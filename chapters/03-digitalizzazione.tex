
\chapter{Digitalizzazione di un segnale audio}

Per poter elaborare un suono tramite strumenti informatici abbiamo prima di tutto bisogno di individuare una sua rappresentazione digitale, cioè numerica.

Esistono molte rappresentazioni possibili e usate nella pratica, ma ce n'è una che possiamo considerare alla base di tutte le altre, e che tra l'altro è quella che viene direttamente prodotta dai convertitori da analogico a digitale (\emph{digital to analog converter} o \emph{DAC}) e accettata dai convertitori da digitale ad analogico (\emph{analog to digital converter} o \emph{ADC}). Si tratta della codifica PCM (\emph{pulse code modulation}, nome adottato più che altro per ragioni storiche, ma non molto rappresentativo dell'effettiva operazione svolta).

Nel caso pratico più semplice di catena audio digitale, il segnale acustico viene in primo luogo convertito in un segnale elettroacustico tramite un microfono o un trasduttore di altro tipo; viene poi preamplificato in maniera da portarlo a un livello di tensione elettrica compatibile con quello accettato dal convertitore ADC; il convertitore produce un flusso di numeri, normalmente rappresentati sotto forma di un segnale digitale elettrico o, talvolta, ottico; questo flusso di numeri viene trasmesso a un convertitore DAC, che lo converte in un segnale elettrico teoricamente identico a quello originale; e questo segnale elettrico viene poi amplificato perché possa guidare un trasduttore come un altoparlante.