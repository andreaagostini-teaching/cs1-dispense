
\chapter{Frequenza}


\section{Frequenza e sue unità di misura}

La nostra rappresentazione cognitiva del suono non avviene nei termini di rapide oscillazioni di pressione dell'aria, ma piuttosto in termini di successioni ed evoluzioni nel tempo di altezze e timbri. Per fare ciò, individuiamo nell'andamento della pressione sonora strutture anche grossolanamente regolari nel tempo e le analizziamo dal punto di vista frequenziale: cioè, in sostanza, valutiamo quante volte tali strutture si ripetono, anche in maniera estremamente approssimativa, in una data unità temporale. Da tale valutazione siamo in grado di trarre informazioni sulle qualità e sull'attività dell'emittente. A complemento di ciò, siamo dotati di un apparato fonatorio in grado di emettere vibrazioni regolari di pressione acustica in maniera estremamente controllata: dalla relazione tra emissione e percezione di tali vibrazioni nasce la parola; la musica, la sua origine, il suo ruolo evolutivo sono fatti più misteriosi e sfuggenti, ma sono basati almeno parzialmente sugli stessi meccanismi.

Introduciamo quindi il concetto di frequenza e la sua misurazione: dato un fenomeno che varia in maniera regolare, possiamo individuare il suo periodo, cioè l'intervallo temporale che intercorre tra due sue ripetizioni. Chiamiamo questo periodo $\Delta t$ e lo misuriamo tramite unità di misura di tempo: millisecondi, secondi, minuti, ore, giorni...

La frequenza indica quante volte la variazione regolare avviene in una data unità di tempo: per esempio, quanti periodi ci sono in un secondo, in un minuto, in un'ora. Possiamo quindi definirla come l'inverso del periodo:
\begin{equation}
f = \frac{1}{\Delta t} = {\Delta t}^{-1}
\end{equation}
e quindi
\begin{equation}
{\Delta t} = \frac{1}{f} = f^{-1}
\end{equation}

Possiamo misurare la frequenza ad esempio in hertz (Hz, cicli per secondo), kilohertz (kHz, migliaia di Hertz o cicli per millisecondo), pulsazioni al minuto (BPM, cicli per minuto). Nota che le formule scritte sopra si possono applicare in maniera letterale solo se le unità del tempo e della frequenza corrispondono; se così non è, sarà necessaria una conversione: per esempio,
\begin{equation}
f_{Hz} = \frac{1}{\Delta t_{s}}
\end{equation}
e
\begin{equation}
f_{kHz} = \frac{1}{\Delta t_{ms}}
\end{equation}
ma
\begin{equation}
f_{kHz} = 1000\frac{1}{\Delta t_{s}} = \frac{1000}{\Delta t_{s}}
\end{equation}
e
\begin{equation}
f_{BPM} = 60\frac{1}{\Delta t_{s}} = \frac{60}{\Delta t_{s}}
\end{equation}

Tutto quanto detto fin qui riguarda in senso stretto la \emph{frequenza fondamentale} di un segnale, che è appunto l'inverso della durata di una delle sue ripetizioni nell'unità di tempo.

Sappiamo però che qualsiasi segnale può essere scomposto in una somma potenzialmente infinita di segnali sinusoidali. Prima di tutto allora abbiamo bisogno di una definizione di sinusoide.


\section{Seni e coseni}

Esistono funzioni matematiche dette periodiche, che ripetono all'infinito un comportamento sempre uguale e che ci riguardano molto da vicino, visto che sono quelle per le quali ha senso definire una frequenza.

Le funzioni periodiche sono molte --- un'infinità, letteralmente --- ma ce ne sono due che hanno alcune proprietà in virtù delle quali esse costituiscono il mattone elementare a partire dal quale costruiremo gran parte del nostro discorso sul suono. Queste funzioni sono il \emph{seno} e il \emph{coseno}, che rappresentano rispettivamente la proiezione sull'asse verticale e orizzontale di un punto della circonferenza goniometrica:%
\footnote{La circonferenza goniometrica (\emph{unit circle} in inglese) è una circonferenza tracciata nel piano cartesiano, con centro corrispondente all'origine degli assi e raggio pari a 1.}
se prendo un raggio della circonferenza che forma un angolo $\alpha$ in senso antiorario rispetto all'asse delle ascisse, il seno di $\alpha$ è la proiezione sull'asse verticale del punto in cui questo raggio tocca la circonferenza; e il coseno di $\alpha$ è la proiezione dello stesso punto sull'asse orizzontale.

Vediamo ora alcuni fatti fondamentali relativi a seno e coseno.

\begin{figure}
    \begin{center}
       \scalebox{0.6} {%% Creator: Matplotlib, PGF backend
%%
%% To include the figure in your LaTeX document, write
%%   \input{<filename>.pgf}
%%
%% Make sure the required packages are loaded in your preamble
%%   \usepackage{pgf}
%%
%% Also ensure that all the required font packages are loaded; for instance,
%% the lmodern package is sometimes necessary when using math font.
%%   \usepackage{lmodern}
%%
%% Figures using additional raster images can only be included by \input if
%% they are in the same directory as the main LaTeX file. For loading figures
%% from other directories you can use the `import` package
%%   \usepackage{import}
%%
%% and then include the figures with
%%   \import{<path to file>}{<filename>.pgf}
%%
%% Matplotlib used the following preamble
%%   
%%   \makeatletter\@ifpackageloaded{underscore}{}{\usepackage[strings]{underscore}}\makeatother
%%
\begingroup%
\makeatletter%
\begin{pgfpicture}%
\pgfpathrectangle{\pgfpointorigin}{\pgfqpoint{6.400000in}{4.800000in}}%
\pgfusepath{use as bounding box, clip}%
\begin{pgfscope}%
\pgfsetbuttcap%
\pgfsetmiterjoin%
\definecolor{currentfill}{rgb}{1.000000,1.000000,1.000000}%
\pgfsetfillcolor{currentfill}%
\pgfsetlinewidth{0.000000pt}%
\definecolor{currentstroke}{rgb}{1.000000,1.000000,1.000000}%
\pgfsetstrokecolor{currentstroke}%
\pgfsetdash{}{0pt}%
\pgfpathmoveto{\pgfqpoint{0.000000in}{0.000000in}}%
\pgfpathlineto{\pgfqpoint{6.400000in}{0.000000in}}%
\pgfpathlineto{\pgfqpoint{6.400000in}{4.800000in}}%
\pgfpathlineto{\pgfqpoint{0.000000in}{4.800000in}}%
\pgfpathlineto{\pgfqpoint{0.000000in}{0.000000in}}%
\pgfpathclose%
\pgfusepath{fill}%
\end{pgfscope}%
\begin{pgfscope}%
\pgfsetbuttcap%
\pgfsetmiterjoin%
\definecolor{currentfill}{rgb}{1.000000,1.000000,1.000000}%
\pgfsetfillcolor{currentfill}%
\pgfsetlinewidth{0.000000pt}%
\definecolor{currentstroke}{rgb}{0.000000,0.000000,0.000000}%
\pgfsetstrokecolor{currentstroke}%
\pgfsetstrokeopacity{0.000000}%
\pgfsetdash{}{0pt}%
\pgfpathmoveto{\pgfqpoint{0.800000in}{0.528000in}}%
\pgfpathlineto{\pgfqpoint{5.760000in}{0.528000in}}%
\pgfpathlineto{\pgfqpoint{5.760000in}{4.224000in}}%
\pgfpathlineto{\pgfqpoint{0.800000in}{4.224000in}}%
\pgfpathlineto{\pgfqpoint{0.800000in}{0.528000in}}%
\pgfpathclose%
\pgfusepath{fill}%
\end{pgfscope}%
\begin{pgfscope}%
\pgfsetbuttcap%
\pgfsetroundjoin%
\definecolor{currentfill}{rgb}{0.000000,0.000000,0.000000}%
\pgfsetfillcolor{currentfill}%
\pgfsetlinewidth{0.803000pt}%
\definecolor{currentstroke}{rgb}{0.000000,0.000000,0.000000}%
\pgfsetstrokecolor{currentstroke}%
\pgfsetdash{}{0pt}%
\pgfsys@defobject{currentmarker}{\pgfqpoint{0.000000in}{-0.048611in}}{\pgfqpoint{0.000000in}{0.000000in}}{%
\pgfpathmoveto{\pgfqpoint{0.000000in}{0.000000in}}%
\pgfpathlineto{\pgfqpoint{0.000000in}{-0.048611in}}%
\pgfusepath{stroke,fill}%
}%
\begin{pgfscope}%
\pgfsys@transformshift{1.025455in}{0.528000in}%
\pgfsys@useobject{currentmarker}{}%
\end{pgfscope}%
\end{pgfscope}%
\begin{pgfscope}%
\definecolor{textcolor}{rgb}{0.000000,0.000000,0.000000}%
\pgfsetstrokecolor{textcolor}%
\pgfsetfillcolor{textcolor}%
\pgftext[x=1.025455in,y=0.430778in,,top]{\color{textcolor}\rmfamily\fontsize{10.000000}{12.000000}\selectfont \(\displaystyle {0}\)}%
\end{pgfscope}%
\begin{pgfscope}%
\pgfsetbuttcap%
\pgfsetroundjoin%
\definecolor{currentfill}{rgb}{0.000000,0.000000,0.000000}%
\pgfsetfillcolor{currentfill}%
\pgfsetlinewidth{0.803000pt}%
\definecolor{currentstroke}{rgb}{0.000000,0.000000,0.000000}%
\pgfsetstrokecolor{currentstroke}%
\pgfsetdash{}{0pt}%
\pgfsys@defobject{currentmarker}{\pgfqpoint{0.000000in}{-0.048611in}}{\pgfqpoint{0.000000in}{0.000000in}}{%
\pgfpathmoveto{\pgfqpoint{0.000000in}{0.000000in}}%
\pgfpathlineto{\pgfqpoint{0.000000in}{-0.048611in}}%
\pgfusepath{stroke,fill}%
}%
\begin{pgfscope}%
\pgfsys@transformshift{1.743099in}{0.528000in}%
\pgfsys@useobject{currentmarker}{}%
\end{pgfscope}%
\end{pgfscope}%
\begin{pgfscope}%
\definecolor{textcolor}{rgb}{0.000000,0.000000,0.000000}%
\pgfsetstrokecolor{textcolor}%
\pgfsetfillcolor{textcolor}%
\pgftext[x=1.743099in,y=0.430778in,,top]{\color{textcolor}\rmfamily\fontsize{10.000000}{12.000000}\selectfont \(\displaystyle {1}\)}%
\end{pgfscope}%
\begin{pgfscope}%
\pgfsetbuttcap%
\pgfsetroundjoin%
\definecolor{currentfill}{rgb}{0.000000,0.000000,0.000000}%
\pgfsetfillcolor{currentfill}%
\pgfsetlinewidth{0.803000pt}%
\definecolor{currentstroke}{rgb}{0.000000,0.000000,0.000000}%
\pgfsetstrokecolor{currentstroke}%
\pgfsetdash{}{0pt}%
\pgfsys@defobject{currentmarker}{\pgfqpoint{0.000000in}{-0.048611in}}{\pgfqpoint{0.000000in}{0.000000in}}{%
\pgfpathmoveto{\pgfqpoint{0.000000in}{0.000000in}}%
\pgfpathlineto{\pgfqpoint{0.000000in}{-0.048611in}}%
\pgfusepath{stroke,fill}%
}%
\begin{pgfscope}%
\pgfsys@transformshift{2.460743in}{0.528000in}%
\pgfsys@useobject{currentmarker}{}%
\end{pgfscope}%
\end{pgfscope}%
\begin{pgfscope}%
\definecolor{textcolor}{rgb}{0.000000,0.000000,0.000000}%
\pgfsetstrokecolor{textcolor}%
\pgfsetfillcolor{textcolor}%
\pgftext[x=2.460743in,y=0.430778in,,top]{\color{textcolor}\rmfamily\fontsize{10.000000}{12.000000}\selectfont \(\displaystyle {2}\)}%
\end{pgfscope}%
\begin{pgfscope}%
\pgfsetbuttcap%
\pgfsetroundjoin%
\definecolor{currentfill}{rgb}{0.000000,0.000000,0.000000}%
\pgfsetfillcolor{currentfill}%
\pgfsetlinewidth{0.803000pt}%
\definecolor{currentstroke}{rgb}{0.000000,0.000000,0.000000}%
\pgfsetstrokecolor{currentstroke}%
\pgfsetdash{}{0pt}%
\pgfsys@defobject{currentmarker}{\pgfqpoint{0.000000in}{-0.048611in}}{\pgfqpoint{0.000000in}{0.000000in}}{%
\pgfpathmoveto{\pgfqpoint{0.000000in}{0.000000in}}%
\pgfpathlineto{\pgfqpoint{0.000000in}{-0.048611in}}%
\pgfusepath{stroke,fill}%
}%
\begin{pgfscope}%
\pgfsys@transformshift{3.178387in}{0.528000in}%
\pgfsys@useobject{currentmarker}{}%
\end{pgfscope}%
\end{pgfscope}%
\begin{pgfscope}%
\definecolor{textcolor}{rgb}{0.000000,0.000000,0.000000}%
\pgfsetstrokecolor{textcolor}%
\pgfsetfillcolor{textcolor}%
\pgftext[x=3.178387in,y=0.430778in,,top]{\color{textcolor}\rmfamily\fontsize{10.000000}{12.000000}\selectfont \(\displaystyle {3}\)}%
\end{pgfscope}%
\begin{pgfscope}%
\pgfsetbuttcap%
\pgfsetroundjoin%
\definecolor{currentfill}{rgb}{0.000000,0.000000,0.000000}%
\pgfsetfillcolor{currentfill}%
\pgfsetlinewidth{0.803000pt}%
\definecolor{currentstroke}{rgb}{0.000000,0.000000,0.000000}%
\pgfsetstrokecolor{currentstroke}%
\pgfsetdash{}{0pt}%
\pgfsys@defobject{currentmarker}{\pgfqpoint{0.000000in}{-0.048611in}}{\pgfqpoint{0.000000in}{0.000000in}}{%
\pgfpathmoveto{\pgfqpoint{0.000000in}{0.000000in}}%
\pgfpathlineto{\pgfqpoint{0.000000in}{-0.048611in}}%
\pgfusepath{stroke,fill}%
}%
\begin{pgfscope}%
\pgfsys@transformshift{3.896031in}{0.528000in}%
\pgfsys@useobject{currentmarker}{}%
\end{pgfscope}%
\end{pgfscope}%
\begin{pgfscope}%
\definecolor{textcolor}{rgb}{0.000000,0.000000,0.000000}%
\pgfsetstrokecolor{textcolor}%
\pgfsetfillcolor{textcolor}%
\pgftext[x=3.896031in,y=0.430778in,,top]{\color{textcolor}\rmfamily\fontsize{10.000000}{12.000000}\selectfont \(\displaystyle {4}\)}%
\end{pgfscope}%
\begin{pgfscope}%
\pgfsetbuttcap%
\pgfsetroundjoin%
\definecolor{currentfill}{rgb}{0.000000,0.000000,0.000000}%
\pgfsetfillcolor{currentfill}%
\pgfsetlinewidth{0.803000pt}%
\definecolor{currentstroke}{rgb}{0.000000,0.000000,0.000000}%
\pgfsetstrokecolor{currentstroke}%
\pgfsetdash{}{0pt}%
\pgfsys@defobject{currentmarker}{\pgfqpoint{0.000000in}{-0.048611in}}{\pgfqpoint{0.000000in}{0.000000in}}{%
\pgfpathmoveto{\pgfqpoint{0.000000in}{0.000000in}}%
\pgfpathlineto{\pgfqpoint{0.000000in}{-0.048611in}}%
\pgfusepath{stroke,fill}%
}%
\begin{pgfscope}%
\pgfsys@transformshift{4.613675in}{0.528000in}%
\pgfsys@useobject{currentmarker}{}%
\end{pgfscope}%
\end{pgfscope}%
\begin{pgfscope}%
\definecolor{textcolor}{rgb}{0.000000,0.000000,0.000000}%
\pgfsetstrokecolor{textcolor}%
\pgfsetfillcolor{textcolor}%
\pgftext[x=4.613675in,y=0.430778in,,top]{\color{textcolor}\rmfamily\fontsize{10.000000}{12.000000}\selectfont \(\displaystyle {5}\)}%
\end{pgfscope}%
\begin{pgfscope}%
\pgfsetbuttcap%
\pgfsetroundjoin%
\definecolor{currentfill}{rgb}{0.000000,0.000000,0.000000}%
\pgfsetfillcolor{currentfill}%
\pgfsetlinewidth{0.803000pt}%
\definecolor{currentstroke}{rgb}{0.000000,0.000000,0.000000}%
\pgfsetstrokecolor{currentstroke}%
\pgfsetdash{}{0pt}%
\pgfsys@defobject{currentmarker}{\pgfqpoint{0.000000in}{-0.048611in}}{\pgfqpoint{0.000000in}{0.000000in}}{%
\pgfpathmoveto{\pgfqpoint{0.000000in}{0.000000in}}%
\pgfpathlineto{\pgfqpoint{0.000000in}{-0.048611in}}%
\pgfusepath{stroke,fill}%
}%
\begin{pgfscope}%
\pgfsys@transformshift{5.331319in}{0.528000in}%
\pgfsys@useobject{currentmarker}{}%
\end{pgfscope}%
\end{pgfscope}%
\begin{pgfscope}%
\definecolor{textcolor}{rgb}{0.000000,0.000000,0.000000}%
\pgfsetstrokecolor{textcolor}%
\pgfsetfillcolor{textcolor}%
\pgftext[x=5.331319in,y=0.430778in,,top]{\color{textcolor}\rmfamily\fontsize{10.000000}{12.000000}\selectfont \(\displaystyle {6}\)}%
\end{pgfscope}%
\begin{pgfscope}%
\definecolor{textcolor}{rgb}{0.000000,0.000000,0.000000}%
\pgfsetstrokecolor{textcolor}%
\pgfsetfillcolor{textcolor}%
\pgftext[x=3.280000in,y=0.251766in,,top]{\color{textcolor}\rmfamily\fontsize{10.000000}{12.000000}\selectfont Angle [rad]}%
\end{pgfscope}%
\begin{pgfscope}%
\pgfsetbuttcap%
\pgfsetroundjoin%
\definecolor{currentfill}{rgb}{0.000000,0.000000,0.000000}%
\pgfsetfillcolor{currentfill}%
\pgfsetlinewidth{0.803000pt}%
\definecolor{currentstroke}{rgb}{0.000000,0.000000,0.000000}%
\pgfsetstrokecolor{currentstroke}%
\pgfsetdash{}{0pt}%
\pgfsys@defobject{currentmarker}{\pgfqpoint{-0.048611in}{0.000000in}}{\pgfqpoint{-0.000000in}{0.000000in}}{%
\pgfpathmoveto{\pgfqpoint{-0.000000in}{0.000000in}}%
\pgfpathlineto{\pgfqpoint{-0.048611in}{0.000000in}}%
\pgfusepath{stroke,fill}%
}%
\begin{pgfscope}%
\pgfsys@transformshift{0.800000in}{0.696000in}%
\pgfsys@useobject{currentmarker}{}%
\end{pgfscope}%
\end{pgfscope}%
\begin{pgfscope}%
\definecolor{textcolor}{rgb}{0.000000,0.000000,0.000000}%
\pgfsetstrokecolor{textcolor}%
\pgfsetfillcolor{textcolor}%
\pgftext[x=0.347838in, y=0.647775in, left, base]{\color{textcolor}\rmfamily\fontsize{10.000000}{12.000000}\selectfont \(\displaystyle {\ensuremath{-}1.00}\)}%
\end{pgfscope}%
\begin{pgfscope}%
\pgfsetbuttcap%
\pgfsetroundjoin%
\definecolor{currentfill}{rgb}{0.000000,0.000000,0.000000}%
\pgfsetfillcolor{currentfill}%
\pgfsetlinewidth{0.803000pt}%
\definecolor{currentstroke}{rgb}{0.000000,0.000000,0.000000}%
\pgfsetstrokecolor{currentstroke}%
\pgfsetdash{}{0pt}%
\pgfsys@defobject{currentmarker}{\pgfqpoint{-0.048611in}{0.000000in}}{\pgfqpoint{-0.000000in}{0.000000in}}{%
\pgfpathmoveto{\pgfqpoint{-0.000000in}{0.000000in}}%
\pgfpathlineto{\pgfqpoint{-0.048611in}{0.000000in}}%
\pgfusepath{stroke,fill}%
}%
\begin{pgfscope}%
\pgfsys@transformshift{0.800000in}{1.116000in}%
\pgfsys@useobject{currentmarker}{}%
\end{pgfscope}%
\end{pgfscope}%
\begin{pgfscope}%
\definecolor{textcolor}{rgb}{0.000000,0.000000,0.000000}%
\pgfsetstrokecolor{textcolor}%
\pgfsetfillcolor{textcolor}%
\pgftext[x=0.347838in, y=1.067775in, left, base]{\color{textcolor}\rmfamily\fontsize{10.000000}{12.000000}\selectfont \(\displaystyle {\ensuremath{-}0.75}\)}%
\end{pgfscope}%
\begin{pgfscope}%
\pgfsetbuttcap%
\pgfsetroundjoin%
\definecolor{currentfill}{rgb}{0.000000,0.000000,0.000000}%
\pgfsetfillcolor{currentfill}%
\pgfsetlinewidth{0.803000pt}%
\definecolor{currentstroke}{rgb}{0.000000,0.000000,0.000000}%
\pgfsetstrokecolor{currentstroke}%
\pgfsetdash{}{0pt}%
\pgfsys@defobject{currentmarker}{\pgfqpoint{-0.048611in}{0.000000in}}{\pgfqpoint{-0.000000in}{0.000000in}}{%
\pgfpathmoveto{\pgfqpoint{-0.000000in}{0.000000in}}%
\pgfpathlineto{\pgfqpoint{-0.048611in}{0.000000in}}%
\pgfusepath{stroke,fill}%
}%
\begin{pgfscope}%
\pgfsys@transformshift{0.800000in}{1.536000in}%
\pgfsys@useobject{currentmarker}{}%
\end{pgfscope}%
\end{pgfscope}%
\begin{pgfscope}%
\definecolor{textcolor}{rgb}{0.000000,0.000000,0.000000}%
\pgfsetstrokecolor{textcolor}%
\pgfsetfillcolor{textcolor}%
\pgftext[x=0.347838in, y=1.487775in, left, base]{\color{textcolor}\rmfamily\fontsize{10.000000}{12.000000}\selectfont \(\displaystyle {\ensuremath{-}0.50}\)}%
\end{pgfscope}%
\begin{pgfscope}%
\pgfsetbuttcap%
\pgfsetroundjoin%
\definecolor{currentfill}{rgb}{0.000000,0.000000,0.000000}%
\pgfsetfillcolor{currentfill}%
\pgfsetlinewidth{0.803000pt}%
\definecolor{currentstroke}{rgb}{0.000000,0.000000,0.000000}%
\pgfsetstrokecolor{currentstroke}%
\pgfsetdash{}{0pt}%
\pgfsys@defobject{currentmarker}{\pgfqpoint{-0.048611in}{0.000000in}}{\pgfqpoint{-0.000000in}{0.000000in}}{%
\pgfpathmoveto{\pgfqpoint{-0.000000in}{0.000000in}}%
\pgfpathlineto{\pgfqpoint{-0.048611in}{0.000000in}}%
\pgfusepath{stroke,fill}%
}%
\begin{pgfscope}%
\pgfsys@transformshift{0.800000in}{1.956000in}%
\pgfsys@useobject{currentmarker}{}%
\end{pgfscope}%
\end{pgfscope}%
\begin{pgfscope}%
\definecolor{textcolor}{rgb}{0.000000,0.000000,0.000000}%
\pgfsetstrokecolor{textcolor}%
\pgfsetfillcolor{textcolor}%
\pgftext[x=0.347838in, y=1.907775in, left, base]{\color{textcolor}\rmfamily\fontsize{10.000000}{12.000000}\selectfont \(\displaystyle {\ensuremath{-}0.25}\)}%
\end{pgfscope}%
\begin{pgfscope}%
\pgfsetbuttcap%
\pgfsetroundjoin%
\definecolor{currentfill}{rgb}{0.000000,0.000000,0.000000}%
\pgfsetfillcolor{currentfill}%
\pgfsetlinewidth{0.803000pt}%
\definecolor{currentstroke}{rgb}{0.000000,0.000000,0.000000}%
\pgfsetstrokecolor{currentstroke}%
\pgfsetdash{}{0pt}%
\pgfsys@defobject{currentmarker}{\pgfqpoint{-0.048611in}{0.000000in}}{\pgfqpoint{-0.000000in}{0.000000in}}{%
\pgfpathmoveto{\pgfqpoint{-0.000000in}{0.000000in}}%
\pgfpathlineto{\pgfqpoint{-0.048611in}{0.000000in}}%
\pgfusepath{stroke,fill}%
}%
\begin{pgfscope}%
\pgfsys@transformshift{0.800000in}{2.376000in}%
\pgfsys@useobject{currentmarker}{}%
\end{pgfscope}%
\end{pgfscope}%
\begin{pgfscope}%
\definecolor{textcolor}{rgb}{0.000000,0.000000,0.000000}%
\pgfsetstrokecolor{textcolor}%
\pgfsetfillcolor{textcolor}%
\pgftext[x=0.455863in, y=2.327775in, left, base]{\color{textcolor}\rmfamily\fontsize{10.000000}{12.000000}\selectfont \(\displaystyle {0.00}\)}%
\end{pgfscope}%
\begin{pgfscope}%
\pgfsetbuttcap%
\pgfsetroundjoin%
\definecolor{currentfill}{rgb}{0.000000,0.000000,0.000000}%
\pgfsetfillcolor{currentfill}%
\pgfsetlinewidth{0.803000pt}%
\definecolor{currentstroke}{rgb}{0.000000,0.000000,0.000000}%
\pgfsetstrokecolor{currentstroke}%
\pgfsetdash{}{0pt}%
\pgfsys@defobject{currentmarker}{\pgfqpoint{-0.048611in}{0.000000in}}{\pgfqpoint{-0.000000in}{0.000000in}}{%
\pgfpathmoveto{\pgfqpoint{-0.000000in}{0.000000in}}%
\pgfpathlineto{\pgfqpoint{-0.048611in}{0.000000in}}%
\pgfusepath{stroke,fill}%
}%
\begin{pgfscope}%
\pgfsys@transformshift{0.800000in}{2.796000in}%
\pgfsys@useobject{currentmarker}{}%
\end{pgfscope}%
\end{pgfscope}%
\begin{pgfscope}%
\definecolor{textcolor}{rgb}{0.000000,0.000000,0.000000}%
\pgfsetstrokecolor{textcolor}%
\pgfsetfillcolor{textcolor}%
\pgftext[x=0.455863in, y=2.747775in, left, base]{\color{textcolor}\rmfamily\fontsize{10.000000}{12.000000}\selectfont \(\displaystyle {0.25}\)}%
\end{pgfscope}%
\begin{pgfscope}%
\pgfsetbuttcap%
\pgfsetroundjoin%
\definecolor{currentfill}{rgb}{0.000000,0.000000,0.000000}%
\pgfsetfillcolor{currentfill}%
\pgfsetlinewidth{0.803000pt}%
\definecolor{currentstroke}{rgb}{0.000000,0.000000,0.000000}%
\pgfsetstrokecolor{currentstroke}%
\pgfsetdash{}{0pt}%
\pgfsys@defobject{currentmarker}{\pgfqpoint{-0.048611in}{0.000000in}}{\pgfqpoint{-0.000000in}{0.000000in}}{%
\pgfpathmoveto{\pgfqpoint{-0.000000in}{0.000000in}}%
\pgfpathlineto{\pgfqpoint{-0.048611in}{0.000000in}}%
\pgfusepath{stroke,fill}%
}%
\begin{pgfscope}%
\pgfsys@transformshift{0.800000in}{3.216000in}%
\pgfsys@useobject{currentmarker}{}%
\end{pgfscope}%
\end{pgfscope}%
\begin{pgfscope}%
\definecolor{textcolor}{rgb}{0.000000,0.000000,0.000000}%
\pgfsetstrokecolor{textcolor}%
\pgfsetfillcolor{textcolor}%
\pgftext[x=0.455863in, y=3.167775in, left, base]{\color{textcolor}\rmfamily\fontsize{10.000000}{12.000000}\selectfont \(\displaystyle {0.50}\)}%
\end{pgfscope}%
\begin{pgfscope}%
\pgfsetbuttcap%
\pgfsetroundjoin%
\definecolor{currentfill}{rgb}{0.000000,0.000000,0.000000}%
\pgfsetfillcolor{currentfill}%
\pgfsetlinewidth{0.803000pt}%
\definecolor{currentstroke}{rgb}{0.000000,0.000000,0.000000}%
\pgfsetstrokecolor{currentstroke}%
\pgfsetdash{}{0pt}%
\pgfsys@defobject{currentmarker}{\pgfqpoint{-0.048611in}{0.000000in}}{\pgfqpoint{-0.000000in}{0.000000in}}{%
\pgfpathmoveto{\pgfqpoint{-0.000000in}{0.000000in}}%
\pgfpathlineto{\pgfqpoint{-0.048611in}{0.000000in}}%
\pgfusepath{stroke,fill}%
}%
\begin{pgfscope}%
\pgfsys@transformshift{0.800000in}{3.636000in}%
\pgfsys@useobject{currentmarker}{}%
\end{pgfscope}%
\end{pgfscope}%
\begin{pgfscope}%
\definecolor{textcolor}{rgb}{0.000000,0.000000,0.000000}%
\pgfsetstrokecolor{textcolor}%
\pgfsetfillcolor{textcolor}%
\pgftext[x=0.455863in, y=3.587775in, left, base]{\color{textcolor}\rmfamily\fontsize{10.000000}{12.000000}\selectfont \(\displaystyle {0.75}\)}%
\end{pgfscope}%
\begin{pgfscope}%
\pgfsetbuttcap%
\pgfsetroundjoin%
\definecolor{currentfill}{rgb}{0.000000,0.000000,0.000000}%
\pgfsetfillcolor{currentfill}%
\pgfsetlinewidth{0.803000pt}%
\definecolor{currentstroke}{rgb}{0.000000,0.000000,0.000000}%
\pgfsetstrokecolor{currentstroke}%
\pgfsetdash{}{0pt}%
\pgfsys@defobject{currentmarker}{\pgfqpoint{-0.048611in}{0.000000in}}{\pgfqpoint{-0.000000in}{0.000000in}}{%
\pgfpathmoveto{\pgfqpoint{-0.000000in}{0.000000in}}%
\pgfpathlineto{\pgfqpoint{-0.048611in}{0.000000in}}%
\pgfusepath{stroke,fill}%
}%
\begin{pgfscope}%
\pgfsys@transformshift{0.800000in}{4.056000in}%
\pgfsys@useobject{currentmarker}{}%
\end{pgfscope}%
\end{pgfscope}%
\begin{pgfscope}%
\definecolor{textcolor}{rgb}{0.000000,0.000000,0.000000}%
\pgfsetstrokecolor{textcolor}%
\pgfsetfillcolor{textcolor}%
\pgftext[x=0.455863in, y=4.007775in, left, base]{\color{textcolor}\rmfamily\fontsize{10.000000}{12.000000}\selectfont \(\displaystyle {1.00}\)}%
\end{pgfscope}%
\begin{pgfscope}%
\definecolor{textcolor}{rgb}{0.000000,0.000000,0.000000}%
\pgfsetstrokecolor{textcolor}%
\pgfsetfillcolor{textcolor}%
\pgftext[x=0.292283in,y=2.376000in,,bottom,rotate=90.000000]{\color{textcolor}\rmfamily\fontsize{10.000000}{12.000000}\selectfont seno(x) and coseno(x)}%
\end{pgfscope}%
\begin{pgfscope}%
\pgfpathrectangle{\pgfqpoint{0.800000in}{0.528000in}}{\pgfqpoint{4.960000in}{3.696000in}}%
\pgfusepath{clip}%
\pgfsetrectcap%
\pgfsetroundjoin%
\pgfsetlinewidth{1.505625pt}%
\definecolor{currentstroke}{rgb}{0.121569,0.466667,0.705882}%
\pgfsetstrokecolor{currentstroke}%
\pgfsetdash{}{0pt}%
\pgfpathmoveto{\pgfqpoint{1.025455in}{2.376000in}}%
\pgfpathlineto{\pgfqpoint{1.160727in}{2.690801in}}%
\pgfpathlineto{\pgfqpoint{1.250909in}{2.895149in}}%
\pgfpathlineto{\pgfqpoint{1.318545in}{3.043208in}}%
\pgfpathlineto{\pgfqpoint{1.386182in}{3.185346in}}%
\pgfpathlineto{\pgfqpoint{1.431273in}{3.276189in}}%
\pgfpathlineto{\pgfqpoint{1.476364in}{3.363479in}}%
\pgfpathlineto{\pgfqpoint{1.521455in}{3.446872in}}%
\pgfpathlineto{\pgfqpoint{1.566545in}{3.526039in}}%
\pgfpathlineto{\pgfqpoint{1.611636in}{3.600667in}}%
\pgfpathlineto{\pgfqpoint{1.656727in}{3.670462in}}%
\pgfpathlineto{\pgfqpoint{1.701818in}{3.735149in}}%
\pgfpathlineto{\pgfqpoint{1.746909in}{3.794471in}}%
\pgfpathlineto{\pgfqpoint{1.792000in}{3.848195in}}%
\pgfpathlineto{\pgfqpoint{1.837091in}{3.896109in}}%
\pgfpathlineto{\pgfqpoint{1.882182in}{3.938024in}}%
\pgfpathlineto{\pgfqpoint{1.904727in}{3.956680in}}%
\pgfpathlineto{\pgfqpoint{1.927273in}{3.973775in}}%
\pgfpathlineto{\pgfqpoint{1.949818in}{3.989293in}}%
\pgfpathlineto{\pgfqpoint{1.972364in}{4.003220in}}%
\pgfpathlineto{\pgfqpoint{1.994909in}{4.015540in}}%
\pgfpathlineto{\pgfqpoint{2.017455in}{4.026243in}}%
\pgfpathlineto{\pgfqpoint{2.040000in}{4.035316in}}%
\pgfpathlineto{\pgfqpoint{2.062545in}{4.042753in}}%
\pgfpathlineto{\pgfqpoint{2.085091in}{4.048544in}}%
\pgfpathlineto{\pgfqpoint{2.107636in}{4.052685in}}%
\pgfpathlineto{\pgfqpoint{2.130182in}{4.055171in}}%
\pgfpathlineto{\pgfqpoint{2.152727in}{4.056000in}}%
\pgfpathlineto{\pgfqpoint{2.175273in}{4.055171in}}%
\pgfpathlineto{\pgfqpoint{2.197818in}{4.052685in}}%
\pgfpathlineto{\pgfqpoint{2.220364in}{4.048544in}}%
\pgfpathlineto{\pgfqpoint{2.242909in}{4.042753in}}%
\pgfpathlineto{\pgfqpoint{2.265455in}{4.035316in}}%
\pgfpathlineto{\pgfqpoint{2.288000in}{4.026243in}}%
\pgfpathlineto{\pgfqpoint{2.310545in}{4.015540in}}%
\pgfpathlineto{\pgfqpoint{2.333091in}{4.003220in}}%
\pgfpathlineto{\pgfqpoint{2.355636in}{3.989293in}}%
\pgfpathlineto{\pgfqpoint{2.378182in}{3.973775in}}%
\pgfpathlineto{\pgfqpoint{2.400727in}{3.956680in}}%
\pgfpathlineto{\pgfqpoint{2.423273in}{3.938024in}}%
\pgfpathlineto{\pgfqpoint{2.445818in}{3.917828in}}%
\pgfpathlineto{\pgfqpoint{2.490909in}{3.872891in}}%
\pgfpathlineto{\pgfqpoint{2.536000in}{3.822047in}}%
\pgfpathlineto{\pgfqpoint{2.581091in}{3.765495in}}%
\pgfpathlineto{\pgfqpoint{2.626182in}{3.703460in}}%
\pgfpathlineto{\pgfqpoint{2.671273in}{3.636187in}}%
\pgfpathlineto{\pgfqpoint{2.716364in}{3.563939in}}%
\pgfpathlineto{\pgfqpoint{2.761455in}{3.487004in}}%
\pgfpathlineto{\pgfqpoint{2.806545in}{3.405684in}}%
\pgfpathlineto{\pgfqpoint{2.851636in}{3.320300in}}%
\pgfpathlineto{\pgfqpoint{2.896727in}{3.231190in}}%
\pgfpathlineto{\pgfqpoint{2.964364in}{3.091309in}}%
\pgfpathlineto{\pgfqpoint{3.032000in}{2.945080in}}%
\pgfpathlineto{\pgfqpoint{3.099636in}{2.793799in}}%
\pgfpathlineto{\pgfqpoint{3.189818in}{2.586560in}}%
\pgfpathlineto{\pgfqpoint{3.482909in}{1.907295in}}%
\pgfpathlineto{\pgfqpoint{3.550545in}{1.757551in}}%
\pgfpathlineto{\pgfqpoint{3.618182in}{1.613296in}}%
\pgfpathlineto{\pgfqpoint{3.685818in}{1.475811in}}%
\pgfpathlineto{\pgfqpoint{3.730909in}{1.388521in}}%
\pgfpathlineto{\pgfqpoint{3.776000in}{1.305128in}}%
\pgfpathlineto{\pgfqpoint{3.821091in}{1.225961in}}%
\pgfpathlineto{\pgfqpoint{3.866182in}{1.151333in}}%
\pgfpathlineto{\pgfqpoint{3.911273in}{1.081538in}}%
\pgfpathlineto{\pgfqpoint{3.956364in}{1.016851in}}%
\pgfpathlineto{\pgfqpoint{4.001455in}{0.957529in}}%
\pgfpathlineto{\pgfqpoint{4.046545in}{0.903805in}}%
\pgfpathlineto{\pgfqpoint{4.091636in}{0.855891in}}%
\pgfpathlineto{\pgfqpoint{4.136727in}{0.813976in}}%
\pgfpathlineto{\pgfqpoint{4.159273in}{0.795320in}}%
\pgfpathlineto{\pgfqpoint{4.181818in}{0.778225in}}%
\pgfpathlineto{\pgfqpoint{4.204364in}{0.762707in}}%
\pgfpathlineto{\pgfqpoint{4.226909in}{0.748780in}}%
\pgfpathlineto{\pgfqpoint{4.249455in}{0.736460in}}%
\pgfpathlineto{\pgfqpoint{4.272000in}{0.725757in}}%
\pgfpathlineto{\pgfqpoint{4.294545in}{0.716684in}}%
\pgfpathlineto{\pgfqpoint{4.317091in}{0.709247in}}%
\pgfpathlineto{\pgfqpoint{4.339636in}{0.703456in}}%
\pgfpathlineto{\pgfqpoint{4.362182in}{0.699315in}}%
\pgfpathlineto{\pgfqpoint{4.384727in}{0.696829in}}%
\pgfpathlineto{\pgfqpoint{4.407273in}{0.696000in}}%
\pgfpathlineto{\pgfqpoint{4.429818in}{0.696829in}}%
\pgfpathlineto{\pgfqpoint{4.452364in}{0.699315in}}%
\pgfpathlineto{\pgfqpoint{4.474909in}{0.703456in}}%
\pgfpathlineto{\pgfqpoint{4.497455in}{0.709247in}}%
\pgfpathlineto{\pgfqpoint{4.520000in}{0.716684in}}%
\pgfpathlineto{\pgfqpoint{4.542545in}{0.725757in}}%
\pgfpathlineto{\pgfqpoint{4.565091in}{0.736460in}}%
\pgfpathlineto{\pgfqpoint{4.587636in}{0.748780in}}%
\pgfpathlineto{\pgfqpoint{4.610182in}{0.762707in}}%
\pgfpathlineto{\pgfqpoint{4.632727in}{0.778225in}}%
\pgfpathlineto{\pgfqpoint{4.655273in}{0.795320in}}%
\pgfpathlineto{\pgfqpoint{4.677818in}{0.813976in}}%
\pgfpathlineto{\pgfqpoint{4.700364in}{0.834172in}}%
\pgfpathlineto{\pgfqpoint{4.745455in}{0.879109in}}%
\pgfpathlineto{\pgfqpoint{4.790545in}{0.929953in}}%
\pgfpathlineto{\pgfqpoint{4.835636in}{0.986505in}}%
\pgfpathlineto{\pgfqpoint{4.880727in}{1.048540in}}%
\pgfpathlineto{\pgfqpoint{4.925818in}{1.115813in}}%
\pgfpathlineto{\pgfqpoint{4.970909in}{1.188061in}}%
\pgfpathlineto{\pgfqpoint{5.016000in}{1.264996in}}%
\pgfpathlineto{\pgfqpoint{5.061091in}{1.346316in}}%
\pgfpathlineto{\pgfqpoint{5.106182in}{1.431700in}}%
\pgfpathlineto{\pgfqpoint{5.151273in}{1.520810in}}%
\pgfpathlineto{\pgfqpoint{5.218909in}{1.660691in}}%
\pgfpathlineto{\pgfqpoint{5.286545in}{1.806920in}}%
\pgfpathlineto{\pgfqpoint{5.354182in}{1.958201in}}%
\pgfpathlineto{\pgfqpoint{5.444364in}{2.165440in}}%
\pgfpathlineto{\pgfqpoint{5.534545in}{2.376000in}}%
\pgfpathlineto{\pgfqpoint{5.534545in}{2.376000in}}%
\pgfusepath{stroke}%
\end{pgfscope}%
\begin{pgfscope}%
\pgfpathrectangle{\pgfqpoint{0.800000in}{0.528000in}}{\pgfqpoint{4.960000in}{3.696000in}}%
\pgfusepath{clip}%
\pgfsetrectcap%
\pgfsetroundjoin%
\pgfsetlinewidth{1.505625pt}%
\definecolor{currentstroke}{rgb}{1.000000,0.498039,0.054902}%
\pgfsetstrokecolor{currentstroke}%
\pgfsetdash{}{0pt}%
\pgfpathmoveto{\pgfqpoint{1.025455in}{4.056000in}}%
\pgfpathlineto{\pgfqpoint{1.048000in}{4.055171in}}%
\pgfpathlineto{\pgfqpoint{1.070545in}{4.052685in}}%
\pgfpathlineto{\pgfqpoint{1.093091in}{4.048544in}}%
\pgfpathlineto{\pgfqpoint{1.115636in}{4.042753in}}%
\pgfpathlineto{\pgfqpoint{1.138182in}{4.035316in}}%
\pgfpathlineto{\pgfqpoint{1.160727in}{4.026243in}}%
\pgfpathlineto{\pgfqpoint{1.183273in}{4.015540in}}%
\pgfpathlineto{\pgfqpoint{1.205818in}{4.003220in}}%
\pgfpathlineto{\pgfqpoint{1.228364in}{3.989293in}}%
\pgfpathlineto{\pgfqpoint{1.250909in}{3.973775in}}%
\pgfpathlineto{\pgfqpoint{1.273455in}{3.956680in}}%
\pgfpathlineto{\pgfqpoint{1.296000in}{3.938024in}}%
\pgfpathlineto{\pgfqpoint{1.318545in}{3.917828in}}%
\pgfpathlineto{\pgfqpoint{1.363636in}{3.872891in}}%
\pgfpathlineto{\pgfqpoint{1.408727in}{3.822047in}}%
\pgfpathlineto{\pgfqpoint{1.453818in}{3.765495in}}%
\pgfpathlineto{\pgfqpoint{1.498909in}{3.703460in}}%
\pgfpathlineto{\pgfqpoint{1.544000in}{3.636187in}}%
\pgfpathlineto{\pgfqpoint{1.589091in}{3.563939in}}%
\pgfpathlineto{\pgfqpoint{1.634182in}{3.487004in}}%
\pgfpathlineto{\pgfqpoint{1.679273in}{3.405684in}}%
\pgfpathlineto{\pgfqpoint{1.724364in}{3.320300in}}%
\pgfpathlineto{\pgfqpoint{1.769455in}{3.231190in}}%
\pgfpathlineto{\pgfqpoint{1.837091in}{3.091309in}}%
\pgfpathlineto{\pgfqpoint{1.904727in}{2.945080in}}%
\pgfpathlineto{\pgfqpoint{1.972364in}{2.793799in}}%
\pgfpathlineto{\pgfqpoint{2.062545in}{2.586560in}}%
\pgfpathlineto{\pgfqpoint{2.355636in}{1.907295in}}%
\pgfpathlineto{\pgfqpoint{2.423273in}{1.757551in}}%
\pgfpathlineto{\pgfqpoint{2.490909in}{1.613296in}}%
\pgfpathlineto{\pgfqpoint{2.558545in}{1.475811in}}%
\pgfpathlineto{\pgfqpoint{2.603636in}{1.388521in}}%
\pgfpathlineto{\pgfqpoint{2.648727in}{1.305128in}}%
\pgfpathlineto{\pgfqpoint{2.693818in}{1.225961in}}%
\pgfpathlineto{\pgfqpoint{2.738909in}{1.151333in}}%
\pgfpathlineto{\pgfqpoint{2.784000in}{1.081538in}}%
\pgfpathlineto{\pgfqpoint{2.829091in}{1.016851in}}%
\pgfpathlineto{\pgfqpoint{2.874182in}{0.957529in}}%
\pgfpathlineto{\pgfqpoint{2.919273in}{0.903805in}}%
\pgfpathlineto{\pgfqpoint{2.964364in}{0.855891in}}%
\pgfpathlineto{\pgfqpoint{3.009455in}{0.813976in}}%
\pgfpathlineto{\pgfqpoint{3.032000in}{0.795320in}}%
\pgfpathlineto{\pgfqpoint{3.054545in}{0.778225in}}%
\pgfpathlineto{\pgfqpoint{3.077091in}{0.762707in}}%
\pgfpathlineto{\pgfqpoint{3.099636in}{0.748780in}}%
\pgfpathlineto{\pgfqpoint{3.122182in}{0.736460in}}%
\pgfpathlineto{\pgfqpoint{3.144727in}{0.725757in}}%
\pgfpathlineto{\pgfqpoint{3.167273in}{0.716684in}}%
\pgfpathlineto{\pgfqpoint{3.189818in}{0.709247in}}%
\pgfpathlineto{\pgfqpoint{3.212364in}{0.703456in}}%
\pgfpathlineto{\pgfqpoint{3.234909in}{0.699315in}}%
\pgfpathlineto{\pgfqpoint{3.257455in}{0.696829in}}%
\pgfpathlineto{\pgfqpoint{3.280000in}{0.696000in}}%
\pgfpathlineto{\pgfqpoint{3.302545in}{0.696829in}}%
\pgfpathlineto{\pgfqpoint{3.325091in}{0.699315in}}%
\pgfpathlineto{\pgfqpoint{3.347636in}{0.703456in}}%
\pgfpathlineto{\pgfqpoint{3.370182in}{0.709247in}}%
\pgfpathlineto{\pgfqpoint{3.392727in}{0.716684in}}%
\pgfpathlineto{\pgfqpoint{3.415273in}{0.725757in}}%
\pgfpathlineto{\pgfqpoint{3.437818in}{0.736460in}}%
\pgfpathlineto{\pgfqpoint{3.460364in}{0.748780in}}%
\pgfpathlineto{\pgfqpoint{3.482909in}{0.762707in}}%
\pgfpathlineto{\pgfqpoint{3.505455in}{0.778225in}}%
\pgfpathlineto{\pgfqpoint{3.528000in}{0.795320in}}%
\pgfpathlineto{\pgfqpoint{3.550545in}{0.813976in}}%
\pgfpathlineto{\pgfqpoint{3.573091in}{0.834172in}}%
\pgfpathlineto{\pgfqpoint{3.618182in}{0.879109in}}%
\pgfpathlineto{\pgfqpoint{3.663273in}{0.929953in}}%
\pgfpathlineto{\pgfqpoint{3.708364in}{0.986505in}}%
\pgfpathlineto{\pgfqpoint{3.753455in}{1.048540in}}%
\pgfpathlineto{\pgfqpoint{3.798545in}{1.115813in}}%
\pgfpathlineto{\pgfqpoint{3.843636in}{1.188061in}}%
\pgfpathlineto{\pgfqpoint{3.888727in}{1.264996in}}%
\pgfpathlineto{\pgfqpoint{3.933818in}{1.346316in}}%
\pgfpathlineto{\pgfqpoint{3.978909in}{1.431700in}}%
\pgfpathlineto{\pgfqpoint{4.024000in}{1.520810in}}%
\pgfpathlineto{\pgfqpoint{4.091636in}{1.660691in}}%
\pgfpathlineto{\pgfqpoint{4.159273in}{1.806920in}}%
\pgfpathlineto{\pgfqpoint{4.226909in}{1.958201in}}%
\pgfpathlineto{\pgfqpoint{4.317091in}{2.165440in}}%
\pgfpathlineto{\pgfqpoint{4.610182in}{2.844705in}}%
\pgfpathlineto{\pgfqpoint{4.677818in}{2.994449in}}%
\pgfpathlineto{\pgfqpoint{4.745455in}{3.138704in}}%
\pgfpathlineto{\pgfqpoint{4.813091in}{3.276189in}}%
\pgfpathlineto{\pgfqpoint{4.858182in}{3.363479in}}%
\pgfpathlineto{\pgfqpoint{4.903273in}{3.446872in}}%
\pgfpathlineto{\pgfqpoint{4.948364in}{3.526039in}}%
\pgfpathlineto{\pgfqpoint{4.993455in}{3.600667in}}%
\pgfpathlineto{\pgfqpoint{5.038545in}{3.670462in}}%
\pgfpathlineto{\pgfqpoint{5.083636in}{3.735149in}}%
\pgfpathlineto{\pgfqpoint{5.128727in}{3.794471in}}%
\pgfpathlineto{\pgfqpoint{5.173818in}{3.848195in}}%
\pgfpathlineto{\pgfqpoint{5.218909in}{3.896109in}}%
\pgfpathlineto{\pgfqpoint{5.264000in}{3.938024in}}%
\pgfpathlineto{\pgfqpoint{5.286545in}{3.956680in}}%
\pgfpathlineto{\pgfqpoint{5.309091in}{3.973775in}}%
\pgfpathlineto{\pgfqpoint{5.331636in}{3.989293in}}%
\pgfpathlineto{\pgfqpoint{5.354182in}{4.003220in}}%
\pgfpathlineto{\pgfqpoint{5.376727in}{4.015540in}}%
\pgfpathlineto{\pgfqpoint{5.399273in}{4.026243in}}%
\pgfpathlineto{\pgfqpoint{5.421818in}{4.035316in}}%
\pgfpathlineto{\pgfqpoint{5.444364in}{4.042753in}}%
\pgfpathlineto{\pgfqpoint{5.466909in}{4.048544in}}%
\pgfpathlineto{\pgfqpoint{5.489455in}{4.052685in}}%
\pgfpathlineto{\pgfqpoint{5.512000in}{4.055171in}}%
\pgfpathlineto{\pgfqpoint{5.534545in}{4.056000in}}%
\pgfpathlineto{\pgfqpoint{5.534545in}{4.056000in}}%
\pgfusepath{stroke}%
\end{pgfscope}%
\begin{pgfscope}%
\pgfpathrectangle{\pgfqpoint{0.800000in}{0.528000in}}{\pgfqpoint{4.960000in}{3.696000in}}%
\pgfusepath{clip}%
\pgfsetrectcap%
\pgfsetroundjoin%
\pgfsetlinewidth{1.505625pt}%
\definecolor{currentstroke}{rgb}{0.172549,0.627451,0.172549}%
\pgfsetstrokecolor{currentstroke}%
\pgfsetdash{}{0pt}%
\pgfpathmoveto{\pgfqpoint{1.025455in}{2.376000in}}%
\pgfpathlineto{\pgfqpoint{1.160727in}{2.690801in}}%
\pgfpathlineto{\pgfqpoint{1.250909in}{2.895149in}}%
\pgfpathlineto{\pgfqpoint{1.318545in}{3.043208in}}%
\pgfpathlineto{\pgfqpoint{1.386182in}{3.185346in}}%
\pgfpathlineto{\pgfqpoint{1.431273in}{3.276189in}}%
\pgfpathlineto{\pgfqpoint{1.476364in}{3.363479in}}%
\pgfpathlineto{\pgfqpoint{1.521455in}{3.446872in}}%
\pgfpathlineto{\pgfqpoint{1.566545in}{3.526039in}}%
\pgfpathlineto{\pgfqpoint{1.611636in}{3.600667in}}%
\pgfpathlineto{\pgfqpoint{1.656727in}{3.670462in}}%
\pgfpathlineto{\pgfqpoint{1.701818in}{3.735149in}}%
\pgfpathlineto{\pgfqpoint{1.746909in}{3.794471in}}%
\pgfpathlineto{\pgfqpoint{1.792000in}{3.848195in}}%
\pgfpathlineto{\pgfqpoint{1.837091in}{3.896109in}}%
\pgfpathlineto{\pgfqpoint{1.882182in}{3.938024in}}%
\pgfpathlineto{\pgfqpoint{1.904727in}{3.956680in}}%
\pgfpathlineto{\pgfqpoint{1.927273in}{3.973775in}}%
\pgfpathlineto{\pgfqpoint{1.949818in}{3.989293in}}%
\pgfpathlineto{\pgfqpoint{1.972364in}{4.003220in}}%
\pgfpathlineto{\pgfqpoint{1.994909in}{4.015540in}}%
\pgfpathlineto{\pgfqpoint{2.017455in}{4.026243in}}%
\pgfpathlineto{\pgfqpoint{2.040000in}{4.035316in}}%
\pgfpathlineto{\pgfqpoint{2.062545in}{4.042753in}}%
\pgfpathlineto{\pgfqpoint{2.085091in}{4.048544in}}%
\pgfpathlineto{\pgfqpoint{2.107636in}{4.052685in}}%
\pgfpathlineto{\pgfqpoint{2.130182in}{4.055171in}}%
\pgfpathlineto{\pgfqpoint{2.152727in}{4.056000in}}%
\pgfpathlineto{\pgfqpoint{2.175273in}{4.055171in}}%
\pgfpathlineto{\pgfqpoint{2.197818in}{4.052685in}}%
\pgfpathlineto{\pgfqpoint{2.220364in}{4.048544in}}%
\pgfpathlineto{\pgfqpoint{2.242909in}{4.042753in}}%
\pgfpathlineto{\pgfqpoint{2.265455in}{4.035316in}}%
\pgfpathlineto{\pgfqpoint{2.288000in}{4.026243in}}%
\pgfpathlineto{\pgfqpoint{2.310545in}{4.015540in}}%
\pgfpathlineto{\pgfqpoint{2.333091in}{4.003220in}}%
\pgfpathlineto{\pgfqpoint{2.355636in}{3.989293in}}%
\pgfpathlineto{\pgfqpoint{2.378182in}{3.973775in}}%
\pgfpathlineto{\pgfqpoint{2.400727in}{3.956680in}}%
\pgfpathlineto{\pgfqpoint{2.423273in}{3.938024in}}%
\pgfpathlineto{\pgfqpoint{2.445818in}{3.917828in}}%
\pgfpathlineto{\pgfqpoint{2.490909in}{3.872891in}}%
\pgfpathlineto{\pgfqpoint{2.536000in}{3.822047in}}%
\pgfpathlineto{\pgfqpoint{2.581091in}{3.765495in}}%
\pgfpathlineto{\pgfqpoint{2.626182in}{3.703460in}}%
\pgfpathlineto{\pgfqpoint{2.671273in}{3.636187in}}%
\pgfpathlineto{\pgfqpoint{2.716364in}{3.563939in}}%
\pgfpathlineto{\pgfqpoint{2.761455in}{3.487004in}}%
\pgfpathlineto{\pgfqpoint{2.806545in}{3.405684in}}%
\pgfpathlineto{\pgfqpoint{2.851636in}{3.320300in}}%
\pgfpathlineto{\pgfqpoint{2.896727in}{3.231190in}}%
\pgfpathlineto{\pgfqpoint{2.964364in}{3.091309in}}%
\pgfpathlineto{\pgfqpoint{3.032000in}{2.945080in}}%
\pgfpathlineto{\pgfqpoint{3.099636in}{2.793799in}}%
\pgfpathlineto{\pgfqpoint{3.189818in}{2.586560in}}%
\pgfpathlineto{\pgfqpoint{3.482909in}{1.907295in}}%
\pgfpathlineto{\pgfqpoint{3.550545in}{1.757551in}}%
\pgfpathlineto{\pgfqpoint{3.618182in}{1.613296in}}%
\pgfpathlineto{\pgfqpoint{3.685818in}{1.475811in}}%
\pgfpathlineto{\pgfqpoint{3.730909in}{1.388521in}}%
\pgfpathlineto{\pgfqpoint{3.776000in}{1.305128in}}%
\pgfpathlineto{\pgfqpoint{3.821091in}{1.225961in}}%
\pgfpathlineto{\pgfqpoint{3.866182in}{1.151333in}}%
\pgfpathlineto{\pgfqpoint{3.911273in}{1.081538in}}%
\pgfpathlineto{\pgfqpoint{3.956364in}{1.016851in}}%
\pgfpathlineto{\pgfqpoint{4.001455in}{0.957529in}}%
\pgfpathlineto{\pgfqpoint{4.046545in}{0.903805in}}%
\pgfpathlineto{\pgfqpoint{4.091636in}{0.855891in}}%
\pgfpathlineto{\pgfqpoint{4.136727in}{0.813976in}}%
\pgfpathlineto{\pgfqpoint{4.159273in}{0.795320in}}%
\pgfpathlineto{\pgfqpoint{4.181818in}{0.778225in}}%
\pgfpathlineto{\pgfqpoint{4.204364in}{0.762707in}}%
\pgfpathlineto{\pgfqpoint{4.226909in}{0.748780in}}%
\pgfpathlineto{\pgfqpoint{4.249455in}{0.736460in}}%
\pgfpathlineto{\pgfqpoint{4.272000in}{0.725757in}}%
\pgfpathlineto{\pgfqpoint{4.294545in}{0.716684in}}%
\pgfpathlineto{\pgfqpoint{4.317091in}{0.709247in}}%
\pgfpathlineto{\pgfqpoint{4.339636in}{0.703456in}}%
\pgfpathlineto{\pgfqpoint{4.362182in}{0.699315in}}%
\pgfpathlineto{\pgfqpoint{4.384727in}{0.696829in}}%
\pgfpathlineto{\pgfqpoint{4.407273in}{0.696000in}}%
\pgfpathlineto{\pgfqpoint{4.429818in}{0.696829in}}%
\pgfpathlineto{\pgfqpoint{4.452364in}{0.699315in}}%
\pgfpathlineto{\pgfqpoint{4.474909in}{0.703456in}}%
\pgfpathlineto{\pgfqpoint{4.497455in}{0.709247in}}%
\pgfpathlineto{\pgfqpoint{4.520000in}{0.716684in}}%
\pgfpathlineto{\pgfqpoint{4.542545in}{0.725757in}}%
\pgfpathlineto{\pgfqpoint{4.565091in}{0.736460in}}%
\pgfpathlineto{\pgfqpoint{4.587636in}{0.748780in}}%
\pgfpathlineto{\pgfqpoint{4.610182in}{0.762707in}}%
\pgfpathlineto{\pgfqpoint{4.632727in}{0.778225in}}%
\pgfpathlineto{\pgfqpoint{4.655273in}{0.795320in}}%
\pgfpathlineto{\pgfqpoint{4.677818in}{0.813976in}}%
\pgfpathlineto{\pgfqpoint{4.700364in}{0.834172in}}%
\pgfpathlineto{\pgfqpoint{4.745455in}{0.879109in}}%
\pgfpathlineto{\pgfqpoint{4.790545in}{0.929953in}}%
\pgfpathlineto{\pgfqpoint{4.835636in}{0.986505in}}%
\pgfpathlineto{\pgfqpoint{4.880727in}{1.048540in}}%
\pgfpathlineto{\pgfqpoint{4.925818in}{1.115813in}}%
\pgfpathlineto{\pgfqpoint{4.970909in}{1.188061in}}%
\pgfpathlineto{\pgfqpoint{5.016000in}{1.264996in}}%
\pgfpathlineto{\pgfqpoint{5.061091in}{1.346316in}}%
\pgfpathlineto{\pgfqpoint{5.106182in}{1.431700in}}%
\pgfpathlineto{\pgfqpoint{5.151273in}{1.520810in}}%
\pgfpathlineto{\pgfqpoint{5.218909in}{1.660691in}}%
\pgfpathlineto{\pgfqpoint{5.286545in}{1.806920in}}%
\pgfpathlineto{\pgfqpoint{5.354182in}{1.958201in}}%
\pgfpathlineto{\pgfqpoint{5.444364in}{2.165440in}}%
\pgfpathlineto{\pgfqpoint{5.534545in}{2.376000in}}%
\pgfpathlineto{\pgfqpoint{5.534545in}{2.376000in}}%
\pgfusepath{stroke}%
\end{pgfscope}%
\begin{pgfscope}%
\pgfpathrectangle{\pgfqpoint{0.800000in}{0.528000in}}{\pgfqpoint{4.960000in}{3.696000in}}%
\pgfusepath{clip}%
\pgfsetrectcap%
\pgfsetroundjoin%
\pgfsetlinewidth{1.505625pt}%
\definecolor{currentstroke}{rgb}{0.839216,0.152941,0.156863}%
\pgfsetstrokecolor{currentstroke}%
\pgfsetdash{}{0pt}%
\pgfpathmoveto{\pgfqpoint{1.025455in}{4.056000in}}%
\pgfpathlineto{\pgfqpoint{1.048000in}{4.055171in}}%
\pgfpathlineto{\pgfqpoint{1.070545in}{4.052685in}}%
\pgfpathlineto{\pgfqpoint{1.093091in}{4.048544in}}%
\pgfpathlineto{\pgfqpoint{1.115636in}{4.042753in}}%
\pgfpathlineto{\pgfqpoint{1.138182in}{4.035316in}}%
\pgfpathlineto{\pgfqpoint{1.160727in}{4.026243in}}%
\pgfpathlineto{\pgfqpoint{1.183273in}{4.015540in}}%
\pgfpathlineto{\pgfqpoint{1.205818in}{4.003220in}}%
\pgfpathlineto{\pgfqpoint{1.228364in}{3.989293in}}%
\pgfpathlineto{\pgfqpoint{1.250909in}{3.973775in}}%
\pgfpathlineto{\pgfqpoint{1.273455in}{3.956680in}}%
\pgfpathlineto{\pgfqpoint{1.296000in}{3.938024in}}%
\pgfpathlineto{\pgfqpoint{1.318545in}{3.917828in}}%
\pgfpathlineto{\pgfqpoint{1.363636in}{3.872891in}}%
\pgfpathlineto{\pgfqpoint{1.408727in}{3.822047in}}%
\pgfpathlineto{\pgfqpoint{1.453818in}{3.765495in}}%
\pgfpathlineto{\pgfqpoint{1.498909in}{3.703460in}}%
\pgfpathlineto{\pgfqpoint{1.544000in}{3.636187in}}%
\pgfpathlineto{\pgfqpoint{1.589091in}{3.563939in}}%
\pgfpathlineto{\pgfqpoint{1.634182in}{3.487004in}}%
\pgfpathlineto{\pgfqpoint{1.679273in}{3.405684in}}%
\pgfpathlineto{\pgfqpoint{1.724364in}{3.320300in}}%
\pgfpathlineto{\pgfqpoint{1.769455in}{3.231190in}}%
\pgfpathlineto{\pgfqpoint{1.837091in}{3.091309in}}%
\pgfpathlineto{\pgfqpoint{1.904727in}{2.945080in}}%
\pgfpathlineto{\pgfqpoint{1.972364in}{2.793799in}}%
\pgfpathlineto{\pgfqpoint{2.062545in}{2.586560in}}%
\pgfpathlineto{\pgfqpoint{2.355636in}{1.907295in}}%
\pgfpathlineto{\pgfqpoint{2.423273in}{1.757551in}}%
\pgfpathlineto{\pgfqpoint{2.490909in}{1.613296in}}%
\pgfpathlineto{\pgfqpoint{2.558545in}{1.475811in}}%
\pgfpathlineto{\pgfqpoint{2.603636in}{1.388521in}}%
\pgfpathlineto{\pgfqpoint{2.648727in}{1.305128in}}%
\pgfpathlineto{\pgfqpoint{2.693818in}{1.225961in}}%
\pgfpathlineto{\pgfqpoint{2.738909in}{1.151333in}}%
\pgfpathlineto{\pgfqpoint{2.784000in}{1.081538in}}%
\pgfpathlineto{\pgfqpoint{2.829091in}{1.016851in}}%
\pgfpathlineto{\pgfqpoint{2.874182in}{0.957529in}}%
\pgfpathlineto{\pgfqpoint{2.919273in}{0.903805in}}%
\pgfpathlineto{\pgfqpoint{2.964364in}{0.855891in}}%
\pgfpathlineto{\pgfqpoint{3.009455in}{0.813976in}}%
\pgfpathlineto{\pgfqpoint{3.032000in}{0.795320in}}%
\pgfpathlineto{\pgfqpoint{3.054545in}{0.778225in}}%
\pgfpathlineto{\pgfqpoint{3.077091in}{0.762707in}}%
\pgfpathlineto{\pgfqpoint{3.099636in}{0.748780in}}%
\pgfpathlineto{\pgfqpoint{3.122182in}{0.736460in}}%
\pgfpathlineto{\pgfqpoint{3.144727in}{0.725757in}}%
\pgfpathlineto{\pgfqpoint{3.167273in}{0.716684in}}%
\pgfpathlineto{\pgfqpoint{3.189818in}{0.709247in}}%
\pgfpathlineto{\pgfqpoint{3.212364in}{0.703456in}}%
\pgfpathlineto{\pgfqpoint{3.234909in}{0.699315in}}%
\pgfpathlineto{\pgfqpoint{3.257455in}{0.696829in}}%
\pgfpathlineto{\pgfqpoint{3.280000in}{0.696000in}}%
\pgfpathlineto{\pgfqpoint{3.302545in}{0.696829in}}%
\pgfpathlineto{\pgfqpoint{3.325091in}{0.699315in}}%
\pgfpathlineto{\pgfqpoint{3.347636in}{0.703456in}}%
\pgfpathlineto{\pgfqpoint{3.370182in}{0.709247in}}%
\pgfpathlineto{\pgfqpoint{3.392727in}{0.716684in}}%
\pgfpathlineto{\pgfqpoint{3.415273in}{0.725757in}}%
\pgfpathlineto{\pgfqpoint{3.437818in}{0.736460in}}%
\pgfpathlineto{\pgfqpoint{3.460364in}{0.748780in}}%
\pgfpathlineto{\pgfqpoint{3.482909in}{0.762707in}}%
\pgfpathlineto{\pgfqpoint{3.505455in}{0.778225in}}%
\pgfpathlineto{\pgfqpoint{3.528000in}{0.795320in}}%
\pgfpathlineto{\pgfqpoint{3.550545in}{0.813976in}}%
\pgfpathlineto{\pgfqpoint{3.573091in}{0.834172in}}%
\pgfpathlineto{\pgfqpoint{3.618182in}{0.879109in}}%
\pgfpathlineto{\pgfqpoint{3.663273in}{0.929953in}}%
\pgfpathlineto{\pgfqpoint{3.708364in}{0.986505in}}%
\pgfpathlineto{\pgfqpoint{3.753455in}{1.048540in}}%
\pgfpathlineto{\pgfqpoint{3.798545in}{1.115813in}}%
\pgfpathlineto{\pgfqpoint{3.843636in}{1.188061in}}%
\pgfpathlineto{\pgfqpoint{3.888727in}{1.264996in}}%
\pgfpathlineto{\pgfqpoint{3.933818in}{1.346316in}}%
\pgfpathlineto{\pgfqpoint{3.978909in}{1.431700in}}%
\pgfpathlineto{\pgfqpoint{4.024000in}{1.520810in}}%
\pgfpathlineto{\pgfqpoint{4.091636in}{1.660691in}}%
\pgfpathlineto{\pgfqpoint{4.159273in}{1.806920in}}%
\pgfpathlineto{\pgfqpoint{4.226909in}{1.958201in}}%
\pgfpathlineto{\pgfqpoint{4.317091in}{2.165440in}}%
\pgfpathlineto{\pgfqpoint{4.610182in}{2.844705in}}%
\pgfpathlineto{\pgfqpoint{4.677818in}{2.994449in}}%
\pgfpathlineto{\pgfqpoint{4.745455in}{3.138704in}}%
\pgfpathlineto{\pgfqpoint{4.813091in}{3.276189in}}%
\pgfpathlineto{\pgfqpoint{4.858182in}{3.363479in}}%
\pgfpathlineto{\pgfqpoint{4.903273in}{3.446872in}}%
\pgfpathlineto{\pgfqpoint{4.948364in}{3.526039in}}%
\pgfpathlineto{\pgfqpoint{4.993455in}{3.600667in}}%
\pgfpathlineto{\pgfqpoint{5.038545in}{3.670462in}}%
\pgfpathlineto{\pgfqpoint{5.083636in}{3.735149in}}%
\pgfpathlineto{\pgfqpoint{5.128727in}{3.794471in}}%
\pgfpathlineto{\pgfqpoint{5.173818in}{3.848195in}}%
\pgfpathlineto{\pgfqpoint{5.218909in}{3.896109in}}%
\pgfpathlineto{\pgfqpoint{5.264000in}{3.938024in}}%
\pgfpathlineto{\pgfqpoint{5.286545in}{3.956680in}}%
\pgfpathlineto{\pgfqpoint{5.309091in}{3.973775in}}%
\pgfpathlineto{\pgfqpoint{5.331636in}{3.989293in}}%
\pgfpathlineto{\pgfqpoint{5.354182in}{4.003220in}}%
\pgfpathlineto{\pgfqpoint{5.376727in}{4.015540in}}%
\pgfpathlineto{\pgfqpoint{5.399273in}{4.026243in}}%
\pgfpathlineto{\pgfqpoint{5.421818in}{4.035316in}}%
\pgfpathlineto{\pgfqpoint{5.444364in}{4.042753in}}%
\pgfpathlineto{\pgfqpoint{5.466909in}{4.048544in}}%
\pgfpathlineto{\pgfqpoint{5.489455in}{4.052685in}}%
\pgfpathlineto{\pgfqpoint{5.512000in}{4.055171in}}%
\pgfpathlineto{\pgfqpoint{5.534545in}{4.056000in}}%
\pgfpathlineto{\pgfqpoint{5.534545in}{4.056000in}}%
\pgfusepath{stroke}%
\end{pgfscope}%
\begin{pgfscope}%
\pgfsetrectcap%
\pgfsetmiterjoin%
\pgfsetlinewidth{0.803000pt}%
\definecolor{currentstroke}{rgb}{0.000000,0.000000,0.000000}%
\pgfsetstrokecolor{currentstroke}%
\pgfsetdash{}{0pt}%
\pgfpathmoveto{\pgfqpoint{0.800000in}{0.528000in}}%
\pgfpathlineto{\pgfqpoint{0.800000in}{4.224000in}}%
\pgfusepath{stroke}%
\end{pgfscope}%
\begin{pgfscope}%
\pgfsetrectcap%
\pgfsetmiterjoin%
\pgfsetlinewidth{0.803000pt}%
\definecolor{currentstroke}{rgb}{0.000000,0.000000,0.000000}%
\pgfsetstrokecolor{currentstroke}%
\pgfsetdash{}{0pt}%
\pgfpathmoveto{\pgfqpoint{5.760000in}{0.528000in}}%
\pgfpathlineto{\pgfqpoint{5.760000in}{4.224000in}}%
\pgfusepath{stroke}%
\end{pgfscope}%
\begin{pgfscope}%
\pgfsetrectcap%
\pgfsetmiterjoin%
\pgfsetlinewidth{0.803000pt}%
\definecolor{currentstroke}{rgb}{0.000000,0.000000,0.000000}%
\pgfsetstrokecolor{currentstroke}%
\pgfsetdash{}{0pt}%
\pgfpathmoveto{\pgfqpoint{0.800000in}{0.528000in}}%
\pgfpathlineto{\pgfqpoint{5.760000in}{0.528000in}}%
\pgfusepath{stroke}%
\end{pgfscope}%
\begin{pgfscope}%
\pgfsetrectcap%
\pgfsetmiterjoin%
\pgfsetlinewidth{0.803000pt}%
\definecolor{currentstroke}{rgb}{0.000000,0.000000,0.000000}%
\pgfsetstrokecolor{currentstroke}%
\pgfsetdash{}{0pt}%
\pgfpathmoveto{\pgfqpoint{0.800000in}{4.224000in}}%
\pgfpathlineto{\pgfqpoint{5.760000in}{4.224000in}}%
\pgfusepath{stroke}%
\end{pgfscope}%
\begin{pgfscope}%
\definecolor{textcolor}{rgb}{0.000000,0.000000,0.000000}%
\pgfsetstrokecolor{textcolor}%
\pgfsetfillcolor{textcolor}%
\pgftext[x=3.280000in,y=4.307333in,,base]{\color{textcolor}\rmfamily\fontsize{12.000000}{14.400000}\selectfont Plot del seno e del coseno da 0 a 2 pi}%
\end{pgfscope}%
\begin{pgfscope}%
\pgfsetbuttcap%
\pgfsetmiterjoin%
\definecolor{currentfill}{rgb}{1.000000,1.000000,1.000000}%
\pgfsetfillcolor{currentfill}%
\pgfsetfillopacity{0.800000}%
\pgfsetlinewidth{1.003750pt}%
\definecolor{currentstroke}{rgb}{0.800000,0.800000,0.800000}%
\pgfsetstrokecolor{currentstroke}%
\pgfsetstrokeopacity{0.800000}%
\pgfsetdash{}{0pt}%
\pgfpathmoveto{\pgfqpoint{0.897222in}{0.597444in}}%
\pgfpathlineto{\pgfqpoint{1.917285in}{0.597444in}}%
\pgfpathquadraticcurveto{\pgfqpoint{1.945063in}{0.597444in}}{\pgfqpoint{1.945063in}{0.625222in}}%
\pgfpathlineto{\pgfqpoint{1.945063in}{1.028000in}}%
\pgfpathquadraticcurveto{\pgfqpoint{1.945063in}{1.055778in}}{\pgfqpoint{1.917285in}{1.055778in}}%
\pgfpathlineto{\pgfqpoint{0.897222in}{1.055778in}}%
\pgfpathquadraticcurveto{\pgfqpoint{0.869444in}{1.055778in}}{\pgfqpoint{0.869444in}{1.028000in}}%
\pgfpathlineto{\pgfqpoint{0.869444in}{0.625222in}}%
\pgfpathquadraticcurveto{\pgfqpoint{0.869444in}{0.597444in}}{\pgfqpoint{0.897222in}{0.597444in}}%
\pgfpathlineto{\pgfqpoint{0.897222in}{0.597444in}}%
\pgfpathclose%
\pgfusepath{stroke,fill}%
\end{pgfscope}%
\begin{pgfscope}%
\pgfsetrectcap%
\pgfsetroundjoin%
\pgfsetlinewidth{1.505625pt}%
\definecolor{currentstroke}{rgb}{0.121569,0.466667,0.705882}%
\pgfsetstrokecolor{currentstroke}%
\pgfsetdash{}{0pt}%
\pgfpathmoveto{\pgfqpoint{0.925000in}{0.944667in}}%
\pgfpathlineto{\pgfqpoint{1.063889in}{0.944667in}}%
\pgfpathlineto{\pgfqpoint{1.202778in}{0.944667in}}%
\pgfusepath{stroke}%
\end{pgfscope}%
\begin{pgfscope}%
\definecolor{textcolor}{rgb}{0.000000,0.000000,0.000000}%
\pgfsetstrokecolor{textcolor}%
\pgfsetfillcolor{textcolor}%
\pgftext[x=1.313889in,y=0.896056in,left,base]{\color{textcolor}\rmfamily\fontsize{10.000000}{12.000000}\selectfont seno(x)}%
\end{pgfscope}%
\begin{pgfscope}%
\pgfsetrectcap%
\pgfsetroundjoin%
\pgfsetlinewidth{1.505625pt}%
\definecolor{currentstroke}{rgb}{1.000000,0.498039,0.054902}%
\pgfsetstrokecolor{currentstroke}%
\pgfsetdash{}{0pt}%
\pgfpathmoveto{\pgfqpoint{0.925000in}{0.736333in}}%
\pgfpathlineto{\pgfqpoint{1.063889in}{0.736333in}}%
\pgfpathlineto{\pgfqpoint{1.202778in}{0.736333in}}%
\pgfusepath{stroke}%
\end{pgfscope}%
\begin{pgfscope}%
\definecolor{textcolor}{rgb}{0.000000,0.000000,0.000000}%
\pgfsetstrokecolor{textcolor}%
\pgfsetfillcolor{textcolor}%
\pgftext[x=1.313889in,y=0.687722in,left,base]{\color{textcolor}\rmfamily\fontsize{10.000000}{12.000000}\selectfont coseno(x)}%
\end{pgfscope}%
\end{pgfpicture}%
\makeatother%
\endgroup%
}
    \end{center}
    \caption{\emph{Seno} e \emph{Cosenoi} : nell'asse delle ascisse viene rappresentata la durata in millisecondi mentre nell'asse delle ordinate l'ampiezza del segnale.}
\end{figure}

\subsection{Radianti}

L'argomento di una funzione seno o coseno è un angolo che viene normalmente espresso in radianti, dove il radiante è una misura adimensionale che rappresenta il rapporto tra la lunghezza di un arco di circonferenza definito da un dato angolo e il raggio della stessa circonferenza. Consideriamo l'arco definito da un angolo di 360°: in questo caso, l'arco corrisponde all'intera circonferenza, la cui lunghezza $c$ possiamo calcolare a partire dal raggio $r$ come
\begin{equation}
c = 2 \pi r
\end{equation}
e quindi l'angolo in radianti corrispondente a 360° sarà
\begin{equation}
360^{\circ} = \frac{2 \pi r}{r} = 2 \pi
\end{equation}

Altri misure di angoli che incontreremo spesso sono
\begin{equation}
90^{\circ} = \frac{\pi}{2}
\end{equation}
e
\begin{equation}
180^{\circ} = \pi
\end{equation}

Nota che un angolo espresso in radianti non porta un'unità di misura.


\subsection{Relazione tra seni e coseni}

Il seno e il coseno hanno esattamente lo stesso andamento, traslato di un quarto di cerchio (cioè 90°). È facile immaginare il perché: un cerchio è sempre ``fatto nello stesso modo'' comunque io lo ruoti, e ruotando tutta la figura di 90° quello che prima era l'asse dell'ordinata si ritrova ora sull'asse dell'ascissa. Quindi
\begin{equation}
\cos \alpha = \sin(\alpha + \frac{\pi}{2})
\end{equation}


\subsection{Codominio}

Il codominio del seno e del coseno --- cioè, l'ambito di valori $y$ che una funzione seno o coseno $y = sin x$ o $y = cos x$ può assumere --- è compreso tra -1 e 1. Anche questo è facile da vedere, per via del fatto che la circonferenza goniometrica ha raggio 1 e quindi tutti i suoi punti sono compresi tra -1 e 1 sia sull'ascissa che sull'ordinata. Quindi, per qualsiasi angolo $x$

\begin{equation}
-1 \leq \sin x \leq 1
\end{equation}
e
\begin{equation}
-1 \leq \cos x \leq 1
\end{equation}

Di conseguenza, l'ampiezza picco-picco delle funzioni seno e coseno è 2.


\subsection{Dominio e periodicità}

Il dominio del seno e del coseno è infinito: angoli più grandi di un giro (cioè di $2\pi$) sono possibili e ben definiti, e i punti che questi definiscono sulla circonferenza corrispondono a punti definiti da angoli più piccoli del giro. Possiamo dire che, continuando a far ruotare il raggio dopo che ha compiuto un giro, questo continua a ripercorrere sempre lo stesso percorso. Possiamo estendere il ragionamento ad angoli negativi, che possono essere visti come angoli calcolati facendo ruotare il raggio in senso orario anziché antiorario.

Questa considerazione in definitiva spiega perché seno e coseno sono funzioni periodiche, che si ripetono uguali a sé stesse a ogni giro di circonferenza, cioè ogni $2 \pi$ radianti. Quindi
\begin{equation}
\forall n \in \mathbb{Z}, \sin(x + 2 \pi n) = \sin x
\end{equation}
e
\begin{equation}
\forall n \in \mathbb{Z}, \cos(x + 2 \pi n) = \cos x
\end{equation}

Per questa ragione, di solito si preferisce utilizzare come argomenti del seno e del coseno valori compresi tra 0 e $2 \pi$ o, talvolta, tra $-\pi$ e $\pi$, a meno che non ci siano ragioni forti per fare altrimenti.





\section{Sinusoidi}

Si può generalizzare la nozione di seno e coseno in quella più astratta di sinusoide. Una sinusoide è una qualsiasi funzione che abbia l'andamento di un seno o di un coseno, ma che abbia un periodo potenzialmente diverso da $2 \pi$; che sia potenzialmente traslata lungo l'asse orizzontale; e la cui ampiezza picco-picco sia potenzialmente diversa da 2, fermo restando che il codominio dev'essere simmetrico rispetto allo 0.

\subsection{Frequenza}

Il termine della frequenza controlla la periodicità della funzione. Ad esempio, possiamo definire una funzione sinusoidale con frequenza 1 (cioè che si ripete esattamente una volta per ogni incremento unitario sull'asse $x$) come
\begin{equation}
s(x) = \sin(2 \pi x)
\end{equation}

Più in generale, possiamo definire una funzione sinusoidale con frequenza $f$ come
\begin{equation}
s(x) = \sin({2 \pi} f x)
\end{equation}

Poiché, nel caso specifico del discorso sul suono, $x$ rappresenta generalmente il tempo, spesso scriviamo
\begin{equation}
s(t) = \sin({2 \pi} f t)
\end{equation}

Alcuni testi, come il Puckette, preferiscono parlare di velocità angolare anziché di frequenza, introducendo un termine
\begin{equation}
\omega = 2 \pi f
\end{equation}
per cui
\begin{equation}
f = \frac{\omega}{2 \pi}
\end{equation}

$\omega$ rappresenta non il periodo, cioè i cicli per unità di tempo, ma i radianti per unità di tempo. Questo rende più semplice ma meno tangibile l'equazione della sinusoide, che diventa
\begin{equation}
s(t) = \sin(\omega t)
\end{equation}

La funzione sinusoidale è considerata, citando Curtis Roads, \emph{sub specie \ae{}ternitatis}: la sua espressione non contiene un termine che rappresenti una sua durata nel tempo. Dal momento che il suo dominio non è limitato, una sinusoide è un fenomeno che non ha un inizio e non ha una fine.


\subsection{Ampiezza}

L'ampiezza di una sinusoide può essere espressa tramite un coefficiente che moltiplica il seno:
\begin{equation}
s(t) = k \cdot \sin(2 \pi f t)
\end{equation}

Il valore assoluto di questo coefficiente rappresenta l'ampiezza di picco della sinusoide; il doppio del valore assoluto, $2k$, ne rappresenta l'ampiezza picco-picco. Se $k$ è negativo, la sinusoide si ritrova rispecchiata rispetto all'asse orizzontale. Se $k$ è 0 stiamo rappresentando il silenzio, e il termine della frequenza è ininfluente e perde significato (come pure quello della fase che vedremo tra poco).


\subsection{Fase}

Aggiungendo all'argomento un termine fisso $\phi$, detto \emph{fase}, si può traslare l'intera funzione sinusoidale sull'asse orizzontale:
\begin{equation}
s(t) = k \cdot \sin(2 \pi f t + \phi)
\end{equation}
In generale, il termine di fase è considerato compreso tra 0 e $2 \pi$ o tra $-\pi$ e $\pi$: qualsiasi fase fuori dall'ambito di riferimento è equivalente a una fase entro esso, dal momento che la sinusoide ha un periodo di $2 \pi$. Il rovesciamento rispetto all'asse orizzontale prodotto da un coefficiente $k$ negativo è equivalente a una rotazione di fase di $\pi$, cioè di mezzo giro:
\begin{equation}
k \cdot \sin(2 \pi f t + \phi) = -k \cdot \sin(2 \pi f t + \phi + \pi) = -k \cdot \sin(2 \pi f t + \phi - \pi)
\end{equation}

Inoltre, una rotazione di fase di $\frac{\pi}{2}$ (un quarto di giro, o 90°) corrisponde alla trasformazione della funzione seno in una funzione coseno:
\begin{equation}
k \cdot \sin(2 \pi f t + \phi) = k \cdot \cos(2 \pi f t + \phi - \frac{\pi}{2})
\end{equation}


\subsection{Somme di sinusoidi alla stessa frequenza}

Sommando due sinusoidi alla stessa frequenza si ottiene un'altra sinusoide, generalmente di ampiezza e fase diversa ma con la stessa frequenza:
\begin{equation}
k_1 \cdot \sin(2 \pi f t + \phi_1) +  k_2 \cdot \sin(2 \pi f t + \phi_2) = k \cdot \sin(2 \pi f t + \phi)
\end{equation}

Inoltre, una sinusoide con qualsiasi fase può essere espressa come la somma di una sinusoide seno e una sinusoide coseno con fase nulla (cioè, di una sinusoide seno con fase nulla e una con termine di fase ruotato di $\pi$) alla stessa frequenza della sinusoide originale:
\begin{equation}
k \cdot \sin(2 \pi f t + \phi) = k_1 \cdot \sin(2 \pi f t) + k_2 \cdot \cos(2 \pi f t)
\end{equation}

Se vuoi approfondire queste due formule, ti consiglio di guardare qui: \url{https://www.dsprelated.com/showarticle/635.php}.



\section{Percezione della frequenza, spazio delle altezze}

La frequenza fondamentale è il parametro essenziale legato alla percezione d'altezza di un suono. In linea di principio, quindi, il concetto di altezza può essere applicato soltanto a suoni periodici. Nella pratica, tuttavia, perché la percezione umana possa attribuire un'altezza a un suono, non serve che questo sia esattamente periodico --- e, naturalmente, non serve che sia illimitato nel tempo: è sufficiente solo una periodicità molto approssimativa ed estesa a un tempo relativamente breve (in generale, qualcosa attorno al decimo di secondo --- di più per suoni gravi, talvolta meno per suoni acuti).

Nella nostra percezione dell'altezza è fondamentale il concetto di intervallo: l'intervallo è, semplificando il concetto, ciò che percepiamo come la ``distanza'' tra due altezze diverse. La riconoscibilità di una struttura melodica o armonica non è tanto data dalle altezze in termini assoluti, ma dalla configurazione (nel tempo, per la melodia; nella simultaneità, per l'armonia) dei suoi intervalli. L'intervallo, da parte sua, può essere definito come un particolare rapporto tra frequenze. Per ragioni storiche e teoriche molto lontane da ciò di cui ci occupiamo qui, certi rapporti semplici hanno nomi specifici: per esempio, un rapporto di frequenze 2:1 è chiamato ottava; un rapporto 3:2 è chiamato quinta; un rapporto 4:3 è chiamato quarta; 5:4 è chiamato terza maggiore; 6:5 terza minore; 9:8 seconda maggiore.%
%\footnote{Nella prassi moderna, nessuno di questi intervalli viene associato esattamente a questi rapporti, ma ad altri rapporti vicini a questi che hanno particolari proprietà matematiche e combinatorie. Di questo abbiamo parlato in Tecniche Compositive 1, e dovremmo parlare in un'altra dispensa.}

Il fatto che nominiamo le distanze tra le altezze in base ai loro rapporti implica che l'organizzazione dell'altezza rispetto alla scala delle frequenze sia di tipo logaritmico, in maniera non dissimile --- ma in realtà molto più precisa nella nostra percezione --- a quanto abbiamo visto a proposito delle ampiezze. Questo vuol dire, tra l'altro, che lo stesso intervallo corrisponderà a una differenza frequenziale molto più piccola nel grave che nell'acuto. Per esempio, la quinta a partire da una frequenza di 110 Hz si trova a 165 Hz,%
\footnote{Questo corrisponde all'intervallo la-mi in chiave di basso.}
e quindi la differenza tra le due frequenze coinvolte è di 55 Hz; d'altra parte, la quinta a partire da una frequenza di 440 Hz si trova a 660 Hz,
\footnote{Questo corrisponde all'intervallo la-mi in chiave di violino.}
con una differenza frequenziale di 220 Hz.

Possiamo quindi dare una preliminare e parziale definizione di \emph{spazio delle altezze}, in relazione logaritmica con lo spazio delle frequenze. Per approfondire tale relazione bisognerebbe addentrarsi nella teoria dei temperamenti e dei sistemi di accordatura. Per il momento, riporto senza dimostrarli solo alcuni punti basilari che stanno alla base della teoria musicale e dell'organologia occidentale moderna:

\begin{itemize}

\item Gli unici intervalli che, nella più diffusa prassi moderna,%
\footnote {Cioè, secondo il temperamento equabile, utilizzato in maniera quasi ubiqua nella maggior parte della musica prodotta a partire dall'inizio del XIX secolo.}
mantengono il rapporto di frequenza semplice enunciato sopra (che deriva da proprietà acustiche fondamentali del suono e che è alla base di molti temperamenti antichi e non solo) sono l'ottava, con rapporto 2:1, e i suoi multipli.

\item L'ottava è divisa in 12 semitoni, tutti definiti dallo stesso rapporto frequenziale (sono i cosiddetti \emph{semitoni temperati}). Tale rapporto frequenziale è pari a $\sqrt[12]{2}$.

\item Tutti gli altri intervalli vengono approssimati elevando a potenza il rapporto del semitono. Per esempio, la seconda maggiore, composta da due semitoni, ha rapporto pari a $(\sqrt[12]{2})^2 = \sqrt[6]{2} \approx 1.122$, un valore leggermente più piccolo di $\frac{9}{8} = 1.125$; la terza maggiore, composta da quattro semitoni, ha rapporto pari a $(\sqrt[12]{2})^4 = \sqrt[3]{2} \approx 1.26$, un valore leggermente più grande di $\frac{5}{4} = 1.25$; l'ottava, coerentemente con la sua definizione data sopra, ha rapporto pari a $(\sqrt[12]{2})^{12} = 2$.

\end{itemize}




\section{Spettro di frequenza}

\subsection{Teorema di Fourier e rappresentazione nel dominio del tempo}

Il \emph{teorema di Fourier} ci dice che qualsiasi segnale può essere rappresentato come una somma di componenti sinusoidali, ciascuna caratterizzata dalla sua frequenza, ampiezza e fase. Questo vuol dire, in senso proprio, che sommando algebricamente tra loro sinusoidi (potenzialmente in numero infinito) con specifiche frequenze, ampiezze e fasi è possibile ottenere qualsiasi segnale, di qualsiasi complessità. Chiamiamo queste sinusoidi le \emph{componenti}, o \emph{parziali}, del segnale originale. Dire che una componente non è presente in un segnale equivale a dire che la sua ampiezza è nulla.

Ci sono alcune ricadute importanti di questa affermazione. In primo luogo, il teorema di Fourier non parla della durata delle componenti: l'espressione matematica di una sinusoide assume che questa sia infinita nel tempo, dal momento che il suo dominio corrisponde all'intero insieme dei numeri reali. In secondo luogo, lo stesso teorema non dice che il segnale debba avere caratteristiche di staticità o periodicità nel tempo: qualsiasi fenomeno, in tutta la sua evoluzione temporale, può essere rappresentato in termini di somma di sinusoidi definite in termini di frequenza, ampiezza e fase. In generale, però, è possibile che questa rappresentazione richieda di considerare sinusoidi a frequenza estremamente bassa.


\subsection{Segnali periodici, armoniche}

Esiste un caso particolare del teorema di Fourier, che si applica a segnali periodici: questi possono essere rappresentati da una somma di sinusoidi le cui frequenze sono tutte multiple della frequenza fondamentale del segnale.%
\footnote{Questo equivale a dire che le componenti a frequenze diverse dai multipli della fondamentale hanno tutte ampiezza nulla.}
L'ampiezza e la fase di ciascuna sinusoide determinano la forma d'onda del segnale originale. Tutte queste sinusoidi sono dette \emph{armoniche} del segnale; diremo che le loro frequenze si trovano in rapporto armonico tra loro, e che il segnale nel suo complesso è armonico. È vero anche il contrario: qualsiasi somma di sinusoidi le cui frequenze siano tutte multiple di una frequenza data (cioè siano tutte in rapporto armonico tra loro) produrrà un segnale periodico, che avrà quest'ultima frequenza come fondamentale. Dunque, \emph{tutti e soli i segnali armonici sono periodici} e \emph{tutti e soli i segnali periodici sono armonici}.

Tra l'altro, i rapporti interi tra le armoniche sono alla base dei rapporti frequenziali semplici corrispondenti agli intervalli. Inoltre, il fatto che l'armonicità di un fenomeno sonoro (qualità di estrema rilevanza e riconoscibilità percettiva) corrisponda a una sua strutturazione definita dai rapporti di frequenza delle sue parziali ha una relazione evidente con il fatto che la nostra percezione classifica le strutture di frequenza in base ai loro rapporti.

Possiamo considerare le armoniche come un caso particolare delle parziali di un segnale, e in generale non è sbagliato chiamarle comunque ``parziali'' o ``componenti'', a meno che questo non generi confusione.

Convenzionalmente, a ciascuna armonica viene associato un indice che corrisponde al rapporto tra la sua frequenza e quella della fondamentale --- cioè, all'intero che, moltiplicando la frequenza della fondamentale, produce la frequenza dell'armonica. Quindi la seconda armonica avrà frequenza doppia rispetto alla fondamentale; la terza armonica frequenza tripla; e così via. Secondo questa convenzione, la prima armonica coinciderà con la fondamentale.

Esiste un'altra convenzione adottata soprattutto da testi più vecchi per cui l'armonica di frequenza doppia rispetto alla fondamentale è chiamata prima armonica; quella di frequenza tripla è chiamata seconda armonica; e così via. Da questa convenzione deriva quella, di uso frequente, per cui la fondamentale viene chiamata $f_0$, anche in contesti in cui questa viene considerata la prima armonica.






\subsection{Rappresentazione nel dominio della frequenza}

La rappresentazione di un segnale in termini delle sinusoidi che lo compongono è detta \emph{nel dominio della frequenza}, e si contrappone alla rappresentazione \emph{nel dominio del tempo} che è quella per cui il segnale è rappresentato in termini delle variazioni di ampiezza nel tempo. In senso stretto, ogni segnale, qualunque sia la durata, può essere descritto in termini di una somma di infinite sinusoidi. 

L'operazione attraverso la quale si ottiene la rappresentazione nel dominio della frequenza o \emph{spettro} a partire da quella del dominio del tempo è la \emph{trasformata di Fourier} (\emph{Fourier transform}, o \emph{FT}). Questo termine indica anche il risultato dell'operazione, per cui possiamo dire che la rappresentazione nel dominio della frequenza è la trasformata di Fourier della rappresentazione nel dominio del tempo. 

La trasformata di Fourier è un'operazione invertibile: dalla rappresentazione nel dominio della frequenza possiamo ottenere quella nel dominio del tempo applicando alla prima la \emph{trasformata inversa di Fourier}. In senso proprio, la trasformata di Fourier si svolge su una rappresentazione analitica del segnale: cioè, è un'operazione che prende un'equazione che descrive il segnale come una funzione del tempo $s(t)$ su dominio infinito e restituisce un'altra equazione che descrive il segnale come una funzione della frequenza $F(f)$ su dominio infinito. Il codominio di questa funzione $F(f)$ è l'insieme dei numeri complessi $\mathbb{C}$. Questo vuol dire che il valore della funzione $F(f)$, corrispondente alla trasformata di Fourier del segnale originale $s(t)$, per una data frequenza $f$ è un numero complesso; questo rappresenta in maniera compatta l'ampiezza e la fase della sinusoide a frequenza $f$ che, sommata a tutte le altre sinusoidi con ampiezze e fasi ottenute per tutte le altre possibili frequenze, produrrà un segnale identico a quello originale. L'ampiezza di cui stiamo parlando qui è, di fatto, il termine di ampiezza dell'equazione sinusoidale presentata sopra, che abbiamo chiamato $k$; possiamo quindi considerarlo come l'ampiezza di picco non segnata della corrispondente sinusoide.

Definiamo formalmente lo spettro come la distribuzione delle ampiezze e delle fasi delle componenti sinusoidali rispetto alla frequenza. In generale, la distribuzione a cui siamo più interessati è quella delle ampiezze, dal momento che la nostra percezione non distingue la fase delle componenti, se non in casi molto particolari. Questa distribuzione può essere rappresentata graficamente come un diagramma cartesiano con le frequenze sull'asse orizzontale e le ampiezze sull'asse verticale. Una singola sinusoide avrà allora una rappresentazione grafica costituita da un solo punto. La somma di due sinusoidi sarà rappresentata da una coppia di punti. Segnali puramente armonici saranno rappresentati come un insieme di punti a posizioni equidistanti sull'asse delle ascisse. Discuteremo più avanti alcune altre categorie fondamentali di configurazioni spettrali, e le relazioni tra le loro rappresentazioni nel dominio della frequenza e del tempo.


 
\subsection{Considerazioni percettive, STFT}

La nostra percezione del suono è basata su una rappresentazione ibrida: da una parte abbiamo la capacità di distinguere timbri diversi indicando le loro caratteristiche spettrali (più scuro, più brillante, più nasale; o anche, per chi è allenato a distinguere e nominare questi dettagli, più ricco di armoniche pari, o dispari, o di alcune armoniche specifiche). Siamo anche in grado di distinguere frequenze e timbri diversi prodotti simultaneamente da corpi diversi (più note suonate simultaneamente da un pianoforte o una chitarra, più voci che parlano o cantano simultaneamente, la compresenza di suoni di origine differente). Alla base di questo tipo di percezione c'è una rappresentazione cognitiva del suono nel dominio della frequenza: il nostro orecchio interno contiene cellule acustiche sensibili a frequenze diverse, e queste cellule compiono un'operazione concettualmente simile a una trasformata di Fourier, salvo il fatto che l'ambito frequenziale a cui esse sono sensibili è limitato a una banda passante compresa approssimativamente (e ottimisticamente) tra i 20 Hz e i 20 kHz.

D'altra parte, la nostra percezione del suono avviene anche in termini di eventi che si susseguono nel tempo --- fonemi, note, articolazioni interne del suono. Questo avviene in particolare per eventi la cui frequenza di produzione è più bassa di 20 Hz. Da questo punto di vista, possiamo dire che nessun fenomeno sonoro che percepiamo è statico: in generale, attribuiremo un inizio e una fine a ciascun fenomeno sonoro; inoltre, le sue qualità ci appariranno come variabili nel tempo: una corda pizzicata, ad esempio, produce un suono che inizia forte e brillante e si estingue lentamente diventando progressivamente più flebile e scuro.%
\footnote{``Brillante'' e ``scuro'' sono sinestesie che fanno riferimento rispettivamente alla maggiore o minore ampiezza delle componenti spettrali acute di un suono.}
Se è vero che la trasformata di Fourier, nella sua formulazione originale e più astratta, tratta il segnale codificando in termini frequenziali il suo comportamento lungo qualsiasi scala temporale, è anche vero che la nostra percezione funziona in maniera molto diversa e, per avere una rappresentazione utile del fenomeno sonoro, è necessario tenerne conto.

Intuitivamente, la dimensione temporale del suono ci appare perfettamente contrapposta rispetto a quella frequenziale e indipendente da essa (potremmo dire \emph{ortogonale}). Una partitura tradizionale ci offre una rappresentazione simbolica di questo paradigma di cognizione del suono: sull'asse orizzontale il tempo, su quello verticale la frequenza --- o, nel caso di più suoni simultanei, le frequenze. Non deve però mai sfuggirci il fatto fondamentale per cui queste due dimensioni percettivamente contrapposte, il tempo e la frequenza, sono in realtà due maniere diverse di percepire fenomeni appartenenti alla stessa categoria, cioè variazioni della pressione dell'aria nel tempo.

Possiamo allora dire che la nostra percezione avviene secondo una rappresentazione frequenziale \emph{a breve termine}: nel breve periodo, cioè per frequenze relativamente alte, si comporta come una rappresentazione nel dominio della frequenza; nel lungo periodo, cioè per frequenze relativamente basse, si comporta come una rappresentazione nel dominio del tempo.

Esiste una variante della trasformata di Fourier, chiamata \emph{trasformata a breve termine di Fourier} (\emph{short-term Fourier transform} o \emph{STFT}), che restituisce una scomposizione sinusoidale di un segnale limitato nel tempo (e, di conseguenza, anche nella frequenza). Non entriamo ora nei suoi dettagli: possiamo dire che, anche se dal punto di vista matematico il suo risultato è \emph{esatto}, questo in generale non coincide però esattamente con la nostra percezione cognitiva del suono, che avviene secondo meccanismi molto più complessi della STFT. Tuttavia, spesso ne fornisce un'approssimazione utile alla descrizione del fenomeno sonoro, e di fatto i vari analizzatori spettrali che utilizziamo sono basati il più delle volte su questo strumento matematico.%
\footnote{In informatica musicale si parla spesso di \emph{fast Fourier transform}, o \emph{FFT}, che è un algoritmo per il calcolo della STFT. Un altro termine che si incontra spesso è \emph{discrete Fourier transform}, o \emph{DFT}, che fa riferimento esplicito al fatto che il calcolo viene svolto su una sequenza di campioni rilevati a intervalli temporali successivi. In senso proprio, la STFT è una DFT applicata a una porzione temporale breve (o ``finestra'') di un segnale.}
Anche la STFT restituisce numeri complessi, che rappresentano i termini di ampiezza non segnata e di fase delle sinusoidi che dobbiamo sommare per ottenere la porzione analizzata del segnale originale. 

La STFT ha alcune limitazioni intrinseche. In primo luogo, il risultato che fornisce non è continuo, ma è costituito da ``finestre'' che discretizzano sia l'asse temporale che quello frequenziale; a una maggiore risoluzione temporale corrisponderà necessariamente una peggiore risoluzione frequenziale, e viceversa. Inoltre, l'asse frequenziale è diviso in maniera lineare: ciascuna finestra frequenziale potrebbe avere, per esempio, una larghezza di 100 Hz. Poiché però la nostra percezione dell'altezza non è lineare ma approssimativamente logaritmica rispetto alla frequenza, questo vuol dire che i 100 Hz nella regione grave copriranno uno spazio percettivamente molto più ampio degli stessi 100 Hz nella regione acuta. Di conseguenza, dal punto di vista percettivo la STFT tende a dare risultati decisamente poco accurati rispetto alle regioni spettrali gravi, e inutilmente dettagliati rispetto a regioni spettrali estremamente acute. Esistono trasformate di tipo diverso, chiamate in generale \emph{constant Q transform}, che organizzano lo spazio delle frequenze in maniera più prossima alla nostra percezione. Dal punto di vista matematico si tratta però di meccanismi molto più complessi, che vengono adottati decisamente più di rado rispetto alla STFT. 

Si può dire, in maniera intuitiva anche se formalmente non precisa, che lo spettro frequenziale restituito dalla STFT non è virtualmente mai esatto rispetto al segnale analizzato. Tuttavia, la STFT è perfettamente invertibile: le inesattezze che essa introduce vengono poi ``metabolizzate'' quando viene calcolata la trasformata inversa a breve termine, e contribuiscono alla ricostruzione del segnale originale. Inoltre, nonostante la sua apparente imprecisione, l'analisi del fenomeno sonoro restituita dalla STFT è abbastanza buona e prossima alla percezione da essere praticamente utile in una grande varietà di casi. La grande maggioranza degli analizzatori di spettro che incontriamo nei vari programmi audio che usiamo sono basati sulla STFT, e in generale funzionano piuttosto bene. Se tutto questo sembra complicato non preoccuparti: lo è, e ci torneremo nelle prossime annualità del corso, quando esploreremo come la STFT e la sua inversa possono essere usate per operare trasformazioni interessanti del suono.

La matematica della STFT non è banale, e non la approfondiremo oltre. Ti consiglio però questo sito \url{
https://jackschaedler.github.io/circles-sines-signals/index.html} e il bellissimo libro per bambini superintelligenti ``Who Is Fourier?'' per avere due introduzioni informali ma rigorose sulla questione.

\subsection{Rappresentazioni grafiche e spettrogrammi}

\begin{figure}
    \begin{center}
       \scalebox{0.6} {%% Creator: Matplotlib, PGF backend
%%
%% To include the figure in your LaTeX document, write
%%   \input{<filename>.pgf}
%%
%% Make sure the required packages are loaded in your preamble
%%   \usepackage{pgf}
%%
%% Also ensure that all the required font packages are loaded; for instance,
%% the lmodern package is sometimes necessary when using math font.
%%   \usepackage{lmodern}
%%
%% Figures using additional raster images can only be included by \input if
%% they are in the same directory as the main LaTeX file. For loading figures
%% from other directories you can use the `import` package
%%   \usepackage{import}
%%
%% and then include the figures with
%%   \import{<path to file>}{<filename>.pgf}
%%
%% Matplotlib used the following preamble
%%   
%%   \usepackage{fontspec}
%%   \setmainfont{DejaVuSerif.ttf}[Path=\detokenize{/opt/homebrew/Caskroom/miniconda/base/envs/label-studio/lib/python3.9/site-packages/matplotlib/mpl-data/fonts/ttf/}]
%%   \setsansfont{DejaVuSans.ttf}[Path=\detokenize{/opt/homebrew/Caskroom/miniconda/base/envs/label-studio/lib/python3.9/site-packages/matplotlib/mpl-data/fonts/ttf/}]
%%   \setmonofont{DejaVuSansMono.ttf}[Path=\detokenize{/opt/homebrew/Caskroom/miniconda/base/envs/label-studio/lib/python3.9/site-packages/matplotlib/mpl-data/fonts/ttf/}]
%%   \makeatletter\@ifpackageloaded{underscore}{}{\usepackage[strings]{underscore}}\makeatother
%%
\begingroup%
\makeatletter%
\begin{pgfpicture}%
\pgfpathrectangle{\pgfpointorigin}{\pgfqpoint{7.000000in}{5.000000in}}%
\pgfusepath{use as bounding box, clip}%
\begin{pgfscope}%
\pgfsetbuttcap%
\pgfsetmiterjoin%
\definecolor{currentfill}{rgb}{1.000000,1.000000,1.000000}%
\pgfsetfillcolor{currentfill}%
\pgfsetlinewidth{0.000000pt}%
\definecolor{currentstroke}{rgb}{1.000000,1.000000,1.000000}%
\pgfsetstrokecolor{currentstroke}%
\pgfsetdash{}{0pt}%
\pgfpathmoveto{\pgfqpoint{0.000000in}{0.000000in}}%
\pgfpathlineto{\pgfqpoint{7.000000in}{0.000000in}}%
\pgfpathlineto{\pgfqpoint{7.000000in}{5.000000in}}%
\pgfpathlineto{\pgfqpoint{0.000000in}{5.000000in}}%
\pgfpathlineto{\pgfqpoint{0.000000in}{0.000000in}}%
\pgfpathclose%
\pgfusepath{fill}%
\end{pgfscope}%
\begin{pgfscope}%
\pgfsetbuttcap%
\pgfsetmiterjoin%
\definecolor{currentfill}{rgb}{1.000000,1.000000,1.000000}%
\pgfsetfillcolor{currentfill}%
\pgfsetlinewidth{0.000000pt}%
\definecolor{currentstroke}{rgb}{0.000000,0.000000,0.000000}%
\pgfsetstrokecolor{currentstroke}%
\pgfsetstrokeopacity{0.000000}%
\pgfsetdash{}{0pt}%
\pgfpathmoveto{\pgfqpoint{0.875000in}{0.550000in}}%
\pgfpathlineto{\pgfqpoint{6.300000in}{0.550000in}}%
\pgfpathlineto{\pgfqpoint{6.300000in}{4.400000in}}%
\pgfpathlineto{\pgfqpoint{0.875000in}{4.400000in}}%
\pgfpathlineto{\pgfqpoint{0.875000in}{0.550000in}}%
\pgfpathclose%
\pgfusepath{fill}%
\end{pgfscope}%
\begin{pgfscope}%
\pgfsetbuttcap%
\pgfsetroundjoin%
\definecolor{currentfill}{rgb}{0.000000,0.000000,0.000000}%
\pgfsetfillcolor{currentfill}%
\pgfsetlinewidth{0.803000pt}%
\definecolor{currentstroke}{rgb}{0.000000,0.000000,0.000000}%
\pgfsetstrokecolor{currentstroke}%
\pgfsetdash{}{0pt}%
\pgfsys@defobject{currentmarker}{\pgfqpoint{0.000000in}{-0.048611in}}{\pgfqpoint{0.000000in}{0.000000in}}{%
\pgfpathmoveto{\pgfqpoint{0.000000in}{0.000000in}}%
\pgfpathlineto{\pgfqpoint{0.000000in}{-0.048611in}}%
\pgfusepath{stroke,fill}%
}%
\begin{pgfscope}%
\pgfsys@transformshift{1.121591in}{0.550000in}%
\pgfsys@useobject{currentmarker}{}%
\end{pgfscope}%
\end{pgfscope}%
\begin{pgfscope}%
\definecolor{textcolor}{rgb}{0.000000,0.000000,0.000000}%
\pgfsetstrokecolor{textcolor}%
\pgfsetfillcolor{textcolor}%
\pgftext[x=1.121591in,y=0.452778in,,top]{\color{textcolor}\sffamily\fontsize{10.000000}{12.000000}\selectfont 0}%
\end{pgfscope}%
\begin{pgfscope}%
\pgfsetbuttcap%
\pgfsetroundjoin%
\definecolor{currentfill}{rgb}{0.000000,0.000000,0.000000}%
\pgfsetfillcolor{currentfill}%
\pgfsetlinewidth{0.803000pt}%
\definecolor{currentstroke}{rgb}{0.000000,0.000000,0.000000}%
\pgfsetstrokecolor{currentstroke}%
\pgfsetdash{}{0pt}%
\pgfsys@defobject{currentmarker}{\pgfqpoint{0.000000in}{-0.048611in}}{\pgfqpoint{0.000000in}{0.000000in}}{%
\pgfpathmoveto{\pgfqpoint{0.000000in}{0.000000in}}%
\pgfpathlineto{\pgfqpoint{0.000000in}{-0.048611in}}%
\pgfusepath{stroke,fill}%
}%
\begin{pgfscope}%
\pgfsys@transformshift{2.037958in}{0.550000in}%
\pgfsys@useobject{currentmarker}{}%
\end{pgfscope}%
\end{pgfscope}%
\begin{pgfscope}%
\definecolor{textcolor}{rgb}{0.000000,0.000000,0.000000}%
\pgfsetstrokecolor{textcolor}%
\pgfsetfillcolor{textcolor}%
\pgftext[x=2.037958in,y=0.452778in,,top]{\color{textcolor}\sffamily\fontsize{10.000000}{12.000000}\selectfont 2000}%
\end{pgfscope}%
\begin{pgfscope}%
\pgfsetbuttcap%
\pgfsetroundjoin%
\definecolor{currentfill}{rgb}{0.000000,0.000000,0.000000}%
\pgfsetfillcolor{currentfill}%
\pgfsetlinewidth{0.803000pt}%
\definecolor{currentstroke}{rgb}{0.000000,0.000000,0.000000}%
\pgfsetstrokecolor{currentstroke}%
\pgfsetdash{}{0pt}%
\pgfsys@defobject{currentmarker}{\pgfqpoint{0.000000in}{-0.048611in}}{\pgfqpoint{0.000000in}{0.000000in}}{%
\pgfpathmoveto{\pgfqpoint{0.000000in}{0.000000in}}%
\pgfpathlineto{\pgfqpoint{0.000000in}{-0.048611in}}%
\pgfusepath{stroke,fill}%
}%
\begin{pgfscope}%
\pgfsys@transformshift{2.954325in}{0.550000in}%
\pgfsys@useobject{currentmarker}{}%
\end{pgfscope}%
\end{pgfscope}%
\begin{pgfscope}%
\definecolor{textcolor}{rgb}{0.000000,0.000000,0.000000}%
\pgfsetstrokecolor{textcolor}%
\pgfsetfillcolor{textcolor}%
\pgftext[x=2.954325in,y=0.452778in,,top]{\color{textcolor}\sffamily\fontsize{10.000000}{12.000000}\selectfont 4000}%
\end{pgfscope}%
\begin{pgfscope}%
\pgfsetbuttcap%
\pgfsetroundjoin%
\definecolor{currentfill}{rgb}{0.000000,0.000000,0.000000}%
\pgfsetfillcolor{currentfill}%
\pgfsetlinewidth{0.803000pt}%
\definecolor{currentstroke}{rgb}{0.000000,0.000000,0.000000}%
\pgfsetstrokecolor{currentstroke}%
\pgfsetdash{}{0pt}%
\pgfsys@defobject{currentmarker}{\pgfqpoint{0.000000in}{-0.048611in}}{\pgfqpoint{0.000000in}{0.000000in}}{%
\pgfpathmoveto{\pgfqpoint{0.000000in}{0.000000in}}%
\pgfpathlineto{\pgfqpoint{0.000000in}{-0.048611in}}%
\pgfusepath{stroke,fill}%
}%
\begin{pgfscope}%
\pgfsys@transformshift{3.870693in}{0.550000in}%
\pgfsys@useobject{currentmarker}{}%
\end{pgfscope}%
\end{pgfscope}%
\begin{pgfscope}%
\definecolor{textcolor}{rgb}{0.000000,0.000000,0.000000}%
\pgfsetstrokecolor{textcolor}%
\pgfsetfillcolor{textcolor}%
\pgftext[x=3.870693in,y=0.452778in,,top]{\color{textcolor}\sffamily\fontsize{10.000000}{12.000000}\selectfont 6000}%
\end{pgfscope}%
\begin{pgfscope}%
\pgfsetbuttcap%
\pgfsetroundjoin%
\definecolor{currentfill}{rgb}{0.000000,0.000000,0.000000}%
\pgfsetfillcolor{currentfill}%
\pgfsetlinewidth{0.803000pt}%
\definecolor{currentstroke}{rgb}{0.000000,0.000000,0.000000}%
\pgfsetstrokecolor{currentstroke}%
\pgfsetdash{}{0pt}%
\pgfsys@defobject{currentmarker}{\pgfqpoint{0.000000in}{-0.048611in}}{\pgfqpoint{0.000000in}{0.000000in}}{%
\pgfpathmoveto{\pgfqpoint{0.000000in}{0.000000in}}%
\pgfpathlineto{\pgfqpoint{0.000000in}{-0.048611in}}%
\pgfusepath{stroke,fill}%
}%
\begin{pgfscope}%
\pgfsys@transformshift{4.787060in}{0.550000in}%
\pgfsys@useobject{currentmarker}{}%
\end{pgfscope}%
\end{pgfscope}%
\begin{pgfscope}%
\definecolor{textcolor}{rgb}{0.000000,0.000000,0.000000}%
\pgfsetstrokecolor{textcolor}%
\pgfsetfillcolor{textcolor}%
\pgftext[x=4.787060in,y=0.452778in,,top]{\color{textcolor}\sffamily\fontsize{10.000000}{12.000000}\selectfont 8000}%
\end{pgfscope}%
\begin{pgfscope}%
\pgfsetbuttcap%
\pgfsetroundjoin%
\definecolor{currentfill}{rgb}{0.000000,0.000000,0.000000}%
\pgfsetfillcolor{currentfill}%
\pgfsetlinewidth{0.803000pt}%
\definecolor{currentstroke}{rgb}{0.000000,0.000000,0.000000}%
\pgfsetstrokecolor{currentstroke}%
\pgfsetdash{}{0pt}%
\pgfsys@defobject{currentmarker}{\pgfqpoint{0.000000in}{-0.048611in}}{\pgfqpoint{0.000000in}{0.000000in}}{%
\pgfpathmoveto{\pgfqpoint{0.000000in}{0.000000in}}%
\pgfpathlineto{\pgfqpoint{0.000000in}{-0.048611in}}%
\pgfusepath{stroke,fill}%
}%
\begin{pgfscope}%
\pgfsys@transformshift{5.703427in}{0.550000in}%
\pgfsys@useobject{currentmarker}{}%
\end{pgfscope}%
\end{pgfscope}%
\begin{pgfscope}%
\definecolor{textcolor}{rgb}{0.000000,0.000000,0.000000}%
\pgfsetstrokecolor{textcolor}%
\pgfsetfillcolor{textcolor}%
\pgftext[x=5.703427in,y=0.452778in,,top]{\color{textcolor}\sffamily\fontsize{10.000000}{12.000000}\selectfont 10000}%
\end{pgfscope}%
\begin{pgfscope}%
\definecolor{textcolor}{rgb}{0.000000,0.000000,0.000000}%
\pgfsetstrokecolor{textcolor}%
\pgfsetfillcolor{textcolor}%
\pgftext[x=3.587500in,y=0.262809in,,top]{\color{textcolor}\sffamily\fontsize{10.000000}{12.000000}\selectfont Frequency (Hz)}%
\end{pgfscope}%
\begin{pgfscope}%
\pgfsetbuttcap%
\pgfsetroundjoin%
\definecolor{currentfill}{rgb}{0.000000,0.000000,0.000000}%
\pgfsetfillcolor{currentfill}%
\pgfsetlinewidth{0.803000pt}%
\definecolor{currentstroke}{rgb}{0.000000,0.000000,0.000000}%
\pgfsetstrokecolor{currentstroke}%
\pgfsetdash{}{0pt}%
\pgfsys@defobject{currentmarker}{\pgfqpoint{-0.048611in}{0.000000in}}{\pgfqpoint{-0.000000in}{0.000000in}}{%
\pgfpathmoveto{\pgfqpoint{-0.000000in}{0.000000in}}%
\pgfpathlineto{\pgfqpoint{-0.048611in}{0.000000in}}%
\pgfusepath{stroke,fill}%
}%
\begin{pgfscope}%
\pgfsys@transformshift{0.875000in}{0.724705in}%
\pgfsys@useobject{currentmarker}{}%
\end{pgfscope}%
\end{pgfscope}%
\begin{pgfscope}%
\definecolor{textcolor}{rgb}{0.000000,0.000000,0.000000}%
\pgfsetstrokecolor{textcolor}%
\pgfsetfillcolor{textcolor}%
\pgftext[x=0.689412in, y=0.671944in, left, base]{\color{textcolor}\sffamily\fontsize{10.000000}{12.000000}\selectfont 0}%
\end{pgfscope}%
\begin{pgfscope}%
\pgfsetbuttcap%
\pgfsetroundjoin%
\definecolor{currentfill}{rgb}{0.000000,0.000000,0.000000}%
\pgfsetfillcolor{currentfill}%
\pgfsetlinewidth{0.803000pt}%
\definecolor{currentstroke}{rgb}{0.000000,0.000000,0.000000}%
\pgfsetstrokecolor{currentstroke}%
\pgfsetdash{}{0pt}%
\pgfsys@defobject{currentmarker}{\pgfqpoint{-0.048611in}{0.000000in}}{\pgfqpoint{-0.000000in}{0.000000in}}{%
\pgfpathmoveto{\pgfqpoint{-0.000000in}{0.000000in}}%
\pgfpathlineto{\pgfqpoint{-0.048611in}{0.000000in}}%
\pgfusepath{stroke,fill}%
}%
\begin{pgfscope}%
\pgfsys@transformshift{0.875000in}{1.226162in}%
\pgfsys@useobject{currentmarker}{}%
\end{pgfscope}%
\end{pgfscope}%
\begin{pgfscope}%
\definecolor{textcolor}{rgb}{0.000000,0.000000,0.000000}%
\pgfsetstrokecolor{textcolor}%
\pgfsetfillcolor{textcolor}%
\pgftext[x=0.512682in, y=1.173400in, left, base]{\color{textcolor}\sffamily\fontsize{10.000000}{12.000000}\selectfont 250}%
\end{pgfscope}%
\begin{pgfscope}%
\pgfsetbuttcap%
\pgfsetroundjoin%
\definecolor{currentfill}{rgb}{0.000000,0.000000,0.000000}%
\pgfsetfillcolor{currentfill}%
\pgfsetlinewidth{0.803000pt}%
\definecolor{currentstroke}{rgb}{0.000000,0.000000,0.000000}%
\pgfsetstrokecolor{currentstroke}%
\pgfsetdash{}{0pt}%
\pgfsys@defobject{currentmarker}{\pgfqpoint{-0.048611in}{0.000000in}}{\pgfqpoint{-0.000000in}{0.000000in}}{%
\pgfpathmoveto{\pgfqpoint{-0.000000in}{0.000000in}}%
\pgfpathlineto{\pgfqpoint{-0.048611in}{0.000000in}}%
\pgfusepath{stroke,fill}%
}%
\begin{pgfscope}%
\pgfsys@transformshift{0.875000in}{1.727619in}%
\pgfsys@useobject{currentmarker}{}%
\end{pgfscope}%
\end{pgfscope}%
\begin{pgfscope}%
\definecolor{textcolor}{rgb}{0.000000,0.000000,0.000000}%
\pgfsetstrokecolor{textcolor}%
\pgfsetfillcolor{textcolor}%
\pgftext[x=0.512682in, y=1.674857in, left, base]{\color{textcolor}\sffamily\fontsize{10.000000}{12.000000}\selectfont 500}%
\end{pgfscope}%
\begin{pgfscope}%
\pgfsetbuttcap%
\pgfsetroundjoin%
\definecolor{currentfill}{rgb}{0.000000,0.000000,0.000000}%
\pgfsetfillcolor{currentfill}%
\pgfsetlinewidth{0.803000pt}%
\definecolor{currentstroke}{rgb}{0.000000,0.000000,0.000000}%
\pgfsetstrokecolor{currentstroke}%
\pgfsetdash{}{0pt}%
\pgfsys@defobject{currentmarker}{\pgfqpoint{-0.048611in}{0.000000in}}{\pgfqpoint{-0.000000in}{0.000000in}}{%
\pgfpathmoveto{\pgfqpoint{-0.000000in}{0.000000in}}%
\pgfpathlineto{\pgfqpoint{-0.048611in}{0.000000in}}%
\pgfusepath{stroke,fill}%
}%
\begin{pgfscope}%
\pgfsys@transformshift{0.875000in}{2.229075in}%
\pgfsys@useobject{currentmarker}{}%
\end{pgfscope}%
\end{pgfscope}%
\begin{pgfscope}%
\definecolor{textcolor}{rgb}{0.000000,0.000000,0.000000}%
\pgfsetstrokecolor{textcolor}%
\pgfsetfillcolor{textcolor}%
\pgftext[x=0.512682in, y=2.176314in, left, base]{\color{textcolor}\sffamily\fontsize{10.000000}{12.000000}\selectfont 750}%
\end{pgfscope}%
\begin{pgfscope}%
\pgfsetbuttcap%
\pgfsetroundjoin%
\definecolor{currentfill}{rgb}{0.000000,0.000000,0.000000}%
\pgfsetfillcolor{currentfill}%
\pgfsetlinewidth{0.803000pt}%
\definecolor{currentstroke}{rgb}{0.000000,0.000000,0.000000}%
\pgfsetstrokecolor{currentstroke}%
\pgfsetdash{}{0pt}%
\pgfsys@defobject{currentmarker}{\pgfqpoint{-0.048611in}{0.000000in}}{\pgfqpoint{-0.000000in}{0.000000in}}{%
\pgfpathmoveto{\pgfqpoint{-0.000000in}{0.000000in}}%
\pgfpathlineto{\pgfqpoint{-0.048611in}{0.000000in}}%
\pgfusepath{stroke,fill}%
}%
\begin{pgfscope}%
\pgfsys@transformshift{0.875000in}{2.730532in}%
\pgfsys@useobject{currentmarker}{}%
\end{pgfscope}%
\end{pgfscope}%
\begin{pgfscope}%
\definecolor{textcolor}{rgb}{0.000000,0.000000,0.000000}%
\pgfsetstrokecolor{textcolor}%
\pgfsetfillcolor{textcolor}%
\pgftext[x=0.424316in, y=2.677770in, left, base]{\color{textcolor}\sffamily\fontsize{10.000000}{12.000000}\selectfont 1000}%
\end{pgfscope}%
\begin{pgfscope}%
\pgfsetbuttcap%
\pgfsetroundjoin%
\definecolor{currentfill}{rgb}{0.000000,0.000000,0.000000}%
\pgfsetfillcolor{currentfill}%
\pgfsetlinewidth{0.803000pt}%
\definecolor{currentstroke}{rgb}{0.000000,0.000000,0.000000}%
\pgfsetstrokecolor{currentstroke}%
\pgfsetdash{}{0pt}%
\pgfsys@defobject{currentmarker}{\pgfqpoint{-0.048611in}{0.000000in}}{\pgfqpoint{-0.000000in}{0.000000in}}{%
\pgfpathmoveto{\pgfqpoint{-0.000000in}{0.000000in}}%
\pgfpathlineto{\pgfqpoint{-0.048611in}{0.000000in}}%
\pgfusepath{stroke,fill}%
}%
\begin{pgfscope}%
\pgfsys@transformshift{0.875000in}{3.231988in}%
\pgfsys@useobject{currentmarker}{}%
\end{pgfscope}%
\end{pgfscope}%
\begin{pgfscope}%
\definecolor{textcolor}{rgb}{0.000000,0.000000,0.000000}%
\pgfsetstrokecolor{textcolor}%
\pgfsetfillcolor{textcolor}%
\pgftext[x=0.424316in, y=3.179227in, left, base]{\color{textcolor}\sffamily\fontsize{10.000000}{12.000000}\selectfont 1250}%
\end{pgfscope}%
\begin{pgfscope}%
\pgfsetbuttcap%
\pgfsetroundjoin%
\definecolor{currentfill}{rgb}{0.000000,0.000000,0.000000}%
\pgfsetfillcolor{currentfill}%
\pgfsetlinewidth{0.803000pt}%
\definecolor{currentstroke}{rgb}{0.000000,0.000000,0.000000}%
\pgfsetstrokecolor{currentstroke}%
\pgfsetdash{}{0pt}%
\pgfsys@defobject{currentmarker}{\pgfqpoint{-0.048611in}{0.000000in}}{\pgfqpoint{-0.000000in}{0.000000in}}{%
\pgfpathmoveto{\pgfqpoint{-0.000000in}{0.000000in}}%
\pgfpathlineto{\pgfqpoint{-0.048611in}{0.000000in}}%
\pgfusepath{stroke,fill}%
}%
\begin{pgfscope}%
\pgfsys@transformshift{0.875000in}{3.733445in}%
\pgfsys@useobject{currentmarker}{}%
\end{pgfscope}%
\end{pgfscope}%
\begin{pgfscope}%
\definecolor{textcolor}{rgb}{0.000000,0.000000,0.000000}%
\pgfsetstrokecolor{textcolor}%
\pgfsetfillcolor{textcolor}%
\pgftext[x=0.424316in, y=3.680684in, left, base]{\color{textcolor}\sffamily\fontsize{10.000000}{12.000000}\selectfont 1500}%
\end{pgfscope}%
\begin{pgfscope}%
\pgfsetbuttcap%
\pgfsetroundjoin%
\definecolor{currentfill}{rgb}{0.000000,0.000000,0.000000}%
\pgfsetfillcolor{currentfill}%
\pgfsetlinewidth{0.803000pt}%
\definecolor{currentstroke}{rgb}{0.000000,0.000000,0.000000}%
\pgfsetstrokecolor{currentstroke}%
\pgfsetdash{}{0pt}%
\pgfsys@defobject{currentmarker}{\pgfqpoint{-0.048611in}{0.000000in}}{\pgfqpoint{-0.000000in}{0.000000in}}{%
\pgfpathmoveto{\pgfqpoint{-0.000000in}{0.000000in}}%
\pgfpathlineto{\pgfqpoint{-0.048611in}{0.000000in}}%
\pgfusepath{stroke,fill}%
}%
\begin{pgfscope}%
\pgfsys@transformshift{0.875000in}{4.234902in}%
\pgfsys@useobject{currentmarker}{}%
\end{pgfscope}%
\end{pgfscope}%
\begin{pgfscope}%
\definecolor{textcolor}{rgb}{0.000000,0.000000,0.000000}%
\pgfsetstrokecolor{textcolor}%
\pgfsetfillcolor{textcolor}%
\pgftext[x=0.424316in, y=4.182140in, left, base]{\color{textcolor}\sffamily\fontsize{10.000000}{12.000000}\selectfont 1750}%
\end{pgfscope}%
\begin{pgfscope}%
\pgfpathrectangle{\pgfqpoint{0.875000in}{0.550000in}}{\pgfqpoint{5.425000in}{3.850000in}}%
\pgfusepath{clip}%
\pgfsetrectcap%
\pgfsetroundjoin%
\pgfsetlinewidth{1.505625pt}%
\definecolor{currentstroke}{rgb}{0.121569,0.466667,0.705882}%
\pgfsetstrokecolor{currentstroke}%
\pgfsetdash{}{0pt}%
\pgfpathmoveto{\pgfqpoint{1.121591in}{0.729078in}}%
\pgfpathlineto{\pgfqpoint{1.146262in}{0.730318in}}%
\pgfpathlineto{\pgfqpoint{1.175868in}{0.734158in}}%
\pgfpathlineto{\pgfqpoint{1.203007in}{0.739781in}}%
\pgfpathlineto{\pgfqpoint{1.222744in}{0.746061in}}%
\pgfpathlineto{\pgfqpoint{1.237547in}{0.753098in}}%
\pgfpathlineto{\pgfqpoint{1.247415in}{0.759772in}}%
\pgfpathlineto{\pgfqpoint{1.257284in}{0.769197in}}%
\pgfpathlineto{\pgfqpoint{1.264685in}{0.779301in}}%
\pgfpathlineto{\pgfqpoint{1.272087in}{0.793974in}}%
\pgfpathlineto{\pgfqpoint{1.277021in}{0.808096in}}%
\pgfpathlineto{\pgfqpoint{1.281955in}{0.828311in}}%
\pgfpathlineto{\pgfqpoint{1.286889in}{0.859630in}}%
\pgfpathlineto{\pgfqpoint{1.291824in}{0.914624in}}%
\pgfpathlineto{\pgfqpoint{1.294291in}{0.961318in}}%
\pgfpathlineto{\pgfqpoint{1.296758in}{1.036307in}}%
\pgfpathlineto{\pgfqpoint{1.299225in}{1.176442in}}%
\pgfpathlineto{\pgfqpoint{1.301692in}{1.531618in}}%
\pgfpathlineto{\pgfqpoint{1.304159in}{4.225000in}}%
\pgfpathlineto{\pgfqpoint{1.306627in}{2.269210in}}%
\pgfpathlineto{\pgfqpoint{1.309094in}{1.365546in}}%
\pgfpathlineto{\pgfqpoint{1.311561in}{1.132277in}}%
\pgfpathlineto{\pgfqpoint{1.316495in}{0.963880in}}%
\pgfpathlineto{\pgfqpoint{1.321429in}{0.896189in}}%
\pgfpathlineto{\pgfqpoint{1.326364in}{0.859732in}}%
\pgfpathlineto{\pgfqpoint{1.331298in}{0.836999in}}%
\pgfpathlineto{\pgfqpoint{1.338699in}{0.815502in}}%
\pgfpathlineto{\pgfqpoint{1.346101in}{0.801908in}}%
\pgfpathlineto{\pgfqpoint{1.353502in}{0.792639in}}%
\pgfpathlineto{\pgfqpoint{1.363371in}{0.784223in}}%
\pgfpathlineto{\pgfqpoint{1.373239in}{0.778601in}}%
\pgfpathlineto{\pgfqpoint{1.385575in}{0.774094in}}%
\pgfpathlineto{\pgfqpoint{1.400378in}{0.771281in}}%
\pgfpathlineto{\pgfqpoint{1.415181in}{0.770830in}}%
\pgfpathlineto{\pgfqpoint{1.427517in}{0.772437in}}%
\pgfpathlineto{\pgfqpoint{1.437385in}{0.775515in}}%
\pgfpathlineto{\pgfqpoint{1.447254in}{0.781151in}}%
\pgfpathlineto{\pgfqpoint{1.454655in}{0.788271in}}%
\pgfpathlineto{\pgfqpoint{1.459589in}{0.795414in}}%
\pgfpathlineto{\pgfqpoint{1.464524in}{0.805840in}}%
\pgfpathlineto{\pgfqpoint{1.469458in}{0.822110in}}%
\pgfpathlineto{\pgfqpoint{1.474392in}{0.850442in}}%
\pgfpathlineto{\pgfqpoint{1.476859in}{0.874028in}}%
\pgfpathlineto{\pgfqpoint{1.479327in}{0.910852in}}%
\pgfpathlineto{\pgfqpoint{1.481794in}{0.976135in}}%
\pgfpathlineto{\pgfqpoint{1.484261in}{1.122972in}}%
\pgfpathlineto{\pgfqpoint{1.489195in}{2.282672in}}%
\pgfpathlineto{\pgfqpoint{1.491662in}{1.153772in}}%
\pgfpathlineto{\pgfqpoint{1.494129in}{0.968862in}}%
\pgfpathlineto{\pgfqpoint{1.496597in}{0.893114in}}%
\pgfpathlineto{\pgfqpoint{1.501531in}{0.826167in}}%
\pgfpathlineto{\pgfqpoint{1.506465in}{0.795695in}}%
\pgfpathlineto{\pgfqpoint{1.511399in}{0.778392in}}%
\pgfpathlineto{\pgfqpoint{1.516334in}{0.767317in}}%
\pgfpathlineto{\pgfqpoint{1.523735in}{0.756699in}}%
\pgfpathlineto{\pgfqpoint{1.531137in}{0.749961in}}%
\pgfpathlineto{\pgfqpoint{1.541005in}{0.744187in}}%
\pgfpathlineto{\pgfqpoint{1.553341in}{0.739783in}}%
\pgfpathlineto{\pgfqpoint{1.570611in}{0.736366in}}%
\pgfpathlineto{\pgfqpoint{1.592815in}{0.734441in}}%
\pgfpathlineto{\pgfqpoint{1.617487in}{0.734391in}}%
\pgfpathlineto{\pgfqpoint{1.634757in}{0.736086in}}%
\pgfpathlineto{\pgfqpoint{1.644625in}{0.738599in}}%
\pgfpathlineto{\pgfqpoint{1.652027in}{0.742455in}}%
\pgfpathlineto{\pgfqpoint{1.656961in}{0.747377in}}%
\pgfpathlineto{\pgfqpoint{1.661895in}{0.757531in}}%
\pgfpathlineto{\pgfqpoint{1.664362in}{0.767937in}}%
\pgfpathlineto{\pgfqpoint{1.666829in}{0.789327in}}%
\pgfpathlineto{\pgfqpoint{1.669297in}{0.857843in}}%
\pgfpathlineto{\pgfqpoint{1.671764in}{2.074559in}}%
\pgfpathlineto{\pgfqpoint{1.674231in}{0.832712in}}%
\pgfpathlineto{\pgfqpoint{1.676698in}{0.780206in}}%
\pgfpathlineto{\pgfqpoint{1.679165in}{0.761755in}}%
\pgfpathlineto{\pgfqpoint{1.684099in}{0.746703in}}%
\pgfpathlineto{\pgfqpoint{1.689034in}{0.740230in}}%
\pgfpathlineto{\pgfqpoint{1.696435in}{0.735437in}}%
\pgfpathlineto{\pgfqpoint{1.706304in}{0.732347in}}%
\pgfpathlineto{\pgfqpoint{1.723574in}{0.729953in}}%
\pgfpathlineto{\pgfqpoint{1.753179in}{0.728595in}}%
\pgfpathlineto{\pgfqpoint{1.785252in}{0.729137in}}%
\pgfpathlineto{\pgfqpoint{1.804989in}{0.731263in}}%
\pgfpathlineto{\pgfqpoint{1.817325in}{0.734235in}}%
\pgfpathlineto{\pgfqpoint{1.827194in}{0.738792in}}%
\pgfpathlineto{\pgfqpoint{1.834595in}{0.745294in}}%
\pgfpathlineto{\pgfqpoint{1.839529in}{0.753209in}}%
\pgfpathlineto{\pgfqpoint{1.844464in}{0.768737in}}%
\pgfpathlineto{\pgfqpoint{1.846931in}{0.783792in}}%
\pgfpathlineto{\pgfqpoint{1.849398in}{0.812483in}}%
\pgfpathlineto{\pgfqpoint{1.851865in}{0.888466in}}%
\pgfpathlineto{\pgfqpoint{1.854332in}{1.670341in}}%
\pgfpathlineto{\pgfqpoint{1.856799in}{0.990421in}}%
\pgfpathlineto{\pgfqpoint{1.859267in}{0.844251in}}%
\pgfpathlineto{\pgfqpoint{1.861734in}{0.803101in}}%
\pgfpathlineto{\pgfqpoint{1.866668in}{0.772466in}}%
\pgfpathlineto{\pgfqpoint{1.871602in}{0.759952in}}%
\pgfpathlineto{\pgfqpoint{1.876537in}{0.753185in}}%
\pgfpathlineto{\pgfqpoint{1.883938in}{0.747414in}}%
\pgfpathlineto{\pgfqpoint{1.893807in}{0.743246in}}%
\pgfpathlineto{\pgfqpoint{1.908609in}{0.740085in}}%
\pgfpathlineto{\pgfqpoint{1.930814in}{0.738172in}}%
\pgfpathlineto{\pgfqpoint{1.957952in}{0.738141in}}%
\pgfpathlineto{\pgfqpoint{1.980157in}{0.740029in}}%
\pgfpathlineto{\pgfqpoint{1.994959in}{0.743160in}}%
\pgfpathlineto{\pgfqpoint{2.004828in}{0.747153in}}%
\pgfpathlineto{\pgfqpoint{2.012229in}{0.752360in}}%
\pgfpathlineto{\pgfqpoint{2.017164in}{0.757990in}}%
\pgfpathlineto{\pgfqpoint{2.022098in}{0.767197in}}%
\pgfpathlineto{\pgfqpoint{2.027032in}{0.784726in}}%
\pgfpathlineto{\pgfqpoint{2.029499in}{0.801044in}}%
\pgfpathlineto{\pgfqpoint{2.031967in}{0.830393in}}%
\pgfpathlineto{\pgfqpoint{2.034434in}{0.898501in}}%
\pgfpathlineto{\pgfqpoint{2.036901in}{1.229290in}}%
\pgfpathlineto{\pgfqpoint{2.039368in}{1.265927in}}%
\pgfpathlineto{\pgfqpoint{2.041835in}{0.899300in}}%
\pgfpathlineto{\pgfqpoint{2.044302in}{0.828327in}}%
\pgfpathlineto{\pgfqpoint{2.049237in}{0.781543in}}%
\pgfpathlineto{\pgfqpoint{2.054171in}{0.763747in}}%
\pgfpathlineto{\pgfqpoint{2.059105in}{0.754430in}}%
\pgfpathlineto{\pgfqpoint{2.066507in}{0.746660in}}%
\pgfpathlineto{\pgfqpoint{2.073908in}{0.742222in}}%
\pgfpathlineto{\pgfqpoint{2.086244in}{0.738050in}}%
\pgfpathlineto{\pgfqpoint{2.103514in}{0.735161in}}%
\pgfpathlineto{\pgfqpoint{2.128185in}{0.733491in}}%
\pgfpathlineto{\pgfqpoint{2.160258in}{0.733546in}}%
\pgfpathlineto{\pgfqpoint{2.182462in}{0.735496in}}%
\pgfpathlineto{\pgfqpoint{2.194798in}{0.738533in}}%
\pgfpathlineto{\pgfqpoint{2.202199in}{0.742426in}}%
\pgfpathlineto{\pgfqpoint{2.207134in}{0.747356in}}%
\pgfpathlineto{\pgfqpoint{2.209601in}{0.751392in}}%
\pgfpathlineto{\pgfqpoint{2.212068in}{0.757569in}}%
\pgfpathlineto{\pgfqpoint{2.214535in}{0.768163in}}%
\pgfpathlineto{\pgfqpoint{2.217002in}{0.790442in}}%
\pgfpathlineto{\pgfqpoint{2.219469in}{0.866991in}}%
\pgfpathlineto{\pgfqpoint{2.221937in}{1.372101in}}%
\pgfpathlineto{\pgfqpoint{2.224404in}{0.819876in}}%
\pgfpathlineto{\pgfqpoint{2.226871in}{0.775060in}}%
\pgfpathlineto{\pgfqpoint{2.229338in}{0.758493in}}%
\pgfpathlineto{\pgfqpoint{2.234272in}{0.744626in}}%
\pgfpathlineto{\pgfqpoint{2.239207in}{0.738551in}}%
\pgfpathlineto{\pgfqpoint{2.246608in}{0.734000in}}%
\pgfpathlineto{\pgfqpoint{2.256477in}{0.731044in}}%
\pgfpathlineto{\pgfqpoint{2.273747in}{0.728772in}}%
\pgfpathlineto{\pgfqpoint{2.305819in}{0.727462in}}%
\pgfpathlineto{\pgfqpoint{2.352695in}{0.727790in}}%
\pgfpathlineto{\pgfqpoint{2.374899in}{0.729799in}}%
\pgfpathlineto{\pgfqpoint{2.384768in}{0.732525in}}%
\pgfpathlineto{\pgfqpoint{2.389702in}{0.735336in}}%
\pgfpathlineto{\pgfqpoint{2.394637in}{0.740976in}}%
\pgfpathlineto{\pgfqpoint{2.397104in}{0.746551in}}%
\pgfpathlineto{\pgfqpoint{2.399571in}{0.757413in}}%
\pgfpathlineto{\pgfqpoint{2.402038in}{0.787637in}}%
\pgfpathlineto{\pgfqpoint{2.404505in}{1.294614in}}%
\pgfpathlineto{\pgfqpoint{2.406972in}{0.809359in}}%
\pgfpathlineto{\pgfqpoint{2.409439in}{0.765028in}}%
\pgfpathlineto{\pgfqpoint{2.411907in}{0.751587in}}%
\pgfpathlineto{\pgfqpoint{2.416841in}{0.741304in}}%
\pgfpathlineto{\pgfqpoint{2.421775in}{0.737052in}}%
\pgfpathlineto{\pgfqpoint{2.429177in}{0.733963in}}%
\pgfpathlineto{\pgfqpoint{2.441512in}{0.731689in}}%
\pgfpathlineto{\pgfqpoint{2.463717in}{0.730240in}}%
\pgfpathlineto{\pgfqpoint{2.503191in}{0.730155in}}%
\pgfpathlineto{\pgfqpoint{2.535264in}{0.732123in}}%
\pgfpathlineto{\pgfqpoint{2.552534in}{0.735332in}}%
\pgfpathlineto{\pgfqpoint{2.562402in}{0.739354in}}%
\pgfpathlineto{\pgfqpoint{2.569804in}{0.745371in}}%
\pgfpathlineto{\pgfqpoint{2.574738in}{0.753175in}}%
\pgfpathlineto{\pgfqpoint{2.577205in}{0.759756in}}%
\pgfpathlineto{\pgfqpoint{2.579672in}{0.770204in}}%
\pgfpathlineto{\pgfqpoint{2.582139in}{0.789290in}}%
\pgfpathlineto{\pgfqpoint{2.584607in}{0.835128in}}%
\pgfpathlineto{\pgfqpoint{2.587074in}{1.093984in}}%
\pgfpathlineto{\pgfqpoint{2.589541in}{1.006356in}}%
\pgfpathlineto{\pgfqpoint{2.592008in}{0.827697in}}%
\pgfpathlineto{\pgfqpoint{2.594475in}{0.788157in}}%
\pgfpathlineto{\pgfqpoint{2.599409in}{0.761064in}}%
\pgfpathlineto{\pgfqpoint{2.604344in}{0.750529in}}%
\pgfpathlineto{\pgfqpoint{2.609278in}{0.744958in}}%
\pgfpathlineto{\pgfqpoint{2.616679in}{0.740278in}}%
\pgfpathlineto{\pgfqpoint{2.626548in}{0.736937in}}%
\pgfpathlineto{\pgfqpoint{2.641351in}{0.734405in}}%
\pgfpathlineto{\pgfqpoint{2.666022in}{0.732681in}}%
\pgfpathlineto{\pgfqpoint{2.700562in}{0.732480in}}%
\pgfpathlineto{\pgfqpoint{2.727701in}{0.734170in}}%
\pgfpathlineto{\pgfqpoint{2.742504in}{0.737011in}}%
\pgfpathlineto{\pgfqpoint{2.749905in}{0.740043in}}%
\pgfpathlineto{\pgfqpoint{2.754839in}{0.743626in}}%
\pgfpathlineto{\pgfqpoint{2.759774in}{0.750302in}}%
\pgfpathlineto{\pgfqpoint{2.762241in}{0.756350in}}%
\pgfpathlineto{\pgfqpoint{2.764708in}{0.766848in}}%
\pgfpathlineto{\pgfqpoint{2.767175in}{0.789487in}}%
\pgfpathlineto{\pgfqpoint{2.769642in}{0.873845in}}%
\pgfpathlineto{\pgfqpoint{2.772109in}{1.127385in}}%
\pgfpathlineto{\pgfqpoint{2.774576in}{0.807024in}}%
\pgfpathlineto{\pgfqpoint{2.777044in}{0.769532in}}%
\pgfpathlineto{\pgfqpoint{2.779511in}{0.755023in}}%
\pgfpathlineto{\pgfqpoint{2.784445in}{0.742573in}}%
\pgfpathlineto{\pgfqpoint{2.789379in}{0.737011in}}%
\pgfpathlineto{\pgfqpoint{2.796781in}{0.732775in}}%
\pgfpathlineto{\pgfqpoint{2.806649in}{0.729957in}}%
\pgfpathlineto{\pgfqpoint{2.823919in}{0.727690in}}%
\pgfpathlineto{\pgfqpoint{2.855992in}{0.726217in}}%
\pgfpathlineto{\pgfqpoint{2.915204in}{0.726001in}}%
\pgfpathlineto{\pgfqpoint{2.939875in}{0.727345in}}%
\pgfpathlineto{\pgfqpoint{2.947276in}{0.729832in}}%
\pgfpathlineto{\pgfqpoint{2.949744in}{0.732349in}}%
\pgfpathlineto{\pgfqpoint{2.952211in}{0.739779in}}%
\pgfpathlineto{\pgfqpoint{2.954678in}{1.134326in}}%
\pgfpathlineto{\pgfqpoint{2.957145in}{0.741336in}}%
\pgfpathlineto{\pgfqpoint{2.959612in}{0.732990in}}%
\pgfpathlineto{\pgfqpoint{2.964546in}{0.728983in}}%
\pgfpathlineto{\pgfqpoint{2.971948in}{0.727346in}}%
\pgfpathlineto{\pgfqpoint{2.991685in}{0.726318in}}%
\pgfpathlineto{\pgfqpoint{3.048429in}{0.726387in}}%
\pgfpathlineto{\pgfqpoint{3.087904in}{0.728189in}}%
\pgfpathlineto{\pgfqpoint{3.105174in}{0.730716in}}%
\pgfpathlineto{\pgfqpoint{3.115042in}{0.733980in}}%
\pgfpathlineto{\pgfqpoint{3.119976in}{0.736984in}}%
\pgfpathlineto{\pgfqpoint{3.124911in}{0.742271in}}%
\pgfpathlineto{\pgfqpoint{3.127378in}{0.746757in}}%
\pgfpathlineto{\pgfqpoint{3.129845in}{0.753940in}}%
\pgfpathlineto{\pgfqpoint{3.132312in}{0.767277in}}%
\pgfpathlineto{\pgfqpoint{3.134779in}{0.800518in}}%
\pgfpathlineto{\pgfqpoint{3.137246in}{1.028058in}}%
\pgfpathlineto{\pgfqpoint{3.142181in}{0.789984in}}%
\pgfpathlineto{\pgfqpoint{3.144648in}{0.766264in}}%
\pgfpathlineto{\pgfqpoint{3.149582in}{0.749413in}}%
\pgfpathlineto{\pgfqpoint{3.154516in}{0.742717in}}%
\pgfpathlineto{\pgfqpoint{3.161918in}{0.737915in}}%
\pgfpathlineto{\pgfqpoint{3.171786in}{0.734856in}}%
\pgfpathlineto{\pgfqpoint{3.189056in}{0.732488in}}%
\pgfpathlineto{\pgfqpoint{3.218662in}{0.731165in}}%
\pgfpathlineto{\pgfqpoint{3.258136in}{0.731567in}}%
\pgfpathlineto{\pgfqpoint{3.282808in}{0.733618in}}%
\pgfpathlineto{\pgfqpoint{3.295144in}{0.736397in}}%
\pgfpathlineto{\pgfqpoint{3.302545in}{0.739914in}}%
\pgfpathlineto{\pgfqpoint{3.307479in}{0.744386in}}%
\pgfpathlineto{\pgfqpoint{3.309946in}{0.748084in}}%
\pgfpathlineto{\pgfqpoint{3.312414in}{0.753824in}}%
\pgfpathlineto{\pgfqpoint{3.314881in}{0.763922in}}%
\pgfpathlineto{\pgfqpoint{3.317348in}{0.786316in}}%
\pgfpathlineto{\pgfqpoint{3.322282in}{0.998555in}}%
\pgfpathlineto{\pgfqpoint{3.324749in}{0.794632in}}%
\pgfpathlineto{\pgfqpoint{3.327216in}{0.764003in}}%
\pgfpathlineto{\pgfqpoint{3.329684in}{0.751653in}}%
\pgfpathlineto{\pgfqpoint{3.334618in}{0.740817in}}%
\pgfpathlineto{\pgfqpoint{3.339552in}{0.735897in}}%
\pgfpathlineto{\pgfqpoint{3.346954in}{0.732110in}}%
\pgfpathlineto{\pgfqpoint{3.359289in}{0.729137in}}%
\pgfpathlineto{\pgfqpoint{3.379026in}{0.727112in}}%
\pgfpathlineto{\pgfqpoint{3.418501in}{0.725751in}}%
\pgfpathlineto{\pgfqpoint{3.475245in}{0.725978in}}%
\pgfpathlineto{\pgfqpoint{3.492515in}{0.727768in}}%
\pgfpathlineto{\pgfqpoint{3.497449in}{0.729906in}}%
\pgfpathlineto{\pgfqpoint{3.499916in}{0.732629in}}%
\pgfpathlineto{\pgfqpoint{3.502384in}{0.741098in}}%
\pgfpathlineto{\pgfqpoint{3.504851in}{1.039544in}}%
\pgfpathlineto{\pgfqpoint{3.507318in}{0.739798in}}%
\pgfpathlineto{\pgfqpoint{3.509785in}{0.732506in}}%
\pgfpathlineto{\pgfqpoint{3.514719in}{0.728728in}}%
\pgfpathlineto{\pgfqpoint{3.524588in}{0.726801in}}%
\pgfpathlineto{\pgfqpoint{3.549259in}{0.725675in}}%
\pgfpathlineto{\pgfqpoint{3.615872in}{0.725227in}}%
\pgfpathlineto{\pgfqpoint{3.650412in}{0.727186in}}%
\pgfpathlineto{\pgfqpoint{3.665215in}{0.729972in}}%
\pgfpathlineto{\pgfqpoint{3.672616in}{0.733407in}}%
\pgfpathlineto{\pgfqpoint{3.677551in}{0.738411in}}%
\pgfpathlineto{\pgfqpoint{3.680018in}{0.743230in}}%
\pgfpathlineto{\pgfqpoint{3.682485in}{0.752334in}}%
\pgfpathlineto{\pgfqpoint{3.684952in}{0.775978in}}%
\pgfpathlineto{\pgfqpoint{3.687419in}{0.985913in}}%
\pgfpathlineto{\pgfqpoint{3.689886in}{0.816402in}}%
\pgfpathlineto{\pgfqpoint{3.692354in}{0.765086in}}%
\pgfpathlineto{\pgfqpoint{3.694821in}{0.751154in}}%
\pgfpathlineto{\pgfqpoint{3.699755in}{0.740923in}}%
\pgfpathlineto{\pgfqpoint{3.704689in}{0.736777in}}%
\pgfpathlineto{\pgfqpoint{3.712091in}{0.733776in}}%
\pgfpathlineto{\pgfqpoint{3.724426in}{0.731546in}}%
\pgfpathlineto{\pgfqpoint{3.749098in}{0.729969in}}%
\pgfpathlineto{\pgfqpoint{3.793506in}{0.729736in}}%
\pgfpathlineto{\pgfqpoint{3.828046in}{0.731429in}}%
\pgfpathlineto{\pgfqpoint{3.842849in}{0.733804in}}%
\pgfpathlineto{\pgfqpoint{3.852718in}{0.737630in}}%
\pgfpathlineto{\pgfqpoint{3.857652in}{0.741664in}}%
\pgfpathlineto{\pgfqpoint{3.860119in}{0.745025in}}%
\pgfpathlineto{\pgfqpoint{3.862586in}{0.750288in}}%
\pgfpathlineto{\pgfqpoint{3.865054in}{0.759686in}}%
\pgfpathlineto{\pgfqpoint{3.867521in}{0.781180in}}%
\pgfpathlineto{\pgfqpoint{3.869988in}{0.879821in}}%
\pgfpathlineto{\pgfqpoint{3.872455in}{0.917217in}}%
\pgfpathlineto{\pgfqpoint{3.874922in}{0.782790in}}%
\pgfpathlineto{\pgfqpoint{3.877389in}{0.758470in}}%
\pgfpathlineto{\pgfqpoint{3.882324in}{0.742711in}}%
\pgfpathlineto{\pgfqpoint{3.887258in}{0.736766in}}%
\pgfpathlineto{\pgfqpoint{3.894659in}{0.732610in}}%
\pgfpathlineto{\pgfqpoint{3.904528in}{0.730006in}}%
\pgfpathlineto{\pgfqpoint{3.921798in}{0.728004in}}%
\pgfpathlineto{\pgfqpoint{3.956338in}{0.726724in}}%
\pgfpathlineto{\pgfqpoint{4.010615in}{0.727011in}}%
\pgfpathlineto{\pgfqpoint{4.032819in}{0.728782in}}%
\pgfpathlineto{\pgfqpoint{4.040221in}{0.730714in}}%
\pgfpathlineto{\pgfqpoint{4.045155in}{0.733720in}}%
\pgfpathlineto{\pgfqpoint{4.047622in}{0.736817in}}%
\pgfpathlineto{\pgfqpoint{4.050089in}{0.743258in}}%
\pgfpathlineto{\pgfqpoint{4.052556in}{0.764691in}}%
\pgfpathlineto{\pgfqpoint{4.055024in}{0.973406in}}%
\pgfpathlineto{\pgfqpoint{4.057491in}{0.754906in}}%
\pgfpathlineto{\pgfqpoint{4.059958in}{0.740785in}}%
\pgfpathlineto{\pgfqpoint{4.064892in}{0.733037in}}%
\pgfpathlineto{\pgfqpoint{4.069826in}{0.730354in}}%
\pgfpathlineto{\pgfqpoint{4.079695in}{0.728189in}}%
\pgfpathlineto{\pgfqpoint{4.099432in}{0.726762in}}%
\pgfpathlineto{\pgfqpoint{4.156176in}{0.725728in}}%
\pgfpathlineto{\pgfqpoint{4.205519in}{0.726331in}}%
\pgfpathlineto{\pgfqpoint{4.220322in}{0.728589in}}%
\pgfpathlineto{\pgfqpoint{4.225256in}{0.730611in}}%
\pgfpathlineto{\pgfqpoint{4.230191in}{0.735248in}}%
\pgfpathlineto{\pgfqpoint{4.232658in}{0.740805in}}%
\pgfpathlineto{\pgfqpoint{4.235125in}{0.755931in}}%
\pgfpathlineto{\pgfqpoint{4.237592in}{0.954019in}}%
\pgfpathlineto{\pgfqpoint{4.240059in}{0.772109in}}%
\pgfpathlineto{\pgfqpoint{4.242526in}{0.747157in}}%
\pgfpathlineto{\pgfqpoint{4.244994in}{0.739822in}}%
\pgfpathlineto{\pgfqpoint{4.249928in}{0.734274in}}%
\pgfpathlineto{\pgfqpoint{4.257329in}{0.731293in}}%
\pgfpathlineto{\pgfqpoint{4.269665in}{0.729473in}}%
\pgfpathlineto{\pgfqpoint{4.296804in}{0.728315in}}%
\pgfpathlineto{\pgfqpoint{4.351081in}{0.728500in}}%
\pgfpathlineto{\pgfqpoint{4.383154in}{0.730310in}}%
\pgfpathlineto{\pgfqpoint{4.397956in}{0.733131in}}%
\pgfpathlineto{\pgfqpoint{4.405358in}{0.736601in}}%
\pgfpathlineto{\pgfqpoint{4.410292in}{0.741555in}}%
\pgfpathlineto{\pgfqpoint{4.412759in}{0.746194in}}%
\pgfpathlineto{\pgfqpoint{4.415226in}{0.754610in}}%
\pgfpathlineto{\pgfqpoint{4.417694in}{0.774514in}}%
\pgfpathlineto{\pgfqpoint{4.420161in}{0.879329in}}%
\pgfpathlineto{\pgfqpoint{4.422628in}{0.860429in}}%
\pgfpathlineto{\pgfqpoint{4.425095in}{0.771421in}}%
\pgfpathlineto{\pgfqpoint{4.427562in}{0.752778in}}%
\pgfpathlineto{\pgfqpoint{4.432496in}{0.740214in}}%
\pgfpathlineto{\pgfqpoint{4.437431in}{0.735377in}}%
\pgfpathlineto{\pgfqpoint{4.444832in}{0.731969in}}%
\pgfpathlineto{\pgfqpoint{4.457168in}{0.729492in}}%
\pgfpathlineto{\pgfqpoint{4.479372in}{0.727843in}}%
\pgfpathlineto{\pgfqpoint{4.523781in}{0.727153in}}%
\pgfpathlineto{\pgfqpoint{4.568189in}{0.728308in}}%
\pgfpathlineto{\pgfqpoint{4.582992in}{0.730252in}}%
\pgfpathlineto{\pgfqpoint{4.590393in}{0.732849in}}%
\pgfpathlineto{\pgfqpoint{4.595328in}{0.736935in}}%
\pgfpathlineto{\pgfqpoint{4.597795in}{0.741213in}}%
\pgfpathlineto{\pgfqpoint{4.600262in}{0.750339in}}%
\pgfpathlineto{\pgfqpoint{4.602729in}{0.783125in}}%
\pgfpathlineto{\pgfqpoint{4.605196in}{0.922860in}}%
\pgfpathlineto{\pgfqpoint{4.607663in}{0.761171in}}%
\pgfpathlineto{\pgfqpoint{4.610131in}{0.744724in}}%
\pgfpathlineto{\pgfqpoint{4.615065in}{0.735206in}}%
\pgfpathlineto{\pgfqpoint{4.619999in}{0.731826in}}%
\pgfpathlineto{\pgfqpoint{4.629868in}{0.729075in}}%
\pgfpathlineto{\pgfqpoint{4.647138in}{0.727404in}}%
\pgfpathlineto{\pgfqpoint{4.689079in}{0.726303in}}%
\pgfpathlineto{\pgfqpoint{4.770495in}{0.726172in}}%
\pgfpathlineto{\pgfqpoint{4.777896in}{0.727556in}}%
\pgfpathlineto{\pgfqpoint{4.782831in}{0.731141in}}%
\pgfpathlineto{\pgfqpoint{4.785298in}{0.738103in}}%
\pgfpathlineto{\pgfqpoint{4.787765in}{0.926897in}}%
\pgfpathlineto{\pgfqpoint{4.790232in}{0.742935in}}%
\pgfpathlineto{\pgfqpoint{4.792699in}{0.734131in}}%
\pgfpathlineto{\pgfqpoint{4.797633in}{0.730027in}}%
\pgfpathlineto{\pgfqpoint{4.805035in}{0.728360in}}%
\pgfpathlineto{\pgfqpoint{4.824772in}{0.727279in}}%
\pgfpathlineto{\pgfqpoint{4.883983in}{0.727165in}}%
\pgfpathlineto{\pgfqpoint{4.930859in}{0.728722in}}%
\pgfpathlineto{\pgfqpoint{4.948129in}{0.731229in}}%
\pgfpathlineto{\pgfqpoint{4.955531in}{0.734075in}}%
\pgfpathlineto{\pgfqpoint{4.960465in}{0.738181in}}%
\pgfpathlineto{\pgfqpoint{4.962932in}{0.742074in}}%
\pgfpathlineto{\pgfqpoint{4.965399in}{0.749253in}}%
\pgfpathlineto{\pgfqpoint{4.967866in}{0.766861in}}%
\pgfpathlineto{\pgfqpoint{4.970333in}{0.877018in}}%
\pgfpathlineto{\pgfqpoint{4.975268in}{0.760387in}}%
\pgfpathlineto{\pgfqpoint{4.977735in}{0.746739in}}%
\pgfpathlineto{\pgfqpoint{4.982669in}{0.737200in}}%
\pgfpathlineto{\pgfqpoint{4.987603in}{0.733453in}}%
\pgfpathlineto{\pgfqpoint{4.995005in}{0.730789in}}%
\pgfpathlineto{\pgfqpoint{5.009808in}{0.728613in}}%
\pgfpathlineto{\pgfqpoint{5.039413in}{0.727302in}}%
\pgfpathlineto{\pgfqpoint{5.096158in}{0.727353in}}%
\pgfpathlineto{\pgfqpoint{5.125763in}{0.729168in}}%
\pgfpathlineto{\pgfqpoint{5.135632in}{0.731274in}}%
\pgfpathlineto{\pgfqpoint{5.140566in}{0.733497in}}%
\pgfpathlineto{\pgfqpoint{5.145501in}{0.738189in}}%
\pgfpathlineto{\pgfqpoint{5.147968in}{0.743187in}}%
\pgfpathlineto{\pgfqpoint{5.150435in}{0.754148in}}%
\pgfpathlineto{\pgfqpoint{5.152902in}{0.797196in}}%
\pgfpathlineto{\pgfqpoint{5.155369in}{0.882481in}}%
\pgfpathlineto{\pgfqpoint{5.157836in}{0.762621in}}%
\pgfpathlineto{\pgfqpoint{5.160303in}{0.746325in}}%
\pgfpathlineto{\pgfqpoint{5.165238in}{0.736424in}}%
\pgfpathlineto{\pgfqpoint{5.170172in}{0.732823in}}%
\pgfpathlineto{\pgfqpoint{5.177573in}{0.730356in}}%
\pgfpathlineto{\pgfqpoint{5.192376in}{0.728391in}}%
\pgfpathlineto{\pgfqpoint{5.224449in}{0.727132in}}%
\pgfpathlineto{\pgfqpoint{5.308332in}{0.726708in}}%
\pgfpathlineto{\pgfqpoint{5.333003in}{0.727500in}}%
\pgfpathlineto{\pgfqpoint{5.335471in}{0.728492in}}%
\pgfpathlineto{\pgfqpoint{5.337938in}{0.901635in}}%
\pgfpathlineto{\pgfqpoint{5.340405in}{0.725678in}}%
\pgfpathlineto{\pgfqpoint{5.377412in}{0.726567in}}%
\pgfpathlineto{\pgfqpoint{5.478565in}{0.728365in}}%
\pgfpathlineto{\pgfqpoint{5.498302in}{0.730523in}}%
\pgfpathlineto{\pgfqpoint{5.505703in}{0.732742in}}%
\pgfpathlineto{\pgfqpoint{5.510638in}{0.735943in}}%
\pgfpathlineto{\pgfqpoint{5.513105in}{0.739001in}}%
\pgfpathlineto{\pgfqpoint{5.515572in}{0.744726in}}%
\pgfpathlineto{\pgfqpoint{5.518039in}{0.759311in}}%
\pgfpathlineto{\pgfqpoint{5.520506in}{0.873607in}}%
\pgfpathlineto{\pgfqpoint{5.522973in}{0.784613in}}%
\pgfpathlineto{\pgfqpoint{5.525441in}{0.748841in}}%
\pgfpathlineto{\pgfqpoint{5.527908in}{0.739493in}}%
\pgfpathlineto{\pgfqpoint{5.532842in}{0.732713in}}%
\pgfpathlineto{\pgfqpoint{5.540243in}{0.729142in}}%
\pgfpathlineto{\pgfqpoint{5.552579in}{0.726949in}}%
\pgfpathlineto{\pgfqpoint{5.577251in}{0.725567in}}%
\pgfpathlineto{\pgfqpoint{5.621659in}{0.725649in}}%
\pgfpathlineto{\pgfqpoint{5.673469in}{0.727557in}}%
\pgfpathlineto{\pgfqpoint{5.688272in}{0.730130in}}%
\pgfpathlineto{\pgfqpoint{5.693206in}{0.732465in}}%
\pgfpathlineto{\pgfqpoint{5.695673in}{0.734603in}}%
\pgfpathlineto{\pgfqpoint{5.698141in}{0.738386in}}%
\pgfpathlineto{\pgfqpoint{5.700608in}{0.746905in}}%
\pgfpathlineto{\pgfqpoint{5.705542in}{0.812078in}}%
\pgfpathlineto{\pgfqpoint{5.708009in}{0.749765in}}%
\pgfpathlineto{\pgfqpoint{5.710476in}{0.739300in}}%
\pgfpathlineto{\pgfqpoint{5.715411in}{0.732627in}}%
\pgfpathlineto{\pgfqpoint{5.722812in}{0.729378in}}%
\pgfpathlineto{\pgfqpoint{5.735148in}{0.727457in}}%
\pgfpathlineto{\pgfqpoint{5.762286in}{0.726136in}}%
\pgfpathlineto{\pgfqpoint{5.838768in}{0.725391in}}%
\pgfpathlineto{\pgfqpoint{5.885643in}{0.725125in}}%
\pgfpathlineto{\pgfqpoint{5.888111in}{0.728070in}}%
\pgfpathlineto{\pgfqpoint{5.890578in}{0.725536in}}%
\pgfpathlineto{\pgfqpoint{5.920183in}{0.725319in}}%
\pgfpathlineto{\pgfqpoint{6.053409in}{0.725305in}}%
\pgfpathlineto{\pgfqpoint{6.053409in}{0.725305in}}%
\pgfusepath{stroke}%
\end{pgfscope}%
\begin{pgfscope}%
\pgfsetrectcap%
\pgfsetmiterjoin%
\pgfsetlinewidth{0.803000pt}%
\definecolor{currentstroke}{rgb}{0.000000,0.000000,0.000000}%
\pgfsetstrokecolor{currentstroke}%
\pgfsetdash{}{0pt}%
\pgfpathmoveto{\pgfqpoint{0.875000in}{0.550000in}}%
\pgfpathlineto{\pgfqpoint{0.875000in}{4.400000in}}%
\pgfusepath{stroke}%
\end{pgfscope}%
\begin{pgfscope}%
\pgfsetrectcap%
\pgfsetmiterjoin%
\pgfsetlinewidth{0.803000pt}%
\definecolor{currentstroke}{rgb}{0.000000,0.000000,0.000000}%
\pgfsetstrokecolor{currentstroke}%
\pgfsetdash{}{0pt}%
\pgfpathmoveto{\pgfqpoint{6.300000in}{0.550000in}}%
\pgfpathlineto{\pgfqpoint{6.300000in}{4.400000in}}%
\pgfusepath{stroke}%
\end{pgfscope}%
\begin{pgfscope}%
\pgfsetrectcap%
\pgfsetmiterjoin%
\pgfsetlinewidth{0.803000pt}%
\definecolor{currentstroke}{rgb}{0.000000,0.000000,0.000000}%
\pgfsetstrokecolor{currentstroke}%
\pgfsetdash{}{0pt}%
\pgfpathmoveto{\pgfqpoint{0.875000in}{0.550000in}}%
\pgfpathlineto{\pgfqpoint{6.300000in}{0.550000in}}%
\pgfusepath{stroke}%
\end{pgfscope}%
\begin{pgfscope}%
\pgfsetrectcap%
\pgfsetmiterjoin%
\pgfsetlinewidth{0.803000pt}%
\definecolor{currentstroke}{rgb}{0.000000,0.000000,0.000000}%
\pgfsetstrokecolor{currentstroke}%
\pgfsetdash{}{0pt}%
\pgfpathmoveto{\pgfqpoint{0.875000in}{4.400000in}}%
\pgfpathlineto{\pgfqpoint{6.300000in}{4.400000in}}%
\pgfusepath{stroke}%
\end{pgfscope}%
\end{pgfpicture}%
\makeatother%
\endgroup%
}
    \end{center}
    \caption{Spettro di un'onda a dente di sega}
\end{figure}

Lo spettro di frequenze di un segnale può essere rappresentato graficamente. Incontriamo generalmente due tipi fondamentali di rappresentazioni grafiche. La prima è la rappresentazione dello spettro istantaneo, che è un grafico bidimensionale con la frequenza sull'asse orizzontale e l'ampiezza, generalmente non segnata, sull'asse verticale;%
\footnote{In certi casi è utile rappresentare l'ampiezza come un valore segnato: il segno negativo, in questo caso, rappresenta un'inversione di fase di una componente rispetto a quelle con segno positivo.}
questa rappresentazione descrive lo spettro di un segnale in maniera indipendente dal tempo, assumendo quindi che questo possa estendersi indefinitamente nel futuro e nel passato; o anche lo spettro di un segnale considerato nel breve termine di una data finestra temporale, come per esempio lo spettro prodotto da una corda di chitarra un secondo dopo che è stata pizzicata.

La seconda rappresentazione è lo spettrogramma, che è un grafico tridimensionale che rappresenta una successione temporale di spettri a breve termine. Lo spettrogramma può essere molto utile per ``leggere'' il comportamento del suono nel tempo e, in generale, è una rappresentazione molto più eloquente e leggibile del grafico della forma d'onda, perché più prossima alla nostra percezione. Lo spettrogramma rappresenta il tempo sull'asse orizzontale, la frequenza sull'asse verticale e l'ampiezza (sempre non negativa) sull'asse della profondità, che spesso è rappresentato tramite una scala di grigi o di colori. Il calcolo dello spettrogramma, che è spesso basato sulla STFT, non fornisce un risultato continuo, ma suddivide sia l'asse del tempo che quello della frequenza in ``finestre'' discrete: maggiore è la risoluzione sull'asse temporale e minore sarà quella sull'asse frequenziale, e viceversa. 

I grafici del segnale nel dominio della frequenza possono mostrare la frequenza e l'ampiezza in maniera lineare o logaritmica (il tempo degli spettrogrammi invece è generalmente mostrato in maniera lineare).

\begin{figure}
    \begin{center}
       %\scalebox{0.6} {\input{figures/spettrogramma_dente_di_sega-img0.png}}
       \scalebox{0.6}{\includegraphics{figures/cherooke.png}}
    \end{center}
    \caption{Spettrogramma vocale (Cherooke dagli esempi audio di Max MSP)}
\end{figure}

Se la frequenza è rappresentata in maniera lineare, una stessa differenza frequenziale corrisponderà a una stessa distanza nel grafico: per esempio, una differenza di 1000 Hz potrebbe corrispondere a 1 cm sull'asse orizzontale. Lo stesso vale per l'ampiezza e le grandezze a essa linearmente correlate: per esempio, una differenza di 0.1 in ampiezza normalizzata, o una differenza di 0.5 V, potrebbero corrispondere a 1 cm sull'asse verticale. 

Se invece la frequenza è rappresentata in maniera logaritmica, uguali rapporti di frequenza corrisponderanno a uguali distanze nel grafico: per esempio, un rapporto frequenziale di 2:1, ovvero un intervallo d'ottava, potrebbe corrispondere a 1 cm nel grafico; questo vuol dire che una rappresentazione frequenziale logaritmica è lineare nello spazio delle altezze, per cui intervalli uguali corrisponderanno a distanze grafiche uguali. Uno spettrogramma con rappresentazione logaritmica delle frequenze è un oggetto piuttosto simile a una partitura. Se l'ampiezza è rappresentata su una scala logaritmica, questo equivale a dire che le differenze in decibel sono mostrate linearmente: per esempio, un rapporto di ampiezze di 2:1, ovvero una differenza di circa 6 dB, potrebbe corrispondere a 1 cm. 





\section{Categorie di spettri}

Possiamo suddividere gli spettri in alcune categorie differenti. 


\subsection{Spettri armonici}

Come già detto, un segnale periodico nel dominio del tempo produce uno spettro armonico. Questo vuol dire che le frequenze di tutte le sue componenti sono multiple della frequenza fondamentale, corrispondente all'inverso del periodo del segnale.

In una rappresentazione grafica nel dominio della frequenza rappresentata linearmente, tutte le sue componenti sono equidistanti sull'asse delle ascisse, e la stessa distanza c'è tra la fondamentale e l'asse delle ordinate.

Il caso più semplice di spettro armonico è quello della sinusoide, che ha una sola componente di ampiezza non nulla, alla frequenza fondamentale.

È utile conoscere le conformazioni spettrali di alcuni altri segnali periodici:

\begin{itemize}

\item L'onda a dente di sega è composta da una somma di sinusoidi armoniche con termine di fase nullo. L'ampiezza di ciascuna armonica è inversamente proporzionale al suo ordine: se la fondamentale (armonica 1) ha ampiezza 1, allora la seconda armonica (di frequenza doppia) ha ampiezza $\frac{1}{2}$, la terza armonica (di frequenza tripla rispetto alla fondamentale) ha ampiezza $\frac{1}{3}$ e così via. In termini percettivi, questo produce un suono molto brillante: la decima armonica ha ampiezza $\frac{1}{10}$, corrispondente a -20 dB, rispetto alla fondamentale; la centesima armonica ha ampiezza $\frac{1}{100}$, corrispondente a -40 dB, il che vuol dire che nelle giuste condizioni può essere ancora chiaramente percepibile.

\begin{figure}
    \begin{center}
       \scalebox{0.6} {%% Creator: Matplotlib, PGF backend
%%
%% To include the figure in your LaTeX document, write
%%   \input{<filename>.pgf}
%%
%% Make sure the required packages are loaded in your preamble
%%   \usepackage{pgf}
%%
%% Also ensure that all the required font packages are loaded; for instance,
%% the lmodern package is sometimes necessary when using math font.
%%   \usepackage{lmodern}
%%
%% Figures using additional raster images can only be included by \input if
%% they are in the same directory as the main LaTeX file. For loading figures
%% from other directories you can use the `import` package
%%   \usepackage{import}
%%
%% and then include the figures with
%%   \import{<path to file>}{<filename>.pgf}
%%
%% Matplotlib used the following preamble
%%   
%%   \makeatletter\@ifpackageloaded{underscore}{}{\usepackage[strings]{underscore}}\makeatother
%%
\begingroup%
\makeatletter%
\begin{pgfpicture}%
\pgfpathrectangle{\pgfpointorigin}{\pgfqpoint{6.400000in}{4.800000in}}%
\pgfusepath{use as bounding box, clip}%
\begin{pgfscope}%
\pgfsetbuttcap%
\pgfsetmiterjoin%
\definecolor{currentfill}{rgb}{1.000000,1.000000,1.000000}%
\pgfsetfillcolor{currentfill}%
\pgfsetlinewidth{0.000000pt}%
\definecolor{currentstroke}{rgb}{1.000000,1.000000,1.000000}%
\pgfsetstrokecolor{currentstroke}%
\pgfsetdash{}{0pt}%
\pgfpathmoveto{\pgfqpoint{0.000000in}{0.000000in}}%
\pgfpathlineto{\pgfqpoint{6.400000in}{0.000000in}}%
\pgfpathlineto{\pgfqpoint{6.400000in}{4.800000in}}%
\pgfpathlineto{\pgfqpoint{0.000000in}{4.800000in}}%
\pgfpathlineto{\pgfqpoint{0.000000in}{0.000000in}}%
\pgfpathclose%
\pgfusepath{fill}%
\end{pgfscope}%
\begin{pgfscope}%
\pgfsetbuttcap%
\pgfsetmiterjoin%
\definecolor{currentfill}{rgb}{1.000000,1.000000,1.000000}%
\pgfsetfillcolor{currentfill}%
\pgfsetlinewidth{0.000000pt}%
\definecolor{currentstroke}{rgb}{0.000000,0.000000,0.000000}%
\pgfsetstrokecolor{currentstroke}%
\pgfsetstrokeopacity{0.000000}%
\pgfsetdash{}{0pt}%
\pgfpathmoveto{\pgfqpoint{0.800000in}{0.528000in}}%
\pgfpathlineto{\pgfqpoint{5.760000in}{0.528000in}}%
\pgfpathlineto{\pgfqpoint{5.760000in}{4.224000in}}%
\pgfpathlineto{\pgfqpoint{0.800000in}{4.224000in}}%
\pgfpathlineto{\pgfqpoint{0.800000in}{0.528000in}}%
\pgfpathclose%
\pgfusepath{fill}%
\end{pgfscope}%
\begin{pgfscope}%
\pgfsetbuttcap%
\pgfsetroundjoin%
\definecolor{currentfill}{rgb}{0.000000,0.000000,0.000000}%
\pgfsetfillcolor{currentfill}%
\pgfsetlinewidth{0.803000pt}%
\definecolor{currentstroke}{rgb}{0.000000,0.000000,0.000000}%
\pgfsetstrokecolor{currentstroke}%
\pgfsetdash{}{0pt}%
\pgfsys@defobject{currentmarker}{\pgfqpoint{0.000000in}{-0.048611in}}{\pgfqpoint{0.000000in}{0.000000in}}{%
\pgfpathmoveto{\pgfqpoint{0.000000in}{0.000000in}}%
\pgfpathlineto{\pgfqpoint{0.000000in}{-0.048611in}}%
\pgfusepath{stroke,fill}%
}%
\begin{pgfscope}%
\pgfsys@transformshift{1.025455in}{0.528000in}%
\pgfsys@useobject{currentmarker}{}%
\end{pgfscope}%
\end{pgfscope}%
\begin{pgfscope}%
\definecolor{textcolor}{rgb}{0.000000,0.000000,0.000000}%
\pgfsetstrokecolor{textcolor}%
\pgfsetfillcolor{textcolor}%
\pgftext[x=1.025455in,y=0.430778in,,top]{\color{textcolor}\rmfamily\fontsize{10.000000}{12.000000}\selectfont \(\displaystyle {0}\)}%
\end{pgfscope}%
\begin{pgfscope}%
\pgfsetbuttcap%
\pgfsetroundjoin%
\definecolor{currentfill}{rgb}{0.000000,0.000000,0.000000}%
\pgfsetfillcolor{currentfill}%
\pgfsetlinewidth{0.803000pt}%
\definecolor{currentstroke}{rgb}{0.000000,0.000000,0.000000}%
\pgfsetstrokecolor{currentstroke}%
\pgfsetdash{}{0pt}%
\pgfsys@defobject{currentmarker}{\pgfqpoint{0.000000in}{-0.048611in}}{\pgfqpoint{0.000000in}{0.000000in}}{%
\pgfpathmoveto{\pgfqpoint{0.000000in}{0.000000in}}%
\pgfpathlineto{\pgfqpoint{0.000000in}{-0.048611in}}%
\pgfusepath{stroke,fill}%
}%
\begin{pgfscope}%
\pgfsys@transformshift{2.047971in}{0.528000in}%
\pgfsys@useobject{currentmarker}{}%
\end{pgfscope}%
\end{pgfscope}%
\begin{pgfscope}%
\definecolor{textcolor}{rgb}{0.000000,0.000000,0.000000}%
\pgfsetstrokecolor{textcolor}%
\pgfsetfillcolor{textcolor}%
\pgftext[x=2.047971in,y=0.430778in,,top]{\color{textcolor}\rmfamily\fontsize{10.000000}{12.000000}\selectfont \(\displaystyle {5000}\)}%
\end{pgfscope}%
\begin{pgfscope}%
\pgfsetbuttcap%
\pgfsetroundjoin%
\definecolor{currentfill}{rgb}{0.000000,0.000000,0.000000}%
\pgfsetfillcolor{currentfill}%
\pgfsetlinewidth{0.803000pt}%
\definecolor{currentstroke}{rgb}{0.000000,0.000000,0.000000}%
\pgfsetstrokecolor{currentstroke}%
\pgfsetdash{}{0pt}%
\pgfsys@defobject{currentmarker}{\pgfqpoint{0.000000in}{-0.048611in}}{\pgfqpoint{0.000000in}{0.000000in}}{%
\pgfpathmoveto{\pgfqpoint{0.000000in}{0.000000in}}%
\pgfpathlineto{\pgfqpoint{0.000000in}{-0.048611in}}%
\pgfusepath{stroke,fill}%
}%
\begin{pgfscope}%
\pgfsys@transformshift{3.070486in}{0.528000in}%
\pgfsys@useobject{currentmarker}{}%
\end{pgfscope}%
\end{pgfscope}%
\begin{pgfscope}%
\definecolor{textcolor}{rgb}{0.000000,0.000000,0.000000}%
\pgfsetstrokecolor{textcolor}%
\pgfsetfillcolor{textcolor}%
\pgftext[x=3.070486in,y=0.430778in,,top]{\color{textcolor}\rmfamily\fontsize{10.000000}{12.000000}\selectfont \(\displaystyle {10000}\)}%
\end{pgfscope}%
\begin{pgfscope}%
\pgfsetbuttcap%
\pgfsetroundjoin%
\definecolor{currentfill}{rgb}{0.000000,0.000000,0.000000}%
\pgfsetfillcolor{currentfill}%
\pgfsetlinewidth{0.803000pt}%
\definecolor{currentstroke}{rgb}{0.000000,0.000000,0.000000}%
\pgfsetstrokecolor{currentstroke}%
\pgfsetdash{}{0pt}%
\pgfsys@defobject{currentmarker}{\pgfqpoint{0.000000in}{-0.048611in}}{\pgfqpoint{0.000000in}{0.000000in}}{%
\pgfpathmoveto{\pgfqpoint{0.000000in}{0.000000in}}%
\pgfpathlineto{\pgfqpoint{0.000000in}{-0.048611in}}%
\pgfusepath{stroke,fill}%
}%
\begin{pgfscope}%
\pgfsys@transformshift{4.093002in}{0.528000in}%
\pgfsys@useobject{currentmarker}{}%
\end{pgfscope}%
\end{pgfscope}%
\begin{pgfscope}%
\definecolor{textcolor}{rgb}{0.000000,0.000000,0.000000}%
\pgfsetstrokecolor{textcolor}%
\pgfsetfillcolor{textcolor}%
\pgftext[x=4.093002in,y=0.430778in,,top]{\color{textcolor}\rmfamily\fontsize{10.000000}{12.000000}\selectfont \(\displaystyle {15000}\)}%
\end{pgfscope}%
\begin{pgfscope}%
\pgfsetbuttcap%
\pgfsetroundjoin%
\definecolor{currentfill}{rgb}{0.000000,0.000000,0.000000}%
\pgfsetfillcolor{currentfill}%
\pgfsetlinewidth{0.803000pt}%
\definecolor{currentstroke}{rgb}{0.000000,0.000000,0.000000}%
\pgfsetstrokecolor{currentstroke}%
\pgfsetdash{}{0pt}%
\pgfsys@defobject{currentmarker}{\pgfqpoint{0.000000in}{-0.048611in}}{\pgfqpoint{0.000000in}{0.000000in}}{%
\pgfpathmoveto{\pgfqpoint{0.000000in}{0.000000in}}%
\pgfpathlineto{\pgfqpoint{0.000000in}{-0.048611in}}%
\pgfusepath{stroke,fill}%
}%
\begin{pgfscope}%
\pgfsys@transformshift{5.115518in}{0.528000in}%
\pgfsys@useobject{currentmarker}{}%
\end{pgfscope}%
\end{pgfscope}%
\begin{pgfscope}%
\definecolor{textcolor}{rgb}{0.000000,0.000000,0.000000}%
\pgfsetstrokecolor{textcolor}%
\pgfsetfillcolor{textcolor}%
\pgftext[x=5.115518in,y=0.430778in,,top]{\color{textcolor}\rmfamily\fontsize{10.000000}{12.000000}\selectfont \(\displaystyle {20000}\)}%
\end{pgfscope}%
\begin{pgfscope}%
\pgfsetbuttcap%
\pgfsetroundjoin%
\definecolor{currentfill}{rgb}{0.000000,0.000000,0.000000}%
\pgfsetfillcolor{currentfill}%
\pgfsetlinewidth{0.803000pt}%
\definecolor{currentstroke}{rgb}{0.000000,0.000000,0.000000}%
\pgfsetstrokecolor{currentstroke}%
\pgfsetdash{}{0pt}%
\pgfsys@defobject{currentmarker}{\pgfqpoint{-0.048611in}{0.000000in}}{\pgfqpoint{-0.000000in}{0.000000in}}{%
\pgfpathmoveto{\pgfqpoint{-0.000000in}{0.000000in}}%
\pgfpathlineto{\pgfqpoint{-0.048611in}{0.000000in}}%
\pgfusepath{stroke,fill}%
}%
\begin{pgfscope}%
\pgfsys@transformshift{0.800000in}{0.546200in}%
\pgfsys@useobject{currentmarker}{}%
\end{pgfscope}%
\end{pgfscope}%
\begin{pgfscope}%
\definecolor{textcolor}{rgb}{0.000000,0.000000,0.000000}%
\pgfsetstrokecolor{textcolor}%
\pgfsetfillcolor{textcolor}%
\pgftext[x=0.417283in, y=0.497975in, left, base]{\color{textcolor}\rmfamily\fontsize{10.000000}{12.000000}\selectfont \(\displaystyle {\ensuremath{-}2.0}\)}%
\end{pgfscope}%
\begin{pgfscope}%
\pgfsetbuttcap%
\pgfsetroundjoin%
\definecolor{currentfill}{rgb}{0.000000,0.000000,0.000000}%
\pgfsetfillcolor{currentfill}%
\pgfsetlinewidth{0.803000pt}%
\definecolor{currentstroke}{rgb}{0.000000,0.000000,0.000000}%
\pgfsetstrokecolor{currentstroke}%
\pgfsetdash{}{0pt}%
\pgfsys@defobject{currentmarker}{\pgfqpoint{-0.048611in}{0.000000in}}{\pgfqpoint{-0.000000in}{0.000000in}}{%
\pgfpathmoveto{\pgfqpoint{-0.000000in}{0.000000in}}%
\pgfpathlineto{\pgfqpoint{-0.048611in}{0.000000in}}%
\pgfusepath{stroke,fill}%
}%
\begin{pgfscope}%
\pgfsys@transformshift{0.800000in}{1.003650in}%
\pgfsys@useobject{currentmarker}{}%
\end{pgfscope}%
\end{pgfscope}%
\begin{pgfscope}%
\definecolor{textcolor}{rgb}{0.000000,0.000000,0.000000}%
\pgfsetstrokecolor{textcolor}%
\pgfsetfillcolor{textcolor}%
\pgftext[x=0.417283in, y=0.955425in, left, base]{\color{textcolor}\rmfamily\fontsize{10.000000}{12.000000}\selectfont \(\displaystyle {\ensuremath{-}1.5}\)}%
\end{pgfscope}%
\begin{pgfscope}%
\pgfsetbuttcap%
\pgfsetroundjoin%
\definecolor{currentfill}{rgb}{0.000000,0.000000,0.000000}%
\pgfsetfillcolor{currentfill}%
\pgfsetlinewidth{0.803000pt}%
\definecolor{currentstroke}{rgb}{0.000000,0.000000,0.000000}%
\pgfsetstrokecolor{currentstroke}%
\pgfsetdash{}{0pt}%
\pgfsys@defobject{currentmarker}{\pgfqpoint{-0.048611in}{0.000000in}}{\pgfqpoint{-0.000000in}{0.000000in}}{%
\pgfpathmoveto{\pgfqpoint{-0.000000in}{0.000000in}}%
\pgfpathlineto{\pgfqpoint{-0.048611in}{0.000000in}}%
\pgfusepath{stroke,fill}%
}%
\begin{pgfscope}%
\pgfsys@transformshift{0.800000in}{1.461100in}%
\pgfsys@useobject{currentmarker}{}%
\end{pgfscope}%
\end{pgfscope}%
\begin{pgfscope}%
\definecolor{textcolor}{rgb}{0.000000,0.000000,0.000000}%
\pgfsetstrokecolor{textcolor}%
\pgfsetfillcolor{textcolor}%
\pgftext[x=0.417283in, y=1.412875in, left, base]{\color{textcolor}\rmfamily\fontsize{10.000000}{12.000000}\selectfont \(\displaystyle {\ensuremath{-}1.0}\)}%
\end{pgfscope}%
\begin{pgfscope}%
\pgfsetbuttcap%
\pgfsetroundjoin%
\definecolor{currentfill}{rgb}{0.000000,0.000000,0.000000}%
\pgfsetfillcolor{currentfill}%
\pgfsetlinewidth{0.803000pt}%
\definecolor{currentstroke}{rgb}{0.000000,0.000000,0.000000}%
\pgfsetstrokecolor{currentstroke}%
\pgfsetdash{}{0pt}%
\pgfsys@defobject{currentmarker}{\pgfqpoint{-0.048611in}{0.000000in}}{\pgfqpoint{-0.000000in}{0.000000in}}{%
\pgfpathmoveto{\pgfqpoint{-0.000000in}{0.000000in}}%
\pgfpathlineto{\pgfqpoint{-0.048611in}{0.000000in}}%
\pgfusepath{stroke,fill}%
}%
\begin{pgfscope}%
\pgfsys@transformshift{0.800000in}{1.918550in}%
\pgfsys@useobject{currentmarker}{}%
\end{pgfscope}%
\end{pgfscope}%
\begin{pgfscope}%
\definecolor{textcolor}{rgb}{0.000000,0.000000,0.000000}%
\pgfsetstrokecolor{textcolor}%
\pgfsetfillcolor{textcolor}%
\pgftext[x=0.417283in, y=1.870325in, left, base]{\color{textcolor}\rmfamily\fontsize{10.000000}{12.000000}\selectfont \(\displaystyle {\ensuremath{-}0.5}\)}%
\end{pgfscope}%
\begin{pgfscope}%
\pgfsetbuttcap%
\pgfsetroundjoin%
\definecolor{currentfill}{rgb}{0.000000,0.000000,0.000000}%
\pgfsetfillcolor{currentfill}%
\pgfsetlinewidth{0.803000pt}%
\definecolor{currentstroke}{rgb}{0.000000,0.000000,0.000000}%
\pgfsetstrokecolor{currentstroke}%
\pgfsetdash{}{0pt}%
\pgfsys@defobject{currentmarker}{\pgfqpoint{-0.048611in}{0.000000in}}{\pgfqpoint{-0.000000in}{0.000000in}}{%
\pgfpathmoveto{\pgfqpoint{-0.000000in}{0.000000in}}%
\pgfpathlineto{\pgfqpoint{-0.048611in}{0.000000in}}%
\pgfusepath{stroke,fill}%
}%
\begin{pgfscope}%
\pgfsys@transformshift{0.800000in}{2.376000in}%
\pgfsys@useobject{currentmarker}{}%
\end{pgfscope}%
\end{pgfscope}%
\begin{pgfscope}%
\definecolor{textcolor}{rgb}{0.000000,0.000000,0.000000}%
\pgfsetstrokecolor{textcolor}%
\pgfsetfillcolor{textcolor}%
\pgftext[x=0.525308in, y=2.327775in, left, base]{\color{textcolor}\rmfamily\fontsize{10.000000}{12.000000}\selectfont \(\displaystyle {0.0}\)}%
\end{pgfscope}%
\begin{pgfscope}%
\pgfsetbuttcap%
\pgfsetroundjoin%
\definecolor{currentfill}{rgb}{0.000000,0.000000,0.000000}%
\pgfsetfillcolor{currentfill}%
\pgfsetlinewidth{0.803000pt}%
\definecolor{currentstroke}{rgb}{0.000000,0.000000,0.000000}%
\pgfsetstrokecolor{currentstroke}%
\pgfsetdash{}{0pt}%
\pgfsys@defobject{currentmarker}{\pgfqpoint{-0.048611in}{0.000000in}}{\pgfqpoint{-0.000000in}{0.000000in}}{%
\pgfpathmoveto{\pgfqpoint{-0.000000in}{0.000000in}}%
\pgfpathlineto{\pgfqpoint{-0.048611in}{0.000000in}}%
\pgfusepath{stroke,fill}%
}%
\begin{pgfscope}%
\pgfsys@transformshift{0.800000in}{2.833450in}%
\pgfsys@useobject{currentmarker}{}%
\end{pgfscope}%
\end{pgfscope}%
\begin{pgfscope}%
\definecolor{textcolor}{rgb}{0.000000,0.000000,0.000000}%
\pgfsetstrokecolor{textcolor}%
\pgfsetfillcolor{textcolor}%
\pgftext[x=0.525308in, y=2.785225in, left, base]{\color{textcolor}\rmfamily\fontsize{10.000000}{12.000000}\selectfont \(\displaystyle {0.5}\)}%
\end{pgfscope}%
\begin{pgfscope}%
\pgfsetbuttcap%
\pgfsetroundjoin%
\definecolor{currentfill}{rgb}{0.000000,0.000000,0.000000}%
\pgfsetfillcolor{currentfill}%
\pgfsetlinewidth{0.803000pt}%
\definecolor{currentstroke}{rgb}{0.000000,0.000000,0.000000}%
\pgfsetstrokecolor{currentstroke}%
\pgfsetdash{}{0pt}%
\pgfsys@defobject{currentmarker}{\pgfqpoint{-0.048611in}{0.000000in}}{\pgfqpoint{-0.000000in}{0.000000in}}{%
\pgfpathmoveto{\pgfqpoint{-0.000000in}{0.000000in}}%
\pgfpathlineto{\pgfqpoint{-0.048611in}{0.000000in}}%
\pgfusepath{stroke,fill}%
}%
\begin{pgfscope}%
\pgfsys@transformshift{0.800000in}{3.290900in}%
\pgfsys@useobject{currentmarker}{}%
\end{pgfscope}%
\end{pgfscope}%
\begin{pgfscope}%
\definecolor{textcolor}{rgb}{0.000000,0.000000,0.000000}%
\pgfsetstrokecolor{textcolor}%
\pgfsetfillcolor{textcolor}%
\pgftext[x=0.525308in, y=3.242675in, left, base]{\color{textcolor}\rmfamily\fontsize{10.000000}{12.000000}\selectfont \(\displaystyle {1.0}\)}%
\end{pgfscope}%
\begin{pgfscope}%
\pgfsetbuttcap%
\pgfsetroundjoin%
\definecolor{currentfill}{rgb}{0.000000,0.000000,0.000000}%
\pgfsetfillcolor{currentfill}%
\pgfsetlinewidth{0.803000pt}%
\definecolor{currentstroke}{rgb}{0.000000,0.000000,0.000000}%
\pgfsetstrokecolor{currentstroke}%
\pgfsetdash{}{0pt}%
\pgfsys@defobject{currentmarker}{\pgfqpoint{-0.048611in}{0.000000in}}{\pgfqpoint{-0.000000in}{0.000000in}}{%
\pgfpathmoveto{\pgfqpoint{-0.000000in}{0.000000in}}%
\pgfpathlineto{\pgfqpoint{-0.048611in}{0.000000in}}%
\pgfusepath{stroke,fill}%
}%
\begin{pgfscope}%
\pgfsys@transformshift{0.800000in}{3.748350in}%
\pgfsys@useobject{currentmarker}{}%
\end{pgfscope}%
\end{pgfscope}%
\begin{pgfscope}%
\definecolor{textcolor}{rgb}{0.000000,0.000000,0.000000}%
\pgfsetstrokecolor{textcolor}%
\pgfsetfillcolor{textcolor}%
\pgftext[x=0.525308in, y=3.700125in, left, base]{\color{textcolor}\rmfamily\fontsize{10.000000}{12.000000}\selectfont \(\displaystyle {1.5}\)}%
\end{pgfscope}%
\begin{pgfscope}%
\pgfsetbuttcap%
\pgfsetroundjoin%
\definecolor{currentfill}{rgb}{0.000000,0.000000,0.000000}%
\pgfsetfillcolor{currentfill}%
\pgfsetlinewidth{0.803000pt}%
\definecolor{currentstroke}{rgb}{0.000000,0.000000,0.000000}%
\pgfsetstrokecolor{currentstroke}%
\pgfsetdash{}{0pt}%
\pgfsys@defobject{currentmarker}{\pgfqpoint{-0.048611in}{0.000000in}}{\pgfqpoint{-0.000000in}{0.000000in}}{%
\pgfpathmoveto{\pgfqpoint{-0.000000in}{0.000000in}}%
\pgfpathlineto{\pgfqpoint{-0.048611in}{0.000000in}}%
\pgfusepath{stroke,fill}%
}%
\begin{pgfscope}%
\pgfsys@transformshift{0.800000in}{4.205800in}%
\pgfsys@useobject{currentmarker}{}%
\end{pgfscope}%
\end{pgfscope}%
\begin{pgfscope}%
\definecolor{textcolor}{rgb}{0.000000,0.000000,0.000000}%
\pgfsetstrokecolor{textcolor}%
\pgfsetfillcolor{textcolor}%
\pgftext[x=0.525308in, y=4.157575in, left, base]{\color{textcolor}\rmfamily\fontsize{10.000000}{12.000000}\selectfont \(\displaystyle {2.0}\)}%
\end{pgfscope}%
\begin{pgfscope}%
\pgfpathrectangle{\pgfqpoint{0.800000in}{0.528000in}}{\pgfqpoint{4.960000in}{3.696000in}}%
\pgfusepath{clip}%
\pgfsetrectcap%
\pgfsetroundjoin%
\pgfsetlinewidth{1.505625pt}%
\definecolor{currentstroke}{rgb}{0.121569,0.466667,0.705882}%
\pgfsetstrokecolor{currentstroke}%
\pgfsetdash{}{0pt}%
\pgfpathmoveto{\pgfqpoint{1.025455in}{2.376000in}}%
\pgfpathlineto{\pgfqpoint{1.031999in}{3.761365in}}%
\pgfpathlineto{\pgfqpoint{1.035066in}{4.026061in}}%
\pgfpathlineto{\pgfqpoint{1.036702in}{4.056000in}}%
\pgfpathlineto{\pgfqpoint{1.036907in}{4.055128in}}%
\pgfpathlineto{\pgfqpoint{1.037929in}{4.037800in}}%
\pgfpathlineto{\pgfqpoint{1.040179in}{3.943639in}}%
\pgfpathlineto{\pgfqpoint{1.046518in}{3.657639in}}%
\pgfpathlineto{\pgfqpoint{1.047950in}{3.644787in}}%
\pgfpathlineto{\pgfqpoint{1.048359in}{3.645488in}}%
\pgfpathlineto{\pgfqpoint{1.049381in}{3.654953in}}%
\pgfpathlineto{\pgfqpoint{1.051631in}{3.705914in}}%
\pgfpathlineto{\pgfqpoint{1.057971in}{3.861843in}}%
\pgfpathlineto{\pgfqpoint{1.058993in}{3.865471in}}%
\pgfpathlineto{\pgfqpoint{1.059402in}{3.864667in}}%
\pgfpathlineto{\pgfqpoint{1.060629in}{3.854985in}}%
\pgfpathlineto{\pgfqpoint{1.063083in}{3.809886in}}%
\pgfpathlineto{\pgfqpoint{1.069423in}{3.687416in}}%
\pgfpathlineto{\pgfqpoint{1.070650in}{3.683819in}}%
\pgfpathlineto{\pgfqpoint{1.070854in}{3.684053in}}%
\pgfpathlineto{\pgfqpoint{1.071877in}{3.688588in}}%
\pgfpathlineto{\pgfqpoint{1.073922in}{3.711636in}}%
\pgfpathlineto{\pgfqpoint{1.080875in}{3.799306in}}%
\pgfpathlineto{\pgfqpoint{1.081488in}{3.799535in}}%
\pgfpathlineto{\pgfqpoint{1.081693in}{3.799228in}}%
\pgfpathlineto{\pgfqpoint{1.082920in}{3.793487in}}%
\pgfpathlineto{\pgfqpoint{1.085169in}{3.768351in}}%
\pgfpathlineto{\pgfqpoint{1.092327in}{3.679883in}}%
\pgfpathlineto{\pgfqpoint{1.093145in}{3.678771in}}%
\pgfpathlineto{\pgfqpoint{1.093554in}{3.679160in}}%
\pgfpathlineto{\pgfqpoint{1.094781in}{3.683853in}}%
\pgfpathlineto{\pgfqpoint{1.097235in}{3.705369in}}%
\pgfpathlineto{\pgfqpoint{1.102757in}{3.753668in}}%
\pgfpathlineto{\pgfqpoint{1.103779in}{3.754728in}}%
\pgfpathlineto{\pgfqpoint{1.103984in}{3.754530in}}%
\pgfpathlineto{\pgfqpoint{1.105211in}{3.750518in}}%
\pgfpathlineto{\pgfqpoint{1.107460in}{3.732277in}}%
\pgfpathlineto{\pgfqpoint{1.115027in}{3.662588in}}%
\pgfpathlineto{\pgfqpoint{1.115845in}{3.662113in}}%
\pgfpathlineto{\pgfqpoint{1.116254in}{3.662576in}}%
\pgfpathlineto{\pgfqpoint{1.117685in}{3.667553in}}%
\pgfpathlineto{\pgfqpoint{1.120753in}{3.689910in}}%
\pgfpathlineto{\pgfqpoint{1.124843in}{3.715702in}}%
\pgfpathlineto{\pgfqpoint{1.126070in}{3.717095in}}%
\pgfpathlineto{\pgfqpoint{1.126275in}{3.716958in}}%
\pgfpathlineto{\pgfqpoint{1.127502in}{3.713919in}}%
\pgfpathlineto{\pgfqpoint{1.129751in}{3.699613in}}%
\pgfpathlineto{\pgfqpoint{1.137727in}{3.640825in}}%
\pgfpathlineto{\pgfqpoint{1.138749in}{3.640935in}}%
\pgfpathlineto{\pgfqpoint{1.138954in}{3.641232in}}%
\pgfpathlineto{\pgfqpoint{1.140590in}{3.646511in}}%
\pgfpathlineto{\pgfqpoint{1.144680in}{3.671206in}}%
\pgfpathlineto{\pgfqpoint{1.147747in}{3.682533in}}%
\pgfpathlineto{\pgfqpoint{1.148770in}{3.682474in}}%
\pgfpathlineto{\pgfqpoint{1.148974in}{3.682201in}}%
\pgfpathlineto{\pgfqpoint{1.150406in}{3.677907in}}%
\pgfpathlineto{\pgfqpoint{1.153065in}{3.660834in}}%
\pgfpathlineto{\pgfqpoint{1.159404in}{3.618397in}}%
\pgfpathlineto{\pgfqpoint{1.161040in}{3.616808in}}%
\pgfpathlineto{\pgfqpoint{1.162267in}{3.618802in}}%
\pgfpathlineto{\pgfqpoint{1.164721in}{3.628825in}}%
\pgfpathlineto{\pgfqpoint{1.169834in}{3.649765in}}%
\pgfpathlineto{\pgfqpoint{1.171061in}{3.649990in}}%
\pgfpathlineto{\pgfqpoint{1.171265in}{3.649771in}}%
\pgfpathlineto{\pgfqpoint{1.172697in}{3.646198in}}%
\pgfpathlineto{\pgfqpoint{1.175355in}{3.631609in}}%
\pgfpathlineto{\pgfqpoint{1.182104in}{3.592602in}}%
\pgfpathlineto{\pgfqpoint{1.183740in}{3.591647in}}%
\pgfpathlineto{\pgfqpoint{1.185172in}{3.594127in}}%
\pgfpathlineto{\pgfqpoint{1.188035in}{3.605038in}}%
\pgfpathlineto{\pgfqpoint{1.192125in}{3.618310in}}%
\pgfpathlineto{\pgfqpoint{1.193556in}{3.618382in}}%
\pgfpathlineto{\pgfqpoint{1.194988in}{3.615344in}}%
\pgfpathlineto{\pgfqpoint{1.197442in}{3.603831in}}%
\pgfpathlineto{\pgfqpoint{1.204804in}{3.566119in}}%
\pgfpathlineto{\pgfqpoint{1.206440in}{3.565635in}}%
\pgfpathlineto{\pgfqpoint{1.208076in}{3.568555in}}%
\pgfpathlineto{\pgfqpoint{1.215438in}{3.587911in}}%
\pgfpathlineto{\pgfqpoint{1.216052in}{3.587463in}}%
\pgfpathlineto{\pgfqpoint{1.217688in}{3.583801in}}%
\pgfpathlineto{\pgfqpoint{1.220551in}{3.570302in}}%
\pgfpathlineto{\pgfqpoint{1.226890in}{3.540090in}}%
\pgfpathlineto{\pgfqpoint{1.228731in}{3.538782in}}%
\pgfpathlineto{\pgfqpoint{1.230367in}{3.540928in}}%
\pgfpathlineto{\pgfqpoint{1.233843in}{3.551049in}}%
\pgfpathlineto{\pgfqpoint{1.237115in}{3.557544in}}%
\pgfpathlineto{\pgfqpoint{1.238547in}{3.557016in}}%
\pgfpathlineto{\pgfqpoint{1.240387in}{3.552731in}}%
\pgfpathlineto{\pgfqpoint{1.243659in}{3.537436in}}%
\pgfpathlineto{\pgfqpoint{1.248977in}{3.513771in}}%
\pgfpathlineto{\pgfqpoint{1.251022in}{3.511632in}}%
\pgfpathlineto{\pgfqpoint{1.252453in}{3.512827in}}%
\pgfpathlineto{\pgfqpoint{1.255112in}{3.518945in}}%
\pgfpathlineto{\pgfqpoint{1.259406in}{3.527577in}}%
\pgfpathlineto{\pgfqpoint{1.260838in}{3.527149in}}%
\pgfpathlineto{\pgfqpoint{1.262678in}{3.523322in}}%
\pgfpathlineto{\pgfqpoint{1.265746in}{3.510357in}}%
\pgfpathlineto{\pgfqpoint{1.271676in}{3.485903in}}%
\pgfpathlineto{\pgfqpoint{1.273721in}{3.484286in}}%
\pgfpathlineto{\pgfqpoint{1.275357in}{3.485860in}}%
\pgfpathlineto{\pgfqpoint{1.279243in}{3.494456in}}%
\pgfpathlineto{\pgfqpoint{1.282106in}{3.497967in}}%
\pgfpathlineto{\pgfqpoint{1.283538in}{3.497045in}}%
\pgfpathlineto{\pgfqpoint{1.285583in}{3.492212in}}%
\pgfpathlineto{\pgfqpoint{1.289468in}{3.475266in}}%
\pgfpathlineto{\pgfqpoint{1.293967in}{3.458638in}}%
\pgfpathlineto{\pgfqpoint{1.296217in}{3.456681in}}%
\pgfpathlineto{\pgfqpoint{1.297853in}{3.458041in}}%
\pgfpathlineto{\pgfqpoint{1.301943in}{3.465973in}}%
\pgfpathlineto{\pgfqpoint{1.304601in}{3.468446in}}%
\pgfpathlineto{\pgfqpoint{1.306237in}{3.467066in}}%
\pgfpathlineto{\pgfqpoint{1.308487in}{3.461235in}}%
\pgfpathlineto{\pgfqpoint{1.318712in}{3.428912in}}%
\pgfpathlineto{\pgfqpoint{1.320553in}{3.430356in}}%
\pgfpathlineto{\pgfqpoint{1.327710in}{3.438753in}}%
\pgfpathlineto{\pgfqpoint{1.329551in}{3.435913in}}%
\pgfpathlineto{\pgfqpoint{1.332414in}{3.426426in}}%
\pgfpathlineto{\pgfqpoint{1.339367in}{3.402218in}}%
\pgfpathlineto{\pgfqpoint{1.341412in}{3.401044in}}%
\pgfpathlineto{\pgfqpoint{1.343457in}{3.402802in}}%
\pgfpathlineto{\pgfqpoint{1.349592in}{3.409756in}}%
\pgfpathlineto{\pgfqpoint{1.351228in}{3.408123in}}%
\pgfpathlineto{\pgfqpoint{1.353682in}{3.401796in}}%
\pgfpathlineto{\pgfqpoint{1.363089in}{3.373123in}}%
\pgfpathlineto{\pgfqpoint{1.364930in}{3.373537in}}%
\pgfpathlineto{\pgfqpoint{1.367997in}{3.377512in}}%
\pgfpathlineto{\pgfqpoint{1.371474in}{3.380678in}}%
\pgfpathlineto{\pgfqpoint{1.373314in}{3.379431in}}%
\pgfpathlineto{\pgfqpoint{1.375564in}{3.374490in}}%
\pgfpathlineto{\pgfqpoint{1.380472in}{3.356282in}}%
\pgfpathlineto{\pgfqpoint{1.384358in}{3.346100in}}%
\pgfpathlineto{\pgfqpoint{1.386607in}{3.344976in}}%
\pgfpathlineto{\pgfqpoint{1.388857in}{3.346790in}}%
\pgfpathlineto{\pgfqpoint{1.393969in}{3.351591in}}%
\pgfpathlineto{\pgfqpoint{1.395810in}{3.350204in}}%
\pgfpathlineto{\pgfqpoint{1.398264in}{3.344683in}}%
\pgfpathlineto{\pgfqpoint{1.408489in}{3.316787in}}%
\pgfpathlineto{\pgfqpoint{1.410534in}{3.317612in}}%
\pgfpathlineto{\pgfqpoint{1.417078in}{3.322300in}}%
\pgfpathlineto{\pgfqpoint{1.419123in}{3.319631in}}%
\pgfpathlineto{\pgfqpoint{1.422191in}{3.310898in}}%
\pgfpathlineto{\pgfqpoint{1.429348in}{3.289705in}}%
\pgfpathlineto{\pgfqpoint{1.431598in}{3.288577in}}%
\pgfpathlineto{\pgfqpoint{1.434052in}{3.290233in}}%
\pgfpathlineto{\pgfqpoint{1.438755in}{3.293604in}}%
\pgfpathlineto{\pgfqpoint{1.440596in}{3.292218in}}%
\pgfpathlineto{\pgfqpoint{1.443050in}{3.287049in}}%
\pgfpathlineto{\pgfqpoint{1.453480in}{3.260320in}}%
\pgfpathlineto{\pgfqpoint{1.455729in}{3.261086in}}%
\pgfpathlineto{\pgfqpoint{1.461660in}{3.264500in}}%
\pgfpathlineto{\pgfqpoint{1.463705in}{3.262197in}}%
\pgfpathlineto{\pgfqpoint{1.466568in}{3.254960in}}%
\pgfpathlineto{\pgfqpoint{1.474544in}{3.232892in}}%
\pgfpathlineto{\pgfqpoint{1.476793in}{3.232017in}}%
\pgfpathlineto{\pgfqpoint{1.479452in}{3.233670in}}%
\pgfpathlineto{\pgfqpoint{1.483542in}{3.235795in}}%
\pgfpathlineto{\pgfqpoint{1.485587in}{3.234138in}}%
\pgfpathlineto{\pgfqpoint{1.488245in}{3.228416in}}%
\pgfpathlineto{\pgfqpoint{1.497857in}{3.203917in}}%
\pgfpathlineto{\pgfqpoint{1.500106in}{3.203920in}}%
\pgfpathlineto{\pgfqpoint{1.507060in}{3.206324in}}%
\pgfpathlineto{\pgfqpoint{1.509309in}{3.202961in}}%
\pgfpathlineto{\pgfqpoint{1.512990in}{3.192483in}}%
\pgfpathlineto{\pgfqpoint{1.518716in}{3.177162in}}%
\pgfpathlineto{\pgfqpoint{1.521375in}{3.175261in}}%
\pgfpathlineto{\pgfqpoint{1.524033in}{3.176329in}}%
\pgfpathlineto{\pgfqpoint{1.528328in}{3.178118in}}%
\pgfpathlineto{\pgfqpoint{1.530373in}{3.176479in}}%
\pgfpathlineto{\pgfqpoint{1.533031in}{3.171039in}}%
\pgfpathlineto{\pgfqpoint{1.543052in}{3.147062in}}%
\pgfpathlineto{\pgfqpoint{1.545506in}{3.147240in}}%
\pgfpathlineto{\pgfqpoint{1.551437in}{3.148990in}}%
\pgfpathlineto{\pgfqpoint{1.553686in}{3.146240in}}%
\pgfpathlineto{\pgfqpoint{1.556958in}{3.138012in}}%
\pgfpathlineto{\pgfqpoint{1.563911in}{3.120017in}}%
\pgfpathlineto{\pgfqpoint{1.566570in}{3.118432in}}%
\pgfpathlineto{\pgfqpoint{1.569638in}{3.119638in}}%
\pgfpathlineto{\pgfqpoint{1.573114in}{3.120540in}}%
\pgfpathlineto{\pgfqpoint{1.575364in}{3.118641in}}%
\pgfpathlineto{\pgfqpoint{1.578227in}{3.112648in}}%
\pgfpathlineto{\pgfqpoint{1.587429in}{3.090563in}}%
\pgfpathlineto{\pgfqpoint{1.589883in}{3.090083in}}%
\pgfpathlineto{\pgfqpoint{1.596836in}{3.090962in}}%
\pgfpathlineto{\pgfqpoint{1.599291in}{3.087196in}}%
\pgfpathlineto{\pgfqpoint{1.603585in}{3.075358in}}%
\pgfpathlineto{\pgfqpoint{1.608698in}{3.063332in}}%
\pgfpathlineto{\pgfqpoint{1.611561in}{3.061525in}}%
\pgfpathlineto{\pgfqpoint{1.614833in}{3.062578in}}%
\pgfpathlineto{\pgfqpoint{1.618105in}{3.062964in}}%
\pgfpathlineto{\pgfqpoint{1.620354in}{3.060879in}}%
\pgfpathlineto{\pgfqpoint{1.623422in}{3.054335in}}%
\pgfpathlineto{\pgfqpoint{1.632011in}{3.033952in}}%
\pgfpathlineto{\pgfqpoint{1.634670in}{3.033075in}}%
\pgfpathlineto{\pgfqpoint{1.641827in}{3.033296in}}%
\pgfpathlineto{\pgfqpoint{1.644486in}{3.028987in}}%
\pgfpathlineto{\pgfqpoint{1.649803in}{3.014348in}}%
\pgfpathlineto{\pgfqpoint{1.654097in}{3.005848in}}%
\pgfpathlineto{\pgfqpoint{1.656756in}{3.004550in}}%
\pgfpathlineto{\pgfqpoint{1.664936in}{3.003756in}}%
\pgfpathlineto{\pgfqpoint{1.668004in}{2.997692in}}%
\pgfpathlineto{\pgfqpoint{1.677206in}{2.976783in}}%
\pgfpathlineto{\pgfqpoint{1.679865in}{2.976084in}}%
\pgfpathlineto{\pgfqpoint{1.686409in}{2.976079in}}%
\pgfpathlineto{\pgfqpoint{1.689067in}{2.972144in}}%
\pgfpathlineto{\pgfqpoint{1.693566in}{2.960336in}}%
\pgfpathlineto{\pgfqpoint{1.698475in}{2.949505in}}%
\pgfpathlineto{\pgfqpoint{1.701338in}{2.947561in}}%
\pgfpathlineto{\pgfqpoint{1.705019in}{2.948227in}}%
\pgfpathlineto{\pgfqpoint{1.708086in}{2.947917in}}%
\pgfpathlineto{\pgfqpoint{1.710540in}{2.945197in}}%
\pgfpathlineto{\pgfqpoint{1.714017in}{2.937324in}}%
\pgfpathlineto{\pgfqpoint{1.721174in}{2.920705in}}%
\pgfpathlineto{\pgfqpoint{1.724037in}{2.919014in}}%
\pgfpathlineto{\pgfqpoint{1.732218in}{2.917596in}}%
\pgfpathlineto{\pgfqpoint{1.735285in}{2.911699in}}%
\pgfpathlineto{\pgfqpoint{1.744692in}{2.891200in}}%
\pgfpathlineto{\pgfqpoint{1.747555in}{2.890499in}}%
\pgfpathlineto{\pgfqpoint{1.753281in}{2.890292in}}%
\pgfpathlineto{\pgfqpoint{1.755940in}{2.886858in}}%
\pgfpathlineto{\pgfqpoint{1.760030in}{2.876916in}}%
\pgfpathlineto{\pgfqpoint{1.765756in}{2.864075in}}%
\pgfpathlineto{\pgfqpoint{1.768824in}{2.861967in}}%
\pgfpathlineto{\pgfqpoint{1.777413in}{2.859740in}}%
\pgfpathlineto{\pgfqpoint{1.780685in}{2.853054in}}%
\pgfpathlineto{\pgfqpoint{1.789069in}{2.834596in}}%
\pgfpathlineto{\pgfqpoint{1.791933in}{2.833365in}}%
\pgfpathlineto{\pgfqpoint{1.798681in}{2.832455in}}%
\pgfpathlineto{\pgfqpoint{1.801544in}{2.828049in}}%
\pgfpathlineto{\pgfqpoint{1.807475in}{2.812978in}}%
\pgfpathlineto{\pgfqpoint{1.811565in}{2.806024in}}%
\pgfpathlineto{\pgfqpoint{1.814632in}{2.804806in}}%
\pgfpathlineto{\pgfqpoint{1.820972in}{2.803911in}}%
\pgfpathlineto{\pgfqpoint{1.823835in}{2.799674in}}%
\pgfpathlineto{\pgfqpoint{1.829152in}{2.786291in}}%
\pgfpathlineto{\pgfqpoint{1.833651in}{2.777845in}}%
\pgfpathlineto{\pgfqpoint{1.836719in}{2.776257in}}%
\pgfpathlineto{\pgfqpoint{1.843876in}{2.774762in}}%
\pgfpathlineto{\pgfqpoint{1.846944in}{2.769584in}}%
\pgfpathlineto{\pgfqpoint{1.857578in}{2.748152in}}%
\pgfpathlineto{\pgfqpoint{1.860850in}{2.747751in}}%
\pgfpathlineto{\pgfqpoint{1.865145in}{2.747131in}}%
\pgfpathlineto{\pgfqpoint{1.867803in}{2.744011in}}%
\pgfpathlineto{\pgfqpoint{1.871689in}{2.735252in}}%
\pgfpathlineto{\pgfqpoint{1.878233in}{2.721100in}}%
\pgfpathlineto{\pgfqpoint{1.881300in}{2.719170in}}%
\pgfpathlineto{\pgfqpoint{1.889072in}{2.717051in}}%
\pgfpathlineto{\pgfqpoint{1.892344in}{2.711086in}}%
\pgfpathlineto{\pgfqpoint{1.901751in}{2.691510in}}%
\pgfpathlineto{\pgfqpoint{1.904818in}{2.690520in}}%
\pgfpathlineto{\pgfqpoint{1.910340in}{2.689582in}}%
\pgfpathlineto{\pgfqpoint{1.913203in}{2.685762in}}%
\pgfpathlineto{\pgfqpoint{1.917906in}{2.674504in}}%
\pgfpathlineto{\pgfqpoint{1.923019in}{2.664110in}}%
\pgfpathlineto{\pgfqpoint{1.926087in}{2.662055in}}%
\pgfpathlineto{\pgfqpoint{1.934062in}{2.659593in}}%
\pgfpathlineto{\pgfqpoint{1.937334in}{2.653537in}}%
\pgfpathlineto{\pgfqpoint{1.946537in}{2.634469in}}%
\pgfpathlineto{\pgfqpoint{1.949604in}{2.633357in}}%
\pgfpathlineto{\pgfqpoint{1.955331in}{2.632193in}}%
\pgfpathlineto{\pgfqpoint{1.958194in}{2.628251in}}%
\pgfpathlineto{\pgfqpoint{1.963102in}{2.616485in}}%
\pgfpathlineto{\pgfqpoint{1.968010in}{2.606856in}}%
\pgfpathlineto{\pgfqpoint{1.971282in}{2.604846in}}%
\pgfpathlineto{\pgfqpoint{1.978644in}{2.602663in}}%
\pgfpathlineto{\pgfqpoint{1.981711in}{2.597399in}}%
\pgfpathlineto{\pgfqpoint{1.992141in}{2.576832in}}%
\pgfpathlineto{\pgfqpoint{1.995618in}{2.576187in}}%
\pgfpathlineto{\pgfqpoint{1.999708in}{2.575308in}}%
\pgfpathlineto{\pgfqpoint{2.002571in}{2.571870in}}%
\pgfpathlineto{\pgfqpoint{2.006865in}{2.562133in}}%
\pgfpathlineto{\pgfqpoint{2.012796in}{2.549821in}}%
\pgfpathlineto{\pgfqpoint{2.016068in}{2.547699in}}%
\pgfpathlineto{\pgfqpoint{2.023635in}{2.545256in}}%
\pgfpathlineto{\pgfqpoint{2.026907in}{2.539427in}}%
\pgfpathlineto{\pgfqpoint{2.036518in}{2.520007in}}%
\pgfpathlineto{\pgfqpoint{2.039790in}{2.518990in}}%
\pgfpathlineto{\pgfqpoint{2.044698in}{2.517974in}}%
\pgfpathlineto{\pgfqpoint{2.047562in}{2.514410in}}%
\pgfpathlineto{\pgfqpoint{2.051856in}{2.504619in}}%
\pgfpathlineto{\pgfqpoint{2.057582in}{2.492769in}}%
\pgfpathlineto{\pgfqpoint{2.060854in}{2.490545in}}%
\pgfpathlineto{\pgfqpoint{2.068625in}{2.487862in}}%
\pgfpathlineto{\pgfqpoint{2.071897in}{2.481923in}}%
\pgfpathlineto{\pgfqpoint{2.081305in}{2.462906in}}%
\pgfpathlineto{\pgfqpoint{2.084577in}{2.461803in}}%
\pgfpathlineto{\pgfqpoint{2.089689in}{2.460651in}}%
\pgfpathlineto{\pgfqpoint{2.092552in}{2.456965in}}%
\pgfpathlineto{\pgfqpoint{2.097051in}{2.446585in}}%
\pgfpathlineto{\pgfqpoint{2.102573in}{2.435453in}}%
\pgfpathlineto{\pgfqpoint{2.105845in}{2.433337in}}%
\pgfpathlineto{\pgfqpoint{2.113207in}{2.430987in}}%
\pgfpathlineto{\pgfqpoint{2.116479in}{2.425354in}}%
\pgfpathlineto{\pgfqpoint{2.126500in}{2.405490in}}%
\pgfpathlineto{\pgfqpoint{2.129772in}{2.404601in}}%
\pgfpathlineto{\pgfqpoint{2.134475in}{2.403502in}}%
\pgfpathlineto{\pgfqpoint{2.137338in}{2.399897in}}%
\pgfpathlineto{\pgfqpoint{2.141837in}{2.389596in}}%
\pgfpathlineto{\pgfqpoint{2.147359in}{2.378363in}}%
\pgfpathlineto{\pgfqpoint{2.150631in}{2.376162in}}%
\pgfpathlineto{\pgfqpoint{2.158198in}{2.373637in}}%
\pgfpathlineto{\pgfqpoint{2.161470in}{2.367878in}}%
\pgfpathlineto{\pgfqpoint{2.171286in}{2.348349in}}%
\pgfpathlineto{\pgfqpoint{2.174558in}{2.347406in}}%
\pgfpathlineto{\pgfqpoint{2.179262in}{2.346366in}}%
\pgfpathlineto{\pgfqpoint{2.182125in}{2.342844in}}%
\pgfpathlineto{\pgfqpoint{2.186419in}{2.333127in}}%
\pgfpathlineto{\pgfqpoint{2.192350in}{2.321013in}}%
\pgfpathlineto{\pgfqpoint{2.195622in}{2.318934in}}%
\pgfpathlineto{\pgfqpoint{2.202984in}{2.316547in}}%
\pgfpathlineto{\pgfqpoint{2.206256in}{2.310870in}}%
\pgfpathlineto{\pgfqpoint{2.216072in}{2.291195in}}%
\pgfpathlineto{\pgfqpoint{2.219344in}{2.290206in}}%
\pgfpathlineto{\pgfqpoint{2.224252in}{2.289094in}}%
\pgfpathlineto{\pgfqpoint{2.227115in}{2.285450in}}%
\pgfpathlineto{\pgfqpoint{2.231614in}{2.275092in}}%
\pgfpathlineto{\pgfqpoint{2.237136in}{2.263887in}}%
\pgfpathlineto{\pgfqpoint{2.240408in}{2.261739in}}%
\pgfpathlineto{\pgfqpoint{2.247975in}{2.259231in}}%
\pgfpathlineto{\pgfqpoint{2.251247in}{2.253416in}}%
\pgfpathlineto{\pgfqpoint{2.260858in}{2.234026in}}%
\pgfpathlineto{\pgfqpoint{2.264130in}{2.233003in}}%
\pgfpathlineto{\pgfqpoint{2.269038in}{2.231993in}}%
\pgfpathlineto{\pgfqpoint{2.271902in}{2.228432in}}%
\pgfpathlineto{\pgfqpoint{2.276196in}{2.218625in}}%
\pgfpathlineto{\pgfqpoint{2.281922in}{2.206744in}}%
\pgfpathlineto{\pgfqpoint{2.285194in}{2.204534in}}%
\pgfpathlineto{\pgfqpoint{2.292965in}{2.201928in}}%
\pgfpathlineto{\pgfqpoint{2.296237in}{2.195969in}}%
\pgfpathlineto{\pgfqpoint{2.305645in}{2.176841in}}%
\pgfpathlineto{\pgfqpoint{2.308917in}{2.175794in}}%
\pgfpathlineto{\pgfqpoint{2.314029in}{2.174754in}}%
\pgfpathlineto{\pgfqpoint{2.316892in}{2.171070in}}%
\pgfpathlineto{\pgfqpoint{2.321391in}{2.160580in}}%
\pgfpathlineto{\pgfqpoint{2.326913in}{2.149337in}}%
\pgfpathlineto{\pgfqpoint{2.330185in}{2.147276in}}%
\pgfpathlineto{\pgfqpoint{2.337547in}{2.145144in}}%
\pgfpathlineto{\pgfqpoint{2.340615in}{2.139920in}}%
\pgfpathlineto{\pgfqpoint{2.351044in}{2.119221in}}%
\pgfpathlineto{\pgfqpoint{2.354316in}{2.118593in}}%
\pgfpathlineto{\pgfqpoint{2.358611in}{2.117833in}}%
\pgfpathlineto{\pgfqpoint{2.361269in}{2.114787in}}%
\pgfpathlineto{\pgfqpoint{2.365155in}{2.106245in}}%
\pgfpathlineto{\pgfqpoint{2.371699in}{2.092153in}}%
\pgfpathlineto{\pgfqpoint{2.374767in}{2.090086in}}%
\pgfpathlineto{\pgfqpoint{2.382947in}{2.087366in}}%
\pgfpathlineto{\pgfqpoint{2.386219in}{2.081089in}}%
\pgfpathlineto{\pgfqpoint{2.395012in}{2.062590in}}%
\pgfpathlineto{\pgfqpoint{2.398080in}{2.061355in}}%
\pgfpathlineto{\pgfqpoint{2.404215in}{2.060144in}}%
\pgfpathlineto{\pgfqpoint{2.407078in}{2.055985in}}%
\pgfpathlineto{\pgfqpoint{2.412395in}{2.042894in}}%
\pgfpathlineto{\pgfqpoint{2.416894in}{2.034476in}}%
\pgfpathlineto{\pgfqpoint{2.419962in}{2.032772in}}%
\pgfpathlineto{\pgfqpoint{2.427119in}{2.031131in}}%
\pgfpathlineto{\pgfqpoint{2.430187in}{2.026029in}}%
\pgfpathlineto{\pgfqpoint{2.441026in}{2.004518in}}%
\pgfpathlineto{\pgfqpoint{2.444298in}{2.004204in}}%
\pgfpathlineto{\pgfqpoint{2.448388in}{2.003609in}}%
\pgfpathlineto{\pgfqpoint{2.451046in}{2.000539in}}%
\pgfpathlineto{\pgfqpoint{2.454932in}{1.991786in}}%
\pgfpathlineto{\pgfqpoint{2.461476in}{1.977473in}}%
\pgfpathlineto{\pgfqpoint{2.464544in}{1.975538in}}%
\pgfpathlineto{\pgfqpoint{2.472519in}{1.973435in}}%
\pgfpathlineto{\pgfqpoint{2.475791in}{1.967229in}}%
\pgfpathlineto{\pgfqpoint{2.484994in}{1.947745in}}%
\pgfpathlineto{\pgfqpoint{2.488061in}{1.946867in}}%
\pgfpathlineto{\pgfqpoint{2.493583in}{1.946317in}}%
\pgfpathlineto{\pgfqpoint{2.496242in}{1.942916in}}%
\pgfpathlineto{\pgfqpoint{2.500332in}{1.933180in}}%
\pgfpathlineto{\pgfqpoint{2.506262in}{1.920215in}}%
\pgfpathlineto{\pgfqpoint{2.509330in}{1.918255in}}%
\pgfpathlineto{\pgfqpoint{2.517510in}{1.916240in}}%
\pgfpathlineto{\pgfqpoint{2.520782in}{1.909828in}}%
\pgfpathlineto{\pgfqpoint{2.529576in}{1.890614in}}%
\pgfpathlineto{\pgfqpoint{2.532439in}{1.889573in}}%
\pgfpathlineto{\pgfqpoint{2.538983in}{1.888839in}}%
\pgfpathlineto{\pgfqpoint{2.541846in}{1.884446in}}%
\pgfpathlineto{\pgfqpoint{2.547367in}{1.870139in}}%
\pgfpathlineto{\pgfqpoint{2.551662in}{1.862240in}}%
\pgfpathlineto{\pgfqpoint{2.554525in}{1.860920in}}%
\pgfpathlineto{\pgfqpoint{2.561683in}{1.860076in}}%
\pgfpathlineto{\pgfqpoint{2.564546in}{1.855361in}}%
\pgfpathlineto{\pgfqpoint{2.575998in}{1.832377in}}%
\pgfpathlineto{\pgfqpoint{2.579270in}{1.832692in}}%
\pgfpathlineto{\pgfqpoint{2.582951in}{1.832529in}}%
\pgfpathlineto{\pgfqpoint{2.585405in}{1.829879in}}%
\pgfpathlineto{\pgfqpoint{2.588881in}{1.822104in}}%
\pgfpathlineto{\pgfqpoint{2.596244in}{1.805118in}}%
\pgfpathlineto{\pgfqpoint{2.599107in}{1.803598in}}%
\pgfpathlineto{\pgfqpoint{2.607082in}{1.802495in}}%
\pgfpathlineto{\pgfqpoint{2.610150in}{1.796611in}}%
\pgfpathlineto{\pgfqpoint{2.619761in}{1.775453in}}%
\pgfpathlineto{\pgfqpoint{2.622420in}{1.775008in}}%
\pgfpathlineto{\pgfqpoint{2.628351in}{1.775217in}}%
\pgfpathlineto{\pgfqpoint{2.630805in}{1.772011in}}%
\pgfpathlineto{\pgfqpoint{2.634690in}{1.762324in}}%
\pgfpathlineto{\pgfqpoint{2.640621in}{1.748244in}}%
\pgfpathlineto{\pgfqpoint{2.643484in}{1.746258in}}%
\pgfpathlineto{\pgfqpoint{2.646756in}{1.746994in}}%
\pgfpathlineto{\pgfqpoint{2.650232in}{1.747104in}}%
\pgfpathlineto{\pgfqpoint{2.652686in}{1.744513in}}%
\pgfpathlineto{\pgfqpoint{2.655959in}{1.737088in}}%
\pgfpathlineto{\pgfqpoint{2.663525in}{1.718920in}}%
\pgfpathlineto{\pgfqpoint{2.666184in}{1.717534in}}%
\pgfpathlineto{\pgfqpoint{2.669865in}{1.718671in}}%
\pgfpathlineto{\pgfqpoint{2.672728in}{1.718618in}}%
\pgfpathlineto{\pgfqpoint{2.675182in}{1.715972in}}%
\pgfpathlineto{\pgfqpoint{2.678454in}{1.708384in}}%
\pgfpathlineto{\pgfqpoint{2.686020in}{1.690086in}}%
\pgfpathlineto{\pgfqpoint{2.688679in}{1.688825in}}%
\pgfpathlineto{\pgfqpoint{2.692565in}{1.690225in}}%
\pgfpathlineto{\pgfqpoint{2.695428in}{1.690030in}}%
\pgfpathlineto{\pgfqpoint{2.697882in}{1.687085in}}%
\pgfpathlineto{\pgfqpoint{2.701358in}{1.678483in}}%
\pgfpathlineto{\pgfqpoint{2.708107in}{1.661702in}}%
\pgfpathlineto{\pgfqpoint{2.710765in}{1.660096in}}%
\pgfpathlineto{\pgfqpoint{2.713833in}{1.661148in}}%
\pgfpathlineto{\pgfqpoint{2.717309in}{1.661869in}}%
\pgfpathlineto{\pgfqpoint{2.719559in}{1.659900in}}%
\pgfpathlineto{\pgfqpoint{2.722422in}{1.653874in}}%
\pgfpathlineto{\pgfqpoint{2.731625in}{1.631861in}}%
\pgfpathlineto{\pgfqpoint{2.734079in}{1.631489in}}%
\pgfpathlineto{\pgfqpoint{2.741032in}{1.632656in}}%
\pgfpathlineto{\pgfqpoint{2.743486in}{1.628798in}}%
\pgfpathlineto{\pgfqpoint{2.747780in}{1.616523in}}%
\pgfpathlineto{\pgfqpoint{2.752689in}{1.604451in}}%
\pgfpathlineto{\pgfqpoint{2.755347in}{1.602592in}}%
\pgfpathlineto{\pgfqpoint{2.758006in}{1.603536in}}%
\pgfpathlineto{\pgfqpoint{2.762300in}{1.605017in}}%
\pgfpathlineto{\pgfqpoint{2.764550in}{1.602968in}}%
\pgfpathlineto{\pgfqpoint{2.767413in}{1.596613in}}%
\pgfpathlineto{\pgfqpoint{2.776206in}{1.574423in}}%
\pgfpathlineto{\pgfqpoint{2.778660in}{1.573939in}}%
\pgfpathlineto{\pgfqpoint{2.786227in}{1.575616in}}%
\pgfpathlineto{\pgfqpoint{2.788681in}{1.571270in}}%
\pgfpathlineto{\pgfqpoint{2.793385in}{1.556861in}}%
\pgfpathlineto{\pgfqpoint{2.797679in}{1.546567in}}%
\pgfpathlineto{\pgfqpoint{2.800133in}{1.544991in}}%
\pgfpathlineto{\pgfqpoint{2.802587in}{1.546054in}}%
\pgfpathlineto{\pgfqpoint{2.807291in}{1.548260in}}%
\pgfpathlineto{\pgfqpoint{2.809336in}{1.546436in}}%
\pgfpathlineto{\pgfqpoint{2.812199in}{1.539984in}}%
\pgfpathlineto{\pgfqpoint{2.821197in}{1.516578in}}%
\pgfpathlineto{\pgfqpoint{2.823447in}{1.516364in}}%
\pgfpathlineto{\pgfqpoint{2.827332in}{1.519317in}}%
\pgfpathlineto{\pgfqpoint{2.829991in}{1.519850in}}%
\pgfpathlineto{\pgfqpoint{2.832036in}{1.517737in}}%
\pgfpathlineto{\pgfqpoint{2.834899in}{1.510773in}}%
\pgfpathlineto{\pgfqpoint{2.843283in}{1.487921in}}%
\pgfpathlineto{\pgfqpoint{2.845533in}{1.487412in}}%
\pgfpathlineto{\pgfqpoint{2.848396in}{1.489626in}}%
\pgfpathlineto{\pgfqpoint{2.852077in}{1.491680in}}%
\pgfpathlineto{\pgfqpoint{2.854122in}{1.490030in}}%
\pgfpathlineto{\pgfqpoint{2.856781in}{1.484066in}}%
\pgfpathlineto{\pgfqpoint{2.866392in}{1.458521in}}%
\pgfpathlineto{\pgfqpoint{2.868437in}{1.458804in}}%
\pgfpathlineto{\pgfqpoint{2.876004in}{1.462531in}}%
\pgfpathlineto{\pgfqpoint{2.878253in}{1.458422in}}%
\pgfpathlineto{\pgfqpoint{2.882344in}{1.444788in}}%
\pgfpathlineto{\pgfqpoint{2.887252in}{1.430960in}}%
\pgfpathlineto{\pgfqpoint{2.889706in}{1.429430in}}%
\pgfpathlineto{\pgfqpoint{2.891955in}{1.430950in}}%
\pgfpathlineto{\pgfqpoint{2.897272in}{1.435168in}}%
\pgfpathlineto{\pgfqpoint{2.899113in}{1.433493in}}%
\pgfpathlineto{\pgfqpoint{2.901567in}{1.427698in}}%
\pgfpathlineto{\pgfqpoint{2.911383in}{1.400447in}}%
\pgfpathlineto{\pgfqpoint{2.913428in}{1.401186in}}%
\pgfpathlineto{\pgfqpoint{2.920586in}{1.406563in}}%
\pgfpathlineto{\pgfqpoint{2.922631in}{1.403273in}}%
\pgfpathlineto{\pgfqpoint{2.925903in}{1.392618in}}%
\pgfpathlineto{\pgfqpoint{2.932038in}{1.372866in}}%
\pgfpathlineto{\pgfqpoint{2.934287in}{1.371335in}}%
\pgfpathlineto{\pgfqpoint{2.936332in}{1.372810in}}%
\pgfpathlineto{\pgfqpoint{2.942467in}{1.378877in}}%
\pgfpathlineto{\pgfqpoint{2.944308in}{1.376817in}}%
\pgfpathlineto{\pgfqpoint{2.946967in}{1.369424in}}%
\pgfpathlineto{\pgfqpoint{2.955351in}{1.342603in}}%
\pgfpathlineto{\pgfqpoint{2.957192in}{1.342383in}}%
\pgfpathlineto{\pgfqpoint{2.959441in}{1.344933in}}%
\pgfpathlineto{\pgfqpoint{2.964554in}{1.350987in}}%
\pgfpathlineto{\pgfqpoint{2.966190in}{1.349782in}}%
\pgfpathlineto{\pgfqpoint{2.968439in}{1.344546in}}%
\pgfpathlineto{\pgfqpoint{2.973143in}{1.325574in}}%
\pgfpathlineto{\pgfqpoint{2.977029in}{1.314149in}}%
\pgfpathlineto{\pgfqpoint{2.979074in}{1.312962in}}%
\pgfpathlineto{\pgfqpoint{2.980914in}{1.314601in}}%
\pgfpathlineto{\pgfqpoint{2.987663in}{1.322912in}}%
\pgfpathlineto{\pgfqpoint{2.989299in}{1.320789in}}%
\pgfpathlineto{\pgfqpoint{2.991957in}{1.312575in}}%
\pgfpathlineto{\pgfqpoint{3.000137in}{1.283934in}}%
\pgfpathlineto{\pgfqpoint{3.001978in}{1.283903in}}%
\pgfpathlineto{\pgfqpoint{3.004227in}{1.287177in}}%
\pgfpathlineto{\pgfqpoint{3.009340in}{1.295319in}}%
\pgfpathlineto{\pgfqpoint{3.010976in}{1.294378in}}%
\pgfpathlineto{\pgfqpoint{3.013021in}{1.289617in}}%
\pgfpathlineto{\pgfqpoint{3.016702in}{1.273779in}}%
\pgfpathlineto{\pgfqpoint{3.021610in}{1.255597in}}%
\pgfpathlineto{\pgfqpoint{3.023655in}{1.254066in}}%
\pgfpathlineto{\pgfqpoint{3.025291in}{1.255628in}}%
\pgfpathlineto{\pgfqpoint{3.028972in}{1.263764in}}%
\pgfpathlineto{\pgfqpoint{3.031835in}{1.267714in}}%
\pgfpathlineto{\pgfqpoint{3.033267in}{1.267041in}}%
\pgfpathlineto{\pgfqpoint{3.035107in}{1.263040in}}%
\pgfpathlineto{\pgfqpoint{3.038379in}{1.248963in}}%
\pgfpathlineto{\pgfqpoint{3.043901in}{1.226103in}}%
\pgfpathlineto{\pgfqpoint{3.045946in}{1.224351in}}%
\pgfpathlineto{\pgfqpoint{3.047582in}{1.226039in}}%
\pgfpathlineto{\pgfqpoint{3.050854in}{1.234175in}}%
\pgfpathlineto{\pgfqpoint{3.054331in}{1.240323in}}%
\pgfpathlineto{\pgfqpoint{3.055762in}{1.239662in}}%
\pgfpathlineto{\pgfqpoint{3.057603in}{1.235376in}}%
\pgfpathlineto{\pgfqpoint{3.060875in}{1.220197in}}%
\pgfpathlineto{\pgfqpoint{3.066396in}{1.195970in}}%
\pgfpathlineto{\pgfqpoint{3.068441in}{1.194456in}}%
\pgfpathlineto{\pgfqpoint{3.070077in}{1.196632in}}%
\pgfpathlineto{\pgfqpoint{3.073554in}{1.206744in}}%
\pgfpathlineto{\pgfqpoint{3.076826in}{1.213218in}}%
\pgfpathlineto{\pgfqpoint{3.078258in}{1.212573in}}%
\pgfpathlineto{\pgfqpoint{3.080098in}{1.207938in}}%
\pgfpathlineto{\pgfqpoint{3.083370in}{1.191407in}}%
\pgfpathlineto{\pgfqpoint{3.088687in}{1.165927in}}%
\pgfpathlineto{\pgfqpoint{3.090732in}{1.164146in}}%
\pgfpathlineto{\pgfqpoint{3.092164in}{1.166091in}}%
\pgfpathlineto{\pgfqpoint{3.095027in}{1.175188in}}%
\pgfpathlineto{\pgfqpoint{3.099117in}{1.186365in}}%
\pgfpathlineto{\pgfqpoint{3.100548in}{1.186149in}}%
\pgfpathlineto{\pgfqpoint{3.102184in}{1.182337in}}%
\pgfpathlineto{\pgfqpoint{3.104843in}{1.169109in}}%
\pgfpathlineto{\pgfqpoint{3.111592in}{1.134170in}}%
\pgfpathlineto{\pgfqpoint{3.113023in}{1.133358in}}%
\pgfpathlineto{\pgfqpoint{3.113228in}{1.133494in}}%
\pgfpathlineto{\pgfqpoint{3.114659in}{1.136051in}}%
\pgfpathlineto{\pgfqpoint{3.117727in}{1.147861in}}%
\pgfpathlineto{\pgfqpoint{3.121612in}{1.160175in}}%
\pgfpathlineto{\pgfqpoint{3.123044in}{1.160045in}}%
\pgfpathlineto{\pgfqpoint{3.124680in}{1.155829in}}%
\pgfpathlineto{\pgfqpoint{3.127338in}{1.141009in}}%
\pgfpathlineto{\pgfqpoint{3.133882in}{1.102889in}}%
\pgfpathlineto{\pgfqpoint{3.135314in}{1.101871in}}%
\pgfpathlineto{\pgfqpoint{3.135518in}{1.102017in}}%
\pgfpathlineto{\pgfqpoint{3.136950in}{1.104934in}}%
\pgfpathlineto{\pgfqpoint{3.139813in}{1.117816in}}%
\pgfpathlineto{\pgfqpoint{3.144108in}{1.134812in}}%
\pgfpathlineto{\pgfqpoint{3.145335in}{1.135034in}}%
\pgfpathlineto{\pgfqpoint{3.145539in}{1.134796in}}%
\pgfpathlineto{\pgfqpoint{3.146971in}{1.130907in}}%
\pgfpathlineto{\pgfqpoint{3.149425in}{1.116245in}}%
\pgfpathlineto{\pgfqpoint{3.156582in}{1.069799in}}%
\pgfpathlineto{\pgfqpoint{3.157605in}{1.069306in}}%
\pgfpathlineto{\pgfqpoint{3.158014in}{1.069713in}}%
\pgfpathlineto{\pgfqpoint{3.159445in}{1.073655in}}%
\pgfpathlineto{\pgfqpoint{3.162513in}{1.091104in}}%
\pgfpathlineto{\pgfqpoint{3.166603in}{1.110768in}}%
\pgfpathlineto{\pgfqpoint{3.167625in}{1.111341in}}%
\pgfpathlineto{\pgfqpoint{3.168035in}{1.110914in}}%
\pgfpathlineto{\pgfqpoint{3.169466in}{1.106415in}}%
\pgfpathlineto{\pgfqpoint{3.171920in}{1.089154in}}%
\pgfpathlineto{\pgfqpoint{3.178873in}{1.035637in}}%
\pgfpathlineto{\pgfqpoint{3.179896in}{1.034950in}}%
\pgfpathlineto{\pgfqpoint{3.180100in}{1.035132in}}%
\pgfpathlineto{\pgfqpoint{3.181327in}{1.038359in}}%
\pgfpathlineto{\pgfqpoint{3.183781in}{1.053715in}}%
\pgfpathlineto{\pgfqpoint{3.189303in}{1.089424in}}%
\pgfpathlineto{\pgfqpoint{3.190121in}{1.089887in}}%
\pgfpathlineto{\pgfqpoint{3.190530in}{1.089412in}}%
\pgfpathlineto{\pgfqpoint{3.191961in}{1.083995in}}%
\pgfpathlineto{\pgfqpoint{3.194415in}{1.062788in}}%
\pgfpathlineto{\pgfqpoint{3.201369in}{0.997805in}}%
\pgfpathlineto{\pgfqpoint{3.202187in}{0.997286in}}%
\pgfpathlineto{\pgfqpoint{3.202596in}{0.997848in}}%
\pgfpathlineto{\pgfqpoint{3.204027in}{1.003935in}}%
\pgfpathlineto{\pgfqpoint{3.206686in}{1.028324in}}%
\pgfpathlineto{\pgfqpoint{3.211594in}{1.071844in}}%
\pgfpathlineto{\pgfqpoint{3.212616in}{1.073190in}}%
\pgfpathlineto{\pgfqpoint{3.213025in}{1.072635in}}%
\pgfpathlineto{\pgfqpoint{3.214252in}{1.067141in}}%
\pgfpathlineto{\pgfqpoint{3.216502in}{1.043533in}}%
\pgfpathlineto{\pgfqpoint{3.224068in}{0.952465in}}%
\pgfpathlineto{\pgfqpoint{3.224682in}{0.952694in}}%
\pgfpathlineto{\pgfqpoint{3.224886in}{0.953156in}}%
\pgfpathlineto{\pgfqpoint{3.226113in}{0.959869in}}%
\pgfpathlineto{\pgfqpoint{3.228567in}{0.990311in}}%
\pgfpathlineto{\pgfqpoint{3.234498in}{1.067482in}}%
\pgfpathlineto{\pgfqpoint{3.235112in}{1.068181in}}%
\pgfpathlineto{\pgfqpoint{3.235521in}{1.067466in}}%
\pgfpathlineto{\pgfqpoint{3.236748in}{1.059538in}}%
\pgfpathlineto{\pgfqpoint{3.238997in}{1.024258in}}%
\pgfpathlineto{\pgfqpoint{3.246564in}{0.886529in}}%
\pgfpathlineto{\pgfqpoint{3.247177in}{0.887730in}}%
\pgfpathlineto{\pgfqpoint{3.248404in}{0.898879in}}%
\pgfpathlineto{\pgfqpoint{3.250654in}{0.946484in}}%
\pgfpathlineto{\pgfqpoint{3.257402in}{1.107095in}}%
\pgfpathlineto{\pgfqpoint{3.257607in}{1.107213in}}%
\pgfpathlineto{\pgfqpoint{3.257811in}{1.106855in}}%
\pgfpathlineto{\pgfqpoint{3.258629in}{1.100527in}}%
\pgfpathlineto{\pgfqpoint{3.260265in}{1.063890in}}%
\pgfpathlineto{\pgfqpoint{3.263128in}{0.934154in}}%
\pgfpathlineto{\pgfqpoint{3.268446in}{0.698659in}}%
\pgfpathlineto{\pgfqpoint{3.268855in}{0.696000in}}%
\pgfpathlineto{\pgfqpoint{3.269468in}{0.699220in}}%
\pgfpathlineto{\pgfqpoint{3.270695in}{0.734735in}}%
\pgfpathlineto{\pgfqpoint{3.272536in}{0.870102in}}%
\pgfpathlineto{\pgfqpoint{3.275399in}{1.287088in}}%
\pgfpathlineto{\pgfqpoint{3.280307in}{2.428131in}}%
\pgfpathlineto{\pgfqpoint{3.286646in}{3.761365in}}%
\pgfpathlineto{\pgfqpoint{3.289714in}{4.026061in}}%
\pgfpathlineto{\pgfqpoint{3.291350in}{4.056000in}}%
\pgfpathlineto{\pgfqpoint{3.291554in}{4.055128in}}%
\pgfpathlineto{\pgfqpoint{3.292577in}{4.037800in}}%
\pgfpathlineto{\pgfqpoint{3.294826in}{3.943639in}}%
\pgfpathlineto{\pgfqpoint{3.301166in}{3.657639in}}%
\pgfpathlineto{\pgfqpoint{3.302598in}{3.644787in}}%
\pgfpathlineto{\pgfqpoint{3.303007in}{3.645488in}}%
\pgfpathlineto{\pgfqpoint{3.304029in}{3.654953in}}%
\pgfpathlineto{\pgfqpoint{3.306279in}{3.705914in}}%
\pgfpathlineto{\pgfqpoint{3.312618in}{3.861843in}}%
\pgfpathlineto{\pgfqpoint{3.313641in}{3.865471in}}%
\pgfpathlineto{\pgfqpoint{3.314050in}{3.864667in}}%
\pgfpathlineto{\pgfqpoint{3.315277in}{3.854985in}}%
\pgfpathlineto{\pgfqpoint{3.317731in}{3.809886in}}%
\pgfpathlineto{\pgfqpoint{3.324070in}{3.687416in}}%
\pgfpathlineto{\pgfqpoint{3.325297in}{3.683819in}}%
\pgfpathlineto{\pgfqpoint{3.325502in}{3.684053in}}%
\pgfpathlineto{\pgfqpoint{3.326524in}{3.688588in}}%
\pgfpathlineto{\pgfqpoint{3.328570in}{3.711636in}}%
\pgfpathlineto{\pgfqpoint{3.335523in}{3.799306in}}%
\pgfpathlineto{\pgfqpoint{3.336136in}{3.799535in}}%
\pgfpathlineto{\pgfqpoint{3.336341in}{3.799228in}}%
\pgfpathlineto{\pgfqpoint{3.337568in}{3.793487in}}%
\pgfpathlineto{\pgfqpoint{3.339817in}{3.768351in}}%
\pgfpathlineto{\pgfqpoint{3.346975in}{3.679883in}}%
\pgfpathlineto{\pgfqpoint{3.347793in}{3.678771in}}%
\pgfpathlineto{\pgfqpoint{3.348202in}{3.679160in}}%
\pgfpathlineto{\pgfqpoint{3.349429in}{3.683853in}}%
\pgfpathlineto{\pgfqpoint{3.351883in}{3.705369in}}%
\pgfpathlineto{\pgfqpoint{3.357404in}{3.753668in}}%
\pgfpathlineto{\pgfqpoint{3.358427in}{3.754728in}}%
\pgfpathlineto{\pgfqpoint{3.358631in}{3.754530in}}%
\pgfpathlineto{\pgfqpoint{3.359858in}{3.750518in}}%
\pgfpathlineto{\pgfqpoint{3.362108in}{3.732277in}}%
\pgfpathlineto{\pgfqpoint{3.369675in}{3.662588in}}%
\pgfpathlineto{\pgfqpoint{3.370493in}{3.662113in}}%
\pgfpathlineto{\pgfqpoint{3.370902in}{3.662576in}}%
\pgfpathlineto{\pgfqpoint{3.372333in}{3.667553in}}%
\pgfpathlineto{\pgfqpoint{3.375401in}{3.689910in}}%
\pgfpathlineto{\pgfqpoint{3.379491in}{3.715702in}}%
\pgfpathlineto{\pgfqpoint{3.380718in}{3.717095in}}%
\pgfpathlineto{\pgfqpoint{3.380922in}{3.716958in}}%
\pgfpathlineto{\pgfqpoint{3.382149in}{3.713919in}}%
\pgfpathlineto{\pgfqpoint{3.384399in}{3.699613in}}%
\pgfpathlineto{\pgfqpoint{3.392375in}{3.640825in}}%
\pgfpathlineto{\pgfqpoint{3.393397in}{3.640935in}}%
\pgfpathlineto{\pgfqpoint{3.393602in}{3.641232in}}%
\pgfpathlineto{\pgfqpoint{3.395238in}{3.646511in}}%
\pgfpathlineto{\pgfqpoint{3.399328in}{3.671206in}}%
\pgfpathlineto{\pgfqpoint{3.402395in}{3.682533in}}%
\pgfpathlineto{\pgfqpoint{3.403418in}{3.682474in}}%
\pgfpathlineto{\pgfqpoint{3.403622in}{3.682201in}}%
\pgfpathlineto{\pgfqpoint{3.405054in}{3.677907in}}%
\pgfpathlineto{\pgfqpoint{3.407712in}{3.660834in}}%
\pgfpathlineto{\pgfqpoint{3.414052in}{3.618397in}}%
\pgfpathlineto{\pgfqpoint{3.415688in}{3.616808in}}%
\pgfpathlineto{\pgfqpoint{3.416915in}{3.618802in}}%
\pgfpathlineto{\pgfqpoint{3.419369in}{3.628825in}}%
\pgfpathlineto{\pgfqpoint{3.424482in}{3.649765in}}%
\pgfpathlineto{\pgfqpoint{3.425709in}{3.649990in}}%
\pgfpathlineto{\pgfqpoint{3.425913in}{3.649771in}}%
\pgfpathlineto{\pgfqpoint{3.427345in}{3.646198in}}%
\pgfpathlineto{\pgfqpoint{3.430003in}{3.631609in}}%
\pgfpathlineto{\pgfqpoint{3.436752in}{3.592602in}}%
\pgfpathlineto{\pgfqpoint{3.438388in}{3.591647in}}%
\pgfpathlineto{\pgfqpoint{3.439819in}{3.594127in}}%
\pgfpathlineto{\pgfqpoint{3.442682in}{3.605038in}}%
\pgfpathlineto{\pgfqpoint{3.446772in}{3.618310in}}%
\pgfpathlineto{\pgfqpoint{3.448204in}{3.618382in}}%
\pgfpathlineto{\pgfqpoint{3.449635in}{3.615344in}}%
\pgfpathlineto{\pgfqpoint{3.452089in}{3.603831in}}%
\pgfpathlineto{\pgfqpoint{3.459452in}{3.566119in}}%
\pgfpathlineto{\pgfqpoint{3.461088in}{3.565635in}}%
\pgfpathlineto{\pgfqpoint{3.462724in}{3.568555in}}%
\pgfpathlineto{\pgfqpoint{3.470086in}{3.587911in}}%
\pgfpathlineto{\pgfqpoint{3.470699in}{3.587463in}}%
\pgfpathlineto{\pgfqpoint{3.472335in}{3.583801in}}%
\pgfpathlineto{\pgfqpoint{3.475198in}{3.570302in}}%
\pgfpathlineto{\pgfqpoint{3.481538in}{3.540090in}}%
\pgfpathlineto{\pgfqpoint{3.483378in}{3.538782in}}%
\pgfpathlineto{\pgfqpoint{3.485014in}{3.540928in}}%
\pgfpathlineto{\pgfqpoint{3.488491in}{3.551049in}}%
\pgfpathlineto{\pgfqpoint{3.491763in}{3.557544in}}%
\pgfpathlineto{\pgfqpoint{3.493195in}{3.557016in}}%
\pgfpathlineto{\pgfqpoint{3.495035in}{3.552731in}}%
\pgfpathlineto{\pgfqpoint{3.498307in}{3.537436in}}%
\pgfpathlineto{\pgfqpoint{3.503624in}{3.513771in}}%
\pgfpathlineto{\pgfqpoint{3.505669in}{3.511632in}}%
\pgfpathlineto{\pgfqpoint{3.507101in}{3.512827in}}%
\pgfpathlineto{\pgfqpoint{3.509759in}{3.518945in}}%
\pgfpathlineto{\pgfqpoint{3.514054in}{3.527577in}}%
\pgfpathlineto{\pgfqpoint{3.515485in}{3.527149in}}%
\pgfpathlineto{\pgfqpoint{3.517326in}{3.523322in}}%
\pgfpathlineto{\pgfqpoint{3.520394in}{3.510357in}}%
\pgfpathlineto{\pgfqpoint{3.526324in}{3.485903in}}%
\pgfpathlineto{\pgfqpoint{3.528369in}{3.484286in}}%
\pgfpathlineto{\pgfqpoint{3.530005in}{3.485860in}}%
\pgfpathlineto{\pgfqpoint{3.533891in}{3.494456in}}%
\pgfpathlineto{\pgfqpoint{3.536754in}{3.497967in}}%
\pgfpathlineto{\pgfqpoint{3.538185in}{3.497045in}}%
\pgfpathlineto{\pgfqpoint{3.540230in}{3.492212in}}%
\pgfpathlineto{\pgfqpoint{3.544116in}{3.475266in}}%
\pgfpathlineto{\pgfqpoint{3.548615in}{3.458638in}}%
\pgfpathlineto{\pgfqpoint{3.550864in}{3.456681in}}%
\pgfpathlineto{\pgfqpoint{3.552501in}{3.458041in}}%
\pgfpathlineto{\pgfqpoint{3.556591in}{3.465973in}}%
\pgfpathlineto{\pgfqpoint{3.559249in}{3.468446in}}%
\pgfpathlineto{\pgfqpoint{3.560885in}{3.467066in}}%
\pgfpathlineto{\pgfqpoint{3.563135in}{3.461235in}}%
\pgfpathlineto{\pgfqpoint{3.573360in}{3.428912in}}%
\pgfpathlineto{\pgfqpoint{3.575200in}{3.430356in}}%
\pgfpathlineto{\pgfqpoint{3.582358in}{3.438753in}}%
\pgfpathlineto{\pgfqpoint{3.584199in}{3.435913in}}%
\pgfpathlineto{\pgfqpoint{3.587062in}{3.426426in}}%
\pgfpathlineto{\pgfqpoint{3.594015in}{3.402218in}}%
\pgfpathlineto{\pgfqpoint{3.596060in}{3.401044in}}%
\pgfpathlineto{\pgfqpoint{3.598105in}{3.402802in}}%
\pgfpathlineto{\pgfqpoint{3.604240in}{3.409756in}}%
\pgfpathlineto{\pgfqpoint{3.605876in}{3.408123in}}%
\pgfpathlineto{\pgfqpoint{3.608330in}{3.401796in}}%
\pgfpathlineto{\pgfqpoint{3.617737in}{3.373123in}}%
\pgfpathlineto{\pgfqpoint{3.619578in}{3.373537in}}%
\pgfpathlineto{\pgfqpoint{3.622645in}{3.377512in}}%
\pgfpathlineto{\pgfqpoint{3.626122in}{3.380678in}}%
\pgfpathlineto{\pgfqpoint{3.627962in}{3.379431in}}%
\pgfpathlineto{\pgfqpoint{3.630212in}{3.374490in}}%
\pgfpathlineto{\pgfqpoint{3.635120in}{3.356282in}}%
\pgfpathlineto{\pgfqpoint{3.639005in}{3.346100in}}%
\pgfpathlineto{\pgfqpoint{3.641255in}{3.344976in}}%
\pgfpathlineto{\pgfqpoint{3.643504in}{3.346790in}}%
\pgfpathlineto{\pgfqpoint{3.648617in}{3.351591in}}%
\pgfpathlineto{\pgfqpoint{3.650458in}{3.350204in}}%
\pgfpathlineto{\pgfqpoint{3.652912in}{3.344683in}}%
\pgfpathlineto{\pgfqpoint{3.663137in}{3.316787in}}%
\pgfpathlineto{\pgfqpoint{3.665182in}{3.317612in}}%
\pgfpathlineto{\pgfqpoint{3.671726in}{3.322300in}}%
\pgfpathlineto{\pgfqpoint{3.673771in}{3.319631in}}%
\pgfpathlineto{\pgfqpoint{3.676838in}{3.310898in}}%
\pgfpathlineto{\pgfqpoint{3.683996in}{3.289705in}}%
\pgfpathlineto{\pgfqpoint{3.686246in}{3.288577in}}%
\pgfpathlineto{\pgfqpoint{3.688700in}{3.290233in}}%
\pgfpathlineto{\pgfqpoint{3.693403in}{3.293604in}}%
\pgfpathlineto{\pgfqpoint{3.695244in}{3.292218in}}%
\pgfpathlineto{\pgfqpoint{3.697698in}{3.287049in}}%
\pgfpathlineto{\pgfqpoint{3.708127in}{3.260320in}}%
\pgfpathlineto{\pgfqpoint{3.710377in}{3.261086in}}%
\pgfpathlineto{\pgfqpoint{3.716308in}{3.264500in}}%
\pgfpathlineto{\pgfqpoint{3.718353in}{3.262197in}}%
\pgfpathlineto{\pgfqpoint{3.721216in}{3.254960in}}%
\pgfpathlineto{\pgfqpoint{3.729191in}{3.232892in}}%
\pgfpathlineto{\pgfqpoint{3.731441in}{3.232017in}}%
\pgfpathlineto{\pgfqpoint{3.734099in}{3.233670in}}%
\pgfpathlineto{\pgfqpoint{3.738189in}{3.235795in}}%
\pgfpathlineto{\pgfqpoint{3.740234in}{3.234138in}}%
\pgfpathlineto{\pgfqpoint{3.742893in}{3.228416in}}%
\pgfpathlineto{\pgfqpoint{3.752505in}{3.203917in}}%
\pgfpathlineto{\pgfqpoint{3.754754in}{3.203920in}}%
\pgfpathlineto{\pgfqpoint{3.761707in}{3.206324in}}%
\pgfpathlineto{\pgfqpoint{3.763957in}{3.202961in}}%
\pgfpathlineto{\pgfqpoint{3.767638in}{3.192483in}}%
\pgfpathlineto{\pgfqpoint{3.773364in}{3.177162in}}%
\pgfpathlineto{\pgfqpoint{3.776022in}{3.175261in}}%
\pgfpathlineto{\pgfqpoint{3.778681in}{3.176329in}}%
\pgfpathlineto{\pgfqpoint{3.782976in}{3.178118in}}%
\pgfpathlineto{\pgfqpoint{3.785021in}{3.176479in}}%
\pgfpathlineto{\pgfqpoint{3.787679in}{3.171039in}}%
\pgfpathlineto{\pgfqpoint{3.797700in}{3.147062in}}%
\pgfpathlineto{\pgfqpoint{3.800154in}{3.147240in}}%
\pgfpathlineto{\pgfqpoint{3.806084in}{3.148990in}}%
\pgfpathlineto{\pgfqpoint{3.808334in}{3.146240in}}%
\pgfpathlineto{\pgfqpoint{3.811606in}{3.138012in}}%
\pgfpathlineto{\pgfqpoint{3.818559in}{3.120017in}}%
\pgfpathlineto{\pgfqpoint{3.821218in}{3.118432in}}%
\pgfpathlineto{\pgfqpoint{3.824285in}{3.119638in}}%
\pgfpathlineto{\pgfqpoint{3.827762in}{3.120540in}}%
\pgfpathlineto{\pgfqpoint{3.830011in}{3.118641in}}%
\pgfpathlineto{\pgfqpoint{3.832874in}{3.112648in}}%
\pgfpathlineto{\pgfqpoint{3.842077in}{3.090563in}}%
\pgfpathlineto{\pgfqpoint{3.844531in}{3.090083in}}%
\pgfpathlineto{\pgfqpoint{3.851484in}{3.090962in}}%
\pgfpathlineto{\pgfqpoint{3.853938in}{3.087196in}}%
\pgfpathlineto{\pgfqpoint{3.858233in}{3.075358in}}%
\pgfpathlineto{\pgfqpoint{3.863345in}{3.063332in}}%
\pgfpathlineto{\pgfqpoint{3.866208in}{3.061525in}}%
\pgfpathlineto{\pgfqpoint{3.869480in}{3.062578in}}%
\pgfpathlineto{\pgfqpoint{3.872753in}{3.062964in}}%
\pgfpathlineto{\pgfqpoint{3.875002in}{3.060879in}}%
\pgfpathlineto{\pgfqpoint{3.878070in}{3.054335in}}%
\pgfpathlineto{\pgfqpoint{3.886659in}{3.033952in}}%
\pgfpathlineto{\pgfqpoint{3.889317in}{3.033075in}}%
\pgfpathlineto{\pgfqpoint{3.896475in}{3.033296in}}%
\pgfpathlineto{\pgfqpoint{3.899133in}{3.028987in}}%
\pgfpathlineto{\pgfqpoint{3.904451in}{3.014348in}}%
\pgfpathlineto{\pgfqpoint{3.908745in}{3.005848in}}%
\pgfpathlineto{\pgfqpoint{3.911404in}{3.004550in}}%
\pgfpathlineto{\pgfqpoint{3.919584in}{3.003756in}}%
\pgfpathlineto{\pgfqpoint{3.922651in}{2.997692in}}%
\pgfpathlineto{\pgfqpoint{3.931854in}{2.976783in}}%
\pgfpathlineto{\pgfqpoint{3.934512in}{2.976084in}}%
\pgfpathlineto{\pgfqpoint{3.941057in}{2.976079in}}%
\pgfpathlineto{\pgfqpoint{3.943715in}{2.972144in}}%
\pgfpathlineto{\pgfqpoint{3.948214in}{2.960336in}}%
\pgfpathlineto{\pgfqpoint{3.953122in}{2.949505in}}%
\pgfpathlineto{\pgfqpoint{3.955985in}{2.947561in}}%
\pgfpathlineto{\pgfqpoint{3.959666in}{2.948227in}}%
\pgfpathlineto{\pgfqpoint{3.962734in}{2.947917in}}%
\pgfpathlineto{\pgfqpoint{3.965188in}{2.945197in}}%
\pgfpathlineto{\pgfqpoint{3.968665in}{2.937324in}}%
\pgfpathlineto{\pgfqpoint{3.975822in}{2.920705in}}%
\pgfpathlineto{\pgfqpoint{3.978685in}{2.919014in}}%
\pgfpathlineto{\pgfqpoint{3.986865in}{2.917596in}}%
\pgfpathlineto{\pgfqpoint{3.989933in}{2.911699in}}%
\pgfpathlineto{\pgfqpoint{3.999340in}{2.891200in}}%
\pgfpathlineto{\pgfqpoint{4.002203in}{2.890499in}}%
\pgfpathlineto{\pgfqpoint{4.007929in}{2.890292in}}%
\pgfpathlineto{\pgfqpoint{4.010588in}{2.886858in}}%
\pgfpathlineto{\pgfqpoint{4.014678in}{2.876916in}}%
\pgfpathlineto{\pgfqpoint{4.020404in}{2.864075in}}%
\pgfpathlineto{\pgfqpoint{4.023471in}{2.861967in}}%
\pgfpathlineto{\pgfqpoint{4.032060in}{2.859740in}}%
\pgfpathlineto{\pgfqpoint{4.035333in}{2.853054in}}%
\pgfpathlineto{\pgfqpoint{4.043717in}{2.834596in}}%
\pgfpathlineto{\pgfqpoint{4.046580in}{2.833365in}}%
\pgfpathlineto{\pgfqpoint{4.053329in}{2.832455in}}%
\pgfpathlineto{\pgfqpoint{4.056192in}{2.828049in}}%
\pgfpathlineto{\pgfqpoint{4.062122in}{2.812978in}}%
\pgfpathlineto{\pgfqpoint{4.066213in}{2.806024in}}%
\pgfpathlineto{\pgfqpoint{4.069280in}{2.804806in}}%
\pgfpathlineto{\pgfqpoint{4.075620in}{2.803911in}}%
\pgfpathlineto{\pgfqpoint{4.078483in}{2.799674in}}%
\pgfpathlineto{\pgfqpoint{4.083800in}{2.786291in}}%
\pgfpathlineto{\pgfqpoint{4.088299in}{2.777845in}}%
\pgfpathlineto{\pgfqpoint{4.091366in}{2.776257in}}%
\pgfpathlineto{\pgfqpoint{4.098524in}{2.774762in}}%
\pgfpathlineto{\pgfqpoint{4.101592in}{2.769584in}}%
\pgfpathlineto{\pgfqpoint{4.112226in}{2.748152in}}%
\pgfpathlineto{\pgfqpoint{4.115498in}{2.747751in}}%
\pgfpathlineto{\pgfqpoint{4.119792in}{2.747131in}}%
\pgfpathlineto{\pgfqpoint{4.122451in}{2.744011in}}%
\pgfpathlineto{\pgfqpoint{4.126336in}{2.735252in}}%
\pgfpathlineto{\pgfqpoint{4.132881in}{2.721100in}}%
\pgfpathlineto{\pgfqpoint{4.135948in}{2.719170in}}%
\pgfpathlineto{\pgfqpoint{4.143719in}{2.717051in}}%
\pgfpathlineto{\pgfqpoint{4.146991in}{2.711086in}}%
\pgfpathlineto{\pgfqpoint{4.156398in}{2.691510in}}%
\pgfpathlineto{\pgfqpoint{4.159466in}{2.690520in}}%
\pgfpathlineto{\pgfqpoint{4.164988in}{2.689582in}}%
\pgfpathlineto{\pgfqpoint{4.167851in}{2.685762in}}%
\pgfpathlineto{\pgfqpoint{4.172554in}{2.674504in}}%
\pgfpathlineto{\pgfqpoint{4.177667in}{2.664110in}}%
\pgfpathlineto{\pgfqpoint{4.180734in}{2.662055in}}%
\pgfpathlineto{\pgfqpoint{4.188710in}{2.659593in}}%
\pgfpathlineto{\pgfqpoint{4.191982in}{2.653537in}}%
\pgfpathlineto{\pgfqpoint{4.201185in}{2.634469in}}%
\pgfpathlineto{\pgfqpoint{4.204252in}{2.633357in}}%
\pgfpathlineto{\pgfqpoint{4.209978in}{2.632193in}}%
\pgfpathlineto{\pgfqpoint{4.212841in}{2.628251in}}%
\pgfpathlineto{\pgfqpoint{4.217749in}{2.616485in}}%
\pgfpathlineto{\pgfqpoint{4.222657in}{2.606856in}}%
\pgfpathlineto{\pgfqpoint{4.225930in}{2.604846in}}%
\pgfpathlineto{\pgfqpoint{4.233292in}{2.602663in}}%
\pgfpathlineto{\pgfqpoint{4.236359in}{2.597399in}}%
\pgfpathlineto{\pgfqpoint{4.246789in}{2.576832in}}%
\pgfpathlineto{\pgfqpoint{4.250265in}{2.576187in}}%
\pgfpathlineto{\pgfqpoint{4.254355in}{2.575308in}}%
\pgfpathlineto{\pgfqpoint{4.257219in}{2.571870in}}%
\pgfpathlineto{\pgfqpoint{4.261513in}{2.562133in}}%
\pgfpathlineto{\pgfqpoint{4.267444in}{2.549821in}}%
\pgfpathlineto{\pgfqpoint{4.270716in}{2.547699in}}%
\pgfpathlineto{\pgfqpoint{4.278282in}{2.545256in}}%
\pgfpathlineto{\pgfqpoint{4.281554in}{2.539427in}}%
\pgfpathlineto{\pgfqpoint{4.291166in}{2.520007in}}%
\pgfpathlineto{\pgfqpoint{4.294438in}{2.518990in}}%
\pgfpathlineto{\pgfqpoint{4.299346in}{2.517974in}}%
\pgfpathlineto{\pgfqpoint{4.302209in}{2.514410in}}%
\pgfpathlineto{\pgfqpoint{4.306504in}{2.504619in}}%
\pgfpathlineto{\pgfqpoint{4.312230in}{2.492769in}}%
\pgfpathlineto{\pgfqpoint{4.315502in}{2.490545in}}%
\pgfpathlineto{\pgfqpoint{4.323273in}{2.487862in}}%
\pgfpathlineto{\pgfqpoint{4.326545in}{2.481923in}}%
\pgfpathlineto{\pgfqpoint{4.335952in}{2.462906in}}%
\pgfpathlineto{\pgfqpoint{4.339224in}{2.461803in}}%
\pgfpathlineto{\pgfqpoint{4.344337in}{2.460651in}}%
\pgfpathlineto{\pgfqpoint{4.347200in}{2.456965in}}%
\pgfpathlineto{\pgfqpoint{4.351699in}{2.446585in}}%
\pgfpathlineto{\pgfqpoint{4.357221in}{2.435453in}}%
\pgfpathlineto{\pgfqpoint{4.360493in}{2.433337in}}%
\pgfpathlineto{\pgfqpoint{4.367855in}{2.430987in}}%
\pgfpathlineto{\pgfqpoint{4.371127in}{2.425354in}}%
\pgfpathlineto{\pgfqpoint{4.381147in}{2.405490in}}%
\pgfpathlineto{\pgfqpoint{4.384419in}{2.404601in}}%
\pgfpathlineto{\pgfqpoint{4.389123in}{2.403502in}}%
\pgfpathlineto{\pgfqpoint{4.391986in}{2.399897in}}%
\pgfpathlineto{\pgfqpoint{4.396485in}{2.389596in}}%
\pgfpathlineto{\pgfqpoint{4.402007in}{2.378363in}}%
\pgfpathlineto{\pgfqpoint{4.405279in}{2.376162in}}%
\pgfpathlineto{\pgfqpoint{4.412845in}{2.373637in}}%
\pgfpathlineto{\pgfqpoint{4.416117in}{2.367878in}}%
\pgfpathlineto{\pgfqpoint{4.425934in}{2.348349in}}%
\pgfpathlineto{\pgfqpoint{4.429206in}{2.347406in}}%
\pgfpathlineto{\pgfqpoint{4.433909in}{2.346366in}}%
\pgfpathlineto{\pgfqpoint{4.436772in}{2.342844in}}%
\pgfpathlineto{\pgfqpoint{4.441067in}{2.333127in}}%
\pgfpathlineto{\pgfqpoint{4.446997in}{2.321013in}}%
\pgfpathlineto{\pgfqpoint{4.450270in}{2.318934in}}%
\pgfpathlineto{\pgfqpoint{4.457632in}{2.316547in}}%
\pgfpathlineto{\pgfqpoint{4.460904in}{2.310870in}}%
\pgfpathlineto{\pgfqpoint{4.470720in}{2.291195in}}%
\pgfpathlineto{\pgfqpoint{4.473992in}{2.290206in}}%
\pgfpathlineto{\pgfqpoint{4.478900in}{2.289094in}}%
\pgfpathlineto{\pgfqpoint{4.481763in}{2.285450in}}%
\pgfpathlineto{\pgfqpoint{4.486262in}{2.275092in}}%
\pgfpathlineto{\pgfqpoint{4.491784in}{2.263887in}}%
\pgfpathlineto{\pgfqpoint{4.495056in}{2.261739in}}%
\pgfpathlineto{\pgfqpoint{4.502622in}{2.259231in}}%
\pgfpathlineto{\pgfqpoint{4.505894in}{2.253416in}}%
\pgfpathlineto{\pgfqpoint{4.515506in}{2.234026in}}%
\pgfpathlineto{\pgfqpoint{4.518778in}{2.233003in}}%
\pgfpathlineto{\pgfqpoint{4.523686in}{2.231993in}}%
\pgfpathlineto{\pgfqpoint{4.526549in}{2.228432in}}%
\pgfpathlineto{\pgfqpoint{4.530844in}{2.218625in}}%
\pgfpathlineto{\pgfqpoint{4.536570in}{2.206744in}}%
\pgfpathlineto{\pgfqpoint{4.539842in}{2.204534in}}%
\pgfpathlineto{\pgfqpoint{4.547613in}{2.201928in}}%
\pgfpathlineto{\pgfqpoint{4.550885in}{2.195969in}}%
\pgfpathlineto{\pgfqpoint{4.560292in}{2.176841in}}%
\pgfpathlineto{\pgfqpoint{4.563564in}{2.175794in}}%
\pgfpathlineto{\pgfqpoint{4.568677in}{2.174754in}}%
\pgfpathlineto{\pgfqpoint{4.571540in}{2.171070in}}%
\pgfpathlineto{\pgfqpoint{4.576039in}{2.160580in}}%
\pgfpathlineto{\pgfqpoint{4.581561in}{2.149337in}}%
\pgfpathlineto{\pgfqpoint{4.584833in}{2.147276in}}%
\pgfpathlineto{\pgfqpoint{4.592195in}{2.145144in}}%
\pgfpathlineto{\pgfqpoint{4.595262in}{2.139920in}}%
\pgfpathlineto{\pgfqpoint{4.605692in}{2.119221in}}%
\pgfpathlineto{\pgfqpoint{4.608964in}{2.118593in}}%
\pgfpathlineto{\pgfqpoint{4.613259in}{2.117833in}}%
\pgfpathlineto{\pgfqpoint{4.615917in}{2.114787in}}%
\pgfpathlineto{\pgfqpoint{4.619803in}{2.106245in}}%
\pgfpathlineto{\pgfqpoint{4.626347in}{2.092153in}}%
\pgfpathlineto{\pgfqpoint{4.629414in}{2.090086in}}%
\pgfpathlineto{\pgfqpoint{4.637594in}{2.087366in}}%
\pgfpathlineto{\pgfqpoint{4.640866in}{2.081089in}}%
\pgfpathlineto{\pgfqpoint{4.649660in}{2.062590in}}%
\pgfpathlineto{\pgfqpoint{4.652728in}{2.061355in}}%
\pgfpathlineto{\pgfqpoint{4.658863in}{2.060144in}}%
\pgfpathlineto{\pgfqpoint{4.661726in}{2.055985in}}%
\pgfpathlineto{\pgfqpoint{4.667043in}{2.042894in}}%
\pgfpathlineto{\pgfqpoint{4.671542in}{2.034476in}}%
\pgfpathlineto{\pgfqpoint{4.674610in}{2.032772in}}%
\pgfpathlineto{\pgfqpoint{4.681767in}{2.031131in}}%
\pgfpathlineto{\pgfqpoint{4.684835in}{2.026029in}}%
\pgfpathlineto{\pgfqpoint{4.695673in}{2.004518in}}%
\pgfpathlineto{\pgfqpoint{4.698945in}{2.004204in}}%
\pgfpathlineto{\pgfqpoint{4.703035in}{2.003609in}}%
\pgfpathlineto{\pgfqpoint{4.705694in}{2.000539in}}%
\pgfpathlineto{\pgfqpoint{4.709580in}{1.991786in}}%
\pgfpathlineto{\pgfqpoint{4.716124in}{1.977473in}}%
\pgfpathlineto{\pgfqpoint{4.719191in}{1.975538in}}%
\pgfpathlineto{\pgfqpoint{4.727167in}{1.973435in}}%
\pgfpathlineto{\pgfqpoint{4.730439in}{1.967229in}}%
\pgfpathlineto{\pgfqpoint{4.739642in}{1.947745in}}%
\pgfpathlineto{\pgfqpoint{4.742709in}{1.946867in}}%
\pgfpathlineto{\pgfqpoint{4.748231in}{1.946317in}}%
\pgfpathlineto{\pgfqpoint{4.750889in}{1.942916in}}%
\pgfpathlineto{\pgfqpoint{4.754979in}{1.933180in}}%
\pgfpathlineto{\pgfqpoint{4.760910in}{1.920215in}}%
\pgfpathlineto{\pgfqpoint{4.763977in}{1.918255in}}%
\pgfpathlineto{\pgfqpoint{4.772158in}{1.916240in}}%
\pgfpathlineto{\pgfqpoint{4.775430in}{1.909828in}}%
\pgfpathlineto{\pgfqpoint{4.784223in}{1.890614in}}%
\pgfpathlineto{\pgfqpoint{4.787086in}{1.889573in}}%
\pgfpathlineto{\pgfqpoint{4.793630in}{1.888839in}}%
\pgfpathlineto{\pgfqpoint{4.796493in}{1.884446in}}%
\pgfpathlineto{\pgfqpoint{4.802015in}{1.870139in}}%
\pgfpathlineto{\pgfqpoint{4.806310in}{1.862240in}}%
\pgfpathlineto{\pgfqpoint{4.809173in}{1.860920in}}%
\pgfpathlineto{\pgfqpoint{4.816330in}{1.860076in}}%
\pgfpathlineto{\pgfqpoint{4.819193in}{1.855361in}}%
\pgfpathlineto{\pgfqpoint{4.830645in}{1.832377in}}%
\pgfpathlineto{\pgfqpoint{4.833918in}{1.832692in}}%
\pgfpathlineto{\pgfqpoint{4.837599in}{1.832529in}}%
\pgfpathlineto{\pgfqpoint{4.840053in}{1.829879in}}%
\pgfpathlineto{\pgfqpoint{4.843529in}{1.822104in}}%
\pgfpathlineto{\pgfqpoint{4.850891in}{1.805118in}}%
\pgfpathlineto{\pgfqpoint{4.853754in}{1.803598in}}%
\pgfpathlineto{\pgfqpoint{4.861730in}{1.802495in}}%
\pgfpathlineto{\pgfqpoint{4.864797in}{1.796611in}}%
\pgfpathlineto{\pgfqpoint{4.874409in}{1.775453in}}%
\pgfpathlineto{\pgfqpoint{4.877068in}{1.775008in}}%
\pgfpathlineto{\pgfqpoint{4.882998in}{1.775217in}}%
\pgfpathlineto{\pgfqpoint{4.885452in}{1.772011in}}%
\pgfpathlineto{\pgfqpoint{4.889338in}{1.762324in}}%
\pgfpathlineto{\pgfqpoint{4.895268in}{1.748244in}}%
\pgfpathlineto{\pgfqpoint{4.898132in}{1.746258in}}%
\pgfpathlineto{\pgfqpoint{4.901404in}{1.746994in}}%
\pgfpathlineto{\pgfqpoint{4.904880in}{1.747104in}}%
\pgfpathlineto{\pgfqpoint{4.907334in}{1.744513in}}%
\pgfpathlineto{\pgfqpoint{4.910606in}{1.737088in}}%
\pgfpathlineto{\pgfqpoint{4.918173in}{1.718920in}}%
\pgfpathlineto{\pgfqpoint{4.920831in}{1.717534in}}%
\pgfpathlineto{\pgfqpoint{4.924512in}{1.718671in}}%
\pgfpathlineto{\pgfqpoint{4.927375in}{1.718618in}}%
\pgfpathlineto{\pgfqpoint{4.929830in}{1.715972in}}%
\pgfpathlineto{\pgfqpoint{4.933102in}{1.708384in}}%
\pgfpathlineto{\pgfqpoint{4.940668in}{1.690086in}}%
\pgfpathlineto{\pgfqpoint{4.943327in}{1.688825in}}%
\pgfpathlineto{\pgfqpoint{4.947212in}{1.690225in}}%
\pgfpathlineto{\pgfqpoint{4.950075in}{1.690030in}}%
\pgfpathlineto{\pgfqpoint{4.952529in}{1.687085in}}%
\pgfpathlineto{\pgfqpoint{4.956006in}{1.678483in}}%
\pgfpathlineto{\pgfqpoint{4.962755in}{1.661702in}}%
\pgfpathlineto{\pgfqpoint{4.965413in}{1.660096in}}%
\pgfpathlineto{\pgfqpoint{4.968481in}{1.661148in}}%
\pgfpathlineto{\pgfqpoint{4.971957in}{1.661869in}}%
\pgfpathlineto{\pgfqpoint{4.974207in}{1.659900in}}%
\pgfpathlineto{\pgfqpoint{4.977070in}{1.653874in}}%
\pgfpathlineto{\pgfqpoint{4.986272in}{1.631861in}}%
\pgfpathlineto{\pgfqpoint{4.988726in}{1.631489in}}%
\pgfpathlineto{\pgfqpoint{4.995680in}{1.632656in}}%
\pgfpathlineto{\pgfqpoint{4.998134in}{1.628798in}}%
\pgfpathlineto{\pgfqpoint{5.002428in}{1.616523in}}%
\pgfpathlineto{\pgfqpoint{5.007336in}{1.604451in}}%
\pgfpathlineto{\pgfqpoint{5.009995in}{1.602592in}}%
\pgfpathlineto{\pgfqpoint{5.012653in}{1.603536in}}%
\pgfpathlineto{\pgfqpoint{5.016948in}{1.605017in}}%
\pgfpathlineto{\pgfqpoint{5.019197in}{1.602968in}}%
\pgfpathlineto{\pgfqpoint{5.022060in}{1.596613in}}%
\pgfpathlineto{\pgfqpoint{5.030854in}{1.574423in}}%
\pgfpathlineto{\pgfqpoint{5.033308in}{1.573939in}}%
\pgfpathlineto{\pgfqpoint{5.040875in}{1.575616in}}%
\pgfpathlineto{\pgfqpoint{5.043329in}{1.571270in}}%
\pgfpathlineto{\pgfqpoint{5.048032in}{1.556861in}}%
\pgfpathlineto{\pgfqpoint{5.052327in}{1.546567in}}%
\pgfpathlineto{\pgfqpoint{5.054781in}{1.544991in}}%
\pgfpathlineto{\pgfqpoint{5.057235in}{1.546054in}}%
\pgfpathlineto{\pgfqpoint{5.061939in}{1.548260in}}%
\pgfpathlineto{\pgfqpoint{5.063984in}{1.546436in}}%
\pgfpathlineto{\pgfqpoint{5.066847in}{1.539984in}}%
\pgfpathlineto{\pgfqpoint{5.075845in}{1.516578in}}%
\pgfpathlineto{\pgfqpoint{5.078094in}{1.516364in}}%
\pgfpathlineto{\pgfqpoint{5.081980in}{1.519317in}}%
\pgfpathlineto{\pgfqpoint{5.084638in}{1.519850in}}%
\pgfpathlineto{\pgfqpoint{5.086683in}{1.517737in}}%
\pgfpathlineto{\pgfqpoint{5.089547in}{1.510773in}}%
\pgfpathlineto{\pgfqpoint{5.097931in}{1.487921in}}%
\pgfpathlineto{\pgfqpoint{5.100181in}{1.487412in}}%
\pgfpathlineto{\pgfqpoint{5.103044in}{1.489626in}}%
\pgfpathlineto{\pgfqpoint{5.106725in}{1.491680in}}%
\pgfpathlineto{\pgfqpoint{5.108770in}{1.490030in}}%
\pgfpathlineto{\pgfqpoint{5.111428in}{1.484066in}}%
\pgfpathlineto{\pgfqpoint{5.121040in}{1.458521in}}%
\pgfpathlineto{\pgfqpoint{5.123085in}{1.458804in}}%
\pgfpathlineto{\pgfqpoint{5.130652in}{1.462531in}}%
\pgfpathlineto{\pgfqpoint{5.132901in}{1.458422in}}%
\pgfpathlineto{\pgfqpoint{5.136991in}{1.444788in}}%
\pgfpathlineto{\pgfqpoint{5.141899in}{1.430960in}}%
\pgfpathlineto{\pgfqpoint{5.144353in}{1.429430in}}%
\pgfpathlineto{\pgfqpoint{5.146603in}{1.430950in}}%
\pgfpathlineto{\pgfqpoint{5.151920in}{1.435168in}}%
\pgfpathlineto{\pgfqpoint{5.153761in}{1.433493in}}%
\pgfpathlineto{\pgfqpoint{5.156215in}{1.427698in}}%
\pgfpathlineto{\pgfqpoint{5.166031in}{1.400447in}}%
\pgfpathlineto{\pgfqpoint{5.168076in}{1.401186in}}%
\pgfpathlineto{\pgfqpoint{5.175233in}{1.406563in}}%
\pgfpathlineto{\pgfqpoint{5.177278in}{1.403273in}}%
\pgfpathlineto{\pgfqpoint{5.180550in}{1.392618in}}%
\pgfpathlineto{\pgfqpoint{5.186686in}{1.372866in}}%
\pgfpathlineto{\pgfqpoint{5.188935in}{1.371335in}}%
\pgfpathlineto{\pgfqpoint{5.190980in}{1.372810in}}%
\pgfpathlineto{\pgfqpoint{5.197115in}{1.378877in}}%
\pgfpathlineto{\pgfqpoint{5.198956in}{1.376817in}}%
\pgfpathlineto{\pgfqpoint{5.201614in}{1.369424in}}%
\pgfpathlineto{\pgfqpoint{5.209999in}{1.342603in}}%
\pgfpathlineto{\pgfqpoint{5.211839in}{1.342383in}}%
\pgfpathlineto{\pgfqpoint{5.214089in}{1.344933in}}%
\pgfpathlineto{\pgfqpoint{5.219202in}{1.350987in}}%
\pgfpathlineto{\pgfqpoint{5.220838in}{1.349782in}}%
\pgfpathlineto{\pgfqpoint{5.223087in}{1.344546in}}%
\pgfpathlineto{\pgfqpoint{5.227791in}{1.325574in}}%
\pgfpathlineto{\pgfqpoint{5.231676in}{1.314149in}}%
\pgfpathlineto{\pgfqpoint{5.233721in}{1.312962in}}%
\pgfpathlineto{\pgfqpoint{5.235562in}{1.314601in}}%
\pgfpathlineto{\pgfqpoint{5.242310in}{1.322912in}}%
\pgfpathlineto{\pgfqpoint{5.243946in}{1.320789in}}%
\pgfpathlineto{\pgfqpoint{5.246605in}{1.312575in}}%
\pgfpathlineto{\pgfqpoint{5.254785in}{1.283934in}}%
\pgfpathlineto{\pgfqpoint{5.256626in}{1.283903in}}%
\pgfpathlineto{\pgfqpoint{5.258875in}{1.287177in}}%
\pgfpathlineto{\pgfqpoint{5.263988in}{1.295319in}}%
\pgfpathlineto{\pgfqpoint{5.265624in}{1.294378in}}%
\pgfpathlineto{\pgfqpoint{5.267669in}{1.289617in}}%
\pgfpathlineto{\pgfqpoint{5.271350in}{1.273779in}}%
\pgfpathlineto{\pgfqpoint{5.276258in}{1.255597in}}%
\pgfpathlineto{\pgfqpoint{5.278303in}{1.254066in}}%
\pgfpathlineto{\pgfqpoint{5.279939in}{1.255628in}}%
\pgfpathlineto{\pgfqpoint{5.283620in}{1.263764in}}%
\pgfpathlineto{\pgfqpoint{5.286483in}{1.267714in}}%
\pgfpathlineto{\pgfqpoint{5.287915in}{1.267041in}}%
\pgfpathlineto{\pgfqpoint{5.289755in}{1.263040in}}%
\pgfpathlineto{\pgfqpoint{5.293027in}{1.248963in}}%
\pgfpathlineto{\pgfqpoint{5.298549in}{1.226103in}}%
\pgfpathlineto{\pgfqpoint{5.300594in}{1.224351in}}%
\pgfpathlineto{\pgfqpoint{5.302230in}{1.226039in}}%
\pgfpathlineto{\pgfqpoint{5.305502in}{1.234175in}}%
\pgfpathlineto{\pgfqpoint{5.308978in}{1.240323in}}%
\pgfpathlineto{\pgfqpoint{5.310410in}{1.239662in}}%
\pgfpathlineto{\pgfqpoint{5.312250in}{1.235376in}}%
\pgfpathlineto{\pgfqpoint{5.315523in}{1.220197in}}%
\pgfpathlineto{\pgfqpoint{5.321044in}{1.195970in}}%
\pgfpathlineto{\pgfqpoint{5.323089in}{1.194456in}}%
\pgfpathlineto{\pgfqpoint{5.324725in}{1.196632in}}%
\pgfpathlineto{\pgfqpoint{5.328202in}{1.206744in}}%
\pgfpathlineto{\pgfqpoint{5.331474in}{1.213218in}}%
\pgfpathlineto{\pgfqpoint{5.332905in}{1.212573in}}%
\pgfpathlineto{\pgfqpoint{5.334746in}{1.207938in}}%
\pgfpathlineto{\pgfqpoint{5.338018in}{1.191407in}}%
\pgfpathlineto{\pgfqpoint{5.343335in}{1.165927in}}%
\pgfpathlineto{\pgfqpoint{5.345380in}{1.164146in}}%
\pgfpathlineto{\pgfqpoint{5.346812in}{1.166091in}}%
\pgfpathlineto{\pgfqpoint{5.349675in}{1.175188in}}%
\pgfpathlineto{\pgfqpoint{5.353765in}{1.186365in}}%
\pgfpathlineto{\pgfqpoint{5.355196in}{1.186149in}}%
\pgfpathlineto{\pgfqpoint{5.356832in}{1.182337in}}%
\pgfpathlineto{\pgfqpoint{5.359491in}{1.169109in}}%
\pgfpathlineto{\pgfqpoint{5.366239in}{1.134170in}}%
\pgfpathlineto{\pgfqpoint{5.367671in}{1.133358in}}%
\pgfpathlineto{\pgfqpoint{5.367875in}{1.133494in}}%
\pgfpathlineto{\pgfqpoint{5.369307in}{1.136051in}}%
\pgfpathlineto{\pgfqpoint{5.372374in}{1.147861in}}%
\pgfpathlineto{\pgfqpoint{5.376260in}{1.160175in}}%
\pgfpathlineto{\pgfqpoint{5.377692in}{1.160045in}}%
\pgfpathlineto{\pgfqpoint{5.379328in}{1.155829in}}%
\pgfpathlineto{\pgfqpoint{5.381986in}{1.141009in}}%
\pgfpathlineto{\pgfqpoint{5.388530in}{1.102889in}}%
\pgfpathlineto{\pgfqpoint{5.389962in}{1.101871in}}%
\pgfpathlineto{\pgfqpoint{5.390166in}{1.102017in}}%
\pgfpathlineto{\pgfqpoint{5.391598in}{1.104934in}}%
\pgfpathlineto{\pgfqpoint{5.394461in}{1.117816in}}%
\pgfpathlineto{\pgfqpoint{5.398755in}{1.134812in}}%
\pgfpathlineto{\pgfqpoint{5.399982in}{1.135034in}}%
\pgfpathlineto{\pgfqpoint{5.400187in}{1.134796in}}%
\pgfpathlineto{\pgfqpoint{5.401618in}{1.130907in}}%
\pgfpathlineto{\pgfqpoint{5.404072in}{1.116245in}}%
\pgfpathlineto{\pgfqpoint{5.411230in}{1.069799in}}%
\pgfpathlineto{\pgfqpoint{5.412253in}{1.069306in}}%
\pgfpathlineto{\pgfqpoint{5.412662in}{1.069713in}}%
\pgfpathlineto{\pgfqpoint{5.414093in}{1.073655in}}%
\pgfpathlineto{\pgfqpoint{5.417161in}{1.091104in}}%
\pgfpathlineto{\pgfqpoint{5.421251in}{1.110768in}}%
\pgfpathlineto{\pgfqpoint{5.422273in}{1.111341in}}%
\pgfpathlineto{\pgfqpoint{5.422682in}{1.110914in}}%
\pgfpathlineto{\pgfqpoint{5.424114in}{1.106415in}}%
\pgfpathlineto{\pgfqpoint{5.426568in}{1.089154in}}%
\pgfpathlineto{\pgfqpoint{5.433521in}{1.035637in}}%
\pgfpathlineto{\pgfqpoint{5.434543in}{1.034950in}}%
\pgfpathlineto{\pgfqpoint{5.434748in}{1.035132in}}%
\pgfpathlineto{\pgfqpoint{5.435975in}{1.038359in}}%
\pgfpathlineto{\pgfqpoint{5.438429in}{1.053715in}}%
\pgfpathlineto{\pgfqpoint{5.443951in}{1.089424in}}%
\pgfpathlineto{\pgfqpoint{5.444769in}{1.089887in}}%
\pgfpathlineto{\pgfqpoint{5.445178in}{1.089412in}}%
\pgfpathlineto{\pgfqpoint{5.446609in}{1.083995in}}%
\pgfpathlineto{\pgfqpoint{5.449063in}{1.062788in}}%
\pgfpathlineto{\pgfqpoint{5.456016in}{0.997805in}}%
\pgfpathlineto{\pgfqpoint{5.456834in}{0.997286in}}%
\pgfpathlineto{\pgfqpoint{5.457243in}{0.997848in}}%
\pgfpathlineto{\pgfqpoint{5.458675in}{1.003935in}}%
\pgfpathlineto{\pgfqpoint{5.461333in}{1.028324in}}%
\pgfpathlineto{\pgfqpoint{5.466241in}{1.071844in}}%
\pgfpathlineto{\pgfqpoint{5.467264in}{1.073190in}}%
\pgfpathlineto{\pgfqpoint{5.467673in}{1.072635in}}%
\pgfpathlineto{\pgfqpoint{5.468900in}{1.067141in}}%
\pgfpathlineto{\pgfqpoint{5.471149in}{1.043533in}}%
\pgfpathlineto{\pgfqpoint{5.478716in}{0.952465in}}%
\pgfpathlineto{\pgfqpoint{5.479330in}{0.952694in}}%
\pgfpathlineto{\pgfqpoint{5.479534in}{0.953156in}}%
\pgfpathlineto{\pgfqpoint{5.480761in}{0.959869in}}%
\pgfpathlineto{\pgfqpoint{5.483215in}{0.990311in}}%
\pgfpathlineto{\pgfqpoint{5.489146in}{1.067482in}}%
\pgfpathlineto{\pgfqpoint{5.489759in}{1.068181in}}%
\pgfpathlineto{\pgfqpoint{5.490168in}{1.067466in}}%
\pgfpathlineto{\pgfqpoint{5.491395in}{1.059538in}}%
\pgfpathlineto{\pgfqpoint{5.493645in}{1.024258in}}%
\pgfpathlineto{\pgfqpoint{5.501211in}{0.886529in}}%
\pgfpathlineto{\pgfqpoint{5.501825in}{0.887730in}}%
\pgfpathlineto{\pgfqpoint{5.503052in}{0.898879in}}%
\pgfpathlineto{\pgfqpoint{5.505301in}{0.946484in}}%
\pgfpathlineto{\pgfqpoint{5.512050in}{1.107095in}}%
\pgfpathlineto{\pgfqpoint{5.512255in}{1.107213in}}%
\pgfpathlineto{\pgfqpoint{5.512459in}{1.106855in}}%
\pgfpathlineto{\pgfqpoint{5.513277in}{1.100527in}}%
\pgfpathlineto{\pgfqpoint{5.514913in}{1.063890in}}%
\pgfpathlineto{\pgfqpoint{5.517776in}{0.934154in}}%
\pgfpathlineto{\pgfqpoint{5.523093in}{0.698659in}}%
\pgfpathlineto{\pgfqpoint{5.523502in}{0.696000in}}%
\pgfpathlineto{\pgfqpoint{5.524116in}{0.699220in}}%
\pgfpathlineto{\pgfqpoint{5.525343in}{0.734735in}}%
\pgfpathlineto{\pgfqpoint{5.527183in}{0.870102in}}%
\pgfpathlineto{\pgfqpoint{5.530046in}{1.287088in}}%
\pgfpathlineto{\pgfqpoint{5.534545in}{2.323869in}}%
\pgfpathlineto{\pgfqpoint{5.534545in}{2.323869in}}%
\pgfusepath{stroke}%
\end{pgfscope}%
\begin{pgfscope}%
\pgfsetrectcap%
\pgfsetmiterjoin%
\pgfsetlinewidth{0.803000pt}%
\definecolor{currentstroke}{rgb}{0.000000,0.000000,0.000000}%
\pgfsetstrokecolor{currentstroke}%
\pgfsetdash{}{0pt}%
\pgfpathmoveto{\pgfqpoint{0.800000in}{0.528000in}}%
\pgfpathlineto{\pgfqpoint{0.800000in}{4.224000in}}%
\pgfusepath{stroke}%
\end{pgfscope}%
\begin{pgfscope}%
\pgfsetrectcap%
\pgfsetmiterjoin%
\pgfsetlinewidth{0.803000pt}%
\definecolor{currentstroke}{rgb}{0.000000,0.000000,0.000000}%
\pgfsetstrokecolor{currentstroke}%
\pgfsetdash{}{0pt}%
\pgfpathmoveto{\pgfqpoint{5.760000in}{0.528000in}}%
\pgfpathlineto{\pgfqpoint{5.760000in}{4.224000in}}%
\pgfusepath{stroke}%
\end{pgfscope}%
\begin{pgfscope}%
\pgfsetrectcap%
\pgfsetmiterjoin%
\pgfsetlinewidth{0.803000pt}%
\definecolor{currentstroke}{rgb}{0.000000,0.000000,0.000000}%
\pgfsetstrokecolor{currentstroke}%
\pgfsetdash{}{0pt}%
\pgfpathmoveto{\pgfqpoint{0.800000in}{0.528000in}}%
\pgfpathlineto{\pgfqpoint{5.760000in}{0.528000in}}%
\pgfusepath{stroke}%
\end{pgfscope}%
\begin{pgfscope}%
\pgfsetrectcap%
\pgfsetmiterjoin%
\pgfsetlinewidth{0.803000pt}%
\definecolor{currentstroke}{rgb}{0.000000,0.000000,0.000000}%
\pgfsetstrokecolor{currentstroke}%
\pgfsetdash{}{0pt}%
\pgfpathmoveto{\pgfqpoint{0.800000in}{4.224000in}}%
\pgfpathlineto{\pgfqpoint{5.760000in}{4.224000in}}%
\pgfusepath{stroke}%
\end{pgfscope}%
\end{pgfpicture}%
\makeatother%
\endgroup%
}
    \end{center}
    \caption{\emph{Onda a dente di sega} ottenuta sommando 100 sinusoidi armoniche, dividendo ciascuna armonica per il proprio indice.}
\end{figure}

\item L'onda quadra è composta da una somma di sinusoidi armoniche con termine di fase nullo. L'ampiezza di ciascuna armonica di ordine dispari è inversamente proporzionale al suo ordine; le armoniche di ordine pari hanno ampiezza nulla. Quindi, se la fondamentale (armonica 1) ha ampiezza 1, allora la terza armonica (di frequenza tripla) ha ampiezza $\frac{1}{3}$, la quinta armonica (di frequenza quintupla rispetto alla fondamentale) ha ampiezza $\frac{1}{5}$ e così via. Le armoniche di ordine 2, 4, 6 eccetera non sono presenti.

\begin{figure}
    \begin{center}
       \scalebox{0.6} {%% Creator: Matplotlib, PGF backend
%%
%% To include the figure in your LaTeX document, write
%%   \input{<filename>.pgf}
%%
%% Make sure the required packages are loaded in your preamble
%%   \usepackage{pgf}
%%
%% Also ensure that all the required font packages are loaded; for instance,
%% the lmodern package is sometimes necessary when using math font.
%%   \usepackage{lmodern}
%%
%% Figures using additional raster images can only be included by \input if
%% they are in the same directory as the main LaTeX file. For loading figures
%% from other directories you can use the `import` package
%%   \usepackage{import}
%%
%% and then include the figures with
%%   \import{<path to file>}{<filename>.pgf}
%%
%% Matplotlib used the following preamble
%%   
%%   \usepackage{fontspec}
%%   \setmainfont{DejaVuSerif.ttf}[Path=\detokenize{/opt/homebrew/Caskroom/miniconda/base/envs/label-studio/lib/python3.9/site-packages/matplotlib/mpl-data/fonts/ttf/}]
%%   \setsansfont{DejaVuSans.ttf}[Path=\detokenize{/opt/homebrew/Caskroom/miniconda/base/envs/label-studio/lib/python3.9/site-packages/matplotlib/mpl-data/fonts/ttf/}]
%%   \setmonofont{DejaVuSansMono.ttf}[Path=\detokenize{/opt/homebrew/Caskroom/miniconda/base/envs/label-studio/lib/python3.9/site-packages/matplotlib/mpl-data/fonts/ttf/}]
%%   \makeatletter\@ifpackageloaded{underscore}{}{\usepackage[strings]{underscore}}\makeatother
%%
\begingroup%
\makeatletter%
\begin{pgfpicture}%
\pgfpathrectangle{\pgfpointorigin}{\pgfqpoint{7.000000in}{4.000000in}}%
\pgfusepath{use as bounding box, clip}%
\begin{pgfscope}%
\pgfsetbuttcap%
\pgfsetmiterjoin%
\definecolor{currentfill}{rgb}{1.000000,1.000000,1.000000}%
\pgfsetfillcolor{currentfill}%
\pgfsetlinewidth{0.000000pt}%
\definecolor{currentstroke}{rgb}{1.000000,1.000000,1.000000}%
\pgfsetstrokecolor{currentstroke}%
\pgfsetdash{}{0pt}%
\pgfpathmoveto{\pgfqpoint{0.000000in}{0.000000in}}%
\pgfpathlineto{\pgfqpoint{7.000000in}{0.000000in}}%
\pgfpathlineto{\pgfqpoint{7.000000in}{4.000000in}}%
\pgfpathlineto{\pgfqpoint{0.000000in}{4.000000in}}%
\pgfpathlineto{\pgfqpoint{0.000000in}{0.000000in}}%
\pgfpathclose%
\pgfusepath{fill}%
\end{pgfscope}%
\begin{pgfscope}%
\pgfsetbuttcap%
\pgfsetmiterjoin%
\definecolor{currentfill}{rgb}{1.000000,1.000000,1.000000}%
\pgfsetfillcolor{currentfill}%
\pgfsetlinewidth{0.000000pt}%
\definecolor{currentstroke}{rgb}{0.000000,0.000000,0.000000}%
\pgfsetstrokecolor{currentstroke}%
\pgfsetstrokeopacity{0.000000}%
\pgfsetdash{}{0pt}%
\pgfpathmoveto{\pgfqpoint{0.467797in}{2.292089in}}%
\pgfpathlineto{\pgfqpoint{6.958330in}{2.292089in}}%
\pgfpathlineto{\pgfqpoint{6.958330in}{3.958330in}}%
\pgfpathlineto{\pgfqpoint{0.467797in}{3.958330in}}%
\pgfpathlineto{\pgfqpoint{0.467797in}{2.292089in}}%
\pgfpathclose%
\pgfusepath{fill}%
\end{pgfscope}%
\begin{pgfscope}%
\pgfsetbuttcap%
\pgfsetroundjoin%
\definecolor{currentfill}{rgb}{0.000000,0.000000,0.000000}%
\pgfsetfillcolor{currentfill}%
\pgfsetlinewidth{0.803000pt}%
\definecolor{currentstroke}{rgb}{0.000000,0.000000,0.000000}%
\pgfsetstrokecolor{currentstroke}%
\pgfsetdash{}{0pt}%
\pgfsys@defobject{currentmarker}{\pgfqpoint{0.000000in}{-0.048611in}}{\pgfqpoint{0.000000in}{0.000000in}}{%
\pgfpathmoveto{\pgfqpoint{0.000000in}{0.000000in}}%
\pgfpathlineto{\pgfqpoint{0.000000in}{-0.048611in}}%
\pgfusepath{stroke,fill}%
}%
\begin{pgfscope}%
\pgfsys@transformshift{0.762821in}{2.292089in}%
\pgfsys@useobject{currentmarker}{}%
\end{pgfscope}%
\end{pgfscope}%
\begin{pgfscope}%
\definecolor{textcolor}{rgb}{0.000000,0.000000,0.000000}%
\pgfsetstrokecolor{textcolor}%
\pgfsetfillcolor{textcolor}%
\pgftext[x=0.762821in,y=2.194867in,,top]{\color{textcolor}\sffamily\fontsize{10.000000}{12.000000}\selectfont 0}%
\end{pgfscope}%
\begin{pgfscope}%
\pgfsetbuttcap%
\pgfsetroundjoin%
\definecolor{currentfill}{rgb}{0.000000,0.000000,0.000000}%
\pgfsetfillcolor{currentfill}%
\pgfsetlinewidth{0.803000pt}%
\definecolor{currentstroke}{rgb}{0.000000,0.000000,0.000000}%
\pgfsetstrokecolor{currentstroke}%
\pgfsetdash{}{0pt}%
\pgfsys@defobject{currentmarker}{\pgfqpoint{0.000000in}{-0.048611in}}{\pgfqpoint{0.000000in}{0.000000in}}{%
\pgfpathmoveto{\pgfqpoint{0.000000in}{0.000000in}}%
\pgfpathlineto{\pgfqpoint{0.000000in}{-0.048611in}}%
\pgfusepath{stroke,fill}%
}%
\begin{pgfscope}%
\pgfsys@transformshift{2.100860in}{2.292089in}%
\pgfsys@useobject{currentmarker}{}%
\end{pgfscope}%
\end{pgfscope}%
\begin{pgfscope}%
\definecolor{textcolor}{rgb}{0.000000,0.000000,0.000000}%
\pgfsetstrokecolor{textcolor}%
\pgfsetfillcolor{textcolor}%
\pgftext[x=2.100860in,y=2.194867in,,top]{\color{textcolor}\sffamily\fontsize{10.000000}{12.000000}\selectfont 5000}%
\end{pgfscope}%
\begin{pgfscope}%
\pgfsetbuttcap%
\pgfsetroundjoin%
\definecolor{currentfill}{rgb}{0.000000,0.000000,0.000000}%
\pgfsetfillcolor{currentfill}%
\pgfsetlinewidth{0.803000pt}%
\definecolor{currentstroke}{rgb}{0.000000,0.000000,0.000000}%
\pgfsetstrokecolor{currentstroke}%
\pgfsetdash{}{0pt}%
\pgfsys@defobject{currentmarker}{\pgfqpoint{0.000000in}{-0.048611in}}{\pgfqpoint{0.000000in}{0.000000in}}{%
\pgfpathmoveto{\pgfqpoint{0.000000in}{0.000000in}}%
\pgfpathlineto{\pgfqpoint{0.000000in}{-0.048611in}}%
\pgfusepath{stroke,fill}%
}%
\begin{pgfscope}%
\pgfsys@transformshift{3.438899in}{2.292089in}%
\pgfsys@useobject{currentmarker}{}%
\end{pgfscope}%
\end{pgfscope}%
\begin{pgfscope}%
\definecolor{textcolor}{rgb}{0.000000,0.000000,0.000000}%
\pgfsetstrokecolor{textcolor}%
\pgfsetfillcolor{textcolor}%
\pgftext[x=3.438899in,y=2.194867in,,top]{\color{textcolor}\sffamily\fontsize{10.000000}{12.000000}\selectfont 10000}%
\end{pgfscope}%
\begin{pgfscope}%
\pgfsetbuttcap%
\pgfsetroundjoin%
\definecolor{currentfill}{rgb}{0.000000,0.000000,0.000000}%
\pgfsetfillcolor{currentfill}%
\pgfsetlinewidth{0.803000pt}%
\definecolor{currentstroke}{rgb}{0.000000,0.000000,0.000000}%
\pgfsetstrokecolor{currentstroke}%
\pgfsetdash{}{0pt}%
\pgfsys@defobject{currentmarker}{\pgfqpoint{0.000000in}{-0.048611in}}{\pgfqpoint{0.000000in}{0.000000in}}{%
\pgfpathmoveto{\pgfqpoint{0.000000in}{0.000000in}}%
\pgfpathlineto{\pgfqpoint{0.000000in}{-0.048611in}}%
\pgfusepath{stroke,fill}%
}%
\begin{pgfscope}%
\pgfsys@transformshift{4.776938in}{2.292089in}%
\pgfsys@useobject{currentmarker}{}%
\end{pgfscope}%
\end{pgfscope}%
\begin{pgfscope}%
\definecolor{textcolor}{rgb}{0.000000,0.000000,0.000000}%
\pgfsetstrokecolor{textcolor}%
\pgfsetfillcolor{textcolor}%
\pgftext[x=4.776938in,y=2.194867in,,top]{\color{textcolor}\sffamily\fontsize{10.000000}{12.000000}\selectfont 15000}%
\end{pgfscope}%
\begin{pgfscope}%
\pgfsetbuttcap%
\pgfsetroundjoin%
\definecolor{currentfill}{rgb}{0.000000,0.000000,0.000000}%
\pgfsetfillcolor{currentfill}%
\pgfsetlinewidth{0.803000pt}%
\definecolor{currentstroke}{rgb}{0.000000,0.000000,0.000000}%
\pgfsetstrokecolor{currentstroke}%
\pgfsetdash{}{0pt}%
\pgfsys@defobject{currentmarker}{\pgfqpoint{0.000000in}{-0.048611in}}{\pgfqpoint{0.000000in}{0.000000in}}{%
\pgfpathmoveto{\pgfqpoint{0.000000in}{0.000000in}}%
\pgfpathlineto{\pgfqpoint{0.000000in}{-0.048611in}}%
\pgfusepath{stroke,fill}%
}%
\begin{pgfscope}%
\pgfsys@transformshift{6.114977in}{2.292089in}%
\pgfsys@useobject{currentmarker}{}%
\end{pgfscope}%
\end{pgfscope}%
\begin{pgfscope}%
\definecolor{textcolor}{rgb}{0.000000,0.000000,0.000000}%
\pgfsetstrokecolor{textcolor}%
\pgfsetfillcolor{textcolor}%
\pgftext[x=6.114977in,y=2.194867in,,top]{\color{textcolor}\sffamily\fontsize{10.000000}{12.000000}\selectfont 20000}%
\end{pgfscope}%
\begin{pgfscope}%
\pgfsetbuttcap%
\pgfsetroundjoin%
\definecolor{currentfill}{rgb}{0.000000,0.000000,0.000000}%
\pgfsetfillcolor{currentfill}%
\pgfsetlinewidth{0.803000pt}%
\definecolor{currentstroke}{rgb}{0.000000,0.000000,0.000000}%
\pgfsetstrokecolor{currentstroke}%
\pgfsetdash{}{0pt}%
\pgfsys@defobject{currentmarker}{\pgfqpoint{-0.048611in}{0.000000in}}{\pgfqpoint{-0.000000in}{0.000000in}}{%
\pgfpathmoveto{\pgfqpoint{-0.000000in}{0.000000in}}%
\pgfpathlineto{\pgfqpoint{-0.048611in}{0.000000in}}%
\pgfusepath{stroke,fill}%
}%
\begin{pgfscope}%
\pgfsys@transformshift{0.467797in}{2.367827in}%
\pgfsys@useobject{currentmarker}{}%
\end{pgfscope}%
\end{pgfscope}%
\begin{pgfscope}%
\definecolor{textcolor}{rgb}{0.000000,0.000000,0.000000}%
\pgfsetstrokecolor{textcolor}%
\pgfsetfillcolor{textcolor}%
\pgftext[x=0.041670in, y=2.315066in, left, base]{\color{textcolor}\sffamily\fontsize{10.000000}{12.000000}\selectfont \ensuremath{-}1.0}%
\end{pgfscope}%
\begin{pgfscope}%
\pgfsetbuttcap%
\pgfsetroundjoin%
\definecolor{currentfill}{rgb}{0.000000,0.000000,0.000000}%
\pgfsetfillcolor{currentfill}%
\pgfsetlinewidth{0.803000pt}%
\definecolor{currentstroke}{rgb}{0.000000,0.000000,0.000000}%
\pgfsetstrokecolor{currentstroke}%
\pgfsetdash{}{0pt}%
\pgfsys@defobject{currentmarker}{\pgfqpoint{-0.048611in}{0.000000in}}{\pgfqpoint{-0.000000in}{0.000000in}}{%
\pgfpathmoveto{\pgfqpoint{-0.000000in}{0.000000in}}%
\pgfpathlineto{\pgfqpoint{-0.048611in}{0.000000in}}%
\pgfusepath{stroke,fill}%
}%
\begin{pgfscope}%
\pgfsys@transformshift{0.467797in}{2.746518in}%
\pgfsys@useobject{currentmarker}{}%
\end{pgfscope}%
\end{pgfscope}%
\begin{pgfscope}%
\definecolor{textcolor}{rgb}{0.000000,0.000000,0.000000}%
\pgfsetstrokecolor{textcolor}%
\pgfsetfillcolor{textcolor}%
\pgftext[x=0.041670in, y=2.693757in, left, base]{\color{textcolor}\sffamily\fontsize{10.000000}{12.000000}\selectfont \ensuremath{-}0.5}%
\end{pgfscope}%
\begin{pgfscope}%
\pgfsetbuttcap%
\pgfsetroundjoin%
\definecolor{currentfill}{rgb}{0.000000,0.000000,0.000000}%
\pgfsetfillcolor{currentfill}%
\pgfsetlinewidth{0.803000pt}%
\definecolor{currentstroke}{rgb}{0.000000,0.000000,0.000000}%
\pgfsetstrokecolor{currentstroke}%
\pgfsetdash{}{0pt}%
\pgfsys@defobject{currentmarker}{\pgfqpoint{-0.048611in}{0.000000in}}{\pgfqpoint{-0.000000in}{0.000000in}}{%
\pgfpathmoveto{\pgfqpoint{-0.000000in}{0.000000in}}%
\pgfpathlineto{\pgfqpoint{-0.048611in}{0.000000in}}%
\pgfusepath{stroke,fill}%
}%
\begin{pgfscope}%
\pgfsys@transformshift{0.467797in}{3.125209in}%
\pgfsys@useobject{currentmarker}{}%
\end{pgfscope}%
\end{pgfscope}%
\begin{pgfscope}%
\definecolor{textcolor}{rgb}{0.000000,0.000000,0.000000}%
\pgfsetstrokecolor{textcolor}%
\pgfsetfillcolor{textcolor}%
\pgftext[x=0.149695in, y=3.072448in, left, base]{\color{textcolor}\sffamily\fontsize{10.000000}{12.000000}\selectfont 0.0}%
\end{pgfscope}%
\begin{pgfscope}%
\pgfsetbuttcap%
\pgfsetroundjoin%
\definecolor{currentfill}{rgb}{0.000000,0.000000,0.000000}%
\pgfsetfillcolor{currentfill}%
\pgfsetlinewidth{0.803000pt}%
\definecolor{currentstroke}{rgb}{0.000000,0.000000,0.000000}%
\pgfsetstrokecolor{currentstroke}%
\pgfsetdash{}{0pt}%
\pgfsys@defobject{currentmarker}{\pgfqpoint{-0.048611in}{0.000000in}}{\pgfqpoint{-0.000000in}{0.000000in}}{%
\pgfpathmoveto{\pgfqpoint{-0.000000in}{0.000000in}}%
\pgfpathlineto{\pgfqpoint{-0.048611in}{0.000000in}}%
\pgfusepath{stroke,fill}%
}%
\begin{pgfscope}%
\pgfsys@transformshift{0.467797in}{3.503901in}%
\pgfsys@useobject{currentmarker}{}%
\end{pgfscope}%
\end{pgfscope}%
\begin{pgfscope}%
\definecolor{textcolor}{rgb}{0.000000,0.000000,0.000000}%
\pgfsetstrokecolor{textcolor}%
\pgfsetfillcolor{textcolor}%
\pgftext[x=0.149695in, y=3.451139in, left, base]{\color{textcolor}\sffamily\fontsize{10.000000}{12.000000}\selectfont 0.5}%
\end{pgfscope}%
\begin{pgfscope}%
\pgfsetbuttcap%
\pgfsetroundjoin%
\definecolor{currentfill}{rgb}{0.000000,0.000000,0.000000}%
\pgfsetfillcolor{currentfill}%
\pgfsetlinewidth{0.803000pt}%
\definecolor{currentstroke}{rgb}{0.000000,0.000000,0.000000}%
\pgfsetstrokecolor{currentstroke}%
\pgfsetdash{}{0pt}%
\pgfsys@defobject{currentmarker}{\pgfqpoint{-0.048611in}{0.000000in}}{\pgfqpoint{-0.000000in}{0.000000in}}{%
\pgfpathmoveto{\pgfqpoint{-0.000000in}{0.000000in}}%
\pgfpathlineto{\pgfqpoint{-0.048611in}{0.000000in}}%
\pgfusepath{stroke,fill}%
}%
\begin{pgfscope}%
\pgfsys@transformshift{0.467797in}{3.882592in}%
\pgfsys@useobject{currentmarker}{}%
\end{pgfscope}%
\end{pgfscope}%
\begin{pgfscope}%
\definecolor{textcolor}{rgb}{0.000000,0.000000,0.000000}%
\pgfsetstrokecolor{textcolor}%
\pgfsetfillcolor{textcolor}%
\pgftext[x=0.149695in, y=3.829830in, left, base]{\color{textcolor}\sffamily\fontsize{10.000000}{12.000000}\selectfont 1.0}%
\end{pgfscope}%
\begin{pgfscope}%
\pgfpathrectangle{\pgfqpoint{0.467797in}{2.292089in}}{\pgfqpoint{6.490533in}{1.666241in}}%
\pgfusepath{clip}%
\pgfsetrectcap%
\pgfsetroundjoin%
\pgfsetlinewidth{1.505625pt}%
\definecolor{currentstroke}{rgb}{0.121569,0.466667,0.705882}%
\pgfsetstrokecolor{currentstroke}%
\pgfsetdash{}{0pt}%
\pgfpathmoveto{\pgfqpoint{0.762821in}{3.125209in}}%
\pgfpathlineto{\pgfqpoint{0.882977in}{3.316905in}}%
\pgfpathlineto{\pgfqpoint{0.946667in}{3.414224in}}%
\pgfpathlineto{\pgfqpoint{0.999386in}{3.490837in}}%
\pgfpathlineto{\pgfqpoint{1.046218in}{3.555062in}}%
\pgfpathlineto{\pgfqpoint{1.088767in}{3.609727in}}%
\pgfpathlineto{\pgfqpoint{1.128373in}{3.657046in}}%
\pgfpathlineto{\pgfqpoint{1.165303in}{3.697768in}}%
\pgfpathlineto{\pgfqpoint{1.200092in}{3.732895in}}%
\pgfpathlineto{\pgfqpoint{1.233275in}{3.763297in}}%
\pgfpathlineto{\pgfqpoint{1.264853in}{3.789272in}}%
\pgfpathlineto{\pgfqpoint{1.295360in}{3.811517in}}%
\pgfpathlineto{\pgfqpoint{1.324530in}{3.830079in}}%
\pgfpathlineto{\pgfqpoint{1.352896in}{3.845523in}}%
\pgfpathlineto{\pgfqpoint{1.380460in}{3.858013in}}%
\pgfpathlineto{\pgfqpoint{1.407221in}{3.867724in}}%
\pgfpathlineto{\pgfqpoint{1.433446in}{3.874902in}}%
\pgfpathlineto{\pgfqpoint{1.459404in}{3.879705in}}%
\pgfpathlineto{\pgfqpoint{1.485094in}{3.882189in}}%
\pgfpathlineto{\pgfqpoint{1.510517in}{3.882417in}}%
\pgfpathlineto{\pgfqpoint{1.535940in}{3.880425in}}%
\pgfpathlineto{\pgfqpoint{1.561630in}{3.876165in}}%
\pgfpathlineto{\pgfqpoint{1.587588in}{3.869578in}}%
\pgfpathlineto{\pgfqpoint{1.613814in}{3.860613in}}%
\pgfpathlineto{\pgfqpoint{1.640575in}{3.849102in}}%
\pgfpathlineto{\pgfqpoint{1.668138in}{3.834789in}}%
\pgfpathlineto{\pgfqpoint{1.696505in}{3.817507in}}%
\pgfpathlineto{\pgfqpoint{1.725674in}{3.797103in}}%
\pgfpathlineto{\pgfqpoint{1.755914in}{3.773215in}}%
\pgfpathlineto{\pgfqpoint{1.787491in}{3.745404in}}%
\pgfpathlineto{\pgfqpoint{1.820675in}{3.713160in}}%
\pgfpathlineto{\pgfqpoint{1.855464in}{3.676208in}}%
\pgfpathlineto{\pgfqpoint{1.892394in}{3.633679in}}%
\pgfpathlineto{\pgfqpoint{1.931732in}{3.584925in}}%
\pgfpathlineto{\pgfqpoint{1.974282in}{3.528572in}}%
\pgfpathlineto{\pgfqpoint{2.020845in}{3.463128in}}%
\pgfpathlineto{\pgfqpoint{2.073296in}{3.385466in}}%
\pgfpathlineto{\pgfqpoint{2.135649in}{3.289005in}}%
\pgfpathlineto{\pgfqpoint{2.224227in}{3.147435in}}%
\pgfpathlineto{\pgfqpoint{2.359637in}{2.931218in}}%
\pgfpathlineto{\pgfqpoint{2.422792in}{2.834799in}}%
\pgfpathlineto{\pgfqpoint{2.475511in}{2.758259in}}%
\pgfpathlineto{\pgfqpoint{2.522075in}{2.694469in}}%
\pgfpathlineto{\pgfqpoint{2.564624in}{2.639863in}}%
\pgfpathlineto{\pgfqpoint{2.603963in}{2.592912in}}%
\pgfpathlineto{\pgfqpoint{2.640893in}{2.552227in}}%
\pgfpathlineto{\pgfqpoint{2.675682in}{2.517137in}}%
\pgfpathlineto{\pgfqpoint{2.708865in}{2.486773in}}%
\pgfpathlineto{\pgfqpoint{2.740443in}{2.460836in}}%
\pgfpathlineto{\pgfqpoint{2.770682in}{2.438811in}}%
\pgfpathlineto{\pgfqpoint{2.799852in}{2.420261in}}%
\pgfpathlineto{\pgfqpoint{2.828218in}{2.404829in}}%
\pgfpathlineto{\pgfqpoint{2.855782in}{2.392352in}}%
\pgfpathlineto{\pgfqpoint{2.882542in}{2.382652in}}%
\pgfpathlineto{\pgfqpoint{2.908768in}{2.375486in}}%
\pgfpathlineto{\pgfqpoint{2.934726in}{2.370695in}}%
\pgfpathlineto{\pgfqpoint{2.960416in}{2.368223in}}%
\pgfpathlineto{\pgfqpoint{2.985839in}{2.368007in}}%
\pgfpathlineto{\pgfqpoint{3.011262in}{2.370010in}}%
\pgfpathlineto{\pgfqpoint{3.036952in}{2.374282in}}%
\pgfpathlineto{\pgfqpoint{3.062910in}{2.380881in}}%
\pgfpathlineto{\pgfqpoint{3.089136in}{2.389857in}}%
\pgfpathlineto{\pgfqpoint{3.115896in}{2.401380in}}%
\pgfpathlineto{\pgfqpoint{3.143460in}{2.415705in}}%
\pgfpathlineto{\pgfqpoint{3.171827in}{2.432999in}}%
\pgfpathlineto{\pgfqpoint{3.200996in}{2.453416in}}%
\pgfpathlineto{\pgfqpoint{3.231235in}{2.477316in}}%
\pgfpathlineto{\pgfqpoint{3.262813in}{2.505139in}}%
\pgfpathlineto{\pgfqpoint{3.295997in}{2.537395in}}%
\pgfpathlineto{\pgfqpoint{3.330786in}{2.574359in}}%
\pgfpathlineto{\pgfqpoint{3.367715in}{2.616900in}}%
\pgfpathlineto{\pgfqpoint{3.407054in}{2.665665in}}%
\pgfpathlineto{\pgfqpoint{3.449603in}{2.722029in}}%
\pgfpathlineto{\pgfqpoint{3.496167in}{2.787484in}}%
\pgfpathlineto{\pgfqpoint{3.548618in}{2.865156in}}%
\pgfpathlineto{\pgfqpoint{3.610971in}{2.961624in}}%
\pgfpathlineto{\pgfqpoint{3.699549in}{3.103199in}}%
\pgfpathlineto{\pgfqpoint{3.834959in}{3.319410in}}%
\pgfpathlineto{\pgfqpoint{3.898114in}{3.415820in}}%
\pgfpathlineto{\pgfqpoint{3.950833in}{3.492348in}}%
\pgfpathlineto{\pgfqpoint{3.997397in}{3.556128in}}%
\pgfpathlineto{\pgfqpoint{4.039946in}{3.610722in}}%
\pgfpathlineto{\pgfqpoint{4.079285in}{3.657660in}}%
\pgfpathlineto{\pgfqpoint{4.116215in}{3.698333in}}%
\pgfpathlineto{\pgfqpoint{4.151004in}{3.733410in}}%
\pgfpathlineto{\pgfqpoint{4.184187in}{3.763762in}}%
\pgfpathlineto{\pgfqpoint{4.215765in}{3.789687in}}%
\pgfpathlineto{\pgfqpoint{4.246004in}{3.811699in}}%
\pgfpathlineto{\pgfqpoint{4.275174in}{3.830237in}}%
\pgfpathlineto{\pgfqpoint{4.303540in}{3.845656in}}%
\pgfpathlineto{\pgfqpoint{4.331104in}{3.858122in}}%
\pgfpathlineto{\pgfqpoint{4.357864in}{3.867809in}}%
\pgfpathlineto{\pgfqpoint{4.384090in}{3.874963in}}%
\pgfpathlineto{\pgfqpoint{4.410048in}{3.879743in}}%
\pgfpathlineto{\pgfqpoint{4.435738in}{3.882203in}}%
\pgfpathlineto{\pgfqpoint{4.461161in}{3.882407in}}%
\pgfpathlineto{\pgfqpoint{4.486584in}{3.880393in}}%
\pgfpathlineto{\pgfqpoint{4.512274in}{3.876109in}}%
\pgfpathlineto{\pgfqpoint{4.538232in}{3.869498in}}%
\pgfpathlineto{\pgfqpoint{4.564458in}{3.860510in}}%
\pgfpathlineto{\pgfqpoint{4.591218in}{3.848975in}}%
\pgfpathlineto{\pgfqpoint{4.618782in}{3.834638in}}%
\pgfpathlineto{\pgfqpoint{4.647148in}{3.817332in}}%
\pgfpathlineto{\pgfqpoint{4.676318in}{3.796904in}}%
\pgfpathlineto{\pgfqpoint{4.706557in}{3.772991in}}%
\pgfpathlineto{\pgfqpoint{4.738135in}{3.745156in}}%
\pgfpathlineto{\pgfqpoint{4.771318in}{3.712888in}}%
\pgfpathlineto{\pgfqpoint{4.806107in}{3.675912in}}%
\pgfpathlineto{\pgfqpoint{4.843037in}{3.633359in}}%
\pgfpathlineto{\pgfqpoint{4.882376in}{3.584582in}}%
\pgfpathlineto{\pgfqpoint{4.924925in}{3.528207in}}%
\pgfpathlineto{\pgfqpoint{4.971489in}{3.462741in}}%
\pgfpathlineto{\pgfqpoint{5.023940in}{3.385060in}}%
\pgfpathlineto{\pgfqpoint{5.086560in}{3.288162in}}%
\pgfpathlineto{\pgfqpoint{5.175406in}{3.146141in}}%
\pgfpathlineto{\pgfqpoint{5.310013in}{2.931218in}}%
\pgfpathlineto{\pgfqpoint{5.373168in}{2.834799in}}%
\pgfpathlineto{\pgfqpoint{5.425887in}{2.758259in}}%
\pgfpathlineto{\pgfqpoint{5.472451in}{2.694469in}}%
\pgfpathlineto{\pgfqpoint{5.515001in}{2.639863in}}%
\pgfpathlineto{\pgfqpoint{5.554339in}{2.592912in}}%
\pgfpathlineto{\pgfqpoint{5.591269in}{2.552227in}}%
\pgfpathlineto{\pgfqpoint{5.626058in}{2.517137in}}%
\pgfpathlineto{\pgfqpoint{5.659241in}{2.486773in}}%
\pgfpathlineto{\pgfqpoint{5.690819in}{2.460836in}}%
\pgfpathlineto{\pgfqpoint{5.721059in}{2.438811in}}%
\pgfpathlineto{\pgfqpoint{5.750228in}{2.420261in}}%
\pgfpathlineto{\pgfqpoint{5.778594in}{2.404829in}}%
\pgfpathlineto{\pgfqpoint{5.806158in}{2.392352in}}%
\pgfpathlineto{\pgfqpoint{5.832919in}{2.382652in}}%
\pgfpathlineto{\pgfqpoint{5.859144in}{2.375486in}}%
\pgfpathlineto{\pgfqpoint{5.885102in}{2.370695in}}%
\pgfpathlineto{\pgfqpoint{5.910793in}{2.368223in}}%
\pgfpathlineto{\pgfqpoint{5.936215in}{2.368007in}}%
\pgfpathlineto{\pgfqpoint{5.961638in}{2.370010in}}%
\pgfpathlineto{\pgfqpoint{5.987328in}{2.374282in}}%
\pgfpathlineto{\pgfqpoint{6.013286in}{2.380881in}}%
\pgfpathlineto{\pgfqpoint{6.039512in}{2.389857in}}%
\pgfpathlineto{\pgfqpoint{6.066273in}{2.401380in}}%
\pgfpathlineto{\pgfqpoint{6.093836in}{2.415705in}}%
\pgfpathlineto{\pgfqpoint{6.122203in}{2.432999in}}%
\pgfpathlineto{\pgfqpoint{6.151372in}{2.453416in}}%
\pgfpathlineto{\pgfqpoint{6.181612in}{2.477316in}}%
\pgfpathlineto{\pgfqpoint{6.213189in}{2.505139in}}%
\pgfpathlineto{\pgfqpoint{6.246373in}{2.537395in}}%
\pgfpathlineto{\pgfqpoint{6.281162in}{2.574359in}}%
\pgfpathlineto{\pgfqpoint{6.318092in}{2.616900in}}%
\pgfpathlineto{\pgfqpoint{6.357430in}{2.665665in}}%
\pgfpathlineto{\pgfqpoint{6.399980in}{2.722029in}}%
\pgfpathlineto{\pgfqpoint{6.446543in}{2.787484in}}%
\pgfpathlineto{\pgfqpoint{6.498995in}{2.865156in}}%
\pgfpathlineto{\pgfqpoint{6.561347in}{2.961624in}}%
\pgfpathlineto{\pgfqpoint{6.649925in}{3.103199in}}%
\pgfpathlineto{\pgfqpoint{6.663306in}{3.124778in}}%
\pgfpathlineto{\pgfqpoint{6.663306in}{3.124778in}}%
\pgfusepath{stroke}%
\end{pgfscope}%
\begin{pgfscope}%
\pgfpathrectangle{\pgfqpoint{0.467797in}{2.292089in}}{\pgfqpoint{6.490533in}{1.666241in}}%
\pgfusepath{clip}%
\pgfsetrectcap%
\pgfsetroundjoin%
\pgfsetlinewidth{1.505625pt}%
\definecolor{currentstroke}{rgb}{1.000000,0.498039,0.054902}%
\pgfsetstrokecolor{currentstroke}%
\pgfsetdash{}{0pt}%
\pgfpathmoveto{\pgfqpoint{0.762821in}{3.125209in}}%
\pgfpathlineto{\pgfqpoint{0.820624in}{3.216338in}}%
\pgfpathlineto{\pgfqpoint{0.851131in}{3.260211in}}%
\pgfpathlineto{\pgfqpoint{0.876287in}{3.292609in}}%
\pgfpathlineto{\pgfqpoint{0.898230in}{3.317375in}}%
\pgfpathlineto{\pgfqpoint{0.918033in}{3.336499in}}%
\pgfpathlineto{\pgfqpoint{0.936231in}{3.351101in}}%
\pgfpathlineto{\pgfqpoint{0.953358in}{3.362061in}}%
\pgfpathlineto{\pgfqpoint{0.969414in}{3.369766in}}%
\pgfpathlineto{\pgfqpoint{0.984935in}{3.374769in}}%
\pgfpathlineto{\pgfqpoint{0.999921in}{3.377275in}}%
\pgfpathlineto{\pgfqpoint{1.014640in}{3.377488in}}%
\pgfpathlineto{\pgfqpoint{1.029358in}{3.375471in}}%
\pgfpathlineto{\pgfqpoint{1.044344in}{3.371147in}}%
\pgfpathlineto{\pgfqpoint{1.059866in}{3.364294in}}%
\pgfpathlineto{\pgfqpoint{1.075922in}{3.354733in}}%
\pgfpathlineto{\pgfqpoint{1.093049in}{3.341879in}}%
\pgfpathlineto{\pgfqpoint{1.111246in}{3.325385in}}%
\pgfpathlineto{\pgfqpoint{1.131049in}{3.304374in}}%
\pgfpathlineto{\pgfqpoint{1.152993in}{3.277761in}}%
\pgfpathlineto{\pgfqpoint{1.178148in}{3.243606in}}%
\pgfpathlineto{\pgfqpoint{1.208388in}{3.198592in}}%
\pgfpathlineto{\pgfqpoint{1.253346in}{3.127152in}}%
\pgfpathlineto{\pgfqpoint{1.312220in}{3.034282in}}%
\pgfpathlineto{\pgfqpoint{1.342727in}{2.990390in}}%
\pgfpathlineto{\pgfqpoint{1.367882in}{2.957971in}}%
\pgfpathlineto{\pgfqpoint{1.389826in}{2.933184in}}%
\pgfpathlineto{\pgfqpoint{1.409629in}{2.914038in}}%
\pgfpathlineto{\pgfqpoint{1.427826in}{2.899414in}}%
\pgfpathlineto{\pgfqpoint{1.444953in}{2.888432in}}%
\pgfpathlineto{\pgfqpoint{1.461010in}{2.880707in}}%
\pgfpathlineto{\pgfqpoint{1.476531in}{2.875682in}}%
\pgfpathlineto{\pgfqpoint{1.491517in}{2.873156in}}%
\pgfpathlineto{\pgfqpoint{1.506235in}{2.872923in}}%
\pgfpathlineto{\pgfqpoint{1.520954in}{2.874919in}}%
\pgfpathlineto{\pgfqpoint{1.535940in}{2.879223in}}%
\pgfpathlineto{\pgfqpoint{1.551461in}{2.886056in}}%
\pgfpathlineto{\pgfqpoint{1.567518in}{2.895596in}}%
\pgfpathlineto{\pgfqpoint{1.584645in}{2.908429in}}%
\pgfpathlineto{\pgfqpoint{1.602842in}{2.924902in}}%
\pgfpathlineto{\pgfqpoint{1.622645in}{2.945893in}}%
\pgfpathlineto{\pgfqpoint{1.644589in}{2.972486in}}%
\pgfpathlineto{\pgfqpoint{1.669744in}{3.006623in}}%
\pgfpathlineto{\pgfqpoint{1.699983in}{3.051621in}}%
\pgfpathlineto{\pgfqpoint{1.744674in}{3.122620in}}%
\pgfpathlineto{\pgfqpoint{1.804083in}{3.216338in}}%
\pgfpathlineto{\pgfqpoint{1.834590in}{3.260211in}}%
\pgfpathlineto{\pgfqpoint{1.859745in}{3.292609in}}%
\pgfpathlineto{\pgfqpoint{1.881689in}{3.317375in}}%
\pgfpathlineto{\pgfqpoint{1.901492in}{3.336499in}}%
\pgfpathlineto{\pgfqpoint{1.919690in}{3.351101in}}%
\pgfpathlineto{\pgfqpoint{1.936816in}{3.362061in}}%
\pgfpathlineto{\pgfqpoint{1.952873in}{3.369766in}}%
\pgfpathlineto{\pgfqpoint{1.968394in}{3.374769in}}%
\pgfpathlineto{\pgfqpoint{1.983380in}{3.377275in}}%
\pgfpathlineto{\pgfqpoint{1.998099in}{3.377488in}}%
\pgfpathlineto{\pgfqpoint{2.012817in}{3.375471in}}%
\pgfpathlineto{\pgfqpoint{2.027803in}{3.371147in}}%
\pgfpathlineto{\pgfqpoint{2.043324in}{3.364294in}}%
\pgfpathlineto{\pgfqpoint{2.059381in}{3.354733in}}%
\pgfpathlineto{\pgfqpoint{2.076508in}{3.341879in}}%
\pgfpathlineto{\pgfqpoint{2.094705in}{3.325385in}}%
\pgfpathlineto{\pgfqpoint{2.114508in}{3.304374in}}%
\pgfpathlineto{\pgfqpoint{2.136452in}{3.277761in}}%
\pgfpathlineto{\pgfqpoint{2.161607in}{3.243606in}}%
\pgfpathlineto{\pgfqpoint{2.191847in}{3.198592in}}%
\pgfpathlineto{\pgfqpoint{2.236805in}{3.127152in}}%
\pgfpathlineto{\pgfqpoint{2.295679in}{3.034282in}}%
\pgfpathlineto{\pgfqpoint{2.326186in}{2.990390in}}%
\pgfpathlineto{\pgfqpoint{2.351341in}{2.957971in}}%
\pgfpathlineto{\pgfqpoint{2.373285in}{2.933184in}}%
\pgfpathlineto{\pgfqpoint{2.393088in}{2.914038in}}%
\pgfpathlineto{\pgfqpoint{2.411285in}{2.899414in}}%
\pgfpathlineto{\pgfqpoint{2.428412in}{2.888432in}}%
\pgfpathlineto{\pgfqpoint{2.444468in}{2.880707in}}%
\pgfpathlineto{\pgfqpoint{2.459990in}{2.875682in}}%
\pgfpathlineto{\pgfqpoint{2.474976in}{2.873156in}}%
\pgfpathlineto{\pgfqpoint{2.489694in}{2.872923in}}%
\pgfpathlineto{\pgfqpoint{2.504413in}{2.874919in}}%
\pgfpathlineto{\pgfqpoint{2.519399in}{2.879223in}}%
\pgfpathlineto{\pgfqpoint{2.534920in}{2.886056in}}%
\pgfpathlineto{\pgfqpoint{2.550976in}{2.895596in}}%
\pgfpathlineto{\pgfqpoint{2.568103in}{2.908429in}}%
\pgfpathlineto{\pgfqpoint{2.586301in}{2.924902in}}%
\pgfpathlineto{\pgfqpoint{2.606104in}{2.945893in}}%
\pgfpathlineto{\pgfqpoint{2.628047in}{2.972486in}}%
\pgfpathlineto{\pgfqpoint{2.653203in}{3.006623in}}%
\pgfpathlineto{\pgfqpoint{2.683442in}{3.051621in}}%
\pgfpathlineto{\pgfqpoint{2.728133in}{3.122620in}}%
\pgfpathlineto{\pgfqpoint{2.787542in}{3.216338in}}%
\pgfpathlineto{\pgfqpoint{2.818049in}{3.260211in}}%
\pgfpathlineto{\pgfqpoint{2.843204in}{3.292609in}}%
\pgfpathlineto{\pgfqpoint{2.865148in}{3.317375in}}%
\pgfpathlineto{\pgfqpoint{2.884951in}{3.336499in}}%
\pgfpathlineto{\pgfqpoint{2.903148in}{3.351101in}}%
\pgfpathlineto{\pgfqpoint{2.920275in}{3.362061in}}%
\pgfpathlineto{\pgfqpoint{2.936332in}{3.369766in}}%
\pgfpathlineto{\pgfqpoint{2.951853in}{3.374769in}}%
\pgfpathlineto{\pgfqpoint{2.966839in}{3.377275in}}%
\pgfpathlineto{\pgfqpoint{2.981557in}{3.377488in}}%
\pgfpathlineto{\pgfqpoint{2.996276in}{3.375471in}}%
\pgfpathlineto{\pgfqpoint{3.011262in}{3.371147in}}%
\pgfpathlineto{\pgfqpoint{3.026783in}{3.364294in}}%
\pgfpathlineto{\pgfqpoint{3.042840in}{3.354733in}}%
\pgfpathlineto{\pgfqpoint{3.059966in}{3.341879in}}%
\pgfpathlineto{\pgfqpoint{3.078164in}{3.325385in}}%
\pgfpathlineto{\pgfqpoint{3.097967in}{3.304374in}}%
\pgfpathlineto{\pgfqpoint{3.119911in}{3.277761in}}%
\pgfpathlineto{\pgfqpoint{3.145066in}{3.243606in}}%
\pgfpathlineto{\pgfqpoint{3.175305in}{3.198592in}}%
\pgfpathlineto{\pgfqpoint{3.220264in}{3.127152in}}%
\pgfpathlineto{\pgfqpoint{3.279137in}{3.034282in}}%
\pgfpathlineto{\pgfqpoint{3.309645in}{2.990390in}}%
\pgfpathlineto{\pgfqpoint{3.334800in}{2.957971in}}%
\pgfpathlineto{\pgfqpoint{3.356744in}{2.933184in}}%
\pgfpathlineto{\pgfqpoint{3.376546in}{2.914038in}}%
\pgfpathlineto{\pgfqpoint{3.394744in}{2.899414in}}%
\pgfpathlineto{\pgfqpoint{3.411871in}{2.888432in}}%
\pgfpathlineto{\pgfqpoint{3.427927in}{2.880707in}}%
\pgfpathlineto{\pgfqpoint{3.443448in}{2.875682in}}%
\pgfpathlineto{\pgfqpoint{3.458434in}{2.873156in}}%
\pgfpathlineto{\pgfqpoint{3.473153in}{2.872923in}}%
\pgfpathlineto{\pgfqpoint{3.487871in}{2.874919in}}%
\pgfpathlineto{\pgfqpoint{3.502857in}{2.879223in}}%
\pgfpathlineto{\pgfqpoint{3.518379in}{2.886056in}}%
\pgfpathlineto{\pgfqpoint{3.534435in}{2.895596in}}%
\pgfpathlineto{\pgfqpoint{3.551562in}{2.908429in}}%
\pgfpathlineto{\pgfqpoint{3.569759in}{2.924902in}}%
\pgfpathlineto{\pgfqpoint{3.589562in}{2.945893in}}%
\pgfpathlineto{\pgfqpoint{3.611506in}{2.972486in}}%
\pgfpathlineto{\pgfqpoint{3.636661in}{3.006623in}}%
\pgfpathlineto{\pgfqpoint{3.666901in}{3.051621in}}%
\pgfpathlineto{\pgfqpoint{3.711591in}{3.122620in}}%
\pgfpathlineto{\pgfqpoint{3.771000in}{3.216338in}}%
\pgfpathlineto{\pgfqpoint{3.801508in}{3.260211in}}%
\pgfpathlineto{\pgfqpoint{3.826663in}{3.292609in}}%
\pgfpathlineto{\pgfqpoint{3.848607in}{3.317375in}}%
\pgfpathlineto{\pgfqpoint{3.868410in}{3.336499in}}%
\pgfpathlineto{\pgfqpoint{3.886607in}{3.351101in}}%
\pgfpathlineto{\pgfqpoint{3.903734in}{3.362061in}}%
\pgfpathlineto{\pgfqpoint{3.919790in}{3.369766in}}%
\pgfpathlineto{\pgfqpoint{3.935312in}{3.374769in}}%
\pgfpathlineto{\pgfqpoint{3.950298in}{3.377275in}}%
\pgfpathlineto{\pgfqpoint{3.965016in}{3.377488in}}%
\pgfpathlineto{\pgfqpoint{3.979735in}{3.375471in}}%
\pgfpathlineto{\pgfqpoint{3.994721in}{3.371147in}}%
\pgfpathlineto{\pgfqpoint{4.010242in}{3.364294in}}%
\pgfpathlineto{\pgfqpoint{4.026298in}{3.354733in}}%
\pgfpathlineto{\pgfqpoint{4.043425in}{3.341879in}}%
\pgfpathlineto{\pgfqpoint{4.061623in}{3.325385in}}%
\pgfpathlineto{\pgfqpoint{4.081425in}{3.304374in}}%
\pgfpathlineto{\pgfqpoint{4.103369in}{3.277761in}}%
\pgfpathlineto{\pgfqpoint{4.128524in}{3.243606in}}%
\pgfpathlineto{\pgfqpoint{4.158764in}{3.198592in}}%
\pgfpathlineto{\pgfqpoint{4.203722in}{3.127152in}}%
\pgfpathlineto{\pgfqpoint{4.262596in}{3.034282in}}%
\pgfpathlineto{\pgfqpoint{4.293103in}{2.990390in}}%
\pgfpathlineto{\pgfqpoint{4.318258in}{2.957971in}}%
\pgfpathlineto{\pgfqpoint{4.340202in}{2.933184in}}%
\pgfpathlineto{\pgfqpoint{4.360005in}{2.914038in}}%
\pgfpathlineto{\pgfqpoint{4.378203in}{2.899414in}}%
\pgfpathlineto{\pgfqpoint{4.395329in}{2.888432in}}%
\pgfpathlineto{\pgfqpoint{4.411386in}{2.880707in}}%
\pgfpathlineto{\pgfqpoint{4.426907in}{2.875682in}}%
\pgfpathlineto{\pgfqpoint{4.441893in}{2.873156in}}%
\pgfpathlineto{\pgfqpoint{4.456612in}{2.872923in}}%
\pgfpathlineto{\pgfqpoint{4.471330in}{2.874919in}}%
\pgfpathlineto{\pgfqpoint{4.486316in}{2.879223in}}%
\pgfpathlineto{\pgfqpoint{4.501837in}{2.886056in}}%
\pgfpathlineto{\pgfqpoint{4.517894in}{2.895596in}}%
\pgfpathlineto{\pgfqpoint{4.535021in}{2.908429in}}%
\pgfpathlineto{\pgfqpoint{4.553218in}{2.924902in}}%
\pgfpathlineto{\pgfqpoint{4.573021in}{2.945893in}}%
\pgfpathlineto{\pgfqpoint{4.594965in}{2.972486in}}%
\pgfpathlineto{\pgfqpoint{4.620120in}{3.006623in}}%
\pgfpathlineto{\pgfqpoint{4.650360in}{3.051621in}}%
\pgfpathlineto{\pgfqpoint{4.695050in}{3.122620in}}%
\pgfpathlineto{\pgfqpoint{4.754459in}{3.216338in}}%
\pgfpathlineto{\pgfqpoint{4.784966in}{3.260211in}}%
\pgfpathlineto{\pgfqpoint{4.810122in}{3.292609in}}%
\pgfpathlineto{\pgfqpoint{4.832065in}{3.317375in}}%
\pgfpathlineto{\pgfqpoint{4.851868in}{3.336499in}}%
\pgfpathlineto{\pgfqpoint{4.870066in}{3.351101in}}%
\pgfpathlineto{\pgfqpoint{4.887193in}{3.362061in}}%
\pgfpathlineto{\pgfqpoint{4.903249in}{3.369766in}}%
\pgfpathlineto{\pgfqpoint{4.918770in}{3.374769in}}%
\pgfpathlineto{\pgfqpoint{4.933756in}{3.377275in}}%
\pgfpathlineto{\pgfqpoint{4.948475in}{3.377488in}}%
\pgfpathlineto{\pgfqpoint{4.963193in}{3.375471in}}%
\pgfpathlineto{\pgfqpoint{4.978179in}{3.371147in}}%
\pgfpathlineto{\pgfqpoint{4.993701in}{3.364294in}}%
\pgfpathlineto{\pgfqpoint{5.009757in}{3.354733in}}%
\pgfpathlineto{\pgfqpoint{5.026884in}{3.341879in}}%
\pgfpathlineto{\pgfqpoint{5.045081in}{3.325385in}}%
\pgfpathlineto{\pgfqpoint{5.064884in}{3.304374in}}%
\pgfpathlineto{\pgfqpoint{5.086828in}{3.277761in}}%
\pgfpathlineto{\pgfqpoint{5.111983in}{3.243606in}}%
\pgfpathlineto{\pgfqpoint{5.142223in}{3.198592in}}%
\pgfpathlineto{\pgfqpoint{5.187181in}{3.127152in}}%
\pgfpathlineto{\pgfqpoint{5.246055in}{3.034282in}}%
\pgfpathlineto{\pgfqpoint{5.276562in}{2.990390in}}%
\pgfpathlineto{\pgfqpoint{5.301717in}{2.957971in}}%
\pgfpathlineto{\pgfqpoint{5.323661in}{2.933184in}}%
\pgfpathlineto{\pgfqpoint{5.343464in}{2.914038in}}%
\pgfpathlineto{\pgfqpoint{5.361661in}{2.899414in}}%
\pgfpathlineto{\pgfqpoint{5.378788in}{2.888432in}}%
\pgfpathlineto{\pgfqpoint{5.394845in}{2.880707in}}%
\pgfpathlineto{\pgfqpoint{5.410366in}{2.875682in}}%
\pgfpathlineto{\pgfqpoint{5.425352in}{2.873156in}}%
\pgfpathlineto{\pgfqpoint{5.440070in}{2.872923in}}%
\pgfpathlineto{\pgfqpoint{5.454789in}{2.874919in}}%
\pgfpathlineto{\pgfqpoint{5.469775in}{2.879223in}}%
\pgfpathlineto{\pgfqpoint{5.485296in}{2.886056in}}%
\pgfpathlineto{\pgfqpoint{5.501353in}{2.895596in}}%
\pgfpathlineto{\pgfqpoint{5.518479in}{2.908429in}}%
\pgfpathlineto{\pgfqpoint{5.536677in}{2.924902in}}%
\pgfpathlineto{\pgfqpoint{5.556480in}{2.945893in}}%
\pgfpathlineto{\pgfqpoint{5.578424in}{2.972486in}}%
\pgfpathlineto{\pgfqpoint{5.603579in}{3.006623in}}%
\pgfpathlineto{\pgfqpoint{5.633818in}{3.051621in}}%
\pgfpathlineto{\pgfqpoint{5.678509in}{3.122620in}}%
\pgfpathlineto{\pgfqpoint{5.737918in}{3.216338in}}%
\pgfpathlineto{\pgfqpoint{5.768425in}{3.260211in}}%
\pgfpathlineto{\pgfqpoint{5.793580in}{3.292609in}}%
\pgfpathlineto{\pgfqpoint{5.815524in}{3.317375in}}%
\pgfpathlineto{\pgfqpoint{5.835327in}{3.336499in}}%
\pgfpathlineto{\pgfqpoint{5.853524in}{3.351101in}}%
\pgfpathlineto{\pgfqpoint{5.870651in}{3.362061in}}%
\pgfpathlineto{\pgfqpoint{5.886708in}{3.369766in}}%
\pgfpathlineto{\pgfqpoint{5.902229in}{3.374769in}}%
\pgfpathlineto{\pgfqpoint{5.917215in}{3.377275in}}%
\pgfpathlineto{\pgfqpoint{5.931934in}{3.377488in}}%
\pgfpathlineto{\pgfqpoint{5.946652in}{3.375471in}}%
\pgfpathlineto{\pgfqpoint{5.961638in}{3.371147in}}%
\pgfpathlineto{\pgfqpoint{5.977159in}{3.364294in}}%
\pgfpathlineto{\pgfqpoint{5.993216in}{3.354733in}}%
\pgfpathlineto{\pgfqpoint{6.010343in}{3.341879in}}%
\pgfpathlineto{\pgfqpoint{6.028540in}{3.325385in}}%
\pgfpathlineto{\pgfqpoint{6.048343in}{3.304374in}}%
\pgfpathlineto{\pgfqpoint{6.070287in}{3.277761in}}%
\pgfpathlineto{\pgfqpoint{6.095442in}{3.243606in}}%
\pgfpathlineto{\pgfqpoint{6.125682in}{3.198592in}}%
\pgfpathlineto{\pgfqpoint{6.170640in}{3.127152in}}%
\pgfpathlineto{\pgfqpoint{6.229513in}{3.034282in}}%
\pgfpathlineto{\pgfqpoint{6.260021in}{2.990390in}}%
\pgfpathlineto{\pgfqpoint{6.285176in}{2.957971in}}%
\pgfpathlineto{\pgfqpoint{6.307120in}{2.933184in}}%
\pgfpathlineto{\pgfqpoint{6.326923in}{2.914038in}}%
\pgfpathlineto{\pgfqpoint{6.345120in}{2.899414in}}%
\pgfpathlineto{\pgfqpoint{6.362247in}{2.888432in}}%
\pgfpathlineto{\pgfqpoint{6.378303in}{2.880707in}}%
\pgfpathlineto{\pgfqpoint{6.393825in}{2.875682in}}%
\pgfpathlineto{\pgfqpoint{6.408811in}{2.873156in}}%
\pgfpathlineto{\pgfqpoint{6.423529in}{2.872923in}}%
\pgfpathlineto{\pgfqpoint{6.438248in}{2.874919in}}%
\pgfpathlineto{\pgfqpoint{6.453234in}{2.879223in}}%
\pgfpathlineto{\pgfqpoint{6.468755in}{2.886056in}}%
\pgfpathlineto{\pgfqpoint{6.484811in}{2.895596in}}%
\pgfpathlineto{\pgfqpoint{6.501938in}{2.908429in}}%
\pgfpathlineto{\pgfqpoint{6.520136in}{2.924902in}}%
\pgfpathlineto{\pgfqpoint{6.539939in}{2.945893in}}%
\pgfpathlineto{\pgfqpoint{6.561882in}{2.972486in}}%
\pgfpathlineto{\pgfqpoint{6.587038in}{3.006623in}}%
\pgfpathlineto{\pgfqpoint{6.617277in}{3.051621in}}%
\pgfpathlineto{\pgfqpoint{6.661968in}{3.122620in}}%
\pgfpathlineto{\pgfqpoint{6.663306in}{3.124778in}}%
\pgfpathlineto{\pgfqpoint{6.663306in}{3.124778in}}%
\pgfusepath{stroke}%
\end{pgfscope}%
\begin{pgfscope}%
\pgfpathrectangle{\pgfqpoint{0.467797in}{2.292089in}}{\pgfqpoint{6.490533in}{1.666241in}}%
\pgfusepath{clip}%
\pgfsetrectcap%
\pgfsetroundjoin%
\pgfsetlinewidth{1.505625pt}%
\definecolor{currentstroke}{rgb}{0.172549,0.627451,0.172549}%
\pgfsetstrokecolor{currentstroke}%
\pgfsetdash{}{0pt}%
\pgfpathmoveto{\pgfqpoint{0.762821in}{3.125209in}}%
\pgfpathlineto{\pgfqpoint{0.803765in}{3.189177in}}%
\pgfpathlineto{\pgfqpoint{0.825441in}{3.218892in}}%
\pgfpathlineto{\pgfqpoint{0.843103in}{3.239495in}}%
\pgfpathlineto{\pgfqpoint{0.858625in}{3.254294in}}%
\pgfpathlineto{\pgfqpoint{0.872540in}{3.264581in}}%
\pgfpathlineto{\pgfqpoint{0.885385in}{3.271370in}}%
\pgfpathlineto{\pgfqpoint{0.897428in}{3.275256in}}%
\pgfpathlineto{\pgfqpoint{0.908935in}{3.276669in}}%
\pgfpathlineto{\pgfqpoint{0.920442in}{3.275810in}}%
\pgfpathlineto{\pgfqpoint{0.931949in}{3.272694in}}%
\pgfpathlineto{\pgfqpoint{0.943991in}{3.267065in}}%
\pgfpathlineto{\pgfqpoint{0.956837in}{3.258497in}}%
\pgfpathlineto{\pgfqpoint{0.970752in}{3.246411in}}%
\pgfpathlineto{\pgfqpoint{0.986006in}{3.230123in}}%
\pgfpathlineto{\pgfqpoint{1.003668in}{3.207845in}}%
\pgfpathlineto{\pgfqpoint{1.025077in}{3.177017in}}%
\pgfpathlineto{\pgfqpoint{1.056922in}{3.126720in}}%
\pgfpathlineto{\pgfqpoint{1.098936in}{3.061046in}}%
\pgfpathlineto{\pgfqpoint{1.120613in}{3.031357in}}%
\pgfpathlineto{\pgfqpoint{1.138275in}{3.010783in}}%
\pgfpathlineto{\pgfqpoint{1.153796in}{2.996012in}}%
\pgfpathlineto{\pgfqpoint{1.167712in}{2.985754in}}%
\pgfpathlineto{\pgfqpoint{1.180557in}{2.978993in}}%
\pgfpathlineto{\pgfqpoint{1.192599in}{2.975133in}}%
\pgfpathlineto{\pgfqpoint{1.204106in}{2.973747in}}%
\pgfpathlineto{\pgfqpoint{1.215613in}{2.974632in}}%
\pgfpathlineto{\pgfqpoint{1.227120in}{2.977775in}}%
\pgfpathlineto{\pgfqpoint{1.239163in}{2.983430in}}%
\pgfpathlineto{\pgfqpoint{1.252008in}{2.992025in}}%
\pgfpathlineto{\pgfqpoint{1.265924in}{3.004138in}}%
\pgfpathlineto{\pgfqpoint{1.281177in}{3.020452in}}%
\pgfpathlineto{\pgfqpoint{1.298839in}{3.042755in}}%
\pgfpathlineto{\pgfqpoint{1.320248in}{3.073604in}}%
\pgfpathlineto{\pgfqpoint{1.352093in}{3.123915in}}%
\pgfpathlineto{\pgfqpoint{1.393840in}{3.189177in}}%
\pgfpathlineto{\pgfqpoint{1.415516in}{3.218892in}}%
\pgfpathlineto{\pgfqpoint{1.433179in}{3.239495in}}%
\pgfpathlineto{\pgfqpoint{1.448700in}{3.254294in}}%
\pgfpathlineto{\pgfqpoint{1.462615in}{3.264581in}}%
\pgfpathlineto{\pgfqpoint{1.475461in}{3.271370in}}%
\pgfpathlineto{\pgfqpoint{1.487503in}{3.275256in}}%
\pgfpathlineto{\pgfqpoint{1.499010in}{3.276669in}}%
\pgfpathlineto{\pgfqpoint{1.510517in}{3.275810in}}%
\pgfpathlineto{\pgfqpoint{1.522024in}{3.272694in}}%
\pgfpathlineto{\pgfqpoint{1.534067in}{3.267065in}}%
\pgfpathlineto{\pgfqpoint{1.546912in}{3.258497in}}%
\pgfpathlineto{\pgfqpoint{1.560827in}{3.246411in}}%
\pgfpathlineto{\pgfqpoint{1.576081in}{3.230123in}}%
\pgfpathlineto{\pgfqpoint{1.593743in}{3.207845in}}%
\pgfpathlineto{\pgfqpoint{1.615152in}{3.177017in}}%
\pgfpathlineto{\pgfqpoint{1.646997in}{3.126720in}}%
\pgfpathlineto{\pgfqpoint{1.689012in}{3.061046in}}%
\pgfpathlineto{\pgfqpoint{1.710688in}{3.031357in}}%
\pgfpathlineto{\pgfqpoint{1.728350in}{3.010783in}}%
\pgfpathlineto{\pgfqpoint{1.743871in}{2.996012in}}%
\pgfpathlineto{\pgfqpoint{1.757787in}{2.985754in}}%
\pgfpathlineto{\pgfqpoint{1.770632in}{2.978993in}}%
\pgfpathlineto{\pgfqpoint{1.782674in}{2.975133in}}%
\pgfpathlineto{\pgfqpoint{1.794181in}{2.973747in}}%
\pgfpathlineto{\pgfqpoint{1.805689in}{2.974632in}}%
\pgfpathlineto{\pgfqpoint{1.817196in}{2.977775in}}%
\pgfpathlineto{\pgfqpoint{1.829238in}{2.983430in}}%
\pgfpathlineto{\pgfqpoint{1.842083in}{2.992025in}}%
\pgfpathlineto{\pgfqpoint{1.855999in}{3.004138in}}%
\pgfpathlineto{\pgfqpoint{1.871253in}{3.020452in}}%
\pgfpathlineto{\pgfqpoint{1.888915in}{3.042755in}}%
\pgfpathlineto{\pgfqpoint{1.910323in}{3.073604in}}%
\pgfpathlineto{\pgfqpoint{1.942169in}{3.123915in}}%
\pgfpathlineto{\pgfqpoint{1.983915in}{3.189177in}}%
\pgfpathlineto{\pgfqpoint{2.005592in}{3.218892in}}%
\pgfpathlineto{\pgfqpoint{2.023254in}{3.239495in}}%
\pgfpathlineto{\pgfqpoint{2.038775in}{3.254294in}}%
\pgfpathlineto{\pgfqpoint{2.052691in}{3.264581in}}%
\pgfpathlineto{\pgfqpoint{2.065536in}{3.271370in}}%
\pgfpathlineto{\pgfqpoint{2.077578in}{3.275256in}}%
\pgfpathlineto{\pgfqpoint{2.089085in}{3.276669in}}%
\pgfpathlineto{\pgfqpoint{2.100592in}{3.275810in}}%
\pgfpathlineto{\pgfqpoint{2.112100in}{3.272694in}}%
\pgfpathlineto{\pgfqpoint{2.124142in}{3.267065in}}%
\pgfpathlineto{\pgfqpoint{2.136987in}{3.258497in}}%
\pgfpathlineto{\pgfqpoint{2.150903in}{3.246411in}}%
\pgfpathlineto{\pgfqpoint{2.166156in}{3.230123in}}%
\pgfpathlineto{\pgfqpoint{2.183818in}{3.207845in}}%
\pgfpathlineto{\pgfqpoint{2.205227in}{3.177017in}}%
\pgfpathlineto{\pgfqpoint{2.237072in}{3.126720in}}%
\pgfpathlineto{\pgfqpoint{2.279087in}{3.061046in}}%
\pgfpathlineto{\pgfqpoint{2.300763in}{3.031357in}}%
\pgfpathlineto{\pgfqpoint{2.318425in}{3.010783in}}%
\pgfpathlineto{\pgfqpoint{2.333946in}{2.996012in}}%
\pgfpathlineto{\pgfqpoint{2.347862in}{2.985754in}}%
\pgfpathlineto{\pgfqpoint{2.360707in}{2.978993in}}%
\pgfpathlineto{\pgfqpoint{2.372750in}{2.975133in}}%
\pgfpathlineto{\pgfqpoint{2.384257in}{2.973747in}}%
\pgfpathlineto{\pgfqpoint{2.395764in}{2.974632in}}%
\pgfpathlineto{\pgfqpoint{2.407271in}{2.977775in}}%
\pgfpathlineto{\pgfqpoint{2.419313in}{2.983430in}}%
\pgfpathlineto{\pgfqpoint{2.432158in}{2.992025in}}%
\pgfpathlineto{\pgfqpoint{2.446074in}{3.004138in}}%
\pgfpathlineto{\pgfqpoint{2.461328in}{3.020452in}}%
\pgfpathlineto{\pgfqpoint{2.478990in}{3.042755in}}%
\pgfpathlineto{\pgfqpoint{2.500398in}{3.073604in}}%
\pgfpathlineto{\pgfqpoint{2.532244in}{3.123915in}}%
\pgfpathlineto{\pgfqpoint{2.573991in}{3.189177in}}%
\pgfpathlineto{\pgfqpoint{2.595667in}{3.218892in}}%
\pgfpathlineto{\pgfqpoint{2.613329in}{3.239495in}}%
\pgfpathlineto{\pgfqpoint{2.628850in}{3.254294in}}%
\pgfpathlineto{\pgfqpoint{2.642766in}{3.264581in}}%
\pgfpathlineto{\pgfqpoint{2.655611in}{3.271370in}}%
\pgfpathlineto{\pgfqpoint{2.667653in}{3.275256in}}%
\pgfpathlineto{\pgfqpoint{2.679161in}{3.276669in}}%
\pgfpathlineto{\pgfqpoint{2.690668in}{3.275810in}}%
\pgfpathlineto{\pgfqpoint{2.702175in}{3.272694in}}%
\pgfpathlineto{\pgfqpoint{2.714217in}{3.267065in}}%
\pgfpathlineto{\pgfqpoint{2.727062in}{3.258497in}}%
\pgfpathlineto{\pgfqpoint{2.740978in}{3.246411in}}%
\pgfpathlineto{\pgfqpoint{2.756232in}{3.230123in}}%
\pgfpathlineto{\pgfqpoint{2.773894in}{3.207845in}}%
\pgfpathlineto{\pgfqpoint{2.795302in}{3.177017in}}%
\pgfpathlineto{\pgfqpoint{2.827148in}{3.126720in}}%
\pgfpathlineto{\pgfqpoint{2.869162in}{3.061046in}}%
\pgfpathlineto{\pgfqpoint{2.890838in}{3.031357in}}%
\pgfpathlineto{\pgfqpoint{2.908500in}{3.010783in}}%
\pgfpathlineto{\pgfqpoint{2.924022in}{2.996012in}}%
\pgfpathlineto{\pgfqpoint{2.937937in}{2.985754in}}%
\pgfpathlineto{\pgfqpoint{2.950782in}{2.978993in}}%
\pgfpathlineto{\pgfqpoint{2.962825in}{2.975133in}}%
\pgfpathlineto{\pgfqpoint{2.974332in}{2.973747in}}%
\pgfpathlineto{\pgfqpoint{2.985839in}{2.974632in}}%
\pgfpathlineto{\pgfqpoint{2.997346in}{2.977775in}}%
\pgfpathlineto{\pgfqpoint{3.009389in}{2.983430in}}%
\pgfpathlineto{\pgfqpoint{3.022234in}{2.992025in}}%
\pgfpathlineto{\pgfqpoint{3.036149in}{3.004138in}}%
\pgfpathlineto{\pgfqpoint{3.051403in}{3.020452in}}%
\pgfpathlineto{\pgfqpoint{3.069065in}{3.042755in}}%
\pgfpathlineto{\pgfqpoint{3.090474in}{3.073604in}}%
\pgfpathlineto{\pgfqpoint{3.122319in}{3.123915in}}%
\pgfpathlineto{\pgfqpoint{3.164066in}{3.189177in}}%
\pgfpathlineto{\pgfqpoint{3.185742in}{3.218892in}}%
\pgfpathlineto{\pgfqpoint{3.203404in}{3.239495in}}%
\pgfpathlineto{\pgfqpoint{3.218925in}{3.254294in}}%
\pgfpathlineto{\pgfqpoint{3.232841in}{3.264581in}}%
\pgfpathlineto{\pgfqpoint{3.245686in}{3.271370in}}%
\pgfpathlineto{\pgfqpoint{3.257729in}{3.275256in}}%
\pgfpathlineto{\pgfqpoint{3.269236in}{3.276669in}}%
\pgfpathlineto{\pgfqpoint{3.280743in}{3.275810in}}%
\pgfpathlineto{\pgfqpoint{3.292250in}{3.272694in}}%
\pgfpathlineto{\pgfqpoint{3.304292in}{3.267065in}}%
\pgfpathlineto{\pgfqpoint{3.317138in}{3.258497in}}%
\pgfpathlineto{\pgfqpoint{3.331053in}{3.246411in}}%
\pgfpathlineto{\pgfqpoint{3.346307in}{3.230123in}}%
\pgfpathlineto{\pgfqpoint{3.363969in}{3.207845in}}%
\pgfpathlineto{\pgfqpoint{3.385378in}{3.177017in}}%
\pgfpathlineto{\pgfqpoint{3.417223in}{3.126720in}}%
\pgfpathlineto{\pgfqpoint{3.459237in}{3.061046in}}%
\pgfpathlineto{\pgfqpoint{3.480914in}{3.031357in}}%
\pgfpathlineto{\pgfqpoint{3.498576in}{3.010783in}}%
\pgfpathlineto{\pgfqpoint{3.514097in}{2.996012in}}%
\pgfpathlineto{\pgfqpoint{3.528013in}{2.985754in}}%
\pgfpathlineto{\pgfqpoint{3.540858in}{2.978993in}}%
\pgfpathlineto{\pgfqpoint{3.552900in}{2.975133in}}%
\pgfpathlineto{\pgfqpoint{3.564407in}{2.973747in}}%
\pgfpathlineto{\pgfqpoint{3.575914in}{2.974632in}}%
\pgfpathlineto{\pgfqpoint{3.587421in}{2.977775in}}%
\pgfpathlineto{\pgfqpoint{3.599464in}{2.983430in}}%
\pgfpathlineto{\pgfqpoint{3.612309in}{2.992025in}}%
\pgfpathlineto{\pgfqpoint{3.626225in}{3.004138in}}%
\pgfpathlineto{\pgfqpoint{3.641478in}{3.020452in}}%
\pgfpathlineto{\pgfqpoint{3.659140in}{3.042755in}}%
\pgfpathlineto{\pgfqpoint{3.680549in}{3.073604in}}%
\pgfpathlineto{\pgfqpoint{3.712394in}{3.123915in}}%
\pgfpathlineto{\pgfqpoint{3.754141in}{3.189177in}}%
\pgfpathlineto{\pgfqpoint{3.775817in}{3.218892in}}%
\pgfpathlineto{\pgfqpoint{3.793479in}{3.239495in}}%
\pgfpathlineto{\pgfqpoint{3.809001in}{3.254294in}}%
\pgfpathlineto{\pgfqpoint{3.822916in}{3.264581in}}%
\pgfpathlineto{\pgfqpoint{3.835762in}{3.271370in}}%
\pgfpathlineto{\pgfqpoint{3.847804in}{3.275256in}}%
\pgfpathlineto{\pgfqpoint{3.859311in}{3.276669in}}%
\pgfpathlineto{\pgfqpoint{3.870818in}{3.275810in}}%
\pgfpathlineto{\pgfqpoint{3.882325in}{3.272694in}}%
\pgfpathlineto{\pgfqpoint{3.894368in}{3.267065in}}%
\pgfpathlineto{\pgfqpoint{3.907213in}{3.258497in}}%
\pgfpathlineto{\pgfqpoint{3.921128in}{3.246411in}}%
\pgfpathlineto{\pgfqpoint{3.936382in}{3.230123in}}%
\pgfpathlineto{\pgfqpoint{3.954044in}{3.207845in}}%
\pgfpathlineto{\pgfqpoint{3.975453in}{3.177017in}}%
\pgfpathlineto{\pgfqpoint{4.007298in}{3.126720in}}%
\pgfpathlineto{\pgfqpoint{4.049313in}{3.061046in}}%
\pgfpathlineto{\pgfqpoint{4.070989in}{3.031357in}}%
\pgfpathlineto{\pgfqpoint{4.088651in}{3.010783in}}%
\pgfpathlineto{\pgfqpoint{4.104172in}{2.996012in}}%
\pgfpathlineto{\pgfqpoint{4.118088in}{2.985754in}}%
\pgfpathlineto{\pgfqpoint{4.130933in}{2.978993in}}%
\pgfpathlineto{\pgfqpoint{4.142975in}{2.975133in}}%
\pgfpathlineto{\pgfqpoint{4.154482in}{2.973747in}}%
\pgfpathlineto{\pgfqpoint{4.165990in}{2.974632in}}%
\pgfpathlineto{\pgfqpoint{4.177497in}{2.977775in}}%
\pgfpathlineto{\pgfqpoint{4.189539in}{2.983430in}}%
\pgfpathlineto{\pgfqpoint{4.202384in}{2.992025in}}%
\pgfpathlineto{\pgfqpoint{4.216300in}{3.004138in}}%
\pgfpathlineto{\pgfqpoint{4.231553in}{3.020452in}}%
\pgfpathlineto{\pgfqpoint{4.249216in}{3.042755in}}%
\pgfpathlineto{\pgfqpoint{4.270624in}{3.073604in}}%
\pgfpathlineto{\pgfqpoint{4.302470in}{3.123915in}}%
\pgfpathlineto{\pgfqpoint{4.344216in}{3.189177in}}%
\pgfpathlineto{\pgfqpoint{4.365893in}{3.218892in}}%
\pgfpathlineto{\pgfqpoint{4.383555in}{3.239495in}}%
\pgfpathlineto{\pgfqpoint{4.399076in}{3.254294in}}%
\pgfpathlineto{\pgfqpoint{4.412992in}{3.264581in}}%
\pgfpathlineto{\pgfqpoint{4.425837in}{3.271370in}}%
\pgfpathlineto{\pgfqpoint{4.437879in}{3.275256in}}%
\pgfpathlineto{\pgfqpoint{4.449386in}{3.276669in}}%
\pgfpathlineto{\pgfqpoint{4.460893in}{3.275810in}}%
\pgfpathlineto{\pgfqpoint{4.472401in}{3.272694in}}%
\pgfpathlineto{\pgfqpoint{4.484443in}{3.267065in}}%
\pgfpathlineto{\pgfqpoint{4.497288in}{3.258497in}}%
\pgfpathlineto{\pgfqpoint{4.511204in}{3.246411in}}%
\pgfpathlineto{\pgfqpoint{4.526457in}{3.230123in}}%
\pgfpathlineto{\pgfqpoint{4.544119in}{3.207845in}}%
\pgfpathlineto{\pgfqpoint{4.565528in}{3.177017in}}%
\pgfpathlineto{\pgfqpoint{4.597373in}{3.126720in}}%
\pgfpathlineto{\pgfqpoint{4.639388in}{3.061046in}}%
\pgfpathlineto{\pgfqpoint{4.661064in}{3.031357in}}%
\pgfpathlineto{\pgfqpoint{4.678726in}{3.010783in}}%
\pgfpathlineto{\pgfqpoint{4.694247in}{2.996012in}}%
\pgfpathlineto{\pgfqpoint{4.708163in}{2.985754in}}%
\pgfpathlineto{\pgfqpoint{4.721008in}{2.978993in}}%
\pgfpathlineto{\pgfqpoint{4.733051in}{2.975133in}}%
\pgfpathlineto{\pgfqpoint{4.744558in}{2.973747in}}%
\pgfpathlineto{\pgfqpoint{4.756065in}{2.974632in}}%
\pgfpathlineto{\pgfqpoint{4.767572in}{2.977775in}}%
\pgfpathlineto{\pgfqpoint{4.779614in}{2.983430in}}%
\pgfpathlineto{\pgfqpoint{4.792459in}{2.992025in}}%
\pgfpathlineto{\pgfqpoint{4.806375in}{3.004138in}}%
\pgfpathlineto{\pgfqpoint{4.821629in}{3.020452in}}%
\pgfpathlineto{\pgfqpoint{4.839291in}{3.042755in}}%
\pgfpathlineto{\pgfqpoint{4.860699in}{3.073604in}}%
\pgfpathlineto{\pgfqpoint{4.892545in}{3.123915in}}%
\pgfpathlineto{\pgfqpoint{4.934292in}{3.189177in}}%
\pgfpathlineto{\pgfqpoint{4.955968in}{3.218892in}}%
\pgfpathlineto{\pgfqpoint{4.973630in}{3.239495in}}%
\pgfpathlineto{\pgfqpoint{4.989151in}{3.254294in}}%
\pgfpathlineto{\pgfqpoint{5.003067in}{3.264581in}}%
\pgfpathlineto{\pgfqpoint{5.015912in}{3.271370in}}%
\pgfpathlineto{\pgfqpoint{5.027954in}{3.275256in}}%
\pgfpathlineto{\pgfqpoint{5.039461in}{3.276669in}}%
\pgfpathlineto{\pgfqpoint{5.050969in}{3.275810in}}%
\pgfpathlineto{\pgfqpoint{5.062476in}{3.272694in}}%
\pgfpathlineto{\pgfqpoint{5.074518in}{3.267065in}}%
\pgfpathlineto{\pgfqpoint{5.087363in}{3.258497in}}%
\pgfpathlineto{\pgfqpoint{5.101279in}{3.246411in}}%
\pgfpathlineto{\pgfqpoint{5.116533in}{3.230123in}}%
\pgfpathlineto{\pgfqpoint{5.134195in}{3.207845in}}%
\pgfpathlineto{\pgfqpoint{5.155603in}{3.177017in}}%
\pgfpathlineto{\pgfqpoint{5.187449in}{3.126720in}}%
\pgfpathlineto{\pgfqpoint{5.229463in}{3.061046in}}%
\pgfpathlineto{\pgfqpoint{5.251139in}{3.031357in}}%
\pgfpathlineto{\pgfqpoint{5.268801in}{3.010783in}}%
\pgfpathlineto{\pgfqpoint{5.284323in}{2.996012in}}%
\pgfpathlineto{\pgfqpoint{5.298238in}{2.985754in}}%
\pgfpathlineto{\pgfqpoint{5.311083in}{2.978993in}}%
\pgfpathlineto{\pgfqpoint{5.323126in}{2.975133in}}%
\pgfpathlineto{\pgfqpoint{5.334633in}{2.973747in}}%
\pgfpathlineto{\pgfqpoint{5.346140in}{2.974632in}}%
\pgfpathlineto{\pgfqpoint{5.357647in}{2.977775in}}%
\pgfpathlineto{\pgfqpoint{5.369690in}{2.983430in}}%
\pgfpathlineto{\pgfqpoint{5.382535in}{2.992025in}}%
\pgfpathlineto{\pgfqpoint{5.396450in}{3.004138in}}%
\pgfpathlineto{\pgfqpoint{5.411704in}{3.020452in}}%
\pgfpathlineto{\pgfqpoint{5.429366in}{3.042755in}}%
\pgfpathlineto{\pgfqpoint{5.450775in}{3.073604in}}%
\pgfpathlineto{\pgfqpoint{5.482620in}{3.123915in}}%
\pgfpathlineto{\pgfqpoint{5.524367in}{3.189177in}}%
\pgfpathlineto{\pgfqpoint{5.546043in}{3.218892in}}%
\pgfpathlineto{\pgfqpoint{5.563705in}{3.239495in}}%
\pgfpathlineto{\pgfqpoint{5.579226in}{3.254294in}}%
\pgfpathlineto{\pgfqpoint{5.593142in}{3.264581in}}%
\pgfpathlineto{\pgfqpoint{5.605987in}{3.271370in}}%
\pgfpathlineto{\pgfqpoint{5.618030in}{3.275256in}}%
\pgfpathlineto{\pgfqpoint{5.629537in}{3.276669in}}%
\pgfpathlineto{\pgfqpoint{5.641044in}{3.275810in}}%
\pgfpathlineto{\pgfqpoint{5.652551in}{3.272694in}}%
\pgfpathlineto{\pgfqpoint{5.664593in}{3.267065in}}%
\pgfpathlineto{\pgfqpoint{5.677439in}{3.258497in}}%
\pgfpathlineto{\pgfqpoint{5.691354in}{3.246411in}}%
\pgfpathlineto{\pgfqpoint{5.706608in}{3.230123in}}%
\pgfpathlineto{\pgfqpoint{5.724270in}{3.207845in}}%
\pgfpathlineto{\pgfqpoint{5.745679in}{3.177017in}}%
\pgfpathlineto{\pgfqpoint{5.777524in}{3.126720in}}%
\pgfpathlineto{\pgfqpoint{5.819538in}{3.061046in}}%
\pgfpathlineto{\pgfqpoint{5.841215in}{3.031357in}}%
\pgfpathlineto{\pgfqpoint{5.858877in}{3.010783in}}%
\pgfpathlineto{\pgfqpoint{5.874398in}{2.996012in}}%
\pgfpathlineto{\pgfqpoint{5.888314in}{2.985754in}}%
\pgfpathlineto{\pgfqpoint{5.901159in}{2.978993in}}%
\pgfpathlineto{\pgfqpoint{5.913201in}{2.975133in}}%
\pgfpathlineto{\pgfqpoint{5.924708in}{2.973747in}}%
\pgfpathlineto{\pgfqpoint{5.936215in}{2.974632in}}%
\pgfpathlineto{\pgfqpoint{5.947722in}{2.977775in}}%
\pgfpathlineto{\pgfqpoint{5.959765in}{2.983430in}}%
\pgfpathlineto{\pgfqpoint{5.972610in}{2.992025in}}%
\pgfpathlineto{\pgfqpoint{5.986526in}{3.004138in}}%
\pgfpathlineto{\pgfqpoint{6.001779in}{3.020452in}}%
\pgfpathlineto{\pgfqpoint{6.019441in}{3.042755in}}%
\pgfpathlineto{\pgfqpoint{6.040850in}{3.073604in}}%
\pgfpathlineto{\pgfqpoint{6.072695in}{3.123915in}}%
\pgfpathlineto{\pgfqpoint{6.114442in}{3.189177in}}%
\pgfpathlineto{\pgfqpoint{6.136118in}{3.218892in}}%
\pgfpathlineto{\pgfqpoint{6.153780in}{3.239495in}}%
\pgfpathlineto{\pgfqpoint{6.169302in}{3.254294in}}%
\pgfpathlineto{\pgfqpoint{6.183217in}{3.264581in}}%
\pgfpathlineto{\pgfqpoint{6.196063in}{3.271370in}}%
\pgfpathlineto{\pgfqpoint{6.208105in}{3.275256in}}%
\pgfpathlineto{\pgfqpoint{6.219612in}{3.276669in}}%
\pgfpathlineto{\pgfqpoint{6.231119in}{3.275810in}}%
\pgfpathlineto{\pgfqpoint{6.242626in}{3.272694in}}%
\pgfpathlineto{\pgfqpoint{6.254669in}{3.267065in}}%
\pgfpathlineto{\pgfqpoint{6.267514in}{3.258497in}}%
\pgfpathlineto{\pgfqpoint{6.281429in}{3.246411in}}%
\pgfpathlineto{\pgfqpoint{6.296683in}{3.230123in}}%
\pgfpathlineto{\pgfqpoint{6.314345in}{3.207845in}}%
\pgfpathlineto{\pgfqpoint{6.335754in}{3.177017in}}%
\pgfpathlineto{\pgfqpoint{6.367599in}{3.126720in}}%
\pgfpathlineto{\pgfqpoint{6.409614in}{3.061046in}}%
\pgfpathlineto{\pgfqpoint{6.431290in}{3.031357in}}%
\pgfpathlineto{\pgfqpoint{6.448952in}{3.010783in}}%
\pgfpathlineto{\pgfqpoint{6.464473in}{2.996012in}}%
\pgfpathlineto{\pgfqpoint{6.478389in}{2.985754in}}%
\pgfpathlineto{\pgfqpoint{6.491234in}{2.978993in}}%
\pgfpathlineto{\pgfqpoint{6.503276in}{2.975133in}}%
\pgfpathlineto{\pgfqpoint{6.514783in}{2.973747in}}%
\pgfpathlineto{\pgfqpoint{6.526291in}{2.974632in}}%
\pgfpathlineto{\pgfqpoint{6.537798in}{2.977775in}}%
\pgfpathlineto{\pgfqpoint{6.549840in}{2.983430in}}%
\pgfpathlineto{\pgfqpoint{6.562685in}{2.992025in}}%
\pgfpathlineto{\pgfqpoint{6.576601in}{3.004138in}}%
\pgfpathlineto{\pgfqpoint{6.591854in}{3.020452in}}%
\pgfpathlineto{\pgfqpoint{6.609517in}{3.042755in}}%
\pgfpathlineto{\pgfqpoint{6.630925in}{3.073604in}}%
\pgfpathlineto{\pgfqpoint{6.662771in}{3.123915in}}%
\pgfpathlineto{\pgfqpoint{6.663306in}{3.124778in}}%
\pgfpathlineto{\pgfqpoint{6.663306in}{3.124778in}}%
\pgfusepath{stroke}%
\end{pgfscope}%
\begin{pgfscope}%
\pgfpathrectangle{\pgfqpoint{0.467797in}{2.292089in}}{\pgfqpoint{6.490533in}{1.666241in}}%
\pgfusepath{clip}%
\pgfsetrectcap%
\pgfsetroundjoin%
\pgfsetlinewidth{1.505625pt}%
\definecolor{currentstroke}{rgb}{0.839216,0.152941,0.156863}%
\pgfsetstrokecolor{currentstroke}%
\pgfsetdash{}{0pt}%
\pgfpathmoveto{\pgfqpoint{0.762821in}{3.125209in}}%
\pgfpathlineto{\pgfqpoint{0.795737in}{3.176196in}}%
\pgfpathlineto{\pgfqpoint{0.812864in}{3.198644in}}%
\pgfpathlineto{\pgfqpoint{0.826779in}{3.213435in}}%
\pgfpathlineto{\pgfqpoint{0.839089in}{3.223383in}}%
\pgfpathlineto{\pgfqpoint{0.850061in}{3.229479in}}%
\pgfpathlineto{\pgfqpoint{0.860230in}{3.232646in}}%
\pgfpathlineto{\pgfqpoint{0.869864in}{3.233373in}}%
\pgfpathlineto{\pgfqpoint{0.879498in}{3.231874in}}%
\pgfpathlineto{\pgfqpoint{0.889399in}{3.228044in}}%
\pgfpathlineto{\pgfqpoint{0.899836in}{3.221590in}}%
\pgfpathlineto{\pgfqpoint{0.911343in}{3.211782in}}%
\pgfpathlineto{\pgfqpoint{0.924456in}{3.197527in}}%
\pgfpathlineto{\pgfqpoint{0.939977in}{3.177145in}}%
\pgfpathlineto{\pgfqpoint{0.960315in}{3.146437in}}%
\pgfpathlineto{\pgfqpoint{1.012767in}{3.065514in}}%
\pgfpathlineto{\pgfqpoint{1.028555in}{3.046116in}}%
\pgfpathlineto{\pgfqpoint{1.041936in}{3.033055in}}%
\pgfpathlineto{\pgfqpoint{1.053711in}{3.024570in}}%
\pgfpathlineto{\pgfqpoint{1.064415in}{3.019536in}}%
\pgfpathlineto{\pgfqpoint{1.074316in}{3.017268in}}%
\pgfpathlineto{\pgfqpoint{1.083950in}{3.017314in}}%
\pgfpathlineto{\pgfqpoint{1.093584in}{3.019583in}}%
\pgfpathlineto{\pgfqpoint{1.103753in}{3.024334in}}%
\pgfpathlineto{\pgfqpoint{1.114725in}{3.032052in}}%
\pgfpathlineto{\pgfqpoint{1.126768in}{3.043375in}}%
\pgfpathlineto{\pgfqpoint{1.140683in}{3.059707in}}%
\pgfpathlineto{\pgfqpoint{1.157542in}{3.083182in}}%
\pgfpathlineto{\pgfqpoint{1.182430in}{3.122188in}}%
\pgfpathlineto{\pgfqpoint{1.217219in}{3.176196in}}%
\pgfpathlineto{\pgfqpoint{1.234346in}{3.198644in}}%
\pgfpathlineto{\pgfqpoint{1.248261in}{3.213435in}}%
\pgfpathlineto{\pgfqpoint{1.260571in}{3.223383in}}%
\pgfpathlineto{\pgfqpoint{1.271543in}{3.229479in}}%
\pgfpathlineto{\pgfqpoint{1.281712in}{3.232646in}}%
\pgfpathlineto{\pgfqpoint{1.291346in}{3.233373in}}%
\pgfpathlineto{\pgfqpoint{1.300980in}{3.231874in}}%
\pgfpathlineto{\pgfqpoint{1.310882in}{3.228044in}}%
\pgfpathlineto{\pgfqpoint{1.321318in}{3.221590in}}%
\pgfpathlineto{\pgfqpoint{1.332826in}{3.211782in}}%
\pgfpathlineto{\pgfqpoint{1.345938in}{3.197527in}}%
\pgfpathlineto{\pgfqpoint{1.361460in}{3.177145in}}%
\pgfpathlineto{\pgfqpoint{1.381798in}{3.146437in}}%
\pgfpathlineto{\pgfqpoint{1.434249in}{3.065514in}}%
\pgfpathlineto{\pgfqpoint{1.450038in}{3.046116in}}%
\pgfpathlineto{\pgfqpoint{1.463418in}{3.033055in}}%
\pgfpathlineto{\pgfqpoint{1.475193in}{3.024570in}}%
\pgfpathlineto{\pgfqpoint{1.485897in}{3.019536in}}%
\pgfpathlineto{\pgfqpoint{1.495799in}{3.017268in}}%
\pgfpathlineto{\pgfqpoint{1.505433in}{3.017314in}}%
\pgfpathlineto{\pgfqpoint{1.515066in}{3.019583in}}%
\pgfpathlineto{\pgfqpoint{1.525236in}{3.024334in}}%
\pgfpathlineto{\pgfqpoint{1.536208in}{3.032052in}}%
\pgfpathlineto{\pgfqpoint{1.548250in}{3.043375in}}%
\pgfpathlineto{\pgfqpoint{1.562165in}{3.059707in}}%
\pgfpathlineto{\pgfqpoint{1.579025in}{3.083182in}}%
\pgfpathlineto{\pgfqpoint{1.603912in}{3.122188in}}%
\pgfpathlineto{\pgfqpoint{1.638701in}{3.176196in}}%
\pgfpathlineto{\pgfqpoint{1.655828in}{3.198644in}}%
\pgfpathlineto{\pgfqpoint{1.669744in}{3.213435in}}%
\pgfpathlineto{\pgfqpoint{1.682054in}{3.223383in}}%
\pgfpathlineto{\pgfqpoint{1.693026in}{3.229479in}}%
\pgfpathlineto{\pgfqpoint{1.703195in}{3.232646in}}%
\pgfpathlineto{\pgfqpoint{1.712829in}{3.233373in}}%
\pgfpathlineto{\pgfqpoint{1.722463in}{3.231874in}}%
\pgfpathlineto{\pgfqpoint{1.732364in}{3.228044in}}%
\pgfpathlineto{\pgfqpoint{1.742801in}{3.221590in}}%
\pgfpathlineto{\pgfqpoint{1.754308in}{3.211782in}}%
\pgfpathlineto{\pgfqpoint{1.767421in}{3.197527in}}%
\pgfpathlineto{\pgfqpoint{1.782942in}{3.177145in}}%
\pgfpathlineto{\pgfqpoint{1.803280in}{3.146437in}}%
\pgfpathlineto{\pgfqpoint{1.855731in}{3.065514in}}%
\pgfpathlineto{\pgfqpoint{1.871520in}{3.046116in}}%
\pgfpathlineto{\pgfqpoint{1.884901in}{3.033055in}}%
\pgfpathlineto{\pgfqpoint{1.896675in}{3.024570in}}%
\pgfpathlineto{\pgfqpoint{1.907380in}{3.019536in}}%
\pgfpathlineto{\pgfqpoint{1.917281in}{3.017268in}}%
\pgfpathlineto{\pgfqpoint{1.926915in}{3.017314in}}%
\pgfpathlineto{\pgfqpoint{1.936549in}{3.019583in}}%
\pgfpathlineto{\pgfqpoint{1.946718in}{3.024334in}}%
\pgfpathlineto{\pgfqpoint{1.957690in}{3.032052in}}%
\pgfpathlineto{\pgfqpoint{1.969732in}{3.043375in}}%
\pgfpathlineto{\pgfqpoint{1.983648in}{3.059707in}}%
\pgfpathlineto{\pgfqpoint{2.000507in}{3.083182in}}%
\pgfpathlineto{\pgfqpoint{2.025395in}{3.122188in}}%
\pgfpathlineto{\pgfqpoint{2.060184in}{3.176196in}}%
\pgfpathlineto{\pgfqpoint{2.077311in}{3.198644in}}%
\pgfpathlineto{\pgfqpoint{2.091226in}{3.213435in}}%
\pgfpathlineto{\pgfqpoint{2.103536in}{3.223383in}}%
\pgfpathlineto{\pgfqpoint{2.114508in}{3.229479in}}%
\pgfpathlineto{\pgfqpoint{2.124677in}{3.232646in}}%
\pgfpathlineto{\pgfqpoint{2.134311in}{3.233373in}}%
\pgfpathlineto{\pgfqpoint{2.143945in}{3.231874in}}%
\pgfpathlineto{\pgfqpoint{2.153846in}{3.228044in}}%
\pgfpathlineto{\pgfqpoint{2.164283in}{3.221590in}}%
\pgfpathlineto{\pgfqpoint{2.175790in}{3.211782in}}%
\pgfpathlineto{\pgfqpoint{2.188903in}{3.197527in}}%
\pgfpathlineto{\pgfqpoint{2.204424in}{3.177145in}}%
\pgfpathlineto{\pgfqpoint{2.224762in}{3.146437in}}%
\pgfpathlineto{\pgfqpoint{2.277214in}{3.065514in}}%
\pgfpathlineto{\pgfqpoint{2.293002in}{3.046116in}}%
\pgfpathlineto{\pgfqpoint{2.306383in}{3.033055in}}%
\pgfpathlineto{\pgfqpoint{2.318158in}{3.024570in}}%
\pgfpathlineto{\pgfqpoint{2.328862in}{3.019536in}}%
\pgfpathlineto{\pgfqpoint{2.338763in}{3.017268in}}%
\pgfpathlineto{\pgfqpoint{2.348397in}{3.017314in}}%
\pgfpathlineto{\pgfqpoint{2.358031in}{3.019583in}}%
\pgfpathlineto{\pgfqpoint{2.368200in}{3.024334in}}%
\pgfpathlineto{\pgfqpoint{2.379172in}{3.032052in}}%
\pgfpathlineto{\pgfqpoint{2.391215in}{3.043375in}}%
\pgfpathlineto{\pgfqpoint{2.405130in}{3.059707in}}%
\pgfpathlineto{\pgfqpoint{2.421989in}{3.083182in}}%
\pgfpathlineto{\pgfqpoint{2.446877in}{3.122188in}}%
\pgfpathlineto{\pgfqpoint{2.481666in}{3.176196in}}%
\pgfpathlineto{\pgfqpoint{2.498793in}{3.198644in}}%
\pgfpathlineto{\pgfqpoint{2.512708in}{3.213435in}}%
\pgfpathlineto{\pgfqpoint{2.525018in}{3.223383in}}%
\pgfpathlineto{\pgfqpoint{2.535990in}{3.229479in}}%
\pgfpathlineto{\pgfqpoint{2.546159in}{3.232646in}}%
\pgfpathlineto{\pgfqpoint{2.555793in}{3.233373in}}%
\pgfpathlineto{\pgfqpoint{2.565427in}{3.231874in}}%
\pgfpathlineto{\pgfqpoint{2.575329in}{3.228044in}}%
\pgfpathlineto{\pgfqpoint{2.585765in}{3.221590in}}%
\pgfpathlineto{\pgfqpoint{2.597273in}{3.211782in}}%
\pgfpathlineto{\pgfqpoint{2.610385in}{3.197527in}}%
\pgfpathlineto{\pgfqpoint{2.625907in}{3.177145in}}%
\pgfpathlineto{\pgfqpoint{2.646245in}{3.146437in}}%
\pgfpathlineto{\pgfqpoint{2.698696in}{3.065514in}}%
\pgfpathlineto{\pgfqpoint{2.714485in}{3.046116in}}%
\pgfpathlineto{\pgfqpoint{2.727865in}{3.033055in}}%
\pgfpathlineto{\pgfqpoint{2.739640in}{3.024570in}}%
\pgfpathlineto{\pgfqpoint{2.750344in}{3.019536in}}%
\pgfpathlineto{\pgfqpoint{2.760246in}{3.017268in}}%
\pgfpathlineto{\pgfqpoint{2.769880in}{3.017314in}}%
\pgfpathlineto{\pgfqpoint{2.779513in}{3.019583in}}%
\pgfpathlineto{\pgfqpoint{2.789683in}{3.024334in}}%
\pgfpathlineto{\pgfqpoint{2.800654in}{3.032052in}}%
\pgfpathlineto{\pgfqpoint{2.812697in}{3.043375in}}%
\pgfpathlineto{\pgfqpoint{2.826612in}{3.059707in}}%
\pgfpathlineto{\pgfqpoint{2.843472in}{3.083182in}}%
\pgfpathlineto{\pgfqpoint{2.868359in}{3.122188in}}%
\pgfpathlineto{\pgfqpoint{2.903148in}{3.176196in}}%
\pgfpathlineto{\pgfqpoint{2.920275in}{3.198644in}}%
\pgfpathlineto{\pgfqpoint{2.934191in}{3.213435in}}%
\pgfpathlineto{\pgfqpoint{2.946501in}{3.223383in}}%
\pgfpathlineto{\pgfqpoint{2.957473in}{3.229479in}}%
\pgfpathlineto{\pgfqpoint{2.967642in}{3.232646in}}%
\pgfpathlineto{\pgfqpoint{2.977276in}{3.233373in}}%
\pgfpathlineto{\pgfqpoint{2.986910in}{3.231874in}}%
\pgfpathlineto{\pgfqpoint{2.996811in}{3.228044in}}%
\pgfpathlineto{\pgfqpoint{3.007248in}{3.221590in}}%
\pgfpathlineto{\pgfqpoint{3.018755in}{3.211782in}}%
\pgfpathlineto{\pgfqpoint{3.031868in}{3.197527in}}%
\pgfpathlineto{\pgfqpoint{3.047389in}{3.177145in}}%
\pgfpathlineto{\pgfqpoint{3.067727in}{3.146437in}}%
\pgfpathlineto{\pgfqpoint{3.120178in}{3.065514in}}%
\pgfpathlineto{\pgfqpoint{3.135967in}{3.046116in}}%
\pgfpathlineto{\pgfqpoint{3.149347in}{3.033055in}}%
\pgfpathlineto{\pgfqpoint{3.161122in}{3.024570in}}%
\pgfpathlineto{\pgfqpoint{3.171827in}{3.019536in}}%
\pgfpathlineto{\pgfqpoint{3.181728in}{3.017268in}}%
\pgfpathlineto{\pgfqpoint{3.191362in}{3.017314in}}%
\pgfpathlineto{\pgfqpoint{3.200996in}{3.019583in}}%
\pgfpathlineto{\pgfqpoint{3.211165in}{3.024334in}}%
\pgfpathlineto{\pgfqpoint{3.222137in}{3.032052in}}%
\pgfpathlineto{\pgfqpoint{3.234179in}{3.043375in}}%
\pgfpathlineto{\pgfqpoint{3.248095in}{3.059707in}}%
\pgfpathlineto{\pgfqpoint{3.264954in}{3.083182in}}%
\pgfpathlineto{\pgfqpoint{3.289842in}{3.122188in}}%
\pgfpathlineto{\pgfqpoint{3.324631in}{3.176196in}}%
\pgfpathlineto{\pgfqpoint{3.341757in}{3.198644in}}%
\pgfpathlineto{\pgfqpoint{3.355673in}{3.213435in}}%
\pgfpathlineto{\pgfqpoint{3.367983in}{3.223383in}}%
\pgfpathlineto{\pgfqpoint{3.378955in}{3.229479in}}%
\pgfpathlineto{\pgfqpoint{3.389124in}{3.232646in}}%
\pgfpathlineto{\pgfqpoint{3.398758in}{3.233373in}}%
\pgfpathlineto{\pgfqpoint{3.408392in}{3.231874in}}%
\pgfpathlineto{\pgfqpoint{3.418293in}{3.228044in}}%
\pgfpathlineto{\pgfqpoint{3.428730in}{3.221590in}}%
\pgfpathlineto{\pgfqpoint{3.440237in}{3.211782in}}%
\pgfpathlineto{\pgfqpoint{3.453350in}{3.197527in}}%
\pgfpathlineto{\pgfqpoint{3.468871in}{3.177145in}}%
\pgfpathlineto{\pgfqpoint{3.489209in}{3.146437in}}%
\pgfpathlineto{\pgfqpoint{3.541661in}{3.065514in}}%
\pgfpathlineto{\pgfqpoint{3.557449in}{3.046116in}}%
\pgfpathlineto{\pgfqpoint{3.570830in}{3.033055in}}%
\pgfpathlineto{\pgfqpoint{3.582605in}{3.024570in}}%
\pgfpathlineto{\pgfqpoint{3.593309in}{3.019536in}}%
\pgfpathlineto{\pgfqpoint{3.603210in}{3.017268in}}%
\pgfpathlineto{\pgfqpoint{3.612844in}{3.017314in}}%
\pgfpathlineto{\pgfqpoint{3.622478in}{3.019583in}}%
\pgfpathlineto{\pgfqpoint{3.632647in}{3.024334in}}%
\pgfpathlineto{\pgfqpoint{3.643619in}{3.032052in}}%
\pgfpathlineto{\pgfqpoint{3.655661in}{3.043375in}}%
\pgfpathlineto{\pgfqpoint{3.669577in}{3.059707in}}%
\pgfpathlineto{\pgfqpoint{3.686436in}{3.083182in}}%
\pgfpathlineto{\pgfqpoint{3.711324in}{3.122188in}}%
\pgfpathlineto{\pgfqpoint{3.746113in}{3.176196in}}%
\pgfpathlineto{\pgfqpoint{3.763240in}{3.198644in}}%
\pgfpathlineto{\pgfqpoint{3.777155in}{3.213435in}}%
\pgfpathlineto{\pgfqpoint{3.789465in}{3.223383in}}%
\pgfpathlineto{\pgfqpoint{3.800437in}{3.229479in}}%
\pgfpathlineto{\pgfqpoint{3.810606in}{3.232646in}}%
\pgfpathlineto{\pgfqpoint{3.820240in}{3.233373in}}%
\pgfpathlineto{\pgfqpoint{3.829874in}{3.231874in}}%
\pgfpathlineto{\pgfqpoint{3.839776in}{3.228044in}}%
\pgfpathlineto{\pgfqpoint{3.850212in}{3.221590in}}%
\pgfpathlineto{\pgfqpoint{3.861719in}{3.211782in}}%
\pgfpathlineto{\pgfqpoint{3.874832in}{3.197527in}}%
\pgfpathlineto{\pgfqpoint{3.890354in}{3.177145in}}%
\pgfpathlineto{\pgfqpoint{3.910692in}{3.146437in}}%
\pgfpathlineto{\pgfqpoint{3.963143in}{3.065514in}}%
\pgfpathlineto{\pgfqpoint{3.978932in}{3.046116in}}%
\pgfpathlineto{\pgfqpoint{3.992312in}{3.033055in}}%
\pgfpathlineto{\pgfqpoint{4.004087in}{3.024570in}}%
\pgfpathlineto{\pgfqpoint{4.014791in}{3.019536in}}%
\pgfpathlineto{\pgfqpoint{4.024693in}{3.017268in}}%
\pgfpathlineto{\pgfqpoint{4.034327in}{3.017314in}}%
\pgfpathlineto{\pgfqpoint{4.043960in}{3.019583in}}%
\pgfpathlineto{\pgfqpoint{4.054130in}{3.024334in}}%
\pgfpathlineto{\pgfqpoint{4.065101in}{3.032052in}}%
\pgfpathlineto{\pgfqpoint{4.077144in}{3.043375in}}%
\pgfpathlineto{\pgfqpoint{4.091059in}{3.059707in}}%
\pgfpathlineto{\pgfqpoint{4.107919in}{3.083182in}}%
\pgfpathlineto{\pgfqpoint{4.132806in}{3.122188in}}%
\pgfpathlineto{\pgfqpoint{4.167595in}{3.176196in}}%
\pgfpathlineto{\pgfqpoint{4.184722in}{3.198644in}}%
\pgfpathlineto{\pgfqpoint{4.198638in}{3.213435in}}%
\pgfpathlineto{\pgfqpoint{4.210948in}{3.223383in}}%
\pgfpathlineto{\pgfqpoint{4.221920in}{3.229479in}}%
\pgfpathlineto{\pgfqpoint{4.232089in}{3.232646in}}%
\pgfpathlineto{\pgfqpoint{4.241723in}{3.233373in}}%
\pgfpathlineto{\pgfqpoint{4.251356in}{3.231874in}}%
\pgfpathlineto{\pgfqpoint{4.261258in}{3.228044in}}%
\pgfpathlineto{\pgfqpoint{4.271695in}{3.221590in}}%
\pgfpathlineto{\pgfqpoint{4.283202in}{3.211782in}}%
\pgfpathlineto{\pgfqpoint{4.296315in}{3.197527in}}%
\pgfpathlineto{\pgfqpoint{4.311836in}{3.177145in}}%
\pgfpathlineto{\pgfqpoint{4.332174in}{3.146437in}}%
\pgfpathlineto{\pgfqpoint{4.384625in}{3.065514in}}%
\pgfpathlineto{\pgfqpoint{4.400414in}{3.046116in}}%
\pgfpathlineto{\pgfqpoint{4.413794in}{3.033055in}}%
\pgfpathlineto{\pgfqpoint{4.425569in}{3.024570in}}%
\pgfpathlineto{\pgfqpoint{4.436273in}{3.019536in}}%
\pgfpathlineto{\pgfqpoint{4.446175in}{3.017268in}}%
\pgfpathlineto{\pgfqpoint{4.455809in}{3.017314in}}%
\pgfpathlineto{\pgfqpoint{4.465443in}{3.019583in}}%
\pgfpathlineto{\pgfqpoint{4.475612in}{3.024334in}}%
\pgfpathlineto{\pgfqpoint{4.486584in}{3.032052in}}%
\pgfpathlineto{\pgfqpoint{4.498626in}{3.043375in}}%
\pgfpathlineto{\pgfqpoint{4.512542in}{3.059707in}}%
\pgfpathlineto{\pgfqpoint{4.529401in}{3.083182in}}%
\pgfpathlineto{\pgfqpoint{4.554289in}{3.122188in}}%
\pgfpathlineto{\pgfqpoint{4.589078in}{3.176196in}}%
\pgfpathlineto{\pgfqpoint{4.606204in}{3.198644in}}%
\pgfpathlineto{\pgfqpoint{4.620120in}{3.213435in}}%
\pgfpathlineto{\pgfqpoint{4.632430in}{3.223383in}}%
\pgfpathlineto{\pgfqpoint{4.643402in}{3.229479in}}%
\pgfpathlineto{\pgfqpoint{4.653571in}{3.232646in}}%
\pgfpathlineto{\pgfqpoint{4.663205in}{3.233373in}}%
\pgfpathlineto{\pgfqpoint{4.672839in}{3.231874in}}%
\pgfpathlineto{\pgfqpoint{4.682740in}{3.228044in}}%
\pgfpathlineto{\pgfqpoint{4.693177in}{3.221590in}}%
\pgfpathlineto{\pgfqpoint{4.704684in}{3.211782in}}%
\pgfpathlineto{\pgfqpoint{4.717797in}{3.197527in}}%
\pgfpathlineto{\pgfqpoint{4.733318in}{3.177145in}}%
\pgfpathlineto{\pgfqpoint{4.753656in}{3.146437in}}%
\pgfpathlineto{\pgfqpoint{4.806107in}{3.065514in}}%
\pgfpathlineto{\pgfqpoint{4.821896in}{3.046116in}}%
\pgfpathlineto{\pgfqpoint{4.835277in}{3.033055in}}%
\pgfpathlineto{\pgfqpoint{4.847051in}{3.024570in}}%
\pgfpathlineto{\pgfqpoint{4.857756in}{3.019536in}}%
\pgfpathlineto{\pgfqpoint{4.867657in}{3.017268in}}%
\pgfpathlineto{\pgfqpoint{4.877291in}{3.017314in}}%
\pgfpathlineto{\pgfqpoint{4.886925in}{3.019583in}}%
\pgfpathlineto{\pgfqpoint{4.897094in}{3.024334in}}%
\pgfpathlineto{\pgfqpoint{4.908066in}{3.032052in}}%
\pgfpathlineto{\pgfqpoint{4.920108in}{3.043375in}}%
\pgfpathlineto{\pgfqpoint{4.934024in}{3.059707in}}%
\pgfpathlineto{\pgfqpoint{4.950883in}{3.083182in}}%
\pgfpathlineto{\pgfqpoint{4.975771in}{3.122188in}}%
\pgfpathlineto{\pgfqpoint{5.010560in}{3.176196in}}%
\pgfpathlineto{\pgfqpoint{5.027687in}{3.198644in}}%
\pgfpathlineto{\pgfqpoint{5.041602in}{3.213435in}}%
\pgfpathlineto{\pgfqpoint{5.053912in}{3.223383in}}%
\pgfpathlineto{\pgfqpoint{5.064884in}{3.229479in}}%
\pgfpathlineto{\pgfqpoint{5.075053in}{3.232646in}}%
\pgfpathlineto{\pgfqpoint{5.084687in}{3.233373in}}%
\pgfpathlineto{\pgfqpoint{5.094321in}{3.231874in}}%
\pgfpathlineto{\pgfqpoint{5.104223in}{3.228044in}}%
\pgfpathlineto{\pgfqpoint{5.114659in}{3.221590in}}%
\pgfpathlineto{\pgfqpoint{5.126166in}{3.211782in}}%
\pgfpathlineto{\pgfqpoint{5.139279in}{3.197527in}}%
\pgfpathlineto{\pgfqpoint{5.154800in}{3.177145in}}%
\pgfpathlineto{\pgfqpoint{5.175139in}{3.146437in}}%
\pgfpathlineto{\pgfqpoint{5.227590in}{3.065514in}}%
\pgfpathlineto{\pgfqpoint{5.243379in}{3.046116in}}%
\pgfpathlineto{\pgfqpoint{5.256759in}{3.033055in}}%
\pgfpathlineto{\pgfqpoint{5.268534in}{3.024570in}}%
\pgfpathlineto{\pgfqpoint{5.279238in}{3.019536in}}%
\pgfpathlineto{\pgfqpoint{5.289140in}{3.017268in}}%
\pgfpathlineto{\pgfqpoint{5.298773in}{3.017314in}}%
\pgfpathlineto{\pgfqpoint{5.308407in}{3.019583in}}%
\pgfpathlineto{\pgfqpoint{5.318576in}{3.024334in}}%
\pgfpathlineto{\pgfqpoint{5.329548in}{3.032052in}}%
\pgfpathlineto{\pgfqpoint{5.341591in}{3.043375in}}%
\pgfpathlineto{\pgfqpoint{5.355506in}{3.059707in}}%
\pgfpathlineto{\pgfqpoint{5.372366in}{3.083182in}}%
\pgfpathlineto{\pgfqpoint{5.397253in}{3.122188in}}%
\pgfpathlineto{\pgfqpoint{5.432042in}{3.176196in}}%
\pgfpathlineto{\pgfqpoint{5.449169in}{3.198644in}}%
\pgfpathlineto{\pgfqpoint{5.463085in}{3.213435in}}%
\pgfpathlineto{\pgfqpoint{5.475395in}{3.223383in}}%
\pgfpathlineto{\pgfqpoint{5.486367in}{3.229479in}}%
\pgfpathlineto{\pgfqpoint{5.496536in}{3.232646in}}%
\pgfpathlineto{\pgfqpoint{5.506170in}{3.233373in}}%
\pgfpathlineto{\pgfqpoint{5.515803in}{3.231874in}}%
\pgfpathlineto{\pgfqpoint{5.525705in}{3.228044in}}%
\pgfpathlineto{\pgfqpoint{5.536142in}{3.221590in}}%
\pgfpathlineto{\pgfqpoint{5.547649in}{3.211782in}}%
\pgfpathlineto{\pgfqpoint{5.560762in}{3.197527in}}%
\pgfpathlineto{\pgfqpoint{5.576283in}{3.177145in}}%
\pgfpathlineto{\pgfqpoint{5.596621in}{3.146437in}}%
\pgfpathlineto{\pgfqpoint{5.649072in}{3.065514in}}%
\pgfpathlineto{\pgfqpoint{5.664861in}{3.046116in}}%
\pgfpathlineto{\pgfqpoint{5.678241in}{3.033055in}}%
\pgfpathlineto{\pgfqpoint{5.690016in}{3.024570in}}%
\pgfpathlineto{\pgfqpoint{5.700720in}{3.019536in}}%
\pgfpathlineto{\pgfqpoint{5.710622in}{3.017268in}}%
\pgfpathlineto{\pgfqpoint{5.720256in}{3.017314in}}%
\pgfpathlineto{\pgfqpoint{5.729890in}{3.019583in}}%
\pgfpathlineto{\pgfqpoint{5.740059in}{3.024334in}}%
\pgfpathlineto{\pgfqpoint{5.751031in}{3.032052in}}%
\pgfpathlineto{\pgfqpoint{5.763073in}{3.043375in}}%
\pgfpathlineto{\pgfqpoint{5.776989in}{3.059707in}}%
\pgfpathlineto{\pgfqpoint{5.793848in}{3.083182in}}%
\pgfpathlineto{\pgfqpoint{5.818735in}{3.122188in}}%
\pgfpathlineto{\pgfqpoint{5.853524in}{3.176196in}}%
\pgfpathlineto{\pgfqpoint{5.870651in}{3.198644in}}%
\pgfpathlineto{\pgfqpoint{5.884567in}{3.213435in}}%
\pgfpathlineto{\pgfqpoint{5.896877in}{3.223383in}}%
\pgfpathlineto{\pgfqpoint{5.907849in}{3.229479in}}%
\pgfpathlineto{\pgfqpoint{5.918018in}{3.232646in}}%
\pgfpathlineto{\pgfqpoint{5.927652in}{3.233373in}}%
\pgfpathlineto{\pgfqpoint{5.937286in}{3.231874in}}%
\pgfpathlineto{\pgfqpoint{5.947187in}{3.228044in}}%
\pgfpathlineto{\pgfqpoint{5.957624in}{3.221590in}}%
\pgfpathlineto{\pgfqpoint{5.969131in}{3.211782in}}%
\pgfpathlineto{\pgfqpoint{5.982244in}{3.197527in}}%
\pgfpathlineto{\pgfqpoint{5.997765in}{3.177145in}}%
\pgfpathlineto{\pgfqpoint{6.018103in}{3.146437in}}%
\pgfpathlineto{\pgfqpoint{6.070554in}{3.065514in}}%
\pgfpathlineto{\pgfqpoint{6.086343in}{3.046116in}}%
\pgfpathlineto{\pgfqpoint{6.099724in}{3.033055in}}%
\pgfpathlineto{\pgfqpoint{6.111498in}{3.024570in}}%
\pgfpathlineto{\pgfqpoint{6.122203in}{3.019536in}}%
\pgfpathlineto{\pgfqpoint{6.132104in}{3.017268in}}%
\pgfpathlineto{\pgfqpoint{6.141738in}{3.017314in}}%
\pgfpathlineto{\pgfqpoint{6.151372in}{3.019583in}}%
\pgfpathlineto{\pgfqpoint{6.161541in}{3.024334in}}%
\pgfpathlineto{\pgfqpoint{6.172513in}{3.032052in}}%
\pgfpathlineto{\pgfqpoint{6.184555in}{3.043375in}}%
\pgfpathlineto{\pgfqpoint{6.198471in}{3.059707in}}%
\pgfpathlineto{\pgfqpoint{6.215330in}{3.083182in}}%
\pgfpathlineto{\pgfqpoint{6.240218in}{3.122188in}}%
\pgfpathlineto{\pgfqpoint{6.275007in}{3.176196in}}%
\pgfpathlineto{\pgfqpoint{6.292134in}{3.198644in}}%
\pgfpathlineto{\pgfqpoint{6.306049in}{3.213435in}}%
\pgfpathlineto{\pgfqpoint{6.318359in}{3.223383in}}%
\pgfpathlineto{\pgfqpoint{6.329331in}{3.229479in}}%
\pgfpathlineto{\pgfqpoint{6.339500in}{3.232646in}}%
\pgfpathlineto{\pgfqpoint{6.349134in}{3.233373in}}%
\pgfpathlineto{\pgfqpoint{6.358768in}{3.231874in}}%
\pgfpathlineto{\pgfqpoint{6.368670in}{3.228044in}}%
\pgfpathlineto{\pgfqpoint{6.379106in}{3.221590in}}%
\pgfpathlineto{\pgfqpoint{6.390613in}{3.211782in}}%
\pgfpathlineto{\pgfqpoint{6.403726in}{3.197527in}}%
\pgfpathlineto{\pgfqpoint{6.419247in}{3.177145in}}%
\pgfpathlineto{\pgfqpoint{6.439586in}{3.146437in}}%
\pgfpathlineto{\pgfqpoint{6.492037in}{3.065514in}}%
\pgfpathlineto{\pgfqpoint{6.507826in}{3.046116in}}%
\pgfpathlineto{\pgfqpoint{6.521206in}{3.033055in}}%
\pgfpathlineto{\pgfqpoint{6.532981in}{3.024570in}}%
\pgfpathlineto{\pgfqpoint{6.543685in}{3.019536in}}%
\pgfpathlineto{\pgfqpoint{6.553587in}{3.017268in}}%
\pgfpathlineto{\pgfqpoint{6.563220in}{3.017314in}}%
\pgfpathlineto{\pgfqpoint{6.572854in}{3.019583in}}%
\pgfpathlineto{\pgfqpoint{6.583023in}{3.024334in}}%
\pgfpathlineto{\pgfqpoint{6.593995in}{3.032052in}}%
\pgfpathlineto{\pgfqpoint{6.606038in}{3.043375in}}%
\pgfpathlineto{\pgfqpoint{6.619953in}{3.059707in}}%
\pgfpathlineto{\pgfqpoint{6.636813in}{3.083182in}}%
\pgfpathlineto{\pgfqpoint{6.661700in}{3.122188in}}%
\pgfpathlineto{\pgfqpoint{6.663306in}{3.124778in}}%
\pgfpathlineto{\pgfqpoint{6.663306in}{3.124778in}}%
\pgfusepath{stroke}%
\end{pgfscope}%
\begin{pgfscope}%
\pgfpathrectangle{\pgfqpoint{0.467797in}{2.292089in}}{\pgfqpoint{6.490533in}{1.666241in}}%
\pgfusepath{clip}%
\pgfsetrectcap%
\pgfsetroundjoin%
\pgfsetlinewidth{1.505625pt}%
\definecolor{currentstroke}{rgb}{0.580392,0.403922,0.741176}%
\pgfsetstrokecolor{currentstroke}%
\pgfsetdash{}{0pt}%
\pgfpathmoveto{\pgfqpoint{0.762821in}{3.125209in}}%
\pgfpathlineto{\pgfqpoint{0.790652in}{3.168001in}}%
\pgfpathlineto{\pgfqpoint{0.805103in}{3.186184in}}%
\pgfpathlineto{\pgfqpoint{0.816878in}{3.197616in}}%
\pgfpathlineto{\pgfqpoint{0.827047in}{3.204551in}}%
\pgfpathlineto{\pgfqpoint{0.836145in}{3.208214in}}%
\pgfpathlineto{\pgfqpoint{0.844709in}{3.209363in}}%
\pgfpathlineto{\pgfqpoint{0.853272in}{3.208250in}}%
\pgfpathlineto{\pgfqpoint{0.862103in}{3.204765in}}%
\pgfpathlineto{\pgfqpoint{0.871470in}{3.198587in}}%
\pgfpathlineto{\pgfqpoint{0.881906in}{3.188937in}}%
\pgfpathlineto{\pgfqpoint{0.894216in}{3.174324in}}%
\pgfpathlineto{\pgfqpoint{0.909738in}{3.152136in}}%
\pgfpathlineto{\pgfqpoint{0.967006in}{3.066510in}}%
\pgfpathlineto{\pgfqpoint{0.979048in}{3.054272in}}%
\pgfpathlineto{\pgfqpoint{0.989485in}{3.046690in}}%
\pgfpathlineto{\pgfqpoint{0.998851in}{3.042546in}}%
\pgfpathlineto{\pgfqpoint{1.007682in}{3.041071in}}%
\pgfpathlineto{\pgfqpoint{1.016246in}{3.041938in}}%
\pgfpathlineto{\pgfqpoint{1.024809in}{3.045042in}}%
\pgfpathlineto{\pgfqpoint{1.034175in}{3.050901in}}%
\pgfpathlineto{\pgfqpoint{1.044612in}{3.060231in}}%
\pgfpathlineto{\pgfqpoint{1.056654in}{3.074187in}}%
\pgfpathlineto{\pgfqpoint{1.071640in}{3.095236in}}%
\pgfpathlineto{\pgfqpoint{1.097598in}{3.136399in}}%
\pgfpathlineto{\pgfqpoint{1.119810in}{3.169845in}}%
\pgfpathlineto{\pgfqpoint{1.133993in}{3.187361in}}%
\pgfpathlineto{\pgfqpoint{1.145500in}{3.198268in}}%
\pgfpathlineto{\pgfqpoint{1.155669in}{3.204973in}}%
\pgfpathlineto{\pgfqpoint{1.164768in}{3.208418in}}%
\pgfpathlineto{\pgfqpoint{1.173331in}{3.209355in}}%
\pgfpathlineto{\pgfqpoint{1.181895in}{3.208030in}}%
\pgfpathlineto{\pgfqpoint{1.190726in}{3.204333in}}%
\pgfpathlineto{\pgfqpoint{1.200092in}{3.197944in}}%
\pgfpathlineto{\pgfqpoint{1.210796in}{3.187796in}}%
\pgfpathlineto{\pgfqpoint{1.223106in}{3.172912in}}%
\pgfpathlineto{\pgfqpoint{1.238895in}{3.150083in}}%
\pgfpathlineto{\pgfqpoint{1.292952in}{3.068713in}}%
\pgfpathlineto{\pgfqpoint{1.305262in}{3.055699in}}%
\pgfpathlineto{\pgfqpoint{1.315966in}{3.047493in}}%
\pgfpathlineto{\pgfqpoint{1.325333in}{3.042978in}}%
\pgfpathlineto{\pgfqpoint{1.334164in}{3.041141in}}%
\pgfpathlineto{\pgfqpoint{1.342727in}{3.041654in}}%
\pgfpathlineto{\pgfqpoint{1.351291in}{3.044412in}}%
\pgfpathlineto{\pgfqpoint{1.360389in}{3.049720in}}%
\pgfpathlineto{\pgfqpoint{1.370558in}{3.058353in}}%
\pgfpathlineto{\pgfqpoint{1.382065in}{3.071153in}}%
\pgfpathlineto{\pgfqpoint{1.396249in}{3.090456in}}%
\pgfpathlineto{\pgfqpoint{1.417657in}{3.123915in}}%
\pgfpathlineto{\pgfqpoint{1.446291in}{3.168001in}}%
\pgfpathlineto{\pgfqpoint{1.460742in}{3.186184in}}%
\pgfpathlineto{\pgfqpoint{1.472517in}{3.197616in}}%
\pgfpathlineto{\pgfqpoint{1.482686in}{3.204551in}}%
\pgfpathlineto{\pgfqpoint{1.491785in}{3.208214in}}%
\pgfpathlineto{\pgfqpoint{1.500348in}{3.209363in}}%
\pgfpathlineto{\pgfqpoint{1.508912in}{3.208250in}}%
\pgfpathlineto{\pgfqpoint{1.517743in}{3.204765in}}%
\pgfpathlineto{\pgfqpoint{1.527109in}{3.198587in}}%
\pgfpathlineto{\pgfqpoint{1.537546in}{3.188937in}}%
\pgfpathlineto{\pgfqpoint{1.549856in}{3.174324in}}%
\pgfpathlineto{\pgfqpoint{1.565377in}{3.152136in}}%
\pgfpathlineto{\pgfqpoint{1.622645in}{3.066510in}}%
\pgfpathlineto{\pgfqpoint{1.634687in}{3.054272in}}%
\pgfpathlineto{\pgfqpoint{1.645124in}{3.046690in}}%
\pgfpathlineto{\pgfqpoint{1.654490in}{3.042546in}}%
\pgfpathlineto{\pgfqpoint{1.663321in}{3.041071in}}%
\pgfpathlineto{\pgfqpoint{1.671885in}{3.041938in}}%
\pgfpathlineto{\pgfqpoint{1.680448in}{3.045042in}}%
\pgfpathlineto{\pgfqpoint{1.689814in}{3.050901in}}%
\pgfpathlineto{\pgfqpoint{1.700251in}{3.060231in}}%
\pgfpathlineto{\pgfqpoint{1.712293in}{3.074187in}}%
\pgfpathlineto{\pgfqpoint{1.727279in}{3.095236in}}%
\pgfpathlineto{\pgfqpoint{1.753237in}{3.136399in}}%
\pgfpathlineto{\pgfqpoint{1.775449in}{3.169845in}}%
\pgfpathlineto{\pgfqpoint{1.789632in}{3.187361in}}%
\pgfpathlineto{\pgfqpoint{1.801139in}{3.198268in}}%
\pgfpathlineto{\pgfqpoint{1.811308in}{3.204973in}}%
\pgfpathlineto{\pgfqpoint{1.820407in}{3.208418in}}%
\pgfpathlineto{\pgfqpoint{1.828970in}{3.209355in}}%
\pgfpathlineto{\pgfqpoint{1.837534in}{3.208030in}}%
\pgfpathlineto{\pgfqpoint{1.846365in}{3.204333in}}%
\pgfpathlineto{\pgfqpoint{1.855731in}{3.197944in}}%
\pgfpathlineto{\pgfqpoint{1.866436in}{3.187796in}}%
\pgfpathlineto{\pgfqpoint{1.878746in}{3.172912in}}%
\pgfpathlineto{\pgfqpoint{1.894534in}{3.150083in}}%
\pgfpathlineto{\pgfqpoint{1.948591in}{3.068713in}}%
\pgfpathlineto{\pgfqpoint{1.960901in}{3.055699in}}%
\pgfpathlineto{\pgfqpoint{1.971605in}{3.047493in}}%
\pgfpathlineto{\pgfqpoint{1.980972in}{3.042978in}}%
\pgfpathlineto{\pgfqpoint{1.989803in}{3.041141in}}%
\pgfpathlineto{\pgfqpoint{1.998366in}{3.041654in}}%
\pgfpathlineto{\pgfqpoint{2.006930in}{3.044412in}}%
\pgfpathlineto{\pgfqpoint{2.016028in}{3.049720in}}%
\pgfpathlineto{\pgfqpoint{2.026197in}{3.058353in}}%
\pgfpathlineto{\pgfqpoint{2.037705in}{3.071153in}}%
\pgfpathlineto{\pgfqpoint{2.051888in}{3.090456in}}%
\pgfpathlineto{\pgfqpoint{2.073296in}{3.123915in}}%
\pgfpathlineto{\pgfqpoint{2.101930in}{3.168001in}}%
\pgfpathlineto{\pgfqpoint{2.116381in}{3.186184in}}%
\pgfpathlineto{\pgfqpoint{2.128156in}{3.197616in}}%
\pgfpathlineto{\pgfqpoint{2.138325in}{3.204551in}}%
\pgfpathlineto{\pgfqpoint{2.147424in}{3.208214in}}%
\pgfpathlineto{\pgfqpoint{2.155987in}{3.209363in}}%
\pgfpathlineto{\pgfqpoint{2.164551in}{3.208250in}}%
\pgfpathlineto{\pgfqpoint{2.173382in}{3.204765in}}%
\pgfpathlineto{\pgfqpoint{2.182748in}{3.198587in}}%
\pgfpathlineto{\pgfqpoint{2.193185in}{3.188937in}}%
\pgfpathlineto{\pgfqpoint{2.205495in}{3.174324in}}%
\pgfpathlineto{\pgfqpoint{2.221016in}{3.152136in}}%
\pgfpathlineto{\pgfqpoint{2.278284in}{3.066510in}}%
\pgfpathlineto{\pgfqpoint{2.290326in}{3.054272in}}%
\pgfpathlineto{\pgfqpoint{2.300763in}{3.046690in}}%
\pgfpathlineto{\pgfqpoint{2.310129in}{3.042546in}}%
\pgfpathlineto{\pgfqpoint{2.318960in}{3.041071in}}%
\pgfpathlineto{\pgfqpoint{2.327524in}{3.041938in}}%
\pgfpathlineto{\pgfqpoint{2.336087in}{3.045042in}}%
\pgfpathlineto{\pgfqpoint{2.345454in}{3.050901in}}%
\pgfpathlineto{\pgfqpoint{2.355890in}{3.060231in}}%
\pgfpathlineto{\pgfqpoint{2.367933in}{3.074187in}}%
\pgfpathlineto{\pgfqpoint{2.382919in}{3.095236in}}%
\pgfpathlineto{\pgfqpoint{2.408877in}{3.136399in}}%
\pgfpathlineto{\pgfqpoint{2.431088in}{3.169845in}}%
\pgfpathlineto{\pgfqpoint{2.445271in}{3.187361in}}%
\pgfpathlineto{\pgfqpoint{2.456778in}{3.198268in}}%
\pgfpathlineto{\pgfqpoint{2.466948in}{3.204973in}}%
\pgfpathlineto{\pgfqpoint{2.476046in}{3.208418in}}%
\pgfpathlineto{\pgfqpoint{2.484610in}{3.209355in}}%
\pgfpathlineto{\pgfqpoint{2.493173in}{3.208030in}}%
\pgfpathlineto{\pgfqpoint{2.502004in}{3.204333in}}%
\pgfpathlineto{\pgfqpoint{2.511370in}{3.197944in}}%
\pgfpathlineto{\pgfqpoint{2.522075in}{3.187796in}}%
\pgfpathlineto{\pgfqpoint{2.534385in}{3.172912in}}%
\pgfpathlineto{\pgfqpoint{2.550174in}{3.150083in}}%
\pgfpathlineto{\pgfqpoint{2.604230in}{3.068713in}}%
\pgfpathlineto{\pgfqpoint{2.616540in}{3.055699in}}%
\pgfpathlineto{\pgfqpoint{2.627245in}{3.047493in}}%
\pgfpathlineto{\pgfqpoint{2.636611in}{3.042978in}}%
\pgfpathlineto{\pgfqpoint{2.645442in}{3.041141in}}%
\pgfpathlineto{\pgfqpoint{2.654005in}{3.041654in}}%
\pgfpathlineto{\pgfqpoint{2.662569in}{3.044412in}}%
\pgfpathlineto{\pgfqpoint{2.671667in}{3.049720in}}%
\pgfpathlineto{\pgfqpoint{2.681837in}{3.058353in}}%
\pgfpathlineto{\pgfqpoint{2.693344in}{3.071153in}}%
\pgfpathlineto{\pgfqpoint{2.707527in}{3.090456in}}%
\pgfpathlineto{\pgfqpoint{2.728936in}{3.123915in}}%
\pgfpathlineto{\pgfqpoint{2.757570in}{3.168001in}}%
\pgfpathlineto{\pgfqpoint{2.772020in}{3.186184in}}%
\pgfpathlineto{\pgfqpoint{2.783795in}{3.197616in}}%
\pgfpathlineto{\pgfqpoint{2.793964in}{3.204551in}}%
\pgfpathlineto{\pgfqpoint{2.803063in}{3.208214in}}%
\pgfpathlineto{\pgfqpoint{2.811626in}{3.209363in}}%
\pgfpathlineto{\pgfqpoint{2.820190in}{3.208250in}}%
\pgfpathlineto{\pgfqpoint{2.829021in}{3.204765in}}%
\pgfpathlineto{\pgfqpoint{2.838387in}{3.198587in}}%
\pgfpathlineto{\pgfqpoint{2.848824in}{3.188937in}}%
\pgfpathlineto{\pgfqpoint{2.861134in}{3.174324in}}%
\pgfpathlineto{\pgfqpoint{2.876655in}{3.152136in}}%
\pgfpathlineto{\pgfqpoint{2.933923in}{3.066510in}}%
\pgfpathlineto{\pgfqpoint{2.945966in}{3.054272in}}%
\pgfpathlineto{\pgfqpoint{2.956402in}{3.046690in}}%
\pgfpathlineto{\pgfqpoint{2.965768in}{3.042546in}}%
\pgfpathlineto{\pgfqpoint{2.974600in}{3.041071in}}%
\pgfpathlineto{\pgfqpoint{2.983163in}{3.041938in}}%
\pgfpathlineto{\pgfqpoint{2.991726in}{3.045042in}}%
\pgfpathlineto{\pgfqpoint{3.001093in}{3.050901in}}%
\pgfpathlineto{\pgfqpoint{3.011529in}{3.060231in}}%
\pgfpathlineto{\pgfqpoint{3.023572in}{3.074187in}}%
\pgfpathlineto{\pgfqpoint{3.038558in}{3.095236in}}%
\pgfpathlineto{\pgfqpoint{3.064516in}{3.136399in}}%
\pgfpathlineto{\pgfqpoint{3.086727in}{3.169845in}}%
\pgfpathlineto{\pgfqpoint{3.100910in}{3.187361in}}%
\pgfpathlineto{\pgfqpoint{3.112418in}{3.198268in}}%
\pgfpathlineto{\pgfqpoint{3.122587in}{3.204973in}}%
\pgfpathlineto{\pgfqpoint{3.131685in}{3.208418in}}%
\pgfpathlineto{\pgfqpoint{3.140249in}{3.209355in}}%
\pgfpathlineto{\pgfqpoint{3.148812in}{3.208030in}}%
\pgfpathlineto{\pgfqpoint{3.157643in}{3.204333in}}%
\pgfpathlineto{\pgfqpoint{3.167010in}{3.197944in}}%
\pgfpathlineto{\pgfqpoint{3.177714in}{3.187796in}}%
\pgfpathlineto{\pgfqpoint{3.190024in}{3.172912in}}%
\pgfpathlineto{\pgfqpoint{3.205813in}{3.150083in}}%
\pgfpathlineto{\pgfqpoint{3.259869in}{3.068713in}}%
\pgfpathlineto{\pgfqpoint{3.272179in}{3.055699in}}%
\pgfpathlineto{\pgfqpoint{3.282884in}{3.047493in}}%
\pgfpathlineto{\pgfqpoint{3.292250in}{3.042978in}}%
\pgfpathlineto{\pgfqpoint{3.301081in}{3.041141in}}%
\pgfpathlineto{\pgfqpoint{3.309645in}{3.041654in}}%
\pgfpathlineto{\pgfqpoint{3.318208in}{3.044412in}}%
\pgfpathlineto{\pgfqpoint{3.327307in}{3.049720in}}%
\pgfpathlineto{\pgfqpoint{3.337476in}{3.058353in}}%
\pgfpathlineto{\pgfqpoint{3.348983in}{3.071153in}}%
\pgfpathlineto{\pgfqpoint{3.363166in}{3.090456in}}%
\pgfpathlineto{\pgfqpoint{3.384575in}{3.123915in}}%
\pgfpathlineto{\pgfqpoint{3.413209in}{3.168001in}}%
\pgfpathlineto{\pgfqpoint{3.427660in}{3.186184in}}%
\pgfpathlineto{\pgfqpoint{3.439434in}{3.197616in}}%
\pgfpathlineto{\pgfqpoint{3.449603in}{3.204551in}}%
\pgfpathlineto{\pgfqpoint{3.458702in}{3.208214in}}%
\pgfpathlineto{\pgfqpoint{3.467266in}{3.209363in}}%
\pgfpathlineto{\pgfqpoint{3.475829in}{3.208250in}}%
\pgfpathlineto{\pgfqpoint{3.484660in}{3.204765in}}%
\pgfpathlineto{\pgfqpoint{3.494026in}{3.198587in}}%
\pgfpathlineto{\pgfqpoint{3.504463in}{3.188937in}}%
\pgfpathlineto{\pgfqpoint{3.516773in}{3.174324in}}%
\pgfpathlineto{\pgfqpoint{3.532294in}{3.152136in}}%
\pgfpathlineto{\pgfqpoint{3.589562in}{3.066510in}}%
\pgfpathlineto{\pgfqpoint{3.601605in}{3.054272in}}%
\pgfpathlineto{\pgfqpoint{3.612041in}{3.046690in}}%
\pgfpathlineto{\pgfqpoint{3.621408in}{3.042546in}}%
\pgfpathlineto{\pgfqpoint{3.630239in}{3.041071in}}%
\pgfpathlineto{\pgfqpoint{3.638802in}{3.041938in}}%
\pgfpathlineto{\pgfqpoint{3.647366in}{3.045042in}}%
\pgfpathlineto{\pgfqpoint{3.656732in}{3.050901in}}%
\pgfpathlineto{\pgfqpoint{3.667169in}{3.060231in}}%
\pgfpathlineto{\pgfqpoint{3.679211in}{3.074187in}}%
\pgfpathlineto{\pgfqpoint{3.694197in}{3.095236in}}%
\pgfpathlineto{\pgfqpoint{3.720155in}{3.136399in}}%
\pgfpathlineto{\pgfqpoint{3.742366in}{3.169845in}}%
\pgfpathlineto{\pgfqpoint{3.756550in}{3.187361in}}%
\pgfpathlineto{\pgfqpoint{3.768057in}{3.198268in}}%
\pgfpathlineto{\pgfqpoint{3.778226in}{3.204973in}}%
\pgfpathlineto{\pgfqpoint{3.787325in}{3.208418in}}%
\pgfpathlineto{\pgfqpoint{3.795888in}{3.209355in}}%
\pgfpathlineto{\pgfqpoint{3.804451in}{3.208030in}}%
\pgfpathlineto{\pgfqpoint{3.813282in}{3.204333in}}%
\pgfpathlineto{\pgfqpoint{3.822649in}{3.197944in}}%
\pgfpathlineto{\pgfqpoint{3.833353in}{3.187796in}}%
\pgfpathlineto{\pgfqpoint{3.845663in}{3.172912in}}%
\pgfpathlineto{\pgfqpoint{3.861452in}{3.150083in}}%
\pgfpathlineto{\pgfqpoint{3.915509in}{3.068713in}}%
\pgfpathlineto{\pgfqpoint{3.927819in}{3.055699in}}%
\pgfpathlineto{\pgfqpoint{3.938523in}{3.047493in}}%
\pgfpathlineto{\pgfqpoint{3.947889in}{3.042978in}}%
\pgfpathlineto{\pgfqpoint{3.956720in}{3.041141in}}%
\pgfpathlineto{\pgfqpoint{3.965284in}{3.041654in}}%
\pgfpathlineto{\pgfqpoint{3.973847in}{3.044412in}}%
\pgfpathlineto{\pgfqpoint{3.982946in}{3.049720in}}%
\pgfpathlineto{\pgfqpoint{3.993115in}{3.058353in}}%
\pgfpathlineto{\pgfqpoint{4.004622in}{3.071153in}}%
\pgfpathlineto{\pgfqpoint{4.018805in}{3.090456in}}%
\pgfpathlineto{\pgfqpoint{4.040214in}{3.123915in}}%
\pgfpathlineto{\pgfqpoint{4.068848in}{3.168001in}}%
\pgfpathlineto{\pgfqpoint{4.083299in}{3.186184in}}%
\pgfpathlineto{\pgfqpoint{4.095073in}{3.197616in}}%
\pgfpathlineto{\pgfqpoint{4.105243in}{3.204551in}}%
\pgfpathlineto{\pgfqpoint{4.114341in}{3.208214in}}%
\pgfpathlineto{\pgfqpoint{4.122905in}{3.209363in}}%
\pgfpathlineto{\pgfqpoint{4.131468in}{3.208250in}}%
\pgfpathlineto{\pgfqpoint{4.140299in}{3.204765in}}%
\pgfpathlineto{\pgfqpoint{4.149665in}{3.198587in}}%
\pgfpathlineto{\pgfqpoint{4.160102in}{3.188937in}}%
\pgfpathlineto{\pgfqpoint{4.172412in}{3.174324in}}%
\pgfpathlineto{\pgfqpoint{4.187933in}{3.152136in}}%
\pgfpathlineto{\pgfqpoint{4.245201in}{3.066510in}}%
\pgfpathlineto{\pgfqpoint{4.257244in}{3.054272in}}%
\pgfpathlineto{\pgfqpoint{4.267681in}{3.046690in}}%
\pgfpathlineto{\pgfqpoint{4.277047in}{3.042546in}}%
\pgfpathlineto{\pgfqpoint{4.285878in}{3.041071in}}%
\pgfpathlineto{\pgfqpoint{4.294441in}{3.041938in}}%
\pgfpathlineto{\pgfqpoint{4.303005in}{3.045042in}}%
\pgfpathlineto{\pgfqpoint{4.312371in}{3.050901in}}%
\pgfpathlineto{\pgfqpoint{4.322808in}{3.060231in}}%
\pgfpathlineto{\pgfqpoint{4.334850in}{3.074187in}}%
\pgfpathlineto{\pgfqpoint{4.349836in}{3.095236in}}%
\pgfpathlineto{\pgfqpoint{4.375794in}{3.136399in}}%
\pgfpathlineto{\pgfqpoint{4.398006in}{3.169845in}}%
\pgfpathlineto{\pgfqpoint{4.412189in}{3.187361in}}%
\pgfpathlineto{\pgfqpoint{4.423696in}{3.198268in}}%
\pgfpathlineto{\pgfqpoint{4.433865in}{3.204973in}}%
\pgfpathlineto{\pgfqpoint{4.442964in}{3.208418in}}%
\pgfpathlineto{\pgfqpoint{4.451527in}{3.209355in}}%
\pgfpathlineto{\pgfqpoint{4.460091in}{3.208030in}}%
\pgfpathlineto{\pgfqpoint{4.468922in}{3.204333in}}%
\pgfpathlineto{\pgfqpoint{4.478288in}{3.197944in}}%
\pgfpathlineto{\pgfqpoint{4.488992in}{3.187796in}}%
\pgfpathlineto{\pgfqpoint{4.501302in}{3.172912in}}%
\pgfpathlineto{\pgfqpoint{4.517091in}{3.150083in}}%
\pgfpathlineto{\pgfqpoint{4.571148in}{3.068713in}}%
\pgfpathlineto{\pgfqpoint{4.583458in}{3.055699in}}%
\pgfpathlineto{\pgfqpoint{4.594162in}{3.047493in}}%
\pgfpathlineto{\pgfqpoint{4.603528in}{3.042978in}}%
\pgfpathlineto{\pgfqpoint{4.612359in}{3.041141in}}%
\pgfpathlineto{\pgfqpoint{4.620923in}{3.041654in}}%
\pgfpathlineto{\pgfqpoint{4.629486in}{3.044412in}}%
\pgfpathlineto{\pgfqpoint{4.638585in}{3.049720in}}%
\pgfpathlineto{\pgfqpoint{4.648754in}{3.058353in}}%
\pgfpathlineto{\pgfqpoint{4.660261in}{3.071153in}}%
\pgfpathlineto{\pgfqpoint{4.674444in}{3.090456in}}%
\pgfpathlineto{\pgfqpoint{4.695853in}{3.123915in}}%
\pgfpathlineto{\pgfqpoint{4.724487in}{3.168001in}}%
\pgfpathlineto{\pgfqpoint{4.738938in}{3.186184in}}%
\pgfpathlineto{\pgfqpoint{4.750713in}{3.197616in}}%
\pgfpathlineto{\pgfqpoint{4.760882in}{3.204551in}}%
\pgfpathlineto{\pgfqpoint{4.769980in}{3.208214in}}%
\pgfpathlineto{\pgfqpoint{4.778544in}{3.209363in}}%
\pgfpathlineto{\pgfqpoint{4.787107in}{3.208250in}}%
\pgfpathlineto{\pgfqpoint{4.795938in}{3.204765in}}%
\pgfpathlineto{\pgfqpoint{4.805305in}{3.198587in}}%
\pgfpathlineto{\pgfqpoint{4.815741in}{3.188937in}}%
\pgfpathlineto{\pgfqpoint{4.828051in}{3.174324in}}%
\pgfpathlineto{\pgfqpoint{4.843573in}{3.152136in}}%
\pgfpathlineto{\pgfqpoint{4.900841in}{3.066510in}}%
\pgfpathlineto{\pgfqpoint{4.912883in}{3.054272in}}%
\pgfpathlineto{\pgfqpoint{4.923320in}{3.046690in}}%
\pgfpathlineto{\pgfqpoint{4.932686in}{3.042546in}}%
\pgfpathlineto{\pgfqpoint{4.941517in}{3.041071in}}%
\pgfpathlineto{\pgfqpoint{4.950080in}{3.041938in}}%
\pgfpathlineto{\pgfqpoint{4.958644in}{3.045042in}}%
\pgfpathlineto{\pgfqpoint{4.968010in}{3.050901in}}%
\pgfpathlineto{\pgfqpoint{4.978447in}{3.060231in}}%
\pgfpathlineto{\pgfqpoint{4.990489in}{3.074187in}}%
\pgfpathlineto{\pgfqpoint{5.005475in}{3.095236in}}%
\pgfpathlineto{\pgfqpoint{5.031433in}{3.136399in}}%
\pgfpathlineto{\pgfqpoint{5.053645in}{3.169845in}}%
\pgfpathlineto{\pgfqpoint{5.067828in}{3.187361in}}%
\pgfpathlineto{\pgfqpoint{5.079335in}{3.198268in}}%
\pgfpathlineto{\pgfqpoint{5.089504in}{3.204973in}}%
\pgfpathlineto{\pgfqpoint{5.098603in}{3.208418in}}%
\pgfpathlineto{\pgfqpoint{5.107166in}{3.209355in}}%
\pgfpathlineto{\pgfqpoint{5.115730in}{3.208030in}}%
\pgfpathlineto{\pgfqpoint{5.124561in}{3.204333in}}%
\pgfpathlineto{\pgfqpoint{5.133927in}{3.197944in}}%
\pgfpathlineto{\pgfqpoint{5.144631in}{3.187796in}}%
\pgfpathlineto{\pgfqpoint{5.156941in}{3.172912in}}%
\pgfpathlineto{\pgfqpoint{5.172730in}{3.150083in}}%
\pgfpathlineto{\pgfqpoint{5.226787in}{3.068713in}}%
\pgfpathlineto{\pgfqpoint{5.239097in}{3.055699in}}%
\pgfpathlineto{\pgfqpoint{5.249801in}{3.047493in}}%
\pgfpathlineto{\pgfqpoint{5.259168in}{3.042978in}}%
\pgfpathlineto{\pgfqpoint{5.267999in}{3.041141in}}%
\pgfpathlineto{\pgfqpoint{5.276562in}{3.041654in}}%
\pgfpathlineto{\pgfqpoint{5.285125in}{3.044412in}}%
\pgfpathlineto{\pgfqpoint{5.294224in}{3.049720in}}%
\pgfpathlineto{\pgfqpoint{5.304393in}{3.058353in}}%
\pgfpathlineto{\pgfqpoint{5.315900in}{3.071153in}}%
\pgfpathlineto{\pgfqpoint{5.330084in}{3.090456in}}%
\pgfpathlineto{\pgfqpoint{5.351492in}{3.123915in}}%
\pgfpathlineto{\pgfqpoint{5.380126in}{3.168001in}}%
\pgfpathlineto{\pgfqpoint{5.394577in}{3.186184in}}%
\pgfpathlineto{\pgfqpoint{5.406352in}{3.197616in}}%
\pgfpathlineto{\pgfqpoint{5.416521in}{3.204551in}}%
\pgfpathlineto{\pgfqpoint{5.425620in}{3.208214in}}%
\pgfpathlineto{\pgfqpoint{5.434183in}{3.209363in}}%
\pgfpathlineto{\pgfqpoint{5.442746in}{3.208250in}}%
\pgfpathlineto{\pgfqpoint{5.451578in}{3.204765in}}%
\pgfpathlineto{\pgfqpoint{5.460944in}{3.198587in}}%
\pgfpathlineto{\pgfqpoint{5.471381in}{3.188937in}}%
\pgfpathlineto{\pgfqpoint{5.483690in}{3.174324in}}%
\pgfpathlineto{\pgfqpoint{5.499212in}{3.152136in}}%
\pgfpathlineto{\pgfqpoint{5.556480in}{3.066510in}}%
\pgfpathlineto{\pgfqpoint{5.568522in}{3.054272in}}%
\pgfpathlineto{\pgfqpoint{5.578959in}{3.046690in}}%
\pgfpathlineto{\pgfqpoint{5.588325in}{3.042546in}}%
\pgfpathlineto{\pgfqpoint{5.597156in}{3.041071in}}%
\pgfpathlineto{\pgfqpoint{5.605720in}{3.041938in}}%
\pgfpathlineto{\pgfqpoint{5.614283in}{3.045042in}}%
\pgfpathlineto{\pgfqpoint{5.623649in}{3.050901in}}%
\pgfpathlineto{\pgfqpoint{5.634086in}{3.060231in}}%
\pgfpathlineto{\pgfqpoint{5.646128in}{3.074187in}}%
\pgfpathlineto{\pgfqpoint{5.661114in}{3.095236in}}%
\pgfpathlineto{\pgfqpoint{5.687072in}{3.136399in}}%
\pgfpathlineto{\pgfqpoint{5.709284in}{3.169845in}}%
\pgfpathlineto{\pgfqpoint{5.723467in}{3.187361in}}%
\pgfpathlineto{\pgfqpoint{5.734974in}{3.198268in}}%
\pgfpathlineto{\pgfqpoint{5.745143in}{3.204973in}}%
\pgfpathlineto{\pgfqpoint{5.754242in}{3.208418in}}%
\pgfpathlineto{\pgfqpoint{5.762805in}{3.209355in}}%
\pgfpathlineto{\pgfqpoint{5.771369in}{3.208030in}}%
\pgfpathlineto{\pgfqpoint{5.780200in}{3.204333in}}%
\pgfpathlineto{\pgfqpoint{5.789566in}{3.197944in}}%
\pgfpathlineto{\pgfqpoint{5.800271in}{3.187796in}}%
\pgfpathlineto{\pgfqpoint{5.812580in}{3.172912in}}%
\pgfpathlineto{\pgfqpoint{5.828369in}{3.150083in}}%
\pgfpathlineto{\pgfqpoint{5.882426in}{3.068713in}}%
\pgfpathlineto{\pgfqpoint{5.894736in}{3.055699in}}%
\pgfpathlineto{\pgfqpoint{5.905440in}{3.047493in}}%
\pgfpathlineto{\pgfqpoint{5.914807in}{3.042978in}}%
\pgfpathlineto{\pgfqpoint{5.923638in}{3.041141in}}%
\pgfpathlineto{\pgfqpoint{5.932201in}{3.041654in}}%
\pgfpathlineto{\pgfqpoint{5.940765in}{3.044412in}}%
\pgfpathlineto{\pgfqpoint{5.949863in}{3.049720in}}%
\pgfpathlineto{\pgfqpoint{5.960032in}{3.058353in}}%
\pgfpathlineto{\pgfqpoint{5.971540in}{3.071153in}}%
\pgfpathlineto{\pgfqpoint{5.985723in}{3.090456in}}%
\pgfpathlineto{\pgfqpoint{6.007131in}{3.123915in}}%
\pgfpathlineto{\pgfqpoint{6.035765in}{3.168001in}}%
\pgfpathlineto{\pgfqpoint{6.050216in}{3.186184in}}%
\pgfpathlineto{\pgfqpoint{6.061991in}{3.197616in}}%
\pgfpathlineto{\pgfqpoint{6.072160in}{3.204551in}}%
\pgfpathlineto{\pgfqpoint{6.081259in}{3.208214in}}%
\pgfpathlineto{\pgfqpoint{6.089822in}{3.209363in}}%
\pgfpathlineto{\pgfqpoint{6.098386in}{3.208250in}}%
\pgfpathlineto{\pgfqpoint{6.107217in}{3.204765in}}%
\pgfpathlineto{\pgfqpoint{6.116583in}{3.198587in}}%
\pgfpathlineto{\pgfqpoint{6.127020in}{3.188937in}}%
\pgfpathlineto{\pgfqpoint{6.139330in}{3.174324in}}%
\pgfpathlineto{\pgfqpoint{6.154851in}{3.152136in}}%
\pgfpathlineto{\pgfqpoint{6.212119in}{3.066510in}}%
\pgfpathlineto{\pgfqpoint{6.224161in}{3.054272in}}%
\pgfpathlineto{\pgfqpoint{6.234598in}{3.046690in}}%
\pgfpathlineto{\pgfqpoint{6.243964in}{3.042546in}}%
\pgfpathlineto{\pgfqpoint{6.252795in}{3.041071in}}%
\pgfpathlineto{\pgfqpoint{6.261359in}{3.041938in}}%
\pgfpathlineto{\pgfqpoint{6.269922in}{3.045042in}}%
\pgfpathlineto{\pgfqpoint{6.279289in}{3.050901in}}%
\pgfpathlineto{\pgfqpoint{6.289725in}{3.060231in}}%
\pgfpathlineto{\pgfqpoint{6.301768in}{3.074187in}}%
\pgfpathlineto{\pgfqpoint{6.316754in}{3.095236in}}%
\pgfpathlineto{\pgfqpoint{6.342712in}{3.136399in}}%
\pgfpathlineto{\pgfqpoint{6.364923in}{3.169845in}}%
\pgfpathlineto{\pgfqpoint{6.379106in}{3.187361in}}%
\pgfpathlineto{\pgfqpoint{6.390613in}{3.198268in}}%
\pgfpathlineto{\pgfqpoint{6.400782in}{3.204973in}}%
\pgfpathlineto{\pgfqpoint{6.409881in}{3.208418in}}%
\pgfpathlineto{\pgfqpoint{6.418445in}{3.209355in}}%
\pgfpathlineto{\pgfqpoint{6.427008in}{3.208030in}}%
\pgfpathlineto{\pgfqpoint{6.435839in}{3.204333in}}%
\pgfpathlineto{\pgfqpoint{6.445205in}{3.197944in}}%
\pgfpathlineto{\pgfqpoint{6.455910in}{3.187796in}}%
\pgfpathlineto{\pgfqpoint{6.468220in}{3.172912in}}%
\pgfpathlineto{\pgfqpoint{6.484009in}{3.150083in}}%
\pgfpathlineto{\pgfqpoint{6.538065in}{3.068713in}}%
\pgfpathlineto{\pgfqpoint{6.550375in}{3.055699in}}%
\pgfpathlineto{\pgfqpoint{6.561080in}{3.047493in}}%
\pgfpathlineto{\pgfqpoint{6.570446in}{3.042978in}}%
\pgfpathlineto{\pgfqpoint{6.579277in}{3.041141in}}%
\pgfpathlineto{\pgfqpoint{6.587840in}{3.041654in}}%
\pgfpathlineto{\pgfqpoint{6.596404in}{3.044412in}}%
\pgfpathlineto{\pgfqpoint{6.605502in}{3.049720in}}%
\pgfpathlineto{\pgfqpoint{6.615672in}{3.058353in}}%
\pgfpathlineto{\pgfqpoint{6.627179in}{3.071153in}}%
\pgfpathlineto{\pgfqpoint{6.641362in}{3.090456in}}%
\pgfpathlineto{\pgfqpoint{6.662771in}{3.123915in}}%
\pgfpathlineto{\pgfqpoint{6.663306in}{3.124778in}}%
\pgfpathlineto{\pgfqpoint{6.663306in}{3.124778in}}%
\pgfusepath{stroke}%
\end{pgfscope}%
\begin{pgfscope}%
\pgfpathrectangle{\pgfqpoint{0.467797in}{2.292089in}}{\pgfqpoint{6.490533in}{1.666241in}}%
\pgfusepath{clip}%
\pgfsetrectcap%
\pgfsetroundjoin%
\pgfsetlinewidth{1.505625pt}%
\definecolor{currentstroke}{rgb}{0.549020,0.337255,0.294118}%
\pgfsetstrokecolor{currentstroke}%
\pgfsetdash{}{0pt}%
\pgfpathmoveto{\pgfqpoint{0.762821in}{3.125209in}}%
\pgfpathlineto{\pgfqpoint{0.787173in}{3.162392in}}%
\pgfpathlineto{\pgfqpoint{0.799751in}{3.177618in}}%
\pgfpathlineto{\pgfqpoint{0.809920in}{3.186675in}}%
\pgfpathlineto{\pgfqpoint{0.818751in}{3.191738in}}%
\pgfpathlineto{\pgfqpoint{0.826779in}{3.193881in}}%
\pgfpathlineto{\pgfqpoint{0.834540in}{3.193652in}}%
\pgfpathlineto{\pgfqpoint{0.842300in}{3.191166in}}%
\pgfpathlineto{\pgfqpoint{0.850596in}{3.186109in}}%
\pgfpathlineto{\pgfqpoint{0.859963in}{3.177656in}}%
\pgfpathlineto{\pgfqpoint{0.870934in}{3.164593in}}%
\pgfpathlineto{\pgfqpoint{0.885385in}{3.143602in}}%
\pgfpathlineto{\pgfqpoint{0.927667in}{3.079805in}}%
\pgfpathlineto{\pgfqpoint{0.938907in}{3.067898in}}%
\pgfpathlineto{\pgfqpoint{0.948541in}{3.060813in}}%
\pgfpathlineto{\pgfqpoint{0.956837in}{3.057319in}}%
\pgfpathlineto{\pgfqpoint{0.964597in}{3.056364in}}%
\pgfpathlineto{\pgfqpoint{0.972358in}{3.057677in}}%
\pgfpathlineto{\pgfqpoint{0.980386in}{3.061377in}}%
\pgfpathlineto{\pgfqpoint{0.989217in}{3.068040in}}%
\pgfpathlineto{\pgfqpoint{0.999386in}{3.078709in}}%
\pgfpathlineto{\pgfqpoint{1.011696in}{3.095070in}}%
\pgfpathlineto{\pgfqpoint{1.030429in}{3.124228in}}%
\pgfpathlineto{\pgfqpoint{1.055316in}{3.162293in}}%
\pgfpathlineto{\pgfqpoint{1.067894in}{3.177542in}}%
\pgfpathlineto{\pgfqpoint{1.078063in}{3.186622in}}%
\pgfpathlineto{\pgfqpoint{1.086894in}{3.191707in}}%
\pgfpathlineto{\pgfqpoint{1.094922in}{3.193873in}}%
\pgfpathlineto{\pgfqpoint{1.102683in}{3.193664in}}%
\pgfpathlineto{\pgfqpoint{1.110443in}{3.191200in}}%
\pgfpathlineto{\pgfqpoint{1.118739in}{3.186163in}}%
\pgfpathlineto{\pgfqpoint{1.128106in}{3.177732in}}%
\pgfpathlineto{\pgfqpoint{1.139078in}{3.164690in}}%
\pgfpathlineto{\pgfqpoint{1.153528in}{3.143716in}}%
\pgfpathlineto{\pgfqpoint{1.196078in}{3.079570in}}%
\pgfpathlineto{\pgfqpoint{1.207318in}{3.067725in}}%
\pgfpathlineto{\pgfqpoint{1.216684in}{3.060854in}}%
\pgfpathlineto{\pgfqpoint{1.224980in}{3.057339in}}%
\pgfpathlineto{\pgfqpoint{1.232740in}{3.056362in}}%
\pgfpathlineto{\pgfqpoint{1.240501in}{3.057654in}}%
\pgfpathlineto{\pgfqpoint{1.248529in}{3.061333in}}%
\pgfpathlineto{\pgfqpoint{1.257360in}{3.067975in}}%
\pgfpathlineto{\pgfqpoint{1.267529in}{3.078623in}}%
\pgfpathlineto{\pgfqpoint{1.279839in}{3.094965in}}%
\pgfpathlineto{\pgfqpoint{1.298304in}{3.123679in}}%
\pgfpathlineto{\pgfqpoint{1.323459in}{3.162194in}}%
\pgfpathlineto{\pgfqpoint{1.336037in}{3.177465in}}%
\pgfpathlineto{\pgfqpoint{1.346206in}{3.186569in}}%
\pgfpathlineto{\pgfqpoint{1.355037in}{3.191677in}}%
\pgfpathlineto{\pgfqpoint{1.363065in}{3.193864in}}%
\pgfpathlineto{\pgfqpoint{1.370826in}{3.193677in}}%
\pgfpathlineto{\pgfqpoint{1.378587in}{3.191233in}}%
\pgfpathlineto{\pgfqpoint{1.386882in}{3.186218in}}%
\pgfpathlineto{\pgfqpoint{1.396249in}{3.177809in}}%
\pgfpathlineto{\pgfqpoint{1.407221in}{3.164786in}}%
\pgfpathlineto{\pgfqpoint{1.421671in}{3.143829in}}%
\pgfpathlineto{\pgfqpoint{1.464489in}{3.079335in}}%
\pgfpathlineto{\pgfqpoint{1.475728in}{3.067553in}}%
\pgfpathlineto{\pgfqpoint{1.485094in}{3.060743in}}%
\pgfpathlineto{\pgfqpoint{1.493390in}{3.057287in}}%
\pgfpathlineto{\pgfqpoint{1.501151in}{3.056367in}}%
\pgfpathlineto{\pgfqpoint{1.508912in}{3.057716in}}%
\pgfpathlineto{\pgfqpoint{1.516940in}{3.061451in}}%
\pgfpathlineto{\pgfqpoint{1.525771in}{3.068150in}}%
\pgfpathlineto{\pgfqpoint{1.535940in}{3.078854in}}%
\pgfpathlineto{\pgfqpoint{1.548250in}{3.095247in}}%
\pgfpathlineto{\pgfqpoint{1.566982in}{3.124425in}}%
\pgfpathlineto{\pgfqpoint{1.591870in}{3.162458in}}%
\pgfpathlineto{\pgfqpoint{1.604448in}{3.177669in}}%
\pgfpathlineto{\pgfqpoint{1.614617in}{3.186711in}}%
\pgfpathlineto{\pgfqpoint{1.623448in}{3.191758in}}%
\pgfpathlineto{\pgfqpoint{1.631476in}{3.193887in}}%
\pgfpathlineto{\pgfqpoint{1.639237in}{3.193643in}}%
\pgfpathlineto{\pgfqpoint{1.646997in}{3.191144in}}%
\pgfpathlineto{\pgfqpoint{1.655293in}{3.186072in}}%
\pgfpathlineto{\pgfqpoint{1.664659in}{3.177605in}}%
\pgfpathlineto{\pgfqpoint{1.675631in}{3.164529in}}%
\pgfpathlineto{\pgfqpoint{1.690082in}{3.143527in}}%
\pgfpathlineto{\pgfqpoint{1.732364in}{3.079746in}}%
\pgfpathlineto{\pgfqpoint{1.743604in}{3.067855in}}%
\pgfpathlineto{\pgfqpoint{1.752970in}{3.060938in}}%
\pgfpathlineto{\pgfqpoint{1.761266in}{3.057379in}}%
\pgfpathlineto{\pgfqpoint{1.769026in}{3.056359in}}%
\pgfpathlineto{\pgfqpoint{1.776787in}{3.057609in}}%
\pgfpathlineto{\pgfqpoint{1.784815in}{3.061246in}}%
\pgfpathlineto{\pgfqpoint{1.793646in}{3.067844in}}%
\pgfpathlineto{\pgfqpoint{1.803815in}{3.078449in}}%
\pgfpathlineto{\pgfqpoint{1.816125in}{3.094753in}}%
\pgfpathlineto{\pgfqpoint{1.834590in}{3.123444in}}%
\pgfpathlineto{\pgfqpoint{1.860013in}{3.162359in}}%
\pgfpathlineto{\pgfqpoint{1.872591in}{3.177593in}}%
\pgfpathlineto{\pgfqpoint{1.882760in}{3.186658in}}%
\pgfpathlineto{\pgfqpoint{1.891591in}{3.191728in}}%
\pgfpathlineto{\pgfqpoint{1.899619in}{3.193879in}}%
\pgfpathlineto{\pgfqpoint{1.907380in}{3.193656in}}%
\pgfpathlineto{\pgfqpoint{1.915140in}{3.191177in}}%
\pgfpathlineto{\pgfqpoint{1.923436in}{3.186127in}}%
\pgfpathlineto{\pgfqpoint{1.932802in}{3.177682in}}%
\pgfpathlineto{\pgfqpoint{1.943774in}{3.164625in}}%
\pgfpathlineto{\pgfqpoint{1.958225in}{3.143640in}}%
\pgfpathlineto{\pgfqpoint{2.000775in}{3.079511in}}%
\pgfpathlineto{\pgfqpoint{2.012014in}{3.067682in}}%
\pgfpathlineto{\pgfqpoint{2.021380in}{3.060827in}}%
\pgfpathlineto{\pgfqpoint{2.029676in}{3.057326in}}%
\pgfpathlineto{\pgfqpoint{2.037437in}{3.056363in}}%
\pgfpathlineto{\pgfqpoint{2.045198in}{3.057670in}}%
\pgfpathlineto{\pgfqpoint{2.053226in}{3.061363in}}%
\pgfpathlineto{\pgfqpoint{2.062057in}{3.068018in}}%
\pgfpathlineto{\pgfqpoint{2.072226in}{3.078680in}}%
\pgfpathlineto{\pgfqpoint{2.084536in}{3.095035in}}%
\pgfpathlineto{\pgfqpoint{2.103268in}{3.124189in}}%
\pgfpathlineto{\pgfqpoint{2.128156in}{3.162260in}}%
\pgfpathlineto{\pgfqpoint{2.140734in}{3.177516in}}%
\pgfpathlineto{\pgfqpoint{2.150903in}{3.186605in}}%
\pgfpathlineto{\pgfqpoint{2.159734in}{3.191697in}}%
\pgfpathlineto{\pgfqpoint{2.167762in}{3.193870in}}%
\pgfpathlineto{\pgfqpoint{2.175523in}{3.193669in}}%
\pgfpathlineto{\pgfqpoint{2.183283in}{3.191211in}}%
\pgfpathlineto{\pgfqpoint{2.191579in}{3.186182in}}%
\pgfpathlineto{\pgfqpoint{2.200945in}{3.177758in}}%
\pgfpathlineto{\pgfqpoint{2.211917in}{3.164722in}}%
\pgfpathlineto{\pgfqpoint{2.226368in}{3.143754in}}%
\pgfpathlineto{\pgfqpoint{2.268918in}{3.079599in}}%
\pgfpathlineto{\pgfqpoint{2.280157in}{3.067747in}}%
\pgfpathlineto{\pgfqpoint{2.289524in}{3.060868in}}%
\pgfpathlineto{\pgfqpoint{2.297819in}{3.057345in}}%
\pgfpathlineto{\pgfqpoint{2.305580in}{3.056362in}}%
\pgfpathlineto{\pgfqpoint{2.313341in}{3.057647in}}%
\pgfpathlineto{\pgfqpoint{2.321369in}{3.061319in}}%
\pgfpathlineto{\pgfqpoint{2.330200in}{3.067953in}}%
\pgfpathlineto{\pgfqpoint{2.340369in}{3.078594in}}%
\pgfpathlineto{\pgfqpoint{2.352679in}{3.094929in}}%
\pgfpathlineto{\pgfqpoint{2.371144in}{3.123640in}}%
\pgfpathlineto{\pgfqpoint{2.396299in}{3.162161in}}%
\pgfpathlineto{\pgfqpoint{2.408877in}{3.177440in}}%
\pgfpathlineto{\pgfqpoint{2.419046in}{3.186551in}}%
\pgfpathlineto{\pgfqpoint{2.427877in}{3.191666in}}%
\pgfpathlineto{\pgfqpoint{2.435905in}{3.193861in}}%
\pgfpathlineto{\pgfqpoint{2.443666in}{3.193681in}}%
\pgfpathlineto{\pgfqpoint{2.451426in}{3.191244in}}%
\pgfpathlineto{\pgfqpoint{2.459722in}{3.186236in}}%
\pgfpathlineto{\pgfqpoint{2.469088in}{3.177834in}}%
\pgfpathlineto{\pgfqpoint{2.480060in}{3.164818in}}%
\pgfpathlineto{\pgfqpoint{2.494244in}{3.144282in}}%
\pgfpathlineto{\pgfqpoint{2.537596in}{3.079043in}}%
\pgfpathlineto{\pgfqpoint{2.548836in}{3.067339in}}%
\pgfpathlineto{\pgfqpoint{2.558202in}{3.060607in}}%
\pgfpathlineto{\pgfqpoint{2.566498in}{3.057223in}}%
\pgfpathlineto{\pgfqpoint{2.574258in}{3.056375in}}%
\pgfpathlineto{\pgfqpoint{2.582019in}{3.057795in}}%
\pgfpathlineto{\pgfqpoint{2.590047in}{3.061600in}}%
\pgfpathlineto{\pgfqpoint{2.598878in}{3.068370in}}%
\pgfpathlineto{\pgfqpoint{2.609047in}{3.079145in}}%
\pgfpathlineto{\pgfqpoint{2.621625in}{3.095991in}}%
\pgfpathlineto{\pgfqpoint{2.640893in}{3.126112in}}%
\pgfpathlineto{\pgfqpoint{2.664710in}{3.162425in}}%
\pgfpathlineto{\pgfqpoint{2.677287in}{3.177644in}}%
\pgfpathlineto{\pgfqpoint{2.687456in}{3.186693in}}%
\pgfpathlineto{\pgfqpoint{2.696287in}{3.191748in}}%
\pgfpathlineto{\pgfqpoint{2.704316in}{3.193884in}}%
\pgfpathlineto{\pgfqpoint{2.712076in}{3.193647in}}%
\pgfpathlineto{\pgfqpoint{2.719837in}{3.191155in}}%
\pgfpathlineto{\pgfqpoint{2.728133in}{3.186090in}}%
\pgfpathlineto{\pgfqpoint{2.737499in}{3.177631in}}%
\pgfpathlineto{\pgfqpoint{2.748471in}{3.164561in}}%
\pgfpathlineto{\pgfqpoint{2.762922in}{3.143565in}}%
\pgfpathlineto{\pgfqpoint{2.805204in}{3.079776in}}%
\pgfpathlineto{\pgfqpoint{2.816443in}{3.067877in}}%
\pgfpathlineto{\pgfqpoint{2.826077in}{3.060799in}}%
\pgfpathlineto{\pgfqpoint{2.834373in}{3.057313in}}%
\pgfpathlineto{\pgfqpoint{2.842134in}{3.056364in}}%
\pgfpathlineto{\pgfqpoint{2.849894in}{3.057685in}}%
\pgfpathlineto{\pgfqpoint{2.857923in}{3.061392in}}%
\pgfpathlineto{\pgfqpoint{2.866754in}{3.068062in}}%
\pgfpathlineto{\pgfqpoint{2.876923in}{3.078738in}}%
\pgfpathlineto{\pgfqpoint{2.889233in}{3.095106in}}%
\pgfpathlineto{\pgfqpoint{2.907965in}{3.124268in}}%
\pgfpathlineto{\pgfqpoint{2.932853in}{3.162326in}}%
\pgfpathlineto{\pgfqpoint{2.945430in}{3.177567in}}%
\pgfpathlineto{\pgfqpoint{2.955599in}{3.186640in}}%
\pgfpathlineto{\pgfqpoint{2.964430in}{3.191718in}}%
\pgfpathlineto{\pgfqpoint{2.972459in}{3.193876in}}%
\pgfpathlineto{\pgfqpoint{2.980219in}{3.193660in}}%
\pgfpathlineto{\pgfqpoint{2.987980in}{3.191189in}}%
\pgfpathlineto{\pgfqpoint{2.996276in}{3.186145in}}%
\pgfpathlineto{\pgfqpoint{3.005642in}{3.177707in}}%
\pgfpathlineto{\pgfqpoint{3.016614in}{3.164658in}}%
\pgfpathlineto{\pgfqpoint{3.031065in}{3.143678in}}%
\pgfpathlineto{\pgfqpoint{3.073614in}{3.079540in}}%
\pgfpathlineto{\pgfqpoint{3.084854in}{3.067703in}}%
\pgfpathlineto{\pgfqpoint{3.094220in}{3.060840in}}%
\pgfpathlineto{\pgfqpoint{3.102516in}{3.057332in}}%
\pgfpathlineto{\pgfqpoint{3.110277in}{3.056363in}}%
\pgfpathlineto{\pgfqpoint{3.118037in}{3.057662in}}%
\pgfpathlineto{\pgfqpoint{3.126066in}{3.061348in}}%
\pgfpathlineto{\pgfqpoint{3.134897in}{3.067996in}}%
\pgfpathlineto{\pgfqpoint{3.145066in}{3.078651in}}%
\pgfpathlineto{\pgfqpoint{3.157376in}{3.095000in}}%
\pgfpathlineto{\pgfqpoint{3.176108in}{3.124150in}}%
\pgfpathlineto{\pgfqpoint{3.200996in}{3.162227in}}%
\pgfpathlineto{\pgfqpoint{3.213573in}{3.177491in}}%
\pgfpathlineto{\pgfqpoint{3.223742in}{3.186587in}}%
\pgfpathlineto{\pgfqpoint{3.232573in}{3.191687in}}%
\pgfpathlineto{\pgfqpoint{3.240602in}{3.193867in}}%
\pgfpathlineto{\pgfqpoint{3.248362in}{3.193673in}}%
\pgfpathlineto{\pgfqpoint{3.256123in}{3.191222in}}%
\pgfpathlineto{\pgfqpoint{3.264419in}{3.186200in}}%
\pgfpathlineto{\pgfqpoint{3.273785in}{3.177783in}}%
\pgfpathlineto{\pgfqpoint{3.284757in}{3.164754in}}%
\pgfpathlineto{\pgfqpoint{3.299208in}{3.143791in}}%
\pgfpathlineto{\pgfqpoint{3.341757in}{3.079628in}}%
\pgfpathlineto{\pgfqpoint{3.352997in}{3.067768in}}%
\pgfpathlineto{\pgfqpoint{3.362363in}{3.060882in}}%
\pgfpathlineto{\pgfqpoint{3.370659in}{3.057352in}}%
\pgfpathlineto{\pgfqpoint{3.378420in}{3.056361in}}%
\pgfpathlineto{\pgfqpoint{3.386180in}{3.057639in}}%
\pgfpathlineto{\pgfqpoint{3.394209in}{3.061304in}}%
\pgfpathlineto{\pgfqpoint{3.403040in}{3.067931in}}%
\pgfpathlineto{\pgfqpoint{3.413209in}{3.078565in}}%
\pgfpathlineto{\pgfqpoint{3.425519in}{3.094894in}}%
\pgfpathlineto{\pgfqpoint{3.443984in}{3.123601in}}%
\pgfpathlineto{\pgfqpoint{3.469406in}{3.162491in}}%
\pgfpathlineto{\pgfqpoint{3.481984in}{3.177694in}}%
\pgfpathlineto{\pgfqpoint{3.492153in}{3.186728in}}%
\pgfpathlineto{\pgfqpoint{3.500984in}{3.191768in}}%
\pgfpathlineto{\pgfqpoint{3.509012in}{3.193890in}}%
\pgfpathlineto{\pgfqpoint{3.516773in}{3.193639in}}%
\pgfpathlineto{\pgfqpoint{3.524534in}{3.191132in}}%
\pgfpathlineto{\pgfqpoint{3.532829in}{3.186054in}}%
\pgfpathlineto{\pgfqpoint{3.542196in}{3.177580in}}%
\pgfpathlineto{\pgfqpoint{3.553168in}{3.164497in}}%
\pgfpathlineto{\pgfqpoint{3.567618in}{3.143489in}}%
\pgfpathlineto{\pgfqpoint{3.609901in}{3.079717in}}%
\pgfpathlineto{\pgfqpoint{3.621140in}{3.067833in}}%
\pgfpathlineto{\pgfqpoint{3.630506in}{3.060924in}}%
\pgfpathlineto{\pgfqpoint{3.638802in}{3.057372in}}%
\pgfpathlineto{\pgfqpoint{3.646563in}{3.056360in}}%
\pgfpathlineto{\pgfqpoint{3.654323in}{3.057617in}}%
\pgfpathlineto{\pgfqpoint{3.662352in}{3.061260in}}%
\pgfpathlineto{\pgfqpoint{3.671183in}{3.067866in}}%
\pgfpathlineto{\pgfqpoint{3.681352in}{3.078478in}}%
\pgfpathlineto{\pgfqpoint{3.693662in}{3.094788in}}%
\pgfpathlineto{\pgfqpoint{3.712127in}{3.123483in}}%
\pgfpathlineto{\pgfqpoint{3.737549in}{3.162392in}}%
\pgfpathlineto{\pgfqpoint{3.750127in}{3.177618in}}%
\pgfpathlineto{\pgfqpoint{3.760296in}{3.186675in}}%
\pgfpathlineto{\pgfqpoint{3.769127in}{3.191738in}}%
\pgfpathlineto{\pgfqpoint{3.777155in}{3.193881in}}%
\pgfpathlineto{\pgfqpoint{3.784916in}{3.193652in}}%
\pgfpathlineto{\pgfqpoint{3.792677in}{3.191166in}}%
\pgfpathlineto{\pgfqpoint{3.800973in}{3.186109in}}%
\pgfpathlineto{\pgfqpoint{3.810339in}{3.177656in}}%
\pgfpathlineto{\pgfqpoint{3.821311in}{3.164593in}}%
\pgfpathlineto{\pgfqpoint{3.835762in}{3.143602in}}%
\pgfpathlineto{\pgfqpoint{3.878044in}{3.079805in}}%
\pgfpathlineto{\pgfqpoint{3.889283in}{3.067898in}}%
\pgfpathlineto{\pgfqpoint{3.898917in}{3.060813in}}%
\pgfpathlineto{\pgfqpoint{3.907213in}{3.057319in}}%
\pgfpathlineto{\pgfqpoint{3.914973in}{3.056364in}}%
\pgfpathlineto{\pgfqpoint{3.922734in}{3.057677in}}%
\pgfpathlineto{\pgfqpoint{3.930762in}{3.061377in}}%
\pgfpathlineto{\pgfqpoint{3.939593in}{3.068040in}}%
\pgfpathlineto{\pgfqpoint{3.949762in}{3.078709in}}%
\pgfpathlineto{\pgfqpoint{3.962072in}{3.095070in}}%
\pgfpathlineto{\pgfqpoint{3.980805in}{3.124228in}}%
\pgfpathlineto{\pgfqpoint{4.005692in}{3.162293in}}%
\pgfpathlineto{\pgfqpoint{4.018270in}{3.177542in}}%
\pgfpathlineto{\pgfqpoint{4.028439in}{3.186622in}}%
\pgfpathlineto{\pgfqpoint{4.037270in}{3.191707in}}%
\pgfpathlineto{\pgfqpoint{4.045298in}{3.193873in}}%
\pgfpathlineto{\pgfqpoint{4.053059in}{3.193664in}}%
\pgfpathlineto{\pgfqpoint{4.060820in}{3.191200in}}%
\pgfpathlineto{\pgfqpoint{4.069116in}{3.186163in}}%
\pgfpathlineto{\pgfqpoint{4.078482in}{3.177732in}}%
\pgfpathlineto{\pgfqpoint{4.089454in}{3.164690in}}%
\pgfpathlineto{\pgfqpoint{4.103905in}{3.143716in}}%
\pgfpathlineto{\pgfqpoint{4.146454in}{3.079570in}}%
\pgfpathlineto{\pgfqpoint{4.157694in}{3.067725in}}%
\pgfpathlineto{\pgfqpoint{4.167060in}{3.060854in}}%
\pgfpathlineto{\pgfqpoint{4.175356in}{3.057339in}}%
\pgfpathlineto{\pgfqpoint{4.183116in}{3.056362in}}%
\pgfpathlineto{\pgfqpoint{4.190877in}{3.057654in}}%
\pgfpathlineto{\pgfqpoint{4.198905in}{3.061333in}}%
\pgfpathlineto{\pgfqpoint{4.207736in}{3.067975in}}%
\pgfpathlineto{\pgfqpoint{4.217905in}{3.078623in}}%
\pgfpathlineto{\pgfqpoint{4.230215in}{3.094965in}}%
\pgfpathlineto{\pgfqpoint{4.248680in}{3.123679in}}%
\pgfpathlineto{\pgfqpoint{4.273836in}{3.162194in}}%
\pgfpathlineto{\pgfqpoint{4.286413in}{3.177465in}}%
\pgfpathlineto{\pgfqpoint{4.296582in}{3.186569in}}%
\pgfpathlineto{\pgfqpoint{4.305413in}{3.191677in}}%
\pgfpathlineto{\pgfqpoint{4.313441in}{3.193864in}}%
\pgfpathlineto{\pgfqpoint{4.321202in}{3.193677in}}%
\pgfpathlineto{\pgfqpoint{4.328963in}{3.191233in}}%
\pgfpathlineto{\pgfqpoint{4.337259in}{3.186218in}}%
\pgfpathlineto{\pgfqpoint{4.346625in}{3.177809in}}%
\pgfpathlineto{\pgfqpoint{4.357597in}{3.164786in}}%
\pgfpathlineto{\pgfqpoint{4.372048in}{3.143829in}}%
\pgfpathlineto{\pgfqpoint{4.414865in}{3.079335in}}%
\pgfpathlineto{\pgfqpoint{4.426104in}{3.067553in}}%
\pgfpathlineto{\pgfqpoint{4.435471in}{3.060743in}}%
\pgfpathlineto{\pgfqpoint{4.443766in}{3.057287in}}%
\pgfpathlineto{\pgfqpoint{4.451527in}{3.056367in}}%
\pgfpathlineto{\pgfqpoint{4.459288in}{3.057716in}}%
\pgfpathlineto{\pgfqpoint{4.467316in}{3.061451in}}%
\pgfpathlineto{\pgfqpoint{4.476147in}{3.068150in}}%
\pgfpathlineto{\pgfqpoint{4.486316in}{3.078854in}}%
\pgfpathlineto{\pgfqpoint{4.498626in}{3.095247in}}%
\pgfpathlineto{\pgfqpoint{4.517359in}{3.124425in}}%
\pgfpathlineto{\pgfqpoint{4.542246in}{3.162458in}}%
\pgfpathlineto{\pgfqpoint{4.554824in}{3.177669in}}%
\pgfpathlineto{\pgfqpoint{4.564993in}{3.186711in}}%
\pgfpathlineto{\pgfqpoint{4.573824in}{3.191758in}}%
\pgfpathlineto{\pgfqpoint{4.581852in}{3.193887in}}%
\pgfpathlineto{\pgfqpoint{4.589613in}{3.193643in}}%
\pgfpathlineto{\pgfqpoint{4.597373in}{3.191144in}}%
\pgfpathlineto{\pgfqpoint{4.605669in}{3.186072in}}%
\pgfpathlineto{\pgfqpoint{4.615035in}{3.177605in}}%
\pgfpathlineto{\pgfqpoint{4.626007in}{3.164529in}}%
\pgfpathlineto{\pgfqpoint{4.640458in}{3.143527in}}%
\pgfpathlineto{\pgfqpoint{4.682740in}{3.079746in}}%
\pgfpathlineto{\pgfqpoint{4.693980in}{3.067855in}}%
\pgfpathlineto{\pgfqpoint{4.703346in}{3.060938in}}%
\pgfpathlineto{\pgfqpoint{4.711642in}{3.057379in}}%
\pgfpathlineto{\pgfqpoint{4.719403in}{3.056359in}}%
\pgfpathlineto{\pgfqpoint{4.727163in}{3.057609in}}%
\pgfpathlineto{\pgfqpoint{4.735191in}{3.061246in}}%
\pgfpathlineto{\pgfqpoint{4.744022in}{3.067844in}}%
\pgfpathlineto{\pgfqpoint{4.754192in}{3.078449in}}%
\pgfpathlineto{\pgfqpoint{4.766502in}{3.094753in}}%
\pgfpathlineto{\pgfqpoint{4.784966in}{3.123444in}}%
\pgfpathlineto{\pgfqpoint{4.810389in}{3.162359in}}%
\pgfpathlineto{\pgfqpoint{4.822967in}{3.177593in}}%
\pgfpathlineto{\pgfqpoint{4.833136in}{3.186658in}}%
\pgfpathlineto{\pgfqpoint{4.841967in}{3.191728in}}%
\pgfpathlineto{\pgfqpoint{4.849995in}{3.193879in}}%
\pgfpathlineto{\pgfqpoint{4.857756in}{3.193656in}}%
\pgfpathlineto{\pgfqpoint{4.865516in}{3.191177in}}%
\pgfpathlineto{\pgfqpoint{4.873812in}{3.186127in}}%
\pgfpathlineto{\pgfqpoint{4.883179in}{3.177682in}}%
\pgfpathlineto{\pgfqpoint{4.894150in}{3.164625in}}%
\pgfpathlineto{\pgfqpoint{4.908601in}{3.143640in}}%
\pgfpathlineto{\pgfqpoint{4.951151in}{3.079511in}}%
\pgfpathlineto{\pgfqpoint{4.962390in}{3.067682in}}%
\pgfpathlineto{\pgfqpoint{4.971757in}{3.060827in}}%
\pgfpathlineto{\pgfqpoint{4.980053in}{3.057326in}}%
\pgfpathlineto{\pgfqpoint{4.987813in}{3.056363in}}%
\pgfpathlineto{\pgfqpoint{4.995574in}{3.057670in}}%
\pgfpathlineto{\pgfqpoint{5.003602in}{3.061363in}}%
\pgfpathlineto{\pgfqpoint{5.012433in}{3.068018in}}%
\pgfpathlineto{\pgfqpoint{5.022602in}{3.078680in}}%
\pgfpathlineto{\pgfqpoint{5.034912in}{3.095035in}}%
\pgfpathlineto{\pgfqpoint{5.053645in}{3.124189in}}%
\pgfpathlineto{\pgfqpoint{5.078532in}{3.162260in}}%
\pgfpathlineto{\pgfqpoint{5.091110in}{3.177516in}}%
\pgfpathlineto{\pgfqpoint{5.101279in}{3.186605in}}%
\pgfpathlineto{\pgfqpoint{5.110110in}{3.191697in}}%
\pgfpathlineto{\pgfqpoint{5.118138in}{3.193870in}}%
\pgfpathlineto{\pgfqpoint{5.125899in}{3.193669in}}%
\pgfpathlineto{\pgfqpoint{5.133659in}{3.191211in}}%
\pgfpathlineto{\pgfqpoint{5.141955in}{3.186182in}}%
\pgfpathlineto{\pgfqpoint{5.151322in}{3.177758in}}%
\pgfpathlineto{\pgfqpoint{5.162293in}{3.164722in}}%
\pgfpathlineto{\pgfqpoint{5.176744in}{3.143754in}}%
\pgfpathlineto{\pgfqpoint{5.219294in}{3.079599in}}%
\pgfpathlineto{\pgfqpoint{5.230533in}{3.067747in}}%
\pgfpathlineto{\pgfqpoint{5.239900in}{3.060868in}}%
\pgfpathlineto{\pgfqpoint{5.248196in}{3.057345in}}%
\pgfpathlineto{\pgfqpoint{5.255956in}{3.056362in}}%
\pgfpathlineto{\pgfqpoint{5.263717in}{3.057647in}}%
\pgfpathlineto{\pgfqpoint{5.271745in}{3.061319in}}%
\pgfpathlineto{\pgfqpoint{5.280576in}{3.067953in}}%
\pgfpathlineto{\pgfqpoint{5.290745in}{3.078594in}}%
\pgfpathlineto{\pgfqpoint{5.303055in}{3.094929in}}%
\pgfpathlineto{\pgfqpoint{5.321520in}{3.123640in}}%
\pgfpathlineto{\pgfqpoint{5.346675in}{3.162161in}}%
\pgfpathlineto{\pgfqpoint{5.359253in}{3.177440in}}%
\pgfpathlineto{\pgfqpoint{5.369422in}{3.186551in}}%
\pgfpathlineto{\pgfqpoint{5.378253in}{3.191666in}}%
\pgfpathlineto{\pgfqpoint{5.386281in}{3.193861in}}%
\pgfpathlineto{\pgfqpoint{5.394042in}{3.193681in}}%
\pgfpathlineto{\pgfqpoint{5.401802in}{3.191244in}}%
\pgfpathlineto{\pgfqpoint{5.410098in}{3.186236in}}%
\pgfpathlineto{\pgfqpoint{5.419465in}{3.177834in}}%
\pgfpathlineto{\pgfqpoint{5.430437in}{3.164818in}}%
\pgfpathlineto{\pgfqpoint{5.444620in}{3.144282in}}%
\pgfpathlineto{\pgfqpoint{5.487972in}{3.079043in}}%
\pgfpathlineto{\pgfqpoint{5.499212in}{3.067339in}}%
\pgfpathlineto{\pgfqpoint{5.508578in}{3.060607in}}%
\pgfpathlineto{\pgfqpoint{5.516874in}{3.057223in}}%
\pgfpathlineto{\pgfqpoint{5.524634in}{3.056375in}}%
\pgfpathlineto{\pgfqpoint{5.532395in}{3.057795in}}%
\pgfpathlineto{\pgfqpoint{5.540423in}{3.061600in}}%
\pgfpathlineto{\pgfqpoint{5.549254in}{3.068370in}}%
\pgfpathlineto{\pgfqpoint{5.559423in}{3.079145in}}%
\pgfpathlineto{\pgfqpoint{5.572001in}{3.095991in}}%
\pgfpathlineto{\pgfqpoint{5.591269in}{3.126112in}}%
\pgfpathlineto{\pgfqpoint{5.615086in}{3.162425in}}%
\pgfpathlineto{\pgfqpoint{5.627663in}{3.177644in}}%
\pgfpathlineto{\pgfqpoint{5.637833in}{3.186693in}}%
\pgfpathlineto{\pgfqpoint{5.646664in}{3.191748in}}%
\pgfpathlineto{\pgfqpoint{5.654692in}{3.193884in}}%
\pgfpathlineto{\pgfqpoint{5.662453in}{3.193647in}}%
\pgfpathlineto{\pgfqpoint{5.670213in}{3.191155in}}%
\pgfpathlineto{\pgfqpoint{5.678509in}{3.186090in}}%
\pgfpathlineto{\pgfqpoint{5.687875in}{3.177631in}}%
\pgfpathlineto{\pgfqpoint{5.698847in}{3.164561in}}%
\pgfpathlineto{\pgfqpoint{5.713298in}{3.143565in}}%
\pgfpathlineto{\pgfqpoint{5.755580in}{3.079776in}}%
\pgfpathlineto{\pgfqpoint{5.766820in}{3.067877in}}%
\pgfpathlineto{\pgfqpoint{5.776453in}{3.060799in}}%
\pgfpathlineto{\pgfqpoint{5.784749in}{3.057313in}}%
\pgfpathlineto{\pgfqpoint{5.792510in}{3.056364in}}%
\pgfpathlineto{\pgfqpoint{5.800271in}{3.057685in}}%
\pgfpathlineto{\pgfqpoint{5.808299in}{3.061392in}}%
\pgfpathlineto{\pgfqpoint{5.817130in}{3.068062in}}%
\pgfpathlineto{\pgfqpoint{5.827299in}{3.078738in}}%
\pgfpathlineto{\pgfqpoint{5.839609in}{3.095106in}}%
\pgfpathlineto{\pgfqpoint{5.858341in}{3.124268in}}%
\pgfpathlineto{\pgfqpoint{5.883229in}{3.162326in}}%
\pgfpathlineto{\pgfqpoint{5.895807in}{3.177567in}}%
\pgfpathlineto{\pgfqpoint{5.905976in}{3.186640in}}%
\pgfpathlineto{\pgfqpoint{5.914807in}{3.191718in}}%
\pgfpathlineto{\pgfqpoint{5.922835in}{3.193876in}}%
\pgfpathlineto{\pgfqpoint{5.930596in}{3.193660in}}%
\pgfpathlineto{\pgfqpoint{5.938356in}{3.191189in}}%
\pgfpathlineto{\pgfqpoint{5.946652in}{3.186145in}}%
\pgfpathlineto{\pgfqpoint{5.956018in}{3.177707in}}%
\pgfpathlineto{\pgfqpoint{5.966990in}{3.164658in}}%
\pgfpathlineto{\pgfqpoint{5.981441in}{3.143678in}}%
\pgfpathlineto{\pgfqpoint{6.023991in}{3.079540in}}%
\pgfpathlineto{\pgfqpoint{6.035230in}{3.067703in}}%
\pgfpathlineto{\pgfqpoint{6.044596in}{3.060840in}}%
\pgfpathlineto{\pgfqpoint{6.052892in}{3.057332in}}%
\pgfpathlineto{\pgfqpoint{6.060653in}{3.056363in}}%
\pgfpathlineto{\pgfqpoint{6.068414in}{3.057662in}}%
\pgfpathlineto{\pgfqpoint{6.076442in}{3.061348in}}%
\pgfpathlineto{\pgfqpoint{6.085273in}{3.067996in}}%
\pgfpathlineto{\pgfqpoint{6.095442in}{3.078651in}}%
\pgfpathlineto{\pgfqpoint{6.107752in}{3.095000in}}%
\pgfpathlineto{\pgfqpoint{6.126484in}{3.124150in}}%
\pgfpathlineto{\pgfqpoint{6.151372in}{3.162227in}}%
\pgfpathlineto{\pgfqpoint{6.163950in}{3.177491in}}%
\pgfpathlineto{\pgfqpoint{6.174119in}{3.186587in}}%
\pgfpathlineto{\pgfqpoint{6.182950in}{3.191687in}}%
\pgfpathlineto{\pgfqpoint{6.190978in}{3.193867in}}%
\pgfpathlineto{\pgfqpoint{6.198739in}{3.193673in}}%
\pgfpathlineto{\pgfqpoint{6.206499in}{3.191222in}}%
\pgfpathlineto{\pgfqpoint{6.214795in}{3.186200in}}%
\pgfpathlineto{\pgfqpoint{6.224161in}{3.177783in}}%
\pgfpathlineto{\pgfqpoint{6.235133in}{3.164754in}}%
\pgfpathlineto{\pgfqpoint{6.249584in}{3.143791in}}%
\pgfpathlineto{\pgfqpoint{6.292134in}{3.079628in}}%
\pgfpathlineto{\pgfqpoint{6.303373in}{3.067768in}}%
\pgfpathlineto{\pgfqpoint{6.312740in}{3.060882in}}%
\pgfpathlineto{\pgfqpoint{6.321035in}{3.057352in}}%
\pgfpathlineto{\pgfqpoint{6.328796in}{3.056361in}}%
\pgfpathlineto{\pgfqpoint{6.336557in}{3.057639in}}%
\pgfpathlineto{\pgfqpoint{6.344585in}{3.061304in}}%
\pgfpathlineto{\pgfqpoint{6.353416in}{3.067931in}}%
\pgfpathlineto{\pgfqpoint{6.363585in}{3.078565in}}%
\pgfpathlineto{\pgfqpoint{6.375895in}{3.094894in}}%
\pgfpathlineto{\pgfqpoint{6.394360in}{3.123601in}}%
\pgfpathlineto{\pgfqpoint{6.419783in}{3.162491in}}%
\pgfpathlineto{\pgfqpoint{6.432360in}{3.177694in}}%
\pgfpathlineto{\pgfqpoint{6.442529in}{3.186728in}}%
\pgfpathlineto{\pgfqpoint{6.451360in}{3.191768in}}%
\pgfpathlineto{\pgfqpoint{6.459389in}{3.193890in}}%
\pgfpathlineto{\pgfqpoint{6.467149in}{3.193639in}}%
\pgfpathlineto{\pgfqpoint{6.474910in}{3.191132in}}%
\pgfpathlineto{\pgfqpoint{6.483206in}{3.186054in}}%
\pgfpathlineto{\pgfqpoint{6.492572in}{3.177580in}}%
\pgfpathlineto{\pgfqpoint{6.503544in}{3.164497in}}%
\pgfpathlineto{\pgfqpoint{6.517995in}{3.143489in}}%
\pgfpathlineto{\pgfqpoint{6.560277in}{3.079717in}}%
\pgfpathlineto{\pgfqpoint{6.571516in}{3.067833in}}%
\pgfpathlineto{\pgfqpoint{6.580883in}{3.060924in}}%
\pgfpathlineto{\pgfqpoint{6.589178in}{3.057372in}}%
\pgfpathlineto{\pgfqpoint{6.596939in}{3.056360in}}%
\pgfpathlineto{\pgfqpoint{6.604700in}{3.057617in}}%
\pgfpathlineto{\pgfqpoint{6.612728in}{3.061260in}}%
\pgfpathlineto{\pgfqpoint{6.621559in}{3.067866in}}%
\pgfpathlineto{\pgfqpoint{6.631728in}{3.078478in}}%
\pgfpathlineto{\pgfqpoint{6.644038in}{3.094788in}}%
\pgfpathlineto{\pgfqpoint{6.662503in}{3.123483in}}%
\pgfpathlineto{\pgfqpoint{6.663306in}{3.124778in}}%
\pgfpathlineto{\pgfqpoint{6.663306in}{3.124778in}}%
\pgfusepath{stroke}%
\end{pgfscope}%
\begin{pgfscope}%
\pgfpathrectangle{\pgfqpoint{0.467797in}{2.292089in}}{\pgfqpoint{6.490533in}{1.666241in}}%
\pgfusepath{clip}%
\pgfsetrectcap%
\pgfsetroundjoin%
\pgfsetlinewidth{1.505625pt}%
\definecolor{currentstroke}{rgb}{0.890196,0.466667,0.760784}%
\pgfsetstrokecolor{currentstroke}%
\pgfsetdash{}{0pt}%
\pgfpathmoveto{\pgfqpoint{0.762821in}{3.125209in}}%
\pgfpathlineto{\pgfqpoint{0.784497in}{3.158111in}}%
\pgfpathlineto{\pgfqpoint{0.795737in}{3.171252in}}%
\pgfpathlineto{\pgfqpoint{0.804835in}{3.178696in}}%
\pgfpathlineto{\pgfqpoint{0.812864in}{3.182472in}}%
\pgfpathlineto{\pgfqpoint{0.820089in}{3.183463in}}%
\pgfpathlineto{\pgfqpoint{0.827047in}{3.182222in}}%
\pgfpathlineto{\pgfqpoint{0.834540in}{3.178530in}}%
\pgfpathlineto{\pgfqpoint{0.842836in}{3.171785in}}%
\pgfpathlineto{\pgfqpoint{0.852470in}{3.160914in}}%
\pgfpathlineto{\pgfqpoint{0.865047in}{3.143063in}}%
\pgfpathlineto{\pgfqpoint{0.904921in}{3.083724in}}%
\pgfpathlineto{\pgfqpoint{0.914822in}{3.074200in}}%
\pgfpathlineto{\pgfqpoint{0.923118in}{3.069131in}}%
\pgfpathlineto{\pgfqpoint{0.930611in}{3.067080in}}%
\pgfpathlineto{\pgfqpoint{0.937569in}{3.067408in}}%
\pgfpathlineto{\pgfqpoint{0.944794in}{3.070010in}}%
\pgfpathlineto{\pgfqpoint{0.952555in}{3.075251in}}%
\pgfpathlineto{\pgfqpoint{0.961386in}{3.083993in}}%
\pgfpathlineto{\pgfqpoint{0.972358in}{3.098196in}}%
\pgfpathlineto{\pgfqpoint{0.988950in}{3.123881in}}%
\pgfpathlineto{\pgfqpoint{1.011429in}{3.158083in}}%
\pgfpathlineto{\pgfqpoint{1.022668in}{3.171231in}}%
\pgfpathlineto{\pgfqpoint{1.031767in}{3.178683in}}%
\pgfpathlineto{\pgfqpoint{1.039795in}{3.182465in}}%
\pgfpathlineto{\pgfqpoint{1.047020in}{3.183464in}}%
\pgfpathlineto{\pgfqpoint{1.053978in}{3.182229in}}%
\pgfpathlineto{\pgfqpoint{1.061471in}{3.178544in}}%
\pgfpathlineto{\pgfqpoint{1.069767in}{3.171805in}}%
\pgfpathlineto{\pgfqpoint{1.079401in}{3.160940in}}%
\pgfpathlineto{\pgfqpoint{1.091979in}{3.143094in}}%
\pgfpathlineto{\pgfqpoint{1.131852in}{3.083747in}}%
\pgfpathlineto{\pgfqpoint{1.141754in}{3.074216in}}%
\pgfpathlineto{\pgfqpoint{1.150049in}{3.069140in}}%
\pgfpathlineto{\pgfqpoint{1.157542in}{3.067083in}}%
\pgfpathlineto{\pgfqpoint{1.164500in}{3.067403in}}%
\pgfpathlineto{\pgfqpoint{1.171726in}{3.069999in}}%
\pgfpathlineto{\pgfqpoint{1.179486in}{3.075234in}}%
\pgfpathlineto{\pgfqpoint{1.188317in}{3.083969in}}%
\pgfpathlineto{\pgfqpoint{1.199289in}{3.098166in}}%
\pgfpathlineto{\pgfqpoint{1.215881in}{3.123848in}}%
\pgfpathlineto{\pgfqpoint{1.238360in}{3.158056in}}%
\pgfpathlineto{\pgfqpoint{1.249600in}{3.171211in}}%
\pgfpathlineto{\pgfqpoint{1.258698in}{3.178670in}}%
\pgfpathlineto{\pgfqpoint{1.266726in}{3.182459in}}%
\pgfpathlineto{\pgfqpoint{1.273952in}{3.183464in}}%
\pgfpathlineto{\pgfqpoint{1.280910in}{3.182236in}}%
\pgfpathlineto{\pgfqpoint{1.288403in}{3.178557in}}%
\pgfpathlineto{\pgfqpoint{1.296699in}{3.171825in}}%
\pgfpathlineto{\pgfqpoint{1.306332in}{3.160967in}}%
\pgfpathlineto{\pgfqpoint{1.318910in}{3.143126in}}%
\pgfpathlineto{\pgfqpoint{1.359051in}{3.083468in}}%
\pgfpathlineto{\pgfqpoint{1.368953in}{3.074024in}}%
\pgfpathlineto{\pgfqpoint{1.377248in}{3.069033in}}%
\pgfpathlineto{\pgfqpoint{1.384741in}{3.067057in}}%
\pgfpathlineto{\pgfqpoint{1.391699in}{3.067454in}}%
\pgfpathlineto{\pgfqpoint{1.398925in}{3.070127in}}%
\pgfpathlineto{\pgfqpoint{1.406685in}{3.075440in}}%
\pgfpathlineto{\pgfqpoint{1.415784in}{3.084560in}}%
\pgfpathlineto{\pgfqpoint{1.427024in}{3.099290in}}%
\pgfpathlineto{\pgfqpoint{1.444418in}{3.126405in}}%
\pgfpathlineto{\pgfqpoint{1.465291in}{3.158029in}}%
\pgfpathlineto{\pgfqpoint{1.476531in}{3.171191in}}%
\pgfpathlineto{\pgfqpoint{1.485630in}{3.178657in}}%
\pgfpathlineto{\pgfqpoint{1.493658in}{3.182453in}}%
\pgfpathlineto{\pgfqpoint{1.500883in}{3.183465in}}%
\pgfpathlineto{\pgfqpoint{1.507841in}{3.182243in}}%
\pgfpathlineto{\pgfqpoint{1.515334in}{3.178570in}}%
\pgfpathlineto{\pgfqpoint{1.523630in}{3.171845in}}%
\pgfpathlineto{\pgfqpoint{1.533264in}{3.160993in}}%
\pgfpathlineto{\pgfqpoint{1.545841in}{3.143158in}}%
\pgfpathlineto{\pgfqpoint{1.585983in}{3.083491in}}%
\pgfpathlineto{\pgfqpoint{1.595884in}{3.074040in}}%
\pgfpathlineto{\pgfqpoint{1.604180in}{3.069042in}}%
\pgfpathlineto{\pgfqpoint{1.611673in}{3.067059in}}%
\pgfpathlineto{\pgfqpoint{1.618631in}{3.067450in}}%
\pgfpathlineto{\pgfqpoint{1.625856in}{3.070117in}}%
\pgfpathlineto{\pgfqpoint{1.633617in}{3.075423in}}%
\pgfpathlineto{\pgfqpoint{1.642715in}{3.084536in}}%
\pgfpathlineto{\pgfqpoint{1.653955in}{3.099260in}}%
\pgfpathlineto{\pgfqpoint{1.671349in}{3.126371in}}%
\pgfpathlineto{\pgfqpoint{1.692490in}{3.158357in}}%
\pgfpathlineto{\pgfqpoint{1.703730in}{3.171434in}}%
\pgfpathlineto{\pgfqpoint{1.712829in}{3.178814in}}%
\pgfpathlineto{\pgfqpoint{1.720589in}{3.182447in}}%
\pgfpathlineto{\pgfqpoint{1.727815in}{3.183465in}}%
\pgfpathlineto{\pgfqpoint{1.734773in}{3.182249in}}%
\pgfpathlineto{\pgfqpoint{1.742266in}{3.178584in}}%
\pgfpathlineto{\pgfqpoint{1.750561in}{3.171865in}}%
\pgfpathlineto{\pgfqpoint{1.760195in}{3.161019in}}%
\pgfpathlineto{\pgfqpoint{1.772773in}{3.143189in}}%
\pgfpathlineto{\pgfqpoint{1.812914in}{3.083514in}}%
\pgfpathlineto{\pgfqpoint{1.822815in}{3.074056in}}%
\pgfpathlineto{\pgfqpoint{1.831111in}{3.069051in}}%
\pgfpathlineto{\pgfqpoint{1.838604in}{3.067061in}}%
\pgfpathlineto{\pgfqpoint{1.845562in}{3.067446in}}%
\pgfpathlineto{\pgfqpoint{1.852788in}{3.070106in}}%
\pgfpathlineto{\pgfqpoint{1.860548in}{3.075406in}}%
\pgfpathlineto{\pgfqpoint{1.869647in}{3.084512in}}%
\pgfpathlineto{\pgfqpoint{1.880886in}{3.099231in}}%
\pgfpathlineto{\pgfqpoint{1.898281in}{3.126338in}}%
\pgfpathlineto{\pgfqpoint{1.919422in}{3.158330in}}%
\pgfpathlineto{\pgfqpoint{1.930661in}{3.171414in}}%
\pgfpathlineto{\pgfqpoint{1.939760in}{3.178801in}}%
\pgfpathlineto{\pgfqpoint{1.947521in}{3.182441in}}%
\pgfpathlineto{\pgfqpoint{1.954746in}{3.183466in}}%
\pgfpathlineto{\pgfqpoint{1.961704in}{3.182256in}}%
\pgfpathlineto{\pgfqpoint{1.969197in}{3.178597in}}%
\pgfpathlineto{\pgfqpoint{1.977493in}{3.171885in}}%
\pgfpathlineto{\pgfqpoint{1.987127in}{3.161045in}}%
\pgfpathlineto{\pgfqpoint{1.999704in}{3.143221in}}%
\pgfpathlineto{\pgfqpoint{2.039845in}{3.083538in}}%
\pgfpathlineto{\pgfqpoint{2.049747in}{3.074072in}}%
\pgfpathlineto{\pgfqpoint{2.058043in}{3.069060in}}%
\pgfpathlineto{\pgfqpoint{2.065536in}{3.067063in}}%
\pgfpathlineto{\pgfqpoint{2.072494in}{3.067441in}}%
\pgfpathlineto{\pgfqpoint{2.079719in}{3.070095in}}%
\pgfpathlineto{\pgfqpoint{2.087480in}{3.075389in}}%
\pgfpathlineto{\pgfqpoint{2.096578in}{3.084488in}}%
\pgfpathlineto{\pgfqpoint{2.107550in}{3.098815in}}%
\pgfpathlineto{\pgfqpoint{2.124410in}{3.125010in}}%
\pgfpathlineto{\pgfqpoint{2.146353in}{3.158302in}}%
\pgfpathlineto{\pgfqpoint{2.157593in}{3.171394in}}%
\pgfpathlineto{\pgfqpoint{2.166692in}{3.178788in}}%
\pgfpathlineto{\pgfqpoint{2.174452in}{3.182435in}}%
\pgfpathlineto{\pgfqpoint{2.181678in}{3.183466in}}%
\pgfpathlineto{\pgfqpoint{2.188635in}{3.182263in}}%
\pgfpathlineto{\pgfqpoint{2.196128in}{3.178610in}}%
\pgfpathlineto{\pgfqpoint{2.204424in}{3.171905in}}%
\pgfpathlineto{\pgfqpoint{2.214058in}{3.161071in}}%
\pgfpathlineto{\pgfqpoint{2.226636in}{3.143252in}}%
\pgfpathlineto{\pgfqpoint{2.266777in}{3.083561in}}%
\pgfpathlineto{\pgfqpoint{2.276678in}{3.074088in}}%
\pgfpathlineto{\pgfqpoint{2.284974in}{3.069069in}}%
\pgfpathlineto{\pgfqpoint{2.292467in}{3.067065in}}%
\pgfpathlineto{\pgfqpoint{2.299425in}{3.067437in}}%
\pgfpathlineto{\pgfqpoint{2.306650in}{3.070084in}}%
\pgfpathlineto{\pgfqpoint{2.314411in}{3.075371in}}%
\pgfpathlineto{\pgfqpoint{2.323510in}{3.084465in}}%
\pgfpathlineto{\pgfqpoint{2.334482in}{3.098786in}}%
\pgfpathlineto{\pgfqpoint{2.351341in}{3.124977in}}%
\pgfpathlineto{\pgfqpoint{2.373285in}{3.158275in}}%
\pgfpathlineto{\pgfqpoint{2.384524in}{3.171373in}}%
\pgfpathlineto{\pgfqpoint{2.393623in}{3.178775in}}%
\pgfpathlineto{\pgfqpoint{2.401384in}{3.182428in}}%
\pgfpathlineto{\pgfqpoint{2.408609in}{3.183466in}}%
\pgfpathlineto{\pgfqpoint{2.415567in}{3.182270in}}%
\pgfpathlineto{\pgfqpoint{2.423060in}{3.178623in}}%
\pgfpathlineto{\pgfqpoint{2.431356in}{3.171925in}}%
\pgfpathlineto{\pgfqpoint{2.440990in}{3.161098in}}%
\pgfpathlineto{\pgfqpoint{2.453567in}{3.143284in}}%
\pgfpathlineto{\pgfqpoint{2.493976in}{3.083283in}}%
\pgfpathlineto{\pgfqpoint{2.503877in}{3.073898in}}%
\pgfpathlineto{\pgfqpoint{2.512173in}{3.068964in}}%
\pgfpathlineto{\pgfqpoint{2.519666in}{3.067041in}}%
\pgfpathlineto{\pgfqpoint{2.526624in}{3.067490in}}%
\pgfpathlineto{\pgfqpoint{2.533849in}{3.070215in}}%
\pgfpathlineto{\pgfqpoint{2.541610in}{3.075579in}}%
\pgfpathlineto{\pgfqpoint{2.550709in}{3.084750in}}%
\pgfpathlineto{\pgfqpoint{2.561948in}{3.099528in}}%
\pgfpathlineto{\pgfqpoint{2.579610in}{3.127102in}}%
\pgfpathlineto{\pgfqpoint{2.600216in}{3.158248in}}%
\pgfpathlineto{\pgfqpoint{2.611456in}{3.171353in}}%
\pgfpathlineto{\pgfqpoint{2.620554in}{3.178762in}}%
\pgfpathlineto{\pgfqpoint{2.628315in}{3.182422in}}%
\pgfpathlineto{\pgfqpoint{2.635540in}{3.183467in}}%
\pgfpathlineto{\pgfqpoint{2.642498in}{3.182276in}}%
\pgfpathlineto{\pgfqpoint{2.649991in}{3.178637in}}%
\pgfpathlineto{\pgfqpoint{2.658287in}{3.171945in}}%
\pgfpathlineto{\pgfqpoint{2.667921in}{3.161124in}}%
\pgfpathlineto{\pgfqpoint{2.680499in}{3.143315in}}%
\pgfpathlineto{\pgfqpoint{2.720907in}{3.083306in}}%
\pgfpathlineto{\pgfqpoint{2.730809in}{3.073914in}}%
\pgfpathlineto{\pgfqpoint{2.739105in}{3.068972in}}%
\pgfpathlineto{\pgfqpoint{2.746598in}{3.067043in}}%
\pgfpathlineto{\pgfqpoint{2.753555in}{3.067485in}}%
\pgfpathlineto{\pgfqpoint{2.760781in}{3.070204in}}%
\pgfpathlineto{\pgfqpoint{2.768542in}{3.075562in}}%
\pgfpathlineto{\pgfqpoint{2.777640in}{3.084727in}}%
\pgfpathlineto{\pgfqpoint{2.788880in}{3.099498in}}%
\pgfpathlineto{\pgfqpoint{2.806274in}{3.126637in}}%
\pgfpathlineto{\pgfqpoint{2.827148in}{3.158220in}}%
\pgfpathlineto{\pgfqpoint{2.838387in}{3.171333in}}%
\pgfpathlineto{\pgfqpoint{2.847486in}{3.178749in}}%
\pgfpathlineto{\pgfqpoint{2.855246in}{3.182416in}}%
\pgfpathlineto{\pgfqpoint{2.862472in}{3.183467in}}%
\pgfpathlineto{\pgfqpoint{2.869430in}{3.182283in}}%
\pgfpathlineto{\pgfqpoint{2.876923in}{3.178650in}}%
\pgfpathlineto{\pgfqpoint{2.885219in}{3.171964in}}%
\pgfpathlineto{\pgfqpoint{2.894852in}{3.161150in}}%
\pgfpathlineto{\pgfqpoint{2.907430in}{3.143347in}}%
\pgfpathlineto{\pgfqpoint{2.947839in}{3.083329in}}%
\pgfpathlineto{\pgfqpoint{2.957740in}{3.073929in}}%
\pgfpathlineto{\pgfqpoint{2.966036in}{3.068981in}}%
\pgfpathlineto{\pgfqpoint{2.973529in}{3.067045in}}%
\pgfpathlineto{\pgfqpoint{2.980487in}{3.067481in}}%
\pgfpathlineto{\pgfqpoint{2.987712in}{3.070193in}}%
\pgfpathlineto{\pgfqpoint{2.995473in}{3.075544in}}%
\pgfpathlineto{\pgfqpoint{3.004572in}{3.084703in}}%
\pgfpathlineto{\pgfqpoint{3.015811in}{3.099469in}}%
\pgfpathlineto{\pgfqpoint{3.033206in}{3.126604in}}%
\pgfpathlineto{\pgfqpoint{3.054079in}{3.158193in}}%
\pgfpathlineto{\pgfqpoint{3.065319in}{3.171313in}}%
\pgfpathlineto{\pgfqpoint{3.074417in}{3.178736in}}%
\pgfpathlineto{\pgfqpoint{3.082178in}{3.182409in}}%
\pgfpathlineto{\pgfqpoint{3.089403in}{3.183467in}}%
\pgfpathlineto{\pgfqpoint{3.096361in}{3.182290in}}%
\pgfpathlineto{\pgfqpoint{3.103854in}{3.178663in}}%
\pgfpathlineto{\pgfqpoint{3.112150in}{3.171984in}}%
\pgfpathlineto{\pgfqpoint{3.121784in}{3.161176in}}%
\pgfpathlineto{\pgfqpoint{3.134361in}{3.143379in}}%
\pgfpathlineto{\pgfqpoint{3.174770in}{3.083352in}}%
\pgfpathlineto{\pgfqpoint{3.184672in}{3.073945in}}%
\pgfpathlineto{\pgfqpoint{3.192968in}{3.068990in}}%
\pgfpathlineto{\pgfqpoint{3.200461in}{3.067047in}}%
\pgfpathlineto{\pgfqpoint{3.207418in}{3.067477in}}%
\pgfpathlineto{\pgfqpoint{3.214644in}{3.070182in}}%
\pgfpathlineto{\pgfqpoint{3.222404in}{3.075527in}}%
\pgfpathlineto{\pgfqpoint{3.231503in}{3.084679in}}%
\pgfpathlineto{\pgfqpoint{3.242743in}{3.099439in}}%
\pgfpathlineto{\pgfqpoint{3.260137in}{3.126571in}}%
\pgfpathlineto{\pgfqpoint{3.281011in}{3.158166in}}%
\pgfpathlineto{\pgfqpoint{3.292250in}{3.171292in}}%
\pgfpathlineto{\pgfqpoint{3.301349in}{3.178722in}}%
\pgfpathlineto{\pgfqpoint{3.309377in}{3.182484in}}%
\pgfpathlineto{\pgfqpoint{3.316602in}{3.183462in}}%
\pgfpathlineto{\pgfqpoint{3.323560in}{3.182209in}}%
\pgfpathlineto{\pgfqpoint{3.331053in}{3.178503in}}%
\pgfpathlineto{\pgfqpoint{3.339349in}{3.171746in}}%
\pgfpathlineto{\pgfqpoint{3.348983in}{3.160862in}}%
\pgfpathlineto{\pgfqpoint{3.361560in}{3.143000in}}%
\pgfpathlineto{\pgfqpoint{3.401434in}{3.083677in}}%
\pgfpathlineto{\pgfqpoint{3.411336in}{3.074168in}}%
\pgfpathlineto{\pgfqpoint{3.419631in}{3.069113in}}%
\pgfpathlineto{\pgfqpoint{3.427124in}{3.067076in}}%
\pgfpathlineto{\pgfqpoint{3.434082in}{3.067416in}}%
\pgfpathlineto{\pgfqpoint{3.441308in}{3.070031in}}%
\pgfpathlineto{\pgfqpoint{3.449068in}{3.075286in}}%
\pgfpathlineto{\pgfqpoint{3.457899in}{3.084040in}}%
\pgfpathlineto{\pgfqpoint{3.468871in}{3.098255in}}%
\pgfpathlineto{\pgfqpoint{3.485463in}{3.123948in}}%
\pgfpathlineto{\pgfqpoint{3.507942in}{3.158138in}}%
\pgfpathlineto{\pgfqpoint{3.519181in}{3.171272in}}%
\pgfpathlineto{\pgfqpoint{3.528280in}{3.178709in}}%
\pgfpathlineto{\pgfqpoint{3.536308in}{3.182478in}}%
\pgfpathlineto{\pgfqpoint{3.543534in}{3.183463in}}%
\pgfpathlineto{\pgfqpoint{3.550492in}{3.182215in}}%
\pgfpathlineto{\pgfqpoint{3.557985in}{3.178517in}}%
\pgfpathlineto{\pgfqpoint{3.566280in}{3.171766in}}%
\pgfpathlineto{\pgfqpoint{3.575914in}{3.160888in}}%
\pgfpathlineto{\pgfqpoint{3.588492in}{3.143031in}}%
\pgfpathlineto{\pgfqpoint{3.628365in}{3.083700in}}%
\pgfpathlineto{\pgfqpoint{3.638267in}{3.074184in}}%
\pgfpathlineto{\pgfqpoint{3.646563in}{3.069122in}}%
\pgfpathlineto{\pgfqpoint{3.654056in}{3.067078in}}%
\pgfpathlineto{\pgfqpoint{3.661014in}{3.067412in}}%
\pgfpathlineto{\pgfqpoint{3.668239in}{3.070020in}}%
\pgfpathlineto{\pgfqpoint{3.676000in}{3.075269in}}%
\pgfpathlineto{\pgfqpoint{3.684831in}{3.084016in}}%
\pgfpathlineto{\pgfqpoint{3.695803in}{3.098225in}}%
\pgfpathlineto{\pgfqpoint{3.712394in}{3.123915in}}%
\pgfpathlineto{\pgfqpoint{3.734873in}{3.158111in}}%
\pgfpathlineto{\pgfqpoint{3.746113in}{3.171252in}}%
\pgfpathlineto{\pgfqpoint{3.755212in}{3.178696in}}%
\pgfpathlineto{\pgfqpoint{3.763240in}{3.182472in}}%
\pgfpathlineto{\pgfqpoint{3.770465in}{3.183463in}}%
\pgfpathlineto{\pgfqpoint{3.777423in}{3.182222in}}%
\pgfpathlineto{\pgfqpoint{3.784916in}{3.178530in}}%
\pgfpathlineto{\pgfqpoint{3.793212in}{3.171785in}}%
\pgfpathlineto{\pgfqpoint{3.802846in}{3.160914in}}%
\pgfpathlineto{\pgfqpoint{3.815423in}{3.143063in}}%
\pgfpathlineto{\pgfqpoint{3.855297in}{3.083724in}}%
\pgfpathlineto{\pgfqpoint{3.865198in}{3.074200in}}%
\pgfpathlineto{\pgfqpoint{3.873494in}{3.069131in}}%
\pgfpathlineto{\pgfqpoint{3.880987in}{3.067080in}}%
\pgfpathlineto{\pgfqpoint{3.887945in}{3.067408in}}%
\pgfpathlineto{\pgfqpoint{3.895170in}{3.070010in}}%
\pgfpathlineto{\pgfqpoint{3.902931in}{3.075251in}}%
\pgfpathlineto{\pgfqpoint{3.911762in}{3.083993in}}%
\pgfpathlineto{\pgfqpoint{3.922734in}{3.098196in}}%
\pgfpathlineto{\pgfqpoint{3.939326in}{3.123881in}}%
\pgfpathlineto{\pgfqpoint{3.961805in}{3.158083in}}%
\pgfpathlineto{\pgfqpoint{3.973044in}{3.171231in}}%
\pgfpathlineto{\pgfqpoint{3.982143in}{3.178683in}}%
\pgfpathlineto{\pgfqpoint{3.990171in}{3.182465in}}%
\pgfpathlineto{\pgfqpoint{3.997397in}{3.183464in}}%
\pgfpathlineto{\pgfqpoint{4.004354in}{3.182229in}}%
\pgfpathlineto{\pgfqpoint{4.011847in}{3.178544in}}%
\pgfpathlineto{\pgfqpoint{4.020143in}{3.171805in}}%
\pgfpathlineto{\pgfqpoint{4.029777in}{3.160940in}}%
\pgfpathlineto{\pgfqpoint{4.042355in}{3.143094in}}%
\pgfpathlineto{\pgfqpoint{4.082228in}{3.083747in}}%
\pgfpathlineto{\pgfqpoint{4.092130in}{3.074216in}}%
\pgfpathlineto{\pgfqpoint{4.100426in}{3.069140in}}%
\pgfpathlineto{\pgfqpoint{4.107919in}{3.067083in}}%
\pgfpathlineto{\pgfqpoint{4.114876in}{3.067403in}}%
\pgfpathlineto{\pgfqpoint{4.122102in}{3.069999in}}%
\pgfpathlineto{\pgfqpoint{4.129863in}{3.075234in}}%
\pgfpathlineto{\pgfqpoint{4.138694in}{3.083969in}}%
\pgfpathlineto{\pgfqpoint{4.149665in}{3.098166in}}%
\pgfpathlineto{\pgfqpoint{4.166257in}{3.123848in}}%
\pgfpathlineto{\pgfqpoint{4.188736in}{3.158056in}}%
\pgfpathlineto{\pgfqpoint{4.199976in}{3.171211in}}%
\pgfpathlineto{\pgfqpoint{4.209074in}{3.178670in}}%
\pgfpathlineto{\pgfqpoint{4.217103in}{3.182459in}}%
\pgfpathlineto{\pgfqpoint{4.224328in}{3.183464in}}%
\pgfpathlineto{\pgfqpoint{4.231286in}{3.182236in}}%
\pgfpathlineto{\pgfqpoint{4.238779in}{3.178557in}}%
\pgfpathlineto{\pgfqpoint{4.247075in}{3.171825in}}%
\pgfpathlineto{\pgfqpoint{4.256709in}{3.160967in}}%
\pgfpathlineto{\pgfqpoint{4.269286in}{3.143126in}}%
\pgfpathlineto{\pgfqpoint{4.309427in}{3.083468in}}%
\pgfpathlineto{\pgfqpoint{4.319329in}{3.074024in}}%
\pgfpathlineto{\pgfqpoint{4.327625in}{3.069033in}}%
\pgfpathlineto{\pgfqpoint{4.335118in}{3.067057in}}%
\pgfpathlineto{\pgfqpoint{4.342076in}{3.067454in}}%
\pgfpathlineto{\pgfqpoint{4.349301in}{3.070127in}}%
\pgfpathlineto{\pgfqpoint{4.357062in}{3.075440in}}%
\pgfpathlineto{\pgfqpoint{4.366160in}{3.084560in}}%
\pgfpathlineto{\pgfqpoint{4.377400in}{3.099290in}}%
\pgfpathlineto{\pgfqpoint{4.394794in}{3.126405in}}%
\pgfpathlineto{\pgfqpoint{4.415668in}{3.158029in}}%
\pgfpathlineto{\pgfqpoint{4.426907in}{3.171191in}}%
\pgfpathlineto{\pgfqpoint{4.436006in}{3.178657in}}%
\pgfpathlineto{\pgfqpoint{4.444034in}{3.182453in}}%
\pgfpathlineto{\pgfqpoint{4.451260in}{3.183465in}}%
\pgfpathlineto{\pgfqpoint{4.458217in}{3.182243in}}%
\pgfpathlineto{\pgfqpoint{4.465710in}{3.178570in}}%
\pgfpathlineto{\pgfqpoint{4.474006in}{3.171845in}}%
\pgfpathlineto{\pgfqpoint{4.483640in}{3.160993in}}%
\pgfpathlineto{\pgfqpoint{4.496218in}{3.143158in}}%
\pgfpathlineto{\pgfqpoint{4.536359in}{3.083491in}}%
\pgfpathlineto{\pgfqpoint{4.546260in}{3.074040in}}%
\pgfpathlineto{\pgfqpoint{4.554556in}{3.069042in}}%
\pgfpathlineto{\pgfqpoint{4.562049in}{3.067059in}}%
\pgfpathlineto{\pgfqpoint{4.569007in}{3.067450in}}%
\pgfpathlineto{\pgfqpoint{4.576232in}{3.070117in}}%
\pgfpathlineto{\pgfqpoint{4.583993in}{3.075423in}}%
\pgfpathlineto{\pgfqpoint{4.593092in}{3.084536in}}%
\pgfpathlineto{\pgfqpoint{4.604331in}{3.099260in}}%
\pgfpathlineto{\pgfqpoint{4.621726in}{3.126371in}}%
\pgfpathlineto{\pgfqpoint{4.642867in}{3.158357in}}%
\pgfpathlineto{\pgfqpoint{4.654106in}{3.171434in}}%
\pgfpathlineto{\pgfqpoint{4.663205in}{3.178814in}}%
\pgfpathlineto{\pgfqpoint{4.670966in}{3.182447in}}%
\pgfpathlineto{\pgfqpoint{4.678191in}{3.183465in}}%
\pgfpathlineto{\pgfqpoint{4.685149in}{3.182249in}}%
\pgfpathlineto{\pgfqpoint{4.692642in}{3.178584in}}%
\pgfpathlineto{\pgfqpoint{4.700938in}{3.171865in}}%
\pgfpathlineto{\pgfqpoint{4.710571in}{3.161019in}}%
\pgfpathlineto{\pgfqpoint{4.723149in}{3.143189in}}%
\pgfpathlineto{\pgfqpoint{4.763290in}{3.083514in}}%
\pgfpathlineto{\pgfqpoint{4.773192in}{3.074056in}}%
\pgfpathlineto{\pgfqpoint{4.781488in}{3.069051in}}%
\pgfpathlineto{\pgfqpoint{4.788981in}{3.067061in}}%
\pgfpathlineto{\pgfqpoint{4.795938in}{3.067446in}}%
\pgfpathlineto{\pgfqpoint{4.803164in}{3.070106in}}%
\pgfpathlineto{\pgfqpoint{4.810924in}{3.075406in}}%
\pgfpathlineto{\pgfqpoint{4.820023in}{3.084512in}}%
\pgfpathlineto{\pgfqpoint{4.831263in}{3.099231in}}%
\pgfpathlineto{\pgfqpoint{4.848657in}{3.126338in}}%
\pgfpathlineto{\pgfqpoint{4.869798in}{3.158330in}}%
\pgfpathlineto{\pgfqpoint{4.881038in}{3.171414in}}%
\pgfpathlineto{\pgfqpoint{4.890136in}{3.178801in}}%
\pgfpathlineto{\pgfqpoint{4.897897in}{3.182441in}}%
\pgfpathlineto{\pgfqpoint{4.905122in}{3.183466in}}%
\pgfpathlineto{\pgfqpoint{4.912080in}{3.182256in}}%
\pgfpathlineto{\pgfqpoint{4.919573in}{3.178597in}}%
\pgfpathlineto{\pgfqpoint{4.927869in}{3.171885in}}%
\pgfpathlineto{\pgfqpoint{4.937503in}{3.161045in}}%
\pgfpathlineto{\pgfqpoint{4.950080in}{3.143221in}}%
\pgfpathlineto{\pgfqpoint{4.990222in}{3.083538in}}%
\pgfpathlineto{\pgfqpoint{5.000123in}{3.074072in}}%
\pgfpathlineto{\pgfqpoint{5.008419in}{3.069060in}}%
\pgfpathlineto{\pgfqpoint{5.015912in}{3.067063in}}%
\pgfpathlineto{\pgfqpoint{5.022870in}{3.067441in}}%
\pgfpathlineto{\pgfqpoint{5.030095in}{3.070095in}}%
\pgfpathlineto{\pgfqpoint{5.037856in}{3.075389in}}%
\pgfpathlineto{\pgfqpoint{5.046955in}{3.084488in}}%
\pgfpathlineto{\pgfqpoint{5.057926in}{3.098815in}}%
\pgfpathlineto{\pgfqpoint{5.074786in}{3.125010in}}%
\pgfpathlineto{\pgfqpoint{5.096730in}{3.158302in}}%
\pgfpathlineto{\pgfqpoint{5.107969in}{3.171394in}}%
\pgfpathlineto{\pgfqpoint{5.117068in}{3.178788in}}%
\pgfpathlineto{\pgfqpoint{5.124828in}{3.182435in}}%
\pgfpathlineto{\pgfqpoint{5.132054in}{3.183466in}}%
\pgfpathlineto{\pgfqpoint{5.139012in}{3.182263in}}%
\pgfpathlineto{\pgfqpoint{5.146505in}{3.178610in}}%
\pgfpathlineto{\pgfqpoint{5.154800in}{3.171905in}}%
\pgfpathlineto{\pgfqpoint{5.164434in}{3.161071in}}%
\pgfpathlineto{\pgfqpoint{5.177012in}{3.143252in}}%
\pgfpathlineto{\pgfqpoint{5.217153in}{3.083561in}}%
\pgfpathlineto{\pgfqpoint{5.227055in}{3.074088in}}%
\pgfpathlineto{\pgfqpoint{5.235350in}{3.069069in}}%
\pgfpathlineto{\pgfqpoint{5.242843in}{3.067065in}}%
\pgfpathlineto{\pgfqpoint{5.249801in}{3.067437in}}%
\pgfpathlineto{\pgfqpoint{5.257027in}{3.070084in}}%
\pgfpathlineto{\pgfqpoint{5.264787in}{3.075371in}}%
\pgfpathlineto{\pgfqpoint{5.273886in}{3.084465in}}%
\pgfpathlineto{\pgfqpoint{5.284858in}{3.098786in}}%
\pgfpathlineto{\pgfqpoint{5.301717in}{3.124977in}}%
\pgfpathlineto{\pgfqpoint{5.323661in}{3.158275in}}%
\pgfpathlineto{\pgfqpoint{5.334901in}{3.171373in}}%
\pgfpathlineto{\pgfqpoint{5.343999in}{3.178775in}}%
\pgfpathlineto{\pgfqpoint{5.351760in}{3.182428in}}%
\pgfpathlineto{\pgfqpoint{5.358985in}{3.183466in}}%
\pgfpathlineto{\pgfqpoint{5.365943in}{3.182270in}}%
\pgfpathlineto{\pgfqpoint{5.373436in}{3.178623in}}%
\pgfpathlineto{\pgfqpoint{5.381732in}{3.171925in}}%
\pgfpathlineto{\pgfqpoint{5.391366in}{3.161098in}}%
\pgfpathlineto{\pgfqpoint{5.403943in}{3.143284in}}%
\pgfpathlineto{\pgfqpoint{5.444352in}{3.083283in}}%
\pgfpathlineto{\pgfqpoint{5.454254in}{3.073898in}}%
\pgfpathlineto{\pgfqpoint{5.462549in}{3.068964in}}%
\pgfpathlineto{\pgfqpoint{5.470042in}{3.067041in}}%
\pgfpathlineto{\pgfqpoint{5.477000in}{3.067490in}}%
\pgfpathlineto{\pgfqpoint{5.484226in}{3.070215in}}%
\pgfpathlineto{\pgfqpoint{5.491986in}{3.075579in}}%
\pgfpathlineto{\pgfqpoint{5.501085in}{3.084750in}}%
\pgfpathlineto{\pgfqpoint{5.512325in}{3.099528in}}%
\pgfpathlineto{\pgfqpoint{5.529987in}{3.127102in}}%
\pgfpathlineto{\pgfqpoint{5.550592in}{3.158248in}}%
\pgfpathlineto{\pgfqpoint{5.561832in}{3.171353in}}%
\pgfpathlineto{\pgfqpoint{5.570931in}{3.178762in}}%
\pgfpathlineto{\pgfqpoint{5.578691in}{3.182422in}}%
\pgfpathlineto{\pgfqpoint{5.585917in}{3.183467in}}%
\pgfpathlineto{\pgfqpoint{5.592874in}{3.182276in}}%
\pgfpathlineto{\pgfqpoint{5.600367in}{3.178637in}}%
\pgfpathlineto{\pgfqpoint{5.608663in}{3.171945in}}%
\pgfpathlineto{\pgfqpoint{5.618297in}{3.161124in}}%
\pgfpathlineto{\pgfqpoint{5.630875in}{3.143315in}}%
\pgfpathlineto{\pgfqpoint{5.671284in}{3.083306in}}%
\pgfpathlineto{\pgfqpoint{5.681185in}{3.073914in}}%
\pgfpathlineto{\pgfqpoint{5.689481in}{3.068972in}}%
\pgfpathlineto{\pgfqpoint{5.696974in}{3.067043in}}%
\pgfpathlineto{\pgfqpoint{5.703932in}{3.067485in}}%
\pgfpathlineto{\pgfqpoint{5.711157in}{3.070204in}}%
\pgfpathlineto{\pgfqpoint{5.718918in}{3.075562in}}%
\pgfpathlineto{\pgfqpoint{5.728016in}{3.084727in}}%
\pgfpathlineto{\pgfqpoint{5.739256in}{3.099498in}}%
\pgfpathlineto{\pgfqpoint{5.756650in}{3.126637in}}%
\pgfpathlineto{\pgfqpoint{5.777524in}{3.158220in}}%
\pgfpathlineto{\pgfqpoint{5.788763in}{3.171333in}}%
\pgfpathlineto{\pgfqpoint{5.797862in}{3.178749in}}%
\pgfpathlineto{\pgfqpoint{5.805623in}{3.182416in}}%
\pgfpathlineto{\pgfqpoint{5.812848in}{3.183467in}}%
\pgfpathlineto{\pgfqpoint{5.819806in}{3.182283in}}%
\pgfpathlineto{\pgfqpoint{5.827299in}{3.178650in}}%
\pgfpathlineto{\pgfqpoint{5.835595in}{3.171964in}}%
\pgfpathlineto{\pgfqpoint{5.845229in}{3.161150in}}%
\pgfpathlineto{\pgfqpoint{5.857806in}{3.143347in}}%
\pgfpathlineto{\pgfqpoint{5.898215in}{3.083329in}}%
\pgfpathlineto{\pgfqpoint{5.908116in}{3.073929in}}%
\pgfpathlineto{\pgfqpoint{5.916412in}{3.068981in}}%
\pgfpathlineto{\pgfqpoint{5.923905in}{3.067045in}}%
\pgfpathlineto{\pgfqpoint{5.930863in}{3.067481in}}%
\pgfpathlineto{\pgfqpoint{5.938089in}{3.070193in}}%
\pgfpathlineto{\pgfqpoint{5.945849in}{3.075544in}}%
\pgfpathlineto{\pgfqpoint{5.954948in}{3.084703in}}%
\pgfpathlineto{\pgfqpoint{5.966187in}{3.099469in}}%
\pgfpathlineto{\pgfqpoint{5.983582in}{3.126604in}}%
\pgfpathlineto{\pgfqpoint{6.004455in}{3.158193in}}%
\pgfpathlineto{\pgfqpoint{6.015695in}{3.171313in}}%
\pgfpathlineto{\pgfqpoint{6.024793in}{3.178736in}}%
\pgfpathlineto{\pgfqpoint{6.032554in}{3.182409in}}%
\pgfpathlineto{\pgfqpoint{6.039780in}{3.183467in}}%
\pgfpathlineto{\pgfqpoint{6.046737in}{3.182290in}}%
\pgfpathlineto{\pgfqpoint{6.054230in}{3.178663in}}%
\pgfpathlineto{\pgfqpoint{6.062526in}{3.171984in}}%
\pgfpathlineto{\pgfqpoint{6.072160in}{3.161176in}}%
\pgfpathlineto{\pgfqpoint{6.084738in}{3.143379in}}%
\pgfpathlineto{\pgfqpoint{6.125146in}{3.083352in}}%
\pgfpathlineto{\pgfqpoint{6.135048in}{3.073945in}}%
\pgfpathlineto{\pgfqpoint{6.143344in}{3.068990in}}%
\pgfpathlineto{\pgfqpoint{6.150837in}{3.067047in}}%
\pgfpathlineto{\pgfqpoint{6.157795in}{3.067477in}}%
\pgfpathlineto{\pgfqpoint{6.165020in}{3.070182in}}%
\pgfpathlineto{\pgfqpoint{6.172781in}{3.075527in}}%
\pgfpathlineto{\pgfqpoint{6.181879in}{3.084679in}}%
\pgfpathlineto{\pgfqpoint{6.193119in}{3.099439in}}%
\pgfpathlineto{\pgfqpoint{6.210513in}{3.126571in}}%
\pgfpathlineto{\pgfqpoint{6.231387in}{3.158166in}}%
\pgfpathlineto{\pgfqpoint{6.242626in}{3.171292in}}%
\pgfpathlineto{\pgfqpoint{6.251725in}{3.178722in}}%
\pgfpathlineto{\pgfqpoint{6.259753in}{3.182484in}}%
\pgfpathlineto{\pgfqpoint{6.266979in}{3.183462in}}%
\pgfpathlineto{\pgfqpoint{6.273936in}{3.182209in}}%
\pgfpathlineto{\pgfqpoint{6.281429in}{3.178503in}}%
\pgfpathlineto{\pgfqpoint{6.289725in}{3.171746in}}%
\pgfpathlineto{\pgfqpoint{6.299359in}{3.160862in}}%
\pgfpathlineto{\pgfqpoint{6.311937in}{3.143000in}}%
\pgfpathlineto{\pgfqpoint{6.351810in}{3.083677in}}%
\pgfpathlineto{\pgfqpoint{6.361712in}{3.074168in}}%
\pgfpathlineto{\pgfqpoint{6.370008in}{3.069113in}}%
\pgfpathlineto{\pgfqpoint{6.377501in}{3.067076in}}%
\pgfpathlineto{\pgfqpoint{6.384458in}{3.067416in}}%
\pgfpathlineto{\pgfqpoint{6.391684in}{3.070031in}}%
\pgfpathlineto{\pgfqpoint{6.399444in}{3.075286in}}%
\pgfpathlineto{\pgfqpoint{6.408276in}{3.084040in}}%
\pgfpathlineto{\pgfqpoint{6.419247in}{3.098255in}}%
\pgfpathlineto{\pgfqpoint{6.435839in}{3.123948in}}%
\pgfpathlineto{\pgfqpoint{6.458318in}{3.158138in}}%
\pgfpathlineto{\pgfqpoint{6.469558in}{3.171272in}}%
\pgfpathlineto{\pgfqpoint{6.478656in}{3.178709in}}%
\pgfpathlineto{\pgfqpoint{6.486685in}{3.182478in}}%
\pgfpathlineto{\pgfqpoint{6.493910in}{3.183463in}}%
\pgfpathlineto{\pgfqpoint{6.500868in}{3.182215in}}%
\pgfpathlineto{\pgfqpoint{6.508361in}{3.178517in}}%
\pgfpathlineto{\pgfqpoint{6.516657in}{3.171766in}}%
\pgfpathlineto{\pgfqpoint{6.526291in}{3.160888in}}%
\pgfpathlineto{\pgfqpoint{6.538868in}{3.143031in}}%
\pgfpathlineto{\pgfqpoint{6.578742in}{3.083700in}}%
\pgfpathlineto{\pgfqpoint{6.588643in}{3.074184in}}%
\pgfpathlineto{\pgfqpoint{6.596939in}{3.069122in}}%
\pgfpathlineto{\pgfqpoint{6.604432in}{3.067078in}}%
\pgfpathlineto{\pgfqpoint{6.611390in}{3.067412in}}%
\pgfpathlineto{\pgfqpoint{6.618615in}{3.070020in}}%
\pgfpathlineto{\pgfqpoint{6.626376in}{3.075269in}}%
\pgfpathlineto{\pgfqpoint{6.635207in}{3.084016in}}%
\pgfpathlineto{\pgfqpoint{6.646179in}{3.098225in}}%
\pgfpathlineto{\pgfqpoint{6.662771in}{3.123915in}}%
\pgfpathlineto{\pgfqpoint{6.663306in}{3.124778in}}%
\pgfpathlineto{\pgfqpoint{6.663306in}{3.124778in}}%
\pgfusepath{stroke}%
\end{pgfscope}%
\begin{pgfscope}%
\pgfpathrectangle{\pgfqpoint{0.467797in}{2.292089in}}{\pgfqpoint{6.490533in}{1.666241in}}%
\pgfusepath{clip}%
\pgfsetrectcap%
\pgfsetroundjoin%
\pgfsetlinewidth{1.505625pt}%
\definecolor{currentstroke}{rgb}{0.498039,0.498039,0.498039}%
\pgfsetstrokecolor{currentstroke}%
\pgfsetdash{}{0pt}%
\pgfpathmoveto{\pgfqpoint{0.762821in}{3.125209in}}%
\pgfpathlineto{\pgfqpoint{0.782624in}{3.155062in}}%
\pgfpathlineto{\pgfqpoint{0.792793in}{3.166498in}}%
\pgfpathlineto{\pgfqpoint{0.801089in}{3.172669in}}%
\pgfpathlineto{\pgfqpoint{0.808314in}{3.175353in}}%
\pgfpathlineto{\pgfqpoint{0.815004in}{3.175468in}}%
\pgfpathlineto{\pgfqpoint{0.821695in}{3.173297in}}%
\pgfpathlineto{\pgfqpoint{0.828920in}{3.168499in}}%
\pgfpathlineto{\pgfqpoint{0.837216in}{3.160181in}}%
\pgfpathlineto{\pgfqpoint{0.847653in}{3.146336in}}%
\pgfpathlineto{\pgfqpoint{0.865850in}{3.117684in}}%
\pgfpathlineto{\pgfqpoint{0.881906in}{3.094152in}}%
\pgfpathlineto{\pgfqpoint{0.891808in}{3.083309in}}%
\pgfpathlineto{\pgfqpoint{0.899836in}{3.077533in}}%
\pgfpathlineto{\pgfqpoint{0.907062in}{3.074994in}}%
\pgfpathlineto{\pgfqpoint{0.913752in}{3.075017in}}%
\pgfpathlineto{\pgfqpoint{0.920442in}{3.077324in}}%
\pgfpathlineto{\pgfqpoint{0.927667in}{3.082257in}}%
\pgfpathlineto{\pgfqpoint{0.936231in}{3.091025in}}%
\pgfpathlineto{\pgfqpoint{0.946935in}{3.105464in}}%
\pgfpathlineto{\pgfqpoint{0.968611in}{3.139679in}}%
\pgfpathlineto{\pgfqpoint{0.982259in}{3.158754in}}%
\pgfpathlineto{\pgfqpoint{0.991893in}{3.168610in}}%
\pgfpathlineto{\pgfqpoint{0.999654in}{3.173615in}}%
\pgfpathlineto{\pgfqpoint{1.006612in}{3.175591in}}%
\pgfpathlineto{\pgfqpoint{1.013302in}{3.175154in}}%
\pgfpathlineto{\pgfqpoint{1.019992in}{3.172444in}}%
\pgfpathlineto{\pgfqpoint{1.027485in}{3.166867in}}%
\pgfpathlineto{\pgfqpoint{1.036316in}{3.157278in}}%
\pgfpathlineto{\pgfqpoint{1.047823in}{3.141120in}}%
\pgfpathlineto{\pgfqpoint{1.083683in}{3.088123in}}%
\pgfpathlineto{\pgfqpoint{1.092514in}{3.080050in}}%
\pgfpathlineto{\pgfqpoint{1.100007in}{3.075983in}}%
\pgfpathlineto{\pgfqpoint{1.106697in}{3.074720in}}%
\pgfpathlineto{\pgfqpoint{1.113387in}{3.075754in}}%
\pgfpathlineto{\pgfqpoint{1.120345in}{3.079215in}}%
\pgfpathlineto{\pgfqpoint{1.128106in}{3.085733in}}%
\pgfpathlineto{\pgfqpoint{1.137472in}{3.096766in}}%
\pgfpathlineto{\pgfqpoint{1.150317in}{3.115769in}}%
\pgfpathlineto{\pgfqpoint{1.177881in}{3.157444in}}%
\pgfpathlineto{\pgfqpoint{1.187782in}{3.167934in}}%
\pgfpathlineto{\pgfqpoint{1.195810in}{3.173362in}}%
\pgfpathlineto{\pgfqpoint{1.202768in}{3.175526in}}%
\pgfpathlineto{\pgfqpoint{1.209458in}{3.175273in}}%
\pgfpathlineto{\pgfqpoint{1.216149in}{3.172742in}}%
\pgfpathlineto{\pgfqpoint{1.223642in}{3.167349in}}%
\pgfpathlineto{\pgfqpoint{1.232205in}{3.158267in}}%
\pgfpathlineto{\pgfqpoint{1.243177in}{3.143153in}}%
\pgfpathlineto{\pgfqpoint{1.282248in}{3.086140in}}%
\pgfpathlineto{\pgfqpoint{1.290811in}{3.078952in}}%
\pgfpathlineto{\pgfqpoint{1.298304in}{3.075472in}}%
\pgfpathlineto{\pgfqpoint{1.304994in}{3.074759in}}%
\pgfpathlineto{\pgfqpoint{1.311685in}{3.076341in}}%
\pgfpathlineto{\pgfqpoint{1.318642in}{3.080344in}}%
\pgfpathlineto{\pgfqpoint{1.326671in}{3.087686in}}%
\pgfpathlineto{\pgfqpoint{1.336572in}{3.100057in}}%
\pgfpathlineto{\pgfqpoint{1.351023in}{3.122190in}}%
\pgfpathlineto{\pgfqpoint{1.372699in}{3.155062in}}%
\pgfpathlineto{\pgfqpoint{1.382868in}{3.166498in}}%
\pgfpathlineto{\pgfqpoint{1.391164in}{3.172669in}}%
\pgfpathlineto{\pgfqpoint{1.398389in}{3.175353in}}%
\pgfpathlineto{\pgfqpoint{1.405080in}{3.175468in}}%
\pgfpathlineto{\pgfqpoint{1.411770in}{3.173297in}}%
\pgfpathlineto{\pgfqpoint{1.418995in}{3.168499in}}%
\pgfpathlineto{\pgfqpoint{1.427291in}{3.160181in}}%
\pgfpathlineto{\pgfqpoint{1.437728in}{3.146336in}}%
\pgfpathlineto{\pgfqpoint{1.455925in}{3.117684in}}%
\pgfpathlineto{\pgfqpoint{1.471982in}{3.094152in}}%
\pgfpathlineto{\pgfqpoint{1.481883in}{3.083309in}}%
\pgfpathlineto{\pgfqpoint{1.489911in}{3.077533in}}%
\pgfpathlineto{\pgfqpoint{1.497137in}{3.074994in}}%
\pgfpathlineto{\pgfqpoint{1.503827in}{3.075017in}}%
\pgfpathlineto{\pgfqpoint{1.510517in}{3.077324in}}%
\pgfpathlineto{\pgfqpoint{1.517743in}{3.082257in}}%
\pgfpathlineto{\pgfqpoint{1.526306in}{3.091025in}}%
\pgfpathlineto{\pgfqpoint{1.537010in}{3.105464in}}%
\pgfpathlineto{\pgfqpoint{1.558687in}{3.139679in}}%
\pgfpathlineto{\pgfqpoint{1.572335in}{3.158754in}}%
\pgfpathlineto{\pgfqpoint{1.581968in}{3.168610in}}%
\pgfpathlineto{\pgfqpoint{1.589729in}{3.173615in}}%
\pgfpathlineto{\pgfqpoint{1.596687in}{3.175591in}}%
\pgfpathlineto{\pgfqpoint{1.603377in}{3.175154in}}%
\pgfpathlineto{\pgfqpoint{1.610067in}{3.172444in}}%
\pgfpathlineto{\pgfqpoint{1.617560in}{3.166867in}}%
\pgfpathlineto{\pgfqpoint{1.626391in}{3.157278in}}%
\pgfpathlineto{\pgfqpoint{1.637898in}{3.141120in}}%
\pgfpathlineto{\pgfqpoint{1.673758in}{3.088123in}}%
\pgfpathlineto{\pgfqpoint{1.682589in}{3.080050in}}%
\pgfpathlineto{\pgfqpoint{1.690082in}{3.075983in}}%
\pgfpathlineto{\pgfqpoint{1.696772in}{3.074720in}}%
\pgfpathlineto{\pgfqpoint{1.703462in}{3.075754in}}%
\pgfpathlineto{\pgfqpoint{1.710420in}{3.079215in}}%
\pgfpathlineto{\pgfqpoint{1.718181in}{3.085733in}}%
\pgfpathlineto{\pgfqpoint{1.727547in}{3.096766in}}%
\pgfpathlineto{\pgfqpoint{1.740392in}{3.115769in}}%
\pgfpathlineto{\pgfqpoint{1.767956in}{3.157444in}}%
\pgfpathlineto{\pgfqpoint{1.777857in}{3.167934in}}%
\pgfpathlineto{\pgfqpoint{1.785886in}{3.173362in}}%
\pgfpathlineto{\pgfqpoint{1.792843in}{3.175526in}}%
\pgfpathlineto{\pgfqpoint{1.799534in}{3.175273in}}%
\pgfpathlineto{\pgfqpoint{1.806224in}{3.172742in}}%
\pgfpathlineto{\pgfqpoint{1.813717in}{3.167349in}}%
\pgfpathlineto{\pgfqpoint{1.822280in}{3.158267in}}%
\pgfpathlineto{\pgfqpoint{1.833252in}{3.143153in}}%
\pgfpathlineto{\pgfqpoint{1.872323in}{3.086140in}}%
\pgfpathlineto{\pgfqpoint{1.880886in}{3.078952in}}%
\pgfpathlineto{\pgfqpoint{1.888379in}{3.075472in}}%
\pgfpathlineto{\pgfqpoint{1.895070in}{3.074759in}}%
\pgfpathlineto{\pgfqpoint{1.901760in}{3.076341in}}%
\pgfpathlineto{\pgfqpoint{1.908718in}{3.080344in}}%
\pgfpathlineto{\pgfqpoint{1.916746in}{3.087686in}}%
\pgfpathlineto{\pgfqpoint{1.926647in}{3.100057in}}%
\pgfpathlineto{\pgfqpoint{1.941098in}{3.122190in}}%
\pgfpathlineto{\pgfqpoint{1.962774in}{3.155062in}}%
\pgfpathlineto{\pgfqpoint{1.972943in}{3.166498in}}%
\pgfpathlineto{\pgfqpoint{1.981239in}{3.172669in}}%
\pgfpathlineto{\pgfqpoint{1.988465in}{3.175353in}}%
\pgfpathlineto{\pgfqpoint{1.995155in}{3.175468in}}%
\pgfpathlineto{\pgfqpoint{2.001845in}{3.173297in}}%
\pgfpathlineto{\pgfqpoint{2.009071in}{3.168499in}}%
\pgfpathlineto{\pgfqpoint{2.017366in}{3.160181in}}%
\pgfpathlineto{\pgfqpoint{2.027803in}{3.146336in}}%
\pgfpathlineto{\pgfqpoint{2.046000in}{3.117684in}}%
\pgfpathlineto{\pgfqpoint{2.062057in}{3.094152in}}%
\pgfpathlineto{\pgfqpoint{2.071958in}{3.083309in}}%
\pgfpathlineto{\pgfqpoint{2.079987in}{3.077533in}}%
\pgfpathlineto{\pgfqpoint{2.087212in}{3.074994in}}%
\pgfpathlineto{\pgfqpoint{2.093902in}{3.075017in}}%
\pgfpathlineto{\pgfqpoint{2.100592in}{3.077324in}}%
\pgfpathlineto{\pgfqpoint{2.107818in}{3.082257in}}%
\pgfpathlineto{\pgfqpoint{2.116381in}{3.091025in}}%
\pgfpathlineto{\pgfqpoint{2.127086in}{3.105464in}}%
\pgfpathlineto{\pgfqpoint{2.148762in}{3.139679in}}%
\pgfpathlineto{\pgfqpoint{2.162410in}{3.158754in}}%
\pgfpathlineto{\pgfqpoint{2.172044in}{3.168610in}}%
\pgfpathlineto{\pgfqpoint{2.179804in}{3.173615in}}%
\pgfpathlineto{\pgfqpoint{2.186762in}{3.175591in}}%
\pgfpathlineto{\pgfqpoint{2.193452in}{3.175154in}}%
\pgfpathlineto{\pgfqpoint{2.200143in}{3.172444in}}%
\pgfpathlineto{\pgfqpoint{2.207636in}{3.166867in}}%
\pgfpathlineto{\pgfqpoint{2.216467in}{3.157278in}}%
\pgfpathlineto{\pgfqpoint{2.227974in}{3.141120in}}%
\pgfpathlineto{\pgfqpoint{2.263833in}{3.088123in}}%
\pgfpathlineto{\pgfqpoint{2.272664in}{3.080050in}}%
\pgfpathlineto{\pgfqpoint{2.280157in}{3.075983in}}%
\pgfpathlineto{\pgfqpoint{2.286847in}{3.074720in}}%
\pgfpathlineto{\pgfqpoint{2.293538in}{3.075754in}}%
\pgfpathlineto{\pgfqpoint{2.300495in}{3.079215in}}%
\pgfpathlineto{\pgfqpoint{2.308256in}{3.085733in}}%
\pgfpathlineto{\pgfqpoint{2.317622in}{3.096766in}}%
\pgfpathlineto{\pgfqpoint{2.330468in}{3.115769in}}%
\pgfpathlineto{\pgfqpoint{2.358031in}{3.157444in}}%
\pgfpathlineto{\pgfqpoint{2.367933in}{3.167934in}}%
\pgfpathlineto{\pgfqpoint{2.375961in}{3.173362in}}%
\pgfpathlineto{\pgfqpoint{2.382919in}{3.175526in}}%
\pgfpathlineto{\pgfqpoint{2.389609in}{3.175273in}}%
\pgfpathlineto{\pgfqpoint{2.396299in}{3.172742in}}%
\pgfpathlineto{\pgfqpoint{2.403792in}{3.167349in}}%
\pgfpathlineto{\pgfqpoint{2.412356in}{3.158267in}}%
\pgfpathlineto{\pgfqpoint{2.423327in}{3.143153in}}%
\pgfpathlineto{\pgfqpoint{2.462398in}{3.086140in}}%
\pgfpathlineto{\pgfqpoint{2.470962in}{3.078952in}}%
\pgfpathlineto{\pgfqpoint{2.478455in}{3.075472in}}%
\pgfpathlineto{\pgfqpoint{2.485145in}{3.074759in}}%
\pgfpathlineto{\pgfqpoint{2.491835in}{3.076341in}}%
\pgfpathlineto{\pgfqpoint{2.498793in}{3.080344in}}%
\pgfpathlineto{\pgfqpoint{2.506821in}{3.087686in}}%
\pgfpathlineto{\pgfqpoint{2.516723in}{3.100057in}}%
\pgfpathlineto{\pgfqpoint{2.531173in}{3.122190in}}%
\pgfpathlineto{\pgfqpoint{2.552850in}{3.155062in}}%
\pgfpathlineto{\pgfqpoint{2.563019in}{3.166498in}}%
\pgfpathlineto{\pgfqpoint{2.571315in}{3.172669in}}%
\pgfpathlineto{\pgfqpoint{2.578540in}{3.175353in}}%
\pgfpathlineto{\pgfqpoint{2.585230in}{3.175468in}}%
\pgfpathlineto{\pgfqpoint{2.591920in}{3.173297in}}%
\pgfpathlineto{\pgfqpoint{2.599146in}{3.168499in}}%
\pgfpathlineto{\pgfqpoint{2.607442in}{3.160181in}}%
\pgfpathlineto{\pgfqpoint{2.617878in}{3.146336in}}%
\pgfpathlineto{\pgfqpoint{2.636076in}{3.117684in}}%
\pgfpathlineto{\pgfqpoint{2.652132in}{3.094152in}}%
\pgfpathlineto{\pgfqpoint{2.662034in}{3.083309in}}%
\pgfpathlineto{\pgfqpoint{2.670062in}{3.077533in}}%
\pgfpathlineto{\pgfqpoint{2.677287in}{3.074994in}}%
\pgfpathlineto{\pgfqpoint{2.683977in}{3.075017in}}%
\pgfpathlineto{\pgfqpoint{2.690668in}{3.077324in}}%
\pgfpathlineto{\pgfqpoint{2.697893in}{3.082257in}}%
\pgfpathlineto{\pgfqpoint{2.706457in}{3.091025in}}%
\pgfpathlineto{\pgfqpoint{2.717161in}{3.105464in}}%
\pgfpathlineto{\pgfqpoint{2.738837in}{3.139679in}}%
\pgfpathlineto{\pgfqpoint{2.752485in}{3.158754in}}%
\pgfpathlineto{\pgfqpoint{2.762119in}{3.168610in}}%
\pgfpathlineto{\pgfqpoint{2.769880in}{3.173615in}}%
\pgfpathlineto{\pgfqpoint{2.776837in}{3.175591in}}%
\pgfpathlineto{\pgfqpoint{2.783528in}{3.175154in}}%
\pgfpathlineto{\pgfqpoint{2.790218in}{3.172444in}}%
\pgfpathlineto{\pgfqpoint{2.797711in}{3.166867in}}%
\pgfpathlineto{\pgfqpoint{2.806542in}{3.157278in}}%
\pgfpathlineto{\pgfqpoint{2.818049in}{3.141120in}}%
\pgfpathlineto{\pgfqpoint{2.853908in}{3.088123in}}%
\pgfpathlineto{\pgfqpoint{2.862739in}{3.080050in}}%
\pgfpathlineto{\pgfqpoint{2.870233in}{3.075983in}}%
\pgfpathlineto{\pgfqpoint{2.876923in}{3.074720in}}%
\pgfpathlineto{\pgfqpoint{2.883613in}{3.075754in}}%
\pgfpathlineto{\pgfqpoint{2.890571in}{3.079215in}}%
\pgfpathlineto{\pgfqpoint{2.898331in}{3.085733in}}%
\pgfpathlineto{\pgfqpoint{2.907698in}{3.096766in}}%
\pgfpathlineto{\pgfqpoint{2.920543in}{3.115769in}}%
\pgfpathlineto{\pgfqpoint{2.948106in}{3.157444in}}%
\pgfpathlineto{\pgfqpoint{2.958008in}{3.167934in}}%
\pgfpathlineto{\pgfqpoint{2.966036in}{3.173362in}}%
\pgfpathlineto{\pgfqpoint{2.972994in}{3.175526in}}%
\pgfpathlineto{\pgfqpoint{2.979684in}{3.175273in}}%
\pgfpathlineto{\pgfqpoint{2.986374in}{3.172742in}}%
\pgfpathlineto{\pgfqpoint{2.993867in}{3.167349in}}%
\pgfpathlineto{\pgfqpoint{3.002431in}{3.158267in}}%
\pgfpathlineto{\pgfqpoint{3.013403in}{3.143153in}}%
\pgfpathlineto{\pgfqpoint{3.052473in}{3.086140in}}%
\pgfpathlineto{\pgfqpoint{3.061037in}{3.078952in}}%
\pgfpathlineto{\pgfqpoint{3.068530in}{3.075472in}}%
\pgfpathlineto{\pgfqpoint{3.075220in}{3.074759in}}%
\pgfpathlineto{\pgfqpoint{3.081910in}{3.076341in}}%
\pgfpathlineto{\pgfqpoint{3.088868in}{3.080344in}}%
\pgfpathlineto{\pgfqpoint{3.096896in}{3.087686in}}%
\pgfpathlineto{\pgfqpoint{3.106798in}{3.100057in}}%
\pgfpathlineto{\pgfqpoint{3.121249in}{3.122190in}}%
\pgfpathlineto{\pgfqpoint{3.142925in}{3.155062in}}%
\pgfpathlineto{\pgfqpoint{3.153094in}{3.166498in}}%
\pgfpathlineto{\pgfqpoint{3.161390in}{3.172669in}}%
\pgfpathlineto{\pgfqpoint{3.168615in}{3.175353in}}%
\pgfpathlineto{\pgfqpoint{3.175305in}{3.175468in}}%
\pgfpathlineto{\pgfqpoint{3.181996in}{3.173297in}}%
\pgfpathlineto{\pgfqpoint{3.189221in}{3.168499in}}%
\pgfpathlineto{\pgfqpoint{3.197517in}{3.160181in}}%
\pgfpathlineto{\pgfqpoint{3.207954in}{3.146336in}}%
\pgfpathlineto{\pgfqpoint{3.226151in}{3.117684in}}%
\pgfpathlineto{\pgfqpoint{3.242207in}{3.094152in}}%
\pgfpathlineto{\pgfqpoint{3.252109in}{3.083309in}}%
\pgfpathlineto{\pgfqpoint{3.260137in}{3.077533in}}%
\pgfpathlineto{\pgfqpoint{3.267363in}{3.074994in}}%
\pgfpathlineto{\pgfqpoint{3.274053in}{3.075017in}}%
\pgfpathlineto{\pgfqpoint{3.280743in}{3.077324in}}%
\pgfpathlineto{\pgfqpoint{3.287968in}{3.082257in}}%
\pgfpathlineto{\pgfqpoint{3.296532in}{3.091025in}}%
\pgfpathlineto{\pgfqpoint{3.307236in}{3.105464in}}%
\pgfpathlineto{\pgfqpoint{3.328912in}{3.139679in}}%
\pgfpathlineto{\pgfqpoint{3.342560in}{3.158754in}}%
\pgfpathlineto{\pgfqpoint{3.352194in}{3.168610in}}%
\pgfpathlineto{\pgfqpoint{3.359955in}{3.173615in}}%
\pgfpathlineto{\pgfqpoint{3.366913in}{3.175591in}}%
\pgfpathlineto{\pgfqpoint{3.373603in}{3.175154in}}%
\pgfpathlineto{\pgfqpoint{3.380293in}{3.172444in}}%
\pgfpathlineto{\pgfqpoint{3.387786in}{3.166867in}}%
\pgfpathlineto{\pgfqpoint{3.396617in}{3.157278in}}%
\pgfpathlineto{\pgfqpoint{3.408124in}{3.141120in}}%
\pgfpathlineto{\pgfqpoint{3.443984in}{3.088123in}}%
\pgfpathlineto{\pgfqpoint{3.452815in}{3.080050in}}%
\pgfpathlineto{\pgfqpoint{3.460308in}{3.075983in}}%
\pgfpathlineto{\pgfqpoint{3.466998in}{3.074720in}}%
\pgfpathlineto{\pgfqpoint{3.473688in}{3.075754in}}%
\pgfpathlineto{\pgfqpoint{3.480646in}{3.079215in}}%
\pgfpathlineto{\pgfqpoint{3.488407in}{3.085733in}}%
\pgfpathlineto{\pgfqpoint{3.497773in}{3.096766in}}%
\pgfpathlineto{\pgfqpoint{3.510618in}{3.115769in}}%
\pgfpathlineto{\pgfqpoint{3.538182in}{3.157444in}}%
\pgfpathlineto{\pgfqpoint{3.548083in}{3.167934in}}%
\pgfpathlineto{\pgfqpoint{3.556111in}{3.173362in}}%
\pgfpathlineto{\pgfqpoint{3.563069in}{3.175526in}}%
\pgfpathlineto{\pgfqpoint{3.569759in}{3.175273in}}%
\pgfpathlineto{\pgfqpoint{3.576450in}{3.172742in}}%
\pgfpathlineto{\pgfqpoint{3.583943in}{3.167349in}}%
\pgfpathlineto{\pgfqpoint{3.592506in}{3.158267in}}%
\pgfpathlineto{\pgfqpoint{3.603478in}{3.143153in}}%
\pgfpathlineto{\pgfqpoint{3.642549in}{3.086140in}}%
\pgfpathlineto{\pgfqpoint{3.651112in}{3.078952in}}%
\pgfpathlineto{\pgfqpoint{3.658605in}{3.075472in}}%
\pgfpathlineto{\pgfqpoint{3.665295in}{3.074759in}}%
\pgfpathlineto{\pgfqpoint{3.671986in}{3.076341in}}%
\pgfpathlineto{\pgfqpoint{3.678943in}{3.080344in}}%
\pgfpathlineto{\pgfqpoint{3.686972in}{3.087686in}}%
\pgfpathlineto{\pgfqpoint{3.696873in}{3.100057in}}%
\pgfpathlineto{\pgfqpoint{3.711324in}{3.122190in}}%
\pgfpathlineto{\pgfqpoint{3.733000in}{3.155062in}}%
\pgfpathlineto{\pgfqpoint{3.743169in}{3.166498in}}%
\pgfpathlineto{\pgfqpoint{3.751465in}{3.172669in}}%
\pgfpathlineto{\pgfqpoint{3.758690in}{3.175353in}}%
\pgfpathlineto{\pgfqpoint{3.765381in}{3.175468in}}%
\pgfpathlineto{\pgfqpoint{3.772071in}{3.173297in}}%
\pgfpathlineto{\pgfqpoint{3.779296in}{3.168499in}}%
\pgfpathlineto{\pgfqpoint{3.787592in}{3.160181in}}%
\pgfpathlineto{\pgfqpoint{3.798029in}{3.146336in}}%
\pgfpathlineto{\pgfqpoint{3.816226in}{3.117684in}}%
\pgfpathlineto{\pgfqpoint{3.832283in}{3.094152in}}%
\pgfpathlineto{\pgfqpoint{3.842184in}{3.083309in}}%
\pgfpathlineto{\pgfqpoint{3.850212in}{3.077533in}}%
\pgfpathlineto{\pgfqpoint{3.857438in}{3.074994in}}%
\pgfpathlineto{\pgfqpoint{3.864128in}{3.075017in}}%
\pgfpathlineto{\pgfqpoint{3.870818in}{3.077324in}}%
\pgfpathlineto{\pgfqpoint{3.878044in}{3.082257in}}%
\pgfpathlineto{\pgfqpoint{3.886607in}{3.091025in}}%
\pgfpathlineto{\pgfqpoint{3.897311in}{3.105464in}}%
\pgfpathlineto{\pgfqpoint{3.918988in}{3.139679in}}%
\pgfpathlineto{\pgfqpoint{3.932636in}{3.158754in}}%
\pgfpathlineto{\pgfqpoint{3.942269in}{3.168610in}}%
\pgfpathlineto{\pgfqpoint{3.950030in}{3.173615in}}%
\pgfpathlineto{\pgfqpoint{3.956988in}{3.175591in}}%
\pgfpathlineto{\pgfqpoint{3.963678in}{3.175154in}}%
\pgfpathlineto{\pgfqpoint{3.970368in}{3.172444in}}%
\pgfpathlineto{\pgfqpoint{3.977861in}{3.166867in}}%
\pgfpathlineto{\pgfqpoint{3.986692in}{3.157278in}}%
\pgfpathlineto{\pgfqpoint{3.998199in}{3.141120in}}%
\pgfpathlineto{\pgfqpoint{4.034059in}{3.088123in}}%
\pgfpathlineto{\pgfqpoint{4.042890in}{3.080050in}}%
\pgfpathlineto{\pgfqpoint{4.050383in}{3.075983in}}%
\pgfpathlineto{\pgfqpoint{4.057073in}{3.074720in}}%
\pgfpathlineto{\pgfqpoint{4.063763in}{3.075754in}}%
\pgfpathlineto{\pgfqpoint{4.070721in}{3.079215in}}%
\pgfpathlineto{\pgfqpoint{4.078482in}{3.085733in}}%
\pgfpathlineto{\pgfqpoint{4.087848in}{3.096766in}}%
\pgfpathlineto{\pgfqpoint{4.100693in}{3.115769in}}%
\pgfpathlineto{\pgfqpoint{4.128257in}{3.157444in}}%
\pgfpathlineto{\pgfqpoint{4.138158in}{3.167934in}}%
\pgfpathlineto{\pgfqpoint{4.146187in}{3.173362in}}%
\pgfpathlineto{\pgfqpoint{4.153144in}{3.175526in}}%
\pgfpathlineto{\pgfqpoint{4.159835in}{3.175273in}}%
\pgfpathlineto{\pgfqpoint{4.166525in}{3.172742in}}%
\pgfpathlineto{\pgfqpoint{4.174018in}{3.167349in}}%
\pgfpathlineto{\pgfqpoint{4.182581in}{3.158267in}}%
\pgfpathlineto{\pgfqpoint{4.193553in}{3.143153in}}%
\pgfpathlineto{\pgfqpoint{4.232624in}{3.086140in}}%
\pgfpathlineto{\pgfqpoint{4.241187in}{3.078952in}}%
\pgfpathlineto{\pgfqpoint{4.248680in}{3.075472in}}%
\pgfpathlineto{\pgfqpoint{4.255371in}{3.074759in}}%
\pgfpathlineto{\pgfqpoint{4.262061in}{3.076341in}}%
\pgfpathlineto{\pgfqpoint{4.269019in}{3.080344in}}%
\pgfpathlineto{\pgfqpoint{4.277047in}{3.087686in}}%
\pgfpathlineto{\pgfqpoint{4.286948in}{3.100057in}}%
\pgfpathlineto{\pgfqpoint{4.301399in}{3.122190in}}%
\pgfpathlineto{\pgfqpoint{4.323075in}{3.155062in}}%
\pgfpathlineto{\pgfqpoint{4.333244in}{3.166498in}}%
\pgfpathlineto{\pgfqpoint{4.341540in}{3.172669in}}%
\pgfpathlineto{\pgfqpoint{4.348766in}{3.175353in}}%
\pgfpathlineto{\pgfqpoint{4.355456in}{3.175468in}}%
\pgfpathlineto{\pgfqpoint{4.362146in}{3.173297in}}%
\pgfpathlineto{\pgfqpoint{4.369372in}{3.168499in}}%
\pgfpathlineto{\pgfqpoint{4.377667in}{3.160181in}}%
\pgfpathlineto{\pgfqpoint{4.388104in}{3.146336in}}%
\pgfpathlineto{\pgfqpoint{4.406301in}{3.117684in}}%
\pgfpathlineto{\pgfqpoint{4.422358in}{3.094152in}}%
\pgfpathlineto{\pgfqpoint{4.432259in}{3.083309in}}%
\pgfpathlineto{\pgfqpoint{4.440288in}{3.077533in}}%
\pgfpathlineto{\pgfqpoint{4.447513in}{3.074994in}}%
\pgfpathlineto{\pgfqpoint{4.454203in}{3.075017in}}%
\pgfpathlineto{\pgfqpoint{4.460893in}{3.077324in}}%
\pgfpathlineto{\pgfqpoint{4.468119in}{3.082257in}}%
\pgfpathlineto{\pgfqpoint{4.476682in}{3.091025in}}%
\pgfpathlineto{\pgfqpoint{4.487387in}{3.105464in}}%
\pgfpathlineto{\pgfqpoint{4.509063in}{3.139679in}}%
\pgfpathlineto{\pgfqpoint{4.522711in}{3.158754in}}%
\pgfpathlineto{\pgfqpoint{4.532345in}{3.168610in}}%
\pgfpathlineto{\pgfqpoint{4.540105in}{3.173615in}}%
\pgfpathlineto{\pgfqpoint{4.547063in}{3.175591in}}%
\pgfpathlineto{\pgfqpoint{4.553753in}{3.175154in}}%
\pgfpathlineto{\pgfqpoint{4.560443in}{3.172444in}}%
\pgfpathlineto{\pgfqpoint{4.567937in}{3.166867in}}%
\pgfpathlineto{\pgfqpoint{4.576768in}{3.157278in}}%
\pgfpathlineto{\pgfqpoint{4.588275in}{3.141120in}}%
\pgfpathlineto{\pgfqpoint{4.624134in}{3.088123in}}%
\pgfpathlineto{\pgfqpoint{4.632965in}{3.080050in}}%
\pgfpathlineto{\pgfqpoint{4.640458in}{3.075983in}}%
\pgfpathlineto{\pgfqpoint{4.647148in}{3.074720in}}%
\pgfpathlineto{\pgfqpoint{4.653839in}{3.075754in}}%
\pgfpathlineto{\pgfqpoint{4.660796in}{3.079215in}}%
\pgfpathlineto{\pgfqpoint{4.668557in}{3.085733in}}%
\pgfpathlineto{\pgfqpoint{4.677923in}{3.096766in}}%
\pgfpathlineto{\pgfqpoint{4.690769in}{3.115769in}}%
\pgfpathlineto{\pgfqpoint{4.718332in}{3.157444in}}%
\pgfpathlineto{\pgfqpoint{4.728234in}{3.167934in}}%
\pgfpathlineto{\pgfqpoint{4.736262in}{3.173362in}}%
\pgfpathlineto{\pgfqpoint{4.743220in}{3.175526in}}%
\pgfpathlineto{\pgfqpoint{4.749910in}{3.175273in}}%
\pgfpathlineto{\pgfqpoint{4.756600in}{3.172742in}}%
\pgfpathlineto{\pgfqpoint{4.764093in}{3.167349in}}%
\pgfpathlineto{\pgfqpoint{4.772657in}{3.158267in}}%
\pgfpathlineto{\pgfqpoint{4.783628in}{3.143153in}}%
\pgfpathlineto{\pgfqpoint{4.822699in}{3.086140in}}%
\pgfpathlineto{\pgfqpoint{4.831263in}{3.078952in}}%
\pgfpathlineto{\pgfqpoint{4.838756in}{3.075472in}}%
\pgfpathlineto{\pgfqpoint{4.845446in}{3.074759in}}%
\pgfpathlineto{\pgfqpoint{4.852136in}{3.076341in}}%
\pgfpathlineto{\pgfqpoint{4.859094in}{3.080344in}}%
\pgfpathlineto{\pgfqpoint{4.867122in}{3.087686in}}%
\pgfpathlineto{\pgfqpoint{4.877024in}{3.100057in}}%
\pgfpathlineto{\pgfqpoint{4.891474in}{3.122190in}}%
\pgfpathlineto{\pgfqpoint{4.913151in}{3.155062in}}%
\pgfpathlineto{\pgfqpoint{4.923320in}{3.166498in}}%
\pgfpathlineto{\pgfqpoint{4.931616in}{3.172669in}}%
\pgfpathlineto{\pgfqpoint{4.938841in}{3.175353in}}%
\pgfpathlineto{\pgfqpoint{4.945531in}{3.175468in}}%
\pgfpathlineto{\pgfqpoint{4.952221in}{3.173297in}}%
\pgfpathlineto{\pgfqpoint{4.959447in}{3.168499in}}%
\pgfpathlineto{\pgfqpoint{4.967743in}{3.160181in}}%
\pgfpathlineto{\pgfqpoint{4.978179in}{3.146336in}}%
\pgfpathlineto{\pgfqpoint{4.996377in}{3.117684in}}%
\pgfpathlineto{\pgfqpoint{5.012433in}{3.094152in}}%
\pgfpathlineto{\pgfqpoint{5.022335in}{3.083309in}}%
\pgfpathlineto{\pgfqpoint{5.030363in}{3.077533in}}%
\pgfpathlineto{\pgfqpoint{5.037588in}{3.074994in}}%
\pgfpathlineto{\pgfqpoint{5.044278in}{3.075017in}}%
\pgfpathlineto{\pgfqpoint{5.050969in}{3.077324in}}%
\pgfpathlineto{\pgfqpoint{5.058194in}{3.082257in}}%
\pgfpathlineto{\pgfqpoint{5.066757in}{3.091025in}}%
\pgfpathlineto{\pgfqpoint{5.077462in}{3.105464in}}%
\pgfpathlineto{\pgfqpoint{5.099138in}{3.139679in}}%
\pgfpathlineto{\pgfqpoint{5.112786in}{3.158754in}}%
\pgfpathlineto{\pgfqpoint{5.122420in}{3.168610in}}%
\pgfpathlineto{\pgfqpoint{5.130181in}{3.173615in}}%
\pgfpathlineto{\pgfqpoint{5.137138in}{3.175591in}}%
\pgfpathlineto{\pgfqpoint{5.143829in}{3.175154in}}%
\pgfpathlineto{\pgfqpoint{5.150519in}{3.172444in}}%
\pgfpathlineto{\pgfqpoint{5.158012in}{3.166867in}}%
\pgfpathlineto{\pgfqpoint{5.166843in}{3.157278in}}%
\pgfpathlineto{\pgfqpoint{5.178350in}{3.141120in}}%
\pgfpathlineto{\pgfqpoint{5.214209in}{3.088123in}}%
\pgfpathlineto{\pgfqpoint{5.223040in}{3.080050in}}%
\pgfpathlineto{\pgfqpoint{5.230533in}{3.075983in}}%
\pgfpathlineto{\pgfqpoint{5.237224in}{3.074720in}}%
\pgfpathlineto{\pgfqpoint{5.243914in}{3.075754in}}%
\pgfpathlineto{\pgfqpoint{5.250872in}{3.079215in}}%
\pgfpathlineto{\pgfqpoint{5.258632in}{3.085733in}}%
\pgfpathlineto{\pgfqpoint{5.267999in}{3.096766in}}%
\pgfpathlineto{\pgfqpoint{5.280844in}{3.115769in}}%
\pgfpathlineto{\pgfqpoint{5.308407in}{3.157444in}}%
\pgfpathlineto{\pgfqpoint{5.318309in}{3.167934in}}%
\pgfpathlineto{\pgfqpoint{5.326337in}{3.173362in}}%
\pgfpathlineto{\pgfqpoint{5.333295in}{3.175526in}}%
\pgfpathlineto{\pgfqpoint{5.339985in}{3.175273in}}%
\pgfpathlineto{\pgfqpoint{5.346675in}{3.172742in}}%
\pgfpathlineto{\pgfqpoint{5.354168in}{3.167349in}}%
\pgfpathlineto{\pgfqpoint{5.362732in}{3.158267in}}%
\pgfpathlineto{\pgfqpoint{5.373704in}{3.143153in}}%
\pgfpathlineto{\pgfqpoint{5.412774in}{3.086140in}}%
\pgfpathlineto{\pgfqpoint{5.421338in}{3.078952in}}%
\pgfpathlineto{\pgfqpoint{5.428831in}{3.075472in}}%
\pgfpathlineto{\pgfqpoint{5.435521in}{3.074759in}}%
\pgfpathlineto{\pgfqpoint{5.442211in}{3.076341in}}%
\pgfpathlineto{\pgfqpoint{5.449169in}{3.080344in}}%
\pgfpathlineto{\pgfqpoint{5.457197in}{3.087686in}}%
\pgfpathlineto{\pgfqpoint{5.467099in}{3.100057in}}%
\pgfpathlineto{\pgfqpoint{5.481550in}{3.122190in}}%
\pgfpathlineto{\pgfqpoint{5.503226in}{3.155062in}}%
\pgfpathlineto{\pgfqpoint{5.513395in}{3.166498in}}%
\pgfpathlineto{\pgfqpoint{5.521691in}{3.172669in}}%
\pgfpathlineto{\pgfqpoint{5.528916in}{3.175353in}}%
\pgfpathlineto{\pgfqpoint{5.535606in}{3.175468in}}%
\pgfpathlineto{\pgfqpoint{5.542297in}{3.173297in}}%
\pgfpathlineto{\pgfqpoint{5.549522in}{3.168499in}}%
\pgfpathlineto{\pgfqpoint{5.557818in}{3.160181in}}%
\pgfpathlineto{\pgfqpoint{5.568255in}{3.146336in}}%
\pgfpathlineto{\pgfqpoint{5.586452in}{3.117684in}}%
\pgfpathlineto{\pgfqpoint{5.602508in}{3.094152in}}%
\pgfpathlineto{\pgfqpoint{5.612410in}{3.083309in}}%
\pgfpathlineto{\pgfqpoint{5.620438in}{3.077533in}}%
\pgfpathlineto{\pgfqpoint{5.627663in}{3.074994in}}%
\pgfpathlineto{\pgfqpoint{5.634354in}{3.075017in}}%
\pgfpathlineto{\pgfqpoint{5.641044in}{3.077324in}}%
\pgfpathlineto{\pgfqpoint{5.648269in}{3.082257in}}%
\pgfpathlineto{\pgfqpoint{5.656833in}{3.091025in}}%
\pgfpathlineto{\pgfqpoint{5.667537in}{3.105464in}}%
\pgfpathlineto{\pgfqpoint{5.689213in}{3.139679in}}%
\pgfpathlineto{\pgfqpoint{5.702861in}{3.158754in}}%
\pgfpathlineto{\pgfqpoint{5.712495in}{3.168610in}}%
\pgfpathlineto{\pgfqpoint{5.720256in}{3.173615in}}%
\pgfpathlineto{\pgfqpoint{5.727214in}{3.175591in}}%
\pgfpathlineto{\pgfqpoint{5.733904in}{3.175154in}}%
\pgfpathlineto{\pgfqpoint{5.740594in}{3.172444in}}%
\pgfpathlineto{\pgfqpoint{5.748087in}{3.166867in}}%
\pgfpathlineto{\pgfqpoint{5.756918in}{3.157278in}}%
\pgfpathlineto{\pgfqpoint{5.768425in}{3.141120in}}%
\pgfpathlineto{\pgfqpoint{5.804285in}{3.088123in}}%
\pgfpathlineto{\pgfqpoint{5.813116in}{3.080050in}}%
\pgfpathlineto{\pgfqpoint{5.820609in}{3.075983in}}%
\pgfpathlineto{\pgfqpoint{5.827299in}{3.074720in}}%
\pgfpathlineto{\pgfqpoint{5.833989in}{3.075754in}}%
\pgfpathlineto{\pgfqpoint{5.840947in}{3.079215in}}%
\pgfpathlineto{\pgfqpoint{5.848708in}{3.085733in}}%
\pgfpathlineto{\pgfqpoint{5.858074in}{3.096766in}}%
\pgfpathlineto{\pgfqpoint{5.870919in}{3.115769in}}%
\pgfpathlineto{\pgfqpoint{5.898483in}{3.157444in}}%
\pgfpathlineto{\pgfqpoint{5.908384in}{3.167934in}}%
\pgfpathlineto{\pgfqpoint{5.916412in}{3.173362in}}%
\pgfpathlineto{\pgfqpoint{5.923370in}{3.175526in}}%
\pgfpathlineto{\pgfqpoint{5.930060in}{3.175273in}}%
\pgfpathlineto{\pgfqpoint{5.936751in}{3.172742in}}%
\pgfpathlineto{\pgfqpoint{5.944244in}{3.167349in}}%
\pgfpathlineto{\pgfqpoint{5.952807in}{3.158267in}}%
\pgfpathlineto{\pgfqpoint{5.963779in}{3.143153in}}%
\pgfpathlineto{\pgfqpoint{6.002850in}{3.086140in}}%
\pgfpathlineto{\pgfqpoint{6.011413in}{3.078952in}}%
\pgfpathlineto{\pgfqpoint{6.018906in}{3.075472in}}%
\pgfpathlineto{\pgfqpoint{6.025596in}{3.074759in}}%
\pgfpathlineto{\pgfqpoint{6.032287in}{3.076341in}}%
\pgfpathlineto{\pgfqpoint{6.039244in}{3.080344in}}%
\pgfpathlineto{\pgfqpoint{6.047273in}{3.087686in}}%
\pgfpathlineto{\pgfqpoint{6.057174in}{3.100057in}}%
\pgfpathlineto{\pgfqpoint{6.071625in}{3.122190in}}%
\pgfpathlineto{\pgfqpoint{6.093301in}{3.155062in}}%
\pgfpathlineto{\pgfqpoint{6.103470in}{3.166498in}}%
\pgfpathlineto{\pgfqpoint{6.111766in}{3.172669in}}%
\pgfpathlineto{\pgfqpoint{6.118991in}{3.175353in}}%
\pgfpathlineto{\pgfqpoint{6.125682in}{3.175468in}}%
\pgfpathlineto{\pgfqpoint{6.132372in}{3.173297in}}%
\pgfpathlineto{\pgfqpoint{6.139597in}{3.168499in}}%
\pgfpathlineto{\pgfqpoint{6.147893in}{3.160181in}}%
\pgfpathlineto{\pgfqpoint{6.158330in}{3.146336in}}%
\pgfpathlineto{\pgfqpoint{6.176527in}{3.117684in}}%
\pgfpathlineto{\pgfqpoint{6.192584in}{3.094152in}}%
\pgfpathlineto{\pgfqpoint{6.202485in}{3.083309in}}%
\pgfpathlineto{\pgfqpoint{6.210513in}{3.077533in}}%
\pgfpathlineto{\pgfqpoint{6.217739in}{3.074994in}}%
\pgfpathlineto{\pgfqpoint{6.224429in}{3.075017in}}%
\pgfpathlineto{\pgfqpoint{6.231119in}{3.077324in}}%
\pgfpathlineto{\pgfqpoint{6.238345in}{3.082257in}}%
\pgfpathlineto{\pgfqpoint{6.246908in}{3.091025in}}%
\pgfpathlineto{\pgfqpoint{6.257612in}{3.105464in}}%
\pgfpathlineto{\pgfqpoint{6.279289in}{3.139679in}}%
\pgfpathlineto{\pgfqpoint{6.292937in}{3.158754in}}%
\pgfpathlineto{\pgfqpoint{6.302570in}{3.168610in}}%
\pgfpathlineto{\pgfqpoint{6.310331in}{3.173615in}}%
\pgfpathlineto{\pgfqpoint{6.317289in}{3.175591in}}%
\pgfpathlineto{\pgfqpoint{6.323979in}{3.175154in}}%
\pgfpathlineto{\pgfqpoint{6.330669in}{3.172444in}}%
\pgfpathlineto{\pgfqpoint{6.338162in}{3.166867in}}%
\pgfpathlineto{\pgfqpoint{6.346993in}{3.157278in}}%
\pgfpathlineto{\pgfqpoint{6.358500in}{3.141120in}}%
\pgfpathlineto{\pgfqpoint{6.394360in}{3.088123in}}%
\pgfpathlineto{\pgfqpoint{6.403191in}{3.080050in}}%
\pgfpathlineto{\pgfqpoint{6.410684in}{3.075983in}}%
\pgfpathlineto{\pgfqpoint{6.417374in}{3.074720in}}%
\pgfpathlineto{\pgfqpoint{6.424064in}{3.075754in}}%
\pgfpathlineto{\pgfqpoint{6.431022in}{3.079215in}}%
\pgfpathlineto{\pgfqpoint{6.438783in}{3.085733in}}%
\pgfpathlineto{\pgfqpoint{6.448149in}{3.096766in}}%
\pgfpathlineto{\pgfqpoint{6.460994in}{3.115769in}}%
\pgfpathlineto{\pgfqpoint{6.488558in}{3.157444in}}%
\pgfpathlineto{\pgfqpoint{6.498459in}{3.167934in}}%
\pgfpathlineto{\pgfqpoint{6.506488in}{3.173362in}}%
\pgfpathlineto{\pgfqpoint{6.513445in}{3.175526in}}%
\pgfpathlineto{\pgfqpoint{6.520136in}{3.175273in}}%
\pgfpathlineto{\pgfqpoint{6.526826in}{3.172742in}}%
\pgfpathlineto{\pgfqpoint{6.534319in}{3.167349in}}%
\pgfpathlineto{\pgfqpoint{6.542882in}{3.158267in}}%
\pgfpathlineto{\pgfqpoint{6.553854in}{3.143153in}}%
\pgfpathlineto{\pgfqpoint{6.592925in}{3.086140in}}%
\pgfpathlineto{\pgfqpoint{6.601488in}{3.078952in}}%
\pgfpathlineto{\pgfqpoint{6.608981in}{3.075472in}}%
\pgfpathlineto{\pgfqpoint{6.615672in}{3.074759in}}%
\pgfpathlineto{\pgfqpoint{6.622362in}{3.076341in}}%
\pgfpathlineto{\pgfqpoint{6.629320in}{3.080344in}}%
\pgfpathlineto{\pgfqpoint{6.637348in}{3.087686in}}%
\pgfpathlineto{\pgfqpoint{6.647249in}{3.100057in}}%
\pgfpathlineto{\pgfqpoint{6.661700in}{3.122190in}}%
\pgfpathlineto{\pgfqpoint{6.663306in}{3.124778in}}%
\pgfpathlineto{\pgfqpoint{6.663306in}{3.124778in}}%
\pgfusepath{stroke}%
\end{pgfscope}%
\begin{pgfscope}%
\pgfpathrectangle{\pgfqpoint{0.467797in}{2.292089in}}{\pgfqpoint{6.490533in}{1.666241in}}%
\pgfusepath{clip}%
\pgfsetrectcap%
\pgfsetroundjoin%
\pgfsetlinewidth{1.505625pt}%
\definecolor{currentstroke}{rgb}{0.737255,0.741176,0.133333}%
\pgfsetstrokecolor{currentstroke}%
\pgfsetdash{}{0pt}%
\pgfpathmoveto{\pgfqpoint{0.762821in}{3.125209in}}%
\pgfpathlineto{\pgfqpoint{0.781018in}{3.152483in}}%
\pgfpathlineto{\pgfqpoint{0.790385in}{3.162648in}}%
\pgfpathlineto{\pgfqpoint{0.797878in}{3.167750in}}%
\pgfpathlineto{\pgfqpoint{0.804568in}{3.169683in}}%
\pgfpathlineto{\pgfqpoint{0.810723in}{3.169168in}}%
\pgfpathlineto{\pgfqpoint{0.817145in}{3.166315in}}%
\pgfpathlineto{\pgfqpoint{0.824371in}{3.160473in}}%
\pgfpathlineto{\pgfqpoint{0.832934in}{3.150484in}}%
\pgfpathlineto{\pgfqpoint{0.844977in}{3.132627in}}%
\pgfpathlineto{\pgfqpoint{0.869061in}{3.096349in}}%
\pgfpathlineto{\pgfqpoint{0.878160in}{3.086922in}}%
\pgfpathlineto{\pgfqpoint{0.885653in}{3.082218in}}%
\pgfpathlineto{\pgfqpoint{0.892075in}{3.080682in}}%
\pgfpathlineto{\pgfqpoint{0.898230in}{3.081459in}}%
\pgfpathlineto{\pgfqpoint{0.904653in}{3.084574in}}%
\pgfpathlineto{\pgfqpoint{0.911878in}{3.090680in}}%
\pgfpathlineto{\pgfqpoint{0.920710in}{3.101278in}}%
\pgfpathlineto{\pgfqpoint{0.933555in}{3.120672in}}%
\pgfpathlineto{\pgfqpoint{0.954963in}{3.152982in}}%
\pgfpathlineto{\pgfqpoint{0.964330in}{3.162988in}}%
\pgfpathlineto{\pgfqpoint{0.971823in}{3.167934in}}%
\pgfpathlineto{\pgfqpoint{0.978245in}{3.169694in}}%
\pgfpathlineto{\pgfqpoint{0.984400in}{3.169134in}}%
\pgfpathlineto{\pgfqpoint{0.990823in}{3.166236in}}%
\pgfpathlineto{\pgfqpoint{0.998048in}{3.160348in}}%
\pgfpathlineto{\pgfqpoint{1.006612in}{3.150317in}}%
\pgfpathlineto{\pgfqpoint{1.018654in}{3.132426in}}%
\pgfpathlineto{\pgfqpoint{1.042471in}{3.096523in}}%
\pgfpathlineto{\pgfqpoint{1.051570in}{3.087039in}}%
\pgfpathlineto{\pgfqpoint{1.059063in}{3.082278in}}%
\pgfpathlineto{\pgfqpoint{1.065485in}{3.080690in}}%
\pgfpathlineto{\pgfqpoint{1.071640in}{3.081416in}}%
\pgfpathlineto{\pgfqpoint{1.078063in}{3.084481in}}%
\pgfpathlineto{\pgfqpoint{1.085288in}{3.090536in}}%
\pgfpathlineto{\pgfqpoint{1.094119in}{3.101086in}}%
\pgfpathlineto{\pgfqpoint{1.106697in}{3.120016in}}%
\pgfpathlineto{\pgfqpoint{1.128641in}{3.153141in}}%
\pgfpathlineto{\pgfqpoint{1.138007in}{3.163095in}}%
\pgfpathlineto{\pgfqpoint{1.145500in}{3.167991in}}%
\pgfpathlineto{\pgfqpoint{1.151923in}{3.169705in}}%
\pgfpathlineto{\pgfqpoint{1.158078in}{3.169100in}}%
\pgfpathlineto{\pgfqpoint{1.164500in}{3.166156in}}%
\pgfpathlineto{\pgfqpoint{1.171726in}{3.160223in}}%
\pgfpathlineto{\pgfqpoint{1.180557in}{3.149790in}}%
\pgfpathlineto{\pgfqpoint{1.192867in}{3.131372in}}%
\pgfpathlineto{\pgfqpoint{1.215613in}{3.097032in}}%
\pgfpathlineto{\pgfqpoint{1.224712in}{3.087384in}}%
\pgfpathlineto{\pgfqpoint{1.232205in}{3.082459in}}%
\pgfpathlineto{\pgfqpoint{1.238628in}{3.080720in}}%
\pgfpathlineto{\pgfqpoint{1.244783in}{3.081300in}}%
\pgfpathlineto{\pgfqpoint{1.251205in}{3.084218in}}%
\pgfpathlineto{\pgfqpoint{1.258431in}{3.090126in}}%
\pgfpathlineto{\pgfqpoint{1.266994in}{3.100175in}}%
\pgfpathlineto{\pgfqpoint{1.279304in}{3.118507in}}%
\pgfpathlineto{\pgfqpoint{1.302586in}{3.153632in}}%
\pgfpathlineto{\pgfqpoint{1.311685in}{3.163202in}}%
\pgfpathlineto{\pgfqpoint{1.319178in}{3.168048in}}%
\pgfpathlineto{\pgfqpoint{1.325600in}{3.169715in}}%
\pgfpathlineto{\pgfqpoint{1.331755in}{3.169065in}}%
\pgfpathlineto{\pgfqpoint{1.338178in}{3.166076in}}%
\pgfpathlineto{\pgfqpoint{1.345403in}{3.160097in}}%
\pgfpathlineto{\pgfqpoint{1.354234in}{3.149620in}}%
\pgfpathlineto{\pgfqpoint{1.366812in}{3.130743in}}%
\pgfpathlineto{\pgfqpoint{1.389023in}{3.097209in}}%
\pgfpathlineto{\pgfqpoint{1.398389in}{3.087277in}}%
\pgfpathlineto{\pgfqpoint{1.405883in}{3.082403in}}%
\pgfpathlineto{\pgfqpoint{1.412305in}{3.080709in}}%
\pgfpathlineto{\pgfqpoint{1.418460in}{3.081334in}}%
\pgfpathlineto{\pgfqpoint{1.424883in}{3.084298in}}%
\pgfpathlineto{\pgfqpoint{1.432108in}{3.090251in}}%
\pgfpathlineto{\pgfqpoint{1.440939in}{3.100703in}}%
\pgfpathlineto{\pgfqpoint{1.453517in}{3.119563in}}%
\pgfpathlineto{\pgfqpoint{1.475996in}{3.153456in}}%
\pgfpathlineto{\pgfqpoint{1.485094in}{3.163082in}}%
\pgfpathlineto{\pgfqpoint{1.492587in}{3.167984in}}%
\pgfpathlineto{\pgfqpoint{1.499010in}{3.169704in}}%
\pgfpathlineto{\pgfqpoint{1.505165in}{3.169104in}}%
\pgfpathlineto{\pgfqpoint{1.511588in}{3.166166in}}%
\pgfpathlineto{\pgfqpoint{1.518813in}{3.160239in}}%
\pgfpathlineto{\pgfqpoint{1.527644in}{3.149811in}}%
\pgfpathlineto{\pgfqpoint{1.539954in}{3.131397in}}%
\pgfpathlineto{\pgfqpoint{1.562701in}{3.097051in}}%
\pgfpathlineto{\pgfqpoint{1.571799in}{3.087397in}}%
\pgfpathlineto{\pgfqpoint{1.579292in}{3.082467in}}%
\pgfpathlineto{\pgfqpoint{1.585715in}{3.080721in}}%
\pgfpathlineto{\pgfqpoint{1.591870in}{3.081295in}}%
\pgfpathlineto{\pgfqpoint{1.598293in}{3.084208in}}%
\pgfpathlineto{\pgfqpoint{1.605518in}{3.090110in}}%
\pgfpathlineto{\pgfqpoint{1.614081in}{3.100154in}}%
\pgfpathlineto{\pgfqpoint{1.626391in}{3.118481in}}%
\pgfpathlineto{\pgfqpoint{1.649673in}{3.153613in}}%
\pgfpathlineto{\pgfqpoint{1.658772in}{3.163189in}}%
\pgfpathlineto{\pgfqpoint{1.666265in}{3.168041in}}%
\pgfpathlineto{\pgfqpoint{1.672688in}{3.169714in}}%
\pgfpathlineto{\pgfqpoint{1.678842in}{3.169069in}}%
\pgfpathlineto{\pgfqpoint{1.685265in}{3.166086in}}%
\pgfpathlineto{\pgfqpoint{1.692490in}{3.160113in}}%
\pgfpathlineto{\pgfqpoint{1.701322in}{3.149642in}}%
\pgfpathlineto{\pgfqpoint{1.713899in}{3.130768in}}%
\pgfpathlineto{\pgfqpoint{1.736111in}{3.097229in}}%
\pgfpathlineto{\pgfqpoint{1.745477in}{3.087290in}}%
\pgfpathlineto{\pgfqpoint{1.752970in}{3.082410in}}%
\pgfpathlineto{\pgfqpoint{1.759392in}{3.080711in}}%
\pgfpathlineto{\pgfqpoint{1.765547in}{3.081330in}}%
\pgfpathlineto{\pgfqpoint{1.771970in}{3.084288in}}%
\pgfpathlineto{\pgfqpoint{1.779195in}{3.090235in}}%
\pgfpathlineto{\pgfqpoint{1.788026in}{3.100682in}}%
\pgfpathlineto{\pgfqpoint{1.800604in}{3.119537in}}%
\pgfpathlineto{\pgfqpoint{1.823083in}{3.153436in}}%
\pgfpathlineto{\pgfqpoint{1.832182in}{3.163069in}}%
\pgfpathlineto{\pgfqpoint{1.839675in}{3.167977in}}%
\pgfpathlineto{\pgfqpoint{1.846097in}{3.169702in}}%
\pgfpathlineto{\pgfqpoint{1.852252in}{3.169109in}}%
\pgfpathlineto{\pgfqpoint{1.858675in}{3.166176in}}%
\pgfpathlineto{\pgfqpoint{1.865900in}{3.160254in}}%
\pgfpathlineto{\pgfqpoint{1.874731in}{3.149832in}}%
\pgfpathlineto{\pgfqpoint{1.887041in}{3.131422in}}%
\pgfpathlineto{\pgfqpoint{1.909788in}{3.097071in}}%
\pgfpathlineto{\pgfqpoint{1.918887in}{3.087410in}}%
\pgfpathlineto{\pgfqpoint{1.926380in}{3.082474in}}%
\pgfpathlineto{\pgfqpoint{1.932802in}{3.080722in}}%
\pgfpathlineto{\pgfqpoint{1.938957in}{3.081291in}}%
\pgfpathlineto{\pgfqpoint{1.945380in}{3.084198in}}%
\pgfpathlineto{\pgfqpoint{1.952605in}{3.090094in}}%
\pgfpathlineto{\pgfqpoint{1.961169in}{3.100133in}}%
\pgfpathlineto{\pgfqpoint{1.973211in}{3.118030in}}%
\pgfpathlineto{\pgfqpoint{1.997028in}{3.153925in}}%
\pgfpathlineto{\pgfqpoint{2.006127in}{3.163400in}}%
\pgfpathlineto{\pgfqpoint{2.013620in}{3.168151in}}%
\pgfpathlineto{\pgfqpoint{2.020042in}{3.169731in}}%
\pgfpathlineto{\pgfqpoint{2.026197in}{3.168996in}}%
\pgfpathlineto{\pgfqpoint{2.032620in}{3.165923in}}%
\pgfpathlineto{\pgfqpoint{2.039845in}{3.159859in}}%
\pgfpathlineto{\pgfqpoint{2.048676in}{3.149301in}}%
\pgfpathlineto{\pgfqpoint{2.061254in}{3.130365in}}%
\pgfpathlineto{\pgfqpoint{2.083198in}{3.097249in}}%
\pgfpathlineto{\pgfqpoint{2.092564in}{3.087303in}}%
\pgfpathlineto{\pgfqpoint{2.100057in}{3.082417in}}%
\pgfpathlineto{\pgfqpoint{2.106480in}{3.080712in}}%
\pgfpathlineto{\pgfqpoint{2.112635in}{3.081326in}}%
\pgfpathlineto{\pgfqpoint{2.119057in}{3.084278in}}%
\pgfpathlineto{\pgfqpoint{2.126283in}{3.090220in}}%
\pgfpathlineto{\pgfqpoint{2.135114in}{3.100661in}}%
\pgfpathlineto{\pgfqpoint{2.147424in}{3.119084in}}%
\pgfpathlineto{\pgfqpoint{2.170170in}{3.153417in}}%
\pgfpathlineto{\pgfqpoint{2.179269in}{3.163055in}}%
\pgfpathlineto{\pgfqpoint{2.186762in}{3.167970in}}%
\pgfpathlineto{\pgfqpoint{2.193185in}{3.169701in}}%
\pgfpathlineto{\pgfqpoint{2.199340in}{3.169113in}}%
\pgfpathlineto{\pgfqpoint{2.205762in}{3.166186in}}%
\pgfpathlineto{\pgfqpoint{2.212988in}{3.160270in}}%
\pgfpathlineto{\pgfqpoint{2.221551in}{3.150212in}}%
\pgfpathlineto{\pgfqpoint{2.233861in}{3.131875in}}%
\pgfpathlineto{\pgfqpoint{2.257143in}{3.096757in}}%
\pgfpathlineto{\pgfqpoint{2.266242in}{3.087197in}}%
\pgfpathlineto{\pgfqpoint{2.273735in}{3.082361in}}%
\pgfpathlineto{\pgfqpoint{2.280157in}{3.080702in}}%
\pgfpathlineto{\pgfqpoint{2.286312in}{3.081361in}}%
\pgfpathlineto{\pgfqpoint{2.292735in}{3.084358in}}%
\pgfpathlineto{\pgfqpoint{2.299960in}{3.090346in}}%
\pgfpathlineto{\pgfqpoint{2.308791in}{3.100830in}}%
\pgfpathlineto{\pgfqpoint{2.321369in}{3.119714in}}%
\pgfpathlineto{\pgfqpoint{2.343580in}{3.153239in}}%
\pgfpathlineto{\pgfqpoint{2.352947in}{3.163162in}}%
\pgfpathlineto{\pgfqpoint{2.360440in}{3.168027in}}%
\pgfpathlineto{\pgfqpoint{2.366862in}{3.169711in}}%
\pgfpathlineto{\pgfqpoint{2.373017in}{3.169078in}}%
\pgfpathlineto{\pgfqpoint{2.379440in}{3.166106in}}%
\pgfpathlineto{\pgfqpoint{2.386665in}{3.160144in}}%
\pgfpathlineto{\pgfqpoint{2.395496in}{3.149684in}}%
\pgfpathlineto{\pgfqpoint{2.408074in}{3.130818in}}%
\pgfpathlineto{\pgfqpoint{2.430285in}{3.097268in}}%
\pgfpathlineto{\pgfqpoint{2.439652in}{3.087317in}}%
\pgfpathlineto{\pgfqpoint{2.447145in}{3.082424in}}%
\pgfpathlineto{\pgfqpoint{2.453567in}{3.080713in}}%
\pgfpathlineto{\pgfqpoint{2.459722in}{3.081321in}}%
\pgfpathlineto{\pgfqpoint{2.466145in}{3.084268in}}%
\pgfpathlineto{\pgfqpoint{2.473370in}{3.090204in}}%
\pgfpathlineto{\pgfqpoint{2.482201in}{3.100639in}}%
\pgfpathlineto{\pgfqpoint{2.494511in}{3.119059in}}%
\pgfpathlineto{\pgfqpoint{2.517258in}{3.153397in}}%
\pgfpathlineto{\pgfqpoint{2.526356in}{3.163042in}}%
\pgfpathlineto{\pgfqpoint{2.533849in}{3.167963in}}%
\pgfpathlineto{\pgfqpoint{2.540272in}{3.169700in}}%
\pgfpathlineto{\pgfqpoint{2.546427in}{3.169117in}}%
\pgfpathlineto{\pgfqpoint{2.552850in}{3.166196in}}%
\pgfpathlineto{\pgfqpoint{2.560075in}{3.160286in}}%
\pgfpathlineto{\pgfqpoint{2.568638in}{3.150233in}}%
\pgfpathlineto{\pgfqpoint{2.580948in}{3.131900in}}%
\pgfpathlineto{\pgfqpoint{2.604230in}{3.096777in}}%
\pgfpathlineto{\pgfqpoint{2.613329in}{3.087210in}}%
\pgfpathlineto{\pgfqpoint{2.620822in}{3.082368in}}%
\pgfpathlineto{\pgfqpoint{2.627245in}{3.080704in}}%
\pgfpathlineto{\pgfqpoint{2.633400in}{3.081357in}}%
\pgfpathlineto{\pgfqpoint{2.639822in}{3.084348in}}%
\pgfpathlineto{\pgfqpoint{2.647048in}{3.090330in}}%
\pgfpathlineto{\pgfqpoint{2.655879in}{3.100809in}}%
\pgfpathlineto{\pgfqpoint{2.668456in}{3.119689in}}%
\pgfpathlineto{\pgfqpoint{2.690668in}{3.153220in}}%
\pgfpathlineto{\pgfqpoint{2.700034in}{3.163149in}}%
\pgfpathlineto{\pgfqpoint{2.707527in}{3.168020in}}%
\pgfpathlineto{\pgfqpoint{2.713950in}{3.169710in}}%
\pgfpathlineto{\pgfqpoint{2.720105in}{3.169082in}}%
\pgfpathlineto{\pgfqpoint{2.726527in}{3.166116in}}%
\pgfpathlineto{\pgfqpoint{2.733753in}{3.160160in}}%
\pgfpathlineto{\pgfqpoint{2.742584in}{3.149705in}}%
\pgfpathlineto{\pgfqpoint{2.755161in}{3.130844in}}%
\pgfpathlineto{\pgfqpoint{2.777640in}{3.096953in}}%
\pgfpathlineto{\pgfqpoint{2.786739in}{3.087330in}}%
\pgfpathlineto{\pgfqpoint{2.794232in}{3.082431in}}%
\pgfpathlineto{\pgfqpoint{2.800654in}{3.080715in}}%
\pgfpathlineto{\pgfqpoint{2.806809in}{3.081317in}}%
\pgfpathlineto{\pgfqpoint{2.813232in}{3.084258in}}%
\pgfpathlineto{\pgfqpoint{2.820457in}{3.090188in}}%
\pgfpathlineto{\pgfqpoint{2.829289in}{3.100618in}}%
\pgfpathlineto{\pgfqpoint{2.841598in}{3.119034in}}%
\pgfpathlineto{\pgfqpoint{2.864345in}{3.153377in}}%
\pgfpathlineto{\pgfqpoint{2.873444in}{3.163029in}}%
\pgfpathlineto{\pgfqpoint{2.880937in}{3.167956in}}%
\pgfpathlineto{\pgfqpoint{2.887359in}{3.169698in}}%
\pgfpathlineto{\pgfqpoint{2.893514in}{3.169121in}}%
\pgfpathlineto{\pgfqpoint{2.899937in}{3.166206in}}%
\pgfpathlineto{\pgfqpoint{2.907162in}{3.160301in}}%
\pgfpathlineto{\pgfqpoint{2.915726in}{3.150254in}}%
\pgfpathlineto{\pgfqpoint{2.928036in}{3.131925in}}%
\pgfpathlineto{\pgfqpoint{2.951318in}{3.096796in}}%
\pgfpathlineto{\pgfqpoint{2.960416in}{3.087224in}}%
\pgfpathlineto{\pgfqpoint{2.967909in}{3.082375in}}%
\pgfpathlineto{\pgfqpoint{2.974332in}{3.080705in}}%
\pgfpathlineto{\pgfqpoint{2.980487in}{3.081352in}}%
\pgfpathlineto{\pgfqpoint{2.986910in}{3.084338in}}%
\pgfpathlineto{\pgfqpoint{2.994135in}{3.090314in}}%
\pgfpathlineto{\pgfqpoint{3.002966in}{3.100788in}}%
\pgfpathlineto{\pgfqpoint{3.015544in}{3.119663in}}%
\pgfpathlineto{\pgfqpoint{3.037755in}{3.153200in}}%
\pgfpathlineto{\pgfqpoint{3.047121in}{3.163136in}}%
\pgfpathlineto{\pgfqpoint{3.054614in}{3.168013in}}%
\pgfpathlineto{\pgfqpoint{3.061037in}{3.169709in}}%
\pgfpathlineto{\pgfqpoint{3.067192in}{3.169087in}}%
\pgfpathlineto{\pgfqpoint{3.073614in}{3.166126in}}%
\pgfpathlineto{\pgfqpoint{3.080840in}{3.160176in}}%
\pgfpathlineto{\pgfqpoint{3.089671in}{3.149726in}}%
\pgfpathlineto{\pgfqpoint{3.102248in}{3.130869in}}%
\pgfpathlineto{\pgfqpoint{3.124728in}{3.096973in}}%
\pgfpathlineto{\pgfqpoint{3.133826in}{3.087343in}}%
\pgfpathlineto{\pgfqpoint{3.141319in}{3.082438in}}%
\pgfpathlineto{\pgfqpoint{3.147742in}{3.080716in}}%
\pgfpathlineto{\pgfqpoint{3.153897in}{3.081313in}}%
\pgfpathlineto{\pgfqpoint{3.160319in}{3.084248in}}%
\pgfpathlineto{\pgfqpoint{3.167545in}{3.090173in}}%
\pgfpathlineto{\pgfqpoint{3.176376in}{3.100597in}}%
\pgfpathlineto{\pgfqpoint{3.188686in}{3.119009in}}%
\pgfpathlineto{\pgfqpoint{3.211432in}{3.153358in}}%
\pgfpathlineto{\pgfqpoint{3.220531in}{3.163015in}}%
\pgfpathlineto{\pgfqpoint{3.228024in}{3.167949in}}%
\pgfpathlineto{\pgfqpoint{3.234447in}{3.169697in}}%
\pgfpathlineto{\pgfqpoint{3.240602in}{3.169126in}}%
\pgfpathlineto{\pgfqpoint{3.247024in}{3.166216in}}%
\pgfpathlineto{\pgfqpoint{3.254250in}{3.160317in}}%
\pgfpathlineto{\pgfqpoint{3.262813in}{3.150275in}}%
\pgfpathlineto{\pgfqpoint{3.274856in}{3.132376in}}%
\pgfpathlineto{\pgfqpoint{3.298673in}{3.096485in}}%
\pgfpathlineto{\pgfqpoint{3.307771in}{3.087013in}}%
\pgfpathlineto{\pgfqpoint{3.315264in}{3.082265in}}%
\pgfpathlineto{\pgfqpoint{3.321687in}{3.080688in}}%
\pgfpathlineto{\pgfqpoint{3.327842in}{3.081425in}}%
\pgfpathlineto{\pgfqpoint{3.334264in}{3.084502in}}%
\pgfpathlineto{\pgfqpoint{3.341490in}{3.090568in}}%
\pgfpathlineto{\pgfqpoint{3.350321in}{3.101129in}}%
\pgfpathlineto{\pgfqpoint{3.362898in}{3.120067in}}%
\pgfpathlineto{\pgfqpoint{3.384842in}{3.153180in}}%
\pgfpathlineto{\pgfqpoint{3.394209in}{3.163122in}}%
\pgfpathlineto{\pgfqpoint{3.401702in}{3.168006in}}%
\pgfpathlineto{\pgfqpoint{3.408124in}{3.169708in}}%
\pgfpathlineto{\pgfqpoint{3.414279in}{3.169091in}}%
\pgfpathlineto{\pgfqpoint{3.420702in}{3.166136in}}%
\pgfpathlineto{\pgfqpoint{3.427927in}{3.160191in}}%
\pgfpathlineto{\pgfqpoint{3.436758in}{3.149748in}}%
\pgfpathlineto{\pgfqpoint{3.449068in}{3.131322in}}%
\pgfpathlineto{\pgfqpoint{3.471815in}{3.096992in}}%
\pgfpathlineto{\pgfqpoint{3.480914in}{3.087357in}}%
\pgfpathlineto{\pgfqpoint{3.488407in}{3.082445in}}%
\pgfpathlineto{\pgfqpoint{3.494829in}{3.080717in}}%
\pgfpathlineto{\pgfqpoint{3.500984in}{3.081308in}}%
\pgfpathlineto{\pgfqpoint{3.507407in}{3.084238in}}%
\pgfpathlineto{\pgfqpoint{3.514632in}{3.090157in}}%
\pgfpathlineto{\pgfqpoint{3.523196in}{3.100217in}}%
\pgfpathlineto{\pgfqpoint{3.535506in}{3.118557in}}%
\pgfpathlineto{\pgfqpoint{3.558787in}{3.153671in}}%
\pgfpathlineto{\pgfqpoint{3.567886in}{3.163228in}}%
\pgfpathlineto{\pgfqpoint{3.575379in}{3.168062in}}%
\pgfpathlineto{\pgfqpoint{3.581802in}{3.169717in}}%
\pgfpathlineto{\pgfqpoint{3.587957in}{3.169056in}}%
\pgfpathlineto{\pgfqpoint{3.594379in}{3.166055in}}%
\pgfpathlineto{\pgfqpoint{3.601605in}{3.160065in}}%
\pgfpathlineto{\pgfqpoint{3.610436in}{3.149578in}}%
\pgfpathlineto{\pgfqpoint{3.623013in}{3.130692in}}%
\pgfpathlineto{\pgfqpoint{3.645225in}{3.097170in}}%
\pgfpathlineto{\pgfqpoint{3.654591in}{3.087250in}}%
\pgfpathlineto{\pgfqpoint{3.662084in}{3.082389in}}%
\pgfpathlineto{\pgfqpoint{3.668507in}{3.080707in}}%
\pgfpathlineto{\pgfqpoint{3.674662in}{3.081343in}}%
\pgfpathlineto{\pgfqpoint{3.681084in}{3.084318in}}%
\pgfpathlineto{\pgfqpoint{3.688310in}{3.090283in}}%
\pgfpathlineto{\pgfqpoint{3.697141in}{3.100745in}}%
\pgfpathlineto{\pgfqpoint{3.709718in}{3.119613in}}%
\pgfpathlineto{\pgfqpoint{3.731930in}{3.153160in}}%
\pgfpathlineto{\pgfqpoint{3.741296in}{3.163109in}}%
\pgfpathlineto{\pgfqpoint{3.748789in}{3.167999in}}%
\pgfpathlineto{\pgfqpoint{3.755212in}{3.169706in}}%
\pgfpathlineto{\pgfqpoint{3.761367in}{3.169095in}}%
\pgfpathlineto{\pgfqpoint{3.767789in}{3.166146in}}%
\pgfpathlineto{\pgfqpoint{3.775015in}{3.160207in}}%
\pgfpathlineto{\pgfqpoint{3.783846in}{3.149769in}}%
\pgfpathlineto{\pgfqpoint{3.796156in}{3.131347in}}%
\pgfpathlineto{\pgfqpoint{3.818902in}{3.097012in}}%
\pgfpathlineto{\pgfqpoint{3.828001in}{3.087370in}}%
\pgfpathlineto{\pgfqpoint{3.835494in}{3.082452in}}%
\pgfpathlineto{\pgfqpoint{3.841916in}{3.080718in}}%
\pgfpathlineto{\pgfqpoint{3.848071in}{3.081304in}}%
\pgfpathlineto{\pgfqpoint{3.854494in}{3.084228in}}%
\pgfpathlineto{\pgfqpoint{3.861719in}{3.090141in}}%
\pgfpathlineto{\pgfqpoint{3.870283in}{3.100196in}}%
\pgfpathlineto{\pgfqpoint{3.882593in}{3.118532in}}%
\pgfpathlineto{\pgfqpoint{3.905875in}{3.153652in}}%
\pgfpathlineto{\pgfqpoint{3.914973in}{3.163215in}}%
\pgfpathlineto{\pgfqpoint{3.922466in}{3.168055in}}%
\pgfpathlineto{\pgfqpoint{3.928889in}{3.169716in}}%
\pgfpathlineto{\pgfqpoint{3.935044in}{3.169060in}}%
\pgfpathlineto{\pgfqpoint{3.941467in}{3.166066in}}%
\pgfpathlineto{\pgfqpoint{3.948692in}{3.160081in}}%
\pgfpathlineto{\pgfqpoint{3.957523in}{3.149599in}}%
\pgfpathlineto{\pgfqpoint{3.970101in}{3.130718in}}%
\pgfpathlineto{\pgfqpoint{3.992312in}{3.097189in}}%
\pgfpathlineto{\pgfqpoint{4.001678in}{3.087263in}}%
\pgfpathlineto{\pgfqpoint{4.009171in}{3.082396in}}%
\pgfpathlineto{\pgfqpoint{4.015594in}{3.080708in}}%
\pgfpathlineto{\pgfqpoint{4.021749in}{3.081339in}}%
\pgfpathlineto{\pgfqpoint{4.028172in}{3.084308in}}%
\pgfpathlineto{\pgfqpoint{4.035397in}{3.090267in}}%
\pgfpathlineto{\pgfqpoint{4.044228in}{3.100724in}}%
\pgfpathlineto{\pgfqpoint{4.056806in}{3.119588in}}%
\pgfpathlineto{\pgfqpoint{4.079017in}{3.153141in}}%
\pgfpathlineto{\pgfqpoint{4.088383in}{3.163095in}}%
\pgfpathlineto{\pgfqpoint{4.095876in}{3.167991in}}%
\pgfpathlineto{\pgfqpoint{4.102299in}{3.169705in}}%
\pgfpathlineto{\pgfqpoint{4.108454in}{3.169100in}}%
\pgfpathlineto{\pgfqpoint{4.114876in}{3.166156in}}%
\pgfpathlineto{\pgfqpoint{4.122102in}{3.160223in}}%
\pgfpathlineto{\pgfqpoint{4.130933in}{3.149790in}}%
\pgfpathlineto{\pgfqpoint{4.143243in}{3.131372in}}%
\pgfpathlineto{\pgfqpoint{4.165990in}{3.097032in}}%
\pgfpathlineto{\pgfqpoint{4.175088in}{3.087384in}}%
\pgfpathlineto{\pgfqpoint{4.182581in}{3.082459in}}%
\pgfpathlineto{\pgfqpoint{4.189004in}{3.080720in}}%
\pgfpathlineto{\pgfqpoint{4.195159in}{3.081300in}}%
\pgfpathlineto{\pgfqpoint{4.201581in}{3.084218in}}%
\pgfpathlineto{\pgfqpoint{4.208807in}{3.090126in}}%
\pgfpathlineto{\pgfqpoint{4.217370in}{3.100175in}}%
\pgfpathlineto{\pgfqpoint{4.229680in}{3.118507in}}%
\pgfpathlineto{\pgfqpoint{4.252962in}{3.153632in}}%
\pgfpathlineto{\pgfqpoint{4.262061in}{3.163202in}}%
\pgfpathlineto{\pgfqpoint{4.269554in}{3.168048in}}%
\pgfpathlineto{\pgfqpoint{4.275976in}{3.169715in}}%
\pgfpathlineto{\pgfqpoint{4.282131in}{3.169065in}}%
\pgfpathlineto{\pgfqpoint{4.288554in}{3.166076in}}%
\pgfpathlineto{\pgfqpoint{4.295779in}{3.160097in}}%
\pgfpathlineto{\pgfqpoint{4.304610in}{3.149620in}}%
\pgfpathlineto{\pgfqpoint{4.317188in}{3.130743in}}%
\pgfpathlineto{\pgfqpoint{4.339399in}{3.097209in}}%
\pgfpathlineto{\pgfqpoint{4.348766in}{3.087277in}}%
\pgfpathlineto{\pgfqpoint{4.356259in}{3.082403in}}%
\pgfpathlineto{\pgfqpoint{4.362681in}{3.080709in}}%
\pgfpathlineto{\pgfqpoint{4.368836in}{3.081334in}}%
\pgfpathlineto{\pgfqpoint{4.375259in}{3.084298in}}%
\pgfpathlineto{\pgfqpoint{4.382484in}{3.090251in}}%
\pgfpathlineto{\pgfqpoint{4.391315in}{3.100703in}}%
\pgfpathlineto{\pgfqpoint{4.403893in}{3.119563in}}%
\pgfpathlineto{\pgfqpoint{4.426372in}{3.153456in}}%
\pgfpathlineto{\pgfqpoint{4.435471in}{3.163082in}}%
\pgfpathlineto{\pgfqpoint{4.442964in}{3.167984in}}%
\pgfpathlineto{\pgfqpoint{4.449386in}{3.169704in}}%
\pgfpathlineto{\pgfqpoint{4.455541in}{3.169104in}}%
\pgfpathlineto{\pgfqpoint{4.461964in}{3.166166in}}%
\pgfpathlineto{\pgfqpoint{4.469189in}{3.160239in}}%
\pgfpathlineto{\pgfqpoint{4.478020in}{3.149811in}}%
\pgfpathlineto{\pgfqpoint{4.490330in}{3.131397in}}%
\pgfpathlineto{\pgfqpoint{4.513077in}{3.097051in}}%
\pgfpathlineto{\pgfqpoint{4.522176in}{3.087397in}}%
\pgfpathlineto{\pgfqpoint{4.529669in}{3.082467in}}%
\pgfpathlineto{\pgfqpoint{4.536091in}{3.080721in}}%
\pgfpathlineto{\pgfqpoint{4.542246in}{3.081295in}}%
\pgfpathlineto{\pgfqpoint{4.548669in}{3.084208in}}%
\pgfpathlineto{\pgfqpoint{4.555894in}{3.090110in}}%
\pgfpathlineto{\pgfqpoint{4.564458in}{3.100154in}}%
\pgfpathlineto{\pgfqpoint{4.576768in}{3.118481in}}%
\pgfpathlineto{\pgfqpoint{4.600049in}{3.153613in}}%
\pgfpathlineto{\pgfqpoint{4.609148in}{3.163189in}}%
\pgfpathlineto{\pgfqpoint{4.616641in}{3.168041in}}%
\pgfpathlineto{\pgfqpoint{4.623064in}{3.169714in}}%
\pgfpathlineto{\pgfqpoint{4.629219in}{3.169069in}}%
\pgfpathlineto{\pgfqpoint{4.635641in}{3.166086in}}%
\pgfpathlineto{\pgfqpoint{4.642867in}{3.160113in}}%
\pgfpathlineto{\pgfqpoint{4.651698in}{3.149642in}}%
\pgfpathlineto{\pgfqpoint{4.664275in}{3.130768in}}%
\pgfpathlineto{\pgfqpoint{4.686487in}{3.097229in}}%
\pgfpathlineto{\pgfqpoint{4.695853in}{3.087290in}}%
\pgfpathlineto{\pgfqpoint{4.703346in}{3.082410in}}%
\pgfpathlineto{\pgfqpoint{4.709769in}{3.080711in}}%
\pgfpathlineto{\pgfqpoint{4.715924in}{3.081330in}}%
\pgfpathlineto{\pgfqpoint{4.722346in}{3.084288in}}%
\pgfpathlineto{\pgfqpoint{4.729572in}{3.090235in}}%
\pgfpathlineto{\pgfqpoint{4.738403in}{3.100682in}}%
\pgfpathlineto{\pgfqpoint{4.750980in}{3.119537in}}%
\pgfpathlineto{\pgfqpoint{4.773459in}{3.153436in}}%
\pgfpathlineto{\pgfqpoint{4.782558in}{3.163069in}}%
\pgfpathlineto{\pgfqpoint{4.790051in}{3.167977in}}%
\pgfpathlineto{\pgfqpoint{4.796474in}{3.169702in}}%
\pgfpathlineto{\pgfqpoint{4.802629in}{3.169109in}}%
\pgfpathlineto{\pgfqpoint{4.809051in}{3.166176in}}%
\pgfpathlineto{\pgfqpoint{4.816277in}{3.160254in}}%
\pgfpathlineto{\pgfqpoint{4.825108in}{3.149832in}}%
\pgfpathlineto{\pgfqpoint{4.837418in}{3.131422in}}%
\pgfpathlineto{\pgfqpoint{4.860164in}{3.097071in}}%
\pgfpathlineto{\pgfqpoint{4.869263in}{3.087410in}}%
\pgfpathlineto{\pgfqpoint{4.876756in}{3.082474in}}%
\pgfpathlineto{\pgfqpoint{4.883179in}{3.080722in}}%
\pgfpathlineto{\pgfqpoint{4.889334in}{3.081291in}}%
\pgfpathlineto{\pgfqpoint{4.895756in}{3.084198in}}%
\pgfpathlineto{\pgfqpoint{4.902982in}{3.090094in}}%
\pgfpathlineto{\pgfqpoint{4.911545in}{3.100133in}}%
\pgfpathlineto{\pgfqpoint{4.923587in}{3.118030in}}%
\pgfpathlineto{\pgfqpoint{4.947404in}{3.153925in}}%
\pgfpathlineto{\pgfqpoint{4.956503in}{3.163400in}}%
\pgfpathlineto{\pgfqpoint{4.963996in}{3.168151in}}%
\pgfpathlineto{\pgfqpoint{4.970419in}{3.169731in}}%
\pgfpathlineto{\pgfqpoint{4.976574in}{3.168996in}}%
\pgfpathlineto{\pgfqpoint{4.982996in}{3.165923in}}%
\pgfpathlineto{\pgfqpoint{4.990222in}{3.159859in}}%
\pgfpathlineto{\pgfqpoint{4.999053in}{3.149301in}}%
\pgfpathlineto{\pgfqpoint{5.011630in}{3.130365in}}%
\pgfpathlineto{\pgfqpoint{5.033574in}{3.097249in}}%
\pgfpathlineto{\pgfqpoint{5.042940in}{3.087303in}}%
\pgfpathlineto{\pgfqpoint{5.050433in}{3.082417in}}%
\pgfpathlineto{\pgfqpoint{5.056856in}{3.080712in}}%
\pgfpathlineto{\pgfqpoint{5.063011in}{3.081326in}}%
\pgfpathlineto{\pgfqpoint{5.069434in}{3.084278in}}%
\pgfpathlineto{\pgfqpoint{5.076659in}{3.090220in}}%
\pgfpathlineto{\pgfqpoint{5.085490in}{3.100661in}}%
\pgfpathlineto{\pgfqpoint{5.097800in}{3.119084in}}%
\pgfpathlineto{\pgfqpoint{5.120547in}{3.153417in}}%
\pgfpathlineto{\pgfqpoint{5.129645in}{3.163055in}}%
\pgfpathlineto{\pgfqpoint{5.137138in}{3.167970in}}%
\pgfpathlineto{\pgfqpoint{5.143561in}{3.169701in}}%
\pgfpathlineto{\pgfqpoint{5.149716in}{3.169113in}}%
\pgfpathlineto{\pgfqpoint{5.156139in}{3.166186in}}%
\pgfpathlineto{\pgfqpoint{5.163364in}{3.160270in}}%
\pgfpathlineto{\pgfqpoint{5.171927in}{3.150212in}}%
\pgfpathlineto{\pgfqpoint{5.184237in}{3.131875in}}%
\pgfpathlineto{\pgfqpoint{5.207519in}{3.096757in}}%
\pgfpathlineto{\pgfqpoint{5.216618in}{3.087197in}}%
\pgfpathlineto{\pgfqpoint{5.224111in}{3.082361in}}%
\pgfpathlineto{\pgfqpoint{5.230533in}{3.080702in}}%
\pgfpathlineto{\pgfqpoint{5.236688in}{3.081361in}}%
\pgfpathlineto{\pgfqpoint{5.243111in}{3.084358in}}%
\pgfpathlineto{\pgfqpoint{5.250336in}{3.090346in}}%
\pgfpathlineto{\pgfqpoint{5.259168in}{3.100830in}}%
\pgfpathlineto{\pgfqpoint{5.271745in}{3.119714in}}%
\pgfpathlineto{\pgfqpoint{5.293957in}{3.153239in}}%
\pgfpathlineto{\pgfqpoint{5.303323in}{3.163162in}}%
\pgfpathlineto{\pgfqpoint{5.310816in}{3.168027in}}%
\pgfpathlineto{\pgfqpoint{5.317238in}{3.169711in}}%
\pgfpathlineto{\pgfqpoint{5.323393in}{3.169078in}}%
\pgfpathlineto{\pgfqpoint{5.329816in}{3.166106in}}%
\pgfpathlineto{\pgfqpoint{5.337041in}{3.160144in}}%
\pgfpathlineto{\pgfqpoint{5.345872in}{3.149684in}}%
\pgfpathlineto{\pgfqpoint{5.358450in}{3.130818in}}%
\pgfpathlineto{\pgfqpoint{5.380661in}{3.097268in}}%
\pgfpathlineto{\pgfqpoint{5.390028in}{3.087317in}}%
\pgfpathlineto{\pgfqpoint{5.397521in}{3.082424in}}%
\pgfpathlineto{\pgfqpoint{5.403943in}{3.080713in}}%
\pgfpathlineto{\pgfqpoint{5.410098in}{3.081321in}}%
\pgfpathlineto{\pgfqpoint{5.416521in}{3.084268in}}%
\pgfpathlineto{\pgfqpoint{5.423746in}{3.090204in}}%
\pgfpathlineto{\pgfqpoint{5.432577in}{3.100639in}}%
\pgfpathlineto{\pgfqpoint{5.444887in}{3.119059in}}%
\pgfpathlineto{\pgfqpoint{5.467634in}{3.153397in}}%
\pgfpathlineto{\pgfqpoint{5.476733in}{3.163042in}}%
\pgfpathlineto{\pgfqpoint{5.484226in}{3.167963in}}%
\pgfpathlineto{\pgfqpoint{5.490648in}{3.169700in}}%
\pgfpathlineto{\pgfqpoint{5.496803in}{3.169117in}}%
\pgfpathlineto{\pgfqpoint{5.503226in}{3.166196in}}%
\pgfpathlineto{\pgfqpoint{5.510451in}{3.160286in}}%
\pgfpathlineto{\pgfqpoint{5.519015in}{3.150233in}}%
\pgfpathlineto{\pgfqpoint{5.531325in}{3.131900in}}%
\pgfpathlineto{\pgfqpoint{5.554607in}{3.096777in}}%
\pgfpathlineto{\pgfqpoint{5.563705in}{3.087210in}}%
\pgfpathlineto{\pgfqpoint{5.571198in}{3.082368in}}%
\pgfpathlineto{\pgfqpoint{5.577621in}{3.080704in}}%
\pgfpathlineto{\pgfqpoint{5.583776in}{3.081357in}}%
\pgfpathlineto{\pgfqpoint{5.590198in}{3.084348in}}%
\pgfpathlineto{\pgfqpoint{5.597424in}{3.090330in}}%
\pgfpathlineto{\pgfqpoint{5.606255in}{3.100809in}}%
\pgfpathlineto{\pgfqpoint{5.618832in}{3.119689in}}%
\pgfpathlineto{\pgfqpoint{5.641044in}{3.153220in}}%
\pgfpathlineto{\pgfqpoint{5.650410in}{3.163149in}}%
\pgfpathlineto{\pgfqpoint{5.657903in}{3.168020in}}%
\pgfpathlineto{\pgfqpoint{5.664326in}{3.169710in}}%
\pgfpathlineto{\pgfqpoint{5.670481in}{3.169082in}}%
\pgfpathlineto{\pgfqpoint{5.676903in}{3.166116in}}%
\pgfpathlineto{\pgfqpoint{5.684129in}{3.160160in}}%
\pgfpathlineto{\pgfqpoint{5.692960in}{3.149705in}}%
\pgfpathlineto{\pgfqpoint{5.705537in}{3.130844in}}%
\pgfpathlineto{\pgfqpoint{5.728016in}{3.096953in}}%
\pgfpathlineto{\pgfqpoint{5.737115in}{3.087330in}}%
\pgfpathlineto{\pgfqpoint{5.744608in}{3.082431in}}%
\pgfpathlineto{\pgfqpoint{5.751031in}{3.080715in}}%
\pgfpathlineto{\pgfqpoint{5.757186in}{3.081317in}}%
\pgfpathlineto{\pgfqpoint{5.763608in}{3.084258in}}%
\pgfpathlineto{\pgfqpoint{5.770834in}{3.090188in}}%
\pgfpathlineto{\pgfqpoint{5.779665in}{3.100618in}}%
\pgfpathlineto{\pgfqpoint{5.791975in}{3.119034in}}%
\pgfpathlineto{\pgfqpoint{5.814721in}{3.153377in}}%
\pgfpathlineto{\pgfqpoint{5.823820in}{3.163029in}}%
\pgfpathlineto{\pgfqpoint{5.831313in}{3.167956in}}%
\pgfpathlineto{\pgfqpoint{5.837736in}{3.169698in}}%
\pgfpathlineto{\pgfqpoint{5.843891in}{3.169121in}}%
\pgfpathlineto{\pgfqpoint{5.850313in}{3.166206in}}%
\pgfpathlineto{\pgfqpoint{5.857539in}{3.160301in}}%
\pgfpathlineto{\pgfqpoint{5.866102in}{3.150254in}}%
\pgfpathlineto{\pgfqpoint{5.878412in}{3.131925in}}%
\pgfpathlineto{\pgfqpoint{5.901694in}{3.096796in}}%
\pgfpathlineto{\pgfqpoint{5.910793in}{3.087224in}}%
\pgfpathlineto{\pgfqpoint{5.918286in}{3.082375in}}%
\pgfpathlineto{\pgfqpoint{5.924708in}{3.080705in}}%
\pgfpathlineto{\pgfqpoint{5.930863in}{3.081352in}}%
\pgfpathlineto{\pgfqpoint{5.937286in}{3.084338in}}%
\pgfpathlineto{\pgfqpoint{5.944511in}{3.090314in}}%
\pgfpathlineto{\pgfqpoint{5.953342in}{3.100788in}}%
\pgfpathlineto{\pgfqpoint{5.965920in}{3.119663in}}%
\pgfpathlineto{\pgfqpoint{5.988131in}{3.153200in}}%
\pgfpathlineto{\pgfqpoint{5.997497in}{3.163136in}}%
\pgfpathlineto{\pgfqpoint{6.004991in}{3.168013in}}%
\pgfpathlineto{\pgfqpoint{6.011413in}{3.169709in}}%
\pgfpathlineto{\pgfqpoint{6.017568in}{3.169087in}}%
\pgfpathlineto{\pgfqpoint{6.023991in}{3.166126in}}%
\pgfpathlineto{\pgfqpoint{6.031216in}{3.160176in}}%
\pgfpathlineto{\pgfqpoint{6.040047in}{3.149726in}}%
\pgfpathlineto{\pgfqpoint{6.052625in}{3.130869in}}%
\pgfpathlineto{\pgfqpoint{6.075104in}{3.096973in}}%
\pgfpathlineto{\pgfqpoint{6.084202in}{3.087343in}}%
\pgfpathlineto{\pgfqpoint{6.091695in}{3.082438in}}%
\pgfpathlineto{\pgfqpoint{6.098118in}{3.080716in}}%
\pgfpathlineto{\pgfqpoint{6.104273in}{3.081313in}}%
\pgfpathlineto{\pgfqpoint{6.110696in}{3.084248in}}%
\pgfpathlineto{\pgfqpoint{6.117921in}{3.090173in}}%
\pgfpathlineto{\pgfqpoint{6.126752in}{3.100597in}}%
\pgfpathlineto{\pgfqpoint{6.139062in}{3.119009in}}%
\pgfpathlineto{\pgfqpoint{6.161809in}{3.153358in}}%
\pgfpathlineto{\pgfqpoint{6.170907in}{3.163015in}}%
\pgfpathlineto{\pgfqpoint{6.178400in}{3.167949in}}%
\pgfpathlineto{\pgfqpoint{6.184823in}{3.169697in}}%
\pgfpathlineto{\pgfqpoint{6.190978in}{3.169126in}}%
\pgfpathlineto{\pgfqpoint{6.197401in}{3.166216in}}%
\pgfpathlineto{\pgfqpoint{6.204626in}{3.160317in}}%
\pgfpathlineto{\pgfqpoint{6.213189in}{3.150275in}}%
\pgfpathlineto{\pgfqpoint{6.225232in}{3.132376in}}%
\pgfpathlineto{\pgfqpoint{6.249049in}{3.096485in}}%
\pgfpathlineto{\pgfqpoint{6.258148in}{3.087013in}}%
\pgfpathlineto{\pgfqpoint{6.265641in}{3.082265in}}%
\pgfpathlineto{\pgfqpoint{6.272063in}{3.080688in}}%
\pgfpathlineto{\pgfqpoint{6.278218in}{3.081425in}}%
\pgfpathlineto{\pgfqpoint{6.284641in}{3.084502in}}%
\pgfpathlineto{\pgfqpoint{6.291866in}{3.090568in}}%
\pgfpathlineto{\pgfqpoint{6.300697in}{3.101129in}}%
\pgfpathlineto{\pgfqpoint{6.313275in}{3.120067in}}%
\pgfpathlineto{\pgfqpoint{6.335219in}{3.153180in}}%
\pgfpathlineto{\pgfqpoint{6.344585in}{3.163122in}}%
\pgfpathlineto{\pgfqpoint{6.352078in}{3.168006in}}%
\pgfpathlineto{\pgfqpoint{6.358500in}{3.169708in}}%
\pgfpathlineto{\pgfqpoint{6.364655in}{3.169091in}}%
\pgfpathlineto{\pgfqpoint{6.371078in}{3.166136in}}%
\pgfpathlineto{\pgfqpoint{6.378303in}{3.160191in}}%
\pgfpathlineto{\pgfqpoint{6.387134in}{3.149748in}}%
\pgfpathlineto{\pgfqpoint{6.399444in}{3.131322in}}%
\pgfpathlineto{\pgfqpoint{6.422191in}{3.096992in}}%
\pgfpathlineto{\pgfqpoint{6.431290in}{3.087357in}}%
\pgfpathlineto{\pgfqpoint{6.438783in}{3.082445in}}%
\pgfpathlineto{\pgfqpoint{6.445205in}{3.080717in}}%
\pgfpathlineto{\pgfqpoint{6.451360in}{3.081308in}}%
\pgfpathlineto{\pgfqpoint{6.457783in}{3.084238in}}%
\pgfpathlineto{\pgfqpoint{6.465008in}{3.090157in}}%
\pgfpathlineto{\pgfqpoint{6.473572in}{3.100217in}}%
\pgfpathlineto{\pgfqpoint{6.485882in}{3.118557in}}%
\pgfpathlineto{\pgfqpoint{6.509164in}{3.153671in}}%
\pgfpathlineto{\pgfqpoint{6.518262in}{3.163228in}}%
\pgfpathlineto{\pgfqpoint{6.525755in}{3.168062in}}%
\pgfpathlineto{\pgfqpoint{6.532178in}{3.169717in}}%
\pgfpathlineto{\pgfqpoint{6.538333in}{3.169056in}}%
\pgfpathlineto{\pgfqpoint{6.544755in}{3.166055in}}%
\pgfpathlineto{\pgfqpoint{6.551981in}{3.160065in}}%
\pgfpathlineto{\pgfqpoint{6.560812in}{3.149578in}}%
\pgfpathlineto{\pgfqpoint{6.573390in}{3.130692in}}%
\pgfpathlineto{\pgfqpoint{6.595601in}{3.097170in}}%
\pgfpathlineto{\pgfqpoint{6.604967in}{3.087250in}}%
\pgfpathlineto{\pgfqpoint{6.612460in}{3.082389in}}%
\pgfpathlineto{\pgfqpoint{6.618883in}{3.080707in}}%
\pgfpathlineto{\pgfqpoint{6.625038in}{3.081343in}}%
\pgfpathlineto{\pgfqpoint{6.631460in}{3.084318in}}%
\pgfpathlineto{\pgfqpoint{6.638686in}{3.090283in}}%
\pgfpathlineto{\pgfqpoint{6.647517in}{3.100745in}}%
\pgfpathlineto{\pgfqpoint{6.660094in}{3.119613in}}%
\pgfpathlineto{\pgfqpoint{6.663306in}{3.124778in}}%
\pgfpathlineto{\pgfqpoint{6.663306in}{3.124778in}}%
\pgfusepath{stroke}%
\end{pgfscope}%
\begin{pgfscope}%
\pgfpathrectangle{\pgfqpoint{0.467797in}{2.292089in}}{\pgfqpoint{6.490533in}{1.666241in}}%
\pgfusepath{clip}%
\pgfsetrectcap%
\pgfsetroundjoin%
\pgfsetlinewidth{1.505625pt}%
\definecolor{currentstroke}{rgb}{0.090196,0.745098,0.811765}%
\pgfsetstrokecolor{currentstroke}%
\pgfsetdash{}{0pt}%
\pgfpathmoveto{\pgfqpoint{0.762821in}{3.125209in}}%
\pgfpathlineto{\pgfqpoint{0.779680in}{3.150342in}}%
\pgfpathlineto{\pgfqpoint{0.788511in}{3.159577in}}%
\pgfpathlineto{\pgfqpoint{0.795469in}{3.163835in}}%
\pgfpathlineto{\pgfqpoint{0.801624in}{3.165072in}}%
\pgfpathlineto{\pgfqpoint{0.807511in}{3.163953in}}%
\pgfpathlineto{\pgfqpoint{0.813666in}{3.160446in}}%
\pgfpathlineto{\pgfqpoint{0.820892in}{3.153578in}}%
\pgfpathlineto{\pgfqpoint{0.829990in}{3.141599in}}%
\pgfpathlineto{\pgfqpoint{0.865047in}{3.091780in}}%
\pgfpathlineto{\pgfqpoint{0.872273in}{3.086940in}}%
\pgfpathlineto{\pgfqpoint{0.878427in}{3.085371in}}%
\pgfpathlineto{\pgfqpoint{0.884315in}{3.086171in}}%
\pgfpathlineto{\pgfqpoint{0.890470in}{3.089362in}}%
\pgfpathlineto{\pgfqpoint{0.897428in}{3.095616in}}%
\pgfpathlineto{\pgfqpoint{0.906259in}{3.106827in}}%
\pgfpathlineto{\pgfqpoint{0.921245in}{3.130262in}}%
\pgfpathlineto{\pgfqpoint{0.935428in}{3.150919in}}%
\pgfpathlineto{\pgfqpoint{0.943991in}{3.159737in}}%
\pgfpathlineto{\pgfqpoint{0.950949in}{3.163912in}}%
\pgfpathlineto{\pgfqpoint{0.957104in}{3.165071in}}%
\pgfpathlineto{\pgfqpoint{0.962992in}{3.163877in}}%
\pgfpathlineto{\pgfqpoint{0.969147in}{3.160296in}}%
\pgfpathlineto{\pgfqpoint{0.976372in}{3.153354in}}%
\pgfpathlineto{\pgfqpoint{0.985738in}{3.140913in}}%
\pgfpathlineto{\pgfqpoint{1.019724in}{3.092322in}}%
\pgfpathlineto{\pgfqpoint{1.026950in}{3.087225in}}%
\pgfpathlineto{\pgfqpoint{1.033105in}{3.085417in}}%
\pgfpathlineto{\pgfqpoint{1.038992in}{3.085985in}}%
\pgfpathlineto{\pgfqpoint{1.045147in}{3.088945in}}%
\pgfpathlineto{\pgfqpoint{1.052105in}{3.094971in}}%
\pgfpathlineto{\pgfqpoint{1.060668in}{3.105589in}}%
\pgfpathlineto{\pgfqpoint{1.074316in}{3.126708in}}%
\pgfpathlineto{\pgfqpoint{1.090373in}{3.150500in}}%
\pgfpathlineto{\pgfqpoint{1.098936in}{3.159461in}}%
\pgfpathlineto{\pgfqpoint{1.105894in}{3.163778in}}%
\pgfpathlineto{\pgfqpoint{1.112049in}{3.165071in}}%
\pgfpathlineto{\pgfqpoint{1.117936in}{3.164005in}}%
\pgfpathlineto{\pgfqpoint{1.124091in}{3.160551in}}%
\pgfpathlineto{\pgfqpoint{1.131317in}{3.153738in}}%
\pgfpathlineto{\pgfqpoint{1.140416in}{3.141806in}}%
\pgfpathlineto{\pgfqpoint{1.160219in}{3.110725in}}%
\pgfpathlineto{\pgfqpoint{1.171190in}{3.096178in}}%
\pgfpathlineto{\pgfqpoint{1.179219in}{3.088978in}}%
\pgfpathlineto{\pgfqpoint{1.185909in}{3.085853in}}%
\pgfpathlineto{\pgfqpoint{1.191796in}{3.085471in}}%
\pgfpathlineto{\pgfqpoint{1.197684in}{3.087333in}}%
\pgfpathlineto{\pgfqpoint{1.204106in}{3.091798in}}%
\pgfpathlineto{\pgfqpoint{1.211867in}{3.100148in}}%
\pgfpathlineto{\pgfqpoint{1.222304in}{3.115054in}}%
\pgfpathlineto{\pgfqpoint{1.248261in}{3.153602in}}%
\pgfpathlineto{\pgfqpoint{1.256290in}{3.161047in}}%
\pgfpathlineto{\pgfqpoint{1.262980in}{3.164410in}}%
\pgfpathlineto{\pgfqpoint{1.268867in}{3.165010in}}%
\pgfpathlineto{\pgfqpoint{1.274755in}{3.163362in}}%
\pgfpathlineto{\pgfqpoint{1.281177in}{3.159113in}}%
\pgfpathlineto{\pgfqpoint{1.288670in}{3.151308in}}%
\pgfpathlineto{\pgfqpoint{1.298572in}{3.137490in}}%
\pgfpathlineto{\pgfqpoint{1.327473in}{3.095097in}}%
\pgfpathlineto{\pgfqpoint{1.335234in}{3.088502in}}%
\pgfpathlineto{\pgfqpoint{1.341657in}{3.085741in}}%
\pgfpathlineto{\pgfqpoint{1.347544in}{3.085537in}}%
\pgfpathlineto{\pgfqpoint{1.353431in}{3.087574in}}%
\pgfpathlineto{\pgfqpoint{1.359854in}{3.092213in}}%
\pgfpathlineto{\pgfqpoint{1.367615in}{3.100735in}}%
\pgfpathlineto{\pgfqpoint{1.378319in}{3.116201in}}%
\pgfpathlineto{\pgfqpoint{1.402939in}{3.152908in}}%
\pgfpathlineto{\pgfqpoint{1.411235in}{3.160805in}}%
\pgfpathlineto{\pgfqpoint{1.417925in}{3.164307in}}%
\pgfpathlineto{\pgfqpoint{1.423812in}{3.165036in}}%
\pgfpathlineto{\pgfqpoint{1.429700in}{3.163516in}}%
\pgfpathlineto{\pgfqpoint{1.436122in}{3.159397in}}%
\pgfpathlineto{\pgfqpoint{1.443615in}{3.151717in}}%
\pgfpathlineto{\pgfqpoint{1.453517in}{3.138007in}}%
\pgfpathlineto{\pgfqpoint{1.482954in}{3.094890in}}%
\pgfpathlineto{\pgfqpoint{1.490714in}{3.088379in}}%
\pgfpathlineto{\pgfqpoint{1.497137in}{3.085697in}}%
\pgfpathlineto{\pgfqpoint{1.503024in}{3.085569in}}%
\pgfpathlineto{\pgfqpoint{1.508912in}{3.087680in}}%
\pgfpathlineto{\pgfqpoint{1.515602in}{3.092639in}}%
\pgfpathlineto{\pgfqpoint{1.523630in}{3.101678in}}%
\pgfpathlineto{\pgfqpoint{1.534602in}{3.117779in}}%
\pgfpathlineto{\pgfqpoint{1.557349in}{3.151878in}}%
\pgfpathlineto{\pgfqpoint{1.565644in}{3.160149in}}%
\pgfpathlineto{\pgfqpoint{1.572602in}{3.164105in}}%
\pgfpathlineto{\pgfqpoint{1.578757in}{3.165056in}}%
\pgfpathlineto{\pgfqpoint{1.584645in}{3.163664in}}%
\pgfpathlineto{\pgfqpoint{1.591067in}{3.159674in}}%
\pgfpathlineto{\pgfqpoint{1.598560in}{3.152122in}}%
\pgfpathlineto{\pgfqpoint{1.608194in}{3.138928in}}%
\pgfpathlineto{\pgfqpoint{1.638969in}{3.094136in}}%
\pgfpathlineto{\pgfqpoint{1.646462in}{3.088099in}}%
\pgfpathlineto{\pgfqpoint{1.652885in}{3.085605in}}%
\pgfpathlineto{\pgfqpoint{1.658772in}{3.085655in}}%
\pgfpathlineto{\pgfqpoint{1.664659in}{3.087939in}}%
\pgfpathlineto{\pgfqpoint{1.671349in}{3.093077in}}%
\pgfpathlineto{\pgfqpoint{1.679378in}{3.102287in}}%
\pgfpathlineto{\pgfqpoint{1.690885in}{3.119369in}}%
\pgfpathlineto{\pgfqpoint{1.712293in}{3.151471in}}%
\pgfpathlineto{\pgfqpoint{1.720857in}{3.160094in}}%
\pgfpathlineto{\pgfqpoint{1.727815in}{3.164080in}}%
\pgfpathlineto{\pgfqpoint{1.733970in}{3.165059in}}%
\pgfpathlineto{\pgfqpoint{1.739857in}{3.163693in}}%
\pgfpathlineto{\pgfqpoint{1.746280in}{3.159731in}}%
\pgfpathlineto{\pgfqpoint{1.753773in}{3.152206in}}%
\pgfpathlineto{\pgfqpoint{1.763407in}{3.139035in}}%
\pgfpathlineto{\pgfqpoint{1.794449in}{3.093938in}}%
\pgfpathlineto{\pgfqpoint{1.801942in}{3.087984in}}%
\pgfpathlineto{\pgfqpoint{1.808365in}{3.085570in}}%
\pgfpathlineto{\pgfqpoint{1.814252in}{3.085696in}}%
\pgfpathlineto{\pgfqpoint{1.820139in}{3.088053in}}%
\pgfpathlineto{\pgfqpoint{1.826830in}{3.093267in}}%
\pgfpathlineto{\pgfqpoint{1.834858in}{3.102548in}}%
\pgfpathlineto{\pgfqpoint{1.846365in}{3.119684in}}%
\pgfpathlineto{\pgfqpoint{1.867506in}{3.151385in}}%
\pgfpathlineto{\pgfqpoint{1.876069in}{3.160039in}}%
\pgfpathlineto{\pgfqpoint{1.883027in}{3.164054in}}%
\pgfpathlineto{\pgfqpoint{1.889182in}{3.165061in}}%
\pgfpathlineto{\pgfqpoint{1.895070in}{3.163723in}}%
\pgfpathlineto{\pgfqpoint{1.901225in}{3.160001in}}%
\pgfpathlineto{\pgfqpoint{1.908450in}{3.152916in}}%
\pgfpathlineto{\pgfqpoint{1.917816in}{3.140347in}}%
\pgfpathlineto{\pgfqpoint{1.951000in}{3.092712in}}%
\pgfpathlineto{\pgfqpoint{1.958225in}{3.087437in}}%
\pgfpathlineto{\pgfqpoint{1.964380in}{3.085463in}}%
\pgfpathlineto{\pgfqpoint{1.970267in}{3.085869in}}%
\pgfpathlineto{\pgfqpoint{1.976422in}{3.088668in}}%
\pgfpathlineto{\pgfqpoint{1.983380in}{3.094531in}}%
\pgfpathlineto{\pgfqpoint{1.991944in}{3.104999in}}%
\pgfpathlineto{\pgfqpoint{2.005056in}{3.125164in}}%
\pgfpathlineto{\pgfqpoint{2.021916in}{3.150307in}}%
\pgfpathlineto{\pgfqpoint{2.030747in}{3.159554in}}%
\pgfpathlineto{\pgfqpoint{2.037705in}{3.163824in}}%
\pgfpathlineto{\pgfqpoint{2.043860in}{3.165072in}}%
\pgfpathlineto{\pgfqpoint{2.049747in}{3.163963in}}%
\pgfpathlineto{\pgfqpoint{2.055902in}{3.160467in}}%
\pgfpathlineto{\pgfqpoint{2.063127in}{3.153610in}}%
\pgfpathlineto{\pgfqpoint{2.072226in}{3.141640in}}%
\pgfpathlineto{\pgfqpoint{2.107550in}{3.091571in}}%
\pgfpathlineto{\pgfqpoint{2.114776in}{3.086834in}}%
\pgfpathlineto{\pgfqpoint{2.120931in}{3.085360in}}%
\pgfpathlineto{\pgfqpoint{2.126818in}{3.086250in}}%
\pgfpathlineto{\pgfqpoint{2.132973in}{3.089532in}}%
\pgfpathlineto{\pgfqpoint{2.139931in}{3.095877in}}%
\pgfpathlineto{\pgfqpoint{2.148762in}{3.107171in}}%
\pgfpathlineto{\pgfqpoint{2.164283in}{3.131499in}}%
\pgfpathlineto{\pgfqpoint{2.177931in}{3.151213in}}%
\pgfpathlineto{\pgfqpoint{2.186495in}{3.159928in}}%
\pgfpathlineto{\pgfqpoint{2.193452in}{3.164003in}}%
\pgfpathlineto{\pgfqpoint{2.199607in}{3.165066in}}%
\pgfpathlineto{\pgfqpoint{2.205495in}{3.163781in}}%
\pgfpathlineto{\pgfqpoint{2.211650in}{3.160111in}}%
\pgfpathlineto{\pgfqpoint{2.218875in}{3.153079in}}%
\pgfpathlineto{\pgfqpoint{2.228241in}{3.140557in}}%
\pgfpathlineto{\pgfqpoint{2.261692in}{3.092594in}}%
\pgfpathlineto{\pgfqpoint{2.268918in}{3.087373in}}%
\pgfpathlineto{\pgfqpoint{2.275073in}{3.085448in}}%
\pgfpathlineto{\pgfqpoint{2.280960in}{3.085903in}}%
\pgfpathlineto{\pgfqpoint{2.287115in}{3.088750in}}%
\pgfpathlineto{\pgfqpoint{2.294073in}{3.094662in}}%
\pgfpathlineto{\pgfqpoint{2.302636in}{3.105175in}}%
\pgfpathlineto{\pgfqpoint{2.316017in}{3.125800in}}%
\pgfpathlineto{\pgfqpoint{2.332608in}{3.150465in}}%
\pgfpathlineto{\pgfqpoint{2.341172in}{3.159438in}}%
\pgfpathlineto{\pgfqpoint{2.348130in}{3.163767in}}%
\pgfpathlineto{\pgfqpoint{2.354285in}{3.165071in}}%
\pgfpathlineto{\pgfqpoint{2.360172in}{3.164016in}}%
\pgfpathlineto{\pgfqpoint{2.366327in}{3.160572in}}%
\pgfpathlineto{\pgfqpoint{2.373552in}{3.153769in}}%
\pgfpathlineto{\pgfqpoint{2.382651in}{3.141847in}}%
\pgfpathlineto{\pgfqpoint{2.402186in}{3.111171in}}%
\pgfpathlineto{\pgfqpoint{2.413426in}{3.096210in}}%
\pgfpathlineto{\pgfqpoint{2.421454in}{3.088997in}}%
\pgfpathlineto{\pgfqpoint{2.428144in}{3.085860in}}%
\pgfpathlineto{\pgfqpoint{2.434032in}{3.085467in}}%
\pgfpathlineto{\pgfqpoint{2.439919in}{3.087319in}}%
\pgfpathlineto{\pgfqpoint{2.446342in}{3.091774in}}%
\pgfpathlineto{\pgfqpoint{2.454102in}{3.100112in}}%
\pgfpathlineto{\pgfqpoint{2.464539in}{3.115010in}}%
\pgfpathlineto{\pgfqpoint{2.490497in}{3.153570in}}%
\pgfpathlineto{\pgfqpoint{2.498525in}{3.161027in}}%
\pgfpathlineto{\pgfqpoint{2.505215in}{3.164401in}}%
\pgfpathlineto{\pgfqpoint{2.511103in}{3.165012in}}%
\pgfpathlineto{\pgfqpoint{2.516990in}{3.163375in}}%
\pgfpathlineto{\pgfqpoint{2.523413in}{3.159137in}}%
\pgfpathlineto{\pgfqpoint{2.530906in}{3.151342in}}%
\pgfpathlineto{\pgfqpoint{2.540807in}{3.137533in}}%
\pgfpathlineto{\pgfqpoint{2.569709in}{3.095127in}}%
\pgfpathlineto{\pgfqpoint{2.577470in}{3.088519in}}%
\pgfpathlineto{\pgfqpoint{2.583892in}{3.085747in}}%
\pgfpathlineto{\pgfqpoint{2.589780in}{3.085532in}}%
\pgfpathlineto{\pgfqpoint{2.595667in}{3.087559in}}%
\pgfpathlineto{\pgfqpoint{2.602089in}{3.092188in}}%
\pgfpathlineto{\pgfqpoint{2.609850in}{3.100699in}}%
\pgfpathlineto{\pgfqpoint{2.620554in}{3.116157in}}%
\pgfpathlineto{\pgfqpoint{2.645174in}{3.152876in}}%
\pgfpathlineto{\pgfqpoint{2.653470in}{3.160785in}}%
\pgfpathlineto{\pgfqpoint{2.660160in}{3.164298in}}%
\pgfpathlineto{\pgfqpoint{2.666048in}{3.165038in}}%
\pgfpathlineto{\pgfqpoint{2.671935in}{3.163529in}}%
\pgfpathlineto{\pgfqpoint{2.678358in}{3.159420in}}%
\pgfpathlineto{\pgfqpoint{2.685851in}{3.151751in}}%
\pgfpathlineto{\pgfqpoint{2.695752in}{3.138050in}}%
\pgfpathlineto{\pgfqpoint{2.725457in}{3.094640in}}%
\pgfpathlineto{\pgfqpoint{2.733217in}{3.088233in}}%
\pgfpathlineto{\pgfqpoint{2.739640in}{3.085648in}}%
\pgfpathlineto{\pgfqpoint{2.745527in}{3.085612in}}%
\pgfpathlineto{\pgfqpoint{2.751415in}{3.087812in}}%
\pgfpathlineto{\pgfqpoint{2.758105in}{3.092864in}}%
\pgfpathlineto{\pgfqpoint{2.766133in}{3.101991in}}%
\pgfpathlineto{\pgfqpoint{2.777373in}{3.118584in}}%
\pgfpathlineto{\pgfqpoint{2.799316in}{3.151522in}}%
\pgfpathlineto{\pgfqpoint{2.807880in}{3.160127in}}%
\pgfpathlineto{\pgfqpoint{2.814838in}{3.164095in}}%
\pgfpathlineto{\pgfqpoint{2.820993in}{3.165057in}}%
\pgfpathlineto{\pgfqpoint{2.826880in}{3.163676in}}%
\pgfpathlineto{\pgfqpoint{2.833303in}{3.159697in}}%
\pgfpathlineto{\pgfqpoint{2.840796in}{3.152156in}}%
\pgfpathlineto{\pgfqpoint{2.850430in}{3.138971in}}%
\pgfpathlineto{\pgfqpoint{2.881472in}{3.093896in}}%
\pgfpathlineto{\pgfqpoint{2.888965in}{3.087959in}}%
\pgfpathlineto{\pgfqpoint{2.895388in}{3.085563in}}%
\pgfpathlineto{\pgfqpoint{2.901275in}{3.085705in}}%
\pgfpathlineto{\pgfqpoint{2.907162in}{3.088078in}}%
\pgfpathlineto{\pgfqpoint{2.913853in}{3.093308in}}%
\pgfpathlineto{\pgfqpoint{2.921881in}{3.102604in}}%
\pgfpathlineto{\pgfqpoint{2.933388in}{3.119752in}}%
\pgfpathlineto{\pgfqpoint{2.954261in}{3.151110in}}%
\pgfpathlineto{\pgfqpoint{2.962825in}{3.159861in}}%
\pgfpathlineto{\pgfqpoint{2.969783in}{3.163971in}}%
\pgfpathlineto{\pgfqpoint{2.975938in}{3.165068in}}%
\pgfpathlineto{\pgfqpoint{2.981825in}{3.163815in}}%
\pgfpathlineto{\pgfqpoint{2.987980in}{3.160176in}}%
\pgfpathlineto{\pgfqpoint{2.995205in}{3.153177in}}%
\pgfpathlineto{\pgfqpoint{3.004572in}{3.140683in}}%
\pgfpathlineto{\pgfqpoint{3.038290in}{3.092425in}}%
\pgfpathlineto{\pgfqpoint{3.045516in}{3.087281in}}%
\pgfpathlineto{\pgfqpoint{3.051671in}{3.085428in}}%
\pgfpathlineto{\pgfqpoint{3.057558in}{3.085953in}}%
\pgfpathlineto{\pgfqpoint{3.063713in}{3.088870in}}%
\pgfpathlineto{\pgfqpoint{3.070671in}{3.094853in}}%
\pgfpathlineto{\pgfqpoint{3.079234in}{3.105431in}}%
\pgfpathlineto{\pgfqpoint{3.092615in}{3.126095in}}%
\pgfpathlineto{\pgfqpoint{3.108939in}{3.150359in}}%
\pgfpathlineto{\pgfqpoint{3.117770in}{3.159588in}}%
\pgfpathlineto{\pgfqpoint{3.124728in}{3.163840in}}%
\pgfpathlineto{\pgfqpoint{3.130883in}{3.165072in}}%
\pgfpathlineto{\pgfqpoint{3.136770in}{3.163947in}}%
\pgfpathlineto{\pgfqpoint{3.142925in}{3.160435in}}%
\pgfpathlineto{\pgfqpoint{3.150150in}{3.153562in}}%
\pgfpathlineto{\pgfqpoint{3.159249in}{3.141578in}}%
\pgfpathlineto{\pgfqpoint{3.194306in}{3.091767in}}%
\pgfpathlineto{\pgfqpoint{3.201531in}{3.086934in}}%
\pgfpathlineto{\pgfqpoint{3.207686in}{3.085370in}}%
\pgfpathlineto{\pgfqpoint{3.213573in}{3.086175in}}%
\pgfpathlineto{\pgfqpoint{3.219728in}{3.089372in}}%
\pgfpathlineto{\pgfqpoint{3.226686in}{3.095632in}}%
\pgfpathlineto{\pgfqpoint{3.235517in}{3.106847in}}%
\pgfpathlineto{\pgfqpoint{3.250503in}{3.130284in}}%
\pgfpathlineto{\pgfqpoint{3.264686in}{3.150937in}}%
\pgfpathlineto{\pgfqpoint{3.273250in}{3.159748in}}%
\pgfpathlineto{\pgfqpoint{3.280208in}{3.163918in}}%
\pgfpathlineto{\pgfqpoint{3.286363in}{3.165070in}}%
\pgfpathlineto{\pgfqpoint{3.292250in}{3.163871in}}%
\pgfpathlineto{\pgfqpoint{3.298405in}{3.160285in}}%
\pgfpathlineto{\pgfqpoint{3.305630in}{3.153338in}}%
\pgfpathlineto{\pgfqpoint{3.314997in}{3.140892in}}%
\pgfpathlineto{\pgfqpoint{3.348983in}{3.092309in}}%
\pgfpathlineto{\pgfqpoint{3.356208in}{3.087218in}}%
\pgfpathlineto{\pgfqpoint{3.362363in}{3.085416in}}%
\pgfpathlineto{\pgfqpoint{3.368251in}{3.085989in}}%
\pgfpathlineto{\pgfqpoint{3.374406in}{3.088955in}}%
\pgfpathlineto{\pgfqpoint{3.381363in}{3.094986in}}%
\pgfpathlineto{\pgfqpoint{3.389927in}{3.105609in}}%
\pgfpathlineto{\pgfqpoint{3.403575in}{3.126731in}}%
\pgfpathlineto{\pgfqpoint{3.419631in}{3.150518in}}%
\pgfpathlineto{\pgfqpoint{3.428195in}{3.159473in}}%
\pgfpathlineto{\pgfqpoint{3.435153in}{3.163784in}}%
\pgfpathlineto{\pgfqpoint{3.441308in}{3.165071in}}%
\pgfpathlineto{\pgfqpoint{3.447195in}{3.164000in}}%
\pgfpathlineto{\pgfqpoint{3.453350in}{3.160541in}}%
\pgfpathlineto{\pgfqpoint{3.460575in}{3.153722in}}%
\pgfpathlineto{\pgfqpoint{3.469674in}{3.141785in}}%
\pgfpathlineto{\pgfqpoint{3.490012in}{3.109904in}}%
\pgfpathlineto{\pgfqpoint{3.500984in}{3.095578in}}%
\pgfpathlineto{\pgfqpoint{3.508745in}{3.088791in}}%
\pgfpathlineto{\pgfqpoint{3.515167in}{3.085849in}}%
\pgfpathlineto{\pgfqpoint{3.521055in}{3.085473in}}%
\pgfpathlineto{\pgfqpoint{3.526942in}{3.087341in}}%
\pgfpathlineto{\pgfqpoint{3.533365in}{3.091811in}}%
\pgfpathlineto{\pgfqpoint{3.541125in}{3.100165in}}%
\pgfpathlineto{\pgfqpoint{3.551562in}{3.115076in}}%
\pgfpathlineto{\pgfqpoint{3.577520in}{3.153618in}}%
\pgfpathlineto{\pgfqpoint{3.585548in}{3.161057in}}%
\pgfpathlineto{\pgfqpoint{3.592238in}{3.164414in}}%
\pgfpathlineto{\pgfqpoint{3.598126in}{3.165009in}}%
\pgfpathlineto{\pgfqpoint{3.604013in}{3.163356in}}%
\pgfpathlineto{\pgfqpoint{3.610436in}{3.159102in}}%
\pgfpathlineto{\pgfqpoint{3.617929in}{3.151291in}}%
\pgfpathlineto{\pgfqpoint{3.627830in}{3.137468in}}%
\pgfpathlineto{\pgfqpoint{3.656464in}{3.095367in}}%
\pgfpathlineto{\pgfqpoint{3.664225in}{3.088663in}}%
\pgfpathlineto{\pgfqpoint{3.670647in}{3.085800in}}%
\pgfpathlineto{\pgfqpoint{3.676535in}{3.085499in}}%
\pgfpathlineto{\pgfqpoint{3.682422in}{3.087441in}}%
\pgfpathlineto{\pgfqpoint{3.688845in}{3.091985in}}%
\pgfpathlineto{\pgfqpoint{3.696605in}{3.100414in}}%
\pgfpathlineto{\pgfqpoint{3.707042in}{3.115384in}}%
\pgfpathlineto{\pgfqpoint{3.732733in}{3.153538in}}%
\pgfpathlineto{\pgfqpoint{3.740761in}{3.161007in}}%
\pgfpathlineto{\pgfqpoint{3.747451in}{3.164393in}}%
\pgfpathlineto{\pgfqpoint{3.753338in}{3.165015in}}%
\pgfpathlineto{\pgfqpoint{3.759226in}{3.163388in}}%
\pgfpathlineto{\pgfqpoint{3.765648in}{3.159161in}}%
\pgfpathlineto{\pgfqpoint{3.773141in}{3.151376in}}%
\pgfpathlineto{\pgfqpoint{3.783043in}{3.137576in}}%
\pgfpathlineto{\pgfqpoint{3.811944in}{3.095157in}}%
\pgfpathlineto{\pgfqpoint{3.819705in}{3.088537in}}%
\pgfpathlineto{\pgfqpoint{3.826128in}{3.085754in}}%
\pgfpathlineto{\pgfqpoint{3.832015in}{3.085528in}}%
\pgfpathlineto{\pgfqpoint{3.837902in}{3.087544in}}%
\pgfpathlineto{\pgfqpoint{3.844325in}{3.092162in}}%
\pgfpathlineto{\pgfqpoint{3.852086in}{3.100663in}}%
\pgfpathlineto{\pgfqpoint{3.862790in}{3.116112in}}%
\pgfpathlineto{\pgfqpoint{3.887410in}{3.152843in}}%
\pgfpathlineto{\pgfqpoint{3.895706in}{3.160764in}}%
\pgfpathlineto{\pgfqpoint{3.902396in}{3.164289in}}%
\pgfpathlineto{\pgfqpoint{3.908283in}{3.165040in}}%
\pgfpathlineto{\pgfqpoint{3.914171in}{3.163542in}}%
\pgfpathlineto{\pgfqpoint{3.920593in}{3.159444in}}%
\pgfpathlineto{\pgfqpoint{3.928086in}{3.151785in}}%
\pgfpathlineto{\pgfqpoint{3.937988in}{3.138093in}}%
\pgfpathlineto{\pgfqpoint{3.967692in}{3.094670in}}%
\pgfpathlineto{\pgfqpoint{3.975453in}{3.088250in}}%
\pgfpathlineto{\pgfqpoint{3.981875in}{3.085654in}}%
\pgfpathlineto{\pgfqpoint{3.987763in}{3.085607in}}%
\pgfpathlineto{\pgfqpoint{3.993650in}{3.087796in}}%
\pgfpathlineto{\pgfqpoint{4.000340in}{3.092837in}}%
\pgfpathlineto{\pgfqpoint{4.008369in}{3.101954in}}%
\pgfpathlineto{\pgfqpoint{4.019608in}{3.118539in}}%
\pgfpathlineto{\pgfqpoint{4.041820in}{3.151811in}}%
\pgfpathlineto{\pgfqpoint{4.050115in}{3.160105in}}%
\pgfpathlineto{\pgfqpoint{4.057073in}{3.164085in}}%
\pgfpathlineto{\pgfqpoint{4.063228in}{3.165058in}}%
\pgfpathlineto{\pgfqpoint{4.069116in}{3.163688in}}%
\pgfpathlineto{\pgfqpoint{4.075538in}{3.159720in}}%
\pgfpathlineto{\pgfqpoint{4.083031in}{3.152189in}}%
\pgfpathlineto{\pgfqpoint{4.092665in}{3.139014in}}%
\pgfpathlineto{\pgfqpoint{4.123708in}{3.093924in}}%
\pgfpathlineto{\pgfqpoint{4.131201in}{3.087976in}}%
\pgfpathlineto{\pgfqpoint{4.137623in}{3.085568in}}%
\pgfpathlineto{\pgfqpoint{4.143511in}{3.085699in}}%
\pgfpathlineto{\pgfqpoint{4.149398in}{3.088061in}}%
\pgfpathlineto{\pgfqpoint{4.156088in}{3.093280in}}%
\pgfpathlineto{\pgfqpoint{4.164116in}{3.102567in}}%
\pgfpathlineto{\pgfqpoint{4.175623in}{3.119707in}}%
\pgfpathlineto{\pgfqpoint{4.196497in}{3.151075in}}%
\pgfpathlineto{\pgfqpoint{4.205060in}{3.159839in}}%
\pgfpathlineto{\pgfqpoint{4.212018in}{3.163961in}}%
\pgfpathlineto{\pgfqpoint{4.218173in}{3.165069in}}%
\pgfpathlineto{\pgfqpoint{4.224060in}{3.163826in}}%
\pgfpathlineto{\pgfqpoint{4.230215in}{3.160198in}}%
\pgfpathlineto{\pgfqpoint{4.237441in}{3.153209in}}%
\pgfpathlineto{\pgfqpoint{4.246807in}{3.140725in}}%
\pgfpathlineto{\pgfqpoint{4.280526in}{3.092451in}}%
\pgfpathlineto{\pgfqpoint{4.287751in}{3.087295in}}%
\pgfpathlineto{\pgfqpoint{4.293906in}{3.085431in}}%
\pgfpathlineto{\pgfqpoint{4.299793in}{3.085945in}}%
\pgfpathlineto{\pgfqpoint{4.305948in}{3.088851in}}%
\pgfpathlineto{\pgfqpoint{4.312906in}{3.094823in}}%
\pgfpathlineto{\pgfqpoint{4.321470in}{3.105392in}}%
\pgfpathlineto{\pgfqpoint{4.334850in}{3.126050in}}%
\pgfpathlineto{\pgfqpoint{4.351174in}{3.150324in}}%
\pgfpathlineto{\pgfqpoint{4.360005in}{3.159565in}}%
\pgfpathlineto{\pgfqpoint{4.366963in}{3.163829in}}%
\pgfpathlineto{\pgfqpoint{4.373118in}{3.165072in}}%
\pgfpathlineto{\pgfqpoint{4.379005in}{3.163958in}}%
\pgfpathlineto{\pgfqpoint{4.385160in}{3.160456in}}%
\pgfpathlineto{\pgfqpoint{4.392386in}{3.153594in}}%
\pgfpathlineto{\pgfqpoint{4.401484in}{3.141620in}}%
\pgfpathlineto{\pgfqpoint{4.436541in}{3.091792in}}%
\pgfpathlineto{\pgfqpoint{4.443766in}{3.086947in}}%
\pgfpathlineto{\pgfqpoint{4.449921in}{3.085372in}}%
\pgfpathlineto{\pgfqpoint{4.455809in}{3.086166in}}%
\pgfpathlineto{\pgfqpoint{4.461964in}{3.089352in}}%
\pgfpathlineto{\pgfqpoint{4.468922in}{3.095601in}}%
\pgfpathlineto{\pgfqpoint{4.477753in}{3.106807in}}%
\pgfpathlineto{\pgfqpoint{4.492739in}{3.130239in}}%
\pgfpathlineto{\pgfqpoint{4.506922in}{3.150902in}}%
\pgfpathlineto{\pgfqpoint{4.515485in}{3.159725in}}%
\pgfpathlineto{\pgfqpoint{4.522443in}{3.163907in}}%
\pgfpathlineto{\pgfqpoint{4.528598in}{3.165071in}}%
\pgfpathlineto{\pgfqpoint{4.534486in}{3.163882in}}%
\pgfpathlineto{\pgfqpoint{4.540641in}{3.160307in}}%
\pgfpathlineto{\pgfqpoint{4.547866in}{3.153370in}}%
\pgfpathlineto{\pgfqpoint{4.557232in}{3.140934in}}%
\pgfpathlineto{\pgfqpoint{4.591218in}{3.092335in}}%
\pgfpathlineto{\pgfqpoint{4.598444in}{3.087232in}}%
\pgfpathlineto{\pgfqpoint{4.604599in}{3.085418in}}%
\pgfpathlineto{\pgfqpoint{4.610486in}{3.085981in}}%
\pgfpathlineto{\pgfqpoint{4.616641in}{3.088936in}}%
\pgfpathlineto{\pgfqpoint{4.623599in}{3.094956in}}%
\pgfpathlineto{\pgfqpoint{4.632162in}{3.105569in}}%
\pgfpathlineto{\pgfqpoint{4.645810in}{3.126686in}}%
\pgfpathlineto{\pgfqpoint{4.661867in}{3.150483in}}%
\pgfpathlineto{\pgfqpoint{4.670430in}{3.159449in}}%
\pgfpathlineto{\pgfqpoint{4.677388in}{3.163772in}}%
\pgfpathlineto{\pgfqpoint{4.683543in}{3.165071in}}%
\pgfpathlineto{\pgfqpoint{4.689430in}{3.164011in}}%
\pgfpathlineto{\pgfqpoint{4.695585in}{3.160562in}}%
\pgfpathlineto{\pgfqpoint{4.702811in}{3.153753in}}%
\pgfpathlineto{\pgfqpoint{4.711910in}{3.141827in}}%
\pgfpathlineto{\pgfqpoint{4.731713in}{3.110747in}}%
\pgfpathlineto{\pgfqpoint{4.742684in}{3.096194in}}%
\pgfpathlineto{\pgfqpoint{4.750713in}{3.088988in}}%
\pgfpathlineto{\pgfqpoint{4.757403in}{3.085856in}}%
\pgfpathlineto{\pgfqpoint{4.763290in}{3.085469in}}%
\pgfpathlineto{\pgfqpoint{4.769178in}{3.087326in}}%
\pgfpathlineto{\pgfqpoint{4.775600in}{3.091786in}}%
\pgfpathlineto{\pgfqpoint{4.783361in}{3.100130in}}%
\pgfpathlineto{\pgfqpoint{4.793798in}{3.115032in}}%
\pgfpathlineto{\pgfqpoint{4.819755in}{3.153586in}}%
\pgfpathlineto{\pgfqpoint{4.827784in}{3.161037in}}%
\pgfpathlineto{\pgfqpoint{4.834474in}{3.164405in}}%
\pgfpathlineto{\pgfqpoint{4.840361in}{3.165011in}}%
\pgfpathlineto{\pgfqpoint{4.846249in}{3.163369in}}%
\pgfpathlineto{\pgfqpoint{4.852671in}{3.159125in}}%
\pgfpathlineto{\pgfqpoint{4.860164in}{3.151325in}}%
\pgfpathlineto{\pgfqpoint{4.870066in}{3.137511in}}%
\pgfpathlineto{\pgfqpoint{4.898967in}{3.095112in}}%
\pgfpathlineto{\pgfqpoint{4.906728in}{3.088511in}}%
\pgfpathlineto{\pgfqpoint{4.913151in}{3.085744in}}%
\pgfpathlineto{\pgfqpoint{4.919038in}{3.085535in}}%
\pgfpathlineto{\pgfqpoint{4.924925in}{3.087566in}}%
\pgfpathlineto{\pgfqpoint{4.931348in}{3.092200in}}%
\pgfpathlineto{\pgfqpoint{4.939109in}{3.100717in}}%
\pgfpathlineto{\pgfqpoint{4.949813in}{3.116179in}}%
\pgfpathlineto{\pgfqpoint{4.974433in}{3.152892in}}%
\pgfpathlineto{\pgfqpoint{4.982729in}{3.160795in}}%
\pgfpathlineto{\pgfqpoint{4.989419in}{3.164303in}}%
\pgfpathlineto{\pgfqpoint{4.995306in}{3.165037in}}%
\pgfpathlineto{\pgfqpoint{5.001194in}{3.163523in}}%
\pgfpathlineto{\pgfqpoint{5.007616in}{3.159409in}}%
\pgfpathlineto{\pgfqpoint{5.015109in}{3.151734in}}%
\pgfpathlineto{\pgfqpoint{5.025011in}{3.138029in}}%
\pgfpathlineto{\pgfqpoint{5.054448in}{3.094904in}}%
\pgfpathlineto{\pgfqpoint{5.062208in}{3.088388in}}%
\pgfpathlineto{\pgfqpoint{5.068631in}{3.085700in}}%
\pgfpathlineto{\pgfqpoint{5.074518in}{3.085567in}}%
\pgfpathlineto{\pgfqpoint{5.080405in}{3.087672in}}%
\pgfpathlineto{\pgfqpoint{5.087096in}{3.092626in}}%
\pgfpathlineto{\pgfqpoint{5.095124in}{3.101660in}}%
\pgfpathlineto{\pgfqpoint{5.106096in}{3.117757in}}%
\pgfpathlineto{\pgfqpoint{5.129110in}{3.152181in}}%
\pgfpathlineto{\pgfqpoint{5.137406in}{3.160344in}}%
\pgfpathlineto{\pgfqpoint{5.144096in}{3.164100in}}%
\pgfpathlineto{\pgfqpoint{5.150251in}{3.165056in}}%
\pgfpathlineto{\pgfqpoint{5.156139in}{3.163670in}}%
\pgfpathlineto{\pgfqpoint{5.162561in}{3.159686in}}%
\pgfpathlineto{\pgfqpoint{5.170054in}{3.152139in}}%
\pgfpathlineto{\pgfqpoint{5.179688in}{3.138950in}}%
\pgfpathlineto{\pgfqpoint{5.210463in}{3.094150in}}%
\pgfpathlineto{\pgfqpoint{5.217956in}{3.088107in}}%
\pgfpathlineto{\pgfqpoint{5.224379in}{3.085608in}}%
\pgfpathlineto{\pgfqpoint{5.230266in}{3.085652in}}%
\pgfpathlineto{\pgfqpoint{5.236153in}{3.087931in}}%
\pgfpathlineto{\pgfqpoint{5.242843in}{3.093064in}}%
\pgfpathlineto{\pgfqpoint{5.250872in}{3.102269in}}%
\pgfpathlineto{\pgfqpoint{5.262379in}{3.119347in}}%
\pgfpathlineto{\pgfqpoint{5.283787in}{3.151453in}}%
\pgfpathlineto{\pgfqpoint{5.292351in}{3.160083in}}%
\pgfpathlineto{\pgfqpoint{5.299309in}{3.164075in}}%
\pgfpathlineto{\pgfqpoint{5.305464in}{3.165059in}}%
\pgfpathlineto{\pgfqpoint{5.311351in}{3.163699in}}%
\pgfpathlineto{\pgfqpoint{5.317774in}{3.159742in}}%
\pgfpathlineto{\pgfqpoint{5.325267in}{3.152223in}}%
\pgfpathlineto{\pgfqpoint{5.334901in}{3.139056in}}%
\pgfpathlineto{\pgfqpoint{5.365943in}{3.093952in}}%
\pgfpathlineto{\pgfqpoint{5.373436in}{3.087992in}}%
\pgfpathlineto{\pgfqpoint{5.379859in}{3.085573in}}%
\pgfpathlineto{\pgfqpoint{5.385746in}{3.085693in}}%
\pgfpathlineto{\pgfqpoint{5.391633in}{3.088045in}}%
\pgfpathlineto{\pgfqpoint{5.398324in}{3.093253in}}%
\pgfpathlineto{\pgfqpoint{5.406352in}{3.102530in}}%
\pgfpathlineto{\pgfqpoint{5.417859in}{3.119662in}}%
\pgfpathlineto{\pgfqpoint{5.439000in}{3.151368in}}%
\pgfpathlineto{\pgfqpoint{5.447563in}{3.160028in}}%
\pgfpathlineto{\pgfqpoint{5.454521in}{3.164049in}}%
\pgfpathlineto{\pgfqpoint{5.460676in}{3.165062in}}%
\pgfpathlineto{\pgfqpoint{5.466564in}{3.163729in}}%
\pgfpathlineto{\pgfqpoint{5.472719in}{3.160012in}}%
\pgfpathlineto{\pgfqpoint{5.479944in}{3.152933in}}%
\pgfpathlineto{\pgfqpoint{5.489310in}{3.140368in}}%
\pgfpathlineto{\pgfqpoint{5.522494in}{3.092725in}}%
\pgfpathlineto{\pgfqpoint{5.529719in}{3.087445in}}%
\pgfpathlineto{\pgfqpoint{5.535874in}{3.085465in}}%
\pgfpathlineto{\pgfqpoint{5.541761in}{3.085865in}}%
\pgfpathlineto{\pgfqpoint{5.547916in}{3.088658in}}%
\pgfpathlineto{\pgfqpoint{5.554874in}{3.094517in}}%
\pgfpathlineto{\pgfqpoint{5.563438in}{3.104979in}}%
\pgfpathlineto{\pgfqpoint{5.576550in}{3.125141in}}%
\pgfpathlineto{\pgfqpoint{5.593410in}{3.150289in}}%
\pgfpathlineto{\pgfqpoint{5.602241in}{3.159542in}}%
\pgfpathlineto{\pgfqpoint{5.609199in}{3.163818in}}%
\pgfpathlineto{\pgfqpoint{5.615354in}{3.165072in}}%
\pgfpathlineto{\pgfqpoint{5.621241in}{3.163969in}}%
\pgfpathlineto{\pgfqpoint{5.627396in}{3.160477in}}%
\pgfpathlineto{\pgfqpoint{5.634621in}{3.153626in}}%
\pgfpathlineto{\pgfqpoint{5.643720in}{3.141661in}}%
\pgfpathlineto{\pgfqpoint{5.679044in}{3.091583in}}%
\pgfpathlineto{\pgfqpoint{5.686270in}{3.086840in}}%
\pgfpathlineto{\pgfqpoint{5.692425in}{3.085360in}}%
\pgfpathlineto{\pgfqpoint{5.698312in}{3.086246in}}%
\pgfpathlineto{\pgfqpoint{5.704467in}{3.089522in}}%
\pgfpathlineto{\pgfqpoint{5.711425in}{3.095861in}}%
\pgfpathlineto{\pgfqpoint{5.720256in}{3.107150in}}%
\pgfpathlineto{\pgfqpoint{5.735777in}{3.131476in}}%
\pgfpathlineto{\pgfqpoint{5.749425in}{3.151196in}}%
\pgfpathlineto{\pgfqpoint{5.757988in}{3.159917in}}%
\pgfpathlineto{\pgfqpoint{5.764946in}{3.163998in}}%
\pgfpathlineto{\pgfqpoint{5.771101in}{3.165066in}}%
\pgfpathlineto{\pgfqpoint{5.776989in}{3.163787in}}%
\pgfpathlineto{\pgfqpoint{5.783144in}{3.160122in}}%
\pgfpathlineto{\pgfqpoint{5.790369in}{3.153096in}}%
\pgfpathlineto{\pgfqpoint{5.799735in}{3.140578in}}%
\pgfpathlineto{\pgfqpoint{5.833186in}{3.092607in}}%
\pgfpathlineto{\pgfqpoint{5.840412in}{3.087380in}}%
\pgfpathlineto{\pgfqpoint{5.846567in}{3.085450in}}%
\pgfpathlineto{\pgfqpoint{5.852454in}{3.085899in}}%
\pgfpathlineto{\pgfqpoint{5.858609in}{3.088741in}}%
\pgfpathlineto{\pgfqpoint{5.865567in}{3.094648in}}%
\pgfpathlineto{\pgfqpoint{5.874130in}{3.105155in}}%
\pgfpathlineto{\pgfqpoint{5.887243in}{3.125346in}}%
\pgfpathlineto{\pgfqpoint{5.904102in}{3.150448in}}%
\pgfpathlineto{\pgfqpoint{5.912666in}{3.159426in}}%
\pgfpathlineto{\pgfqpoint{5.919624in}{3.163761in}}%
\pgfpathlineto{\pgfqpoint{5.925779in}{3.165070in}}%
\pgfpathlineto{\pgfqpoint{5.931666in}{3.164021in}}%
\pgfpathlineto{\pgfqpoint{5.937821in}{3.160583in}}%
\pgfpathlineto{\pgfqpoint{5.945046in}{3.153785in}}%
\pgfpathlineto{\pgfqpoint{5.954145in}{3.141868in}}%
\pgfpathlineto{\pgfqpoint{5.973413in}{3.111597in}}%
\pgfpathlineto{\pgfqpoint{5.984652in}{3.096523in}}%
\pgfpathlineto{\pgfqpoint{5.992681in}{3.089190in}}%
\pgfpathlineto{\pgfqpoint{5.999371in}{3.085935in}}%
\pgfpathlineto{\pgfqpoint{6.005258in}{3.085435in}}%
\pgfpathlineto{\pgfqpoint{6.011145in}{3.087181in}}%
\pgfpathlineto{\pgfqpoint{6.017568in}{3.091528in}}%
\pgfpathlineto{\pgfqpoint{6.025061in}{3.099430in}}%
\pgfpathlineto{\pgfqpoint{6.035230in}{3.113742in}}%
\pgfpathlineto{\pgfqpoint{6.062794in}{3.154450in}}%
\pgfpathlineto{\pgfqpoint{6.070822in}{3.161567in}}%
\pgfpathlineto{\pgfqpoint{6.077512in}{3.164614in}}%
\pgfpathlineto{\pgfqpoint{6.083400in}{3.164923in}}%
\pgfpathlineto{\pgfqpoint{6.089287in}{3.162989in}}%
\pgfpathlineto{\pgfqpoint{6.095710in}{3.158452in}}%
\pgfpathlineto{\pgfqpoint{6.103470in}{3.150032in}}%
\pgfpathlineto{\pgfqpoint{6.113907in}{3.135068in}}%
\pgfpathlineto{\pgfqpoint{6.139597in}{3.096904in}}%
\pgfpathlineto{\pgfqpoint{6.147625in}{3.089426in}}%
\pgfpathlineto{\pgfqpoint{6.154316in}{3.086032in}}%
\pgfpathlineto{\pgfqpoint{6.160203in}{3.085402in}}%
\pgfpathlineto{\pgfqpoint{6.166090in}{3.087021in}}%
\pgfpathlineto{\pgfqpoint{6.172513in}{3.091240in}}%
\pgfpathlineto{\pgfqpoint{6.180006in}{3.099017in}}%
\pgfpathlineto{\pgfqpoint{6.189908in}{3.112810in}}%
\pgfpathlineto{\pgfqpoint{6.218809in}{3.155240in}}%
\pgfpathlineto{\pgfqpoint{6.226570in}{3.161868in}}%
\pgfpathlineto{\pgfqpoint{6.232992in}{3.164660in}}%
\pgfpathlineto{\pgfqpoint{6.238880in}{3.164894in}}%
\pgfpathlineto{\pgfqpoint{6.244767in}{3.162886in}}%
\pgfpathlineto{\pgfqpoint{6.251190in}{3.158276in}}%
\pgfpathlineto{\pgfqpoint{6.258950in}{3.149782in}}%
\pgfpathlineto{\pgfqpoint{6.269655in}{3.134340in}}%
\pgfpathlineto{\pgfqpoint{6.294275in}{3.097601in}}%
\pgfpathlineto{\pgfqpoint{6.302570in}{3.089670in}}%
\pgfpathlineto{\pgfqpoint{6.309261in}{3.086136in}}%
\pgfpathlineto{\pgfqpoint{6.315148in}{3.085377in}}%
\pgfpathlineto{\pgfqpoint{6.321035in}{3.086868in}}%
\pgfpathlineto{\pgfqpoint{6.327458in}{3.090958in}}%
\pgfpathlineto{\pgfqpoint{6.334951in}{3.098608in}}%
\pgfpathlineto{\pgfqpoint{6.344852in}{3.112293in}}%
\pgfpathlineto{\pgfqpoint{6.374557in}{3.155727in}}%
\pgfpathlineto{\pgfqpoint{6.382318in}{3.162156in}}%
\pgfpathlineto{\pgfqpoint{6.388740in}{3.164761in}}%
\pgfpathlineto{\pgfqpoint{6.394628in}{3.164816in}}%
\pgfpathlineto{\pgfqpoint{6.400515in}{3.162635in}}%
\pgfpathlineto{\pgfqpoint{6.407205in}{3.157602in}}%
\pgfpathlineto{\pgfqpoint{6.415233in}{3.148493in}}%
\pgfpathlineto{\pgfqpoint{6.426473in}{3.131913in}}%
\pgfpathlineto{\pgfqpoint{6.448684in}{3.098634in}}%
\pgfpathlineto{\pgfqpoint{6.456980in}{3.090330in}}%
\pgfpathlineto{\pgfqpoint{6.463938in}{3.086342in}}%
\pgfpathlineto{\pgfqpoint{6.470093in}{3.085360in}}%
\pgfpathlineto{\pgfqpoint{6.475980in}{3.086722in}}%
\pgfpathlineto{\pgfqpoint{6.482403in}{3.090682in}}%
\pgfpathlineto{\pgfqpoint{6.489896in}{3.098205in}}%
\pgfpathlineto{\pgfqpoint{6.499530in}{3.111373in}}%
\pgfpathlineto{\pgfqpoint{6.530572in}{3.156474in}}%
\pgfpathlineto{\pgfqpoint{6.538065in}{3.162431in}}%
\pgfpathlineto{\pgfqpoint{6.544488in}{3.164847in}}%
\pgfpathlineto{\pgfqpoint{6.550375in}{3.164724in}}%
\pgfpathlineto{\pgfqpoint{6.556263in}{3.162370in}}%
\pgfpathlineto{\pgfqpoint{6.562953in}{3.157159in}}%
\pgfpathlineto{\pgfqpoint{6.570981in}{3.147880in}}%
\pgfpathlineto{\pgfqpoint{6.582488in}{3.130746in}}%
\pgfpathlineto{\pgfqpoint{6.603629in}{3.099042in}}%
\pgfpathlineto{\pgfqpoint{6.612193in}{3.090385in}}%
\pgfpathlineto{\pgfqpoint{6.619150in}{3.086367in}}%
\pgfpathlineto{\pgfqpoint{6.625305in}{3.085357in}}%
\pgfpathlineto{\pgfqpoint{6.631193in}{3.086693in}}%
\pgfpathlineto{\pgfqpoint{6.637348in}{3.090413in}}%
\pgfpathlineto{\pgfqpoint{6.644573in}{3.097494in}}%
\pgfpathlineto{\pgfqpoint{6.653939in}{3.110061in}}%
\pgfpathlineto{\pgfqpoint{6.663306in}{3.124778in}}%
\pgfpathlineto{\pgfqpoint{6.663306in}{3.124778in}}%
\pgfusepath{stroke}%
\end{pgfscope}%
\begin{pgfscope}%
\pgfpathrectangle{\pgfqpoint{0.467797in}{2.292089in}}{\pgfqpoint{6.490533in}{1.666241in}}%
\pgfusepath{clip}%
\pgfsetrectcap%
\pgfsetroundjoin%
\pgfsetlinewidth{1.505625pt}%
\definecolor{currentstroke}{rgb}{0.121569,0.466667,0.705882}%
\pgfsetstrokecolor{currentstroke}%
\pgfsetdash{}{0pt}%
\pgfpathmoveto{\pgfqpoint{0.762821in}{3.125209in}}%
\pgfpathlineto{\pgfqpoint{0.778610in}{3.148612in}}%
\pgfpathlineto{\pgfqpoint{0.786638in}{3.156762in}}%
\pgfpathlineto{\pgfqpoint{0.793061in}{3.160418in}}%
\pgfpathlineto{\pgfqpoint{0.798680in}{3.161256in}}%
\pgfpathlineto{\pgfqpoint{0.804300in}{3.159828in}}%
\pgfpathlineto{\pgfqpoint{0.810455in}{3.155776in}}%
\pgfpathlineto{\pgfqpoint{0.817681in}{3.148115in}}%
\pgfpathlineto{\pgfqpoint{0.827582in}{3.133969in}}%
\pgfpathlineto{\pgfqpoint{0.850864in}{3.099441in}}%
\pgfpathlineto{\pgfqpoint{0.858625in}{3.092395in}}%
\pgfpathlineto{\pgfqpoint{0.864779in}{3.089563in}}%
\pgfpathlineto{\pgfqpoint{0.870399in}{3.089319in}}%
\pgfpathlineto{\pgfqpoint{0.876019in}{3.091331in}}%
\pgfpathlineto{\pgfqpoint{0.882174in}{3.095968in}}%
\pgfpathlineto{\pgfqpoint{0.889667in}{3.104538in}}%
\pgfpathlineto{\pgfqpoint{0.900639in}{3.120903in}}%
\pgfpathlineto{\pgfqpoint{0.919639in}{3.149262in}}%
\pgfpathlineto{\pgfqpoint{0.927667in}{3.157171in}}%
\pgfpathlineto{\pgfqpoint{0.934090in}{3.160595in}}%
\pgfpathlineto{\pgfqpoint{0.939710in}{3.161217in}}%
\pgfpathlineto{\pgfqpoint{0.945329in}{3.159576in}}%
\pgfpathlineto{\pgfqpoint{0.951484in}{3.155309in}}%
\pgfpathlineto{\pgfqpoint{0.958977in}{3.147101in}}%
\pgfpathlineto{\pgfqpoint{0.969414in}{3.131861in}}%
\pgfpathlineto{\pgfqpoint{0.990555in}{3.100364in}}%
\pgfpathlineto{\pgfqpoint{0.998316in}{3.092953in}}%
\pgfpathlineto{\pgfqpoint{1.004738in}{3.089704in}}%
\pgfpathlineto{\pgfqpoint{1.010358in}{3.089244in}}%
\pgfpathlineto{\pgfqpoint{1.015978in}{3.091045in}}%
\pgfpathlineto{\pgfqpoint{1.022133in}{3.095471in}}%
\pgfpathlineto{\pgfqpoint{1.029626in}{3.103836in}}%
\pgfpathlineto{\pgfqpoint{1.040330in}{3.119621in}}%
\pgfpathlineto{\pgfqpoint{1.060401in}{3.149582in}}%
\pgfpathlineto{\pgfqpoint{1.068429in}{3.157369in}}%
\pgfpathlineto{\pgfqpoint{1.074852in}{3.160676in}}%
\pgfpathlineto{\pgfqpoint{1.080471in}{3.161190in}}%
\pgfpathlineto{\pgfqpoint{1.086091in}{3.159443in}}%
\pgfpathlineto{\pgfqpoint{1.092246in}{3.155069in}}%
\pgfpathlineto{\pgfqpoint{1.099739in}{3.146756in}}%
\pgfpathlineto{\pgfqpoint{1.110176in}{3.131437in}}%
\pgfpathlineto{\pgfqpoint{1.130782in}{3.100678in}}%
\pgfpathlineto{\pgfqpoint{1.138810in}{3.092953in}}%
\pgfpathlineto{\pgfqpoint{1.145232in}{3.089704in}}%
\pgfpathlineto{\pgfqpoint{1.150852in}{3.089244in}}%
\pgfpathlineto{\pgfqpoint{1.156472in}{3.091045in}}%
\pgfpathlineto{\pgfqpoint{1.162627in}{3.095471in}}%
\pgfpathlineto{\pgfqpoint{1.170120in}{3.103836in}}%
\pgfpathlineto{\pgfqpoint{1.180824in}{3.119621in}}%
\pgfpathlineto{\pgfqpoint{1.200895in}{3.149582in}}%
\pgfpathlineto{\pgfqpoint{1.208923in}{3.157369in}}%
\pgfpathlineto{\pgfqpoint{1.215346in}{3.160676in}}%
\pgfpathlineto{\pgfqpoint{1.220965in}{3.161190in}}%
\pgfpathlineto{\pgfqpoint{1.226585in}{3.159443in}}%
\pgfpathlineto{\pgfqpoint{1.232740in}{3.155069in}}%
\pgfpathlineto{\pgfqpoint{1.240233in}{3.146756in}}%
\pgfpathlineto{\pgfqpoint{1.250670in}{3.131437in}}%
\pgfpathlineto{\pgfqpoint{1.271276in}{3.100678in}}%
\pgfpathlineto{\pgfqpoint{1.279304in}{3.092953in}}%
\pgfpathlineto{\pgfqpoint{1.285727in}{3.089704in}}%
\pgfpathlineto{\pgfqpoint{1.291346in}{3.089244in}}%
\pgfpathlineto{\pgfqpoint{1.296966in}{3.091045in}}%
\pgfpathlineto{\pgfqpoint{1.303121in}{3.095471in}}%
\pgfpathlineto{\pgfqpoint{1.310614in}{3.103836in}}%
\pgfpathlineto{\pgfqpoint{1.321318in}{3.119621in}}%
\pgfpathlineto{\pgfqpoint{1.341389in}{3.149582in}}%
\pgfpathlineto{\pgfqpoint{1.349417in}{3.157369in}}%
\pgfpathlineto{\pgfqpoint{1.355840in}{3.160676in}}%
\pgfpathlineto{\pgfqpoint{1.361460in}{3.161190in}}%
\pgfpathlineto{\pgfqpoint{1.367079in}{3.159443in}}%
\pgfpathlineto{\pgfqpoint{1.373234in}{3.155069in}}%
\pgfpathlineto{\pgfqpoint{1.380727in}{3.146756in}}%
\pgfpathlineto{\pgfqpoint{1.391164in}{3.131437in}}%
\pgfpathlineto{\pgfqpoint{1.411770in}{3.100678in}}%
\pgfpathlineto{\pgfqpoint{1.419798in}{3.092953in}}%
\pgfpathlineto{\pgfqpoint{1.426221in}{3.089704in}}%
\pgfpathlineto{\pgfqpoint{1.431840in}{3.089244in}}%
\pgfpathlineto{\pgfqpoint{1.437460in}{3.091045in}}%
\pgfpathlineto{\pgfqpoint{1.443615in}{3.095471in}}%
\pgfpathlineto{\pgfqpoint{1.451108in}{3.103836in}}%
\pgfpathlineto{\pgfqpoint{1.461813in}{3.119621in}}%
\pgfpathlineto{\pgfqpoint{1.481883in}{3.149582in}}%
\pgfpathlineto{\pgfqpoint{1.489911in}{3.157369in}}%
\pgfpathlineto{\pgfqpoint{1.496334in}{3.160676in}}%
\pgfpathlineto{\pgfqpoint{1.501954in}{3.161190in}}%
\pgfpathlineto{\pgfqpoint{1.507573in}{3.159443in}}%
\pgfpathlineto{\pgfqpoint{1.513728in}{3.155069in}}%
\pgfpathlineto{\pgfqpoint{1.521221in}{3.146756in}}%
\pgfpathlineto{\pgfqpoint{1.531658in}{3.131437in}}%
\pgfpathlineto{\pgfqpoint{1.552264in}{3.100678in}}%
\pgfpathlineto{\pgfqpoint{1.560292in}{3.092953in}}%
\pgfpathlineto{\pgfqpoint{1.566715in}{3.089704in}}%
\pgfpathlineto{\pgfqpoint{1.572335in}{3.089244in}}%
\pgfpathlineto{\pgfqpoint{1.577954in}{3.091045in}}%
\pgfpathlineto{\pgfqpoint{1.584109in}{3.095471in}}%
\pgfpathlineto{\pgfqpoint{1.591602in}{3.103836in}}%
\pgfpathlineto{\pgfqpoint{1.602307in}{3.119621in}}%
\pgfpathlineto{\pgfqpoint{1.622377in}{3.149582in}}%
\pgfpathlineto{\pgfqpoint{1.630405in}{3.157369in}}%
\pgfpathlineto{\pgfqpoint{1.636828in}{3.160676in}}%
\pgfpathlineto{\pgfqpoint{1.642448in}{3.161190in}}%
\pgfpathlineto{\pgfqpoint{1.648068in}{3.159443in}}%
\pgfpathlineto{\pgfqpoint{1.654223in}{3.155069in}}%
\pgfpathlineto{\pgfqpoint{1.661716in}{3.146756in}}%
\pgfpathlineto{\pgfqpoint{1.672152in}{3.131437in}}%
\pgfpathlineto{\pgfqpoint{1.692758in}{3.100678in}}%
\pgfpathlineto{\pgfqpoint{1.700786in}{3.092953in}}%
\pgfpathlineto{\pgfqpoint{1.707209in}{3.089704in}}%
\pgfpathlineto{\pgfqpoint{1.712829in}{3.089244in}}%
\pgfpathlineto{\pgfqpoint{1.718448in}{3.091045in}}%
\pgfpathlineto{\pgfqpoint{1.724603in}{3.095471in}}%
\pgfpathlineto{\pgfqpoint{1.732096in}{3.103836in}}%
\pgfpathlineto{\pgfqpoint{1.742801in}{3.119621in}}%
\pgfpathlineto{\pgfqpoint{1.762871in}{3.149582in}}%
\pgfpathlineto{\pgfqpoint{1.770900in}{3.157369in}}%
\pgfpathlineto{\pgfqpoint{1.777322in}{3.160676in}}%
\pgfpathlineto{\pgfqpoint{1.782942in}{3.161190in}}%
\pgfpathlineto{\pgfqpoint{1.788562in}{3.159443in}}%
\pgfpathlineto{\pgfqpoint{1.794717in}{3.155069in}}%
\pgfpathlineto{\pgfqpoint{1.802210in}{3.146756in}}%
\pgfpathlineto{\pgfqpoint{1.812646in}{3.131437in}}%
\pgfpathlineto{\pgfqpoint{1.833252in}{3.100678in}}%
\pgfpathlineto{\pgfqpoint{1.841280in}{3.092953in}}%
\pgfpathlineto{\pgfqpoint{1.847703in}{3.089704in}}%
\pgfpathlineto{\pgfqpoint{1.853323in}{3.089244in}}%
\pgfpathlineto{\pgfqpoint{1.858943in}{3.091045in}}%
\pgfpathlineto{\pgfqpoint{1.865098in}{3.095471in}}%
\pgfpathlineto{\pgfqpoint{1.872591in}{3.103836in}}%
\pgfpathlineto{\pgfqpoint{1.883295in}{3.119621in}}%
\pgfpathlineto{\pgfqpoint{1.903365in}{3.149582in}}%
\pgfpathlineto{\pgfqpoint{1.911394in}{3.157369in}}%
\pgfpathlineto{\pgfqpoint{1.917816in}{3.160676in}}%
\pgfpathlineto{\pgfqpoint{1.923436in}{3.161190in}}%
\pgfpathlineto{\pgfqpoint{1.929056in}{3.159443in}}%
\pgfpathlineto{\pgfqpoint{1.935211in}{3.155069in}}%
\pgfpathlineto{\pgfqpoint{1.942704in}{3.146756in}}%
\pgfpathlineto{\pgfqpoint{1.953140in}{3.131437in}}%
\pgfpathlineto{\pgfqpoint{1.973746in}{3.100678in}}%
\pgfpathlineto{\pgfqpoint{1.981775in}{3.092953in}}%
\pgfpathlineto{\pgfqpoint{1.988197in}{3.089704in}}%
\pgfpathlineto{\pgfqpoint{1.993817in}{3.089244in}}%
\pgfpathlineto{\pgfqpoint{1.999437in}{3.091045in}}%
\pgfpathlineto{\pgfqpoint{2.005592in}{3.095471in}}%
\pgfpathlineto{\pgfqpoint{2.013085in}{3.103836in}}%
\pgfpathlineto{\pgfqpoint{2.023789in}{3.119621in}}%
\pgfpathlineto{\pgfqpoint{2.043860in}{3.149582in}}%
\pgfpathlineto{\pgfqpoint{2.051888in}{3.157369in}}%
\pgfpathlineto{\pgfqpoint{2.058310in}{3.160676in}}%
\pgfpathlineto{\pgfqpoint{2.063930in}{3.161190in}}%
\pgfpathlineto{\pgfqpoint{2.069550in}{3.159443in}}%
\pgfpathlineto{\pgfqpoint{2.075705in}{3.155069in}}%
\pgfpathlineto{\pgfqpoint{2.083198in}{3.146756in}}%
\pgfpathlineto{\pgfqpoint{2.093635in}{3.131437in}}%
\pgfpathlineto{\pgfqpoint{2.114240in}{3.100678in}}%
\pgfpathlineto{\pgfqpoint{2.122269in}{3.092953in}}%
\pgfpathlineto{\pgfqpoint{2.128691in}{3.089704in}}%
\pgfpathlineto{\pgfqpoint{2.134311in}{3.089244in}}%
\pgfpathlineto{\pgfqpoint{2.139931in}{3.091045in}}%
\pgfpathlineto{\pgfqpoint{2.146086in}{3.095471in}}%
\pgfpathlineto{\pgfqpoint{2.153579in}{3.103836in}}%
\pgfpathlineto{\pgfqpoint{2.164283in}{3.119621in}}%
\pgfpathlineto{\pgfqpoint{2.184354in}{3.149582in}}%
\pgfpathlineto{\pgfqpoint{2.192382in}{3.157369in}}%
\pgfpathlineto{\pgfqpoint{2.198804in}{3.160676in}}%
\pgfpathlineto{\pgfqpoint{2.204424in}{3.161190in}}%
\pgfpathlineto{\pgfqpoint{2.210044in}{3.159443in}}%
\pgfpathlineto{\pgfqpoint{2.216199in}{3.155069in}}%
\pgfpathlineto{\pgfqpoint{2.223692in}{3.146756in}}%
\pgfpathlineto{\pgfqpoint{2.234129in}{3.131437in}}%
\pgfpathlineto{\pgfqpoint{2.254735in}{3.100678in}}%
\pgfpathlineto{\pgfqpoint{2.262763in}{3.092953in}}%
\pgfpathlineto{\pgfqpoint{2.269185in}{3.089704in}}%
\pgfpathlineto{\pgfqpoint{2.274805in}{3.089244in}}%
\pgfpathlineto{\pgfqpoint{2.280425in}{3.091045in}}%
\pgfpathlineto{\pgfqpoint{2.286580in}{3.095471in}}%
\pgfpathlineto{\pgfqpoint{2.294073in}{3.103836in}}%
\pgfpathlineto{\pgfqpoint{2.304777in}{3.119621in}}%
\pgfpathlineto{\pgfqpoint{2.324848in}{3.149582in}}%
\pgfpathlineto{\pgfqpoint{2.332876in}{3.157369in}}%
\pgfpathlineto{\pgfqpoint{2.339299in}{3.160676in}}%
\pgfpathlineto{\pgfqpoint{2.344918in}{3.161190in}}%
\pgfpathlineto{\pgfqpoint{2.350538in}{3.159443in}}%
\pgfpathlineto{\pgfqpoint{2.356693in}{3.155069in}}%
\pgfpathlineto{\pgfqpoint{2.364186in}{3.146756in}}%
\pgfpathlineto{\pgfqpoint{2.374623in}{3.131437in}}%
\pgfpathlineto{\pgfqpoint{2.395229in}{3.100678in}}%
\pgfpathlineto{\pgfqpoint{2.403257in}{3.092953in}}%
\pgfpathlineto{\pgfqpoint{2.409679in}{3.089704in}}%
\pgfpathlineto{\pgfqpoint{2.415299in}{3.089244in}}%
\pgfpathlineto{\pgfqpoint{2.420919in}{3.091045in}}%
\pgfpathlineto{\pgfqpoint{2.427074in}{3.095471in}}%
\pgfpathlineto{\pgfqpoint{2.434567in}{3.103836in}}%
\pgfpathlineto{\pgfqpoint{2.445271in}{3.119621in}}%
\pgfpathlineto{\pgfqpoint{2.465342in}{3.149582in}}%
\pgfpathlineto{\pgfqpoint{2.473370in}{3.157369in}}%
\pgfpathlineto{\pgfqpoint{2.479793in}{3.160676in}}%
\pgfpathlineto{\pgfqpoint{2.485412in}{3.161190in}}%
\pgfpathlineto{\pgfqpoint{2.491032in}{3.159443in}}%
\pgfpathlineto{\pgfqpoint{2.497187in}{3.155069in}}%
\pgfpathlineto{\pgfqpoint{2.504680in}{3.146756in}}%
\pgfpathlineto{\pgfqpoint{2.515117in}{3.131437in}}%
\pgfpathlineto{\pgfqpoint{2.535723in}{3.100678in}}%
\pgfpathlineto{\pgfqpoint{2.543751in}{3.092953in}}%
\pgfpathlineto{\pgfqpoint{2.550174in}{3.089704in}}%
\pgfpathlineto{\pgfqpoint{2.555793in}{3.089244in}}%
\pgfpathlineto{\pgfqpoint{2.561413in}{3.091045in}}%
\pgfpathlineto{\pgfqpoint{2.567568in}{3.095471in}}%
\pgfpathlineto{\pgfqpoint{2.575061in}{3.103836in}}%
\pgfpathlineto{\pgfqpoint{2.585765in}{3.119621in}}%
\pgfpathlineto{\pgfqpoint{2.605836in}{3.149582in}}%
\pgfpathlineto{\pgfqpoint{2.613864in}{3.157369in}}%
\pgfpathlineto{\pgfqpoint{2.620287in}{3.160676in}}%
\pgfpathlineto{\pgfqpoint{2.625907in}{3.161190in}}%
\pgfpathlineto{\pgfqpoint{2.631526in}{3.159443in}}%
\pgfpathlineto{\pgfqpoint{2.637681in}{3.155069in}}%
\pgfpathlineto{\pgfqpoint{2.645174in}{3.146756in}}%
\pgfpathlineto{\pgfqpoint{2.655611in}{3.131437in}}%
\pgfpathlineto{\pgfqpoint{2.676217in}{3.100678in}}%
\pgfpathlineto{\pgfqpoint{2.684245in}{3.092953in}}%
\pgfpathlineto{\pgfqpoint{2.690668in}{3.089704in}}%
\pgfpathlineto{\pgfqpoint{2.696287in}{3.089244in}}%
\pgfpathlineto{\pgfqpoint{2.701907in}{3.091045in}}%
\pgfpathlineto{\pgfqpoint{2.708062in}{3.095471in}}%
\pgfpathlineto{\pgfqpoint{2.715555in}{3.103836in}}%
\pgfpathlineto{\pgfqpoint{2.726259in}{3.119621in}}%
\pgfpathlineto{\pgfqpoint{2.746330in}{3.149582in}}%
\pgfpathlineto{\pgfqpoint{2.754358in}{3.157369in}}%
\pgfpathlineto{\pgfqpoint{2.760781in}{3.160676in}}%
\pgfpathlineto{\pgfqpoint{2.766401in}{3.161190in}}%
\pgfpathlineto{\pgfqpoint{2.772020in}{3.159443in}}%
\pgfpathlineto{\pgfqpoint{2.778175in}{3.155069in}}%
\pgfpathlineto{\pgfqpoint{2.785668in}{3.146756in}}%
\pgfpathlineto{\pgfqpoint{2.796105in}{3.131437in}}%
\pgfpathlineto{\pgfqpoint{2.816711in}{3.100678in}}%
\pgfpathlineto{\pgfqpoint{2.824739in}{3.092953in}}%
\pgfpathlineto{\pgfqpoint{2.831162in}{3.089704in}}%
\pgfpathlineto{\pgfqpoint{2.836782in}{3.089244in}}%
\pgfpathlineto{\pgfqpoint{2.842401in}{3.091045in}}%
\pgfpathlineto{\pgfqpoint{2.848556in}{3.095471in}}%
\pgfpathlineto{\pgfqpoint{2.856049in}{3.103836in}}%
\pgfpathlineto{\pgfqpoint{2.866754in}{3.119621in}}%
\pgfpathlineto{\pgfqpoint{2.886824in}{3.149582in}}%
\pgfpathlineto{\pgfqpoint{2.894852in}{3.157369in}}%
\pgfpathlineto{\pgfqpoint{2.901275in}{3.160676in}}%
\pgfpathlineto{\pgfqpoint{2.906895in}{3.161190in}}%
\pgfpathlineto{\pgfqpoint{2.912515in}{3.159443in}}%
\pgfpathlineto{\pgfqpoint{2.918670in}{3.155069in}}%
\pgfpathlineto{\pgfqpoint{2.926163in}{3.146756in}}%
\pgfpathlineto{\pgfqpoint{2.936599in}{3.131437in}}%
\pgfpathlineto{\pgfqpoint{2.957205in}{3.100678in}}%
\pgfpathlineto{\pgfqpoint{2.965233in}{3.092953in}}%
\pgfpathlineto{\pgfqpoint{2.971656in}{3.089704in}}%
\pgfpathlineto{\pgfqpoint{2.977276in}{3.089244in}}%
\pgfpathlineto{\pgfqpoint{2.982895in}{3.091045in}}%
\pgfpathlineto{\pgfqpoint{2.989050in}{3.095471in}}%
\pgfpathlineto{\pgfqpoint{2.996543in}{3.103836in}}%
\pgfpathlineto{\pgfqpoint{3.007248in}{3.119621in}}%
\pgfpathlineto{\pgfqpoint{3.027318in}{3.149582in}}%
\pgfpathlineto{\pgfqpoint{3.035347in}{3.157369in}}%
\pgfpathlineto{\pgfqpoint{3.041769in}{3.160676in}}%
\pgfpathlineto{\pgfqpoint{3.047389in}{3.161190in}}%
\pgfpathlineto{\pgfqpoint{3.053009in}{3.159443in}}%
\pgfpathlineto{\pgfqpoint{3.059164in}{3.155069in}}%
\pgfpathlineto{\pgfqpoint{3.066657in}{3.146756in}}%
\pgfpathlineto{\pgfqpoint{3.077093in}{3.131437in}}%
\pgfpathlineto{\pgfqpoint{3.097699in}{3.100678in}}%
\pgfpathlineto{\pgfqpoint{3.105727in}{3.092953in}}%
\pgfpathlineto{\pgfqpoint{3.112150in}{3.089704in}}%
\pgfpathlineto{\pgfqpoint{3.117770in}{3.089244in}}%
\pgfpathlineto{\pgfqpoint{3.123389in}{3.091045in}}%
\pgfpathlineto{\pgfqpoint{3.129544in}{3.095471in}}%
\pgfpathlineto{\pgfqpoint{3.137037in}{3.103836in}}%
\pgfpathlineto{\pgfqpoint{3.147742in}{3.119621in}}%
\pgfpathlineto{\pgfqpoint{3.167812in}{3.149582in}}%
\pgfpathlineto{\pgfqpoint{3.175841in}{3.157369in}}%
\pgfpathlineto{\pgfqpoint{3.182263in}{3.160676in}}%
\pgfpathlineto{\pgfqpoint{3.187883in}{3.161190in}}%
\pgfpathlineto{\pgfqpoint{3.193503in}{3.159443in}}%
\pgfpathlineto{\pgfqpoint{3.199658in}{3.155069in}}%
\pgfpathlineto{\pgfqpoint{3.207151in}{3.146756in}}%
\pgfpathlineto{\pgfqpoint{3.217587in}{3.131437in}}%
\pgfpathlineto{\pgfqpoint{3.238193in}{3.100678in}}%
\pgfpathlineto{\pgfqpoint{3.246221in}{3.092953in}}%
\pgfpathlineto{\pgfqpoint{3.252644in}{3.089704in}}%
\pgfpathlineto{\pgfqpoint{3.258264in}{3.089244in}}%
\pgfpathlineto{\pgfqpoint{3.263884in}{3.091045in}}%
\pgfpathlineto{\pgfqpoint{3.270039in}{3.095471in}}%
\pgfpathlineto{\pgfqpoint{3.277532in}{3.103836in}}%
\pgfpathlineto{\pgfqpoint{3.288236in}{3.119621in}}%
\pgfpathlineto{\pgfqpoint{3.308307in}{3.149582in}}%
\pgfpathlineto{\pgfqpoint{3.316335in}{3.157369in}}%
\pgfpathlineto{\pgfqpoint{3.322757in}{3.160676in}}%
\pgfpathlineto{\pgfqpoint{3.328377in}{3.161190in}}%
\pgfpathlineto{\pgfqpoint{3.333997in}{3.159443in}}%
\pgfpathlineto{\pgfqpoint{3.340152in}{3.155069in}}%
\pgfpathlineto{\pgfqpoint{3.347645in}{3.146756in}}%
\pgfpathlineto{\pgfqpoint{3.358082in}{3.131437in}}%
\pgfpathlineto{\pgfqpoint{3.378687in}{3.100678in}}%
\pgfpathlineto{\pgfqpoint{3.386716in}{3.092953in}}%
\pgfpathlineto{\pgfqpoint{3.393138in}{3.089704in}}%
\pgfpathlineto{\pgfqpoint{3.398758in}{3.089244in}}%
\pgfpathlineto{\pgfqpoint{3.404378in}{3.091045in}}%
\pgfpathlineto{\pgfqpoint{3.410533in}{3.095471in}}%
\pgfpathlineto{\pgfqpoint{3.418026in}{3.103836in}}%
\pgfpathlineto{\pgfqpoint{3.428730in}{3.119621in}}%
\pgfpathlineto{\pgfqpoint{3.448801in}{3.149582in}}%
\pgfpathlineto{\pgfqpoint{3.456829in}{3.157369in}}%
\pgfpathlineto{\pgfqpoint{3.463251in}{3.160676in}}%
\pgfpathlineto{\pgfqpoint{3.468871in}{3.161190in}}%
\pgfpathlineto{\pgfqpoint{3.474491in}{3.159443in}}%
\pgfpathlineto{\pgfqpoint{3.480646in}{3.155069in}}%
\pgfpathlineto{\pgfqpoint{3.488139in}{3.146756in}}%
\pgfpathlineto{\pgfqpoint{3.498576in}{3.131437in}}%
\pgfpathlineto{\pgfqpoint{3.519181in}{3.100678in}}%
\pgfpathlineto{\pgfqpoint{3.527210in}{3.092953in}}%
\pgfpathlineto{\pgfqpoint{3.533632in}{3.089704in}}%
\pgfpathlineto{\pgfqpoint{3.539252in}{3.089244in}}%
\pgfpathlineto{\pgfqpoint{3.544872in}{3.091045in}}%
\pgfpathlineto{\pgfqpoint{3.551027in}{3.095471in}}%
\pgfpathlineto{\pgfqpoint{3.558520in}{3.103836in}}%
\pgfpathlineto{\pgfqpoint{3.569224in}{3.119621in}}%
\pgfpathlineto{\pgfqpoint{3.589295in}{3.149582in}}%
\pgfpathlineto{\pgfqpoint{3.597323in}{3.157369in}}%
\pgfpathlineto{\pgfqpoint{3.603746in}{3.160676in}}%
\pgfpathlineto{\pgfqpoint{3.609365in}{3.161190in}}%
\pgfpathlineto{\pgfqpoint{3.614985in}{3.159443in}}%
\pgfpathlineto{\pgfqpoint{3.621140in}{3.155069in}}%
\pgfpathlineto{\pgfqpoint{3.628633in}{3.146756in}}%
\pgfpathlineto{\pgfqpoint{3.639070in}{3.131437in}}%
\pgfpathlineto{\pgfqpoint{3.659676in}{3.100678in}}%
\pgfpathlineto{\pgfqpoint{3.667704in}{3.092953in}}%
\pgfpathlineto{\pgfqpoint{3.674126in}{3.089704in}}%
\pgfpathlineto{\pgfqpoint{3.679746in}{3.089244in}}%
\pgfpathlineto{\pgfqpoint{3.685366in}{3.091045in}}%
\pgfpathlineto{\pgfqpoint{3.691521in}{3.095471in}}%
\pgfpathlineto{\pgfqpoint{3.699014in}{3.103836in}}%
\pgfpathlineto{\pgfqpoint{3.709718in}{3.119621in}}%
\pgfpathlineto{\pgfqpoint{3.729789in}{3.149582in}}%
\pgfpathlineto{\pgfqpoint{3.737817in}{3.157369in}}%
\pgfpathlineto{\pgfqpoint{3.744240in}{3.160676in}}%
\pgfpathlineto{\pgfqpoint{3.749859in}{3.161190in}}%
\pgfpathlineto{\pgfqpoint{3.755479in}{3.159443in}}%
\pgfpathlineto{\pgfqpoint{3.761634in}{3.155069in}}%
\pgfpathlineto{\pgfqpoint{3.769127in}{3.146756in}}%
\pgfpathlineto{\pgfqpoint{3.779564in}{3.131437in}}%
\pgfpathlineto{\pgfqpoint{3.800170in}{3.100678in}}%
\pgfpathlineto{\pgfqpoint{3.808198in}{3.092953in}}%
\pgfpathlineto{\pgfqpoint{3.814621in}{3.089704in}}%
\pgfpathlineto{\pgfqpoint{3.820240in}{3.089244in}}%
\pgfpathlineto{\pgfqpoint{3.825860in}{3.091045in}}%
\pgfpathlineto{\pgfqpoint{3.832015in}{3.095471in}}%
\pgfpathlineto{\pgfqpoint{3.839508in}{3.103836in}}%
\pgfpathlineto{\pgfqpoint{3.850212in}{3.119621in}}%
\pgfpathlineto{\pgfqpoint{3.870283in}{3.149582in}}%
\pgfpathlineto{\pgfqpoint{3.878311in}{3.157369in}}%
\pgfpathlineto{\pgfqpoint{3.884734in}{3.160676in}}%
\pgfpathlineto{\pgfqpoint{3.890354in}{3.161190in}}%
\pgfpathlineto{\pgfqpoint{3.895973in}{3.159443in}}%
\pgfpathlineto{\pgfqpoint{3.902128in}{3.155069in}}%
\pgfpathlineto{\pgfqpoint{3.909621in}{3.146756in}}%
\pgfpathlineto{\pgfqpoint{3.920058in}{3.131437in}}%
\pgfpathlineto{\pgfqpoint{3.940664in}{3.100678in}}%
\pgfpathlineto{\pgfqpoint{3.948692in}{3.092953in}}%
\pgfpathlineto{\pgfqpoint{3.955115in}{3.089704in}}%
\pgfpathlineto{\pgfqpoint{3.960734in}{3.089244in}}%
\pgfpathlineto{\pgfqpoint{3.966354in}{3.091045in}}%
\pgfpathlineto{\pgfqpoint{3.972509in}{3.095471in}}%
\pgfpathlineto{\pgfqpoint{3.980002in}{3.103836in}}%
\pgfpathlineto{\pgfqpoint{3.990706in}{3.119621in}}%
\pgfpathlineto{\pgfqpoint{4.010777in}{3.149582in}}%
\pgfpathlineto{\pgfqpoint{4.018805in}{3.157369in}}%
\pgfpathlineto{\pgfqpoint{4.025228in}{3.160676in}}%
\pgfpathlineto{\pgfqpoint{4.030848in}{3.161190in}}%
\pgfpathlineto{\pgfqpoint{4.036467in}{3.159443in}}%
\pgfpathlineto{\pgfqpoint{4.042622in}{3.155069in}}%
\pgfpathlineto{\pgfqpoint{4.050115in}{3.146756in}}%
\pgfpathlineto{\pgfqpoint{4.060552in}{3.131437in}}%
\pgfpathlineto{\pgfqpoint{4.081158in}{3.100678in}}%
\pgfpathlineto{\pgfqpoint{4.089186in}{3.092953in}}%
\pgfpathlineto{\pgfqpoint{4.095609in}{3.089704in}}%
\pgfpathlineto{\pgfqpoint{4.101228in}{3.089244in}}%
\pgfpathlineto{\pgfqpoint{4.106848in}{3.091045in}}%
\pgfpathlineto{\pgfqpoint{4.113003in}{3.095471in}}%
\pgfpathlineto{\pgfqpoint{4.120496in}{3.103836in}}%
\pgfpathlineto{\pgfqpoint{4.131201in}{3.119621in}}%
\pgfpathlineto{\pgfqpoint{4.151271in}{3.149582in}}%
\pgfpathlineto{\pgfqpoint{4.159299in}{3.157369in}}%
\pgfpathlineto{\pgfqpoint{4.165722in}{3.160676in}}%
\pgfpathlineto{\pgfqpoint{4.171342in}{3.161190in}}%
\pgfpathlineto{\pgfqpoint{4.176961in}{3.159443in}}%
\pgfpathlineto{\pgfqpoint{4.183116in}{3.155069in}}%
\pgfpathlineto{\pgfqpoint{4.190609in}{3.146756in}}%
\pgfpathlineto{\pgfqpoint{4.201046in}{3.131437in}}%
\pgfpathlineto{\pgfqpoint{4.221652in}{3.100678in}}%
\pgfpathlineto{\pgfqpoint{4.229680in}{3.092953in}}%
\pgfpathlineto{\pgfqpoint{4.236103in}{3.089704in}}%
\pgfpathlineto{\pgfqpoint{4.241723in}{3.089244in}}%
\pgfpathlineto{\pgfqpoint{4.247342in}{3.091045in}}%
\pgfpathlineto{\pgfqpoint{4.253497in}{3.095471in}}%
\pgfpathlineto{\pgfqpoint{4.260990in}{3.103836in}}%
\pgfpathlineto{\pgfqpoint{4.271695in}{3.119621in}}%
\pgfpathlineto{\pgfqpoint{4.291765in}{3.149582in}}%
\pgfpathlineto{\pgfqpoint{4.299793in}{3.157369in}}%
\pgfpathlineto{\pgfqpoint{4.306216in}{3.160676in}}%
\pgfpathlineto{\pgfqpoint{4.311836in}{3.161190in}}%
\pgfpathlineto{\pgfqpoint{4.317456in}{3.159443in}}%
\pgfpathlineto{\pgfqpoint{4.323611in}{3.155069in}}%
\pgfpathlineto{\pgfqpoint{4.331104in}{3.146756in}}%
\pgfpathlineto{\pgfqpoint{4.341540in}{3.131437in}}%
\pgfpathlineto{\pgfqpoint{4.362146in}{3.100678in}}%
\pgfpathlineto{\pgfqpoint{4.370174in}{3.092953in}}%
\pgfpathlineto{\pgfqpoint{4.376597in}{3.089704in}}%
\pgfpathlineto{\pgfqpoint{4.382217in}{3.089244in}}%
\pgfpathlineto{\pgfqpoint{4.387836in}{3.091045in}}%
\pgfpathlineto{\pgfqpoint{4.393991in}{3.095471in}}%
\pgfpathlineto{\pgfqpoint{4.401484in}{3.103836in}}%
\pgfpathlineto{\pgfqpoint{4.412189in}{3.119621in}}%
\pgfpathlineto{\pgfqpoint{4.432259in}{3.149582in}}%
\pgfpathlineto{\pgfqpoint{4.440288in}{3.157369in}}%
\pgfpathlineto{\pgfqpoint{4.446710in}{3.160676in}}%
\pgfpathlineto{\pgfqpoint{4.452330in}{3.161190in}}%
\pgfpathlineto{\pgfqpoint{4.457950in}{3.159443in}}%
\pgfpathlineto{\pgfqpoint{4.464105in}{3.155069in}}%
\pgfpathlineto{\pgfqpoint{4.471598in}{3.146756in}}%
\pgfpathlineto{\pgfqpoint{4.482034in}{3.131437in}}%
\pgfpathlineto{\pgfqpoint{4.502640in}{3.100678in}}%
\pgfpathlineto{\pgfqpoint{4.510668in}{3.092953in}}%
\pgfpathlineto{\pgfqpoint{4.517091in}{3.089704in}}%
\pgfpathlineto{\pgfqpoint{4.522711in}{3.089244in}}%
\pgfpathlineto{\pgfqpoint{4.528331in}{3.091045in}}%
\pgfpathlineto{\pgfqpoint{4.534486in}{3.095471in}}%
\pgfpathlineto{\pgfqpoint{4.541979in}{3.103836in}}%
\pgfpathlineto{\pgfqpoint{4.552683in}{3.119621in}}%
\pgfpathlineto{\pgfqpoint{4.572753in}{3.149582in}}%
\pgfpathlineto{\pgfqpoint{4.580782in}{3.157369in}}%
\pgfpathlineto{\pgfqpoint{4.587204in}{3.160676in}}%
\pgfpathlineto{\pgfqpoint{4.592824in}{3.161190in}}%
\pgfpathlineto{\pgfqpoint{4.598444in}{3.159443in}}%
\pgfpathlineto{\pgfqpoint{4.604599in}{3.155069in}}%
\pgfpathlineto{\pgfqpoint{4.612092in}{3.146756in}}%
\pgfpathlineto{\pgfqpoint{4.622529in}{3.131437in}}%
\pgfpathlineto{\pgfqpoint{4.643134in}{3.100678in}}%
\pgfpathlineto{\pgfqpoint{4.651163in}{3.092953in}}%
\pgfpathlineto{\pgfqpoint{4.657585in}{3.089704in}}%
\pgfpathlineto{\pgfqpoint{4.663205in}{3.089244in}}%
\pgfpathlineto{\pgfqpoint{4.668825in}{3.091045in}}%
\pgfpathlineto{\pgfqpoint{4.674980in}{3.095471in}}%
\pgfpathlineto{\pgfqpoint{4.682473in}{3.103836in}}%
\pgfpathlineto{\pgfqpoint{4.693177in}{3.119621in}}%
\pgfpathlineto{\pgfqpoint{4.713248in}{3.149582in}}%
\pgfpathlineto{\pgfqpoint{4.721276in}{3.157369in}}%
\pgfpathlineto{\pgfqpoint{4.727698in}{3.160676in}}%
\pgfpathlineto{\pgfqpoint{4.733318in}{3.161190in}}%
\pgfpathlineto{\pgfqpoint{4.738938in}{3.159443in}}%
\pgfpathlineto{\pgfqpoint{4.745093in}{3.155069in}}%
\pgfpathlineto{\pgfqpoint{4.752586in}{3.146756in}}%
\pgfpathlineto{\pgfqpoint{4.763023in}{3.131437in}}%
\pgfpathlineto{\pgfqpoint{4.783628in}{3.100678in}}%
\pgfpathlineto{\pgfqpoint{4.791657in}{3.092953in}}%
\pgfpathlineto{\pgfqpoint{4.798079in}{3.089704in}}%
\pgfpathlineto{\pgfqpoint{4.803699in}{3.089244in}}%
\pgfpathlineto{\pgfqpoint{4.809319in}{3.091045in}}%
\pgfpathlineto{\pgfqpoint{4.815474in}{3.095471in}}%
\pgfpathlineto{\pgfqpoint{4.822967in}{3.103836in}}%
\pgfpathlineto{\pgfqpoint{4.833671in}{3.119621in}}%
\pgfpathlineto{\pgfqpoint{4.853742in}{3.149582in}}%
\pgfpathlineto{\pgfqpoint{4.861770in}{3.157369in}}%
\pgfpathlineto{\pgfqpoint{4.868192in}{3.160676in}}%
\pgfpathlineto{\pgfqpoint{4.873812in}{3.161190in}}%
\pgfpathlineto{\pgfqpoint{4.879432in}{3.159443in}}%
\pgfpathlineto{\pgfqpoint{4.885587in}{3.155069in}}%
\pgfpathlineto{\pgfqpoint{4.893080in}{3.146756in}}%
\pgfpathlineto{\pgfqpoint{4.903517in}{3.131437in}}%
\pgfpathlineto{\pgfqpoint{4.924123in}{3.100678in}}%
\pgfpathlineto{\pgfqpoint{4.932151in}{3.092953in}}%
\pgfpathlineto{\pgfqpoint{4.938573in}{3.089704in}}%
\pgfpathlineto{\pgfqpoint{4.944193in}{3.089244in}}%
\pgfpathlineto{\pgfqpoint{4.949813in}{3.091045in}}%
\pgfpathlineto{\pgfqpoint{4.955968in}{3.095471in}}%
\pgfpathlineto{\pgfqpoint{4.963461in}{3.103836in}}%
\pgfpathlineto{\pgfqpoint{4.974165in}{3.119621in}}%
\pgfpathlineto{\pgfqpoint{4.994236in}{3.149582in}}%
\pgfpathlineto{\pgfqpoint{5.002264in}{3.157369in}}%
\pgfpathlineto{\pgfqpoint{5.008687in}{3.160676in}}%
\pgfpathlineto{\pgfqpoint{5.014306in}{3.161190in}}%
\pgfpathlineto{\pgfqpoint{5.019926in}{3.159443in}}%
\pgfpathlineto{\pgfqpoint{5.026081in}{3.155069in}}%
\pgfpathlineto{\pgfqpoint{5.033574in}{3.146756in}}%
\pgfpathlineto{\pgfqpoint{5.044011in}{3.131437in}}%
\pgfpathlineto{\pgfqpoint{5.064617in}{3.100678in}}%
\pgfpathlineto{\pgfqpoint{5.072645in}{3.092953in}}%
\pgfpathlineto{\pgfqpoint{5.079067in}{3.089704in}}%
\pgfpathlineto{\pgfqpoint{5.084687in}{3.089244in}}%
\pgfpathlineto{\pgfqpoint{5.090307in}{3.091045in}}%
\pgfpathlineto{\pgfqpoint{5.096462in}{3.095471in}}%
\pgfpathlineto{\pgfqpoint{5.103955in}{3.103836in}}%
\pgfpathlineto{\pgfqpoint{5.114659in}{3.119621in}}%
\pgfpathlineto{\pgfqpoint{5.134730in}{3.149582in}}%
\pgfpathlineto{\pgfqpoint{5.142758in}{3.157369in}}%
\pgfpathlineto{\pgfqpoint{5.149181in}{3.160676in}}%
\pgfpathlineto{\pgfqpoint{5.154800in}{3.161190in}}%
\pgfpathlineto{\pgfqpoint{5.160420in}{3.159443in}}%
\pgfpathlineto{\pgfqpoint{5.166575in}{3.155069in}}%
\pgfpathlineto{\pgfqpoint{5.174068in}{3.146756in}}%
\pgfpathlineto{\pgfqpoint{5.184505in}{3.131437in}}%
\pgfpathlineto{\pgfqpoint{5.205111in}{3.100678in}}%
\pgfpathlineto{\pgfqpoint{5.213139in}{3.092953in}}%
\pgfpathlineto{\pgfqpoint{5.219562in}{3.089704in}}%
\pgfpathlineto{\pgfqpoint{5.225181in}{3.089244in}}%
\pgfpathlineto{\pgfqpoint{5.230801in}{3.091045in}}%
\pgfpathlineto{\pgfqpoint{5.236956in}{3.095471in}}%
\pgfpathlineto{\pgfqpoint{5.244449in}{3.103836in}}%
\pgfpathlineto{\pgfqpoint{5.255153in}{3.119621in}}%
\pgfpathlineto{\pgfqpoint{5.275224in}{3.149582in}}%
\pgfpathlineto{\pgfqpoint{5.283252in}{3.157369in}}%
\pgfpathlineto{\pgfqpoint{5.289675in}{3.160676in}}%
\pgfpathlineto{\pgfqpoint{5.295295in}{3.161190in}}%
\pgfpathlineto{\pgfqpoint{5.300914in}{3.159443in}}%
\pgfpathlineto{\pgfqpoint{5.307069in}{3.155069in}}%
\pgfpathlineto{\pgfqpoint{5.314562in}{3.146756in}}%
\pgfpathlineto{\pgfqpoint{5.324999in}{3.131437in}}%
\pgfpathlineto{\pgfqpoint{5.345605in}{3.100678in}}%
\pgfpathlineto{\pgfqpoint{5.353633in}{3.092953in}}%
\pgfpathlineto{\pgfqpoint{5.360056in}{3.089704in}}%
\pgfpathlineto{\pgfqpoint{5.365675in}{3.089244in}}%
\pgfpathlineto{\pgfqpoint{5.371295in}{3.091045in}}%
\pgfpathlineto{\pgfqpoint{5.377450in}{3.095471in}}%
\pgfpathlineto{\pgfqpoint{5.384943in}{3.103836in}}%
\pgfpathlineto{\pgfqpoint{5.395648in}{3.119621in}}%
\pgfpathlineto{\pgfqpoint{5.415718in}{3.149582in}}%
\pgfpathlineto{\pgfqpoint{5.423746in}{3.157369in}}%
\pgfpathlineto{\pgfqpoint{5.430169in}{3.160676in}}%
\pgfpathlineto{\pgfqpoint{5.435789in}{3.161190in}}%
\pgfpathlineto{\pgfqpoint{5.441408in}{3.159443in}}%
\pgfpathlineto{\pgfqpoint{5.447563in}{3.155069in}}%
\pgfpathlineto{\pgfqpoint{5.455056in}{3.146756in}}%
\pgfpathlineto{\pgfqpoint{5.465493in}{3.131437in}}%
\pgfpathlineto{\pgfqpoint{5.486099in}{3.100678in}}%
\pgfpathlineto{\pgfqpoint{5.494127in}{3.092953in}}%
\pgfpathlineto{\pgfqpoint{5.500550in}{3.089704in}}%
\pgfpathlineto{\pgfqpoint{5.506170in}{3.089244in}}%
\pgfpathlineto{\pgfqpoint{5.511789in}{3.091045in}}%
\pgfpathlineto{\pgfqpoint{5.517944in}{3.095471in}}%
\pgfpathlineto{\pgfqpoint{5.525437in}{3.103836in}}%
\pgfpathlineto{\pgfqpoint{5.536142in}{3.119621in}}%
\pgfpathlineto{\pgfqpoint{5.556212in}{3.149582in}}%
\pgfpathlineto{\pgfqpoint{5.564240in}{3.157369in}}%
\pgfpathlineto{\pgfqpoint{5.570663in}{3.160676in}}%
\pgfpathlineto{\pgfqpoint{5.576283in}{3.161190in}}%
\pgfpathlineto{\pgfqpoint{5.581903in}{3.159443in}}%
\pgfpathlineto{\pgfqpoint{5.588058in}{3.155069in}}%
\pgfpathlineto{\pgfqpoint{5.595551in}{3.146756in}}%
\pgfpathlineto{\pgfqpoint{5.605987in}{3.131437in}}%
\pgfpathlineto{\pgfqpoint{5.626593in}{3.100678in}}%
\pgfpathlineto{\pgfqpoint{5.634621in}{3.092953in}}%
\pgfpathlineto{\pgfqpoint{5.641044in}{3.089704in}}%
\pgfpathlineto{\pgfqpoint{5.646664in}{3.089244in}}%
\pgfpathlineto{\pgfqpoint{5.652283in}{3.091045in}}%
\pgfpathlineto{\pgfqpoint{5.658438in}{3.095471in}}%
\pgfpathlineto{\pgfqpoint{5.665931in}{3.103836in}}%
\pgfpathlineto{\pgfqpoint{5.676636in}{3.119621in}}%
\pgfpathlineto{\pgfqpoint{5.696706in}{3.149582in}}%
\pgfpathlineto{\pgfqpoint{5.704735in}{3.157369in}}%
\pgfpathlineto{\pgfqpoint{5.711157in}{3.160676in}}%
\pgfpathlineto{\pgfqpoint{5.716777in}{3.161190in}}%
\pgfpathlineto{\pgfqpoint{5.722397in}{3.159443in}}%
\pgfpathlineto{\pgfqpoint{5.728552in}{3.155069in}}%
\pgfpathlineto{\pgfqpoint{5.736045in}{3.146756in}}%
\pgfpathlineto{\pgfqpoint{5.746481in}{3.131437in}}%
\pgfpathlineto{\pgfqpoint{5.767087in}{3.100678in}}%
\pgfpathlineto{\pgfqpoint{5.775115in}{3.092953in}}%
\pgfpathlineto{\pgfqpoint{5.781538in}{3.089704in}}%
\pgfpathlineto{\pgfqpoint{5.787158in}{3.089244in}}%
\pgfpathlineto{\pgfqpoint{5.792778in}{3.091045in}}%
\pgfpathlineto{\pgfqpoint{5.798932in}{3.095471in}}%
\pgfpathlineto{\pgfqpoint{5.806426in}{3.103836in}}%
\pgfpathlineto{\pgfqpoint{5.817130in}{3.119621in}}%
\pgfpathlineto{\pgfqpoint{5.837200in}{3.149582in}}%
\pgfpathlineto{\pgfqpoint{5.845229in}{3.157369in}}%
\pgfpathlineto{\pgfqpoint{5.851651in}{3.160676in}}%
\pgfpathlineto{\pgfqpoint{5.857271in}{3.161190in}}%
\pgfpathlineto{\pgfqpoint{5.862891in}{3.159443in}}%
\pgfpathlineto{\pgfqpoint{5.869046in}{3.155069in}}%
\pgfpathlineto{\pgfqpoint{5.876539in}{3.146756in}}%
\pgfpathlineto{\pgfqpoint{5.886975in}{3.131437in}}%
\pgfpathlineto{\pgfqpoint{5.907581in}{3.100678in}}%
\pgfpathlineto{\pgfqpoint{5.915610in}{3.092953in}}%
\pgfpathlineto{\pgfqpoint{5.922032in}{3.089704in}}%
\pgfpathlineto{\pgfqpoint{5.927652in}{3.089244in}}%
\pgfpathlineto{\pgfqpoint{5.933272in}{3.091045in}}%
\pgfpathlineto{\pgfqpoint{5.939427in}{3.095471in}}%
\pgfpathlineto{\pgfqpoint{5.946920in}{3.103836in}}%
\pgfpathlineto{\pgfqpoint{5.957624in}{3.119621in}}%
\pgfpathlineto{\pgfqpoint{5.977695in}{3.149582in}}%
\pgfpathlineto{\pgfqpoint{5.985723in}{3.157369in}}%
\pgfpathlineto{\pgfqpoint{5.992145in}{3.160676in}}%
\pgfpathlineto{\pgfqpoint{5.997765in}{3.161190in}}%
\pgfpathlineto{\pgfqpoint{6.003385in}{3.159443in}}%
\pgfpathlineto{\pgfqpoint{6.009540in}{3.155069in}}%
\pgfpathlineto{\pgfqpoint{6.017033in}{3.146756in}}%
\pgfpathlineto{\pgfqpoint{6.027470in}{3.131437in}}%
\pgfpathlineto{\pgfqpoint{6.048075in}{3.100678in}}%
\pgfpathlineto{\pgfqpoint{6.056104in}{3.092953in}}%
\pgfpathlineto{\pgfqpoint{6.062526in}{3.089704in}}%
\pgfpathlineto{\pgfqpoint{6.068146in}{3.089244in}}%
\pgfpathlineto{\pgfqpoint{6.073766in}{3.091045in}}%
\pgfpathlineto{\pgfqpoint{6.079921in}{3.095471in}}%
\pgfpathlineto{\pgfqpoint{6.087414in}{3.103836in}}%
\pgfpathlineto{\pgfqpoint{6.098118in}{3.119621in}}%
\pgfpathlineto{\pgfqpoint{6.118189in}{3.149582in}}%
\pgfpathlineto{\pgfqpoint{6.126217in}{3.157369in}}%
\pgfpathlineto{\pgfqpoint{6.132639in}{3.160676in}}%
\pgfpathlineto{\pgfqpoint{6.138259in}{3.161190in}}%
\pgfpathlineto{\pgfqpoint{6.143879in}{3.159443in}}%
\pgfpathlineto{\pgfqpoint{6.150034in}{3.155069in}}%
\pgfpathlineto{\pgfqpoint{6.157527in}{3.146756in}}%
\pgfpathlineto{\pgfqpoint{6.167964in}{3.131437in}}%
\pgfpathlineto{\pgfqpoint{6.188569in}{3.100678in}}%
\pgfpathlineto{\pgfqpoint{6.196598in}{3.092953in}}%
\pgfpathlineto{\pgfqpoint{6.203020in}{3.089704in}}%
\pgfpathlineto{\pgfqpoint{6.208640in}{3.089244in}}%
\pgfpathlineto{\pgfqpoint{6.214260in}{3.091045in}}%
\pgfpathlineto{\pgfqpoint{6.220415in}{3.095471in}}%
\pgfpathlineto{\pgfqpoint{6.227908in}{3.103836in}}%
\pgfpathlineto{\pgfqpoint{6.238612in}{3.119621in}}%
\pgfpathlineto{\pgfqpoint{6.258683in}{3.149582in}}%
\pgfpathlineto{\pgfqpoint{6.266711in}{3.157369in}}%
\pgfpathlineto{\pgfqpoint{6.273134in}{3.160676in}}%
\pgfpathlineto{\pgfqpoint{6.278753in}{3.161190in}}%
\pgfpathlineto{\pgfqpoint{6.284373in}{3.159443in}}%
\pgfpathlineto{\pgfqpoint{6.290528in}{3.155069in}}%
\pgfpathlineto{\pgfqpoint{6.298021in}{3.146756in}}%
\pgfpathlineto{\pgfqpoint{6.308458in}{3.131437in}}%
\pgfpathlineto{\pgfqpoint{6.329064in}{3.100678in}}%
\pgfpathlineto{\pgfqpoint{6.337092in}{3.092953in}}%
\pgfpathlineto{\pgfqpoint{6.343514in}{3.089704in}}%
\pgfpathlineto{\pgfqpoint{6.349134in}{3.089244in}}%
\pgfpathlineto{\pgfqpoint{6.354754in}{3.091045in}}%
\pgfpathlineto{\pgfqpoint{6.360909in}{3.095471in}}%
\pgfpathlineto{\pgfqpoint{6.368402in}{3.103836in}}%
\pgfpathlineto{\pgfqpoint{6.379106in}{3.119621in}}%
\pgfpathlineto{\pgfqpoint{6.399177in}{3.149582in}}%
\pgfpathlineto{\pgfqpoint{6.407205in}{3.157369in}}%
\pgfpathlineto{\pgfqpoint{6.413628in}{3.160676in}}%
\pgfpathlineto{\pgfqpoint{6.419247in}{3.161190in}}%
\pgfpathlineto{\pgfqpoint{6.424867in}{3.159443in}}%
\pgfpathlineto{\pgfqpoint{6.431022in}{3.155069in}}%
\pgfpathlineto{\pgfqpoint{6.438515in}{3.146756in}}%
\pgfpathlineto{\pgfqpoint{6.448952in}{3.131437in}}%
\pgfpathlineto{\pgfqpoint{6.469558in}{3.100678in}}%
\pgfpathlineto{\pgfqpoint{6.477586in}{3.092953in}}%
\pgfpathlineto{\pgfqpoint{6.484009in}{3.089704in}}%
\pgfpathlineto{\pgfqpoint{6.489628in}{3.089244in}}%
\pgfpathlineto{\pgfqpoint{6.495248in}{3.091045in}}%
\pgfpathlineto{\pgfqpoint{6.501403in}{3.095471in}}%
\pgfpathlineto{\pgfqpoint{6.508896in}{3.103836in}}%
\pgfpathlineto{\pgfqpoint{6.519600in}{3.119621in}}%
\pgfpathlineto{\pgfqpoint{6.539671in}{3.149582in}}%
\pgfpathlineto{\pgfqpoint{6.547699in}{3.157369in}}%
\pgfpathlineto{\pgfqpoint{6.554122in}{3.160676in}}%
\pgfpathlineto{\pgfqpoint{6.559742in}{3.161190in}}%
\pgfpathlineto{\pgfqpoint{6.565361in}{3.159443in}}%
\pgfpathlineto{\pgfqpoint{6.571516in}{3.155069in}}%
\pgfpathlineto{\pgfqpoint{6.579009in}{3.146756in}}%
\pgfpathlineto{\pgfqpoint{6.589446in}{3.131437in}}%
\pgfpathlineto{\pgfqpoint{6.610052in}{3.100678in}}%
\pgfpathlineto{\pgfqpoint{6.618080in}{3.092953in}}%
\pgfpathlineto{\pgfqpoint{6.624503in}{3.089704in}}%
\pgfpathlineto{\pgfqpoint{6.630122in}{3.089244in}}%
\pgfpathlineto{\pgfqpoint{6.635742in}{3.091045in}}%
\pgfpathlineto{\pgfqpoint{6.641897in}{3.095471in}}%
\pgfpathlineto{\pgfqpoint{6.649390in}{3.103836in}}%
\pgfpathlineto{\pgfqpoint{6.660094in}{3.119621in}}%
\pgfpathlineto{\pgfqpoint{6.663306in}{3.124778in}}%
\pgfpathlineto{\pgfqpoint{6.663306in}{3.124778in}}%
\pgfusepath{stroke}%
\end{pgfscope}%
\begin{pgfscope}%
\pgfpathrectangle{\pgfqpoint{0.467797in}{2.292089in}}{\pgfqpoint{6.490533in}{1.666241in}}%
\pgfusepath{clip}%
\pgfsetrectcap%
\pgfsetroundjoin%
\pgfsetlinewidth{1.505625pt}%
\definecolor{currentstroke}{rgb}{1.000000,0.498039,0.054902}%
\pgfsetstrokecolor{currentstroke}%
\pgfsetdash{}{0pt}%
\pgfpathmoveto{\pgfqpoint{0.762821in}{3.125209in}}%
\pgfpathlineto{\pgfqpoint{0.777539in}{3.146946in}}%
\pgfpathlineto{\pgfqpoint{0.785300in}{3.154572in}}%
\pgfpathlineto{\pgfqpoint{0.791455in}{3.157674in}}%
\pgfpathlineto{\pgfqpoint{0.796807in}{3.157994in}}%
\pgfpathlineto{\pgfqpoint{0.802159in}{3.156074in}}%
\pgfpathlineto{\pgfqpoint{0.808314in}{3.151273in}}%
\pgfpathlineto{\pgfqpoint{0.815807in}{3.142316in}}%
\pgfpathlineto{\pgfqpoint{0.827850in}{3.123774in}}%
\pgfpathlineto{\pgfqpoint{0.841765in}{3.103368in}}%
\pgfpathlineto{\pgfqpoint{0.849526in}{3.095783in}}%
\pgfpathlineto{\pgfqpoint{0.855681in}{3.092722in}}%
\pgfpathlineto{\pgfqpoint{0.861033in}{3.092438in}}%
\pgfpathlineto{\pgfqpoint{0.866385in}{3.094394in}}%
\pgfpathlineto{\pgfqpoint{0.872540in}{3.099232in}}%
\pgfpathlineto{\pgfqpoint{0.880033in}{3.108223in}}%
\pgfpathlineto{\pgfqpoint{0.892075in}{3.126785in}}%
\pgfpathlineto{\pgfqpoint{0.905991in}{3.147156in}}%
\pgfpathlineto{\pgfqpoint{0.913484in}{3.154504in}}%
\pgfpathlineto{\pgfqpoint{0.919639in}{3.157649in}}%
\pgfpathlineto{\pgfqpoint{0.924991in}{3.158008in}}%
\pgfpathlineto{\pgfqpoint{0.930343in}{3.156126in}}%
\pgfpathlineto{\pgfqpoint{0.936231in}{3.151625in}}%
\pgfpathlineto{\pgfqpoint{0.943724in}{3.142810in}}%
\pgfpathlineto{\pgfqpoint{0.955231in}{3.125219in}}%
\pgfpathlineto{\pgfqpoint{0.969949in}{3.103480in}}%
\pgfpathlineto{\pgfqpoint{0.977710in}{3.095851in}}%
\pgfpathlineto{\pgfqpoint{0.983865in}{3.092746in}}%
\pgfpathlineto{\pgfqpoint{0.989217in}{3.092424in}}%
\pgfpathlineto{\pgfqpoint{0.994569in}{3.094342in}}%
\pgfpathlineto{\pgfqpoint{1.000724in}{3.099140in}}%
\pgfpathlineto{\pgfqpoint{1.008217in}{3.108095in}}%
\pgfpathlineto{\pgfqpoint{1.020260in}{3.126635in}}%
\pgfpathlineto{\pgfqpoint{1.034175in}{3.147044in}}%
\pgfpathlineto{\pgfqpoint{1.041936in}{3.154632in}}%
\pgfpathlineto{\pgfqpoint{1.048091in}{3.157696in}}%
\pgfpathlineto{\pgfqpoint{1.053443in}{3.157981in}}%
\pgfpathlineto{\pgfqpoint{1.058795in}{3.156028in}}%
\pgfpathlineto{\pgfqpoint{1.064950in}{3.151193in}}%
\pgfpathlineto{\pgfqpoint{1.072443in}{3.142204in}}%
\pgfpathlineto{\pgfqpoint{1.084486in}{3.123643in}}%
\pgfpathlineto{\pgfqpoint{1.098401in}{3.103269in}}%
\pgfpathlineto{\pgfqpoint{1.105894in}{3.095919in}}%
\pgfpathlineto{\pgfqpoint{1.112049in}{3.092772in}}%
\pgfpathlineto{\pgfqpoint{1.117401in}{3.092410in}}%
\pgfpathlineto{\pgfqpoint{1.122753in}{3.094290in}}%
\pgfpathlineto{\pgfqpoint{1.128641in}{3.098788in}}%
\pgfpathlineto{\pgfqpoint{1.136134in}{3.107601in}}%
\pgfpathlineto{\pgfqpoint{1.147641in}{3.125191in}}%
\pgfpathlineto{\pgfqpoint{1.162627in}{3.147254in}}%
\pgfpathlineto{\pgfqpoint{1.170120in}{3.154564in}}%
\pgfpathlineto{\pgfqpoint{1.176275in}{3.157671in}}%
\pgfpathlineto{\pgfqpoint{1.181627in}{3.157996in}}%
\pgfpathlineto{\pgfqpoint{1.186979in}{3.156080in}}%
\pgfpathlineto{\pgfqpoint{1.193134in}{3.151285in}}%
\pgfpathlineto{\pgfqpoint{1.200627in}{3.142332in}}%
\pgfpathlineto{\pgfqpoint{1.212670in}{3.123793in}}%
\pgfpathlineto{\pgfqpoint{1.226585in}{3.103382in}}%
\pgfpathlineto{\pgfqpoint{1.234346in}{3.095792in}}%
\pgfpathlineto{\pgfqpoint{1.240501in}{3.092725in}}%
\pgfpathlineto{\pgfqpoint{1.245853in}{3.092436in}}%
\pgfpathlineto{\pgfqpoint{1.251205in}{3.094388in}}%
\pgfpathlineto{\pgfqpoint{1.257360in}{3.099220in}}%
\pgfpathlineto{\pgfqpoint{1.264853in}{3.108207in}}%
\pgfpathlineto{\pgfqpoint{1.276896in}{3.126766in}}%
\pgfpathlineto{\pgfqpoint{1.290811in}{3.147142in}}%
\pgfpathlineto{\pgfqpoint{1.298304in}{3.154495in}}%
\pgfpathlineto{\pgfqpoint{1.304459in}{3.157645in}}%
\pgfpathlineto{\pgfqpoint{1.309811in}{3.158009in}}%
\pgfpathlineto{\pgfqpoint{1.315163in}{3.156132in}}%
\pgfpathlineto{\pgfqpoint{1.321051in}{3.151636in}}%
\pgfpathlineto{\pgfqpoint{1.328544in}{3.142826in}}%
\pgfpathlineto{\pgfqpoint{1.340051in}{3.125238in}}%
\pgfpathlineto{\pgfqpoint{1.355037in}{3.103172in}}%
\pgfpathlineto{\pgfqpoint{1.362530in}{3.095859in}}%
\pgfpathlineto{\pgfqpoint{1.368685in}{3.092750in}}%
\pgfpathlineto{\pgfqpoint{1.374037in}{3.092422in}}%
\pgfpathlineto{\pgfqpoint{1.379389in}{3.094335in}}%
\pgfpathlineto{\pgfqpoint{1.385277in}{3.098867in}}%
\pgfpathlineto{\pgfqpoint{1.392770in}{3.107712in}}%
\pgfpathlineto{\pgfqpoint{1.404277in}{3.125322in}}%
\pgfpathlineto{\pgfqpoint{1.418995in}{3.147030in}}%
\pgfpathlineto{\pgfqpoint{1.426756in}{3.154623in}}%
\pgfpathlineto{\pgfqpoint{1.432911in}{3.157693in}}%
\pgfpathlineto{\pgfqpoint{1.438263in}{3.157983in}}%
\pgfpathlineto{\pgfqpoint{1.443615in}{3.156034in}}%
\pgfpathlineto{\pgfqpoint{1.449770in}{3.151205in}}%
\pgfpathlineto{\pgfqpoint{1.457263in}{3.142220in}}%
\pgfpathlineto{\pgfqpoint{1.469306in}{3.123662in}}%
\pgfpathlineto{\pgfqpoint{1.483221in}{3.103283in}}%
\pgfpathlineto{\pgfqpoint{1.490714in}{3.095928in}}%
\pgfpathlineto{\pgfqpoint{1.496869in}{3.092775in}}%
\pgfpathlineto{\pgfqpoint{1.502221in}{3.092409in}}%
\pgfpathlineto{\pgfqpoint{1.507573in}{3.094283in}}%
\pgfpathlineto{\pgfqpoint{1.513461in}{3.098777in}}%
\pgfpathlineto{\pgfqpoint{1.520954in}{3.107585in}}%
\pgfpathlineto{\pgfqpoint{1.532461in}{3.125172in}}%
\pgfpathlineto{\pgfqpoint{1.547447in}{3.147240in}}%
\pgfpathlineto{\pgfqpoint{1.554940in}{3.154555in}}%
\pgfpathlineto{\pgfqpoint{1.561095in}{3.157668in}}%
\pgfpathlineto{\pgfqpoint{1.566447in}{3.157998in}}%
\pgfpathlineto{\pgfqpoint{1.571799in}{3.156087in}}%
\pgfpathlineto{\pgfqpoint{1.577687in}{3.151558in}}%
\pgfpathlineto{\pgfqpoint{1.585180in}{3.142715in}}%
\pgfpathlineto{\pgfqpoint{1.596687in}{3.125106in}}%
\pgfpathlineto{\pgfqpoint{1.611405in}{3.103396in}}%
\pgfpathlineto{\pgfqpoint{1.619166in}{3.095800in}}%
\pgfpathlineto{\pgfqpoint{1.625321in}{3.092728in}}%
\pgfpathlineto{\pgfqpoint{1.630673in}{3.092435in}}%
\pgfpathlineto{\pgfqpoint{1.636025in}{3.094381in}}%
\pgfpathlineto{\pgfqpoint{1.642180in}{3.099209in}}%
\pgfpathlineto{\pgfqpoint{1.649673in}{3.108191in}}%
\pgfpathlineto{\pgfqpoint{1.661716in}{3.126748in}}%
\pgfpathlineto{\pgfqpoint{1.675631in}{3.147128in}}%
\pgfpathlineto{\pgfqpoint{1.683124in}{3.154487in}}%
\pgfpathlineto{\pgfqpoint{1.689279in}{3.157642in}}%
\pgfpathlineto{\pgfqpoint{1.694631in}{3.158011in}}%
\pgfpathlineto{\pgfqpoint{1.699983in}{3.156139in}}%
\pgfpathlineto{\pgfqpoint{1.705871in}{3.151647in}}%
\pgfpathlineto{\pgfqpoint{1.713364in}{3.142842in}}%
\pgfpathlineto{\pgfqpoint{1.724871in}{3.125256in}}%
\pgfpathlineto{\pgfqpoint{1.739857in}{3.103186in}}%
\pgfpathlineto{\pgfqpoint{1.747350in}{3.095868in}}%
\pgfpathlineto{\pgfqpoint{1.753505in}{3.092753in}}%
\pgfpathlineto{\pgfqpoint{1.758857in}{3.092420in}}%
\pgfpathlineto{\pgfqpoint{1.764209in}{3.094329in}}%
\pgfpathlineto{\pgfqpoint{1.770097in}{3.098856in}}%
\pgfpathlineto{\pgfqpoint{1.777590in}{3.107696in}}%
\pgfpathlineto{\pgfqpoint{1.789097in}{3.125303in}}%
\pgfpathlineto{\pgfqpoint{1.803815in}{3.147016in}}%
\pgfpathlineto{\pgfqpoint{1.811576in}{3.154615in}}%
\pgfpathlineto{\pgfqpoint{1.817731in}{3.157690in}}%
\pgfpathlineto{\pgfqpoint{1.823083in}{3.157985in}}%
\pgfpathlineto{\pgfqpoint{1.828435in}{3.156041in}}%
\pgfpathlineto{\pgfqpoint{1.834590in}{3.151216in}}%
\pgfpathlineto{\pgfqpoint{1.842083in}{3.142236in}}%
\pgfpathlineto{\pgfqpoint{1.854126in}{3.123680in}}%
\pgfpathlineto{\pgfqpoint{1.868041in}{3.103297in}}%
\pgfpathlineto{\pgfqpoint{1.875802in}{3.095741in}}%
\pgfpathlineto{\pgfqpoint{1.881957in}{3.092706in}}%
\pgfpathlineto{\pgfqpoint{1.887309in}{3.092448in}}%
\pgfpathlineto{\pgfqpoint{1.892661in}{3.094428in}}%
\pgfpathlineto{\pgfqpoint{1.898816in}{3.099289in}}%
\pgfpathlineto{\pgfqpoint{1.906309in}{3.108304in}}%
\pgfpathlineto{\pgfqpoint{1.918619in}{3.127310in}}%
\pgfpathlineto{\pgfqpoint{1.932267in}{3.147226in}}%
\pgfpathlineto{\pgfqpoint{1.939760in}{3.154547in}}%
\pgfpathlineto{\pgfqpoint{1.945915in}{3.157665in}}%
\pgfpathlineto{\pgfqpoint{1.951267in}{3.157999in}}%
\pgfpathlineto{\pgfqpoint{1.956619in}{3.156093in}}%
\pgfpathlineto{\pgfqpoint{1.962507in}{3.151569in}}%
\pgfpathlineto{\pgfqpoint{1.970000in}{3.142731in}}%
\pgfpathlineto{\pgfqpoint{1.981507in}{3.125125in}}%
\pgfpathlineto{\pgfqpoint{1.996225in}{3.103410in}}%
\pgfpathlineto{\pgfqpoint{2.003986in}{3.095808in}}%
\pgfpathlineto{\pgfqpoint{2.010141in}{3.092731in}}%
\pgfpathlineto{\pgfqpoint{2.015493in}{3.092433in}}%
\pgfpathlineto{\pgfqpoint{2.020845in}{3.094375in}}%
\pgfpathlineto{\pgfqpoint{2.027000in}{3.099197in}}%
\pgfpathlineto{\pgfqpoint{2.034493in}{3.108175in}}%
\pgfpathlineto{\pgfqpoint{2.046536in}{3.126729in}}%
\pgfpathlineto{\pgfqpoint{2.060451in}{3.147114in}}%
\pgfpathlineto{\pgfqpoint{2.068212in}{3.154674in}}%
\pgfpathlineto{\pgfqpoint{2.074367in}{3.157711in}}%
\pgfpathlineto{\pgfqpoint{2.079719in}{3.157972in}}%
\pgfpathlineto{\pgfqpoint{2.085071in}{3.155995in}}%
\pgfpathlineto{\pgfqpoint{2.091226in}{3.151135in}}%
\pgfpathlineto{\pgfqpoint{2.098719in}{3.142123in}}%
\pgfpathlineto{\pgfqpoint{2.110762in}{3.123549in}}%
\pgfpathlineto{\pgfqpoint{2.124677in}{3.103200in}}%
\pgfpathlineto{\pgfqpoint{2.132170in}{3.095876in}}%
\pgfpathlineto{\pgfqpoint{2.138325in}{3.092756in}}%
\pgfpathlineto{\pgfqpoint{2.143677in}{3.092419in}}%
\pgfpathlineto{\pgfqpoint{2.149029in}{3.094322in}}%
\pgfpathlineto{\pgfqpoint{2.154917in}{3.098844in}}%
\pgfpathlineto{\pgfqpoint{2.162410in}{3.107680in}}%
\pgfpathlineto{\pgfqpoint{2.173917in}{3.125284in}}%
\pgfpathlineto{\pgfqpoint{2.188635in}{3.147002in}}%
\pgfpathlineto{\pgfqpoint{2.196396in}{3.154606in}}%
\pgfpathlineto{\pgfqpoint{2.202551in}{3.157687in}}%
\pgfpathlineto{\pgfqpoint{2.207903in}{3.157987in}}%
\pgfpathlineto{\pgfqpoint{2.213255in}{3.156048in}}%
\pgfpathlineto{\pgfqpoint{2.219410in}{3.151228in}}%
\pgfpathlineto{\pgfqpoint{2.226903in}{3.142252in}}%
\pgfpathlineto{\pgfqpoint{2.238946in}{3.123699in}}%
\pgfpathlineto{\pgfqpoint{2.252861in}{3.103312in}}%
\pgfpathlineto{\pgfqpoint{2.260622in}{3.095749in}}%
\pgfpathlineto{\pgfqpoint{2.266777in}{3.092709in}}%
\pgfpathlineto{\pgfqpoint{2.272129in}{3.092446in}}%
\pgfpathlineto{\pgfqpoint{2.277481in}{3.094421in}}%
\pgfpathlineto{\pgfqpoint{2.283636in}{3.099278in}}%
\pgfpathlineto{\pgfqpoint{2.291129in}{3.108288in}}%
\pgfpathlineto{\pgfqpoint{2.303172in}{3.126860in}}%
\pgfpathlineto{\pgfqpoint{2.317087in}{3.147212in}}%
\pgfpathlineto{\pgfqpoint{2.324580in}{3.154538in}}%
\pgfpathlineto{\pgfqpoint{2.330735in}{3.157661in}}%
\pgfpathlineto{\pgfqpoint{2.336087in}{3.158001in}}%
\pgfpathlineto{\pgfqpoint{2.341439in}{3.156100in}}%
\pgfpathlineto{\pgfqpoint{2.347327in}{3.151580in}}%
\pgfpathlineto{\pgfqpoint{2.354820in}{3.142747in}}%
\pgfpathlineto{\pgfqpoint{2.366327in}{3.125144in}}%
\pgfpathlineto{\pgfqpoint{2.381045in}{3.103424in}}%
\pgfpathlineto{\pgfqpoint{2.388806in}{3.095817in}}%
\pgfpathlineto{\pgfqpoint{2.394961in}{3.092734in}}%
\pgfpathlineto{\pgfqpoint{2.400313in}{3.092431in}}%
\pgfpathlineto{\pgfqpoint{2.405665in}{3.094368in}}%
\pgfpathlineto{\pgfqpoint{2.411820in}{3.099186in}}%
\pgfpathlineto{\pgfqpoint{2.419313in}{3.108159in}}%
\pgfpathlineto{\pgfqpoint{2.431356in}{3.126710in}}%
\pgfpathlineto{\pgfqpoint{2.445271in}{3.147100in}}%
\pgfpathlineto{\pgfqpoint{2.453032in}{3.154665in}}%
\pgfpathlineto{\pgfqpoint{2.459187in}{3.157708in}}%
\pgfpathlineto{\pgfqpoint{2.464539in}{3.157974in}}%
\pgfpathlineto{\pgfqpoint{2.469891in}{3.156001in}}%
\pgfpathlineto{\pgfqpoint{2.476046in}{3.151147in}}%
\pgfpathlineto{\pgfqpoint{2.483539in}{3.142139in}}%
\pgfpathlineto{\pgfqpoint{2.495582in}{3.123568in}}%
\pgfpathlineto{\pgfqpoint{2.509497in}{3.103214in}}%
\pgfpathlineto{\pgfqpoint{2.516990in}{3.095885in}}%
\pgfpathlineto{\pgfqpoint{2.523145in}{3.092759in}}%
\pgfpathlineto{\pgfqpoint{2.528497in}{3.092417in}}%
\pgfpathlineto{\pgfqpoint{2.533849in}{3.094316in}}%
\pgfpathlineto{\pgfqpoint{2.539737in}{3.098833in}}%
\pgfpathlineto{\pgfqpoint{2.547230in}{3.107664in}}%
\pgfpathlineto{\pgfqpoint{2.558737in}{3.125266in}}%
\pgfpathlineto{\pgfqpoint{2.573455in}{3.146988in}}%
\pgfpathlineto{\pgfqpoint{2.581216in}{3.154598in}}%
\pgfpathlineto{\pgfqpoint{2.587371in}{3.157683in}}%
\pgfpathlineto{\pgfqpoint{2.592723in}{3.157989in}}%
\pgfpathlineto{\pgfqpoint{2.598075in}{3.156054in}}%
\pgfpathlineto{\pgfqpoint{2.604230in}{3.151239in}}%
\pgfpathlineto{\pgfqpoint{2.611723in}{3.142268in}}%
\pgfpathlineto{\pgfqpoint{2.623766in}{3.123718in}}%
\pgfpathlineto{\pgfqpoint{2.637681in}{3.103326in}}%
\pgfpathlineto{\pgfqpoint{2.645442in}{3.095758in}}%
\pgfpathlineto{\pgfqpoint{2.651597in}{3.092712in}}%
\pgfpathlineto{\pgfqpoint{2.656949in}{3.092444in}}%
\pgfpathlineto{\pgfqpoint{2.662301in}{3.094414in}}%
\pgfpathlineto{\pgfqpoint{2.668456in}{3.099266in}}%
\pgfpathlineto{\pgfqpoint{2.675949in}{3.108272in}}%
\pgfpathlineto{\pgfqpoint{2.687992in}{3.126841in}}%
\pgfpathlineto{\pgfqpoint{2.701907in}{3.147198in}}%
\pgfpathlineto{\pgfqpoint{2.709400in}{3.154530in}}%
\pgfpathlineto{\pgfqpoint{2.715555in}{3.157658in}}%
\pgfpathlineto{\pgfqpoint{2.720907in}{3.158003in}}%
\pgfpathlineto{\pgfqpoint{2.726259in}{3.156106in}}%
\pgfpathlineto{\pgfqpoint{2.732147in}{3.151591in}}%
\pgfpathlineto{\pgfqpoint{2.739640in}{3.142763in}}%
\pgfpathlineto{\pgfqpoint{2.751147in}{3.125163in}}%
\pgfpathlineto{\pgfqpoint{2.765865in}{3.103438in}}%
\pgfpathlineto{\pgfqpoint{2.773626in}{3.095825in}}%
\pgfpathlineto{\pgfqpoint{2.779781in}{3.092737in}}%
\pgfpathlineto{\pgfqpoint{2.785133in}{3.092429in}}%
\pgfpathlineto{\pgfqpoint{2.790485in}{3.094361in}}%
\pgfpathlineto{\pgfqpoint{2.796640in}{3.099174in}}%
\pgfpathlineto{\pgfqpoint{2.804133in}{3.108143in}}%
\pgfpathlineto{\pgfqpoint{2.816176in}{3.126692in}}%
\pgfpathlineto{\pgfqpoint{2.830091in}{3.147086in}}%
\pgfpathlineto{\pgfqpoint{2.837852in}{3.154657in}}%
\pgfpathlineto{\pgfqpoint{2.844007in}{3.157705in}}%
\pgfpathlineto{\pgfqpoint{2.849359in}{3.157976in}}%
\pgfpathlineto{\pgfqpoint{2.854711in}{3.156008in}}%
\pgfpathlineto{\pgfqpoint{2.860866in}{3.151158in}}%
\pgfpathlineto{\pgfqpoint{2.868359in}{3.142155in}}%
\pgfpathlineto{\pgfqpoint{2.880402in}{3.123587in}}%
\pgfpathlineto{\pgfqpoint{2.894317in}{3.103228in}}%
\pgfpathlineto{\pgfqpoint{2.901810in}{3.095893in}}%
\pgfpathlineto{\pgfqpoint{2.907965in}{3.092762in}}%
\pgfpathlineto{\pgfqpoint{2.913317in}{3.092415in}}%
\pgfpathlineto{\pgfqpoint{2.918670in}{3.094309in}}%
\pgfpathlineto{\pgfqpoint{2.924557in}{3.098822in}}%
\pgfpathlineto{\pgfqpoint{2.932050in}{3.107648in}}%
\pgfpathlineto{\pgfqpoint{2.943557in}{3.125247in}}%
\pgfpathlineto{\pgfqpoint{2.958275in}{3.146974in}}%
\pgfpathlineto{\pgfqpoint{2.966036in}{3.154589in}}%
\pgfpathlineto{\pgfqpoint{2.972191in}{3.157680in}}%
\pgfpathlineto{\pgfqpoint{2.977543in}{3.157991in}}%
\pgfpathlineto{\pgfqpoint{2.982895in}{3.156061in}}%
\pgfpathlineto{\pgfqpoint{2.989050in}{3.151251in}}%
\pgfpathlineto{\pgfqpoint{2.996543in}{3.142284in}}%
\pgfpathlineto{\pgfqpoint{3.008586in}{3.123737in}}%
\pgfpathlineto{\pgfqpoint{3.022501in}{3.103340in}}%
\pgfpathlineto{\pgfqpoint{3.030262in}{3.095766in}}%
\pgfpathlineto{\pgfqpoint{3.036417in}{3.092715in}}%
\pgfpathlineto{\pgfqpoint{3.041769in}{3.092442in}}%
\pgfpathlineto{\pgfqpoint{3.047121in}{3.094408in}}%
\pgfpathlineto{\pgfqpoint{3.053276in}{3.099255in}}%
\pgfpathlineto{\pgfqpoint{3.060769in}{3.108255in}}%
\pgfpathlineto{\pgfqpoint{3.072812in}{3.126823in}}%
\pgfpathlineto{\pgfqpoint{3.086727in}{3.147184in}}%
\pgfpathlineto{\pgfqpoint{3.094220in}{3.154521in}}%
\pgfpathlineto{\pgfqpoint{3.100375in}{3.157655in}}%
\pgfpathlineto{\pgfqpoint{3.105727in}{3.158004in}}%
\pgfpathlineto{\pgfqpoint{3.111080in}{3.156113in}}%
\pgfpathlineto{\pgfqpoint{3.116967in}{3.151603in}}%
\pgfpathlineto{\pgfqpoint{3.124460in}{3.142779in}}%
\pgfpathlineto{\pgfqpoint{3.135967in}{3.125181in}}%
\pgfpathlineto{\pgfqpoint{3.150685in}{3.103452in}}%
\pgfpathlineto{\pgfqpoint{3.158446in}{3.095834in}}%
\pgfpathlineto{\pgfqpoint{3.164601in}{3.092740in}}%
\pgfpathlineto{\pgfqpoint{3.169953in}{3.092427in}}%
\pgfpathlineto{\pgfqpoint{3.175305in}{3.094355in}}%
\pgfpathlineto{\pgfqpoint{3.181460in}{3.099163in}}%
\pgfpathlineto{\pgfqpoint{3.188953in}{3.108127in}}%
\pgfpathlineto{\pgfqpoint{3.200996in}{3.126673in}}%
\pgfpathlineto{\pgfqpoint{3.214911in}{3.147072in}}%
\pgfpathlineto{\pgfqpoint{3.222672in}{3.154648in}}%
\pgfpathlineto{\pgfqpoint{3.228827in}{3.157702in}}%
\pgfpathlineto{\pgfqpoint{3.234179in}{3.157978in}}%
\pgfpathlineto{\pgfqpoint{3.239531in}{3.156015in}}%
\pgfpathlineto{\pgfqpoint{3.245686in}{3.151170in}}%
\pgfpathlineto{\pgfqpoint{3.253179in}{3.142171in}}%
\pgfpathlineto{\pgfqpoint{3.265222in}{3.123606in}}%
\pgfpathlineto{\pgfqpoint{3.279137in}{3.103242in}}%
\pgfpathlineto{\pgfqpoint{3.286630in}{3.095902in}}%
\pgfpathlineto{\pgfqpoint{3.292785in}{3.092765in}}%
\pgfpathlineto{\pgfqpoint{3.298137in}{3.092414in}}%
\pgfpathlineto{\pgfqpoint{3.303490in}{3.094303in}}%
\pgfpathlineto{\pgfqpoint{3.309377in}{3.098811in}}%
\pgfpathlineto{\pgfqpoint{3.316870in}{3.107632in}}%
\pgfpathlineto{\pgfqpoint{3.328377in}{3.125228in}}%
\pgfpathlineto{\pgfqpoint{3.343096in}{3.146960in}}%
\pgfpathlineto{\pgfqpoint{3.350856in}{3.154581in}}%
\pgfpathlineto{\pgfqpoint{3.357011in}{3.157677in}}%
\pgfpathlineto{\pgfqpoint{3.362363in}{3.157992in}}%
\pgfpathlineto{\pgfqpoint{3.367715in}{3.156067in}}%
\pgfpathlineto{\pgfqpoint{3.373870in}{3.151262in}}%
\pgfpathlineto{\pgfqpoint{3.381363in}{3.142300in}}%
\pgfpathlineto{\pgfqpoint{3.393406in}{3.123755in}}%
\pgfpathlineto{\pgfqpoint{3.407321in}{3.103354in}}%
\pgfpathlineto{\pgfqpoint{3.415082in}{3.095775in}}%
\pgfpathlineto{\pgfqpoint{3.421237in}{3.092718in}}%
\pgfpathlineto{\pgfqpoint{3.426589in}{3.092440in}}%
\pgfpathlineto{\pgfqpoint{3.431941in}{3.094401in}}%
\pgfpathlineto{\pgfqpoint{3.438096in}{3.099243in}}%
\pgfpathlineto{\pgfqpoint{3.445589in}{3.108239in}}%
\pgfpathlineto{\pgfqpoint{3.457632in}{3.126804in}}%
\pgfpathlineto{\pgfqpoint{3.471547in}{3.147170in}}%
\pgfpathlineto{\pgfqpoint{3.479040in}{3.154513in}}%
\pgfpathlineto{\pgfqpoint{3.485195in}{3.157652in}}%
\pgfpathlineto{\pgfqpoint{3.490547in}{3.158006in}}%
\pgfpathlineto{\pgfqpoint{3.495900in}{3.156119in}}%
\pgfpathlineto{\pgfqpoint{3.501787in}{3.151614in}}%
\pgfpathlineto{\pgfqpoint{3.509280in}{3.142795in}}%
\pgfpathlineto{\pgfqpoint{3.520787in}{3.125200in}}%
\pgfpathlineto{\pgfqpoint{3.535506in}{3.103466in}}%
\pgfpathlineto{\pgfqpoint{3.543266in}{3.095842in}}%
\pgfpathlineto{\pgfqpoint{3.549421in}{3.092743in}}%
\pgfpathlineto{\pgfqpoint{3.554773in}{3.092426in}}%
\pgfpathlineto{\pgfqpoint{3.560125in}{3.094348in}}%
\pgfpathlineto{\pgfqpoint{3.566280in}{3.099151in}}%
\pgfpathlineto{\pgfqpoint{3.573773in}{3.108111in}}%
\pgfpathlineto{\pgfqpoint{3.585816in}{3.126654in}}%
\pgfpathlineto{\pgfqpoint{3.599731in}{3.147058in}}%
\pgfpathlineto{\pgfqpoint{3.607492in}{3.154640in}}%
\pgfpathlineto{\pgfqpoint{3.613647in}{3.157699in}}%
\pgfpathlineto{\pgfqpoint{3.618999in}{3.157980in}}%
\pgfpathlineto{\pgfqpoint{3.624351in}{3.156021in}}%
\pgfpathlineto{\pgfqpoint{3.630506in}{3.151181in}}%
\pgfpathlineto{\pgfqpoint{3.637999in}{3.142188in}}%
\pgfpathlineto{\pgfqpoint{3.650042in}{3.123624in}}%
\pgfpathlineto{\pgfqpoint{3.663957in}{3.103255in}}%
\pgfpathlineto{\pgfqpoint{3.671450in}{3.095911in}}%
\pgfpathlineto{\pgfqpoint{3.677605in}{3.092769in}}%
\pgfpathlineto{\pgfqpoint{3.682957in}{3.092412in}}%
\pgfpathlineto{\pgfqpoint{3.688310in}{3.094296in}}%
\pgfpathlineto{\pgfqpoint{3.694197in}{3.098800in}}%
\pgfpathlineto{\pgfqpoint{3.701690in}{3.107616in}}%
\pgfpathlineto{\pgfqpoint{3.713197in}{3.125209in}}%
\pgfpathlineto{\pgfqpoint{3.727916in}{3.146946in}}%
\pgfpathlineto{\pgfqpoint{3.735676in}{3.154572in}}%
\pgfpathlineto{\pgfqpoint{3.741831in}{3.157674in}}%
\pgfpathlineto{\pgfqpoint{3.747183in}{3.157994in}}%
\pgfpathlineto{\pgfqpoint{3.752535in}{3.156074in}}%
\pgfpathlineto{\pgfqpoint{3.758690in}{3.151273in}}%
\pgfpathlineto{\pgfqpoint{3.766183in}{3.142316in}}%
\pgfpathlineto{\pgfqpoint{3.778226in}{3.123774in}}%
\pgfpathlineto{\pgfqpoint{3.792141in}{3.103368in}}%
\pgfpathlineto{\pgfqpoint{3.799902in}{3.095783in}}%
\pgfpathlineto{\pgfqpoint{3.806057in}{3.092722in}}%
\pgfpathlineto{\pgfqpoint{3.811409in}{3.092438in}}%
\pgfpathlineto{\pgfqpoint{3.816761in}{3.094394in}}%
\pgfpathlineto{\pgfqpoint{3.822916in}{3.099232in}}%
\pgfpathlineto{\pgfqpoint{3.830409in}{3.108223in}}%
\pgfpathlineto{\pgfqpoint{3.842452in}{3.126785in}}%
\pgfpathlineto{\pgfqpoint{3.856367in}{3.147156in}}%
\pgfpathlineto{\pgfqpoint{3.863860in}{3.154504in}}%
\pgfpathlineto{\pgfqpoint{3.870015in}{3.157649in}}%
\pgfpathlineto{\pgfqpoint{3.875367in}{3.158008in}}%
\pgfpathlineto{\pgfqpoint{3.880720in}{3.156126in}}%
\pgfpathlineto{\pgfqpoint{3.886607in}{3.151625in}}%
\pgfpathlineto{\pgfqpoint{3.894100in}{3.142810in}}%
\pgfpathlineto{\pgfqpoint{3.905607in}{3.125219in}}%
\pgfpathlineto{\pgfqpoint{3.920326in}{3.103480in}}%
\pgfpathlineto{\pgfqpoint{3.928086in}{3.095851in}}%
\pgfpathlineto{\pgfqpoint{3.934241in}{3.092746in}}%
\pgfpathlineto{\pgfqpoint{3.939593in}{3.092424in}}%
\pgfpathlineto{\pgfqpoint{3.944946in}{3.094342in}}%
\pgfpathlineto{\pgfqpoint{3.951100in}{3.099140in}}%
\pgfpathlineto{\pgfqpoint{3.958594in}{3.108095in}}%
\pgfpathlineto{\pgfqpoint{3.970636in}{3.126635in}}%
\pgfpathlineto{\pgfqpoint{3.984551in}{3.147044in}}%
\pgfpathlineto{\pgfqpoint{3.992312in}{3.154632in}}%
\pgfpathlineto{\pgfqpoint{3.998467in}{3.157696in}}%
\pgfpathlineto{\pgfqpoint{4.003819in}{3.157981in}}%
\pgfpathlineto{\pgfqpoint{4.009171in}{3.156028in}}%
\pgfpathlineto{\pgfqpoint{4.015326in}{3.151193in}}%
\pgfpathlineto{\pgfqpoint{4.022819in}{3.142204in}}%
\pgfpathlineto{\pgfqpoint{4.034862in}{3.123643in}}%
\pgfpathlineto{\pgfqpoint{4.048777in}{3.103269in}}%
\pgfpathlineto{\pgfqpoint{4.056270in}{3.095919in}}%
\pgfpathlineto{\pgfqpoint{4.062425in}{3.092772in}}%
\pgfpathlineto{\pgfqpoint{4.067778in}{3.092410in}}%
\pgfpathlineto{\pgfqpoint{4.073130in}{3.094290in}}%
\pgfpathlineto{\pgfqpoint{4.079017in}{3.098788in}}%
\pgfpathlineto{\pgfqpoint{4.086510in}{3.107601in}}%
\pgfpathlineto{\pgfqpoint{4.098017in}{3.125191in}}%
\pgfpathlineto{\pgfqpoint{4.113003in}{3.147254in}}%
\pgfpathlineto{\pgfqpoint{4.120496in}{3.154564in}}%
\pgfpathlineto{\pgfqpoint{4.126651in}{3.157671in}}%
\pgfpathlineto{\pgfqpoint{4.132003in}{3.157996in}}%
\pgfpathlineto{\pgfqpoint{4.137356in}{3.156080in}}%
\pgfpathlineto{\pgfqpoint{4.143511in}{3.151285in}}%
\pgfpathlineto{\pgfqpoint{4.151004in}{3.142332in}}%
\pgfpathlineto{\pgfqpoint{4.163046in}{3.123793in}}%
\pgfpathlineto{\pgfqpoint{4.176961in}{3.103382in}}%
\pgfpathlineto{\pgfqpoint{4.184722in}{3.095792in}}%
\pgfpathlineto{\pgfqpoint{4.190877in}{3.092725in}}%
\pgfpathlineto{\pgfqpoint{4.196229in}{3.092436in}}%
\pgfpathlineto{\pgfqpoint{4.201581in}{3.094388in}}%
\pgfpathlineto{\pgfqpoint{4.207736in}{3.099220in}}%
\pgfpathlineto{\pgfqpoint{4.215229in}{3.108207in}}%
\pgfpathlineto{\pgfqpoint{4.227272in}{3.126766in}}%
\pgfpathlineto{\pgfqpoint{4.241187in}{3.147142in}}%
\pgfpathlineto{\pgfqpoint{4.248680in}{3.154495in}}%
\pgfpathlineto{\pgfqpoint{4.254835in}{3.157645in}}%
\pgfpathlineto{\pgfqpoint{4.260188in}{3.158009in}}%
\pgfpathlineto{\pgfqpoint{4.265540in}{3.156132in}}%
\pgfpathlineto{\pgfqpoint{4.271427in}{3.151636in}}%
\pgfpathlineto{\pgfqpoint{4.278920in}{3.142826in}}%
\pgfpathlineto{\pgfqpoint{4.290427in}{3.125238in}}%
\pgfpathlineto{\pgfqpoint{4.305413in}{3.103172in}}%
\pgfpathlineto{\pgfqpoint{4.312906in}{3.095859in}}%
\pgfpathlineto{\pgfqpoint{4.319061in}{3.092750in}}%
\pgfpathlineto{\pgfqpoint{4.324413in}{3.092422in}}%
\pgfpathlineto{\pgfqpoint{4.329766in}{3.094335in}}%
\pgfpathlineto{\pgfqpoint{4.335653in}{3.098867in}}%
\pgfpathlineto{\pgfqpoint{4.343146in}{3.107712in}}%
\pgfpathlineto{\pgfqpoint{4.354653in}{3.125322in}}%
\pgfpathlineto{\pgfqpoint{4.369372in}{3.147030in}}%
\pgfpathlineto{\pgfqpoint{4.377132in}{3.154623in}}%
\pgfpathlineto{\pgfqpoint{4.383287in}{3.157693in}}%
\pgfpathlineto{\pgfqpoint{4.388639in}{3.157983in}}%
\pgfpathlineto{\pgfqpoint{4.393991in}{3.156034in}}%
\pgfpathlineto{\pgfqpoint{4.400146in}{3.151205in}}%
\pgfpathlineto{\pgfqpoint{4.407639in}{3.142220in}}%
\pgfpathlineto{\pgfqpoint{4.419682in}{3.123662in}}%
\pgfpathlineto{\pgfqpoint{4.433597in}{3.103283in}}%
\pgfpathlineto{\pgfqpoint{4.441090in}{3.095928in}}%
\pgfpathlineto{\pgfqpoint{4.447245in}{3.092775in}}%
\pgfpathlineto{\pgfqpoint{4.452598in}{3.092409in}}%
\pgfpathlineto{\pgfqpoint{4.457950in}{3.094283in}}%
\pgfpathlineto{\pgfqpoint{4.463837in}{3.098777in}}%
\pgfpathlineto{\pgfqpoint{4.471330in}{3.107585in}}%
\pgfpathlineto{\pgfqpoint{4.482837in}{3.125172in}}%
\pgfpathlineto{\pgfqpoint{4.497823in}{3.147240in}}%
\pgfpathlineto{\pgfqpoint{4.505316in}{3.154555in}}%
\pgfpathlineto{\pgfqpoint{4.511471in}{3.157668in}}%
\pgfpathlineto{\pgfqpoint{4.516823in}{3.157998in}}%
\pgfpathlineto{\pgfqpoint{4.522176in}{3.156087in}}%
\pgfpathlineto{\pgfqpoint{4.528063in}{3.151558in}}%
\pgfpathlineto{\pgfqpoint{4.535556in}{3.142715in}}%
\pgfpathlineto{\pgfqpoint{4.547063in}{3.125106in}}%
\pgfpathlineto{\pgfqpoint{4.561782in}{3.103396in}}%
\pgfpathlineto{\pgfqpoint{4.569542in}{3.095800in}}%
\pgfpathlineto{\pgfqpoint{4.575697in}{3.092728in}}%
\pgfpathlineto{\pgfqpoint{4.581049in}{3.092435in}}%
\pgfpathlineto{\pgfqpoint{4.586401in}{3.094381in}}%
\pgfpathlineto{\pgfqpoint{4.592556in}{3.099209in}}%
\pgfpathlineto{\pgfqpoint{4.600049in}{3.108191in}}%
\pgfpathlineto{\pgfqpoint{4.612092in}{3.126748in}}%
\pgfpathlineto{\pgfqpoint{4.626007in}{3.147128in}}%
\pgfpathlineto{\pgfqpoint{4.633500in}{3.154487in}}%
\pgfpathlineto{\pgfqpoint{4.639655in}{3.157642in}}%
\pgfpathlineto{\pgfqpoint{4.645008in}{3.158011in}}%
\pgfpathlineto{\pgfqpoint{4.650360in}{3.156139in}}%
\pgfpathlineto{\pgfqpoint{4.656247in}{3.151647in}}%
\pgfpathlineto{\pgfqpoint{4.663740in}{3.142842in}}%
\pgfpathlineto{\pgfqpoint{4.675247in}{3.125256in}}%
\pgfpathlineto{\pgfqpoint{4.690233in}{3.103186in}}%
\pgfpathlineto{\pgfqpoint{4.697726in}{3.095868in}}%
\pgfpathlineto{\pgfqpoint{4.703881in}{3.092753in}}%
\pgfpathlineto{\pgfqpoint{4.709233in}{3.092420in}}%
\pgfpathlineto{\pgfqpoint{4.714586in}{3.094329in}}%
\pgfpathlineto{\pgfqpoint{4.720473in}{3.098856in}}%
\pgfpathlineto{\pgfqpoint{4.727966in}{3.107696in}}%
\pgfpathlineto{\pgfqpoint{4.739473in}{3.125303in}}%
\pgfpathlineto{\pgfqpoint{4.754192in}{3.147016in}}%
\pgfpathlineto{\pgfqpoint{4.761952in}{3.154615in}}%
\pgfpathlineto{\pgfqpoint{4.768107in}{3.157690in}}%
\pgfpathlineto{\pgfqpoint{4.773459in}{3.157985in}}%
\pgfpathlineto{\pgfqpoint{4.778811in}{3.156041in}}%
\pgfpathlineto{\pgfqpoint{4.784966in}{3.151216in}}%
\pgfpathlineto{\pgfqpoint{4.792459in}{3.142236in}}%
\pgfpathlineto{\pgfqpoint{4.804502in}{3.123680in}}%
\pgfpathlineto{\pgfqpoint{4.818417in}{3.103297in}}%
\pgfpathlineto{\pgfqpoint{4.826178in}{3.095741in}}%
\pgfpathlineto{\pgfqpoint{4.832333in}{3.092706in}}%
\pgfpathlineto{\pgfqpoint{4.837685in}{3.092448in}}%
\pgfpathlineto{\pgfqpoint{4.843037in}{3.094428in}}%
\pgfpathlineto{\pgfqpoint{4.849192in}{3.099289in}}%
\pgfpathlineto{\pgfqpoint{4.856685in}{3.108304in}}%
\pgfpathlineto{\pgfqpoint{4.868995in}{3.127310in}}%
\pgfpathlineto{\pgfqpoint{4.882643in}{3.147226in}}%
\pgfpathlineto{\pgfqpoint{4.890136in}{3.154547in}}%
\pgfpathlineto{\pgfqpoint{4.896291in}{3.157665in}}%
\pgfpathlineto{\pgfqpoint{4.901643in}{3.157999in}}%
\pgfpathlineto{\pgfqpoint{4.906996in}{3.156093in}}%
\pgfpathlineto{\pgfqpoint{4.912883in}{3.151569in}}%
\pgfpathlineto{\pgfqpoint{4.920376in}{3.142731in}}%
\pgfpathlineto{\pgfqpoint{4.931883in}{3.125125in}}%
\pgfpathlineto{\pgfqpoint{4.946602in}{3.103410in}}%
\pgfpathlineto{\pgfqpoint{4.954362in}{3.095808in}}%
\pgfpathlineto{\pgfqpoint{4.960517in}{3.092731in}}%
\pgfpathlineto{\pgfqpoint{4.965869in}{3.092433in}}%
\pgfpathlineto{\pgfqpoint{4.971222in}{3.094375in}}%
\pgfpathlineto{\pgfqpoint{4.977376in}{3.099197in}}%
\pgfpathlineto{\pgfqpoint{4.984870in}{3.108175in}}%
\pgfpathlineto{\pgfqpoint{4.996912in}{3.126729in}}%
\pgfpathlineto{\pgfqpoint{5.010827in}{3.147114in}}%
\pgfpathlineto{\pgfqpoint{5.018588in}{3.154674in}}%
\pgfpathlineto{\pgfqpoint{5.024743in}{3.157711in}}%
\pgfpathlineto{\pgfqpoint{5.030095in}{3.157972in}}%
\pgfpathlineto{\pgfqpoint{5.035447in}{3.155995in}}%
\pgfpathlineto{\pgfqpoint{5.041602in}{3.151135in}}%
\pgfpathlineto{\pgfqpoint{5.049095in}{3.142123in}}%
\pgfpathlineto{\pgfqpoint{5.061138in}{3.123549in}}%
\pgfpathlineto{\pgfqpoint{5.075053in}{3.103200in}}%
\pgfpathlineto{\pgfqpoint{5.082546in}{3.095876in}}%
\pgfpathlineto{\pgfqpoint{5.088701in}{3.092756in}}%
\pgfpathlineto{\pgfqpoint{5.094053in}{3.092419in}}%
\pgfpathlineto{\pgfqpoint{5.099406in}{3.094322in}}%
\pgfpathlineto{\pgfqpoint{5.105293in}{3.098844in}}%
\pgfpathlineto{\pgfqpoint{5.112786in}{3.107680in}}%
\pgfpathlineto{\pgfqpoint{5.124293in}{3.125284in}}%
\pgfpathlineto{\pgfqpoint{5.139012in}{3.147002in}}%
\pgfpathlineto{\pgfqpoint{5.146772in}{3.154606in}}%
\pgfpathlineto{\pgfqpoint{5.152927in}{3.157687in}}%
\pgfpathlineto{\pgfqpoint{5.158279in}{3.157987in}}%
\pgfpathlineto{\pgfqpoint{5.163632in}{3.156048in}}%
\pgfpathlineto{\pgfqpoint{5.169787in}{3.151228in}}%
\pgfpathlineto{\pgfqpoint{5.177280in}{3.142252in}}%
\pgfpathlineto{\pgfqpoint{5.189322in}{3.123699in}}%
\pgfpathlineto{\pgfqpoint{5.203237in}{3.103312in}}%
\pgfpathlineto{\pgfqpoint{5.210998in}{3.095749in}}%
\pgfpathlineto{\pgfqpoint{5.217153in}{3.092709in}}%
\pgfpathlineto{\pgfqpoint{5.222505in}{3.092446in}}%
\pgfpathlineto{\pgfqpoint{5.227857in}{3.094421in}}%
\pgfpathlineto{\pgfqpoint{5.234012in}{3.099278in}}%
\pgfpathlineto{\pgfqpoint{5.241505in}{3.108288in}}%
\pgfpathlineto{\pgfqpoint{5.253548in}{3.126860in}}%
\pgfpathlineto{\pgfqpoint{5.267463in}{3.147212in}}%
\pgfpathlineto{\pgfqpoint{5.274956in}{3.154538in}}%
\pgfpathlineto{\pgfqpoint{5.281111in}{3.157661in}}%
\pgfpathlineto{\pgfqpoint{5.286464in}{3.158001in}}%
\pgfpathlineto{\pgfqpoint{5.291816in}{3.156100in}}%
\pgfpathlineto{\pgfqpoint{5.297703in}{3.151580in}}%
\pgfpathlineto{\pgfqpoint{5.305196in}{3.142747in}}%
\pgfpathlineto{\pgfqpoint{5.316703in}{3.125144in}}%
\pgfpathlineto{\pgfqpoint{5.331422in}{3.103424in}}%
\pgfpathlineto{\pgfqpoint{5.339182in}{3.095817in}}%
\pgfpathlineto{\pgfqpoint{5.345337in}{3.092734in}}%
\pgfpathlineto{\pgfqpoint{5.350689in}{3.092431in}}%
\pgfpathlineto{\pgfqpoint{5.356042in}{3.094368in}}%
\pgfpathlineto{\pgfqpoint{5.362197in}{3.099186in}}%
\pgfpathlineto{\pgfqpoint{5.369690in}{3.108159in}}%
\pgfpathlineto{\pgfqpoint{5.381732in}{3.126710in}}%
\pgfpathlineto{\pgfqpoint{5.395648in}{3.147100in}}%
\pgfpathlineto{\pgfqpoint{5.403408in}{3.154665in}}%
\pgfpathlineto{\pgfqpoint{5.409563in}{3.157708in}}%
\pgfpathlineto{\pgfqpoint{5.414915in}{3.157974in}}%
\pgfpathlineto{\pgfqpoint{5.420267in}{3.156001in}}%
\pgfpathlineto{\pgfqpoint{5.426422in}{3.151147in}}%
\pgfpathlineto{\pgfqpoint{5.433915in}{3.142139in}}%
\pgfpathlineto{\pgfqpoint{5.445958in}{3.123568in}}%
\pgfpathlineto{\pgfqpoint{5.459873in}{3.103214in}}%
\pgfpathlineto{\pgfqpoint{5.467366in}{3.095885in}}%
\pgfpathlineto{\pgfqpoint{5.473521in}{3.092759in}}%
\pgfpathlineto{\pgfqpoint{5.478874in}{3.092417in}}%
\pgfpathlineto{\pgfqpoint{5.484226in}{3.094316in}}%
\pgfpathlineto{\pgfqpoint{5.490113in}{3.098833in}}%
\pgfpathlineto{\pgfqpoint{5.497606in}{3.107664in}}%
\pgfpathlineto{\pgfqpoint{5.509113in}{3.125266in}}%
\pgfpathlineto{\pgfqpoint{5.523832in}{3.146988in}}%
\pgfpathlineto{\pgfqpoint{5.531592in}{3.154598in}}%
\pgfpathlineto{\pgfqpoint{5.537747in}{3.157683in}}%
\pgfpathlineto{\pgfqpoint{5.543099in}{3.157989in}}%
\pgfpathlineto{\pgfqpoint{5.548452in}{3.156054in}}%
\pgfpathlineto{\pgfqpoint{5.554607in}{3.151239in}}%
\pgfpathlineto{\pgfqpoint{5.562100in}{3.142268in}}%
\pgfpathlineto{\pgfqpoint{5.574142in}{3.123718in}}%
\pgfpathlineto{\pgfqpoint{5.588058in}{3.103326in}}%
\pgfpathlineto{\pgfqpoint{5.595818in}{3.095758in}}%
\pgfpathlineto{\pgfqpoint{5.601973in}{3.092712in}}%
\pgfpathlineto{\pgfqpoint{5.607325in}{3.092444in}}%
\pgfpathlineto{\pgfqpoint{5.612677in}{3.094414in}}%
\pgfpathlineto{\pgfqpoint{5.618832in}{3.099266in}}%
\pgfpathlineto{\pgfqpoint{5.626325in}{3.108272in}}%
\pgfpathlineto{\pgfqpoint{5.638368in}{3.126841in}}%
\pgfpathlineto{\pgfqpoint{5.652283in}{3.147198in}}%
\pgfpathlineto{\pgfqpoint{5.659776in}{3.154530in}}%
\pgfpathlineto{\pgfqpoint{5.665931in}{3.157658in}}%
\pgfpathlineto{\pgfqpoint{5.671284in}{3.158003in}}%
\pgfpathlineto{\pgfqpoint{5.676636in}{3.156106in}}%
\pgfpathlineto{\pgfqpoint{5.682523in}{3.151591in}}%
\pgfpathlineto{\pgfqpoint{5.690016in}{3.142763in}}%
\pgfpathlineto{\pgfqpoint{5.701523in}{3.125163in}}%
\pgfpathlineto{\pgfqpoint{5.716242in}{3.103438in}}%
\pgfpathlineto{\pgfqpoint{5.724002in}{3.095825in}}%
\pgfpathlineto{\pgfqpoint{5.730157in}{3.092737in}}%
\pgfpathlineto{\pgfqpoint{5.735509in}{3.092429in}}%
\pgfpathlineto{\pgfqpoint{5.740862in}{3.094361in}}%
\pgfpathlineto{\pgfqpoint{5.747017in}{3.099174in}}%
\pgfpathlineto{\pgfqpoint{5.754510in}{3.108143in}}%
\pgfpathlineto{\pgfqpoint{5.766552in}{3.126692in}}%
\pgfpathlineto{\pgfqpoint{5.780468in}{3.147086in}}%
\pgfpathlineto{\pgfqpoint{5.788228in}{3.154657in}}%
\pgfpathlineto{\pgfqpoint{5.794383in}{3.157705in}}%
\pgfpathlineto{\pgfqpoint{5.799735in}{3.157976in}}%
\pgfpathlineto{\pgfqpoint{5.805087in}{3.156008in}}%
\pgfpathlineto{\pgfqpoint{5.811242in}{3.151158in}}%
\pgfpathlineto{\pgfqpoint{5.818735in}{3.142155in}}%
\pgfpathlineto{\pgfqpoint{5.830778in}{3.123587in}}%
\pgfpathlineto{\pgfqpoint{5.844693in}{3.103228in}}%
\pgfpathlineto{\pgfqpoint{5.852186in}{3.095893in}}%
\pgfpathlineto{\pgfqpoint{5.858341in}{3.092762in}}%
\pgfpathlineto{\pgfqpoint{5.863694in}{3.092415in}}%
\pgfpathlineto{\pgfqpoint{5.869046in}{3.094309in}}%
\pgfpathlineto{\pgfqpoint{5.874933in}{3.098822in}}%
\pgfpathlineto{\pgfqpoint{5.882426in}{3.107648in}}%
\pgfpathlineto{\pgfqpoint{5.893933in}{3.125247in}}%
\pgfpathlineto{\pgfqpoint{5.908652in}{3.146974in}}%
\pgfpathlineto{\pgfqpoint{5.916412in}{3.154589in}}%
\pgfpathlineto{\pgfqpoint{5.922567in}{3.157680in}}%
\pgfpathlineto{\pgfqpoint{5.927919in}{3.157991in}}%
\pgfpathlineto{\pgfqpoint{5.933272in}{3.156061in}}%
\pgfpathlineto{\pgfqpoint{5.939427in}{3.151251in}}%
\pgfpathlineto{\pgfqpoint{5.946920in}{3.142284in}}%
\pgfpathlineto{\pgfqpoint{5.958962in}{3.123737in}}%
\pgfpathlineto{\pgfqpoint{5.972878in}{3.103340in}}%
\pgfpathlineto{\pgfqpoint{5.980638in}{3.095766in}}%
\pgfpathlineto{\pgfqpoint{5.986793in}{3.092715in}}%
\pgfpathlineto{\pgfqpoint{5.992145in}{3.092442in}}%
\pgfpathlineto{\pgfqpoint{5.997497in}{3.094408in}}%
\pgfpathlineto{\pgfqpoint{6.003652in}{3.099255in}}%
\pgfpathlineto{\pgfqpoint{6.011145in}{3.108255in}}%
\pgfpathlineto{\pgfqpoint{6.023188in}{3.126823in}}%
\pgfpathlineto{\pgfqpoint{6.037103in}{3.147184in}}%
\pgfpathlineto{\pgfqpoint{6.044596in}{3.154521in}}%
\pgfpathlineto{\pgfqpoint{6.050751in}{3.157655in}}%
\pgfpathlineto{\pgfqpoint{6.056104in}{3.158004in}}%
\pgfpathlineto{\pgfqpoint{6.061456in}{3.156113in}}%
\pgfpathlineto{\pgfqpoint{6.067343in}{3.151603in}}%
\pgfpathlineto{\pgfqpoint{6.074836in}{3.142779in}}%
\pgfpathlineto{\pgfqpoint{6.086343in}{3.125181in}}%
\pgfpathlineto{\pgfqpoint{6.101062in}{3.103452in}}%
\pgfpathlineto{\pgfqpoint{6.108822in}{3.095834in}}%
\pgfpathlineto{\pgfqpoint{6.114977in}{3.092740in}}%
\pgfpathlineto{\pgfqpoint{6.120329in}{3.092427in}}%
\pgfpathlineto{\pgfqpoint{6.125682in}{3.094355in}}%
\pgfpathlineto{\pgfqpoint{6.131837in}{3.099163in}}%
\pgfpathlineto{\pgfqpoint{6.139330in}{3.108127in}}%
\pgfpathlineto{\pgfqpoint{6.151372in}{3.126673in}}%
\pgfpathlineto{\pgfqpoint{6.165288in}{3.147072in}}%
\pgfpathlineto{\pgfqpoint{6.173048in}{3.154648in}}%
\pgfpathlineto{\pgfqpoint{6.179203in}{3.157702in}}%
\pgfpathlineto{\pgfqpoint{6.184555in}{3.157978in}}%
\pgfpathlineto{\pgfqpoint{6.189908in}{3.156015in}}%
\pgfpathlineto{\pgfqpoint{6.196063in}{3.151170in}}%
\pgfpathlineto{\pgfqpoint{6.203556in}{3.142171in}}%
\pgfpathlineto{\pgfqpoint{6.215598in}{3.123606in}}%
\pgfpathlineto{\pgfqpoint{6.229513in}{3.103242in}}%
\pgfpathlineto{\pgfqpoint{6.237006in}{3.095902in}}%
\pgfpathlineto{\pgfqpoint{6.243161in}{3.092765in}}%
\pgfpathlineto{\pgfqpoint{6.248514in}{3.092414in}}%
\pgfpathlineto{\pgfqpoint{6.253866in}{3.094303in}}%
\pgfpathlineto{\pgfqpoint{6.259753in}{3.098811in}}%
\pgfpathlineto{\pgfqpoint{6.267246in}{3.107632in}}%
\pgfpathlineto{\pgfqpoint{6.278753in}{3.125228in}}%
\pgfpathlineto{\pgfqpoint{6.293472in}{3.146960in}}%
\pgfpathlineto{\pgfqpoint{6.301232in}{3.154581in}}%
\pgfpathlineto{\pgfqpoint{6.307387in}{3.157677in}}%
\pgfpathlineto{\pgfqpoint{6.312740in}{3.157992in}}%
\pgfpathlineto{\pgfqpoint{6.318092in}{3.156067in}}%
\pgfpathlineto{\pgfqpoint{6.324247in}{3.151262in}}%
\pgfpathlineto{\pgfqpoint{6.331740in}{3.142300in}}%
\pgfpathlineto{\pgfqpoint{6.343782in}{3.123755in}}%
\pgfpathlineto{\pgfqpoint{6.357698in}{3.103354in}}%
\pgfpathlineto{\pgfqpoint{6.365458in}{3.095775in}}%
\pgfpathlineto{\pgfqpoint{6.371613in}{3.092718in}}%
\pgfpathlineto{\pgfqpoint{6.376965in}{3.092440in}}%
\pgfpathlineto{\pgfqpoint{6.382318in}{3.094401in}}%
\pgfpathlineto{\pgfqpoint{6.388473in}{3.099243in}}%
\pgfpathlineto{\pgfqpoint{6.395966in}{3.108239in}}%
\pgfpathlineto{\pgfqpoint{6.408008in}{3.126804in}}%
\pgfpathlineto{\pgfqpoint{6.421924in}{3.147170in}}%
\pgfpathlineto{\pgfqpoint{6.429417in}{3.154513in}}%
\pgfpathlineto{\pgfqpoint{6.435572in}{3.157652in}}%
\pgfpathlineto{\pgfqpoint{6.440924in}{3.158006in}}%
\pgfpathlineto{\pgfqpoint{6.446276in}{3.156119in}}%
\pgfpathlineto{\pgfqpoint{6.452163in}{3.151614in}}%
\pgfpathlineto{\pgfqpoint{6.459656in}{3.142795in}}%
\pgfpathlineto{\pgfqpoint{6.471163in}{3.125200in}}%
\pgfpathlineto{\pgfqpoint{6.485882in}{3.103466in}}%
\pgfpathlineto{\pgfqpoint{6.493642in}{3.095842in}}%
\pgfpathlineto{\pgfqpoint{6.499797in}{3.092743in}}%
\pgfpathlineto{\pgfqpoint{6.505150in}{3.092426in}}%
\pgfpathlineto{\pgfqpoint{6.510502in}{3.094348in}}%
\pgfpathlineto{\pgfqpoint{6.516657in}{3.099151in}}%
\pgfpathlineto{\pgfqpoint{6.524150in}{3.108111in}}%
\pgfpathlineto{\pgfqpoint{6.536192in}{3.126654in}}%
\pgfpathlineto{\pgfqpoint{6.550108in}{3.147058in}}%
\pgfpathlineto{\pgfqpoint{6.557868in}{3.154640in}}%
\pgfpathlineto{\pgfqpoint{6.564023in}{3.157699in}}%
\pgfpathlineto{\pgfqpoint{6.569375in}{3.157980in}}%
\pgfpathlineto{\pgfqpoint{6.574728in}{3.156021in}}%
\pgfpathlineto{\pgfqpoint{6.580883in}{3.151181in}}%
\pgfpathlineto{\pgfqpoint{6.588376in}{3.142188in}}%
\pgfpathlineto{\pgfqpoint{6.600418in}{3.123624in}}%
\pgfpathlineto{\pgfqpoint{6.614334in}{3.103255in}}%
\pgfpathlineto{\pgfqpoint{6.621827in}{3.095911in}}%
\pgfpathlineto{\pgfqpoint{6.627982in}{3.092769in}}%
\pgfpathlineto{\pgfqpoint{6.633334in}{3.092412in}}%
\pgfpathlineto{\pgfqpoint{6.638686in}{3.094296in}}%
\pgfpathlineto{\pgfqpoint{6.644573in}{3.098800in}}%
\pgfpathlineto{\pgfqpoint{6.652066in}{3.107616in}}%
\pgfpathlineto{\pgfqpoint{6.663306in}{3.124778in}}%
\pgfpathlineto{\pgfqpoint{6.663306in}{3.124778in}}%
\pgfusepath{stroke}%
\end{pgfscope}%
\begin{pgfscope}%
\pgfpathrectangle{\pgfqpoint{0.467797in}{2.292089in}}{\pgfqpoint{6.490533in}{1.666241in}}%
\pgfusepath{clip}%
\pgfsetrectcap%
\pgfsetroundjoin%
\pgfsetlinewidth{1.505625pt}%
\definecolor{currentstroke}{rgb}{0.172549,0.627451,0.172549}%
\pgfsetstrokecolor{currentstroke}%
\pgfsetdash{}{0pt}%
\pgfpathmoveto{\pgfqpoint{0.762821in}{3.125209in}}%
\pgfpathlineto{\pgfqpoint{0.776737in}{3.145657in}}%
\pgfpathlineto{\pgfqpoint{0.783962in}{3.152551in}}%
\pgfpathlineto{\pgfqpoint{0.789849in}{3.155242in}}%
\pgfpathlineto{\pgfqpoint{0.794934in}{3.155213in}}%
\pgfpathlineto{\pgfqpoint{0.800286in}{3.152824in}}%
\pgfpathlineto{\pgfqpoint{0.806441in}{3.147344in}}%
\pgfpathlineto{\pgfqpoint{0.814469in}{3.136778in}}%
\pgfpathlineto{\pgfqpoint{0.840427in}{3.099878in}}%
\pgfpathlineto{\pgfqpoint{0.846582in}{3.095878in}}%
\pgfpathlineto{\pgfqpoint{0.851934in}{3.094930in}}%
\pgfpathlineto{\pgfqpoint{0.857019in}{3.096292in}}%
\pgfpathlineto{\pgfqpoint{0.862639in}{3.100239in}}%
\pgfpathlineto{\pgfqpoint{0.869596in}{3.108143in}}%
\pgfpathlineto{\pgfqpoint{0.880033in}{3.123915in}}%
\pgfpathlineto{\pgfqpoint{0.895019in}{3.145973in}}%
\pgfpathlineto{\pgfqpoint{0.902245in}{3.152734in}}%
\pgfpathlineto{\pgfqpoint{0.907864in}{3.155242in}}%
\pgfpathlineto{\pgfqpoint{0.912949in}{3.155213in}}%
\pgfpathlineto{\pgfqpoint{0.918301in}{3.152824in}}%
\pgfpathlineto{\pgfqpoint{0.924456in}{3.147344in}}%
\pgfpathlineto{\pgfqpoint{0.932484in}{3.136778in}}%
\pgfpathlineto{\pgfqpoint{0.958442in}{3.099878in}}%
\pgfpathlineto{\pgfqpoint{0.964597in}{3.095878in}}%
\pgfpathlineto{\pgfqpoint{0.969949in}{3.094930in}}%
\pgfpathlineto{\pgfqpoint{0.975034in}{3.096292in}}%
\pgfpathlineto{\pgfqpoint{0.980654in}{3.100239in}}%
\pgfpathlineto{\pgfqpoint{0.987611in}{3.108143in}}%
\pgfpathlineto{\pgfqpoint{0.998048in}{3.123915in}}%
\pgfpathlineto{\pgfqpoint{1.013034in}{3.145973in}}%
\pgfpathlineto{\pgfqpoint{1.020260in}{3.152734in}}%
\pgfpathlineto{\pgfqpoint{1.025879in}{3.155242in}}%
\pgfpathlineto{\pgfqpoint{1.030964in}{3.155213in}}%
\pgfpathlineto{\pgfqpoint{1.036316in}{3.152824in}}%
\pgfpathlineto{\pgfqpoint{1.042471in}{3.147344in}}%
\pgfpathlineto{\pgfqpoint{1.050499in}{3.136778in}}%
\pgfpathlineto{\pgfqpoint{1.076457in}{3.099878in}}%
\pgfpathlineto{\pgfqpoint{1.082612in}{3.095878in}}%
\pgfpathlineto{\pgfqpoint{1.087964in}{3.094930in}}%
\pgfpathlineto{\pgfqpoint{1.093049in}{3.096292in}}%
\pgfpathlineto{\pgfqpoint{1.098669in}{3.100239in}}%
\pgfpathlineto{\pgfqpoint{1.105627in}{3.108143in}}%
\pgfpathlineto{\pgfqpoint{1.116063in}{3.123915in}}%
\pgfpathlineto{\pgfqpoint{1.131049in}{3.145973in}}%
\pgfpathlineto{\pgfqpoint{1.138275in}{3.152734in}}%
\pgfpathlineto{\pgfqpoint{1.143894in}{3.155242in}}%
\pgfpathlineto{\pgfqpoint{1.148979in}{3.155213in}}%
\pgfpathlineto{\pgfqpoint{1.154331in}{3.152824in}}%
\pgfpathlineto{\pgfqpoint{1.160486in}{3.147344in}}%
\pgfpathlineto{\pgfqpoint{1.168514in}{3.136778in}}%
\pgfpathlineto{\pgfqpoint{1.194472in}{3.099878in}}%
\pgfpathlineto{\pgfqpoint{1.200627in}{3.095878in}}%
\pgfpathlineto{\pgfqpoint{1.205979in}{3.094930in}}%
\pgfpathlineto{\pgfqpoint{1.211064in}{3.096292in}}%
\pgfpathlineto{\pgfqpoint{1.216684in}{3.100239in}}%
\pgfpathlineto{\pgfqpoint{1.223642in}{3.108143in}}%
\pgfpathlineto{\pgfqpoint{1.234078in}{3.123915in}}%
\pgfpathlineto{\pgfqpoint{1.249064in}{3.145973in}}%
\pgfpathlineto{\pgfqpoint{1.256290in}{3.152734in}}%
\pgfpathlineto{\pgfqpoint{1.261909in}{3.155242in}}%
\pgfpathlineto{\pgfqpoint{1.266994in}{3.155213in}}%
\pgfpathlineto{\pgfqpoint{1.272346in}{3.152824in}}%
\pgfpathlineto{\pgfqpoint{1.278501in}{3.147344in}}%
\pgfpathlineto{\pgfqpoint{1.286529in}{3.136778in}}%
\pgfpathlineto{\pgfqpoint{1.312487in}{3.099878in}}%
\pgfpathlineto{\pgfqpoint{1.318642in}{3.095878in}}%
\pgfpathlineto{\pgfqpoint{1.323995in}{3.094930in}}%
\pgfpathlineto{\pgfqpoint{1.329079in}{3.096292in}}%
\pgfpathlineto{\pgfqpoint{1.334699in}{3.100239in}}%
\pgfpathlineto{\pgfqpoint{1.341657in}{3.108143in}}%
\pgfpathlineto{\pgfqpoint{1.352093in}{3.123915in}}%
\pgfpathlineto{\pgfqpoint{1.367079in}{3.145973in}}%
\pgfpathlineto{\pgfqpoint{1.374305in}{3.152734in}}%
\pgfpathlineto{\pgfqpoint{1.379925in}{3.155242in}}%
\pgfpathlineto{\pgfqpoint{1.385009in}{3.155213in}}%
\pgfpathlineto{\pgfqpoint{1.390361in}{3.152824in}}%
\pgfpathlineto{\pgfqpoint{1.396516in}{3.147344in}}%
\pgfpathlineto{\pgfqpoint{1.404544in}{3.136778in}}%
\pgfpathlineto{\pgfqpoint{1.430502in}{3.099878in}}%
\pgfpathlineto{\pgfqpoint{1.436657in}{3.095878in}}%
\pgfpathlineto{\pgfqpoint{1.442010in}{3.094930in}}%
\pgfpathlineto{\pgfqpoint{1.447094in}{3.096292in}}%
\pgfpathlineto{\pgfqpoint{1.452714in}{3.100239in}}%
\pgfpathlineto{\pgfqpoint{1.459672in}{3.108143in}}%
\pgfpathlineto{\pgfqpoint{1.470108in}{3.123915in}}%
\pgfpathlineto{\pgfqpoint{1.485094in}{3.145973in}}%
\pgfpathlineto{\pgfqpoint{1.492320in}{3.152734in}}%
\pgfpathlineto{\pgfqpoint{1.497940in}{3.155242in}}%
\pgfpathlineto{\pgfqpoint{1.503024in}{3.155213in}}%
\pgfpathlineto{\pgfqpoint{1.508376in}{3.152824in}}%
\pgfpathlineto{\pgfqpoint{1.514531in}{3.147344in}}%
\pgfpathlineto{\pgfqpoint{1.522560in}{3.136778in}}%
\pgfpathlineto{\pgfqpoint{1.548517in}{3.099878in}}%
\pgfpathlineto{\pgfqpoint{1.554672in}{3.095878in}}%
\pgfpathlineto{\pgfqpoint{1.560025in}{3.094930in}}%
\pgfpathlineto{\pgfqpoint{1.565109in}{3.096292in}}%
\pgfpathlineto{\pgfqpoint{1.570729in}{3.100239in}}%
\pgfpathlineto{\pgfqpoint{1.577687in}{3.108143in}}%
\pgfpathlineto{\pgfqpoint{1.588123in}{3.123915in}}%
\pgfpathlineto{\pgfqpoint{1.603109in}{3.145973in}}%
\pgfpathlineto{\pgfqpoint{1.610335in}{3.152734in}}%
\pgfpathlineto{\pgfqpoint{1.615955in}{3.155242in}}%
\pgfpathlineto{\pgfqpoint{1.621039in}{3.155213in}}%
\pgfpathlineto{\pgfqpoint{1.626391in}{3.152824in}}%
\pgfpathlineto{\pgfqpoint{1.632546in}{3.147344in}}%
\pgfpathlineto{\pgfqpoint{1.640575in}{3.136778in}}%
\pgfpathlineto{\pgfqpoint{1.666533in}{3.099878in}}%
\pgfpathlineto{\pgfqpoint{1.672688in}{3.095878in}}%
\pgfpathlineto{\pgfqpoint{1.678040in}{3.094930in}}%
\pgfpathlineto{\pgfqpoint{1.683124in}{3.096292in}}%
\pgfpathlineto{\pgfqpoint{1.688744in}{3.100239in}}%
\pgfpathlineto{\pgfqpoint{1.695702in}{3.108143in}}%
\pgfpathlineto{\pgfqpoint{1.706138in}{3.123915in}}%
\pgfpathlineto{\pgfqpoint{1.721125in}{3.145973in}}%
\pgfpathlineto{\pgfqpoint{1.728350in}{3.152734in}}%
\pgfpathlineto{\pgfqpoint{1.733970in}{3.155242in}}%
\pgfpathlineto{\pgfqpoint{1.739054in}{3.155213in}}%
\pgfpathlineto{\pgfqpoint{1.744406in}{3.152824in}}%
\pgfpathlineto{\pgfqpoint{1.750561in}{3.147344in}}%
\pgfpathlineto{\pgfqpoint{1.758590in}{3.136778in}}%
\pgfpathlineto{\pgfqpoint{1.784548in}{3.099878in}}%
\pgfpathlineto{\pgfqpoint{1.790703in}{3.095878in}}%
\pgfpathlineto{\pgfqpoint{1.796055in}{3.094930in}}%
\pgfpathlineto{\pgfqpoint{1.801139in}{3.096292in}}%
\pgfpathlineto{\pgfqpoint{1.806759in}{3.100239in}}%
\pgfpathlineto{\pgfqpoint{1.813717in}{3.108143in}}%
\pgfpathlineto{\pgfqpoint{1.824154in}{3.123915in}}%
\pgfpathlineto{\pgfqpoint{1.839140in}{3.145973in}}%
\pgfpathlineto{\pgfqpoint{1.846365in}{3.152734in}}%
\pgfpathlineto{\pgfqpoint{1.851985in}{3.155242in}}%
\pgfpathlineto{\pgfqpoint{1.857069in}{3.155213in}}%
\pgfpathlineto{\pgfqpoint{1.862421in}{3.152824in}}%
\pgfpathlineto{\pgfqpoint{1.868576in}{3.147344in}}%
\pgfpathlineto{\pgfqpoint{1.876605in}{3.136778in}}%
\pgfpathlineto{\pgfqpoint{1.902563in}{3.099878in}}%
\pgfpathlineto{\pgfqpoint{1.908718in}{3.095878in}}%
\pgfpathlineto{\pgfqpoint{1.914070in}{3.094930in}}%
\pgfpathlineto{\pgfqpoint{1.919154in}{3.096292in}}%
\pgfpathlineto{\pgfqpoint{1.924774in}{3.100239in}}%
\pgfpathlineto{\pgfqpoint{1.931732in}{3.108143in}}%
\pgfpathlineto{\pgfqpoint{1.942169in}{3.123915in}}%
\pgfpathlineto{\pgfqpoint{1.957155in}{3.145973in}}%
\pgfpathlineto{\pgfqpoint{1.964380in}{3.152734in}}%
\pgfpathlineto{\pgfqpoint{1.970000in}{3.155242in}}%
\pgfpathlineto{\pgfqpoint{1.975084in}{3.155213in}}%
\pgfpathlineto{\pgfqpoint{1.980436in}{3.152824in}}%
\pgfpathlineto{\pgfqpoint{1.986591in}{3.147344in}}%
\pgfpathlineto{\pgfqpoint{1.994620in}{3.136778in}}%
\pgfpathlineto{\pgfqpoint{2.020578in}{3.099878in}}%
\pgfpathlineto{\pgfqpoint{2.026733in}{3.095878in}}%
\pgfpathlineto{\pgfqpoint{2.032085in}{3.094930in}}%
\pgfpathlineto{\pgfqpoint{2.037169in}{3.096292in}}%
\pgfpathlineto{\pgfqpoint{2.042789in}{3.100239in}}%
\pgfpathlineto{\pgfqpoint{2.049747in}{3.108143in}}%
\pgfpathlineto{\pgfqpoint{2.060184in}{3.123915in}}%
\pgfpathlineto{\pgfqpoint{2.075170in}{3.145973in}}%
\pgfpathlineto{\pgfqpoint{2.082395in}{3.152734in}}%
\pgfpathlineto{\pgfqpoint{2.088015in}{3.155242in}}%
\pgfpathlineto{\pgfqpoint{2.093099in}{3.155213in}}%
\pgfpathlineto{\pgfqpoint{2.098452in}{3.152824in}}%
\pgfpathlineto{\pgfqpoint{2.104607in}{3.147344in}}%
\pgfpathlineto{\pgfqpoint{2.112635in}{3.136778in}}%
\pgfpathlineto{\pgfqpoint{2.138593in}{3.099878in}}%
\pgfpathlineto{\pgfqpoint{2.144748in}{3.095878in}}%
\pgfpathlineto{\pgfqpoint{2.150100in}{3.094930in}}%
\pgfpathlineto{\pgfqpoint{2.155184in}{3.096292in}}%
\pgfpathlineto{\pgfqpoint{2.160804in}{3.100239in}}%
\pgfpathlineto{\pgfqpoint{2.167762in}{3.108143in}}%
\pgfpathlineto{\pgfqpoint{2.178199in}{3.123915in}}%
\pgfpathlineto{\pgfqpoint{2.193185in}{3.145973in}}%
\pgfpathlineto{\pgfqpoint{2.200410in}{3.152734in}}%
\pgfpathlineto{\pgfqpoint{2.206030in}{3.155242in}}%
\pgfpathlineto{\pgfqpoint{2.211114in}{3.155213in}}%
\pgfpathlineto{\pgfqpoint{2.216467in}{3.152824in}}%
\pgfpathlineto{\pgfqpoint{2.222622in}{3.147344in}}%
\pgfpathlineto{\pgfqpoint{2.230650in}{3.136778in}}%
\pgfpathlineto{\pgfqpoint{2.256608in}{3.099878in}}%
\pgfpathlineto{\pgfqpoint{2.262763in}{3.095878in}}%
\pgfpathlineto{\pgfqpoint{2.268115in}{3.094930in}}%
\pgfpathlineto{\pgfqpoint{2.273199in}{3.096292in}}%
\pgfpathlineto{\pgfqpoint{2.278819in}{3.100239in}}%
\pgfpathlineto{\pgfqpoint{2.285777in}{3.108143in}}%
\pgfpathlineto{\pgfqpoint{2.296214in}{3.123915in}}%
\pgfpathlineto{\pgfqpoint{2.311200in}{3.145973in}}%
\pgfpathlineto{\pgfqpoint{2.318425in}{3.152734in}}%
\pgfpathlineto{\pgfqpoint{2.324045in}{3.155242in}}%
\pgfpathlineto{\pgfqpoint{2.329129in}{3.155213in}}%
\pgfpathlineto{\pgfqpoint{2.334482in}{3.152824in}}%
\pgfpathlineto{\pgfqpoint{2.340637in}{3.147344in}}%
\pgfpathlineto{\pgfqpoint{2.348665in}{3.136778in}}%
\pgfpathlineto{\pgfqpoint{2.374623in}{3.099878in}}%
\pgfpathlineto{\pgfqpoint{2.380778in}{3.095878in}}%
\pgfpathlineto{\pgfqpoint{2.386130in}{3.094930in}}%
\pgfpathlineto{\pgfqpoint{2.391215in}{3.096292in}}%
\pgfpathlineto{\pgfqpoint{2.396834in}{3.100239in}}%
\pgfpathlineto{\pgfqpoint{2.403792in}{3.108143in}}%
\pgfpathlineto{\pgfqpoint{2.414229in}{3.123915in}}%
\pgfpathlineto{\pgfqpoint{2.429215in}{3.145973in}}%
\pgfpathlineto{\pgfqpoint{2.436440in}{3.152734in}}%
\pgfpathlineto{\pgfqpoint{2.442060in}{3.155242in}}%
\pgfpathlineto{\pgfqpoint{2.447145in}{3.155213in}}%
\pgfpathlineto{\pgfqpoint{2.452497in}{3.152824in}}%
\pgfpathlineto{\pgfqpoint{2.458652in}{3.147344in}}%
\pgfpathlineto{\pgfqpoint{2.466680in}{3.136778in}}%
\pgfpathlineto{\pgfqpoint{2.492638in}{3.099878in}}%
\pgfpathlineto{\pgfqpoint{2.498793in}{3.095878in}}%
\pgfpathlineto{\pgfqpoint{2.504145in}{3.094930in}}%
\pgfpathlineto{\pgfqpoint{2.509230in}{3.096292in}}%
\pgfpathlineto{\pgfqpoint{2.514849in}{3.100239in}}%
\pgfpathlineto{\pgfqpoint{2.521807in}{3.108143in}}%
\pgfpathlineto{\pgfqpoint{2.532244in}{3.123915in}}%
\pgfpathlineto{\pgfqpoint{2.547230in}{3.145973in}}%
\pgfpathlineto{\pgfqpoint{2.554455in}{3.152734in}}%
\pgfpathlineto{\pgfqpoint{2.560075in}{3.155242in}}%
\pgfpathlineto{\pgfqpoint{2.565160in}{3.155213in}}%
\pgfpathlineto{\pgfqpoint{2.570512in}{3.152824in}}%
\pgfpathlineto{\pgfqpoint{2.576667in}{3.147344in}}%
\pgfpathlineto{\pgfqpoint{2.584695in}{3.136778in}}%
\pgfpathlineto{\pgfqpoint{2.610653in}{3.099878in}}%
\pgfpathlineto{\pgfqpoint{2.616808in}{3.095878in}}%
\pgfpathlineto{\pgfqpoint{2.622160in}{3.094930in}}%
\pgfpathlineto{\pgfqpoint{2.627245in}{3.096292in}}%
\pgfpathlineto{\pgfqpoint{2.632864in}{3.100239in}}%
\pgfpathlineto{\pgfqpoint{2.639822in}{3.108143in}}%
\pgfpathlineto{\pgfqpoint{2.650259in}{3.123915in}}%
\pgfpathlineto{\pgfqpoint{2.665245in}{3.145973in}}%
\pgfpathlineto{\pgfqpoint{2.672470in}{3.152734in}}%
\pgfpathlineto{\pgfqpoint{2.678090in}{3.155242in}}%
\pgfpathlineto{\pgfqpoint{2.683175in}{3.155213in}}%
\pgfpathlineto{\pgfqpoint{2.688527in}{3.152824in}}%
\pgfpathlineto{\pgfqpoint{2.694682in}{3.147344in}}%
\pgfpathlineto{\pgfqpoint{2.702710in}{3.136778in}}%
\pgfpathlineto{\pgfqpoint{2.728668in}{3.099878in}}%
\pgfpathlineto{\pgfqpoint{2.734823in}{3.095878in}}%
\pgfpathlineto{\pgfqpoint{2.740175in}{3.094930in}}%
\pgfpathlineto{\pgfqpoint{2.745260in}{3.096292in}}%
\pgfpathlineto{\pgfqpoint{2.750879in}{3.100239in}}%
\pgfpathlineto{\pgfqpoint{2.757837in}{3.108143in}}%
\pgfpathlineto{\pgfqpoint{2.768274in}{3.123915in}}%
\pgfpathlineto{\pgfqpoint{2.783260in}{3.145973in}}%
\pgfpathlineto{\pgfqpoint{2.790485in}{3.152734in}}%
\pgfpathlineto{\pgfqpoint{2.796105in}{3.155242in}}%
\pgfpathlineto{\pgfqpoint{2.801190in}{3.155213in}}%
\pgfpathlineto{\pgfqpoint{2.806542in}{3.152824in}}%
\pgfpathlineto{\pgfqpoint{2.812697in}{3.147344in}}%
\pgfpathlineto{\pgfqpoint{2.820725in}{3.136778in}}%
\pgfpathlineto{\pgfqpoint{2.846683in}{3.099878in}}%
\pgfpathlineto{\pgfqpoint{2.852838in}{3.095878in}}%
\pgfpathlineto{\pgfqpoint{2.858190in}{3.094930in}}%
\pgfpathlineto{\pgfqpoint{2.863275in}{3.096292in}}%
\pgfpathlineto{\pgfqpoint{2.868894in}{3.100239in}}%
\pgfpathlineto{\pgfqpoint{2.875852in}{3.108143in}}%
\pgfpathlineto{\pgfqpoint{2.886289in}{3.123915in}}%
\pgfpathlineto{\pgfqpoint{2.901275in}{3.145973in}}%
\pgfpathlineto{\pgfqpoint{2.908500in}{3.152734in}}%
\pgfpathlineto{\pgfqpoint{2.914120in}{3.155242in}}%
\pgfpathlineto{\pgfqpoint{2.919205in}{3.155213in}}%
\pgfpathlineto{\pgfqpoint{2.924557in}{3.152824in}}%
\pgfpathlineto{\pgfqpoint{2.930712in}{3.147344in}}%
\pgfpathlineto{\pgfqpoint{2.938740in}{3.136778in}}%
\pgfpathlineto{\pgfqpoint{2.964698in}{3.099878in}}%
\pgfpathlineto{\pgfqpoint{2.970853in}{3.095878in}}%
\pgfpathlineto{\pgfqpoint{2.976205in}{3.094930in}}%
\pgfpathlineto{\pgfqpoint{2.981290in}{3.096292in}}%
\pgfpathlineto{\pgfqpoint{2.986910in}{3.100239in}}%
\pgfpathlineto{\pgfqpoint{2.993867in}{3.108143in}}%
\pgfpathlineto{\pgfqpoint{3.004304in}{3.123915in}}%
\pgfpathlineto{\pgfqpoint{3.019290in}{3.145973in}}%
\pgfpathlineto{\pgfqpoint{3.026515in}{3.152734in}}%
\pgfpathlineto{\pgfqpoint{3.032135in}{3.155242in}}%
\pgfpathlineto{\pgfqpoint{3.037220in}{3.155213in}}%
\pgfpathlineto{\pgfqpoint{3.042572in}{3.152824in}}%
\pgfpathlineto{\pgfqpoint{3.048727in}{3.147344in}}%
\pgfpathlineto{\pgfqpoint{3.056755in}{3.136778in}}%
\pgfpathlineto{\pgfqpoint{3.082713in}{3.099878in}}%
\pgfpathlineto{\pgfqpoint{3.088868in}{3.095878in}}%
\pgfpathlineto{\pgfqpoint{3.094220in}{3.094930in}}%
\pgfpathlineto{\pgfqpoint{3.099305in}{3.096292in}}%
\pgfpathlineto{\pgfqpoint{3.104925in}{3.100239in}}%
\pgfpathlineto{\pgfqpoint{3.111882in}{3.108143in}}%
\pgfpathlineto{\pgfqpoint{3.122319in}{3.123915in}}%
\pgfpathlineto{\pgfqpoint{3.137305in}{3.145973in}}%
\pgfpathlineto{\pgfqpoint{3.144531in}{3.152734in}}%
\pgfpathlineto{\pgfqpoint{3.150150in}{3.155242in}}%
\pgfpathlineto{\pgfqpoint{3.155235in}{3.155213in}}%
\pgfpathlineto{\pgfqpoint{3.160587in}{3.152824in}}%
\pgfpathlineto{\pgfqpoint{3.166742in}{3.147344in}}%
\pgfpathlineto{\pgfqpoint{3.174770in}{3.136778in}}%
\pgfpathlineto{\pgfqpoint{3.200728in}{3.099878in}}%
\pgfpathlineto{\pgfqpoint{3.206883in}{3.095878in}}%
\pgfpathlineto{\pgfqpoint{3.212235in}{3.094930in}}%
\pgfpathlineto{\pgfqpoint{3.217320in}{3.096292in}}%
\pgfpathlineto{\pgfqpoint{3.222940in}{3.100239in}}%
\pgfpathlineto{\pgfqpoint{3.229897in}{3.108143in}}%
\pgfpathlineto{\pgfqpoint{3.240334in}{3.123915in}}%
\pgfpathlineto{\pgfqpoint{3.255320in}{3.145973in}}%
\pgfpathlineto{\pgfqpoint{3.262546in}{3.152734in}}%
\pgfpathlineto{\pgfqpoint{3.268165in}{3.155242in}}%
\pgfpathlineto{\pgfqpoint{3.273250in}{3.155213in}}%
\pgfpathlineto{\pgfqpoint{3.278602in}{3.152824in}}%
\pgfpathlineto{\pgfqpoint{3.284757in}{3.147344in}}%
\pgfpathlineto{\pgfqpoint{3.292785in}{3.136778in}}%
\pgfpathlineto{\pgfqpoint{3.318743in}{3.099878in}}%
\pgfpathlineto{\pgfqpoint{3.324898in}{3.095878in}}%
\pgfpathlineto{\pgfqpoint{3.330250in}{3.094930in}}%
\pgfpathlineto{\pgfqpoint{3.335335in}{3.096292in}}%
\pgfpathlineto{\pgfqpoint{3.340955in}{3.100239in}}%
\pgfpathlineto{\pgfqpoint{3.347912in}{3.108143in}}%
\pgfpathlineto{\pgfqpoint{3.358349in}{3.123915in}}%
\pgfpathlineto{\pgfqpoint{3.373335in}{3.145973in}}%
\pgfpathlineto{\pgfqpoint{3.380561in}{3.152734in}}%
\pgfpathlineto{\pgfqpoint{3.386180in}{3.155242in}}%
\pgfpathlineto{\pgfqpoint{3.391265in}{3.155213in}}%
\pgfpathlineto{\pgfqpoint{3.396617in}{3.152824in}}%
\pgfpathlineto{\pgfqpoint{3.402772in}{3.147344in}}%
\pgfpathlineto{\pgfqpoint{3.410800in}{3.136778in}}%
\pgfpathlineto{\pgfqpoint{3.436758in}{3.099878in}}%
\pgfpathlineto{\pgfqpoint{3.442913in}{3.095878in}}%
\pgfpathlineto{\pgfqpoint{3.448265in}{3.094930in}}%
\pgfpathlineto{\pgfqpoint{3.453350in}{3.096292in}}%
\pgfpathlineto{\pgfqpoint{3.458970in}{3.100239in}}%
\pgfpathlineto{\pgfqpoint{3.465928in}{3.108143in}}%
\pgfpathlineto{\pgfqpoint{3.476364in}{3.123915in}}%
\pgfpathlineto{\pgfqpoint{3.491350in}{3.145973in}}%
\pgfpathlineto{\pgfqpoint{3.498576in}{3.152734in}}%
\pgfpathlineto{\pgfqpoint{3.504195in}{3.155242in}}%
\pgfpathlineto{\pgfqpoint{3.509280in}{3.155213in}}%
\pgfpathlineto{\pgfqpoint{3.514632in}{3.152824in}}%
\pgfpathlineto{\pgfqpoint{3.520787in}{3.147344in}}%
\pgfpathlineto{\pgfqpoint{3.528815in}{3.136778in}}%
\pgfpathlineto{\pgfqpoint{3.554773in}{3.099878in}}%
\pgfpathlineto{\pgfqpoint{3.560928in}{3.095878in}}%
\pgfpathlineto{\pgfqpoint{3.566280in}{3.094930in}}%
\pgfpathlineto{\pgfqpoint{3.571365in}{3.096292in}}%
\pgfpathlineto{\pgfqpoint{3.576985in}{3.100239in}}%
\pgfpathlineto{\pgfqpoint{3.583943in}{3.108143in}}%
\pgfpathlineto{\pgfqpoint{3.594379in}{3.123915in}}%
\pgfpathlineto{\pgfqpoint{3.609365in}{3.145973in}}%
\pgfpathlineto{\pgfqpoint{3.616591in}{3.152734in}}%
\pgfpathlineto{\pgfqpoint{3.622210in}{3.155242in}}%
\pgfpathlineto{\pgfqpoint{3.627295in}{3.155213in}}%
\pgfpathlineto{\pgfqpoint{3.632647in}{3.152824in}}%
\pgfpathlineto{\pgfqpoint{3.638802in}{3.147344in}}%
\pgfpathlineto{\pgfqpoint{3.646830in}{3.136778in}}%
\pgfpathlineto{\pgfqpoint{3.672788in}{3.099878in}}%
\pgfpathlineto{\pgfqpoint{3.678943in}{3.095878in}}%
\pgfpathlineto{\pgfqpoint{3.684295in}{3.094930in}}%
\pgfpathlineto{\pgfqpoint{3.689380in}{3.096292in}}%
\pgfpathlineto{\pgfqpoint{3.695000in}{3.100239in}}%
\pgfpathlineto{\pgfqpoint{3.701958in}{3.108143in}}%
\pgfpathlineto{\pgfqpoint{3.712394in}{3.123915in}}%
\pgfpathlineto{\pgfqpoint{3.727380in}{3.145973in}}%
\pgfpathlineto{\pgfqpoint{3.734606in}{3.152734in}}%
\pgfpathlineto{\pgfqpoint{3.740226in}{3.155242in}}%
\pgfpathlineto{\pgfqpoint{3.745310in}{3.155213in}}%
\pgfpathlineto{\pgfqpoint{3.750662in}{3.152824in}}%
\pgfpathlineto{\pgfqpoint{3.756817in}{3.147344in}}%
\pgfpathlineto{\pgfqpoint{3.764845in}{3.136778in}}%
\pgfpathlineto{\pgfqpoint{3.790803in}{3.099878in}}%
\pgfpathlineto{\pgfqpoint{3.796958in}{3.095878in}}%
\pgfpathlineto{\pgfqpoint{3.802311in}{3.094930in}}%
\pgfpathlineto{\pgfqpoint{3.807395in}{3.096292in}}%
\pgfpathlineto{\pgfqpoint{3.813015in}{3.100239in}}%
\pgfpathlineto{\pgfqpoint{3.819973in}{3.108143in}}%
\pgfpathlineto{\pgfqpoint{3.830409in}{3.123915in}}%
\pgfpathlineto{\pgfqpoint{3.845395in}{3.145973in}}%
\pgfpathlineto{\pgfqpoint{3.852621in}{3.152734in}}%
\pgfpathlineto{\pgfqpoint{3.858241in}{3.155242in}}%
\pgfpathlineto{\pgfqpoint{3.863325in}{3.155213in}}%
\pgfpathlineto{\pgfqpoint{3.868677in}{3.152824in}}%
\pgfpathlineto{\pgfqpoint{3.874832in}{3.147344in}}%
\pgfpathlineto{\pgfqpoint{3.882860in}{3.136778in}}%
\pgfpathlineto{\pgfqpoint{3.908818in}{3.099878in}}%
\pgfpathlineto{\pgfqpoint{3.914973in}{3.095878in}}%
\pgfpathlineto{\pgfqpoint{3.920326in}{3.094930in}}%
\pgfpathlineto{\pgfqpoint{3.925410in}{3.096292in}}%
\pgfpathlineto{\pgfqpoint{3.931030in}{3.100239in}}%
\pgfpathlineto{\pgfqpoint{3.937988in}{3.108143in}}%
\pgfpathlineto{\pgfqpoint{3.948424in}{3.123915in}}%
\pgfpathlineto{\pgfqpoint{3.963410in}{3.145973in}}%
\pgfpathlineto{\pgfqpoint{3.970636in}{3.152734in}}%
\pgfpathlineto{\pgfqpoint{3.976256in}{3.155242in}}%
\pgfpathlineto{\pgfqpoint{3.981340in}{3.155213in}}%
\pgfpathlineto{\pgfqpoint{3.986692in}{3.152824in}}%
\pgfpathlineto{\pgfqpoint{3.992847in}{3.147344in}}%
\pgfpathlineto{\pgfqpoint{4.000876in}{3.136778in}}%
\pgfpathlineto{\pgfqpoint{4.026834in}{3.099878in}}%
\pgfpathlineto{\pgfqpoint{4.032988in}{3.095878in}}%
\pgfpathlineto{\pgfqpoint{4.038341in}{3.094930in}}%
\pgfpathlineto{\pgfqpoint{4.043425in}{3.096292in}}%
\pgfpathlineto{\pgfqpoint{4.049045in}{3.100239in}}%
\pgfpathlineto{\pgfqpoint{4.056003in}{3.108143in}}%
\pgfpathlineto{\pgfqpoint{4.066439in}{3.123915in}}%
\pgfpathlineto{\pgfqpoint{4.081425in}{3.145973in}}%
\pgfpathlineto{\pgfqpoint{4.088651in}{3.152734in}}%
\pgfpathlineto{\pgfqpoint{4.094271in}{3.155242in}}%
\pgfpathlineto{\pgfqpoint{4.099355in}{3.155213in}}%
\pgfpathlineto{\pgfqpoint{4.104707in}{3.152824in}}%
\pgfpathlineto{\pgfqpoint{4.110862in}{3.147344in}}%
\pgfpathlineto{\pgfqpoint{4.118891in}{3.136778in}}%
\pgfpathlineto{\pgfqpoint{4.144849in}{3.099878in}}%
\pgfpathlineto{\pgfqpoint{4.151004in}{3.095878in}}%
\pgfpathlineto{\pgfqpoint{4.156356in}{3.094930in}}%
\pgfpathlineto{\pgfqpoint{4.161440in}{3.096292in}}%
\pgfpathlineto{\pgfqpoint{4.167060in}{3.100239in}}%
\pgfpathlineto{\pgfqpoint{4.174018in}{3.108143in}}%
\pgfpathlineto{\pgfqpoint{4.184455in}{3.123915in}}%
\pgfpathlineto{\pgfqpoint{4.199441in}{3.145973in}}%
\pgfpathlineto{\pgfqpoint{4.206666in}{3.152734in}}%
\pgfpathlineto{\pgfqpoint{4.212286in}{3.155242in}}%
\pgfpathlineto{\pgfqpoint{4.217370in}{3.155213in}}%
\pgfpathlineto{\pgfqpoint{4.222722in}{3.152824in}}%
\pgfpathlineto{\pgfqpoint{4.228877in}{3.147344in}}%
\pgfpathlineto{\pgfqpoint{4.236906in}{3.136778in}}%
\pgfpathlineto{\pgfqpoint{4.262864in}{3.099878in}}%
\pgfpathlineto{\pgfqpoint{4.269019in}{3.095878in}}%
\pgfpathlineto{\pgfqpoint{4.274371in}{3.094930in}}%
\pgfpathlineto{\pgfqpoint{4.279455in}{3.096292in}}%
\pgfpathlineto{\pgfqpoint{4.285075in}{3.100239in}}%
\pgfpathlineto{\pgfqpoint{4.292033in}{3.108143in}}%
\pgfpathlineto{\pgfqpoint{4.302470in}{3.123915in}}%
\pgfpathlineto{\pgfqpoint{4.317456in}{3.145973in}}%
\pgfpathlineto{\pgfqpoint{4.324681in}{3.152734in}}%
\pgfpathlineto{\pgfqpoint{4.330301in}{3.155242in}}%
\pgfpathlineto{\pgfqpoint{4.335385in}{3.155213in}}%
\pgfpathlineto{\pgfqpoint{4.340737in}{3.152824in}}%
\pgfpathlineto{\pgfqpoint{4.346892in}{3.147344in}}%
\pgfpathlineto{\pgfqpoint{4.354921in}{3.136778in}}%
\pgfpathlineto{\pgfqpoint{4.380879in}{3.099878in}}%
\pgfpathlineto{\pgfqpoint{4.387034in}{3.095878in}}%
\pgfpathlineto{\pgfqpoint{4.392386in}{3.094930in}}%
\pgfpathlineto{\pgfqpoint{4.397470in}{3.096292in}}%
\pgfpathlineto{\pgfqpoint{4.403090in}{3.100239in}}%
\pgfpathlineto{\pgfqpoint{4.410048in}{3.108143in}}%
\pgfpathlineto{\pgfqpoint{4.420485in}{3.123915in}}%
\pgfpathlineto{\pgfqpoint{4.435471in}{3.145973in}}%
\pgfpathlineto{\pgfqpoint{4.442696in}{3.152734in}}%
\pgfpathlineto{\pgfqpoint{4.448316in}{3.155242in}}%
\pgfpathlineto{\pgfqpoint{4.453400in}{3.155213in}}%
\pgfpathlineto{\pgfqpoint{4.458753in}{3.152824in}}%
\pgfpathlineto{\pgfqpoint{4.464908in}{3.147344in}}%
\pgfpathlineto{\pgfqpoint{4.472936in}{3.136778in}}%
\pgfpathlineto{\pgfqpoint{4.498894in}{3.099878in}}%
\pgfpathlineto{\pgfqpoint{4.505049in}{3.095878in}}%
\pgfpathlineto{\pgfqpoint{4.510401in}{3.094930in}}%
\pgfpathlineto{\pgfqpoint{4.515485in}{3.096292in}}%
\pgfpathlineto{\pgfqpoint{4.521105in}{3.100239in}}%
\pgfpathlineto{\pgfqpoint{4.528063in}{3.108143in}}%
\pgfpathlineto{\pgfqpoint{4.538500in}{3.123915in}}%
\pgfpathlineto{\pgfqpoint{4.553486in}{3.145973in}}%
\pgfpathlineto{\pgfqpoint{4.560711in}{3.152734in}}%
\pgfpathlineto{\pgfqpoint{4.566331in}{3.155242in}}%
\pgfpathlineto{\pgfqpoint{4.571415in}{3.155213in}}%
\pgfpathlineto{\pgfqpoint{4.576768in}{3.152824in}}%
\pgfpathlineto{\pgfqpoint{4.582923in}{3.147344in}}%
\pgfpathlineto{\pgfqpoint{4.590951in}{3.136778in}}%
\pgfpathlineto{\pgfqpoint{4.616909in}{3.099878in}}%
\pgfpathlineto{\pgfqpoint{4.623064in}{3.095878in}}%
\pgfpathlineto{\pgfqpoint{4.628416in}{3.094930in}}%
\pgfpathlineto{\pgfqpoint{4.633500in}{3.096292in}}%
\pgfpathlineto{\pgfqpoint{4.639120in}{3.100239in}}%
\pgfpathlineto{\pgfqpoint{4.646078in}{3.108143in}}%
\pgfpathlineto{\pgfqpoint{4.656515in}{3.123915in}}%
\pgfpathlineto{\pgfqpoint{4.671501in}{3.145973in}}%
\pgfpathlineto{\pgfqpoint{4.678726in}{3.152734in}}%
\pgfpathlineto{\pgfqpoint{4.684346in}{3.155242in}}%
\pgfpathlineto{\pgfqpoint{4.689430in}{3.155213in}}%
\pgfpathlineto{\pgfqpoint{4.694783in}{3.152824in}}%
\pgfpathlineto{\pgfqpoint{4.700938in}{3.147344in}}%
\pgfpathlineto{\pgfqpoint{4.708966in}{3.136778in}}%
\pgfpathlineto{\pgfqpoint{4.734924in}{3.099878in}}%
\pgfpathlineto{\pgfqpoint{4.741079in}{3.095878in}}%
\pgfpathlineto{\pgfqpoint{4.746431in}{3.094930in}}%
\pgfpathlineto{\pgfqpoint{4.751515in}{3.096292in}}%
\pgfpathlineto{\pgfqpoint{4.757135in}{3.100239in}}%
\pgfpathlineto{\pgfqpoint{4.764093in}{3.108143in}}%
\pgfpathlineto{\pgfqpoint{4.774530in}{3.123915in}}%
\pgfpathlineto{\pgfqpoint{4.789516in}{3.145973in}}%
\pgfpathlineto{\pgfqpoint{4.796741in}{3.152734in}}%
\pgfpathlineto{\pgfqpoint{4.802361in}{3.155242in}}%
\pgfpathlineto{\pgfqpoint{4.807446in}{3.155213in}}%
\pgfpathlineto{\pgfqpoint{4.812798in}{3.152824in}}%
\pgfpathlineto{\pgfqpoint{4.818953in}{3.147344in}}%
\pgfpathlineto{\pgfqpoint{4.826981in}{3.136778in}}%
\pgfpathlineto{\pgfqpoint{4.852939in}{3.099878in}}%
\pgfpathlineto{\pgfqpoint{4.859094in}{3.095878in}}%
\pgfpathlineto{\pgfqpoint{4.864446in}{3.094930in}}%
\pgfpathlineto{\pgfqpoint{4.869531in}{3.096292in}}%
\pgfpathlineto{\pgfqpoint{4.875150in}{3.100239in}}%
\pgfpathlineto{\pgfqpoint{4.882108in}{3.108143in}}%
\pgfpathlineto{\pgfqpoint{4.892545in}{3.123915in}}%
\pgfpathlineto{\pgfqpoint{4.907531in}{3.145973in}}%
\pgfpathlineto{\pgfqpoint{4.914756in}{3.152734in}}%
\pgfpathlineto{\pgfqpoint{4.920376in}{3.155242in}}%
\pgfpathlineto{\pgfqpoint{4.925461in}{3.155213in}}%
\pgfpathlineto{\pgfqpoint{4.930813in}{3.152824in}}%
\pgfpathlineto{\pgfqpoint{4.936968in}{3.147344in}}%
\pgfpathlineto{\pgfqpoint{4.944996in}{3.136778in}}%
\pgfpathlineto{\pgfqpoint{4.970954in}{3.099878in}}%
\pgfpathlineto{\pgfqpoint{4.977109in}{3.095878in}}%
\pgfpathlineto{\pgfqpoint{4.982461in}{3.094930in}}%
\pgfpathlineto{\pgfqpoint{4.987546in}{3.096292in}}%
\pgfpathlineto{\pgfqpoint{4.993165in}{3.100239in}}%
\pgfpathlineto{\pgfqpoint{5.000123in}{3.108143in}}%
\pgfpathlineto{\pgfqpoint{5.010560in}{3.123915in}}%
\pgfpathlineto{\pgfqpoint{5.025546in}{3.145973in}}%
\pgfpathlineto{\pgfqpoint{5.032771in}{3.152734in}}%
\pgfpathlineto{\pgfqpoint{5.038391in}{3.155242in}}%
\pgfpathlineto{\pgfqpoint{5.043476in}{3.155213in}}%
\pgfpathlineto{\pgfqpoint{5.048828in}{3.152824in}}%
\pgfpathlineto{\pgfqpoint{5.054983in}{3.147344in}}%
\pgfpathlineto{\pgfqpoint{5.063011in}{3.136778in}}%
\pgfpathlineto{\pgfqpoint{5.088969in}{3.099878in}}%
\pgfpathlineto{\pgfqpoint{5.095124in}{3.095878in}}%
\pgfpathlineto{\pgfqpoint{5.100476in}{3.094930in}}%
\pgfpathlineto{\pgfqpoint{5.105561in}{3.096292in}}%
\pgfpathlineto{\pgfqpoint{5.111180in}{3.100239in}}%
\pgfpathlineto{\pgfqpoint{5.118138in}{3.108143in}}%
\pgfpathlineto{\pgfqpoint{5.128575in}{3.123915in}}%
\pgfpathlineto{\pgfqpoint{5.143561in}{3.145973in}}%
\pgfpathlineto{\pgfqpoint{5.150786in}{3.152734in}}%
\pgfpathlineto{\pgfqpoint{5.156406in}{3.155242in}}%
\pgfpathlineto{\pgfqpoint{5.161491in}{3.155213in}}%
\pgfpathlineto{\pgfqpoint{5.166843in}{3.152824in}}%
\pgfpathlineto{\pgfqpoint{5.172998in}{3.147344in}}%
\pgfpathlineto{\pgfqpoint{5.181026in}{3.136778in}}%
\pgfpathlineto{\pgfqpoint{5.206984in}{3.099878in}}%
\pgfpathlineto{\pgfqpoint{5.213139in}{3.095878in}}%
\pgfpathlineto{\pgfqpoint{5.218491in}{3.094930in}}%
\pgfpathlineto{\pgfqpoint{5.223576in}{3.096292in}}%
\pgfpathlineto{\pgfqpoint{5.229195in}{3.100239in}}%
\pgfpathlineto{\pgfqpoint{5.236153in}{3.108143in}}%
\pgfpathlineto{\pgfqpoint{5.246590in}{3.123915in}}%
\pgfpathlineto{\pgfqpoint{5.261576in}{3.145973in}}%
\pgfpathlineto{\pgfqpoint{5.268801in}{3.152734in}}%
\pgfpathlineto{\pgfqpoint{5.274421in}{3.155242in}}%
\pgfpathlineto{\pgfqpoint{5.279506in}{3.155213in}}%
\pgfpathlineto{\pgfqpoint{5.284858in}{3.152824in}}%
\pgfpathlineto{\pgfqpoint{5.291013in}{3.147344in}}%
\pgfpathlineto{\pgfqpoint{5.299041in}{3.136778in}}%
\pgfpathlineto{\pgfqpoint{5.324999in}{3.099878in}}%
\pgfpathlineto{\pgfqpoint{5.331154in}{3.095878in}}%
\pgfpathlineto{\pgfqpoint{5.336506in}{3.094930in}}%
\pgfpathlineto{\pgfqpoint{5.341591in}{3.096292in}}%
\pgfpathlineto{\pgfqpoint{5.347210in}{3.100239in}}%
\pgfpathlineto{\pgfqpoint{5.354168in}{3.108143in}}%
\pgfpathlineto{\pgfqpoint{5.364605in}{3.123915in}}%
\pgfpathlineto{\pgfqpoint{5.379591in}{3.145973in}}%
\pgfpathlineto{\pgfqpoint{5.386816in}{3.152734in}}%
\pgfpathlineto{\pgfqpoint{5.392436in}{3.155242in}}%
\pgfpathlineto{\pgfqpoint{5.397521in}{3.155213in}}%
\pgfpathlineto{\pgfqpoint{5.402873in}{3.152824in}}%
\pgfpathlineto{\pgfqpoint{5.409028in}{3.147344in}}%
\pgfpathlineto{\pgfqpoint{5.417056in}{3.136778in}}%
\pgfpathlineto{\pgfqpoint{5.443014in}{3.099878in}}%
\pgfpathlineto{\pgfqpoint{5.449169in}{3.095878in}}%
\pgfpathlineto{\pgfqpoint{5.454521in}{3.094930in}}%
\pgfpathlineto{\pgfqpoint{5.459606in}{3.096292in}}%
\pgfpathlineto{\pgfqpoint{5.465226in}{3.100239in}}%
\pgfpathlineto{\pgfqpoint{5.472183in}{3.108143in}}%
\pgfpathlineto{\pgfqpoint{5.482620in}{3.123915in}}%
\pgfpathlineto{\pgfqpoint{5.497606in}{3.145973in}}%
\pgfpathlineto{\pgfqpoint{5.504831in}{3.152734in}}%
\pgfpathlineto{\pgfqpoint{5.510451in}{3.155242in}}%
\pgfpathlineto{\pgfqpoint{5.515536in}{3.155213in}}%
\pgfpathlineto{\pgfqpoint{5.520888in}{3.152824in}}%
\pgfpathlineto{\pgfqpoint{5.527043in}{3.147344in}}%
\pgfpathlineto{\pgfqpoint{5.535071in}{3.136778in}}%
\pgfpathlineto{\pgfqpoint{5.561029in}{3.099878in}}%
\pgfpathlineto{\pgfqpoint{5.567184in}{3.095878in}}%
\pgfpathlineto{\pgfqpoint{5.572536in}{3.094930in}}%
\pgfpathlineto{\pgfqpoint{5.577621in}{3.096292in}}%
\pgfpathlineto{\pgfqpoint{5.583241in}{3.100239in}}%
\pgfpathlineto{\pgfqpoint{5.590198in}{3.108143in}}%
\pgfpathlineto{\pgfqpoint{5.600635in}{3.123915in}}%
\pgfpathlineto{\pgfqpoint{5.615621in}{3.145973in}}%
\pgfpathlineto{\pgfqpoint{5.622847in}{3.152734in}}%
\pgfpathlineto{\pgfqpoint{5.628466in}{3.155242in}}%
\pgfpathlineto{\pgfqpoint{5.633551in}{3.155213in}}%
\pgfpathlineto{\pgfqpoint{5.638903in}{3.152824in}}%
\pgfpathlineto{\pgfqpoint{5.645058in}{3.147344in}}%
\pgfpathlineto{\pgfqpoint{5.653086in}{3.136778in}}%
\pgfpathlineto{\pgfqpoint{5.679044in}{3.099878in}}%
\pgfpathlineto{\pgfqpoint{5.685199in}{3.095878in}}%
\pgfpathlineto{\pgfqpoint{5.690551in}{3.094930in}}%
\pgfpathlineto{\pgfqpoint{5.695636in}{3.096292in}}%
\pgfpathlineto{\pgfqpoint{5.701256in}{3.100239in}}%
\pgfpathlineto{\pgfqpoint{5.708213in}{3.108143in}}%
\pgfpathlineto{\pgfqpoint{5.718650in}{3.123915in}}%
\pgfpathlineto{\pgfqpoint{5.733636in}{3.145973in}}%
\pgfpathlineto{\pgfqpoint{5.740862in}{3.152734in}}%
\pgfpathlineto{\pgfqpoint{5.746481in}{3.155242in}}%
\pgfpathlineto{\pgfqpoint{5.751566in}{3.155213in}}%
\pgfpathlineto{\pgfqpoint{5.756918in}{3.152824in}}%
\pgfpathlineto{\pgfqpoint{5.763073in}{3.147344in}}%
\pgfpathlineto{\pgfqpoint{5.771101in}{3.136778in}}%
\pgfpathlineto{\pgfqpoint{5.797059in}{3.099878in}}%
\pgfpathlineto{\pgfqpoint{5.803214in}{3.095878in}}%
\pgfpathlineto{\pgfqpoint{5.808566in}{3.094930in}}%
\pgfpathlineto{\pgfqpoint{5.813651in}{3.096292in}}%
\pgfpathlineto{\pgfqpoint{5.819271in}{3.100239in}}%
\pgfpathlineto{\pgfqpoint{5.826228in}{3.108143in}}%
\pgfpathlineto{\pgfqpoint{5.836665in}{3.123915in}}%
\pgfpathlineto{\pgfqpoint{5.851651in}{3.145973in}}%
\pgfpathlineto{\pgfqpoint{5.858877in}{3.152734in}}%
\pgfpathlineto{\pgfqpoint{5.864496in}{3.155242in}}%
\pgfpathlineto{\pgfqpoint{5.869581in}{3.155213in}}%
\pgfpathlineto{\pgfqpoint{5.874933in}{3.152824in}}%
\pgfpathlineto{\pgfqpoint{5.881088in}{3.147344in}}%
\pgfpathlineto{\pgfqpoint{5.889116in}{3.136778in}}%
\pgfpathlineto{\pgfqpoint{5.915074in}{3.099878in}}%
\pgfpathlineto{\pgfqpoint{5.921229in}{3.095878in}}%
\pgfpathlineto{\pgfqpoint{5.926581in}{3.094930in}}%
\pgfpathlineto{\pgfqpoint{5.931666in}{3.096292in}}%
\pgfpathlineto{\pgfqpoint{5.937286in}{3.100239in}}%
\pgfpathlineto{\pgfqpoint{5.944244in}{3.108143in}}%
\pgfpathlineto{\pgfqpoint{5.954680in}{3.123915in}}%
\pgfpathlineto{\pgfqpoint{5.969666in}{3.145973in}}%
\pgfpathlineto{\pgfqpoint{5.976892in}{3.152734in}}%
\pgfpathlineto{\pgfqpoint{5.982511in}{3.155242in}}%
\pgfpathlineto{\pgfqpoint{5.987596in}{3.155213in}}%
\pgfpathlineto{\pgfqpoint{5.992948in}{3.152824in}}%
\pgfpathlineto{\pgfqpoint{5.999103in}{3.147344in}}%
\pgfpathlineto{\pgfqpoint{6.007131in}{3.136778in}}%
\pgfpathlineto{\pgfqpoint{6.033089in}{3.099878in}}%
\pgfpathlineto{\pgfqpoint{6.039244in}{3.095878in}}%
\pgfpathlineto{\pgfqpoint{6.044596in}{3.094930in}}%
\pgfpathlineto{\pgfqpoint{6.049681in}{3.096292in}}%
\pgfpathlineto{\pgfqpoint{6.055301in}{3.100239in}}%
\pgfpathlineto{\pgfqpoint{6.062259in}{3.108143in}}%
\pgfpathlineto{\pgfqpoint{6.072695in}{3.123915in}}%
\pgfpathlineto{\pgfqpoint{6.087681in}{3.145973in}}%
\pgfpathlineto{\pgfqpoint{6.094907in}{3.152734in}}%
\pgfpathlineto{\pgfqpoint{6.100527in}{3.155242in}}%
\pgfpathlineto{\pgfqpoint{6.105611in}{3.155213in}}%
\pgfpathlineto{\pgfqpoint{6.110963in}{3.152824in}}%
\pgfpathlineto{\pgfqpoint{6.117118in}{3.147344in}}%
\pgfpathlineto{\pgfqpoint{6.125146in}{3.136778in}}%
\pgfpathlineto{\pgfqpoint{6.151104in}{3.099878in}}%
\pgfpathlineto{\pgfqpoint{6.157259in}{3.095878in}}%
\pgfpathlineto{\pgfqpoint{6.162612in}{3.094930in}}%
\pgfpathlineto{\pgfqpoint{6.167696in}{3.096292in}}%
\pgfpathlineto{\pgfqpoint{6.173316in}{3.100239in}}%
\pgfpathlineto{\pgfqpoint{6.180274in}{3.108143in}}%
\pgfpathlineto{\pgfqpoint{6.190710in}{3.123915in}}%
\pgfpathlineto{\pgfqpoint{6.205696in}{3.145973in}}%
\pgfpathlineto{\pgfqpoint{6.212922in}{3.152734in}}%
\pgfpathlineto{\pgfqpoint{6.218542in}{3.155242in}}%
\pgfpathlineto{\pgfqpoint{6.223626in}{3.155213in}}%
\pgfpathlineto{\pgfqpoint{6.228978in}{3.152824in}}%
\pgfpathlineto{\pgfqpoint{6.235133in}{3.147344in}}%
\pgfpathlineto{\pgfqpoint{6.243161in}{3.136778in}}%
\pgfpathlineto{\pgfqpoint{6.269119in}{3.099878in}}%
\pgfpathlineto{\pgfqpoint{6.275274in}{3.095878in}}%
\pgfpathlineto{\pgfqpoint{6.280627in}{3.094930in}}%
\pgfpathlineto{\pgfqpoint{6.285711in}{3.096292in}}%
\pgfpathlineto{\pgfqpoint{6.291331in}{3.100239in}}%
\pgfpathlineto{\pgfqpoint{6.298289in}{3.108143in}}%
\pgfpathlineto{\pgfqpoint{6.308725in}{3.123915in}}%
\pgfpathlineto{\pgfqpoint{6.323711in}{3.145973in}}%
\pgfpathlineto{\pgfqpoint{6.330937in}{3.152734in}}%
\pgfpathlineto{\pgfqpoint{6.336557in}{3.155242in}}%
\pgfpathlineto{\pgfqpoint{6.341641in}{3.155213in}}%
\pgfpathlineto{\pgfqpoint{6.346993in}{3.152824in}}%
\pgfpathlineto{\pgfqpoint{6.353148in}{3.147344in}}%
\pgfpathlineto{\pgfqpoint{6.361177in}{3.136778in}}%
\pgfpathlineto{\pgfqpoint{6.387134in}{3.099878in}}%
\pgfpathlineto{\pgfqpoint{6.393289in}{3.095878in}}%
\pgfpathlineto{\pgfqpoint{6.398642in}{3.094930in}}%
\pgfpathlineto{\pgfqpoint{6.403726in}{3.096292in}}%
\pgfpathlineto{\pgfqpoint{6.409346in}{3.100239in}}%
\pgfpathlineto{\pgfqpoint{6.416304in}{3.108143in}}%
\pgfpathlineto{\pgfqpoint{6.426740in}{3.123915in}}%
\pgfpathlineto{\pgfqpoint{6.441726in}{3.145973in}}%
\pgfpathlineto{\pgfqpoint{6.448952in}{3.152734in}}%
\pgfpathlineto{\pgfqpoint{6.454572in}{3.155242in}}%
\pgfpathlineto{\pgfqpoint{6.459656in}{3.155213in}}%
\pgfpathlineto{\pgfqpoint{6.465008in}{3.152824in}}%
\pgfpathlineto{\pgfqpoint{6.471163in}{3.147344in}}%
\pgfpathlineto{\pgfqpoint{6.479192in}{3.136778in}}%
\pgfpathlineto{\pgfqpoint{6.505150in}{3.099878in}}%
\pgfpathlineto{\pgfqpoint{6.511305in}{3.095878in}}%
\pgfpathlineto{\pgfqpoint{6.516657in}{3.094930in}}%
\pgfpathlineto{\pgfqpoint{6.521741in}{3.096292in}}%
\pgfpathlineto{\pgfqpoint{6.527361in}{3.100239in}}%
\pgfpathlineto{\pgfqpoint{6.534319in}{3.108143in}}%
\pgfpathlineto{\pgfqpoint{6.544755in}{3.123915in}}%
\pgfpathlineto{\pgfqpoint{6.559742in}{3.145973in}}%
\pgfpathlineto{\pgfqpoint{6.566967in}{3.152734in}}%
\pgfpathlineto{\pgfqpoint{6.572587in}{3.155242in}}%
\pgfpathlineto{\pgfqpoint{6.577671in}{3.155213in}}%
\pgfpathlineto{\pgfqpoint{6.583023in}{3.152824in}}%
\pgfpathlineto{\pgfqpoint{6.589178in}{3.147344in}}%
\pgfpathlineto{\pgfqpoint{6.597207in}{3.136778in}}%
\pgfpathlineto{\pgfqpoint{6.623165in}{3.099878in}}%
\pgfpathlineto{\pgfqpoint{6.629320in}{3.095878in}}%
\pgfpathlineto{\pgfqpoint{6.634672in}{3.094930in}}%
\pgfpathlineto{\pgfqpoint{6.639756in}{3.096292in}}%
\pgfpathlineto{\pgfqpoint{6.645376in}{3.100239in}}%
\pgfpathlineto{\pgfqpoint{6.652334in}{3.108143in}}%
\pgfpathlineto{\pgfqpoint{6.662771in}{3.123915in}}%
\pgfpathlineto{\pgfqpoint{6.663306in}{3.124778in}}%
\pgfpathlineto{\pgfqpoint{6.663306in}{3.124778in}}%
\pgfusepath{stroke}%
\end{pgfscope}%
\begin{pgfscope}%
\pgfpathrectangle{\pgfqpoint{0.467797in}{2.292089in}}{\pgfqpoint{6.490533in}{1.666241in}}%
\pgfusepath{clip}%
\pgfsetrectcap%
\pgfsetroundjoin%
\pgfsetlinewidth{1.505625pt}%
\definecolor{currentstroke}{rgb}{0.839216,0.152941,0.156863}%
\pgfsetstrokecolor{currentstroke}%
\pgfsetdash{}{0pt}%
\pgfpathmoveto{\pgfqpoint{0.762821in}{3.125209in}}%
\pgfpathlineto{\pgfqpoint{0.776201in}{3.144724in}}%
\pgfpathlineto{\pgfqpoint{0.782891in}{3.150860in}}%
\pgfpathlineto{\pgfqpoint{0.788244in}{3.153094in}}%
\pgfpathlineto{\pgfqpoint{0.793061in}{3.152866in}}%
\pgfpathlineto{\pgfqpoint{0.798145in}{3.150341in}}%
\pgfpathlineto{\pgfqpoint{0.804033in}{3.144776in}}%
\pgfpathlineto{\pgfqpoint{0.812061in}{3.133775in}}%
\pgfpathlineto{\pgfqpoint{0.833202in}{3.103144in}}%
\pgfpathlineto{\pgfqpoint{0.839357in}{3.098509in}}%
\pgfpathlineto{\pgfqpoint{0.844709in}{3.097158in}}%
\pgfpathlineto{\pgfqpoint{0.849526in}{3.098198in}}%
\pgfpathlineto{\pgfqpoint{0.854878in}{3.101759in}}%
\pgfpathlineto{\pgfqpoint{0.861301in}{3.108896in}}%
\pgfpathlineto{\pgfqpoint{0.871202in}{3.123771in}}%
\pgfpathlineto{\pgfqpoint{0.885385in}{3.144621in}}%
\pgfpathlineto{\pgfqpoint{0.892343in}{3.150975in}}%
\pgfpathlineto{\pgfqpoint{0.897695in}{3.153124in}}%
\pgfpathlineto{\pgfqpoint{0.902512in}{3.152816in}}%
\pgfpathlineto{\pgfqpoint{0.907597in}{3.150211in}}%
\pgfpathlineto{\pgfqpoint{0.913484in}{3.144569in}}%
\pgfpathlineto{\pgfqpoint{0.921512in}{3.133501in}}%
\pgfpathlineto{\pgfqpoint{0.942386in}{3.103233in}}%
\pgfpathlineto{\pgfqpoint{0.948808in}{3.098422in}}%
\pgfpathlineto{\pgfqpoint{0.953893in}{3.097159in}}%
\pgfpathlineto{\pgfqpoint{0.958710in}{3.098160in}}%
\pgfpathlineto{\pgfqpoint{0.964062in}{3.101680in}}%
\pgfpathlineto{\pgfqpoint{0.970485in}{3.108780in}}%
\pgfpathlineto{\pgfqpoint{0.980386in}{3.123628in}}%
\pgfpathlineto{\pgfqpoint{0.994837in}{3.144827in}}%
\pgfpathlineto{\pgfqpoint{1.001527in}{3.150918in}}%
\pgfpathlineto{\pgfqpoint{1.006879in}{3.153109in}}%
\pgfpathlineto{\pgfqpoint{1.011696in}{3.152841in}}%
\pgfpathlineto{\pgfqpoint{1.016781in}{3.150276in}}%
\pgfpathlineto{\pgfqpoint{1.022668in}{3.144672in}}%
\pgfpathlineto{\pgfqpoint{1.030696in}{3.133638in}}%
\pgfpathlineto{\pgfqpoint{1.051837in}{3.103056in}}%
\pgfpathlineto{\pgfqpoint{1.057992in}{3.098465in}}%
\pgfpathlineto{\pgfqpoint{1.063077in}{3.097161in}}%
\pgfpathlineto{\pgfqpoint{1.067894in}{3.098122in}}%
\pgfpathlineto{\pgfqpoint{1.073246in}{3.101602in}}%
\pgfpathlineto{\pgfqpoint{1.079669in}{3.108663in}}%
\pgfpathlineto{\pgfqpoint{1.089302in}{3.123053in}}%
\pgfpathlineto{\pgfqpoint{1.104021in}{3.144724in}}%
\pgfpathlineto{\pgfqpoint{1.110711in}{3.150860in}}%
\pgfpathlineto{\pgfqpoint{1.116063in}{3.153094in}}%
\pgfpathlineto{\pgfqpoint{1.120880in}{3.152866in}}%
\pgfpathlineto{\pgfqpoint{1.125965in}{3.150341in}}%
\pgfpathlineto{\pgfqpoint{1.131852in}{3.144776in}}%
\pgfpathlineto{\pgfqpoint{1.139880in}{3.133775in}}%
\pgfpathlineto{\pgfqpoint{1.161021in}{3.103144in}}%
\pgfpathlineto{\pgfqpoint{1.167176in}{3.098509in}}%
\pgfpathlineto{\pgfqpoint{1.172528in}{3.097158in}}%
\pgfpathlineto{\pgfqpoint{1.177345in}{3.098198in}}%
\pgfpathlineto{\pgfqpoint{1.182698in}{3.101759in}}%
\pgfpathlineto{\pgfqpoint{1.189120in}{3.108896in}}%
\pgfpathlineto{\pgfqpoint{1.199022in}{3.123771in}}%
\pgfpathlineto{\pgfqpoint{1.213205in}{3.144621in}}%
\pgfpathlineto{\pgfqpoint{1.220163in}{3.150975in}}%
\pgfpathlineto{\pgfqpoint{1.225515in}{3.153124in}}%
\pgfpathlineto{\pgfqpoint{1.230332in}{3.152816in}}%
\pgfpathlineto{\pgfqpoint{1.235416in}{3.150211in}}%
\pgfpathlineto{\pgfqpoint{1.241304in}{3.144569in}}%
\pgfpathlineto{\pgfqpoint{1.249332in}{3.133501in}}%
\pgfpathlineto{\pgfqpoint{1.270205in}{3.103233in}}%
\pgfpathlineto{\pgfqpoint{1.276628in}{3.098422in}}%
\pgfpathlineto{\pgfqpoint{1.281712in}{3.097159in}}%
\pgfpathlineto{\pgfqpoint{1.286529in}{3.098160in}}%
\pgfpathlineto{\pgfqpoint{1.291882in}{3.101680in}}%
\pgfpathlineto{\pgfqpoint{1.298304in}{3.108780in}}%
\pgfpathlineto{\pgfqpoint{1.308206in}{3.123628in}}%
\pgfpathlineto{\pgfqpoint{1.322656in}{3.144827in}}%
\pgfpathlineto{\pgfqpoint{1.329347in}{3.150918in}}%
\pgfpathlineto{\pgfqpoint{1.334699in}{3.153109in}}%
\pgfpathlineto{\pgfqpoint{1.339516in}{3.152841in}}%
\pgfpathlineto{\pgfqpoint{1.344600in}{3.150276in}}%
\pgfpathlineto{\pgfqpoint{1.350488in}{3.144672in}}%
\pgfpathlineto{\pgfqpoint{1.358516in}{3.133638in}}%
\pgfpathlineto{\pgfqpoint{1.379657in}{3.103056in}}%
\pgfpathlineto{\pgfqpoint{1.385812in}{3.098465in}}%
\pgfpathlineto{\pgfqpoint{1.390896in}{3.097161in}}%
\pgfpathlineto{\pgfqpoint{1.395713in}{3.098122in}}%
\pgfpathlineto{\pgfqpoint{1.401066in}{3.101602in}}%
\pgfpathlineto{\pgfqpoint{1.407488in}{3.108663in}}%
\pgfpathlineto{\pgfqpoint{1.417122in}{3.123053in}}%
\pgfpathlineto{\pgfqpoint{1.431840in}{3.144724in}}%
\pgfpathlineto{\pgfqpoint{1.438531in}{3.150860in}}%
\pgfpathlineto{\pgfqpoint{1.443883in}{3.153094in}}%
\pgfpathlineto{\pgfqpoint{1.448700in}{3.152866in}}%
\pgfpathlineto{\pgfqpoint{1.453784in}{3.150341in}}%
\pgfpathlineto{\pgfqpoint{1.459672in}{3.144776in}}%
\pgfpathlineto{\pgfqpoint{1.467700in}{3.133775in}}%
\pgfpathlineto{\pgfqpoint{1.488841in}{3.103144in}}%
\pgfpathlineto{\pgfqpoint{1.494996in}{3.098509in}}%
\pgfpathlineto{\pgfqpoint{1.500348in}{3.097158in}}%
\pgfpathlineto{\pgfqpoint{1.505165in}{3.098198in}}%
\pgfpathlineto{\pgfqpoint{1.510517in}{3.101759in}}%
\pgfpathlineto{\pgfqpoint{1.516940in}{3.108896in}}%
\pgfpathlineto{\pgfqpoint{1.526841in}{3.123771in}}%
\pgfpathlineto{\pgfqpoint{1.541024in}{3.144621in}}%
\pgfpathlineto{\pgfqpoint{1.547982in}{3.150975in}}%
\pgfpathlineto{\pgfqpoint{1.553334in}{3.153124in}}%
\pgfpathlineto{\pgfqpoint{1.558151in}{3.152816in}}%
\pgfpathlineto{\pgfqpoint{1.563236in}{3.150211in}}%
\pgfpathlineto{\pgfqpoint{1.569123in}{3.144569in}}%
\pgfpathlineto{\pgfqpoint{1.577152in}{3.133501in}}%
\pgfpathlineto{\pgfqpoint{1.598025in}{3.103233in}}%
\pgfpathlineto{\pgfqpoint{1.604448in}{3.098422in}}%
\pgfpathlineto{\pgfqpoint{1.609532in}{3.097159in}}%
\pgfpathlineto{\pgfqpoint{1.614349in}{3.098160in}}%
\pgfpathlineto{\pgfqpoint{1.619701in}{3.101680in}}%
\pgfpathlineto{\pgfqpoint{1.626124in}{3.108780in}}%
\pgfpathlineto{\pgfqpoint{1.636025in}{3.123628in}}%
\pgfpathlineto{\pgfqpoint{1.650476in}{3.144827in}}%
\pgfpathlineto{\pgfqpoint{1.657166in}{3.150918in}}%
\pgfpathlineto{\pgfqpoint{1.662518in}{3.153109in}}%
\pgfpathlineto{\pgfqpoint{1.667335in}{3.152841in}}%
\pgfpathlineto{\pgfqpoint{1.672420in}{3.150276in}}%
\pgfpathlineto{\pgfqpoint{1.678307in}{3.144672in}}%
\pgfpathlineto{\pgfqpoint{1.686335in}{3.133638in}}%
\pgfpathlineto{\pgfqpoint{1.707477in}{3.103056in}}%
\pgfpathlineto{\pgfqpoint{1.713631in}{3.098465in}}%
\pgfpathlineto{\pgfqpoint{1.718716in}{3.097161in}}%
\pgfpathlineto{\pgfqpoint{1.723533in}{3.098122in}}%
\pgfpathlineto{\pgfqpoint{1.728885in}{3.101602in}}%
\pgfpathlineto{\pgfqpoint{1.735308in}{3.108663in}}%
\pgfpathlineto{\pgfqpoint{1.744942in}{3.123053in}}%
\pgfpathlineto{\pgfqpoint{1.759660in}{3.144724in}}%
\pgfpathlineto{\pgfqpoint{1.766350in}{3.150860in}}%
\pgfpathlineto{\pgfqpoint{1.771702in}{3.153094in}}%
\pgfpathlineto{\pgfqpoint{1.776519in}{3.152866in}}%
\pgfpathlineto{\pgfqpoint{1.781604in}{3.150341in}}%
\pgfpathlineto{\pgfqpoint{1.787491in}{3.144776in}}%
\pgfpathlineto{\pgfqpoint{1.795519in}{3.133775in}}%
\pgfpathlineto{\pgfqpoint{1.816661in}{3.103144in}}%
\pgfpathlineto{\pgfqpoint{1.822815in}{3.098509in}}%
\pgfpathlineto{\pgfqpoint{1.828168in}{3.097158in}}%
\pgfpathlineto{\pgfqpoint{1.832985in}{3.098198in}}%
\pgfpathlineto{\pgfqpoint{1.838337in}{3.101759in}}%
\pgfpathlineto{\pgfqpoint{1.844759in}{3.108896in}}%
\pgfpathlineto{\pgfqpoint{1.854661in}{3.123771in}}%
\pgfpathlineto{\pgfqpoint{1.868844in}{3.144621in}}%
\pgfpathlineto{\pgfqpoint{1.875802in}{3.150975in}}%
\pgfpathlineto{\pgfqpoint{1.881154in}{3.153124in}}%
\pgfpathlineto{\pgfqpoint{1.885971in}{3.152816in}}%
\pgfpathlineto{\pgfqpoint{1.891055in}{3.150211in}}%
\pgfpathlineto{\pgfqpoint{1.896943in}{3.144569in}}%
\pgfpathlineto{\pgfqpoint{1.904971in}{3.133501in}}%
\pgfpathlineto{\pgfqpoint{1.925844in}{3.103233in}}%
\pgfpathlineto{\pgfqpoint{1.932267in}{3.098422in}}%
\pgfpathlineto{\pgfqpoint{1.937352in}{3.097159in}}%
\pgfpathlineto{\pgfqpoint{1.942169in}{3.098160in}}%
\pgfpathlineto{\pgfqpoint{1.947521in}{3.101680in}}%
\pgfpathlineto{\pgfqpoint{1.953943in}{3.108780in}}%
\pgfpathlineto{\pgfqpoint{1.963845in}{3.123628in}}%
\pgfpathlineto{\pgfqpoint{1.978296in}{3.144827in}}%
\pgfpathlineto{\pgfqpoint{1.984986in}{3.150918in}}%
\pgfpathlineto{\pgfqpoint{1.990338in}{3.153109in}}%
\pgfpathlineto{\pgfqpoint{1.995155in}{3.152841in}}%
\pgfpathlineto{\pgfqpoint{2.000239in}{3.150276in}}%
\pgfpathlineto{\pgfqpoint{2.006127in}{3.144672in}}%
\pgfpathlineto{\pgfqpoint{2.014155in}{3.133638in}}%
\pgfpathlineto{\pgfqpoint{2.035296in}{3.103056in}}%
\pgfpathlineto{\pgfqpoint{2.041451in}{3.098465in}}%
\pgfpathlineto{\pgfqpoint{2.046536in}{3.097161in}}%
\pgfpathlineto{\pgfqpoint{2.051353in}{3.098122in}}%
\pgfpathlineto{\pgfqpoint{2.056705in}{3.101602in}}%
\pgfpathlineto{\pgfqpoint{2.063127in}{3.108663in}}%
\pgfpathlineto{\pgfqpoint{2.072761in}{3.123053in}}%
\pgfpathlineto{\pgfqpoint{2.087480in}{3.144724in}}%
\pgfpathlineto{\pgfqpoint{2.094170in}{3.150860in}}%
\pgfpathlineto{\pgfqpoint{2.099522in}{3.153094in}}%
\pgfpathlineto{\pgfqpoint{2.104339in}{3.152866in}}%
\pgfpathlineto{\pgfqpoint{2.109423in}{3.150341in}}%
\pgfpathlineto{\pgfqpoint{2.115311in}{3.144776in}}%
\pgfpathlineto{\pgfqpoint{2.123339in}{3.133775in}}%
\pgfpathlineto{\pgfqpoint{2.144480in}{3.103144in}}%
\pgfpathlineto{\pgfqpoint{2.150635in}{3.098509in}}%
\pgfpathlineto{\pgfqpoint{2.155987in}{3.097158in}}%
\pgfpathlineto{\pgfqpoint{2.160804in}{3.098198in}}%
\pgfpathlineto{\pgfqpoint{2.166156in}{3.101759in}}%
\pgfpathlineto{\pgfqpoint{2.172579in}{3.108896in}}%
\pgfpathlineto{\pgfqpoint{2.182480in}{3.123771in}}%
\pgfpathlineto{\pgfqpoint{2.196664in}{3.144621in}}%
\pgfpathlineto{\pgfqpoint{2.203621in}{3.150975in}}%
\pgfpathlineto{\pgfqpoint{2.208974in}{3.153124in}}%
\pgfpathlineto{\pgfqpoint{2.213791in}{3.152816in}}%
\pgfpathlineto{\pgfqpoint{2.218875in}{3.150211in}}%
\pgfpathlineto{\pgfqpoint{2.224762in}{3.144569in}}%
\pgfpathlineto{\pgfqpoint{2.232791in}{3.133501in}}%
\pgfpathlineto{\pgfqpoint{2.253664in}{3.103233in}}%
\pgfpathlineto{\pgfqpoint{2.260087in}{3.098422in}}%
\pgfpathlineto{\pgfqpoint{2.265171in}{3.097159in}}%
\pgfpathlineto{\pgfqpoint{2.269988in}{3.098160in}}%
\pgfpathlineto{\pgfqpoint{2.275340in}{3.101680in}}%
\pgfpathlineto{\pgfqpoint{2.281763in}{3.108780in}}%
\pgfpathlineto{\pgfqpoint{2.291664in}{3.123628in}}%
\pgfpathlineto{\pgfqpoint{2.306115in}{3.144827in}}%
\pgfpathlineto{\pgfqpoint{2.312805in}{3.150918in}}%
\pgfpathlineto{\pgfqpoint{2.318158in}{3.153109in}}%
\pgfpathlineto{\pgfqpoint{2.322975in}{3.152841in}}%
\pgfpathlineto{\pgfqpoint{2.328059in}{3.150276in}}%
\pgfpathlineto{\pgfqpoint{2.333946in}{3.144672in}}%
\pgfpathlineto{\pgfqpoint{2.341975in}{3.133638in}}%
\pgfpathlineto{\pgfqpoint{2.363116in}{3.103056in}}%
\pgfpathlineto{\pgfqpoint{2.369271in}{3.098465in}}%
\pgfpathlineto{\pgfqpoint{2.374355in}{3.097161in}}%
\pgfpathlineto{\pgfqpoint{2.379172in}{3.098122in}}%
\pgfpathlineto{\pgfqpoint{2.384524in}{3.101602in}}%
\pgfpathlineto{\pgfqpoint{2.390947in}{3.108663in}}%
\pgfpathlineto{\pgfqpoint{2.400581in}{3.123053in}}%
\pgfpathlineto{\pgfqpoint{2.415299in}{3.144724in}}%
\pgfpathlineto{\pgfqpoint{2.421989in}{3.150860in}}%
\pgfpathlineto{\pgfqpoint{2.427342in}{3.153094in}}%
\pgfpathlineto{\pgfqpoint{2.432158in}{3.152866in}}%
\pgfpathlineto{\pgfqpoint{2.437243in}{3.150341in}}%
\pgfpathlineto{\pgfqpoint{2.443130in}{3.144776in}}%
\pgfpathlineto{\pgfqpoint{2.451159in}{3.133775in}}%
\pgfpathlineto{\pgfqpoint{2.472300in}{3.103144in}}%
\pgfpathlineto{\pgfqpoint{2.478455in}{3.098509in}}%
\pgfpathlineto{\pgfqpoint{2.483807in}{3.097158in}}%
\pgfpathlineto{\pgfqpoint{2.488624in}{3.098198in}}%
\pgfpathlineto{\pgfqpoint{2.493976in}{3.101759in}}%
\pgfpathlineto{\pgfqpoint{2.500398in}{3.108896in}}%
\pgfpathlineto{\pgfqpoint{2.510300in}{3.123771in}}%
\pgfpathlineto{\pgfqpoint{2.524483in}{3.144621in}}%
\pgfpathlineto{\pgfqpoint{2.531441in}{3.150975in}}%
\pgfpathlineto{\pgfqpoint{2.536793in}{3.153124in}}%
\pgfpathlineto{\pgfqpoint{2.541610in}{3.152816in}}%
\pgfpathlineto{\pgfqpoint{2.546695in}{3.150211in}}%
\pgfpathlineto{\pgfqpoint{2.552582in}{3.144569in}}%
\pgfpathlineto{\pgfqpoint{2.560610in}{3.133501in}}%
\pgfpathlineto{\pgfqpoint{2.581484in}{3.103233in}}%
\pgfpathlineto{\pgfqpoint{2.587906in}{3.098422in}}%
\pgfpathlineto{\pgfqpoint{2.592991in}{3.097159in}}%
\pgfpathlineto{\pgfqpoint{2.597808in}{3.098160in}}%
\pgfpathlineto{\pgfqpoint{2.603160in}{3.101680in}}%
\pgfpathlineto{\pgfqpoint{2.609582in}{3.108780in}}%
\pgfpathlineto{\pgfqpoint{2.619484in}{3.123628in}}%
\pgfpathlineto{\pgfqpoint{2.633935in}{3.144827in}}%
\pgfpathlineto{\pgfqpoint{2.640625in}{3.150918in}}%
\pgfpathlineto{\pgfqpoint{2.645977in}{3.153109in}}%
\pgfpathlineto{\pgfqpoint{2.650794in}{3.152841in}}%
\pgfpathlineto{\pgfqpoint{2.655879in}{3.150276in}}%
\pgfpathlineto{\pgfqpoint{2.661766in}{3.144672in}}%
\pgfpathlineto{\pgfqpoint{2.669794in}{3.133638in}}%
\pgfpathlineto{\pgfqpoint{2.690935in}{3.103056in}}%
\pgfpathlineto{\pgfqpoint{2.697090in}{3.098465in}}%
\pgfpathlineto{\pgfqpoint{2.702175in}{3.097161in}}%
\pgfpathlineto{\pgfqpoint{2.706992in}{3.098122in}}%
\pgfpathlineto{\pgfqpoint{2.712344in}{3.101602in}}%
\pgfpathlineto{\pgfqpoint{2.718766in}{3.108663in}}%
\pgfpathlineto{\pgfqpoint{2.728400in}{3.123053in}}%
\pgfpathlineto{\pgfqpoint{2.743119in}{3.144724in}}%
\pgfpathlineto{\pgfqpoint{2.749809in}{3.150860in}}%
\pgfpathlineto{\pgfqpoint{2.755161in}{3.153094in}}%
\pgfpathlineto{\pgfqpoint{2.759978in}{3.152866in}}%
\pgfpathlineto{\pgfqpoint{2.765063in}{3.150341in}}%
\pgfpathlineto{\pgfqpoint{2.770950in}{3.144776in}}%
\pgfpathlineto{\pgfqpoint{2.778978in}{3.133775in}}%
\pgfpathlineto{\pgfqpoint{2.800119in}{3.103144in}}%
\pgfpathlineto{\pgfqpoint{2.806274in}{3.098509in}}%
\pgfpathlineto{\pgfqpoint{2.811626in}{3.097158in}}%
\pgfpathlineto{\pgfqpoint{2.816443in}{3.098198in}}%
\pgfpathlineto{\pgfqpoint{2.821795in}{3.101759in}}%
\pgfpathlineto{\pgfqpoint{2.828218in}{3.108896in}}%
\pgfpathlineto{\pgfqpoint{2.838120in}{3.123771in}}%
\pgfpathlineto{\pgfqpoint{2.852303in}{3.144621in}}%
\pgfpathlineto{\pgfqpoint{2.859261in}{3.150975in}}%
\pgfpathlineto{\pgfqpoint{2.864613in}{3.153124in}}%
\pgfpathlineto{\pgfqpoint{2.869430in}{3.152816in}}%
\pgfpathlineto{\pgfqpoint{2.874514in}{3.150211in}}%
\pgfpathlineto{\pgfqpoint{2.880402in}{3.144569in}}%
\pgfpathlineto{\pgfqpoint{2.888430in}{3.133501in}}%
\pgfpathlineto{\pgfqpoint{2.909303in}{3.103233in}}%
\pgfpathlineto{\pgfqpoint{2.915726in}{3.098422in}}%
\pgfpathlineto{\pgfqpoint{2.920810in}{3.097159in}}%
\pgfpathlineto{\pgfqpoint{2.925627in}{3.098160in}}%
\pgfpathlineto{\pgfqpoint{2.930979in}{3.101680in}}%
\pgfpathlineto{\pgfqpoint{2.937402in}{3.108780in}}%
\pgfpathlineto{\pgfqpoint{2.947304in}{3.123628in}}%
\pgfpathlineto{\pgfqpoint{2.961754in}{3.144827in}}%
\pgfpathlineto{\pgfqpoint{2.968445in}{3.150918in}}%
\pgfpathlineto{\pgfqpoint{2.973797in}{3.153109in}}%
\pgfpathlineto{\pgfqpoint{2.978614in}{3.152841in}}%
\pgfpathlineto{\pgfqpoint{2.983698in}{3.150276in}}%
\pgfpathlineto{\pgfqpoint{2.989586in}{3.144672in}}%
\pgfpathlineto{\pgfqpoint{2.997614in}{3.133638in}}%
\pgfpathlineto{\pgfqpoint{3.018755in}{3.103056in}}%
\pgfpathlineto{\pgfqpoint{3.024910in}{3.098465in}}%
\pgfpathlineto{\pgfqpoint{3.029994in}{3.097161in}}%
\pgfpathlineto{\pgfqpoint{3.034811in}{3.098122in}}%
\pgfpathlineto{\pgfqpoint{3.040163in}{3.101602in}}%
\pgfpathlineto{\pgfqpoint{3.046586in}{3.108663in}}%
\pgfpathlineto{\pgfqpoint{3.056220in}{3.123053in}}%
\pgfpathlineto{\pgfqpoint{3.070938in}{3.144724in}}%
\pgfpathlineto{\pgfqpoint{3.077629in}{3.150860in}}%
\pgfpathlineto{\pgfqpoint{3.082981in}{3.153094in}}%
\pgfpathlineto{\pgfqpoint{3.087798in}{3.152866in}}%
\pgfpathlineto{\pgfqpoint{3.092882in}{3.150341in}}%
\pgfpathlineto{\pgfqpoint{3.098770in}{3.144776in}}%
\pgfpathlineto{\pgfqpoint{3.106798in}{3.133775in}}%
\pgfpathlineto{\pgfqpoint{3.127939in}{3.103144in}}%
\pgfpathlineto{\pgfqpoint{3.134094in}{3.098509in}}%
\pgfpathlineto{\pgfqpoint{3.139446in}{3.097158in}}%
\pgfpathlineto{\pgfqpoint{3.144263in}{3.098198in}}%
\pgfpathlineto{\pgfqpoint{3.149615in}{3.101759in}}%
\pgfpathlineto{\pgfqpoint{3.156038in}{3.108896in}}%
\pgfpathlineto{\pgfqpoint{3.165939in}{3.123771in}}%
\pgfpathlineto{\pgfqpoint{3.180122in}{3.144621in}}%
\pgfpathlineto{\pgfqpoint{3.187080in}{3.150975in}}%
\pgfpathlineto{\pgfqpoint{3.192432in}{3.153124in}}%
\pgfpathlineto{\pgfqpoint{3.197249in}{3.152816in}}%
\pgfpathlineto{\pgfqpoint{3.202334in}{3.150211in}}%
\pgfpathlineto{\pgfqpoint{3.208221in}{3.144569in}}%
\pgfpathlineto{\pgfqpoint{3.216249in}{3.133501in}}%
\pgfpathlineto{\pgfqpoint{3.237123in}{3.103233in}}%
\pgfpathlineto{\pgfqpoint{3.243545in}{3.098422in}}%
\pgfpathlineto{\pgfqpoint{3.248630in}{3.097159in}}%
\pgfpathlineto{\pgfqpoint{3.253447in}{3.098160in}}%
\pgfpathlineto{\pgfqpoint{3.258799in}{3.101680in}}%
\pgfpathlineto{\pgfqpoint{3.265222in}{3.108780in}}%
\pgfpathlineto{\pgfqpoint{3.275123in}{3.123628in}}%
\pgfpathlineto{\pgfqpoint{3.289574in}{3.144827in}}%
\pgfpathlineto{\pgfqpoint{3.296264in}{3.150918in}}%
\pgfpathlineto{\pgfqpoint{3.301616in}{3.153109in}}%
\pgfpathlineto{\pgfqpoint{3.306433in}{3.152841in}}%
\pgfpathlineto{\pgfqpoint{3.311518in}{3.150276in}}%
\pgfpathlineto{\pgfqpoint{3.317405in}{3.144672in}}%
\pgfpathlineto{\pgfqpoint{3.325433in}{3.133638in}}%
\pgfpathlineto{\pgfqpoint{3.346574in}{3.103056in}}%
\pgfpathlineto{\pgfqpoint{3.352729in}{3.098465in}}%
\pgfpathlineto{\pgfqpoint{3.357814in}{3.097161in}}%
\pgfpathlineto{\pgfqpoint{3.362631in}{3.098122in}}%
\pgfpathlineto{\pgfqpoint{3.367983in}{3.101602in}}%
\pgfpathlineto{\pgfqpoint{3.374406in}{3.108663in}}%
\pgfpathlineto{\pgfqpoint{3.384040in}{3.123053in}}%
\pgfpathlineto{\pgfqpoint{3.398758in}{3.144724in}}%
\pgfpathlineto{\pgfqpoint{3.405448in}{3.150860in}}%
\pgfpathlineto{\pgfqpoint{3.410800in}{3.153094in}}%
\pgfpathlineto{\pgfqpoint{3.415617in}{3.152866in}}%
\pgfpathlineto{\pgfqpoint{3.420702in}{3.150341in}}%
\pgfpathlineto{\pgfqpoint{3.426589in}{3.144776in}}%
\pgfpathlineto{\pgfqpoint{3.434617in}{3.133775in}}%
\pgfpathlineto{\pgfqpoint{3.455758in}{3.103144in}}%
\pgfpathlineto{\pgfqpoint{3.461913in}{3.098509in}}%
\pgfpathlineto{\pgfqpoint{3.467266in}{3.097158in}}%
\pgfpathlineto{\pgfqpoint{3.472082in}{3.098198in}}%
\pgfpathlineto{\pgfqpoint{3.477435in}{3.101759in}}%
\pgfpathlineto{\pgfqpoint{3.483857in}{3.108896in}}%
\pgfpathlineto{\pgfqpoint{3.493759in}{3.123771in}}%
\pgfpathlineto{\pgfqpoint{3.507942in}{3.144621in}}%
\pgfpathlineto{\pgfqpoint{3.514900in}{3.150975in}}%
\pgfpathlineto{\pgfqpoint{3.520252in}{3.153124in}}%
\pgfpathlineto{\pgfqpoint{3.525069in}{3.152816in}}%
\pgfpathlineto{\pgfqpoint{3.530153in}{3.150211in}}%
\pgfpathlineto{\pgfqpoint{3.536041in}{3.144569in}}%
\pgfpathlineto{\pgfqpoint{3.544069in}{3.133501in}}%
\pgfpathlineto{\pgfqpoint{3.564942in}{3.103233in}}%
\pgfpathlineto{\pgfqpoint{3.571365in}{3.098422in}}%
\pgfpathlineto{\pgfqpoint{3.576450in}{3.097159in}}%
\pgfpathlineto{\pgfqpoint{3.581266in}{3.098160in}}%
\pgfpathlineto{\pgfqpoint{3.586619in}{3.101680in}}%
\pgfpathlineto{\pgfqpoint{3.593041in}{3.108780in}}%
\pgfpathlineto{\pgfqpoint{3.602943in}{3.123628in}}%
\pgfpathlineto{\pgfqpoint{3.617394in}{3.144827in}}%
\pgfpathlineto{\pgfqpoint{3.624084in}{3.150918in}}%
\pgfpathlineto{\pgfqpoint{3.629436in}{3.153109in}}%
\pgfpathlineto{\pgfqpoint{3.634253in}{3.152841in}}%
\pgfpathlineto{\pgfqpoint{3.639337in}{3.150276in}}%
\pgfpathlineto{\pgfqpoint{3.645225in}{3.144672in}}%
\pgfpathlineto{\pgfqpoint{3.653253in}{3.133638in}}%
\pgfpathlineto{\pgfqpoint{3.674394in}{3.103056in}}%
\pgfpathlineto{\pgfqpoint{3.680549in}{3.098465in}}%
\pgfpathlineto{\pgfqpoint{3.685634in}{3.097161in}}%
\pgfpathlineto{\pgfqpoint{3.690450in}{3.098122in}}%
\pgfpathlineto{\pgfqpoint{3.695803in}{3.101602in}}%
\pgfpathlineto{\pgfqpoint{3.702225in}{3.108663in}}%
\pgfpathlineto{\pgfqpoint{3.711859in}{3.123053in}}%
\pgfpathlineto{\pgfqpoint{3.726578in}{3.144724in}}%
\pgfpathlineto{\pgfqpoint{3.733268in}{3.150860in}}%
\pgfpathlineto{\pgfqpoint{3.738620in}{3.153094in}}%
\pgfpathlineto{\pgfqpoint{3.743437in}{3.152866in}}%
\pgfpathlineto{\pgfqpoint{3.748521in}{3.150341in}}%
\pgfpathlineto{\pgfqpoint{3.754409in}{3.144776in}}%
\pgfpathlineto{\pgfqpoint{3.762437in}{3.133775in}}%
\pgfpathlineto{\pgfqpoint{3.783578in}{3.103144in}}%
\pgfpathlineto{\pgfqpoint{3.789733in}{3.098509in}}%
\pgfpathlineto{\pgfqpoint{3.795085in}{3.097158in}}%
\pgfpathlineto{\pgfqpoint{3.799902in}{3.098198in}}%
\pgfpathlineto{\pgfqpoint{3.805254in}{3.101759in}}%
\pgfpathlineto{\pgfqpoint{3.811677in}{3.108896in}}%
\pgfpathlineto{\pgfqpoint{3.821578in}{3.123771in}}%
\pgfpathlineto{\pgfqpoint{3.835762in}{3.144621in}}%
\pgfpathlineto{\pgfqpoint{3.842719in}{3.150975in}}%
\pgfpathlineto{\pgfqpoint{3.848071in}{3.153124in}}%
\pgfpathlineto{\pgfqpoint{3.852888in}{3.152816in}}%
\pgfpathlineto{\pgfqpoint{3.857973in}{3.150211in}}%
\pgfpathlineto{\pgfqpoint{3.863860in}{3.144569in}}%
\pgfpathlineto{\pgfqpoint{3.871889in}{3.133501in}}%
\pgfpathlineto{\pgfqpoint{3.892762in}{3.103233in}}%
\pgfpathlineto{\pgfqpoint{3.899185in}{3.098422in}}%
\pgfpathlineto{\pgfqpoint{3.904269in}{3.097159in}}%
\pgfpathlineto{\pgfqpoint{3.909086in}{3.098160in}}%
\pgfpathlineto{\pgfqpoint{3.914438in}{3.101680in}}%
\pgfpathlineto{\pgfqpoint{3.920861in}{3.108780in}}%
\pgfpathlineto{\pgfqpoint{3.930762in}{3.123628in}}%
\pgfpathlineto{\pgfqpoint{3.945213in}{3.144827in}}%
\pgfpathlineto{\pgfqpoint{3.951903in}{3.150918in}}%
\pgfpathlineto{\pgfqpoint{3.957255in}{3.153109in}}%
\pgfpathlineto{\pgfqpoint{3.962072in}{3.152841in}}%
\pgfpathlineto{\pgfqpoint{3.967157in}{3.150276in}}%
\pgfpathlineto{\pgfqpoint{3.973044in}{3.144672in}}%
\pgfpathlineto{\pgfqpoint{3.981073in}{3.133638in}}%
\pgfpathlineto{\pgfqpoint{4.002214in}{3.103056in}}%
\pgfpathlineto{\pgfqpoint{4.008369in}{3.098465in}}%
\pgfpathlineto{\pgfqpoint{4.013453in}{3.097161in}}%
\pgfpathlineto{\pgfqpoint{4.018270in}{3.098122in}}%
\pgfpathlineto{\pgfqpoint{4.023622in}{3.101602in}}%
\pgfpathlineto{\pgfqpoint{4.030045in}{3.108663in}}%
\pgfpathlineto{\pgfqpoint{4.039679in}{3.123053in}}%
\pgfpathlineto{\pgfqpoint{4.054397in}{3.144724in}}%
\pgfpathlineto{\pgfqpoint{4.061087in}{3.150860in}}%
\pgfpathlineto{\pgfqpoint{4.066439in}{3.153094in}}%
\pgfpathlineto{\pgfqpoint{4.071256in}{3.152866in}}%
\pgfpathlineto{\pgfqpoint{4.076341in}{3.150341in}}%
\pgfpathlineto{\pgfqpoint{4.082228in}{3.144776in}}%
\pgfpathlineto{\pgfqpoint{4.090257in}{3.133775in}}%
\pgfpathlineto{\pgfqpoint{4.111398in}{3.103144in}}%
\pgfpathlineto{\pgfqpoint{4.117553in}{3.098509in}}%
\pgfpathlineto{\pgfqpoint{4.122905in}{3.097158in}}%
\pgfpathlineto{\pgfqpoint{4.127722in}{3.098198in}}%
\pgfpathlineto{\pgfqpoint{4.133074in}{3.101759in}}%
\pgfpathlineto{\pgfqpoint{4.139496in}{3.108896in}}%
\pgfpathlineto{\pgfqpoint{4.149398in}{3.123771in}}%
\pgfpathlineto{\pgfqpoint{4.163581in}{3.144621in}}%
\pgfpathlineto{\pgfqpoint{4.170539in}{3.150975in}}%
\pgfpathlineto{\pgfqpoint{4.175891in}{3.153124in}}%
\pgfpathlineto{\pgfqpoint{4.180708in}{3.152816in}}%
\pgfpathlineto{\pgfqpoint{4.185793in}{3.150211in}}%
\pgfpathlineto{\pgfqpoint{4.191680in}{3.144569in}}%
\pgfpathlineto{\pgfqpoint{4.199708in}{3.133501in}}%
\pgfpathlineto{\pgfqpoint{4.220582in}{3.103233in}}%
\pgfpathlineto{\pgfqpoint{4.227004in}{3.098422in}}%
\pgfpathlineto{\pgfqpoint{4.232089in}{3.097159in}}%
\pgfpathlineto{\pgfqpoint{4.236906in}{3.098160in}}%
\pgfpathlineto{\pgfqpoint{4.242258in}{3.101680in}}%
\pgfpathlineto{\pgfqpoint{4.248680in}{3.108780in}}%
\pgfpathlineto{\pgfqpoint{4.258582in}{3.123628in}}%
\pgfpathlineto{\pgfqpoint{4.273033in}{3.144827in}}%
\pgfpathlineto{\pgfqpoint{4.279723in}{3.150918in}}%
\pgfpathlineto{\pgfqpoint{4.285075in}{3.153109in}}%
\pgfpathlineto{\pgfqpoint{4.289892in}{3.152841in}}%
\pgfpathlineto{\pgfqpoint{4.294977in}{3.150276in}}%
\pgfpathlineto{\pgfqpoint{4.300864in}{3.144672in}}%
\pgfpathlineto{\pgfqpoint{4.308892in}{3.133638in}}%
\pgfpathlineto{\pgfqpoint{4.330033in}{3.103056in}}%
\pgfpathlineto{\pgfqpoint{4.336188in}{3.098465in}}%
\pgfpathlineto{\pgfqpoint{4.341273in}{3.097161in}}%
\pgfpathlineto{\pgfqpoint{4.346090in}{3.098122in}}%
\pgfpathlineto{\pgfqpoint{4.351442in}{3.101602in}}%
\pgfpathlineto{\pgfqpoint{4.357864in}{3.108663in}}%
\pgfpathlineto{\pgfqpoint{4.367498in}{3.123053in}}%
\pgfpathlineto{\pgfqpoint{4.382217in}{3.144724in}}%
\pgfpathlineto{\pgfqpoint{4.388907in}{3.150860in}}%
\pgfpathlineto{\pgfqpoint{4.394259in}{3.153094in}}%
\pgfpathlineto{\pgfqpoint{4.399076in}{3.152866in}}%
\pgfpathlineto{\pgfqpoint{4.404161in}{3.150341in}}%
\pgfpathlineto{\pgfqpoint{4.410048in}{3.144776in}}%
\pgfpathlineto{\pgfqpoint{4.418076in}{3.133775in}}%
\pgfpathlineto{\pgfqpoint{4.439217in}{3.103144in}}%
\pgfpathlineto{\pgfqpoint{4.445372in}{3.098509in}}%
\pgfpathlineto{\pgfqpoint{4.450724in}{3.097158in}}%
\pgfpathlineto{\pgfqpoint{4.455541in}{3.098198in}}%
\pgfpathlineto{\pgfqpoint{4.460893in}{3.101759in}}%
\pgfpathlineto{\pgfqpoint{4.467316in}{3.108896in}}%
\pgfpathlineto{\pgfqpoint{4.477217in}{3.123771in}}%
\pgfpathlineto{\pgfqpoint{4.491401in}{3.144621in}}%
\pgfpathlineto{\pgfqpoint{4.498358in}{3.150975in}}%
\pgfpathlineto{\pgfqpoint{4.503711in}{3.153124in}}%
\pgfpathlineto{\pgfqpoint{4.508528in}{3.152816in}}%
\pgfpathlineto{\pgfqpoint{4.513612in}{3.150211in}}%
\pgfpathlineto{\pgfqpoint{4.519500in}{3.144569in}}%
\pgfpathlineto{\pgfqpoint{4.527528in}{3.133501in}}%
\pgfpathlineto{\pgfqpoint{4.548401in}{3.103233in}}%
\pgfpathlineto{\pgfqpoint{4.554824in}{3.098422in}}%
\pgfpathlineto{\pgfqpoint{4.559908in}{3.097159in}}%
\pgfpathlineto{\pgfqpoint{4.564725in}{3.098160in}}%
\pgfpathlineto{\pgfqpoint{4.570077in}{3.101680in}}%
\pgfpathlineto{\pgfqpoint{4.576500in}{3.108780in}}%
\pgfpathlineto{\pgfqpoint{4.586401in}{3.123628in}}%
\pgfpathlineto{\pgfqpoint{4.600852in}{3.144827in}}%
\pgfpathlineto{\pgfqpoint{4.607542in}{3.150918in}}%
\pgfpathlineto{\pgfqpoint{4.612895in}{3.153109in}}%
\pgfpathlineto{\pgfqpoint{4.617712in}{3.152841in}}%
\pgfpathlineto{\pgfqpoint{4.622796in}{3.150276in}}%
\pgfpathlineto{\pgfqpoint{4.628683in}{3.144672in}}%
\pgfpathlineto{\pgfqpoint{4.636712in}{3.133638in}}%
\pgfpathlineto{\pgfqpoint{4.657853in}{3.103056in}}%
\pgfpathlineto{\pgfqpoint{4.664008in}{3.098465in}}%
\pgfpathlineto{\pgfqpoint{4.669092in}{3.097161in}}%
\pgfpathlineto{\pgfqpoint{4.673909in}{3.098122in}}%
\pgfpathlineto{\pgfqpoint{4.679261in}{3.101602in}}%
\pgfpathlineto{\pgfqpoint{4.685684in}{3.108663in}}%
\pgfpathlineto{\pgfqpoint{4.695318in}{3.123053in}}%
\pgfpathlineto{\pgfqpoint{4.710036in}{3.144724in}}%
\pgfpathlineto{\pgfqpoint{4.716726in}{3.150860in}}%
\pgfpathlineto{\pgfqpoint{4.722079in}{3.153094in}}%
\pgfpathlineto{\pgfqpoint{4.726896in}{3.152866in}}%
\pgfpathlineto{\pgfqpoint{4.731980in}{3.150341in}}%
\pgfpathlineto{\pgfqpoint{4.737867in}{3.144776in}}%
\pgfpathlineto{\pgfqpoint{4.745896in}{3.133775in}}%
\pgfpathlineto{\pgfqpoint{4.767037in}{3.103144in}}%
\pgfpathlineto{\pgfqpoint{4.773192in}{3.098509in}}%
\pgfpathlineto{\pgfqpoint{4.778544in}{3.097158in}}%
\pgfpathlineto{\pgfqpoint{4.783361in}{3.098198in}}%
\pgfpathlineto{\pgfqpoint{4.788713in}{3.101759in}}%
\pgfpathlineto{\pgfqpoint{4.795136in}{3.108896in}}%
\pgfpathlineto{\pgfqpoint{4.805037in}{3.123771in}}%
\pgfpathlineto{\pgfqpoint{4.819220in}{3.144621in}}%
\pgfpathlineto{\pgfqpoint{4.826178in}{3.150975in}}%
\pgfpathlineto{\pgfqpoint{4.831530in}{3.153124in}}%
\pgfpathlineto{\pgfqpoint{4.836347in}{3.152816in}}%
\pgfpathlineto{\pgfqpoint{4.841432in}{3.150211in}}%
\pgfpathlineto{\pgfqpoint{4.847319in}{3.144569in}}%
\pgfpathlineto{\pgfqpoint{4.855347in}{3.133501in}}%
\pgfpathlineto{\pgfqpoint{4.876221in}{3.103233in}}%
\pgfpathlineto{\pgfqpoint{4.882643in}{3.098422in}}%
\pgfpathlineto{\pgfqpoint{4.887728in}{3.097159in}}%
\pgfpathlineto{\pgfqpoint{4.892545in}{3.098160in}}%
\pgfpathlineto{\pgfqpoint{4.897897in}{3.101680in}}%
\pgfpathlineto{\pgfqpoint{4.904320in}{3.108780in}}%
\pgfpathlineto{\pgfqpoint{4.914221in}{3.123628in}}%
\pgfpathlineto{\pgfqpoint{4.928672in}{3.144827in}}%
\pgfpathlineto{\pgfqpoint{4.935362in}{3.150918in}}%
\pgfpathlineto{\pgfqpoint{4.940714in}{3.153109in}}%
\pgfpathlineto{\pgfqpoint{4.945531in}{3.152841in}}%
\pgfpathlineto{\pgfqpoint{4.950616in}{3.150276in}}%
\pgfpathlineto{\pgfqpoint{4.956503in}{3.144672in}}%
\pgfpathlineto{\pgfqpoint{4.964531in}{3.133638in}}%
\pgfpathlineto{\pgfqpoint{4.985672in}{3.103056in}}%
\pgfpathlineto{\pgfqpoint{4.991827in}{3.098465in}}%
\pgfpathlineto{\pgfqpoint{4.996912in}{3.097161in}}%
\pgfpathlineto{\pgfqpoint{5.001729in}{3.098122in}}%
\pgfpathlineto{\pgfqpoint{5.007081in}{3.101602in}}%
\pgfpathlineto{\pgfqpoint{5.013504in}{3.108663in}}%
\pgfpathlineto{\pgfqpoint{5.023137in}{3.123053in}}%
\pgfpathlineto{\pgfqpoint{5.037856in}{3.144724in}}%
\pgfpathlineto{\pgfqpoint{5.044546in}{3.150860in}}%
\pgfpathlineto{\pgfqpoint{5.049898in}{3.153094in}}%
\pgfpathlineto{\pgfqpoint{5.054715in}{3.152866in}}%
\pgfpathlineto{\pgfqpoint{5.059800in}{3.150341in}}%
\pgfpathlineto{\pgfqpoint{5.065687in}{3.144776in}}%
\pgfpathlineto{\pgfqpoint{5.073715in}{3.133775in}}%
\pgfpathlineto{\pgfqpoint{5.094856in}{3.103144in}}%
\pgfpathlineto{\pgfqpoint{5.101011in}{3.098509in}}%
\pgfpathlineto{\pgfqpoint{5.106363in}{3.097158in}}%
\pgfpathlineto{\pgfqpoint{5.111180in}{3.098198in}}%
\pgfpathlineto{\pgfqpoint{5.116533in}{3.101759in}}%
\pgfpathlineto{\pgfqpoint{5.122955in}{3.108896in}}%
\pgfpathlineto{\pgfqpoint{5.132857in}{3.123771in}}%
\pgfpathlineto{\pgfqpoint{5.147040in}{3.144621in}}%
\pgfpathlineto{\pgfqpoint{5.153998in}{3.150975in}}%
\pgfpathlineto{\pgfqpoint{5.159350in}{3.153124in}}%
\pgfpathlineto{\pgfqpoint{5.164167in}{3.152816in}}%
\pgfpathlineto{\pgfqpoint{5.169251in}{3.150211in}}%
\pgfpathlineto{\pgfqpoint{5.175139in}{3.144569in}}%
\pgfpathlineto{\pgfqpoint{5.183167in}{3.133501in}}%
\pgfpathlineto{\pgfqpoint{5.204040in}{3.103233in}}%
\pgfpathlineto{\pgfqpoint{5.210463in}{3.098422in}}%
\pgfpathlineto{\pgfqpoint{5.215547in}{3.097159in}}%
\pgfpathlineto{\pgfqpoint{5.220364in}{3.098160in}}%
\pgfpathlineto{\pgfqpoint{5.225717in}{3.101680in}}%
\pgfpathlineto{\pgfqpoint{5.232139in}{3.108780in}}%
\pgfpathlineto{\pgfqpoint{5.242041in}{3.123628in}}%
\pgfpathlineto{\pgfqpoint{5.256491in}{3.144827in}}%
\pgfpathlineto{\pgfqpoint{5.263182in}{3.150918in}}%
\pgfpathlineto{\pgfqpoint{5.268534in}{3.153109in}}%
\pgfpathlineto{\pgfqpoint{5.273351in}{3.152841in}}%
\pgfpathlineto{\pgfqpoint{5.278435in}{3.150276in}}%
\pgfpathlineto{\pgfqpoint{5.284323in}{3.144672in}}%
\pgfpathlineto{\pgfqpoint{5.292351in}{3.133638in}}%
\pgfpathlineto{\pgfqpoint{5.313492in}{3.103056in}}%
\pgfpathlineto{\pgfqpoint{5.319647in}{3.098465in}}%
\pgfpathlineto{\pgfqpoint{5.324731in}{3.097161in}}%
\pgfpathlineto{\pgfqpoint{5.329548in}{3.098122in}}%
\pgfpathlineto{\pgfqpoint{5.334901in}{3.101602in}}%
\pgfpathlineto{\pgfqpoint{5.341323in}{3.108663in}}%
\pgfpathlineto{\pgfqpoint{5.350957in}{3.123053in}}%
\pgfpathlineto{\pgfqpoint{5.365675in}{3.144724in}}%
\pgfpathlineto{\pgfqpoint{5.372366in}{3.150860in}}%
\pgfpathlineto{\pgfqpoint{5.377718in}{3.153094in}}%
\pgfpathlineto{\pgfqpoint{5.382535in}{3.152866in}}%
\pgfpathlineto{\pgfqpoint{5.387619in}{3.150341in}}%
\pgfpathlineto{\pgfqpoint{5.393507in}{3.144776in}}%
\pgfpathlineto{\pgfqpoint{5.401535in}{3.133775in}}%
\pgfpathlineto{\pgfqpoint{5.422676in}{3.103144in}}%
\pgfpathlineto{\pgfqpoint{5.428831in}{3.098509in}}%
\pgfpathlineto{\pgfqpoint{5.434183in}{3.097158in}}%
\pgfpathlineto{\pgfqpoint{5.439000in}{3.098198in}}%
\pgfpathlineto{\pgfqpoint{5.444352in}{3.101759in}}%
\pgfpathlineto{\pgfqpoint{5.450775in}{3.108896in}}%
\pgfpathlineto{\pgfqpoint{5.460676in}{3.123771in}}%
\pgfpathlineto{\pgfqpoint{5.474859in}{3.144621in}}%
\pgfpathlineto{\pgfqpoint{5.481817in}{3.150975in}}%
\pgfpathlineto{\pgfqpoint{5.487169in}{3.153124in}}%
\pgfpathlineto{\pgfqpoint{5.491986in}{3.152816in}}%
\pgfpathlineto{\pgfqpoint{5.497071in}{3.150211in}}%
\pgfpathlineto{\pgfqpoint{5.502958in}{3.144569in}}%
\pgfpathlineto{\pgfqpoint{5.510986in}{3.133501in}}%
\pgfpathlineto{\pgfqpoint{5.531860in}{3.103233in}}%
\pgfpathlineto{\pgfqpoint{5.538282in}{3.098422in}}%
\pgfpathlineto{\pgfqpoint{5.543367in}{3.097159in}}%
\pgfpathlineto{\pgfqpoint{5.548184in}{3.098160in}}%
\pgfpathlineto{\pgfqpoint{5.553536in}{3.101680in}}%
\pgfpathlineto{\pgfqpoint{5.559959in}{3.108780in}}%
\pgfpathlineto{\pgfqpoint{5.569860in}{3.123628in}}%
\pgfpathlineto{\pgfqpoint{5.584311in}{3.144827in}}%
\pgfpathlineto{\pgfqpoint{5.591001in}{3.150918in}}%
\pgfpathlineto{\pgfqpoint{5.596353in}{3.153109in}}%
\pgfpathlineto{\pgfqpoint{5.601170in}{3.152841in}}%
\pgfpathlineto{\pgfqpoint{5.606255in}{3.150276in}}%
\pgfpathlineto{\pgfqpoint{5.612142in}{3.144672in}}%
\pgfpathlineto{\pgfqpoint{5.620170in}{3.133638in}}%
\pgfpathlineto{\pgfqpoint{5.641311in}{3.103056in}}%
\pgfpathlineto{\pgfqpoint{5.647466in}{3.098465in}}%
\pgfpathlineto{\pgfqpoint{5.652551in}{3.097161in}}%
\pgfpathlineto{\pgfqpoint{5.657368in}{3.098122in}}%
\pgfpathlineto{\pgfqpoint{5.662720in}{3.101602in}}%
\pgfpathlineto{\pgfqpoint{5.669143in}{3.108663in}}%
\pgfpathlineto{\pgfqpoint{5.678777in}{3.123053in}}%
\pgfpathlineto{\pgfqpoint{5.693495in}{3.144724in}}%
\pgfpathlineto{\pgfqpoint{5.700185in}{3.150860in}}%
\pgfpathlineto{\pgfqpoint{5.705537in}{3.153094in}}%
\pgfpathlineto{\pgfqpoint{5.710354in}{3.152866in}}%
\pgfpathlineto{\pgfqpoint{5.715439in}{3.150341in}}%
\pgfpathlineto{\pgfqpoint{5.721326in}{3.144776in}}%
\pgfpathlineto{\pgfqpoint{5.729354in}{3.133775in}}%
\pgfpathlineto{\pgfqpoint{5.750495in}{3.103144in}}%
\pgfpathlineto{\pgfqpoint{5.756650in}{3.098509in}}%
\pgfpathlineto{\pgfqpoint{5.762003in}{3.097158in}}%
\pgfpathlineto{\pgfqpoint{5.766820in}{3.098198in}}%
\pgfpathlineto{\pgfqpoint{5.772172in}{3.101759in}}%
\pgfpathlineto{\pgfqpoint{5.778594in}{3.108896in}}%
\pgfpathlineto{\pgfqpoint{5.788496in}{3.123771in}}%
\pgfpathlineto{\pgfqpoint{5.802679in}{3.144621in}}%
\pgfpathlineto{\pgfqpoint{5.809637in}{3.150975in}}%
\pgfpathlineto{\pgfqpoint{5.814989in}{3.153124in}}%
\pgfpathlineto{\pgfqpoint{5.819806in}{3.152816in}}%
\pgfpathlineto{\pgfqpoint{5.824890in}{3.150211in}}%
\pgfpathlineto{\pgfqpoint{5.830778in}{3.144569in}}%
\pgfpathlineto{\pgfqpoint{5.838806in}{3.133501in}}%
\pgfpathlineto{\pgfqpoint{5.859679in}{3.103233in}}%
\pgfpathlineto{\pgfqpoint{5.866102in}{3.098422in}}%
\pgfpathlineto{\pgfqpoint{5.871187in}{3.097159in}}%
\pgfpathlineto{\pgfqpoint{5.876004in}{3.098160in}}%
\pgfpathlineto{\pgfqpoint{5.881356in}{3.101680in}}%
\pgfpathlineto{\pgfqpoint{5.887778in}{3.108780in}}%
\pgfpathlineto{\pgfqpoint{5.897680in}{3.123628in}}%
\pgfpathlineto{\pgfqpoint{5.912131in}{3.144827in}}%
\pgfpathlineto{\pgfqpoint{5.918821in}{3.150918in}}%
\pgfpathlineto{\pgfqpoint{5.924173in}{3.153109in}}%
\pgfpathlineto{\pgfqpoint{5.928990in}{3.152841in}}%
\pgfpathlineto{\pgfqpoint{5.934074in}{3.150276in}}%
\pgfpathlineto{\pgfqpoint{5.939962in}{3.144672in}}%
\pgfpathlineto{\pgfqpoint{5.947990in}{3.133638in}}%
\pgfpathlineto{\pgfqpoint{5.969131in}{3.103056in}}%
\pgfpathlineto{\pgfqpoint{5.975286in}{3.098465in}}%
\pgfpathlineto{\pgfqpoint{5.980371in}{3.097161in}}%
\pgfpathlineto{\pgfqpoint{5.985188in}{3.098122in}}%
\pgfpathlineto{\pgfqpoint{5.990540in}{3.101602in}}%
\pgfpathlineto{\pgfqpoint{5.996962in}{3.108663in}}%
\pgfpathlineto{\pgfqpoint{6.006596in}{3.123053in}}%
\pgfpathlineto{\pgfqpoint{6.021315in}{3.144724in}}%
\pgfpathlineto{\pgfqpoint{6.028005in}{3.150860in}}%
\pgfpathlineto{\pgfqpoint{6.033357in}{3.153094in}}%
\pgfpathlineto{\pgfqpoint{6.038174in}{3.152866in}}%
\pgfpathlineto{\pgfqpoint{6.043258in}{3.150341in}}%
\pgfpathlineto{\pgfqpoint{6.049146in}{3.144776in}}%
\pgfpathlineto{\pgfqpoint{6.057174in}{3.133775in}}%
\pgfpathlineto{\pgfqpoint{6.078315in}{3.103144in}}%
\pgfpathlineto{\pgfqpoint{6.084470in}{3.098509in}}%
\pgfpathlineto{\pgfqpoint{6.089822in}{3.097158in}}%
\pgfpathlineto{\pgfqpoint{6.094639in}{3.098198in}}%
\pgfpathlineto{\pgfqpoint{6.099991in}{3.101759in}}%
\pgfpathlineto{\pgfqpoint{6.106414in}{3.108896in}}%
\pgfpathlineto{\pgfqpoint{6.116315in}{3.123771in}}%
\pgfpathlineto{\pgfqpoint{6.130499in}{3.144621in}}%
\pgfpathlineto{\pgfqpoint{6.137456in}{3.150975in}}%
\pgfpathlineto{\pgfqpoint{6.142809in}{3.153124in}}%
\pgfpathlineto{\pgfqpoint{6.147625in}{3.152816in}}%
\pgfpathlineto{\pgfqpoint{6.152710in}{3.150211in}}%
\pgfpathlineto{\pgfqpoint{6.158597in}{3.144569in}}%
\pgfpathlineto{\pgfqpoint{6.166626in}{3.133501in}}%
\pgfpathlineto{\pgfqpoint{6.187499in}{3.103233in}}%
\pgfpathlineto{\pgfqpoint{6.193922in}{3.098422in}}%
\pgfpathlineto{\pgfqpoint{6.199006in}{3.097159in}}%
\pgfpathlineto{\pgfqpoint{6.203823in}{3.098160in}}%
\pgfpathlineto{\pgfqpoint{6.209175in}{3.101680in}}%
\pgfpathlineto{\pgfqpoint{6.215598in}{3.108780in}}%
\pgfpathlineto{\pgfqpoint{6.225499in}{3.123628in}}%
\pgfpathlineto{\pgfqpoint{6.239950in}{3.144827in}}%
\pgfpathlineto{\pgfqpoint{6.246640in}{3.150918in}}%
\pgfpathlineto{\pgfqpoint{6.251993in}{3.153109in}}%
\pgfpathlineto{\pgfqpoint{6.256809in}{3.152841in}}%
\pgfpathlineto{\pgfqpoint{6.261894in}{3.150276in}}%
\pgfpathlineto{\pgfqpoint{6.267781in}{3.144672in}}%
\pgfpathlineto{\pgfqpoint{6.275810in}{3.133638in}}%
\pgfpathlineto{\pgfqpoint{6.296951in}{3.103056in}}%
\pgfpathlineto{\pgfqpoint{6.303106in}{3.098465in}}%
\pgfpathlineto{\pgfqpoint{6.308190in}{3.097161in}}%
\pgfpathlineto{\pgfqpoint{6.313007in}{3.098122in}}%
\pgfpathlineto{\pgfqpoint{6.318359in}{3.101602in}}%
\pgfpathlineto{\pgfqpoint{6.324782in}{3.108663in}}%
\pgfpathlineto{\pgfqpoint{6.334416in}{3.123053in}}%
\pgfpathlineto{\pgfqpoint{6.349134in}{3.144724in}}%
\pgfpathlineto{\pgfqpoint{6.355824in}{3.150860in}}%
\pgfpathlineto{\pgfqpoint{6.361177in}{3.153094in}}%
\pgfpathlineto{\pgfqpoint{6.365993in}{3.152866in}}%
\pgfpathlineto{\pgfqpoint{6.371078in}{3.150341in}}%
\pgfpathlineto{\pgfqpoint{6.376965in}{3.144776in}}%
\pgfpathlineto{\pgfqpoint{6.384994in}{3.133775in}}%
\pgfpathlineto{\pgfqpoint{6.406135in}{3.103144in}}%
\pgfpathlineto{\pgfqpoint{6.412290in}{3.098509in}}%
\pgfpathlineto{\pgfqpoint{6.417642in}{3.097158in}}%
\pgfpathlineto{\pgfqpoint{6.422459in}{3.098198in}}%
\pgfpathlineto{\pgfqpoint{6.427811in}{3.101759in}}%
\pgfpathlineto{\pgfqpoint{6.434233in}{3.108896in}}%
\pgfpathlineto{\pgfqpoint{6.444135in}{3.123771in}}%
\pgfpathlineto{\pgfqpoint{6.458318in}{3.144621in}}%
\pgfpathlineto{\pgfqpoint{6.465276in}{3.150975in}}%
\pgfpathlineto{\pgfqpoint{6.470628in}{3.153124in}}%
\pgfpathlineto{\pgfqpoint{6.475445in}{3.152816in}}%
\pgfpathlineto{\pgfqpoint{6.480530in}{3.150211in}}%
\pgfpathlineto{\pgfqpoint{6.486417in}{3.144569in}}%
\pgfpathlineto{\pgfqpoint{6.494445in}{3.133501in}}%
\pgfpathlineto{\pgfqpoint{6.515319in}{3.103233in}}%
\pgfpathlineto{\pgfqpoint{6.521741in}{3.098422in}}%
\pgfpathlineto{\pgfqpoint{6.526826in}{3.097159in}}%
\pgfpathlineto{\pgfqpoint{6.531643in}{3.098160in}}%
\pgfpathlineto{\pgfqpoint{6.536995in}{3.101680in}}%
\pgfpathlineto{\pgfqpoint{6.543417in}{3.108780in}}%
\pgfpathlineto{\pgfqpoint{6.553319in}{3.123628in}}%
\pgfpathlineto{\pgfqpoint{6.567770in}{3.144827in}}%
\pgfpathlineto{\pgfqpoint{6.574460in}{3.150918in}}%
\pgfpathlineto{\pgfqpoint{6.579812in}{3.153109in}}%
\pgfpathlineto{\pgfqpoint{6.584629in}{3.152841in}}%
\pgfpathlineto{\pgfqpoint{6.589714in}{3.150276in}}%
\pgfpathlineto{\pgfqpoint{6.595601in}{3.144672in}}%
\pgfpathlineto{\pgfqpoint{6.603629in}{3.133638in}}%
\pgfpathlineto{\pgfqpoint{6.624770in}{3.103056in}}%
\pgfpathlineto{\pgfqpoint{6.630925in}{3.098465in}}%
\pgfpathlineto{\pgfqpoint{6.636010in}{3.097161in}}%
\pgfpathlineto{\pgfqpoint{6.640827in}{3.098122in}}%
\pgfpathlineto{\pgfqpoint{6.646179in}{3.101602in}}%
\pgfpathlineto{\pgfqpoint{6.652601in}{3.108663in}}%
\pgfpathlineto{\pgfqpoint{6.662235in}{3.123053in}}%
\pgfpathlineto{\pgfqpoint{6.663306in}{3.124778in}}%
\pgfpathlineto{\pgfqpoint{6.663306in}{3.124778in}}%
\pgfusepath{stroke}%
\end{pgfscope}%
\begin{pgfscope}%
\pgfpathrectangle{\pgfqpoint{0.467797in}{2.292089in}}{\pgfqpoint{6.490533in}{1.666241in}}%
\pgfusepath{clip}%
\pgfsetrectcap%
\pgfsetroundjoin%
\pgfsetlinewidth{1.505625pt}%
\definecolor{currentstroke}{rgb}{0.580392,0.403922,0.741176}%
\pgfsetstrokecolor{currentstroke}%
\pgfsetdash{}{0pt}%
\pgfpathmoveto{\pgfqpoint{0.762821in}{3.125209in}}%
\pgfpathlineto{\pgfqpoint{0.775666in}{3.143822in}}%
\pgfpathlineto{\pgfqpoint{0.782089in}{3.149455in}}%
\pgfpathlineto{\pgfqpoint{0.787173in}{3.151268in}}%
\pgfpathlineto{\pgfqpoint{0.791990in}{3.150634in}}%
\pgfpathlineto{\pgfqpoint{0.797075in}{3.147547in}}%
\pgfpathlineto{\pgfqpoint{0.803230in}{3.140931in}}%
\pgfpathlineto{\pgfqpoint{0.812328in}{3.127402in}}%
\pgfpathlineto{\pgfqpoint{0.826512in}{3.106623in}}%
\pgfpathlineto{\pgfqpoint{0.832934in}{3.100978in}}%
\pgfpathlineto{\pgfqpoint{0.838019in}{3.099154in}}%
\pgfpathlineto{\pgfqpoint{0.842836in}{3.099776in}}%
\pgfpathlineto{\pgfqpoint{0.847920in}{3.102853in}}%
\pgfpathlineto{\pgfqpoint{0.854075in}{3.109458in}}%
\pgfpathlineto{\pgfqpoint{0.863174in}{3.122980in}}%
\pgfpathlineto{\pgfqpoint{0.877357in}{3.143770in}}%
\pgfpathlineto{\pgfqpoint{0.883780in}{3.149427in}}%
\pgfpathlineto{\pgfqpoint{0.888864in}{3.151263in}}%
\pgfpathlineto{\pgfqpoint{0.893681in}{3.150651in}}%
\pgfpathlineto{\pgfqpoint{0.898766in}{3.147585in}}%
\pgfpathlineto{\pgfqpoint{0.904921in}{3.140991in}}%
\pgfpathlineto{\pgfqpoint{0.914019in}{3.127476in}}%
\pgfpathlineto{\pgfqpoint{0.928203in}{3.106675in}}%
\pgfpathlineto{\pgfqpoint{0.934625in}{3.101006in}}%
\pgfpathlineto{\pgfqpoint{0.939710in}{3.099159in}}%
\pgfpathlineto{\pgfqpoint{0.944527in}{3.099759in}}%
\pgfpathlineto{\pgfqpoint{0.949611in}{3.102814in}}%
\pgfpathlineto{\pgfqpoint{0.955766in}{3.109399in}}%
\pgfpathlineto{\pgfqpoint{0.964865in}{3.122905in}}%
\pgfpathlineto{\pgfqpoint{0.979048in}{3.143717in}}%
\pgfpathlineto{\pgfqpoint{0.985471in}{3.149399in}}%
\pgfpathlineto{\pgfqpoint{0.990555in}{3.151257in}}%
\pgfpathlineto{\pgfqpoint{0.995372in}{3.150668in}}%
\pgfpathlineto{\pgfqpoint{1.000457in}{3.147623in}}%
\pgfpathlineto{\pgfqpoint{1.006612in}{3.141050in}}%
\pgfpathlineto{\pgfqpoint{1.015710in}{3.127551in}}%
\pgfpathlineto{\pgfqpoint{1.029894in}{3.106728in}}%
\pgfpathlineto{\pgfqpoint{1.036316in}{3.101034in}}%
\pgfpathlineto{\pgfqpoint{1.041401in}{3.099164in}}%
\pgfpathlineto{\pgfqpoint{1.046218in}{3.099743in}}%
\pgfpathlineto{\pgfqpoint{1.051302in}{3.102776in}}%
\pgfpathlineto{\pgfqpoint{1.057457in}{3.109339in}}%
\pgfpathlineto{\pgfqpoint{1.066556in}{3.122831in}}%
\pgfpathlineto{\pgfqpoint{1.081007in}{3.143968in}}%
\pgfpathlineto{\pgfqpoint{1.087429in}{3.149532in}}%
\pgfpathlineto{\pgfqpoint{1.092514in}{3.151281in}}%
\pgfpathlineto{\pgfqpoint{1.097331in}{3.150586in}}%
\pgfpathlineto{\pgfqpoint{1.102415in}{3.147438in}}%
\pgfpathlineto{\pgfqpoint{1.108570in}{3.140764in}}%
\pgfpathlineto{\pgfqpoint{1.117936in}{3.126764in}}%
\pgfpathlineto{\pgfqpoint{1.131584in}{3.106780in}}%
\pgfpathlineto{\pgfqpoint{1.138007in}{3.101062in}}%
\pgfpathlineto{\pgfqpoint{1.143092in}{3.099170in}}%
\pgfpathlineto{\pgfqpoint{1.147909in}{3.099726in}}%
\pgfpathlineto{\pgfqpoint{1.152993in}{3.102738in}}%
\pgfpathlineto{\pgfqpoint{1.159148in}{3.109280in}}%
\pgfpathlineto{\pgfqpoint{1.168247in}{3.122757in}}%
\pgfpathlineto{\pgfqpoint{1.182698in}{3.143916in}}%
\pgfpathlineto{\pgfqpoint{1.189120in}{3.149504in}}%
\pgfpathlineto{\pgfqpoint{1.194205in}{3.151276in}}%
\pgfpathlineto{\pgfqpoint{1.199022in}{3.150603in}}%
\pgfpathlineto{\pgfqpoint{1.204106in}{3.147477in}}%
\pgfpathlineto{\pgfqpoint{1.210261in}{3.140824in}}%
\pgfpathlineto{\pgfqpoint{1.219627in}{3.126838in}}%
\pgfpathlineto{\pgfqpoint{1.233543in}{3.106529in}}%
\pgfpathlineto{\pgfqpoint{1.239966in}{3.100928in}}%
\pgfpathlineto{\pgfqpoint{1.245050in}{3.099145in}}%
\pgfpathlineto{\pgfqpoint{1.249867in}{3.099807in}}%
\pgfpathlineto{\pgfqpoint{1.254952in}{3.102922in}}%
\pgfpathlineto{\pgfqpoint{1.261107in}{3.109565in}}%
\pgfpathlineto{\pgfqpoint{1.270473in}{3.123544in}}%
\pgfpathlineto{\pgfqpoint{1.284389in}{3.143864in}}%
\pgfpathlineto{\pgfqpoint{1.290811in}{3.149477in}}%
\pgfpathlineto{\pgfqpoint{1.295896in}{3.151272in}}%
\pgfpathlineto{\pgfqpoint{1.300713in}{3.150621in}}%
\pgfpathlineto{\pgfqpoint{1.305797in}{3.147516in}}%
\pgfpathlineto{\pgfqpoint{1.311952in}{3.140884in}}%
\pgfpathlineto{\pgfqpoint{1.321318in}{3.126912in}}%
\pgfpathlineto{\pgfqpoint{1.335234in}{3.106581in}}%
\pgfpathlineto{\pgfqpoint{1.341657in}{3.100956in}}%
\pgfpathlineto{\pgfqpoint{1.346741in}{3.099150in}}%
\pgfpathlineto{\pgfqpoint{1.351558in}{3.099790in}}%
\pgfpathlineto{\pgfqpoint{1.356643in}{3.102884in}}%
\pgfpathlineto{\pgfqpoint{1.362798in}{3.109505in}}%
\pgfpathlineto{\pgfqpoint{1.372164in}{3.123469in}}%
\pgfpathlineto{\pgfqpoint{1.386080in}{3.143812in}}%
\pgfpathlineto{\pgfqpoint{1.392502in}{3.149449in}}%
\pgfpathlineto{\pgfqpoint{1.397587in}{3.151267in}}%
\pgfpathlineto{\pgfqpoint{1.402404in}{3.150638in}}%
\pgfpathlineto{\pgfqpoint{1.407488in}{3.147554in}}%
\pgfpathlineto{\pgfqpoint{1.413643in}{3.140943in}}%
\pgfpathlineto{\pgfqpoint{1.422742in}{3.127417in}}%
\pgfpathlineto{\pgfqpoint{1.436925in}{3.106633in}}%
\pgfpathlineto{\pgfqpoint{1.443348in}{3.100983in}}%
\pgfpathlineto{\pgfqpoint{1.448432in}{3.099155in}}%
\pgfpathlineto{\pgfqpoint{1.453249in}{3.099773in}}%
\pgfpathlineto{\pgfqpoint{1.458334in}{3.102845in}}%
\pgfpathlineto{\pgfqpoint{1.464489in}{3.109446in}}%
\pgfpathlineto{\pgfqpoint{1.473587in}{3.122965in}}%
\pgfpathlineto{\pgfqpoint{1.487770in}{3.143759in}}%
\pgfpathlineto{\pgfqpoint{1.494193in}{3.149422in}}%
\pgfpathlineto{\pgfqpoint{1.499278in}{3.151262in}}%
\pgfpathlineto{\pgfqpoint{1.504095in}{3.150655in}}%
\pgfpathlineto{\pgfqpoint{1.509179in}{3.147593in}}%
\pgfpathlineto{\pgfqpoint{1.515334in}{3.141002in}}%
\pgfpathlineto{\pgfqpoint{1.524433in}{3.127491in}}%
\pgfpathlineto{\pgfqpoint{1.538616in}{3.106686in}}%
\pgfpathlineto{\pgfqpoint{1.545039in}{3.101011in}}%
\pgfpathlineto{\pgfqpoint{1.550123in}{3.099160in}}%
\pgfpathlineto{\pgfqpoint{1.554940in}{3.099756in}}%
\pgfpathlineto{\pgfqpoint{1.560025in}{3.102807in}}%
\pgfpathlineto{\pgfqpoint{1.566180in}{3.109387in}}%
\pgfpathlineto{\pgfqpoint{1.575278in}{3.122891in}}%
\pgfpathlineto{\pgfqpoint{1.589461in}{3.143707in}}%
\pgfpathlineto{\pgfqpoint{1.595884in}{3.149394in}}%
\pgfpathlineto{\pgfqpoint{1.600969in}{3.151256in}}%
\pgfpathlineto{\pgfqpoint{1.605786in}{3.150671in}}%
\pgfpathlineto{\pgfqpoint{1.610870in}{3.147631in}}%
\pgfpathlineto{\pgfqpoint{1.617025in}{3.141062in}}%
\pgfpathlineto{\pgfqpoint{1.626124in}{3.127565in}}%
\pgfpathlineto{\pgfqpoint{1.640575in}{3.106436in}}%
\pgfpathlineto{\pgfqpoint{1.646997in}{3.100879in}}%
\pgfpathlineto{\pgfqpoint{1.652082in}{3.099137in}}%
\pgfpathlineto{\pgfqpoint{1.656899in}{3.099838in}}%
\pgfpathlineto{\pgfqpoint{1.661983in}{3.102993in}}%
\pgfpathlineto{\pgfqpoint{1.668138in}{3.109672in}}%
\pgfpathlineto{\pgfqpoint{1.677504in}{3.123677in}}%
\pgfpathlineto{\pgfqpoint{1.691152in}{3.143654in}}%
\pgfpathlineto{\pgfqpoint{1.697575in}{3.149365in}}%
\pgfpathlineto{\pgfqpoint{1.702660in}{3.151251in}}%
\pgfpathlineto{\pgfqpoint{1.707477in}{3.150688in}}%
\pgfpathlineto{\pgfqpoint{1.712561in}{3.147669in}}%
\pgfpathlineto{\pgfqpoint{1.718716in}{3.141121in}}%
\pgfpathlineto{\pgfqpoint{1.727815in}{3.127639in}}%
\pgfpathlineto{\pgfqpoint{1.742266in}{3.106487in}}%
\pgfpathlineto{\pgfqpoint{1.748688in}{3.100906in}}%
\pgfpathlineto{\pgfqpoint{1.753773in}{3.099141in}}%
\pgfpathlineto{\pgfqpoint{1.758590in}{3.099821in}}%
\pgfpathlineto{\pgfqpoint{1.763674in}{3.102954in}}%
\pgfpathlineto{\pgfqpoint{1.769829in}{3.109613in}}%
\pgfpathlineto{\pgfqpoint{1.779195in}{3.123603in}}%
\pgfpathlineto{\pgfqpoint{1.792843in}{3.143602in}}%
\pgfpathlineto{\pgfqpoint{1.799266in}{3.149337in}}%
\pgfpathlineto{\pgfqpoint{1.804351in}{3.151245in}}%
\pgfpathlineto{\pgfqpoint{1.809167in}{3.150704in}}%
\pgfpathlineto{\pgfqpoint{1.814252in}{3.147707in}}%
\pgfpathlineto{\pgfqpoint{1.820407in}{3.141180in}}%
\pgfpathlineto{\pgfqpoint{1.829506in}{3.127714in}}%
\pgfpathlineto{\pgfqpoint{1.843957in}{3.106539in}}%
\pgfpathlineto{\pgfqpoint{1.850379in}{3.100934in}}%
\pgfpathlineto{\pgfqpoint{1.855464in}{3.099146in}}%
\pgfpathlineto{\pgfqpoint{1.860281in}{3.099803in}}%
\pgfpathlineto{\pgfqpoint{1.865365in}{3.102915in}}%
\pgfpathlineto{\pgfqpoint{1.871520in}{3.109553in}}%
\pgfpathlineto{\pgfqpoint{1.880886in}{3.123529in}}%
\pgfpathlineto{\pgfqpoint{1.894802in}{3.143853in}}%
\pgfpathlineto{\pgfqpoint{1.901225in}{3.149471in}}%
\pgfpathlineto{\pgfqpoint{1.906309in}{3.151271in}}%
\pgfpathlineto{\pgfqpoint{1.911126in}{3.150624in}}%
\pgfpathlineto{\pgfqpoint{1.916211in}{3.147524in}}%
\pgfpathlineto{\pgfqpoint{1.922366in}{3.140896in}}%
\pgfpathlineto{\pgfqpoint{1.931732in}{3.126927in}}%
\pgfpathlineto{\pgfqpoint{1.945647in}{3.106592in}}%
\pgfpathlineto{\pgfqpoint{1.952070in}{3.100961in}}%
\pgfpathlineto{\pgfqpoint{1.957155in}{3.099151in}}%
\pgfpathlineto{\pgfqpoint{1.961972in}{3.099786in}}%
\pgfpathlineto{\pgfqpoint{1.967056in}{3.102876in}}%
\pgfpathlineto{\pgfqpoint{1.973211in}{3.109494in}}%
\pgfpathlineto{\pgfqpoint{1.982310in}{3.123024in}}%
\pgfpathlineto{\pgfqpoint{1.996493in}{3.143801in}}%
\pgfpathlineto{\pgfqpoint{2.002916in}{3.149444in}}%
\pgfpathlineto{\pgfqpoint{2.008000in}{3.151266in}}%
\pgfpathlineto{\pgfqpoint{2.012817in}{3.150641in}}%
\pgfpathlineto{\pgfqpoint{2.017902in}{3.147562in}}%
\pgfpathlineto{\pgfqpoint{2.024057in}{3.140955in}}%
\pgfpathlineto{\pgfqpoint{2.033155in}{3.127432in}}%
\pgfpathlineto{\pgfqpoint{2.047338in}{3.106644in}}%
\pgfpathlineto{\pgfqpoint{2.053761in}{3.100989in}}%
\pgfpathlineto{\pgfqpoint{2.058846in}{3.099156in}}%
\pgfpathlineto{\pgfqpoint{2.063663in}{3.099769in}}%
\pgfpathlineto{\pgfqpoint{2.068747in}{3.102837in}}%
\pgfpathlineto{\pgfqpoint{2.074902in}{3.109434in}}%
\pgfpathlineto{\pgfqpoint{2.084001in}{3.122950in}}%
\pgfpathlineto{\pgfqpoint{2.098184in}{3.143749in}}%
\pgfpathlineto{\pgfqpoint{2.104607in}{3.149416in}}%
\pgfpathlineto{\pgfqpoint{2.109691in}{3.151261in}}%
\pgfpathlineto{\pgfqpoint{2.114508in}{3.150658in}}%
\pgfpathlineto{\pgfqpoint{2.119593in}{3.147601in}}%
\pgfpathlineto{\pgfqpoint{2.125748in}{3.141014in}}%
\pgfpathlineto{\pgfqpoint{2.134846in}{3.127506in}}%
\pgfpathlineto{\pgfqpoint{2.149029in}{3.106696in}}%
\pgfpathlineto{\pgfqpoint{2.155452in}{3.101017in}}%
\pgfpathlineto{\pgfqpoint{2.160537in}{3.099161in}}%
\pgfpathlineto{\pgfqpoint{2.165353in}{3.099753in}}%
\pgfpathlineto{\pgfqpoint{2.170438in}{3.102799in}}%
\pgfpathlineto{\pgfqpoint{2.176593in}{3.109375in}}%
\pgfpathlineto{\pgfqpoint{2.185692in}{3.122876in}}%
\pgfpathlineto{\pgfqpoint{2.199875in}{3.143696in}}%
\pgfpathlineto{\pgfqpoint{2.206297in}{3.149388in}}%
\pgfpathlineto{\pgfqpoint{2.211382in}{3.151255in}}%
\pgfpathlineto{\pgfqpoint{2.216199in}{3.150675in}}%
\pgfpathlineto{\pgfqpoint{2.221284in}{3.147639in}}%
\pgfpathlineto{\pgfqpoint{2.227439in}{3.141074in}}%
\pgfpathlineto{\pgfqpoint{2.236537in}{3.127580in}}%
\pgfpathlineto{\pgfqpoint{2.250988in}{3.106446in}}%
\pgfpathlineto{\pgfqpoint{2.257411in}{3.100885in}}%
\pgfpathlineto{\pgfqpoint{2.262495in}{3.099138in}}%
\pgfpathlineto{\pgfqpoint{2.267312in}{3.099835in}}%
\pgfpathlineto{\pgfqpoint{2.272397in}{3.102985in}}%
\pgfpathlineto{\pgfqpoint{2.278552in}{3.109660in}}%
\pgfpathlineto{\pgfqpoint{2.287918in}{3.123662in}}%
\pgfpathlineto{\pgfqpoint{2.301566in}{3.143644in}}%
\pgfpathlineto{\pgfqpoint{2.307988in}{3.149360in}}%
\pgfpathlineto{\pgfqpoint{2.313073in}{3.151250in}}%
\pgfpathlineto{\pgfqpoint{2.317890in}{3.150691in}}%
\pgfpathlineto{\pgfqpoint{2.322975in}{3.147677in}}%
\pgfpathlineto{\pgfqpoint{2.329129in}{3.141133in}}%
\pgfpathlineto{\pgfqpoint{2.338228in}{3.127654in}}%
\pgfpathlineto{\pgfqpoint{2.352679in}{3.106498in}}%
\pgfpathlineto{\pgfqpoint{2.359102in}{3.100912in}}%
\pgfpathlineto{\pgfqpoint{2.364186in}{3.099142in}}%
\pgfpathlineto{\pgfqpoint{2.369003in}{3.099817in}}%
\pgfpathlineto{\pgfqpoint{2.374088in}{3.102946in}}%
\pgfpathlineto{\pgfqpoint{2.380243in}{3.109601in}}%
\pgfpathlineto{\pgfqpoint{2.389609in}{3.123588in}}%
\pgfpathlineto{\pgfqpoint{2.403524in}{3.143895in}}%
\pgfpathlineto{\pgfqpoint{2.409947in}{3.149493in}}%
\pgfpathlineto{\pgfqpoint{2.415032in}{3.151275in}}%
\pgfpathlineto{\pgfqpoint{2.419849in}{3.150610in}}%
\pgfpathlineto{\pgfqpoint{2.424933in}{3.147493in}}%
\pgfpathlineto{\pgfqpoint{2.431088in}{3.140848in}}%
\pgfpathlineto{\pgfqpoint{2.440454in}{3.126868in}}%
\pgfpathlineto{\pgfqpoint{2.454370in}{3.106550in}}%
\pgfpathlineto{\pgfqpoint{2.460793in}{3.100939in}}%
\pgfpathlineto{\pgfqpoint{2.465877in}{3.099147in}}%
\pgfpathlineto{\pgfqpoint{2.470694in}{3.099800in}}%
\pgfpathlineto{\pgfqpoint{2.475779in}{3.102907in}}%
\pgfpathlineto{\pgfqpoint{2.481934in}{3.109541in}}%
\pgfpathlineto{\pgfqpoint{2.491300in}{3.123514in}}%
\pgfpathlineto{\pgfqpoint{2.505215in}{3.143843in}}%
\pgfpathlineto{\pgfqpoint{2.511638in}{3.149466in}}%
\pgfpathlineto{\pgfqpoint{2.516723in}{3.151270in}}%
\pgfpathlineto{\pgfqpoint{2.521540in}{3.150628in}}%
\pgfpathlineto{\pgfqpoint{2.526624in}{3.147531in}}%
\pgfpathlineto{\pgfqpoint{2.532779in}{3.140907in}}%
\pgfpathlineto{\pgfqpoint{2.542145in}{3.126942in}}%
\pgfpathlineto{\pgfqpoint{2.556061in}{3.106602in}}%
\pgfpathlineto{\pgfqpoint{2.562484in}{3.100967in}}%
\pgfpathlineto{\pgfqpoint{2.567568in}{3.099152in}}%
\pgfpathlineto{\pgfqpoint{2.572385in}{3.099783in}}%
\pgfpathlineto{\pgfqpoint{2.577470in}{3.102868in}}%
\pgfpathlineto{\pgfqpoint{2.583625in}{3.109482in}}%
\pgfpathlineto{\pgfqpoint{2.592723in}{3.123009in}}%
\pgfpathlineto{\pgfqpoint{2.606906in}{3.143791in}}%
\pgfpathlineto{\pgfqpoint{2.613329in}{3.149438in}}%
\pgfpathlineto{\pgfqpoint{2.618414in}{3.151265in}}%
\pgfpathlineto{\pgfqpoint{2.623230in}{3.150645in}}%
\pgfpathlineto{\pgfqpoint{2.628315in}{3.147570in}}%
\pgfpathlineto{\pgfqpoint{2.634470in}{3.140967in}}%
\pgfpathlineto{\pgfqpoint{2.643569in}{3.127447in}}%
\pgfpathlineto{\pgfqpoint{2.657752in}{3.106654in}}%
\pgfpathlineto{\pgfqpoint{2.664174in}{3.100995in}}%
\pgfpathlineto{\pgfqpoint{2.669259in}{3.099157in}}%
\pgfpathlineto{\pgfqpoint{2.674076in}{3.099766in}}%
\pgfpathlineto{\pgfqpoint{2.679161in}{3.102830in}}%
\pgfpathlineto{\pgfqpoint{2.685315in}{3.109422in}}%
\pgfpathlineto{\pgfqpoint{2.694414in}{3.122935in}}%
\pgfpathlineto{\pgfqpoint{2.708597in}{3.143738in}}%
\pgfpathlineto{\pgfqpoint{2.715020in}{3.149410in}}%
\pgfpathlineto{\pgfqpoint{2.720105in}{3.151260in}}%
\pgfpathlineto{\pgfqpoint{2.724921in}{3.150661in}}%
\pgfpathlineto{\pgfqpoint{2.730006in}{3.147608in}}%
\pgfpathlineto{\pgfqpoint{2.736161in}{3.141026in}}%
\pgfpathlineto{\pgfqpoint{2.745260in}{3.127521in}}%
\pgfpathlineto{\pgfqpoint{2.759443in}{3.106707in}}%
\pgfpathlineto{\pgfqpoint{2.765865in}{3.101022in}}%
\pgfpathlineto{\pgfqpoint{2.770950in}{3.099162in}}%
\pgfpathlineto{\pgfqpoint{2.775767in}{3.099749in}}%
\pgfpathlineto{\pgfqpoint{2.780851in}{3.102792in}}%
\pgfpathlineto{\pgfqpoint{2.787006in}{3.109363in}}%
\pgfpathlineto{\pgfqpoint{2.796105in}{3.122861in}}%
\pgfpathlineto{\pgfqpoint{2.810556in}{3.143988in}}%
\pgfpathlineto{\pgfqpoint{2.816979in}{3.149542in}}%
\pgfpathlineto{\pgfqpoint{2.822063in}{3.151283in}}%
\pgfpathlineto{\pgfqpoint{2.826880in}{3.150579in}}%
\pgfpathlineto{\pgfqpoint{2.831965in}{3.147422in}}%
\pgfpathlineto{\pgfqpoint{2.838120in}{3.140740in}}%
\pgfpathlineto{\pgfqpoint{2.847486in}{3.126734in}}%
\pgfpathlineto{\pgfqpoint{2.861134in}{3.106759in}}%
\pgfpathlineto{\pgfqpoint{2.867556in}{3.101051in}}%
\pgfpathlineto{\pgfqpoint{2.872641in}{3.099168in}}%
\pgfpathlineto{\pgfqpoint{2.877458in}{3.099733in}}%
\pgfpathlineto{\pgfqpoint{2.882542in}{3.102753in}}%
\pgfpathlineto{\pgfqpoint{2.888697in}{3.109304in}}%
\pgfpathlineto{\pgfqpoint{2.897796in}{3.122787in}}%
\pgfpathlineto{\pgfqpoint{2.912247in}{3.143937in}}%
\pgfpathlineto{\pgfqpoint{2.918670in}{3.149515in}}%
\pgfpathlineto{\pgfqpoint{2.923754in}{3.151278in}}%
\pgfpathlineto{\pgfqpoint{2.928571in}{3.150596in}}%
\pgfpathlineto{\pgfqpoint{2.933656in}{3.147461in}}%
\pgfpathlineto{\pgfqpoint{2.939811in}{3.140800in}}%
\pgfpathlineto{\pgfqpoint{2.949177in}{3.126808in}}%
\pgfpathlineto{\pgfqpoint{2.962825in}{3.106812in}}%
\pgfpathlineto{\pgfqpoint{2.969247in}{3.101079in}}%
\pgfpathlineto{\pgfqpoint{2.974332in}{3.099173in}}%
\pgfpathlineto{\pgfqpoint{2.979149in}{3.099717in}}%
\pgfpathlineto{\pgfqpoint{2.984233in}{3.102716in}}%
\pgfpathlineto{\pgfqpoint{2.990388in}{3.109245in}}%
\pgfpathlineto{\pgfqpoint{2.999487in}{3.122713in}}%
\pgfpathlineto{\pgfqpoint{3.013938in}{3.143885in}}%
\pgfpathlineto{\pgfqpoint{3.020360in}{3.149488in}}%
\pgfpathlineto{\pgfqpoint{3.025445in}{3.151274in}}%
\pgfpathlineto{\pgfqpoint{3.030262in}{3.150614in}}%
\pgfpathlineto{\pgfqpoint{3.035347in}{3.147500in}}%
\pgfpathlineto{\pgfqpoint{3.041502in}{3.140860in}}%
\pgfpathlineto{\pgfqpoint{3.050868in}{3.126883in}}%
\pgfpathlineto{\pgfqpoint{3.064783in}{3.106560in}}%
\pgfpathlineto{\pgfqpoint{3.071206in}{3.100945in}}%
\pgfpathlineto{\pgfqpoint{3.076291in}{3.099148in}}%
\pgfpathlineto{\pgfqpoint{3.081107in}{3.099796in}}%
\pgfpathlineto{\pgfqpoint{3.086192in}{3.102899in}}%
\pgfpathlineto{\pgfqpoint{3.092347in}{3.109529in}}%
\pgfpathlineto{\pgfqpoint{3.101713in}{3.123499in}}%
\pgfpathlineto{\pgfqpoint{3.115629in}{3.143833in}}%
\pgfpathlineto{\pgfqpoint{3.122051in}{3.149460in}}%
\pgfpathlineto{\pgfqpoint{3.127136in}{3.151269in}}%
\pgfpathlineto{\pgfqpoint{3.131953in}{3.150631in}}%
\pgfpathlineto{\pgfqpoint{3.137037in}{3.147539in}}%
\pgfpathlineto{\pgfqpoint{3.143192in}{3.140919in}}%
\pgfpathlineto{\pgfqpoint{3.152291in}{3.127387in}}%
\pgfpathlineto{\pgfqpoint{3.166474in}{3.106612in}}%
\pgfpathlineto{\pgfqpoint{3.172897in}{3.100972in}}%
\pgfpathlineto{\pgfqpoint{3.177981in}{3.099153in}}%
\pgfpathlineto{\pgfqpoint{3.182798in}{3.099779in}}%
\pgfpathlineto{\pgfqpoint{3.187883in}{3.102861in}}%
\pgfpathlineto{\pgfqpoint{3.194038in}{3.109470in}}%
\pgfpathlineto{\pgfqpoint{3.203137in}{3.122994in}}%
\pgfpathlineto{\pgfqpoint{3.217320in}{3.143780in}}%
\pgfpathlineto{\pgfqpoint{3.223742in}{3.149433in}}%
\pgfpathlineto{\pgfqpoint{3.228827in}{3.151264in}}%
\pgfpathlineto{\pgfqpoint{3.233644in}{3.150648in}}%
\pgfpathlineto{\pgfqpoint{3.238728in}{3.147578in}}%
\pgfpathlineto{\pgfqpoint{3.244883in}{3.140979in}}%
\pgfpathlineto{\pgfqpoint{3.253982in}{3.127462in}}%
\pgfpathlineto{\pgfqpoint{3.268165in}{3.106665in}}%
\pgfpathlineto{\pgfqpoint{3.274588in}{3.101000in}}%
\pgfpathlineto{\pgfqpoint{3.279672in}{3.099158in}}%
\pgfpathlineto{\pgfqpoint{3.284489in}{3.099763in}}%
\pgfpathlineto{\pgfqpoint{3.289574in}{3.102822in}}%
\pgfpathlineto{\pgfqpoint{3.295729in}{3.109410in}}%
\pgfpathlineto{\pgfqpoint{3.304828in}{3.122920in}}%
\pgfpathlineto{\pgfqpoint{3.319011in}{3.143728in}}%
\pgfpathlineto{\pgfqpoint{3.325433in}{3.149405in}}%
\pgfpathlineto{\pgfqpoint{3.330518in}{3.151258in}}%
\pgfpathlineto{\pgfqpoint{3.335335in}{3.150665in}}%
\pgfpathlineto{\pgfqpoint{3.340419in}{3.147616in}}%
\pgfpathlineto{\pgfqpoint{3.346574in}{3.141038in}}%
\pgfpathlineto{\pgfqpoint{3.355673in}{3.127536in}}%
\pgfpathlineto{\pgfqpoint{3.369856in}{3.106717in}}%
\pgfpathlineto{\pgfqpoint{3.376279in}{3.101028in}}%
\pgfpathlineto{\pgfqpoint{3.381363in}{3.099163in}}%
\pgfpathlineto{\pgfqpoint{3.386180in}{3.099746in}}%
\pgfpathlineto{\pgfqpoint{3.391265in}{3.102784in}}%
\pgfpathlineto{\pgfqpoint{3.397420in}{3.109351in}}%
\pgfpathlineto{\pgfqpoint{3.406519in}{3.122846in}}%
\pgfpathlineto{\pgfqpoint{3.420969in}{3.143978in}}%
\pgfpathlineto{\pgfqpoint{3.427392in}{3.149537in}}%
\pgfpathlineto{\pgfqpoint{3.432477in}{3.151282in}}%
\pgfpathlineto{\pgfqpoint{3.437293in}{3.150582in}}%
\pgfpathlineto{\pgfqpoint{3.442378in}{3.147430in}}%
\pgfpathlineto{\pgfqpoint{3.448533in}{3.140752in}}%
\pgfpathlineto{\pgfqpoint{3.457899in}{3.126749in}}%
\pgfpathlineto{\pgfqpoint{3.471547in}{3.106770in}}%
\pgfpathlineto{\pgfqpoint{3.477970in}{3.101056in}}%
\pgfpathlineto{\pgfqpoint{3.483054in}{3.099169in}}%
\pgfpathlineto{\pgfqpoint{3.487871in}{3.099730in}}%
\pgfpathlineto{\pgfqpoint{3.492956in}{3.102746in}}%
\pgfpathlineto{\pgfqpoint{3.499111in}{3.109292in}}%
\pgfpathlineto{\pgfqpoint{3.508210in}{3.122772in}}%
\pgfpathlineto{\pgfqpoint{3.522660in}{3.143926in}}%
\pgfpathlineto{\pgfqpoint{3.529083in}{3.149510in}}%
\pgfpathlineto{\pgfqpoint{3.534168in}{3.151277in}}%
\pgfpathlineto{\pgfqpoint{3.538984in}{3.150600in}}%
\pgfpathlineto{\pgfqpoint{3.544069in}{3.147469in}}%
\pgfpathlineto{\pgfqpoint{3.550224in}{3.140812in}}%
\pgfpathlineto{\pgfqpoint{3.559590in}{3.126823in}}%
\pgfpathlineto{\pgfqpoint{3.573238in}{3.106823in}}%
\pgfpathlineto{\pgfqpoint{3.579661in}{3.101085in}}%
\pgfpathlineto{\pgfqpoint{3.584745in}{3.099174in}}%
\pgfpathlineto{\pgfqpoint{3.589562in}{3.099713in}}%
\pgfpathlineto{\pgfqpoint{3.594647in}{3.102708in}}%
\pgfpathlineto{\pgfqpoint{3.600802in}{3.109233in}}%
\pgfpathlineto{\pgfqpoint{3.609901in}{3.122698in}}%
\pgfpathlineto{\pgfqpoint{3.624351in}{3.143874in}}%
\pgfpathlineto{\pgfqpoint{3.630774in}{3.149482in}}%
\pgfpathlineto{\pgfqpoint{3.635858in}{3.151273in}}%
\pgfpathlineto{\pgfqpoint{3.640675in}{3.150617in}}%
\pgfpathlineto{\pgfqpoint{3.645760in}{3.147508in}}%
\pgfpathlineto{\pgfqpoint{3.651915in}{3.140872in}}%
\pgfpathlineto{\pgfqpoint{3.661281in}{3.126898in}}%
\pgfpathlineto{\pgfqpoint{3.675197in}{3.106571in}}%
\pgfpathlineto{\pgfqpoint{3.681619in}{3.100950in}}%
\pgfpathlineto{\pgfqpoint{3.686704in}{3.099149in}}%
\pgfpathlineto{\pgfqpoint{3.691521in}{3.099793in}}%
\pgfpathlineto{\pgfqpoint{3.696605in}{3.102891in}}%
\pgfpathlineto{\pgfqpoint{3.702760in}{3.109517in}}%
\pgfpathlineto{\pgfqpoint{3.712127in}{3.123484in}}%
\pgfpathlineto{\pgfqpoint{3.726042in}{3.143822in}}%
\pgfpathlineto{\pgfqpoint{3.732465in}{3.149455in}}%
\pgfpathlineto{\pgfqpoint{3.737549in}{3.151268in}}%
\pgfpathlineto{\pgfqpoint{3.742366in}{3.150634in}}%
\pgfpathlineto{\pgfqpoint{3.747451in}{3.147547in}}%
\pgfpathlineto{\pgfqpoint{3.753606in}{3.140931in}}%
\pgfpathlineto{\pgfqpoint{3.762705in}{3.127402in}}%
\pgfpathlineto{\pgfqpoint{3.776888in}{3.106623in}}%
\pgfpathlineto{\pgfqpoint{3.783310in}{3.100978in}}%
\pgfpathlineto{\pgfqpoint{3.788395in}{3.099154in}}%
\pgfpathlineto{\pgfqpoint{3.793212in}{3.099776in}}%
\pgfpathlineto{\pgfqpoint{3.798296in}{3.102853in}}%
\pgfpathlineto{\pgfqpoint{3.804451in}{3.109458in}}%
\pgfpathlineto{\pgfqpoint{3.813550in}{3.122980in}}%
\pgfpathlineto{\pgfqpoint{3.827733in}{3.143770in}}%
\pgfpathlineto{\pgfqpoint{3.834156in}{3.149427in}}%
\pgfpathlineto{\pgfqpoint{3.839240in}{3.151263in}}%
\pgfpathlineto{\pgfqpoint{3.844057in}{3.150651in}}%
\pgfpathlineto{\pgfqpoint{3.849142in}{3.147585in}}%
\pgfpathlineto{\pgfqpoint{3.855297in}{3.140991in}}%
\pgfpathlineto{\pgfqpoint{3.864396in}{3.127476in}}%
\pgfpathlineto{\pgfqpoint{3.878579in}{3.106675in}}%
\pgfpathlineto{\pgfqpoint{3.885001in}{3.101006in}}%
\pgfpathlineto{\pgfqpoint{3.890086in}{3.099159in}}%
\pgfpathlineto{\pgfqpoint{3.894903in}{3.099759in}}%
\pgfpathlineto{\pgfqpoint{3.899987in}{3.102814in}}%
\pgfpathlineto{\pgfqpoint{3.906142in}{3.109399in}}%
\pgfpathlineto{\pgfqpoint{3.915241in}{3.122905in}}%
\pgfpathlineto{\pgfqpoint{3.929424in}{3.143717in}}%
\pgfpathlineto{\pgfqpoint{3.935847in}{3.149399in}}%
\pgfpathlineto{\pgfqpoint{3.940931in}{3.151257in}}%
\pgfpathlineto{\pgfqpoint{3.945748in}{3.150668in}}%
\pgfpathlineto{\pgfqpoint{3.950833in}{3.147623in}}%
\pgfpathlineto{\pgfqpoint{3.956988in}{3.141050in}}%
\pgfpathlineto{\pgfqpoint{3.966087in}{3.127551in}}%
\pgfpathlineto{\pgfqpoint{3.980270in}{3.106728in}}%
\pgfpathlineto{\pgfqpoint{3.986692in}{3.101034in}}%
\pgfpathlineto{\pgfqpoint{3.991777in}{3.099164in}}%
\pgfpathlineto{\pgfqpoint{3.996594in}{3.099743in}}%
\pgfpathlineto{\pgfqpoint{4.001678in}{3.102776in}}%
\pgfpathlineto{\pgfqpoint{4.007833in}{3.109339in}}%
\pgfpathlineto{\pgfqpoint{4.016932in}{3.122831in}}%
\pgfpathlineto{\pgfqpoint{4.031383in}{3.143968in}}%
\pgfpathlineto{\pgfqpoint{4.037805in}{3.149532in}}%
\pgfpathlineto{\pgfqpoint{4.042890in}{3.151281in}}%
\pgfpathlineto{\pgfqpoint{4.047707in}{3.150586in}}%
\pgfpathlineto{\pgfqpoint{4.052791in}{3.147438in}}%
\pgfpathlineto{\pgfqpoint{4.058946in}{3.140764in}}%
\pgfpathlineto{\pgfqpoint{4.068313in}{3.126764in}}%
\pgfpathlineto{\pgfqpoint{4.081961in}{3.106780in}}%
\pgfpathlineto{\pgfqpoint{4.088383in}{3.101062in}}%
\pgfpathlineto{\pgfqpoint{4.093468in}{3.099170in}}%
\pgfpathlineto{\pgfqpoint{4.098285in}{3.099726in}}%
\pgfpathlineto{\pgfqpoint{4.103369in}{3.102738in}}%
\pgfpathlineto{\pgfqpoint{4.109524in}{3.109280in}}%
\pgfpathlineto{\pgfqpoint{4.118623in}{3.122757in}}%
\pgfpathlineto{\pgfqpoint{4.133074in}{3.143916in}}%
\pgfpathlineto{\pgfqpoint{4.139496in}{3.149504in}}%
\pgfpathlineto{\pgfqpoint{4.144581in}{3.151276in}}%
\pgfpathlineto{\pgfqpoint{4.149398in}{3.150603in}}%
\pgfpathlineto{\pgfqpoint{4.154482in}{3.147477in}}%
\pgfpathlineto{\pgfqpoint{4.160637in}{3.140824in}}%
\pgfpathlineto{\pgfqpoint{4.170004in}{3.126838in}}%
\pgfpathlineto{\pgfqpoint{4.183919in}{3.106529in}}%
\pgfpathlineto{\pgfqpoint{4.190342in}{3.100928in}}%
\pgfpathlineto{\pgfqpoint{4.195426in}{3.099145in}}%
\pgfpathlineto{\pgfqpoint{4.200243in}{3.099807in}}%
\pgfpathlineto{\pgfqpoint{4.205328in}{3.102922in}}%
\pgfpathlineto{\pgfqpoint{4.211483in}{3.109565in}}%
\pgfpathlineto{\pgfqpoint{4.220849in}{3.123544in}}%
\pgfpathlineto{\pgfqpoint{4.234765in}{3.143864in}}%
\pgfpathlineto{\pgfqpoint{4.241187in}{3.149477in}}%
\pgfpathlineto{\pgfqpoint{4.246272in}{3.151272in}}%
\pgfpathlineto{\pgfqpoint{4.251089in}{3.150621in}}%
\pgfpathlineto{\pgfqpoint{4.256173in}{3.147516in}}%
\pgfpathlineto{\pgfqpoint{4.262328in}{3.140884in}}%
\pgfpathlineto{\pgfqpoint{4.271695in}{3.126912in}}%
\pgfpathlineto{\pgfqpoint{4.285610in}{3.106581in}}%
\pgfpathlineto{\pgfqpoint{4.292033in}{3.100956in}}%
\pgfpathlineto{\pgfqpoint{4.297117in}{3.099150in}}%
\pgfpathlineto{\pgfqpoint{4.301934in}{3.099790in}}%
\pgfpathlineto{\pgfqpoint{4.307019in}{3.102884in}}%
\pgfpathlineto{\pgfqpoint{4.313174in}{3.109505in}}%
\pgfpathlineto{\pgfqpoint{4.322540in}{3.123469in}}%
\pgfpathlineto{\pgfqpoint{4.336456in}{3.143812in}}%
\pgfpathlineto{\pgfqpoint{4.342878in}{3.149449in}}%
\pgfpathlineto{\pgfqpoint{4.347963in}{3.151267in}}%
\pgfpathlineto{\pgfqpoint{4.352780in}{3.150638in}}%
\pgfpathlineto{\pgfqpoint{4.357864in}{3.147554in}}%
\pgfpathlineto{\pgfqpoint{4.364019in}{3.140943in}}%
\pgfpathlineto{\pgfqpoint{4.373118in}{3.127417in}}%
\pgfpathlineto{\pgfqpoint{4.387301in}{3.106633in}}%
\pgfpathlineto{\pgfqpoint{4.393724in}{3.100983in}}%
\pgfpathlineto{\pgfqpoint{4.398808in}{3.099155in}}%
\pgfpathlineto{\pgfqpoint{4.403625in}{3.099773in}}%
\pgfpathlineto{\pgfqpoint{4.408710in}{3.102845in}}%
\pgfpathlineto{\pgfqpoint{4.414865in}{3.109446in}}%
\pgfpathlineto{\pgfqpoint{4.423964in}{3.122965in}}%
\pgfpathlineto{\pgfqpoint{4.438147in}{3.143759in}}%
\pgfpathlineto{\pgfqpoint{4.444569in}{3.149422in}}%
\pgfpathlineto{\pgfqpoint{4.449654in}{3.151262in}}%
\pgfpathlineto{\pgfqpoint{4.454471in}{3.150655in}}%
\pgfpathlineto{\pgfqpoint{4.459555in}{3.147593in}}%
\pgfpathlineto{\pgfqpoint{4.465710in}{3.141002in}}%
\pgfpathlineto{\pgfqpoint{4.474809in}{3.127491in}}%
\pgfpathlineto{\pgfqpoint{4.488992in}{3.106686in}}%
\pgfpathlineto{\pgfqpoint{4.495415in}{3.101011in}}%
\pgfpathlineto{\pgfqpoint{4.500499in}{3.099160in}}%
\pgfpathlineto{\pgfqpoint{4.505316in}{3.099756in}}%
\pgfpathlineto{\pgfqpoint{4.510401in}{3.102807in}}%
\pgfpathlineto{\pgfqpoint{4.516556in}{3.109387in}}%
\pgfpathlineto{\pgfqpoint{4.525654in}{3.122891in}}%
\pgfpathlineto{\pgfqpoint{4.539838in}{3.143707in}}%
\pgfpathlineto{\pgfqpoint{4.546260in}{3.149394in}}%
\pgfpathlineto{\pgfqpoint{4.551345in}{3.151256in}}%
\pgfpathlineto{\pgfqpoint{4.556162in}{3.150671in}}%
\pgfpathlineto{\pgfqpoint{4.561246in}{3.147631in}}%
\pgfpathlineto{\pgfqpoint{4.567401in}{3.141062in}}%
\pgfpathlineto{\pgfqpoint{4.576500in}{3.127565in}}%
\pgfpathlineto{\pgfqpoint{4.590951in}{3.106436in}}%
\pgfpathlineto{\pgfqpoint{4.597373in}{3.100879in}}%
\pgfpathlineto{\pgfqpoint{4.602458in}{3.099137in}}%
\pgfpathlineto{\pgfqpoint{4.607275in}{3.099838in}}%
\pgfpathlineto{\pgfqpoint{4.612359in}{3.102993in}}%
\pgfpathlineto{\pgfqpoint{4.618514in}{3.109672in}}%
\pgfpathlineto{\pgfqpoint{4.627881in}{3.123677in}}%
\pgfpathlineto{\pgfqpoint{4.641529in}{3.143654in}}%
\pgfpathlineto{\pgfqpoint{4.647951in}{3.149365in}}%
\pgfpathlineto{\pgfqpoint{4.653036in}{3.151251in}}%
\pgfpathlineto{\pgfqpoint{4.657853in}{3.150688in}}%
\pgfpathlineto{\pgfqpoint{4.662937in}{3.147669in}}%
\pgfpathlineto{\pgfqpoint{4.669092in}{3.141121in}}%
\pgfpathlineto{\pgfqpoint{4.678191in}{3.127639in}}%
\pgfpathlineto{\pgfqpoint{4.692642in}{3.106487in}}%
\pgfpathlineto{\pgfqpoint{4.699064in}{3.100906in}}%
\pgfpathlineto{\pgfqpoint{4.704149in}{3.099141in}}%
\pgfpathlineto{\pgfqpoint{4.708966in}{3.099821in}}%
\pgfpathlineto{\pgfqpoint{4.714050in}{3.102954in}}%
\pgfpathlineto{\pgfqpoint{4.720205in}{3.109613in}}%
\pgfpathlineto{\pgfqpoint{4.729572in}{3.123603in}}%
\pgfpathlineto{\pgfqpoint{4.743220in}{3.143602in}}%
\pgfpathlineto{\pgfqpoint{4.749642in}{3.149337in}}%
\pgfpathlineto{\pgfqpoint{4.754727in}{3.151245in}}%
\pgfpathlineto{\pgfqpoint{4.759544in}{3.150704in}}%
\pgfpathlineto{\pgfqpoint{4.764628in}{3.147707in}}%
\pgfpathlineto{\pgfqpoint{4.770783in}{3.141180in}}%
\pgfpathlineto{\pgfqpoint{4.779882in}{3.127714in}}%
\pgfpathlineto{\pgfqpoint{4.794333in}{3.106539in}}%
\pgfpathlineto{\pgfqpoint{4.800755in}{3.100934in}}%
\pgfpathlineto{\pgfqpoint{4.805840in}{3.099146in}}%
\pgfpathlineto{\pgfqpoint{4.810657in}{3.099803in}}%
\pgfpathlineto{\pgfqpoint{4.815741in}{3.102915in}}%
\pgfpathlineto{\pgfqpoint{4.821896in}{3.109553in}}%
\pgfpathlineto{\pgfqpoint{4.831263in}{3.123529in}}%
\pgfpathlineto{\pgfqpoint{4.845178in}{3.143853in}}%
\pgfpathlineto{\pgfqpoint{4.851601in}{3.149471in}}%
\pgfpathlineto{\pgfqpoint{4.856685in}{3.151271in}}%
\pgfpathlineto{\pgfqpoint{4.861502in}{3.150624in}}%
\pgfpathlineto{\pgfqpoint{4.866587in}{3.147524in}}%
\pgfpathlineto{\pgfqpoint{4.872742in}{3.140896in}}%
\pgfpathlineto{\pgfqpoint{4.882108in}{3.126927in}}%
\pgfpathlineto{\pgfqpoint{4.896024in}{3.106592in}}%
\pgfpathlineto{\pgfqpoint{4.902446in}{3.100961in}}%
\pgfpathlineto{\pgfqpoint{4.907531in}{3.099151in}}%
\pgfpathlineto{\pgfqpoint{4.912348in}{3.099786in}}%
\pgfpathlineto{\pgfqpoint{4.917432in}{3.102876in}}%
\pgfpathlineto{\pgfqpoint{4.923587in}{3.109494in}}%
\pgfpathlineto{\pgfqpoint{4.932686in}{3.123024in}}%
\pgfpathlineto{\pgfqpoint{4.946869in}{3.143801in}}%
\pgfpathlineto{\pgfqpoint{4.953292in}{3.149444in}}%
\pgfpathlineto{\pgfqpoint{4.958376in}{3.151266in}}%
\pgfpathlineto{\pgfqpoint{4.963193in}{3.150641in}}%
\pgfpathlineto{\pgfqpoint{4.968278in}{3.147562in}}%
\pgfpathlineto{\pgfqpoint{4.974433in}{3.140955in}}%
\pgfpathlineto{\pgfqpoint{4.983531in}{3.127432in}}%
\pgfpathlineto{\pgfqpoint{4.997715in}{3.106644in}}%
\pgfpathlineto{\pgfqpoint{5.004137in}{3.100989in}}%
\pgfpathlineto{\pgfqpoint{5.009222in}{3.099156in}}%
\pgfpathlineto{\pgfqpoint{5.014039in}{3.099769in}}%
\pgfpathlineto{\pgfqpoint{5.019123in}{3.102837in}}%
\pgfpathlineto{\pgfqpoint{5.025278in}{3.109434in}}%
\pgfpathlineto{\pgfqpoint{5.034377in}{3.122950in}}%
\pgfpathlineto{\pgfqpoint{5.048560in}{3.143749in}}%
\pgfpathlineto{\pgfqpoint{5.054983in}{3.149416in}}%
\pgfpathlineto{\pgfqpoint{5.060067in}{3.151261in}}%
\pgfpathlineto{\pgfqpoint{5.064884in}{3.150658in}}%
\pgfpathlineto{\pgfqpoint{5.069969in}{3.147601in}}%
\pgfpathlineto{\pgfqpoint{5.076124in}{3.141014in}}%
\pgfpathlineto{\pgfqpoint{5.085222in}{3.127506in}}%
\pgfpathlineto{\pgfqpoint{5.099406in}{3.106696in}}%
\pgfpathlineto{\pgfqpoint{5.105828in}{3.101017in}}%
\pgfpathlineto{\pgfqpoint{5.110913in}{3.099161in}}%
\pgfpathlineto{\pgfqpoint{5.115730in}{3.099753in}}%
\pgfpathlineto{\pgfqpoint{5.120814in}{3.102799in}}%
\pgfpathlineto{\pgfqpoint{5.126969in}{3.109375in}}%
\pgfpathlineto{\pgfqpoint{5.136068in}{3.122876in}}%
\pgfpathlineto{\pgfqpoint{5.150251in}{3.143696in}}%
\pgfpathlineto{\pgfqpoint{5.156674in}{3.149388in}}%
\pgfpathlineto{\pgfqpoint{5.161758in}{3.151255in}}%
\pgfpathlineto{\pgfqpoint{5.166575in}{3.150675in}}%
\pgfpathlineto{\pgfqpoint{5.171660in}{3.147639in}}%
\pgfpathlineto{\pgfqpoint{5.177815in}{3.141074in}}%
\pgfpathlineto{\pgfqpoint{5.186913in}{3.127580in}}%
\pgfpathlineto{\pgfqpoint{5.201364in}{3.106446in}}%
\pgfpathlineto{\pgfqpoint{5.207787in}{3.100885in}}%
\pgfpathlineto{\pgfqpoint{5.212871in}{3.099138in}}%
\pgfpathlineto{\pgfqpoint{5.217688in}{3.099835in}}%
\pgfpathlineto{\pgfqpoint{5.222773in}{3.102985in}}%
\pgfpathlineto{\pgfqpoint{5.228928in}{3.109660in}}%
\pgfpathlineto{\pgfqpoint{5.238294in}{3.123662in}}%
\pgfpathlineto{\pgfqpoint{5.251942in}{3.143644in}}%
\pgfpathlineto{\pgfqpoint{5.258365in}{3.149360in}}%
\pgfpathlineto{\pgfqpoint{5.263449in}{3.151250in}}%
\pgfpathlineto{\pgfqpoint{5.268266in}{3.150691in}}%
\pgfpathlineto{\pgfqpoint{5.273351in}{3.147677in}}%
\pgfpathlineto{\pgfqpoint{5.279506in}{3.141133in}}%
\pgfpathlineto{\pgfqpoint{5.288604in}{3.127654in}}%
\pgfpathlineto{\pgfqpoint{5.303055in}{3.106498in}}%
\pgfpathlineto{\pgfqpoint{5.309478in}{3.100912in}}%
\pgfpathlineto{\pgfqpoint{5.314562in}{3.099142in}}%
\pgfpathlineto{\pgfqpoint{5.319379in}{3.099817in}}%
\pgfpathlineto{\pgfqpoint{5.324464in}{3.102946in}}%
\pgfpathlineto{\pgfqpoint{5.330619in}{3.109601in}}%
\pgfpathlineto{\pgfqpoint{5.339985in}{3.123588in}}%
\pgfpathlineto{\pgfqpoint{5.353901in}{3.143895in}}%
\pgfpathlineto{\pgfqpoint{5.360323in}{3.149493in}}%
\pgfpathlineto{\pgfqpoint{5.365408in}{3.151275in}}%
\pgfpathlineto{\pgfqpoint{5.370225in}{3.150610in}}%
\pgfpathlineto{\pgfqpoint{5.375309in}{3.147493in}}%
\pgfpathlineto{\pgfqpoint{5.381464in}{3.140848in}}%
\pgfpathlineto{\pgfqpoint{5.390831in}{3.126868in}}%
\pgfpathlineto{\pgfqpoint{5.404746in}{3.106550in}}%
\pgfpathlineto{\pgfqpoint{5.411169in}{3.100939in}}%
\pgfpathlineto{\pgfqpoint{5.416253in}{3.099147in}}%
\pgfpathlineto{\pgfqpoint{5.421070in}{3.099800in}}%
\pgfpathlineto{\pgfqpoint{5.426155in}{3.102907in}}%
\pgfpathlineto{\pgfqpoint{5.432310in}{3.109541in}}%
\pgfpathlineto{\pgfqpoint{5.441676in}{3.123514in}}%
\pgfpathlineto{\pgfqpoint{5.455592in}{3.143843in}}%
\pgfpathlineto{\pgfqpoint{5.462014in}{3.149466in}}%
\pgfpathlineto{\pgfqpoint{5.467099in}{3.151270in}}%
\pgfpathlineto{\pgfqpoint{5.471916in}{3.150628in}}%
\pgfpathlineto{\pgfqpoint{5.477000in}{3.147531in}}%
\pgfpathlineto{\pgfqpoint{5.483155in}{3.140907in}}%
\pgfpathlineto{\pgfqpoint{5.492522in}{3.126942in}}%
\pgfpathlineto{\pgfqpoint{5.506437in}{3.106602in}}%
\pgfpathlineto{\pgfqpoint{5.512860in}{3.100967in}}%
\pgfpathlineto{\pgfqpoint{5.517944in}{3.099152in}}%
\pgfpathlineto{\pgfqpoint{5.522761in}{3.099783in}}%
\pgfpathlineto{\pgfqpoint{5.527846in}{3.102868in}}%
\pgfpathlineto{\pgfqpoint{5.534001in}{3.109482in}}%
\pgfpathlineto{\pgfqpoint{5.543099in}{3.123009in}}%
\pgfpathlineto{\pgfqpoint{5.557283in}{3.143791in}}%
\pgfpathlineto{\pgfqpoint{5.563705in}{3.149438in}}%
\pgfpathlineto{\pgfqpoint{5.568790in}{3.151265in}}%
\pgfpathlineto{\pgfqpoint{5.573607in}{3.150645in}}%
\pgfpathlineto{\pgfqpoint{5.578691in}{3.147570in}}%
\pgfpathlineto{\pgfqpoint{5.584846in}{3.140967in}}%
\pgfpathlineto{\pgfqpoint{5.593945in}{3.127447in}}%
\pgfpathlineto{\pgfqpoint{5.608128in}{3.106654in}}%
\pgfpathlineto{\pgfqpoint{5.614551in}{3.100995in}}%
\pgfpathlineto{\pgfqpoint{5.619635in}{3.099157in}}%
\pgfpathlineto{\pgfqpoint{5.624452in}{3.099766in}}%
\pgfpathlineto{\pgfqpoint{5.629537in}{3.102830in}}%
\pgfpathlineto{\pgfqpoint{5.635692in}{3.109422in}}%
\pgfpathlineto{\pgfqpoint{5.644790in}{3.122935in}}%
\pgfpathlineto{\pgfqpoint{5.658974in}{3.143738in}}%
\pgfpathlineto{\pgfqpoint{5.665396in}{3.149410in}}%
\pgfpathlineto{\pgfqpoint{5.670481in}{3.151260in}}%
\pgfpathlineto{\pgfqpoint{5.675298in}{3.150661in}}%
\pgfpathlineto{\pgfqpoint{5.680382in}{3.147608in}}%
\pgfpathlineto{\pgfqpoint{5.686537in}{3.141026in}}%
\pgfpathlineto{\pgfqpoint{5.695636in}{3.127521in}}%
\pgfpathlineto{\pgfqpoint{5.709819in}{3.106707in}}%
\pgfpathlineto{\pgfqpoint{5.716242in}{3.101022in}}%
\pgfpathlineto{\pgfqpoint{5.721326in}{3.099162in}}%
\pgfpathlineto{\pgfqpoint{5.726143in}{3.099749in}}%
\pgfpathlineto{\pgfqpoint{5.731228in}{3.102792in}}%
\pgfpathlineto{\pgfqpoint{5.737383in}{3.109363in}}%
\pgfpathlineto{\pgfqpoint{5.746481in}{3.122861in}}%
\pgfpathlineto{\pgfqpoint{5.760932in}{3.143988in}}%
\pgfpathlineto{\pgfqpoint{5.767355in}{3.149542in}}%
\pgfpathlineto{\pgfqpoint{5.772439in}{3.151283in}}%
\pgfpathlineto{\pgfqpoint{5.777256in}{3.150579in}}%
\pgfpathlineto{\pgfqpoint{5.782341in}{3.147422in}}%
\pgfpathlineto{\pgfqpoint{5.788496in}{3.140740in}}%
\pgfpathlineto{\pgfqpoint{5.797862in}{3.126734in}}%
\pgfpathlineto{\pgfqpoint{5.811510in}{3.106759in}}%
\pgfpathlineto{\pgfqpoint{5.817933in}{3.101051in}}%
\pgfpathlineto{\pgfqpoint{5.823017in}{3.099168in}}%
\pgfpathlineto{\pgfqpoint{5.827834in}{3.099733in}}%
\pgfpathlineto{\pgfqpoint{5.832919in}{3.102753in}}%
\pgfpathlineto{\pgfqpoint{5.839074in}{3.109304in}}%
\pgfpathlineto{\pgfqpoint{5.848172in}{3.122787in}}%
\pgfpathlineto{\pgfqpoint{5.862623in}{3.143937in}}%
\pgfpathlineto{\pgfqpoint{5.869046in}{3.149515in}}%
\pgfpathlineto{\pgfqpoint{5.874130in}{3.151278in}}%
\pgfpathlineto{\pgfqpoint{5.878947in}{3.150596in}}%
\pgfpathlineto{\pgfqpoint{5.884032in}{3.147461in}}%
\pgfpathlineto{\pgfqpoint{5.890187in}{3.140800in}}%
\pgfpathlineto{\pgfqpoint{5.899553in}{3.126808in}}%
\pgfpathlineto{\pgfqpoint{5.913201in}{3.106812in}}%
\pgfpathlineto{\pgfqpoint{5.919624in}{3.101079in}}%
\pgfpathlineto{\pgfqpoint{5.924708in}{3.099173in}}%
\pgfpathlineto{\pgfqpoint{5.929525in}{3.099717in}}%
\pgfpathlineto{\pgfqpoint{5.934610in}{3.102716in}}%
\pgfpathlineto{\pgfqpoint{5.940765in}{3.109245in}}%
\pgfpathlineto{\pgfqpoint{5.949863in}{3.122713in}}%
\pgfpathlineto{\pgfqpoint{5.964314in}{3.143885in}}%
\pgfpathlineto{\pgfqpoint{5.970737in}{3.149488in}}%
\pgfpathlineto{\pgfqpoint{5.975821in}{3.151274in}}%
\pgfpathlineto{\pgfqpoint{5.980638in}{3.150614in}}%
\pgfpathlineto{\pgfqpoint{5.985723in}{3.147500in}}%
\pgfpathlineto{\pgfqpoint{5.991878in}{3.140860in}}%
\pgfpathlineto{\pgfqpoint{6.001244in}{3.126883in}}%
\pgfpathlineto{\pgfqpoint{6.015160in}{3.106560in}}%
\pgfpathlineto{\pgfqpoint{6.021582in}{3.100945in}}%
\pgfpathlineto{\pgfqpoint{6.026667in}{3.099148in}}%
\pgfpathlineto{\pgfqpoint{6.031484in}{3.099796in}}%
\pgfpathlineto{\pgfqpoint{6.036568in}{3.102899in}}%
\pgfpathlineto{\pgfqpoint{6.042723in}{3.109529in}}%
\pgfpathlineto{\pgfqpoint{6.052089in}{3.123499in}}%
\pgfpathlineto{\pgfqpoint{6.066005in}{3.143833in}}%
\pgfpathlineto{\pgfqpoint{6.072428in}{3.149460in}}%
\pgfpathlineto{\pgfqpoint{6.077512in}{3.151269in}}%
\pgfpathlineto{\pgfqpoint{6.082329in}{3.150631in}}%
\pgfpathlineto{\pgfqpoint{6.087414in}{3.147539in}}%
\pgfpathlineto{\pgfqpoint{6.093569in}{3.140919in}}%
\pgfpathlineto{\pgfqpoint{6.102667in}{3.127387in}}%
\pgfpathlineto{\pgfqpoint{6.116851in}{3.106612in}}%
\pgfpathlineto{\pgfqpoint{6.123273in}{3.100972in}}%
\pgfpathlineto{\pgfqpoint{6.128358in}{3.099153in}}%
\pgfpathlineto{\pgfqpoint{6.133175in}{3.099779in}}%
\pgfpathlineto{\pgfqpoint{6.138259in}{3.102861in}}%
\pgfpathlineto{\pgfqpoint{6.144414in}{3.109470in}}%
\pgfpathlineto{\pgfqpoint{6.153513in}{3.122994in}}%
\pgfpathlineto{\pgfqpoint{6.167696in}{3.143780in}}%
\pgfpathlineto{\pgfqpoint{6.174119in}{3.149433in}}%
\pgfpathlineto{\pgfqpoint{6.179203in}{3.151264in}}%
\pgfpathlineto{\pgfqpoint{6.184020in}{3.150648in}}%
\pgfpathlineto{\pgfqpoint{6.189105in}{3.147578in}}%
\pgfpathlineto{\pgfqpoint{6.195260in}{3.140979in}}%
\pgfpathlineto{\pgfqpoint{6.204358in}{3.127462in}}%
\pgfpathlineto{\pgfqpoint{6.218542in}{3.106665in}}%
\pgfpathlineto{\pgfqpoint{6.224964in}{3.101000in}}%
\pgfpathlineto{\pgfqpoint{6.230049in}{3.099158in}}%
\pgfpathlineto{\pgfqpoint{6.234866in}{3.099763in}}%
\pgfpathlineto{\pgfqpoint{6.239950in}{3.102822in}}%
\pgfpathlineto{\pgfqpoint{6.246105in}{3.109410in}}%
\pgfpathlineto{\pgfqpoint{6.255204in}{3.122920in}}%
\pgfpathlineto{\pgfqpoint{6.269387in}{3.143728in}}%
\pgfpathlineto{\pgfqpoint{6.275810in}{3.149405in}}%
\pgfpathlineto{\pgfqpoint{6.280894in}{3.151258in}}%
\pgfpathlineto{\pgfqpoint{6.285711in}{3.150665in}}%
\pgfpathlineto{\pgfqpoint{6.290796in}{3.147616in}}%
\pgfpathlineto{\pgfqpoint{6.296951in}{3.141038in}}%
\pgfpathlineto{\pgfqpoint{6.306049in}{3.127536in}}%
\pgfpathlineto{\pgfqpoint{6.320233in}{3.106717in}}%
\pgfpathlineto{\pgfqpoint{6.326655in}{3.101028in}}%
\pgfpathlineto{\pgfqpoint{6.331740in}{3.099163in}}%
\pgfpathlineto{\pgfqpoint{6.336557in}{3.099746in}}%
\pgfpathlineto{\pgfqpoint{6.341641in}{3.102784in}}%
\pgfpathlineto{\pgfqpoint{6.347796in}{3.109351in}}%
\pgfpathlineto{\pgfqpoint{6.356895in}{3.122846in}}%
\pgfpathlineto{\pgfqpoint{6.371346in}{3.143978in}}%
\pgfpathlineto{\pgfqpoint{6.377768in}{3.149537in}}%
\pgfpathlineto{\pgfqpoint{6.382853in}{3.151282in}}%
\pgfpathlineto{\pgfqpoint{6.387670in}{3.150582in}}%
\pgfpathlineto{\pgfqpoint{6.392754in}{3.147430in}}%
\pgfpathlineto{\pgfqpoint{6.398909in}{3.140752in}}%
\pgfpathlineto{\pgfqpoint{6.408276in}{3.126749in}}%
\pgfpathlineto{\pgfqpoint{6.421924in}{3.106770in}}%
\pgfpathlineto{\pgfqpoint{6.428346in}{3.101056in}}%
\pgfpathlineto{\pgfqpoint{6.433431in}{3.099169in}}%
\pgfpathlineto{\pgfqpoint{6.438248in}{3.099730in}}%
\pgfpathlineto{\pgfqpoint{6.443332in}{3.102746in}}%
\pgfpathlineto{\pgfqpoint{6.449487in}{3.109292in}}%
\pgfpathlineto{\pgfqpoint{6.458586in}{3.122772in}}%
\pgfpathlineto{\pgfqpoint{6.473037in}{3.143926in}}%
\pgfpathlineto{\pgfqpoint{6.479459in}{3.149510in}}%
\pgfpathlineto{\pgfqpoint{6.484544in}{3.151277in}}%
\pgfpathlineto{\pgfqpoint{6.489361in}{3.150600in}}%
\pgfpathlineto{\pgfqpoint{6.494445in}{3.147469in}}%
\pgfpathlineto{\pgfqpoint{6.500600in}{3.140812in}}%
\pgfpathlineto{\pgfqpoint{6.509966in}{3.126823in}}%
\pgfpathlineto{\pgfqpoint{6.523614in}{3.106823in}}%
\pgfpathlineto{\pgfqpoint{6.530037in}{3.101085in}}%
\pgfpathlineto{\pgfqpoint{6.535122in}{3.099174in}}%
\pgfpathlineto{\pgfqpoint{6.539939in}{3.099713in}}%
\pgfpathlineto{\pgfqpoint{6.545023in}{3.102708in}}%
\pgfpathlineto{\pgfqpoint{6.551178in}{3.109233in}}%
\pgfpathlineto{\pgfqpoint{6.560277in}{3.122698in}}%
\pgfpathlineto{\pgfqpoint{6.574728in}{3.143874in}}%
\pgfpathlineto{\pgfqpoint{6.581150in}{3.149482in}}%
\pgfpathlineto{\pgfqpoint{6.586235in}{3.151273in}}%
\pgfpathlineto{\pgfqpoint{6.591052in}{3.150617in}}%
\pgfpathlineto{\pgfqpoint{6.596136in}{3.147508in}}%
\pgfpathlineto{\pgfqpoint{6.602291in}{3.140872in}}%
\pgfpathlineto{\pgfqpoint{6.611657in}{3.126898in}}%
\pgfpathlineto{\pgfqpoint{6.625573in}{3.106571in}}%
\pgfpathlineto{\pgfqpoint{6.631996in}{3.100950in}}%
\pgfpathlineto{\pgfqpoint{6.637080in}{3.099149in}}%
\pgfpathlineto{\pgfqpoint{6.641897in}{3.099793in}}%
\pgfpathlineto{\pgfqpoint{6.646982in}{3.102891in}}%
\pgfpathlineto{\pgfqpoint{6.653137in}{3.109517in}}%
\pgfpathlineto{\pgfqpoint{6.662503in}{3.123484in}}%
\pgfpathlineto{\pgfqpoint{6.663306in}{3.124778in}}%
\pgfpathlineto{\pgfqpoint{6.663306in}{3.124778in}}%
\pgfusepath{stroke}%
\end{pgfscope}%
\begin{pgfscope}%
\pgfpathrectangle{\pgfqpoint{0.467797in}{2.292089in}}{\pgfqpoint{6.490533in}{1.666241in}}%
\pgfusepath{clip}%
\pgfsetrectcap%
\pgfsetroundjoin%
\pgfsetlinewidth{1.505625pt}%
\definecolor{currentstroke}{rgb}{0.549020,0.337255,0.294118}%
\pgfsetstrokecolor{currentstroke}%
\pgfsetdash{}{0pt}%
\pgfpathmoveto{\pgfqpoint{0.762821in}{3.125209in}}%
\pgfpathlineto{\pgfqpoint{0.774863in}{3.142651in}}%
\pgfpathlineto{\pgfqpoint{0.781018in}{3.147993in}}%
\pgfpathlineto{\pgfqpoint{0.786103in}{3.149627in}}%
\pgfpathlineto{\pgfqpoint{0.790652in}{3.148778in}}%
\pgfpathlineto{\pgfqpoint{0.795737in}{3.145343in}}%
\pgfpathlineto{\pgfqpoint{0.801892in}{3.138233in}}%
\pgfpathlineto{\pgfqpoint{0.812596in}{3.121692in}}%
\pgfpathlineto{\pgfqpoint{0.823033in}{3.107123in}}%
\pgfpathlineto{\pgfqpoint{0.829188in}{3.102104in}}%
\pgfpathlineto{\pgfqpoint{0.834005in}{3.100780in}}%
\pgfpathlineto{\pgfqpoint{0.838554in}{3.101780in}}%
\pgfpathlineto{\pgfqpoint{0.843638in}{3.105368in}}%
\pgfpathlineto{\pgfqpoint{0.850061in}{3.112991in}}%
\pgfpathlineto{\pgfqpoint{0.862371in}{3.132171in}}%
\pgfpathlineto{\pgfqpoint{0.871470in}{3.144189in}}%
\pgfpathlineto{\pgfqpoint{0.877357in}{3.148603in}}%
\pgfpathlineto{\pgfqpoint{0.882174in}{3.149633in}}%
\pgfpathlineto{\pgfqpoint{0.886723in}{3.148356in}}%
\pgfpathlineto{\pgfqpoint{0.891808in}{3.144488in}}%
\pgfpathlineto{\pgfqpoint{0.898498in}{3.136222in}}%
\pgfpathlineto{\pgfqpoint{0.922583in}{3.103193in}}%
\pgfpathlineto{\pgfqpoint{0.927667in}{3.100933in}}%
\pgfpathlineto{\pgfqpoint{0.932217in}{3.101207in}}%
\pgfpathlineto{\pgfqpoint{0.937034in}{3.103835in}}%
\pgfpathlineto{\pgfqpoint{0.942921in}{3.109914in}}%
\pgfpathlineto{\pgfqpoint{0.951484in}{3.122500in}}%
\pgfpathlineto{\pgfqpoint{0.965668in}{3.143159in}}%
\pgfpathlineto{\pgfqpoint{0.971823in}{3.148249in}}%
\pgfpathlineto{\pgfqpoint{0.976640in}{3.149636in}}%
\pgfpathlineto{\pgfqpoint{0.981189in}{3.148696in}}%
\pgfpathlineto{\pgfqpoint{0.986273in}{3.145168in}}%
\pgfpathlineto{\pgfqpoint{0.992696in}{3.137603in}}%
\pgfpathlineto{\pgfqpoint{1.004471in}{3.119275in}}%
\pgfpathlineto{\pgfqpoint{1.013837in}{3.106635in}}%
\pgfpathlineto{\pgfqpoint{1.019724in}{3.102006in}}%
\pgfpathlineto{\pgfqpoint{1.024541in}{3.100778in}}%
\pgfpathlineto{\pgfqpoint{1.029091in}{3.101868in}}%
\pgfpathlineto{\pgfqpoint{1.034175in}{3.105549in}}%
\pgfpathlineto{\pgfqpoint{1.040598in}{3.113257in}}%
\pgfpathlineto{\pgfqpoint{1.053978in}{3.134093in}}%
\pgfpathlineto{\pgfqpoint{1.062274in}{3.144645in}}%
\pgfpathlineto{\pgfqpoint{1.068161in}{3.148805in}}%
\pgfpathlineto{\pgfqpoint{1.072978in}{3.149603in}}%
\pgfpathlineto{\pgfqpoint{1.077528in}{3.148109in}}%
\pgfpathlineto{\pgfqpoint{1.082880in}{3.143747in}}%
\pgfpathlineto{\pgfqpoint{1.089838in}{3.134770in}}%
\pgfpathlineto{\pgfqpoint{1.111782in}{3.104058in}}%
\pgfpathlineto{\pgfqpoint{1.117134in}{3.101133in}}%
\pgfpathlineto{\pgfqpoint{1.121683in}{3.100982in}}%
\pgfpathlineto{\pgfqpoint{1.126500in}{3.103184in}}%
\pgfpathlineto{\pgfqpoint{1.132120in}{3.108515in}}%
\pgfpathlineto{\pgfqpoint{1.139880in}{3.119404in}}%
\pgfpathlineto{\pgfqpoint{1.156740in}{3.143932in}}%
\pgfpathlineto{\pgfqpoint{1.162627in}{3.148484in}}%
\pgfpathlineto{\pgfqpoint{1.167444in}{3.149640in}}%
\pgfpathlineto{\pgfqpoint{1.171993in}{3.148482in}}%
\pgfpathlineto{\pgfqpoint{1.177078in}{3.144733in}}%
\pgfpathlineto{\pgfqpoint{1.183500in}{3.136961in}}%
\pgfpathlineto{\pgfqpoint{1.208923in}{3.102652in}}%
\pgfpathlineto{\pgfqpoint{1.214008in}{3.100820in}}%
\pgfpathlineto{\pgfqpoint{1.218557in}{3.101488in}}%
\pgfpathlineto{\pgfqpoint{1.223374in}{3.104507in}}%
\pgfpathlineto{\pgfqpoint{1.229529in}{3.111320in}}%
\pgfpathlineto{\pgfqpoint{1.239163in}{3.125975in}}%
\pgfpathlineto{\pgfqpoint{1.250938in}{3.142883in}}%
\pgfpathlineto{\pgfqpoint{1.257093in}{3.148111in}}%
\pgfpathlineto{\pgfqpoint{1.262177in}{3.149636in}}%
\pgfpathlineto{\pgfqpoint{1.266726in}{3.148688in}}%
\pgfpathlineto{\pgfqpoint{1.271811in}{3.145152in}}%
\pgfpathlineto{\pgfqpoint{1.278234in}{3.137579in}}%
\pgfpathlineto{\pgfqpoint{1.290008in}{3.119248in}}%
\pgfpathlineto{\pgfqpoint{1.299375in}{3.106617in}}%
\pgfpathlineto{\pgfqpoint{1.305262in}{3.101997in}}%
\pgfpathlineto{\pgfqpoint{1.310079in}{3.100778in}}%
\pgfpathlineto{\pgfqpoint{1.314628in}{3.101877in}}%
\pgfpathlineto{\pgfqpoint{1.319713in}{3.105565in}}%
\pgfpathlineto{\pgfqpoint{1.326135in}{3.113281in}}%
\pgfpathlineto{\pgfqpoint{1.339516in}{3.134119in}}%
\pgfpathlineto{\pgfqpoint{1.347812in}{3.144662in}}%
\pgfpathlineto{\pgfqpoint{1.353699in}{3.148813in}}%
\pgfpathlineto{\pgfqpoint{1.358516in}{3.149602in}}%
\pgfpathlineto{\pgfqpoint{1.363065in}{3.148099in}}%
\pgfpathlineto{\pgfqpoint{1.368417in}{3.143729in}}%
\pgfpathlineto{\pgfqpoint{1.375375in}{3.134744in}}%
\pgfpathlineto{\pgfqpoint{1.397051in}{3.104263in}}%
\pgfpathlineto{\pgfqpoint{1.402404in}{3.101205in}}%
\pgfpathlineto{\pgfqpoint{1.406953in}{3.100934in}}%
\pgfpathlineto{\pgfqpoint{1.411502in}{3.102835in}}%
\pgfpathlineto{\pgfqpoint{1.416854in}{3.107613in}}%
\pgfpathlineto{\pgfqpoint{1.424080in}{3.117359in}}%
\pgfpathlineto{\pgfqpoint{1.443615in}{3.145260in}}%
\pgfpathlineto{\pgfqpoint{1.449235in}{3.148957in}}%
\pgfpathlineto{\pgfqpoint{1.454052in}{3.149561in}}%
\pgfpathlineto{\pgfqpoint{1.458601in}{3.147888in}}%
\pgfpathlineto{\pgfqpoint{1.463953in}{3.143342in}}%
\pgfpathlineto{\pgfqpoint{1.470911in}{3.134203in}}%
\pgfpathlineto{\pgfqpoint{1.491785in}{3.104637in}}%
\pgfpathlineto{\pgfqpoint{1.497404in}{3.101259in}}%
\pgfpathlineto{\pgfqpoint{1.502221in}{3.100951in}}%
\pgfpathlineto{\pgfqpoint{1.507038in}{3.103076in}}%
\pgfpathlineto{\pgfqpoint{1.512658in}{3.108333in}}%
\pgfpathlineto{\pgfqpoint{1.520419in}{3.119160in}}%
\pgfpathlineto{\pgfqpoint{1.537546in}{3.144048in}}%
\pgfpathlineto{\pgfqpoint{1.543433in}{3.148538in}}%
\pgfpathlineto{\pgfqpoint{1.548250in}{3.149638in}}%
\pgfpathlineto{\pgfqpoint{1.552799in}{3.148426in}}%
\pgfpathlineto{\pgfqpoint{1.557884in}{3.144624in}}%
\pgfpathlineto{\pgfqpoint{1.564306in}{3.136802in}}%
\pgfpathlineto{\pgfqpoint{1.589461in}{3.102749in}}%
\pgfpathlineto{\pgfqpoint{1.594546in}{3.100836in}}%
\pgfpathlineto{\pgfqpoint{1.599095in}{3.101430in}}%
\pgfpathlineto{\pgfqpoint{1.603912in}{3.104375in}}%
\pgfpathlineto{\pgfqpoint{1.609800in}{3.110764in}}%
\pgfpathlineto{\pgfqpoint{1.618898in}{3.124430in}}%
\pgfpathlineto{\pgfqpoint{1.631476in}{3.142709in}}%
\pgfpathlineto{\pgfqpoint{1.637631in}{3.148023in}}%
\pgfpathlineto{\pgfqpoint{1.642715in}{3.149630in}}%
\pgfpathlineto{\pgfqpoint{1.647265in}{3.148756in}}%
\pgfpathlineto{\pgfqpoint{1.652349in}{3.145295in}}%
\pgfpathlineto{\pgfqpoint{1.658504in}{3.138162in}}%
\pgfpathlineto{\pgfqpoint{1.669209in}{3.121609in}}%
\pgfpathlineto{\pgfqpoint{1.679645in}{3.107067in}}%
\pgfpathlineto{\pgfqpoint{1.685800in}{3.102077in}}%
\pgfpathlineto{\pgfqpoint{1.690617in}{3.100779in}}%
\pgfpathlineto{\pgfqpoint{1.695167in}{3.101803in}}%
\pgfpathlineto{\pgfqpoint{1.700251in}{3.105417in}}%
\pgfpathlineto{\pgfqpoint{1.706674in}{3.113063in}}%
\pgfpathlineto{\pgfqpoint{1.719251in}{3.132663in}}%
\pgfpathlineto{\pgfqpoint{1.728082in}{3.144241in}}%
\pgfpathlineto{\pgfqpoint{1.733970in}{3.148627in}}%
\pgfpathlineto{\pgfqpoint{1.738787in}{3.149631in}}%
\pgfpathlineto{\pgfqpoint{1.743336in}{3.148329in}}%
\pgfpathlineto{\pgfqpoint{1.748421in}{3.144436in}}%
\pgfpathlineto{\pgfqpoint{1.755111in}{3.136147in}}%
\pgfpathlineto{\pgfqpoint{1.779195in}{3.103157in}}%
\pgfpathlineto{\pgfqpoint{1.784280in}{3.100923in}}%
\pgfpathlineto{\pgfqpoint{1.788829in}{3.101222in}}%
\pgfpathlineto{\pgfqpoint{1.793646in}{3.103876in}}%
\pgfpathlineto{\pgfqpoint{1.799534in}{3.109979in}}%
\pgfpathlineto{\pgfqpoint{1.808097in}{3.122583in}}%
\pgfpathlineto{\pgfqpoint{1.822013in}{3.142921in}}%
\pgfpathlineto{\pgfqpoint{1.828168in}{3.148131in}}%
\pgfpathlineto{\pgfqpoint{1.833252in}{3.149637in}}%
\pgfpathlineto{\pgfqpoint{1.837802in}{3.148672in}}%
\pgfpathlineto{\pgfqpoint{1.842886in}{3.145119in}}%
\pgfpathlineto{\pgfqpoint{1.849309in}{3.137531in}}%
\pgfpathlineto{\pgfqpoint{1.861351in}{3.118777in}}%
\pgfpathlineto{\pgfqpoint{1.870450in}{3.106581in}}%
\pgfpathlineto{\pgfqpoint{1.876337in}{3.101980in}}%
\pgfpathlineto{\pgfqpoint{1.881154in}{3.100778in}}%
\pgfpathlineto{\pgfqpoint{1.885703in}{3.101893in}}%
\pgfpathlineto{\pgfqpoint{1.890788in}{3.105598in}}%
\pgfpathlineto{\pgfqpoint{1.897210in}{3.113330in}}%
\pgfpathlineto{\pgfqpoint{1.910858in}{3.134571in}}%
\pgfpathlineto{\pgfqpoint{1.919154in}{3.144953in}}%
\pgfpathlineto{\pgfqpoint{1.924774in}{3.148827in}}%
\pgfpathlineto{\pgfqpoint{1.929591in}{3.149598in}}%
\pgfpathlineto{\pgfqpoint{1.934140in}{3.148080in}}%
\pgfpathlineto{\pgfqpoint{1.939492in}{3.143693in}}%
\pgfpathlineto{\pgfqpoint{1.946450in}{3.134693in}}%
\pgfpathlineto{\pgfqpoint{1.968127in}{3.104234in}}%
\pgfpathlineto{\pgfqpoint{1.973479in}{3.101195in}}%
\pgfpathlineto{\pgfqpoint{1.978028in}{3.100940in}}%
\pgfpathlineto{\pgfqpoint{1.982577in}{3.102858in}}%
\pgfpathlineto{\pgfqpoint{1.987930in}{3.107652in}}%
\pgfpathlineto{\pgfqpoint{1.995155in}{3.117412in}}%
\pgfpathlineto{\pgfqpoint{2.014423in}{3.145042in}}%
\pgfpathlineto{\pgfqpoint{2.020042in}{3.148866in}}%
\pgfpathlineto{\pgfqpoint{2.024859in}{3.149589in}}%
\pgfpathlineto{\pgfqpoint{2.029409in}{3.148025in}}%
\pgfpathlineto{\pgfqpoint{2.034761in}{3.143592in}}%
\pgfpathlineto{\pgfqpoint{2.041719in}{3.134552in}}%
\pgfpathlineto{\pgfqpoint{2.063127in}{3.104378in}}%
\pgfpathlineto{\pgfqpoint{2.068479in}{3.101248in}}%
\pgfpathlineto{\pgfqpoint{2.073296in}{3.100958in}}%
\pgfpathlineto{\pgfqpoint{2.078113in}{3.103100in}}%
\pgfpathlineto{\pgfqpoint{2.083733in}{3.108373in}}%
\pgfpathlineto{\pgfqpoint{2.091494in}{3.119214in}}%
\pgfpathlineto{\pgfqpoint{2.108621in}{3.144083in}}%
\pgfpathlineto{\pgfqpoint{2.114508in}{3.148555in}}%
\pgfpathlineto{\pgfqpoint{2.119325in}{3.149637in}}%
\pgfpathlineto{\pgfqpoint{2.123874in}{3.148409in}}%
\pgfpathlineto{\pgfqpoint{2.128959in}{3.144590in}}%
\pgfpathlineto{\pgfqpoint{2.135381in}{3.136753in}}%
\pgfpathlineto{\pgfqpoint{2.160537in}{3.102727in}}%
\pgfpathlineto{\pgfqpoint{2.165621in}{3.100832in}}%
\pgfpathlineto{\pgfqpoint{2.170170in}{3.101442in}}%
\pgfpathlineto{\pgfqpoint{2.174987in}{3.104404in}}%
\pgfpathlineto{\pgfqpoint{2.180875in}{3.110809in}}%
\pgfpathlineto{\pgfqpoint{2.190241in}{3.124917in}}%
\pgfpathlineto{\pgfqpoint{2.202551in}{3.142748in}}%
\pgfpathlineto{\pgfqpoint{2.208706in}{3.148043in}}%
\pgfpathlineto{\pgfqpoint{2.213791in}{3.149631in}}%
\pgfpathlineto{\pgfqpoint{2.218340in}{3.148741in}}%
\pgfpathlineto{\pgfqpoint{2.223424in}{3.145264in}}%
\pgfpathlineto{\pgfqpoint{2.229579in}{3.138115in}}%
\pgfpathlineto{\pgfqpoint{2.240551in}{3.121128in}}%
\pgfpathlineto{\pgfqpoint{2.250720in}{3.107030in}}%
\pgfpathlineto{\pgfqpoint{2.256875in}{3.102059in}}%
\pgfpathlineto{\pgfqpoint{2.261692in}{3.100778in}}%
\pgfpathlineto{\pgfqpoint{2.266242in}{3.101819in}}%
\pgfpathlineto{\pgfqpoint{2.271326in}{3.105450in}}%
\pgfpathlineto{\pgfqpoint{2.277749in}{3.113111in}}%
\pgfpathlineto{\pgfqpoint{2.290326in}{3.132716in}}%
\pgfpathlineto{\pgfqpoint{2.299157in}{3.144276in}}%
\pgfpathlineto{\pgfqpoint{2.305045in}{3.148643in}}%
\pgfpathlineto{\pgfqpoint{2.309862in}{3.149629in}}%
\pgfpathlineto{\pgfqpoint{2.314411in}{3.148311in}}%
\pgfpathlineto{\pgfqpoint{2.319496in}{3.144402in}}%
\pgfpathlineto{\pgfqpoint{2.326186in}{3.136098in}}%
\pgfpathlineto{\pgfqpoint{2.350003in}{3.103321in}}%
\pgfpathlineto{\pgfqpoint{2.355087in}{3.100967in}}%
\pgfpathlineto{\pgfqpoint{2.359637in}{3.101154in}}%
\pgfpathlineto{\pgfqpoint{2.364454in}{3.103695in}}%
\pgfpathlineto{\pgfqpoint{2.370073in}{3.109356in}}%
\pgfpathlineto{\pgfqpoint{2.378369in}{3.121355in}}%
\pgfpathlineto{\pgfqpoint{2.393355in}{3.143253in}}%
\pgfpathlineto{\pgfqpoint{2.399510in}{3.148295in}}%
\pgfpathlineto{\pgfqpoint{2.404327in}{3.149638in}}%
\pgfpathlineto{\pgfqpoint{2.408877in}{3.148657in}}%
\pgfpathlineto{\pgfqpoint{2.413961in}{3.145087in}}%
\pgfpathlineto{\pgfqpoint{2.420384in}{3.137483in}}%
\pgfpathlineto{\pgfqpoint{2.432426in}{3.118723in}}%
\pgfpathlineto{\pgfqpoint{2.441525in}{3.106545in}}%
\pgfpathlineto{\pgfqpoint{2.447412in}{3.101963in}}%
\pgfpathlineto{\pgfqpoint{2.452229in}{3.100778in}}%
\pgfpathlineto{\pgfqpoint{2.456778in}{3.101910in}}%
\pgfpathlineto{\pgfqpoint{2.461863in}{3.105632in}}%
\pgfpathlineto{\pgfqpoint{2.468286in}{3.113378in}}%
\pgfpathlineto{\pgfqpoint{2.482469in}{3.135413in}}%
\pgfpathlineto{\pgfqpoint{2.490497in}{3.145236in}}%
\pgfpathlineto{\pgfqpoint{2.496117in}{3.148947in}}%
\pgfpathlineto{\pgfqpoint{2.500934in}{3.149565in}}%
\pgfpathlineto{\pgfqpoint{2.505483in}{3.147904in}}%
\pgfpathlineto{\pgfqpoint{2.510835in}{3.143370in}}%
\pgfpathlineto{\pgfqpoint{2.517793in}{3.134242in}}%
\pgfpathlineto{\pgfqpoint{2.538934in}{3.104430in}}%
\pgfpathlineto{\pgfqpoint{2.544286in}{3.101267in}}%
\pgfpathlineto{\pgfqpoint{2.549103in}{3.100946in}}%
\pgfpathlineto{\pgfqpoint{2.553920in}{3.103059in}}%
\pgfpathlineto{\pgfqpoint{2.559540in}{3.108303in}}%
\pgfpathlineto{\pgfqpoint{2.567300in}{3.119120in}}%
\pgfpathlineto{\pgfqpoint{2.584695in}{3.144294in}}%
\pgfpathlineto{\pgfqpoint{2.590582in}{3.148651in}}%
\pgfpathlineto{\pgfqpoint{2.595399in}{3.149628in}}%
\pgfpathlineto{\pgfqpoint{2.599949in}{3.148302in}}%
\pgfpathlineto{\pgfqpoint{2.605033in}{3.144384in}}%
\pgfpathlineto{\pgfqpoint{2.611723in}{3.136073in}}%
\pgfpathlineto{\pgfqpoint{2.635540in}{3.103309in}}%
\pgfpathlineto{\pgfqpoint{2.640625in}{3.100964in}}%
\pgfpathlineto{\pgfqpoint{2.645174in}{3.101159in}}%
\pgfpathlineto{\pgfqpoint{2.649991in}{3.103708in}}%
\pgfpathlineto{\pgfqpoint{2.655611in}{3.109377in}}%
\pgfpathlineto{\pgfqpoint{2.663907in}{3.121382in}}%
\pgfpathlineto{\pgfqpoint{2.678893in}{3.143272in}}%
\pgfpathlineto{\pgfqpoint{2.685048in}{3.148304in}}%
\pgfpathlineto{\pgfqpoint{2.689865in}{3.149639in}}%
\pgfpathlineto{\pgfqpoint{2.694414in}{3.148649in}}%
\pgfpathlineto{\pgfqpoint{2.699499in}{3.145071in}}%
\pgfpathlineto{\pgfqpoint{2.705921in}{3.137458in}}%
\pgfpathlineto{\pgfqpoint{2.718231in}{3.118281in}}%
\pgfpathlineto{\pgfqpoint{2.727330in}{3.106252in}}%
\pgfpathlineto{\pgfqpoint{2.733217in}{3.101826in}}%
\pgfpathlineto{\pgfqpoint{2.738034in}{3.100785in}}%
\pgfpathlineto{\pgfqpoint{2.742584in}{3.102052in}}%
\pgfpathlineto{\pgfqpoint{2.747668in}{3.105910in}}%
\pgfpathlineto{\pgfqpoint{2.754358in}{3.114166in}}%
\pgfpathlineto{\pgfqpoint{2.778443in}{3.147211in}}%
\pgfpathlineto{\pgfqpoint{2.783528in}{3.149482in}}%
\pgfpathlineto{\pgfqpoint{2.788077in}{3.149219in}}%
\pgfpathlineto{\pgfqpoint{2.792894in}{3.146601in}}%
\pgfpathlineto{\pgfqpoint{2.798781in}{3.140532in}}%
\pgfpathlineto{\pgfqpoint{2.807345in}{3.127954in}}%
\pgfpathlineto{\pgfqpoint{2.821528in}{3.107283in}}%
\pgfpathlineto{\pgfqpoint{2.827683in}{3.102182in}}%
\pgfpathlineto{\pgfqpoint{2.832500in}{3.100784in}}%
\pgfpathlineto{\pgfqpoint{2.837049in}{3.101714in}}%
\pgfpathlineto{\pgfqpoint{2.842134in}{3.105231in}}%
\pgfpathlineto{\pgfqpoint{2.848556in}{3.112786in}}%
\pgfpathlineto{\pgfqpoint{2.860331in}{3.131110in}}%
\pgfpathlineto{\pgfqpoint{2.869697in}{3.143761in}}%
\pgfpathlineto{\pgfqpoint{2.875585in}{3.148402in}}%
\pgfpathlineto{\pgfqpoint{2.880402in}{3.149641in}}%
\pgfpathlineto{\pgfqpoint{2.884951in}{3.148561in}}%
\pgfpathlineto{\pgfqpoint{2.890035in}{3.144891in}}%
\pgfpathlineto{\pgfqpoint{2.896458in}{3.137192in}}%
\pgfpathlineto{\pgfqpoint{2.909571in}{3.116762in}}%
\pgfpathlineto{\pgfqpoint{2.918134in}{3.105795in}}%
\pgfpathlineto{\pgfqpoint{2.924022in}{3.101622in}}%
\pgfpathlineto{\pgfqpoint{2.928839in}{3.100814in}}%
\pgfpathlineto{\pgfqpoint{2.933388in}{3.102298in}}%
\pgfpathlineto{\pgfqpoint{2.938472in}{3.106371in}}%
\pgfpathlineto{\pgfqpoint{2.945163in}{3.114829in}}%
\pgfpathlineto{\pgfqpoint{2.968177in}{3.146763in}}%
\pgfpathlineto{\pgfqpoint{2.973529in}{3.149412in}}%
\pgfpathlineto{\pgfqpoint{2.978078in}{3.149316in}}%
\pgfpathlineto{\pgfqpoint{2.982895in}{3.146864in}}%
\pgfpathlineto{\pgfqpoint{2.988515in}{3.141289in}}%
\pgfpathlineto{\pgfqpoint{2.996543in}{3.129784in}}%
\pgfpathlineto{\pgfqpoint{3.012332in}{3.106790in}}%
\pgfpathlineto{\pgfqpoint{3.018220in}{3.102081in}}%
\pgfpathlineto{\pgfqpoint{3.023037in}{3.100779in}}%
\pgfpathlineto{\pgfqpoint{3.027586in}{3.101799in}}%
\pgfpathlineto{\pgfqpoint{3.032670in}{3.105409in}}%
\pgfpathlineto{\pgfqpoint{3.039093in}{3.113051in}}%
\pgfpathlineto{\pgfqpoint{3.051671in}{3.132650in}}%
\pgfpathlineto{\pgfqpoint{3.060502in}{3.144233in}}%
\pgfpathlineto{\pgfqpoint{3.066389in}{3.148623in}}%
\pgfpathlineto{\pgfqpoint{3.071206in}{3.149631in}}%
\pgfpathlineto{\pgfqpoint{3.075755in}{3.148333in}}%
\pgfpathlineto{\pgfqpoint{3.080840in}{3.144445in}}%
\pgfpathlineto{\pgfqpoint{3.087530in}{3.136160in}}%
\pgfpathlineto{\pgfqpoint{3.111615in}{3.103163in}}%
\pgfpathlineto{\pgfqpoint{3.116699in}{3.100925in}}%
\pgfpathlineto{\pgfqpoint{3.121249in}{3.101220in}}%
\pgfpathlineto{\pgfqpoint{3.126066in}{3.103869in}}%
\pgfpathlineto{\pgfqpoint{3.131953in}{3.109968in}}%
\pgfpathlineto{\pgfqpoint{3.140516in}{3.122569in}}%
\pgfpathlineto{\pgfqpoint{3.154432in}{3.142912in}}%
\pgfpathlineto{\pgfqpoint{3.160587in}{3.148126in}}%
\pgfpathlineto{\pgfqpoint{3.165672in}{3.149637in}}%
\pgfpathlineto{\pgfqpoint{3.170221in}{3.148676in}}%
\pgfpathlineto{\pgfqpoint{3.175305in}{3.145127in}}%
\pgfpathlineto{\pgfqpoint{3.181728in}{3.137543in}}%
\pgfpathlineto{\pgfqpoint{3.193770in}{3.118790in}}%
\pgfpathlineto{\pgfqpoint{3.202869in}{3.106590in}}%
\pgfpathlineto{\pgfqpoint{3.208756in}{3.101984in}}%
\pgfpathlineto{\pgfqpoint{3.213573in}{3.100778in}}%
\pgfpathlineto{\pgfqpoint{3.218123in}{3.101889in}}%
\pgfpathlineto{\pgfqpoint{3.223207in}{3.105590in}}%
\pgfpathlineto{\pgfqpoint{3.229630in}{3.113318in}}%
\pgfpathlineto{\pgfqpoint{3.243278in}{3.134558in}}%
\pgfpathlineto{\pgfqpoint{3.251574in}{3.144944in}}%
\pgfpathlineto{\pgfqpoint{3.257193in}{3.148823in}}%
\pgfpathlineto{\pgfqpoint{3.262010in}{3.149599in}}%
\pgfpathlineto{\pgfqpoint{3.266560in}{3.148084in}}%
\pgfpathlineto{\pgfqpoint{3.271912in}{3.143702in}}%
\pgfpathlineto{\pgfqpoint{3.278870in}{3.134706in}}%
\pgfpathlineto{\pgfqpoint{3.300546in}{3.104241in}}%
\pgfpathlineto{\pgfqpoint{3.305898in}{3.101198in}}%
\pgfpathlineto{\pgfqpoint{3.310447in}{3.100938in}}%
\pgfpathlineto{\pgfqpoint{3.314997in}{3.102852in}}%
\pgfpathlineto{\pgfqpoint{3.320349in}{3.107642in}}%
\pgfpathlineto{\pgfqpoint{3.327574in}{3.117399in}}%
\pgfpathlineto{\pgfqpoint{3.346842in}{3.145034in}}%
\pgfpathlineto{\pgfqpoint{3.352462in}{3.148862in}}%
\pgfpathlineto{\pgfqpoint{3.357279in}{3.149590in}}%
\pgfpathlineto{\pgfqpoint{3.361828in}{3.148030in}}%
\pgfpathlineto{\pgfqpoint{3.367180in}{3.143601in}}%
\pgfpathlineto{\pgfqpoint{3.374138in}{3.134565in}}%
\pgfpathlineto{\pgfqpoint{3.395547in}{3.104386in}}%
\pgfpathlineto{\pgfqpoint{3.400899in}{3.101251in}}%
\pgfpathlineto{\pgfqpoint{3.405716in}{3.100956in}}%
\pgfpathlineto{\pgfqpoint{3.410533in}{3.103094in}}%
\pgfpathlineto{\pgfqpoint{3.416152in}{3.108363in}}%
\pgfpathlineto{\pgfqpoint{3.423913in}{3.119201in}}%
\pgfpathlineto{\pgfqpoint{3.441040in}{3.144074in}}%
\pgfpathlineto{\pgfqpoint{3.446927in}{3.148551in}}%
\pgfpathlineto{\pgfqpoint{3.451744in}{3.149637in}}%
\pgfpathlineto{\pgfqpoint{3.456294in}{3.148413in}}%
\pgfpathlineto{\pgfqpoint{3.461378in}{3.144598in}}%
\pgfpathlineto{\pgfqpoint{3.467801in}{3.136765in}}%
\pgfpathlineto{\pgfqpoint{3.492956in}{3.102733in}}%
\pgfpathlineto{\pgfqpoint{3.498040in}{3.100833in}}%
\pgfpathlineto{\pgfqpoint{3.502590in}{3.101439in}}%
\pgfpathlineto{\pgfqpoint{3.507407in}{3.104397in}}%
\pgfpathlineto{\pgfqpoint{3.513294in}{3.110798in}}%
\pgfpathlineto{\pgfqpoint{3.522660in}{3.124903in}}%
\pgfpathlineto{\pgfqpoint{3.534970in}{3.142738in}}%
\pgfpathlineto{\pgfqpoint{3.541125in}{3.148038in}}%
\pgfpathlineto{\pgfqpoint{3.546210in}{3.149631in}}%
\pgfpathlineto{\pgfqpoint{3.550759in}{3.148745in}}%
\pgfpathlineto{\pgfqpoint{3.555844in}{3.145271in}}%
\pgfpathlineto{\pgfqpoint{3.561999in}{3.138127in}}%
\pgfpathlineto{\pgfqpoint{3.572971in}{3.121142in}}%
\pgfpathlineto{\pgfqpoint{3.583140in}{3.107039in}}%
\pgfpathlineto{\pgfqpoint{3.589295in}{3.102063in}}%
\pgfpathlineto{\pgfqpoint{3.594112in}{3.100778in}}%
\pgfpathlineto{\pgfqpoint{3.598661in}{3.101815in}}%
\pgfpathlineto{\pgfqpoint{3.603746in}{3.105442in}}%
\pgfpathlineto{\pgfqpoint{3.610168in}{3.113099in}}%
\pgfpathlineto{\pgfqpoint{3.622746in}{3.132703in}}%
\pgfpathlineto{\pgfqpoint{3.631577in}{3.144267in}}%
\pgfpathlineto{\pgfqpoint{3.637464in}{3.148639in}}%
\pgfpathlineto{\pgfqpoint{3.642281in}{3.149630in}}%
\pgfpathlineto{\pgfqpoint{3.646830in}{3.148315in}}%
\pgfpathlineto{\pgfqpoint{3.651915in}{3.144410in}}%
\pgfpathlineto{\pgfqpoint{3.658605in}{3.136110in}}%
\pgfpathlineto{\pgfqpoint{3.682690in}{3.103139in}}%
\pgfpathlineto{\pgfqpoint{3.687774in}{3.100919in}}%
\pgfpathlineto{\pgfqpoint{3.692324in}{3.101230in}}%
\pgfpathlineto{\pgfqpoint{3.697141in}{3.103896in}}%
\pgfpathlineto{\pgfqpoint{3.703028in}{3.110012in}}%
\pgfpathlineto{\pgfqpoint{3.711591in}{3.122624in}}%
\pgfpathlineto{\pgfqpoint{3.725507in}{3.142950in}}%
\pgfpathlineto{\pgfqpoint{3.731662in}{3.148145in}}%
\pgfpathlineto{\pgfqpoint{3.736479in}{3.149627in}}%
\pgfpathlineto{\pgfqpoint{3.741028in}{3.148778in}}%
\pgfpathlineto{\pgfqpoint{3.746113in}{3.145343in}}%
\pgfpathlineto{\pgfqpoint{3.752268in}{3.138233in}}%
\pgfpathlineto{\pgfqpoint{3.762972in}{3.121692in}}%
\pgfpathlineto{\pgfqpoint{3.773409in}{3.107123in}}%
\pgfpathlineto{\pgfqpoint{3.779564in}{3.102104in}}%
\pgfpathlineto{\pgfqpoint{3.784381in}{3.100780in}}%
\pgfpathlineto{\pgfqpoint{3.788930in}{3.101780in}}%
\pgfpathlineto{\pgfqpoint{3.794015in}{3.105368in}}%
\pgfpathlineto{\pgfqpoint{3.800437in}{3.112991in}}%
\pgfpathlineto{\pgfqpoint{3.812747in}{3.132171in}}%
\pgfpathlineto{\pgfqpoint{3.821846in}{3.144189in}}%
\pgfpathlineto{\pgfqpoint{3.827733in}{3.148603in}}%
\pgfpathlineto{\pgfqpoint{3.832550in}{3.149633in}}%
\pgfpathlineto{\pgfqpoint{3.837100in}{3.148356in}}%
\pgfpathlineto{\pgfqpoint{3.842184in}{3.144488in}}%
\pgfpathlineto{\pgfqpoint{3.848874in}{3.136222in}}%
\pgfpathlineto{\pgfqpoint{3.872959in}{3.103193in}}%
\pgfpathlineto{\pgfqpoint{3.878044in}{3.100933in}}%
\pgfpathlineto{\pgfqpoint{3.882593in}{3.101207in}}%
\pgfpathlineto{\pgfqpoint{3.887410in}{3.103835in}}%
\pgfpathlineto{\pgfqpoint{3.893297in}{3.109914in}}%
\pgfpathlineto{\pgfqpoint{3.901861in}{3.122500in}}%
\pgfpathlineto{\pgfqpoint{3.916044in}{3.143159in}}%
\pgfpathlineto{\pgfqpoint{3.922199in}{3.148249in}}%
\pgfpathlineto{\pgfqpoint{3.927016in}{3.149636in}}%
\pgfpathlineto{\pgfqpoint{3.931565in}{3.148696in}}%
\pgfpathlineto{\pgfqpoint{3.936650in}{3.145168in}}%
\pgfpathlineto{\pgfqpoint{3.943072in}{3.137603in}}%
\pgfpathlineto{\pgfqpoint{3.954847in}{3.119275in}}%
\pgfpathlineto{\pgfqpoint{3.964213in}{3.106635in}}%
\pgfpathlineto{\pgfqpoint{3.970101in}{3.102006in}}%
\pgfpathlineto{\pgfqpoint{3.974918in}{3.100778in}}%
\pgfpathlineto{\pgfqpoint{3.979467in}{3.101868in}}%
\pgfpathlineto{\pgfqpoint{3.984551in}{3.105549in}}%
\pgfpathlineto{\pgfqpoint{3.990974in}{3.113257in}}%
\pgfpathlineto{\pgfqpoint{4.004354in}{3.134093in}}%
\pgfpathlineto{\pgfqpoint{4.012650in}{3.144645in}}%
\pgfpathlineto{\pgfqpoint{4.018538in}{3.148805in}}%
\pgfpathlineto{\pgfqpoint{4.023355in}{3.149603in}}%
\pgfpathlineto{\pgfqpoint{4.027904in}{3.148109in}}%
\pgfpathlineto{\pgfqpoint{4.033256in}{3.143747in}}%
\pgfpathlineto{\pgfqpoint{4.040214in}{3.134770in}}%
\pgfpathlineto{\pgfqpoint{4.062158in}{3.104058in}}%
\pgfpathlineto{\pgfqpoint{4.067510in}{3.101133in}}%
\pgfpathlineto{\pgfqpoint{4.072059in}{3.100982in}}%
\pgfpathlineto{\pgfqpoint{4.076876in}{3.103184in}}%
\pgfpathlineto{\pgfqpoint{4.082496in}{3.108515in}}%
\pgfpathlineto{\pgfqpoint{4.090257in}{3.119404in}}%
\pgfpathlineto{\pgfqpoint{4.107116in}{3.143932in}}%
\pgfpathlineto{\pgfqpoint{4.113003in}{3.148484in}}%
\pgfpathlineto{\pgfqpoint{4.117820in}{3.149640in}}%
\pgfpathlineto{\pgfqpoint{4.122369in}{3.148482in}}%
\pgfpathlineto{\pgfqpoint{4.127454in}{3.144733in}}%
\pgfpathlineto{\pgfqpoint{4.133877in}{3.136961in}}%
\pgfpathlineto{\pgfqpoint{4.159299in}{3.102652in}}%
\pgfpathlineto{\pgfqpoint{4.164384in}{3.100820in}}%
\pgfpathlineto{\pgfqpoint{4.168933in}{3.101488in}}%
\pgfpathlineto{\pgfqpoint{4.173750in}{3.104507in}}%
\pgfpathlineto{\pgfqpoint{4.179905in}{3.111320in}}%
\pgfpathlineto{\pgfqpoint{4.189539in}{3.125975in}}%
\pgfpathlineto{\pgfqpoint{4.201314in}{3.142883in}}%
\pgfpathlineto{\pgfqpoint{4.207469in}{3.148111in}}%
\pgfpathlineto{\pgfqpoint{4.212553in}{3.149636in}}%
\pgfpathlineto{\pgfqpoint{4.217103in}{3.148688in}}%
\pgfpathlineto{\pgfqpoint{4.222187in}{3.145152in}}%
\pgfpathlineto{\pgfqpoint{4.228610in}{3.137579in}}%
\pgfpathlineto{\pgfqpoint{4.240385in}{3.119248in}}%
\pgfpathlineto{\pgfqpoint{4.249751in}{3.106617in}}%
\pgfpathlineto{\pgfqpoint{4.255638in}{3.101997in}}%
\pgfpathlineto{\pgfqpoint{4.260455in}{3.100778in}}%
\pgfpathlineto{\pgfqpoint{4.265004in}{3.101877in}}%
\pgfpathlineto{\pgfqpoint{4.270089in}{3.105565in}}%
\pgfpathlineto{\pgfqpoint{4.276512in}{3.113281in}}%
\pgfpathlineto{\pgfqpoint{4.289892in}{3.134119in}}%
\pgfpathlineto{\pgfqpoint{4.298188in}{3.144662in}}%
\pgfpathlineto{\pgfqpoint{4.304075in}{3.148813in}}%
\pgfpathlineto{\pgfqpoint{4.308892in}{3.149602in}}%
\pgfpathlineto{\pgfqpoint{4.313441in}{3.148099in}}%
\pgfpathlineto{\pgfqpoint{4.318794in}{3.143729in}}%
\pgfpathlineto{\pgfqpoint{4.325751in}{3.134744in}}%
\pgfpathlineto{\pgfqpoint{4.347428in}{3.104263in}}%
\pgfpathlineto{\pgfqpoint{4.352780in}{3.101205in}}%
\pgfpathlineto{\pgfqpoint{4.357329in}{3.100934in}}%
\pgfpathlineto{\pgfqpoint{4.361878in}{3.102835in}}%
\pgfpathlineto{\pgfqpoint{4.367231in}{3.107613in}}%
\pgfpathlineto{\pgfqpoint{4.374456in}{3.117359in}}%
\pgfpathlineto{\pgfqpoint{4.393991in}{3.145260in}}%
\pgfpathlineto{\pgfqpoint{4.399611in}{3.148957in}}%
\pgfpathlineto{\pgfqpoint{4.404428in}{3.149561in}}%
\pgfpathlineto{\pgfqpoint{4.408977in}{3.147888in}}%
\pgfpathlineto{\pgfqpoint{4.414330in}{3.143342in}}%
\pgfpathlineto{\pgfqpoint{4.421287in}{3.134203in}}%
\pgfpathlineto{\pgfqpoint{4.442161in}{3.104637in}}%
\pgfpathlineto{\pgfqpoint{4.447781in}{3.101259in}}%
\pgfpathlineto{\pgfqpoint{4.452598in}{3.100951in}}%
\pgfpathlineto{\pgfqpoint{4.457414in}{3.103076in}}%
\pgfpathlineto{\pgfqpoint{4.463034in}{3.108333in}}%
\pgfpathlineto{\pgfqpoint{4.470795in}{3.119160in}}%
\pgfpathlineto{\pgfqpoint{4.487922in}{3.144048in}}%
\pgfpathlineto{\pgfqpoint{4.493809in}{3.148538in}}%
\pgfpathlineto{\pgfqpoint{4.498626in}{3.149638in}}%
\pgfpathlineto{\pgfqpoint{4.503175in}{3.148426in}}%
\pgfpathlineto{\pgfqpoint{4.508260in}{3.144624in}}%
\pgfpathlineto{\pgfqpoint{4.514683in}{3.136802in}}%
\pgfpathlineto{\pgfqpoint{4.539838in}{3.102749in}}%
\pgfpathlineto{\pgfqpoint{4.544922in}{3.100836in}}%
\pgfpathlineto{\pgfqpoint{4.549472in}{3.101430in}}%
\pgfpathlineto{\pgfqpoint{4.554289in}{3.104375in}}%
\pgfpathlineto{\pgfqpoint{4.560176in}{3.110764in}}%
\pgfpathlineto{\pgfqpoint{4.569275in}{3.124430in}}%
\pgfpathlineto{\pgfqpoint{4.581852in}{3.142709in}}%
\pgfpathlineto{\pgfqpoint{4.588007in}{3.148023in}}%
\pgfpathlineto{\pgfqpoint{4.593092in}{3.149630in}}%
\pgfpathlineto{\pgfqpoint{4.597641in}{3.148756in}}%
\pgfpathlineto{\pgfqpoint{4.602726in}{3.145295in}}%
\pgfpathlineto{\pgfqpoint{4.608881in}{3.138162in}}%
\pgfpathlineto{\pgfqpoint{4.619585in}{3.121609in}}%
\pgfpathlineto{\pgfqpoint{4.630022in}{3.107067in}}%
\pgfpathlineto{\pgfqpoint{4.636177in}{3.102077in}}%
\pgfpathlineto{\pgfqpoint{4.640993in}{3.100779in}}%
\pgfpathlineto{\pgfqpoint{4.645543in}{3.101803in}}%
\pgfpathlineto{\pgfqpoint{4.650627in}{3.105417in}}%
\pgfpathlineto{\pgfqpoint{4.657050in}{3.113063in}}%
\pgfpathlineto{\pgfqpoint{4.669627in}{3.132663in}}%
\pgfpathlineto{\pgfqpoint{4.678459in}{3.144241in}}%
\pgfpathlineto{\pgfqpoint{4.684346in}{3.148627in}}%
\pgfpathlineto{\pgfqpoint{4.689163in}{3.149631in}}%
\pgfpathlineto{\pgfqpoint{4.693712in}{3.148329in}}%
\pgfpathlineto{\pgfqpoint{4.698797in}{3.144436in}}%
\pgfpathlineto{\pgfqpoint{4.705487in}{3.136147in}}%
\pgfpathlineto{\pgfqpoint{4.729572in}{3.103157in}}%
\pgfpathlineto{\pgfqpoint{4.734656in}{3.100923in}}%
\pgfpathlineto{\pgfqpoint{4.739206in}{3.101222in}}%
\pgfpathlineto{\pgfqpoint{4.744022in}{3.103876in}}%
\pgfpathlineto{\pgfqpoint{4.749910in}{3.109979in}}%
\pgfpathlineto{\pgfqpoint{4.758473in}{3.122583in}}%
\pgfpathlineto{\pgfqpoint{4.772389in}{3.142921in}}%
\pgfpathlineto{\pgfqpoint{4.778544in}{3.148131in}}%
\pgfpathlineto{\pgfqpoint{4.783628in}{3.149637in}}%
\pgfpathlineto{\pgfqpoint{4.788178in}{3.148672in}}%
\pgfpathlineto{\pgfqpoint{4.793262in}{3.145119in}}%
\pgfpathlineto{\pgfqpoint{4.799685in}{3.137531in}}%
\pgfpathlineto{\pgfqpoint{4.811727in}{3.118777in}}%
\pgfpathlineto{\pgfqpoint{4.820826in}{3.106581in}}%
\pgfpathlineto{\pgfqpoint{4.826713in}{3.101980in}}%
\pgfpathlineto{\pgfqpoint{4.831530in}{3.100778in}}%
\pgfpathlineto{\pgfqpoint{4.836080in}{3.101893in}}%
\pgfpathlineto{\pgfqpoint{4.841164in}{3.105598in}}%
\pgfpathlineto{\pgfqpoint{4.847587in}{3.113330in}}%
\pgfpathlineto{\pgfqpoint{4.861235in}{3.134571in}}%
\pgfpathlineto{\pgfqpoint{4.869531in}{3.144953in}}%
\pgfpathlineto{\pgfqpoint{4.875150in}{3.148827in}}%
\pgfpathlineto{\pgfqpoint{4.879967in}{3.149598in}}%
\pgfpathlineto{\pgfqpoint{4.884517in}{3.148080in}}%
\pgfpathlineto{\pgfqpoint{4.889869in}{3.143693in}}%
\pgfpathlineto{\pgfqpoint{4.896827in}{3.134693in}}%
\pgfpathlineto{\pgfqpoint{4.918503in}{3.104234in}}%
\pgfpathlineto{\pgfqpoint{4.923855in}{3.101195in}}%
\pgfpathlineto{\pgfqpoint{4.928404in}{3.100940in}}%
\pgfpathlineto{\pgfqpoint{4.932954in}{3.102858in}}%
\pgfpathlineto{\pgfqpoint{4.938306in}{3.107652in}}%
\pgfpathlineto{\pgfqpoint{4.945531in}{3.117412in}}%
\pgfpathlineto{\pgfqpoint{4.964799in}{3.145042in}}%
\pgfpathlineto{\pgfqpoint{4.970419in}{3.148866in}}%
\pgfpathlineto{\pgfqpoint{4.975236in}{3.149589in}}%
\pgfpathlineto{\pgfqpoint{4.979785in}{3.148025in}}%
\pgfpathlineto{\pgfqpoint{4.985137in}{3.143592in}}%
\pgfpathlineto{\pgfqpoint{4.992095in}{3.134552in}}%
\pgfpathlineto{\pgfqpoint{5.013504in}{3.104378in}}%
\pgfpathlineto{\pgfqpoint{5.018856in}{3.101248in}}%
\pgfpathlineto{\pgfqpoint{5.023673in}{3.100958in}}%
\pgfpathlineto{\pgfqpoint{5.028490in}{3.103100in}}%
\pgfpathlineto{\pgfqpoint{5.034109in}{3.108373in}}%
\pgfpathlineto{\pgfqpoint{5.041870in}{3.119214in}}%
\pgfpathlineto{\pgfqpoint{5.058997in}{3.144083in}}%
\pgfpathlineto{\pgfqpoint{5.064884in}{3.148555in}}%
\pgfpathlineto{\pgfqpoint{5.069701in}{3.149637in}}%
\pgfpathlineto{\pgfqpoint{5.074251in}{3.148409in}}%
\pgfpathlineto{\pgfqpoint{5.079335in}{3.144590in}}%
\pgfpathlineto{\pgfqpoint{5.085758in}{3.136753in}}%
\pgfpathlineto{\pgfqpoint{5.110913in}{3.102727in}}%
\pgfpathlineto{\pgfqpoint{5.115997in}{3.100832in}}%
\pgfpathlineto{\pgfqpoint{5.120547in}{3.101442in}}%
\pgfpathlineto{\pgfqpoint{5.125364in}{3.104404in}}%
\pgfpathlineto{\pgfqpoint{5.131251in}{3.110809in}}%
\pgfpathlineto{\pgfqpoint{5.140617in}{3.124917in}}%
\pgfpathlineto{\pgfqpoint{5.152927in}{3.142748in}}%
\pgfpathlineto{\pgfqpoint{5.159082in}{3.148043in}}%
\pgfpathlineto{\pgfqpoint{5.164167in}{3.149631in}}%
\pgfpathlineto{\pgfqpoint{5.168716in}{3.148741in}}%
\pgfpathlineto{\pgfqpoint{5.173801in}{3.145264in}}%
\pgfpathlineto{\pgfqpoint{5.179956in}{3.138115in}}%
\pgfpathlineto{\pgfqpoint{5.190928in}{3.121128in}}%
\pgfpathlineto{\pgfqpoint{5.201097in}{3.107030in}}%
\pgfpathlineto{\pgfqpoint{5.207252in}{3.102059in}}%
\pgfpathlineto{\pgfqpoint{5.212069in}{3.100778in}}%
\pgfpathlineto{\pgfqpoint{5.216618in}{3.101819in}}%
\pgfpathlineto{\pgfqpoint{5.221702in}{3.105450in}}%
\pgfpathlineto{\pgfqpoint{5.228125in}{3.113111in}}%
\pgfpathlineto{\pgfqpoint{5.240703in}{3.132716in}}%
\pgfpathlineto{\pgfqpoint{5.249534in}{3.144276in}}%
\pgfpathlineto{\pgfqpoint{5.255421in}{3.148643in}}%
\pgfpathlineto{\pgfqpoint{5.260238in}{3.149629in}}%
\pgfpathlineto{\pgfqpoint{5.264787in}{3.148311in}}%
\pgfpathlineto{\pgfqpoint{5.269872in}{3.144402in}}%
\pgfpathlineto{\pgfqpoint{5.276562in}{3.136098in}}%
\pgfpathlineto{\pgfqpoint{5.300379in}{3.103321in}}%
\pgfpathlineto{\pgfqpoint{5.305464in}{3.100967in}}%
\pgfpathlineto{\pgfqpoint{5.310013in}{3.101154in}}%
\pgfpathlineto{\pgfqpoint{5.314830in}{3.103695in}}%
\pgfpathlineto{\pgfqpoint{5.320450in}{3.109356in}}%
\pgfpathlineto{\pgfqpoint{5.328746in}{3.121355in}}%
\pgfpathlineto{\pgfqpoint{5.343732in}{3.143253in}}%
\pgfpathlineto{\pgfqpoint{5.349887in}{3.148295in}}%
\pgfpathlineto{\pgfqpoint{5.354704in}{3.149638in}}%
\pgfpathlineto{\pgfqpoint{5.359253in}{3.148657in}}%
\pgfpathlineto{\pgfqpoint{5.364337in}{3.145087in}}%
\pgfpathlineto{\pgfqpoint{5.370760in}{3.137483in}}%
\pgfpathlineto{\pgfqpoint{5.382802in}{3.118723in}}%
\pgfpathlineto{\pgfqpoint{5.391901in}{3.106545in}}%
\pgfpathlineto{\pgfqpoint{5.397788in}{3.101963in}}%
\pgfpathlineto{\pgfqpoint{5.402605in}{3.100778in}}%
\pgfpathlineto{\pgfqpoint{5.407155in}{3.101910in}}%
\pgfpathlineto{\pgfqpoint{5.412239in}{3.105632in}}%
\pgfpathlineto{\pgfqpoint{5.418662in}{3.113378in}}%
\pgfpathlineto{\pgfqpoint{5.432845in}{3.135413in}}%
\pgfpathlineto{\pgfqpoint{5.440873in}{3.145236in}}%
\pgfpathlineto{\pgfqpoint{5.446493in}{3.148947in}}%
\pgfpathlineto{\pgfqpoint{5.451310in}{3.149565in}}%
\pgfpathlineto{\pgfqpoint{5.455859in}{3.147904in}}%
\pgfpathlineto{\pgfqpoint{5.461211in}{3.143370in}}%
\pgfpathlineto{\pgfqpoint{5.468169in}{3.134242in}}%
\pgfpathlineto{\pgfqpoint{5.489310in}{3.104430in}}%
\pgfpathlineto{\pgfqpoint{5.494662in}{3.101267in}}%
\pgfpathlineto{\pgfqpoint{5.499479in}{3.100946in}}%
\pgfpathlineto{\pgfqpoint{5.504296in}{3.103059in}}%
\pgfpathlineto{\pgfqpoint{5.509916in}{3.108303in}}%
\pgfpathlineto{\pgfqpoint{5.517677in}{3.119120in}}%
\pgfpathlineto{\pgfqpoint{5.535071in}{3.144294in}}%
\pgfpathlineto{\pgfqpoint{5.540959in}{3.148651in}}%
\pgfpathlineto{\pgfqpoint{5.545775in}{3.149628in}}%
\pgfpathlineto{\pgfqpoint{5.550325in}{3.148302in}}%
\pgfpathlineto{\pgfqpoint{5.555409in}{3.144384in}}%
\pgfpathlineto{\pgfqpoint{5.562100in}{3.136073in}}%
\pgfpathlineto{\pgfqpoint{5.585917in}{3.103309in}}%
\pgfpathlineto{\pgfqpoint{5.591001in}{3.100964in}}%
\pgfpathlineto{\pgfqpoint{5.595551in}{3.101159in}}%
\pgfpathlineto{\pgfqpoint{5.600367in}{3.103708in}}%
\pgfpathlineto{\pgfqpoint{5.605987in}{3.109377in}}%
\pgfpathlineto{\pgfqpoint{5.614283in}{3.121382in}}%
\pgfpathlineto{\pgfqpoint{5.629269in}{3.143272in}}%
\pgfpathlineto{\pgfqpoint{5.635424in}{3.148304in}}%
\pgfpathlineto{\pgfqpoint{5.640241in}{3.149639in}}%
\pgfpathlineto{\pgfqpoint{5.644790in}{3.148649in}}%
\pgfpathlineto{\pgfqpoint{5.649875in}{3.145071in}}%
\pgfpathlineto{\pgfqpoint{5.656298in}{3.137458in}}%
\pgfpathlineto{\pgfqpoint{5.668607in}{3.118281in}}%
\pgfpathlineto{\pgfqpoint{5.677706in}{3.106252in}}%
\pgfpathlineto{\pgfqpoint{5.683594in}{3.101826in}}%
\pgfpathlineto{\pgfqpoint{5.688410in}{3.100785in}}%
\pgfpathlineto{\pgfqpoint{5.692960in}{3.102052in}}%
\pgfpathlineto{\pgfqpoint{5.698044in}{3.105910in}}%
\pgfpathlineto{\pgfqpoint{5.704735in}{3.114166in}}%
\pgfpathlineto{\pgfqpoint{5.728819in}{3.147211in}}%
\pgfpathlineto{\pgfqpoint{5.733904in}{3.149482in}}%
\pgfpathlineto{\pgfqpoint{5.738453in}{3.149219in}}%
\pgfpathlineto{\pgfqpoint{5.743270in}{3.146601in}}%
\pgfpathlineto{\pgfqpoint{5.749157in}{3.140532in}}%
\pgfpathlineto{\pgfqpoint{5.757721in}{3.127954in}}%
\pgfpathlineto{\pgfqpoint{5.771904in}{3.107283in}}%
\pgfpathlineto{\pgfqpoint{5.778059in}{3.102182in}}%
\pgfpathlineto{\pgfqpoint{5.782876in}{3.100784in}}%
\pgfpathlineto{\pgfqpoint{5.787425in}{3.101714in}}%
\pgfpathlineto{\pgfqpoint{5.792510in}{3.105231in}}%
\pgfpathlineto{\pgfqpoint{5.798932in}{3.112786in}}%
\pgfpathlineto{\pgfqpoint{5.810707in}{3.131110in}}%
\pgfpathlineto{\pgfqpoint{5.820074in}{3.143761in}}%
\pgfpathlineto{\pgfqpoint{5.825961in}{3.148402in}}%
\pgfpathlineto{\pgfqpoint{5.830778in}{3.149641in}}%
\pgfpathlineto{\pgfqpoint{5.835327in}{3.148561in}}%
\pgfpathlineto{\pgfqpoint{5.840412in}{3.144891in}}%
\pgfpathlineto{\pgfqpoint{5.846834in}{3.137192in}}%
\pgfpathlineto{\pgfqpoint{5.859947in}{3.116762in}}%
\pgfpathlineto{\pgfqpoint{5.868511in}{3.105795in}}%
\pgfpathlineto{\pgfqpoint{5.874398in}{3.101622in}}%
\pgfpathlineto{\pgfqpoint{5.879215in}{3.100814in}}%
\pgfpathlineto{\pgfqpoint{5.883764in}{3.102298in}}%
\pgfpathlineto{\pgfqpoint{5.888849in}{3.106371in}}%
\pgfpathlineto{\pgfqpoint{5.895539in}{3.114829in}}%
\pgfpathlineto{\pgfqpoint{5.918553in}{3.146763in}}%
\pgfpathlineto{\pgfqpoint{5.923905in}{3.149412in}}%
\pgfpathlineto{\pgfqpoint{5.928455in}{3.149316in}}%
\pgfpathlineto{\pgfqpoint{5.933272in}{3.146864in}}%
\pgfpathlineto{\pgfqpoint{5.938891in}{3.141289in}}%
\pgfpathlineto{\pgfqpoint{5.946920in}{3.129784in}}%
\pgfpathlineto{\pgfqpoint{5.962708in}{3.106790in}}%
\pgfpathlineto{\pgfqpoint{5.968596in}{3.102081in}}%
\pgfpathlineto{\pgfqpoint{5.973413in}{3.100779in}}%
\pgfpathlineto{\pgfqpoint{5.977962in}{3.101799in}}%
\pgfpathlineto{\pgfqpoint{5.983047in}{3.105409in}}%
\pgfpathlineto{\pgfqpoint{5.989469in}{3.113051in}}%
\pgfpathlineto{\pgfqpoint{6.002047in}{3.132650in}}%
\pgfpathlineto{\pgfqpoint{6.010878in}{3.144233in}}%
\pgfpathlineto{\pgfqpoint{6.016765in}{3.148623in}}%
\pgfpathlineto{\pgfqpoint{6.021582in}{3.149631in}}%
\pgfpathlineto{\pgfqpoint{6.026132in}{3.148333in}}%
\pgfpathlineto{\pgfqpoint{6.031216in}{3.144445in}}%
\pgfpathlineto{\pgfqpoint{6.037906in}{3.136160in}}%
\pgfpathlineto{\pgfqpoint{6.061991in}{3.103163in}}%
\pgfpathlineto{\pgfqpoint{6.067076in}{3.100925in}}%
\pgfpathlineto{\pgfqpoint{6.071625in}{3.101220in}}%
\pgfpathlineto{\pgfqpoint{6.076442in}{3.103869in}}%
\pgfpathlineto{\pgfqpoint{6.082329in}{3.109968in}}%
\pgfpathlineto{\pgfqpoint{6.090893in}{3.122569in}}%
\pgfpathlineto{\pgfqpoint{6.104808in}{3.142912in}}%
\pgfpathlineto{\pgfqpoint{6.110963in}{3.148126in}}%
\pgfpathlineto{\pgfqpoint{6.116048in}{3.149637in}}%
\pgfpathlineto{\pgfqpoint{6.120597in}{3.148676in}}%
\pgfpathlineto{\pgfqpoint{6.125682in}{3.145127in}}%
\pgfpathlineto{\pgfqpoint{6.132104in}{3.137543in}}%
\pgfpathlineto{\pgfqpoint{6.144147in}{3.118790in}}%
\pgfpathlineto{\pgfqpoint{6.153245in}{3.106590in}}%
\pgfpathlineto{\pgfqpoint{6.159133in}{3.101984in}}%
\pgfpathlineto{\pgfqpoint{6.163950in}{3.100778in}}%
\pgfpathlineto{\pgfqpoint{6.168499in}{3.101889in}}%
\pgfpathlineto{\pgfqpoint{6.173583in}{3.105590in}}%
\pgfpathlineto{\pgfqpoint{6.180006in}{3.113318in}}%
\pgfpathlineto{\pgfqpoint{6.193654in}{3.134558in}}%
\pgfpathlineto{\pgfqpoint{6.201950in}{3.144944in}}%
\pgfpathlineto{\pgfqpoint{6.207570in}{3.148823in}}%
\pgfpathlineto{\pgfqpoint{6.212387in}{3.149599in}}%
\pgfpathlineto{\pgfqpoint{6.216936in}{3.148084in}}%
\pgfpathlineto{\pgfqpoint{6.222288in}{3.143702in}}%
\pgfpathlineto{\pgfqpoint{6.229246in}{3.134706in}}%
\pgfpathlineto{\pgfqpoint{6.250922in}{3.104241in}}%
\pgfpathlineto{\pgfqpoint{6.256274in}{3.101198in}}%
\pgfpathlineto{\pgfqpoint{6.260824in}{3.100938in}}%
\pgfpathlineto{\pgfqpoint{6.265373in}{3.102852in}}%
\pgfpathlineto{\pgfqpoint{6.270725in}{3.107642in}}%
\pgfpathlineto{\pgfqpoint{6.277950in}{3.117399in}}%
\pgfpathlineto{\pgfqpoint{6.297218in}{3.145034in}}%
\pgfpathlineto{\pgfqpoint{6.302838in}{3.148862in}}%
\pgfpathlineto{\pgfqpoint{6.307655in}{3.149590in}}%
\pgfpathlineto{\pgfqpoint{6.312204in}{3.148030in}}%
\pgfpathlineto{\pgfqpoint{6.317556in}{3.143601in}}%
\pgfpathlineto{\pgfqpoint{6.324514in}{3.134565in}}%
\pgfpathlineto{\pgfqpoint{6.345923in}{3.104386in}}%
\pgfpathlineto{\pgfqpoint{6.351275in}{3.101251in}}%
\pgfpathlineto{\pgfqpoint{6.356092in}{3.100956in}}%
\pgfpathlineto{\pgfqpoint{6.360909in}{3.103094in}}%
\pgfpathlineto{\pgfqpoint{6.366529in}{3.108363in}}%
\pgfpathlineto{\pgfqpoint{6.374289in}{3.119201in}}%
\pgfpathlineto{\pgfqpoint{6.391416in}{3.144074in}}%
\pgfpathlineto{\pgfqpoint{6.397304in}{3.148551in}}%
\pgfpathlineto{\pgfqpoint{6.402121in}{3.149637in}}%
\pgfpathlineto{\pgfqpoint{6.406670in}{3.148413in}}%
\pgfpathlineto{\pgfqpoint{6.411754in}{3.144598in}}%
\pgfpathlineto{\pgfqpoint{6.418177in}{3.136765in}}%
\pgfpathlineto{\pgfqpoint{6.443332in}{3.102733in}}%
\pgfpathlineto{\pgfqpoint{6.448417in}{3.100833in}}%
\pgfpathlineto{\pgfqpoint{6.452966in}{3.101439in}}%
\pgfpathlineto{\pgfqpoint{6.457783in}{3.104397in}}%
\pgfpathlineto{\pgfqpoint{6.463670in}{3.110798in}}%
\pgfpathlineto{\pgfqpoint{6.473037in}{3.124903in}}%
\pgfpathlineto{\pgfqpoint{6.485347in}{3.142738in}}%
\pgfpathlineto{\pgfqpoint{6.491502in}{3.148038in}}%
\pgfpathlineto{\pgfqpoint{6.496586in}{3.149631in}}%
\pgfpathlineto{\pgfqpoint{6.501135in}{3.148745in}}%
\pgfpathlineto{\pgfqpoint{6.506220in}{3.145271in}}%
\pgfpathlineto{\pgfqpoint{6.512375in}{3.138127in}}%
\pgfpathlineto{\pgfqpoint{6.523347in}{3.121142in}}%
\pgfpathlineto{\pgfqpoint{6.533516in}{3.107039in}}%
\pgfpathlineto{\pgfqpoint{6.539671in}{3.102063in}}%
\pgfpathlineto{\pgfqpoint{6.544488in}{3.100778in}}%
\pgfpathlineto{\pgfqpoint{6.549037in}{3.101815in}}%
\pgfpathlineto{\pgfqpoint{6.554122in}{3.105442in}}%
\pgfpathlineto{\pgfqpoint{6.560544in}{3.113099in}}%
\pgfpathlineto{\pgfqpoint{6.573122in}{3.132703in}}%
\pgfpathlineto{\pgfqpoint{6.581953in}{3.144267in}}%
\pgfpathlineto{\pgfqpoint{6.587840in}{3.148639in}}%
\pgfpathlineto{\pgfqpoint{6.592657in}{3.149630in}}%
\pgfpathlineto{\pgfqpoint{6.597207in}{3.148315in}}%
\pgfpathlineto{\pgfqpoint{6.602291in}{3.144410in}}%
\pgfpathlineto{\pgfqpoint{6.608981in}{3.136110in}}%
\pgfpathlineto{\pgfqpoint{6.633066in}{3.103139in}}%
\pgfpathlineto{\pgfqpoint{6.638151in}{3.100919in}}%
\pgfpathlineto{\pgfqpoint{6.642700in}{3.101230in}}%
\pgfpathlineto{\pgfqpoint{6.647517in}{3.103896in}}%
\pgfpathlineto{\pgfqpoint{6.653404in}{3.110012in}}%
\pgfpathlineto{\pgfqpoint{6.661968in}{3.122624in}}%
\pgfpathlineto{\pgfqpoint{6.663306in}{3.124778in}}%
\pgfpathlineto{\pgfqpoint{6.663306in}{3.124778in}}%
\pgfusepath{stroke}%
\end{pgfscope}%
\begin{pgfscope}%
\pgfpathrectangle{\pgfqpoint{0.467797in}{2.292089in}}{\pgfqpoint{6.490533in}{1.666241in}}%
\pgfusepath{clip}%
\pgfsetrectcap%
\pgfsetroundjoin%
\pgfsetlinewidth{1.505625pt}%
\definecolor{currentstroke}{rgb}{0.890196,0.466667,0.760784}%
\pgfsetstrokecolor{currentstroke}%
\pgfsetdash{}{0pt}%
\pgfpathmoveto{\pgfqpoint{0.762821in}{3.125209in}}%
\pgfpathlineto{\pgfqpoint{0.774596in}{3.142107in}}%
\pgfpathlineto{\pgfqpoint{0.780483in}{3.146925in}}%
\pgfpathlineto{\pgfqpoint{0.785300in}{3.148159in}}%
\pgfpathlineto{\pgfqpoint{0.789849in}{3.146932in}}%
\pgfpathlineto{\pgfqpoint{0.794934in}{3.142968in}}%
\pgfpathlineto{\pgfqpoint{0.801356in}{3.134848in}}%
\pgfpathlineto{\pgfqpoint{0.823033in}{3.104861in}}%
\pgfpathlineto{\pgfqpoint{0.828117in}{3.102433in}}%
\pgfpathlineto{\pgfqpoint{0.832667in}{3.102699in}}%
\pgfpathlineto{\pgfqpoint{0.837216in}{3.105246in}}%
\pgfpathlineto{\pgfqpoint{0.842836in}{3.111139in}}%
\pgfpathlineto{\pgfqpoint{0.851399in}{3.123876in}}%
\pgfpathlineto{\pgfqpoint{0.863977in}{3.142080in}}%
\pgfpathlineto{\pgfqpoint{0.869864in}{3.146913in}}%
\pgfpathlineto{\pgfqpoint{0.874681in}{3.148160in}}%
\pgfpathlineto{\pgfqpoint{0.879230in}{3.146944in}}%
\pgfpathlineto{\pgfqpoint{0.884315in}{3.142993in}}%
\pgfpathlineto{\pgfqpoint{0.890737in}{3.134884in}}%
\pgfpathlineto{\pgfqpoint{0.912414in}{3.104879in}}%
\pgfpathlineto{\pgfqpoint{0.917498in}{3.102438in}}%
\pgfpathlineto{\pgfqpoint{0.922048in}{3.102691in}}%
\pgfpathlineto{\pgfqpoint{0.926597in}{3.105226in}}%
\pgfpathlineto{\pgfqpoint{0.932217in}{3.111108in}}%
\pgfpathlineto{\pgfqpoint{0.940780in}{3.123837in}}%
\pgfpathlineto{\pgfqpoint{0.953358in}{3.142054in}}%
\pgfpathlineto{\pgfqpoint{0.959245in}{3.146900in}}%
\pgfpathlineto{\pgfqpoint{0.964062in}{3.148160in}}%
\pgfpathlineto{\pgfqpoint{0.968611in}{3.146957in}}%
\pgfpathlineto{\pgfqpoint{0.973696in}{3.143018in}}%
\pgfpathlineto{\pgfqpoint{0.980118in}{3.134919in}}%
\pgfpathlineto{\pgfqpoint{1.001795in}{3.104898in}}%
\pgfpathlineto{\pgfqpoint{1.006879in}{3.102443in}}%
\pgfpathlineto{\pgfqpoint{1.011429in}{3.102684in}}%
\pgfpathlineto{\pgfqpoint{1.015978in}{3.105207in}}%
\pgfpathlineto{\pgfqpoint{1.021598in}{3.111077in}}%
\pgfpathlineto{\pgfqpoint{1.030161in}{3.123798in}}%
\pgfpathlineto{\pgfqpoint{1.042739in}{3.142027in}}%
\pgfpathlineto{\pgfqpoint{1.048626in}{3.146887in}}%
\pgfpathlineto{\pgfqpoint{1.053443in}{3.148160in}}%
\pgfpathlineto{\pgfqpoint{1.057992in}{3.146969in}}%
\pgfpathlineto{\pgfqpoint{1.063077in}{3.143043in}}%
\pgfpathlineto{\pgfqpoint{1.069499in}{3.134955in}}%
\pgfpathlineto{\pgfqpoint{1.091443in}{3.104718in}}%
\pgfpathlineto{\pgfqpoint{1.096528in}{3.102397in}}%
\pgfpathlineto{\pgfqpoint{1.101077in}{3.102762in}}%
\pgfpathlineto{\pgfqpoint{1.105894in}{3.105624in}}%
\pgfpathlineto{\pgfqpoint{1.111782in}{3.112087in}}%
\pgfpathlineto{\pgfqpoint{1.121148in}{3.126347in}}%
\pgfpathlineto{\pgfqpoint{1.132120in}{3.142000in}}%
\pgfpathlineto{\pgfqpoint{1.138007in}{3.146874in}}%
\pgfpathlineto{\pgfqpoint{1.142824in}{3.148160in}}%
\pgfpathlineto{\pgfqpoint{1.147373in}{3.146982in}}%
\pgfpathlineto{\pgfqpoint{1.152458in}{3.143067in}}%
\pgfpathlineto{\pgfqpoint{1.158880in}{3.134990in}}%
\pgfpathlineto{\pgfqpoint{1.180824in}{3.104736in}}%
\pgfpathlineto{\pgfqpoint{1.185909in}{3.102401in}}%
\pgfpathlineto{\pgfqpoint{1.190458in}{3.102754in}}%
\pgfpathlineto{\pgfqpoint{1.195275in}{3.105603in}}%
\pgfpathlineto{\pgfqpoint{1.201163in}{3.112055in}}%
\pgfpathlineto{\pgfqpoint{1.210529in}{3.126308in}}%
\pgfpathlineto{\pgfqpoint{1.221501in}{3.141974in}}%
\pgfpathlineto{\pgfqpoint{1.227388in}{3.146861in}}%
\pgfpathlineto{\pgfqpoint{1.232205in}{3.148160in}}%
\pgfpathlineto{\pgfqpoint{1.236754in}{3.146994in}}%
\pgfpathlineto{\pgfqpoint{1.241839in}{3.143092in}}%
\pgfpathlineto{\pgfqpoint{1.248261in}{3.135026in}}%
\pgfpathlineto{\pgfqpoint{1.270205in}{3.104753in}}%
\pgfpathlineto{\pgfqpoint{1.275290in}{3.102406in}}%
\pgfpathlineto{\pgfqpoint{1.279839in}{3.102746in}}%
\pgfpathlineto{\pgfqpoint{1.284656in}{3.105583in}}%
\pgfpathlineto{\pgfqpoint{1.290544in}{3.112023in}}%
\pgfpathlineto{\pgfqpoint{1.299910in}{3.126269in}}%
\pgfpathlineto{\pgfqpoint{1.310882in}{3.141947in}}%
\pgfpathlineto{\pgfqpoint{1.316769in}{3.146848in}}%
\pgfpathlineto{\pgfqpoint{1.321586in}{3.148160in}}%
\pgfpathlineto{\pgfqpoint{1.326135in}{3.147007in}}%
\pgfpathlineto{\pgfqpoint{1.331220in}{3.143116in}}%
\pgfpathlineto{\pgfqpoint{1.337643in}{3.135061in}}%
\pgfpathlineto{\pgfqpoint{1.359586in}{3.104771in}}%
\pgfpathlineto{\pgfqpoint{1.364671in}{3.102410in}}%
\pgfpathlineto{\pgfqpoint{1.369220in}{3.102738in}}%
\pgfpathlineto{\pgfqpoint{1.374037in}{3.105563in}}%
\pgfpathlineto{\pgfqpoint{1.379925in}{3.111990in}}%
\pgfpathlineto{\pgfqpoint{1.389291in}{3.126229in}}%
\pgfpathlineto{\pgfqpoint{1.400263in}{3.141920in}}%
\pgfpathlineto{\pgfqpoint{1.406150in}{3.146835in}}%
\pgfpathlineto{\pgfqpoint{1.410967in}{3.148160in}}%
\pgfpathlineto{\pgfqpoint{1.415516in}{3.147019in}}%
\pgfpathlineto{\pgfqpoint{1.420601in}{3.143141in}}%
\pgfpathlineto{\pgfqpoint{1.427024in}{3.135097in}}%
\pgfpathlineto{\pgfqpoint{1.448967in}{3.104789in}}%
\pgfpathlineto{\pgfqpoint{1.454052in}{3.102415in}}%
\pgfpathlineto{\pgfqpoint{1.458601in}{3.102730in}}%
\pgfpathlineto{\pgfqpoint{1.463151in}{3.105323in}}%
\pgfpathlineto{\pgfqpoint{1.468770in}{3.111263in}}%
\pgfpathlineto{\pgfqpoint{1.477601in}{3.124464in}}%
\pgfpathlineto{\pgfqpoint{1.489644in}{3.141893in}}%
\pgfpathlineto{\pgfqpoint{1.495531in}{3.146822in}}%
\pgfpathlineto{\pgfqpoint{1.500348in}{3.148160in}}%
\pgfpathlineto{\pgfqpoint{1.504897in}{3.147031in}}%
\pgfpathlineto{\pgfqpoint{1.509714in}{3.143431in}}%
\pgfpathlineto{\pgfqpoint{1.516137in}{3.135520in}}%
\pgfpathlineto{\pgfqpoint{1.538884in}{3.104426in}}%
\pgfpathlineto{\pgfqpoint{1.543968in}{3.102333in}}%
\pgfpathlineto{\pgfqpoint{1.548517in}{3.102910in}}%
\pgfpathlineto{\pgfqpoint{1.553334in}{3.105980in}}%
\pgfpathlineto{\pgfqpoint{1.559222in}{3.112640in}}%
\pgfpathlineto{\pgfqpoint{1.569391in}{3.128300in}}%
\pgfpathlineto{\pgfqpoint{1.579560in}{3.142448in}}%
\pgfpathlineto{\pgfqpoint{1.585447in}{3.147085in}}%
\pgfpathlineto{\pgfqpoint{1.590264in}{3.148149in}}%
\pgfpathlineto{\pgfqpoint{1.594814in}{3.146762in}}%
\pgfpathlineto{\pgfqpoint{1.599898in}{3.142641in}}%
\pgfpathlineto{\pgfqpoint{1.606588in}{3.133986in}}%
\pgfpathlineto{\pgfqpoint{1.626927in}{3.105452in}}%
\pgfpathlineto{\pgfqpoint{1.632279in}{3.102543in}}%
\pgfpathlineto{\pgfqpoint{1.636828in}{3.102559in}}%
\pgfpathlineto{\pgfqpoint{1.641377in}{3.104870in}}%
\pgfpathlineto{\pgfqpoint{1.646997in}{3.110527in}}%
\pgfpathlineto{\pgfqpoint{1.655293in}{3.122664in}}%
\pgfpathlineto{\pgfqpoint{1.668673in}{3.142133in}}%
\pgfpathlineto{\pgfqpoint{1.674561in}{3.146938in}}%
\pgfpathlineto{\pgfqpoint{1.679378in}{3.148159in}}%
\pgfpathlineto{\pgfqpoint{1.683927in}{3.146919in}}%
\pgfpathlineto{\pgfqpoint{1.689012in}{3.142943in}}%
\pgfpathlineto{\pgfqpoint{1.695702in}{3.134419in}}%
\pgfpathlineto{\pgfqpoint{1.716843in}{3.105046in}}%
\pgfpathlineto{\pgfqpoint{1.721927in}{3.102485in}}%
\pgfpathlineto{\pgfqpoint{1.726477in}{3.102626in}}%
\pgfpathlineto{\pgfqpoint{1.731026in}{3.105055in}}%
\pgfpathlineto{\pgfqpoint{1.736646in}{3.110831in}}%
\pgfpathlineto{\pgfqpoint{1.744942in}{3.123054in}}%
\pgfpathlineto{\pgfqpoint{1.758054in}{3.142107in}}%
\pgfpathlineto{\pgfqpoint{1.763942in}{3.146925in}}%
\pgfpathlineto{\pgfqpoint{1.768759in}{3.148159in}}%
\pgfpathlineto{\pgfqpoint{1.773308in}{3.146932in}}%
\pgfpathlineto{\pgfqpoint{1.778393in}{3.142968in}}%
\pgfpathlineto{\pgfqpoint{1.784815in}{3.134848in}}%
\pgfpathlineto{\pgfqpoint{1.806491in}{3.104861in}}%
\pgfpathlineto{\pgfqpoint{1.811576in}{3.102433in}}%
\pgfpathlineto{\pgfqpoint{1.816125in}{3.102699in}}%
\pgfpathlineto{\pgfqpoint{1.820675in}{3.105246in}}%
\pgfpathlineto{\pgfqpoint{1.826294in}{3.111139in}}%
\pgfpathlineto{\pgfqpoint{1.834858in}{3.123876in}}%
\pgfpathlineto{\pgfqpoint{1.847435in}{3.142080in}}%
\pgfpathlineto{\pgfqpoint{1.853323in}{3.146913in}}%
\pgfpathlineto{\pgfqpoint{1.858140in}{3.148160in}}%
\pgfpathlineto{\pgfqpoint{1.862689in}{3.146944in}}%
\pgfpathlineto{\pgfqpoint{1.867774in}{3.142993in}}%
\pgfpathlineto{\pgfqpoint{1.874196in}{3.134884in}}%
\pgfpathlineto{\pgfqpoint{1.895872in}{3.104879in}}%
\pgfpathlineto{\pgfqpoint{1.900957in}{3.102438in}}%
\pgfpathlineto{\pgfqpoint{1.905506in}{3.102691in}}%
\pgfpathlineto{\pgfqpoint{1.910056in}{3.105226in}}%
\pgfpathlineto{\pgfqpoint{1.915675in}{3.111108in}}%
\pgfpathlineto{\pgfqpoint{1.924239in}{3.123837in}}%
\pgfpathlineto{\pgfqpoint{1.936816in}{3.142054in}}%
\pgfpathlineto{\pgfqpoint{1.942704in}{3.146900in}}%
\pgfpathlineto{\pgfqpoint{1.947521in}{3.148160in}}%
\pgfpathlineto{\pgfqpoint{1.952070in}{3.146957in}}%
\pgfpathlineto{\pgfqpoint{1.957155in}{3.143018in}}%
\pgfpathlineto{\pgfqpoint{1.963577in}{3.134919in}}%
\pgfpathlineto{\pgfqpoint{1.985253in}{3.104898in}}%
\pgfpathlineto{\pgfqpoint{1.990338in}{3.102443in}}%
\pgfpathlineto{\pgfqpoint{1.994887in}{3.102684in}}%
\pgfpathlineto{\pgfqpoint{1.999437in}{3.105207in}}%
\pgfpathlineto{\pgfqpoint{2.005056in}{3.111077in}}%
\pgfpathlineto{\pgfqpoint{2.013620in}{3.123798in}}%
\pgfpathlineto{\pgfqpoint{2.026197in}{3.142027in}}%
\pgfpathlineto{\pgfqpoint{2.032085in}{3.146887in}}%
\pgfpathlineto{\pgfqpoint{2.036902in}{3.148160in}}%
\pgfpathlineto{\pgfqpoint{2.041451in}{3.146969in}}%
\pgfpathlineto{\pgfqpoint{2.046536in}{3.143043in}}%
\pgfpathlineto{\pgfqpoint{2.052958in}{3.134955in}}%
\pgfpathlineto{\pgfqpoint{2.074902in}{3.104718in}}%
\pgfpathlineto{\pgfqpoint{2.079987in}{3.102397in}}%
\pgfpathlineto{\pgfqpoint{2.084536in}{3.102762in}}%
\pgfpathlineto{\pgfqpoint{2.089353in}{3.105624in}}%
\pgfpathlineto{\pgfqpoint{2.095240in}{3.112087in}}%
\pgfpathlineto{\pgfqpoint{2.104607in}{3.126347in}}%
\pgfpathlineto{\pgfqpoint{2.115578in}{3.142000in}}%
\pgfpathlineto{\pgfqpoint{2.121466in}{3.146874in}}%
\pgfpathlineto{\pgfqpoint{2.126283in}{3.148160in}}%
\pgfpathlineto{\pgfqpoint{2.130832in}{3.146982in}}%
\pgfpathlineto{\pgfqpoint{2.135917in}{3.143067in}}%
\pgfpathlineto{\pgfqpoint{2.142339in}{3.134990in}}%
\pgfpathlineto{\pgfqpoint{2.164283in}{3.104736in}}%
\pgfpathlineto{\pgfqpoint{2.169368in}{3.102401in}}%
\pgfpathlineto{\pgfqpoint{2.173917in}{3.102754in}}%
\pgfpathlineto{\pgfqpoint{2.178734in}{3.105603in}}%
\pgfpathlineto{\pgfqpoint{2.184621in}{3.112055in}}%
\pgfpathlineto{\pgfqpoint{2.193988in}{3.126308in}}%
\pgfpathlineto{\pgfqpoint{2.204959in}{3.141974in}}%
\pgfpathlineto{\pgfqpoint{2.210847in}{3.146861in}}%
\pgfpathlineto{\pgfqpoint{2.215664in}{3.148160in}}%
\pgfpathlineto{\pgfqpoint{2.220213in}{3.146994in}}%
\pgfpathlineto{\pgfqpoint{2.225298in}{3.143092in}}%
\pgfpathlineto{\pgfqpoint{2.231720in}{3.135026in}}%
\pgfpathlineto{\pgfqpoint{2.253664in}{3.104753in}}%
\pgfpathlineto{\pgfqpoint{2.258749in}{3.102406in}}%
\pgfpathlineto{\pgfqpoint{2.263298in}{3.102746in}}%
\pgfpathlineto{\pgfqpoint{2.268115in}{3.105583in}}%
\pgfpathlineto{\pgfqpoint{2.274002in}{3.112023in}}%
\pgfpathlineto{\pgfqpoint{2.283369in}{3.126269in}}%
\pgfpathlineto{\pgfqpoint{2.294340in}{3.141947in}}%
\pgfpathlineto{\pgfqpoint{2.300228in}{3.146848in}}%
\pgfpathlineto{\pgfqpoint{2.305045in}{3.148160in}}%
\pgfpathlineto{\pgfqpoint{2.309594in}{3.147007in}}%
\pgfpathlineto{\pgfqpoint{2.314679in}{3.143116in}}%
\pgfpathlineto{\pgfqpoint{2.321101in}{3.135061in}}%
\pgfpathlineto{\pgfqpoint{2.343045in}{3.104771in}}%
\pgfpathlineto{\pgfqpoint{2.348130in}{3.102410in}}%
\pgfpathlineto{\pgfqpoint{2.352679in}{3.102738in}}%
\pgfpathlineto{\pgfqpoint{2.357496in}{3.105563in}}%
\pgfpathlineto{\pgfqpoint{2.363383in}{3.111990in}}%
\pgfpathlineto{\pgfqpoint{2.372750in}{3.126229in}}%
\pgfpathlineto{\pgfqpoint{2.383721in}{3.141920in}}%
\pgfpathlineto{\pgfqpoint{2.389609in}{3.146835in}}%
\pgfpathlineto{\pgfqpoint{2.394426in}{3.148160in}}%
\pgfpathlineto{\pgfqpoint{2.398975in}{3.147019in}}%
\pgfpathlineto{\pgfqpoint{2.404060in}{3.143141in}}%
\pgfpathlineto{\pgfqpoint{2.410482in}{3.135097in}}%
\pgfpathlineto{\pgfqpoint{2.432426in}{3.104789in}}%
\pgfpathlineto{\pgfqpoint{2.437511in}{3.102415in}}%
\pgfpathlineto{\pgfqpoint{2.442060in}{3.102730in}}%
\pgfpathlineto{\pgfqpoint{2.446609in}{3.105323in}}%
\pgfpathlineto{\pgfqpoint{2.452229in}{3.111263in}}%
\pgfpathlineto{\pgfqpoint{2.461060in}{3.124464in}}%
\pgfpathlineto{\pgfqpoint{2.473102in}{3.141893in}}%
\pgfpathlineto{\pgfqpoint{2.478990in}{3.146822in}}%
\pgfpathlineto{\pgfqpoint{2.483807in}{3.148160in}}%
\pgfpathlineto{\pgfqpoint{2.488356in}{3.147031in}}%
\pgfpathlineto{\pgfqpoint{2.493173in}{3.143431in}}%
\pgfpathlineto{\pgfqpoint{2.499596in}{3.135520in}}%
\pgfpathlineto{\pgfqpoint{2.522342in}{3.104426in}}%
\pgfpathlineto{\pgfqpoint{2.527427in}{3.102333in}}%
\pgfpathlineto{\pgfqpoint{2.531976in}{3.102910in}}%
\pgfpathlineto{\pgfqpoint{2.536793in}{3.105980in}}%
\pgfpathlineto{\pgfqpoint{2.542681in}{3.112640in}}%
\pgfpathlineto{\pgfqpoint{2.552850in}{3.128300in}}%
\pgfpathlineto{\pgfqpoint{2.563019in}{3.142448in}}%
\pgfpathlineto{\pgfqpoint{2.568906in}{3.147085in}}%
\pgfpathlineto{\pgfqpoint{2.573723in}{3.148149in}}%
\pgfpathlineto{\pgfqpoint{2.578272in}{3.146762in}}%
\pgfpathlineto{\pgfqpoint{2.583357in}{3.142641in}}%
\pgfpathlineto{\pgfqpoint{2.590047in}{3.133986in}}%
\pgfpathlineto{\pgfqpoint{2.610385in}{3.105452in}}%
\pgfpathlineto{\pgfqpoint{2.615737in}{3.102543in}}%
\pgfpathlineto{\pgfqpoint{2.620287in}{3.102559in}}%
\pgfpathlineto{\pgfqpoint{2.624836in}{3.104870in}}%
\pgfpathlineto{\pgfqpoint{2.630456in}{3.110527in}}%
\pgfpathlineto{\pgfqpoint{2.638752in}{3.122664in}}%
\pgfpathlineto{\pgfqpoint{2.652132in}{3.142133in}}%
\pgfpathlineto{\pgfqpoint{2.658019in}{3.146938in}}%
\pgfpathlineto{\pgfqpoint{2.662836in}{3.148159in}}%
\pgfpathlineto{\pgfqpoint{2.667386in}{3.146919in}}%
\pgfpathlineto{\pgfqpoint{2.672470in}{3.142943in}}%
\pgfpathlineto{\pgfqpoint{2.679161in}{3.134419in}}%
\pgfpathlineto{\pgfqpoint{2.700302in}{3.105046in}}%
\pgfpathlineto{\pgfqpoint{2.705386in}{3.102485in}}%
\pgfpathlineto{\pgfqpoint{2.709935in}{3.102626in}}%
\pgfpathlineto{\pgfqpoint{2.714485in}{3.105055in}}%
\pgfpathlineto{\pgfqpoint{2.720105in}{3.110831in}}%
\pgfpathlineto{\pgfqpoint{2.728400in}{3.123054in}}%
\pgfpathlineto{\pgfqpoint{2.741513in}{3.142107in}}%
\pgfpathlineto{\pgfqpoint{2.747401in}{3.146925in}}%
\pgfpathlineto{\pgfqpoint{2.752217in}{3.148159in}}%
\pgfpathlineto{\pgfqpoint{2.756767in}{3.146932in}}%
\pgfpathlineto{\pgfqpoint{2.761851in}{3.142968in}}%
\pgfpathlineto{\pgfqpoint{2.768274in}{3.134848in}}%
\pgfpathlineto{\pgfqpoint{2.789950in}{3.104861in}}%
\pgfpathlineto{\pgfqpoint{2.795035in}{3.102433in}}%
\pgfpathlineto{\pgfqpoint{2.799584in}{3.102699in}}%
\pgfpathlineto{\pgfqpoint{2.804133in}{3.105246in}}%
\pgfpathlineto{\pgfqpoint{2.809753in}{3.111139in}}%
\pgfpathlineto{\pgfqpoint{2.818317in}{3.123876in}}%
\pgfpathlineto{\pgfqpoint{2.830894in}{3.142080in}}%
\pgfpathlineto{\pgfqpoint{2.836782in}{3.146913in}}%
\pgfpathlineto{\pgfqpoint{2.841598in}{3.148160in}}%
\pgfpathlineto{\pgfqpoint{2.846148in}{3.146944in}}%
\pgfpathlineto{\pgfqpoint{2.851232in}{3.142993in}}%
\pgfpathlineto{\pgfqpoint{2.857655in}{3.134884in}}%
\pgfpathlineto{\pgfqpoint{2.879331in}{3.104879in}}%
\pgfpathlineto{\pgfqpoint{2.884416in}{3.102438in}}%
\pgfpathlineto{\pgfqpoint{2.888965in}{3.102691in}}%
\pgfpathlineto{\pgfqpoint{2.893514in}{3.105226in}}%
\pgfpathlineto{\pgfqpoint{2.899134in}{3.111108in}}%
\pgfpathlineto{\pgfqpoint{2.907698in}{3.123837in}}%
\pgfpathlineto{\pgfqpoint{2.920275in}{3.142054in}}%
\pgfpathlineto{\pgfqpoint{2.926163in}{3.146900in}}%
\pgfpathlineto{\pgfqpoint{2.930979in}{3.148160in}}%
\pgfpathlineto{\pgfqpoint{2.935529in}{3.146957in}}%
\pgfpathlineto{\pgfqpoint{2.940613in}{3.143018in}}%
\pgfpathlineto{\pgfqpoint{2.947036in}{3.134919in}}%
\pgfpathlineto{\pgfqpoint{2.968712in}{3.104898in}}%
\pgfpathlineto{\pgfqpoint{2.973797in}{3.102443in}}%
\pgfpathlineto{\pgfqpoint{2.978346in}{3.102684in}}%
\pgfpathlineto{\pgfqpoint{2.982895in}{3.105207in}}%
\pgfpathlineto{\pgfqpoint{2.988515in}{3.111077in}}%
\pgfpathlineto{\pgfqpoint{2.997079in}{3.123798in}}%
\pgfpathlineto{\pgfqpoint{3.009656in}{3.142027in}}%
\pgfpathlineto{\pgfqpoint{3.015544in}{3.146887in}}%
\pgfpathlineto{\pgfqpoint{3.020360in}{3.148160in}}%
\pgfpathlineto{\pgfqpoint{3.024910in}{3.146969in}}%
\pgfpathlineto{\pgfqpoint{3.029994in}{3.143043in}}%
\pgfpathlineto{\pgfqpoint{3.036417in}{3.134955in}}%
\pgfpathlineto{\pgfqpoint{3.058361in}{3.104718in}}%
\pgfpathlineto{\pgfqpoint{3.063445in}{3.102397in}}%
\pgfpathlineto{\pgfqpoint{3.067995in}{3.102762in}}%
\pgfpathlineto{\pgfqpoint{3.072812in}{3.105624in}}%
\pgfpathlineto{\pgfqpoint{3.078699in}{3.112087in}}%
\pgfpathlineto{\pgfqpoint{3.088065in}{3.126347in}}%
\pgfpathlineto{\pgfqpoint{3.099037in}{3.142000in}}%
\pgfpathlineto{\pgfqpoint{3.104925in}{3.146874in}}%
\pgfpathlineto{\pgfqpoint{3.109742in}{3.148160in}}%
\pgfpathlineto{\pgfqpoint{3.114291in}{3.146982in}}%
\pgfpathlineto{\pgfqpoint{3.119375in}{3.143067in}}%
\pgfpathlineto{\pgfqpoint{3.125798in}{3.134990in}}%
\pgfpathlineto{\pgfqpoint{3.147742in}{3.104736in}}%
\pgfpathlineto{\pgfqpoint{3.152826in}{3.102401in}}%
\pgfpathlineto{\pgfqpoint{3.157376in}{3.102754in}}%
\pgfpathlineto{\pgfqpoint{3.162193in}{3.105603in}}%
\pgfpathlineto{\pgfqpoint{3.168080in}{3.112055in}}%
\pgfpathlineto{\pgfqpoint{3.177446in}{3.126308in}}%
\pgfpathlineto{\pgfqpoint{3.188418in}{3.141974in}}%
\pgfpathlineto{\pgfqpoint{3.194306in}{3.146861in}}%
\pgfpathlineto{\pgfqpoint{3.199123in}{3.148160in}}%
\pgfpathlineto{\pgfqpoint{3.203672in}{3.146994in}}%
\pgfpathlineto{\pgfqpoint{3.208756in}{3.143092in}}%
\pgfpathlineto{\pgfqpoint{3.215179in}{3.135026in}}%
\pgfpathlineto{\pgfqpoint{3.237123in}{3.104753in}}%
\pgfpathlineto{\pgfqpoint{3.242207in}{3.102406in}}%
\pgfpathlineto{\pgfqpoint{3.246757in}{3.102746in}}%
\pgfpathlineto{\pgfqpoint{3.251574in}{3.105583in}}%
\pgfpathlineto{\pgfqpoint{3.257461in}{3.112023in}}%
\pgfpathlineto{\pgfqpoint{3.266827in}{3.126269in}}%
\pgfpathlineto{\pgfqpoint{3.277799in}{3.141947in}}%
\pgfpathlineto{\pgfqpoint{3.283687in}{3.146848in}}%
\pgfpathlineto{\pgfqpoint{3.288504in}{3.148160in}}%
\pgfpathlineto{\pgfqpoint{3.293053in}{3.147007in}}%
\pgfpathlineto{\pgfqpoint{3.298137in}{3.143116in}}%
\pgfpathlineto{\pgfqpoint{3.304560in}{3.135061in}}%
\pgfpathlineto{\pgfqpoint{3.326504in}{3.104771in}}%
\pgfpathlineto{\pgfqpoint{3.331588in}{3.102410in}}%
\pgfpathlineto{\pgfqpoint{3.336138in}{3.102738in}}%
\pgfpathlineto{\pgfqpoint{3.340955in}{3.105563in}}%
\pgfpathlineto{\pgfqpoint{3.346842in}{3.111990in}}%
\pgfpathlineto{\pgfqpoint{3.356208in}{3.126229in}}%
\pgfpathlineto{\pgfqpoint{3.367180in}{3.141920in}}%
\pgfpathlineto{\pgfqpoint{3.373068in}{3.146835in}}%
\pgfpathlineto{\pgfqpoint{3.377885in}{3.148160in}}%
\pgfpathlineto{\pgfqpoint{3.382434in}{3.147019in}}%
\pgfpathlineto{\pgfqpoint{3.387518in}{3.143141in}}%
\pgfpathlineto{\pgfqpoint{3.393941in}{3.135097in}}%
\pgfpathlineto{\pgfqpoint{3.415885in}{3.104789in}}%
\pgfpathlineto{\pgfqpoint{3.420969in}{3.102415in}}%
\pgfpathlineto{\pgfqpoint{3.425519in}{3.102730in}}%
\pgfpathlineto{\pgfqpoint{3.430068in}{3.105323in}}%
\pgfpathlineto{\pgfqpoint{3.435688in}{3.111263in}}%
\pgfpathlineto{\pgfqpoint{3.444519in}{3.124464in}}%
\pgfpathlineto{\pgfqpoint{3.456561in}{3.141893in}}%
\pgfpathlineto{\pgfqpoint{3.462449in}{3.146822in}}%
\pgfpathlineto{\pgfqpoint{3.467266in}{3.148160in}}%
\pgfpathlineto{\pgfqpoint{3.471815in}{3.147031in}}%
\pgfpathlineto{\pgfqpoint{3.476632in}{3.143431in}}%
\pgfpathlineto{\pgfqpoint{3.483054in}{3.135520in}}%
\pgfpathlineto{\pgfqpoint{3.505801in}{3.104426in}}%
\pgfpathlineto{\pgfqpoint{3.510886in}{3.102333in}}%
\pgfpathlineto{\pgfqpoint{3.515435in}{3.102910in}}%
\pgfpathlineto{\pgfqpoint{3.520252in}{3.105980in}}%
\pgfpathlineto{\pgfqpoint{3.526139in}{3.112640in}}%
\pgfpathlineto{\pgfqpoint{3.536308in}{3.128300in}}%
\pgfpathlineto{\pgfqpoint{3.546477in}{3.142448in}}%
\pgfpathlineto{\pgfqpoint{3.552365in}{3.147085in}}%
\pgfpathlineto{\pgfqpoint{3.557182in}{3.148149in}}%
\pgfpathlineto{\pgfqpoint{3.561731in}{3.146762in}}%
\pgfpathlineto{\pgfqpoint{3.566816in}{3.142641in}}%
\pgfpathlineto{\pgfqpoint{3.573506in}{3.133986in}}%
\pgfpathlineto{\pgfqpoint{3.593844in}{3.105452in}}%
\pgfpathlineto{\pgfqpoint{3.599196in}{3.102543in}}%
\pgfpathlineto{\pgfqpoint{3.603746in}{3.102559in}}%
\pgfpathlineto{\pgfqpoint{3.608295in}{3.104870in}}%
\pgfpathlineto{\pgfqpoint{3.613915in}{3.110527in}}%
\pgfpathlineto{\pgfqpoint{3.622210in}{3.122664in}}%
\pgfpathlineto{\pgfqpoint{3.635591in}{3.142133in}}%
\pgfpathlineto{\pgfqpoint{3.641478in}{3.146938in}}%
\pgfpathlineto{\pgfqpoint{3.646295in}{3.148159in}}%
\pgfpathlineto{\pgfqpoint{3.650845in}{3.146919in}}%
\pgfpathlineto{\pgfqpoint{3.655929in}{3.142943in}}%
\pgfpathlineto{\pgfqpoint{3.662619in}{3.134419in}}%
\pgfpathlineto{\pgfqpoint{3.683760in}{3.105046in}}%
\pgfpathlineto{\pgfqpoint{3.688845in}{3.102485in}}%
\pgfpathlineto{\pgfqpoint{3.693394in}{3.102626in}}%
\pgfpathlineto{\pgfqpoint{3.697943in}{3.105055in}}%
\pgfpathlineto{\pgfqpoint{3.703563in}{3.110831in}}%
\pgfpathlineto{\pgfqpoint{3.711859in}{3.123054in}}%
\pgfpathlineto{\pgfqpoint{3.724972in}{3.142107in}}%
\pgfpathlineto{\pgfqpoint{3.730859in}{3.146925in}}%
\pgfpathlineto{\pgfqpoint{3.735676in}{3.148159in}}%
\pgfpathlineto{\pgfqpoint{3.740226in}{3.146932in}}%
\pgfpathlineto{\pgfqpoint{3.745310in}{3.142968in}}%
\pgfpathlineto{\pgfqpoint{3.751733in}{3.134848in}}%
\pgfpathlineto{\pgfqpoint{3.773409in}{3.104861in}}%
\pgfpathlineto{\pgfqpoint{3.778493in}{3.102433in}}%
\pgfpathlineto{\pgfqpoint{3.783043in}{3.102699in}}%
\pgfpathlineto{\pgfqpoint{3.787592in}{3.105246in}}%
\pgfpathlineto{\pgfqpoint{3.793212in}{3.111139in}}%
\pgfpathlineto{\pgfqpoint{3.801775in}{3.123876in}}%
\pgfpathlineto{\pgfqpoint{3.814353in}{3.142080in}}%
\pgfpathlineto{\pgfqpoint{3.820240in}{3.146913in}}%
\pgfpathlineto{\pgfqpoint{3.825057in}{3.148160in}}%
\pgfpathlineto{\pgfqpoint{3.829607in}{3.146944in}}%
\pgfpathlineto{\pgfqpoint{3.834691in}{3.142993in}}%
\pgfpathlineto{\pgfqpoint{3.841114in}{3.134884in}}%
\pgfpathlineto{\pgfqpoint{3.862790in}{3.104879in}}%
\pgfpathlineto{\pgfqpoint{3.867874in}{3.102438in}}%
\pgfpathlineto{\pgfqpoint{3.872424in}{3.102691in}}%
\pgfpathlineto{\pgfqpoint{3.876973in}{3.105226in}}%
\pgfpathlineto{\pgfqpoint{3.882593in}{3.111108in}}%
\pgfpathlineto{\pgfqpoint{3.891156in}{3.123837in}}%
\pgfpathlineto{\pgfqpoint{3.903734in}{3.142054in}}%
\pgfpathlineto{\pgfqpoint{3.909621in}{3.146900in}}%
\pgfpathlineto{\pgfqpoint{3.914438in}{3.148160in}}%
\pgfpathlineto{\pgfqpoint{3.918988in}{3.146957in}}%
\pgfpathlineto{\pgfqpoint{3.924072in}{3.143018in}}%
\pgfpathlineto{\pgfqpoint{3.930495in}{3.134919in}}%
\pgfpathlineto{\pgfqpoint{3.952171in}{3.104898in}}%
\pgfpathlineto{\pgfqpoint{3.957255in}{3.102443in}}%
\pgfpathlineto{\pgfqpoint{3.961805in}{3.102684in}}%
\pgfpathlineto{\pgfqpoint{3.966354in}{3.105207in}}%
\pgfpathlineto{\pgfqpoint{3.971974in}{3.111077in}}%
\pgfpathlineto{\pgfqpoint{3.980537in}{3.123798in}}%
\pgfpathlineto{\pgfqpoint{3.993115in}{3.142027in}}%
\pgfpathlineto{\pgfqpoint{3.999002in}{3.146887in}}%
\pgfpathlineto{\pgfqpoint{4.003819in}{3.148160in}}%
\pgfpathlineto{\pgfqpoint{4.008369in}{3.146969in}}%
\pgfpathlineto{\pgfqpoint{4.013453in}{3.143043in}}%
\pgfpathlineto{\pgfqpoint{4.019876in}{3.134955in}}%
\pgfpathlineto{\pgfqpoint{4.041820in}{3.104718in}}%
\pgfpathlineto{\pgfqpoint{4.046904in}{3.102397in}}%
\pgfpathlineto{\pgfqpoint{4.051453in}{3.102762in}}%
\pgfpathlineto{\pgfqpoint{4.056270in}{3.105624in}}%
\pgfpathlineto{\pgfqpoint{4.062158in}{3.112087in}}%
\pgfpathlineto{\pgfqpoint{4.071524in}{3.126347in}}%
\pgfpathlineto{\pgfqpoint{4.082496in}{3.142000in}}%
\pgfpathlineto{\pgfqpoint{4.088383in}{3.146874in}}%
\pgfpathlineto{\pgfqpoint{4.093200in}{3.148160in}}%
\pgfpathlineto{\pgfqpoint{4.097750in}{3.146982in}}%
\pgfpathlineto{\pgfqpoint{4.102834in}{3.143067in}}%
\pgfpathlineto{\pgfqpoint{4.109257in}{3.134990in}}%
\pgfpathlineto{\pgfqpoint{4.131201in}{3.104736in}}%
\pgfpathlineto{\pgfqpoint{4.136285in}{3.102401in}}%
\pgfpathlineto{\pgfqpoint{4.140834in}{3.102754in}}%
\pgfpathlineto{\pgfqpoint{4.145651in}{3.105603in}}%
\pgfpathlineto{\pgfqpoint{4.151539in}{3.112055in}}%
\pgfpathlineto{\pgfqpoint{4.160905in}{3.126308in}}%
\pgfpathlineto{\pgfqpoint{4.171877in}{3.141974in}}%
\pgfpathlineto{\pgfqpoint{4.177764in}{3.146861in}}%
\pgfpathlineto{\pgfqpoint{4.182581in}{3.148160in}}%
\pgfpathlineto{\pgfqpoint{4.187131in}{3.146994in}}%
\pgfpathlineto{\pgfqpoint{4.192215in}{3.143092in}}%
\pgfpathlineto{\pgfqpoint{4.198638in}{3.135026in}}%
\pgfpathlineto{\pgfqpoint{4.220582in}{3.104753in}}%
\pgfpathlineto{\pgfqpoint{4.225666in}{3.102406in}}%
\pgfpathlineto{\pgfqpoint{4.230215in}{3.102746in}}%
\pgfpathlineto{\pgfqpoint{4.235032in}{3.105583in}}%
\pgfpathlineto{\pgfqpoint{4.240920in}{3.112023in}}%
\pgfpathlineto{\pgfqpoint{4.250286in}{3.126269in}}%
\pgfpathlineto{\pgfqpoint{4.261258in}{3.141947in}}%
\pgfpathlineto{\pgfqpoint{4.267145in}{3.146848in}}%
\pgfpathlineto{\pgfqpoint{4.271962in}{3.148160in}}%
\pgfpathlineto{\pgfqpoint{4.276512in}{3.147007in}}%
\pgfpathlineto{\pgfqpoint{4.281596in}{3.143116in}}%
\pgfpathlineto{\pgfqpoint{4.288019in}{3.135061in}}%
\pgfpathlineto{\pgfqpoint{4.309963in}{3.104771in}}%
\pgfpathlineto{\pgfqpoint{4.315047in}{3.102410in}}%
\pgfpathlineto{\pgfqpoint{4.319596in}{3.102738in}}%
\pgfpathlineto{\pgfqpoint{4.324413in}{3.105563in}}%
\pgfpathlineto{\pgfqpoint{4.330301in}{3.111990in}}%
\pgfpathlineto{\pgfqpoint{4.339667in}{3.126229in}}%
\pgfpathlineto{\pgfqpoint{4.350639in}{3.141920in}}%
\pgfpathlineto{\pgfqpoint{4.356526in}{3.146835in}}%
\pgfpathlineto{\pgfqpoint{4.361343in}{3.148160in}}%
\pgfpathlineto{\pgfqpoint{4.365893in}{3.147019in}}%
\pgfpathlineto{\pgfqpoint{4.370977in}{3.143141in}}%
\pgfpathlineto{\pgfqpoint{4.377400in}{3.135097in}}%
\pgfpathlineto{\pgfqpoint{4.399344in}{3.104789in}}%
\pgfpathlineto{\pgfqpoint{4.404428in}{3.102415in}}%
\pgfpathlineto{\pgfqpoint{4.408977in}{3.102730in}}%
\pgfpathlineto{\pgfqpoint{4.413527in}{3.105323in}}%
\pgfpathlineto{\pgfqpoint{4.419147in}{3.111263in}}%
\pgfpathlineto{\pgfqpoint{4.427978in}{3.124464in}}%
\pgfpathlineto{\pgfqpoint{4.440020in}{3.141893in}}%
\pgfpathlineto{\pgfqpoint{4.445907in}{3.146822in}}%
\pgfpathlineto{\pgfqpoint{4.450724in}{3.148160in}}%
\pgfpathlineto{\pgfqpoint{4.455274in}{3.147031in}}%
\pgfpathlineto{\pgfqpoint{4.460091in}{3.143431in}}%
\pgfpathlineto{\pgfqpoint{4.466513in}{3.135520in}}%
\pgfpathlineto{\pgfqpoint{4.489260in}{3.104426in}}%
\pgfpathlineto{\pgfqpoint{4.494344in}{3.102333in}}%
\pgfpathlineto{\pgfqpoint{4.498894in}{3.102910in}}%
\pgfpathlineto{\pgfqpoint{4.503711in}{3.105980in}}%
\pgfpathlineto{\pgfqpoint{4.509598in}{3.112640in}}%
\pgfpathlineto{\pgfqpoint{4.519767in}{3.128300in}}%
\pgfpathlineto{\pgfqpoint{4.529936in}{3.142448in}}%
\pgfpathlineto{\pgfqpoint{4.535824in}{3.147085in}}%
\pgfpathlineto{\pgfqpoint{4.540641in}{3.148149in}}%
\pgfpathlineto{\pgfqpoint{4.545190in}{3.146762in}}%
\pgfpathlineto{\pgfqpoint{4.550274in}{3.142641in}}%
\pgfpathlineto{\pgfqpoint{4.556965in}{3.133986in}}%
\pgfpathlineto{\pgfqpoint{4.577303in}{3.105452in}}%
\pgfpathlineto{\pgfqpoint{4.582655in}{3.102543in}}%
\pgfpathlineto{\pgfqpoint{4.587204in}{3.102559in}}%
\pgfpathlineto{\pgfqpoint{4.591754in}{3.104870in}}%
\pgfpathlineto{\pgfqpoint{4.597373in}{3.110527in}}%
\pgfpathlineto{\pgfqpoint{4.605669in}{3.122664in}}%
\pgfpathlineto{\pgfqpoint{4.619050in}{3.142133in}}%
\pgfpathlineto{\pgfqpoint{4.624937in}{3.146938in}}%
\pgfpathlineto{\pgfqpoint{4.629754in}{3.148159in}}%
\pgfpathlineto{\pgfqpoint{4.634303in}{3.146919in}}%
\pgfpathlineto{\pgfqpoint{4.639388in}{3.142943in}}%
\pgfpathlineto{\pgfqpoint{4.646078in}{3.134419in}}%
\pgfpathlineto{\pgfqpoint{4.667219in}{3.105046in}}%
\pgfpathlineto{\pgfqpoint{4.672304in}{3.102485in}}%
\pgfpathlineto{\pgfqpoint{4.676853in}{3.102626in}}%
\pgfpathlineto{\pgfqpoint{4.681402in}{3.105055in}}%
\pgfpathlineto{\pgfqpoint{4.687022in}{3.110831in}}%
\pgfpathlineto{\pgfqpoint{4.695318in}{3.123054in}}%
\pgfpathlineto{\pgfqpoint{4.708431in}{3.142107in}}%
\pgfpathlineto{\pgfqpoint{4.714318in}{3.146925in}}%
\pgfpathlineto{\pgfqpoint{4.719135in}{3.148159in}}%
\pgfpathlineto{\pgfqpoint{4.723684in}{3.146932in}}%
\pgfpathlineto{\pgfqpoint{4.728769in}{3.142968in}}%
\pgfpathlineto{\pgfqpoint{4.735191in}{3.134848in}}%
\pgfpathlineto{\pgfqpoint{4.756868in}{3.104861in}}%
\pgfpathlineto{\pgfqpoint{4.761952in}{3.102433in}}%
\pgfpathlineto{\pgfqpoint{4.766502in}{3.102699in}}%
\pgfpathlineto{\pgfqpoint{4.771051in}{3.105246in}}%
\pgfpathlineto{\pgfqpoint{4.776671in}{3.111139in}}%
\pgfpathlineto{\pgfqpoint{4.785234in}{3.123876in}}%
\pgfpathlineto{\pgfqpoint{4.797812in}{3.142080in}}%
\pgfpathlineto{\pgfqpoint{4.803699in}{3.146913in}}%
\pgfpathlineto{\pgfqpoint{4.808516in}{3.148160in}}%
\pgfpathlineto{\pgfqpoint{4.813065in}{3.146944in}}%
\pgfpathlineto{\pgfqpoint{4.818150in}{3.142993in}}%
\pgfpathlineto{\pgfqpoint{4.824572in}{3.134884in}}%
\pgfpathlineto{\pgfqpoint{4.846249in}{3.104879in}}%
\pgfpathlineto{\pgfqpoint{4.851333in}{3.102438in}}%
\pgfpathlineto{\pgfqpoint{4.855883in}{3.102691in}}%
\pgfpathlineto{\pgfqpoint{4.860432in}{3.105226in}}%
\pgfpathlineto{\pgfqpoint{4.866052in}{3.111108in}}%
\pgfpathlineto{\pgfqpoint{4.874615in}{3.123837in}}%
\pgfpathlineto{\pgfqpoint{4.887193in}{3.142054in}}%
\pgfpathlineto{\pgfqpoint{4.893080in}{3.146900in}}%
\pgfpathlineto{\pgfqpoint{4.897897in}{3.148160in}}%
\pgfpathlineto{\pgfqpoint{4.902446in}{3.146957in}}%
\pgfpathlineto{\pgfqpoint{4.907531in}{3.143018in}}%
\pgfpathlineto{\pgfqpoint{4.913953in}{3.134919in}}%
\pgfpathlineto{\pgfqpoint{4.935630in}{3.104898in}}%
\pgfpathlineto{\pgfqpoint{4.940714in}{3.102443in}}%
\pgfpathlineto{\pgfqpoint{4.945264in}{3.102684in}}%
\pgfpathlineto{\pgfqpoint{4.949813in}{3.105207in}}%
\pgfpathlineto{\pgfqpoint{4.955433in}{3.111077in}}%
\pgfpathlineto{\pgfqpoint{4.963996in}{3.123798in}}%
\pgfpathlineto{\pgfqpoint{4.976574in}{3.142027in}}%
\pgfpathlineto{\pgfqpoint{4.982461in}{3.146887in}}%
\pgfpathlineto{\pgfqpoint{4.987278in}{3.148160in}}%
\pgfpathlineto{\pgfqpoint{4.991827in}{3.146969in}}%
\pgfpathlineto{\pgfqpoint{4.996912in}{3.143043in}}%
\pgfpathlineto{\pgfqpoint{5.003334in}{3.134955in}}%
\pgfpathlineto{\pgfqpoint{5.025278in}{3.104718in}}%
\pgfpathlineto{\pgfqpoint{5.030363in}{3.102397in}}%
\pgfpathlineto{\pgfqpoint{5.034912in}{3.102762in}}%
\pgfpathlineto{\pgfqpoint{5.039729in}{3.105624in}}%
\pgfpathlineto{\pgfqpoint{5.045616in}{3.112087in}}%
\pgfpathlineto{\pgfqpoint{5.054983in}{3.126347in}}%
\pgfpathlineto{\pgfqpoint{5.065955in}{3.142000in}}%
\pgfpathlineto{\pgfqpoint{5.071842in}{3.146874in}}%
\pgfpathlineto{\pgfqpoint{5.076659in}{3.148160in}}%
\pgfpathlineto{\pgfqpoint{5.081208in}{3.146982in}}%
\pgfpathlineto{\pgfqpoint{5.086293in}{3.143067in}}%
\pgfpathlineto{\pgfqpoint{5.092715in}{3.134990in}}%
\pgfpathlineto{\pgfqpoint{5.114659in}{3.104736in}}%
\pgfpathlineto{\pgfqpoint{5.119744in}{3.102401in}}%
\pgfpathlineto{\pgfqpoint{5.124293in}{3.102754in}}%
\pgfpathlineto{\pgfqpoint{5.129110in}{3.105603in}}%
\pgfpathlineto{\pgfqpoint{5.134997in}{3.112055in}}%
\pgfpathlineto{\pgfqpoint{5.144364in}{3.126308in}}%
\pgfpathlineto{\pgfqpoint{5.155336in}{3.141974in}}%
\pgfpathlineto{\pgfqpoint{5.161223in}{3.146861in}}%
\pgfpathlineto{\pgfqpoint{5.166040in}{3.148160in}}%
\pgfpathlineto{\pgfqpoint{5.170589in}{3.146994in}}%
\pgfpathlineto{\pgfqpoint{5.175674in}{3.143092in}}%
\pgfpathlineto{\pgfqpoint{5.182096in}{3.135026in}}%
\pgfpathlineto{\pgfqpoint{5.204040in}{3.104753in}}%
\pgfpathlineto{\pgfqpoint{5.209125in}{3.102406in}}%
\pgfpathlineto{\pgfqpoint{5.213674in}{3.102746in}}%
\pgfpathlineto{\pgfqpoint{5.218491in}{3.105583in}}%
\pgfpathlineto{\pgfqpoint{5.224379in}{3.112023in}}%
\pgfpathlineto{\pgfqpoint{5.233745in}{3.126269in}}%
\pgfpathlineto{\pgfqpoint{5.244717in}{3.141947in}}%
\pgfpathlineto{\pgfqpoint{5.250604in}{3.146848in}}%
\pgfpathlineto{\pgfqpoint{5.255421in}{3.148160in}}%
\pgfpathlineto{\pgfqpoint{5.259970in}{3.147007in}}%
\pgfpathlineto{\pgfqpoint{5.265055in}{3.143116in}}%
\pgfpathlineto{\pgfqpoint{5.271477in}{3.135061in}}%
\pgfpathlineto{\pgfqpoint{5.293421in}{3.104771in}}%
\pgfpathlineto{\pgfqpoint{5.298506in}{3.102410in}}%
\pgfpathlineto{\pgfqpoint{5.303055in}{3.102738in}}%
\pgfpathlineto{\pgfqpoint{5.307872in}{3.105563in}}%
\pgfpathlineto{\pgfqpoint{5.313760in}{3.111990in}}%
\pgfpathlineto{\pgfqpoint{5.323126in}{3.126229in}}%
\pgfpathlineto{\pgfqpoint{5.334098in}{3.141920in}}%
\pgfpathlineto{\pgfqpoint{5.339985in}{3.146835in}}%
\pgfpathlineto{\pgfqpoint{5.344802in}{3.148160in}}%
\pgfpathlineto{\pgfqpoint{5.349351in}{3.147019in}}%
\pgfpathlineto{\pgfqpoint{5.354436in}{3.143141in}}%
\pgfpathlineto{\pgfqpoint{5.360858in}{3.135097in}}%
\pgfpathlineto{\pgfqpoint{5.382802in}{3.104789in}}%
\pgfpathlineto{\pgfqpoint{5.387887in}{3.102415in}}%
\pgfpathlineto{\pgfqpoint{5.392436in}{3.102730in}}%
\pgfpathlineto{\pgfqpoint{5.396986in}{3.105323in}}%
\pgfpathlineto{\pgfqpoint{5.402605in}{3.111263in}}%
\pgfpathlineto{\pgfqpoint{5.411436in}{3.124464in}}%
\pgfpathlineto{\pgfqpoint{5.423479in}{3.141893in}}%
\pgfpathlineto{\pgfqpoint{5.429366in}{3.146822in}}%
\pgfpathlineto{\pgfqpoint{5.434183in}{3.148160in}}%
\pgfpathlineto{\pgfqpoint{5.438732in}{3.147031in}}%
\pgfpathlineto{\pgfqpoint{5.443549in}{3.143431in}}%
\pgfpathlineto{\pgfqpoint{5.449972in}{3.135520in}}%
\pgfpathlineto{\pgfqpoint{5.472719in}{3.104426in}}%
\pgfpathlineto{\pgfqpoint{5.477803in}{3.102333in}}%
\pgfpathlineto{\pgfqpoint{5.482352in}{3.102910in}}%
\pgfpathlineto{\pgfqpoint{5.487169in}{3.105980in}}%
\pgfpathlineto{\pgfqpoint{5.493057in}{3.112640in}}%
\pgfpathlineto{\pgfqpoint{5.503226in}{3.128300in}}%
\pgfpathlineto{\pgfqpoint{5.513395in}{3.142448in}}%
\pgfpathlineto{\pgfqpoint{5.519282in}{3.147085in}}%
\pgfpathlineto{\pgfqpoint{5.524099in}{3.148149in}}%
\pgfpathlineto{\pgfqpoint{5.528649in}{3.146762in}}%
\pgfpathlineto{\pgfqpoint{5.533733in}{3.142641in}}%
\pgfpathlineto{\pgfqpoint{5.540423in}{3.133986in}}%
\pgfpathlineto{\pgfqpoint{5.560762in}{3.105452in}}%
\pgfpathlineto{\pgfqpoint{5.566114in}{3.102543in}}%
\pgfpathlineto{\pgfqpoint{5.570663in}{3.102559in}}%
\pgfpathlineto{\pgfqpoint{5.575212in}{3.104870in}}%
\pgfpathlineto{\pgfqpoint{5.580832in}{3.110527in}}%
\pgfpathlineto{\pgfqpoint{5.589128in}{3.122664in}}%
\pgfpathlineto{\pgfqpoint{5.602508in}{3.142133in}}%
\pgfpathlineto{\pgfqpoint{5.608396in}{3.146938in}}%
\pgfpathlineto{\pgfqpoint{5.613213in}{3.148159in}}%
\pgfpathlineto{\pgfqpoint{5.617762in}{3.146919in}}%
\pgfpathlineto{\pgfqpoint{5.622847in}{3.142943in}}%
\pgfpathlineto{\pgfqpoint{5.629537in}{3.134419in}}%
\pgfpathlineto{\pgfqpoint{5.650678in}{3.105046in}}%
\pgfpathlineto{\pgfqpoint{5.655762in}{3.102485in}}%
\pgfpathlineto{\pgfqpoint{5.660312in}{3.102626in}}%
\pgfpathlineto{\pgfqpoint{5.664861in}{3.105055in}}%
\pgfpathlineto{\pgfqpoint{5.670481in}{3.110831in}}%
\pgfpathlineto{\pgfqpoint{5.678777in}{3.123054in}}%
\pgfpathlineto{\pgfqpoint{5.691889in}{3.142107in}}%
\pgfpathlineto{\pgfqpoint{5.697777in}{3.146925in}}%
\pgfpathlineto{\pgfqpoint{5.702594in}{3.148159in}}%
\pgfpathlineto{\pgfqpoint{5.707143in}{3.146932in}}%
\pgfpathlineto{\pgfqpoint{5.712228in}{3.142968in}}%
\pgfpathlineto{\pgfqpoint{5.718650in}{3.134848in}}%
\pgfpathlineto{\pgfqpoint{5.740326in}{3.104861in}}%
\pgfpathlineto{\pgfqpoint{5.745411in}{3.102433in}}%
\pgfpathlineto{\pgfqpoint{5.749960in}{3.102699in}}%
\pgfpathlineto{\pgfqpoint{5.754510in}{3.105246in}}%
\pgfpathlineto{\pgfqpoint{5.760129in}{3.111139in}}%
\pgfpathlineto{\pgfqpoint{5.768693in}{3.123876in}}%
\pgfpathlineto{\pgfqpoint{5.781270in}{3.142080in}}%
\pgfpathlineto{\pgfqpoint{5.787158in}{3.146913in}}%
\pgfpathlineto{\pgfqpoint{5.791975in}{3.148160in}}%
\pgfpathlineto{\pgfqpoint{5.796524in}{3.146944in}}%
\pgfpathlineto{\pgfqpoint{5.801609in}{3.142993in}}%
\pgfpathlineto{\pgfqpoint{5.808031in}{3.134884in}}%
\pgfpathlineto{\pgfqpoint{5.829707in}{3.104879in}}%
\pgfpathlineto{\pgfqpoint{5.834792in}{3.102438in}}%
\pgfpathlineto{\pgfqpoint{5.839341in}{3.102691in}}%
\pgfpathlineto{\pgfqpoint{5.843891in}{3.105226in}}%
\pgfpathlineto{\pgfqpoint{5.849510in}{3.111108in}}%
\pgfpathlineto{\pgfqpoint{5.858074in}{3.123837in}}%
\pgfpathlineto{\pgfqpoint{5.870651in}{3.142054in}}%
\pgfpathlineto{\pgfqpoint{5.876539in}{3.146900in}}%
\pgfpathlineto{\pgfqpoint{5.881356in}{3.148160in}}%
\pgfpathlineto{\pgfqpoint{5.885905in}{3.146957in}}%
\pgfpathlineto{\pgfqpoint{5.890990in}{3.143018in}}%
\pgfpathlineto{\pgfqpoint{5.897412in}{3.134919in}}%
\pgfpathlineto{\pgfqpoint{5.919088in}{3.104898in}}%
\pgfpathlineto{\pgfqpoint{5.924173in}{3.102443in}}%
\pgfpathlineto{\pgfqpoint{5.928722in}{3.102684in}}%
\pgfpathlineto{\pgfqpoint{5.933272in}{3.105207in}}%
\pgfpathlineto{\pgfqpoint{5.938891in}{3.111077in}}%
\pgfpathlineto{\pgfqpoint{5.947455in}{3.123798in}}%
\pgfpathlineto{\pgfqpoint{5.960032in}{3.142027in}}%
\pgfpathlineto{\pgfqpoint{5.965920in}{3.146887in}}%
\pgfpathlineto{\pgfqpoint{5.970737in}{3.148160in}}%
\pgfpathlineto{\pgfqpoint{5.975286in}{3.146969in}}%
\pgfpathlineto{\pgfqpoint{5.980371in}{3.143043in}}%
\pgfpathlineto{\pgfqpoint{5.986793in}{3.134955in}}%
\pgfpathlineto{\pgfqpoint{6.008737in}{3.104718in}}%
\pgfpathlineto{\pgfqpoint{6.013822in}{3.102397in}}%
\pgfpathlineto{\pgfqpoint{6.018371in}{3.102762in}}%
\pgfpathlineto{\pgfqpoint{6.023188in}{3.105624in}}%
\pgfpathlineto{\pgfqpoint{6.029075in}{3.112087in}}%
\pgfpathlineto{\pgfqpoint{6.038441in}{3.126347in}}%
\pgfpathlineto{\pgfqpoint{6.049413in}{3.142000in}}%
\pgfpathlineto{\pgfqpoint{6.055301in}{3.146874in}}%
\pgfpathlineto{\pgfqpoint{6.060118in}{3.148160in}}%
\pgfpathlineto{\pgfqpoint{6.064667in}{3.146982in}}%
\pgfpathlineto{\pgfqpoint{6.069752in}{3.143067in}}%
\pgfpathlineto{\pgfqpoint{6.076174in}{3.134990in}}%
\pgfpathlineto{\pgfqpoint{6.098118in}{3.104736in}}%
\pgfpathlineto{\pgfqpoint{6.103203in}{3.102401in}}%
\pgfpathlineto{\pgfqpoint{6.107752in}{3.102754in}}%
\pgfpathlineto{\pgfqpoint{6.112569in}{3.105603in}}%
\pgfpathlineto{\pgfqpoint{6.118456in}{3.112055in}}%
\pgfpathlineto{\pgfqpoint{6.127823in}{3.126308in}}%
\pgfpathlineto{\pgfqpoint{6.138794in}{3.141974in}}%
\pgfpathlineto{\pgfqpoint{6.144682in}{3.146861in}}%
\pgfpathlineto{\pgfqpoint{6.149499in}{3.148160in}}%
\pgfpathlineto{\pgfqpoint{6.154048in}{3.146994in}}%
\pgfpathlineto{\pgfqpoint{6.159133in}{3.143092in}}%
\pgfpathlineto{\pgfqpoint{6.165555in}{3.135026in}}%
\pgfpathlineto{\pgfqpoint{6.187499in}{3.104753in}}%
\pgfpathlineto{\pgfqpoint{6.192584in}{3.102406in}}%
\pgfpathlineto{\pgfqpoint{6.197133in}{3.102746in}}%
\pgfpathlineto{\pgfqpoint{6.201950in}{3.105583in}}%
\pgfpathlineto{\pgfqpoint{6.207837in}{3.112023in}}%
\pgfpathlineto{\pgfqpoint{6.217204in}{3.126269in}}%
\pgfpathlineto{\pgfqpoint{6.228175in}{3.141947in}}%
\pgfpathlineto{\pgfqpoint{6.234063in}{3.146848in}}%
\pgfpathlineto{\pgfqpoint{6.238880in}{3.148160in}}%
\pgfpathlineto{\pgfqpoint{6.243429in}{3.147007in}}%
\pgfpathlineto{\pgfqpoint{6.248514in}{3.143116in}}%
\pgfpathlineto{\pgfqpoint{6.254936in}{3.135061in}}%
\pgfpathlineto{\pgfqpoint{6.276880in}{3.104771in}}%
\pgfpathlineto{\pgfqpoint{6.281965in}{3.102410in}}%
\pgfpathlineto{\pgfqpoint{6.286514in}{3.102738in}}%
\pgfpathlineto{\pgfqpoint{6.291331in}{3.105563in}}%
\pgfpathlineto{\pgfqpoint{6.297218in}{3.111990in}}%
\pgfpathlineto{\pgfqpoint{6.306585in}{3.126229in}}%
\pgfpathlineto{\pgfqpoint{6.317556in}{3.141920in}}%
\pgfpathlineto{\pgfqpoint{6.323444in}{3.146835in}}%
\pgfpathlineto{\pgfqpoint{6.328261in}{3.148160in}}%
\pgfpathlineto{\pgfqpoint{6.332810in}{3.147019in}}%
\pgfpathlineto{\pgfqpoint{6.337895in}{3.143141in}}%
\pgfpathlineto{\pgfqpoint{6.344317in}{3.135097in}}%
\pgfpathlineto{\pgfqpoint{6.366261in}{3.104789in}}%
\pgfpathlineto{\pgfqpoint{6.371346in}{3.102415in}}%
\pgfpathlineto{\pgfqpoint{6.375895in}{3.102730in}}%
\pgfpathlineto{\pgfqpoint{6.380444in}{3.105323in}}%
\pgfpathlineto{\pgfqpoint{6.386064in}{3.111263in}}%
\pgfpathlineto{\pgfqpoint{6.394895in}{3.124464in}}%
\pgfpathlineto{\pgfqpoint{6.406937in}{3.141893in}}%
\pgfpathlineto{\pgfqpoint{6.412825in}{3.146822in}}%
\pgfpathlineto{\pgfqpoint{6.417642in}{3.148160in}}%
\pgfpathlineto{\pgfqpoint{6.422191in}{3.147031in}}%
\pgfpathlineto{\pgfqpoint{6.427008in}{3.143431in}}%
\pgfpathlineto{\pgfqpoint{6.433431in}{3.135520in}}%
\pgfpathlineto{\pgfqpoint{6.456177in}{3.104426in}}%
\pgfpathlineto{\pgfqpoint{6.461262in}{3.102333in}}%
\pgfpathlineto{\pgfqpoint{6.465811in}{3.102910in}}%
\pgfpathlineto{\pgfqpoint{6.470628in}{3.105980in}}%
\pgfpathlineto{\pgfqpoint{6.476515in}{3.112640in}}%
\pgfpathlineto{\pgfqpoint{6.486685in}{3.128300in}}%
\pgfpathlineto{\pgfqpoint{6.496854in}{3.142448in}}%
\pgfpathlineto{\pgfqpoint{6.502741in}{3.147085in}}%
\pgfpathlineto{\pgfqpoint{6.507558in}{3.148149in}}%
\pgfpathlineto{\pgfqpoint{6.512107in}{3.146762in}}%
\pgfpathlineto{\pgfqpoint{6.517192in}{3.142641in}}%
\pgfpathlineto{\pgfqpoint{6.523882in}{3.133986in}}%
\pgfpathlineto{\pgfqpoint{6.544220in}{3.105452in}}%
\pgfpathlineto{\pgfqpoint{6.549572in}{3.102543in}}%
\pgfpathlineto{\pgfqpoint{6.554122in}{3.102559in}}%
\pgfpathlineto{\pgfqpoint{6.558671in}{3.104870in}}%
\pgfpathlineto{\pgfqpoint{6.564291in}{3.110527in}}%
\pgfpathlineto{\pgfqpoint{6.572587in}{3.122664in}}%
\pgfpathlineto{\pgfqpoint{6.585967in}{3.142133in}}%
\pgfpathlineto{\pgfqpoint{6.591854in}{3.146938in}}%
\pgfpathlineto{\pgfqpoint{6.596671in}{3.148159in}}%
\pgfpathlineto{\pgfqpoint{6.601221in}{3.146919in}}%
\pgfpathlineto{\pgfqpoint{6.606305in}{3.142943in}}%
\pgfpathlineto{\pgfqpoint{6.612995in}{3.134419in}}%
\pgfpathlineto{\pgfqpoint{6.634137in}{3.105046in}}%
\pgfpathlineto{\pgfqpoint{6.639221in}{3.102485in}}%
\pgfpathlineto{\pgfqpoint{6.643770in}{3.102626in}}%
\pgfpathlineto{\pgfqpoint{6.648320in}{3.105055in}}%
\pgfpathlineto{\pgfqpoint{6.653939in}{3.110831in}}%
\pgfpathlineto{\pgfqpoint{6.662235in}{3.123054in}}%
\pgfpathlineto{\pgfqpoint{6.663306in}{3.124778in}}%
\pgfpathlineto{\pgfqpoint{6.663306in}{3.124778in}}%
\pgfusepath{stroke}%
\end{pgfscope}%
\begin{pgfscope}%
\pgfpathrectangle{\pgfqpoint{0.467797in}{2.292089in}}{\pgfqpoint{6.490533in}{1.666241in}}%
\pgfusepath{clip}%
\pgfsetrectcap%
\pgfsetroundjoin%
\pgfsetlinewidth{1.505625pt}%
\definecolor{currentstroke}{rgb}{0.498039,0.498039,0.498039}%
\pgfsetstrokecolor{currentstroke}%
\pgfsetdash{}{0pt}%
\pgfpathmoveto{\pgfqpoint{0.762821in}{3.125209in}}%
\pgfpathlineto{\pgfqpoint{0.774060in}{3.141291in}}%
\pgfpathlineto{\pgfqpoint{0.779680in}{3.145790in}}%
\pgfpathlineto{\pgfqpoint{0.784230in}{3.146842in}}%
\pgfpathlineto{\pgfqpoint{0.788511in}{3.145581in}}%
\pgfpathlineto{\pgfqpoint{0.793328in}{3.141717in}}%
\pgfpathlineto{\pgfqpoint{0.799751in}{3.133416in}}%
\pgfpathlineto{\pgfqpoint{0.819019in}{3.106469in}}%
\pgfpathlineto{\pgfqpoint{0.824103in}{3.103796in}}%
\pgfpathlineto{\pgfqpoint{0.828385in}{3.103899in}}%
\pgfpathlineto{\pgfqpoint{0.832934in}{3.106362in}}%
\pgfpathlineto{\pgfqpoint{0.838554in}{3.112316in}}%
\pgfpathlineto{\pgfqpoint{0.847385in}{3.125641in}}%
\pgfpathlineto{\pgfqpoint{0.858357in}{3.141291in}}%
\pgfpathlineto{\pgfqpoint{0.863977in}{3.145790in}}%
\pgfpathlineto{\pgfqpoint{0.868526in}{3.146842in}}%
\pgfpathlineto{\pgfqpoint{0.872808in}{3.145581in}}%
\pgfpathlineto{\pgfqpoint{0.877625in}{3.141717in}}%
\pgfpathlineto{\pgfqpoint{0.884047in}{3.133416in}}%
\pgfpathlineto{\pgfqpoint{0.903315in}{3.106469in}}%
\pgfpathlineto{\pgfqpoint{0.908400in}{3.103796in}}%
\pgfpathlineto{\pgfqpoint{0.912681in}{3.103899in}}%
\pgfpathlineto{\pgfqpoint{0.917231in}{3.106362in}}%
\pgfpathlineto{\pgfqpoint{0.922850in}{3.112316in}}%
\pgfpathlineto{\pgfqpoint{0.931681in}{3.125641in}}%
\pgfpathlineto{\pgfqpoint{0.942653in}{3.141291in}}%
\pgfpathlineto{\pgfqpoint{0.948273in}{3.145790in}}%
\pgfpathlineto{\pgfqpoint{0.952822in}{3.146842in}}%
\pgfpathlineto{\pgfqpoint{0.957104in}{3.145581in}}%
\pgfpathlineto{\pgfqpoint{0.961921in}{3.141717in}}%
\pgfpathlineto{\pgfqpoint{0.968344in}{3.133416in}}%
\pgfpathlineto{\pgfqpoint{0.987611in}{3.106469in}}%
\pgfpathlineto{\pgfqpoint{0.992696in}{3.103796in}}%
\pgfpathlineto{\pgfqpoint{0.996978in}{3.103899in}}%
\pgfpathlineto{\pgfqpoint{1.001527in}{3.106362in}}%
\pgfpathlineto{\pgfqpoint{1.007147in}{3.112316in}}%
\pgfpathlineto{\pgfqpoint{1.015978in}{3.125641in}}%
\pgfpathlineto{\pgfqpoint{1.026950in}{3.141291in}}%
\pgfpathlineto{\pgfqpoint{1.032570in}{3.145790in}}%
\pgfpathlineto{\pgfqpoint{1.037119in}{3.146842in}}%
\pgfpathlineto{\pgfqpoint{1.041401in}{3.145581in}}%
\pgfpathlineto{\pgfqpoint{1.046218in}{3.141717in}}%
\pgfpathlineto{\pgfqpoint{1.052640in}{3.133416in}}%
\pgfpathlineto{\pgfqpoint{1.071908in}{3.106469in}}%
\pgfpathlineto{\pgfqpoint{1.076992in}{3.103796in}}%
\pgfpathlineto{\pgfqpoint{1.081274in}{3.103899in}}%
\pgfpathlineto{\pgfqpoint{1.085824in}{3.106362in}}%
\pgfpathlineto{\pgfqpoint{1.091443in}{3.112316in}}%
\pgfpathlineto{\pgfqpoint{1.100274in}{3.125641in}}%
\pgfpathlineto{\pgfqpoint{1.111246in}{3.141291in}}%
\pgfpathlineto{\pgfqpoint{1.116866in}{3.145790in}}%
\pgfpathlineto{\pgfqpoint{1.121415in}{3.146842in}}%
\pgfpathlineto{\pgfqpoint{1.125697in}{3.145581in}}%
\pgfpathlineto{\pgfqpoint{1.130514in}{3.141717in}}%
\pgfpathlineto{\pgfqpoint{1.136937in}{3.133416in}}%
\pgfpathlineto{\pgfqpoint{1.156204in}{3.106469in}}%
\pgfpathlineto{\pgfqpoint{1.161289in}{3.103796in}}%
\pgfpathlineto{\pgfqpoint{1.165571in}{3.103899in}}%
\pgfpathlineto{\pgfqpoint{1.170120in}{3.106362in}}%
\pgfpathlineto{\pgfqpoint{1.175740in}{3.112316in}}%
\pgfpathlineto{\pgfqpoint{1.184571in}{3.125641in}}%
\pgfpathlineto{\pgfqpoint{1.195543in}{3.141291in}}%
\pgfpathlineto{\pgfqpoint{1.201163in}{3.145790in}}%
\pgfpathlineto{\pgfqpoint{1.205712in}{3.146842in}}%
\pgfpathlineto{\pgfqpoint{1.209994in}{3.145581in}}%
\pgfpathlineto{\pgfqpoint{1.214811in}{3.141717in}}%
\pgfpathlineto{\pgfqpoint{1.221233in}{3.133416in}}%
\pgfpathlineto{\pgfqpoint{1.240501in}{3.106469in}}%
\pgfpathlineto{\pgfqpoint{1.245585in}{3.103796in}}%
\pgfpathlineto{\pgfqpoint{1.249867in}{3.103899in}}%
\pgfpathlineto{\pgfqpoint{1.254416in}{3.106362in}}%
\pgfpathlineto{\pgfqpoint{1.260036in}{3.112316in}}%
\pgfpathlineto{\pgfqpoint{1.268867in}{3.125641in}}%
\pgfpathlineto{\pgfqpoint{1.279839in}{3.141291in}}%
\pgfpathlineto{\pgfqpoint{1.285459in}{3.145790in}}%
\pgfpathlineto{\pgfqpoint{1.290008in}{3.146842in}}%
\pgfpathlineto{\pgfqpoint{1.294290in}{3.145581in}}%
\pgfpathlineto{\pgfqpoint{1.299107in}{3.141717in}}%
\pgfpathlineto{\pgfqpoint{1.305530in}{3.133416in}}%
\pgfpathlineto{\pgfqpoint{1.324797in}{3.106469in}}%
\pgfpathlineto{\pgfqpoint{1.329882in}{3.103796in}}%
\pgfpathlineto{\pgfqpoint{1.334164in}{3.103899in}}%
\pgfpathlineto{\pgfqpoint{1.338713in}{3.106362in}}%
\pgfpathlineto{\pgfqpoint{1.344333in}{3.112316in}}%
\pgfpathlineto{\pgfqpoint{1.353164in}{3.125641in}}%
\pgfpathlineto{\pgfqpoint{1.364136in}{3.141291in}}%
\pgfpathlineto{\pgfqpoint{1.369755in}{3.145790in}}%
\pgfpathlineto{\pgfqpoint{1.374305in}{3.146842in}}%
\pgfpathlineto{\pgfqpoint{1.378587in}{3.145581in}}%
\pgfpathlineto{\pgfqpoint{1.383403in}{3.141717in}}%
\pgfpathlineto{\pgfqpoint{1.389826in}{3.133416in}}%
\pgfpathlineto{\pgfqpoint{1.409094in}{3.106469in}}%
\pgfpathlineto{\pgfqpoint{1.414178in}{3.103796in}}%
\pgfpathlineto{\pgfqpoint{1.418460in}{3.103899in}}%
\pgfpathlineto{\pgfqpoint{1.423009in}{3.106362in}}%
\pgfpathlineto{\pgfqpoint{1.428629in}{3.112316in}}%
\pgfpathlineto{\pgfqpoint{1.437460in}{3.125641in}}%
\pgfpathlineto{\pgfqpoint{1.448432in}{3.141291in}}%
\pgfpathlineto{\pgfqpoint{1.454052in}{3.145790in}}%
\pgfpathlineto{\pgfqpoint{1.458601in}{3.146842in}}%
\pgfpathlineto{\pgfqpoint{1.462883in}{3.145581in}}%
\pgfpathlineto{\pgfqpoint{1.467700in}{3.141717in}}%
\pgfpathlineto{\pgfqpoint{1.474122in}{3.133416in}}%
\pgfpathlineto{\pgfqpoint{1.493390in}{3.106469in}}%
\pgfpathlineto{\pgfqpoint{1.498475in}{3.103796in}}%
\pgfpathlineto{\pgfqpoint{1.502757in}{3.103899in}}%
\pgfpathlineto{\pgfqpoint{1.507306in}{3.106362in}}%
\pgfpathlineto{\pgfqpoint{1.512926in}{3.112316in}}%
\pgfpathlineto{\pgfqpoint{1.521757in}{3.125641in}}%
\pgfpathlineto{\pgfqpoint{1.532729in}{3.141291in}}%
\pgfpathlineto{\pgfqpoint{1.538348in}{3.145790in}}%
\pgfpathlineto{\pgfqpoint{1.542898in}{3.146842in}}%
\pgfpathlineto{\pgfqpoint{1.547179in}{3.145581in}}%
\pgfpathlineto{\pgfqpoint{1.551996in}{3.141717in}}%
\pgfpathlineto{\pgfqpoint{1.558419in}{3.133416in}}%
\pgfpathlineto{\pgfqpoint{1.577687in}{3.106469in}}%
\pgfpathlineto{\pgfqpoint{1.582771in}{3.103796in}}%
\pgfpathlineto{\pgfqpoint{1.587053in}{3.103899in}}%
\pgfpathlineto{\pgfqpoint{1.591602in}{3.106362in}}%
\pgfpathlineto{\pgfqpoint{1.597222in}{3.112316in}}%
\pgfpathlineto{\pgfqpoint{1.606053in}{3.125641in}}%
\pgfpathlineto{\pgfqpoint{1.617025in}{3.141291in}}%
\pgfpathlineto{\pgfqpoint{1.622645in}{3.145790in}}%
\pgfpathlineto{\pgfqpoint{1.627194in}{3.146842in}}%
\pgfpathlineto{\pgfqpoint{1.631476in}{3.145581in}}%
\pgfpathlineto{\pgfqpoint{1.636293in}{3.141717in}}%
\pgfpathlineto{\pgfqpoint{1.642715in}{3.133416in}}%
\pgfpathlineto{\pgfqpoint{1.661983in}{3.106469in}}%
\pgfpathlineto{\pgfqpoint{1.667068in}{3.103796in}}%
\pgfpathlineto{\pgfqpoint{1.671349in}{3.103899in}}%
\pgfpathlineto{\pgfqpoint{1.675899in}{3.106362in}}%
\pgfpathlineto{\pgfqpoint{1.681519in}{3.112316in}}%
\pgfpathlineto{\pgfqpoint{1.690350in}{3.125641in}}%
\pgfpathlineto{\pgfqpoint{1.701322in}{3.141291in}}%
\pgfpathlineto{\pgfqpoint{1.706941in}{3.145790in}}%
\pgfpathlineto{\pgfqpoint{1.711491in}{3.146842in}}%
\pgfpathlineto{\pgfqpoint{1.715772in}{3.145581in}}%
\pgfpathlineto{\pgfqpoint{1.720589in}{3.141717in}}%
\pgfpathlineto{\pgfqpoint{1.727012in}{3.133416in}}%
\pgfpathlineto{\pgfqpoint{1.746280in}{3.106469in}}%
\pgfpathlineto{\pgfqpoint{1.751364in}{3.103796in}}%
\pgfpathlineto{\pgfqpoint{1.755646in}{3.103899in}}%
\pgfpathlineto{\pgfqpoint{1.760195in}{3.106362in}}%
\pgfpathlineto{\pgfqpoint{1.765815in}{3.112316in}}%
\pgfpathlineto{\pgfqpoint{1.774646in}{3.125641in}}%
\pgfpathlineto{\pgfqpoint{1.785618in}{3.141291in}}%
\pgfpathlineto{\pgfqpoint{1.791238in}{3.145790in}}%
\pgfpathlineto{\pgfqpoint{1.795787in}{3.146842in}}%
\pgfpathlineto{\pgfqpoint{1.800069in}{3.145581in}}%
\pgfpathlineto{\pgfqpoint{1.804886in}{3.141717in}}%
\pgfpathlineto{\pgfqpoint{1.811308in}{3.133416in}}%
\pgfpathlineto{\pgfqpoint{1.830576in}{3.106469in}}%
\pgfpathlineto{\pgfqpoint{1.835661in}{3.103796in}}%
\pgfpathlineto{\pgfqpoint{1.839942in}{3.103899in}}%
\pgfpathlineto{\pgfqpoint{1.844492in}{3.106362in}}%
\pgfpathlineto{\pgfqpoint{1.850111in}{3.112316in}}%
\pgfpathlineto{\pgfqpoint{1.858943in}{3.125641in}}%
\pgfpathlineto{\pgfqpoint{1.869914in}{3.141291in}}%
\pgfpathlineto{\pgfqpoint{1.875534in}{3.145790in}}%
\pgfpathlineto{\pgfqpoint{1.880084in}{3.146842in}}%
\pgfpathlineto{\pgfqpoint{1.884365in}{3.145581in}}%
\pgfpathlineto{\pgfqpoint{1.889182in}{3.141717in}}%
\pgfpathlineto{\pgfqpoint{1.895605in}{3.133416in}}%
\pgfpathlineto{\pgfqpoint{1.914873in}{3.106469in}}%
\pgfpathlineto{\pgfqpoint{1.919957in}{3.103796in}}%
\pgfpathlineto{\pgfqpoint{1.924239in}{3.103899in}}%
\pgfpathlineto{\pgfqpoint{1.928788in}{3.106362in}}%
\pgfpathlineto{\pgfqpoint{1.934408in}{3.112316in}}%
\pgfpathlineto{\pgfqpoint{1.943239in}{3.125641in}}%
\pgfpathlineto{\pgfqpoint{1.954211in}{3.141291in}}%
\pgfpathlineto{\pgfqpoint{1.959831in}{3.145790in}}%
\pgfpathlineto{\pgfqpoint{1.964380in}{3.146842in}}%
\pgfpathlineto{\pgfqpoint{1.968662in}{3.145581in}}%
\pgfpathlineto{\pgfqpoint{1.973479in}{3.141717in}}%
\pgfpathlineto{\pgfqpoint{1.979901in}{3.133416in}}%
\pgfpathlineto{\pgfqpoint{1.999169in}{3.106469in}}%
\pgfpathlineto{\pgfqpoint{2.004254in}{3.103796in}}%
\pgfpathlineto{\pgfqpoint{2.008535in}{3.103899in}}%
\pgfpathlineto{\pgfqpoint{2.013085in}{3.106362in}}%
\pgfpathlineto{\pgfqpoint{2.018704in}{3.112316in}}%
\pgfpathlineto{\pgfqpoint{2.027535in}{3.125641in}}%
\pgfpathlineto{\pgfqpoint{2.038507in}{3.141291in}}%
\pgfpathlineto{\pgfqpoint{2.044127in}{3.145790in}}%
\pgfpathlineto{\pgfqpoint{2.048676in}{3.146842in}}%
\pgfpathlineto{\pgfqpoint{2.052958in}{3.145581in}}%
\pgfpathlineto{\pgfqpoint{2.057775in}{3.141717in}}%
\pgfpathlineto{\pgfqpoint{2.064198in}{3.133416in}}%
\pgfpathlineto{\pgfqpoint{2.083466in}{3.106469in}}%
\pgfpathlineto{\pgfqpoint{2.088550in}{3.103796in}}%
\pgfpathlineto{\pgfqpoint{2.092832in}{3.103899in}}%
\pgfpathlineto{\pgfqpoint{2.097381in}{3.106362in}}%
\pgfpathlineto{\pgfqpoint{2.103001in}{3.112316in}}%
\pgfpathlineto{\pgfqpoint{2.111832in}{3.125641in}}%
\pgfpathlineto{\pgfqpoint{2.122804in}{3.141291in}}%
\pgfpathlineto{\pgfqpoint{2.128424in}{3.145790in}}%
\pgfpathlineto{\pgfqpoint{2.132973in}{3.146842in}}%
\pgfpathlineto{\pgfqpoint{2.137255in}{3.145581in}}%
\pgfpathlineto{\pgfqpoint{2.142072in}{3.141717in}}%
\pgfpathlineto{\pgfqpoint{2.148494in}{3.133416in}}%
\pgfpathlineto{\pgfqpoint{2.167762in}{3.106469in}}%
\pgfpathlineto{\pgfqpoint{2.172847in}{3.103796in}}%
\pgfpathlineto{\pgfqpoint{2.177128in}{3.103899in}}%
\pgfpathlineto{\pgfqpoint{2.181678in}{3.106362in}}%
\pgfpathlineto{\pgfqpoint{2.187297in}{3.112316in}}%
\pgfpathlineto{\pgfqpoint{2.196128in}{3.125641in}}%
\pgfpathlineto{\pgfqpoint{2.207100in}{3.141291in}}%
\pgfpathlineto{\pgfqpoint{2.212720in}{3.145790in}}%
\pgfpathlineto{\pgfqpoint{2.217269in}{3.146842in}}%
\pgfpathlineto{\pgfqpoint{2.221551in}{3.145581in}}%
\pgfpathlineto{\pgfqpoint{2.226368in}{3.141717in}}%
\pgfpathlineto{\pgfqpoint{2.232791in}{3.133416in}}%
\pgfpathlineto{\pgfqpoint{2.252058in}{3.106469in}}%
\pgfpathlineto{\pgfqpoint{2.257143in}{3.103796in}}%
\pgfpathlineto{\pgfqpoint{2.261425in}{3.103899in}}%
\pgfpathlineto{\pgfqpoint{2.265974in}{3.106362in}}%
\pgfpathlineto{\pgfqpoint{2.271594in}{3.112316in}}%
\pgfpathlineto{\pgfqpoint{2.280425in}{3.125641in}}%
\pgfpathlineto{\pgfqpoint{2.291397in}{3.141291in}}%
\pgfpathlineto{\pgfqpoint{2.297017in}{3.145790in}}%
\pgfpathlineto{\pgfqpoint{2.301566in}{3.146842in}}%
\pgfpathlineto{\pgfqpoint{2.305848in}{3.145581in}}%
\pgfpathlineto{\pgfqpoint{2.310665in}{3.141717in}}%
\pgfpathlineto{\pgfqpoint{2.317087in}{3.133416in}}%
\pgfpathlineto{\pgfqpoint{2.336355in}{3.106469in}}%
\pgfpathlineto{\pgfqpoint{2.341439in}{3.103796in}}%
\pgfpathlineto{\pgfqpoint{2.345721in}{3.103899in}}%
\pgfpathlineto{\pgfqpoint{2.350271in}{3.106362in}}%
\pgfpathlineto{\pgfqpoint{2.355890in}{3.112316in}}%
\pgfpathlineto{\pgfqpoint{2.364721in}{3.125641in}}%
\pgfpathlineto{\pgfqpoint{2.375693in}{3.141291in}}%
\pgfpathlineto{\pgfqpoint{2.381313in}{3.145790in}}%
\pgfpathlineto{\pgfqpoint{2.385862in}{3.146842in}}%
\pgfpathlineto{\pgfqpoint{2.390144in}{3.145581in}}%
\pgfpathlineto{\pgfqpoint{2.394961in}{3.141717in}}%
\pgfpathlineto{\pgfqpoint{2.401384in}{3.133416in}}%
\pgfpathlineto{\pgfqpoint{2.420651in}{3.106469in}}%
\pgfpathlineto{\pgfqpoint{2.425736in}{3.103796in}}%
\pgfpathlineto{\pgfqpoint{2.430018in}{3.103899in}}%
\pgfpathlineto{\pgfqpoint{2.434567in}{3.106362in}}%
\pgfpathlineto{\pgfqpoint{2.440187in}{3.112316in}}%
\pgfpathlineto{\pgfqpoint{2.449018in}{3.125641in}}%
\pgfpathlineto{\pgfqpoint{2.459990in}{3.141291in}}%
\pgfpathlineto{\pgfqpoint{2.465609in}{3.145790in}}%
\pgfpathlineto{\pgfqpoint{2.470159in}{3.146842in}}%
\pgfpathlineto{\pgfqpoint{2.474441in}{3.145581in}}%
\pgfpathlineto{\pgfqpoint{2.479257in}{3.141717in}}%
\pgfpathlineto{\pgfqpoint{2.485680in}{3.133416in}}%
\pgfpathlineto{\pgfqpoint{2.504948in}{3.106469in}}%
\pgfpathlineto{\pgfqpoint{2.510032in}{3.103796in}}%
\pgfpathlineto{\pgfqpoint{2.514314in}{3.103899in}}%
\pgfpathlineto{\pgfqpoint{2.518863in}{3.106362in}}%
\pgfpathlineto{\pgfqpoint{2.524483in}{3.112316in}}%
\pgfpathlineto{\pgfqpoint{2.533314in}{3.125641in}}%
\pgfpathlineto{\pgfqpoint{2.544286in}{3.141291in}}%
\pgfpathlineto{\pgfqpoint{2.549906in}{3.145790in}}%
\pgfpathlineto{\pgfqpoint{2.554455in}{3.146842in}}%
\pgfpathlineto{\pgfqpoint{2.558737in}{3.145581in}}%
\pgfpathlineto{\pgfqpoint{2.563554in}{3.141717in}}%
\pgfpathlineto{\pgfqpoint{2.569977in}{3.133416in}}%
\pgfpathlineto{\pgfqpoint{2.589244in}{3.106469in}}%
\pgfpathlineto{\pgfqpoint{2.594329in}{3.103796in}}%
\pgfpathlineto{\pgfqpoint{2.598611in}{3.103899in}}%
\pgfpathlineto{\pgfqpoint{2.603160in}{3.106362in}}%
\pgfpathlineto{\pgfqpoint{2.608780in}{3.112316in}}%
\pgfpathlineto{\pgfqpoint{2.617611in}{3.125641in}}%
\pgfpathlineto{\pgfqpoint{2.628583in}{3.141291in}}%
\pgfpathlineto{\pgfqpoint{2.634202in}{3.145790in}}%
\pgfpathlineto{\pgfqpoint{2.638752in}{3.146842in}}%
\pgfpathlineto{\pgfqpoint{2.643033in}{3.145581in}}%
\pgfpathlineto{\pgfqpoint{2.647850in}{3.141717in}}%
\pgfpathlineto{\pgfqpoint{2.654273in}{3.133416in}}%
\pgfpathlineto{\pgfqpoint{2.673541in}{3.106469in}}%
\pgfpathlineto{\pgfqpoint{2.678625in}{3.103796in}}%
\pgfpathlineto{\pgfqpoint{2.682907in}{3.103899in}}%
\pgfpathlineto{\pgfqpoint{2.687456in}{3.106362in}}%
\pgfpathlineto{\pgfqpoint{2.693076in}{3.112316in}}%
\pgfpathlineto{\pgfqpoint{2.701907in}{3.125641in}}%
\pgfpathlineto{\pgfqpoint{2.712879in}{3.141291in}}%
\pgfpathlineto{\pgfqpoint{2.718499in}{3.145790in}}%
\pgfpathlineto{\pgfqpoint{2.723048in}{3.146842in}}%
\pgfpathlineto{\pgfqpoint{2.727330in}{3.145581in}}%
\pgfpathlineto{\pgfqpoint{2.732147in}{3.141717in}}%
\pgfpathlineto{\pgfqpoint{2.738569in}{3.133416in}}%
\pgfpathlineto{\pgfqpoint{2.757837in}{3.106469in}}%
\pgfpathlineto{\pgfqpoint{2.762922in}{3.103796in}}%
\pgfpathlineto{\pgfqpoint{2.767203in}{3.103899in}}%
\pgfpathlineto{\pgfqpoint{2.771753in}{3.106362in}}%
\pgfpathlineto{\pgfqpoint{2.777373in}{3.112316in}}%
\pgfpathlineto{\pgfqpoint{2.786204in}{3.125641in}}%
\pgfpathlineto{\pgfqpoint{2.797176in}{3.141291in}}%
\pgfpathlineto{\pgfqpoint{2.802795in}{3.145790in}}%
\pgfpathlineto{\pgfqpoint{2.807345in}{3.146842in}}%
\pgfpathlineto{\pgfqpoint{2.811626in}{3.145581in}}%
\pgfpathlineto{\pgfqpoint{2.816443in}{3.141717in}}%
\pgfpathlineto{\pgfqpoint{2.822866in}{3.133416in}}%
\pgfpathlineto{\pgfqpoint{2.842134in}{3.106469in}}%
\pgfpathlineto{\pgfqpoint{2.847218in}{3.103796in}}%
\pgfpathlineto{\pgfqpoint{2.851500in}{3.103899in}}%
\pgfpathlineto{\pgfqpoint{2.856049in}{3.106362in}}%
\pgfpathlineto{\pgfqpoint{2.861669in}{3.112316in}}%
\pgfpathlineto{\pgfqpoint{2.870500in}{3.125641in}}%
\pgfpathlineto{\pgfqpoint{2.881472in}{3.141291in}}%
\pgfpathlineto{\pgfqpoint{2.887092in}{3.145790in}}%
\pgfpathlineto{\pgfqpoint{2.891641in}{3.146842in}}%
\pgfpathlineto{\pgfqpoint{2.895923in}{3.145581in}}%
\pgfpathlineto{\pgfqpoint{2.900740in}{3.141717in}}%
\pgfpathlineto{\pgfqpoint{2.907162in}{3.133416in}}%
\pgfpathlineto{\pgfqpoint{2.926430in}{3.106469in}}%
\pgfpathlineto{\pgfqpoint{2.931515in}{3.103796in}}%
\pgfpathlineto{\pgfqpoint{2.935796in}{3.103899in}}%
\pgfpathlineto{\pgfqpoint{2.940346in}{3.106362in}}%
\pgfpathlineto{\pgfqpoint{2.945966in}{3.112316in}}%
\pgfpathlineto{\pgfqpoint{2.954797in}{3.125641in}}%
\pgfpathlineto{\pgfqpoint{2.965768in}{3.141291in}}%
\pgfpathlineto{\pgfqpoint{2.971388in}{3.145790in}}%
\pgfpathlineto{\pgfqpoint{2.975938in}{3.146842in}}%
\pgfpathlineto{\pgfqpoint{2.980219in}{3.145581in}}%
\pgfpathlineto{\pgfqpoint{2.985036in}{3.141717in}}%
\pgfpathlineto{\pgfqpoint{2.991459in}{3.133416in}}%
\pgfpathlineto{\pgfqpoint{3.010727in}{3.106469in}}%
\pgfpathlineto{\pgfqpoint{3.015811in}{3.103796in}}%
\pgfpathlineto{\pgfqpoint{3.020093in}{3.103899in}}%
\pgfpathlineto{\pgfqpoint{3.024642in}{3.106362in}}%
\pgfpathlineto{\pgfqpoint{3.030262in}{3.112316in}}%
\pgfpathlineto{\pgfqpoint{3.039093in}{3.125641in}}%
\pgfpathlineto{\pgfqpoint{3.050065in}{3.141291in}}%
\pgfpathlineto{\pgfqpoint{3.055685in}{3.145790in}}%
\pgfpathlineto{\pgfqpoint{3.060234in}{3.146842in}}%
\pgfpathlineto{\pgfqpoint{3.064516in}{3.145581in}}%
\pgfpathlineto{\pgfqpoint{3.069333in}{3.141717in}}%
\pgfpathlineto{\pgfqpoint{3.075755in}{3.133416in}}%
\pgfpathlineto{\pgfqpoint{3.095023in}{3.106469in}}%
\pgfpathlineto{\pgfqpoint{3.100108in}{3.103796in}}%
\pgfpathlineto{\pgfqpoint{3.104389in}{3.103899in}}%
\pgfpathlineto{\pgfqpoint{3.108939in}{3.106362in}}%
\pgfpathlineto{\pgfqpoint{3.114558in}{3.112316in}}%
\pgfpathlineto{\pgfqpoint{3.123389in}{3.125641in}}%
\pgfpathlineto{\pgfqpoint{3.134361in}{3.141291in}}%
\pgfpathlineto{\pgfqpoint{3.139981in}{3.145790in}}%
\pgfpathlineto{\pgfqpoint{3.144531in}{3.146842in}}%
\pgfpathlineto{\pgfqpoint{3.148812in}{3.145581in}}%
\pgfpathlineto{\pgfqpoint{3.153629in}{3.141717in}}%
\pgfpathlineto{\pgfqpoint{3.160052in}{3.133416in}}%
\pgfpathlineto{\pgfqpoint{3.179320in}{3.106469in}}%
\pgfpathlineto{\pgfqpoint{3.184404in}{3.103796in}}%
\pgfpathlineto{\pgfqpoint{3.188686in}{3.103899in}}%
\pgfpathlineto{\pgfqpoint{3.193235in}{3.106362in}}%
\pgfpathlineto{\pgfqpoint{3.198855in}{3.112316in}}%
\pgfpathlineto{\pgfqpoint{3.207686in}{3.125641in}}%
\pgfpathlineto{\pgfqpoint{3.218658in}{3.141291in}}%
\pgfpathlineto{\pgfqpoint{3.224278in}{3.145790in}}%
\pgfpathlineto{\pgfqpoint{3.228827in}{3.146842in}}%
\pgfpathlineto{\pgfqpoint{3.233109in}{3.145581in}}%
\pgfpathlineto{\pgfqpoint{3.237926in}{3.141717in}}%
\pgfpathlineto{\pgfqpoint{3.244348in}{3.133416in}}%
\pgfpathlineto{\pgfqpoint{3.263616in}{3.106469in}}%
\pgfpathlineto{\pgfqpoint{3.268701in}{3.103796in}}%
\pgfpathlineto{\pgfqpoint{3.272982in}{3.103899in}}%
\pgfpathlineto{\pgfqpoint{3.277532in}{3.106362in}}%
\pgfpathlineto{\pgfqpoint{3.283151in}{3.112316in}}%
\pgfpathlineto{\pgfqpoint{3.291982in}{3.125641in}}%
\pgfpathlineto{\pgfqpoint{3.302954in}{3.141291in}}%
\pgfpathlineto{\pgfqpoint{3.308574in}{3.145790in}}%
\pgfpathlineto{\pgfqpoint{3.313123in}{3.146842in}}%
\pgfpathlineto{\pgfqpoint{3.317405in}{3.145581in}}%
\pgfpathlineto{\pgfqpoint{3.322222in}{3.141717in}}%
\pgfpathlineto{\pgfqpoint{3.328645in}{3.133416in}}%
\pgfpathlineto{\pgfqpoint{3.347912in}{3.106469in}}%
\pgfpathlineto{\pgfqpoint{3.352997in}{3.103796in}}%
\pgfpathlineto{\pgfqpoint{3.357279in}{3.103899in}}%
\pgfpathlineto{\pgfqpoint{3.361828in}{3.106362in}}%
\pgfpathlineto{\pgfqpoint{3.367448in}{3.112316in}}%
\pgfpathlineto{\pgfqpoint{3.376279in}{3.125641in}}%
\pgfpathlineto{\pgfqpoint{3.387251in}{3.141291in}}%
\pgfpathlineto{\pgfqpoint{3.392871in}{3.145790in}}%
\pgfpathlineto{\pgfqpoint{3.397420in}{3.146842in}}%
\pgfpathlineto{\pgfqpoint{3.401702in}{3.145581in}}%
\pgfpathlineto{\pgfqpoint{3.406519in}{3.141717in}}%
\pgfpathlineto{\pgfqpoint{3.412941in}{3.133416in}}%
\pgfpathlineto{\pgfqpoint{3.432209in}{3.106469in}}%
\pgfpathlineto{\pgfqpoint{3.437293in}{3.103796in}}%
\pgfpathlineto{\pgfqpoint{3.441575in}{3.103899in}}%
\pgfpathlineto{\pgfqpoint{3.446125in}{3.106362in}}%
\pgfpathlineto{\pgfqpoint{3.451744in}{3.112316in}}%
\pgfpathlineto{\pgfqpoint{3.460575in}{3.125641in}}%
\pgfpathlineto{\pgfqpoint{3.471547in}{3.141291in}}%
\pgfpathlineto{\pgfqpoint{3.477167in}{3.145790in}}%
\pgfpathlineto{\pgfqpoint{3.481716in}{3.146842in}}%
\pgfpathlineto{\pgfqpoint{3.485998in}{3.145581in}}%
\pgfpathlineto{\pgfqpoint{3.490815in}{3.141717in}}%
\pgfpathlineto{\pgfqpoint{3.497238in}{3.133416in}}%
\pgfpathlineto{\pgfqpoint{3.516505in}{3.106469in}}%
\pgfpathlineto{\pgfqpoint{3.521590in}{3.103796in}}%
\pgfpathlineto{\pgfqpoint{3.525872in}{3.103899in}}%
\pgfpathlineto{\pgfqpoint{3.530421in}{3.106362in}}%
\pgfpathlineto{\pgfqpoint{3.536041in}{3.112316in}}%
\pgfpathlineto{\pgfqpoint{3.544872in}{3.125641in}}%
\pgfpathlineto{\pgfqpoint{3.555844in}{3.141291in}}%
\pgfpathlineto{\pgfqpoint{3.561464in}{3.145790in}}%
\pgfpathlineto{\pgfqpoint{3.566013in}{3.146842in}}%
\pgfpathlineto{\pgfqpoint{3.570295in}{3.145581in}}%
\pgfpathlineto{\pgfqpoint{3.575112in}{3.141717in}}%
\pgfpathlineto{\pgfqpoint{3.581534in}{3.133416in}}%
\pgfpathlineto{\pgfqpoint{3.600802in}{3.106469in}}%
\pgfpathlineto{\pgfqpoint{3.605886in}{3.103796in}}%
\pgfpathlineto{\pgfqpoint{3.610168in}{3.103899in}}%
\pgfpathlineto{\pgfqpoint{3.614717in}{3.106362in}}%
\pgfpathlineto{\pgfqpoint{3.620337in}{3.112316in}}%
\pgfpathlineto{\pgfqpoint{3.629168in}{3.125641in}}%
\pgfpathlineto{\pgfqpoint{3.640140in}{3.141291in}}%
\pgfpathlineto{\pgfqpoint{3.645760in}{3.145790in}}%
\pgfpathlineto{\pgfqpoint{3.650309in}{3.146842in}}%
\pgfpathlineto{\pgfqpoint{3.654591in}{3.145581in}}%
\pgfpathlineto{\pgfqpoint{3.659408in}{3.141717in}}%
\pgfpathlineto{\pgfqpoint{3.665831in}{3.133416in}}%
\pgfpathlineto{\pgfqpoint{3.685098in}{3.106469in}}%
\pgfpathlineto{\pgfqpoint{3.690183in}{3.103796in}}%
\pgfpathlineto{\pgfqpoint{3.694465in}{3.103899in}}%
\pgfpathlineto{\pgfqpoint{3.699014in}{3.106362in}}%
\pgfpathlineto{\pgfqpoint{3.704634in}{3.112316in}}%
\pgfpathlineto{\pgfqpoint{3.713465in}{3.125641in}}%
\pgfpathlineto{\pgfqpoint{3.724437in}{3.141291in}}%
\pgfpathlineto{\pgfqpoint{3.730056in}{3.145790in}}%
\pgfpathlineto{\pgfqpoint{3.734606in}{3.146842in}}%
\pgfpathlineto{\pgfqpoint{3.738887in}{3.145581in}}%
\pgfpathlineto{\pgfqpoint{3.743704in}{3.141717in}}%
\pgfpathlineto{\pgfqpoint{3.750127in}{3.133416in}}%
\pgfpathlineto{\pgfqpoint{3.769395in}{3.106469in}}%
\pgfpathlineto{\pgfqpoint{3.774479in}{3.103796in}}%
\pgfpathlineto{\pgfqpoint{3.778761in}{3.103899in}}%
\pgfpathlineto{\pgfqpoint{3.783310in}{3.106362in}}%
\pgfpathlineto{\pgfqpoint{3.788930in}{3.112316in}}%
\pgfpathlineto{\pgfqpoint{3.797761in}{3.125641in}}%
\pgfpathlineto{\pgfqpoint{3.808733in}{3.141291in}}%
\pgfpathlineto{\pgfqpoint{3.814353in}{3.145790in}}%
\pgfpathlineto{\pgfqpoint{3.818902in}{3.146842in}}%
\pgfpathlineto{\pgfqpoint{3.823184in}{3.145581in}}%
\pgfpathlineto{\pgfqpoint{3.828001in}{3.141717in}}%
\pgfpathlineto{\pgfqpoint{3.834423in}{3.133416in}}%
\pgfpathlineto{\pgfqpoint{3.853691in}{3.106469in}}%
\pgfpathlineto{\pgfqpoint{3.858776in}{3.103796in}}%
\pgfpathlineto{\pgfqpoint{3.863058in}{3.103899in}}%
\pgfpathlineto{\pgfqpoint{3.867607in}{3.106362in}}%
\pgfpathlineto{\pgfqpoint{3.873227in}{3.112316in}}%
\pgfpathlineto{\pgfqpoint{3.882058in}{3.125641in}}%
\pgfpathlineto{\pgfqpoint{3.893030in}{3.141291in}}%
\pgfpathlineto{\pgfqpoint{3.898649in}{3.145790in}}%
\pgfpathlineto{\pgfqpoint{3.903199in}{3.146842in}}%
\pgfpathlineto{\pgfqpoint{3.907480in}{3.145581in}}%
\pgfpathlineto{\pgfqpoint{3.912297in}{3.141717in}}%
\pgfpathlineto{\pgfqpoint{3.918720in}{3.133416in}}%
\pgfpathlineto{\pgfqpoint{3.937988in}{3.106469in}}%
\pgfpathlineto{\pgfqpoint{3.943072in}{3.103796in}}%
\pgfpathlineto{\pgfqpoint{3.947354in}{3.103899in}}%
\pgfpathlineto{\pgfqpoint{3.951903in}{3.106362in}}%
\pgfpathlineto{\pgfqpoint{3.957523in}{3.112316in}}%
\pgfpathlineto{\pgfqpoint{3.966354in}{3.125641in}}%
\pgfpathlineto{\pgfqpoint{3.977326in}{3.141291in}}%
\pgfpathlineto{\pgfqpoint{3.982946in}{3.145790in}}%
\pgfpathlineto{\pgfqpoint{3.987495in}{3.146842in}}%
\pgfpathlineto{\pgfqpoint{3.991777in}{3.145581in}}%
\pgfpathlineto{\pgfqpoint{3.996594in}{3.141717in}}%
\pgfpathlineto{\pgfqpoint{4.003016in}{3.133416in}}%
\pgfpathlineto{\pgfqpoint{4.022284in}{3.106469in}}%
\pgfpathlineto{\pgfqpoint{4.027369in}{3.103796in}}%
\pgfpathlineto{\pgfqpoint{4.031650in}{3.103899in}}%
\pgfpathlineto{\pgfqpoint{4.036200in}{3.106362in}}%
\pgfpathlineto{\pgfqpoint{4.041820in}{3.112316in}}%
\pgfpathlineto{\pgfqpoint{4.050651in}{3.125641in}}%
\pgfpathlineto{\pgfqpoint{4.061623in}{3.141291in}}%
\pgfpathlineto{\pgfqpoint{4.067242in}{3.145790in}}%
\pgfpathlineto{\pgfqpoint{4.071792in}{3.146842in}}%
\pgfpathlineto{\pgfqpoint{4.076073in}{3.145581in}}%
\pgfpathlineto{\pgfqpoint{4.080890in}{3.141717in}}%
\pgfpathlineto{\pgfqpoint{4.087313in}{3.133416in}}%
\pgfpathlineto{\pgfqpoint{4.106581in}{3.106469in}}%
\pgfpathlineto{\pgfqpoint{4.111665in}{3.103796in}}%
\pgfpathlineto{\pgfqpoint{4.115947in}{3.103899in}}%
\pgfpathlineto{\pgfqpoint{4.120496in}{3.106362in}}%
\pgfpathlineto{\pgfqpoint{4.126116in}{3.112316in}}%
\pgfpathlineto{\pgfqpoint{4.134947in}{3.125641in}}%
\pgfpathlineto{\pgfqpoint{4.145919in}{3.141291in}}%
\pgfpathlineto{\pgfqpoint{4.151539in}{3.145790in}}%
\pgfpathlineto{\pgfqpoint{4.156088in}{3.146842in}}%
\pgfpathlineto{\pgfqpoint{4.160370in}{3.145581in}}%
\pgfpathlineto{\pgfqpoint{4.165187in}{3.141717in}}%
\pgfpathlineto{\pgfqpoint{4.171609in}{3.133416in}}%
\pgfpathlineto{\pgfqpoint{4.190877in}{3.106469in}}%
\pgfpathlineto{\pgfqpoint{4.195962in}{3.103796in}}%
\pgfpathlineto{\pgfqpoint{4.200243in}{3.103899in}}%
\pgfpathlineto{\pgfqpoint{4.204793in}{3.106362in}}%
\pgfpathlineto{\pgfqpoint{4.210412in}{3.112316in}}%
\pgfpathlineto{\pgfqpoint{4.219244in}{3.125641in}}%
\pgfpathlineto{\pgfqpoint{4.230215in}{3.141291in}}%
\pgfpathlineto{\pgfqpoint{4.235835in}{3.145790in}}%
\pgfpathlineto{\pgfqpoint{4.240385in}{3.146842in}}%
\pgfpathlineto{\pgfqpoint{4.244666in}{3.145581in}}%
\pgfpathlineto{\pgfqpoint{4.249483in}{3.141717in}}%
\pgfpathlineto{\pgfqpoint{4.255906in}{3.133416in}}%
\pgfpathlineto{\pgfqpoint{4.275174in}{3.106469in}}%
\pgfpathlineto{\pgfqpoint{4.280258in}{3.103796in}}%
\pgfpathlineto{\pgfqpoint{4.284540in}{3.103899in}}%
\pgfpathlineto{\pgfqpoint{4.289089in}{3.106362in}}%
\pgfpathlineto{\pgfqpoint{4.294709in}{3.112316in}}%
\pgfpathlineto{\pgfqpoint{4.303540in}{3.125641in}}%
\pgfpathlineto{\pgfqpoint{4.314512in}{3.141291in}}%
\pgfpathlineto{\pgfqpoint{4.320132in}{3.145790in}}%
\pgfpathlineto{\pgfqpoint{4.324681in}{3.146842in}}%
\pgfpathlineto{\pgfqpoint{4.328963in}{3.145581in}}%
\pgfpathlineto{\pgfqpoint{4.333780in}{3.141717in}}%
\pgfpathlineto{\pgfqpoint{4.340202in}{3.133416in}}%
\pgfpathlineto{\pgfqpoint{4.359470in}{3.106469in}}%
\pgfpathlineto{\pgfqpoint{4.364555in}{3.103796in}}%
\pgfpathlineto{\pgfqpoint{4.368836in}{3.103899in}}%
\pgfpathlineto{\pgfqpoint{4.373386in}{3.106362in}}%
\pgfpathlineto{\pgfqpoint{4.379005in}{3.112316in}}%
\pgfpathlineto{\pgfqpoint{4.387836in}{3.125641in}}%
\pgfpathlineto{\pgfqpoint{4.398808in}{3.141291in}}%
\pgfpathlineto{\pgfqpoint{4.404428in}{3.145790in}}%
\pgfpathlineto{\pgfqpoint{4.408977in}{3.146842in}}%
\pgfpathlineto{\pgfqpoint{4.413259in}{3.145581in}}%
\pgfpathlineto{\pgfqpoint{4.418076in}{3.141717in}}%
\pgfpathlineto{\pgfqpoint{4.424499in}{3.133416in}}%
\pgfpathlineto{\pgfqpoint{4.443766in}{3.106469in}}%
\pgfpathlineto{\pgfqpoint{4.448851in}{3.103796in}}%
\pgfpathlineto{\pgfqpoint{4.453133in}{3.103899in}}%
\pgfpathlineto{\pgfqpoint{4.457682in}{3.106362in}}%
\pgfpathlineto{\pgfqpoint{4.463302in}{3.112316in}}%
\pgfpathlineto{\pgfqpoint{4.472133in}{3.125641in}}%
\pgfpathlineto{\pgfqpoint{4.483105in}{3.141291in}}%
\pgfpathlineto{\pgfqpoint{4.488725in}{3.145790in}}%
\pgfpathlineto{\pgfqpoint{4.493274in}{3.146842in}}%
\pgfpathlineto{\pgfqpoint{4.497556in}{3.145581in}}%
\pgfpathlineto{\pgfqpoint{4.502373in}{3.141717in}}%
\pgfpathlineto{\pgfqpoint{4.508795in}{3.133416in}}%
\pgfpathlineto{\pgfqpoint{4.528063in}{3.106469in}}%
\pgfpathlineto{\pgfqpoint{4.533148in}{3.103796in}}%
\pgfpathlineto{\pgfqpoint{4.537429in}{3.103899in}}%
\pgfpathlineto{\pgfqpoint{4.541979in}{3.106362in}}%
\pgfpathlineto{\pgfqpoint{4.547598in}{3.112316in}}%
\pgfpathlineto{\pgfqpoint{4.556429in}{3.125641in}}%
\pgfpathlineto{\pgfqpoint{4.567401in}{3.141291in}}%
\pgfpathlineto{\pgfqpoint{4.573021in}{3.145790in}}%
\pgfpathlineto{\pgfqpoint{4.577570in}{3.146842in}}%
\pgfpathlineto{\pgfqpoint{4.581852in}{3.145581in}}%
\pgfpathlineto{\pgfqpoint{4.586669in}{3.141717in}}%
\pgfpathlineto{\pgfqpoint{4.593092in}{3.133416in}}%
\pgfpathlineto{\pgfqpoint{4.612359in}{3.106469in}}%
\pgfpathlineto{\pgfqpoint{4.617444in}{3.103796in}}%
\pgfpathlineto{\pgfqpoint{4.621726in}{3.103899in}}%
\pgfpathlineto{\pgfqpoint{4.626275in}{3.106362in}}%
\pgfpathlineto{\pgfqpoint{4.631895in}{3.112316in}}%
\pgfpathlineto{\pgfqpoint{4.640726in}{3.125641in}}%
\pgfpathlineto{\pgfqpoint{4.651698in}{3.141291in}}%
\pgfpathlineto{\pgfqpoint{4.657318in}{3.145790in}}%
\pgfpathlineto{\pgfqpoint{4.661867in}{3.146842in}}%
\pgfpathlineto{\pgfqpoint{4.666149in}{3.145581in}}%
\pgfpathlineto{\pgfqpoint{4.670966in}{3.141717in}}%
\pgfpathlineto{\pgfqpoint{4.677388in}{3.133416in}}%
\pgfpathlineto{\pgfqpoint{4.696656in}{3.106469in}}%
\pgfpathlineto{\pgfqpoint{4.701740in}{3.103796in}}%
\pgfpathlineto{\pgfqpoint{4.706022in}{3.103899in}}%
\pgfpathlineto{\pgfqpoint{4.710571in}{3.106362in}}%
\pgfpathlineto{\pgfqpoint{4.716191in}{3.112316in}}%
\pgfpathlineto{\pgfqpoint{4.725022in}{3.125641in}}%
\pgfpathlineto{\pgfqpoint{4.735994in}{3.141291in}}%
\pgfpathlineto{\pgfqpoint{4.741614in}{3.145790in}}%
\pgfpathlineto{\pgfqpoint{4.746163in}{3.146842in}}%
\pgfpathlineto{\pgfqpoint{4.750445in}{3.145581in}}%
\pgfpathlineto{\pgfqpoint{4.755262in}{3.141717in}}%
\pgfpathlineto{\pgfqpoint{4.761685in}{3.133416in}}%
\pgfpathlineto{\pgfqpoint{4.780952in}{3.106469in}}%
\pgfpathlineto{\pgfqpoint{4.786037in}{3.103796in}}%
\pgfpathlineto{\pgfqpoint{4.790319in}{3.103899in}}%
\pgfpathlineto{\pgfqpoint{4.794868in}{3.106362in}}%
\pgfpathlineto{\pgfqpoint{4.800488in}{3.112316in}}%
\pgfpathlineto{\pgfqpoint{4.809319in}{3.125641in}}%
\pgfpathlineto{\pgfqpoint{4.820291in}{3.141291in}}%
\pgfpathlineto{\pgfqpoint{4.825910in}{3.145790in}}%
\pgfpathlineto{\pgfqpoint{4.830460in}{3.146842in}}%
\pgfpathlineto{\pgfqpoint{4.834742in}{3.145581in}}%
\pgfpathlineto{\pgfqpoint{4.839558in}{3.141717in}}%
\pgfpathlineto{\pgfqpoint{4.845981in}{3.133416in}}%
\pgfpathlineto{\pgfqpoint{4.865249in}{3.106469in}}%
\pgfpathlineto{\pgfqpoint{4.870333in}{3.103796in}}%
\pgfpathlineto{\pgfqpoint{4.874615in}{3.103899in}}%
\pgfpathlineto{\pgfqpoint{4.879164in}{3.106362in}}%
\pgfpathlineto{\pgfqpoint{4.884784in}{3.112316in}}%
\pgfpathlineto{\pgfqpoint{4.893615in}{3.125641in}}%
\pgfpathlineto{\pgfqpoint{4.904587in}{3.141291in}}%
\pgfpathlineto{\pgfqpoint{4.910207in}{3.145790in}}%
\pgfpathlineto{\pgfqpoint{4.914756in}{3.146842in}}%
\pgfpathlineto{\pgfqpoint{4.919038in}{3.145581in}}%
\pgfpathlineto{\pgfqpoint{4.923855in}{3.141717in}}%
\pgfpathlineto{\pgfqpoint{4.930278in}{3.133416in}}%
\pgfpathlineto{\pgfqpoint{4.949545in}{3.106469in}}%
\pgfpathlineto{\pgfqpoint{4.954630in}{3.103796in}}%
\pgfpathlineto{\pgfqpoint{4.958912in}{3.103899in}}%
\pgfpathlineto{\pgfqpoint{4.963461in}{3.106362in}}%
\pgfpathlineto{\pgfqpoint{4.969081in}{3.112316in}}%
\pgfpathlineto{\pgfqpoint{4.977912in}{3.125641in}}%
\pgfpathlineto{\pgfqpoint{4.988884in}{3.141291in}}%
\pgfpathlineto{\pgfqpoint{4.994503in}{3.145790in}}%
\pgfpathlineto{\pgfqpoint{4.999053in}{3.146842in}}%
\pgfpathlineto{\pgfqpoint{5.003334in}{3.145581in}}%
\pgfpathlineto{\pgfqpoint{5.008151in}{3.141717in}}%
\pgfpathlineto{\pgfqpoint{5.014574in}{3.133416in}}%
\pgfpathlineto{\pgfqpoint{5.033842in}{3.106469in}}%
\pgfpathlineto{\pgfqpoint{5.038926in}{3.103796in}}%
\pgfpathlineto{\pgfqpoint{5.043208in}{3.103899in}}%
\pgfpathlineto{\pgfqpoint{5.047757in}{3.106362in}}%
\pgfpathlineto{\pgfqpoint{5.053377in}{3.112316in}}%
\pgfpathlineto{\pgfqpoint{5.062208in}{3.125641in}}%
\pgfpathlineto{\pgfqpoint{5.073180in}{3.141291in}}%
\pgfpathlineto{\pgfqpoint{5.078800in}{3.145790in}}%
\pgfpathlineto{\pgfqpoint{5.083349in}{3.146842in}}%
\pgfpathlineto{\pgfqpoint{5.087631in}{3.145581in}}%
\pgfpathlineto{\pgfqpoint{5.092448in}{3.141717in}}%
\pgfpathlineto{\pgfqpoint{5.098870in}{3.133416in}}%
\pgfpathlineto{\pgfqpoint{5.118138in}{3.106469in}}%
\pgfpathlineto{\pgfqpoint{5.123223in}{3.103796in}}%
\pgfpathlineto{\pgfqpoint{5.127504in}{3.103899in}}%
\pgfpathlineto{\pgfqpoint{5.132054in}{3.106362in}}%
\pgfpathlineto{\pgfqpoint{5.137674in}{3.112316in}}%
\pgfpathlineto{\pgfqpoint{5.146505in}{3.125641in}}%
\pgfpathlineto{\pgfqpoint{5.157477in}{3.141291in}}%
\pgfpathlineto{\pgfqpoint{5.163096in}{3.145790in}}%
\pgfpathlineto{\pgfqpoint{5.167646in}{3.146842in}}%
\pgfpathlineto{\pgfqpoint{5.171927in}{3.145581in}}%
\pgfpathlineto{\pgfqpoint{5.176744in}{3.141717in}}%
\pgfpathlineto{\pgfqpoint{5.183167in}{3.133416in}}%
\pgfpathlineto{\pgfqpoint{5.202435in}{3.106469in}}%
\pgfpathlineto{\pgfqpoint{5.207519in}{3.103796in}}%
\pgfpathlineto{\pgfqpoint{5.211801in}{3.103899in}}%
\pgfpathlineto{\pgfqpoint{5.216350in}{3.106362in}}%
\pgfpathlineto{\pgfqpoint{5.221970in}{3.112316in}}%
\pgfpathlineto{\pgfqpoint{5.230801in}{3.125641in}}%
\pgfpathlineto{\pgfqpoint{5.241773in}{3.141291in}}%
\pgfpathlineto{\pgfqpoint{5.247393in}{3.145790in}}%
\pgfpathlineto{\pgfqpoint{5.251942in}{3.146842in}}%
\pgfpathlineto{\pgfqpoint{5.256224in}{3.145581in}}%
\pgfpathlineto{\pgfqpoint{5.261041in}{3.141717in}}%
\pgfpathlineto{\pgfqpoint{5.267463in}{3.133416in}}%
\pgfpathlineto{\pgfqpoint{5.286731in}{3.106469in}}%
\pgfpathlineto{\pgfqpoint{5.291816in}{3.103796in}}%
\pgfpathlineto{\pgfqpoint{5.296097in}{3.103899in}}%
\pgfpathlineto{\pgfqpoint{5.300647in}{3.106362in}}%
\pgfpathlineto{\pgfqpoint{5.306266in}{3.112316in}}%
\pgfpathlineto{\pgfqpoint{5.315098in}{3.125641in}}%
\pgfpathlineto{\pgfqpoint{5.326069in}{3.141291in}}%
\pgfpathlineto{\pgfqpoint{5.331689in}{3.145790in}}%
\pgfpathlineto{\pgfqpoint{5.336239in}{3.146842in}}%
\pgfpathlineto{\pgfqpoint{5.340520in}{3.145581in}}%
\pgfpathlineto{\pgfqpoint{5.345337in}{3.141717in}}%
\pgfpathlineto{\pgfqpoint{5.351760in}{3.133416in}}%
\pgfpathlineto{\pgfqpoint{5.371028in}{3.106469in}}%
\pgfpathlineto{\pgfqpoint{5.376112in}{3.103796in}}%
\pgfpathlineto{\pgfqpoint{5.380394in}{3.103899in}}%
\pgfpathlineto{\pgfqpoint{5.384943in}{3.106362in}}%
\pgfpathlineto{\pgfqpoint{5.390563in}{3.112316in}}%
\pgfpathlineto{\pgfqpoint{5.399394in}{3.125641in}}%
\pgfpathlineto{\pgfqpoint{5.410366in}{3.141291in}}%
\pgfpathlineto{\pgfqpoint{5.415986in}{3.145790in}}%
\pgfpathlineto{\pgfqpoint{5.420535in}{3.146842in}}%
\pgfpathlineto{\pgfqpoint{5.424817in}{3.145581in}}%
\pgfpathlineto{\pgfqpoint{5.429634in}{3.141717in}}%
\pgfpathlineto{\pgfqpoint{5.436056in}{3.133416in}}%
\pgfpathlineto{\pgfqpoint{5.455324in}{3.106469in}}%
\pgfpathlineto{\pgfqpoint{5.460409in}{3.103796in}}%
\pgfpathlineto{\pgfqpoint{5.464690in}{3.103899in}}%
\pgfpathlineto{\pgfqpoint{5.469240in}{3.106362in}}%
\pgfpathlineto{\pgfqpoint{5.474859in}{3.112316in}}%
\pgfpathlineto{\pgfqpoint{5.483690in}{3.125641in}}%
\pgfpathlineto{\pgfqpoint{5.494662in}{3.141291in}}%
\pgfpathlineto{\pgfqpoint{5.500282in}{3.145790in}}%
\pgfpathlineto{\pgfqpoint{5.504831in}{3.146842in}}%
\pgfpathlineto{\pgfqpoint{5.509113in}{3.145581in}}%
\pgfpathlineto{\pgfqpoint{5.513930in}{3.141717in}}%
\pgfpathlineto{\pgfqpoint{5.520353in}{3.133416in}}%
\pgfpathlineto{\pgfqpoint{5.539621in}{3.106469in}}%
\pgfpathlineto{\pgfqpoint{5.544705in}{3.103796in}}%
\pgfpathlineto{\pgfqpoint{5.548987in}{3.103899in}}%
\pgfpathlineto{\pgfqpoint{5.553536in}{3.106362in}}%
\pgfpathlineto{\pgfqpoint{5.559156in}{3.112316in}}%
\pgfpathlineto{\pgfqpoint{5.567987in}{3.125641in}}%
\pgfpathlineto{\pgfqpoint{5.578959in}{3.141291in}}%
\pgfpathlineto{\pgfqpoint{5.584579in}{3.145790in}}%
\pgfpathlineto{\pgfqpoint{5.589128in}{3.146842in}}%
\pgfpathlineto{\pgfqpoint{5.593410in}{3.145581in}}%
\pgfpathlineto{\pgfqpoint{5.598227in}{3.141717in}}%
\pgfpathlineto{\pgfqpoint{5.604649in}{3.133416in}}%
\pgfpathlineto{\pgfqpoint{5.623917in}{3.106469in}}%
\pgfpathlineto{\pgfqpoint{5.629002in}{3.103796in}}%
\pgfpathlineto{\pgfqpoint{5.633283in}{3.103899in}}%
\pgfpathlineto{\pgfqpoint{5.637833in}{3.106362in}}%
\pgfpathlineto{\pgfqpoint{5.643452in}{3.112316in}}%
\pgfpathlineto{\pgfqpoint{5.652283in}{3.125641in}}%
\pgfpathlineto{\pgfqpoint{5.663255in}{3.141291in}}%
\pgfpathlineto{\pgfqpoint{5.668875in}{3.145790in}}%
\pgfpathlineto{\pgfqpoint{5.673424in}{3.146842in}}%
\pgfpathlineto{\pgfqpoint{5.677706in}{3.145581in}}%
\pgfpathlineto{\pgfqpoint{5.682523in}{3.141717in}}%
\pgfpathlineto{\pgfqpoint{5.688946in}{3.133416in}}%
\pgfpathlineto{\pgfqpoint{5.708213in}{3.106469in}}%
\pgfpathlineto{\pgfqpoint{5.713298in}{3.103796in}}%
\pgfpathlineto{\pgfqpoint{5.717580in}{3.103899in}}%
\pgfpathlineto{\pgfqpoint{5.722129in}{3.106362in}}%
\pgfpathlineto{\pgfqpoint{5.727749in}{3.112316in}}%
\pgfpathlineto{\pgfqpoint{5.736580in}{3.125641in}}%
\pgfpathlineto{\pgfqpoint{5.747552in}{3.141291in}}%
\pgfpathlineto{\pgfqpoint{5.753172in}{3.145790in}}%
\pgfpathlineto{\pgfqpoint{5.757721in}{3.146842in}}%
\pgfpathlineto{\pgfqpoint{5.762003in}{3.145581in}}%
\pgfpathlineto{\pgfqpoint{5.766820in}{3.141717in}}%
\pgfpathlineto{\pgfqpoint{5.773242in}{3.133416in}}%
\pgfpathlineto{\pgfqpoint{5.792510in}{3.106469in}}%
\pgfpathlineto{\pgfqpoint{5.797594in}{3.103796in}}%
\pgfpathlineto{\pgfqpoint{5.801876in}{3.103899in}}%
\pgfpathlineto{\pgfqpoint{5.806426in}{3.106362in}}%
\pgfpathlineto{\pgfqpoint{5.812045in}{3.112316in}}%
\pgfpathlineto{\pgfqpoint{5.820876in}{3.125641in}}%
\pgfpathlineto{\pgfqpoint{5.831848in}{3.141291in}}%
\pgfpathlineto{\pgfqpoint{5.837468in}{3.145790in}}%
\pgfpathlineto{\pgfqpoint{5.842017in}{3.146842in}}%
\pgfpathlineto{\pgfqpoint{5.846299in}{3.145581in}}%
\pgfpathlineto{\pgfqpoint{5.851116in}{3.141717in}}%
\pgfpathlineto{\pgfqpoint{5.857539in}{3.133416in}}%
\pgfpathlineto{\pgfqpoint{5.876806in}{3.106469in}}%
\pgfpathlineto{\pgfqpoint{5.881891in}{3.103796in}}%
\pgfpathlineto{\pgfqpoint{5.886173in}{3.103899in}}%
\pgfpathlineto{\pgfqpoint{5.890722in}{3.106362in}}%
\pgfpathlineto{\pgfqpoint{5.896342in}{3.112316in}}%
\pgfpathlineto{\pgfqpoint{5.905173in}{3.125641in}}%
\pgfpathlineto{\pgfqpoint{5.916145in}{3.141291in}}%
\pgfpathlineto{\pgfqpoint{5.921764in}{3.145790in}}%
\pgfpathlineto{\pgfqpoint{5.926314in}{3.146842in}}%
\pgfpathlineto{\pgfqpoint{5.930596in}{3.145581in}}%
\pgfpathlineto{\pgfqpoint{5.935412in}{3.141717in}}%
\pgfpathlineto{\pgfqpoint{5.941835in}{3.133416in}}%
\pgfpathlineto{\pgfqpoint{5.961103in}{3.106469in}}%
\pgfpathlineto{\pgfqpoint{5.966187in}{3.103796in}}%
\pgfpathlineto{\pgfqpoint{5.970469in}{3.103899in}}%
\pgfpathlineto{\pgfqpoint{5.975018in}{3.106362in}}%
\pgfpathlineto{\pgfqpoint{5.980638in}{3.112316in}}%
\pgfpathlineto{\pgfqpoint{5.989469in}{3.125641in}}%
\pgfpathlineto{\pgfqpoint{6.000441in}{3.141291in}}%
\pgfpathlineto{\pgfqpoint{6.006061in}{3.145790in}}%
\pgfpathlineto{\pgfqpoint{6.010610in}{3.146842in}}%
\pgfpathlineto{\pgfqpoint{6.014892in}{3.145581in}}%
\pgfpathlineto{\pgfqpoint{6.019709in}{3.141717in}}%
\pgfpathlineto{\pgfqpoint{6.026132in}{3.133416in}}%
\pgfpathlineto{\pgfqpoint{6.045399in}{3.106469in}}%
\pgfpathlineto{\pgfqpoint{6.050484in}{3.103796in}}%
\pgfpathlineto{\pgfqpoint{6.054766in}{3.103899in}}%
\pgfpathlineto{\pgfqpoint{6.059315in}{3.106362in}}%
\pgfpathlineto{\pgfqpoint{6.064935in}{3.112316in}}%
\pgfpathlineto{\pgfqpoint{6.073766in}{3.125641in}}%
\pgfpathlineto{\pgfqpoint{6.084738in}{3.141291in}}%
\pgfpathlineto{\pgfqpoint{6.090357in}{3.145790in}}%
\pgfpathlineto{\pgfqpoint{6.094907in}{3.146842in}}%
\pgfpathlineto{\pgfqpoint{6.099188in}{3.145581in}}%
\pgfpathlineto{\pgfqpoint{6.104005in}{3.141717in}}%
\pgfpathlineto{\pgfqpoint{6.110428in}{3.133416in}}%
\pgfpathlineto{\pgfqpoint{6.129696in}{3.106469in}}%
\pgfpathlineto{\pgfqpoint{6.134780in}{3.103796in}}%
\pgfpathlineto{\pgfqpoint{6.139062in}{3.103899in}}%
\pgfpathlineto{\pgfqpoint{6.143611in}{3.106362in}}%
\pgfpathlineto{\pgfqpoint{6.149231in}{3.112316in}}%
\pgfpathlineto{\pgfqpoint{6.158062in}{3.125641in}}%
\pgfpathlineto{\pgfqpoint{6.169034in}{3.141291in}}%
\pgfpathlineto{\pgfqpoint{6.174654in}{3.145790in}}%
\pgfpathlineto{\pgfqpoint{6.179203in}{3.146842in}}%
\pgfpathlineto{\pgfqpoint{6.183485in}{3.145581in}}%
\pgfpathlineto{\pgfqpoint{6.188302in}{3.141717in}}%
\pgfpathlineto{\pgfqpoint{6.194724in}{3.133416in}}%
\pgfpathlineto{\pgfqpoint{6.213992in}{3.106469in}}%
\pgfpathlineto{\pgfqpoint{6.219077in}{3.103796in}}%
\pgfpathlineto{\pgfqpoint{6.223358in}{3.103899in}}%
\pgfpathlineto{\pgfqpoint{6.227908in}{3.106362in}}%
\pgfpathlineto{\pgfqpoint{6.233528in}{3.112316in}}%
\pgfpathlineto{\pgfqpoint{6.242359in}{3.125641in}}%
\pgfpathlineto{\pgfqpoint{6.253331in}{3.141291in}}%
\pgfpathlineto{\pgfqpoint{6.258950in}{3.145790in}}%
\pgfpathlineto{\pgfqpoint{6.263500in}{3.146842in}}%
\pgfpathlineto{\pgfqpoint{6.267781in}{3.145581in}}%
\pgfpathlineto{\pgfqpoint{6.272598in}{3.141717in}}%
\pgfpathlineto{\pgfqpoint{6.279021in}{3.133416in}}%
\pgfpathlineto{\pgfqpoint{6.298289in}{3.106469in}}%
\pgfpathlineto{\pgfqpoint{6.303373in}{3.103796in}}%
\pgfpathlineto{\pgfqpoint{6.307655in}{3.103899in}}%
\pgfpathlineto{\pgfqpoint{6.312204in}{3.106362in}}%
\pgfpathlineto{\pgfqpoint{6.317824in}{3.112316in}}%
\pgfpathlineto{\pgfqpoint{6.326655in}{3.125641in}}%
\pgfpathlineto{\pgfqpoint{6.337627in}{3.141291in}}%
\pgfpathlineto{\pgfqpoint{6.343247in}{3.145790in}}%
\pgfpathlineto{\pgfqpoint{6.347796in}{3.146842in}}%
\pgfpathlineto{\pgfqpoint{6.352078in}{3.145581in}}%
\pgfpathlineto{\pgfqpoint{6.356895in}{3.141717in}}%
\pgfpathlineto{\pgfqpoint{6.363317in}{3.133416in}}%
\pgfpathlineto{\pgfqpoint{6.382585in}{3.106469in}}%
\pgfpathlineto{\pgfqpoint{6.387670in}{3.103796in}}%
\pgfpathlineto{\pgfqpoint{6.391951in}{3.103899in}}%
\pgfpathlineto{\pgfqpoint{6.396501in}{3.106362in}}%
\pgfpathlineto{\pgfqpoint{6.402121in}{3.112316in}}%
\pgfpathlineto{\pgfqpoint{6.410952in}{3.125641in}}%
\pgfpathlineto{\pgfqpoint{6.421924in}{3.141291in}}%
\pgfpathlineto{\pgfqpoint{6.427543in}{3.145790in}}%
\pgfpathlineto{\pgfqpoint{6.432093in}{3.146842in}}%
\pgfpathlineto{\pgfqpoint{6.436374in}{3.145581in}}%
\pgfpathlineto{\pgfqpoint{6.441191in}{3.141717in}}%
\pgfpathlineto{\pgfqpoint{6.447614in}{3.133416in}}%
\pgfpathlineto{\pgfqpoint{6.466882in}{3.106469in}}%
\pgfpathlineto{\pgfqpoint{6.471966in}{3.103796in}}%
\pgfpathlineto{\pgfqpoint{6.476248in}{3.103899in}}%
\pgfpathlineto{\pgfqpoint{6.480797in}{3.106362in}}%
\pgfpathlineto{\pgfqpoint{6.486417in}{3.112316in}}%
\pgfpathlineto{\pgfqpoint{6.495248in}{3.125641in}}%
\pgfpathlineto{\pgfqpoint{6.506220in}{3.141291in}}%
\pgfpathlineto{\pgfqpoint{6.511840in}{3.145790in}}%
\pgfpathlineto{\pgfqpoint{6.516389in}{3.146842in}}%
\pgfpathlineto{\pgfqpoint{6.520671in}{3.145581in}}%
\pgfpathlineto{\pgfqpoint{6.525488in}{3.141717in}}%
\pgfpathlineto{\pgfqpoint{6.531910in}{3.133416in}}%
\pgfpathlineto{\pgfqpoint{6.551178in}{3.106469in}}%
\pgfpathlineto{\pgfqpoint{6.556263in}{3.103796in}}%
\pgfpathlineto{\pgfqpoint{6.560544in}{3.103899in}}%
\pgfpathlineto{\pgfqpoint{6.565094in}{3.106362in}}%
\pgfpathlineto{\pgfqpoint{6.570713in}{3.112316in}}%
\pgfpathlineto{\pgfqpoint{6.579545in}{3.125641in}}%
\pgfpathlineto{\pgfqpoint{6.590516in}{3.141291in}}%
\pgfpathlineto{\pgfqpoint{6.596136in}{3.145790in}}%
\pgfpathlineto{\pgfqpoint{6.600686in}{3.146842in}}%
\pgfpathlineto{\pgfqpoint{6.604967in}{3.145581in}}%
\pgfpathlineto{\pgfqpoint{6.609784in}{3.141717in}}%
\pgfpathlineto{\pgfqpoint{6.616207in}{3.133416in}}%
\pgfpathlineto{\pgfqpoint{6.635475in}{3.106469in}}%
\pgfpathlineto{\pgfqpoint{6.640559in}{3.103796in}}%
\pgfpathlineto{\pgfqpoint{6.644841in}{3.103899in}}%
\pgfpathlineto{\pgfqpoint{6.649390in}{3.106362in}}%
\pgfpathlineto{\pgfqpoint{6.655010in}{3.112316in}}%
\pgfpathlineto{\pgfqpoint{6.663306in}{3.124778in}}%
\pgfpathlineto{\pgfqpoint{6.663306in}{3.124778in}}%
\pgfusepath{stroke}%
\end{pgfscope}%
\begin{pgfscope}%
\pgfpathrectangle{\pgfqpoint{0.467797in}{2.292089in}}{\pgfqpoint{6.490533in}{1.666241in}}%
\pgfusepath{clip}%
\pgfsetrectcap%
\pgfsetroundjoin%
\pgfsetlinewidth{1.505625pt}%
\definecolor{currentstroke}{rgb}{0.737255,0.741176,0.133333}%
\pgfsetstrokecolor{currentstroke}%
\pgfsetdash{}{0pt}%
\pgfpathmoveto{\pgfqpoint{0.762821in}{3.125209in}}%
\pgfpathlineto{\pgfqpoint{0.773525in}{3.140499in}}%
\pgfpathlineto{\pgfqpoint{0.779145in}{3.144856in}}%
\pgfpathlineto{\pgfqpoint{0.783427in}{3.145651in}}%
\pgfpathlineto{\pgfqpoint{0.787708in}{3.144140in}}%
\pgfpathlineto{\pgfqpoint{0.792525in}{3.139908in}}%
\pgfpathlineto{\pgfqpoint{0.799216in}{3.130745in}}%
\pgfpathlineto{\pgfqpoint{0.814737in}{3.108569in}}%
\pgfpathlineto{\pgfqpoint{0.819821in}{3.105237in}}%
\pgfpathlineto{\pgfqpoint{0.824103in}{3.104878in}}%
\pgfpathlineto{\pgfqpoint{0.828385in}{3.106811in}}%
\pgfpathlineto{\pgfqpoint{0.833469in}{3.111768in}}%
\pgfpathlineto{\pgfqpoint{0.840962in}{3.122638in}}%
\pgfpathlineto{\pgfqpoint{0.853540in}{3.140790in}}%
\pgfpathlineto{\pgfqpoint{0.858892in}{3.144860in}}%
\pgfpathlineto{\pgfqpoint{0.863174in}{3.145650in}}%
\pgfpathlineto{\pgfqpoint{0.867456in}{3.144136in}}%
\pgfpathlineto{\pgfqpoint{0.872273in}{3.139900in}}%
\pgfpathlineto{\pgfqpoint{0.878963in}{3.130734in}}%
\pgfpathlineto{\pgfqpoint{0.894484in}{3.108562in}}%
\pgfpathlineto{\pgfqpoint{0.899569in}{3.105235in}}%
\pgfpathlineto{\pgfqpoint{0.903850in}{3.104880in}}%
\pgfpathlineto{\pgfqpoint{0.908132in}{3.106816in}}%
\pgfpathlineto{\pgfqpoint{0.913217in}{3.111777in}}%
\pgfpathlineto{\pgfqpoint{0.920710in}{3.122650in}}%
\pgfpathlineto{\pgfqpoint{0.933287in}{3.140798in}}%
\pgfpathlineto{\pgfqpoint{0.938639in}{3.144863in}}%
\pgfpathlineto{\pgfqpoint{0.942921in}{3.145649in}}%
\pgfpathlineto{\pgfqpoint{0.947203in}{3.144131in}}%
\pgfpathlineto{\pgfqpoint{0.952020in}{3.139892in}}%
\pgfpathlineto{\pgfqpoint{0.958710in}{3.130723in}}%
\pgfpathlineto{\pgfqpoint{0.974231in}{3.108555in}}%
\pgfpathlineto{\pgfqpoint{0.979316in}{3.105232in}}%
\pgfpathlineto{\pgfqpoint{0.983597in}{3.104881in}}%
\pgfpathlineto{\pgfqpoint{0.987879in}{3.106822in}}%
\pgfpathlineto{\pgfqpoint{0.992964in}{3.111786in}}%
\pgfpathlineto{\pgfqpoint{1.000457in}{3.122661in}}%
\pgfpathlineto{\pgfqpoint{1.013034in}{3.140806in}}%
\pgfpathlineto{\pgfqpoint{1.018386in}{3.144866in}}%
\pgfpathlineto{\pgfqpoint{1.022668in}{3.145649in}}%
\pgfpathlineto{\pgfqpoint{1.026950in}{3.144127in}}%
\pgfpathlineto{\pgfqpoint{1.031767in}{3.139883in}}%
\pgfpathlineto{\pgfqpoint{1.038725in}{3.130294in}}%
\pgfpathlineto{\pgfqpoint{1.053711in}{3.108803in}}%
\pgfpathlineto{\pgfqpoint{1.059063in}{3.105230in}}%
\pgfpathlineto{\pgfqpoint{1.063344in}{3.104882in}}%
\pgfpathlineto{\pgfqpoint{1.067626in}{3.106827in}}%
\pgfpathlineto{\pgfqpoint{1.072711in}{3.111794in}}%
\pgfpathlineto{\pgfqpoint{1.080204in}{3.122673in}}%
\pgfpathlineto{\pgfqpoint{1.092781in}{3.140813in}}%
\pgfpathlineto{\pgfqpoint{1.098134in}{3.144869in}}%
\pgfpathlineto{\pgfqpoint{1.102415in}{3.145648in}}%
\pgfpathlineto{\pgfqpoint{1.106697in}{3.144122in}}%
\pgfpathlineto{\pgfqpoint{1.111514in}{3.139875in}}%
\pgfpathlineto{\pgfqpoint{1.118472in}{3.130283in}}%
\pgfpathlineto{\pgfqpoint{1.133458in}{3.108796in}}%
\pgfpathlineto{\pgfqpoint{1.138810in}{3.105227in}}%
\pgfpathlineto{\pgfqpoint{1.143092in}{3.104884in}}%
\pgfpathlineto{\pgfqpoint{1.147373in}{3.106832in}}%
\pgfpathlineto{\pgfqpoint{1.152458in}{3.111803in}}%
\pgfpathlineto{\pgfqpoint{1.159951in}{3.122684in}}%
\pgfpathlineto{\pgfqpoint{1.172528in}{3.140821in}}%
\pgfpathlineto{\pgfqpoint{1.177881in}{3.144873in}}%
\pgfpathlineto{\pgfqpoint{1.182162in}{3.145647in}}%
\pgfpathlineto{\pgfqpoint{1.186444in}{3.144118in}}%
\pgfpathlineto{\pgfqpoint{1.191261in}{3.139867in}}%
\pgfpathlineto{\pgfqpoint{1.198219in}{3.130272in}}%
\pgfpathlineto{\pgfqpoint{1.213205in}{3.108789in}}%
\pgfpathlineto{\pgfqpoint{1.218557in}{3.105225in}}%
\pgfpathlineto{\pgfqpoint{1.222839in}{3.104885in}}%
\pgfpathlineto{\pgfqpoint{1.227120in}{3.106837in}}%
\pgfpathlineto{\pgfqpoint{1.232205in}{3.111812in}}%
\pgfpathlineto{\pgfqpoint{1.239698in}{3.122696in}}%
\pgfpathlineto{\pgfqpoint{1.252276in}{3.140828in}}%
\pgfpathlineto{\pgfqpoint{1.257628in}{3.144876in}}%
\pgfpathlineto{\pgfqpoint{1.261909in}{3.145647in}}%
\pgfpathlineto{\pgfqpoint{1.266191in}{3.144114in}}%
\pgfpathlineto{\pgfqpoint{1.271008in}{3.139859in}}%
\pgfpathlineto{\pgfqpoint{1.277966in}{3.130261in}}%
\pgfpathlineto{\pgfqpoint{1.292952in}{3.108782in}}%
\pgfpathlineto{\pgfqpoint{1.298304in}{3.105222in}}%
\pgfpathlineto{\pgfqpoint{1.302586in}{3.104886in}}%
\pgfpathlineto{\pgfqpoint{1.306868in}{3.106842in}}%
\pgfpathlineto{\pgfqpoint{1.311952in}{3.111821in}}%
\pgfpathlineto{\pgfqpoint{1.319713in}{3.123136in}}%
\pgfpathlineto{\pgfqpoint{1.332023in}{3.140836in}}%
\pgfpathlineto{\pgfqpoint{1.337375in}{3.144879in}}%
\pgfpathlineto{\pgfqpoint{1.341657in}{3.145646in}}%
\pgfpathlineto{\pgfqpoint{1.345938in}{3.144109in}}%
\pgfpathlineto{\pgfqpoint{1.350755in}{3.139851in}}%
\pgfpathlineto{\pgfqpoint{1.357713in}{3.130249in}}%
\pgfpathlineto{\pgfqpoint{1.372699in}{3.108775in}}%
\pgfpathlineto{\pgfqpoint{1.378051in}{3.105220in}}%
\pgfpathlineto{\pgfqpoint{1.382333in}{3.104888in}}%
\pgfpathlineto{\pgfqpoint{1.386615in}{3.106847in}}%
\pgfpathlineto{\pgfqpoint{1.391699in}{3.111830in}}%
\pgfpathlineto{\pgfqpoint{1.399460in}{3.123148in}}%
\pgfpathlineto{\pgfqpoint{1.411770in}{3.140843in}}%
\pgfpathlineto{\pgfqpoint{1.417122in}{3.144882in}}%
\pgfpathlineto{\pgfqpoint{1.421404in}{3.145645in}}%
\pgfpathlineto{\pgfqpoint{1.425685in}{3.144105in}}%
\pgfpathlineto{\pgfqpoint{1.430502in}{3.139843in}}%
\pgfpathlineto{\pgfqpoint{1.437460in}{3.130238in}}%
\pgfpathlineto{\pgfqpoint{1.452446in}{3.108768in}}%
\pgfpathlineto{\pgfqpoint{1.457798in}{3.105217in}}%
\pgfpathlineto{\pgfqpoint{1.462080in}{3.104889in}}%
\pgfpathlineto{\pgfqpoint{1.466362in}{3.106852in}}%
\pgfpathlineto{\pgfqpoint{1.471446in}{3.111838in}}%
\pgfpathlineto{\pgfqpoint{1.479207in}{3.123160in}}%
\pgfpathlineto{\pgfqpoint{1.491517in}{3.140851in}}%
\pgfpathlineto{\pgfqpoint{1.496869in}{3.144885in}}%
\pgfpathlineto{\pgfqpoint{1.501151in}{3.145645in}}%
\pgfpathlineto{\pgfqpoint{1.505433in}{3.144100in}}%
\pgfpathlineto{\pgfqpoint{1.510250in}{3.139835in}}%
\pgfpathlineto{\pgfqpoint{1.517207in}{3.130227in}}%
\pgfpathlineto{\pgfqpoint{1.532193in}{3.108761in}}%
\pgfpathlineto{\pgfqpoint{1.537546in}{3.105215in}}%
\pgfpathlineto{\pgfqpoint{1.541827in}{3.104891in}}%
\pgfpathlineto{\pgfqpoint{1.546109in}{3.106858in}}%
\pgfpathlineto{\pgfqpoint{1.551194in}{3.111847in}}%
\pgfpathlineto{\pgfqpoint{1.558954in}{3.123171in}}%
\pgfpathlineto{\pgfqpoint{1.571264in}{3.140858in}}%
\pgfpathlineto{\pgfqpoint{1.576616in}{3.144889in}}%
\pgfpathlineto{\pgfqpoint{1.580898in}{3.145644in}}%
\pgfpathlineto{\pgfqpoint{1.585180in}{3.144096in}}%
\pgfpathlineto{\pgfqpoint{1.589997in}{3.139826in}}%
\pgfpathlineto{\pgfqpoint{1.596954in}{3.130215in}}%
\pgfpathlineto{\pgfqpoint{1.611941in}{3.108754in}}%
\pgfpathlineto{\pgfqpoint{1.617293in}{3.105212in}}%
\pgfpathlineto{\pgfqpoint{1.621574in}{3.104892in}}%
\pgfpathlineto{\pgfqpoint{1.625856in}{3.106863in}}%
\pgfpathlineto{\pgfqpoint{1.630941in}{3.111856in}}%
\pgfpathlineto{\pgfqpoint{1.638701in}{3.123183in}}%
\pgfpathlineto{\pgfqpoint{1.651011in}{3.140866in}}%
\pgfpathlineto{\pgfqpoint{1.656363in}{3.144892in}}%
\pgfpathlineto{\pgfqpoint{1.660645in}{3.145643in}}%
\pgfpathlineto{\pgfqpoint{1.664927in}{3.144091in}}%
\pgfpathlineto{\pgfqpoint{1.669744in}{3.139818in}}%
\pgfpathlineto{\pgfqpoint{1.676702in}{3.130204in}}%
\pgfpathlineto{\pgfqpoint{1.691688in}{3.108747in}}%
\pgfpathlineto{\pgfqpoint{1.697040in}{3.105210in}}%
\pgfpathlineto{\pgfqpoint{1.701322in}{3.104893in}}%
\pgfpathlineto{\pgfqpoint{1.705603in}{3.106868in}}%
\pgfpathlineto{\pgfqpoint{1.710688in}{3.111865in}}%
\pgfpathlineto{\pgfqpoint{1.718448in}{3.123195in}}%
\pgfpathlineto{\pgfqpoint{1.730758in}{3.140873in}}%
\pgfpathlineto{\pgfqpoint{1.736111in}{3.144895in}}%
\pgfpathlineto{\pgfqpoint{1.740392in}{3.145643in}}%
\pgfpathlineto{\pgfqpoint{1.744674in}{3.144087in}}%
\pgfpathlineto{\pgfqpoint{1.749491in}{3.139810in}}%
\pgfpathlineto{\pgfqpoint{1.756449in}{3.130193in}}%
\pgfpathlineto{\pgfqpoint{1.771435in}{3.108740in}}%
\pgfpathlineto{\pgfqpoint{1.776787in}{3.105207in}}%
\pgfpathlineto{\pgfqpoint{1.781069in}{3.104895in}}%
\pgfpathlineto{\pgfqpoint{1.785350in}{3.106873in}}%
\pgfpathlineto{\pgfqpoint{1.790435in}{3.111874in}}%
\pgfpathlineto{\pgfqpoint{1.798196in}{3.123206in}}%
\pgfpathlineto{\pgfqpoint{1.810238in}{3.140600in}}%
\pgfpathlineto{\pgfqpoint{1.815858in}{3.144898in}}%
\pgfpathlineto{\pgfqpoint{1.820139in}{3.145642in}}%
\pgfpathlineto{\pgfqpoint{1.824421in}{3.144082in}}%
\pgfpathlineto{\pgfqpoint{1.829238in}{3.139802in}}%
\pgfpathlineto{\pgfqpoint{1.836196in}{3.130181in}}%
\pgfpathlineto{\pgfqpoint{1.851182in}{3.108733in}}%
\pgfpathlineto{\pgfqpoint{1.856534in}{3.105205in}}%
\pgfpathlineto{\pgfqpoint{1.860816in}{3.104896in}}%
\pgfpathlineto{\pgfqpoint{1.865098in}{3.106878in}}%
\pgfpathlineto{\pgfqpoint{1.870182in}{3.111883in}}%
\pgfpathlineto{\pgfqpoint{1.877943in}{3.123218in}}%
\pgfpathlineto{\pgfqpoint{1.889985in}{3.140607in}}%
\pgfpathlineto{\pgfqpoint{1.895605in}{3.144901in}}%
\pgfpathlineto{\pgfqpoint{1.899887in}{3.145641in}}%
\pgfpathlineto{\pgfqpoint{1.904168in}{3.144078in}}%
\pgfpathlineto{\pgfqpoint{1.908985in}{3.139794in}}%
\pgfpathlineto{\pgfqpoint{1.915943in}{3.130170in}}%
\pgfpathlineto{\pgfqpoint{1.930661in}{3.108986in}}%
\pgfpathlineto{\pgfqpoint{1.936014in}{3.105298in}}%
\pgfpathlineto{\pgfqpoint{1.940295in}{3.104849in}}%
\pgfpathlineto{\pgfqpoint{1.944577in}{3.106695in}}%
\pgfpathlineto{\pgfqpoint{1.949662in}{3.111567in}}%
\pgfpathlineto{\pgfqpoint{1.957155in}{3.122372in}}%
\pgfpathlineto{\pgfqpoint{1.970000in}{3.140896in}}%
\pgfpathlineto{\pgfqpoint{1.975352in}{3.144905in}}%
\pgfpathlineto{\pgfqpoint{1.979634in}{3.145641in}}%
\pgfpathlineto{\pgfqpoint{1.983915in}{3.144073in}}%
\pgfpathlineto{\pgfqpoint{1.988732in}{3.139785in}}%
\pgfpathlineto{\pgfqpoint{1.995690in}{3.130159in}}%
\pgfpathlineto{\pgfqpoint{2.010409in}{3.108979in}}%
\pgfpathlineto{\pgfqpoint{2.015761in}{3.105295in}}%
\pgfpathlineto{\pgfqpoint{2.020042in}{3.104850in}}%
\pgfpathlineto{\pgfqpoint{2.024324in}{3.106700in}}%
\pgfpathlineto{\pgfqpoint{2.029409in}{3.111575in}}%
\pgfpathlineto{\pgfqpoint{2.036902in}{3.122384in}}%
\pgfpathlineto{\pgfqpoint{2.049747in}{3.140903in}}%
\pgfpathlineto{\pgfqpoint{2.055099in}{3.144908in}}%
\pgfpathlineto{\pgfqpoint{2.059381in}{3.145640in}}%
\pgfpathlineto{\pgfqpoint{2.063663in}{3.144068in}}%
\pgfpathlineto{\pgfqpoint{2.068479in}{3.139777in}}%
\pgfpathlineto{\pgfqpoint{2.075437in}{3.130147in}}%
\pgfpathlineto{\pgfqpoint{2.090156in}{3.108972in}}%
\pgfpathlineto{\pgfqpoint{2.095508in}{3.105293in}}%
\pgfpathlineto{\pgfqpoint{2.099790in}{3.104851in}}%
\pgfpathlineto{\pgfqpoint{2.104071in}{3.106705in}}%
\pgfpathlineto{\pgfqpoint{2.109156in}{3.111584in}}%
\pgfpathlineto{\pgfqpoint{2.116649in}{3.122395in}}%
\pgfpathlineto{\pgfqpoint{2.129494in}{3.140911in}}%
\pgfpathlineto{\pgfqpoint{2.134846in}{3.144911in}}%
\pgfpathlineto{\pgfqpoint{2.139128in}{3.145639in}}%
\pgfpathlineto{\pgfqpoint{2.143410in}{3.144064in}}%
\pgfpathlineto{\pgfqpoint{2.148227in}{3.139769in}}%
\pgfpathlineto{\pgfqpoint{2.155184in}{3.130136in}}%
\pgfpathlineto{\pgfqpoint{2.169903in}{3.108965in}}%
\pgfpathlineto{\pgfqpoint{2.175255in}{3.105290in}}%
\pgfpathlineto{\pgfqpoint{2.179537in}{3.104852in}}%
\pgfpathlineto{\pgfqpoint{2.183818in}{3.106710in}}%
\pgfpathlineto{\pgfqpoint{2.188903in}{3.111593in}}%
\pgfpathlineto{\pgfqpoint{2.196396in}{3.122407in}}%
\pgfpathlineto{\pgfqpoint{2.209241in}{3.140918in}}%
\pgfpathlineto{\pgfqpoint{2.214593in}{3.144914in}}%
\pgfpathlineto{\pgfqpoint{2.218875in}{3.145638in}}%
\pgfpathlineto{\pgfqpoint{2.223157in}{3.144059in}}%
\pgfpathlineto{\pgfqpoint{2.227974in}{3.139761in}}%
\pgfpathlineto{\pgfqpoint{2.234932in}{3.130125in}}%
\pgfpathlineto{\pgfqpoint{2.249650in}{3.108958in}}%
\pgfpathlineto{\pgfqpoint{2.255002in}{3.105287in}}%
\pgfpathlineto{\pgfqpoint{2.259284in}{3.104854in}}%
\pgfpathlineto{\pgfqpoint{2.263566in}{3.106715in}}%
\pgfpathlineto{\pgfqpoint{2.268650in}{3.111602in}}%
\pgfpathlineto{\pgfqpoint{2.276143in}{3.122418in}}%
\pgfpathlineto{\pgfqpoint{2.288988in}{3.140926in}}%
\pgfpathlineto{\pgfqpoint{2.294340in}{3.144917in}}%
\pgfpathlineto{\pgfqpoint{2.298622in}{3.145638in}}%
\pgfpathlineto{\pgfqpoint{2.302904in}{3.144055in}}%
\pgfpathlineto{\pgfqpoint{2.307721in}{3.139753in}}%
\pgfpathlineto{\pgfqpoint{2.314679in}{3.130113in}}%
\pgfpathlineto{\pgfqpoint{2.329397in}{3.108951in}}%
\pgfpathlineto{\pgfqpoint{2.334749in}{3.105284in}}%
\pgfpathlineto{\pgfqpoint{2.339031in}{3.104855in}}%
\pgfpathlineto{\pgfqpoint{2.343313in}{3.106720in}}%
\pgfpathlineto{\pgfqpoint{2.348397in}{3.111610in}}%
\pgfpathlineto{\pgfqpoint{2.355890in}{3.122430in}}%
\pgfpathlineto{\pgfqpoint{2.368735in}{3.140933in}}%
\pgfpathlineto{\pgfqpoint{2.374088in}{3.144920in}}%
\pgfpathlineto{\pgfqpoint{2.378369in}{3.145637in}}%
\pgfpathlineto{\pgfqpoint{2.382651in}{3.144050in}}%
\pgfpathlineto{\pgfqpoint{2.387468in}{3.139744in}}%
\pgfpathlineto{\pgfqpoint{2.394426in}{3.130102in}}%
\pgfpathlineto{\pgfqpoint{2.409144in}{3.108944in}}%
\pgfpathlineto{\pgfqpoint{2.414496in}{3.105282in}}%
\pgfpathlineto{\pgfqpoint{2.418778in}{3.104856in}}%
\pgfpathlineto{\pgfqpoint{2.423060in}{3.106725in}}%
\pgfpathlineto{\pgfqpoint{2.428144in}{3.111619in}}%
\pgfpathlineto{\pgfqpoint{2.435637in}{3.122441in}}%
\pgfpathlineto{\pgfqpoint{2.448483in}{3.140941in}}%
\pgfpathlineto{\pgfqpoint{2.453835in}{3.144924in}}%
\pgfpathlineto{\pgfqpoint{2.458116in}{3.145636in}}%
\pgfpathlineto{\pgfqpoint{2.462398in}{3.144046in}}%
\pgfpathlineto{\pgfqpoint{2.467215in}{3.139736in}}%
\pgfpathlineto{\pgfqpoint{2.474173in}{3.130091in}}%
\pgfpathlineto{\pgfqpoint{2.488891in}{3.108936in}}%
\pgfpathlineto{\pgfqpoint{2.494244in}{3.105279in}}%
\pgfpathlineto{\pgfqpoint{2.498525in}{3.104857in}}%
\pgfpathlineto{\pgfqpoint{2.502807in}{3.106730in}}%
\pgfpathlineto{\pgfqpoint{2.507892in}{3.111628in}}%
\pgfpathlineto{\pgfqpoint{2.515385in}{3.122453in}}%
\pgfpathlineto{\pgfqpoint{2.528230in}{3.140948in}}%
\pgfpathlineto{\pgfqpoint{2.533582in}{3.144927in}}%
\pgfpathlineto{\pgfqpoint{2.537864in}{3.145635in}}%
\pgfpathlineto{\pgfqpoint{2.542145in}{3.144041in}}%
\pgfpathlineto{\pgfqpoint{2.546962in}{3.139728in}}%
\pgfpathlineto{\pgfqpoint{2.553920in}{3.130079in}}%
\pgfpathlineto{\pgfqpoint{2.568638in}{3.108929in}}%
\pgfpathlineto{\pgfqpoint{2.573991in}{3.105276in}}%
\pgfpathlineto{\pgfqpoint{2.578272in}{3.104859in}}%
\pgfpathlineto{\pgfqpoint{2.582554in}{3.106735in}}%
\pgfpathlineto{\pgfqpoint{2.587639in}{3.111636in}}%
\pgfpathlineto{\pgfqpoint{2.595132in}{3.122465in}}%
\pgfpathlineto{\pgfqpoint{2.607977in}{3.140956in}}%
\pgfpathlineto{\pgfqpoint{2.613329in}{3.144930in}}%
\pgfpathlineto{\pgfqpoint{2.617611in}{3.145635in}}%
\pgfpathlineto{\pgfqpoint{2.621892in}{3.144037in}}%
\pgfpathlineto{\pgfqpoint{2.626709in}{3.139720in}}%
\pgfpathlineto{\pgfqpoint{2.633667in}{3.130068in}}%
\pgfpathlineto{\pgfqpoint{2.648386in}{3.108922in}}%
\pgfpathlineto{\pgfqpoint{2.653738in}{3.105274in}}%
\pgfpathlineto{\pgfqpoint{2.658019in}{3.104860in}}%
\pgfpathlineto{\pgfqpoint{2.662301in}{3.106740in}}%
\pgfpathlineto{\pgfqpoint{2.667386in}{3.111645in}}%
\pgfpathlineto{\pgfqpoint{2.674879in}{3.122476in}}%
\pgfpathlineto{\pgfqpoint{2.687724in}{3.140963in}}%
\pgfpathlineto{\pgfqpoint{2.693076in}{3.144933in}}%
\pgfpathlineto{\pgfqpoint{2.697358in}{3.145634in}}%
\pgfpathlineto{\pgfqpoint{2.701640in}{3.144032in}}%
\pgfpathlineto{\pgfqpoint{2.706457in}{3.139712in}}%
\pgfpathlineto{\pgfqpoint{2.713414in}{3.130057in}}%
\pgfpathlineto{\pgfqpoint{2.728133in}{3.108915in}}%
\pgfpathlineto{\pgfqpoint{2.733485in}{3.105271in}}%
\pgfpathlineto{\pgfqpoint{2.737767in}{3.104861in}}%
\pgfpathlineto{\pgfqpoint{2.742048in}{3.106745in}}%
\pgfpathlineto{\pgfqpoint{2.747133in}{3.111654in}}%
\pgfpathlineto{\pgfqpoint{2.754626in}{3.122488in}}%
\pgfpathlineto{\pgfqpoint{2.767471in}{3.140971in}}%
\pgfpathlineto{\pgfqpoint{2.772823in}{3.144936in}}%
\pgfpathlineto{\pgfqpoint{2.777105in}{3.145633in}}%
\pgfpathlineto{\pgfqpoint{2.781387in}{3.144027in}}%
\pgfpathlineto{\pgfqpoint{2.786471in}{3.139395in}}%
\pgfpathlineto{\pgfqpoint{2.793697in}{3.129203in}}%
\pgfpathlineto{\pgfqpoint{2.807612in}{3.109173in}}%
\pgfpathlineto{\pgfqpoint{2.812964in}{3.105370in}}%
\pgfpathlineto{\pgfqpoint{2.817246in}{3.104820in}}%
\pgfpathlineto{\pgfqpoint{2.821528in}{3.106568in}}%
\pgfpathlineto{\pgfqpoint{2.826612in}{3.111342in}}%
\pgfpathlineto{\pgfqpoint{2.833838in}{3.121646in}}%
\pgfpathlineto{\pgfqpoint{2.847218in}{3.140978in}}%
\pgfpathlineto{\pgfqpoint{2.852570in}{3.144939in}}%
\pgfpathlineto{\pgfqpoint{2.856852in}{3.145632in}}%
\pgfpathlineto{\pgfqpoint{2.861134in}{3.144023in}}%
\pgfpathlineto{\pgfqpoint{2.866218in}{3.139387in}}%
\pgfpathlineto{\pgfqpoint{2.873444in}{3.129191in}}%
\pgfpathlineto{\pgfqpoint{2.887359in}{3.109166in}}%
\pgfpathlineto{\pgfqpoint{2.892712in}{3.105368in}}%
\pgfpathlineto{\pgfqpoint{2.896993in}{3.104821in}}%
\pgfpathlineto{\pgfqpoint{2.901275in}{3.106573in}}%
\pgfpathlineto{\pgfqpoint{2.906360in}{3.111351in}}%
\pgfpathlineto{\pgfqpoint{2.913585in}{3.121658in}}%
\pgfpathlineto{\pgfqpoint{2.926965in}{3.140985in}}%
\pgfpathlineto{\pgfqpoint{2.932318in}{3.144942in}}%
\pgfpathlineto{\pgfqpoint{2.936599in}{3.145632in}}%
\pgfpathlineto{\pgfqpoint{2.940881in}{3.144018in}}%
\pgfpathlineto{\pgfqpoint{2.945966in}{3.139379in}}%
\pgfpathlineto{\pgfqpoint{2.953191in}{3.129180in}}%
\pgfpathlineto{\pgfqpoint{2.967107in}{3.109158in}}%
\pgfpathlineto{\pgfqpoint{2.972459in}{3.105365in}}%
\pgfpathlineto{\pgfqpoint{2.976740in}{3.104822in}}%
\pgfpathlineto{\pgfqpoint{2.981022in}{3.106578in}}%
\pgfpathlineto{\pgfqpoint{2.986107in}{3.111359in}}%
\pgfpathlineto{\pgfqpoint{2.993332in}{3.121669in}}%
\pgfpathlineto{\pgfqpoint{3.006712in}{3.140993in}}%
\pgfpathlineto{\pgfqpoint{3.012065in}{3.144945in}}%
\pgfpathlineto{\pgfqpoint{3.016346in}{3.145631in}}%
\pgfpathlineto{\pgfqpoint{3.020628in}{3.144014in}}%
\pgfpathlineto{\pgfqpoint{3.025713in}{3.139370in}}%
\pgfpathlineto{\pgfqpoint{3.032938in}{3.129168in}}%
\pgfpathlineto{\pgfqpoint{3.046854in}{3.109151in}}%
\pgfpathlineto{\pgfqpoint{3.052206in}{3.105362in}}%
\pgfpathlineto{\pgfqpoint{3.056488in}{3.104823in}}%
\pgfpathlineto{\pgfqpoint{3.060769in}{3.106582in}}%
\pgfpathlineto{\pgfqpoint{3.065854in}{3.111368in}}%
\pgfpathlineto{\pgfqpoint{3.073079in}{3.121681in}}%
\pgfpathlineto{\pgfqpoint{3.086460in}{3.141000in}}%
\pgfpathlineto{\pgfqpoint{3.091812in}{3.144949in}}%
\pgfpathlineto{\pgfqpoint{3.096094in}{3.145630in}}%
\pgfpathlineto{\pgfqpoint{3.100375in}{3.144009in}}%
\pgfpathlineto{\pgfqpoint{3.105460in}{3.139362in}}%
\pgfpathlineto{\pgfqpoint{3.112685in}{3.129157in}}%
\pgfpathlineto{\pgfqpoint{3.126601in}{3.109144in}}%
\pgfpathlineto{\pgfqpoint{3.131953in}{3.105359in}}%
\pgfpathlineto{\pgfqpoint{3.136235in}{3.104824in}}%
\pgfpathlineto{\pgfqpoint{3.140516in}{3.106587in}}%
\pgfpathlineto{\pgfqpoint{3.145601in}{3.111376in}}%
\pgfpathlineto{\pgfqpoint{3.152826in}{3.121692in}}%
\pgfpathlineto{\pgfqpoint{3.166207in}{3.141008in}}%
\pgfpathlineto{\pgfqpoint{3.171559in}{3.144952in}}%
\pgfpathlineto{\pgfqpoint{3.175841in}{3.145629in}}%
\pgfpathlineto{\pgfqpoint{3.180122in}{3.144004in}}%
\pgfpathlineto{\pgfqpoint{3.185207in}{3.139353in}}%
\pgfpathlineto{\pgfqpoint{3.192432in}{3.129145in}}%
\pgfpathlineto{\pgfqpoint{3.206348in}{3.109137in}}%
\pgfpathlineto{\pgfqpoint{3.211700in}{3.105356in}}%
\pgfpathlineto{\pgfqpoint{3.215982in}{3.104825in}}%
\pgfpathlineto{\pgfqpoint{3.220264in}{3.106592in}}%
\pgfpathlineto{\pgfqpoint{3.225348in}{3.111385in}}%
\pgfpathlineto{\pgfqpoint{3.232573in}{3.121704in}}%
\pgfpathlineto{\pgfqpoint{3.245954in}{3.141015in}}%
\pgfpathlineto{\pgfqpoint{3.251306in}{3.144955in}}%
\pgfpathlineto{\pgfqpoint{3.255588in}{3.145628in}}%
\pgfpathlineto{\pgfqpoint{3.259869in}{3.144000in}}%
\pgfpathlineto{\pgfqpoint{3.264954in}{3.139345in}}%
\pgfpathlineto{\pgfqpoint{3.272179in}{3.129134in}}%
\pgfpathlineto{\pgfqpoint{3.286095in}{3.109129in}}%
\pgfpathlineto{\pgfqpoint{3.291447in}{3.105353in}}%
\pgfpathlineto{\pgfqpoint{3.295729in}{3.104826in}}%
\pgfpathlineto{\pgfqpoint{3.300011in}{3.106597in}}%
\pgfpathlineto{\pgfqpoint{3.305095in}{3.111394in}}%
\pgfpathlineto{\pgfqpoint{3.312321in}{3.121715in}}%
\pgfpathlineto{\pgfqpoint{3.325701in}{3.141023in}}%
\pgfpathlineto{\pgfqpoint{3.331053in}{3.144958in}}%
\pgfpathlineto{\pgfqpoint{3.335335in}{3.145628in}}%
\pgfpathlineto{\pgfqpoint{3.339617in}{3.143995in}}%
\pgfpathlineto{\pgfqpoint{3.344701in}{3.139336in}}%
\pgfpathlineto{\pgfqpoint{3.351927in}{3.129122in}}%
\pgfpathlineto{\pgfqpoint{3.365842in}{3.109122in}}%
\pgfpathlineto{\pgfqpoint{3.371194in}{3.105351in}}%
\pgfpathlineto{\pgfqpoint{3.375476in}{3.104827in}}%
\pgfpathlineto{\pgfqpoint{3.379758in}{3.106602in}}%
\pgfpathlineto{\pgfqpoint{3.384842in}{3.111402in}}%
\pgfpathlineto{\pgfqpoint{3.392068in}{3.121727in}}%
\pgfpathlineto{\pgfqpoint{3.405448in}{3.141030in}}%
\pgfpathlineto{\pgfqpoint{3.410800in}{3.144961in}}%
\pgfpathlineto{\pgfqpoint{3.415082in}{3.145627in}}%
\pgfpathlineto{\pgfqpoint{3.419364in}{3.143990in}}%
\pgfpathlineto{\pgfqpoint{3.424448in}{3.139328in}}%
\pgfpathlineto{\pgfqpoint{3.431674in}{3.129111in}}%
\pgfpathlineto{\pgfqpoint{3.445589in}{3.109115in}}%
\pgfpathlineto{\pgfqpoint{3.450941in}{3.105348in}}%
\pgfpathlineto{\pgfqpoint{3.455223in}{3.104828in}}%
\pgfpathlineto{\pgfqpoint{3.459505in}{3.106607in}}%
\pgfpathlineto{\pgfqpoint{3.464589in}{3.111411in}}%
\pgfpathlineto{\pgfqpoint{3.471815in}{3.121738in}}%
\pgfpathlineto{\pgfqpoint{3.485195in}{3.141037in}}%
\pgfpathlineto{\pgfqpoint{3.490547in}{3.144964in}}%
\pgfpathlineto{\pgfqpoint{3.494829in}{3.145626in}}%
\pgfpathlineto{\pgfqpoint{3.499111in}{3.143986in}}%
\pgfpathlineto{\pgfqpoint{3.504195in}{3.139319in}}%
\pgfpathlineto{\pgfqpoint{3.511421in}{3.129100in}}%
\pgfpathlineto{\pgfqpoint{3.525069in}{3.109378in}}%
\pgfpathlineto{\pgfqpoint{3.530421in}{3.105453in}}%
\pgfpathlineto{\pgfqpoint{3.534703in}{3.104793in}}%
\pgfpathlineto{\pgfqpoint{3.538984in}{3.106435in}}%
\pgfpathlineto{\pgfqpoint{3.544069in}{3.111104in}}%
\pgfpathlineto{\pgfqpoint{3.551294in}{3.121325in}}%
\pgfpathlineto{\pgfqpoint{3.564942in}{3.141045in}}%
\pgfpathlineto{\pgfqpoint{3.570295in}{3.144967in}}%
\pgfpathlineto{\pgfqpoint{3.574576in}{3.145625in}}%
\pgfpathlineto{\pgfqpoint{3.578858in}{3.143981in}}%
\pgfpathlineto{\pgfqpoint{3.583943in}{3.139311in}}%
\pgfpathlineto{\pgfqpoint{3.591168in}{3.129088in}}%
\pgfpathlineto{\pgfqpoint{3.604816in}{3.109370in}}%
\pgfpathlineto{\pgfqpoint{3.610168in}{3.105450in}}%
\pgfpathlineto{\pgfqpoint{3.614450in}{3.104794in}}%
\pgfpathlineto{\pgfqpoint{3.618732in}{3.106440in}}%
\pgfpathlineto{\pgfqpoint{3.623816in}{3.111112in}}%
\pgfpathlineto{\pgfqpoint{3.631042in}{3.121337in}}%
\pgfpathlineto{\pgfqpoint{3.644690in}{3.141052in}}%
\pgfpathlineto{\pgfqpoint{3.650042in}{3.144970in}}%
\pgfpathlineto{\pgfqpoint{3.654323in}{3.145624in}}%
\pgfpathlineto{\pgfqpoint{3.658605in}{3.143977in}}%
\pgfpathlineto{\pgfqpoint{3.663690in}{3.139303in}}%
\pgfpathlineto{\pgfqpoint{3.670915in}{3.129077in}}%
\pgfpathlineto{\pgfqpoint{3.684563in}{3.109363in}}%
\pgfpathlineto{\pgfqpoint{3.689915in}{3.105447in}}%
\pgfpathlineto{\pgfqpoint{3.694197in}{3.104795in}}%
\pgfpathlineto{\pgfqpoint{3.698479in}{3.106445in}}%
\pgfpathlineto{\pgfqpoint{3.703563in}{3.111121in}}%
\pgfpathlineto{\pgfqpoint{3.710789in}{3.121348in}}%
\pgfpathlineto{\pgfqpoint{3.724437in}{3.141059in}}%
\pgfpathlineto{\pgfqpoint{3.729789in}{3.144973in}}%
\pgfpathlineto{\pgfqpoint{3.734071in}{3.145623in}}%
\pgfpathlineto{\pgfqpoint{3.738352in}{3.143972in}}%
\pgfpathlineto{\pgfqpoint{3.743437in}{3.139294in}}%
\pgfpathlineto{\pgfqpoint{3.750662in}{3.129065in}}%
\pgfpathlineto{\pgfqpoint{3.764310in}{3.109356in}}%
\pgfpathlineto{\pgfqpoint{3.769662in}{3.105444in}}%
\pgfpathlineto{\pgfqpoint{3.773944in}{3.104796in}}%
\pgfpathlineto{\pgfqpoint{3.778226in}{3.106449in}}%
\pgfpathlineto{\pgfqpoint{3.783310in}{3.111129in}}%
\pgfpathlineto{\pgfqpoint{3.790536in}{3.121359in}}%
\pgfpathlineto{\pgfqpoint{3.804184in}{3.141067in}}%
\pgfpathlineto{\pgfqpoint{3.809536in}{3.144976in}}%
\pgfpathlineto{\pgfqpoint{3.813818in}{3.145622in}}%
\pgfpathlineto{\pgfqpoint{3.818099in}{3.143967in}}%
\pgfpathlineto{\pgfqpoint{3.823184in}{3.139286in}}%
\pgfpathlineto{\pgfqpoint{3.830409in}{3.129054in}}%
\pgfpathlineto{\pgfqpoint{3.844057in}{3.109348in}}%
\pgfpathlineto{\pgfqpoint{3.849410in}{3.105441in}}%
\pgfpathlineto{\pgfqpoint{3.853691in}{3.104797in}}%
\pgfpathlineto{\pgfqpoint{3.857973in}{3.106454in}}%
\pgfpathlineto{\pgfqpoint{3.863058in}{3.111137in}}%
\pgfpathlineto{\pgfqpoint{3.870283in}{3.121371in}}%
\pgfpathlineto{\pgfqpoint{3.883931in}{3.141074in}}%
\pgfpathlineto{\pgfqpoint{3.889283in}{3.144979in}}%
\pgfpathlineto{\pgfqpoint{3.893565in}{3.145622in}}%
\pgfpathlineto{\pgfqpoint{3.897847in}{3.143963in}}%
\pgfpathlineto{\pgfqpoint{3.902931in}{3.139277in}}%
\pgfpathlineto{\pgfqpoint{3.910156in}{3.129042in}}%
\pgfpathlineto{\pgfqpoint{3.923804in}{3.109341in}}%
\pgfpathlineto{\pgfqpoint{3.929157in}{3.105438in}}%
\pgfpathlineto{\pgfqpoint{3.933438in}{3.104798in}}%
\pgfpathlineto{\pgfqpoint{3.937720in}{3.106459in}}%
\pgfpathlineto{\pgfqpoint{3.942805in}{3.111146in}}%
\pgfpathlineto{\pgfqpoint{3.950030in}{3.121382in}}%
\pgfpathlineto{\pgfqpoint{3.963678in}{3.141082in}}%
\pgfpathlineto{\pgfqpoint{3.969030in}{3.144982in}}%
\pgfpathlineto{\pgfqpoint{3.973312in}{3.145621in}}%
\pgfpathlineto{\pgfqpoint{3.977594in}{3.143958in}}%
\pgfpathlineto{\pgfqpoint{3.982678in}{3.139269in}}%
\pgfpathlineto{\pgfqpoint{3.989904in}{3.129031in}}%
\pgfpathlineto{\pgfqpoint{4.003552in}{3.109334in}}%
\pgfpathlineto{\pgfqpoint{4.008904in}{3.105435in}}%
\pgfpathlineto{\pgfqpoint{4.013186in}{3.104799in}}%
\pgfpathlineto{\pgfqpoint{4.017467in}{3.106463in}}%
\pgfpathlineto{\pgfqpoint{4.022552in}{3.111154in}}%
\pgfpathlineto{\pgfqpoint{4.029777in}{3.121394in}}%
\pgfpathlineto{\pgfqpoint{4.043425in}{3.141089in}}%
\pgfpathlineto{\pgfqpoint{4.048777in}{3.144985in}}%
\pgfpathlineto{\pgfqpoint{4.053059in}{3.145620in}}%
\pgfpathlineto{\pgfqpoint{4.057341in}{3.143953in}}%
\pgfpathlineto{\pgfqpoint{4.062425in}{3.139260in}}%
\pgfpathlineto{\pgfqpoint{4.069651in}{3.129019in}}%
\pgfpathlineto{\pgfqpoint{4.083299in}{3.109326in}}%
\pgfpathlineto{\pgfqpoint{4.088651in}{3.105432in}}%
\pgfpathlineto{\pgfqpoint{4.092933in}{3.104800in}}%
\pgfpathlineto{\pgfqpoint{4.097214in}{3.106468in}}%
\pgfpathlineto{\pgfqpoint{4.102299in}{3.111163in}}%
\pgfpathlineto{\pgfqpoint{4.109524in}{3.121405in}}%
\pgfpathlineto{\pgfqpoint{4.123172in}{3.141096in}}%
\pgfpathlineto{\pgfqpoint{4.128524in}{3.144988in}}%
\pgfpathlineto{\pgfqpoint{4.132806in}{3.145619in}}%
\pgfpathlineto{\pgfqpoint{4.137088in}{3.143948in}}%
\pgfpathlineto{\pgfqpoint{4.142172in}{3.139252in}}%
\pgfpathlineto{\pgfqpoint{4.149398in}{3.129008in}}%
\pgfpathlineto{\pgfqpoint{4.163046in}{3.109319in}}%
\pgfpathlineto{\pgfqpoint{4.168398in}{3.105429in}}%
\pgfpathlineto{\pgfqpoint{4.172680in}{3.104800in}}%
\pgfpathlineto{\pgfqpoint{4.176961in}{3.106473in}}%
\pgfpathlineto{\pgfqpoint{4.182046in}{3.111171in}}%
\pgfpathlineto{\pgfqpoint{4.189271in}{3.121417in}}%
\pgfpathlineto{\pgfqpoint{4.202919in}{3.141104in}}%
\pgfpathlineto{\pgfqpoint{4.208272in}{3.144991in}}%
\pgfpathlineto{\pgfqpoint{4.212553in}{3.145618in}}%
\pgfpathlineto{\pgfqpoint{4.216835in}{3.143944in}}%
\pgfpathlineto{\pgfqpoint{4.221920in}{3.139243in}}%
\pgfpathlineto{\pgfqpoint{4.229145in}{3.128996in}}%
\pgfpathlineto{\pgfqpoint{4.242793in}{3.109312in}}%
\pgfpathlineto{\pgfqpoint{4.248145in}{3.105426in}}%
\pgfpathlineto{\pgfqpoint{4.252427in}{3.104801in}}%
\pgfpathlineto{\pgfqpoint{4.256709in}{3.106477in}}%
\pgfpathlineto{\pgfqpoint{4.261793in}{3.111180in}}%
\pgfpathlineto{\pgfqpoint{4.269019in}{3.121428in}}%
\pgfpathlineto{\pgfqpoint{4.282667in}{3.141111in}}%
\pgfpathlineto{\pgfqpoint{4.288019in}{3.144994in}}%
\pgfpathlineto{\pgfqpoint{4.292300in}{3.145617in}}%
\pgfpathlineto{\pgfqpoint{4.296582in}{3.143939in}}%
\pgfpathlineto{\pgfqpoint{4.301667in}{3.139235in}}%
\pgfpathlineto{\pgfqpoint{4.308892in}{3.128985in}}%
\pgfpathlineto{\pgfqpoint{4.322540in}{3.109304in}}%
\pgfpathlineto{\pgfqpoint{4.327892in}{3.105423in}}%
\pgfpathlineto{\pgfqpoint{4.332174in}{3.104802in}}%
\pgfpathlineto{\pgfqpoint{4.336456in}{3.106482in}}%
\pgfpathlineto{\pgfqpoint{4.341540in}{3.111188in}}%
\pgfpathlineto{\pgfqpoint{4.348766in}{3.121440in}}%
\pgfpathlineto{\pgfqpoint{4.362414in}{3.141118in}}%
\pgfpathlineto{\pgfqpoint{4.367766in}{3.144997in}}%
\pgfpathlineto{\pgfqpoint{4.372048in}{3.145616in}}%
\pgfpathlineto{\pgfqpoint{4.376329in}{3.143934in}}%
\pgfpathlineto{\pgfqpoint{4.381414in}{3.139226in}}%
\pgfpathlineto{\pgfqpoint{4.388639in}{3.128973in}}%
\pgfpathlineto{\pgfqpoint{4.402287in}{3.109297in}}%
\pgfpathlineto{\pgfqpoint{4.407639in}{3.105420in}}%
\pgfpathlineto{\pgfqpoint{4.411921in}{3.104803in}}%
\pgfpathlineto{\pgfqpoint{4.416203in}{3.106487in}}%
\pgfpathlineto{\pgfqpoint{4.421287in}{3.111197in}}%
\pgfpathlineto{\pgfqpoint{4.428513in}{3.121451in}}%
\pgfpathlineto{\pgfqpoint{4.442161in}{3.141126in}}%
\pgfpathlineto{\pgfqpoint{4.447513in}{3.145000in}}%
\pgfpathlineto{\pgfqpoint{4.451795in}{3.145615in}}%
\pgfpathlineto{\pgfqpoint{4.456076in}{3.143930in}}%
\pgfpathlineto{\pgfqpoint{4.461161in}{3.139218in}}%
\pgfpathlineto{\pgfqpoint{4.468386in}{3.128962in}}%
\pgfpathlineto{\pgfqpoint{4.482034in}{3.109289in}}%
\pgfpathlineto{\pgfqpoint{4.487387in}{3.105417in}}%
\pgfpathlineto{\pgfqpoint{4.491668in}{3.104804in}}%
\pgfpathlineto{\pgfqpoint{4.495950in}{3.106492in}}%
\pgfpathlineto{\pgfqpoint{4.501035in}{3.111205in}}%
\pgfpathlineto{\pgfqpoint{4.508260in}{3.121463in}}%
\pgfpathlineto{\pgfqpoint{4.521908in}{3.141133in}}%
\pgfpathlineto{\pgfqpoint{4.527260in}{3.145003in}}%
\pgfpathlineto{\pgfqpoint{4.531542in}{3.145614in}}%
\pgfpathlineto{\pgfqpoint{4.535824in}{3.143925in}}%
\pgfpathlineto{\pgfqpoint{4.540908in}{3.139209in}}%
\pgfpathlineto{\pgfqpoint{4.548134in}{3.128951in}}%
\pgfpathlineto{\pgfqpoint{4.561782in}{3.109282in}}%
\pgfpathlineto{\pgfqpoint{4.567134in}{3.105414in}}%
\pgfpathlineto{\pgfqpoint{4.571415in}{3.104805in}}%
\pgfpathlineto{\pgfqpoint{4.575697in}{3.106496in}}%
\pgfpathlineto{\pgfqpoint{4.580782in}{3.111214in}}%
\pgfpathlineto{\pgfqpoint{4.588007in}{3.121474in}}%
\pgfpathlineto{\pgfqpoint{4.601655in}{3.141140in}}%
\pgfpathlineto{\pgfqpoint{4.607007in}{3.145006in}}%
\pgfpathlineto{\pgfqpoint{4.611289in}{3.145613in}}%
\pgfpathlineto{\pgfqpoint{4.615571in}{3.143920in}}%
\pgfpathlineto{\pgfqpoint{4.620655in}{3.139201in}}%
\pgfpathlineto{\pgfqpoint{4.627881in}{3.128939in}}%
\pgfpathlineto{\pgfqpoint{4.641529in}{3.109275in}}%
\pgfpathlineto{\pgfqpoint{4.646881in}{3.105411in}}%
\pgfpathlineto{\pgfqpoint{4.651163in}{3.104806in}}%
\pgfpathlineto{\pgfqpoint{4.655444in}{3.106501in}}%
\pgfpathlineto{\pgfqpoint{4.660529in}{3.111222in}}%
\pgfpathlineto{\pgfqpoint{4.667754in}{3.121486in}}%
\pgfpathlineto{\pgfqpoint{4.681402in}{3.141148in}}%
\pgfpathlineto{\pgfqpoint{4.686754in}{3.145009in}}%
\pgfpathlineto{\pgfqpoint{4.691036in}{3.145612in}}%
\pgfpathlineto{\pgfqpoint{4.695318in}{3.143915in}}%
\pgfpathlineto{\pgfqpoint{4.700402in}{3.139192in}}%
\pgfpathlineto{\pgfqpoint{4.707628in}{3.128928in}}%
\pgfpathlineto{\pgfqpoint{4.721276in}{3.109268in}}%
\pgfpathlineto{\pgfqpoint{4.726628in}{3.105408in}}%
\pgfpathlineto{\pgfqpoint{4.730910in}{3.104807in}}%
\pgfpathlineto{\pgfqpoint{4.735191in}{3.106506in}}%
\pgfpathlineto{\pgfqpoint{4.740276in}{3.111231in}}%
\pgfpathlineto{\pgfqpoint{4.747501in}{3.121497in}}%
\pgfpathlineto{\pgfqpoint{4.761149in}{3.141155in}}%
\pgfpathlineto{\pgfqpoint{4.766502in}{3.145012in}}%
\pgfpathlineto{\pgfqpoint{4.770783in}{3.145612in}}%
\pgfpathlineto{\pgfqpoint{4.775065in}{3.143911in}}%
\pgfpathlineto{\pgfqpoint{4.780150in}{3.139184in}}%
\pgfpathlineto{\pgfqpoint{4.787375in}{3.128916in}}%
\pgfpathlineto{\pgfqpoint{4.801023in}{3.109260in}}%
\pgfpathlineto{\pgfqpoint{4.806375in}{3.105405in}}%
\pgfpathlineto{\pgfqpoint{4.810657in}{3.104808in}}%
\pgfpathlineto{\pgfqpoint{4.814939in}{3.106511in}}%
\pgfpathlineto{\pgfqpoint{4.820023in}{3.111239in}}%
\pgfpathlineto{\pgfqpoint{4.827248in}{3.121509in}}%
\pgfpathlineto{\pgfqpoint{4.840896in}{3.141162in}}%
\pgfpathlineto{\pgfqpoint{4.846249in}{3.145015in}}%
\pgfpathlineto{\pgfqpoint{4.850530in}{3.145611in}}%
\pgfpathlineto{\pgfqpoint{4.854812in}{3.143906in}}%
\pgfpathlineto{\pgfqpoint{4.859897in}{3.139175in}}%
\pgfpathlineto{\pgfqpoint{4.867122in}{3.128905in}}%
\pgfpathlineto{\pgfqpoint{4.880770in}{3.109253in}}%
\pgfpathlineto{\pgfqpoint{4.886122in}{3.105402in}}%
\pgfpathlineto{\pgfqpoint{4.890404in}{3.104809in}}%
\pgfpathlineto{\pgfqpoint{4.894686in}{3.106515in}}%
\pgfpathlineto{\pgfqpoint{4.899770in}{3.111248in}}%
\pgfpathlineto{\pgfqpoint{4.906996in}{3.121520in}}%
\pgfpathlineto{\pgfqpoint{4.920644in}{3.141170in}}%
\pgfpathlineto{\pgfqpoint{4.925996in}{3.145018in}}%
\pgfpathlineto{\pgfqpoint{4.930278in}{3.145610in}}%
\pgfpathlineto{\pgfqpoint{4.934559in}{3.143901in}}%
\pgfpathlineto{\pgfqpoint{4.939644in}{3.139167in}}%
\pgfpathlineto{\pgfqpoint{4.946869in}{3.128893in}}%
\pgfpathlineto{\pgfqpoint{4.960517in}{3.109246in}}%
\pgfpathlineto{\pgfqpoint{4.965869in}{3.105400in}}%
\pgfpathlineto{\pgfqpoint{4.970151in}{3.104810in}}%
\pgfpathlineto{\pgfqpoint{4.974433in}{3.106520in}}%
\pgfpathlineto{\pgfqpoint{4.979517in}{3.111257in}}%
\pgfpathlineto{\pgfqpoint{4.986743in}{3.121531in}}%
\pgfpathlineto{\pgfqpoint{5.000391in}{3.141177in}}%
\pgfpathlineto{\pgfqpoint{5.005743in}{3.145021in}}%
\pgfpathlineto{\pgfqpoint{5.010025in}{3.145609in}}%
\pgfpathlineto{\pgfqpoint{5.014306in}{3.143896in}}%
\pgfpathlineto{\pgfqpoint{5.019391in}{3.139158in}}%
\pgfpathlineto{\pgfqpoint{5.026616in}{3.128882in}}%
\pgfpathlineto{\pgfqpoint{5.040264in}{3.109238in}}%
\pgfpathlineto{\pgfqpoint{5.045616in}{3.105397in}}%
\pgfpathlineto{\pgfqpoint{5.049898in}{3.104811in}}%
\pgfpathlineto{\pgfqpoint{5.054180in}{3.106525in}}%
\pgfpathlineto{\pgfqpoint{5.059264in}{3.111265in}}%
\pgfpathlineto{\pgfqpoint{5.066490in}{3.121543in}}%
\pgfpathlineto{\pgfqpoint{5.080138in}{3.141184in}}%
\pgfpathlineto{\pgfqpoint{5.085490in}{3.145024in}}%
\pgfpathlineto{\pgfqpoint{5.089772in}{3.145608in}}%
\pgfpathlineto{\pgfqpoint{5.094053in}{3.143892in}}%
\pgfpathlineto{\pgfqpoint{5.099138in}{3.139150in}}%
\pgfpathlineto{\pgfqpoint{5.106363in}{3.128870in}}%
\pgfpathlineto{\pgfqpoint{5.120011in}{3.109231in}}%
\pgfpathlineto{\pgfqpoint{5.125364in}{3.105394in}}%
\pgfpathlineto{\pgfqpoint{5.129645in}{3.104812in}}%
\pgfpathlineto{\pgfqpoint{5.133927in}{3.106530in}}%
\pgfpathlineto{\pgfqpoint{5.139012in}{3.111274in}}%
\pgfpathlineto{\pgfqpoint{5.146237in}{3.121554in}}%
\pgfpathlineto{\pgfqpoint{5.159885in}{3.141192in}}%
\pgfpathlineto{\pgfqpoint{5.165237in}{3.145027in}}%
\pgfpathlineto{\pgfqpoint{5.169519in}{3.145607in}}%
\pgfpathlineto{\pgfqpoint{5.173801in}{3.143887in}}%
\pgfpathlineto{\pgfqpoint{5.178885in}{3.139141in}}%
\pgfpathlineto{\pgfqpoint{5.186111in}{3.128859in}}%
\pgfpathlineto{\pgfqpoint{5.199759in}{3.109224in}}%
\pgfpathlineto{\pgfqpoint{5.205111in}{3.105391in}}%
\pgfpathlineto{\pgfqpoint{5.209392in}{3.104813in}}%
\pgfpathlineto{\pgfqpoint{5.213674in}{3.106534in}}%
\pgfpathlineto{\pgfqpoint{5.218759in}{3.111282in}}%
\pgfpathlineto{\pgfqpoint{5.225984in}{3.121566in}}%
\pgfpathlineto{\pgfqpoint{5.239632in}{3.141199in}}%
\pgfpathlineto{\pgfqpoint{5.244984in}{3.145030in}}%
\pgfpathlineto{\pgfqpoint{5.249266in}{3.145606in}}%
\pgfpathlineto{\pgfqpoint{5.253548in}{3.143882in}}%
\pgfpathlineto{\pgfqpoint{5.258632in}{3.139132in}}%
\pgfpathlineto{\pgfqpoint{5.265858in}{3.128847in}}%
\pgfpathlineto{\pgfqpoint{5.279506in}{3.109216in}}%
\pgfpathlineto{\pgfqpoint{5.284858in}{3.105388in}}%
\pgfpathlineto{\pgfqpoint{5.289140in}{3.104814in}}%
\pgfpathlineto{\pgfqpoint{5.293421in}{3.106539in}}%
\pgfpathlineto{\pgfqpoint{5.298506in}{3.111291in}}%
\pgfpathlineto{\pgfqpoint{5.305731in}{3.121577in}}%
\pgfpathlineto{\pgfqpoint{5.319379in}{3.141206in}}%
\pgfpathlineto{\pgfqpoint{5.324731in}{3.145032in}}%
\pgfpathlineto{\pgfqpoint{5.329013in}{3.145605in}}%
\pgfpathlineto{\pgfqpoint{5.333295in}{3.143877in}}%
\pgfpathlineto{\pgfqpoint{5.338379in}{3.139124in}}%
\pgfpathlineto{\pgfqpoint{5.345605in}{3.128836in}}%
\pgfpathlineto{\pgfqpoint{5.359253in}{3.109209in}}%
\pgfpathlineto{\pgfqpoint{5.364605in}{3.105385in}}%
\pgfpathlineto{\pgfqpoint{5.368887in}{3.104815in}}%
\pgfpathlineto{\pgfqpoint{5.373168in}{3.106544in}}%
\pgfpathlineto{\pgfqpoint{5.378253in}{3.111299in}}%
\pgfpathlineto{\pgfqpoint{5.385478in}{3.121589in}}%
\pgfpathlineto{\pgfqpoint{5.399126in}{3.141213in}}%
\pgfpathlineto{\pgfqpoint{5.404479in}{3.145035in}}%
\pgfpathlineto{\pgfqpoint{5.408760in}{3.145604in}}%
\pgfpathlineto{\pgfqpoint{5.413042in}{3.143873in}}%
\pgfpathlineto{\pgfqpoint{5.418127in}{3.139115in}}%
\pgfpathlineto{\pgfqpoint{5.425352in}{3.128824in}}%
\pgfpathlineto{\pgfqpoint{5.439000in}{3.109202in}}%
\pgfpathlineto{\pgfqpoint{5.444352in}{3.105382in}}%
\pgfpathlineto{\pgfqpoint{5.448634in}{3.104816in}}%
\pgfpathlineto{\pgfqpoint{5.452916in}{3.106549in}}%
\pgfpathlineto{\pgfqpoint{5.458000in}{3.111308in}}%
\pgfpathlineto{\pgfqpoint{5.465226in}{3.121600in}}%
\pgfpathlineto{\pgfqpoint{5.478874in}{3.141221in}}%
\pgfpathlineto{\pgfqpoint{5.484226in}{3.145038in}}%
\pgfpathlineto{\pgfqpoint{5.488507in}{3.145603in}}%
\pgfpathlineto{\pgfqpoint{5.492789in}{3.143868in}}%
\pgfpathlineto{\pgfqpoint{5.497874in}{3.139107in}}%
\pgfpathlineto{\pgfqpoint{5.505099in}{3.128813in}}%
\pgfpathlineto{\pgfqpoint{5.518479in}{3.109467in}}%
\pgfpathlineto{\pgfqpoint{5.523832in}{3.105491in}}%
\pgfpathlineto{\pgfqpoint{5.528113in}{3.104784in}}%
\pgfpathlineto{\pgfqpoint{5.532395in}{3.106380in}}%
\pgfpathlineto{\pgfqpoint{5.537212in}{3.110695in}}%
\pgfpathlineto{\pgfqpoint{5.544170in}{3.120345in}}%
\pgfpathlineto{\pgfqpoint{5.558888in}{3.141493in}}%
\pgfpathlineto{\pgfqpoint{5.564240in}{3.145144in}}%
\pgfpathlineto{\pgfqpoint{5.568522in}{3.145560in}}%
\pgfpathlineto{\pgfqpoint{5.572804in}{3.143681in}}%
\pgfpathlineto{\pgfqpoint{5.577888in}{3.138778in}}%
\pgfpathlineto{\pgfqpoint{5.585381in}{3.127948in}}%
\pgfpathlineto{\pgfqpoint{5.598227in}{3.109460in}}%
\pgfpathlineto{\pgfqpoint{5.603579in}{3.105487in}}%
\pgfpathlineto{\pgfqpoint{5.607861in}{3.104785in}}%
\pgfpathlineto{\pgfqpoint{5.612142in}{3.106385in}}%
\pgfpathlineto{\pgfqpoint{5.616959in}{3.110703in}}%
\pgfpathlineto{\pgfqpoint{5.623917in}{3.120356in}}%
\pgfpathlineto{\pgfqpoint{5.638635in}{3.141500in}}%
\pgfpathlineto{\pgfqpoint{5.643988in}{3.145146in}}%
\pgfpathlineto{\pgfqpoint{5.648269in}{3.145558in}}%
\pgfpathlineto{\pgfqpoint{5.652551in}{3.143676in}}%
\pgfpathlineto{\pgfqpoint{5.657636in}{3.138769in}}%
\pgfpathlineto{\pgfqpoint{5.665129in}{3.127937in}}%
\pgfpathlineto{\pgfqpoint{5.677974in}{3.109452in}}%
\pgfpathlineto{\pgfqpoint{5.683326in}{3.105484in}}%
\pgfpathlineto{\pgfqpoint{5.687608in}{3.104785in}}%
\pgfpathlineto{\pgfqpoint{5.691889in}{3.106389in}}%
\pgfpathlineto{\pgfqpoint{5.696706in}{3.110711in}}%
\pgfpathlineto{\pgfqpoint{5.703664in}{3.120368in}}%
\pgfpathlineto{\pgfqpoint{5.718383in}{3.141507in}}%
\pgfpathlineto{\pgfqpoint{5.723735in}{3.145149in}}%
\pgfpathlineto{\pgfqpoint{5.728016in}{3.145557in}}%
\pgfpathlineto{\pgfqpoint{5.732298in}{3.143671in}}%
\pgfpathlineto{\pgfqpoint{5.737383in}{3.138761in}}%
\pgfpathlineto{\pgfqpoint{5.744876in}{3.127925in}}%
\pgfpathlineto{\pgfqpoint{5.757721in}{3.109445in}}%
\pgfpathlineto{\pgfqpoint{5.763073in}{3.105481in}}%
\pgfpathlineto{\pgfqpoint{5.767355in}{3.104786in}}%
\pgfpathlineto{\pgfqpoint{5.771636in}{3.106394in}}%
\pgfpathlineto{\pgfqpoint{5.776721in}{3.111028in}}%
\pgfpathlineto{\pgfqpoint{5.783946in}{3.121222in}}%
\pgfpathlineto{\pgfqpoint{5.797862in}{3.141250in}}%
\pgfpathlineto{\pgfqpoint{5.803214in}{3.145050in}}%
\pgfpathlineto{\pgfqpoint{5.807496in}{3.145599in}}%
\pgfpathlineto{\pgfqpoint{5.811778in}{3.143849in}}%
\pgfpathlineto{\pgfqpoint{5.816862in}{3.139072in}}%
\pgfpathlineto{\pgfqpoint{5.824088in}{3.128767in}}%
\pgfpathlineto{\pgfqpoint{5.837468in}{3.109437in}}%
\pgfpathlineto{\pgfqpoint{5.842820in}{3.105478in}}%
\pgfpathlineto{\pgfqpoint{5.847102in}{3.104787in}}%
\pgfpathlineto{\pgfqpoint{5.851384in}{3.106398in}}%
\pgfpathlineto{\pgfqpoint{5.856468in}{3.111036in}}%
\pgfpathlineto{\pgfqpoint{5.863694in}{3.121233in}}%
\pgfpathlineto{\pgfqpoint{5.877609in}{3.141257in}}%
\pgfpathlineto{\pgfqpoint{5.882961in}{3.145053in}}%
\pgfpathlineto{\pgfqpoint{5.887243in}{3.145598in}}%
\pgfpathlineto{\pgfqpoint{5.891525in}{3.143844in}}%
\pgfpathlineto{\pgfqpoint{5.896609in}{3.139064in}}%
\pgfpathlineto{\pgfqpoint{5.903835in}{3.128755in}}%
\pgfpathlineto{\pgfqpoint{5.917215in}{3.109430in}}%
\pgfpathlineto{\pgfqpoint{5.922567in}{3.105475in}}%
\pgfpathlineto{\pgfqpoint{5.926849in}{3.104788in}}%
\pgfpathlineto{\pgfqpoint{5.931131in}{3.106403in}}%
\pgfpathlineto{\pgfqpoint{5.936215in}{3.111045in}}%
\pgfpathlineto{\pgfqpoint{5.943441in}{3.121245in}}%
\pgfpathlineto{\pgfqpoint{5.957356in}{3.141264in}}%
\pgfpathlineto{\pgfqpoint{5.962708in}{3.145056in}}%
\pgfpathlineto{\pgfqpoint{5.966990in}{3.145597in}}%
\pgfpathlineto{\pgfqpoint{5.971272in}{3.143839in}}%
\pgfpathlineto{\pgfqpoint{5.976356in}{3.139055in}}%
\pgfpathlineto{\pgfqpoint{5.983582in}{3.128744in}}%
\pgfpathlineto{\pgfqpoint{5.996962in}{3.109422in}}%
\pgfpathlineto{\pgfqpoint{6.002314in}{3.105472in}}%
\pgfpathlineto{\pgfqpoint{6.006596in}{3.104788in}}%
\pgfpathlineto{\pgfqpoint{6.010878in}{3.106408in}}%
\pgfpathlineto{\pgfqpoint{6.015962in}{3.111053in}}%
\pgfpathlineto{\pgfqpoint{6.023188in}{3.121256in}}%
\pgfpathlineto{\pgfqpoint{6.037103in}{3.141271in}}%
\pgfpathlineto{\pgfqpoint{6.042456in}{3.145058in}}%
\pgfpathlineto{\pgfqpoint{6.046737in}{3.145596in}}%
\pgfpathlineto{\pgfqpoint{6.051019in}{3.143834in}}%
\pgfpathlineto{\pgfqpoint{6.056104in}{3.139047in}}%
\pgfpathlineto{\pgfqpoint{6.063329in}{3.128732in}}%
\pgfpathlineto{\pgfqpoint{6.076709in}{3.109415in}}%
\pgfpathlineto{\pgfqpoint{6.082062in}{3.105469in}}%
\pgfpathlineto{\pgfqpoint{6.086343in}{3.104789in}}%
\pgfpathlineto{\pgfqpoint{6.090625in}{3.106412in}}%
\pgfpathlineto{\pgfqpoint{6.095710in}{3.111061in}}%
\pgfpathlineto{\pgfqpoint{6.102935in}{3.121268in}}%
\pgfpathlineto{\pgfqpoint{6.116851in}{3.141279in}}%
\pgfpathlineto{\pgfqpoint{6.122203in}{3.145061in}}%
\pgfpathlineto{\pgfqpoint{6.126484in}{3.145594in}}%
\pgfpathlineto{\pgfqpoint{6.130766in}{3.143829in}}%
\pgfpathlineto{\pgfqpoint{6.135851in}{3.139038in}}%
\pgfpathlineto{\pgfqpoint{6.143076in}{3.128721in}}%
\pgfpathlineto{\pgfqpoint{6.156457in}{3.109407in}}%
\pgfpathlineto{\pgfqpoint{6.161809in}{3.105466in}}%
\pgfpathlineto{\pgfqpoint{6.166090in}{3.104790in}}%
\pgfpathlineto{\pgfqpoint{6.170372in}{3.106417in}}%
\pgfpathlineto{\pgfqpoint{6.175457in}{3.111070in}}%
\pgfpathlineto{\pgfqpoint{6.182682in}{3.121279in}}%
\pgfpathlineto{\pgfqpoint{6.196598in}{3.141286in}}%
\pgfpathlineto{\pgfqpoint{6.201950in}{3.145064in}}%
\pgfpathlineto{\pgfqpoint{6.206232in}{3.145593in}}%
\pgfpathlineto{\pgfqpoint{6.210513in}{3.143824in}}%
\pgfpathlineto{\pgfqpoint{6.215598in}{3.139029in}}%
\pgfpathlineto{\pgfqpoint{6.222823in}{3.128709in}}%
\pgfpathlineto{\pgfqpoint{6.236204in}{3.109400in}}%
\pgfpathlineto{\pgfqpoint{6.241556in}{3.105463in}}%
\pgfpathlineto{\pgfqpoint{6.245838in}{3.104791in}}%
\pgfpathlineto{\pgfqpoint{6.250119in}{3.106421in}}%
\pgfpathlineto{\pgfqpoint{6.255204in}{3.111078in}}%
\pgfpathlineto{\pgfqpoint{6.262429in}{3.121291in}}%
\pgfpathlineto{\pgfqpoint{6.276345in}{3.141293in}}%
\pgfpathlineto{\pgfqpoint{6.281697in}{3.145067in}}%
\pgfpathlineto{\pgfqpoint{6.285979in}{3.145592in}}%
\pgfpathlineto{\pgfqpoint{6.290260in}{3.143819in}}%
\pgfpathlineto{\pgfqpoint{6.295345in}{3.139021in}}%
\pgfpathlineto{\pgfqpoint{6.302570in}{3.128698in}}%
\pgfpathlineto{\pgfqpoint{6.315951in}{3.109393in}}%
\pgfpathlineto{\pgfqpoint{6.321303in}{3.105460in}}%
\pgfpathlineto{\pgfqpoint{6.325585in}{3.104792in}}%
\pgfpathlineto{\pgfqpoint{6.329866in}{3.106426in}}%
\pgfpathlineto{\pgfqpoint{6.334951in}{3.111087in}}%
\pgfpathlineto{\pgfqpoint{6.342176in}{3.121302in}}%
\pgfpathlineto{\pgfqpoint{6.356092in}{3.141300in}}%
\pgfpathlineto{\pgfqpoint{6.361444in}{3.145070in}}%
\pgfpathlineto{\pgfqpoint{6.365726in}{3.145591in}}%
\pgfpathlineto{\pgfqpoint{6.370008in}{3.143815in}}%
\pgfpathlineto{\pgfqpoint{6.375092in}{3.139012in}}%
\pgfpathlineto{\pgfqpoint{6.382318in}{3.128686in}}%
\pgfpathlineto{\pgfqpoint{6.395698in}{3.109385in}}%
\pgfpathlineto{\pgfqpoint{6.401050in}{3.105457in}}%
\pgfpathlineto{\pgfqpoint{6.405332in}{3.104793in}}%
\pgfpathlineto{\pgfqpoint{6.409614in}{3.106431in}}%
\pgfpathlineto{\pgfqpoint{6.414698in}{3.111095in}}%
\pgfpathlineto{\pgfqpoint{6.421924in}{3.121314in}}%
\pgfpathlineto{\pgfqpoint{6.435572in}{3.141037in}}%
\pgfpathlineto{\pgfqpoint{6.440924in}{3.144964in}}%
\pgfpathlineto{\pgfqpoint{6.445205in}{3.145626in}}%
\pgfpathlineto{\pgfqpoint{6.449487in}{3.143986in}}%
\pgfpathlineto{\pgfqpoint{6.454572in}{3.139319in}}%
\pgfpathlineto{\pgfqpoint{6.461797in}{3.129100in}}%
\pgfpathlineto{\pgfqpoint{6.475445in}{3.109378in}}%
\pgfpathlineto{\pgfqpoint{6.480797in}{3.105453in}}%
\pgfpathlineto{\pgfqpoint{6.485079in}{3.104793in}}%
\pgfpathlineto{\pgfqpoint{6.489361in}{3.106435in}}%
\pgfpathlineto{\pgfqpoint{6.494445in}{3.111104in}}%
\pgfpathlineto{\pgfqpoint{6.501671in}{3.121325in}}%
\pgfpathlineto{\pgfqpoint{6.515319in}{3.141045in}}%
\pgfpathlineto{\pgfqpoint{6.520671in}{3.144967in}}%
\pgfpathlineto{\pgfqpoint{6.524953in}{3.145625in}}%
\pgfpathlineto{\pgfqpoint{6.529234in}{3.143981in}}%
\pgfpathlineto{\pgfqpoint{6.534319in}{3.139311in}}%
\pgfpathlineto{\pgfqpoint{6.541544in}{3.129088in}}%
\pgfpathlineto{\pgfqpoint{6.555192in}{3.109370in}}%
\pgfpathlineto{\pgfqpoint{6.560544in}{3.105450in}}%
\pgfpathlineto{\pgfqpoint{6.564826in}{3.104794in}}%
\pgfpathlineto{\pgfqpoint{6.569108in}{3.106440in}}%
\pgfpathlineto{\pgfqpoint{6.574192in}{3.111112in}}%
\pgfpathlineto{\pgfqpoint{6.581418in}{3.121337in}}%
\pgfpathlineto{\pgfqpoint{6.595066in}{3.141052in}}%
\pgfpathlineto{\pgfqpoint{6.600418in}{3.144970in}}%
\pgfpathlineto{\pgfqpoint{6.604700in}{3.145624in}}%
\pgfpathlineto{\pgfqpoint{6.608981in}{3.143977in}}%
\pgfpathlineto{\pgfqpoint{6.614066in}{3.139303in}}%
\pgfpathlineto{\pgfqpoint{6.621291in}{3.129077in}}%
\pgfpathlineto{\pgfqpoint{6.634939in}{3.109363in}}%
\pgfpathlineto{\pgfqpoint{6.640291in}{3.105447in}}%
\pgfpathlineto{\pgfqpoint{6.644573in}{3.104795in}}%
\pgfpathlineto{\pgfqpoint{6.648855in}{3.106445in}}%
\pgfpathlineto{\pgfqpoint{6.653939in}{3.111121in}}%
\pgfpathlineto{\pgfqpoint{6.661165in}{3.121348in}}%
\pgfpathlineto{\pgfqpoint{6.663306in}{3.124778in}}%
\pgfpathlineto{\pgfqpoint{6.663306in}{3.124778in}}%
\pgfusepath{stroke}%
\end{pgfscope}%
\begin{pgfscope}%
\pgfpathrectangle{\pgfqpoint{0.467797in}{2.292089in}}{\pgfqpoint{6.490533in}{1.666241in}}%
\pgfusepath{clip}%
\pgfsetrectcap%
\pgfsetroundjoin%
\pgfsetlinewidth{1.505625pt}%
\definecolor{currentstroke}{rgb}{0.090196,0.745098,0.811765}%
\pgfsetstrokecolor{currentstroke}%
\pgfsetdash{}{0pt}%
\pgfpathmoveto{\pgfqpoint{0.762821in}{3.125209in}}%
\pgfpathlineto{\pgfqpoint{0.773258in}{3.140013in}}%
\pgfpathlineto{\pgfqpoint{0.778610in}{3.143980in}}%
\pgfpathlineto{\pgfqpoint{0.782891in}{3.144540in}}%
\pgfpathlineto{\pgfqpoint{0.787173in}{3.142681in}}%
\pgfpathlineto{\pgfqpoint{0.792258in}{3.137671in}}%
\pgfpathlineto{\pgfqpoint{0.800018in}{3.126222in}}%
\pgfpathlineto{\pgfqpoint{0.810990in}{3.110503in}}%
\pgfpathlineto{\pgfqpoint{0.816342in}{3.106478in}}%
\pgfpathlineto{\pgfqpoint{0.820624in}{3.105865in}}%
\pgfpathlineto{\pgfqpoint{0.824906in}{3.107673in}}%
\pgfpathlineto{\pgfqpoint{0.829990in}{3.112633in}}%
\pgfpathlineto{\pgfqpoint{0.837483in}{3.123617in}}%
\pgfpathlineto{\pgfqpoint{0.848991in}{3.140099in}}%
\pgfpathlineto{\pgfqpoint{0.854343in}{3.144013in}}%
\pgfpathlineto{\pgfqpoint{0.858625in}{3.144527in}}%
\pgfpathlineto{\pgfqpoint{0.862906in}{3.142623in}}%
\pgfpathlineto{\pgfqpoint{0.867991in}{3.137569in}}%
\pgfpathlineto{\pgfqpoint{0.875751in}{3.126089in}}%
\pgfpathlineto{\pgfqpoint{0.886723in}{3.110417in}}%
\pgfpathlineto{\pgfqpoint{0.892075in}{3.106444in}}%
\pgfpathlineto{\pgfqpoint{0.896357in}{3.105878in}}%
\pgfpathlineto{\pgfqpoint{0.900639in}{3.107731in}}%
\pgfpathlineto{\pgfqpoint{0.905723in}{3.112735in}}%
\pgfpathlineto{\pgfqpoint{0.913484in}{3.124181in}}%
\pgfpathlineto{\pgfqpoint{0.924456in}{3.139905in}}%
\pgfpathlineto{\pgfqpoint{0.929808in}{3.143936in}}%
\pgfpathlineto{\pgfqpoint{0.934090in}{3.144555in}}%
\pgfpathlineto{\pgfqpoint{0.938372in}{3.142753in}}%
\pgfpathlineto{\pgfqpoint{0.943456in}{3.137798in}}%
\pgfpathlineto{\pgfqpoint{0.950949in}{3.126818in}}%
\pgfpathlineto{\pgfqpoint{0.962456in}{3.110331in}}%
\pgfpathlineto{\pgfqpoint{0.967809in}{3.106410in}}%
\pgfpathlineto{\pgfqpoint{0.972090in}{3.105891in}}%
\pgfpathlineto{\pgfqpoint{0.976372in}{3.107789in}}%
\pgfpathlineto{\pgfqpoint{0.981456in}{3.112837in}}%
\pgfpathlineto{\pgfqpoint{0.989217in}{3.124313in}}%
\pgfpathlineto{\pgfqpoint{1.000189in}{3.139991in}}%
\pgfpathlineto{\pgfqpoint{1.005541in}{3.143971in}}%
\pgfpathlineto{\pgfqpoint{1.009823in}{3.144543in}}%
\pgfpathlineto{\pgfqpoint{1.014105in}{3.142695in}}%
\pgfpathlineto{\pgfqpoint{1.019189in}{3.137697in}}%
\pgfpathlineto{\pgfqpoint{1.026950in}{3.126255in}}%
\pgfpathlineto{\pgfqpoint{1.038189in}{3.110246in}}%
\pgfpathlineto{\pgfqpoint{1.043542in}{3.106377in}}%
\pgfpathlineto{\pgfqpoint{1.047823in}{3.105905in}}%
\pgfpathlineto{\pgfqpoint{1.052105in}{3.107848in}}%
\pgfpathlineto{\pgfqpoint{1.057190in}{3.112940in}}%
\pgfpathlineto{\pgfqpoint{1.064950in}{3.124446in}}%
\pgfpathlineto{\pgfqpoint{1.075922in}{3.140077in}}%
\pgfpathlineto{\pgfqpoint{1.081274in}{3.144005in}}%
\pgfpathlineto{\pgfqpoint{1.085556in}{3.144530in}}%
\pgfpathlineto{\pgfqpoint{1.089838in}{3.142637in}}%
\pgfpathlineto{\pgfqpoint{1.094922in}{3.137595in}}%
\pgfpathlineto{\pgfqpoint{1.102683in}{3.126122in}}%
\pgfpathlineto{\pgfqpoint{1.113655in}{3.110438in}}%
\pgfpathlineto{\pgfqpoint{1.119007in}{3.106452in}}%
\pgfpathlineto{\pgfqpoint{1.123289in}{3.105874in}}%
\pgfpathlineto{\pgfqpoint{1.127570in}{3.107716in}}%
\pgfpathlineto{\pgfqpoint{1.132655in}{3.112710in}}%
\pgfpathlineto{\pgfqpoint{1.140416in}{3.124147in}}%
\pgfpathlineto{\pgfqpoint{1.151655in}{3.140162in}}%
\pgfpathlineto{\pgfqpoint{1.157007in}{3.144038in}}%
\pgfpathlineto{\pgfqpoint{1.161289in}{3.144516in}}%
\pgfpathlineto{\pgfqpoint{1.165571in}{3.142578in}}%
\pgfpathlineto{\pgfqpoint{1.170655in}{3.137492in}}%
\pgfpathlineto{\pgfqpoint{1.178416in}{3.125989in}}%
\pgfpathlineto{\pgfqpoint{1.189388in}{3.110352in}}%
\pgfpathlineto{\pgfqpoint{1.194740in}{3.106418in}}%
\pgfpathlineto{\pgfqpoint{1.199022in}{3.105887in}}%
\pgfpathlineto{\pgfqpoint{1.203303in}{3.107774in}}%
\pgfpathlineto{\pgfqpoint{1.208388in}{3.112811in}}%
\pgfpathlineto{\pgfqpoint{1.216149in}{3.124280in}}%
\pgfpathlineto{\pgfqpoint{1.227120in}{3.139970in}}%
\pgfpathlineto{\pgfqpoint{1.232473in}{3.143962in}}%
\pgfpathlineto{\pgfqpoint{1.236754in}{3.144546in}}%
\pgfpathlineto{\pgfqpoint{1.241036in}{3.142710in}}%
\pgfpathlineto{\pgfqpoint{1.246121in}{3.137722in}}%
\pgfpathlineto{\pgfqpoint{1.253881in}{3.126288in}}%
\pgfpathlineto{\pgfqpoint{1.265121in}{3.110267in}}%
\pgfpathlineto{\pgfqpoint{1.270473in}{3.106385in}}%
\pgfpathlineto{\pgfqpoint{1.274755in}{3.105901in}}%
\pgfpathlineto{\pgfqpoint{1.279036in}{3.107833in}}%
\pgfpathlineto{\pgfqpoint{1.284121in}{3.112914in}}%
\pgfpathlineto{\pgfqpoint{1.291882in}{3.124413in}}%
\pgfpathlineto{\pgfqpoint{1.302853in}{3.140056in}}%
\pgfpathlineto{\pgfqpoint{1.308206in}{3.143996in}}%
\pgfpathlineto{\pgfqpoint{1.312487in}{3.144533in}}%
\pgfpathlineto{\pgfqpoint{1.316769in}{3.142652in}}%
\pgfpathlineto{\pgfqpoint{1.321854in}{3.137620in}}%
\pgfpathlineto{\pgfqpoint{1.329614in}{3.126155in}}%
\pgfpathlineto{\pgfqpoint{1.340586in}{3.110460in}}%
\pgfpathlineto{\pgfqpoint{1.345938in}{3.106461in}}%
\pgfpathlineto{\pgfqpoint{1.350220in}{3.105871in}}%
\pgfpathlineto{\pgfqpoint{1.354502in}{3.107702in}}%
\pgfpathlineto{\pgfqpoint{1.359586in}{3.112684in}}%
\pgfpathlineto{\pgfqpoint{1.367079in}{3.123684in}}%
\pgfpathlineto{\pgfqpoint{1.378587in}{3.140141in}}%
\pgfpathlineto{\pgfqpoint{1.383939in}{3.144030in}}%
\pgfpathlineto{\pgfqpoint{1.388220in}{3.144520in}}%
\pgfpathlineto{\pgfqpoint{1.392502in}{3.142593in}}%
\pgfpathlineto{\pgfqpoint{1.397587in}{3.137518in}}%
\pgfpathlineto{\pgfqpoint{1.405347in}{3.126023in}}%
\pgfpathlineto{\pgfqpoint{1.416319in}{3.110374in}}%
\pgfpathlineto{\pgfqpoint{1.421671in}{3.106427in}}%
\pgfpathlineto{\pgfqpoint{1.425953in}{3.105884in}}%
\pgfpathlineto{\pgfqpoint{1.430235in}{3.107760in}}%
\pgfpathlineto{\pgfqpoint{1.435319in}{3.112786in}}%
\pgfpathlineto{\pgfqpoint{1.443080in}{3.124247in}}%
\pgfpathlineto{\pgfqpoint{1.454052in}{3.139948in}}%
\pgfpathlineto{\pgfqpoint{1.459404in}{3.143954in}}%
\pgfpathlineto{\pgfqpoint{1.463686in}{3.144549in}}%
\pgfpathlineto{\pgfqpoint{1.467968in}{3.142724in}}%
\pgfpathlineto{\pgfqpoint{1.473052in}{3.137747in}}%
\pgfpathlineto{\pgfqpoint{1.480545in}{3.126752in}}%
\pgfpathlineto{\pgfqpoint{1.492052in}{3.110288in}}%
\pgfpathlineto{\pgfqpoint{1.497404in}{3.106393in}}%
\pgfpathlineto{\pgfqpoint{1.501686in}{3.105898in}}%
\pgfpathlineto{\pgfqpoint{1.505968in}{3.107818in}}%
\pgfpathlineto{\pgfqpoint{1.511052in}{3.112888in}}%
\pgfpathlineto{\pgfqpoint{1.518813in}{3.124380in}}%
\pgfpathlineto{\pgfqpoint{1.529785in}{3.140034in}}%
\pgfpathlineto{\pgfqpoint{1.535137in}{3.143988in}}%
\pgfpathlineto{\pgfqpoint{1.539419in}{3.144537in}}%
\pgfpathlineto{\pgfqpoint{1.543701in}{3.142666in}}%
\pgfpathlineto{\pgfqpoint{1.548785in}{3.137646in}}%
\pgfpathlineto{\pgfqpoint{1.556546in}{3.126188in}}%
\pgfpathlineto{\pgfqpoint{1.567518in}{3.110481in}}%
\pgfpathlineto{\pgfqpoint{1.572870in}{3.106469in}}%
\pgfpathlineto{\pgfqpoint{1.577152in}{3.105868in}}%
\pgfpathlineto{\pgfqpoint{1.581433in}{3.107687in}}%
\pgfpathlineto{\pgfqpoint{1.586518in}{3.112659in}}%
\pgfpathlineto{\pgfqpoint{1.594011in}{3.123651in}}%
\pgfpathlineto{\pgfqpoint{1.605518in}{3.140120in}}%
\pgfpathlineto{\pgfqpoint{1.610870in}{3.144021in}}%
\pgfpathlineto{\pgfqpoint{1.615152in}{3.144523in}}%
\pgfpathlineto{\pgfqpoint{1.619434in}{3.142608in}}%
\pgfpathlineto{\pgfqpoint{1.624518in}{3.137543in}}%
\pgfpathlineto{\pgfqpoint{1.632279in}{3.126056in}}%
\pgfpathlineto{\pgfqpoint{1.643251in}{3.110395in}}%
\pgfpathlineto{\pgfqpoint{1.648603in}{3.106435in}}%
\pgfpathlineto{\pgfqpoint{1.652885in}{3.105881in}}%
\pgfpathlineto{\pgfqpoint{1.657166in}{3.107745in}}%
\pgfpathlineto{\pgfqpoint{1.662251in}{3.112760in}}%
\pgfpathlineto{\pgfqpoint{1.670011in}{3.124214in}}%
\pgfpathlineto{\pgfqpoint{1.680983in}{3.139927in}}%
\pgfpathlineto{\pgfqpoint{1.686335in}{3.143945in}}%
\pgfpathlineto{\pgfqpoint{1.690617in}{3.144552in}}%
\pgfpathlineto{\pgfqpoint{1.694899in}{3.142739in}}%
\pgfpathlineto{\pgfqpoint{1.699983in}{3.137773in}}%
\pgfpathlineto{\pgfqpoint{1.707477in}{3.126785in}}%
\pgfpathlineto{\pgfqpoint{1.718984in}{3.110310in}}%
\pgfpathlineto{\pgfqpoint{1.724336in}{3.106402in}}%
\pgfpathlineto{\pgfqpoint{1.728618in}{3.105894in}}%
\pgfpathlineto{\pgfqpoint{1.732899in}{3.107804in}}%
\pgfpathlineto{\pgfqpoint{1.737984in}{3.112863in}}%
\pgfpathlineto{\pgfqpoint{1.745744in}{3.124346in}}%
\pgfpathlineto{\pgfqpoint{1.756716in}{3.140013in}}%
\pgfpathlineto{\pgfqpoint{1.762069in}{3.143980in}}%
\pgfpathlineto{\pgfqpoint{1.766350in}{3.144540in}}%
\pgfpathlineto{\pgfqpoint{1.770632in}{3.142681in}}%
\pgfpathlineto{\pgfqpoint{1.775717in}{3.137671in}}%
\pgfpathlineto{\pgfqpoint{1.783477in}{3.126222in}}%
\pgfpathlineto{\pgfqpoint{1.794449in}{3.110503in}}%
\pgfpathlineto{\pgfqpoint{1.799801in}{3.106478in}}%
\pgfpathlineto{\pgfqpoint{1.804083in}{3.105865in}}%
\pgfpathlineto{\pgfqpoint{1.808365in}{3.107673in}}%
\pgfpathlineto{\pgfqpoint{1.813449in}{3.112633in}}%
\pgfpathlineto{\pgfqpoint{1.820942in}{3.123617in}}%
\pgfpathlineto{\pgfqpoint{1.832449in}{3.140099in}}%
\pgfpathlineto{\pgfqpoint{1.837802in}{3.144013in}}%
\pgfpathlineto{\pgfqpoint{1.842083in}{3.144527in}}%
\pgfpathlineto{\pgfqpoint{1.846365in}{3.142623in}}%
\pgfpathlineto{\pgfqpoint{1.851450in}{3.137569in}}%
\pgfpathlineto{\pgfqpoint{1.859210in}{3.126089in}}%
\pgfpathlineto{\pgfqpoint{1.870182in}{3.110417in}}%
\pgfpathlineto{\pgfqpoint{1.875534in}{3.106444in}}%
\pgfpathlineto{\pgfqpoint{1.879816in}{3.105878in}}%
\pgfpathlineto{\pgfqpoint{1.884098in}{3.107731in}}%
\pgfpathlineto{\pgfqpoint{1.889182in}{3.112735in}}%
\pgfpathlineto{\pgfqpoint{1.896943in}{3.124181in}}%
\pgfpathlineto{\pgfqpoint{1.907915in}{3.139905in}}%
\pgfpathlineto{\pgfqpoint{1.913267in}{3.143936in}}%
\pgfpathlineto{\pgfqpoint{1.917549in}{3.144555in}}%
\pgfpathlineto{\pgfqpoint{1.921830in}{3.142753in}}%
\pgfpathlineto{\pgfqpoint{1.926915in}{3.137798in}}%
\pgfpathlineto{\pgfqpoint{1.934408in}{3.126818in}}%
\pgfpathlineto{\pgfqpoint{1.945915in}{3.110331in}}%
\pgfpathlineto{\pgfqpoint{1.951267in}{3.106410in}}%
\pgfpathlineto{\pgfqpoint{1.955549in}{3.105891in}}%
\pgfpathlineto{\pgfqpoint{1.959831in}{3.107789in}}%
\pgfpathlineto{\pgfqpoint{1.964915in}{3.112837in}}%
\pgfpathlineto{\pgfqpoint{1.972676in}{3.124313in}}%
\pgfpathlineto{\pgfqpoint{1.983648in}{3.139991in}}%
\pgfpathlineto{\pgfqpoint{1.989000in}{3.143971in}}%
\pgfpathlineto{\pgfqpoint{1.993282in}{3.144543in}}%
\pgfpathlineto{\pgfqpoint{1.997563in}{3.142695in}}%
\pgfpathlineto{\pgfqpoint{2.002648in}{3.137697in}}%
\pgfpathlineto{\pgfqpoint{2.010409in}{3.126255in}}%
\pgfpathlineto{\pgfqpoint{2.021648in}{3.110246in}}%
\pgfpathlineto{\pgfqpoint{2.027000in}{3.106377in}}%
\pgfpathlineto{\pgfqpoint{2.031282in}{3.105905in}}%
\pgfpathlineto{\pgfqpoint{2.035564in}{3.107848in}}%
\pgfpathlineto{\pgfqpoint{2.040648in}{3.112940in}}%
\pgfpathlineto{\pgfqpoint{2.048409in}{3.124446in}}%
\pgfpathlineto{\pgfqpoint{2.059381in}{3.140077in}}%
\pgfpathlineto{\pgfqpoint{2.064733in}{3.144005in}}%
\pgfpathlineto{\pgfqpoint{2.069015in}{3.144530in}}%
\pgfpathlineto{\pgfqpoint{2.073296in}{3.142637in}}%
\pgfpathlineto{\pgfqpoint{2.078381in}{3.137595in}}%
\pgfpathlineto{\pgfqpoint{2.086142in}{3.126122in}}%
\pgfpathlineto{\pgfqpoint{2.097114in}{3.110438in}}%
\pgfpathlineto{\pgfqpoint{2.102466in}{3.106452in}}%
\pgfpathlineto{\pgfqpoint{2.106747in}{3.105874in}}%
\pgfpathlineto{\pgfqpoint{2.111029in}{3.107716in}}%
\pgfpathlineto{\pgfqpoint{2.116114in}{3.112710in}}%
\pgfpathlineto{\pgfqpoint{2.123874in}{3.124147in}}%
\pgfpathlineto{\pgfqpoint{2.135114in}{3.140162in}}%
\pgfpathlineto{\pgfqpoint{2.140466in}{3.144038in}}%
\pgfpathlineto{\pgfqpoint{2.144748in}{3.144516in}}%
\pgfpathlineto{\pgfqpoint{2.149029in}{3.142578in}}%
\pgfpathlineto{\pgfqpoint{2.154114in}{3.137492in}}%
\pgfpathlineto{\pgfqpoint{2.161875in}{3.125989in}}%
\pgfpathlineto{\pgfqpoint{2.172847in}{3.110352in}}%
\pgfpathlineto{\pgfqpoint{2.178199in}{3.106418in}}%
\pgfpathlineto{\pgfqpoint{2.182480in}{3.105887in}}%
\pgfpathlineto{\pgfqpoint{2.186762in}{3.107774in}}%
\pgfpathlineto{\pgfqpoint{2.191847in}{3.112811in}}%
\pgfpathlineto{\pgfqpoint{2.199607in}{3.124280in}}%
\pgfpathlineto{\pgfqpoint{2.210579in}{3.139970in}}%
\pgfpathlineto{\pgfqpoint{2.215931in}{3.143962in}}%
\pgfpathlineto{\pgfqpoint{2.220213in}{3.144546in}}%
\pgfpathlineto{\pgfqpoint{2.224495in}{3.142710in}}%
\pgfpathlineto{\pgfqpoint{2.229579in}{3.137722in}}%
\pgfpathlineto{\pgfqpoint{2.237340in}{3.126288in}}%
\pgfpathlineto{\pgfqpoint{2.248580in}{3.110267in}}%
\pgfpathlineto{\pgfqpoint{2.253932in}{3.106385in}}%
\pgfpathlineto{\pgfqpoint{2.258213in}{3.105901in}}%
\pgfpathlineto{\pgfqpoint{2.262495in}{3.107833in}}%
\pgfpathlineto{\pgfqpoint{2.267580in}{3.112914in}}%
\pgfpathlineto{\pgfqpoint{2.275340in}{3.124413in}}%
\pgfpathlineto{\pgfqpoint{2.286312in}{3.140056in}}%
\pgfpathlineto{\pgfqpoint{2.291664in}{3.143996in}}%
\pgfpathlineto{\pgfqpoint{2.295946in}{3.144533in}}%
\pgfpathlineto{\pgfqpoint{2.300228in}{3.142652in}}%
\pgfpathlineto{\pgfqpoint{2.305312in}{3.137620in}}%
\pgfpathlineto{\pgfqpoint{2.313073in}{3.126155in}}%
\pgfpathlineto{\pgfqpoint{2.324045in}{3.110460in}}%
\pgfpathlineto{\pgfqpoint{2.329397in}{3.106461in}}%
\pgfpathlineto{\pgfqpoint{2.333679in}{3.105871in}}%
\pgfpathlineto{\pgfqpoint{2.337961in}{3.107702in}}%
\pgfpathlineto{\pgfqpoint{2.343045in}{3.112684in}}%
\pgfpathlineto{\pgfqpoint{2.350538in}{3.123684in}}%
\pgfpathlineto{\pgfqpoint{2.362045in}{3.140141in}}%
\pgfpathlineto{\pgfqpoint{2.367397in}{3.144030in}}%
\pgfpathlineto{\pgfqpoint{2.371679in}{3.144520in}}%
\pgfpathlineto{\pgfqpoint{2.375961in}{3.142593in}}%
\pgfpathlineto{\pgfqpoint{2.381045in}{3.137518in}}%
\pgfpathlineto{\pgfqpoint{2.388806in}{3.126023in}}%
\pgfpathlineto{\pgfqpoint{2.399778in}{3.110374in}}%
\pgfpathlineto{\pgfqpoint{2.405130in}{3.106427in}}%
\pgfpathlineto{\pgfqpoint{2.409412in}{3.105884in}}%
\pgfpathlineto{\pgfqpoint{2.413694in}{3.107760in}}%
\pgfpathlineto{\pgfqpoint{2.418778in}{3.112786in}}%
\pgfpathlineto{\pgfqpoint{2.426539in}{3.124247in}}%
\pgfpathlineto{\pgfqpoint{2.437511in}{3.139948in}}%
\pgfpathlineto{\pgfqpoint{2.442863in}{3.143954in}}%
\pgfpathlineto{\pgfqpoint{2.447145in}{3.144549in}}%
\pgfpathlineto{\pgfqpoint{2.451426in}{3.142724in}}%
\pgfpathlineto{\pgfqpoint{2.456511in}{3.137747in}}%
\pgfpathlineto{\pgfqpoint{2.464004in}{3.126752in}}%
\pgfpathlineto{\pgfqpoint{2.475511in}{3.110288in}}%
\pgfpathlineto{\pgfqpoint{2.480863in}{3.106393in}}%
\pgfpathlineto{\pgfqpoint{2.485145in}{3.105898in}}%
\pgfpathlineto{\pgfqpoint{2.489427in}{3.107818in}}%
\pgfpathlineto{\pgfqpoint{2.494511in}{3.112888in}}%
\pgfpathlineto{\pgfqpoint{2.502272in}{3.124380in}}%
\pgfpathlineto{\pgfqpoint{2.513244in}{3.140034in}}%
\pgfpathlineto{\pgfqpoint{2.518596in}{3.143988in}}%
\pgfpathlineto{\pgfqpoint{2.522878in}{3.144537in}}%
\pgfpathlineto{\pgfqpoint{2.527159in}{3.142666in}}%
\pgfpathlineto{\pgfqpoint{2.532244in}{3.137646in}}%
\pgfpathlineto{\pgfqpoint{2.540004in}{3.126188in}}%
\pgfpathlineto{\pgfqpoint{2.550976in}{3.110481in}}%
\pgfpathlineto{\pgfqpoint{2.556329in}{3.106469in}}%
\pgfpathlineto{\pgfqpoint{2.560610in}{3.105868in}}%
\pgfpathlineto{\pgfqpoint{2.564892in}{3.107687in}}%
\pgfpathlineto{\pgfqpoint{2.569977in}{3.112659in}}%
\pgfpathlineto{\pgfqpoint{2.577470in}{3.123651in}}%
\pgfpathlineto{\pgfqpoint{2.588977in}{3.140120in}}%
\pgfpathlineto{\pgfqpoint{2.594329in}{3.144021in}}%
\pgfpathlineto{\pgfqpoint{2.598611in}{3.144523in}}%
\pgfpathlineto{\pgfqpoint{2.602892in}{3.142608in}}%
\pgfpathlineto{\pgfqpoint{2.607977in}{3.137543in}}%
\pgfpathlineto{\pgfqpoint{2.615737in}{3.126056in}}%
\pgfpathlineto{\pgfqpoint{2.626709in}{3.110395in}}%
\pgfpathlineto{\pgfqpoint{2.632062in}{3.106435in}}%
\pgfpathlineto{\pgfqpoint{2.636343in}{3.105881in}}%
\pgfpathlineto{\pgfqpoint{2.640625in}{3.107745in}}%
\pgfpathlineto{\pgfqpoint{2.645710in}{3.112760in}}%
\pgfpathlineto{\pgfqpoint{2.653470in}{3.124214in}}%
\pgfpathlineto{\pgfqpoint{2.664442in}{3.139927in}}%
\pgfpathlineto{\pgfqpoint{2.669794in}{3.143945in}}%
\pgfpathlineto{\pgfqpoint{2.674076in}{3.144552in}}%
\pgfpathlineto{\pgfqpoint{2.678358in}{3.142739in}}%
\pgfpathlineto{\pgfqpoint{2.683442in}{3.137773in}}%
\pgfpathlineto{\pgfqpoint{2.690935in}{3.126785in}}%
\pgfpathlineto{\pgfqpoint{2.702442in}{3.110310in}}%
\pgfpathlineto{\pgfqpoint{2.707795in}{3.106402in}}%
\pgfpathlineto{\pgfqpoint{2.712076in}{3.105894in}}%
\pgfpathlineto{\pgfqpoint{2.716358in}{3.107804in}}%
\pgfpathlineto{\pgfqpoint{2.721443in}{3.112863in}}%
\pgfpathlineto{\pgfqpoint{2.729203in}{3.124346in}}%
\pgfpathlineto{\pgfqpoint{2.740175in}{3.140013in}}%
\pgfpathlineto{\pgfqpoint{2.745527in}{3.143980in}}%
\pgfpathlineto{\pgfqpoint{2.749809in}{3.144540in}}%
\pgfpathlineto{\pgfqpoint{2.754091in}{3.142681in}}%
\pgfpathlineto{\pgfqpoint{2.759175in}{3.137671in}}%
\pgfpathlineto{\pgfqpoint{2.766936in}{3.126222in}}%
\pgfpathlineto{\pgfqpoint{2.777908in}{3.110503in}}%
\pgfpathlineto{\pgfqpoint{2.783260in}{3.106478in}}%
\pgfpathlineto{\pgfqpoint{2.787542in}{3.105865in}}%
\pgfpathlineto{\pgfqpoint{2.791823in}{3.107673in}}%
\pgfpathlineto{\pgfqpoint{2.796908in}{3.112633in}}%
\pgfpathlineto{\pgfqpoint{2.804401in}{3.123617in}}%
\pgfpathlineto{\pgfqpoint{2.815908in}{3.140099in}}%
\pgfpathlineto{\pgfqpoint{2.821260in}{3.144013in}}%
\pgfpathlineto{\pgfqpoint{2.825542in}{3.144527in}}%
\pgfpathlineto{\pgfqpoint{2.829824in}{3.142623in}}%
\pgfpathlineto{\pgfqpoint{2.834908in}{3.137569in}}%
\pgfpathlineto{\pgfqpoint{2.842669in}{3.126089in}}%
\pgfpathlineto{\pgfqpoint{2.853641in}{3.110417in}}%
\pgfpathlineto{\pgfqpoint{2.858993in}{3.106444in}}%
\pgfpathlineto{\pgfqpoint{2.863275in}{3.105878in}}%
\pgfpathlineto{\pgfqpoint{2.867556in}{3.107731in}}%
\pgfpathlineto{\pgfqpoint{2.872641in}{3.112735in}}%
\pgfpathlineto{\pgfqpoint{2.880402in}{3.124181in}}%
\pgfpathlineto{\pgfqpoint{2.891374in}{3.139905in}}%
\pgfpathlineto{\pgfqpoint{2.896726in}{3.143936in}}%
\pgfpathlineto{\pgfqpoint{2.901007in}{3.144555in}}%
\pgfpathlineto{\pgfqpoint{2.905289in}{3.142753in}}%
\pgfpathlineto{\pgfqpoint{2.910374in}{3.137798in}}%
\pgfpathlineto{\pgfqpoint{2.917867in}{3.126818in}}%
\pgfpathlineto{\pgfqpoint{2.929374in}{3.110331in}}%
\pgfpathlineto{\pgfqpoint{2.934726in}{3.106410in}}%
\pgfpathlineto{\pgfqpoint{2.939008in}{3.105891in}}%
\pgfpathlineto{\pgfqpoint{2.943289in}{3.107789in}}%
\pgfpathlineto{\pgfqpoint{2.948374in}{3.112837in}}%
\pgfpathlineto{\pgfqpoint{2.956135in}{3.124313in}}%
\pgfpathlineto{\pgfqpoint{2.967107in}{3.139991in}}%
\pgfpathlineto{\pgfqpoint{2.972459in}{3.143971in}}%
\pgfpathlineto{\pgfqpoint{2.976740in}{3.144543in}}%
\pgfpathlineto{\pgfqpoint{2.981022in}{3.142695in}}%
\pgfpathlineto{\pgfqpoint{2.986107in}{3.137697in}}%
\pgfpathlineto{\pgfqpoint{2.993867in}{3.126255in}}%
\pgfpathlineto{\pgfqpoint{3.005107in}{3.110246in}}%
\pgfpathlineto{\pgfqpoint{3.010459in}{3.106377in}}%
\pgfpathlineto{\pgfqpoint{3.014741in}{3.105905in}}%
\pgfpathlineto{\pgfqpoint{3.019022in}{3.107848in}}%
\pgfpathlineto{\pgfqpoint{3.024107in}{3.112940in}}%
\pgfpathlineto{\pgfqpoint{3.031868in}{3.124446in}}%
\pgfpathlineto{\pgfqpoint{3.042840in}{3.140077in}}%
\pgfpathlineto{\pgfqpoint{3.048192in}{3.144005in}}%
\pgfpathlineto{\pgfqpoint{3.052473in}{3.144530in}}%
\pgfpathlineto{\pgfqpoint{3.056755in}{3.142637in}}%
\pgfpathlineto{\pgfqpoint{3.061840in}{3.137595in}}%
\pgfpathlineto{\pgfqpoint{3.069600in}{3.126122in}}%
\pgfpathlineto{\pgfqpoint{3.080572in}{3.110438in}}%
\pgfpathlineto{\pgfqpoint{3.085924in}{3.106452in}}%
\pgfpathlineto{\pgfqpoint{3.090206in}{3.105874in}}%
\pgfpathlineto{\pgfqpoint{3.094488in}{3.107716in}}%
\pgfpathlineto{\pgfqpoint{3.099572in}{3.112710in}}%
\pgfpathlineto{\pgfqpoint{3.107333in}{3.124147in}}%
\pgfpathlineto{\pgfqpoint{3.118573in}{3.140162in}}%
\pgfpathlineto{\pgfqpoint{3.123925in}{3.144038in}}%
\pgfpathlineto{\pgfqpoint{3.128206in}{3.144516in}}%
\pgfpathlineto{\pgfqpoint{3.132488in}{3.142578in}}%
\pgfpathlineto{\pgfqpoint{3.137573in}{3.137492in}}%
\pgfpathlineto{\pgfqpoint{3.145333in}{3.125989in}}%
\pgfpathlineto{\pgfqpoint{3.156305in}{3.110352in}}%
\pgfpathlineto{\pgfqpoint{3.161657in}{3.106418in}}%
\pgfpathlineto{\pgfqpoint{3.165939in}{3.105887in}}%
\pgfpathlineto{\pgfqpoint{3.170221in}{3.107774in}}%
\pgfpathlineto{\pgfqpoint{3.175305in}{3.112811in}}%
\pgfpathlineto{\pgfqpoint{3.183066in}{3.124280in}}%
\pgfpathlineto{\pgfqpoint{3.194038in}{3.139970in}}%
\pgfpathlineto{\pgfqpoint{3.199390in}{3.143962in}}%
\pgfpathlineto{\pgfqpoint{3.203672in}{3.144546in}}%
\pgfpathlineto{\pgfqpoint{3.207954in}{3.142710in}}%
\pgfpathlineto{\pgfqpoint{3.213038in}{3.137722in}}%
\pgfpathlineto{\pgfqpoint{3.220799in}{3.126288in}}%
\pgfpathlineto{\pgfqpoint{3.232038in}{3.110267in}}%
\pgfpathlineto{\pgfqpoint{3.237390in}{3.106385in}}%
\pgfpathlineto{\pgfqpoint{3.241672in}{3.105901in}}%
\pgfpathlineto{\pgfqpoint{3.245954in}{3.107833in}}%
\pgfpathlineto{\pgfqpoint{3.251038in}{3.112914in}}%
\pgfpathlineto{\pgfqpoint{3.258799in}{3.124413in}}%
\pgfpathlineto{\pgfqpoint{3.269771in}{3.140056in}}%
\pgfpathlineto{\pgfqpoint{3.275123in}{3.143996in}}%
\pgfpathlineto{\pgfqpoint{3.279405in}{3.144533in}}%
\pgfpathlineto{\pgfqpoint{3.283687in}{3.142652in}}%
\pgfpathlineto{\pgfqpoint{3.288771in}{3.137620in}}%
\pgfpathlineto{\pgfqpoint{3.296532in}{3.126155in}}%
\pgfpathlineto{\pgfqpoint{3.307504in}{3.110460in}}%
\pgfpathlineto{\pgfqpoint{3.312856in}{3.106461in}}%
\pgfpathlineto{\pgfqpoint{3.317138in}{3.105871in}}%
\pgfpathlineto{\pgfqpoint{3.321419in}{3.107702in}}%
\pgfpathlineto{\pgfqpoint{3.326504in}{3.112684in}}%
\pgfpathlineto{\pgfqpoint{3.333997in}{3.123684in}}%
\pgfpathlineto{\pgfqpoint{3.345504in}{3.140141in}}%
\pgfpathlineto{\pgfqpoint{3.350856in}{3.144030in}}%
\pgfpathlineto{\pgfqpoint{3.355138in}{3.144520in}}%
\pgfpathlineto{\pgfqpoint{3.359420in}{3.142593in}}%
\pgfpathlineto{\pgfqpoint{3.364504in}{3.137518in}}%
\pgfpathlineto{\pgfqpoint{3.372265in}{3.126023in}}%
\pgfpathlineto{\pgfqpoint{3.383237in}{3.110374in}}%
\pgfpathlineto{\pgfqpoint{3.388589in}{3.106427in}}%
\pgfpathlineto{\pgfqpoint{3.392871in}{3.105884in}}%
\pgfpathlineto{\pgfqpoint{3.397152in}{3.107760in}}%
\pgfpathlineto{\pgfqpoint{3.402237in}{3.112786in}}%
\pgfpathlineto{\pgfqpoint{3.409997in}{3.124247in}}%
\pgfpathlineto{\pgfqpoint{3.420969in}{3.139948in}}%
\pgfpathlineto{\pgfqpoint{3.426322in}{3.143954in}}%
\pgfpathlineto{\pgfqpoint{3.430603in}{3.144549in}}%
\pgfpathlineto{\pgfqpoint{3.434885in}{3.142724in}}%
\pgfpathlineto{\pgfqpoint{3.439970in}{3.137747in}}%
\pgfpathlineto{\pgfqpoint{3.447463in}{3.126752in}}%
\pgfpathlineto{\pgfqpoint{3.458970in}{3.110288in}}%
\pgfpathlineto{\pgfqpoint{3.464322in}{3.106393in}}%
\pgfpathlineto{\pgfqpoint{3.468604in}{3.105898in}}%
\pgfpathlineto{\pgfqpoint{3.472885in}{3.107818in}}%
\pgfpathlineto{\pgfqpoint{3.477970in}{3.112888in}}%
\pgfpathlineto{\pgfqpoint{3.485730in}{3.124380in}}%
\pgfpathlineto{\pgfqpoint{3.496702in}{3.140034in}}%
\pgfpathlineto{\pgfqpoint{3.502055in}{3.143988in}}%
\pgfpathlineto{\pgfqpoint{3.506336in}{3.144537in}}%
\pgfpathlineto{\pgfqpoint{3.510618in}{3.142666in}}%
\pgfpathlineto{\pgfqpoint{3.515703in}{3.137646in}}%
\pgfpathlineto{\pgfqpoint{3.523463in}{3.126188in}}%
\pgfpathlineto{\pgfqpoint{3.534435in}{3.110481in}}%
\pgfpathlineto{\pgfqpoint{3.539787in}{3.106469in}}%
\pgfpathlineto{\pgfqpoint{3.544069in}{3.105868in}}%
\pgfpathlineto{\pgfqpoint{3.548351in}{3.107687in}}%
\pgfpathlineto{\pgfqpoint{3.553435in}{3.112659in}}%
\pgfpathlineto{\pgfqpoint{3.560928in}{3.123651in}}%
\pgfpathlineto{\pgfqpoint{3.572435in}{3.140120in}}%
\pgfpathlineto{\pgfqpoint{3.577788in}{3.144021in}}%
\pgfpathlineto{\pgfqpoint{3.582069in}{3.144523in}}%
\pgfpathlineto{\pgfqpoint{3.586351in}{3.142608in}}%
\pgfpathlineto{\pgfqpoint{3.591436in}{3.137543in}}%
\pgfpathlineto{\pgfqpoint{3.599196in}{3.126056in}}%
\pgfpathlineto{\pgfqpoint{3.610168in}{3.110395in}}%
\pgfpathlineto{\pgfqpoint{3.615520in}{3.106435in}}%
\pgfpathlineto{\pgfqpoint{3.619802in}{3.105881in}}%
\pgfpathlineto{\pgfqpoint{3.624084in}{3.107745in}}%
\pgfpathlineto{\pgfqpoint{3.629168in}{3.112760in}}%
\pgfpathlineto{\pgfqpoint{3.636929in}{3.124214in}}%
\pgfpathlineto{\pgfqpoint{3.647901in}{3.139927in}}%
\pgfpathlineto{\pgfqpoint{3.653253in}{3.143945in}}%
\pgfpathlineto{\pgfqpoint{3.657535in}{3.144552in}}%
\pgfpathlineto{\pgfqpoint{3.661816in}{3.142739in}}%
\pgfpathlineto{\pgfqpoint{3.666901in}{3.137773in}}%
\pgfpathlineto{\pgfqpoint{3.674394in}{3.126785in}}%
\pgfpathlineto{\pgfqpoint{3.685901in}{3.110310in}}%
\pgfpathlineto{\pgfqpoint{3.691253in}{3.106402in}}%
\pgfpathlineto{\pgfqpoint{3.695535in}{3.105894in}}%
\pgfpathlineto{\pgfqpoint{3.699817in}{3.107804in}}%
\pgfpathlineto{\pgfqpoint{3.704901in}{3.112863in}}%
\pgfpathlineto{\pgfqpoint{3.712662in}{3.124346in}}%
\pgfpathlineto{\pgfqpoint{3.723634in}{3.140013in}}%
\pgfpathlineto{\pgfqpoint{3.728986in}{3.143980in}}%
\pgfpathlineto{\pgfqpoint{3.733268in}{3.144540in}}%
\pgfpathlineto{\pgfqpoint{3.737549in}{3.142681in}}%
\pgfpathlineto{\pgfqpoint{3.742634in}{3.137671in}}%
\pgfpathlineto{\pgfqpoint{3.750395in}{3.126222in}}%
\pgfpathlineto{\pgfqpoint{3.761367in}{3.110503in}}%
\pgfpathlineto{\pgfqpoint{3.766719in}{3.106478in}}%
\pgfpathlineto{\pgfqpoint{3.771000in}{3.105865in}}%
\pgfpathlineto{\pgfqpoint{3.775282in}{3.107673in}}%
\pgfpathlineto{\pgfqpoint{3.780367in}{3.112633in}}%
\pgfpathlineto{\pgfqpoint{3.787860in}{3.123617in}}%
\pgfpathlineto{\pgfqpoint{3.799367in}{3.140099in}}%
\pgfpathlineto{\pgfqpoint{3.804719in}{3.144013in}}%
\pgfpathlineto{\pgfqpoint{3.809001in}{3.144527in}}%
\pgfpathlineto{\pgfqpoint{3.813282in}{3.142623in}}%
\pgfpathlineto{\pgfqpoint{3.818367in}{3.137569in}}%
\pgfpathlineto{\pgfqpoint{3.826128in}{3.126089in}}%
\pgfpathlineto{\pgfqpoint{3.837100in}{3.110417in}}%
\pgfpathlineto{\pgfqpoint{3.842452in}{3.106444in}}%
\pgfpathlineto{\pgfqpoint{3.846733in}{3.105878in}}%
\pgfpathlineto{\pgfqpoint{3.851015in}{3.107731in}}%
\pgfpathlineto{\pgfqpoint{3.856100in}{3.112735in}}%
\pgfpathlineto{\pgfqpoint{3.863860in}{3.124181in}}%
\pgfpathlineto{\pgfqpoint{3.874832in}{3.139905in}}%
\pgfpathlineto{\pgfqpoint{3.880184in}{3.143936in}}%
\pgfpathlineto{\pgfqpoint{3.884466in}{3.144555in}}%
\pgfpathlineto{\pgfqpoint{3.888748in}{3.142753in}}%
\pgfpathlineto{\pgfqpoint{3.893832in}{3.137798in}}%
\pgfpathlineto{\pgfqpoint{3.901325in}{3.126818in}}%
\pgfpathlineto{\pgfqpoint{3.912833in}{3.110331in}}%
\pgfpathlineto{\pgfqpoint{3.918185in}{3.106410in}}%
\pgfpathlineto{\pgfqpoint{3.922466in}{3.105891in}}%
\pgfpathlineto{\pgfqpoint{3.926748in}{3.107789in}}%
\pgfpathlineto{\pgfqpoint{3.931833in}{3.112837in}}%
\pgfpathlineto{\pgfqpoint{3.939593in}{3.124313in}}%
\pgfpathlineto{\pgfqpoint{3.950565in}{3.139991in}}%
\pgfpathlineto{\pgfqpoint{3.955917in}{3.143971in}}%
\pgfpathlineto{\pgfqpoint{3.960199in}{3.144543in}}%
\pgfpathlineto{\pgfqpoint{3.964481in}{3.142695in}}%
\pgfpathlineto{\pgfqpoint{3.969565in}{3.137697in}}%
\pgfpathlineto{\pgfqpoint{3.977326in}{3.126255in}}%
\pgfpathlineto{\pgfqpoint{3.988566in}{3.110246in}}%
\pgfpathlineto{\pgfqpoint{3.993918in}{3.106377in}}%
\pgfpathlineto{\pgfqpoint{3.998199in}{3.105905in}}%
\pgfpathlineto{\pgfqpoint{4.002481in}{3.107848in}}%
\pgfpathlineto{\pgfqpoint{4.007566in}{3.112940in}}%
\pgfpathlineto{\pgfqpoint{4.015326in}{3.124446in}}%
\pgfpathlineto{\pgfqpoint{4.026298in}{3.140077in}}%
\pgfpathlineto{\pgfqpoint{4.031650in}{3.144005in}}%
\pgfpathlineto{\pgfqpoint{4.035932in}{3.144530in}}%
\pgfpathlineto{\pgfqpoint{4.040214in}{3.142637in}}%
\pgfpathlineto{\pgfqpoint{4.045298in}{3.137595in}}%
\pgfpathlineto{\pgfqpoint{4.053059in}{3.126122in}}%
\pgfpathlineto{\pgfqpoint{4.064031in}{3.110438in}}%
\pgfpathlineto{\pgfqpoint{4.069383in}{3.106452in}}%
\pgfpathlineto{\pgfqpoint{4.073665in}{3.105874in}}%
\pgfpathlineto{\pgfqpoint{4.077947in}{3.107716in}}%
\pgfpathlineto{\pgfqpoint{4.083031in}{3.112710in}}%
\pgfpathlineto{\pgfqpoint{4.090792in}{3.124147in}}%
\pgfpathlineto{\pgfqpoint{4.102031in}{3.140162in}}%
\pgfpathlineto{\pgfqpoint{4.107383in}{3.144038in}}%
\pgfpathlineto{\pgfqpoint{4.111665in}{3.144516in}}%
\pgfpathlineto{\pgfqpoint{4.115947in}{3.142578in}}%
\pgfpathlineto{\pgfqpoint{4.121031in}{3.137492in}}%
\pgfpathlineto{\pgfqpoint{4.128792in}{3.125989in}}%
\pgfpathlineto{\pgfqpoint{4.139764in}{3.110352in}}%
\pgfpathlineto{\pgfqpoint{4.145116in}{3.106418in}}%
\pgfpathlineto{\pgfqpoint{4.149398in}{3.105887in}}%
\pgfpathlineto{\pgfqpoint{4.153680in}{3.107774in}}%
\pgfpathlineto{\pgfqpoint{4.158764in}{3.112811in}}%
\pgfpathlineto{\pgfqpoint{4.166525in}{3.124280in}}%
\pgfpathlineto{\pgfqpoint{4.177497in}{3.139970in}}%
\pgfpathlineto{\pgfqpoint{4.182849in}{3.143962in}}%
\pgfpathlineto{\pgfqpoint{4.187131in}{3.144546in}}%
\pgfpathlineto{\pgfqpoint{4.191412in}{3.142710in}}%
\pgfpathlineto{\pgfqpoint{4.196497in}{3.137722in}}%
\pgfpathlineto{\pgfqpoint{4.204257in}{3.126288in}}%
\pgfpathlineto{\pgfqpoint{4.215497in}{3.110267in}}%
\pgfpathlineto{\pgfqpoint{4.220849in}{3.106385in}}%
\pgfpathlineto{\pgfqpoint{4.225131in}{3.105901in}}%
\pgfpathlineto{\pgfqpoint{4.229413in}{3.107833in}}%
\pgfpathlineto{\pgfqpoint{4.234497in}{3.112914in}}%
\pgfpathlineto{\pgfqpoint{4.242258in}{3.124413in}}%
\pgfpathlineto{\pgfqpoint{4.253230in}{3.140056in}}%
\pgfpathlineto{\pgfqpoint{4.258582in}{3.143996in}}%
\pgfpathlineto{\pgfqpoint{4.262864in}{3.144533in}}%
\pgfpathlineto{\pgfqpoint{4.267145in}{3.142652in}}%
\pgfpathlineto{\pgfqpoint{4.272230in}{3.137620in}}%
\pgfpathlineto{\pgfqpoint{4.279991in}{3.126155in}}%
\pgfpathlineto{\pgfqpoint{4.290962in}{3.110460in}}%
\pgfpathlineto{\pgfqpoint{4.296315in}{3.106461in}}%
\pgfpathlineto{\pgfqpoint{4.300596in}{3.105871in}}%
\pgfpathlineto{\pgfqpoint{4.304878in}{3.107702in}}%
\pgfpathlineto{\pgfqpoint{4.309963in}{3.112684in}}%
\pgfpathlineto{\pgfqpoint{4.317456in}{3.123684in}}%
\pgfpathlineto{\pgfqpoint{4.328963in}{3.140141in}}%
\pgfpathlineto{\pgfqpoint{4.334315in}{3.144030in}}%
\pgfpathlineto{\pgfqpoint{4.338597in}{3.144520in}}%
\pgfpathlineto{\pgfqpoint{4.342878in}{3.142593in}}%
\pgfpathlineto{\pgfqpoint{4.347963in}{3.137518in}}%
\pgfpathlineto{\pgfqpoint{4.355724in}{3.126023in}}%
\pgfpathlineto{\pgfqpoint{4.366695in}{3.110374in}}%
\pgfpathlineto{\pgfqpoint{4.372048in}{3.106427in}}%
\pgfpathlineto{\pgfqpoint{4.376329in}{3.105884in}}%
\pgfpathlineto{\pgfqpoint{4.380611in}{3.107760in}}%
\pgfpathlineto{\pgfqpoint{4.385696in}{3.112786in}}%
\pgfpathlineto{\pgfqpoint{4.393456in}{3.124247in}}%
\pgfpathlineto{\pgfqpoint{4.404428in}{3.139948in}}%
\pgfpathlineto{\pgfqpoint{4.409780in}{3.143954in}}%
\pgfpathlineto{\pgfqpoint{4.414062in}{3.144549in}}%
\pgfpathlineto{\pgfqpoint{4.418344in}{3.142724in}}%
\pgfpathlineto{\pgfqpoint{4.423428in}{3.137747in}}%
\pgfpathlineto{\pgfqpoint{4.430921in}{3.126752in}}%
\pgfpathlineto{\pgfqpoint{4.442428in}{3.110288in}}%
\pgfpathlineto{\pgfqpoint{4.447781in}{3.106393in}}%
\pgfpathlineto{\pgfqpoint{4.452062in}{3.105898in}}%
\pgfpathlineto{\pgfqpoint{4.456344in}{3.107818in}}%
\pgfpathlineto{\pgfqpoint{4.461429in}{3.112888in}}%
\pgfpathlineto{\pgfqpoint{4.469189in}{3.124380in}}%
\pgfpathlineto{\pgfqpoint{4.480161in}{3.140034in}}%
\pgfpathlineto{\pgfqpoint{4.485513in}{3.143988in}}%
\pgfpathlineto{\pgfqpoint{4.489795in}{3.144537in}}%
\pgfpathlineto{\pgfqpoint{4.494077in}{3.142666in}}%
\pgfpathlineto{\pgfqpoint{4.499161in}{3.137646in}}%
\pgfpathlineto{\pgfqpoint{4.506922in}{3.126188in}}%
\pgfpathlineto{\pgfqpoint{4.517894in}{3.110481in}}%
\pgfpathlineto{\pgfqpoint{4.523246in}{3.106469in}}%
\pgfpathlineto{\pgfqpoint{4.527528in}{3.105868in}}%
\pgfpathlineto{\pgfqpoint{4.531809in}{3.107687in}}%
\pgfpathlineto{\pgfqpoint{4.536894in}{3.112659in}}%
\pgfpathlineto{\pgfqpoint{4.544387in}{3.123651in}}%
\pgfpathlineto{\pgfqpoint{4.555894in}{3.140120in}}%
\pgfpathlineto{\pgfqpoint{4.561246in}{3.144021in}}%
\pgfpathlineto{\pgfqpoint{4.565528in}{3.144523in}}%
\pgfpathlineto{\pgfqpoint{4.569810in}{3.142608in}}%
\pgfpathlineto{\pgfqpoint{4.574894in}{3.137543in}}%
\pgfpathlineto{\pgfqpoint{4.582655in}{3.126056in}}%
\pgfpathlineto{\pgfqpoint{4.593627in}{3.110395in}}%
\pgfpathlineto{\pgfqpoint{4.598979in}{3.106435in}}%
\pgfpathlineto{\pgfqpoint{4.603261in}{3.105881in}}%
\pgfpathlineto{\pgfqpoint{4.607542in}{3.107745in}}%
\pgfpathlineto{\pgfqpoint{4.612627in}{3.112760in}}%
\pgfpathlineto{\pgfqpoint{4.620388in}{3.124214in}}%
\pgfpathlineto{\pgfqpoint{4.631360in}{3.139927in}}%
\pgfpathlineto{\pgfqpoint{4.636712in}{3.143945in}}%
\pgfpathlineto{\pgfqpoint{4.640993in}{3.144552in}}%
\pgfpathlineto{\pgfqpoint{4.645275in}{3.142739in}}%
\pgfpathlineto{\pgfqpoint{4.650360in}{3.137773in}}%
\pgfpathlineto{\pgfqpoint{4.657853in}{3.126785in}}%
\pgfpathlineto{\pgfqpoint{4.669360in}{3.110310in}}%
\pgfpathlineto{\pgfqpoint{4.674712in}{3.106402in}}%
\pgfpathlineto{\pgfqpoint{4.678994in}{3.105894in}}%
\pgfpathlineto{\pgfqpoint{4.683275in}{3.107804in}}%
\pgfpathlineto{\pgfqpoint{4.688360in}{3.112863in}}%
\pgfpathlineto{\pgfqpoint{4.696121in}{3.124346in}}%
\pgfpathlineto{\pgfqpoint{4.707093in}{3.140013in}}%
\pgfpathlineto{\pgfqpoint{4.712445in}{3.143980in}}%
\pgfpathlineto{\pgfqpoint{4.716726in}{3.144540in}}%
\pgfpathlineto{\pgfqpoint{4.721008in}{3.142681in}}%
\pgfpathlineto{\pgfqpoint{4.726093in}{3.137671in}}%
\pgfpathlineto{\pgfqpoint{4.733853in}{3.126222in}}%
\pgfpathlineto{\pgfqpoint{4.744825in}{3.110503in}}%
\pgfpathlineto{\pgfqpoint{4.750177in}{3.106478in}}%
\pgfpathlineto{\pgfqpoint{4.754459in}{3.105865in}}%
\pgfpathlineto{\pgfqpoint{4.758741in}{3.107673in}}%
\pgfpathlineto{\pgfqpoint{4.763825in}{3.112633in}}%
\pgfpathlineto{\pgfqpoint{4.771318in}{3.123617in}}%
\pgfpathlineto{\pgfqpoint{4.782826in}{3.140099in}}%
\pgfpathlineto{\pgfqpoint{4.788178in}{3.144013in}}%
\pgfpathlineto{\pgfqpoint{4.792459in}{3.144527in}}%
\pgfpathlineto{\pgfqpoint{4.796741in}{3.142623in}}%
\pgfpathlineto{\pgfqpoint{4.801826in}{3.137569in}}%
\pgfpathlineto{\pgfqpoint{4.809586in}{3.126089in}}%
\pgfpathlineto{\pgfqpoint{4.820558in}{3.110417in}}%
\pgfpathlineto{\pgfqpoint{4.825910in}{3.106444in}}%
\pgfpathlineto{\pgfqpoint{4.830192in}{3.105878in}}%
\pgfpathlineto{\pgfqpoint{4.834474in}{3.107731in}}%
\pgfpathlineto{\pgfqpoint{4.839558in}{3.112735in}}%
\pgfpathlineto{\pgfqpoint{4.847319in}{3.124181in}}%
\pgfpathlineto{\pgfqpoint{4.858291in}{3.139905in}}%
\pgfpathlineto{\pgfqpoint{4.863643in}{3.143936in}}%
\pgfpathlineto{\pgfqpoint{4.867925in}{3.144555in}}%
\pgfpathlineto{\pgfqpoint{4.872207in}{3.142753in}}%
\pgfpathlineto{\pgfqpoint{4.877291in}{3.137798in}}%
\pgfpathlineto{\pgfqpoint{4.884784in}{3.126818in}}%
\pgfpathlineto{\pgfqpoint{4.896291in}{3.110331in}}%
\pgfpathlineto{\pgfqpoint{4.901643in}{3.106410in}}%
\pgfpathlineto{\pgfqpoint{4.905925in}{3.105891in}}%
\pgfpathlineto{\pgfqpoint{4.910207in}{3.107789in}}%
\pgfpathlineto{\pgfqpoint{4.915291in}{3.112837in}}%
\pgfpathlineto{\pgfqpoint{4.923052in}{3.124313in}}%
\pgfpathlineto{\pgfqpoint{4.934024in}{3.139991in}}%
\pgfpathlineto{\pgfqpoint{4.939376in}{3.143971in}}%
\pgfpathlineto{\pgfqpoint{4.943658in}{3.144543in}}%
\pgfpathlineto{\pgfqpoint{4.947940in}{3.142695in}}%
\pgfpathlineto{\pgfqpoint{4.953024in}{3.137697in}}%
\pgfpathlineto{\pgfqpoint{4.960785in}{3.126255in}}%
\pgfpathlineto{\pgfqpoint{4.972024in}{3.110246in}}%
\pgfpathlineto{\pgfqpoint{4.977376in}{3.106377in}}%
\pgfpathlineto{\pgfqpoint{4.981658in}{3.105905in}}%
\pgfpathlineto{\pgfqpoint{4.985940in}{3.107848in}}%
\pgfpathlineto{\pgfqpoint{4.991024in}{3.112940in}}%
\pgfpathlineto{\pgfqpoint{4.998785in}{3.124446in}}%
\pgfpathlineto{\pgfqpoint{5.009757in}{3.140077in}}%
\pgfpathlineto{\pgfqpoint{5.015109in}{3.144005in}}%
\pgfpathlineto{\pgfqpoint{5.019391in}{3.144530in}}%
\pgfpathlineto{\pgfqpoint{5.023673in}{3.142637in}}%
\pgfpathlineto{\pgfqpoint{5.028757in}{3.137595in}}%
\pgfpathlineto{\pgfqpoint{5.036518in}{3.126122in}}%
\pgfpathlineto{\pgfqpoint{5.047490in}{3.110438in}}%
\pgfpathlineto{\pgfqpoint{5.052842in}{3.106452in}}%
\pgfpathlineto{\pgfqpoint{5.057124in}{3.105874in}}%
\pgfpathlineto{\pgfqpoint{5.061405in}{3.107716in}}%
\pgfpathlineto{\pgfqpoint{5.066490in}{3.112710in}}%
\pgfpathlineto{\pgfqpoint{5.074251in}{3.124147in}}%
\pgfpathlineto{\pgfqpoint{5.085490in}{3.140162in}}%
\pgfpathlineto{\pgfqpoint{5.090842in}{3.144038in}}%
\pgfpathlineto{\pgfqpoint{5.095124in}{3.144516in}}%
\pgfpathlineto{\pgfqpoint{5.099406in}{3.142578in}}%
\pgfpathlineto{\pgfqpoint{5.104490in}{3.137492in}}%
\pgfpathlineto{\pgfqpoint{5.112251in}{3.125989in}}%
\pgfpathlineto{\pgfqpoint{5.123223in}{3.110352in}}%
\pgfpathlineto{\pgfqpoint{5.128575in}{3.106418in}}%
\pgfpathlineto{\pgfqpoint{5.132857in}{3.105887in}}%
\pgfpathlineto{\pgfqpoint{5.137138in}{3.107774in}}%
\pgfpathlineto{\pgfqpoint{5.142223in}{3.112811in}}%
\pgfpathlineto{\pgfqpoint{5.149984in}{3.124280in}}%
\pgfpathlineto{\pgfqpoint{5.160955in}{3.139970in}}%
\pgfpathlineto{\pgfqpoint{5.166308in}{3.143962in}}%
\pgfpathlineto{\pgfqpoint{5.170589in}{3.144546in}}%
\pgfpathlineto{\pgfqpoint{5.174871in}{3.142710in}}%
\pgfpathlineto{\pgfqpoint{5.179956in}{3.137722in}}%
\pgfpathlineto{\pgfqpoint{5.187716in}{3.126288in}}%
\pgfpathlineto{\pgfqpoint{5.198956in}{3.110267in}}%
\pgfpathlineto{\pgfqpoint{5.204308in}{3.106385in}}%
\pgfpathlineto{\pgfqpoint{5.208590in}{3.105901in}}%
\pgfpathlineto{\pgfqpoint{5.212871in}{3.107833in}}%
\pgfpathlineto{\pgfqpoint{5.217956in}{3.112914in}}%
\pgfpathlineto{\pgfqpoint{5.225717in}{3.124413in}}%
\pgfpathlineto{\pgfqpoint{5.236688in}{3.140056in}}%
\pgfpathlineto{\pgfqpoint{5.242041in}{3.143996in}}%
\pgfpathlineto{\pgfqpoint{5.246322in}{3.144533in}}%
\pgfpathlineto{\pgfqpoint{5.250604in}{3.142652in}}%
\pgfpathlineto{\pgfqpoint{5.255689in}{3.137620in}}%
\pgfpathlineto{\pgfqpoint{5.263449in}{3.126155in}}%
\pgfpathlineto{\pgfqpoint{5.274421in}{3.110460in}}%
\pgfpathlineto{\pgfqpoint{5.279773in}{3.106461in}}%
\pgfpathlineto{\pgfqpoint{5.284055in}{3.105871in}}%
\pgfpathlineto{\pgfqpoint{5.288337in}{3.107702in}}%
\pgfpathlineto{\pgfqpoint{5.293421in}{3.112684in}}%
\pgfpathlineto{\pgfqpoint{5.300914in}{3.123684in}}%
\pgfpathlineto{\pgfqpoint{5.312421in}{3.140141in}}%
\pgfpathlineto{\pgfqpoint{5.317774in}{3.144030in}}%
\pgfpathlineto{\pgfqpoint{5.322055in}{3.144520in}}%
\pgfpathlineto{\pgfqpoint{5.326337in}{3.142593in}}%
\pgfpathlineto{\pgfqpoint{5.331422in}{3.137518in}}%
\pgfpathlineto{\pgfqpoint{5.339182in}{3.126023in}}%
\pgfpathlineto{\pgfqpoint{5.350154in}{3.110374in}}%
\pgfpathlineto{\pgfqpoint{5.355506in}{3.106427in}}%
\pgfpathlineto{\pgfqpoint{5.359788in}{3.105884in}}%
\pgfpathlineto{\pgfqpoint{5.364070in}{3.107760in}}%
\pgfpathlineto{\pgfqpoint{5.369154in}{3.112786in}}%
\pgfpathlineto{\pgfqpoint{5.376915in}{3.124247in}}%
\pgfpathlineto{\pgfqpoint{5.387887in}{3.139948in}}%
\pgfpathlineto{\pgfqpoint{5.393239in}{3.143954in}}%
\pgfpathlineto{\pgfqpoint{5.397521in}{3.144549in}}%
\pgfpathlineto{\pgfqpoint{5.401802in}{3.142724in}}%
\pgfpathlineto{\pgfqpoint{5.406887in}{3.137747in}}%
\pgfpathlineto{\pgfqpoint{5.414380in}{3.126752in}}%
\pgfpathlineto{\pgfqpoint{5.425887in}{3.110288in}}%
\pgfpathlineto{\pgfqpoint{5.431239in}{3.106393in}}%
\pgfpathlineto{\pgfqpoint{5.435521in}{3.105898in}}%
\pgfpathlineto{\pgfqpoint{5.439803in}{3.107818in}}%
\pgfpathlineto{\pgfqpoint{5.444887in}{3.112888in}}%
\pgfpathlineto{\pgfqpoint{5.452648in}{3.124380in}}%
\pgfpathlineto{\pgfqpoint{5.463620in}{3.140034in}}%
\pgfpathlineto{\pgfqpoint{5.468972in}{3.143988in}}%
\pgfpathlineto{\pgfqpoint{5.473254in}{3.144537in}}%
\pgfpathlineto{\pgfqpoint{5.477536in}{3.142666in}}%
\pgfpathlineto{\pgfqpoint{5.482620in}{3.137646in}}%
\pgfpathlineto{\pgfqpoint{5.490381in}{3.126188in}}%
\pgfpathlineto{\pgfqpoint{5.501353in}{3.110481in}}%
\pgfpathlineto{\pgfqpoint{5.506705in}{3.106469in}}%
\pgfpathlineto{\pgfqpoint{5.510986in}{3.105868in}}%
\pgfpathlineto{\pgfqpoint{5.515268in}{3.107687in}}%
\pgfpathlineto{\pgfqpoint{5.520353in}{3.112659in}}%
\pgfpathlineto{\pgfqpoint{5.527846in}{3.123651in}}%
\pgfpathlineto{\pgfqpoint{5.539353in}{3.140120in}}%
\pgfpathlineto{\pgfqpoint{5.544705in}{3.144021in}}%
\pgfpathlineto{\pgfqpoint{5.548987in}{3.144523in}}%
\pgfpathlineto{\pgfqpoint{5.553269in}{3.142608in}}%
\pgfpathlineto{\pgfqpoint{5.558353in}{3.137543in}}%
\pgfpathlineto{\pgfqpoint{5.566114in}{3.126056in}}%
\pgfpathlineto{\pgfqpoint{5.577086in}{3.110395in}}%
\pgfpathlineto{\pgfqpoint{5.582438in}{3.106435in}}%
\pgfpathlineto{\pgfqpoint{5.586719in}{3.105881in}}%
\pgfpathlineto{\pgfqpoint{5.591001in}{3.107745in}}%
\pgfpathlineto{\pgfqpoint{5.596086in}{3.112760in}}%
\pgfpathlineto{\pgfqpoint{5.603846in}{3.124214in}}%
\pgfpathlineto{\pgfqpoint{5.614818in}{3.139927in}}%
\pgfpathlineto{\pgfqpoint{5.620170in}{3.143945in}}%
\pgfpathlineto{\pgfqpoint{5.624452in}{3.144552in}}%
\pgfpathlineto{\pgfqpoint{5.628734in}{3.142739in}}%
\pgfpathlineto{\pgfqpoint{5.633818in}{3.137773in}}%
\pgfpathlineto{\pgfqpoint{5.641311in}{3.126785in}}%
\pgfpathlineto{\pgfqpoint{5.652819in}{3.110310in}}%
\pgfpathlineto{\pgfqpoint{5.658171in}{3.106402in}}%
\pgfpathlineto{\pgfqpoint{5.662453in}{3.105894in}}%
\pgfpathlineto{\pgfqpoint{5.666734in}{3.107804in}}%
\pgfpathlineto{\pgfqpoint{5.671819in}{3.112863in}}%
\pgfpathlineto{\pgfqpoint{5.679579in}{3.124346in}}%
\pgfpathlineto{\pgfqpoint{5.690551in}{3.140013in}}%
\pgfpathlineto{\pgfqpoint{5.695903in}{3.143980in}}%
\pgfpathlineto{\pgfqpoint{5.700185in}{3.144540in}}%
\pgfpathlineto{\pgfqpoint{5.704467in}{3.142681in}}%
\pgfpathlineto{\pgfqpoint{5.709551in}{3.137671in}}%
\pgfpathlineto{\pgfqpoint{5.717312in}{3.126222in}}%
\pgfpathlineto{\pgfqpoint{5.728284in}{3.110503in}}%
\pgfpathlineto{\pgfqpoint{5.733636in}{3.106478in}}%
\pgfpathlineto{\pgfqpoint{5.737918in}{3.105865in}}%
\pgfpathlineto{\pgfqpoint{5.742200in}{3.107673in}}%
\pgfpathlineto{\pgfqpoint{5.747284in}{3.112633in}}%
\pgfpathlineto{\pgfqpoint{5.754777in}{3.123617in}}%
\pgfpathlineto{\pgfqpoint{5.766284in}{3.140099in}}%
\pgfpathlineto{\pgfqpoint{5.771636in}{3.144013in}}%
\pgfpathlineto{\pgfqpoint{5.775918in}{3.144527in}}%
\pgfpathlineto{\pgfqpoint{5.780200in}{3.142623in}}%
\pgfpathlineto{\pgfqpoint{5.785284in}{3.137569in}}%
\pgfpathlineto{\pgfqpoint{5.793045in}{3.126089in}}%
\pgfpathlineto{\pgfqpoint{5.804017in}{3.110417in}}%
\pgfpathlineto{\pgfqpoint{5.809369in}{3.106444in}}%
\pgfpathlineto{\pgfqpoint{5.813651in}{3.105878in}}%
\pgfpathlineto{\pgfqpoint{5.817933in}{3.107731in}}%
\pgfpathlineto{\pgfqpoint{5.823017in}{3.112735in}}%
\pgfpathlineto{\pgfqpoint{5.830778in}{3.124181in}}%
\pgfpathlineto{\pgfqpoint{5.841750in}{3.139905in}}%
\pgfpathlineto{\pgfqpoint{5.847102in}{3.143936in}}%
\pgfpathlineto{\pgfqpoint{5.851384in}{3.144555in}}%
\pgfpathlineto{\pgfqpoint{5.855665in}{3.142753in}}%
\pgfpathlineto{\pgfqpoint{5.860750in}{3.137798in}}%
\pgfpathlineto{\pgfqpoint{5.868243in}{3.126818in}}%
\pgfpathlineto{\pgfqpoint{5.879750in}{3.110331in}}%
\pgfpathlineto{\pgfqpoint{5.885102in}{3.106410in}}%
\pgfpathlineto{\pgfqpoint{5.889384in}{3.105891in}}%
\pgfpathlineto{\pgfqpoint{5.893666in}{3.107789in}}%
\pgfpathlineto{\pgfqpoint{5.898750in}{3.112837in}}%
\pgfpathlineto{\pgfqpoint{5.906511in}{3.124313in}}%
\pgfpathlineto{\pgfqpoint{5.917483in}{3.139991in}}%
\pgfpathlineto{\pgfqpoint{5.922835in}{3.143971in}}%
\pgfpathlineto{\pgfqpoint{5.927117in}{3.144543in}}%
\pgfpathlineto{\pgfqpoint{5.931398in}{3.142695in}}%
\pgfpathlineto{\pgfqpoint{5.936483in}{3.137697in}}%
\pgfpathlineto{\pgfqpoint{5.944244in}{3.126255in}}%
\pgfpathlineto{\pgfqpoint{5.955483in}{3.110246in}}%
\pgfpathlineto{\pgfqpoint{5.960835in}{3.106377in}}%
\pgfpathlineto{\pgfqpoint{5.965117in}{3.105905in}}%
\pgfpathlineto{\pgfqpoint{5.969399in}{3.107848in}}%
\pgfpathlineto{\pgfqpoint{5.974483in}{3.112940in}}%
\pgfpathlineto{\pgfqpoint{5.982244in}{3.124446in}}%
\pgfpathlineto{\pgfqpoint{5.993216in}{3.140077in}}%
\pgfpathlineto{\pgfqpoint{5.998568in}{3.144005in}}%
\pgfpathlineto{\pgfqpoint{6.002850in}{3.144530in}}%
\pgfpathlineto{\pgfqpoint{6.007131in}{3.142637in}}%
\pgfpathlineto{\pgfqpoint{6.012216in}{3.137595in}}%
\pgfpathlineto{\pgfqpoint{6.019977in}{3.126122in}}%
\pgfpathlineto{\pgfqpoint{6.030948in}{3.110438in}}%
\pgfpathlineto{\pgfqpoint{6.036301in}{3.106452in}}%
\pgfpathlineto{\pgfqpoint{6.040582in}{3.105874in}}%
\pgfpathlineto{\pgfqpoint{6.044864in}{3.107716in}}%
\pgfpathlineto{\pgfqpoint{6.049949in}{3.112710in}}%
\pgfpathlineto{\pgfqpoint{6.057709in}{3.124147in}}%
\pgfpathlineto{\pgfqpoint{6.068949in}{3.140162in}}%
\pgfpathlineto{\pgfqpoint{6.074301in}{3.144038in}}%
\pgfpathlineto{\pgfqpoint{6.078583in}{3.144516in}}%
\pgfpathlineto{\pgfqpoint{6.082864in}{3.142578in}}%
\pgfpathlineto{\pgfqpoint{6.087949in}{3.137492in}}%
\pgfpathlineto{\pgfqpoint{6.095710in}{3.125989in}}%
\pgfpathlineto{\pgfqpoint{6.106681in}{3.110352in}}%
\pgfpathlineto{\pgfqpoint{6.112034in}{3.106418in}}%
\pgfpathlineto{\pgfqpoint{6.116315in}{3.105887in}}%
\pgfpathlineto{\pgfqpoint{6.120597in}{3.107774in}}%
\pgfpathlineto{\pgfqpoint{6.125682in}{3.112811in}}%
\pgfpathlineto{\pgfqpoint{6.133442in}{3.124280in}}%
\pgfpathlineto{\pgfqpoint{6.144414in}{3.139970in}}%
\pgfpathlineto{\pgfqpoint{6.149766in}{3.143962in}}%
\pgfpathlineto{\pgfqpoint{6.154048in}{3.144546in}}%
\pgfpathlineto{\pgfqpoint{6.158330in}{3.142710in}}%
\pgfpathlineto{\pgfqpoint{6.163414in}{3.137722in}}%
\pgfpathlineto{\pgfqpoint{6.171175in}{3.126288in}}%
\pgfpathlineto{\pgfqpoint{6.182415in}{3.110267in}}%
\pgfpathlineto{\pgfqpoint{6.187767in}{3.106385in}}%
\pgfpathlineto{\pgfqpoint{6.192048in}{3.105901in}}%
\pgfpathlineto{\pgfqpoint{6.196330in}{3.107833in}}%
\pgfpathlineto{\pgfqpoint{6.201415in}{3.112914in}}%
\pgfpathlineto{\pgfqpoint{6.209175in}{3.124413in}}%
\pgfpathlineto{\pgfqpoint{6.220147in}{3.140056in}}%
\pgfpathlineto{\pgfqpoint{6.225499in}{3.143996in}}%
\pgfpathlineto{\pgfqpoint{6.229781in}{3.144533in}}%
\pgfpathlineto{\pgfqpoint{6.234063in}{3.142652in}}%
\pgfpathlineto{\pgfqpoint{6.239147in}{3.137620in}}%
\pgfpathlineto{\pgfqpoint{6.246908in}{3.126155in}}%
\pgfpathlineto{\pgfqpoint{6.257880in}{3.110460in}}%
\pgfpathlineto{\pgfqpoint{6.263232in}{3.106461in}}%
\pgfpathlineto{\pgfqpoint{6.267514in}{3.105871in}}%
\pgfpathlineto{\pgfqpoint{6.271796in}{3.107702in}}%
\pgfpathlineto{\pgfqpoint{6.276880in}{3.112684in}}%
\pgfpathlineto{\pgfqpoint{6.284373in}{3.123684in}}%
\pgfpathlineto{\pgfqpoint{6.295880in}{3.140141in}}%
\pgfpathlineto{\pgfqpoint{6.301232in}{3.144030in}}%
\pgfpathlineto{\pgfqpoint{6.305514in}{3.144520in}}%
\pgfpathlineto{\pgfqpoint{6.309796in}{3.142593in}}%
\pgfpathlineto{\pgfqpoint{6.314880in}{3.137518in}}%
\pgfpathlineto{\pgfqpoint{6.322641in}{3.126023in}}%
\pgfpathlineto{\pgfqpoint{6.333613in}{3.110374in}}%
\pgfpathlineto{\pgfqpoint{6.338965in}{3.106427in}}%
\pgfpathlineto{\pgfqpoint{6.343247in}{3.105884in}}%
\pgfpathlineto{\pgfqpoint{6.347529in}{3.107760in}}%
\pgfpathlineto{\pgfqpoint{6.352613in}{3.112786in}}%
\pgfpathlineto{\pgfqpoint{6.360374in}{3.124247in}}%
\pgfpathlineto{\pgfqpoint{6.371346in}{3.139948in}}%
\pgfpathlineto{\pgfqpoint{6.376698in}{3.143954in}}%
\pgfpathlineto{\pgfqpoint{6.380980in}{3.144549in}}%
\pgfpathlineto{\pgfqpoint{6.385261in}{3.142724in}}%
\pgfpathlineto{\pgfqpoint{6.390346in}{3.137747in}}%
\pgfpathlineto{\pgfqpoint{6.397839in}{3.126752in}}%
\pgfpathlineto{\pgfqpoint{6.409346in}{3.110288in}}%
\pgfpathlineto{\pgfqpoint{6.414698in}{3.106393in}}%
\pgfpathlineto{\pgfqpoint{6.418980in}{3.105898in}}%
\pgfpathlineto{\pgfqpoint{6.423262in}{3.107818in}}%
\pgfpathlineto{\pgfqpoint{6.428346in}{3.112888in}}%
\pgfpathlineto{\pgfqpoint{6.436107in}{3.124380in}}%
\pgfpathlineto{\pgfqpoint{6.447079in}{3.140034in}}%
\pgfpathlineto{\pgfqpoint{6.452431in}{3.143988in}}%
\pgfpathlineto{\pgfqpoint{6.456713in}{3.144537in}}%
\pgfpathlineto{\pgfqpoint{6.460994in}{3.142666in}}%
\pgfpathlineto{\pgfqpoint{6.466079in}{3.137646in}}%
\pgfpathlineto{\pgfqpoint{6.473839in}{3.126188in}}%
\pgfpathlineto{\pgfqpoint{6.484811in}{3.110481in}}%
\pgfpathlineto{\pgfqpoint{6.490163in}{3.106469in}}%
\pgfpathlineto{\pgfqpoint{6.494445in}{3.105868in}}%
\pgfpathlineto{\pgfqpoint{6.498727in}{3.107687in}}%
\pgfpathlineto{\pgfqpoint{6.503811in}{3.112659in}}%
\pgfpathlineto{\pgfqpoint{6.511305in}{3.123651in}}%
\pgfpathlineto{\pgfqpoint{6.522812in}{3.140120in}}%
\pgfpathlineto{\pgfqpoint{6.528164in}{3.144021in}}%
\pgfpathlineto{\pgfqpoint{6.532446in}{3.144523in}}%
\pgfpathlineto{\pgfqpoint{6.536727in}{3.142608in}}%
\pgfpathlineto{\pgfqpoint{6.541812in}{3.137543in}}%
\pgfpathlineto{\pgfqpoint{6.549572in}{3.126056in}}%
\pgfpathlineto{\pgfqpoint{6.560544in}{3.110395in}}%
\pgfpathlineto{\pgfqpoint{6.565897in}{3.106435in}}%
\pgfpathlineto{\pgfqpoint{6.570178in}{3.105881in}}%
\pgfpathlineto{\pgfqpoint{6.574460in}{3.107745in}}%
\pgfpathlineto{\pgfqpoint{6.579545in}{3.112760in}}%
\pgfpathlineto{\pgfqpoint{6.587305in}{3.124214in}}%
\pgfpathlineto{\pgfqpoint{6.598277in}{3.139927in}}%
\pgfpathlineto{\pgfqpoint{6.603629in}{3.143945in}}%
\pgfpathlineto{\pgfqpoint{6.607911in}{3.144552in}}%
\pgfpathlineto{\pgfqpoint{6.612193in}{3.142739in}}%
\pgfpathlineto{\pgfqpoint{6.617277in}{3.137773in}}%
\pgfpathlineto{\pgfqpoint{6.624770in}{3.126785in}}%
\pgfpathlineto{\pgfqpoint{6.636277in}{3.110310in}}%
\pgfpathlineto{\pgfqpoint{6.641630in}{3.106402in}}%
\pgfpathlineto{\pgfqpoint{6.645911in}{3.105894in}}%
\pgfpathlineto{\pgfqpoint{6.650193in}{3.107804in}}%
\pgfpathlineto{\pgfqpoint{6.655278in}{3.112863in}}%
\pgfpathlineto{\pgfqpoint{6.663038in}{3.124346in}}%
\pgfpathlineto{\pgfqpoint{6.663306in}{3.124778in}}%
\pgfpathlineto{\pgfqpoint{6.663306in}{3.124778in}}%
\pgfusepath{stroke}%
\end{pgfscope}%
\begin{pgfscope}%
\pgfpathrectangle{\pgfqpoint{0.467797in}{2.292089in}}{\pgfqpoint{6.490533in}{1.666241in}}%
\pgfusepath{clip}%
\pgfsetrectcap%
\pgfsetroundjoin%
\pgfsetlinewidth{1.505625pt}%
\definecolor{currentstroke}{rgb}{0.121569,0.466667,0.705882}%
\pgfsetstrokecolor{currentstroke}%
\pgfsetdash{}{0pt}%
\pgfpathmoveto{\pgfqpoint{0.762821in}{3.125209in}}%
\pgfpathlineto{\pgfqpoint{0.772990in}{3.139540in}}%
\pgfpathlineto{\pgfqpoint{0.778075in}{3.143157in}}%
\pgfpathlineto{\pgfqpoint{0.782089in}{3.143567in}}%
\pgfpathlineto{\pgfqpoint{0.786103in}{3.141745in}}%
\pgfpathlineto{\pgfqpoint{0.790920in}{3.136942in}}%
\pgfpathlineto{\pgfqpoint{0.798680in}{3.125404in}}%
\pgfpathlineto{\pgfqpoint{0.808849in}{3.111003in}}%
\pgfpathlineto{\pgfqpoint{0.813934in}{3.107309in}}%
\pgfpathlineto{\pgfqpoint{0.817948in}{3.106831in}}%
\pgfpathlineto{\pgfqpoint{0.821962in}{3.108588in}}%
\pgfpathlineto{\pgfqpoint{0.826779in}{3.113327in}}%
\pgfpathlineto{\pgfqpoint{0.834272in}{3.124389in}}%
\pgfpathlineto{\pgfqpoint{0.844977in}{3.139566in}}%
\pgfpathlineto{\pgfqpoint{0.850061in}{3.143167in}}%
\pgfpathlineto{\pgfqpoint{0.854075in}{3.143563in}}%
\pgfpathlineto{\pgfqpoint{0.858089in}{3.141726in}}%
\pgfpathlineto{\pgfqpoint{0.862906in}{3.136909in}}%
\pgfpathlineto{\pgfqpoint{0.870667in}{3.125362in}}%
\pgfpathlineto{\pgfqpoint{0.880836in}{3.110976in}}%
\pgfpathlineto{\pgfqpoint{0.885921in}{3.107298in}}%
\pgfpathlineto{\pgfqpoint{0.889935in}{3.106835in}}%
\pgfpathlineto{\pgfqpoint{0.893949in}{3.108606in}}%
\pgfpathlineto{\pgfqpoint{0.898766in}{3.113360in}}%
\pgfpathlineto{\pgfqpoint{0.906259in}{3.124431in}}%
\pgfpathlineto{\pgfqpoint{0.916963in}{3.139593in}}%
\pgfpathlineto{\pgfqpoint{0.922048in}{3.143177in}}%
\pgfpathlineto{\pgfqpoint{0.926062in}{3.143558in}}%
\pgfpathlineto{\pgfqpoint{0.930076in}{3.141707in}}%
\pgfpathlineto{\pgfqpoint{0.934893in}{3.136877in}}%
\pgfpathlineto{\pgfqpoint{0.942653in}{3.125320in}}%
\pgfpathlineto{\pgfqpoint{0.952822in}{3.110949in}}%
\pgfpathlineto{\pgfqpoint{0.957907in}{3.107288in}}%
\pgfpathlineto{\pgfqpoint{0.961921in}{3.106840in}}%
\pgfpathlineto{\pgfqpoint{0.965935in}{3.108625in}}%
\pgfpathlineto{\pgfqpoint{0.970752in}{3.113392in}}%
\pgfpathlineto{\pgfqpoint{0.978245in}{3.124473in}}%
\pgfpathlineto{\pgfqpoint{0.988682in}{3.139345in}}%
\pgfpathlineto{\pgfqpoint{0.993766in}{3.143083in}}%
\pgfpathlineto{\pgfqpoint{0.997781in}{3.143599in}}%
\pgfpathlineto{\pgfqpoint{1.001795in}{3.141879in}}%
\pgfpathlineto{\pgfqpoint{1.006612in}{3.137176in}}%
\pgfpathlineto{\pgfqpoint{1.014105in}{3.126141in}}%
\pgfpathlineto{\pgfqpoint{1.024809in}{3.110922in}}%
\pgfpathlineto{\pgfqpoint{1.029894in}{3.107278in}}%
\pgfpathlineto{\pgfqpoint{1.033908in}{3.106844in}}%
\pgfpathlineto{\pgfqpoint{1.037922in}{3.108643in}}%
\pgfpathlineto{\pgfqpoint{1.042739in}{3.113424in}}%
\pgfpathlineto{\pgfqpoint{1.050232in}{3.124515in}}%
\pgfpathlineto{\pgfqpoint{1.060668in}{3.139372in}}%
\pgfpathlineto{\pgfqpoint{1.065753in}{3.143093in}}%
\pgfpathlineto{\pgfqpoint{1.069767in}{3.143595in}}%
\pgfpathlineto{\pgfqpoint{1.073781in}{3.141861in}}%
\pgfpathlineto{\pgfqpoint{1.078598in}{3.137144in}}%
\pgfpathlineto{\pgfqpoint{1.086091in}{3.126099in}}%
\pgfpathlineto{\pgfqpoint{1.096795in}{3.110896in}}%
\pgfpathlineto{\pgfqpoint{1.101880in}{3.107268in}}%
\pgfpathlineto{\pgfqpoint{1.105894in}{3.106849in}}%
\pgfpathlineto{\pgfqpoint{1.109908in}{3.108662in}}%
\pgfpathlineto{\pgfqpoint{1.114725in}{3.113457in}}%
\pgfpathlineto{\pgfqpoint{1.122486in}{3.124988in}}%
\pgfpathlineto{\pgfqpoint{1.132655in}{3.139399in}}%
\pgfpathlineto{\pgfqpoint{1.137739in}{3.143104in}}%
\pgfpathlineto{\pgfqpoint{1.141754in}{3.143591in}}%
\pgfpathlineto{\pgfqpoint{1.145768in}{3.141843in}}%
\pgfpathlineto{\pgfqpoint{1.150585in}{3.137112in}}%
\pgfpathlineto{\pgfqpoint{1.158078in}{3.126057in}}%
\pgfpathlineto{\pgfqpoint{1.168782in}{3.110869in}}%
\pgfpathlineto{\pgfqpoint{1.173867in}{3.107258in}}%
\pgfpathlineto{\pgfqpoint{1.177881in}{3.106853in}}%
\pgfpathlineto{\pgfqpoint{1.181895in}{3.108681in}}%
\pgfpathlineto{\pgfqpoint{1.186712in}{3.113489in}}%
\pgfpathlineto{\pgfqpoint{1.194472in}{3.125030in}}%
\pgfpathlineto{\pgfqpoint{1.204641in}{3.139426in}}%
\pgfpathlineto{\pgfqpoint{1.209726in}{3.143114in}}%
\pgfpathlineto{\pgfqpoint{1.213740in}{3.143586in}}%
\pgfpathlineto{\pgfqpoint{1.217754in}{3.141824in}}%
\pgfpathlineto{\pgfqpoint{1.222571in}{3.137079in}}%
\pgfpathlineto{\pgfqpoint{1.230064in}{3.126015in}}%
\pgfpathlineto{\pgfqpoint{1.240768in}{3.110843in}}%
\pgfpathlineto{\pgfqpoint{1.245853in}{3.107248in}}%
\pgfpathlineto{\pgfqpoint{1.249867in}{3.106858in}}%
\pgfpathlineto{\pgfqpoint{1.253881in}{3.108700in}}%
\pgfpathlineto{\pgfqpoint{1.258698in}{3.113522in}}%
\pgfpathlineto{\pgfqpoint{1.266459in}{3.125073in}}%
\pgfpathlineto{\pgfqpoint{1.276628in}{3.139453in}}%
\pgfpathlineto{\pgfqpoint{1.281712in}{3.143125in}}%
\pgfpathlineto{\pgfqpoint{1.285727in}{3.143582in}}%
\pgfpathlineto{\pgfqpoint{1.289741in}{3.141806in}}%
\pgfpathlineto{\pgfqpoint{1.294558in}{3.137047in}}%
\pgfpathlineto{\pgfqpoint{1.302051in}{3.125972in}}%
\pgfpathlineto{\pgfqpoint{1.312755in}{3.110816in}}%
\pgfpathlineto{\pgfqpoint{1.317840in}{3.107238in}}%
\pgfpathlineto{\pgfqpoint{1.321854in}{3.106863in}}%
\pgfpathlineto{\pgfqpoint{1.325868in}{3.108719in}}%
\pgfpathlineto{\pgfqpoint{1.330685in}{3.113555in}}%
\pgfpathlineto{\pgfqpoint{1.338445in}{3.125115in}}%
\pgfpathlineto{\pgfqpoint{1.348614in}{3.139480in}}%
\pgfpathlineto{\pgfqpoint{1.353699in}{3.143135in}}%
\pgfpathlineto{\pgfqpoint{1.357713in}{3.143578in}}%
\pgfpathlineto{\pgfqpoint{1.361727in}{3.141787in}}%
\pgfpathlineto{\pgfqpoint{1.366544in}{3.137015in}}%
\pgfpathlineto{\pgfqpoint{1.374037in}{3.125930in}}%
\pgfpathlineto{\pgfqpoint{1.384474in}{3.111064in}}%
\pgfpathlineto{\pgfqpoint{1.389558in}{3.107332in}}%
\pgfpathlineto{\pgfqpoint{1.393573in}{3.106822in}}%
\pgfpathlineto{\pgfqpoint{1.397587in}{3.108547in}}%
\pgfpathlineto{\pgfqpoint{1.402404in}{3.113255in}}%
\pgfpathlineto{\pgfqpoint{1.409897in}{3.124294in}}%
\pgfpathlineto{\pgfqpoint{1.420601in}{3.139506in}}%
\pgfpathlineto{\pgfqpoint{1.425685in}{3.143145in}}%
\pgfpathlineto{\pgfqpoint{1.429700in}{3.143573in}}%
\pgfpathlineto{\pgfqpoint{1.433714in}{3.141769in}}%
\pgfpathlineto{\pgfqpoint{1.438531in}{3.136982in}}%
\pgfpathlineto{\pgfqpoint{1.446024in}{3.125888in}}%
\pgfpathlineto{\pgfqpoint{1.456460in}{3.111037in}}%
\pgfpathlineto{\pgfqpoint{1.461545in}{3.107322in}}%
\pgfpathlineto{\pgfqpoint{1.465559in}{3.106826in}}%
\pgfpathlineto{\pgfqpoint{1.469573in}{3.108565in}}%
\pgfpathlineto{\pgfqpoint{1.474390in}{3.113287in}}%
\pgfpathlineto{\pgfqpoint{1.481883in}{3.124336in}}%
\pgfpathlineto{\pgfqpoint{1.492587in}{3.139533in}}%
\pgfpathlineto{\pgfqpoint{1.497672in}{3.143155in}}%
\pgfpathlineto{\pgfqpoint{1.501686in}{3.143569in}}%
\pgfpathlineto{\pgfqpoint{1.505700in}{3.141750in}}%
\pgfpathlineto{\pgfqpoint{1.510517in}{3.136950in}}%
\pgfpathlineto{\pgfqpoint{1.518278in}{3.125415in}}%
\pgfpathlineto{\pgfqpoint{1.528447in}{3.111010in}}%
\pgfpathlineto{\pgfqpoint{1.533531in}{3.107311in}}%
\pgfpathlineto{\pgfqpoint{1.537546in}{3.106830in}}%
\pgfpathlineto{\pgfqpoint{1.541560in}{3.108583in}}%
\pgfpathlineto{\pgfqpoint{1.546377in}{3.113319in}}%
\pgfpathlineto{\pgfqpoint{1.553870in}{3.124378in}}%
\pgfpathlineto{\pgfqpoint{1.564574in}{3.139560in}}%
\pgfpathlineto{\pgfqpoint{1.569658in}{3.143165in}}%
\pgfpathlineto{\pgfqpoint{1.573673in}{3.143564in}}%
\pgfpathlineto{\pgfqpoint{1.577687in}{3.141731in}}%
\pgfpathlineto{\pgfqpoint{1.582504in}{3.136917in}}%
\pgfpathlineto{\pgfqpoint{1.590264in}{3.125373in}}%
\pgfpathlineto{\pgfqpoint{1.600433in}{3.110983in}}%
\pgfpathlineto{\pgfqpoint{1.605518in}{3.107301in}}%
\pgfpathlineto{\pgfqpoint{1.609532in}{3.106834in}}%
\pgfpathlineto{\pgfqpoint{1.613546in}{3.108602in}}%
\pgfpathlineto{\pgfqpoint{1.618363in}{3.113352in}}%
\pgfpathlineto{\pgfqpoint{1.625856in}{3.124420in}}%
\pgfpathlineto{\pgfqpoint{1.636560in}{3.139586in}}%
\pgfpathlineto{\pgfqpoint{1.641645in}{3.143175in}}%
\pgfpathlineto{\pgfqpoint{1.645659in}{3.143559in}}%
\pgfpathlineto{\pgfqpoint{1.649673in}{3.141712in}}%
\pgfpathlineto{\pgfqpoint{1.654490in}{3.136885in}}%
\pgfpathlineto{\pgfqpoint{1.662251in}{3.125331in}}%
\pgfpathlineto{\pgfqpoint{1.672420in}{3.110956in}}%
\pgfpathlineto{\pgfqpoint{1.677504in}{3.107291in}}%
\pgfpathlineto{\pgfqpoint{1.681519in}{3.106838in}}%
\pgfpathlineto{\pgfqpoint{1.685533in}{3.108620in}}%
\pgfpathlineto{\pgfqpoint{1.690350in}{3.113384in}}%
\pgfpathlineto{\pgfqpoint{1.697843in}{3.124462in}}%
\pgfpathlineto{\pgfqpoint{1.708547in}{3.139613in}}%
\pgfpathlineto{\pgfqpoint{1.713631in}{3.143184in}}%
\pgfpathlineto{\pgfqpoint{1.717646in}{3.143554in}}%
\pgfpathlineto{\pgfqpoint{1.721660in}{3.141693in}}%
\pgfpathlineto{\pgfqpoint{1.726477in}{3.136852in}}%
\pgfpathlineto{\pgfqpoint{1.734237in}{3.125288in}}%
\pgfpathlineto{\pgfqpoint{1.744406in}{3.110929in}}%
\pgfpathlineto{\pgfqpoint{1.749491in}{3.107280in}}%
\pgfpathlineto{\pgfqpoint{1.753505in}{3.106843in}}%
\pgfpathlineto{\pgfqpoint{1.757519in}{3.108639in}}%
\pgfpathlineto{\pgfqpoint{1.762336in}{3.113416in}}%
\pgfpathlineto{\pgfqpoint{1.769829in}{3.124504in}}%
\pgfpathlineto{\pgfqpoint{1.780266in}{3.139365in}}%
\pgfpathlineto{\pgfqpoint{1.785350in}{3.143091in}}%
\pgfpathlineto{\pgfqpoint{1.789365in}{3.143596in}}%
\pgfpathlineto{\pgfqpoint{1.793379in}{3.141865in}}%
\pgfpathlineto{\pgfqpoint{1.798196in}{3.137152in}}%
\pgfpathlineto{\pgfqpoint{1.805689in}{3.126109in}}%
\pgfpathlineto{\pgfqpoint{1.816393in}{3.110902in}}%
\pgfpathlineto{\pgfqpoint{1.821477in}{3.107270in}}%
\pgfpathlineto{\pgfqpoint{1.825492in}{3.106847in}}%
\pgfpathlineto{\pgfqpoint{1.829506in}{3.108657in}}%
\pgfpathlineto{\pgfqpoint{1.834323in}{3.113449in}}%
\pgfpathlineto{\pgfqpoint{1.841816in}{3.124546in}}%
\pgfpathlineto{\pgfqpoint{1.852252in}{3.139392in}}%
\pgfpathlineto{\pgfqpoint{1.857337in}{3.143101in}}%
\pgfpathlineto{\pgfqpoint{1.861351in}{3.143592in}}%
\pgfpathlineto{\pgfqpoint{1.865365in}{3.141847in}}%
\pgfpathlineto{\pgfqpoint{1.870182in}{3.137120in}}%
\pgfpathlineto{\pgfqpoint{1.877675in}{3.126067in}}%
\pgfpathlineto{\pgfqpoint{1.888379in}{3.110876in}}%
\pgfpathlineto{\pgfqpoint{1.893464in}{3.107260in}}%
\pgfpathlineto{\pgfqpoint{1.897478in}{3.106852in}}%
\pgfpathlineto{\pgfqpoint{1.901492in}{3.108676in}}%
\pgfpathlineto{\pgfqpoint{1.906309in}{3.113481in}}%
\pgfpathlineto{\pgfqpoint{1.914070in}{3.125020in}}%
\pgfpathlineto{\pgfqpoint{1.924239in}{3.139419in}}%
\pgfpathlineto{\pgfqpoint{1.929323in}{3.143112in}}%
\pgfpathlineto{\pgfqpoint{1.933338in}{3.143587in}}%
\pgfpathlineto{\pgfqpoint{1.937352in}{3.141829in}}%
\pgfpathlineto{\pgfqpoint{1.942169in}{3.137088in}}%
\pgfpathlineto{\pgfqpoint{1.949662in}{3.126025in}}%
\pgfpathlineto{\pgfqpoint{1.960366in}{3.110849in}}%
\pgfpathlineto{\pgfqpoint{1.965450in}{3.107250in}}%
\pgfpathlineto{\pgfqpoint{1.969465in}{3.106857in}}%
\pgfpathlineto{\pgfqpoint{1.973479in}{3.108695in}}%
\pgfpathlineto{\pgfqpoint{1.978296in}{3.113514in}}%
\pgfpathlineto{\pgfqpoint{1.986056in}{3.125062in}}%
\pgfpathlineto{\pgfqpoint{1.996225in}{3.139446in}}%
\pgfpathlineto{\pgfqpoint{2.001310in}{3.143122in}}%
\pgfpathlineto{\pgfqpoint{2.005324in}{3.143583in}}%
\pgfpathlineto{\pgfqpoint{2.009338in}{3.141810in}}%
\pgfpathlineto{\pgfqpoint{2.014155in}{3.137055in}}%
\pgfpathlineto{\pgfqpoint{2.021648in}{3.125983in}}%
\pgfpathlineto{\pgfqpoint{2.032352in}{3.110823in}}%
\pgfpathlineto{\pgfqpoint{2.037437in}{3.107241in}}%
\pgfpathlineto{\pgfqpoint{2.041451in}{3.106862in}}%
\pgfpathlineto{\pgfqpoint{2.045465in}{3.108714in}}%
\pgfpathlineto{\pgfqpoint{2.050282in}{3.113546in}}%
\pgfpathlineto{\pgfqpoint{2.058043in}{3.125104in}}%
\pgfpathlineto{\pgfqpoint{2.068212in}{3.139473in}}%
\pgfpathlineto{\pgfqpoint{2.073296in}{3.143132in}}%
\pgfpathlineto{\pgfqpoint{2.077311in}{3.143579in}}%
\pgfpathlineto{\pgfqpoint{2.081325in}{3.141792in}}%
\pgfpathlineto{\pgfqpoint{2.086142in}{3.137023in}}%
\pgfpathlineto{\pgfqpoint{2.093635in}{3.125941in}}%
\pgfpathlineto{\pgfqpoint{2.104071in}{3.111070in}}%
\pgfpathlineto{\pgfqpoint{2.109156in}{3.107335in}}%
\pgfpathlineto{\pgfqpoint{2.113170in}{3.106821in}}%
\pgfpathlineto{\pgfqpoint{2.117184in}{3.108542in}}%
\pgfpathlineto{\pgfqpoint{2.122001in}{3.113247in}}%
\pgfpathlineto{\pgfqpoint{2.129494in}{3.124283in}}%
\pgfpathlineto{\pgfqpoint{2.140198in}{3.139500in}}%
\pgfpathlineto{\pgfqpoint{2.145283in}{3.143142in}}%
\pgfpathlineto{\pgfqpoint{2.149297in}{3.143574in}}%
\pgfpathlineto{\pgfqpoint{2.153311in}{3.141773in}}%
\pgfpathlineto{\pgfqpoint{2.158128in}{3.136990in}}%
\pgfpathlineto{\pgfqpoint{2.165621in}{3.125899in}}%
\pgfpathlineto{\pgfqpoint{2.176058in}{3.111043in}}%
\pgfpathlineto{\pgfqpoint{2.181142in}{3.107324in}}%
\pgfpathlineto{\pgfqpoint{2.185156in}{3.106825in}}%
\pgfpathlineto{\pgfqpoint{2.189171in}{3.108560in}}%
\pgfpathlineto{\pgfqpoint{2.193988in}{3.113279in}}%
\pgfpathlineto{\pgfqpoint{2.201481in}{3.124325in}}%
\pgfpathlineto{\pgfqpoint{2.212185in}{3.139526in}}%
\pgfpathlineto{\pgfqpoint{2.217269in}{3.143152in}}%
\pgfpathlineto{\pgfqpoint{2.221284in}{3.143570in}}%
\pgfpathlineto{\pgfqpoint{2.225298in}{3.141755in}}%
\pgfpathlineto{\pgfqpoint{2.230115in}{3.136958in}}%
\pgfpathlineto{\pgfqpoint{2.237875in}{3.125425in}}%
\pgfpathlineto{\pgfqpoint{2.248044in}{3.111016in}}%
\pgfpathlineto{\pgfqpoint{2.253129in}{3.107314in}}%
\pgfpathlineto{\pgfqpoint{2.257143in}{3.106829in}}%
\pgfpathlineto{\pgfqpoint{2.261157in}{3.108579in}}%
\pgfpathlineto{\pgfqpoint{2.265974in}{3.113311in}}%
\pgfpathlineto{\pgfqpoint{2.273467in}{3.124368in}}%
\pgfpathlineto{\pgfqpoint{2.284171in}{3.139553in}}%
\pgfpathlineto{\pgfqpoint{2.289256in}{3.143162in}}%
\pgfpathlineto{\pgfqpoint{2.293270in}{3.143565in}}%
\pgfpathlineto{\pgfqpoint{2.297284in}{3.141736in}}%
\pgfpathlineto{\pgfqpoint{2.302101in}{3.136925in}}%
\pgfpathlineto{\pgfqpoint{2.309862in}{3.125383in}}%
\pgfpathlineto{\pgfqpoint{2.320031in}{3.110989in}}%
\pgfpathlineto{\pgfqpoint{2.325115in}{3.107303in}}%
\pgfpathlineto{\pgfqpoint{2.329129in}{3.106833in}}%
\pgfpathlineto{\pgfqpoint{2.333144in}{3.108597in}}%
\pgfpathlineto{\pgfqpoint{2.337961in}{3.113343in}}%
\pgfpathlineto{\pgfqpoint{2.345454in}{3.124410in}}%
\pgfpathlineto{\pgfqpoint{2.356158in}{3.139580in}}%
\pgfpathlineto{\pgfqpoint{2.361242in}{3.143172in}}%
\pgfpathlineto{\pgfqpoint{2.365257in}{3.143560in}}%
\pgfpathlineto{\pgfqpoint{2.369271in}{3.141717in}}%
\pgfpathlineto{\pgfqpoint{2.374088in}{3.136893in}}%
\pgfpathlineto{\pgfqpoint{2.381848in}{3.125341in}}%
\pgfpathlineto{\pgfqpoint{2.392017in}{3.110963in}}%
\pgfpathlineto{\pgfqpoint{2.397102in}{3.107293in}}%
\pgfpathlineto{\pgfqpoint{2.401116in}{3.106837in}}%
\pgfpathlineto{\pgfqpoint{2.405130in}{3.108615in}}%
\pgfpathlineto{\pgfqpoint{2.409947in}{3.113376in}}%
\pgfpathlineto{\pgfqpoint{2.417440in}{3.124452in}}%
\pgfpathlineto{\pgfqpoint{2.428144in}{3.139606in}}%
\pgfpathlineto{\pgfqpoint{2.433229in}{3.143182in}}%
\pgfpathlineto{\pgfqpoint{2.437243in}{3.143555in}}%
\pgfpathlineto{\pgfqpoint{2.441257in}{3.141698in}}%
\pgfpathlineto{\pgfqpoint{2.446074in}{3.136860in}}%
\pgfpathlineto{\pgfqpoint{2.453835in}{3.125299in}}%
\pgfpathlineto{\pgfqpoint{2.464004in}{3.110936in}}%
\pgfpathlineto{\pgfqpoint{2.469088in}{3.107283in}}%
\pgfpathlineto{\pgfqpoint{2.473102in}{3.106842in}}%
\pgfpathlineto{\pgfqpoint{2.477117in}{3.108634in}}%
\pgfpathlineto{\pgfqpoint{2.481934in}{3.113408in}}%
\pgfpathlineto{\pgfqpoint{2.489427in}{3.124494in}}%
\pgfpathlineto{\pgfqpoint{2.499863in}{3.139359in}}%
\pgfpathlineto{\pgfqpoint{2.504948in}{3.143088in}}%
\pgfpathlineto{\pgfqpoint{2.508962in}{3.143597in}}%
\pgfpathlineto{\pgfqpoint{2.512976in}{3.141870in}}%
\pgfpathlineto{\pgfqpoint{2.517793in}{3.137160in}}%
\pgfpathlineto{\pgfqpoint{2.525286in}{3.126120in}}%
\pgfpathlineto{\pgfqpoint{2.535990in}{3.110909in}}%
\pgfpathlineto{\pgfqpoint{2.541075in}{3.107273in}}%
\pgfpathlineto{\pgfqpoint{2.545089in}{3.106846in}}%
\pgfpathlineto{\pgfqpoint{2.549103in}{3.108653in}}%
\pgfpathlineto{\pgfqpoint{2.553920in}{3.113441in}}%
\pgfpathlineto{\pgfqpoint{2.561413in}{3.124536in}}%
\pgfpathlineto{\pgfqpoint{2.571850in}{3.139386in}}%
\pgfpathlineto{\pgfqpoint{2.576934in}{3.143099in}}%
\pgfpathlineto{\pgfqpoint{2.580948in}{3.143593in}}%
\pgfpathlineto{\pgfqpoint{2.584963in}{3.141852in}}%
\pgfpathlineto{\pgfqpoint{2.589780in}{3.137128in}}%
\pgfpathlineto{\pgfqpoint{2.597273in}{3.126078in}}%
\pgfpathlineto{\pgfqpoint{2.607977in}{3.110882in}}%
\pgfpathlineto{\pgfqpoint{2.613061in}{3.107263in}}%
\pgfpathlineto{\pgfqpoint{2.617076in}{3.106851in}}%
\pgfpathlineto{\pgfqpoint{2.621090in}{3.108671in}}%
\pgfpathlineto{\pgfqpoint{2.625907in}{3.113473in}}%
\pgfpathlineto{\pgfqpoint{2.633667in}{3.125009in}}%
\pgfpathlineto{\pgfqpoint{2.643836in}{3.139413in}}%
\pgfpathlineto{\pgfqpoint{2.648921in}{3.143109in}}%
\pgfpathlineto{\pgfqpoint{2.652935in}{3.143589in}}%
\pgfpathlineto{\pgfqpoint{2.656949in}{3.141833in}}%
\pgfpathlineto{\pgfqpoint{2.661766in}{3.137096in}}%
\pgfpathlineto{\pgfqpoint{2.669259in}{3.126036in}}%
\pgfpathlineto{\pgfqpoint{2.679963in}{3.110856in}}%
\pgfpathlineto{\pgfqpoint{2.685048in}{3.107253in}}%
\pgfpathlineto{\pgfqpoint{2.689062in}{3.106856in}}%
\pgfpathlineto{\pgfqpoint{2.693076in}{3.108690in}}%
\pgfpathlineto{\pgfqpoint{2.697893in}{3.113506in}}%
\pgfpathlineto{\pgfqpoint{2.705654in}{3.125052in}}%
\pgfpathlineto{\pgfqpoint{2.715823in}{3.139440in}}%
\pgfpathlineto{\pgfqpoint{2.720907in}{3.143119in}}%
\pgfpathlineto{\pgfqpoint{2.724921in}{3.143584in}}%
\pgfpathlineto{\pgfqpoint{2.728936in}{3.141815in}}%
\pgfpathlineto{\pgfqpoint{2.733753in}{3.137063in}}%
\pgfpathlineto{\pgfqpoint{2.741246in}{3.125994in}}%
\pgfpathlineto{\pgfqpoint{2.751950in}{3.110829in}}%
\pgfpathlineto{\pgfqpoint{2.757034in}{3.107243in}}%
\pgfpathlineto{\pgfqpoint{2.761049in}{3.106860in}}%
\pgfpathlineto{\pgfqpoint{2.765063in}{3.108709in}}%
\pgfpathlineto{\pgfqpoint{2.769880in}{3.113538in}}%
\pgfpathlineto{\pgfqpoint{2.777640in}{3.125094in}}%
\pgfpathlineto{\pgfqpoint{2.787809in}{3.139466in}}%
\pgfpathlineto{\pgfqpoint{2.792894in}{3.143130in}}%
\pgfpathlineto{\pgfqpoint{2.796908in}{3.143580in}}%
\pgfpathlineto{\pgfqpoint{2.800922in}{3.141796in}}%
\pgfpathlineto{\pgfqpoint{2.805739in}{3.137031in}}%
\pgfpathlineto{\pgfqpoint{2.813232in}{3.125951in}}%
\pgfpathlineto{\pgfqpoint{2.823669in}{3.111077in}}%
\pgfpathlineto{\pgfqpoint{2.828753in}{3.107338in}}%
\pgfpathlineto{\pgfqpoint{2.832767in}{3.106820in}}%
\pgfpathlineto{\pgfqpoint{2.836782in}{3.108538in}}%
\pgfpathlineto{\pgfqpoint{2.841598in}{3.113239in}}%
\pgfpathlineto{\pgfqpoint{2.849091in}{3.124273in}}%
\pgfpathlineto{\pgfqpoint{2.859796in}{3.139493in}}%
\pgfpathlineto{\pgfqpoint{2.864880in}{3.143140in}}%
\pgfpathlineto{\pgfqpoint{2.868894in}{3.143575in}}%
\pgfpathlineto{\pgfqpoint{2.872909in}{3.141778in}}%
\pgfpathlineto{\pgfqpoint{2.877726in}{3.136999in}}%
\pgfpathlineto{\pgfqpoint{2.885219in}{3.125909in}}%
\pgfpathlineto{\pgfqpoint{2.895655in}{3.111050in}}%
\pgfpathlineto{\pgfqpoint{2.900740in}{3.107327in}}%
\pgfpathlineto{\pgfqpoint{2.904754in}{3.106824in}}%
\pgfpathlineto{\pgfqpoint{2.908768in}{3.108556in}}%
\pgfpathlineto{\pgfqpoint{2.913585in}{3.113271in}}%
\pgfpathlineto{\pgfqpoint{2.921078in}{3.124315in}}%
\pgfpathlineto{\pgfqpoint{2.931782in}{3.139520in}}%
\pgfpathlineto{\pgfqpoint{2.936867in}{3.143150in}}%
\pgfpathlineto{\pgfqpoint{2.940881in}{3.143571in}}%
\pgfpathlineto{\pgfqpoint{2.944895in}{3.141759in}}%
\pgfpathlineto{\pgfqpoint{2.949712in}{3.136966in}}%
\pgfpathlineto{\pgfqpoint{2.957473in}{3.125436in}}%
\pgfpathlineto{\pgfqpoint{2.967642in}{3.111023in}}%
\pgfpathlineto{\pgfqpoint{2.972726in}{3.107316in}}%
\pgfpathlineto{\pgfqpoint{2.976740in}{3.106828in}}%
\pgfpathlineto{\pgfqpoint{2.980755in}{3.108574in}}%
\pgfpathlineto{\pgfqpoint{2.985571in}{3.113303in}}%
\pgfpathlineto{\pgfqpoint{2.993064in}{3.124357in}}%
\pgfpathlineto{\pgfqpoint{3.003769in}{3.139546in}}%
\pgfpathlineto{\pgfqpoint{3.008853in}{3.143160in}}%
\pgfpathlineto{\pgfqpoint{3.012867in}{3.143566in}}%
\pgfpathlineto{\pgfqpoint{3.016882in}{3.141740in}}%
\pgfpathlineto{\pgfqpoint{3.021699in}{3.136934in}}%
\pgfpathlineto{\pgfqpoint{3.029459in}{3.125394in}}%
\pgfpathlineto{\pgfqpoint{3.039628in}{3.110996in}}%
\pgfpathlineto{\pgfqpoint{3.044713in}{3.107306in}}%
\pgfpathlineto{\pgfqpoint{3.048727in}{3.106832in}}%
\pgfpathlineto{\pgfqpoint{3.052741in}{3.108592in}}%
\pgfpathlineto{\pgfqpoint{3.057558in}{3.113335in}}%
\pgfpathlineto{\pgfqpoint{3.065051in}{3.124399in}}%
\pgfpathlineto{\pgfqpoint{3.075755in}{3.139573in}}%
\pgfpathlineto{\pgfqpoint{3.080840in}{3.143170in}}%
\pgfpathlineto{\pgfqpoint{3.084854in}{3.143561in}}%
\pgfpathlineto{\pgfqpoint{3.088868in}{3.141722in}}%
\pgfpathlineto{\pgfqpoint{3.093685in}{3.136901in}}%
\pgfpathlineto{\pgfqpoint{3.101446in}{3.125352in}}%
\pgfpathlineto{\pgfqpoint{3.111615in}{3.110969in}}%
\pgfpathlineto{\pgfqpoint{3.116699in}{3.107296in}}%
\pgfpathlineto{\pgfqpoint{3.120713in}{3.106836in}}%
\pgfpathlineto{\pgfqpoint{3.124728in}{3.108611in}}%
\pgfpathlineto{\pgfqpoint{3.129544in}{3.113368in}}%
\pgfpathlineto{\pgfqpoint{3.137037in}{3.124441in}}%
\pgfpathlineto{\pgfqpoint{3.147742in}{3.139599in}}%
\pgfpathlineto{\pgfqpoint{3.152826in}{3.143180in}}%
\pgfpathlineto{\pgfqpoint{3.156840in}{3.143557in}}%
\pgfpathlineto{\pgfqpoint{3.160855in}{3.141703in}}%
\pgfpathlineto{\pgfqpoint{3.165672in}{3.136868in}}%
\pgfpathlineto{\pgfqpoint{3.173432in}{3.125309in}}%
\pgfpathlineto{\pgfqpoint{3.183601in}{3.110942in}}%
\pgfpathlineto{\pgfqpoint{3.188686in}{3.107285in}}%
\pgfpathlineto{\pgfqpoint{3.192700in}{3.106841in}}%
\pgfpathlineto{\pgfqpoint{3.196714in}{3.108629in}}%
\pgfpathlineto{\pgfqpoint{3.201531in}{3.113400in}}%
\pgfpathlineto{\pgfqpoint{3.209024in}{3.124483in}}%
\pgfpathlineto{\pgfqpoint{3.219461in}{3.139352in}}%
\pgfpathlineto{\pgfqpoint{3.224545in}{3.143085in}}%
\pgfpathlineto{\pgfqpoint{3.228559in}{3.143598in}}%
\pgfpathlineto{\pgfqpoint{3.232573in}{3.141874in}}%
\pgfpathlineto{\pgfqpoint{3.237390in}{3.137168in}}%
\pgfpathlineto{\pgfqpoint{3.244883in}{3.126130in}}%
\pgfpathlineto{\pgfqpoint{3.255588in}{3.110916in}}%
\pgfpathlineto{\pgfqpoint{3.260672in}{3.107275in}}%
\pgfpathlineto{\pgfqpoint{3.264686in}{3.106845in}}%
\pgfpathlineto{\pgfqpoint{3.268701in}{3.108648in}}%
\pgfpathlineto{\pgfqpoint{3.273517in}{3.113432in}}%
\pgfpathlineto{\pgfqpoint{3.281011in}{3.124525in}}%
\pgfpathlineto{\pgfqpoint{3.291447in}{3.139379in}}%
\pgfpathlineto{\pgfqpoint{3.296532in}{3.143096in}}%
\pgfpathlineto{\pgfqpoint{3.300546in}{3.143594in}}%
\pgfpathlineto{\pgfqpoint{3.304560in}{3.141856in}}%
\pgfpathlineto{\pgfqpoint{3.309377in}{3.137136in}}%
\pgfpathlineto{\pgfqpoint{3.316870in}{3.126088in}}%
\pgfpathlineto{\pgfqpoint{3.327574in}{3.110889in}}%
\pgfpathlineto{\pgfqpoint{3.332659in}{3.107265in}}%
\pgfpathlineto{\pgfqpoint{3.336673in}{3.106850in}}%
\pgfpathlineto{\pgfqpoint{3.340687in}{3.108667in}}%
\pgfpathlineto{\pgfqpoint{3.345504in}{3.113465in}}%
\pgfpathlineto{\pgfqpoint{3.353265in}{3.124999in}}%
\pgfpathlineto{\pgfqpoint{3.363434in}{3.139406in}}%
\pgfpathlineto{\pgfqpoint{3.368518in}{3.143106in}}%
\pgfpathlineto{\pgfqpoint{3.372532in}{3.143590in}}%
\pgfpathlineto{\pgfqpoint{3.376546in}{3.141838in}}%
\pgfpathlineto{\pgfqpoint{3.381363in}{3.137104in}}%
\pgfpathlineto{\pgfqpoint{3.388856in}{3.126046in}}%
\pgfpathlineto{\pgfqpoint{3.399561in}{3.110863in}}%
\pgfpathlineto{\pgfqpoint{3.404645in}{3.107255in}}%
\pgfpathlineto{\pgfqpoint{3.408659in}{3.106854in}}%
\pgfpathlineto{\pgfqpoint{3.412674in}{3.108685in}}%
\pgfpathlineto{\pgfqpoint{3.417490in}{3.113497in}}%
\pgfpathlineto{\pgfqpoint{3.425251in}{3.125041in}}%
\pgfpathlineto{\pgfqpoint{3.435420in}{3.139433in}}%
\pgfpathlineto{\pgfqpoint{3.440505in}{3.143117in}}%
\pgfpathlineto{\pgfqpoint{3.444519in}{3.143585in}}%
\pgfpathlineto{\pgfqpoint{3.448533in}{3.141820in}}%
\pgfpathlineto{\pgfqpoint{3.453350in}{3.137071in}}%
\pgfpathlineto{\pgfqpoint{3.460843in}{3.126004in}}%
\pgfpathlineto{\pgfqpoint{3.471547in}{3.110836in}}%
\pgfpathlineto{\pgfqpoint{3.476632in}{3.107245in}}%
\pgfpathlineto{\pgfqpoint{3.480646in}{3.106859in}}%
\pgfpathlineto{\pgfqpoint{3.484660in}{3.108704in}}%
\pgfpathlineto{\pgfqpoint{3.489477in}{3.113530in}}%
\pgfpathlineto{\pgfqpoint{3.497238in}{3.125083in}}%
\pgfpathlineto{\pgfqpoint{3.507407in}{3.139460in}}%
\pgfpathlineto{\pgfqpoint{3.512491in}{3.143127in}}%
\pgfpathlineto{\pgfqpoint{3.516505in}{3.143581in}}%
\pgfpathlineto{\pgfqpoint{3.520520in}{3.141801in}}%
\pgfpathlineto{\pgfqpoint{3.525336in}{3.137039in}}%
\pgfpathlineto{\pgfqpoint{3.532829in}{3.125962in}}%
\pgfpathlineto{\pgfqpoint{3.543534in}{3.110810in}}%
\pgfpathlineto{\pgfqpoint{3.548618in}{3.107236in}}%
\pgfpathlineto{\pgfqpoint{3.552632in}{3.106864in}}%
\pgfpathlineto{\pgfqpoint{3.556647in}{3.108723in}}%
\pgfpathlineto{\pgfqpoint{3.561464in}{3.113563in}}%
\pgfpathlineto{\pgfqpoint{3.569224in}{3.125125in}}%
\pgfpathlineto{\pgfqpoint{3.579393in}{3.139486in}}%
\pgfpathlineto{\pgfqpoint{3.584478in}{3.143137in}}%
\pgfpathlineto{\pgfqpoint{3.588492in}{3.143577in}}%
\pgfpathlineto{\pgfqpoint{3.592506in}{3.141783in}}%
\pgfpathlineto{\pgfqpoint{3.597323in}{3.137007in}}%
\pgfpathlineto{\pgfqpoint{3.604816in}{3.125920in}}%
\pgfpathlineto{\pgfqpoint{3.615253in}{3.111057in}}%
\pgfpathlineto{\pgfqpoint{3.620337in}{3.107330in}}%
\pgfpathlineto{\pgfqpoint{3.624351in}{3.106823in}}%
\pgfpathlineto{\pgfqpoint{3.628365in}{3.108551in}}%
\pgfpathlineto{\pgfqpoint{3.633182in}{3.113263in}}%
\pgfpathlineto{\pgfqpoint{3.640675in}{3.124304in}}%
\pgfpathlineto{\pgfqpoint{3.651380in}{3.139513in}}%
\pgfpathlineto{\pgfqpoint{3.656464in}{3.143147in}}%
\pgfpathlineto{\pgfqpoint{3.660478in}{3.143572in}}%
\pgfpathlineto{\pgfqpoint{3.664493in}{3.141764in}}%
\pgfpathlineto{\pgfqpoint{3.669309in}{3.136974in}}%
\pgfpathlineto{\pgfqpoint{3.676802in}{3.125878in}}%
\pgfpathlineto{\pgfqpoint{3.687239in}{3.111030in}}%
\pgfpathlineto{\pgfqpoint{3.692324in}{3.107319in}}%
\pgfpathlineto{\pgfqpoint{3.696338in}{3.106827in}}%
\pgfpathlineto{\pgfqpoint{3.700352in}{3.108569in}}%
\pgfpathlineto{\pgfqpoint{3.705169in}{3.113295in}}%
\pgfpathlineto{\pgfqpoint{3.712662in}{3.124346in}}%
\pgfpathlineto{\pgfqpoint{3.723366in}{3.139540in}}%
\pgfpathlineto{\pgfqpoint{3.728451in}{3.143157in}}%
\pgfpathlineto{\pgfqpoint{3.732465in}{3.143567in}}%
\pgfpathlineto{\pgfqpoint{3.736479in}{3.141745in}}%
\pgfpathlineto{\pgfqpoint{3.741296in}{3.136942in}}%
\pgfpathlineto{\pgfqpoint{3.749057in}{3.125404in}}%
\pgfpathlineto{\pgfqpoint{3.759226in}{3.111003in}}%
\pgfpathlineto{\pgfqpoint{3.764310in}{3.107309in}}%
\pgfpathlineto{\pgfqpoint{3.768324in}{3.106831in}}%
\pgfpathlineto{\pgfqpoint{3.772338in}{3.108588in}}%
\pgfpathlineto{\pgfqpoint{3.777155in}{3.113327in}}%
\pgfpathlineto{\pgfqpoint{3.784648in}{3.124389in}}%
\pgfpathlineto{\pgfqpoint{3.795353in}{3.139566in}}%
\pgfpathlineto{\pgfqpoint{3.800437in}{3.143167in}}%
\pgfpathlineto{\pgfqpoint{3.804451in}{3.143563in}}%
\pgfpathlineto{\pgfqpoint{3.808466in}{3.141726in}}%
\pgfpathlineto{\pgfqpoint{3.813282in}{3.136909in}}%
\pgfpathlineto{\pgfqpoint{3.821043in}{3.125362in}}%
\pgfpathlineto{\pgfqpoint{3.831212in}{3.110976in}}%
\pgfpathlineto{\pgfqpoint{3.836297in}{3.107298in}}%
\pgfpathlineto{\pgfqpoint{3.840311in}{3.106835in}}%
\pgfpathlineto{\pgfqpoint{3.844325in}{3.108606in}}%
\pgfpathlineto{\pgfqpoint{3.849142in}{3.113360in}}%
\pgfpathlineto{\pgfqpoint{3.856635in}{3.124431in}}%
\pgfpathlineto{\pgfqpoint{3.867339in}{3.139593in}}%
\pgfpathlineto{\pgfqpoint{3.872424in}{3.143177in}}%
\pgfpathlineto{\pgfqpoint{3.876438in}{3.143558in}}%
\pgfpathlineto{\pgfqpoint{3.880452in}{3.141707in}}%
\pgfpathlineto{\pgfqpoint{3.885269in}{3.136877in}}%
\pgfpathlineto{\pgfqpoint{3.893030in}{3.125320in}}%
\pgfpathlineto{\pgfqpoint{3.903199in}{3.110949in}}%
\pgfpathlineto{\pgfqpoint{3.908283in}{3.107288in}}%
\pgfpathlineto{\pgfqpoint{3.912297in}{3.106840in}}%
\pgfpathlineto{\pgfqpoint{3.916311in}{3.108625in}}%
\pgfpathlineto{\pgfqpoint{3.921128in}{3.113392in}}%
\pgfpathlineto{\pgfqpoint{3.928621in}{3.124473in}}%
\pgfpathlineto{\pgfqpoint{3.939058in}{3.139345in}}%
\pgfpathlineto{\pgfqpoint{3.944143in}{3.143083in}}%
\pgfpathlineto{\pgfqpoint{3.948157in}{3.143599in}}%
\pgfpathlineto{\pgfqpoint{3.952171in}{3.141879in}}%
\pgfpathlineto{\pgfqpoint{3.956988in}{3.137176in}}%
\pgfpathlineto{\pgfqpoint{3.964481in}{3.126141in}}%
\pgfpathlineto{\pgfqpoint{3.975185in}{3.110922in}}%
\pgfpathlineto{\pgfqpoint{3.980270in}{3.107278in}}%
\pgfpathlineto{\pgfqpoint{3.984284in}{3.106844in}}%
\pgfpathlineto{\pgfqpoint{3.988298in}{3.108643in}}%
\pgfpathlineto{\pgfqpoint{3.993115in}{3.113424in}}%
\pgfpathlineto{\pgfqpoint{4.000608in}{3.124515in}}%
\pgfpathlineto{\pgfqpoint{4.011045in}{3.139372in}}%
\pgfpathlineto{\pgfqpoint{4.016129in}{3.143093in}}%
\pgfpathlineto{\pgfqpoint{4.020143in}{3.143595in}}%
\pgfpathlineto{\pgfqpoint{4.024157in}{3.141861in}}%
\pgfpathlineto{\pgfqpoint{4.028974in}{3.137144in}}%
\pgfpathlineto{\pgfqpoint{4.036467in}{3.126099in}}%
\pgfpathlineto{\pgfqpoint{4.047172in}{3.110896in}}%
\pgfpathlineto{\pgfqpoint{4.052256in}{3.107268in}}%
\pgfpathlineto{\pgfqpoint{4.056270in}{3.106849in}}%
\pgfpathlineto{\pgfqpoint{4.060284in}{3.108662in}}%
\pgfpathlineto{\pgfqpoint{4.065101in}{3.113457in}}%
\pgfpathlineto{\pgfqpoint{4.072862in}{3.124988in}}%
\pgfpathlineto{\pgfqpoint{4.083031in}{3.139399in}}%
\pgfpathlineto{\pgfqpoint{4.088116in}{3.143104in}}%
\pgfpathlineto{\pgfqpoint{4.092130in}{3.143591in}}%
\pgfpathlineto{\pgfqpoint{4.096144in}{3.141843in}}%
\pgfpathlineto{\pgfqpoint{4.100961in}{3.137112in}}%
\pgfpathlineto{\pgfqpoint{4.108454in}{3.126057in}}%
\pgfpathlineto{\pgfqpoint{4.119158in}{3.110869in}}%
\pgfpathlineto{\pgfqpoint{4.124243in}{3.107258in}}%
\pgfpathlineto{\pgfqpoint{4.128257in}{3.106853in}}%
\pgfpathlineto{\pgfqpoint{4.132271in}{3.108681in}}%
\pgfpathlineto{\pgfqpoint{4.137088in}{3.113489in}}%
\pgfpathlineto{\pgfqpoint{4.144849in}{3.125030in}}%
\pgfpathlineto{\pgfqpoint{4.155018in}{3.139426in}}%
\pgfpathlineto{\pgfqpoint{4.160102in}{3.143114in}}%
\pgfpathlineto{\pgfqpoint{4.164116in}{3.143586in}}%
\pgfpathlineto{\pgfqpoint{4.168130in}{3.141824in}}%
\pgfpathlineto{\pgfqpoint{4.172947in}{3.137079in}}%
\pgfpathlineto{\pgfqpoint{4.180440in}{3.126015in}}%
\pgfpathlineto{\pgfqpoint{4.191145in}{3.110843in}}%
\pgfpathlineto{\pgfqpoint{4.196229in}{3.107248in}}%
\pgfpathlineto{\pgfqpoint{4.200243in}{3.106858in}}%
\pgfpathlineto{\pgfqpoint{4.204257in}{3.108700in}}%
\pgfpathlineto{\pgfqpoint{4.209074in}{3.113522in}}%
\pgfpathlineto{\pgfqpoint{4.216835in}{3.125073in}}%
\pgfpathlineto{\pgfqpoint{4.227004in}{3.139453in}}%
\pgfpathlineto{\pgfqpoint{4.232089in}{3.143125in}}%
\pgfpathlineto{\pgfqpoint{4.236103in}{3.143582in}}%
\pgfpathlineto{\pgfqpoint{4.240117in}{3.141806in}}%
\pgfpathlineto{\pgfqpoint{4.244934in}{3.137047in}}%
\pgfpathlineto{\pgfqpoint{4.252427in}{3.125972in}}%
\pgfpathlineto{\pgfqpoint{4.263131in}{3.110816in}}%
\pgfpathlineto{\pgfqpoint{4.268216in}{3.107238in}}%
\pgfpathlineto{\pgfqpoint{4.272230in}{3.106863in}}%
\pgfpathlineto{\pgfqpoint{4.276244in}{3.108719in}}%
\pgfpathlineto{\pgfqpoint{4.281061in}{3.113555in}}%
\pgfpathlineto{\pgfqpoint{4.288822in}{3.125115in}}%
\pgfpathlineto{\pgfqpoint{4.298991in}{3.139480in}}%
\pgfpathlineto{\pgfqpoint{4.304075in}{3.143135in}}%
\pgfpathlineto{\pgfqpoint{4.308089in}{3.143578in}}%
\pgfpathlineto{\pgfqpoint{4.312103in}{3.141787in}}%
\pgfpathlineto{\pgfqpoint{4.316920in}{3.137015in}}%
\pgfpathlineto{\pgfqpoint{4.324413in}{3.125930in}}%
\pgfpathlineto{\pgfqpoint{4.334850in}{3.111064in}}%
\pgfpathlineto{\pgfqpoint{4.339935in}{3.107332in}}%
\pgfpathlineto{\pgfqpoint{4.343949in}{3.106822in}}%
\pgfpathlineto{\pgfqpoint{4.347963in}{3.108547in}}%
\pgfpathlineto{\pgfqpoint{4.352780in}{3.113255in}}%
\pgfpathlineto{\pgfqpoint{4.360273in}{3.124294in}}%
\pgfpathlineto{\pgfqpoint{4.370977in}{3.139506in}}%
\pgfpathlineto{\pgfqpoint{4.376062in}{3.143145in}}%
\pgfpathlineto{\pgfqpoint{4.380076in}{3.143573in}}%
\pgfpathlineto{\pgfqpoint{4.384090in}{3.141769in}}%
\pgfpathlineto{\pgfqpoint{4.388907in}{3.136982in}}%
\pgfpathlineto{\pgfqpoint{4.396400in}{3.125888in}}%
\pgfpathlineto{\pgfqpoint{4.406837in}{3.111037in}}%
\pgfpathlineto{\pgfqpoint{4.411921in}{3.107322in}}%
\pgfpathlineto{\pgfqpoint{4.415935in}{3.106826in}}%
\pgfpathlineto{\pgfqpoint{4.419949in}{3.108565in}}%
\pgfpathlineto{\pgfqpoint{4.424766in}{3.113287in}}%
\pgfpathlineto{\pgfqpoint{4.432259in}{3.124336in}}%
\pgfpathlineto{\pgfqpoint{4.442964in}{3.139533in}}%
\pgfpathlineto{\pgfqpoint{4.448048in}{3.143155in}}%
\pgfpathlineto{\pgfqpoint{4.452062in}{3.143569in}}%
\pgfpathlineto{\pgfqpoint{4.456076in}{3.141750in}}%
\pgfpathlineto{\pgfqpoint{4.460893in}{3.136950in}}%
\pgfpathlineto{\pgfqpoint{4.468654in}{3.125415in}}%
\pgfpathlineto{\pgfqpoint{4.478823in}{3.111010in}}%
\pgfpathlineto{\pgfqpoint{4.483908in}{3.107311in}}%
\pgfpathlineto{\pgfqpoint{4.487922in}{3.106830in}}%
\pgfpathlineto{\pgfqpoint{4.491936in}{3.108583in}}%
\pgfpathlineto{\pgfqpoint{4.496753in}{3.113319in}}%
\pgfpathlineto{\pgfqpoint{4.504246in}{3.124378in}}%
\pgfpathlineto{\pgfqpoint{4.514950in}{3.139560in}}%
\pgfpathlineto{\pgfqpoint{4.520035in}{3.143165in}}%
\pgfpathlineto{\pgfqpoint{4.524049in}{3.143564in}}%
\pgfpathlineto{\pgfqpoint{4.528063in}{3.141731in}}%
\pgfpathlineto{\pgfqpoint{4.532880in}{3.136917in}}%
\pgfpathlineto{\pgfqpoint{4.540641in}{3.125373in}}%
\pgfpathlineto{\pgfqpoint{4.550810in}{3.110983in}}%
\pgfpathlineto{\pgfqpoint{4.555894in}{3.107301in}}%
\pgfpathlineto{\pgfqpoint{4.559908in}{3.106834in}}%
\pgfpathlineto{\pgfqpoint{4.563922in}{3.108602in}}%
\pgfpathlineto{\pgfqpoint{4.568739in}{3.113352in}}%
\pgfpathlineto{\pgfqpoint{4.576232in}{3.124420in}}%
\pgfpathlineto{\pgfqpoint{4.586937in}{3.139586in}}%
\pgfpathlineto{\pgfqpoint{4.592021in}{3.143175in}}%
\pgfpathlineto{\pgfqpoint{4.596035in}{3.143559in}}%
\pgfpathlineto{\pgfqpoint{4.600049in}{3.141712in}}%
\pgfpathlineto{\pgfqpoint{4.604866in}{3.136885in}}%
\pgfpathlineto{\pgfqpoint{4.612627in}{3.125331in}}%
\pgfpathlineto{\pgfqpoint{4.622796in}{3.110956in}}%
\pgfpathlineto{\pgfqpoint{4.627881in}{3.107291in}}%
\pgfpathlineto{\pgfqpoint{4.631895in}{3.106838in}}%
\pgfpathlineto{\pgfqpoint{4.635909in}{3.108620in}}%
\pgfpathlineto{\pgfqpoint{4.640726in}{3.113384in}}%
\pgfpathlineto{\pgfqpoint{4.648219in}{3.124462in}}%
\pgfpathlineto{\pgfqpoint{4.658923in}{3.139613in}}%
\pgfpathlineto{\pgfqpoint{4.664008in}{3.143184in}}%
\pgfpathlineto{\pgfqpoint{4.668022in}{3.143554in}}%
\pgfpathlineto{\pgfqpoint{4.672036in}{3.141693in}}%
\pgfpathlineto{\pgfqpoint{4.676853in}{3.136852in}}%
\pgfpathlineto{\pgfqpoint{4.684614in}{3.125288in}}%
\pgfpathlineto{\pgfqpoint{4.694783in}{3.110929in}}%
\pgfpathlineto{\pgfqpoint{4.699867in}{3.107280in}}%
\pgfpathlineto{\pgfqpoint{4.703881in}{3.106843in}}%
\pgfpathlineto{\pgfqpoint{4.707895in}{3.108639in}}%
\pgfpathlineto{\pgfqpoint{4.712712in}{3.113416in}}%
\pgfpathlineto{\pgfqpoint{4.720205in}{3.124504in}}%
\pgfpathlineto{\pgfqpoint{4.730642in}{3.139365in}}%
\pgfpathlineto{\pgfqpoint{4.735727in}{3.143091in}}%
\pgfpathlineto{\pgfqpoint{4.739741in}{3.143596in}}%
\pgfpathlineto{\pgfqpoint{4.743755in}{3.141865in}}%
\pgfpathlineto{\pgfqpoint{4.748572in}{3.137152in}}%
\pgfpathlineto{\pgfqpoint{4.756065in}{3.126109in}}%
\pgfpathlineto{\pgfqpoint{4.766769in}{3.110902in}}%
\pgfpathlineto{\pgfqpoint{4.771854in}{3.107270in}}%
\pgfpathlineto{\pgfqpoint{4.775868in}{3.106847in}}%
\pgfpathlineto{\pgfqpoint{4.779882in}{3.108657in}}%
\pgfpathlineto{\pgfqpoint{4.784699in}{3.113449in}}%
\pgfpathlineto{\pgfqpoint{4.792192in}{3.124546in}}%
\pgfpathlineto{\pgfqpoint{4.802629in}{3.139392in}}%
\pgfpathlineto{\pgfqpoint{4.807713in}{3.143101in}}%
\pgfpathlineto{\pgfqpoint{4.811727in}{3.143592in}}%
\pgfpathlineto{\pgfqpoint{4.815741in}{3.141847in}}%
\pgfpathlineto{\pgfqpoint{4.820558in}{3.137120in}}%
\pgfpathlineto{\pgfqpoint{4.828051in}{3.126067in}}%
\pgfpathlineto{\pgfqpoint{4.838756in}{3.110876in}}%
\pgfpathlineto{\pgfqpoint{4.843840in}{3.107260in}}%
\pgfpathlineto{\pgfqpoint{4.847854in}{3.106852in}}%
\pgfpathlineto{\pgfqpoint{4.851868in}{3.108676in}}%
\pgfpathlineto{\pgfqpoint{4.856685in}{3.113481in}}%
\pgfpathlineto{\pgfqpoint{4.864446in}{3.125020in}}%
\pgfpathlineto{\pgfqpoint{4.874615in}{3.139419in}}%
\pgfpathlineto{\pgfqpoint{4.879700in}{3.143112in}}%
\pgfpathlineto{\pgfqpoint{4.883714in}{3.143587in}}%
\pgfpathlineto{\pgfqpoint{4.887728in}{3.141829in}}%
\pgfpathlineto{\pgfqpoint{4.892545in}{3.137088in}}%
\pgfpathlineto{\pgfqpoint{4.900038in}{3.126025in}}%
\pgfpathlineto{\pgfqpoint{4.910742in}{3.110849in}}%
\pgfpathlineto{\pgfqpoint{4.915827in}{3.107250in}}%
\pgfpathlineto{\pgfqpoint{4.919841in}{3.106857in}}%
\pgfpathlineto{\pgfqpoint{4.923855in}{3.108695in}}%
\pgfpathlineto{\pgfqpoint{4.928672in}{3.113514in}}%
\pgfpathlineto{\pgfqpoint{4.936432in}{3.125062in}}%
\pgfpathlineto{\pgfqpoint{4.946602in}{3.139446in}}%
\pgfpathlineto{\pgfqpoint{4.951686in}{3.143122in}}%
\pgfpathlineto{\pgfqpoint{4.955700in}{3.143583in}}%
\pgfpathlineto{\pgfqpoint{4.959714in}{3.141810in}}%
\pgfpathlineto{\pgfqpoint{4.964531in}{3.137055in}}%
\pgfpathlineto{\pgfqpoint{4.972024in}{3.125983in}}%
\pgfpathlineto{\pgfqpoint{4.982729in}{3.110823in}}%
\pgfpathlineto{\pgfqpoint{4.987813in}{3.107241in}}%
\pgfpathlineto{\pgfqpoint{4.991827in}{3.106862in}}%
\pgfpathlineto{\pgfqpoint{4.995841in}{3.108714in}}%
\pgfpathlineto{\pgfqpoint{5.000658in}{3.113546in}}%
\pgfpathlineto{\pgfqpoint{5.008419in}{3.125104in}}%
\pgfpathlineto{\pgfqpoint{5.018588in}{3.139473in}}%
\pgfpathlineto{\pgfqpoint{5.023673in}{3.143132in}}%
\pgfpathlineto{\pgfqpoint{5.027687in}{3.143579in}}%
\pgfpathlineto{\pgfqpoint{5.031701in}{3.141792in}}%
\pgfpathlineto{\pgfqpoint{5.036518in}{3.137023in}}%
\pgfpathlineto{\pgfqpoint{5.044011in}{3.125941in}}%
\pgfpathlineto{\pgfqpoint{5.054448in}{3.111070in}}%
\pgfpathlineto{\pgfqpoint{5.059532in}{3.107335in}}%
\pgfpathlineto{\pgfqpoint{5.063546in}{3.106821in}}%
\pgfpathlineto{\pgfqpoint{5.067560in}{3.108542in}}%
\pgfpathlineto{\pgfqpoint{5.072377in}{3.113247in}}%
\pgfpathlineto{\pgfqpoint{5.079870in}{3.124283in}}%
\pgfpathlineto{\pgfqpoint{5.090575in}{3.139500in}}%
\pgfpathlineto{\pgfqpoint{5.095659in}{3.143142in}}%
\pgfpathlineto{\pgfqpoint{5.099673in}{3.143574in}}%
\pgfpathlineto{\pgfqpoint{5.103687in}{3.141773in}}%
\pgfpathlineto{\pgfqpoint{5.108504in}{3.136990in}}%
\pgfpathlineto{\pgfqpoint{5.115997in}{3.125899in}}%
\pgfpathlineto{\pgfqpoint{5.126434in}{3.111043in}}%
\pgfpathlineto{\pgfqpoint{5.131519in}{3.107324in}}%
\pgfpathlineto{\pgfqpoint{5.135533in}{3.106825in}}%
\pgfpathlineto{\pgfqpoint{5.139547in}{3.108560in}}%
\pgfpathlineto{\pgfqpoint{5.144364in}{3.113279in}}%
\pgfpathlineto{\pgfqpoint{5.151857in}{3.124325in}}%
\pgfpathlineto{\pgfqpoint{5.162561in}{3.139526in}}%
\pgfpathlineto{\pgfqpoint{5.167646in}{3.143152in}}%
\pgfpathlineto{\pgfqpoint{5.171660in}{3.143570in}}%
\pgfpathlineto{\pgfqpoint{5.175674in}{3.141755in}}%
\pgfpathlineto{\pgfqpoint{5.180491in}{3.136958in}}%
\pgfpathlineto{\pgfqpoint{5.188251in}{3.125425in}}%
\pgfpathlineto{\pgfqpoint{5.198421in}{3.111016in}}%
\pgfpathlineto{\pgfqpoint{5.203505in}{3.107314in}}%
\pgfpathlineto{\pgfqpoint{5.207519in}{3.106829in}}%
\pgfpathlineto{\pgfqpoint{5.211533in}{3.108579in}}%
\pgfpathlineto{\pgfqpoint{5.216350in}{3.113311in}}%
\pgfpathlineto{\pgfqpoint{5.223843in}{3.124368in}}%
\pgfpathlineto{\pgfqpoint{5.234548in}{3.139553in}}%
\pgfpathlineto{\pgfqpoint{5.239632in}{3.143162in}}%
\pgfpathlineto{\pgfqpoint{5.243646in}{3.143565in}}%
\pgfpathlineto{\pgfqpoint{5.247660in}{3.141736in}}%
\pgfpathlineto{\pgfqpoint{5.252477in}{3.136925in}}%
\pgfpathlineto{\pgfqpoint{5.260238in}{3.125383in}}%
\pgfpathlineto{\pgfqpoint{5.270407in}{3.110989in}}%
\pgfpathlineto{\pgfqpoint{5.275492in}{3.107303in}}%
\pgfpathlineto{\pgfqpoint{5.279506in}{3.106833in}}%
\pgfpathlineto{\pgfqpoint{5.283520in}{3.108597in}}%
\pgfpathlineto{\pgfqpoint{5.288337in}{3.113343in}}%
\pgfpathlineto{\pgfqpoint{5.295830in}{3.124410in}}%
\pgfpathlineto{\pgfqpoint{5.306534in}{3.139580in}}%
\pgfpathlineto{\pgfqpoint{5.311619in}{3.143172in}}%
\pgfpathlineto{\pgfqpoint{5.315633in}{3.143560in}}%
\pgfpathlineto{\pgfqpoint{5.319647in}{3.141717in}}%
\pgfpathlineto{\pgfqpoint{5.324464in}{3.136893in}}%
\pgfpathlineto{\pgfqpoint{5.332224in}{3.125341in}}%
\pgfpathlineto{\pgfqpoint{5.342394in}{3.110963in}}%
\pgfpathlineto{\pgfqpoint{5.347478in}{3.107293in}}%
\pgfpathlineto{\pgfqpoint{5.351492in}{3.106837in}}%
\pgfpathlineto{\pgfqpoint{5.355506in}{3.108615in}}%
\pgfpathlineto{\pgfqpoint{5.360323in}{3.113376in}}%
\pgfpathlineto{\pgfqpoint{5.367816in}{3.124452in}}%
\pgfpathlineto{\pgfqpoint{5.378521in}{3.139606in}}%
\pgfpathlineto{\pgfqpoint{5.383605in}{3.143182in}}%
\pgfpathlineto{\pgfqpoint{5.387619in}{3.143555in}}%
\pgfpathlineto{\pgfqpoint{5.391633in}{3.141698in}}%
\pgfpathlineto{\pgfqpoint{5.396450in}{3.136860in}}%
\pgfpathlineto{\pgfqpoint{5.404211in}{3.125299in}}%
\pgfpathlineto{\pgfqpoint{5.414380in}{3.110936in}}%
\pgfpathlineto{\pgfqpoint{5.419465in}{3.107283in}}%
\pgfpathlineto{\pgfqpoint{5.423479in}{3.106842in}}%
\pgfpathlineto{\pgfqpoint{5.427493in}{3.108634in}}%
\pgfpathlineto{\pgfqpoint{5.432310in}{3.113408in}}%
\pgfpathlineto{\pgfqpoint{5.439803in}{3.124494in}}%
\pgfpathlineto{\pgfqpoint{5.450240in}{3.139359in}}%
\pgfpathlineto{\pgfqpoint{5.455324in}{3.143088in}}%
\pgfpathlineto{\pgfqpoint{5.459338in}{3.143597in}}%
\pgfpathlineto{\pgfqpoint{5.463352in}{3.141870in}}%
\pgfpathlineto{\pgfqpoint{5.468169in}{3.137160in}}%
\pgfpathlineto{\pgfqpoint{5.475662in}{3.126120in}}%
\pgfpathlineto{\pgfqpoint{5.486367in}{3.110909in}}%
\pgfpathlineto{\pgfqpoint{5.491451in}{3.107273in}}%
\pgfpathlineto{\pgfqpoint{5.495465in}{3.106846in}}%
\pgfpathlineto{\pgfqpoint{5.499479in}{3.108653in}}%
\pgfpathlineto{\pgfqpoint{5.504296in}{3.113441in}}%
\pgfpathlineto{\pgfqpoint{5.511789in}{3.124536in}}%
\pgfpathlineto{\pgfqpoint{5.522226in}{3.139386in}}%
\pgfpathlineto{\pgfqpoint{5.527311in}{3.143099in}}%
\pgfpathlineto{\pgfqpoint{5.531325in}{3.143593in}}%
\pgfpathlineto{\pgfqpoint{5.535339in}{3.141852in}}%
\pgfpathlineto{\pgfqpoint{5.540156in}{3.137128in}}%
\pgfpathlineto{\pgfqpoint{5.547649in}{3.126078in}}%
\pgfpathlineto{\pgfqpoint{5.558353in}{3.110882in}}%
\pgfpathlineto{\pgfqpoint{5.563438in}{3.107263in}}%
\pgfpathlineto{\pgfqpoint{5.567452in}{3.106851in}}%
\pgfpathlineto{\pgfqpoint{5.571466in}{3.108671in}}%
\pgfpathlineto{\pgfqpoint{5.576283in}{3.113473in}}%
\pgfpathlineto{\pgfqpoint{5.584043in}{3.125009in}}%
\pgfpathlineto{\pgfqpoint{5.594213in}{3.139413in}}%
\pgfpathlineto{\pgfqpoint{5.599297in}{3.143109in}}%
\pgfpathlineto{\pgfqpoint{5.603311in}{3.143589in}}%
\pgfpathlineto{\pgfqpoint{5.607325in}{3.141833in}}%
\pgfpathlineto{\pgfqpoint{5.612142in}{3.137096in}}%
\pgfpathlineto{\pgfqpoint{5.619635in}{3.126036in}}%
\pgfpathlineto{\pgfqpoint{5.630340in}{3.110856in}}%
\pgfpathlineto{\pgfqpoint{5.635424in}{3.107253in}}%
\pgfpathlineto{\pgfqpoint{5.639438in}{3.106856in}}%
\pgfpathlineto{\pgfqpoint{5.643452in}{3.108690in}}%
\pgfpathlineto{\pgfqpoint{5.648269in}{3.113506in}}%
\pgfpathlineto{\pgfqpoint{5.656030in}{3.125052in}}%
\pgfpathlineto{\pgfqpoint{5.666199in}{3.139440in}}%
\pgfpathlineto{\pgfqpoint{5.671284in}{3.143119in}}%
\pgfpathlineto{\pgfqpoint{5.675298in}{3.143584in}}%
\pgfpathlineto{\pgfqpoint{5.679312in}{3.141815in}}%
\pgfpathlineto{\pgfqpoint{5.684129in}{3.137063in}}%
\pgfpathlineto{\pgfqpoint{5.691622in}{3.125994in}}%
\pgfpathlineto{\pgfqpoint{5.702326in}{3.110829in}}%
\pgfpathlineto{\pgfqpoint{5.707411in}{3.107243in}}%
\pgfpathlineto{\pgfqpoint{5.711425in}{3.106860in}}%
\pgfpathlineto{\pgfqpoint{5.715439in}{3.108709in}}%
\pgfpathlineto{\pgfqpoint{5.720256in}{3.113538in}}%
\pgfpathlineto{\pgfqpoint{5.728016in}{3.125094in}}%
\pgfpathlineto{\pgfqpoint{5.738186in}{3.139466in}}%
\pgfpathlineto{\pgfqpoint{5.743270in}{3.143130in}}%
\pgfpathlineto{\pgfqpoint{5.747284in}{3.143580in}}%
\pgfpathlineto{\pgfqpoint{5.751298in}{3.141796in}}%
\pgfpathlineto{\pgfqpoint{5.756115in}{3.137031in}}%
\pgfpathlineto{\pgfqpoint{5.763608in}{3.125951in}}%
\pgfpathlineto{\pgfqpoint{5.774045in}{3.111077in}}%
\pgfpathlineto{\pgfqpoint{5.779130in}{3.107338in}}%
\pgfpathlineto{\pgfqpoint{5.783144in}{3.106820in}}%
\pgfpathlineto{\pgfqpoint{5.787158in}{3.108538in}}%
\pgfpathlineto{\pgfqpoint{5.791975in}{3.113239in}}%
\pgfpathlineto{\pgfqpoint{5.799468in}{3.124273in}}%
\pgfpathlineto{\pgfqpoint{5.810172in}{3.139493in}}%
\pgfpathlineto{\pgfqpoint{5.815257in}{3.143140in}}%
\pgfpathlineto{\pgfqpoint{5.819271in}{3.143575in}}%
\pgfpathlineto{\pgfqpoint{5.823285in}{3.141778in}}%
\pgfpathlineto{\pgfqpoint{5.828102in}{3.136999in}}%
\pgfpathlineto{\pgfqpoint{5.835595in}{3.125909in}}%
\pgfpathlineto{\pgfqpoint{5.846031in}{3.111050in}}%
\pgfpathlineto{\pgfqpoint{5.851116in}{3.107327in}}%
\pgfpathlineto{\pgfqpoint{5.855130in}{3.106824in}}%
\pgfpathlineto{\pgfqpoint{5.859144in}{3.108556in}}%
\pgfpathlineto{\pgfqpoint{5.863961in}{3.113271in}}%
\pgfpathlineto{\pgfqpoint{5.871454in}{3.124315in}}%
\pgfpathlineto{\pgfqpoint{5.882159in}{3.139520in}}%
\pgfpathlineto{\pgfqpoint{5.887243in}{3.143150in}}%
\pgfpathlineto{\pgfqpoint{5.891257in}{3.143571in}}%
\pgfpathlineto{\pgfqpoint{5.895271in}{3.141759in}}%
\pgfpathlineto{\pgfqpoint{5.900088in}{3.136966in}}%
\pgfpathlineto{\pgfqpoint{5.907849in}{3.125436in}}%
\pgfpathlineto{\pgfqpoint{5.918018in}{3.111023in}}%
\pgfpathlineto{\pgfqpoint{5.923103in}{3.107316in}}%
\pgfpathlineto{\pgfqpoint{5.927117in}{3.106828in}}%
\pgfpathlineto{\pgfqpoint{5.931131in}{3.108574in}}%
\pgfpathlineto{\pgfqpoint{5.935948in}{3.113303in}}%
\pgfpathlineto{\pgfqpoint{5.943441in}{3.124357in}}%
\pgfpathlineto{\pgfqpoint{5.954145in}{3.139546in}}%
\pgfpathlineto{\pgfqpoint{5.959230in}{3.143160in}}%
\pgfpathlineto{\pgfqpoint{5.963244in}{3.143566in}}%
\pgfpathlineto{\pgfqpoint{5.967258in}{3.141740in}}%
\pgfpathlineto{\pgfqpoint{5.972075in}{3.136934in}}%
\pgfpathlineto{\pgfqpoint{5.979835in}{3.125394in}}%
\pgfpathlineto{\pgfqpoint{5.990004in}{3.110996in}}%
\pgfpathlineto{\pgfqpoint{5.995089in}{3.107306in}}%
\pgfpathlineto{\pgfqpoint{5.999103in}{3.106832in}}%
\pgfpathlineto{\pgfqpoint{6.003117in}{3.108592in}}%
\pgfpathlineto{\pgfqpoint{6.007934in}{3.113335in}}%
\pgfpathlineto{\pgfqpoint{6.015427in}{3.124399in}}%
\pgfpathlineto{\pgfqpoint{6.026132in}{3.139573in}}%
\pgfpathlineto{\pgfqpoint{6.031216in}{3.143170in}}%
\pgfpathlineto{\pgfqpoint{6.035230in}{3.143561in}}%
\pgfpathlineto{\pgfqpoint{6.039244in}{3.141722in}}%
\pgfpathlineto{\pgfqpoint{6.044061in}{3.136901in}}%
\pgfpathlineto{\pgfqpoint{6.051822in}{3.125352in}}%
\pgfpathlineto{\pgfqpoint{6.061991in}{3.110969in}}%
\pgfpathlineto{\pgfqpoint{6.067076in}{3.107296in}}%
\pgfpathlineto{\pgfqpoint{6.071090in}{3.106836in}}%
\pgfpathlineto{\pgfqpoint{6.075104in}{3.108611in}}%
\pgfpathlineto{\pgfqpoint{6.079921in}{3.113368in}}%
\pgfpathlineto{\pgfqpoint{6.087414in}{3.124441in}}%
\pgfpathlineto{\pgfqpoint{6.098118in}{3.139599in}}%
\pgfpathlineto{\pgfqpoint{6.103203in}{3.143180in}}%
\pgfpathlineto{\pgfqpoint{6.107217in}{3.143557in}}%
\pgfpathlineto{\pgfqpoint{6.111231in}{3.141703in}}%
\pgfpathlineto{\pgfqpoint{6.116048in}{3.136868in}}%
\pgfpathlineto{\pgfqpoint{6.123808in}{3.125309in}}%
\pgfpathlineto{\pgfqpoint{6.133977in}{3.110942in}}%
\pgfpathlineto{\pgfqpoint{6.139062in}{3.107285in}}%
\pgfpathlineto{\pgfqpoint{6.143076in}{3.106841in}}%
\pgfpathlineto{\pgfqpoint{6.147090in}{3.108629in}}%
\pgfpathlineto{\pgfqpoint{6.151907in}{3.113400in}}%
\pgfpathlineto{\pgfqpoint{6.159400in}{3.124483in}}%
\pgfpathlineto{\pgfqpoint{6.169837in}{3.139352in}}%
\pgfpathlineto{\pgfqpoint{6.174921in}{3.143085in}}%
\pgfpathlineto{\pgfqpoint{6.178936in}{3.143598in}}%
\pgfpathlineto{\pgfqpoint{6.182950in}{3.141874in}}%
\pgfpathlineto{\pgfqpoint{6.187767in}{3.137168in}}%
\pgfpathlineto{\pgfqpoint{6.195260in}{3.126130in}}%
\pgfpathlineto{\pgfqpoint{6.205964in}{3.110916in}}%
\pgfpathlineto{\pgfqpoint{6.211049in}{3.107275in}}%
\pgfpathlineto{\pgfqpoint{6.215063in}{3.106845in}}%
\pgfpathlineto{\pgfqpoint{6.219077in}{3.108648in}}%
\pgfpathlineto{\pgfqpoint{6.223894in}{3.113432in}}%
\pgfpathlineto{\pgfqpoint{6.231387in}{3.124525in}}%
\pgfpathlineto{\pgfqpoint{6.241823in}{3.139379in}}%
\pgfpathlineto{\pgfqpoint{6.246908in}{3.143096in}}%
\pgfpathlineto{\pgfqpoint{6.250922in}{3.143594in}}%
\pgfpathlineto{\pgfqpoint{6.254936in}{3.141856in}}%
\pgfpathlineto{\pgfqpoint{6.259753in}{3.137136in}}%
\pgfpathlineto{\pgfqpoint{6.267246in}{3.126088in}}%
\pgfpathlineto{\pgfqpoint{6.277950in}{3.110889in}}%
\pgfpathlineto{\pgfqpoint{6.283035in}{3.107265in}}%
\pgfpathlineto{\pgfqpoint{6.287049in}{3.106850in}}%
\pgfpathlineto{\pgfqpoint{6.291063in}{3.108667in}}%
\pgfpathlineto{\pgfqpoint{6.295880in}{3.113465in}}%
\pgfpathlineto{\pgfqpoint{6.303641in}{3.124999in}}%
\pgfpathlineto{\pgfqpoint{6.313810in}{3.139406in}}%
\pgfpathlineto{\pgfqpoint{6.318894in}{3.143106in}}%
\pgfpathlineto{\pgfqpoint{6.322909in}{3.143590in}}%
\pgfpathlineto{\pgfqpoint{6.326923in}{3.141838in}}%
\pgfpathlineto{\pgfqpoint{6.331740in}{3.137104in}}%
\pgfpathlineto{\pgfqpoint{6.339233in}{3.126046in}}%
\pgfpathlineto{\pgfqpoint{6.349937in}{3.110863in}}%
\pgfpathlineto{\pgfqpoint{6.355022in}{3.107255in}}%
\pgfpathlineto{\pgfqpoint{6.359036in}{3.106854in}}%
\pgfpathlineto{\pgfqpoint{6.363050in}{3.108685in}}%
\pgfpathlineto{\pgfqpoint{6.367867in}{3.113497in}}%
\pgfpathlineto{\pgfqpoint{6.375627in}{3.125041in}}%
\pgfpathlineto{\pgfqpoint{6.385796in}{3.139433in}}%
\pgfpathlineto{\pgfqpoint{6.390881in}{3.143117in}}%
\pgfpathlineto{\pgfqpoint{6.394895in}{3.143585in}}%
\pgfpathlineto{\pgfqpoint{6.398909in}{3.141820in}}%
\pgfpathlineto{\pgfqpoint{6.403726in}{3.137071in}}%
\pgfpathlineto{\pgfqpoint{6.411219in}{3.126004in}}%
\pgfpathlineto{\pgfqpoint{6.421924in}{3.110836in}}%
\pgfpathlineto{\pgfqpoint{6.427008in}{3.107245in}}%
\pgfpathlineto{\pgfqpoint{6.431022in}{3.106859in}}%
\pgfpathlineto{\pgfqpoint{6.435036in}{3.108704in}}%
\pgfpathlineto{\pgfqpoint{6.439853in}{3.113530in}}%
\pgfpathlineto{\pgfqpoint{6.447614in}{3.125083in}}%
\pgfpathlineto{\pgfqpoint{6.457783in}{3.139460in}}%
\pgfpathlineto{\pgfqpoint{6.462867in}{3.143127in}}%
\pgfpathlineto{\pgfqpoint{6.466882in}{3.143581in}}%
\pgfpathlineto{\pgfqpoint{6.470896in}{3.141801in}}%
\pgfpathlineto{\pgfqpoint{6.475713in}{3.137039in}}%
\pgfpathlineto{\pgfqpoint{6.483206in}{3.125962in}}%
\pgfpathlineto{\pgfqpoint{6.493910in}{3.110810in}}%
\pgfpathlineto{\pgfqpoint{6.498995in}{3.107236in}}%
\pgfpathlineto{\pgfqpoint{6.503009in}{3.106864in}}%
\pgfpathlineto{\pgfqpoint{6.507023in}{3.108723in}}%
\pgfpathlineto{\pgfqpoint{6.511840in}{3.113563in}}%
\pgfpathlineto{\pgfqpoint{6.519600in}{3.125125in}}%
\pgfpathlineto{\pgfqpoint{6.529769in}{3.139486in}}%
\pgfpathlineto{\pgfqpoint{6.534854in}{3.143137in}}%
\pgfpathlineto{\pgfqpoint{6.538868in}{3.143577in}}%
\pgfpathlineto{\pgfqpoint{6.542882in}{3.141783in}}%
\pgfpathlineto{\pgfqpoint{6.547699in}{3.137007in}}%
\pgfpathlineto{\pgfqpoint{6.555192in}{3.125920in}}%
\pgfpathlineto{\pgfqpoint{6.565629in}{3.111057in}}%
\pgfpathlineto{\pgfqpoint{6.570713in}{3.107330in}}%
\pgfpathlineto{\pgfqpoint{6.574728in}{3.106823in}}%
\pgfpathlineto{\pgfqpoint{6.578742in}{3.108551in}}%
\pgfpathlineto{\pgfqpoint{6.583559in}{3.113263in}}%
\pgfpathlineto{\pgfqpoint{6.591052in}{3.124304in}}%
\pgfpathlineto{\pgfqpoint{6.601756in}{3.139513in}}%
\pgfpathlineto{\pgfqpoint{6.606841in}{3.143147in}}%
\pgfpathlineto{\pgfqpoint{6.610855in}{3.143572in}}%
\pgfpathlineto{\pgfqpoint{6.614869in}{3.141764in}}%
\pgfpathlineto{\pgfqpoint{6.619686in}{3.136974in}}%
\pgfpathlineto{\pgfqpoint{6.627179in}{3.125878in}}%
\pgfpathlineto{\pgfqpoint{6.637615in}{3.111030in}}%
\pgfpathlineto{\pgfqpoint{6.642700in}{3.107319in}}%
\pgfpathlineto{\pgfqpoint{6.646714in}{3.106827in}}%
\pgfpathlineto{\pgfqpoint{6.650728in}{3.108569in}}%
\pgfpathlineto{\pgfqpoint{6.655545in}{3.113295in}}%
\pgfpathlineto{\pgfqpoint{6.663038in}{3.124346in}}%
\pgfpathlineto{\pgfqpoint{6.663306in}{3.124778in}}%
\pgfpathlineto{\pgfqpoint{6.663306in}{3.124778in}}%
\pgfusepath{stroke}%
\end{pgfscope}%
\begin{pgfscope}%
\pgfpathrectangle{\pgfqpoint{0.467797in}{2.292089in}}{\pgfqpoint{6.490533in}{1.666241in}}%
\pgfusepath{clip}%
\pgfsetrectcap%
\pgfsetroundjoin%
\pgfsetlinewidth{1.505625pt}%
\definecolor{currentstroke}{rgb}{1.000000,0.498039,0.054902}%
\pgfsetstrokecolor{currentstroke}%
\pgfsetdash{}{0pt}%
\pgfpathmoveto{\pgfqpoint{0.762821in}{3.125209in}}%
\pgfpathlineto{\pgfqpoint{0.772722in}{3.139080in}}%
\pgfpathlineto{\pgfqpoint{0.777539in}{3.142387in}}%
\pgfpathlineto{\pgfqpoint{0.781553in}{3.142639in}}%
\pgfpathlineto{\pgfqpoint{0.785568in}{3.140563in}}%
\pgfpathlineto{\pgfqpoint{0.790652in}{3.135053in}}%
\pgfpathlineto{\pgfqpoint{0.800018in}{3.120601in}}%
\pgfpathlineto{\pgfqpoint{0.807511in}{3.110873in}}%
\pgfpathlineto{\pgfqpoint{0.812328in}{3.107877in}}%
\pgfpathlineto{\pgfqpoint{0.816342in}{3.107909in}}%
\pgfpathlineto{\pgfqpoint{0.820357in}{3.110252in}}%
\pgfpathlineto{\pgfqpoint{0.825441in}{3.116021in}}%
\pgfpathlineto{\pgfqpoint{0.845512in}{3.142129in}}%
\pgfpathlineto{\pgfqpoint{0.849526in}{3.142758in}}%
\pgfpathlineto{\pgfqpoint{0.853540in}{3.141043in}}%
\pgfpathlineto{\pgfqpoint{0.858357in}{3.136232in}}%
\pgfpathlineto{\pgfqpoint{0.866118in}{3.124602in}}%
\pgfpathlineto{\pgfqpoint{0.875484in}{3.111498in}}%
\pgfpathlineto{\pgfqpoint{0.880568in}{3.107994in}}%
\pgfpathlineto{\pgfqpoint{0.884582in}{3.107806in}}%
\pgfpathlineto{\pgfqpoint{0.888597in}{3.109943in}}%
\pgfpathlineto{\pgfqpoint{0.893681in}{3.115512in}}%
\pgfpathlineto{\pgfqpoint{0.903583in}{3.130812in}}%
\pgfpathlineto{\pgfqpoint{0.910808in}{3.139890in}}%
\pgfpathlineto{\pgfqpoint{0.915625in}{3.142640in}}%
\pgfpathlineto{\pgfqpoint{0.919639in}{3.142386in}}%
\pgfpathlineto{\pgfqpoint{0.923921in}{3.139592in}}%
\pgfpathlineto{\pgfqpoint{0.929273in}{3.133113in}}%
\pgfpathlineto{\pgfqpoint{0.947203in}{3.108947in}}%
\pgfpathlineto{\pgfqpoint{0.951484in}{3.107596in}}%
\pgfpathlineto{\pgfqpoint{0.955499in}{3.108759in}}%
\pgfpathlineto{\pgfqpoint{0.960048in}{3.112713in}}%
\pgfpathlineto{\pgfqpoint{0.966470in}{3.121700in}}%
\pgfpathlineto{\pgfqpoint{0.979048in}{3.139549in}}%
\pgfpathlineto{\pgfqpoint{0.983865in}{3.142543in}}%
\pgfpathlineto{\pgfqpoint{0.987879in}{3.142509in}}%
\pgfpathlineto{\pgfqpoint{0.991893in}{3.140164in}}%
\pgfpathlineto{\pgfqpoint{0.996978in}{3.134394in}}%
\pgfpathlineto{\pgfqpoint{1.017048in}{3.108289in}}%
\pgfpathlineto{\pgfqpoint{1.021062in}{3.107661in}}%
\pgfpathlineto{\pgfqpoint{1.025077in}{3.109378in}}%
\pgfpathlineto{\pgfqpoint{1.029894in}{3.114190in}}%
\pgfpathlineto{\pgfqpoint{1.037654in}{3.125822in}}%
\pgfpathlineto{\pgfqpoint{1.047020in}{3.138924in}}%
\pgfpathlineto{\pgfqpoint{1.052105in}{3.142426in}}%
\pgfpathlineto{\pgfqpoint{1.056119in}{3.142612in}}%
\pgfpathlineto{\pgfqpoint{1.060133in}{3.140473in}}%
\pgfpathlineto{\pgfqpoint{1.065218in}{3.134902in}}%
\pgfpathlineto{\pgfqpoint{1.075119in}{3.119602in}}%
\pgfpathlineto{\pgfqpoint{1.082345in}{3.110526in}}%
\pgfpathlineto{\pgfqpoint{1.087162in}{3.107778in}}%
\pgfpathlineto{\pgfqpoint{1.091176in}{3.108034in}}%
\pgfpathlineto{\pgfqpoint{1.095457in}{3.110830in}}%
\pgfpathlineto{\pgfqpoint{1.100810in}{3.117310in}}%
\pgfpathlineto{\pgfqpoint{1.118739in}{3.141474in}}%
\pgfpathlineto{\pgfqpoint{1.123021in}{3.142823in}}%
\pgfpathlineto{\pgfqpoint{1.127035in}{3.141658in}}%
\pgfpathlineto{\pgfqpoint{1.131584in}{3.137702in}}%
\pgfpathlineto{\pgfqpoint{1.138007in}{3.128714in}}%
\pgfpathlineto{\pgfqpoint{1.150585in}{3.110867in}}%
\pgfpathlineto{\pgfqpoint{1.155402in}{3.107875in}}%
\pgfpathlineto{\pgfqpoint{1.159416in}{3.107911in}}%
\pgfpathlineto{\pgfqpoint{1.163430in}{3.110258in}}%
\pgfpathlineto{\pgfqpoint{1.168514in}{3.116029in}}%
\pgfpathlineto{\pgfqpoint{1.188585in}{3.142132in}}%
\pgfpathlineto{\pgfqpoint{1.192599in}{3.142757in}}%
\pgfpathlineto{\pgfqpoint{1.196613in}{3.141038in}}%
\pgfpathlineto{\pgfqpoint{1.201430in}{3.136225in}}%
\pgfpathlineto{\pgfqpoint{1.209191in}{3.124592in}}%
\pgfpathlineto{\pgfqpoint{1.218557in}{3.111492in}}%
\pgfpathlineto{\pgfqpoint{1.223642in}{3.107992in}}%
\pgfpathlineto{\pgfqpoint{1.227656in}{3.107807in}}%
\pgfpathlineto{\pgfqpoint{1.231670in}{3.109948in}}%
\pgfpathlineto{\pgfqpoint{1.236754in}{3.115521in}}%
\pgfpathlineto{\pgfqpoint{1.246656in}{3.130821in}}%
\pgfpathlineto{\pgfqpoint{1.253881in}{3.139895in}}%
\pgfpathlineto{\pgfqpoint{1.258698in}{3.142641in}}%
\pgfpathlineto{\pgfqpoint{1.262712in}{3.142384in}}%
\pgfpathlineto{\pgfqpoint{1.266994in}{3.139586in}}%
\pgfpathlineto{\pgfqpoint{1.272346in}{3.133104in}}%
\pgfpathlineto{\pgfqpoint{1.290276in}{3.108943in}}%
\pgfpathlineto{\pgfqpoint{1.294558in}{3.107596in}}%
\pgfpathlineto{\pgfqpoint{1.298572in}{3.108763in}}%
\pgfpathlineto{\pgfqpoint{1.303121in}{3.112720in}}%
\pgfpathlineto{\pgfqpoint{1.309544in}{3.121710in}}%
\pgfpathlineto{\pgfqpoint{1.321854in}{3.139300in}}%
\pgfpathlineto{\pgfqpoint{1.326671in}{3.142463in}}%
\pgfpathlineto{\pgfqpoint{1.330685in}{3.142583in}}%
\pgfpathlineto{\pgfqpoint{1.334699in}{3.140382in}}%
\pgfpathlineto{\pgfqpoint{1.339783in}{3.134751in}}%
\pgfpathlineto{\pgfqpoint{1.350220in}{3.118623in}}%
\pgfpathlineto{\pgfqpoint{1.357178in}{3.110197in}}%
\pgfpathlineto{\pgfqpoint{1.361727in}{3.107753in}}%
\pgfpathlineto{\pgfqpoint{1.365741in}{3.108075in}}%
\pgfpathlineto{\pgfqpoint{1.370023in}{3.110935in}}%
\pgfpathlineto{\pgfqpoint{1.375643in}{3.117862in}}%
\pgfpathlineto{\pgfqpoint{1.392770in}{3.141200in}}%
\pgfpathlineto{\pgfqpoint{1.397051in}{3.142808in}}%
\pgfpathlineto{\pgfqpoint{1.401066in}{3.141890in}}%
\pgfpathlineto{\pgfqpoint{1.405347in}{3.138462in}}%
\pgfpathlineto{\pgfqpoint{1.411502in}{3.130214in}}%
\pgfpathlineto{\pgfqpoint{1.425418in}{3.110521in}}%
\pgfpathlineto{\pgfqpoint{1.430235in}{3.107777in}}%
\pgfpathlineto{\pgfqpoint{1.434249in}{3.108036in}}%
\pgfpathlineto{\pgfqpoint{1.438531in}{3.110835in}}%
\pgfpathlineto{\pgfqpoint{1.443883in}{3.117319in}}%
\pgfpathlineto{\pgfqpoint{1.461813in}{3.141478in}}%
\pgfpathlineto{\pgfqpoint{1.466094in}{3.142823in}}%
\pgfpathlineto{\pgfqpoint{1.470108in}{3.141654in}}%
\pgfpathlineto{\pgfqpoint{1.474658in}{3.137695in}}%
\pgfpathlineto{\pgfqpoint{1.481080in}{3.128704in}}%
\pgfpathlineto{\pgfqpoint{1.493390in}{3.111116in}}%
\pgfpathlineto{\pgfqpoint{1.498207in}{3.107955in}}%
\pgfpathlineto{\pgfqpoint{1.502221in}{3.107836in}}%
\pgfpathlineto{\pgfqpoint{1.506235in}{3.110039in}}%
\pgfpathlineto{\pgfqpoint{1.511320in}{3.115672in}}%
\pgfpathlineto{\pgfqpoint{1.521757in}{3.131801in}}%
\pgfpathlineto{\pgfqpoint{1.528714in}{3.140224in}}%
\pgfpathlineto{\pgfqpoint{1.533264in}{3.142666in}}%
\pgfpathlineto{\pgfqpoint{1.537278in}{3.142343in}}%
\pgfpathlineto{\pgfqpoint{1.541560in}{3.139481in}}%
\pgfpathlineto{\pgfqpoint{1.547179in}{3.132552in}}%
\pgfpathlineto{\pgfqpoint{1.564306in}{3.109217in}}%
\pgfpathlineto{\pgfqpoint{1.568588in}{3.107610in}}%
\pgfpathlineto{\pgfqpoint{1.572602in}{3.108531in}}%
\pgfpathlineto{\pgfqpoint{1.576884in}{3.111960in}}%
\pgfpathlineto{\pgfqpoint{1.583039in}{3.120210in}}%
\pgfpathlineto{\pgfqpoint{1.596954in}{3.139901in}}%
\pgfpathlineto{\pgfqpoint{1.601771in}{3.142643in}}%
\pgfpathlineto{\pgfqpoint{1.605786in}{3.142381in}}%
\pgfpathlineto{\pgfqpoint{1.610067in}{3.139581in}}%
\pgfpathlineto{\pgfqpoint{1.615419in}{3.133095in}}%
\pgfpathlineto{\pgfqpoint{1.633349in}{3.108939in}}%
\pgfpathlineto{\pgfqpoint{1.637631in}{3.107596in}}%
\pgfpathlineto{\pgfqpoint{1.641645in}{3.108766in}}%
\pgfpathlineto{\pgfqpoint{1.646194in}{3.112727in}}%
\pgfpathlineto{\pgfqpoint{1.652617in}{3.121719in}}%
\pgfpathlineto{\pgfqpoint{1.664927in}{3.139306in}}%
\pgfpathlineto{\pgfqpoint{1.669744in}{3.142465in}}%
\pgfpathlineto{\pgfqpoint{1.673758in}{3.142582in}}%
\pgfpathlineto{\pgfqpoint{1.677772in}{3.140377in}}%
\pgfpathlineto{\pgfqpoint{1.682857in}{3.134742in}}%
\pgfpathlineto{\pgfqpoint{1.693293in}{3.118614in}}%
\pgfpathlineto{\pgfqpoint{1.700251in}{3.110192in}}%
\pgfpathlineto{\pgfqpoint{1.704800in}{3.107752in}}%
\pgfpathlineto{\pgfqpoint{1.708815in}{3.108077in}}%
\pgfpathlineto{\pgfqpoint{1.713096in}{3.110941in}}%
\pgfpathlineto{\pgfqpoint{1.718716in}{3.117872in}}%
\pgfpathlineto{\pgfqpoint{1.735843in}{3.141204in}}%
\pgfpathlineto{\pgfqpoint{1.740125in}{3.142809in}}%
\pgfpathlineto{\pgfqpoint{1.744139in}{3.141887in}}%
\pgfpathlineto{\pgfqpoint{1.748421in}{3.138456in}}%
\pgfpathlineto{\pgfqpoint{1.754575in}{3.130204in}}%
\pgfpathlineto{\pgfqpoint{1.768491in}{3.110515in}}%
\pgfpathlineto{\pgfqpoint{1.773308in}{3.107775in}}%
\pgfpathlineto{\pgfqpoint{1.777322in}{3.108039in}}%
\pgfpathlineto{\pgfqpoint{1.781604in}{3.110841in}}%
\pgfpathlineto{\pgfqpoint{1.786956in}{3.117328in}}%
\pgfpathlineto{\pgfqpoint{1.804886in}{3.141482in}}%
\pgfpathlineto{\pgfqpoint{1.809167in}{3.142823in}}%
\pgfpathlineto{\pgfqpoint{1.813182in}{3.141651in}}%
\pgfpathlineto{\pgfqpoint{1.817731in}{3.137688in}}%
\pgfpathlineto{\pgfqpoint{1.824154in}{3.128695in}}%
\pgfpathlineto{\pgfqpoint{1.836463in}{3.111110in}}%
\pgfpathlineto{\pgfqpoint{1.841280in}{3.107953in}}%
\pgfpathlineto{\pgfqpoint{1.845295in}{3.107838in}}%
\pgfpathlineto{\pgfqpoint{1.849309in}{3.110044in}}%
\pgfpathlineto{\pgfqpoint{1.854393in}{3.115681in}}%
\pgfpathlineto{\pgfqpoint{1.864830in}{3.131810in}}%
\pgfpathlineto{\pgfqpoint{1.871788in}{3.140230in}}%
\pgfpathlineto{\pgfqpoint{1.876337in}{3.142668in}}%
\pgfpathlineto{\pgfqpoint{1.880351in}{3.142340in}}%
\pgfpathlineto{\pgfqpoint{1.884633in}{3.139475in}}%
\pgfpathlineto{\pgfqpoint{1.890253in}{3.132543in}}%
\pgfpathlineto{\pgfqpoint{1.907380in}{3.109213in}}%
\pgfpathlineto{\pgfqpoint{1.911661in}{3.107610in}}%
\pgfpathlineto{\pgfqpoint{1.915675in}{3.108534in}}%
\pgfpathlineto{\pgfqpoint{1.919957in}{3.111967in}}%
\pgfpathlineto{\pgfqpoint{1.926112in}{3.120220in}}%
\pgfpathlineto{\pgfqpoint{1.940028in}{3.139906in}}%
\pgfpathlineto{\pgfqpoint{1.944845in}{3.142644in}}%
\pgfpathlineto{\pgfqpoint{1.948859in}{3.142379in}}%
\pgfpathlineto{\pgfqpoint{1.953140in}{3.139575in}}%
\pgfpathlineto{\pgfqpoint{1.958493in}{3.133086in}}%
\pgfpathlineto{\pgfqpoint{1.976422in}{3.108935in}}%
\pgfpathlineto{\pgfqpoint{1.980704in}{3.107596in}}%
\pgfpathlineto{\pgfqpoint{1.984718in}{3.108770in}}%
\pgfpathlineto{\pgfqpoint{1.989268in}{3.112734in}}%
\pgfpathlineto{\pgfqpoint{1.995690in}{3.121729in}}%
\pgfpathlineto{\pgfqpoint{2.008000in}{3.139312in}}%
\pgfpathlineto{\pgfqpoint{2.012817in}{3.142467in}}%
\pgfpathlineto{\pgfqpoint{2.016831in}{3.142580in}}%
\pgfpathlineto{\pgfqpoint{2.020845in}{3.140372in}}%
\pgfpathlineto{\pgfqpoint{2.025930in}{3.134734in}}%
\pgfpathlineto{\pgfqpoint{2.036367in}{3.118604in}}%
\pgfpathlineto{\pgfqpoint{2.043324in}{3.110187in}}%
\pgfpathlineto{\pgfqpoint{2.047874in}{3.107751in}}%
\pgfpathlineto{\pgfqpoint{2.051888in}{3.108080in}}%
\pgfpathlineto{\pgfqpoint{2.056170in}{3.110947in}}%
\pgfpathlineto{\pgfqpoint{2.061789in}{3.117881in}}%
\pgfpathlineto{\pgfqpoint{2.078916in}{3.141208in}}%
\pgfpathlineto{\pgfqpoint{2.083198in}{3.142809in}}%
\pgfpathlineto{\pgfqpoint{2.087212in}{3.141883in}}%
\pgfpathlineto{\pgfqpoint{2.091494in}{3.138449in}}%
\pgfpathlineto{\pgfqpoint{2.097649in}{3.130195in}}%
\pgfpathlineto{\pgfqpoint{2.111564in}{3.110510in}}%
\pgfpathlineto{\pgfqpoint{2.116381in}{3.107774in}}%
\pgfpathlineto{\pgfqpoint{2.120395in}{3.108041in}}%
\pgfpathlineto{\pgfqpoint{2.124677in}{3.110847in}}%
\pgfpathlineto{\pgfqpoint{2.130029in}{3.117337in}}%
\pgfpathlineto{\pgfqpoint{2.147959in}{3.141485in}}%
\pgfpathlineto{\pgfqpoint{2.152241in}{3.142823in}}%
\pgfpathlineto{\pgfqpoint{2.156255in}{3.141647in}}%
\pgfpathlineto{\pgfqpoint{2.160804in}{3.137681in}}%
\pgfpathlineto{\pgfqpoint{2.167227in}{3.128685in}}%
\pgfpathlineto{\pgfqpoint{2.179537in}{3.111104in}}%
\pgfpathlineto{\pgfqpoint{2.184354in}{3.107951in}}%
\pgfpathlineto{\pgfqpoint{2.188368in}{3.107840in}}%
\pgfpathlineto{\pgfqpoint{2.192382in}{3.110049in}}%
\pgfpathlineto{\pgfqpoint{2.197466in}{3.115689in}}%
\pgfpathlineto{\pgfqpoint{2.207903in}{3.131819in}}%
\pgfpathlineto{\pgfqpoint{2.214861in}{3.140235in}}%
\pgfpathlineto{\pgfqpoint{2.219410in}{3.142669in}}%
\pgfpathlineto{\pgfqpoint{2.223424in}{3.142338in}}%
\pgfpathlineto{\pgfqpoint{2.227706in}{3.139469in}}%
\pgfpathlineto{\pgfqpoint{2.233326in}{3.132534in}}%
\pgfpathlineto{\pgfqpoint{2.250453in}{3.109209in}}%
\pgfpathlineto{\pgfqpoint{2.254735in}{3.107609in}}%
\pgfpathlineto{\pgfqpoint{2.258749in}{3.108537in}}%
\pgfpathlineto{\pgfqpoint{2.263030in}{3.111973in}}%
\pgfpathlineto{\pgfqpoint{2.269185in}{3.120229in}}%
\pgfpathlineto{\pgfqpoint{2.283101in}{3.139912in}}%
\pgfpathlineto{\pgfqpoint{2.287918in}{3.142646in}}%
\pgfpathlineto{\pgfqpoint{2.291932in}{3.142377in}}%
\pgfpathlineto{\pgfqpoint{2.296214in}{3.139569in}}%
\pgfpathlineto{\pgfqpoint{2.301566in}{3.133077in}}%
\pgfpathlineto{\pgfqpoint{2.319496in}{3.108932in}}%
\pgfpathlineto{\pgfqpoint{2.323777in}{3.107596in}}%
\pgfpathlineto{\pgfqpoint{2.327791in}{3.108773in}}%
\pgfpathlineto{\pgfqpoint{2.332341in}{3.112741in}}%
\pgfpathlineto{\pgfqpoint{2.338763in}{3.121739in}}%
\pgfpathlineto{\pgfqpoint{2.351073in}{3.139318in}}%
\pgfpathlineto{\pgfqpoint{2.355890in}{3.142469in}}%
\pgfpathlineto{\pgfqpoint{2.359904in}{3.142578in}}%
\pgfpathlineto{\pgfqpoint{2.363919in}{3.140367in}}%
\pgfpathlineto{\pgfqpoint{2.369003in}{3.134726in}}%
\pgfpathlineto{\pgfqpoint{2.379440in}{3.118595in}}%
\pgfpathlineto{\pgfqpoint{2.386398in}{3.110181in}}%
\pgfpathlineto{\pgfqpoint{2.390947in}{3.107749in}}%
\pgfpathlineto{\pgfqpoint{2.394961in}{3.108082in}}%
\pgfpathlineto{\pgfqpoint{2.399243in}{3.110952in}}%
\pgfpathlineto{\pgfqpoint{2.404862in}{3.117890in}}%
\pgfpathlineto{\pgfqpoint{2.421989in}{3.141212in}}%
\pgfpathlineto{\pgfqpoint{2.426271in}{3.142810in}}%
\pgfpathlineto{\pgfqpoint{2.430285in}{3.141880in}}%
\pgfpathlineto{\pgfqpoint{2.434567in}{3.138442in}}%
\pgfpathlineto{\pgfqpoint{2.440722in}{3.130185in}}%
\pgfpathlineto{\pgfqpoint{2.454638in}{3.110504in}}%
\pgfpathlineto{\pgfqpoint{2.459454in}{3.107773in}}%
\pgfpathlineto{\pgfqpoint{2.463469in}{3.108043in}}%
\pgfpathlineto{\pgfqpoint{2.467750in}{3.110853in}}%
\pgfpathlineto{\pgfqpoint{2.473102in}{3.117346in}}%
\pgfpathlineto{\pgfqpoint{2.491032in}{3.141489in}}%
\pgfpathlineto{\pgfqpoint{2.495314in}{3.142823in}}%
\pgfpathlineto{\pgfqpoint{2.499328in}{3.141644in}}%
\pgfpathlineto{\pgfqpoint{2.503877in}{3.137674in}}%
\pgfpathlineto{\pgfqpoint{2.510300in}{3.128675in}}%
\pgfpathlineto{\pgfqpoint{2.522610in}{3.111098in}}%
\pgfpathlineto{\pgfqpoint{2.527427in}{3.107949in}}%
\pgfpathlineto{\pgfqpoint{2.531441in}{3.107841in}}%
\pgfpathlineto{\pgfqpoint{2.535455in}{3.110054in}}%
\pgfpathlineto{\pgfqpoint{2.540540in}{3.115698in}}%
\pgfpathlineto{\pgfqpoint{2.550976in}{3.131829in}}%
\pgfpathlineto{\pgfqpoint{2.557934in}{3.140240in}}%
\pgfpathlineto{\pgfqpoint{2.562484in}{3.142670in}}%
\pgfpathlineto{\pgfqpoint{2.566498in}{3.142336in}}%
\pgfpathlineto{\pgfqpoint{2.570779in}{3.139464in}}%
\pgfpathlineto{\pgfqpoint{2.576399in}{3.132524in}}%
\pgfpathlineto{\pgfqpoint{2.593526in}{3.109205in}}%
\pgfpathlineto{\pgfqpoint{2.597808in}{3.107609in}}%
\pgfpathlineto{\pgfqpoint{2.601822in}{3.108540in}}%
\pgfpathlineto{\pgfqpoint{2.606104in}{3.111980in}}%
\pgfpathlineto{\pgfqpoint{2.612259in}{3.120239in}}%
\pgfpathlineto{\pgfqpoint{2.626174in}{3.139918in}}%
\pgfpathlineto{\pgfqpoint{2.630991in}{3.142647in}}%
\pgfpathlineto{\pgfqpoint{2.635005in}{3.142375in}}%
\pgfpathlineto{\pgfqpoint{2.639287in}{3.139563in}}%
\pgfpathlineto{\pgfqpoint{2.644639in}{3.133068in}}%
\pgfpathlineto{\pgfqpoint{2.662569in}{3.108928in}}%
\pgfpathlineto{\pgfqpoint{2.666851in}{3.107596in}}%
\pgfpathlineto{\pgfqpoint{2.670865in}{3.108777in}}%
\pgfpathlineto{\pgfqpoint{2.675414in}{3.112749in}}%
\pgfpathlineto{\pgfqpoint{2.681837in}{3.121749in}}%
\pgfpathlineto{\pgfqpoint{2.694147in}{3.139324in}}%
\pgfpathlineto{\pgfqpoint{2.698963in}{3.142471in}}%
\pgfpathlineto{\pgfqpoint{2.702978in}{3.142577in}}%
\pgfpathlineto{\pgfqpoint{2.706992in}{3.140362in}}%
\pgfpathlineto{\pgfqpoint{2.712076in}{3.134717in}}%
\pgfpathlineto{\pgfqpoint{2.722781in}{3.118188in}}%
\pgfpathlineto{\pgfqpoint{2.729471in}{3.110176in}}%
\pgfpathlineto{\pgfqpoint{2.734020in}{3.107748in}}%
\pgfpathlineto{\pgfqpoint{2.738034in}{3.108084in}}%
\pgfpathlineto{\pgfqpoint{2.742316in}{3.110958in}}%
\pgfpathlineto{\pgfqpoint{2.747936in}{3.117899in}}%
\pgfpathlineto{\pgfqpoint{2.765063in}{3.141216in}}%
\pgfpathlineto{\pgfqpoint{2.769344in}{3.142810in}}%
\pgfpathlineto{\pgfqpoint{2.773358in}{3.141877in}}%
\pgfpathlineto{\pgfqpoint{2.777640in}{3.138436in}}%
\pgfpathlineto{\pgfqpoint{2.783795in}{3.130175in}}%
\pgfpathlineto{\pgfqpoint{2.797711in}{3.110499in}}%
\pgfpathlineto{\pgfqpoint{2.802528in}{3.107771in}}%
\pgfpathlineto{\pgfqpoint{2.806542in}{3.108045in}}%
\pgfpathlineto{\pgfqpoint{2.810824in}{3.110859in}}%
\pgfpathlineto{\pgfqpoint{2.816176in}{3.117355in}}%
\pgfpathlineto{\pgfqpoint{2.834105in}{3.141493in}}%
\pgfpathlineto{\pgfqpoint{2.838387in}{3.142823in}}%
\pgfpathlineto{\pgfqpoint{2.842401in}{3.141640in}}%
\pgfpathlineto{\pgfqpoint{2.846951in}{3.137667in}}%
\pgfpathlineto{\pgfqpoint{2.853373in}{3.128665in}}%
\pgfpathlineto{\pgfqpoint{2.865683in}{3.111092in}}%
\pgfpathlineto{\pgfqpoint{2.870500in}{3.107947in}}%
\pgfpathlineto{\pgfqpoint{2.874514in}{3.107843in}}%
\pgfpathlineto{\pgfqpoint{2.878528in}{3.110060in}}%
\pgfpathlineto{\pgfqpoint{2.883613in}{3.115706in}}%
\pgfpathlineto{\pgfqpoint{2.894317in}{3.132236in}}%
\pgfpathlineto{\pgfqpoint{2.901007in}{3.140245in}}%
\pgfpathlineto{\pgfqpoint{2.905557in}{3.142672in}}%
\pgfpathlineto{\pgfqpoint{2.909571in}{3.142333in}}%
\pgfpathlineto{\pgfqpoint{2.913853in}{3.139458in}}%
\pgfpathlineto{\pgfqpoint{2.919472in}{3.132515in}}%
\pgfpathlineto{\pgfqpoint{2.936599in}{3.109200in}}%
\pgfpathlineto{\pgfqpoint{2.940881in}{3.107609in}}%
\pgfpathlineto{\pgfqpoint{2.944895in}{3.108544in}}%
\pgfpathlineto{\pgfqpoint{2.949177in}{3.111987in}}%
\pgfpathlineto{\pgfqpoint{2.955332in}{3.120248in}}%
\pgfpathlineto{\pgfqpoint{2.969247in}{3.139923in}}%
\pgfpathlineto{\pgfqpoint{2.974064in}{3.142648in}}%
\pgfpathlineto{\pgfqpoint{2.978078in}{3.142372in}}%
\pgfpathlineto{\pgfqpoint{2.982360in}{3.139557in}}%
\pgfpathlineto{\pgfqpoint{2.987712in}{3.133059in}}%
\pgfpathlineto{\pgfqpoint{3.005642in}{3.108924in}}%
\pgfpathlineto{\pgfqpoint{3.009924in}{3.107596in}}%
\pgfpathlineto{\pgfqpoint{3.013938in}{3.108781in}}%
\pgfpathlineto{\pgfqpoint{3.018487in}{3.112756in}}%
\pgfpathlineto{\pgfqpoint{3.024910in}{3.121759in}}%
\pgfpathlineto{\pgfqpoint{3.037220in}{3.139330in}}%
\pgfpathlineto{\pgfqpoint{3.042037in}{3.142473in}}%
\pgfpathlineto{\pgfqpoint{3.046051in}{3.142575in}}%
\pgfpathlineto{\pgfqpoint{3.050065in}{3.140357in}}%
\pgfpathlineto{\pgfqpoint{3.055150in}{3.134709in}}%
\pgfpathlineto{\pgfqpoint{3.065854in}{3.118179in}}%
\pgfpathlineto{\pgfqpoint{3.072544in}{3.110171in}}%
\pgfpathlineto{\pgfqpoint{3.077093in}{3.107747in}}%
\pgfpathlineto{\pgfqpoint{3.081107in}{3.108087in}}%
\pgfpathlineto{\pgfqpoint{3.085389in}{3.110964in}}%
\pgfpathlineto{\pgfqpoint{3.091009in}{3.117908in}}%
\pgfpathlineto{\pgfqpoint{3.108136in}{3.141221in}}%
\pgfpathlineto{\pgfqpoint{3.112418in}{3.142810in}}%
\pgfpathlineto{\pgfqpoint{3.116432in}{3.141874in}}%
\pgfpathlineto{\pgfqpoint{3.120713in}{3.138429in}}%
\pgfpathlineto{\pgfqpoint{3.126868in}{3.130166in}}%
\pgfpathlineto{\pgfqpoint{3.140784in}{3.110493in}}%
\pgfpathlineto{\pgfqpoint{3.145601in}{3.107770in}}%
\pgfpathlineto{\pgfqpoint{3.149615in}{3.108048in}}%
\pgfpathlineto{\pgfqpoint{3.153897in}{3.110865in}}%
\pgfpathlineto{\pgfqpoint{3.159249in}{3.117364in}}%
\pgfpathlineto{\pgfqpoint{3.177179in}{3.141497in}}%
\pgfpathlineto{\pgfqpoint{3.181460in}{3.142823in}}%
\pgfpathlineto{\pgfqpoint{3.185475in}{3.141636in}}%
\pgfpathlineto{\pgfqpoint{3.190024in}{3.137660in}}%
\pgfpathlineto{\pgfqpoint{3.196446in}{3.128655in}}%
\pgfpathlineto{\pgfqpoint{3.208756in}{3.111086in}}%
\pgfpathlineto{\pgfqpoint{3.213573in}{3.107945in}}%
\pgfpathlineto{\pgfqpoint{3.217587in}{3.107845in}}%
\pgfpathlineto{\pgfqpoint{3.221602in}{3.110065in}}%
\pgfpathlineto{\pgfqpoint{3.226686in}{3.115714in}}%
\pgfpathlineto{\pgfqpoint{3.237390in}{3.132245in}}%
\pgfpathlineto{\pgfqpoint{3.244081in}{3.140251in}}%
\pgfpathlineto{\pgfqpoint{3.248630in}{3.142673in}}%
\pgfpathlineto{\pgfqpoint{3.252644in}{3.142331in}}%
\pgfpathlineto{\pgfqpoint{3.256926in}{3.139452in}}%
\pgfpathlineto{\pgfqpoint{3.262546in}{3.132506in}}%
\pgfpathlineto{\pgfqpoint{3.279672in}{3.109196in}}%
\pgfpathlineto{\pgfqpoint{3.283954in}{3.107608in}}%
\pgfpathlineto{\pgfqpoint{3.287968in}{3.108547in}}%
\pgfpathlineto{\pgfqpoint{3.292250in}{3.111993in}}%
\pgfpathlineto{\pgfqpoint{3.298405in}{3.120258in}}%
\pgfpathlineto{\pgfqpoint{3.312321in}{3.139929in}}%
\pgfpathlineto{\pgfqpoint{3.317138in}{3.142650in}}%
\pgfpathlineto{\pgfqpoint{3.321152in}{3.142370in}}%
\pgfpathlineto{\pgfqpoint{3.325433in}{3.139551in}}%
\pgfpathlineto{\pgfqpoint{3.330786in}{3.133050in}}%
\pgfpathlineto{\pgfqpoint{3.348715in}{3.108920in}}%
\pgfpathlineto{\pgfqpoint{3.352997in}{3.107596in}}%
\pgfpathlineto{\pgfqpoint{3.357011in}{3.108784in}}%
\pgfpathlineto{\pgfqpoint{3.361560in}{3.112763in}}%
\pgfpathlineto{\pgfqpoint{3.367983in}{3.121769in}}%
\pgfpathlineto{\pgfqpoint{3.380293in}{3.139336in}}%
\pgfpathlineto{\pgfqpoint{3.385110in}{3.142475in}}%
\pgfpathlineto{\pgfqpoint{3.389124in}{3.142573in}}%
\pgfpathlineto{\pgfqpoint{3.393138in}{3.140352in}}%
\pgfpathlineto{\pgfqpoint{3.398223in}{3.134700in}}%
\pgfpathlineto{\pgfqpoint{3.408927in}{3.118169in}}%
\pgfpathlineto{\pgfqpoint{3.415617in}{3.110166in}}%
\pgfpathlineto{\pgfqpoint{3.420167in}{3.107745in}}%
\pgfpathlineto{\pgfqpoint{3.424181in}{3.108089in}}%
\pgfpathlineto{\pgfqpoint{3.428462in}{3.110970in}}%
\pgfpathlineto{\pgfqpoint{3.434082in}{3.117917in}}%
\pgfpathlineto{\pgfqpoint{3.451209in}{3.141225in}}%
\pgfpathlineto{\pgfqpoint{3.455491in}{3.142811in}}%
\pgfpathlineto{\pgfqpoint{3.459505in}{3.141870in}}%
\pgfpathlineto{\pgfqpoint{3.463787in}{3.138422in}}%
\pgfpathlineto{\pgfqpoint{3.469942in}{3.130156in}}%
\pgfpathlineto{\pgfqpoint{3.483857in}{3.110488in}}%
\pgfpathlineto{\pgfqpoint{3.488674in}{3.107768in}}%
\pgfpathlineto{\pgfqpoint{3.492688in}{3.108050in}}%
\pgfpathlineto{\pgfqpoint{3.496970in}{3.110870in}}%
\pgfpathlineto{\pgfqpoint{3.502322in}{3.117373in}}%
\pgfpathlineto{\pgfqpoint{3.520252in}{3.141501in}}%
\pgfpathlineto{\pgfqpoint{3.524534in}{3.142823in}}%
\pgfpathlineto{\pgfqpoint{3.528548in}{3.141633in}}%
\pgfpathlineto{\pgfqpoint{3.533097in}{3.137653in}}%
\pgfpathlineto{\pgfqpoint{3.539520in}{3.128645in}}%
\pgfpathlineto{\pgfqpoint{3.551830in}{3.111080in}}%
\pgfpathlineto{\pgfqpoint{3.556647in}{3.107943in}}%
\pgfpathlineto{\pgfqpoint{3.560661in}{3.107846in}}%
\pgfpathlineto{\pgfqpoint{3.564675in}{3.110070in}}%
\pgfpathlineto{\pgfqpoint{3.569759in}{3.115723in}}%
\pgfpathlineto{\pgfqpoint{3.580464in}{3.132254in}}%
\pgfpathlineto{\pgfqpoint{3.587154in}{3.140256in}}%
\pgfpathlineto{\pgfqpoint{3.591703in}{3.142674in}}%
\pgfpathlineto{\pgfqpoint{3.595717in}{3.142329in}}%
\pgfpathlineto{\pgfqpoint{3.599999in}{3.139446in}}%
\pgfpathlineto{\pgfqpoint{3.605619in}{3.132497in}}%
\pgfpathlineto{\pgfqpoint{3.622478in}{3.109376in}}%
\pgfpathlineto{\pgfqpoint{3.626760in}{3.107629in}}%
\pgfpathlineto{\pgfqpoint{3.630774in}{3.108415in}}%
\pgfpathlineto{\pgfqpoint{3.635056in}{3.111718in}}%
\pgfpathlineto{\pgfqpoint{3.640943in}{3.119445in}}%
\pgfpathlineto{\pgfqpoint{3.655929in}{3.140390in}}%
\pgfpathlineto{\pgfqpoint{3.660478in}{3.142706in}}%
\pgfpathlineto{\pgfqpoint{3.664493in}{3.142265in}}%
\pgfpathlineto{\pgfqpoint{3.668774in}{3.139291in}}%
\pgfpathlineto{\pgfqpoint{3.674394in}{3.132259in}}%
\pgfpathlineto{\pgfqpoint{3.690986in}{3.109452in}}%
\pgfpathlineto{\pgfqpoint{3.695267in}{3.107640in}}%
\pgfpathlineto{\pgfqpoint{3.699282in}{3.108364in}}%
\pgfpathlineto{\pgfqpoint{3.703563in}{3.111609in}}%
\pgfpathlineto{\pgfqpoint{3.709451in}{3.119284in}}%
\pgfpathlineto{\pgfqpoint{3.724437in}{3.140303in}}%
\pgfpathlineto{\pgfqpoint{3.728986in}{3.142686in}}%
\pgfpathlineto{\pgfqpoint{3.733000in}{3.142307in}}%
\pgfpathlineto{\pgfqpoint{3.737282in}{3.139392in}}%
\pgfpathlineto{\pgfqpoint{3.742902in}{3.132415in}}%
\pgfpathlineto{\pgfqpoint{3.759761in}{3.109337in}}%
\pgfpathlineto{\pgfqpoint{3.764043in}{3.107624in}}%
\pgfpathlineto{\pgfqpoint{3.768057in}{3.108443in}}%
\pgfpathlineto{\pgfqpoint{3.772338in}{3.111777in}}%
\pgfpathlineto{\pgfqpoint{3.778226in}{3.119531in}}%
\pgfpathlineto{\pgfqpoint{3.792944in}{3.140214in}}%
\pgfpathlineto{\pgfqpoint{3.797494in}{3.142664in}}%
\pgfpathlineto{\pgfqpoint{3.801508in}{3.142347in}}%
\pgfpathlineto{\pgfqpoint{3.805789in}{3.139493in}}%
\pgfpathlineto{\pgfqpoint{3.811409in}{3.132570in}}%
\pgfpathlineto{\pgfqpoint{3.828536in}{3.109226in}}%
\pgfpathlineto{\pgfqpoint{3.832818in}{3.107611in}}%
\pgfpathlineto{\pgfqpoint{3.836832in}{3.108524in}}%
\pgfpathlineto{\pgfqpoint{3.841114in}{3.111947in}}%
\pgfpathlineto{\pgfqpoint{3.847269in}{3.120191in}}%
\pgfpathlineto{\pgfqpoint{3.861184in}{3.139890in}}%
\pgfpathlineto{\pgfqpoint{3.866001in}{3.142640in}}%
\pgfpathlineto{\pgfqpoint{3.870015in}{3.142386in}}%
\pgfpathlineto{\pgfqpoint{3.874297in}{3.139592in}}%
\pgfpathlineto{\pgfqpoint{3.879649in}{3.133113in}}%
\pgfpathlineto{\pgfqpoint{3.897579in}{3.108947in}}%
\pgfpathlineto{\pgfqpoint{3.901861in}{3.107596in}}%
\pgfpathlineto{\pgfqpoint{3.905875in}{3.108759in}}%
\pgfpathlineto{\pgfqpoint{3.910424in}{3.112713in}}%
\pgfpathlineto{\pgfqpoint{3.916847in}{3.121700in}}%
\pgfpathlineto{\pgfqpoint{3.929424in}{3.139549in}}%
\pgfpathlineto{\pgfqpoint{3.934241in}{3.142543in}}%
\pgfpathlineto{\pgfqpoint{3.938255in}{3.142509in}}%
\pgfpathlineto{\pgfqpoint{3.942269in}{3.140164in}}%
\pgfpathlineto{\pgfqpoint{3.947354in}{3.134394in}}%
\pgfpathlineto{\pgfqpoint{3.967425in}{3.108289in}}%
\pgfpathlineto{\pgfqpoint{3.971439in}{3.107661in}}%
\pgfpathlineto{\pgfqpoint{3.975453in}{3.109378in}}%
\pgfpathlineto{\pgfqpoint{3.980270in}{3.114190in}}%
\pgfpathlineto{\pgfqpoint{3.988030in}{3.125822in}}%
\pgfpathlineto{\pgfqpoint{3.997397in}{3.138924in}}%
\pgfpathlineto{\pgfqpoint{4.002481in}{3.142426in}}%
\pgfpathlineto{\pgfqpoint{4.006495in}{3.142612in}}%
\pgfpathlineto{\pgfqpoint{4.010509in}{3.140473in}}%
\pgfpathlineto{\pgfqpoint{4.015594in}{3.134902in}}%
\pgfpathlineto{\pgfqpoint{4.025495in}{3.119602in}}%
\pgfpathlineto{\pgfqpoint{4.032721in}{3.110526in}}%
\pgfpathlineto{\pgfqpoint{4.037538in}{3.107778in}}%
\pgfpathlineto{\pgfqpoint{4.041552in}{3.108034in}}%
\pgfpathlineto{\pgfqpoint{4.045834in}{3.110830in}}%
\pgfpathlineto{\pgfqpoint{4.051186in}{3.117310in}}%
\pgfpathlineto{\pgfqpoint{4.069116in}{3.141474in}}%
\pgfpathlineto{\pgfqpoint{4.073397in}{3.142823in}}%
\pgfpathlineto{\pgfqpoint{4.077411in}{3.141658in}}%
\pgfpathlineto{\pgfqpoint{4.081961in}{3.137702in}}%
\pgfpathlineto{\pgfqpoint{4.088383in}{3.128714in}}%
\pgfpathlineto{\pgfqpoint{4.100961in}{3.110867in}}%
\pgfpathlineto{\pgfqpoint{4.105778in}{3.107875in}}%
\pgfpathlineto{\pgfqpoint{4.109792in}{3.107911in}}%
\pgfpathlineto{\pgfqpoint{4.113806in}{3.110258in}}%
\pgfpathlineto{\pgfqpoint{4.118891in}{3.116029in}}%
\pgfpathlineto{\pgfqpoint{4.138961in}{3.142132in}}%
\pgfpathlineto{\pgfqpoint{4.142975in}{3.142757in}}%
\pgfpathlineto{\pgfqpoint{4.146989in}{3.141038in}}%
\pgfpathlineto{\pgfqpoint{4.151806in}{3.136225in}}%
\pgfpathlineto{\pgfqpoint{4.159567in}{3.124592in}}%
\pgfpathlineto{\pgfqpoint{4.168933in}{3.111492in}}%
\pgfpathlineto{\pgfqpoint{4.174018in}{3.107992in}}%
\pgfpathlineto{\pgfqpoint{4.178032in}{3.107807in}}%
\pgfpathlineto{\pgfqpoint{4.182046in}{3.109948in}}%
\pgfpathlineto{\pgfqpoint{4.187131in}{3.115521in}}%
\pgfpathlineto{\pgfqpoint{4.197032in}{3.130821in}}%
\pgfpathlineto{\pgfqpoint{4.204257in}{3.139895in}}%
\pgfpathlineto{\pgfqpoint{4.209074in}{3.142641in}}%
\pgfpathlineto{\pgfqpoint{4.213089in}{3.142384in}}%
\pgfpathlineto{\pgfqpoint{4.217370in}{3.139586in}}%
\pgfpathlineto{\pgfqpoint{4.222722in}{3.133104in}}%
\pgfpathlineto{\pgfqpoint{4.240652in}{3.108943in}}%
\pgfpathlineto{\pgfqpoint{4.244934in}{3.107596in}}%
\pgfpathlineto{\pgfqpoint{4.248948in}{3.108763in}}%
\pgfpathlineto{\pgfqpoint{4.253497in}{3.112720in}}%
\pgfpathlineto{\pgfqpoint{4.259920in}{3.121710in}}%
\pgfpathlineto{\pgfqpoint{4.272230in}{3.139300in}}%
\pgfpathlineto{\pgfqpoint{4.277047in}{3.142463in}}%
\pgfpathlineto{\pgfqpoint{4.281061in}{3.142583in}}%
\pgfpathlineto{\pgfqpoint{4.285075in}{3.140382in}}%
\pgfpathlineto{\pgfqpoint{4.290160in}{3.134751in}}%
\pgfpathlineto{\pgfqpoint{4.300596in}{3.118623in}}%
\pgfpathlineto{\pgfqpoint{4.307554in}{3.110197in}}%
\pgfpathlineto{\pgfqpoint{4.312103in}{3.107753in}}%
\pgfpathlineto{\pgfqpoint{4.316118in}{3.108075in}}%
\pgfpathlineto{\pgfqpoint{4.320399in}{3.110935in}}%
\pgfpathlineto{\pgfqpoint{4.326019in}{3.117862in}}%
\pgfpathlineto{\pgfqpoint{4.343146in}{3.141200in}}%
\pgfpathlineto{\pgfqpoint{4.347428in}{3.142808in}}%
\pgfpathlineto{\pgfqpoint{4.351442in}{3.141890in}}%
\pgfpathlineto{\pgfqpoint{4.355724in}{3.138462in}}%
\pgfpathlineto{\pgfqpoint{4.361878in}{3.130214in}}%
\pgfpathlineto{\pgfqpoint{4.375794in}{3.110521in}}%
\pgfpathlineto{\pgfqpoint{4.380611in}{3.107777in}}%
\pgfpathlineto{\pgfqpoint{4.384625in}{3.108036in}}%
\pgfpathlineto{\pgfqpoint{4.388907in}{3.110835in}}%
\pgfpathlineto{\pgfqpoint{4.394259in}{3.117319in}}%
\pgfpathlineto{\pgfqpoint{4.412189in}{3.141478in}}%
\pgfpathlineto{\pgfqpoint{4.416470in}{3.142823in}}%
\pgfpathlineto{\pgfqpoint{4.420485in}{3.141654in}}%
\pgfpathlineto{\pgfqpoint{4.425034in}{3.137695in}}%
\pgfpathlineto{\pgfqpoint{4.431457in}{3.128704in}}%
\pgfpathlineto{\pgfqpoint{4.443766in}{3.111116in}}%
\pgfpathlineto{\pgfqpoint{4.448583in}{3.107955in}}%
\pgfpathlineto{\pgfqpoint{4.452598in}{3.107836in}}%
\pgfpathlineto{\pgfqpoint{4.456612in}{3.110039in}}%
\pgfpathlineto{\pgfqpoint{4.461696in}{3.115672in}}%
\pgfpathlineto{\pgfqpoint{4.472133in}{3.131801in}}%
\pgfpathlineto{\pgfqpoint{4.479091in}{3.140224in}}%
\pgfpathlineto{\pgfqpoint{4.483640in}{3.142666in}}%
\pgfpathlineto{\pgfqpoint{4.487654in}{3.142343in}}%
\pgfpathlineto{\pgfqpoint{4.491936in}{3.139481in}}%
\pgfpathlineto{\pgfqpoint{4.497556in}{3.132552in}}%
\pgfpathlineto{\pgfqpoint{4.514683in}{3.109217in}}%
\pgfpathlineto{\pgfqpoint{4.518964in}{3.107610in}}%
\pgfpathlineto{\pgfqpoint{4.522978in}{3.108531in}}%
\pgfpathlineto{\pgfqpoint{4.527260in}{3.111960in}}%
\pgfpathlineto{\pgfqpoint{4.533415in}{3.120210in}}%
\pgfpathlineto{\pgfqpoint{4.547331in}{3.139901in}}%
\pgfpathlineto{\pgfqpoint{4.552148in}{3.142643in}}%
\pgfpathlineto{\pgfqpoint{4.556162in}{3.142381in}}%
\pgfpathlineto{\pgfqpoint{4.560443in}{3.139581in}}%
\pgfpathlineto{\pgfqpoint{4.565796in}{3.133095in}}%
\pgfpathlineto{\pgfqpoint{4.583725in}{3.108939in}}%
\pgfpathlineto{\pgfqpoint{4.588007in}{3.107596in}}%
\pgfpathlineto{\pgfqpoint{4.592021in}{3.108766in}}%
\pgfpathlineto{\pgfqpoint{4.596571in}{3.112727in}}%
\pgfpathlineto{\pgfqpoint{4.602993in}{3.121719in}}%
\pgfpathlineto{\pgfqpoint{4.615303in}{3.139306in}}%
\pgfpathlineto{\pgfqpoint{4.620120in}{3.142465in}}%
\pgfpathlineto{\pgfqpoint{4.624134in}{3.142582in}}%
\pgfpathlineto{\pgfqpoint{4.628148in}{3.140377in}}%
\pgfpathlineto{\pgfqpoint{4.633233in}{3.134742in}}%
\pgfpathlineto{\pgfqpoint{4.643670in}{3.118614in}}%
\pgfpathlineto{\pgfqpoint{4.650627in}{3.110192in}}%
\pgfpathlineto{\pgfqpoint{4.655177in}{3.107752in}}%
\pgfpathlineto{\pgfqpoint{4.659191in}{3.108077in}}%
\pgfpathlineto{\pgfqpoint{4.663473in}{3.110941in}}%
\pgfpathlineto{\pgfqpoint{4.669092in}{3.117872in}}%
\pgfpathlineto{\pgfqpoint{4.686219in}{3.141204in}}%
\pgfpathlineto{\pgfqpoint{4.690501in}{3.142809in}}%
\pgfpathlineto{\pgfqpoint{4.694515in}{3.141887in}}%
\pgfpathlineto{\pgfqpoint{4.698797in}{3.138456in}}%
\pgfpathlineto{\pgfqpoint{4.704952in}{3.130204in}}%
\pgfpathlineto{\pgfqpoint{4.718867in}{3.110515in}}%
\pgfpathlineto{\pgfqpoint{4.723684in}{3.107775in}}%
\pgfpathlineto{\pgfqpoint{4.727698in}{3.108039in}}%
\pgfpathlineto{\pgfqpoint{4.731980in}{3.110841in}}%
\pgfpathlineto{\pgfqpoint{4.737332in}{3.117328in}}%
\pgfpathlineto{\pgfqpoint{4.755262in}{3.141482in}}%
\pgfpathlineto{\pgfqpoint{4.759544in}{3.142823in}}%
\pgfpathlineto{\pgfqpoint{4.763558in}{3.141651in}}%
\pgfpathlineto{\pgfqpoint{4.768107in}{3.137688in}}%
\pgfpathlineto{\pgfqpoint{4.774530in}{3.128695in}}%
\pgfpathlineto{\pgfqpoint{4.786840in}{3.111110in}}%
\pgfpathlineto{\pgfqpoint{4.791657in}{3.107953in}}%
\pgfpathlineto{\pgfqpoint{4.795671in}{3.107838in}}%
\pgfpathlineto{\pgfqpoint{4.799685in}{3.110044in}}%
\pgfpathlineto{\pgfqpoint{4.804769in}{3.115681in}}%
\pgfpathlineto{\pgfqpoint{4.815206in}{3.131810in}}%
\pgfpathlineto{\pgfqpoint{4.822164in}{3.140230in}}%
\pgfpathlineto{\pgfqpoint{4.826713in}{3.142668in}}%
\pgfpathlineto{\pgfqpoint{4.830727in}{3.142340in}}%
\pgfpathlineto{\pgfqpoint{4.835009in}{3.139475in}}%
\pgfpathlineto{\pgfqpoint{4.840629in}{3.132543in}}%
\pgfpathlineto{\pgfqpoint{4.857756in}{3.109213in}}%
\pgfpathlineto{\pgfqpoint{4.862038in}{3.107610in}}%
\pgfpathlineto{\pgfqpoint{4.866052in}{3.108534in}}%
\pgfpathlineto{\pgfqpoint{4.870333in}{3.111967in}}%
\pgfpathlineto{\pgfqpoint{4.876488in}{3.120220in}}%
\pgfpathlineto{\pgfqpoint{4.890404in}{3.139906in}}%
\pgfpathlineto{\pgfqpoint{4.895221in}{3.142644in}}%
\pgfpathlineto{\pgfqpoint{4.899235in}{3.142379in}}%
\pgfpathlineto{\pgfqpoint{4.903517in}{3.139575in}}%
\pgfpathlineto{\pgfqpoint{4.908869in}{3.133086in}}%
\pgfpathlineto{\pgfqpoint{4.926799in}{3.108935in}}%
\pgfpathlineto{\pgfqpoint{4.931080in}{3.107596in}}%
\pgfpathlineto{\pgfqpoint{4.935094in}{3.108770in}}%
\pgfpathlineto{\pgfqpoint{4.939644in}{3.112734in}}%
\pgfpathlineto{\pgfqpoint{4.946066in}{3.121729in}}%
\pgfpathlineto{\pgfqpoint{4.958376in}{3.139312in}}%
\pgfpathlineto{\pgfqpoint{4.963193in}{3.142467in}}%
\pgfpathlineto{\pgfqpoint{4.967207in}{3.142580in}}%
\pgfpathlineto{\pgfqpoint{4.971222in}{3.140372in}}%
\pgfpathlineto{\pgfqpoint{4.976306in}{3.134734in}}%
\pgfpathlineto{\pgfqpoint{4.986743in}{3.118604in}}%
\pgfpathlineto{\pgfqpoint{4.993701in}{3.110187in}}%
\pgfpathlineto{\pgfqpoint{4.998250in}{3.107751in}}%
\pgfpathlineto{\pgfqpoint{5.002264in}{3.108080in}}%
\pgfpathlineto{\pgfqpoint{5.006546in}{3.110947in}}%
\pgfpathlineto{\pgfqpoint{5.012166in}{3.117881in}}%
\pgfpathlineto{\pgfqpoint{5.029292in}{3.141208in}}%
\pgfpathlineto{\pgfqpoint{5.033574in}{3.142809in}}%
\pgfpathlineto{\pgfqpoint{5.037588in}{3.141883in}}%
\pgfpathlineto{\pgfqpoint{5.041870in}{3.138449in}}%
\pgfpathlineto{\pgfqpoint{5.048025in}{3.130195in}}%
\pgfpathlineto{\pgfqpoint{5.061941in}{3.110510in}}%
\pgfpathlineto{\pgfqpoint{5.066757in}{3.107774in}}%
\pgfpathlineto{\pgfqpoint{5.070772in}{3.108041in}}%
\pgfpathlineto{\pgfqpoint{5.075053in}{3.110847in}}%
\pgfpathlineto{\pgfqpoint{5.080405in}{3.117337in}}%
\pgfpathlineto{\pgfqpoint{5.098335in}{3.141485in}}%
\pgfpathlineto{\pgfqpoint{5.102617in}{3.142823in}}%
\pgfpathlineto{\pgfqpoint{5.106631in}{3.141647in}}%
\pgfpathlineto{\pgfqpoint{5.111180in}{3.137681in}}%
\pgfpathlineto{\pgfqpoint{5.117603in}{3.128685in}}%
\pgfpathlineto{\pgfqpoint{5.129913in}{3.111104in}}%
\pgfpathlineto{\pgfqpoint{5.134730in}{3.107951in}}%
\pgfpathlineto{\pgfqpoint{5.138744in}{3.107840in}}%
\pgfpathlineto{\pgfqpoint{5.142758in}{3.110049in}}%
\pgfpathlineto{\pgfqpoint{5.147843in}{3.115689in}}%
\pgfpathlineto{\pgfqpoint{5.158279in}{3.131819in}}%
\pgfpathlineto{\pgfqpoint{5.165237in}{3.140235in}}%
\pgfpathlineto{\pgfqpoint{5.169787in}{3.142669in}}%
\pgfpathlineto{\pgfqpoint{5.173801in}{3.142338in}}%
\pgfpathlineto{\pgfqpoint{5.178082in}{3.139469in}}%
\pgfpathlineto{\pgfqpoint{5.183702in}{3.132534in}}%
\pgfpathlineto{\pgfqpoint{5.200829in}{3.109209in}}%
\pgfpathlineto{\pgfqpoint{5.205111in}{3.107609in}}%
\pgfpathlineto{\pgfqpoint{5.209125in}{3.108537in}}%
\pgfpathlineto{\pgfqpoint{5.213407in}{3.111973in}}%
\pgfpathlineto{\pgfqpoint{5.219562in}{3.120229in}}%
\pgfpathlineto{\pgfqpoint{5.233477in}{3.139912in}}%
\pgfpathlineto{\pgfqpoint{5.238294in}{3.142646in}}%
\pgfpathlineto{\pgfqpoint{5.242308in}{3.142377in}}%
\pgfpathlineto{\pgfqpoint{5.246590in}{3.139569in}}%
\pgfpathlineto{\pgfqpoint{5.251942in}{3.133077in}}%
\pgfpathlineto{\pgfqpoint{5.269872in}{3.108932in}}%
\pgfpathlineto{\pgfqpoint{5.274154in}{3.107596in}}%
\pgfpathlineto{\pgfqpoint{5.278168in}{3.108773in}}%
\pgfpathlineto{\pgfqpoint{5.282717in}{3.112741in}}%
\pgfpathlineto{\pgfqpoint{5.289140in}{3.121739in}}%
\pgfpathlineto{\pgfqpoint{5.301450in}{3.139318in}}%
\pgfpathlineto{\pgfqpoint{5.306266in}{3.142469in}}%
\pgfpathlineto{\pgfqpoint{5.310281in}{3.142578in}}%
\pgfpathlineto{\pgfqpoint{5.314295in}{3.140367in}}%
\pgfpathlineto{\pgfqpoint{5.319379in}{3.134726in}}%
\pgfpathlineto{\pgfqpoint{5.329816in}{3.118595in}}%
\pgfpathlineto{\pgfqpoint{5.336774in}{3.110181in}}%
\pgfpathlineto{\pgfqpoint{5.341323in}{3.107749in}}%
\pgfpathlineto{\pgfqpoint{5.345337in}{3.108082in}}%
\pgfpathlineto{\pgfqpoint{5.349619in}{3.110952in}}%
\pgfpathlineto{\pgfqpoint{5.355239in}{3.117890in}}%
\pgfpathlineto{\pgfqpoint{5.372366in}{3.141212in}}%
\pgfpathlineto{\pgfqpoint{5.376647in}{3.142810in}}%
\pgfpathlineto{\pgfqpoint{5.380661in}{3.141880in}}%
\pgfpathlineto{\pgfqpoint{5.384943in}{3.138442in}}%
\pgfpathlineto{\pgfqpoint{5.391098in}{3.130185in}}%
\pgfpathlineto{\pgfqpoint{5.405014in}{3.110504in}}%
\pgfpathlineto{\pgfqpoint{5.409831in}{3.107773in}}%
\pgfpathlineto{\pgfqpoint{5.413845in}{3.108043in}}%
\pgfpathlineto{\pgfqpoint{5.418127in}{3.110853in}}%
\pgfpathlineto{\pgfqpoint{5.423479in}{3.117346in}}%
\pgfpathlineto{\pgfqpoint{5.441408in}{3.141489in}}%
\pgfpathlineto{\pgfqpoint{5.445690in}{3.142823in}}%
\pgfpathlineto{\pgfqpoint{5.449704in}{3.141644in}}%
\pgfpathlineto{\pgfqpoint{5.454254in}{3.137674in}}%
\pgfpathlineto{\pgfqpoint{5.460676in}{3.128675in}}%
\pgfpathlineto{\pgfqpoint{5.472986in}{3.111098in}}%
\pgfpathlineto{\pgfqpoint{5.477803in}{3.107949in}}%
\pgfpathlineto{\pgfqpoint{5.481817in}{3.107841in}}%
\pgfpathlineto{\pgfqpoint{5.485831in}{3.110054in}}%
\pgfpathlineto{\pgfqpoint{5.490916in}{3.115698in}}%
\pgfpathlineto{\pgfqpoint{5.501353in}{3.131829in}}%
\pgfpathlineto{\pgfqpoint{5.508310in}{3.140240in}}%
\pgfpathlineto{\pgfqpoint{5.512860in}{3.142670in}}%
\pgfpathlineto{\pgfqpoint{5.516874in}{3.142336in}}%
\pgfpathlineto{\pgfqpoint{5.521156in}{3.139464in}}%
\pgfpathlineto{\pgfqpoint{5.526775in}{3.132524in}}%
\pgfpathlineto{\pgfqpoint{5.543902in}{3.109205in}}%
\pgfpathlineto{\pgfqpoint{5.548184in}{3.107609in}}%
\pgfpathlineto{\pgfqpoint{5.552198in}{3.108540in}}%
\pgfpathlineto{\pgfqpoint{5.556480in}{3.111980in}}%
\pgfpathlineto{\pgfqpoint{5.562635in}{3.120239in}}%
\pgfpathlineto{\pgfqpoint{5.576550in}{3.139918in}}%
\pgfpathlineto{\pgfqpoint{5.581367in}{3.142647in}}%
\pgfpathlineto{\pgfqpoint{5.585381in}{3.142375in}}%
\pgfpathlineto{\pgfqpoint{5.589663in}{3.139563in}}%
\pgfpathlineto{\pgfqpoint{5.595015in}{3.133068in}}%
\pgfpathlineto{\pgfqpoint{5.612945in}{3.108928in}}%
\pgfpathlineto{\pgfqpoint{5.617227in}{3.107596in}}%
\pgfpathlineto{\pgfqpoint{5.621241in}{3.108777in}}%
\pgfpathlineto{\pgfqpoint{5.625790in}{3.112749in}}%
\pgfpathlineto{\pgfqpoint{5.632213in}{3.121749in}}%
\pgfpathlineto{\pgfqpoint{5.644523in}{3.139324in}}%
\pgfpathlineto{\pgfqpoint{5.649340in}{3.142471in}}%
\pgfpathlineto{\pgfqpoint{5.653354in}{3.142577in}}%
\pgfpathlineto{\pgfqpoint{5.657368in}{3.140362in}}%
\pgfpathlineto{\pgfqpoint{5.662453in}{3.134717in}}%
\pgfpathlineto{\pgfqpoint{5.673157in}{3.118188in}}%
\pgfpathlineto{\pgfqpoint{5.679847in}{3.110176in}}%
\pgfpathlineto{\pgfqpoint{5.684396in}{3.107748in}}%
\pgfpathlineto{\pgfqpoint{5.688410in}{3.108084in}}%
\pgfpathlineto{\pgfqpoint{5.692692in}{3.110958in}}%
\pgfpathlineto{\pgfqpoint{5.698312in}{3.117899in}}%
\pgfpathlineto{\pgfqpoint{5.715439in}{3.141216in}}%
\pgfpathlineto{\pgfqpoint{5.719721in}{3.142810in}}%
\pgfpathlineto{\pgfqpoint{5.723735in}{3.141877in}}%
\pgfpathlineto{\pgfqpoint{5.728016in}{3.138436in}}%
\pgfpathlineto{\pgfqpoint{5.734171in}{3.130175in}}%
\pgfpathlineto{\pgfqpoint{5.748087in}{3.110499in}}%
\pgfpathlineto{\pgfqpoint{5.752904in}{3.107771in}}%
\pgfpathlineto{\pgfqpoint{5.756918in}{3.108045in}}%
\pgfpathlineto{\pgfqpoint{5.761200in}{3.110859in}}%
\pgfpathlineto{\pgfqpoint{5.766552in}{3.117355in}}%
\pgfpathlineto{\pgfqpoint{5.784482in}{3.141493in}}%
\pgfpathlineto{\pgfqpoint{5.788763in}{3.142823in}}%
\pgfpathlineto{\pgfqpoint{5.792778in}{3.141640in}}%
\pgfpathlineto{\pgfqpoint{5.797327in}{3.137667in}}%
\pgfpathlineto{\pgfqpoint{5.803749in}{3.128665in}}%
\pgfpathlineto{\pgfqpoint{5.816059in}{3.111092in}}%
\pgfpathlineto{\pgfqpoint{5.820876in}{3.107947in}}%
\pgfpathlineto{\pgfqpoint{5.824890in}{3.107843in}}%
\pgfpathlineto{\pgfqpoint{5.828905in}{3.110060in}}%
\pgfpathlineto{\pgfqpoint{5.833989in}{3.115706in}}%
\pgfpathlineto{\pgfqpoint{5.844693in}{3.132236in}}%
\pgfpathlineto{\pgfqpoint{5.851384in}{3.140245in}}%
\pgfpathlineto{\pgfqpoint{5.855933in}{3.142672in}}%
\pgfpathlineto{\pgfqpoint{5.859947in}{3.142333in}}%
\pgfpathlineto{\pgfqpoint{5.864229in}{3.139458in}}%
\pgfpathlineto{\pgfqpoint{5.869849in}{3.132515in}}%
\pgfpathlineto{\pgfqpoint{5.886975in}{3.109200in}}%
\pgfpathlineto{\pgfqpoint{5.891257in}{3.107609in}}%
\pgfpathlineto{\pgfqpoint{5.895271in}{3.108544in}}%
\pgfpathlineto{\pgfqpoint{5.899553in}{3.111987in}}%
\pgfpathlineto{\pgfqpoint{5.905708in}{3.120248in}}%
\pgfpathlineto{\pgfqpoint{5.919624in}{3.139923in}}%
\pgfpathlineto{\pgfqpoint{5.924441in}{3.142648in}}%
\pgfpathlineto{\pgfqpoint{5.928455in}{3.142372in}}%
\pgfpathlineto{\pgfqpoint{5.932736in}{3.139557in}}%
\pgfpathlineto{\pgfqpoint{5.938089in}{3.133059in}}%
\pgfpathlineto{\pgfqpoint{5.956018in}{3.108924in}}%
\pgfpathlineto{\pgfqpoint{5.960300in}{3.107596in}}%
\pgfpathlineto{\pgfqpoint{5.964314in}{3.108781in}}%
\pgfpathlineto{\pgfqpoint{5.968863in}{3.112756in}}%
\pgfpathlineto{\pgfqpoint{5.975286in}{3.121759in}}%
\pgfpathlineto{\pgfqpoint{5.987596in}{3.139330in}}%
\pgfpathlineto{\pgfqpoint{5.992413in}{3.142473in}}%
\pgfpathlineto{\pgfqpoint{5.996427in}{3.142575in}}%
\pgfpathlineto{\pgfqpoint{6.000441in}{3.140357in}}%
\pgfpathlineto{\pgfqpoint{6.005526in}{3.134709in}}%
\pgfpathlineto{\pgfqpoint{6.016230in}{3.118179in}}%
\pgfpathlineto{\pgfqpoint{6.022920in}{3.110171in}}%
\pgfpathlineto{\pgfqpoint{6.027470in}{3.107747in}}%
\pgfpathlineto{\pgfqpoint{6.031484in}{3.108087in}}%
\pgfpathlineto{\pgfqpoint{6.035765in}{3.110964in}}%
\pgfpathlineto{\pgfqpoint{6.041385in}{3.117908in}}%
\pgfpathlineto{\pgfqpoint{6.058512in}{3.141221in}}%
\pgfpathlineto{\pgfqpoint{6.062794in}{3.142810in}}%
\pgfpathlineto{\pgfqpoint{6.066808in}{3.141874in}}%
\pgfpathlineto{\pgfqpoint{6.071090in}{3.138429in}}%
\pgfpathlineto{\pgfqpoint{6.077245in}{3.130166in}}%
\pgfpathlineto{\pgfqpoint{6.091160in}{3.110493in}}%
\pgfpathlineto{\pgfqpoint{6.095977in}{3.107770in}}%
\pgfpathlineto{\pgfqpoint{6.099991in}{3.108048in}}%
\pgfpathlineto{\pgfqpoint{6.104273in}{3.110865in}}%
\pgfpathlineto{\pgfqpoint{6.109625in}{3.117364in}}%
\pgfpathlineto{\pgfqpoint{6.127555in}{3.141497in}}%
\pgfpathlineto{\pgfqpoint{6.131837in}{3.142823in}}%
\pgfpathlineto{\pgfqpoint{6.135851in}{3.141636in}}%
\pgfpathlineto{\pgfqpoint{6.140400in}{3.137660in}}%
\pgfpathlineto{\pgfqpoint{6.146823in}{3.128655in}}%
\pgfpathlineto{\pgfqpoint{6.159133in}{3.111086in}}%
\pgfpathlineto{\pgfqpoint{6.163950in}{3.107945in}}%
\pgfpathlineto{\pgfqpoint{6.167964in}{3.107845in}}%
\pgfpathlineto{\pgfqpoint{6.171978in}{3.110065in}}%
\pgfpathlineto{\pgfqpoint{6.177062in}{3.115714in}}%
\pgfpathlineto{\pgfqpoint{6.187767in}{3.132245in}}%
\pgfpathlineto{\pgfqpoint{6.194457in}{3.140251in}}%
\pgfpathlineto{\pgfqpoint{6.199006in}{3.142673in}}%
\pgfpathlineto{\pgfqpoint{6.203020in}{3.142331in}}%
\pgfpathlineto{\pgfqpoint{6.207302in}{3.139452in}}%
\pgfpathlineto{\pgfqpoint{6.212922in}{3.132506in}}%
\pgfpathlineto{\pgfqpoint{6.230049in}{3.109196in}}%
\pgfpathlineto{\pgfqpoint{6.234330in}{3.107608in}}%
\pgfpathlineto{\pgfqpoint{6.238345in}{3.108547in}}%
\pgfpathlineto{\pgfqpoint{6.242626in}{3.111993in}}%
\pgfpathlineto{\pgfqpoint{6.248781in}{3.120258in}}%
\pgfpathlineto{\pgfqpoint{6.262697in}{3.139929in}}%
\pgfpathlineto{\pgfqpoint{6.267514in}{3.142650in}}%
\pgfpathlineto{\pgfqpoint{6.271528in}{3.142370in}}%
\pgfpathlineto{\pgfqpoint{6.275810in}{3.139551in}}%
\pgfpathlineto{\pgfqpoint{6.281162in}{3.133050in}}%
\pgfpathlineto{\pgfqpoint{6.299092in}{3.108920in}}%
\pgfpathlineto{\pgfqpoint{6.303373in}{3.107596in}}%
\pgfpathlineto{\pgfqpoint{6.307387in}{3.108784in}}%
\pgfpathlineto{\pgfqpoint{6.311937in}{3.112763in}}%
\pgfpathlineto{\pgfqpoint{6.318359in}{3.121769in}}%
\pgfpathlineto{\pgfqpoint{6.330669in}{3.139336in}}%
\pgfpathlineto{\pgfqpoint{6.335486in}{3.142475in}}%
\pgfpathlineto{\pgfqpoint{6.339500in}{3.142573in}}%
\pgfpathlineto{\pgfqpoint{6.343514in}{3.140352in}}%
\pgfpathlineto{\pgfqpoint{6.348599in}{3.134700in}}%
\pgfpathlineto{\pgfqpoint{6.359303in}{3.118169in}}%
\pgfpathlineto{\pgfqpoint{6.365993in}{3.110166in}}%
\pgfpathlineto{\pgfqpoint{6.370543in}{3.107745in}}%
\pgfpathlineto{\pgfqpoint{6.374557in}{3.108089in}}%
\pgfpathlineto{\pgfqpoint{6.378839in}{3.110970in}}%
\pgfpathlineto{\pgfqpoint{6.384458in}{3.117917in}}%
\pgfpathlineto{\pgfqpoint{6.401585in}{3.141225in}}%
\pgfpathlineto{\pgfqpoint{6.405867in}{3.142811in}}%
\pgfpathlineto{\pgfqpoint{6.409881in}{3.141870in}}%
\pgfpathlineto{\pgfqpoint{6.414163in}{3.138422in}}%
\pgfpathlineto{\pgfqpoint{6.420318in}{3.130156in}}%
\pgfpathlineto{\pgfqpoint{6.434233in}{3.110488in}}%
\pgfpathlineto{\pgfqpoint{6.439050in}{3.107768in}}%
\pgfpathlineto{\pgfqpoint{6.443065in}{3.108050in}}%
\pgfpathlineto{\pgfqpoint{6.447346in}{3.110870in}}%
\pgfpathlineto{\pgfqpoint{6.452698in}{3.117373in}}%
\pgfpathlineto{\pgfqpoint{6.470628in}{3.141501in}}%
\pgfpathlineto{\pgfqpoint{6.474910in}{3.142823in}}%
\pgfpathlineto{\pgfqpoint{6.478924in}{3.141633in}}%
\pgfpathlineto{\pgfqpoint{6.483473in}{3.137653in}}%
\pgfpathlineto{\pgfqpoint{6.489896in}{3.128645in}}%
\pgfpathlineto{\pgfqpoint{6.502206in}{3.111080in}}%
\pgfpathlineto{\pgfqpoint{6.507023in}{3.107943in}}%
\pgfpathlineto{\pgfqpoint{6.511037in}{3.107846in}}%
\pgfpathlineto{\pgfqpoint{6.515051in}{3.110070in}}%
\pgfpathlineto{\pgfqpoint{6.520136in}{3.115723in}}%
\pgfpathlineto{\pgfqpoint{6.530840in}{3.132254in}}%
\pgfpathlineto{\pgfqpoint{6.537530in}{3.140256in}}%
\pgfpathlineto{\pgfqpoint{6.542079in}{3.142674in}}%
\pgfpathlineto{\pgfqpoint{6.546094in}{3.142329in}}%
\pgfpathlineto{\pgfqpoint{6.550375in}{3.139446in}}%
\pgfpathlineto{\pgfqpoint{6.555995in}{3.132497in}}%
\pgfpathlineto{\pgfqpoint{6.572854in}{3.109376in}}%
\pgfpathlineto{\pgfqpoint{6.577136in}{3.107629in}}%
\pgfpathlineto{\pgfqpoint{6.581150in}{3.108415in}}%
\pgfpathlineto{\pgfqpoint{6.585432in}{3.111718in}}%
\pgfpathlineto{\pgfqpoint{6.591319in}{3.119445in}}%
\pgfpathlineto{\pgfqpoint{6.606305in}{3.140390in}}%
\pgfpathlineto{\pgfqpoint{6.610855in}{3.142706in}}%
\pgfpathlineto{\pgfqpoint{6.614869in}{3.142265in}}%
\pgfpathlineto{\pgfqpoint{6.619150in}{3.139291in}}%
\pgfpathlineto{\pgfqpoint{6.624770in}{3.132259in}}%
\pgfpathlineto{\pgfqpoint{6.641362in}{3.109452in}}%
\pgfpathlineto{\pgfqpoint{6.645644in}{3.107640in}}%
\pgfpathlineto{\pgfqpoint{6.649658in}{3.108364in}}%
\pgfpathlineto{\pgfqpoint{6.653939in}{3.111609in}}%
\pgfpathlineto{\pgfqpoint{6.659827in}{3.119284in}}%
\pgfpathlineto{\pgfqpoint{6.663306in}{3.124778in}}%
\pgfpathlineto{\pgfqpoint{6.663306in}{3.124778in}}%
\pgfusepath{stroke}%
\end{pgfscope}%
\begin{pgfscope}%
\pgfpathrectangle{\pgfqpoint{0.467797in}{2.292089in}}{\pgfqpoint{6.490533in}{1.666241in}}%
\pgfusepath{clip}%
\pgfsetrectcap%
\pgfsetroundjoin%
\pgfsetlinewidth{1.505625pt}%
\definecolor{currentstroke}{rgb}{0.172549,0.627451,0.172549}%
\pgfsetstrokecolor{currentstroke}%
\pgfsetdash{}{0pt}%
\pgfpathmoveto{\pgfqpoint{0.762821in}{3.125209in}}%
\pgfpathlineto{\pgfqpoint{0.772187in}{3.138368in}}%
\pgfpathlineto{\pgfqpoint{0.777004in}{3.141665in}}%
\pgfpathlineto{\pgfqpoint{0.781018in}{3.141789in}}%
\pgfpathlineto{\pgfqpoint{0.785032in}{3.139489in}}%
\pgfpathlineto{\pgfqpoint{0.790117in}{3.133656in}}%
\pgfpathlineto{\pgfqpoint{0.808849in}{3.109137in}}%
\pgfpathlineto{\pgfqpoint{0.812864in}{3.108437in}}%
\pgfpathlineto{\pgfqpoint{0.816610in}{3.109999in}}%
\pgfpathlineto{\pgfqpoint{0.821159in}{3.114465in}}%
\pgfpathlineto{\pgfqpoint{0.828652in}{3.125641in}}%
\pgfpathlineto{\pgfqpoint{0.838019in}{3.138633in}}%
\pgfpathlineto{\pgfqpoint{0.842836in}{3.141750in}}%
\pgfpathlineto{\pgfqpoint{0.846582in}{3.141789in}}%
\pgfpathlineto{\pgfqpoint{0.850596in}{3.139489in}}%
\pgfpathlineto{\pgfqpoint{0.855681in}{3.133656in}}%
\pgfpathlineto{\pgfqpoint{0.874413in}{3.109137in}}%
\pgfpathlineto{\pgfqpoint{0.878427in}{3.108437in}}%
\pgfpathlineto{\pgfqpoint{0.882174in}{3.109999in}}%
\pgfpathlineto{\pgfqpoint{0.886723in}{3.114465in}}%
\pgfpathlineto{\pgfqpoint{0.894216in}{3.125641in}}%
\pgfpathlineto{\pgfqpoint{0.903583in}{3.138633in}}%
\pgfpathlineto{\pgfqpoint{0.908400in}{3.141750in}}%
\pgfpathlineto{\pgfqpoint{0.912146in}{3.141789in}}%
\pgfpathlineto{\pgfqpoint{0.916160in}{3.139489in}}%
\pgfpathlineto{\pgfqpoint{0.921245in}{3.133656in}}%
\pgfpathlineto{\pgfqpoint{0.939977in}{3.109137in}}%
\pgfpathlineto{\pgfqpoint{0.943991in}{3.108437in}}%
\pgfpathlineto{\pgfqpoint{0.947738in}{3.109999in}}%
\pgfpathlineto{\pgfqpoint{0.952287in}{3.114465in}}%
\pgfpathlineto{\pgfqpoint{0.959780in}{3.125641in}}%
\pgfpathlineto{\pgfqpoint{0.969147in}{3.138633in}}%
\pgfpathlineto{\pgfqpoint{0.973963in}{3.141750in}}%
\pgfpathlineto{\pgfqpoint{0.977710in}{3.141789in}}%
\pgfpathlineto{\pgfqpoint{0.981724in}{3.139489in}}%
\pgfpathlineto{\pgfqpoint{0.986809in}{3.133656in}}%
\pgfpathlineto{\pgfqpoint{1.005541in}{3.109137in}}%
\pgfpathlineto{\pgfqpoint{1.009555in}{3.108437in}}%
\pgfpathlineto{\pgfqpoint{1.013302in}{3.109999in}}%
\pgfpathlineto{\pgfqpoint{1.017851in}{3.114465in}}%
\pgfpathlineto{\pgfqpoint{1.025344in}{3.125641in}}%
\pgfpathlineto{\pgfqpoint{1.034710in}{3.138633in}}%
\pgfpathlineto{\pgfqpoint{1.039527in}{3.141750in}}%
\pgfpathlineto{\pgfqpoint{1.043274in}{3.141789in}}%
\pgfpathlineto{\pgfqpoint{1.047288in}{3.139489in}}%
\pgfpathlineto{\pgfqpoint{1.052373in}{3.133656in}}%
\pgfpathlineto{\pgfqpoint{1.071105in}{3.109137in}}%
\pgfpathlineto{\pgfqpoint{1.075119in}{3.108437in}}%
\pgfpathlineto{\pgfqpoint{1.078866in}{3.109999in}}%
\pgfpathlineto{\pgfqpoint{1.083415in}{3.114465in}}%
\pgfpathlineto{\pgfqpoint{1.090908in}{3.125641in}}%
\pgfpathlineto{\pgfqpoint{1.100274in}{3.138633in}}%
\pgfpathlineto{\pgfqpoint{1.105091in}{3.141750in}}%
\pgfpathlineto{\pgfqpoint{1.108838in}{3.141789in}}%
\pgfpathlineto{\pgfqpoint{1.112852in}{3.139489in}}%
\pgfpathlineto{\pgfqpoint{1.117936in}{3.133656in}}%
\pgfpathlineto{\pgfqpoint{1.136669in}{3.109137in}}%
\pgfpathlineto{\pgfqpoint{1.140683in}{3.108437in}}%
\pgfpathlineto{\pgfqpoint{1.144430in}{3.109999in}}%
\pgfpathlineto{\pgfqpoint{1.148979in}{3.114465in}}%
\pgfpathlineto{\pgfqpoint{1.156472in}{3.125641in}}%
\pgfpathlineto{\pgfqpoint{1.165838in}{3.138633in}}%
\pgfpathlineto{\pgfqpoint{1.170655in}{3.141750in}}%
\pgfpathlineto{\pgfqpoint{1.174402in}{3.141789in}}%
\pgfpathlineto{\pgfqpoint{1.178416in}{3.139489in}}%
\pgfpathlineto{\pgfqpoint{1.183500in}{3.133656in}}%
\pgfpathlineto{\pgfqpoint{1.202233in}{3.109137in}}%
\pgfpathlineto{\pgfqpoint{1.206247in}{3.108437in}}%
\pgfpathlineto{\pgfqpoint{1.209994in}{3.109999in}}%
\pgfpathlineto{\pgfqpoint{1.214543in}{3.114465in}}%
\pgfpathlineto{\pgfqpoint{1.222036in}{3.125641in}}%
\pgfpathlineto{\pgfqpoint{1.231402in}{3.138633in}}%
\pgfpathlineto{\pgfqpoint{1.236219in}{3.141750in}}%
\pgfpathlineto{\pgfqpoint{1.239966in}{3.141789in}}%
\pgfpathlineto{\pgfqpoint{1.243980in}{3.139489in}}%
\pgfpathlineto{\pgfqpoint{1.249064in}{3.133656in}}%
\pgfpathlineto{\pgfqpoint{1.267797in}{3.109137in}}%
\pgfpathlineto{\pgfqpoint{1.271811in}{3.108437in}}%
\pgfpathlineto{\pgfqpoint{1.275557in}{3.109999in}}%
\pgfpathlineto{\pgfqpoint{1.280107in}{3.114465in}}%
\pgfpathlineto{\pgfqpoint{1.287600in}{3.125641in}}%
\pgfpathlineto{\pgfqpoint{1.296966in}{3.138633in}}%
\pgfpathlineto{\pgfqpoint{1.301783in}{3.141750in}}%
\pgfpathlineto{\pgfqpoint{1.305530in}{3.141789in}}%
\pgfpathlineto{\pgfqpoint{1.309544in}{3.139489in}}%
\pgfpathlineto{\pgfqpoint{1.314628in}{3.133656in}}%
\pgfpathlineto{\pgfqpoint{1.333361in}{3.109137in}}%
\pgfpathlineto{\pgfqpoint{1.337375in}{3.108437in}}%
\pgfpathlineto{\pgfqpoint{1.341121in}{3.109999in}}%
\pgfpathlineto{\pgfqpoint{1.345671in}{3.114465in}}%
\pgfpathlineto{\pgfqpoint{1.353164in}{3.125641in}}%
\pgfpathlineto{\pgfqpoint{1.362530in}{3.138633in}}%
\pgfpathlineto{\pgfqpoint{1.367347in}{3.141750in}}%
\pgfpathlineto{\pgfqpoint{1.371093in}{3.141789in}}%
\pgfpathlineto{\pgfqpoint{1.375108in}{3.139489in}}%
\pgfpathlineto{\pgfqpoint{1.380192in}{3.133656in}}%
\pgfpathlineto{\pgfqpoint{1.398925in}{3.109137in}}%
\pgfpathlineto{\pgfqpoint{1.402939in}{3.108437in}}%
\pgfpathlineto{\pgfqpoint{1.406685in}{3.109999in}}%
\pgfpathlineto{\pgfqpoint{1.411235in}{3.114465in}}%
\pgfpathlineto{\pgfqpoint{1.418728in}{3.125641in}}%
\pgfpathlineto{\pgfqpoint{1.428094in}{3.138633in}}%
\pgfpathlineto{\pgfqpoint{1.432911in}{3.141750in}}%
\pgfpathlineto{\pgfqpoint{1.436657in}{3.141789in}}%
\pgfpathlineto{\pgfqpoint{1.440672in}{3.139489in}}%
\pgfpathlineto{\pgfqpoint{1.445756in}{3.133656in}}%
\pgfpathlineto{\pgfqpoint{1.464489in}{3.109137in}}%
\pgfpathlineto{\pgfqpoint{1.468503in}{3.108437in}}%
\pgfpathlineto{\pgfqpoint{1.472249in}{3.109999in}}%
\pgfpathlineto{\pgfqpoint{1.476799in}{3.114465in}}%
\pgfpathlineto{\pgfqpoint{1.484292in}{3.125641in}}%
\pgfpathlineto{\pgfqpoint{1.493658in}{3.138633in}}%
\pgfpathlineto{\pgfqpoint{1.498475in}{3.141750in}}%
\pgfpathlineto{\pgfqpoint{1.502221in}{3.141789in}}%
\pgfpathlineto{\pgfqpoint{1.506235in}{3.139489in}}%
\pgfpathlineto{\pgfqpoint{1.511320in}{3.133656in}}%
\pgfpathlineto{\pgfqpoint{1.530053in}{3.109137in}}%
\pgfpathlineto{\pgfqpoint{1.534067in}{3.108437in}}%
\pgfpathlineto{\pgfqpoint{1.537813in}{3.109999in}}%
\pgfpathlineto{\pgfqpoint{1.542362in}{3.114465in}}%
\pgfpathlineto{\pgfqpoint{1.549856in}{3.125641in}}%
\pgfpathlineto{\pgfqpoint{1.559222in}{3.138633in}}%
\pgfpathlineto{\pgfqpoint{1.564039in}{3.141750in}}%
\pgfpathlineto{\pgfqpoint{1.567785in}{3.141789in}}%
\pgfpathlineto{\pgfqpoint{1.571799in}{3.139489in}}%
\pgfpathlineto{\pgfqpoint{1.576884in}{3.133656in}}%
\pgfpathlineto{\pgfqpoint{1.595616in}{3.109137in}}%
\pgfpathlineto{\pgfqpoint{1.599631in}{3.108437in}}%
\pgfpathlineto{\pgfqpoint{1.603377in}{3.109999in}}%
\pgfpathlineto{\pgfqpoint{1.607926in}{3.114465in}}%
\pgfpathlineto{\pgfqpoint{1.615419in}{3.125641in}}%
\pgfpathlineto{\pgfqpoint{1.624786in}{3.138633in}}%
\pgfpathlineto{\pgfqpoint{1.629603in}{3.141750in}}%
\pgfpathlineto{\pgfqpoint{1.633349in}{3.141789in}}%
\pgfpathlineto{\pgfqpoint{1.637363in}{3.139489in}}%
\pgfpathlineto{\pgfqpoint{1.642448in}{3.133656in}}%
\pgfpathlineto{\pgfqpoint{1.661180in}{3.109137in}}%
\pgfpathlineto{\pgfqpoint{1.665194in}{3.108437in}}%
\pgfpathlineto{\pgfqpoint{1.668941in}{3.109999in}}%
\pgfpathlineto{\pgfqpoint{1.673490in}{3.114465in}}%
\pgfpathlineto{\pgfqpoint{1.680983in}{3.125641in}}%
\pgfpathlineto{\pgfqpoint{1.690350in}{3.138633in}}%
\pgfpathlineto{\pgfqpoint{1.695167in}{3.141750in}}%
\pgfpathlineto{\pgfqpoint{1.698913in}{3.141789in}}%
\pgfpathlineto{\pgfqpoint{1.702927in}{3.139489in}}%
\pgfpathlineto{\pgfqpoint{1.708012in}{3.133656in}}%
\pgfpathlineto{\pgfqpoint{1.726744in}{3.109137in}}%
\pgfpathlineto{\pgfqpoint{1.730758in}{3.108437in}}%
\pgfpathlineto{\pgfqpoint{1.734505in}{3.109999in}}%
\pgfpathlineto{\pgfqpoint{1.739054in}{3.114465in}}%
\pgfpathlineto{\pgfqpoint{1.746547in}{3.125641in}}%
\pgfpathlineto{\pgfqpoint{1.755914in}{3.138633in}}%
\pgfpathlineto{\pgfqpoint{1.760730in}{3.141750in}}%
\pgfpathlineto{\pgfqpoint{1.764477in}{3.141789in}}%
\pgfpathlineto{\pgfqpoint{1.768491in}{3.139489in}}%
\pgfpathlineto{\pgfqpoint{1.773576in}{3.133656in}}%
\pgfpathlineto{\pgfqpoint{1.792308in}{3.109137in}}%
\pgfpathlineto{\pgfqpoint{1.796322in}{3.108437in}}%
\pgfpathlineto{\pgfqpoint{1.800069in}{3.109999in}}%
\pgfpathlineto{\pgfqpoint{1.804618in}{3.114465in}}%
\pgfpathlineto{\pgfqpoint{1.812111in}{3.125641in}}%
\pgfpathlineto{\pgfqpoint{1.821477in}{3.138633in}}%
\pgfpathlineto{\pgfqpoint{1.826294in}{3.141750in}}%
\pgfpathlineto{\pgfqpoint{1.830041in}{3.141789in}}%
\pgfpathlineto{\pgfqpoint{1.834055in}{3.139489in}}%
\pgfpathlineto{\pgfqpoint{1.839140in}{3.133656in}}%
\pgfpathlineto{\pgfqpoint{1.857872in}{3.109137in}}%
\pgfpathlineto{\pgfqpoint{1.861886in}{3.108437in}}%
\pgfpathlineto{\pgfqpoint{1.865633in}{3.109999in}}%
\pgfpathlineto{\pgfqpoint{1.870182in}{3.114465in}}%
\pgfpathlineto{\pgfqpoint{1.877675in}{3.125641in}}%
\pgfpathlineto{\pgfqpoint{1.887041in}{3.138633in}}%
\pgfpathlineto{\pgfqpoint{1.891858in}{3.141750in}}%
\pgfpathlineto{\pgfqpoint{1.895605in}{3.141789in}}%
\pgfpathlineto{\pgfqpoint{1.899619in}{3.139489in}}%
\pgfpathlineto{\pgfqpoint{1.904703in}{3.133656in}}%
\pgfpathlineto{\pgfqpoint{1.923436in}{3.109137in}}%
\pgfpathlineto{\pgfqpoint{1.927450in}{3.108437in}}%
\pgfpathlineto{\pgfqpoint{1.931197in}{3.109999in}}%
\pgfpathlineto{\pgfqpoint{1.935746in}{3.114465in}}%
\pgfpathlineto{\pgfqpoint{1.943239in}{3.125641in}}%
\pgfpathlineto{\pgfqpoint{1.952605in}{3.138633in}}%
\pgfpathlineto{\pgfqpoint{1.957422in}{3.141750in}}%
\pgfpathlineto{\pgfqpoint{1.961169in}{3.141789in}}%
\pgfpathlineto{\pgfqpoint{1.965183in}{3.139489in}}%
\pgfpathlineto{\pgfqpoint{1.970267in}{3.133656in}}%
\pgfpathlineto{\pgfqpoint{1.989000in}{3.109137in}}%
\pgfpathlineto{\pgfqpoint{1.993014in}{3.108437in}}%
\pgfpathlineto{\pgfqpoint{1.996761in}{3.109999in}}%
\pgfpathlineto{\pgfqpoint{2.001310in}{3.114465in}}%
\pgfpathlineto{\pgfqpoint{2.008803in}{3.125641in}}%
\pgfpathlineto{\pgfqpoint{2.018169in}{3.138633in}}%
\pgfpathlineto{\pgfqpoint{2.022986in}{3.141750in}}%
\pgfpathlineto{\pgfqpoint{2.026733in}{3.141789in}}%
\pgfpathlineto{\pgfqpoint{2.030747in}{3.139489in}}%
\pgfpathlineto{\pgfqpoint{2.035831in}{3.133656in}}%
\pgfpathlineto{\pgfqpoint{2.054564in}{3.109137in}}%
\pgfpathlineto{\pgfqpoint{2.058578in}{3.108437in}}%
\pgfpathlineto{\pgfqpoint{2.062324in}{3.109999in}}%
\pgfpathlineto{\pgfqpoint{2.066874in}{3.114465in}}%
\pgfpathlineto{\pgfqpoint{2.074367in}{3.125641in}}%
\pgfpathlineto{\pgfqpoint{2.083733in}{3.138633in}}%
\pgfpathlineto{\pgfqpoint{2.088550in}{3.141750in}}%
\pgfpathlineto{\pgfqpoint{2.092297in}{3.141789in}}%
\pgfpathlineto{\pgfqpoint{2.096311in}{3.139489in}}%
\pgfpathlineto{\pgfqpoint{2.101395in}{3.133656in}}%
\pgfpathlineto{\pgfqpoint{2.120128in}{3.109137in}}%
\pgfpathlineto{\pgfqpoint{2.124142in}{3.108437in}}%
\pgfpathlineto{\pgfqpoint{2.127888in}{3.109999in}}%
\pgfpathlineto{\pgfqpoint{2.132438in}{3.114465in}}%
\pgfpathlineto{\pgfqpoint{2.139931in}{3.125641in}}%
\pgfpathlineto{\pgfqpoint{2.149297in}{3.138633in}}%
\pgfpathlineto{\pgfqpoint{2.154114in}{3.141750in}}%
\pgfpathlineto{\pgfqpoint{2.157860in}{3.141789in}}%
\pgfpathlineto{\pgfqpoint{2.161875in}{3.139489in}}%
\pgfpathlineto{\pgfqpoint{2.166959in}{3.133656in}}%
\pgfpathlineto{\pgfqpoint{2.185692in}{3.109137in}}%
\pgfpathlineto{\pgfqpoint{2.189706in}{3.108437in}}%
\pgfpathlineto{\pgfqpoint{2.193452in}{3.109999in}}%
\pgfpathlineto{\pgfqpoint{2.198002in}{3.114465in}}%
\pgfpathlineto{\pgfqpoint{2.205495in}{3.125641in}}%
\pgfpathlineto{\pgfqpoint{2.214861in}{3.138633in}}%
\pgfpathlineto{\pgfqpoint{2.219678in}{3.141750in}}%
\pgfpathlineto{\pgfqpoint{2.223424in}{3.141789in}}%
\pgfpathlineto{\pgfqpoint{2.227439in}{3.139489in}}%
\pgfpathlineto{\pgfqpoint{2.232523in}{3.133656in}}%
\pgfpathlineto{\pgfqpoint{2.251256in}{3.109137in}}%
\pgfpathlineto{\pgfqpoint{2.255270in}{3.108437in}}%
\pgfpathlineto{\pgfqpoint{2.259016in}{3.109999in}}%
\pgfpathlineto{\pgfqpoint{2.263566in}{3.114465in}}%
\pgfpathlineto{\pgfqpoint{2.271059in}{3.125641in}}%
\pgfpathlineto{\pgfqpoint{2.280425in}{3.138633in}}%
\pgfpathlineto{\pgfqpoint{2.285242in}{3.141750in}}%
\pgfpathlineto{\pgfqpoint{2.288988in}{3.141789in}}%
\pgfpathlineto{\pgfqpoint{2.293002in}{3.139489in}}%
\pgfpathlineto{\pgfqpoint{2.298087in}{3.133656in}}%
\pgfpathlineto{\pgfqpoint{2.316820in}{3.109137in}}%
\pgfpathlineto{\pgfqpoint{2.320834in}{3.108437in}}%
\pgfpathlineto{\pgfqpoint{2.324580in}{3.109999in}}%
\pgfpathlineto{\pgfqpoint{2.329129in}{3.114465in}}%
\pgfpathlineto{\pgfqpoint{2.336623in}{3.125641in}}%
\pgfpathlineto{\pgfqpoint{2.345989in}{3.138633in}}%
\pgfpathlineto{\pgfqpoint{2.350806in}{3.141750in}}%
\pgfpathlineto{\pgfqpoint{2.354552in}{3.141789in}}%
\pgfpathlineto{\pgfqpoint{2.358566in}{3.139489in}}%
\pgfpathlineto{\pgfqpoint{2.363651in}{3.133656in}}%
\pgfpathlineto{\pgfqpoint{2.382383in}{3.109137in}}%
\pgfpathlineto{\pgfqpoint{2.386398in}{3.108437in}}%
\pgfpathlineto{\pgfqpoint{2.390144in}{3.109999in}}%
\pgfpathlineto{\pgfqpoint{2.394693in}{3.114465in}}%
\pgfpathlineto{\pgfqpoint{2.402186in}{3.125641in}}%
\pgfpathlineto{\pgfqpoint{2.411553in}{3.138633in}}%
\pgfpathlineto{\pgfqpoint{2.416370in}{3.141750in}}%
\pgfpathlineto{\pgfqpoint{2.420116in}{3.141789in}}%
\pgfpathlineto{\pgfqpoint{2.424130in}{3.139489in}}%
\pgfpathlineto{\pgfqpoint{2.429215in}{3.133656in}}%
\pgfpathlineto{\pgfqpoint{2.447947in}{3.109137in}}%
\pgfpathlineto{\pgfqpoint{2.451961in}{3.108437in}}%
\pgfpathlineto{\pgfqpoint{2.455708in}{3.109999in}}%
\pgfpathlineto{\pgfqpoint{2.460257in}{3.114465in}}%
\pgfpathlineto{\pgfqpoint{2.467750in}{3.125641in}}%
\pgfpathlineto{\pgfqpoint{2.477117in}{3.138633in}}%
\pgfpathlineto{\pgfqpoint{2.481934in}{3.141750in}}%
\pgfpathlineto{\pgfqpoint{2.485680in}{3.141789in}}%
\pgfpathlineto{\pgfqpoint{2.489694in}{3.139489in}}%
\pgfpathlineto{\pgfqpoint{2.494779in}{3.133656in}}%
\pgfpathlineto{\pgfqpoint{2.513511in}{3.109137in}}%
\pgfpathlineto{\pgfqpoint{2.517525in}{3.108437in}}%
\pgfpathlineto{\pgfqpoint{2.521272in}{3.109999in}}%
\pgfpathlineto{\pgfqpoint{2.525821in}{3.114465in}}%
\pgfpathlineto{\pgfqpoint{2.533314in}{3.125641in}}%
\pgfpathlineto{\pgfqpoint{2.542681in}{3.138633in}}%
\pgfpathlineto{\pgfqpoint{2.547497in}{3.141750in}}%
\pgfpathlineto{\pgfqpoint{2.551244in}{3.141789in}}%
\pgfpathlineto{\pgfqpoint{2.555258in}{3.139489in}}%
\pgfpathlineto{\pgfqpoint{2.560343in}{3.133656in}}%
\pgfpathlineto{\pgfqpoint{2.579075in}{3.109137in}}%
\pgfpathlineto{\pgfqpoint{2.583089in}{3.108437in}}%
\pgfpathlineto{\pgfqpoint{2.586836in}{3.109999in}}%
\pgfpathlineto{\pgfqpoint{2.591385in}{3.114465in}}%
\pgfpathlineto{\pgfqpoint{2.598878in}{3.125641in}}%
\pgfpathlineto{\pgfqpoint{2.608244in}{3.138633in}}%
\pgfpathlineto{\pgfqpoint{2.613061in}{3.141750in}}%
\pgfpathlineto{\pgfqpoint{2.616808in}{3.141789in}}%
\pgfpathlineto{\pgfqpoint{2.620822in}{3.139489in}}%
\pgfpathlineto{\pgfqpoint{2.625907in}{3.133656in}}%
\pgfpathlineto{\pgfqpoint{2.644639in}{3.109137in}}%
\pgfpathlineto{\pgfqpoint{2.648653in}{3.108437in}}%
\pgfpathlineto{\pgfqpoint{2.652400in}{3.109999in}}%
\pgfpathlineto{\pgfqpoint{2.656949in}{3.114465in}}%
\pgfpathlineto{\pgfqpoint{2.664442in}{3.125641in}}%
\pgfpathlineto{\pgfqpoint{2.673808in}{3.138633in}}%
\pgfpathlineto{\pgfqpoint{2.678625in}{3.141750in}}%
\pgfpathlineto{\pgfqpoint{2.682372in}{3.141789in}}%
\pgfpathlineto{\pgfqpoint{2.686386in}{3.139489in}}%
\pgfpathlineto{\pgfqpoint{2.691470in}{3.133656in}}%
\pgfpathlineto{\pgfqpoint{2.710203in}{3.109137in}}%
\pgfpathlineto{\pgfqpoint{2.714217in}{3.108437in}}%
\pgfpathlineto{\pgfqpoint{2.717964in}{3.109999in}}%
\pgfpathlineto{\pgfqpoint{2.722513in}{3.114465in}}%
\pgfpathlineto{\pgfqpoint{2.730006in}{3.125641in}}%
\pgfpathlineto{\pgfqpoint{2.739372in}{3.138633in}}%
\pgfpathlineto{\pgfqpoint{2.744189in}{3.141750in}}%
\pgfpathlineto{\pgfqpoint{2.747936in}{3.141789in}}%
\pgfpathlineto{\pgfqpoint{2.751950in}{3.139489in}}%
\pgfpathlineto{\pgfqpoint{2.757034in}{3.133656in}}%
\pgfpathlineto{\pgfqpoint{2.775767in}{3.109137in}}%
\pgfpathlineto{\pgfqpoint{2.779781in}{3.108437in}}%
\pgfpathlineto{\pgfqpoint{2.783528in}{3.109999in}}%
\pgfpathlineto{\pgfqpoint{2.788077in}{3.114465in}}%
\pgfpathlineto{\pgfqpoint{2.795570in}{3.125641in}}%
\pgfpathlineto{\pgfqpoint{2.804936in}{3.138633in}}%
\pgfpathlineto{\pgfqpoint{2.809753in}{3.141750in}}%
\pgfpathlineto{\pgfqpoint{2.813500in}{3.141789in}}%
\pgfpathlineto{\pgfqpoint{2.817514in}{3.139489in}}%
\pgfpathlineto{\pgfqpoint{2.822598in}{3.133656in}}%
\pgfpathlineto{\pgfqpoint{2.841331in}{3.109137in}}%
\pgfpathlineto{\pgfqpoint{2.845345in}{3.108437in}}%
\pgfpathlineto{\pgfqpoint{2.849091in}{3.109999in}}%
\pgfpathlineto{\pgfqpoint{2.853641in}{3.114465in}}%
\pgfpathlineto{\pgfqpoint{2.861134in}{3.125641in}}%
\pgfpathlineto{\pgfqpoint{2.870500in}{3.138633in}}%
\pgfpathlineto{\pgfqpoint{2.875317in}{3.141750in}}%
\pgfpathlineto{\pgfqpoint{2.879064in}{3.141789in}}%
\pgfpathlineto{\pgfqpoint{2.883078in}{3.139489in}}%
\pgfpathlineto{\pgfqpoint{2.888162in}{3.133656in}}%
\pgfpathlineto{\pgfqpoint{2.906895in}{3.109137in}}%
\pgfpathlineto{\pgfqpoint{2.910909in}{3.108437in}}%
\pgfpathlineto{\pgfqpoint{2.914655in}{3.109999in}}%
\pgfpathlineto{\pgfqpoint{2.919205in}{3.114465in}}%
\pgfpathlineto{\pgfqpoint{2.926698in}{3.125641in}}%
\pgfpathlineto{\pgfqpoint{2.936064in}{3.138633in}}%
\pgfpathlineto{\pgfqpoint{2.940881in}{3.141750in}}%
\pgfpathlineto{\pgfqpoint{2.944627in}{3.141789in}}%
\pgfpathlineto{\pgfqpoint{2.948642in}{3.139489in}}%
\pgfpathlineto{\pgfqpoint{2.953726in}{3.133656in}}%
\pgfpathlineto{\pgfqpoint{2.972459in}{3.109137in}}%
\pgfpathlineto{\pgfqpoint{2.976473in}{3.108437in}}%
\pgfpathlineto{\pgfqpoint{2.980219in}{3.109999in}}%
\pgfpathlineto{\pgfqpoint{2.984769in}{3.114465in}}%
\pgfpathlineto{\pgfqpoint{2.992262in}{3.125641in}}%
\pgfpathlineto{\pgfqpoint{3.001628in}{3.138633in}}%
\pgfpathlineto{\pgfqpoint{3.006445in}{3.141750in}}%
\pgfpathlineto{\pgfqpoint{3.010191in}{3.141789in}}%
\pgfpathlineto{\pgfqpoint{3.014206in}{3.139489in}}%
\pgfpathlineto{\pgfqpoint{3.019290in}{3.133656in}}%
\pgfpathlineto{\pgfqpoint{3.038023in}{3.109137in}}%
\pgfpathlineto{\pgfqpoint{3.042037in}{3.108437in}}%
\pgfpathlineto{\pgfqpoint{3.045783in}{3.109999in}}%
\pgfpathlineto{\pgfqpoint{3.050333in}{3.114465in}}%
\pgfpathlineto{\pgfqpoint{3.057826in}{3.125641in}}%
\pgfpathlineto{\pgfqpoint{3.067192in}{3.138633in}}%
\pgfpathlineto{\pgfqpoint{3.072009in}{3.141750in}}%
\pgfpathlineto{\pgfqpoint{3.075755in}{3.141789in}}%
\pgfpathlineto{\pgfqpoint{3.079769in}{3.139489in}}%
\pgfpathlineto{\pgfqpoint{3.084854in}{3.133656in}}%
\pgfpathlineto{\pgfqpoint{3.103587in}{3.109137in}}%
\pgfpathlineto{\pgfqpoint{3.107601in}{3.108437in}}%
\pgfpathlineto{\pgfqpoint{3.111347in}{3.109999in}}%
\pgfpathlineto{\pgfqpoint{3.115896in}{3.114465in}}%
\pgfpathlineto{\pgfqpoint{3.123389in}{3.125641in}}%
\pgfpathlineto{\pgfqpoint{3.132756in}{3.138633in}}%
\pgfpathlineto{\pgfqpoint{3.137573in}{3.141750in}}%
\pgfpathlineto{\pgfqpoint{3.141319in}{3.141789in}}%
\pgfpathlineto{\pgfqpoint{3.145333in}{3.139489in}}%
\pgfpathlineto{\pgfqpoint{3.150418in}{3.133656in}}%
\pgfpathlineto{\pgfqpoint{3.169150in}{3.109137in}}%
\pgfpathlineto{\pgfqpoint{3.173165in}{3.108437in}}%
\pgfpathlineto{\pgfqpoint{3.176911in}{3.109999in}}%
\pgfpathlineto{\pgfqpoint{3.181460in}{3.114465in}}%
\pgfpathlineto{\pgfqpoint{3.188953in}{3.125641in}}%
\pgfpathlineto{\pgfqpoint{3.198320in}{3.138633in}}%
\pgfpathlineto{\pgfqpoint{3.203137in}{3.141750in}}%
\pgfpathlineto{\pgfqpoint{3.206883in}{3.141789in}}%
\pgfpathlineto{\pgfqpoint{3.210897in}{3.139489in}}%
\pgfpathlineto{\pgfqpoint{3.215982in}{3.133656in}}%
\pgfpathlineto{\pgfqpoint{3.234714in}{3.109137in}}%
\pgfpathlineto{\pgfqpoint{3.238728in}{3.108437in}}%
\pgfpathlineto{\pgfqpoint{3.242475in}{3.109999in}}%
\pgfpathlineto{\pgfqpoint{3.247024in}{3.114465in}}%
\pgfpathlineto{\pgfqpoint{3.254517in}{3.125641in}}%
\pgfpathlineto{\pgfqpoint{3.263884in}{3.138633in}}%
\pgfpathlineto{\pgfqpoint{3.268701in}{3.141750in}}%
\pgfpathlineto{\pgfqpoint{3.272447in}{3.141789in}}%
\pgfpathlineto{\pgfqpoint{3.276461in}{3.139489in}}%
\pgfpathlineto{\pgfqpoint{3.281546in}{3.133656in}}%
\pgfpathlineto{\pgfqpoint{3.300278in}{3.109137in}}%
\pgfpathlineto{\pgfqpoint{3.304292in}{3.108437in}}%
\pgfpathlineto{\pgfqpoint{3.308039in}{3.109999in}}%
\pgfpathlineto{\pgfqpoint{3.312588in}{3.114465in}}%
\pgfpathlineto{\pgfqpoint{3.320081in}{3.125641in}}%
\pgfpathlineto{\pgfqpoint{3.329448in}{3.138633in}}%
\pgfpathlineto{\pgfqpoint{3.334264in}{3.141750in}}%
\pgfpathlineto{\pgfqpoint{3.338011in}{3.141789in}}%
\pgfpathlineto{\pgfqpoint{3.342025in}{3.139489in}}%
\pgfpathlineto{\pgfqpoint{3.347110in}{3.133656in}}%
\pgfpathlineto{\pgfqpoint{3.365842in}{3.109137in}}%
\pgfpathlineto{\pgfqpoint{3.369856in}{3.108437in}}%
\pgfpathlineto{\pgfqpoint{3.373603in}{3.109999in}}%
\pgfpathlineto{\pgfqpoint{3.378152in}{3.114465in}}%
\pgfpathlineto{\pgfqpoint{3.385645in}{3.125641in}}%
\pgfpathlineto{\pgfqpoint{3.395011in}{3.138633in}}%
\pgfpathlineto{\pgfqpoint{3.399828in}{3.141750in}}%
\pgfpathlineto{\pgfqpoint{3.403575in}{3.141789in}}%
\pgfpathlineto{\pgfqpoint{3.407589in}{3.139489in}}%
\pgfpathlineto{\pgfqpoint{3.412674in}{3.133656in}}%
\pgfpathlineto{\pgfqpoint{3.431406in}{3.109137in}}%
\pgfpathlineto{\pgfqpoint{3.435420in}{3.108437in}}%
\pgfpathlineto{\pgfqpoint{3.439167in}{3.109999in}}%
\pgfpathlineto{\pgfqpoint{3.443716in}{3.114465in}}%
\pgfpathlineto{\pgfqpoint{3.451209in}{3.125641in}}%
\pgfpathlineto{\pgfqpoint{3.460575in}{3.138633in}}%
\pgfpathlineto{\pgfqpoint{3.465392in}{3.141750in}}%
\pgfpathlineto{\pgfqpoint{3.469139in}{3.141789in}}%
\pgfpathlineto{\pgfqpoint{3.473153in}{3.139489in}}%
\pgfpathlineto{\pgfqpoint{3.478237in}{3.133656in}}%
\pgfpathlineto{\pgfqpoint{3.496970in}{3.109137in}}%
\pgfpathlineto{\pgfqpoint{3.500984in}{3.108437in}}%
\pgfpathlineto{\pgfqpoint{3.504731in}{3.109999in}}%
\pgfpathlineto{\pgfqpoint{3.509280in}{3.114465in}}%
\pgfpathlineto{\pgfqpoint{3.516773in}{3.125641in}}%
\pgfpathlineto{\pgfqpoint{3.526139in}{3.138633in}}%
\pgfpathlineto{\pgfqpoint{3.530956in}{3.141750in}}%
\pgfpathlineto{\pgfqpoint{3.534703in}{3.141789in}}%
\pgfpathlineto{\pgfqpoint{3.538717in}{3.139489in}}%
\pgfpathlineto{\pgfqpoint{3.543801in}{3.133656in}}%
\pgfpathlineto{\pgfqpoint{3.562534in}{3.109137in}}%
\pgfpathlineto{\pgfqpoint{3.566548in}{3.108437in}}%
\pgfpathlineto{\pgfqpoint{3.570295in}{3.109999in}}%
\pgfpathlineto{\pgfqpoint{3.574844in}{3.114465in}}%
\pgfpathlineto{\pgfqpoint{3.582337in}{3.125641in}}%
\pgfpathlineto{\pgfqpoint{3.591703in}{3.138633in}}%
\pgfpathlineto{\pgfqpoint{3.596520in}{3.141750in}}%
\pgfpathlineto{\pgfqpoint{3.600267in}{3.141789in}}%
\pgfpathlineto{\pgfqpoint{3.604281in}{3.139489in}}%
\pgfpathlineto{\pgfqpoint{3.609365in}{3.133656in}}%
\pgfpathlineto{\pgfqpoint{3.628098in}{3.109137in}}%
\pgfpathlineto{\pgfqpoint{3.632112in}{3.108437in}}%
\pgfpathlineto{\pgfqpoint{3.635858in}{3.109999in}}%
\pgfpathlineto{\pgfqpoint{3.640408in}{3.114465in}}%
\pgfpathlineto{\pgfqpoint{3.647901in}{3.125641in}}%
\pgfpathlineto{\pgfqpoint{3.657267in}{3.138633in}}%
\pgfpathlineto{\pgfqpoint{3.662084in}{3.141750in}}%
\pgfpathlineto{\pgfqpoint{3.665831in}{3.141789in}}%
\pgfpathlineto{\pgfqpoint{3.669845in}{3.139489in}}%
\pgfpathlineto{\pgfqpoint{3.674929in}{3.133656in}}%
\pgfpathlineto{\pgfqpoint{3.693662in}{3.109137in}}%
\pgfpathlineto{\pgfqpoint{3.697676in}{3.108437in}}%
\pgfpathlineto{\pgfqpoint{3.701422in}{3.109999in}}%
\pgfpathlineto{\pgfqpoint{3.705972in}{3.114465in}}%
\pgfpathlineto{\pgfqpoint{3.713465in}{3.125641in}}%
\pgfpathlineto{\pgfqpoint{3.722831in}{3.138633in}}%
\pgfpathlineto{\pgfqpoint{3.727648in}{3.141750in}}%
\pgfpathlineto{\pgfqpoint{3.731394in}{3.141789in}}%
\pgfpathlineto{\pgfqpoint{3.735409in}{3.139489in}}%
\pgfpathlineto{\pgfqpoint{3.740493in}{3.133656in}}%
\pgfpathlineto{\pgfqpoint{3.759226in}{3.109137in}}%
\pgfpathlineto{\pgfqpoint{3.763240in}{3.108437in}}%
\pgfpathlineto{\pgfqpoint{3.766986in}{3.109999in}}%
\pgfpathlineto{\pgfqpoint{3.771536in}{3.114465in}}%
\pgfpathlineto{\pgfqpoint{3.779029in}{3.125641in}}%
\pgfpathlineto{\pgfqpoint{3.788395in}{3.138633in}}%
\pgfpathlineto{\pgfqpoint{3.793212in}{3.141750in}}%
\pgfpathlineto{\pgfqpoint{3.796958in}{3.141789in}}%
\pgfpathlineto{\pgfqpoint{3.800973in}{3.139489in}}%
\pgfpathlineto{\pgfqpoint{3.806057in}{3.133656in}}%
\pgfpathlineto{\pgfqpoint{3.824790in}{3.109137in}}%
\pgfpathlineto{\pgfqpoint{3.828804in}{3.108437in}}%
\pgfpathlineto{\pgfqpoint{3.832550in}{3.109999in}}%
\pgfpathlineto{\pgfqpoint{3.837100in}{3.114465in}}%
\pgfpathlineto{\pgfqpoint{3.844593in}{3.125641in}}%
\pgfpathlineto{\pgfqpoint{3.853959in}{3.138633in}}%
\pgfpathlineto{\pgfqpoint{3.858776in}{3.141750in}}%
\pgfpathlineto{\pgfqpoint{3.862522in}{3.141789in}}%
\pgfpathlineto{\pgfqpoint{3.866536in}{3.139489in}}%
\pgfpathlineto{\pgfqpoint{3.871621in}{3.133656in}}%
\pgfpathlineto{\pgfqpoint{3.890354in}{3.109137in}}%
\pgfpathlineto{\pgfqpoint{3.894368in}{3.108437in}}%
\pgfpathlineto{\pgfqpoint{3.898114in}{3.109999in}}%
\pgfpathlineto{\pgfqpoint{3.902663in}{3.114465in}}%
\pgfpathlineto{\pgfqpoint{3.910156in}{3.125641in}}%
\pgfpathlineto{\pgfqpoint{3.919523in}{3.138633in}}%
\pgfpathlineto{\pgfqpoint{3.924340in}{3.141750in}}%
\pgfpathlineto{\pgfqpoint{3.928086in}{3.141789in}}%
\pgfpathlineto{\pgfqpoint{3.932100in}{3.139489in}}%
\pgfpathlineto{\pgfqpoint{3.937185in}{3.133656in}}%
\pgfpathlineto{\pgfqpoint{3.955917in}{3.109137in}}%
\pgfpathlineto{\pgfqpoint{3.959932in}{3.108437in}}%
\pgfpathlineto{\pgfqpoint{3.963678in}{3.109999in}}%
\pgfpathlineto{\pgfqpoint{3.968227in}{3.114465in}}%
\pgfpathlineto{\pgfqpoint{3.975720in}{3.125641in}}%
\pgfpathlineto{\pgfqpoint{3.985087in}{3.138633in}}%
\pgfpathlineto{\pgfqpoint{3.989904in}{3.141750in}}%
\pgfpathlineto{\pgfqpoint{3.993650in}{3.141789in}}%
\pgfpathlineto{\pgfqpoint{3.997664in}{3.139489in}}%
\pgfpathlineto{\pgfqpoint{4.002749in}{3.133656in}}%
\pgfpathlineto{\pgfqpoint{4.021481in}{3.109137in}}%
\pgfpathlineto{\pgfqpoint{4.025495in}{3.108437in}}%
\pgfpathlineto{\pgfqpoint{4.029242in}{3.109999in}}%
\pgfpathlineto{\pgfqpoint{4.033791in}{3.114465in}}%
\pgfpathlineto{\pgfqpoint{4.041284in}{3.125641in}}%
\pgfpathlineto{\pgfqpoint{4.050651in}{3.138633in}}%
\pgfpathlineto{\pgfqpoint{4.055468in}{3.141750in}}%
\pgfpathlineto{\pgfqpoint{4.059214in}{3.141789in}}%
\pgfpathlineto{\pgfqpoint{4.063228in}{3.139489in}}%
\pgfpathlineto{\pgfqpoint{4.068313in}{3.133656in}}%
\pgfpathlineto{\pgfqpoint{4.087045in}{3.109137in}}%
\pgfpathlineto{\pgfqpoint{4.091059in}{3.108437in}}%
\pgfpathlineto{\pgfqpoint{4.094806in}{3.109999in}}%
\pgfpathlineto{\pgfqpoint{4.099355in}{3.114465in}}%
\pgfpathlineto{\pgfqpoint{4.106848in}{3.125641in}}%
\pgfpathlineto{\pgfqpoint{4.116215in}{3.138633in}}%
\pgfpathlineto{\pgfqpoint{4.121031in}{3.141750in}}%
\pgfpathlineto{\pgfqpoint{4.124778in}{3.141789in}}%
\pgfpathlineto{\pgfqpoint{4.128792in}{3.139489in}}%
\pgfpathlineto{\pgfqpoint{4.133877in}{3.133656in}}%
\pgfpathlineto{\pgfqpoint{4.152609in}{3.109137in}}%
\pgfpathlineto{\pgfqpoint{4.156623in}{3.108437in}}%
\pgfpathlineto{\pgfqpoint{4.160370in}{3.109999in}}%
\pgfpathlineto{\pgfqpoint{4.164919in}{3.114465in}}%
\pgfpathlineto{\pgfqpoint{4.172412in}{3.125641in}}%
\pgfpathlineto{\pgfqpoint{4.181778in}{3.138633in}}%
\pgfpathlineto{\pgfqpoint{4.186595in}{3.141750in}}%
\pgfpathlineto{\pgfqpoint{4.190342in}{3.141789in}}%
\pgfpathlineto{\pgfqpoint{4.194356in}{3.139489in}}%
\pgfpathlineto{\pgfqpoint{4.199441in}{3.133656in}}%
\pgfpathlineto{\pgfqpoint{4.218173in}{3.109137in}}%
\pgfpathlineto{\pgfqpoint{4.222187in}{3.108437in}}%
\pgfpathlineto{\pgfqpoint{4.225934in}{3.109999in}}%
\pgfpathlineto{\pgfqpoint{4.230483in}{3.114465in}}%
\pgfpathlineto{\pgfqpoint{4.237976in}{3.125641in}}%
\pgfpathlineto{\pgfqpoint{4.247342in}{3.138633in}}%
\pgfpathlineto{\pgfqpoint{4.252159in}{3.141750in}}%
\pgfpathlineto{\pgfqpoint{4.255906in}{3.141789in}}%
\pgfpathlineto{\pgfqpoint{4.259920in}{3.139489in}}%
\pgfpathlineto{\pgfqpoint{4.265004in}{3.133656in}}%
\pgfpathlineto{\pgfqpoint{4.283737in}{3.109137in}}%
\pgfpathlineto{\pgfqpoint{4.287751in}{3.108437in}}%
\pgfpathlineto{\pgfqpoint{4.291498in}{3.109999in}}%
\pgfpathlineto{\pgfqpoint{4.296047in}{3.114465in}}%
\pgfpathlineto{\pgfqpoint{4.303540in}{3.125641in}}%
\pgfpathlineto{\pgfqpoint{4.312906in}{3.138633in}}%
\pgfpathlineto{\pgfqpoint{4.317723in}{3.141750in}}%
\pgfpathlineto{\pgfqpoint{4.321470in}{3.141789in}}%
\pgfpathlineto{\pgfqpoint{4.325484in}{3.139489in}}%
\pgfpathlineto{\pgfqpoint{4.330568in}{3.133656in}}%
\pgfpathlineto{\pgfqpoint{4.349301in}{3.109137in}}%
\pgfpathlineto{\pgfqpoint{4.353315in}{3.108437in}}%
\pgfpathlineto{\pgfqpoint{4.357062in}{3.109999in}}%
\pgfpathlineto{\pgfqpoint{4.361611in}{3.114465in}}%
\pgfpathlineto{\pgfqpoint{4.369104in}{3.125641in}}%
\pgfpathlineto{\pgfqpoint{4.378470in}{3.138633in}}%
\pgfpathlineto{\pgfqpoint{4.383287in}{3.141750in}}%
\pgfpathlineto{\pgfqpoint{4.387034in}{3.141789in}}%
\pgfpathlineto{\pgfqpoint{4.391048in}{3.139489in}}%
\pgfpathlineto{\pgfqpoint{4.396132in}{3.133656in}}%
\pgfpathlineto{\pgfqpoint{4.414865in}{3.109137in}}%
\pgfpathlineto{\pgfqpoint{4.418879in}{3.108437in}}%
\pgfpathlineto{\pgfqpoint{4.422625in}{3.109999in}}%
\pgfpathlineto{\pgfqpoint{4.427175in}{3.114465in}}%
\pgfpathlineto{\pgfqpoint{4.434668in}{3.125641in}}%
\pgfpathlineto{\pgfqpoint{4.444034in}{3.138633in}}%
\pgfpathlineto{\pgfqpoint{4.448851in}{3.141750in}}%
\pgfpathlineto{\pgfqpoint{4.452598in}{3.141789in}}%
\pgfpathlineto{\pgfqpoint{4.456612in}{3.139489in}}%
\pgfpathlineto{\pgfqpoint{4.461696in}{3.133656in}}%
\pgfpathlineto{\pgfqpoint{4.480429in}{3.109137in}}%
\pgfpathlineto{\pgfqpoint{4.484443in}{3.108437in}}%
\pgfpathlineto{\pgfqpoint{4.488189in}{3.109999in}}%
\pgfpathlineto{\pgfqpoint{4.492739in}{3.114465in}}%
\pgfpathlineto{\pgfqpoint{4.500232in}{3.125641in}}%
\pgfpathlineto{\pgfqpoint{4.509598in}{3.138633in}}%
\pgfpathlineto{\pgfqpoint{4.514415in}{3.141750in}}%
\pgfpathlineto{\pgfqpoint{4.518161in}{3.141789in}}%
\pgfpathlineto{\pgfqpoint{4.522176in}{3.139489in}}%
\pgfpathlineto{\pgfqpoint{4.527260in}{3.133656in}}%
\pgfpathlineto{\pgfqpoint{4.545993in}{3.109137in}}%
\pgfpathlineto{\pgfqpoint{4.550007in}{3.108437in}}%
\pgfpathlineto{\pgfqpoint{4.553753in}{3.109999in}}%
\pgfpathlineto{\pgfqpoint{4.558303in}{3.114465in}}%
\pgfpathlineto{\pgfqpoint{4.565796in}{3.125641in}}%
\pgfpathlineto{\pgfqpoint{4.575162in}{3.138633in}}%
\pgfpathlineto{\pgfqpoint{4.579979in}{3.141750in}}%
\pgfpathlineto{\pgfqpoint{4.583725in}{3.141789in}}%
\pgfpathlineto{\pgfqpoint{4.587739in}{3.139489in}}%
\pgfpathlineto{\pgfqpoint{4.592824in}{3.133656in}}%
\pgfpathlineto{\pgfqpoint{4.611557in}{3.109137in}}%
\pgfpathlineto{\pgfqpoint{4.615571in}{3.108437in}}%
\pgfpathlineto{\pgfqpoint{4.619317in}{3.109999in}}%
\pgfpathlineto{\pgfqpoint{4.623867in}{3.114465in}}%
\pgfpathlineto{\pgfqpoint{4.631360in}{3.125641in}}%
\pgfpathlineto{\pgfqpoint{4.640726in}{3.138633in}}%
\pgfpathlineto{\pgfqpoint{4.645543in}{3.141750in}}%
\pgfpathlineto{\pgfqpoint{4.649289in}{3.141789in}}%
\pgfpathlineto{\pgfqpoint{4.653303in}{3.139489in}}%
\pgfpathlineto{\pgfqpoint{4.658388in}{3.133656in}}%
\pgfpathlineto{\pgfqpoint{4.677121in}{3.109137in}}%
\pgfpathlineto{\pgfqpoint{4.681135in}{3.108437in}}%
\pgfpathlineto{\pgfqpoint{4.684881in}{3.109999in}}%
\pgfpathlineto{\pgfqpoint{4.689430in}{3.114465in}}%
\pgfpathlineto{\pgfqpoint{4.696923in}{3.125641in}}%
\pgfpathlineto{\pgfqpoint{4.706290in}{3.138633in}}%
\pgfpathlineto{\pgfqpoint{4.711107in}{3.141750in}}%
\pgfpathlineto{\pgfqpoint{4.714853in}{3.141789in}}%
\pgfpathlineto{\pgfqpoint{4.718867in}{3.139489in}}%
\pgfpathlineto{\pgfqpoint{4.723952in}{3.133656in}}%
\pgfpathlineto{\pgfqpoint{4.742684in}{3.109137in}}%
\pgfpathlineto{\pgfqpoint{4.746699in}{3.108437in}}%
\pgfpathlineto{\pgfqpoint{4.750445in}{3.109999in}}%
\pgfpathlineto{\pgfqpoint{4.754994in}{3.114465in}}%
\pgfpathlineto{\pgfqpoint{4.762487in}{3.125641in}}%
\pgfpathlineto{\pgfqpoint{4.771854in}{3.138633in}}%
\pgfpathlineto{\pgfqpoint{4.776671in}{3.141750in}}%
\pgfpathlineto{\pgfqpoint{4.780417in}{3.141789in}}%
\pgfpathlineto{\pgfqpoint{4.784431in}{3.139489in}}%
\pgfpathlineto{\pgfqpoint{4.789516in}{3.133656in}}%
\pgfpathlineto{\pgfqpoint{4.808248in}{3.109137in}}%
\pgfpathlineto{\pgfqpoint{4.812262in}{3.108437in}}%
\pgfpathlineto{\pgfqpoint{4.816009in}{3.109999in}}%
\pgfpathlineto{\pgfqpoint{4.820558in}{3.114465in}}%
\pgfpathlineto{\pgfqpoint{4.828051in}{3.125641in}}%
\pgfpathlineto{\pgfqpoint{4.837418in}{3.138633in}}%
\pgfpathlineto{\pgfqpoint{4.842235in}{3.141750in}}%
\pgfpathlineto{\pgfqpoint{4.845981in}{3.141789in}}%
\pgfpathlineto{\pgfqpoint{4.849995in}{3.139489in}}%
\pgfpathlineto{\pgfqpoint{4.855080in}{3.133656in}}%
\pgfpathlineto{\pgfqpoint{4.873812in}{3.109137in}}%
\pgfpathlineto{\pgfqpoint{4.877826in}{3.108437in}}%
\pgfpathlineto{\pgfqpoint{4.881573in}{3.109999in}}%
\pgfpathlineto{\pgfqpoint{4.886122in}{3.114465in}}%
\pgfpathlineto{\pgfqpoint{4.893615in}{3.125641in}}%
\pgfpathlineto{\pgfqpoint{4.902982in}{3.138633in}}%
\pgfpathlineto{\pgfqpoint{4.907798in}{3.141750in}}%
\pgfpathlineto{\pgfqpoint{4.911545in}{3.141789in}}%
\pgfpathlineto{\pgfqpoint{4.915559in}{3.139489in}}%
\pgfpathlineto{\pgfqpoint{4.920644in}{3.133656in}}%
\pgfpathlineto{\pgfqpoint{4.939376in}{3.109137in}}%
\pgfpathlineto{\pgfqpoint{4.943390in}{3.108437in}}%
\pgfpathlineto{\pgfqpoint{4.947137in}{3.109999in}}%
\pgfpathlineto{\pgfqpoint{4.951686in}{3.114465in}}%
\pgfpathlineto{\pgfqpoint{4.959179in}{3.125641in}}%
\pgfpathlineto{\pgfqpoint{4.968545in}{3.138633in}}%
\pgfpathlineto{\pgfqpoint{4.973362in}{3.141750in}}%
\pgfpathlineto{\pgfqpoint{4.977109in}{3.141789in}}%
\pgfpathlineto{\pgfqpoint{4.981123in}{3.139489in}}%
\pgfpathlineto{\pgfqpoint{4.986208in}{3.133656in}}%
\pgfpathlineto{\pgfqpoint{5.004940in}{3.109137in}}%
\pgfpathlineto{\pgfqpoint{5.008954in}{3.108437in}}%
\pgfpathlineto{\pgfqpoint{5.012701in}{3.109999in}}%
\pgfpathlineto{\pgfqpoint{5.017250in}{3.114465in}}%
\pgfpathlineto{\pgfqpoint{5.024743in}{3.125641in}}%
\pgfpathlineto{\pgfqpoint{5.034109in}{3.138633in}}%
\pgfpathlineto{\pgfqpoint{5.038926in}{3.141750in}}%
\pgfpathlineto{\pgfqpoint{5.042673in}{3.141789in}}%
\pgfpathlineto{\pgfqpoint{5.046687in}{3.139489in}}%
\pgfpathlineto{\pgfqpoint{5.051771in}{3.133656in}}%
\pgfpathlineto{\pgfqpoint{5.070504in}{3.109137in}}%
\pgfpathlineto{\pgfqpoint{5.074518in}{3.108437in}}%
\pgfpathlineto{\pgfqpoint{5.078265in}{3.109999in}}%
\pgfpathlineto{\pgfqpoint{5.082814in}{3.114465in}}%
\pgfpathlineto{\pgfqpoint{5.090307in}{3.125641in}}%
\pgfpathlineto{\pgfqpoint{5.099673in}{3.138633in}}%
\pgfpathlineto{\pgfqpoint{5.104490in}{3.141750in}}%
\pgfpathlineto{\pgfqpoint{5.108237in}{3.141789in}}%
\pgfpathlineto{\pgfqpoint{5.112251in}{3.139489in}}%
\pgfpathlineto{\pgfqpoint{5.117335in}{3.133656in}}%
\pgfpathlineto{\pgfqpoint{5.136068in}{3.109137in}}%
\pgfpathlineto{\pgfqpoint{5.140082in}{3.108437in}}%
\pgfpathlineto{\pgfqpoint{5.143829in}{3.109999in}}%
\pgfpathlineto{\pgfqpoint{5.148378in}{3.114465in}}%
\pgfpathlineto{\pgfqpoint{5.155871in}{3.125641in}}%
\pgfpathlineto{\pgfqpoint{5.165237in}{3.138633in}}%
\pgfpathlineto{\pgfqpoint{5.170054in}{3.141750in}}%
\pgfpathlineto{\pgfqpoint{5.173801in}{3.141789in}}%
\pgfpathlineto{\pgfqpoint{5.177815in}{3.139489in}}%
\pgfpathlineto{\pgfqpoint{5.182899in}{3.133656in}}%
\pgfpathlineto{\pgfqpoint{5.201632in}{3.109137in}}%
\pgfpathlineto{\pgfqpoint{5.205646in}{3.108437in}}%
\pgfpathlineto{\pgfqpoint{5.209392in}{3.109999in}}%
\pgfpathlineto{\pgfqpoint{5.213942in}{3.114465in}}%
\pgfpathlineto{\pgfqpoint{5.221435in}{3.125641in}}%
\pgfpathlineto{\pgfqpoint{5.230801in}{3.138633in}}%
\pgfpathlineto{\pgfqpoint{5.235618in}{3.141750in}}%
\pgfpathlineto{\pgfqpoint{5.239365in}{3.141789in}}%
\pgfpathlineto{\pgfqpoint{5.243379in}{3.139489in}}%
\pgfpathlineto{\pgfqpoint{5.248463in}{3.133656in}}%
\pgfpathlineto{\pgfqpoint{5.267196in}{3.109137in}}%
\pgfpathlineto{\pgfqpoint{5.271210in}{3.108437in}}%
\pgfpathlineto{\pgfqpoint{5.274956in}{3.109999in}}%
\pgfpathlineto{\pgfqpoint{5.279506in}{3.114465in}}%
\pgfpathlineto{\pgfqpoint{5.286999in}{3.125641in}}%
\pgfpathlineto{\pgfqpoint{5.296365in}{3.138633in}}%
\pgfpathlineto{\pgfqpoint{5.301182in}{3.141750in}}%
\pgfpathlineto{\pgfqpoint{5.304928in}{3.141789in}}%
\pgfpathlineto{\pgfqpoint{5.308943in}{3.139489in}}%
\pgfpathlineto{\pgfqpoint{5.314027in}{3.133656in}}%
\pgfpathlineto{\pgfqpoint{5.332760in}{3.109137in}}%
\pgfpathlineto{\pgfqpoint{5.336774in}{3.108437in}}%
\pgfpathlineto{\pgfqpoint{5.340520in}{3.109999in}}%
\pgfpathlineto{\pgfqpoint{5.345070in}{3.114465in}}%
\pgfpathlineto{\pgfqpoint{5.352563in}{3.125641in}}%
\pgfpathlineto{\pgfqpoint{5.361929in}{3.138633in}}%
\pgfpathlineto{\pgfqpoint{5.366746in}{3.141750in}}%
\pgfpathlineto{\pgfqpoint{5.370492in}{3.141789in}}%
\pgfpathlineto{\pgfqpoint{5.374506in}{3.139489in}}%
\pgfpathlineto{\pgfqpoint{5.379591in}{3.133656in}}%
\pgfpathlineto{\pgfqpoint{5.398324in}{3.109137in}}%
\pgfpathlineto{\pgfqpoint{5.402338in}{3.108437in}}%
\pgfpathlineto{\pgfqpoint{5.406084in}{3.109999in}}%
\pgfpathlineto{\pgfqpoint{5.410634in}{3.114465in}}%
\pgfpathlineto{\pgfqpoint{5.418127in}{3.125641in}}%
\pgfpathlineto{\pgfqpoint{5.427493in}{3.138633in}}%
\pgfpathlineto{\pgfqpoint{5.432310in}{3.141750in}}%
\pgfpathlineto{\pgfqpoint{5.436056in}{3.141789in}}%
\pgfpathlineto{\pgfqpoint{5.440070in}{3.139489in}}%
\pgfpathlineto{\pgfqpoint{5.445155in}{3.133656in}}%
\pgfpathlineto{\pgfqpoint{5.463888in}{3.109137in}}%
\pgfpathlineto{\pgfqpoint{5.467902in}{3.108437in}}%
\pgfpathlineto{\pgfqpoint{5.471648in}{3.109999in}}%
\pgfpathlineto{\pgfqpoint{5.476197in}{3.114465in}}%
\pgfpathlineto{\pgfqpoint{5.483690in}{3.125641in}}%
\pgfpathlineto{\pgfqpoint{5.493057in}{3.138633in}}%
\pgfpathlineto{\pgfqpoint{5.497874in}{3.141750in}}%
\pgfpathlineto{\pgfqpoint{5.501620in}{3.141789in}}%
\pgfpathlineto{\pgfqpoint{5.505634in}{3.139489in}}%
\pgfpathlineto{\pgfqpoint{5.510719in}{3.133656in}}%
\pgfpathlineto{\pgfqpoint{5.529451in}{3.109137in}}%
\pgfpathlineto{\pgfqpoint{5.533466in}{3.108437in}}%
\pgfpathlineto{\pgfqpoint{5.537212in}{3.109999in}}%
\pgfpathlineto{\pgfqpoint{5.541761in}{3.114465in}}%
\pgfpathlineto{\pgfqpoint{5.549254in}{3.125641in}}%
\pgfpathlineto{\pgfqpoint{5.558621in}{3.138633in}}%
\pgfpathlineto{\pgfqpoint{5.563438in}{3.141750in}}%
\pgfpathlineto{\pgfqpoint{5.567184in}{3.141789in}}%
\pgfpathlineto{\pgfqpoint{5.571198in}{3.139489in}}%
\pgfpathlineto{\pgfqpoint{5.576283in}{3.133656in}}%
\pgfpathlineto{\pgfqpoint{5.595015in}{3.109137in}}%
\pgfpathlineto{\pgfqpoint{5.599029in}{3.108437in}}%
\pgfpathlineto{\pgfqpoint{5.602776in}{3.109999in}}%
\pgfpathlineto{\pgfqpoint{5.607325in}{3.114465in}}%
\pgfpathlineto{\pgfqpoint{5.614818in}{3.125641in}}%
\pgfpathlineto{\pgfqpoint{5.624185in}{3.138633in}}%
\pgfpathlineto{\pgfqpoint{5.629002in}{3.141750in}}%
\pgfpathlineto{\pgfqpoint{5.632748in}{3.141789in}}%
\pgfpathlineto{\pgfqpoint{5.636762in}{3.139489in}}%
\pgfpathlineto{\pgfqpoint{5.641847in}{3.133656in}}%
\pgfpathlineto{\pgfqpoint{5.660579in}{3.109137in}}%
\pgfpathlineto{\pgfqpoint{5.664593in}{3.108437in}}%
\pgfpathlineto{\pgfqpoint{5.668340in}{3.109999in}}%
\pgfpathlineto{\pgfqpoint{5.672889in}{3.114465in}}%
\pgfpathlineto{\pgfqpoint{5.680382in}{3.125641in}}%
\pgfpathlineto{\pgfqpoint{5.689749in}{3.138633in}}%
\pgfpathlineto{\pgfqpoint{5.694565in}{3.141750in}}%
\pgfpathlineto{\pgfqpoint{5.698312in}{3.141789in}}%
\pgfpathlineto{\pgfqpoint{5.702326in}{3.139489in}}%
\pgfpathlineto{\pgfqpoint{5.707411in}{3.133656in}}%
\pgfpathlineto{\pgfqpoint{5.726143in}{3.109137in}}%
\pgfpathlineto{\pgfqpoint{5.730157in}{3.108437in}}%
\pgfpathlineto{\pgfqpoint{5.733904in}{3.109999in}}%
\pgfpathlineto{\pgfqpoint{5.738453in}{3.114465in}}%
\pgfpathlineto{\pgfqpoint{5.745946in}{3.125641in}}%
\pgfpathlineto{\pgfqpoint{5.755312in}{3.138633in}}%
\pgfpathlineto{\pgfqpoint{5.760129in}{3.141750in}}%
\pgfpathlineto{\pgfqpoint{5.763876in}{3.141789in}}%
\pgfpathlineto{\pgfqpoint{5.767890in}{3.139489in}}%
\pgfpathlineto{\pgfqpoint{5.772975in}{3.133656in}}%
\pgfpathlineto{\pgfqpoint{5.791707in}{3.109137in}}%
\pgfpathlineto{\pgfqpoint{5.795721in}{3.108437in}}%
\pgfpathlineto{\pgfqpoint{5.799468in}{3.109999in}}%
\pgfpathlineto{\pgfqpoint{5.804017in}{3.114465in}}%
\pgfpathlineto{\pgfqpoint{5.811510in}{3.125641in}}%
\pgfpathlineto{\pgfqpoint{5.820876in}{3.138633in}}%
\pgfpathlineto{\pgfqpoint{5.825693in}{3.141750in}}%
\pgfpathlineto{\pgfqpoint{5.829440in}{3.141789in}}%
\pgfpathlineto{\pgfqpoint{5.833454in}{3.139489in}}%
\pgfpathlineto{\pgfqpoint{5.838538in}{3.133656in}}%
\pgfpathlineto{\pgfqpoint{5.857271in}{3.109137in}}%
\pgfpathlineto{\pgfqpoint{5.861285in}{3.108437in}}%
\pgfpathlineto{\pgfqpoint{5.865032in}{3.109999in}}%
\pgfpathlineto{\pgfqpoint{5.869581in}{3.114465in}}%
\pgfpathlineto{\pgfqpoint{5.877074in}{3.125641in}}%
\pgfpathlineto{\pgfqpoint{5.886440in}{3.138633in}}%
\pgfpathlineto{\pgfqpoint{5.891257in}{3.141750in}}%
\pgfpathlineto{\pgfqpoint{5.895004in}{3.141789in}}%
\pgfpathlineto{\pgfqpoint{5.899018in}{3.139489in}}%
\pgfpathlineto{\pgfqpoint{5.904102in}{3.133656in}}%
\pgfpathlineto{\pgfqpoint{5.922835in}{3.109137in}}%
\pgfpathlineto{\pgfqpoint{5.926849in}{3.108437in}}%
\pgfpathlineto{\pgfqpoint{5.930596in}{3.109999in}}%
\pgfpathlineto{\pgfqpoint{5.935145in}{3.114465in}}%
\pgfpathlineto{\pgfqpoint{5.942638in}{3.125641in}}%
\pgfpathlineto{\pgfqpoint{5.952004in}{3.138633in}}%
\pgfpathlineto{\pgfqpoint{5.956821in}{3.141750in}}%
\pgfpathlineto{\pgfqpoint{5.960568in}{3.141789in}}%
\pgfpathlineto{\pgfqpoint{5.964582in}{3.139489in}}%
\pgfpathlineto{\pgfqpoint{5.969666in}{3.133656in}}%
\pgfpathlineto{\pgfqpoint{5.988399in}{3.109137in}}%
\pgfpathlineto{\pgfqpoint{5.992413in}{3.108437in}}%
\pgfpathlineto{\pgfqpoint{5.996159in}{3.109999in}}%
\pgfpathlineto{\pgfqpoint{6.000709in}{3.114465in}}%
\pgfpathlineto{\pgfqpoint{6.008202in}{3.125641in}}%
\pgfpathlineto{\pgfqpoint{6.017568in}{3.138633in}}%
\pgfpathlineto{\pgfqpoint{6.022385in}{3.141750in}}%
\pgfpathlineto{\pgfqpoint{6.026132in}{3.141789in}}%
\pgfpathlineto{\pgfqpoint{6.030146in}{3.139489in}}%
\pgfpathlineto{\pgfqpoint{6.035230in}{3.133656in}}%
\pgfpathlineto{\pgfqpoint{6.053963in}{3.109137in}}%
\pgfpathlineto{\pgfqpoint{6.057977in}{3.108437in}}%
\pgfpathlineto{\pgfqpoint{6.061723in}{3.109999in}}%
\pgfpathlineto{\pgfqpoint{6.066273in}{3.114465in}}%
\pgfpathlineto{\pgfqpoint{6.073766in}{3.125641in}}%
\pgfpathlineto{\pgfqpoint{6.083132in}{3.138633in}}%
\pgfpathlineto{\pgfqpoint{6.087949in}{3.141750in}}%
\pgfpathlineto{\pgfqpoint{6.091695in}{3.141789in}}%
\pgfpathlineto{\pgfqpoint{6.095710in}{3.139489in}}%
\pgfpathlineto{\pgfqpoint{6.100794in}{3.133656in}}%
\pgfpathlineto{\pgfqpoint{6.119527in}{3.109137in}}%
\pgfpathlineto{\pgfqpoint{6.123541in}{3.108437in}}%
\pgfpathlineto{\pgfqpoint{6.127287in}{3.109999in}}%
\pgfpathlineto{\pgfqpoint{6.131837in}{3.114465in}}%
\pgfpathlineto{\pgfqpoint{6.139330in}{3.125641in}}%
\pgfpathlineto{\pgfqpoint{6.148696in}{3.138633in}}%
\pgfpathlineto{\pgfqpoint{6.153513in}{3.141750in}}%
\pgfpathlineto{\pgfqpoint{6.157259in}{3.141789in}}%
\pgfpathlineto{\pgfqpoint{6.161273in}{3.139489in}}%
\pgfpathlineto{\pgfqpoint{6.166358in}{3.133656in}}%
\pgfpathlineto{\pgfqpoint{6.185091in}{3.109137in}}%
\pgfpathlineto{\pgfqpoint{6.189105in}{3.108437in}}%
\pgfpathlineto{\pgfqpoint{6.192851in}{3.109999in}}%
\pgfpathlineto{\pgfqpoint{6.197401in}{3.114465in}}%
\pgfpathlineto{\pgfqpoint{6.204894in}{3.125641in}}%
\pgfpathlineto{\pgfqpoint{6.214260in}{3.138633in}}%
\pgfpathlineto{\pgfqpoint{6.219077in}{3.141750in}}%
\pgfpathlineto{\pgfqpoint{6.222823in}{3.141789in}}%
\pgfpathlineto{\pgfqpoint{6.226837in}{3.139489in}}%
\pgfpathlineto{\pgfqpoint{6.231922in}{3.133656in}}%
\pgfpathlineto{\pgfqpoint{6.250654in}{3.109137in}}%
\pgfpathlineto{\pgfqpoint{6.254669in}{3.108437in}}%
\pgfpathlineto{\pgfqpoint{6.258415in}{3.109999in}}%
\pgfpathlineto{\pgfqpoint{6.262964in}{3.114465in}}%
\pgfpathlineto{\pgfqpoint{6.270457in}{3.125641in}}%
\pgfpathlineto{\pgfqpoint{6.279824in}{3.138633in}}%
\pgfpathlineto{\pgfqpoint{6.284641in}{3.141750in}}%
\pgfpathlineto{\pgfqpoint{6.288387in}{3.141789in}}%
\pgfpathlineto{\pgfqpoint{6.292401in}{3.139489in}}%
\pgfpathlineto{\pgfqpoint{6.297486in}{3.133656in}}%
\pgfpathlineto{\pgfqpoint{6.316218in}{3.109137in}}%
\pgfpathlineto{\pgfqpoint{6.320233in}{3.108437in}}%
\pgfpathlineto{\pgfqpoint{6.323979in}{3.109999in}}%
\pgfpathlineto{\pgfqpoint{6.328528in}{3.114465in}}%
\pgfpathlineto{\pgfqpoint{6.336021in}{3.125641in}}%
\pgfpathlineto{\pgfqpoint{6.345388in}{3.138633in}}%
\pgfpathlineto{\pgfqpoint{6.350205in}{3.141750in}}%
\pgfpathlineto{\pgfqpoint{6.353951in}{3.141789in}}%
\pgfpathlineto{\pgfqpoint{6.357965in}{3.139489in}}%
\pgfpathlineto{\pgfqpoint{6.363050in}{3.133656in}}%
\pgfpathlineto{\pgfqpoint{6.381782in}{3.109137in}}%
\pgfpathlineto{\pgfqpoint{6.385796in}{3.108437in}}%
\pgfpathlineto{\pgfqpoint{6.389543in}{3.109999in}}%
\pgfpathlineto{\pgfqpoint{6.394092in}{3.114465in}}%
\pgfpathlineto{\pgfqpoint{6.401585in}{3.125641in}}%
\pgfpathlineto{\pgfqpoint{6.410952in}{3.138633in}}%
\pgfpathlineto{\pgfqpoint{6.415769in}{3.141750in}}%
\pgfpathlineto{\pgfqpoint{6.419515in}{3.141789in}}%
\pgfpathlineto{\pgfqpoint{6.423529in}{3.139489in}}%
\pgfpathlineto{\pgfqpoint{6.428614in}{3.133656in}}%
\pgfpathlineto{\pgfqpoint{6.447346in}{3.109137in}}%
\pgfpathlineto{\pgfqpoint{6.451360in}{3.108437in}}%
\pgfpathlineto{\pgfqpoint{6.455107in}{3.109999in}}%
\pgfpathlineto{\pgfqpoint{6.459656in}{3.114465in}}%
\pgfpathlineto{\pgfqpoint{6.467149in}{3.125641in}}%
\pgfpathlineto{\pgfqpoint{6.476515in}{3.138633in}}%
\pgfpathlineto{\pgfqpoint{6.481332in}{3.141750in}}%
\pgfpathlineto{\pgfqpoint{6.485079in}{3.141789in}}%
\pgfpathlineto{\pgfqpoint{6.489093in}{3.139489in}}%
\pgfpathlineto{\pgfqpoint{6.494178in}{3.133656in}}%
\pgfpathlineto{\pgfqpoint{6.512910in}{3.109137in}}%
\pgfpathlineto{\pgfqpoint{6.516924in}{3.108437in}}%
\pgfpathlineto{\pgfqpoint{6.520671in}{3.109999in}}%
\pgfpathlineto{\pgfqpoint{6.525220in}{3.114465in}}%
\pgfpathlineto{\pgfqpoint{6.532713in}{3.125641in}}%
\pgfpathlineto{\pgfqpoint{6.542079in}{3.138633in}}%
\pgfpathlineto{\pgfqpoint{6.546896in}{3.141750in}}%
\pgfpathlineto{\pgfqpoint{6.550643in}{3.141789in}}%
\pgfpathlineto{\pgfqpoint{6.554657in}{3.139489in}}%
\pgfpathlineto{\pgfqpoint{6.559742in}{3.133656in}}%
\pgfpathlineto{\pgfqpoint{6.578474in}{3.109137in}}%
\pgfpathlineto{\pgfqpoint{6.582488in}{3.108437in}}%
\pgfpathlineto{\pgfqpoint{6.586235in}{3.109999in}}%
\pgfpathlineto{\pgfqpoint{6.590784in}{3.114465in}}%
\pgfpathlineto{\pgfqpoint{6.598277in}{3.125641in}}%
\pgfpathlineto{\pgfqpoint{6.607643in}{3.138633in}}%
\pgfpathlineto{\pgfqpoint{6.612460in}{3.141750in}}%
\pgfpathlineto{\pgfqpoint{6.616207in}{3.141789in}}%
\pgfpathlineto{\pgfqpoint{6.620221in}{3.139489in}}%
\pgfpathlineto{\pgfqpoint{6.625305in}{3.133656in}}%
\pgfpathlineto{\pgfqpoint{6.644038in}{3.109137in}}%
\pgfpathlineto{\pgfqpoint{6.648052in}{3.108437in}}%
\pgfpathlineto{\pgfqpoint{6.651799in}{3.109999in}}%
\pgfpathlineto{\pgfqpoint{6.656348in}{3.114465in}}%
\pgfpathlineto{\pgfqpoint{6.663306in}{3.124778in}}%
\pgfpathlineto{\pgfqpoint{6.663306in}{3.124778in}}%
\pgfusepath{stroke}%
\end{pgfscope}%
\begin{pgfscope}%
\pgfpathrectangle{\pgfqpoint{0.467797in}{2.292089in}}{\pgfqpoint{6.490533in}{1.666241in}}%
\pgfusepath{clip}%
\pgfsetrectcap%
\pgfsetroundjoin%
\pgfsetlinewidth{1.505625pt}%
\definecolor{currentstroke}{rgb}{0.839216,0.152941,0.156863}%
\pgfsetstrokecolor{currentstroke}%
\pgfsetdash{}{0pt}%
\pgfpathmoveto{\pgfqpoint{0.762821in}{3.125209in}}%
\pgfpathlineto{\pgfqpoint{0.771920in}{3.137939in}}%
\pgfpathlineto{\pgfqpoint{0.776469in}{3.140987in}}%
\pgfpathlineto{\pgfqpoint{0.780215in}{3.141091in}}%
\pgfpathlineto{\pgfqpoint{0.784230in}{3.138758in}}%
\pgfpathlineto{\pgfqpoint{0.789314in}{3.132791in}}%
\pgfpathlineto{\pgfqpoint{0.806441in}{3.110052in}}%
\pgfpathlineto{\pgfqpoint{0.810455in}{3.109120in}}%
\pgfpathlineto{\pgfqpoint{0.814202in}{3.110565in}}%
\pgfpathlineto{\pgfqpoint{0.818751in}{3.115014in}}%
\pgfpathlineto{\pgfqpoint{0.826244in}{3.126256in}}%
\pgfpathlineto{\pgfqpoint{0.834807in}{3.138051in}}%
\pgfpathlineto{\pgfqpoint{0.839357in}{3.141024in}}%
\pgfpathlineto{\pgfqpoint{0.843103in}{3.141059in}}%
\pgfpathlineto{\pgfqpoint{0.847117in}{3.138658in}}%
\pgfpathlineto{\pgfqpoint{0.852202in}{3.132628in}}%
\pgfpathlineto{\pgfqpoint{0.869061in}{3.110138in}}%
\pgfpathlineto{\pgfqpoint{0.873075in}{3.109108in}}%
\pgfpathlineto{\pgfqpoint{0.876822in}{3.110463in}}%
\pgfpathlineto{\pgfqpoint{0.881371in}{3.114823in}}%
\pgfpathlineto{\pgfqpoint{0.888597in}{3.125577in}}%
\pgfpathlineto{\pgfqpoint{0.897695in}{3.138161in}}%
\pgfpathlineto{\pgfqpoint{0.902245in}{3.141058in}}%
\pgfpathlineto{\pgfqpoint{0.905991in}{3.141025in}}%
\pgfpathlineto{\pgfqpoint{0.910005in}{3.138556in}}%
\pgfpathlineto{\pgfqpoint{0.915090in}{3.132465in}}%
\pgfpathlineto{\pgfqpoint{0.931681in}{3.110227in}}%
\pgfpathlineto{\pgfqpoint{0.935696in}{3.109100in}}%
\pgfpathlineto{\pgfqpoint{0.939442in}{3.110365in}}%
\pgfpathlineto{\pgfqpoint{0.943724in}{3.114313in}}%
\pgfpathlineto{\pgfqpoint{0.950414in}{3.124035in}}%
\pgfpathlineto{\pgfqpoint{0.960315in}{3.138012in}}%
\pgfpathlineto{\pgfqpoint{0.964865in}{3.141011in}}%
\pgfpathlineto{\pgfqpoint{0.968611in}{3.141070in}}%
\pgfpathlineto{\pgfqpoint{0.972625in}{3.138693in}}%
\pgfpathlineto{\pgfqpoint{0.977710in}{3.132685in}}%
\pgfpathlineto{\pgfqpoint{0.994569in}{3.110161in}}%
\pgfpathlineto{\pgfqpoint{0.998583in}{3.109105in}}%
\pgfpathlineto{\pgfqpoint{1.002330in}{3.110437in}}%
\pgfpathlineto{\pgfqpoint{1.006879in}{3.114774in}}%
\pgfpathlineto{\pgfqpoint{1.014105in}{3.125512in}}%
\pgfpathlineto{\pgfqpoint{1.023203in}{3.138122in}}%
\pgfpathlineto{\pgfqpoint{1.027753in}{3.141046in}}%
\pgfpathlineto{\pgfqpoint{1.031499in}{3.141037in}}%
\pgfpathlineto{\pgfqpoint{1.035513in}{3.138592in}}%
\pgfpathlineto{\pgfqpoint{1.040598in}{3.132522in}}%
\pgfpathlineto{\pgfqpoint{1.057190in}{3.110251in}}%
\pgfpathlineto{\pgfqpoint{1.061204in}{3.109098in}}%
\pgfpathlineto{\pgfqpoint{1.064950in}{3.110340in}}%
\pgfpathlineto{\pgfqpoint{1.069232in}{3.114266in}}%
\pgfpathlineto{\pgfqpoint{1.075922in}{3.123971in}}%
\pgfpathlineto{\pgfqpoint{1.085824in}{3.137973in}}%
\pgfpathlineto{\pgfqpoint{1.090373in}{3.140998in}}%
\pgfpathlineto{\pgfqpoint{1.094119in}{3.141082in}}%
\pgfpathlineto{\pgfqpoint{1.098134in}{3.138729in}}%
\pgfpathlineto{\pgfqpoint{1.103218in}{3.132742in}}%
\pgfpathlineto{\pgfqpoint{1.120345in}{3.110033in}}%
\pgfpathlineto{\pgfqpoint{1.124359in}{3.109123in}}%
\pgfpathlineto{\pgfqpoint{1.128106in}{3.110588in}}%
\pgfpathlineto{\pgfqpoint{1.132655in}{3.115057in}}%
\pgfpathlineto{\pgfqpoint{1.140148in}{3.126311in}}%
\pgfpathlineto{\pgfqpoint{1.148711in}{3.138084in}}%
\pgfpathlineto{\pgfqpoint{1.153261in}{3.141034in}}%
\pgfpathlineto{\pgfqpoint{1.157007in}{3.141049in}}%
\pgfpathlineto{\pgfqpoint{1.161021in}{3.138628in}}%
\pgfpathlineto{\pgfqpoint{1.166106in}{3.132579in}}%
\pgfpathlineto{\pgfqpoint{1.182965in}{3.110118in}}%
\pgfpathlineto{\pgfqpoint{1.186979in}{3.109110in}}%
\pgfpathlineto{\pgfqpoint{1.190726in}{3.110485in}}%
\pgfpathlineto{\pgfqpoint{1.195275in}{3.114865in}}%
\pgfpathlineto{\pgfqpoint{1.202501in}{3.125632in}}%
\pgfpathlineto{\pgfqpoint{1.211599in}{3.138194in}}%
\pgfpathlineto{\pgfqpoint{1.216149in}{3.141068in}}%
\pgfpathlineto{\pgfqpoint{1.219895in}{3.141014in}}%
\pgfpathlineto{\pgfqpoint{1.223909in}{3.138525in}}%
\pgfpathlineto{\pgfqpoint{1.228994in}{3.132416in}}%
\pgfpathlineto{\pgfqpoint{1.245585in}{3.110207in}}%
\pgfpathlineto{\pgfqpoint{1.249600in}{3.109101in}}%
\pgfpathlineto{\pgfqpoint{1.253346in}{3.110386in}}%
\pgfpathlineto{\pgfqpoint{1.257895in}{3.114676in}}%
\pgfpathlineto{\pgfqpoint{1.264853in}{3.124952in}}%
\pgfpathlineto{\pgfqpoint{1.274219in}{3.138045in}}%
\pgfpathlineto{\pgfqpoint{1.278769in}{3.141022in}}%
\pgfpathlineto{\pgfqpoint{1.282515in}{3.141060in}}%
\pgfpathlineto{\pgfqpoint{1.286529in}{3.138663in}}%
\pgfpathlineto{\pgfqpoint{1.291614in}{3.132637in}}%
\pgfpathlineto{\pgfqpoint{1.308473in}{3.110141in}}%
\pgfpathlineto{\pgfqpoint{1.312487in}{3.109107in}}%
\pgfpathlineto{\pgfqpoint{1.316234in}{3.110459in}}%
\pgfpathlineto{\pgfqpoint{1.320783in}{3.114816in}}%
\pgfpathlineto{\pgfqpoint{1.328009in}{3.125568in}}%
\pgfpathlineto{\pgfqpoint{1.337107in}{3.138155in}}%
\pgfpathlineto{\pgfqpoint{1.341657in}{3.141056in}}%
\pgfpathlineto{\pgfqpoint{1.345403in}{3.141026in}}%
\pgfpathlineto{\pgfqpoint{1.349417in}{3.138561in}}%
\pgfpathlineto{\pgfqpoint{1.354502in}{3.132473in}}%
\pgfpathlineto{\pgfqpoint{1.371093in}{3.110231in}}%
\pgfpathlineto{\pgfqpoint{1.375108in}{3.109100in}}%
\pgfpathlineto{\pgfqpoint{1.378854in}{3.110361in}}%
\pgfpathlineto{\pgfqpoint{1.383136in}{3.114306in}}%
\pgfpathlineto{\pgfqpoint{1.389826in}{3.124026in}}%
\pgfpathlineto{\pgfqpoint{1.399728in}{3.138006in}}%
\pgfpathlineto{\pgfqpoint{1.404277in}{3.141009in}}%
\pgfpathlineto{\pgfqpoint{1.408023in}{3.141072in}}%
\pgfpathlineto{\pgfqpoint{1.412037in}{3.138699in}}%
\pgfpathlineto{\pgfqpoint{1.417122in}{3.132694in}}%
\pgfpathlineto{\pgfqpoint{1.433981in}{3.110164in}}%
\pgfpathlineto{\pgfqpoint{1.437995in}{3.109105in}}%
\pgfpathlineto{\pgfqpoint{1.441742in}{3.110434in}}%
\pgfpathlineto{\pgfqpoint{1.446291in}{3.114767in}}%
\pgfpathlineto{\pgfqpoint{1.453517in}{3.125503in}}%
\pgfpathlineto{\pgfqpoint{1.462615in}{3.138117in}}%
\pgfpathlineto{\pgfqpoint{1.467165in}{3.141045in}}%
\pgfpathlineto{\pgfqpoint{1.470911in}{3.141039in}}%
\pgfpathlineto{\pgfqpoint{1.474925in}{3.138597in}}%
\pgfpathlineto{\pgfqpoint{1.480010in}{3.132530in}}%
\pgfpathlineto{\pgfqpoint{1.496602in}{3.110255in}}%
\pgfpathlineto{\pgfqpoint{1.500616in}{3.109098in}}%
\pgfpathlineto{\pgfqpoint{1.504362in}{3.110336in}}%
\pgfpathlineto{\pgfqpoint{1.508644in}{3.114259in}}%
\pgfpathlineto{\pgfqpoint{1.515334in}{3.123962in}}%
\pgfpathlineto{\pgfqpoint{1.525503in}{3.138226in}}%
\pgfpathlineto{\pgfqpoint{1.530053in}{3.141078in}}%
\pgfpathlineto{\pgfqpoint{1.533799in}{3.141003in}}%
\pgfpathlineto{\pgfqpoint{1.537813in}{3.138494in}}%
\pgfpathlineto{\pgfqpoint{1.542898in}{3.132366in}}%
\pgfpathlineto{\pgfqpoint{1.559489in}{3.110187in}}%
\pgfpathlineto{\pgfqpoint{1.563504in}{3.109103in}}%
\pgfpathlineto{\pgfqpoint{1.567250in}{3.110408in}}%
\pgfpathlineto{\pgfqpoint{1.571799in}{3.114718in}}%
\pgfpathlineto{\pgfqpoint{1.578757in}{3.125007in}}%
\pgfpathlineto{\pgfqpoint{1.588123in}{3.138078in}}%
\pgfpathlineto{\pgfqpoint{1.592673in}{3.141032in}}%
\pgfpathlineto{\pgfqpoint{1.596419in}{3.141050in}}%
\pgfpathlineto{\pgfqpoint{1.600433in}{3.138633in}}%
\pgfpathlineto{\pgfqpoint{1.605518in}{3.132588in}}%
\pgfpathlineto{\pgfqpoint{1.622377in}{3.110122in}}%
\pgfpathlineto{\pgfqpoint{1.626391in}{3.109110in}}%
\pgfpathlineto{\pgfqpoint{1.630138in}{3.110482in}}%
\pgfpathlineto{\pgfqpoint{1.634687in}{3.114858in}}%
\pgfpathlineto{\pgfqpoint{1.641913in}{3.125623in}}%
\pgfpathlineto{\pgfqpoint{1.651011in}{3.138188in}}%
\pgfpathlineto{\pgfqpoint{1.655561in}{3.141066in}}%
\pgfpathlineto{\pgfqpoint{1.659307in}{3.141016in}}%
\pgfpathlineto{\pgfqpoint{1.663321in}{3.138530in}}%
\pgfpathlineto{\pgfqpoint{1.668406in}{3.132424in}}%
\pgfpathlineto{\pgfqpoint{1.684997in}{3.110211in}}%
\pgfpathlineto{\pgfqpoint{1.689012in}{3.109101in}}%
\pgfpathlineto{\pgfqpoint{1.692758in}{3.110383in}}%
\pgfpathlineto{\pgfqpoint{1.697307in}{3.114669in}}%
\pgfpathlineto{\pgfqpoint{1.704265in}{3.124943in}}%
\pgfpathlineto{\pgfqpoint{1.713631in}{3.138040in}}%
\pgfpathlineto{\pgfqpoint{1.718181in}{3.141020in}}%
\pgfpathlineto{\pgfqpoint{1.721927in}{3.141062in}}%
\pgfpathlineto{\pgfqpoint{1.725941in}{3.138668in}}%
\pgfpathlineto{\pgfqpoint{1.731026in}{3.132645in}}%
\pgfpathlineto{\pgfqpoint{1.747885in}{3.110144in}}%
\pgfpathlineto{\pgfqpoint{1.751899in}{3.109107in}}%
\pgfpathlineto{\pgfqpoint{1.755646in}{3.110456in}}%
\pgfpathlineto{\pgfqpoint{1.760195in}{3.114809in}}%
\pgfpathlineto{\pgfqpoint{1.767421in}{3.125558in}}%
\pgfpathlineto{\pgfqpoint{1.776519in}{3.138150in}}%
\pgfpathlineto{\pgfqpoint{1.781069in}{3.141055in}}%
\pgfpathlineto{\pgfqpoint{1.784815in}{3.141028in}}%
\pgfpathlineto{\pgfqpoint{1.788829in}{3.138566in}}%
\pgfpathlineto{\pgfqpoint{1.793914in}{3.132481in}}%
\pgfpathlineto{\pgfqpoint{1.810506in}{3.110234in}}%
\pgfpathlineto{\pgfqpoint{1.814520in}{3.109099in}}%
\pgfpathlineto{\pgfqpoint{1.818266in}{3.110358in}}%
\pgfpathlineto{\pgfqpoint{1.822548in}{3.114299in}}%
\pgfpathlineto{\pgfqpoint{1.829238in}{3.124017in}}%
\pgfpathlineto{\pgfqpoint{1.839140in}{3.138001in}}%
\pgfpathlineto{\pgfqpoint{1.843689in}{3.141008in}}%
\pgfpathlineto{\pgfqpoint{1.847435in}{3.141074in}}%
\pgfpathlineto{\pgfqpoint{1.851450in}{3.138704in}}%
\pgfpathlineto{\pgfqpoint{1.856534in}{3.132702in}}%
\pgfpathlineto{\pgfqpoint{1.873661in}{3.110018in}}%
\pgfpathlineto{\pgfqpoint{1.877675in}{3.109126in}}%
\pgfpathlineto{\pgfqpoint{1.881422in}{3.110607in}}%
\pgfpathlineto{\pgfqpoint{1.885971in}{3.115092in}}%
\pgfpathlineto{\pgfqpoint{1.893464in}{3.126356in}}%
\pgfpathlineto{\pgfqpoint{1.902027in}{3.138111in}}%
\pgfpathlineto{\pgfqpoint{1.906577in}{3.141043in}}%
\pgfpathlineto{\pgfqpoint{1.910323in}{3.141040in}}%
\pgfpathlineto{\pgfqpoint{1.914337in}{3.138602in}}%
\pgfpathlineto{\pgfqpoint{1.919422in}{3.132539in}}%
\pgfpathlineto{\pgfqpoint{1.936281in}{3.110102in}}%
\pgfpathlineto{\pgfqpoint{1.940295in}{3.109112in}}%
\pgfpathlineto{\pgfqpoint{1.944042in}{3.110504in}}%
\pgfpathlineto{\pgfqpoint{1.948591in}{3.114901in}}%
\pgfpathlineto{\pgfqpoint{1.955817in}{3.125678in}}%
\pgfpathlineto{\pgfqpoint{1.964648in}{3.137961in}}%
\pgfpathlineto{\pgfqpoint{1.969197in}{3.140995in}}%
\pgfpathlineto{\pgfqpoint{1.972943in}{3.141085in}}%
\pgfpathlineto{\pgfqpoint{1.976958in}{3.138739in}}%
\pgfpathlineto{\pgfqpoint{1.982042in}{3.132759in}}%
\pgfpathlineto{\pgfqpoint{1.999169in}{3.110039in}}%
\pgfpathlineto{\pgfqpoint{2.003183in}{3.109122in}}%
\pgfpathlineto{\pgfqpoint{2.006930in}{3.110580in}}%
\pgfpathlineto{\pgfqpoint{2.011479in}{3.115042in}}%
\pgfpathlineto{\pgfqpoint{2.018972in}{3.126292in}}%
\pgfpathlineto{\pgfqpoint{2.027535in}{3.138073in}}%
\pgfpathlineto{\pgfqpoint{2.032085in}{3.141031in}}%
\pgfpathlineto{\pgfqpoint{2.035831in}{3.141052in}}%
\pgfpathlineto{\pgfqpoint{2.039845in}{3.138638in}}%
\pgfpathlineto{\pgfqpoint{2.044930in}{3.132596in}}%
\pgfpathlineto{\pgfqpoint{2.061789in}{3.110125in}}%
\pgfpathlineto{\pgfqpoint{2.065803in}{3.109109in}}%
\pgfpathlineto{\pgfqpoint{2.069550in}{3.110478in}}%
\pgfpathlineto{\pgfqpoint{2.074099in}{3.114851in}}%
\pgfpathlineto{\pgfqpoint{2.081325in}{3.125613in}}%
\pgfpathlineto{\pgfqpoint{2.090423in}{3.138183in}}%
\pgfpathlineto{\pgfqpoint{2.094973in}{3.141065in}}%
\pgfpathlineto{\pgfqpoint{2.098719in}{3.141017in}}%
\pgfpathlineto{\pgfqpoint{2.102733in}{3.138536in}}%
\pgfpathlineto{\pgfqpoint{2.107818in}{3.132432in}}%
\pgfpathlineto{\pgfqpoint{2.124410in}{3.110214in}}%
\pgfpathlineto{\pgfqpoint{2.128424in}{3.109101in}}%
\pgfpathlineto{\pgfqpoint{2.132170in}{3.110379in}}%
\pgfpathlineto{\pgfqpoint{2.136719in}{3.114663in}}%
\pgfpathlineto{\pgfqpoint{2.143677in}{3.124934in}}%
\pgfpathlineto{\pgfqpoint{2.153044in}{3.138034in}}%
\pgfpathlineto{\pgfqpoint{2.157593in}{3.141018in}}%
\pgfpathlineto{\pgfqpoint{2.161339in}{3.141064in}}%
\pgfpathlineto{\pgfqpoint{2.165353in}{3.138673in}}%
\pgfpathlineto{\pgfqpoint{2.170438in}{3.132653in}}%
\pgfpathlineto{\pgfqpoint{2.187297in}{3.110148in}}%
\pgfpathlineto{\pgfqpoint{2.191311in}{3.109107in}}%
\pgfpathlineto{\pgfqpoint{2.195058in}{3.110452in}}%
\pgfpathlineto{\pgfqpoint{2.199607in}{3.114802in}}%
\pgfpathlineto{\pgfqpoint{2.206833in}{3.125549in}}%
\pgfpathlineto{\pgfqpoint{2.215931in}{3.138144in}}%
\pgfpathlineto{\pgfqpoint{2.220481in}{3.141053in}}%
\pgfpathlineto{\pgfqpoint{2.224227in}{3.141030in}}%
\pgfpathlineto{\pgfqpoint{2.228241in}{3.138572in}}%
\pgfpathlineto{\pgfqpoint{2.233326in}{3.132489in}}%
\pgfpathlineto{\pgfqpoint{2.249918in}{3.110238in}}%
\pgfpathlineto{\pgfqpoint{2.253932in}{3.109099in}}%
\pgfpathlineto{\pgfqpoint{2.257678in}{3.110354in}}%
\pgfpathlineto{\pgfqpoint{2.261960in}{3.114293in}}%
\pgfpathlineto{\pgfqpoint{2.268650in}{3.124007in}}%
\pgfpathlineto{\pgfqpoint{2.278552in}{3.137995in}}%
\pgfpathlineto{\pgfqpoint{2.283101in}{3.141006in}}%
\pgfpathlineto{\pgfqpoint{2.286847in}{3.141075in}}%
\pgfpathlineto{\pgfqpoint{2.290862in}{3.138709in}}%
\pgfpathlineto{\pgfqpoint{2.295946in}{3.132710in}}%
\pgfpathlineto{\pgfqpoint{2.313073in}{3.110021in}}%
\pgfpathlineto{\pgfqpoint{2.317087in}{3.109125in}}%
\pgfpathlineto{\pgfqpoint{2.320834in}{3.110603in}}%
\pgfpathlineto{\pgfqpoint{2.325383in}{3.115085in}}%
\pgfpathlineto{\pgfqpoint{2.332876in}{3.126347in}}%
\pgfpathlineto{\pgfqpoint{2.341439in}{3.138106in}}%
\pgfpathlineto{\pgfqpoint{2.345989in}{3.141041in}}%
\pgfpathlineto{\pgfqpoint{2.349735in}{3.141042in}}%
\pgfpathlineto{\pgfqpoint{2.353749in}{3.138607in}}%
\pgfpathlineto{\pgfqpoint{2.358834in}{3.132547in}}%
\pgfpathlineto{\pgfqpoint{2.375693in}{3.110106in}}%
\pgfpathlineto{\pgfqpoint{2.379707in}{3.109112in}}%
\pgfpathlineto{\pgfqpoint{2.383454in}{3.110500in}}%
\pgfpathlineto{\pgfqpoint{2.388003in}{3.114893in}}%
\pgfpathlineto{\pgfqpoint{2.395229in}{3.125669in}}%
\pgfpathlineto{\pgfqpoint{2.404060in}{3.137956in}}%
\pgfpathlineto{\pgfqpoint{2.408609in}{3.140993in}}%
\pgfpathlineto{\pgfqpoint{2.412356in}{3.141086in}}%
\pgfpathlineto{\pgfqpoint{2.416370in}{3.138744in}}%
\pgfpathlineto{\pgfqpoint{2.421454in}{3.132767in}}%
\pgfpathlineto{\pgfqpoint{2.438581in}{3.110043in}}%
\pgfpathlineto{\pgfqpoint{2.442595in}{3.109121in}}%
\pgfpathlineto{\pgfqpoint{2.446342in}{3.110576in}}%
\pgfpathlineto{\pgfqpoint{2.450891in}{3.115035in}}%
\pgfpathlineto{\pgfqpoint{2.458384in}{3.126283in}}%
\pgfpathlineto{\pgfqpoint{2.466948in}{3.138067in}}%
\pgfpathlineto{\pgfqpoint{2.471497in}{3.141029in}}%
\pgfpathlineto{\pgfqpoint{2.475243in}{3.141054in}}%
\pgfpathlineto{\pgfqpoint{2.479257in}{3.138643in}}%
\pgfpathlineto{\pgfqpoint{2.484342in}{3.132604in}}%
\pgfpathlineto{\pgfqpoint{2.501201in}{3.110128in}}%
\pgfpathlineto{\pgfqpoint{2.505215in}{3.109109in}}%
\pgfpathlineto{\pgfqpoint{2.508962in}{3.110474in}}%
\pgfpathlineto{\pgfqpoint{2.513511in}{3.114844in}}%
\pgfpathlineto{\pgfqpoint{2.520737in}{3.125604in}}%
\pgfpathlineto{\pgfqpoint{2.529835in}{3.138177in}}%
\pgfpathlineto{\pgfqpoint{2.534385in}{3.141063in}}%
\pgfpathlineto{\pgfqpoint{2.538131in}{3.141019in}}%
\pgfpathlineto{\pgfqpoint{2.542145in}{3.138541in}}%
\pgfpathlineto{\pgfqpoint{2.547230in}{3.132440in}}%
\pgfpathlineto{\pgfqpoint{2.563822in}{3.110217in}}%
\pgfpathlineto{\pgfqpoint{2.567836in}{3.109101in}}%
\pgfpathlineto{\pgfqpoint{2.571582in}{3.110375in}}%
\pgfpathlineto{\pgfqpoint{2.576132in}{3.114656in}}%
\pgfpathlineto{\pgfqpoint{2.583089in}{3.124925in}}%
\pgfpathlineto{\pgfqpoint{2.592456in}{3.138028in}}%
\pgfpathlineto{\pgfqpoint{2.597005in}{3.141017in}}%
\pgfpathlineto{\pgfqpoint{2.600751in}{3.141065in}}%
\pgfpathlineto{\pgfqpoint{2.604766in}{3.138678in}}%
\pgfpathlineto{\pgfqpoint{2.609850in}{3.132661in}}%
\pgfpathlineto{\pgfqpoint{2.626709in}{3.110151in}}%
\pgfpathlineto{\pgfqpoint{2.630724in}{3.109106in}}%
\pgfpathlineto{\pgfqpoint{2.634470in}{3.110448in}}%
\pgfpathlineto{\pgfqpoint{2.639019in}{3.114795in}}%
\pgfpathlineto{\pgfqpoint{2.646245in}{3.125540in}}%
\pgfpathlineto{\pgfqpoint{2.655343in}{3.138139in}}%
\pgfpathlineto{\pgfqpoint{2.659893in}{3.141051in}}%
\pgfpathlineto{\pgfqpoint{2.663639in}{3.141032in}}%
\pgfpathlineto{\pgfqpoint{2.667653in}{3.138577in}}%
\pgfpathlineto{\pgfqpoint{2.672738in}{3.132498in}}%
\pgfpathlineto{\pgfqpoint{2.689330in}{3.110241in}}%
\pgfpathlineto{\pgfqpoint{2.693344in}{3.109099in}}%
\pgfpathlineto{\pgfqpoint{2.697090in}{3.110350in}}%
\pgfpathlineto{\pgfqpoint{2.701372in}{3.114286in}}%
\pgfpathlineto{\pgfqpoint{2.708062in}{3.123998in}}%
\pgfpathlineto{\pgfqpoint{2.717964in}{3.137989in}}%
\pgfpathlineto{\pgfqpoint{2.722513in}{3.141004in}}%
\pgfpathlineto{\pgfqpoint{2.726259in}{3.141077in}}%
\pgfpathlineto{\pgfqpoint{2.730274in}{3.138714in}}%
\pgfpathlineto{\pgfqpoint{2.735358in}{3.132718in}}%
\pgfpathlineto{\pgfqpoint{2.752485in}{3.110024in}}%
\pgfpathlineto{\pgfqpoint{2.756499in}{3.109124in}}%
\pgfpathlineto{\pgfqpoint{2.760246in}{3.110599in}}%
\pgfpathlineto{\pgfqpoint{2.764795in}{3.115078in}}%
\pgfpathlineto{\pgfqpoint{2.772288in}{3.126338in}}%
\pgfpathlineto{\pgfqpoint{2.780851in}{3.138100in}}%
\pgfpathlineto{\pgfqpoint{2.785401in}{3.141039in}}%
\pgfpathlineto{\pgfqpoint{2.789147in}{3.141044in}}%
\pgfpathlineto{\pgfqpoint{2.793161in}{3.138612in}}%
\pgfpathlineto{\pgfqpoint{2.798246in}{3.132555in}}%
\pgfpathlineto{\pgfqpoint{2.815105in}{3.110109in}}%
\pgfpathlineto{\pgfqpoint{2.819119in}{3.109111in}}%
\pgfpathlineto{\pgfqpoint{2.822866in}{3.110497in}}%
\pgfpathlineto{\pgfqpoint{2.827415in}{3.114886in}}%
\pgfpathlineto{\pgfqpoint{2.834641in}{3.125659in}}%
\pgfpathlineto{\pgfqpoint{2.843472in}{3.137950in}}%
\pgfpathlineto{\pgfqpoint{2.848021in}{3.140991in}}%
\pgfpathlineto{\pgfqpoint{2.851768in}{3.141088in}}%
\pgfpathlineto{\pgfqpoint{2.855782in}{3.138749in}}%
\pgfpathlineto{\pgfqpoint{2.860866in}{3.132775in}}%
\pgfpathlineto{\pgfqpoint{2.877993in}{3.110046in}}%
\pgfpathlineto{\pgfqpoint{2.882007in}{3.109121in}}%
\pgfpathlineto{\pgfqpoint{2.885754in}{3.110572in}}%
\pgfpathlineto{\pgfqpoint{2.890303in}{3.115028in}}%
\pgfpathlineto{\pgfqpoint{2.897796in}{3.126274in}}%
\pgfpathlineto{\pgfqpoint{2.906360in}{3.138062in}}%
\pgfpathlineto{\pgfqpoint{2.910909in}{3.141027in}}%
\pgfpathlineto{\pgfqpoint{2.914655in}{3.141055in}}%
\pgfpathlineto{\pgfqpoint{2.918670in}{3.138648in}}%
\pgfpathlineto{\pgfqpoint{2.923754in}{3.132612in}}%
\pgfpathlineto{\pgfqpoint{2.940613in}{3.110131in}}%
\pgfpathlineto{\pgfqpoint{2.944627in}{3.109109in}}%
\pgfpathlineto{\pgfqpoint{2.948374in}{3.110470in}}%
\pgfpathlineto{\pgfqpoint{2.952923in}{3.114837in}}%
\pgfpathlineto{\pgfqpoint{2.960149in}{3.125595in}}%
\pgfpathlineto{\pgfqpoint{2.969247in}{3.138172in}}%
\pgfpathlineto{\pgfqpoint{2.973797in}{3.141061in}}%
\pgfpathlineto{\pgfqpoint{2.977543in}{3.141021in}}%
\pgfpathlineto{\pgfqpoint{2.981557in}{3.138546in}}%
\pgfpathlineto{\pgfqpoint{2.986642in}{3.132448in}}%
\pgfpathlineto{\pgfqpoint{3.003234in}{3.110221in}}%
\pgfpathlineto{\pgfqpoint{3.007248in}{3.109100in}}%
\pgfpathlineto{\pgfqpoint{3.010994in}{3.110372in}}%
\pgfpathlineto{\pgfqpoint{3.015544in}{3.114649in}}%
\pgfpathlineto{\pgfqpoint{3.022501in}{3.124916in}}%
\pgfpathlineto{\pgfqpoint{3.031868in}{3.138023in}}%
\pgfpathlineto{\pgfqpoint{3.036417in}{3.141015in}}%
\pgfpathlineto{\pgfqpoint{3.040163in}{3.141067in}}%
\pgfpathlineto{\pgfqpoint{3.044178in}{3.138683in}}%
\pgfpathlineto{\pgfqpoint{3.049262in}{3.132669in}}%
\pgfpathlineto{\pgfqpoint{3.066121in}{3.110154in}}%
\pgfpathlineto{\pgfqpoint{3.070136in}{3.109106in}}%
\pgfpathlineto{\pgfqpoint{3.073882in}{3.110445in}}%
\pgfpathlineto{\pgfqpoint{3.078431in}{3.114788in}}%
\pgfpathlineto{\pgfqpoint{3.085657in}{3.125531in}}%
\pgfpathlineto{\pgfqpoint{3.094755in}{3.138133in}}%
\pgfpathlineto{\pgfqpoint{3.099305in}{3.141050in}}%
\pgfpathlineto{\pgfqpoint{3.103051in}{3.141033in}}%
\pgfpathlineto{\pgfqpoint{3.107065in}{3.138582in}}%
\pgfpathlineto{\pgfqpoint{3.112150in}{3.132506in}}%
\pgfpathlineto{\pgfqpoint{3.128742in}{3.110244in}}%
\pgfpathlineto{\pgfqpoint{3.132756in}{3.109099in}}%
\pgfpathlineto{\pgfqpoint{3.136502in}{3.110347in}}%
\pgfpathlineto{\pgfqpoint{3.140784in}{3.114279in}}%
\pgfpathlineto{\pgfqpoint{3.147474in}{3.123989in}}%
\pgfpathlineto{\pgfqpoint{3.157376in}{3.137984in}}%
\pgfpathlineto{\pgfqpoint{3.161925in}{3.141002in}}%
\pgfpathlineto{\pgfqpoint{3.165672in}{3.141078in}}%
\pgfpathlineto{\pgfqpoint{3.169686in}{3.138719in}}%
\pgfpathlineto{\pgfqpoint{3.174770in}{3.132726in}}%
\pgfpathlineto{\pgfqpoint{3.191897in}{3.110027in}}%
\pgfpathlineto{\pgfqpoint{3.195911in}{3.109124in}}%
\pgfpathlineto{\pgfqpoint{3.199658in}{3.110596in}}%
\pgfpathlineto{\pgfqpoint{3.204207in}{3.115071in}}%
\pgfpathlineto{\pgfqpoint{3.211700in}{3.126329in}}%
\pgfpathlineto{\pgfqpoint{3.220264in}{3.138095in}}%
\pgfpathlineto{\pgfqpoint{3.224813in}{3.141038in}}%
\pgfpathlineto{\pgfqpoint{3.228559in}{3.141045in}}%
\pgfpathlineto{\pgfqpoint{3.232573in}{3.138618in}}%
\pgfpathlineto{\pgfqpoint{3.237658in}{3.132563in}}%
\pgfpathlineto{\pgfqpoint{3.254517in}{3.110112in}}%
\pgfpathlineto{\pgfqpoint{3.258531in}{3.109111in}}%
\pgfpathlineto{\pgfqpoint{3.262278in}{3.110493in}}%
\pgfpathlineto{\pgfqpoint{3.266827in}{3.114879in}}%
\pgfpathlineto{\pgfqpoint{3.274053in}{3.125650in}}%
\pgfpathlineto{\pgfqpoint{3.282884in}{3.137945in}}%
\pgfpathlineto{\pgfqpoint{3.287433in}{3.140989in}}%
\pgfpathlineto{\pgfqpoint{3.291180in}{3.141089in}}%
\pgfpathlineto{\pgfqpoint{3.295194in}{3.138754in}}%
\pgfpathlineto{\pgfqpoint{3.300278in}{3.132783in}}%
\pgfpathlineto{\pgfqpoint{3.317405in}{3.110049in}}%
\pgfpathlineto{\pgfqpoint{3.321419in}{3.109120in}}%
\pgfpathlineto{\pgfqpoint{3.325166in}{3.110569in}}%
\pgfpathlineto{\pgfqpoint{3.329715in}{3.115021in}}%
\pgfpathlineto{\pgfqpoint{3.337208in}{3.126265in}}%
\pgfpathlineto{\pgfqpoint{3.345772in}{3.138056in}}%
\pgfpathlineto{\pgfqpoint{3.350321in}{3.141025in}}%
\pgfpathlineto{\pgfqpoint{3.354067in}{3.141057in}}%
\pgfpathlineto{\pgfqpoint{3.358082in}{3.138653in}}%
\pgfpathlineto{\pgfqpoint{3.363166in}{3.132620in}}%
\pgfpathlineto{\pgfqpoint{3.380025in}{3.110135in}}%
\pgfpathlineto{\pgfqpoint{3.384040in}{3.109108in}}%
\pgfpathlineto{\pgfqpoint{3.387786in}{3.110467in}}%
\pgfpathlineto{\pgfqpoint{3.392335in}{3.114830in}}%
\pgfpathlineto{\pgfqpoint{3.399561in}{3.125586in}}%
\pgfpathlineto{\pgfqpoint{3.408659in}{3.138166in}}%
\pgfpathlineto{\pgfqpoint{3.413209in}{3.141060in}}%
\pgfpathlineto{\pgfqpoint{3.416955in}{3.141023in}}%
\pgfpathlineto{\pgfqpoint{3.420969in}{3.138551in}}%
\pgfpathlineto{\pgfqpoint{3.426054in}{3.132457in}}%
\pgfpathlineto{\pgfqpoint{3.442646in}{3.110224in}}%
\pgfpathlineto{\pgfqpoint{3.446660in}{3.109100in}}%
\pgfpathlineto{\pgfqpoint{3.450406in}{3.110368in}}%
\pgfpathlineto{\pgfqpoint{3.454688in}{3.114320in}}%
\pgfpathlineto{\pgfqpoint{3.461378in}{3.124044in}}%
\pgfpathlineto{\pgfqpoint{3.471280in}{3.138017in}}%
\pgfpathlineto{\pgfqpoint{3.475829in}{3.141013in}}%
\pgfpathlineto{\pgfqpoint{3.479576in}{3.141069in}}%
\pgfpathlineto{\pgfqpoint{3.483590in}{3.138688in}}%
\pgfpathlineto{\pgfqpoint{3.488674in}{3.132677in}}%
\pgfpathlineto{\pgfqpoint{3.505533in}{3.110157in}}%
\pgfpathlineto{\pgfqpoint{3.509548in}{3.109106in}}%
\pgfpathlineto{\pgfqpoint{3.513294in}{3.110441in}}%
\pgfpathlineto{\pgfqpoint{3.517843in}{3.114781in}}%
\pgfpathlineto{\pgfqpoint{3.525069in}{3.125522in}}%
\pgfpathlineto{\pgfqpoint{3.534168in}{3.138128in}}%
\pgfpathlineto{\pgfqpoint{3.538717in}{3.141048in}}%
\pgfpathlineto{\pgfqpoint{3.542463in}{3.141035in}}%
\pgfpathlineto{\pgfqpoint{3.546477in}{3.138587in}}%
\pgfpathlineto{\pgfqpoint{3.551562in}{3.132514in}}%
\pgfpathlineto{\pgfqpoint{3.568154in}{3.110248in}}%
\pgfpathlineto{\pgfqpoint{3.572168in}{3.109099in}}%
\pgfpathlineto{\pgfqpoint{3.575914in}{3.110343in}}%
\pgfpathlineto{\pgfqpoint{3.580196in}{3.114272in}}%
\pgfpathlineto{\pgfqpoint{3.586886in}{3.123980in}}%
\pgfpathlineto{\pgfqpoint{3.596788in}{3.137978in}}%
\pgfpathlineto{\pgfqpoint{3.601337in}{3.141000in}}%
\pgfpathlineto{\pgfqpoint{3.605084in}{3.141080in}}%
\pgfpathlineto{\pgfqpoint{3.609098in}{3.138724in}}%
\pgfpathlineto{\pgfqpoint{3.614182in}{3.132734in}}%
\pgfpathlineto{\pgfqpoint{3.631309in}{3.110030in}}%
\pgfpathlineto{\pgfqpoint{3.635323in}{3.109123in}}%
\pgfpathlineto{\pgfqpoint{3.639070in}{3.110592in}}%
\pgfpathlineto{\pgfqpoint{3.643619in}{3.115064in}}%
\pgfpathlineto{\pgfqpoint{3.651112in}{3.126320in}}%
\pgfpathlineto{\pgfqpoint{3.659676in}{3.138089in}}%
\pgfpathlineto{\pgfqpoint{3.664225in}{3.141036in}}%
\pgfpathlineto{\pgfqpoint{3.667971in}{3.141047in}}%
\pgfpathlineto{\pgfqpoint{3.671986in}{3.138623in}}%
\pgfpathlineto{\pgfqpoint{3.677070in}{3.132571in}}%
\pgfpathlineto{\pgfqpoint{3.693929in}{3.110115in}}%
\pgfpathlineto{\pgfqpoint{3.697943in}{3.109111in}}%
\pgfpathlineto{\pgfqpoint{3.701690in}{3.110489in}}%
\pgfpathlineto{\pgfqpoint{3.706239in}{3.114872in}}%
\pgfpathlineto{\pgfqpoint{3.713465in}{3.125641in}}%
\pgfpathlineto{\pgfqpoint{3.722563in}{3.138199in}}%
\pgfpathlineto{\pgfqpoint{3.727113in}{3.141069in}}%
\pgfpathlineto{\pgfqpoint{3.730859in}{3.141012in}}%
\pgfpathlineto{\pgfqpoint{3.734873in}{3.138520in}}%
\pgfpathlineto{\pgfqpoint{3.739958in}{3.132407in}}%
\pgfpathlineto{\pgfqpoint{3.756550in}{3.110204in}}%
\pgfpathlineto{\pgfqpoint{3.760564in}{3.109102in}}%
\pgfpathlineto{\pgfqpoint{3.764310in}{3.110390in}}%
\pgfpathlineto{\pgfqpoint{3.768860in}{3.114683in}}%
\pgfpathlineto{\pgfqpoint{3.775817in}{3.124961in}}%
\pgfpathlineto{\pgfqpoint{3.785184in}{3.138051in}}%
\pgfpathlineto{\pgfqpoint{3.789733in}{3.141024in}}%
\pgfpathlineto{\pgfqpoint{3.793479in}{3.141059in}}%
\pgfpathlineto{\pgfqpoint{3.797494in}{3.138658in}}%
\pgfpathlineto{\pgfqpoint{3.802578in}{3.132628in}}%
\pgfpathlineto{\pgfqpoint{3.819437in}{3.110138in}}%
\pgfpathlineto{\pgfqpoint{3.823452in}{3.109108in}}%
\pgfpathlineto{\pgfqpoint{3.827198in}{3.110463in}}%
\pgfpathlineto{\pgfqpoint{3.831747in}{3.114823in}}%
\pgfpathlineto{\pgfqpoint{3.838973in}{3.125577in}}%
\pgfpathlineto{\pgfqpoint{3.848071in}{3.138161in}}%
\pgfpathlineto{\pgfqpoint{3.852621in}{3.141058in}}%
\pgfpathlineto{\pgfqpoint{3.856367in}{3.141025in}}%
\pgfpathlineto{\pgfqpoint{3.860381in}{3.138556in}}%
\pgfpathlineto{\pgfqpoint{3.865466in}{3.132465in}}%
\pgfpathlineto{\pgfqpoint{3.882058in}{3.110227in}}%
\pgfpathlineto{\pgfqpoint{3.886072in}{3.109100in}}%
\pgfpathlineto{\pgfqpoint{3.889818in}{3.110365in}}%
\pgfpathlineto{\pgfqpoint{3.894100in}{3.114313in}}%
\pgfpathlineto{\pgfqpoint{3.900790in}{3.124035in}}%
\pgfpathlineto{\pgfqpoint{3.910692in}{3.138012in}}%
\pgfpathlineto{\pgfqpoint{3.915241in}{3.141011in}}%
\pgfpathlineto{\pgfqpoint{3.918988in}{3.141070in}}%
\pgfpathlineto{\pgfqpoint{3.923002in}{3.138693in}}%
\pgfpathlineto{\pgfqpoint{3.928086in}{3.132685in}}%
\pgfpathlineto{\pgfqpoint{3.944946in}{3.110161in}}%
\pgfpathlineto{\pgfqpoint{3.948960in}{3.109105in}}%
\pgfpathlineto{\pgfqpoint{3.952706in}{3.110437in}}%
\pgfpathlineto{\pgfqpoint{3.957255in}{3.114774in}}%
\pgfpathlineto{\pgfqpoint{3.964481in}{3.125512in}}%
\pgfpathlineto{\pgfqpoint{3.973580in}{3.138122in}}%
\pgfpathlineto{\pgfqpoint{3.978129in}{3.141046in}}%
\pgfpathlineto{\pgfqpoint{3.981875in}{3.141037in}}%
\pgfpathlineto{\pgfqpoint{3.985890in}{3.138592in}}%
\pgfpathlineto{\pgfqpoint{3.990974in}{3.132522in}}%
\pgfpathlineto{\pgfqpoint{4.007566in}{3.110251in}}%
\pgfpathlineto{\pgfqpoint{4.011580in}{3.109098in}}%
\pgfpathlineto{\pgfqpoint{4.015326in}{3.110340in}}%
\pgfpathlineto{\pgfqpoint{4.019608in}{3.114266in}}%
\pgfpathlineto{\pgfqpoint{4.026298in}{3.123971in}}%
\pgfpathlineto{\pgfqpoint{4.036200in}{3.137973in}}%
\pgfpathlineto{\pgfqpoint{4.040749in}{3.140998in}}%
\pgfpathlineto{\pgfqpoint{4.044496in}{3.141082in}}%
\pgfpathlineto{\pgfqpoint{4.048510in}{3.138729in}}%
\pgfpathlineto{\pgfqpoint{4.053594in}{3.132742in}}%
\pgfpathlineto{\pgfqpoint{4.070721in}{3.110033in}}%
\pgfpathlineto{\pgfqpoint{4.074735in}{3.109123in}}%
\pgfpathlineto{\pgfqpoint{4.078482in}{3.110588in}}%
\pgfpathlineto{\pgfqpoint{4.083031in}{3.115057in}}%
\pgfpathlineto{\pgfqpoint{4.090524in}{3.126311in}}%
\pgfpathlineto{\pgfqpoint{4.099088in}{3.138084in}}%
\pgfpathlineto{\pgfqpoint{4.103637in}{3.141034in}}%
\pgfpathlineto{\pgfqpoint{4.107383in}{3.141049in}}%
\pgfpathlineto{\pgfqpoint{4.111398in}{3.138628in}}%
\pgfpathlineto{\pgfqpoint{4.116482in}{3.132579in}}%
\pgfpathlineto{\pgfqpoint{4.133341in}{3.110118in}}%
\pgfpathlineto{\pgfqpoint{4.137356in}{3.109110in}}%
\pgfpathlineto{\pgfqpoint{4.141102in}{3.110485in}}%
\pgfpathlineto{\pgfqpoint{4.145651in}{3.114865in}}%
\pgfpathlineto{\pgfqpoint{4.152877in}{3.125632in}}%
\pgfpathlineto{\pgfqpoint{4.161975in}{3.138194in}}%
\pgfpathlineto{\pgfqpoint{4.166525in}{3.141068in}}%
\pgfpathlineto{\pgfqpoint{4.170271in}{3.141014in}}%
\pgfpathlineto{\pgfqpoint{4.174285in}{3.138525in}}%
\pgfpathlineto{\pgfqpoint{4.179370in}{3.132416in}}%
\pgfpathlineto{\pgfqpoint{4.195962in}{3.110207in}}%
\pgfpathlineto{\pgfqpoint{4.199976in}{3.109101in}}%
\pgfpathlineto{\pgfqpoint{4.203722in}{3.110386in}}%
\pgfpathlineto{\pgfqpoint{4.208272in}{3.114676in}}%
\pgfpathlineto{\pgfqpoint{4.215229in}{3.124952in}}%
\pgfpathlineto{\pgfqpoint{4.224596in}{3.138045in}}%
\pgfpathlineto{\pgfqpoint{4.229145in}{3.141022in}}%
\pgfpathlineto{\pgfqpoint{4.232892in}{3.141060in}}%
\pgfpathlineto{\pgfqpoint{4.236906in}{3.138663in}}%
\pgfpathlineto{\pgfqpoint{4.241990in}{3.132637in}}%
\pgfpathlineto{\pgfqpoint{4.258849in}{3.110141in}}%
\pgfpathlineto{\pgfqpoint{4.262864in}{3.109107in}}%
\pgfpathlineto{\pgfqpoint{4.266610in}{3.110459in}}%
\pgfpathlineto{\pgfqpoint{4.271159in}{3.114816in}}%
\pgfpathlineto{\pgfqpoint{4.278385in}{3.125568in}}%
\pgfpathlineto{\pgfqpoint{4.287484in}{3.138155in}}%
\pgfpathlineto{\pgfqpoint{4.292033in}{3.141056in}}%
\pgfpathlineto{\pgfqpoint{4.295779in}{3.141026in}}%
\pgfpathlineto{\pgfqpoint{4.299793in}{3.138561in}}%
\pgfpathlineto{\pgfqpoint{4.304878in}{3.132473in}}%
\pgfpathlineto{\pgfqpoint{4.321470in}{3.110231in}}%
\pgfpathlineto{\pgfqpoint{4.325484in}{3.109100in}}%
\pgfpathlineto{\pgfqpoint{4.329230in}{3.110361in}}%
\pgfpathlineto{\pgfqpoint{4.333512in}{3.114306in}}%
\pgfpathlineto{\pgfqpoint{4.340202in}{3.124026in}}%
\pgfpathlineto{\pgfqpoint{4.350104in}{3.138006in}}%
\pgfpathlineto{\pgfqpoint{4.354653in}{3.141009in}}%
\pgfpathlineto{\pgfqpoint{4.358400in}{3.141072in}}%
\pgfpathlineto{\pgfqpoint{4.362414in}{3.138699in}}%
\pgfpathlineto{\pgfqpoint{4.367498in}{3.132694in}}%
\pgfpathlineto{\pgfqpoint{4.384358in}{3.110164in}}%
\pgfpathlineto{\pgfqpoint{4.388372in}{3.109105in}}%
\pgfpathlineto{\pgfqpoint{4.392118in}{3.110434in}}%
\pgfpathlineto{\pgfqpoint{4.396668in}{3.114767in}}%
\pgfpathlineto{\pgfqpoint{4.403893in}{3.125503in}}%
\pgfpathlineto{\pgfqpoint{4.412992in}{3.138117in}}%
\pgfpathlineto{\pgfqpoint{4.417541in}{3.141045in}}%
\pgfpathlineto{\pgfqpoint{4.421287in}{3.141039in}}%
\pgfpathlineto{\pgfqpoint{4.425302in}{3.138597in}}%
\pgfpathlineto{\pgfqpoint{4.430386in}{3.132530in}}%
\pgfpathlineto{\pgfqpoint{4.446978in}{3.110255in}}%
\pgfpathlineto{\pgfqpoint{4.450992in}{3.109098in}}%
\pgfpathlineto{\pgfqpoint{4.454738in}{3.110336in}}%
\pgfpathlineto{\pgfqpoint{4.459020in}{3.114259in}}%
\pgfpathlineto{\pgfqpoint{4.465710in}{3.123962in}}%
\pgfpathlineto{\pgfqpoint{4.475879in}{3.138226in}}%
\pgfpathlineto{\pgfqpoint{4.480429in}{3.141078in}}%
\pgfpathlineto{\pgfqpoint{4.484175in}{3.141003in}}%
\pgfpathlineto{\pgfqpoint{4.488189in}{3.138494in}}%
\pgfpathlineto{\pgfqpoint{4.493274in}{3.132366in}}%
\pgfpathlineto{\pgfqpoint{4.509866in}{3.110187in}}%
\pgfpathlineto{\pgfqpoint{4.513880in}{3.109103in}}%
\pgfpathlineto{\pgfqpoint{4.517626in}{3.110408in}}%
\pgfpathlineto{\pgfqpoint{4.522176in}{3.114718in}}%
\pgfpathlineto{\pgfqpoint{4.529133in}{3.125007in}}%
\pgfpathlineto{\pgfqpoint{4.538500in}{3.138078in}}%
\pgfpathlineto{\pgfqpoint{4.543049in}{3.141032in}}%
\pgfpathlineto{\pgfqpoint{4.546796in}{3.141050in}}%
\pgfpathlineto{\pgfqpoint{4.550810in}{3.138633in}}%
\pgfpathlineto{\pgfqpoint{4.555894in}{3.132588in}}%
\pgfpathlineto{\pgfqpoint{4.572753in}{3.110122in}}%
\pgfpathlineto{\pgfqpoint{4.576768in}{3.109110in}}%
\pgfpathlineto{\pgfqpoint{4.580514in}{3.110482in}}%
\pgfpathlineto{\pgfqpoint{4.585063in}{3.114858in}}%
\pgfpathlineto{\pgfqpoint{4.592289in}{3.125623in}}%
\pgfpathlineto{\pgfqpoint{4.601387in}{3.138188in}}%
\pgfpathlineto{\pgfqpoint{4.605937in}{3.141066in}}%
\pgfpathlineto{\pgfqpoint{4.609683in}{3.141016in}}%
\pgfpathlineto{\pgfqpoint{4.613697in}{3.138530in}}%
\pgfpathlineto{\pgfqpoint{4.618782in}{3.132424in}}%
\pgfpathlineto{\pgfqpoint{4.635374in}{3.110211in}}%
\pgfpathlineto{\pgfqpoint{4.639388in}{3.109101in}}%
\pgfpathlineto{\pgfqpoint{4.643134in}{3.110383in}}%
\pgfpathlineto{\pgfqpoint{4.647684in}{3.114669in}}%
\pgfpathlineto{\pgfqpoint{4.654641in}{3.124943in}}%
\pgfpathlineto{\pgfqpoint{4.664008in}{3.138040in}}%
\pgfpathlineto{\pgfqpoint{4.668557in}{3.141020in}}%
\pgfpathlineto{\pgfqpoint{4.672304in}{3.141062in}}%
\pgfpathlineto{\pgfqpoint{4.676318in}{3.138668in}}%
\pgfpathlineto{\pgfqpoint{4.681402in}{3.132645in}}%
\pgfpathlineto{\pgfqpoint{4.698262in}{3.110144in}}%
\pgfpathlineto{\pgfqpoint{4.702276in}{3.109107in}}%
\pgfpathlineto{\pgfqpoint{4.706022in}{3.110456in}}%
\pgfpathlineto{\pgfqpoint{4.710571in}{3.114809in}}%
\pgfpathlineto{\pgfqpoint{4.717797in}{3.125558in}}%
\pgfpathlineto{\pgfqpoint{4.726896in}{3.138150in}}%
\pgfpathlineto{\pgfqpoint{4.731445in}{3.141055in}}%
\pgfpathlineto{\pgfqpoint{4.735191in}{3.141028in}}%
\pgfpathlineto{\pgfqpoint{4.739206in}{3.138566in}}%
\pgfpathlineto{\pgfqpoint{4.744290in}{3.132481in}}%
\pgfpathlineto{\pgfqpoint{4.760882in}{3.110234in}}%
\pgfpathlineto{\pgfqpoint{4.764896in}{3.109099in}}%
\pgfpathlineto{\pgfqpoint{4.768642in}{3.110358in}}%
\pgfpathlineto{\pgfqpoint{4.772924in}{3.114299in}}%
\pgfpathlineto{\pgfqpoint{4.779614in}{3.124017in}}%
\pgfpathlineto{\pgfqpoint{4.789516in}{3.138001in}}%
\pgfpathlineto{\pgfqpoint{4.794065in}{3.141008in}}%
\pgfpathlineto{\pgfqpoint{4.797812in}{3.141074in}}%
\pgfpathlineto{\pgfqpoint{4.801826in}{3.138704in}}%
\pgfpathlineto{\pgfqpoint{4.806910in}{3.132702in}}%
\pgfpathlineto{\pgfqpoint{4.824037in}{3.110018in}}%
\pgfpathlineto{\pgfqpoint{4.828051in}{3.109126in}}%
\pgfpathlineto{\pgfqpoint{4.831798in}{3.110607in}}%
\pgfpathlineto{\pgfqpoint{4.836347in}{3.115092in}}%
\pgfpathlineto{\pgfqpoint{4.843840in}{3.126356in}}%
\pgfpathlineto{\pgfqpoint{4.852404in}{3.138111in}}%
\pgfpathlineto{\pgfqpoint{4.856953in}{3.141043in}}%
\pgfpathlineto{\pgfqpoint{4.860699in}{3.141040in}}%
\pgfpathlineto{\pgfqpoint{4.864714in}{3.138602in}}%
\pgfpathlineto{\pgfqpoint{4.869798in}{3.132539in}}%
\pgfpathlineto{\pgfqpoint{4.886657in}{3.110102in}}%
\pgfpathlineto{\pgfqpoint{4.890672in}{3.109112in}}%
\pgfpathlineto{\pgfqpoint{4.894418in}{3.110504in}}%
\pgfpathlineto{\pgfqpoint{4.898967in}{3.114901in}}%
\pgfpathlineto{\pgfqpoint{4.906193in}{3.125678in}}%
\pgfpathlineto{\pgfqpoint{4.915024in}{3.137961in}}%
\pgfpathlineto{\pgfqpoint{4.919573in}{3.140995in}}%
\pgfpathlineto{\pgfqpoint{4.923320in}{3.141085in}}%
\pgfpathlineto{\pgfqpoint{4.927334in}{3.138739in}}%
\pgfpathlineto{\pgfqpoint{4.932418in}{3.132759in}}%
\pgfpathlineto{\pgfqpoint{4.949545in}{3.110039in}}%
\pgfpathlineto{\pgfqpoint{4.953559in}{3.109122in}}%
\pgfpathlineto{\pgfqpoint{4.957306in}{3.110580in}}%
\pgfpathlineto{\pgfqpoint{4.961855in}{3.115042in}}%
\pgfpathlineto{\pgfqpoint{4.969348in}{3.126292in}}%
\pgfpathlineto{\pgfqpoint{4.977912in}{3.138073in}}%
\pgfpathlineto{\pgfqpoint{4.982461in}{3.141031in}}%
\pgfpathlineto{\pgfqpoint{4.986208in}{3.141052in}}%
\pgfpathlineto{\pgfqpoint{4.990222in}{3.138638in}}%
\pgfpathlineto{\pgfqpoint{4.995306in}{3.132596in}}%
\pgfpathlineto{\pgfqpoint{5.012166in}{3.110125in}}%
\pgfpathlineto{\pgfqpoint{5.016180in}{3.109109in}}%
\pgfpathlineto{\pgfqpoint{5.019926in}{3.110478in}}%
\pgfpathlineto{\pgfqpoint{5.024475in}{3.114851in}}%
\pgfpathlineto{\pgfqpoint{5.031701in}{3.125613in}}%
\pgfpathlineto{\pgfqpoint{5.040800in}{3.138183in}}%
\pgfpathlineto{\pgfqpoint{5.045349in}{3.141065in}}%
\pgfpathlineto{\pgfqpoint{5.049095in}{3.141017in}}%
\pgfpathlineto{\pgfqpoint{5.053109in}{3.138536in}}%
\pgfpathlineto{\pgfqpoint{5.058194in}{3.132432in}}%
\pgfpathlineto{\pgfqpoint{5.074786in}{3.110214in}}%
\pgfpathlineto{\pgfqpoint{5.078800in}{3.109101in}}%
\pgfpathlineto{\pgfqpoint{5.082546in}{3.110379in}}%
\pgfpathlineto{\pgfqpoint{5.087096in}{3.114663in}}%
\pgfpathlineto{\pgfqpoint{5.094053in}{3.124934in}}%
\pgfpathlineto{\pgfqpoint{5.103420in}{3.138034in}}%
\pgfpathlineto{\pgfqpoint{5.107969in}{3.141018in}}%
\pgfpathlineto{\pgfqpoint{5.111716in}{3.141064in}}%
\pgfpathlineto{\pgfqpoint{5.115730in}{3.138673in}}%
\pgfpathlineto{\pgfqpoint{5.120814in}{3.132653in}}%
\pgfpathlineto{\pgfqpoint{5.137674in}{3.110148in}}%
\pgfpathlineto{\pgfqpoint{5.141688in}{3.109107in}}%
\pgfpathlineto{\pgfqpoint{5.145434in}{3.110452in}}%
\pgfpathlineto{\pgfqpoint{5.149984in}{3.114802in}}%
\pgfpathlineto{\pgfqpoint{5.157209in}{3.125549in}}%
\pgfpathlineto{\pgfqpoint{5.166308in}{3.138144in}}%
\pgfpathlineto{\pgfqpoint{5.170857in}{3.141053in}}%
\pgfpathlineto{\pgfqpoint{5.174603in}{3.141030in}}%
\pgfpathlineto{\pgfqpoint{5.178618in}{3.138572in}}%
\pgfpathlineto{\pgfqpoint{5.183702in}{3.132489in}}%
\pgfpathlineto{\pgfqpoint{5.200294in}{3.110238in}}%
\pgfpathlineto{\pgfqpoint{5.204308in}{3.109099in}}%
\pgfpathlineto{\pgfqpoint{5.208054in}{3.110354in}}%
\pgfpathlineto{\pgfqpoint{5.212336in}{3.114293in}}%
\pgfpathlineto{\pgfqpoint{5.219026in}{3.124007in}}%
\pgfpathlineto{\pgfqpoint{5.228928in}{3.137995in}}%
\pgfpathlineto{\pgfqpoint{5.233477in}{3.141006in}}%
\pgfpathlineto{\pgfqpoint{5.237224in}{3.141075in}}%
\pgfpathlineto{\pgfqpoint{5.241238in}{3.138709in}}%
\pgfpathlineto{\pgfqpoint{5.246322in}{3.132710in}}%
\pgfpathlineto{\pgfqpoint{5.263449in}{3.110021in}}%
\pgfpathlineto{\pgfqpoint{5.267463in}{3.109125in}}%
\pgfpathlineto{\pgfqpoint{5.271210in}{3.110603in}}%
\pgfpathlineto{\pgfqpoint{5.275759in}{3.115085in}}%
\pgfpathlineto{\pgfqpoint{5.283252in}{3.126347in}}%
\pgfpathlineto{\pgfqpoint{5.291816in}{3.138106in}}%
\pgfpathlineto{\pgfqpoint{5.296365in}{3.141041in}}%
\pgfpathlineto{\pgfqpoint{5.300112in}{3.141042in}}%
\pgfpathlineto{\pgfqpoint{5.304126in}{3.138607in}}%
\pgfpathlineto{\pgfqpoint{5.309210in}{3.132547in}}%
\pgfpathlineto{\pgfqpoint{5.326069in}{3.110106in}}%
\pgfpathlineto{\pgfqpoint{5.330084in}{3.109112in}}%
\pgfpathlineto{\pgfqpoint{5.333830in}{3.110500in}}%
\pgfpathlineto{\pgfqpoint{5.338379in}{3.114893in}}%
\pgfpathlineto{\pgfqpoint{5.345605in}{3.125669in}}%
\pgfpathlineto{\pgfqpoint{5.354436in}{3.137956in}}%
\pgfpathlineto{\pgfqpoint{5.358985in}{3.140993in}}%
\pgfpathlineto{\pgfqpoint{5.362732in}{3.141086in}}%
\pgfpathlineto{\pgfqpoint{5.366746in}{3.138744in}}%
\pgfpathlineto{\pgfqpoint{5.371830in}{3.132767in}}%
\pgfpathlineto{\pgfqpoint{5.388957in}{3.110043in}}%
\pgfpathlineto{\pgfqpoint{5.392971in}{3.109121in}}%
\pgfpathlineto{\pgfqpoint{5.396718in}{3.110576in}}%
\pgfpathlineto{\pgfqpoint{5.401267in}{3.115035in}}%
\pgfpathlineto{\pgfqpoint{5.408760in}{3.126283in}}%
\pgfpathlineto{\pgfqpoint{5.417324in}{3.138067in}}%
\pgfpathlineto{\pgfqpoint{5.421873in}{3.141029in}}%
\pgfpathlineto{\pgfqpoint{5.425620in}{3.141054in}}%
\pgfpathlineto{\pgfqpoint{5.429634in}{3.138643in}}%
\pgfpathlineto{\pgfqpoint{5.434718in}{3.132604in}}%
\pgfpathlineto{\pgfqpoint{5.451578in}{3.110128in}}%
\pgfpathlineto{\pgfqpoint{5.455592in}{3.109109in}}%
\pgfpathlineto{\pgfqpoint{5.459338in}{3.110474in}}%
\pgfpathlineto{\pgfqpoint{5.463888in}{3.114844in}}%
\pgfpathlineto{\pgfqpoint{5.471113in}{3.125604in}}%
\pgfpathlineto{\pgfqpoint{5.480212in}{3.138177in}}%
\pgfpathlineto{\pgfqpoint{5.484761in}{3.141063in}}%
\pgfpathlineto{\pgfqpoint{5.488507in}{3.141019in}}%
\pgfpathlineto{\pgfqpoint{5.492522in}{3.138541in}}%
\pgfpathlineto{\pgfqpoint{5.497606in}{3.132440in}}%
\pgfpathlineto{\pgfqpoint{5.514198in}{3.110217in}}%
\pgfpathlineto{\pgfqpoint{5.518212in}{3.109101in}}%
\pgfpathlineto{\pgfqpoint{5.521958in}{3.110375in}}%
\pgfpathlineto{\pgfqpoint{5.526508in}{3.114656in}}%
\pgfpathlineto{\pgfqpoint{5.533466in}{3.124925in}}%
\pgfpathlineto{\pgfqpoint{5.542832in}{3.138028in}}%
\pgfpathlineto{\pgfqpoint{5.547381in}{3.141017in}}%
\pgfpathlineto{\pgfqpoint{5.551128in}{3.141065in}}%
\pgfpathlineto{\pgfqpoint{5.555142in}{3.138678in}}%
\pgfpathlineto{\pgfqpoint{5.560226in}{3.132661in}}%
\pgfpathlineto{\pgfqpoint{5.577086in}{3.110151in}}%
\pgfpathlineto{\pgfqpoint{5.581100in}{3.109106in}}%
\pgfpathlineto{\pgfqpoint{5.584846in}{3.110448in}}%
\pgfpathlineto{\pgfqpoint{5.589396in}{3.114795in}}%
\pgfpathlineto{\pgfqpoint{5.596621in}{3.125540in}}%
\pgfpathlineto{\pgfqpoint{5.605720in}{3.138139in}}%
\pgfpathlineto{\pgfqpoint{5.610269in}{3.141051in}}%
\pgfpathlineto{\pgfqpoint{5.614015in}{3.141032in}}%
\pgfpathlineto{\pgfqpoint{5.618030in}{3.138577in}}%
\pgfpathlineto{\pgfqpoint{5.623114in}{3.132498in}}%
\pgfpathlineto{\pgfqpoint{5.639706in}{3.110241in}}%
\pgfpathlineto{\pgfqpoint{5.643720in}{3.109099in}}%
\pgfpathlineto{\pgfqpoint{5.647466in}{3.110350in}}%
\pgfpathlineto{\pgfqpoint{5.651748in}{3.114286in}}%
\pgfpathlineto{\pgfqpoint{5.658438in}{3.123998in}}%
\pgfpathlineto{\pgfqpoint{5.668340in}{3.137989in}}%
\pgfpathlineto{\pgfqpoint{5.672889in}{3.141004in}}%
\pgfpathlineto{\pgfqpoint{5.676636in}{3.141077in}}%
\pgfpathlineto{\pgfqpoint{5.680650in}{3.138714in}}%
\pgfpathlineto{\pgfqpoint{5.685734in}{3.132718in}}%
\pgfpathlineto{\pgfqpoint{5.702861in}{3.110024in}}%
\pgfpathlineto{\pgfqpoint{5.706875in}{3.109124in}}%
\pgfpathlineto{\pgfqpoint{5.710622in}{3.110599in}}%
\pgfpathlineto{\pgfqpoint{5.715171in}{3.115078in}}%
\pgfpathlineto{\pgfqpoint{5.722664in}{3.126338in}}%
\pgfpathlineto{\pgfqpoint{5.731228in}{3.138100in}}%
\pgfpathlineto{\pgfqpoint{5.735777in}{3.141039in}}%
\pgfpathlineto{\pgfqpoint{5.739524in}{3.141044in}}%
\pgfpathlineto{\pgfqpoint{5.743538in}{3.138612in}}%
\pgfpathlineto{\pgfqpoint{5.748622in}{3.132555in}}%
\pgfpathlineto{\pgfqpoint{5.765482in}{3.110109in}}%
\pgfpathlineto{\pgfqpoint{5.769496in}{3.109111in}}%
\pgfpathlineto{\pgfqpoint{5.773242in}{3.110497in}}%
\pgfpathlineto{\pgfqpoint{5.777791in}{3.114886in}}%
\pgfpathlineto{\pgfqpoint{5.785017in}{3.125659in}}%
\pgfpathlineto{\pgfqpoint{5.793848in}{3.137950in}}%
\pgfpathlineto{\pgfqpoint{5.798397in}{3.140991in}}%
\pgfpathlineto{\pgfqpoint{5.802144in}{3.141088in}}%
\pgfpathlineto{\pgfqpoint{5.806158in}{3.138749in}}%
\pgfpathlineto{\pgfqpoint{5.811242in}{3.132775in}}%
\pgfpathlineto{\pgfqpoint{5.828369in}{3.110046in}}%
\pgfpathlineto{\pgfqpoint{5.832383in}{3.109121in}}%
\pgfpathlineto{\pgfqpoint{5.836130in}{3.110572in}}%
\pgfpathlineto{\pgfqpoint{5.840679in}{3.115028in}}%
\pgfpathlineto{\pgfqpoint{5.848172in}{3.126274in}}%
\pgfpathlineto{\pgfqpoint{5.856736in}{3.138062in}}%
\pgfpathlineto{\pgfqpoint{5.861285in}{3.141027in}}%
\pgfpathlineto{\pgfqpoint{5.865032in}{3.141055in}}%
\pgfpathlineto{\pgfqpoint{5.869046in}{3.138648in}}%
\pgfpathlineto{\pgfqpoint{5.874130in}{3.132612in}}%
\pgfpathlineto{\pgfqpoint{5.890990in}{3.110131in}}%
\pgfpathlineto{\pgfqpoint{5.895004in}{3.109109in}}%
\pgfpathlineto{\pgfqpoint{5.898750in}{3.110470in}}%
\pgfpathlineto{\pgfqpoint{5.903300in}{3.114837in}}%
\pgfpathlineto{\pgfqpoint{5.910525in}{3.125595in}}%
\pgfpathlineto{\pgfqpoint{5.919624in}{3.138172in}}%
\pgfpathlineto{\pgfqpoint{5.924173in}{3.141061in}}%
\pgfpathlineto{\pgfqpoint{5.927919in}{3.141021in}}%
\pgfpathlineto{\pgfqpoint{5.931934in}{3.138546in}}%
\pgfpathlineto{\pgfqpoint{5.937018in}{3.132448in}}%
\pgfpathlineto{\pgfqpoint{5.953610in}{3.110221in}}%
\pgfpathlineto{\pgfqpoint{5.957624in}{3.109100in}}%
\pgfpathlineto{\pgfqpoint{5.961370in}{3.110372in}}%
\pgfpathlineto{\pgfqpoint{5.965920in}{3.114649in}}%
\pgfpathlineto{\pgfqpoint{5.972878in}{3.124916in}}%
\pgfpathlineto{\pgfqpoint{5.982244in}{3.138023in}}%
\pgfpathlineto{\pgfqpoint{5.986793in}{3.141015in}}%
\pgfpathlineto{\pgfqpoint{5.990540in}{3.141067in}}%
\pgfpathlineto{\pgfqpoint{5.994554in}{3.138683in}}%
\pgfpathlineto{\pgfqpoint{5.999638in}{3.132669in}}%
\pgfpathlineto{\pgfqpoint{6.016498in}{3.110154in}}%
\pgfpathlineto{\pgfqpoint{6.020512in}{3.109106in}}%
\pgfpathlineto{\pgfqpoint{6.024258in}{3.110445in}}%
\pgfpathlineto{\pgfqpoint{6.028808in}{3.114788in}}%
\pgfpathlineto{\pgfqpoint{6.036033in}{3.125531in}}%
\pgfpathlineto{\pgfqpoint{6.045132in}{3.138133in}}%
\pgfpathlineto{\pgfqpoint{6.049681in}{3.141050in}}%
\pgfpathlineto{\pgfqpoint{6.053428in}{3.141033in}}%
\pgfpathlineto{\pgfqpoint{6.057442in}{3.138582in}}%
\pgfpathlineto{\pgfqpoint{6.062526in}{3.132506in}}%
\pgfpathlineto{\pgfqpoint{6.079118in}{3.110244in}}%
\pgfpathlineto{\pgfqpoint{6.083132in}{3.109099in}}%
\pgfpathlineto{\pgfqpoint{6.086879in}{3.110347in}}%
\pgfpathlineto{\pgfqpoint{6.091160in}{3.114279in}}%
\pgfpathlineto{\pgfqpoint{6.097850in}{3.123989in}}%
\pgfpathlineto{\pgfqpoint{6.107752in}{3.137984in}}%
\pgfpathlineto{\pgfqpoint{6.112301in}{3.141002in}}%
\pgfpathlineto{\pgfqpoint{6.116048in}{3.141078in}}%
\pgfpathlineto{\pgfqpoint{6.120062in}{3.138719in}}%
\pgfpathlineto{\pgfqpoint{6.125146in}{3.132726in}}%
\pgfpathlineto{\pgfqpoint{6.142273in}{3.110027in}}%
\pgfpathlineto{\pgfqpoint{6.146287in}{3.109124in}}%
\pgfpathlineto{\pgfqpoint{6.150034in}{3.110596in}}%
\pgfpathlineto{\pgfqpoint{6.154583in}{3.115071in}}%
\pgfpathlineto{\pgfqpoint{6.162076in}{3.126329in}}%
\pgfpathlineto{\pgfqpoint{6.170640in}{3.138095in}}%
\pgfpathlineto{\pgfqpoint{6.175189in}{3.141038in}}%
\pgfpathlineto{\pgfqpoint{6.178936in}{3.141045in}}%
\pgfpathlineto{\pgfqpoint{6.182950in}{3.138618in}}%
\pgfpathlineto{\pgfqpoint{6.188034in}{3.132563in}}%
\pgfpathlineto{\pgfqpoint{6.204894in}{3.110112in}}%
\pgfpathlineto{\pgfqpoint{6.208908in}{3.109111in}}%
\pgfpathlineto{\pgfqpoint{6.212654in}{3.110493in}}%
\pgfpathlineto{\pgfqpoint{6.217204in}{3.114879in}}%
\pgfpathlineto{\pgfqpoint{6.224429in}{3.125650in}}%
\pgfpathlineto{\pgfqpoint{6.233260in}{3.137945in}}%
\pgfpathlineto{\pgfqpoint{6.237809in}{3.140989in}}%
\pgfpathlineto{\pgfqpoint{6.241556in}{3.141089in}}%
\pgfpathlineto{\pgfqpoint{6.245570in}{3.138754in}}%
\pgfpathlineto{\pgfqpoint{6.250654in}{3.132783in}}%
\pgfpathlineto{\pgfqpoint{6.267781in}{3.110049in}}%
\pgfpathlineto{\pgfqpoint{6.271796in}{3.109120in}}%
\pgfpathlineto{\pgfqpoint{6.275542in}{3.110569in}}%
\pgfpathlineto{\pgfqpoint{6.280091in}{3.115021in}}%
\pgfpathlineto{\pgfqpoint{6.287584in}{3.126265in}}%
\pgfpathlineto{\pgfqpoint{6.296148in}{3.138056in}}%
\pgfpathlineto{\pgfqpoint{6.300697in}{3.141025in}}%
\pgfpathlineto{\pgfqpoint{6.304444in}{3.141057in}}%
\pgfpathlineto{\pgfqpoint{6.308458in}{3.138653in}}%
\pgfpathlineto{\pgfqpoint{6.313542in}{3.132620in}}%
\pgfpathlineto{\pgfqpoint{6.330402in}{3.110135in}}%
\pgfpathlineto{\pgfqpoint{6.334416in}{3.109108in}}%
\pgfpathlineto{\pgfqpoint{6.338162in}{3.110467in}}%
\pgfpathlineto{\pgfqpoint{6.342712in}{3.114830in}}%
\pgfpathlineto{\pgfqpoint{6.349937in}{3.125586in}}%
\pgfpathlineto{\pgfqpoint{6.359036in}{3.138166in}}%
\pgfpathlineto{\pgfqpoint{6.363585in}{3.141060in}}%
\pgfpathlineto{\pgfqpoint{6.367332in}{3.141023in}}%
\pgfpathlineto{\pgfqpoint{6.371346in}{3.138551in}}%
\pgfpathlineto{\pgfqpoint{6.376430in}{3.132457in}}%
\pgfpathlineto{\pgfqpoint{6.393022in}{3.110224in}}%
\pgfpathlineto{\pgfqpoint{6.397036in}{3.109100in}}%
\pgfpathlineto{\pgfqpoint{6.400782in}{3.110368in}}%
\pgfpathlineto{\pgfqpoint{6.405064in}{3.114320in}}%
\pgfpathlineto{\pgfqpoint{6.411754in}{3.124044in}}%
\pgfpathlineto{\pgfqpoint{6.421656in}{3.138017in}}%
\pgfpathlineto{\pgfqpoint{6.426205in}{3.141013in}}%
\pgfpathlineto{\pgfqpoint{6.429952in}{3.141069in}}%
\pgfpathlineto{\pgfqpoint{6.433966in}{3.138688in}}%
\pgfpathlineto{\pgfqpoint{6.439050in}{3.132677in}}%
\pgfpathlineto{\pgfqpoint{6.455910in}{3.110157in}}%
\pgfpathlineto{\pgfqpoint{6.459924in}{3.109106in}}%
\pgfpathlineto{\pgfqpoint{6.463670in}{3.110441in}}%
\pgfpathlineto{\pgfqpoint{6.468220in}{3.114781in}}%
\pgfpathlineto{\pgfqpoint{6.475445in}{3.125522in}}%
\pgfpathlineto{\pgfqpoint{6.484544in}{3.138128in}}%
\pgfpathlineto{\pgfqpoint{6.489093in}{3.141048in}}%
\pgfpathlineto{\pgfqpoint{6.492840in}{3.141035in}}%
\pgfpathlineto{\pgfqpoint{6.496854in}{3.138587in}}%
\pgfpathlineto{\pgfqpoint{6.501938in}{3.132514in}}%
\pgfpathlineto{\pgfqpoint{6.518530in}{3.110248in}}%
\pgfpathlineto{\pgfqpoint{6.522544in}{3.109099in}}%
\pgfpathlineto{\pgfqpoint{6.526291in}{3.110343in}}%
\pgfpathlineto{\pgfqpoint{6.530572in}{3.114272in}}%
\pgfpathlineto{\pgfqpoint{6.537262in}{3.123980in}}%
\pgfpathlineto{\pgfqpoint{6.547164in}{3.137978in}}%
\pgfpathlineto{\pgfqpoint{6.551713in}{3.141000in}}%
\pgfpathlineto{\pgfqpoint{6.555460in}{3.141080in}}%
\pgfpathlineto{\pgfqpoint{6.559474in}{3.138724in}}%
\pgfpathlineto{\pgfqpoint{6.564558in}{3.132734in}}%
\pgfpathlineto{\pgfqpoint{6.581685in}{3.110030in}}%
\pgfpathlineto{\pgfqpoint{6.585699in}{3.109123in}}%
\pgfpathlineto{\pgfqpoint{6.589446in}{3.110592in}}%
\pgfpathlineto{\pgfqpoint{6.593995in}{3.115064in}}%
\pgfpathlineto{\pgfqpoint{6.601488in}{3.126320in}}%
\pgfpathlineto{\pgfqpoint{6.610052in}{3.138089in}}%
\pgfpathlineto{\pgfqpoint{6.614601in}{3.141036in}}%
\pgfpathlineto{\pgfqpoint{6.618348in}{3.141047in}}%
\pgfpathlineto{\pgfqpoint{6.622362in}{3.138623in}}%
\pgfpathlineto{\pgfqpoint{6.627446in}{3.132571in}}%
\pgfpathlineto{\pgfqpoint{6.644306in}{3.110115in}}%
\pgfpathlineto{\pgfqpoint{6.648320in}{3.109111in}}%
\pgfpathlineto{\pgfqpoint{6.652066in}{3.110489in}}%
\pgfpathlineto{\pgfqpoint{6.656616in}{3.114872in}}%
\pgfpathlineto{\pgfqpoint{6.663306in}{3.124778in}}%
\pgfpathlineto{\pgfqpoint{6.663306in}{3.124778in}}%
\pgfusepath{stroke}%
\end{pgfscope}%
\begin{pgfscope}%
\pgfpathrectangle{\pgfqpoint{0.467797in}{2.292089in}}{\pgfqpoint{6.490533in}{1.666241in}}%
\pgfusepath{clip}%
\pgfsetrectcap%
\pgfsetroundjoin%
\pgfsetlinewidth{1.505625pt}%
\definecolor{currentstroke}{rgb}{0.580392,0.403922,0.741176}%
\pgfsetstrokecolor{currentstroke}%
\pgfsetdash{}{0pt}%
\pgfpathmoveto{\pgfqpoint{0.762821in}{3.125209in}}%
\pgfpathlineto{\pgfqpoint{0.771920in}{3.137777in}}%
\pgfpathlineto{\pgfqpoint{0.776469in}{3.140500in}}%
\pgfpathlineto{\pgfqpoint{0.780215in}{3.140207in}}%
\pgfpathlineto{\pgfqpoint{0.784230in}{3.137390in}}%
\pgfpathlineto{\pgfqpoint{0.789582in}{3.130496in}}%
\pgfpathlineto{\pgfqpoint{0.803230in}{3.111613in}}%
\pgfpathlineto{\pgfqpoint{0.807511in}{3.109771in}}%
\pgfpathlineto{\pgfqpoint{0.811258in}{3.110648in}}%
\pgfpathlineto{\pgfqpoint{0.815540in}{3.114318in}}%
\pgfpathlineto{\pgfqpoint{0.821695in}{3.123058in}}%
\pgfpathlineto{\pgfqpoint{0.832131in}{3.137777in}}%
\pgfpathlineto{\pgfqpoint{0.836681in}{3.140500in}}%
\pgfpathlineto{\pgfqpoint{0.840427in}{3.140207in}}%
\pgfpathlineto{\pgfqpoint{0.844441in}{3.137390in}}%
\pgfpathlineto{\pgfqpoint{0.849793in}{3.130496in}}%
\pgfpathlineto{\pgfqpoint{0.863441in}{3.111613in}}%
\pgfpathlineto{\pgfqpoint{0.867723in}{3.109771in}}%
\pgfpathlineto{\pgfqpoint{0.871470in}{3.110648in}}%
\pgfpathlineto{\pgfqpoint{0.875751in}{3.114318in}}%
\pgfpathlineto{\pgfqpoint{0.881906in}{3.123058in}}%
\pgfpathlineto{\pgfqpoint{0.892343in}{3.137777in}}%
\pgfpathlineto{\pgfqpoint{0.896892in}{3.140500in}}%
\pgfpathlineto{\pgfqpoint{0.900639in}{3.140207in}}%
\pgfpathlineto{\pgfqpoint{0.904653in}{3.137390in}}%
\pgfpathlineto{\pgfqpoint{0.910005in}{3.130496in}}%
\pgfpathlineto{\pgfqpoint{0.923653in}{3.111613in}}%
\pgfpathlineto{\pgfqpoint{0.927935in}{3.109771in}}%
\pgfpathlineto{\pgfqpoint{0.931681in}{3.110648in}}%
\pgfpathlineto{\pgfqpoint{0.935963in}{3.114318in}}%
\pgfpathlineto{\pgfqpoint{0.942118in}{3.123058in}}%
\pgfpathlineto{\pgfqpoint{0.952555in}{3.137777in}}%
\pgfpathlineto{\pgfqpoint{0.957104in}{3.140500in}}%
\pgfpathlineto{\pgfqpoint{0.960851in}{3.140207in}}%
\pgfpathlineto{\pgfqpoint{0.964865in}{3.137390in}}%
\pgfpathlineto{\pgfqpoint{0.970217in}{3.130496in}}%
\pgfpathlineto{\pgfqpoint{0.983865in}{3.111613in}}%
\pgfpathlineto{\pgfqpoint{0.988147in}{3.109771in}}%
\pgfpathlineto{\pgfqpoint{0.991893in}{3.110648in}}%
\pgfpathlineto{\pgfqpoint{0.996175in}{3.114318in}}%
\pgfpathlineto{\pgfqpoint{1.002330in}{3.123058in}}%
\pgfpathlineto{\pgfqpoint{1.012767in}{3.137777in}}%
\pgfpathlineto{\pgfqpoint{1.017316in}{3.140500in}}%
\pgfpathlineto{\pgfqpoint{1.021062in}{3.140207in}}%
\pgfpathlineto{\pgfqpoint{1.025077in}{3.137390in}}%
\pgfpathlineto{\pgfqpoint{1.030429in}{3.130496in}}%
\pgfpathlineto{\pgfqpoint{1.044077in}{3.111613in}}%
\pgfpathlineto{\pgfqpoint{1.048358in}{3.109771in}}%
\pgfpathlineto{\pgfqpoint{1.052105in}{3.110648in}}%
\pgfpathlineto{\pgfqpoint{1.056387in}{3.114318in}}%
\pgfpathlineto{\pgfqpoint{1.062542in}{3.123058in}}%
\pgfpathlineto{\pgfqpoint{1.072978in}{3.137777in}}%
\pgfpathlineto{\pgfqpoint{1.077528in}{3.140500in}}%
\pgfpathlineto{\pgfqpoint{1.081274in}{3.140207in}}%
\pgfpathlineto{\pgfqpoint{1.085288in}{3.137390in}}%
\pgfpathlineto{\pgfqpoint{1.090640in}{3.130496in}}%
\pgfpathlineto{\pgfqpoint{1.104288in}{3.111613in}}%
\pgfpathlineto{\pgfqpoint{1.108570in}{3.109771in}}%
\pgfpathlineto{\pgfqpoint{1.112317in}{3.110648in}}%
\pgfpathlineto{\pgfqpoint{1.116598in}{3.114318in}}%
\pgfpathlineto{\pgfqpoint{1.122753in}{3.123058in}}%
\pgfpathlineto{\pgfqpoint{1.133190in}{3.137777in}}%
\pgfpathlineto{\pgfqpoint{1.137739in}{3.140500in}}%
\pgfpathlineto{\pgfqpoint{1.141486in}{3.140207in}}%
\pgfpathlineto{\pgfqpoint{1.145500in}{3.137390in}}%
\pgfpathlineto{\pgfqpoint{1.150852in}{3.130496in}}%
\pgfpathlineto{\pgfqpoint{1.164500in}{3.111613in}}%
\pgfpathlineto{\pgfqpoint{1.168782in}{3.109771in}}%
\pgfpathlineto{\pgfqpoint{1.172528in}{3.110648in}}%
\pgfpathlineto{\pgfqpoint{1.176810in}{3.114318in}}%
\pgfpathlineto{\pgfqpoint{1.182965in}{3.123058in}}%
\pgfpathlineto{\pgfqpoint{1.193402in}{3.137777in}}%
\pgfpathlineto{\pgfqpoint{1.197951in}{3.140500in}}%
\pgfpathlineto{\pgfqpoint{1.201698in}{3.140207in}}%
\pgfpathlineto{\pgfqpoint{1.205712in}{3.137390in}}%
\pgfpathlineto{\pgfqpoint{1.211064in}{3.130496in}}%
\pgfpathlineto{\pgfqpoint{1.224712in}{3.111613in}}%
\pgfpathlineto{\pgfqpoint{1.228994in}{3.109771in}}%
\pgfpathlineto{\pgfqpoint{1.232740in}{3.110648in}}%
\pgfpathlineto{\pgfqpoint{1.237022in}{3.114318in}}%
\pgfpathlineto{\pgfqpoint{1.243177in}{3.123058in}}%
\pgfpathlineto{\pgfqpoint{1.253614in}{3.137777in}}%
\pgfpathlineto{\pgfqpoint{1.258163in}{3.140500in}}%
\pgfpathlineto{\pgfqpoint{1.261909in}{3.140207in}}%
\pgfpathlineto{\pgfqpoint{1.265924in}{3.137390in}}%
\pgfpathlineto{\pgfqpoint{1.271276in}{3.130496in}}%
\pgfpathlineto{\pgfqpoint{1.284924in}{3.111613in}}%
\pgfpathlineto{\pgfqpoint{1.289205in}{3.109771in}}%
\pgfpathlineto{\pgfqpoint{1.292952in}{3.110648in}}%
\pgfpathlineto{\pgfqpoint{1.297234in}{3.114318in}}%
\pgfpathlineto{\pgfqpoint{1.303389in}{3.123058in}}%
\pgfpathlineto{\pgfqpoint{1.313825in}{3.137777in}}%
\pgfpathlineto{\pgfqpoint{1.318375in}{3.140500in}}%
\pgfpathlineto{\pgfqpoint{1.322121in}{3.140207in}}%
\pgfpathlineto{\pgfqpoint{1.326135in}{3.137390in}}%
\pgfpathlineto{\pgfqpoint{1.331488in}{3.130496in}}%
\pgfpathlineto{\pgfqpoint{1.345136in}{3.111613in}}%
\pgfpathlineto{\pgfqpoint{1.349417in}{3.109771in}}%
\pgfpathlineto{\pgfqpoint{1.353164in}{3.110648in}}%
\pgfpathlineto{\pgfqpoint{1.357445in}{3.114318in}}%
\pgfpathlineto{\pgfqpoint{1.363600in}{3.123058in}}%
\pgfpathlineto{\pgfqpoint{1.374037in}{3.137777in}}%
\pgfpathlineto{\pgfqpoint{1.378587in}{3.140500in}}%
\pgfpathlineto{\pgfqpoint{1.382333in}{3.140207in}}%
\pgfpathlineto{\pgfqpoint{1.386347in}{3.137390in}}%
\pgfpathlineto{\pgfqpoint{1.391699in}{3.130496in}}%
\pgfpathlineto{\pgfqpoint{1.405347in}{3.111613in}}%
\pgfpathlineto{\pgfqpoint{1.409629in}{3.109771in}}%
\pgfpathlineto{\pgfqpoint{1.413376in}{3.110648in}}%
\pgfpathlineto{\pgfqpoint{1.417657in}{3.114318in}}%
\pgfpathlineto{\pgfqpoint{1.423812in}{3.123058in}}%
\pgfpathlineto{\pgfqpoint{1.434249in}{3.137777in}}%
\pgfpathlineto{\pgfqpoint{1.438798in}{3.140500in}}%
\pgfpathlineto{\pgfqpoint{1.442545in}{3.140207in}}%
\pgfpathlineto{\pgfqpoint{1.446559in}{3.137390in}}%
\pgfpathlineto{\pgfqpoint{1.451911in}{3.130496in}}%
\pgfpathlineto{\pgfqpoint{1.465559in}{3.111613in}}%
\pgfpathlineto{\pgfqpoint{1.469841in}{3.109771in}}%
\pgfpathlineto{\pgfqpoint{1.473587in}{3.110648in}}%
\pgfpathlineto{\pgfqpoint{1.477869in}{3.114318in}}%
\pgfpathlineto{\pgfqpoint{1.484024in}{3.123058in}}%
\pgfpathlineto{\pgfqpoint{1.494461in}{3.137777in}}%
\pgfpathlineto{\pgfqpoint{1.499010in}{3.140500in}}%
\pgfpathlineto{\pgfqpoint{1.502757in}{3.140207in}}%
\pgfpathlineto{\pgfqpoint{1.506771in}{3.137390in}}%
\pgfpathlineto{\pgfqpoint{1.512123in}{3.130496in}}%
\pgfpathlineto{\pgfqpoint{1.525771in}{3.111613in}}%
\pgfpathlineto{\pgfqpoint{1.530053in}{3.109771in}}%
\pgfpathlineto{\pgfqpoint{1.533799in}{3.110648in}}%
\pgfpathlineto{\pgfqpoint{1.538081in}{3.114318in}}%
\pgfpathlineto{\pgfqpoint{1.544236in}{3.123058in}}%
\pgfpathlineto{\pgfqpoint{1.554672in}{3.137777in}}%
\pgfpathlineto{\pgfqpoint{1.559222in}{3.140500in}}%
\pgfpathlineto{\pgfqpoint{1.562968in}{3.140207in}}%
\pgfpathlineto{\pgfqpoint{1.566982in}{3.137390in}}%
\pgfpathlineto{\pgfqpoint{1.572335in}{3.130496in}}%
\pgfpathlineto{\pgfqpoint{1.585983in}{3.111613in}}%
\pgfpathlineto{\pgfqpoint{1.590264in}{3.109771in}}%
\pgfpathlineto{\pgfqpoint{1.594011in}{3.110648in}}%
\pgfpathlineto{\pgfqpoint{1.598293in}{3.114318in}}%
\pgfpathlineto{\pgfqpoint{1.604448in}{3.123058in}}%
\pgfpathlineto{\pgfqpoint{1.614884in}{3.137777in}}%
\pgfpathlineto{\pgfqpoint{1.619434in}{3.140500in}}%
\pgfpathlineto{\pgfqpoint{1.623180in}{3.140207in}}%
\pgfpathlineto{\pgfqpoint{1.627194in}{3.137390in}}%
\pgfpathlineto{\pgfqpoint{1.632546in}{3.130496in}}%
\pgfpathlineto{\pgfqpoint{1.646194in}{3.111613in}}%
\pgfpathlineto{\pgfqpoint{1.650476in}{3.109771in}}%
\pgfpathlineto{\pgfqpoint{1.654223in}{3.110648in}}%
\pgfpathlineto{\pgfqpoint{1.658504in}{3.114318in}}%
\pgfpathlineto{\pgfqpoint{1.664659in}{3.123058in}}%
\pgfpathlineto{\pgfqpoint{1.675096in}{3.137777in}}%
\pgfpathlineto{\pgfqpoint{1.679645in}{3.140500in}}%
\pgfpathlineto{\pgfqpoint{1.683392in}{3.140207in}}%
\pgfpathlineto{\pgfqpoint{1.687406in}{3.137390in}}%
\pgfpathlineto{\pgfqpoint{1.692758in}{3.130496in}}%
\pgfpathlineto{\pgfqpoint{1.706406in}{3.111613in}}%
\pgfpathlineto{\pgfqpoint{1.710688in}{3.109771in}}%
\pgfpathlineto{\pgfqpoint{1.714434in}{3.110648in}}%
\pgfpathlineto{\pgfqpoint{1.718716in}{3.114318in}}%
\pgfpathlineto{\pgfqpoint{1.724871in}{3.123058in}}%
\pgfpathlineto{\pgfqpoint{1.735308in}{3.137777in}}%
\pgfpathlineto{\pgfqpoint{1.739857in}{3.140500in}}%
\pgfpathlineto{\pgfqpoint{1.743604in}{3.140207in}}%
\pgfpathlineto{\pgfqpoint{1.747618in}{3.137390in}}%
\pgfpathlineto{\pgfqpoint{1.752970in}{3.130496in}}%
\pgfpathlineto{\pgfqpoint{1.766618in}{3.111613in}}%
\pgfpathlineto{\pgfqpoint{1.770900in}{3.109771in}}%
\pgfpathlineto{\pgfqpoint{1.774646in}{3.110648in}}%
\pgfpathlineto{\pgfqpoint{1.778928in}{3.114318in}}%
\pgfpathlineto{\pgfqpoint{1.785083in}{3.123058in}}%
\pgfpathlineto{\pgfqpoint{1.795519in}{3.137777in}}%
\pgfpathlineto{\pgfqpoint{1.800069in}{3.140500in}}%
\pgfpathlineto{\pgfqpoint{1.803815in}{3.140207in}}%
\pgfpathlineto{\pgfqpoint{1.807829in}{3.137390in}}%
\pgfpathlineto{\pgfqpoint{1.813182in}{3.130496in}}%
\pgfpathlineto{\pgfqpoint{1.826830in}{3.111613in}}%
\pgfpathlineto{\pgfqpoint{1.831111in}{3.109771in}}%
\pgfpathlineto{\pgfqpoint{1.834858in}{3.110648in}}%
\pgfpathlineto{\pgfqpoint{1.839140in}{3.114318in}}%
\pgfpathlineto{\pgfqpoint{1.845295in}{3.123058in}}%
\pgfpathlineto{\pgfqpoint{1.855731in}{3.137777in}}%
\pgfpathlineto{\pgfqpoint{1.860281in}{3.140500in}}%
\pgfpathlineto{\pgfqpoint{1.864027in}{3.140207in}}%
\pgfpathlineto{\pgfqpoint{1.868041in}{3.137390in}}%
\pgfpathlineto{\pgfqpoint{1.873393in}{3.130496in}}%
\pgfpathlineto{\pgfqpoint{1.887041in}{3.111613in}}%
\pgfpathlineto{\pgfqpoint{1.891323in}{3.109771in}}%
\pgfpathlineto{\pgfqpoint{1.895070in}{3.110648in}}%
\pgfpathlineto{\pgfqpoint{1.899351in}{3.114318in}}%
\pgfpathlineto{\pgfqpoint{1.905506in}{3.123058in}}%
\pgfpathlineto{\pgfqpoint{1.915943in}{3.137777in}}%
\pgfpathlineto{\pgfqpoint{1.920492in}{3.140500in}}%
\pgfpathlineto{\pgfqpoint{1.924239in}{3.140207in}}%
\pgfpathlineto{\pgfqpoint{1.928253in}{3.137390in}}%
\pgfpathlineto{\pgfqpoint{1.933605in}{3.130496in}}%
\pgfpathlineto{\pgfqpoint{1.947253in}{3.111613in}}%
\pgfpathlineto{\pgfqpoint{1.951535in}{3.109771in}}%
\pgfpathlineto{\pgfqpoint{1.955281in}{3.110648in}}%
\pgfpathlineto{\pgfqpoint{1.959563in}{3.114318in}}%
\pgfpathlineto{\pgfqpoint{1.965718in}{3.123058in}}%
\pgfpathlineto{\pgfqpoint{1.976155in}{3.137777in}}%
\pgfpathlineto{\pgfqpoint{1.980704in}{3.140500in}}%
\pgfpathlineto{\pgfqpoint{1.984451in}{3.140207in}}%
\pgfpathlineto{\pgfqpoint{1.988465in}{3.137390in}}%
\pgfpathlineto{\pgfqpoint{1.993817in}{3.130496in}}%
\pgfpathlineto{\pgfqpoint{2.007465in}{3.111613in}}%
\pgfpathlineto{\pgfqpoint{2.011747in}{3.109771in}}%
\pgfpathlineto{\pgfqpoint{2.015493in}{3.110648in}}%
\pgfpathlineto{\pgfqpoint{2.019775in}{3.114318in}}%
\pgfpathlineto{\pgfqpoint{2.025930in}{3.123058in}}%
\pgfpathlineto{\pgfqpoint{2.036367in}{3.137777in}}%
\pgfpathlineto{\pgfqpoint{2.040916in}{3.140500in}}%
\pgfpathlineto{\pgfqpoint{2.044662in}{3.140207in}}%
\pgfpathlineto{\pgfqpoint{2.048676in}{3.137390in}}%
\pgfpathlineto{\pgfqpoint{2.054029in}{3.130496in}}%
\pgfpathlineto{\pgfqpoint{2.067677in}{3.111613in}}%
\pgfpathlineto{\pgfqpoint{2.071958in}{3.109771in}}%
\pgfpathlineto{\pgfqpoint{2.075705in}{3.110648in}}%
\pgfpathlineto{\pgfqpoint{2.079987in}{3.114318in}}%
\pgfpathlineto{\pgfqpoint{2.086142in}{3.123058in}}%
\pgfpathlineto{\pgfqpoint{2.096578in}{3.137777in}}%
\pgfpathlineto{\pgfqpoint{2.101128in}{3.140500in}}%
\pgfpathlineto{\pgfqpoint{2.104874in}{3.140207in}}%
\pgfpathlineto{\pgfqpoint{2.108888in}{3.137390in}}%
\pgfpathlineto{\pgfqpoint{2.114240in}{3.130496in}}%
\pgfpathlineto{\pgfqpoint{2.127888in}{3.111613in}}%
\pgfpathlineto{\pgfqpoint{2.132170in}{3.109771in}}%
\pgfpathlineto{\pgfqpoint{2.135917in}{3.110648in}}%
\pgfpathlineto{\pgfqpoint{2.140198in}{3.114318in}}%
\pgfpathlineto{\pgfqpoint{2.146353in}{3.123058in}}%
\pgfpathlineto{\pgfqpoint{2.156790in}{3.137777in}}%
\pgfpathlineto{\pgfqpoint{2.161339in}{3.140500in}}%
\pgfpathlineto{\pgfqpoint{2.165086in}{3.140207in}}%
\pgfpathlineto{\pgfqpoint{2.169100in}{3.137390in}}%
\pgfpathlineto{\pgfqpoint{2.174452in}{3.130496in}}%
\pgfpathlineto{\pgfqpoint{2.188100in}{3.111613in}}%
\pgfpathlineto{\pgfqpoint{2.192382in}{3.109771in}}%
\pgfpathlineto{\pgfqpoint{2.196128in}{3.110648in}}%
\pgfpathlineto{\pgfqpoint{2.200410in}{3.114318in}}%
\pgfpathlineto{\pgfqpoint{2.206565in}{3.123058in}}%
\pgfpathlineto{\pgfqpoint{2.217002in}{3.137777in}}%
\pgfpathlineto{\pgfqpoint{2.221551in}{3.140500in}}%
\pgfpathlineto{\pgfqpoint{2.225298in}{3.140207in}}%
\pgfpathlineto{\pgfqpoint{2.229312in}{3.137390in}}%
\pgfpathlineto{\pgfqpoint{2.234664in}{3.130496in}}%
\pgfpathlineto{\pgfqpoint{2.248312in}{3.111613in}}%
\pgfpathlineto{\pgfqpoint{2.252594in}{3.109771in}}%
\pgfpathlineto{\pgfqpoint{2.256340in}{3.110648in}}%
\pgfpathlineto{\pgfqpoint{2.260622in}{3.114318in}}%
\pgfpathlineto{\pgfqpoint{2.266777in}{3.123058in}}%
\pgfpathlineto{\pgfqpoint{2.277214in}{3.137777in}}%
\pgfpathlineto{\pgfqpoint{2.281763in}{3.140500in}}%
\pgfpathlineto{\pgfqpoint{2.285509in}{3.140207in}}%
\pgfpathlineto{\pgfqpoint{2.289524in}{3.137390in}}%
\pgfpathlineto{\pgfqpoint{2.294876in}{3.130496in}}%
\pgfpathlineto{\pgfqpoint{2.308524in}{3.111613in}}%
\pgfpathlineto{\pgfqpoint{2.312805in}{3.109771in}}%
\pgfpathlineto{\pgfqpoint{2.316552in}{3.110648in}}%
\pgfpathlineto{\pgfqpoint{2.320834in}{3.114318in}}%
\pgfpathlineto{\pgfqpoint{2.326989in}{3.123058in}}%
\pgfpathlineto{\pgfqpoint{2.337425in}{3.137777in}}%
\pgfpathlineto{\pgfqpoint{2.341975in}{3.140500in}}%
\pgfpathlineto{\pgfqpoint{2.345721in}{3.140207in}}%
\pgfpathlineto{\pgfqpoint{2.349735in}{3.137390in}}%
\pgfpathlineto{\pgfqpoint{2.355087in}{3.130496in}}%
\pgfpathlineto{\pgfqpoint{2.368735in}{3.111613in}}%
\pgfpathlineto{\pgfqpoint{2.373017in}{3.109771in}}%
\pgfpathlineto{\pgfqpoint{2.376764in}{3.110648in}}%
\pgfpathlineto{\pgfqpoint{2.381045in}{3.114318in}}%
\pgfpathlineto{\pgfqpoint{2.387200in}{3.123058in}}%
\pgfpathlineto{\pgfqpoint{2.397637in}{3.137777in}}%
\pgfpathlineto{\pgfqpoint{2.402186in}{3.140500in}}%
\pgfpathlineto{\pgfqpoint{2.405933in}{3.140207in}}%
\pgfpathlineto{\pgfqpoint{2.409947in}{3.137390in}}%
\pgfpathlineto{\pgfqpoint{2.415299in}{3.130496in}}%
\pgfpathlineto{\pgfqpoint{2.428947in}{3.111613in}}%
\pgfpathlineto{\pgfqpoint{2.433229in}{3.109771in}}%
\pgfpathlineto{\pgfqpoint{2.436975in}{3.110648in}}%
\pgfpathlineto{\pgfqpoint{2.441257in}{3.114318in}}%
\pgfpathlineto{\pgfqpoint{2.447412in}{3.123058in}}%
\pgfpathlineto{\pgfqpoint{2.457849in}{3.137777in}}%
\pgfpathlineto{\pgfqpoint{2.462398in}{3.140500in}}%
\pgfpathlineto{\pgfqpoint{2.466145in}{3.140207in}}%
\pgfpathlineto{\pgfqpoint{2.470159in}{3.137390in}}%
\pgfpathlineto{\pgfqpoint{2.475511in}{3.130496in}}%
\pgfpathlineto{\pgfqpoint{2.489159in}{3.111613in}}%
\pgfpathlineto{\pgfqpoint{2.493441in}{3.109771in}}%
\pgfpathlineto{\pgfqpoint{2.497187in}{3.110648in}}%
\pgfpathlineto{\pgfqpoint{2.501469in}{3.114318in}}%
\pgfpathlineto{\pgfqpoint{2.507624in}{3.123058in}}%
\pgfpathlineto{\pgfqpoint{2.518061in}{3.137777in}}%
\pgfpathlineto{\pgfqpoint{2.522610in}{3.140500in}}%
\pgfpathlineto{\pgfqpoint{2.526356in}{3.140207in}}%
\pgfpathlineto{\pgfqpoint{2.530371in}{3.137390in}}%
\pgfpathlineto{\pgfqpoint{2.535723in}{3.130496in}}%
\pgfpathlineto{\pgfqpoint{2.549371in}{3.111613in}}%
\pgfpathlineto{\pgfqpoint{2.553652in}{3.109771in}}%
\pgfpathlineto{\pgfqpoint{2.557399in}{3.110648in}}%
\pgfpathlineto{\pgfqpoint{2.561681in}{3.114318in}}%
\pgfpathlineto{\pgfqpoint{2.567836in}{3.123058in}}%
\pgfpathlineto{\pgfqpoint{2.578272in}{3.137777in}}%
\pgfpathlineto{\pgfqpoint{2.582822in}{3.140500in}}%
\pgfpathlineto{\pgfqpoint{2.586568in}{3.140207in}}%
\pgfpathlineto{\pgfqpoint{2.590582in}{3.137390in}}%
\pgfpathlineto{\pgfqpoint{2.595934in}{3.130496in}}%
\pgfpathlineto{\pgfqpoint{2.609582in}{3.111613in}}%
\pgfpathlineto{\pgfqpoint{2.613864in}{3.109771in}}%
\pgfpathlineto{\pgfqpoint{2.617611in}{3.110648in}}%
\pgfpathlineto{\pgfqpoint{2.621892in}{3.114318in}}%
\pgfpathlineto{\pgfqpoint{2.628047in}{3.123058in}}%
\pgfpathlineto{\pgfqpoint{2.638484in}{3.137777in}}%
\pgfpathlineto{\pgfqpoint{2.643033in}{3.140500in}}%
\pgfpathlineto{\pgfqpoint{2.646780in}{3.140207in}}%
\pgfpathlineto{\pgfqpoint{2.650794in}{3.137390in}}%
\pgfpathlineto{\pgfqpoint{2.656146in}{3.130496in}}%
\pgfpathlineto{\pgfqpoint{2.669794in}{3.111613in}}%
\pgfpathlineto{\pgfqpoint{2.674076in}{3.109771in}}%
\pgfpathlineto{\pgfqpoint{2.677822in}{3.110648in}}%
\pgfpathlineto{\pgfqpoint{2.682104in}{3.114318in}}%
\pgfpathlineto{\pgfqpoint{2.688259in}{3.123058in}}%
\pgfpathlineto{\pgfqpoint{2.698696in}{3.137777in}}%
\pgfpathlineto{\pgfqpoint{2.703245in}{3.140500in}}%
\pgfpathlineto{\pgfqpoint{2.706992in}{3.140207in}}%
\pgfpathlineto{\pgfqpoint{2.711006in}{3.137390in}}%
\pgfpathlineto{\pgfqpoint{2.716358in}{3.130496in}}%
\pgfpathlineto{\pgfqpoint{2.730006in}{3.111613in}}%
\pgfpathlineto{\pgfqpoint{2.734288in}{3.109771in}}%
\pgfpathlineto{\pgfqpoint{2.738034in}{3.110648in}}%
\pgfpathlineto{\pgfqpoint{2.742316in}{3.114318in}}%
\pgfpathlineto{\pgfqpoint{2.748471in}{3.123058in}}%
\pgfpathlineto{\pgfqpoint{2.758908in}{3.137777in}}%
\pgfpathlineto{\pgfqpoint{2.763457in}{3.140500in}}%
\pgfpathlineto{\pgfqpoint{2.767203in}{3.140207in}}%
\pgfpathlineto{\pgfqpoint{2.771218in}{3.137390in}}%
\pgfpathlineto{\pgfqpoint{2.776570in}{3.130496in}}%
\pgfpathlineto{\pgfqpoint{2.790218in}{3.111613in}}%
\pgfpathlineto{\pgfqpoint{2.794499in}{3.109771in}}%
\pgfpathlineto{\pgfqpoint{2.798246in}{3.110648in}}%
\pgfpathlineto{\pgfqpoint{2.802528in}{3.114318in}}%
\pgfpathlineto{\pgfqpoint{2.808683in}{3.123058in}}%
\pgfpathlineto{\pgfqpoint{2.819119in}{3.137777in}}%
\pgfpathlineto{\pgfqpoint{2.823669in}{3.140500in}}%
\pgfpathlineto{\pgfqpoint{2.827415in}{3.140207in}}%
\pgfpathlineto{\pgfqpoint{2.831429in}{3.137390in}}%
\pgfpathlineto{\pgfqpoint{2.836782in}{3.130496in}}%
\pgfpathlineto{\pgfqpoint{2.850430in}{3.111613in}}%
\pgfpathlineto{\pgfqpoint{2.854711in}{3.109771in}}%
\pgfpathlineto{\pgfqpoint{2.858458in}{3.110648in}}%
\pgfpathlineto{\pgfqpoint{2.862739in}{3.114318in}}%
\pgfpathlineto{\pgfqpoint{2.868894in}{3.123058in}}%
\pgfpathlineto{\pgfqpoint{2.879331in}{3.137777in}}%
\pgfpathlineto{\pgfqpoint{2.883880in}{3.140500in}}%
\pgfpathlineto{\pgfqpoint{2.887627in}{3.140207in}}%
\pgfpathlineto{\pgfqpoint{2.891641in}{3.137390in}}%
\pgfpathlineto{\pgfqpoint{2.896993in}{3.130496in}}%
\pgfpathlineto{\pgfqpoint{2.910641in}{3.111613in}}%
\pgfpathlineto{\pgfqpoint{2.914923in}{3.109771in}}%
\pgfpathlineto{\pgfqpoint{2.918670in}{3.110648in}}%
\pgfpathlineto{\pgfqpoint{2.922951in}{3.114318in}}%
\pgfpathlineto{\pgfqpoint{2.929106in}{3.123058in}}%
\pgfpathlineto{\pgfqpoint{2.939543in}{3.137777in}}%
\pgfpathlineto{\pgfqpoint{2.944092in}{3.140500in}}%
\pgfpathlineto{\pgfqpoint{2.947839in}{3.140207in}}%
\pgfpathlineto{\pgfqpoint{2.951853in}{3.137390in}}%
\pgfpathlineto{\pgfqpoint{2.957205in}{3.130496in}}%
\pgfpathlineto{\pgfqpoint{2.970853in}{3.111613in}}%
\pgfpathlineto{\pgfqpoint{2.975135in}{3.109771in}}%
\pgfpathlineto{\pgfqpoint{2.978881in}{3.110648in}}%
\pgfpathlineto{\pgfqpoint{2.983163in}{3.114318in}}%
\pgfpathlineto{\pgfqpoint{2.989318in}{3.123058in}}%
\pgfpathlineto{\pgfqpoint{2.999755in}{3.137777in}}%
\pgfpathlineto{\pgfqpoint{3.004304in}{3.140500in}}%
\pgfpathlineto{\pgfqpoint{3.008051in}{3.140207in}}%
\pgfpathlineto{\pgfqpoint{3.012065in}{3.137390in}}%
\pgfpathlineto{\pgfqpoint{3.017417in}{3.130496in}}%
\pgfpathlineto{\pgfqpoint{3.031065in}{3.111613in}}%
\pgfpathlineto{\pgfqpoint{3.035347in}{3.109771in}}%
\pgfpathlineto{\pgfqpoint{3.039093in}{3.110648in}}%
\pgfpathlineto{\pgfqpoint{3.043375in}{3.114318in}}%
\pgfpathlineto{\pgfqpoint{3.049530in}{3.123058in}}%
\pgfpathlineto{\pgfqpoint{3.059966in}{3.137777in}}%
\pgfpathlineto{\pgfqpoint{3.064516in}{3.140500in}}%
\pgfpathlineto{\pgfqpoint{3.068262in}{3.140207in}}%
\pgfpathlineto{\pgfqpoint{3.072276in}{3.137390in}}%
\pgfpathlineto{\pgfqpoint{3.077629in}{3.130496in}}%
\pgfpathlineto{\pgfqpoint{3.091277in}{3.111613in}}%
\pgfpathlineto{\pgfqpoint{3.095558in}{3.109771in}}%
\pgfpathlineto{\pgfqpoint{3.099305in}{3.110648in}}%
\pgfpathlineto{\pgfqpoint{3.103587in}{3.114318in}}%
\pgfpathlineto{\pgfqpoint{3.109742in}{3.123058in}}%
\pgfpathlineto{\pgfqpoint{3.120178in}{3.137777in}}%
\pgfpathlineto{\pgfqpoint{3.124728in}{3.140500in}}%
\pgfpathlineto{\pgfqpoint{3.128474in}{3.140207in}}%
\pgfpathlineto{\pgfqpoint{3.132488in}{3.137390in}}%
\pgfpathlineto{\pgfqpoint{3.137840in}{3.130496in}}%
\pgfpathlineto{\pgfqpoint{3.151488in}{3.111613in}}%
\pgfpathlineto{\pgfqpoint{3.155770in}{3.109771in}}%
\pgfpathlineto{\pgfqpoint{3.159517in}{3.110648in}}%
\pgfpathlineto{\pgfqpoint{3.163798in}{3.114318in}}%
\pgfpathlineto{\pgfqpoint{3.169953in}{3.123058in}}%
\pgfpathlineto{\pgfqpoint{3.180390in}{3.137777in}}%
\pgfpathlineto{\pgfqpoint{3.184939in}{3.140500in}}%
\pgfpathlineto{\pgfqpoint{3.188686in}{3.140207in}}%
\pgfpathlineto{\pgfqpoint{3.192700in}{3.137390in}}%
\pgfpathlineto{\pgfqpoint{3.198052in}{3.130496in}}%
\pgfpathlineto{\pgfqpoint{3.211700in}{3.111613in}}%
\pgfpathlineto{\pgfqpoint{3.215982in}{3.109771in}}%
\pgfpathlineto{\pgfqpoint{3.219728in}{3.110648in}}%
\pgfpathlineto{\pgfqpoint{3.224010in}{3.114318in}}%
\pgfpathlineto{\pgfqpoint{3.230165in}{3.123058in}}%
\pgfpathlineto{\pgfqpoint{3.240602in}{3.137777in}}%
\pgfpathlineto{\pgfqpoint{3.245151in}{3.140500in}}%
\pgfpathlineto{\pgfqpoint{3.248898in}{3.140207in}}%
\pgfpathlineto{\pgfqpoint{3.252912in}{3.137390in}}%
\pgfpathlineto{\pgfqpoint{3.258264in}{3.130496in}}%
\pgfpathlineto{\pgfqpoint{3.271912in}{3.111613in}}%
\pgfpathlineto{\pgfqpoint{3.276194in}{3.109771in}}%
\pgfpathlineto{\pgfqpoint{3.279940in}{3.110648in}}%
\pgfpathlineto{\pgfqpoint{3.284222in}{3.114318in}}%
\pgfpathlineto{\pgfqpoint{3.290377in}{3.123058in}}%
\pgfpathlineto{\pgfqpoint{3.300813in}{3.137777in}}%
\pgfpathlineto{\pgfqpoint{3.305363in}{3.140500in}}%
\pgfpathlineto{\pgfqpoint{3.309109in}{3.140207in}}%
\pgfpathlineto{\pgfqpoint{3.313123in}{3.137390in}}%
\pgfpathlineto{\pgfqpoint{3.318476in}{3.130496in}}%
\pgfpathlineto{\pgfqpoint{3.332124in}{3.111613in}}%
\pgfpathlineto{\pgfqpoint{3.336405in}{3.109771in}}%
\pgfpathlineto{\pgfqpoint{3.340152in}{3.110648in}}%
\pgfpathlineto{\pgfqpoint{3.344434in}{3.114318in}}%
\pgfpathlineto{\pgfqpoint{3.350589in}{3.123058in}}%
\pgfpathlineto{\pgfqpoint{3.361025in}{3.137777in}}%
\pgfpathlineto{\pgfqpoint{3.365575in}{3.140500in}}%
\pgfpathlineto{\pgfqpoint{3.369321in}{3.140207in}}%
\pgfpathlineto{\pgfqpoint{3.373335in}{3.137390in}}%
\pgfpathlineto{\pgfqpoint{3.378687in}{3.130496in}}%
\pgfpathlineto{\pgfqpoint{3.392335in}{3.111613in}}%
\pgfpathlineto{\pgfqpoint{3.396617in}{3.109771in}}%
\pgfpathlineto{\pgfqpoint{3.400364in}{3.110648in}}%
\pgfpathlineto{\pgfqpoint{3.404645in}{3.114318in}}%
\pgfpathlineto{\pgfqpoint{3.410800in}{3.123058in}}%
\pgfpathlineto{\pgfqpoint{3.421237in}{3.137777in}}%
\pgfpathlineto{\pgfqpoint{3.425786in}{3.140500in}}%
\pgfpathlineto{\pgfqpoint{3.429533in}{3.140207in}}%
\pgfpathlineto{\pgfqpoint{3.433547in}{3.137390in}}%
\pgfpathlineto{\pgfqpoint{3.438899in}{3.130496in}}%
\pgfpathlineto{\pgfqpoint{3.452547in}{3.111613in}}%
\pgfpathlineto{\pgfqpoint{3.456829in}{3.109771in}}%
\pgfpathlineto{\pgfqpoint{3.460575in}{3.110648in}}%
\pgfpathlineto{\pgfqpoint{3.464857in}{3.114318in}}%
\pgfpathlineto{\pgfqpoint{3.471012in}{3.123058in}}%
\pgfpathlineto{\pgfqpoint{3.481449in}{3.137777in}}%
\pgfpathlineto{\pgfqpoint{3.485998in}{3.140500in}}%
\pgfpathlineto{\pgfqpoint{3.489745in}{3.140207in}}%
\pgfpathlineto{\pgfqpoint{3.493759in}{3.137390in}}%
\pgfpathlineto{\pgfqpoint{3.499111in}{3.130496in}}%
\pgfpathlineto{\pgfqpoint{3.512759in}{3.111613in}}%
\pgfpathlineto{\pgfqpoint{3.517041in}{3.109771in}}%
\pgfpathlineto{\pgfqpoint{3.520787in}{3.110648in}}%
\pgfpathlineto{\pgfqpoint{3.525069in}{3.114318in}}%
\pgfpathlineto{\pgfqpoint{3.531224in}{3.123058in}}%
\pgfpathlineto{\pgfqpoint{3.541661in}{3.137777in}}%
\pgfpathlineto{\pgfqpoint{3.546210in}{3.140500in}}%
\pgfpathlineto{\pgfqpoint{3.549956in}{3.140207in}}%
\pgfpathlineto{\pgfqpoint{3.553970in}{3.137390in}}%
\pgfpathlineto{\pgfqpoint{3.559323in}{3.130496in}}%
\pgfpathlineto{\pgfqpoint{3.572971in}{3.111613in}}%
\pgfpathlineto{\pgfqpoint{3.577252in}{3.109771in}}%
\pgfpathlineto{\pgfqpoint{3.580999in}{3.110648in}}%
\pgfpathlineto{\pgfqpoint{3.585281in}{3.114318in}}%
\pgfpathlineto{\pgfqpoint{3.591436in}{3.123058in}}%
\pgfpathlineto{\pgfqpoint{3.601872in}{3.137777in}}%
\pgfpathlineto{\pgfqpoint{3.606422in}{3.140500in}}%
\pgfpathlineto{\pgfqpoint{3.610168in}{3.140207in}}%
\pgfpathlineto{\pgfqpoint{3.614182in}{3.137390in}}%
\pgfpathlineto{\pgfqpoint{3.619534in}{3.130496in}}%
\pgfpathlineto{\pgfqpoint{3.633182in}{3.111613in}}%
\pgfpathlineto{\pgfqpoint{3.637464in}{3.109771in}}%
\pgfpathlineto{\pgfqpoint{3.641211in}{3.110648in}}%
\pgfpathlineto{\pgfqpoint{3.645492in}{3.114318in}}%
\pgfpathlineto{\pgfqpoint{3.651647in}{3.123058in}}%
\pgfpathlineto{\pgfqpoint{3.662084in}{3.137777in}}%
\pgfpathlineto{\pgfqpoint{3.666633in}{3.140500in}}%
\pgfpathlineto{\pgfqpoint{3.670380in}{3.140207in}}%
\pgfpathlineto{\pgfqpoint{3.674394in}{3.137390in}}%
\pgfpathlineto{\pgfqpoint{3.679746in}{3.130496in}}%
\pgfpathlineto{\pgfqpoint{3.693394in}{3.111613in}}%
\pgfpathlineto{\pgfqpoint{3.697676in}{3.109771in}}%
\pgfpathlineto{\pgfqpoint{3.701422in}{3.110648in}}%
\pgfpathlineto{\pgfqpoint{3.705704in}{3.114318in}}%
\pgfpathlineto{\pgfqpoint{3.711859in}{3.123058in}}%
\pgfpathlineto{\pgfqpoint{3.722296in}{3.137777in}}%
\pgfpathlineto{\pgfqpoint{3.726845in}{3.140500in}}%
\pgfpathlineto{\pgfqpoint{3.730592in}{3.140207in}}%
\pgfpathlineto{\pgfqpoint{3.734606in}{3.137390in}}%
\pgfpathlineto{\pgfqpoint{3.739958in}{3.130496in}}%
\pgfpathlineto{\pgfqpoint{3.753606in}{3.111613in}}%
\pgfpathlineto{\pgfqpoint{3.757888in}{3.109771in}}%
\pgfpathlineto{\pgfqpoint{3.761634in}{3.110648in}}%
\pgfpathlineto{\pgfqpoint{3.765916in}{3.114318in}}%
\pgfpathlineto{\pgfqpoint{3.772071in}{3.123058in}}%
\pgfpathlineto{\pgfqpoint{3.782508in}{3.137777in}}%
\pgfpathlineto{\pgfqpoint{3.787057in}{3.140500in}}%
\pgfpathlineto{\pgfqpoint{3.790803in}{3.140207in}}%
\pgfpathlineto{\pgfqpoint{3.794818in}{3.137390in}}%
\pgfpathlineto{\pgfqpoint{3.800170in}{3.130496in}}%
\pgfpathlineto{\pgfqpoint{3.813818in}{3.111613in}}%
\pgfpathlineto{\pgfqpoint{3.818099in}{3.109771in}}%
\pgfpathlineto{\pgfqpoint{3.821846in}{3.110648in}}%
\pgfpathlineto{\pgfqpoint{3.826128in}{3.114318in}}%
\pgfpathlineto{\pgfqpoint{3.832283in}{3.123058in}}%
\pgfpathlineto{\pgfqpoint{3.842719in}{3.137777in}}%
\pgfpathlineto{\pgfqpoint{3.847269in}{3.140500in}}%
\pgfpathlineto{\pgfqpoint{3.851015in}{3.140207in}}%
\pgfpathlineto{\pgfqpoint{3.855029in}{3.137390in}}%
\pgfpathlineto{\pgfqpoint{3.860381in}{3.130496in}}%
\pgfpathlineto{\pgfqpoint{3.874029in}{3.111613in}}%
\pgfpathlineto{\pgfqpoint{3.878311in}{3.109771in}}%
\pgfpathlineto{\pgfqpoint{3.882058in}{3.110648in}}%
\pgfpathlineto{\pgfqpoint{3.886339in}{3.114318in}}%
\pgfpathlineto{\pgfqpoint{3.892494in}{3.123058in}}%
\pgfpathlineto{\pgfqpoint{3.902931in}{3.137777in}}%
\pgfpathlineto{\pgfqpoint{3.907480in}{3.140500in}}%
\pgfpathlineto{\pgfqpoint{3.911227in}{3.140207in}}%
\pgfpathlineto{\pgfqpoint{3.915241in}{3.137390in}}%
\pgfpathlineto{\pgfqpoint{3.920593in}{3.130496in}}%
\pgfpathlineto{\pgfqpoint{3.934241in}{3.111613in}}%
\pgfpathlineto{\pgfqpoint{3.938523in}{3.109771in}}%
\pgfpathlineto{\pgfqpoint{3.942269in}{3.110648in}}%
\pgfpathlineto{\pgfqpoint{3.946551in}{3.114318in}}%
\pgfpathlineto{\pgfqpoint{3.952706in}{3.123058in}}%
\pgfpathlineto{\pgfqpoint{3.963143in}{3.137777in}}%
\pgfpathlineto{\pgfqpoint{3.967692in}{3.140500in}}%
\pgfpathlineto{\pgfqpoint{3.971439in}{3.140207in}}%
\pgfpathlineto{\pgfqpoint{3.975453in}{3.137390in}}%
\pgfpathlineto{\pgfqpoint{3.980805in}{3.130496in}}%
\pgfpathlineto{\pgfqpoint{3.994453in}{3.111613in}}%
\pgfpathlineto{\pgfqpoint{3.998735in}{3.109771in}}%
\pgfpathlineto{\pgfqpoint{4.002481in}{3.110648in}}%
\pgfpathlineto{\pgfqpoint{4.006763in}{3.114318in}}%
\pgfpathlineto{\pgfqpoint{4.012918in}{3.123058in}}%
\pgfpathlineto{\pgfqpoint{4.023355in}{3.137777in}}%
\pgfpathlineto{\pgfqpoint{4.027904in}{3.140500in}}%
\pgfpathlineto{\pgfqpoint{4.031650in}{3.140207in}}%
\pgfpathlineto{\pgfqpoint{4.035665in}{3.137390in}}%
\pgfpathlineto{\pgfqpoint{4.041017in}{3.130496in}}%
\pgfpathlineto{\pgfqpoint{4.054665in}{3.111613in}}%
\pgfpathlineto{\pgfqpoint{4.058946in}{3.109771in}}%
\pgfpathlineto{\pgfqpoint{4.062693in}{3.110648in}}%
\pgfpathlineto{\pgfqpoint{4.066975in}{3.114318in}}%
\pgfpathlineto{\pgfqpoint{4.073130in}{3.123058in}}%
\pgfpathlineto{\pgfqpoint{4.083566in}{3.137777in}}%
\pgfpathlineto{\pgfqpoint{4.088116in}{3.140500in}}%
\pgfpathlineto{\pgfqpoint{4.091862in}{3.140207in}}%
\pgfpathlineto{\pgfqpoint{4.095876in}{3.137390in}}%
\pgfpathlineto{\pgfqpoint{4.101228in}{3.130496in}}%
\pgfpathlineto{\pgfqpoint{4.114876in}{3.111613in}}%
\pgfpathlineto{\pgfqpoint{4.119158in}{3.109771in}}%
\pgfpathlineto{\pgfqpoint{4.122905in}{3.110648in}}%
\pgfpathlineto{\pgfqpoint{4.127186in}{3.114318in}}%
\pgfpathlineto{\pgfqpoint{4.133341in}{3.123058in}}%
\pgfpathlineto{\pgfqpoint{4.143778in}{3.137777in}}%
\pgfpathlineto{\pgfqpoint{4.148327in}{3.140500in}}%
\pgfpathlineto{\pgfqpoint{4.152074in}{3.140207in}}%
\pgfpathlineto{\pgfqpoint{4.156088in}{3.137390in}}%
\pgfpathlineto{\pgfqpoint{4.161440in}{3.130496in}}%
\pgfpathlineto{\pgfqpoint{4.175088in}{3.111613in}}%
\pgfpathlineto{\pgfqpoint{4.179370in}{3.109771in}}%
\pgfpathlineto{\pgfqpoint{4.183116in}{3.110648in}}%
\pgfpathlineto{\pgfqpoint{4.187398in}{3.114318in}}%
\pgfpathlineto{\pgfqpoint{4.193553in}{3.123058in}}%
\pgfpathlineto{\pgfqpoint{4.203990in}{3.137777in}}%
\pgfpathlineto{\pgfqpoint{4.208539in}{3.140500in}}%
\pgfpathlineto{\pgfqpoint{4.212286in}{3.140207in}}%
\pgfpathlineto{\pgfqpoint{4.216300in}{3.137390in}}%
\pgfpathlineto{\pgfqpoint{4.221652in}{3.130496in}}%
\pgfpathlineto{\pgfqpoint{4.235300in}{3.111613in}}%
\pgfpathlineto{\pgfqpoint{4.239582in}{3.109771in}}%
\pgfpathlineto{\pgfqpoint{4.243328in}{3.110648in}}%
\pgfpathlineto{\pgfqpoint{4.247610in}{3.114318in}}%
\pgfpathlineto{\pgfqpoint{4.253765in}{3.123058in}}%
\pgfpathlineto{\pgfqpoint{4.264202in}{3.137777in}}%
\pgfpathlineto{\pgfqpoint{4.268751in}{3.140500in}}%
\pgfpathlineto{\pgfqpoint{4.272497in}{3.140207in}}%
\pgfpathlineto{\pgfqpoint{4.276512in}{3.137390in}}%
\pgfpathlineto{\pgfqpoint{4.281864in}{3.130496in}}%
\pgfpathlineto{\pgfqpoint{4.295512in}{3.111613in}}%
\pgfpathlineto{\pgfqpoint{4.299793in}{3.109771in}}%
\pgfpathlineto{\pgfqpoint{4.303540in}{3.110648in}}%
\pgfpathlineto{\pgfqpoint{4.307822in}{3.114318in}}%
\pgfpathlineto{\pgfqpoint{4.313977in}{3.123058in}}%
\pgfpathlineto{\pgfqpoint{4.324413in}{3.137777in}}%
\pgfpathlineto{\pgfqpoint{4.328963in}{3.140500in}}%
\pgfpathlineto{\pgfqpoint{4.332709in}{3.140207in}}%
\pgfpathlineto{\pgfqpoint{4.336723in}{3.137390in}}%
\pgfpathlineto{\pgfqpoint{4.342076in}{3.130496in}}%
\pgfpathlineto{\pgfqpoint{4.355724in}{3.111613in}}%
\pgfpathlineto{\pgfqpoint{4.360005in}{3.109771in}}%
\pgfpathlineto{\pgfqpoint{4.363752in}{3.110648in}}%
\pgfpathlineto{\pgfqpoint{4.368033in}{3.114318in}}%
\pgfpathlineto{\pgfqpoint{4.374188in}{3.123058in}}%
\pgfpathlineto{\pgfqpoint{4.384625in}{3.137777in}}%
\pgfpathlineto{\pgfqpoint{4.389174in}{3.140500in}}%
\pgfpathlineto{\pgfqpoint{4.392921in}{3.140207in}}%
\pgfpathlineto{\pgfqpoint{4.396935in}{3.137390in}}%
\pgfpathlineto{\pgfqpoint{4.402287in}{3.130496in}}%
\pgfpathlineto{\pgfqpoint{4.415935in}{3.111613in}}%
\pgfpathlineto{\pgfqpoint{4.420217in}{3.109771in}}%
\pgfpathlineto{\pgfqpoint{4.423964in}{3.110648in}}%
\pgfpathlineto{\pgfqpoint{4.428245in}{3.114318in}}%
\pgfpathlineto{\pgfqpoint{4.434400in}{3.123058in}}%
\pgfpathlineto{\pgfqpoint{4.444837in}{3.137777in}}%
\pgfpathlineto{\pgfqpoint{4.449386in}{3.140500in}}%
\pgfpathlineto{\pgfqpoint{4.453133in}{3.140207in}}%
\pgfpathlineto{\pgfqpoint{4.457147in}{3.137390in}}%
\pgfpathlineto{\pgfqpoint{4.462499in}{3.130496in}}%
\pgfpathlineto{\pgfqpoint{4.476147in}{3.111613in}}%
\pgfpathlineto{\pgfqpoint{4.480429in}{3.109771in}}%
\pgfpathlineto{\pgfqpoint{4.484175in}{3.110648in}}%
\pgfpathlineto{\pgfqpoint{4.488457in}{3.114318in}}%
\pgfpathlineto{\pgfqpoint{4.494612in}{3.123058in}}%
\pgfpathlineto{\pgfqpoint{4.505049in}{3.137777in}}%
\pgfpathlineto{\pgfqpoint{4.509598in}{3.140500in}}%
\pgfpathlineto{\pgfqpoint{4.513345in}{3.140207in}}%
\pgfpathlineto{\pgfqpoint{4.517359in}{3.137390in}}%
\pgfpathlineto{\pgfqpoint{4.522711in}{3.130496in}}%
\pgfpathlineto{\pgfqpoint{4.536359in}{3.111613in}}%
\pgfpathlineto{\pgfqpoint{4.540641in}{3.109771in}}%
\pgfpathlineto{\pgfqpoint{4.544387in}{3.110648in}}%
\pgfpathlineto{\pgfqpoint{4.548669in}{3.114318in}}%
\pgfpathlineto{\pgfqpoint{4.554824in}{3.123058in}}%
\pgfpathlineto{\pgfqpoint{4.565260in}{3.137777in}}%
\pgfpathlineto{\pgfqpoint{4.569810in}{3.140500in}}%
\pgfpathlineto{\pgfqpoint{4.573556in}{3.140207in}}%
\pgfpathlineto{\pgfqpoint{4.577570in}{3.137390in}}%
\pgfpathlineto{\pgfqpoint{4.582923in}{3.130496in}}%
\pgfpathlineto{\pgfqpoint{4.596571in}{3.111613in}}%
\pgfpathlineto{\pgfqpoint{4.600852in}{3.109771in}}%
\pgfpathlineto{\pgfqpoint{4.604599in}{3.110648in}}%
\pgfpathlineto{\pgfqpoint{4.608881in}{3.114318in}}%
\pgfpathlineto{\pgfqpoint{4.615035in}{3.123058in}}%
\pgfpathlineto{\pgfqpoint{4.625472in}{3.137777in}}%
\pgfpathlineto{\pgfqpoint{4.630022in}{3.140500in}}%
\pgfpathlineto{\pgfqpoint{4.633768in}{3.140207in}}%
\pgfpathlineto{\pgfqpoint{4.637782in}{3.137390in}}%
\pgfpathlineto{\pgfqpoint{4.643134in}{3.130496in}}%
\pgfpathlineto{\pgfqpoint{4.656782in}{3.111613in}}%
\pgfpathlineto{\pgfqpoint{4.661064in}{3.109771in}}%
\pgfpathlineto{\pgfqpoint{4.664811in}{3.110648in}}%
\pgfpathlineto{\pgfqpoint{4.669092in}{3.114318in}}%
\pgfpathlineto{\pgfqpoint{4.675247in}{3.123058in}}%
\pgfpathlineto{\pgfqpoint{4.685684in}{3.137777in}}%
\pgfpathlineto{\pgfqpoint{4.690233in}{3.140500in}}%
\pgfpathlineto{\pgfqpoint{4.693980in}{3.140207in}}%
\pgfpathlineto{\pgfqpoint{4.697994in}{3.137390in}}%
\pgfpathlineto{\pgfqpoint{4.703346in}{3.130496in}}%
\pgfpathlineto{\pgfqpoint{4.716994in}{3.111613in}}%
\pgfpathlineto{\pgfqpoint{4.721276in}{3.109771in}}%
\pgfpathlineto{\pgfqpoint{4.725022in}{3.110648in}}%
\pgfpathlineto{\pgfqpoint{4.729304in}{3.114318in}}%
\pgfpathlineto{\pgfqpoint{4.735459in}{3.123058in}}%
\pgfpathlineto{\pgfqpoint{4.745896in}{3.137777in}}%
\pgfpathlineto{\pgfqpoint{4.750445in}{3.140500in}}%
\pgfpathlineto{\pgfqpoint{4.754192in}{3.140207in}}%
\pgfpathlineto{\pgfqpoint{4.758206in}{3.137390in}}%
\pgfpathlineto{\pgfqpoint{4.763558in}{3.130496in}}%
\pgfpathlineto{\pgfqpoint{4.777206in}{3.111613in}}%
\pgfpathlineto{\pgfqpoint{4.781488in}{3.109771in}}%
\pgfpathlineto{\pgfqpoint{4.785234in}{3.110648in}}%
\pgfpathlineto{\pgfqpoint{4.789516in}{3.114318in}}%
\pgfpathlineto{\pgfqpoint{4.795671in}{3.123058in}}%
\pgfpathlineto{\pgfqpoint{4.806107in}{3.137777in}}%
\pgfpathlineto{\pgfqpoint{4.810657in}{3.140500in}}%
\pgfpathlineto{\pgfqpoint{4.814403in}{3.140207in}}%
\pgfpathlineto{\pgfqpoint{4.818417in}{3.137390in}}%
\pgfpathlineto{\pgfqpoint{4.823770in}{3.130496in}}%
\pgfpathlineto{\pgfqpoint{4.837418in}{3.111613in}}%
\pgfpathlineto{\pgfqpoint{4.841699in}{3.109771in}}%
\pgfpathlineto{\pgfqpoint{4.845446in}{3.110648in}}%
\pgfpathlineto{\pgfqpoint{4.849728in}{3.114318in}}%
\pgfpathlineto{\pgfqpoint{4.855883in}{3.123058in}}%
\pgfpathlineto{\pgfqpoint{4.866319in}{3.137777in}}%
\pgfpathlineto{\pgfqpoint{4.870869in}{3.140500in}}%
\pgfpathlineto{\pgfqpoint{4.874615in}{3.140207in}}%
\pgfpathlineto{\pgfqpoint{4.878629in}{3.137390in}}%
\pgfpathlineto{\pgfqpoint{4.883981in}{3.130496in}}%
\pgfpathlineto{\pgfqpoint{4.897629in}{3.111613in}}%
\pgfpathlineto{\pgfqpoint{4.901911in}{3.109771in}}%
\pgfpathlineto{\pgfqpoint{4.905658in}{3.110648in}}%
\pgfpathlineto{\pgfqpoint{4.909939in}{3.114318in}}%
\pgfpathlineto{\pgfqpoint{4.916094in}{3.123058in}}%
\pgfpathlineto{\pgfqpoint{4.926531in}{3.137777in}}%
\pgfpathlineto{\pgfqpoint{4.931080in}{3.140500in}}%
\pgfpathlineto{\pgfqpoint{4.934827in}{3.140207in}}%
\pgfpathlineto{\pgfqpoint{4.938841in}{3.137390in}}%
\pgfpathlineto{\pgfqpoint{4.944193in}{3.130496in}}%
\pgfpathlineto{\pgfqpoint{4.957841in}{3.111613in}}%
\pgfpathlineto{\pgfqpoint{4.962123in}{3.109771in}}%
\pgfpathlineto{\pgfqpoint{4.965869in}{3.110648in}}%
\pgfpathlineto{\pgfqpoint{4.970151in}{3.114318in}}%
\pgfpathlineto{\pgfqpoint{4.976306in}{3.123058in}}%
\pgfpathlineto{\pgfqpoint{4.986743in}{3.137777in}}%
\pgfpathlineto{\pgfqpoint{4.991292in}{3.140500in}}%
\pgfpathlineto{\pgfqpoint{4.995039in}{3.140207in}}%
\pgfpathlineto{\pgfqpoint{4.999053in}{3.137390in}}%
\pgfpathlineto{\pgfqpoint{5.004405in}{3.130496in}}%
\pgfpathlineto{\pgfqpoint{5.018053in}{3.111613in}}%
\pgfpathlineto{\pgfqpoint{5.022335in}{3.109771in}}%
\pgfpathlineto{\pgfqpoint{5.026081in}{3.110648in}}%
\pgfpathlineto{\pgfqpoint{5.030363in}{3.114318in}}%
\pgfpathlineto{\pgfqpoint{5.036518in}{3.123058in}}%
\pgfpathlineto{\pgfqpoint{5.046955in}{3.137777in}}%
\pgfpathlineto{\pgfqpoint{5.051504in}{3.140500in}}%
\pgfpathlineto{\pgfqpoint{5.055250in}{3.140207in}}%
\pgfpathlineto{\pgfqpoint{5.059264in}{3.137390in}}%
\pgfpathlineto{\pgfqpoint{5.064617in}{3.130496in}}%
\pgfpathlineto{\pgfqpoint{5.078265in}{3.111613in}}%
\pgfpathlineto{\pgfqpoint{5.082546in}{3.109771in}}%
\pgfpathlineto{\pgfqpoint{5.086293in}{3.110648in}}%
\pgfpathlineto{\pgfqpoint{5.090575in}{3.114318in}}%
\pgfpathlineto{\pgfqpoint{5.096730in}{3.123058in}}%
\pgfpathlineto{\pgfqpoint{5.107166in}{3.137777in}}%
\pgfpathlineto{\pgfqpoint{5.111716in}{3.140500in}}%
\pgfpathlineto{\pgfqpoint{5.115462in}{3.140207in}}%
\pgfpathlineto{\pgfqpoint{5.119476in}{3.137390in}}%
\pgfpathlineto{\pgfqpoint{5.124828in}{3.130496in}}%
\pgfpathlineto{\pgfqpoint{5.138476in}{3.111613in}}%
\pgfpathlineto{\pgfqpoint{5.142758in}{3.109771in}}%
\pgfpathlineto{\pgfqpoint{5.146505in}{3.110648in}}%
\pgfpathlineto{\pgfqpoint{5.150786in}{3.114318in}}%
\pgfpathlineto{\pgfqpoint{5.156941in}{3.123058in}}%
\pgfpathlineto{\pgfqpoint{5.167378in}{3.137777in}}%
\pgfpathlineto{\pgfqpoint{5.171927in}{3.140500in}}%
\pgfpathlineto{\pgfqpoint{5.175674in}{3.140207in}}%
\pgfpathlineto{\pgfqpoint{5.179688in}{3.137390in}}%
\pgfpathlineto{\pgfqpoint{5.185040in}{3.130496in}}%
\pgfpathlineto{\pgfqpoint{5.198688in}{3.111613in}}%
\pgfpathlineto{\pgfqpoint{5.202970in}{3.109771in}}%
\pgfpathlineto{\pgfqpoint{5.206716in}{3.110648in}}%
\pgfpathlineto{\pgfqpoint{5.210998in}{3.114318in}}%
\pgfpathlineto{\pgfqpoint{5.217153in}{3.123058in}}%
\pgfpathlineto{\pgfqpoint{5.227590in}{3.137777in}}%
\pgfpathlineto{\pgfqpoint{5.232139in}{3.140500in}}%
\pgfpathlineto{\pgfqpoint{5.235886in}{3.140207in}}%
\pgfpathlineto{\pgfqpoint{5.239900in}{3.137390in}}%
\pgfpathlineto{\pgfqpoint{5.245252in}{3.130496in}}%
\pgfpathlineto{\pgfqpoint{5.258900in}{3.111613in}}%
\pgfpathlineto{\pgfqpoint{5.263182in}{3.109771in}}%
\pgfpathlineto{\pgfqpoint{5.266928in}{3.110648in}}%
\pgfpathlineto{\pgfqpoint{5.271210in}{3.114318in}}%
\pgfpathlineto{\pgfqpoint{5.277365in}{3.123058in}}%
\pgfpathlineto{\pgfqpoint{5.287802in}{3.137777in}}%
\pgfpathlineto{\pgfqpoint{5.292351in}{3.140500in}}%
\pgfpathlineto{\pgfqpoint{5.296097in}{3.140207in}}%
\pgfpathlineto{\pgfqpoint{5.300112in}{3.137390in}}%
\pgfpathlineto{\pgfqpoint{5.305464in}{3.130496in}}%
\pgfpathlineto{\pgfqpoint{5.319112in}{3.111613in}}%
\pgfpathlineto{\pgfqpoint{5.323393in}{3.109771in}}%
\pgfpathlineto{\pgfqpoint{5.327140in}{3.110648in}}%
\pgfpathlineto{\pgfqpoint{5.331422in}{3.114318in}}%
\pgfpathlineto{\pgfqpoint{5.337577in}{3.123058in}}%
\pgfpathlineto{\pgfqpoint{5.348013in}{3.137777in}}%
\pgfpathlineto{\pgfqpoint{5.352563in}{3.140500in}}%
\pgfpathlineto{\pgfqpoint{5.356309in}{3.140207in}}%
\pgfpathlineto{\pgfqpoint{5.360323in}{3.137390in}}%
\pgfpathlineto{\pgfqpoint{5.365675in}{3.130496in}}%
\pgfpathlineto{\pgfqpoint{5.379323in}{3.111613in}}%
\pgfpathlineto{\pgfqpoint{5.383605in}{3.109771in}}%
\pgfpathlineto{\pgfqpoint{5.387352in}{3.110648in}}%
\pgfpathlineto{\pgfqpoint{5.391633in}{3.114318in}}%
\pgfpathlineto{\pgfqpoint{5.397788in}{3.123058in}}%
\pgfpathlineto{\pgfqpoint{5.408225in}{3.137777in}}%
\pgfpathlineto{\pgfqpoint{5.412774in}{3.140500in}}%
\pgfpathlineto{\pgfqpoint{5.416521in}{3.140207in}}%
\pgfpathlineto{\pgfqpoint{5.420535in}{3.137390in}}%
\pgfpathlineto{\pgfqpoint{5.425887in}{3.130496in}}%
\pgfpathlineto{\pgfqpoint{5.439535in}{3.111613in}}%
\pgfpathlineto{\pgfqpoint{5.443817in}{3.109771in}}%
\pgfpathlineto{\pgfqpoint{5.447563in}{3.110648in}}%
\pgfpathlineto{\pgfqpoint{5.451845in}{3.114318in}}%
\pgfpathlineto{\pgfqpoint{5.458000in}{3.123058in}}%
\pgfpathlineto{\pgfqpoint{5.468437in}{3.137777in}}%
\pgfpathlineto{\pgfqpoint{5.472986in}{3.140500in}}%
\pgfpathlineto{\pgfqpoint{5.476733in}{3.140207in}}%
\pgfpathlineto{\pgfqpoint{5.480747in}{3.137390in}}%
\pgfpathlineto{\pgfqpoint{5.486099in}{3.130496in}}%
\pgfpathlineto{\pgfqpoint{5.499747in}{3.111613in}}%
\pgfpathlineto{\pgfqpoint{5.504029in}{3.109771in}}%
\pgfpathlineto{\pgfqpoint{5.507775in}{3.110648in}}%
\pgfpathlineto{\pgfqpoint{5.512057in}{3.114318in}}%
\pgfpathlineto{\pgfqpoint{5.518212in}{3.123058in}}%
\pgfpathlineto{\pgfqpoint{5.528649in}{3.137777in}}%
\pgfpathlineto{\pgfqpoint{5.533198in}{3.140500in}}%
\pgfpathlineto{\pgfqpoint{5.536944in}{3.140207in}}%
\pgfpathlineto{\pgfqpoint{5.540959in}{3.137390in}}%
\pgfpathlineto{\pgfqpoint{5.546311in}{3.130496in}}%
\pgfpathlineto{\pgfqpoint{5.559959in}{3.111613in}}%
\pgfpathlineto{\pgfqpoint{5.564240in}{3.109771in}}%
\pgfpathlineto{\pgfqpoint{5.567987in}{3.110648in}}%
\pgfpathlineto{\pgfqpoint{5.572269in}{3.114318in}}%
\pgfpathlineto{\pgfqpoint{5.578424in}{3.123058in}}%
\pgfpathlineto{\pgfqpoint{5.588860in}{3.137777in}}%
\pgfpathlineto{\pgfqpoint{5.593410in}{3.140500in}}%
\pgfpathlineto{\pgfqpoint{5.597156in}{3.140207in}}%
\pgfpathlineto{\pgfqpoint{5.601170in}{3.137390in}}%
\pgfpathlineto{\pgfqpoint{5.606522in}{3.130496in}}%
\pgfpathlineto{\pgfqpoint{5.620170in}{3.111613in}}%
\pgfpathlineto{\pgfqpoint{5.624452in}{3.109771in}}%
\pgfpathlineto{\pgfqpoint{5.628199in}{3.110648in}}%
\pgfpathlineto{\pgfqpoint{5.632480in}{3.114318in}}%
\pgfpathlineto{\pgfqpoint{5.638635in}{3.123058in}}%
\pgfpathlineto{\pgfqpoint{5.649072in}{3.137777in}}%
\pgfpathlineto{\pgfqpoint{5.653621in}{3.140500in}}%
\pgfpathlineto{\pgfqpoint{5.657368in}{3.140207in}}%
\pgfpathlineto{\pgfqpoint{5.661382in}{3.137390in}}%
\pgfpathlineto{\pgfqpoint{5.666734in}{3.130496in}}%
\pgfpathlineto{\pgfqpoint{5.680382in}{3.111613in}}%
\pgfpathlineto{\pgfqpoint{5.684664in}{3.109771in}}%
\pgfpathlineto{\pgfqpoint{5.688410in}{3.110648in}}%
\pgfpathlineto{\pgfqpoint{5.692692in}{3.114318in}}%
\pgfpathlineto{\pgfqpoint{5.698847in}{3.123058in}}%
\pgfpathlineto{\pgfqpoint{5.709284in}{3.137777in}}%
\pgfpathlineto{\pgfqpoint{5.713833in}{3.140500in}}%
\pgfpathlineto{\pgfqpoint{5.717580in}{3.140207in}}%
\pgfpathlineto{\pgfqpoint{5.721594in}{3.137390in}}%
\pgfpathlineto{\pgfqpoint{5.726946in}{3.130496in}}%
\pgfpathlineto{\pgfqpoint{5.740594in}{3.111613in}}%
\pgfpathlineto{\pgfqpoint{5.744876in}{3.109771in}}%
\pgfpathlineto{\pgfqpoint{5.748622in}{3.110648in}}%
\pgfpathlineto{\pgfqpoint{5.752904in}{3.114318in}}%
\pgfpathlineto{\pgfqpoint{5.759059in}{3.123058in}}%
\pgfpathlineto{\pgfqpoint{5.769496in}{3.137777in}}%
\pgfpathlineto{\pgfqpoint{5.774045in}{3.140500in}}%
\pgfpathlineto{\pgfqpoint{5.777791in}{3.140207in}}%
\pgfpathlineto{\pgfqpoint{5.781806in}{3.137390in}}%
\pgfpathlineto{\pgfqpoint{5.787158in}{3.130496in}}%
\pgfpathlineto{\pgfqpoint{5.800806in}{3.111613in}}%
\pgfpathlineto{\pgfqpoint{5.805087in}{3.109771in}}%
\pgfpathlineto{\pgfqpoint{5.808834in}{3.110648in}}%
\pgfpathlineto{\pgfqpoint{5.813116in}{3.114318in}}%
\pgfpathlineto{\pgfqpoint{5.819271in}{3.123058in}}%
\pgfpathlineto{\pgfqpoint{5.829707in}{3.137777in}}%
\pgfpathlineto{\pgfqpoint{5.834257in}{3.140500in}}%
\pgfpathlineto{\pgfqpoint{5.838003in}{3.140207in}}%
\pgfpathlineto{\pgfqpoint{5.842017in}{3.137390in}}%
\pgfpathlineto{\pgfqpoint{5.847370in}{3.130496in}}%
\pgfpathlineto{\pgfqpoint{5.861018in}{3.111613in}}%
\pgfpathlineto{\pgfqpoint{5.865299in}{3.109771in}}%
\pgfpathlineto{\pgfqpoint{5.869046in}{3.110648in}}%
\pgfpathlineto{\pgfqpoint{5.873327in}{3.114318in}}%
\pgfpathlineto{\pgfqpoint{5.879482in}{3.123058in}}%
\pgfpathlineto{\pgfqpoint{5.889919in}{3.137777in}}%
\pgfpathlineto{\pgfqpoint{5.894468in}{3.140500in}}%
\pgfpathlineto{\pgfqpoint{5.898215in}{3.140207in}}%
\pgfpathlineto{\pgfqpoint{5.902229in}{3.137390in}}%
\pgfpathlineto{\pgfqpoint{5.907581in}{3.130496in}}%
\pgfpathlineto{\pgfqpoint{5.921229in}{3.111613in}}%
\pgfpathlineto{\pgfqpoint{5.925511in}{3.109771in}}%
\pgfpathlineto{\pgfqpoint{5.929258in}{3.110648in}}%
\pgfpathlineto{\pgfqpoint{5.933539in}{3.114318in}}%
\pgfpathlineto{\pgfqpoint{5.939694in}{3.123058in}}%
\pgfpathlineto{\pgfqpoint{5.950131in}{3.137777in}}%
\pgfpathlineto{\pgfqpoint{5.954680in}{3.140500in}}%
\pgfpathlineto{\pgfqpoint{5.958427in}{3.140207in}}%
\pgfpathlineto{\pgfqpoint{5.962441in}{3.137390in}}%
\pgfpathlineto{\pgfqpoint{5.967793in}{3.130496in}}%
\pgfpathlineto{\pgfqpoint{5.981441in}{3.111613in}}%
\pgfpathlineto{\pgfqpoint{5.985723in}{3.109771in}}%
\pgfpathlineto{\pgfqpoint{5.989469in}{3.110648in}}%
\pgfpathlineto{\pgfqpoint{5.993751in}{3.114318in}}%
\pgfpathlineto{\pgfqpoint{5.999906in}{3.123058in}}%
\pgfpathlineto{\pgfqpoint{6.010343in}{3.137777in}}%
\pgfpathlineto{\pgfqpoint{6.014892in}{3.140500in}}%
\pgfpathlineto{\pgfqpoint{6.018639in}{3.140207in}}%
\pgfpathlineto{\pgfqpoint{6.022653in}{3.137390in}}%
\pgfpathlineto{\pgfqpoint{6.028005in}{3.130496in}}%
\pgfpathlineto{\pgfqpoint{6.041653in}{3.111613in}}%
\pgfpathlineto{\pgfqpoint{6.045935in}{3.109771in}}%
\pgfpathlineto{\pgfqpoint{6.049681in}{3.110648in}}%
\pgfpathlineto{\pgfqpoint{6.053963in}{3.114318in}}%
\pgfpathlineto{\pgfqpoint{6.060118in}{3.123058in}}%
\pgfpathlineto{\pgfqpoint{6.070554in}{3.137777in}}%
\pgfpathlineto{\pgfqpoint{6.075104in}{3.140500in}}%
\pgfpathlineto{\pgfqpoint{6.078850in}{3.140207in}}%
\pgfpathlineto{\pgfqpoint{6.082864in}{3.137390in}}%
\pgfpathlineto{\pgfqpoint{6.088217in}{3.130496in}}%
\pgfpathlineto{\pgfqpoint{6.101865in}{3.111613in}}%
\pgfpathlineto{\pgfqpoint{6.106146in}{3.109771in}}%
\pgfpathlineto{\pgfqpoint{6.109893in}{3.110648in}}%
\pgfpathlineto{\pgfqpoint{6.114175in}{3.114318in}}%
\pgfpathlineto{\pgfqpoint{6.120329in}{3.123058in}}%
\pgfpathlineto{\pgfqpoint{6.130766in}{3.137777in}}%
\pgfpathlineto{\pgfqpoint{6.135316in}{3.140500in}}%
\pgfpathlineto{\pgfqpoint{6.139062in}{3.140207in}}%
\pgfpathlineto{\pgfqpoint{6.143076in}{3.137390in}}%
\pgfpathlineto{\pgfqpoint{6.148428in}{3.130496in}}%
\pgfpathlineto{\pgfqpoint{6.162076in}{3.111613in}}%
\pgfpathlineto{\pgfqpoint{6.166358in}{3.109771in}}%
\pgfpathlineto{\pgfqpoint{6.170105in}{3.110648in}}%
\pgfpathlineto{\pgfqpoint{6.174386in}{3.114318in}}%
\pgfpathlineto{\pgfqpoint{6.180541in}{3.123058in}}%
\pgfpathlineto{\pgfqpoint{6.190978in}{3.137777in}}%
\pgfpathlineto{\pgfqpoint{6.195527in}{3.140500in}}%
\pgfpathlineto{\pgfqpoint{6.199274in}{3.140207in}}%
\pgfpathlineto{\pgfqpoint{6.203288in}{3.137390in}}%
\pgfpathlineto{\pgfqpoint{6.208640in}{3.130496in}}%
\pgfpathlineto{\pgfqpoint{6.222288in}{3.111613in}}%
\pgfpathlineto{\pgfqpoint{6.226570in}{3.109771in}}%
\pgfpathlineto{\pgfqpoint{6.230316in}{3.110648in}}%
\pgfpathlineto{\pgfqpoint{6.234598in}{3.114318in}}%
\pgfpathlineto{\pgfqpoint{6.240753in}{3.123058in}}%
\pgfpathlineto{\pgfqpoint{6.251190in}{3.137777in}}%
\pgfpathlineto{\pgfqpoint{6.255739in}{3.140500in}}%
\pgfpathlineto{\pgfqpoint{6.259486in}{3.140207in}}%
\pgfpathlineto{\pgfqpoint{6.263500in}{3.137390in}}%
\pgfpathlineto{\pgfqpoint{6.268852in}{3.130496in}}%
\pgfpathlineto{\pgfqpoint{6.282500in}{3.111613in}}%
\pgfpathlineto{\pgfqpoint{6.286782in}{3.109771in}}%
\pgfpathlineto{\pgfqpoint{6.290528in}{3.110648in}}%
\pgfpathlineto{\pgfqpoint{6.294810in}{3.114318in}}%
\pgfpathlineto{\pgfqpoint{6.300965in}{3.123058in}}%
\pgfpathlineto{\pgfqpoint{6.311401in}{3.137777in}}%
\pgfpathlineto{\pgfqpoint{6.315951in}{3.140500in}}%
\pgfpathlineto{\pgfqpoint{6.319697in}{3.140207in}}%
\pgfpathlineto{\pgfqpoint{6.323711in}{3.137390in}}%
\pgfpathlineto{\pgfqpoint{6.329064in}{3.130496in}}%
\pgfpathlineto{\pgfqpoint{6.342712in}{3.111613in}}%
\pgfpathlineto{\pgfqpoint{6.346993in}{3.109771in}}%
\pgfpathlineto{\pgfqpoint{6.350740in}{3.110648in}}%
\pgfpathlineto{\pgfqpoint{6.355022in}{3.114318in}}%
\pgfpathlineto{\pgfqpoint{6.361177in}{3.123058in}}%
\pgfpathlineto{\pgfqpoint{6.371613in}{3.137777in}}%
\pgfpathlineto{\pgfqpoint{6.376163in}{3.140500in}}%
\pgfpathlineto{\pgfqpoint{6.379909in}{3.140207in}}%
\pgfpathlineto{\pgfqpoint{6.383923in}{3.137390in}}%
\pgfpathlineto{\pgfqpoint{6.389275in}{3.130496in}}%
\pgfpathlineto{\pgfqpoint{6.402923in}{3.111613in}}%
\pgfpathlineto{\pgfqpoint{6.407205in}{3.109771in}}%
\pgfpathlineto{\pgfqpoint{6.410952in}{3.110648in}}%
\pgfpathlineto{\pgfqpoint{6.415233in}{3.114318in}}%
\pgfpathlineto{\pgfqpoint{6.421388in}{3.123058in}}%
\pgfpathlineto{\pgfqpoint{6.431825in}{3.137777in}}%
\pgfpathlineto{\pgfqpoint{6.436374in}{3.140500in}}%
\pgfpathlineto{\pgfqpoint{6.440121in}{3.140207in}}%
\pgfpathlineto{\pgfqpoint{6.444135in}{3.137390in}}%
\pgfpathlineto{\pgfqpoint{6.449487in}{3.130496in}}%
\pgfpathlineto{\pgfqpoint{6.463135in}{3.111613in}}%
\pgfpathlineto{\pgfqpoint{6.467417in}{3.109771in}}%
\pgfpathlineto{\pgfqpoint{6.471163in}{3.110648in}}%
\pgfpathlineto{\pgfqpoint{6.475445in}{3.114318in}}%
\pgfpathlineto{\pgfqpoint{6.481600in}{3.123058in}}%
\pgfpathlineto{\pgfqpoint{6.492037in}{3.137777in}}%
\pgfpathlineto{\pgfqpoint{6.496586in}{3.140500in}}%
\pgfpathlineto{\pgfqpoint{6.500333in}{3.140207in}}%
\pgfpathlineto{\pgfqpoint{6.504347in}{3.137390in}}%
\pgfpathlineto{\pgfqpoint{6.509699in}{3.130496in}}%
\pgfpathlineto{\pgfqpoint{6.523347in}{3.111613in}}%
\pgfpathlineto{\pgfqpoint{6.527629in}{3.109771in}}%
\pgfpathlineto{\pgfqpoint{6.531375in}{3.110648in}}%
\pgfpathlineto{\pgfqpoint{6.535657in}{3.114318in}}%
\pgfpathlineto{\pgfqpoint{6.541812in}{3.123058in}}%
\pgfpathlineto{\pgfqpoint{6.552249in}{3.137777in}}%
\pgfpathlineto{\pgfqpoint{6.556798in}{3.140500in}}%
\pgfpathlineto{\pgfqpoint{6.560544in}{3.140207in}}%
\pgfpathlineto{\pgfqpoint{6.564558in}{3.137390in}}%
\pgfpathlineto{\pgfqpoint{6.569911in}{3.130496in}}%
\pgfpathlineto{\pgfqpoint{6.583559in}{3.111613in}}%
\pgfpathlineto{\pgfqpoint{6.587840in}{3.109771in}}%
\pgfpathlineto{\pgfqpoint{6.591587in}{3.110648in}}%
\pgfpathlineto{\pgfqpoint{6.595869in}{3.114318in}}%
\pgfpathlineto{\pgfqpoint{6.602024in}{3.123058in}}%
\pgfpathlineto{\pgfqpoint{6.612460in}{3.137777in}}%
\pgfpathlineto{\pgfqpoint{6.617010in}{3.140500in}}%
\pgfpathlineto{\pgfqpoint{6.620756in}{3.140207in}}%
\pgfpathlineto{\pgfqpoint{6.624770in}{3.137390in}}%
\pgfpathlineto{\pgfqpoint{6.630122in}{3.130496in}}%
\pgfpathlineto{\pgfqpoint{6.643770in}{3.111613in}}%
\pgfpathlineto{\pgfqpoint{6.648052in}{3.109771in}}%
\pgfpathlineto{\pgfqpoint{6.651799in}{3.110648in}}%
\pgfpathlineto{\pgfqpoint{6.656080in}{3.114318in}}%
\pgfpathlineto{\pgfqpoint{6.662235in}{3.123058in}}%
\pgfpathlineto{\pgfqpoint{6.663306in}{3.124778in}}%
\pgfpathlineto{\pgfqpoint{6.663306in}{3.124778in}}%
\pgfusepath{stroke}%
\end{pgfscope}%
\begin{pgfscope}%
\pgfpathrectangle{\pgfqpoint{0.467797in}{2.292089in}}{\pgfqpoint{6.490533in}{1.666241in}}%
\pgfusepath{clip}%
\pgfsetrectcap%
\pgfsetroundjoin%
\pgfsetlinewidth{1.505625pt}%
\definecolor{currentstroke}{rgb}{0.549020,0.337255,0.294118}%
\pgfsetstrokecolor{currentstroke}%
\pgfsetdash{}{0pt}%
\pgfpathmoveto{\pgfqpoint{0.762821in}{3.125209in}}%
\pgfpathlineto{\pgfqpoint{0.771652in}{3.137368in}}%
\pgfpathlineto{\pgfqpoint{0.775934in}{3.139901in}}%
\pgfpathlineto{\pgfqpoint{0.779413in}{3.139665in}}%
\pgfpathlineto{\pgfqpoint{0.783159in}{3.137137in}}%
\pgfpathlineto{\pgfqpoint{0.788244in}{3.130723in}}%
\pgfpathlineto{\pgfqpoint{0.802159in}{3.111772in}}%
\pgfpathlineto{\pgfqpoint{0.806173in}{3.110359in}}%
\pgfpathlineto{\pgfqpoint{0.809652in}{3.111385in}}%
\pgfpathlineto{\pgfqpoint{0.813934in}{3.115287in}}%
\pgfpathlineto{\pgfqpoint{0.820624in}{3.125133in}}%
\pgfpathlineto{\pgfqpoint{0.829455in}{3.137324in}}%
\pgfpathlineto{\pgfqpoint{0.833737in}{3.139889in}}%
\pgfpathlineto{\pgfqpoint{0.837483in}{3.139579in}}%
\pgfpathlineto{\pgfqpoint{0.841498in}{3.136652in}}%
\pgfpathlineto{\pgfqpoint{0.846850in}{3.129575in}}%
\pgfpathlineto{\pgfqpoint{0.859160in}{3.112412in}}%
\pgfpathlineto{\pgfqpoint{0.863441in}{3.110392in}}%
\pgfpathlineto{\pgfqpoint{0.866920in}{3.111070in}}%
\pgfpathlineto{\pgfqpoint{0.870934in}{3.114310in}}%
\pgfpathlineto{\pgfqpoint{0.876822in}{3.122483in}}%
\pgfpathlineto{\pgfqpoint{0.887526in}{3.137526in}}%
\pgfpathlineto{\pgfqpoint{0.891808in}{3.139939in}}%
\pgfpathlineto{\pgfqpoint{0.895287in}{3.139598in}}%
\pgfpathlineto{\pgfqpoint{0.899301in}{3.136700in}}%
\pgfpathlineto{\pgfqpoint{0.904653in}{3.129647in}}%
\pgfpathlineto{\pgfqpoint{0.917231in}{3.112236in}}%
\pgfpathlineto{\pgfqpoint{0.921245in}{3.110398in}}%
\pgfpathlineto{\pgfqpoint{0.924724in}{3.111047in}}%
\pgfpathlineto{\pgfqpoint{0.928738in}{3.114259in}}%
\pgfpathlineto{\pgfqpoint{0.934625in}{3.122408in}}%
\pgfpathlineto{\pgfqpoint{0.945329in}{3.137483in}}%
\pgfpathlineto{\pgfqpoint{0.949611in}{3.139929in}}%
\pgfpathlineto{\pgfqpoint{0.953090in}{3.139617in}}%
\pgfpathlineto{\pgfqpoint{0.956837in}{3.137015in}}%
\pgfpathlineto{\pgfqpoint{0.962189in}{3.130129in}}%
\pgfpathlineto{\pgfqpoint{0.975302in}{3.112066in}}%
\pgfpathlineto{\pgfqpoint{0.979316in}{3.110376in}}%
\pgfpathlineto{\pgfqpoint{0.982795in}{3.111158in}}%
\pgfpathlineto{\pgfqpoint{0.986809in}{3.114502in}}%
\pgfpathlineto{\pgfqpoint{0.992696in}{3.122758in}}%
\pgfpathlineto{\pgfqpoint{1.003133in}{3.137440in}}%
\pgfpathlineto{\pgfqpoint{1.007414in}{3.139919in}}%
\pgfpathlineto{\pgfqpoint{1.010893in}{3.139635in}}%
\pgfpathlineto{\pgfqpoint{1.014640in}{3.137061in}}%
\pgfpathlineto{\pgfqpoint{1.019992in}{3.130201in}}%
\pgfpathlineto{\pgfqpoint{1.033105in}{3.112102in}}%
\pgfpathlineto{\pgfqpoint{1.037119in}{3.110380in}}%
\pgfpathlineto{\pgfqpoint{1.040598in}{3.111133in}}%
\pgfpathlineto{\pgfqpoint{1.044612in}{3.114449in}}%
\pgfpathlineto{\pgfqpoint{1.050499in}{3.122683in}}%
\pgfpathlineto{\pgfqpoint{1.060936in}{3.137397in}}%
\pgfpathlineto{\pgfqpoint{1.065218in}{3.139908in}}%
\pgfpathlineto{\pgfqpoint{1.068697in}{3.139653in}}%
\pgfpathlineto{\pgfqpoint{1.072443in}{3.137107in}}%
\pgfpathlineto{\pgfqpoint{1.077528in}{3.130676in}}%
\pgfpathlineto{\pgfqpoint{1.091176in}{3.111939in}}%
\pgfpathlineto{\pgfqpoint{1.095190in}{3.110365in}}%
\pgfpathlineto{\pgfqpoint{1.098669in}{3.111251in}}%
\pgfpathlineto{\pgfqpoint{1.102683in}{3.114697in}}%
\pgfpathlineto{\pgfqpoint{1.108838in}{3.123462in}}%
\pgfpathlineto{\pgfqpoint{1.118739in}{3.137353in}}%
\pgfpathlineto{\pgfqpoint{1.123021in}{3.139897in}}%
\pgfpathlineto{\pgfqpoint{1.126500in}{3.139671in}}%
\pgfpathlineto{\pgfqpoint{1.130246in}{3.137153in}}%
\pgfpathlineto{\pgfqpoint{1.135331in}{3.130747in}}%
\pgfpathlineto{\pgfqpoint{1.149247in}{3.111783in}}%
\pgfpathlineto{\pgfqpoint{1.153261in}{3.110359in}}%
\pgfpathlineto{\pgfqpoint{1.156740in}{3.111376in}}%
\pgfpathlineto{\pgfqpoint{1.161021in}{3.115268in}}%
\pgfpathlineto{\pgfqpoint{1.167712in}{3.125108in}}%
\pgfpathlineto{\pgfqpoint{1.176543in}{3.137309in}}%
\pgfpathlineto{\pgfqpoint{1.180824in}{3.139886in}}%
\pgfpathlineto{\pgfqpoint{1.184571in}{3.139586in}}%
\pgfpathlineto{\pgfqpoint{1.188585in}{3.136668in}}%
\pgfpathlineto{\pgfqpoint{1.193937in}{3.129599in}}%
\pgfpathlineto{\pgfqpoint{1.206247in}{3.112425in}}%
\pgfpathlineto{\pgfqpoint{1.210529in}{3.110394in}}%
\pgfpathlineto{\pgfqpoint{1.214008in}{3.111062in}}%
\pgfpathlineto{\pgfqpoint{1.218022in}{3.114293in}}%
\pgfpathlineto{\pgfqpoint{1.223909in}{3.122458in}}%
\pgfpathlineto{\pgfqpoint{1.234613in}{3.137512in}}%
\pgfpathlineto{\pgfqpoint{1.238895in}{3.139936in}}%
\pgfpathlineto{\pgfqpoint{1.242374in}{3.139604in}}%
\pgfpathlineto{\pgfqpoint{1.246121in}{3.136984in}}%
\pgfpathlineto{\pgfqpoint{1.251473in}{3.130081in}}%
\pgfpathlineto{\pgfqpoint{1.264318in}{3.112248in}}%
\pgfpathlineto{\pgfqpoint{1.268332in}{3.110399in}}%
\pgfpathlineto{\pgfqpoint{1.271811in}{3.111039in}}%
\pgfpathlineto{\pgfqpoint{1.275825in}{3.114241in}}%
\pgfpathlineto{\pgfqpoint{1.281712in}{3.122383in}}%
\pgfpathlineto{\pgfqpoint{1.292417in}{3.137469in}}%
\pgfpathlineto{\pgfqpoint{1.296699in}{3.139926in}}%
\pgfpathlineto{\pgfqpoint{1.300177in}{3.139623in}}%
\pgfpathlineto{\pgfqpoint{1.303924in}{3.137031in}}%
\pgfpathlineto{\pgfqpoint{1.309276in}{3.130153in}}%
\pgfpathlineto{\pgfqpoint{1.322389in}{3.112078in}}%
\pgfpathlineto{\pgfqpoint{1.326403in}{3.110377in}}%
\pgfpathlineto{\pgfqpoint{1.329882in}{3.111150in}}%
\pgfpathlineto{\pgfqpoint{1.333896in}{3.114484in}}%
\pgfpathlineto{\pgfqpoint{1.339783in}{3.122733in}}%
\pgfpathlineto{\pgfqpoint{1.350220in}{3.137426in}}%
\pgfpathlineto{\pgfqpoint{1.354502in}{3.139915in}}%
\pgfpathlineto{\pgfqpoint{1.357981in}{3.139641in}}%
\pgfpathlineto{\pgfqpoint{1.361727in}{3.137077in}}%
\pgfpathlineto{\pgfqpoint{1.366812in}{3.130629in}}%
\pgfpathlineto{\pgfqpoint{1.380460in}{3.111916in}}%
\pgfpathlineto{\pgfqpoint{1.384474in}{3.110364in}}%
\pgfpathlineto{\pgfqpoint{1.387953in}{3.111268in}}%
\pgfpathlineto{\pgfqpoint{1.391967in}{3.114733in}}%
\pgfpathlineto{\pgfqpoint{1.398122in}{3.123512in}}%
\pgfpathlineto{\pgfqpoint{1.408023in}{3.137382in}}%
\pgfpathlineto{\pgfqpoint{1.412305in}{3.139904in}}%
\pgfpathlineto{\pgfqpoint{1.415784in}{3.139659in}}%
\pgfpathlineto{\pgfqpoint{1.419531in}{3.137122in}}%
\pgfpathlineto{\pgfqpoint{1.424615in}{3.130700in}}%
\pgfpathlineto{\pgfqpoint{1.438263in}{3.111950in}}%
\pgfpathlineto{\pgfqpoint{1.442277in}{3.110366in}}%
\pgfpathlineto{\pgfqpoint{1.445756in}{3.111242in}}%
\pgfpathlineto{\pgfqpoint{1.449770in}{3.114679in}}%
\pgfpathlineto{\pgfqpoint{1.455925in}{3.123436in}}%
\pgfpathlineto{\pgfqpoint{1.465827in}{3.137338in}}%
\pgfpathlineto{\pgfqpoint{1.470108in}{3.139893in}}%
\pgfpathlineto{\pgfqpoint{1.473855in}{3.139573in}}%
\pgfpathlineto{\pgfqpoint{1.477869in}{3.136636in}}%
\pgfpathlineto{\pgfqpoint{1.483221in}{3.129550in}}%
\pgfpathlineto{\pgfqpoint{1.495531in}{3.112399in}}%
\pgfpathlineto{\pgfqpoint{1.499813in}{3.110391in}}%
\pgfpathlineto{\pgfqpoint{1.503292in}{3.111078in}}%
\pgfpathlineto{\pgfqpoint{1.507306in}{3.114327in}}%
\pgfpathlineto{\pgfqpoint{1.513193in}{3.122508in}}%
\pgfpathlineto{\pgfqpoint{1.523898in}{3.137540in}}%
\pgfpathlineto{\pgfqpoint{1.528179in}{3.139942in}}%
\pgfpathlineto{\pgfqpoint{1.531658in}{3.139592in}}%
\pgfpathlineto{\pgfqpoint{1.535672in}{3.136684in}}%
\pgfpathlineto{\pgfqpoint{1.541024in}{3.129623in}}%
\pgfpathlineto{\pgfqpoint{1.553334in}{3.112438in}}%
\pgfpathlineto{\pgfqpoint{1.557616in}{3.110396in}}%
\pgfpathlineto{\pgfqpoint{1.561095in}{3.111055in}}%
\pgfpathlineto{\pgfqpoint{1.565109in}{3.114276in}}%
\pgfpathlineto{\pgfqpoint{1.570997in}{3.122433in}}%
\pgfpathlineto{\pgfqpoint{1.581701in}{3.137497in}}%
\pgfpathlineto{\pgfqpoint{1.585983in}{3.139932in}}%
\pgfpathlineto{\pgfqpoint{1.589461in}{3.139611in}}%
\pgfpathlineto{\pgfqpoint{1.593208in}{3.137000in}}%
\pgfpathlineto{\pgfqpoint{1.598560in}{3.130105in}}%
\pgfpathlineto{\pgfqpoint{1.611673in}{3.112055in}}%
\pgfpathlineto{\pgfqpoint{1.615687in}{3.110375in}}%
\pgfpathlineto{\pgfqpoint{1.619166in}{3.111166in}}%
\pgfpathlineto{\pgfqpoint{1.623180in}{3.114519in}}%
\pgfpathlineto{\pgfqpoint{1.629067in}{3.122783in}}%
\pgfpathlineto{\pgfqpoint{1.639504in}{3.137455in}}%
\pgfpathlineto{\pgfqpoint{1.643786in}{3.139922in}}%
\pgfpathlineto{\pgfqpoint{1.647265in}{3.139629in}}%
\pgfpathlineto{\pgfqpoint{1.651011in}{3.137046in}}%
\pgfpathlineto{\pgfqpoint{1.656363in}{3.130177in}}%
\pgfpathlineto{\pgfqpoint{1.669476in}{3.112090in}}%
\pgfpathlineto{\pgfqpoint{1.673490in}{3.110379in}}%
\pgfpathlineto{\pgfqpoint{1.676969in}{3.111142in}}%
\pgfpathlineto{\pgfqpoint{1.680983in}{3.114467in}}%
\pgfpathlineto{\pgfqpoint{1.686871in}{3.122708in}}%
\pgfpathlineto{\pgfqpoint{1.697307in}{3.137411in}}%
\pgfpathlineto{\pgfqpoint{1.701589in}{3.139912in}}%
\pgfpathlineto{\pgfqpoint{1.705068in}{3.139647in}}%
\pgfpathlineto{\pgfqpoint{1.708815in}{3.137092in}}%
\pgfpathlineto{\pgfqpoint{1.713899in}{3.130653in}}%
\pgfpathlineto{\pgfqpoint{1.727547in}{3.111927in}}%
\pgfpathlineto{\pgfqpoint{1.731561in}{3.110365in}}%
\pgfpathlineto{\pgfqpoint{1.735040in}{3.111259in}}%
\pgfpathlineto{\pgfqpoint{1.739054in}{3.114715in}}%
\pgfpathlineto{\pgfqpoint{1.745209in}{3.123487in}}%
\pgfpathlineto{\pgfqpoint{1.755111in}{3.137368in}}%
\pgfpathlineto{\pgfqpoint{1.759392in}{3.139901in}}%
\pgfpathlineto{\pgfqpoint{1.762871in}{3.139665in}}%
\pgfpathlineto{\pgfqpoint{1.766618in}{3.137137in}}%
\pgfpathlineto{\pgfqpoint{1.771702in}{3.130723in}}%
\pgfpathlineto{\pgfqpoint{1.785618in}{3.111772in}}%
\pgfpathlineto{\pgfqpoint{1.789632in}{3.110359in}}%
\pgfpathlineto{\pgfqpoint{1.793111in}{3.111385in}}%
\pgfpathlineto{\pgfqpoint{1.797393in}{3.115287in}}%
\pgfpathlineto{\pgfqpoint{1.804083in}{3.125133in}}%
\pgfpathlineto{\pgfqpoint{1.812914in}{3.137324in}}%
\pgfpathlineto{\pgfqpoint{1.817196in}{3.139889in}}%
\pgfpathlineto{\pgfqpoint{1.820942in}{3.139579in}}%
\pgfpathlineto{\pgfqpoint{1.824956in}{3.136652in}}%
\pgfpathlineto{\pgfqpoint{1.830309in}{3.129575in}}%
\pgfpathlineto{\pgfqpoint{1.842618in}{3.112412in}}%
\pgfpathlineto{\pgfqpoint{1.846900in}{3.110392in}}%
\pgfpathlineto{\pgfqpoint{1.850379in}{3.111070in}}%
\pgfpathlineto{\pgfqpoint{1.854393in}{3.114310in}}%
\pgfpathlineto{\pgfqpoint{1.860281in}{3.122483in}}%
\pgfpathlineto{\pgfqpoint{1.870985in}{3.137526in}}%
\pgfpathlineto{\pgfqpoint{1.875267in}{3.139939in}}%
\pgfpathlineto{\pgfqpoint{1.878746in}{3.139598in}}%
\pgfpathlineto{\pgfqpoint{1.882760in}{3.136700in}}%
\pgfpathlineto{\pgfqpoint{1.888112in}{3.129647in}}%
\pgfpathlineto{\pgfqpoint{1.900689in}{3.112236in}}%
\pgfpathlineto{\pgfqpoint{1.904703in}{3.110398in}}%
\pgfpathlineto{\pgfqpoint{1.908182in}{3.111047in}}%
\pgfpathlineto{\pgfqpoint{1.912197in}{3.114259in}}%
\pgfpathlineto{\pgfqpoint{1.918084in}{3.122408in}}%
\pgfpathlineto{\pgfqpoint{1.928788in}{3.137483in}}%
\pgfpathlineto{\pgfqpoint{1.933070in}{3.139929in}}%
\pgfpathlineto{\pgfqpoint{1.936549in}{3.139617in}}%
\pgfpathlineto{\pgfqpoint{1.940295in}{3.137015in}}%
\pgfpathlineto{\pgfqpoint{1.945647in}{3.130129in}}%
\pgfpathlineto{\pgfqpoint{1.958760in}{3.112066in}}%
\pgfpathlineto{\pgfqpoint{1.962774in}{3.110376in}}%
\pgfpathlineto{\pgfqpoint{1.966253in}{3.111158in}}%
\pgfpathlineto{\pgfqpoint{1.970267in}{3.114502in}}%
\pgfpathlineto{\pgfqpoint{1.976155in}{3.122758in}}%
\pgfpathlineto{\pgfqpoint{1.986591in}{3.137440in}}%
\pgfpathlineto{\pgfqpoint{1.990873in}{3.139919in}}%
\pgfpathlineto{\pgfqpoint{1.994352in}{3.139635in}}%
\pgfpathlineto{\pgfqpoint{1.998099in}{3.137061in}}%
\pgfpathlineto{\pgfqpoint{2.003451in}{3.130201in}}%
\pgfpathlineto{\pgfqpoint{2.016564in}{3.112102in}}%
\pgfpathlineto{\pgfqpoint{2.020578in}{3.110380in}}%
\pgfpathlineto{\pgfqpoint{2.024057in}{3.111133in}}%
\pgfpathlineto{\pgfqpoint{2.028071in}{3.114449in}}%
\pgfpathlineto{\pgfqpoint{2.033958in}{3.122683in}}%
\pgfpathlineto{\pgfqpoint{2.044395in}{3.137397in}}%
\pgfpathlineto{\pgfqpoint{2.048676in}{3.139908in}}%
\pgfpathlineto{\pgfqpoint{2.052155in}{3.139653in}}%
\pgfpathlineto{\pgfqpoint{2.055902in}{3.137107in}}%
\pgfpathlineto{\pgfqpoint{2.060986in}{3.130676in}}%
\pgfpathlineto{\pgfqpoint{2.074634in}{3.111939in}}%
\pgfpathlineto{\pgfqpoint{2.078649in}{3.110365in}}%
\pgfpathlineto{\pgfqpoint{2.082127in}{3.111251in}}%
\pgfpathlineto{\pgfqpoint{2.086142in}{3.114697in}}%
\pgfpathlineto{\pgfqpoint{2.092297in}{3.123462in}}%
\pgfpathlineto{\pgfqpoint{2.102198in}{3.137353in}}%
\pgfpathlineto{\pgfqpoint{2.106480in}{3.139897in}}%
\pgfpathlineto{\pgfqpoint{2.109959in}{3.139671in}}%
\pgfpathlineto{\pgfqpoint{2.113705in}{3.137153in}}%
\pgfpathlineto{\pgfqpoint{2.118790in}{3.130747in}}%
\pgfpathlineto{\pgfqpoint{2.132705in}{3.111783in}}%
\pgfpathlineto{\pgfqpoint{2.136719in}{3.110359in}}%
\pgfpathlineto{\pgfqpoint{2.140198in}{3.111376in}}%
\pgfpathlineto{\pgfqpoint{2.144480in}{3.115268in}}%
\pgfpathlineto{\pgfqpoint{2.151170in}{3.125108in}}%
\pgfpathlineto{\pgfqpoint{2.160001in}{3.137309in}}%
\pgfpathlineto{\pgfqpoint{2.164283in}{3.139886in}}%
\pgfpathlineto{\pgfqpoint{2.168030in}{3.139586in}}%
\pgfpathlineto{\pgfqpoint{2.172044in}{3.136668in}}%
\pgfpathlineto{\pgfqpoint{2.177396in}{3.129599in}}%
\pgfpathlineto{\pgfqpoint{2.189706in}{3.112425in}}%
\pgfpathlineto{\pgfqpoint{2.193988in}{3.110394in}}%
\pgfpathlineto{\pgfqpoint{2.197466in}{3.111062in}}%
\pgfpathlineto{\pgfqpoint{2.201481in}{3.114293in}}%
\pgfpathlineto{\pgfqpoint{2.207368in}{3.122458in}}%
\pgfpathlineto{\pgfqpoint{2.218072in}{3.137512in}}%
\pgfpathlineto{\pgfqpoint{2.222354in}{3.139936in}}%
\pgfpathlineto{\pgfqpoint{2.225833in}{3.139604in}}%
\pgfpathlineto{\pgfqpoint{2.229579in}{3.136984in}}%
\pgfpathlineto{\pgfqpoint{2.234932in}{3.130081in}}%
\pgfpathlineto{\pgfqpoint{2.247777in}{3.112248in}}%
\pgfpathlineto{\pgfqpoint{2.251791in}{3.110399in}}%
\pgfpathlineto{\pgfqpoint{2.255270in}{3.111039in}}%
\pgfpathlineto{\pgfqpoint{2.259284in}{3.114241in}}%
\pgfpathlineto{\pgfqpoint{2.265171in}{3.122383in}}%
\pgfpathlineto{\pgfqpoint{2.275876in}{3.137469in}}%
\pgfpathlineto{\pgfqpoint{2.280157in}{3.139926in}}%
\pgfpathlineto{\pgfqpoint{2.283636in}{3.139623in}}%
\pgfpathlineto{\pgfqpoint{2.287383in}{3.137031in}}%
\pgfpathlineto{\pgfqpoint{2.292735in}{3.130153in}}%
\pgfpathlineto{\pgfqpoint{2.305848in}{3.112078in}}%
\pgfpathlineto{\pgfqpoint{2.309862in}{3.110377in}}%
\pgfpathlineto{\pgfqpoint{2.313341in}{3.111150in}}%
\pgfpathlineto{\pgfqpoint{2.317355in}{3.114484in}}%
\pgfpathlineto{\pgfqpoint{2.323242in}{3.122733in}}%
\pgfpathlineto{\pgfqpoint{2.333679in}{3.137426in}}%
\pgfpathlineto{\pgfqpoint{2.337961in}{3.139915in}}%
\pgfpathlineto{\pgfqpoint{2.341439in}{3.139641in}}%
\pgfpathlineto{\pgfqpoint{2.345186in}{3.137077in}}%
\pgfpathlineto{\pgfqpoint{2.350271in}{3.130629in}}%
\pgfpathlineto{\pgfqpoint{2.363919in}{3.111916in}}%
\pgfpathlineto{\pgfqpoint{2.367933in}{3.110364in}}%
\pgfpathlineto{\pgfqpoint{2.371412in}{3.111268in}}%
\pgfpathlineto{\pgfqpoint{2.375426in}{3.114733in}}%
\pgfpathlineto{\pgfqpoint{2.381581in}{3.123512in}}%
\pgfpathlineto{\pgfqpoint{2.391482in}{3.137382in}}%
\pgfpathlineto{\pgfqpoint{2.395764in}{3.139904in}}%
\pgfpathlineto{\pgfqpoint{2.399243in}{3.139659in}}%
\pgfpathlineto{\pgfqpoint{2.402989in}{3.137122in}}%
\pgfpathlineto{\pgfqpoint{2.408074in}{3.130700in}}%
\pgfpathlineto{\pgfqpoint{2.421722in}{3.111950in}}%
\pgfpathlineto{\pgfqpoint{2.425736in}{3.110366in}}%
\pgfpathlineto{\pgfqpoint{2.429215in}{3.111242in}}%
\pgfpathlineto{\pgfqpoint{2.433229in}{3.114679in}}%
\pgfpathlineto{\pgfqpoint{2.439384in}{3.123436in}}%
\pgfpathlineto{\pgfqpoint{2.449285in}{3.137338in}}%
\pgfpathlineto{\pgfqpoint{2.453567in}{3.139893in}}%
\pgfpathlineto{\pgfqpoint{2.457314in}{3.139573in}}%
\pgfpathlineto{\pgfqpoint{2.461328in}{3.136636in}}%
\pgfpathlineto{\pgfqpoint{2.466680in}{3.129550in}}%
\pgfpathlineto{\pgfqpoint{2.478990in}{3.112399in}}%
\pgfpathlineto{\pgfqpoint{2.483272in}{3.110391in}}%
\pgfpathlineto{\pgfqpoint{2.486750in}{3.111078in}}%
\pgfpathlineto{\pgfqpoint{2.490765in}{3.114327in}}%
\pgfpathlineto{\pgfqpoint{2.496652in}{3.122508in}}%
\pgfpathlineto{\pgfqpoint{2.507356in}{3.137540in}}%
\pgfpathlineto{\pgfqpoint{2.511638in}{3.139942in}}%
\pgfpathlineto{\pgfqpoint{2.515117in}{3.139592in}}%
\pgfpathlineto{\pgfqpoint{2.519131in}{3.136684in}}%
\pgfpathlineto{\pgfqpoint{2.524483in}{3.129623in}}%
\pgfpathlineto{\pgfqpoint{2.536793in}{3.112438in}}%
\pgfpathlineto{\pgfqpoint{2.541075in}{3.110396in}}%
\pgfpathlineto{\pgfqpoint{2.544554in}{3.111055in}}%
\pgfpathlineto{\pgfqpoint{2.548568in}{3.114276in}}%
\pgfpathlineto{\pgfqpoint{2.554455in}{3.122433in}}%
\pgfpathlineto{\pgfqpoint{2.565160in}{3.137497in}}%
\pgfpathlineto{\pgfqpoint{2.569441in}{3.139932in}}%
\pgfpathlineto{\pgfqpoint{2.572920in}{3.139611in}}%
\pgfpathlineto{\pgfqpoint{2.576667in}{3.137000in}}%
\pgfpathlineto{\pgfqpoint{2.582019in}{3.130105in}}%
\pgfpathlineto{\pgfqpoint{2.595132in}{3.112055in}}%
\pgfpathlineto{\pgfqpoint{2.599146in}{3.110375in}}%
\pgfpathlineto{\pgfqpoint{2.602625in}{3.111166in}}%
\pgfpathlineto{\pgfqpoint{2.606639in}{3.114519in}}%
\pgfpathlineto{\pgfqpoint{2.612526in}{3.122783in}}%
\pgfpathlineto{\pgfqpoint{2.622963in}{3.137455in}}%
\pgfpathlineto{\pgfqpoint{2.627245in}{3.139922in}}%
\pgfpathlineto{\pgfqpoint{2.630724in}{3.139629in}}%
\pgfpathlineto{\pgfqpoint{2.634470in}{3.137046in}}%
\pgfpathlineto{\pgfqpoint{2.639822in}{3.130177in}}%
\pgfpathlineto{\pgfqpoint{2.652935in}{3.112090in}}%
\pgfpathlineto{\pgfqpoint{2.656949in}{3.110379in}}%
\pgfpathlineto{\pgfqpoint{2.660428in}{3.111142in}}%
\pgfpathlineto{\pgfqpoint{2.664442in}{3.114467in}}%
\pgfpathlineto{\pgfqpoint{2.670329in}{3.122708in}}%
\pgfpathlineto{\pgfqpoint{2.680766in}{3.137411in}}%
\pgfpathlineto{\pgfqpoint{2.685048in}{3.139912in}}%
\pgfpathlineto{\pgfqpoint{2.688527in}{3.139647in}}%
\pgfpathlineto{\pgfqpoint{2.692273in}{3.137092in}}%
\pgfpathlineto{\pgfqpoint{2.697358in}{3.130653in}}%
\pgfpathlineto{\pgfqpoint{2.711006in}{3.111927in}}%
\pgfpathlineto{\pgfqpoint{2.715020in}{3.110365in}}%
\pgfpathlineto{\pgfqpoint{2.718499in}{3.111259in}}%
\pgfpathlineto{\pgfqpoint{2.722513in}{3.114715in}}%
\pgfpathlineto{\pgfqpoint{2.728668in}{3.123487in}}%
\pgfpathlineto{\pgfqpoint{2.738569in}{3.137368in}}%
\pgfpathlineto{\pgfqpoint{2.742851in}{3.139901in}}%
\pgfpathlineto{\pgfqpoint{2.746330in}{3.139665in}}%
\pgfpathlineto{\pgfqpoint{2.750077in}{3.137137in}}%
\pgfpathlineto{\pgfqpoint{2.755161in}{3.130723in}}%
\pgfpathlineto{\pgfqpoint{2.769077in}{3.111772in}}%
\pgfpathlineto{\pgfqpoint{2.773091in}{3.110359in}}%
\pgfpathlineto{\pgfqpoint{2.776570in}{3.111385in}}%
\pgfpathlineto{\pgfqpoint{2.780851in}{3.115287in}}%
\pgfpathlineto{\pgfqpoint{2.787542in}{3.125133in}}%
\pgfpathlineto{\pgfqpoint{2.796373in}{3.137324in}}%
\pgfpathlineto{\pgfqpoint{2.800654in}{3.139889in}}%
\pgfpathlineto{\pgfqpoint{2.804401in}{3.139579in}}%
\pgfpathlineto{\pgfqpoint{2.808415in}{3.136652in}}%
\pgfpathlineto{\pgfqpoint{2.813767in}{3.129575in}}%
\pgfpathlineto{\pgfqpoint{2.826077in}{3.112412in}}%
\pgfpathlineto{\pgfqpoint{2.830359in}{3.110392in}}%
\pgfpathlineto{\pgfqpoint{2.833838in}{3.111070in}}%
\pgfpathlineto{\pgfqpoint{2.837852in}{3.114310in}}%
\pgfpathlineto{\pgfqpoint{2.843739in}{3.122483in}}%
\pgfpathlineto{\pgfqpoint{2.854444in}{3.137526in}}%
\pgfpathlineto{\pgfqpoint{2.858725in}{3.139939in}}%
\pgfpathlineto{\pgfqpoint{2.862204in}{3.139598in}}%
\pgfpathlineto{\pgfqpoint{2.866218in}{3.136700in}}%
\pgfpathlineto{\pgfqpoint{2.871571in}{3.129647in}}%
\pgfpathlineto{\pgfqpoint{2.884148in}{3.112236in}}%
\pgfpathlineto{\pgfqpoint{2.888162in}{3.110398in}}%
\pgfpathlineto{\pgfqpoint{2.891641in}{3.111047in}}%
\pgfpathlineto{\pgfqpoint{2.895655in}{3.114259in}}%
\pgfpathlineto{\pgfqpoint{2.901543in}{3.122408in}}%
\pgfpathlineto{\pgfqpoint{2.912247in}{3.137483in}}%
\pgfpathlineto{\pgfqpoint{2.916529in}{3.139929in}}%
\pgfpathlineto{\pgfqpoint{2.920008in}{3.139617in}}%
\pgfpathlineto{\pgfqpoint{2.923754in}{3.137015in}}%
\pgfpathlineto{\pgfqpoint{2.929106in}{3.130129in}}%
\pgfpathlineto{\pgfqpoint{2.942219in}{3.112066in}}%
\pgfpathlineto{\pgfqpoint{2.946233in}{3.110376in}}%
\pgfpathlineto{\pgfqpoint{2.949712in}{3.111158in}}%
\pgfpathlineto{\pgfqpoint{2.953726in}{3.114502in}}%
\pgfpathlineto{\pgfqpoint{2.959614in}{3.122758in}}%
\pgfpathlineto{\pgfqpoint{2.970050in}{3.137440in}}%
\pgfpathlineto{\pgfqpoint{2.974332in}{3.139919in}}%
\pgfpathlineto{\pgfqpoint{2.977811in}{3.139635in}}%
\pgfpathlineto{\pgfqpoint{2.981557in}{3.137061in}}%
\pgfpathlineto{\pgfqpoint{2.986910in}{3.130201in}}%
\pgfpathlineto{\pgfqpoint{3.000022in}{3.112102in}}%
\pgfpathlineto{\pgfqpoint{3.004036in}{3.110380in}}%
\pgfpathlineto{\pgfqpoint{3.007515in}{3.111133in}}%
\pgfpathlineto{\pgfqpoint{3.011529in}{3.114449in}}%
\pgfpathlineto{\pgfqpoint{3.017417in}{3.122683in}}%
\pgfpathlineto{\pgfqpoint{3.027854in}{3.137397in}}%
\pgfpathlineto{\pgfqpoint{3.032135in}{3.139908in}}%
\pgfpathlineto{\pgfqpoint{3.035614in}{3.139653in}}%
\pgfpathlineto{\pgfqpoint{3.039361in}{3.137107in}}%
\pgfpathlineto{\pgfqpoint{3.044445in}{3.130676in}}%
\pgfpathlineto{\pgfqpoint{3.058093in}{3.111939in}}%
\pgfpathlineto{\pgfqpoint{3.062107in}{3.110365in}}%
\pgfpathlineto{\pgfqpoint{3.065586in}{3.111251in}}%
\pgfpathlineto{\pgfqpoint{3.069600in}{3.114697in}}%
\pgfpathlineto{\pgfqpoint{3.075755in}{3.123462in}}%
\pgfpathlineto{\pgfqpoint{3.085657in}{3.137353in}}%
\pgfpathlineto{\pgfqpoint{3.089939in}{3.139897in}}%
\pgfpathlineto{\pgfqpoint{3.093417in}{3.139671in}}%
\pgfpathlineto{\pgfqpoint{3.097164in}{3.137153in}}%
\pgfpathlineto{\pgfqpoint{3.102248in}{3.130747in}}%
\pgfpathlineto{\pgfqpoint{3.116164in}{3.111783in}}%
\pgfpathlineto{\pgfqpoint{3.120178in}{3.110359in}}%
\pgfpathlineto{\pgfqpoint{3.123657in}{3.111376in}}%
\pgfpathlineto{\pgfqpoint{3.127939in}{3.115268in}}%
\pgfpathlineto{\pgfqpoint{3.134629in}{3.125108in}}%
\pgfpathlineto{\pgfqpoint{3.143460in}{3.137309in}}%
\pgfpathlineto{\pgfqpoint{3.147742in}{3.139886in}}%
\pgfpathlineto{\pgfqpoint{3.151488in}{3.139586in}}%
\pgfpathlineto{\pgfqpoint{3.155502in}{3.136668in}}%
\pgfpathlineto{\pgfqpoint{3.160855in}{3.129599in}}%
\pgfpathlineto{\pgfqpoint{3.173165in}{3.112425in}}%
\pgfpathlineto{\pgfqpoint{3.177446in}{3.110394in}}%
\pgfpathlineto{\pgfqpoint{3.180925in}{3.111062in}}%
\pgfpathlineto{\pgfqpoint{3.184939in}{3.114293in}}%
\pgfpathlineto{\pgfqpoint{3.190827in}{3.122458in}}%
\pgfpathlineto{\pgfqpoint{3.201531in}{3.137512in}}%
\pgfpathlineto{\pgfqpoint{3.205813in}{3.139936in}}%
\pgfpathlineto{\pgfqpoint{3.209292in}{3.139604in}}%
\pgfpathlineto{\pgfqpoint{3.213038in}{3.136984in}}%
\pgfpathlineto{\pgfqpoint{3.218390in}{3.130081in}}%
\pgfpathlineto{\pgfqpoint{3.231235in}{3.112248in}}%
\pgfpathlineto{\pgfqpoint{3.235250in}{3.110399in}}%
\pgfpathlineto{\pgfqpoint{3.238728in}{3.111039in}}%
\pgfpathlineto{\pgfqpoint{3.242743in}{3.114241in}}%
\pgfpathlineto{\pgfqpoint{3.248630in}{3.122383in}}%
\pgfpathlineto{\pgfqpoint{3.259334in}{3.137469in}}%
\pgfpathlineto{\pgfqpoint{3.263616in}{3.139926in}}%
\pgfpathlineto{\pgfqpoint{3.267095in}{3.139623in}}%
\pgfpathlineto{\pgfqpoint{3.270841in}{3.137031in}}%
\pgfpathlineto{\pgfqpoint{3.276194in}{3.130153in}}%
\pgfpathlineto{\pgfqpoint{3.289306in}{3.112078in}}%
\pgfpathlineto{\pgfqpoint{3.293320in}{3.110377in}}%
\pgfpathlineto{\pgfqpoint{3.296799in}{3.111150in}}%
\pgfpathlineto{\pgfqpoint{3.300813in}{3.114484in}}%
\pgfpathlineto{\pgfqpoint{3.306701in}{3.122733in}}%
\pgfpathlineto{\pgfqpoint{3.317138in}{3.137426in}}%
\pgfpathlineto{\pgfqpoint{3.321419in}{3.139915in}}%
\pgfpathlineto{\pgfqpoint{3.324898in}{3.139641in}}%
\pgfpathlineto{\pgfqpoint{3.328645in}{3.137077in}}%
\pgfpathlineto{\pgfqpoint{3.333729in}{3.130629in}}%
\pgfpathlineto{\pgfqpoint{3.347377in}{3.111916in}}%
\pgfpathlineto{\pgfqpoint{3.351391in}{3.110364in}}%
\pgfpathlineto{\pgfqpoint{3.354870in}{3.111268in}}%
\pgfpathlineto{\pgfqpoint{3.358884in}{3.114733in}}%
\pgfpathlineto{\pgfqpoint{3.365039in}{3.123512in}}%
\pgfpathlineto{\pgfqpoint{3.374941in}{3.137382in}}%
\pgfpathlineto{\pgfqpoint{3.379223in}{3.139904in}}%
\pgfpathlineto{\pgfqpoint{3.382701in}{3.139659in}}%
\pgfpathlineto{\pgfqpoint{3.386448in}{3.137122in}}%
\pgfpathlineto{\pgfqpoint{3.391533in}{3.130700in}}%
\pgfpathlineto{\pgfqpoint{3.405181in}{3.111950in}}%
\pgfpathlineto{\pgfqpoint{3.409195in}{3.110366in}}%
\pgfpathlineto{\pgfqpoint{3.412674in}{3.111242in}}%
\pgfpathlineto{\pgfqpoint{3.416688in}{3.114679in}}%
\pgfpathlineto{\pgfqpoint{3.422843in}{3.123436in}}%
\pgfpathlineto{\pgfqpoint{3.432744in}{3.137338in}}%
\pgfpathlineto{\pgfqpoint{3.437026in}{3.139893in}}%
\pgfpathlineto{\pgfqpoint{3.440772in}{3.139573in}}%
\pgfpathlineto{\pgfqpoint{3.444786in}{3.136636in}}%
\pgfpathlineto{\pgfqpoint{3.450139in}{3.129550in}}%
\pgfpathlineto{\pgfqpoint{3.462449in}{3.112399in}}%
\pgfpathlineto{\pgfqpoint{3.466730in}{3.110391in}}%
\pgfpathlineto{\pgfqpoint{3.470209in}{3.111078in}}%
\pgfpathlineto{\pgfqpoint{3.474223in}{3.114327in}}%
\pgfpathlineto{\pgfqpoint{3.480111in}{3.122508in}}%
\pgfpathlineto{\pgfqpoint{3.490815in}{3.137540in}}%
\pgfpathlineto{\pgfqpoint{3.495097in}{3.139942in}}%
\pgfpathlineto{\pgfqpoint{3.498576in}{3.139592in}}%
\pgfpathlineto{\pgfqpoint{3.502590in}{3.136684in}}%
\pgfpathlineto{\pgfqpoint{3.507942in}{3.129623in}}%
\pgfpathlineto{\pgfqpoint{3.520252in}{3.112438in}}%
\pgfpathlineto{\pgfqpoint{3.524534in}{3.110396in}}%
\pgfpathlineto{\pgfqpoint{3.528013in}{3.111055in}}%
\pgfpathlineto{\pgfqpoint{3.532027in}{3.114276in}}%
\pgfpathlineto{\pgfqpoint{3.537914in}{3.122433in}}%
\pgfpathlineto{\pgfqpoint{3.548618in}{3.137497in}}%
\pgfpathlineto{\pgfqpoint{3.552900in}{3.139932in}}%
\pgfpathlineto{\pgfqpoint{3.556379in}{3.139611in}}%
\pgfpathlineto{\pgfqpoint{3.560125in}{3.137000in}}%
\pgfpathlineto{\pgfqpoint{3.565478in}{3.130105in}}%
\pgfpathlineto{\pgfqpoint{3.578590in}{3.112055in}}%
\pgfpathlineto{\pgfqpoint{3.582605in}{3.110375in}}%
\pgfpathlineto{\pgfqpoint{3.586083in}{3.111166in}}%
\pgfpathlineto{\pgfqpoint{3.590098in}{3.114519in}}%
\pgfpathlineto{\pgfqpoint{3.595985in}{3.122783in}}%
\pgfpathlineto{\pgfqpoint{3.606422in}{3.137455in}}%
\pgfpathlineto{\pgfqpoint{3.610703in}{3.139922in}}%
\pgfpathlineto{\pgfqpoint{3.614182in}{3.139629in}}%
\pgfpathlineto{\pgfqpoint{3.617929in}{3.137046in}}%
\pgfpathlineto{\pgfqpoint{3.623281in}{3.130177in}}%
\pgfpathlineto{\pgfqpoint{3.636394in}{3.112090in}}%
\pgfpathlineto{\pgfqpoint{3.640408in}{3.110379in}}%
\pgfpathlineto{\pgfqpoint{3.643887in}{3.111142in}}%
\pgfpathlineto{\pgfqpoint{3.647901in}{3.114467in}}%
\pgfpathlineto{\pgfqpoint{3.653788in}{3.122708in}}%
\pgfpathlineto{\pgfqpoint{3.664225in}{3.137411in}}%
\pgfpathlineto{\pgfqpoint{3.668507in}{3.139912in}}%
\pgfpathlineto{\pgfqpoint{3.671986in}{3.139647in}}%
\pgfpathlineto{\pgfqpoint{3.675732in}{3.137092in}}%
\pgfpathlineto{\pgfqpoint{3.680817in}{3.130653in}}%
\pgfpathlineto{\pgfqpoint{3.694465in}{3.111927in}}%
\pgfpathlineto{\pgfqpoint{3.698479in}{3.110365in}}%
\pgfpathlineto{\pgfqpoint{3.701958in}{3.111259in}}%
\pgfpathlineto{\pgfqpoint{3.705972in}{3.114715in}}%
\pgfpathlineto{\pgfqpoint{3.712127in}{3.123487in}}%
\pgfpathlineto{\pgfqpoint{3.722028in}{3.137368in}}%
\pgfpathlineto{\pgfqpoint{3.726310in}{3.139901in}}%
\pgfpathlineto{\pgfqpoint{3.729789in}{3.139665in}}%
\pgfpathlineto{\pgfqpoint{3.733535in}{3.137137in}}%
\pgfpathlineto{\pgfqpoint{3.738620in}{3.130723in}}%
\pgfpathlineto{\pgfqpoint{3.752535in}{3.111772in}}%
\pgfpathlineto{\pgfqpoint{3.756550in}{3.110359in}}%
\pgfpathlineto{\pgfqpoint{3.760029in}{3.111385in}}%
\pgfpathlineto{\pgfqpoint{3.764310in}{3.115287in}}%
\pgfpathlineto{\pgfqpoint{3.771000in}{3.125133in}}%
\pgfpathlineto{\pgfqpoint{3.779831in}{3.137324in}}%
\pgfpathlineto{\pgfqpoint{3.784113in}{3.139889in}}%
\pgfpathlineto{\pgfqpoint{3.787860in}{3.139579in}}%
\pgfpathlineto{\pgfqpoint{3.791874in}{3.136652in}}%
\pgfpathlineto{\pgfqpoint{3.797226in}{3.129575in}}%
\pgfpathlineto{\pgfqpoint{3.809536in}{3.112412in}}%
\pgfpathlineto{\pgfqpoint{3.813818in}{3.110392in}}%
\pgfpathlineto{\pgfqpoint{3.817297in}{3.111070in}}%
\pgfpathlineto{\pgfqpoint{3.821311in}{3.114310in}}%
\pgfpathlineto{\pgfqpoint{3.827198in}{3.122483in}}%
\pgfpathlineto{\pgfqpoint{3.837902in}{3.137526in}}%
\pgfpathlineto{\pgfqpoint{3.842184in}{3.139939in}}%
\pgfpathlineto{\pgfqpoint{3.845663in}{3.139598in}}%
\pgfpathlineto{\pgfqpoint{3.849677in}{3.136700in}}%
\pgfpathlineto{\pgfqpoint{3.855029in}{3.129647in}}%
\pgfpathlineto{\pgfqpoint{3.867607in}{3.112236in}}%
\pgfpathlineto{\pgfqpoint{3.871621in}{3.110398in}}%
\pgfpathlineto{\pgfqpoint{3.875100in}{3.111047in}}%
\pgfpathlineto{\pgfqpoint{3.879114in}{3.114259in}}%
\pgfpathlineto{\pgfqpoint{3.885001in}{3.122408in}}%
\pgfpathlineto{\pgfqpoint{3.895706in}{3.137483in}}%
\pgfpathlineto{\pgfqpoint{3.899987in}{3.139929in}}%
\pgfpathlineto{\pgfqpoint{3.903466in}{3.139617in}}%
\pgfpathlineto{\pgfqpoint{3.907213in}{3.137015in}}%
\pgfpathlineto{\pgfqpoint{3.912565in}{3.130129in}}%
\pgfpathlineto{\pgfqpoint{3.925678in}{3.112066in}}%
\pgfpathlineto{\pgfqpoint{3.929692in}{3.110376in}}%
\pgfpathlineto{\pgfqpoint{3.933171in}{3.111158in}}%
\pgfpathlineto{\pgfqpoint{3.937185in}{3.114502in}}%
\pgfpathlineto{\pgfqpoint{3.943072in}{3.122758in}}%
\pgfpathlineto{\pgfqpoint{3.953509in}{3.137440in}}%
\pgfpathlineto{\pgfqpoint{3.957791in}{3.139919in}}%
\pgfpathlineto{\pgfqpoint{3.961270in}{3.139635in}}%
\pgfpathlineto{\pgfqpoint{3.965016in}{3.137061in}}%
\pgfpathlineto{\pgfqpoint{3.970368in}{3.130201in}}%
\pgfpathlineto{\pgfqpoint{3.983481in}{3.112102in}}%
\pgfpathlineto{\pgfqpoint{3.987495in}{3.110380in}}%
\pgfpathlineto{\pgfqpoint{3.990974in}{3.111133in}}%
\pgfpathlineto{\pgfqpoint{3.994988in}{3.114449in}}%
\pgfpathlineto{\pgfqpoint{4.000876in}{3.122683in}}%
\pgfpathlineto{\pgfqpoint{4.011312in}{3.137397in}}%
\pgfpathlineto{\pgfqpoint{4.015594in}{3.139908in}}%
\pgfpathlineto{\pgfqpoint{4.019073in}{3.139653in}}%
\pgfpathlineto{\pgfqpoint{4.022819in}{3.137107in}}%
\pgfpathlineto{\pgfqpoint{4.027904in}{3.130676in}}%
\pgfpathlineto{\pgfqpoint{4.041552in}{3.111939in}}%
\pgfpathlineto{\pgfqpoint{4.045566in}{3.110365in}}%
\pgfpathlineto{\pgfqpoint{4.049045in}{3.111251in}}%
\pgfpathlineto{\pgfqpoint{4.053059in}{3.114697in}}%
\pgfpathlineto{\pgfqpoint{4.059214in}{3.123462in}}%
\pgfpathlineto{\pgfqpoint{4.069116in}{3.137353in}}%
\pgfpathlineto{\pgfqpoint{4.073397in}{3.139897in}}%
\pgfpathlineto{\pgfqpoint{4.076876in}{3.139671in}}%
\pgfpathlineto{\pgfqpoint{4.080623in}{3.137153in}}%
\pgfpathlineto{\pgfqpoint{4.085707in}{3.130747in}}%
\pgfpathlineto{\pgfqpoint{4.099623in}{3.111783in}}%
\pgfpathlineto{\pgfqpoint{4.103637in}{3.110359in}}%
\pgfpathlineto{\pgfqpoint{4.107116in}{3.111376in}}%
\pgfpathlineto{\pgfqpoint{4.111398in}{3.115268in}}%
\pgfpathlineto{\pgfqpoint{4.118088in}{3.125108in}}%
\pgfpathlineto{\pgfqpoint{4.126919in}{3.137309in}}%
\pgfpathlineto{\pgfqpoint{4.131201in}{3.139886in}}%
\pgfpathlineto{\pgfqpoint{4.134947in}{3.139586in}}%
\pgfpathlineto{\pgfqpoint{4.138961in}{3.136668in}}%
\pgfpathlineto{\pgfqpoint{4.144313in}{3.129599in}}%
\pgfpathlineto{\pgfqpoint{4.156623in}{3.112425in}}%
\pgfpathlineto{\pgfqpoint{4.160905in}{3.110394in}}%
\pgfpathlineto{\pgfqpoint{4.164384in}{3.111062in}}%
\pgfpathlineto{\pgfqpoint{4.168398in}{3.114293in}}%
\pgfpathlineto{\pgfqpoint{4.174285in}{3.122458in}}%
\pgfpathlineto{\pgfqpoint{4.184990in}{3.137512in}}%
\pgfpathlineto{\pgfqpoint{4.189271in}{3.139936in}}%
\pgfpathlineto{\pgfqpoint{4.192750in}{3.139604in}}%
\pgfpathlineto{\pgfqpoint{4.196497in}{3.136984in}}%
\pgfpathlineto{\pgfqpoint{4.201849in}{3.130081in}}%
\pgfpathlineto{\pgfqpoint{4.214694in}{3.112248in}}%
\pgfpathlineto{\pgfqpoint{4.218708in}{3.110399in}}%
\pgfpathlineto{\pgfqpoint{4.222187in}{3.111039in}}%
\pgfpathlineto{\pgfqpoint{4.226201in}{3.114241in}}%
\pgfpathlineto{\pgfqpoint{4.232089in}{3.122383in}}%
\pgfpathlineto{\pgfqpoint{4.242793in}{3.137469in}}%
\pgfpathlineto{\pgfqpoint{4.247075in}{3.139926in}}%
\pgfpathlineto{\pgfqpoint{4.250554in}{3.139623in}}%
\pgfpathlineto{\pgfqpoint{4.254300in}{3.137031in}}%
\pgfpathlineto{\pgfqpoint{4.259652in}{3.130153in}}%
\pgfpathlineto{\pgfqpoint{4.272765in}{3.112078in}}%
\pgfpathlineto{\pgfqpoint{4.276779in}{3.110377in}}%
\pgfpathlineto{\pgfqpoint{4.280258in}{3.111150in}}%
\pgfpathlineto{\pgfqpoint{4.284272in}{3.114484in}}%
\pgfpathlineto{\pgfqpoint{4.290160in}{3.122733in}}%
\pgfpathlineto{\pgfqpoint{4.300596in}{3.137426in}}%
\pgfpathlineto{\pgfqpoint{4.304878in}{3.139915in}}%
\pgfpathlineto{\pgfqpoint{4.308357in}{3.139641in}}%
\pgfpathlineto{\pgfqpoint{4.312103in}{3.137077in}}%
\pgfpathlineto{\pgfqpoint{4.317188in}{3.130629in}}%
\pgfpathlineto{\pgfqpoint{4.330836in}{3.111916in}}%
\pgfpathlineto{\pgfqpoint{4.334850in}{3.110364in}}%
\pgfpathlineto{\pgfqpoint{4.338329in}{3.111268in}}%
\pgfpathlineto{\pgfqpoint{4.342343in}{3.114733in}}%
\pgfpathlineto{\pgfqpoint{4.348498in}{3.123512in}}%
\pgfpathlineto{\pgfqpoint{4.358400in}{3.137382in}}%
\pgfpathlineto{\pgfqpoint{4.362681in}{3.139904in}}%
\pgfpathlineto{\pgfqpoint{4.366160in}{3.139659in}}%
\pgfpathlineto{\pgfqpoint{4.369907in}{3.137122in}}%
\pgfpathlineto{\pgfqpoint{4.374991in}{3.130700in}}%
\pgfpathlineto{\pgfqpoint{4.388639in}{3.111950in}}%
\pgfpathlineto{\pgfqpoint{4.392653in}{3.110366in}}%
\pgfpathlineto{\pgfqpoint{4.396132in}{3.111242in}}%
\pgfpathlineto{\pgfqpoint{4.400146in}{3.114679in}}%
\pgfpathlineto{\pgfqpoint{4.406301in}{3.123436in}}%
\pgfpathlineto{\pgfqpoint{4.416203in}{3.137338in}}%
\pgfpathlineto{\pgfqpoint{4.420485in}{3.139893in}}%
\pgfpathlineto{\pgfqpoint{4.424231in}{3.139573in}}%
\pgfpathlineto{\pgfqpoint{4.428245in}{3.136636in}}%
\pgfpathlineto{\pgfqpoint{4.433597in}{3.129550in}}%
\pgfpathlineto{\pgfqpoint{4.445907in}{3.112399in}}%
\pgfpathlineto{\pgfqpoint{4.450189in}{3.110391in}}%
\pgfpathlineto{\pgfqpoint{4.453668in}{3.111078in}}%
\pgfpathlineto{\pgfqpoint{4.457682in}{3.114327in}}%
\pgfpathlineto{\pgfqpoint{4.463569in}{3.122508in}}%
\pgfpathlineto{\pgfqpoint{4.474274in}{3.137540in}}%
\pgfpathlineto{\pgfqpoint{4.478556in}{3.139942in}}%
\pgfpathlineto{\pgfqpoint{4.482034in}{3.139592in}}%
\pgfpathlineto{\pgfqpoint{4.486049in}{3.136684in}}%
\pgfpathlineto{\pgfqpoint{4.491401in}{3.129623in}}%
\pgfpathlineto{\pgfqpoint{4.503711in}{3.112438in}}%
\pgfpathlineto{\pgfqpoint{4.507992in}{3.110396in}}%
\pgfpathlineto{\pgfqpoint{4.511471in}{3.111055in}}%
\pgfpathlineto{\pgfqpoint{4.515485in}{3.114276in}}%
\pgfpathlineto{\pgfqpoint{4.521373in}{3.122433in}}%
\pgfpathlineto{\pgfqpoint{4.532077in}{3.137497in}}%
\pgfpathlineto{\pgfqpoint{4.536359in}{3.139932in}}%
\pgfpathlineto{\pgfqpoint{4.539838in}{3.139611in}}%
\pgfpathlineto{\pgfqpoint{4.543584in}{3.137000in}}%
\pgfpathlineto{\pgfqpoint{4.548936in}{3.130105in}}%
\pgfpathlineto{\pgfqpoint{4.562049in}{3.112055in}}%
\pgfpathlineto{\pgfqpoint{4.566063in}{3.110375in}}%
\pgfpathlineto{\pgfqpoint{4.569542in}{3.111166in}}%
\pgfpathlineto{\pgfqpoint{4.573556in}{3.114519in}}%
\pgfpathlineto{\pgfqpoint{4.579444in}{3.122783in}}%
\pgfpathlineto{\pgfqpoint{4.589880in}{3.137455in}}%
\pgfpathlineto{\pgfqpoint{4.594162in}{3.139922in}}%
\pgfpathlineto{\pgfqpoint{4.597641in}{3.139629in}}%
\pgfpathlineto{\pgfqpoint{4.601387in}{3.137046in}}%
\pgfpathlineto{\pgfqpoint{4.606740in}{3.130177in}}%
\pgfpathlineto{\pgfqpoint{4.619852in}{3.112090in}}%
\pgfpathlineto{\pgfqpoint{4.623867in}{3.110379in}}%
\pgfpathlineto{\pgfqpoint{4.627345in}{3.111142in}}%
\pgfpathlineto{\pgfqpoint{4.631360in}{3.114467in}}%
\pgfpathlineto{\pgfqpoint{4.637247in}{3.122708in}}%
\pgfpathlineto{\pgfqpoint{4.647684in}{3.137411in}}%
\pgfpathlineto{\pgfqpoint{4.651965in}{3.139912in}}%
\pgfpathlineto{\pgfqpoint{4.655444in}{3.139647in}}%
\pgfpathlineto{\pgfqpoint{4.659191in}{3.137092in}}%
\pgfpathlineto{\pgfqpoint{4.664275in}{3.130653in}}%
\pgfpathlineto{\pgfqpoint{4.677923in}{3.111927in}}%
\pgfpathlineto{\pgfqpoint{4.681937in}{3.110365in}}%
\pgfpathlineto{\pgfqpoint{4.685416in}{3.111259in}}%
\pgfpathlineto{\pgfqpoint{4.689430in}{3.114715in}}%
\pgfpathlineto{\pgfqpoint{4.695585in}{3.123487in}}%
\pgfpathlineto{\pgfqpoint{4.705487in}{3.137368in}}%
\pgfpathlineto{\pgfqpoint{4.709769in}{3.139901in}}%
\pgfpathlineto{\pgfqpoint{4.713248in}{3.139665in}}%
\pgfpathlineto{\pgfqpoint{4.716994in}{3.137137in}}%
\pgfpathlineto{\pgfqpoint{4.722079in}{3.130723in}}%
\pgfpathlineto{\pgfqpoint{4.735994in}{3.111772in}}%
\pgfpathlineto{\pgfqpoint{4.740008in}{3.110359in}}%
\pgfpathlineto{\pgfqpoint{4.743487in}{3.111385in}}%
\pgfpathlineto{\pgfqpoint{4.747769in}{3.115287in}}%
\pgfpathlineto{\pgfqpoint{4.754459in}{3.125133in}}%
\pgfpathlineto{\pgfqpoint{4.763290in}{3.137324in}}%
\pgfpathlineto{\pgfqpoint{4.767572in}{3.139889in}}%
\pgfpathlineto{\pgfqpoint{4.771318in}{3.139579in}}%
\pgfpathlineto{\pgfqpoint{4.775333in}{3.136652in}}%
\pgfpathlineto{\pgfqpoint{4.780685in}{3.129575in}}%
\pgfpathlineto{\pgfqpoint{4.792995in}{3.112412in}}%
\pgfpathlineto{\pgfqpoint{4.797276in}{3.110392in}}%
\pgfpathlineto{\pgfqpoint{4.800755in}{3.111070in}}%
\pgfpathlineto{\pgfqpoint{4.804769in}{3.114310in}}%
\pgfpathlineto{\pgfqpoint{4.810657in}{3.122483in}}%
\pgfpathlineto{\pgfqpoint{4.821361in}{3.137526in}}%
\pgfpathlineto{\pgfqpoint{4.825643in}{3.139939in}}%
\pgfpathlineto{\pgfqpoint{4.829122in}{3.139598in}}%
\pgfpathlineto{\pgfqpoint{4.833136in}{3.136700in}}%
\pgfpathlineto{\pgfqpoint{4.838488in}{3.129647in}}%
\pgfpathlineto{\pgfqpoint{4.851066in}{3.112236in}}%
\pgfpathlineto{\pgfqpoint{4.855080in}{3.110398in}}%
\pgfpathlineto{\pgfqpoint{4.858559in}{3.111047in}}%
\pgfpathlineto{\pgfqpoint{4.862573in}{3.114259in}}%
\pgfpathlineto{\pgfqpoint{4.868460in}{3.122408in}}%
\pgfpathlineto{\pgfqpoint{4.879164in}{3.137483in}}%
\pgfpathlineto{\pgfqpoint{4.883446in}{3.139929in}}%
\pgfpathlineto{\pgfqpoint{4.886925in}{3.139617in}}%
\pgfpathlineto{\pgfqpoint{4.890672in}{3.137015in}}%
\pgfpathlineto{\pgfqpoint{4.896024in}{3.130129in}}%
\pgfpathlineto{\pgfqpoint{4.909136in}{3.112066in}}%
\pgfpathlineto{\pgfqpoint{4.913151in}{3.110376in}}%
\pgfpathlineto{\pgfqpoint{4.916630in}{3.111158in}}%
\pgfpathlineto{\pgfqpoint{4.920644in}{3.114502in}}%
\pgfpathlineto{\pgfqpoint{4.926531in}{3.122758in}}%
\pgfpathlineto{\pgfqpoint{4.936968in}{3.137440in}}%
\pgfpathlineto{\pgfqpoint{4.941249in}{3.139919in}}%
\pgfpathlineto{\pgfqpoint{4.944728in}{3.139635in}}%
\pgfpathlineto{\pgfqpoint{4.948475in}{3.137061in}}%
\pgfpathlineto{\pgfqpoint{4.953827in}{3.130201in}}%
\pgfpathlineto{\pgfqpoint{4.966940in}{3.112102in}}%
\pgfpathlineto{\pgfqpoint{4.970954in}{3.110380in}}%
\pgfpathlineto{\pgfqpoint{4.974433in}{3.111133in}}%
\pgfpathlineto{\pgfqpoint{4.978447in}{3.114449in}}%
\pgfpathlineto{\pgfqpoint{4.984334in}{3.122683in}}%
\pgfpathlineto{\pgfqpoint{4.994771in}{3.137397in}}%
\pgfpathlineto{\pgfqpoint{4.999053in}{3.139908in}}%
\pgfpathlineto{\pgfqpoint{5.002532in}{3.139653in}}%
\pgfpathlineto{\pgfqpoint{5.006278in}{3.137107in}}%
\pgfpathlineto{\pgfqpoint{5.011363in}{3.130676in}}%
\pgfpathlineto{\pgfqpoint{5.025011in}{3.111939in}}%
\pgfpathlineto{\pgfqpoint{5.029025in}{3.110365in}}%
\pgfpathlineto{\pgfqpoint{5.032504in}{3.111251in}}%
\pgfpathlineto{\pgfqpoint{5.036518in}{3.114697in}}%
\pgfpathlineto{\pgfqpoint{5.042673in}{3.123462in}}%
\pgfpathlineto{\pgfqpoint{5.052574in}{3.137353in}}%
\pgfpathlineto{\pgfqpoint{5.056856in}{3.139897in}}%
\pgfpathlineto{\pgfqpoint{5.060335in}{3.139671in}}%
\pgfpathlineto{\pgfqpoint{5.064081in}{3.137153in}}%
\pgfpathlineto{\pgfqpoint{5.069166in}{3.130747in}}%
\pgfpathlineto{\pgfqpoint{5.083082in}{3.111783in}}%
\pgfpathlineto{\pgfqpoint{5.087096in}{3.110359in}}%
\pgfpathlineto{\pgfqpoint{5.090575in}{3.111376in}}%
\pgfpathlineto{\pgfqpoint{5.094856in}{3.115268in}}%
\pgfpathlineto{\pgfqpoint{5.101547in}{3.125108in}}%
\pgfpathlineto{\pgfqpoint{5.110378in}{3.137309in}}%
\pgfpathlineto{\pgfqpoint{5.114659in}{3.139886in}}%
\pgfpathlineto{\pgfqpoint{5.118406in}{3.139586in}}%
\pgfpathlineto{\pgfqpoint{5.122420in}{3.136668in}}%
\pgfpathlineto{\pgfqpoint{5.127772in}{3.129599in}}%
\pgfpathlineto{\pgfqpoint{5.140082in}{3.112425in}}%
\pgfpathlineto{\pgfqpoint{5.144364in}{3.110394in}}%
\pgfpathlineto{\pgfqpoint{5.147843in}{3.111062in}}%
\pgfpathlineto{\pgfqpoint{5.151857in}{3.114293in}}%
\pgfpathlineto{\pgfqpoint{5.157744in}{3.122458in}}%
\pgfpathlineto{\pgfqpoint{5.168448in}{3.137512in}}%
\pgfpathlineto{\pgfqpoint{5.172730in}{3.139936in}}%
\pgfpathlineto{\pgfqpoint{5.176209in}{3.139604in}}%
\pgfpathlineto{\pgfqpoint{5.179956in}{3.136984in}}%
\pgfpathlineto{\pgfqpoint{5.185308in}{3.130081in}}%
\pgfpathlineto{\pgfqpoint{5.198153in}{3.112248in}}%
\pgfpathlineto{\pgfqpoint{5.202167in}{3.110399in}}%
\pgfpathlineto{\pgfqpoint{5.205646in}{3.111039in}}%
\pgfpathlineto{\pgfqpoint{5.209660in}{3.114241in}}%
\pgfpathlineto{\pgfqpoint{5.215547in}{3.122383in}}%
\pgfpathlineto{\pgfqpoint{5.226252in}{3.137469in}}%
\pgfpathlineto{\pgfqpoint{5.230533in}{3.139926in}}%
\pgfpathlineto{\pgfqpoint{5.234012in}{3.139623in}}%
\pgfpathlineto{\pgfqpoint{5.237759in}{3.137031in}}%
\pgfpathlineto{\pgfqpoint{5.243111in}{3.130153in}}%
\pgfpathlineto{\pgfqpoint{5.256224in}{3.112078in}}%
\pgfpathlineto{\pgfqpoint{5.260238in}{3.110377in}}%
\pgfpathlineto{\pgfqpoint{5.263717in}{3.111150in}}%
\pgfpathlineto{\pgfqpoint{5.267731in}{3.114484in}}%
\pgfpathlineto{\pgfqpoint{5.273618in}{3.122733in}}%
\pgfpathlineto{\pgfqpoint{5.284055in}{3.137426in}}%
\pgfpathlineto{\pgfqpoint{5.288337in}{3.139915in}}%
\pgfpathlineto{\pgfqpoint{5.291816in}{3.139641in}}%
\pgfpathlineto{\pgfqpoint{5.295562in}{3.137077in}}%
\pgfpathlineto{\pgfqpoint{5.300647in}{3.130629in}}%
\pgfpathlineto{\pgfqpoint{5.314295in}{3.111916in}}%
\pgfpathlineto{\pgfqpoint{5.318309in}{3.110364in}}%
\pgfpathlineto{\pgfqpoint{5.321788in}{3.111268in}}%
\pgfpathlineto{\pgfqpoint{5.325802in}{3.114733in}}%
\pgfpathlineto{\pgfqpoint{5.331957in}{3.123512in}}%
\pgfpathlineto{\pgfqpoint{5.341858in}{3.137382in}}%
\pgfpathlineto{\pgfqpoint{5.346140in}{3.139904in}}%
\pgfpathlineto{\pgfqpoint{5.349619in}{3.139659in}}%
\pgfpathlineto{\pgfqpoint{5.353365in}{3.137122in}}%
\pgfpathlineto{\pgfqpoint{5.358450in}{3.130700in}}%
\pgfpathlineto{\pgfqpoint{5.372098in}{3.111950in}}%
\pgfpathlineto{\pgfqpoint{5.376112in}{3.110366in}}%
\pgfpathlineto{\pgfqpoint{5.379591in}{3.111242in}}%
\pgfpathlineto{\pgfqpoint{5.383605in}{3.114679in}}%
\pgfpathlineto{\pgfqpoint{5.389760in}{3.123436in}}%
\pgfpathlineto{\pgfqpoint{5.399662in}{3.137338in}}%
\pgfpathlineto{\pgfqpoint{5.403943in}{3.139893in}}%
\pgfpathlineto{\pgfqpoint{5.407690in}{3.139573in}}%
\pgfpathlineto{\pgfqpoint{5.411704in}{3.136636in}}%
\pgfpathlineto{\pgfqpoint{5.417056in}{3.129550in}}%
\pgfpathlineto{\pgfqpoint{5.429366in}{3.112399in}}%
\pgfpathlineto{\pgfqpoint{5.433648in}{3.110391in}}%
\pgfpathlineto{\pgfqpoint{5.437127in}{3.111078in}}%
\pgfpathlineto{\pgfqpoint{5.441141in}{3.114327in}}%
\pgfpathlineto{\pgfqpoint{5.447028in}{3.122508in}}%
\pgfpathlineto{\pgfqpoint{5.457733in}{3.137540in}}%
\pgfpathlineto{\pgfqpoint{5.462014in}{3.139942in}}%
\pgfpathlineto{\pgfqpoint{5.465493in}{3.139592in}}%
\pgfpathlineto{\pgfqpoint{5.469507in}{3.136684in}}%
\pgfpathlineto{\pgfqpoint{5.474859in}{3.129623in}}%
\pgfpathlineto{\pgfqpoint{5.487169in}{3.112438in}}%
\pgfpathlineto{\pgfqpoint{5.491451in}{3.110396in}}%
\pgfpathlineto{\pgfqpoint{5.494930in}{3.111055in}}%
\pgfpathlineto{\pgfqpoint{5.498944in}{3.114276in}}%
\pgfpathlineto{\pgfqpoint{5.504831in}{3.122433in}}%
\pgfpathlineto{\pgfqpoint{5.515536in}{3.137497in}}%
\pgfpathlineto{\pgfqpoint{5.519818in}{3.139932in}}%
\pgfpathlineto{\pgfqpoint{5.523296in}{3.139611in}}%
\pgfpathlineto{\pgfqpoint{5.527043in}{3.137000in}}%
\pgfpathlineto{\pgfqpoint{5.532395in}{3.130105in}}%
\pgfpathlineto{\pgfqpoint{5.545508in}{3.112055in}}%
\pgfpathlineto{\pgfqpoint{5.549522in}{3.110375in}}%
\pgfpathlineto{\pgfqpoint{5.553001in}{3.111166in}}%
\pgfpathlineto{\pgfqpoint{5.557015in}{3.114519in}}%
\pgfpathlineto{\pgfqpoint{5.562902in}{3.122783in}}%
\pgfpathlineto{\pgfqpoint{5.573339in}{3.137455in}}%
\pgfpathlineto{\pgfqpoint{5.577621in}{3.139922in}}%
\pgfpathlineto{\pgfqpoint{5.581100in}{3.139629in}}%
\pgfpathlineto{\pgfqpoint{5.584846in}{3.137046in}}%
\pgfpathlineto{\pgfqpoint{5.590198in}{3.130177in}}%
\pgfpathlineto{\pgfqpoint{5.603311in}{3.112090in}}%
\pgfpathlineto{\pgfqpoint{5.607325in}{3.110379in}}%
\pgfpathlineto{\pgfqpoint{5.610804in}{3.111142in}}%
\pgfpathlineto{\pgfqpoint{5.614818in}{3.114467in}}%
\pgfpathlineto{\pgfqpoint{5.620706in}{3.122708in}}%
\pgfpathlineto{\pgfqpoint{5.631142in}{3.137411in}}%
\pgfpathlineto{\pgfqpoint{5.635424in}{3.139912in}}%
\pgfpathlineto{\pgfqpoint{5.638903in}{3.139647in}}%
\pgfpathlineto{\pgfqpoint{5.642650in}{3.137092in}}%
\pgfpathlineto{\pgfqpoint{5.647734in}{3.130653in}}%
\pgfpathlineto{\pgfqpoint{5.661382in}{3.111927in}}%
\pgfpathlineto{\pgfqpoint{5.665396in}{3.110365in}}%
\pgfpathlineto{\pgfqpoint{5.668875in}{3.111259in}}%
\pgfpathlineto{\pgfqpoint{5.672889in}{3.114715in}}%
\pgfpathlineto{\pgfqpoint{5.679044in}{3.123487in}}%
\pgfpathlineto{\pgfqpoint{5.688946in}{3.137368in}}%
\pgfpathlineto{\pgfqpoint{5.693227in}{3.139901in}}%
\pgfpathlineto{\pgfqpoint{5.696706in}{3.139665in}}%
\pgfpathlineto{\pgfqpoint{5.700453in}{3.137137in}}%
\pgfpathlineto{\pgfqpoint{5.705537in}{3.130723in}}%
\pgfpathlineto{\pgfqpoint{5.719453in}{3.111772in}}%
\pgfpathlineto{\pgfqpoint{5.723467in}{3.110359in}}%
\pgfpathlineto{\pgfqpoint{5.726946in}{3.111385in}}%
\pgfpathlineto{\pgfqpoint{5.731228in}{3.115287in}}%
\pgfpathlineto{\pgfqpoint{5.737918in}{3.125133in}}%
\pgfpathlineto{\pgfqpoint{5.746749in}{3.137324in}}%
\pgfpathlineto{\pgfqpoint{5.751031in}{3.139889in}}%
\pgfpathlineto{\pgfqpoint{5.754777in}{3.139579in}}%
\pgfpathlineto{\pgfqpoint{5.758791in}{3.136652in}}%
\pgfpathlineto{\pgfqpoint{5.764143in}{3.129575in}}%
\pgfpathlineto{\pgfqpoint{5.776453in}{3.112412in}}%
\pgfpathlineto{\pgfqpoint{5.780735in}{3.110392in}}%
\pgfpathlineto{\pgfqpoint{5.784214in}{3.111070in}}%
\pgfpathlineto{\pgfqpoint{5.788228in}{3.114310in}}%
\pgfpathlineto{\pgfqpoint{5.794116in}{3.122483in}}%
\pgfpathlineto{\pgfqpoint{5.804820in}{3.137526in}}%
\pgfpathlineto{\pgfqpoint{5.809102in}{3.139939in}}%
\pgfpathlineto{\pgfqpoint{5.812580in}{3.139598in}}%
\pgfpathlineto{\pgfqpoint{5.816595in}{3.136700in}}%
\pgfpathlineto{\pgfqpoint{5.821947in}{3.129647in}}%
\pgfpathlineto{\pgfqpoint{5.834524in}{3.112236in}}%
\pgfpathlineto{\pgfqpoint{5.838538in}{3.110398in}}%
\pgfpathlineto{\pgfqpoint{5.842017in}{3.111047in}}%
\pgfpathlineto{\pgfqpoint{5.846031in}{3.114259in}}%
\pgfpathlineto{\pgfqpoint{5.851919in}{3.122408in}}%
\pgfpathlineto{\pgfqpoint{5.862623in}{3.137483in}}%
\pgfpathlineto{\pgfqpoint{5.866905in}{3.139929in}}%
\pgfpathlineto{\pgfqpoint{5.870384in}{3.139617in}}%
\pgfpathlineto{\pgfqpoint{5.874130in}{3.137015in}}%
\pgfpathlineto{\pgfqpoint{5.879482in}{3.130129in}}%
\pgfpathlineto{\pgfqpoint{5.892595in}{3.112066in}}%
\pgfpathlineto{\pgfqpoint{5.896609in}{3.110376in}}%
\pgfpathlineto{\pgfqpoint{5.900088in}{3.111158in}}%
\pgfpathlineto{\pgfqpoint{5.904102in}{3.114502in}}%
\pgfpathlineto{\pgfqpoint{5.909990in}{3.122758in}}%
\pgfpathlineto{\pgfqpoint{5.920426in}{3.137440in}}%
\pgfpathlineto{\pgfqpoint{5.924708in}{3.139919in}}%
\pgfpathlineto{\pgfqpoint{5.928187in}{3.139635in}}%
\pgfpathlineto{\pgfqpoint{5.931934in}{3.137061in}}%
\pgfpathlineto{\pgfqpoint{5.937286in}{3.130201in}}%
\pgfpathlineto{\pgfqpoint{5.950399in}{3.112102in}}%
\pgfpathlineto{\pgfqpoint{5.954413in}{3.110380in}}%
\pgfpathlineto{\pgfqpoint{5.957892in}{3.111133in}}%
\pgfpathlineto{\pgfqpoint{5.961906in}{3.114449in}}%
\pgfpathlineto{\pgfqpoint{5.967793in}{3.122683in}}%
\pgfpathlineto{\pgfqpoint{5.978230in}{3.137397in}}%
\pgfpathlineto{\pgfqpoint{5.982511in}{3.139908in}}%
\pgfpathlineto{\pgfqpoint{5.985990in}{3.139653in}}%
\pgfpathlineto{\pgfqpoint{5.989737in}{3.137107in}}%
\pgfpathlineto{\pgfqpoint{5.994821in}{3.130676in}}%
\pgfpathlineto{\pgfqpoint{6.008469in}{3.111939in}}%
\pgfpathlineto{\pgfqpoint{6.012484in}{3.110365in}}%
\pgfpathlineto{\pgfqpoint{6.015962in}{3.111251in}}%
\pgfpathlineto{\pgfqpoint{6.019977in}{3.114697in}}%
\pgfpathlineto{\pgfqpoint{6.026132in}{3.123462in}}%
\pgfpathlineto{\pgfqpoint{6.036033in}{3.137353in}}%
\pgfpathlineto{\pgfqpoint{6.040315in}{3.139897in}}%
\pgfpathlineto{\pgfqpoint{6.043794in}{3.139671in}}%
\pgfpathlineto{\pgfqpoint{6.047540in}{3.137153in}}%
\pgfpathlineto{\pgfqpoint{6.052625in}{3.130747in}}%
\pgfpathlineto{\pgfqpoint{6.066540in}{3.111783in}}%
\pgfpathlineto{\pgfqpoint{6.070554in}{3.110359in}}%
\pgfpathlineto{\pgfqpoint{6.074033in}{3.111376in}}%
\pgfpathlineto{\pgfqpoint{6.078315in}{3.115268in}}%
\pgfpathlineto{\pgfqpoint{6.085005in}{3.125108in}}%
\pgfpathlineto{\pgfqpoint{6.093836in}{3.137309in}}%
\pgfpathlineto{\pgfqpoint{6.098118in}{3.139886in}}%
\pgfpathlineto{\pgfqpoint{6.101865in}{3.139586in}}%
\pgfpathlineto{\pgfqpoint{6.105879in}{3.136668in}}%
\pgfpathlineto{\pgfqpoint{6.111231in}{3.129599in}}%
\pgfpathlineto{\pgfqpoint{6.123541in}{3.112425in}}%
\pgfpathlineto{\pgfqpoint{6.127823in}{3.110394in}}%
\pgfpathlineto{\pgfqpoint{6.131301in}{3.111062in}}%
\pgfpathlineto{\pgfqpoint{6.135316in}{3.114293in}}%
\pgfpathlineto{\pgfqpoint{6.141203in}{3.122458in}}%
\pgfpathlineto{\pgfqpoint{6.151907in}{3.137512in}}%
\pgfpathlineto{\pgfqpoint{6.156189in}{3.139936in}}%
\pgfpathlineto{\pgfqpoint{6.159668in}{3.139604in}}%
\pgfpathlineto{\pgfqpoint{6.163414in}{3.136984in}}%
\pgfpathlineto{\pgfqpoint{6.168767in}{3.130081in}}%
\pgfpathlineto{\pgfqpoint{6.181612in}{3.112248in}}%
\pgfpathlineto{\pgfqpoint{6.185626in}{3.110399in}}%
\pgfpathlineto{\pgfqpoint{6.189105in}{3.111039in}}%
\pgfpathlineto{\pgfqpoint{6.193119in}{3.114241in}}%
\pgfpathlineto{\pgfqpoint{6.199006in}{3.122383in}}%
\pgfpathlineto{\pgfqpoint{6.209711in}{3.137469in}}%
\pgfpathlineto{\pgfqpoint{6.213992in}{3.139926in}}%
\pgfpathlineto{\pgfqpoint{6.217471in}{3.139623in}}%
\pgfpathlineto{\pgfqpoint{6.221218in}{3.137031in}}%
\pgfpathlineto{\pgfqpoint{6.226570in}{3.130153in}}%
\pgfpathlineto{\pgfqpoint{6.239683in}{3.112078in}}%
\pgfpathlineto{\pgfqpoint{6.243697in}{3.110377in}}%
\pgfpathlineto{\pgfqpoint{6.247176in}{3.111150in}}%
\pgfpathlineto{\pgfqpoint{6.251190in}{3.114484in}}%
\pgfpathlineto{\pgfqpoint{6.257077in}{3.122733in}}%
\pgfpathlineto{\pgfqpoint{6.267514in}{3.137426in}}%
\pgfpathlineto{\pgfqpoint{6.271796in}{3.139915in}}%
\pgfpathlineto{\pgfqpoint{6.275274in}{3.139641in}}%
\pgfpathlineto{\pgfqpoint{6.279021in}{3.137077in}}%
\pgfpathlineto{\pgfqpoint{6.284105in}{3.130629in}}%
\pgfpathlineto{\pgfqpoint{6.297753in}{3.111916in}}%
\pgfpathlineto{\pgfqpoint{6.301768in}{3.110364in}}%
\pgfpathlineto{\pgfqpoint{6.305246in}{3.111268in}}%
\pgfpathlineto{\pgfqpoint{6.309261in}{3.114733in}}%
\pgfpathlineto{\pgfqpoint{6.315416in}{3.123512in}}%
\pgfpathlineto{\pgfqpoint{6.325317in}{3.137382in}}%
\pgfpathlineto{\pgfqpoint{6.329599in}{3.139904in}}%
\pgfpathlineto{\pgfqpoint{6.333078in}{3.139659in}}%
\pgfpathlineto{\pgfqpoint{6.336824in}{3.137122in}}%
\pgfpathlineto{\pgfqpoint{6.341909in}{3.130700in}}%
\pgfpathlineto{\pgfqpoint{6.355557in}{3.111950in}}%
\pgfpathlineto{\pgfqpoint{6.359571in}{3.110366in}}%
\pgfpathlineto{\pgfqpoint{6.363050in}{3.111242in}}%
\pgfpathlineto{\pgfqpoint{6.367064in}{3.114679in}}%
\pgfpathlineto{\pgfqpoint{6.373219in}{3.123436in}}%
\pgfpathlineto{\pgfqpoint{6.383120in}{3.137338in}}%
\pgfpathlineto{\pgfqpoint{6.387402in}{3.139893in}}%
\pgfpathlineto{\pgfqpoint{6.391149in}{3.139573in}}%
\pgfpathlineto{\pgfqpoint{6.395163in}{3.136636in}}%
\pgfpathlineto{\pgfqpoint{6.400515in}{3.129550in}}%
\pgfpathlineto{\pgfqpoint{6.412825in}{3.112399in}}%
\pgfpathlineto{\pgfqpoint{6.417107in}{3.110391in}}%
\pgfpathlineto{\pgfqpoint{6.420585in}{3.111078in}}%
\pgfpathlineto{\pgfqpoint{6.424600in}{3.114327in}}%
\pgfpathlineto{\pgfqpoint{6.430487in}{3.122508in}}%
\pgfpathlineto{\pgfqpoint{6.441191in}{3.137540in}}%
\pgfpathlineto{\pgfqpoint{6.445473in}{3.139942in}}%
\pgfpathlineto{\pgfqpoint{6.448952in}{3.139592in}}%
\pgfpathlineto{\pgfqpoint{6.452966in}{3.136684in}}%
\pgfpathlineto{\pgfqpoint{6.458318in}{3.129623in}}%
\pgfpathlineto{\pgfqpoint{6.470628in}{3.112438in}}%
\pgfpathlineto{\pgfqpoint{6.474910in}{3.110396in}}%
\pgfpathlineto{\pgfqpoint{6.478389in}{3.111055in}}%
\pgfpathlineto{\pgfqpoint{6.482403in}{3.114276in}}%
\pgfpathlineto{\pgfqpoint{6.488290in}{3.122433in}}%
\pgfpathlineto{\pgfqpoint{6.498995in}{3.137497in}}%
\pgfpathlineto{\pgfqpoint{6.503276in}{3.139932in}}%
\pgfpathlineto{\pgfqpoint{6.506755in}{3.139611in}}%
\pgfpathlineto{\pgfqpoint{6.510502in}{3.137000in}}%
\pgfpathlineto{\pgfqpoint{6.515854in}{3.130105in}}%
\pgfpathlineto{\pgfqpoint{6.528967in}{3.112055in}}%
\pgfpathlineto{\pgfqpoint{6.532981in}{3.110375in}}%
\pgfpathlineto{\pgfqpoint{6.536460in}{3.111166in}}%
\pgfpathlineto{\pgfqpoint{6.540474in}{3.114519in}}%
\pgfpathlineto{\pgfqpoint{6.546361in}{3.122783in}}%
\pgfpathlineto{\pgfqpoint{6.556798in}{3.137455in}}%
\pgfpathlineto{\pgfqpoint{6.561080in}{3.139922in}}%
\pgfpathlineto{\pgfqpoint{6.564558in}{3.139629in}}%
\pgfpathlineto{\pgfqpoint{6.568305in}{3.137046in}}%
\pgfpathlineto{\pgfqpoint{6.573657in}{3.130177in}}%
\pgfpathlineto{\pgfqpoint{6.586770in}{3.112090in}}%
\pgfpathlineto{\pgfqpoint{6.590784in}{3.110379in}}%
\pgfpathlineto{\pgfqpoint{6.594263in}{3.111142in}}%
\pgfpathlineto{\pgfqpoint{6.598277in}{3.114467in}}%
\pgfpathlineto{\pgfqpoint{6.604164in}{3.122708in}}%
\pgfpathlineto{\pgfqpoint{6.614601in}{3.137411in}}%
\pgfpathlineto{\pgfqpoint{6.618883in}{3.139912in}}%
\pgfpathlineto{\pgfqpoint{6.622362in}{3.139647in}}%
\pgfpathlineto{\pgfqpoint{6.626108in}{3.137092in}}%
\pgfpathlineto{\pgfqpoint{6.631193in}{3.130653in}}%
\pgfpathlineto{\pgfqpoint{6.644841in}{3.111927in}}%
\pgfpathlineto{\pgfqpoint{6.648855in}{3.110365in}}%
\pgfpathlineto{\pgfqpoint{6.652334in}{3.111259in}}%
\pgfpathlineto{\pgfqpoint{6.656348in}{3.114715in}}%
\pgfpathlineto{\pgfqpoint{6.662503in}{3.123487in}}%
\pgfpathlineto{\pgfqpoint{6.663306in}{3.124778in}}%
\pgfpathlineto{\pgfqpoint{6.663306in}{3.124778in}}%
\pgfusepath{stroke}%
\end{pgfscope}%
\begin{pgfscope}%
\pgfpathrectangle{\pgfqpoint{0.467797in}{2.292089in}}{\pgfqpoint{6.490533in}{1.666241in}}%
\pgfusepath{clip}%
\pgfsetrectcap%
\pgfsetroundjoin%
\pgfsetlinewidth{1.505625pt}%
\definecolor{currentstroke}{rgb}{0.890196,0.466667,0.760784}%
\pgfsetstrokecolor{currentstroke}%
\pgfsetdash{}{0pt}%
\pgfpathmoveto{\pgfqpoint{0.762821in}{3.125209in}}%
\pgfpathlineto{\pgfqpoint{0.771384in}{3.136969in}}%
\pgfpathlineto{\pgfqpoint{0.775666in}{3.139395in}}%
\pgfpathlineto{\pgfqpoint{0.779145in}{3.138975in}}%
\pgfpathlineto{\pgfqpoint{0.782891in}{3.136189in}}%
\pgfpathlineto{\pgfqpoint{0.788244in}{3.129050in}}%
\pgfpathlineto{\pgfqpoint{0.799751in}{3.112983in}}%
\pgfpathlineto{\pgfqpoint{0.803765in}{3.110978in}}%
\pgfpathlineto{\pgfqpoint{0.807244in}{3.111564in}}%
\pgfpathlineto{\pgfqpoint{0.811258in}{3.114799in}}%
\pgfpathlineto{\pgfqpoint{0.817145in}{3.123051in}}%
\pgfpathlineto{\pgfqpoint{0.827047in}{3.136965in}}%
\pgfpathlineto{\pgfqpoint{0.831329in}{3.139394in}}%
\pgfpathlineto{\pgfqpoint{0.834807in}{3.138978in}}%
\pgfpathlineto{\pgfqpoint{0.838554in}{3.136194in}}%
\pgfpathlineto{\pgfqpoint{0.843906in}{3.129058in}}%
\pgfpathlineto{\pgfqpoint{0.855413in}{3.112987in}}%
\pgfpathlineto{\pgfqpoint{0.859427in}{3.110979in}}%
\pgfpathlineto{\pgfqpoint{0.862906in}{3.111562in}}%
\pgfpathlineto{\pgfqpoint{0.866920in}{3.114794in}}%
\pgfpathlineto{\pgfqpoint{0.872808in}{3.123043in}}%
\pgfpathlineto{\pgfqpoint{0.882977in}{3.137200in}}%
\pgfpathlineto{\pgfqpoint{0.887259in}{3.139439in}}%
\pgfpathlineto{\pgfqpoint{0.890737in}{3.138858in}}%
\pgfpathlineto{\pgfqpoint{0.894752in}{3.135628in}}%
\pgfpathlineto{\pgfqpoint{0.900639in}{3.127380in}}%
\pgfpathlineto{\pgfqpoint{0.910808in}{3.113221in}}%
\pgfpathlineto{\pgfqpoint{0.915090in}{3.110980in}}%
\pgfpathlineto{\pgfqpoint{0.918569in}{3.111559in}}%
\pgfpathlineto{\pgfqpoint{0.922583in}{3.114788in}}%
\pgfpathlineto{\pgfqpoint{0.928470in}{3.123035in}}%
\pgfpathlineto{\pgfqpoint{0.938639in}{3.137196in}}%
\pgfpathlineto{\pgfqpoint{0.942921in}{3.139439in}}%
\pgfpathlineto{\pgfqpoint{0.946400in}{3.138861in}}%
\pgfpathlineto{\pgfqpoint{0.950414in}{3.135634in}}%
\pgfpathlineto{\pgfqpoint{0.956301in}{3.127388in}}%
\pgfpathlineto{\pgfqpoint{0.966470in}{3.113225in}}%
\pgfpathlineto{\pgfqpoint{0.970752in}{3.110981in}}%
\pgfpathlineto{\pgfqpoint{0.974231in}{3.111557in}}%
\pgfpathlineto{\pgfqpoint{0.978245in}{3.114783in}}%
\pgfpathlineto{\pgfqpoint{0.984133in}{3.123027in}}%
\pgfpathlineto{\pgfqpoint{0.994302in}{3.137191in}}%
\pgfpathlineto{\pgfqpoint{0.998583in}{3.139438in}}%
\pgfpathlineto{\pgfqpoint{1.002062in}{3.138863in}}%
\pgfpathlineto{\pgfqpoint{1.006076in}{3.135639in}}%
\pgfpathlineto{\pgfqpoint{1.011964in}{3.127396in}}%
\pgfpathlineto{\pgfqpoint{1.022133in}{3.113230in}}%
\pgfpathlineto{\pgfqpoint{1.026415in}{3.110981in}}%
\pgfpathlineto{\pgfqpoint{1.029894in}{3.111555in}}%
\pgfpathlineto{\pgfqpoint{1.033908in}{3.114777in}}%
\pgfpathlineto{\pgfqpoint{1.039795in}{3.123019in}}%
\pgfpathlineto{\pgfqpoint{1.049964in}{3.137187in}}%
\pgfpathlineto{\pgfqpoint{1.054246in}{3.139437in}}%
\pgfpathlineto{\pgfqpoint{1.057725in}{3.138865in}}%
\pgfpathlineto{\pgfqpoint{1.061739in}{3.135645in}}%
\pgfpathlineto{\pgfqpoint{1.067626in}{3.127404in}}%
\pgfpathlineto{\pgfqpoint{1.077795in}{3.113234in}}%
\pgfpathlineto{\pgfqpoint{1.082077in}{3.110982in}}%
\pgfpathlineto{\pgfqpoint{1.085556in}{3.111552in}}%
\pgfpathlineto{\pgfqpoint{1.089570in}{3.114771in}}%
\pgfpathlineto{\pgfqpoint{1.095457in}{3.123011in}}%
\pgfpathlineto{\pgfqpoint{1.105627in}{3.137183in}}%
\pgfpathlineto{\pgfqpoint{1.109908in}{3.139436in}}%
\pgfpathlineto{\pgfqpoint{1.113387in}{3.138868in}}%
\pgfpathlineto{\pgfqpoint{1.117401in}{3.135650in}}%
\pgfpathlineto{\pgfqpoint{1.123289in}{3.127412in}}%
\pgfpathlineto{\pgfqpoint{1.133458in}{3.113239in}}%
\pgfpathlineto{\pgfqpoint{1.137739in}{3.110983in}}%
\pgfpathlineto{\pgfqpoint{1.141218in}{3.111550in}}%
\pgfpathlineto{\pgfqpoint{1.145232in}{3.114766in}}%
\pgfpathlineto{\pgfqpoint{1.151120in}{3.123003in}}%
\pgfpathlineto{\pgfqpoint{1.161289in}{3.137178in}}%
\pgfpathlineto{\pgfqpoint{1.165571in}{3.139436in}}%
\pgfpathlineto{\pgfqpoint{1.169050in}{3.138870in}}%
\pgfpathlineto{\pgfqpoint{1.173064in}{3.135656in}}%
\pgfpathlineto{\pgfqpoint{1.178951in}{3.127420in}}%
\pgfpathlineto{\pgfqpoint{1.189120in}{3.113243in}}%
\pgfpathlineto{\pgfqpoint{1.193402in}{3.110984in}}%
\pgfpathlineto{\pgfqpoint{1.196881in}{3.111547in}}%
\pgfpathlineto{\pgfqpoint{1.200895in}{3.114760in}}%
\pgfpathlineto{\pgfqpoint{1.206782in}{3.122995in}}%
\pgfpathlineto{\pgfqpoint{1.216951in}{3.137174in}}%
\pgfpathlineto{\pgfqpoint{1.221233in}{3.139435in}}%
\pgfpathlineto{\pgfqpoint{1.224712in}{3.138873in}}%
\pgfpathlineto{\pgfqpoint{1.228726in}{3.135661in}}%
\pgfpathlineto{\pgfqpoint{1.234613in}{3.127428in}}%
\pgfpathlineto{\pgfqpoint{1.244783in}{3.113247in}}%
\pgfpathlineto{\pgfqpoint{1.249064in}{3.110984in}}%
\pgfpathlineto{\pgfqpoint{1.252543in}{3.111545in}}%
\pgfpathlineto{\pgfqpoint{1.256557in}{3.114755in}}%
\pgfpathlineto{\pgfqpoint{1.262445in}{3.122987in}}%
\pgfpathlineto{\pgfqpoint{1.272614in}{3.137169in}}%
\pgfpathlineto{\pgfqpoint{1.276896in}{3.139434in}}%
\pgfpathlineto{\pgfqpoint{1.280374in}{3.138875in}}%
\pgfpathlineto{\pgfqpoint{1.284389in}{3.135667in}}%
\pgfpathlineto{\pgfqpoint{1.290276in}{3.127436in}}%
\pgfpathlineto{\pgfqpoint{1.300445in}{3.113252in}}%
\pgfpathlineto{\pgfqpoint{1.304727in}{3.110985in}}%
\pgfpathlineto{\pgfqpoint{1.308206in}{3.111543in}}%
\pgfpathlineto{\pgfqpoint{1.312220in}{3.114749in}}%
\pgfpathlineto{\pgfqpoint{1.318107in}{3.122979in}}%
\pgfpathlineto{\pgfqpoint{1.328276in}{3.137165in}}%
\pgfpathlineto{\pgfqpoint{1.332558in}{3.139433in}}%
\pgfpathlineto{\pgfqpoint{1.336037in}{3.138877in}}%
\pgfpathlineto{\pgfqpoint{1.340051in}{3.135672in}}%
\pgfpathlineto{\pgfqpoint{1.345938in}{3.127444in}}%
\pgfpathlineto{\pgfqpoint{1.356107in}{3.113256in}}%
\pgfpathlineto{\pgfqpoint{1.360389in}{3.110986in}}%
\pgfpathlineto{\pgfqpoint{1.363868in}{3.111540in}}%
\pgfpathlineto{\pgfqpoint{1.367882in}{3.114744in}}%
\pgfpathlineto{\pgfqpoint{1.373770in}{3.122971in}}%
\pgfpathlineto{\pgfqpoint{1.383939in}{3.137160in}}%
\pgfpathlineto{\pgfqpoint{1.388220in}{3.139432in}}%
\pgfpathlineto{\pgfqpoint{1.391699in}{3.138880in}}%
\pgfpathlineto{\pgfqpoint{1.395713in}{3.135678in}}%
\pgfpathlineto{\pgfqpoint{1.401601in}{3.127452in}}%
\pgfpathlineto{\pgfqpoint{1.411770in}{3.113261in}}%
\pgfpathlineto{\pgfqpoint{1.416052in}{3.110987in}}%
\pgfpathlineto{\pgfqpoint{1.419531in}{3.111538in}}%
\pgfpathlineto{\pgfqpoint{1.423545in}{3.114738in}}%
\pgfpathlineto{\pgfqpoint{1.429432in}{3.122963in}}%
\pgfpathlineto{\pgfqpoint{1.439601in}{3.137156in}}%
\pgfpathlineto{\pgfqpoint{1.443883in}{3.139432in}}%
\pgfpathlineto{\pgfqpoint{1.447362in}{3.138882in}}%
\pgfpathlineto{\pgfqpoint{1.451376in}{3.135684in}}%
\pgfpathlineto{\pgfqpoint{1.457263in}{3.127460in}}%
\pgfpathlineto{\pgfqpoint{1.467432in}{3.113265in}}%
\pgfpathlineto{\pgfqpoint{1.471714in}{3.110988in}}%
\pgfpathlineto{\pgfqpoint{1.475193in}{3.111536in}}%
\pgfpathlineto{\pgfqpoint{1.479207in}{3.114733in}}%
\pgfpathlineto{\pgfqpoint{1.485094in}{3.122955in}}%
\pgfpathlineto{\pgfqpoint{1.495264in}{3.137151in}}%
\pgfpathlineto{\pgfqpoint{1.499545in}{3.139431in}}%
\pgfpathlineto{\pgfqpoint{1.503024in}{3.138884in}}%
\pgfpathlineto{\pgfqpoint{1.507038in}{3.135689in}}%
\pgfpathlineto{\pgfqpoint{1.512926in}{3.127468in}}%
\pgfpathlineto{\pgfqpoint{1.523095in}{3.113270in}}%
\pgfpathlineto{\pgfqpoint{1.527376in}{3.110988in}}%
\pgfpathlineto{\pgfqpoint{1.530855in}{3.111533in}}%
\pgfpathlineto{\pgfqpoint{1.534869in}{3.114727in}}%
\pgfpathlineto{\pgfqpoint{1.540757in}{3.122947in}}%
\pgfpathlineto{\pgfqpoint{1.550926in}{3.137147in}}%
\pgfpathlineto{\pgfqpoint{1.555208in}{3.139430in}}%
\pgfpathlineto{\pgfqpoint{1.558687in}{3.138887in}}%
\pgfpathlineto{\pgfqpoint{1.562701in}{3.135695in}}%
\pgfpathlineto{\pgfqpoint{1.568588in}{3.127476in}}%
\pgfpathlineto{\pgfqpoint{1.578757in}{3.113274in}}%
\pgfpathlineto{\pgfqpoint{1.583039in}{3.110989in}}%
\pgfpathlineto{\pgfqpoint{1.586518in}{3.111531in}}%
\pgfpathlineto{\pgfqpoint{1.590532in}{3.114721in}}%
\pgfpathlineto{\pgfqpoint{1.596152in}{3.122514in}}%
\pgfpathlineto{\pgfqpoint{1.606588in}{3.137142in}}%
\pgfpathlineto{\pgfqpoint{1.610870in}{3.139429in}}%
\pgfpathlineto{\pgfqpoint{1.614349in}{3.138889in}}%
\pgfpathlineto{\pgfqpoint{1.618363in}{3.135700in}}%
\pgfpathlineto{\pgfqpoint{1.623983in}{3.127909in}}%
\pgfpathlineto{\pgfqpoint{1.634420in}{3.113279in}}%
\pgfpathlineto{\pgfqpoint{1.638701in}{3.110990in}}%
\pgfpathlineto{\pgfqpoint{1.642180in}{3.111529in}}%
\pgfpathlineto{\pgfqpoint{1.646194in}{3.114716in}}%
\pgfpathlineto{\pgfqpoint{1.651814in}{3.122506in}}%
\pgfpathlineto{\pgfqpoint{1.662251in}{3.137138in}}%
\pgfpathlineto{\pgfqpoint{1.666533in}{3.139428in}}%
\pgfpathlineto{\pgfqpoint{1.670011in}{3.138892in}}%
\pgfpathlineto{\pgfqpoint{1.674026in}{3.135706in}}%
\pgfpathlineto{\pgfqpoint{1.679645in}{3.127917in}}%
\pgfpathlineto{\pgfqpoint{1.690082in}{3.113283in}}%
\pgfpathlineto{\pgfqpoint{1.694364in}{3.110991in}}%
\pgfpathlineto{\pgfqpoint{1.697843in}{3.111526in}}%
\pgfpathlineto{\pgfqpoint{1.701857in}{3.114710in}}%
\pgfpathlineto{\pgfqpoint{1.707477in}{3.122498in}}%
\pgfpathlineto{\pgfqpoint{1.717913in}{3.137133in}}%
\pgfpathlineto{\pgfqpoint{1.722195in}{3.139428in}}%
\pgfpathlineto{\pgfqpoint{1.725674in}{3.138894in}}%
\pgfpathlineto{\pgfqpoint{1.729688in}{3.135711in}}%
\pgfpathlineto{\pgfqpoint{1.735308in}{3.127925in}}%
\pgfpathlineto{\pgfqpoint{1.745744in}{3.113288in}}%
\pgfpathlineto{\pgfqpoint{1.750026in}{3.110992in}}%
\pgfpathlineto{\pgfqpoint{1.753505in}{3.111524in}}%
\pgfpathlineto{\pgfqpoint{1.757519in}{3.114705in}}%
\pgfpathlineto{\pgfqpoint{1.763139in}{3.122490in}}%
\pgfpathlineto{\pgfqpoint{1.773576in}{3.137129in}}%
\pgfpathlineto{\pgfqpoint{1.777857in}{3.139427in}}%
\pgfpathlineto{\pgfqpoint{1.781336in}{3.138896in}}%
\pgfpathlineto{\pgfqpoint{1.785350in}{3.135717in}}%
\pgfpathlineto{\pgfqpoint{1.790970in}{3.127933in}}%
\pgfpathlineto{\pgfqpoint{1.801407in}{3.113292in}}%
\pgfpathlineto{\pgfqpoint{1.805689in}{3.110992in}}%
\pgfpathlineto{\pgfqpoint{1.809167in}{3.111521in}}%
\pgfpathlineto{\pgfqpoint{1.813182in}{3.114699in}}%
\pgfpathlineto{\pgfqpoint{1.818801in}{3.122482in}}%
\pgfpathlineto{\pgfqpoint{1.829238in}{3.137124in}}%
\pgfpathlineto{\pgfqpoint{1.833520in}{3.139426in}}%
\pgfpathlineto{\pgfqpoint{1.836999in}{3.138899in}}%
\pgfpathlineto{\pgfqpoint{1.841013in}{3.135722in}}%
\pgfpathlineto{\pgfqpoint{1.846633in}{3.127941in}}%
\pgfpathlineto{\pgfqpoint{1.857069in}{3.113297in}}%
\pgfpathlineto{\pgfqpoint{1.861351in}{3.110993in}}%
\pgfpathlineto{\pgfqpoint{1.864830in}{3.111519in}}%
\pgfpathlineto{\pgfqpoint{1.868844in}{3.114694in}}%
\pgfpathlineto{\pgfqpoint{1.874464in}{3.122474in}}%
\pgfpathlineto{\pgfqpoint{1.884901in}{3.137120in}}%
\pgfpathlineto{\pgfqpoint{1.889182in}{3.139425in}}%
\pgfpathlineto{\pgfqpoint{1.892661in}{3.138901in}}%
\pgfpathlineto{\pgfqpoint{1.896675in}{3.135728in}}%
\pgfpathlineto{\pgfqpoint{1.902295in}{3.127949in}}%
\pgfpathlineto{\pgfqpoint{1.912732in}{3.113301in}}%
\pgfpathlineto{\pgfqpoint{1.917013in}{3.110994in}}%
\pgfpathlineto{\pgfqpoint{1.920492in}{3.111517in}}%
\pgfpathlineto{\pgfqpoint{1.924506in}{3.114688in}}%
\pgfpathlineto{\pgfqpoint{1.930126in}{3.122466in}}%
\pgfpathlineto{\pgfqpoint{1.940563in}{3.137115in}}%
\pgfpathlineto{\pgfqpoint{1.944845in}{3.139424in}}%
\pgfpathlineto{\pgfqpoint{1.948324in}{3.138903in}}%
\pgfpathlineto{\pgfqpoint{1.952338in}{3.135733in}}%
\pgfpathlineto{\pgfqpoint{1.957957in}{3.127957in}}%
\pgfpathlineto{\pgfqpoint{1.968394in}{3.113306in}}%
\pgfpathlineto{\pgfqpoint{1.972676in}{3.110995in}}%
\pgfpathlineto{\pgfqpoint{1.976155in}{3.111514in}}%
\pgfpathlineto{\pgfqpoint{1.980169in}{3.114683in}}%
\pgfpathlineto{\pgfqpoint{1.985789in}{3.122458in}}%
\pgfpathlineto{\pgfqpoint{1.996225in}{3.137111in}}%
\pgfpathlineto{\pgfqpoint{2.000507in}{3.139423in}}%
\pgfpathlineto{\pgfqpoint{2.003986in}{3.138906in}}%
\pgfpathlineto{\pgfqpoint{2.008000in}{3.135739in}}%
\pgfpathlineto{\pgfqpoint{2.013620in}{3.127965in}}%
\pgfpathlineto{\pgfqpoint{2.024057in}{3.113310in}}%
\pgfpathlineto{\pgfqpoint{2.028338in}{3.110996in}}%
\pgfpathlineto{\pgfqpoint{2.031817in}{3.111512in}}%
\pgfpathlineto{\pgfqpoint{2.035831in}{3.114677in}}%
\pgfpathlineto{\pgfqpoint{2.041451in}{3.122450in}}%
\pgfpathlineto{\pgfqpoint{2.051888in}{3.137106in}}%
\pgfpathlineto{\pgfqpoint{2.056170in}{3.139423in}}%
\pgfpathlineto{\pgfqpoint{2.059648in}{3.138908in}}%
\pgfpathlineto{\pgfqpoint{2.063663in}{3.135744in}}%
\pgfpathlineto{\pgfqpoint{2.069282in}{3.127973in}}%
\pgfpathlineto{\pgfqpoint{2.079719in}{3.113315in}}%
\pgfpathlineto{\pgfqpoint{2.084001in}{3.110997in}}%
\pgfpathlineto{\pgfqpoint{2.087480in}{3.111510in}}%
\pgfpathlineto{\pgfqpoint{2.091494in}{3.114672in}}%
\pgfpathlineto{\pgfqpoint{2.097114in}{3.122442in}}%
\pgfpathlineto{\pgfqpoint{2.107550in}{3.137102in}}%
\pgfpathlineto{\pgfqpoint{2.111832in}{3.139422in}}%
\pgfpathlineto{\pgfqpoint{2.115311in}{3.138910in}}%
\pgfpathlineto{\pgfqpoint{2.119325in}{3.135750in}}%
\pgfpathlineto{\pgfqpoint{2.124945in}{3.127981in}}%
\pgfpathlineto{\pgfqpoint{2.135381in}{3.113319in}}%
\pgfpathlineto{\pgfqpoint{2.139663in}{3.110997in}}%
\pgfpathlineto{\pgfqpoint{2.143142in}{3.111508in}}%
\pgfpathlineto{\pgfqpoint{2.147156in}{3.114666in}}%
\pgfpathlineto{\pgfqpoint{2.152776in}{3.122434in}}%
\pgfpathlineto{\pgfqpoint{2.163213in}{3.137097in}}%
\pgfpathlineto{\pgfqpoint{2.167494in}{3.139421in}}%
\pgfpathlineto{\pgfqpoint{2.170973in}{3.138912in}}%
\pgfpathlineto{\pgfqpoint{2.174987in}{3.135755in}}%
\pgfpathlineto{\pgfqpoint{2.180607in}{3.127989in}}%
\pgfpathlineto{\pgfqpoint{2.191044in}{3.113324in}}%
\pgfpathlineto{\pgfqpoint{2.195326in}{3.110998in}}%
\pgfpathlineto{\pgfqpoint{2.198804in}{3.111505in}}%
\pgfpathlineto{\pgfqpoint{2.202819in}{3.114661in}}%
\pgfpathlineto{\pgfqpoint{2.208438in}{3.122426in}}%
\pgfpathlineto{\pgfqpoint{2.219143in}{3.137327in}}%
\pgfpathlineto{\pgfqpoint{2.223424in}{3.139459in}}%
\pgfpathlineto{\pgfqpoint{2.226903in}{3.138786in}}%
\pgfpathlineto{\pgfqpoint{2.230917in}{3.135465in}}%
\pgfpathlineto{\pgfqpoint{2.236805in}{3.127146in}}%
\pgfpathlineto{\pgfqpoint{2.246706in}{3.113328in}}%
\pgfpathlineto{\pgfqpoint{2.250988in}{3.110999in}}%
\pgfpathlineto{\pgfqpoint{2.254467in}{3.111503in}}%
\pgfpathlineto{\pgfqpoint{2.258481in}{3.114655in}}%
\pgfpathlineto{\pgfqpoint{2.264101in}{3.122418in}}%
\pgfpathlineto{\pgfqpoint{2.274805in}{3.137323in}}%
\pgfpathlineto{\pgfqpoint{2.279087in}{3.139458in}}%
\pgfpathlineto{\pgfqpoint{2.282566in}{3.138789in}}%
\pgfpathlineto{\pgfqpoint{2.286580in}{3.135471in}}%
\pgfpathlineto{\pgfqpoint{2.292467in}{3.127154in}}%
\pgfpathlineto{\pgfqpoint{2.302369in}{3.113333in}}%
\pgfpathlineto{\pgfqpoint{2.306650in}{3.111000in}}%
\pgfpathlineto{\pgfqpoint{2.310129in}{3.111501in}}%
\pgfpathlineto{\pgfqpoint{2.314143in}{3.114650in}}%
\pgfpathlineto{\pgfqpoint{2.319763in}{3.122410in}}%
\pgfpathlineto{\pgfqpoint{2.330468in}{3.137319in}}%
\pgfpathlineto{\pgfqpoint{2.334749in}{3.139458in}}%
\pgfpathlineto{\pgfqpoint{2.338228in}{3.138791in}}%
\pgfpathlineto{\pgfqpoint{2.342242in}{3.135476in}}%
\pgfpathlineto{\pgfqpoint{2.348130in}{3.127162in}}%
\pgfpathlineto{\pgfqpoint{2.358031in}{3.113337in}}%
\pgfpathlineto{\pgfqpoint{2.362313in}{3.111001in}}%
\pgfpathlineto{\pgfqpoint{2.365792in}{3.111498in}}%
\pgfpathlineto{\pgfqpoint{2.369806in}{3.114644in}}%
\pgfpathlineto{\pgfqpoint{2.375426in}{3.122402in}}%
\pgfpathlineto{\pgfqpoint{2.386130in}{3.137314in}}%
\pgfpathlineto{\pgfqpoint{2.390412in}{3.139457in}}%
\pgfpathlineto{\pgfqpoint{2.393891in}{3.138794in}}%
\pgfpathlineto{\pgfqpoint{2.397905in}{3.135482in}}%
\pgfpathlineto{\pgfqpoint{2.403792in}{3.127170in}}%
\pgfpathlineto{\pgfqpoint{2.413694in}{3.113342in}}%
\pgfpathlineto{\pgfqpoint{2.417975in}{3.111002in}}%
\pgfpathlineto{\pgfqpoint{2.421454in}{3.111496in}}%
\pgfpathlineto{\pgfqpoint{2.425468in}{3.114639in}}%
\pgfpathlineto{\pgfqpoint{2.431088in}{3.122394in}}%
\pgfpathlineto{\pgfqpoint{2.441792in}{3.137310in}}%
\pgfpathlineto{\pgfqpoint{2.446074in}{3.139457in}}%
\pgfpathlineto{\pgfqpoint{2.449553in}{3.138796in}}%
\pgfpathlineto{\pgfqpoint{2.453567in}{3.135488in}}%
\pgfpathlineto{\pgfqpoint{2.459454in}{3.127178in}}%
\pgfpathlineto{\pgfqpoint{2.469356in}{3.113346in}}%
\pgfpathlineto{\pgfqpoint{2.473638in}{3.111003in}}%
\pgfpathlineto{\pgfqpoint{2.477117in}{3.111494in}}%
\pgfpathlineto{\pgfqpoint{2.481131in}{3.114633in}}%
\pgfpathlineto{\pgfqpoint{2.486750in}{3.122386in}}%
\pgfpathlineto{\pgfqpoint{2.497455in}{3.137306in}}%
\pgfpathlineto{\pgfqpoint{2.501737in}{3.139456in}}%
\pgfpathlineto{\pgfqpoint{2.505215in}{3.138799in}}%
\pgfpathlineto{\pgfqpoint{2.509230in}{3.135493in}}%
\pgfpathlineto{\pgfqpoint{2.515117in}{3.127186in}}%
\pgfpathlineto{\pgfqpoint{2.525018in}{3.113351in}}%
\pgfpathlineto{\pgfqpoint{2.529300in}{3.111004in}}%
\pgfpathlineto{\pgfqpoint{2.532779in}{3.111491in}}%
\pgfpathlineto{\pgfqpoint{2.536793in}{3.114628in}}%
\pgfpathlineto{\pgfqpoint{2.542413in}{3.122378in}}%
\pgfpathlineto{\pgfqpoint{2.553117in}{3.137301in}}%
\pgfpathlineto{\pgfqpoint{2.557399in}{3.139455in}}%
\pgfpathlineto{\pgfqpoint{2.560878in}{3.138802in}}%
\pgfpathlineto{\pgfqpoint{2.564892in}{3.135499in}}%
\pgfpathlineto{\pgfqpoint{2.570779in}{3.127194in}}%
\pgfpathlineto{\pgfqpoint{2.580681in}{3.113355in}}%
\pgfpathlineto{\pgfqpoint{2.584963in}{3.111004in}}%
\pgfpathlineto{\pgfqpoint{2.588441in}{3.111489in}}%
\pgfpathlineto{\pgfqpoint{2.592456in}{3.114622in}}%
\pgfpathlineto{\pgfqpoint{2.598075in}{3.122370in}}%
\pgfpathlineto{\pgfqpoint{2.608780in}{3.137297in}}%
\pgfpathlineto{\pgfqpoint{2.613061in}{3.139455in}}%
\pgfpathlineto{\pgfqpoint{2.616540in}{3.138804in}}%
\pgfpathlineto{\pgfqpoint{2.620554in}{3.135505in}}%
\pgfpathlineto{\pgfqpoint{2.626442in}{3.127202in}}%
\pgfpathlineto{\pgfqpoint{2.636343in}{3.113360in}}%
\pgfpathlineto{\pgfqpoint{2.640625in}{3.111005in}}%
\pgfpathlineto{\pgfqpoint{2.644104in}{3.111487in}}%
\pgfpathlineto{\pgfqpoint{2.647850in}{3.114332in}}%
\pgfpathlineto{\pgfqpoint{2.653203in}{3.121522in}}%
\pgfpathlineto{\pgfqpoint{2.664710in}{3.137517in}}%
\pgfpathlineto{\pgfqpoint{2.668724in}{3.139454in}}%
\pgfpathlineto{\pgfqpoint{2.672203in}{3.138807in}}%
\pgfpathlineto{\pgfqpoint{2.676217in}{3.135510in}}%
\pgfpathlineto{\pgfqpoint{2.682104in}{3.127210in}}%
\pgfpathlineto{\pgfqpoint{2.692006in}{3.113365in}}%
\pgfpathlineto{\pgfqpoint{2.696287in}{3.111006in}}%
\pgfpathlineto{\pgfqpoint{2.699766in}{3.111485in}}%
\pgfpathlineto{\pgfqpoint{2.703513in}{3.114327in}}%
\pgfpathlineto{\pgfqpoint{2.708865in}{3.121514in}}%
\pgfpathlineto{\pgfqpoint{2.720372in}{3.137513in}}%
\pgfpathlineto{\pgfqpoint{2.724386in}{3.139453in}}%
\pgfpathlineto{\pgfqpoint{2.727865in}{3.138809in}}%
\pgfpathlineto{\pgfqpoint{2.731879in}{3.135516in}}%
\pgfpathlineto{\pgfqpoint{2.737767in}{3.127218in}}%
\pgfpathlineto{\pgfqpoint{2.747668in}{3.113369in}}%
\pgfpathlineto{\pgfqpoint{2.751950in}{3.111007in}}%
\pgfpathlineto{\pgfqpoint{2.755429in}{3.111482in}}%
\pgfpathlineto{\pgfqpoint{2.759175in}{3.114322in}}%
\pgfpathlineto{\pgfqpoint{2.764527in}{3.121506in}}%
\pgfpathlineto{\pgfqpoint{2.776035in}{3.137509in}}%
\pgfpathlineto{\pgfqpoint{2.780049in}{3.139453in}}%
\pgfpathlineto{\pgfqpoint{2.783528in}{3.138812in}}%
\pgfpathlineto{\pgfqpoint{2.787542in}{3.135521in}}%
\pgfpathlineto{\pgfqpoint{2.793429in}{3.127226in}}%
\pgfpathlineto{\pgfqpoint{2.803331in}{3.113374in}}%
\pgfpathlineto{\pgfqpoint{2.807612in}{3.111008in}}%
\pgfpathlineto{\pgfqpoint{2.811091in}{3.111480in}}%
\pgfpathlineto{\pgfqpoint{2.814838in}{3.114316in}}%
\pgfpathlineto{\pgfqpoint{2.820190in}{3.121498in}}%
\pgfpathlineto{\pgfqpoint{2.831697in}{3.137505in}}%
\pgfpathlineto{\pgfqpoint{2.835711in}{3.139452in}}%
\pgfpathlineto{\pgfqpoint{2.839190in}{3.138814in}}%
\pgfpathlineto{\pgfqpoint{2.843204in}{3.135527in}}%
\pgfpathlineto{\pgfqpoint{2.849091in}{3.127235in}}%
\pgfpathlineto{\pgfqpoint{2.858993in}{3.113378in}}%
\pgfpathlineto{\pgfqpoint{2.863275in}{3.111009in}}%
\pgfpathlineto{\pgfqpoint{2.866754in}{3.111478in}}%
\pgfpathlineto{\pgfqpoint{2.870500in}{3.114311in}}%
\pgfpathlineto{\pgfqpoint{2.875852in}{3.121490in}}%
\pgfpathlineto{\pgfqpoint{2.887359in}{3.137501in}}%
\pgfpathlineto{\pgfqpoint{2.891374in}{3.139451in}}%
\pgfpathlineto{\pgfqpoint{2.894852in}{3.138817in}}%
\pgfpathlineto{\pgfqpoint{2.898867in}{3.135533in}}%
\pgfpathlineto{\pgfqpoint{2.904754in}{3.127243in}}%
\pgfpathlineto{\pgfqpoint{2.914655in}{3.113383in}}%
\pgfpathlineto{\pgfqpoint{2.918937in}{3.111010in}}%
\pgfpathlineto{\pgfqpoint{2.922416in}{3.111476in}}%
\pgfpathlineto{\pgfqpoint{2.926163in}{3.114306in}}%
\pgfpathlineto{\pgfqpoint{2.931515in}{3.121482in}}%
\pgfpathlineto{\pgfqpoint{2.943022in}{3.137497in}}%
\pgfpathlineto{\pgfqpoint{2.947036in}{3.139451in}}%
\pgfpathlineto{\pgfqpoint{2.950515in}{3.138819in}}%
\pgfpathlineto{\pgfqpoint{2.954529in}{3.135538in}}%
\pgfpathlineto{\pgfqpoint{2.960416in}{3.127251in}}%
\pgfpathlineto{\pgfqpoint{2.970318in}{3.113387in}}%
\pgfpathlineto{\pgfqpoint{2.974600in}{3.111011in}}%
\pgfpathlineto{\pgfqpoint{2.978078in}{3.111473in}}%
\pgfpathlineto{\pgfqpoint{2.981825in}{3.114301in}}%
\pgfpathlineto{\pgfqpoint{2.987177in}{3.121474in}}%
\pgfpathlineto{\pgfqpoint{2.998684in}{3.137493in}}%
\pgfpathlineto{\pgfqpoint{3.002698in}{3.139450in}}%
\pgfpathlineto{\pgfqpoint{3.006177in}{3.138821in}}%
\pgfpathlineto{\pgfqpoint{3.010191in}{3.135544in}}%
\pgfpathlineto{\pgfqpoint{3.016079in}{3.127259in}}%
\pgfpathlineto{\pgfqpoint{3.025980in}{3.113392in}}%
\pgfpathlineto{\pgfqpoint{3.030262in}{3.111012in}}%
\pgfpathlineto{\pgfqpoint{3.033741in}{3.111471in}}%
\pgfpathlineto{\pgfqpoint{3.037487in}{3.114295in}}%
\pgfpathlineto{\pgfqpoint{3.042840in}{3.121467in}}%
\pgfpathlineto{\pgfqpoint{3.054347in}{3.137488in}}%
\pgfpathlineto{\pgfqpoint{3.058361in}{3.139449in}}%
\pgfpathlineto{\pgfqpoint{3.061840in}{3.138824in}}%
\pgfpathlineto{\pgfqpoint{3.065854in}{3.135550in}}%
\pgfpathlineto{\pgfqpoint{3.071741in}{3.127267in}}%
\pgfpathlineto{\pgfqpoint{3.081643in}{3.113397in}}%
\pgfpathlineto{\pgfqpoint{3.085924in}{3.111013in}}%
\pgfpathlineto{\pgfqpoint{3.089403in}{3.111469in}}%
\pgfpathlineto{\pgfqpoint{3.093150in}{3.114290in}}%
\pgfpathlineto{\pgfqpoint{3.098502in}{3.121459in}}%
\pgfpathlineto{\pgfqpoint{3.110009in}{3.137484in}}%
\pgfpathlineto{\pgfqpoint{3.114023in}{3.139449in}}%
\pgfpathlineto{\pgfqpoint{3.117502in}{3.138826in}}%
\pgfpathlineto{\pgfqpoint{3.121516in}{3.135555in}}%
\pgfpathlineto{\pgfqpoint{3.127404in}{3.127275in}}%
\pgfpathlineto{\pgfqpoint{3.137305in}{3.113401in}}%
\pgfpathlineto{\pgfqpoint{3.141587in}{3.111014in}}%
\pgfpathlineto{\pgfqpoint{3.145066in}{3.111467in}}%
\pgfpathlineto{\pgfqpoint{3.148812in}{3.114285in}}%
\pgfpathlineto{\pgfqpoint{3.154164in}{3.121451in}}%
\pgfpathlineto{\pgfqpoint{3.165672in}{3.137480in}}%
\pgfpathlineto{\pgfqpoint{3.169686in}{3.139448in}}%
\pgfpathlineto{\pgfqpoint{3.173165in}{3.138829in}}%
\pgfpathlineto{\pgfqpoint{3.177179in}{3.135561in}}%
\pgfpathlineto{\pgfqpoint{3.183066in}{3.127283in}}%
\pgfpathlineto{\pgfqpoint{3.192968in}{3.113406in}}%
\pgfpathlineto{\pgfqpoint{3.197249in}{3.111014in}}%
\pgfpathlineto{\pgfqpoint{3.200728in}{3.111464in}}%
\pgfpathlineto{\pgfqpoint{3.204475in}{3.114280in}}%
\pgfpathlineto{\pgfqpoint{3.209827in}{3.121443in}}%
\pgfpathlineto{\pgfqpoint{3.221334in}{3.137476in}}%
\pgfpathlineto{\pgfqpoint{3.225348in}{3.139447in}}%
\pgfpathlineto{\pgfqpoint{3.228827in}{3.138831in}}%
\pgfpathlineto{\pgfqpoint{3.232841in}{3.135566in}}%
\pgfpathlineto{\pgfqpoint{3.238728in}{3.127291in}}%
\pgfpathlineto{\pgfqpoint{3.248630in}{3.113410in}}%
\pgfpathlineto{\pgfqpoint{3.252912in}{3.111015in}}%
\pgfpathlineto{\pgfqpoint{3.256391in}{3.111462in}}%
\pgfpathlineto{\pgfqpoint{3.260137in}{3.114274in}}%
\pgfpathlineto{\pgfqpoint{3.265489in}{3.121435in}}%
\pgfpathlineto{\pgfqpoint{3.276996in}{3.137472in}}%
\pgfpathlineto{\pgfqpoint{3.281011in}{3.139447in}}%
\pgfpathlineto{\pgfqpoint{3.284489in}{3.138834in}}%
\pgfpathlineto{\pgfqpoint{3.288504in}{3.135572in}}%
\pgfpathlineto{\pgfqpoint{3.294391in}{3.127299in}}%
\pgfpathlineto{\pgfqpoint{3.304292in}{3.113415in}}%
\pgfpathlineto{\pgfqpoint{3.308574in}{3.111016in}}%
\pgfpathlineto{\pgfqpoint{3.312053in}{3.111460in}}%
\pgfpathlineto{\pgfqpoint{3.315800in}{3.114269in}}%
\pgfpathlineto{\pgfqpoint{3.321152in}{3.121427in}}%
\pgfpathlineto{\pgfqpoint{3.332659in}{3.137467in}}%
\pgfpathlineto{\pgfqpoint{3.336673in}{3.139446in}}%
\pgfpathlineto{\pgfqpoint{3.340152in}{3.138836in}}%
\pgfpathlineto{\pgfqpoint{3.344166in}{3.135578in}}%
\pgfpathlineto{\pgfqpoint{3.350053in}{3.127307in}}%
\pgfpathlineto{\pgfqpoint{3.359955in}{3.113420in}}%
\pgfpathlineto{\pgfqpoint{3.364237in}{3.111017in}}%
\pgfpathlineto{\pgfqpoint{3.367715in}{3.111458in}}%
\pgfpathlineto{\pgfqpoint{3.371462in}{3.114264in}}%
\pgfpathlineto{\pgfqpoint{3.376814in}{3.121419in}}%
\pgfpathlineto{\pgfqpoint{3.388321in}{3.137463in}}%
\pgfpathlineto{\pgfqpoint{3.392335in}{3.139445in}}%
\pgfpathlineto{\pgfqpoint{3.395814in}{3.138839in}}%
\pgfpathlineto{\pgfqpoint{3.399828in}{3.135583in}}%
\pgfpathlineto{\pgfqpoint{3.405716in}{3.127315in}}%
\pgfpathlineto{\pgfqpoint{3.415617in}{3.113424in}}%
\pgfpathlineto{\pgfqpoint{3.419899in}{3.111018in}}%
\pgfpathlineto{\pgfqpoint{3.423378in}{3.111456in}}%
\pgfpathlineto{\pgfqpoint{3.427124in}{3.114259in}}%
\pgfpathlineto{\pgfqpoint{3.432477in}{3.121412in}}%
\pgfpathlineto{\pgfqpoint{3.443984in}{3.137459in}}%
\pgfpathlineto{\pgfqpoint{3.447998in}{3.139444in}}%
\pgfpathlineto{\pgfqpoint{3.451477in}{3.138841in}}%
\pgfpathlineto{\pgfqpoint{3.455491in}{3.135589in}}%
\pgfpathlineto{\pgfqpoint{3.461378in}{3.127323in}}%
\pgfpathlineto{\pgfqpoint{3.471280in}{3.113429in}}%
\pgfpathlineto{\pgfqpoint{3.475561in}{3.111019in}}%
\pgfpathlineto{\pgfqpoint{3.479040in}{3.111453in}}%
\pgfpathlineto{\pgfqpoint{3.482787in}{3.114253in}}%
\pgfpathlineto{\pgfqpoint{3.488139in}{3.121404in}}%
\pgfpathlineto{\pgfqpoint{3.499646in}{3.137455in}}%
\pgfpathlineto{\pgfqpoint{3.503660in}{3.139444in}}%
\pgfpathlineto{\pgfqpoint{3.507139in}{3.138844in}}%
\pgfpathlineto{\pgfqpoint{3.511153in}{3.135594in}}%
\pgfpathlineto{\pgfqpoint{3.517041in}{3.127331in}}%
\pgfpathlineto{\pgfqpoint{3.526942in}{3.113433in}}%
\pgfpathlineto{\pgfqpoint{3.531224in}{3.111020in}}%
\pgfpathlineto{\pgfqpoint{3.534703in}{3.111451in}}%
\pgfpathlineto{\pgfqpoint{3.538449in}{3.114248in}}%
\pgfpathlineto{\pgfqpoint{3.543801in}{3.121396in}}%
\pgfpathlineto{\pgfqpoint{3.555309in}{3.137451in}}%
\pgfpathlineto{\pgfqpoint{3.559323in}{3.139443in}}%
\pgfpathlineto{\pgfqpoint{3.562802in}{3.138846in}}%
\pgfpathlineto{\pgfqpoint{3.566816in}{3.135600in}}%
\pgfpathlineto{\pgfqpoint{3.572703in}{3.127339in}}%
\pgfpathlineto{\pgfqpoint{3.582605in}{3.113438in}}%
\pgfpathlineto{\pgfqpoint{3.586886in}{3.111021in}}%
\pgfpathlineto{\pgfqpoint{3.590365in}{3.111449in}}%
\pgfpathlineto{\pgfqpoint{3.594112in}{3.114243in}}%
\pgfpathlineto{\pgfqpoint{3.599464in}{3.121388in}}%
\pgfpathlineto{\pgfqpoint{3.610971in}{3.137446in}}%
\pgfpathlineto{\pgfqpoint{3.614985in}{3.139442in}}%
\pgfpathlineto{\pgfqpoint{3.618464in}{3.138848in}}%
\pgfpathlineto{\pgfqpoint{3.622478in}{3.135606in}}%
\pgfpathlineto{\pgfqpoint{3.628365in}{3.127347in}}%
\pgfpathlineto{\pgfqpoint{3.638267in}{3.113443in}}%
\pgfpathlineto{\pgfqpoint{3.642549in}{3.111022in}}%
\pgfpathlineto{\pgfqpoint{3.646028in}{3.111447in}}%
\pgfpathlineto{\pgfqpoint{3.649774in}{3.114238in}}%
\pgfpathlineto{\pgfqpoint{3.655126in}{3.121380in}}%
\pgfpathlineto{\pgfqpoint{3.666633in}{3.137442in}}%
\pgfpathlineto{\pgfqpoint{3.670647in}{3.139442in}}%
\pgfpathlineto{\pgfqpoint{3.674126in}{3.138851in}}%
\pgfpathlineto{\pgfqpoint{3.678141in}{3.135611in}}%
\pgfpathlineto{\pgfqpoint{3.684028in}{3.127355in}}%
\pgfpathlineto{\pgfqpoint{3.693929in}{3.113447in}}%
\pgfpathlineto{\pgfqpoint{3.698211in}{3.111023in}}%
\pgfpathlineto{\pgfqpoint{3.701690in}{3.111445in}}%
\pgfpathlineto{\pgfqpoint{3.705437in}{3.114232in}}%
\pgfpathlineto{\pgfqpoint{3.710789in}{3.121372in}}%
\pgfpathlineto{\pgfqpoint{3.722296in}{3.137438in}}%
\pgfpathlineto{\pgfqpoint{3.726310in}{3.139441in}}%
\pgfpathlineto{\pgfqpoint{3.729789in}{3.138853in}}%
\pgfpathlineto{\pgfqpoint{3.733803in}{3.135617in}}%
\pgfpathlineto{\pgfqpoint{3.739690in}{3.127363in}}%
\pgfpathlineto{\pgfqpoint{3.749592in}{3.113452in}}%
\pgfpathlineto{\pgfqpoint{3.753874in}{3.111024in}}%
\pgfpathlineto{\pgfqpoint{3.757352in}{3.111442in}}%
\pgfpathlineto{\pgfqpoint{3.761099in}{3.114227in}}%
\pgfpathlineto{\pgfqpoint{3.766451in}{3.121365in}}%
\pgfpathlineto{\pgfqpoint{3.777958in}{3.137434in}}%
\pgfpathlineto{\pgfqpoint{3.781972in}{3.139440in}}%
\pgfpathlineto{\pgfqpoint{3.785451in}{3.138856in}}%
\pgfpathlineto{\pgfqpoint{3.789465in}{3.135622in}}%
\pgfpathlineto{\pgfqpoint{3.795353in}{3.127371in}}%
\pgfpathlineto{\pgfqpoint{3.805522in}{3.113216in}}%
\pgfpathlineto{\pgfqpoint{3.809804in}{3.110979in}}%
\pgfpathlineto{\pgfqpoint{3.813282in}{3.111562in}}%
\pgfpathlineto{\pgfqpoint{3.817297in}{3.114794in}}%
\pgfpathlineto{\pgfqpoint{3.823184in}{3.123043in}}%
\pgfpathlineto{\pgfqpoint{3.833353in}{3.137200in}}%
\pgfpathlineto{\pgfqpoint{3.837635in}{3.139439in}}%
\pgfpathlineto{\pgfqpoint{3.841114in}{3.138858in}}%
\pgfpathlineto{\pgfqpoint{3.845128in}{3.135628in}}%
\pgfpathlineto{\pgfqpoint{3.851015in}{3.127380in}}%
\pgfpathlineto{\pgfqpoint{3.861184in}{3.113221in}}%
\pgfpathlineto{\pgfqpoint{3.865466in}{3.110980in}}%
\pgfpathlineto{\pgfqpoint{3.868945in}{3.111559in}}%
\pgfpathlineto{\pgfqpoint{3.872959in}{3.114788in}}%
\pgfpathlineto{\pgfqpoint{3.878846in}{3.123035in}}%
\pgfpathlineto{\pgfqpoint{3.889015in}{3.137196in}}%
\pgfpathlineto{\pgfqpoint{3.893297in}{3.139439in}}%
\pgfpathlineto{\pgfqpoint{3.896776in}{3.138861in}}%
\pgfpathlineto{\pgfqpoint{3.900790in}{3.135634in}}%
\pgfpathlineto{\pgfqpoint{3.906678in}{3.127388in}}%
\pgfpathlineto{\pgfqpoint{3.916847in}{3.113225in}}%
\pgfpathlineto{\pgfqpoint{3.921128in}{3.110981in}}%
\pgfpathlineto{\pgfqpoint{3.924607in}{3.111557in}}%
\pgfpathlineto{\pgfqpoint{3.928621in}{3.114783in}}%
\pgfpathlineto{\pgfqpoint{3.934509in}{3.123027in}}%
\pgfpathlineto{\pgfqpoint{3.944678in}{3.137191in}}%
\pgfpathlineto{\pgfqpoint{3.948960in}{3.139438in}}%
\pgfpathlineto{\pgfqpoint{3.952439in}{3.138863in}}%
\pgfpathlineto{\pgfqpoint{3.956453in}{3.135639in}}%
\pgfpathlineto{\pgfqpoint{3.962340in}{3.127396in}}%
\pgfpathlineto{\pgfqpoint{3.972509in}{3.113230in}}%
\pgfpathlineto{\pgfqpoint{3.976791in}{3.110981in}}%
\pgfpathlineto{\pgfqpoint{3.980270in}{3.111555in}}%
\pgfpathlineto{\pgfqpoint{3.984284in}{3.114777in}}%
\pgfpathlineto{\pgfqpoint{3.990171in}{3.123019in}}%
\pgfpathlineto{\pgfqpoint{4.000340in}{3.137187in}}%
\pgfpathlineto{\pgfqpoint{4.004622in}{3.139437in}}%
\pgfpathlineto{\pgfqpoint{4.008101in}{3.138865in}}%
\pgfpathlineto{\pgfqpoint{4.012115in}{3.135645in}}%
\pgfpathlineto{\pgfqpoint{4.018002in}{3.127404in}}%
\pgfpathlineto{\pgfqpoint{4.028172in}{3.113234in}}%
\pgfpathlineto{\pgfqpoint{4.032453in}{3.110982in}}%
\pgfpathlineto{\pgfqpoint{4.035932in}{3.111552in}}%
\pgfpathlineto{\pgfqpoint{4.039946in}{3.114771in}}%
\pgfpathlineto{\pgfqpoint{4.045834in}{3.123011in}}%
\pgfpathlineto{\pgfqpoint{4.056003in}{3.137183in}}%
\pgfpathlineto{\pgfqpoint{4.060284in}{3.139436in}}%
\pgfpathlineto{\pgfqpoint{4.063763in}{3.138868in}}%
\pgfpathlineto{\pgfqpoint{4.067778in}{3.135650in}}%
\pgfpathlineto{\pgfqpoint{4.073665in}{3.127412in}}%
\pgfpathlineto{\pgfqpoint{4.083834in}{3.113239in}}%
\pgfpathlineto{\pgfqpoint{4.088116in}{3.110983in}}%
\pgfpathlineto{\pgfqpoint{4.091595in}{3.111550in}}%
\pgfpathlineto{\pgfqpoint{4.095609in}{3.114766in}}%
\pgfpathlineto{\pgfqpoint{4.101496in}{3.123003in}}%
\pgfpathlineto{\pgfqpoint{4.111665in}{3.137178in}}%
\pgfpathlineto{\pgfqpoint{4.115947in}{3.139436in}}%
\pgfpathlineto{\pgfqpoint{4.119426in}{3.138870in}}%
\pgfpathlineto{\pgfqpoint{4.123440in}{3.135656in}}%
\pgfpathlineto{\pgfqpoint{4.129327in}{3.127420in}}%
\pgfpathlineto{\pgfqpoint{4.139496in}{3.113243in}}%
\pgfpathlineto{\pgfqpoint{4.143778in}{3.110984in}}%
\pgfpathlineto{\pgfqpoint{4.147257in}{3.111547in}}%
\pgfpathlineto{\pgfqpoint{4.151271in}{3.114760in}}%
\pgfpathlineto{\pgfqpoint{4.157159in}{3.122995in}}%
\pgfpathlineto{\pgfqpoint{4.167328in}{3.137174in}}%
\pgfpathlineto{\pgfqpoint{4.171609in}{3.139435in}}%
\pgfpathlineto{\pgfqpoint{4.175088in}{3.138873in}}%
\pgfpathlineto{\pgfqpoint{4.179102in}{3.135661in}}%
\pgfpathlineto{\pgfqpoint{4.184990in}{3.127428in}}%
\pgfpathlineto{\pgfqpoint{4.195159in}{3.113247in}}%
\pgfpathlineto{\pgfqpoint{4.199441in}{3.110984in}}%
\pgfpathlineto{\pgfqpoint{4.202919in}{3.111545in}}%
\pgfpathlineto{\pgfqpoint{4.206934in}{3.114755in}}%
\pgfpathlineto{\pgfqpoint{4.212821in}{3.122987in}}%
\pgfpathlineto{\pgfqpoint{4.222990in}{3.137169in}}%
\pgfpathlineto{\pgfqpoint{4.227272in}{3.139434in}}%
\pgfpathlineto{\pgfqpoint{4.230751in}{3.138875in}}%
\pgfpathlineto{\pgfqpoint{4.234765in}{3.135667in}}%
\pgfpathlineto{\pgfqpoint{4.240652in}{3.127436in}}%
\pgfpathlineto{\pgfqpoint{4.250821in}{3.113252in}}%
\pgfpathlineto{\pgfqpoint{4.255103in}{3.110985in}}%
\pgfpathlineto{\pgfqpoint{4.258582in}{3.111543in}}%
\pgfpathlineto{\pgfqpoint{4.262596in}{3.114749in}}%
\pgfpathlineto{\pgfqpoint{4.268483in}{3.122979in}}%
\pgfpathlineto{\pgfqpoint{4.278652in}{3.137165in}}%
\pgfpathlineto{\pgfqpoint{4.282934in}{3.139433in}}%
\pgfpathlineto{\pgfqpoint{4.286413in}{3.138877in}}%
\pgfpathlineto{\pgfqpoint{4.290427in}{3.135672in}}%
\pgfpathlineto{\pgfqpoint{4.296315in}{3.127444in}}%
\pgfpathlineto{\pgfqpoint{4.306484in}{3.113256in}}%
\pgfpathlineto{\pgfqpoint{4.310765in}{3.110986in}}%
\pgfpathlineto{\pgfqpoint{4.314244in}{3.111540in}}%
\pgfpathlineto{\pgfqpoint{4.318258in}{3.114744in}}%
\pgfpathlineto{\pgfqpoint{4.324146in}{3.122971in}}%
\pgfpathlineto{\pgfqpoint{4.334315in}{3.137160in}}%
\pgfpathlineto{\pgfqpoint{4.338597in}{3.139432in}}%
\pgfpathlineto{\pgfqpoint{4.342076in}{3.138880in}}%
\pgfpathlineto{\pgfqpoint{4.346090in}{3.135678in}}%
\pgfpathlineto{\pgfqpoint{4.351977in}{3.127452in}}%
\pgfpathlineto{\pgfqpoint{4.362146in}{3.113261in}}%
\pgfpathlineto{\pgfqpoint{4.366428in}{3.110987in}}%
\pgfpathlineto{\pgfqpoint{4.369907in}{3.111538in}}%
\pgfpathlineto{\pgfqpoint{4.373921in}{3.114738in}}%
\pgfpathlineto{\pgfqpoint{4.379808in}{3.122963in}}%
\pgfpathlineto{\pgfqpoint{4.389977in}{3.137156in}}%
\pgfpathlineto{\pgfqpoint{4.394259in}{3.139432in}}%
\pgfpathlineto{\pgfqpoint{4.397738in}{3.138882in}}%
\pgfpathlineto{\pgfqpoint{4.401752in}{3.135684in}}%
\pgfpathlineto{\pgfqpoint{4.407639in}{3.127460in}}%
\pgfpathlineto{\pgfqpoint{4.417809in}{3.113265in}}%
\pgfpathlineto{\pgfqpoint{4.422090in}{3.110988in}}%
\pgfpathlineto{\pgfqpoint{4.425569in}{3.111536in}}%
\pgfpathlineto{\pgfqpoint{4.429583in}{3.114733in}}%
\pgfpathlineto{\pgfqpoint{4.435471in}{3.122955in}}%
\pgfpathlineto{\pgfqpoint{4.445640in}{3.137151in}}%
\pgfpathlineto{\pgfqpoint{4.449921in}{3.139431in}}%
\pgfpathlineto{\pgfqpoint{4.453400in}{3.138884in}}%
\pgfpathlineto{\pgfqpoint{4.457414in}{3.135689in}}%
\pgfpathlineto{\pgfqpoint{4.463302in}{3.127468in}}%
\pgfpathlineto{\pgfqpoint{4.473471in}{3.113270in}}%
\pgfpathlineto{\pgfqpoint{4.477753in}{3.110988in}}%
\pgfpathlineto{\pgfqpoint{4.481232in}{3.111533in}}%
\pgfpathlineto{\pgfqpoint{4.485246in}{3.114727in}}%
\pgfpathlineto{\pgfqpoint{4.491133in}{3.122947in}}%
\pgfpathlineto{\pgfqpoint{4.501302in}{3.137147in}}%
\pgfpathlineto{\pgfqpoint{4.505584in}{3.139430in}}%
\pgfpathlineto{\pgfqpoint{4.509063in}{3.138887in}}%
\pgfpathlineto{\pgfqpoint{4.513077in}{3.135695in}}%
\pgfpathlineto{\pgfqpoint{4.518964in}{3.127476in}}%
\pgfpathlineto{\pgfqpoint{4.529133in}{3.113274in}}%
\pgfpathlineto{\pgfqpoint{4.533415in}{3.110989in}}%
\pgfpathlineto{\pgfqpoint{4.536894in}{3.111531in}}%
\pgfpathlineto{\pgfqpoint{4.540908in}{3.114721in}}%
\pgfpathlineto{\pgfqpoint{4.546528in}{3.122514in}}%
\pgfpathlineto{\pgfqpoint{4.556965in}{3.137142in}}%
\pgfpathlineto{\pgfqpoint{4.561246in}{3.139429in}}%
\pgfpathlineto{\pgfqpoint{4.564725in}{3.138889in}}%
\pgfpathlineto{\pgfqpoint{4.568739in}{3.135700in}}%
\pgfpathlineto{\pgfqpoint{4.574359in}{3.127909in}}%
\pgfpathlineto{\pgfqpoint{4.584796in}{3.113279in}}%
\pgfpathlineto{\pgfqpoint{4.589078in}{3.110990in}}%
\pgfpathlineto{\pgfqpoint{4.592556in}{3.111529in}}%
\pgfpathlineto{\pgfqpoint{4.596571in}{3.114716in}}%
\pgfpathlineto{\pgfqpoint{4.602190in}{3.122506in}}%
\pgfpathlineto{\pgfqpoint{4.612627in}{3.137138in}}%
\pgfpathlineto{\pgfqpoint{4.616909in}{3.139428in}}%
\pgfpathlineto{\pgfqpoint{4.620388in}{3.138892in}}%
\pgfpathlineto{\pgfqpoint{4.624402in}{3.135706in}}%
\pgfpathlineto{\pgfqpoint{4.630022in}{3.127917in}}%
\pgfpathlineto{\pgfqpoint{4.640458in}{3.113283in}}%
\pgfpathlineto{\pgfqpoint{4.644740in}{3.110991in}}%
\pgfpathlineto{\pgfqpoint{4.648219in}{3.111526in}}%
\pgfpathlineto{\pgfqpoint{4.652233in}{3.114710in}}%
\pgfpathlineto{\pgfqpoint{4.657853in}{3.122498in}}%
\pgfpathlineto{\pgfqpoint{4.668289in}{3.137133in}}%
\pgfpathlineto{\pgfqpoint{4.672571in}{3.139428in}}%
\pgfpathlineto{\pgfqpoint{4.676050in}{3.138894in}}%
\pgfpathlineto{\pgfqpoint{4.680064in}{3.135711in}}%
\pgfpathlineto{\pgfqpoint{4.685684in}{3.127925in}}%
\pgfpathlineto{\pgfqpoint{4.696121in}{3.113288in}}%
\pgfpathlineto{\pgfqpoint{4.700402in}{3.110992in}}%
\pgfpathlineto{\pgfqpoint{4.703881in}{3.111524in}}%
\pgfpathlineto{\pgfqpoint{4.707895in}{3.114705in}}%
\pgfpathlineto{\pgfqpoint{4.713515in}{3.122490in}}%
\pgfpathlineto{\pgfqpoint{4.723952in}{3.137129in}}%
\pgfpathlineto{\pgfqpoint{4.728234in}{3.139427in}}%
\pgfpathlineto{\pgfqpoint{4.731713in}{3.138896in}}%
\pgfpathlineto{\pgfqpoint{4.735727in}{3.135717in}}%
\pgfpathlineto{\pgfqpoint{4.741346in}{3.127933in}}%
\pgfpathlineto{\pgfqpoint{4.751783in}{3.113292in}}%
\pgfpathlineto{\pgfqpoint{4.756065in}{3.110992in}}%
\pgfpathlineto{\pgfqpoint{4.759544in}{3.111521in}}%
\pgfpathlineto{\pgfqpoint{4.763558in}{3.114699in}}%
\pgfpathlineto{\pgfqpoint{4.769178in}{3.122482in}}%
\pgfpathlineto{\pgfqpoint{4.779614in}{3.137124in}}%
\pgfpathlineto{\pgfqpoint{4.783896in}{3.139426in}}%
\pgfpathlineto{\pgfqpoint{4.787375in}{3.138899in}}%
\pgfpathlineto{\pgfqpoint{4.791389in}{3.135722in}}%
\pgfpathlineto{\pgfqpoint{4.797009in}{3.127941in}}%
\pgfpathlineto{\pgfqpoint{4.807446in}{3.113297in}}%
\pgfpathlineto{\pgfqpoint{4.811727in}{3.110993in}}%
\pgfpathlineto{\pgfqpoint{4.815206in}{3.111519in}}%
\pgfpathlineto{\pgfqpoint{4.819220in}{3.114694in}}%
\pgfpathlineto{\pgfqpoint{4.824840in}{3.122474in}}%
\pgfpathlineto{\pgfqpoint{4.835277in}{3.137120in}}%
\pgfpathlineto{\pgfqpoint{4.839558in}{3.139425in}}%
\pgfpathlineto{\pgfqpoint{4.843037in}{3.138901in}}%
\pgfpathlineto{\pgfqpoint{4.847051in}{3.135728in}}%
\pgfpathlineto{\pgfqpoint{4.852671in}{3.127949in}}%
\pgfpathlineto{\pgfqpoint{4.863108in}{3.113301in}}%
\pgfpathlineto{\pgfqpoint{4.867390in}{3.110994in}}%
\pgfpathlineto{\pgfqpoint{4.870869in}{3.111517in}}%
\pgfpathlineto{\pgfqpoint{4.874883in}{3.114688in}}%
\pgfpathlineto{\pgfqpoint{4.880502in}{3.122466in}}%
\pgfpathlineto{\pgfqpoint{4.890939in}{3.137115in}}%
\pgfpathlineto{\pgfqpoint{4.895221in}{3.139424in}}%
\pgfpathlineto{\pgfqpoint{4.898700in}{3.138903in}}%
\pgfpathlineto{\pgfqpoint{4.902714in}{3.135733in}}%
\pgfpathlineto{\pgfqpoint{4.908334in}{3.127957in}}%
\pgfpathlineto{\pgfqpoint{4.918770in}{3.113306in}}%
\pgfpathlineto{\pgfqpoint{4.923052in}{3.110995in}}%
\pgfpathlineto{\pgfqpoint{4.926531in}{3.111514in}}%
\pgfpathlineto{\pgfqpoint{4.930545in}{3.114683in}}%
\pgfpathlineto{\pgfqpoint{4.936165in}{3.122458in}}%
\pgfpathlineto{\pgfqpoint{4.946602in}{3.137111in}}%
\pgfpathlineto{\pgfqpoint{4.950883in}{3.139423in}}%
\pgfpathlineto{\pgfqpoint{4.954362in}{3.138906in}}%
\pgfpathlineto{\pgfqpoint{4.958376in}{3.135739in}}%
\pgfpathlineto{\pgfqpoint{4.963996in}{3.127965in}}%
\pgfpathlineto{\pgfqpoint{4.974433in}{3.113310in}}%
\pgfpathlineto{\pgfqpoint{4.978715in}{3.110996in}}%
\pgfpathlineto{\pgfqpoint{4.982193in}{3.111512in}}%
\pgfpathlineto{\pgfqpoint{4.986208in}{3.114677in}}%
\pgfpathlineto{\pgfqpoint{4.991827in}{3.122450in}}%
\pgfpathlineto{\pgfqpoint{5.002264in}{3.137106in}}%
\pgfpathlineto{\pgfqpoint{5.006546in}{3.139423in}}%
\pgfpathlineto{\pgfqpoint{5.010025in}{3.138908in}}%
\pgfpathlineto{\pgfqpoint{5.014039in}{3.135744in}}%
\pgfpathlineto{\pgfqpoint{5.019659in}{3.127973in}}%
\pgfpathlineto{\pgfqpoint{5.030095in}{3.113315in}}%
\pgfpathlineto{\pgfqpoint{5.034377in}{3.110997in}}%
\pgfpathlineto{\pgfqpoint{5.037856in}{3.111510in}}%
\pgfpathlineto{\pgfqpoint{5.041870in}{3.114672in}}%
\pgfpathlineto{\pgfqpoint{5.047490in}{3.122442in}}%
\pgfpathlineto{\pgfqpoint{5.057926in}{3.137102in}}%
\pgfpathlineto{\pgfqpoint{5.062208in}{3.139422in}}%
\pgfpathlineto{\pgfqpoint{5.065687in}{3.138910in}}%
\pgfpathlineto{\pgfqpoint{5.069701in}{3.135750in}}%
\pgfpathlineto{\pgfqpoint{5.075321in}{3.127981in}}%
\pgfpathlineto{\pgfqpoint{5.085758in}{3.113319in}}%
\pgfpathlineto{\pgfqpoint{5.090039in}{3.110997in}}%
\pgfpathlineto{\pgfqpoint{5.093518in}{3.111508in}}%
\pgfpathlineto{\pgfqpoint{5.097532in}{3.114666in}}%
\pgfpathlineto{\pgfqpoint{5.103152in}{3.122434in}}%
\pgfpathlineto{\pgfqpoint{5.113589in}{3.137097in}}%
\pgfpathlineto{\pgfqpoint{5.117871in}{3.139421in}}%
\pgfpathlineto{\pgfqpoint{5.121349in}{3.138912in}}%
\pgfpathlineto{\pgfqpoint{5.125364in}{3.135755in}}%
\pgfpathlineto{\pgfqpoint{5.130983in}{3.127989in}}%
\pgfpathlineto{\pgfqpoint{5.141420in}{3.113324in}}%
\pgfpathlineto{\pgfqpoint{5.145702in}{3.110998in}}%
\pgfpathlineto{\pgfqpoint{5.149181in}{3.111505in}}%
\pgfpathlineto{\pgfqpoint{5.153195in}{3.114661in}}%
\pgfpathlineto{\pgfqpoint{5.158815in}{3.122426in}}%
\pgfpathlineto{\pgfqpoint{5.169519in}{3.137327in}}%
\pgfpathlineto{\pgfqpoint{5.173801in}{3.139459in}}%
\pgfpathlineto{\pgfqpoint{5.177280in}{3.138786in}}%
\pgfpathlineto{\pgfqpoint{5.181294in}{3.135465in}}%
\pgfpathlineto{\pgfqpoint{5.187181in}{3.127146in}}%
\pgfpathlineto{\pgfqpoint{5.197083in}{3.113328in}}%
\pgfpathlineto{\pgfqpoint{5.201364in}{3.110999in}}%
\pgfpathlineto{\pgfqpoint{5.204843in}{3.111503in}}%
\pgfpathlineto{\pgfqpoint{5.208857in}{3.114655in}}%
\pgfpathlineto{\pgfqpoint{5.214477in}{3.122418in}}%
\pgfpathlineto{\pgfqpoint{5.225181in}{3.137323in}}%
\pgfpathlineto{\pgfqpoint{5.229463in}{3.139458in}}%
\pgfpathlineto{\pgfqpoint{5.232942in}{3.138789in}}%
\pgfpathlineto{\pgfqpoint{5.236956in}{3.135471in}}%
\pgfpathlineto{\pgfqpoint{5.242843in}{3.127154in}}%
\pgfpathlineto{\pgfqpoint{5.252745in}{3.113333in}}%
\pgfpathlineto{\pgfqpoint{5.257027in}{3.111000in}}%
\pgfpathlineto{\pgfqpoint{5.260506in}{3.111501in}}%
\pgfpathlineto{\pgfqpoint{5.264520in}{3.114650in}}%
\pgfpathlineto{\pgfqpoint{5.270139in}{3.122410in}}%
\pgfpathlineto{\pgfqpoint{5.280844in}{3.137319in}}%
\pgfpathlineto{\pgfqpoint{5.285125in}{3.139458in}}%
\pgfpathlineto{\pgfqpoint{5.288604in}{3.138791in}}%
\pgfpathlineto{\pgfqpoint{5.292618in}{3.135476in}}%
\pgfpathlineto{\pgfqpoint{5.298506in}{3.127162in}}%
\pgfpathlineto{\pgfqpoint{5.308407in}{3.113337in}}%
\pgfpathlineto{\pgfqpoint{5.312689in}{3.111001in}}%
\pgfpathlineto{\pgfqpoint{5.316168in}{3.111498in}}%
\pgfpathlineto{\pgfqpoint{5.320182in}{3.114644in}}%
\pgfpathlineto{\pgfqpoint{5.325802in}{3.122402in}}%
\pgfpathlineto{\pgfqpoint{5.336506in}{3.137314in}}%
\pgfpathlineto{\pgfqpoint{5.340788in}{3.139457in}}%
\pgfpathlineto{\pgfqpoint{5.344267in}{3.138794in}}%
\pgfpathlineto{\pgfqpoint{5.348281in}{3.135482in}}%
\pgfpathlineto{\pgfqpoint{5.354168in}{3.127170in}}%
\pgfpathlineto{\pgfqpoint{5.364070in}{3.113342in}}%
\pgfpathlineto{\pgfqpoint{5.368352in}{3.111002in}}%
\pgfpathlineto{\pgfqpoint{5.371830in}{3.111496in}}%
\pgfpathlineto{\pgfqpoint{5.375845in}{3.114639in}}%
\pgfpathlineto{\pgfqpoint{5.381464in}{3.122394in}}%
\pgfpathlineto{\pgfqpoint{5.392169in}{3.137310in}}%
\pgfpathlineto{\pgfqpoint{5.396450in}{3.139457in}}%
\pgfpathlineto{\pgfqpoint{5.399929in}{3.138796in}}%
\pgfpathlineto{\pgfqpoint{5.403943in}{3.135488in}}%
\pgfpathlineto{\pgfqpoint{5.409831in}{3.127178in}}%
\pgfpathlineto{\pgfqpoint{5.419732in}{3.113346in}}%
\pgfpathlineto{\pgfqpoint{5.424014in}{3.111003in}}%
\pgfpathlineto{\pgfqpoint{5.427493in}{3.111494in}}%
\pgfpathlineto{\pgfqpoint{5.431507in}{3.114633in}}%
\pgfpathlineto{\pgfqpoint{5.437127in}{3.122386in}}%
\pgfpathlineto{\pgfqpoint{5.447831in}{3.137306in}}%
\pgfpathlineto{\pgfqpoint{5.452113in}{3.139456in}}%
\pgfpathlineto{\pgfqpoint{5.455592in}{3.138799in}}%
\pgfpathlineto{\pgfqpoint{5.459606in}{3.135493in}}%
\pgfpathlineto{\pgfqpoint{5.465493in}{3.127186in}}%
\pgfpathlineto{\pgfqpoint{5.475395in}{3.113351in}}%
\pgfpathlineto{\pgfqpoint{5.479676in}{3.111004in}}%
\pgfpathlineto{\pgfqpoint{5.483155in}{3.111491in}}%
\pgfpathlineto{\pgfqpoint{5.487169in}{3.114628in}}%
\pgfpathlineto{\pgfqpoint{5.492789in}{3.122378in}}%
\pgfpathlineto{\pgfqpoint{5.503493in}{3.137301in}}%
\pgfpathlineto{\pgfqpoint{5.507775in}{3.139455in}}%
\pgfpathlineto{\pgfqpoint{5.511254in}{3.138802in}}%
\pgfpathlineto{\pgfqpoint{5.515268in}{3.135499in}}%
\pgfpathlineto{\pgfqpoint{5.521156in}{3.127194in}}%
\pgfpathlineto{\pgfqpoint{5.531057in}{3.113355in}}%
\pgfpathlineto{\pgfqpoint{5.535339in}{3.111004in}}%
\pgfpathlineto{\pgfqpoint{5.538818in}{3.111489in}}%
\pgfpathlineto{\pgfqpoint{5.542832in}{3.114622in}}%
\pgfpathlineto{\pgfqpoint{5.548452in}{3.122370in}}%
\pgfpathlineto{\pgfqpoint{5.559156in}{3.137297in}}%
\pgfpathlineto{\pgfqpoint{5.563438in}{3.139455in}}%
\pgfpathlineto{\pgfqpoint{5.566917in}{3.138804in}}%
\pgfpathlineto{\pgfqpoint{5.570931in}{3.135505in}}%
\pgfpathlineto{\pgfqpoint{5.576818in}{3.127202in}}%
\pgfpathlineto{\pgfqpoint{5.586719in}{3.113360in}}%
\pgfpathlineto{\pgfqpoint{5.591001in}{3.111005in}}%
\pgfpathlineto{\pgfqpoint{5.594480in}{3.111487in}}%
\pgfpathlineto{\pgfqpoint{5.598227in}{3.114332in}}%
\pgfpathlineto{\pgfqpoint{5.603579in}{3.121522in}}%
\pgfpathlineto{\pgfqpoint{5.615086in}{3.137517in}}%
\pgfpathlineto{\pgfqpoint{5.619100in}{3.139454in}}%
\pgfpathlineto{\pgfqpoint{5.622579in}{3.138807in}}%
\pgfpathlineto{\pgfqpoint{5.626593in}{3.135510in}}%
\pgfpathlineto{\pgfqpoint{5.632480in}{3.127210in}}%
\pgfpathlineto{\pgfqpoint{5.642382in}{3.113365in}}%
\pgfpathlineto{\pgfqpoint{5.646664in}{3.111006in}}%
\pgfpathlineto{\pgfqpoint{5.650143in}{3.111485in}}%
\pgfpathlineto{\pgfqpoint{5.653889in}{3.114327in}}%
\pgfpathlineto{\pgfqpoint{5.659241in}{3.121514in}}%
\pgfpathlineto{\pgfqpoint{5.670748in}{3.137513in}}%
\pgfpathlineto{\pgfqpoint{5.674762in}{3.139453in}}%
\pgfpathlineto{\pgfqpoint{5.678241in}{3.138809in}}%
\pgfpathlineto{\pgfqpoint{5.682255in}{3.135516in}}%
\pgfpathlineto{\pgfqpoint{5.688143in}{3.127218in}}%
\pgfpathlineto{\pgfqpoint{5.698044in}{3.113369in}}%
\pgfpathlineto{\pgfqpoint{5.702326in}{3.111007in}}%
\pgfpathlineto{\pgfqpoint{5.705805in}{3.111482in}}%
\pgfpathlineto{\pgfqpoint{5.709551in}{3.114322in}}%
\pgfpathlineto{\pgfqpoint{5.714904in}{3.121506in}}%
\pgfpathlineto{\pgfqpoint{5.726411in}{3.137509in}}%
\pgfpathlineto{\pgfqpoint{5.730425in}{3.139453in}}%
\pgfpathlineto{\pgfqpoint{5.733904in}{3.138812in}}%
\pgfpathlineto{\pgfqpoint{5.737918in}{3.135521in}}%
\pgfpathlineto{\pgfqpoint{5.743805in}{3.127226in}}%
\pgfpathlineto{\pgfqpoint{5.753707in}{3.113374in}}%
\pgfpathlineto{\pgfqpoint{5.757988in}{3.111008in}}%
\pgfpathlineto{\pgfqpoint{5.761467in}{3.111480in}}%
\pgfpathlineto{\pgfqpoint{5.765214in}{3.114316in}}%
\pgfpathlineto{\pgfqpoint{5.770566in}{3.121498in}}%
\pgfpathlineto{\pgfqpoint{5.782073in}{3.137505in}}%
\pgfpathlineto{\pgfqpoint{5.786087in}{3.139452in}}%
\pgfpathlineto{\pgfqpoint{5.789566in}{3.138814in}}%
\pgfpathlineto{\pgfqpoint{5.793580in}{3.135527in}}%
\pgfpathlineto{\pgfqpoint{5.799468in}{3.127235in}}%
\pgfpathlineto{\pgfqpoint{5.809369in}{3.113378in}}%
\pgfpathlineto{\pgfqpoint{5.813651in}{3.111009in}}%
\pgfpathlineto{\pgfqpoint{5.817130in}{3.111478in}}%
\pgfpathlineto{\pgfqpoint{5.820876in}{3.114311in}}%
\pgfpathlineto{\pgfqpoint{5.826228in}{3.121490in}}%
\pgfpathlineto{\pgfqpoint{5.837736in}{3.137501in}}%
\pgfpathlineto{\pgfqpoint{5.841750in}{3.139451in}}%
\pgfpathlineto{\pgfqpoint{5.845229in}{3.138817in}}%
\pgfpathlineto{\pgfqpoint{5.849243in}{3.135533in}}%
\pgfpathlineto{\pgfqpoint{5.855130in}{3.127243in}}%
\pgfpathlineto{\pgfqpoint{5.865032in}{3.113383in}}%
\pgfpathlineto{\pgfqpoint{5.869313in}{3.111010in}}%
\pgfpathlineto{\pgfqpoint{5.872792in}{3.111476in}}%
\pgfpathlineto{\pgfqpoint{5.876539in}{3.114306in}}%
\pgfpathlineto{\pgfqpoint{5.881891in}{3.121482in}}%
\pgfpathlineto{\pgfqpoint{5.893398in}{3.137497in}}%
\pgfpathlineto{\pgfqpoint{5.897412in}{3.139451in}}%
\pgfpathlineto{\pgfqpoint{5.900891in}{3.138819in}}%
\pgfpathlineto{\pgfqpoint{5.904905in}{3.135538in}}%
\pgfpathlineto{\pgfqpoint{5.910793in}{3.127251in}}%
\pgfpathlineto{\pgfqpoint{5.920694in}{3.113387in}}%
\pgfpathlineto{\pgfqpoint{5.924976in}{3.111011in}}%
\pgfpathlineto{\pgfqpoint{5.928455in}{3.111473in}}%
\pgfpathlineto{\pgfqpoint{5.932201in}{3.114301in}}%
\pgfpathlineto{\pgfqpoint{5.937553in}{3.121474in}}%
\pgfpathlineto{\pgfqpoint{5.949060in}{3.137493in}}%
\pgfpathlineto{\pgfqpoint{5.953075in}{3.139450in}}%
\pgfpathlineto{\pgfqpoint{5.956554in}{3.138821in}}%
\pgfpathlineto{\pgfqpoint{5.960568in}{3.135544in}}%
\pgfpathlineto{\pgfqpoint{5.966455in}{3.127259in}}%
\pgfpathlineto{\pgfqpoint{5.976356in}{3.113392in}}%
\pgfpathlineto{\pgfqpoint{5.980638in}{3.111012in}}%
\pgfpathlineto{\pgfqpoint{5.984117in}{3.111471in}}%
\pgfpathlineto{\pgfqpoint{5.987864in}{3.114295in}}%
\pgfpathlineto{\pgfqpoint{5.993216in}{3.121467in}}%
\pgfpathlineto{\pgfqpoint{6.004723in}{3.137488in}}%
\pgfpathlineto{\pgfqpoint{6.008737in}{3.139449in}}%
\pgfpathlineto{\pgfqpoint{6.012216in}{3.138824in}}%
\pgfpathlineto{\pgfqpoint{6.016230in}{3.135550in}}%
\pgfpathlineto{\pgfqpoint{6.022117in}{3.127267in}}%
\pgfpathlineto{\pgfqpoint{6.032019in}{3.113397in}}%
\pgfpathlineto{\pgfqpoint{6.036301in}{3.111013in}}%
\pgfpathlineto{\pgfqpoint{6.039780in}{3.111469in}}%
\pgfpathlineto{\pgfqpoint{6.043526in}{3.114290in}}%
\pgfpathlineto{\pgfqpoint{6.048878in}{3.121459in}}%
\pgfpathlineto{\pgfqpoint{6.060385in}{3.137484in}}%
\pgfpathlineto{\pgfqpoint{6.064399in}{3.139449in}}%
\pgfpathlineto{\pgfqpoint{6.067878in}{3.138826in}}%
\pgfpathlineto{\pgfqpoint{6.071892in}{3.135555in}}%
\pgfpathlineto{\pgfqpoint{6.077780in}{3.127275in}}%
\pgfpathlineto{\pgfqpoint{6.087681in}{3.113401in}}%
\pgfpathlineto{\pgfqpoint{6.091963in}{3.111014in}}%
\pgfpathlineto{\pgfqpoint{6.095442in}{3.111467in}}%
\pgfpathlineto{\pgfqpoint{6.099188in}{3.114285in}}%
\pgfpathlineto{\pgfqpoint{6.104541in}{3.121451in}}%
\pgfpathlineto{\pgfqpoint{6.116048in}{3.137480in}}%
\pgfpathlineto{\pgfqpoint{6.120062in}{3.139448in}}%
\pgfpathlineto{\pgfqpoint{6.123541in}{3.138829in}}%
\pgfpathlineto{\pgfqpoint{6.127555in}{3.135561in}}%
\pgfpathlineto{\pgfqpoint{6.133442in}{3.127283in}}%
\pgfpathlineto{\pgfqpoint{6.143344in}{3.113406in}}%
\pgfpathlineto{\pgfqpoint{6.147625in}{3.111014in}}%
\pgfpathlineto{\pgfqpoint{6.151104in}{3.111464in}}%
\pgfpathlineto{\pgfqpoint{6.154851in}{3.114280in}}%
\pgfpathlineto{\pgfqpoint{6.160203in}{3.121443in}}%
\pgfpathlineto{\pgfqpoint{6.171710in}{3.137476in}}%
\pgfpathlineto{\pgfqpoint{6.175724in}{3.139447in}}%
\pgfpathlineto{\pgfqpoint{6.179203in}{3.138831in}}%
\pgfpathlineto{\pgfqpoint{6.183217in}{3.135566in}}%
\pgfpathlineto{\pgfqpoint{6.189105in}{3.127291in}}%
\pgfpathlineto{\pgfqpoint{6.199006in}{3.113410in}}%
\pgfpathlineto{\pgfqpoint{6.203288in}{3.111015in}}%
\pgfpathlineto{\pgfqpoint{6.206767in}{3.111462in}}%
\pgfpathlineto{\pgfqpoint{6.210513in}{3.114274in}}%
\pgfpathlineto{\pgfqpoint{6.215865in}{3.121435in}}%
\pgfpathlineto{\pgfqpoint{6.227373in}{3.137472in}}%
\pgfpathlineto{\pgfqpoint{6.231387in}{3.139447in}}%
\pgfpathlineto{\pgfqpoint{6.234866in}{3.138834in}}%
\pgfpathlineto{\pgfqpoint{6.238880in}{3.135572in}}%
\pgfpathlineto{\pgfqpoint{6.244767in}{3.127299in}}%
\pgfpathlineto{\pgfqpoint{6.254669in}{3.113415in}}%
\pgfpathlineto{\pgfqpoint{6.258950in}{3.111016in}}%
\pgfpathlineto{\pgfqpoint{6.262429in}{3.111460in}}%
\pgfpathlineto{\pgfqpoint{6.266176in}{3.114269in}}%
\pgfpathlineto{\pgfqpoint{6.271528in}{3.121427in}}%
\pgfpathlineto{\pgfqpoint{6.283035in}{3.137467in}}%
\pgfpathlineto{\pgfqpoint{6.287049in}{3.139446in}}%
\pgfpathlineto{\pgfqpoint{6.290528in}{3.138836in}}%
\pgfpathlineto{\pgfqpoint{6.294542in}{3.135578in}}%
\pgfpathlineto{\pgfqpoint{6.300430in}{3.127307in}}%
\pgfpathlineto{\pgfqpoint{6.310331in}{3.113420in}}%
\pgfpathlineto{\pgfqpoint{6.314613in}{3.111017in}}%
\pgfpathlineto{\pgfqpoint{6.318092in}{3.111458in}}%
\pgfpathlineto{\pgfqpoint{6.321838in}{3.114264in}}%
\pgfpathlineto{\pgfqpoint{6.327190in}{3.121419in}}%
\pgfpathlineto{\pgfqpoint{6.338697in}{3.137463in}}%
\pgfpathlineto{\pgfqpoint{6.342712in}{3.139445in}}%
\pgfpathlineto{\pgfqpoint{6.346190in}{3.138839in}}%
\pgfpathlineto{\pgfqpoint{6.350205in}{3.135583in}}%
\pgfpathlineto{\pgfqpoint{6.356092in}{3.127315in}}%
\pgfpathlineto{\pgfqpoint{6.365993in}{3.113424in}}%
\pgfpathlineto{\pgfqpoint{6.370275in}{3.111018in}}%
\pgfpathlineto{\pgfqpoint{6.373754in}{3.111456in}}%
\pgfpathlineto{\pgfqpoint{6.377501in}{3.114259in}}%
\pgfpathlineto{\pgfqpoint{6.382853in}{3.121412in}}%
\pgfpathlineto{\pgfqpoint{6.394360in}{3.137459in}}%
\pgfpathlineto{\pgfqpoint{6.398374in}{3.139444in}}%
\pgfpathlineto{\pgfqpoint{6.401853in}{3.138841in}}%
\pgfpathlineto{\pgfqpoint{6.405867in}{3.135589in}}%
\pgfpathlineto{\pgfqpoint{6.411754in}{3.127323in}}%
\pgfpathlineto{\pgfqpoint{6.421656in}{3.113429in}}%
\pgfpathlineto{\pgfqpoint{6.425938in}{3.111019in}}%
\pgfpathlineto{\pgfqpoint{6.429417in}{3.111453in}}%
\pgfpathlineto{\pgfqpoint{6.433163in}{3.114253in}}%
\pgfpathlineto{\pgfqpoint{6.438515in}{3.121404in}}%
\pgfpathlineto{\pgfqpoint{6.450022in}{3.137455in}}%
\pgfpathlineto{\pgfqpoint{6.454036in}{3.139444in}}%
\pgfpathlineto{\pgfqpoint{6.457515in}{3.138844in}}%
\pgfpathlineto{\pgfqpoint{6.461529in}{3.135594in}}%
\pgfpathlineto{\pgfqpoint{6.467417in}{3.127331in}}%
\pgfpathlineto{\pgfqpoint{6.477318in}{3.113433in}}%
\pgfpathlineto{\pgfqpoint{6.481600in}{3.111020in}}%
\pgfpathlineto{\pgfqpoint{6.485079in}{3.111451in}}%
\pgfpathlineto{\pgfqpoint{6.488825in}{3.114248in}}%
\pgfpathlineto{\pgfqpoint{6.494178in}{3.121396in}}%
\pgfpathlineto{\pgfqpoint{6.505685in}{3.137451in}}%
\pgfpathlineto{\pgfqpoint{6.509699in}{3.139443in}}%
\pgfpathlineto{\pgfqpoint{6.513178in}{3.138846in}}%
\pgfpathlineto{\pgfqpoint{6.517192in}{3.135600in}}%
\pgfpathlineto{\pgfqpoint{6.523079in}{3.127339in}}%
\pgfpathlineto{\pgfqpoint{6.532981in}{3.113438in}}%
\pgfpathlineto{\pgfqpoint{6.537262in}{3.111021in}}%
\pgfpathlineto{\pgfqpoint{6.540741in}{3.111449in}}%
\pgfpathlineto{\pgfqpoint{6.544488in}{3.114243in}}%
\pgfpathlineto{\pgfqpoint{6.549840in}{3.121388in}}%
\pgfpathlineto{\pgfqpoint{6.561347in}{3.137446in}}%
\pgfpathlineto{\pgfqpoint{6.565361in}{3.139442in}}%
\pgfpathlineto{\pgfqpoint{6.568840in}{3.138848in}}%
\pgfpathlineto{\pgfqpoint{6.572854in}{3.135606in}}%
\pgfpathlineto{\pgfqpoint{6.578742in}{3.127347in}}%
\pgfpathlineto{\pgfqpoint{6.588643in}{3.113443in}}%
\pgfpathlineto{\pgfqpoint{6.592925in}{3.111022in}}%
\pgfpathlineto{\pgfqpoint{6.596404in}{3.111447in}}%
\pgfpathlineto{\pgfqpoint{6.600150in}{3.114238in}}%
\pgfpathlineto{\pgfqpoint{6.605502in}{3.121380in}}%
\pgfpathlineto{\pgfqpoint{6.617010in}{3.137442in}}%
\pgfpathlineto{\pgfqpoint{6.621024in}{3.139442in}}%
\pgfpathlineto{\pgfqpoint{6.624503in}{3.138851in}}%
\pgfpathlineto{\pgfqpoint{6.628517in}{3.135611in}}%
\pgfpathlineto{\pgfqpoint{6.634404in}{3.127355in}}%
\pgfpathlineto{\pgfqpoint{6.644306in}{3.113447in}}%
\pgfpathlineto{\pgfqpoint{6.648587in}{3.111023in}}%
\pgfpathlineto{\pgfqpoint{6.652066in}{3.111445in}}%
\pgfpathlineto{\pgfqpoint{6.655813in}{3.114232in}}%
\pgfpathlineto{\pgfqpoint{6.661165in}{3.121372in}}%
\pgfpathlineto{\pgfqpoint{6.663306in}{3.124778in}}%
\pgfpathlineto{\pgfqpoint{6.663306in}{3.124778in}}%
\pgfusepath{stroke}%
\end{pgfscope}%
\begin{pgfscope}%
\pgfpathrectangle{\pgfqpoint{0.467797in}{2.292089in}}{\pgfqpoint{6.490533in}{1.666241in}}%
\pgfusepath{clip}%
\pgfsetrectcap%
\pgfsetroundjoin%
\pgfsetlinewidth{1.505625pt}%
\definecolor{currentstroke}{rgb}{0.498039,0.498039,0.498039}%
\pgfsetstrokecolor{currentstroke}%
\pgfsetdash{}{0pt}%
\pgfpathmoveto{\pgfqpoint{0.762821in}{3.125209in}}%
\pgfpathlineto{\pgfqpoint{0.771117in}{3.136582in}}%
\pgfpathlineto{\pgfqpoint{0.775398in}{3.138914in}}%
\pgfpathlineto{\pgfqpoint{0.778877in}{3.138324in}}%
\pgfpathlineto{\pgfqpoint{0.782624in}{3.135297in}}%
\pgfpathlineto{\pgfqpoint{0.788244in}{3.127456in}}%
\pgfpathlineto{\pgfqpoint{0.798145in}{3.113652in}}%
\pgfpathlineto{\pgfqpoint{0.802159in}{3.111514in}}%
\pgfpathlineto{\pgfqpoint{0.805638in}{3.112065in}}%
\pgfpathlineto{\pgfqpoint{0.809385in}{3.115055in}}%
\pgfpathlineto{\pgfqpoint{0.815004in}{3.122867in}}%
\pgfpathlineto{\pgfqpoint{0.824906in}{3.136713in}}%
\pgfpathlineto{\pgfqpoint{0.828920in}{3.138894in}}%
\pgfpathlineto{\pgfqpoint{0.832399in}{3.138383in}}%
\pgfpathlineto{\pgfqpoint{0.836145in}{3.135430in}}%
\pgfpathlineto{\pgfqpoint{0.841765in}{3.127649in}}%
\pgfpathlineto{\pgfqpoint{0.851934in}{3.113526in}}%
\pgfpathlineto{\pgfqpoint{0.855948in}{3.111492in}}%
\pgfpathlineto{\pgfqpoint{0.859427in}{3.112137in}}%
\pgfpathlineto{\pgfqpoint{0.863441in}{3.115517in}}%
\pgfpathlineto{\pgfqpoint{0.869329in}{3.123956in}}%
\pgfpathlineto{\pgfqpoint{0.878427in}{3.136604in}}%
\pgfpathlineto{\pgfqpoint{0.882709in}{3.138918in}}%
\pgfpathlineto{\pgfqpoint{0.886188in}{3.138312in}}%
\pgfpathlineto{\pgfqpoint{0.889935in}{3.135271in}}%
\pgfpathlineto{\pgfqpoint{0.895554in}{3.127417in}}%
\pgfpathlineto{\pgfqpoint{0.905456in}{3.113631in}}%
\pgfpathlineto{\pgfqpoint{0.909470in}{3.111510in}}%
\pgfpathlineto{\pgfqpoint{0.912949in}{3.112077in}}%
\pgfpathlineto{\pgfqpoint{0.916695in}{3.115082in}}%
\pgfpathlineto{\pgfqpoint{0.922315in}{3.122905in}}%
\pgfpathlineto{\pgfqpoint{0.932217in}{3.136734in}}%
\pgfpathlineto{\pgfqpoint{0.936231in}{3.138898in}}%
\pgfpathlineto{\pgfqpoint{0.939710in}{3.138371in}}%
\pgfpathlineto{\pgfqpoint{0.943456in}{3.135403in}}%
\pgfpathlineto{\pgfqpoint{0.949076in}{3.127610in}}%
\pgfpathlineto{\pgfqpoint{0.959245in}{3.113505in}}%
\pgfpathlineto{\pgfqpoint{0.963259in}{3.111488in}}%
\pgfpathlineto{\pgfqpoint{0.966738in}{3.112150in}}%
\pgfpathlineto{\pgfqpoint{0.970752in}{3.115545in}}%
\pgfpathlineto{\pgfqpoint{0.976907in}{3.124425in}}%
\pgfpathlineto{\pgfqpoint{0.985738in}{3.136626in}}%
\pgfpathlineto{\pgfqpoint{0.990020in}{3.138922in}}%
\pgfpathlineto{\pgfqpoint{0.993499in}{3.138300in}}%
\pgfpathlineto{\pgfqpoint{0.997513in}{3.134943in}}%
\pgfpathlineto{\pgfqpoint{1.003400in}{3.126522in}}%
\pgfpathlineto{\pgfqpoint{1.012499in}{3.113848in}}%
\pgfpathlineto{\pgfqpoint{1.016781in}{3.111506in}}%
\pgfpathlineto{\pgfqpoint{1.020260in}{3.112089in}}%
\pgfpathlineto{\pgfqpoint{1.024006in}{3.115108in}}%
\pgfpathlineto{\pgfqpoint{1.029626in}{3.122944in}}%
\pgfpathlineto{\pgfqpoint{1.039527in}{3.136756in}}%
\pgfpathlineto{\pgfqpoint{1.043542in}{3.138903in}}%
\pgfpathlineto{\pgfqpoint{1.047020in}{3.138360in}}%
\pgfpathlineto{\pgfqpoint{1.050767in}{3.135377in}}%
\pgfpathlineto{\pgfqpoint{1.056387in}{3.127572in}}%
\pgfpathlineto{\pgfqpoint{1.066288in}{3.113717in}}%
\pgfpathlineto{\pgfqpoint{1.070302in}{3.111527in}}%
\pgfpathlineto{\pgfqpoint{1.073781in}{3.112031in}}%
\pgfpathlineto{\pgfqpoint{1.077528in}{3.114976in}}%
\pgfpathlineto{\pgfqpoint{1.083147in}{3.122751in}}%
\pgfpathlineto{\pgfqpoint{1.093317in}{3.136882in}}%
\pgfpathlineto{\pgfqpoint{1.097331in}{3.138925in}}%
\pgfpathlineto{\pgfqpoint{1.100810in}{3.138288in}}%
\pgfpathlineto{\pgfqpoint{1.104824in}{3.134915in}}%
\pgfpathlineto{\pgfqpoint{1.110711in}{3.126483in}}%
\pgfpathlineto{\pgfqpoint{1.119810in}{3.113826in}}%
\pgfpathlineto{\pgfqpoint{1.124091in}{3.111502in}}%
\pgfpathlineto{\pgfqpoint{1.127570in}{3.112101in}}%
\pgfpathlineto{\pgfqpoint{1.131317in}{3.115135in}}%
\pgfpathlineto{\pgfqpoint{1.136937in}{3.122983in}}%
\pgfpathlineto{\pgfqpoint{1.146838in}{3.136777in}}%
\pgfpathlineto{\pgfqpoint{1.150852in}{3.138907in}}%
\pgfpathlineto{\pgfqpoint{1.154331in}{3.138348in}}%
\pgfpathlineto{\pgfqpoint{1.158078in}{3.135351in}}%
\pgfpathlineto{\pgfqpoint{1.163697in}{3.127533in}}%
\pgfpathlineto{\pgfqpoint{1.173599in}{3.113695in}}%
\pgfpathlineto{\pgfqpoint{1.177613in}{3.111523in}}%
\pgfpathlineto{\pgfqpoint{1.181092in}{3.112042in}}%
\pgfpathlineto{\pgfqpoint{1.184838in}{3.115002in}}%
\pgfpathlineto{\pgfqpoint{1.190458in}{3.122789in}}%
\pgfpathlineto{\pgfqpoint{1.200627in}{3.136903in}}%
\pgfpathlineto{\pgfqpoint{1.204641in}{3.138929in}}%
\pgfpathlineto{\pgfqpoint{1.208120in}{3.138275in}}%
\pgfpathlineto{\pgfqpoint{1.212134in}{3.134888in}}%
\pgfpathlineto{\pgfqpoint{1.218289in}{3.126013in}}%
\pgfpathlineto{\pgfqpoint{1.227120in}{3.113804in}}%
\pgfpathlineto{\pgfqpoint{1.231402in}{3.111499in}}%
\pgfpathlineto{\pgfqpoint{1.234881in}{3.112113in}}%
\pgfpathlineto{\pgfqpoint{1.238895in}{3.115462in}}%
\pgfpathlineto{\pgfqpoint{1.244783in}{3.123877in}}%
\pgfpathlineto{\pgfqpoint{1.253881in}{3.136560in}}%
\pgfpathlineto{\pgfqpoint{1.258163in}{3.138911in}}%
\pgfpathlineto{\pgfqpoint{1.261642in}{3.138336in}}%
\pgfpathlineto{\pgfqpoint{1.265388in}{3.135324in}}%
\pgfpathlineto{\pgfqpoint{1.271008in}{3.127494in}}%
\pgfpathlineto{\pgfqpoint{1.280910in}{3.113674in}}%
\pgfpathlineto{\pgfqpoint{1.284924in}{3.111518in}}%
\pgfpathlineto{\pgfqpoint{1.288403in}{3.112054in}}%
\pgfpathlineto{\pgfqpoint{1.292149in}{3.115029in}}%
\pgfpathlineto{\pgfqpoint{1.297769in}{3.122828in}}%
\pgfpathlineto{\pgfqpoint{1.307670in}{3.136691in}}%
\pgfpathlineto{\pgfqpoint{1.311685in}{3.138890in}}%
\pgfpathlineto{\pgfqpoint{1.315163in}{3.138394in}}%
\pgfpathlineto{\pgfqpoint{1.318910in}{3.135456in}}%
\pgfpathlineto{\pgfqpoint{1.324530in}{3.127688in}}%
\pgfpathlineto{\pgfqpoint{1.334699in}{3.113547in}}%
\pgfpathlineto{\pgfqpoint{1.338713in}{3.111495in}}%
\pgfpathlineto{\pgfqpoint{1.342192in}{3.112125in}}%
\pgfpathlineto{\pgfqpoint{1.346206in}{3.115490in}}%
\pgfpathlineto{\pgfqpoint{1.352093in}{3.123916in}}%
\pgfpathlineto{\pgfqpoint{1.361192in}{3.136582in}}%
\pgfpathlineto{\pgfqpoint{1.365474in}{3.138914in}}%
\pgfpathlineto{\pgfqpoint{1.368953in}{3.138324in}}%
\pgfpathlineto{\pgfqpoint{1.372699in}{3.135297in}}%
\pgfpathlineto{\pgfqpoint{1.378319in}{3.127456in}}%
\pgfpathlineto{\pgfqpoint{1.388220in}{3.113652in}}%
\pgfpathlineto{\pgfqpoint{1.392235in}{3.111514in}}%
\pgfpathlineto{\pgfqpoint{1.395713in}{3.112065in}}%
\pgfpathlineto{\pgfqpoint{1.399460in}{3.115055in}}%
\pgfpathlineto{\pgfqpoint{1.405080in}{3.122867in}}%
\pgfpathlineto{\pgfqpoint{1.414981in}{3.136713in}}%
\pgfpathlineto{\pgfqpoint{1.418995in}{3.138894in}}%
\pgfpathlineto{\pgfqpoint{1.422474in}{3.138383in}}%
\pgfpathlineto{\pgfqpoint{1.426221in}{3.135430in}}%
\pgfpathlineto{\pgfqpoint{1.431840in}{3.127649in}}%
\pgfpathlineto{\pgfqpoint{1.442010in}{3.113526in}}%
\pgfpathlineto{\pgfqpoint{1.446024in}{3.111492in}}%
\pgfpathlineto{\pgfqpoint{1.449503in}{3.112137in}}%
\pgfpathlineto{\pgfqpoint{1.453517in}{3.115517in}}%
\pgfpathlineto{\pgfqpoint{1.459404in}{3.123956in}}%
\pgfpathlineto{\pgfqpoint{1.468503in}{3.136604in}}%
\pgfpathlineto{\pgfqpoint{1.472784in}{3.138918in}}%
\pgfpathlineto{\pgfqpoint{1.476263in}{3.138312in}}%
\pgfpathlineto{\pgfqpoint{1.480010in}{3.135271in}}%
\pgfpathlineto{\pgfqpoint{1.485630in}{3.127417in}}%
\pgfpathlineto{\pgfqpoint{1.495531in}{3.113631in}}%
\pgfpathlineto{\pgfqpoint{1.499545in}{3.111510in}}%
\pgfpathlineto{\pgfqpoint{1.503024in}{3.112077in}}%
\pgfpathlineto{\pgfqpoint{1.506771in}{3.115082in}}%
\pgfpathlineto{\pgfqpoint{1.512390in}{3.122905in}}%
\pgfpathlineto{\pgfqpoint{1.522292in}{3.136734in}}%
\pgfpathlineto{\pgfqpoint{1.526306in}{3.138898in}}%
\pgfpathlineto{\pgfqpoint{1.529785in}{3.138371in}}%
\pgfpathlineto{\pgfqpoint{1.533531in}{3.135403in}}%
\pgfpathlineto{\pgfqpoint{1.539151in}{3.127610in}}%
\pgfpathlineto{\pgfqpoint{1.549320in}{3.113505in}}%
\pgfpathlineto{\pgfqpoint{1.553334in}{3.111488in}}%
\pgfpathlineto{\pgfqpoint{1.556813in}{3.112150in}}%
\pgfpathlineto{\pgfqpoint{1.560827in}{3.115545in}}%
\pgfpathlineto{\pgfqpoint{1.566982in}{3.124425in}}%
\pgfpathlineto{\pgfqpoint{1.575813in}{3.136626in}}%
\pgfpathlineto{\pgfqpoint{1.580095in}{3.138922in}}%
\pgfpathlineto{\pgfqpoint{1.583574in}{3.138300in}}%
\pgfpathlineto{\pgfqpoint{1.587588in}{3.134943in}}%
\pgfpathlineto{\pgfqpoint{1.593476in}{3.126522in}}%
\pgfpathlineto{\pgfqpoint{1.602574in}{3.113848in}}%
\pgfpathlineto{\pgfqpoint{1.606856in}{3.111506in}}%
\pgfpathlineto{\pgfqpoint{1.610335in}{3.112089in}}%
\pgfpathlineto{\pgfqpoint{1.614081in}{3.115108in}}%
\pgfpathlineto{\pgfqpoint{1.619701in}{3.122944in}}%
\pgfpathlineto{\pgfqpoint{1.629603in}{3.136756in}}%
\pgfpathlineto{\pgfqpoint{1.633617in}{3.138903in}}%
\pgfpathlineto{\pgfqpoint{1.637096in}{3.138360in}}%
\pgfpathlineto{\pgfqpoint{1.640842in}{3.135377in}}%
\pgfpathlineto{\pgfqpoint{1.646462in}{3.127572in}}%
\pgfpathlineto{\pgfqpoint{1.656363in}{3.113717in}}%
\pgfpathlineto{\pgfqpoint{1.660378in}{3.111527in}}%
\pgfpathlineto{\pgfqpoint{1.663856in}{3.112031in}}%
\pgfpathlineto{\pgfqpoint{1.667603in}{3.114976in}}%
\pgfpathlineto{\pgfqpoint{1.673223in}{3.122751in}}%
\pgfpathlineto{\pgfqpoint{1.683392in}{3.136882in}}%
\pgfpathlineto{\pgfqpoint{1.687406in}{3.138925in}}%
\pgfpathlineto{\pgfqpoint{1.690885in}{3.138288in}}%
\pgfpathlineto{\pgfqpoint{1.694899in}{3.134915in}}%
\pgfpathlineto{\pgfqpoint{1.700786in}{3.126483in}}%
\pgfpathlineto{\pgfqpoint{1.709885in}{3.113826in}}%
\pgfpathlineto{\pgfqpoint{1.714167in}{3.111502in}}%
\pgfpathlineto{\pgfqpoint{1.717646in}{3.112101in}}%
\pgfpathlineto{\pgfqpoint{1.721392in}{3.115135in}}%
\pgfpathlineto{\pgfqpoint{1.727012in}{3.122983in}}%
\pgfpathlineto{\pgfqpoint{1.736913in}{3.136777in}}%
\pgfpathlineto{\pgfqpoint{1.740927in}{3.138907in}}%
\pgfpathlineto{\pgfqpoint{1.744406in}{3.138348in}}%
\pgfpathlineto{\pgfqpoint{1.748153in}{3.135351in}}%
\pgfpathlineto{\pgfqpoint{1.753773in}{3.127533in}}%
\pgfpathlineto{\pgfqpoint{1.763674in}{3.113695in}}%
\pgfpathlineto{\pgfqpoint{1.767688in}{3.111523in}}%
\pgfpathlineto{\pgfqpoint{1.771167in}{3.112042in}}%
\pgfpathlineto{\pgfqpoint{1.774914in}{3.115002in}}%
\pgfpathlineto{\pgfqpoint{1.780533in}{3.122789in}}%
\pgfpathlineto{\pgfqpoint{1.790703in}{3.136903in}}%
\pgfpathlineto{\pgfqpoint{1.794717in}{3.138929in}}%
\pgfpathlineto{\pgfqpoint{1.798196in}{3.138275in}}%
\pgfpathlineto{\pgfqpoint{1.802210in}{3.134888in}}%
\pgfpathlineto{\pgfqpoint{1.808365in}{3.126013in}}%
\pgfpathlineto{\pgfqpoint{1.817196in}{3.113804in}}%
\pgfpathlineto{\pgfqpoint{1.821477in}{3.111499in}}%
\pgfpathlineto{\pgfqpoint{1.824956in}{3.112113in}}%
\pgfpathlineto{\pgfqpoint{1.828970in}{3.115462in}}%
\pgfpathlineto{\pgfqpoint{1.834858in}{3.123877in}}%
\pgfpathlineto{\pgfqpoint{1.843957in}{3.136560in}}%
\pgfpathlineto{\pgfqpoint{1.848238in}{3.138911in}}%
\pgfpathlineto{\pgfqpoint{1.851717in}{3.138336in}}%
\pgfpathlineto{\pgfqpoint{1.855464in}{3.135324in}}%
\pgfpathlineto{\pgfqpoint{1.861083in}{3.127494in}}%
\pgfpathlineto{\pgfqpoint{1.870985in}{3.113674in}}%
\pgfpathlineto{\pgfqpoint{1.874999in}{3.111518in}}%
\pgfpathlineto{\pgfqpoint{1.878478in}{3.112054in}}%
\pgfpathlineto{\pgfqpoint{1.882224in}{3.115029in}}%
\pgfpathlineto{\pgfqpoint{1.887844in}{3.122828in}}%
\pgfpathlineto{\pgfqpoint{1.897746in}{3.136691in}}%
\pgfpathlineto{\pgfqpoint{1.901760in}{3.138890in}}%
\pgfpathlineto{\pgfqpoint{1.905239in}{3.138394in}}%
\pgfpathlineto{\pgfqpoint{1.908985in}{3.135456in}}%
\pgfpathlineto{\pgfqpoint{1.914605in}{3.127688in}}%
\pgfpathlineto{\pgfqpoint{1.924774in}{3.113547in}}%
\pgfpathlineto{\pgfqpoint{1.928788in}{3.111495in}}%
\pgfpathlineto{\pgfqpoint{1.932267in}{3.112125in}}%
\pgfpathlineto{\pgfqpoint{1.936281in}{3.115490in}}%
\pgfpathlineto{\pgfqpoint{1.942169in}{3.123916in}}%
\pgfpathlineto{\pgfqpoint{1.951267in}{3.136582in}}%
\pgfpathlineto{\pgfqpoint{1.955549in}{3.138914in}}%
\pgfpathlineto{\pgfqpoint{1.959028in}{3.138324in}}%
\pgfpathlineto{\pgfqpoint{1.962774in}{3.135297in}}%
\pgfpathlineto{\pgfqpoint{1.968394in}{3.127456in}}%
\pgfpathlineto{\pgfqpoint{1.978296in}{3.113652in}}%
\pgfpathlineto{\pgfqpoint{1.982310in}{3.111514in}}%
\pgfpathlineto{\pgfqpoint{1.985789in}{3.112065in}}%
\pgfpathlineto{\pgfqpoint{1.989535in}{3.115055in}}%
\pgfpathlineto{\pgfqpoint{1.995155in}{3.122867in}}%
\pgfpathlineto{\pgfqpoint{2.005056in}{3.136713in}}%
\pgfpathlineto{\pgfqpoint{2.009071in}{3.138894in}}%
\pgfpathlineto{\pgfqpoint{2.012549in}{3.138383in}}%
\pgfpathlineto{\pgfqpoint{2.016296in}{3.135430in}}%
\pgfpathlineto{\pgfqpoint{2.021916in}{3.127649in}}%
\pgfpathlineto{\pgfqpoint{2.032085in}{3.113526in}}%
\pgfpathlineto{\pgfqpoint{2.036099in}{3.111492in}}%
\pgfpathlineto{\pgfqpoint{2.039578in}{3.112137in}}%
\pgfpathlineto{\pgfqpoint{2.043592in}{3.115517in}}%
\pgfpathlineto{\pgfqpoint{2.049479in}{3.123956in}}%
\pgfpathlineto{\pgfqpoint{2.058578in}{3.136604in}}%
\pgfpathlineto{\pgfqpoint{2.062860in}{3.138918in}}%
\pgfpathlineto{\pgfqpoint{2.066339in}{3.138312in}}%
\pgfpathlineto{\pgfqpoint{2.070085in}{3.135271in}}%
\pgfpathlineto{\pgfqpoint{2.075705in}{3.127417in}}%
\pgfpathlineto{\pgfqpoint{2.085606in}{3.113631in}}%
\pgfpathlineto{\pgfqpoint{2.089620in}{3.111510in}}%
\pgfpathlineto{\pgfqpoint{2.093099in}{3.112077in}}%
\pgfpathlineto{\pgfqpoint{2.096846in}{3.115082in}}%
\pgfpathlineto{\pgfqpoint{2.102466in}{3.122905in}}%
\pgfpathlineto{\pgfqpoint{2.112367in}{3.136734in}}%
\pgfpathlineto{\pgfqpoint{2.116381in}{3.138898in}}%
\pgfpathlineto{\pgfqpoint{2.119860in}{3.138371in}}%
\pgfpathlineto{\pgfqpoint{2.123607in}{3.135403in}}%
\pgfpathlineto{\pgfqpoint{2.129226in}{3.127610in}}%
\pgfpathlineto{\pgfqpoint{2.139396in}{3.113505in}}%
\pgfpathlineto{\pgfqpoint{2.143410in}{3.111488in}}%
\pgfpathlineto{\pgfqpoint{2.146889in}{3.112150in}}%
\pgfpathlineto{\pgfqpoint{2.150903in}{3.115545in}}%
\pgfpathlineto{\pgfqpoint{2.157058in}{3.124425in}}%
\pgfpathlineto{\pgfqpoint{2.165889in}{3.136626in}}%
\pgfpathlineto{\pgfqpoint{2.170170in}{3.138922in}}%
\pgfpathlineto{\pgfqpoint{2.173649in}{3.138300in}}%
\pgfpathlineto{\pgfqpoint{2.177663in}{3.134943in}}%
\pgfpathlineto{\pgfqpoint{2.183551in}{3.126522in}}%
\pgfpathlineto{\pgfqpoint{2.192649in}{3.113848in}}%
\pgfpathlineto{\pgfqpoint{2.196931in}{3.111506in}}%
\pgfpathlineto{\pgfqpoint{2.200410in}{3.112089in}}%
\pgfpathlineto{\pgfqpoint{2.204157in}{3.115108in}}%
\pgfpathlineto{\pgfqpoint{2.209776in}{3.122944in}}%
\pgfpathlineto{\pgfqpoint{2.219678in}{3.136756in}}%
\pgfpathlineto{\pgfqpoint{2.223692in}{3.138903in}}%
\pgfpathlineto{\pgfqpoint{2.227171in}{3.138360in}}%
\pgfpathlineto{\pgfqpoint{2.230917in}{3.135377in}}%
\pgfpathlineto{\pgfqpoint{2.236537in}{3.127572in}}%
\pgfpathlineto{\pgfqpoint{2.246439in}{3.113717in}}%
\pgfpathlineto{\pgfqpoint{2.250453in}{3.111527in}}%
\pgfpathlineto{\pgfqpoint{2.253932in}{3.112031in}}%
\pgfpathlineto{\pgfqpoint{2.257678in}{3.114976in}}%
\pgfpathlineto{\pgfqpoint{2.263298in}{3.122751in}}%
\pgfpathlineto{\pgfqpoint{2.273467in}{3.136882in}}%
\pgfpathlineto{\pgfqpoint{2.277481in}{3.138925in}}%
\pgfpathlineto{\pgfqpoint{2.280960in}{3.138288in}}%
\pgfpathlineto{\pgfqpoint{2.284974in}{3.134915in}}%
\pgfpathlineto{\pgfqpoint{2.290862in}{3.126483in}}%
\pgfpathlineto{\pgfqpoint{2.299960in}{3.113826in}}%
\pgfpathlineto{\pgfqpoint{2.304242in}{3.111502in}}%
\pgfpathlineto{\pgfqpoint{2.307721in}{3.112101in}}%
\pgfpathlineto{\pgfqpoint{2.311467in}{3.115135in}}%
\pgfpathlineto{\pgfqpoint{2.317087in}{3.122983in}}%
\pgfpathlineto{\pgfqpoint{2.326989in}{3.136777in}}%
\pgfpathlineto{\pgfqpoint{2.331003in}{3.138907in}}%
\pgfpathlineto{\pgfqpoint{2.334482in}{3.138348in}}%
\pgfpathlineto{\pgfqpoint{2.338228in}{3.135351in}}%
\pgfpathlineto{\pgfqpoint{2.343848in}{3.127533in}}%
\pgfpathlineto{\pgfqpoint{2.353749in}{3.113695in}}%
\pgfpathlineto{\pgfqpoint{2.357764in}{3.111523in}}%
\pgfpathlineto{\pgfqpoint{2.361242in}{3.112042in}}%
\pgfpathlineto{\pgfqpoint{2.364989in}{3.115002in}}%
\pgfpathlineto{\pgfqpoint{2.370609in}{3.122789in}}%
\pgfpathlineto{\pgfqpoint{2.380778in}{3.136903in}}%
\pgfpathlineto{\pgfqpoint{2.384792in}{3.138929in}}%
\pgfpathlineto{\pgfqpoint{2.388271in}{3.138275in}}%
\pgfpathlineto{\pgfqpoint{2.392285in}{3.134888in}}%
\pgfpathlineto{\pgfqpoint{2.398440in}{3.126013in}}%
\pgfpathlineto{\pgfqpoint{2.407271in}{3.113804in}}%
\pgfpathlineto{\pgfqpoint{2.411553in}{3.111499in}}%
\pgfpathlineto{\pgfqpoint{2.415032in}{3.112113in}}%
\pgfpathlineto{\pgfqpoint{2.419046in}{3.115462in}}%
\pgfpathlineto{\pgfqpoint{2.424933in}{3.123877in}}%
\pgfpathlineto{\pgfqpoint{2.434032in}{3.136560in}}%
\pgfpathlineto{\pgfqpoint{2.438313in}{3.138911in}}%
\pgfpathlineto{\pgfqpoint{2.441792in}{3.138336in}}%
\pgfpathlineto{\pgfqpoint{2.445539in}{3.135324in}}%
\pgfpathlineto{\pgfqpoint{2.451159in}{3.127494in}}%
\pgfpathlineto{\pgfqpoint{2.461060in}{3.113674in}}%
\pgfpathlineto{\pgfqpoint{2.465074in}{3.111518in}}%
\pgfpathlineto{\pgfqpoint{2.468553in}{3.112054in}}%
\pgfpathlineto{\pgfqpoint{2.472300in}{3.115029in}}%
\pgfpathlineto{\pgfqpoint{2.477919in}{3.122828in}}%
\pgfpathlineto{\pgfqpoint{2.487821in}{3.136691in}}%
\pgfpathlineto{\pgfqpoint{2.491835in}{3.138890in}}%
\pgfpathlineto{\pgfqpoint{2.495314in}{3.138394in}}%
\pgfpathlineto{\pgfqpoint{2.499060in}{3.135456in}}%
\pgfpathlineto{\pgfqpoint{2.504680in}{3.127688in}}%
\pgfpathlineto{\pgfqpoint{2.514849in}{3.113547in}}%
\pgfpathlineto{\pgfqpoint{2.518863in}{3.111495in}}%
\pgfpathlineto{\pgfqpoint{2.522342in}{3.112125in}}%
\pgfpathlineto{\pgfqpoint{2.526356in}{3.115490in}}%
\pgfpathlineto{\pgfqpoint{2.532244in}{3.123916in}}%
\pgfpathlineto{\pgfqpoint{2.541342in}{3.136582in}}%
\pgfpathlineto{\pgfqpoint{2.545624in}{3.138914in}}%
\pgfpathlineto{\pgfqpoint{2.549103in}{3.138324in}}%
\pgfpathlineto{\pgfqpoint{2.552850in}{3.135297in}}%
\pgfpathlineto{\pgfqpoint{2.558469in}{3.127456in}}%
\pgfpathlineto{\pgfqpoint{2.568371in}{3.113652in}}%
\pgfpathlineto{\pgfqpoint{2.572385in}{3.111514in}}%
\pgfpathlineto{\pgfqpoint{2.575864in}{3.112065in}}%
\pgfpathlineto{\pgfqpoint{2.579610in}{3.115055in}}%
\pgfpathlineto{\pgfqpoint{2.585230in}{3.122867in}}%
\pgfpathlineto{\pgfqpoint{2.595132in}{3.136713in}}%
\pgfpathlineto{\pgfqpoint{2.599146in}{3.138894in}}%
\pgfpathlineto{\pgfqpoint{2.602625in}{3.138383in}}%
\pgfpathlineto{\pgfqpoint{2.606371in}{3.135430in}}%
\pgfpathlineto{\pgfqpoint{2.611991in}{3.127649in}}%
\pgfpathlineto{\pgfqpoint{2.622160in}{3.113526in}}%
\pgfpathlineto{\pgfqpoint{2.626174in}{3.111492in}}%
\pgfpathlineto{\pgfqpoint{2.629653in}{3.112137in}}%
\pgfpathlineto{\pgfqpoint{2.633667in}{3.115517in}}%
\pgfpathlineto{\pgfqpoint{2.639555in}{3.123956in}}%
\pgfpathlineto{\pgfqpoint{2.648653in}{3.136604in}}%
\pgfpathlineto{\pgfqpoint{2.652935in}{3.138918in}}%
\pgfpathlineto{\pgfqpoint{2.656414in}{3.138312in}}%
\pgfpathlineto{\pgfqpoint{2.660160in}{3.135271in}}%
\pgfpathlineto{\pgfqpoint{2.665780in}{3.127417in}}%
\pgfpathlineto{\pgfqpoint{2.675682in}{3.113631in}}%
\pgfpathlineto{\pgfqpoint{2.679696in}{3.111510in}}%
\pgfpathlineto{\pgfqpoint{2.683175in}{3.112077in}}%
\pgfpathlineto{\pgfqpoint{2.686921in}{3.115082in}}%
\pgfpathlineto{\pgfqpoint{2.692541in}{3.122905in}}%
\pgfpathlineto{\pgfqpoint{2.702442in}{3.136734in}}%
\pgfpathlineto{\pgfqpoint{2.706457in}{3.138898in}}%
\pgfpathlineto{\pgfqpoint{2.709935in}{3.138371in}}%
\pgfpathlineto{\pgfqpoint{2.713682in}{3.135403in}}%
\pgfpathlineto{\pgfqpoint{2.719302in}{3.127610in}}%
\pgfpathlineto{\pgfqpoint{2.729471in}{3.113505in}}%
\pgfpathlineto{\pgfqpoint{2.733485in}{3.111488in}}%
\pgfpathlineto{\pgfqpoint{2.736964in}{3.112150in}}%
\pgfpathlineto{\pgfqpoint{2.740978in}{3.115545in}}%
\pgfpathlineto{\pgfqpoint{2.747133in}{3.124425in}}%
\pgfpathlineto{\pgfqpoint{2.755964in}{3.136626in}}%
\pgfpathlineto{\pgfqpoint{2.760246in}{3.138922in}}%
\pgfpathlineto{\pgfqpoint{2.763725in}{3.138300in}}%
\pgfpathlineto{\pgfqpoint{2.767739in}{3.134943in}}%
\pgfpathlineto{\pgfqpoint{2.773626in}{3.126522in}}%
\pgfpathlineto{\pgfqpoint{2.782725in}{3.113848in}}%
\pgfpathlineto{\pgfqpoint{2.787006in}{3.111506in}}%
\pgfpathlineto{\pgfqpoint{2.790485in}{3.112089in}}%
\pgfpathlineto{\pgfqpoint{2.794232in}{3.115108in}}%
\pgfpathlineto{\pgfqpoint{2.799852in}{3.122944in}}%
\pgfpathlineto{\pgfqpoint{2.809753in}{3.136756in}}%
\pgfpathlineto{\pgfqpoint{2.813767in}{3.138903in}}%
\pgfpathlineto{\pgfqpoint{2.817246in}{3.138360in}}%
\pgfpathlineto{\pgfqpoint{2.820993in}{3.135377in}}%
\pgfpathlineto{\pgfqpoint{2.826612in}{3.127572in}}%
\pgfpathlineto{\pgfqpoint{2.836514in}{3.113717in}}%
\pgfpathlineto{\pgfqpoint{2.840528in}{3.111527in}}%
\pgfpathlineto{\pgfqpoint{2.844007in}{3.112031in}}%
\pgfpathlineto{\pgfqpoint{2.847753in}{3.114976in}}%
\pgfpathlineto{\pgfqpoint{2.853373in}{3.122751in}}%
\pgfpathlineto{\pgfqpoint{2.863542in}{3.136882in}}%
\pgfpathlineto{\pgfqpoint{2.867556in}{3.138925in}}%
\pgfpathlineto{\pgfqpoint{2.871035in}{3.138288in}}%
\pgfpathlineto{\pgfqpoint{2.875049in}{3.134915in}}%
\pgfpathlineto{\pgfqpoint{2.880937in}{3.126483in}}%
\pgfpathlineto{\pgfqpoint{2.890035in}{3.113826in}}%
\pgfpathlineto{\pgfqpoint{2.894317in}{3.111502in}}%
\pgfpathlineto{\pgfqpoint{2.897796in}{3.112101in}}%
\pgfpathlineto{\pgfqpoint{2.901543in}{3.115135in}}%
\pgfpathlineto{\pgfqpoint{2.907162in}{3.122983in}}%
\pgfpathlineto{\pgfqpoint{2.917064in}{3.136777in}}%
\pgfpathlineto{\pgfqpoint{2.921078in}{3.138907in}}%
\pgfpathlineto{\pgfqpoint{2.924557in}{3.138348in}}%
\pgfpathlineto{\pgfqpoint{2.928303in}{3.135351in}}%
\pgfpathlineto{\pgfqpoint{2.933923in}{3.127533in}}%
\pgfpathlineto{\pgfqpoint{2.943825in}{3.113695in}}%
\pgfpathlineto{\pgfqpoint{2.947839in}{3.111523in}}%
\pgfpathlineto{\pgfqpoint{2.951318in}{3.112042in}}%
\pgfpathlineto{\pgfqpoint{2.955064in}{3.115002in}}%
\pgfpathlineto{\pgfqpoint{2.960684in}{3.122789in}}%
\pgfpathlineto{\pgfqpoint{2.970853in}{3.136903in}}%
\pgfpathlineto{\pgfqpoint{2.974867in}{3.138929in}}%
\pgfpathlineto{\pgfqpoint{2.978346in}{3.138275in}}%
\pgfpathlineto{\pgfqpoint{2.982360in}{3.134888in}}%
\pgfpathlineto{\pgfqpoint{2.988515in}{3.126013in}}%
\pgfpathlineto{\pgfqpoint{2.997346in}{3.113804in}}%
\pgfpathlineto{\pgfqpoint{3.001628in}{3.111499in}}%
\pgfpathlineto{\pgfqpoint{3.005107in}{3.112113in}}%
\pgfpathlineto{\pgfqpoint{3.009121in}{3.115462in}}%
\pgfpathlineto{\pgfqpoint{3.015008in}{3.123877in}}%
\pgfpathlineto{\pgfqpoint{3.024107in}{3.136560in}}%
\pgfpathlineto{\pgfqpoint{3.028389in}{3.138911in}}%
\pgfpathlineto{\pgfqpoint{3.031868in}{3.138336in}}%
\pgfpathlineto{\pgfqpoint{3.035614in}{3.135324in}}%
\pgfpathlineto{\pgfqpoint{3.041234in}{3.127494in}}%
\pgfpathlineto{\pgfqpoint{3.051135in}{3.113674in}}%
\pgfpathlineto{\pgfqpoint{3.055150in}{3.111518in}}%
\pgfpathlineto{\pgfqpoint{3.058628in}{3.112054in}}%
\pgfpathlineto{\pgfqpoint{3.062375in}{3.115029in}}%
\pgfpathlineto{\pgfqpoint{3.067995in}{3.122828in}}%
\pgfpathlineto{\pgfqpoint{3.077896in}{3.136691in}}%
\pgfpathlineto{\pgfqpoint{3.081910in}{3.138890in}}%
\pgfpathlineto{\pgfqpoint{3.085389in}{3.138394in}}%
\pgfpathlineto{\pgfqpoint{3.089136in}{3.135456in}}%
\pgfpathlineto{\pgfqpoint{3.094755in}{3.127688in}}%
\pgfpathlineto{\pgfqpoint{3.104925in}{3.113547in}}%
\pgfpathlineto{\pgfqpoint{3.108939in}{3.111495in}}%
\pgfpathlineto{\pgfqpoint{3.112418in}{3.112125in}}%
\pgfpathlineto{\pgfqpoint{3.116432in}{3.115490in}}%
\pgfpathlineto{\pgfqpoint{3.122319in}{3.123916in}}%
\pgfpathlineto{\pgfqpoint{3.131418in}{3.136582in}}%
\pgfpathlineto{\pgfqpoint{3.135699in}{3.138914in}}%
\pgfpathlineto{\pgfqpoint{3.139178in}{3.138324in}}%
\pgfpathlineto{\pgfqpoint{3.142925in}{3.135297in}}%
\pgfpathlineto{\pgfqpoint{3.148545in}{3.127456in}}%
\pgfpathlineto{\pgfqpoint{3.158446in}{3.113652in}}%
\pgfpathlineto{\pgfqpoint{3.162460in}{3.111514in}}%
\pgfpathlineto{\pgfqpoint{3.165939in}{3.112065in}}%
\pgfpathlineto{\pgfqpoint{3.169686in}{3.115055in}}%
\pgfpathlineto{\pgfqpoint{3.175305in}{3.122867in}}%
\pgfpathlineto{\pgfqpoint{3.185207in}{3.136713in}}%
\pgfpathlineto{\pgfqpoint{3.189221in}{3.138894in}}%
\pgfpathlineto{\pgfqpoint{3.192700in}{3.138383in}}%
\pgfpathlineto{\pgfqpoint{3.196446in}{3.135430in}}%
\pgfpathlineto{\pgfqpoint{3.202066in}{3.127649in}}%
\pgfpathlineto{\pgfqpoint{3.212235in}{3.113526in}}%
\pgfpathlineto{\pgfqpoint{3.216249in}{3.111492in}}%
\pgfpathlineto{\pgfqpoint{3.219728in}{3.112137in}}%
\pgfpathlineto{\pgfqpoint{3.223742in}{3.115517in}}%
\pgfpathlineto{\pgfqpoint{3.229630in}{3.123956in}}%
\pgfpathlineto{\pgfqpoint{3.238728in}{3.136604in}}%
\pgfpathlineto{\pgfqpoint{3.243010in}{3.138918in}}%
\pgfpathlineto{\pgfqpoint{3.246489in}{3.138312in}}%
\pgfpathlineto{\pgfqpoint{3.250236in}{3.135271in}}%
\pgfpathlineto{\pgfqpoint{3.255855in}{3.127417in}}%
\pgfpathlineto{\pgfqpoint{3.265757in}{3.113631in}}%
\pgfpathlineto{\pgfqpoint{3.269771in}{3.111510in}}%
\pgfpathlineto{\pgfqpoint{3.273250in}{3.112077in}}%
\pgfpathlineto{\pgfqpoint{3.276996in}{3.115082in}}%
\pgfpathlineto{\pgfqpoint{3.282616in}{3.122905in}}%
\pgfpathlineto{\pgfqpoint{3.292518in}{3.136734in}}%
\pgfpathlineto{\pgfqpoint{3.296532in}{3.138898in}}%
\pgfpathlineto{\pgfqpoint{3.300011in}{3.138371in}}%
\pgfpathlineto{\pgfqpoint{3.303757in}{3.135403in}}%
\pgfpathlineto{\pgfqpoint{3.309377in}{3.127610in}}%
\pgfpathlineto{\pgfqpoint{3.319546in}{3.113505in}}%
\pgfpathlineto{\pgfqpoint{3.323560in}{3.111488in}}%
\pgfpathlineto{\pgfqpoint{3.327039in}{3.112150in}}%
\pgfpathlineto{\pgfqpoint{3.331053in}{3.115545in}}%
\pgfpathlineto{\pgfqpoint{3.337208in}{3.124425in}}%
\pgfpathlineto{\pgfqpoint{3.346039in}{3.136626in}}%
\pgfpathlineto{\pgfqpoint{3.350321in}{3.138922in}}%
\pgfpathlineto{\pgfqpoint{3.353800in}{3.138300in}}%
\pgfpathlineto{\pgfqpoint{3.357814in}{3.134943in}}%
\pgfpathlineto{\pgfqpoint{3.363701in}{3.126522in}}%
\pgfpathlineto{\pgfqpoint{3.372800in}{3.113848in}}%
\pgfpathlineto{\pgfqpoint{3.377082in}{3.111506in}}%
\pgfpathlineto{\pgfqpoint{3.380561in}{3.112089in}}%
\pgfpathlineto{\pgfqpoint{3.384307in}{3.115108in}}%
\pgfpathlineto{\pgfqpoint{3.389927in}{3.122944in}}%
\pgfpathlineto{\pgfqpoint{3.399828in}{3.136756in}}%
\pgfpathlineto{\pgfqpoint{3.403842in}{3.138903in}}%
\pgfpathlineto{\pgfqpoint{3.407321in}{3.138360in}}%
\pgfpathlineto{\pgfqpoint{3.411068in}{3.135377in}}%
\pgfpathlineto{\pgfqpoint{3.416688in}{3.127572in}}%
\pgfpathlineto{\pgfqpoint{3.426589in}{3.113717in}}%
\pgfpathlineto{\pgfqpoint{3.430603in}{3.111527in}}%
\pgfpathlineto{\pgfqpoint{3.434082in}{3.112031in}}%
\pgfpathlineto{\pgfqpoint{3.437829in}{3.114976in}}%
\pgfpathlineto{\pgfqpoint{3.443448in}{3.122751in}}%
\pgfpathlineto{\pgfqpoint{3.453618in}{3.136882in}}%
\pgfpathlineto{\pgfqpoint{3.457632in}{3.138925in}}%
\pgfpathlineto{\pgfqpoint{3.461111in}{3.138288in}}%
\pgfpathlineto{\pgfqpoint{3.465125in}{3.134915in}}%
\pgfpathlineto{\pgfqpoint{3.471012in}{3.126483in}}%
\pgfpathlineto{\pgfqpoint{3.480111in}{3.113826in}}%
\pgfpathlineto{\pgfqpoint{3.484392in}{3.111502in}}%
\pgfpathlineto{\pgfqpoint{3.487871in}{3.112101in}}%
\pgfpathlineto{\pgfqpoint{3.491618in}{3.115135in}}%
\pgfpathlineto{\pgfqpoint{3.497238in}{3.122983in}}%
\pgfpathlineto{\pgfqpoint{3.507139in}{3.136777in}}%
\pgfpathlineto{\pgfqpoint{3.511153in}{3.138907in}}%
\pgfpathlineto{\pgfqpoint{3.514632in}{3.138348in}}%
\pgfpathlineto{\pgfqpoint{3.518379in}{3.135351in}}%
\pgfpathlineto{\pgfqpoint{3.523998in}{3.127533in}}%
\pgfpathlineto{\pgfqpoint{3.533900in}{3.113695in}}%
\pgfpathlineto{\pgfqpoint{3.537914in}{3.111523in}}%
\pgfpathlineto{\pgfqpoint{3.541393in}{3.112042in}}%
\pgfpathlineto{\pgfqpoint{3.545139in}{3.115002in}}%
\pgfpathlineto{\pgfqpoint{3.550759in}{3.122789in}}%
\pgfpathlineto{\pgfqpoint{3.560928in}{3.136903in}}%
\pgfpathlineto{\pgfqpoint{3.564942in}{3.138929in}}%
\pgfpathlineto{\pgfqpoint{3.568421in}{3.138275in}}%
\pgfpathlineto{\pgfqpoint{3.572435in}{3.134888in}}%
\pgfpathlineto{\pgfqpoint{3.578590in}{3.126013in}}%
\pgfpathlineto{\pgfqpoint{3.587421in}{3.113804in}}%
\pgfpathlineto{\pgfqpoint{3.591703in}{3.111499in}}%
\pgfpathlineto{\pgfqpoint{3.595182in}{3.112113in}}%
\pgfpathlineto{\pgfqpoint{3.599196in}{3.115462in}}%
\pgfpathlineto{\pgfqpoint{3.605084in}{3.123877in}}%
\pgfpathlineto{\pgfqpoint{3.614182in}{3.136560in}}%
\pgfpathlineto{\pgfqpoint{3.618464in}{3.138911in}}%
\pgfpathlineto{\pgfqpoint{3.621943in}{3.138336in}}%
\pgfpathlineto{\pgfqpoint{3.625689in}{3.135324in}}%
\pgfpathlineto{\pgfqpoint{3.631309in}{3.127494in}}%
\pgfpathlineto{\pgfqpoint{3.641211in}{3.113674in}}%
\pgfpathlineto{\pgfqpoint{3.645225in}{3.111518in}}%
\pgfpathlineto{\pgfqpoint{3.648704in}{3.112054in}}%
\pgfpathlineto{\pgfqpoint{3.652450in}{3.115029in}}%
\pgfpathlineto{\pgfqpoint{3.658070in}{3.122828in}}%
\pgfpathlineto{\pgfqpoint{3.667971in}{3.136691in}}%
\pgfpathlineto{\pgfqpoint{3.671986in}{3.138890in}}%
\pgfpathlineto{\pgfqpoint{3.675464in}{3.138394in}}%
\pgfpathlineto{\pgfqpoint{3.679211in}{3.135456in}}%
\pgfpathlineto{\pgfqpoint{3.684831in}{3.127688in}}%
\pgfpathlineto{\pgfqpoint{3.695000in}{3.113547in}}%
\pgfpathlineto{\pgfqpoint{3.699014in}{3.111495in}}%
\pgfpathlineto{\pgfqpoint{3.702493in}{3.112125in}}%
\pgfpathlineto{\pgfqpoint{3.706507in}{3.115490in}}%
\pgfpathlineto{\pgfqpoint{3.712394in}{3.123916in}}%
\pgfpathlineto{\pgfqpoint{3.721493in}{3.136582in}}%
\pgfpathlineto{\pgfqpoint{3.725775in}{3.138914in}}%
\pgfpathlineto{\pgfqpoint{3.729254in}{3.138324in}}%
\pgfpathlineto{\pgfqpoint{3.733000in}{3.135297in}}%
\pgfpathlineto{\pgfqpoint{3.738620in}{3.127456in}}%
\pgfpathlineto{\pgfqpoint{3.748521in}{3.113652in}}%
\pgfpathlineto{\pgfqpoint{3.752535in}{3.111514in}}%
\pgfpathlineto{\pgfqpoint{3.756014in}{3.112065in}}%
\pgfpathlineto{\pgfqpoint{3.759761in}{3.115055in}}%
\pgfpathlineto{\pgfqpoint{3.765381in}{3.122867in}}%
\pgfpathlineto{\pgfqpoint{3.775282in}{3.136713in}}%
\pgfpathlineto{\pgfqpoint{3.779296in}{3.138894in}}%
\pgfpathlineto{\pgfqpoint{3.782775in}{3.138383in}}%
\pgfpathlineto{\pgfqpoint{3.786522in}{3.135430in}}%
\pgfpathlineto{\pgfqpoint{3.792141in}{3.127649in}}%
\pgfpathlineto{\pgfqpoint{3.802311in}{3.113526in}}%
\pgfpathlineto{\pgfqpoint{3.806325in}{3.111492in}}%
\pgfpathlineto{\pgfqpoint{3.809804in}{3.112137in}}%
\pgfpathlineto{\pgfqpoint{3.813818in}{3.115517in}}%
\pgfpathlineto{\pgfqpoint{3.819705in}{3.123956in}}%
\pgfpathlineto{\pgfqpoint{3.828804in}{3.136604in}}%
\pgfpathlineto{\pgfqpoint{3.833085in}{3.138918in}}%
\pgfpathlineto{\pgfqpoint{3.836564in}{3.138312in}}%
\pgfpathlineto{\pgfqpoint{3.840311in}{3.135271in}}%
\pgfpathlineto{\pgfqpoint{3.845931in}{3.127417in}}%
\pgfpathlineto{\pgfqpoint{3.855832in}{3.113631in}}%
\pgfpathlineto{\pgfqpoint{3.859846in}{3.111510in}}%
\pgfpathlineto{\pgfqpoint{3.863325in}{3.112077in}}%
\pgfpathlineto{\pgfqpoint{3.867072in}{3.115082in}}%
\pgfpathlineto{\pgfqpoint{3.872691in}{3.122905in}}%
\pgfpathlineto{\pgfqpoint{3.882593in}{3.136734in}}%
\pgfpathlineto{\pgfqpoint{3.886607in}{3.138898in}}%
\pgfpathlineto{\pgfqpoint{3.890086in}{3.138371in}}%
\pgfpathlineto{\pgfqpoint{3.893832in}{3.135403in}}%
\pgfpathlineto{\pgfqpoint{3.899452in}{3.127610in}}%
\pgfpathlineto{\pgfqpoint{3.909621in}{3.113505in}}%
\pgfpathlineto{\pgfqpoint{3.913635in}{3.111488in}}%
\pgfpathlineto{\pgfqpoint{3.917114in}{3.112150in}}%
\pgfpathlineto{\pgfqpoint{3.921128in}{3.115545in}}%
\pgfpathlineto{\pgfqpoint{3.927283in}{3.124425in}}%
\pgfpathlineto{\pgfqpoint{3.936114in}{3.136626in}}%
\pgfpathlineto{\pgfqpoint{3.940396in}{3.138922in}}%
\pgfpathlineto{\pgfqpoint{3.943875in}{3.138300in}}%
\pgfpathlineto{\pgfqpoint{3.947889in}{3.134943in}}%
\pgfpathlineto{\pgfqpoint{3.953777in}{3.126522in}}%
\pgfpathlineto{\pgfqpoint{3.962875in}{3.113848in}}%
\pgfpathlineto{\pgfqpoint{3.967157in}{3.111506in}}%
\pgfpathlineto{\pgfqpoint{3.970636in}{3.112089in}}%
\pgfpathlineto{\pgfqpoint{3.974382in}{3.115108in}}%
\pgfpathlineto{\pgfqpoint{3.980002in}{3.122944in}}%
\pgfpathlineto{\pgfqpoint{3.989904in}{3.136756in}}%
\pgfpathlineto{\pgfqpoint{3.993918in}{3.138903in}}%
\pgfpathlineto{\pgfqpoint{3.997397in}{3.138360in}}%
\pgfpathlineto{\pgfqpoint{4.001143in}{3.135377in}}%
\pgfpathlineto{\pgfqpoint{4.006763in}{3.127572in}}%
\pgfpathlineto{\pgfqpoint{4.016664in}{3.113717in}}%
\pgfpathlineto{\pgfqpoint{4.020679in}{3.111527in}}%
\pgfpathlineto{\pgfqpoint{4.024157in}{3.112031in}}%
\pgfpathlineto{\pgfqpoint{4.027904in}{3.114976in}}%
\pgfpathlineto{\pgfqpoint{4.033524in}{3.122751in}}%
\pgfpathlineto{\pgfqpoint{4.043693in}{3.136882in}}%
\pgfpathlineto{\pgfqpoint{4.047707in}{3.138925in}}%
\pgfpathlineto{\pgfqpoint{4.051186in}{3.138288in}}%
\pgfpathlineto{\pgfqpoint{4.055200in}{3.134915in}}%
\pgfpathlineto{\pgfqpoint{4.061087in}{3.126483in}}%
\pgfpathlineto{\pgfqpoint{4.070186in}{3.113826in}}%
\pgfpathlineto{\pgfqpoint{4.074468in}{3.111502in}}%
\pgfpathlineto{\pgfqpoint{4.077947in}{3.112101in}}%
\pgfpathlineto{\pgfqpoint{4.081693in}{3.115135in}}%
\pgfpathlineto{\pgfqpoint{4.087313in}{3.122983in}}%
\pgfpathlineto{\pgfqpoint{4.097214in}{3.136777in}}%
\pgfpathlineto{\pgfqpoint{4.101228in}{3.138907in}}%
\pgfpathlineto{\pgfqpoint{4.104707in}{3.138348in}}%
\pgfpathlineto{\pgfqpoint{4.108454in}{3.135351in}}%
\pgfpathlineto{\pgfqpoint{4.114074in}{3.127533in}}%
\pgfpathlineto{\pgfqpoint{4.123975in}{3.113695in}}%
\pgfpathlineto{\pgfqpoint{4.127989in}{3.111523in}}%
\pgfpathlineto{\pgfqpoint{4.131468in}{3.112042in}}%
\pgfpathlineto{\pgfqpoint{4.135215in}{3.115002in}}%
\pgfpathlineto{\pgfqpoint{4.140834in}{3.122789in}}%
\pgfpathlineto{\pgfqpoint{4.151004in}{3.136903in}}%
\pgfpathlineto{\pgfqpoint{4.155018in}{3.138929in}}%
\pgfpathlineto{\pgfqpoint{4.158497in}{3.138275in}}%
\pgfpathlineto{\pgfqpoint{4.162511in}{3.134888in}}%
\pgfpathlineto{\pgfqpoint{4.168666in}{3.126013in}}%
\pgfpathlineto{\pgfqpoint{4.177497in}{3.113804in}}%
\pgfpathlineto{\pgfqpoint{4.181778in}{3.111499in}}%
\pgfpathlineto{\pgfqpoint{4.185257in}{3.112113in}}%
\pgfpathlineto{\pgfqpoint{4.189271in}{3.115462in}}%
\pgfpathlineto{\pgfqpoint{4.195159in}{3.123877in}}%
\pgfpathlineto{\pgfqpoint{4.204257in}{3.136560in}}%
\pgfpathlineto{\pgfqpoint{4.208539in}{3.138911in}}%
\pgfpathlineto{\pgfqpoint{4.212018in}{3.138336in}}%
\pgfpathlineto{\pgfqpoint{4.215765in}{3.135324in}}%
\pgfpathlineto{\pgfqpoint{4.221384in}{3.127494in}}%
\pgfpathlineto{\pgfqpoint{4.231286in}{3.113674in}}%
\pgfpathlineto{\pgfqpoint{4.235300in}{3.111518in}}%
\pgfpathlineto{\pgfqpoint{4.238779in}{3.112054in}}%
\pgfpathlineto{\pgfqpoint{4.242525in}{3.115029in}}%
\pgfpathlineto{\pgfqpoint{4.248145in}{3.122828in}}%
\pgfpathlineto{\pgfqpoint{4.258047in}{3.136691in}}%
\pgfpathlineto{\pgfqpoint{4.262061in}{3.138890in}}%
\pgfpathlineto{\pgfqpoint{4.265540in}{3.138394in}}%
\pgfpathlineto{\pgfqpoint{4.269286in}{3.135456in}}%
\pgfpathlineto{\pgfqpoint{4.274906in}{3.127688in}}%
\pgfpathlineto{\pgfqpoint{4.285075in}{3.113547in}}%
\pgfpathlineto{\pgfqpoint{4.289089in}{3.111495in}}%
\pgfpathlineto{\pgfqpoint{4.292568in}{3.112125in}}%
\pgfpathlineto{\pgfqpoint{4.296582in}{3.115490in}}%
\pgfpathlineto{\pgfqpoint{4.302470in}{3.123916in}}%
\pgfpathlineto{\pgfqpoint{4.311568in}{3.136582in}}%
\pgfpathlineto{\pgfqpoint{4.315850in}{3.138914in}}%
\pgfpathlineto{\pgfqpoint{4.319329in}{3.138324in}}%
\pgfpathlineto{\pgfqpoint{4.323075in}{3.135297in}}%
\pgfpathlineto{\pgfqpoint{4.328695in}{3.127456in}}%
\pgfpathlineto{\pgfqpoint{4.338597in}{3.113652in}}%
\pgfpathlineto{\pgfqpoint{4.342611in}{3.111514in}}%
\pgfpathlineto{\pgfqpoint{4.346090in}{3.112065in}}%
\pgfpathlineto{\pgfqpoint{4.349836in}{3.115055in}}%
\pgfpathlineto{\pgfqpoint{4.355456in}{3.122867in}}%
\pgfpathlineto{\pgfqpoint{4.365357in}{3.136713in}}%
\pgfpathlineto{\pgfqpoint{4.369372in}{3.138894in}}%
\pgfpathlineto{\pgfqpoint{4.372850in}{3.138383in}}%
\pgfpathlineto{\pgfqpoint{4.376597in}{3.135430in}}%
\pgfpathlineto{\pgfqpoint{4.382217in}{3.127649in}}%
\pgfpathlineto{\pgfqpoint{4.392386in}{3.113526in}}%
\pgfpathlineto{\pgfqpoint{4.396400in}{3.111492in}}%
\pgfpathlineto{\pgfqpoint{4.399879in}{3.112137in}}%
\pgfpathlineto{\pgfqpoint{4.403893in}{3.115517in}}%
\pgfpathlineto{\pgfqpoint{4.409780in}{3.123956in}}%
\pgfpathlineto{\pgfqpoint{4.418879in}{3.136604in}}%
\pgfpathlineto{\pgfqpoint{4.423161in}{3.138918in}}%
\pgfpathlineto{\pgfqpoint{4.426640in}{3.138312in}}%
\pgfpathlineto{\pgfqpoint{4.430386in}{3.135271in}}%
\pgfpathlineto{\pgfqpoint{4.436006in}{3.127417in}}%
\pgfpathlineto{\pgfqpoint{4.445907in}{3.113631in}}%
\pgfpathlineto{\pgfqpoint{4.449921in}{3.111510in}}%
\pgfpathlineto{\pgfqpoint{4.453400in}{3.112077in}}%
\pgfpathlineto{\pgfqpoint{4.457147in}{3.115082in}}%
\pgfpathlineto{\pgfqpoint{4.462767in}{3.122905in}}%
\pgfpathlineto{\pgfqpoint{4.472668in}{3.136734in}}%
\pgfpathlineto{\pgfqpoint{4.476682in}{3.138898in}}%
\pgfpathlineto{\pgfqpoint{4.480161in}{3.138371in}}%
\pgfpathlineto{\pgfqpoint{4.483908in}{3.135403in}}%
\pgfpathlineto{\pgfqpoint{4.489527in}{3.127610in}}%
\pgfpathlineto{\pgfqpoint{4.499697in}{3.113505in}}%
\pgfpathlineto{\pgfqpoint{4.503711in}{3.111488in}}%
\pgfpathlineto{\pgfqpoint{4.507190in}{3.112150in}}%
\pgfpathlineto{\pgfqpoint{4.511204in}{3.115545in}}%
\pgfpathlineto{\pgfqpoint{4.517359in}{3.124425in}}%
\pgfpathlineto{\pgfqpoint{4.526190in}{3.136626in}}%
\pgfpathlineto{\pgfqpoint{4.530471in}{3.138922in}}%
\pgfpathlineto{\pgfqpoint{4.533950in}{3.138300in}}%
\pgfpathlineto{\pgfqpoint{4.537964in}{3.134943in}}%
\pgfpathlineto{\pgfqpoint{4.543852in}{3.126522in}}%
\pgfpathlineto{\pgfqpoint{4.552950in}{3.113848in}}%
\pgfpathlineto{\pgfqpoint{4.557232in}{3.111506in}}%
\pgfpathlineto{\pgfqpoint{4.560711in}{3.112089in}}%
\pgfpathlineto{\pgfqpoint{4.564458in}{3.115108in}}%
\pgfpathlineto{\pgfqpoint{4.570077in}{3.122944in}}%
\pgfpathlineto{\pgfqpoint{4.579979in}{3.136756in}}%
\pgfpathlineto{\pgfqpoint{4.583993in}{3.138903in}}%
\pgfpathlineto{\pgfqpoint{4.587472in}{3.138360in}}%
\pgfpathlineto{\pgfqpoint{4.591218in}{3.135377in}}%
\pgfpathlineto{\pgfqpoint{4.596838in}{3.127572in}}%
\pgfpathlineto{\pgfqpoint{4.606740in}{3.113717in}}%
\pgfpathlineto{\pgfqpoint{4.610754in}{3.111527in}}%
\pgfpathlineto{\pgfqpoint{4.614233in}{3.112031in}}%
\pgfpathlineto{\pgfqpoint{4.617979in}{3.114976in}}%
\pgfpathlineto{\pgfqpoint{4.623599in}{3.122751in}}%
\pgfpathlineto{\pgfqpoint{4.633768in}{3.136882in}}%
\pgfpathlineto{\pgfqpoint{4.637782in}{3.138925in}}%
\pgfpathlineto{\pgfqpoint{4.641261in}{3.138288in}}%
\pgfpathlineto{\pgfqpoint{4.645275in}{3.134915in}}%
\pgfpathlineto{\pgfqpoint{4.651163in}{3.126483in}}%
\pgfpathlineto{\pgfqpoint{4.660261in}{3.113826in}}%
\pgfpathlineto{\pgfqpoint{4.664543in}{3.111502in}}%
\pgfpathlineto{\pgfqpoint{4.668022in}{3.112101in}}%
\pgfpathlineto{\pgfqpoint{4.671768in}{3.115135in}}%
\pgfpathlineto{\pgfqpoint{4.677388in}{3.122983in}}%
\pgfpathlineto{\pgfqpoint{4.687290in}{3.136777in}}%
\pgfpathlineto{\pgfqpoint{4.691304in}{3.138907in}}%
\pgfpathlineto{\pgfqpoint{4.694783in}{3.138348in}}%
\pgfpathlineto{\pgfqpoint{4.698529in}{3.135351in}}%
\pgfpathlineto{\pgfqpoint{4.704149in}{3.127533in}}%
\pgfpathlineto{\pgfqpoint{4.714050in}{3.113695in}}%
\pgfpathlineto{\pgfqpoint{4.718065in}{3.111523in}}%
\pgfpathlineto{\pgfqpoint{4.721543in}{3.112042in}}%
\pgfpathlineto{\pgfqpoint{4.725290in}{3.115002in}}%
\pgfpathlineto{\pgfqpoint{4.730910in}{3.122789in}}%
\pgfpathlineto{\pgfqpoint{4.741079in}{3.136903in}}%
\pgfpathlineto{\pgfqpoint{4.745093in}{3.138929in}}%
\pgfpathlineto{\pgfqpoint{4.748572in}{3.138275in}}%
\pgfpathlineto{\pgfqpoint{4.752586in}{3.134888in}}%
\pgfpathlineto{\pgfqpoint{4.758741in}{3.126013in}}%
\pgfpathlineto{\pgfqpoint{4.767572in}{3.113804in}}%
\pgfpathlineto{\pgfqpoint{4.771854in}{3.111499in}}%
\pgfpathlineto{\pgfqpoint{4.775333in}{3.112113in}}%
\pgfpathlineto{\pgfqpoint{4.779347in}{3.115462in}}%
\pgfpathlineto{\pgfqpoint{4.785234in}{3.123877in}}%
\pgfpathlineto{\pgfqpoint{4.794333in}{3.136560in}}%
\pgfpathlineto{\pgfqpoint{4.798614in}{3.138911in}}%
\pgfpathlineto{\pgfqpoint{4.802093in}{3.138336in}}%
\pgfpathlineto{\pgfqpoint{4.805840in}{3.135324in}}%
\pgfpathlineto{\pgfqpoint{4.811460in}{3.127494in}}%
\pgfpathlineto{\pgfqpoint{4.821361in}{3.113674in}}%
\pgfpathlineto{\pgfqpoint{4.825375in}{3.111518in}}%
\pgfpathlineto{\pgfqpoint{4.828854in}{3.112054in}}%
\pgfpathlineto{\pgfqpoint{4.832601in}{3.115029in}}%
\pgfpathlineto{\pgfqpoint{4.838220in}{3.122828in}}%
\pgfpathlineto{\pgfqpoint{4.848122in}{3.136691in}}%
\pgfpathlineto{\pgfqpoint{4.852136in}{3.138890in}}%
\pgfpathlineto{\pgfqpoint{4.855615in}{3.138394in}}%
\pgfpathlineto{\pgfqpoint{4.859361in}{3.135456in}}%
\pgfpathlineto{\pgfqpoint{4.864981in}{3.127688in}}%
\pgfpathlineto{\pgfqpoint{4.875150in}{3.113547in}}%
\pgfpathlineto{\pgfqpoint{4.879164in}{3.111495in}}%
\pgfpathlineto{\pgfqpoint{4.882643in}{3.112125in}}%
\pgfpathlineto{\pgfqpoint{4.886657in}{3.115490in}}%
\pgfpathlineto{\pgfqpoint{4.892545in}{3.123916in}}%
\pgfpathlineto{\pgfqpoint{4.901643in}{3.136582in}}%
\pgfpathlineto{\pgfqpoint{4.905925in}{3.138914in}}%
\pgfpathlineto{\pgfqpoint{4.909404in}{3.138324in}}%
\pgfpathlineto{\pgfqpoint{4.913151in}{3.135297in}}%
\pgfpathlineto{\pgfqpoint{4.918770in}{3.127456in}}%
\pgfpathlineto{\pgfqpoint{4.928672in}{3.113652in}}%
\pgfpathlineto{\pgfqpoint{4.932686in}{3.111514in}}%
\pgfpathlineto{\pgfqpoint{4.936165in}{3.112065in}}%
\pgfpathlineto{\pgfqpoint{4.939911in}{3.115055in}}%
\pgfpathlineto{\pgfqpoint{4.945531in}{3.122867in}}%
\pgfpathlineto{\pgfqpoint{4.955433in}{3.136713in}}%
\pgfpathlineto{\pgfqpoint{4.959447in}{3.138894in}}%
\pgfpathlineto{\pgfqpoint{4.962926in}{3.138383in}}%
\pgfpathlineto{\pgfqpoint{4.966672in}{3.135430in}}%
\pgfpathlineto{\pgfqpoint{4.972292in}{3.127649in}}%
\pgfpathlineto{\pgfqpoint{4.982461in}{3.113526in}}%
\pgfpathlineto{\pgfqpoint{4.986475in}{3.111492in}}%
\pgfpathlineto{\pgfqpoint{4.989954in}{3.112137in}}%
\pgfpathlineto{\pgfqpoint{4.993968in}{3.115517in}}%
\pgfpathlineto{\pgfqpoint{4.999856in}{3.123956in}}%
\pgfpathlineto{\pgfqpoint{5.008954in}{3.136604in}}%
\pgfpathlineto{\pgfqpoint{5.013236in}{3.138918in}}%
\pgfpathlineto{\pgfqpoint{5.016715in}{3.138312in}}%
\pgfpathlineto{\pgfqpoint{5.020461in}{3.135271in}}%
\pgfpathlineto{\pgfqpoint{5.026081in}{3.127417in}}%
\pgfpathlineto{\pgfqpoint{5.035983in}{3.113631in}}%
\pgfpathlineto{\pgfqpoint{5.039997in}{3.111510in}}%
\pgfpathlineto{\pgfqpoint{5.043476in}{3.112077in}}%
\pgfpathlineto{\pgfqpoint{5.047222in}{3.115082in}}%
\pgfpathlineto{\pgfqpoint{5.052842in}{3.122905in}}%
\pgfpathlineto{\pgfqpoint{5.062743in}{3.136734in}}%
\pgfpathlineto{\pgfqpoint{5.066757in}{3.138898in}}%
\pgfpathlineto{\pgfqpoint{5.070236in}{3.138371in}}%
\pgfpathlineto{\pgfqpoint{5.073983in}{3.135403in}}%
\pgfpathlineto{\pgfqpoint{5.079603in}{3.127610in}}%
\pgfpathlineto{\pgfqpoint{5.089772in}{3.113505in}}%
\pgfpathlineto{\pgfqpoint{5.093786in}{3.111488in}}%
\pgfpathlineto{\pgfqpoint{5.097265in}{3.112150in}}%
\pgfpathlineto{\pgfqpoint{5.101279in}{3.115545in}}%
\pgfpathlineto{\pgfqpoint{5.107434in}{3.124425in}}%
\pgfpathlineto{\pgfqpoint{5.116265in}{3.136626in}}%
\pgfpathlineto{\pgfqpoint{5.120547in}{3.138922in}}%
\pgfpathlineto{\pgfqpoint{5.124026in}{3.138300in}}%
\pgfpathlineto{\pgfqpoint{5.128040in}{3.134943in}}%
\pgfpathlineto{\pgfqpoint{5.133927in}{3.126522in}}%
\pgfpathlineto{\pgfqpoint{5.143026in}{3.113848in}}%
\pgfpathlineto{\pgfqpoint{5.147307in}{3.111506in}}%
\pgfpathlineto{\pgfqpoint{5.150786in}{3.112089in}}%
\pgfpathlineto{\pgfqpoint{5.154533in}{3.115108in}}%
\pgfpathlineto{\pgfqpoint{5.160153in}{3.122944in}}%
\pgfpathlineto{\pgfqpoint{5.170054in}{3.136756in}}%
\pgfpathlineto{\pgfqpoint{5.174068in}{3.138903in}}%
\pgfpathlineto{\pgfqpoint{5.177547in}{3.138360in}}%
\pgfpathlineto{\pgfqpoint{5.181294in}{3.135377in}}%
\pgfpathlineto{\pgfqpoint{5.186913in}{3.127572in}}%
\pgfpathlineto{\pgfqpoint{5.196815in}{3.113717in}}%
\pgfpathlineto{\pgfqpoint{5.200829in}{3.111527in}}%
\pgfpathlineto{\pgfqpoint{5.204308in}{3.112031in}}%
\pgfpathlineto{\pgfqpoint{5.208054in}{3.114976in}}%
\pgfpathlineto{\pgfqpoint{5.213674in}{3.122751in}}%
\pgfpathlineto{\pgfqpoint{5.223843in}{3.136882in}}%
\pgfpathlineto{\pgfqpoint{5.227857in}{3.138925in}}%
\pgfpathlineto{\pgfqpoint{5.231336in}{3.138288in}}%
\pgfpathlineto{\pgfqpoint{5.235350in}{3.134915in}}%
\pgfpathlineto{\pgfqpoint{5.241238in}{3.126483in}}%
\pgfpathlineto{\pgfqpoint{5.250336in}{3.113826in}}%
\pgfpathlineto{\pgfqpoint{5.254618in}{3.111502in}}%
\pgfpathlineto{\pgfqpoint{5.258097in}{3.112101in}}%
\pgfpathlineto{\pgfqpoint{5.261844in}{3.115135in}}%
\pgfpathlineto{\pgfqpoint{5.267463in}{3.122983in}}%
\pgfpathlineto{\pgfqpoint{5.277365in}{3.136777in}}%
\pgfpathlineto{\pgfqpoint{5.281379in}{3.138907in}}%
\pgfpathlineto{\pgfqpoint{5.284858in}{3.138348in}}%
\pgfpathlineto{\pgfqpoint{5.288604in}{3.135351in}}%
\pgfpathlineto{\pgfqpoint{5.294224in}{3.127533in}}%
\pgfpathlineto{\pgfqpoint{5.304126in}{3.113695in}}%
\pgfpathlineto{\pgfqpoint{5.308140in}{3.111523in}}%
\pgfpathlineto{\pgfqpoint{5.311619in}{3.112042in}}%
\pgfpathlineto{\pgfqpoint{5.315365in}{3.115002in}}%
\pgfpathlineto{\pgfqpoint{5.320985in}{3.122789in}}%
\pgfpathlineto{\pgfqpoint{5.331154in}{3.136903in}}%
\pgfpathlineto{\pgfqpoint{5.335168in}{3.138929in}}%
\pgfpathlineto{\pgfqpoint{5.338647in}{3.138275in}}%
\pgfpathlineto{\pgfqpoint{5.342661in}{3.134888in}}%
\pgfpathlineto{\pgfqpoint{5.348816in}{3.126013in}}%
\pgfpathlineto{\pgfqpoint{5.357647in}{3.113804in}}%
\pgfpathlineto{\pgfqpoint{5.361929in}{3.111499in}}%
\pgfpathlineto{\pgfqpoint{5.365408in}{3.112113in}}%
\pgfpathlineto{\pgfqpoint{5.369422in}{3.115462in}}%
\pgfpathlineto{\pgfqpoint{5.375309in}{3.123877in}}%
\pgfpathlineto{\pgfqpoint{5.384408in}{3.136560in}}%
\pgfpathlineto{\pgfqpoint{5.388690in}{3.138911in}}%
\pgfpathlineto{\pgfqpoint{5.392169in}{3.138336in}}%
\pgfpathlineto{\pgfqpoint{5.395915in}{3.135324in}}%
\pgfpathlineto{\pgfqpoint{5.401535in}{3.127494in}}%
\pgfpathlineto{\pgfqpoint{5.411436in}{3.113674in}}%
\pgfpathlineto{\pgfqpoint{5.415450in}{3.111518in}}%
\pgfpathlineto{\pgfqpoint{5.418929in}{3.112054in}}%
\pgfpathlineto{\pgfqpoint{5.422676in}{3.115029in}}%
\pgfpathlineto{\pgfqpoint{5.428296in}{3.122828in}}%
\pgfpathlineto{\pgfqpoint{5.438197in}{3.136691in}}%
\pgfpathlineto{\pgfqpoint{5.442211in}{3.138890in}}%
\pgfpathlineto{\pgfqpoint{5.445690in}{3.138394in}}%
\pgfpathlineto{\pgfqpoint{5.449437in}{3.135456in}}%
\pgfpathlineto{\pgfqpoint{5.455056in}{3.127688in}}%
\pgfpathlineto{\pgfqpoint{5.465226in}{3.113547in}}%
\pgfpathlineto{\pgfqpoint{5.469240in}{3.111495in}}%
\pgfpathlineto{\pgfqpoint{5.472719in}{3.112125in}}%
\pgfpathlineto{\pgfqpoint{5.476733in}{3.115490in}}%
\pgfpathlineto{\pgfqpoint{5.482620in}{3.123916in}}%
\pgfpathlineto{\pgfqpoint{5.491719in}{3.136582in}}%
\pgfpathlineto{\pgfqpoint{5.496000in}{3.138914in}}%
\pgfpathlineto{\pgfqpoint{5.499479in}{3.138324in}}%
\pgfpathlineto{\pgfqpoint{5.503226in}{3.135297in}}%
\pgfpathlineto{\pgfqpoint{5.508846in}{3.127456in}}%
\pgfpathlineto{\pgfqpoint{5.518747in}{3.113652in}}%
\pgfpathlineto{\pgfqpoint{5.522761in}{3.111514in}}%
\pgfpathlineto{\pgfqpoint{5.526240in}{3.112065in}}%
\pgfpathlineto{\pgfqpoint{5.529987in}{3.115055in}}%
\pgfpathlineto{\pgfqpoint{5.535606in}{3.122867in}}%
\pgfpathlineto{\pgfqpoint{5.545508in}{3.136713in}}%
\pgfpathlineto{\pgfqpoint{5.549522in}{3.138894in}}%
\pgfpathlineto{\pgfqpoint{5.553001in}{3.138383in}}%
\pgfpathlineto{\pgfqpoint{5.556747in}{3.135430in}}%
\pgfpathlineto{\pgfqpoint{5.562367in}{3.127649in}}%
\pgfpathlineto{\pgfqpoint{5.572536in}{3.113526in}}%
\pgfpathlineto{\pgfqpoint{5.576550in}{3.111492in}}%
\pgfpathlineto{\pgfqpoint{5.580029in}{3.112137in}}%
\pgfpathlineto{\pgfqpoint{5.584043in}{3.115517in}}%
\pgfpathlineto{\pgfqpoint{5.589931in}{3.123956in}}%
\pgfpathlineto{\pgfqpoint{5.599029in}{3.136604in}}%
\pgfpathlineto{\pgfqpoint{5.603311in}{3.138918in}}%
\pgfpathlineto{\pgfqpoint{5.606790in}{3.138312in}}%
\pgfpathlineto{\pgfqpoint{5.610537in}{3.135271in}}%
\pgfpathlineto{\pgfqpoint{5.616156in}{3.127417in}}%
\pgfpathlineto{\pgfqpoint{5.626058in}{3.113631in}}%
\pgfpathlineto{\pgfqpoint{5.630072in}{3.111510in}}%
\pgfpathlineto{\pgfqpoint{5.633551in}{3.112077in}}%
\pgfpathlineto{\pgfqpoint{5.637297in}{3.115082in}}%
\pgfpathlineto{\pgfqpoint{5.642917in}{3.122905in}}%
\pgfpathlineto{\pgfqpoint{5.652819in}{3.136734in}}%
\pgfpathlineto{\pgfqpoint{5.656833in}{3.138898in}}%
\pgfpathlineto{\pgfqpoint{5.660312in}{3.138371in}}%
\pgfpathlineto{\pgfqpoint{5.664058in}{3.135403in}}%
\pgfpathlineto{\pgfqpoint{5.669678in}{3.127610in}}%
\pgfpathlineto{\pgfqpoint{5.679847in}{3.113505in}}%
\pgfpathlineto{\pgfqpoint{5.683861in}{3.111488in}}%
\pgfpathlineto{\pgfqpoint{5.687340in}{3.112150in}}%
\pgfpathlineto{\pgfqpoint{5.691354in}{3.115545in}}%
\pgfpathlineto{\pgfqpoint{5.697509in}{3.124425in}}%
\pgfpathlineto{\pgfqpoint{5.706340in}{3.136626in}}%
\pgfpathlineto{\pgfqpoint{5.710622in}{3.138922in}}%
\pgfpathlineto{\pgfqpoint{5.714101in}{3.138300in}}%
\pgfpathlineto{\pgfqpoint{5.718115in}{3.134943in}}%
\pgfpathlineto{\pgfqpoint{5.724002in}{3.126522in}}%
\pgfpathlineto{\pgfqpoint{5.733101in}{3.113848in}}%
\pgfpathlineto{\pgfqpoint{5.737383in}{3.111506in}}%
\pgfpathlineto{\pgfqpoint{5.740862in}{3.112089in}}%
\pgfpathlineto{\pgfqpoint{5.744608in}{3.115108in}}%
\pgfpathlineto{\pgfqpoint{5.750228in}{3.122944in}}%
\pgfpathlineto{\pgfqpoint{5.760129in}{3.136756in}}%
\pgfpathlineto{\pgfqpoint{5.764143in}{3.138903in}}%
\pgfpathlineto{\pgfqpoint{5.767622in}{3.138360in}}%
\pgfpathlineto{\pgfqpoint{5.771369in}{3.135377in}}%
\pgfpathlineto{\pgfqpoint{5.776989in}{3.127572in}}%
\pgfpathlineto{\pgfqpoint{5.786890in}{3.113717in}}%
\pgfpathlineto{\pgfqpoint{5.790904in}{3.111527in}}%
\pgfpathlineto{\pgfqpoint{5.794383in}{3.112031in}}%
\pgfpathlineto{\pgfqpoint{5.798130in}{3.114976in}}%
\pgfpathlineto{\pgfqpoint{5.803749in}{3.122751in}}%
\pgfpathlineto{\pgfqpoint{5.813919in}{3.136882in}}%
\pgfpathlineto{\pgfqpoint{5.817933in}{3.138925in}}%
\pgfpathlineto{\pgfqpoint{5.821412in}{3.138288in}}%
\pgfpathlineto{\pgfqpoint{5.825426in}{3.134915in}}%
\pgfpathlineto{\pgfqpoint{5.831313in}{3.126483in}}%
\pgfpathlineto{\pgfqpoint{5.840412in}{3.113826in}}%
\pgfpathlineto{\pgfqpoint{5.844693in}{3.111502in}}%
\pgfpathlineto{\pgfqpoint{5.848172in}{3.112101in}}%
\pgfpathlineto{\pgfqpoint{5.851919in}{3.115135in}}%
\pgfpathlineto{\pgfqpoint{5.857539in}{3.122983in}}%
\pgfpathlineto{\pgfqpoint{5.867440in}{3.136777in}}%
\pgfpathlineto{\pgfqpoint{5.871454in}{3.138907in}}%
\pgfpathlineto{\pgfqpoint{5.874933in}{3.138348in}}%
\pgfpathlineto{\pgfqpoint{5.878680in}{3.135351in}}%
\pgfpathlineto{\pgfqpoint{5.884299in}{3.127533in}}%
\pgfpathlineto{\pgfqpoint{5.894201in}{3.113695in}}%
\pgfpathlineto{\pgfqpoint{5.898215in}{3.111523in}}%
\pgfpathlineto{\pgfqpoint{5.901694in}{3.112042in}}%
\pgfpathlineto{\pgfqpoint{5.905440in}{3.115002in}}%
\pgfpathlineto{\pgfqpoint{5.911060in}{3.122789in}}%
\pgfpathlineto{\pgfqpoint{5.921229in}{3.136903in}}%
\pgfpathlineto{\pgfqpoint{5.925243in}{3.138929in}}%
\pgfpathlineto{\pgfqpoint{5.928722in}{3.138275in}}%
\pgfpathlineto{\pgfqpoint{5.932736in}{3.134888in}}%
\pgfpathlineto{\pgfqpoint{5.938891in}{3.126013in}}%
\pgfpathlineto{\pgfqpoint{5.947722in}{3.113804in}}%
\pgfpathlineto{\pgfqpoint{5.952004in}{3.111499in}}%
\pgfpathlineto{\pgfqpoint{5.955483in}{3.112113in}}%
\pgfpathlineto{\pgfqpoint{5.959497in}{3.115462in}}%
\pgfpathlineto{\pgfqpoint{5.965385in}{3.123877in}}%
\pgfpathlineto{\pgfqpoint{5.974483in}{3.136560in}}%
\pgfpathlineto{\pgfqpoint{5.978765in}{3.138911in}}%
\pgfpathlineto{\pgfqpoint{5.982244in}{3.138336in}}%
\pgfpathlineto{\pgfqpoint{5.985990in}{3.135324in}}%
\pgfpathlineto{\pgfqpoint{5.991610in}{3.127494in}}%
\pgfpathlineto{\pgfqpoint{6.001512in}{3.113674in}}%
\pgfpathlineto{\pgfqpoint{6.005526in}{3.111518in}}%
\pgfpathlineto{\pgfqpoint{6.009005in}{3.112054in}}%
\pgfpathlineto{\pgfqpoint{6.012751in}{3.115029in}}%
\pgfpathlineto{\pgfqpoint{6.018371in}{3.122828in}}%
\pgfpathlineto{\pgfqpoint{6.028272in}{3.136691in}}%
\pgfpathlineto{\pgfqpoint{6.032287in}{3.138890in}}%
\pgfpathlineto{\pgfqpoint{6.035765in}{3.138394in}}%
\pgfpathlineto{\pgfqpoint{6.039512in}{3.135456in}}%
\pgfpathlineto{\pgfqpoint{6.045132in}{3.127688in}}%
\pgfpathlineto{\pgfqpoint{6.055301in}{3.113547in}}%
\pgfpathlineto{\pgfqpoint{6.059315in}{3.111495in}}%
\pgfpathlineto{\pgfqpoint{6.062794in}{3.112125in}}%
\pgfpathlineto{\pgfqpoint{6.066808in}{3.115490in}}%
\pgfpathlineto{\pgfqpoint{6.072695in}{3.123916in}}%
\pgfpathlineto{\pgfqpoint{6.081794in}{3.136582in}}%
\pgfpathlineto{\pgfqpoint{6.086076in}{3.138914in}}%
\pgfpathlineto{\pgfqpoint{6.089555in}{3.138324in}}%
\pgfpathlineto{\pgfqpoint{6.093301in}{3.135297in}}%
\pgfpathlineto{\pgfqpoint{6.098921in}{3.127456in}}%
\pgfpathlineto{\pgfqpoint{6.108822in}{3.113652in}}%
\pgfpathlineto{\pgfqpoint{6.112836in}{3.111514in}}%
\pgfpathlineto{\pgfqpoint{6.116315in}{3.112065in}}%
\pgfpathlineto{\pgfqpoint{6.120062in}{3.115055in}}%
\pgfpathlineto{\pgfqpoint{6.125682in}{3.122867in}}%
\pgfpathlineto{\pgfqpoint{6.135583in}{3.136713in}}%
\pgfpathlineto{\pgfqpoint{6.139597in}{3.138894in}}%
\pgfpathlineto{\pgfqpoint{6.143076in}{3.138383in}}%
\pgfpathlineto{\pgfqpoint{6.146823in}{3.135430in}}%
\pgfpathlineto{\pgfqpoint{6.152442in}{3.127649in}}%
\pgfpathlineto{\pgfqpoint{6.162612in}{3.113526in}}%
\pgfpathlineto{\pgfqpoint{6.166626in}{3.111492in}}%
\pgfpathlineto{\pgfqpoint{6.170105in}{3.112137in}}%
\pgfpathlineto{\pgfqpoint{6.174119in}{3.115517in}}%
\pgfpathlineto{\pgfqpoint{6.180006in}{3.123956in}}%
\pgfpathlineto{\pgfqpoint{6.189105in}{3.136604in}}%
\pgfpathlineto{\pgfqpoint{6.193386in}{3.138918in}}%
\pgfpathlineto{\pgfqpoint{6.196865in}{3.138312in}}%
\pgfpathlineto{\pgfqpoint{6.200612in}{3.135271in}}%
\pgfpathlineto{\pgfqpoint{6.206232in}{3.127417in}}%
\pgfpathlineto{\pgfqpoint{6.216133in}{3.113631in}}%
\pgfpathlineto{\pgfqpoint{6.220147in}{3.111510in}}%
\pgfpathlineto{\pgfqpoint{6.223626in}{3.112077in}}%
\pgfpathlineto{\pgfqpoint{6.227373in}{3.115082in}}%
\pgfpathlineto{\pgfqpoint{6.232992in}{3.122905in}}%
\pgfpathlineto{\pgfqpoint{6.242894in}{3.136734in}}%
\pgfpathlineto{\pgfqpoint{6.246908in}{3.138898in}}%
\pgfpathlineto{\pgfqpoint{6.250387in}{3.138371in}}%
\pgfpathlineto{\pgfqpoint{6.254133in}{3.135403in}}%
\pgfpathlineto{\pgfqpoint{6.259753in}{3.127610in}}%
\pgfpathlineto{\pgfqpoint{6.269922in}{3.113505in}}%
\pgfpathlineto{\pgfqpoint{6.273936in}{3.111488in}}%
\pgfpathlineto{\pgfqpoint{6.277415in}{3.112150in}}%
\pgfpathlineto{\pgfqpoint{6.281429in}{3.115545in}}%
\pgfpathlineto{\pgfqpoint{6.287584in}{3.124425in}}%
\pgfpathlineto{\pgfqpoint{6.296415in}{3.136626in}}%
\pgfpathlineto{\pgfqpoint{6.300697in}{3.138922in}}%
\pgfpathlineto{\pgfqpoint{6.304176in}{3.138300in}}%
\pgfpathlineto{\pgfqpoint{6.308190in}{3.134943in}}%
\pgfpathlineto{\pgfqpoint{6.314078in}{3.126522in}}%
\pgfpathlineto{\pgfqpoint{6.323176in}{3.113848in}}%
\pgfpathlineto{\pgfqpoint{6.327458in}{3.111506in}}%
\pgfpathlineto{\pgfqpoint{6.330937in}{3.112089in}}%
\pgfpathlineto{\pgfqpoint{6.334683in}{3.115108in}}%
\pgfpathlineto{\pgfqpoint{6.340303in}{3.122944in}}%
\pgfpathlineto{\pgfqpoint{6.350205in}{3.136756in}}%
\pgfpathlineto{\pgfqpoint{6.354219in}{3.138903in}}%
\pgfpathlineto{\pgfqpoint{6.357698in}{3.138360in}}%
\pgfpathlineto{\pgfqpoint{6.361444in}{3.135377in}}%
\pgfpathlineto{\pgfqpoint{6.367064in}{3.127572in}}%
\pgfpathlineto{\pgfqpoint{6.376965in}{3.113717in}}%
\pgfpathlineto{\pgfqpoint{6.380980in}{3.111527in}}%
\pgfpathlineto{\pgfqpoint{6.384458in}{3.112031in}}%
\pgfpathlineto{\pgfqpoint{6.388205in}{3.114976in}}%
\pgfpathlineto{\pgfqpoint{6.393825in}{3.122751in}}%
\pgfpathlineto{\pgfqpoint{6.403994in}{3.136882in}}%
\pgfpathlineto{\pgfqpoint{6.408008in}{3.138925in}}%
\pgfpathlineto{\pgfqpoint{6.411487in}{3.138288in}}%
\pgfpathlineto{\pgfqpoint{6.415501in}{3.134915in}}%
\pgfpathlineto{\pgfqpoint{6.421388in}{3.126483in}}%
\pgfpathlineto{\pgfqpoint{6.430487in}{3.113826in}}%
\pgfpathlineto{\pgfqpoint{6.434769in}{3.111502in}}%
\pgfpathlineto{\pgfqpoint{6.438248in}{3.112101in}}%
\pgfpathlineto{\pgfqpoint{6.441994in}{3.115135in}}%
\pgfpathlineto{\pgfqpoint{6.447614in}{3.122983in}}%
\pgfpathlineto{\pgfqpoint{6.457515in}{3.136777in}}%
\pgfpathlineto{\pgfqpoint{6.461529in}{3.138907in}}%
\pgfpathlineto{\pgfqpoint{6.465008in}{3.138348in}}%
\pgfpathlineto{\pgfqpoint{6.468755in}{3.135351in}}%
\pgfpathlineto{\pgfqpoint{6.474375in}{3.127533in}}%
\pgfpathlineto{\pgfqpoint{6.484276in}{3.113695in}}%
\pgfpathlineto{\pgfqpoint{6.488290in}{3.111523in}}%
\pgfpathlineto{\pgfqpoint{6.491769in}{3.112042in}}%
\pgfpathlineto{\pgfqpoint{6.495516in}{3.115002in}}%
\pgfpathlineto{\pgfqpoint{6.501135in}{3.122789in}}%
\pgfpathlineto{\pgfqpoint{6.511305in}{3.136903in}}%
\pgfpathlineto{\pgfqpoint{6.515319in}{3.138929in}}%
\pgfpathlineto{\pgfqpoint{6.518798in}{3.138275in}}%
\pgfpathlineto{\pgfqpoint{6.522812in}{3.134888in}}%
\pgfpathlineto{\pgfqpoint{6.528967in}{3.126013in}}%
\pgfpathlineto{\pgfqpoint{6.537798in}{3.113804in}}%
\pgfpathlineto{\pgfqpoint{6.542079in}{3.111499in}}%
\pgfpathlineto{\pgfqpoint{6.545558in}{3.112113in}}%
\pgfpathlineto{\pgfqpoint{6.549572in}{3.115462in}}%
\pgfpathlineto{\pgfqpoint{6.555460in}{3.123877in}}%
\pgfpathlineto{\pgfqpoint{6.564558in}{3.136560in}}%
\pgfpathlineto{\pgfqpoint{6.568840in}{3.138911in}}%
\pgfpathlineto{\pgfqpoint{6.572319in}{3.138336in}}%
\pgfpathlineto{\pgfqpoint{6.576066in}{3.135324in}}%
\pgfpathlineto{\pgfqpoint{6.581685in}{3.127494in}}%
\pgfpathlineto{\pgfqpoint{6.591587in}{3.113674in}}%
\pgfpathlineto{\pgfqpoint{6.595601in}{3.111518in}}%
\pgfpathlineto{\pgfqpoint{6.599080in}{3.112054in}}%
\pgfpathlineto{\pgfqpoint{6.602826in}{3.115029in}}%
\pgfpathlineto{\pgfqpoint{6.608446in}{3.122828in}}%
\pgfpathlineto{\pgfqpoint{6.618348in}{3.136691in}}%
\pgfpathlineto{\pgfqpoint{6.622362in}{3.138890in}}%
\pgfpathlineto{\pgfqpoint{6.625841in}{3.138394in}}%
\pgfpathlineto{\pgfqpoint{6.629587in}{3.135456in}}%
\pgfpathlineto{\pgfqpoint{6.635207in}{3.127688in}}%
\pgfpathlineto{\pgfqpoint{6.645376in}{3.113547in}}%
\pgfpathlineto{\pgfqpoint{6.649390in}{3.111495in}}%
\pgfpathlineto{\pgfqpoint{6.652869in}{3.112125in}}%
\pgfpathlineto{\pgfqpoint{6.656883in}{3.115490in}}%
\pgfpathlineto{\pgfqpoint{6.662771in}{3.123916in}}%
\pgfpathlineto{\pgfqpoint{6.663306in}{3.124778in}}%
\pgfpathlineto{\pgfqpoint{6.663306in}{3.124778in}}%
\pgfusepath{stroke}%
\end{pgfscope}%
\begin{pgfscope}%
\pgfpathrectangle{\pgfqpoint{0.467797in}{2.292089in}}{\pgfqpoint{6.490533in}{1.666241in}}%
\pgfusepath{clip}%
\pgfsetrectcap%
\pgfsetroundjoin%
\pgfsetlinewidth{1.505625pt}%
\definecolor{currentstroke}{rgb}{0.737255,0.741176,0.133333}%
\pgfsetstrokecolor{currentstroke}%
\pgfsetdash{}{0pt}%
\pgfpathmoveto{\pgfqpoint{0.762821in}{3.125209in}}%
\pgfpathlineto{\pgfqpoint{0.770849in}{3.136204in}}%
\pgfpathlineto{\pgfqpoint{0.774863in}{3.138418in}}%
\pgfpathlineto{\pgfqpoint{0.778342in}{3.137850in}}%
\pgfpathlineto{\pgfqpoint{0.782089in}{3.134766in}}%
\pgfpathlineto{\pgfqpoint{0.787976in}{3.126378in}}%
\pgfpathlineto{\pgfqpoint{0.796807in}{3.114145in}}%
\pgfpathlineto{\pgfqpoint{0.800821in}{3.111988in}}%
\pgfpathlineto{\pgfqpoint{0.804300in}{3.112608in}}%
\pgfpathlineto{\pgfqpoint{0.808047in}{3.115740in}}%
\pgfpathlineto{\pgfqpoint{0.813934in}{3.124165in}}%
\pgfpathlineto{\pgfqpoint{0.822765in}{3.136342in}}%
\pgfpathlineto{\pgfqpoint{0.826779in}{3.138443in}}%
\pgfpathlineto{\pgfqpoint{0.830258in}{3.137771in}}%
\pgfpathlineto{\pgfqpoint{0.834005in}{3.134591in}}%
\pgfpathlineto{\pgfqpoint{0.839892in}{3.126129in}}%
\pgfpathlineto{\pgfqpoint{0.848723in}{3.114009in}}%
\pgfpathlineto{\pgfqpoint{0.852737in}{3.111965in}}%
\pgfpathlineto{\pgfqpoint{0.856216in}{3.112689in}}%
\pgfpathlineto{\pgfqpoint{0.860230in}{3.116230in}}%
\pgfpathlineto{\pgfqpoint{0.866653in}{3.125709in}}%
\pgfpathlineto{\pgfqpoint{0.874413in}{3.136242in}}%
\pgfpathlineto{\pgfqpoint{0.878427in}{3.138425in}}%
\pgfpathlineto{\pgfqpoint{0.881906in}{3.137829in}}%
\pgfpathlineto{\pgfqpoint{0.885653in}{3.134719in}}%
\pgfpathlineto{\pgfqpoint{0.891540in}{3.126310in}}%
\pgfpathlineto{\pgfqpoint{0.900371in}{3.114108in}}%
\pgfpathlineto{\pgfqpoint{0.904385in}{3.111981in}}%
\pgfpathlineto{\pgfqpoint{0.907864in}{3.112630in}}%
\pgfpathlineto{\pgfqpoint{0.911611in}{3.115788in}}%
\pgfpathlineto{\pgfqpoint{0.917498in}{3.124233in}}%
\pgfpathlineto{\pgfqpoint{0.926329in}{3.136379in}}%
\pgfpathlineto{\pgfqpoint{0.930343in}{3.138449in}}%
\pgfpathlineto{\pgfqpoint{0.933822in}{3.137748in}}%
\pgfpathlineto{\pgfqpoint{0.937569in}{3.134543in}}%
\pgfpathlineto{\pgfqpoint{0.943456in}{3.126061in}}%
\pgfpathlineto{\pgfqpoint{0.952020in}{3.114209in}}%
\pgfpathlineto{\pgfqpoint{0.956034in}{3.112000in}}%
\pgfpathlineto{\pgfqpoint{0.959513in}{3.112572in}}%
\pgfpathlineto{\pgfqpoint{0.963259in}{3.115660in}}%
\pgfpathlineto{\pgfqpoint{0.969147in}{3.124052in}}%
\pgfpathlineto{\pgfqpoint{0.977978in}{3.136280in}}%
\pgfpathlineto{\pgfqpoint{0.981992in}{3.138432in}}%
\pgfpathlineto{\pgfqpoint{0.985471in}{3.137807in}}%
\pgfpathlineto{\pgfqpoint{0.989217in}{3.134671in}}%
\pgfpathlineto{\pgfqpoint{0.995104in}{3.126242in}}%
\pgfpathlineto{\pgfqpoint{1.003936in}{3.114070in}}%
\pgfpathlineto{\pgfqpoint{1.007950in}{3.111975in}}%
\pgfpathlineto{\pgfqpoint{1.011429in}{3.112652in}}%
\pgfpathlineto{\pgfqpoint{1.015175in}{3.115836in}}%
\pgfpathlineto{\pgfqpoint{1.021062in}{3.124301in}}%
\pgfpathlineto{\pgfqpoint{1.029626in}{3.136178in}}%
\pgfpathlineto{\pgfqpoint{1.033640in}{3.138413in}}%
\pgfpathlineto{\pgfqpoint{1.037119in}{3.137864in}}%
\pgfpathlineto{\pgfqpoint{1.040865in}{3.134798in}}%
\pgfpathlineto{\pgfqpoint{1.046485in}{3.126852in}}%
\pgfpathlineto{\pgfqpoint{1.055584in}{3.114170in}}%
\pgfpathlineto{\pgfqpoint{1.059598in}{3.111992in}}%
\pgfpathlineto{\pgfqpoint{1.063077in}{3.112594in}}%
\pgfpathlineto{\pgfqpoint{1.066823in}{3.115708in}}%
\pgfpathlineto{\pgfqpoint{1.072711in}{3.124120in}}%
\pgfpathlineto{\pgfqpoint{1.081542in}{3.136317in}}%
\pgfpathlineto{\pgfqpoint{1.085556in}{3.138439in}}%
\pgfpathlineto{\pgfqpoint{1.089035in}{3.137786in}}%
\pgfpathlineto{\pgfqpoint{1.092781in}{3.134623in}}%
\pgfpathlineto{\pgfqpoint{1.098669in}{3.126174in}}%
\pgfpathlineto{\pgfqpoint{1.107500in}{3.114033in}}%
\pgfpathlineto{\pgfqpoint{1.111514in}{3.111969in}}%
\pgfpathlineto{\pgfqpoint{1.114993in}{3.112674in}}%
\pgfpathlineto{\pgfqpoint{1.118739in}{3.115884in}}%
\pgfpathlineto{\pgfqpoint{1.124627in}{3.124369in}}%
\pgfpathlineto{\pgfqpoint{1.133190in}{3.136217in}}%
\pgfpathlineto{\pgfqpoint{1.137204in}{3.138420in}}%
\pgfpathlineto{\pgfqpoint{1.140683in}{3.137843in}}%
\pgfpathlineto{\pgfqpoint{1.144430in}{3.134750in}}%
\pgfpathlineto{\pgfqpoint{1.150317in}{3.126355in}}%
\pgfpathlineto{\pgfqpoint{1.159148in}{3.114133in}}%
\pgfpathlineto{\pgfqpoint{1.163162in}{3.111986in}}%
\pgfpathlineto{\pgfqpoint{1.166641in}{3.112615in}}%
\pgfpathlineto{\pgfqpoint{1.170388in}{3.115756in}}%
\pgfpathlineto{\pgfqpoint{1.176275in}{3.124188in}}%
\pgfpathlineto{\pgfqpoint{1.185106in}{3.136355in}}%
\pgfpathlineto{\pgfqpoint{1.189120in}{3.138445in}}%
\pgfpathlineto{\pgfqpoint{1.192599in}{3.137763in}}%
\pgfpathlineto{\pgfqpoint{1.196346in}{3.134575in}}%
\pgfpathlineto{\pgfqpoint{1.202233in}{3.126106in}}%
\pgfpathlineto{\pgfqpoint{1.210796in}{3.114234in}}%
\pgfpathlineto{\pgfqpoint{1.214811in}{3.112005in}}%
\pgfpathlineto{\pgfqpoint{1.218289in}{3.112558in}}%
\pgfpathlineto{\pgfqpoint{1.222036in}{3.115629in}}%
\pgfpathlineto{\pgfqpoint{1.227656in}{3.123578in}}%
\pgfpathlineto{\pgfqpoint{1.236754in}{3.136255in}}%
\pgfpathlineto{\pgfqpoint{1.240768in}{3.138428in}}%
\pgfpathlineto{\pgfqpoint{1.244247in}{3.137822in}}%
\pgfpathlineto{\pgfqpoint{1.247994in}{3.134703in}}%
\pgfpathlineto{\pgfqpoint{1.253881in}{3.126287in}}%
\pgfpathlineto{\pgfqpoint{1.262712in}{3.114095in}}%
\pgfpathlineto{\pgfqpoint{1.266726in}{3.111979in}}%
\pgfpathlineto{\pgfqpoint{1.270205in}{3.112637in}}%
\pgfpathlineto{\pgfqpoint{1.273952in}{3.115804in}}%
\pgfpathlineto{\pgfqpoint{1.279839in}{3.124256in}}%
\pgfpathlineto{\pgfqpoint{1.288670in}{3.136392in}}%
\pgfpathlineto{\pgfqpoint{1.292684in}{3.138451in}}%
\pgfpathlineto{\pgfqpoint{1.296163in}{3.137741in}}%
\pgfpathlineto{\pgfqpoint{1.299910in}{3.134526in}}%
\pgfpathlineto{\pgfqpoint{1.305797in}{3.126038in}}%
\pgfpathlineto{\pgfqpoint{1.314361in}{3.114196in}}%
\pgfpathlineto{\pgfqpoint{1.318375in}{3.111997in}}%
\pgfpathlineto{\pgfqpoint{1.321854in}{3.112579in}}%
\pgfpathlineto{\pgfqpoint{1.325600in}{3.115676in}}%
\pgfpathlineto{\pgfqpoint{1.331488in}{3.124075in}}%
\pgfpathlineto{\pgfqpoint{1.340319in}{3.136292in}}%
\pgfpathlineto{\pgfqpoint{1.344333in}{3.138434in}}%
\pgfpathlineto{\pgfqpoint{1.347812in}{3.137800in}}%
\pgfpathlineto{\pgfqpoint{1.351558in}{3.134655in}}%
\pgfpathlineto{\pgfqpoint{1.357445in}{3.126219in}}%
\pgfpathlineto{\pgfqpoint{1.366277in}{3.114058in}}%
\pgfpathlineto{\pgfqpoint{1.370291in}{3.111973in}}%
\pgfpathlineto{\pgfqpoint{1.373770in}{3.112659in}}%
\pgfpathlineto{\pgfqpoint{1.377516in}{3.115852in}}%
\pgfpathlineto{\pgfqpoint{1.383403in}{3.124324in}}%
\pgfpathlineto{\pgfqpoint{1.391967in}{3.136191in}}%
\pgfpathlineto{\pgfqpoint{1.395981in}{3.138416in}}%
\pgfpathlineto{\pgfqpoint{1.399460in}{3.137857in}}%
\pgfpathlineto{\pgfqpoint{1.403206in}{3.134782in}}%
\pgfpathlineto{\pgfqpoint{1.408826in}{3.126830in}}%
\pgfpathlineto{\pgfqpoint{1.417925in}{3.114158in}}%
\pgfpathlineto{\pgfqpoint{1.421939in}{3.111990in}}%
\pgfpathlineto{\pgfqpoint{1.425418in}{3.112601in}}%
\pgfpathlineto{\pgfqpoint{1.429164in}{3.115724in}}%
\pgfpathlineto{\pgfqpoint{1.435052in}{3.124143in}}%
\pgfpathlineto{\pgfqpoint{1.443883in}{3.136330in}}%
\pgfpathlineto{\pgfqpoint{1.447897in}{3.138441in}}%
\pgfpathlineto{\pgfqpoint{1.451376in}{3.137778in}}%
\pgfpathlineto{\pgfqpoint{1.455122in}{3.134607in}}%
\pgfpathlineto{\pgfqpoint{1.461010in}{3.126151in}}%
\pgfpathlineto{\pgfqpoint{1.469841in}{3.114021in}}%
\pgfpathlineto{\pgfqpoint{1.473855in}{3.111967in}}%
\pgfpathlineto{\pgfqpoint{1.477334in}{3.112682in}}%
\pgfpathlineto{\pgfqpoint{1.481080in}{3.115901in}}%
\pgfpathlineto{\pgfqpoint{1.486968in}{3.124392in}}%
\pgfpathlineto{\pgfqpoint{1.495531in}{3.136229in}}%
\pgfpathlineto{\pgfqpoint{1.499545in}{3.138423in}}%
\pgfpathlineto{\pgfqpoint{1.503024in}{3.137836in}}%
\pgfpathlineto{\pgfqpoint{1.506771in}{3.134735in}}%
\pgfpathlineto{\pgfqpoint{1.512658in}{3.126333in}}%
\pgfpathlineto{\pgfqpoint{1.521489in}{3.114120in}}%
\pgfpathlineto{\pgfqpoint{1.525503in}{3.111983in}}%
\pgfpathlineto{\pgfqpoint{1.528982in}{3.112622in}}%
\pgfpathlineto{\pgfqpoint{1.532729in}{3.115772in}}%
\pgfpathlineto{\pgfqpoint{1.538616in}{3.124211in}}%
\pgfpathlineto{\pgfqpoint{1.547447in}{3.136367in}}%
\pgfpathlineto{\pgfqpoint{1.551461in}{3.138447in}}%
\pgfpathlineto{\pgfqpoint{1.554940in}{3.137756in}}%
\pgfpathlineto{\pgfqpoint{1.558687in}{3.134559in}}%
\pgfpathlineto{\pgfqpoint{1.564574in}{3.126083in}}%
\pgfpathlineto{\pgfqpoint{1.573137in}{3.114221in}}%
\pgfpathlineto{\pgfqpoint{1.577152in}{3.112002in}}%
\pgfpathlineto{\pgfqpoint{1.580630in}{3.112565in}}%
\pgfpathlineto{\pgfqpoint{1.584377in}{3.115645in}}%
\pgfpathlineto{\pgfqpoint{1.589997in}{3.123600in}}%
\pgfpathlineto{\pgfqpoint{1.599095in}{3.136267in}}%
\pgfpathlineto{\pgfqpoint{1.603109in}{3.138430in}}%
\pgfpathlineto{\pgfqpoint{1.606588in}{3.137815in}}%
\pgfpathlineto{\pgfqpoint{1.610335in}{3.134687in}}%
\pgfpathlineto{\pgfqpoint{1.616222in}{3.126265in}}%
\pgfpathlineto{\pgfqpoint{1.625053in}{3.114083in}}%
\pgfpathlineto{\pgfqpoint{1.629067in}{3.111977in}}%
\pgfpathlineto{\pgfqpoint{1.632546in}{3.112644in}}%
\pgfpathlineto{\pgfqpoint{1.636293in}{3.115820in}}%
\pgfpathlineto{\pgfqpoint{1.642180in}{3.124279in}}%
\pgfpathlineto{\pgfqpoint{1.651011in}{3.136404in}}%
\pgfpathlineto{\pgfqpoint{1.655025in}{3.138453in}}%
\pgfpathlineto{\pgfqpoint{1.658504in}{3.137733in}}%
\pgfpathlineto{\pgfqpoint{1.662518in}{3.134197in}}%
\pgfpathlineto{\pgfqpoint{1.668941in}{3.124721in}}%
\pgfpathlineto{\pgfqpoint{1.676702in}{3.114183in}}%
\pgfpathlineto{\pgfqpoint{1.680716in}{3.111995in}}%
\pgfpathlineto{\pgfqpoint{1.684195in}{3.112586in}}%
\pgfpathlineto{\pgfqpoint{1.687941in}{3.115692in}}%
\pgfpathlineto{\pgfqpoint{1.693829in}{3.124098in}}%
\pgfpathlineto{\pgfqpoint{1.702660in}{3.136305in}}%
\pgfpathlineto{\pgfqpoint{1.706674in}{3.138437in}}%
\pgfpathlineto{\pgfqpoint{1.710153in}{3.137793in}}%
\pgfpathlineto{\pgfqpoint{1.713899in}{3.134639in}}%
\pgfpathlineto{\pgfqpoint{1.719786in}{3.126197in}}%
\pgfpathlineto{\pgfqpoint{1.728618in}{3.114046in}}%
\pgfpathlineto{\pgfqpoint{1.732632in}{3.111971in}}%
\pgfpathlineto{\pgfqpoint{1.736111in}{3.112667in}}%
\pgfpathlineto{\pgfqpoint{1.739857in}{3.115868in}}%
\pgfpathlineto{\pgfqpoint{1.745744in}{3.124347in}}%
\pgfpathlineto{\pgfqpoint{1.754308in}{3.136204in}}%
\pgfpathlineto{\pgfqpoint{1.758322in}{3.138418in}}%
\pgfpathlineto{\pgfqpoint{1.761801in}{3.137850in}}%
\pgfpathlineto{\pgfqpoint{1.765547in}{3.134766in}}%
\pgfpathlineto{\pgfqpoint{1.771435in}{3.126378in}}%
\pgfpathlineto{\pgfqpoint{1.780266in}{3.114145in}}%
\pgfpathlineto{\pgfqpoint{1.784280in}{3.111988in}}%
\pgfpathlineto{\pgfqpoint{1.787759in}{3.112608in}}%
\pgfpathlineto{\pgfqpoint{1.791505in}{3.115740in}}%
\pgfpathlineto{\pgfqpoint{1.797393in}{3.124165in}}%
\pgfpathlineto{\pgfqpoint{1.806224in}{3.136342in}}%
\pgfpathlineto{\pgfqpoint{1.810238in}{3.138443in}}%
\pgfpathlineto{\pgfqpoint{1.813717in}{3.137771in}}%
\pgfpathlineto{\pgfqpoint{1.817463in}{3.134591in}}%
\pgfpathlineto{\pgfqpoint{1.823351in}{3.126129in}}%
\pgfpathlineto{\pgfqpoint{1.832182in}{3.114009in}}%
\pgfpathlineto{\pgfqpoint{1.836196in}{3.111965in}}%
\pgfpathlineto{\pgfqpoint{1.839675in}{3.112689in}}%
\pgfpathlineto{\pgfqpoint{1.843689in}{3.116230in}}%
\pgfpathlineto{\pgfqpoint{1.850111in}{3.125709in}}%
\pgfpathlineto{\pgfqpoint{1.857872in}{3.136242in}}%
\pgfpathlineto{\pgfqpoint{1.861886in}{3.138425in}}%
\pgfpathlineto{\pgfqpoint{1.865365in}{3.137829in}}%
\pgfpathlineto{\pgfqpoint{1.869112in}{3.134719in}}%
\pgfpathlineto{\pgfqpoint{1.874999in}{3.126310in}}%
\pgfpathlineto{\pgfqpoint{1.883830in}{3.114108in}}%
\pgfpathlineto{\pgfqpoint{1.887844in}{3.111981in}}%
\pgfpathlineto{\pgfqpoint{1.891323in}{3.112630in}}%
\pgfpathlineto{\pgfqpoint{1.895070in}{3.115788in}}%
\pgfpathlineto{\pgfqpoint{1.900957in}{3.124233in}}%
\pgfpathlineto{\pgfqpoint{1.909788in}{3.136379in}}%
\pgfpathlineto{\pgfqpoint{1.913802in}{3.138449in}}%
\pgfpathlineto{\pgfqpoint{1.917281in}{3.137748in}}%
\pgfpathlineto{\pgfqpoint{1.921028in}{3.134543in}}%
\pgfpathlineto{\pgfqpoint{1.926915in}{3.126061in}}%
\pgfpathlineto{\pgfqpoint{1.935478in}{3.114209in}}%
\pgfpathlineto{\pgfqpoint{1.939492in}{3.112000in}}%
\pgfpathlineto{\pgfqpoint{1.942971in}{3.112572in}}%
\pgfpathlineto{\pgfqpoint{1.946718in}{3.115660in}}%
\pgfpathlineto{\pgfqpoint{1.952605in}{3.124052in}}%
\pgfpathlineto{\pgfqpoint{1.961436in}{3.136280in}}%
\pgfpathlineto{\pgfqpoint{1.965450in}{3.138432in}}%
\pgfpathlineto{\pgfqpoint{1.968929in}{3.137807in}}%
\pgfpathlineto{\pgfqpoint{1.972676in}{3.134671in}}%
\pgfpathlineto{\pgfqpoint{1.978563in}{3.126242in}}%
\pgfpathlineto{\pgfqpoint{1.987394in}{3.114070in}}%
\pgfpathlineto{\pgfqpoint{1.991408in}{3.111975in}}%
\pgfpathlineto{\pgfqpoint{1.994887in}{3.112652in}}%
\pgfpathlineto{\pgfqpoint{1.998634in}{3.115836in}}%
\pgfpathlineto{\pgfqpoint{2.004521in}{3.124301in}}%
\pgfpathlineto{\pgfqpoint{2.013085in}{3.136178in}}%
\pgfpathlineto{\pgfqpoint{2.017099in}{3.138413in}}%
\pgfpathlineto{\pgfqpoint{2.020578in}{3.137864in}}%
\pgfpathlineto{\pgfqpoint{2.024324in}{3.134798in}}%
\pgfpathlineto{\pgfqpoint{2.029944in}{3.126852in}}%
\pgfpathlineto{\pgfqpoint{2.039043in}{3.114170in}}%
\pgfpathlineto{\pgfqpoint{2.043057in}{3.111992in}}%
\pgfpathlineto{\pgfqpoint{2.046536in}{3.112594in}}%
\pgfpathlineto{\pgfqpoint{2.050282in}{3.115708in}}%
\pgfpathlineto{\pgfqpoint{2.056170in}{3.124120in}}%
\pgfpathlineto{\pgfqpoint{2.065001in}{3.136317in}}%
\pgfpathlineto{\pgfqpoint{2.069015in}{3.138439in}}%
\pgfpathlineto{\pgfqpoint{2.072494in}{3.137786in}}%
\pgfpathlineto{\pgfqpoint{2.076240in}{3.134623in}}%
\pgfpathlineto{\pgfqpoint{2.082127in}{3.126174in}}%
\pgfpathlineto{\pgfqpoint{2.090959in}{3.114033in}}%
\pgfpathlineto{\pgfqpoint{2.094973in}{3.111969in}}%
\pgfpathlineto{\pgfqpoint{2.098452in}{3.112674in}}%
\pgfpathlineto{\pgfqpoint{2.102198in}{3.115884in}}%
\pgfpathlineto{\pgfqpoint{2.108085in}{3.124369in}}%
\pgfpathlineto{\pgfqpoint{2.116649in}{3.136217in}}%
\pgfpathlineto{\pgfqpoint{2.120663in}{3.138420in}}%
\pgfpathlineto{\pgfqpoint{2.124142in}{3.137843in}}%
\pgfpathlineto{\pgfqpoint{2.127888in}{3.134750in}}%
\pgfpathlineto{\pgfqpoint{2.133776in}{3.126355in}}%
\pgfpathlineto{\pgfqpoint{2.142607in}{3.114133in}}%
\pgfpathlineto{\pgfqpoint{2.146621in}{3.111986in}}%
\pgfpathlineto{\pgfqpoint{2.150100in}{3.112615in}}%
\pgfpathlineto{\pgfqpoint{2.153846in}{3.115756in}}%
\pgfpathlineto{\pgfqpoint{2.159734in}{3.124188in}}%
\pgfpathlineto{\pgfqpoint{2.168565in}{3.136355in}}%
\pgfpathlineto{\pgfqpoint{2.172579in}{3.138445in}}%
\pgfpathlineto{\pgfqpoint{2.176058in}{3.137763in}}%
\pgfpathlineto{\pgfqpoint{2.179804in}{3.134575in}}%
\pgfpathlineto{\pgfqpoint{2.185692in}{3.126106in}}%
\pgfpathlineto{\pgfqpoint{2.194255in}{3.114234in}}%
\pgfpathlineto{\pgfqpoint{2.198269in}{3.112005in}}%
\pgfpathlineto{\pgfqpoint{2.201748in}{3.112558in}}%
\pgfpathlineto{\pgfqpoint{2.205495in}{3.115629in}}%
\pgfpathlineto{\pgfqpoint{2.211114in}{3.123578in}}%
\pgfpathlineto{\pgfqpoint{2.220213in}{3.136255in}}%
\pgfpathlineto{\pgfqpoint{2.224227in}{3.138428in}}%
\pgfpathlineto{\pgfqpoint{2.227706in}{3.137822in}}%
\pgfpathlineto{\pgfqpoint{2.231453in}{3.134703in}}%
\pgfpathlineto{\pgfqpoint{2.237340in}{3.126287in}}%
\pgfpathlineto{\pgfqpoint{2.246171in}{3.114095in}}%
\pgfpathlineto{\pgfqpoint{2.250185in}{3.111979in}}%
\pgfpathlineto{\pgfqpoint{2.253664in}{3.112637in}}%
\pgfpathlineto{\pgfqpoint{2.257411in}{3.115804in}}%
\pgfpathlineto{\pgfqpoint{2.263298in}{3.124256in}}%
\pgfpathlineto{\pgfqpoint{2.272129in}{3.136392in}}%
\pgfpathlineto{\pgfqpoint{2.276143in}{3.138451in}}%
\pgfpathlineto{\pgfqpoint{2.279622in}{3.137741in}}%
\pgfpathlineto{\pgfqpoint{2.283369in}{3.134526in}}%
\pgfpathlineto{\pgfqpoint{2.289256in}{3.126038in}}%
\pgfpathlineto{\pgfqpoint{2.297819in}{3.114196in}}%
\pgfpathlineto{\pgfqpoint{2.301833in}{3.111997in}}%
\pgfpathlineto{\pgfqpoint{2.305312in}{3.112579in}}%
\pgfpathlineto{\pgfqpoint{2.309059in}{3.115676in}}%
\pgfpathlineto{\pgfqpoint{2.314946in}{3.124075in}}%
\pgfpathlineto{\pgfqpoint{2.323777in}{3.136292in}}%
\pgfpathlineto{\pgfqpoint{2.327791in}{3.138434in}}%
\pgfpathlineto{\pgfqpoint{2.331270in}{3.137800in}}%
\pgfpathlineto{\pgfqpoint{2.335017in}{3.134655in}}%
\pgfpathlineto{\pgfqpoint{2.340904in}{3.126219in}}%
\pgfpathlineto{\pgfqpoint{2.349735in}{3.114058in}}%
\pgfpathlineto{\pgfqpoint{2.353749in}{3.111973in}}%
\pgfpathlineto{\pgfqpoint{2.357228in}{3.112659in}}%
\pgfpathlineto{\pgfqpoint{2.360975in}{3.115852in}}%
\pgfpathlineto{\pgfqpoint{2.366862in}{3.124324in}}%
\pgfpathlineto{\pgfqpoint{2.375426in}{3.136191in}}%
\pgfpathlineto{\pgfqpoint{2.379440in}{3.138416in}}%
\pgfpathlineto{\pgfqpoint{2.382919in}{3.137857in}}%
\pgfpathlineto{\pgfqpoint{2.386665in}{3.134782in}}%
\pgfpathlineto{\pgfqpoint{2.392285in}{3.126830in}}%
\pgfpathlineto{\pgfqpoint{2.401384in}{3.114158in}}%
\pgfpathlineto{\pgfqpoint{2.405398in}{3.111990in}}%
\pgfpathlineto{\pgfqpoint{2.408877in}{3.112601in}}%
\pgfpathlineto{\pgfqpoint{2.412623in}{3.115724in}}%
\pgfpathlineto{\pgfqpoint{2.418510in}{3.124143in}}%
\pgfpathlineto{\pgfqpoint{2.427342in}{3.136330in}}%
\pgfpathlineto{\pgfqpoint{2.431356in}{3.138441in}}%
\pgfpathlineto{\pgfqpoint{2.434835in}{3.137778in}}%
\pgfpathlineto{\pgfqpoint{2.438581in}{3.134607in}}%
\pgfpathlineto{\pgfqpoint{2.444468in}{3.126151in}}%
\pgfpathlineto{\pgfqpoint{2.453300in}{3.114021in}}%
\pgfpathlineto{\pgfqpoint{2.457314in}{3.111967in}}%
\pgfpathlineto{\pgfqpoint{2.460793in}{3.112682in}}%
\pgfpathlineto{\pgfqpoint{2.464539in}{3.115901in}}%
\pgfpathlineto{\pgfqpoint{2.470426in}{3.124392in}}%
\pgfpathlineto{\pgfqpoint{2.478990in}{3.136229in}}%
\pgfpathlineto{\pgfqpoint{2.483004in}{3.138423in}}%
\pgfpathlineto{\pgfqpoint{2.486483in}{3.137836in}}%
\pgfpathlineto{\pgfqpoint{2.490229in}{3.134735in}}%
\pgfpathlineto{\pgfqpoint{2.496117in}{3.126333in}}%
\pgfpathlineto{\pgfqpoint{2.504948in}{3.114120in}}%
\pgfpathlineto{\pgfqpoint{2.508962in}{3.111983in}}%
\pgfpathlineto{\pgfqpoint{2.512441in}{3.112622in}}%
\pgfpathlineto{\pgfqpoint{2.516187in}{3.115772in}}%
\pgfpathlineto{\pgfqpoint{2.522075in}{3.124211in}}%
\pgfpathlineto{\pgfqpoint{2.530906in}{3.136367in}}%
\pgfpathlineto{\pgfqpoint{2.534920in}{3.138447in}}%
\pgfpathlineto{\pgfqpoint{2.538399in}{3.137756in}}%
\pgfpathlineto{\pgfqpoint{2.542145in}{3.134559in}}%
\pgfpathlineto{\pgfqpoint{2.548033in}{3.126083in}}%
\pgfpathlineto{\pgfqpoint{2.556596in}{3.114221in}}%
\pgfpathlineto{\pgfqpoint{2.560610in}{3.112002in}}%
\pgfpathlineto{\pgfqpoint{2.564089in}{3.112565in}}%
\pgfpathlineto{\pgfqpoint{2.567836in}{3.115645in}}%
\pgfpathlineto{\pgfqpoint{2.573455in}{3.123600in}}%
\pgfpathlineto{\pgfqpoint{2.582554in}{3.136267in}}%
\pgfpathlineto{\pgfqpoint{2.586568in}{3.138430in}}%
\pgfpathlineto{\pgfqpoint{2.590047in}{3.137815in}}%
\pgfpathlineto{\pgfqpoint{2.593794in}{3.134687in}}%
\pgfpathlineto{\pgfqpoint{2.599681in}{3.126265in}}%
\pgfpathlineto{\pgfqpoint{2.608512in}{3.114083in}}%
\pgfpathlineto{\pgfqpoint{2.612526in}{3.111977in}}%
\pgfpathlineto{\pgfqpoint{2.616005in}{3.112644in}}%
\pgfpathlineto{\pgfqpoint{2.619752in}{3.115820in}}%
\pgfpathlineto{\pgfqpoint{2.625639in}{3.124279in}}%
\pgfpathlineto{\pgfqpoint{2.634470in}{3.136404in}}%
\pgfpathlineto{\pgfqpoint{2.638484in}{3.138453in}}%
\pgfpathlineto{\pgfqpoint{2.641963in}{3.137733in}}%
\pgfpathlineto{\pgfqpoint{2.645977in}{3.134197in}}%
\pgfpathlineto{\pgfqpoint{2.652400in}{3.124721in}}%
\pgfpathlineto{\pgfqpoint{2.660160in}{3.114183in}}%
\pgfpathlineto{\pgfqpoint{2.664174in}{3.111995in}}%
\pgfpathlineto{\pgfqpoint{2.667653in}{3.112586in}}%
\pgfpathlineto{\pgfqpoint{2.671400in}{3.115692in}}%
\pgfpathlineto{\pgfqpoint{2.677287in}{3.124098in}}%
\pgfpathlineto{\pgfqpoint{2.686118in}{3.136305in}}%
\pgfpathlineto{\pgfqpoint{2.690132in}{3.138437in}}%
\pgfpathlineto{\pgfqpoint{2.693611in}{3.137793in}}%
\pgfpathlineto{\pgfqpoint{2.697358in}{3.134639in}}%
\pgfpathlineto{\pgfqpoint{2.703245in}{3.126197in}}%
\pgfpathlineto{\pgfqpoint{2.712076in}{3.114046in}}%
\pgfpathlineto{\pgfqpoint{2.716090in}{3.111971in}}%
\pgfpathlineto{\pgfqpoint{2.719569in}{3.112667in}}%
\pgfpathlineto{\pgfqpoint{2.723316in}{3.115868in}}%
\pgfpathlineto{\pgfqpoint{2.729203in}{3.124347in}}%
\pgfpathlineto{\pgfqpoint{2.737767in}{3.136204in}}%
\pgfpathlineto{\pgfqpoint{2.741781in}{3.138418in}}%
\pgfpathlineto{\pgfqpoint{2.745260in}{3.137850in}}%
\pgfpathlineto{\pgfqpoint{2.749006in}{3.134766in}}%
\pgfpathlineto{\pgfqpoint{2.754894in}{3.126378in}}%
\pgfpathlineto{\pgfqpoint{2.763725in}{3.114145in}}%
\pgfpathlineto{\pgfqpoint{2.767739in}{3.111988in}}%
\pgfpathlineto{\pgfqpoint{2.771218in}{3.112608in}}%
\pgfpathlineto{\pgfqpoint{2.774964in}{3.115740in}}%
\pgfpathlineto{\pgfqpoint{2.780851in}{3.124165in}}%
\pgfpathlineto{\pgfqpoint{2.789683in}{3.136342in}}%
\pgfpathlineto{\pgfqpoint{2.793697in}{3.138443in}}%
\pgfpathlineto{\pgfqpoint{2.797176in}{3.137771in}}%
\pgfpathlineto{\pgfqpoint{2.800922in}{3.134591in}}%
\pgfpathlineto{\pgfqpoint{2.806809in}{3.126129in}}%
\pgfpathlineto{\pgfqpoint{2.815641in}{3.114009in}}%
\pgfpathlineto{\pgfqpoint{2.819655in}{3.111965in}}%
\pgfpathlineto{\pgfqpoint{2.823134in}{3.112689in}}%
\pgfpathlineto{\pgfqpoint{2.827148in}{3.116230in}}%
\pgfpathlineto{\pgfqpoint{2.833570in}{3.125709in}}%
\pgfpathlineto{\pgfqpoint{2.841331in}{3.136242in}}%
\pgfpathlineto{\pgfqpoint{2.845345in}{3.138425in}}%
\pgfpathlineto{\pgfqpoint{2.848824in}{3.137829in}}%
\pgfpathlineto{\pgfqpoint{2.852570in}{3.134719in}}%
\pgfpathlineto{\pgfqpoint{2.858458in}{3.126310in}}%
\pgfpathlineto{\pgfqpoint{2.867289in}{3.114108in}}%
\pgfpathlineto{\pgfqpoint{2.871303in}{3.111981in}}%
\pgfpathlineto{\pgfqpoint{2.874782in}{3.112630in}}%
\pgfpathlineto{\pgfqpoint{2.878528in}{3.115788in}}%
\pgfpathlineto{\pgfqpoint{2.884416in}{3.124233in}}%
\pgfpathlineto{\pgfqpoint{2.893247in}{3.136379in}}%
\pgfpathlineto{\pgfqpoint{2.897261in}{3.138449in}}%
\pgfpathlineto{\pgfqpoint{2.900740in}{3.137748in}}%
\pgfpathlineto{\pgfqpoint{2.904486in}{3.134543in}}%
\pgfpathlineto{\pgfqpoint{2.910374in}{3.126061in}}%
\pgfpathlineto{\pgfqpoint{2.918937in}{3.114209in}}%
\pgfpathlineto{\pgfqpoint{2.922951in}{3.112000in}}%
\pgfpathlineto{\pgfqpoint{2.926430in}{3.112572in}}%
\pgfpathlineto{\pgfqpoint{2.930177in}{3.115660in}}%
\pgfpathlineto{\pgfqpoint{2.936064in}{3.124052in}}%
\pgfpathlineto{\pgfqpoint{2.944895in}{3.136280in}}%
\pgfpathlineto{\pgfqpoint{2.948909in}{3.138432in}}%
\pgfpathlineto{\pgfqpoint{2.952388in}{3.137807in}}%
\pgfpathlineto{\pgfqpoint{2.956135in}{3.134671in}}%
\pgfpathlineto{\pgfqpoint{2.962022in}{3.126242in}}%
\pgfpathlineto{\pgfqpoint{2.970853in}{3.114070in}}%
\pgfpathlineto{\pgfqpoint{2.974867in}{3.111975in}}%
\pgfpathlineto{\pgfqpoint{2.978346in}{3.112652in}}%
\pgfpathlineto{\pgfqpoint{2.982093in}{3.115836in}}%
\pgfpathlineto{\pgfqpoint{2.987980in}{3.124301in}}%
\pgfpathlineto{\pgfqpoint{2.996543in}{3.136178in}}%
\pgfpathlineto{\pgfqpoint{3.000558in}{3.138413in}}%
\pgfpathlineto{\pgfqpoint{3.004036in}{3.137864in}}%
\pgfpathlineto{\pgfqpoint{3.007783in}{3.134798in}}%
\pgfpathlineto{\pgfqpoint{3.013403in}{3.126852in}}%
\pgfpathlineto{\pgfqpoint{3.022501in}{3.114170in}}%
\pgfpathlineto{\pgfqpoint{3.026515in}{3.111992in}}%
\pgfpathlineto{\pgfqpoint{3.029994in}{3.112594in}}%
\pgfpathlineto{\pgfqpoint{3.033741in}{3.115708in}}%
\pgfpathlineto{\pgfqpoint{3.039628in}{3.124120in}}%
\pgfpathlineto{\pgfqpoint{3.048459in}{3.136317in}}%
\pgfpathlineto{\pgfqpoint{3.052473in}{3.138439in}}%
\pgfpathlineto{\pgfqpoint{3.055952in}{3.137786in}}%
\pgfpathlineto{\pgfqpoint{3.059699in}{3.134623in}}%
\pgfpathlineto{\pgfqpoint{3.065586in}{3.126174in}}%
\pgfpathlineto{\pgfqpoint{3.074417in}{3.114033in}}%
\pgfpathlineto{\pgfqpoint{3.078431in}{3.111969in}}%
\pgfpathlineto{\pgfqpoint{3.081910in}{3.112674in}}%
\pgfpathlineto{\pgfqpoint{3.085657in}{3.115884in}}%
\pgfpathlineto{\pgfqpoint{3.091544in}{3.124369in}}%
\pgfpathlineto{\pgfqpoint{3.100108in}{3.136217in}}%
\pgfpathlineto{\pgfqpoint{3.104122in}{3.138420in}}%
\pgfpathlineto{\pgfqpoint{3.107601in}{3.137843in}}%
\pgfpathlineto{\pgfqpoint{3.111347in}{3.134750in}}%
\pgfpathlineto{\pgfqpoint{3.117235in}{3.126355in}}%
\pgfpathlineto{\pgfqpoint{3.126066in}{3.114133in}}%
\pgfpathlineto{\pgfqpoint{3.130080in}{3.111986in}}%
\pgfpathlineto{\pgfqpoint{3.133559in}{3.112615in}}%
\pgfpathlineto{\pgfqpoint{3.137305in}{3.115756in}}%
\pgfpathlineto{\pgfqpoint{3.143192in}{3.124188in}}%
\pgfpathlineto{\pgfqpoint{3.152024in}{3.136355in}}%
\pgfpathlineto{\pgfqpoint{3.156038in}{3.138445in}}%
\pgfpathlineto{\pgfqpoint{3.159517in}{3.137763in}}%
\pgfpathlineto{\pgfqpoint{3.163263in}{3.134575in}}%
\pgfpathlineto{\pgfqpoint{3.169150in}{3.126106in}}%
\pgfpathlineto{\pgfqpoint{3.177714in}{3.114234in}}%
\pgfpathlineto{\pgfqpoint{3.181728in}{3.112005in}}%
\pgfpathlineto{\pgfqpoint{3.185207in}{3.112558in}}%
\pgfpathlineto{\pgfqpoint{3.188953in}{3.115629in}}%
\pgfpathlineto{\pgfqpoint{3.194573in}{3.123578in}}%
\pgfpathlineto{\pgfqpoint{3.203672in}{3.136255in}}%
\pgfpathlineto{\pgfqpoint{3.207686in}{3.138428in}}%
\pgfpathlineto{\pgfqpoint{3.211165in}{3.137822in}}%
\pgfpathlineto{\pgfqpoint{3.214911in}{3.134703in}}%
\pgfpathlineto{\pgfqpoint{3.220799in}{3.126287in}}%
\pgfpathlineto{\pgfqpoint{3.229630in}{3.114095in}}%
\pgfpathlineto{\pgfqpoint{3.233644in}{3.111979in}}%
\pgfpathlineto{\pgfqpoint{3.237123in}{3.112637in}}%
\pgfpathlineto{\pgfqpoint{3.240869in}{3.115804in}}%
\pgfpathlineto{\pgfqpoint{3.246757in}{3.124256in}}%
\pgfpathlineto{\pgfqpoint{3.255588in}{3.136392in}}%
\pgfpathlineto{\pgfqpoint{3.259602in}{3.138451in}}%
\pgfpathlineto{\pgfqpoint{3.263081in}{3.137741in}}%
\pgfpathlineto{\pgfqpoint{3.266827in}{3.134526in}}%
\pgfpathlineto{\pgfqpoint{3.272715in}{3.126038in}}%
\pgfpathlineto{\pgfqpoint{3.281278in}{3.114196in}}%
\pgfpathlineto{\pgfqpoint{3.285292in}{3.111997in}}%
\pgfpathlineto{\pgfqpoint{3.288771in}{3.112579in}}%
\pgfpathlineto{\pgfqpoint{3.292518in}{3.115676in}}%
\pgfpathlineto{\pgfqpoint{3.298405in}{3.124075in}}%
\pgfpathlineto{\pgfqpoint{3.307236in}{3.136292in}}%
\pgfpathlineto{\pgfqpoint{3.311250in}{3.138434in}}%
\pgfpathlineto{\pgfqpoint{3.314729in}{3.137800in}}%
\pgfpathlineto{\pgfqpoint{3.318476in}{3.134655in}}%
\pgfpathlineto{\pgfqpoint{3.324363in}{3.126219in}}%
\pgfpathlineto{\pgfqpoint{3.333194in}{3.114058in}}%
\pgfpathlineto{\pgfqpoint{3.337208in}{3.111973in}}%
\pgfpathlineto{\pgfqpoint{3.340687in}{3.112659in}}%
\pgfpathlineto{\pgfqpoint{3.344434in}{3.115852in}}%
\pgfpathlineto{\pgfqpoint{3.350321in}{3.124324in}}%
\pgfpathlineto{\pgfqpoint{3.358884in}{3.136191in}}%
\pgfpathlineto{\pgfqpoint{3.362898in}{3.138416in}}%
\pgfpathlineto{\pgfqpoint{3.366377in}{3.137857in}}%
\pgfpathlineto{\pgfqpoint{3.370124in}{3.134782in}}%
\pgfpathlineto{\pgfqpoint{3.375744in}{3.126830in}}%
\pgfpathlineto{\pgfqpoint{3.384842in}{3.114158in}}%
\pgfpathlineto{\pgfqpoint{3.388856in}{3.111990in}}%
\pgfpathlineto{\pgfqpoint{3.392335in}{3.112601in}}%
\pgfpathlineto{\pgfqpoint{3.396082in}{3.115724in}}%
\pgfpathlineto{\pgfqpoint{3.401969in}{3.124143in}}%
\pgfpathlineto{\pgfqpoint{3.410800in}{3.136330in}}%
\pgfpathlineto{\pgfqpoint{3.414814in}{3.138441in}}%
\pgfpathlineto{\pgfqpoint{3.418293in}{3.137778in}}%
\pgfpathlineto{\pgfqpoint{3.422040in}{3.134607in}}%
\pgfpathlineto{\pgfqpoint{3.427927in}{3.126151in}}%
\pgfpathlineto{\pgfqpoint{3.436758in}{3.114021in}}%
\pgfpathlineto{\pgfqpoint{3.440772in}{3.111967in}}%
\pgfpathlineto{\pgfqpoint{3.444251in}{3.112682in}}%
\pgfpathlineto{\pgfqpoint{3.447998in}{3.115901in}}%
\pgfpathlineto{\pgfqpoint{3.453885in}{3.124392in}}%
\pgfpathlineto{\pgfqpoint{3.462449in}{3.136229in}}%
\pgfpathlineto{\pgfqpoint{3.466463in}{3.138423in}}%
\pgfpathlineto{\pgfqpoint{3.469942in}{3.137836in}}%
\pgfpathlineto{\pgfqpoint{3.473688in}{3.134735in}}%
\pgfpathlineto{\pgfqpoint{3.479576in}{3.126333in}}%
\pgfpathlineto{\pgfqpoint{3.488407in}{3.114120in}}%
\pgfpathlineto{\pgfqpoint{3.492421in}{3.111983in}}%
\pgfpathlineto{\pgfqpoint{3.495900in}{3.112622in}}%
\pgfpathlineto{\pgfqpoint{3.499646in}{3.115772in}}%
\pgfpathlineto{\pgfqpoint{3.505533in}{3.124211in}}%
\pgfpathlineto{\pgfqpoint{3.514365in}{3.136367in}}%
\pgfpathlineto{\pgfqpoint{3.518379in}{3.138447in}}%
\pgfpathlineto{\pgfqpoint{3.521858in}{3.137756in}}%
\pgfpathlineto{\pgfqpoint{3.525604in}{3.134559in}}%
\pgfpathlineto{\pgfqpoint{3.531491in}{3.126083in}}%
\pgfpathlineto{\pgfqpoint{3.540055in}{3.114221in}}%
\pgfpathlineto{\pgfqpoint{3.544069in}{3.112002in}}%
\pgfpathlineto{\pgfqpoint{3.547548in}{3.112565in}}%
\pgfpathlineto{\pgfqpoint{3.551294in}{3.115645in}}%
\pgfpathlineto{\pgfqpoint{3.556914in}{3.123600in}}%
\pgfpathlineto{\pgfqpoint{3.566013in}{3.136267in}}%
\pgfpathlineto{\pgfqpoint{3.570027in}{3.138430in}}%
\pgfpathlineto{\pgfqpoint{3.573506in}{3.137815in}}%
\pgfpathlineto{\pgfqpoint{3.577252in}{3.134687in}}%
\pgfpathlineto{\pgfqpoint{3.583140in}{3.126265in}}%
\pgfpathlineto{\pgfqpoint{3.591971in}{3.114083in}}%
\pgfpathlineto{\pgfqpoint{3.595985in}{3.111977in}}%
\pgfpathlineto{\pgfqpoint{3.599464in}{3.112644in}}%
\pgfpathlineto{\pgfqpoint{3.603210in}{3.115820in}}%
\pgfpathlineto{\pgfqpoint{3.609098in}{3.124279in}}%
\pgfpathlineto{\pgfqpoint{3.617929in}{3.136404in}}%
\pgfpathlineto{\pgfqpoint{3.621943in}{3.138453in}}%
\pgfpathlineto{\pgfqpoint{3.625422in}{3.137733in}}%
\pgfpathlineto{\pgfqpoint{3.629436in}{3.134197in}}%
\pgfpathlineto{\pgfqpoint{3.635858in}{3.124721in}}%
\pgfpathlineto{\pgfqpoint{3.643619in}{3.114183in}}%
\pgfpathlineto{\pgfqpoint{3.647633in}{3.111995in}}%
\pgfpathlineto{\pgfqpoint{3.651112in}{3.112586in}}%
\pgfpathlineto{\pgfqpoint{3.654859in}{3.115692in}}%
\pgfpathlineto{\pgfqpoint{3.660746in}{3.124098in}}%
\pgfpathlineto{\pgfqpoint{3.669577in}{3.136305in}}%
\pgfpathlineto{\pgfqpoint{3.673591in}{3.138437in}}%
\pgfpathlineto{\pgfqpoint{3.677070in}{3.137793in}}%
\pgfpathlineto{\pgfqpoint{3.680817in}{3.134639in}}%
\pgfpathlineto{\pgfqpoint{3.686704in}{3.126197in}}%
\pgfpathlineto{\pgfqpoint{3.695535in}{3.114046in}}%
\pgfpathlineto{\pgfqpoint{3.699549in}{3.111971in}}%
\pgfpathlineto{\pgfqpoint{3.703028in}{3.112667in}}%
\pgfpathlineto{\pgfqpoint{3.706775in}{3.115868in}}%
\pgfpathlineto{\pgfqpoint{3.712662in}{3.124347in}}%
\pgfpathlineto{\pgfqpoint{3.721225in}{3.136204in}}%
\pgfpathlineto{\pgfqpoint{3.725239in}{3.138418in}}%
\pgfpathlineto{\pgfqpoint{3.728718in}{3.137850in}}%
\pgfpathlineto{\pgfqpoint{3.732465in}{3.134766in}}%
\pgfpathlineto{\pgfqpoint{3.738352in}{3.126378in}}%
\pgfpathlineto{\pgfqpoint{3.747183in}{3.114145in}}%
\pgfpathlineto{\pgfqpoint{3.751197in}{3.111988in}}%
\pgfpathlineto{\pgfqpoint{3.754676in}{3.112608in}}%
\pgfpathlineto{\pgfqpoint{3.758423in}{3.115740in}}%
\pgfpathlineto{\pgfqpoint{3.764310in}{3.124165in}}%
\pgfpathlineto{\pgfqpoint{3.773141in}{3.136342in}}%
\pgfpathlineto{\pgfqpoint{3.777155in}{3.138443in}}%
\pgfpathlineto{\pgfqpoint{3.780634in}{3.137771in}}%
\pgfpathlineto{\pgfqpoint{3.784381in}{3.134591in}}%
\pgfpathlineto{\pgfqpoint{3.790268in}{3.126129in}}%
\pgfpathlineto{\pgfqpoint{3.799099in}{3.114009in}}%
\pgfpathlineto{\pgfqpoint{3.803113in}{3.111965in}}%
\pgfpathlineto{\pgfqpoint{3.806592in}{3.112689in}}%
\pgfpathlineto{\pgfqpoint{3.810606in}{3.116230in}}%
\pgfpathlineto{\pgfqpoint{3.817029in}{3.125709in}}%
\pgfpathlineto{\pgfqpoint{3.824790in}{3.136242in}}%
\pgfpathlineto{\pgfqpoint{3.828804in}{3.138425in}}%
\pgfpathlineto{\pgfqpoint{3.832283in}{3.137829in}}%
\pgfpathlineto{\pgfqpoint{3.836029in}{3.134719in}}%
\pgfpathlineto{\pgfqpoint{3.841916in}{3.126310in}}%
\pgfpathlineto{\pgfqpoint{3.850748in}{3.114108in}}%
\pgfpathlineto{\pgfqpoint{3.854762in}{3.111981in}}%
\pgfpathlineto{\pgfqpoint{3.858241in}{3.112630in}}%
\pgfpathlineto{\pgfqpoint{3.861987in}{3.115788in}}%
\pgfpathlineto{\pgfqpoint{3.867874in}{3.124233in}}%
\pgfpathlineto{\pgfqpoint{3.876706in}{3.136379in}}%
\pgfpathlineto{\pgfqpoint{3.880720in}{3.138449in}}%
\pgfpathlineto{\pgfqpoint{3.884199in}{3.137748in}}%
\pgfpathlineto{\pgfqpoint{3.887945in}{3.134543in}}%
\pgfpathlineto{\pgfqpoint{3.893832in}{3.126061in}}%
\pgfpathlineto{\pgfqpoint{3.902396in}{3.114209in}}%
\pgfpathlineto{\pgfqpoint{3.906410in}{3.112000in}}%
\pgfpathlineto{\pgfqpoint{3.909889in}{3.112572in}}%
\pgfpathlineto{\pgfqpoint{3.913635in}{3.115660in}}%
\pgfpathlineto{\pgfqpoint{3.919523in}{3.124052in}}%
\pgfpathlineto{\pgfqpoint{3.928354in}{3.136280in}}%
\pgfpathlineto{\pgfqpoint{3.932368in}{3.138432in}}%
\pgfpathlineto{\pgfqpoint{3.935847in}{3.137807in}}%
\pgfpathlineto{\pgfqpoint{3.939593in}{3.134671in}}%
\pgfpathlineto{\pgfqpoint{3.945481in}{3.126242in}}%
\pgfpathlineto{\pgfqpoint{3.954312in}{3.114070in}}%
\pgfpathlineto{\pgfqpoint{3.958326in}{3.111975in}}%
\pgfpathlineto{\pgfqpoint{3.961805in}{3.112652in}}%
\pgfpathlineto{\pgfqpoint{3.965551in}{3.115836in}}%
\pgfpathlineto{\pgfqpoint{3.971439in}{3.124301in}}%
\pgfpathlineto{\pgfqpoint{3.980002in}{3.136178in}}%
\pgfpathlineto{\pgfqpoint{3.984016in}{3.138413in}}%
\pgfpathlineto{\pgfqpoint{3.987495in}{3.137864in}}%
\pgfpathlineto{\pgfqpoint{3.991242in}{3.134798in}}%
\pgfpathlineto{\pgfqpoint{3.996861in}{3.126852in}}%
\pgfpathlineto{\pgfqpoint{4.005960in}{3.114170in}}%
\pgfpathlineto{\pgfqpoint{4.009974in}{3.111992in}}%
\pgfpathlineto{\pgfqpoint{4.013453in}{3.112594in}}%
\pgfpathlineto{\pgfqpoint{4.017200in}{3.115708in}}%
\pgfpathlineto{\pgfqpoint{4.023087in}{3.124120in}}%
\pgfpathlineto{\pgfqpoint{4.031918in}{3.136317in}}%
\pgfpathlineto{\pgfqpoint{4.035932in}{3.138439in}}%
\pgfpathlineto{\pgfqpoint{4.039411in}{3.137786in}}%
\pgfpathlineto{\pgfqpoint{4.043158in}{3.134623in}}%
\pgfpathlineto{\pgfqpoint{4.049045in}{3.126174in}}%
\pgfpathlineto{\pgfqpoint{4.057876in}{3.114033in}}%
\pgfpathlineto{\pgfqpoint{4.061890in}{3.111969in}}%
\pgfpathlineto{\pgfqpoint{4.065369in}{3.112674in}}%
\pgfpathlineto{\pgfqpoint{4.069116in}{3.115884in}}%
\pgfpathlineto{\pgfqpoint{4.075003in}{3.124369in}}%
\pgfpathlineto{\pgfqpoint{4.083566in}{3.136217in}}%
\pgfpathlineto{\pgfqpoint{4.087580in}{3.138420in}}%
\pgfpathlineto{\pgfqpoint{4.091059in}{3.137843in}}%
\pgfpathlineto{\pgfqpoint{4.094806in}{3.134750in}}%
\pgfpathlineto{\pgfqpoint{4.100693in}{3.126355in}}%
\pgfpathlineto{\pgfqpoint{4.109524in}{3.114133in}}%
\pgfpathlineto{\pgfqpoint{4.113538in}{3.111986in}}%
\pgfpathlineto{\pgfqpoint{4.117017in}{3.112615in}}%
\pgfpathlineto{\pgfqpoint{4.120764in}{3.115756in}}%
\pgfpathlineto{\pgfqpoint{4.126651in}{3.124188in}}%
\pgfpathlineto{\pgfqpoint{4.135482in}{3.136355in}}%
\pgfpathlineto{\pgfqpoint{4.139496in}{3.138445in}}%
\pgfpathlineto{\pgfqpoint{4.142975in}{3.137763in}}%
\pgfpathlineto{\pgfqpoint{4.146722in}{3.134575in}}%
\pgfpathlineto{\pgfqpoint{4.152609in}{3.126106in}}%
\pgfpathlineto{\pgfqpoint{4.161173in}{3.114234in}}%
\pgfpathlineto{\pgfqpoint{4.165187in}{3.112005in}}%
\pgfpathlineto{\pgfqpoint{4.168666in}{3.112558in}}%
\pgfpathlineto{\pgfqpoint{4.172412in}{3.115629in}}%
\pgfpathlineto{\pgfqpoint{4.178032in}{3.123578in}}%
\pgfpathlineto{\pgfqpoint{4.187131in}{3.136255in}}%
\pgfpathlineto{\pgfqpoint{4.191145in}{3.138428in}}%
\pgfpathlineto{\pgfqpoint{4.194624in}{3.137822in}}%
\pgfpathlineto{\pgfqpoint{4.198370in}{3.134703in}}%
\pgfpathlineto{\pgfqpoint{4.204257in}{3.126287in}}%
\pgfpathlineto{\pgfqpoint{4.213089in}{3.114095in}}%
\pgfpathlineto{\pgfqpoint{4.217103in}{3.111979in}}%
\pgfpathlineto{\pgfqpoint{4.220582in}{3.112637in}}%
\pgfpathlineto{\pgfqpoint{4.224328in}{3.115804in}}%
\pgfpathlineto{\pgfqpoint{4.230215in}{3.124256in}}%
\pgfpathlineto{\pgfqpoint{4.239047in}{3.136392in}}%
\pgfpathlineto{\pgfqpoint{4.243061in}{3.138451in}}%
\pgfpathlineto{\pgfqpoint{4.246540in}{3.137741in}}%
\pgfpathlineto{\pgfqpoint{4.250286in}{3.134526in}}%
\pgfpathlineto{\pgfqpoint{4.256173in}{3.126038in}}%
\pgfpathlineto{\pgfqpoint{4.264737in}{3.114196in}}%
\pgfpathlineto{\pgfqpoint{4.268751in}{3.111997in}}%
\pgfpathlineto{\pgfqpoint{4.272230in}{3.112579in}}%
\pgfpathlineto{\pgfqpoint{4.275976in}{3.115676in}}%
\pgfpathlineto{\pgfqpoint{4.281864in}{3.124075in}}%
\pgfpathlineto{\pgfqpoint{4.290695in}{3.136292in}}%
\pgfpathlineto{\pgfqpoint{4.294709in}{3.138434in}}%
\pgfpathlineto{\pgfqpoint{4.298188in}{3.137800in}}%
\pgfpathlineto{\pgfqpoint{4.301934in}{3.134655in}}%
\pgfpathlineto{\pgfqpoint{4.307822in}{3.126219in}}%
\pgfpathlineto{\pgfqpoint{4.316653in}{3.114058in}}%
\pgfpathlineto{\pgfqpoint{4.320667in}{3.111973in}}%
\pgfpathlineto{\pgfqpoint{4.324146in}{3.112659in}}%
\pgfpathlineto{\pgfqpoint{4.327892in}{3.115852in}}%
\pgfpathlineto{\pgfqpoint{4.333780in}{3.124324in}}%
\pgfpathlineto{\pgfqpoint{4.342343in}{3.136191in}}%
\pgfpathlineto{\pgfqpoint{4.346357in}{3.138416in}}%
\pgfpathlineto{\pgfqpoint{4.349836in}{3.137857in}}%
\pgfpathlineto{\pgfqpoint{4.353583in}{3.134782in}}%
\pgfpathlineto{\pgfqpoint{4.359202in}{3.126830in}}%
\pgfpathlineto{\pgfqpoint{4.368301in}{3.114158in}}%
\pgfpathlineto{\pgfqpoint{4.372315in}{3.111990in}}%
\pgfpathlineto{\pgfqpoint{4.375794in}{3.112601in}}%
\pgfpathlineto{\pgfqpoint{4.379541in}{3.115724in}}%
\pgfpathlineto{\pgfqpoint{4.385428in}{3.124143in}}%
\pgfpathlineto{\pgfqpoint{4.394259in}{3.136330in}}%
\pgfpathlineto{\pgfqpoint{4.398273in}{3.138441in}}%
\pgfpathlineto{\pgfqpoint{4.401752in}{3.137778in}}%
\pgfpathlineto{\pgfqpoint{4.405499in}{3.134607in}}%
\pgfpathlineto{\pgfqpoint{4.411386in}{3.126151in}}%
\pgfpathlineto{\pgfqpoint{4.420217in}{3.114021in}}%
\pgfpathlineto{\pgfqpoint{4.424231in}{3.111967in}}%
\pgfpathlineto{\pgfqpoint{4.427710in}{3.112682in}}%
\pgfpathlineto{\pgfqpoint{4.431457in}{3.115901in}}%
\pgfpathlineto{\pgfqpoint{4.437344in}{3.124392in}}%
\pgfpathlineto{\pgfqpoint{4.445907in}{3.136229in}}%
\pgfpathlineto{\pgfqpoint{4.449921in}{3.138423in}}%
\pgfpathlineto{\pgfqpoint{4.453400in}{3.137836in}}%
\pgfpathlineto{\pgfqpoint{4.457147in}{3.134735in}}%
\pgfpathlineto{\pgfqpoint{4.463034in}{3.126333in}}%
\pgfpathlineto{\pgfqpoint{4.471865in}{3.114120in}}%
\pgfpathlineto{\pgfqpoint{4.475879in}{3.111983in}}%
\pgfpathlineto{\pgfqpoint{4.479358in}{3.112622in}}%
\pgfpathlineto{\pgfqpoint{4.483105in}{3.115772in}}%
\pgfpathlineto{\pgfqpoint{4.488992in}{3.124211in}}%
\pgfpathlineto{\pgfqpoint{4.497823in}{3.136367in}}%
\pgfpathlineto{\pgfqpoint{4.501837in}{3.138447in}}%
\pgfpathlineto{\pgfqpoint{4.505316in}{3.137756in}}%
\pgfpathlineto{\pgfqpoint{4.509063in}{3.134559in}}%
\pgfpathlineto{\pgfqpoint{4.514950in}{3.126083in}}%
\pgfpathlineto{\pgfqpoint{4.523514in}{3.114221in}}%
\pgfpathlineto{\pgfqpoint{4.527528in}{3.112002in}}%
\pgfpathlineto{\pgfqpoint{4.531007in}{3.112565in}}%
\pgfpathlineto{\pgfqpoint{4.534753in}{3.115645in}}%
\pgfpathlineto{\pgfqpoint{4.540373in}{3.123600in}}%
\pgfpathlineto{\pgfqpoint{4.549472in}{3.136267in}}%
\pgfpathlineto{\pgfqpoint{4.553486in}{3.138430in}}%
\pgfpathlineto{\pgfqpoint{4.556965in}{3.137815in}}%
\pgfpathlineto{\pgfqpoint{4.560711in}{3.134687in}}%
\pgfpathlineto{\pgfqpoint{4.566598in}{3.126265in}}%
\pgfpathlineto{\pgfqpoint{4.575430in}{3.114083in}}%
\pgfpathlineto{\pgfqpoint{4.579444in}{3.111977in}}%
\pgfpathlineto{\pgfqpoint{4.582923in}{3.112644in}}%
\pgfpathlineto{\pgfqpoint{4.586669in}{3.115820in}}%
\pgfpathlineto{\pgfqpoint{4.592556in}{3.124279in}}%
\pgfpathlineto{\pgfqpoint{4.601387in}{3.136404in}}%
\pgfpathlineto{\pgfqpoint{4.605402in}{3.138453in}}%
\pgfpathlineto{\pgfqpoint{4.608881in}{3.137733in}}%
\pgfpathlineto{\pgfqpoint{4.612895in}{3.134197in}}%
\pgfpathlineto{\pgfqpoint{4.619317in}{3.124721in}}%
\pgfpathlineto{\pgfqpoint{4.627078in}{3.114183in}}%
\pgfpathlineto{\pgfqpoint{4.631092in}{3.111995in}}%
\pgfpathlineto{\pgfqpoint{4.634571in}{3.112586in}}%
\pgfpathlineto{\pgfqpoint{4.638317in}{3.115692in}}%
\pgfpathlineto{\pgfqpoint{4.644205in}{3.124098in}}%
\pgfpathlineto{\pgfqpoint{4.653036in}{3.136305in}}%
\pgfpathlineto{\pgfqpoint{4.657050in}{3.138437in}}%
\pgfpathlineto{\pgfqpoint{4.660529in}{3.137793in}}%
\pgfpathlineto{\pgfqpoint{4.664275in}{3.134639in}}%
\pgfpathlineto{\pgfqpoint{4.670163in}{3.126197in}}%
\pgfpathlineto{\pgfqpoint{4.678994in}{3.114046in}}%
\pgfpathlineto{\pgfqpoint{4.683008in}{3.111971in}}%
\pgfpathlineto{\pgfqpoint{4.686487in}{3.112667in}}%
\pgfpathlineto{\pgfqpoint{4.690233in}{3.115868in}}%
\pgfpathlineto{\pgfqpoint{4.696121in}{3.124347in}}%
\pgfpathlineto{\pgfqpoint{4.704684in}{3.136204in}}%
\pgfpathlineto{\pgfqpoint{4.708698in}{3.138418in}}%
\pgfpathlineto{\pgfqpoint{4.712177in}{3.137850in}}%
\pgfpathlineto{\pgfqpoint{4.715924in}{3.134766in}}%
\pgfpathlineto{\pgfqpoint{4.721811in}{3.126378in}}%
\pgfpathlineto{\pgfqpoint{4.730642in}{3.114145in}}%
\pgfpathlineto{\pgfqpoint{4.734656in}{3.111988in}}%
\pgfpathlineto{\pgfqpoint{4.738135in}{3.112608in}}%
\pgfpathlineto{\pgfqpoint{4.741882in}{3.115740in}}%
\pgfpathlineto{\pgfqpoint{4.747769in}{3.124165in}}%
\pgfpathlineto{\pgfqpoint{4.756600in}{3.136342in}}%
\pgfpathlineto{\pgfqpoint{4.760614in}{3.138443in}}%
\pgfpathlineto{\pgfqpoint{4.764093in}{3.137771in}}%
\pgfpathlineto{\pgfqpoint{4.767840in}{3.134591in}}%
\pgfpathlineto{\pgfqpoint{4.773727in}{3.126129in}}%
\pgfpathlineto{\pgfqpoint{4.782558in}{3.114009in}}%
\pgfpathlineto{\pgfqpoint{4.786572in}{3.111965in}}%
\pgfpathlineto{\pgfqpoint{4.790051in}{3.112689in}}%
\pgfpathlineto{\pgfqpoint{4.794065in}{3.116230in}}%
\pgfpathlineto{\pgfqpoint{4.800488in}{3.125709in}}%
\pgfpathlineto{\pgfqpoint{4.808248in}{3.136242in}}%
\pgfpathlineto{\pgfqpoint{4.812262in}{3.138425in}}%
\pgfpathlineto{\pgfqpoint{4.815741in}{3.137829in}}%
\pgfpathlineto{\pgfqpoint{4.819488in}{3.134719in}}%
\pgfpathlineto{\pgfqpoint{4.825375in}{3.126310in}}%
\pgfpathlineto{\pgfqpoint{4.834206in}{3.114108in}}%
\pgfpathlineto{\pgfqpoint{4.838220in}{3.111981in}}%
\pgfpathlineto{\pgfqpoint{4.841699in}{3.112630in}}%
\pgfpathlineto{\pgfqpoint{4.845446in}{3.115788in}}%
\pgfpathlineto{\pgfqpoint{4.851333in}{3.124233in}}%
\pgfpathlineto{\pgfqpoint{4.860164in}{3.136379in}}%
\pgfpathlineto{\pgfqpoint{4.864178in}{3.138449in}}%
\pgfpathlineto{\pgfqpoint{4.867657in}{3.137748in}}%
\pgfpathlineto{\pgfqpoint{4.871404in}{3.134543in}}%
\pgfpathlineto{\pgfqpoint{4.877291in}{3.126061in}}%
\pgfpathlineto{\pgfqpoint{4.885855in}{3.114209in}}%
\pgfpathlineto{\pgfqpoint{4.889869in}{3.112000in}}%
\pgfpathlineto{\pgfqpoint{4.893348in}{3.112572in}}%
\pgfpathlineto{\pgfqpoint{4.897094in}{3.115660in}}%
\pgfpathlineto{\pgfqpoint{4.902982in}{3.124052in}}%
\pgfpathlineto{\pgfqpoint{4.911813in}{3.136280in}}%
\pgfpathlineto{\pgfqpoint{4.915827in}{3.138432in}}%
\pgfpathlineto{\pgfqpoint{4.919306in}{3.137807in}}%
\pgfpathlineto{\pgfqpoint{4.923052in}{3.134671in}}%
\pgfpathlineto{\pgfqpoint{4.928939in}{3.126242in}}%
\pgfpathlineto{\pgfqpoint{4.937771in}{3.114070in}}%
\pgfpathlineto{\pgfqpoint{4.941785in}{3.111975in}}%
\pgfpathlineto{\pgfqpoint{4.945264in}{3.112652in}}%
\pgfpathlineto{\pgfqpoint{4.949010in}{3.115836in}}%
\pgfpathlineto{\pgfqpoint{4.954897in}{3.124301in}}%
\pgfpathlineto{\pgfqpoint{4.963461in}{3.136178in}}%
\pgfpathlineto{\pgfqpoint{4.967475in}{3.138413in}}%
\pgfpathlineto{\pgfqpoint{4.970954in}{3.137864in}}%
\pgfpathlineto{\pgfqpoint{4.974700in}{3.134798in}}%
\pgfpathlineto{\pgfqpoint{4.980320in}{3.126852in}}%
\pgfpathlineto{\pgfqpoint{4.989419in}{3.114170in}}%
\pgfpathlineto{\pgfqpoint{4.993433in}{3.111992in}}%
\pgfpathlineto{\pgfqpoint{4.996912in}{3.112594in}}%
\pgfpathlineto{\pgfqpoint{5.000658in}{3.115708in}}%
\pgfpathlineto{\pgfqpoint{5.006546in}{3.124120in}}%
\pgfpathlineto{\pgfqpoint{5.015377in}{3.136317in}}%
\pgfpathlineto{\pgfqpoint{5.019391in}{3.138439in}}%
\pgfpathlineto{\pgfqpoint{5.022870in}{3.137786in}}%
\pgfpathlineto{\pgfqpoint{5.026616in}{3.134623in}}%
\pgfpathlineto{\pgfqpoint{5.032504in}{3.126174in}}%
\pgfpathlineto{\pgfqpoint{5.041335in}{3.114033in}}%
\pgfpathlineto{\pgfqpoint{5.045349in}{3.111969in}}%
\pgfpathlineto{\pgfqpoint{5.048828in}{3.112674in}}%
\pgfpathlineto{\pgfqpoint{5.052574in}{3.115884in}}%
\pgfpathlineto{\pgfqpoint{5.058462in}{3.124369in}}%
\pgfpathlineto{\pgfqpoint{5.067025in}{3.136217in}}%
\pgfpathlineto{\pgfqpoint{5.071039in}{3.138420in}}%
\pgfpathlineto{\pgfqpoint{5.074518in}{3.137843in}}%
\pgfpathlineto{\pgfqpoint{5.078265in}{3.134750in}}%
\pgfpathlineto{\pgfqpoint{5.084152in}{3.126355in}}%
\pgfpathlineto{\pgfqpoint{5.092983in}{3.114133in}}%
\pgfpathlineto{\pgfqpoint{5.096997in}{3.111986in}}%
\pgfpathlineto{\pgfqpoint{5.100476in}{3.112615in}}%
\pgfpathlineto{\pgfqpoint{5.104223in}{3.115756in}}%
\pgfpathlineto{\pgfqpoint{5.110110in}{3.124188in}}%
\pgfpathlineto{\pgfqpoint{5.118941in}{3.136355in}}%
\pgfpathlineto{\pgfqpoint{5.122955in}{3.138445in}}%
\pgfpathlineto{\pgfqpoint{5.126434in}{3.137763in}}%
\pgfpathlineto{\pgfqpoint{5.130181in}{3.134575in}}%
\pgfpathlineto{\pgfqpoint{5.136068in}{3.126106in}}%
\pgfpathlineto{\pgfqpoint{5.144631in}{3.114234in}}%
\pgfpathlineto{\pgfqpoint{5.148645in}{3.112005in}}%
\pgfpathlineto{\pgfqpoint{5.152124in}{3.112558in}}%
\pgfpathlineto{\pgfqpoint{5.155871in}{3.115629in}}%
\pgfpathlineto{\pgfqpoint{5.161491in}{3.123578in}}%
\pgfpathlineto{\pgfqpoint{5.170589in}{3.136255in}}%
\pgfpathlineto{\pgfqpoint{5.174603in}{3.138428in}}%
\pgfpathlineto{\pgfqpoint{5.178082in}{3.137822in}}%
\pgfpathlineto{\pgfqpoint{5.181829in}{3.134703in}}%
\pgfpathlineto{\pgfqpoint{5.187716in}{3.126287in}}%
\pgfpathlineto{\pgfqpoint{5.196547in}{3.114095in}}%
\pgfpathlineto{\pgfqpoint{5.200561in}{3.111979in}}%
\pgfpathlineto{\pgfqpoint{5.204040in}{3.112637in}}%
\pgfpathlineto{\pgfqpoint{5.207787in}{3.115804in}}%
\pgfpathlineto{\pgfqpoint{5.213674in}{3.124256in}}%
\pgfpathlineto{\pgfqpoint{5.222505in}{3.136392in}}%
\pgfpathlineto{\pgfqpoint{5.226519in}{3.138451in}}%
\pgfpathlineto{\pgfqpoint{5.229998in}{3.137741in}}%
\pgfpathlineto{\pgfqpoint{5.233745in}{3.134526in}}%
\pgfpathlineto{\pgfqpoint{5.239632in}{3.126038in}}%
\pgfpathlineto{\pgfqpoint{5.248196in}{3.114196in}}%
\pgfpathlineto{\pgfqpoint{5.252210in}{3.111997in}}%
\pgfpathlineto{\pgfqpoint{5.255689in}{3.112579in}}%
\pgfpathlineto{\pgfqpoint{5.259435in}{3.115676in}}%
\pgfpathlineto{\pgfqpoint{5.265322in}{3.124075in}}%
\pgfpathlineto{\pgfqpoint{5.274154in}{3.136292in}}%
\pgfpathlineto{\pgfqpoint{5.278168in}{3.138434in}}%
\pgfpathlineto{\pgfqpoint{5.281647in}{3.137800in}}%
\pgfpathlineto{\pgfqpoint{5.285393in}{3.134655in}}%
\pgfpathlineto{\pgfqpoint{5.291280in}{3.126219in}}%
\pgfpathlineto{\pgfqpoint{5.300112in}{3.114058in}}%
\pgfpathlineto{\pgfqpoint{5.304126in}{3.111973in}}%
\pgfpathlineto{\pgfqpoint{5.307605in}{3.112659in}}%
\pgfpathlineto{\pgfqpoint{5.311351in}{3.115852in}}%
\pgfpathlineto{\pgfqpoint{5.317238in}{3.124324in}}%
\pgfpathlineto{\pgfqpoint{5.325802in}{3.136191in}}%
\pgfpathlineto{\pgfqpoint{5.329816in}{3.138416in}}%
\pgfpathlineto{\pgfqpoint{5.333295in}{3.137857in}}%
\pgfpathlineto{\pgfqpoint{5.337041in}{3.134782in}}%
\pgfpathlineto{\pgfqpoint{5.342661in}{3.126830in}}%
\pgfpathlineto{\pgfqpoint{5.351760in}{3.114158in}}%
\pgfpathlineto{\pgfqpoint{5.355774in}{3.111990in}}%
\pgfpathlineto{\pgfqpoint{5.359253in}{3.112601in}}%
\pgfpathlineto{\pgfqpoint{5.362999in}{3.115724in}}%
\pgfpathlineto{\pgfqpoint{5.368887in}{3.124143in}}%
\pgfpathlineto{\pgfqpoint{5.377718in}{3.136330in}}%
\pgfpathlineto{\pgfqpoint{5.381732in}{3.138441in}}%
\pgfpathlineto{\pgfqpoint{5.385211in}{3.137778in}}%
\pgfpathlineto{\pgfqpoint{5.388957in}{3.134607in}}%
\pgfpathlineto{\pgfqpoint{5.394845in}{3.126151in}}%
\pgfpathlineto{\pgfqpoint{5.403676in}{3.114021in}}%
\pgfpathlineto{\pgfqpoint{5.407690in}{3.111967in}}%
\pgfpathlineto{\pgfqpoint{5.411169in}{3.112682in}}%
\pgfpathlineto{\pgfqpoint{5.414915in}{3.115901in}}%
\pgfpathlineto{\pgfqpoint{5.420803in}{3.124392in}}%
\pgfpathlineto{\pgfqpoint{5.429366in}{3.136229in}}%
\pgfpathlineto{\pgfqpoint{5.433380in}{3.138423in}}%
\pgfpathlineto{\pgfqpoint{5.436859in}{3.137836in}}%
\pgfpathlineto{\pgfqpoint{5.440606in}{3.134735in}}%
\pgfpathlineto{\pgfqpoint{5.446493in}{3.126333in}}%
\pgfpathlineto{\pgfqpoint{5.455324in}{3.114120in}}%
\pgfpathlineto{\pgfqpoint{5.459338in}{3.111983in}}%
\pgfpathlineto{\pgfqpoint{5.462817in}{3.112622in}}%
\pgfpathlineto{\pgfqpoint{5.466564in}{3.115772in}}%
\pgfpathlineto{\pgfqpoint{5.472451in}{3.124211in}}%
\pgfpathlineto{\pgfqpoint{5.481282in}{3.136367in}}%
\pgfpathlineto{\pgfqpoint{5.485296in}{3.138447in}}%
\pgfpathlineto{\pgfqpoint{5.488775in}{3.137756in}}%
\pgfpathlineto{\pgfqpoint{5.492522in}{3.134559in}}%
\pgfpathlineto{\pgfqpoint{5.498409in}{3.126083in}}%
\pgfpathlineto{\pgfqpoint{5.506972in}{3.114221in}}%
\pgfpathlineto{\pgfqpoint{5.510986in}{3.112002in}}%
\pgfpathlineto{\pgfqpoint{5.514465in}{3.112565in}}%
\pgfpathlineto{\pgfqpoint{5.518212in}{3.115645in}}%
\pgfpathlineto{\pgfqpoint{5.523832in}{3.123600in}}%
\pgfpathlineto{\pgfqpoint{5.532930in}{3.136267in}}%
\pgfpathlineto{\pgfqpoint{5.536944in}{3.138430in}}%
\pgfpathlineto{\pgfqpoint{5.540423in}{3.137815in}}%
\pgfpathlineto{\pgfqpoint{5.544170in}{3.134687in}}%
\pgfpathlineto{\pgfqpoint{5.550057in}{3.126265in}}%
\pgfpathlineto{\pgfqpoint{5.558888in}{3.114083in}}%
\pgfpathlineto{\pgfqpoint{5.562902in}{3.111977in}}%
\pgfpathlineto{\pgfqpoint{5.566381in}{3.112644in}}%
\pgfpathlineto{\pgfqpoint{5.570128in}{3.115820in}}%
\pgfpathlineto{\pgfqpoint{5.576015in}{3.124279in}}%
\pgfpathlineto{\pgfqpoint{5.584846in}{3.136404in}}%
\pgfpathlineto{\pgfqpoint{5.588860in}{3.138453in}}%
\pgfpathlineto{\pgfqpoint{5.592339in}{3.137733in}}%
\pgfpathlineto{\pgfqpoint{5.596353in}{3.134197in}}%
\pgfpathlineto{\pgfqpoint{5.602776in}{3.124721in}}%
\pgfpathlineto{\pgfqpoint{5.610537in}{3.114183in}}%
\pgfpathlineto{\pgfqpoint{5.614551in}{3.111995in}}%
\pgfpathlineto{\pgfqpoint{5.618030in}{3.112586in}}%
\pgfpathlineto{\pgfqpoint{5.621776in}{3.115692in}}%
\pgfpathlineto{\pgfqpoint{5.627663in}{3.124098in}}%
\pgfpathlineto{\pgfqpoint{5.636495in}{3.136305in}}%
\pgfpathlineto{\pgfqpoint{5.640509in}{3.138437in}}%
\pgfpathlineto{\pgfqpoint{5.643988in}{3.137793in}}%
\pgfpathlineto{\pgfqpoint{5.647734in}{3.134639in}}%
\pgfpathlineto{\pgfqpoint{5.653621in}{3.126197in}}%
\pgfpathlineto{\pgfqpoint{5.662453in}{3.114046in}}%
\pgfpathlineto{\pgfqpoint{5.666467in}{3.111971in}}%
\pgfpathlineto{\pgfqpoint{5.669946in}{3.112667in}}%
\pgfpathlineto{\pgfqpoint{5.673692in}{3.115868in}}%
\pgfpathlineto{\pgfqpoint{5.679579in}{3.124347in}}%
\pgfpathlineto{\pgfqpoint{5.688143in}{3.136204in}}%
\pgfpathlineto{\pgfqpoint{5.692157in}{3.138418in}}%
\pgfpathlineto{\pgfqpoint{5.695636in}{3.137850in}}%
\pgfpathlineto{\pgfqpoint{5.699382in}{3.134766in}}%
\pgfpathlineto{\pgfqpoint{5.705270in}{3.126378in}}%
\pgfpathlineto{\pgfqpoint{5.714101in}{3.114145in}}%
\pgfpathlineto{\pgfqpoint{5.718115in}{3.111988in}}%
\pgfpathlineto{\pgfqpoint{5.721594in}{3.112608in}}%
\pgfpathlineto{\pgfqpoint{5.725340in}{3.115740in}}%
\pgfpathlineto{\pgfqpoint{5.731228in}{3.124165in}}%
\pgfpathlineto{\pgfqpoint{5.740059in}{3.136342in}}%
\pgfpathlineto{\pgfqpoint{5.744073in}{3.138443in}}%
\pgfpathlineto{\pgfqpoint{5.747552in}{3.137771in}}%
\pgfpathlineto{\pgfqpoint{5.751298in}{3.134591in}}%
\pgfpathlineto{\pgfqpoint{5.757186in}{3.126129in}}%
\pgfpathlineto{\pgfqpoint{5.766017in}{3.114009in}}%
\pgfpathlineto{\pgfqpoint{5.770031in}{3.111965in}}%
\pgfpathlineto{\pgfqpoint{5.773510in}{3.112689in}}%
\pgfpathlineto{\pgfqpoint{5.777524in}{3.116230in}}%
\pgfpathlineto{\pgfqpoint{5.783946in}{3.125709in}}%
\pgfpathlineto{\pgfqpoint{5.791707in}{3.136242in}}%
\pgfpathlineto{\pgfqpoint{5.795721in}{3.138425in}}%
\pgfpathlineto{\pgfqpoint{5.799200in}{3.137829in}}%
\pgfpathlineto{\pgfqpoint{5.802947in}{3.134719in}}%
\pgfpathlineto{\pgfqpoint{5.808834in}{3.126310in}}%
\pgfpathlineto{\pgfqpoint{5.817665in}{3.114108in}}%
\pgfpathlineto{\pgfqpoint{5.821679in}{3.111981in}}%
\pgfpathlineto{\pgfqpoint{5.825158in}{3.112630in}}%
\pgfpathlineto{\pgfqpoint{5.828905in}{3.115788in}}%
\pgfpathlineto{\pgfqpoint{5.834792in}{3.124233in}}%
\pgfpathlineto{\pgfqpoint{5.843623in}{3.136379in}}%
\pgfpathlineto{\pgfqpoint{5.847637in}{3.138449in}}%
\pgfpathlineto{\pgfqpoint{5.851116in}{3.137748in}}%
\pgfpathlineto{\pgfqpoint{5.854863in}{3.134543in}}%
\pgfpathlineto{\pgfqpoint{5.860750in}{3.126061in}}%
\pgfpathlineto{\pgfqpoint{5.869313in}{3.114209in}}%
\pgfpathlineto{\pgfqpoint{5.873327in}{3.112000in}}%
\pgfpathlineto{\pgfqpoint{5.876806in}{3.112572in}}%
\pgfpathlineto{\pgfqpoint{5.880553in}{3.115660in}}%
\pgfpathlineto{\pgfqpoint{5.886440in}{3.124052in}}%
\pgfpathlineto{\pgfqpoint{5.895271in}{3.136280in}}%
\pgfpathlineto{\pgfqpoint{5.899285in}{3.138432in}}%
\pgfpathlineto{\pgfqpoint{5.902764in}{3.137807in}}%
\pgfpathlineto{\pgfqpoint{5.906511in}{3.134671in}}%
\pgfpathlineto{\pgfqpoint{5.912398in}{3.126242in}}%
\pgfpathlineto{\pgfqpoint{5.921229in}{3.114070in}}%
\pgfpathlineto{\pgfqpoint{5.925243in}{3.111975in}}%
\pgfpathlineto{\pgfqpoint{5.928722in}{3.112652in}}%
\pgfpathlineto{\pgfqpoint{5.932469in}{3.115836in}}%
\pgfpathlineto{\pgfqpoint{5.938356in}{3.124301in}}%
\pgfpathlineto{\pgfqpoint{5.946920in}{3.136178in}}%
\pgfpathlineto{\pgfqpoint{5.950934in}{3.138413in}}%
\pgfpathlineto{\pgfqpoint{5.954413in}{3.137864in}}%
\pgfpathlineto{\pgfqpoint{5.958159in}{3.134798in}}%
\pgfpathlineto{\pgfqpoint{5.963779in}{3.126852in}}%
\pgfpathlineto{\pgfqpoint{5.972878in}{3.114170in}}%
\pgfpathlineto{\pgfqpoint{5.976892in}{3.111992in}}%
\pgfpathlineto{\pgfqpoint{5.980371in}{3.112594in}}%
\pgfpathlineto{\pgfqpoint{5.984117in}{3.115708in}}%
\pgfpathlineto{\pgfqpoint{5.990004in}{3.124120in}}%
\pgfpathlineto{\pgfqpoint{5.998836in}{3.136317in}}%
\pgfpathlineto{\pgfqpoint{6.002850in}{3.138439in}}%
\pgfpathlineto{\pgfqpoint{6.006329in}{3.137786in}}%
\pgfpathlineto{\pgfqpoint{6.010075in}{3.134623in}}%
\pgfpathlineto{\pgfqpoint{6.015962in}{3.126174in}}%
\pgfpathlineto{\pgfqpoint{6.024793in}{3.114033in}}%
\pgfpathlineto{\pgfqpoint{6.028808in}{3.111969in}}%
\pgfpathlineto{\pgfqpoint{6.032287in}{3.112674in}}%
\pgfpathlineto{\pgfqpoint{6.036033in}{3.115884in}}%
\pgfpathlineto{\pgfqpoint{6.041920in}{3.124369in}}%
\pgfpathlineto{\pgfqpoint{6.050484in}{3.136217in}}%
\pgfpathlineto{\pgfqpoint{6.054498in}{3.138420in}}%
\pgfpathlineto{\pgfqpoint{6.057977in}{3.137843in}}%
\pgfpathlineto{\pgfqpoint{6.061723in}{3.134750in}}%
\pgfpathlineto{\pgfqpoint{6.067611in}{3.126355in}}%
\pgfpathlineto{\pgfqpoint{6.076442in}{3.114133in}}%
\pgfpathlineto{\pgfqpoint{6.080456in}{3.111986in}}%
\pgfpathlineto{\pgfqpoint{6.083935in}{3.112615in}}%
\pgfpathlineto{\pgfqpoint{6.087681in}{3.115756in}}%
\pgfpathlineto{\pgfqpoint{6.093569in}{3.124188in}}%
\pgfpathlineto{\pgfqpoint{6.102400in}{3.136355in}}%
\pgfpathlineto{\pgfqpoint{6.106414in}{3.138445in}}%
\pgfpathlineto{\pgfqpoint{6.109893in}{3.137763in}}%
\pgfpathlineto{\pgfqpoint{6.113639in}{3.134575in}}%
\pgfpathlineto{\pgfqpoint{6.119527in}{3.126106in}}%
\pgfpathlineto{\pgfqpoint{6.128090in}{3.114234in}}%
\pgfpathlineto{\pgfqpoint{6.132104in}{3.112005in}}%
\pgfpathlineto{\pgfqpoint{6.135583in}{3.112558in}}%
\pgfpathlineto{\pgfqpoint{6.139330in}{3.115629in}}%
\pgfpathlineto{\pgfqpoint{6.144949in}{3.123578in}}%
\pgfpathlineto{\pgfqpoint{6.154048in}{3.136255in}}%
\pgfpathlineto{\pgfqpoint{6.158062in}{3.138428in}}%
\pgfpathlineto{\pgfqpoint{6.161541in}{3.137822in}}%
\pgfpathlineto{\pgfqpoint{6.165288in}{3.134703in}}%
\pgfpathlineto{\pgfqpoint{6.171175in}{3.126287in}}%
\pgfpathlineto{\pgfqpoint{6.180006in}{3.114095in}}%
\pgfpathlineto{\pgfqpoint{6.184020in}{3.111979in}}%
\pgfpathlineto{\pgfqpoint{6.187499in}{3.112637in}}%
\pgfpathlineto{\pgfqpoint{6.191246in}{3.115804in}}%
\pgfpathlineto{\pgfqpoint{6.197133in}{3.124256in}}%
\pgfpathlineto{\pgfqpoint{6.205964in}{3.136392in}}%
\pgfpathlineto{\pgfqpoint{6.209978in}{3.138451in}}%
\pgfpathlineto{\pgfqpoint{6.213457in}{3.137741in}}%
\pgfpathlineto{\pgfqpoint{6.217204in}{3.134526in}}%
\pgfpathlineto{\pgfqpoint{6.223091in}{3.126038in}}%
\pgfpathlineto{\pgfqpoint{6.231654in}{3.114196in}}%
\pgfpathlineto{\pgfqpoint{6.235668in}{3.111997in}}%
\pgfpathlineto{\pgfqpoint{6.239147in}{3.112579in}}%
\pgfpathlineto{\pgfqpoint{6.242894in}{3.115676in}}%
\pgfpathlineto{\pgfqpoint{6.248781in}{3.124075in}}%
\pgfpathlineto{\pgfqpoint{6.257612in}{3.136292in}}%
\pgfpathlineto{\pgfqpoint{6.261626in}{3.138434in}}%
\pgfpathlineto{\pgfqpoint{6.265105in}{3.137800in}}%
\pgfpathlineto{\pgfqpoint{6.268852in}{3.134655in}}%
\pgfpathlineto{\pgfqpoint{6.274739in}{3.126219in}}%
\pgfpathlineto{\pgfqpoint{6.283570in}{3.114058in}}%
\pgfpathlineto{\pgfqpoint{6.287584in}{3.111973in}}%
\pgfpathlineto{\pgfqpoint{6.291063in}{3.112659in}}%
\pgfpathlineto{\pgfqpoint{6.294810in}{3.115852in}}%
\pgfpathlineto{\pgfqpoint{6.300697in}{3.124324in}}%
\pgfpathlineto{\pgfqpoint{6.309261in}{3.136191in}}%
\pgfpathlineto{\pgfqpoint{6.313275in}{3.138416in}}%
\pgfpathlineto{\pgfqpoint{6.316754in}{3.137857in}}%
\pgfpathlineto{\pgfqpoint{6.320500in}{3.134782in}}%
\pgfpathlineto{\pgfqpoint{6.326120in}{3.126830in}}%
\pgfpathlineto{\pgfqpoint{6.335219in}{3.114158in}}%
\pgfpathlineto{\pgfqpoint{6.339233in}{3.111990in}}%
\pgfpathlineto{\pgfqpoint{6.342712in}{3.112601in}}%
\pgfpathlineto{\pgfqpoint{6.346458in}{3.115724in}}%
\pgfpathlineto{\pgfqpoint{6.352345in}{3.124143in}}%
\pgfpathlineto{\pgfqpoint{6.361177in}{3.136330in}}%
\pgfpathlineto{\pgfqpoint{6.365191in}{3.138441in}}%
\pgfpathlineto{\pgfqpoint{6.368670in}{3.137778in}}%
\pgfpathlineto{\pgfqpoint{6.372416in}{3.134607in}}%
\pgfpathlineto{\pgfqpoint{6.378303in}{3.126151in}}%
\pgfpathlineto{\pgfqpoint{6.387134in}{3.114021in}}%
\pgfpathlineto{\pgfqpoint{6.391149in}{3.111967in}}%
\pgfpathlineto{\pgfqpoint{6.394628in}{3.112682in}}%
\pgfpathlineto{\pgfqpoint{6.398374in}{3.115901in}}%
\pgfpathlineto{\pgfqpoint{6.404261in}{3.124392in}}%
\pgfpathlineto{\pgfqpoint{6.412825in}{3.136229in}}%
\pgfpathlineto{\pgfqpoint{6.416839in}{3.138423in}}%
\pgfpathlineto{\pgfqpoint{6.420318in}{3.137836in}}%
\pgfpathlineto{\pgfqpoint{6.424064in}{3.134735in}}%
\pgfpathlineto{\pgfqpoint{6.429952in}{3.126333in}}%
\pgfpathlineto{\pgfqpoint{6.438783in}{3.114120in}}%
\pgfpathlineto{\pgfqpoint{6.442797in}{3.111983in}}%
\pgfpathlineto{\pgfqpoint{6.446276in}{3.112622in}}%
\pgfpathlineto{\pgfqpoint{6.450022in}{3.115772in}}%
\pgfpathlineto{\pgfqpoint{6.455910in}{3.124211in}}%
\pgfpathlineto{\pgfqpoint{6.464741in}{3.136367in}}%
\pgfpathlineto{\pgfqpoint{6.468755in}{3.138447in}}%
\pgfpathlineto{\pgfqpoint{6.472234in}{3.137756in}}%
\pgfpathlineto{\pgfqpoint{6.475980in}{3.134559in}}%
\pgfpathlineto{\pgfqpoint{6.481868in}{3.126083in}}%
\pgfpathlineto{\pgfqpoint{6.490431in}{3.114221in}}%
\pgfpathlineto{\pgfqpoint{6.494445in}{3.112002in}}%
\pgfpathlineto{\pgfqpoint{6.497924in}{3.112565in}}%
\pgfpathlineto{\pgfqpoint{6.501671in}{3.115645in}}%
\pgfpathlineto{\pgfqpoint{6.507290in}{3.123600in}}%
\pgfpathlineto{\pgfqpoint{6.516389in}{3.136267in}}%
\pgfpathlineto{\pgfqpoint{6.520403in}{3.138430in}}%
\pgfpathlineto{\pgfqpoint{6.523882in}{3.137815in}}%
\pgfpathlineto{\pgfqpoint{6.527629in}{3.134687in}}%
\pgfpathlineto{\pgfqpoint{6.533516in}{3.126265in}}%
\pgfpathlineto{\pgfqpoint{6.542347in}{3.114083in}}%
\pgfpathlineto{\pgfqpoint{6.546361in}{3.111977in}}%
\pgfpathlineto{\pgfqpoint{6.549840in}{3.112644in}}%
\pgfpathlineto{\pgfqpoint{6.553587in}{3.115820in}}%
\pgfpathlineto{\pgfqpoint{6.559474in}{3.124279in}}%
\pgfpathlineto{\pgfqpoint{6.568305in}{3.136404in}}%
\pgfpathlineto{\pgfqpoint{6.572319in}{3.138453in}}%
\pgfpathlineto{\pgfqpoint{6.575798in}{3.137733in}}%
\pgfpathlineto{\pgfqpoint{6.579812in}{3.134197in}}%
\pgfpathlineto{\pgfqpoint{6.586235in}{3.124721in}}%
\pgfpathlineto{\pgfqpoint{6.593995in}{3.114183in}}%
\pgfpathlineto{\pgfqpoint{6.598009in}{3.111995in}}%
\pgfpathlineto{\pgfqpoint{6.601488in}{3.112586in}}%
\pgfpathlineto{\pgfqpoint{6.605235in}{3.115692in}}%
\pgfpathlineto{\pgfqpoint{6.611122in}{3.124098in}}%
\pgfpathlineto{\pgfqpoint{6.619953in}{3.136305in}}%
\pgfpathlineto{\pgfqpoint{6.623967in}{3.138437in}}%
\pgfpathlineto{\pgfqpoint{6.627446in}{3.137793in}}%
\pgfpathlineto{\pgfqpoint{6.631193in}{3.134639in}}%
\pgfpathlineto{\pgfqpoint{6.637080in}{3.126197in}}%
\pgfpathlineto{\pgfqpoint{6.645911in}{3.114046in}}%
\pgfpathlineto{\pgfqpoint{6.649925in}{3.111971in}}%
\pgfpathlineto{\pgfqpoint{6.653404in}{3.112667in}}%
\pgfpathlineto{\pgfqpoint{6.657151in}{3.115868in}}%
\pgfpathlineto{\pgfqpoint{6.663038in}{3.124347in}}%
\pgfpathlineto{\pgfqpoint{6.663306in}{3.124778in}}%
\pgfpathlineto{\pgfqpoint{6.663306in}{3.124778in}}%
\pgfusepath{stroke}%
\end{pgfscope}%
\begin{pgfscope}%
\pgfpathrectangle{\pgfqpoint{0.467797in}{2.292089in}}{\pgfqpoint{6.490533in}{1.666241in}}%
\pgfusepath{clip}%
\pgfsetrectcap%
\pgfsetroundjoin%
\pgfsetlinewidth{1.505625pt}%
\definecolor{currentstroke}{rgb}{0.090196,0.745098,0.811765}%
\pgfsetstrokecolor{currentstroke}%
\pgfsetdash{}{0pt}%
\pgfpathmoveto{\pgfqpoint{0.762821in}{3.125209in}}%
\pgfpathlineto{\pgfqpoint{0.770849in}{3.136072in}}%
\pgfpathlineto{\pgfqpoint{0.774863in}{3.138025in}}%
\pgfpathlineto{\pgfqpoint{0.778075in}{3.137287in}}%
\pgfpathlineto{\pgfqpoint{0.781821in}{3.134000in}}%
\pgfpathlineto{\pgfqpoint{0.787976in}{3.124964in}}%
\pgfpathlineto{\pgfqpoint{0.795737in}{3.114448in}}%
\pgfpathlineto{\pgfqpoint{0.799751in}{3.112406in}}%
\pgfpathlineto{\pgfqpoint{0.802962in}{3.113070in}}%
\pgfpathlineto{\pgfqpoint{0.806709in}{3.116284in}}%
\pgfpathlineto{\pgfqpoint{0.812596in}{3.124836in}}%
\pgfpathlineto{\pgfqpoint{0.820624in}{3.135868in}}%
\pgfpathlineto{\pgfqpoint{0.824638in}{3.137998in}}%
\pgfpathlineto{\pgfqpoint{0.827850in}{3.137408in}}%
\pgfpathlineto{\pgfqpoint{0.831596in}{3.134268in}}%
\pgfpathlineto{\pgfqpoint{0.837483in}{3.125769in}}%
\pgfpathlineto{\pgfqpoint{0.845779in}{3.114416in}}%
\pgfpathlineto{\pgfqpoint{0.849793in}{3.112402in}}%
\pgfpathlineto{\pgfqpoint{0.853005in}{3.113089in}}%
\pgfpathlineto{\pgfqpoint{0.856751in}{3.116326in}}%
\pgfpathlineto{\pgfqpoint{0.862906in}{3.125326in}}%
\pgfpathlineto{\pgfqpoint{0.870667in}{3.135901in}}%
\pgfpathlineto{\pgfqpoint{0.874681in}{3.138003in}}%
\pgfpathlineto{\pgfqpoint{0.877892in}{3.137390in}}%
\pgfpathlineto{\pgfqpoint{0.881639in}{3.134226in}}%
\pgfpathlineto{\pgfqpoint{0.887526in}{3.125710in}}%
\pgfpathlineto{\pgfqpoint{0.895822in}{3.114385in}}%
\pgfpathlineto{\pgfqpoint{0.899836in}{3.112398in}}%
\pgfpathlineto{\pgfqpoint{0.903047in}{3.113109in}}%
\pgfpathlineto{\pgfqpoint{0.906794in}{3.116369in}}%
\pgfpathlineto{\pgfqpoint{0.912949in}{3.125385in}}%
\pgfpathlineto{\pgfqpoint{0.920710in}{3.135933in}}%
\pgfpathlineto{\pgfqpoint{0.924724in}{3.138008in}}%
\pgfpathlineto{\pgfqpoint{0.927935in}{3.137371in}}%
\pgfpathlineto{\pgfqpoint{0.931681in}{3.134184in}}%
\pgfpathlineto{\pgfqpoint{0.937569in}{3.125652in}}%
\pgfpathlineto{\pgfqpoint{0.945865in}{3.114353in}}%
\pgfpathlineto{\pgfqpoint{0.949879in}{3.112394in}}%
\pgfpathlineto{\pgfqpoint{0.953090in}{3.113129in}}%
\pgfpathlineto{\pgfqpoint{0.956837in}{3.116411in}}%
\pgfpathlineto{\pgfqpoint{0.962992in}{3.125444in}}%
\pgfpathlineto{\pgfqpoint{0.970752in}{3.135965in}}%
\pgfpathlineto{\pgfqpoint{0.974766in}{3.138012in}}%
\pgfpathlineto{\pgfqpoint{0.977978in}{3.137352in}}%
\pgfpathlineto{\pgfqpoint{0.981724in}{3.134143in}}%
\pgfpathlineto{\pgfqpoint{0.987611in}{3.125593in}}%
\pgfpathlineto{\pgfqpoint{0.995907in}{3.114322in}}%
\pgfpathlineto{\pgfqpoint{0.999921in}{3.112391in}}%
\pgfpathlineto{\pgfqpoint{1.003133in}{3.113148in}}%
\pgfpathlineto{\pgfqpoint{1.006879in}{3.116454in}}%
\pgfpathlineto{\pgfqpoint{1.013034in}{3.125502in}}%
\pgfpathlineto{\pgfqpoint{1.020795in}{3.135997in}}%
\pgfpathlineto{\pgfqpoint{1.024809in}{3.138016in}}%
\pgfpathlineto{\pgfqpoint{1.028020in}{3.137333in}}%
\pgfpathlineto{\pgfqpoint{1.031767in}{3.134100in}}%
\pgfpathlineto{\pgfqpoint{1.037654in}{3.125535in}}%
\pgfpathlineto{\pgfqpoint{1.045682in}{3.114524in}}%
\pgfpathlineto{\pgfqpoint{1.049696in}{3.112417in}}%
\pgfpathlineto{\pgfqpoint{1.052908in}{3.113026in}}%
\pgfpathlineto{\pgfqpoint{1.056654in}{3.116185in}}%
\pgfpathlineto{\pgfqpoint{1.062542in}{3.124697in}}%
\pgfpathlineto{\pgfqpoint{1.070838in}{3.136028in}}%
\pgfpathlineto{\pgfqpoint{1.074852in}{3.138020in}}%
\pgfpathlineto{\pgfqpoint{1.078063in}{3.137314in}}%
\pgfpathlineto{\pgfqpoint{1.081809in}{3.134058in}}%
\pgfpathlineto{\pgfqpoint{1.087964in}{3.125045in}}%
\pgfpathlineto{\pgfqpoint{1.095725in}{3.114492in}}%
\pgfpathlineto{\pgfqpoint{1.099739in}{3.112412in}}%
\pgfpathlineto{\pgfqpoint{1.102950in}{3.113044in}}%
\pgfpathlineto{\pgfqpoint{1.106697in}{3.116227in}}%
\pgfpathlineto{\pgfqpoint{1.112584in}{3.124756in}}%
\pgfpathlineto{\pgfqpoint{1.120880in}{3.136060in}}%
\pgfpathlineto{\pgfqpoint{1.124894in}{3.138024in}}%
\pgfpathlineto{\pgfqpoint{1.128106in}{3.137294in}}%
\pgfpathlineto{\pgfqpoint{1.131852in}{3.134016in}}%
\pgfpathlineto{\pgfqpoint{1.138007in}{3.124986in}}%
\pgfpathlineto{\pgfqpoint{1.145768in}{3.114460in}}%
\pgfpathlineto{\pgfqpoint{1.149782in}{3.112408in}}%
\pgfpathlineto{\pgfqpoint{1.152993in}{3.113063in}}%
\pgfpathlineto{\pgfqpoint{1.156740in}{3.116268in}}%
\pgfpathlineto{\pgfqpoint{1.162627in}{3.124814in}}%
\pgfpathlineto{\pgfqpoint{1.170923in}{3.136091in}}%
\pgfpathlineto{\pgfqpoint{1.174937in}{3.138027in}}%
\pgfpathlineto{\pgfqpoint{1.178148in}{3.137274in}}%
\pgfpathlineto{\pgfqpoint{1.181895in}{3.133973in}}%
\pgfpathlineto{\pgfqpoint{1.188050in}{3.124928in}}%
\pgfpathlineto{\pgfqpoint{1.195810in}{3.114428in}}%
\pgfpathlineto{\pgfqpoint{1.199824in}{3.112403in}}%
\pgfpathlineto{\pgfqpoint{1.203036in}{3.113082in}}%
\pgfpathlineto{\pgfqpoint{1.206782in}{3.116311in}}%
\pgfpathlineto{\pgfqpoint{1.212670in}{3.124873in}}%
\pgfpathlineto{\pgfqpoint{1.220698in}{3.135888in}}%
\pgfpathlineto{\pgfqpoint{1.224712in}{3.138001in}}%
\pgfpathlineto{\pgfqpoint{1.227923in}{3.137396in}}%
\pgfpathlineto{\pgfqpoint{1.231670in}{3.134242in}}%
\pgfpathlineto{\pgfqpoint{1.237557in}{3.125732in}}%
\pgfpathlineto{\pgfqpoint{1.245853in}{3.114396in}}%
\pgfpathlineto{\pgfqpoint{1.249867in}{3.112399in}}%
\pgfpathlineto{\pgfqpoint{1.253078in}{3.113102in}}%
\pgfpathlineto{\pgfqpoint{1.256825in}{3.116353in}}%
\pgfpathlineto{\pgfqpoint{1.262980in}{3.125363in}}%
\pgfpathlineto{\pgfqpoint{1.270741in}{3.135921in}}%
\pgfpathlineto{\pgfqpoint{1.274755in}{3.138006in}}%
\pgfpathlineto{\pgfqpoint{1.277966in}{3.137378in}}%
\pgfpathlineto{\pgfqpoint{1.281712in}{3.134200in}}%
\pgfpathlineto{\pgfqpoint{1.287600in}{3.125674in}}%
\pgfpathlineto{\pgfqpoint{1.295896in}{3.114365in}}%
\pgfpathlineto{\pgfqpoint{1.299910in}{3.112396in}}%
\pgfpathlineto{\pgfqpoint{1.303121in}{3.113121in}}%
\pgfpathlineto{\pgfqpoint{1.306868in}{3.116395in}}%
\pgfpathlineto{\pgfqpoint{1.313023in}{3.125422in}}%
\pgfpathlineto{\pgfqpoint{1.320783in}{3.135953in}}%
\pgfpathlineto{\pgfqpoint{1.324797in}{3.138011in}}%
\pgfpathlineto{\pgfqpoint{1.328009in}{3.137359in}}%
\pgfpathlineto{\pgfqpoint{1.331755in}{3.134158in}}%
\pgfpathlineto{\pgfqpoint{1.337643in}{3.125615in}}%
\pgfpathlineto{\pgfqpoint{1.345938in}{3.114334in}}%
\pgfpathlineto{\pgfqpoint{1.349952in}{3.112392in}}%
\pgfpathlineto{\pgfqpoint{1.353164in}{3.113141in}}%
\pgfpathlineto{\pgfqpoint{1.356910in}{3.116438in}}%
\pgfpathlineto{\pgfqpoint{1.363065in}{3.125480in}}%
\pgfpathlineto{\pgfqpoint{1.370826in}{3.135985in}}%
\pgfpathlineto{\pgfqpoint{1.374840in}{3.138015in}}%
\pgfpathlineto{\pgfqpoint{1.378051in}{3.137340in}}%
\pgfpathlineto{\pgfqpoint{1.381798in}{3.134116in}}%
\pgfpathlineto{\pgfqpoint{1.387685in}{3.125557in}}%
\pgfpathlineto{\pgfqpoint{1.395713in}{3.114537in}}%
\pgfpathlineto{\pgfqpoint{1.399728in}{3.112419in}}%
\pgfpathlineto{\pgfqpoint{1.402939in}{3.113019in}}%
\pgfpathlineto{\pgfqpoint{1.406685in}{3.116169in}}%
\pgfpathlineto{\pgfqpoint{1.412573in}{3.124676in}}%
\pgfpathlineto{\pgfqpoint{1.420869in}{3.136017in}}%
\pgfpathlineto{\pgfqpoint{1.424883in}{3.138019in}}%
\pgfpathlineto{\pgfqpoint{1.428094in}{3.137321in}}%
\pgfpathlineto{\pgfqpoint{1.431840in}{3.134074in}}%
\pgfpathlineto{\pgfqpoint{1.437995in}{3.125067in}}%
\pgfpathlineto{\pgfqpoint{1.445756in}{3.114504in}}%
\pgfpathlineto{\pgfqpoint{1.449770in}{3.112414in}}%
\pgfpathlineto{\pgfqpoint{1.452981in}{3.113037in}}%
\pgfpathlineto{\pgfqpoint{1.456728in}{3.116211in}}%
\pgfpathlineto{\pgfqpoint{1.462615in}{3.124734in}}%
\pgfpathlineto{\pgfqpoint{1.470911in}{3.136048in}}%
\pgfpathlineto{\pgfqpoint{1.474925in}{3.138022in}}%
\pgfpathlineto{\pgfqpoint{1.478137in}{3.137301in}}%
\pgfpathlineto{\pgfqpoint{1.481883in}{3.134032in}}%
\pgfpathlineto{\pgfqpoint{1.488038in}{3.125008in}}%
\pgfpathlineto{\pgfqpoint{1.495799in}{3.114472in}}%
\pgfpathlineto{\pgfqpoint{1.499813in}{3.112409in}}%
\pgfpathlineto{\pgfqpoint{1.503024in}{3.113056in}}%
\pgfpathlineto{\pgfqpoint{1.506771in}{3.116253in}}%
\pgfpathlineto{\pgfqpoint{1.512658in}{3.124793in}}%
\pgfpathlineto{\pgfqpoint{1.520954in}{3.136079in}}%
\pgfpathlineto{\pgfqpoint{1.524968in}{3.138026in}}%
\pgfpathlineto{\pgfqpoint{1.528179in}{3.137282in}}%
\pgfpathlineto{\pgfqpoint{1.531926in}{3.133989in}}%
\pgfpathlineto{\pgfqpoint{1.538081in}{3.124950in}}%
\pgfpathlineto{\pgfqpoint{1.545841in}{3.114440in}}%
\pgfpathlineto{\pgfqpoint{1.549856in}{3.112405in}}%
\pgfpathlineto{\pgfqpoint{1.553067in}{3.113075in}}%
\pgfpathlineto{\pgfqpoint{1.556813in}{3.116295in}}%
\pgfpathlineto{\pgfqpoint{1.562701in}{3.124851in}}%
\pgfpathlineto{\pgfqpoint{1.570729in}{3.135876in}}%
\pgfpathlineto{\pgfqpoint{1.574743in}{3.137999in}}%
\pgfpathlineto{\pgfqpoint{1.577954in}{3.137403in}}%
\pgfpathlineto{\pgfqpoint{1.581701in}{3.134257in}}%
\pgfpathlineto{\pgfqpoint{1.587588in}{3.125754in}}%
\pgfpathlineto{\pgfqpoint{1.595884in}{3.114408in}}%
\pgfpathlineto{\pgfqpoint{1.599898in}{3.112401in}}%
\pgfpathlineto{\pgfqpoint{1.603109in}{3.113094in}}%
\pgfpathlineto{\pgfqpoint{1.606856in}{3.116337in}}%
\pgfpathlineto{\pgfqpoint{1.613011in}{3.125341in}}%
\pgfpathlineto{\pgfqpoint{1.620772in}{3.135909in}}%
\pgfpathlineto{\pgfqpoint{1.624786in}{3.138004in}}%
\pgfpathlineto{\pgfqpoint{1.627997in}{3.137385in}}%
\pgfpathlineto{\pgfqpoint{1.631744in}{3.134216in}}%
\pgfpathlineto{\pgfqpoint{1.637631in}{3.125696in}}%
\pgfpathlineto{\pgfqpoint{1.645927in}{3.114377in}}%
\pgfpathlineto{\pgfqpoint{1.649941in}{3.112397in}}%
\pgfpathlineto{\pgfqpoint{1.653152in}{3.113114in}}%
\pgfpathlineto{\pgfqpoint{1.656899in}{3.116379in}}%
\pgfpathlineto{\pgfqpoint{1.663054in}{3.125400in}}%
\pgfpathlineto{\pgfqpoint{1.670814in}{3.135941in}}%
\pgfpathlineto{\pgfqpoint{1.674828in}{3.138009in}}%
\pgfpathlineto{\pgfqpoint{1.678040in}{3.137366in}}%
\pgfpathlineto{\pgfqpoint{1.681786in}{3.134174in}}%
\pgfpathlineto{\pgfqpoint{1.687674in}{3.125637in}}%
\pgfpathlineto{\pgfqpoint{1.695969in}{3.114345in}}%
\pgfpathlineto{\pgfqpoint{1.699983in}{3.112394in}}%
\pgfpathlineto{\pgfqpoint{1.703195in}{3.113134in}}%
\pgfpathlineto{\pgfqpoint{1.706941in}{3.116422in}}%
\pgfpathlineto{\pgfqpoint{1.713096in}{3.125458in}}%
\pgfpathlineto{\pgfqpoint{1.720857in}{3.135973in}}%
\pgfpathlineto{\pgfqpoint{1.724871in}{3.138013in}}%
\pgfpathlineto{\pgfqpoint{1.728082in}{3.137347in}}%
\pgfpathlineto{\pgfqpoint{1.731829in}{3.134132in}}%
\pgfpathlineto{\pgfqpoint{1.737716in}{3.125579in}}%
\pgfpathlineto{\pgfqpoint{1.745744in}{3.114549in}}%
\pgfpathlineto{\pgfqpoint{1.749759in}{3.112420in}}%
\pgfpathlineto{\pgfqpoint{1.752970in}{3.113012in}}%
\pgfpathlineto{\pgfqpoint{1.756716in}{3.116154in}}%
\pgfpathlineto{\pgfqpoint{1.762604in}{3.124654in}}%
\pgfpathlineto{\pgfqpoint{1.770900in}{3.136005in}}%
\pgfpathlineto{\pgfqpoint{1.774914in}{3.138017in}}%
\pgfpathlineto{\pgfqpoint{1.778125in}{3.137328in}}%
\pgfpathlineto{\pgfqpoint{1.781871in}{3.134090in}}%
\pgfpathlineto{\pgfqpoint{1.788026in}{3.125089in}}%
\pgfpathlineto{\pgfqpoint{1.795787in}{3.114516in}}%
\pgfpathlineto{\pgfqpoint{1.799801in}{3.112415in}}%
\pgfpathlineto{\pgfqpoint{1.803013in}{3.113030in}}%
\pgfpathlineto{\pgfqpoint{1.806759in}{3.116195in}}%
\pgfpathlineto{\pgfqpoint{1.812646in}{3.124712in}}%
\pgfpathlineto{\pgfqpoint{1.820942in}{3.136036in}}%
\pgfpathlineto{\pgfqpoint{1.824956in}{3.138021in}}%
\pgfpathlineto{\pgfqpoint{1.828168in}{3.137309in}}%
\pgfpathlineto{\pgfqpoint{1.831914in}{3.134047in}}%
\pgfpathlineto{\pgfqpoint{1.838069in}{3.125030in}}%
\pgfpathlineto{\pgfqpoint{1.845830in}{3.114484in}}%
\pgfpathlineto{\pgfqpoint{1.849844in}{3.112411in}}%
\pgfpathlineto{\pgfqpoint{1.853055in}{3.113049in}}%
\pgfpathlineto{\pgfqpoint{1.856802in}{3.116237in}}%
\pgfpathlineto{\pgfqpoint{1.862689in}{3.124771in}}%
\pgfpathlineto{\pgfqpoint{1.870985in}{3.136068in}}%
\pgfpathlineto{\pgfqpoint{1.874999in}{3.138025in}}%
\pgfpathlineto{\pgfqpoint{1.878210in}{3.137289in}}%
\pgfpathlineto{\pgfqpoint{1.881957in}{3.134005in}}%
\pgfpathlineto{\pgfqpoint{1.888112in}{3.124972in}}%
\pgfpathlineto{\pgfqpoint{1.895872in}{3.114452in}}%
\pgfpathlineto{\pgfqpoint{1.899887in}{3.112406in}}%
\pgfpathlineto{\pgfqpoint{1.903098in}{3.113068in}}%
\pgfpathlineto{\pgfqpoint{1.906844in}{3.116279in}}%
\pgfpathlineto{\pgfqpoint{1.912732in}{3.124829in}}%
\pgfpathlineto{\pgfqpoint{1.920760in}{3.135864in}}%
\pgfpathlineto{\pgfqpoint{1.924774in}{3.137997in}}%
\pgfpathlineto{\pgfqpoint{1.927985in}{3.137410in}}%
\pgfpathlineto{\pgfqpoint{1.931732in}{3.134273in}}%
\pgfpathlineto{\pgfqpoint{1.937619in}{3.125776in}}%
\pgfpathlineto{\pgfqpoint{1.945915in}{3.114420in}}%
\pgfpathlineto{\pgfqpoint{1.949929in}{3.112402in}}%
\pgfpathlineto{\pgfqpoint{1.953140in}{3.113087in}}%
\pgfpathlineto{\pgfqpoint{1.956887in}{3.116321in}}%
\pgfpathlineto{\pgfqpoint{1.962774in}{3.124888in}}%
\pgfpathlineto{\pgfqpoint{1.970803in}{3.135897in}}%
\pgfpathlineto{\pgfqpoint{1.974817in}{3.138002in}}%
\pgfpathlineto{\pgfqpoint{1.978028in}{3.137392in}}%
\pgfpathlineto{\pgfqpoint{1.981775in}{3.134231in}}%
\pgfpathlineto{\pgfqpoint{1.987662in}{3.125718in}}%
\pgfpathlineto{\pgfqpoint{1.995958in}{3.114388in}}%
\pgfpathlineto{\pgfqpoint{1.999972in}{3.112398in}}%
\pgfpathlineto{\pgfqpoint{2.003183in}{3.113106in}}%
\pgfpathlineto{\pgfqpoint{2.006930in}{3.116363in}}%
\pgfpathlineto{\pgfqpoint{2.013085in}{3.125378in}}%
\pgfpathlineto{\pgfqpoint{2.020845in}{3.135929in}}%
\pgfpathlineto{\pgfqpoint{2.024859in}{3.138007in}}%
\pgfpathlineto{\pgfqpoint{2.028071in}{3.137373in}}%
\pgfpathlineto{\pgfqpoint{2.031817in}{3.134190in}}%
\pgfpathlineto{\pgfqpoint{2.037705in}{3.125659in}}%
\pgfpathlineto{\pgfqpoint{2.046000in}{3.114357in}}%
\pgfpathlineto{\pgfqpoint{2.050015in}{3.112395in}}%
\pgfpathlineto{\pgfqpoint{2.053226in}{3.113126in}}%
\pgfpathlineto{\pgfqpoint{2.056972in}{3.116406in}}%
\pgfpathlineto{\pgfqpoint{2.063127in}{3.125436in}}%
\pgfpathlineto{\pgfqpoint{2.070888in}{3.135961in}}%
\pgfpathlineto{\pgfqpoint{2.074902in}{3.138012in}}%
\pgfpathlineto{\pgfqpoint{2.078113in}{3.137354in}}%
\pgfpathlineto{\pgfqpoint{2.081860in}{3.134148in}}%
\pgfpathlineto{\pgfqpoint{2.087747in}{3.125601in}}%
\pgfpathlineto{\pgfqpoint{2.096043in}{3.114326in}}%
\pgfpathlineto{\pgfqpoint{2.100057in}{3.112392in}}%
\pgfpathlineto{\pgfqpoint{2.103268in}{3.113146in}}%
\pgfpathlineto{\pgfqpoint{2.107015in}{3.116449in}}%
\pgfpathlineto{\pgfqpoint{2.113170in}{3.125495in}}%
\pgfpathlineto{\pgfqpoint{2.120931in}{3.135993in}}%
\pgfpathlineto{\pgfqpoint{2.124945in}{3.138016in}}%
\pgfpathlineto{\pgfqpoint{2.128156in}{3.137335in}}%
\pgfpathlineto{\pgfqpoint{2.131903in}{3.134106in}}%
\pgfpathlineto{\pgfqpoint{2.137790in}{3.125542in}}%
\pgfpathlineto{\pgfqpoint{2.145818in}{3.114528in}}%
\pgfpathlineto{\pgfqpoint{2.149832in}{3.112417in}}%
\pgfpathlineto{\pgfqpoint{2.153044in}{3.113024in}}%
\pgfpathlineto{\pgfqpoint{2.156790in}{3.116180in}}%
\pgfpathlineto{\pgfqpoint{2.162677in}{3.124690in}}%
\pgfpathlineto{\pgfqpoint{2.170973in}{3.136024in}}%
\pgfpathlineto{\pgfqpoint{2.174987in}{3.138020in}}%
\pgfpathlineto{\pgfqpoint{2.178199in}{3.137316in}}%
\pgfpathlineto{\pgfqpoint{2.181945in}{3.134063in}}%
\pgfpathlineto{\pgfqpoint{2.188100in}{3.125052in}}%
\pgfpathlineto{\pgfqpoint{2.195861in}{3.114496in}}%
\pgfpathlineto{\pgfqpoint{2.199875in}{3.112413in}}%
\pgfpathlineto{\pgfqpoint{2.203086in}{3.113042in}}%
\pgfpathlineto{\pgfqpoint{2.206833in}{3.116221in}}%
\pgfpathlineto{\pgfqpoint{2.212720in}{3.124749in}}%
\pgfpathlineto{\pgfqpoint{2.221016in}{3.136056in}}%
\pgfpathlineto{\pgfqpoint{2.225030in}{3.138023in}}%
\pgfpathlineto{\pgfqpoint{2.228241in}{3.137296in}}%
\pgfpathlineto{\pgfqpoint{2.231988in}{3.134021in}}%
\pgfpathlineto{\pgfqpoint{2.238143in}{3.124994in}}%
\pgfpathlineto{\pgfqpoint{2.245903in}{3.114464in}}%
\pgfpathlineto{\pgfqpoint{2.249918in}{3.112408in}}%
\pgfpathlineto{\pgfqpoint{2.253129in}{3.113061in}}%
\pgfpathlineto{\pgfqpoint{2.256875in}{3.116263in}}%
\pgfpathlineto{\pgfqpoint{2.262763in}{3.124807in}}%
\pgfpathlineto{\pgfqpoint{2.271059in}{3.136087in}}%
\pgfpathlineto{\pgfqpoint{2.275073in}{3.138027in}}%
\pgfpathlineto{\pgfqpoint{2.278284in}{3.137277in}}%
\pgfpathlineto{\pgfqpoint{2.282031in}{3.133978in}}%
\pgfpathlineto{\pgfqpoint{2.288185in}{3.124935in}}%
\pgfpathlineto{\pgfqpoint{2.295946in}{3.114432in}}%
\pgfpathlineto{\pgfqpoint{2.299960in}{3.112404in}}%
\pgfpathlineto{\pgfqpoint{2.303172in}{3.113080in}}%
\pgfpathlineto{\pgfqpoint{2.306918in}{3.116305in}}%
\pgfpathlineto{\pgfqpoint{2.312805in}{3.124866in}}%
\pgfpathlineto{\pgfqpoint{2.320834in}{3.135884in}}%
\pgfpathlineto{\pgfqpoint{2.324848in}{3.138001in}}%
\pgfpathlineto{\pgfqpoint{2.328059in}{3.137399in}}%
\pgfpathlineto{\pgfqpoint{2.331806in}{3.134247in}}%
\pgfpathlineto{\pgfqpoint{2.337693in}{3.125740in}}%
\pgfpathlineto{\pgfqpoint{2.345989in}{3.114400in}}%
\pgfpathlineto{\pgfqpoint{2.350003in}{3.112400in}}%
\pgfpathlineto{\pgfqpoint{2.353214in}{3.113099in}}%
\pgfpathlineto{\pgfqpoint{2.356961in}{3.116348in}}%
\pgfpathlineto{\pgfqpoint{2.363116in}{3.125356in}}%
\pgfpathlineto{\pgfqpoint{2.370876in}{3.135917in}}%
\pgfpathlineto{\pgfqpoint{2.374890in}{3.138005in}}%
\pgfpathlineto{\pgfqpoint{2.378102in}{3.137380in}}%
\pgfpathlineto{\pgfqpoint{2.381848in}{3.134205in}}%
\pgfpathlineto{\pgfqpoint{2.387736in}{3.125681in}}%
\pgfpathlineto{\pgfqpoint{2.396031in}{3.114369in}}%
\pgfpathlineto{\pgfqpoint{2.400046in}{3.112396in}}%
\pgfpathlineto{\pgfqpoint{2.403257in}{3.113119in}}%
\pgfpathlineto{\pgfqpoint{2.407003in}{3.116390in}}%
\pgfpathlineto{\pgfqpoint{2.413158in}{3.125414in}}%
\pgfpathlineto{\pgfqpoint{2.420919in}{3.135949in}}%
\pgfpathlineto{\pgfqpoint{2.424933in}{3.138010in}}%
\pgfpathlineto{\pgfqpoint{2.428144in}{3.137362in}}%
\pgfpathlineto{\pgfqpoint{2.431891in}{3.134164in}}%
\pgfpathlineto{\pgfqpoint{2.437778in}{3.125623in}}%
\pgfpathlineto{\pgfqpoint{2.446074in}{3.114338in}}%
\pgfpathlineto{\pgfqpoint{2.450088in}{3.112393in}}%
\pgfpathlineto{\pgfqpoint{2.453300in}{3.113138in}}%
\pgfpathlineto{\pgfqpoint{2.457046in}{3.116433in}}%
\pgfpathlineto{\pgfqpoint{2.463201in}{3.125473in}}%
\pgfpathlineto{\pgfqpoint{2.470962in}{3.135981in}}%
\pgfpathlineto{\pgfqpoint{2.474976in}{3.138014in}}%
\pgfpathlineto{\pgfqpoint{2.478187in}{3.137343in}}%
\pgfpathlineto{\pgfqpoint{2.481934in}{3.134121in}}%
\pgfpathlineto{\pgfqpoint{2.487821in}{3.125564in}}%
\pgfpathlineto{\pgfqpoint{2.495849in}{3.114541in}}%
\pgfpathlineto{\pgfqpoint{2.499863in}{3.112419in}}%
\pgfpathlineto{\pgfqpoint{2.503075in}{3.113017in}}%
\pgfpathlineto{\pgfqpoint{2.506821in}{3.116164in}}%
\pgfpathlineto{\pgfqpoint{2.512708in}{3.124668in}}%
\pgfpathlineto{\pgfqpoint{2.521004in}{3.136013in}}%
\pgfpathlineto{\pgfqpoint{2.525018in}{3.138018in}}%
\pgfpathlineto{\pgfqpoint{2.528230in}{3.137323in}}%
\pgfpathlineto{\pgfqpoint{2.531976in}{3.134079in}}%
\pgfpathlineto{\pgfqpoint{2.538131in}{3.125074in}}%
\pgfpathlineto{\pgfqpoint{2.545892in}{3.114508in}}%
\pgfpathlineto{\pgfqpoint{2.549906in}{3.112414in}}%
\pgfpathlineto{\pgfqpoint{2.553117in}{3.113035in}}%
\pgfpathlineto{\pgfqpoint{2.556864in}{3.116206in}}%
\pgfpathlineto{\pgfqpoint{2.562751in}{3.124727in}}%
\pgfpathlineto{\pgfqpoint{2.571047in}{3.136044in}}%
\pgfpathlineto{\pgfqpoint{2.575061in}{3.138022in}}%
\pgfpathlineto{\pgfqpoint{2.578272in}{3.137304in}}%
\pgfpathlineto{\pgfqpoint{2.582019in}{3.134037in}}%
\pgfpathlineto{\pgfqpoint{2.588174in}{3.125016in}}%
\pgfpathlineto{\pgfqpoint{2.595934in}{3.114476in}}%
\pgfpathlineto{\pgfqpoint{2.599949in}{3.112410in}}%
\pgfpathlineto{\pgfqpoint{2.603160in}{3.113054in}}%
\pgfpathlineto{\pgfqpoint{2.606906in}{3.116247in}}%
\pgfpathlineto{\pgfqpoint{2.612794in}{3.124785in}}%
\pgfpathlineto{\pgfqpoint{2.621090in}{3.136075in}}%
\pgfpathlineto{\pgfqpoint{2.625104in}{3.138025in}}%
\pgfpathlineto{\pgfqpoint{2.628315in}{3.137284in}}%
\pgfpathlineto{\pgfqpoint{2.632062in}{3.133994in}}%
\pgfpathlineto{\pgfqpoint{2.638217in}{3.124957in}}%
\pgfpathlineto{\pgfqpoint{2.645977in}{3.114444in}}%
\pgfpathlineto{\pgfqpoint{2.649991in}{3.112405in}}%
\pgfpathlineto{\pgfqpoint{2.653203in}{3.113073in}}%
\pgfpathlineto{\pgfqpoint{2.656949in}{3.116289in}}%
\pgfpathlineto{\pgfqpoint{2.662836in}{3.124844in}}%
\pgfpathlineto{\pgfqpoint{2.670865in}{3.135872in}}%
\pgfpathlineto{\pgfqpoint{2.674879in}{3.137999in}}%
\pgfpathlineto{\pgfqpoint{2.678090in}{3.137406in}}%
\pgfpathlineto{\pgfqpoint{2.681837in}{3.134263in}}%
\pgfpathlineto{\pgfqpoint{2.687724in}{3.125762in}}%
\pgfpathlineto{\pgfqpoint{2.696020in}{3.114412in}}%
\pgfpathlineto{\pgfqpoint{2.700034in}{3.112401in}}%
\pgfpathlineto{\pgfqpoint{2.703245in}{3.113092in}}%
\pgfpathlineto{\pgfqpoint{2.706992in}{3.116332in}}%
\pgfpathlineto{\pgfqpoint{2.713147in}{3.125334in}}%
\pgfpathlineto{\pgfqpoint{2.720907in}{3.135905in}}%
\pgfpathlineto{\pgfqpoint{2.724921in}{3.138004in}}%
\pgfpathlineto{\pgfqpoint{2.728133in}{3.137387in}}%
\pgfpathlineto{\pgfqpoint{2.731879in}{3.134221in}}%
\pgfpathlineto{\pgfqpoint{2.737767in}{3.125703in}}%
\pgfpathlineto{\pgfqpoint{2.746062in}{3.114381in}}%
\pgfpathlineto{\pgfqpoint{2.750077in}{3.112398in}}%
\pgfpathlineto{\pgfqpoint{2.753288in}{3.113111in}}%
\pgfpathlineto{\pgfqpoint{2.757034in}{3.116374in}}%
\pgfpathlineto{\pgfqpoint{2.763189in}{3.125392in}}%
\pgfpathlineto{\pgfqpoint{2.770950in}{3.135937in}}%
\pgfpathlineto{\pgfqpoint{2.774964in}{3.138008in}}%
\pgfpathlineto{\pgfqpoint{2.778175in}{3.137369in}}%
\pgfpathlineto{\pgfqpoint{2.781922in}{3.134179in}}%
\pgfpathlineto{\pgfqpoint{2.787809in}{3.125645in}}%
\pgfpathlineto{\pgfqpoint{2.796105in}{3.114349in}}%
\pgfpathlineto{\pgfqpoint{2.800119in}{3.112394in}}%
\pgfpathlineto{\pgfqpoint{2.803331in}{3.113131in}}%
\pgfpathlineto{\pgfqpoint{2.807077in}{3.116417in}}%
\pgfpathlineto{\pgfqpoint{2.813232in}{3.125451in}}%
\pgfpathlineto{\pgfqpoint{2.820993in}{3.135969in}}%
\pgfpathlineto{\pgfqpoint{2.825007in}{3.138013in}}%
\pgfpathlineto{\pgfqpoint{2.828218in}{3.137350in}}%
\pgfpathlineto{\pgfqpoint{2.831965in}{3.134137in}}%
\pgfpathlineto{\pgfqpoint{2.837852in}{3.125586in}}%
\pgfpathlineto{\pgfqpoint{2.845880in}{3.114553in}}%
\pgfpathlineto{\pgfqpoint{2.849894in}{3.112421in}}%
\pgfpathlineto{\pgfqpoint{2.853106in}{3.113010in}}%
\pgfpathlineto{\pgfqpoint{2.856852in}{3.116148in}}%
\pgfpathlineto{\pgfqpoint{2.862739in}{3.124646in}}%
\pgfpathlineto{\pgfqpoint{2.871035in}{3.136001in}}%
\pgfpathlineto{\pgfqpoint{2.875049in}{3.138017in}}%
\pgfpathlineto{\pgfqpoint{2.878261in}{3.137331in}}%
\pgfpathlineto{\pgfqpoint{2.882007in}{3.134095in}}%
\pgfpathlineto{\pgfqpoint{2.888162in}{3.125096in}}%
\pgfpathlineto{\pgfqpoint{2.895923in}{3.114520in}}%
\pgfpathlineto{\pgfqpoint{2.899937in}{3.112416in}}%
\pgfpathlineto{\pgfqpoint{2.903148in}{3.113028in}}%
\pgfpathlineto{\pgfqpoint{2.906895in}{3.116190in}}%
\pgfpathlineto{\pgfqpoint{2.912782in}{3.124705in}}%
\pgfpathlineto{\pgfqpoint{2.921078in}{3.136032in}}%
\pgfpathlineto{\pgfqpoint{2.925092in}{3.138021in}}%
\pgfpathlineto{\pgfqpoint{2.928303in}{3.137311in}}%
\pgfpathlineto{\pgfqpoint{2.932050in}{3.134053in}}%
\pgfpathlineto{\pgfqpoint{2.938205in}{3.125038in}}%
\pgfpathlineto{\pgfqpoint{2.945966in}{3.114488in}}%
\pgfpathlineto{\pgfqpoint{2.949980in}{3.112411in}}%
\pgfpathlineto{\pgfqpoint{2.953191in}{3.113047in}}%
\pgfpathlineto{\pgfqpoint{2.956937in}{3.116232in}}%
\pgfpathlineto{\pgfqpoint{2.962825in}{3.124763in}}%
\pgfpathlineto{\pgfqpoint{2.971121in}{3.136064in}}%
\pgfpathlineto{\pgfqpoint{2.975135in}{3.138024in}}%
\pgfpathlineto{\pgfqpoint{2.978346in}{3.137292in}}%
\pgfpathlineto{\pgfqpoint{2.982093in}{3.134010in}}%
\pgfpathlineto{\pgfqpoint{2.988248in}{3.124979in}}%
\pgfpathlineto{\pgfqpoint{2.996008in}{3.114456in}}%
\pgfpathlineto{\pgfqpoint{3.000022in}{3.112407in}}%
\pgfpathlineto{\pgfqpoint{3.003234in}{3.113066in}}%
\pgfpathlineto{\pgfqpoint{3.006980in}{3.116274in}}%
\pgfpathlineto{\pgfqpoint{3.012867in}{3.124822in}}%
\pgfpathlineto{\pgfqpoint{3.021163in}{3.136095in}}%
\pgfpathlineto{\pgfqpoint{3.025177in}{3.138028in}}%
\pgfpathlineto{\pgfqpoint{3.028389in}{3.137272in}}%
\pgfpathlineto{\pgfqpoint{3.032135in}{3.133968in}}%
\pgfpathlineto{\pgfqpoint{3.038290in}{3.124920in}}%
\pgfpathlineto{\pgfqpoint{3.046051in}{3.114424in}}%
\pgfpathlineto{\pgfqpoint{3.050065in}{3.112403in}}%
\pgfpathlineto{\pgfqpoint{3.053276in}{3.113085in}}%
\pgfpathlineto{\pgfqpoint{3.057023in}{3.116316in}}%
\pgfpathlineto{\pgfqpoint{3.062910in}{3.124880in}}%
\pgfpathlineto{\pgfqpoint{3.070938in}{3.135893in}}%
\pgfpathlineto{\pgfqpoint{3.074952in}{3.138002in}}%
\pgfpathlineto{\pgfqpoint{3.078164in}{3.137394in}}%
\pgfpathlineto{\pgfqpoint{3.081910in}{3.134237in}}%
\pgfpathlineto{\pgfqpoint{3.087798in}{3.125725in}}%
\pgfpathlineto{\pgfqpoint{3.096094in}{3.114392in}}%
\pgfpathlineto{\pgfqpoint{3.100108in}{3.112399in}}%
\pgfpathlineto{\pgfqpoint{3.103319in}{3.113104in}}%
\pgfpathlineto{\pgfqpoint{3.107065in}{3.116358in}}%
\pgfpathlineto{\pgfqpoint{3.113220in}{3.125370in}}%
\pgfpathlineto{\pgfqpoint{3.120981in}{3.135925in}}%
\pgfpathlineto{\pgfqpoint{3.124995in}{3.138007in}}%
\pgfpathlineto{\pgfqpoint{3.128206in}{3.137376in}}%
\pgfpathlineto{\pgfqpoint{3.131953in}{3.134195in}}%
\pgfpathlineto{\pgfqpoint{3.137840in}{3.125667in}}%
\pgfpathlineto{\pgfqpoint{3.146136in}{3.114361in}}%
\pgfpathlineto{\pgfqpoint{3.150150in}{3.112395in}}%
\pgfpathlineto{\pgfqpoint{3.153362in}{3.113124in}}%
\pgfpathlineto{\pgfqpoint{3.157108in}{3.116401in}}%
\pgfpathlineto{\pgfqpoint{3.163263in}{3.125429in}}%
\pgfpathlineto{\pgfqpoint{3.171024in}{3.135957in}}%
\pgfpathlineto{\pgfqpoint{3.175038in}{3.138011in}}%
\pgfpathlineto{\pgfqpoint{3.178249in}{3.137357in}}%
\pgfpathlineto{\pgfqpoint{3.181996in}{3.134153in}}%
\pgfpathlineto{\pgfqpoint{3.187883in}{3.125608in}}%
\pgfpathlineto{\pgfqpoint{3.196179in}{3.114330in}}%
\pgfpathlineto{\pgfqpoint{3.200193in}{3.112392in}}%
\pgfpathlineto{\pgfqpoint{3.203404in}{3.113143in}}%
\pgfpathlineto{\pgfqpoint{3.207151in}{3.116443in}}%
\pgfpathlineto{\pgfqpoint{3.213306in}{3.125487in}}%
\pgfpathlineto{\pgfqpoint{3.221066in}{3.135989in}}%
\pgfpathlineto{\pgfqpoint{3.225080in}{3.138015in}}%
\pgfpathlineto{\pgfqpoint{3.228292in}{3.137338in}}%
\pgfpathlineto{\pgfqpoint{3.232038in}{3.134111in}}%
\pgfpathlineto{\pgfqpoint{3.237926in}{3.125550in}}%
\pgfpathlineto{\pgfqpoint{3.245954in}{3.114532in}}%
\pgfpathlineto{\pgfqpoint{3.249968in}{3.112418in}}%
\pgfpathlineto{\pgfqpoint{3.253179in}{3.113021in}}%
\pgfpathlineto{\pgfqpoint{3.256926in}{3.116174in}}%
\pgfpathlineto{\pgfqpoint{3.262813in}{3.124683in}}%
\pgfpathlineto{\pgfqpoint{3.271109in}{3.136021in}}%
\pgfpathlineto{\pgfqpoint{3.275123in}{3.138019in}}%
\pgfpathlineto{\pgfqpoint{3.278334in}{3.137318in}}%
\pgfpathlineto{\pgfqpoint{3.282081in}{3.134069in}}%
\pgfpathlineto{\pgfqpoint{3.288236in}{3.125059in}}%
\pgfpathlineto{\pgfqpoint{3.295997in}{3.114500in}}%
\pgfpathlineto{\pgfqpoint{3.300011in}{3.112413in}}%
\pgfpathlineto{\pgfqpoint{3.303222in}{3.113040in}}%
\pgfpathlineto{\pgfqpoint{3.306968in}{3.116216in}}%
\pgfpathlineto{\pgfqpoint{3.312856in}{3.124741in}}%
\pgfpathlineto{\pgfqpoint{3.321152in}{3.136052in}}%
\pgfpathlineto{\pgfqpoint{3.325166in}{3.138023in}}%
\pgfpathlineto{\pgfqpoint{3.328377in}{3.137299in}}%
\pgfpathlineto{\pgfqpoint{3.332124in}{3.134026in}}%
\pgfpathlineto{\pgfqpoint{3.338279in}{3.125001in}}%
\pgfpathlineto{\pgfqpoint{3.346039in}{3.114468in}}%
\pgfpathlineto{\pgfqpoint{3.350053in}{3.112409in}}%
\pgfpathlineto{\pgfqpoint{3.353265in}{3.113058in}}%
\pgfpathlineto{\pgfqpoint{3.357011in}{3.116258in}}%
\pgfpathlineto{\pgfqpoint{3.362898in}{3.124800in}}%
\pgfpathlineto{\pgfqpoint{3.371194in}{3.136083in}}%
\pgfpathlineto{\pgfqpoint{3.375208in}{3.138026in}}%
\pgfpathlineto{\pgfqpoint{3.378420in}{3.137279in}}%
\pgfpathlineto{\pgfqpoint{3.382166in}{3.133984in}}%
\pgfpathlineto{\pgfqpoint{3.388321in}{3.124942in}}%
\pgfpathlineto{\pgfqpoint{3.396082in}{3.114436in}}%
\pgfpathlineto{\pgfqpoint{3.400096in}{3.112404in}}%
\pgfpathlineto{\pgfqpoint{3.403307in}{3.113077in}}%
\pgfpathlineto{\pgfqpoint{3.407054in}{3.116300in}}%
\pgfpathlineto{\pgfqpoint{3.412941in}{3.124858in}}%
\pgfpathlineto{\pgfqpoint{3.420969in}{3.135880in}}%
\pgfpathlineto{\pgfqpoint{3.424984in}{3.138000in}}%
\pgfpathlineto{\pgfqpoint{3.428195in}{3.137401in}}%
\pgfpathlineto{\pgfqpoint{3.431941in}{3.134252in}}%
\pgfpathlineto{\pgfqpoint{3.437829in}{3.125747in}}%
\pgfpathlineto{\pgfqpoint{3.446125in}{3.114404in}}%
\pgfpathlineto{\pgfqpoint{3.450139in}{3.112400in}}%
\pgfpathlineto{\pgfqpoint{3.453350in}{3.113097in}}%
\pgfpathlineto{\pgfqpoint{3.457096in}{3.116342in}}%
\pgfpathlineto{\pgfqpoint{3.463251in}{3.125348in}}%
\pgfpathlineto{\pgfqpoint{3.471012in}{3.135913in}}%
\pgfpathlineto{\pgfqpoint{3.475026in}{3.138005in}}%
\pgfpathlineto{\pgfqpoint{3.478237in}{3.137383in}}%
\pgfpathlineto{\pgfqpoint{3.481984in}{3.134211in}}%
\pgfpathlineto{\pgfqpoint{3.487871in}{3.125689in}}%
\pgfpathlineto{\pgfqpoint{3.496167in}{3.114373in}}%
\pgfpathlineto{\pgfqpoint{3.500181in}{3.112397in}}%
\pgfpathlineto{\pgfqpoint{3.503393in}{3.113116in}}%
\pgfpathlineto{\pgfqpoint{3.507139in}{3.116385in}}%
\pgfpathlineto{\pgfqpoint{3.513294in}{3.125407in}}%
\pgfpathlineto{\pgfqpoint{3.521055in}{3.135945in}}%
\pgfpathlineto{\pgfqpoint{3.525069in}{3.138009in}}%
\pgfpathlineto{\pgfqpoint{3.528280in}{3.137364in}}%
\pgfpathlineto{\pgfqpoint{3.532027in}{3.134169in}}%
\pgfpathlineto{\pgfqpoint{3.537914in}{3.125630in}}%
\pgfpathlineto{\pgfqpoint{3.546210in}{3.114342in}}%
\pgfpathlineto{\pgfqpoint{3.550224in}{3.112393in}}%
\pgfpathlineto{\pgfqpoint{3.553435in}{3.113136in}}%
\pgfpathlineto{\pgfqpoint{3.557182in}{3.116427in}}%
\pgfpathlineto{\pgfqpoint{3.563337in}{3.125465in}}%
\pgfpathlineto{\pgfqpoint{3.571097in}{3.135977in}}%
\pgfpathlineto{\pgfqpoint{3.575112in}{3.138014in}}%
\pgfpathlineto{\pgfqpoint{3.578323in}{3.137345in}}%
\pgfpathlineto{\pgfqpoint{3.582069in}{3.134127in}}%
\pgfpathlineto{\pgfqpoint{3.587957in}{3.125572in}}%
\pgfpathlineto{\pgfqpoint{3.595985in}{3.114545in}}%
\pgfpathlineto{\pgfqpoint{3.599999in}{3.112420in}}%
\pgfpathlineto{\pgfqpoint{3.603210in}{3.113014in}}%
\pgfpathlineto{\pgfqpoint{3.606957in}{3.116159in}}%
\pgfpathlineto{\pgfqpoint{3.612844in}{3.124661in}}%
\pgfpathlineto{\pgfqpoint{3.621140in}{3.136009in}}%
\pgfpathlineto{\pgfqpoint{3.625154in}{3.138018in}}%
\pgfpathlineto{\pgfqpoint{3.628365in}{3.137326in}}%
\pgfpathlineto{\pgfqpoint{3.632112in}{3.134085in}}%
\pgfpathlineto{\pgfqpoint{3.638267in}{3.125081in}}%
\pgfpathlineto{\pgfqpoint{3.646028in}{3.114512in}}%
\pgfpathlineto{\pgfqpoint{3.650042in}{3.112415in}}%
\pgfpathlineto{\pgfqpoint{3.653253in}{3.113033in}}%
\pgfpathlineto{\pgfqpoint{3.656999in}{3.116200in}}%
\pgfpathlineto{\pgfqpoint{3.662887in}{3.124719in}}%
\pgfpathlineto{\pgfqpoint{3.671183in}{3.136040in}}%
\pgfpathlineto{\pgfqpoint{3.675197in}{3.138022in}}%
\pgfpathlineto{\pgfqpoint{3.678408in}{3.137306in}}%
\pgfpathlineto{\pgfqpoint{3.682155in}{3.134042in}}%
\pgfpathlineto{\pgfqpoint{3.688310in}{3.125023in}}%
\pgfpathlineto{\pgfqpoint{3.696070in}{3.114480in}}%
\pgfpathlineto{\pgfqpoint{3.700084in}{3.112410in}}%
\pgfpathlineto{\pgfqpoint{3.703296in}{3.113051in}}%
\pgfpathlineto{\pgfqpoint{3.707042in}{3.116242in}}%
\pgfpathlineto{\pgfqpoint{3.712930in}{3.124778in}}%
\pgfpathlineto{\pgfqpoint{3.721225in}{3.136072in}}%
\pgfpathlineto{\pgfqpoint{3.725239in}{3.138025in}}%
\pgfpathlineto{\pgfqpoint{3.728451in}{3.137287in}}%
\pgfpathlineto{\pgfqpoint{3.732197in}{3.134000in}}%
\pgfpathlineto{\pgfqpoint{3.738352in}{3.124964in}}%
\pgfpathlineto{\pgfqpoint{3.746113in}{3.114448in}}%
\pgfpathlineto{\pgfqpoint{3.750127in}{3.112406in}}%
\pgfpathlineto{\pgfqpoint{3.753338in}{3.113070in}}%
\pgfpathlineto{\pgfqpoint{3.757085in}{3.116284in}}%
\pgfpathlineto{\pgfqpoint{3.762972in}{3.124836in}}%
\pgfpathlineto{\pgfqpoint{3.771000in}{3.135868in}}%
\pgfpathlineto{\pgfqpoint{3.775015in}{3.137998in}}%
\pgfpathlineto{\pgfqpoint{3.778226in}{3.137408in}}%
\pgfpathlineto{\pgfqpoint{3.781972in}{3.134268in}}%
\pgfpathlineto{\pgfqpoint{3.787860in}{3.125769in}}%
\pgfpathlineto{\pgfqpoint{3.796156in}{3.114416in}}%
\pgfpathlineto{\pgfqpoint{3.800170in}{3.112402in}}%
\pgfpathlineto{\pgfqpoint{3.803381in}{3.113089in}}%
\pgfpathlineto{\pgfqpoint{3.807127in}{3.116326in}}%
\pgfpathlineto{\pgfqpoint{3.813282in}{3.125326in}}%
\pgfpathlineto{\pgfqpoint{3.821043in}{3.135901in}}%
\pgfpathlineto{\pgfqpoint{3.825057in}{3.138003in}}%
\pgfpathlineto{\pgfqpoint{3.828269in}{3.137390in}}%
\pgfpathlineto{\pgfqpoint{3.832015in}{3.134226in}}%
\pgfpathlineto{\pgfqpoint{3.837902in}{3.125710in}}%
\pgfpathlineto{\pgfqpoint{3.846198in}{3.114385in}}%
\pgfpathlineto{\pgfqpoint{3.850212in}{3.112398in}}%
\pgfpathlineto{\pgfqpoint{3.853424in}{3.113109in}}%
\pgfpathlineto{\pgfqpoint{3.857170in}{3.116369in}}%
\pgfpathlineto{\pgfqpoint{3.863325in}{3.125385in}}%
\pgfpathlineto{\pgfqpoint{3.871086in}{3.135933in}}%
\pgfpathlineto{\pgfqpoint{3.875100in}{3.138008in}}%
\pgfpathlineto{\pgfqpoint{3.878311in}{3.137371in}}%
\pgfpathlineto{\pgfqpoint{3.882058in}{3.134184in}}%
\pgfpathlineto{\pgfqpoint{3.887945in}{3.125652in}}%
\pgfpathlineto{\pgfqpoint{3.896241in}{3.114353in}}%
\pgfpathlineto{\pgfqpoint{3.900255in}{3.112394in}}%
\pgfpathlineto{\pgfqpoint{3.903466in}{3.113129in}}%
\pgfpathlineto{\pgfqpoint{3.907213in}{3.116411in}}%
\pgfpathlineto{\pgfqpoint{3.913368in}{3.125444in}}%
\pgfpathlineto{\pgfqpoint{3.921128in}{3.135965in}}%
\pgfpathlineto{\pgfqpoint{3.925143in}{3.138012in}}%
\pgfpathlineto{\pgfqpoint{3.928354in}{3.137352in}}%
\pgfpathlineto{\pgfqpoint{3.932100in}{3.134143in}}%
\pgfpathlineto{\pgfqpoint{3.937988in}{3.125593in}}%
\pgfpathlineto{\pgfqpoint{3.946284in}{3.114322in}}%
\pgfpathlineto{\pgfqpoint{3.950298in}{3.112391in}}%
\pgfpathlineto{\pgfqpoint{3.953509in}{3.113148in}}%
\pgfpathlineto{\pgfqpoint{3.957255in}{3.116454in}}%
\pgfpathlineto{\pgfqpoint{3.963410in}{3.125502in}}%
\pgfpathlineto{\pgfqpoint{3.971171in}{3.135997in}}%
\pgfpathlineto{\pgfqpoint{3.975185in}{3.138016in}}%
\pgfpathlineto{\pgfqpoint{3.978396in}{3.137333in}}%
\pgfpathlineto{\pgfqpoint{3.982143in}{3.134100in}}%
\pgfpathlineto{\pgfqpoint{3.988030in}{3.125535in}}%
\pgfpathlineto{\pgfqpoint{3.996059in}{3.114524in}}%
\pgfpathlineto{\pgfqpoint{4.000073in}{3.112417in}}%
\pgfpathlineto{\pgfqpoint{4.003284in}{3.113026in}}%
\pgfpathlineto{\pgfqpoint{4.007031in}{3.116185in}}%
\pgfpathlineto{\pgfqpoint{4.012918in}{3.124697in}}%
\pgfpathlineto{\pgfqpoint{4.021214in}{3.136028in}}%
\pgfpathlineto{\pgfqpoint{4.025228in}{3.138020in}}%
\pgfpathlineto{\pgfqpoint{4.028439in}{3.137314in}}%
\pgfpathlineto{\pgfqpoint{4.032186in}{3.134058in}}%
\pgfpathlineto{\pgfqpoint{4.038341in}{3.125045in}}%
\pgfpathlineto{\pgfqpoint{4.046101in}{3.114492in}}%
\pgfpathlineto{\pgfqpoint{4.050115in}{3.112412in}}%
\pgfpathlineto{\pgfqpoint{4.053327in}{3.113044in}}%
\pgfpathlineto{\pgfqpoint{4.057073in}{3.116227in}}%
\pgfpathlineto{\pgfqpoint{4.062961in}{3.124756in}}%
\pgfpathlineto{\pgfqpoint{4.071256in}{3.136060in}}%
\pgfpathlineto{\pgfqpoint{4.075271in}{3.138024in}}%
\pgfpathlineto{\pgfqpoint{4.078482in}{3.137294in}}%
\pgfpathlineto{\pgfqpoint{4.082228in}{3.134016in}}%
\pgfpathlineto{\pgfqpoint{4.088383in}{3.124986in}}%
\pgfpathlineto{\pgfqpoint{4.096144in}{3.114460in}}%
\pgfpathlineto{\pgfqpoint{4.100158in}{3.112408in}}%
\pgfpathlineto{\pgfqpoint{4.103369in}{3.113063in}}%
\pgfpathlineto{\pgfqpoint{4.107116in}{3.116268in}}%
\pgfpathlineto{\pgfqpoint{4.113003in}{3.124814in}}%
\pgfpathlineto{\pgfqpoint{4.121299in}{3.136091in}}%
\pgfpathlineto{\pgfqpoint{4.125313in}{3.138027in}}%
\pgfpathlineto{\pgfqpoint{4.128524in}{3.137274in}}%
\pgfpathlineto{\pgfqpoint{4.132271in}{3.133973in}}%
\pgfpathlineto{\pgfqpoint{4.138426in}{3.124928in}}%
\pgfpathlineto{\pgfqpoint{4.146187in}{3.114428in}}%
\pgfpathlineto{\pgfqpoint{4.150201in}{3.112403in}}%
\pgfpathlineto{\pgfqpoint{4.153412in}{3.113082in}}%
\pgfpathlineto{\pgfqpoint{4.157159in}{3.116311in}}%
\pgfpathlineto{\pgfqpoint{4.163046in}{3.124873in}}%
\pgfpathlineto{\pgfqpoint{4.171074in}{3.135888in}}%
\pgfpathlineto{\pgfqpoint{4.175088in}{3.138001in}}%
\pgfpathlineto{\pgfqpoint{4.178300in}{3.137396in}}%
\pgfpathlineto{\pgfqpoint{4.182046in}{3.134242in}}%
\pgfpathlineto{\pgfqpoint{4.187933in}{3.125732in}}%
\pgfpathlineto{\pgfqpoint{4.196229in}{3.114396in}}%
\pgfpathlineto{\pgfqpoint{4.200243in}{3.112399in}}%
\pgfpathlineto{\pgfqpoint{4.203455in}{3.113102in}}%
\pgfpathlineto{\pgfqpoint{4.207201in}{3.116353in}}%
\pgfpathlineto{\pgfqpoint{4.213356in}{3.125363in}}%
\pgfpathlineto{\pgfqpoint{4.221117in}{3.135921in}}%
\pgfpathlineto{\pgfqpoint{4.225131in}{3.138006in}}%
\pgfpathlineto{\pgfqpoint{4.228342in}{3.137378in}}%
\pgfpathlineto{\pgfqpoint{4.232089in}{3.134200in}}%
\pgfpathlineto{\pgfqpoint{4.237976in}{3.125674in}}%
\pgfpathlineto{\pgfqpoint{4.246272in}{3.114365in}}%
\pgfpathlineto{\pgfqpoint{4.250286in}{3.112396in}}%
\pgfpathlineto{\pgfqpoint{4.253497in}{3.113121in}}%
\pgfpathlineto{\pgfqpoint{4.257244in}{3.116395in}}%
\pgfpathlineto{\pgfqpoint{4.263399in}{3.125422in}}%
\pgfpathlineto{\pgfqpoint{4.271159in}{3.135953in}}%
\pgfpathlineto{\pgfqpoint{4.275174in}{3.138011in}}%
\pgfpathlineto{\pgfqpoint{4.278385in}{3.137359in}}%
\pgfpathlineto{\pgfqpoint{4.282131in}{3.134158in}}%
\pgfpathlineto{\pgfqpoint{4.288019in}{3.125615in}}%
\pgfpathlineto{\pgfqpoint{4.296315in}{3.114334in}}%
\pgfpathlineto{\pgfqpoint{4.300329in}{3.112392in}}%
\pgfpathlineto{\pgfqpoint{4.303540in}{3.113141in}}%
\pgfpathlineto{\pgfqpoint{4.307287in}{3.116438in}}%
\pgfpathlineto{\pgfqpoint{4.313441in}{3.125480in}}%
\pgfpathlineto{\pgfqpoint{4.321202in}{3.135985in}}%
\pgfpathlineto{\pgfqpoint{4.325216in}{3.138015in}}%
\pgfpathlineto{\pgfqpoint{4.328428in}{3.137340in}}%
\pgfpathlineto{\pgfqpoint{4.332174in}{3.134116in}}%
\pgfpathlineto{\pgfqpoint{4.338061in}{3.125557in}}%
\pgfpathlineto{\pgfqpoint{4.346090in}{3.114537in}}%
\pgfpathlineto{\pgfqpoint{4.350104in}{3.112419in}}%
\pgfpathlineto{\pgfqpoint{4.353315in}{3.113019in}}%
\pgfpathlineto{\pgfqpoint{4.357062in}{3.116169in}}%
\pgfpathlineto{\pgfqpoint{4.362949in}{3.124676in}}%
\pgfpathlineto{\pgfqpoint{4.371245in}{3.136017in}}%
\pgfpathlineto{\pgfqpoint{4.375259in}{3.138019in}}%
\pgfpathlineto{\pgfqpoint{4.378470in}{3.137321in}}%
\pgfpathlineto{\pgfqpoint{4.382217in}{3.134074in}}%
\pgfpathlineto{\pgfqpoint{4.388372in}{3.125067in}}%
\pgfpathlineto{\pgfqpoint{4.396132in}{3.114504in}}%
\pgfpathlineto{\pgfqpoint{4.400146in}{3.112414in}}%
\pgfpathlineto{\pgfqpoint{4.403358in}{3.113037in}}%
\pgfpathlineto{\pgfqpoint{4.407104in}{3.116211in}}%
\pgfpathlineto{\pgfqpoint{4.412992in}{3.124734in}}%
\pgfpathlineto{\pgfqpoint{4.421287in}{3.136048in}}%
\pgfpathlineto{\pgfqpoint{4.425302in}{3.138022in}}%
\pgfpathlineto{\pgfqpoint{4.428513in}{3.137301in}}%
\pgfpathlineto{\pgfqpoint{4.432259in}{3.134032in}}%
\pgfpathlineto{\pgfqpoint{4.438414in}{3.125008in}}%
\pgfpathlineto{\pgfqpoint{4.446175in}{3.114472in}}%
\pgfpathlineto{\pgfqpoint{4.450189in}{3.112409in}}%
\pgfpathlineto{\pgfqpoint{4.453400in}{3.113056in}}%
\pgfpathlineto{\pgfqpoint{4.457147in}{3.116253in}}%
\pgfpathlineto{\pgfqpoint{4.463034in}{3.124793in}}%
\pgfpathlineto{\pgfqpoint{4.471330in}{3.136079in}}%
\pgfpathlineto{\pgfqpoint{4.475344in}{3.138026in}}%
\pgfpathlineto{\pgfqpoint{4.478556in}{3.137282in}}%
\pgfpathlineto{\pgfqpoint{4.482302in}{3.133989in}}%
\pgfpathlineto{\pgfqpoint{4.488457in}{3.124950in}}%
\pgfpathlineto{\pgfqpoint{4.496218in}{3.114440in}}%
\pgfpathlineto{\pgfqpoint{4.500232in}{3.112405in}}%
\pgfpathlineto{\pgfqpoint{4.503443in}{3.113075in}}%
\pgfpathlineto{\pgfqpoint{4.507190in}{3.116295in}}%
\pgfpathlineto{\pgfqpoint{4.513077in}{3.124851in}}%
\pgfpathlineto{\pgfqpoint{4.521105in}{3.135876in}}%
\pgfpathlineto{\pgfqpoint{4.525119in}{3.137999in}}%
\pgfpathlineto{\pgfqpoint{4.528331in}{3.137403in}}%
\pgfpathlineto{\pgfqpoint{4.532077in}{3.134257in}}%
\pgfpathlineto{\pgfqpoint{4.537964in}{3.125754in}}%
\pgfpathlineto{\pgfqpoint{4.546260in}{3.114408in}}%
\pgfpathlineto{\pgfqpoint{4.550274in}{3.112401in}}%
\pgfpathlineto{\pgfqpoint{4.553486in}{3.113094in}}%
\pgfpathlineto{\pgfqpoint{4.557232in}{3.116337in}}%
\pgfpathlineto{\pgfqpoint{4.563387in}{3.125341in}}%
\pgfpathlineto{\pgfqpoint{4.571148in}{3.135909in}}%
\pgfpathlineto{\pgfqpoint{4.575162in}{3.138004in}}%
\pgfpathlineto{\pgfqpoint{4.578373in}{3.137385in}}%
\pgfpathlineto{\pgfqpoint{4.582120in}{3.134216in}}%
\pgfpathlineto{\pgfqpoint{4.588007in}{3.125696in}}%
\pgfpathlineto{\pgfqpoint{4.596303in}{3.114377in}}%
\pgfpathlineto{\pgfqpoint{4.600317in}{3.112397in}}%
\pgfpathlineto{\pgfqpoint{4.603528in}{3.113114in}}%
\pgfpathlineto{\pgfqpoint{4.607275in}{3.116379in}}%
\pgfpathlineto{\pgfqpoint{4.613430in}{3.125400in}}%
\pgfpathlineto{\pgfqpoint{4.621190in}{3.135941in}}%
\pgfpathlineto{\pgfqpoint{4.625205in}{3.138009in}}%
\pgfpathlineto{\pgfqpoint{4.628416in}{3.137366in}}%
\pgfpathlineto{\pgfqpoint{4.632162in}{3.134174in}}%
\pgfpathlineto{\pgfqpoint{4.638050in}{3.125637in}}%
\pgfpathlineto{\pgfqpoint{4.646346in}{3.114345in}}%
\pgfpathlineto{\pgfqpoint{4.650360in}{3.112394in}}%
\pgfpathlineto{\pgfqpoint{4.653571in}{3.113134in}}%
\pgfpathlineto{\pgfqpoint{4.657318in}{3.116422in}}%
\pgfpathlineto{\pgfqpoint{4.663473in}{3.125458in}}%
\pgfpathlineto{\pgfqpoint{4.671233in}{3.135973in}}%
\pgfpathlineto{\pgfqpoint{4.675247in}{3.138013in}}%
\pgfpathlineto{\pgfqpoint{4.678459in}{3.137347in}}%
\pgfpathlineto{\pgfqpoint{4.682205in}{3.134132in}}%
\pgfpathlineto{\pgfqpoint{4.688092in}{3.125579in}}%
\pgfpathlineto{\pgfqpoint{4.696121in}{3.114549in}}%
\pgfpathlineto{\pgfqpoint{4.700135in}{3.112420in}}%
\pgfpathlineto{\pgfqpoint{4.703346in}{3.113012in}}%
\pgfpathlineto{\pgfqpoint{4.707093in}{3.116154in}}%
\pgfpathlineto{\pgfqpoint{4.712980in}{3.124654in}}%
\pgfpathlineto{\pgfqpoint{4.721276in}{3.136005in}}%
\pgfpathlineto{\pgfqpoint{4.725290in}{3.138017in}}%
\pgfpathlineto{\pgfqpoint{4.728501in}{3.137328in}}%
\pgfpathlineto{\pgfqpoint{4.732248in}{3.134090in}}%
\pgfpathlineto{\pgfqpoint{4.738403in}{3.125089in}}%
\pgfpathlineto{\pgfqpoint{4.746163in}{3.114516in}}%
\pgfpathlineto{\pgfqpoint{4.750177in}{3.112415in}}%
\pgfpathlineto{\pgfqpoint{4.753389in}{3.113030in}}%
\pgfpathlineto{\pgfqpoint{4.757135in}{3.116195in}}%
\pgfpathlineto{\pgfqpoint{4.763023in}{3.124712in}}%
\pgfpathlineto{\pgfqpoint{4.771318in}{3.136036in}}%
\pgfpathlineto{\pgfqpoint{4.775333in}{3.138021in}}%
\pgfpathlineto{\pgfqpoint{4.778544in}{3.137309in}}%
\pgfpathlineto{\pgfqpoint{4.782290in}{3.134047in}}%
\pgfpathlineto{\pgfqpoint{4.788445in}{3.125030in}}%
\pgfpathlineto{\pgfqpoint{4.796206in}{3.114484in}}%
\pgfpathlineto{\pgfqpoint{4.800220in}{3.112411in}}%
\pgfpathlineto{\pgfqpoint{4.803431in}{3.113049in}}%
\pgfpathlineto{\pgfqpoint{4.807178in}{3.116237in}}%
\pgfpathlineto{\pgfqpoint{4.813065in}{3.124771in}}%
\pgfpathlineto{\pgfqpoint{4.821361in}{3.136068in}}%
\pgfpathlineto{\pgfqpoint{4.825375in}{3.138025in}}%
\pgfpathlineto{\pgfqpoint{4.828587in}{3.137289in}}%
\pgfpathlineto{\pgfqpoint{4.832333in}{3.134005in}}%
\pgfpathlineto{\pgfqpoint{4.838488in}{3.124972in}}%
\pgfpathlineto{\pgfqpoint{4.846249in}{3.114452in}}%
\pgfpathlineto{\pgfqpoint{4.850263in}{3.112406in}}%
\pgfpathlineto{\pgfqpoint{4.853474in}{3.113068in}}%
\pgfpathlineto{\pgfqpoint{4.857221in}{3.116279in}}%
\pgfpathlineto{\pgfqpoint{4.863108in}{3.124829in}}%
\pgfpathlineto{\pgfqpoint{4.871136in}{3.135864in}}%
\pgfpathlineto{\pgfqpoint{4.875150in}{3.137997in}}%
\pgfpathlineto{\pgfqpoint{4.878362in}{3.137410in}}%
\pgfpathlineto{\pgfqpoint{4.882108in}{3.134273in}}%
\pgfpathlineto{\pgfqpoint{4.887995in}{3.125776in}}%
\pgfpathlineto{\pgfqpoint{4.896291in}{3.114420in}}%
\pgfpathlineto{\pgfqpoint{4.900305in}{3.112402in}}%
\pgfpathlineto{\pgfqpoint{4.903517in}{3.113087in}}%
\pgfpathlineto{\pgfqpoint{4.907263in}{3.116321in}}%
\pgfpathlineto{\pgfqpoint{4.913151in}{3.124888in}}%
\pgfpathlineto{\pgfqpoint{4.921179in}{3.135897in}}%
\pgfpathlineto{\pgfqpoint{4.925193in}{3.138002in}}%
\pgfpathlineto{\pgfqpoint{4.928404in}{3.137392in}}%
\pgfpathlineto{\pgfqpoint{4.932151in}{3.134231in}}%
\pgfpathlineto{\pgfqpoint{4.938038in}{3.125718in}}%
\pgfpathlineto{\pgfqpoint{4.946334in}{3.114388in}}%
\pgfpathlineto{\pgfqpoint{4.950348in}{3.112398in}}%
\pgfpathlineto{\pgfqpoint{4.953559in}{3.113106in}}%
\pgfpathlineto{\pgfqpoint{4.957306in}{3.116363in}}%
\pgfpathlineto{\pgfqpoint{4.963461in}{3.125378in}}%
\pgfpathlineto{\pgfqpoint{4.971222in}{3.135929in}}%
\pgfpathlineto{\pgfqpoint{4.975236in}{3.138007in}}%
\pgfpathlineto{\pgfqpoint{4.978447in}{3.137373in}}%
\pgfpathlineto{\pgfqpoint{4.982193in}{3.134190in}}%
\pgfpathlineto{\pgfqpoint{4.988081in}{3.125659in}}%
\pgfpathlineto{\pgfqpoint{4.996377in}{3.114357in}}%
\pgfpathlineto{\pgfqpoint{5.000391in}{3.112395in}}%
\pgfpathlineto{\pgfqpoint{5.003602in}{3.113126in}}%
\pgfpathlineto{\pgfqpoint{5.007349in}{3.116406in}}%
\pgfpathlineto{\pgfqpoint{5.013504in}{3.125436in}}%
\pgfpathlineto{\pgfqpoint{5.021264in}{3.135961in}}%
\pgfpathlineto{\pgfqpoint{5.025278in}{3.138012in}}%
\pgfpathlineto{\pgfqpoint{5.028490in}{3.137354in}}%
\pgfpathlineto{\pgfqpoint{5.032236in}{3.134148in}}%
\pgfpathlineto{\pgfqpoint{5.038123in}{3.125601in}}%
\pgfpathlineto{\pgfqpoint{5.046419in}{3.114326in}}%
\pgfpathlineto{\pgfqpoint{5.050433in}{3.112392in}}%
\pgfpathlineto{\pgfqpoint{5.053645in}{3.113146in}}%
\pgfpathlineto{\pgfqpoint{5.057391in}{3.116449in}}%
\pgfpathlineto{\pgfqpoint{5.063546in}{3.125495in}}%
\pgfpathlineto{\pgfqpoint{5.071307in}{3.135993in}}%
\pgfpathlineto{\pgfqpoint{5.075321in}{3.138016in}}%
\pgfpathlineto{\pgfqpoint{5.078532in}{3.137335in}}%
\pgfpathlineto{\pgfqpoint{5.082279in}{3.134106in}}%
\pgfpathlineto{\pgfqpoint{5.088166in}{3.125542in}}%
\pgfpathlineto{\pgfqpoint{5.096194in}{3.114528in}}%
\pgfpathlineto{\pgfqpoint{5.100208in}{3.112417in}}%
\pgfpathlineto{\pgfqpoint{5.103420in}{3.113024in}}%
\pgfpathlineto{\pgfqpoint{5.107166in}{3.116180in}}%
\pgfpathlineto{\pgfqpoint{5.113054in}{3.124690in}}%
\pgfpathlineto{\pgfqpoint{5.121349in}{3.136024in}}%
\pgfpathlineto{\pgfqpoint{5.125364in}{3.138020in}}%
\pgfpathlineto{\pgfqpoint{5.128575in}{3.137316in}}%
\pgfpathlineto{\pgfqpoint{5.132321in}{3.134063in}}%
\pgfpathlineto{\pgfqpoint{5.138476in}{3.125052in}}%
\pgfpathlineto{\pgfqpoint{5.146237in}{3.114496in}}%
\pgfpathlineto{\pgfqpoint{5.150251in}{3.112413in}}%
\pgfpathlineto{\pgfqpoint{5.153462in}{3.113042in}}%
\pgfpathlineto{\pgfqpoint{5.157209in}{3.116221in}}%
\pgfpathlineto{\pgfqpoint{5.163096in}{3.124749in}}%
\pgfpathlineto{\pgfqpoint{5.171392in}{3.136056in}}%
\pgfpathlineto{\pgfqpoint{5.175406in}{3.138023in}}%
\pgfpathlineto{\pgfqpoint{5.178618in}{3.137296in}}%
\pgfpathlineto{\pgfqpoint{5.182364in}{3.134021in}}%
\pgfpathlineto{\pgfqpoint{5.188519in}{3.124994in}}%
\pgfpathlineto{\pgfqpoint{5.196280in}{3.114464in}}%
\pgfpathlineto{\pgfqpoint{5.200294in}{3.112408in}}%
\pgfpathlineto{\pgfqpoint{5.203505in}{3.113061in}}%
\pgfpathlineto{\pgfqpoint{5.207252in}{3.116263in}}%
\pgfpathlineto{\pgfqpoint{5.213139in}{3.124807in}}%
\pgfpathlineto{\pgfqpoint{5.221435in}{3.136087in}}%
\pgfpathlineto{\pgfqpoint{5.225449in}{3.138027in}}%
\pgfpathlineto{\pgfqpoint{5.228660in}{3.137277in}}%
\pgfpathlineto{\pgfqpoint{5.232407in}{3.133978in}}%
\pgfpathlineto{\pgfqpoint{5.238562in}{3.124935in}}%
\pgfpathlineto{\pgfqpoint{5.246322in}{3.114432in}}%
\pgfpathlineto{\pgfqpoint{5.250336in}{3.112404in}}%
\pgfpathlineto{\pgfqpoint{5.253548in}{3.113080in}}%
\pgfpathlineto{\pgfqpoint{5.257294in}{3.116305in}}%
\pgfpathlineto{\pgfqpoint{5.263182in}{3.124866in}}%
\pgfpathlineto{\pgfqpoint{5.271210in}{3.135884in}}%
\pgfpathlineto{\pgfqpoint{5.275224in}{3.138001in}}%
\pgfpathlineto{\pgfqpoint{5.278435in}{3.137399in}}%
\pgfpathlineto{\pgfqpoint{5.282182in}{3.134247in}}%
\pgfpathlineto{\pgfqpoint{5.288069in}{3.125740in}}%
\pgfpathlineto{\pgfqpoint{5.296365in}{3.114400in}}%
\pgfpathlineto{\pgfqpoint{5.300379in}{3.112400in}}%
\pgfpathlineto{\pgfqpoint{5.303590in}{3.113099in}}%
\pgfpathlineto{\pgfqpoint{5.307337in}{3.116348in}}%
\pgfpathlineto{\pgfqpoint{5.313492in}{3.125356in}}%
\pgfpathlineto{\pgfqpoint{5.321253in}{3.135917in}}%
\pgfpathlineto{\pgfqpoint{5.325267in}{3.138005in}}%
\pgfpathlineto{\pgfqpoint{5.328478in}{3.137380in}}%
\pgfpathlineto{\pgfqpoint{5.332224in}{3.134205in}}%
\pgfpathlineto{\pgfqpoint{5.338112in}{3.125681in}}%
\pgfpathlineto{\pgfqpoint{5.346408in}{3.114369in}}%
\pgfpathlineto{\pgfqpoint{5.350422in}{3.112396in}}%
\pgfpathlineto{\pgfqpoint{5.353633in}{3.113119in}}%
\pgfpathlineto{\pgfqpoint{5.357380in}{3.116390in}}%
\pgfpathlineto{\pgfqpoint{5.363535in}{3.125414in}}%
\pgfpathlineto{\pgfqpoint{5.371295in}{3.135949in}}%
\pgfpathlineto{\pgfqpoint{5.375309in}{3.138010in}}%
\pgfpathlineto{\pgfqpoint{5.378521in}{3.137362in}}%
\pgfpathlineto{\pgfqpoint{5.382267in}{3.134164in}}%
\pgfpathlineto{\pgfqpoint{5.388154in}{3.125623in}}%
\pgfpathlineto{\pgfqpoint{5.396450in}{3.114338in}}%
\pgfpathlineto{\pgfqpoint{5.400464in}{3.112393in}}%
\pgfpathlineto{\pgfqpoint{5.403676in}{3.113138in}}%
\pgfpathlineto{\pgfqpoint{5.407422in}{3.116433in}}%
\pgfpathlineto{\pgfqpoint{5.413577in}{3.125473in}}%
\pgfpathlineto{\pgfqpoint{5.421338in}{3.135981in}}%
\pgfpathlineto{\pgfqpoint{5.425352in}{3.138014in}}%
\pgfpathlineto{\pgfqpoint{5.428563in}{3.137343in}}%
\pgfpathlineto{\pgfqpoint{5.432310in}{3.134121in}}%
\pgfpathlineto{\pgfqpoint{5.438197in}{3.125564in}}%
\pgfpathlineto{\pgfqpoint{5.446225in}{3.114541in}}%
\pgfpathlineto{\pgfqpoint{5.450240in}{3.112419in}}%
\pgfpathlineto{\pgfqpoint{5.453451in}{3.113017in}}%
\pgfpathlineto{\pgfqpoint{5.457197in}{3.116164in}}%
\pgfpathlineto{\pgfqpoint{5.463085in}{3.124668in}}%
\pgfpathlineto{\pgfqpoint{5.471381in}{3.136013in}}%
\pgfpathlineto{\pgfqpoint{5.475395in}{3.138018in}}%
\pgfpathlineto{\pgfqpoint{5.478606in}{3.137323in}}%
\pgfpathlineto{\pgfqpoint{5.482352in}{3.134079in}}%
\pgfpathlineto{\pgfqpoint{5.488507in}{3.125074in}}%
\pgfpathlineto{\pgfqpoint{5.496268in}{3.114508in}}%
\pgfpathlineto{\pgfqpoint{5.500282in}{3.112414in}}%
\pgfpathlineto{\pgfqpoint{5.503493in}{3.113035in}}%
\pgfpathlineto{\pgfqpoint{5.507240in}{3.116206in}}%
\pgfpathlineto{\pgfqpoint{5.513127in}{3.124727in}}%
\pgfpathlineto{\pgfqpoint{5.521423in}{3.136044in}}%
\pgfpathlineto{\pgfqpoint{5.525437in}{3.138022in}}%
\pgfpathlineto{\pgfqpoint{5.528649in}{3.137304in}}%
\pgfpathlineto{\pgfqpoint{5.532395in}{3.134037in}}%
\pgfpathlineto{\pgfqpoint{5.538550in}{3.125016in}}%
\pgfpathlineto{\pgfqpoint{5.546311in}{3.114476in}}%
\pgfpathlineto{\pgfqpoint{5.550325in}{3.112410in}}%
\pgfpathlineto{\pgfqpoint{5.553536in}{3.113054in}}%
\pgfpathlineto{\pgfqpoint{5.557283in}{3.116247in}}%
\pgfpathlineto{\pgfqpoint{5.563170in}{3.124785in}}%
\pgfpathlineto{\pgfqpoint{5.571466in}{3.136075in}}%
\pgfpathlineto{\pgfqpoint{5.575480in}{3.138025in}}%
\pgfpathlineto{\pgfqpoint{5.578691in}{3.137284in}}%
\pgfpathlineto{\pgfqpoint{5.582438in}{3.133994in}}%
\pgfpathlineto{\pgfqpoint{5.588593in}{3.124957in}}%
\pgfpathlineto{\pgfqpoint{5.596353in}{3.114444in}}%
\pgfpathlineto{\pgfqpoint{5.600367in}{3.112405in}}%
\pgfpathlineto{\pgfqpoint{5.603579in}{3.113073in}}%
\pgfpathlineto{\pgfqpoint{5.607325in}{3.116289in}}%
\pgfpathlineto{\pgfqpoint{5.613213in}{3.124844in}}%
\pgfpathlineto{\pgfqpoint{5.621241in}{3.135872in}}%
\pgfpathlineto{\pgfqpoint{5.625255in}{3.137999in}}%
\pgfpathlineto{\pgfqpoint{5.628466in}{3.137406in}}%
\pgfpathlineto{\pgfqpoint{5.632213in}{3.134263in}}%
\pgfpathlineto{\pgfqpoint{5.638100in}{3.125762in}}%
\pgfpathlineto{\pgfqpoint{5.646396in}{3.114412in}}%
\pgfpathlineto{\pgfqpoint{5.650410in}{3.112401in}}%
\pgfpathlineto{\pgfqpoint{5.653621in}{3.113092in}}%
\pgfpathlineto{\pgfqpoint{5.657368in}{3.116332in}}%
\pgfpathlineto{\pgfqpoint{5.663523in}{3.125334in}}%
\pgfpathlineto{\pgfqpoint{5.671284in}{3.135905in}}%
\pgfpathlineto{\pgfqpoint{5.675298in}{3.138004in}}%
\pgfpathlineto{\pgfqpoint{5.678509in}{3.137387in}}%
\pgfpathlineto{\pgfqpoint{5.682255in}{3.134221in}}%
\pgfpathlineto{\pgfqpoint{5.688143in}{3.125703in}}%
\pgfpathlineto{\pgfqpoint{5.696439in}{3.114381in}}%
\pgfpathlineto{\pgfqpoint{5.700453in}{3.112398in}}%
\pgfpathlineto{\pgfqpoint{5.703664in}{3.113111in}}%
\pgfpathlineto{\pgfqpoint{5.707411in}{3.116374in}}%
\pgfpathlineto{\pgfqpoint{5.713566in}{3.125392in}}%
\pgfpathlineto{\pgfqpoint{5.721326in}{3.135937in}}%
\pgfpathlineto{\pgfqpoint{5.725340in}{3.138008in}}%
\pgfpathlineto{\pgfqpoint{5.728552in}{3.137369in}}%
\pgfpathlineto{\pgfqpoint{5.732298in}{3.134179in}}%
\pgfpathlineto{\pgfqpoint{5.738186in}{3.125645in}}%
\pgfpathlineto{\pgfqpoint{5.746481in}{3.114349in}}%
\pgfpathlineto{\pgfqpoint{5.750495in}{3.112394in}}%
\pgfpathlineto{\pgfqpoint{5.753707in}{3.113131in}}%
\pgfpathlineto{\pgfqpoint{5.757453in}{3.116417in}}%
\pgfpathlineto{\pgfqpoint{5.763608in}{3.125451in}}%
\pgfpathlineto{\pgfqpoint{5.771369in}{3.135969in}}%
\pgfpathlineto{\pgfqpoint{5.775383in}{3.138013in}}%
\pgfpathlineto{\pgfqpoint{5.778594in}{3.137350in}}%
\pgfpathlineto{\pgfqpoint{5.782341in}{3.134137in}}%
\pgfpathlineto{\pgfqpoint{5.788228in}{3.125586in}}%
\pgfpathlineto{\pgfqpoint{5.796256in}{3.114553in}}%
\pgfpathlineto{\pgfqpoint{5.800271in}{3.112421in}}%
\pgfpathlineto{\pgfqpoint{5.803482in}{3.113010in}}%
\pgfpathlineto{\pgfqpoint{5.807228in}{3.116148in}}%
\pgfpathlineto{\pgfqpoint{5.813116in}{3.124646in}}%
\pgfpathlineto{\pgfqpoint{5.821412in}{3.136001in}}%
\pgfpathlineto{\pgfqpoint{5.825426in}{3.138017in}}%
\pgfpathlineto{\pgfqpoint{5.828637in}{3.137331in}}%
\pgfpathlineto{\pgfqpoint{5.832383in}{3.134095in}}%
\pgfpathlineto{\pgfqpoint{5.838538in}{3.125096in}}%
\pgfpathlineto{\pgfqpoint{5.846299in}{3.114520in}}%
\pgfpathlineto{\pgfqpoint{5.850313in}{3.112416in}}%
\pgfpathlineto{\pgfqpoint{5.853524in}{3.113028in}}%
\pgfpathlineto{\pgfqpoint{5.857271in}{3.116190in}}%
\pgfpathlineto{\pgfqpoint{5.863158in}{3.124705in}}%
\pgfpathlineto{\pgfqpoint{5.871454in}{3.136032in}}%
\pgfpathlineto{\pgfqpoint{5.875468in}{3.138021in}}%
\pgfpathlineto{\pgfqpoint{5.878680in}{3.137311in}}%
\pgfpathlineto{\pgfqpoint{5.882426in}{3.134053in}}%
\pgfpathlineto{\pgfqpoint{5.888581in}{3.125038in}}%
\pgfpathlineto{\pgfqpoint{5.896342in}{3.114488in}}%
\pgfpathlineto{\pgfqpoint{5.900356in}{3.112411in}}%
\pgfpathlineto{\pgfqpoint{5.903567in}{3.113047in}}%
\pgfpathlineto{\pgfqpoint{5.907314in}{3.116232in}}%
\pgfpathlineto{\pgfqpoint{5.913201in}{3.124763in}}%
\pgfpathlineto{\pgfqpoint{5.921497in}{3.136064in}}%
\pgfpathlineto{\pgfqpoint{5.925511in}{3.138024in}}%
\pgfpathlineto{\pgfqpoint{5.928722in}{3.137292in}}%
\pgfpathlineto{\pgfqpoint{5.932469in}{3.134010in}}%
\pgfpathlineto{\pgfqpoint{5.938624in}{3.124979in}}%
\pgfpathlineto{\pgfqpoint{5.946384in}{3.114456in}}%
\pgfpathlineto{\pgfqpoint{5.950399in}{3.112407in}}%
\pgfpathlineto{\pgfqpoint{5.953610in}{3.113066in}}%
\pgfpathlineto{\pgfqpoint{5.957356in}{3.116274in}}%
\pgfpathlineto{\pgfqpoint{5.963244in}{3.124822in}}%
\pgfpathlineto{\pgfqpoint{5.971540in}{3.136095in}}%
\pgfpathlineto{\pgfqpoint{5.975554in}{3.138028in}}%
\pgfpathlineto{\pgfqpoint{5.978765in}{3.137272in}}%
\pgfpathlineto{\pgfqpoint{5.982511in}{3.133968in}}%
\pgfpathlineto{\pgfqpoint{5.988666in}{3.124920in}}%
\pgfpathlineto{\pgfqpoint{5.996427in}{3.114424in}}%
\pgfpathlineto{\pgfqpoint{6.000441in}{3.112403in}}%
\pgfpathlineto{\pgfqpoint{6.003652in}{3.113085in}}%
\pgfpathlineto{\pgfqpoint{6.007399in}{3.116316in}}%
\pgfpathlineto{\pgfqpoint{6.013286in}{3.124880in}}%
\pgfpathlineto{\pgfqpoint{6.021315in}{3.135893in}}%
\pgfpathlineto{\pgfqpoint{6.025329in}{3.138002in}}%
\pgfpathlineto{\pgfqpoint{6.028540in}{3.137394in}}%
\pgfpathlineto{\pgfqpoint{6.032287in}{3.134237in}}%
\pgfpathlineto{\pgfqpoint{6.038174in}{3.125725in}}%
\pgfpathlineto{\pgfqpoint{6.046470in}{3.114392in}}%
\pgfpathlineto{\pgfqpoint{6.050484in}{3.112399in}}%
\pgfpathlineto{\pgfqpoint{6.053695in}{3.113104in}}%
\pgfpathlineto{\pgfqpoint{6.057442in}{3.116358in}}%
\pgfpathlineto{\pgfqpoint{6.063597in}{3.125370in}}%
\pgfpathlineto{\pgfqpoint{6.071357in}{3.135925in}}%
\pgfpathlineto{\pgfqpoint{6.075371in}{3.138007in}}%
\pgfpathlineto{\pgfqpoint{6.078583in}{3.137376in}}%
\pgfpathlineto{\pgfqpoint{6.082329in}{3.134195in}}%
\pgfpathlineto{\pgfqpoint{6.088217in}{3.125667in}}%
\pgfpathlineto{\pgfqpoint{6.096512in}{3.114361in}}%
\pgfpathlineto{\pgfqpoint{6.100527in}{3.112395in}}%
\pgfpathlineto{\pgfqpoint{6.103738in}{3.113124in}}%
\pgfpathlineto{\pgfqpoint{6.107484in}{3.116401in}}%
\pgfpathlineto{\pgfqpoint{6.113639in}{3.125429in}}%
\pgfpathlineto{\pgfqpoint{6.121400in}{3.135957in}}%
\pgfpathlineto{\pgfqpoint{6.125414in}{3.138011in}}%
\pgfpathlineto{\pgfqpoint{6.128625in}{3.137357in}}%
\pgfpathlineto{\pgfqpoint{6.132372in}{3.134153in}}%
\pgfpathlineto{\pgfqpoint{6.138259in}{3.125608in}}%
\pgfpathlineto{\pgfqpoint{6.146555in}{3.114330in}}%
\pgfpathlineto{\pgfqpoint{6.150569in}{3.112392in}}%
\pgfpathlineto{\pgfqpoint{6.153780in}{3.113143in}}%
\pgfpathlineto{\pgfqpoint{6.157527in}{3.116443in}}%
\pgfpathlineto{\pgfqpoint{6.163682in}{3.125487in}}%
\pgfpathlineto{\pgfqpoint{6.171443in}{3.135989in}}%
\pgfpathlineto{\pgfqpoint{6.175457in}{3.138015in}}%
\pgfpathlineto{\pgfqpoint{6.178668in}{3.137338in}}%
\pgfpathlineto{\pgfqpoint{6.182415in}{3.134111in}}%
\pgfpathlineto{\pgfqpoint{6.188302in}{3.125550in}}%
\pgfpathlineto{\pgfqpoint{6.196330in}{3.114532in}}%
\pgfpathlineto{\pgfqpoint{6.200344in}{3.112418in}}%
\pgfpathlineto{\pgfqpoint{6.203556in}{3.113021in}}%
\pgfpathlineto{\pgfqpoint{6.207302in}{3.116174in}}%
\pgfpathlineto{\pgfqpoint{6.213189in}{3.124683in}}%
\pgfpathlineto{\pgfqpoint{6.221485in}{3.136021in}}%
\pgfpathlineto{\pgfqpoint{6.225499in}{3.138019in}}%
\pgfpathlineto{\pgfqpoint{6.228711in}{3.137318in}}%
\pgfpathlineto{\pgfqpoint{6.232457in}{3.134069in}}%
\pgfpathlineto{\pgfqpoint{6.238612in}{3.125059in}}%
\pgfpathlineto{\pgfqpoint{6.246373in}{3.114500in}}%
\pgfpathlineto{\pgfqpoint{6.250387in}{3.112413in}}%
\pgfpathlineto{\pgfqpoint{6.253598in}{3.113040in}}%
\pgfpathlineto{\pgfqpoint{6.257345in}{3.116216in}}%
\pgfpathlineto{\pgfqpoint{6.263232in}{3.124741in}}%
\pgfpathlineto{\pgfqpoint{6.271528in}{3.136052in}}%
\pgfpathlineto{\pgfqpoint{6.275542in}{3.138023in}}%
\pgfpathlineto{\pgfqpoint{6.278753in}{3.137299in}}%
\pgfpathlineto{\pgfqpoint{6.282500in}{3.134026in}}%
\pgfpathlineto{\pgfqpoint{6.288655in}{3.125001in}}%
\pgfpathlineto{\pgfqpoint{6.296415in}{3.114468in}}%
\pgfpathlineto{\pgfqpoint{6.300430in}{3.112409in}}%
\pgfpathlineto{\pgfqpoint{6.303641in}{3.113058in}}%
\pgfpathlineto{\pgfqpoint{6.307387in}{3.116258in}}%
\pgfpathlineto{\pgfqpoint{6.313275in}{3.124800in}}%
\pgfpathlineto{\pgfqpoint{6.321571in}{3.136083in}}%
\pgfpathlineto{\pgfqpoint{6.325585in}{3.138026in}}%
\pgfpathlineto{\pgfqpoint{6.328796in}{3.137279in}}%
\pgfpathlineto{\pgfqpoint{6.332542in}{3.133984in}}%
\pgfpathlineto{\pgfqpoint{6.338697in}{3.124942in}}%
\pgfpathlineto{\pgfqpoint{6.346458in}{3.114436in}}%
\pgfpathlineto{\pgfqpoint{6.350472in}{3.112404in}}%
\pgfpathlineto{\pgfqpoint{6.353684in}{3.113077in}}%
\pgfpathlineto{\pgfqpoint{6.357430in}{3.116300in}}%
\pgfpathlineto{\pgfqpoint{6.363317in}{3.124858in}}%
\pgfpathlineto{\pgfqpoint{6.371346in}{3.135880in}}%
\pgfpathlineto{\pgfqpoint{6.375360in}{3.138000in}}%
\pgfpathlineto{\pgfqpoint{6.378571in}{3.137401in}}%
\pgfpathlineto{\pgfqpoint{6.382318in}{3.134252in}}%
\pgfpathlineto{\pgfqpoint{6.388205in}{3.125747in}}%
\pgfpathlineto{\pgfqpoint{6.396501in}{3.114404in}}%
\pgfpathlineto{\pgfqpoint{6.400515in}{3.112400in}}%
\pgfpathlineto{\pgfqpoint{6.403726in}{3.113097in}}%
\pgfpathlineto{\pgfqpoint{6.407473in}{3.116342in}}%
\pgfpathlineto{\pgfqpoint{6.413628in}{3.125348in}}%
\pgfpathlineto{\pgfqpoint{6.421388in}{3.135913in}}%
\pgfpathlineto{\pgfqpoint{6.425402in}{3.138005in}}%
\pgfpathlineto{\pgfqpoint{6.428614in}{3.137383in}}%
\pgfpathlineto{\pgfqpoint{6.432360in}{3.134211in}}%
\pgfpathlineto{\pgfqpoint{6.438248in}{3.125689in}}%
\pgfpathlineto{\pgfqpoint{6.446543in}{3.114373in}}%
\pgfpathlineto{\pgfqpoint{6.450558in}{3.112397in}}%
\pgfpathlineto{\pgfqpoint{6.453769in}{3.113116in}}%
\pgfpathlineto{\pgfqpoint{6.457515in}{3.116385in}}%
\pgfpathlineto{\pgfqpoint{6.463670in}{3.125407in}}%
\pgfpathlineto{\pgfqpoint{6.471431in}{3.135945in}}%
\pgfpathlineto{\pgfqpoint{6.475445in}{3.138009in}}%
\pgfpathlineto{\pgfqpoint{6.478656in}{3.137364in}}%
\pgfpathlineto{\pgfqpoint{6.482403in}{3.134169in}}%
\pgfpathlineto{\pgfqpoint{6.488290in}{3.125630in}}%
\pgfpathlineto{\pgfqpoint{6.496586in}{3.114342in}}%
\pgfpathlineto{\pgfqpoint{6.500600in}{3.112393in}}%
\pgfpathlineto{\pgfqpoint{6.503811in}{3.113136in}}%
\pgfpathlineto{\pgfqpoint{6.507558in}{3.116427in}}%
\pgfpathlineto{\pgfqpoint{6.513713in}{3.125465in}}%
\pgfpathlineto{\pgfqpoint{6.521474in}{3.135977in}}%
\pgfpathlineto{\pgfqpoint{6.525488in}{3.138014in}}%
\pgfpathlineto{\pgfqpoint{6.528699in}{3.137345in}}%
\pgfpathlineto{\pgfqpoint{6.532446in}{3.134127in}}%
\pgfpathlineto{\pgfqpoint{6.538333in}{3.125572in}}%
\pgfpathlineto{\pgfqpoint{6.546361in}{3.114545in}}%
\pgfpathlineto{\pgfqpoint{6.550375in}{3.112420in}}%
\pgfpathlineto{\pgfqpoint{6.553587in}{3.113014in}}%
\pgfpathlineto{\pgfqpoint{6.557333in}{3.116159in}}%
\pgfpathlineto{\pgfqpoint{6.563220in}{3.124661in}}%
\pgfpathlineto{\pgfqpoint{6.571516in}{3.136009in}}%
\pgfpathlineto{\pgfqpoint{6.575530in}{3.138018in}}%
\pgfpathlineto{\pgfqpoint{6.578742in}{3.137326in}}%
\pgfpathlineto{\pgfqpoint{6.582488in}{3.134085in}}%
\pgfpathlineto{\pgfqpoint{6.588643in}{3.125081in}}%
\pgfpathlineto{\pgfqpoint{6.596404in}{3.114512in}}%
\pgfpathlineto{\pgfqpoint{6.600418in}{3.112415in}}%
\pgfpathlineto{\pgfqpoint{6.603629in}{3.113033in}}%
\pgfpathlineto{\pgfqpoint{6.607376in}{3.116200in}}%
\pgfpathlineto{\pgfqpoint{6.613263in}{3.124719in}}%
\pgfpathlineto{\pgfqpoint{6.621559in}{3.136040in}}%
\pgfpathlineto{\pgfqpoint{6.625573in}{3.138022in}}%
\pgfpathlineto{\pgfqpoint{6.628784in}{3.137306in}}%
\pgfpathlineto{\pgfqpoint{6.632531in}{3.134042in}}%
\pgfpathlineto{\pgfqpoint{6.638686in}{3.125023in}}%
\pgfpathlineto{\pgfqpoint{6.646446in}{3.114480in}}%
\pgfpathlineto{\pgfqpoint{6.650461in}{3.112410in}}%
\pgfpathlineto{\pgfqpoint{6.653672in}{3.113051in}}%
\pgfpathlineto{\pgfqpoint{6.657418in}{3.116242in}}%
\pgfpathlineto{\pgfqpoint{6.663306in}{3.124778in}}%
\pgfpathlineto{\pgfqpoint{6.663306in}{3.124778in}}%
\pgfusepath{stroke}%
\end{pgfscope}%
\begin{pgfscope}%
\pgfpathrectangle{\pgfqpoint{0.467797in}{2.292089in}}{\pgfqpoint{6.490533in}{1.666241in}}%
\pgfusepath{clip}%
\pgfsetrectcap%
\pgfsetroundjoin%
\pgfsetlinewidth{1.505625pt}%
\definecolor{currentstroke}{rgb}{0.121569,0.466667,0.705882}%
\pgfsetstrokecolor{currentstroke}%
\pgfsetdash{}{0pt}%
\pgfpathmoveto{\pgfqpoint{0.762821in}{3.125209in}}%
\pgfpathlineto{\pgfqpoint{0.770582in}{3.135712in}}%
\pgfpathlineto{\pgfqpoint{0.774328in}{3.137590in}}%
\pgfpathlineto{\pgfqpoint{0.777539in}{3.136910in}}%
\pgfpathlineto{\pgfqpoint{0.781286in}{3.133608in}}%
\pgfpathlineto{\pgfqpoint{0.787441in}{3.124506in}}%
\pgfpathlineto{\pgfqpoint{0.794666in}{3.114793in}}%
\pgfpathlineto{\pgfqpoint{0.798680in}{3.112811in}}%
\pgfpathlineto{\pgfqpoint{0.801892in}{3.113603in}}%
\pgfpathlineto{\pgfqpoint{0.805638in}{3.117014in}}%
\pgfpathlineto{\pgfqpoint{0.812061in}{3.126615in}}%
\pgfpathlineto{\pgfqpoint{0.819019in}{3.135772in}}%
\pgfpathlineto{\pgfqpoint{0.822765in}{3.137598in}}%
\pgfpathlineto{\pgfqpoint{0.825976in}{3.136871in}}%
\pgfpathlineto{\pgfqpoint{0.829723in}{3.133524in}}%
\pgfpathlineto{\pgfqpoint{0.836145in}{3.123963in}}%
\pgfpathlineto{\pgfqpoint{0.843103in}{3.114732in}}%
\pgfpathlineto{\pgfqpoint{0.847117in}{3.112806in}}%
\pgfpathlineto{\pgfqpoint{0.850329in}{3.113644in}}%
\pgfpathlineto{\pgfqpoint{0.854075in}{3.117099in}}%
\pgfpathlineto{\pgfqpoint{0.860765in}{3.127154in}}%
\pgfpathlineto{\pgfqpoint{0.867456in}{3.135831in}}%
\pgfpathlineto{\pgfqpoint{0.871202in}{3.137605in}}%
\pgfpathlineto{\pgfqpoint{0.874413in}{3.136832in}}%
\pgfpathlineto{\pgfqpoint{0.878160in}{3.133440in}}%
\pgfpathlineto{\pgfqpoint{0.884582in}{3.123850in}}%
\pgfpathlineto{\pgfqpoint{0.891540in}{3.114672in}}%
\pgfpathlineto{\pgfqpoint{0.895287in}{3.112824in}}%
\pgfpathlineto{\pgfqpoint{0.898498in}{3.113532in}}%
\pgfpathlineto{\pgfqpoint{0.902245in}{3.116861in}}%
\pgfpathlineto{\pgfqpoint{0.908400in}{3.125980in}}%
\pgfpathlineto{\pgfqpoint{0.915625in}{3.135662in}}%
\pgfpathlineto{\pgfqpoint{0.919639in}{3.137611in}}%
\pgfpathlineto{\pgfqpoint{0.922850in}{3.136792in}}%
\pgfpathlineto{\pgfqpoint{0.926597in}{3.133354in}}%
\pgfpathlineto{\pgfqpoint{0.933019in}{3.123738in}}%
\pgfpathlineto{\pgfqpoint{0.939977in}{3.114612in}}%
\pgfpathlineto{\pgfqpoint{0.943724in}{3.112817in}}%
\pgfpathlineto{\pgfqpoint{0.946935in}{3.113571in}}%
\pgfpathlineto{\pgfqpoint{0.950682in}{3.116945in}}%
\pgfpathlineto{\pgfqpoint{0.957104in}{3.126523in}}%
\pgfpathlineto{\pgfqpoint{0.964062in}{3.135723in}}%
\pgfpathlineto{\pgfqpoint{0.967809in}{3.137591in}}%
\pgfpathlineto{\pgfqpoint{0.971020in}{3.136903in}}%
\pgfpathlineto{\pgfqpoint{0.974766in}{3.133592in}}%
\pgfpathlineto{\pgfqpoint{0.980921in}{3.124485in}}%
\pgfpathlineto{\pgfqpoint{0.988147in}{3.114782in}}%
\pgfpathlineto{\pgfqpoint{0.992161in}{3.112810in}}%
\pgfpathlineto{\pgfqpoint{0.995372in}{3.113611in}}%
\pgfpathlineto{\pgfqpoint{0.999119in}{3.117030in}}%
\pgfpathlineto{\pgfqpoint{1.005541in}{3.126636in}}%
\pgfpathlineto{\pgfqpoint{1.012499in}{3.135783in}}%
\pgfpathlineto{\pgfqpoint{1.016246in}{3.137599in}}%
\pgfpathlineto{\pgfqpoint{1.019457in}{3.136864in}}%
\pgfpathlineto{\pgfqpoint{1.023203in}{3.133508in}}%
\pgfpathlineto{\pgfqpoint{1.029626in}{3.123942in}}%
\pgfpathlineto{\pgfqpoint{1.036584in}{3.114720in}}%
\pgfpathlineto{\pgfqpoint{1.040330in}{3.112831in}}%
\pgfpathlineto{\pgfqpoint{1.043542in}{3.113501in}}%
\pgfpathlineto{\pgfqpoint{1.047288in}{3.116793in}}%
\pgfpathlineto{\pgfqpoint{1.053443in}{3.125888in}}%
\pgfpathlineto{\pgfqpoint{1.060936in}{3.135842in}}%
\pgfpathlineto{\pgfqpoint{1.064683in}{3.137606in}}%
\pgfpathlineto{\pgfqpoint{1.067894in}{3.136824in}}%
\pgfpathlineto{\pgfqpoint{1.071640in}{3.133424in}}%
\pgfpathlineto{\pgfqpoint{1.078063in}{3.123829in}}%
\pgfpathlineto{\pgfqpoint{1.085021in}{3.114660in}}%
\pgfpathlineto{\pgfqpoint{1.088767in}{3.112823in}}%
\pgfpathlineto{\pgfqpoint{1.091979in}{3.113539in}}%
\pgfpathlineto{\pgfqpoint{1.095725in}{3.116876in}}%
\pgfpathlineto{\pgfqpoint{1.101880in}{3.126001in}}%
\pgfpathlineto{\pgfqpoint{1.109105in}{3.135674in}}%
\pgfpathlineto{\pgfqpoint{1.113120in}{3.137612in}}%
\pgfpathlineto{\pgfqpoint{1.116331in}{3.136784in}}%
\pgfpathlineto{\pgfqpoint{1.120077in}{3.133338in}}%
\pgfpathlineto{\pgfqpoint{1.126500in}{3.123716in}}%
\pgfpathlineto{\pgfqpoint{1.133458in}{3.114601in}}%
\pgfpathlineto{\pgfqpoint{1.137204in}{3.112815in}}%
\pgfpathlineto{\pgfqpoint{1.140416in}{3.113578in}}%
\pgfpathlineto{\pgfqpoint{1.144162in}{3.116961in}}%
\pgfpathlineto{\pgfqpoint{1.150585in}{3.126544in}}%
\pgfpathlineto{\pgfqpoint{1.157542in}{3.135734in}}%
\pgfpathlineto{\pgfqpoint{1.161289in}{3.137593in}}%
\pgfpathlineto{\pgfqpoint{1.164500in}{3.136895in}}%
\pgfpathlineto{\pgfqpoint{1.168247in}{3.133576in}}%
\pgfpathlineto{\pgfqpoint{1.174402in}{3.124463in}}%
\pgfpathlineto{\pgfqpoint{1.181627in}{3.114770in}}%
\pgfpathlineto{\pgfqpoint{1.185641in}{3.112809in}}%
\pgfpathlineto{\pgfqpoint{1.188853in}{3.113618in}}%
\pgfpathlineto{\pgfqpoint{1.192599in}{3.117046in}}%
\pgfpathlineto{\pgfqpoint{1.199022in}{3.126657in}}%
\pgfpathlineto{\pgfqpoint{1.205979in}{3.135794in}}%
\pgfpathlineto{\pgfqpoint{1.209726in}{3.137601in}}%
\pgfpathlineto{\pgfqpoint{1.212937in}{3.136857in}}%
\pgfpathlineto{\pgfqpoint{1.216684in}{3.133492in}}%
\pgfpathlineto{\pgfqpoint{1.223106in}{3.123920in}}%
\pgfpathlineto{\pgfqpoint{1.230064in}{3.114709in}}%
\pgfpathlineto{\pgfqpoint{1.233811in}{3.112829in}}%
\pgfpathlineto{\pgfqpoint{1.237022in}{3.113508in}}%
\pgfpathlineto{\pgfqpoint{1.240768in}{3.116809in}}%
\pgfpathlineto{\pgfqpoint{1.246923in}{3.125910in}}%
\pgfpathlineto{\pgfqpoint{1.254149in}{3.135624in}}%
\pgfpathlineto{\pgfqpoint{1.258163in}{3.137607in}}%
\pgfpathlineto{\pgfqpoint{1.261374in}{3.136817in}}%
\pgfpathlineto{\pgfqpoint{1.265121in}{3.133408in}}%
\pgfpathlineto{\pgfqpoint{1.271543in}{3.123808in}}%
\pgfpathlineto{\pgfqpoint{1.278501in}{3.114649in}}%
\pgfpathlineto{\pgfqpoint{1.282248in}{3.112821in}}%
\pgfpathlineto{\pgfqpoint{1.285459in}{3.113546in}}%
\pgfpathlineto{\pgfqpoint{1.289205in}{3.116892in}}%
\pgfpathlineto{\pgfqpoint{1.295360in}{3.126023in}}%
\pgfpathlineto{\pgfqpoint{1.302586in}{3.135685in}}%
\pgfpathlineto{\pgfqpoint{1.306600in}{3.137613in}}%
\pgfpathlineto{\pgfqpoint{1.309811in}{3.136776in}}%
\pgfpathlineto{\pgfqpoint{1.313558in}{3.133322in}}%
\pgfpathlineto{\pgfqpoint{1.320248in}{3.123268in}}%
\pgfpathlineto{\pgfqpoint{1.326938in}{3.114590in}}%
\pgfpathlineto{\pgfqpoint{1.330685in}{3.112814in}}%
\pgfpathlineto{\pgfqpoint{1.333896in}{3.113586in}}%
\pgfpathlineto{\pgfqpoint{1.337643in}{3.116977in}}%
\pgfpathlineto{\pgfqpoint{1.344065in}{3.126565in}}%
\pgfpathlineto{\pgfqpoint{1.351023in}{3.135745in}}%
\pgfpathlineto{\pgfqpoint{1.354769in}{3.137594in}}%
\pgfpathlineto{\pgfqpoint{1.357981in}{3.136888in}}%
\pgfpathlineto{\pgfqpoint{1.361727in}{3.133561in}}%
\pgfpathlineto{\pgfqpoint{1.367882in}{3.124442in}}%
\pgfpathlineto{\pgfqpoint{1.375108in}{3.114759in}}%
\pgfpathlineto{\pgfqpoint{1.379122in}{3.112808in}}%
\pgfpathlineto{\pgfqpoint{1.382333in}{3.113626in}}%
\pgfpathlineto{\pgfqpoint{1.386080in}{3.117062in}}%
\pgfpathlineto{\pgfqpoint{1.392502in}{3.126678in}}%
\pgfpathlineto{\pgfqpoint{1.399460in}{3.135805in}}%
\pgfpathlineto{\pgfqpoint{1.403206in}{3.137602in}}%
\pgfpathlineto{\pgfqpoint{1.406418in}{3.136849in}}%
\pgfpathlineto{\pgfqpoint{1.410164in}{3.133477in}}%
\pgfpathlineto{\pgfqpoint{1.416587in}{3.123899in}}%
\pgfpathlineto{\pgfqpoint{1.423545in}{3.114698in}}%
\pgfpathlineto{\pgfqpoint{1.427291in}{3.112828in}}%
\pgfpathlineto{\pgfqpoint{1.430502in}{3.113515in}}%
\pgfpathlineto{\pgfqpoint{1.434249in}{3.116824in}}%
\pgfpathlineto{\pgfqpoint{1.440404in}{3.125931in}}%
\pgfpathlineto{\pgfqpoint{1.447629in}{3.135635in}}%
\pgfpathlineto{\pgfqpoint{1.451643in}{3.137608in}}%
\pgfpathlineto{\pgfqpoint{1.454855in}{3.136809in}}%
\pgfpathlineto{\pgfqpoint{1.458601in}{3.133392in}}%
\pgfpathlineto{\pgfqpoint{1.465024in}{3.123787in}}%
\pgfpathlineto{\pgfqpoint{1.471982in}{3.114638in}}%
\pgfpathlineto{\pgfqpoint{1.475728in}{3.112820in}}%
\pgfpathlineto{\pgfqpoint{1.478939in}{3.113554in}}%
\pgfpathlineto{\pgfqpoint{1.482686in}{3.116908in}}%
\pgfpathlineto{\pgfqpoint{1.489109in}{3.126474in}}%
\pgfpathlineto{\pgfqpoint{1.496066in}{3.135696in}}%
\pgfpathlineto{\pgfqpoint{1.499813in}{3.137588in}}%
\pgfpathlineto{\pgfqpoint{1.503024in}{3.136919in}}%
\pgfpathlineto{\pgfqpoint{1.506771in}{3.133629in}}%
\pgfpathlineto{\pgfqpoint{1.512926in}{3.124534in}}%
\pgfpathlineto{\pgfqpoint{1.520419in}{3.114579in}}%
\pgfpathlineto{\pgfqpoint{1.524165in}{3.112813in}}%
\pgfpathlineto{\pgfqpoint{1.527376in}{3.113593in}}%
\pgfpathlineto{\pgfqpoint{1.531123in}{3.116993in}}%
\pgfpathlineto{\pgfqpoint{1.537546in}{3.126586in}}%
\pgfpathlineto{\pgfqpoint{1.544503in}{3.135757in}}%
\pgfpathlineto{\pgfqpoint{1.548250in}{3.137596in}}%
\pgfpathlineto{\pgfqpoint{1.551461in}{3.136881in}}%
\pgfpathlineto{\pgfqpoint{1.555208in}{3.133545in}}%
\pgfpathlineto{\pgfqpoint{1.561363in}{3.124421in}}%
\pgfpathlineto{\pgfqpoint{1.568588in}{3.114747in}}%
\pgfpathlineto{\pgfqpoint{1.572602in}{3.112807in}}%
\pgfpathlineto{\pgfqpoint{1.575813in}{3.113634in}}%
\pgfpathlineto{\pgfqpoint{1.579560in}{3.117078in}}%
\pgfpathlineto{\pgfqpoint{1.585983in}{3.126699in}}%
\pgfpathlineto{\pgfqpoint{1.592940in}{3.135816in}}%
\pgfpathlineto{\pgfqpoint{1.596687in}{3.137603in}}%
\pgfpathlineto{\pgfqpoint{1.599898in}{3.136842in}}%
\pgfpathlineto{\pgfqpoint{1.603645in}{3.133461in}}%
\pgfpathlineto{\pgfqpoint{1.610067in}{3.123878in}}%
\pgfpathlineto{\pgfqpoint{1.617025in}{3.114687in}}%
\pgfpathlineto{\pgfqpoint{1.620772in}{3.112826in}}%
\pgfpathlineto{\pgfqpoint{1.623983in}{3.113522in}}%
\pgfpathlineto{\pgfqpoint{1.627729in}{3.116840in}}%
\pgfpathlineto{\pgfqpoint{1.633884in}{3.125952in}}%
\pgfpathlineto{\pgfqpoint{1.641110in}{3.135647in}}%
\pgfpathlineto{\pgfqpoint{1.645124in}{3.137609in}}%
\pgfpathlineto{\pgfqpoint{1.648335in}{3.136802in}}%
\pgfpathlineto{\pgfqpoint{1.652082in}{3.133376in}}%
\pgfpathlineto{\pgfqpoint{1.658504in}{3.123766in}}%
\pgfpathlineto{\pgfqpoint{1.665462in}{3.114627in}}%
\pgfpathlineto{\pgfqpoint{1.669209in}{3.112819in}}%
\pgfpathlineto{\pgfqpoint{1.672420in}{3.113561in}}%
\pgfpathlineto{\pgfqpoint{1.676166in}{3.116924in}}%
\pgfpathlineto{\pgfqpoint{1.682589in}{3.126495in}}%
\pgfpathlineto{\pgfqpoint{1.689547in}{3.135708in}}%
\pgfpathlineto{\pgfqpoint{1.693293in}{3.137589in}}%
\pgfpathlineto{\pgfqpoint{1.696505in}{3.136912in}}%
\pgfpathlineto{\pgfqpoint{1.700251in}{3.133613in}}%
\pgfpathlineto{\pgfqpoint{1.706406in}{3.124513in}}%
\pgfpathlineto{\pgfqpoint{1.713631in}{3.114797in}}%
\pgfpathlineto{\pgfqpoint{1.717646in}{3.112812in}}%
\pgfpathlineto{\pgfqpoint{1.720857in}{3.113601in}}%
\pgfpathlineto{\pgfqpoint{1.724603in}{3.117008in}}%
\pgfpathlineto{\pgfqpoint{1.731026in}{3.126608in}}%
\pgfpathlineto{\pgfqpoint{1.737984in}{3.135768in}}%
\pgfpathlineto{\pgfqpoint{1.741730in}{3.137597in}}%
\pgfpathlineto{\pgfqpoint{1.744942in}{3.136874in}}%
\pgfpathlineto{\pgfqpoint{1.748688in}{3.133529in}}%
\pgfpathlineto{\pgfqpoint{1.754843in}{3.124400in}}%
\pgfpathlineto{\pgfqpoint{1.762069in}{3.114736in}}%
\pgfpathlineto{\pgfqpoint{1.766083in}{3.112806in}}%
\pgfpathlineto{\pgfqpoint{1.769294in}{3.113641in}}%
\pgfpathlineto{\pgfqpoint{1.773040in}{3.117094in}}%
\pgfpathlineto{\pgfqpoint{1.779731in}{3.127147in}}%
\pgfpathlineto{\pgfqpoint{1.786421in}{3.135827in}}%
\pgfpathlineto{\pgfqpoint{1.790167in}{3.137604in}}%
\pgfpathlineto{\pgfqpoint{1.793379in}{3.136834in}}%
\pgfpathlineto{\pgfqpoint{1.797125in}{3.133445in}}%
\pgfpathlineto{\pgfqpoint{1.803548in}{3.123857in}}%
\pgfpathlineto{\pgfqpoint{1.810506in}{3.114675in}}%
\pgfpathlineto{\pgfqpoint{1.814252in}{3.112825in}}%
\pgfpathlineto{\pgfqpoint{1.817463in}{3.113529in}}%
\pgfpathlineto{\pgfqpoint{1.821210in}{3.116856in}}%
\pgfpathlineto{\pgfqpoint{1.827365in}{3.125973in}}%
\pgfpathlineto{\pgfqpoint{1.834590in}{3.135658in}}%
\pgfpathlineto{\pgfqpoint{1.838604in}{3.137611in}}%
\pgfpathlineto{\pgfqpoint{1.841816in}{3.136794in}}%
\pgfpathlineto{\pgfqpoint{1.845562in}{3.133360in}}%
\pgfpathlineto{\pgfqpoint{1.851985in}{3.123745in}}%
\pgfpathlineto{\pgfqpoint{1.858943in}{3.114616in}}%
\pgfpathlineto{\pgfqpoint{1.862689in}{3.112817in}}%
\pgfpathlineto{\pgfqpoint{1.865900in}{3.113568in}}%
\pgfpathlineto{\pgfqpoint{1.869647in}{3.116940in}}%
\pgfpathlineto{\pgfqpoint{1.876069in}{3.126516in}}%
\pgfpathlineto{\pgfqpoint{1.883027in}{3.135719in}}%
\pgfpathlineto{\pgfqpoint{1.886774in}{3.137591in}}%
\pgfpathlineto{\pgfqpoint{1.889985in}{3.136905in}}%
\pgfpathlineto{\pgfqpoint{1.893732in}{3.133597in}}%
\pgfpathlineto{\pgfqpoint{1.899887in}{3.124492in}}%
\pgfpathlineto{\pgfqpoint{1.907112in}{3.114785in}}%
\pgfpathlineto{\pgfqpoint{1.911126in}{3.112811in}}%
\pgfpathlineto{\pgfqpoint{1.914337in}{3.113608in}}%
\pgfpathlineto{\pgfqpoint{1.918084in}{3.117024in}}%
\pgfpathlineto{\pgfqpoint{1.924506in}{3.126629in}}%
\pgfpathlineto{\pgfqpoint{1.931464in}{3.135779in}}%
\pgfpathlineto{\pgfqpoint{1.935211in}{3.137599in}}%
\pgfpathlineto{\pgfqpoint{1.938422in}{3.136866in}}%
\pgfpathlineto{\pgfqpoint{1.942169in}{3.133513in}}%
\pgfpathlineto{\pgfqpoint{1.948591in}{3.123949in}}%
\pgfpathlineto{\pgfqpoint{1.955549in}{3.114724in}}%
\pgfpathlineto{\pgfqpoint{1.959295in}{3.112832in}}%
\pgfpathlineto{\pgfqpoint{1.962507in}{3.113499in}}%
\pgfpathlineto{\pgfqpoint{1.966253in}{3.116788in}}%
\pgfpathlineto{\pgfqpoint{1.972408in}{3.125881in}}%
\pgfpathlineto{\pgfqpoint{1.979901in}{3.135838in}}%
\pgfpathlineto{\pgfqpoint{1.983648in}{3.137606in}}%
\pgfpathlineto{\pgfqpoint{1.986859in}{3.136827in}}%
\pgfpathlineto{\pgfqpoint{1.990606in}{3.133429in}}%
\pgfpathlineto{\pgfqpoint{1.997028in}{3.123836in}}%
\pgfpathlineto{\pgfqpoint{2.003986in}{3.114664in}}%
\pgfpathlineto{\pgfqpoint{2.007732in}{3.112823in}}%
\pgfpathlineto{\pgfqpoint{2.010944in}{3.113537in}}%
\pgfpathlineto{\pgfqpoint{2.014690in}{3.116871in}}%
\pgfpathlineto{\pgfqpoint{2.020845in}{3.125994in}}%
\pgfpathlineto{\pgfqpoint{2.028071in}{3.135670in}}%
\pgfpathlineto{\pgfqpoint{2.032085in}{3.137612in}}%
\pgfpathlineto{\pgfqpoint{2.035296in}{3.136787in}}%
\pgfpathlineto{\pgfqpoint{2.039043in}{3.133344in}}%
\pgfpathlineto{\pgfqpoint{2.045465in}{3.123724in}}%
\pgfpathlineto{\pgfqpoint{2.052423in}{3.114605in}}%
\pgfpathlineto{\pgfqpoint{2.056170in}{3.112816in}}%
\pgfpathlineto{\pgfqpoint{2.059381in}{3.113576in}}%
\pgfpathlineto{\pgfqpoint{2.063127in}{3.116955in}}%
\pgfpathlineto{\pgfqpoint{2.069550in}{3.126537in}}%
\pgfpathlineto{\pgfqpoint{2.076508in}{3.135730in}}%
\pgfpathlineto{\pgfqpoint{2.080254in}{3.137592in}}%
\pgfpathlineto{\pgfqpoint{2.083466in}{3.136898in}}%
\pgfpathlineto{\pgfqpoint{2.087212in}{3.133582in}}%
\pgfpathlineto{\pgfqpoint{2.093367in}{3.124470in}}%
\pgfpathlineto{\pgfqpoint{2.100592in}{3.114774in}}%
\pgfpathlineto{\pgfqpoint{2.104607in}{3.112810in}}%
\pgfpathlineto{\pgfqpoint{2.107818in}{3.113616in}}%
\pgfpathlineto{\pgfqpoint{2.111564in}{3.117040in}}%
\pgfpathlineto{\pgfqpoint{2.117987in}{3.126650in}}%
\pgfpathlineto{\pgfqpoint{2.124945in}{3.135790in}}%
\pgfpathlineto{\pgfqpoint{2.128691in}{3.137600in}}%
\pgfpathlineto{\pgfqpoint{2.131903in}{3.136859in}}%
\pgfpathlineto{\pgfqpoint{2.135649in}{3.133498in}}%
\pgfpathlineto{\pgfqpoint{2.142072in}{3.123927in}}%
\pgfpathlineto{\pgfqpoint{2.149029in}{3.114713in}}%
\pgfpathlineto{\pgfqpoint{2.152776in}{3.112830in}}%
\pgfpathlineto{\pgfqpoint{2.155987in}{3.113506in}}%
\pgfpathlineto{\pgfqpoint{2.159734in}{3.116803in}}%
\pgfpathlineto{\pgfqpoint{2.165889in}{3.125903in}}%
\pgfpathlineto{\pgfqpoint{2.173114in}{3.135620in}}%
\pgfpathlineto{\pgfqpoint{2.177128in}{3.137607in}}%
\pgfpathlineto{\pgfqpoint{2.180340in}{3.136819in}}%
\pgfpathlineto{\pgfqpoint{2.184086in}{3.133413in}}%
\pgfpathlineto{\pgfqpoint{2.190509in}{3.123815in}}%
\pgfpathlineto{\pgfqpoint{2.197466in}{3.114653in}}%
\pgfpathlineto{\pgfqpoint{2.201213in}{3.112822in}}%
\pgfpathlineto{\pgfqpoint{2.204424in}{3.113544in}}%
\pgfpathlineto{\pgfqpoint{2.208171in}{3.116887in}}%
\pgfpathlineto{\pgfqpoint{2.214326in}{3.126016in}}%
\pgfpathlineto{\pgfqpoint{2.221551in}{3.135681in}}%
\pgfpathlineto{\pgfqpoint{2.225565in}{3.137613in}}%
\pgfpathlineto{\pgfqpoint{2.228777in}{3.136779in}}%
\pgfpathlineto{\pgfqpoint{2.232523in}{3.133328in}}%
\pgfpathlineto{\pgfqpoint{2.239213in}{3.123275in}}%
\pgfpathlineto{\pgfqpoint{2.245903in}{3.114594in}}%
\pgfpathlineto{\pgfqpoint{2.249650in}{3.112815in}}%
\pgfpathlineto{\pgfqpoint{2.252861in}{3.113583in}}%
\pgfpathlineto{\pgfqpoint{2.256608in}{3.116971in}}%
\pgfpathlineto{\pgfqpoint{2.263030in}{3.126558in}}%
\pgfpathlineto{\pgfqpoint{2.269988in}{3.135742in}}%
\pgfpathlineto{\pgfqpoint{2.273735in}{3.137594in}}%
\pgfpathlineto{\pgfqpoint{2.276946in}{3.136891in}}%
\pgfpathlineto{\pgfqpoint{2.280692in}{3.133566in}}%
\pgfpathlineto{\pgfqpoint{2.286847in}{3.124449in}}%
\pgfpathlineto{\pgfqpoint{2.294073in}{3.114762in}}%
\pgfpathlineto{\pgfqpoint{2.298087in}{3.112808in}}%
\pgfpathlineto{\pgfqpoint{2.301298in}{3.113623in}}%
\pgfpathlineto{\pgfqpoint{2.305045in}{3.117056in}}%
\pgfpathlineto{\pgfqpoint{2.311467in}{3.126671in}}%
\pgfpathlineto{\pgfqpoint{2.318425in}{3.135801in}}%
\pgfpathlineto{\pgfqpoint{2.322172in}{3.137601in}}%
\pgfpathlineto{\pgfqpoint{2.325383in}{3.136852in}}%
\pgfpathlineto{\pgfqpoint{2.329129in}{3.133482in}}%
\pgfpathlineto{\pgfqpoint{2.335552in}{3.123906in}}%
\pgfpathlineto{\pgfqpoint{2.342510in}{3.114702in}}%
\pgfpathlineto{\pgfqpoint{2.346256in}{3.112828in}}%
\pgfpathlineto{\pgfqpoint{2.349468in}{3.113513in}}%
\pgfpathlineto{\pgfqpoint{2.353214in}{3.116819in}}%
\pgfpathlineto{\pgfqpoint{2.359369in}{3.125924in}}%
\pgfpathlineto{\pgfqpoint{2.366595in}{3.135632in}}%
\pgfpathlineto{\pgfqpoint{2.370609in}{3.137608in}}%
\pgfpathlineto{\pgfqpoint{2.373820in}{3.136812in}}%
\pgfpathlineto{\pgfqpoint{2.377567in}{3.133397in}}%
\pgfpathlineto{\pgfqpoint{2.383989in}{3.123794in}}%
\pgfpathlineto{\pgfqpoint{2.390947in}{3.114642in}}%
\pgfpathlineto{\pgfqpoint{2.394693in}{3.112820in}}%
\pgfpathlineto{\pgfqpoint{2.397905in}{3.113551in}}%
\pgfpathlineto{\pgfqpoint{2.401651in}{3.116903in}}%
\pgfpathlineto{\pgfqpoint{2.408074in}{3.126467in}}%
\pgfpathlineto{\pgfqpoint{2.415032in}{3.135693in}}%
\pgfpathlineto{\pgfqpoint{2.418778in}{3.137587in}}%
\pgfpathlineto{\pgfqpoint{2.421989in}{3.136921in}}%
\pgfpathlineto{\pgfqpoint{2.425736in}{3.133634in}}%
\pgfpathlineto{\pgfqpoint{2.431891in}{3.124541in}}%
\pgfpathlineto{\pgfqpoint{2.439384in}{3.114583in}}%
\pgfpathlineto{\pgfqpoint{2.443130in}{3.112813in}}%
\pgfpathlineto{\pgfqpoint{2.446342in}{3.113591in}}%
\pgfpathlineto{\pgfqpoint{2.450088in}{3.116987in}}%
\pgfpathlineto{\pgfqpoint{2.456511in}{3.126579in}}%
\pgfpathlineto{\pgfqpoint{2.463469in}{3.135753in}}%
\pgfpathlineto{\pgfqpoint{2.467215in}{3.137595in}}%
\pgfpathlineto{\pgfqpoint{2.470426in}{3.136883in}}%
\pgfpathlineto{\pgfqpoint{2.474173in}{3.133550in}}%
\pgfpathlineto{\pgfqpoint{2.480328in}{3.124428in}}%
\pgfpathlineto{\pgfqpoint{2.487553in}{3.114751in}}%
\pgfpathlineto{\pgfqpoint{2.491567in}{3.112807in}}%
\pgfpathlineto{\pgfqpoint{2.494779in}{3.113631in}}%
\pgfpathlineto{\pgfqpoint{2.498525in}{3.117072in}}%
\pgfpathlineto{\pgfqpoint{2.504948in}{3.126692in}}%
\pgfpathlineto{\pgfqpoint{2.511906in}{3.135812in}}%
\pgfpathlineto{\pgfqpoint{2.515652in}{3.137603in}}%
\pgfpathlineto{\pgfqpoint{2.518863in}{3.136844in}}%
\pgfpathlineto{\pgfqpoint{2.522610in}{3.133466in}}%
\pgfpathlineto{\pgfqpoint{2.529033in}{3.123885in}}%
\pgfpathlineto{\pgfqpoint{2.535990in}{3.114690in}}%
\pgfpathlineto{\pgfqpoint{2.539737in}{3.112827in}}%
\pgfpathlineto{\pgfqpoint{2.542948in}{3.113520in}}%
\pgfpathlineto{\pgfqpoint{2.546695in}{3.116835in}}%
\pgfpathlineto{\pgfqpoint{2.552850in}{3.125945in}}%
\pgfpathlineto{\pgfqpoint{2.560075in}{3.135643in}}%
\pgfpathlineto{\pgfqpoint{2.564089in}{3.137609in}}%
\pgfpathlineto{\pgfqpoint{2.567300in}{3.136804in}}%
\pgfpathlineto{\pgfqpoint{2.571047in}{3.133381in}}%
\pgfpathlineto{\pgfqpoint{2.577470in}{3.123773in}}%
\pgfpathlineto{\pgfqpoint{2.584427in}{3.114631in}}%
\pgfpathlineto{\pgfqpoint{2.588174in}{3.112819in}}%
\pgfpathlineto{\pgfqpoint{2.591385in}{3.113559in}}%
\pgfpathlineto{\pgfqpoint{2.595132in}{3.116919in}}%
\pgfpathlineto{\pgfqpoint{2.601554in}{3.126488in}}%
\pgfpathlineto{\pgfqpoint{2.608512in}{3.135704in}}%
\pgfpathlineto{\pgfqpoint{2.612259in}{3.137589in}}%
\pgfpathlineto{\pgfqpoint{2.615470in}{3.136914in}}%
\pgfpathlineto{\pgfqpoint{2.619216in}{3.133618in}}%
\pgfpathlineto{\pgfqpoint{2.625371in}{3.124520in}}%
\pgfpathlineto{\pgfqpoint{2.632597in}{3.114801in}}%
\pgfpathlineto{\pgfqpoint{2.636611in}{3.112812in}}%
\pgfpathlineto{\pgfqpoint{2.639822in}{3.113598in}}%
\pgfpathlineto{\pgfqpoint{2.643569in}{3.117003in}}%
\pgfpathlineto{\pgfqpoint{2.649991in}{3.126600in}}%
\pgfpathlineto{\pgfqpoint{2.656949in}{3.135764in}}%
\pgfpathlineto{\pgfqpoint{2.660696in}{3.137597in}}%
\pgfpathlineto{\pgfqpoint{2.663907in}{3.136876in}}%
\pgfpathlineto{\pgfqpoint{2.667653in}{3.133535in}}%
\pgfpathlineto{\pgfqpoint{2.673808in}{3.124407in}}%
\pgfpathlineto{\pgfqpoint{2.681034in}{3.114739in}}%
\pgfpathlineto{\pgfqpoint{2.685048in}{3.112806in}}%
\pgfpathlineto{\pgfqpoint{2.688259in}{3.113639in}}%
\pgfpathlineto{\pgfqpoint{2.692006in}{3.117088in}}%
\pgfpathlineto{\pgfqpoint{2.698696in}{3.127140in}}%
\pgfpathlineto{\pgfqpoint{2.705386in}{3.135823in}}%
\pgfpathlineto{\pgfqpoint{2.709133in}{3.137604in}}%
\pgfpathlineto{\pgfqpoint{2.712344in}{3.136837in}}%
\pgfpathlineto{\pgfqpoint{2.716090in}{3.133450in}}%
\pgfpathlineto{\pgfqpoint{2.722513in}{3.123864in}}%
\pgfpathlineto{\pgfqpoint{2.729471in}{3.114679in}}%
\pgfpathlineto{\pgfqpoint{2.733217in}{3.112825in}}%
\pgfpathlineto{\pgfqpoint{2.736429in}{3.113527in}}%
\pgfpathlineto{\pgfqpoint{2.740175in}{3.116850in}}%
\pgfpathlineto{\pgfqpoint{2.746330in}{3.125966in}}%
\pgfpathlineto{\pgfqpoint{2.753555in}{3.135655in}}%
\pgfpathlineto{\pgfqpoint{2.757570in}{3.137610in}}%
\pgfpathlineto{\pgfqpoint{2.760781in}{3.136797in}}%
\pgfpathlineto{\pgfqpoint{2.764527in}{3.133365in}}%
\pgfpathlineto{\pgfqpoint{2.770950in}{3.123752in}}%
\pgfpathlineto{\pgfqpoint{2.777908in}{3.114619in}}%
\pgfpathlineto{\pgfqpoint{2.781654in}{3.112818in}}%
\pgfpathlineto{\pgfqpoint{2.784866in}{3.113566in}}%
\pgfpathlineto{\pgfqpoint{2.788612in}{3.116934in}}%
\pgfpathlineto{\pgfqpoint{2.795035in}{3.126509in}}%
\pgfpathlineto{\pgfqpoint{2.801993in}{3.135715in}}%
\pgfpathlineto{\pgfqpoint{2.805739in}{3.137590in}}%
\pgfpathlineto{\pgfqpoint{2.808950in}{3.136907in}}%
\pgfpathlineto{\pgfqpoint{2.812697in}{3.133603in}}%
\pgfpathlineto{\pgfqpoint{2.818852in}{3.124499in}}%
\pgfpathlineto{\pgfqpoint{2.826077in}{3.114789in}}%
\pgfpathlineto{\pgfqpoint{2.830091in}{3.112811in}}%
\pgfpathlineto{\pgfqpoint{2.833303in}{3.113606in}}%
\pgfpathlineto{\pgfqpoint{2.837049in}{3.117019in}}%
\pgfpathlineto{\pgfqpoint{2.843472in}{3.126622in}}%
\pgfpathlineto{\pgfqpoint{2.850430in}{3.135775in}}%
\pgfpathlineto{\pgfqpoint{2.854176in}{3.137598in}}%
\pgfpathlineto{\pgfqpoint{2.857387in}{3.136869in}}%
\pgfpathlineto{\pgfqpoint{2.861134in}{3.133519in}}%
\pgfpathlineto{\pgfqpoint{2.867556in}{3.123956in}}%
\pgfpathlineto{\pgfqpoint{2.874514in}{3.114728in}}%
\pgfpathlineto{\pgfqpoint{2.878261in}{3.112832in}}%
\pgfpathlineto{\pgfqpoint{2.881472in}{3.113496in}}%
\pgfpathlineto{\pgfqpoint{2.885219in}{3.116783in}}%
\pgfpathlineto{\pgfqpoint{2.891374in}{3.125874in}}%
\pgfpathlineto{\pgfqpoint{2.898867in}{3.135834in}}%
\pgfpathlineto{\pgfqpoint{2.902613in}{3.137605in}}%
\pgfpathlineto{\pgfqpoint{2.905824in}{3.136829in}}%
\pgfpathlineto{\pgfqpoint{2.909571in}{3.133434in}}%
\pgfpathlineto{\pgfqpoint{2.915993in}{3.123843in}}%
\pgfpathlineto{\pgfqpoint{2.922951in}{3.114668in}}%
\pgfpathlineto{\pgfqpoint{2.926698in}{3.112824in}}%
\pgfpathlineto{\pgfqpoint{2.929909in}{3.113534in}}%
\pgfpathlineto{\pgfqpoint{2.933656in}{3.116866in}}%
\pgfpathlineto{\pgfqpoint{2.939811in}{3.125987in}}%
\pgfpathlineto{\pgfqpoint{2.947036in}{3.135666in}}%
\pgfpathlineto{\pgfqpoint{2.951050in}{3.137611in}}%
\pgfpathlineto{\pgfqpoint{2.954261in}{3.136789in}}%
\pgfpathlineto{\pgfqpoint{2.958008in}{3.133349in}}%
\pgfpathlineto{\pgfqpoint{2.964430in}{3.123731in}}%
\pgfpathlineto{\pgfqpoint{2.971388in}{3.114608in}}%
\pgfpathlineto{\pgfqpoint{2.975135in}{3.112816in}}%
\pgfpathlineto{\pgfqpoint{2.978346in}{3.113573in}}%
\pgfpathlineto{\pgfqpoint{2.982093in}{3.116950in}}%
\pgfpathlineto{\pgfqpoint{2.988515in}{3.126530in}}%
\pgfpathlineto{\pgfqpoint{2.995473in}{3.135727in}}%
\pgfpathlineto{\pgfqpoint{2.999219in}{3.137592in}}%
\pgfpathlineto{\pgfqpoint{3.002431in}{3.136900in}}%
\pgfpathlineto{\pgfqpoint{3.006177in}{3.133587in}}%
\pgfpathlineto{\pgfqpoint{3.012332in}{3.124477in}}%
\pgfpathlineto{\pgfqpoint{3.019558in}{3.114778in}}%
\pgfpathlineto{\pgfqpoint{3.023572in}{3.112810in}}%
\pgfpathlineto{\pgfqpoint{3.026783in}{3.113613in}}%
\pgfpathlineto{\pgfqpoint{3.030530in}{3.117035in}}%
\pgfpathlineto{\pgfqpoint{3.036952in}{3.126643in}}%
\pgfpathlineto{\pgfqpoint{3.043910in}{3.135786in}}%
\pgfpathlineto{\pgfqpoint{3.047656in}{3.137600in}}%
\pgfpathlineto{\pgfqpoint{3.050868in}{3.136862in}}%
\pgfpathlineto{\pgfqpoint{3.054614in}{3.133503in}}%
\pgfpathlineto{\pgfqpoint{3.061037in}{3.123934in}}%
\pgfpathlineto{\pgfqpoint{3.067995in}{3.114717in}}%
\pgfpathlineto{\pgfqpoint{3.071741in}{3.112830in}}%
\pgfpathlineto{\pgfqpoint{3.074952in}{3.113503in}}%
\pgfpathlineto{\pgfqpoint{3.078699in}{3.116798in}}%
\pgfpathlineto{\pgfqpoint{3.084854in}{3.125895in}}%
\pgfpathlineto{\pgfqpoint{3.092347in}{3.135845in}}%
\pgfpathlineto{\pgfqpoint{3.096094in}{3.137606in}}%
\pgfpathlineto{\pgfqpoint{3.099305in}{3.136822in}}%
\pgfpathlineto{\pgfqpoint{3.103051in}{3.133418in}}%
\pgfpathlineto{\pgfqpoint{3.109474in}{3.123822in}}%
\pgfpathlineto{\pgfqpoint{3.116432in}{3.114657in}}%
\pgfpathlineto{\pgfqpoint{3.120178in}{3.112822in}}%
\pgfpathlineto{\pgfqpoint{3.123389in}{3.113542in}}%
\pgfpathlineto{\pgfqpoint{3.127136in}{3.116882in}}%
\pgfpathlineto{\pgfqpoint{3.133291in}{3.126008in}}%
\pgfpathlineto{\pgfqpoint{3.140516in}{3.135678in}}%
\pgfpathlineto{\pgfqpoint{3.144531in}{3.137612in}}%
\pgfpathlineto{\pgfqpoint{3.147742in}{3.136781in}}%
\pgfpathlineto{\pgfqpoint{3.151488in}{3.133333in}}%
\pgfpathlineto{\pgfqpoint{3.157911in}{3.123709in}}%
\pgfpathlineto{\pgfqpoint{3.164869in}{3.114597in}}%
\pgfpathlineto{\pgfqpoint{3.168615in}{3.112815in}}%
\pgfpathlineto{\pgfqpoint{3.171827in}{3.113581in}}%
\pgfpathlineto{\pgfqpoint{3.175573in}{3.116966in}}%
\pgfpathlineto{\pgfqpoint{3.181996in}{3.126551in}}%
\pgfpathlineto{\pgfqpoint{3.188953in}{3.135738in}}%
\pgfpathlineto{\pgfqpoint{3.192700in}{3.137593in}}%
\pgfpathlineto{\pgfqpoint{3.195911in}{3.136893in}}%
\pgfpathlineto{\pgfqpoint{3.199658in}{3.133571in}}%
\pgfpathlineto{\pgfqpoint{3.205813in}{3.124456in}}%
\pgfpathlineto{\pgfqpoint{3.213038in}{3.114766in}}%
\pgfpathlineto{\pgfqpoint{3.217052in}{3.112809in}}%
\pgfpathlineto{\pgfqpoint{3.220264in}{3.113621in}}%
\pgfpathlineto{\pgfqpoint{3.224010in}{3.117051in}}%
\pgfpathlineto{\pgfqpoint{3.230433in}{3.126664in}}%
\pgfpathlineto{\pgfqpoint{3.237390in}{3.135798in}}%
\pgfpathlineto{\pgfqpoint{3.241137in}{3.137601in}}%
\pgfpathlineto{\pgfqpoint{3.244348in}{3.136854in}}%
\pgfpathlineto{\pgfqpoint{3.248095in}{3.133487in}}%
\pgfpathlineto{\pgfqpoint{3.254517in}{3.123913in}}%
\pgfpathlineto{\pgfqpoint{3.261475in}{3.114705in}}%
\pgfpathlineto{\pgfqpoint{3.265222in}{3.112829in}}%
\pgfpathlineto{\pgfqpoint{3.268433in}{3.113510in}}%
\pgfpathlineto{\pgfqpoint{3.272179in}{3.116814in}}%
\pgfpathlineto{\pgfqpoint{3.278334in}{3.125917in}}%
\pgfpathlineto{\pgfqpoint{3.285560in}{3.135628in}}%
\pgfpathlineto{\pgfqpoint{3.289574in}{3.137608in}}%
\pgfpathlineto{\pgfqpoint{3.292785in}{3.136814in}}%
\pgfpathlineto{\pgfqpoint{3.296532in}{3.133402in}}%
\pgfpathlineto{\pgfqpoint{3.302954in}{3.123801in}}%
\pgfpathlineto{\pgfqpoint{3.309912in}{3.114645in}}%
\pgfpathlineto{\pgfqpoint{3.313659in}{3.112821in}}%
\pgfpathlineto{\pgfqpoint{3.316870in}{3.113549in}}%
\pgfpathlineto{\pgfqpoint{3.320616in}{3.116897in}}%
\pgfpathlineto{\pgfqpoint{3.327039in}{3.126460in}}%
\pgfpathlineto{\pgfqpoint{3.333997in}{3.135689in}}%
\pgfpathlineto{\pgfqpoint{3.337743in}{3.137586in}}%
\pgfpathlineto{\pgfqpoint{3.340955in}{3.136924in}}%
\pgfpathlineto{\pgfqpoint{3.344701in}{3.133639in}}%
\pgfpathlineto{\pgfqpoint{3.350856in}{3.124548in}}%
\pgfpathlineto{\pgfqpoint{3.358349in}{3.114586in}}%
\pgfpathlineto{\pgfqpoint{3.362096in}{3.112814in}}%
\pgfpathlineto{\pgfqpoint{3.365307in}{3.113588in}}%
\pgfpathlineto{\pgfqpoint{3.369053in}{3.116982in}}%
\pgfpathlineto{\pgfqpoint{3.375476in}{3.126572in}}%
\pgfpathlineto{\pgfqpoint{3.382434in}{3.135749in}}%
\pgfpathlineto{\pgfqpoint{3.386180in}{3.137595in}}%
\pgfpathlineto{\pgfqpoint{3.389392in}{3.136886in}}%
\pgfpathlineto{\pgfqpoint{3.393138in}{3.133555in}}%
\pgfpathlineto{\pgfqpoint{3.399293in}{3.124435in}}%
\pgfpathlineto{\pgfqpoint{3.406519in}{3.114755in}}%
\pgfpathlineto{\pgfqpoint{3.410533in}{3.112808in}}%
\pgfpathlineto{\pgfqpoint{3.413744in}{3.113628in}}%
\pgfpathlineto{\pgfqpoint{3.417490in}{3.117067in}}%
\pgfpathlineto{\pgfqpoint{3.423913in}{3.126685in}}%
\pgfpathlineto{\pgfqpoint{3.430871in}{3.135809in}}%
\pgfpathlineto{\pgfqpoint{3.434617in}{3.137602in}}%
\pgfpathlineto{\pgfqpoint{3.437829in}{3.136847in}}%
\pgfpathlineto{\pgfqpoint{3.441575in}{3.133471in}}%
\pgfpathlineto{\pgfqpoint{3.447998in}{3.123892in}}%
\pgfpathlineto{\pgfqpoint{3.454956in}{3.114694in}}%
\pgfpathlineto{\pgfqpoint{3.458702in}{3.112827in}}%
\pgfpathlineto{\pgfqpoint{3.461913in}{3.113518in}}%
\pgfpathlineto{\pgfqpoint{3.465660in}{3.116829in}}%
\pgfpathlineto{\pgfqpoint{3.471815in}{3.125938in}}%
\pgfpathlineto{\pgfqpoint{3.479040in}{3.135639in}}%
\pgfpathlineto{\pgfqpoint{3.483054in}{3.137609in}}%
\pgfpathlineto{\pgfqpoint{3.486266in}{3.136807in}}%
\pgfpathlineto{\pgfqpoint{3.490012in}{3.133386in}}%
\pgfpathlineto{\pgfqpoint{3.496435in}{3.123780in}}%
\pgfpathlineto{\pgfqpoint{3.503393in}{3.114634in}}%
\pgfpathlineto{\pgfqpoint{3.507139in}{3.112819in}}%
\pgfpathlineto{\pgfqpoint{3.510350in}{3.113556in}}%
\pgfpathlineto{\pgfqpoint{3.514097in}{3.116913in}}%
\pgfpathlineto{\pgfqpoint{3.520520in}{3.126481in}}%
\pgfpathlineto{\pgfqpoint{3.527477in}{3.135700in}}%
\pgfpathlineto{\pgfqpoint{3.531224in}{3.137588in}}%
\pgfpathlineto{\pgfqpoint{3.534435in}{3.136917in}}%
\pgfpathlineto{\pgfqpoint{3.538182in}{3.133623in}}%
\pgfpathlineto{\pgfqpoint{3.544337in}{3.124527in}}%
\pgfpathlineto{\pgfqpoint{3.551830in}{3.114575in}}%
\pgfpathlineto{\pgfqpoint{3.555576in}{3.112813in}}%
\pgfpathlineto{\pgfqpoint{3.558787in}{3.113596in}}%
\pgfpathlineto{\pgfqpoint{3.562534in}{3.116998in}}%
\pgfpathlineto{\pgfqpoint{3.568957in}{3.126593in}}%
\pgfpathlineto{\pgfqpoint{3.575914in}{3.135760in}}%
\pgfpathlineto{\pgfqpoint{3.579661in}{3.137596in}}%
\pgfpathlineto{\pgfqpoint{3.582872in}{3.136879in}}%
\pgfpathlineto{\pgfqpoint{3.586619in}{3.133540in}}%
\pgfpathlineto{\pgfqpoint{3.592774in}{3.124414in}}%
\pgfpathlineto{\pgfqpoint{3.599999in}{3.114743in}}%
\pgfpathlineto{\pgfqpoint{3.604013in}{3.112807in}}%
\pgfpathlineto{\pgfqpoint{3.607224in}{3.113636in}}%
\pgfpathlineto{\pgfqpoint{3.610971in}{3.117083in}}%
\pgfpathlineto{\pgfqpoint{3.617394in}{3.126706in}}%
\pgfpathlineto{\pgfqpoint{3.624351in}{3.135820in}}%
\pgfpathlineto{\pgfqpoint{3.628098in}{3.137604in}}%
\pgfpathlineto{\pgfqpoint{3.631309in}{3.136839in}}%
\pgfpathlineto{\pgfqpoint{3.635056in}{3.133455in}}%
\pgfpathlineto{\pgfqpoint{3.641478in}{3.123871in}}%
\pgfpathlineto{\pgfqpoint{3.648436in}{3.114683in}}%
\pgfpathlineto{\pgfqpoint{3.652183in}{3.112826in}}%
\pgfpathlineto{\pgfqpoint{3.655394in}{3.113525in}}%
\pgfpathlineto{\pgfqpoint{3.659140in}{3.116845in}}%
\pgfpathlineto{\pgfqpoint{3.665295in}{3.125959in}}%
\pgfpathlineto{\pgfqpoint{3.672521in}{3.135651in}}%
\pgfpathlineto{\pgfqpoint{3.676535in}{3.137610in}}%
\pgfpathlineto{\pgfqpoint{3.679746in}{3.136799in}}%
\pgfpathlineto{\pgfqpoint{3.683493in}{3.133370in}}%
\pgfpathlineto{\pgfqpoint{3.689915in}{3.123759in}}%
\pgfpathlineto{\pgfqpoint{3.696873in}{3.114623in}}%
\pgfpathlineto{\pgfqpoint{3.700620in}{3.112818in}}%
\pgfpathlineto{\pgfqpoint{3.703831in}{3.113563in}}%
\pgfpathlineto{\pgfqpoint{3.707577in}{3.116929in}}%
\pgfpathlineto{\pgfqpoint{3.714000in}{3.126502in}}%
\pgfpathlineto{\pgfqpoint{3.720958in}{3.135712in}}%
\pgfpathlineto{\pgfqpoint{3.724704in}{3.137590in}}%
\pgfpathlineto{\pgfqpoint{3.727916in}{3.136910in}}%
\pgfpathlineto{\pgfqpoint{3.731662in}{3.133608in}}%
\pgfpathlineto{\pgfqpoint{3.737817in}{3.124506in}}%
\pgfpathlineto{\pgfqpoint{3.745042in}{3.114793in}}%
\pgfpathlineto{\pgfqpoint{3.749057in}{3.112811in}}%
\pgfpathlineto{\pgfqpoint{3.752268in}{3.113603in}}%
\pgfpathlineto{\pgfqpoint{3.756014in}{3.117014in}}%
\pgfpathlineto{\pgfqpoint{3.762437in}{3.126615in}}%
\pgfpathlineto{\pgfqpoint{3.769395in}{3.135772in}}%
\pgfpathlineto{\pgfqpoint{3.773141in}{3.137598in}}%
\pgfpathlineto{\pgfqpoint{3.776353in}{3.136871in}}%
\pgfpathlineto{\pgfqpoint{3.780099in}{3.133524in}}%
\pgfpathlineto{\pgfqpoint{3.786522in}{3.123963in}}%
\pgfpathlineto{\pgfqpoint{3.793479in}{3.114732in}}%
\pgfpathlineto{\pgfqpoint{3.797494in}{3.112806in}}%
\pgfpathlineto{\pgfqpoint{3.800705in}{3.113644in}}%
\pgfpathlineto{\pgfqpoint{3.804451in}{3.117099in}}%
\pgfpathlineto{\pgfqpoint{3.811142in}{3.127154in}}%
\pgfpathlineto{\pgfqpoint{3.817832in}{3.135831in}}%
\pgfpathlineto{\pgfqpoint{3.821578in}{3.137605in}}%
\pgfpathlineto{\pgfqpoint{3.824790in}{3.136832in}}%
\pgfpathlineto{\pgfqpoint{3.828536in}{3.133440in}}%
\pgfpathlineto{\pgfqpoint{3.834959in}{3.123850in}}%
\pgfpathlineto{\pgfqpoint{3.841916in}{3.114672in}}%
\pgfpathlineto{\pgfqpoint{3.845663in}{3.112824in}}%
\pgfpathlineto{\pgfqpoint{3.848874in}{3.113532in}}%
\pgfpathlineto{\pgfqpoint{3.852621in}{3.116861in}}%
\pgfpathlineto{\pgfqpoint{3.858776in}{3.125980in}}%
\pgfpathlineto{\pgfqpoint{3.866001in}{3.135662in}}%
\pgfpathlineto{\pgfqpoint{3.870015in}{3.137611in}}%
\pgfpathlineto{\pgfqpoint{3.873227in}{3.136792in}}%
\pgfpathlineto{\pgfqpoint{3.876973in}{3.133354in}}%
\pgfpathlineto{\pgfqpoint{3.883396in}{3.123738in}}%
\pgfpathlineto{\pgfqpoint{3.890354in}{3.114612in}}%
\pgfpathlineto{\pgfqpoint{3.894100in}{3.112817in}}%
\pgfpathlineto{\pgfqpoint{3.897311in}{3.113571in}}%
\pgfpathlineto{\pgfqpoint{3.901058in}{3.116945in}}%
\pgfpathlineto{\pgfqpoint{3.907480in}{3.126523in}}%
\pgfpathlineto{\pgfqpoint{3.914438in}{3.135723in}}%
\pgfpathlineto{\pgfqpoint{3.918185in}{3.137591in}}%
\pgfpathlineto{\pgfqpoint{3.921396in}{3.136903in}}%
\pgfpathlineto{\pgfqpoint{3.925143in}{3.133592in}}%
\pgfpathlineto{\pgfqpoint{3.931298in}{3.124485in}}%
\pgfpathlineto{\pgfqpoint{3.938523in}{3.114782in}}%
\pgfpathlineto{\pgfqpoint{3.942537in}{3.112810in}}%
\pgfpathlineto{\pgfqpoint{3.945748in}{3.113611in}}%
\pgfpathlineto{\pgfqpoint{3.949495in}{3.117030in}}%
\pgfpathlineto{\pgfqpoint{3.955917in}{3.126636in}}%
\pgfpathlineto{\pgfqpoint{3.962875in}{3.135783in}}%
\pgfpathlineto{\pgfqpoint{3.966622in}{3.137599in}}%
\pgfpathlineto{\pgfqpoint{3.969833in}{3.136864in}}%
\pgfpathlineto{\pgfqpoint{3.973580in}{3.133508in}}%
\pgfpathlineto{\pgfqpoint{3.980002in}{3.123942in}}%
\pgfpathlineto{\pgfqpoint{3.986960in}{3.114720in}}%
\pgfpathlineto{\pgfqpoint{3.990706in}{3.112831in}}%
\pgfpathlineto{\pgfqpoint{3.993918in}{3.113501in}}%
\pgfpathlineto{\pgfqpoint{3.997664in}{3.116793in}}%
\pgfpathlineto{\pgfqpoint{4.003819in}{3.125888in}}%
\pgfpathlineto{\pgfqpoint{4.011312in}{3.135842in}}%
\pgfpathlineto{\pgfqpoint{4.015059in}{3.137606in}}%
\pgfpathlineto{\pgfqpoint{4.018270in}{3.136824in}}%
\pgfpathlineto{\pgfqpoint{4.022017in}{3.133424in}}%
\pgfpathlineto{\pgfqpoint{4.028439in}{3.123829in}}%
\pgfpathlineto{\pgfqpoint{4.035397in}{3.114660in}}%
\pgfpathlineto{\pgfqpoint{4.039143in}{3.112823in}}%
\pgfpathlineto{\pgfqpoint{4.042355in}{3.113539in}}%
\pgfpathlineto{\pgfqpoint{4.046101in}{3.116876in}}%
\pgfpathlineto{\pgfqpoint{4.052256in}{3.126001in}}%
\pgfpathlineto{\pgfqpoint{4.059482in}{3.135674in}}%
\pgfpathlineto{\pgfqpoint{4.063496in}{3.137612in}}%
\pgfpathlineto{\pgfqpoint{4.066707in}{3.136784in}}%
\pgfpathlineto{\pgfqpoint{4.070454in}{3.133338in}}%
\pgfpathlineto{\pgfqpoint{4.076876in}{3.123716in}}%
\pgfpathlineto{\pgfqpoint{4.083834in}{3.114601in}}%
\pgfpathlineto{\pgfqpoint{4.087580in}{3.112815in}}%
\pgfpathlineto{\pgfqpoint{4.090792in}{3.113578in}}%
\pgfpathlineto{\pgfqpoint{4.094538in}{3.116961in}}%
\pgfpathlineto{\pgfqpoint{4.100961in}{3.126544in}}%
\pgfpathlineto{\pgfqpoint{4.107919in}{3.135734in}}%
\pgfpathlineto{\pgfqpoint{4.111665in}{3.137593in}}%
\pgfpathlineto{\pgfqpoint{4.114876in}{3.136895in}}%
\pgfpathlineto{\pgfqpoint{4.118623in}{3.133576in}}%
\pgfpathlineto{\pgfqpoint{4.124778in}{3.124463in}}%
\pgfpathlineto{\pgfqpoint{4.132003in}{3.114770in}}%
\pgfpathlineto{\pgfqpoint{4.136017in}{3.112809in}}%
\pgfpathlineto{\pgfqpoint{4.139229in}{3.113618in}}%
\pgfpathlineto{\pgfqpoint{4.142975in}{3.117046in}}%
\pgfpathlineto{\pgfqpoint{4.149398in}{3.126657in}}%
\pgfpathlineto{\pgfqpoint{4.156356in}{3.135794in}}%
\pgfpathlineto{\pgfqpoint{4.160102in}{3.137601in}}%
\pgfpathlineto{\pgfqpoint{4.163313in}{3.136857in}}%
\pgfpathlineto{\pgfqpoint{4.167060in}{3.133492in}}%
\pgfpathlineto{\pgfqpoint{4.173483in}{3.123920in}}%
\pgfpathlineto{\pgfqpoint{4.180440in}{3.114709in}}%
\pgfpathlineto{\pgfqpoint{4.184187in}{3.112829in}}%
\pgfpathlineto{\pgfqpoint{4.187398in}{3.113508in}}%
\pgfpathlineto{\pgfqpoint{4.191145in}{3.116809in}}%
\pgfpathlineto{\pgfqpoint{4.197300in}{3.125910in}}%
\pgfpathlineto{\pgfqpoint{4.204525in}{3.135624in}}%
\pgfpathlineto{\pgfqpoint{4.208539in}{3.137607in}}%
\pgfpathlineto{\pgfqpoint{4.211751in}{3.136817in}}%
\pgfpathlineto{\pgfqpoint{4.215497in}{3.133408in}}%
\pgfpathlineto{\pgfqpoint{4.221920in}{3.123808in}}%
\pgfpathlineto{\pgfqpoint{4.228877in}{3.114649in}}%
\pgfpathlineto{\pgfqpoint{4.232624in}{3.112821in}}%
\pgfpathlineto{\pgfqpoint{4.235835in}{3.113546in}}%
\pgfpathlineto{\pgfqpoint{4.239582in}{3.116892in}}%
\pgfpathlineto{\pgfqpoint{4.245737in}{3.126023in}}%
\pgfpathlineto{\pgfqpoint{4.252962in}{3.135685in}}%
\pgfpathlineto{\pgfqpoint{4.256976in}{3.137613in}}%
\pgfpathlineto{\pgfqpoint{4.260188in}{3.136776in}}%
\pgfpathlineto{\pgfqpoint{4.263934in}{3.133322in}}%
\pgfpathlineto{\pgfqpoint{4.270624in}{3.123268in}}%
\pgfpathlineto{\pgfqpoint{4.277314in}{3.114590in}}%
\pgfpathlineto{\pgfqpoint{4.281061in}{3.112814in}}%
\pgfpathlineto{\pgfqpoint{4.284272in}{3.113586in}}%
\pgfpathlineto{\pgfqpoint{4.288019in}{3.116977in}}%
\pgfpathlineto{\pgfqpoint{4.294441in}{3.126565in}}%
\pgfpathlineto{\pgfqpoint{4.301399in}{3.135745in}}%
\pgfpathlineto{\pgfqpoint{4.305146in}{3.137594in}}%
\pgfpathlineto{\pgfqpoint{4.308357in}{3.136888in}}%
\pgfpathlineto{\pgfqpoint{4.312103in}{3.133561in}}%
\pgfpathlineto{\pgfqpoint{4.318258in}{3.124442in}}%
\pgfpathlineto{\pgfqpoint{4.325484in}{3.114759in}}%
\pgfpathlineto{\pgfqpoint{4.329498in}{3.112808in}}%
\pgfpathlineto{\pgfqpoint{4.332709in}{3.113626in}}%
\pgfpathlineto{\pgfqpoint{4.336456in}{3.117062in}}%
\pgfpathlineto{\pgfqpoint{4.342878in}{3.126678in}}%
\pgfpathlineto{\pgfqpoint{4.349836in}{3.135805in}}%
\pgfpathlineto{\pgfqpoint{4.353583in}{3.137602in}}%
\pgfpathlineto{\pgfqpoint{4.356794in}{3.136849in}}%
\pgfpathlineto{\pgfqpoint{4.360540in}{3.133477in}}%
\pgfpathlineto{\pgfqpoint{4.366963in}{3.123899in}}%
\pgfpathlineto{\pgfqpoint{4.373921in}{3.114698in}}%
\pgfpathlineto{\pgfqpoint{4.377667in}{3.112828in}}%
\pgfpathlineto{\pgfqpoint{4.380879in}{3.113515in}}%
\pgfpathlineto{\pgfqpoint{4.384625in}{3.116824in}}%
\pgfpathlineto{\pgfqpoint{4.390780in}{3.125931in}}%
\pgfpathlineto{\pgfqpoint{4.398006in}{3.135635in}}%
\pgfpathlineto{\pgfqpoint{4.402020in}{3.137608in}}%
\pgfpathlineto{\pgfqpoint{4.405231in}{3.136809in}}%
\pgfpathlineto{\pgfqpoint{4.408977in}{3.133392in}}%
\pgfpathlineto{\pgfqpoint{4.415400in}{3.123787in}}%
\pgfpathlineto{\pgfqpoint{4.422358in}{3.114638in}}%
\pgfpathlineto{\pgfqpoint{4.426104in}{3.112820in}}%
\pgfpathlineto{\pgfqpoint{4.429316in}{3.113554in}}%
\pgfpathlineto{\pgfqpoint{4.433062in}{3.116908in}}%
\pgfpathlineto{\pgfqpoint{4.439485in}{3.126474in}}%
\pgfpathlineto{\pgfqpoint{4.446443in}{3.135696in}}%
\pgfpathlineto{\pgfqpoint{4.450189in}{3.137588in}}%
\pgfpathlineto{\pgfqpoint{4.453400in}{3.136919in}}%
\pgfpathlineto{\pgfqpoint{4.457147in}{3.133629in}}%
\pgfpathlineto{\pgfqpoint{4.463302in}{3.124534in}}%
\pgfpathlineto{\pgfqpoint{4.470795in}{3.114579in}}%
\pgfpathlineto{\pgfqpoint{4.474541in}{3.112813in}}%
\pgfpathlineto{\pgfqpoint{4.477753in}{3.113593in}}%
\pgfpathlineto{\pgfqpoint{4.481499in}{3.116993in}}%
\pgfpathlineto{\pgfqpoint{4.487922in}{3.126586in}}%
\pgfpathlineto{\pgfqpoint{4.494880in}{3.135757in}}%
\pgfpathlineto{\pgfqpoint{4.498626in}{3.137596in}}%
\pgfpathlineto{\pgfqpoint{4.501837in}{3.136881in}}%
\pgfpathlineto{\pgfqpoint{4.505584in}{3.133545in}}%
\pgfpathlineto{\pgfqpoint{4.511739in}{3.124421in}}%
\pgfpathlineto{\pgfqpoint{4.518964in}{3.114747in}}%
\pgfpathlineto{\pgfqpoint{4.522978in}{3.112807in}}%
\pgfpathlineto{\pgfqpoint{4.526190in}{3.113634in}}%
\pgfpathlineto{\pgfqpoint{4.529936in}{3.117078in}}%
\pgfpathlineto{\pgfqpoint{4.536359in}{3.126699in}}%
\pgfpathlineto{\pgfqpoint{4.543317in}{3.135816in}}%
\pgfpathlineto{\pgfqpoint{4.547063in}{3.137603in}}%
\pgfpathlineto{\pgfqpoint{4.550274in}{3.136842in}}%
\pgfpathlineto{\pgfqpoint{4.554021in}{3.133461in}}%
\pgfpathlineto{\pgfqpoint{4.560443in}{3.123878in}}%
\pgfpathlineto{\pgfqpoint{4.567401in}{3.114687in}}%
\pgfpathlineto{\pgfqpoint{4.571148in}{3.112826in}}%
\pgfpathlineto{\pgfqpoint{4.574359in}{3.113522in}}%
\pgfpathlineto{\pgfqpoint{4.578106in}{3.116840in}}%
\pgfpathlineto{\pgfqpoint{4.584261in}{3.125952in}}%
\pgfpathlineto{\pgfqpoint{4.591486in}{3.135647in}}%
\pgfpathlineto{\pgfqpoint{4.595500in}{3.137609in}}%
\pgfpathlineto{\pgfqpoint{4.598711in}{3.136802in}}%
\pgfpathlineto{\pgfqpoint{4.602458in}{3.133376in}}%
\pgfpathlineto{\pgfqpoint{4.608881in}{3.123766in}}%
\pgfpathlineto{\pgfqpoint{4.615838in}{3.114627in}}%
\pgfpathlineto{\pgfqpoint{4.619585in}{3.112819in}}%
\pgfpathlineto{\pgfqpoint{4.622796in}{3.113561in}}%
\pgfpathlineto{\pgfqpoint{4.626543in}{3.116924in}}%
\pgfpathlineto{\pgfqpoint{4.632965in}{3.126495in}}%
\pgfpathlineto{\pgfqpoint{4.639923in}{3.135708in}}%
\pgfpathlineto{\pgfqpoint{4.643670in}{3.137589in}}%
\pgfpathlineto{\pgfqpoint{4.646881in}{3.136912in}}%
\pgfpathlineto{\pgfqpoint{4.650627in}{3.133613in}}%
\pgfpathlineto{\pgfqpoint{4.656782in}{3.124513in}}%
\pgfpathlineto{\pgfqpoint{4.664008in}{3.114797in}}%
\pgfpathlineto{\pgfqpoint{4.668022in}{3.112812in}}%
\pgfpathlineto{\pgfqpoint{4.671233in}{3.113601in}}%
\pgfpathlineto{\pgfqpoint{4.674980in}{3.117008in}}%
\pgfpathlineto{\pgfqpoint{4.681402in}{3.126608in}}%
\pgfpathlineto{\pgfqpoint{4.688360in}{3.135768in}}%
\pgfpathlineto{\pgfqpoint{4.692107in}{3.137597in}}%
\pgfpathlineto{\pgfqpoint{4.695318in}{3.136874in}}%
\pgfpathlineto{\pgfqpoint{4.699064in}{3.133529in}}%
\pgfpathlineto{\pgfqpoint{4.705219in}{3.124400in}}%
\pgfpathlineto{\pgfqpoint{4.712445in}{3.114736in}}%
\pgfpathlineto{\pgfqpoint{4.716459in}{3.112806in}}%
\pgfpathlineto{\pgfqpoint{4.719670in}{3.113641in}}%
\pgfpathlineto{\pgfqpoint{4.723417in}{3.117094in}}%
\pgfpathlineto{\pgfqpoint{4.730107in}{3.127147in}}%
\pgfpathlineto{\pgfqpoint{4.736797in}{3.135827in}}%
\pgfpathlineto{\pgfqpoint{4.740544in}{3.137604in}}%
\pgfpathlineto{\pgfqpoint{4.743755in}{3.136834in}}%
\pgfpathlineto{\pgfqpoint{4.747501in}{3.133445in}}%
\pgfpathlineto{\pgfqpoint{4.753924in}{3.123857in}}%
\pgfpathlineto{\pgfqpoint{4.760882in}{3.114675in}}%
\pgfpathlineto{\pgfqpoint{4.764628in}{3.112825in}}%
\pgfpathlineto{\pgfqpoint{4.767840in}{3.113529in}}%
\pgfpathlineto{\pgfqpoint{4.771586in}{3.116856in}}%
\pgfpathlineto{\pgfqpoint{4.777741in}{3.125973in}}%
\pgfpathlineto{\pgfqpoint{4.784966in}{3.135658in}}%
\pgfpathlineto{\pgfqpoint{4.788981in}{3.137611in}}%
\pgfpathlineto{\pgfqpoint{4.792192in}{3.136794in}}%
\pgfpathlineto{\pgfqpoint{4.795938in}{3.133360in}}%
\pgfpathlineto{\pgfqpoint{4.802361in}{3.123745in}}%
\pgfpathlineto{\pgfqpoint{4.809319in}{3.114616in}}%
\pgfpathlineto{\pgfqpoint{4.813065in}{3.112817in}}%
\pgfpathlineto{\pgfqpoint{4.816277in}{3.113568in}}%
\pgfpathlineto{\pgfqpoint{4.820023in}{3.116940in}}%
\pgfpathlineto{\pgfqpoint{4.826446in}{3.126516in}}%
\pgfpathlineto{\pgfqpoint{4.833403in}{3.135719in}}%
\pgfpathlineto{\pgfqpoint{4.837150in}{3.137591in}}%
\pgfpathlineto{\pgfqpoint{4.840361in}{3.136905in}}%
\pgfpathlineto{\pgfqpoint{4.844108in}{3.133597in}}%
\pgfpathlineto{\pgfqpoint{4.850263in}{3.124492in}}%
\pgfpathlineto{\pgfqpoint{4.857488in}{3.114785in}}%
\pgfpathlineto{\pgfqpoint{4.861502in}{3.112811in}}%
\pgfpathlineto{\pgfqpoint{4.864714in}{3.113608in}}%
\pgfpathlineto{\pgfqpoint{4.868460in}{3.117024in}}%
\pgfpathlineto{\pgfqpoint{4.874883in}{3.126629in}}%
\pgfpathlineto{\pgfqpoint{4.881840in}{3.135779in}}%
\pgfpathlineto{\pgfqpoint{4.885587in}{3.137599in}}%
\pgfpathlineto{\pgfqpoint{4.888798in}{3.136866in}}%
\pgfpathlineto{\pgfqpoint{4.892545in}{3.133513in}}%
\pgfpathlineto{\pgfqpoint{4.898967in}{3.123949in}}%
\pgfpathlineto{\pgfqpoint{4.905925in}{3.114724in}}%
\pgfpathlineto{\pgfqpoint{4.909672in}{3.112832in}}%
\pgfpathlineto{\pgfqpoint{4.912883in}{3.113499in}}%
\pgfpathlineto{\pgfqpoint{4.916630in}{3.116788in}}%
\pgfpathlineto{\pgfqpoint{4.922784in}{3.125881in}}%
\pgfpathlineto{\pgfqpoint{4.930278in}{3.135838in}}%
\pgfpathlineto{\pgfqpoint{4.934024in}{3.137606in}}%
\pgfpathlineto{\pgfqpoint{4.937235in}{3.136827in}}%
\pgfpathlineto{\pgfqpoint{4.940982in}{3.133429in}}%
\pgfpathlineto{\pgfqpoint{4.947404in}{3.123836in}}%
\pgfpathlineto{\pgfqpoint{4.954362in}{3.114664in}}%
\pgfpathlineto{\pgfqpoint{4.958109in}{3.112823in}}%
\pgfpathlineto{\pgfqpoint{4.961320in}{3.113537in}}%
\pgfpathlineto{\pgfqpoint{4.965067in}{3.116871in}}%
\pgfpathlineto{\pgfqpoint{4.971222in}{3.125994in}}%
\pgfpathlineto{\pgfqpoint{4.978447in}{3.135670in}}%
\pgfpathlineto{\pgfqpoint{4.982461in}{3.137612in}}%
\pgfpathlineto{\pgfqpoint{4.985672in}{3.136787in}}%
\pgfpathlineto{\pgfqpoint{4.989419in}{3.133344in}}%
\pgfpathlineto{\pgfqpoint{4.995841in}{3.123724in}}%
\pgfpathlineto{\pgfqpoint{5.002799in}{3.114605in}}%
\pgfpathlineto{\pgfqpoint{5.006546in}{3.112816in}}%
\pgfpathlineto{\pgfqpoint{5.009757in}{3.113576in}}%
\pgfpathlineto{\pgfqpoint{5.013504in}{3.116955in}}%
\pgfpathlineto{\pgfqpoint{5.019926in}{3.126537in}}%
\pgfpathlineto{\pgfqpoint{5.026884in}{3.135730in}}%
\pgfpathlineto{\pgfqpoint{5.030630in}{3.137592in}}%
\pgfpathlineto{\pgfqpoint{5.033842in}{3.136898in}}%
\pgfpathlineto{\pgfqpoint{5.037588in}{3.133582in}}%
\pgfpathlineto{\pgfqpoint{5.043743in}{3.124470in}}%
\pgfpathlineto{\pgfqpoint{5.050969in}{3.114774in}}%
\pgfpathlineto{\pgfqpoint{5.054983in}{3.112810in}}%
\pgfpathlineto{\pgfqpoint{5.058194in}{3.113616in}}%
\pgfpathlineto{\pgfqpoint{5.061941in}{3.117040in}}%
\pgfpathlineto{\pgfqpoint{5.068363in}{3.126650in}}%
\pgfpathlineto{\pgfqpoint{5.075321in}{3.135790in}}%
\pgfpathlineto{\pgfqpoint{5.079067in}{3.137600in}}%
\pgfpathlineto{\pgfqpoint{5.082279in}{3.136859in}}%
\pgfpathlineto{\pgfqpoint{5.086025in}{3.133498in}}%
\pgfpathlineto{\pgfqpoint{5.092448in}{3.123927in}}%
\pgfpathlineto{\pgfqpoint{5.099406in}{3.114713in}}%
\pgfpathlineto{\pgfqpoint{5.103152in}{3.112830in}}%
\pgfpathlineto{\pgfqpoint{5.106363in}{3.113506in}}%
\pgfpathlineto{\pgfqpoint{5.110110in}{3.116803in}}%
\pgfpathlineto{\pgfqpoint{5.116265in}{3.125903in}}%
\pgfpathlineto{\pgfqpoint{5.123490in}{3.135620in}}%
\pgfpathlineto{\pgfqpoint{5.127504in}{3.137607in}}%
\pgfpathlineto{\pgfqpoint{5.130716in}{3.136819in}}%
\pgfpathlineto{\pgfqpoint{5.134462in}{3.133413in}}%
\pgfpathlineto{\pgfqpoint{5.140885in}{3.123815in}}%
\pgfpathlineto{\pgfqpoint{5.147843in}{3.114653in}}%
\pgfpathlineto{\pgfqpoint{5.151589in}{3.112822in}}%
\pgfpathlineto{\pgfqpoint{5.154800in}{3.113544in}}%
\pgfpathlineto{\pgfqpoint{5.158547in}{3.116887in}}%
\pgfpathlineto{\pgfqpoint{5.164702in}{3.126016in}}%
\pgfpathlineto{\pgfqpoint{5.171927in}{3.135681in}}%
\pgfpathlineto{\pgfqpoint{5.175941in}{3.137613in}}%
\pgfpathlineto{\pgfqpoint{5.179153in}{3.136779in}}%
\pgfpathlineto{\pgfqpoint{5.182899in}{3.133328in}}%
\pgfpathlineto{\pgfqpoint{5.189589in}{3.123275in}}%
\pgfpathlineto{\pgfqpoint{5.196280in}{3.114594in}}%
\pgfpathlineto{\pgfqpoint{5.200026in}{3.112815in}}%
\pgfpathlineto{\pgfqpoint{5.203237in}{3.113583in}}%
\pgfpathlineto{\pgfqpoint{5.206984in}{3.116971in}}%
\pgfpathlineto{\pgfqpoint{5.213407in}{3.126558in}}%
\pgfpathlineto{\pgfqpoint{5.220364in}{3.135742in}}%
\pgfpathlineto{\pgfqpoint{5.224111in}{3.137594in}}%
\pgfpathlineto{\pgfqpoint{5.227322in}{3.136891in}}%
\pgfpathlineto{\pgfqpoint{5.231069in}{3.133566in}}%
\pgfpathlineto{\pgfqpoint{5.237224in}{3.124449in}}%
\pgfpathlineto{\pgfqpoint{5.244449in}{3.114762in}}%
\pgfpathlineto{\pgfqpoint{5.248463in}{3.112808in}}%
\pgfpathlineto{\pgfqpoint{5.251675in}{3.113623in}}%
\pgfpathlineto{\pgfqpoint{5.255421in}{3.117056in}}%
\pgfpathlineto{\pgfqpoint{5.261844in}{3.126671in}}%
\pgfpathlineto{\pgfqpoint{5.268801in}{3.135801in}}%
\pgfpathlineto{\pgfqpoint{5.272548in}{3.137601in}}%
\pgfpathlineto{\pgfqpoint{5.275759in}{3.136852in}}%
\pgfpathlineto{\pgfqpoint{5.279506in}{3.133482in}}%
\pgfpathlineto{\pgfqpoint{5.285928in}{3.123906in}}%
\pgfpathlineto{\pgfqpoint{5.292886in}{3.114702in}}%
\pgfpathlineto{\pgfqpoint{5.296633in}{3.112828in}}%
\pgfpathlineto{\pgfqpoint{5.299844in}{3.113513in}}%
\pgfpathlineto{\pgfqpoint{5.303590in}{3.116819in}}%
\pgfpathlineto{\pgfqpoint{5.309745in}{3.125924in}}%
\pgfpathlineto{\pgfqpoint{5.316971in}{3.135632in}}%
\pgfpathlineto{\pgfqpoint{5.320985in}{3.137608in}}%
\pgfpathlineto{\pgfqpoint{5.324196in}{3.136812in}}%
\pgfpathlineto{\pgfqpoint{5.327943in}{3.133397in}}%
\pgfpathlineto{\pgfqpoint{5.334365in}{3.123794in}}%
\pgfpathlineto{\pgfqpoint{5.341323in}{3.114642in}}%
\pgfpathlineto{\pgfqpoint{5.345070in}{3.112820in}}%
\pgfpathlineto{\pgfqpoint{5.348281in}{3.113551in}}%
\pgfpathlineto{\pgfqpoint{5.352027in}{3.116903in}}%
\pgfpathlineto{\pgfqpoint{5.358450in}{3.126467in}}%
\pgfpathlineto{\pgfqpoint{5.365408in}{3.135693in}}%
\pgfpathlineto{\pgfqpoint{5.369154in}{3.137587in}}%
\pgfpathlineto{\pgfqpoint{5.372366in}{3.136921in}}%
\pgfpathlineto{\pgfqpoint{5.376112in}{3.133634in}}%
\pgfpathlineto{\pgfqpoint{5.382267in}{3.124541in}}%
\pgfpathlineto{\pgfqpoint{5.389760in}{3.114583in}}%
\pgfpathlineto{\pgfqpoint{5.393507in}{3.112813in}}%
\pgfpathlineto{\pgfqpoint{5.396718in}{3.113591in}}%
\pgfpathlineto{\pgfqpoint{5.400464in}{3.116987in}}%
\pgfpathlineto{\pgfqpoint{5.406887in}{3.126579in}}%
\pgfpathlineto{\pgfqpoint{5.413845in}{3.135753in}}%
\pgfpathlineto{\pgfqpoint{5.417591in}{3.137595in}}%
\pgfpathlineto{\pgfqpoint{5.420803in}{3.136883in}}%
\pgfpathlineto{\pgfqpoint{5.424549in}{3.133550in}}%
\pgfpathlineto{\pgfqpoint{5.430704in}{3.124428in}}%
\pgfpathlineto{\pgfqpoint{5.437930in}{3.114751in}}%
\pgfpathlineto{\pgfqpoint{5.441944in}{3.112807in}}%
\pgfpathlineto{\pgfqpoint{5.445155in}{3.113631in}}%
\pgfpathlineto{\pgfqpoint{5.448901in}{3.117072in}}%
\pgfpathlineto{\pgfqpoint{5.455324in}{3.126692in}}%
\pgfpathlineto{\pgfqpoint{5.462282in}{3.135812in}}%
\pgfpathlineto{\pgfqpoint{5.466028in}{3.137603in}}%
\pgfpathlineto{\pgfqpoint{5.469240in}{3.136844in}}%
\pgfpathlineto{\pgfqpoint{5.472986in}{3.133466in}}%
\pgfpathlineto{\pgfqpoint{5.479409in}{3.123885in}}%
\pgfpathlineto{\pgfqpoint{5.486367in}{3.114690in}}%
\pgfpathlineto{\pgfqpoint{5.490113in}{3.112827in}}%
\pgfpathlineto{\pgfqpoint{5.493324in}{3.113520in}}%
\pgfpathlineto{\pgfqpoint{5.497071in}{3.116835in}}%
\pgfpathlineto{\pgfqpoint{5.503226in}{3.125945in}}%
\pgfpathlineto{\pgfqpoint{5.510451in}{3.135643in}}%
\pgfpathlineto{\pgfqpoint{5.514465in}{3.137609in}}%
\pgfpathlineto{\pgfqpoint{5.517677in}{3.136804in}}%
\pgfpathlineto{\pgfqpoint{5.521423in}{3.133381in}}%
\pgfpathlineto{\pgfqpoint{5.527846in}{3.123773in}}%
\pgfpathlineto{\pgfqpoint{5.534804in}{3.114631in}}%
\pgfpathlineto{\pgfqpoint{5.538550in}{3.112819in}}%
\pgfpathlineto{\pgfqpoint{5.541761in}{3.113559in}}%
\pgfpathlineto{\pgfqpoint{5.545508in}{3.116919in}}%
\pgfpathlineto{\pgfqpoint{5.551930in}{3.126488in}}%
\pgfpathlineto{\pgfqpoint{5.558888in}{3.135704in}}%
\pgfpathlineto{\pgfqpoint{5.562635in}{3.137589in}}%
\pgfpathlineto{\pgfqpoint{5.565846in}{3.136914in}}%
\pgfpathlineto{\pgfqpoint{5.569593in}{3.133618in}}%
\pgfpathlineto{\pgfqpoint{5.575748in}{3.124520in}}%
\pgfpathlineto{\pgfqpoint{5.582973in}{3.114801in}}%
\pgfpathlineto{\pgfqpoint{5.586987in}{3.112812in}}%
\pgfpathlineto{\pgfqpoint{5.590198in}{3.113598in}}%
\pgfpathlineto{\pgfqpoint{5.593945in}{3.117003in}}%
\pgfpathlineto{\pgfqpoint{5.600367in}{3.126600in}}%
\pgfpathlineto{\pgfqpoint{5.607325in}{3.135764in}}%
\pgfpathlineto{\pgfqpoint{5.611072in}{3.137597in}}%
\pgfpathlineto{\pgfqpoint{5.614283in}{3.136876in}}%
\pgfpathlineto{\pgfqpoint{5.618030in}{3.133535in}}%
\pgfpathlineto{\pgfqpoint{5.624185in}{3.124407in}}%
\pgfpathlineto{\pgfqpoint{5.631410in}{3.114739in}}%
\pgfpathlineto{\pgfqpoint{5.635424in}{3.112806in}}%
\pgfpathlineto{\pgfqpoint{5.638635in}{3.113639in}}%
\pgfpathlineto{\pgfqpoint{5.642382in}{3.117088in}}%
\pgfpathlineto{\pgfqpoint{5.649072in}{3.127140in}}%
\pgfpathlineto{\pgfqpoint{5.655762in}{3.135823in}}%
\pgfpathlineto{\pgfqpoint{5.659509in}{3.137604in}}%
\pgfpathlineto{\pgfqpoint{5.662720in}{3.136837in}}%
\pgfpathlineto{\pgfqpoint{5.666467in}{3.133450in}}%
\pgfpathlineto{\pgfqpoint{5.672889in}{3.123864in}}%
\pgfpathlineto{\pgfqpoint{5.679847in}{3.114679in}}%
\pgfpathlineto{\pgfqpoint{5.683594in}{3.112825in}}%
\pgfpathlineto{\pgfqpoint{5.686805in}{3.113527in}}%
\pgfpathlineto{\pgfqpoint{5.690551in}{3.116850in}}%
\pgfpathlineto{\pgfqpoint{5.696706in}{3.125966in}}%
\pgfpathlineto{\pgfqpoint{5.703932in}{3.135655in}}%
\pgfpathlineto{\pgfqpoint{5.707946in}{3.137610in}}%
\pgfpathlineto{\pgfqpoint{5.711157in}{3.136797in}}%
\pgfpathlineto{\pgfqpoint{5.714904in}{3.133365in}}%
\pgfpathlineto{\pgfqpoint{5.721326in}{3.123752in}}%
\pgfpathlineto{\pgfqpoint{5.728284in}{3.114619in}}%
\pgfpathlineto{\pgfqpoint{5.732031in}{3.112818in}}%
\pgfpathlineto{\pgfqpoint{5.735242in}{3.113566in}}%
\pgfpathlineto{\pgfqpoint{5.738988in}{3.116934in}}%
\pgfpathlineto{\pgfqpoint{5.745411in}{3.126509in}}%
\pgfpathlineto{\pgfqpoint{5.752369in}{3.135715in}}%
\pgfpathlineto{\pgfqpoint{5.756115in}{3.137590in}}%
\pgfpathlineto{\pgfqpoint{5.759327in}{3.136907in}}%
\pgfpathlineto{\pgfqpoint{5.763073in}{3.133603in}}%
\pgfpathlineto{\pgfqpoint{5.769228in}{3.124499in}}%
\pgfpathlineto{\pgfqpoint{5.776453in}{3.114789in}}%
\pgfpathlineto{\pgfqpoint{5.780468in}{3.112811in}}%
\pgfpathlineto{\pgfqpoint{5.783679in}{3.113606in}}%
\pgfpathlineto{\pgfqpoint{5.787425in}{3.117019in}}%
\pgfpathlineto{\pgfqpoint{5.793848in}{3.126622in}}%
\pgfpathlineto{\pgfqpoint{5.800806in}{3.135775in}}%
\pgfpathlineto{\pgfqpoint{5.804552in}{3.137598in}}%
\pgfpathlineto{\pgfqpoint{5.807764in}{3.136869in}}%
\pgfpathlineto{\pgfqpoint{5.811510in}{3.133519in}}%
\pgfpathlineto{\pgfqpoint{5.817933in}{3.123956in}}%
\pgfpathlineto{\pgfqpoint{5.824890in}{3.114728in}}%
\pgfpathlineto{\pgfqpoint{5.828637in}{3.112832in}}%
\pgfpathlineto{\pgfqpoint{5.831848in}{3.113496in}}%
\pgfpathlineto{\pgfqpoint{5.835595in}{3.116783in}}%
\pgfpathlineto{\pgfqpoint{5.841750in}{3.125874in}}%
\pgfpathlineto{\pgfqpoint{5.849243in}{3.135834in}}%
\pgfpathlineto{\pgfqpoint{5.852989in}{3.137605in}}%
\pgfpathlineto{\pgfqpoint{5.856201in}{3.136829in}}%
\pgfpathlineto{\pgfqpoint{5.859947in}{3.133434in}}%
\pgfpathlineto{\pgfqpoint{5.866370in}{3.123843in}}%
\pgfpathlineto{\pgfqpoint{5.873327in}{3.114668in}}%
\pgfpathlineto{\pgfqpoint{5.877074in}{3.112824in}}%
\pgfpathlineto{\pgfqpoint{5.880285in}{3.113534in}}%
\pgfpathlineto{\pgfqpoint{5.884032in}{3.116866in}}%
\pgfpathlineto{\pgfqpoint{5.890187in}{3.125987in}}%
\pgfpathlineto{\pgfqpoint{5.897412in}{3.135666in}}%
\pgfpathlineto{\pgfqpoint{5.901426in}{3.137611in}}%
\pgfpathlineto{\pgfqpoint{5.904638in}{3.136789in}}%
\pgfpathlineto{\pgfqpoint{5.908384in}{3.133349in}}%
\pgfpathlineto{\pgfqpoint{5.914807in}{3.123731in}}%
\pgfpathlineto{\pgfqpoint{5.921764in}{3.114608in}}%
\pgfpathlineto{\pgfqpoint{5.925511in}{3.112816in}}%
\pgfpathlineto{\pgfqpoint{5.928722in}{3.113573in}}%
\pgfpathlineto{\pgfqpoint{5.932469in}{3.116950in}}%
\pgfpathlineto{\pgfqpoint{5.938891in}{3.126530in}}%
\pgfpathlineto{\pgfqpoint{5.945849in}{3.135727in}}%
\pgfpathlineto{\pgfqpoint{5.949596in}{3.137592in}}%
\pgfpathlineto{\pgfqpoint{5.952807in}{3.136900in}}%
\pgfpathlineto{\pgfqpoint{5.956554in}{3.133587in}}%
\pgfpathlineto{\pgfqpoint{5.962708in}{3.124477in}}%
\pgfpathlineto{\pgfqpoint{5.969934in}{3.114778in}}%
\pgfpathlineto{\pgfqpoint{5.973948in}{3.112810in}}%
\pgfpathlineto{\pgfqpoint{5.977159in}{3.113613in}}%
\pgfpathlineto{\pgfqpoint{5.980906in}{3.117035in}}%
\pgfpathlineto{\pgfqpoint{5.987328in}{3.126643in}}%
\pgfpathlineto{\pgfqpoint{5.994286in}{3.135786in}}%
\pgfpathlineto{\pgfqpoint{5.998033in}{3.137600in}}%
\pgfpathlineto{\pgfqpoint{6.001244in}{3.136862in}}%
\pgfpathlineto{\pgfqpoint{6.004991in}{3.133503in}}%
\pgfpathlineto{\pgfqpoint{6.011413in}{3.123934in}}%
\pgfpathlineto{\pgfqpoint{6.018371in}{3.114717in}}%
\pgfpathlineto{\pgfqpoint{6.022117in}{3.112830in}}%
\pgfpathlineto{\pgfqpoint{6.025329in}{3.113503in}}%
\pgfpathlineto{\pgfqpoint{6.029075in}{3.116798in}}%
\pgfpathlineto{\pgfqpoint{6.035230in}{3.125895in}}%
\pgfpathlineto{\pgfqpoint{6.042723in}{3.135845in}}%
\pgfpathlineto{\pgfqpoint{6.046470in}{3.137606in}}%
\pgfpathlineto{\pgfqpoint{6.049681in}{3.136822in}}%
\pgfpathlineto{\pgfqpoint{6.053428in}{3.133418in}}%
\pgfpathlineto{\pgfqpoint{6.059850in}{3.123822in}}%
\pgfpathlineto{\pgfqpoint{6.066808in}{3.114657in}}%
\pgfpathlineto{\pgfqpoint{6.070554in}{3.112822in}}%
\pgfpathlineto{\pgfqpoint{6.073766in}{3.113542in}}%
\pgfpathlineto{\pgfqpoint{6.077512in}{3.116882in}}%
\pgfpathlineto{\pgfqpoint{6.083667in}{3.126008in}}%
\pgfpathlineto{\pgfqpoint{6.090893in}{3.135678in}}%
\pgfpathlineto{\pgfqpoint{6.094907in}{3.137612in}}%
\pgfpathlineto{\pgfqpoint{6.098118in}{3.136781in}}%
\pgfpathlineto{\pgfqpoint{6.101865in}{3.133333in}}%
\pgfpathlineto{\pgfqpoint{6.108287in}{3.123709in}}%
\pgfpathlineto{\pgfqpoint{6.115245in}{3.114597in}}%
\pgfpathlineto{\pgfqpoint{6.118991in}{3.112815in}}%
\pgfpathlineto{\pgfqpoint{6.122203in}{3.113581in}}%
\pgfpathlineto{\pgfqpoint{6.125949in}{3.116966in}}%
\pgfpathlineto{\pgfqpoint{6.132372in}{3.126551in}}%
\pgfpathlineto{\pgfqpoint{6.139330in}{3.135738in}}%
\pgfpathlineto{\pgfqpoint{6.143076in}{3.137593in}}%
\pgfpathlineto{\pgfqpoint{6.146287in}{3.136893in}}%
\pgfpathlineto{\pgfqpoint{6.150034in}{3.133571in}}%
\pgfpathlineto{\pgfqpoint{6.156189in}{3.124456in}}%
\pgfpathlineto{\pgfqpoint{6.163414in}{3.114766in}}%
\pgfpathlineto{\pgfqpoint{6.167428in}{3.112809in}}%
\pgfpathlineto{\pgfqpoint{6.170640in}{3.113621in}}%
\pgfpathlineto{\pgfqpoint{6.174386in}{3.117051in}}%
\pgfpathlineto{\pgfqpoint{6.180809in}{3.126664in}}%
\pgfpathlineto{\pgfqpoint{6.187767in}{3.135798in}}%
\pgfpathlineto{\pgfqpoint{6.191513in}{3.137601in}}%
\pgfpathlineto{\pgfqpoint{6.194724in}{3.136854in}}%
\pgfpathlineto{\pgfqpoint{6.198471in}{3.133487in}}%
\pgfpathlineto{\pgfqpoint{6.204894in}{3.123913in}}%
\pgfpathlineto{\pgfqpoint{6.211851in}{3.114705in}}%
\pgfpathlineto{\pgfqpoint{6.215598in}{3.112829in}}%
\pgfpathlineto{\pgfqpoint{6.218809in}{3.113510in}}%
\pgfpathlineto{\pgfqpoint{6.222556in}{3.116814in}}%
\pgfpathlineto{\pgfqpoint{6.228711in}{3.125917in}}%
\pgfpathlineto{\pgfqpoint{6.235936in}{3.135628in}}%
\pgfpathlineto{\pgfqpoint{6.239950in}{3.137608in}}%
\pgfpathlineto{\pgfqpoint{6.243161in}{3.136814in}}%
\pgfpathlineto{\pgfqpoint{6.246908in}{3.133402in}}%
\pgfpathlineto{\pgfqpoint{6.253331in}{3.123801in}}%
\pgfpathlineto{\pgfqpoint{6.260288in}{3.114645in}}%
\pgfpathlineto{\pgfqpoint{6.264035in}{3.112821in}}%
\pgfpathlineto{\pgfqpoint{6.267246in}{3.113549in}}%
\pgfpathlineto{\pgfqpoint{6.270993in}{3.116897in}}%
\pgfpathlineto{\pgfqpoint{6.277415in}{3.126460in}}%
\pgfpathlineto{\pgfqpoint{6.284373in}{3.135689in}}%
\pgfpathlineto{\pgfqpoint{6.288120in}{3.137586in}}%
\pgfpathlineto{\pgfqpoint{6.291331in}{3.136924in}}%
\pgfpathlineto{\pgfqpoint{6.295077in}{3.133639in}}%
\pgfpathlineto{\pgfqpoint{6.301232in}{3.124548in}}%
\pgfpathlineto{\pgfqpoint{6.308725in}{3.114586in}}%
\pgfpathlineto{\pgfqpoint{6.312472in}{3.112814in}}%
\pgfpathlineto{\pgfqpoint{6.315683in}{3.113588in}}%
\pgfpathlineto{\pgfqpoint{6.319430in}{3.116982in}}%
\pgfpathlineto{\pgfqpoint{6.325852in}{3.126572in}}%
\pgfpathlineto{\pgfqpoint{6.332810in}{3.135749in}}%
\pgfpathlineto{\pgfqpoint{6.336557in}{3.137595in}}%
\pgfpathlineto{\pgfqpoint{6.339768in}{3.136886in}}%
\pgfpathlineto{\pgfqpoint{6.343514in}{3.133555in}}%
\pgfpathlineto{\pgfqpoint{6.349669in}{3.124435in}}%
\pgfpathlineto{\pgfqpoint{6.356895in}{3.114755in}}%
\pgfpathlineto{\pgfqpoint{6.360909in}{3.112808in}}%
\pgfpathlineto{\pgfqpoint{6.364120in}{3.113628in}}%
\pgfpathlineto{\pgfqpoint{6.367867in}{3.117067in}}%
\pgfpathlineto{\pgfqpoint{6.374289in}{3.126685in}}%
\pgfpathlineto{\pgfqpoint{6.381247in}{3.135809in}}%
\pgfpathlineto{\pgfqpoint{6.384994in}{3.137602in}}%
\pgfpathlineto{\pgfqpoint{6.388205in}{3.136847in}}%
\pgfpathlineto{\pgfqpoint{6.391951in}{3.133471in}}%
\pgfpathlineto{\pgfqpoint{6.398374in}{3.123892in}}%
\pgfpathlineto{\pgfqpoint{6.405332in}{3.114694in}}%
\pgfpathlineto{\pgfqpoint{6.409078in}{3.112827in}}%
\pgfpathlineto{\pgfqpoint{6.412290in}{3.113518in}}%
\pgfpathlineto{\pgfqpoint{6.416036in}{3.116829in}}%
\pgfpathlineto{\pgfqpoint{6.422191in}{3.125938in}}%
\pgfpathlineto{\pgfqpoint{6.429417in}{3.135639in}}%
\pgfpathlineto{\pgfqpoint{6.433431in}{3.137609in}}%
\pgfpathlineto{\pgfqpoint{6.436642in}{3.136807in}}%
\pgfpathlineto{\pgfqpoint{6.440388in}{3.133386in}}%
\pgfpathlineto{\pgfqpoint{6.446811in}{3.123780in}}%
\pgfpathlineto{\pgfqpoint{6.453769in}{3.114634in}}%
\pgfpathlineto{\pgfqpoint{6.457515in}{3.112819in}}%
\pgfpathlineto{\pgfqpoint{6.460727in}{3.113556in}}%
\pgfpathlineto{\pgfqpoint{6.464473in}{3.116913in}}%
\pgfpathlineto{\pgfqpoint{6.470896in}{3.126481in}}%
\pgfpathlineto{\pgfqpoint{6.477854in}{3.135700in}}%
\pgfpathlineto{\pgfqpoint{6.481600in}{3.137588in}}%
\pgfpathlineto{\pgfqpoint{6.484811in}{3.136917in}}%
\pgfpathlineto{\pgfqpoint{6.488558in}{3.133623in}}%
\pgfpathlineto{\pgfqpoint{6.494713in}{3.124527in}}%
\pgfpathlineto{\pgfqpoint{6.502206in}{3.114575in}}%
\pgfpathlineto{\pgfqpoint{6.505952in}{3.112813in}}%
\pgfpathlineto{\pgfqpoint{6.509164in}{3.113596in}}%
\pgfpathlineto{\pgfqpoint{6.512910in}{3.116998in}}%
\pgfpathlineto{\pgfqpoint{6.519333in}{3.126593in}}%
\pgfpathlineto{\pgfqpoint{6.526291in}{3.135760in}}%
\pgfpathlineto{\pgfqpoint{6.530037in}{3.137596in}}%
\pgfpathlineto{\pgfqpoint{6.533248in}{3.136879in}}%
\pgfpathlineto{\pgfqpoint{6.536995in}{3.133540in}}%
\pgfpathlineto{\pgfqpoint{6.543150in}{3.124414in}}%
\pgfpathlineto{\pgfqpoint{6.550375in}{3.114743in}}%
\pgfpathlineto{\pgfqpoint{6.554389in}{3.112807in}}%
\pgfpathlineto{\pgfqpoint{6.557601in}{3.113636in}}%
\pgfpathlineto{\pgfqpoint{6.561347in}{3.117083in}}%
\pgfpathlineto{\pgfqpoint{6.567770in}{3.126706in}}%
\pgfpathlineto{\pgfqpoint{6.574728in}{3.135820in}}%
\pgfpathlineto{\pgfqpoint{6.578474in}{3.137604in}}%
\pgfpathlineto{\pgfqpoint{6.581685in}{3.136839in}}%
\pgfpathlineto{\pgfqpoint{6.585432in}{3.133455in}}%
\pgfpathlineto{\pgfqpoint{6.591854in}{3.123871in}}%
\pgfpathlineto{\pgfqpoint{6.598812in}{3.114683in}}%
\pgfpathlineto{\pgfqpoint{6.602559in}{3.112826in}}%
\pgfpathlineto{\pgfqpoint{6.605770in}{3.113525in}}%
\pgfpathlineto{\pgfqpoint{6.609517in}{3.116845in}}%
\pgfpathlineto{\pgfqpoint{6.615672in}{3.125959in}}%
\pgfpathlineto{\pgfqpoint{6.622897in}{3.135651in}}%
\pgfpathlineto{\pgfqpoint{6.626911in}{3.137610in}}%
\pgfpathlineto{\pgfqpoint{6.630122in}{3.136799in}}%
\pgfpathlineto{\pgfqpoint{6.633869in}{3.133370in}}%
\pgfpathlineto{\pgfqpoint{6.640291in}{3.123759in}}%
\pgfpathlineto{\pgfqpoint{6.647249in}{3.114623in}}%
\pgfpathlineto{\pgfqpoint{6.650996in}{3.112818in}}%
\pgfpathlineto{\pgfqpoint{6.654207in}{3.113563in}}%
\pgfpathlineto{\pgfqpoint{6.657954in}{3.116929in}}%
\pgfpathlineto{\pgfqpoint{6.663306in}{3.124778in}}%
\pgfpathlineto{\pgfqpoint{6.663306in}{3.124778in}}%
\pgfusepath{stroke}%
\end{pgfscope}%
\begin{pgfscope}%
\pgfpathrectangle{\pgfqpoint{0.467797in}{2.292089in}}{\pgfqpoint{6.490533in}{1.666241in}}%
\pgfusepath{clip}%
\pgfsetrectcap%
\pgfsetroundjoin%
\pgfsetlinewidth{1.505625pt}%
\definecolor{currentstroke}{rgb}{1.000000,0.498039,0.054902}%
\pgfsetstrokecolor{currentstroke}%
\pgfsetdash{}{0pt}%
\pgfpathmoveto{\pgfqpoint{0.762821in}{3.125209in}}%
\pgfpathlineto{\pgfqpoint{0.770314in}{3.135360in}}%
\pgfpathlineto{\pgfqpoint{0.774060in}{3.137208in}}%
\pgfpathlineto{\pgfqpoint{0.777272in}{3.136426in}}%
\pgfpathlineto{\pgfqpoint{0.781018in}{3.132955in}}%
\pgfpathlineto{\pgfqpoint{0.787708in}{3.122851in}}%
\pgfpathlineto{\pgfqpoint{0.794131in}{3.114727in}}%
\pgfpathlineto{\pgfqpoint{0.797878in}{3.113188in}}%
\pgfpathlineto{\pgfqpoint{0.801089in}{3.114242in}}%
\pgfpathlineto{\pgfqpoint{0.805103in}{3.118319in}}%
\pgfpathlineto{\pgfqpoint{0.820357in}{3.137123in}}%
\pgfpathlineto{\pgfqpoint{0.823568in}{3.136708in}}%
\pgfpathlineto{\pgfqpoint{0.827047in}{3.133899in}}%
\pgfpathlineto{\pgfqpoint{0.832667in}{3.125857in}}%
\pgfpathlineto{\pgfqpoint{0.840695in}{3.114945in}}%
\pgfpathlineto{\pgfqpoint{0.844441in}{3.113200in}}%
\pgfpathlineto{\pgfqpoint{0.847653in}{3.114072in}}%
\pgfpathlineto{\pgfqpoint{0.851399in}{3.117630in}}%
\pgfpathlineto{\pgfqpoint{0.858357in}{3.128199in}}%
\pgfpathlineto{\pgfqpoint{0.864512in}{3.135796in}}%
\pgfpathlineto{\pgfqpoint{0.868258in}{3.137231in}}%
\pgfpathlineto{\pgfqpoint{0.871470in}{3.136087in}}%
\pgfpathlineto{\pgfqpoint{0.875484in}{3.131922in}}%
\pgfpathlineto{\pgfqpoint{0.890470in}{3.113327in}}%
\pgfpathlineto{\pgfqpoint{0.893681in}{3.113650in}}%
\pgfpathlineto{\pgfqpoint{0.897160in}{3.116372in}}%
\pgfpathlineto{\pgfqpoint{0.902512in}{3.123917in}}%
\pgfpathlineto{\pgfqpoint{0.911076in}{3.135585in}}%
\pgfpathlineto{\pgfqpoint{0.914822in}{3.137227in}}%
\pgfpathlineto{\pgfqpoint{0.918033in}{3.136264in}}%
\pgfpathlineto{\pgfqpoint{0.921780in}{3.132620in}}%
\pgfpathlineto{\pgfqpoint{0.929273in}{3.121188in}}%
\pgfpathlineto{\pgfqpoint{0.934893in}{3.114522in}}%
\pgfpathlineto{\pgfqpoint{0.938639in}{3.113192in}}%
\pgfpathlineto{\pgfqpoint{0.941851in}{3.114425in}}%
\pgfpathlineto{\pgfqpoint{0.945865in}{3.118677in}}%
\pgfpathlineto{\pgfqpoint{0.960583in}{3.137057in}}%
\pgfpathlineto{\pgfqpoint{0.963794in}{3.136826in}}%
\pgfpathlineto{\pgfqpoint{0.967273in}{3.134191in}}%
\pgfpathlineto{\pgfqpoint{0.972625in}{3.126716in}}%
\pgfpathlineto{\pgfqpoint{0.981189in}{3.114945in}}%
\pgfpathlineto{\pgfqpoint{0.984935in}{3.113200in}}%
\pgfpathlineto{\pgfqpoint{0.988147in}{3.114072in}}%
\pgfpathlineto{\pgfqpoint{0.991893in}{3.117630in}}%
\pgfpathlineto{\pgfqpoint{0.998851in}{3.128199in}}%
\pgfpathlineto{\pgfqpoint{1.005006in}{3.135796in}}%
\pgfpathlineto{\pgfqpoint{1.008752in}{3.137231in}}%
\pgfpathlineto{\pgfqpoint{1.011964in}{3.136087in}}%
\pgfpathlineto{\pgfqpoint{1.015978in}{3.131922in}}%
\pgfpathlineto{\pgfqpoint{1.030964in}{3.113327in}}%
\pgfpathlineto{\pgfqpoint{1.034175in}{3.113650in}}%
\pgfpathlineto{\pgfqpoint{1.037654in}{3.116372in}}%
\pgfpathlineto{\pgfqpoint{1.043006in}{3.123917in}}%
\pgfpathlineto{\pgfqpoint{1.051570in}{3.135585in}}%
\pgfpathlineto{\pgfqpoint{1.055316in}{3.137227in}}%
\pgfpathlineto{\pgfqpoint{1.058528in}{3.136264in}}%
\pgfpathlineto{\pgfqpoint{1.062274in}{3.132620in}}%
\pgfpathlineto{\pgfqpoint{1.069767in}{3.121188in}}%
\pgfpathlineto{\pgfqpoint{1.075387in}{3.114522in}}%
\pgfpathlineto{\pgfqpoint{1.079133in}{3.113192in}}%
\pgfpathlineto{\pgfqpoint{1.082345in}{3.114425in}}%
\pgfpathlineto{\pgfqpoint{1.086359in}{3.118677in}}%
\pgfpathlineto{\pgfqpoint{1.101077in}{3.137057in}}%
\pgfpathlineto{\pgfqpoint{1.104288in}{3.136826in}}%
\pgfpathlineto{\pgfqpoint{1.107767in}{3.134191in}}%
\pgfpathlineto{\pgfqpoint{1.113120in}{3.126716in}}%
\pgfpathlineto{\pgfqpoint{1.121683in}{3.114945in}}%
\pgfpathlineto{\pgfqpoint{1.125430in}{3.113200in}}%
\pgfpathlineto{\pgfqpoint{1.128641in}{3.114072in}}%
\pgfpathlineto{\pgfqpoint{1.132387in}{3.117630in}}%
\pgfpathlineto{\pgfqpoint{1.139345in}{3.128199in}}%
\pgfpathlineto{\pgfqpoint{1.145500in}{3.135796in}}%
\pgfpathlineto{\pgfqpoint{1.149247in}{3.137231in}}%
\pgfpathlineto{\pgfqpoint{1.152458in}{3.136087in}}%
\pgfpathlineto{\pgfqpoint{1.156472in}{3.131922in}}%
\pgfpathlineto{\pgfqpoint{1.171458in}{3.113327in}}%
\pgfpathlineto{\pgfqpoint{1.174669in}{3.113650in}}%
\pgfpathlineto{\pgfqpoint{1.178148in}{3.116372in}}%
\pgfpathlineto{\pgfqpoint{1.183500in}{3.123917in}}%
\pgfpathlineto{\pgfqpoint{1.192064in}{3.135585in}}%
\pgfpathlineto{\pgfqpoint{1.195810in}{3.137227in}}%
\pgfpathlineto{\pgfqpoint{1.199022in}{3.136264in}}%
\pgfpathlineto{\pgfqpoint{1.202768in}{3.132620in}}%
\pgfpathlineto{\pgfqpoint{1.210261in}{3.121188in}}%
\pgfpathlineto{\pgfqpoint{1.215881in}{3.114522in}}%
\pgfpathlineto{\pgfqpoint{1.219627in}{3.113192in}}%
\pgfpathlineto{\pgfqpoint{1.222839in}{3.114425in}}%
\pgfpathlineto{\pgfqpoint{1.226853in}{3.118677in}}%
\pgfpathlineto{\pgfqpoint{1.241571in}{3.137057in}}%
\pgfpathlineto{\pgfqpoint{1.244783in}{3.136826in}}%
\pgfpathlineto{\pgfqpoint{1.248261in}{3.134191in}}%
\pgfpathlineto{\pgfqpoint{1.253614in}{3.126716in}}%
\pgfpathlineto{\pgfqpoint{1.262177in}{3.114945in}}%
\pgfpathlineto{\pgfqpoint{1.265924in}{3.113200in}}%
\pgfpathlineto{\pgfqpoint{1.269135in}{3.114072in}}%
\pgfpathlineto{\pgfqpoint{1.272881in}{3.117630in}}%
\pgfpathlineto{\pgfqpoint{1.279839in}{3.128199in}}%
\pgfpathlineto{\pgfqpoint{1.285994in}{3.135796in}}%
\pgfpathlineto{\pgfqpoint{1.289741in}{3.137231in}}%
\pgfpathlineto{\pgfqpoint{1.292952in}{3.136087in}}%
\pgfpathlineto{\pgfqpoint{1.296966in}{3.131922in}}%
\pgfpathlineto{\pgfqpoint{1.311952in}{3.113327in}}%
\pgfpathlineto{\pgfqpoint{1.315163in}{3.113650in}}%
\pgfpathlineto{\pgfqpoint{1.318642in}{3.116372in}}%
\pgfpathlineto{\pgfqpoint{1.323995in}{3.123917in}}%
\pgfpathlineto{\pgfqpoint{1.332558in}{3.135585in}}%
\pgfpathlineto{\pgfqpoint{1.336304in}{3.137227in}}%
\pgfpathlineto{\pgfqpoint{1.339516in}{3.136264in}}%
\pgfpathlineto{\pgfqpoint{1.343262in}{3.132620in}}%
\pgfpathlineto{\pgfqpoint{1.350755in}{3.121188in}}%
\pgfpathlineto{\pgfqpoint{1.356375in}{3.114522in}}%
\pgfpathlineto{\pgfqpoint{1.360122in}{3.113192in}}%
\pgfpathlineto{\pgfqpoint{1.363333in}{3.114425in}}%
\pgfpathlineto{\pgfqpoint{1.367347in}{3.118677in}}%
\pgfpathlineto{\pgfqpoint{1.382065in}{3.137057in}}%
\pgfpathlineto{\pgfqpoint{1.385277in}{3.136826in}}%
\pgfpathlineto{\pgfqpoint{1.388756in}{3.134191in}}%
\pgfpathlineto{\pgfqpoint{1.394108in}{3.126716in}}%
\pgfpathlineto{\pgfqpoint{1.402671in}{3.114945in}}%
\pgfpathlineto{\pgfqpoint{1.406418in}{3.113200in}}%
\pgfpathlineto{\pgfqpoint{1.409629in}{3.114072in}}%
\pgfpathlineto{\pgfqpoint{1.413376in}{3.117630in}}%
\pgfpathlineto{\pgfqpoint{1.420333in}{3.128199in}}%
\pgfpathlineto{\pgfqpoint{1.426488in}{3.135796in}}%
\pgfpathlineto{\pgfqpoint{1.430235in}{3.137231in}}%
\pgfpathlineto{\pgfqpoint{1.433446in}{3.136087in}}%
\pgfpathlineto{\pgfqpoint{1.437460in}{3.131922in}}%
\pgfpathlineto{\pgfqpoint{1.452446in}{3.113327in}}%
\pgfpathlineto{\pgfqpoint{1.455658in}{3.113650in}}%
\pgfpathlineto{\pgfqpoint{1.459136in}{3.116372in}}%
\pgfpathlineto{\pgfqpoint{1.464489in}{3.123917in}}%
\pgfpathlineto{\pgfqpoint{1.473052in}{3.135585in}}%
\pgfpathlineto{\pgfqpoint{1.476799in}{3.137227in}}%
\pgfpathlineto{\pgfqpoint{1.480010in}{3.136264in}}%
\pgfpathlineto{\pgfqpoint{1.483756in}{3.132620in}}%
\pgfpathlineto{\pgfqpoint{1.491249in}{3.121188in}}%
\pgfpathlineto{\pgfqpoint{1.496869in}{3.114522in}}%
\pgfpathlineto{\pgfqpoint{1.500616in}{3.113192in}}%
\pgfpathlineto{\pgfqpoint{1.503827in}{3.114425in}}%
\pgfpathlineto{\pgfqpoint{1.507841in}{3.118677in}}%
\pgfpathlineto{\pgfqpoint{1.522560in}{3.137057in}}%
\pgfpathlineto{\pgfqpoint{1.525771in}{3.136826in}}%
\pgfpathlineto{\pgfqpoint{1.529250in}{3.134191in}}%
\pgfpathlineto{\pgfqpoint{1.534602in}{3.126716in}}%
\pgfpathlineto{\pgfqpoint{1.543165in}{3.114945in}}%
\pgfpathlineto{\pgfqpoint{1.546912in}{3.113200in}}%
\pgfpathlineto{\pgfqpoint{1.550123in}{3.114072in}}%
\pgfpathlineto{\pgfqpoint{1.553870in}{3.117630in}}%
\pgfpathlineto{\pgfqpoint{1.560827in}{3.128199in}}%
\pgfpathlineto{\pgfqpoint{1.566982in}{3.135796in}}%
\pgfpathlineto{\pgfqpoint{1.570729in}{3.137231in}}%
\pgfpathlineto{\pgfqpoint{1.573940in}{3.136087in}}%
\pgfpathlineto{\pgfqpoint{1.577954in}{3.131922in}}%
\pgfpathlineto{\pgfqpoint{1.592940in}{3.113327in}}%
\pgfpathlineto{\pgfqpoint{1.596152in}{3.113650in}}%
\pgfpathlineto{\pgfqpoint{1.599631in}{3.116372in}}%
\pgfpathlineto{\pgfqpoint{1.604983in}{3.123917in}}%
\pgfpathlineto{\pgfqpoint{1.613546in}{3.135585in}}%
\pgfpathlineto{\pgfqpoint{1.617293in}{3.137227in}}%
\pgfpathlineto{\pgfqpoint{1.620504in}{3.136264in}}%
\pgfpathlineto{\pgfqpoint{1.624250in}{3.132620in}}%
\pgfpathlineto{\pgfqpoint{1.631744in}{3.121188in}}%
\pgfpathlineto{\pgfqpoint{1.637363in}{3.114522in}}%
\pgfpathlineto{\pgfqpoint{1.641110in}{3.113192in}}%
\pgfpathlineto{\pgfqpoint{1.644321in}{3.114425in}}%
\pgfpathlineto{\pgfqpoint{1.648335in}{3.118677in}}%
\pgfpathlineto{\pgfqpoint{1.663054in}{3.137057in}}%
\pgfpathlineto{\pgfqpoint{1.666265in}{3.136826in}}%
\pgfpathlineto{\pgfqpoint{1.669744in}{3.134191in}}%
\pgfpathlineto{\pgfqpoint{1.675096in}{3.126716in}}%
\pgfpathlineto{\pgfqpoint{1.683659in}{3.114945in}}%
\pgfpathlineto{\pgfqpoint{1.687406in}{3.113200in}}%
\pgfpathlineto{\pgfqpoint{1.690617in}{3.114072in}}%
\pgfpathlineto{\pgfqpoint{1.694364in}{3.117630in}}%
\pgfpathlineto{\pgfqpoint{1.701322in}{3.128199in}}%
\pgfpathlineto{\pgfqpoint{1.707477in}{3.135796in}}%
\pgfpathlineto{\pgfqpoint{1.711223in}{3.137231in}}%
\pgfpathlineto{\pgfqpoint{1.714434in}{3.136087in}}%
\pgfpathlineto{\pgfqpoint{1.718448in}{3.131922in}}%
\pgfpathlineto{\pgfqpoint{1.733434in}{3.113327in}}%
\pgfpathlineto{\pgfqpoint{1.736646in}{3.113650in}}%
\pgfpathlineto{\pgfqpoint{1.740125in}{3.116372in}}%
\pgfpathlineto{\pgfqpoint{1.745477in}{3.123917in}}%
\pgfpathlineto{\pgfqpoint{1.754040in}{3.135585in}}%
\pgfpathlineto{\pgfqpoint{1.757787in}{3.137227in}}%
\pgfpathlineto{\pgfqpoint{1.760998in}{3.136264in}}%
\pgfpathlineto{\pgfqpoint{1.764745in}{3.132620in}}%
\pgfpathlineto{\pgfqpoint{1.772238in}{3.121188in}}%
\pgfpathlineto{\pgfqpoint{1.777857in}{3.114522in}}%
\pgfpathlineto{\pgfqpoint{1.781604in}{3.113192in}}%
\pgfpathlineto{\pgfqpoint{1.784815in}{3.114425in}}%
\pgfpathlineto{\pgfqpoint{1.788829in}{3.118677in}}%
\pgfpathlineto{\pgfqpoint{1.803548in}{3.137057in}}%
\pgfpathlineto{\pgfqpoint{1.806759in}{3.136826in}}%
\pgfpathlineto{\pgfqpoint{1.810238in}{3.134191in}}%
\pgfpathlineto{\pgfqpoint{1.815590in}{3.126716in}}%
\pgfpathlineto{\pgfqpoint{1.824154in}{3.114945in}}%
\pgfpathlineto{\pgfqpoint{1.827900in}{3.113200in}}%
\pgfpathlineto{\pgfqpoint{1.831111in}{3.114072in}}%
\pgfpathlineto{\pgfqpoint{1.834858in}{3.117630in}}%
\pgfpathlineto{\pgfqpoint{1.841816in}{3.128199in}}%
\pgfpathlineto{\pgfqpoint{1.847971in}{3.135796in}}%
\pgfpathlineto{\pgfqpoint{1.851717in}{3.137231in}}%
\pgfpathlineto{\pgfqpoint{1.854928in}{3.136087in}}%
\pgfpathlineto{\pgfqpoint{1.858943in}{3.131922in}}%
\pgfpathlineto{\pgfqpoint{1.873929in}{3.113327in}}%
\pgfpathlineto{\pgfqpoint{1.877140in}{3.113650in}}%
\pgfpathlineto{\pgfqpoint{1.880619in}{3.116372in}}%
\pgfpathlineto{\pgfqpoint{1.885971in}{3.123917in}}%
\pgfpathlineto{\pgfqpoint{1.894534in}{3.135585in}}%
\pgfpathlineto{\pgfqpoint{1.898281in}{3.137227in}}%
\pgfpathlineto{\pgfqpoint{1.901492in}{3.136264in}}%
\pgfpathlineto{\pgfqpoint{1.905239in}{3.132620in}}%
\pgfpathlineto{\pgfqpoint{1.912732in}{3.121188in}}%
\pgfpathlineto{\pgfqpoint{1.918351in}{3.114522in}}%
\pgfpathlineto{\pgfqpoint{1.922098in}{3.113192in}}%
\pgfpathlineto{\pgfqpoint{1.925309in}{3.114425in}}%
\pgfpathlineto{\pgfqpoint{1.929323in}{3.118677in}}%
\pgfpathlineto{\pgfqpoint{1.944042in}{3.137057in}}%
\pgfpathlineto{\pgfqpoint{1.947253in}{3.136826in}}%
\pgfpathlineto{\pgfqpoint{1.950732in}{3.134191in}}%
\pgfpathlineto{\pgfqpoint{1.956084in}{3.126716in}}%
\pgfpathlineto{\pgfqpoint{1.964648in}{3.114945in}}%
\pgfpathlineto{\pgfqpoint{1.968394in}{3.113200in}}%
\pgfpathlineto{\pgfqpoint{1.971605in}{3.114072in}}%
\pgfpathlineto{\pgfqpoint{1.975352in}{3.117630in}}%
\pgfpathlineto{\pgfqpoint{1.982310in}{3.128199in}}%
\pgfpathlineto{\pgfqpoint{1.988465in}{3.135796in}}%
\pgfpathlineto{\pgfqpoint{1.992211in}{3.137231in}}%
\pgfpathlineto{\pgfqpoint{1.995423in}{3.136087in}}%
\pgfpathlineto{\pgfqpoint{1.999437in}{3.131922in}}%
\pgfpathlineto{\pgfqpoint{2.014423in}{3.113327in}}%
\pgfpathlineto{\pgfqpoint{2.017634in}{3.113650in}}%
\pgfpathlineto{\pgfqpoint{2.021113in}{3.116372in}}%
\pgfpathlineto{\pgfqpoint{2.026465in}{3.123917in}}%
\pgfpathlineto{\pgfqpoint{2.035028in}{3.135585in}}%
\pgfpathlineto{\pgfqpoint{2.038775in}{3.137227in}}%
\pgfpathlineto{\pgfqpoint{2.041986in}{3.136264in}}%
\pgfpathlineto{\pgfqpoint{2.045733in}{3.132620in}}%
\pgfpathlineto{\pgfqpoint{2.053226in}{3.121188in}}%
\pgfpathlineto{\pgfqpoint{2.058846in}{3.114522in}}%
\pgfpathlineto{\pgfqpoint{2.062592in}{3.113192in}}%
\pgfpathlineto{\pgfqpoint{2.065803in}{3.114425in}}%
\pgfpathlineto{\pgfqpoint{2.069818in}{3.118677in}}%
\pgfpathlineto{\pgfqpoint{2.084536in}{3.137057in}}%
\pgfpathlineto{\pgfqpoint{2.087747in}{3.136826in}}%
\pgfpathlineto{\pgfqpoint{2.091226in}{3.134191in}}%
\pgfpathlineto{\pgfqpoint{2.096578in}{3.126716in}}%
\pgfpathlineto{\pgfqpoint{2.105142in}{3.114945in}}%
\pgfpathlineto{\pgfqpoint{2.108888in}{3.113200in}}%
\pgfpathlineto{\pgfqpoint{2.112100in}{3.114072in}}%
\pgfpathlineto{\pgfqpoint{2.115846in}{3.117630in}}%
\pgfpathlineto{\pgfqpoint{2.122804in}{3.128199in}}%
\pgfpathlineto{\pgfqpoint{2.128959in}{3.135796in}}%
\pgfpathlineto{\pgfqpoint{2.132705in}{3.137231in}}%
\pgfpathlineto{\pgfqpoint{2.135917in}{3.136087in}}%
\pgfpathlineto{\pgfqpoint{2.139931in}{3.131922in}}%
\pgfpathlineto{\pgfqpoint{2.154917in}{3.113327in}}%
\pgfpathlineto{\pgfqpoint{2.158128in}{3.113650in}}%
\pgfpathlineto{\pgfqpoint{2.161607in}{3.116372in}}%
\pgfpathlineto{\pgfqpoint{2.166959in}{3.123917in}}%
\pgfpathlineto{\pgfqpoint{2.175523in}{3.135585in}}%
\pgfpathlineto{\pgfqpoint{2.179269in}{3.137227in}}%
\pgfpathlineto{\pgfqpoint{2.182480in}{3.136264in}}%
\pgfpathlineto{\pgfqpoint{2.186227in}{3.132620in}}%
\pgfpathlineto{\pgfqpoint{2.193720in}{3.121188in}}%
\pgfpathlineto{\pgfqpoint{2.199340in}{3.114522in}}%
\pgfpathlineto{\pgfqpoint{2.203086in}{3.113192in}}%
\pgfpathlineto{\pgfqpoint{2.206297in}{3.114425in}}%
\pgfpathlineto{\pgfqpoint{2.210312in}{3.118677in}}%
\pgfpathlineto{\pgfqpoint{2.225030in}{3.137057in}}%
\pgfpathlineto{\pgfqpoint{2.228241in}{3.136826in}}%
\pgfpathlineto{\pgfqpoint{2.231720in}{3.134191in}}%
\pgfpathlineto{\pgfqpoint{2.237072in}{3.126716in}}%
\pgfpathlineto{\pgfqpoint{2.245636in}{3.114945in}}%
\pgfpathlineto{\pgfqpoint{2.249382in}{3.113200in}}%
\pgfpathlineto{\pgfqpoint{2.252594in}{3.114072in}}%
\pgfpathlineto{\pgfqpoint{2.256340in}{3.117630in}}%
\pgfpathlineto{\pgfqpoint{2.263298in}{3.128199in}}%
\pgfpathlineto{\pgfqpoint{2.269453in}{3.135796in}}%
\pgfpathlineto{\pgfqpoint{2.273199in}{3.137231in}}%
\pgfpathlineto{\pgfqpoint{2.276411in}{3.136087in}}%
\pgfpathlineto{\pgfqpoint{2.280425in}{3.131922in}}%
\pgfpathlineto{\pgfqpoint{2.295411in}{3.113327in}}%
\pgfpathlineto{\pgfqpoint{2.298622in}{3.113650in}}%
\pgfpathlineto{\pgfqpoint{2.302101in}{3.116372in}}%
\pgfpathlineto{\pgfqpoint{2.307453in}{3.123917in}}%
\pgfpathlineto{\pgfqpoint{2.316017in}{3.135585in}}%
\pgfpathlineto{\pgfqpoint{2.319763in}{3.137227in}}%
\pgfpathlineto{\pgfqpoint{2.322975in}{3.136264in}}%
\pgfpathlineto{\pgfqpoint{2.326721in}{3.132620in}}%
\pgfpathlineto{\pgfqpoint{2.334214in}{3.121188in}}%
\pgfpathlineto{\pgfqpoint{2.339834in}{3.114522in}}%
\pgfpathlineto{\pgfqpoint{2.343580in}{3.113192in}}%
\pgfpathlineto{\pgfqpoint{2.346792in}{3.114425in}}%
\pgfpathlineto{\pgfqpoint{2.350806in}{3.118677in}}%
\pgfpathlineto{\pgfqpoint{2.365524in}{3.137057in}}%
\pgfpathlineto{\pgfqpoint{2.368735in}{3.136826in}}%
\pgfpathlineto{\pgfqpoint{2.372214in}{3.134191in}}%
\pgfpathlineto{\pgfqpoint{2.377567in}{3.126716in}}%
\pgfpathlineto{\pgfqpoint{2.386130in}{3.114945in}}%
\pgfpathlineto{\pgfqpoint{2.389876in}{3.113200in}}%
\pgfpathlineto{\pgfqpoint{2.393088in}{3.114072in}}%
\pgfpathlineto{\pgfqpoint{2.396834in}{3.117630in}}%
\pgfpathlineto{\pgfqpoint{2.403792in}{3.128199in}}%
\pgfpathlineto{\pgfqpoint{2.409947in}{3.135796in}}%
\pgfpathlineto{\pgfqpoint{2.413694in}{3.137231in}}%
\pgfpathlineto{\pgfqpoint{2.416905in}{3.136087in}}%
\pgfpathlineto{\pgfqpoint{2.420919in}{3.131922in}}%
\pgfpathlineto{\pgfqpoint{2.435905in}{3.113327in}}%
\pgfpathlineto{\pgfqpoint{2.439116in}{3.113650in}}%
\pgfpathlineto{\pgfqpoint{2.442595in}{3.116372in}}%
\pgfpathlineto{\pgfqpoint{2.447947in}{3.123917in}}%
\pgfpathlineto{\pgfqpoint{2.456511in}{3.135585in}}%
\pgfpathlineto{\pgfqpoint{2.460257in}{3.137227in}}%
\pgfpathlineto{\pgfqpoint{2.463469in}{3.136264in}}%
\pgfpathlineto{\pgfqpoint{2.467215in}{3.132620in}}%
\pgfpathlineto{\pgfqpoint{2.474708in}{3.121188in}}%
\pgfpathlineto{\pgfqpoint{2.480328in}{3.114522in}}%
\pgfpathlineto{\pgfqpoint{2.484074in}{3.113192in}}%
\pgfpathlineto{\pgfqpoint{2.487286in}{3.114425in}}%
\pgfpathlineto{\pgfqpoint{2.491300in}{3.118677in}}%
\pgfpathlineto{\pgfqpoint{2.506018in}{3.137057in}}%
\pgfpathlineto{\pgfqpoint{2.509230in}{3.136826in}}%
\pgfpathlineto{\pgfqpoint{2.512708in}{3.134191in}}%
\pgfpathlineto{\pgfqpoint{2.518061in}{3.126716in}}%
\pgfpathlineto{\pgfqpoint{2.526624in}{3.114945in}}%
\pgfpathlineto{\pgfqpoint{2.530371in}{3.113200in}}%
\pgfpathlineto{\pgfqpoint{2.533582in}{3.114072in}}%
\pgfpathlineto{\pgfqpoint{2.537328in}{3.117630in}}%
\pgfpathlineto{\pgfqpoint{2.544286in}{3.128199in}}%
\pgfpathlineto{\pgfqpoint{2.550441in}{3.135796in}}%
\pgfpathlineto{\pgfqpoint{2.554188in}{3.137231in}}%
\pgfpathlineto{\pgfqpoint{2.557399in}{3.136087in}}%
\pgfpathlineto{\pgfqpoint{2.561413in}{3.131922in}}%
\pgfpathlineto{\pgfqpoint{2.576399in}{3.113327in}}%
\pgfpathlineto{\pgfqpoint{2.579610in}{3.113650in}}%
\pgfpathlineto{\pgfqpoint{2.583089in}{3.116372in}}%
\pgfpathlineto{\pgfqpoint{2.588441in}{3.123917in}}%
\pgfpathlineto{\pgfqpoint{2.597005in}{3.135585in}}%
\pgfpathlineto{\pgfqpoint{2.600751in}{3.137227in}}%
\pgfpathlineto{\pgfqpoint{2.603963in}{3.136264in}}%
\pgfpathlineto{\pgfqpoint{2.607709in}{3.132620in}}%
\pgfpathlineto{\pgfqpoint{2.615202in}{3.121188in}}%
\pgfpathlineto{\pgfqpoint{2.620822in}{3.114522in}}%
\pgfpathlineto{\pgfqpoint{2.624569in}{3.113192in}}%
\pgfpathlineto{\pgfqpoint{2.627780in}{3.114425in}}%
\pgfpathlineto{\pgfqpoint{2.631794in}{3.118677in}}%
\pgfpathlineto{\pgfqpoint{2.646512in}{3.137057in}}%
\pgfpathlineto{\pgfqpoint{2.649724in}{3.136826in}}%
\pgfpathlineto{\pgfqpoint{2.653203in}{3.134191in}}%
\pgfpathlineto{\pgfqpoint{2.658555in}{3.126716in}}%
\pgfpathlineto{\pgfqpoint{2.667118in}{3.114945in}}%
\pgfpathlineto{\pgfqpoint{2.670865in}{3.113200in}}%
\pgfpathlineto{\pgfqpoint{2.674076in}{3.114072in}}%
\pgfpathlineto{\pgfqpoint{2.677822in}{3.117630in}}%
\pgfpathlineto{\pgfqpoint{2.684780in}{3.128199in}}%
\pgfpathlineto{\pgfqpoint{2.690935in}{3.135796in}}%
\pgfpathlineto{\pgfqpoint{2.694682in}{3.137231in}}%
\pgfpathlineto{\pgfqpoint{2.697893in}{3.136087in}}%
\pgfpathlineto{\pgfqpoint{2.701907in}{3.131922in}}%
\pgfpathlineto{\pgfqpoint{2.716893in}{3.113327in}}%
\pgfpathlineto{\pgfqpoint{2.720105in}{3.113650in}}%
\pgfpathlineto{\pgfqpoint{2.723583in}{3.116372in}}%
\pgfpathlineto{\pgfqpoint{2.728936in}{3.123917in}}%
\pgfpathlineto{\pgfqpoint{2.737499in}{3.135585in}}%
\pgfpathlineto{\pgfqpoint{2.741246in}{3.137227in}}%
\pgfpathlineto{\pgfqpoint{2.744457in}{3.136264in}}%
\pgfpathlineto{\pgfqpoint{2.748203in}{3.132620in}}%
\pgfpathlineto{\pgfqpoint{2.755696in}{3.121188in}}%
\pgfpathlineto{\pgfqpoint{2.761316in}{3.114522in}}%
\pgfpathlineto{\pgfqpoint{2.765063in}{3.113192in}}%
\pgfpathlineto{\pgfqpoint{2.768274in}{3.114425in}}%
\pgfpathlineto{\pgfqpoint{2.772288in}{3.118677in}}%
\pgfpathlineto{\pgfqpoint{2.787006in}{3.137057in}}%
\pgfpathlineto{\pgfqpoint{2.790218in}{3.136826in}}%
\pgfpathlineto{\pgfqpoint{2.793697in}{3.134191in}}%
\pgfpathlineto{\pgfqpoint{2.799049in}{3.126716in}}%
\pgfpathlineto{\pgfqpoint{2.807612in}{3.114945in}}%
\pgfpathlineto{\pgfqpoint{2.811359in}{3.113200in}}%
\pgfpathlineto{\pgfqpoint{2.814570in}{3.114072in}}%
\pgfpathlineto{\pgfqpoint{2.818317in}{3.117630in}}%
\pgfpathlineto{\pgfqpoint{2.825274in}{3.128199in}}%
\pgfpathlineto{\pgfqpoint{2.831429in}{3.135796in}}%
\pgfpathlineto{\pgfqpoint{2.835176in}{3.137231in}}%
\pgfpathlineto{\pgfqpoint{2.838387in}{3.136087in}}%
\pgfpathlineto{\pgfqpoint{2.842401in}{3.131922in}}%
\pgfpathlineto{\pgfqpoint{2.857387in}{3.113327in}}%
\pgfpathlineto{\pgfqpoint{2.860599in}{3.113650in}}%
\pgfpathlineto{\pgfqpoint{2.864078in}{3.116372in}}%
\pgfpathlineto{\pgfqpoint{2.869430in}{3.123917in}}%
\pgfpathlineto{\pgfqpoint{2.877993in}{3.135585in}}%
\pgfpathlineto{\pgfqpoint{2.881740in}{3.137227in}}%
\pgfpathlineto{\pgfqpoint{2.884951in}{3.136264in}}%
\pgfpathlineto{\pgfqpoint{2.888697in}{3.132620in}}%
\pgfpathlineto{\pgfqpoint{2.896190in}{3.121188in}}%
\pgfpathlineto{\pgfqpoint{2.901810in}{3.114522in}}%
\pgfpathlineto{\pgfqpoint{2.905557in}{3.113192in}}%
\pgfpathlineto{\pgfqpoint{2.908768in}{3.114425in}}%
\pgfpathlineto{\pgfqpoint{2.912782in}{3.118677in}}%
\pgfpathlineto{\pgfqpoint{2.927501in}{3.137057in}}%
\pgfpathlineto{\pgfqpoint{2.930712in}{3.136826in}}%
\pgfpathlineto{\pgfqpoint{2.934191in}{3.134191in}}%
\pgfpathlineto{\pgfqpoint{2.939543in}{3.126716in}}%
\pgfpathlineto{\pgfqpoint{2.948106in}{3.114945in}}%
\pgfpathlineto{\pgfqpoint{2.951853in}{3.113200in}}%
\pgfpathlineto{\pgfqpoint{2.955064in}{3.114072in}}%
\pgfpathlineto{\pgfqpoint{2.958811in}{3.117630in}}%
\pgfpathlineto{\pgfqpoint{2.965768in}{3.128199in}}%
\pgfpathlineto{\pgfqpoint{2.971923in}{3.135796in}}%
\pgfpathlineto{\pgfqpoint{2.975670in}{3.137231in}}%
\pgfpathlineto{\pgfqpoint{2.978881in}{3.136087in}}%
\pgfpathlineto{\pgfqpoint{2.982895in}{3.131922in}}%
\pgfpathlineto{\pgfqpoint{2.997881in}{3.113327in}}%
\pgfpathlineto{\pgfqpoint{3.001093in}{3.113650in}}%
\pgfpathlineto{\pgfqpoint{3.004572in}{3.116372in}}%
\pgfpathlineto{\pgfqpoint{3.009924in}{3.123917in}}%
\pgfpathlineto{\pgfqpoint{3.018487in}{3.135585in}}%
\pgfpathlineto{\pgfqpoint{3.022234in}{3.137227in}}%
\pgfpathlineto{\pgfqpoint{3.025445in}{3.136264in}}%
\pgfpathlineto{\pgfqpoint{3.029192in}{3.132620in}}%
\pgfpathlineto{\pgfqpoint{3.036685in}{3.121188in}}%
\pgfpathlineto{\pgfqpoint{3.042304in}{3.114522in}}%
\pgfpathlineto{\pgfqpoint{3.046051in}{3.113192in}}%
\pgfpathlineto{\pgfqpoint{3.049262in}{3.114425in}}%
\pgfpathlineto{\pgfqpoint{3.053276in}{3.118677in}}%
\pgfpathlineto{\pgfqpoint{3.067995in}{3.137057in}}%
\pgfpathlineto{\pgfqpoint{3.071206in}{3.136826in}}%
\pgfpathlineto{\pgfqpoint{3.074685in}{3.134191in}}%
\pgfpathlineto{\pgfqpoint{3.080037in}{3.126716in}}%
\pgfpathlineto{\pgfqpoint{3.088600in}{3.114945in}}%
\pgfpathlineto{\pgfqpoint{3.092347in}{3.113200in}}%
\pgfpathlineto{\pgfqpoint{3.095558in}{3.114072in}}%
\pgfpathlineto{\pgfqpoint{3.099305in}{3.117630in}}%
\pgfpathlineto{\pgfqpoint{3.106263in}{3.128199in}}%
\pgfpathlineto{\pgfqpoint{3.112418in}{3.135796in}}%
\pgfpathlineto{\pgfqpoint{3.116164in}{3.137231in}}%
\pgfpathlineto{\pgfqpoint{3.119375in}{3.136087in}}%
\pgfpathlineto{\pgfqpoint{3.123389in}{3.131922in}}%
\pgfpathlineto{\pgfqpoint{3.138376in}{3.113327in}}%
\pgfpathlineto{\pgfqpoint{3.141587in}{3.113650in}}%
\pgfpathlineto{\pgfqpoint{3.145066in}{3.116372in}}%
\pgfpathlineto{\pgfqpoint{3.150418in}{3.123917in}}%
\pgfpathlineto{\pgfqpoint{3.158981in}{3.135585in}}%
\pgfpathlineto{\pgfqpoint{3.162728in}{3.137227in}}%
\pgfpathlineto{\pgfqpoint{3.165939in}{3.136264in}}%
\pgfpathlineto{\pgfqpoint{3.169686in}{3.132620in}}%
\pgfpathlineto{\pgfqpoint{3.177179in}{3.121188in}}%
\pgfpathlineto{\pgfqpoint{3.182798in}{3.114522in}}%
\pgfpathlineto{\pgfqpoint{3.186545in}{3.113192in}}%
\pgfpathlineto{\pgfqpoint{3.189756in}{3.114425in}}%
\pgfpathlineto{\pgfqpoint{3.193770in}{3.118677in}}%
\pgfpathlineto{\pgfqpoint{3.208489in}{3.137057in}}%
\pgfpathlineto{\pgfqpoint{3.211700in}{3.136826in}}%
\pgfpathlineto{\pgfqpoint{3.215179in}{3.134191in}}%
\pgfpathlineto{\pgfqpoint{3.220531in}{3.126716in}}%
\pgfpathlineto{\pgfqpoint{3.229095in}{3.114945in}}%
\pgfpathlineto{\pgfqpoint{3.232841in}{3.113200in}}%
\pgfpathlineto{\pgfqpoint{3.236052in}{3.114072in}}%
\pgfpathlineto{\pgfqpoint{3.239799in}{3.117630in}}%
\pgfpathlineto{\pgfqpoint{3.246757in}{3.128199in}}%
\pgfpathlineto{\pgfqpoint{3.252912in}{3.135796in}}%
\pgfpathlineto{\pgfqpoint{3.256658in}{3.137231in}}%
\pgfpathlineto{\pgfqpoint{3.259869in}{3.136087in}}%
\pgfpathlineto{\pgfqpoint{3.263884in}{3.131922in}}%
\pgfpathlineto{\pgfqpoint{3.278870in}{3.113327in}}%
\pgfpathlineto{\pgfqpoint{3.282081in}{3.113650in}}%
\pgfpathlineto{\pgfqpoint{3.285560in}{3.116372in}}%
\pgfpathlineto{\pgfqpoint{3.290912in}{3.123917in}}%
\pgfpathlineto{\pgfqpoint{3.299475in}{3.135585in}}%
\pgfpathlineto{\pgfqpoint{3.303222in}{3.137227in}}%
\pgfpathlineto{\pgfqpoint{3.306433in}{3.136264in}}%
\pgfpathlineto{\pgfqpoint{3.310180in}{3.132620in}}%
\pgfpathlineto{\pgfqpoint{3.317673in}{3.121188in}}%
\pgfpathlineto{\pgfqpoint{3.323293in}{3.114522in}}%
\pgfpathlineto{\pgfqpoint{3.327039in}{3.113192in}}%
\pgfpathlineto{\pgfqpoint{3.330250in}{3.114425in}}%
\pgfpathlineto{\pgfqpoint{3.334264in}{3.118677in}}%
\pgfpathlineto{\pgfqpoint{3.348983in}{3.137057in}}%
\pgfpathlineto{\pgfqpoint{3.352194in}{3.136826in}}%
\pgfpathlineto{\pgfqpoint{3.355673in}{3.134191in}}%
\pgfpathlineto{\pgfqpoint{3.361025in}{3.126716in}}%
\pgfpathlineto{\pgfqpoint{3.369589in}{3.114945in}}%
\pgfpathlineto{\pgfqpoint{3.373335in}{3.113200in}}%
\pgfpathlineto{\pgfqpoint{3.376546in}{3.114072in}}%
\pgfpathlineto{\pgfqpoint{3.380293in}{3.117630in}}%
\pgfpathlineto{\pgfqpoint{3.387251in}{3.128199in}}%
\pgfpathlineto{\pgfqpoint{3.393406in}{3.135796in}}%
\pgfpathlineto{\pgfqpoint{3.397152in}{3.137231in}}%
\pgfpathlineto{\pgfqpoint{3.400364in}{3.136087in}}%
\pgfpathlineto{\pgfqpoint{3.404378in}{3.131922in}}%
\pgfpathlineto{\pgfqpoint{3.419364in}{3.113327in}}%
\pgfpathlineto{\pgfqpoint{3.422575in}{3.113650in}}%
\pgfpathlineto{\pgfqpoint{3.426054in}{3.116372in}}%
\pgfpathlineto{\pgfqpoint{3.431406in}{3.123917in}}%
\pgfpathlineto{\pgfqpoint{3.439970in}{3.135585in}}%
\pgfpathlineto{\pgfqpoint{3.443716in}{3.137227in}}%
\pgfpathlineto{\pgfqpoint{3.446927in}{3.136264in}}%
\pgfpathlineto{\pgfqpoint{3.450674in}{3.132620in}}%
\pgfpathlineto{\pgfqpoint{3.458167in}{3.121188in}}%
\pgfpathlineto{\pgfqpoint{3.463787in}{3.114522in}}%
\pgfpathlineto{\pgfqpoint{3.467533in}{3.113192in}}%
\pgfpathlineto{\pgfqpoint{3.470744in}{3.114425in}}%
\pgfpathlineto{\pgfqpoint{3.474759in}{3.118677in}}%
\pgfpathlineto{\pgfqpoint{3.489477in}{3.137057in}}%
\pgfpathlineto{\pgfqpoint{3.492688in}{3.136826in}}%
\pgfpathlineto{\pgfqpoint{3.496167in}{3.134191in}}%
\pgfpathlineto{\pgfqpoint{3.501519in}{3.126716in}}%
\pgfpathlineto{\pgfqpoint{3.510083in}{3.114945in}}%
\pgfpathlineto{\pgfqpoint{3.513829in}{3.113200in}}%
\pgfpathlineto{\pgfqpoint{3.517041in}{3.114072in}}%
\pgfpathlineto{\pgfqpoint{3.520787in}{3.117630in}}%
\pgfpathlineto{\pgfqpoint{3.527745in}{3.128199in}}%
\pgfpathlineto{\pgfqpoint{3.533900in}{3.135796in}}%
\pgfpathlineto{\pgfqpoint{3.537646in}{3.137231in}}%
\pgfpathlineto{\pgfqpoint{3.540858in}{3.136087in}}%
\pgfpathlineto{\pgfqpoint{3.544872in}{3.131922in}}%
\pgfpathlineto{\pgfqpoint{3.559858in}{3.113327in}}%
\pgfpathlineto{\pgfqpoint{3.563069in}{3.113650in}}%
\pgfpathlineto{\pgfqpoint{3.566548in}{3.116372in}}%
\pgfpathlineto{\pgfqpoint{3.571900in}{3.123917in}}%
\pgfpathlineto{\pgfqpoint{3.580464in}{3.135585in}}%
\pgfpathlineto{\pgfqpoint{3.584210in}{3.137227in}}%
\pgfpathlineto{\pgfqpoint{3.587421in}{3.136264in}}%
\pgfpathlineto{\pgfqpoint{3.591168in}{3.132620in}}%
\pgfpathlineto{\pgfqpoint{3.598661in}{3.121188in}}%
\pgfpathlineto{\pgfqpoint{3.604281in}{3.114522in}}%
\pgfpathlineto{\pgfqpoint{3.608027in}{3.113192in}}%
\pgfpathlineto{\pgfqpoint{3.611239in}{3.114425in}}%
\pgfpathlineto{\pgfqpoint{3.615253in}{3.118677in}}%
\pgfpathlineto{\pgfqpoint{3.629971in}{3.137057in}}%
\pgfpathlineto{\pgfqpoint{3.633182in}{3.136826in}}%
\pgfpathlineto{\pgfqpoint{3.636661in}{3.134191in}}%
\pgfpathlineto{\pgfqpoint{3.642013in}{3.126716in}}%
\pgfpathlineto{\pgfqpoint{3.650577in}{3.114945in}}%
\pgfpathlineto{\pgfqpoint{3.654323in}{3.113200in}}%
\pgfpathlineto{\pgfqpoint{3.657535in}{3.114072in}}%
\pgfpathlineto{\pgfqpoint{3.661281in}{3.117630in}}%
\pgfpathlineto{\pgfqpoint{3.668239in}{3.128199in}}%
\pgfpathlineto{\pgfqpoint{3.674394in}{3.135796in}}%
\pgfpathlineto{\pgfqpoint{3.678141in}{3.137231in}}%
\pgfpathlineto{\pgfqpoint{3.681352in}{3.136087in}}%
\pgfpathlineto{\pgfqpoint{3.685366in}{3.131922in}}%
\pgfpathlineto{\pgfqpoint{3.700352in}{3.113327in}}%
\pgfpathlineto{\pgfqpoint{3.703563in}{3.113650in}}%
\pgfpathlineto{\pgfqpoint{3.707042in}{3.116372in}}%
\pgfpathlineto{\pgfqpoint{3.712394in}{3.123917in}}%
\pgfpathlineto{\pgfqpoint{3.720958in}{3.135585in}}%
\pgfpathlineto{\pgfqpoint{3.724704in}{3.137227in}}%
\pgfpathlineto{\pgfqpoint{3.727916in}{3.136264in}}%
\pgfpathlineto{\pgfqpoint{3.731662in}{3.132620in}}%
\pgfpathlineto{\pgfqpoint{3.739155in}{3.121188in}}%
\pgfpathlineto{\pgfqpoint{3.744775in}{3.114522in}}%
\pgfpathlineto{\pgfqpoint{3.748521in}{3.113192in}}%
\pgfpathlineto{\pgfqpoint{3.751733in}{3.114425in}}%
\pgfpathlineto{\pgfqpoint{3.755747in}{3.118677in}}%
\pgfpathlineto{\pgfqpoint{3.770465in}{3.137057in}}%
\pgfpathlineto{\pgfqpoint{3.773677in}{3.136826in}}%
\pgfpathlineto{\pgfqpoint{3.777155in}{3.134191in}}%
\pgfpathlineto{\pgfqpoint{3.782508in}{3.126716in}}%
\pgfpathlineto{\pgfqpoint{3.791071in}{3.114945in}}%
\pgfpathlineto{\pgfqpoint{3.794818in}{3.113200in}}%
\pgfpathlineto{\pgfqpoint{3.798029in}{3.114072in}}%
\pgfpathlineto{\pgfqpoint{3.801775in}{3.117630in}}%
\pgfpathlineto{\pgfqpoint{3.808733in}{3.128199in}}%
\pgfpathlineto{\pgfqpoint{3.814888in}{3.135796in}}%
\pgfpathlineto{\pgfqpoint{3.818635in}{3.137231in}}%
\pgfpathlineto{\pgfqpoint{3.821846in}{3.136087in}}%
\pgfpathlineto{\pgfqpoint{3.825860in}{3.131922in}}%
\pgfpathlineto{\pgfqpoint{3.840846in}{3.113327in}}%
\pgfpathlineto{\pgfqpoint{3.844057in}{3.113650in}}%
\pgfpathlineto{\pgfqpoint{3.847536in}{3.116372in}}%
\pgfpathlineto{\pgfqpoint{3.852888in}{3.123917in}}%
\pgfpathlineto{\pgfqpoint{3.861452in}{3.135585in}}%
\pgfpathlineto{\pgfqpoint{3.865198in}{3.137227in}}%
\pgfpathlineto{\pgfqpoint{3.868410in}{3.136264in}}%
\pgfpathlineto{\pgfqpoint{3.872156in}{3.132620in}}%
\pgfpathlineto{\pgfqpoint{3.879649in}{3.121188in}}%
\pgfpathlineto{\pgfqpoint{3.885269in}{3.114522in}}%
\pgfpathlineto{\pgfqpoint{3.889015in}{3.113192in}}%
\pgfpathlineto{\pgfqpoint{3.892227in}{3.114425in}}%
\pgfpathlineto{\pgfqpoint{3.896241in}{3.118677in}}%
\pgfpathlineto{\pgfqpoint{3.910959in}{3.137057in}}%
\pgfpathlineto{\pgfqpoint{3.914171in}{3.136826in}}%
\pgfpathlineto{\pgfqpoint{3.917650in}{3.134191in}}%
\pgfpathlineto{\pgfqpoint{3.923002in}{3.126716in}}%
\pgfpathlineto{\pgfqpoint{3.931565in}{3.114945in}}%
\pgfpathlineto{\pgfqpoint{3.935312in}{3.113200in}}%
\pgfpathlineto{\pgfqpoint{3.938523in}{3.114072in}}%
\pgfpathlineto{\pgfqpoint{3.942269in}{3.117630in}}%
\pgfpathlineto{\pgfqpoint{3.949227in}{3.128199in}}%
\pgfpathlineto{\pgfqpoint{3.955382in}{3.135796in}}%
\pgfpathlineto{\pgfqpoint{3.959129in}{3.137231in}}%
\pgfpathlineto{\pgfqpoint{3.962340in}{3.136087in}}%
\pgfpathlineto{\pgfqpoint{3.966354in}{3.131922in}}%
\pgfpathlineto{\pgfqpoint{3.981340in}{3.113327in}}%
\pgfpathlineto{\pgfqpoint{3.984551in}{3.113650in}}%
\pgfpathlineto{\pgfqpoint{3.988030in}{3.116372in}}%
\pgfpathlineto{\pgfqpoint{3.993383in}{3.123917in}}%
\pgfpathlineto{\pgfqpoint{4.001946in}{3.135585in}}%
\pgfpathlineto{\pgfqpoint{4.005692in}{3.137227in}}%
\pgfpathlineto{\pgfqpoint{4.008904in}{3.136264in}}%
\pgfpathlineto{\pgfqpoint{4.012650in}{3.132620in}}%
\pgfpathlineto{\pgfqpoint{4.020143in}{3.121188in}}%
\pgfpathlineto{\pgfqpoint{4.025763in}{3.114522in}}%
\pgfpathlineto{\pgfqpoint{4.029510in}{3.113192in}}%
\pgfpathlineto{\pgfqpoint{4.032721in}{3.114425in}}%
\pgfpathlineto{\pgfqpoint{4.036735in}{3.118677in}}%
\pgfpathlineto{\pgfqpoint{4.051453in}{3.137057in}}%
\pgfpathlineto{\pgfqpoint{4.054665in}{3.136826in}}%
\pgfpathlineto{\pgfqpoint{4.058144in}{3.134191in}}%
\pgfpathlineto{\pgfqpoint{4.063496in}{3.126716in}}%
\pgfpathlineto{\pgfqpoint{4.072059in}{3.114945in}}%
\pgfpathlineto{\pgfqpoint{4.075806in}{3.113200in}}%
\pgfpathlineto{\pgfqpoint{4.079017in}{3.114072in}}%
\pgfpathlineto{\pgfqpoint{4.082764in}{3.117630in}}%
\pgfpathlineto{\pgfqpoint{4.089721in}{3.128199in}}%
\pgfpathlineto{\pgfqpoint{4.095876in}{3.135796in}}%
\pgfpathlineto{\pgfqpoint{4.099623in}{3.137231in}}%
\pgfpathlineto{\pgfqpoint{4.102834in}{3.136087in}}%
\pgfpathlineto{\pgfqpoint{4.106848in}{3.131922in}}%
\pgfpathlineto{\pgfqpoint{4.121834in}{3.113327in}}%
\pgfpathlineto{\pgfqpoint{4.125046in}{3.113650in}}%
\pgfpathlineto{\pgfqpoint{4.128524in}{3.116372in}}%
\pgfpathlineto{\pgfqpoint{4.133877in}{3.123917in}}%
\pgfpathlineto{\pgfqpoint{4.142440in}{3.135585in}}%
\pgfpathlineto{\pgfqpoint{4.146187in}{3.137227in}}%
\pgfpathlineto{\pgfqpoint{4.149398in}{3.136264in}}%
\pgfpathlineto{\pgfqpoint{4.153144in}{3.132620in}}%
\pgfpathlineto{\pgfqpoint{4.160637in}{3.121188in}}%
\pgfpathlineto{\pgfqpoint{4.166257in}{3.114522in}}%
\pgfpathlineto{\pgfqpoint{4.170004in}{3.113192in}}%
\pgfpathlineto{\pgfqpoint{4.173215in}{3.114425in}}%
\pgfpathlineto{\pgfqpoint{4.177229in}{3.118677in}}%
\pgfpathlineto{\pgfqpoint{4.191948in}{3.137057in}}%
\pgfpathlineto{\pgfqpoint{4.195159in}{3.136826in}}%
\pgfpathlineto{\pgfqpoint{4.198638in}{3.134191in}}%
\pgfpathlineto{\pgfqpoint{4.203990in}{3.126716in}}%
\pgfpathlineto{\pgfqpoint{4.212553in}{3.114945in}}%
\pgfpathlineto{\pgfqpoint{4.216300in}{3.113200in}}%
\pgfpathlineto{\pgfqpoint{4.219511in}{3.114072in}}%
\pgfpathlineto{\pgfqpoint{4.223258in}{3.117630in}}%
\pgfpathlineto{\pgfqpoint{4.230215in}{3.128199in}}%
\pgfpathlineto{\pgfqpoint{4.236370in}{3.135796in}}%
\pgfpathlineto{\pgfqpoint{4.240117in}{3.137231in}}%
\pgfpathlineto{\pgfqpoint{4.243328in}{3.136087in}}%
\pgfpathlineto{\pgfqpoint{4.247342in}{3.131922in}}%
\pgfpathlineto{\pgfqpoint{4.262328in}{3.113327in}}%
\pgfpathlineto{\pgfqpoint{4.265540in}{3.113650in}}%
\pgfpathlineto{\pgfqpoint{4.269019in}{3.116372in}}%
\pgfpathlineto{\pgfqpoint{4.274371in}{3.123917in}}%
\pgfpathlineto{\pgfqpoint{4.282934in}{3.135585in}}%
\pgfpathlineto{\pgfqpoint{4.286681in}{3.137227in}}%
\pgfpathlineto{\pgfqpoint{4.289892in}{3.136264in}}%
\pgfpathlineto{\pgfqpoint{4.293639in}{3.132620in}}%
\pgfpathlineto{\pgfqpoint{4.301132in}{3.121188in}}%
\pgfpathlineto{\pgfqpoint{4.306751in}{3.114522in}}%
\pgfpathlineto{\pgfqpoint{4.310498in}{3.113192in}}%
\pgfpathlineto{\pgfqpoint{4.313709in}{3.114425in}}%
\pgfpathlineto{\pgfqpoint{4.317723in}{3.118677in}}%
\pgfpathlineto{\pgfqpoint{4.332442in}{3.137057in}}%
\pgfpathlineto{\pgfqpoint{4.335653in}{3.136826in}}%
\pgfpathlineto{\pgfqpoint{4.339132in}{3.134191in}}%
\pgfpathlineto{\pgfqpoint{4.344484in}{3.126716in}}%
\pgfpathlineto{\pgfqpoint{4.353047in}{3.114945in}}%
\pgfpathlineto{\pgfqpoint{4.356794in}{3.113200in}}%
\pgfpathlineto{\pgfqpoint{4.360005in}{3.114072in}}%
\pgfpathlineto{\pgfqpoint{4.363752in}{3.117630in}}%
\pgfpathlineto{\pgfqpoint{4.370710in}{3.128199in}}%
\pgfpathlineto{\pgfqpoint{4.376865in}{3.135796in}}%
\pgfpathlineto{\pgfqpoint{4.380611in}{3.137231in}}%
\pgfpathlineto{\pgfqpoint{4.383822in}{3.136087in}}%
\pgfpathlineto{\pgfqpoint{4.387836in}{3.131922in}}%
\pgfpathlineto{\pgfqpoint{4.402822in}{3.113327in}}%
\pgfpathlineto{\pgfqpoint{4.406034in}{3.113650in}}%
\pgfpathlineto{\pgfqpoint{4.409513in}{3.116372in}}%
\pgfpathlineto{\pgfqpoint{4.414865in}{3.123917in}}%
\pgfpathlineto{\pgfqpoint{4.423428in}{3.135585in}}%
\pgfpathlineto{\pgfqpoint{4.427175in}{3.137227in}}%
\pgfpathlineto{\pgfqpoint{4.430386in}{3.136264in}}%
\pgfpathlineto{\pgfqpoint{4.434133in}{3.132620in}}%
\pgfpathlineto{\pgfqpoint{4.441626in}{3.121188in}}%
\pgfpathlineto{\pgfqpoint{4.447245in}{3.114522in}}%
\pgfpathlineto{\pgfqpoint{4.450992in}{3.113192in}}%
\pgfpathlineto{\pgfqpoint{4.454203in}{3.114425in}}%
\pgfpathlineto{\pgfqpoint{4.458217in}{3.118677in}}%
\pgfpathlineto{\pgfqpoint{4.472936in}{3.137057in}}%
\pgfpathlineto{\pgfqpoint{4.476147in}{3.136826in}}%
\pgfpathlineto{\pgfqpoint{4.479626in}{3.134191in}}%
\pgfpathlineto{\pgfqpoint{4.484978in}{3.126716in}}%
\pgfpathlineto{\pgfqpoint{4.493542in}{3.114945in}}%
\pgfpathlineto{\pgfqpoint{4.497288in}{3.113200in}}%
\pgfpathlineto{\pgfqpoint{4.500499in}{3.114072in}}%
\pgfpathlineto{\pgfqpoint{4.504246in}{3.117630in}}%
\pgfpathlineto{\pgfqpoint{4.511204in}{3.128199in}}%
\pgfpathlineto{\pgfqpoint{4.517359in}{3.135796in}}%
\pgfpathlineto{\pgfqpoint{4.521105in}{3.137231in}}%
\pgfpathlineto{\pgfqpoint{4.524316in}{3.136087in}}%
\pgfpathlineto{\pgfqpoint{4.528331in}{3.131922in}}%
\pgfpathlineto{\pgfqpoint{4.543317in}{3.113327in}}%
\pgfpathlineto{\pgfqpoint{4.546528in}{3.113650in}}%
\pgfpathlineto{\pgfqpoint{4.550007in}{3.116372in}}%
\pgfpathlineto{\pgfqpoint{4.555359in}{3.123917in}}%
\pgfpathlineto{\pgfqpoint{4.563922in}{3.135585in}}%
\pgfpathlineto{\pgfqpoint{4.567669in}{3.137227in}}%
\pgfpathlineto{\pgfqpoint{4.570880in}{3.136264in}}%
\pgfpathlineto{\pgfqpoint{4.574627in}{3.132620in}}%
\pgfpathlineto{\pgfqpoint{4.582120in}{3.121188in}}%
\pgfpathlineto{\pgfqpoint{4.587739in}{3.114522in}}%
\pgfpathlineto{\pgfqpoint{4.591486in}{3.113192in}}%
\pgfpathlineto{\pgfqpoint{4.594697in}{3.114425in}}%
\pgfpathlineto{\pgfqpoint{4.598711in}{3.118677in}}%
\pgfpathlineto{\pgfqpoint{4.613430in}{3.137057in}}%
\pgfpathlineto{\pgfqpoint{4.616641in}{3.136826in}}%
\pgfpathlineto{\pgfqpoint{4.620120in}{3.134191in}}%
\pgfpathlineto{\pgfqpoint{4.625472in}{3.126716in}}%
\pgfpathlineto{\pgfqpoint{4.634036in}{3.114945in}}%
\pgfpathlineto{\pgfqpoint{4.637782in}{3.113200in}}%
\pgfpathlineto{\pgfqpoint{4.640993in}{3.114072in}}%
\pgfpathlineto{\pgfqpoint{4.644740in}{3.117630in}}%
\pgfpathlineto{\pgfqpoint{4.651698in}{3.128199in}}%
\pgfpathlineto{\pgfqpoint{4.657853in}{3.135796in}}%
\pgfpathlineto{\pgfqpoint{4.661599in}{3.137231in}}%
\pgfpathlineto{\pgfqpoint{4.664811in}{3.136087in}}%
\pgfpathlineto{\pgfqpoint{4.668825in}{3.131922in}}%
\pgfpathlineto{\pgfqpoint{4.683811in}{3.113327in}}%
\pgfpathlineto{\pgfqpoint{4.687022in}{3.113650in}}%
\pgfpathlineto{\pgfqpoint{4.690501in}{3.116372in}}%
\pgfpathlineto{\pgfqpoint{4.695853in}{3.123917in}}%
\pgfpathlineto{\pgfqpoint{4.704417in}{3.135585in}}%
\pgfpathlineto{\pgfqpoint{4.708163in}{3.137227in}}%
\pgfpathlineto{\pgfqpoint{4.711374in}{3.136264in}}%
\pgfpathlineto{\pgfqpoint{4.715121in}{3.132620in}}%
\pgfpathlineto{\pgfqpoint{4.722614in}{3.121188in}}%
\pgfpathlineto{\pgfqpoint{4.728234in}{3.114522in}}%
\pgfpathlineto{\pgfqpoint{4.731980in}{3.113192in}}%
\pgfpathlineto{\pgfqpoint{4.735191in}{3.114425in}}%
\pgfpathlineto{\pgfqpoint{4.739206in}{3.118677in}}%
\pgfpathlineto{\pgfqpoint{4.753924in}{3.137057in}}%
\pgfpathlineto{\pgfqpoint{4.757135in}{3.136826in}}%
\pgfpathlineto{\pgfqpoint{4.760614in}{3.134191in}}%
\pgfpathlineto{\pgfqpoint{4.765966in}{3.126716in}}%
\pgfpathlineto{\pgfqpoint{4.774530in}{3.114945in}}%
\pgfpathlineto{\pgfqpoint{4.778276in}{3.113200in}}%
\pgfpathlineto{\pgfqpoint{4.781488in}{3.114072in}}%
\pgfpathlineto{\pgfqpoint{4.785234in}{3.117630in}}%
\pgfpathlineto{\pgfqpoint{4.792192in}{3.128199in}}%
\pgfpathlineto{\pgfqpoint{4.798347in}{3.135796in}}%
\pgfpathlineto{\pgfqpoint{4.802093in}{3.137231in}}%
\pgfpathlineto{\pgfqpoint{4.805305in}{3.136087in}}%
\pgfpathlineto{\pgfqpoint{4.809319in}{3.131922in}}%
\pgfpathlineto{\pgfqpoint{4.824305in}{3.113327in}}%
\pgfpathlineto{\pgfqpoint{4.827516in}{3.113650in}}%
\pgfpathlineto{\pgfqpoint{4.830995in}{3.116372in}}%
\pgfpathlineto{\pgfqpoint{4.836347in}{3.123917in}}%
\pgfpathlineto{\pgfqpoint{4.844911in}{3.135585in}}%
\pgfpathlineto{\pgfqpoint{4.848657in}{3.137227in}}%
\pgfpathlineto{\pgfqpoint{4.851868in}{3.136264in}}%
\pgfpathlineto{\pgfqpoint{4.855615in}{3.132620in}}%
\pgfpathlineto{\pgfqpoint{4.863108in}{3.121188in}}%
\pgfpathlineto{\pgfqpoint{4.868728in}{3.114522in}}%
\pgfpathlineto{\pgfqpoint{4.872474in}{3.113192in}}%
\pgfpathlineto{\pgfqpoint{4.875686in}{3.114425in}}%
\pgfpathlineto{\pgfqpoint{4.879700in}{3.118677in}}%
\pgfpathlineto{\pgfqpoint{4.894418in}{3.137057in}}%
\pgfpathlineto{\pgfqpoint{4.897629in}{3.136826in}}%
\pgfpathlineto{\pgfqpoint{4.901108in}{3.134191in}}%
\pgfpathlineto{\pgfqpoint{4.906460in}{3.126716in}}%
\pgfpathlineto{\pgfqpoint{4.915024in}{3.114945in}}%
\pgfpathlineto{\pgfqpoint{4.918770in}{3.113200in}}%
\pgfpathlineto{\pgfqpoint{4.921982in}{3.114072in}}%
\pgfpathlineto{\pgfqpoint{4.925728in}{3.117630in}}%
\pgfpathlineto{\pgfqpoint{4.932686in}{3.128199in}}%
\pgfpathlineto{\pgfqpoint{4.938841in}{3.135796in}}%
\pgfpathlineto{\pgfqpoint{4.942587in}{3.137231in}}%
\pgfpathlineto{\pgfqpoint{4.945799in}{3.136087in}}%
\pgfpathlineto{\pgfqpoint{4.949813in}{3.131922in}}%
\pgfpathlineto{\pgfqpoint{4.964799in}{3.113327in}}%
\pgfpathlineto{\pgfqpoint{4.968010in}{3.113650in}}%
\pgfpathlineto{\pgfqpoint{4.971489in}{3.116372in}}%
\pgfpathlineto{\pgfqpoint{4.976841in}{3.123917in}}%
\pgfpathlineto{\pgfqpoint{4.985405in}{3.135585in}}%
\pgfpathlineto{\pgfqpoint{4.989151in}{3.137227in}}%
\pgfpathlineto{\pgfqpoint{4.992363in}{3.136264in}}%
\pgfpathlineto{\pgfqpoint{4.996109in}{3.132620in}}%
\pgfpathlineto{\pgfqpoint{5.003602in}{3.121188in}}%
\pgfpathlineto{\pgfqpoint{5.009222in}{3.114522in}}%
\pgfpathlineto{\pgfqpoint{5.012968in}{3.113192in}}%
\pgfpathlineto{\pgfqpoint{5.016180in}{3.114425in}}%
\pgfpathlineto{\pgfqpoint{5.020194in}{3.118677in}}%
\pgfpathlineto{\pgfqpoint{5.034912in}{3.137057in}}%
\pgfpathlineto{\pgfqpoint{5.038123in}{3.136826in}}%
\pgfpathlineto{\pgfqpoint{5.041602in}{3.134191in}}%
\pgfpathlineto{\pgfqpoint{5.046955in}{3.126716in}}%
\pgfpathlineto{\pgfqpoint{5.055518in}{3.114945in}}%
\pgfpathlineto{\pgfqpoint{5.059264in}{3.113200in}}%
\pgfpathlineto{\pgfqpoint{5.062476in}{3.114072in}}%
\pgfpathlineto{\pgfqpoint{5.066222in}{3.117630in}}%
\pgfpathlineto{\pgfqpoint{5.073180in}{3.128199in}}%
\pgfpathlineto{\pgfqpoint{5.079335in}{3.135796in}}%
\pgfpathlineto{\pgfqpoint{5.083082in}{3.137231in}}%
\pgfpathlineto{\pgfqpoint{5.086293in}{3.136087in}}%
\pgfpathlineto{\pgfqpoint{5.090307in}{3.131922in}}%
\pgfpathlineto{\pgfqpoint{5.105293in}{3.113327in}}%
\pgfpathlineto{\pgfqpoint{5.108504in}{3.113650in}}%
\pgfpathlineto{\pgfqpoint{5.111983in}{3.116372in}}%
\pgfpathlineto{\pgfqpoint{5.117335in}{3.123917in}}%
\pgfpathlineto{\pgfqpoint{5.125899in}{3.135585in}}%
\pgfpathlineto{\pgfqpoint{5.129645in}{3.137227in}}%
\pgfpathlineto{\pgfqpoint{5.132857in}{3.136264in}}%
\pgfpathlineto{\pgfqpoint{5.136603in}{3.132620in}}%
\pgfpathlineto{\pgfqpoint{5.144096in}{3.121188in}}%
\pgfpathlineto{\pgfqpoint{5.149716in}{3.114522in}}%
\pgfpathlineto{\pgfqpoint{5.153462in}{3.113192in}}%
\pgfpathlineto{\pgfqpoint{5.156674in}{3.114425in}}%
\pgfpathlineto{\pgfqpoint{5.160688in}{3.118677in}}%
\pgfpathlineto{\pgfqpoint{5.175406in}{3.137057in}}%
\pgfpathlineto{\pgfqpoint{5.178618in}{3.136826in}}%
\pgfpathlineto{\pgfqpoint{5.182096in}{3.134191in}}%
\pgfpathlineto{\pgfqpoint{5.187449in}{3.126716in}}%
\pgfpathlineto{\pgfqpoint{5.196012in}{3.114945in}}%
\pgfpathlineto{\pgfqpoint{5.199759in}{3.113200in}}%
\pgfpathlineto{\pgfqpoint{5.202970in}{3.114072in}}%
\pgfpathlineto{\pgfqpoint{5.206716in}{3.117630in}}%
\pgfpathlineto{\pgfqpoint{5.213674in}{3.128199in}}%
\pgfpathlineto{\pgfqpoint{5.219829in}{3.135796in}}%
\pgfpathlineto{\pgfqpoint{5.223576in}{3.137231in}}%
\pgfpathlineto{\pgfqpoint{5.226787in}{3.136087in}}%
\pgfpathlineto{\pgfqpoint{5.230801in}{3.131922in}}%
\pgfpathlineto{\pgfqpoint{5.245787in}{3.113327in}}%
\pgfpathlineto{\pgfqpoint{5.248998in}{3.113650in}}%
\pgfpathlineto{\pgfqpoint{5.252477in}{3.116372in}}%
\pgfpathlineto{\pgfqpoint{5.257829in}{3.123917in}}%
\pgfpathlineto{\pgfqpoint{5.266393in}{3.135585in}}%
\pgfpathlineto{\pgfqpoint{5.270139in}{3.137227in}}%
\pgfpathlineto{\pgfqpoint{5.273351in}{3.136264in}}%
\pgfpathlineto{\pgfqpoint{5.277097in}{3.132620in}}%
\pgfpathlineto{\pgfqpoint{5.284590in}{3.121188in}}%
\pgfpathlineto{\pgfqpoint{5.290210in}{3.114522in}}%
\pgfpathlineto{\pgfqpoint{5.293957in}{3.113192in}}%
\pgfpathlineto{\pgfqpoint{5.297168in}{3.114425in}}%
\pgfpathlineto{\pgfqpoint{5.301182in}{3.118677in}}%
\pgfpathlineto{\pgfqpoint{5.315900in}{3.137057in}}%
\pgfpathlineto{\pgfqpoint{5.319112in}{3.136826in}}%
\pgfpathlineto{\pgfqpoint{5.322591in}{3.134191in}}%
\pgfpathlineto{\pgfqpoint{5.327943in}{3.126716in}}%
\pgfpathlineto{\pgfqpoint{5.336506in}{3.114945in}}%
\pgfpathlineto{\pgfqpoint{5.340253in}{3.113200in}}%
\pgfpathlineto{\pgfqpoint{5.343464in}{3.114072in}}%
\pgfpathlineto{\pgfqpoint{5.347210in}{3.117630in}}%
\pgfpathlineto{\pgfqpoint{5.354168in}{3.128199in}}%
\pgfpathlineto{\pgfqpoint{5.360323in}{3.135796in}}%
\pgfpathlineto{\pgfqpoint{5.364070in}{3.137231in}}%
\pgfpathlineto{\pgfqpoint{5.367281in}{3.136087in}}%
\pgfpathlineto{\pgfqpoint{5.371295in}{3.131922in}}%
\pgfpathlineto{\pgfqpoint{5.386281in}{3.113327in}}%
\pgfpathlineto{\pgfqpoint{5.389493in}{3.113650in}}%
\pgfpathlineto{\pgfqpoint{5.392971in}{3.116372in}}%
\pgfpathlineto{\pgfqpoint{5.398324in}{3.123917in}}%
\pgfpathlineto{\pgfqpoint{5.406887in}{3.135585in}}%
\pgfpathlineto{\pgfqpoint{5.410634in}{3.137227in}}%
\pgfpathlineto{\pgfqpoint{5.413845in}{3.136264in}}%
\pgfpathlineto{\pgfqpoint{5.417591in}{3.132620in}}%
\pgfpathlineto{\pgfqpoint{5.425084in}{3.121188in}}%
\pgfpathlineto{\pgfqpoint{5.430704in}{3.114522in}}%
\pgfpathlineto{\pgfqpoint{5.434451in}{3.113192in}}%
\pgfpathlineto{\pgfqpoint{5.437662in}{3.114425in}}%
\pgfpathlineto{\pgfqpoint{5.441676in}{3.118677in}}%
\pgfpathlineto{\pgfqpoint{5.456394in}{3.137057in}}%
\pgfpathlineto{\pgfqpoint{5.459606in}{3.136826in}}%
\pgfpathlineto{\pgfqpoint{5.463085in}{3.134191in}}%
\pgfpathlineto{\pgfqpoint{5.468437in}{3.126716in}}%
\pgfpathlineto{\pgfqpoint{5.477000in}{3.114945in}}%
\pgfpathlineto{\pgfqpoint{5.480747in}{3.113200in}}%
\pgfpathlineto{\pgfqpoint{5.483958in}{3.114072in}}%
\pgfpathlineto{\pgfqpoint{5.487705in}{3.117630in}}%
\pgfpathlineto{\pgfqpoint{5.494662in}{3.128199in}}%
\pgfpathlineto{\pgfqpoint{5.500817in}{3.135796in}}%
\pgfpathlineto{\pgfqpoint{5.504564in}{3.137231in}}%
\pgfpathlineto{\pgfqpoint{5.507775in}{3.136087in}}%
\pgfpathlineto{\pgfqpoint{5.511789in}{3.131922in}}%
\pgfpathlineto{\pgfqpoint{5.526775in}{3.113327in}}%
\pgfpathlineto{\pgfqpoint{5.529987in}{3.113650in}}%
\pgfpathlineto{\pgfqpoint{5.533466in}{3.116372in}}%
\pgfpathlineto{\pgfqpoint{5.538818in}{3.123917in}}%
\pgfpathlineto{\pgfqpoint{5.547381in}{3.135585in}}%
\pgfpathlineto{\pgfqpoint{5.551128in}{3.137227in}}%
\pgfpathlineto{\pgfqpoint{5.554339in}{3.136264in}}%
\pgfpathlineto{\pgfqpoint{5.558085in}{3.132620in}}%
\pgfpathlineto{\pgfqpoint{5.565578in}{3.121188in}}%
\pgfpathlineto{\pgfqpoint{5.571198in}{3.114522in}}%
\pgfpathlineto{\pgfqpoint{5.574945in}{3.113192in}}%
\pgfpathlineto{\pgfqpoint{5.578156in}{3.114425in}}%
\pgfpathlineto{\pgfqpoint{5.582170in}{3.118677in}}%
\pgfpathlineto{\pgfqpoint{5.596889in}{3.137057in}}%
\pgfpathlineto{\pgfqpoint{5.600100in}{3.136826in}}%
\pgfpathlineto{\pgfqpoint{5.603579in}{3.134191in}}%
\pgfpathlineto{\pgfqpoint{5.608931in}{3.126716in}}%
\pgfpathlineto{\pgfqpoint{5.617494in}{3.114945in}}%
\pgfpathlineto{\pgfqpoint{5.621241in}{3.113200in}}%
\pgfpathlineto{\pgfqpoint{5.624452in}{3.114072in}}%
\pgfpathlineto{\pgfqpoint{5.628199in}{3.117630in}}%
\pgfpathlineto{\pgfqpoint{5.635157in}{3.128199in}}%
\pgfpathlineto{\pgfqpoint{5.641311in}{3.135796in}}%
\pgfpathlineto{\pgfqpoint{5.645058in}{3.137231in}}%
\pgfpathlineto{\pgfqpoint{5.648269in}{3.136087in}}%
\pgfpathlineto{\pgfqpoint{5.652283in}{3.131922in}}%
\pgfpathlineto{\pgfqpoint{5.667269in}{3.113327in}}%
\pgfpathlineto{\pgfqpoint{5.670481in}{3.113650in}}%
\pgfpathlineto{\pgfqpoint{5.673960in}{3.116372in}}%
\pgfpathlineto{\pgfqpoint{5.679312in}{3.123917in}}%
\pgfpathlineto{\pgfqpoint{5.687875in}{3.135585in}}%
\pgfpathlineto{\pgfqpoint{5.691622in}{3.137227in}}%
\pgfpathlineto{\pgfqpoint{5.694833in}{3.136264in}}%
\pgfpathlineto{\pgfqpoint{5.698580in}{3.132620in}}%
\pgfpathlineto{\pgfqpoint{5.706073in}{3.121188in}}%
\pgfpathlineto{\pgfqpoint{5.711692in}{3.114522in}}%
\pgfpathlineto{\pgfqpoint{5.715439in}{3.113192in}}%
\pgfpathlineto{\pgfqpoint{5.718650in}{3.114425in}}%
\pgfpathlineto{\pgfqpoint{5.722664in}{3.118677in}}%
\pgfpathlineto{\pgfqpoint{5.737383in}{3.137057in}}%
\pgfpathlineto{\pgfqpoint{5.740594in}{3.136826in}}%
\pgfpathlineto{\pgfqpoint{5.744073in}{3.134191in}}%
\pgfpathlineto{\pgfqpoint{5.749425in}{3.126716in}}%
\pgfpathlineto{\pgfqpoint{5.757988in}{3.114945in}}%
\pgfpathlineto{\pgfqpoint{5.761735in}{3.113200in}}%
\pgfpathlineto{\pgfqpoint{5.764946in}{3.114072in}}%
\pgfpathlineto{\pgfqpoint{5.768693in}{3.117630in}}%
\pgfpathlineto{\pgfqpoint{5.775651in}{3.128199in}}%
\pgfpathlineto{\pgfqpoint{5.781806in}{3.135796in}}%
\pgfpathlineto{\pgfqpoint{5.785552in}{3.137231in}}%
\pgfpathlineto{\pgfqpoint{5.788763in}{3.136087in}}%
\pgfpathlineto{\pgfqpoint{5.792778in}{3.131922in}}%
\pgfpathlineto{\pgfqpoint{5.807764in}{3.113327in}}%
\pgfpathlineto{\pgfqpoint{5.810975in}{3.113650in}}%
\pgfpathlineto{\pgfqpoint{5.814454in}{3.116372in}}%
\pgfpathlineto{\pgfqpoint{5.819806in}{3.123917in}}%
\pgfpathlineto{\pgfqpoint{5.828369in}{3.135585in}}%
\pgfpathlineto{\pgfqpoint{5.832116in}{3.137227in}}%
\pgfpathlineto{\pgfqpoint{5.835327in}{3.136264in}}%
\pgfpathlineto{\pgfqpoint{5.839074in}{3.132620in}}%
\pgfpathlineto{\pgfqpoint{5.846567in}{3.121188in}}%
\pgfpathlineto{\pgfqpoint{5.852186in}{3.114522in}}%
\pgfpathlineto{\pgfqpoint{5.855933in}{3.113192in}}%
\pgfpathlineto{\pgfqpoint{5.859144in}{3.114425in}}%
\pgfpathlineto{\pgfqpoint{5.863158in}{3.118677in}}%
\pgfpathlineto{\pgfqpoint{5.877877in}{3.137057in}}%
\pgfpathlineto{\pgfqpoint{5.881088in}{3.136826in}}%
\pgfpathlineto{\pgfqpoint{5.884567in}{3.134191in}}%
\pgfpathlineto{\pgfqpoint{5.889919in}{3.126716in}}%
\pgfpathlineto{\pgfqpoint{5.898483in}{3.114945in}}%
\pgfpathlineto{\pgfqpoint{5.902229in}{3.113200in}}%
\pgfpathlineto{\pgfqpoint{5.905440in}{3.114072in}}%
\pgfpathlineto{\pgfqpoint{5.909187in}{3.117630in}}%
\pgfpathlineto{\pgfqpoint{5.916145in}{3.128199in}}%
\pgfpathlineto{\pgfqpoint{5.922300in}{3.135796in}}%
\pgfpathlineto{\pgfqpoint{5.926046in}{3.137231in}}%
\pgfpathlineto{\pgfqpoint{5.929258in}{3.136087in}}%
\pgfpathlineto{\pgfqpoint{5.933272in}{3.131922in}}%
\pgfpathlineto{\pgfqpoint{5.948258in}{3.113327in}}%
\pgfpathlineto{\pgfqpoint{5.951469in}{3.113650in}}%
\pgfpathlineto{\pgfqpoint{5.954948in}{3.116372in}}%
\pgfpathlineto{\pgfqpoint{5.960300in}{3.123917in}}%
\pgfpathlineto{\pgfqpoint{5.968863in}{3.135585in}}%
\pgfpathlineto{\pgfqpoint{5.972610in}{3.137227in}}%
\pgfpathlineto{\pgfqpoint{5.975821in}{3.136264in}}%
\pgfpathlineto{\pgfqpoint{5.979568in}{3.132620in}}%
\pgfpathlineto{\pgfqpoint{5.987061in}{3.121188in}}%
\pgfpathlineto{\pgfqpoint{5.992681in}{3.114522in}}%
\pgfpathlineto{\pgfqpoint{5.996427in}{3.113192in}}%
\pgfpathlineto{\pgfqpoint{5.999638in}{3.114425in}}%
\pgfpathlineto{\pgfqpoint{6.003652in}{3.118677in}}%
\pgfpathlineto{\pgfqpoint{6.018371in}{3.137057in}}%
\pgfpathlineto{\pgfqpoint{6.021582in}{3.136826in}}%
\pgfpathlineto{\pgfqpoint{6.025061in}{3.134191in}}%
\pgfpathlineto{\pgfqpoint{6.030413in}{3.126716in}}%
\pgfpathlineto{\pgfqpoint{6.038977in}{3.114945in}}%
\pgfpathlineto{\pgfqpoint{6.042723in}{3.113200in}}%
\pgfpathlineto{\pgfqpoint{6.045935in}{3.114072in}}%
\pgfpathlineto{\pgfqpoint{6.049681in}{3.117630in}}%
\pgfpathlineto{\pgfqpoint{6.056639in}{3.128199in}}%
\pgfpathlineto{\pgfqpoint{6.062794in}{3.135796in}}%
\pgfpathlineto{\pgfqpoint{6.066540in}{3.137231in}}%
\pgfpathlineto{\pgfqpoint{6.069752in}{3.136087in}}%
\pgfpathlineto{\pgfqpoint{6.073766in}{3.131922in}}%
\pgfpathlineto{\pgfqpoint{6.088752in}{3.113327in}}%
\pgfpathlineto{\pgfqpoint{6.091963in}{3.113650in}}%
\pgfpathlineto{\pgfqpoint{6.095442in}{3.116372in}}%
\pgfpathlineto{\pgfqpoint{6.100794in}{3.123917in}}%
\pgfpathlineto{\pgfqpoint{6.109358in}{3.135585in}}%
\pgfpathlineto{\pgfqpoint{6.113104in}{3.137227in}}%
\pgfpathlineto{\pgfqpoint{6.116315in}{3.136264in}}%
\pgfpathlineto{\pgfqpoint{6.120062in}{3.132620in}}%
\pgfpathlineto{\pgfqpoint{6.127555in}{3.121188in}}%
\pgfpathlineto{\pgfqpoint{6.133175in}{3.114522in}}%
\pgfpathlineto{\pgfqpoint{6.136921in}{3.113192in}}%
\pgfpathlineto{\pgfqpoint{6.140132in}{3.114425in}}%
\pgfpathlineto{\pgfqpoint{6.144147in}{3.118677in}}%
\pgfpathlineto{\pgfqpoint{6.158865in}{3.137057in}}%
\pgfpathlineto{\pgfqpoint{6.162076in}{3.136826in}}%
\pgfpathlineto{\pgfqpoint{6.165555in}{3.134191in}}%
\pgfpathlineto{\pgfqpoint{6.170907in}{3.126716in}}%
\pgfpathlineto{\pgfqpoint{6.179471in}{3.114945in}}%
\pgfpathlineto{\pgfqpoint{6.183217in}{3.113200in}}%
\pgfpathlineto{\pgfqpoint{6.186429in}{3.114072in}}%
\pgfpathlineto{\pgfqpoint{6.190175in}{3.117630in}}%
\pgfpathlineto{\pgfqpoint{6.197133in}{3.128199in}}%
\pgfpathlineto{\pgfqpoint{6.203288in}{3.135796in}}%
\pgfpathlineto{\pgfqpoint{6.207034in}{3.137231in}}%
\pgfpathlineto{\pgfqpoint{6.210246in}{3.136087in}}%
\pgfpathlineto{\pgfqpoint{6.214260in}{3.131922in}}%
\pgfpathlineto{\pgfqpoint{6.229246in}{3.113327in}}%
\pgfpathlineto{\pgfqpoint{6.232457in}{3.113650in}}%
\pgfpathlineto{\pgfqpoint{6.235936in}{3.116372in}}%
\pgfpathlineto{\pgfqpoint{6.241288in}{3.123917in}}%
\pgfpathlineto{\pgfqpoint{6.249852in}{3.135585in}}%
\pgfpathlineto{\pgfqpoint{6.253598in}{3.137227in}}%
\pgfpathlineto{\pgfqpoint{6.256809in}{3.136264in}}%
\pgfpathlineto{\pgfqpoint{6.260556in}{3.132620in}}%
\pgfpathlineto{\pgfqpoint{6.268049in}{3.121188in}}%
\pgfpathlineto{\pgfqpoint{6.273669in}{3.114522in}}%
\pgfpathlineto{\pgfqpoint{6.277415in}{3.113192in}}%
\pgfpathlineto{\pgfqpoint{6.280627in}{3.114425in}}%
\pgfpathlineto{\pgfqpoint{6.284641in}{3.118677in}}%
\pgfpathlineto{\pgfqpoint{6.299359in}{3.137057in}}%
\pgfpathlineto{\pgfqpoint{6.302570in}{3.136826in}}%
\pgfpathlineto{\pgfqpoint{6.306049in}{3.134191in}}%
\pgfpathlineto{\pgfqpoint{6.311401in}{3.126716in}}%
\pgfpathlineto{\pgfqpoint{6.319965in}{3.114945in}}%
\pgfpathlineto{\pgfqpoint{6.323711in}{3.113200in}}%
\pgfpathlineto{\pgfqpoint{6.326923in}{3.114072in}}%
\pgfpathlineto{\pgfqpoint{6.330669in}{3.117630in}}%
\pgfpathlineto{\pgfqpoint{6.337627in}{3.128199in}}%
\pgfpathlineto{\pgfqpoint{6.343782in}{3.135796in}}%
\pgfpathlineto{\pgfqpoint{6.347529in}{3.137231in}}%
\pgfpathlineto{\pgfqpoint{6.350740in}{3.136087in}}%
\pgfpathlineto{\pgfqpoint{6.354754in}{3.131922in}}%
\pgfpathlineto{\pgfqpoint{6.369740in}{3.113327in}}%
\pgfpathlineto{\pgfqpoint{6.372951in}{3.113650in}}%
\pgfpathlineto{\pgfqpoint{6.376430in}{3.116372in}}%
\pgfpathlineto{\pgfqpoint{6.381782in}{3.123917in}}%
\pgfpathlineto{\pgfqpoint{6.390346in}{3.135585in}}%
\pgfpathlineto{\pgfqpoint{6.394092in}{3.137227in}}%
\pgfpathlineto{\pgfqpoint{6.397304in}{3.136264in}}%
\pgfpathlineto{\pgfqpoint{6.401050in}{3.132620in}}%
\pgfpathlineto{\pgfqpoint{6.408543in}{3.121188in}}%
\pgfpathlineto{\pgfqpoint{6.414163in}{3.114522in}}%
\pgfpathlineto{\pgfqpoint{6.417909in}{3.113192in}}%
\pgfpathlineto{\pgfqpoint{6.421121in}{3.114425in}}%
\pgfpathlineto{\pgfqpoint{6.425135in}{3.118677in}}%
\pgfpathlineto{\pgfqpoint{6.439853in}{3.137057in}}%
\pgfpathlineto{\pgfqpoint{6.443065in}{3.136826in}}%
\pgfpathlineto{\pgfqpoint{6.446543in}{3.134191in}}%
\pgfpathlineto{\pgfqpoint{6.451896in}{3.126716in}}%
\pgfpathlineto{\pgfqpoint{6.460459in}{3.114945in}}%
\pgfpathlineto{\pgfqpoint{6.464206in}{3.113200in}}%
\pgfpathlineto{\pgfqpoint{6.467417in}{3.114072in}}%
\pgfpathlineto{\pgfqpoint{6.471163in}{3.117630in}}%
\pgfpathlineto{\pgfqpoint{6.478121in}{3.128199in}}%
\pgfpathlineto{\pgfqpoint{6.484276in}{3.135796in}}%
\pgfpathlineto{\pgfqpoint{6.488023in}{3.137231in}}%
\pgfpathlineto{\pgfqpoint{6.491234in}{3.136087in}}%
\pgfpathlineto{\pgfqpoint{6.495248in}{3.131922in}}%
\pgfpathlineto{\pgfqpoint{6.510234in}{3.113327in}}%
\pgfpathlineto{\pgfqpoint{6.513445in}{3.113650in}}%
\pgfpathlineto{\pgfqpoint{6.516924in}{3.116372in}}%
\pgfpathlineto{\pgfqpoint{6.522276in}{3.123917in}}%
\pgfpathlineto{\pgfqpoint{6.530840in}{3.135585in}}%
\pgfpathlineto{\pgfqpoint{6.534586in}{3.137227in}}%
\pgfpathlineto{\pgfqpoint{6.537798in}{3.136264in}}%
\pgfpathlineto{\pgfqpoint{6.541544in}{3.132620in}}%
\pgfpathlineto{\pgfqpoint{6.549037in}{3.121188in}}%
\pgfpathlineto{\pgfqpoint{6.554657in}{3.114522in}}%
\pgfpathlineto{\pgfqpoint{6.558403in}{3.113192in}}%
\pgfpathlineto{\pgfqpoint{6.561615in}{3.114425in}}%
\pgfpathlineto{\pgfqpoint{6.565629in}{3.118677in}}%
\pgfpathlineto{\pgfqpoint{6.580347in}{3.137057in}}%
\pgfpathlineto{\pgfqpoint{6.583559in}{3.136826in}}%
\pgfpathlineto{\pgfqpoint{6.587038in}{3.134191in}}%
\pgfpathlineto{\pgfqpoint{6.592390in}{3.126716in}}%
\pgfpathlineto{\pgfqpoint{6.600953in}{3.114945in}}%
\pgfpathlineto{\pgfqpoint{6.604700in}{3.113200in}}%
\pgfpathlineto{\pgfqpoint{6.607911in}{3.114072in}}%
\pgfpathlineto{\pgfqpoint{6.611657in}{3.117630in}}%
\pgfpathlineto{\pgfqpoint{6.618615in}{3.128199in}}%
\pgfpathlineto{\pgfqpoint{6.624770in}{3.135796in}}%
\pgfpathlineto{\pgfqpoint{6.628517in}{3.137231in}}%
\pgfpathlineto{\pgfqpoint{6.631728in}{3.136087in}}%
\pgfpathlineto{\pgfqpoint{6.635742in}{3.131922in}}%
\pgfpathlineto{\pgfqpoint{6.650728in}{3.113327in}}%
\pgfpathlineto{\pgfqpoint{6.653939in}{3.113650in}}%
\pgfpathlineto{\pgfqpoint{6.657418in}{3.116372in}}%
\pgfpathlineto{\pgfqpoint{6.662771in}{3.123917in}}%
\pgfpathlineto{\pgfqpoint{6.663306in}{3.124778in}}%
\pgfpathlineto{\pgfqpoint{6.663306in}{3.124778in}}%
\pgfusepath{stroke}%
\end{pgfscope}%
\begin{pgfscope}%
\pgfpathrectangle{\pgfqpoint{0.467797in}{2.292089in}}{\pgfqpoint{6.490533in}{1.666241in}}%
\pgfusepath{clip}%
\pgfsetrectcap%
\pgfsetroundjoin%
\pgfsetlinewidth{1.505625pt}%
\definecolor{currentstroke}{rgb}{0.172549,0.627451,0.172549}%
\pgfsetstrokecolor{currentstroke}%
\pgfsetdash{}{0pt}%
\pgfpathmoveto{\pgfqpoint{0.762821in}{3.125209in}}%
\pgfpathlineto{\pgfqpoint{0.770314in}{3.135242in}}%
\pgfpathlineto{\pgfqpoint{0.774060in}{3.136860in}}%
\pgfpathlineto{\pgfqpoint{0.777272in}{3.135803in}}%
\pgfpathlineto{\pgfqpoint{0.781286in}{3.131649in}}%
\pgfpathlineto{\pgfqpoint{0.795737in}{3.113699in}}%
\pgfpathlineto{\pgfqpoint{0.798948in}{3.114039in}}%
\pgfpathlineto{\pgfqpoint{0.802427in}{3.116845in}}%
\pgfpathlineto{\pgfqpoint{0.808047in}{3.124944in}}%
\pgfpathlineto{\pgfqpoint{0.815540in}{3.135104in}}%
\pgfpathlineto{\pgfqpoint{0.819286in}{3.136853in}}%
\pgfpathlineto{\pgfqpoint{0.822497in}{3.135911in}}%
\pgfpathlineto{\pgfqpoint{0.826244in}{3.132218in}}%
\pgfpathlineto{\pgfqpoint{0.834807in}{3.119219in}}%
\pgfpathlineto{\pgfqpoint{0.839624in}{3.114321in}}%
\pgfpathlineto{\pgfqpoint{0.843103in}{3.113638in}}%
\pgfpathlineto{\pgfqpoint{0.846315in}{3.115350in}}%
\pgfpathlineto{\pgfqpoint{0.850596in}{3.120500in}}%
\pgfpathlineto{\pgfqpoint{0.862371in}{3.136127in}}%
\pgfpathlineto{\pgfqpoint{0.865850in}{3.136771in}}%
\pgfpathlineto{\pgfqpoint{0.869061in}{3.135024in}}%
\pgfpathlineto{\pgfqpoint{0.873343in}{3.129842in}}%
\pgfpathlineto{\pgfqpoint{0.885118in}{3.114263in}}%
\pgfpathlineto{\pgfqpoint{0.888597in}{3.113658in}}%
\pgfpathlineto{\pgfqpoint{0.891808in}{3.115439in}}%
\pgfpathlineto{\pgfqpoint{0.896090in}{3.120653in}}%
\pgfpathlineto{\pgfqpoint{0.907597in}{3.136032in}}%
\pgfpathlineto{\pgfqpoint{0.911076in}{3.136801in}}%
\pgfpathlineto{\pgfqpoint{0.914287in}{3.135165in}}%
\pgfpathlineto{\pgfqpoint{0.918569in}{3.130085in}}%
\pgfpathlineto{\pgfqpoint{0.930611in}{3.114207in}}%
\pgfpathlineto{\pgfqpoint{0.934090in}{3.113681in}}%
\pgfpathlineto{\pgfqpoint{0.937301in}{3.115531in}}%
\pgfpathlineto{\pgfqpoint{0.941851in}{3.121209in}}%
\pgfpathlineto{\pgfqpoint{0.952822in}{3.135930in}}%
\pgfpathlineto{\pgfqpoint{0.956301in}{3.136825in}}%
\pgfpathlineto{\pgfqpoint{0.959513in}{3.135300in}}%
\pgfpathlineto{\pgfqpoint{0.963794in}{3.130325in}}%
\pgfpathlineto{\pgfqpoint{0.976104in}{3.114154in}}%
\pgfpathlineto{\pgfqpoint{0.979583in}{3.113707in}}%
\pgfpathlineto{\pgfqpoint{0.982795in}{3.115624in}}%
\pgfpathlineto{\pgfqpoint{0.987344in}{3.121365in}}%
\pgfpathlineto{\pgfqpoint{0.998048in}{3.135823in}}%
\pgfpathlineto{\pgfqpoint{1.001527in}{3.136843in}}%
\pgfpathlineto{\pgfqpoint{1.004738in}{3.135431in}}%
\pgfpathlineto{\pgfqpoint{1.008752in}{3.130942in}}%
\pgfpathlineto{\pgfqpoint{1.022133in}{3.113872in}}%
\pgfpathlineto{\pgfqpoint{1.025344in}{3.113817in}}%
\pgfpathlineto{\pgfqpoint{1.028823in}{3.116246in}}%
\pgfpathlineto{\pgfqpoint{1.033908in}{3.123194in}}%
\pgfpathlineto{\pgfqpoint{1.042739in}{3.135309in}}%
\pgfpathlineto{\pgfqpoint{1.046485in}{3.136861in}}%
\pgfpathlineto{\pgfqpoint{1.049696in}{3.135747in}}%
\pgfpathlineto{\pgfqpoint{1.053711in}{3.131538in}}%
\pgfpathlineto{\pgfqpoint{1.067894in}{3.113749in}}%
\pgfpathlineto{\pgfqpoint{1.071105in}{3.113958in}}%
\pgfpathlineto{\pgfqpoint{1.074584in}{3.116640in}}%
\pgfpathlineto{\pgfqpoint{1.079936in}{3.124215in}}%
\pgfpathlineto{\pgfqpoint{1.087964in}{3.135174in}}%
\pgfpathlineto{\pgfqpoint{1.091711in}{3.136857in}}%
\pgfpathlineto{\pgfqpoint{1.094922in}{3.135857in}}%
\pgfpathlineto{\pgfqpoint{1.098669in}{3.132112in}}%
\pgfpathlineto{\pgfqpoint{1.113922in}{3.113608in}}%
\pgfpathlineto{\pgfqpoint{1.117134in}{3.114269in}}%
\pgfpathlineto{\pgfqpoint{1.120880in}{3.117694in}}%
\pgfpathlineto{\pgfqpoint{1.127570in}{3.127810in}}%
\pgfpathlineto{\pgfqpoint{1.133725in}{3.135470in}}%
\pgfpathlineto{\pgfqpoint{1.137472in}{3.136858in}}%
\pgfpathlineto{\pgfqpoint{1.140683in}{3.135601in}}%
\pgfpathlineto{\pgfqpoint{1.144697in}{3.131257in}}%
\pgfpathlineto{\pgfqpoint{1.158345in}{3.113857in}}%
\pgfpathlineto{\pgfqpoint{1.161557in}{3.113831in}}%
\pgfpathlineto{\pgfqpoint{1.165035in}{3.116289in}}%
\pgfpathlineto{\pgfqpoint{1.170120in}{3.123260in}}%
\pgfpathlineto{\pgfqpoint{1.178951in}{3.135342in}}%
\pgfpathlineto{\pgfqpoint{1.182698in}{3.136861in}}%
\pgfpathlineto{\pgfqpoint{1.185909in}{3.135718in}}%
\pgfpathlineto{\pgfqpoint{1.189923in}{3.131482in}}%
\pgfpathlineto{\pgfqpoint{1.204106in}{3.113737in}}%
\pgfpathlineto{\pgfqpoint{1.207318in}{3.113975in}}%
\pgfpathlineto{\pgfqpoint{1.210796in}{3.116685in}}%
\pgfpathlineto{\pgfqpoint{1.216149in}{3.124281in}}%
\pgfpathlineto{\pgfqpoint{1.224177in}{3.135208in}}%
\pgfpathlineto{\pgfqpoint{1.227923in}{3.136859in}}%
\pgfpathlineto{\pgfqpoint{1.231135in}{3.135830in}}%
\pgfpathlineto{\pgfqpoint{1.234881in}{3.132058in}}%
\pgfpathlineto{\pgfqpoint{1.250135in}{3.113602in}}%
\pgfpathlineto{\pgfqpoint{1.253346in}{3.114292in}}%
\pgfpathlineto{\pgfqpoint{1.257093in}{3.117745in}}%
\pgfpathlineto{\pgfqpoint{1.263783in}{3.127875in}}%
\pgfpathlineto{\pgfqpoint{1.269938in}{3.135501in}}%
\pgfpathlineto{\pgfqpoint{1.273417in}{3.136861in}}%
\pgfpathlineto{\pgfqpoint{1.276628in}{3.135761in}}%
\pgfpathlineto{\pgfqpoint{1.280642in}{3.131566in}}%
\pgfpathlineto{\pgfqpoint{1.294825in}{3.113755in}}%
\pgfpathlineto{\pgfqpoint{1.298037in}{3.113949in}}%
\pgfpathlineto{\pgfqpoint{1.301515in}{3.116617in}}%
\pgfpathlineto{\pgfqpoint{1.306868in}{3.124181in}}%
\pgfpathlineto{\pgfqpoint{1.314896in}{3.135156in}}%
\pgfpathlineto{\pgfqpoint{1.318642in}{3.136856in}}%
\pgfpathlineto{\pgfqpoint{1.321854in}{3.135871in}}%
\pgfpathlineto{\pgfqpoint{1.325600in}{3.132139in}}%
\pgfpathlineto{\pgfqpoint{1.340854in}{3.113611in}}%
\pgfpathlineto{\pgfqpoint{1.344065in}{3.114257in}}%
\pgfpathlineto{\pgfqpoint{1.347812in}{3.117669in}}%
\pgfpathlineto{\pgfqpoint{1.354502in}{3.127778in}}%
\pgfpathlineto{\pgfqpoint{1.360657in}{3.135454in}}%
\pgfpathlineto{\pgfqpoint{1.364403in}{3.136859in}}%
\pgfpathlineto{\pgfqpoint{1.367615in}{3.135616in}}%
\pgfpathlineto{\pgfqpoint{1.371629in}{3.131285in}}%
\pgfpathlineto{\pgfqpoint{1.385544in}{3.113774in}}%
\pgfpathlineto{\pgfqpoint{1.388756in}{3.113924in}}%
\pgfpathlineto{\pgfqpoint{1.392235in}{3.116550in}}%
\pgfpathlineto{\pgfqpoint{1.397587in}{3.124082in}}%
\pgfpathlineto{\pgfqpoint{1.405883in}{3.135325in}}%
\pgfpathlineto{\pgfqpoint{1.409629in}{3.136861in}}%
\pgfpathlineto{\pgfqpoint{1.412840in}{3.135733in}}%
\pgfpathlineto{\pgfqpoint{1.416854in}{3.131510in}}%
\pgfpathlineto{\pgfqpoint{1.431038in}{3.113743in}}%
\pgfpathlineto{\pgfqpoint{1.434249in}{3.113966in}}%
\pgfpathlineto{\pgfqpoint{1.437728in}{3.116662in}}%
\pgfpathlineto{\pgfqpoint{1.443080in}{3.124248in}}%
\pgfpathlineto{\pgfqpoint{1.451108in}{3.135191in}}%
\pgfpathlineto{\pgfqpoint{1.454855in}{3.136858in}}%
\pgfpathlineto{\pgfqpoint{1.458066in}{3.135844in}}%
\pgfpathlineto{\pgfqpoint{1.461813in}{3.132085in}}%
\pgfpathlineto{\pgfqpoint{1.477066in}{3.113605in}}%
\pgfpathlineto{\pgfqpoint{1.480277in}{3.114280in}}%
\pgfpathlineto{\pgfqpoint{1.484024in}{3.117720in}}%
\pgfpathlineto{\pgfqpoint{1.490714in}{3.127843in}}%
\pgfpathlineto{\pgfqpoint{1.496869in}{3.135486in}}%
\pgfpathlineto{\pgfqpoint{1.500616in}{3.136857in}}%
\pgfpathlineto{\pgfqpoint{1.503827in}{3.135586in}}%
\pgfpathlineto{\pgfqpoint{1.507841in}{3.131228in}}%
\pgfpathlineto{\pgfqpoint{1.521489in}{3.113850in}}%
\pgfpathlineto{\pgfqpoint{1.524700in}{3.113839in}}%
\pgfpathlineto{\pgfqpoint{1.528179in}{3.116310in}}%
\pgfpathlineto{\pgfqpoint{1.533264in}{3.123292in}}%
\pgfpathlineto{\pgfqpoint{1.542095in}{3.135358in}}%
\pgfpathlineto{\pgfqpoint{1.545841in}{3.136861in}}%
\pgfpathlineto{\pgfqpoint{1.549053in}{3.135704in}}%
\pgfpathlineto{\pgfqpoint{1.553067in}{3.131454in}}%
\pgfpathlineto{\pgfqpoint{1.567250in}{3.113732in}}%
\pgfpathlineto{\pgfqpoint{1.570461in}{3.113984in}}%
\pgfpathlineto{\pgfqpoint{1.573940in}{3.116708in}}%
\pgfpathlineto{\pgfqpoint{1.579292in}{3.124314in}}%
\pgfpathlineto{\pgfqpoint{1.587321in}{3.135225in}}%
\pgfpathlineto{\pgfqpoint{1.591067in}{3.136860in}}%
\pgfpathlineto{\pgfqpoint{1.594278in}{3.135817in}}%
\pgfpathlineto{\pgfqpoint{1.598293in}{3.131677in}}%
\pgfpathlineto{\pgfqpoint{1.612743in}{3.113704in}}%
\pgfpathlineto{\pgfqpoint{1.615955in}{3.114030in}}%
\pgfpathlineto{\pgfqpoint{1.619434in}{3.116822in}}%
\pgfpathlineto{\pgfqpoint{1.625053in}{3.124911in}}%
\pgfpathlineto{\pgfqpoint{1.632546in}{3.135087in}}%
\pgfpathlineto{\pgfqpoint{1.636293in}{3.136852in}}%
\pgfpathlineto{\pgfqpoint{1.639504in}{3.135924in}}%
\pgfpathlineto{\pgfqpoint{1.643251in}{3.132245in}}%
\pgfpathlineto{\pgfqpoint{1.651546in}{3.119622in}}%
\pgfpathlineto{\pgfqpoint{1.656631in}{3.114333in}}%
\pgfpathlineto{\pgfqpoint{1.660110in}{3.113634in}}%
\pgfpathlineto{\pgfqpoint{1.663321in}{3.115332in}}%
\pgfpathlineto{\pgfqpoint{1.667603in}{3.120470in}}%
\pgfpathlineto{\pgfqpoint{1.679378in}{3.136116in}}%
\pgfpathlineto{\pgfqpoint{1.682857in}{3.136775in}}%
\pgfpathlineto{\pgfqpoint{1.686068in}{3.135042in}}%
\pgfpathlineto{\pgfqpoint{1.690350in}{3.129873in}}%
\pgfpathlineto{\pgfqpoint{1.702124in}{3.114274in}}%
\pgfpathlineto{\pgfqpoint{1.705603in}{3.113654in}}%
\pgfpathlineto{\pgfqpoint{1.708815in}{3.115421in}}%
\pgfpathlineto{\pgfqpoint{1.713096in}{3.120622in}}%
\pgfpathlineto{\pgfqpoint{1.724603in}{3.136019in}}%
\pgfpathlineto{\pgfqpoint{1.728082in}{3.136804in}}%
\pgfpathlineto{\pgfqpoint{1.731294in}{3.135182in}}%
\pgfpathlineto{\pgfqpoint{1.735575in}{3.130115in}}%
\pgfpathlineto{\pgfqpoint{1.747618in}{3.114218in}}%
\pgfpathlineto{\pgfqpoint{1.751097in}{3.113677in}}%
\pgfpathlineto{\pgfqpoint{1.754308in}{3.115512in}}%
\pgfpathlineto{\pgfqpoint{1.758857in}{3.121177in}}%
\pgfpathlineto{\pgfqpoint{1.769829in}{3.135917in}}%
\pgfpathlineto{\pgfqpoint{1.773308in}{3.136828in}}%
\pgfpathlineto{\pgfqpoint{1.776519in}{3.135317in}}%
\pgfpathlineto{\pgfqpoint{1.780801in}{3.130355in}}%
\pgfpathlineto{\pgfqpoint{1.793111in}{3.114164in}}%
\pgfpathlineto{\pgfqpoint{1.796590in}{3.113702in}}%
\pgfpathlineto{\pgfqpoint{1.799801in}{3.115606in}}%
\pgfpathlineto{\pgfqpoint{1.804351in}{3.121334in}}%
\pgfpathlineto{\pgfqpoint{1.815055in}{3.135810in}}%
\pgfpathlineto{\pgfqpoint{1.818534in}{3.136845in}}%
\pgfpathlineto{\pgfqpoint{1.821745in}{3.135446in}}%
\pgfpathlineto{\pgfqpoint{1.825759in}{3.130971in}}%
\pgfpathlineto{\pgfqpoint{1.839140in}{3.113880in}}%
\pgfpathlineto{\pgfqpoint{1.842351in}{3.113810in}}%
\pgfpathlineto{\pgfqpoint{1.845830in}{3.116225in}}%
\pgfpathlineto{\pgfqpoint{1.850914in}{3.123162in}}%
\pgfpathlineto{\pgfqpoint{1.859745in}{3.135292in}}%
\pgfpathlineto{\pgfqpoint{1.863492in}{3.136861in}}%
\pgfpathlineto{\pgfqpoint{1.866703in}{3.135761in}}%
\pgfpathlineto{\pgfqpoint{1.870717in}{3.131566in}}%
\pgfpathlineto{\pgfqpoint{1.884901in}{3.113755in}}%
\pgfpathlineto{\pgfqpoint{1.888112in}{3.113949in}}%
\pgfpathlineto{\pgfqpoint{1.891591in}{3.116617in}}%
\pgfpathlineto{\pgfqpoint{1.896943in}{3.124181in}}%
\pgfpathlineto{\pgfqpoint{1.904971in}{3.135156in}}%
\pgfpathlineto{\pgfqpoint{1.908718in}{3.136856in}}%
\pgfpathlineto{\pgfqpoint{1.911929in}{3.135871in}}%
\pgfpathlineto{\pgfqpoint{1.915675in}{3.132139in}}%
\pgfpathlineto{\pgfqpoint{1.930929in}{3.113611in}}%
\pgfpathlineto{\pgfqpoint{1.934140in}{3.114257in}}%
\pgfpathlineto{\pgfqpoint{1.937887in}{3.117669in}}%
\pgfpathlineto{\pgfqpoint{1.944577in}{3.127778in}}%
\pgfpathlineto{\pgfqpoint{1.950732in}{3.135454in}}%
\pgfpathlineto{\pgfqpoint{1.954479in}{3.136859in}}%
\pgfpathlineto{\pgfqpoint{1.957690in}{3.135616in}}%
\pgfpathlineto{\pgfqpoint{1.961704in}{3.131285in}}%
\pgfpathlineto{\pgfqpoint{1.975620in}{3.113774in}}%
\pgfpathlineto{\pgfqpoint{1.978831in}{3.113924in}}%
\pgfpathlineto{\pgfqpoint{1.982310in}{3.116550in}}%
\pgfpathlineto{\pgfqpoint{1.987662in}{3.124082in}}%
\pgfpathlineto{\pgfqpoint{1.995958in}{3.135325in}}%
\pgfpathlineto{\pgfqpoint{1.999704in}{3.136861in}}%
\pgfpathlineto{\pgfqpoint{2.002916in}{3.135733in}}%
\pgfpathlineto{\pgfqpoint{2.006930in}{3.131510in}}%
\pgfpathlineto{\pgfqpoint{2.021113in}{3.113743in}}%
\pgfpathlineto{\pgfqpoint{2.024324in}{3.113966in}}%
\pgfpathlineto{\pgfqpoint{2.027803in}{3.116662in}}%
\pgfpathlineto{\pgfqpoint{2.033155in}{3.124248in}}%
\pgfpathlineto{\pgfqpoint{2.041183in}{3.135191in}}%
\pgfpathlineto{\pgfqpoint{2.044930in}{3.136858in}}%
\pgfpathlineto{\pgfqpoint{2.048141in}{3.135844in}}%
\pgfpathlineto{\pgfqpoint{2.051888in}{3.132085in}}%
\pgfpathlineto{\pgfqpoint{2.067141in}{3.113605in}}%
\pgfpathlineto{\pgfqpoint{2.070353in}{3.114280in}}%
\pgfpathlineto{\pgfqpoint{2.074099in}{3.117720in}}%
\pgfpathlineto{\pgfqpoint{2.080789in}{3.127843in}}%
\pgfpathlineto{\pgfqpoint{2.086944in}{3.135486in}}%
\pgfpathlineto{\pgfqpoint{2.090691in}{3.136857in}}%
\pgfpathlineto{\pgfqpoint{2.093902in}{3.135586in}}%
\pgfpathlineto{\pgfqpoint{2.097916in}{3.131228in}}%
\pgfpathlineto{\pgfqpoint{2.111564in}{3.113850in}}%
\pgfpathlineto{\pgfqpoint{2.114776in}{3.113839in}}%
\pgfpathlineto{\pgfqpoint{2.118255in}{3.116310in}}%
\pgfpathlineto{\pgfqpoint{2.123339in}{3.123292in}}%
\pgfpathlineto{\pgfqpoint{2.132170in}{3.135358in}}%
\pgfpathlineto{\pgfqpoint{2.135917in}{3.136861in}}%
\pgfpathlineto{\pgfqpoint{2.139128in}{3.135704in}}%
\pgfpathlineto{\pgfqpoint{2.143142in}{3.131454in}}%
\pgfpathlineto{\pgfqpoint{2.157325in}{3.113732in}}%
\pgfpathlineto{\pgfqpoint{2.160537in}{3.113984in}}%
\pgfpathlineto{\pgfqpoint{2.164015in}{3.116708in}}%
\pgfpathlineto{\pgfqpoint{2.169368in}{3.124314in}}%
\pgfpathlineto{\pgfqpoint{2.177396in}{3.135225in}}%
\pgfpathlineto{\pgfqpoint{2.181142in}{3.136860in}}%
\pgfpathlineto{\pgfqpoint{2.184354in}{3.135817in}}%
\pgfpathlineto{\pgfqpoint{2.188368in}{3.131677in}}%
\pgfpathlineto{\pgfqpoint{2.202819in}{3.113704in}}%
\pgfpathlineto{\pgfqpoint{2.206030in}{3.114030in}}%
\pgfpathlineto{\pgfqpoint{2.209509in}{3.116822in}}%
\pgfpathlineto{\pgfqpoint{2.215129in}{3.124911in}}%
\pgfpathlineto{\pgfqpoint{2.222622in}{3.135087in}}%
\pgfpathlineto{\pgfqpoint{2.226368in}{3.136852in}}%
\pgfpathlineto{\pgfqpoint{2.229579in}{3.135924in}}%
\pgfpathlineto{\pgfqpoint{2.233326in}{3.132245in}}%
\pgfpathlineto{\pgfqpoint{2.241622in}{3.119622in}}%
\pgfpathlineto{\pgfqpoint{2.246706in}{3.114333in}}%
\pgfpathlineto{\pgfqpoint{2.250185in}{3.113634in}}%
\pgfpathlineto{\pgfqpoint{2.253396in}{3.115332in}}%
\pgfpathlineto{\pgfqpoint{2.257678in}{3.120470in}}%
\pgfpathlineto{\pgfqpoint{2.269453in}{3.136116in}}%
\pgfpathlineto{\pgfqpoint{2.272932in}{3.136775in}}%
\pgfpathlineto{\pgfqpoint{2.276143in}{3.135042in}}%
\pgfpathlineto{\pgfqpoint{2.280425in}{3.129873in}}%
\pgfpathlineto{\pgfqpoint{2.292200in}{3.114274in}}%
\pgfpathlineto{\pgfqpoint{2.295679in}{3.113654in}}%
\pgfpathlineto{\pgfqpoint{2.298890in}{3.115421in}}%
\pgfpathlineto{\pgfqpoint{2.303172in}{3.120622in}}%
\pgfpathlineto{\pgfqpoint{2.314679in}{3.136019in}}%
\pgfpathlineto{\pgfqpoint{2.318158in}{3.136804in}}%
\pgfpathlineto{\pgfqpoint{2.321369in}{3.135182in}}%
\pgfpathlineto{\pgfqpoint{2.325651in}{3.130115in}}%
\pgfpathlineto{\pgfqpoint{2.337693in}{3.114218in}}%
\pgfpathlineto{\pgfqpoint{2.341172in}{3.113677in}}%
\pgfpathlineto{\pgfqpoint{2.344383in}{3.115512in}}%
\pgfpathlineto{\pgfqpoint{2.348932in}{3.121177in}}%
\pgfpathlineto{\pgfqpoint{2.359904in}{3.135917in}}%
\pgfpathlineto{\pgfqpoint{2.363383in}{3.136828in}}%
\pgfpathlineto{\pgfqpoint{2.366595in}{3.135317in}}%
\pgfpathlineto{\pgfqpoint{2.370876in}{3.130355in}}%
\pgfpathlineto{\pgfqpoint{2.383186in}{3.114164in}}%
\pgfpathlineto{\pgfqpoint{2.386665in}{3.113702in}}%
\pgfpathlineto{\pgfqpoint{2.389876in}{3.115606in}}%
\pgfpathlineto{\pgfqpoint{2.394426in}{3.121334in}}%
\pgfpathlineto{\pgfqpoint{2.405130in}{3.135810in}}%
\pgfpathlineto{\pgfqpoint{2.408609in}{3.136845in}}%
\pgfpathlineto{\pgfqpoint{2.411820in}{3.135446in}}%
\pgfpathlineto{\pgfqpoint{2.415834in}{3.130971in}}%
\pgfpathlineto{\pgfqpoint{2.429215in}{3.113880in}}%
\pgfpathlineto{\pgfqpoint{2.432426in}{3.113810in}}%
\pgfpathlineto{\pgfqpoint{2.435905in}{3.116225in}}%
\pgfpathlineto{\pgfqpoint{2.440990in}{3.123162in}}%
\pgfpathlineto{\pgfqpoint{2.449821in}{3.135292in}}%
\pgfpathlineto{\pgfqpoint{2.453567in}{3.136861in}}%
\pgfpathlineto{\pgfqpoint{2.456778in}{3.135761in}}%
\pgfpathlineto{\pgfqpoint{2.460793in}{3.131566in}}%
\pgfpathlineto{\pgfqpoint{2.474976in}{3.113755in}}%
\pgfpathlineto{\pgfqpoint{2.478187in}{3.113949in}}%
\pgfpathlineto{\pgfqpoint{2.481666in}{3.116617in}}%
\pgfpathlineto{\pgfqpoint{2.487018in}{3.124181in}}%
\pgfpathlineto{\pgfqpoint{2.495046in}{3.135156in}}%
\pgfpathlineto{\pgfqpoint{2.498793in}{3.136856in}}%
\pgfpathlineto{\pgfqpoint{2.502004in}{3.135871in}}%
\pgfpathlineto{\pgfqpoint{2.505751in}{3.132139in}}%
\pgfpathlineto{\pgfqpoint{2.521004in}{3.113611in}}%
\pgfpathlineto{\pgfqpoint{2.524216in}{3.114257in}}%
\pgfpathlineto{\pgfqpoint{2.527962in}{3.117669in}}%
\pgfpathlineto{\pgfqpoint{2.534652in}{3.127778in}}%
\pgfpathlineto{\pgfqpoint{2.540807in}{3.135454in}}%
\pgfpathlineto{\pgfqpoint{2.544554in}{3.136859in}}%
\pgfpathlineto{\pgfqpoint{2.547765in}{3.135616in}}%
\pgfpathlineto{\pgfqpoint{2.551779in}{3.131285in}}%
\pgfpathlineto{\pgfqpoint{2.565695in}{3.113774in}}%
\pgfpathlineto{\pgfqpoint{2.568906in}{3.113924in}}%
\pgfpathlineto{\pgfqpoint{2.572385in}{3.116550in}}%
\pgfpathlineto{\pgfqpoint{2.577737in}{3.124082in}}%
\pgfpathlineto{\pgfqpoint{2.586033in}{3.135325in}}%
\pgfpathlineto{\pgfqpoint{2.589780in}{3.136861in}}%
\pgfpathlineto{\pgfqpoint{2.592991in}{3.135733in}}%
\pgfpathlineto{\pgfqpoint{2.597005in}{3.131510in}}%
\pgfpathlineto{\pgfqpoint{2.611188in}{3.113743in}}%
\pgfpathlineto{\pgfqpoint{2.614399in}{3.113966in}}%
\pgfpathlineto{\pgfqpoint{2.617878in}{3.116662in}}%
\pgfpathlineto{\pgfqpoint{2.623230in}{3.124248in}}%
\pgfpathlineto{\pgfqpoint{2.631259in}{3.135191in}}%
\pgfpathlineto{\pgfqpoint{2.635005in}{3.136858in}}%
\pgfpathlineto{\pgfqpoint{2.638217in}{3.135844in}}%
\pgfpathlineto{\pgfqpoint{2.641963in}{3.132085in}}%
\pgfpathlineto{\pgfqpoint{2.657217in}{3.113605in}}%
\pgfpathlineto{\pgfqpoint{2.660428in}{3.114280in}}%
\pgfpathlineto{\pgfqpoint{2.664174in}{3.117720in}}%
\pgfpathlineto{\pgfqpoint{2.670865in}{3.127843in}}%
\pgfpathlineto{\pgfqpoint{2.677020in}{3.135486in}}%
\pgfpathlineto{\pgfqpoint{2.680766in}{3.136857in}}%
\pgfpathlineto{\pgfqpoint{2.683977in}{3.135586in}}%
\pgfpathlineto{\pgfqpoint{2.687992in}{3.131228in}}%
\pgfpathlineto{\pgfqpoint{2.701640in}{3.113850in}}%
\pgfpathlineto{\pgfqpoint{2.704851in}{3.113839in}}%
\pgfpathlineto{\pgfqpoint{2.708330in}{3.116310in}}%
\pgfpathlineto{\pgfqpoint{2.713414in}{3.123292in}}%
\pgfpathlineto{\pgfqpoint{2.722245in}{3.135358in}}%
\pgfpathlineto{\pgfqpoint{2.725992in}{3.136861in}}%
\pgfpathlineto{\pgfqpoint{2.729203in}{3.135704in}}%
\pgfpathlineto{\pgfqpoint{2.733217in}{3.131454in}}%
\pgfpathlineto{\pgfqpoint{2.747401in}{3.113732in}}%
\pgfpathlineto{\pgfqpoint{2.750612in}{3.113984in}}%
\pgfpathlineto{\pgfqpoint{2.754091in}{3.116708in}}%
\pgfpathlineto{\pgfqpoint{2.759443in}{3.124314in}}%
\pgfpathlineto{\pgfqpoint{2.767471in}{3.135225in}}%
\pgfpathlineto{\pgfqpoint{2.771218in}{3.136860in}}%
\pgfpathlineto{\pgfqpoint{2.774429in}{3.135817in}}%
\pgfpathlineto{\pgfqpoint{2.778443in}{3.131677in}}%
\pgfpathlineto{\pgfqpoint{2.792894in}{3.113704in}}%
\pgfpathlineto{\pgfqpoint{2.796105in}{3.114030in}}%
\pgfpathlineto{\pgfqpoint{2.799584in}{3.116822in}}%
\pgfpathlineto{\pgfqpoint{2.805204in}{3.124911in}}%
\pgfpathlineto{\pgfqpoint{2.812697in}{3.135087in}}%
\pgfpathlineto{\pgfqpoint{2.816443in}{3.136852in}}%
\pgfpathlineto{\pgfqpoint{2.819655in}{3.135924in}}%
\pgfpathlineto{\pgfqpoint{2.823401in}{3.132245in}}%
\pgfpathlineto{\pgfqpoint{2.831697in}{3.119622in}}%
\pgfpathlineto{\pgfqpoint{2.836782in}{3.114333in}}%
\pgfpathlineto{\pgfqpoint{2.840260in}{3.113634in}}%
\pgfpathlineto{\pgfqpoint{2.843472in}{3.115332in}}%
\pgfpathlineto{\pgfqpoint{2.847753in}{3.120470in}}%
\pgfpathlineto{\pgfqpoint{2.859528in}{3.136116in}}%
\pgfpathlineto{\pgfqpoint{2.863007in}{3.136775in}}%
\pgfpathlineto{\pgfqpoint{2.866218in}{3.135042in}}%
\pgfpathlineto{\pgfqpoint{2.870500in}{3.129873in}}%
\pgfpathlineto{\pgfqpoint{2.882275in}{3.114274in}}%
\pgfpathlineto{\pgfqpoint{2.885754in}{3.113654in}}%
\pgfpathlineto{\pgfqpoint{2.888965in}{3.115421in}}%
\pgfpathlineto{\pgfqpoint{2.893247in}{3.120622in}}%
\pgfpathlineto{\pgfqpoint{2.904754in}{3.136019in}}%
\pgfpathlineto{\pgfqpoint{2.908233in}{3.136804in}}%
\pgfpathlineto{\pgfqpoint{2.911444in}{3.135182in}}%
\pgfpathlineto{\pgfqpoint{2.915726in}{3.130115in}}%
\pgfpathlineto{\pgfqpoint{2.927768in}{3.114218in}}%
\pgfpathlineto{\pgfqpoint{2.931247in}{3.113677in}}%
\pgfpathlineto{\pgfqpoint{2.934458in}{3.115512in}}%
\pgfpathlineto{\pgfqpoint{2.939008in}{3.121177in}}%
\pgfpathlineto{\pgfqpoint{2.949980in}{3.135917in}}%
\pgfpathlineto{\pgfqpoint{2.953459in}{3.136828in}}%
\pgfpathlineto{\pgfqpoint{2.956670in}{3.135317in}}%
\pgfpathlineto{\pgfqpoint{2.960952in}{3.130355in}}%
\pgfpathlineto{\pgfqpoint{2.973262in}{3.114164in}}%
\pgfpathlineto{\pgfqpoint{2.976740in}{3.113702in}}%
\pgfpathlineto{\pgfqpoint{2.979952in}{3.115606in}}%
\pgfpathlineto{\pgfqpoint{2.984501in}{3.121334in}}%
\pgfpathlineto{\pgfqpoint{2.995205in}{3.135810in}}%
\pgfpathlineto{\pgfqpoint{2.998684in}{3.136845in}}%
\pgfpathlineto{\pgfqpoint{3.001896in}{3.135446in}}%
\pgfpathlineto{\pgfqpoint{3.005910in}{3.130971in}}%
\pgfpathlineto{\pgfqpoint{3.019290in}{3.113880in}}%
\pgfpathlineto{\pgfqpoint{3.022501in}{3.113810in}}%
\pgfpathlineto{\pgfqpoint{3.025980in}{3.116225in}}%
\pgfpathlineto{\pgfqpoint{3.031065in}{3.123162in}}%
\pgfpathlineto{\pgfqpoint{3.039896in}{3.135292in}}%
\pgfpathlineto{\pgfqpoint{3.043642in}{3.136861in}}%
\pgfpathlineto{\pgfqpoint{3.046854in}{3.135761in}}%
\pgfpathlineto{\pgfqpoint{3.050868in}{3.131566in}}%
\pgfpathlineto{\pgfqpoint{3.065051in}{3.113755in}}%
\pgfpathlineto{\pgfqpoint{3.068262in}{3.113949in}}%
\pgfpathlineto{\pgfqpoint{3.071741in}{3.116617in}}%
\pgfpathlineto{\pgfqpoint{3.077093in}{3.124181in}}%
\pgfpathlineto{\pgfqpoint{3.085122in}{3.135156in}}%
\pgfpathlineto{\pgfqpoint{3.088868in}{3.136856in}}%
\pgfpathlineto{\pgfqpoint{3.092079in}{3.135871in}}%
\pgfpathlineto{\pgfqpoint{3.095826in}{3.132139in}}%
\pgfpathlineto{\pgfqpoint{3.111080in}{3.113611in}}%
\pgfpathlineto{\pgfqpoint{3.114291in}{3.114257in}}%
\pgfpathlineto{\pgfqpoint{3.118037in}{3.117669in}}%
\pgfpathlineto{\pgfqpoint{3.124728in}{3.127778in}}%
\pgfpathlineto{\pgfqpoint{3.130883in}{3.135454in}}%
\pgfpathlineto{\pgfqpoint{3.134629in}{3.136859in}}%
\pgfpathlineto{\pgfqpoint{3.137840in}{3.135616in}}%
\pgfpathlineto{\pgfqpoint{3.141854in}{3.131285in}}%
\pgfpathlineto{\pgfqpoint{3.155770in}{3.113774in}}%
\pgfpathlineto{\pgfqpoint{3.158981in}{3.113924in}}%
\pgfpathlineto{\pgfqpoint{3.162460in}{3.116550in}}%
\pgfpathlineto{\pgfqpoint{3.167812in}{3.124082in}}%
\pgfpathlineto{\pgfqpoint{3.176108in}{3.135325in}}%
\pgfpathlineto{\pgfqpoint{3.179855in}{3.136861in}}%
\pgfpathlineto{\pgfqpoint{3.183066in}{3.135733in}}%
\pgfpathlineto{\pgfqpoint{3.187080in}{3.131510in}}%
\pgfpathlineto{\pgfqpoint{3.201263in}{3.113743in}}%
\pgfpathlineto{\pgfqpoint{3.204475in}{3.113966in}}%
\pgfpathlineto{\pgfqpoint{3.207954in}{3.116662in}}%
\pgfpathlineto{\pgfqpoint{3.213306in}{3.124248in}}%
\pgfpathlineto{\pgfqpoint{3.221334in}{3.135191in}}%
\pgfpathlineto{\pgfqpoint{3.225080in}{3.136858in}}%
\pgfpathlineto{\pgfqpoint{3.228292in}{3.135844in}}%
\pgfpathlineto{\pgfqpoint{3.232038in}{3.132085in}}%
\pgfpathlineto{\pgfqpoint{3.247292in}{3.113605in}}%
\pgfpathlineto{\pgfqpoint{3.250503in}{3.114280in}}%
\pgfpathlineto{\pgfqpoint{3.254250in}{3.117720in}}%
\pgfpathlineto{\pgfqpoint{3.260940in}{3.127843in}}%
\pgfpathlineto{\pgfqpoint{3.267095in}{3.135486in}}%
\pgfpathlineto{\pgfqpoint{3.270841in}{3.136857in}}%
\pgfpathlineto{\pgfqpoint{3.274053in}{3.135586in}}%
\pgfpathlineto{\pgfqpoint{3.278067in}{3.131228in}}%
\pgfpathlineto{\pgfqpoint{3.291715in}{3.113850in}}%
\pgfpathlineto{\pgfqpoint{3.294926in}{3.113839in}}%
\pgfpathlineto{\pgfqpoint{3.298405in}{3.116310in}}%
\pgfpathlineto{\pgfqpoint{3.303490in}{3.123292in}}%
\pgfpathlineto{\pgfqpoint{3.312321in}{3.135358in}}%
\pgfpathlineto{\pgfqpoint{3.316067in}{3.136861in}}%
\pgfpathlineto{\pgfqpoint{3.319278in}{3.135704in}}%
\pgfpathlineto{\pgfqpoint{3.323293in}{3.131454in}}%
\pgfpathlineto{\pgfqpoint{3.337476in}{3.113732in}}%
\pgfpathlineto{\pgfqpoint{3.340687in}{3.113984in}}%
\pgfpathlineto{\pgfqpoint{3.344166in}{3.116708in}}%
\pgfpathlineto{\pgfqpoint{3.349518in}{3.124314in}}%
\pgfpathlineto{\pgfqpoint{3.357546in}{3.135225in}}%
\pgfpathlineto{\pgfqpoint{3.361293in}{3.136860in}}%
\pgfpathlineto{\pgfqpoint{3.364504in}{3.135817in}}%
\pgfpathlineto{\pgfqpoint{3.368518in}{3.131677in}}%
\pgfpathlineto{\pgfqpoint{3.382969in}{3.113704in}}%
\pgfpathlineto{\pgfqpoint{3.386180in}{3.114030in}}%
\pgfpathlineto{\pgfqpoint{3.389659in}{3.116822in}}%
\pgfpathlineto{\pgfqpoint{3.395279in}{3.124911in}}%
\pgfpathlineto{\pgfqpoint{3.402772in}{3.135087in}}%
\pgfpathlineto{\pgfqpoint{3.406519in}{3.136852in}}%
\pgfpathlineto{\pgfqpoint{3.409730in}{3.135924in}}%
\pgfpathlineto{\pgfqpoint{3.413476in}{3.132245in}}%
\pgfpathlineto{\pgfqpoint{3.421772in}{3.119622in}}%
\pgfpathlineto{\pgfqpoint{3.426857in}{3.114333in}}%
\pgfpathlineto{\pgfqpoint{3.430336in}{3.113634in}}%
\pgfpathlineto{\pgfqpoint{3.433547in}{3.115332in}}%
\pgfpathlineto{\pgfqpoint{3.437829in}{3.120470in}}%
\pgfpathlineto{\pgfqpoint{3.449603in}{3.136116in}}%
\pgfpathlineto{\pgfqpoint{3.453082in}{3.136775in}}%
\pgfpathlineto{\pgfqpoint{3.456294in}{3.135042in}}%
\pgfpathlineto{\pgfqpoint{3.460575in}{3.129873in}}%
\pgfpathlineto{\pgfqpoint{3.472350in}{3.114274in}}%
\pgfpathlineto{\pgfqpoint{3.475829in}{3.113654in}}%
\pgfpathlineto{\pgfqpoint{3.479040in}{3.115421in}}%
\pgfpathlineto{\pgfqpoint{3.483322in}{3.120622in}}%
\pgfpathlineto{\pgfqpoint{3.494829in}{3.136019in}}%
\pgfpathlineto{\pgfqpoint{3.498308in}{3.136804in}}%
\pgfpathlineto{\pgfqpoint{3.501519in}{3.135182in}}%
\pgfpathlineto{\pgfqpoint{3.505801in}{3.130115in}}%
\pgfpathlineto{\pgfqpoint{3.517843in}{3.114218in}}%
\pgfpathlineto{\pgfqpoint{3.521322in}{3.113677in}}%
\pgfpathlineto{\pgfqpoint{3.524534in}{3.115512in}}%
\pgfpathlineto{\pgfqpoint{3.529083in}{3.121177in}}%
\pgfpathlineto{\pgfqpoint{3.540055in}{3.135917in}}%
\pgfpathlineto{\pgfqpoint{3.543534in}{3.136828in}}%
\pgfpathlineto{\pgfqpoint{3.546745in}{3.135317in}}%
\pgfpathlineto{\pgfqpoint{3.551027in}{3.130355in}}%
\pgfpathlineto{\pgfqpoint{3.563337in}{3.114164in}}%
\pgfpathlineto{\pgfqpoint{3.566816in}{3.113702in}}%
\pgfpathlineto{\pgfqpoint{3.570027in}{3.115606in}}%
\pgfpathlineto{\pgfqpoint{3.574576in}{3.121334in}}%
\pgfpathlineto{\pgfqpoint{3.585281in}{3.135810in}}%
\pgfpathlineto{\pgfqpoint{3.588760in}{3.136845in}}%
\pgfpathlineto{\pgfqpoint{3.591971in}{3.135446in}}%
\pgfpathlineto{\pgfqpoint{3.595985in}{3.130971in}}%
\pgfpathlineto{\pgfqpoint{3.609365in}{3.113880in}}%
\pgfpathlineto{\pgfqpoint{3.612577in}{3.113810in}}%
\pgfpathlineto{\pgfqpoint{3.616055in}{3.116225in}}%
\pgfpathlineto{\pgfqpoint{3.621140in}{3.123162in}}%
\pgfpathlineto{\pgfqpoint{3.629971in}{3.135292in}}%
\pgfpathlineto{\pgfqpoint{3.633718in}{3.136861in}}%
\pgfpathlineto{\pgfqpoint{3.636929in}{3.135761in}}%
\pgfpathlineto{\pgfqpoint{3.640943in}{3.131566in}}%
\pgfpathlineto{\pgfqpoint{3.655126in}{3.113755in}}%
\pgfpathlineto{\pgfqpoint{3.658338in}{3.113949in}}%
\pgfpathlineto{\pgfqpoint{3.661816in}{3.116617in}}%
\pgfpathlineto{\pgfqpoint{3.667169in}{3.124181in}}%
\pgfpathlineto{\pgfqpoint{3.675197in}{3.135156in}}%
\pgfpathlineto{\pgfqpoint{3.678943in}{3.136856in}}%
\pgfpathlineto{\pgfqpoint{3.682155in}{3.135871in}}%
\pgfpathlineto{\pgfqpoint{3.685901in}{3.132139in}}%
\pgfpathlineto{\pgfqpoint{3.701155in}{3.113611in}}%
\pgfpathlineto{\pgfqpoint{3.704366in}{3.114257in}}%
\pgfpathlineto{\pgfqpoint{3.708113in}{3.117669in}}%
\pgfpathlineto{\pgfqpoint{3.714803in}{3.127778in}}%
\pgfpathlineto{\pgfqpoint{3.720958in}{3.135454in}}%
\pgfpathlineto{\pgfqpoint{3.724704in}{3.136859in}}%
\pgfpathlineto{\pgfqpoint{3.727916in}{3.135616in}}%
\pgfpathlineto{\pgfqpoint{3.731930in}{3.131285in}}%
\pgfpathlineto{\pgfqpoint{3.745845in}{3.113774in}}%
\pgfpathlineto{\pgfqpoint{3.749057in}{3.113924in}}%
\pgfpathlineto{\pgfqpoint{3.752535in}{3.116550in}}%
\pgfpathlineto{\pgfqpoint{3.757888in}{3.124082in}}%
\pgfpathlineto{\pgfqpoint{3.766183in}{3.135325in}}%
\pgfpathlineto{\pgfqpoint{3.769930in}{3.136861in}}%
\pgfpathlineto{\pgfqpoint{3.773141in}{3.135733in}}%
\pgfpathlineto{\pgfqpoint{3.777155in}{3.131510in}}%
\pgfpathlineto{\pgfqpoint{3.791339in}{3.113743in}}%
\pgfpathlineto{\pgfqpoint{3.794550in}{3.113966in}}%
\pgfpathlineto{\pgfqpoint{3.798029in}{3.116662in}}%
\pgfpathlineto{\pgfqpoint{3.803381in}{3.124248in}}%
\pgfpathlineto{\pgfqpoint{3.811409in}{3.135191in}}%
\pgfpathlineto{\pgfqpoint{3.815156in}{3.136858in}}%
\pgfpathlineto{\pgfqpoint{3.818367in}{3.135844in}}%
\pgfpathlineto{\pgfqpoint{3.822114in}{3.132085in}}%
\pgfpathlineto{\pgfqpoint{3.837367in}{3.113605in}}%
\pgfpathlineto{\pgfqpoint{3.840578in}{3.114280in}}%
\pgfpathlineto{\pgfqpoint{3.844325in}{3.117720in}}%
\pgfpathlineto{\pgfqpoint{3.851015in}{3.127843in}}%
\pgfpathlineto{\pgfqpoint{3.857170in}{3.135486in}}%
\pgfpathlineto{\pgfqpoint{3.860917in}{3.136857in}}%
\pgfpathlineto{\pgfqpoint{3.864128in}{3.135586in}}%
\pgfpathlineto{\pgfqpoint{3.868142in}{3.131228in}}%
\pgfpathlineto{\pgfqpoint{3.881790in}{3.113850in}}%
\pgfpathlineto{\pgfqpoint{3.885001in}{3.113839in}}%
\pgfpathlineto{\pgfqpoint{3.888480in}{3.116310in}}%
\pgfpathlineto{\pgfqpoint{3.893565in}{3.123292in}}%
\pgfpathlineto{\pgfqpoint{3.902396in}{3.135358in}}%
\pgfpathlineto{\pgfqpoint{3.906142in}{3.136861in}}%
\pgfpathlineto{\pgfqpoint{3.909354in}{3.135704in}}%
\pgfpathlineto{\pgfqpoint{3.913368in}{3.131454in}}%
\pgfpathlineto{\pgfqpoint{3.927551in}{3.113732in}}%
\pgfpathlineto{\pgfqpoint{3.930762in}{3.113984in}}%
\pgfpathlineto{\pgfqpoint{3.934241in}{3.116708in}}%
\pgfpathlineto{\pgfqpoint{3.939593in}{3.124314in}}%
\pgfpathlineto{\pgfqpoint{3.947622in}{3.135225in}}%
\pgfpathlineto{\pgfqpoint{3.951368in}{3.136860in}}%
\pgfpathlineto{\pgfqpoint{3.954579in}{3.135817in}}%
\pgfpathlineto{\pgfqpoint{3.958594in}{3.131677in}}%
\pgfpathlineto{\pgfqpoint{3.973044in}{3.113704in}}%
\pgfpathlineto{\pgfqpoint{3.976256in}{3.114030in}}%
\pgfpathlineto{\pgfqpoint{3.979735in}{3.116822in}}%
\pgfpathlineto{\pgfqpoint{3.985354in}{3.124911in}}%
\pgfpathlineto{\pgfqpoint{3.992847in}{3.135087in}}%
\pgfpathlineto{\pgfqpoint{3.996594in}{3.136852in}}%
\pgfpathlineto{\pgfqpoint{3.999805in}{3.135924in}}%
\pgfpathlineto{\pgfqpoint{4.003552in}{3.132245in}}%
\pgfpathlineto{\pgfqpoint{4.011847in}{3.119622in}}%
\pgfpathlineto{\pgfqpoint{4.016932in}{3.114333in}}%
\pgfpathlineto{\pgfqpoint{4.020411in}{3.113634in}}%
\pgfpathlineto{\pgfqpoint{4.023622in}{3.115332in}}%
\pgfpathlineto{\pgfqpoint{4.027904in}{3.120470in}}%
\pgfpathlineto{\pgfqpoint{4.039679in}{3.136116in}}%
\pgfpathlineto{\pgfqpoint{4.043158in}{3.136775in}}%
\pgfpathlineto{\pgfqpoint{4.046369in}{3.135042in}}%
\pgfpathlineto{\pgfqpoint{4.050651in}{3.129873in}}%
\pgfpathlineto{\pgfqpoint{4.062425in}{3.114274in}}%
\pgfpathlineto{\pgfqpoint{4.065904in}{3.113654in}}%
\pgfpathlineto{\pgfqpoint{4.069116in}{3.115421in}}%
\pgfpathlineto{\pgfqpoint{4.073397in}{3.120622in}}%
\pgfpathlineto{\pgfqpoint{4.084904in}{3.136019in}}%
\pgfpathlineto{\pgfqpoint{4.088383in}{3.136804in}}%
\pgfpathlineto{\pgfqpoint{4.091595in}{3.135182in}}%
\pgfpathlineto{\pgfqpoint{4.095876in}{3.130115in}}%
\pgfpathlineto{\pgfqpoint{4.107919in}{3.114218in}}%
\pgfpathlineto{\pgfqpoint{4.111398in}{3.113677in}}%
\pgfpathlineto{\pgfqpoint{4.114609in}{3.115512in}}%
\pgfpathlineto{\pgfqpoint{4.119158in}{3.121177in}}%
\pgfpathlineto{\pgfqpoint{4.130130in}{3.135917in}}%
\pgfpathlineto{\pgfqpoint{4.133609in}{3.136828in}}%
\pgfpathlineto{\pgfqpoint{4.136820in}{3.135317in}}%
\pgfpathlineto{\pgfqpoint{4.141102in}{3.130355in}}%
\pgfpathlineto{\pgfqpoint{4.153412in}{3.114164in}}%
\pgfpathlineto{\pgfqpoint{4.156891in}{3.113702in}}%
\pgfpathlineto{\pgfqpoint{4.160102in}{3.115606in}}%
\pgfpathlineto{\pgfqpoint{4.164652in}{3.121334in}}%
\pgfpathlineto{\pgfqpoint{4.175356in}{3.135810in}}%
\pgfpathlineto{\pgfqpoint{4.178835in}{3.136845in}}%
\pgfpathlineto{\pgfqpoint{4.182046in}{3.135446in}}%
\pgfpathlineto{\pgfqpoint{4.186060in}{3.130971in}}%
\pgfpathlineto{\pgfqpoint{4.199441in}{3.113880in}}%
\pgfpathlineto{\pgfqpoint{4.202652in}{3.113810in}}%
\pgfpathlineto{\pgfqpoint{4.206131in}{3.116225in}}%
\pgfpathlineto{\pgfqpoint{4.211215in}{3.123162in}}%
\pgfpathlineto{\pgfqpoint{4.220046in}{3.135292in}}%
\pgfpathlineto{\pgfqpoint{4.223793in}{3.136861in}}%
\pgfpathlineto{\pgfqpoint{4.227004in}{3.135761in}}%
\pgfpathlineto{\pgfqpoint{4.231018in}{3.131566in}}%
\pgfpathlineto{\pgfqpoint{4.245201in}{3.113755in}}%
\pgfpathlineto{\pgfqpoint{4.248413in}{3.113949in}}%
\pgfpathlineto{\pgfqpoint{4.251892in}{3.116617in}}%
\pgfpathlineto{\pgfqpoint{4.257244in}{3.124181in}}%
\pgfpathlineto{\pgfqpoint{4.265272in}{3.135156in}}%
\pgfpathlineto{\pgfqpoint{4.269019in}{3.136856in}}%
\pgfpathlineto{\pgfqpoint{4.272230in}{3.135871in}}%
\pgfpathlineto{\pgfqpoint{4.275976in}{3.132139in}}%
\pgfpathlineto{\pgfqpoint{4.291230in}{3.113611in}}%
\pgfpathlineto{\pgfqpoint{4.294441in}{3.114257in}}%
\pgfpathlineto{\pgfqpoint{4.298188in}{3.117669in}}%
\pgfpathlineto{\pgfqpoint{4.304878in}{3.127778in}}%
\pgfpathlineto{\pgfqpoint{4.311033in}{3.135454in}}%
\pgfpathlineto{\pgfqpoint{4.314780in}{3.136859in}}%
\pgfpathlineto{\pgfqpoint{4.317991in}{3.135616in}}%
\pgfpathlineto{\pgfqpoint{4.322005in}{3.131285in}}%
\pgfpathlineto{\pgfqpoint{4.335921in}{3.113774in}}%
\pgfpathlineto{\pgfqpoint{4.339132in}{3.113924in}}%
\pgfpathlineto{\pgfqpoint{4.342611in}{3.116550in}}%
\pgfpathlineto{\pgfqpoint{4.347963in}{3.124082in}}%
\pgfpathlineto{\pgfqpoint{4.356259in}{3.135325in}}%
\pgfpathlineto{\pgfqpoint{4.360005in}{3.136861in}}%
\pgfpathlineto{\pgfqpoint{4.363217in}{3.135733in}}%
\pgfpathlineto{\pgfqpoint{4.367231in}{3.131510in}}%
\pgfpathlineto{\pgfqpoint{4.381414in}{3.113743in}}%
\pgfpathlineto{\pgfqpoint{4.384625in}{3.113966in}}%
\pgfpathlineto{\pgfqpoint{4.388104in}{3.116662in}}%
\pgfpathlineto{\pgfqpoint{4.393456in}{3.124248in}}%
\pgfpathlineto{\pgfqpoint{4.401484in}{3.135191in}}%
\pgfpathlineto{\pgfqpoint{4.405231in}{3.136858in}}%
\pgfpathlineto{\pgfqpoint{4.408442in}{3.135844in}}%
\pgfpathlineto{\pgfqpoint{4.412189in}{3.132085in}}%
\pgfpathlineto{\pgfqpoint{4.427442in}{3.113605in}}%
\pgfpathlineto{\pgfqpoint{4.430654in}{3.114280in}}%
\pgfpathlineto{\pgfqpoint{4.434400in}{3.117720in}}%
\pgfpathlineto{\pgfqpoint{4.441090in}{3.127843in}}%
\pgfpathlineto{\pgfqpoint{4.447245in}{3.135486in}}%
\pgfpathlineto{\pgfqpoint{4.450992in}{3.136857in}}%
\pgfpathlineto{\pgfqpoint{4.454203in}{3.135586in}}%
\pgfpathlineto{\pgfqpoint{4.458217in}{3.131228in}}%
\pgfpathlineto{\pgfqpoint{4.471865in}{3.113850in}}%
\pgfpathlineto{\pgfqpoint{4.475077in}{3.113839in}}%
\pgfpathlineto{\pgfqpoint{4.478556in}{3.116310in}}%
\pgfpathlineto{\pgfqpoint{4.483640in}{3.123292in}}%
\pgfpathlineto{\pgfqpoint{4.492471in}{3.135358in}}%
\pgfpathlineto{\pgfqpoint{4.496218in}{3.136861in}}%
\pgfpathlineto{\pgfqpoint{4.499429in}{3.135704in}}%
\pgfpathlineto{\pgfqpoint{4.503443in}{3.131454in}}%
\pgfpathlineto{\pgfqpoint{4.517626in}{3.113732in}}%
\pgfpathlineto{\pgfqpoint{4.520838in}{3.113984in}}%
\pgfpathlineto{\pgfqpoint{4.524316in}{3.116708in}}%
\pgfpathlineto{\pgfqpoint{4.529669in}{3.124314in}}%
\pgfpathlineto{\pgfqpoint{4.537697in}{3.135225in}}%
\pgfpathlineto{\pgfqpoint{4.541443in}{3.136860in}}%
\pgfpathlineto{\pgfqpoint{4.544655in}{3.135817in}}%
\pgfpathlineto{\pgfqpoint{4.548669in}{3.131677in}}%
\pgfpathlineto{\pgfqpoint{4.563120in}{3.113704in}}%
\pgfpathlineto{\pgfqpoint{4.566331in}{3.114030in}}%
\pgfpathlineto{\pgfqpoint{4.569810in}{3.116822in}}%
\pgfpathlineto{\pgfqpoint{4.575430in}{3.124911in}}%
\pgfpathlineto{\pgfqpoint{4.582923in}{3.135087in}}%
\pgfpathlineto{\pgfqpoint{4.586669in}{3.136852in}}%
\pgfpathlineto{\pgfqpoint{4.589880in}{3.135924in}}%
\pgfpathlineto{\pgfqpoint{4.593627in}{3.132245in}}%
\pgfpathlineto{\pgfqpoint{4.601923in}{3.119622in}}%
\pgfpathlineto{\pgfqpoint{4.607007in}{3.114333in}}%
\pgfpathlineto{\pgfqpoint{4.610486in}{3.113634in}}%
\pgfpathlineto{\pgfqpoint{4.613697in}{3.115332in}}%
\pgfpathlineto{\pgfqpoint{4.617979in}{3.120470in}}%
\pgfpathlineto{\pgfqpoint{4.629754in}{3.136116in}}%
\pgfpathlineto{\pgfqpoint{4.633233in}{3.136775in}}%
\pgfpathlineto{\pgfqpoint{4.636444in}{3.135042in}}%
\pgfpathlineto{\pgfqpoint{4.640726in}{3.129873in}}%
\pgfpathlineto{\pgfqpoint{4.652501in}{3.114274in}}%
\pgfpathlineto{\pgfqpoint{4.655979in}{3.113654in}}%
\pgfpathlineto{\pgfqpoint{4.659191in}{3.115421in}}%
\pgfpathlineto{\pgfqpoint{4.663473in}{3.120622in}}%
\pgfpathlineto{\pgfqpoint{4.674980in}{3.136019in}}%
\pgfpathlineto{\pgfqpoint{4.678459in}{3.136804in}}%
\pgfpathlineto{\pgfqpoint{4.681670in}{3.135182in}}%
\pgfpathlineto{\pgfqpoint{4.685952in}{3.130115in}}%
\pgfpathlineto{\pgfqpoint{4.697994in}{3.114218in}}%
\pgfpathlineto{\pgfqpoint{4.701473in}{3.113677in}}%
\pgfpathlineto{\pgfqpoint{4.704684in}{3.115512in}}%
\pgfpathlineto{\pgfqpoint{4.709233in}{3.121177in}}%
\pgfpathlineto{\pgfqpoint{4.720205in}{3.135917in}}%
\pgfpathlineto{\pgfqpoint{4.723684in}{3.136828in}}%
\pgfpathlineto{\pgfqpoint{4.726896in}{3.135317in}}%
\pgfpathlineto{\pgfqpoint{4.731177in}{3.130355in}}%
\pgfpathlineto{\pgfqpoint{4.743487in}{3.114164in}}%
\pgfpathlineto{\pgfqpoint{4.746966in}{3.113702in}}%
\pgfpathlineto{\pgfqpoint{4.750177in}{3.115606in}}%
\pgfpathlineto{\pgfqpoint{4.754727in}{3.121334in}}%
\pgfpathlineto{\pgfqpoint{4.765431in}{3.135810in}}%
\pgfpathlineto{\pgfqpoint{4.768910in}{3.136845in}}%
\pgfpathlineto{\pgfqpoint{4.772121in}{3.135446in}}%
\pgfpathlineto{\pgfqpoint{4.776135in}{3.130971in}}%
\pgfpathlineto{\pgfqpoint{4.789516in}{3.113880in}}%
\pgfpathlineto{\pgfqpoint{4.792727in}{3.113810in}}%
\pgfpathlineto{\pgfqpoint{4.796206in}{3.116225in}}%
\pgfpathlineto{\pgfqpoint{4.801291in}{3.123162in}}%
\pgfpathlineto{\pgfqpoint{4.810122in}{3.135292in}}%
\pgfpathlineto{\pgfqpoint{4.813868in}{3.136861in}}%
\pgfpathlineto{\pgfqpoint{4.817079in}{3.135761in}}%
\pgfpathlineto{\pgfqpoint{4.821094in}{3.131566in}}%
\pgfpathlineto{\pgfqpoint{4.835277in}{3.113755in}}%
\pgfpathlineto{\pgfqpoint{4.838488in}{3.113949in}}%
\pgfpathlineto{\pgfqpoint{4.841967in}{3.116617in}}%
\pgfpathlineto{\pgfqpoint{4.847319in}{3.124181in}}%
\pgfpathlineto{\pgfqpoint{4.855347in}{3.135156in}}%
\pgfpathlineto{\pgfqpoint{4.859094in}{3.136856in}}%
\pgfpathlineto{\pgfqpoint{4.862305in}{3.135871in}}%
\pgfpathlineto{\pgfqpoint{4.866052in}{3.132139in}}%
\pgfpathlineto{\pgfqpoint{4.881305in}{3.113611in}}%
\pgfpathlineto{\pgfqpoint{4.884517in}{3.114257in}}%
\pgfpathlineto{\pgfqpoint{4.888263in}{3.117669in}}%
\pgfpathlineto{\pgfqpoint{4.894953in}{3.127778in}}%
\pgfpathlineto{\pgfqpoint{4.901108in}{3.135454in}}%
\pgfpathlineto{\pgfqpoint{4.904855in}{3.136859in}}%
\pgfpathlineto{\pgfqpoint{4.908066in}{3.135616in}}%
\pgfpathlineto{\pgfqpoint{4.912080in}{3.131285in}}%
\pgfpathlineto{\pgfqpoint{4.925996in}{3.113774in}}%
\pgfpathlineto{\pgfqpoint{4.929207in}{3.113924in}}%
\pgfpathlineto{\pgfqpoint{4.932686in}{3.116550in}}%
\pgfpathlineto{\pgfqpoint{4.938038in}{3.124082in}}%
\pgfpathlineto{\pgfqpoint{4.946334in}{3.135325in}}%
\pgfpathlineto{\pgfqpoint{4.950080in}{3.136861in}}%
\pgfpathlineto{\pgfqpoint{4.953292in}{3.135733in}}%
\pgfpathlineto{\pgfqpoint{4.957306in}{3.131510in}}%
\pgfpathlineto{\pgfqpoint{4.971489in}{3.113743in}}%
\pgfpathlineto{\pgfqpoint{4.974700in}{3.113966in}}%
\pgfpathlineto{\pgfqpoint{4.978179in}{3.116662in}}%
\pgfpathlineto{\pgfqpoint{4.983531in}{3.124248in}}%
\pgfpathlineto{\pgfqpoint{4.991560in}{3.135191in}}%
\pgfpathlineto{\pgfqpoint{4.995306in}{3.136858in}}%
\pgfpathlineto{\pgfqpoint{4.998518in}{3.135844in}}%
\pgfpathlineto{\pgfqpoint{5.002264in}{3.132085in}}%
\pgfpathlineto{\pgfqpoint{5.017518in}{3.113605in}}%
\pgfpathlineto{\pgfqpoint{5.020729in}{3.114280in}}%
\pgfpathlineto{\pgfqpoint{5.024475in}{3.117720in}}%
\pgfpathlineto{\pgfqpoint{5.031166in}{3.127843in}}%
\pgfpathlineto{\pgfqpoint{5.037321in}{3.135486in}}%
\pgfpathlineto{\pgfqpoint{5.041067in}{3.136857in}}%
\pgfpathlineto{\pgfqpoint{5.044278in}{3.135586in}}%
\pgfpathlineto{\pgfqpoint{5.048293in}{3.131228in}}%
\pgfpathlineto{\pgfqpoint{5.061941in}{3.113850in}}%
\pgfpathlineto{\pgfqpoint{5.065152in}{3.113839in}}%
\pgfpathlineto{\pgfqpoint{5.068631in}{3.116310in}}%
\pgfpathlineto{\pgfqpoint{5.073715in}{3.123292in}}%
\pgfpathlineto{\pgfqpoint{5.082546in}{3.135358in}}%
\pgfpathlineto{\pgfqpoint{5.086293in}{3.136861in}}%
\pgfpathlineto{\pgfqpoint{5.089504in}{3.135704in}}%
\pgfpathlineto{\pgfqpoint{5.093518in}{3.131454in}}%
\pgfpathlineto{\pgfqpoint{5.107701in}{3.113732in}}%
\pgfpathlineto{\pgfqpoint{5.110913in}{3.113984in}}%
\pgfpathlineto{\pgfqpoint{5.114392in}{3.116708in}}%
\pgfpathlineto{\pgfqpoint{5.119744in}{3.124314in}}%
\pgfpathlineto{\pgfqpoint{5.127772in}{3.135225in}}%
\pgfpathlineto{\pgfqpoint{5.131519in}{3.136860in}}%
\pgfpathlineto{\pgfqpoint{5.134730in}{3.135817in}}%
\pgfpathlineto{\pgfqpoint{5.138744in}{3.131677in}}%
\pgfpathlineto{\pgfqpoint{5.153195in}{3.113704in}}%
\pgfpathlineto{\pgfqpoint{5.156406in}{3.114030in}}%
\pgfpathlineto{\pgfqpoint{5.159885in}{3.116822in}}%
\pgfpathlineto{\pgfqpoint{5.165505in}{3.124911in}}%
\pgfpathlineto{\pgfqpoint{5.172998in}{3.135087in}}%
\pgfpathlineto{\pgfqpoint{5.176744in}{3.136852in}}%
\pgfpathlineto{\pgfqpoint{5.179956in}{3.135924in}}%
\pgfpathlineto{\pgfqpoint{5.183702in}{3.132245in}}%
\pgfpathlineto{\pgfqpoint{5.191998in}{3.119622in}}%
\pgfpathlineto{\pgfqpoint{5.197083in}{3.114333in}}%
\pgfpathlineto{\pgfqpoint{5.200561in}{3.113634in}}%
\pgfpathlineto{\pgfqpoint{5.203773in}{3.115332in}}%
\pgfpathlineto{\pgfqpoint{5.208054in}{3.120470in}}%
\pgfpathlineto{\pgfqpoint{5.219829in}{3.136116in}}%
\pgfpathlineto{\pgfqpoint{5.223308in}{3.136775in}}%
\pgfpathlineto{\pgfqpoint{5.226519in}{3.135042in}}%
\pgfpathlineto{\pgfqpoint{5.230801in}{3.129873in}}%
\pgfpathlineto{\pgfqpoint{5.242576in}{3.114274in}}%
\pgfpathlineto{\pgfqpoint{5.246055in}{3.113654in}}%
\pgfpathlineto{\pgfqpoint{5.249266in}{3.115421in}}%
\pgfpathlineto{\pgfqpoint{5.253548in}{3.120622in}}%
\pgfpathlineto{\pgfqpoint{5.265055in}{3.136019in}}%
\pgfpathlineto{\pgfqpoint{5.268534in}{3.136804in}}%
\pgfpathlineto{\pgfqpoint{5.271745in}{3.135182in}}%
\pgfpathlineto{\pgfqpoint{5.276027in}{3.130115in}}%
\pgfpathlineto{\pgfqpoint{5.288069in}{3.114218in}}%
\pgfpathlineto{\pgfqpoint{5.291548in}{3.113677in}}%
\pgfpathlineto{\pgfqpoint{5.294759in}{3.115512in}}%
\pgfpathlineto{\pgfqpoint{5.299309in}{3.121177in}}%
\pgfpathlineto{\pgfqpoint{5.310281in}{3.135917in}}%
\pgfpathlineto{\pgfqpoint{5.313760in}{3.136828in}}%
\pgfpathlineto{\pgfqpoint{5.316971in}{3.135317in}}%
\pgfpathlineto{\pgfqpoint{5.321253in}{3.130355in}}%
\pgfpathlineto{\pgfqpoint{5.333562in}{3.114164in}}%
\pgfpathlineto{\pgfqpoint{5.337041in}{3.113702in}}%
\pgfpathlineto{\pgfqpoint{5.340253in}{3.115606in}}%
\pgfpathlineto{\pgfqpoint{5.344802in}{3.121334in}}%
\pgfpathlineto{\pgfqpoint{5.355506in}{3.135810in}}%
\pgfpathlineto{\pgfqpoint{5.358985in}{3.136845in}}%
\pgfpathlineto{\pgfqpoint{5.362197in}{3.135446in}}%
\pgfpathlineto{\pgfqpoint{5.366211in}{3.130971in}}%
\pgfpathlineto{\pgfqpoint{5.379591in}{3.113880in}}%
\pgfpathlineto{\pgfqpoint{5.382802in}{3.113810in}}%
\pgfpathlineto{\pgfqpoint{5.386281in}{3.116225in}}%
\pgfpathlineto{\pgfqpoint{5.391366in}{3.123162in}}%
\pgfpathlineto{\pgfqpoint{5.400197in}{3.135292in}}%
\pgfpathlineto{\pgfqpoint{5.403943in}{3.136861in}}%
\pgfpathlineto{\pgfqpoint{5.407155in}{3.135761in}}%
\pgfpathlineto{\pgfqpoint{5.411169in}{3.131566in}}%
\pgfpathlineto{\pgfqpoint{5.425352in}{3.113755in}}%
\pgfpathlineto{\pgfqpoint{5.428563in}{3.113949in}}%
\pgfpathlineto{\pgfqpoint{5.432042in}{3.116617in}}%
\pgfpathlineto{\pgfqpoint{5.437394in}{3.124181in}}%
\pgfpathlineto{\pgfqpoint{5.445423in}{3.135156in}}%
\pgfpathlineto{\pgfqpoint{5.449169in}{3.136856in}}%
\pgfpathlineto{\pgfqpoint{5.452380in}{3.135871in}}%
\pgfpathlineto{\pgfqpoint{5.456127in}{3.132139in}}%
\pgfpathlineto{\pgfqpoint{5.471381in}{3.113611in}}%
\pgfpathlineto{\pgfqpoint{5.474592in}{3.114257in}}%
\pgfpathlineto{\pgfqpoint{5.478338in}{3.117669in}}%
\pgfpathlineto{\pgfqpoint{5.485029in}{3.127778in}}%
\pgfpathlineto{\pgfqpoint{5.491184in}{3.135454in}}%
\pgfpathlineto{\pgfqpoint{5.494930in}{3.136859in}}%
\pgfpathlineto{\pgfqpoint{5.498141in}{3.135616in}}%
\pgfpathlineto{\pgfqpoint{5.502155in}{3.131285in}}%
\pgfpathlineto{\pgfqpoint{5.516071in}{3.113774in}}%
\pgfpathlineto{\pgfqpoint{5.519282in}{3.113924in}}%
\pgfpathlineto{\pgfqpoint{5.522761in}{3.116550in}}%
\pgfpathlineto{\pgfqpoint{5.528113in}{3.124082in}}%
\pgfpathlineto{\pgfqpoint{5.536409in}{3.135325in}}%
\pgfpathlineto{\pgfqpoint{5.540156in}{3.136861in}}%
\pgfpathlineto{\pgfqpoint{5.543367in}{3.135733in}}%
\pgfpathlineto{\pgfqpoint{5.547381in}{3.131510in}}%
\pgfpathlineto{\pgfqpoint{5.561564in}{3.113743in}}%
\pgfpathlineto{\pgfqpoint{5.564776in}{3.113966in}}%
\pgfpathlineto{\pgfqpoint{5.568255in}{3.116662in}}%
\pgfpathlineto{\pgfqpoint{5.573607in}{3.124248in}}%
\pgfpathlineto{\pgfqpoint{5.581635in}{3.135191in}}%
\pgfpathlineto{\pgfqpoint{5.585381in}{3.136858in}}%
\pgfpathlineto{\pgfqpoint{5.588593in}{3.135844in}}%
\pgfpathlineto{\pgfqpoint{5.592339in}{3.132085in}}%
\pgfpathlineto{\pgfqpoint{5.607593in}{3.113605in}}%
\pgfpathlineto{\pgfqpoint{5.610804in}{3.114280in}}%
\pgfpathlineto{\pgfqpoint{5.614551in}{3.117720in}}%
\pgfpathlineto{\pgfqpoint{5.621241in}{3.127843in}}%
\pgfpathlineto{\pgfqpoint{5.627396in}{3.135486in}}%
\pgfpathlineto{\pgfqpoint{5.631142in}{3.136857in}}%
\pgfpathlineto{\pgfqpoint{5.634354in}{3.135586in}}%
\pgfpathlineto{\pgfqpoint{5.638368in}{3.131228in}}%
\pgfpathlineto{\pgfqpoint{5.652016in}{3.113850in}}%
\pgfpathlineto{\pgfqpoint{5.655227in}{3.113839in}}%
\pgfpathlineto{\pgfqpoint{5.658706in}{3.116310in}}%
\pgfpathlineto{\pgfqpoint{5.663791in}{3.123292in}}%
\pgfpathlineto{\pgfqpoint{5.672622in}{3.135358in}}%
\pgfpathlineto{\pgfqpoint{5.676368in}{3.136861in}}%
\pgfpathlineto{\pgfqpoint{5.679579in}{3.135704in}}%
\pgfpathlineto{\pgfqpoint{5.683594in}{3.131454in}}%
\pgfpathlineto{\pgfqpoint{5.697777in}{3.113732in}}%
\pgfpathlineto{\pgfqpoint{5.700988in}{3.113984in}}%
\pgfpathlineto{\pgfqpoint{5.704467in}{3.116708in}}%
\pgfpathlineto{\pgfqpoint{5.709819in}{3.124314in}}%
\pgfpathlineto{\pgfqpoint{5.717847in}{3.135225in}}%
\pgfpathlineto{\pgfqpoint{5.721594in}{3.136860in}}%
\pgfpathlineto{\pgfqpoint{5.724805in}{3.135817in}}%
\pgfpathlineto{\pgfqpoint{5.728819in}{3.131677in}}%
\pgfpathlineto{\pgfqpoint{5.743270in}{3.113704in}}%
\pgfpathlineto{\pgfqpoint{5.746481in}{3.114030in}}%
\pgfpathlineto{\pgfqpoint{5.749960in}{3.116822in}}%
\pgfpathlineto{\pgfqpoint{5.755580in}{3.124911in}}%
\pgfpathlineto{\pgfqpoint{5.763073in}{3.135087in}}%
\pgfpathlineto{\pgfqpoint{5.766820in}{3.136852in}}%
\pgfpathlineto{\pgfqpoint{5.770031in}{3.135924in}}%
\pgfpathlineto{\pgfqpoint{5.773777in}{3.132245in}}%
\pgfpathlineto{\pgfqpoint{5.782073in}{3.119622in}}%
\pgfpathlineto{\pgfqpoint{5.787158in}{3.114333in}}%
\pgfpathlineto{\pgfqpoint{5.790637in}{3.113634in}}%
\pgfpathlineto{\pgfqpoint{5.793848in}{3.115332in}}%
\pgfpathlineto{\pgfqpoint{5.798130in}{3.120470in}}%
\pgfpathlineto{\pgfqpoint{5.809904in}{3.136116in}}%
\pgfpathlineto{\pgfqpoint{5.813383in}{3.136775in}}%
\pgfpathlineto{\pgfqpoint{5.816595in}{3.135042in}}%
\pgfpathlineto{\pgfqpoint{5.820876in}{3.129873in}}%
\pgfpathlineto{\pgfqpoint{5.832651in}{3.114274in}}%
\pgfpathlineto{\pgfqpoint{5.836130in}{3.113654in}}%
\pgfpathlineto{\pgfqpoint{5.839341in}{3.115421in}}%
\pgfpathlineto{\pgfqpoint{5.843623in}{3.120622in}}%
\pgfpathlineto{\pgfqpoint{5.855130in}{3.136019in}}%
\pgfpathlineto{\pgfqpoint{5.858609in}{3.136804in}}%
\pgfpathlineto{\pgfqpoint{5.861820in}{3.135182in}}%
\pgfpathlineto{\pgfqpoint{5.866102in}{3.130115in}}%
\pgfpathlineto{\pgfqpoint{5.878144in}{3.114218in}}%
\pgfpathlineto{\pgfqpoint{5.881623in}{3.113677in}}%
\pgfpathlineto{\pgfqpoint{5.884835in}{3.115512in}}%
\pgfpathlineto{\pgfqpoint{5.889384in}{3.121177in}}%
\pgfpathlineto{\pgfqpoint{5.900356in}{3.135917in}}%
\pgfpathlineto{\pgfqpoint{5.903835in}{3.136828in}}%
\pgfpathlineto{\pgfqpoint{5.907046in}{3.135317in}}%
\pgfpathlineto{\pgfqpoint{5.911328in}{3.130355in}}%
\pgfpathlineto{\pgfqpoint{5.923638in}{3.114164in}}%
\pgfpathlineto{\pgfqpoint{5.927117in}{3.113702in}}%
\pgfpathlineto{\pgfqpoint{5.930328in}{3.115606in}}%
\pgfpathlineto{\pgfqpoint{5.934877in}{3.121334in}}%
\pgfpathlineto{\pgfqpoint{5.945582in}{3.135810in}}%
\pgfpathlineto{\pgfqpoint{5.949060in}{3.136845in}}%
\pgfpathlineto{\pgfqpoint{5.952272in}{3.135446in}}%
\pgfpathlineto{\pgfqpoint{5.956286in}{3.130971in}}%
\pgfpathlineto{\pgfqpoint{5.969666in}{3.113880in}}%
\pgfpathlineto{\pgfqpoint{5.972878in}{3.113810in}}%
\pgfpathlineto{\pgfqpoint{5.976356in}{3.116225in}}%
\pgfpathlineto{\pgfqpoint{5.981441in}{3.123162in}}%
\pgfpathlineto{\pgfqpoint{5.990272in}{3.135292in}}%
\pgfpathlineto{\pgfqpoint{5.994019in}{3.136861in}}%
\pgfpathlineto{\pgfqpoint{5.997230in}{3.135761in}}%
\pgfpathlineto{\pgfqpoint{6.001244in}{3.131566in}}%
\pgfpathlineto{\pgfqpoint{6.015427in}{3.113755in}}%
\pgfpathlineto{\pgfqpoint{6.018639in}{3.113949in}}%
\pgfpathlineto{\pgfqpoint{6.022117in}{3.116617in}}%
\pgfpathlineto{\pgfqpoint{6.027470in}{3.124181in}}%
\pgfpathlineto{\pgfqpoint{6.035498in}{3.135156in}}%
\pgfpathlineto{\pgfqpoint{6.039244in}{3.136856in}}%
\pgfpathlineto{\pgfqpoint{6.042456in}{3.135871in}}%
\pgfpathlineto{\pgfqpoint{6.046202in}{3.132139in}}%
\pgfpathlineto{\pgfqpoint{6.061456in}{3.113611in}}%
\pgfpathlineto{\pgfqpoint{6.064667in}{3.114257in}}%
\pgfpathlineto{\pgfqpoint{6.068414in}{3.117669in}}%
\pgfpathlineto{\pgfqpoint{6.075104in}{3.127778in}}%
\pgfpathlineto{\pgfqpoint{6.081259in}{3.135454in}}%
\pgfpathlineto{\pgfqpoint{6.085005in}{3.136859in}}%
\pgfpathlineto{\pgfqpoint{6.088217in}{3.135616in}}%
\pgfpathlineto{\pgfqpoint{6.092231in}{3.131285in}}%
\pgfpathlineto{\pgfqpoint{6.106146in}{3.113774in}}%
\pgfpathlineto{\pgfqpoint{6.109358in}{3.113924in}}%
\pgfpathlineto{\pgfqpoint{6.112836in}{3.116550in}}%
\pgfpathlineto{\pgfqpoint{6.118189in}{3.124082in}}%
\pgfpathlineto{\pgfqpoint{6.126484in}{3.135325in}}%
\pgfpathlineto{\pgfqpoint{6.130231in}{3.136861in}}%
\pgfpathlineto{\pgfqpoint{6.133442in}{3.135733in}}%
\pgfpathlineto{\pgfqpoint{6.137456in}{3.131510in}}%
\pgfpathlineto{\pgfqpoint{6.151640in}{3.113743in}}%
\pgfpathlineto{\pgfqpoint{6.154851in}{3.113966in}}%
\pgfpathlineto{\pgfqpoint{6.158330in}{3.116662in}}%
\pgfpathlineto{\pgfqpoint{6.163682in}{3.124248in}}%
\pgfpathlineto{\pgfqpoint{6.171710in}{3.135191in}}%
\pgfpathlineto{\pgfqpoint{6.175457in}{3.136858in}}%
\pgfpathlineto{\pgfqpoint{6.178668in}{3.135844in}}%
\pgfpathlineto{\pgfqpoint{6.182415in}{3.132085in}}%
\pgfpathlineto{\pgfqpoint{6.197668in}{3.113605in}}%
\pgfpathlineto{\pgfqpoint{6.200879in}{3.114280in}}%
\pgfpathlineto{\pgfqpoint{6.204626in}{3.117720in}}%
\pgfpathlineto{\pgfqpoint{6.211316in}{3.127843in}}%
\pgfpathlineto{\pgfqpoint{6.217471in}{3.135486in}}%
\pgfpathlineto{\pgfqpoint{6.221218in}{3.136857in}}%
\pgfpathlineto{\pgfqpoint{6.224429in}{3.135586in}}%
\pgfpathlineto{\pgfqpoint{6.228443in}{3.131228in}}%
\pgfpathlineto{\pgfqpoint{6.242091in}{3.113850in}}%
\pgfpathlineto{\pgfqpoint{6.245302in}{3.113839in}}%
\pgfpathlineto{\pgfqpoint{6.248781in}{3.116310in}}%
\pgfpathlineto{\pgfqpoint{6.253866in}{3.123292in}}%
\pgfpathlineto{\pgfqpoint{6.262697in}{3.135358in}}%
\pgfpathlineto{\pgfqpoint{6.266443in}{3.136861in}}%
\pgfpathlineto{\pgfqpoint{6.269655in}{3.135704in}}%
\pgfpathlineto{\pgfqpoint{6.273669in}{3.131454in}}%
\pgfpathlineto{\pgfqpoint{6.287852in}{3.113732in}}%
\pgfpathlineto{\pgfqpoint{6.291063in}{3.113984in}}%
\pgfpathlineto{\pgfqpoint{6.294542in}{3.116708in}}%
\pgfpathlineto{\pgfqpoint{6.299894in}{3.124314in}}%
\pgfpathlineto{\pgfqpoint{6.307923in}{3.135225in}}%
\pgfpathlineto{\pgfqpoint{6.311669in}{3.136860in}}%
\pgfpathlineto{\pgfqpoint{6.314880in}{3.135817in}}%
\pgfpathlineto{\pgfqpoint{6.318894in}{3.131677in}}%
\pgfpathlineto{\pgfqpoint{6.333345in}{3.113704in}}%
\pgfpathlineto{\pgfqpoint{6.336557in}{3.114030in}}%
\pgfpathlineto{\pgfqpoint{6.340036in}{3.116822in}}%
\pgfpathlineto{\pgfqpoint{6.345655in}{3.124911in}}%
\pgfpathlineto{\pgfqpoint{6.353148in}{3.135087in}}%
\pgfpathlineto{\pgfqpoint{6.356895in}{3.136852in}}%
\pgfpathlineto{\pgfqpoint{6.360106in}{3.135924in}}%
\pgfpathlineto{\pgfqpoint{6.363853in}{3.132245in}}%
\pgfpathlineto{\pgfqpoint{6.372148in}{3.119622in}}%
\pgfpathlineto{\pgfqpoint{6.377233in}{3.114333in}}%
\pgfpathlineto{\pgfqpoint{6.380712in}{3.113634in}}%
\pgfpathlineto{\pgfqpoint{6.383923in}{3.115332in}}%
\pgfpathlineto{\pgfqpoint{6.388205in}{3.120470in}}%
\pgfpathlineto{\pgfqpoint{6.399980in}{3.136116in}}%
\pgfpathlineto{\pgfqpoint{6.403459in}{3.136775in}}%
\pgfpathlineto{\pgfqpoint{6.406670in}{3.135042in}}%
\pgfpathlineto{\pgfqpoint{6.410952in}{3.129873in}}%
\pgfpathlineto{\pgfqpoint{6.422726in}{3.114274in}}%
\pgfpathlineto{\pgfqpoint{6.426205in}{3.113654in}}%
\pgfpathlineto{\pgfqpoint{6.429417in}{3.115421in}}%
\pgfpathlineto{\pgfqpoint{6.433698in}{3.120622in}}%
\pgfpathlineto{\pgfqpoint{6.445205in}{3.136019in}}%
\pgfpathlineto{\pgfqpoint{6.448684in}{3.136804in}}%
\pgfpathlineto{\pgfqpoint{6.451896in}{3.135182in}}%
\pgfpathlineto{\pgfqpoint{6.456177in}{3.130115in}}%
\pgfpathlineto{\pgfqpoint{6.468220in}{3.114218in}}%
\pgfpathlineto{\pgfqpoint{6.471699in}{3.113677in}}%
\pgfpathlineto{\pgfqpoint{6.474910in}{3.115512in}}%
\pgfpathlineto{\pgfqpoint{6.479459in}{3.121177in}}%
\pgfpathlineto{\pgfqpoint{6.490431in}{3.135917in}}%
\pgfpathlineto{\pgfqpoint{6.493910in}{3.136828in}}%
\pgfpathlineto{\pgfqpoint{6.497121in}{3.135317in}}%
\pgfpathlineto{\pgfqpoint{6.501403in}{3.130355in}}%
\pgfpathlineto{\pgfqpoint{6.513713in}{3.114164in}}%
\pgfpathlineto{\pgfqpoint{6.517192in}{3.113702in}}%
\pgfpathlineto{\pgfqpoint{6.520403in}{3.115606in}}%
\pgfpathlineto{\pgfqpoint{6.524953in}{3.121334in}}%
\pgfpathlineto{\pgfqpoint{6.535657in}{3.135810in}}%
\pgfpathlineto{\pgfqpoint{6.539136in}{3.136845in}}%
\pgfpathlineto{\pgfqpoint{6.542347in}{3.135446in}}%
\pgfpathlineto{\pgfqpoint{6.546361in}{3.130971in}}%
\pgfpathlineto{\pgfqpoint{6.559742in}{3.113880in}}%
\pgfpathlineto{\pgfqpoint{6.562953in}{3.113810in}}%
\pgfpathlineto{\pgfqpoint{6.566432in}{3.116225in}}%
\pgfpathlineto{\pgfqpoint{6.571516in}{3.123162in}}%
\pgfpathlineto{\pgfqpoint{6.580347in}{3.135292in}}%
\pgfpathlineto{\pgfqpoint{6.584094in}{3.136861in}}%
\pgfpathlineto{\pgfqpoint{6.587305in}{3.135761in}}%
\pgfpathlineto{\pgfqpoint{6.591319in}{3.131566in}}%
\pgfpathlineto{\pgfqpoint{6.605502in}{3.113755in}}%
\pgfpathlineto{\pgfqpoint{6.608714in}{3.113949in}}%
\pgfpathlineto{\pgfqpoint{6.612193in}{3.116617in}}%
\pgfpathlineto{\pgfqpoint{6.617545in}{3.124181in}}%
\pgfpathlineto{\pgfqpoint{6.625573in}{3.135156in}}%
\pgfpathlineto{\pgfqpoint{6.629320in}{3.136856in}}%
\pgfpathlineto{\pgfqpoint{6.632531in}{3.135871in}}%
\pgfpathlineto{\pgfqpoint{6.636277in}{3.132139in}}%
\pgfpathlineto{\pgfqpoint{6.651531in}{3.113611in}}%
\pgfpathlineto{\pgfqpoint{6.654742in}{3.114257in}}%
\pgfpathlineto{\pgfqpoint{6.658489in}{3.117669in}}%
\pgfpathlineto{\pgfqpoint{6.663306in}{3.124778in}}%
\pgfpathlineto{\pgfqpoint{6.663306in}{3.124778in}}%
\pgfusepath{stroke}%
\end{pgfscope}%
\begin{pgfscope}%
\pgfpathrectangle{\pgfqpoint{0.467797in}{2.292089in}}{\pgfqpoint{6.490533in}{1.666241in}}%
\pgfusepath{clip}%
\pgfsetrectcap%
\pgfsetroundjoin%
\pgfsetlinewidth{1.505625pt}%
\definecolor{currentstroke}{rgb}{0.839216,0.152941,0.156863}%
\pgfsetstrokecolor{currentstroke}%
\pgfsetdash{}{0pt}%
\pgfpathmoveto{\pgfqpoint{0.762821in}{3.125209in}}%
\pgfpathlineto{\pgfqpoint{0.770046in}{3.134906in}}%
\pgfpathlineto{\pgfqpoint{0.773793in}{3.136513in}}%
\pgfpathlineto{\pgfqpoint{0.777004in}{3.135374in}}%
\pgfpathlineto{\pgfqpoint{0.781018in}{3.131071in}}%
\pgfpathlineto{\pgfqpoint{0.794399in}{3.114146in}}%
\pgfpathlineto{\pgfqpoint{0.797610in}{3.114261in}}%
\pgfpathlineto{\pgfqpoint{0.801089in}{3.116922in}}%
\pgfpathlineto{\pgfqpoint{0.806441in}{3.124540in}}%
\pgfpathlineto{\pgfqpoint{0.814202in}{3.135004in}}%
\pgfpathlineto{\pgfqpoint{0.817948in}{3.136513in}}%
\pgfpathlineto{\pgfqpoint{0.821159in}{3.135288in}}%
\pgfpathlineto{\pgfqpoint{0.825174in}{3.130905in}}%
\pgfpathlineto{\pgfqpoint{0.838286in}{3.114197in}}%
\pgfpathlineto{\pgfqpoint{0.841498in}{3.114204in}}%
\pgfpathlineto{\pgfqpoint{0.844977in}{3.116762in}}%
\pgfpathlineto{\pgfqpoint{0.850061in}{3.123873in}}%
\pgfpathlineto{\pgfqpoint{0.858357in}{3.135099in}}%
\pgfpathlineto{\pgfqpoint{0.861836in}{3.136513in}}%
\pgfpathlineto{\pgfqpoint{0.865047in}{3.135393in}}%
\pgfpathlineto{\pgfqpoint{0.869061in}{3.131109in}}%
\pgfpathlineto{\pgfqpoint{0.882442in}{3.114155in}}%
\pgfpathlineto{\pgfqpoint{0.885653in}{3.114250in}}%
\pgfpathlineto{\pgfqpoint{0.889132in}{3.116891in}}%
\pgfpathlineto{\pgfqpoint{0.894484in}{3.124495in}}%
\pgfpathlineto{\pgfqpoint{0.902245in}{3.134981in}}%
\pgfpathlineto{\pgfqpoint{0.905991in}{3.136513in}}%
\pgfpathlineto{\pgfqpoint{0.909202in}{3.135308in}}%
\pgfpathlineto{\pgfqpoint{0.913217in}{3.130944in}}%
\pgfpathlineto{\pgfqpoint{0.926329in}{3.114208in}}%
\pgfpathlineto{\pgfqpoint{0.929541in}{3.114194in}}%
\pgfpathlineto{\pgfqpoint{0.933019in}{3.116732in}}%
\pgfpathlineto{\pgfqpoint{0.938104in}{3.123828in}}%
\pgfpathlineto{\pgfqpoint{0.946400in}{3.135077in}}%
\pgfpathlineto{\pgfqpoint{0.949879in}{3.136513in}}%
\pgfpathlineto{\pgfqpoint{0.953090in}{3.135413in}}%
\pgfpathlineto{\pgfqpoint{0.957104in}{3.131148in}}%
\pgfpathlineto{\pgfqpoint{0.970752in}{3.114081in}}%
\pgfpathlineto{\pgfqpoint{0.973963in}{3.114351in}}%
\pgfpathlineto{\pgfqpoint{0.977442in}{3.117158in}}%
\pgfpathlineto{\pgfqpoint{0.983062in}{3.125313in}}%
\pgfpathlineto{\pgfqpoint{0.990288in}{3.134959in}}%
\pgfpathlineto{\pgfqpoint{0.994034in}{3.136514in}}%
\pgfpathlineto{\pgfqpoint{0.997245in}{3.135328in}}%
\pgfpathlineto{\pgfqpoint{1.001259in}{3.130982in}}%
\pgfpathlineto{\pgfqpoint{1.014640in}{3.114125in}}%
\pgfpathlineto{\pgfqpoint{1.017851in}{3.114287in}}%
\pgfpathlineto{\pgfqpoint{1.021330in}{3.116992in}}%
\pgfpathlineto{\pgfqpoint{1.026682in}{3.124643in}}%
\pgfpathlineto{\pgfqpoint{1.034443in}{3.135055in}}%
\pgfpathlineto{\pgfqpoint{1.037922in}{3.136512in}}%
\pgfpathlineto{\pgfqpoint{1.041133in}{3.135432in}}%
\pgfpathlineto{\pgfqpoint{1.044880in}{3.131548in}}%
\pgfpathlineto{\pgfqpoint{1.059063in}{3.114020in}}%
\pgfpathlineto{\pgfqpoint{1.062274in}{3.114464in}}%
\pgfpathlineto{\pgfqpoint{1.065753in}{3.117434in}}%
\pgfpathlineto{\pgfqpoint{1.071640in}{3.126130in}}%
\pgfpathlineto{\pgfqpoint{1.078331in}{3.134936in}}%
\pgfpathlineto{\pgfqpoint{1.082077in}{3.136514in}}%
\pgfpathlineto{\pgfqpoint{1.085288in}{3.135348in}}%
\pgfpathlineto{\pgfqpoint{1.089302in}{3.131021in}}%
\pgfpathlineto{\pgfqpoint{1.102683in}{3.114134in}}%
\pgfpathlineto{\pgfqpoint{1.105894in}{3.114275in}}%
\pgfpathlineto{\pgfqpoint{1.109373in}{3.116961in}}%
\pgfpathlineto{\pgfqpoint{1.114725in}{3.124598in}}%
\pgfpathlineto{\pgfqpoint{1.122486in}{3.135033in}}%
\pgfpathlineto{\pgfqpoint{1.125965in}{3.136511in}}%
\pgfpathlineto{\pgfqpoint{1.129176in}{3.135451in}}%
\pgfpathlineto{\pgfqpoint{1.132923in}{3.131585in}}%
\pgfpathlineto{\pgfqpoint{1.147106in}{3.114026in}}%
\pgfpathlineto{\pgfqpoint{1.150317in}{3.114451in}}%
\pgfpathlineto{\pgfqpoint{1.153796in}{3.117401in}}%
\pgfpathlineto{\pgfqpoint{1.159683in}{3.126085in}}%
\pgfpathlineto{\pgfqpoint{1.166374in}{3.134913in}}%
\pgfpathlineto{\pgfqpoint{1.170120in}{3.136514in}}%
\pgfpathlineto{\pgfqpoint{1.173331in}{3.135368in}}%
\pgfpathlineto{\pgfqpoint{1.177345in}{3.131060in}}%
\pgfpathlineto{\pgfqpoint{1.190726in}{3.114143in}}%
\pgfpathlineto{\pgfqpoint{1.193937in}{3.114264in}}%
\pgfpathlineto{\pgfqpoint{1.197416in}{3.116931in}}%
\pgfpathlineto{\pgfqpoint{1.202768in}{3.124553in}}%
\pgfpathlineto{\pgfqpoint{1.210529in}{3.135010in}}%
\pgfpathlineto{\pgfqpoint{1.214008in}{3.136510in}}%
\pgfpathlineto{\pgfqpoint{1.217219in}{3.135470in}}%
\pgfpathlineto{\pgfqpoint{1.220965in}{3.131622in}}%
\pgfpathlineto{\pgfqpoint{1.235416in}{3.113976in}}%
\pgfpathlineto{\pgfqpoint{1.238628in}{3.114575in}}%
\pgfpathlineto{\pgfqpoint{1.242374in}{3.118013in}}%
\pgfpathlineto{\pgfqpoint{1.249332in}{3.128578in}}%
\pgfpathlineto{\pgfqpoint{1.254952in}{3.135307in}}%
\pgfpathlineto{\pgfqpoint{1.258431in}{3.136509in}}%
\pgfpathlineto{\pgfqpoint{1.261642in}{3.135193in}}%
\pgfpathlineto{\pgfqpoint{1.265656in}{3.130726in}}%
\pgfpathlineto{\pgfqpoint{1.278501in}{3.114250in}}%
\pgfpathlineto{\pgfqpoint{1.281712in}{3.114155in}}%
\pgfpathlineto{\pgfqpoint{1.285191in}{3.116613in}}%
\pgfpathlineto{\pgfqpoint{1.290276in}{3.123649in}}%
\pgfpathlineto{\pgfqpoint{1.298572in}{3.134988in}}%
\pgfpathlineto{\pgfqpoint{1.302318in}{3.136513in}}%
\pgfpathlineto{\pgfqpoint{1.305530in}{3.135302in}}%
\pgfpathlineto{\pgfqpoint{1.309544in}{3.130933in}}%
\pgfpathlineto{\pgfqpoint{1.322656in}{3.114205in}}%
\pgfpathlineto{\pgfqpoint{1.325868in}{3.114197in}}%
\pgfpathlineto{\pgfqpoint{1.329347in}{3.116740in}}%
\pgfpathlineto{\pgfqpoint{1.334431in}{3.123841in}}%
\pgfpathlineto{\pgfqpoint{1.342727in}{3.135083in}}%
\pgfpathlineto{\pgfqpoint{1.346206in}{3.136513in}}%
\pgfpathlineto{\pgfqpoint{1.349417in}{3.135407in}}%
\pgfpathlineto{\pgfqpoint{1.353431in}{3.131137in}}%
\pgfpathlineto{\pgfqpoint{1.366812in}{3.114162in}}%
\pgfpathlineto{\pgfqpoint{1.370023in}{3.114242in}}%
\pgfpathlineto{\pgfqpoint{1.373502in}{3.116869in}}%
\pgfpathlineto{\pgfqpoint{1.378854in}{3.124463in}}%
\pgfpathlineto{\pgfqpoint{1.386615in}{3.134965in}}%
\pgfpathlineto{\pgfqpoint{1.390361in}{3.136514in}}%
\pgfpathlineto{\pgfqpoint{1.393573in}{3.135323in}}%
\pgfpathlineto{\pgfqpoint{1.397587in}{3.130971in}}%
\pgfpathlineto{\pgfqpoint{1.410699in}{3.114215in}}%
\pgfpathlineto{\pgfqpoint{1.413911in}{3.114186in}}%
\pgfpathlineto{\pgfqpoint{1.417390in}{3.116710in}}%
\pgfpathlineto{\pgfqpoint{1.422474in}{3.123796in}}%
\pgfpathlineto{\pgfqpoint{1.430770in}{3.135061in}}%
\pgfpathlineto{\pgfqpoint{1.434249in}{3.136512in}}%
\pgfpathlineto{\pgfqpoint{1.437460in}{3.135427in}}%
\pgfpathlineto{\pgfqpoint{1.441207in}{3.131537in}}%
\pgfpathlineto{\pgfqpoint{1.455390in}{3.114018in}}%
\pgfpathlineto{\pgfqpoint{1.458601in}{3.114468in}}%
\pgfpathlineto{\pgfqpoint{1.462080in}{3.117443in}}%
\pgfpathlineto{\pgfqpoint{1.467968in}{3.126143in}}%
\pgfpathlineto{\pgfqpoint{1.474658in}{3.134942in}}%
\pgfpathlineto{\pgfqpoint{1.478404in}{3.136514in}}%
\pgfpathlineto{\pgfqpoint{1.481616in}{3.135343in}}%
\pgfpathlineto{\pgfqpoint{1.485630in}{3.131010in}}%
\pgfpathlineto{\pgfqpoint{1.499010in}{3.114132in}}%
\pgfpathlineto{\pgfqpoint{1.502221in}{3.114279in}}%
\pgfpathlineto{\pgfqpoint{1.505700in}{3.116970in}}%
\pgfpathlineto{\pgfqpoint{1.511052in}{3.124611in}}%
\pgfpathlineto{\pgfqpoint{1.518813in}{3.135039in}}%
\pgfpathlineto{\pgfqpoint{1.522292in}{3.136511in}}%
\pgfpathlineto{\pgfqpoint{1.525503in}{3.135446in}}%
\pgfpathlineto{\pgfqpoint{1.529250in}{3.131575in}}%
\pgfpathlineto{\pgfqpoint{1.543433in}{3.114024in}}%
\pgfpathlineto{\pgfqpoint{1.546644in}{3.114455in}}%
\pgfpathlineto{\pgfqpoint{1.550123in}{3.117410in}}%
\pgfpathlineto{\pgfqpoint{1.556010in}{3.126098in}}%
\pgfpathlineto{\pgfqpoint{1.562701in}{3.134919in}}%
\pgfpathlineto{\pgfqpoint{1.566447in}{3.136514in}}%
\pgfpathlineto{\pgfqpoint{1.569658in}{3.135363in}}%
\pgfpathlineto{\pgfqpoint{1.573673in}{3.131049in}}%
\pgfpathlineto{\pgfqpoint{1.587053in}{3.114141in}}%
\pgfpathlineto{\pgfqpoint{1.590264in}{3.114267in}}%
\pgfpathlineto{\pgfqpoint{1.593743in}{3.116939in}}%
\pgfpathlineto{\pgfqpoint{1.599095in}{3.124566in}}%
\pgfpathlineto{\pgfqpoint{1.606856in}{3.135017in}}%
\pgfpathlineto{\pgfqpoint{1.610335in}{3.136510in}}%
\pgfpathlineto{\pgfqpoint{1.613546in}{3.135465in}}%
\pgfpathlineto{\pgfqpoint{1.617293in}{3.131612in}}%
\pgfpathlineto{\pgfqpoint{1.631744in}{3.113975in}}%
\pgfpathlineto{\pgfqpoint{1.634955in}{3.114580in}}%
\pgfpathlineto{\pgfqpoint{1.638701in}{3.118022in}}%
\pgfpathlineto{\pgfqpoint{1.645659in}{3.128591in}}%
\pgfpathlineto{\pgfqpoint{1.651279in}{3.135312in}}%
\pgfpathlineto{\pgfqpoint{1.654758in}{3.136508in}}%
\pgfpathlineto{\pgfqpoint{1.657969in}{3.135187in}}%
\pgfpathlineto{\pgfqpoint{1.661983in}{3.130714in}}%
\pgfpathlineto{\pgfqpoint{1.674828in}{3.114247in}}%
\pgfpathlineto{\pgfqpoint{1.678040in}{3.114157in}}%
\pgfpathlineto{\pgfqpoint{1.681519in}{3.116622in}}%
\pgfpathlineto{\pgfqpoint{1.686603in}{3.123662in}}%
\pgfpathlineto{\pgfqpoint{1.694899in}{3.134994in}}%
\pgfpathlineto{\pgfqpoint{1.698645in}{3.136513in}}%
\pgfpathlineto{\pgfqpoint{1.701857in}{3.135297in}}%
\pgfpathlineto{\pgfqpoint{1.705871in}{3.130921in}}%
\pgfpathlineto{\pgfqpoint{1.718984in}{3.114202in}}%
\pgfpathlineto{\pgfqpoint{1.722195in}{3.114199in}}%
\pgfpathlineto{\pgfqpoint{1.725674in}{3.116749in}}%
\pgfpathlineto{\pgfqpoint{1.730758in}{3.123853in}}%
\pgfpathlineto{\pgfqpoint{1.739054in}{3.135090in}}%
\pgfpathlineto{\pgfqpoint{1.742533in}{3.136513in}}%
\pgfpathlineto{\pgfqpoint{1.745744in}{3.135402in}}%
\pgfpathlineto{\pgfqpoint{1.749759in}{3.131126in}}%
\pgfpathlineto{\pgfqpoint{1.763139in}{3.114159in}}%
\pgfpathlineto{\pgfqpoint{1.766350in}{3.114245in}}%
\pgfpathlineto{\pgfqpoint{1.769829in}{3.116878in}}%
\pgfpathlineto{\pgfqpoint{1.775181in}{3.124476in}}%
\pgfpathlineto{\pgfqpoint{1.782942in}{3.134972in}}%
\pgfpathlineto{\pgfqpoint{1.786688in}{3.136513in}}%
\pgfpathlineto{\pgfqpoint{1.789900in}{3.135317in}}%
\pgfpathlineto{\pgfqpoint{1.793914in}{3.130960in}}%
\pgfpathlineto{\pgfqpoint{1.807027in}{3.114212in}}%
\pgfpathlineto{\pgfqpoint{1.810238in}{3.114189in}}%
\pgfpathlineto{\pgfqpoint{1.813717in}{3.116719in}}%
\pgfpathlineto{\pgfqpoint{1.818801in}{3.123809in}}%
\pgfpathlineto{\pgfqpoint{1.827097in}{3.135068in}}%
\pgfpathlineto{\pgfqpoint{1.830576in}{3.136512in}}%
\pgfpathlineto{\pgfqpoint{1.833787in}{3.135421in}}%
\pgfpathlineto{\pgfqpoint{1.837534in}{3.131527in}}%
\pgfpathlineto{\pgfqpoint{1.851717in}{3.114016in}}%
\pgfpathlineto{\pgfqpoint{1.854928in}{3.114473in}}%
\pgfpathlineto{\pgfqpoint{1.858407in}{3.117452in}}%
\pgfpathlineto{\pgfqpoint{1.864295in}{3.126155in}}%
\pgfpathlineto{\pgfqpoint{1.870985in}{3.134949in}}%
\pgfpathlineto{\pgfqpoint{1.874731in}{3.136514in}}%
\pgfpathlineto{\pgfqpoint{1.877943in}{3.135337in}}%
\pgfpathlineto{\pgfqpoint{1.881957in}{3.130999in}}%
\pgfpathlineto{\pgfqpoint{1.895337in}{3.114129in}}%
\pgfpathlineto{\pgfqpoint{1.898549in}{3.114282in}}%
\pgfpathlineto{\pgfqpoint{1.902027in}{3.116979in}}%
\pgfpathlineto{\pgfqpoint{1.907380in}{3.124623in}}%
\pgfpathlineto{\pgfqpoint{1.915140in}{3.135046in}}%
\pgfpathlineto{\pgfqpoint{1.918619in}{3.136512in}}%
\pgfpathlineto{\pgfqpoint{1.921830in}{3.135441in}}%
\pgfpathlineto{\pgfqpoint{1.925577in}{3.131564in}}%
\pgfpathlineto{\pgfqpoint{1.939760in}{3.114022in}}%
\pgfpathlineto{\pgfqpoint{1.942971in}{3.114458in}}%
\pgfpathlineto{\pgfqpoint{1.946450in}{3.117420in}}%
\pgfpathlineto{\pgfqpoint{1.952338in}{3.126110in}}%
\pgfpathlineto{\pgfqpoint{1.959028in}{3.134926in}}%
\pgfpathlineto{\pgfqpoint{1.962774in}{3.136514in}}%
\pgfpathlineto{\pgfqpoint{1.965986in}{3.135357in}}%
\pgfpathlineto{\pgfqpoint{1.970000in}{3.131038in}}%
\pgfpathlineto{\pgfqpoint{1.983380in}{3.114138in}}%
\pgfpathlineto{\pgfqpoint{1.986591in}{3.114270in}}%
\pgfpathlineto{\pgfqpoint{1.990070in}{3.116948in}}%
\pgfpathlineto{\pgfqpoint{1.995423in}{3.124578in}}%
\pgfpathlineto{\pgfqpoint{2.003183in}{3.135023in}}%
\pgfpathlineto{\pgfqpoint{2.006662in}{3.136511in}}%
\pgfpathlineto{\pgfqpoint{2.009873in}{3.135460in}}%
\pgfpathlineto{\pgfqpoint{2.013620in}{3.131601in}}%
\pgfpathlineto{\pgfqpoint{2.028071in}{3.113973in}}%
\pgfpathlineto{\pgfqpoint{2.031282in}{3.114584in}}%
\pgfpathlineto{\pgfqpoint{2.035028in}{3.118032in}}%
\pgfpathlineto{\pgfqpoint{2.041986in}{3.128603in}}%
\pgfpathlineto{\pgfqpoint{2.047606in}{3.135318in}}%
\pgfpathlineto{\pgfqpoint{2.051085in}{3.136508in}}%
\pgfpathlineto{\pgfqpoint{2.054296in}{3.135181in}}%
\pgfpathlineto{\pgfqpoint{2.058310in}{3.130703in}}%
\pgfpathlineto{\pgfqpoint{2.071156in}{3.114244in}}%
\pgfpathlineto{\pgfqpoint{2.074367in}{3.114160in}}%
\pgfpathlineto{\pgfqpoint{2.077846in}{3.116630in}}%
\pgfpathlineto{\pgfqpoint{2.082930in}{3.123674in}}%
\pgfpathlineto{\pgfqpoint{2.091226in}{3.135001in}}%
\pgfpathlineto{\pgfqpoint{2.094973in}{3.136513in}}%
\pgfpathlineto{\pgfqpoint{2.098184in}{3.135291in}}%
\pgfpathlineto{\pgfqpoint{2.102198in}{3.130910in}}%
\pgfpathlineto{\pgfqpoint{2.115311in}{3.114199in}}%
\pgfpathlineto{\pgfqpoint{2.118522in}{3.114202in}}%
\pgfpathlineto{\pgfqpoint{2.122001in}{3.116757in}}%
\pgfpathlineto{\pgfqpoint{2.127086in}{3.123866in}}%
\pgfpathlineto{\pgfqpoint{2.135381in}{3.135096in}}%
\pgfpathlineto{\pgfqpoint{2.138860in}{3.136513in}}%
\pgfpathlineto{\pgfqpoint{2.142072in}{3.135396in}}%
\pgfpathlineto{\pgfqpoint{2.146086in}{3.131115in}}%
\pgfpathlineto{\pgfqpoint{2.159466in}{3.114157in}}%
\pgfpathlineto{\pgfqpoint{2.162677in}{3.114248in}}%
\pgfpathlineto{\pgfqpoint{2.166156in}{3.116887in}}%
\pgfpathlineto{\pgfqpoint{2.171508in}{3.124488in}}%
\pgfpathlineto{\pgfqpoint{2.179269in}{3.134978in}}%
\pgfpathlineto{\pgfqpoint{2.183016in}{3.136513in}}%
\pgfpathlineto{\pgfqpoint{2.186227in}{3.135311in}}%
\pgfpathlineto{\pgfqpoint{2.190241in}{3.130949in}}%
\pgfpathlineto{\pgfqpoint{2.203354in}{3.114209in}}%
\pgfpathlineto{\pgfqpoint{2.206565in}{3.114192in}}%
\pgfpathlineto{\pgfqpoint{2.210044in}{3.116727in}}%
\pgfpathlineto{\pgfqpoint{2.215129in}{3.123821in}}%
\pgfpathlineto{\pgfqpoint{2.223424in}{3.135074in}}%
\pgfpathlineto{\pgfqpoint{2.226903in}{3.136513in}}%
\pgfpathlineto{\pgfqpoint{2.230115in}{3.135416in}}%
\pgfpathlineto{\pgfqpoint{2.234129in}{3.131153in}}%
\pgfpathlineto{\pgfqpoint{2.247777in}{3.114082in}}%
\pgfpathlineto{\pgfqpoint{2.250988in}{3.114349in}}%
\pgfpathlineto{\pgfqpoint{2.254467in}{3.117153in}}%
\pgfpathlineto{\pgfqpoint{2.260087in}{3.125306in}}%
\pgfpathlineto{\pgfqpoint{2.267312in}{3.134955in}}%
\pgfpathlineto{\pgfqpoint{2.271059in}{3.136514in}}%
\pgfpathlineto{\pgfqpoint{2.274270in}{3.135331in}}%
\pgfpathlineto{\pgfqpoint{2.278284in}{3.130988in}}%
\pgfpathlineto{\pgfqpoint{2.291664in}{3.114126in}}%
\pgfpathlineto{\pgfqpoint{2.294876in}{3.114285in}}%
\pgfpathlineto{\pgfqpoint{2.298355in}{3.116988in}}%
\pgfpathlineto{\pgfqpoint{2.303707in}{3.124636in}}%
\pgfpathlineto{\pgfqpoint{2.311467in}{3.135052in}}%
\pgfpathlineto{\pgfqpoint{2.314946in}{3.136512in}}%
\pgfpathlineto{\pgfqpoint{2.318158in}{3.135435in}}%
\pgfpathlineto{\pgfqpoint{2.321904in}{3.131553in}}%
\pgfpathlineto{\pgfqpoint{2.336087in}{3.114020in}}%
\pgfpathlineto{\pgfqpoint{2.339299in}{3.114462in}}%
\pgfpathlineto{\pgfqpoint{2.342777in}{3.117429in}}%
\pgfpathlineto{\pgfqpoint{2.348665in}{3.126123in}}%
\pgfpathlineto{\pgfqpoint{2.355355in}{3.134932in}}%
\pgfpathlineto{\pgfqpoint{2.359102in}{3.136514in}}%
\pgfpathlineto{\pgfqpoint{2.362313in}{3.135351in}}%
\pgfpathlineto{\pgfqpoint{2.366327in}{3.131027in}}%
\pgfpathlineto{\pgfqpoint{2.379707in}{3.114135in}}%
\pgfpathlineto{\pgfqpoint{2.382919in}{3.114274in}}%
\pgfpathlineto{\pgfqpoint{2.386398in}{3.116957in}}%
\pgfpathlineto{\pgfqpoint{2.391750in}{3.124591in}}%
\pgfpathlineto{\pgfqpoint{2.399510in}{3.135030in}}%
\pgfpathlineto{\pgfqpoint{2.402989in}{3.136511in}}%
\pgfpathlineto{\pgfqpoint{2.406201in}{3.135454in}}%
\pgfpathlineto{\pgfqpoint{2.409947in}{3.131591in}}%
\pgfpathlineto{\pgfqpoint{2.424130in}{3.114027in}}%
\pgfpathlineto{\pgfqpoint{2.427342in}{3.114449in}}%
\pgfpathlineto{\pgfqpoint{2.430820in}{3.117396in}}%
\pgfpathlineto{\pgfqpoint{2.436708in}{3.126078in}}%
\pgfpathlineto{\pgfqpoint{2.443398in}{3.134909in}}%
\pgfpathlineto{\pgfqpoint{2.447145in}{3.136514in}}%
\pgfpathlineto{\pgfqpoint{2.450356in}{3.135371in}}%
\pgfpathlineto{\pgfqpoint{2.454370in}{3.131065in}}%
\pgfpathlineto{\pgfqpoint{2.467750in}{3.114145in}}%
\pgfpathlineto{\pgfqpoint{2.470962in}{3.114262in}}%
\pgfpathlineto{\pgfqpoint{2.474441in}{3.116926in}}%
\pgfpathlineto{\pgfqpoint{2.479793in}{3.124546in}}%
\pgfpathlineto{\pgfqpoint{2.487553in}{3.135007in}}%
\pgfpathlineto{\pgfqpoint{2.491032in}{3.136510in}}%
\pgfpathlineto{\pgfqpoint{2.494244in}{3.135473in}}%
\pgfpathlineto{\pgfqpoint{2.497990in}{3.131628in}}%
\pgfpathlineto{\pgfqpoint{2.512441in}{3.113977in}}%
\pgfpathlineto{\pgfqpoint{2.515652in}{3.114573in}}%
\pgfpathlineto{\pgfqpoint{2.519399in}{3.118008in}}%
\pgfpathlineto{\pgfqpoint{2.526356in}{3.128572in}}%
\pgfpathlineto{\pgfqpoint{2.531976in}{3.135304in}}%
\pgfpathlineto{\pgfqpoint{2.535455in}{3.136509in}}%
\pgfpathlineto{\pgfqpoint{2.538666in}{3.135196in}}%
\pgfpathlineto{\pgfqpoint{2.542681in}{3.130731in}}%
\pgfpathlineto{\pgfqpoint{2.555526in}{3.114252in}}%
\pgfpathlineto{\pgfqpoint{2.558737in}{3.114153in}}%
\pgfpathlineto{\pgfqpoint{2.562216in}{3.116609in}}%
\pgfpathlineto{\pgfqpoint{2.567300in}{3.123643in}}%
\pgfpathlineto{\pgfqpoint{2.575596in}{3.134985in}}%
\pgfpathlineto{\pgfqpoint{2.579343in}{3.136513in}}%
\pgfpathlineto{\pgfqpoint{2.582554in}{3.135305in}}%
\pgfpathlineto{\pgfqpoint{2.586568in}{3.130938in}}%
\pgfpathlineto{\pgfqpoint{2.599681in}{3.114206in}}%
\pgfpathlineto{\pgfqpoint{2.602892in}{3.114195in}}%
\pgfpathlineto{\pgfqpoint{2.606371in}{3.116736in}}%
\pgfpathlineto{\pgfqpoint{2.611456in}{3.123834in}}%
\pgfpathlineto{\pgfqpoint{2.619752in}{3.135080in}}%
\pgfpathlineto{\pgfqpoint{2.623230in}{3.136513in}}%
\pgfpathlineto{\pgfqpoint{2.626442in}{3.135410in}}%
\pgfpathlineto{\pgfqpoint{2.630456in}{3.131142in}}%
\pgfpathlineto{\pgfqpoint{2.644104in}{3.114080in}}%
\pgfpathlineto{\pgfqpoint{2.647315in}{3.114352in}}%
\pgfpathlineto{\pgfqpoint{2.650794in}{3.117162in}}%
\pgfpathlineto{\pgfqpoint{2.656414in}{3.125319in}}%
\pgfpathlineto{\pgfqpoint{2.663639in}{3.134962in}}%
\pgfpathlineto{\pgfqpoint{2.667386in}{3.136514in}}%
\pgfpathlineto{\pgfqpoint{2.670597in}{3.135325in}}%
\pgfpathlineto{\pgfqpoint{2.674611in}{3.130977in}}%
\pgfpathlineto{\pgfqpoint{2.687724in}{3.114217in}}%
\pgfpathlineto{\pgfqpoint{2.690935in}{3.114185in}}%
\pgfpathlineto{\pgfqpoint{2.694414in}{3.116706in}}%
\pgfpathlineto{\pgfqpoint{2.699499in}{3.123789in}}%
\pgfpathlineto{\pgfqpoint{2.707795in}{3.135058in}}%
\pgfpathlineto{\pgfqpoint{2.711273in}{3.136512in}}%
\pgfpathlineto{\pgfqpoint{2.714485in}{3.135430in}}%
\pgfpathlineto{\pgfqpoint{2.718231in}{3.131543in}}%
\pgfpathlineto{\pgfqpoint{2.732414in}{3.114019in}}%
\pgfpathlineto{\pgfqpoint{2.735626in}{3.114466in}}%
\pgfpathlineto{\pgfqpoint{2.739105in}{3.117438in}}%
\pgfpathlineto{\pgfqpoint{2.744992in}{3.126136in}}%
\pgfpathlineto{\pgfqpoint{2.751682in}{3.134939in}}%
\pgfpathlineto{\pgfqpoint{2.755429in}{3.136514in}}%
\pgfpathlineto{\pgfqpoint{2.758640in}{3.135345in}}%
\pgfpathlineto{\pgfqpoint{2.762654in}{3.131016in}}%
\pgfpathlineto{\pgfqpoint{2.776035in}{3.114133in}}%
\pgfpathlineto{\pgfqpoint{2.779246in}{3.114277in}}%
\pgfpathlineto{\pgfqpoint{2.782725in}{3.116966in}}%
\pgfpathlineto{\pgfqpoint{2.788077in}{3.124604in}}%
\pgfpathlineto{\pgfqpoint{2.795838in}{3.135036in}}%
\pgfpathlineto{\pgfqpoint{2.799316in}{3.136511in}}%
\pgfpathlineto{\pgfqpoint{2.802528in}{3.135449in}}%
\pgfpathlineto{\pgfqpoint{2.806274in}{3.131580in}}%
\pgfpathlineto{\pgfqpoint{2.820457in}{3.114025in}}%
\pgfpathlineto{\pgfqpoint{2.823669in}{3.114453in}}%
\pgfpathlineto{\pgfqpoint{2.827148in}{3.117406in}}%
\pgfpathlineto{\pgfqpoint{2.833035in}{3.126091in}}%
\pgfpathlineto{\pgfqpoint{2.839725in}{3.134916in}}%
\pgfpathlineto{\pgfqpoint{2.843472in}{3.136514in}}%
\pgfpathlineto{\pgfqpoint{2.846683in}{3.135365in}}%
\pgfpathlineto{\pgfqpoint{2.850697in}{3.131054in}}%
\pgfpathlineto{\pgfqpoint{2.864078in}{3.114142in}}%
\pgfpathlineto{\pgfqpoint{2.867289in}{3.114266in}}%
\pgfpathlineto{\pgfqpoint{2.870768in}{3.116935in}}%
\pgfpathlineto{\pgfqpoint{2.876120in}{3.124559in}}%
\pgfpathlineto{\pgfqpoint{2.883880in}{3.135014in}}%
\pgfpathlineto{\pgfqpoint{2.887359in}{3.136510in}}%
\pgfpathlineto{\pgfqpoint{2.890571in}{3.135468in}}%
\pgfpathlineto{\pgfqpoint{2.894317in}{3.131617in}}%
\pgfpathlineto{\pgfqpoint{2.908768in}{3.113975in}}%
\pgfpathlineto{\pgfqpoint{2.911979in}{3.114578in}}%
\pgfpathlineto{\pgfqpoint{2.915726in}{3.118018in}}%
\pgfpathlineto{\pgfqpoint{2.922684in}{3.128584in}}%
\pgfpathlineto{\pgfqpoint{2.928303in}{3.135310in}}%
\pgfpathlineto{\pgfqpoint{2.931782in}{3.136508in}}%
\pgfpathlineto{\pgfqpoint{2.934994in}{3.135190in}}%
\pgfpathlineto{\pgfqpoint{2.939008in}{3.130720in}}%
\pgfpathlineto{\pgfqpoint{2.951853in}{3.114249in}}%
\pgfpathlineto{\pgfqpoint{2.955064in}{3.114156in}}%
\pgfpathlineto{\pgfqpoint{2.958543in}{3.116618in}}%
\pgfpathlineto{\pgfqpoint{2.963628in}{3.123655in}}%
\pgfpathlineto{\pgfqpoint{2.971923in}{3.134991in}}%
\pgfpathlineto{\pgfqpoint{2.975670in}{3.136513in}}%
\pgfpathlineto{\pgfqpoint{2.978881in}{3.135299in}}%
\pgfpathlineto{\pgfqpoint{2.982895in}{3.130927in}}%
\pgfpathlineto{\pgfqpoint{2.996008in}{3.114203in}}%
\pgfpathlineto{\pgfqpoint{2.999219in}{3.114198in}}%
\pgfpathlineto{\pgfqpoint{3.002698in}{3.116744in}}%
\pgfpathlineto{\pgfqpoint{3.007783in}{3.123847in}}%
\pgfpathlineto{\pgfqpoint{3.016079in}{3.135087in}}%
\pgfpathlineto{\pgfqpoint{3.019558in}{3.136513in}}%
\pgfpathlineto{\pgfqpoint{3.022769in}{3.135405in}}%
\pgfpathlineto{\pgfqpoint{3.026783in}{3.131131in}}%
\pgfpathlineto{\pgfqpoint{3.040163in}{3.114161in}}%
\pgfpathlineto{\pgfqpoint{3.043375in}{3.114243in}}%
\pgfpathlineto{\pgfqpoint{3.046854in}{3.116874in}}%
\pgfpathlineto{\pgfqpoint{3.052206in}{3.124469in}}%
\pgfpathlineto{\pgfqpoint{3.059966in}{3.134968in}}%
\pgfpathlineto{\pgfqpoint{3.063713in}{3.136513in}}%
\pgfpathlineto{\pgfqpoint{3.066924in}{3.135320in}}%
\pgfpathlineto{\pgfqpoint{3.070938in}{3.130966in}}%
\pgfpathlineto{\pgfqpoint{3.084051in}{3.114214in}}%
\pgfpathlineto{\pgfqpoint{3.087262in}{3.114188in}}%
\pgfpathlineto{\pgfqpoint{3.090741in}{3.116715in}}%
\pgfpathlineto{\pgfqpoint{3.095826in}{3.123802in}}%
\pgfpathlineto{\pgfqpoint{3.104122in}{3.135065in}}%
\pgfpathlineto{\pgfqpoint{3.107601in}{3.136512in}}%
\pgfpathlineto{\pgfqpoint{3.110812in}{3.135424in}}%
\pgfpathlineto{\pgfqpoint{3.114558in}{3.131532in}}%
\pgfpathlineto{\pgfqpoint{3.128742in}{3.114017in}}%
\pgfpathlineto{\pgfqpoint{3.131953in}{3.114470in}}%
\pgfpathlineto{\pgfqpoint{3.135432in}{3.117448in}}%
\pgfpathlineto{\pgfqpoint{3.141319in}{3.126149in}}%
\pgfpathlineto{\pgfqpoint{3.148009in}{3.134946in}}%
\pgfpathlineto{\pgfqpoint{3.151756in}{3.136514in}}%
\pgfpathlineto{\pgfqpoint{3.154967in}{3.135340in}}%
\pgfpathlineto{\pgfqpoint{3.158981in}{3.131005in}}%
\pgfpathlineto{\pgfqpoint{3.172362in}{3.114130in}}%
\pgfpathlineto{\pgfqpoint{3.175573in}{3.114280in}}%
\pgfpathlineto{\pgfqpoint{3.179052in}{3.116975in}}%
\pgfpathlineto{\pgfqpoint{3.184404in}{3.124617in}}%
\pgfpathlineto{\pgfqpoint{3.192165in}{3.135042in}}%
\pgfpathlineto{\pgfqpoint{3.195644in}{3.136511in}}%
\pgfpathlineto{\pgfqpoint{3.198855in}{3.135443in}}%
\pgfpathlineto{\pgfqpoint{3.202601in}{3.131569in}}%
\pgfpathlineto{\pgfqpoint{3.216785in}{3.114023in}}%
\pgfpathlineto{\pgfqpoint{3.219996in}{3.114456in}}%
\pgfpathlineto{\pgfqpoint{3.223475in}{3.117415in}}%
\pgfpathlineto{\pgfqpoint{3.229362in}{3.126104in}}%
\pgfpathlineto{\pgfqpoint{3.236052in}{3.134923in}}%
\pgfpathlineto{\pgfqpoint{3.239799in}{3.136514in}}%
\pgfpathlineto{\pgfqpoint{3.243010in}{3.135360in}}%
\pgfpathlineto{\pgfqpoint{3.247024in}{3.131043in}}%
\pgfpathlineto{\pgfqpoint{3.260405in}{3.114139in}}%
\pgfpathlineto{\pgfqpoint{3.263616in}{3.114269in}}%
\pgfpathlineto{\pgfqpoint{3.267095in}{3.116944in}}%
\pgfpathlineto{\pgfqpoint{3.272447in}{3.124572in}}%
\pgfpathlineto{\pgfqpoint{3.280208in}{3.135020in}}%
\pgfpathlineto{\pgfqpoint{3.283687in}{3.136511in}}%
\pgfpathlineto{\pgfqpoint{3.286898in}{3.135462in}}%
\pgfpathlineto{\pgfqpoint{3.290644in}{3.131607in}}%
\pgfpathlineto{\pgfqpoint{3.305095in}{3.113974in}}%
\pgfpathlineto{\pgfqpoint{3.308307in}{3.114582in}}%
\pgfpathlineto{\pgfqpoint{3.312053in}{3.118027in}}%
\pgfpathlineto{\pgfqpoint{3.319011in}{3.128597in}}%
\pgfpathlineto{\pgfqpoint{3.324631in}{3.135315in}}%
\pgfpathlineto{\pgfqpoint{3.328109in}{3.136508in}}%
\pgfpathlineto{\pgfqpoint{3.331321in}{3.135184in}}%
\pgfpathlineto{\pgfqpoint{3.335335in}{3.130709in}}%
\pgfpathlineto{\pgfqpoint{3.348180in}{3.114246in}}%
\pgfpathlineto{\pgfqpoint{3.351391in}{3.114159in}}%
\pgfpathlineto{\pgfqpoint{3.354870in}{3.116626in}}%
\pgfpathlineto{\pgfqpoint{3.359955in}{3.123668in}}%
\pgfpathlineto{\pgfqpoint{3.368251in}{3.134998in}}%
\pgfpathlineto{\pgfqpoint{3.371997in}{3.136513in}}%
\pgfpathlineto{\pgfqpoint{3.375208in}{3.135294in}}%
\pgfpathlineto{\pgfqpoint{3.379223in}{3.130916in}}%
\pgfpathlineto{\pgfqpoint{3.392335in}{3.114200in}}%
\pgfpathlineto{\pgfqpoint{3.395547in}{3.114201in}}%
\pgfpathlineto{\pgfqpoint{3.399026in}{3.116753in}}%
\pgfpathlineto{\pgfqpoint{3.404110in}{3.123860in}}%
\pgfpathlineto{\pgfqpoint{3.412406in}{3.135093in}}%
\pgfpathlineto{\pgfqpoint{3.415885in}{3.136513in}}%
\pgfpathlineto{\pgfqpoint{3.419096in}{3.135399in}}%
\pgfpathlineto{\pgfqpoint{3.423110in}{3.131120in}}%
\pgfpathlineto{\pgfqpoint{3.436491in}{3.114158in}}%
\pgfpathlineto{\pgfqpoint{3.439702in}{3.114246in}}%
\pgfpathlineto{\pgfqpoint{3.443181in}{3.116882in}}%
\pgfpathlineto{\pgfqpoint{3.448533in}{3.124482in}}%
\pgfpathlineto{\pgfqpoint{3.456294in}{3.134975in}}%
\pgfpathlineto{\pgfqpoint{3.460040in}{3.136513in}}%
\pgfpathlineto{\pgfqpoint{3.463251in}{3.135314in}}%
\pgfpathlineto{\pgfqpoint{3.467266in}{3.130955in}}%
\pgfpathlineto{\pgfqpoint{3.480378in}{3.114211in}}%
\pgfpathlineto{\pgfqpoint{3.483590in}{3.114191in}}%
\pgfpathlineto{\pgfqpoint{3.487069in}{3.116723in}}%
\pgfpathlineto{\pgfqpoint{3.492153in}{3.123815in}}%
\pgfpathlineto{\pgfqpoint{3.500449in}{3.135071in}}%
\pgfpathlineto{\pgfqpoint{3.503928in}{3.136512in}}%
\pgfpathlineto{\pgfqpoint{3.507139in}{3.135419in}}%
\pgfpathlineto{\pgfqpoint{3.511153in}{3.131159in}}%
\pgfpathlineto{\pgfqpoint{3.524801in}{3.114083in}}%
\pgfpathlineto{\pgfqpoint{3.528013in}{3.114347in}}%
\pgfpathlineto{\pgfqpoint{3.531491in}{3.117149in}}%
\pgfpathlineto{\pgfqpoint{3.537111in}{3.125300in}}%
\pgfpathlineto{\pgfqpoint{3.544337in}{3.134952in}}%
\pgfpathlineto{\pgfqpoint{3.548083in}{3.136514in}}%
\pgfpathlineto{\pgfqpoint{3.551294in}{3.135334in}}%
\pgfpathlineto{\pgfqpoint{3.555309in}{3.130994in}}%
\pgfpathlineto{\pgfqpoint{3.568689in}{3.114128in}}%
\pgfpathlineto{\pgfqpoint{3.571900in}{3.114284in}}%
\pgfpathlineto{\pgfqpoint{3.575379in}{3.116983in}}%
\pgfpathlineto{\pgfqpoint{3.580731in}{3.124630in}}%
\pgfpathlineto{\pgfqpoint{3.588492in}{3.135049in}}%
\pgfpathlineto{\pgfqpoint{3.591971in}{3.136512in}}%
\pgfpathlineto{\pgfqpoint{3.595182in}{3.135438in}}%
\pgfpathlineto{\pgfqpoint{3.598929in}{3.131559in}}%
\pgfpathlineto{\pgfqpoint{3.613112in}{3.114021in}}%
\pgfpathlineto{\pgfqpoint{3.616323in}{3.114460in}}%
\pgfpathlineto{\pgfqpoint{3.619802in}{3.117424in}}%
\pgfpathlineto{\pgfqpoint{3.625689in}{3.126117in}}%
\pgfpathlineto{\pgfqpoint{3.632380in}{3.134929in}}%
\pgfpathlineto{\pgfqpoint{3.636126in}{3.136514in}}%
\pgfpathlineto{\pgfqpoint{3.639337in}{3.135354in}}%
\pgfpathlineto{\pgfqpoint{3.643351in}{3.131032in}}%
\pgfpathlineto{\pgfqpoint{3.656732in}{3.114137in}}%
\pgfpathlineto{\pgfqpoint{3.659943in}{3.114272in}}%
\pgfpathlineto{\pgfqpoint{3.663422in}{3.116952in}}%
\pgfpathlineto{\pgfqpoint{3.668774in}{3.124585in}}%
\pgfpathlineto{\pgfqpoint{3.676535in}{3.135026in}}%
\pgfpathlineto{\pgfqpoint{3.680014in}{3.136511in}}%
\pgfpathlineto{\pgfqpoint{3.683225in}{3.135457in}}%
\pgfpathlineto{\pgfqpoint{3.686972in}{3.131596in}}%
\pgfpathlineto{\pgfqpoint{3.701422in}{3.113973in}}%
\pgfpathlineto{\pgfqpoint{3.704634in}{3.114586in}}%
\pgfpathlineto{\pgfqpoint{3.708380in}{3.118037in}}%
\pgfpathlineto{\pgfqpoint{3.715338in}{3.128609in}}%
\pgfpathlineto{\pgfqpoint{3.720958in}{3.135321in}}%
\pgfpathlineto{\pgfqpoint{3.724437in}{3.136508in}}%
\pgfpathlineto{\pgfqpoint{3.727648in}{3.135178in}}%
\pgfpathlineto{\pgfqpoint{3.731662in}{3.130698in}}%
\pgfpathlineto{\pgfqpoint{3.744507in}{3.114243in}}%
\pgfpathlineto{\pgfqpoint{3.747719in}{3.114161in}}%
\pgfpathlineto{\pgfqpoint{3.751197in}{3.116634in}}%
\pgfpathlineto{\pgfqpoint{3.756282in}{3.123681in}}%
\pgfpathlineto{\pgfqpoint{3.764578in}{3.135004in}}%
\pgfpathlineto{\pgfqpoint{3.768324in}{3.136513in}}%
\pgfpathlineto{\pgfqpoint{3.771536in}{3.135288in}}%
\pgfpathlineto{\pgfqpoint{3.775550in}{3.130905in}}%
\pgfpathlineto{\pgfqpoint{3.788663in}{3.114197in}}%
\pgfpathlineto{\pgfqpoint{3.791874in}{3.114204in}}%
\pgfpathlineto{\pgfqpoint{3.795353in}{3.116762in}}%
\pgfpathlineto{\pgfqpoint{3.800437in}{3.123873in}}%
\pgfpathlineto{\pgfqpoint{3.808733in}{3.135099in}}%
\pgfpathlineto{\pgfqpoint{3.812212in}{3.136513in}}%
\pgfpathlineto{\pgfqpoint{3.815423in}{3.135393in}}%
\pgfpathlineto{\pgfqpoint{3.819437in}{3.131109in}}%
\pgfpathlineto{\pgfqpoint{3.832818in}{3.114155in}}%
\pgfpathlineto{\pgfqpoint{3.836029in}{3.114250in}}%
\pgfpathlineto{\pgfqpoint{3.839508in}{3.116891in}}%
\pgfpathlineto{\pgfqpoint{3.844860in}{3.124495in}}%
\pgfpathlineto{\pgfqpoint{3.852621in}{3.134981in}}%
\pgfpathlineto{\pgfqpoint{3.856367in}{3.136513in}}%
\pgfpathlineto{\pgfqpoint{3.859579in}{3.135308in}}%
\pgfpathlineto{\pgfqpoint{3.863593in}{3.130944in}}%
\pgfpathlineto{\pgfqpoint{3.876706in}{3.114208in}}%
\pgfpathlineto{\pgfqpoint{3.879917in}{3.114194in}}%
\pgfpathlineto{\pgfqpoint{3.883396in}{3.116732in}}%
\pgfpathlineto{\pgfqpoint{3.888480in}{3.123828in}}%
\pgfpathlineto{\pgfqpoint{3.896776in}{3.135077in}}%
\pgfpathlineto{\pgfqpoint{3.900255in}{3.136513in}}%
\pgfpathlineto{\pgfqpoint{3.903466in}{3.135413in}}%
\pgfpathlineto{\pgfqpoint{3.907480in}{3.131148in}}%
\pgfpathlineto{\pgfqpoint{3.921128in}{3.114081in}}%
\pgfpathlineto{\pgfqpoint{3.924340in}{3.114351in}}%
\pgfpathlineto{\pgfqpoint{3.927819in}{3.117158in}}%
\pgfpathlineto{\pgfqpoint{3.933438in}{3.125313in}}%
\pgfpathlineto{\pgfqpoint{3.940664in}{3.134959in}}%
\pgfpathlineto{\pgfqpoint{3.944410in}{3.136514in}}%
\pgfpathlineto{\pgfqpoint{3.947622in}{3.135328in}}%
\pgfpathlineto{\pgfqpoint{3.951636in}{3.130982in}}%
\pgfpathlineto{\pgfqpoint{3.965016in}{3.114125in}}%
\pgfpathlineto{\pgfqpoint{3.968227in}{3.114287in}}%
\pgfpathlineto{\pgfqpoint{3.971706in}{3.116992in}}%
\pgfpathlineto{\pgfqpoint{3.977058in}{3.124643in}}%
\pgfpathlineto{\pgfqpoint{3.984819in}{3.135055in}}%
\pgfpathlineto{\pgfqpoint{3.988298in}{3.136512in}}%
\pgfpathlineto{\pgfqpoint{3.991509in}{3.135432in}}%
\pgfpathlineto{\pgfqpoint{3.995256in}{3.131548in}}%
\pgfpathlineto{\pgfqpoint{4.009439in}{3.114020in}}%
\pgfpathlineto{\pgfqpoint{4.012650in}{3.114464in}}%
\pgfpathlineto{\pgfqpoint{4.016129in}{3.117434in}}%
\pgfpathlineto{\pgfqpoint{4.022017in}{3.126130in}}%
\pgfpathlineto{\pgfqpoint{4.028707in}{3.134936in}}%
\pgfpathlineto{\pgfqpoint{4.032453in}{3.136514in}}%
\pgfpathlineto{\pgfqpoint{4.035665in}{3.135348in}}%
\pgfpathlineto{\pgfqpoint{4.039679in}{3.131021in}}%
\pgfpathlineto{\pgfqpoint{4.053059in}{3.114134in}}%
\pgfpathlineto{\pgfqpoint{4.056270in}{3.114275in}}%
\pgfpathlineto{\pgfqpoint{4.059749in}{3.116961in}}%
\pgfpathlineto{\pgfqpoint{4.065101in}{3.124598in}}%
\pgfpathlineto{\pgfqpoint{4.072862in}{3.135033in}}%
\pgfpathlineto{\pgfqpoint{4.076341in}{3.136511in}}%
\pgfpathlineto{\pgfqpoint{4.079552in}{3.135451in}}%
\pgfpathlineto{\pgfqpoint{4.083299in}{3.131585in}}%
\pgfpathlineto{\pgfqpoint{4.097482in}{3.114026in}}%
\pgfpathlineto{\pgfqpoint{4.100693in}{3.114451in}}%
\pgfpathlineto{\pgfqpoint{4.104172in}{3.117401in}}%
\pgfpathlineto{\pgfqpoint{4.110060in}{3.126085in}}%
\pgfpathlineto{\pgfqpoint{4.116750in}{3.134913in}}%
\pgfpathlineto{\pgfqpoint{4.120496in}{3.136514in}}%
\pgfpathlineto{\pgfqpoint{4.123708in}{3.135368in}}%
\pgfpathlineto{\pgfqpoint{4.127722in}{3.131060in}}%
\pgfpathlineto{\pgfqpoint{4.141102in}{3.114143in}}%
\pgfpathlineto{\pgfqpoint{4.144313in}{3.114264in}}%
\pgfpathlineto{\pgfqpoint{4.147792in}{3.116931in}}%
\pgfpathlineto{\pgfqpoint{4.153144in}{3.124553in}}%
\pgfpathlineto{\pgfqpoint{4.160905in}{3.135010in}}%
\pgfpathlineto{\pgfqpoint{4.164384in}{3.136510in}}%
\pgfpathlineto{\pgfqpoint{4.167595in}{3.135470in}}%
\pgfpathlineto{\pgfqpoint{4.171342in}{3.131622in}}%
\pgfpathlineto{\pgfqpoint{4.185793in}{3.113976in}}%
\pgfpathlineto{\pgfqpoint{4.189004in}{3.114575in}}%
\pgfpathlineto{\pgfqpoint{4.192750in}{3.118013in}}%
\pgfpathlineto{\pgfqpoint{4.199708in}{3.128578in}}%
\pgfpathlineto{\pgfqpoint{4.205328in}{3.135307in}}%
\pgfpathlineto{\pgfqpoint{4.208807in}{3.136509in}}%
\pgfpathlineto{\pgfqpoint{4.212018in}{3.135193in}}%
\pgfpathlineto{\pgfqpoint{4.216032in}{3.130726in}}%
\pgfpathlineto{\pgfqpoint{4.228877in}{3.114250in}}%
\pgfpathlineto{\pgfqpoint{4.232089in}{3.114155in}}%
\pgfpathlineto{\pgfqpoint{4.235568in}{3.116613in}}%
\pgfpathlineto{\pgfqpoint{4.240652in}{3.123649in}}%
\pgfpathlineto{\pgfqpoint{4.248948in}{3.134988in}}%
\pgfpathlineto{\pgfqpoint{4.252695in}{3.136513in}}%
\pgfpathlineto{\pgfqpoint{4.255906in}{3.135302in}}%
\pgfpathlineto{\pgfqpoint{4.259920in}{3.130933in}}%
\pgfpathlineto{\pgfqpoint{4.273033in}{3.114205in}}%
\pgfpathlineto{\pgfqpoint{4.276244in}{3.114197in}}%
\pgfpathlineto{\pgfqpoint{4.279723in}{3.116740in}}%
\pgfpathlineto{\pgfqpoint{4.284807in}{3.123841in}}%
\pgfpathlineto{\pgfqpoint{4.293103in}{3.135083in}}%
\pgfpathlineto{\pgfqpoint{4.296582in}{3.136513in}}%
\pgfpathlineto{\pgfqpoint{4.299793in}{3.135407in}}%
\pgfpathlineto{\pgfqpoint{4.303808in}{3.131137in}}%
\pgfpathlineto{\pgfqpoint{4.317188in}{3.114162in}}%
\pgfpathlineto{\pgfqpoint{4.320399in}{3.114242in}}%
\pgfpathlineto{\pgfqpoint{4.323878in}{3.116869in}}%
\pgfpathlineto{\pgfqpoint{4.329230in}{3.124463in}}%
\pgfpathlineto{\pgfqpoint{4.336991in}{3.134965in}}%
\pgfpathlineto{\pgfqpoint{4.340737in}{3.136514in}}%
\pgfpathlineto{\pgfqpoint{4.343949in}{3.135323in}}%
\pgfpathlineto{\pgfqpoint{4.347963in}{3.130971in}}%
\pgfpathlineto{\pgfqpoint{4.361076in}{3.114215in}}%
\pgfpathlineto{\pgfqpoint{4.364287in}{3.114186in}}%
\pgfpathlineto{\pgfqpoint{4.367766in}{3.116710in}}%
\pgfpathlineto{\pgfqpoint{4.372850in}{3.123796in}}%
\pgfpathlineto{\pgfqpoint{4.381146in}{3.135061in}}%
\pgfpathlineto{\pgfqpoint{4.384625in}{3.136512in}}%
\pgfpathlineto{\pgfqpoint{4.387836in}{3.135427in}}%
\pgfpathlineto{\pgfqpoint{4.391583in}{3.131537in}}%
\pgfpathlineto{\pgfqpoint{4.405766in}{3.114018in}}%
\pgfpathlineto{\pgfqpoint{4.408977in}{3.114468in}}%
\pgfpathlineto{\pgfqpoint{4.412456in}{3.117443in}}%
\pgfpathlineto{\pgfqpoint{4.418344in}{3.126143in}}%
\pgfpathlineto{\pgfqpoint{4.425034in}{3.134942in}}%
\pgfpathlineto{\pgfqpoint{4.428780in}{3.136514in}}%
\pgfpathlineto{\pgfqpoint{4.431992in}{3.135343in}}%
\pgfpathlineto{\pgfqpoint{4.436006in}{3.131010in}}%
\pgfpathlineto{\pgfqpoint{4.449386in}{3.114132in}}%
\pgfpathlineto{\pgfqpoint{4.452598in}{3.114279in}}%
\pgfpathlineto{\pgfqpoint{4.456076in}{3.116970in}}%
\pgfpathlineto{\pgfqpoint{4.461429in}{3.124611in}}%
\pgfpathlineto{\pgfqpoint{4.469189in}{3.135039in}}%
\pgfpathlineto{\pgfqpoint{4.472668in}{3.136511in}}%
\pgfpathlineto{\pgfqpoint{4.475879in}{3.135446in}}%
\pgfpathlineto{\pgfqpoint{4.479626in}{3.131575in}}%
\pgfpathlineto{\pgfqpoint{4.493809in}{3.114024in}}%
\pgfpathlineto{\pgfqpoint{4.497020in}{3.114455in}}%
\pgfpathlineto{\pgfqpoint{4.500499in}{3.117410in}}%
\pgfpathlineto{\pgfqpoint{4.506387in}{3.126098in}}%
\pgfpathlineto{\pgfqpoint{4.513077in}{3.134919in}}%
\pgfpathlineto{\pgfqpoint{4.516823in}{3.136514in}}%
\pgfpathlineto{\pgfqpoint{4.520035in}{3.135363in}}%
\pgfpathlineto{\pgfqpoint{4.524049in}{3.131049in}}%
\pgfpathlineto{\pgfqpoint{4.537429in}{3.114141in}}%
\pgfpathlineto{\pgfqpoint{4.540641in}{3.114267in}}%
\pgfpathlineto{\pgfqpoint{4.544119in}{3.116939in}}%
\pgfpathlineto{\pgfqpoint{4.549472in}{3.124566in}}%
\pgfpathlineto{\pgfqpoint{4.557232in}{3.135017in}}%
\pgfpathlineto{\pgfqpoint{4.560711in}{3.136510in}}%
\pgfpathlineto{\pgfqpoint{4.563922in}{3.135465in}}%
\pgfpathlineto{\pgfqpoint{4.567669in}{3.131612in}}%
\pgfpathlineto{\pgfqpoint{4.582120in}{3.113975in}}%
\pgfpathlineto{\pgfqpoint{4.585331in}{3.114580in}}%
\pgfpathlineto{\pgfqpoint{4.589078in}{3.118022in}}%
\pgfpathlineto{\pgfqpoint{4.596035in}{3.128591in}}%
\pgfpathlineto{\pgfqpoint{4.601655in}{3.135312in}}%
\pgfpathlineto{\pgfqpoint{4.605134in}{3.136508in}}%
\pgfpathlineto{\pgfqpoint{4.608345in}{3.135187in}}%
\pgfpathlineto{\pgfqpoint{4.612359in}{3.130714in}}%
\pgfpathlineto{\pgfqpoint{4.625205in}{3.114247in}}%
\pgfpathlineto{\pgfqpoint{4.628416in}{3.114157in}}%
\pgfpathlineto{\pgfqpoint{4.631895in}{3.116622in}}%
\pgfpathlineto{\pgfqpoint{4.636979in}{3.123662in}}%
\pgfpathlineto{\pgfqpoint{4.645275in}{3.134994in}}%
\pgfpathlineto{\pgfqpoint{4.649022in}{3.136513in}}%
\pgfpathlineto{\pgfqpoint{4.652233in}{3.135297in}}%
\pgfpathlineto{\pgfqpoint{4.656247in}{3.130921in}}%
\pgfpathlineto{\pgfqpoint{4.669360in}{3.114202in}}%
\pgfpathlineto{\pgfqpoint{4.672571in}{3.114199in}}%
\pgfpathlineto{\pgfqpoint{4.676050in}{3.116749in}}%
\pgfpathlineto{\pgfqpoint{4.681135in}{3.123853in}}%
\pgfpathlineto{\pgfqpoint{4.689430in}{3.135090in}}%
\pgfpathlineto{\pgfqpoint{4.692909in}{3.136513in}}%
\pgfpathlineto{\pgfqpoint{4.696121in}{3.135402in}}%
\pgfpathlineto{\pgfqpoint{4.700135in}{3.131126in}}%
\pgfpathlineto{\pgfqpoint{4.713515in}{3.114159in}}%
\pgfpathlineto{\pgfqpoint{4.716726in}{3.114245in}}%
\pgfpathlineto{\pgfqpoint{4.720205in}{3.116878in}}%
\pgfpathlineto{\pgfqpoint{4.725558in}{3.124476in}}%
\pgfpathlineto{\pgfqpoint{4.733318in}{3.134972in}}%
\pgfpathlineto{\pgfqpoint{4.737065in}{3.136513in}}%
\pgfpathlineto{\pgfqpoint{4.740276in}{3.135317in}}%
\pgfpathlineto{\pgfqpoint{4.744290in}{3.130960in}}%
\pgfpathlineto{\pgfqpoint{4.757403in}{3.114212in}}%
\pgfpathlineto{\pgfqpoint{4.760614in}{3.114189in}}%
\pgfpathlineto{\pgfqpoint{4.764093in}{3.116719in}}%
\pgfpathlineto{\pgfqpoint{4.769178in}{3.123809in}}%
\pgfpathlineto{\pgfqpoint{4.777473in}{3.135068in}}%
\pgfpathlineto{\pgfqpoint{4.780952in}{3.136512in}}%
\pgfpathlineto{\pgfqpoint{4.784164in}{3.135421in}}%
\pgfpathlineto{\pgfqpoint{4.787910in}{3.131527in}}%
\pgfpathlineto{\pgfqpoint{4.802093in}{3.114016in}}%
\pgfpathlineto{\pgfqpoint{4.805305in}{3.114473in}}%
\pgfpathlineto{\pgfqpoint{4.808784in}{3.117452in}}%
\pgfpathlineto{\pgfqpoint{4.814671in}{3.126155in}}%
\pgfpathlineto{\pgfqpoint{4.821361in}{3.134949in}}%
\pgfpathlineto{\pgfqpoint{4.825108in}{3.136514in}}%
\pgfpathlineto{\pgfqpoint{4.828319in}{3.135337in}}%
\pgfpathlineto{\pgfqpoint{4.832333in}{3.130999in}}%
\pgfpathlineto{\pgfqpoint{4.845713in}{3.114129in}}%
\pgfpathlineto{\pgfqpoint{4.848925in}{3.114282in}}%
\pgfpathlineto{\pgfqpoint{4.852404in}{3.116979in}}%
\pgfpathlineto{\pgfqpoint{4.857756in}{3.124623in}}%
\pgfpathlineto{\pgfqpoint{4.865516in}{3.135046in}}%
\pgfpathlineto{\pgfqpoint{4.868995in}{3.136512in}}%
\pgfpathlineto{\pgfqpoint{4.872207in}{3.135441in}}%
\pgfpathlineto{\pgfqpoint{4.875953in}{3.131564in}}%
\pgfpathlineto{\pgfqpoint{4.890136in}{3.114022in}}%
\pgfpathlineto{\pgfqpoint{4.893348in}{3.114458in}}%
\pgfpathlineto{\pgfqpoint{4.896827in}{3.117420in}}%
\pgfpathlineto{\pgfqpoint{4.902714in}{3.126110in}}%
\pgfpathlineto{\pgfqpoint{4.909404in}{3.134926in}}%
\pgfpathlineto{\pgfqpoint{4.913151in}{3.136514in}}%
\pgfpathlineto{\pgfqpoint{4.916362in}{3.135357in}}%
\pgfpathlineto{\pgfqpoint{4.920376in}{3.131038in}}%
\pgfpathlineto{\pgfqpoint{4.933756in}{3.114138in}}%
\pgfpathlineto{\pgfqpoint{4.936968in}{3.114270in}}%
\pgfpathlineto{\pgfqpoint{4.940447in}{3.116948in}}%
\pgfpathlineto{\pgfqpoint{4.945799in}{3.124578in}}%
\pgfpathlineto{\pgfqpoint{4.953559in}{3.135023in}}%
\pgfpathlineto{\pgfqpoint{4.957038in}{3.136511in}}%
\pgfpathlineto{\pgfqpoint{4.960250in}{3.135460in}}%
\pgfpathlineto{\pgfqpoint{4.963996in}{3.131601in}}%
\pgfpathlineto{\pgfqpoint{4.978447in}{3.113973in}}%
\pgfpathlineto{\pgfqpoint{4.981658in}{3.114584in}}%
\pgfpathlineto{\pgfqpoint{4.985405in}{3.118032in}}%
\pgfpathlineto{\pgfqpoint{4.992363in}{3.128603in}}%
\pgfpathlineto{\pgfqpoint{4.997982in}{3.135318in}}%
\pgfpathlineto{\pgfqpoint{5.001461in}{3.136508in}}%
\pgfpathlineto{\pgfqpoint{5.004672in}{3.135181in}}%
\pgfpathlineto{\pgfqpoint{5.008687in}{3.130703in}}%
\pgfpathlineto{\pgfqpoint{5.021532in}{3.114244in}}%
\pgfpathlineto{\pgfqpoint{5.024743in}{3.114160in}}%
\pgfpathlineto{\pgfqpoint{5.028222in}{3.116630in}}%
\pgfpathlineto{\pgfqpoint{5.033307in}{3.123674in}}%
\pgfpathlineto{\pgfqpoint{5.041602in}{3.135001in}}%
\pgfpathlineto{\pgfqpoint{5.045349in}{3.136513in}}%
\pgfpathlineto{\pgfqpoint{5.048560in}{3.135291in}}%
\pgfpathlineto{\pgfqpoint{5.052574in}{3.130910in}}%
\pgfpathlineto{\pgfqpoint{5.065687in}{3.114199in}}%
\pgfpathlineto{\pgfqpoint{5.068898in}{3.114202in}}%
\pgfpathlineto{\pgfqpoint{5.072377in}{3.116757in}}%
\pgfpathlineto{\pgfqpoint{5.077462in}{3.123866in}}%
\pgfpathlineto{\pgfqpoint{5.085758in}{3.135096in}}%
\pgfpathlineto{\pgfqpoint{5.089237in}{3.136513in}}%
\pgfpathlineto{\pgfqpoint{5.092448in}{3.135396in}}%
\pgfpathlineto{\pgfqpoint{5.096462in}{3.131115in}}%
\pgfpathlineto{\pgfqpoint{5.109842in}{3.114157in}}%
\pgfpathlineto{\pgfqpoint{5.113054in}{3.114248in}}%
\pgfpathlineto{\pgfqpoint{5.116533in}{3.116887in}}%
\pgfpathlineto{\pgfqpoint{5.121885in}{3.124488in}}%
\pgfpathlineto{\pgfqpoint{5.129645in}{3.134978in}}%
\pgfpathlineto{\pgfqpoint{5.133392in}{3.136513in}}%
\pgfpathlineto{\pgfqpoint{5.136603in}{3.135311in}}%
\pgfpathlineto{\pgfqpoint{5.140617in}{3.130949in}}%
\pgfpathlineto{\pgfqpoint{5.153730in}{3.114209in}}%
\pgfpathlineto{\pgfqpoint{5.156941in}{3.114192in}}%
\pgfpathlineto{\pgfqpoint{5.160420in}{3.116727in}}%
\pgfpathlineto{\pgfqpoint{5.165505in}{3.123821in}}%
\pgfpathlineto{\pgfqpoint{5.173801in}{3.135074in}}%
\pgfpathlineto{\pgfqpoint{5.177280in}{3.136513in}}%
\pgfpathlineto{\pgfqpoint{5.180491in}{3.135416in}}%
\pgfpathlineto{\pgfqpoint{5.184505in}{3.131153in}}%
\pgfpathlineto{\pgfqpoint{5.198153in}{3.114082in}}%
\pgfpathlineto{\pgfqpoint{5.201364in}{3.114349in}}%
\pgfpathlineto{\pgfqpoint{5.204843in}{3.117153in}}%
\pgfpathlineto{\pgfqpoint{5.210463in}{3.125306in}}%
\pgfpathlineto{\pgfqpoint{5.217688in}{3.134955in}}%
\pgfpathlineto{\pgfqpoint{5.221435in}{3.136514in}}%
\pgfpathlineto{\pgfqpoint{5.224646in}{3.135331in}}%
\pgfpathlineto{\pgfqpoint{5.228660in}{3.130988in}}%
\pgfpathlineto{\pgfqpoint{5.242041in}{3.114126in}}%
\pgfpathlineto{\pgfqpoint{5.245252in}{3.114285in}}%
\pgfpathlineto{\pgfqpoint{5.248731in}{3.116988in}}%
\pgfpathlineto{\pgfqpoint{5.254083in}{3.124636in}}%
\pgfpathlineto{\pgfqpoint{5.261844in}{3.135052in}}%
\pgfpathlineto{\pgfqpoint{5.265322in}{3.136512in}}%
\pgfpathlineto{\pgfqpoint{5.268534in}{3.135435in}}%
\pgfpathlineto{\pgfqpoint{5.272280in}{3.131553in}}%
\pgfpathlineto{\pgfqpoint{5.286464in}{3.114020in}}%
\pgfpathlineto{\pgfqpoint{5.289675in}{3.114462in}}%
\pgfpathlineto{\pgfqpoint{5.293154in}{3.117429in}}%
\pgfpathlineto{\pgfqpoint{5.299041in}{3.126123in}}%
\pgfpathlineto{\pgfqpoint{5.305731in}{3.134932in}}%
\pgfpathlineto{\pgfqpoint{5.309478in}{3.136514in}}%
\pgfpathlineto{\pgfqpoint{5.312689in}{3.135351in}}%
\pgfpathlineto{\pgfqpoint{5.316703in}{3.131027in}}%
\pgfpathlineto{\pgfqpoint{5.330084in}{3.114135in}}%
\pgfpathlineto{\pgfqpoint{5.333295in}{3.114274in}}%
\pgfpathlineto{\pgfqpoint{5.336774in}{3.116957in}}%
\pgfpathlineto{\pgfqpoint{5.342126in}{3.124591in}}%
\pgfpathlineto{\pgfqpoint{5.349887in}{3.135030in}}%
\pgfpathlineto{\pgfqpoint{5.353365in}{3.136511in}}%
\pgfpathlineto{\pgfqpoint{5.356577in}{3.135454in}}%
\pgfpathlineto{\pgfqpoint{5.360323in}{3.131591in}}%
\pgfpathlineto{\pgfqpoint{5.374506in}{3.114027in}}%
\pgfpathlineto{\pgfqpoint{5.377718in}{3.114449in}}%
\pgfpathlineto{\pgfqpoint{5.381197in}{3.117396in}}%
\pgfpathlineto{\pgfqpoint{5.387084in}{3.126078in}}%
\pgfpathlineto{\pgfqpoint{5.393774in}{3.134909in}}%
\pgfpathlineto{\pgfqpoint{5.397521in}{3.136514in}}%
\pgfpathlineto{\pgfqpoint{5.400732in}{3.135371in}}%
\pgfpathlineto{\pgfqpoint{5.404746in}{3.131065in}}%
\pgfpathlineto{\pgfqpoint{5.418127in}{3.114145in}}%
\pgfpathlineto{\pgfqpoint{5.421338in}{3.114262in}}%
\pgfpathlineto{\pgfqpoint{5.424817in}{3.116926in}}%
\pgfpathlineto{\pgfqpoint{5.430169in}{3.124546in}}%
\pgfpathlineto{\pgfqpoint{5.437930in}{3.135007in}}%
\pgfpathlineto{\pgfqpoint{5.441408in}{3.136510in}}%
\pgfpathlineto{\pgfqpoint{5.444620in}{3.135473in}}%
\pgfpathlineto{\pgfqpoint{5.448366in}{3.131628in}}%
\pgfpathlineto{\pgfqpoint{5.462817in}{3.113977in}}%
\pgfpathlineto{\pgfqpoint{5.466028in}{3.114573in}}%
\pgfpathlineto{\pgfqpoint{5.469775in}{3.118008in}}%
\pgfpathlineto{\pgfqpoint{5.476733in}{3.128572in}}%
\pgfpathlineto{\pgfqpoint{5.482352in}{3.135304in}}%
\pgfpathlineto{\pgfqpoint{5.485831in}{3.136509in}}%
\pgfpathlineto{\pgfqpoint{5.489043in}{3.135196in}}%
\pgfpathlineto{\pgfqpoint{5.493057in}{3.130731in}}%
\pgfpathlineto{\pgfqpoint{5.505902in}{3.114252in}}%
\pgfpathlineto{\pgfqpoint{5.509113in}{3.114153in}}%
\pgfpathlineto{\pgfqpoint{5.512592in}{3.116609in}}%
\pgfpathlineto{\pgfqpoint{5.517677in}{3.123643in}}%
\pgfpathlineto{\pgfqpoint{5.525973in}{3.134985in}}%
\pgfpathlineto{\pgfqpoint{5.529719in}{3.136513in}}%
\pgfpathlineto{\pgfqpoint{5.532930in}{3.135305in}}%
\pgfpathlineto{\pgfqpoint{5.536944in}{3.130938in}}%
\pgfpathlineto{\pgfqpoint{5.550057in}{3.114206in}}%
\pgfpathlineto{\pgfqpoint{5.553269in}{3.114195in}}%
\pgfpathlineto{\pgfqpoint{5.556747in}{3.116736in}}%
\pgfpathlineto{\pgfqpoint{5.561832in}{3.123834in}}%
\pgfpathlineto{\pgfqpoint{5.570128in}{3.135080in}}%
\pgfpathlineto{\pgfqpoint{5.573607in}{3.136513in}}%
\pgfpathlineto{\pgfqpoint{5.576818in}{3.135410in}}%
\pgfpathlineto{\pgfqpoint{5.580832in}{3.131142in}}%
\pgfpathlineto{\pgfqpoint{5.594480in}{3.114080in}}%
\pgfpathlineto{\pgfqpoint{5.597691in}{3.114352in}}%
\pgfpathlineto{\pgfqpoint{5.601170in}{3.117162in}}%
\pgfpathlineto{\pgfqpoint{5.606790in}{3.125319in}}%
\pgfpathlineto{\pgfqpoint{5.614015in}{3.134962in}}%
\pgfpathlineto{\pgfqpoint{5.617762in}{3.136514in}}%
\pgfpathlineto{\pgfqpoint{5.620973in}{3.135325in}}%
\pgfpathlineto{\pgfqpoint{5.624987in}{3.130977in}}%
\pgfpathlineto{\pgfqpoint{5.638100in}{3.114217in}}%
\pgfpathlineto{\pgfqpoint{5.641311in}{3.114185in}}%
\pgfpathlineto{\pgfqpoint{5.644790in}{3.116706in}}%
\pgfpathlineto{\pgfqpoint{5.649875in}{3.123789in}}%
\pgfpathlineto{\pgfqpoint{5.658171in}{3.135058in}}%
\pgfpathlineto{\pgfqpoint{5.661650in}{3.136512in}}%
\pgfpathlineto{\pgfqpoint{5.664861in}{3.135430in}}%
\pgfpathlineto{\pgfqpoint{5.668607in}{3.131543in}}%
\pgfpathlineto{\pgfqpoint{5.682791in}{3.114019in}}%
\pgfpathlineto{\pgfqpoint{5.686002in}{3.114466in}}%
\pgfpathlineto{\pgfqpoint{5.689481in}{3.117438in}}%
\pgfpathlineto{\pgfqpoint{5.695368in}{3.126136in}}%
\pgfpathlineto{\pgfqpoint{5.702058in}{3.134939in}}%
\pgfpathlineto{\pgfqpoint{5.705805in}{3.136514in}}%
\pgfpathlineto{\pgfqpoint{5.709016in}{3.135345in}}%
\pgfpathlineto{\pgfqpoint{5.713030in}{3.131016in}}%
\pgfpathlineto{\pgfqpoint{5.726411in}{3.114133in}}%
\pgfpathlineto{\pgfqpoint{5.729622in}{3.114277in}}%
\pgfpathlineto{\pgfqpoint{5.733101in}{3.116966in}}%
\pgfpathlineto{\pgfqpoint{5.738453in}{3.124604in}}%
\pgfpathlineto{\pgfqpoint{5.746214in}{3.135036in}}%
\pgfpathlineto{\pgfqpoint{5.749693in}{3.136511in}}%
\pgfpathlineto{\pgfqpoint{5.752904in}{3.135449in}}%
\pgfpathlineto{\pgfqpoint{5.756650in}{3.131580in}}%
\pgfpathlineto{\pgfqpoint{5.770834in}{3.114025in}}%
\pgfpathlineto{\pgfqpoint{5.774045in}{3.114453in}}%
\pgfpathlineto{\pgfqpoint{5.777524in}{3.117406in}}%
\pgfpathlineto{\pgfqpoint{5.783411in}{3.126091in}}%
\pgfpathlineto{\pgfqpoint{5.790101in}{3.134916in}}%
\pgfpathlineto{\pgfqpoint{5.793848in}{3.136514in}}%
\pgfpathlineto{\pgfqpoint{5.797059in}{3.135365in}}%
\pgfpathlineto{\pgfqpoint{5.801073in}{3.131054in}}%
\pgfpathlineto{\pgfqpoint{5.814454in}{3.114142in}}%
\pgfpathlineto{\pgfqpoint{5.817665in}{3.114266in}}%
\pgfpathlineto{\pgfqpoint{5.821144in}{3.116935in}}%
\pgfpathlineto{\pgfqpoint{5.826496in}{3.124559in}}%
\pgfpathlineto{\pgfqpoint{5.834257in}{3.135014in}}%
\pgfpathlineto{\pgfqpoint{5.837736in}{3.136510in}}%
\pgfpathlineto{\pgfqpoint{5.840947in}{3.135468in}}%
\pgfpathlineto{\pgfqpoint{5.844693in}{3.131617in}}%
\pgfpathlineto{\pgfqpoint{5.859144in}{3.113975in}}%
\pgfpathlineto{\pgfqpoint{5.862356in}{3.114578in}}%
\pgfpathlineto{\pgfqpoint{5.866102in}{3.118018in}}%
\pgfpathlineto{\pgfqpoint{5.873060in}{3.128584in}}%
\pgfpathlineto{\pgfqpoint{5.878680in}{3.135310in}}%
\pgfpathlineto{\pgfqpoint{5.882159in}{3.136508in}}%
\pgfpathlineto{\pgfqpoint{5.885370in}{3.135190in}}%
\pgfpathlineto{\pgfqpoint{5.889384in}{3.130720in}}%
\pgfpathlineto{\pgfqpoint{5.902229in}{3.114249in}}%
\pgfpathlineto{\pgfqpoint{5.905440in}{3.114156in}}%
\pgfpathlineto{\pgfqpoint{5.908919in}{3.116618in}}%
\pgfpathlineto{\pgfqpoint{5.914004in}{3.123655in}}%
\pgfpathlineto{\pgfqpoint{5.922300in}{3.134991in}}%
\pgfpathlineto{\pgfqpoint{5.926046in}{3.136513in}}%
\pgfpathlineto{\pgfqpoint{5.929258in}{3.135299in}}%
\pgfpathlineto{\pgfqpoint{5.933272in}{3.130927in}}%
\pgfpathlineto{\pgfqpoint{5.946384in}{3.114203in}}%
\pgfpathlineto{\pgfqpoint{5.949596in}{3.114198in}}%
\pgfpathlineto{\pgfqpoint{5.953075in}{3.116744in}}%
\pgfpathlineto{\pgfqpoint{5.958159in}{3.123847in}}%
\pgfpathlineto{\pgfqpoint{5.966455in}{3.135087in}}%
\pgfpathlineto{\pgfqpoint{5.969934in}{3.136513in}}%
\pgfpathlineto{\pgfqpoint{5.973145in}{3.135405in}}%
\pgfpathlineto{\pgfqpoint{5.977159in}{3.131131in}}%
\pgfpathlineto{\pgfqpoint{5.990540in}{3.114161in}}%
\pgfpathlineto{\pgfqpoint{5.993751in}{3.114243in}}%
\pgfpathlineto{\pgfqpoint{5.997230in}{3.116874in}}%
\pgfpathlineto{\pgfqpoint{6.002582in}{3.124469in}}%
\pgfpathlineto{\pgfqpoint{6.010343in}{3.134968in}}%
\pgfpathlineto{\pgfqpoint{6.014089in}{3.136513in}}%
\pgfpathlineto{\pgfqpoint{6.017300in}{3.135320in}}%
\pgfpathlineto{\pgfqpoint{6.021315in}{3.130966in}}%
\pgfpathlineto{\pgfqpoint{6.034427in}{3.114214in}}%
\pgfpathlineto{\pgfqpoint{6.037639in}{3.114188in}}%
\pgfpathlineto{\pgfqpoint{6.041118in}{3.116715in}}%
\pgfpathlineto{\pgfqpoint{6.046202in}{3.123802in}}%
\pgfpathlineto{\pgfqpoint{6.054498in}{3.135065in}}%
\pgfpathlineto{\pgfqpoint{6.057977in}{3.136512in}}%
\pgfpathlineto{\pgfqpoint{6.061188in}{3.135424in}}%
\pgfpathlineto{\pgfqpoint{6.064935in}{3.131532in}}%
\pgfpathlineto{\pgfqpoint{6.079118in}{3.114017in}}%
\pgfpathlineto{\pgfqpoint{6.082329in}{3.114470in}}%
\pgfpathlineto{\pgfqpoint{6.085808in}{3.117448in}}%
\pgfpathlineto{\pgfqpoint{6.091695in}{3.126149in}}%
\pgfpathlineto{\pgfqpoint{6.098386in}{3.134946in}}%
\pgfpathlineto{\pgfqpoint{6.102132in}{3.136514in}}%
\pgfpathlineto{\pgfqpoint{6.105343in}{3.135340in}}%
\pgfpathlineto{\pgfqpoint{6.109358in}{3.131005in}}%
\pgfpathlineto{\pgfqpoint{6.122738in}{3.114130in}}%
\pgfpathlineto{\pgfqpoint{6.125949in}{3.114280in}}%
\pgfpathlineto{\pgfqpoint{6.129428in}{3.116975in}}%
\pgfpathlineto{\pgfqpoint{6.134780in}{3.124617in}}%
\pgfpathlineto{\pgfqpoint{6.142541in}{3.135042in}}%
\pgfpathlineto{\pgfqpoint{6.146020in}{3.136511in}}%
\pgfpathlineto{\pgfqpoint{6.149231in}{3.135443in}}%
\pgfpathlineto{\pgfqpoint{6.152978in}{3.131569in}}%
\pgfpathlineto{\pgfqpoint{6.167161in}{3.114023in}}%
\pgfpathlineto{\pgfqpoint{6.170372in}{3.114456in}}%
\pgfpathlineto{\pgfqpoint{6.173851in}{3.117415in}}%
\pgfpathlineto{\pgfqpoint{6.179738in}{3.126104in}}%
\pgfpathlineto{\pgfqpoint{6.186429in}{3.134923in}}%
\pgfpathlineto{\pgfqpoint{6.190175in}{3.136514in}}%
\pgfpathlineto{\pgfqpoint{6.193386in}{3.135360in}}%
\pgfpathlineto{\pgfqpoint{6.197401in}{3.131043in}}%
\pgfpathlineto{\pgfqpoint{6.210781in}{3.114139in}}%
\pgfpathlineto{\pgfqpoint{6.213992in}{3.114269in}}%
\pgfpathlineto{\pgfqpoint{6.217471in}{3.116944in}}%
\pgfpathlineto{\pgfqpoint{6.222823in}{3.124572in}}%
\pgfpathlineto{\pgfqpoint{6.230584in}{3.135020in}}%
\pgfpathlineto{\pgfqpoint{6.234063in}{3.136511in}}%
\pgfpathlineto{\pgfqpoint{6.237274in}{3.135462in}}%
\pgfpathlineto{\pgfqpoint{6.241021in}{3.131607in}}%
\pgfpathlineto{\pgfqpoint{6.255471in}{3.113974in}}%
\pgfpathlineto{\pgfqpoint{6.258683in}{3.114582in}}%
\pgfpathlineto{\pgfqpoint{6.262429in}{3.118027in}}%
\pgfpathlineto{\pgfqpoint{6.269387in}{3.128597in}}%
\pgfpathlineto{\pgfqpoint{6.275007in}{3.135315in}}%
\pgfpathlineto{\pgfqpoint{6.278486in}{3.136508in}}%
\pgfpathlineto{\pgfqpoint{6.281697in}{3.135184in}}%
\pgfpathlineto{\pgfqpoint{6.285711in}{3.130709in}}%
\pgfpathlineto{\pgfqpoint{6.298556in}{3.114246in}}%
\pgfpathlineto{\pgfqpoint{6.301768in}{3.114159in}}%
\pgfpathlineto{\pgfqpoint{6.305246in}{3.116626in}}%
\pgfpathlineto{\pgfqpoint{6.310331in}{3.123668in}}%
\pgfpathlineto{\pgfqpoint{6.318627in}{3.134998in}}%
\pgfpathlineto{\pgfqpoint{6.322373in}{3.136513in}}%
\pgfpathlineto{\pgfqpoint{6.325585in}{3.135294in}}%
\pgfpathlineto{\pgfqpoint{6.329599in}{3.130916in}}%
\pgfpathlineto{\pgfqpoint{6.342712in}{3.114200in}}%
\pgfpathlineto{\pgfqpoint{6.345923in}{3.114201in}}%
\pgfpathlineto{\pgfqpoint{6.349402in}{3.116753in}}%
\pgfpathlineto{\pgfqpoint{6.354486in}{3.123860in}}%
\pgfpathlineto{\pgfqpoint{6.362782in}{3.135093in}}%
\pgfpathlineto{\pgfqpoint{6.366261in}{3.136513in}}%
\pgfpathlineto{\pgfqpoint{6.369472in}{3.135399in}}%
\pgfpathlineto{\pgfqpoint{6.373486in}{3.131120in}}%
\pgfpathlineto{\pgfqpoint{6.386867in}{3.114158in}}%
\pgfpathlineto{\pgfqpoint{6.390078in}{3.114246in}}%
\pgfpathlineto{\pgfqpoint{6.393557in}{3.116882in}}%
\pgfpathlineto{\pgfqpoint{6.398909in}{3.124482in}}%
\pgfpathlineto{\pgfqpoint{6.406670in}{3.134975in}}%
\pgfpathlineto{\pgfqpoint{6.410416in}{3.136513in}}%
\pgfpathlineto{\pgfqpoint{6.413628in}{3.135314in}}%
\pgfpathlineto{\pgfqpoint{6.417642in}{3.130955in}}%
\pgfpathlineto{\pgfqpoint{6.430755in}{3.114211in}}%
\pgfpathlineto{\pgfqpoint{6.433966in}{3.114191in}}%
\pgfpathlineto{\pgfqpoint{6.437445in}{3.116723in}}%
\pgfpathlineto{\pgfqpoint{6.442529in}{3.123815in}}%
\pgfpathlineto{\pgfqpoint{6.450825in}{3.135071in}}%
\pgfpathlineto{\pgfqpoint{6.454304in}{3.136512in}}%
\pgfpathlineto{\pgfqpoint{6.457515in}{3.135419in}}%
\pgfpathlineto{\pgfqpoint{6.461529in}{3.131159in}}%
\pgfpathlineto{\pgfqpoint{6.475177in}{3.114083in}}%
\pgfpathlineto{\pgfqpoint{6.478389in}{3.114347in}}%
\pgfpathlineto{\pgfqpoint{6.481868in}{3.117149in}}%
\pgfpathlineto{\pgfqpoint{6.487487in}{3.125300in}}%
\pgfpathlineto{\pgfqpoint{6.494713in}{3.134952in}}%
\pgfpathlineto{\pgfqpoint{6.498459in}{3.136514in}}%
\pgfpathlineto{\pgfqpoint{6.501671in}{3.135334in}}%
\pgfpathlineto{\pgfqpoint{6.505685in}{3.130994in}}%
\pgfpathlineto{\pgfqpoint{6.519065in}{3.114128in}}%
\pgfpathlineto{\pgfqpoint{6.522276in}{3.114284in}}%
\pgfpathlineto{\pgfqpoint{6.525755in}{3.116983in}}%
\pgfpathlineto{\pgfqpoint{6.531107in}{3.124630in}}%
\pgfpathlineto{\pgfqpoint{6.538868in}{3.135049in}}%
\pgfpathlineto{\pgfqpoint{6.542347in}{3.136512in}}%
\pgfpathlineto{\pgfqpoint{6.545558in}{3.135438in}}%
\pgfpathlineto{\pgfqpoint{6.549305in}{3.131559in}}%
\pgfpathlineto{\pgfqpoint{6.563488in}{3.114021in}}%
\pgfpathlineto{\pgfqpoint{6.566699in}{3.114460in}}%
\pgfpathlineto{\pgfqpoint{6.570178in}{3.117424in}}%
\pgfpathlineto{\pgfqpoint{6.576066in}{3.126117in}}%
\pgfpathlineto{\pgfqpoint{6.582756in}{3.134929in}}%
\pgfpathlineto{\pgfqpoint{6.586502in}{3.136514in}}%
\pgfpathlineto{\pgfqpoint{6.589714in}{3.135354in}}%
\pgfpathlineto{\pgfqpoint{6.593728in}{3.131032in}}%
\pgfpathlineto{\pgfqpoint{6.607108in}{3.114137in}}%
\pgfpathlineto{\pgfqpoint{6.610319in}{3.114272in}}%
\pgfpathlineto{\pgfqpoint{6.613798in}{3.116952in}}%
\pgfpathlineto{\pgfqpoint{6.619150in}{3.124585in}}%
\pgfpathlineto{\pgfqpoint{6.626911in}{3.135026in}}%
\pgfpathlineto{\pgfqpoint{6.630390in}{3.136511in}}%
\pgfpathlineto{\pgfqpoint{6.633601in}{3.135457in}}%
\pgfpathlineto{\pgfqpoint{6.637348in}{3.131596in}}%
\pgfpathlineto{\pgfqpoint{6.651799in}{3.113973in}}%
\pgfpathlineto{\pgfqpoint{6.655010in}{3.114586in}}%
\pgfpathlineto{\pgfqpoint{6.658756in}{3.118037in}}%
\pgfpathlineto{\pgfqpoint{6.663306in}{3.124778in}}%
\pgfpathlineto{\pgfqpoint{6.663306in}{3.124778in}}%
\pgfusepath{stroke}%
\end{pgfscope}%
\begin{pgfscope}%
\pgfpathrectangle{\pgfqpoint{0.467797in}{2.292089in}}{\pgfqpoint{6.490533in}{1.666241in}}%
\pgfusepath{clip}%
\pgfsetrectcap%
\pgfsetroundjoin%
\pgfsetlinewidth{1.505625pt}%
\definecolor{currentstroke}{rgb}{0.580392,0.403922,0.741176}%
\pgfsetstrokecolor{currentstroke}%
\pgfsetdash{}{0pt}%
\pgfpathmoveto{\pgfqpoint{0.762821in}{3.125209in}}%
\pgfpathlineto{\pgfqpoint{0.770046in}{3.134794in}}%
\pgfpathlineto{\pgfqpoint{0.773525in}{3.136186in}}%
\pgfpathlineto{\pgfqpoint{0.776737in}{3.134976in}}%
\pgfpathlineto{\pgfqpoint{0.780751in}{3.130538in}}%
\pgfpathlineto{\pgfqpoint{0.793328in}{3.114521in}}%
\pgfpathlineto{\pgfqpoint{0.796539in}{3.114554in}}%
\pgfpathlineto{\pgfqpoint{0.800018in}{3.117205in}}%
\pgfpathlineto{\pgfqpoint{0.805371in}{3.124872in}}%
\pgfpathlineto{\pgfqpoint{0.812596in}{3.134625in}}%
\pgfpathlineto{\pgfqpoint{0.816075in}{3.136181in}}%
\pgfpathlineto{\pgfqpoint{0.819286in}{3.135125in}}%
\pgfpathlineto{\pgfqpoint{0.823033in}{3.131198in}}%
\pgfpathlineto{\pgfqpoint{0.836681in}{3.114344in}}%
\pgfpathlineto{\pgfqpoint{0.839892in}{3.114824in}}%
\pgfpathlineto{\pgfqpoint{0.843371in}{3.117889in}}%
\pgfpathlineto{\pgfqpoint{0.849526in}{3.127114in}}%
\pgfpathlineto{\pgfqpoint{0.855681in}{3.134884in}}%
\pgfpathlineto{\pgfqpoint{0.859160in}{3.136184in}}%
\pgfpathlineto{\pgfqpoint{0.862371in}{3.134889in}}%
\pgfpathlineto{\pgfqpoint{0.866385in}{3.130374in}}%
\pgfpathlineto{\pgfqpoint{0.878695in}{3.114579in}}%
\pgfpathlineto{\pgfqpoint{0.881906in}{3.114498in}}%
\pgfpathlineto{\pgfqpoint{0.885118in}{3.116758in}}%
\pgfpathlineto{\pgfqpoint{0.889935in}{3.123342in}}%
\pgfpathlineto{\pgfqpoint{0.898498in}{3.134928in}}%
\pgfpathlineto{\pgfqpoint{0.901977in}{3.136182in}}%
\pgfpathlineto{\pgfqpoint{0.905188in}{3.134844in}}%
\pgfpathlineto{\pgfqpoint{0.909202in}{3.130291in}}%
\pgfpathlineto{\pgfqpoint{0.921245in}{3.114668in}}%
\pgfpathlineto{\pgfqpoint{0.924456in}{3.114429in}}%
\pgfpathlineto{\pgfqpoint{0.927667in}{3.116546in}}%
\pgfpathlineto{\pgfqpoint{0.932484in}{3.123010in}}%
\pgfpathlineto{\pgfqpoint{0.941315in}{3.134971in}}%
\pgfpathlineto{\pgfqpoint{0.944794in}{3.136179in}}%
\pgfpathlineto{\pgfqpoint{0.948006in}{3.134799in}}%
\pgfpathlineto{\pgfqpoint{0.952020in}{3.130207in}}%
\pgfpathlineto{\pgfqpoint{0.964062in}{3.114642in}}%
\pgfpathlineto{\pgfqpoint{0.967273in}{3.114447in}}%
\pgfpathlineto{\pgfqpoint{0.970485in}{3.116604in}}%
\pgfpathlineto{\pgfqpoint{0.975302in}{3.123102in}}%
\pgfpathlineto{\pgfqpoint{0.983865in}{3.134812in}}%
\pgfpathlineto{\pgfqpoint{0.987344in}{3.136186in}}%
\pgfpathlineto{\pgfqpoint{0.990555in}{3.134959in}}%
\pgfpathlineto{\pgfqpoint{0.994569in}{3.130506in}}%
\pgfpathlineto{\pgfqpoint{1.006879in}{3.114617in}}%
\pgfpathlineto{\pgfqpoint{1.010091in}{3.114466in}}%
\pgfpathlineto{\pgfqpoint{1.013302in}{3.116663in}}%
\pgfpathlineto{\pgfqpoint{1.018119in}{3.123194in}}%
\pgfpathlineto{\pgfqpoint{1.026682in}{3.134857in}}%
\pgfpathlineto{\pgfqpoint{1.030161in}{3.136185in}}%
\pgfpathlineto{\pgfqpoint{1.033372in}{3.134915in}}%
\pgfpathlineto{\pgfqpoint{1.037387in}{3.130423in}}%
\pgfpathlineto{\pgfqpoint{1.049696in}{3.114593in}}%
\pgfpathlineto{\pgfqpoint{1.052908in}{3.114486in}}%
\pgfpathlineto{\pgfqpoint{1.056119in}{3.116722in}}%
\pgfpathlineto{\pgfqpoint{1.060936in}{3.123286in}}%
\pgfpathlineto{\pgfqpoint{1.069499in}{3.134902in}}%
\pgfpathlineto{\pgfqpoint{1.072978in}{3.136183in}}%
\pgfpathlineto{\pgfqpoint{1.076190in}{3.134871in}}%
\pgfpathlineto{\pgfqpoint{1.080204in}{3.130341in}}%
\pgfpathlineto{\pgfqpoint{1.092246in}{3.114684in}}%
\pgfpathlineto{\pgfqpoint{1.095457in}{3.114419in}}%
\pgfpathlineto{\pgfqpoint{1.098669in}{3.116512in}}%
\pgfpathlineto{\pgfqpoint{1.103218in}{3.122534in}}%
\pgfpathlineto{\pgfqpoint{1.112317in}{3.134946in}}%
\pgfpathlineto{\pgfqpoint{1.115796in}{3.136181in}}%
\pgfpathlineto{\pgfqpoint{1.119007in}{3.134826in}}%
\pgfpathlineto{\pgfqpoint{1.123021in}{3.130257in}}%
\pgfpathlineto{\pgfqpoint{1.135063in}{3.114658in}}%
\pgfpathlineto{\pgfqpoint{1.138275in}{3.114436in}}%
\pgfpathlineto{\pgfqpoint{1.141486in}{3.116569in}}%
\pgfpathlineto{\pgfqpoint{1.146303in}{3.123047in}}%
\pgfpathlineto{\pgfqpoint{1.155134in}{3.134989in}}%
\pgfpathlineto{\pgfqpoint{1.158613in}{3.136177in}}%
\pgfpathlineto{\pgfqpoint{1.161824in}{3.134780in}}%
\pgfpathlineto{\pgfqpoint{1.165838in}{3.130174in}}%
\pgfpathlineto{\pgfqpoint{1.177881in}{3.114632in}}%
\pgfpathlineto{\pgfqpoint{1.181092in}{3.114455in}}%
\pgfpathlineto{\pgfqpoint{1.184303in}{3.116628in}}%
\pgfpathlineto{\pgfqpoint{1.189120in}{3.123139in}}%
\pgfpathlineto{\pgfqpoint{1.197684in}{3.134831in}}%
\pgfpathlineto{\pgfqpoint{1.201163in}{3.136186in}}%
\pgfpathlineto{\pgfqpoint{1.204374in}{3.134941in}}%
\pgfpathlineto{\pgfqpoint{1.208388in}{3.130473in}}%
\pgfpathlineto{\pgfqpoint{1.220698in}{3.114608in}}%
\pgfpathlineto{\pgfqpoint{1.223909in}{3.114474in}}%
\pgfpathlineto{\pgfqpoint{1.227120in}{3.116686in}}%
\pgfpathlineto{\pgfqpoint{1.231937in}{3.123231in}}%
\pgfpathlineto{\pgfqpoint{1.240501in}{3.134875in}}%
\pgfpathlineto{\pgfqpoint{1.243980in}{3.136184in}}%
\pgfpathlineto{\pgfqpoint{1.247191in}{3.134897in}}%
\pgfpathlineto{\pgfqpoint{1.251205in}{3.130390in}}%
\pgfpathlineto{\pgfqpoint{1.263515in}{3.114584in}}%
\pgfpathlineto{\pgfqpoint{1.266726in}{3.114494in}}%
\pgfpathlineto{\pgfqpoint{1.269938in}{3.116746in}}%
\pgfpathlineto{\pgfqpoint{1.274755in}{3.123323in}}%
\pgfpathlineto{\pgfqpoint{1.283318in}{3.134919in}}%
\pgfpathlineto{\pgfqpoint{1.286797in}{3.136182in}}%
\pgfpathlineto{\pgfqpoint{1.290008in}{3.134853in}}%
\pgfpathlineto{\pgfqpoint{1.294022in}{3.130307in}}%
\pgfpathlineto{\pgfqpoint{1.306065in}{3.114673in}}%
\pgfpathlineto{\pgfqpoint{1.309276in}{3.114426in}}%
\pgfpathlineto{\pgfqpoint{1.312487in}{3.116535in}}%
\pgfpathlineto{\pgfqpoint{1.317304in}{3.122992in}}%
\pgfpathlineto{\pgfqpoint{1.326135in}{3.134963in}}%
\pgfpathlineto{\pgfqpoint{1.329614in}{3.136179in}}%
\pgfpathlineto{\pgfqpoint{1.332826in}{3.134808in}}%
\pgfpathlineto{\pgfqpoint{1.336840in}{3.130224in}}%
\pgfpathlineto{\pgfqpoint{1.348882in}{3.114647in}}%
\pgfpathlineto{\pgfqpoint{1.352093in}{3.114443in}}%
\pgfpathlineto{\pgfqpoint{1.355305in}{3.116593in}}%
\pgfpathlineto{\pgfqpoint{1.360122in}{3.123084in}}%
\pgfpathlineto{\pgfqpoint{1.368685in}{3.134803in}}%
\pgfpathlineto{\pgfqpoint{1.372164in}{3.136186in}}%
\pgfpathlineto{\pgfqpoint{1.375375in}{3.134967in}}%
\pgfpathlineto{\pgfqpoint{1.379389in}{3.130522in}}%
\pgfpathlineto{\pgfqpoint{1.391967in}{3.114516in}}%
\pgfpathlineto{\pgfqpoint{1.395178in}{3.114558in}}%
\pgfpathlineto{\pgfqpoint{1.398657in}{3.117218in}}%
\pgfpathlineto{\pgfqpoint{1.404009in}{3.124890in}}%
\pgfpathlineto{\pgfqpoint{1.411235in}{3.134635in}}%
\pgfpathlineto{\pgfqpoint{1.414714in}{3.136182in}}%
\pgfpathlineto{\pgfqpoint{1.417925in}{3.135117in}}%
\pgfpathlineto{\pgfqpoint{1.421671in}{3.131182in}}%
\pgfpathlineto{\pgfqpoint{1.435319in}{3.114341in}}%
\pgfpathlineto{\pgfqpoint{1.438531in}{3.114830in}}%
\pgfpathlineto{\pgfqpoint{1.442010in}{3.117903in}}%
\pgfpathlineto{\pgfqpoint{1.448165in}{3.127132in}}%
\pgfpathlineto{\pgfqpoint{1.454320in}{3.134893in}}%
\pgfpathlineto{\pgfqpoint{1.457798in}{3.136184in}}%
\pgfpathlineto{\pgfqpoint{1.461010in}{3.134880in}}%
\pgfpathlineto{\pgfqpoint{1.465024in}{3.130357in}}%
\pgfpathlineto{\pgfqpoint{1.477334in}{3.114574in}}%
\pgfpathlineto{\pgfqpoint{1.480545in}{3.114502in}}%
\pgfpathlineto{\pgfqpoint{1.483756in}{3.116770in}}%
\pgfpathlineto{\pgfqpoint{1.488573in}{3.123360in}}%
\pgfpathlineto{\pgfqpoint{1.497137in}{3.134937in}}%
\pgfpathlineto{\pgfqpoint{1.500616in}{3.136181in}}%
\pgfpathlineto{\pgfqpoint{1.503827in}{3.134835in}}%
\pgfpathlineto{\pgfqpoint{1.507841in}{3.130274in}}%
\pgfpathlineto{\pgfqpoint{1.519883in}{3.114663in}}%
\pgfpathlineto{\pgfqpoint{1.523095in}{3.114433in}}%
\pgfpathlineto{\pgfqpoint{1.526306in}{3.116558in}}%
\pgfpathlineto{\pgfqpoint{1.531123in}{3.123028in}}%
\pgfpathlineto{\pgfqpoint{1.539954in}{3.134980in}}%
\pgfpathlineto{\pgfqpoint{1.543433in}{3.136178in}}%
\pgfpathlineto{\pgfqpoint{1.546644in}{3.134790in}}%
\pgfpathlineto{\pgfqpoint{1.550658in}{3.130191in}}%
\pgfpathlineto{\pgfqpoint{1.562701in}{3.114637in}}%
\pgfpathlineto{\pgfqpoint{1.565912in}{3.114451in}}%
\pgfpathlineto{\pgfqpoint{1.569123in}{3.116616in}}%
\pgfpathlineto{\pgfqpoint{1.573940in}{3.123120in}}%
\pgfpathlineto{\pgfqpoint{1.582504in}{3.134821in}}%
\pgfpathlineto{\pgfqpoint{1.585983in}{3.136186in}}%
\pgfpathlineto{\pgfqpoint{1.589194in}{3.134950in}}%
\pgfpathlineto{\pgfqpoint{1.593208in}{3.130489in}}%
\pgfpathlineto{\pgfqpoint{1.605518in}{3.114612in}}%
\pgfpathlineto{\pgfqpoint{1.608729in}{3.114470in}}%
\pgfpathlineto{\pgfqpoint{1.611941in}{3.116675in}}%
\pgfpathlineto{\pgfqpoint{1.616757in}{3.123213in}}%
\pgfpathlineto{\pgfqpoint{1.625321in}{3.134866in}}%
\pgfpathlineto{\pgfqpoint{1.628800in}{3.136185in}}%
\pgfpathlineto{\pgfqpoint{1.632011in}{3.134906in}}%
\pgfpathlineto{\pgfqpoint{1.636025in}{3.130407in}}%
\pgfpathlineto{\pgfqpoint{1.648335in}{3.114588in}}%
\pgfpathlineto{\pgfqpoint{1.651546in}{3.114490in}}%
\pgfpathlineto{\pgfqpoint{1.654758in}{3.116734in}}%
\pgfpathlineto{\pgfqpoint{1.659575in}{3.123305in}}%
\pgfpathlineto{\pgfqpoint{1.668138in}{3.134911in}}%
\pgfpathlineto{\pgfqpoint{1.671617in}{3.136183in}}%
\pgfpathlineto{\pgfqpoint{1.674828in}{3.134862in}}%
\pgfpathlineto{\pgfqpoint{1.678842in}{3.130324in}}%
\pgfpathlineto{\pgfqpoint{1.690885in}{3.114679in}}%
\pgfpathlineto{\pgfqpoint{1.694096in}{3.114422in}}%
\pgfpathlineto{\pgfqpoint{1.697307in}{3.116523in}}%
\pgfpathlineto{\pgfqpoint{1.701857in}{3.122552in}}%
\pgfpathlineto{\pgfqpoint{1.710955in}{3.134954in}}%
\pgfpathlineto{\pgfqpoint{1.714434in}{3.136180in}}%
\pgfpathlineto{\pgfqpoint{1.717646in}{3.134817in}}%
\pgfpathlineto{\pgfqpoint{1.721660in}{3.130241in}}%
\pgfpathlineto{\pgfqpoint{1.733702in}{3.114653in}}%
\pgfpathlineto{\pgfqpoint{1.736913in}{3.114440in}}%
\pgfpathlineto{\pgfqpoint{1.740125in}{3.116581in}}%
\pgfpathlineto{\pgfqpoint{1.744942in}{3.123065in}}%
\pgfpathlineto{\pgfqpoint{1.753505in}{3.134794in}}%
\pgfpathlineto{\pgfqpoint{1.756984in}{3.136186in}}%
\pgfpathlineto{\pgfqpoint{1.760195in}{3.134976in}}%
\pgfpathlineto{\pgfqpoint{1.764209in}{3.130538in}}%
\pgfpathlineto{\pgfqpoint{1.776787in}{3.114521in}}%
\pgfpathlineto{\pgfqpoint{1.779998in}{3.114554in}}%
\pgfpathlineto{\pgfqpoint{1.783477in}{3.117205in}}%
\pgfpathlineto{\pgfqpoint{1.788829in}{3.124872in}}%
\pgfpathlineto{\pgfqpoint{1.796055in}{3.134625in}}%
\pgfpathlineto{\pgfqpoint{1.799534in}{3.136181in}}%
\pgfpathlineto{\pgfqpoint{1.802745in}{3.135125in}}%
\pgfpathlineto{\pgfqpoint{1.806491in}{3.131198in}}%
\pgfpathlineto{\pgfqpoint{1.820139in}{3.114344in}}%
\pgfpathlineto{\pgfqpoint{1.823351in}{3.114824in}}%
\pgfpathlineto{\pgfqpoint{1.826830in}{3.117889in}}%
\pgfpathlineto{\pgfqpoint{1.832985in}{3.127114in}}%
\pgfpathlineto{\pgfqpoint{1.839140in}{3.134884in}}%
\pgfpathlineto{\pgfqpoint{1.842618in}{3.136184in}}%
\pgfpathlineto{\pgfqpoint{1.845830in}{3.134889in}}%
\pgfpathlineto{\pgfqpoint{1.849844in}{3.130374in}}%
\pgfpathlineto{\pgfqpoint{1.862154in}{3.114579in}}%
\pgfpathlineto{\pgfqpoint{1.865365in}{3.114498in}}%
\pgfpathlineto{\pgfqpoint{1.868576in}{3.116758in}}%
\pgfpathlineto{\pgfqpoint{1.873393in}{3.123342in}}%
\pgfpathlineto{\pgfqpoint{1.881957in}{3.134928in}}%
\pgfpathlineto{\pgfqpoint{1.885436in}{3.136182in}}%
\pgfpathlineto{\pgfqpoint{1.888647in}{3.134844in}}%
\pgfpathlineto{\pgfqpoint{1.892661in}{3.130291in}}%
\pgfpathlineto{\pgfqpoint{1.904703in}{3.114668in}}%
\pgfpathlineto{\pgfqpoint{1.907915in}{3.114429in}}%
\pgfpathlineto{\pgfqpoint{1.911126in}{3.116546in}}%
\pgfpathlineto{\pgfqpoint{1.915943in}{3.123010in}}%
\pgfpathlineto{\pgfqpoint{1.924774in}{3.134971in}}%
\pgfpathlineto{\pgfqpoint{1.928253in}{3.136179in}}%
\pgfpathlineto{\pgfqpoint{1.931464in}{3.134799in}}%
\pgfpathlineto{\pgfqpoint{1.935478in}{3.130207in}}%
\pgfpathlineto{\pgfqpoint{1.947521in}{3.114642in}}%
\pgfpathlineto{\pgfqpoint{1.950732in}{3.114447in}}%
\pgfpathlineto{\pgfqpoint{1.953943in}{3.116604in}}%
\pgfpathlineto{\pgfqpoint{1.958760in}{3.123102in}}%
\pgfpathlineto{\pgfqpoint{1.967324in}{3.134812in}}%
\pgfpathlineto{\pgfqpoint{1.970803in}{3.136186in}}%
\pgfpathlineto{\pgfqpoint{1.974014in}{3.134959in}}%
\pgfpathlineto{\pgfqpoint{1.978028in}{3.130506in}}%
\pgfpathlineto{\pgfqpoint{1.990338in}{3.114617in}}%
\pgfpathlineto{\pgfqpoint{1.993549in}{3.114466in}}%
\pgfpathlineto{\pgfqpoint{1.996761in}{3.116663in}}%
\pgfpathlineto{\pgfqpoint{2.001578in}{3.123194in}}%
\pgfpathlineto{\pgfqpoint{2.010141in}{3.134857in}}%
\pgfpathlineto{\pgfqpoint{2.013620in}{3.136185in}}%
\pgfpathlineto{\pgfqpoint{2.016831in}{3.134915in}}%
\pgfpathlineto{\pgfqpoint{2.020845in}{3.130423in}}%
\pgfpathlineto{\pgfqpoint{2.033155in}{3.114593in}}%
\pgfpathlineto{\pgfqpoint{2.036367in}{3.114486in}}%
\pgfpathlineto{\pgfqpoint{2.039578in}{3.116722in}}%
\pgfpathlineto{\pgfqpoint{2.044395in}{3.123286in}}%
\pgfpathlineto{\pgfqpoint{2.052958in}{3.134902in}}%
\pgfpathlineto{\pgfqpoint{2.056437in}{3.136183in}}%
\pgfpathlineto{\pgfqpoint{2.059648in}{3.134871in}}%
\pgfpathlineto{\pgfqpoint{2.063663in}{3.130341in}}%
\pgfpathlineto{\pgfqpoint{2.075705in}{3.114684in}}%
\pgfpathlineto{\pgfqpoint{2.078916in}{3.114419in}}%
\pgfpathlineto{\pgfqpoint{2.082127in}{3.116512in}}%
\pgfpathlineto{\pgfqpoint{2.086677in}{3.122534in}}%
\pgfpathlineto{\pgfqpoint{2.095775in}{3.134946in}}%
\pgfpathlineto{\pgfqpoint{2.099254in}{3.136181in}}%
\pgfpathlineto{\pgfqpoint{2.102466in}{3.134826in}}%
\pgfpathlineto{\pgfqpoint{2.106480in}{3.130257in}}%
\pgfpathlineto{\pgfqpoint{2.118522in}{3.114658in}}%
\pgfpathlineto{\pgfqpoint{2.121733in}{3.114436in}}%
\pgfpathlineto{\pgfqpoint{2.124945in}{3.116569in}}%
\pgfpathlineto{\pgfqpoint{2.129762in}{3.123047in}}%
\pgfpathlineto{\pgfqpoint{2.138593in}{3.134989in}}%
\pgfpathlineto{\pgfqpoint{2.142072in}{3.136177in}}%
\pgfpathlineto{\pgfqpoint{2.145283in}{3.134780in}}%
\pgfpathlineto{\pgfqpoint{2.149297in}{3.130174in}}%
\pgfpathlineto{\pgfqpoint{2.161339in}{3.114632in}}%
\pgfpathlineto{\pgfqpoint{2.164551in}{3.114455in}}%
\pgfpathlineto{\pgfqpoint{2.167762in}{3.116628in}}%
\pgfpathlineto{\pgfqpoint{2.172579in}{3.123139in}}%
\pgfpathlineto{\pgfqpoint{2.181142in}{3.134831in}}%
\pgfpathlineto{\pgfqpoint{2.184621in}{3.136186in}}%
\pgfpathlineto{\pgfqpoint{2.187833in}{3.134941in}}%
\pgfpathlineto{\pgfqpoint{2.191847in}{3.130473in}}%
\pgfpathlineto{\pgfqpoint{2.204157in}{3.114608in}}%
\pgfpathlineto{\pgfqpoint{2.207368in}{3.114474in}}%
\pgfpathlineto{\pgfqpoint{2.210579in}{3.116686in}}%
\pgfpathlineto{\pgfqpoint{2.215396in}{3.123231in}}%
\pgfpathlineto{\pgfqpoint{2.223960in}{3.134875in}}%
\pgfpathlineto{\pgfqpoint{2.227439in}{3.136184in}}%
\pgfpathlineto{\pgfqpoint{2.230650in}{3.134897in}}%
\pgfpathlineto{\pgfqpoint{2.234664in}{3.130390in}}%
\pgfpathlineto{\pgfqpoint{2.246974in}{3.114584in}}%
\pgfpathlineto{\pgfqpoint{2.250185in}{3.114494in}}%
\pgfpathlineto{\pgfqpoint{2.253396in}{3.116746in}}%
\pgfpathlineto{\pgfqpoint{2.258213in}{3.123323in}}%
\pgfpathlineto{\pgfqpoint{2.266777in}{3.134919in}}%
\pgfpathlineto{\pgfqpoint{2.270256in}{3.136182in}}%
\pgfpathlineto{\pgfqpoint{2.273467in}{3.134853in}}%
\pgfpathlineto{\pgfqpoint{2.277481in}{3.130307in}}%
\pgfpathlineto{\pgfqpoint{2.289524in}{3.114673in}}%
\pgfpathlineto{\pgfqpoint{2.292735in}{3.114426in}}%
\pgfpathlineto{\pgfqpoint{2.295946in}{3.116535in}}%
\pgfpathlineto{\pgfqpoint{2.300763in}{3.122992in}}%
\pgfpathlineto{\pgfqpoint{2.309594in}{3.134963in}}%
\pgfpathlineto{\pgfqpoint{2.313073in}{3.136179in}}%
\pgfpathlineto{\pgfqpoint{2.316284in}{3.134808in}}%
\pgfpathlineto{\pgfqpoint{2.320298in}{3.130224in}}%
\pgfpathlineto{\pgfqpoint{2.332341in}{3.114647in}}%
\pgfpathlineto{\pgfqpoint{2.335552in}{3.114443in}}%
\pgfpathlineto{\pgfqpoint{2.338763in}{3.116593in}}%
\pgfpathlineto{\pgfqpoint{2.343580in}{3.123084in}}%
\pgfpathlineto{\pgfqpoint{2.352144in}{3.134803in}}%
\pgfpathlineto{\pgfqpoint{2.355623in}{3.136186in}}%
\pgfpathlineto{\pgfqpoint{2.358834in}{3.134967in}}%
\pgfpathlineto{\pgfqpoint{2.362848in}{3.130522in}}%
\pgfpathlineto{\pgfqpoint{2.375426in}{3.114516in}}%
\pgfpathlineto{\pgfqpoint{2.378637in}{3.114558in}}%
\pgfpathlineto{\pgfqpoint{2.382116in}{3.117218in}}%
\pgfpathlineto{\pgfqpoint{2.387468in}{3.124890in}}%
\pgfpathlineto{\pgfqpoint{2.394693in}{3.134635in}}%
\pgfpathlineto{\pgfqpoint{2.398172in}{3.136182in}}%
\pgfpathlineto{\pgfqpoint{2.401384in}{3.135117in}}%
\pgfpathlineto{\pgfqpoint{2.405130in}{3.131182in}}%
\pgfpathlineto{\pgfqpoint{2.418778in}{3.114341in}}%
\pgfpathlineto{\pgfqpoint{2.421989in}{3.114830in}}%
\pgfpathlineto{\pgfqpoint{2.425468in}{3.117903in}}%
\pgfpathlineto{\pgfqpoint{2.431623in}{3.127132in}}%
\pgfpathlineto{\pgfqpoint{2.437778in}{3.134893in}}%
\pgfpathlineto{\pgfqpoint{2.441257in}{3.136184in}}%
\pgfpathlineto{\pgfqpoint{2.444468in}{3.134880in}}%
\pgfpathlineto{\pgfqpoint{2.448483in}{3.130357in}}%
\pgfpathlineto{\pgfqpoint{2.460793in}{3.114574in}}%
\pgfpathlineto{\pgfqpoint{2.464004in}{3.114502in}}%
\pgfpathlineto{\pgfqpoint{2.467215in}{3.116770in}}%
\pgfpathlineto{\pgfqpoint{2.472032in}{3.123360in}}%
\pgfpathlineto{\pgfqpoint{2.480596in}{3.134937in}}%
\pgfpathlineto{\pgfqpoint{2.484074in}{3.136181in}}%
\pgfpathlineto{\pgfqpoint{2.487286in}{3.134835in}}%
\pgfpathlineto{\pgfqpoint{2.491300in}{3.130274in}}%
\pgfpathlineto{\pgfqpoint{2.503342in}{3.114663in}}%
\pgfpathlineto{\pgfqpoint{2.506553in}{3.114433in}}%
\pgfpathlineto{\pgfqpoint{2.509765in}{3.116558in}}%
\pgfpathlineto{\pgfqpoint{2.514582in}{3.123028in}}%
\pgfpathlineto{\pgfqpoint{2.523413in}{3.134980in}}%
\pgfpathlineto{\pgfqpoint{2.526892in}{3.136178in}}%
\pgfpathlineto{\pgfqpoint{2.530103in}{3.134790in}}%
\pgfpathlineto{\pgfqpoint{2.534117in}{3.130191in}}%
\pgfpathlineto{\pgfqpoint{2.546159in}{3.114637in}}%
\pgfpathlineto{\pgfqpoint{2.549371in}{3.114451in}}%
\pgfpathlineto{\pgfqpoint{2.552582in}{3.116616in}}%
\pgfpathlineto{\pgfqpoint{2.557399in}{3.123120in}}%
\pgfpathlineto{\pgfqpoint{2.565962in}{3.134821in}}%
\pgfpathlineto{\pgfqpoint{2.569441in}{3.136186in}}%
\pgfpathlineto{\pgfqpoint{2.572653in}{3.134950in}}%
\pgfpathlineto{\pgfqpoint{2.576667in}{3.130489in}}%
\pgfpathlineto{\pgfqpoint{2.588977in}{3.114612in}}%
\pgfpathlineto{\pgfqpoint{2.592188in}{3.114470in}}%
\pgfpathlineto{\pgfqpoint{2.595399in}{3.116675in}}%
\pgfpathlineto{\pgfqpoint{2.600216in}{3.123213in}}%
\pgfpathlineto{\pgfqpoint{2.608780in}{3.134866in}}%
\pgfpathlineto{\pgfqpoint{2.612259in}{3.136185in}}%
\pgfpathlineto{\pgfqpoint{2.615470in}{3.134906in}}%
\pgfpathlineto{\pgfqpoint{2.619484in}{3.130407in}}%
\pgfpathlineto{\pgfqpoint{2.631794in}{3.114588in}}%
\pgfpathlineto{\pgfqpoint{2.635005in}{3.114490in}}%
\pgfpathlineto{\pgfqpoint{2.638217in}{3.116734in}}%
\pgfpathlineto{\pgfqpoint{2.643033in}{3.123305in}}%
\pgfpathlineto{\pgfqpoint{2.651597in}{3.134911in}}%
\pgfpathlineto{\pgfqpoint{2.655076in}{3.136183in}}%
\pgfpathlineto{\pgfqpoint{2.658287in}{3.134862in}}%
\pgfpathlineto{\pgfqpoint{2.662301in}{3.130324in}}%
\pgfpathlineto{\pgfqpoint{2.674344in}{3.114679in}}%
\pgfpathlineto{\pgfqpoint{2.677555in}{3.114422in}}%
\pgfpathlineto{\pgfqpoint{2.680766in}{3.116523in}}%
\pgfpathlineto{\pgfqpoint{2.685315in}{3.122552in}}%
\pgfpathlineto{\pgfqpoint{2.694414in}{3.134954in}}%
\pgfpathlineto{\pgfqpoint{2.697893in}{3.136180in}}%
\pgfpathlineto{\pgfqpoint{2.701104in}{3.134817in}}%
\pgfpathlineto{\pgfqpoint{2.705118in}{3.130241in}}%
\pgfpathlineto{\pgfqpoint{2.717161in}{3.114653in}}%
\pgfpathlineto{\pgfqpoint{2.720372in}{3.114440in}}%
\pgfpathlineto{\pgfqpoint{2.723583in}{3.116581in}}%
\pgfpathlineto{\pgfqpoint{2.728400in}{3.123065in}}%
\pgfpathlineto{\pgfqpoint{2.736964in}{3.134794in}}%
\pgfpathlineto{\pgfqpoint{2.740443in}{3.136186in}}%
\pgfpathlineto{\pgfqpoint{2.743654in}{3.134976in}}%
\pgfpathlineto{\pgfqpoint{2.747668in}{3.130538in}}%
\pgfpathlineto{\pgfqpoint{2.760246in}{3.114521in}}%
\pgfpathlineto{\pgfqpoint{2.763457in}{3.114554in}}%
\pgfpathlineto{\pgfqpoint{2.766936in}{3.117205in}}%
\pgfpathlineto{\pgfqpoint{2.772288in}{3.124872in}}%
\pgfpathlineto{\pgfqpoint{2.779513in}{3.134625in}}%
\pgfpathlineto{\pgfqpoint{2.782992in}{3.136181in}}%
\pgfpathlineto{\pgfqpoint{2.786204in}{3.135125in}}%
\pgfpathlineto{\pgfqpoint{2.789950in}{3.131198in}}%
\pgfpathlineto{\pgfqpoint{2.803598in}{3.114344in}}%
\pgfpathlineto{\pgfqpoint{2.806809in}{3.114824in}}%
\pgfpathlineto{\pgfqpoint{2.810288in}{3.117889in}}%
\pgfpathlineto{\pgfqpoint{2.816443in}{3.127114in}}%
\pgfpathlineto{\pgfqpoint{2.822598in}{3.134884in}}%
\pgfpathlineto{\pgfqpoint{2.826077in}{3.136184in}}%
\pgfpathlineto{\pgfqpoint{2.829289in}{3.134889in}}%
\pgfpathlineto{\pgfqpoint{2.833303in}{3.130374in}}%
\pgfpathlineto{\pgfqpoint{2.845613in}{3.114579in}}%
\pgfpathlineto{\pgfqpoint{2.848824in}{3.114498in}}%
\pgfpathlineto{\pgfqpoint{2.852035in}{3.116758in}}%
\pgfpathlineto{\pgfqpoint{2.856852in}{3.123342in}}%
\pgfpathlineto{\pgfqpoint{2.865416in}{3.134928in}}%
\pgfpathlineto{\pgfqpoint{2.868894in}{3.136182in}}%
\pgfpathlineto{\pgfqpoint{2.872106in}{3.134844in}}%
\pgfpathlineto{\pgfqpoint{2.876120in}{3.130291in}}%
\pgfpathlineto{\pgfqpoint{2.888162in}{3.114668in}}%
\pgfpathlineto{\pgfqpoint{2.891374in}{3.114429in}}%
\pgfpathlineto{\pgfqpoint{2.894585in}{3.116546in}}%
\pgfpathlineto{\pgfqpoint{2.899402in}{3.123010in}}%
\pgfpathlineto{\pgfqpoint{2.908233in}{3.134971in}}%
\pgfpathlineto{\pgfqpoint{2.911712in}{3.136179in}}%
\pgfpathlineto{\pgfqpoint{2.914923in}{3.134799in}}%
\pgfpathlineto{\pgfqpoint{2.918937in}{3.130207in}}%
\pgfpathlineto{\pgfqpoint{2.930979in}{3.114642in}}%
\pgfpathlineto{\pgfqpoint{2.934191in}{3.114447in}}%
\pgfpathlineto{\pgfqpoint{2.937402in}{3.116604in}}%
\pgfpathlineto{\pgfqpoint{2.942219in}{3.123102in}}%
\pgfpathlineto{\pgfqpoint{2.950782in}{3.134812in}}%
\pgfpathlineto{\pgfqpoint{2.954261in}{3.136186in}}%
\pgfpathlineto{\pgfqpoint{2.957473in}{3.134959in}}%
\pgfpathlineto{\pgfqpoint{2.961487in}{3.130506in}}%
\pgfpathlineto{\pgfqpoint{2.973797in}{3.114617in}}%
\pgfpathlineto{\pgfqpoint{2.977008in}{3.114466in}}%
\pgfpathlineto{\pgfqpoint{2.980219in}{3.116663in}}%
\pgfpathlineto{\pgfqpoint{2.985036in}{3.123194in}}%
\pgfpathlineto{\pgfqpoint{2.993600in}{3.134857in}}%
\pgfpathlineto{\pgfqpoint{2.997079in}{3.136185in}}%
\pgfpathlineto{\pgfqpoint{3.000290in}{3.134915in}}%
\pgfpathlineto{\pgfqpoint{3.004304in}{3.130423in}}%
\pgfpathlineto{\pgfqpoint{3.016614in}{3.114593in}}%
\pgfpathlineto{\pgfqpoint{3.019825in}{3.114486in}}%
\pgfpathlineto{\pgfqpoint{3.023037in}{3.116722in}}%
\pgfpathlineto{\pgfqpoint{3.027854in}{3.123286in}}%
\pgfpathlineto{\pgfqpoint{3.036417in}{3.134902in}}%
\pgfpathlineto{\pgfqpoint{3.039896in}{3.136183in}}%
\pgfpathlineto{\pgfqpoint{3.043107in}{3.134871in}}%
\pgfpathlineto{\pgfqpoint{3.047121in}{3.130341in}}%
\pgfpathlineto{\pgfqpoint{3.059164in}{3.114684in}}%
\pgfpathlineto{\pgfqpoint{3.062375in}{3.114419in}}%
\pgfpathlineto{\pgfqpoint{3.065586in}{3.116512in}}%
\pgfpathlineto{\pgfqpoint{3.070136in}{3.122534in}}%
\pgfpathlineto{\pgfqpoint{3.079234in}{3.134946in}}%
\pgfpathlineto{\pgfqpoint{3.082713in}{3.136181in}}%
\pgfpathlineto{\pgfqpoint{3.085924in}{3.134826in}}%
\pgfpathlineto{\pgfqpoint{3.089939in}{3.130257in}}%
\pgfpathlineto{\pgfqpoint{3.101981in}{3.114658in}}%
\pgfpathlineto{\pgfqpoint{3.105192in}{3.114436in}}%
\pgfpathlineto{\pgfqpoint{3.108403in}{3.116569in}}%
\pgfpathlineto{\pgfqpoint{3.113220in}{3.123047in}}%
\pgfpathlineto{\pgfqpoint{3.122051in}{3.134989in}}%
\pgfpathlineto{\pgfqpoint{3.125530in}{3.136177in}}%
\pgfpathlineto{\pgfqpoint{3.128742in}{3.134780in}}%
\pgfpathlineto{\pgfqpoint{3.132756in}{3.130174in}}%
\pgfpathlineto{\pgfqpoint{3.144798in}{3.114632in}}%
\pgfpathlineto{\pgfqpoint{3.148009in}{3.114455in}}%
\pgfpathlineto{\pgfqpoint{3.151221in}{3.116628in}}%
\pgfpathlineto{\pgfqpoint{3.156038in}{3.123139in}}%
\pgfpathlineto{\pgfqpoint{3.164601in}{3.134831in}}%
\pgfpathlineto{\pgfqpoint{3.168080in}{3.136186in}}%
\pgfpathlineto{\pgfqpoint{3.171291in}{3.134941in}}%
\pgfpathlineto{\pgfqpoint{3.175305in}{3.130473in}}%
\pgfpathlineto{\pgfqpoint{3.187615in}{3.114608in}}%
\pgfpathlineto{\pgfqpoint{3.190827in}{3.114474in}}%
\pgfpathlineto{\pgfqpoint{3.194038in}{3.116686in}}%
\pgfpathlineto{\pgfqpoint{3.198855in}{3.123231in}}%
\pgfpathlineto{\pgfqpoint{3.207418in}{3.134875in}}%
\pgfpathlineto{\pgfqpoint{3.210897in}{3.136184in}}%
\pgfpathlineto{\pgfqpoint{3.214109in}{3.134897in}}%
\pgfpathlineto{\pgfqpoint{3.218123in}{3.130390in}}%
\pgfpathlineto{\pgfqpoint{3.230433in}{3.114584in}}%
\pgfpathlineto{\pgfqpoint{3.233644in}{3.114494in}}%
\pgfpathlineto{\pgfqpoint{3.236855in}{3.116746in}}%
\pgfpathlineto{\pgfqpoint{3.241672in}{3.123323in}}%
\pgfpathlineto{\pgfqpoint{3.250236in}{3.134919in}}%
\pgfpathlineto{\pgfqpoint{3.253715in}{3.136182in}}%
\pgfpathlineto{\pgfqpoint{3.256926in}{3.134853in}}%
\pgfpathlineto{\pgfqpoint{3.260940in}{3.130307in}}%
\pgfpathlineto{\pgfqpoint{3.272982in}{3.114673in}}%
\pgfpathlineto{\pgfqpoint{3.276194in}{3.114426in}}%
\pgfpathlineto{\pgfqpoint{3.279405in}{3.116535in}}%
\pgfpathlineto{\pgfqpoint{3.284222in}{3.122992in}}%
\pgfpathlineto{\pgfqpoint{3.293053in}{3.134963in}}%
\pgfpathlineto{\pgfqpoint{3.296532in}{3.136179in}}%
\pgfpathlineto{\pgfqpoint{3.299743in}{3.134808in}}%
\pgfpathlineto{\pgfqpoint{3.303757in}{3.130224in}}%
\pgfpathlineto{\pgfqpoint{3.315800in}{3.114647in}}%
\pgfpathlineto{\pgfqpoint{3.319011in}{3.114443in}}%
\pgfpathlineto{\pgfqpoint{3.322222in}{3.116593in}}%
\pgfpathlineto{\pgfqpoint{3.327039in}{3.123084in}}%
\pgfpathlineto{\pgfqpoint{3.335603in}{3.134803in}}%
\pgfpathlineto{\pgfqpoint{3.339081in}{3.136186in}}%
\pgfpathlineto{\pgfqpoint{3.342293in}{3.134967in}}%
\pgfpathlineto{\pgfqpoint{3.346307in}{3.130522in}}%
\pgfpathlineto{\pgfqpoint{3.358884in}{3.114516in}}%
\pgfpathlineto{\pgfqpoint{3.362096in}{3.114558in}}%
\pgfpathlineto{\pgfqpoint{3.365575in}{3.117218in}}%
\pgfpathlineto{\pgfqpoint{3.370927in}{3.124890in}}%
\pgfpathlineto{\pgfqpoint{3.378152in}{3.134635in}}%
\pgfpathlineto{\pgfqpoint{3.381631in}{3.136182in}}%
\pgfpathlineto{\pgfqpoint{3.384842in}{3.135117in}}%
\pgfpathlineto{\pgfqpoint{3.388589in}{3.131182in}}%
\pgfpathlineto{\pgfqpoint{3.402237in}{3.114341in}}%
\pgfpathlineto{\pgfqpoint{3.405448in}{3.114830in}}%
\pgfpathlineto{\pgfqpoint{3.408927in}{3.117903in}}%
\pgfpathlineto{\pgfqpoint{3.415082in}{3.127132in}}%
\pgfpathlineto{\pgfqpoint{3.421237in}{3.134893in}}%
\pgfpathlineto{\pgfqpoint{3.424716in}{3.136184in}}%
\pgfpathlineto{\pgfqpoint{3.427927in}{3.134880in}}%
\pgfpathlineto{\pgfqpoint{3.431941in}{3.130357in}}%
\pgfpathlineto{\pgfqpoint{3.444251in}{3.114574in}}%
\pgfpathlineto{\pgfqpoint{3.447463in}{3.114502in}}%
\pgfpathlineto{\pgfqpoint{3.450674in}{3.116770in}}%
\pgfpathlineto{\pgfqpoint{3.455491in}{3.123360in}}%
\pgfpathlineto{\pgfqpoint{3.464054in}{3.134937in}}%
\pgfpathlineto{\pgfqpoint{3.467533in}{3.136181in}}%
\pgfpathlineto{\pgfqpoint{3.470744in}{3.134835in}}%
\pgfpathlineto{\pgfqpoint{3.474759in}{3.130274in}}%
\pgfpathlineto{\pgfqpoint{3.486801in}{3.114663in}}%
\pgfpathlineto{\pgfqpoint{3.490012in}{3.114433in}}%
\pgfpathlineto{\pgfqpoint{3.493224in}{3.116558in}}%
\pgfpathlineto{\pgfqpoint{3.498040in}{3.123028in}}%
\pgfpathlineto{\pgfqpoint{3.506872in}{3.134980in}}%
\pgfpathlineto{\pgfqpoint{3.510350in}{3.136178in}}%
\pgfpathlineto{\pgfqpoint{3.513562in}{3.134790in}}%
\pgfpathlineto{\pgfqpoint{3.517576in}{3.130191in}}%
\pgfpathlineto{\pgfqpoint{3.529618in}{3.114637in}}%
\pgfpathlineto{\pgfqpoint{3.532829in}{3.114451in}}%
\pgfpathlineto{\pgfqpoint{3.536041in}{3.116616in}}%
\pgfpathlineto{\pgfqpoint{3.540858in}{3.123120in}}%
\pgfpathlineto{\pgfqpoint{3.549421in}{3.134821in}}%
\pgfpathlineto{\pgfqpoint{3.552900in}{3.136186in}}%
\pgfpathlineto{\pgfqpoint{3.556111in}{3.134950in}}%
\pgfpathlineto{\pgfqpoint{3.560125in}{3.130489in}}%
\pgfpathlineto{\pgfqpoint{3.572435in}{3.114612in}}%
\pgfpathlineto{\pgfqpoint{3.575647in}{3.114470in}}%
\pgfpathlineto{\pgfqpoint{3.578858in}{3.116675in}}%
\pgfpathlineto{\pgfqpoint{3.583675in}{3.123213in}}%
\pgfpathlineto{\pgfqpoint{3.592238in}{3.134866in}}%
\pgfpathlineto{\pgfqpoint{3.595717in}{3.136185in}}%
\pgfpathlineto{\pgfqpoint{3.598929in}{3.134906in}}%
\pgfpathlineto{\pgfqpoint{3.602943in}{3.130407in}}%
\pgfpathlineto{\pgfqpoint{3.615253in}{3.114588in}}%
\pgfpathlineto{\pgfqpoint{3.618464in}{3.114490in}}%
\pgfpathlineto{\pgfqpoint{3.621675in}{3.116734in}}%
\pgfpathlineto{\pgfqpoint{3.626492in}{3.123305in}}%
\pgfpathlineto{\pgfqpoint{3.635056in}{3.134911in}}%
\pgfpathlineto{\pgfqpoint{3.638535in}{3.136183in}}%
\pgfpathlineto{\pgfqpoint{3.641746in}{3.134862in}}%
\pgfpathlineto{\pgfqpoint{3.645760in}{3.130324in}}%
\pgfpathlineto{\pgfqpoint{3.657802in}{3.114679in}}%
\pgfpathlineto{\pgfqpoint{3.661014in}{3.114422in}}%
\pgfpathlineto{\pgfqpoint{3.664225in}{3.116523in}}%
\pgfpathlineto{\pgfqpoint{3.668774in}{3.122552in}}%
\pgfpathlineto{\pgfqpoint{3.677873in}{3.134954in}}%
\pgfpathlineto{\pgfqpoint{3.681352in}{3.136180in}}%
\pgfpathlineto{\pgfqpoint{3.684563in}{3.134817in}}%
\pgfpathlineto{\pgfqpoint{3.688577in}{3.130241in}}%
\pgfpathlineto{\pgfqpoint{3.700620in}{3.114653in}}%
\pgfpathlineto{\pgfqpoint{3.703831in}{3.114440in}}%
\pgfpathlineto{\pgfqpoint{3.707042in}{3.116581in}}%
\pgfpathlineto{\pgfqpoint{3.711859in}{3.123065in}}%
\pgfpathlineto{\pgfqpoint{3.720423in}{3.134794in}}%
\pgfpathlineto{\pgfqpoint{3.723901in}{3.136186in}}%
\pgfpathlineto{\pgfqpoint{3.727113in}{3.134976in}}%
\pgfpathlineto{\pgfqpoint{3.731127in}{3.130538in}}%
\pgfpathlineto{\pgfqpoint{3.743704in}{3.114521in}}%
\pgfpathlineto{\pgfqpoint{3.746916in}{3.114554in}}%
\pgfpathlineto{\pgfqpoint{3.750395in}{3.117205in}}%
\pgfpathlineto{\pgfqpoint{3.755747in}{3.124872in}}%
\pgfpathlineto{\pgfqpoint{3.762972in}{3.134625in}}%
\pgfpathlineto{\pgfqpoint{3.766451in}{3.136181in}}%
\pgfpathlineto{\pgfqpoint{3.769662in}{3.135125in}}%
\pgfpathlineto{\pgfqpoint{3.773409in}{3.131198in}}%
\pgfpathlineto{\pgfqpoint{3.787057in}{3.114344in}}%
\pgfpathlineto{\pgfqpoint{3.790268in}{3.114824in}}%
\pgfpathlineto{\pgfqpoint{3.793747in}{3.117889in}}%
\pgfpathlineto{\pgfqpoint{3.799902in}{3.127114in}}%
\pgfpathlineto{\pgfqpoint{3.806057in}{3.134884in}}%
\pgfpathlineto{\pgfqpoint{3.809536in}{3.136184in}}%
\pgfpathlineto{\pgfqpoint{3.812747in}{3.134889in}}%
\pgfpathlineto{\pgfqpoint{3.816761in}{3.130374in}}%
\pgfpathlineto{\pgfqpoint{3.829071in}{3.114579in}}%
\pgfpathlineto{\pgfqpoint{3.832283in}{3.114498in}}%
\pgfpathlineto{\pgfqpoint{3.835494in}{3.116758in}}%
\pgfpathlineto{\pgfqpoint{3.840311in}{3.123342in}}%
\pgfpathlineto{\pgfqpoint{3.848874in}{3.134928in}}%
\pgfpathlineto{\pgfqpoint{3.852353in}{3.136182in}}%
\pgfpathlineto{\pgfqpoint{3.855564in}{3.134844in}}%
\pgfpathlineto{\pgfqpoint{3.859579in}{3.130291in}}%
\pgfpathlineto{\pgfqpoint{3.871621in}{3.114668in}}%
\pgfpathlineto{\pgfqpoint{3.874832in}{3.114429in}}%
\pgfpathlineto{\pgfqpoint{3.878044in}{3.116546in}}%
\pgfpathlineto{\pgfqpoint{3.882860in}{3.123010in}}%
\pgfpathlineto{\pgfqpoint{3.891692in}{3.134971in}}%
\pgfpathlineto{\pgfqpoint{3.895170in}{3.136179in}}%
\pgfpathlineto{\pgfqpoint{3.898382in}{3.134799in}}%
\pgfpathlineto{\pgfqpoint{3.902396in}{3.130207in}}%
\pgfpathlineto{\pgfqpoint{3.914438in}{3.114642in}}%
\pgfpathlineto{\pgfqpoint{3.917650in}{3.114447in}}%
\pgfpathlineto{\pgfqpoint{3.920861in}{3.116604in}}%
\pgfpathlineto{\pgfqpoint{3.925678in}{3.123102in}}%
\pgfpathlineto{\pgfqpoint{3.934241in}{3.134812in}}%
\pgfpathlineto{\pgfqpoint{3.937720in}{3.136186in}}%
\pgfpathlineto{\pgfqpoint{3.940931in}{3.134959in}}%
\pgfpathlineto{\pgfqpoint{3.944946in}{3.130506in}}%
\pgfpathlineto{\pgfqpoint{3.957255in}{3.114617in}}%
\pgfpathlineto{\pgfqpoint{3.960467in}{3.114466in}}%
\pgfpathlineto{\pgfqpoint{3.963678in}{3.116663in}}%
\pgfpathlineto{\pgfqpoint{3.968495in}{3.123194in}}%
\pgfpathlineto{\pgfqpoint{3.977058in}{3.134857in}}%
\pgfpathlineto{\pgfqpoint{3.980537in}{3.136185in}}%
\pgfpathlineto{\pgfqpoint{3.983749in}{3.134915in}}%
\pgfpathlineto{\pgfqpoint{3.987763in}{3.130423in}}%
\pgfpathlineto{\pgfqpoint{4.000073in}{3.114593in}}%
\pgfpathlineto{\pgfqpoint{4.003284in}{3.114486in}}%
\pgfpathlineto{\pgfqpoint{4.006495in}{3.116722in}}%
\pgfpathlineto{\pgfqpoint{4.011312in}{3.123286in}}%
\pgfpathlineto{\pgfqpoint{4.019876in}{3.134902in}}%
\pgfpathlineto{\pgfqpoint{4.023355in}{3.136183in}}%
\pgfpathlineto{\pgfqpoint{4.026566in}{3.134871in}}%
\pgfpathlineto{\pgfqpoint{4.030580in}{3.130341in}}%
\pgfpathlineto{\pgfqpoint{4.042622in}{3.114684in}}%
\pgfpathlineto{\pgfqpoint{4.045834in}{3.114419in}}%
\pgfpathlineto{\pgfqpoint{4.049045in}{3.116512in}}%
\pgfpathlineto{\pgfqpoint{4.053594in}{3.122534in}}%
\pgfpathlineto{\pgfqpoint{4.062693in}{3.134946in}}%
\pgfpathlineto{\pgfqpoint{4.066172in}{3.136181in}}%
\pgfpathlineto{\pgfqpoint{4.069383in}{3.134826in}}%
\pgfpathlineto{\pgfqpoint{4.073397in}{3.130257in}}%
\pgfpathlineto{\pgfqpoint{4.085440in}{3.114658in}}%
\pgfpathlineto{\pgfqpoint{4.088651in}{3.114436in}}%
\pgfpathlineto{\pgfqpoint{4.091862in}{3.116569in}}%
\pgfpathlineto{\pgfqpoint{4.096679in}{3.123047in}}%
\pgfpathlineto{\pgfqpoint{4.105510in}{3.134989in}}%
\pgfpathlineto{\pgfqpoint{4.108989in}{3.136177in}}%
\pgfpathlineto{\pgfqpoint{4.112200in}{3.134780in}}%
\pgfpathlineto{\pgfqpoint{4.116215in}{3.130174in}}%
\pgfpathlineto{\pgfqpoint{4.128257in}{3.114632in}}%
\pgfpathlineto{\pgfqpoint{4.131468in}{3.114455in}}%
\pgfpathlineto{\pgfqpoint{4.134679in}{3.116628in}}%
\pgfpathlineto{\pgfqpoint{4.139496in}{3.123139in}}%
\pgfpathlineto{\pgfqpoint{4.148060in}{3.134831in}}%
\pgfpathlineto{\pgfqpoint{4.151539in}{3.136186in}}%
\pgfpathlineto{\pgfqpoint{4.154750in}{3.134941in}}%
\pgfpathlineto{\pgfqpoint{4.158764in}{3.130473in}}%
\pgfpathlineto{\pgfqpoint{4.171074in}{3.114608in}}%
\pgfpathlineto{\pgfqpoint{4.174285in}{3.114474in}}%
\pgfpathlineto{\pgfqpoint{4.177497in}{3.116686in}}%
\pgfpathlineto{\pgfqpoint{4.182314in}{3.123231in}}%
\pgfpathlineto{\pgfqpoint{4.190877in}{3.134875in}}%
\pgfpathlineto{\pgfqpoint{4.194356in}{3.136184in}}%
\pgfpathlineto{\pgfqpoint{4.197567in}{3.134897in}}%
\pgfpathlineto{\pgfqpoint{4.201581in}{3.130390in}}%
\pgfpathlineto{\pgfqpoint{4.213891in}{3.114584in}}%
\pgfpathlineto{\pgfqpoint{4.217103in}{3.114494in}}%
\pgfpathlineto{\pgfqpoint{4.220314in}{3.116746in}}%
\pgfpathlineto{\pgfqpoint{4.225131in}{3.123323in}}%
\pgfpathlineto{\pgfqpoint{4.233694in}{3.134919in}}%
\pgfpathlineto{\pgfqpoint{4.237173in}{3.136182in}}%
\pgfpathlineto{\pgfqpoint{4.240385in}{3.134853in}}%
\pgfpathlineto{\pgfqpoint{4.244399in}{3.130307in}}%
\pgfpathlineto{\pgfqpoint{4.256441in}{3.114673in}}%
\pgfpathlineto{\pgfqpoint{4.259652in}{3.114426in}}%
\pgfpathlineto{\pgfqpoint{4.262864in}{3.116535in}}%
\pgfpathlineto{\pgfqpoint{4.267681in}{3.122992in}}%
\pgfpathlineto{\pgfqpoint{4.276512in}{3.134963in}}%
\pgfpathlineto{\pgfqpoint{4.279991in}{3.136179in}}%
\pgfpathlineto{\pgfqpoint{4.283202in}{3.134808in}}%
\pgfpathlineto{\pgfqpoint{4.287216in}{3.130224in}}%
\pgfpathlineto{\pgfqpoint{4.299258in}{3.114647in}}%
\pgfpathlineto{\pgfqpoint{4.302470in}{3.114443in}}%
\pgfpathlineto{\pgfqpoint{4.305681in}{3.116593in}}%
\pgfpathlineto{\pgfqpoint{4.310498in}{3.123084in}}%
\pgfpathlineto{\pgfqpoint{4.319061in}{3.134803in}}%
\pgfpathlineto{\pgfqpoint{4.322540in}{3.136186in}}%
\pgfpathlineto{\pgfqpoint{4.325751in}{3.134967in}}%
\pgfpathlineto{\pgfqpoint{4.329766in}{3.130522in}}%
\pgfpathlineto{\pgfqpoint{4.342343in}{3.114516in}}%
\pgfpathlineto{\pgfqpoint{4.345554in}{3.114558in}}%
\pgfpathlineto{\pgfqpoint{4.349033in}{3.117218in}}%
\pgfpathlineto{\pgfqpoint{4.354385in}{3.124890in}}%
\pgfpathlineto{\pgfqpoint{4.361611in}{3.134635in}}%
\pgfpathlineto{\pgfqpoint{4.365090in}{3.136182in}}%
\pgfpathlineto{\pgfqpoint{4.368301in}{3.135117in}}%
\pgfpathlineto{\pgfqpoint{4.372048in}{3.131182in}}%
\pgfpathlineto{\pgfqpoint{4.385696in}{3.114341in}}%
\pgfpathlineto{\pgfqpoint{4.388907in}{3.114830in}}%
\pgfpathlineto{\pgfqpoint{4.392386in}{3.117903in}}%
\pgfpathlineto{\pgfqpoint{4.398541in}{3.127132in}}%
\pgfpathlineto{\pgfqpoint{4.404696in}{3.134893in}}%
\pgfpathlineto{\pgfqpoint{4.408175in}{3.136184in}}%
\pgfpathlineto{\pgfqpoint{4.411386in}{3.134880in}}%
\pgfpathlineto{\pgfqpoint{4.415400in}{3.130357in}}%
\pgfpathlineto{\pgfqpoint{4.427710in}{3.114574in}}%
\pgfpathlineto{\pgfqpoint{4.430921in}{3.114502in}}%
\pgfpathlineto{\pgfqpoint{4.434133in}{3.116770in}}%
\pgfpathlineto{\pgfqpoint{4.438950in}{3.123360in}}%
\pgfpathlineto{\pgfqpoint{4.447513in}{3.134937in}}%
\pgfpathlineto{\pgfqpoint{4.450992in}{3.136181in}}%
\pgfpathlineto{\pgfqpoint{4.454203in}{3.134835in}}%
\pgfpathlineto{\pgfqpoint{4.458217in}{3.130274in}}%
\pgfpathlineto{\pgfqpoint{4.470260in}{3.114663in}}%
\pgfpathlineto{\pgfqpoint{4.473471in}{3.114433in}}%
\pgfpathlineto{\pgfqpoint{4.476682in}{3.116558in}}%
\pgfpathlineto{\pgfqpoint{4.481499in}{3.123028in}}%
\pgfpathlineto{\pgfqpoint{4.490330in}{3.134980in}}%
\pgfpathlineto{\pgfqpoint{4.493809in}{3.136178in}}%
\pgfpathlineto{\pgfqpoint{4.497020in}{3.134790in}}%
\pgfpathlineto{\pgfqpoint{4.501035in}{3.130191in}}%
\pgfpathlineto{\pgfqpoint{4.513077in}{3.114637in}}%
\pgfpathlineto{\pgfqpoint{4.516288in}{3.114451in}}%
\pgfpathlineto{\pgfqpoint{4.519500in}{3.116616in}}%
\pgfpathlineto{\pgfqpoint{4.524316in}{3.123120in}}%
\pgfpathlineto{\pgfqpoint{4.532880in}{3.134821in}}%
\pgfpathlineto{\pgfqpoint{4.536359in}{3.136186in}}%
\pgfpathlineto{\pgfqpoint{4.539570in}{3.134950in}}%
\pgfpathlineto{\pgfqpoint{4.543584in}{3.130489in}}%
\pgfpathlineto{\pgfqpoint{4.555894in}{3.114612in}}%
\pgfpathlineto{\pgfqpoint{4.559105in}{3.114470in}}%
\pgfpathlineto{\pgfqpoint{4.562317in}{3.116675in}}%
\pgfpathlineto{\pgfqpoint{4.567134in}{3.123213in}}%
\pgfpathlineto{\pgfqpoint{4.575697in}{3.134866in}}%
\pgfpathlineto{\pgfqpoint{4.579176in}{3.136185in}}%
\pgfpathlineto{\pgfqpoint{4.582387in}{3.134906in}}%
\pgfpathlineto{\pgfqpoint{4.586401in}{3.130407in}}%
\pgfpathlineto{\pgfqpoint{4.598711in}{3.114588in}}%
\pgfpathlineto{\pgfqpoint{4.601923in}{3.114490in}}%
\pgfpathlineto{\pgfqpoint{4.605134in}{3.116734in}}%
\pgfpathlineto{\pgfqpoint{4.609951in}{3.123305in}}%
\pgfpathlineto{\pgfqpoint{4.618514in}{3.134911in}}%
\pgfpathlineto{\pgfqpoint{4.621993in}{3.136183in}}%
\pgfpathlineto{\pgfqpoint{4.625205in}{3.134862in}}%
\pgfpathlineto{\pgfqpoint{4.629219in}{3.130324in}}%
\pgfpathlineto{\pgfqpoint{4.641261in}{3.114679in}}%
\pgfpathlineto{\pgfqpoint{4.644472in}{3.114422in}}%
\pgfpathlineto{\pgfqpoint{4.647684in}{3.116523in}}%
\pgfpathlineto{\pgfqpoint{4.652233in}{3.122552in}}%
\pgfpathlineto{\pgfqpoint{4.661332in}{3.134954in}}%
\pgfpathlineto{\pgfqpoint{4.664811in}{3.136180in}}%
\pgfpathlineto{\pgfqpoint{4.668022in}{3.134817in}}%
\pgfpathlineto{\pgfqpoint{4.672036in}{3.130241in}}%
\pgfpathlineto{\pgfqpoint{4.684078in}{3.114653in}}%
\pgfpathlineto{\pgfqpoint{4.687290in}{3.114440in}}%
\pgfpathlineto{\pgfqpoint{4.690501in}{3.116581in}}%
\pgfpathlineto{\pgfqpoint{4.695318in}{3.123065in}}%
\pgfpathlineto{\pgfqpoint{4.703881in}{3.134794in}}%
\pgfpathlineto{\pgfqpoint{4.707360in}{3.136186in}}%
\pgfpathlineto{\pgfqpoint{4.710571in}{3.134976in}}%
\pgfpathlineto{\pgfqpoint{4.714586in}{3.130538in}}%
\pgfpathlineto{\pgfqpoint{4.727163in}{3.114521in}}%
\pgfpathlineto{\pgfqpoint{4.730374in}{3.114554in}}%
\pgfpathlineto{\pgfqpoint{4.733853in}{3.117205in}}%
\pgfpathlineto{\pgfqpoint{4.739206in}{3.124872in}}%
\pgfpathlineto{\pgfqpoint{4.746431in}{3.134625in}}%
\pgfpathlineto{\pgfqpoint{4.749910in}{3.136181in}}%
\pgfpathlineto{\pgfqpoint{4.753121in}{3.135125in}}%
\pgfpathlineto{\pgfqpoint{4.756868in}{3.131198in}}%
\pgfpathlineto{\pgfqpoint{4.770516in}{3.114344in}}%
\pgfpathlineto{\pgfqpoint{4.773727in}{3.114824in}}%
\pgfpathlineto{\pgfqpoint{4.777206in}{3.117889in}}%
\pgfpathlineto{\pgfqpoint{4.783361in}{3.127114in}}%
\pgfpathlineto{\pgfqpoint{4.789516in}{3.134884in}}%
\pgfpathlineto{\pgfqpoint{4.792995in}{3.136184in}}%
\pgfpathlineto{\pgfqpoint{4.796206in}{3.134889in}}%
\pgfpathlineto{\pgfqpoint{4.800220in}{3.130374in}}%
\pgfpathlineto{\pgfqpoint{4.812530in}{3.114579in}}%
\pgfpathlineto{\pgfqpoint{4.815741in}{3.114498in}}%
\pgfpathlineto{\pgfqpoint{4.818953in}{3.116758in}}%
\pgfpathlineto{\pgfqpoint{4.823770in}{3.123342in}}%
\pgfpathlineto{\pgfqpoint{4.832333in}{3.134928in}}%
\pgfpathlineto{\pgfqpoint{4.835812in}{3.136182in}}%
\pgfpathlineto{\pgfqpoint{4.839023in}{3.134844in}}%
\pgfpathlineto{\pgfqpoint{4.843037in}{3.130291in}}%
\pgfpathlineto{\pgfqpoint{4.855080in}{3.114668in}}%
\pgfpathlineto{\pgfqpoint{4.858291in}{3.114429in}}%
\pgfpathlineto{\pgfqpoint{4.861502in}{3.116546in}}%
\pgfpathlineto{\pgfqpoint{4.866319in}{3.123010in}}%
\pgfpathlineto{\pgfqpoint{4.875150in}{3.134971in}}%
\pgfpathlineto{\pgfqpoint{4.878629in}{3.136179in}}%
\pgfpathlineto{\pgfqpoint{4.881840in}{3.134799in}}%
\pgfpathlineto{\pgfqpoint{4.885855in}{3.130207in}}%
\pgfpathlineto{\pgfqpoint{4.897897in}{3.114642in}}%
\pgfpathlineto{\pgfqpoint{4.901108in}{3.114447in}}%
\pgfpathlineto{\pgfqpoint{4.904320in}{3.116604in}}%
\pgfpathlineto{\pgfqpoint{4.909136in}{3.123102in}}%
\pgfpathlineto{\pgfqpoint{4.917700in}{3.134812in}}%
\pgfpathlineto{\pgfqpoint{4.921179in}{3.136186in}}%
\pgfpathlineto{\pgfqpoint{4.924390in}{3.134959in}}%
\pgfpathlineto{\pgfqpoint{4.928404in}{3.130506in}}%
\pgfpathlineto{\pgfqpoint{4.940714in}{3.114617in}}%
\pgfpathlineto{\pgfqpoint{4.943926in}{3.114466in}}%
\pgfpathlineto{\pgfqpoint{4.947137in}{3.116663in}}%
\pgfpathlineto{\pgfqpoint{4.951954in}{3.123194in}}%
\pgfpathlineto{\pgfqpoint{4.960517in}{3.134857in}}%
\pgfpathlineto{\pgfqpoint{4.963996in}{3.136185in}}%
\pgfpathlineto{\pgfqpoint{4.967207in}{3.134915in}}%
\pgfpathlineto{\pgfqpoint{4.971222in}{3.130423in}}%
\pgfpathlineto{\pgfqpoint{4.983531in}{3.114593in}}%
\pgfpathlineto{\pgfqpoint{4.986743in}{3.114486in}}%
\pgfpathlineto{\pgfqpoint{4.989954in}{3.116722in}}%
\pgfpathlineto{\pgfqpoint{4.994771in}{3.123286in}}%
\pgfpathlineto{\pgfqpoint{5.003334in}{3.134902in}}%
\pgfpathlineto{\pgfqpoint{5.006813in}{3.136183in}}%
\pgfpathlineto{\pgfqpoint{5.010025in}{3.134871in}}%
\pgfpathlineto{\pgfqpoint{5.014039in}{3.130341in}}%
\pgfpathlineto{\pgfqpoint{5.026081in}{3.114684in}}%
\pgfpathlineto{\pgfqpoint{5.029292in}{3.114419in}}%
\pgfpathlineto{\pgfqpoint{5.032504in}{3.116512in}}%
\pgfpathlineto{\pgfqpoint{5.037053in}{3.122534in}}%
\pgfpathlineto{\pgfqpoint{5.046152in}{3.134946in}}%
\pgfpathlineto{\pgfqpoint{5.049631in}{3.136181in}}%
\pgfpathlineto{\pgfqpoint{5.052842in}{3.134826in}}%
\pgfpathlineto{\pgfqpoint{5.056856in}{3.130257in}}%
\pgfpathlineto{\pgfqpoint{5.068898in}{3.114658in}}%
\pgfpathlineto{\pgfqpoint{5.072110in}{3.114436in}}%
\pgfpathlineto{\pgfqpoint{5.075321in}{3.116569in}}%
\pgfpathlineto{\pgfqpoint{5.080138in}{3.123047in}}%
\pgfpathlineto{\pgfqpoint{5.088969in}{3.134989in}}%
\pgfpathlineto{\pgfqpoint{5.092448in}{3.136177in}}%
\pgfpathlineto{\pgfqpoint{5.095659in}{3.134780in}}%
\pgfpathlineto{\pgfqpoint{5.099673in}{3.130174in}}%
\pgfpathlineto{\pgfqpoint{5.111716in}{3.114632in}}%
\pgfpathlineto{\pgfqpoint{5.114927in}{3.114455in}}%
\pgfpathlineto{\pgfqpoint{5.118138in}{3.116628in}}%
\pgfpathlineto{\pgfqpoint{5.122955in}{3.123139in}}%
\pgfpathlineto{\pgfqpoint{5.131519in}{3.134831in}}%
\pgfpathlineto{\pgfqpoint{5.134997in}{3.136186in}}%
\pgfpathlineto{\pgfqpoint{5.138209in}{3.134941in}}%
\pgfpathlineto{\pgfqpoint{5.142223in}{3.130473in}}%
\pgfpathlineto{\pgfqpoint{5.154533in}{3.114608in}}%
\pgfpathlineto{\pgfqpoint{5.157744in}{3.114474in}}%
\pgfpathlineto{\pgfqpoint{5.160955in}{3.116686in}}%
\pgfpathlineto{\pgfqpoint{5.165772in}{3.123231in}}%
\pgfpathlineto{\pgfqpoint{5.174336in}{3.134875in}}%
\pgfpathlineto{\pgfqpoint{5.177815in}{3.136184in}}%
\pgfpathlineto{\pgfqpoint{5.181026in}{3.134897in}}%
\pgfpathlineto{\pgfqpoint{5.185040in}{3.130390in}}%
\pgfpathlineto{\pgfqpoint{5.197350in}{3.114584in}}%
\pgfpathlineto{\pgfqpoint{5.200561in}{3.114494in}}%
\pgfpathlineto{\pgfqpoint{5.203773in}{3.116746in}}%
\pgfpathlineto{\pgfqpoint{5.208590in}{3.123323in}}%
\pgfpathlineto{\pgfqpoint{5.217153in}{3.134919in}}%
\pgfpathlineto{\pgfqpoint{5.220632in}{3.136182in}}%
\pgfpathlineto{\pgfqpoint{5.223843in}{3.134853in}}%
\pgfpathlineto{\pgfqpoint{5.227857in}{3.130307in}}%
\pgfpathlineto{\pgfqpoint{5.239900in}{3.114673in}}%
\pgfpathlineto{\pgfqpoint{5.243111in}{3.114426in}}%
\pgfpathlineto{\pgfqpoint{5.246322in}{3.116535in}}%
\pgfpathlineto{\pgfqpoint{5.251139in}{3.122992in}}%
\pgfpathlineto{\pgfqpoint{5.259970in}{3.134963in}}%
\pgfpathlineto{\pgfqpoint{5.263449in}{3.136179in}}%
\pgfpathlineto{\pgfqpoint{5.266661in}{3.134808in}}%
\pgfpathlineto{\pgfqpoint{5.270675in}{3.130224in}}%
\pgfpathlineto{\pgfqpoint{5.282717in}{3.114647in}}%
\pgfpathlineto{\pgfqpoint{5.285928in}{3.114443in}}%
\pgfpathlineto{\pgfqpoint{5.289140in}{3.116593in}}%
\pgfpathlineto{\pgfqpoint{5.293957in}{3.123084in}}%
\pgfpathlineto{\pgfqpoint{5.302520in}{3.134803in}}%
\pgfpathlineto{\pgfqpoint{5.305999in}{3.136186in}}%
\pgfpathlineto{\pgfqpoint{5.309210in}{3.134967in}}%
\pgfpathlineto{\pgfqpoint{5.313224in}{3.130522in}}%
\pgfpathlineto{\pgfqpoint{5.325802in}{3.114516in}}%
\pgfpathlineto{\pgfqpoint{5.329013in}{3.114558in}}%
\pgfpathlineto{\pgfqpoint{5.332492in}{3.117218in}}%
\pgfpathlineto{\pgfqpoint{5.337844in}{3.124890in}}%
\pgfpathlineto{\pgfqpoint{5.345070in}{3.134635in}}%
\pgfpathlineto{\pgfqpoint{5.348549in}{3.136182in}}%
\pgfpathlineto{\pgfqpoint{5.351760in}{3.135117in}}%
\pgfpathlineto{\pgfqpoint{5.355506in}{3.131182in}}%
\pgfpathlineto{\pgfqpoint{5.369154in}{3.114341in}}%
\pgfpathlineto{\pgfqpoint{5.372366in}{3.114830in}}%
\pgfpathlineto{\pgfqpoint{5.375845in}{3.117903in}}%
\pgfpathlineto{\pgfqpoint{5.382000in}{3.127132in}}%
\pgfpathlineto{\pgfqpoint{5.388154in}{3.134893in}}%
\pgfpathlineto{\pgfqpoint{5.391633in}{3.136184in}}%
\pgfpathlineto{\pgfqpoint{5.394845in}{3.134880in}}%
\pgfpathlineto{\pgfqpoint{5.398859in}{3.130357in}}%
\pgfpathlineto{\pgfqpoint{5.411169in}{3.114574in}}%
\pgfpathlineto{\pgfqpoint{5.414380in}{3.114502in}}%
\pgfpathlineto{\pgfqpoint{5.417591in}{3.116770in}}%
\pgfpathlineto{\pgfqpoint{5.422408in}{3.123360in}}%
\pgfpathlineto{\pgfqpoint{5.430972in}{3.134937in}}%
\pgfpathlineto{\pgfqpoint{5.434451in}{3.136181in}}%
\pgfpathlineto{\pgfqpoint{5.437662in}{3.134835in}}%
\pgfpathlineto{\pgfqpoint{5.441676in}{3.130274in}}%
\pgfpathlineto{\pgfqpoint{5.453718in}{3.114663in}}%
\pgfpathlineto{\pgfqpoint{5.456930in}{3.114433in}}%
\pgfpathlineto{\pgfqpoint{5.460141in}{3.116558in}}%
\pgfpathlineto{\pgfqpoint{5.464958in}{3.123028in}}%
\pgfpathlineto{\pgfqpoint{5.473789in}{3.134980in}}%
\pgfpathlineto{\pgfqpoint{5.477268in}{3.136178in}}%
\pgfpathlineto{\pgfqpoint{5.480479in}{3.134790in}}%
\pgfpathlineto{\pgfqpoint{5.484493in}{3.130191in}}%
\pgfpathlineto{\pgfqpoint{5.496536in}{3.114637in}}%
\pgfpathlineto{\pgfqpoint{5.499747in}{3.114451in}}%
\pgfpathlineto{\pgfqpoint{5.502958in}{3.116616in}}%
\pgfpathlineto{\pgfqpoint{5.507775in}{3.123120in}}%
\pgfpathlineto{\pgfqpoint{5.516339in}{3.134821in}}%
\pgfpathlineto{\pgfqpoint{5.519818in}{3.136186in}}%
\pgfpathlineto{\pgfqpoint{5.523029in}{3.134950in}}%
\pgfpathlineto{\pgfqpoint{5.527043in}{3.130489in}}%
\pgfpathlineto{\pgfqpoint{5.539353in}{3.114612in}}%
\pgfpathlineto{\pgfqpoint{5.542564in}{3.114470in}}%
\pgfpathlineto{\pgfqpoint{5.545775in}{3.116675in}}%
\pgfpathlineto{\pgfqpoint{5.550592in}{3.123213in}}%
\pgfpathlineto{\pgfqpoint{5.559156in}{3.134866in}}%
\pgfpathlineto{\pgfqpoint{5.562635in}{3.136185in}}%
\pgfpathlineto{\pgfqpoint{5.565846in}{3.134906in}}%
\pgfpathlineto{\pgfqpoint{5.569860in}{3.130407in}}%
\pgfpathlineto{\pgfqpoint{5.582170in}{3.114588in}}%
\pgfpathlineto{\pgfqpoint{5.585381in}{3.114490in}}%
\pgfpathlineto{\pgfqpoint{5.588593in}{3.116734in}}%
\pgfpathlineto{\pgfqpoint{5.593410in}{3.123305in}}%
\pgfpathlineto{\pgfqpoint{5.601973in}{3.134911in}}%
\pgfpathlineto{\pgfqpoint{5.605452in}{3.136183in}}%
\pgfpathlineto{\pgfqpoint{5.608663in}{3.134862in}}%
\pgfpathlineto{\pgfqpoint{5.612677in}{3.130324in}}%
\pgfpathlineto{\pgfqpoint{5.624720in}{3.114679in}}%
\pgfpathlineto{\pgfqpoint{5.627931in}{3.114422in}}%
\pgfpathlineto{\pgfqpoint{5.631142in}{3.116523in}}%
\pgfpathlineto{\pgfqpoint{5.635692in}{3.122552in}}%
\pgfpathlineto{\pgfqpoint{5.644790in}{3.134954in}}%
\pgfpathlineto{\pgfqpoint{5.648269in}{3.136180in}}%
\pgfpathlineto{\pgfqpoint{5.651481in}{3.134817in}}%
\pgfpathlineto{\pgfqpoint{5.655495in}{3.130241in}}%
\pgfpathlineto{\pgfqpoint{5.667537in}{3.114653in}}%
\pgfpathlineto{\pgfqpoint{5.670748in}{3.114440in}}%
\pgfpathlineto{\pgfqpoint{5.673960in}{3.116581in}}%
\pgfpathlineto{\pgfqpoint{5.678777in}{3.123065in}}%
\pgfpathlineto{\pgfqpoint{5.687340in}{3.134794in}}%
\pgfpathlineto{\pgfqpoint{5.690819in}{3.136186in}}%
\pgfpathlineto{\pgfqpoint{5.694030in}{3.134976in}}%
\pgfpathlineto{\pgfqpoint{5.698044in}{3.130538in}}%
\pgfpathlineto{\pgfqpoint{5.710622in}{3.114521in}}%
\pgfpathlineto{\pgfqpoint{5.713833in}{3.114554in}}%
\pgfpathlineto{\pgfqpoint{5.717312in}{3.117205in}}%
\pgfpathlineto{\pgfqpoint{5.722664in}{3.124872in}}%
\pgfpathlineto{\pgfqpoint{5.729890in}{3.134625in}}%
\pgfpathlineto{\pgfqpoint{5.733369in}{3.136181in}}%
\pgfpathlineto{\pgfqpoint{5.736580in}{3.135125in}}%
\pgfpathlineto{\pgfqpoint{5.740326in}{3.131198in}}%
\pgfpathlineto{\pgfqpoint{5.753974in}{3.114344in}}%
\pgfpathlineto{\pgfqpoint{5.757186in}{3.114824in}}%
\pgfpathlineto{\pgfqpoint{5.760665in}{3.117889in}}%
\pgfpathlineto{\pgfqpoint{5.766820in}{3.127114in}}%
\pgfpathlineto{\pgfqpoint{5.772975in}{3.134884in}}%
\pgfpathlineto{\pgfqpoint{5.776453in}{3.136184in}}%
\pgfpathlineto{\pgfqpoint{5.779665in}{3.134889in}}%
\pgfpathlineto{\pgfqpoint{5.783679in}{3.130374in}}%
\pgfpathlineto{\pgfqpoint{5.795989in}{3.114579in}}%
\pgfpathlineto{\pgfqpoint{5.799200in}{3.114498in}}%
\pgfpathlineto{\pgfqpoint{5.802411in}{3.116758in}}%
\pgfpathlineto{\pgfqpoint{5.807228in}{3.123342in}}%
\pgfpathlineto{\pgfqpoint{5.815792in}{3.134928in}}%
\pgfpathlineto{\pgfqpoint{5.819271in}{3.136182in}}%
\pgfpathlineto{\pgfqpoint{5.822482in}{3.134844in}}%
\pgfpathlineto{\pgfqpoint{5.826496in}{3.130291in}}%
\pgfpathlineto{\pgfqpoint{5.838538in}{3.114668in}}%
\pgfpathlineto{\pgfqpoint{5.841750in}{3.114429in}}%
\pgfpathlineto{\pgfqpoint{5.844961in}{3.116546in}}%
\pgfpathlineto{\pgfqpoint{5.849778in}{3.123010in}}%
\pgfpathlineto{\pgfqpoint{5.858609in}{3.134971in}}%
\pgfpathlineto{\pgfqpoint{5.862088in}{3.136179in}}%
\pgfpathlineto{\pgfqpoint{5.865299in}{3.134799in}}%
\pgfpathlineto{\pgfqpoint{5.869313in}{3.130207in}}%
\pgfpathlineto{\pgfqpoint{5.881356in}{3.114642in}}%
\pgfpathlineto{\pgfqpoint{5.884567in}{3.114447in}}%
\pgfpathlineto{\pgfqpoint{5.887778in}{3.116604in}}%
\pgfpathlineto{\pgfqpoint{5.892595in}{3.123102in}}%
\pgfpathlineto{\pgfqpoint{5.901159in}{3.134812in}}%
\pgfpathlineto{\pgfqpoint{5.904638in}{3.136186in}}%
\pgfpathlineto{\pgfqpoint{5.907849in}{3.134959in}}%
\pgfpathlineto{\pgfqpoint{5.911863in}{3.130506in}}%
\pgfpathlineto{\pgfqpoint{5.924173in}{3.114617in}}%
\pgfpathlineto{\pgfqpoint{5.927384in}{3.114466in}}%
\pgfpathlineto{\pgfqpoint{5.930596in}{3.116663in}}%
\pgfpathlineto{\pgfqpoint{5.935412in}{3.123194in}}%
\pgfpathlineto{\pgfqpoint{5.943976in}{3.134857in}}%
\pgfpathlineto{\pgfqpoint{5.947455in}{3.136185in}}%
\pgfpathlineto{\pgfqpoint{5.950666in}{3.134915in}}%
\pgfpathlineto{\pgfqpoint{5.954680in}{3.130423in}}%
\pgfpathlineto{\pgfqpoint{5.966990in}{3.114593in}}%
\pgfpathlineto{\pgfqpoint{5.970202in}{3.114486in}}%
\pgfpathlineto{\pgfqpoint{5.973413in}{3.116722in}}%
\pgfpathlineto{\pgfqpoint{5.978230in}{3.123286in}}%
\pgfpathlineto{\pgfqpoint{5.986793in}{3.134902in}}%
\pgfpathlineto{\pgfqpoint{5.990272in}{3.136183in}}%
\pgfpathlineto{\pgfqpoint{5.993483in}{3.134871in}}%
\pgfpathlineto{\pgfqpoint{5.997497in}{3.130341in}}%
\pgfpathlineto{\pgfqpoint{6.009540in}{3.114684in}}%
\pgfpathlineto{\pgfqpoint{6.012751in}{3.114419in}}%
\pgfpathlineto{\pgfqpoint{6.015962in}{3.116512in}}%
\pgfpathlineto{\pgfqpoint{6.020512in}{3.122534in}}%
\pgfpathlineto{\pgfqpoint{6.029610in}{3.134946in}}%
\pgfpathlineto{\pgfqpoint{6.033089in}{3.136181in}}%
\pgfpathlineto{\pgfqpoint{6.036301in}{3.134826in}}%
\pgfpathlineto{\pgfqpoint{6.040315in}{3.130257in}}%
\pgfpathlineto{\pgfqpoint{6.052357in}{3.114658in}}%
\pgfpathlineto{\pgfqpoint{6.055568in}{3.114436in}}%
\pgfpathlineto{\pgfqpoint{6.058780in}{3.116569in}}%
\pgfpathlineto{\pgfqpoint{6.063597in}{3.123047in}}%
\pgfpathlineto{\pgfqpoint{6.072428in}{3.134989in}}%
\pgfpathlineto{\pgfqpoint{6.075907in}{3.136177in}}%
\pgfpathlineto{\pgfqpoint{6.079118in}{3.134780in}}%
\pgfpathlineto{\pgfqpoint{6.083132in}{3.130174in}}%
\pgfpathlineto{\pgfqpoint{6.095174in}{3.114632in}}%
\pgfpathlineto{\pgfqpoint{6.098386in}{3.114455in}}%
\pgfpathlineto{\pgfqpoint{6.101597in}{3.116628in}}%
\pgfpathlineto{\pgfqpoint{6.106414in}{3.123139in}}%
\pgfpathlineto{\pgfqpoint{6.114977in}{3.134831in}}%
\pgfpathlineto{\pgfqpoint{6.118456in}{3.136186in}}%
\pgfpathlineto{\pgfqpoint{6.121668in}{3.134941in}}%
\pgfpathlineto{\pgfqpoint{6.125682in}{3.130473in}}%
\pgfpathlineto{\pgfqpoint{6.137992in}{3.114608in}}%
\pgfpathlineto{\pgfqpoint{6.141203in}{3.114474in}}%
\pgfpathlineto{\pgfqpoint{6.144414in}{3.116686in}}%
\pgfpathlineto{\pgfqpoint{6.149231in}{3.123231in}}%
\pgfpathlineto{\pgfqpoint{6.157795in}{3.134875in}}%
\pgfpathlineto{\pgfqpoint{6.161273in}{3.136184in}}%
\pgfpathlineto{\pgfqpoint{6.164485in}{3.134897in}}%
\pgfpathlineto{\pgfqpoint{6.168499in}{3.130390in}}%
\pgfpathlineto{\pgfqpoint{6.180809in}{3.114584in}}%
\pgfpathlineto{\pgfqpoint{6.184020in}{3.114494in}}%
\pgfpathlineto{\pgfqpoint{6.187231in}{3.116746in}}%
\pgfpathlineto{\pgfqpoint{6.192048in}{3.123323in}}%
\pgfpathlineto{\pgfqpoint{6.200612in}{3.134919in}}%
\pgfpathlineto{\pgfqpoint{6.204091in}{3.136182in}}%
\pgfpathlineto{\pgfqpoint{6.207302in}{3.134853in}}%
\pgfpathlineto{\pgfqpoint{6.211316in}{3.130307in}}%
\pgfpathlineto{\pgfqpoint{6.223358in}{3.114673in}}%
\pgfpathlineto{\pgfqpoint{6.226570in}{3.114426in}}%
\pgfpathlineto{\pgfqpoint{6.229781in}{3.116535in}}%
\pgfpathlineto{\pgfqpoint{6.234598in}{3.122992in}}%
\pgfpathlineto{\pgfqpoint{6.243429in}{3.134963in}}%
\pgfpathlineto{\pgfqpoint{6.246908in}{3.136179in}}%
\pgfpathlineto{\pgfqpoint{6.250119in}{3.134808in}}%
\pgfpathlineto{\pgfqpoint{6.254133in}{3.130224in}}%
\pgfpathlineto{\pgfqpoint{6.266176in}{3.114647in}}%
\pgfpathlineto{\pgfqpoint{6.269387in}{3.114443in}}%
\pgfpathlineto{\pgfqpoint{6.272598in}{3.116593in}}%
\pgfpathlineto{\pgfqpoint{6.277415in}{3.123084in}}%
\pgfpathlineto{\pgfqpoint{6.285979in}{3.134803in}}%
\pgfpathlineto{\pgfqpoint{6.289458in}{3.136186in}}%
\pgfpathlineto{\pgfqpoint{6.292669in}{3.134967in}}%
\pgfpathlineto{\pgfqpoint{6.296683in}{3.130522in}}%
\pgfpathlineto{\pgfqpoint{6.309261in}{3.114516in}}%
\pgfpathlineto{\pgfqpoint{6.312472in}{3.114558in}}%
\pgfpathlineto{\pgfqpoint{6.315951in}{3.117218in}}%
\pgfpathlineto{\pgfqpoint{6.321303in}{3.124890in}}%
\pgfpathlineto{\pgfqpoint{6.328528in}{3.134635in}}%
\pgfpathlineto{\pgfqpoint{6.332007in}{3.136182in}}%
\pgfpathlineto{\pgfqpoint{6.335219in}{3.135117in}}%
\pgfpathlineto{\pgfqpoint{6.338965in}{3.131182in}}%
\pgfpathlineto{\pgfqpoint{6.352613in}{3.114341in}}%
\pgfpathlineto{\pgfqpoint{6.355824in}{3.114830in}}%
\pgfpathlineto{\pgfqpoint{6.359303in}{3.117903in}}%
\pgfpathlineto{\pgfqpoint{6.365458in}{3.127132in}}%
\pgfpathlineto{\pgfqpoint{6.371613in}{3.134893in}}%
\pgfpathlineto{\pgfqpoint{6.375092in}{3.136184in}}%
\pgfpathlineto{\pgfqpoint{6.378303in}{3.134880in}}%
\pgfpathlineto{\pgfqpoint{6.382318in}{3.130357in}}%
\pgfpathlineto{\pgfqpoint{6.394628in}{3.114574in}}%
\pgfpathlineto{\pgfqpoint{6.397839in}{3.114502in}}%
\pgfpathlineto{\pgfqpoint{6.401050in}{3.116770in}}%
\pgfpathlineto{\pgfqpoint{6.405867in}{3.123360in}}%
\pgfpathlineto{\pgfqpoint{6.414430in}{3.134937in}}%
\pgfpathlineto{\pgfqpoint{6.417909in}{3.136181in}}%
\pgfpathlineto{\pgfqpoint{6.421121in}{3.134835in}}%
\pgfpathlineto{\pgfqpoint{6.425135in}{3.130274in}}%
\pgfpathlineto{\pgfqpoint{6.437177in}{3.114663in}}%
\pgfpathlineto{\pgfqpoint{6.440388in}{3.114433in}}%
\pgfpathlineto{\pgfqpoint{6.443600in}{3.116558in}}%
\pgfpathlineto{\pgfqpoint{6.448417in}{3.123028in}}%
\pgfpathlineto{\pgfqpoint{6.457248in}{3.134980in}}%
\pgfpathlineto{\pgfqpoint{6.460727in}{3.136178in}}%
\pgfpathlineto{\pgfqpoint{6.463938in}{3.134790in}}%
\pgfpathlineto{\pgfqpoint{6.467952in}{3.130191in}}%
\pgfpathlineto{\pgfqpoint{6.479994in}{3.114637in}}%
\pgfpathlineto{\pgfqpoint{6.483206in}{3.114451in}}%
\pgfpathlineto{\pgfqpoint{6.486417in}{3.116616in}}%
\pgfpathlineto{\pgfqpoint{6.491234in}{3.123120in}}%
\pgfpathlineto{\pgfqpoint{6.499797in}{3.134821in}}%
\pgfpathlineto{\pgfqpoint{6.503276in}{3.136186in}}%
\pgfpathlineto{\pgfqpoint{6.506488in}{3.134950in}}%
\pgfpathlineto{\pgfqpoint{6.510502in}{3.130489in}}%
\pgfpathlineto{\pgfqpoint{6.522812in}{3.114612in}}%
\pgfpathlineto{\pgfqpoint{6.526023in}{3.114470in}}%
\pgfpathlineto{\pgfqpoint{6.529234in}{3.116675in}}%
\pgfpathlineto{\pgfqpoint{6.534051in}{3.123213in}}%
\pgfpathlineto{\pgfqpoint{6.542615in}{3.134866in}}%
\pgfpathlineto{\pgfqpoint{6.546094in}{3.136185in}}%
\pgfpathlineto{\pgfqpoint{6.549305in}{3.134906in}}%
\pgfpathlineto{\pgfqpoint{6.553319in}{3.130407in}}%
\pgfpathlineto{\pgfqpoint{6.565629in}{3.114588in}}%
\pgfpathlineto{\pgfqpoint{6.568840in}{3.114490in}}%
\pgfpathlineto{\pgfqpoint{6.572051in}{3.116734in}}%
\pgfpathlineto{\pgfqpoint{6.576868in}{3.123305in}}%
\pgfpathlineto{\pgfqpoint{6.585432in}{3.134911in}}%
\pgfpathlineto{\pgfqpoint{6.588911in}{3.136183in}}%
\pgfpathlineto{\pgfqpoint{6.592122in}{3.134862in}}%
\pgfpathlineto{\pgfqpoint{6.596136in}{3.130324in}}%
\pgfpathlineto{\pgfqpoint{6.608179in}{3.114679in}}%
\pgfpathlineto{\pgfqpoint{6.611390in}{3.114422in}}%
\pgfpathlineto{\pgfqpoint{6.614601in}{3.116523in}}%
\pgfpathlineto{\pgfqpoint{6.619150in}{3.122552in}}%
\pgfpathlineto{\pgfqpoint{6.628249in}{3.134954in}}%
\pgfpathlineto{\pgfqpoint{6.631728in}{3.136180in}}%
\pgfpathlineto{\pgfqpoint{6.634939in}{3.134817in}}%
\pgfpathlineto{\pgfqpoint{6.638953in}{3.130241in}}%
\pgfpathlineto{\pgfqpoint{6.650996in}{3.114653in}}%
\pgfpathlineto{\pgfqpoint{6.654207in}{3.114440in}}%
\pgfpathlineto{\pgfqpoint{6.657418in}{3.116581in}}%
\pgfpathlineto{\pgfqpoint{6.662235in}{3.123065in}}%
\pgfpathlineto{\pgfqpoint{6.663306in}{3.124778in}}%
\pgfpathlineto{\pgfqpoint{6.663306in}{3.124778in}}%
\pgfusepath{stroke}%
\end{pgfscope}%
\begin{pgfscope}%
\pgfpathrectangle{\pgfqpoint{0.467797in}{2.292089in}}{\pgfqpoint{6.490533in}{1.666241in}}%
\pgfusepath{clip}%
\pgfsetrectcap%
\pgfsetroundjoin%
\pgfsetlinewidth{1.505625pt}%
\definecolor{currentstroke}{rgb}{0.549020,0.337255,0.294118}%
\pgfsetstrokecolor{currentstroke}%
\pgfsetdash{}{0pt}%
\pgfpathmoveto{\pgfqpoint{0.762821in}{3.125209in}}%
\pgfpathlineto{\pgfqpoint{0.769779in}{3.134473in}}%
\pgfpathlineto{\pgfqpoint{0.773258in}{3.135877in}}%
\pgfpathlineto{\pgfqpoint{0.776201in}{3.134804in}}%
\pgfpathlineto{\pgfqpoint{0.779948in}{3.130803in}}%
\pgfpathlineto{\pgfqpoint{0.792793in}{3.114715in}}%
\pgfpathlineto{\pgfqpoint{0.796004in}{3.115035in}}%
\pgfpathlineto{\pgfqpoint{0.799483in}{3.118019in}}%
\pgfpathlineto{\pgfqpoint{0.805371in}{3.126808in}}%
\pgfpathlineto{\pgfqpoint{0.811526in}{3.134623in}}%
\pgfpathlineto{\pgfqpoint{0.815004in}{3.135870in}}%
\pgfpathlineto{\pgfqpoint{0.817948in}{3.134664in}}%
\pgfpathlineto{\pgfqpoint{0.821962in}{3.130158in}}%
\pgfpathlineto{\pgfqpoint{0.833737in}{3.114937in}}%
\pgfpathlineto{\pgfqpoint{0.836948in}{3.114783in}}%
\pgfpathlineto{\pgfqpoint{0.840160in}{3.117039in}}%
\pgfpathlineto{\pgfqpoint{0.844977in}{3.123677in}}%
\pgfpathlineto{\pgfqpoint{0.853005in}{3.134565in}}%
\pgfpathlineto{\pgfqpoint{0.856484in}{3.135873in}}%
\pgfpathlineto{\pgfqpoint{0.859427in}{3.134720in}}%
\pgfpathlineto{\pgfqpoint{0.863174in}{3.130642in}}%
\pgfpathlineto{\pgfqpoint{0.875751in}{3.114762in}}%
\pgfpathlineto{\pgfqpoint{0.878963in}{3.114964in}}%
\pgfpathlineto{\pgfqpoint{0.882442in}{3.117841in}}%
\pgfpathlineto{\pgfqpoint{0.888061in}{3.126138in}}%
\pgfpathlineto{\pgfqpoint{0.894484in}{3.134506in}}%
\pgfpathlineto{\pgfqpoint{0.897963in}{3.135876in}}%
\pgfpathlineto{\pgfqpoint{0.900907in}{3.134774in}}%
\pgfpathlineto{\pgfqpoint{0.904653in}{3.130746in}}%
\pgfpathlineto{\pgfqpoint{0.917498in}{3.114704in}}%
\pgfpathlineto{\pgfqpoint{0.920442in}{3.114931in}}%
\pgfpathlineto{\pgfqpoint{0.923921in}{3.117753in}}%
\pgfpathlineto{\pgfqpoint{0.929541in}{3.126017in}}%
\pgfpathlineto{\pgfqpoint{0.935963in}{3.134446in}}%
\pgfpathlineto{\pgfqpoint{0.939442in}{3.135877in}}%
\pgfpathlineto{\pgfqpoint{0.942386in}{3.134828in}}%
\pgfpathlineto{\pgfqpoint{0.946132in}{3.130849in}}%
\pgfpathlineto{\pgfqpoint{0.958977in}{3.114725in}}%
\pgfpathlineto{\pgfqpoint{0.962189in}{3.115018in}}%
\pgfpathlineto{\pgfqpoint{0.965668in}{3.117978in}}%
\pgfpathlineto{\pgfqpoint{0.971555in}{3.126754in}}%
\pgfpathlineto{\pgfqpoint{0.977710in}{3.134597in}}%
\pgfpathlineto{\pgfqpoint{0.981189in}{3.135872in}}%
\pgfpathlineto{\pgfqpoint{0.984133in}{3.134690in}}%
\pgfpathlineto{\pgfqpoint{0.988147in}{3.130207in}}%
\pgfpathlineto{\pgfqpoint{0.999921in}{3.114952in}}%
\pgfpathlineto{\pgfqpoint{1.003133in}{3.114771in}}%
\pgfpathlineto{\pgfqpoint{1.006344in}{3.117004in}}%
\pgfpathlineto{\pgfqpoint{1.011161in}{3.123623in}}%
\pgfpathlineto{\pgfqpoint{1.019189in}{3.134539in}}%
\pgfpathlineto{\pgfqpoint{1.022668in}{3.135875in}}%
\pgfpathlineto{\pgfqpoint{1.025612in}{3.134745in}}%
\pgfpathlineto{\pgfqpoint{1.029358in}{3.130689in}}%
\pgfpathlineto{\pgfqpoint{1.041936in}{3.114773in}}%
\pgfpathlineto{\pgfqpoint{1.045147in}{3.114949in}}%
\pgfpathlineto{\pgfqpoint{1.048626in}{3.117801in}}%
\pgfpathlineto{\pgfqpoint{1.054246in}{3.126084in}}%
\pgfpathlineto{\pgfqpoint{1.060668in}{3.134479in}}%
\pgfpathlineto{\pgfqpoint{1.064147in}{3.135876in}}%
\pgfpathlineto{\pgfqpoint{1.067091in}{3.134799in}}%
\pgfpathlineto{\pgfqpoint{1.070838in}{3.130793in}}%
\pgfpathlineto{\pgfqpoint{1.083683in}{3.114713in}}%
\pgfpathlineto{\pgfqpoint{1.086894in}{3.115038in}}%
\pgfpathlineto{\pgfqpoint{1.090373in}{3.118028in}}%
\pgfpathlineto{\pgfqpoint{1.096528in}{3.127246in}}%
\pgfpathlineto{\pgfqpoint{1.102415in}{3.134629in}}%
\pgfpathlineto{\pgfqpoint{1.105894in}{3.135869in}}%
\pgfpathlineto{\pgfqpoint{1.108838in}{3.134659in}}%
\pgfpathlineto{\pgfqpoint{1.112852in}{3.130148in}}%
\pgfpathlineto{\pgfqpoint{1.124627in}{3.114933in}}%
\pgfpathlineto{\pgfqpoint{1.127838in}{3.114785in}}%
\pgfpathlineto{\pgfqpoint{1.131049in}{3.117047in}}%
\pgfpathlineto{\pgfqpoint{1.135866in}{3.123689in}}%
\pgfpathlineto{\pgfqpoint{1.143894in}{3.134571in}}%
\pgfpathlineto{\pgfqpoint{1.147373in}{3.135873in}}%
\pgfpathlineto{\pgfqpoint{1.150317in}{3.134714in}}%
\pgfpathlineto{\pgfqpoint{1.154064in}{3.130631in}}%
\pgfpathlineto{\pgfqpoint{1.166641in}{3.114759in}}%
\pgfpathlineto{\pgfqpoint{1.169852in}{3.114968in}}%
\pgfpathlineto{\pgfqpoint{1.173331in}{3.117850in}}%
\pgfpathlineto{\pgfqpoint{1.179219in}{3.126580in}}%
\pgfpathlineto{\pgfqpoint{1.185374in}{3.134512in}}%
\pgfpathlineto{\pgfqpoint{1.188853in}{3.135876in}}%
\pgfpathlineto{\pgfqpoint{1.191796in}{3.134769in}}%
\pgfpathlineto{\pgfqpoint{1.195543in}{3.130735in}}%
\pgfpathlineto{\pgfqpoint{1.208388in}{3.114702in}}%
\pgfpathlineto{\pgfqpoint{1.211332in}{3.114934in}}%
\pgfpathlineto{\pgfqpoint{1.214811in}{3.117762in}}%
\pgfpathlineto{\pgfqpoint{1.220430in}{3.126029in}}%
\pgfpathlineto{\pgfqpoint{1.226853in}{3.134452in}}%
\pgfpathlineto{\pgfqpoint{1.230332in}{3.135877in}}%
\pgfpathlineto{\pgfqpoint{1.233275in}{3.134822in}}%
\pgfpathlineto{\pgfqpoint{1.237022in}{3.130839in}}%
\pgfpathlineto{\pgfqpoint{1.249867in}{3.114723in}}%
\pgfpathlineto{\pgfqpoint{1.253078in}{3.115022in}}%
\pgfpathlineto{\pgfqpoint{1.256557in}{3.117987in}}%
\pgfpathlineto{\pgfqpoint{1.262445in}{3.126766in}}%
\pgfpathlineto{\pgfqpoint{1.268600in}{3.134603in}}%
\pgfpathlineto{\pgfqpoint{1.272079in}{3.135871in}}%
\pgfpathlineto{\pgfqpoint{1.275022in}{3.134684in}}%
\pgfpathlineto{\pgfqpoint{1.279036in}{3.130196in}}%
\pgfpathlineto{\pgfqpoint{1.290811in}{3.114948in}}%
\pgfpathlineto{\pgfqpoint{1.294022in}{3.114774in}}%
\pgfpathlineto{\pgfqpoint{1.297234in}{3.117011in}}%
\pgfpathlineto{\pgfqpoint{1.302051in}{3.123635in}}%
\pgfpathlineto{\pgfqpoint{1.310079in}{3.134545in}}%
\pgfpathlineto{\pgfqpoint{1.313558in}{3.135874in}}%
\pgfpathlineto{\pgfqpoint{1.316501in}{3.134739in}}%
\pgfpathlineto{\pgfqpoint{1.320248in}{3.130678in}}%
\pgfpathlineto{\pgfqpoint{1.332826in}{3.114770in}}%
\pgfpathlineto{\pgfqpoint{1.336037in}{3.114952in}}%
\pgfpathlineto{\pgfqpoint{1.339516in}{3.117810in}}%
\pgfpathlineto{\pgfqpoint{1.345136in}{3.126096in}}%
\pgfpathlineto{\pgfqpoint{1.351558in}{3.134485in}}%
\pgfpathlineto{\pgfqpoint{1.355037in}{3.135876in}}%
\pgfpathlineto{\pgfqpoint{1.357981in}{3.134793in}}%
\pgfpathlineto{\pgfqpoint{1.361727in}{3.130782in}}%
\pgfpathlineto{\pgfqpoint{1.374572in}{3.114711in}}%
\pgfpathlineto{\pgfqpoint{1.377784in}{3.115042in}}%
\pgfpathlineto{\pgfqpoint{1.381263in}{3.118037in}}%
\pgfpathlineto{\pgfqpoint{1.387418in}{3.127258in}}%
\pgfpathlineto{\pgfqpoint{1.393305in}{3.134635in}}%
\pgfpathlineto{\pgfqpoint{1.396784in}{3.135869in}}%
\pgfpathlineto{\pgfqpoint{1.399728in}{3.134653in}}%
\pgfpathlineto{\pgfqpoint{1.403742in}{3.130137in}}%
\pgfpathlineto{\pgfqpoint{1.415516in}{3.114930in}}%
\pgfpathlineto{\pgfqpoint{1.418728in}{3.114788in}}%
\pgfpathlineto{\pgfqpoint{1.421939in}{3.117054in}}%
\pgfpathlineto{\pgfqpoint{1.426756in}{3.123701in}}%
\pgfpathlineto{\pgfqpoint{1.434784in}{3.134577in}}%
\pgfpathlineto{\pgfqpoint{1.438263in}{3.135873in}}%
\pgfpathlineto{\pgfqpoint{1.441207in}{3.134709in}}%
\pgfpathlineto{\pgfqpoint{1.444953in}{3.130621in}}%
\pgfpathlineto{\pgfqpoint{1.457531in}{3.114757in}}%
\pgfpathlineto{\pgfqpoint{1.460742in}{3.114971in}}%
\pgfpathlineto{\pgfqpoint{1.464221in}{3.117859in}}%
\pgfpathlineto{\pgfqpoint{1.470108in}{3.126592in}}%
\pgfpathlineto{\pgfqpoint{1.476263in}{3.134518in}}%
\pgfpathlineto{\pgfqpoint{1.479742in}{3.135875in}}%
\pgfpathlineto{\pgfqpoint{1.482686in}{3.134764in}}%
\pgfpathlineto{\pgfqpoint{1.486432in}{3.130725in}}%
\pgfpathlineto{\pgfqpoint{1.499010in}{3.114782in}}%
\pgfpathlineto{\pgfqpoint{1.502221in}{3.114937in}}%
\pgfpathlineto{\pgfqpoint{1.505700in}{3.117771in}}%
\pgfpathlineto{\pgfqpoint{1.511320in}{3.126041in}}%
\pgfpathlineto{\pgfqpoint{1.517743in}{3.134458in}}%
\pgfpathlineto{\pgfqpoint{1.521221in}{3.135877in}}%
\pgfpathlineto{\pgfqpoint{1.524165in}{3.134817in}}%
\pgfpathlineto{\pgfqpoint{1.527912in}{3.130829in}}%
\pgfpathlineto{\pgfqpoint{1.540757in}{3.114721in}}%
\pgfpathlineto{\pgfqpoint{1.543968in}{3.115025in}}%
\pgfpathlineto{\pgfqpoint{1.547447in}{3.117996in}}%
\pgfpathlineto{\pgfqpoint{1.553334in}{3.126778in}}%
\pgfpathlineto{\pgfqpoint{1.559489in}{3.134609in}}%
\pgfpathlineto{\pgfqpoint{1.562968in}{3.135871in}}%
\pgfpathlineto{\pgfqpoint{1.565912in}{3.134678in}}%
\pgfpathlineto{\pgfqpoint{1.569926in}{3.130185in}}%
\pgfpathlineto{\pgfqpoint{1.581701in}{3.114945in}}%
\pgfpathlineto{\pgfqpoint{1.584912in}{3.114776in}}%
\pgfpathlineto{\pgfqpoint{1.588123in}{3.117019in}}%
\pgfpathlineto{\pgfqpoint{1.592940in}{3.123647in}}%
\pgfpathlineto{\pgfqpoint{1.600969in}{3.134551in}}%
\pgfpathlineto{\pgfqpoint{1.604448in}{3.135874in}}%
\pgfpathlineto{\pgfqpoint{1.607391in}{3.134734in}}%
\pgfpathlineto{\pgfqpoint{1.611138in}{3.130668in}}%
\pgfpathlineto{\pgfqpoint{1.623715in}{3.114768in}}%
\pgfpathlineto{\pgfqpoint{1.626927in}{3.114956in}}%
\pgfpathlineto{\pgfqpoint{1.630405in}{3.117819in}}%
\pgfpathlineto{\pgfqpoint{1.636025in}{3.126108in}}%
\pgfpathlineto{\pgfqpoint{1.642448in}{3.134491in}}%
\pgfpathlineto{\pgfqpoint{1.645927in}{3.135876in}}%
\pgfpathlineto{\pgfqpoint{1.648870in}{3.134788in}}%
\pgfpathlineto{\pgfqpoint{1.652617in}{3.130772in}}%
\pgfpathlineto{\pgfqpoint{1.665462in}{3.114709in}}%
\pgfpathlineto{\pgfqpoint{1.668673in}{3.115046in}}%
\pgfpathlineto{\pgfqpoint{1.672152in}{3.118046in}}%
\pgfpathlineto{\pgfqpoint{1.678307in}{3.127269in}}%
\pgfpathlineto{\pgfqpoint{1.684195in}{3.134640in}}%
\pgfpathlineto{\pgfqpoint{1.687674in}{3.135868in}}%
\pgfpathlineto{\pgfqpoint{1.690617in}{3.134647in}}%
\pgfpathlineto{\pgfqpoint{1.694631in}{3.130126in}}%
\pgfpathlineto{\pgfqpoint{1.706406in}{3.114927in}}%
\pgfpathlineto{\pgfqpoint{1.709617in}{3.114790in}}%
\pgfpathlineto{\pgfqpoint{1.712829in}{3.117062in}}%
\pgfpathlineto{\pgfqpoint{1.717646in}{3.123713in}}%
\pgfpathlineto{\pgfqpoint{1.725674in}{3.134583in}}%
\pgfpathlineto{\pgfqpoint{1.729153in}{3.135873in}}%
\pgfpathlineto{\pgfqpoint{1.732096in}{3.134703in}}%
\pgfpathlineto{\pgfqpoint{1.735843in}{3.130610in}}%
\pgfpathlineto{\pgfqpoint{1.748421in}{3.114754in}}%
\pgfpathlineto{\pgfqpoint{1.751632in}{3.114974in}}%
\pgfpathlineto{\pgfqpoint{1.755111in}{3.117867in}}%
\pgfpathlineto{\pgfqpoint{1.760998in}{3.126604in}}%
\pgfpathlineto{\pgfqpoint{1.767153in}{3.134524in}}%
\pgfpathlineto{\pgfqpoint{1.770632in}{3.135875in}}%
\pgfpathlineto{\pgfqpoint{1.773576in}{3.134758in}}%
\pgfpathlineto{\pgfqpoint{1.777322in}{3.130715in}}%
\pgfpathlineto{\pgfqpoint{1.789900in}{3.114779in}}%
\pgfpathlineto{\pgfqpoint{1.793111in}{3.114941in}}%
\pgfpathlineto{\pgfqpoint{1.796590in}{3.117780in}}%
\pgfpathlineto{\pgfqpoint{1.802210in}{3.126054in}}%
\pgfpathlineto{\pgfqpoint{1.808632in}{3.134464in}}%
\pgfpathlineto{\pgfqpoint{1.812111in}{3.135877in}}%
\pgfpathlineto{\pgfqpoint{1.815055in}{3.134812in}}%
\pgfpathlineto{\pgfqpoint{1.818801in}{3.130818in}}%
\pgfpathlineto{\pgfqpoint{1.831647in}{3.114719in}}%
\pgfpathlineto{\pgfqpoint{1.834858in}{3.115029in}}%
\pgfpathlineto{\pgfqpoint{1.838337in}{3.118005in}}%
\pgfpathlineto{\pgfqpoint{1.844224in}{3.126790in}}%
\pgfpathlineto{\pgfqpoint{1.850379in}{3.134615in}}%
\pgfpathlineto{\pgfqpoint{1.853858in}{3.135870in}}%
\pgfpathlineto{\pgfqpoint{1.856802in}{3.134673in}}%
\pgfpathlineto{\pgfqpoint{1.860816in}{3.130174in}}%
\pgfpathlineto{\pgfqpoint{1.872591in}{3.114942in}}%
\pgfpathlineto{\pgfqpoint{1.875802in}{3.114779in}}%
\pgfpathlineto{\pgfqpoint{1.879013in}{3.117027in}}%
\pgfpathlineto{\pgfqpoint{1.883830in}{3.123659in}}%
\pgfpathlineto{\pgfqpoint{1.891858in}{3.134557in}}%
\pgfpathlineto{\pgfqpoint{1.895337in}{3.135874in}}%
\pgfpathlineto{\pgfqpoint{1.898281in}{3.134728in}}%
\pgfpathlineto{\pgfqpoint{1.902027in}{3.130657in}}%
\pgfpathlineto{\pgfqpoint{1.914605in}{3.114765in}}%
\pgfpathlineto{\pgfqpoint{1.917816in}{3.114959in}}%
\pgfpathlineto{\pgfqpoint{1.921295in}{3.117828in}}%
\pgfpathlineto{\pgfqpoint{1.926915in}{3.126120in}}%
\pgfpathlineto{\pgfqpoint{1.933338in}{3.134497in}}%
\pgfpathlineto{\pgfqpoint{1.936816in}{3.135876in}}%
\pgfpathlineto{\pgfqpoint{1.939760in}{3.134782in}}%
\pgfpathlineto{\pgfqpoint{1.943507in}{3.130761in}}%
\pgfpathlineto{\pgfqpoint{1.956352in}{3.114707in}}%
\pgfpathlineto{\pgfqpoint{1.959295in}{3.114926in}}%
\pgfpathlineto{\pgfqpoint{1.962774in}{3.117740in}}%
\pgfpathlineto{\pgfqpoint{1.968394in}{3.125999in}}%
\pgfpathlineto{\pgfqpoint{1.975084in}{3.134646in}}%
\pgfpathlineto{\pgfqpoint{1.978563in}{3.135868in}}%
\pgfpathlineto{\pgfqpoint{1.981507in}{3.134642in}}%
\pgfpathlineto{\pgfqpoint{1.985521in}{3.130115in}}%
\pgfpathlineto{\pgfqpoint{1.997296in}{3.114924in}}%
\pgfpathlineto{\pgfqpoint{2.000507in}{3.114793in}}%
\pgfpathlineto{\pgfqpoint{2.003718in}{3.117070in}}%
\pgfpathlineto{\pgfqpoint{2.008535in}{3.123725in}}%
\pgfpathlineto{\pgfqpoint{2.016564in}{3.134589in}}%
\pgfpathlineto{\pgfqpoint{2.020042in}{3.135872in}}%
\pgfpathlineto{\pgfqpoint{2.022986in}{3.134698in}}%
\pgfpathlineto{\pgfqpoint{2.026733in}{3.130600in}}%
\pgfpathlineto{\pgfqpoint{2.039310in}{3.114752in}}%
\pgfpathlineto{\pgfqpoint{2.042522in}{3.114978in}}%
\pgfpathlineto{\pgfqpoint{2.046000in}{3.117876in}}%
\pgfpathlineto{\pgfqpoint{2.051888in}{3.126616in}}%
\pgfpathlineto{\pgfqpoint{2.058043in}{3.134530in}}%
\pgfpathlineto{\pgfqpoint{2.061522in}{3.135875in}}%
\pgfpathlineto{\pgfqpoint{2.064465in}{3.134753in}}%
\pgfpathlineto{\pgfqpoint{2.068212in}{3.130704in}}%
\pgfpathlineto{\pgfqpoint{2.080789in}{3.114777in}}%
\pgfpathlineto{\pgfqpoint{2.084001in}{3.114944in}}%
\pgfpathlineto{\pgfqpoint{2.087480in}{3.117788in}}%
\pgfpathlineto{\pgfqpoint{2.093099in}{3.126066in}}%
\pgfpathlineto{\pgfqpoint{2.099522in}{3.134470in}}%
\pgfpathlineto{\pgfqpoint{2.103001in}{3.135877in}}%
\pgfpathlineto{\pgfqpoint{2.105945in}{3.134806in}}%
\pgfpathlineto{\pgfqpoint{2.109691in}{3.130808in}}%
\pgfpathlineto{\pgfqpoint{2.122536in}{3.114717in}}%
\pgfpathlineto{\pgfqpoint{2.125748in}{3.115033in}}%
\pgfpathlineto{\pgfqpoint{2.129226in}{3.118014in}}%
\pgfpathlineto{\pgfqpoint{2.135114in}{3.126802in}}%
\pgfpathlineto{\pgfqpoint{2.141269in}{3.134620in}}%
\pgfpathlineto{\pgfqpoint{2.144748in}{3.135870in}}%
\pgfpathlineto{\pgfqpoint{2.147691in}{3.134667in}}%
\pgfpathlineto{\pgfqpoint{2.151706in}{3.130164in}}%
\pgfpathlineto{\pgfqpoint{2.163480in}{3.114938in}}%
\pgfpathlineto{\pgfqpoint{2.166692in}{3.114781in}}%
\pgfpathlineto{\pgfqpoint{2.169903in}{3.117035in}}%
\pgfpathlineto{\pgfqpoint{2.174720in}{3.123671in}}%
\pgfpathlineto{\pgfqpoint{2.182748in}{3.134562in}}%
\pgfpathlineto{\pgfqpoint{2.186227in}{3.135874in}}%
\pgfpathlineto{\pgfqpoint{2.189171in}{3.134723in}}%
\pgfpathlineto{\pgfqpoint{2.192917in}{3.130647in}}%
\pgfpathlineto{\pgfqpoint{2.205495in}{3.114763in}}%
\pgfpathlineto{\pgfqpoint{2.208706in}{3.114962in}}%
\pgfpathlineto{\pgfqpoint{2.212185in}{3.117837in}}%
\pgfpathlineto{\pgfqpoint{2.217805in}{3.126132in}}%
\pgfpathlineto{\pgfqpoint{2.224227in}{3.134503in}}%
\pgfpathlineto{\pgfqpoint{2.227706in}{3.135876in}}%
\pgfpathlineto{\pgfqpoint{2.230650in}{3.134777in}}%
\pgfpathlineto{\pgfqpoint{2.234396in}{3.130751in}}%
\pgfpathlineto{\pgfqpoint{2.247241in}{3.114705in}}%
\pgfpathlineto{\pgfqpoint{2.250185in}{3.114929in}}%
\pgfpathlineto{\pgfqpoint{2.253664in}{3.117749in}}%
\pgfpathlineto{\pgfqpoint{2.259284in}{3.126011in}}%
\pgfpathlineto{\pgfqpoint{2.265974in}{3.134652in}}%
\pgfpathlineto{\pgfqpoint{2.269453in}{3.135867in}}%
\pgfpathlineto{\pgfqpoint{2.272397in}{3.134636in}}%
\pgfpathlineto{\pgfqpoint{2.276411in}{3.130104in}}%
\pgfpathlineto{\pgfqpoint{2.288185in}{3.114920in}}%
\pgfpathlineto{\pgfqpoint{2.291397in}{3.114796in}}%
\pgfpathlineto{\pgfqpoint{2.294608in}{3.117078in}}%
\pgfpathlineto{\pgfqpoint{2.299425in}{3.123737in}}%
\pgfpathlineto{\pgfqpoint{2.307453in}{3.134594in}}%
\pgfpathlineto{\pgfqpoint{2.310932in}{3.135872in}}%
\pgfpathlineto{\pgfqpoint{2.313876in}{3.134692in}}%
\pgfpathlineto{\pgfqpoint{2.317890in}{3.130212in}}%
\pgfpathlineto{\pgfqpoint{2.329665in}{3.114953in}}%
\pgfpathlineto{\pgfqpoint{2.332876in}{3.114770in}}%
\pgfpathlineto{\pgfqpoint{2.336087in}{3.117000in}}%
\pgfpathlineto{\pgfqpoint{2.340904in}{3.123617in}}%
\pgfpathlineto{\pgfqpoint{2.348932in}{3.134536in}}%
\pgfpathlineto{\pgfqpoint{2.352411in}{3.135875in}}%
\pgfpathlineto{\pgfqpoint{2.355355in}{3.134747in}}%
\pgfpathlineto{\pgfqpoint{2.359102in}{3.130694in}}%
\pgfpathlineto{\pgfqpoint{2.371679in}{3.114774in}}%
\pgfpathlineto{\pgfqpoint{2.374890in}{3.114947in}}%
\pgfpathlineto{\pgfqpoint{2.378369in}{3.117797in}}%
\pgfpathlineto{\pgfqpoint{2.383989in}{3.126078in}}%
\pgfpathlineto{\pgfqpoint{2.390412in}{3.134476in}}%
\pgfpathlineto{\pgfqpoint{2.393891in}{3.135876in}}%
\pgfpathlineto{\pgfqpoint{2.396834in}{3.134801in}}%
\pgfpathlineto{\pgfqpoint{2.400581in}{3.130798in}}%
\pgfpathlineto{\pgfqpoint{2.413426in}{3.114714in}}%
\pgfpathlineto{\pgfqpoint{2.416637in}{3.115036in}}%
\pgfpathlineto{\pgfqpoint{2.420116in}{3.118023in}}%
\pgfpathlineto{\pgfqpoint{2.426271in}{3.127240in}}%
\pgfpathlineto{\pgfqpoint{2.432158in}{3.134626in}}%
\pgfpathlineto{\pgfqpoint{2.435637in}{3.135870in}}%
\pgfpathlineto{\pgfqpoint{2.438581in}{3.134661in}}%
\pgfpathlineto{\pgfqpoint{2.442595in}{3.130153in}}%
\pgfpathlineto{\pgfqpoint{2.454370in}{3.114935in}}%
\pgfpathlineto{\pgfqpoint{2.457581in}{3.114784in}}%
\pgfpathlineto{\pgfqpoint{2.460793in}{3.117043in}}%
\pgfpathlineto{\pgfqpoint{2.465609in}{3.123683in}}%
\pgfpathlineto{\pgfqpoint{2.473638in}{3.134568in}}%
\pgfpathlineto{\pgfqpoint{2.477117in}{3.135873in}}%
\pgfpathlineto{\pgfqpoint{2.480060in}{3.134717in}}%
\pgfpathlineto{\pgfqpoint{2.483807in}{3.130636in}}%
\pgfpathlineto{\pgfqpoint{2.496384in}{3.114761in}}%
\pgfpathlineto{\pgfqpoint{2.499596in}{3.114966in}}%
\pgfpathlineto{\pgfqpoint{2.503075in}{3.117845in}}%
\pgfpathlineto{\pgfqpoint{2.508694in}{3.126144in}}%
\pgfpathlineto{\pgfqpoint{2.515117in}{3.134509in}}%
\pgfpathlineto{\pgfqpoint{2.518596in}{3.135876in}}%
\pgfpathlineto{\pgfqpoint{2.521540in}{3.134772in}}%
\pgfpathlineto{\pgfqpoint{2.525286in}{3.130741in}}%
\pgfpathlineto{\pgfqpoint{2.538131in}{3.114703in}}%
\pgfpathlineto{\pgfqpoint{2.541075in}{3.114933in}}%
\pgfpathlineto{\pgfqpoint{2.544554in}{3.117758in}}%
\pgfpathlineto{\pgfqpoint{2.550174in}{3.126023in}}%
\pgfpathlineto{\pgfqpoint{2.556596in}{3.134449in}}%
\pgfpathlineto{\pgfqpoint{2.560075in}{3.135877in}}%
\pgfpathlineto{\pgfqpoint{2.563019in}{3.134825in}}%
\pgfpathlineto{\pgfqpoint{2.566765in}{3.130844in}}%
\pgfpathlineto{\pgfqpoint{2.579610in}{3.114724in}}%
\pgfpathlineto{\pgfqpoint{2.582822in}{3.115020in}}%
\pgfpathlineto{\pgfqpoint{2.586301in}{3.117983in}}%
\pgfpathlineto{\pgfqpoint{2.592188in}{3.126760in}}%
\pgfpathlineto{\pgfqpoint{2.598343in}{3.134600in}}%
\pgfpathlineto{\pgfqpoint{2.601822in}{3.135871in}}%
\pgfpathlineto{\pgfqpoint{2.604766in}{3.134687in}}%
\pgfpathlineto{\pgfqpoint{2.608780in}{3.130201in}}%
\pgfpathlineto{\pgfqpoint{2.620554in}{3.114950in}}%
\pgfpathlineto{\pgfqpoint{2.623766in}{3.114772in}}%
\pgfpathlineto{\pgfqpoint{2.626977in}{3.117008in}}%
\pgfpathlineto{\pgfqpoint{2.631794in}{3.123629in}}%
\pgfpathlineto{\pgfqpoint{2.639822in}{3.134542in}}%
\pgfpathlineto{\pgfqpoint{2.643301in}{3.135875in}}%
\pgfpathlineto{\pgfqpoint{2.646245in}{3.134742in}}%
\pgfpathlineto{\pgfqpoint{2.649991in}{3.130683in}}%
\pgfpathlineto{\pgfqpoint{2.662569in}{3.114772in}}%
\pgfpathlineto{\pgfqpoint{2.665780in}{3.114951in}}%
\pgfpathlineto{\pgfqpoint{2.669259in}{3.117806in}}%
\pgfpathlineto{\pgfqpoint{2.674879in}{3.126090in}}%
\pgfpathlineto{\pgfqpoint{2.681301in}{3.134482in}}%
\pgfpathlineto{\pgfqpoint{2.684780in}{3.135876in}}%
\pgfpathlineto{\pgfqpoint{2.687724in}{3.134796in}}%
\pgfpathlineto{\pgfqpoint{2.691470in}{3.130787in}}%
\pgfpathlineto{\pgfqpoint{2.704316in}{3.114712in}}%
\pgfpathlineto{\pgfqpoint{2.707527in}{3.115040in}}%
\pgfpathlineto{\pgfqpoint{2.711006in}{3.118032in}}%
\pgfpathlineto{\pgfqpoint{2.717161in}{3.127252in}}%
\pgfpathlineto{\pgfqpoint{2.723048in}{3.134632in}}%
\pgfpathlineto{\pgfqpoint{2.726527in}{3.135869in}}%
\pgfpathlineto{\pgfqpoint{2.729471in}{3.134656in}}%
\pgfpathlineto{\pgfqpoint{2.733485in}{3.130142in}}%
\pgfpathlineto{\pgfqpoint{2.745260in}{3.114932in}}%
\pgfpathlineto{\pgfqpoint{2.748471in}{3.114786in}}%
\pgfpathlineto{\pgfqpoint{2.751682in}{3.117050in}}%
\pgfpathlineto{\pgfqpoint{2.756499in}{3.123695in}}%
\pgfpathlineto{\pgfqpoint{2.764527in}{3.134574in}}%
\pgfpathlineto{\pgfqpoint{2.768006in}{3.135873in}}%
\pgfpathlineto{\pgfqpoint{2.770950in}{3.134712in}}%
\pgfpathlineto{\pgfqpoint{2.774697in}{3.130626in}}%
\pgfpathlineto{\pgfqpoint{2.787274in}{3.114758in}}%
\pgfpathlineto{\pgfqpoint{2.790485in}{3.114969in}}%
\pgfpathlineto{\pgfqpoint{2.793964in}{3.117854in}}%
\pgfpathlineto{\pgfqpoint{2.799852in}{3.126586in}}%
\pgfpathlineto{\pgfqpoint{2.806007in}{3.134515in}}%
\pgfpathlineto{\pgfqpoint{2.809486in}{3.135876in}}%
\pgfpathlineto{\pgfqpoint{2.812429in}{3.134766in}}%
\pgfpathlineto{\pgfqpoint{2.816176in}{3.130730in}}%
\pgfpathlineto{\pgfqpoint{2.828753in}{3.114783in}}%
\pgfpathlineto{\pgfqpoint{2.831965in}{3.114936in}}%
\pgfpathlineto{\pgfqpoint{2.835443in}{3.117767in}}%
\pgfpathlineto{\pgfqpoint{2.841063in}{3.126035in}}%
\pgfpathlineto{\pgfqpoint{2.847486in}{3.134455in}}%
\pgfpathlineto{\pgfqpoint{2.850965in}{3.135877in}}%
\pgfpathlineto{\pgfqpoint{2.853908in}{3.134820in}}%
\pgfpathlineto{\pgfqpoint{2.857655in}{3.130834in}}%
\pgfpathlineto{\pgfqpoint{2.870500in}{3.114722in}}%
\pgfpathlineto{\pgfqpoint{2.873711in}{3.115024in}}%
\pgfpathlineto{\pgfqpoint{2.877190in}{3.117992in}}%
\pgfpathlineto{\pgfqpoint{2.883078in}{3.126772in}}%
\pgfpathlineto{\pgfqpoint{2.889233in}{3.134606in}}%
\pgfpathlineto{\pgfqpoint{2.892712in}{3.135871in}}%
\pgfpathlineto{\pgfqpoint{2.895655in}{3.134681in}}%
\pgfpathlineto{\pgfqpoint{2.899669in}{3.130191in}}%
\pgfpathlineto{\pgfqpoint{2.911444in}{3.114947in}}%
\pgfpathlineto{\pgfqpoint{2.914655in}{3.114775in}}%
\pgfpathlineto{\pgfqpoint{2.917867in}{3.117015in}}%
\pgfpathlineto{\pgfqpoint{2.922684in}{3.123641in}}%
\pgfpathlineto{\pgfqpoint{2.930712in}{3.134548in}}%
\pgfpathlineto{\pgfqpoint{2.934191in}{3.135874in}}%
\pgfpathlineto{\pgfqpoint{2.937134in}{3.134736in}}%
\pgfpathlineto{\pgfqpoint{2.940881in}{3.130673in}}%
\pgfpathlineto{\pgfqpoint{2.953459in}{3.114769in}}%
\pgfpathlineto{\pgfqpoint{2.956670in}{3.114954in}}%
\pgfpathlineto{\pgfqpoint{2.960149in}{3.117815in}}%
\pgfpathlineto{\pgfqpoint{2.965768in}{3.126102in}}%
\pgfpathlineto{\pgfqpoint{2.972191in}{3.134488in}}%
\pgfpathlineto{\pgfqpoint{2.975670in}{3.135876in}}%
\pgfpathlineto{\pgfqpoint{2.978614in}{3.134791in}}%
\pgfpathlineto{\pgfqpoint{2.982360in}{3.130777in}}%
\pgfpathlineto{\pgfqpoint{2.995205in}{3.114710in}}%
\pgfpathlineto{\pgfqpoint{2.998417in}{3.115044in}}%
\pgfpathlineto{\pgfqpoint{3.001896in}{3.118041in}}%
\pgfpathlineto{\pgfqpoint{3.008051in}{3.127264in}}%
\pgfpathlineto{\pgfqpoint{3.013938in}{3.134637in}}%
\pgfpathlineto{\pgfqpoint{3.017417in}{3.135869in}}%
\pgfpathlineto{\pgfqpoint{3.020360in}{3.134650in}}%
\pgfpathlineto{\pgfqpoint{3.024375in}{3.130131in}}%
\pgfpathlineto{\pgfqpoint{3.036149in}{3.114929in}}%
\pgfpathlineto{\pgfqpoint{3.039361in}{3.114789in}}%
\pgfpathlineto{\pgfqpoint{3.042572in}{3.117058in}}%
\pgfpathlineto{\pgfqpoint{3.047389in}{3.123707in}}%
\pgfpathlineto{\pgfqpoint{3.055417in}{3.134580in}}%
\pgfpathlineto{\pgfqpoint{3.058896in}{3.135873in}}%
\pgfpathlineto{\pgfqpoint{3.061840in}{3.134706in}}%
\pgfpathlineto{\pgfqpoint{3.065586in}{3.130615in}}%
\pgfpathlineto{\pgfqpoint{3.078164in}{3.114756in}}%
\pgfpathlineto{\pgfqpoint{3.081375in}{3.114973in}}%
\pgfpathlineto{\pgfqpoint{3.084854in}{3.117863in}}%
\pgfpathlineto{\pgfqpoint{3.090741in}{3.126598in}}%
\pgfpathlineto{\pgfqpoint{3.096896in}{3.134521in}}%
\pgfpathlineto{\pgfqpoint{3.100375in}{3.135875in}}%
\pgfpathlineto{\pgfqpoint{3.103319in}{3.134761in}}%
\pgfpathlineto{\pgfqpoint{3.107065in}{3.130720in}}%
\pgfpathlineto{\pgfqpoint{3.119643in}{3.114781in}}%
\pgfpathlineto{\pgfqpoint{3.122854in}{3.114939in}}%
\pgfpathlineto{\pgfqpoint{3.126333in}{3.117775in}}%
\pgfpathlineto{\pgfqpoint{3.131953in}{3.126048in}}%
\pgfpathlineto{\pgfqpoint{3.138376in}{3.134461in}}%
\pgfpathlineto{\pgfqpoint{3.141854in}{3.135877in}}%
\pgfpathlineto{\pgfqpoint{3.144798in}{3.134814in}}%
\pgfpathlineto{\pgfqpoint{3.148545in}{3.130824in}}%
\pgfpathlineto{\pgfqpoint{3.161390in}{3.114720in}}%
\pgfpathlineto{\pgfqpoint{3.164601in}{3.115027in}}%
\pgfpathlineto{\pgfqpoint{3.168080in}{3.118001in}}%
\pgfpathlineto{\pgfqpoint{3.173967in}{3.126784in}}%
\pgfpathlineto{\pgfqpoint{3.180122in}{3.134612in}}%
\pgfpathlineto{\pgfqpoint{3.183601in}{3.135871in}}%
\pgfpathlineto{\pgfqpoint{3.186545in}{3.134676in}}%
\pgfpathlineto{\pgfqpoint{3.190559in}{3.130180in}}%
\pgfpathlineto{\pgfqpoint{3.202334in}{3.114943in}}%
\pgfpathlineto{\pgfqpoint{3.205545in}{3.114777in}}%
\pgfpathlineto{\pgfqpoint{3.208756in}{3.117023in}}%
\pgfpathlineto{\pgfqpoint{3.213573in}{3.123653in}}%
\pgfpathlineto{\pgfqpoint{3.221602in}{3.134554in}}%
\pgfpathlineto{\pgfqpoint{3.225080in}{3.135874in}}%
\pgfpathlineto{\pgfqpoint{3.228024in}{3.134731in}}%
\pgfpathlineto{\pgfqpoint{3.231771in}{3.130662in}}%
\pgfpathlineto{\pgfqpoint{3.244348in}{3.114767in}}%
\pgfpathlineto{\pgfqpoint{3.247560in}{3.114957in}}%
\pgfpathlineto{\pgfqpoint{3.251038in}{3.117823in}}%
\pgfpathlineto{\pgfqpoint{3.256658in}{3.126114in}}%
\pgfpathlineto{\pgfqpoint{3.263081in}{3.134494in}}%
\pgfpathlineto{\pgfqpoint{3.266560in}{3.135876in}}%
\pgfpathlineto{\pgfqpoint{3.269503in}{3.134785in}}%
\pgfpathlineto{\pgfqpoint{3.273250in}{3.130767in}}%
\pgfpathlineto{\pgfqpoint{3.286095in}{3.114708in}}%
\pgfpathlineto{\pgfqpoint{3.289039in}{3.114924in}}%
\pgfpathlineto{\pgfqpoint{3.292518in}{3.117736in}}%
\pgfpathlineto{\pgfqpoint{3.298137in}{3.125993in}}%
\pgfpathlineto{\pgfqpoint{3.304828in}{3.134643in}}%
\pgfpathlineto{\pgfqpoint{3.308307in}{3.135868in}}%
\pgfpathlineto{\pgfqpoint{3.311250in}{3.134645in}}%
\pgfpathlineto{\pgfqpoint{3.315264in}{3.130121in}}%
\pgfpathlineto{\pgfqpoint{3.327039in}{3.114925in}}%
\pgfpathlineto{\pgfqpoint{3.330250in}{3.114792in}}%
\pgfpathlineto{\pgfqpoint{3.333462in}{3.117066in}}%
\pgfpathlineto{\pgfqpoint{3.338279in}{3.123719in}}%
\pgfpathlineto{\pgfqpoint{3.346307in}{3.134586in}}%
\pgfpathlineto{\pgfqpoint{3.349786in}{3.135872in}}%
\pgfpathlineto{\pgfqpoint{3.352729in}{3.134701in}}%
\pgfpathlineto{\pgfqpoint{3.356476in}{3.130605in}}%
\pgfpathlineto{\pgfqpoint{3.369053in}{3.114753in}}%
\pgfpathlineto{\pgfqpoint{3.372265in}{3.114976in}}%
\pgfpathlineto{\pgfqpoint{3.375744in}{3.117872in}}%
\pgfpathlineto{\pgfqpoint{3.381631in}{3.126610in}}%
\pgfpathlineto{\pgfqpoint{3.387786in}{3.134527in}}%
\pgfpathlineto{\pgfqpoint{3.391265in}{3.135875in}}%
\pgfpathlineto{\pgfqpoint{3.394209in}{3.134756in}}%
\pgfpathlineto{\pgfqpoint{3.397955in}{3.130709in}}%
\pgfpathlineto{\pgfqpoint{3.410533in}{3.114778in}}%
\pgfpathlineto{\pgfqpoint{3.413744in}{3.114942in}}%
\pgfpathlineto{\pgfqpoint{3.417223in}{3.117784in}}%
\pgfpathlineto{\pgfqpoint{3.422843in}{3.126060in}}%
\pgfpathlineto{\pgfqpoint{3.429265in}{3.134467in}}%
\pgfpathlineto{\pgfqpoint{3.432744in}{3.135877in}}%
\pgfpathlineto{\pgfqpoint{3.435688in}{3.134809in}}%
\pgfpathlineto{\pgfqpoint{3.439434in}{3.130813in}}%
\pgfpathlineto{\pgfqpoint{3.452280in}{3.114718in}}%
\pgfpathlineto{\pgfqpoint{3.455491in}{3.115031in}}%
\pgfpathlineto{\pgfqpoint{3.458970in}{3.118010in}}%
\pgfpathlineto{\pgfqpoint{3.464857in}{3.126796in}}%
\pgfpathlineto{\pgfqpoint{3.471012in}{3.134617in}}%
\pgfpathlineto{\pgfqpoint{3.474491in}{3.135870in}}%
\pgfpathlineto{\pgfqpoint{3.477435in}{3.134670in}}%
\pgfpathlineto{\pgfqpoint{3.481449in}{3.130169in}}%
\pgfpathlineto{\pgfqpoint{3.493224in}{3.114940in}}%
\pgfpathlineto{\pgfqpoint{3.496435in}{3.114780in}}%
\pgfpathlineto{\pgfqpoint{3.499646in}{3.117031in}}%
\pgfpathlineto{\pgfqpoint{3.504463in}{3.123665in}}%
\pgfpathlineto{\pgfqpoint{3.512491in}{3.134560in}}%
\pgfpathlineto{\pgfqpoint{3.515970in}{3.135874in}}%
\pgfpathlineto{\pgfqpoint{3.518914in}{3.134725in}}%
\pgfpathlineto{\pgfqpoint{3.522660in}{3.130652in}}%
\pgfpathlineto{\pgfqpoint{3.535238in}{3.114764in}}%
\pgfpathlineto{\pgfqpoint{3.538449in}{3.114961in}}%
\pgfpathlineto{\pgfqpoint{3.541928in}{3.117832in}}%
\pgfpathlineto{\pgfqpoint{3.547548in}{3.126126in}}%
\pgfpathlineto{\pgfqpoint{3.553970in}{3.134500in}}%
\pgfpathlineto{\pgfqpoint{3.557449in}{3.135876in}}%
\pgfpathlineto{\pgfqpoint{3.560393in}{3.134780in}}%
\pgfpathlineto{\pgfqpoint{3.564140in}{3.130756in}}%
\pgfpathlineto{\pgfqpoint{3.576985in}{3.114706in}}%
\pgfpathlineto{\pgfqpoint{3.579928in}{3.114928in}}%
\pgfpathlineto{\pgfqpoint{3.583407in}{3.117745in}}%
\pgfpathlineto{\pgfqpoint{3.589027in}{3.126005in}}%
\pgfpathlineto{\pgfqpoint{3.595717in}{3.134649in}}%
\pgfpathlineto{\pgfqpoint{3.599196in}{3.135868in}}%
\pgfpathlineto{\pgfqpoint{3.602140in}{3.134639in}}%
\pgfpathlineto{\pgfqpoint{3.606154in}{3.130110in}}%
\pgfpathlineto{\pgfqpoint{3.617929in}{3.114922in}}%
\pgfpathlineto{\pgfqpoint{3.621140in}{3.114794in}}%
\pgfpathlineto{\pgfqpoint{3.624351in}{3.117074in}}%
\pgfpathlineto{\pgfqpoint{3.629168in}{3.123731in}}%
\pgfpathlineto{\pgfqpoint{3.637197in}{3.134592in}}%
\pgfpathlineto{\pgfqpoint{3.640675in}{3.135872in}}%
\pgfpathlineto{\pgfqpoint{3.643619in}{3.134695in}}%
\pgfpathlineto{\pgfqpoint{3.647633in}{3.130217in}}%
\pgfpathlineto{\pgfqpoint{3.659408in}{3.114955in}}%
\pgfpathlineto{\pgfqpoint{3.662619in}{3.114769in}}%
\pgfpathlineto{\pgfqpoint{3.665831in}{3.116996in}}%
\pgfpathlineto{\pgfqpoint{3.670647in}{3.123611in}}%
\pgfpathlineto{\pgfqpoint{3.678676in}{3.134533in}}%
\pgfpathlineto{\pgfqpoint{3.682155in}{3.135875in}}%
\pgfpathlineto{\pgfqpoint{3.685098in}{3.134750in}}%
\pgfpathlineto{\pgfqpoint{3.688845in}{3.130699in}}%
\pgfpathlineto{\pgfqpoint{3.701422in}{3.114775in}}%
\pgfpathlineto{\pgfqpoint{3.704634in}{3.114946in}}%
\pgfpathlineto{\pgfqpoint{3.708113in}{3.117793in}}%
\pgfpathlineto{\pgfqpoint{3.713732in}{3.126072in}}%
\pgfpathlineto{\pgfqpoint{3.720155in}{3.134473in}}%
\pgfpathlineto{\pgfqpoint{3.723634in}{3.135877in}}%
\pgfpathlineto{\pgfqpoint{3.726578in}{3.134804in}}%
\pgfpathlineto{\pgfqpoint{3.730324in}{3.130803in}}%
\pgfpathlineto{\pgfqpoint{3.743169in}{3.114715in}}%
\pgfpathlineto{\pgfqpoint{3.746381in}{3.115035in}}%
\pgfpathlineto{\pgfqpoint{3.749859in}{3.118019in}}%
\pgfpathlineto{\pgfqpoint{3.755747in}{3.126808in}}%
\pgfpathlineto{\pgfqpoint{3.761902in}{3.134623in}}%
\pgfpathlineto{\pgfqpoint{3.765381in}{3.135870in}}%
\pgfpathlineto{\pgfqpoint{3.768324in}{3.134664in}}%
\pgfpathlineto{\pgfqpoint{3.772338in}{3.130158in}}%
\pgfpathlineto{\pgfqpoint{3.784113in}{3.114937in}}%
\pgfpathlineto{\pgfqpoint{3.787325in}{3.114783in}}%
\pgfpathlineto{\pgfqpoint{3.790536in}{3.117039in}}%
\pgfpathlineto{\pgfqpoint{3.795353in}{3.123677in}}%
\pgfpathlineto{\pgfqpoint{3.803381in}{3.134565in}}%
\pgfpathlineto{\pgfqpoint{3.806860in}{3.135873in}}%
\pgfpathlineto{\pgfqpoint{3.809804in}{3.134720in}}%
\pgfpathlineto{\pgfqpoint{3.813550in}{3.130642in}}%
\pgfpathlineto{\pgfqpoint{3.826128in}{3.114762in}}%
\pgfpathlineto{\pgfqpoint{3.829339in}{3.114964in}}%
\pgfpathlineto{\pgfqpoint{3.832818in}{3.117841in}}%
\pgfpathlineto{\pgfqpoint{3.838438in}{3.126138in}}%
\pgfpathlineto{\pgfqpoint{3.844860in}{3.134506in}}%
\pgfpathlineto{\pgfqpoint{3.848339in}{3.135876in}}%
\pgfpathlineto{\pgfqpoint{3.851283in}{3.134774in}}%
\pgfpathlineto{\pgfqpoint{3.855029in}{3.130746in}}%
\pgfpathlineto{\pgfqpoint{3.867874in}{3.114704in}}%
\pgfpathlineto{\pgfqpoint{3.870818in}{3.114931in}}%
\pgfpathlineto{\pgfqpoint{3.874297in}{3.117753in}}%
\pgfpathlineto{\pgfqpoint{3.879917in}{3.126017in}}%
\pgfpathlineto{\pgfqpoint{3.886339in}{3.134446in}}%
\pgfpathlineto{\pgfqpoint{3.889818in}{3.135877in}}%
\pgfpathlineto{\pgfqpoint{3.892762in}{3.134828in}}%
\pgfpathlineto{\pgfqpoint{3.896508in}{3.130849in}}%
\pgfpathlineto{\pgfqpoint{3.909354in}{3.114725in}}%
\pgfpathlineto{\pgfqpoint{3.912565in}{3.115018in}}%
\pgfpathlineto{\pgfqpoint{3.916044in}{3.117978in}}%
\pgfpathlineto{\pgfqpoint{3.921931in}{3.126754in}}%
\pgfpathlineto{\pgfqpoint{3.928086in}{3.134597in}}%
\pgfpathlineto{\pgfqpoint{3.931565in}{3.135872in}}%
\pgfpathlineto{\pgfqpoint{3.934509in}{3.134690in}}%
\pgfpathlineto{\pgfqpoint{3.938523in}{3.130207in}}%
\pgfpathlineto{\pgfqpoint{3.950298in}{3.114952in}}%
\pgfpathlineto{\pgfqpoint{3.953509in}{3.114771in}}%
\pgfpathlineto{\pgfqpoint{3.956720in}{3.117004in}}%
\pgfpathlineto{\pgfqpoint{3.961537in}{3.123623in}}%
\pgfpathlineto{\pgfqpoint{3.969565in}{3.134539in}}%
\pgfpathlineto{\pgfqpoint{3.973044in}{3.135875in}}%
\pgfpathlineto{\pgfqpoint{3.975988in}{3.134745in}}%
\pgfpathlineto{\pgfqpoint{3.979735in}{3.130689in}}%
\pgfpathlineto{\pgfqpoint{3.992312in}{3.114773in}}%
\pgfpathlineto{\pgfqpoint{3.995523in}{3.114949in}}%
\pgfpathlineto{\pgfqpoint{3.999002in}{3.117801in}}%
\pgfpathlineto{\pgfqpoint{4.004622in}{3.126084in}}%
\pgfpathlineto{\pgfqpoint{4.011045in}{3.134479in}}%
\pgfpathlineto{\pgfqpoint{4.014524in}{3.135876in}}%
\pgfpathlineto{\pgfqpoint{4.017467in}{3.134799in}}%
\pgfpathlineto{\pgfqpoint{4.021214in}{3.130793in}}%
\pgfpathlineto{\pgfqpoint{4.034059in}{3.114713in}}%
\pgfpathlineto{\pgfqpoint{4.037270in}{3.115038in}}%
\pgfpathlineto{\pgfqpoint{4.040749in}{3.118028in}}%
\pgfpathlineto{\pgfqpoint{4.046904in}{3.127246in}}%
\pgfpathlineto{\pgfqpoint{4.052791in}{3.134629in}}%
\pgfpathlineto{\pgfqpoint{4.056270in}{3.135869in}}%
\pgfpathlineto{\pgfqpoint{4.059214in}{3.134659in}}%
\pgfpathlineto{\pgfqpoint{4.063228in}{3.130148in}}%
\pgfpathlineto{\pgfqpoint{4.075003in}{3.114933in}}%
\pgfpathlineto{\pgfqpoint{4.078214in}{3.114785in}}%
\pgfpathlineto{\pgfqpoint{4.081425in}{3.117047in}}%
\pgfpathlineto{\pgfqpoint{4.086242in}{3.123689in}}%
\pgfpathlineto{\pgfqpoint{4.094271in}{3.134571in}}%
\pgfpathlineto{\pgfqpoint{4.097750in}{3.135873in}}%
\pgfpathlineto{\pgfqpoint{4.100693in}{3.134714in}}%
\pgfpathlineto{\pgfqpoint{4.104440in}{3.130631in}}%
\pgfpathlineto{\pgfqpoint{4.117017in}{3.114759in}}%
\pgfpathlineto{\pgfqpoint{4.120229in}{3.114968in}}%
\pgfpathlineto{\pgfqpoint{4.123708in}{3.117850in}}%
\pgfpathlineto{\pgfqpoint{4.129595in}{3.126580in}}%
\pgfpathlineto{\pgfqpoint{4.135750in}{3.134512in}}%
\pgfpathlineto{\pgfqpoint{4.139229in}{3.135876in}}%
\pgfpathlineto{\pgfqpoint{4.142172in}{3.134769in}}%
\pgfpathlineto{\pgfqpoint{4.145919in}{3.130735in}}%
\pgfpathlineto{\pgfqpoint{4.158764in}{3.114702in}}%
\pgfpathlineto{\pgfqpoint{4.161708in}{3.114934in}}%
\pgfpathlineto{\pgfqpoint{4.165187in}{3.117762in}}%
\pgfpathlineto{\pgfqpoint{4.170807in}{3.126029in}}%
\pgfpathlineto{\pgfqpoint{4.177229in}{3.134452in}}%
\pgfpathlineto{\pgfqpoint{4.180708in}{3.135877in}}%
\pgfpathlineto{\pgfqpoint{4.183652in}{3.134822in}}%
\pgfpathlineto{\pgfqpoint{4.187398in}{3.130839in}}%
\pgfpathlineto{\pgfqpoint{4.200243in}{3.114723in}}%
\pgfpathlineto{\pgfqpoint{4.203455in}{3.115022in}}%
\pgfpathlineto{\pgfqpoint{4.206934in}{3.117987in}}%
\pgfpathlineto{\pgfqpoint{4.212821in}{3.126766in}}%
\pgfpathlineto{\pgfqpoint{4.218976in}{3.134603in}}%
\pgfpathlineto{\pgfqpoint{4.222455in}{3.135871in}}%
\pgfpathlineto{\pgfqpoint{4.225399in}{3.134684in}}%
\pgfpathlineto{\pgfqpoint{4.229413in}{3.130196in}}%
\pgfpathlineto{\pgfqpoint{4.241187in}{3.114948in}}%
\pgfpathlineto{\pgfqpoint{4.244399in}{3.114774in}}%
\pgfpathlineto{\pgfqpoint{4.247610in}{3.117011in}}%
\pgfpathlineto{\pgfqpoint{4.252427in}{3.123635in}}%
\pgfpathlineto{\pgfqpoint{4.260455in}{3.134545in}}%
\pgfpathlineto{\pgfqpoint{4.263934in}{3.135874in}}%
\pgfpathlineto{\pgfqpoint{4.266878in}{3.134739in}}%
\pgfpathlineto{\pgfqpoint{4.270624in}{3.130678in}}%
\pgfpathlineto{\pgfqpoint{4.283202in}{3.114770in}}%
\pgfpathlineto{\pgfqpoint{4.286413in}{3.114952in}}%
\pgfpathlineto{\pgfqpoint{4.289892in}{3.117810in}}%
\pgfpathlineto{\pgfqpoint{4.295512in}{3.126096in}}%
\pgfpathlineto{\pgfqpoint{4.301934in}{3.134485in}}%
\pgfpathlineto{\pgfqpoint{4.305413in}{3.135876in}}%
\pgfpathlineto{\pgfqpoint{4.308357in}{3.134793in}}%
\pgfpathlineto{\pgfqpoint{4.312103in}{3.130782in}}%
\pgfpathlineto{\pgfqpoint{4.324949in}{3.114711in}}%
\pgfpathlineto{\pgfqpoint{4.328160in}{3.115042in}}%
\pgfpathlineto{\pgfqpoint{4.331639in}{3.118037in}}%
\pgfpathlineto{\pgfqpoint{4.337794in}{3.127258in}}%
\pgfpathlineto{\pgfqpoint{4.343681in}{3.134635in}}%
\pgfpathlineto{\pgfqpoint{4.347160in}{3.135869in}}%
\pgfpathlineto{\pgfqpoint{4.350104in}{3.134653in}}%
\pgfpathlineto{\pgfqpoint{4.354118in}{3.130137in}}%
\pgfpathlineto{\pgfqpoint{4.365893in}{3.114930in}}%
\pgfpathlineto{\pgfqpoint{4.369104in}{3.114788in}}%
\pgfpathlineto{\pgfqpoint{4.372315in}{3.117054in}}%
\pgfpathlineto{\pgfqpoint{4.377132in}{3.123701in}}%
\pgfpathlineto{\pgfqpoint{4.385160in}{3.134577in}}%
\pgfpathlineto{\pgfqpoint{4.388639in}{3.135873in}}%
\pgfpathlineto{\pgfqpoint{4.391583in}{3.134709in}}%
\pgfpathlineto{\pgfqpoint{4.395329in}{3.130621in}}%
\pgfpathlineto{\pgfqpoint{4.407907in}{3.114757in}}%
\pgfpathlineto{\pgfqpoint{4.411118in}{3.114971in}}%
\pgfpathlineto{\pgfqpoint{4.414597in}{3.117859in}}%
\pgfpathlineto{\pgfqpoint{4.420485in}{3.126592in}}%
\pgfpathlineto{\pgfqpoint{4.426640in}{3.134518in}}%
\pgfpathlineto{\pgfqpoint{4.430118in}{3.135875in}}%
\pgfpathlineto{\pgfqpoint{4.433062in}{3.134764in}}%
\pgfpathlineto{\pgfqpoint{4.436809in}{3.130725in}}%
\pgfpathlineto{\pgfqpoint{4.449386in}{3.114782in}}%
\pgfpathlineto{\pgfqpoint{4.452598in}{3.114937in}}%
\pgfpathlineto{\pgfqpoint{4.456076in}{3.117771in}}%
\pgfpathlineto{\pgfqpoint{4.461696in}{3.126041in}}%
\pgfpathlineto{\pgfqpoint{4.468119in}{3.134458in}}%
\pgfpathlineto{\pgfqpoint{4.471598in}{3.135877in}}%
\pgfpathlineto{\pgfqpoint{4.474541in}{3.134817in}}%
\pgfpathlineto{\pgfqpoint{4.478288in}{3.130829in}}%
\pgfpathlineto{\pgfqpoint{4.491133in}{3.114721in}}%
\pgfpathlineto{\pgfqpoint{4.494344in}{3.115025in}}%
\pgfpathlineto{\pgfqpoint{4.497823in}{3.117996in}}%
\pgfpathlineto{\pgfqpoint{4.503711in}{3.126778in}}%
\pgfpathlineto{\pgfqpoint{4.509866in}{3.134609in}}%
\pgfpathlineto{\pgfqpoint{4.513345in}{3.135871in}}%
\pgfpathlineto{\pgfqpoint{4.516288in}{3.134678in}}%
\pgfpathlineto{\pgfqpoint{4.520302in}{3.130185in}}%
\pgfpathlineto{\pgfqpoint{4.532077in}{3.114945in}}%
\pgfpathlineto{\pgfqpoint{4.535288in}{3.114776in}}%
\pgfpathlineto{\pgfqpoint{4.538500in}{3.117019in}}%
\pgfpathlineto{\pgfqpoint{4.543317in}{3.123647in}}%
\pgfpathlineto{\pgfqpoint{4.551345in}{3.134551in}}%
\pgfpathlineto{\pgfqpoint{4.554824in}{3.135874in}}%
\pgfpathlineto{\pgfqpoint{4.557767in}{3.134734in}}%
\pgfpathlineto{\pgfqpoint{4.561514in}{3.130668in}}%
\pgfpathlineto{\pgfqpoint{4.574091in}{3.114768in}}%
\pgfpathlineto{\pgfqpoint{4.577303in}{3.114956in}}%
\pgfpathlineto{\pgfqpoint{4.580782in}{3.117819in}}%
\pgfpathlineto{\pgfqpoint{4.586401in}{3.126108in}}%
\pgfpathlineto{\pgfqpoint{4.592824in}{3.134491in}}%
\pgfpathlineto{\pgfqpoint{4.596303in}{3.135876in}}%
\pgfpathlineto{\pgfqpoint{4.599247in}{3.134788in}}%
\pgfpathlineto{\pgfqpoint{4.602993in}{3.130772in}}%
\pgfpathlineto{\pgfqpoint{4.615838in}{3.114709in}}%
\pgfpathlineto{\pgfqpoint{4.619050in}{3.115046in}}%
\pgfpathlineto{\pgfqpoint{4.622529in}{3.118046in}}%
\pgfpathlineto{\pgfqpoint{4.628683in}{3.127269in}}%
\pgfpathlineto{\pgfqpoint{4.634571in}{3.134640in}}%
\pgfpathlineto{\pgfqpoint{4.638050in}{3.135868in}}%
\pgfpathlineto{\pgfqpoint{4.640993in}{3.134647in}}%
\pgfpathlineto{\pgfqpoint{4.645008in}{3.130126in}}%
\pgfpathlineto{\pgfqpoint{4.656782in}{3.114927in}}%
\pgfpathlineto{\pgfqpoint{4.659994in}{3.114790in}}%
\pgfpathlineto{\pgfqpoint{4.663205in}{3.117062in}}%
\pgfpathlineto{\pgfqpoint{4.668022in}{3.123713in}}%
\pgfpathlineto{\pgfqpoint{4.676050in}{3.134583in}}%
\pgfpathlineto{\pgfqpoint{4.679529in}{3.135873in}}%
\pgfpathlineto{\pgfqpoint{4.682473in}{3.134703in}}%
\pgfpathlineto{\pgfqpoint{4.686219in}{3.130610in}}%
\pgfpathlineto{\pgfqpoint{4.698797in}{3.114754in}}%
\pgfpathlineto{\pgfqpoint{4.702008in}{3.114974in}}%
\pgfpathlineto{\pgfqpoint{4.705487in}{3.117867in}}%
\pgfpathlineto{\pgfqpoint{4.711374in}{3.126604in}}%
\pgfpathlineto{\pgfqpoint{4.717529in}{3.134524in}}%
\pgfpathlineto{\pgfqpoint{4.721008in}{3.135875in}}%
\pgfpathlineto{\pgfqpoint{4.723952in}{3.134758in}}%
\pgfpathlineto{\pgfqpoint{4.727698in}{3.130715in}}%
\pgfpathlineto{\pgfqpoint{4.740276in}{3.114779in}}%
\pgfpathlineto{\pgfqpoint{4.743487in}{3.114941in}}%
\pgfpathlineto{\pgfqpoint{4.746966in}{3.117780in}}%
\pgfpathlineto{\pgfqpoint{4.752586in}{3.126054in}}%
\pgfpathlineto{\pgfqpoint{4.759009in}{3.134464in}}%
\pgfpathlineto{\pgfqpoint{4.762487in}{3.135877in}}%
\pgfpathlineto{\pgfqpoint{4.765431in}{3.134812in}}%
\pgfpathlineto{\pgfqpoint{4.769178in}{3.130818in}}%
\pgfpathlineto{\pgfqpoint{4.782023in}{3.114719in}}%
\pgfpathlineto{\pgfqpoint{4.785234in}{3.115029in}}%
\pgfpathlineto{\pgfqpoint{4.788713in}{3.118005in}}%
\pgfpathlineto{\pgfqpoint{4.794600in}{3.126790in}}%
\pgfpathlineto{\pgfqpoint{4.800755in}{3.134615in}}%
\pgfpathlineto{\pgfqpoint{4.804234in}{3.135870in}}%
\pgfpathlineto{\pgfqpoint{4.807178in}{3.134673in}}%
\pgfpathlineto{\pgfqpoint{4.811192in}{3.130174in}}%
\pgfpathlineto{\pgfqpoint{4.822967in}{3.114942in}}%
\pgfpathlineto{\pgfqpoint{4.826178in}{3.114779in}}%
\pgfpathlineto{\pgfqpoint{4.829389in}{3.117027in}}%
\pgfpathlineto{\pgfqpoint{4.834206in}{3.123659in}}%
\pgfpathlineto{\pgfqpoint{4.842235in}{3.134557in}}%
\pgfpathlineto{\pgfqpoint{4.845713in}{3.135874in}}%
\pgfpathlineto{\pgfqpoint{4.848657in}{3.134728in}}%
\pgfpathlineto{\pgfqpoint{4.852404in}{3.130657in}}%
\pgfpathlineto{\pgfqpoint{4.864981in}{3.114765in}}%
\pgfpathlineto{\pgfqpoint{4.868192in}{3.114959in}}%
\pgfpathlineto{\pgfqpoint{4.871671in}{3.117828in}}%
\pgfpathlineto{\pgfqpoint{4.877291in}{3.126120in}}%
\pgfpathlineto{\pgfqpoint{4.883714in}{3.134497in}}%
\pgfpathlineto{\pgfqpoint{4.887193in}{3.135876in}}%
\pgfpathlineto{\pgfqpoint{4.890136in}{3.134782in}}%
\pgfpathlineto{\pgfqpoint{4.893883in}{3.130761in}}%
\pgfpathlineto{\pgfqpoint{4.906728in}{3.114707in}}%
\pgfpathlineto{\pgfqpoint{4.909672in}{3.114926in}}%
\pgfpathlineto{\pgfqpoint{4.913151in}{3.117740in}}%
\pgfpathlineto{\pgfqpoint{4.918770in}{3.125999in}}%
\pgfpathlineto{\pgfqpoint{4.925461in}{3.134646in}}%
\pgfpathlineto{\pgfqpoint{4.928939in}{3.135868in}}%
\pgfpathlineto{\pgfqpoint{4.931883in}{3.134642in}}%
\pgfpathlineto{\pgfqpoint{4.935897in}{3.130115in}}%
\pgfpathlineto{\pgfqpoint{4.947672in}{3.114924in}}%
\pgfpathlineto{\pgfqpoint{4.950883in}{3.114793in}}%
\pgfpathlineto{\pgfqpoint{4.954095in}{3.117070in}}%
\pgfpathlineto{\pgfqpoint{4.958912in}{3.123725in}}%
\pgfpathlineto{\pgfqpoint{4.966940in}{3.134589in}}%
\pgfpathlineto{\pgfqpoint{4.970419in}{3.135872in}}%
\pgfpathlineto{\pgfqpoint{4.973362in}{3.134698in}}%
\pgfpathlineto{\pgfqpoint{4.977109in}{3.130600in}}%
\pgfpathlineto{\pgfqpoint{4.989686in}{3.114752in}}%
\pgfpathlineto{\pgfqpoint{4.992898in}{3.114978in}}%
\pgfpathlineto{\pgfqpoint{4.996377in}{3.117876in}}%
\pgfpathlineto{\pgfqpoint{5.002264in}{3.126616in}}%
\pgfpathlineto{\pgfqpoint{5.008419in}{3.134530in}}%
\pgfpathlineto{\pgfqpoint{5.011898in}{3.135875in}}%
\pgfpathlineto{\pgfqpoint{5.014842in}{3.134753in}}%
\pgfpathlineto{\pgfqpoint{5.018588in}{3.130704in}}%
\pgfpathlineto{\pgfqpoint{5.031166in}{3.114777in}}%
\pgfpathlineto{\pgfqpoint{5.034377in}{3.114944in}}%
\pgfpathlineto{\pgfqpoint{5.037856in}{3.117788in}}%
\pgfpathlineto{\pgfqpoint{5.043476in}{3.126066in}}%
\pgfpathlineto{\pgfqpoint{5.049898in}{3.134470in}}%
\pgfpathlineto{\pgfqpoint{5.053377in}{3.135877in}}%
\pgfpathlineto{\pgfqpoint{5.056321in}{3.134806in}}%
\pgfpathlineto{\pgfqpoint{5.060067in}{3.130808in}}%
\pgfpathlineto{\pgfqpoint{5.072912in}{3.114717in}}%
\pgfpathlineto{\pgfqpoint{5.076124in}{3.115033in}}%
\pgfpathlineto{\pgfqpoint{5.079603in}{3.118014in}}%
\pgfpathlineto{\pgfqpoint{5.085490in}{3.126802in}}%
\pgfpathlineto{\pgfqpoint{5.091645in}{3.134620in}}%
\pgfpathlineto{\pgfqpoint{5.095124in}{3.135870in}}%
\pgfpathlineto{\pgfqpoint{5.098068in}{3.134667in}}%
\pgfpathlineto{\pgfqpoint{5.102082in}{3.130164in}}%
\pgfpathlineto{\pgfqpoint{5.113856in}{3.114938in}}%
\pgfpathlineto{\pgfqpoint{5.117068in}{3.114781in}}%
\pgfpathlineto{\pgfqpoint{5.120279in}{3.117035in}}%
\pgfpathlineto{\pgfqpoint{5.125096in}{3.123671in}}%
\pgfpathlineto{\pgfqpoint{5.133124in}{3.134562in}}%
\pgfpathlineto{\pgfqpoint{5.136603in}{3.135874in}}%
\pgfpathlineto{\pgfqpoint{5.139547in}{3.134723in}}%
\pgfpathlineto{\pgfqpoint{5.143293in}{3.130647in}}%
\pgfpathlineto{\pgfqpoint{5.155871in}{3.114763in}}%
\pgfpathlineto{\pgfqpoint{5.159082in}{3.114962in}}%
\pgfpathlineto{\pgfqpoint{5.162561in}{3.117837in}}%
\pgfpathlineto{\pgfqpoint{5.168181in}{3.126132in}}%
\pgfpathlineto{\pgfqpoint{5.174603in}{3.134503in}}%
\pgfpathlineto{\pgfqpoint{5.178082in}{3.135876in}}%
\pgfpathlineto{\pgfqpoint{5.181026in}{3.134777in}}%
\pgfpathlineto{\pgfqpoint{5.184773in}{3.130751in}}%
\pgfpathlineto{\pgfqpoint{5.197618in}{3.114705in}}%
\pgfpathlineto{\pgfqpoint{5.200561in}{3.114929in}}%
\pgfpathlineto{\pgfqpoint{5.204040in}{3.117749in}}%
\pgfpathlineto{\pgfqpoint{5.209660in}{3.126011in}}%
\pgfpathlineto{\pgfqpoint{5.216350in}{3.134652in}}%
\pgfpathlineto{\pgfqpoint{5.219829in}{3.135867in}}%
\pgfpathlineto{\pgfqpoint{5.222773in}{3.134636in}}%
\pgfpathlineto{\pgfqpoint{5.226787in}{3.130104in}}%
\pgfpathlineto{\pgfqpoint{5.238562in}{3.114920in}}%
\pgfpathlineto{\pgfqpoint{5.241773in}{3.114796in}}%
\pgfpathlineto{\pgfqpoint{5.244984in}{3.117078in}}%
\pgfpathlineto{\pgfqpoint{5.249801in}{3.123737in}}%
\pgfpathlineto{\pgfqpoint{5.257829in}{3.134594in}}%
\pgfpathlineto{\pgfqpoint{5.261308in}{3.135872in}}%
\pgfpathlineto{\pgfqpoint{5.264252in}{3.134692in}}%
\pgfpathlineto{\pgfqpoint{5.268266in}{3.130212in}}%
\pgfpathlineto{\pgfqpoint{5.280041in}{3.114953in}}%
\pgfpathlineto{\pgfqpoint{5.283252in}{3.114770in}}%
\pgfpathlineto{\pgfqpoint{5.286464in}{3.117000in}}%
\pgfpathlineto{\pgfqpoint{5.291280in}{3.123617in}}%
\pgfpathlineto{\pgfqpoint{5.299309in}{3.134536in}}%
\pgfpathlineto{\pgfqpoint{5.302788in}{3.135875in}}%
\pgfpathlineto{\pgfqpoint{5.305731in}{3.134747in}}%
\pgfpathlineto{\pgfqpoint{5.309478in}{3.130694in}}%
\pgfpathlineto{\pgfqpoint{5.322055in}{3.114774in}}%
\pgfpathlineto{\pgfqpoint{5.325267in}{3.114947in}}%
\pgfpathlineto{\pgfqpoint{5.328746in}{3.117797in}}%
\pgfpathlineto{\pgfqpoint{5.334365in}{3.126078in}}%
\pgfpathlineto{\pgfqpoint{5.340788in}{3.134476in}}%
\pgfpathlineto{\pgfqpoint{5.344267in}{3.135876in}}%
\pgfpathlineto{\pgfqpoint{5.347210in}{3.134801in}}%
\pgfpathlineto{\pgfqpoint{5.350957in}{3.130798in}}%
\pgfpathlineto{\pgfqpoint{5.363802in}{3.114714in}}%
\pgfpathlineto{\pgfqpoint{5.367013in}{3.115036in}}%
\pgfpathlineto{\pgfqpoint{5.370492in}{3.118023in}}%
\pgfpathlineto{\pgfqpoint{5.376647in}{3.127240in}}%
\pgfpathlineto{\pgfqpoint{5.382535in}{3.134626in}}%
\pgfpathlineto{\pgfqpoint{5.386014in}{3.135870in}}%
\pgfpathlineto{\pgfqpoint{5.388957in}{3.134661in}}%
\pgfpathlineto{\pgfqpoint{5.392971in}{3.130153in}}%
\pgfpathlineto{\pgfqpoint{5.404746in}{3.114935in}}%
\pgfpathlineto{\pgfqpoint{5.407957in}{3.114784in}}%
\pgfpathlineto{\pgfqpoint{5.411169in}{3.117043in}}%
\pgfpathlineto{\pgfqpoint{5.415986in}{3.123683in}}%
\pgfpathlineto{\pgfqpoint{5.424014in}{3.134568in}}%
\pgfpathlineto{\pgfqpoint{5.427493in}{3.135873in}}%
\pgfpathlineto{\pgfqpoint{5.430437in}{3.134717in}}%
\pgfpathlineto{\pgfqpoint{5.434183in}{3.130636in}}%
\pgfpathlineto{\pgfqpoint{5.446761in}{3.114761in}}%
\pgfpathlineto{\pgfqpoint{5.449972in}{3.114966in}}%
\pgfpathlineto{\pgfqpoint{5.453451in}{3.117845in}}%
\pgfpathlineto{\pgfqpoint{5.459071in}{3.126144in}}%
\pgfpathlineto{\pgfqpoint{5.465493in}{3.134509in}}%
\pgfpathlineto{\pgfqpoint{5.468972in}{3.135876in}}%
\pgfpathlineto{\pgfqpoint{5.471916in}{3.134772in}}%
\pgfpathlineto{\pgfqpoint{5.475662in}{3.130741in}}%
\pgfpathlineto{\pgfqpoint{5.488507in}{3.114703in}}%
\pgfpathlineto{\pgfqpoint{5.491451in}{3.114933in}}%
\pgfpathlineto{\pgfqpoint{5.494930in}{3.117758in}}%
\pgfpathlineto{\pgfqpoint{5.500550in}{3.126023in}}%
\pgfpathlineto{\pgfqpoint{5.506972in}{3.134449in}}%
\pgfpathlineto{\pgfqpoint{5.510451in}{3.135877in}}%
\pgfpathlineto{\pgfqpoint{5.513395in}{3.134825in}}%
\pgfpathlineto{\pgfqpoint{5.517141in}{3.130844in}}%
\pgfpathlineto{\pgfqpoint{5.529987in}{3.114724in}}%
\pgfpathlineto{\pgfqpoint{5.533198in}{3.115020in}}%
\pgfpathlineto{\pgfqpoint{5.536677in}{3.117983in}}%
\pgfpathlineto{\pgfqpoint{5.542564in}{3.126760in}}%
\pgfpathlineto{\pgfqpoint{5.548719in}{3.134600in}}%
\pgfpathlineto{\pgfqpoint{5.552198in}{3.135871in}}%
\pgfpathlineto{\pgfqpoint{5.555142in}{3.134687in}}%
\pgfpathlineto{\pgfqpoint{5.559156in}{3.130201in}}%
\pgfpathlineto{\pgfqpoint{5.570931in}{3.114950in}}%
\pgfpathlineto{\pgfqpoint{5.574142in}{3.114772in}}%
\pgfpathlineto{\pgfqpoint{5.577353in}{3.117008in}}%
\pgfpathlineto{\pgfqpoint{5.582170in}{3.123629in}}%
\pgfpathlineto{\pgfqpoint{5.590198in}{3.134542in}}%
\pgfpathlineto{\pgfqpoint{5.593677in}{3.135875in}}%
\pgfpathlineto{\pgfqpoint{5.596621in}{3.134742in}}%
\pgfpathlineto{\pgfqpoint{5.600367in}{3.130683in}}%
\pgfpathlineto{\pgfqpoint{5.612945in}{3.114772in}}%
\pgfpathlineto{\pgfqpoint{5.616156in}{3.114951in}}%
\pgfpathlineto{\pgfqpoint{5.619635in}{3.117806in}}%
\pgfpathlineto{\pgfqpoint{5.625255in}{3.126090in}}%
\pgfpathlineto{\pgfqpoint{5.631678in}{3.134482in}}%
\pgfpathlineto{\pgfqpoint{5.635157in}{3.135876in}}%
\pgfpathlineto{\pgfqpoint{5.638100in}{3.134796in}}%
\pgfpathlineto{\pgfqpoint{5.641847in}{3.130787in}}%
\pgfpathlineto{\pgfqpoint{5.654692in}{3.114712in}}%
\pgfpathlineto{\pgfqpoint{5.657903in}{3.115040in}}%
\pgfpathlineto{\pgfqpoint{5.661382in}{3.118032in}}%
\pgfpathlineto{\pgfqpoint{5.667537in}{3.127252in}}%
\pgfpathlineto{\pgfqpoint{5.673424in}{3.134632in}}%
\pgfpathlineto{\pgfqpoint{5.676903in}{3.135869in}}%
\pgfpathlineto{\pgfqpoint{5.679847in}{3.134656in}}%
\pgfpathlineto{\pgfqpoint{5.683861in}{3.130142in}}%
\pgfpathlineto{\pgfqpoint{5.695636in}{3.114932in}}%
\pgfpathlineto{\pgfqpoint{5.698847in}{3.114786in}}%
\pgfpathlineto{\pgfqpoint{5.702058in}{3.117050in}}%
\pgfpathlineto{\pgfqpoint{5.706875in}{3.123695in}}%
\pgfpathlineto{\pgfqpoint{5.714904in}{3.134574in}}%
\pgfpathlineto{\pgfqpoint{5.718383in}{3.135873in}}%
\pgfpathlineto{\pgfqpoint{5.721326in}{3.134712in}}%
\pgfpathlineto{\pgfqpoint{5.725073in}{3.130626in}}%
\pgfpathlineto{\pgfqpoint{5.737650in}{3.114758in}}%
\pgfpathlineto{\pgfqpoint{5.740862in}{3.114969in}}%
\pgfpathlineto{\pgfqpoint{5.744340in}{3.117854in}}%
\pgfpathlineto{\pgfqpoint{5.750228in}{3.126586in}}%
\pgfpathlineto{\pgfqpoint{5.756383in}{3.134515in}}%
\pgfpathlineto{\pgfqpoint{5.759862in}{3.135876in}}%
\pgfpathlineto{\pgfqpoint{5.762805in}{3.134766in}}%
\pgfpathlineto{\pgfqpoint{5.766552in}{3.130730in}}%
\pgfpathlineto{\pgfqpoint{5.779130in}{3.114783in}}%
\pgfpathlineto{\pgfqpoint{5.782341in}{3.114936in}}%
\pgfpathlineto{\pgfqpoint{5.785820in}{3.117767in}}%
\pgfpathlineto{\pgfqpoint{5.791439in}{3.126035in}}%
\pgfpathlineto{\pgfqpoint{5.797862in}{3.134455in}}%
\pgfpathlineto{\pgfqpoint{5.801341in}{3.135877in}}%
\pgfpathlineto{\pgfqpoint{5.804285in}{3.134820in}}%
\pgfpathlineto{\pgfqpoint{5.808031in}{3.130834in}}%
\pgfpathlineto{\pgfqpoint{5.820876in}{3.114722in}}%
\pgfpathlineto{\pgfqpoint{5.824088in}{3.115024in}}%
\pgfpathlineto{\pgfqpoint{5.827567in}{3.117992in}}%
\pgfpathlineto{\pgfqpoint{5.833454in}{3.126772in}}%
\pgfpathlineto{\pgfqpoint{5.839609in}{3.134606in}}%
\pgfpathlineto{\pgfqpoint{5.843088in}{3.135871in}}%
\pgfpathlineto{\pgfqpoint{5.846031in}{3.134681in}}%
\pgfpathlineto{\pgfqpoint{5.850046in}{3.130191in}}%
\pgfpathlineto{\pgfqpoint{5.861820in}{3.114947in}}%
\pgfpathlineto{\pgfqpoint{5.865032in}{3.114775in}}%
\pgfpathlineto{\pgfqpoint{5.868243in}{3.117015in}}%
\pgfpathlineto{\pgfqpoint{5.873060in}{3.123641in}}%
\pgfpathlineto{\pgfqpoint{5.881088in}{3.134548in}}%
\pgfpathlineto{\pgfqpoint{5.884567in}{3.135874in}}%
\pgfpathlineto{\pgfqpoint{5.887511in}{3.134736in}}%
\pgfpathlineto{\pgfqpoint{5.891257in}{3.130673in}}%
\pgfpathlineto{\pgfqpoint{5.903835in}{3.114769in}}%
\pgfpathlineto{\pgfqpoint{5.907046in}{3.114954in}}%
\pgfpathlineto{\pgfqpoint{5.910525in}{3.117815in}}%
\pgfpathlineto{\pgfqpoint{5.916145in}{3.126102in}}%
\pgfpathlineto{\pgfqpoint{5.922567in}{3.134488in}}%
\pgfpathlineto{\pgfqpoint{5.926046in}{3.135876in}}%
\pgfpathlineto{\pgfqpoint{5.928990in}{3.134791in}}%
\pgfpathlineto{\pgfqpoint{5.932736in}{3.130777in}}%
\pgfpathlineto{\pgfqpoint{5.945582in}{3.114710in}}%
\pgfpathlineto{\pgfqpoint{5.948793in}{3.115044in}}%
\pgfpathlineto{\pgfqpoint{5.952272in}{3.118041in}}%
\pgfpathlineto{\pgfqpoint{5.958427in}{3.127264in}}%
\pgfpathlineto{\pgfqpoint{5.964314in}{3.134637in}}%
\pgfpathlineto{\pgfqpoint{5.967793in}{3.135869in}}%
\pgfpathlineto{\pgfqpoint{5.970737in}{3.134650in}}%
\pgfpathlineto{\pgfqpoint{5.974751in}{3.130131in}}%
\pgfpathlineto{\pgfqpoint{5.986526in}{3.114929in}}%
\pgfpathlineto{\pgfqpoint{5.989737in}{3.114789in}}%
\pgfpathlineto{\pgfqpoint{5.992948in}{3.117058in}}%
\pgfpathlineto{\pgfqpoint{5.997765in}{3.123707in}}%
\pgfpathlineto{\pgfqpoint{6.005793in}{3.134580in}}%
\pgfpathlineto{\pgfqpoint{6.009272in}{3.135873in}}%
\pgfpathlineto{\pgfqpoint{6.012216in}{3.134706in}}%
\pgfpathlineto{\pgfqpoint{6.015962in}{3.130615in}}%
\pgfpathlineto{\pgfqpoint{6.028540in}{3.114756in}}%
\pgfpathlineto{\pgfqpoint{6.031751in}{3.114973in}}%
\pgfpathlineto{\pgfqpoint{6.035230in}{3.117863in}}%
\pgfpathlineto{\pgfqpoint{6.041118in}{3.126598in}}%
\pgfpathlineto{\pgfqpoint{6.047273in}{3.134521in}}%
\pgfpathlineto{\pgfqpoint{6.050751in}{3.135875in}}%
\pgfpathlineto{\pgfqpoint{6.053695in}{3.134761in}}%
\pgfpathlineto{\pgfqpoint{6.057442in}{3.130720in}}%
\pgfpathlineto{\pgfqpoint{6.070019in}{3.114781in}}%
\pgfpathlineto{\pgfqpoint{6.073231in}{3.114939in}}%
\pgfpathlineto{\pgfqpoint{6.076709in}{3.117775in}}%
\pgfpathlineto{\pgfqpoint{6.082329in}{3.126048in}}%
\pgfpathlineto{\pgfqpoint{6.088752in}{3.134461in}}%
\pgfpathlineto{\pgfqpoint{6.092231in}{3.135877in}}%
\pgfpathlineto{\pgfqpoint{6.095174in}{3.134814in}}%
\pgfpathlineto{\pgfqpoint{6.098921in}{3.130824in}}%
\pgfpathlineto{\pgfqpoint{6.111766in}{3.114720in}}%
\pgfpathlineto{\pgfqpoint{6.114977in}{3.115027in}}%
\pgfpathlineto{\pgfqpoint{6.118456in}{3.118001in}}%
\pgfpathlineto{\pgfqpoint{6.124344in}{3.126784in}}%
\pgfpathlineto{\pgfqpoint{6.130499in}{3.134612in}}%
\pgfpathlineto{\pgfqpoint{6.133977in}{3.135871in}}%
\pgfpathlineto{\pgfqpoint{6.136921in}{3.134676in}}%
\pgfpathlineto{\pgfqpoint{6.140935in}{3.130180in}}%
\pgfpathlineto{\pgfqpoint{6.152710in}{3.114943in}}%
\pgfpathlineto{\pgfqpoint{6.155921in}{3.114777in}}%
\pgfpathlineto{\pgfqpoint{6.159133in}{3.117023in}}%
\pgfpathlineto{\pgfqpoint{6.163950in}{3.123653in}}%
\pgfpathlineto{\pgfqpoint{6.171978in}{3.134554in}}%
\pgfpathlineto{\pgfqpoint{6.175457in}{3.135874in}}%
\pgfpathlineto{\pgfqpoint{6.178400in}{3.134731in}}%
\pgfpathlineto{\pgfqpoint{6.182147in}{3.130662in}}%
\pgfpathlineto{\pgfqpoint{6.194724in}{3.114767in}}%
\pgfpathlineto{\pgfqpoint{6.197936in}{3.114957in}}%
\pgfpathlineto{\pgfqpoint{6.201415in}{3.117823in}}%
\pgfpathlineto{\pgfqpoint{6.207034in}{3.126114in}}%
\pgfpathlineto{\pgfqpoint{6.213457in}{3.134494in}}%
\pgfpathlineto{\pgfqpoint{6.216936in}{3.135876in}}%
\pgfpathlineto{\pgfqpoint{6.219880in}{3.134785in}}%
\pgfpathlineto{\pgfqpoint{6.223626in}{3.130767in}}%
\pgfpathlineto{\pgfqpoint{6.236471in}{3.114708in}}%
\pgfpathlineto{\pgfqpoint{6.239415in}{3.114924in}}%
\pgfpathlineto{\pgfqpoint{6.242894in}{3.117736in}}%
\pgfpathlineto{\pgfqpoint{6.248514in}{3.125993in}}%
\pgfpathlineto{\pgfqpoint{6.255204in}{3.134643in}}%
\pgfpathlineto{\pgfqpoint{6.258683in}{3.135868in}}%
\pgfpathlineto{\pgfqpoint{6.261626in}{3.134645in}}%
\pgfpathlineto{\pgfqpoint{6.265641in}{3.130121in}}%
\pgfpathlineto{\pgfqpoint{6.277415in}{3.114925in}}%
\pgfpathlineto{\pgfqpoint{6.280627in}{3.114792in}}%
\pgfpathlineto{\pgfqpoint{6.283838in}{3.117066in}}%
\pgfpathlineto{\pgfqpoint{6.288655in}{3.123719in}}%
\pgfpathlineto{\pgfqpoint{6.296683in}{3.134586in}}%
\pgfpathlineto{\pgfqpoint{6.300162in}{3.135872in}}%
\pgfpathlineto{\pgfqpoint{6.303106in}{3.134701in}}%
\pgfpathlineto{\pgfqpoint{6.306852in}{3.130605in}}%
\pgfpathlineto{\pgfqpoint{6.319430in}{3.114753in}}%
\pgfpathlineto{\pgfqpoint{6.322641in}{3.114976in}}%
\pgfpathlineto{\pgfqpoint{6.326120in}{3.117872in}}%
\pgfpathlineto{\pgfqpoint{6.332007in}{3.126610in}}%
\pgfpathlineto{\pgfqpoint{6.338162in}{3.134527in}}%
\pgfpathlineto{\pgfqpoint{6.341641in}{3.135875in}}%
\pgfpathlineto{\pgfqpoint{6.344585in}{3.134756in}}%
\pgfpathlineto{\pgfqpoint{6.348331in}{3.130709in}}%
\pgfpathlineto{\pgfqpoint{6.360909in}{3.114778in}}%
\pgfpathlineto{\pgfqpoint{6.364120in}{3.114942in}}%
\pgfpathlineto{\pgfqpoint{6.367599in}{3.117784in}}%
\pgfpathlineto{\pgfqpoint{6.373219in}{3.126060in}}%
\pgfpathlineto{\pgfqpoint{6.379641in}{3.134467in}}%
\pgfpathlineto{\pgfqpoint{6.383120in}{3.135877in}}%
\pgfpathlineto{\pgfqpoint{6.386064in}{3.134809in}}%
\pgfpathlineto{\pgfqpoint{6.389811in}{3.130813in}}%
\pgfpathlineto{\pgfqpoint{6.402656in}{3.114718in}}%
\pgfpathlineto{\pgfqpoint{6.405867in}{3.115031in}}%
\pgfpathlineto{\pgfqpoint{6.409346in}{3.118010in}}%
\pgfpathlineto{\pgfqpoint{6.415233in}{3.126796in}}%
\pgfpathlineto{\pgfqpoint{6.421388in}{3.134617in}}%
\pgfpathlineto{\pgfqpoint{6.424867in}{3.135870in}}%
\pgfpathlineto{\pgfqpoint{6.427811in}{3.134670in}}%
\pgfpathlineto{\pgfqpoint{6.431825in}{3.130169in}}%
\pgfpathlineto{\pgfqpoint{6.443600in}{3.114940in}}%
\pgfpathlineto{\pgfqpoint{6.446811in}{3.114780in}}%
\pgfpathlineto{\pgfqpoint{6.450022in}{3.117031in}}%
\pgfpathlineto{\pgfqpoint{6.454839in}{3.123665in}}%
\pgfpathlineto{\pgfqpoint{6.462867in}{3.134560in}}%
\pgfpathlineto{\pgfqpoint{6.466346in}{3.135874in}}%
\pgfpathlineto{\pgfqpoint{6.469290in}{3.134725in}}%
\pgfpathlineto{\pgfqpoint{6.473037in}{3.130652in}}%
\pgfpathlineto{\pgfqpoint{6.485614in}{3.114764in}}%
\pgfpathlineto{\pgfqpoint{6.488825in}{3.114961in}}%
\pgfpathlineto{\pgfqpoint{6.492304in}{3.117832in}}%
\pgfpathlineto{\pgfqpoint{6.497924in}{3.126126in}}%
\pgfpathlineto{\pgfqpoint{6.504347in}{3.134500in}}%
\pgfpathlineto{\pgfqpoint{6.507826in}{3.135876in}}%
\pgfpathlineto{\pgfqpoint{6.510769in}{3.134780in}}%
\pgfpathlineto{\pgfqpoint{6.514516in}{3.130756in}}%
\pgfpathlineto{\pgfqpoint{6.527361in}{3.114706in}}%
\pgfpathlineto{\pgfqpoint{6.530305in}{3.114928in}}%
\pgfpathlineto{\pgfqpoint{6.533784in}{3.117745in}}%
\pgfpathlineto{\pgfqpoint{6.539403in}{3.126005in}}%
\pgfpathlineto{\pgfqpoint{6.546094in}{3.134649in}}%
\pgfpathlineto{\pgfqpoint{6.549572in}{3.135868in}}%
\pgfpathlineto{\pgfqpoint{6.552516in}{3.134639in}}%
\pgfpathlineto{\pgfqpoint{6.556530in}{3.130110in}}%
\pgfpathlineto{\pgfqpoint{6.568305in}{3.114922in}}%
\pgfpathlineto{\pgfqpoint{6.571516in}{3.114794in}}%
\pgfpathlineto{\pgfqpoint{6.574728in}{3.117074in}}%
\pgfpathlineto{\pgfqpoint{6.579545in}{3.123731in}}%
\pgfpathlineto{\pgfqpoint{6.587573in}{3.134592in}}%
\pgfpathlineto{\pgfqpoint{6.591052in}{3.135872in}}%
\pgfpathlineto{\pgfqpoint{6.593995in}{3.134695in}}%
\pgfpathlineto{\pgfqpoint{6.598009in}{3.130217in}}%
\pgfpathlineto{\pgfqpoint{6.609784in}{3.114955in}}%
\pgfpathlineto{\pgfqpoint{6.612995in}{3.114769in}}%
\pgfpathlineto{\pgfqpoint{6.616207in}{3.116996in}}%
\pgfpathlineto{\pgfqpoint{6.621024in}{3.123611in}}%
\pgfpathlineto{\pgfqpoint{6.629052in}{3.134533in}}%
\pgfpathlineto{\pgfqpoint{6.632531in}{3.135875in}}%
\pgfpathlineto{\pgfqpoint{6.635475in}{3.134750in}}%
\pgfpathlineto{\pgfqpoint{6.639221in}{3.130699in}}%
\pgfpathlineto{\pgfqpoint{6.651799in}{3.114775in}}%
\pgfpathlineto{\pgfqpoint{6.655010in}{3.114946in}}%
\pgfpathlineto{\pgfqpoint{6.658489in}{3.117793in}}%
\pgfpathlineto{\pgfqpoint{6.663306in}{3.124778in}}%
\pgfpathlineto{\pgfqpoint{6.663306in}{3.124778in}}%
\pgfusepath{stroke}%
\end{pgfscope}%
\begin{pgfscope}%
\pgfpathrectangle{\pgfqpoint{0.467797in}{2.292089in}}{\pgfqpoint{6.490533in}{1.666241in}}%
\pgfusepath{clip}%
\pgfsetrectcap%
\pgfsetroundjoin%
\pgfsetlinewidth{1.505625pt}%
\definecolor{currentstroke}{rgb}{0.890196,0.466667,0.760784}%
\pgfsetstrokecolor{currentstroke}%
\pgfsetdash{}{0pt}%
\pgfpathmoveto{\pgfqpoint{0.762821in}{3.125209in}}%
\pgfpathlineto{\pgfqpoint{0.769779in}{3.134368in}}%
\pgfpathlineto{\pgfqpoint{0.773258in}{3.135571in}}%
\pgfpathlineto{\pgfqpoint{0.776201in}{3.134268in}}%
\pgfpathlineto{\pgfqpoint{0.780215in}{3.129604in}}%
\pgfpathlineto{\pgfqpoint{0.791187in}{3.115305in}}%
\pgfpathlineto{\pgfqpoint{0.794399in}{3.115035in}}%
\pgfpathlineto{\pgfqpoint{0.797610in}{3.117247in}}%
\pgfpathlineto{\pgfqpoint{0.802427in}{3.123906in}}%
\pgfpathlineto{\pgfqpoint{0.810187in}{3.134362in}}%
\pgfpathlineto{\pgfqpoint{0.813666in}{3.135571in}}%
\pgfpathlineto{\pgfqpoint{0.816610in}{3.134273in}}%
\pgfpathlineto{\pgfqpoint{0.820624in}{3.129615in}}%
\pgfpathlineto{\pgfqpoint{0.831596in}{3.115309in}}%
\pgfpathlineto{\pgfqpoint{0.834807in}{3.115032in}}%
\pgfpathlineto{\pgfqpoint{0.838019in}{3.117240in}}%
\pgfpathlineto{\pgfqpoint{0.842836in}{3.123894in}}%
\pgfpathlineto{\pgfqpoint{0.850596in}{3.134357in}}%
\pgfpathlineto{\pgfqpoint{0.854075in}{3.135572in}}%
\pgfpathlineto{\pgfqpoint{0.857019in}{3.134279in}}%
\pgfpathlineto{\pgfqpoint{0.861033in}{3.129626in}}%
\pgfpathlineto{\pgfqpoint{0.872005in}{3.115312in}}%
\pgfpathlineto{\pgfqpoint{0.875216in}{3.115030in}}%
\pgfpathlineto{\pgfqpoint{0.878427in}{3.117232in}}%
\pgfpathlineto{\pgfqpoint{0.883244in}{3.123883in}}%
\pgfpathlineto{\pgfqpoint{0.891005in}{3.134351in}}%
\pgfpathlineto{\pgfqpoint{0.894484in}{3.135572in}}%
\pgfpathlineto{\pgfqpoint{0.897428in}{3.134285in}}%
\pgfpathlineto{\pgfqpoint{0.901442in}{3.129636in}}%
\pgfpathlineto{\pgfqpoint{0.912414in}{3.115316in}}%
\pgfpathlineto{\pgfqpoint{0.915625in}{3.115028in}}%
\pgfpathlineto{\pgfqpoint{0.918836in}{3.117225in}}%
\pgfpathlineto{\pgfqpoint{0.923653in}{3.123871in}}%
\pgfpathlineto{\pgfqpoint{0.931414in}{3.134346in}}%
\pgfpathlineto{\pgfqpoint{0.934893in}{3.135573in}}%
\pgfpathlineto{\pgfqpoint{0.937836in}{3.134291in}}%
\pgfpathlineto{\pgfqpoint{0.941851in}{3.129647in}}%
\pgfpathlineto{\pgfqpoint{0.952822in}{3.115320in}}%
\pgfpathlineto{\pgfqpoint{0.956034in}{3.115025in}}%
\pgfpathlineto{\pgfqpoint{0.959245in}{3.117217in}}%
\pgfpathlineto{\pgfqpoint{0.964062in}{3.123859in}}%
\pgfpathlineto{\pgfqpoint{0.971823in}{3.134340in}}%
\pgfpathlineto{\pgfqpoint{0.975302in}{3.135574in}}%
\pgfpathlineto{\pgfqpoint{0.978245in}{3.134296in}}%
\pgfpathlineto{\pgfqpoint{0.982259in}{3.129658in}}%
\pgfpathlineto{\pgfqpoint{0.993231in}{3.115323in}}%
\pgfpathlineto{\pgfqpoint{0.996443in}{3.115023in}}%
\pgfpathlineto{\pgfqpoint{0.999654in}{3.117210in}}%
\pgfpathlineto{\pgfqpoint{1.004471in}{3.123848in}}%
\pgfpathlineto{\pgfqpoint{1.012231in}{3.134334in}}%
\pgfpathlineto{\pgfqpoint{1.015710in}{3.135574in}}%
\pgfpathlineto{\pgfqpoint{1.018654in}{3.134302in}}%
\pgfpathlineto{\pgfqpoint{1.022668in}{3.129668in}}%
\pgfpathlineto{\pgfqpoint{1.033640in}{3.115327in}}%
\pgfpathlineto{\pgfqpoint{1.036851in}{3.115021in}}%
\pgfpathlineto{\pgfqpoint{1.040063in}{3.117202in}}%
\pgfpathlineto{\pgfqpoint{1.044880in}{3.123836in}}%
\pgfpathlineto{\pgfqpoint{1.052640in}{3.134329in}}%
\pgfpathlineto{\pgfqpoint{1.056119in}{3.135575in}}%
\pgfpathlineto{\pgfqpoint{1.059063in}{3.134308in}}%
\pgfpathlineto{\pgfqpoint{1.063077in}{3.129679in}}%
\pgfpathlineto{\pgfqpoint{1.074049in}{3.115330in}}%
\pgfpathlineto{\pgfqpoint{1.077260in}{3.115019in}}%
\pgfpathlineto{\pgfqpoint{1.080471in}{3.117195in}}%
\pgfpathlineto{\pgfqpoint{1.085288in}{3.123824in}}%
\pgfpathlineto{\pgfqpoint{1.093049in}{3.134323in}}%
\pgfpathlineto{\pgfqpoint{1.096528in}{3.135575in}}%
\pgfpathlineto{\pgfqpoint{1.099472in}{3.134313in}}%
\pgfpathlineto{\pgfqpoint{1.103486in}{3.129690in}}%
\pgfpathlineto{\pgfqpoint{1.114458in}{3.115334in}}%
\pgfpathlineto{\pgfqpoint{1.117669in}{3.115016in}}%
\pgfpathlineto{\pgfqpoint{1.120880in}{3.117187in}}%
\pgfpathlineto{\pgfqpoint{1.125697in}{3.123812in}}%
\pgfpathlineto{\pgfqpoint{1.133458in}{3.134318in}}%
\pgfpathlineto{\pgfqpoint{1.136937in}{3.135576in}}%
\pgfpathlineto{\pgfqpoint{1.139880in}{3.134319in}}%
\pgfpathlineto{\pgfqpoint{1.143894in}{3.129700in}}%
\pgfpathlineto{\pgfqpoint{1.154866in}{3.115338in}}%
\pgfpathlineto{\pgfqpoint{1.158078in}{3.115014in}}%
\pgfpathlineto{\pgfqpoint{1.161289in}{3.117180in}}%
\pgfpathlineto{\pgfqpoint{1.166106in}{3.123801in}}%
\pgfpathlineto{\pgfqpoint{1.173867in}{3.134312in}}%
\pgfpathlineto{\pgfqpoint{1.177345in}{3.135576in}}%
\pgfpathlineto{\pgfqpoint{1.180289in}{3.134325in}}%
\pgfpathlineto{\pgfqpoint{1.184303in}{3.129711in}}%
\pgfpathlineto{\pgfqpoint{1.195275in}{3.115341in}}%
\pgfpathlineto{\pgfqpoint{1.198486in}{3.115012in}}%
\pgfpathlineto{\pgfqpoint{1.201698in}{3.117172in}}%
\pgfpathlineto{\pgfqpoint{1.206515in}{3.123789in}}%
\pgfpathlineto{\pgfqpoint{1.214275in}{3.134306in}}%
\pgfpathlineto{\pgfqpoint{1.217754in}{3.135577in}}%
\pgfpathlineto{\pgfqpoint{1.220698in}{3.134330in}}%
\pgfpathlineto{\pgfqpoint{1.224712in}{3.129722in}}%
\pgfpathlineto{\pgfqpoint{1.235684in}{3.115345in}}%
\pgfpathlineto{\pgfqpoint{1.238895in}{3.115010in}}%
\pgfpathlineto{\pgfqpoint{1.242107in}{3.117165in}}%
\pgfpathlineto{\pgfqpoint{1.246923in}{3.123777in}}%
\pgfpathlineto{\pgfqpoint{1.254684in}{3.134300in}}%
\pgfpathlineto{\pgfqpoint{1.258163in}{3.135577in}}%
\pgfpathlineto{\pgfqpoint{1.261107in}{3.134336in}}%
\pgfpathlineto{\pgfqpoint{1.265121in}{3.129732in}}%
\pgfpathlineto{\pgfqpoint{1.276360in}{3.115223in}}%
\pgfpathlineto{\pgfqpoint{1.279572in}{3.115095in}}%
\pgfpathlineto{\pgfqpoint{1.282783in}{3.117436in}}%
\pgfpathlineto{\pgfqpoint{1.287867in}{3.124624in}}%
\pgfpathlineto{\pgfqpoint{1.295093in}{3.134295in}}%
\pgfpathlineto{\pgfqpoint{1.298572in}{3.135577in}}%
\pgfpathlineto{\pgfqpoint{1.301515in}{3.134341in}}%
\pgfpathlineto{\pgfqpoint{1.305530in}{3.129743in}}%
\pgfpathlineto{\pgfqpoint{1.316769in}{3.115226in}}%
\pgfpathlineto{\pgfqpoint{1.319980in}{3.115092in}}%
\pgfpathlineto{\pgfqpoint{1.323192in}{3.117428in}}%
\pgfpathlineto{\pgfqpoint{1.328276in}{3.124613in}}%
\pgfpathlineto{\pgfqpoint{1.335502in}{3.134289in}}%
\pgfpathlineto{\pgfqpoint{1.338981in}{3.135578in}}%
\pgfpathlineto{\pgfqpoint{1.341924in}{3.134347in}}%
\pgfpathlineto{\pgfqpoint{1.345938in}{3.129754in}}%
\pgfpathlineto{\pgfqpoint{1.357178in}{3.115229in}}%
\pgfpathlineto{\pgfqpoint{1.360389in}{3.115090in}}%
\pgfpathlineto{\pgfqpoint{1.363600in}{3.117421in}}%
\pgfpathlineto{\pgfqpoint{1.368685in}{3.124601in}}%
\pgfpathlineto{\pgfqpoint{1.375910in}{3.134283in}}%
\pgfpathlineto{\pgfqpoint{1.379389in}{3.135578in}}%
\pgfpathlineto{\pgfqpoint{1.382333in}{3.134353in}}%
\pgfpathlineto{\pgfqpoint{1.386347in}{3.129764in}}%
\pgfpathlineto{\pgfqpoint{1.397587in}{3.115233in}}%
\pgfpathlineto{\pgfqpoint{1.400798in}{3.115087in}}%
\pgfpathlineto{\pgfqpoint{1.404009in}{3.117413in}}%
\pgfpathlineto{\pgfqpoint{1.409094in}{3.124589in}}%
\pgfpathlineto{\pgfqpoint{1.416319in}{3.134278in}}%
\pgfpathlineto{\pgfqpoint{1.419798in}{3.135579in}}%
\pgfpathlineto{\pgfqpoint{1.422742in}{3.134358in}}%
\pgfpathlineto{\pgfqpoint{1.426756in}{3.129775in}}%
\pgfpathlineto{\pgfqpoint{1.437995in}{3.115236in}}%
\pgfpathlineto{\pgfqpoint{1.441207in}{3.115085in}}%
\pgfpathlineto{\pgfqpoint{1.444418in}{3.117405in}}%
\pgfpathlineto{\pgfqpoint{1.449503in}{3.124577in}}%
\pgfpathlineto{\pgfqpoint{1.456728in}{3.134272in}}%
\pgfpathlineto{\pgfqpoint{1.460207in}{3.135579in}}%
\pgfpathlineto{\pgfqpoint{1.463151in}{3.134364in}}%
\pgfpathlineto{\pgfqpoint{1.467165in}{3.129785in}}%
\pgfpathlineto{\pgfqpoint{1.478404in}{3.115239in}}%
\pgfpathlineto{\pgfqpoint{1.481616in}{3.115082in}}%
\pgfpathlineto{\pgfqpoint{1.484827in}{3.117397in}}%
\pgfpathlineto{\pgfqpoint{1.489644in}{3.124135in}}%
\pgfpathlineto{\pgfqpoint{1.497137in}{3.134266in}}%
\pgfpathlineto{\pgfqpoint{1.500616in}{3.135579in}}%
\pgfpathlineto{\pgfqpoint{1.503559in}{3.134369in}}%
\pgfpathlineto{\pgfqpoint{1.507306in}{3.130179in}}%
\pgfpathlineto{\pgfqpoint{1.519081in}{3.115131in}}%
\pgfpathlineto{\pgfqpoint{1.522292in}{3.115182in}}%
\pgfpathlineto{\pgfqpoint{1.525503in}{3.117680in}}%
\pgfpathlineto{\pgfqpoint{1.530855in}{3.125416in}}%
\pgfpathlineto{\pgfqpoint{1.537546in}{3.134260in}}%
\pgfpathlineto{\pgfqpoint{1.541024in}{3.135580in}}%
\pgfpathlineto{\pgfqpoint{1.543968in}{3.134375in}}%
\pgfpathlineto{\pgfqpoint{1.547715in}{3.130190in}}%
\pgfpathlineto{\pgfqpoint{1.559489in}{3.115134in}}%
\pgfpathlineto{\pgfqpoint{1.562701in}{3.115179in}}%
\pgfpathlineto{\pgfqpoint{1.565912in}{3.117672in}}%
\pgfpathlineto{\pgfqpoint{1.571264in}{3.125405in}}%
\pgfpathlineto{\pgfqpoint{1.577954in}{3.134255in}}%
\pgfpathlineto{\pgfqpoint{1.581433in}{3.135580in}}%
\pgfpathlineto{\pgfqpoint{1.584377in}{3.134380in}}%
\pgfpathlineto{\pgfqpoint{1.588123in}{3.130200in}}%
\pgfpathlineto{\pgfqpoint{1.599898in}{3.115137in}}%
\pgfpathlineto{\pgfqpoint{1.603109in}{3.115176in}}%
\pgfpathlineto{\pgfqpoint{1.606321in}{3.117664in}}%
\pgfpathlineto{\pgfqpoint{1.611673in}{3.125393in}}%
\pgfpathlineto{\pgfqpoint{1.618363in}{3.134249in}}%
\pgfpathlineto{\pgfqpoint{1.621842in}{3.135581in}}%
\pgfpathlineto{\pgfqpoint{1.624786in}{3.134386in}}%
\pgfpathlineto{\pgfqpoint{1.628532in}{3.130210in}}%
\pgfpathlineto{\pgfqpoint{1.640307in}{3.115140in}}%
\pgfpathlineto{\pgfqpoint{1.643518in}{3.115173in}}%
\pgfpathlineto{\pgfqpoint{1.646730in}{3.117655in}}%
\pgfpathlineto{\pgfqpoint{1.651814in}{3.124949in}}%
\pgfpathlineto{\pgfqpoint{1.658772in}{3.134243in}}%
\pgfpathlineto{\pgfqpoint{1.662251in}{3.135581in}}%
\pgfpathlineto{\pgfqpoint{1.665194in}{3.134392in}}%
\pgfpathlineto{\pgfqpoint{1.668941in}{3.130221in}}%
\pgfpathlineto{\pgfqpoint{1.680716in}{3.115143in}}%
\pgfpathlineto{\pgfqpoint{1.683927in}{3.115170in}}%
\pgfpathlineto{\pgfqpoint{1.687138in}{3.117647in}}%
\pgfpathlineto{\pgfqpoint{1.692223in}{3.124937in}}%
\pgfpathlineto{\pgfqpoint{1.699181in}{3.134237in}}%
\pgfpathlineto{\pgfqpoint{1.702660in}{3.135581in}}%
\pgfpathlineto{\pgfqpoint{1.705603in}{3.134397in}}%
\pgfpathlineto{\pgfqpoint{1.709350in}{3.130231in}}%
\pgfpathlineto{\pgfqpoint{1.721125in}{3.115145in}}%
\pgfpathlineto{\pgfqpoint{1.724336in}{3.115167in}}%
\pgfpathlineto{\pgfqpoint{1.727547in}{3.117639in}}%
\pgfpathlineto{\pgfqpoint{1.732632in}{3.124926in}}%
\pgfpathlineto{\pgfqpoint{1.739589in}{3.134231in}}%
\pgfpathlineto{\pgfqpoint{1.743068in}{3.135581in}}%
\pgfpathlineto{\pgfqpoint{1.746012in}{3.134403in}}%
\pgfpathlineto{\pgfqpoint{1.749759in}{3.130241in}}%
\pgfpathlineto{\pgfqpoint{1.761533in}{3.115148in}}%
\pgfpathlineto{\pgfqpoint{1.764745in}{3.115164in}}%
\pgfpathlineto{\pgfqpoint{1.767956in}{3.117631in}}%
\pgfpathlineto{\pgfqpoint{1.773040in}{3.124914in}}%
\pgfpathlineto{\pgfqpoint{1.779998in}{3.134225in}}%
\pgfpathlineto{\pgfqpoint{1.783477in}{3.135582in}}%
\pgfpathlineto{\pgfqpoint{1.786421in}{3.134408in}}%
\pgfpathlineto{\pgfqpoint{1.790167in}{3.130252in}}%
\pgfpathlineto{\pgfqpoint{1.801942in}{3.115151in}}%
\pgfpathlineto{\pgfqpoint{1.805153in}{3.115161in}}%
\pgfpathlineto{\pgfqpoint{1.808365in}{3.117623in}}%
\pgfpathlineto{\pgfqpoint{1.813449in}{3.124902in}}%
\pgfpathlineto{\pgfqpoint{1.820407in}{3.134220in}}%
\pgfpathlineto{\pgfqpoint{1.823886in}{3.135582in}}%
\pgfpathlineto{\pgfqpoint{1.826830in}{3.134413in}}%
\pgfpathlineto{\pgfqpoint{1.830576in}{3.130262in}}%
\pgfpathlineto{\pgfqpoint{1.842351in}{3.115154in}}%
\pgfpathlineto{\pgfqpoint{1.845562in}{3.115158in}}%
\pgfpathlineto{\pgfqpoint{1.848773in}{3.117615in}}%
\pgfpathlineto{\pgfqpoint{1.853858in}{3.124890in}}%
\pgfpathlineto{\pgfqpoint{1.860816in}{3.134214in}}%
\pgfpathlineto{\pgfqpoint{1.864295in}{3.135582in}}%
\pgfpathlineto{\pgfqpoint{1.867238in}{3.134419in}}%
\pgfpathlineto{\pgfqpoint{1.870985in}{3.130272in}}%
\pgfpathlineto{\pgfqpoint{1.883027in}{3.115059in}}%
\pgfpathlineto{\pgfqpoint{1.885971in}{3.115155in}}%
\pgfpathlineto{\pgfqpoint{1.889182in}{3.117607in}}%
\pgfpathlineto{\pgfqpoint{1.894267in}{3.124878in}}%
\pgfpathlineto{\pgfqpoint{1.901225in}{3.134208in}}%
\pgfpathlineto{\pgfqpoint{1.904703in}{3.135582in}}%
\pgfpathlineto{\pgfqpoint{1.907647in}{3.134424in}}%
\pgfpathlineto{\pgfqpoint{1.911394in}{3.130283in}}%
\pgfpathlineto{\pgfqpoint{1.923436in}{3.115061in}}%
\pgfpathlineto{\pgfqpoint{1.926380in}{3.115152in}}%
\pgfpathlineto{\pgfqpoint{1.929591in}{3.117599in}}%
\pgfpathlineto{\pgfqpoint{1.934676in}{3.124867in}}%
\pgfpathlineto{\pgfqpoint{1.941633in}{3.134202in}}%
\pgfpathlineto{\pgfqpoint{1.945112in}{3.135583in}}%
\pgfpathlineto{\pgfqpoint{1.948056in}{3.134430in}}%
\pgfpathlineto{\pgfqpoint{1.951802in}{3.130293in}}%
\pgfpathlineto{\pgfqpoint{1.963845in}{3.115064in}}%
\pgfpathlineto{\pgfqpoint{1.966788in}{3.115149in}}%
\pgfpathlineto{\pgfqpoint{1.970000in}{3.117591in}}%
\pgfpathlineto{\pgfqpoint{1.975084in}{3.124855in}}%
\pgfpathlineto{\pgfqpoint{1.982310in}{3.134404in}}%
\pgfpathlineto{\pgfqpoint{1.985789in}{3.135566in}}%
\pgfpathlineto{\pgfqpoint{1.988732in}{3.134230in}}%
\pgfpathlineto{\pgfqpoint{1.992746in}{3.129534in}}%
\pgfpathlineto{\pgfqpoint{2.003718in}{3.115283in}}%
\pgfpathlineto{\pgfqpoint{2.006930in}{3.115050in}}%
\pgfpathlineto{\pgfqpoint{2.010141in}{3.117297in}}%
\pgfpathlineto{\pgfqpoint{2.014958in}{3.123982in}}%
\pgfpathlineto{\pgfqpoint{2.022719in}{3.134398in}}%
\pgfpathlineto{\pgfqpoint{2.026197in}{3.135567in}}%
\pgfpathlineto{\pgfqpoint{2.029141in}{3.134236in}}%
\pgfpathlineto{\pgfqpoint{2.033155in}{3.129545in}}%
\pgfpathlineto{\pgfqpoint{2.044127in}{3.115286in}}%
\pgfpathlineto{\pgfqpoint{2.047338in}{3.115047in}}%
\pgfpathlineto{\pgfqpoint{2.050550in}{3.117289in}}%
\pgfpathlineto{\pgfqpoint{2.055367in}{3.123971in}}%
\pgfpathlineto{\pgfqpoint{2.063127in}{3.134393in}}%
\pgfpathlineto{\pgfqpoint{2.066606in}{3.135568in}}%
\pgfpathlineto{\pgfqpoint{2.069550in}{3.134241in}}%
\pgfpathlineto{\pgfqpoint{2.073564in}{3.129556in}}%
\pgfpathlineto{\pgfqpoint{2.084536in}{3.115290in}}%
\pgfpathlineto{\pgfqpoint{2.087747in}{3.115045in}}%
\pgfpathlineto{\pgfqpoint{2.090959in}{3.117282in}}%
\pgfpathlineto{\pgfqpoint{2.095775in}{3.123959in}}%
\pgfpathlineto{\pgfqpoint{2.103536in}{3.134387in}}%
\pgfpathlineto{\pgfqpoint{2.107015in}{3.135568in}}%
\pgfpathlineto{\pgfqpoint{2.109959in}{3.134247in}}%
\pgfpathlineto{\pgfqpoint{2.113973in}{3.129567in}}%
\pgfpathlineto{\pgfqpoint{2.124945in}{3.115293in}}%
\pgfpathlineto{\pgfqpoint{2.128156in}{3.115043in}}%
\pgfpathlineto{\pgfqpoint{2.131367in}{3.117274in}}%
\pgfpathlineto{\pgfqpoint{2.136184in}{3.123947in}}%
\pgfpathlineto{\pgfqpoint{2.143945in}{3.134382in}}%
\pgfpathlineto{\pgfqpoint{2.147424in}{3.135569in}}%
\pgfpathlineto{\pgfqpoint{2.150367in}{3.134253in}}%
\pgfpathlineto{\pgfqpoint{2.154382in}{3.129577in}}%
\pgfpathlineto{\pgfqpoint{2.165353in}{3.115297in}}%
\pgfpathlineto{\pgfqpoint{2.168565in}{3.115040in}}%
\pgfpathlineto{\pgfqpoint{2.171776in}{3.117266in}}%
\pgfpathlineto{\pgfqpoint{2.176593in}{3.123935in}}%
\pgfpathlineto{\pgfqpoint{2.184354in}{3.134376in}}%
\pgfpathlineto{\pgfqpoint{2.187833in}{3.135570in}}%
\pgfpathlineto{\pgfqpoint{2.190776in}{3.134259in}}%
\pgfpathlineto{\pgfqpoint{2.194790in}{3.129588in}}%
\pgfpathlineto{\pgfqpoint{2.205762in}{3.115300in}}%
\pgfpathlineto{\pgfqpoint{2.208974in}{3.115038in}}%
\pgfpathlineto{\pgfqpoint{2.212185in}{3.117259in}}%
\pgfpathlineto{\pgfqpoint{2.217002in}{3.123924in}}%
\pgfpathlineto{\pgfqpoint{2.224762in}{3.134371in}}%
\pgfpathlineto{\pgfqpoint{2.228241in}{3.135570in}}%
\pgfpathlineto{\pgfqpoint{2.231185in}{3.134265in}}%
\pgfpathlineto{\pgfqpoint{2.235199in}{3.129599in}}%
\pgfpathlineto{\pgfqpoint{2.246171in}{3.115304in}}%
\pgfpathlineto{\pgfqpoint{2.249382in}{3.115036in}}%
\pgfpathlineto{\pgfqpoint{2.252594in}{3.117251in}}%
\pgfpathlineto{\pgfqpoint{2.257411in}{3.123912in}}%
\pgfpathlineto{\pgfqpoint{2.265171in}{3.134365in}}%
\pgfpathlineto{\pgfqpoint{2.268650in}{3.135571in}}%
\pgfpathlineto{\pgfqpoint{2.271594in}{3.134270in}}%
\pgfpathlineto{\pgfqpoint{2.275608in}{3.129610in}}%
\pgfpathlineto{\pgfqpoint{2.286580in}{3.115307in}}%
\pgfpathlineto{\pgfqpoint{2.289791in}{3.115033in}}%
\pgfpathlineto{\pgfqpoint{2.293002in}{3.117244in}}%
\pgfpathlineto{\pgfqpoint{2.297819in}{3.123900in}}%
\pgfpathlineto{\pgfqpoint{2.305580in}{3.134360in}}%
\pgfpathlineto{\pgfqpoint{2.309059in}{3.135572in}}%
\pgfpathlineto{\pgfqpoint{2.312003in}{3.134276in}}%
\pgfpathlineto{\pgfqpoint{2.316017in}{3.129620in}}%
\pgfpathlineto{\pgfqpoint{2.326989in}{3.115311in}}%
\pgfpathlineto{\pgfqpoint{2.330200in}{3.115031in}}%
\pgfpathlineto{\pgfqpoint{2.333411in}{3.117236in}}%
\pgfpathlineto{\pgfqpoint{2.338228in}{3.123889in}}%
\pgfpathlineto{\pgfqpoint{2.345989in}{3.134354in}}%
\pgfpathlineto{\pgfqpoint{2.349468in}{3.135572in}}%
\pgfpathlineto{\pgfqpoint{2.352411in}{3.134282in}}%
\pgfpathlineto{\pgfqpoint{2.356425in}{3.129631in}}%
\pgfpathlineto{\pgfqpoint{2.367397in}{3.115314in}}%
\pgfpathlineto{\pgfqpoint{2.370609in}{3.115029in}}%
\pgfpathlineto{\pgfqpoint{2.373820in}{3.117228in}}%
\pgfpathlineto{\pgfqpoint{2.378637in}{3.123877in}}%
\pgfpathlineto{\pgfqpoint{2.386398in}{3.134349in}}%
\pgfpathlineto{\pgfqpoint{2.389876in}{3.135573in}}%
\pgfpathlineto{\pgfqpoint{2.392820in}{3.134288in}}%
\pgfpathlineto{\pgfqpoint{2.396834in}{3.129642in}}%
\pgfpathlineto{\pgfqpoint{2.407806in}{3.115318in}}%
\pgfpathlineto{\pgfqpoint{2.411017in}{3.115027in}}%
\pgfpathlineto{\pgfqpoint{2.414229in}{3.117221in}}%
\pgfpathlineto{\pgfqpoint{2.419046in}{3.123865in}}%
\pgfpathlineto{\pgfqpoint{2.426806in}{3.134343in}}%
\pgfpathlineto{\pgfqpoint{2.430285in}{3.135573in}}%
\pgfpathlineto{\pgfqpoint{2.433229in}{3.134293in}}%
\pgfpathlineto{\pgfqpoint{2.437243in}{3.129652in}}%
\pgfpathlineto{\pgfqpoint{2.448215in}{3.115321in}}%
\pgfpathlineto{\pgfqpoint{2.451426in}{3.115024in}}%
\pgfpathlineto{\pgfqpoint{2.454638in}{3.117213in}}%
\pgfpathlineto{\pgfqpoint{2.459454in}{3.123853in}}%
\pgfpathlineto{\pgfqpoint{2.467215in}{3.134337in}}%
\pgfpathlineto{\pgfqpoint{2.470694in}{3.135574in}}%
\pgfpathlineto{\pgfqpoint{2.473638in}{3.134299in}}%
\pgfpathlineto{\pgfqpoint{2.477652in}{3.129663in}}%
\pgfpathlineto{\pgfqpoint{2.488624in}{3.115325in}}%
\pgfpathlineto{\pgfqpoint{2.491835in}{3.115022in}}%
\pgfpathlineto{\pgfqpoint{2.495046in}{3.117206in}}%
\pgfpathlineto{\pgfqpoint{2.499863in}{3.123842in}}%
\pgfpathlineto{\pgfqpoint{2.507624in}{3.134332in}}%
\pgfpathlineto{\pgfqpoint{2.511103in}{3.135574in}}%
\pgfpathlineto{\pgfqpoint{2.514046in}{3.134305in}}%
\pgfpathlineto{\pgfqpoint{2.518061in}{3.129674in}}%
\pgfpathlineto{\pgfqpoint{2.529033in}{3.115329in}}%
\pgfpathlineto{\pgfqpoint{2.532244in}{3.115020in}}%
\pgfpathlineto{\pgfqpoint{2.535455in}{3.117198in}}%
\pgfpathlineto{\pgfqpoint{2.540272in}{3.123830in}}%
\pgfpathlineto{\pgfqpoint{2.548033in}{3.134326in}}%
\pgfpathlineto{\pgfqpoint{2.551512in}{3.135575in}}%
\pgfpathlineto{\pgfqpoint{2.554455in}{3.134310in}}%
\pgfpathlineto{\pgfqpoint{2.558469in}{3.129684in}}%
\pgfpathlineto{\pgfqpoint{2.569441in}{3.115332in}}%
\pgfpathlineto{\pgfqpoint{2.572653in}{3.115018in}}%
\pgfpathlineto{\pgfqpoint{2.575864in}{3.117191in}}%
\pgfpathlineto{\pgfqpoint{2.580681in}{3.123818in}}%
\pgfpathlineto{\pgfqpoint{2.588441in}{3.134320in}}%
\pgfpathlineto{\pgfqpoint{2.591920in}{3.135575in}}%
\pgfpathlineto{\pgfqpoint{2.594864in}{3.134316in}}%
\pgfpathlineto{\pgfqpoint{2.598878in}{3.129695in}}%
\pgfpathlineto{\pgfqpoint{2.609850in}{3.115336in}}%
\pgfpathlineto{\pgfqpoint{2.613061in}{3.115015in}}%
\pgfpathlineto{\pgfqpoint{2.616273in}{3.117183in}}%
\pgfpathlineto{\pgfqpoint{2.621090in}{3.123806in}}%
\pgfpathlineto{\pgfqpoint{2.628850in}{3.134315in}}%
\pgfpathlineto{\pgfqpoint{2.632329in}{3.135576in}}%
\pgfpathlineto{\pgfqpoint{2.635273in}{3.134322in}}%
\pgfpathlineto{\pgfqpoint{2.639287in}{3.129706in}}%
\pgfpathlineto{\pgfqpoint{2.650259in}{3.115339in}}%
\pgfpathlineto{\pgfqpoint{2.653470in}{3.115013in}}%
\pgfpathlineto{\pgfqpoint{2.656681in}{3.117176in}}%
\pgfpathlineto{\pgfqpoint{2.661498in}{3.123795in}}%
\pgfpathlineto{\pgfqpoint{2.669259in}{3.134309in}}%
\pgfpathlineto{\pgfqpoint{2.672738in}{3.135576in}}%
\pgfpathlineto{\pgfqpoint{2.675682in}{3.134327in}}%
\pgfpathlineto{\pgfqpoint{2.679696in}{3.129716in}}%
\pgfpathlineto{\pgfqpoint{2.690668in}{3.115343in}}%
\pgfpathlineto{\pgfqpoint{2.693879in}{3.115011in}}%
\pgfpathlineto{\pgfqpoint{2.697090in}{3.117168in}}%
\pgfpathlineto{\pgfqpoint{2.701907in}{3.123783in}}%
\pgfpathlineto{\pgfqpoint{2.709668in}{3.134303in}}%
\pgfpathlineto{\pgfqpoint{2.713147in}{3.135577in}}%
\pgfpathlineto{\pgfqpoint{2.716090in}{3.134333in}}%
\pgfpathlineto{\pgfqpoint{2.720105in}{3.129727in}}%
\pgfpathlineto{\pgfqpoint{2.731076in}{3.115347in}}%
\pgfpathlineto{\pgfqpoint{2.734288in}{3.115009in}}%
\pgfpathlineto{\pgfqpoint{2.737499in}{3.117161in}}%
\pgfpathlineto{\pgfqpoint{2.742316in}{3.123771in}}%
\pgfpathlineto{\pgfqpoint{2.750077in}{3.134298in}}%
\pgfpathlineto{\pgfqpoint{2.753555in}{3.135577in}}%
\pgfpathlineto{\pgfqpoint{2.756499in}{3.134339in}}%
\pgfpathlineto{\pgfqpoint{2.760513in}{3.129738in}}%
\pgfpathlineto{\pgfqpoint{2.771753in}{3.115225in}}%
\pgfpathlineto{\pgfqpoint{2.774964in}{3.115094in}}%
\pgfpathlineto{\pgfqpoint{2.778175in}{3.117432in}}%
\pgfpathlineto{\pgfqpoint{2.783260in}{3.124618in}}%
\pgfpathlineto{\pgfqpoint{2.790485in}{3.134292in}}%
\pgfpathlineto{\pgfqpoint{2.793964in}{3.135578in}}%
\pgfpathlineto{\pgfqpoint{2.796908in}{3.134344in}}%
\pgfpathlineto{\pgfqpoint{2.800922in}{3.129748in}}%
\pgfpathlineto{\pgfqpoint{2.812162in}{3.115228in}}%
\pgfpathlineto{\pgfqpoint{2.815373in}{3.115091in}}%
\pgfpathlineto{\pgfqpoint{2.818584in}{3.117425in}}%
\pgfpathlineto{\pgfqpoint{2.823669in}{3.124607in}}%
\pgfpathlineto{\pgfqpoint{2.830894in}{3.134286in}}%
\pgfpathlineto{\pgfqpoint{2.834373in}{3.135578in}}%
\pgfpathlineto{\pgfqpoint{2.837317in}{3.134350in}}%
\pgfpathlineto{\pgfqpoint{2.841331in}{3.129759in}}%
\pgfpathlineto{\pgfqpoint{2.852570in}{3.115231in}}%
\pgfpathlineto{\pgfqpoint{2.855782in}{3.115089in}}%
\pgfpathlineto{\pgfqpoint{2.858993in}{3.117417in}}%
\pgfpathlineto{\pgfqpoint{2.864078in}{3.124595in}}%
\pgfpathlineto{\pgfqpoint{2.871303in}{3.134280in}}%
\pgfpathlineto{\pgfqpoint{2.874782in}{3.135579in}}%
\pgfpathlineto{\pgfqpoint{2.877726in}{3.134355in}}%
\pgfpathlineto{\pgfqpoint{2.881740in}{3.129770in}}%
\pgfpathlineto{\pgfqpoint{2.892979in}{3.115234in}}%
\pgfpathlineto{\pgfqpoint{2.896190in}{3.115086in}}%
\pgfpathlineto{\pgfqpoint{2.899402in}{3.117409in}}%
\pgfpathlineto{\pgfqpoint{2.904486in}{3.124583in}}%
\pgfpathlineto{\pgfqpoint{2.911712in}{3.134275in}}%
\pgfpathlineto{\pgfqpoint{2.915191in}{3.135579in}}%
\pgfpathlineto{\pgfqpoint{2.918134in}{3.134361in}}%
\pgfpathlineto{\pgfqpoint{2.922148in}{3.129780in}}%
\pgfpathlineto{\pgfqpoint{2.933388in}{3.115237in}}%
\pgfpathlineto{\pgfqpoint{2.936599in}{3.115083in}}%
\pgfpathlineto{\pgfqpoint{2.939811in}{3.117401in}}%
\pgfpathlineto{\pgfqpoint{2.944627in}{3.124141in}}%
\pgfpathlineto{\pgfqpoint{2.952120in}{3.134269in}}%
\pgfpathlineto{\pgfqpoint{2.955599in}{3.135579in}}%
\pgfpathlineto{\pgfqpoint{2.958543in}{3.134367in}}%
\pgfpathlineto{\pgfqpoint{2.962557in}{3.129791in}}%
\pgfpathlineto{\pgfqpoint{2.973797in}{3.115241in}}%
\pgfpathlineto{\pgfqpoint{2.977008in}{3.115081in}}%
\pgfpathlineto{\pgfqpoint{2.980219in}{3.117393in}}%
\pgfpathlineto{\pgfqpoint{2.985036in}{3.124129in}}%
\pgfpathlineto{\pgfqpoint{2.992529in}{3.134263in}}%
\pgfpathlineto{\pgfqpoint{2.996008in}{3.135580in}}%
\pgfpathlineto{\pgfqpoint{2.998952in}{3.134372in}}%
\pgfpathlineto{\pgfqpoint{3.002698in}{3.130184in}}%
\pgfpathlineto{\pgfqpoint{3.014473in}{3.115133in}}%
\pgfpathlineto{\pgfqpoint{3.017684in}{3.115180in}}%
\pgfpathlineto{\pgfqpoint{3.020896in}{3.117676in}}%
\pgfpathlineto{\pgfqpoint{3.026248in}{3.125410in}}%
\pgfpathlineto{\pgfqpoint{3.032938in}{3.134257in}}%
\pgfpathlineto{\pgfqpoint{3.036417in}{3.135580in}}%
\pgfpathlineto{\pgfqpoint{3.039361in}{3.134378in}}%
\pgfpathlineto{\pgfqpoint{3.043107in}{3.130195in}}%
\pgfpathlineto{\pgfqpoint{3.054882in}{3.115135in}}%
\pgfpathlineto{\pgfqpoint{3.058093in}{3.115177in}}%
\pgfpathlineto{\pgfqpoint{3.061304in}{3.117668in}}%
\pgfpathlineto{\pgfqpoint{3.066657in}{3.125399in}}%
\pgfpathlineto{\pgfqpoint{3.073347in}{3.134252in}}%
\pgfpathlineto{\pgfqpoint{3.076826in}{3.135580in}}%
\pgfpathlineto{\pgfqpoint{3.079769in}{3.134383in}}%
\pgfpathlineto{\pgfqpoint{3.083516in}{3.130205in}}%
\pgfpathlineto{\pgfqpoint{3.095291in}{3.115138in}}%
\pgfpathlineto{\pgfqpoint{3.098502in}{3.115174in}}%
\pgfpathlineto{\pgfqpoint{3.101713in}{3.117659in}}%
\pgfpathlineto{\pgfqpoint{3.106798in}{3.124955in}}%
\pgfpathlineto{\pgfqpoint{3.113756in}{3.134246in}}%
\pgfpathlineto{\pgfqpoint{3.117235in}{3.135581in}}%
\pgfpathlineto{\pgfqpoint{3.120178in}{3.134389in}}%
\pgfpathlineto{\pgfqpoint{3.123925in}{3.130215in}}%
\pgfpathlineto{\pgfqpoint{3.135699in}{3.115141in}}%
\pgfpathlineto{\pgfqpoint{3.138911in}{3.115171in}}%
\pgfpathlineto{\pgfqpoint{3.142122in}{3.117651in}}%
\pgfpathlineto{\pgfqpoint{3.147207in}{3.124943in}}%
\pgfpathlineto{\pgfqpoint{3.154164in}{3.134240in}}%
\pgfpathlineto{\pgfqpoint{3.157643in}{3.135581in}}%
\pgfpathlineto{\pgfqpoint{3.160587in}{3.134394in}}%
\pgfpathlineto{\pgfqpoint{3.164333in}{3.130226in}}%
\pgfpathlineto{\pgfqpoint{3.176108in}{3.115144in}}%
\pgfpathlineto{\pgfqpoint{3.179320in}{3.115168in}}%
\pgfpathlineto{\pgfqpoint{3.182531in}{3.117643in}}%
\pgfpathlineto{\pgfqpoint{3.187615in}{3.124932in}}%
\pgfpathlineto{\pgfqpoint{3.194573in}{3.134234in}}%
\pgfpathlineto{\pgfqpoint{3.198052in}{3.135581in}}%
\pgfpathlineto{\pgfqpoint{3.200996in}{3.134400in}}%
\pgfpathlineto{\pgfqpoint{3.204742in}{3.130236in}}%
\pgfpathlineto{\pgfqpoint{3.216517in}{3.115147in}}%
\pgfpathlineto{\pgfqpoint{3.219728in}{3.115165in}}%
\pgfpathlineto{\pgfqpoint{3.222940in}{3.117635in}}%
\pgfpathlineto{\pgfqpoint{3.228024in}{3.124920in}}%
\pgfpathlineto{\pgfqpoint{3.234982in}{3.134228in}}%
\pgfpathlineto{\pgfqpoint{3.238461in}{3.135582in}}%
\pgfpathlineto{\pgfqpoint{3.241405in}{3.134405in}}%
\pgfpathlineto{\pgfqpoint{3.245151in}{3.130246in}}%
\pgfpathlineto{\pgfqpoint{3.256926in}{3.115150in}}%
\pgfpathlineto{\pgfqpoint{3.260137in}{3.115162in}}%
\pgfpathlineto{\pgfqpoint{3.263348in}{3.117627in}}%
\pgfpathlineto{\pgfqpoint{3.268433in}{3.124908in}}%
\pgfpathlineto{\pgfqpoint{3.275391in}{3.134222in}}%
\pgfpathlineto{\pgfqpoint{3.278870in}{3.135582in}}%
\pgfpathlineto{\pgfqpoint{3.281813in}{3.134411in}}%
\pgfpathlineto{\pgfqpoint{3.285560in}{3.130257in}}%
\pgfpathlineto{\pgfqpoint{3.297335in}{3.115153in}}%
\pgfpathlineto{\pgfqpoint{3.300546in}{3.115159in}}%
\pgfpathlineto{\pgfqpoint{3.303757in}{3.117619in}}%
\pgfpathlineto{\pgfqpoint{3.308842in}{3.124896in}}%
\pgfpathlineto{\pgfqpoint{3.315800in}{3.134217in}}%
\pgfpathlineto{\pgfqpoint{3.319278in}{3.135582in}}%
\pgfpathlineto{\pgfqpoint{3.322222in}{3.134416in}}%
\pgfpathlineto{\pgfqpoint{3.325969in}{3.130267in}}%
\pgfpathlineto{\pgfqpoint{3.337743in}{3.115156in}}%
\pgfpathlineto{\pgfqpoint{3.340955in}{3.115156in}}%
\pgfpathlineto{\pgfqpoint{3.344166in}{3.117611in}}%
\pgfpathlineto{\pgfqpoint{3.349251in}{3.124884in}}%
\pgfpathlineto{\pgfqpoint{3.356208in}{3.134211in}}%
\pgfpathlineto{\pgfqpoint{3.359687in}{3.135582in}}%
\pgfpathlineto{\pgfqpoint{3.362631in}{3.134422in}}%
\pgfpathlineto{\pgfqpoint{3.366377in}{3.130277in}}%
\pgfpathlineto{\pgfqpoint{3.378420in}{3.115060in}}%
\pgfpathlineto{\pgfqpoint{3.381363in}{3.115153in}}%
\pgfpathlineto{\pgfqpoint{3.384575in}{3.117603in}}%
\pgfpathlineto{\pgfqpoint{3.389659in}{3.124872in}}%
\pgfpathlineto{\pgfqpoint{3.396617in}{3.134205in}}%
\pgfpathlineto{\pgfqpoint{3.400096in}{3.135583in}}%
\pgfpathlineto{\pgfqpoint{3.403040in}{3.134427in}}%
\pgfpathlineto{\pgfqpoint{3.406786in}{3.130288in}}%
\pgfpathlineto{\pgfqpoint{3.418829in}{3.115063in}}%
\pgfpathlineto{\pgfqpoint{3.421772in}{3.115150in}}%
\pgfpathlineto{\pgfqpoint{3.424984in}{3.117595in}}%
\pgfpathlineto{\pgfqpoint{3.430068in}{3.124861in}}%
\pgfpathlineto{\pgfqpoint{3.437293in}{3.134407in}}%
\pgfpathlineto{\pgfqpoint{3.440772in}{3.135566in}}%
\pgfpathlineto{\pgfqpoint{3.443716in}{3.134227in}}%
\pgfpathlineto{\pgfqpoint{3.447730in}{3.129529in}}%
\pgfpathlineto{\pgfqpoint{3.458434in}{3.115415in}}%
\pgfpathlineto{\pgfqpoint{3.461646in}{3.114972in}}%
\pgfpathlineto{\pgfqpoint{3.464857in}{3.117028in}}%
\pgfpathlineto{\pgfqpoint{3.469406in}{3.123136in}}%
\pgfpathlineto{\pgfqpoint{3.477702in}{3.134401in}}%
\pgfpathlineto{\pgfqpoint{3.481181in}{3.135567in}}%
\pgfpathlineto{\pgfqpoint{3.484125in}{3.134233in}}%
\pgfpathlineto{\pgfqpoint{3.488139in}{3.129540in}}%
\pgfpathlineto{\pgfqpoint{3.499111in}{3.115284in}}%
\pgfpathlineto{\pgfqpoint{3.502322in}{3.115049in}}%
\pgfpathlineto{\pgfqpoint{3.505533in}{3.117293in}}%
\pgfpathlineto{\pgfqpoint{3.510350in}{3.123977in}}%
\pgfpathlineto{\pgfqpoint{3.518111in}{3.134396in}}%
\pgfpathlineto{\pgfqpoint{3.521590in}{3.135567in}}%
\pgfpathlineto{\pgfqpoint{3.524534in}{3.134239in}}%
\pgfpathlineto{\pgfqpoint{3.528548in}{3.129551in}}%
\pgfpathlineto{\pgfqpoint{3.539520in}{3.115288in}}%
\pgfpathlineto{\pgfqpoint{3.542731in}{3.115046in}}%
\pgfpathlineto{\pgfqpoint{3.545942in}{3.117285in}}%
\pgfpathlineto{\pgfqpoint{3.550759in}{3.123965in}}%
\pgfpathlineto{\pgfqpoint{3.558520in}{3.134390in}}%
\pgfpathlineto{\pgfqpoint{3.561999in}{3.135568in}}%
\pgfpathlineto{\pgfqpoint{3.564942in}{3.134244in}}%
\pgfpathlineto{\pgfqpoint{3.568957in}{3.129561in}}%
\pgfpathlineto{\pgfqpoint{3.579928in}{3.115291in}}%
\pgfpathlineto{\pgfqpoint{3.583140in}{3.115044in}}%
\pgfpathlineto{\pgfqpoint{3.586351in}{3.117278in}}%
\pgfpathlineto{\pgfqpoint{3.591168in}{3.123953in}}%
\pgfpathlineto{\pgfqpoint{3.598929in}{3.134385in}}%
\pgfpathlineto{\pgfqpoint{3.602407in}{3.135569in}}%
\pgfpathlineto{\pgfqpoint{3.605351in}{3.134250in}}%
\pgfpathlineto{\pgfqpoint{3.609365in}{3.129572in}}%
\pgfpathlineto{\pgfqpoint{3.620337in}{3.115295in}}%
\pgfpathlineto{\pgfqpoint{3.623549in}{3.115042in}}%
\pgfpathlineto{\pgfqpoint{3.626760in}{3.117270in}}%
\pgfpathlineto{\pgfqpoint{3.631577in}{3.123941in}}%
\pgfpathlineto{\pgfqpoint{3.639337in}{3.134379in}}%
\pgfpathlineto{\pgfqpoint{3.642816in}{3.135569in}}%
\pgfpathlineto{\pgfqpoint{3.645760in}{3.134256in}}%
\pgfpathlineto{\pgfqpoint{3.649774in}{3.129583in}}%
\pgfpathlineto{\pgfqpoint{3.660746in}{3.115298in}}%
\pgfpathlineto{\pgfqpoint{3.663957in}{3.115039in}}%
\pgfpathlineto{\pgfqpoint{3.667169in}{3.117263in}}%
\pgfpathlineto{\pgfqpoint{3.671986in}{3.123930in}}%
\pgfpathlineto{\pgfqpoint{3.679746in}{3.134374in}}%
\pgfpathlineto{\pgfqpoint{3.683225in}{3.135570in}}%
\pgfpathlineto{\pgfqpoint{3.686169in}{3.134262in}}%
\pgfpathlineto{\pgfqpoint{3.690183in}{3.129593in}}%
\pgfpathlineto{\pgfqpoint{3.701155in}{3.115302in}}%
\pgfpathlineto{\pgfqpoint{3.704366in}{3.115037in}}%
\pgfpathlineto{\pgfqpoint{3.707577in}{3.117255in}}%
\pgfpathlineto{\pgfqpoint{3.712394in}{3.123918in}}%
\pgfpathlineto{\pgfqpoint{3.720155in}{3.134368in}}%
\pgfpathlineto{\pgfqpoint{3.723634in}{3.135571in}}%
\pgfpathlineto{\pgfqpoint{3.726578in}{3.134268in}}%
\pgfpathlineto{\pgfqpoint{3.730592in}{3.129604in}}%
\pgfpathlineto{\pgfqpoint{3.741564in}{3.115305in}}%
\pgfpathlineto{\pgfqpoint{3.744775in}{3.115035in}}%
\pgfpathlineto{\pgfqpoint{3.747986in}{3.117247in}}%
\pgfpathlineto{\pgfqpoint{3.752803in}{3.123906in}}%
\pgfpathlineto{\pgfqpoint{3.760564in}{3.134362in}}%
\pgfpathlineto{\pgfqpoint{3.764043in}{3.135571in}}%
\pgfpathlineto{\pgfqpoint{3.766986in}{3.134273in}}%
\pgfpathlineto{\pgfqpoint{3.771000in}{3.129615in}}%
\pgfpathlineto{\pgfqpoint{3.781972in}{3.115309in}}%
\pgfpathlineto{\pgfqpoint{3.785184in}{3.115032in}}%
\pgfpathlineto{\pgfqpoint{3.788395in}{3.117240in}}%
\pgfpathlineto{\pgfqpoint{3.793212in}{3.123894in}}%
\pgfpathlineto{\pgfqpoint{3.800973in}{3.134357in}}%
\pgfpathlineto{\pgfqpoint{3.804451in}{3.135572in}}%
\pgfpathlineto{\pgfqpoint{3.807395in}{3.134279in}}%
\pgfpathlineto{\pgfqpoint{3.811409in}{3.129626in}}%
\pgfpathlineto{\pgfqpoint{3.822381in}{3.115312in}}%
\pgfpathlineto{\pgfqpoint{3.825592in}{3.115030in}}%
\pgfpathlineto{\pgfqpoint{3.828804in}{3.117232in}}%
\pgfpathlineto{\pgfqpoint{3.833621in}{3.123883in}}%
\pgfpathlineto{\pgfqpoint{3.841381in}{3.134351in}}%
\pgfpathlineto{\pgfqpoint{3.844860in}{3.135572in}}%
\pgfpathlineto{\pgfqpoint{3.847804in}{3.134285in}}%
\pgfpathlineto{\pgfqpoint{3.851818in}{3.129636in}}%
\pgfpathlineto{\pgfqpoint{3.862790in}{3.115316in}}%
\pgfpathlineto{\pgfqpoint{3.866001in}{3.115028in}}%
\pgfpathlineto{\pgfqpoint{3.869212in}{3.117225in}}%
\pgfpathlineto{\pgfqpoint{3.874029in}{3.123871in}}%
\pgfpathlineto{\pgfqpoint{3.881790in}{3.134346in}}%
\pgfpathlineto{\pgfqpoint{3.885269in}{3.135573in}}%
\pgfpathlineto{\pgfqpoint{3.888213in}{3.134291in}}%
\pgfpathlineto{\pgfqpoint{3.892227in}{3.129647in}}%
\pgfpathlineto{\pgfqpoint{3.903199in}{3.115320in}}%
\pgfpathlineto{\pgfqpoint{3.906410in}{3.115025in}}%
\pgfpathlineto{\pgfqpoint{3.909621in}{3.117217in}}%
\pgfpathlineto{\pgfqpoint{3.914438in}{3.123859in}}%
\pgfpathlineto{\pgfqpoint{3.922199in}{3.134340in}}%
\pgfpathlineto{\pgfqpoint{3.925678in}{3.135574in}}%
\pgfpathlineto{\pgfqpoint{3.928621in}{3.134296in}}%
\pgfpathlineto{\pgfqpoint{3.932636in}{3.129658in}}%
\pgfpathlineto{\pgfqpoint{3.943607in}{3.115323in}}%
\pgfpathlineto{\pgfqpoint{3.946819in}{3.115023in}}%
\pgfpathlineto{\pgfqpoint{3.950030in}{3.117210in}}%
\pgfpathlineto{\pgfqpoint{3.954847in}{3.123848in}}%
\pgfpathlineto{\pgfqpoint{3.962608in}{3.134334in}}%
\pgfpathlineto{\pgfqpoint{3.966087in}{3.135574in}}%
\pgfpathlineto{\pgfqpoint{3.969030in}{3.134302in}}%
\pgfpathlineto{\pgfqpoint{3.973044in}{3.129668in}}%
\pgfpathlineto{\pgfqpoint{3.984016in}{3.115327in}}%
\pgfpathlineto{\pgfqpoint{3.987228in}{3.115021in}}%
\pgfpathlineto{\pgfqpoint{3.990439in}{3.117202in}}%
\pgfpathlineto{\pgfqpoint{3.995256in}{3.123836in}}%
\pgfpathlineto{\pgfqpoint{4.003016in}{3.134329in}}%
\pgfpathlineto{\pgfqpoint{4.006495in}{3.135575in}}%
\pgfpathlineto{\pgfqpoint{4.009439in}{3.134308in}}%
\pgfpathlineto{\pgfqpoint{4.013453in}{3.129679in}}%
\pgfpathlineto{\pgfqpoint{4.024425in}{3.115330in}}%
\pgfpathlineto{\pgfqpoint{4.027636in}{3.115019in}}%
\pgfpathlineto{\pgfqpoint{4.030848in}{3.117195in}}%
\pgfpathlineto{\pgfqpoint{4.035665in}{3.123824in}}%
\pgfpathlineto{\pgfqpoint{4.043425in}{3.134323in}}%
\pgfpathlineto{\pgfqpoint{4.046904in}{3.135575in}}%
\pgfpathlineto{\pgfqpoint{4.049848in}{3.134313in}}%
\pgfpathlineto{\pgfqpoint{4.053862in}{3.129690in}}%
\pgfpathlineto{\pgfqpoint{4.064834in}{3.115334in}}%
\pgfpathlineto{\pgfqpoint{4.068045in}{3.115016in}}%
\pgfpathlineto{\pgfqpoint{4.071256in}{3.117187in}}%
\pgfpathlineto{\pgfqpoint{4.076073in}{3.123812in}}%
\pgfpathlineto{\pgfqpoint{4.083834in}{3.134318in}}%
\pgfpathlineto{\pgfqpoint{4.087313in}{3.135576in}}%
\pgfpathlineto{\pgfqpoint{4.090257in}{3.134319in}}%
\pgfpathlineto{\pgfqpoint{4.094271in}{3.129700in}}%
\pgfpathlineto{\pgfqpoint{4.105243in}{3.115338in}}%
\pgfpathlineto{\pgfqpoint{4.108454in}{3.115014in}}%
\pgfpathlineto{\pgfqpoint{4.111665in}{3.117180in}}%
\pgfpathlineto{\pgfqpoint{4.116482in}{3.123801in}}%
\pgfpathlineto{\pgfqpoint{4.124243in}{3.134312in}}%
\pgfpathlineto{\pgfqpoint{4.127722in}{3.135576in}}%
\pgfpathlineto{\pgfqpoint{4.130665in}{3.134325in}}%
\pgfpathlineto{\pgfqpoint{4.134679in}{3.129711in}}%
\pgfpathlineto{\pgfqpoint{4.145651in}{3.115341in}}%
\pgfpathlineto{\pgfqpoint{4.148863in}{3.115012in}}%
\pgfpathlineto{\pgfqpoint{4.152074in}{3.117172in}}%
\pgfpathlineto{\pgfqpoint{4.156891in}{3.123789in}}%
\pgfpathlineto{\pgfqpoint{4.164652in}{3.134306in}}%
\pgfpathlineto{\pgfqpoint{4.168130in}{3.135577in}}%
\pgfpathlineto{\pgfqpoint{4.171074in}{3.134330in}}%
\pgfpathlineto{\pgfqpoint{4.175088in}{3.129722in}}%
\pgfpathlineto{\pgfqpoint{4.186060in}{3.115345in}}%
\pgfpathlineto{\pgfqpoint{4.189271in}{3.115010in}}%
\pgfpathlineto{\pgfqpoint{4.192483in}{3.117165in}}%
\pgfpathlineto{\pgfqpoint{4.197300in}{3.123777in}}%
\pgfpathlineto{\pgfqpoint{4.205060in}{3.134300in}}%
\pgfpathlineto{\pgfqpoint{4.208539in}{3.135577in}}%
\pgfpathlineto{\pgfqpoint{4.211483in}{3.134336in}}%
\pgfpathlineto{\pgfqpoint{4.215497in}{3.129732in}}%
\pgfpathlineto{\pgfqpoint{4.226737in}{3.115223in}}%
\pgfpathlineto{\pgfqpoint{4.229948in}{3.115095in}}%
\pgfpathlineto{\pgfqpoint{4.233159in}{3.117436in}}%
\pgfpathlineto{\pgfqpoint{4.238244in}{3.124624in}}%
\pgfpathlineto{\pgfqpoint{4.245469in}{3.134295in}}%
\pgfpathlineto{\pgfqpoint{4.248948in}{3.135577in}}%
\pgfpathlineto{\pgfqpoint{4.251892in}{3.134341in}}%
\pgfpathlineto{\pgfqpoint{4.255906in}{3.129743in}}%
\pgfpathlineto{\pgfqpoint{4.267145in}{3.115226in}}%
\pgfpathlineto{\pgfqpoint{4.270357in}{3.115092in}}%
\pgfpathlineto{\pgfqpoint{4.273568in}{3.117428in}}%
\pgfpathlineto{\pgfqpoint{4.278652in}{3.124613in}}%
\pgfpathlineto{\pgfqpoint{4.285878in}{3.134289in}}%
\pgfpathlineto{\pgfqpoint{4.289357in}{3.135578in}}%
\pgfpathlineto{\pgfqpoint{4.292300in}{3.134347in}}%
\pgfpathlineto{\pgfqpoint{4.296315in}{3.129754in}}%
\pgfpathlineto{\pgfqpoint{4.307554in}{3.115229in}}%
\pgfpathlineto{\pgfqpoint{4.310765in}{3.115090in}}%
\pgfpathlineto{\pgfqpoint{4.313977in}{3.117421in}}%
\pgfpathlineto{\pgfqpoint{4.319061in}{3.124601in}}%
\pgfpathlineto{\pgfqpoint{4.326287in}{3.134283in}}%
\pgfpathlineto{\pgfqpoint{4.329766in}{3.135578in}}%
\pgfpathlineto{\pgfqpoint{4.332709in}{3.134353in}}%
\pgfpathlineto{\pgfqpoint{4.336723in}{3.129764in}}%
\pgfpathlineto{\pgfqpoint{4.347963in}{3.115233in}}%
\pgfpathlineto{\pgfqpoint{4.351174in}{3.115087in}}%
\pgfpathlineto{\pgfqpoint{4.354385in}{3.117413in}}%
\pgfpathlineto{\pgfqpoint{4.359470in}{3.124589in}}%
\pgfpathlineto{\pgfqpoint{4.366695in}{3.134278in}}%
\pgfpathlineto{\pgfqpoint{4.370174in}{3.135579in}}%
\pgfpathlineto{\pgfqpoint{4.373118in}{3.134358in}}%
\pgfpathlineto{\pgfqpoint{4.377132in}{3.129775in}}%
\pgfpathlineto{\pgfqpoint{4.388372in}{3.115236in}}%
\pgfpathlineto{\pgfqpoint{4.391583in}{3.115085in}}%
\pgfpathlineto{\pgfqpoint{4.394794in}{3.117405in}}%
\pgfpathlineto{\pgfqpoint{4.399879in}{3.124577in}}%
\pgfpathlineto{\pgfqpoint{4.407104in}{3.134272in}}%
\pgfpathlineto{\pgfqpoint{4.410583in}{3.135579in}}%
\pgfpathlineto{\pgfqpoint{4.413527in}{3.134364in}}%
\pgfpathlineto{\pgfqpoint{4.417541in}{3.129785in}}%
\pgfpathlineto{\pgfqpoint{4.428780in}{3.115239in}}%
\pgfpathlineto{\pgfqpoint{4.431992in}{3.115082in}}%
\pgfpathlineto{\pgfqpoint{4.435203in}{3.117397in}}%
\pgfpathlineto{\pgfqpoint{4.440020in}{3.124135in}}%
\pgfpathlineto{\pgfqpoint{4.447513in}{3.134266in}}%
\pgfpathlineto{\pgfqpoint{4.450992in}{3.135579in}}%
\pgfpathlineto{\pgfqpoint{4.453936in}{3.134369in}}%
\pgfpathlineto{\pgfqpoint{4.457682in}{3.130179in}}%
\pgfpathlineto{\pgfqpoint{4.469457in}{3.115131in}}%
\pgfpathlineto{\pgfqpoint{4.472668in}{3.115182in}}%
\pgfpathlineto{\pgfqpoint{4.475879in}{3.117680in}}%
\pgfpathlineto{\pgfqpoint{4.481232in}{3.125416in}}%
\pgfpathlineto{\pgfqpoint{4.487922in}{3.134260in}}%
\pgfpathlineto{\pgfqpoint{4.491401in}{3.135580in}}%
\pgfpathlineto{\pgfqpoint{4.494344in}{3.134375in}}%
\pgfpathlineto{\pgfqpoint{4.498091in}{3.130190in}}%
\pgfpathlineto{\pgfqpoint{4.509866in}{3.115134in}}%
\pgfpathlineto{\pgfqpoint{4.513077in}{3.115179in}}%
\pgfpathlineto{\pgfqpoint{4.516288in}{3.117672in}}%
\pgfpathlineto{\pgfqpoint{4.521640in}{3.125405in}}%
\pgfpathlineto{\pgfqpoint{4.528331in}{3.134255in}}%
\pgfpathlineto{\pgfqpoint{4.531809in}{3.135580in}}%
\pgfpathlineto{\pgfqpoint{4.534753in}{3.134380in}}%
\pgfpathlineto{\pgfqpoint{4.538500in}{3.130200in}}%
\pgfpathlineto{\pgfqpoint{4.550274in}{3.115137in}}%
\pgfpathlineto{\pgfqpoint{4.553486in}{3.115176in}}%
\pgfpathlineto{\pgfqpoint{4.556697in}{3.117664in}}%
\pgfpathlineto{\pgfqpoint{4.562049in}{3.125393in}}%
\pgfpathlineto{\pgfqpoint{4.568739in}{3.134249in}}%
\pgfpathlineto{\pgfqpoint{4.572218in}{3.135581in}}%
\pgfpathlineto{\pgfqpoint{4.575162in}{3.134386in}}%
\pgfpathlineto{\pgfqpoint{4.578908in}{3.130210in}}%
\pgfpathlineto{\pgfqpoint{4.590683in}{3.115140in}}%
\pgfpathlineto{\pgfqpoint{4.593894in}{3.115173in}}%
\pgfpathlineto{\pgfqpoint{4.597106in}{3.117655in}}%
\pgfpathlineto{\pgfqpoint{4.602190in}{3.124949in}}%
\pgfpathlineto{\pgfqpoint{4.609148in}{3.134243in}}%
\pgfpathlineto{\pgfqpoint{4.612627in}{3.135581in}}%
\pgfpathlineto{\pgfqpoint{4.615571in}{3.134392in}}%
\pgfpathlineto{\pgfqpoint{4.619317in}{3.130221in}}%
\pgfpathlineto{\pgfqpoint{4.631092in}{3.115143in}}%
\pgfpathlineto{\pgfqpoint{4.634303in}{3.115170in}}%
\pgfpathlineto{\pgfqpoint{4.637515in}{3.117647in}}%
\pgfpathlineto{\pgfqpoint{4.642599in}{3.124937in}}%
\pgfpathlineto{\pgfqpoint{4.649557in}{3.134237in}}%
\pgfpathlineto{\pgfqpoint{4.653036in}{3.135581in}}%
\pgfpathlineto{\pgfqpoint{4.655979in}{3.134397in}}%
\pgfpathlineto{\pgfqpoint{4.659726in}{3.130231in}}%
\pgfpathlineto{\pgfqpoint{4.671501in}{3.115145in}}%
\pgfpathlineto{\pgfqpoint{4.674712in}{3.115167in}}%
\pgfpathlineto{\pgfqpoint{4.677923in}{3.117639in}}%
\pgfpathlineto{\pgfqpoint{4.683008in}{3.124926in}}%
\pgfpathlineto{\pgfqpoint{4.689966in}{3.134231in}}%
\pgfpathlineto{\pgfqpoint{4.693445in}{3.135581in}}%
\pgfpathlineto{\pgfqpoint{4.696388in}{3.134403in}}%
\pgfpathlineto{\pgfqpoint{4.700135in}{3.130241in}}%
\pgfpathlineto{\pgfqpoint{4.711910in}{3.115148in}}%
\pgfpathlineto{\pgfqpoint{4.715121in}{3.115164in}}%
\pgfpathlineto{\pgfqpoint{4.718332in}{3.117631in}}%
\pgfpathlineto{\pgfqpoint{4.723417in}{3.124914in}}%
\pgfpathlineto{\pgfqpoint{4.730374in}{3.134225in}}%
\pgfpathlineto{\pgfqpoint{4.733853in}{3.135582in}}%
\pgfpathlineto{\pgfqpoint{4.736797in}{3.134408in}}%
\pgfpathlineto{\pgfqpoint{4.740544in}{3.130252in}}%
\pgfpathlineto{\pgfqpoint{4.752318in}{3.115151in}}%
\pgfpathlineto{\pgfqpoint{4.755530in}{3.115161in}}%
\pgfpathlineto{\pgfqpoint{4.758741in}{3.117623in}}%
\pgfpathlineto{\pgfqpoint{4.763825in}{3.124902in}}%
\pgfpathlineto{\pgfqpoint{4.770783in}{3.134220in}}%
\pgfpathlineto{\pgfqpoint{4.774262in}{3.135582in}}%
\pgfpathlineto{\pgfqpoint{4.777206in}{3.134413in}}%
\pgfpathlineto{\pgfqpoint{4.780952in}{3.130262in}}%
\pgfpathlineto{\pgfqpoint{4.792727in}{3.115154in}}%
\pgfpathlineto{\pgfqpoint{4.795938in}{3.115158in}}%
\pgfpathlineto{\pgfqpoint{4.799150in}{3.117615in}}%
\pgfpathlineto{\pgfqpoint{4.804234in}{3.124890in}}%
\pgfpathlineto{\pgfqpoint{4.811192in}{3.134214in}}%
\pgfpathlineto{\pgfqpoint{4.814671in}{3.135582in}}%
\pgfpathlineto{\pgfqpoint{4.817615in}{3.134419in}}%
\pgfpathlineto{\pgfqpoint{4.821361in}{3.130272in}}%
\pgfpathlineto{\pgfqpoint{4.833403in}{3.115059in}}%
\pgfpathlineto{\pgfqpoint{4.836347in}{3.115155in}}%
\pgfpathlineto{\pgfqpoint{4.839558in}{3.117607in}}%
\pgfpathlineto{\pgfqpoint{4.844643in}{3.124878in}}%
\pgfpathlineto{\pgfqpoint{4.851601in}{3.134208in}}%
\pgfpathlineto{\pgfqpoint{4.855080in}{3.135582in}}%
\pgfpathlineto{\pgfqpoint{4.858023in}{3.134424in}}%
\pgfpathlineto{\pgfqpoint{4.861770in}{3.130283in}}%
\pgfpathlineto{\pgfqpoint{4.873812in}{3.115061in}}%
\pgfpathlineto{\pgfqpoint{4.876756in}{3.115152in}}%
\pgfpathlineto{\pgfqpoint{4.879967in}{3.117599in}}%
\pgfpathlineto{\pgfqpoint{4.885052in}{3.124867in}}%
\pgfpathlineto{\pgfqpoint{4.892010in}{3.134202in}}%
\pgfpathlineto{\pgfqpoint{4.895488in}{3.135583in}}%
\pgfpathlineto{\pgfqpoint{4.898432in}{3.134430in}}%
\pgfpathlineto{\pgfqpoint{4.902179in}{3.130293in}}%
\pgfpathlineto{\pgfqpoint{4.914221in}{3.115064in}}%
\pgfpathlineto{\pgfqpoint{4.917165in}{3.115149in}}%
\pgfpathlineto{\pgfqpoint{4.920376in}{3.117591in}}%
\pgfpathlineto{\pgfqpoint{4.925461in}{3.124855in}}%
\pgfpathlineto{\pgfqpoint{4.932686in}{3.134404in}}%
\pgfpathlineto{\pgfqpoint{4.936165in}{3.135566in}}%
\pgfpathlineto{\pgfqpoint{4.939109in}{3.134230in}}%
\pgfpathlineto{\pgfqpoint{4.943123in}{3.129534in}}%
\pgfpathlineto{\pgfqpoint{4.954095in}{3.115283in}}%
\pgfpathlineto{\pgfqpoint{4.957306in}{3.115050in}}%
\pgfpathlineto{\pgfqpoint{4.960517in}{3.117297in}}%
\pgfpathlineto{\pgfqpoint{4.965334in}{3.123982in}}%
\pgfpathlineto{\pgfqpoint{4.973095in}{3.134398in}}%
\pgfpathlineto{\pgfqpoint{4.976574in}{3.135567in}}%
\pgfpathlineto{\pgfqpoint{4.979517in}{3.134236in}}%
\pgfpathlineto{\pgfqpoint{4.983531in}{3.129545in}}%
\pgfpathlineto{\pgfqpoint{4.994503in}{3.115286in}}%
\pgfpathlineto{\pgfqpoint{4.997715in}{3.115047in}}%
\pgfpathlineto{\pgfqpoint{5.000926in}{3.117289in}}%
\pgfpathlineto{\pgfqpoint{5.005743in}{3.123971in}}%
\pgfpathlineto{\pgfqpoint{5.013504in}{3.134393in}}%
\pgfpathlineto{\pgfqpoint{5.016982in}{3.135568in}}%
\pgfpathlineto{\pgfqpoint{5.019926in}{3.134241in}}%
\pgfpathlineto{\pgfqpoint{5.023940in}{3.129556in}}%
\pgfpathlineto{\pgfqpoint{5.034912in}{3.115290in}}%
\pgfpathlineto{\pgfqpoint{5.038123in}{3.115045in}}%
\pgfpathlineto{\pgfqpoint{5.041335in}{3.117282in}}%
\pgfpathlineto{\pgfqpoint{5.046152in}{3.123959in}}%
\pgfpathlineto{\pgfqpoint{5.053912in}{3.134387in}}%
\pgfpathlineto{\pgfqpoint{5.057391in}{3.135568in}}%
\pgfpathlineto{\pgfqpoint{5.060335in}{3.134247in}}%
\pgfpathlineto{\pgfqpoint{5.064349in}{3.129567in}}%
\pgfpathlineto{\pgfqpoint{5.075321in}{3.115293in}}%
\pgfpathlineto{\pgfqpoint{5.078532in}{3.115043in}}%
\pgfpathlineto{\pgfqpoint{5.081744in}{3.117274in}}%
\pgfpathlineto{\pgfqpoint{5.086560in}{3.123947in}}%
\pgfpathlineto{\pgfqpoint{5.094321in}{3.134382in}}%
\pgfpathlineto{\pgfqpoint{5.097800in}{3.135569in}}%
\pgfpathlineto{\pgfqpoint{5.100744in}{3.134253in}}%
\pgfpathlineto{\pgfqpoint{5.104758in}{3.129577in}}%
\pgfpathlineto{\pgfqpoint{5.115730in}{3.115297in}}%
\pgfpathlineto{\pgfqpoint{5.118941in}{3.115040in}}%
\pgfpathlineto{\pgfqpoint{5.122152in}{3.117266in}}%
\pgfpathlineto{\pgfqpoint{5.126969in}{3.123935in}}%
\pgfpathlineto{\pgfqpoint{5.134730in}{3.134376in}}%
\pgfpathlineto{\pgfqpoint{5.138209in}{3.135570in}}%
\pgfpathlineto{\pgfqpoint{5.141152in}{3.134259in}}%
\pgfpathlineto{\pgfqpoint{5.145167in}{3.129588in}}%
\pgfpathlineto{\pgfqpoint{5.156139in}{3.115300in}}%
\pgfpathlineto{\pgfqpoint{5.159350in}{3.115038in}}%
\pgfpathlineto{\pgfqpoint{5.162561in}{3.117259in}}%
\pgfpathlineto{\pgfqpoint{5.167378in}{3.123924in}}%
\pgfpathlineto{\pgfqpoint{5.175139in}{3.134371in}}%
\pgfpathlineto{\pgfqpoint{5.178618in}{3.135570in}}%
\pgfpathlineto{\pgfqpoint{5.181561in}{3.134265in}}%
\pgfpathlineto{\pgfqpoint{5.185575in}{3.129599in}}%
\pgfpathlineto{\pgfqpoint{5.196547in}{3.115304in}}%
\pgfpathlineto{\pgfqpoint{5.199759in}{3.115036in}}%
\pgfpathlineto{\pgfqpoint{5.202970in}{3.117251in}}%
\pgfpathlineto{\pgfqpoint{5.207787in}{3.123912in}}%
\pgfpathlineto{\pgfqpoint{5.215547in}{3.134365in}}%
\pgfpathlineto{\pgfqpoint{5.219026in}{3.135571in}}%
\pgfpathlineto{\pgfqpoint{5.221970in}{3.134270in}}%
\pgfpathlineto{\pgfqpoint{5.225984in}{3.129610in}}%
\pgfpathlineto{\pgfqpoint{5.236956in}{3.115307in}}%
\pgfpathlineto{\pgfqpoint{5.240167in}{3.115033in}}%
\pgfpathlineto{\pgfqpoint{5.243379in}{3.117244in}}%
\pgfpathlineto{\pgfqpoint{5.248196in}{3.123900in}}%
\pgfpathlineto{\pgfqpoint{5.255956in}{3.134360in}}%
\pgfpathlineto{\pgfqpoint{5.259435in}{3.135572in}}%
\pgfpathlineto{\pgfqpoint{5.262379in}{3.134276in}}%
\pgfpathlineto{\pgfqpoint{5.266393in}{3.129620in}}%
\pgfpathlineto{\pgfqpoint{5.277365in}{3.115311in}}%
\pgfpathlineto{\pgfqpoint{5.280576in}{3.115031in}}%
\pgfpathlineto{\pgfqpoint{5.283787in}{3.117236in}}%
\pgfpathlineto{\pgfqpoint{5.288604in}{3.123889in}}%
\pgfpathlineto{\pgfqpoint{5.296365in}{3.134354in}}%
\pgfpathlineto{\pgfqpoint{5.299844in}{3.135572in}}%
\pgfpathlineto{\pgfqpoint{5.302788in}{3.134282in}}%
\pgfpathlineto{\pgfqpoint{5.306802in}{3.129631in}}%
\pgfpathlineto{\pgfqpoint{5.317774in}{3.115314in}}%
\pgfpathlineto{\pgfqpoint{5.320985in}{3.115029in}}%
\pgfpathlineto{\pgfqpoint{5.324196in}{3.117228in}}%
\pgfpathlineto{\pgfqpoint{5.329013in}{3.123877in}}%
\pgfpathlineto{\pgfqpoint{5.336774in}{3.134349in}}%
\pgfpathlineto{\pgfqpoint{5.340253in}{3.135573in}}%
\pgfpathlineto{\pgfqpoint{5.343196in}{3.134288in}}%
\pgfpathlineto{\pgfqpoint{5.347210in}{3.129642in}}%
\pgfpathlineto{\pgfqpoint{5.358182in}{3.115318in}}%
\pgfpathlineto{\pgfqpoint{5.361394in}{3.115027in}}%
\pgfpathlineto{\pgfqpoint{5.364605in}{3.117221in}}%
\pgfpathlineto{\pgfqpoint{5.369422in}{3.123865in}}%
\pgfpathlineto{\pgfqpoint{5.377183in}{3.134343in}}%
\pgfpathlineto{\pgfqpoint{5.380661in}{3.135573in}}%
\pgfpathlineto{\pgfqpoint{5.383605in}{3.134293in}}%
\pgfpathlineto{\pgfqpoint{5.387619in}{3.129652in}}%
\pgfpathlineto{\pgfqpoint{5.398591in}{3.115321in}}%
\pgfpathlineto{\pgfqpoint{5.401802in}{3.115024in}}%
\pgfpathlineto{\pgfqpoint{5.405014in}{3.117213in}}%
\pgfpathlineto{\pgfqpoint{5.409831in}{3.123853in}}%
\pgfpathlineto{\pgfqpoint{5.417591in}{3.134337in}}%
\pgfpathlineto{\pgfqpoint{5.421070in}{3.135574in}}%
\pgfpathlineto{\pgfqpoint{5.424014in}{3.134299in}}%
\pgfpathlineto{\pgfqpoint{5.428028in}{3.129663in}}%
\pgfpathlineto{\pgfqpoint{5.439000in}{3.115325in}}%
\pgfpathlineto{\pgfqpoint{5.442211in}{3.115022in}}%
\pgfpathlineto{\pgfqpoint{5.445423in}{3.117206in}}%
\pgfpathlineto{\pgfqpoint{5.450240in}{3.123842in}}%
\pgfpathlineto{\pgfqpoint{5.458000in}{3.134332in}}%
\pgfpathlineto{\pgfqpoint{5.461479in}{3.135574in}}%
\pgfpathlineto{\pgfqpoint{5.464423in}{3.134305in}}%
\pgfpathlineto{\pgfqpoint{5.468437in}{3.129674in}}%
\pgfpathlineto{\pgfqpoint{5.479409in}{3.115329in}}%
\pgfpathlineto{\pgfqpoint{5.482620in}{3.115020in}}%
\pgfpathlineto{\pgfqpoint{5.485831in}{3.117198in}}%
\pgfpathlineto{\pgfqpoint{5.490648in}{3.123830in}}%
\pgfpathlineto{\pgfqpoint{5.498409in}{3.134326in}}%
\pgfpathlineto{\pgfqpoint{5.501888in}{3.135575in}}%
\pgfpathlineto{\pgfqpoint{5.504831in}{3.134310in}}%
\pgfpathlineto{\pgfqpoint{5.508846in}{3.129684in}}%
\pgfpathlineto{\pgfqpoint{5.519818in}{3.115332in}}%
\pgfpathlineto{\pgfqpoint{5.523029in}{3.115018in}}%
\pgfpathlineto{\pgfqpoint{5.526240in}{3.117191in}}%
\pgfpathlineto{\pgfqpoint{5.531057in}{3.123818in}}%
\pgfpathlineto{\pgfqpoint{5.538818in}{3.134320in}}%
\pgfpathlineto{\pgfqpoint{5.542297in}{3.135575in}}%
\pgfpathlineto{\pgfqpoint{5.545240in}{3.134316in}}%
\pgfpathlineto{\pgfqpoint{5.549254in}{3.129695in}}%
\pgfpathlineto{\pgfqpoint{5.560226in}{3.115336in}}%
\pgfpathlineto{\pgfqpoint{5.563438in}{3.115015in}}%
\pgfpathlineto{\pgfqpoint{5.566649in}{3.117183in}}%
\pgfpathlineto{\pgfqpoint{5.571466in}{3.123806in}}%
\pgfpathlineto{\pgfqpoint{5.579226in}{3.134315in}}%
\pgfpathlineto{\pgfqpoint{5.582705in}{3.135576in}}%
\pgfpathlineto{\pgfqpoint{5.585649in}{3.134322in}}%
\pgfpathlineto{\pgfqpoint{5.589663in}{3.129706in}}%
\pgfpathlineto{\pgfqpoint{5.600635in}{3.115339in}}%
\pgfpathlineto{\pgfqpoint{5.603846in}{3.115013in}}%
\pgfpathlineto{\pgfqpoint{5.607058in}{3.117176in}}%
\pgfpathlineto{\pgfqpoint{5.611875in}{3.123795in}}%
\pgfpathlineto{\pgfqpoint{5.619635in}{3.134309in}}%
\pgfpathlineto{\pgfqpoint{5.623114in}{3.135576in}}%
\pgfpathlineto{\pgfqpoint{5.626058in}{3.134327in}}%
\pgfpathlineto{\pgfqpoint{5.630072in}{3.129716in}}%
\pgfpathlineto{\pgfqpoint{5.641044in}{3.115343in}}%
\pgfpathlineto{\pgfqpoint{5.644255in}{3.115011in}}%
\pgfpathlineto{\pgfqpoint{5.647466in}{3.117168in}}%
\pgfpathlineto{\pgfqpoint{5.652283in}{3.123783in}}%
\pgfpathlineto{\pgfqpoint{5.660044in}{3.134303in}}%
\pgfpathlineto{\pgfqpoint{5.663523in}{3.135577in}}%
\pgfpathlineto{\pgfqpoint{5.666467in}{3.134333in}}%
\pgfpathlineto{\pgfqpoint{5.670481in}{3.129727in}}%
\pgfpathlineto{\pgfqpoint{5.681453in}{3.115347in}}%
\pgfpathlineto{\pgfqpoint{5.684664in}{3.115009in}}%
\pgfpathlineto{\pgfqpoint{5.687875in}{3.117161in}}%
\pgfpathlineto{\pgfqpoint{5.692692in}{3.123771in}}%
\pgfpathlineto{\pgfqpoint{5.700453in}{3.134298in}}%
\pgfpathlineto{\pgfqpoint{5.703932in}{3.135577in}}%
\pgfpathlineto{\pgfqpoint{5.706875in}{3.134339in}}%
\pgfpathlineto{\pgfqpoint{5.710890in}{3.129738in}}%
\pgfpathlineto{\pgfqpoint{5.722129in}{3.115225in}}%
\pgfpathlineto{\pgfqpoint{5.725340in}{3.115094in}}%
\pgfpathlineto{\pgfqpoint{5.728552in}{3.117432in}}%
\pgfpathlineto{\pgfqpoint{5.733636in}{3.124618in}}%
\pgfpathlineto{\pgfqpoint{5.740862in}{3.134292in}}%
\pgfpathlineto{\pgfqpoint{5.744340in}{3.135578in}}%
\pgfpathlineto{\pgfqpoint{5.747284in}{3.134344in}}%
\pgfpathlineto{\pgfqpoint{5.751298in}{3.129748in}}%
\pgfpathlineto{\pgfqpoint{5.762538in}{3.115228in}}%
\pgfpathlineto{\pgfqpoint{5.765749in}{3.115091in}}%
\pgfpathlineto{\pgfqpoint{5.768960in}{3.117425in}}%
\pgfpathlineto{\pgfqpoint{5.774045in}{3.124607in}}%
\pgfpathlineto{\pgfqpoint{5.781270in}{3.134286in}}%
\pgfpathlineto{\pgfqpoint{5.784749in}{3.135578in}}%
\pgfpathlineto{\pgfqpoint{5.787693in}{3.134350in}}%
\pgfpathlineto{\pgfqpoint{5.791707in}{3.129759in}}%
\pgfpathlineto{\pgfqpoint{5.802947in}{3.115231in}}%
\pgfpathlineto{\pgfqpoint{5.806158in}{3.115089in}}%
\pgfpathlineto{\pgfqpoint{5.809369in}{3.117417in}}%
\pgfpathlineto{\pgfqpoint{5.814454in}{3.124595in}}%
\pgfpathlineto{\pgfqpoint{5.821679in}{3.134280in}}%
\pgfpathlineto{\pgfqpoint{5.825158in}{3.135579in}}%
\pgfpathlineto{\pgfqpoint{5.828102in}{3.134355in}}%
\pgfpathlineto{\pgfqpoint{5.832116in}{3.129770in}}%
\pgfpathlineto{\pgfqpoint{5.843355in}{3.115234in}}%
\pgfpathlineto{\pgfqpoint{5.846567in}{3.115086in}}%
\pgfpathlineto{\pgfqpoint{5.849778in}{3.117409in}}%
\pgfpathlineto{\pgfqpoint{5.854863in}{3.124583in}}%
\pgfpathlineto{\pgfqpoint{5.862088in}{3.134275in}}%
\pgfpathlineto{\pgfqpoint{5.865567in}{3.135579in}}%
\pgfpathlineto{\pgfqpoint{5.868511in}{3.134361in}}%
\pgfpathlineto{\pgfqpoint{5.872525in}{3.129780in}}%
\pgfpathlineto{\pgfqpoint{5.883764in}{3.115237in}}%
\pgfpathlineto{\pgfqpoint{5.886975in}{3.115083in}}%
\pgfpathlineto{\pgfqpoint{5.890187in}{3.117401in}}%
\pgfpathlineto{\pgfqpoint{5.895004in}{3.124141in}}%
\pgfpathlineto{\pgfqpoint{5.902497in}{3.134269in}}%
\pgfpathlineto{\pgfqpoint{5.905976in}{3.135579in}}%
\pgfpathlineto{\pgfqpoint{5.908919in}{3.134367in}}%
\pgfpathlineto{\pgfqpoint{5.912933in}{3.129791in}}%
\pgfpathlineto{\pgfqpoint{5.924173in}{3.115241in}}%
\pgfpathlineto{\pgfqpoint{5.927384in}{3.115081in}}%
\pgfpathlineto{\pgfqpoint{5.930596in}{3.117393in}}%
\pgfpathlineto{\pgfqpoint{5.935412in}{3.124129in}}%
\pgfpathlineto{\pgfqpoint{5.942906in}{3.134263in}}%
\pgfpathlineto{\pgfqpoint{5.946384in}{3.135580in}}%
\pgfpathlineto{\pgfqpoint{5.949328in}{3.134372in}}%
\pgfpathlineto{\pgfqpoint{5.953075in}{3.130184in}}%
\pgfpathlineto{\pgfqpoint{5.964849in}{3.115133in}}%
\pgfpathlineto{\pgfqpoint{5.968061in}{3.115180in}}%
\pgfpathlineto{\pgfqpoint{5.971272in}{3.117676in}}%
\pgfpathlineto{\pgfqpoint{5.976624in}{3.125410in}}%
\pgfpathlineto{\pgfqpoint{5.983314in}{3.134257in}}%
\pgfpathlineto{\pgfqpoint{5.986793in}{3.135580in}}%
\pgfpathlineto{\pgfqpoint{5.989737in}{3.134378in}}%
\pgfpathlineto{\pgfqpoint{5.993483in}{3.130195in}}%
\pgfpathlineto{\pgfqpoint{6.005258in}{3.115135in}}%
\pgfpathlineto{\pgfqpoint{6.008469in}{3.115177in}}%
\pgfpathlineto{\pgfqpoint{6.011681in}{3.117668in}}%
\pgfpathlineto{\pgfqpoint{6.017033in}{3.125399in}}%
\pgfpathlineto{\pgfqpoint{6.023723in}{3.134252in}}%
\pgfpathlineto{\pgfqpoint{6.027202in}{3.135580in}}%
\pgfpathlineto{\pgfqpoint{6.030146in}{3.134383in}}%
\pgfpathlineto{\pgfqpoint{6.033892in}{3.130205in}}%
\pgfpathlineto{\pgfqpoint{6.045667in}{3.115138in}}%
\pgfpathlineto{\pgfqpoint{6.048878in}{3.115174in}}%
\pgfpathlineto{\pgfqpoint{6.052089in}{3.117659in}}%
\pgfpathlineto{\pgfqpoint{6.057174in}{3.124955in}}%
\pgfpathlineto{\pgfqpoint{6.064132in}{3.134246in}}%
\pgfpathlineto{\pgfqpoint{6.067611in}{3.135581in}}%
\pgfpathlineto{\pgfqpoint{6.070554in}{3.134389in}}%
\pgfpathlineto{\pgfqpoint{6.074301in}{3.130215in}}%
\pgfpathlineto{\pgfqpoint{6.086076in}{3.115141in}}%
\pgfpathlineto{\pgfqpoint{6.089287in}{3.115171in}}%
\pgfpathlineto{\pgfqpoint{6.092498in}{3.117651in}}%
\pgfpathlineto{\pgfqpoint{6.097583in}{3.124943in}}%
\pgfpathlineto{\pgfqpoint{6.104541in}{3.134240in}}%
\pgfpathlineto{\pgfqpoint{6.108020in}{3.135581in}}%
\pgfpathlineto{\pgfqpoint{6.110963in}{3.134394in}}%
\pgfpathlineto{\pgfqpoint{6.114710in}{3.130226in}}%
\pgfpathlineto{\pgfqpoint{6.126484in}{3.115144in}}%
\pgfpathlineto{\pgfqpoint{6.129696in}{3.115168in}}%
\pgfpathlineto{\pgfqpoint{6.132907in}{3.117643in}}%
\pgfpathlineto{\pgfqpoint{6.137992in}{3.124932in}}%
\pgfpathlineto{\pgfqpoint{6.144949in}{3.134234in}}%
\pgfpathlineto{\pgfqpoint{6.148428in}{3.135581in}}%
\pgfpathlineto{\pgfqpoint{6.151372in}{3.134400in}}%
\pgfpathlineto{\pgfqpoint{6.155119in}{3.130236in}}%
\pgfpathlineto{\pgfqpoint{6.166893in}{3.115147in}}%
\pgfpathlineto{\pgfqpoint{6.170105in}{3.115165in}}%
\pgfpathlineto{\pgfqpoint{6.173316in}{3.117635in}}%
\pgfpathlineto{\pgfqpoint{6.178400in}{3.124920in}}%
\pgfpathlineto{\pgfqpoint{6.185358in}{3.134228in}}%
\pgfpathlineto{\pgfqpoint{6.188837in}{3.135582in}}%
\pgfpathlineto{\pgfqpoint{6.191781in}{3.134405in}}%
\pgfpathlineto{\pgfqpoint{6.195527in}{3.130246in}}%
\pgfpathlineto{\pgfqpoint{6.207302in}{3.115150in}}%
\pgfpathlineto{\pgfqpoint{6.210513in}{3.115162in}}%
\pgfpathlineto{\pgfqpoint{6.213725in}{3.117627in}}%
\pgfpathlineto{\pgfqpoint{6.218809in}{3.124908in}}%
\pgfpathlineto{\pgfqpoint{6.225767in}{3.134222in}}%
\pgfpathlineto{\pgfqpoint{6.229246in}{3.135582in}}%
\pgfpathlineto{\pgfqpoint{6.232190in}{3.134411in}}%
\pgfpathlineto{\pgfqpoint{6.235936in}{3.130257in}}%
\pgfpathlineto{\pgfqpoint{6.247711in}{3.115153in}}%
\pgfpathlineto{\pgfqpoint{6.250922in}{3.115159in}}%
\pgfpathlineto{\pgfqpoint{6.254133in}{3.117619in}}%
\pgfpathlineto{\pgfqpoint{6.259218in}{3.124896in}}%
\pgfpathlineto{\pgfqpoint{6.266176in}{3.134217in}}%
\pgfpathlineto{\pgfqpoint{6.269655in}{3.135582in}}%
\pgfpathlineto{\pgfqpoint{6.272598in}{3.134416in}}%
\pgfpathlineto{\pgfqpoint{6.276345in}{3.130267in}}%
\pgfpathlineto{\pgfqpoint{6.288120in}{3.115156in}}%
\pgfpathlineto{\pgfqpoint{6.291331in}{3.115156in}}%
\pgfpathlineto{\pgfqpoint{6.294542in}{3.117611in}}%
\pgfpathlineto{\pgfqpoint{6.299627in}{3.124884in}}%
\pgfpathlineto{\pgfqpoint{6.306585in}{3.134211in}}%
\pgfpathlineto{\pgfqpoint{6.310063in}{3.135582in}}%
\pgfpathlineto{\pgfqpoint{6.313007in}{3.134422in}}%
\pgfpathlineto{\pgfqpoint{6.316754in}{3.130277in}}%
\pgfpathlineto{\pgfqpoint{6.328796in}{3.115060in}}%
\pgfpathlineto{\pgfqpoint{6.331740in}{3.115153in}}%
\pgfpathlineto{\pgfqpoint{6.334951in}{3.117603in}}%
\pgfpathlineto{\pgfqpoint{6.340036in}{3.124872in}}%
\pgfpathlineto{\pgfqpoint{6.346993in}{3.134205in}}%
\pgfpathlineto{\pgfqpoint{6.350472in}{3.135583in}}%
\pgfpathlineto{\pgfqpoint{6.353416in}{3.134427in}}%
\pgfpathlineto{\pgfqpoint{6.357162in}{3.130288in}}%
\pgfpathlineto{\pgfqpoint{6.369205in}{3.115063in}}%
\pgfpathlineto{\pgfqpoint{6.372148in}{3.115150in}}%
\pgfpathlineto{\pgfqpoint{6.375360in}{3.117595in}}%
\pgfpathlineto{\pgfqpoint{6.380444in}{3.124861in}}%
\pgfpathlineto{\pgfqpoint{6.387670in}{3.134407in}}%
\pgfpathlineto{\pgfqpoint{6.391149in}{3.135566in}}%
\pgfpathlineto{\pgfqpoint{6.394092in}{3.134227in}}%
\pgfpathlineto{\pgfqpoint{6.398106in}{3.129529in}}%
\pgfpathlineto{\pgfqpoint{6.408811in}{3.115415in}}%
\pgfpathlineto{\pgfqpoint{6.412022in}{3.114972in}}%
\pgfpathlineto{\pgfqpoint{6.415233in}{3.117028in}}%
\pgfpathlineto{\pgfqpoint{6.419783in}{3.123136in}}%
\pgfpathlineto{\pgfqpoint{6.428078in}{3.134401in}}%
\pgfpathlineto{\pgfqpoint{6.431557in}{3.135567in}}%
\pgfpathlineto{\pgfqpoint{6.434501in}{3.134233in}}%
\pgfpathlineto{\pgfqpoint{6.438515in}{3.129540in}}%
\pgfpathlineto{\pgfqpoint{6.449487in}{3.115284in}}%
\pgfpathlineto{\pgfqpoint{6.452698in}{3.115049in}}%
\pgfpathlineto{\pgfqpoint{6.455910in}{3.117293in}}%
\pgfpathlineto{\pgfqpoint{6.460727in}{3.123977in}}%
\pgfpathlineto{\pgfqpoint{6.468487in}{3.134396in}}%
\pgfpathlineto{\pgfqpoint{6.471966in}{3.135567in}}%
\pgfpathlineto{\pgfqpoint{6.474910in}{3.134239in}}%
\pgfpathlineto{\pgfqpoint{6.478924in}{3.129551in}}%
\pgfpathlineto{\pgfqpoint{6.489896in}{3.115288in}}%
\pgfpathlineto{\pgfqpoint{6.493107in}{3.115046in}}%
\pgfpathlineto{\pgfqpoint{6.496318in}{3.117285in}}%
\pgfpathlineto{\pgfqpoint{6.501135in}{3.123965in}}%
\pgfpathlineto{\pgfqpoint{6.508896in}{3.134390in}}%
\pgfpathlineto{\pgfqpoint{6.512375in}{3.135568in}}%
\pgfpathlineto{\pgfqpoint{6.515319in}{3.134244in}}%
\pgfpathlineto{\pgfqpoint{6.519333in}{3.129561in}}%
\pgfpathlineto{\pgfqpoint{6.530305in}{3.115291in}}%
\pgfpathlineto{\pgfqpoint{6.533516in}{3.115044in}}%
\pgfpathlineto{\pgfqpoint{6.536727in}{3.117278in}}%
\pgfpathlineto{\pgfqpoint{6.541544in}{3.123953in}}%
\pgfpathlineto{\pgfqpoint{6.549305in}{3.134385in}}%
\pgfpathlineto{\pgfqpoint{6.552784in}{3.135569in}}%
\pgfpathlineto{\pgfqpoint{6.555727in}{3.134250in}}%
\pgfpathlineto{\pgfqpoint{6.559742in}{3.129572in}}%
\pgfpathlineto{\pgfqpoint{6.570713in}{3.115295in}}%
\pgfpathlineto{\pgfqpoint{6.573925in}{3.115042in}}%
\pgfpathlineto{\pgfqpoint{6.577136in}{3.117270in}}%
\pgfpathlineto{\pgfqpoint{6.581953in}{3.123941in}}%
\pgfpathlineto{\pgfqpoint{6.589714in}{3.134379in}}%
\pgfpathlineto{\pgfqpoint{6.593193in}{3.135569in}}%
\pgfpathlineto{\pgfqpoint{6.596136in}{3.134256in}}%
\pgfpathlineto{\pgfqpoint{6.600150in}{3.129583in}}%
\pgfpathlineto{\pgfqpoint{6.611122in}{3.115298in}}%
\pgfpathlineto{\pgfqpoint{6.614334in}{3.115039in}}%
\pgfpathlineto{\pgfqpoint{6.617545in}{3.117263in}}%
\pgfpathlineto{\pgfqpoint{6.622362in}{3.123930in}}%
\pgfpathlineto{\pgfqpoint{6.630122in}{3.134374in}}%
\pgfpathlineto{\pgfqpoint{6.633601in}{3.135570in}}%
\pgfpathlineto{\pgfqpoint{6.636545in}{3.134262in}}%
\pgfpathlineto{\pgfqpoint{6.640559in}{3.129593in}}%
\pgfpathlineto{\pgfqpoint{6.651531in}{3.115302in}}%
\pgfpathlineto{\pgfqpoint{6.654742in}{3.115037in}}%
\pgfpathlineto{\pgfqpoint{6.657954in}{3.117255in}}%
\pgfpathlineto{\pgfqpoint{6.662771in}{3.123918in}}%
\pgfpathlineto{\pgfqpoint{6.663306in}{3.124778in}}%
\pgfpathlineto{\pgfqpoint{6.663306in}{3.124778in}}%
\pgfusepath{stroke}%
\end{pgfscope}%
\begin{pgfscope}%
\pgfpathrectangle{\pgfqpoint{0.467797in}{2.292089in}}{\pgfqpoint{6.490533in}{1.666241in}}%
\pgfusepath{clip}%
\pgfsetrectcap%
\pgfsetroundjoin%
\pgfsetlinewidth{1.505625pt}%
\definecolor{currentstroke}{rgb}{0.498039,0.498039,0.498039}%
\pgfsetstrokecolor{currentstroke}%
\pgfsetdash{}{0pt}%
\pgfpathmoveto{\pgfqpoint{0.762821in}{3.125209in}}%
\pgfpathlineto{\pgfqpoint{0.769511in}{3.134061in}}%
\pgfpathlineto{\pgfqpoint{0.772990in}{3.135293in}}%
\pgfpathlineto{\pgfqpoint{0.775934in}{3.133955in}}%
\pgfpathlineto{\pgfqpoint{0.779948in}{3.129198in}}%
\pgfpathlineto{\pgfqpoint{0.790385in}{3.115592in}}%
\pgfpathlineto{\pgfqpoint{0.793596in}{3.115318in}}%
\pgfpathlineto{\pgfqpoint{0.796807in}{3.117590in}}%
\pgfpathlineto{\pgfqpoint{0.801624in}{3.124347in}}%
\pgfpathlineto{\pgfqpoint{0.808849in}{3.134061in}}%
\pgfpathlineto{\pgfqpoint{0.812328in}{3.135293in}}%
\pgfpathlineto{\pgfqpoint{0.815272in}{3.133955in}}%
\pgfpathlineto{\pgfqpoint{0.819286in}{3.129198in}}%
\pgfpathlineto{\pgfqpoint{0.829723in}{3.115592in}}%
\pgfpathlineto{\pgfqpoint{0.832934in}{3.115318in}}%
\pgfpathlineto{\pgfqpoint{0.836145in}{3.117590in}}%
\pgfpathlineto{\pgfqpoint{0.840962in}{3.124347in}}%
\pgfpathlineto{\pgfqpoint{0.848188in}{3.134061in}}%
\pgfpathlineto{\pgfqpoint{0.851667in}{3.135293in}}%
\pgfpathlineto{\pgfqpoint{0.854610in}{3.133955in}}%
\pgfpathlineto{\pgfqpoint{0.858625in}{3.129198in}}%
\pgfpathlineto{\pgfqpoint{0.869061in}{3.115592in}}%
\pgfpathlineto{\pgfqpoint{0.872273in}{3.115318in}}%
\pgfpathlineto{\pgfqpoint{0.875484in}{3.117590in}}%
\pgfpathlineto{\pgfqpoint{0.880301in}{3.124347in}}%
\pgfpathlineto{\pgfqpoint{0.887526in}{3.134061in}}%
\pgfpathlineto{\pgfqpoint{0.891005in}{3.135293in}}%
\pgfpathlineto{\pgfqpoint{0.893949in}{3.133955in}}%
\pgfpathlineto{\pgfqpoint{0.897963in}{3.129198in}}%
\pgfpathlineto{\pgfqpoint{0.908400in}{3.115592in}}%
\pgfpathlineto{\pgfqpoint{0.911611in}{3.115318in}}%
\pgfpathlineto{\pgfqpoint{0.914822in}{3.117590in}}%
\pgfpathlineto{\pgfqpoint{0.919639in}{3.124347in}}%
\pgfpathlineto{\pgfqpoint{0.926865in}{3.134061in}}%
\pgfpathlineto{\pgfqpoint{0.930343in}{3.135293in}}%
\pgfpathlineto{\pgfqpoint{0.933287in}{3.133955in}}%
\pgfpathlineto{\pgfqpoint{0.937301in}{3.129198in}}%
\pgfpathlineto{\pgfqpoint{0.947738in}{3.115592in}}%
\pgfpathlineto{\pgfqpoint{0.950949in}{3.115318in}}%
\pgfpathlineto{\pgfqpoint{0.954161in}{3.117590in}}%
\pgfpathlineto{\pgfqpoint{0.958977in}{3.124347in}}%
\pgfpathlineto{\pgfqpoint{0.966203in}{3.134061in}}%
\pgfpathlineto{\pgfqpoint{0.969682in}{3.135293in}}%
\pgfpathlineto{\pgfqpoint{0.972625in}{3.133955in}}%
\pgfpathlineto{\pgfqpoint{0.976640in}{3.129198in}}%
\pgfpathlineto{\pgfqpoint{0.987076in}{3.115592in}}%
\pgfpathlineto{\pgfqpoint{0.990288in}{3.115318in}}%
\pgfpathlineto{\pgfqpoint{0.993499in}{3.117590in}}%
\pgfpathlineto{\pgfqpoint{0.998316in}{3.124347in}}%
\pgfpathlineto{\pgfqpoint{1.005541in}{3.134061in}}%
\pgfpathlineto{\pgfqpoint{1.009020in}{3.135293in}}%
\pgfpathlineto{\pgfqpoint{1.011964in}{3.133955in}}%
\pgfpathlineto{\pgfqpoint{1.015978in}{3.129198in}}%
\pgfpathlineto{\pgfqpoint{1.026415in}{3.115592in}}%
\pgfpathlineto{\pgfqpoint{1.029626in}{3.115318in}}%
\pgfpathlineto{\pgfqpoint{1.032837in}{3.117590in}}%
\pgfpathlineto{\pgfqpoint{1.037654in}{3.124347in}}%
\pgfpathlineto{\pgfqpoint{1.044880in}{3.134061in}}%
\pgfpathlineto{\pgfqpoint{1.048358in}{3.135293in}}%
\pgfpathlineto{\pgfqpoint{1.051302in}{3.133955in}}%
\pgfpathlineto{\pgfqpoint{1.055316in}{3.129198in}}%
\pgfpathlineto{\pgfqpoint{1.065753in}{3.115592in}}%
\pgfpathlineto{\pgfqpoint{1.068964in}{3.115318in}}%
\pgfpathlineto{\pgfqpoint{1.072176in}{3.117590in}}%
\pgfpathlineto{\pgfqpoint{1.076992in}{3.124347in}}%
\pgfpathlineto{\pgfqpoint{1.084218in}{3.134061in}}%
\pgfpathlineto{\pgfqpoint{1.087697in}{3.135293in}}%
\pgfpathlineto{\pgfqpoint{1.090640in}{3.133955in}}%
\pgfpathlineto{\pgfqpoint{1.094655in}{3.129198in}}%
\pgfpathlineto{\pgfqpoint{1.105091in}{3.115592in}}%
\pgfpathlineto{\pgfqpoint{1.108303in}{3.115318in}}%
\pgfpathlineto{\pgfqpoint{1.111514in}{3.117590in}}%
\pgfpathlineto{\pgfqpoint{1.116331in}{3.124347in}}%
\pgfpathlineto{\pgfqpoint{1.123556in}{3.134061in}}%
\pgfpathlineto{\pgfqpoint{1.127035in}{3.135293in}}%
\pgfpathlineto{\pgfqpoint{1.129979in}{3.133955in}}%
\pgfpathlineto{\pgfqpoint{1.133993in}{3.129198in}}%
\pgfpathlineto{\pgfqpoint{1.144430in}{3.115592in}}%
\pgfpathlineto{\pgfqpoint{1.147641in}{3.115318in}}%
\pgfpathlineto{\pgfqpoint{1.150852in}{3.117590in}}%
\pgfpathlineto{\pgfqpoint{1.155669in}{3.124347in}}%
\pgfpathlineto{\pgfqpoint{1.162895in}{3.134061in}}%
\pgfpathlineto{\pgfqpoint{1.166374in}{3.135293in}}%
\pgfpathlineto{\pgfqpoint{1.169317in}{3.133955in}}%
\pgfpathlineto{\pgfqpoint{1.173331in}{3.129198in}}%
\pgfpathlineto{\pgfqpoint{1.183768in}{3.115592in}}%
\pgfpathlineto{\pgfqpoint{1.186979in}{3.115318in}}%
\pgfpathlineto{\pgfqpoint{1.190191in}{3.117590in}}%
\pgfpathlineto{\pgfqpoint{1.195008in}{3.124347in}}%
\pgfpathlineto{\pgfqpoint{1.202233in}{3.134061in}}%
\pgfpathlineto{\pgfqpoint{1.205712in}{3.135293in}}%
\pgfpathlineto{\pgfqpoint{1.208656in}{3.133955in}}%
\pgfpathlineto{\pgfqpoint{1.212670in}{3.129198in}}%
\pgfpathlineto{\pgfqpoint{1.223106in}{3.115592in}}%
\pgfpathlineto{\pgfqpoint{1.226318in}{3.115318in}}%
\pgfpathlineto{\pgfqpoint{1.229529in}{3.117590in}}%
\pgfpathlineto{\pgfqpoint{1.234346in}{3.124347in}}%
\pgfpathlineto{\pgfqpoint{1.241571in}{3.134061in}}%
\pgfpathlineto{\pgfqpoint{1.245050in}{3.135293in}}%
\pgfpathlineto{\pgfqpoint{1.247994in}{3.133955in}}%
\pgfpathlineto{\pgfqpoint{1.252008in}{3.129198in}}%
\pgfpathlineto{\pgfqpoint{1.262445in}{3.115592in}}%
\pgfpathlineto{\pgfqpoint{1.265656in}{3.115318in}}%
\pgfpathlineto{\pgfqpoint{1.268867in}{3.117590in}}%
\pgfpathlineto{\pgfqpoint{1.273684in}{3.124347in}}%
\pgfpathlineto{\pgfqpoint{1.280910in}{3.134061in}}%
\pgfpathlineto{\pgfqpoint{1.284389in}{3.135293in}}%
\pgfpathlineto{\pgfqpoint{1.287332in}{3.133955in}}%
\pgfpathlineto{\pgfqpoint{1.291346in}{3.129198in}}%
\pgfpathlineto{\pgfqpoint{1.301783in}{3.115592in}}%
\pgfpathlineto{\pgfqpoint{1.304994in}{3.115318in}}%
\pgfpathlineto{\pgfqpoint{1.308206in}{3.117590in}}%
\pgfpathlineto{\pgfqpoint{1.313023in}{3.124347in}}%
\pgfpathlineto{\pgfqpoint{1.320248in}{3.134061in}}%
\pgfpathlineto{\pgfqpoint{1.323727in}{3.135293in}}%
\pgfpathlineto{\pgfqpoint{1.326671in}{3.133955in}}%
\pgfpathlineto{\pgfqpoint{1.330685in}{3.129198in}}%
\pgfpathlineto{\pgfqpoint{1.341121in}{3.115592in}}%
\pgfpathlineto{\pgfqpoint{1.344333in}{3.115318in}}%
\pgfpathlineto{\pgfqpoint{1.347544in}{3.117590in}}%
\pgfpathlineto{\pgfqpoint{1.352361in}{3.124347in}}%
\pgfpathlineto{\pgfqpoint{1.359586in}{3.134061in}}%
\pgfpathlineto{\pgfqpoint{1.363065in}{3.135293in}}%
\pgfpathlineto{\pgfqpoint{1.366009in}{3.133955in}}%
\pgfpathlineto{\pgfqpoint{1.370023in}{3.129198in}}%
\pgfpathlineto{\pgfqpoint{1.380460in}{3.115592in}}%
\pgfpathlineto{\pgfqpoint{1.383671in}{3.115318in}}%
\pgfpathlineto{\pgfqpoint{1.386882in}{3.117590in}}%
\pgfpathlineto{\pgfqpoint{1.391699in}{3.124347in}}%
\pgfpathlineto{\pgfqpoint{1.398925in}{3.134061in}}%
\pgfpathlineto{\pgfqpoint{1.402404in}{3.135293in}}%
\pgfpathlineto{\pgfqpoint{1.405347in}{3.133955in}}%
\pgfpathlineto{\pgfqpoint{1.409361in}{3.129198in}}%
\pgfpathlineto{\pgfqpoint{1.419798in}{3.115592in}}%
\pgfpathlineto{\pgfqpoint{1.423009in}{3.115318in}}%
\pgfpathlineto{\pgfqpoint{1.426221in}{3.117590in}}%
\pgfpathlineto{\pgfqpoint{1.431038in}{3.124347in}}%
\pgfpathlineto{\pgfqpoint{1.438263in}{3.134061in}}%
\pgfpathlineto{\pgfqpoint{1.441742in}{3.135293in}}%
\pgfpathlineto{\pgfqpoint{1.444686in}{3.133955in}}%
\pgfpathlineto{\pgfqpoint{1.448700in}{3.129198in}}%
\pgfpathlineto{\pgfqpoint{1.459136in}{3.115592in}}%
\pgfpathlineto{\pgfqpoint{1.462348in}{3.115318in}}%
\pgfpathlineto{\pgfqpoint{1.465559in}{3.117590in}}%
\pgfpathlineto{\pgfqpoint{1.470376in}{3.124347in}}%
\pgfpathlineto{\pgfqpoint{1.477601in}{3.134061in}}%
\pgfpathlineto{\pgfqpoint{1.481080in}{3.135293in}}%
\pgfpathlineto{\pgfqpoint{1.484024in}{3.133955in}}%
\pgfpathlineto{\pgfqpoint{1.488038in}{3.129198in}}%
\pgfpathlineto{\pgfqpoint{1.498475in}{3.115592in}}%
\pgfpathlineto{\pgfqpoint{1.501686in}{3.115318in}}%
\pgfpathlineto{\pgfqpoint{1.504897in}{3.117590in}}%
\pgfpathlineto{\pgfqpoint{1.509714in}{3.124347in}}%
\pgfpathlineto{\pgfqpoint{1.516940in}{3.134061in}}%
\pgfpathlineto{\pgfqpoint{1.520419in}{3.135293in}}%
\pgfpathlineto{\pgfqpoint{1.523362in}{3.133955in}}%
\pgfpathlineto{\pgfqpoint{1.527376in}{3.129198in}}%
\pgfpathlineto{\pgfqpoint{1.537813in}{3.115592in}}%
\pgfpathlineto{\pgfqpoint{1.541024in}{3.115318in}}%
\pgfpathlineto{\pgfqpoint{1.544236in}{3.117590in}}%
\pgfpathlineto{\pgfqpoint{1.549053in}{3.124347in}}%
\pgfpathlineto{\pgfqpoint{1.556278in}{3.134061in}}%
\pgfpathlineto{\pgfqpoint{1.559757in}{3.135293in}}%
\pgfpathlineto{\pgfqpoint{1.562701in}{3.133955in}}%
\pgfpathlineto{\pgfqpoint{1.566715in}{3.129198in}}%
\pgfpathlineto{\pgfqpoint{1.577152in}{3.115592in}}%
\pgfpathlineto{\pgfqpoint{1.580363in}{3.115318in}}%
\pgfpathlineto{\pgfqpoint{1.583574in}{3.117590in}}%
\pgfpathlineto{\pgfqpoint{1.588391in}{3.124347in}}%
\pgfpathlineto{\pgfqpoint{1.595616in}{3.134061in}}%
\pgfpathlineto{\pgfqpoint{1.599095in}{3.135293in}}%
\pgfpathlineto{\pgfqpoint{1.602039in}{3.133955in}}%
\pgfpathlineto{\pgfqpoint{1.606053in}{3.129198in}}%
\pgfpathlineto{\pgfqpoint{1.616490in}{3.115592in}}%
\pgfpathlineto{\pgfqpoint{1.619701in}{3.115318in}}%
\pgfpathlineto{\pgfqpoint{1.622912in}{3.117590in}}%
\pgfpathlineto{\pgfqpoint{1.627729in}{3.124347in}}%
\pgfpathlineto{\pgfqpoint{1.634955in}{3.134061in}}%
\pgfpathlineto{\pgfqpoint{1.638434in}{3.135293in}}%
\pgfpathlineto{\pgfqpoint{1.641377in}{3.133955in}}%
\pgfpathlineto{\pgfqpoint{1.645392in}{3.129198in}}%
\pgfpathlineto{\pgfqpoint{1.655828in}{3.115592in}}%
\pgfpathlineto{\pgfqpoint{1.659040in}{3.115318in}}%
\pgfpathlineto{\pgfqpoint{1.662251in}{3.117590in}}%
\pgfpathlineto{\pgfqpoint{1.667068in}{3.124347in}}%
\pgfpathlineto{\pgfqpoint{1.674293in}{3.134061in}}%
\pgfpathlineto{\pgfqpoint{1.677772in}{3.135293in}}%
\pgfpathlineto{\pgfqpoint{1.680716in}{3.133955in}}%
\pgfpathlineto{\pgfqpoint{1.684730in}{3.129198in}}%
\pgfpathlineto{\pgfqpoint{1.695167in}{3.115592in}}%
\pgfpathlineto{\pgfqpoint{1.698378in}{3.115318in}}%
\pgfpathlineto{\pgfqpoint{1.701589in}{3.117590in}}%
\pgfpathlineto{\pgfqpoint{1.706406in}{3.124347in}}%
\pgfpathlineto{\pgfqpoint{1.713631in}{3.134061in}}%
\pgfpathlineto{\pgfqpoint{1.717110in}{3.135293in}}%
\pgfpathlineto{\pgfqpoint{1.720054in}{3.133955in}}%
\pgfpathlineto{\pgfqpoint{1.724068in}{3.129198in}}%
\pgfpathlineto{\pgfqpoint{1.734505in}{3.115592in}}%
\pgfpathlineto{\pgfqpoint{1.737716in}{3.115318in}}%
\pgfpathlineto{\pgfqpoint{1.740927in}{3.117590in}}%
\pgfpathlineto{\pgfqpoint{1.745744in}{3.124347in}}%
\pgfpathlineto{\pgfqpoint{1.752970in}{3.134061in}}%
\pgfpathlineto{\pgfqpoint{1.756449in}{3.135293in}}%
\pgfpathlineto{\pgfqpoint{1.759392in}{3.133955in}}%
\pgfpathlineto{\pgfqpoint{1.763407in}{3.129198in}}%
\pgfpathlineto{\pgfqpoint{1.773843in}{3.115592in}}%
\pgfpathlineto{\pgfqpoint{1.777055in}{3.115318in}}%
\pgfpathlineto{\pgfqpoint{1.780266in}{3.117590in}}%
\pgfpathlineto{\pgfqpoint{1.785083in}{3.124347in}}%
\pgfpathlineto{\pgfqpoint{1.792308in}{3.134061in}}%
\pgfpathlineto{\pgfqpoint{1.795787in}{3.135293in}}%
\pgfpathlineto{\pgfqpoint{1.798731in}{3.133955in}}%
\pgfpathlineto{\pgfqpoint{1.802745in}{3.129198in}}%
\pgfpathlineto{\pgfqpoint{1.813182in}{3.115592in}}%
\pgfpathlineto{\pgfqpoint{1.816393in}{3.115318in}}%
\pgfpathlineto{\pgfqpoint{1.819604in}{3.117590in}}%
\pgfpathlineto{\pgfqpoint{1.824421in}{3.124347in}}%
\pgfpathlineto{\pgfqpoint{1.831647in}{3.134061in}}%
\pgfpathlineto{\pgfqpoint{1.835125in}{3.135293in}}%
\pgfpathlineto{\pgfqpoint{1.838069in}{3.133955in}}%
\pgfpathlineto{\pgfqpoint{1.842083in}{3.129198in}}%
\pgfpathlineto{\pgfqpoint{1.852520in}{3.115592in}}%
\pgfpathlineto{\pgfqpoint{1.855731in}{3.115318in}}%
\pgfpathlineto{\pgfqpoint{1.858943in}{3.117590in}}%
\pgfpathlineto{\pgfqpoint{1.863759in}{3.124347in}}%
\pgfpathlineto{\pgfqpoint{1.870985in}{3.134061in}}%
\pgfpathlineto{\pgfqpoint{1.874464in}{3.135293in}}%
\pgfpathlineto{\pgfqpoint{1.877407in}{3.133955in}}%
\pgfpathlineto{\pgfqpoint{1.881422in}{3.129198in}}%
\pgfpathlineto{\pgfqpoint{1.891858in}{3.115592in}}%
\pgfpathlineto{\pgfqpoint{1.895070in}{3.115318in}}%
\pgfpathlineto{\pgfqpoint{1.898281in}{3.117590in}}%
\pgfpathlineto{\pgfqpoint{1.903098in}{3.124347in}}%
\pgfpathlineto{\pgfqpoint{1.910323in}{3.134061in}}%
\pgfpathlineto{\pgfqpoint{1.913802in}{3.135293in}}%
\pgfpathlineto{\pgfqpoint{1.916746in}{3.133955in}}%
\pgfpathlineto{\pgfqpoint{1.920760in}{3.129198in}}%
\pgfpathlineto{\pgfqpoint{1.931197in}{3.115592in}}%
\pgfpathlineto{\pgfqpoint{1.934408in}{3.115318in}}%
\pgfpathlineto{\pgfqpoint{1.937619in}{3.117590in}}%
\pgfpathlineto{\pgfqpoint{1.942436in}{3.124347in}}%
\pgfpathlineto{\pgfqpoint{1.949662in}{3.134061in}}%
\pgfpathlineto{\pgfqpoint{1.953140in}{3.135293in}}%
\pgfpathlineto{\pgfqpoint{1.956084in}{3.133955in}}%
\pgfpathlineto{\pgfqpoint{1.960098in}{3.129198in}}%
\pgfpathlineto{\pgfqpoint{1.970535in}{3.115592in}}%
\pgfpathlineto{\pgfqpoint{1.973746in}{3.115318in}}%
\pgfpathlineto{\pgfqpoint{1.976958in}{3.117590in}}%
\pgfpathlineto{\pgfqpoint{1.981775in}{3.124347in}}%
\pgfpathlineto{\pgfqpoint{1.989000in}{3.134061in}}%
\pgfpathlineto{\pgfqpoint{1.992479in}{3.135293in}}%
\pgfpathlineto{\pgfqpoint{1.995423in}{3.133955in}}%
\pgfpathlineto{\pgfqpoint{1.999437in}{3.129198in}}%
\pgfpathlineto{\pgfqpoint{2.009873in}{3.115592in}}%
\pgfpathlineto{\pgfqpoint{2.013085in}{3.115318in}}%
\pgfpathlineto{\pgfqpoint{2.016296in}{3.117590in}}%
\pgfpathlineto{\pgfqpoint{2.021113in}{3.124347in}}%
\pgfpathlineto{\pgfqpoint{2.028338in}{3.134061in}}%
\pgfpathlineto{\pgfqpoint{2.031817in}{3.135293in}}%
\pgfpathlineto{\pgfqpoint{2.034761in}{3.133955in}}%
\pgfpathlineto{\pgfqpoint{2.038775in}{3.129198in}}%
\pgfpathlineto{\pgfqpoint{2.049212in}{3.115592in}}%
\pgfpathlineto{\pgfqpoint{2.052423in}{3.115318in}}%
\pgfpathlineto{\pgfqpoint{2.055634in}{3.117590in}}%
\pgfpathlineto{\pgfqpoint{2.060451in}{3.124347in}}%
\pgfpathlineto{\pgfqpoint{2.067677in}{3.134061in}}%
\pgfpathlineto{\pgfqpoint{2.071156in}{3.135293in}}%
\pgfpathlineto{\pgfqpoint{2.074099in}{3.133955in}}%
\pgfpathlineto{\pgfqpoint{2.078113in}{3.129198in}}%
\pgfpathlineto{\pgfqpoint{2.088550in}{3.115592in}}%
\pgfpathlineto{\pgfqpoint{2.091761in}{3.115318in}}%
\pgfpathlineto{\pgfqpoint{2.094973in}{3.117590in}}%
\pgfpathlineto{\pgfqpoint{2.099790in}{3.124347in}}%
\pgfpathlineto{\pgfqpoint{2.107015in}{3.134061in}}%
\pgfpathlineto{\pgfqpoint{2.110494in}{3.135293in}}%
\pgfpathlineto{\pgfqpoint{2.113438in}{3.133955in}}%
\pgfpathlineto{\pgfqpoint{2.117452in}{3.129198in}}%
\pgfpathlineto{\pgfqpoint{2.127888in}{3.115592in}}%
\pgfpathlineto{\pgfqpoint{2.131100in}{3.115318in}}%
\pgfpathlineto{\pgfqpoint{2.134311in}{3.117590in}}%
\pgfpathlineto{\pgfqpoint{2.139128in}{3.124347in}}%
\pgfpathlineto{\pgfqpoint{2.146353in}{3.134061in}}%
\pgfpathlineto{\pgfqpoint{2.149832in}{3.135293in}}%
\pgfpathlineto{\pgfqpoint{2.152776in}{3.133955in}}%
\pgfpathlineto{\pgfqpoint{2.156790in}{3.129198in}}%
\pgfpathlineto{\pgfqpoint{2.167227in}{3.115592in}}%
\pgfpathlineto{\pgfqpoint{2.170438in}{3.115318in}}%
\pgfpathlineto{\pgfqpoint{2.173649in}{3.117590in}}%
\pgfpathlineto{\pgfqpoint{2.178466in}{3.124347in}}%
\pgfpathlineto{\pgfqpoint{2.185692in}{3.134061in}}%
\pgfpathlineto{\pgfqpoint{2.189171in}{3.135293in}}%
\pgfpathlineto{\pgfqpoint{2.192114in}{3.133955in}}%
\pgfpathlineto{\pgfqpoint{2.196128in}{3.129198in}}%
\pgfpathlineto{\pgfqpoint{2.206565in}{3.115592in}}%
\pgfpathlineto{\pgfqpoint{2.209776in}{3.115318in}}%
\pgfpathlineto{\pgfqpoint{2.212988in}{3.117590in}}%
\pgfpathlineto{\pgfqpoint{2.217805in}{3.124347in}}%
\pgfpathlineto{\pgfqpoint{2.225030in}{3.134061in}}%
\pgfpathlineto{\pgfqpoint{2.228509in}{3.135293in}}%
\pgfpathlineto{\pgfqpoint{2.231453in}{3.133955in}}%
\pgfpathlineto{\pgfqpoint{2.235467in}{3.129198in}}%
\pgfpathlineto{\pgfqpoint{2.245903in}{3.115592in}}%
\pgfpathlineto{\pgfqpoint{2.249115in}{3.115318in}}%
\pgfpathlineto{\pgfqpoint{2.252326in}{3.117590in}}%
\pgfpathlineto{\pgfqpoint{2.257143in}{3.124347in}}%
\pgfpathlineto{\pgfqpoint{2.264368in}{3.134061in}}%
\pgfpathlineto{\pgfqpoint{2.267847in}{3.135293in}}%
\pgfpathlineto{\pgfqpoint{2.270791in}{3.133955in}}%
\pgfpathlineto{\pgfqpoint{2.274805in}{3.129198in}}%
\pgfpathlineto{\pgfqpoint{2.285242in}{3.115592in}}%
\pgfpathlineto{\pgfqpoint{2.288453in}{3.115318in}}%
\pgfpathlineto{\pgfqpoint{2.291664in}{3.117590in}}%
\pgfpathlineto{\pgfqpoint{2.296481in}{3.124347in}}%
\pgfpathlineto{\pgfqpoint{2.303707in}{3.134061in}}%
\pgfpathlineto{\pgfqpoint{2.307186in}{3.135293in}}%
\pgfpathlineto{\pgfqpoint{2.310129in}{3.133955in}}%
\pgfpathlineto{\pgfqpoint{2.314143in}{3.129198in}}%
\pgfpathlineto{\pgfqpoint{2.324580in}{3.115592in}}%
\pgfpathlineto{\pgfqpoint{2.327791in}{3.115318in}}%
\pgfpathlineto{\pgfqpoint{2.331003in}{3.117590in}}%
\pgfpathlineto{\pgfqpoint{2.335820in}{3.124347in}}%
\pgfpathlineto{\pgfqpoint{2.343045in}{3.134061in}}%
\pgfpathlineto{\pgfqpoint{2.346524in}{3.135293in}}%
\pgfpathlineto{\pgfqpoint{2.349468in}{3.133955in}}%
\pgfpathlineto{\pgfqpoint{2.353482in}{3.129198in}}%
\pgfpathlineto{\pgfqpoint{2.363919in}{3.115592in}}%
\pgfpathlineto{\pgfqpoint{2.367130in}{3.115318in}}%
\pgfpathlineto{\pgfqpoint{2.370341in}{3.117590in}}%
\pgfpathlineto{\pgfqpoint{2.375158in}{3.124347in}}%
\pgfpathlineto{\pgfqpoint{2.382383in}{3.134061in}}%
\pgfpathlineto{\pgfqpoint{2.385862in}{3.135293in}}%
\pgfpathlineto{\pgfqpoint{2.388806in}{3.133955in}}%
\pgfpathlineto{\pgfqpoint{2.392820in}{3.129198in}}%
\pgfpathlineto{\pgfqpoint{2.403257in}{3.115592in}}%
\pgfpathlineto{\pgfqpoint{2.406468in}{3.115318in}}%
\pgfpathlineto{\pgfqpoint{2.409679in}{3.117590in}}%
\pgfpathlineto{\pgfqpoint{2.414496in}{3.124347in}}%
\pgfpathlineto{\pgfqpoint{2.421722in}{3.134061in}}%
\pgfpathlineto{\pgfqpoint{2.425201in}{3.135293in}}%
\pgfpathlineto{\pgfqpoint{2.428144in}{3.133955in}}%
\pgfpathlineto{\pgfqpoint{2.432158in}{3.129198in}}%
\pgfpathlineto{\pgfqpoint{2.442595in}{3.115592in}}%
\pgfpathlineto{\pgfqpoint{2.445806in}{3.115318in}}%
\pgfpathlineto{\pgfqpoint{2.449018in}{3.117590in}}%
\pgfpathlineto{\pgfqpoint{2.453835in}{3.124347in}}%
\pgfpathlineto{\pgfqpoint{2.461060in}{3.134061in}}%
\pgfpathlineto{\pgfqpoint{2.464539in}{3.135293in}}%
\pgfpathlineto{\pgfqpoint{2.467483in}{3.133955in}}%
\pgfpathlineto{\pgfqpoint{2.471497in}{3.129198in}}%
\pgfpathlineto{\pgfqpoint{2.481934in}{3.115592in}}%
\pgfpathlineto{\pgfqpoint{2.485145in}{3.115318in}}%
\pgfpathlineto{\pgfqpoint{2.488356in}{3.117590in}}%
\pgfpathlineto{\pgfqpoint{2.493173in}{3.124347in}}%
\pgfpathlineto{\pgfqpoint{2.500398in}{3.134061in}}%
\pgfpathlineto{\pgfqpoint{2.503877in}{3.135293in}}%
\pgfpathlineto{\pgfqpoint{2.506821in}{3.133955in}}%
\pgfpathlineto{\pgfqpoint{2.510835in}{3.129198in}}%
\pgfpathlineto{\pgfqpoint{2.521272in}{3.115592in}}%
\pgfpathlineto{\pgfqpoint{2.524483in}{3.115318in}}%
\pgfpathlineto{\pgfqpoint{2.527694in}{3.117590in}}%
\pgfpathlineto{\pgfqpoint{2.532511in}{3.124347in}}%
\pgfpathlineto{\pgfqpoint{2.539737in}{3.134061in}}%
\pgfpathlineto{\pgfqpoint{2.543216in}{3.135293in}}%
\pgfpathlineto{\pgfqpoint{2.546159in}{3.133955in}}%
\pgfpathlineto{\pgfqpoint{2.550174in}{3.129198in}}%
\pgfpathlineto{\pgfqpoint{2.560610in}{3.115592in}}%
\pgfpathlineto{\pgfqpoint{2.563822in}{3.115318in}}%
\pgfpathlineto{\pgfqpoint{2.567033in}{3.117590in}}%
\pgfpathlineto{\pgfqpoint{2.571850in}{3.124347in}}%
\pgfpathlineto{\pgfqpoint{2.579075in}{3.134061in}}%
\pgfpathlineto{\pgfqpoint{2.582554in}{3.135293in}}%
\pgfpathlineto{\pgfqpoint{2.585498in}{3.133955in}}%
\pgfpathlineto{\pgfqpoint{2.589512in}{3.129198in}}%
\pgfpathlineto{\pgfqpoint{2.599949in}{3.115592in}}%
\pgfpathlineto{\pgfqpoint{2.603160in}{3.115318in}}%
\pgfpathlineto{\pgfqpoint{2.606371in}{3.117590in}}%
\pgfpathlineto{\pgfqpoint{2.611188in}{3.124347in}}%
\pgfpathlineto{\pgfqpoint{2.618414in}{3.134061in}}%
\pgfpathlineto{\pgfqpoint{2.621892in}{3.135293in}}%
\pgfpathlineto{\pgfqpoint{2.624836in}{3.133955in}}%
\pgfpathlineto{\pgfqpoint{2.628850in}{3.129198in}}%
\pgfpathlineto{\pgfqpoint{2.639287in}{3.115592in}}%
\pgfpathlineto{\pgfqpoint{2.642498in}{3.115318in}}%
\pgfpathlineto{\pgfqpoint{2.645710in}{3.117590in}}%
\pgfpathlineto{\pgfqpoint{2.650526in}{3.124347in}}%
\pgfpathlineto{\pgfqpoint{2.657752in}{3.134061in}}%
\pgfpathlineto{\pgfqpoint{2.661231in}{3.135293in}}%
\pgfpathlineto{\pgfqpoint{2.664174in}{3.133955in}}%
\pgfpathlineto{\pgfqpoint{2.668189in}{3.129198in}}%
\pgfpathlineto{\pgfqpoint{2.678625in}{3.115592in}}%
\pgfpathlineto{\pgfqpoint{2.681837in}{3.115318in}}%
\pgfpathlineto{\pgfqpoint{2.685048in}{3.117590in}}%
\pgfpathlineto{\pgfqpoint{2.689865in}{3.124347in}}%
\pgfpathlineto{\pgfqpoint{2.697090in}{3.134061in}}%
\pgfpathlineto{\pgfqpoint{2.700569in}{3.135293in}}%
\pgfpathlineto{\pgfqpoint{2.703513in}{3.133955in}}%
\pgfpathlineto{\pgfqpoint{2.707527in}{3.129198in}}%
\pgfpathlineto{\pgfqpoint{2.717964in}{3.115592in}}%
\pgfpathlineto{\pgfqpoint{2.721175in}{3.115318in}}%
\pgfpathlineto{\pgfqpoint{2.724386in}{3.117590in}}%
\pgfpathlineto{\pgfqpoint{2.729203in}{3.124347in}}%
\pgfpathlineto{\pgfqpoint{2.736429in}{3.134061in}}%
\pgfpathlineto{\pgfqpoint{2.739907in}{3.135293in}}%
\pgfpathlineto{\pgfqpoint{2.742851in}{3.133955in}}%
\pgfpathlineto{\pgfqpoint{2.746865in}{3.129198in}}%
\pgfpathlineto{\pgfqpoint{2.757302in}{3.115592in}}%
\pgfpathlineto{\pgfqpoint{2.760513in}{3.115318in}}%
\pgfpathlineto{\pgfqpoint{2.763725in}{3.117590in}}%
\pgfpathlineto{\pgfqpoint{2.768542in}{3.124347in}}%
\pgfpathlineto{\pgfqpoint{2.775767in}{3.134061in}}%
\pgfpathlineto{\pgfqpoint{2.779246in}{3.135293in}}%
\pgfpathlineto{\pgfqpoint{2.782190in}{3.133955in}}%
\pgfpathlineto{\pgfqpoint{2.786204in}{3.129198in}}%
\pgfpathlineto{\pgfqpoint{2.796640in}{3.115592in}}%
\pgfpathlineto{\pgfqpoint{2.799852in}{3.115318in}}%
\pgfpathlineto{\pgfqpoint{2.803063in}{3.117590in}}%
\pgfpathlineto{\pgfqpoint{2.807880in}{3.124347in}}%
\pgfpathlineto{\pgfqpoint{2.815105in}{3.134061in}}%
\pgfpathlineto{\pgfqpoint{2.818584in}{3.135293in}}%
\pgfpathlineto{\pgfqpoint{2.821528in}{3.133955in}}%
\pgfpathlineto{\pgfqpoint{2.825542in}{3.129198in}}%
\pgfpathlineto{\pgfqpoint{2.835979in}{3.115592in}}%
\pgfpathlineto{\pgfqpoint{2.839190in}{3.115318in}}%
\pgfpathlineto{\pgfqpoint{2.842401in}{3.117590in}}%
\pgfpathlineto{\pgfqpoint{2.847218in}{3.124347in}}%
\pgfpathlineto{\pgfqpoint{2.854444in}{3.134061in}}%
\pgfpathlineto{\pgfqpoint{2.857923in}{3.135293in}}%
\pgfpathlineto{\pgfqpoint{2.860866in}{3.133955in}}%
\pgfpathlineto{\pgfqpoint{2.864880in}{3.129198in}}%
\pgfpathlineto{\pgfqpoint{2.875317in}{3.115592in}}%
\pgfpathlineto{\pgfqpoint{2.878528in}{3.115318in}}%
\pgfpathlineto{\pgfqpoint{2.881740in}{3.117590in}}%
\pgfpathlineto{\pgfqpoint{2.886557in}{3.124347in}}%
\pgfpathlineto{\pgfqpoint{2.893782in}{3.134061in}}%
\pgfpathlineto{\pgfqpoint{2.897261in}{3.135293in}}%
\pgfpathlineto{\pgfqpoint{2.900205in}{3.133955in}}%
\pgfpathlineto{\pgfqpoint{2.904219in}{3.129198in}}%
\pgfpathlineto{\pgfqpoint{2.914655in}{3.115592in}}%
\pgfpathlineto{\pgfqpoint{2.917867in}{3.115318in}}%
\pgfpathlineto{\pgfqpoint{2.921078in}{3.117590in}}%
\pgfpathlineto{\pgfqpoint{2.925895in}{3.124347in}}%
\pgfpathlineto{\pgfqpoint{2.933120in}{3.134061in}}%
\pgfpathlineto{\pgfqpoint{2.936599in}{3.135293in}}%
\pgfpathlineto{\pgfqpoint{2.939543in}{3.133955in}}%
\pgfpathlineto{\pgfqpoint{2.943557in}{3.129198in}}%
\pgfpathlineto{\pgfqpoint{2.953994in}{3.115592in}}%
\pgfpathlineto{\pgfqpoint{2.957205in}{3.115318in}}%
\pgfpathlineto{\pgfqpoint{2.960416in}{3.117590in}}%
\pgfpathlineto{\pgfqpoint{2.965233in}{3.124347in}}%
\pgfpathlineto{\pgfqpoint{2.972459in}{3.134061in}}%
\pgfpathlineto{\pgfqpoint{2.975938in}{3.135293in}}%
\pgfpathlineto{\pgfqpoint{2.978881in}{3.133955in}}%
\pgfpathlineto{\pgfqpoint{2.982895in}{3.129198in}}%
\pgfpathlineto{\pgfqpoint{2.993332in}{3.115592in}}%
\pgfpathlineto{\pgfqpoint{2.996543in}{3.115318in}}%
\pgfpathlineto{\pgfqpoint{2.999755in}{3.117590in}}%
\pgfpathlineto{\pgfqpoint{3.004572in}{3.124347in}}%
\pgfpathlineto{\pgfqpoint{3.011797in}{3.134061in}}%
\pgfpathlineto{\pgfqpoint{3.015276in}{3.135293in}}%
\pgfpathlineto{\pgfqpoint{3.018220in}{3.133955in}}%
\pgfpathlineto{\pgfqpoint{3.022234in}{3.129198in}}%
\pgfpathlineto{\pgfqpoint{3.032670in}{3.115592in}}%
\pgfpathlineto{\pgfqpoint{3.035882in}{3.115318in}}%
\pgfpathlineto{\pgfqpoint{3.039093in}{3.117590in}}%
\pgfpathlineto{\pgfqpoint{3.043910in}{3.124347in}}%
\pgfpathlineto{\pgfqpoint{3.051135in}{3.134061in}}%
\pgfpathlineto{\pgfqpoint{3.054614in}{3.135293in}}%
\pgfpathlineto{\pgfqpoint{3.057558in}{3.133955in}}%
\pgfpathlineto{\pgfqpoint{3.061572in}{3.129198in}}%
\pgfpathlineto{\pgfqpoint{3.072009in}{3.115592in}}%
\pgfpathlineto{\pgfqpoint{3.075220in}{3.115318in}}%
\pgfpathlineto{\pgfqpoint{3.078431in}{3.117590in}}%
\pgfpathlineto{\pgfqpoint{3.083248in}{3.124347in}}%
\pgfpathlineto{\pgfqpoint{3.090474in}{3.134061in}}%
\pgfpathlineto{\pgfqpoint{3.093953in}{3.135293in}}%
\pgfpathlineto{\pgfqpoint{3.096896in}{3.133955in}}%
\pgfpathlineto{\pgfqpoint{3.100910in}{3.129198in}}%
\pgfpathlineto{\pgfqpoint{3.111347in}{3.115592in}}%
\pgfpathlineto{\pgfqpoint{3.114558in}{3.115318in}}%
\pgfpathlineto{\pgfqpoint{3.117770in}{3.117590in}}%
\pgfpathlineto{\pgfqpoint{3.122587in}{3.124347in}}%
\pgfpathlineto{\pgfqpoint{3.129812in}{3.134061in}}%
\pgfpathlineto{\pgfqpoint{3.133291in}{3.135293in}}%
\pgfpathlineto{\pgfqpoint{3.136235in}{3.133955in}}%
\pgfpathlineto{\pgfqpoint{3.140249in}{3.129198in}}%
\pgfpathlineto{\pgfqpoint{3.150685in}{3.115592in}}%
\pgfpathlineto{\pgfqpoint{3.153897in}{3.115318in}}%
\pgfpathlineto{\pgfqpoint{3.157108in}{3.117590in}}%
\pgfpathlineto{\pgfqpoint{3.161925in}{3.124347in}}%
\pgfpathlineto{\pgfqpoint{3.169150in}{3.134061in}}%
\pgfpathlineto{\pgfqpoint{3.172629in}{3.135293in}}%
\pgfpathlineto{\pgfqpoint{3.175573in}{3.133955in}}%
\pgfpathlineto{\pgfqpoint{3.179587in}{3.129198in}}%
\pgfpathlineto{\pgfqpoint{3.190024in}{3.115592in}}%
\pgfpathlineto{\pgfqpoint{3.193235in}{3.115318in}}%
\pgfpathlineto{\pgfqpoint{3.196446in}{3.117590in}}%
\pgfpathlineto{\pgfqpoint{3.201263in}{3.124347in}}%
\pgfpathlineto{\pgfqpoint{3.208489in}{3.134061in}}%
\pgfpathlineto{\pgfqpoint{3.211968in}{3.135293in}}%
\pgfpathlineto{\pgfqpoint{3.214911in}{3.133955in}}%
\pgfpathlineto{\pgfqpoint{3.218925in}{3.129198in}}%
\pgfpathlineto{\pgfqpoint{3.229362in}{3.115592in}}%
\pgfpathlineto{\pgfqpoint{3.232573in}{3.115318in}}%
\pgfpathlineto{\pgfqpoint{3.235785in}{3.117590in}}%
\pgfpathlineto{\pgfqpoint{3.240602in}{3.124347in}}%
\pgfpathlineto{\pgfqpoint{3.247827in}{3.134061in}}%
\pgfpathlineto{\pgfqpoint{3.251306in}{3.135293in}}%
\pgfpathlineto{\pgfqpoint{3.254250in}{3.133955in}}%
\pgfpathlineto{\pgfqpoint{3.258264in}{3.129198in}}%
\pgfpathlineto{\pgfqpoint{3.268701in}{3.115592in}}%
\pgfpathlineto{\pgfqpoint{3.271912in}{3.115318in}}%
\pgfpathlineto{\pgfqpoint{3.275123in}{3.117590in}}%
\pgfpathlineto{\pgfqpoint{3.279940in}{3.124347in}}%
\pgfpathlineto{\pgfqpoint{3.287165in}{3.134061in}}%
\pgfpathlineto{\pgfqpoint{3.290644in}{3.135293in}}%
\pgfpathlineto{\pgfqpoint{3.293588in}{3.133955in}}%
\pgfpathlineto{\pgfqpoint{3.297602in}{3.129198in}}%
\pgfpathlineto{\pgfqpoint{3.308039in}{3.115592in}}%
\pgfpathlineto{\pgfqpoint{3.311250in}{3.115318in}}%
\pgfpathlineto{\pgfqpoint{3.314461in}{3.117590in}}%
\pgfpathlineto{\pgfqpoint{3.319278in}{3.124347in}}%
\pgfpathlineto{\pgfqpoint{3.326504in}{3.134061in}}%
\pgfpathlineto{\pgfqpoint{3.329983in}{3.135293in}}%
\pgfpathlineto{\pgfqpoint{3.332926in}{3.133955in}}%
\pgfpathlineto{\pgfqpoint{3.336941in}{3.129198in}}%
\pgfpathlineto{\pgfqpoint{3.347377in}{3.115592in}}%
\pgfpathlineto{\pgfqpoint{3.350589in}{3.115318in}}%
\pgfpathlineto{\pgfqpoint{3.353800in}{3.117590in}}%
\pgfpathlineto{\pgfqpoint{3.358617in}{3.124347in}}%
\pgfpathlineto{\pgfqpoint{3.365842in}{3.134061in}}%
\pgfpathlineto{\pgfqpoint{3.369321in}{3.135293in}}%
\pgfpathlineto{\pgfqpoint{3.372265in}{3.133955in}}%
\pgfpathlineto{\pgfqpoint{3.376279in}{3.129198in}}%
\pgfpathlineto{\pgfqpoint{3.386716in}{3.115592in}}%
\pgfpathlineto{\pgfqpoint{3.389927in}{3.115318in}}%
\pgfpathlineto{\pgfqpoint{3.393138in}{3.117590in}}%
\pgfpathlineto{\pgfqpoint{3.397955in}{3.124347in}}%
\pgfpathlineto{\pgfqpoint{3.405181in}{3.134061in}}%
\pgfpathlineto{\pgfqpoint{3.408659in}{3.135293in}}%
\pgfpathlineto{\pgfqpoint{3.411603in}{3.133955in}}%
\pgfpathlineto{\pgfqpoint{3.415617in}{3.129198in}}%
\pgfpathlineto{\pgfqpoint{3.426054in}{3.115592in}}%
\pgfpathlineto{\pgfqpoint{3.429265in}{3.115318in}}%
\pgfpathlineto{\pgfqpoint{3.432477in}{3.117590in}}%
\pgfpathlineto{\pgfqpoint{3.437293in}{3.124347in}}%
\pgfpathlineto{\pgfqpoint{3.444519in}{3.134061in}}%
\pgfpathlineto{\pgfqpoint{3.447998in}{3.135293in}}%
\pgfpathlineto{\pgfqpoint{3.450941in}{3.133955in}}%
\pgfpathlineto{\pgfqpoint{3.454956in}{3.129198in}}%
\pgfpathlineto{\pgfqpoint{3.465392in}{3.115592in}}%
\pgfpathlineto{\pgfqpoint{3.468604in}{3.115318in}}%
\pgfpathlineto{\pgfqpoint{3.471815in}{3.117590in}}%
\pgfpathlineto{\pgfqpoint{3.476632in}{3.124347in}}%
\pgfpathlineto{\pgfqpoint{3.483857in}{3.134061in}}%
\pgfpathlineto{\pgfqpoint{3.487336in}{3.135293in}}%
\pgfpathlineto{\pgfqpoint{3.490280in}{3.133955in}}%
\pgfpathlineto{\pgfqpoint{3.494294in}{3.129198in}}%
\pgfpathlineto{\pgfqpoint{3.504731in}{3.115592in}}%
\pgfpathlineto{\pgfqpoint{3.507942in}{3.115318in}}%
\pgfpathlineto{\pgfqpoint{3.511153in}{3.117590in}}%
\pgfpathlineto{\pgfqpoint{3.515970in}{3.124347in}}%
\pgfpathlineto{\pgfqpoint{3.523196in}{3.134061in}}%
\pgfpathlineto{\pgfqpoint{3.526674in}{3.135293in}}%
\pgfpathlineto{\pgfqpoint{3.529618in}{3.133955in}}%
\pgfpathlineto{\pgfqpoint{3.533632in}{3.129198in}}%
\pgfpathlineto{\pgfqpoint{3.544069in}{3.115592in}}%
\pgfpathlineto{\pgfqpoint{3.547280in}{3.115318in}}%
\pgfpathlineto{\pgfqpoint{3.550492in}{3.117590in}}%
\pgfpathlineto{\pgfqpoint{3.555309in}{3.124347in}}%
\pgfpathlineto{\pgfqpoint{3.562534in}{3.134061in}}%
\pgfpathlineto{\pgfqpoint{3.566013in}{3.135293in}}%
\pgfpathlineto{\pgfqpoint{3.568957in}{3.133955in}}%
\pgfpathlineto{\pgfqpoint{3.572971in}{3.129198in}}%
\pgfpathlineto{\pgfqpoint{3.583407in}{3.115592in}}%
\pgfpathlineto{\pgfqpoint{3.586619in}{3.115318in}}%
\pgfpathlineto{\pgfqpoint{3.589830in}{3.117590in}}%
\pgfpathlineto{\pgfqpoint{3.594647in}{3.124347in}}%
\pgfpathlineto{\pgfqpoint{3.601872in}{3.134061in}}%
\pgfpathlineto{\pgfqpoint{3.605351in}{3.135293in}}%
\pgfpathlineto{\pgfqpoint{3.608295in}{3.133955in}}%
\pgfpathlineto{\pgfqpoint{3.612309in}{3.129198in}}%
\pgfpathlineto{\pgfqpoint{3.622746in}{3.115592in}}%
\pgfpathlineto{\pgfqpoint{3.625957in}{3.115318in}}%
\pgfpathlineto{\pgfqpoint{3.629168in}{3.117590in}}%
\pgfpathlineto{\pgfqpoint{3.633985in}{3.124347in}}%
\pgfpathlineto{\pgfqpoint{3.641211in}{3.134061in}}%
\pgfpathlineto{\pgfqpoint{3.644690in}{3.135293in}}%
\pgfpathlineto{\pgfqpoint{3.647633in}{3.133955in}}%
\pgfpathlineto{\pgfqpoint{3.651647in}{3.129198in}}%
\pgfpathlineto{\pgfqpoint{3.662084in}{3.115592in}}%
\pgfpathlineto{\pgfqpoint{3.665295in}{3.115318in}}%
\pgfpathlineto{\pgfqpoint{3.668507in}{3.117590in}}%
\pgfpathlineto{\pgfqpoint{3.673324in}{3.124347in}}%
\pgfpathlineto{\pgfqpoint{3.680549in}{3.134061in}}%
\pgfpathlineto{\pgfqpoint{3.684028in}{3.135293in}}%
\pgfpathlineto{\pgfqpoint{3.686972in}{3.133955in}}%
\pgfpathlineto{\pgfqpoint{3.690986in}{3.129198in}}%
\pgfpathlineto{\pgfqpoint{3.701422in}{3.115592in}}%
\pgfpathlineto{\pgfqpoint{3.704634in}{3.115318in}}%
\pgfpathlineto{\pgfqpoint{3.707845in}{3.117590in}}%
\pgfpathlineto{\pgfqpoint{3.712662in}{3.124347in}}%
\pgfpathlineto{\pgfqpoint{3.719887in}{3.134061in}}%
\pgfpathlineto{\pgfqpoint{3.723366in}{3.135293in}}%
\pgfpathlineto{\pgfqpoint{3.726310in}{3.133955in}}%
\pgfpathlineto{\pgfqpoint{3.730324in}{3.129198in}}%
\pgfpathlineto{\pgfqpoint{3.740761in}{3.115592in}}%
\pgfpathlineto{\pgfqpoint{3.743972in}{3.115318in}}%
\pgfpathlineto{\pgfqpoint{3.747183in}{3.117590in}}%
\pgfpathlineto{\pgfqpoint{3.752000in}{3.124347in}}%
\pgfpathlineto{\pgfqpoint{3.759226in}{3.134061in}}%
\pgfpathlineto{\pgfqpoint{3.762705in}{3.135293in}}%
\pgfpathlineto{\pgfqpoint{3.765648in}{3.133955in}}%
\pgfpathlineto{\pgfqpoint{3.769662in}{3.129198in}}%
\pgfpathlineto{\pgfqpoint{3.780099in}{3.115592in}}%
\pgfpathlineto{\pgfqpoint{3.783310in}{3.115318in}}%
\pgfpathlineto{\pgfqpoint{3.786522in}{3.117590in}}%
\pgfpathlineto{\pgfqpoint{3.791339in}{3.124347in}}%
\pgfpathlineto{\pgfqpoint{3.798564in}{3.134061in}}%
\pgfpathlineto{\pgfqpoint{3.802043in}{3.135293in}}%
\pgfpathlineto{\pgfqpoint{3.804987in}{3.133955in}}%
\pgfpathlineto{\pgfqpoint{3.809001in}{3.129198in}}%
\pgfpathlineto{\pgfqpoint{3.819437in}{3.115592in}}%
\pgfpathlineto{\pgfqpoint{3.822649in}{3.115318in}}%
\pgfpathlineto{\pgfqpoint{3.825860in}{3.117590in}}%
\pgfpathlineto{\pgfqpoint{3.830677in}{3.124347in}}%
\pgfpathlineto{\pgfqpoint{3.837902in}{3.134061in}}%
\pgfpathlineto{\pgfqpoint{3.841381in}{3.135293in}}%
\pgfpathlineto{\pgfqpoint{3.844325in}{3.133955in}}%
\pgfpathlineto{\pgfqpoint{3.848339in}{3.129198in}}%
\pgfpathlineto{\pgfqpoint{3.858776in}{3.115592in}}%
\pgfpathlineto{\pgfqpoint{3.861987in}{3.115318in}}%
\pgfpathlineto{\pgfqpoint{3.865198in}{3.117590in}}%
\pgfpathlineto{\pgfqpoint{3.870015in}{3.124347in}}%
\pgfpathlineto{\pgfqpoint{3.877241in}{3.134061in}}%
\pgfpathlineto{\pgfqpoint{3.880720in}{3.135293in}}%
\pgfpathlineto{\pgfqpoint{3.883663in}{3.133955in}}%
\pgfpathlineto{\pgfqpoint{3.887677in}{3.129198in}}%
\pgfpathlineto{\pgfqpoint{3.898114in}{3.115592in}}%
\pgfpathlineto{\pgfqpoint{3.901325in}{3.115318in}}%
\pgfpathlineto{\pgfqpoint{3.904537in}{3.117590in}}%
\pgfpathlineto{\pgfqpoint{3.909354in}{3.124347in}}%
\pgfpathlineto{\pgfqpoint{3.916579in}{3.134061in}}%
\pgfpathlineto{\pgfqpoint{3.920058in}{3.135293in}}%
\pgfpathlineto{\pgfqpoint{3.923002in}{3.133955in}}%
\pgfpathlineto{\pgfqpoint{3.927016in}{3.129198in}}%
\pgfpathlineto{\pgfqpoint{3.937452in}{3.115592in}}%
\pgfpathlineto{\pgfqpoint{3.940664in}{3.115318in}}%
\pgfpathlineto{\pgfqpoint{3.943875in}{3.117590in}}%
\pgfpathlineto{\pgfqpoint{3.948692in}{3.124347in}}%
\pgfpathlineto{\pgfqpoint{3.955917in}{3.134061in}}%
\pgfpathlineto{\pgfqpoint{3.959396in}{3.135293in}}%
\pgfpathlineto{\pgfqpoint{3.962340in}{3.133955in}}%
\pgfpathlineto{\pgfqpoint{3.966354in}{3.129198in}}%
\pgfpathlineto{\pgfqpoint{3.976791in}{3.115592in}}%
\pgfpathlineto{\pgfqpoint{3.980002in}{3.115318in}}%
\pgfpathlineto{\pgfqpoint{3.983213in}{3.117590in}}%
\pgfpathlineto{\pgfqpoint{3.988030in}{3.124347in}}%
\pgfpathlineto{\pgfqpoint{3.995256in}{3.134061in}}%
\pgfpathlineto{\pgfqpoint{3.998735in}{3.135293in}}%
\pgfpathlineto{\pgfqpoint{4.001678in}{3.133955in}}%
\pgfpathlineto{\pgfqpoint{4.005692in}{3.129198in}}%
\pgfpathlineto{\pgfqpoint{4.016129in}{3.115592in}}%
\pgfpathlineto{\pgfqpoint{4.019340in}{3.115318in}}%
\pgfpathlineto{\pgfqpoint{4.022552in}{3.117590in}}%
\pgfpathlineto{\pgfqpoint{4.027369in}{3.124347in}}%
\pgfpathlineto{\pgfqpoint{4.034594in}{3.134061in}}%
\pgfpathlineto{\pgfqpoint{4.038073in}{3.135293in}}%
\pgfpathlineto{\pgfqpoint{4.041017in}{3.133955in}}%
\pgfpathlineto{\pgfqpoint{4.045031in}{3.129198in}}%
\pgfpathlineto{\pgfqpoint{4.055468in}{3.115592in}}%
\pgfpathlineto{\pgfqpoint{4.058679in}{3.115318in}}%
\pgfpathlineto{\pgfqpoint{4.061890in}{3.117590in}}%
\pgfpathlineto{\pgfqpoint{4.066707in}{3.124347in}}%
\pgfpathlineto{\pgfqpoint{4.073932in}{3.134061in}}%
\pgfpathlineto{\pgfqpoint{4.077411in}{3.135293in}}%
\pgfpathlineto{\pgfqpoint{4.080355in}{3.133955in}}%
\pgfpathlineto{\pgfqpoint{4.084369in}{3.129198in}}%
\pgfpathlineto{\pgfqpoint{4.094806in}{3.115592in}}%
\pgfpathlineto{\pgfqpoint{4.098017in}{3.115318in}}%
\pgfpathlineto{\pgfqpoint{4.101228in}{3.117590in}}%
\pgfpathlineto{\pgfqpoint{4.106045in}{3.124347in}}%
\pgfpathlineto{\pgfqpoint{4.113271in}{3.134061in}}%
\pgfpathlineto{\pgfqpoint{4.116750in}{3.135293in}}%
\pgfpathlineto{\pgfqpoint{4.119693in}{3.133955in}}%
\pgfpathlineto{\pgfqpoint{4.123708in}{3.129198in}}%
\pgfpathlineto{\pgfqpoint{4.134144in}{3.115592in}}%
\pgfpathlineto{\pgfqpoint{4.137356in}{3.115318in}}%
\pgfpathlineto{\pgfqpoint{4.140567in}{3.117590in}}%
\pgfpathlineto{\pgfqpoint{4.145384in}{3.124347in}}%
\pgfpathlineto{\pgfqpoint{4.152609in}{3.134061in}}%
\pgfpathlineto{\pgfqpoint{4.156088in}{3.135293in}}%
\pgfpathlineto{\pgfqpoint{4.159032in}{3.133955in}}%
\pgfpathlineto{\pgfqpoint{4.163046in}{3.129198in}}%
\pgfpathlineto{\pgfqpoint{4.173483in}{3.115592in}}%
\pgfpathlineto{\pgfqpoint{4.176694in}{3.115318in}}%
\pgfpathlineto{\pgfqpoint{4.179905in}{3.117590in}}%
\pgfpathlineto{\pgfqpoint{4.184722in}{3.124347in}}%
\pgfpathlineto{\pgfqpoint{4.191948in}{3.134061in}}%
\pgfpathlineto{\pgfqpoint{4.195426in}{3.135293in}}%
\pgfpathlineto{\pgfqpoint{4.198370in}{3.133955in}}%
\pgfpathlineto{\pgfqpoint{4.202384in}{3.129198in}}%
\pgfpathlineto{\pgfqpoint{4.212821in}{3.115592in}}%
\pgfpathlineto{\pgfqpoint{4.216032in}{3.115318in}}%
\pgfpathlineto{\pgfqpoint{4.219244in}{3.117590in}}%
\pgfpathlineto{\pgfqpoint{4.224060in}{3.124347in}}%
\pgfpathlineto{\pgfqpoint{4.231286in}{3.134061in}}%
\pgfpathlineto{\pgfqpoint{4.234765in}{3.135293in}}%
\pgfpathlineto{\pgfqpoint{4.237708in}{3.133955in}}%
\pgfpathlineto{\pgfqpoint{4.241723in}{3.129198in}}%
\pgfpathlineto{\pgfqpoint{4.252159in}{3.115592in}}%
\pgfpathlineto{\pgfqpoint{4.255371in}{3.115318in}}%
\pgfpathlineto{\pgfqpoint{4.258582in}{3.117590in}}%
\pgfpathlineto{\pgfqpoint{4.263399in}{3.124347in}}%
\pgfpathlineto{\pgfqpoint{4.270624in}{3.134061in}}%
\pgfpathlineto{\pgfqpoint{4.274103in}{3.135293in}}%
\pgfpathlineto{\pgfqpoint{4.277047in}{3.133955in}}%
\pgfpathlineto{\pgfqpoint{4.281061in}{3.129198in}}%
\pgfpathlineto{\pgfqpoint{4.291498in}{3.115592in}}%
\pgfpathlineto{\pgfqpoint{4.294709in}{3.115318in}}%
\pgfpathlineto{\pgfqpoint{4.297920in}{3.117590in}}%
\pgfpathlineto{\pgfqpoint{4.302737in}{3.124347in}}%
\pgfpathlineto{\pgfqpoint{4.309963in}{3.134061in}}%
\pgfpathlineto{\pgfqpoint{4.313441in}{3.135293in}}%
\pgfpathlineto{\pgfqpoint{4.316385in}{3.133955in}}%
\pgfpathlineto{\pgfqpoint{4.320399in}{3.129198in}}%
\pgfpathlineto{\pgfqpoint{4.330836in}{3.115592in}}%
\pgfpathlineto{\pgfqpoint{4.334047in}{3.115318in}}%
\pgfpathlineto{\pgfqpoint{4.337259in}{3.117590in}}%
\pgfpathlineto{\pgfqpoint{4.342076in}{3.124347in}}%
\pgfpathlineto{\pgfqpoint{4.349301in}{3.134061in}}%
\pgfpathlineto{\pgfqpoint{4.352780in}{3.135293in}}%
\pgfpathlineto{\pgfqpoint{4.355724in}{3.133955in}}%
\pgfpathlineto{\pgfqpoint{4.359738in}{3.129198in}}%
\pgfpathlineto{\pgfqpoint{4.370174in}{3.115592in}}%
\pgfpathlineto{\pgfqpoint{4.373386in}{3.115318in}}%
\pgfpathlineto{\pgfqpoint{4.376597in}{3.117590in}}%
\pgfpathlineto{\pgfqpoint{4.381414in}{3.124347in}}%
\pgfpathlineto{\pgfqpoint{4.388639in}{3.134061in}}%
\pgfpathlineto{\pgfqpoint{4.392118in}{3.135293in}}%
\pgfpathlineto{\pgfqpoint{4.395062in}{3.133955in}}%
\pgfpathlineto{\pgfqpoint{4.399076in}{3.129198in}}%
\pgfpathlineto{\pgfqpoint{4.409513in}{3.115592in}}%
\pgfpathlineto{\pgfqpoint{4.412724in}{3.115318in}}%
\pgfpathlineto{\pgfqpoint{4.415935in}{3.117590in}}%
\pgfpathlineto{\pgfqpoint{4.420752in}{3.124347in}}%
\pgfpathlineto{\pgfqpoint{4.427978in}{3.134061in}}%
\pgfpathlineto{\pgfqpoint{4.431457in}{3.135293in}}%
\pgfpathlineto{\pgfqpoint{4.434400in}{3.133955in}}%
\pgfpathlineto{\pgfqpoint{4.438414in}{3.129198in}}%
\pgfpathlineto{\pgfqpoint{4.448851in}{3.115592in}}%
\pgfpathlineto{\pgfqpoint{4.452062in}{3.115318in}}%
\pgfpathlineto{\pgfqpoint{4.455274in}{3.117590in}}%
\pgfpathlineto{\pgfqpoint{4.460091in}{3.124347in}}%
\pgfpathlineto{\pgfqpoint{4.467316in}{3.134061in}}%
\pgfpathlineto{\pgfqpoint{4.470795in}{3.135293in}}%
\pgfpathlineto{\pgfqpoint{4.473739in}{3.133955in}}%
\pgfpathlineto{\pgfqpoint{4.477753in}{3.129198in}}%
\pgfpathlineto{\pgfqpoint{4.488189in}{3.115592in}}%
\pgfpathlineto{\pgfqpoint{4.491401in}{3.115318in}}%
\pgfpathlineto{\pgfqpoint{4.494612in}{3.117590in}}%
\pgfpathlineto{\pgfqpoint{4.499429in}{3.124347in}}%
\pgfpathlineto{\pgfqpoint{4.506654in}{3.134061in}}%
\pgfpathlineto{\pgfqpoint{4.510133in}{3.135293in}}%
\pgfpathlineto{\pgfqpoint{4.513077in}{3.133955in}}%
\pgfpathlineto{\pgfqpoint{4.517091in}{3.129198in}}%
\pgfpathlineto{\pgfqpoint{4.527528in}{3.115592in}}%
\pgfpathlineto{\pgfqpoint{4.530739in}{3.115318in}}%
\pgfpathlineto{\pgfqpoint{4.533950in}{3.117590in}}%
\pgfpathlineto{\pgfqpoint{4.538767in}{3.124347in}}%
\pgfpathlineto{\pgfqpoint{4.545993in}{3.134061in}}%
\pgfpathlineto{\pgfqpoint{4.549472in}{3.135293in}}%
\pgfpathlineto{\pgfqpoint{4.552415in}{3.133955in}}%
\pgfpathlineto{\pgfqpoint{4.556429in}{3.129198in}}%
\pgfpathlineto{\pgfqpoint{4.566866in}{3.115592in}}%
\pgfpathlineto{\pgfqpoint{4.570077in}{3.115318in}}%
\pgfpathlineto{\pgfqpoint{4.573289in}{3.117590in}}%
\pgfpathlineto{\pgfqpoint{4.578106in}{3.124347in}}%
\pgfpathlineto{\pgfqpoint{4.585331in}{3.134061in}}%
\pgfpathlineto{\pgfqpoint{4.588810in}{3.135293in}}%
\pgfpathlineto{\pgfqpoint{4.591754in}{3.133955in}}%
\pgfpathlineto{\pgfqpoint{4.595768in}{3.129198in}}%
\pgfpathlineto{\pgfqpoint{4.606204in}{3.115592in}}%
\pgfpathlineto{\pgfqpoint{4.609416in}{3.115318in}}%
\pgfpathlineto{\pgfqpoint{4.612627in}{3.117590in}}%
\pgfpathlineto{\pgfqpoint{4.617444in}{3.124347in}}%
\pgfpathlineto{\pgfqpoint{4.624669in}{3.134061in}}%
\pgfpathlineto{\pgfqpoint{4.628148in}{3.135293in}}%
\pgfpathlineto{\pgfqpoint{4.631092in}{3.133955in}}%
\pgfpathlineto{\pgfqpoint{4.635106in}{3.129198in}}%
\pgfpathlineto{\pgfqpoint{4.645543in}{3.115592in}}%
\pgfpathlineto{\pgfqpoint{4.648754in}{3.115318in}}%
\pgfpathlineto{\pgfqpoint{4.651965in}{3.117590in}}%
\pgfpathlineto{\pgfqpoint{4.656782in}{3.124347in}}%
\pgfpathlineto{\pgfqpoint{4.664008in}{3.134061in}}%
\pgfpathlineto{\pgfqpoint{4.667487in}{3.135293in}}%
\pgfpathlineto{\pgfqpoint{4.670430in}{3.133955in}}%
\pgfpathlineto{\pgfqpoint{4.674444in}{3.129198in}}%
\pgfpathlineto{\pgfqpoint{4.684881in}{3.115592in}}%
\pgfpathlineto{\pgfqpoint{4.688092in}{3.115318in}}%
\pgfpathlineto{\pgfqpoint{4.691304in}{3.117590in}}%
\pgfpathlineto{\pgfqpoint{4.696121in}{3.124347in}}%
\pgfpathlineto{\pgfqpoint{4.703346in}{3.134061in}}%
\pgfpathlineto{\pgfqpoint{4.706825in}{3.135293in}}%
\pgfpathlineto{\pgfqpoint{4.709769in}{3.133955in}}%
\pgfpathlineto{\pgfqpoint{4.713783in}{3.129198in}}%
\pgfpathlineto{\pgfqpoint{4.724219in}{3.115592in}}%
\pgfpathlineto{\pgfqpoint{4.727431in}{3.115318in}}%
\pgfpathlineto{\pgfqpoint{4.730642in}{3.117590in}}%
\pgfpathlineto{\pgfqpoint{4.735459in}{3.124347in}}%
\pgfpathlineto{\pgfqpoint{4.742684in}{3.134061in}}%
\pgfpathlineto{\pgfqpoint{4.746163in}{3.135293in}}%
\pgfpathlineto{\pgfqpoint{4.749107in}{3.133955in}}%
\pgfpathlineto{\pgfqpoint{4.753121in}{3.129198in}}%
\pgfpathlineto{\pgfqpoint{4.763558in}{3.115592in}}%
\pgfpathlineto{\pgfqpoint{4.766769in}{3.115318in}}%
\pgfpathlineto{\pgfqpoint{4.769980in}{3.117590in}}%
\pgfpathlineto{\pgfqpoint{4.774797in}{3.124347in}}%
\pgfpathlineto{\pgfqpoint{4.782023in}{3.134061in}}%
\pgfpathlineto{\pgfqpoint{4.785502in}{3.135293in}}%
\pgfpathlineto{\pgfqpoint{4.788445in}{3.133955in}}%
\pgfpathlineto{\pgfqpoint{4.792459in}{3.129198in}}%
\pgfpathlineto{\pgfqpoint{4.802896in}{3.115592in}}%
\pgfpathlineto{\pgfqpoint{4.806107in}{3.115318in}}%
\pgfpathlineto{\pgfqpoint{4.809319in}{3.117590in}}%
\pgfpathlineto{\pgfqpoint{4.814136in}{3.124347in}}%
\pgfpathlineto{\pgfqpoint{4.821361in}{3.134061in}}%
\pgfpathlineto{\pgfqpoint{4.824840in}{3.135293in}}%
\pgfpathlineto{\pgfqpoint{4.827784in}{3.133955in}}%
\pgfpathlineto{\pgfqpoint{4.831798in}{3.129198in}}%
\pgfpathlineto{\pgfqpoint{4.842235in}{3.115592in}}%
\pgfpathlineto{\pgfqpoint{4.845446in}{3.115318in}}%
\pgfpathlineto{\pgfqpoint{4.848657in}{3.117590in}}%
\pgfpathlineto{\pgfqpoint{4.853474in}{3.124347in}}%
\pgfpathlineto{\pgfqpoint{4.860699in}{3.134061in}}%
\pgfpathlineto{\pgfqpoint{4.864178in}{3.135293in}}%
\pgfpathlineto{\pgfqpoint{4.867122in}{3.133955in}}%
\pgfpathlineto{\pgfqpoint{4.871136in}{3.129198in}}%
\pgfpathlineto{\pgfqpoint{4.881573in}{3.115592in}}%
\pgfpathlineto{\pgfqpoint{4.884784in}{3.115318in}}%
\pgfpathlineto{\pgfqpoint{4.887995in}{3.117590in}}%
\pgfpathlineto{\pgfqpoint{4.892812in}{3.124347in}}%
\pgfpathlineto{\pgfqpoint{4.900038in}{3.134061in}}%
\pgfpathlineto{\pgfqpoint{4.903517in}{3.135293in}}%
\pgfpathlineto{\pgfqpoint{4.906460in}{3.133955in}}%
\pgfpathlineto{\pgfqpoint{4.910475in}{3.129198in}}%
\pgfpathlineto{\pgfqpoint{4.920911in}{3.115592in}}%
\pgfpathlineto{\pgfqpoint{4.924123in}{3.115318in}}%
\pgfpathlineto{\pgfqpoint{4.927334in}{3.117590in}}%
\pgfpathlineto{\pgfqpoint{4.932151in}{3.124347in}}%
\pgfpathlineto{\pgfqpoint{4.939376in}{3.134061in}}%
\pgfpathlineto{\pgfqpoint{4.942855in}{3.135293in}}%
\pgfpathlineto{\pgfqpoint{4.945799in}{3.133955in}}%
\pgfpathlineto{\pgfqpoint{4.949813in}{3.129198in}}%
\pgfpathlineto{\pgfqpoint{4.960250in}{3.115592in}}%
\pgfpathlineto{\pgfqpoint{4.963461in}{3.115318in}}%
\pgfpathlineto{\pgfqpoint{4.966672in}{3.117590in}}%
\pgfpathlineto{\pgfqpoint{4.971489in}{3.124347in}}%
\pgfpathlineto{\pgfqpoint{4.978715in}{3.134061in}}%
\pgfpathlineto{\pgfqpoint{4.982193in}{3.135293in}}%
\pgfpathlineto{\pgfqpoint{4.985137in}{3.133955in}}%
\pgfpathlineto{\pgfqpoint{4.989151in}{3.129198in}}%
\pgfpathlineto{\pgfqpoint{4.999588in}{3.115592in}}%
\pgfpathlineto{\pgfqpoint{5.002799in}{3.115318in}}%
\pgfpathlineto{\pgfqpoint{5.006011in}{3.117590in}}%
\pgfpathlineto{\pgfqpoint{5.010827in}{3.124347in}}%
\pgfpathlineto{\pgfqpoint{5.018053in}{3.134061in}}%
\pgfpathlineto{\pgfqpoint{5.021532in}{3.135293in}}%
\pgfpathlineto{\pgfqpoint{5.024475in}{3.133955in}}%
\pgfpathlineto{\pgfqpoint{5.028490in}{3.129198in}}%
\pgfpathlineto{\pgfqpoint{5.038926in}{3.115592in}}%
\pgfpathlineto{\pgfqpoint{5.042138in}{3.115318in}}%
\pgfpathlineto{\pgfqpoint{5.045349in}{3.117590in}}%
\pgfpathlineto{\pgfqpoint{5.050166in}{3.124347in}}%
\pgfpathlineto{\pgfqpoint{5.057391in}{3.134061in}}%
\pgfpathlineto{\pgfqpoint{5.060870in}{3.135293in}}%
\pgfpathlineto{\pgfqpoint{5.063814in}{3.133955in}}%
\pgfpathlineto{\pgfqpoint{5.067828in}{3.129198in}}%
\pgfpathlineto{\pgfqpoint{5.078265in}{3.115592in}}%
\pgfpathlineto{\pgfqpoint{5.081476in}{3.115318in}}%
\pgfpathlineto{\pgfqpoint{5.084687in}{3.117590in}}%
\pgfpathlineto{\pgfqpoint{5.089504in}{3.124347in}}%
\pgfpathlineto{\pgfqpoint{5.096730in}{3.134061in}}%
\pgfpathlineto{\pgfqpoint{5.100208in}{3.135293in}}%
\pgfpathlineto{\pgfqpoint{5.103152in}{3.133955in}}%
\pgfpathlineto{\pgfqpoint{5.107166in}{3.129198in}}%
\pgfpathlineto{\pgfqpoint{5.117603in}{3.115592in}}%
\pgfpathlineto{\pgfqpoint{5.120814in}{3.115318in}}%
\pgfpathlineto{\pgfqpoint{5.124026in}{3.117590in}}%
\pgfpathlineto{\pgfqpoint{5.128843in}{3.124347in}}%
\pgfpathlineto{\pgfqpoint{5.136068in}{3.134061in}}%
\pgfpathlineto{\pgfqpoint{5.139547in}{3.135293in}}%
\pgfpathlineto{\pgfqpoint{5.142491in}{3.133955in}}%
\pgfpathlineto{\pgfqpoint{5.146505in}{3.129198in}}%
\pgfpathlineto{\pgfqpoint{5.156941in}{3.115592in}}%
\pgfpathlineto{\pgfqpoint{5.160153in}{3.115318in}}%
\pgfpathlineto{\pgfqpoint{5.163364in}{3.117590in}}%
\pgfpathlineto{\pgfqpoint{5.168181in}{3.124347in}}%
\pgfpathlineto{\pgfqpoint{5.175406in}{3.134061in}}%
\pgfpathlineto{\pgfqpoint{5.178885in}{3.135293in}}%
\pgfpathlineto{\pgfqpoint{5.181829in}{3.133955in}}%
\pgfpathlineto{\pgfqpoint{5.185843in}{3.129198in}}%
\pgfpathlineto{\pgfqpoint{5.196280in}{3.115592in}}%
\pgfpathlineto{\pgfqpoint{5.199491in}{3.115318in}}%
\pgfpathlineto{\pgfqpoint{5.202702in}{3.117590in}}%
\pgfpathlineto{\pgfqpoint{5.207519in}{3.124347in}}%
\pgfpathlineto{\pgfqpoint{5.214745in}{3.134061in}}%
\pgfpathlineto{\pgfqpoint{5.218224in}{3.135293in}}%
\pgfpathlineto{\pgfqpoint{5.221167in}{3.133955in}}%
\pgfpathlineto{\pgfqpoint{5.225181in}{3.129198in}}%
\pgfpathlineto{\pgfqpoint{5.235618in}{3.115592in}}%
\pgfpathlineto{\pgfqpoint{5.238829in}{3.115318in}}%
\pgfpathlineto{\pgfqpoint{5.242041in}{3.117590in}}%
\pgfpathlineto{\pgfqpoint{5.246858in}{3.124347in}}%
\pgfpathlineto{\pgfqpoint{5.254083in}{3.134061in}}%
\pgfpathlineto{\pgfqpoint{5.257562in}{3.135293in}}%
\pgfpathlineto{\pgfqpoint{5.260506in}{3.133955in}}%
\pgfpathlineto{\pgfqpoint{5.264520in}{3.129198in}}%
\pgfpathlineto{\pgfqpoint{5.274956in}{3.115592in}}%
\pgfpathlineto{\pgfqpoint{5.278168in}{3.115318in}}%
\pgfpathlineto{\pgfqpoint{5.281379in}{3.117590in}}%
\pgfpathlineto{\pgfqpoint{5.286196in}{3.124347in}}%
\pgfpathlineto{\pgfqpoint{5.293421in}{3.134061in}}%
\pgfpathlineto{\pgfqpoint{5.296900in}{3.135293in}}%
\pgfpathlineto{\pgfqpoint{5.299844in}{3.133955in}}%
\pgfpathlineto{\pgfqpoint{5.303858in}{3.129198in}}%
\pgfpathlineto{\pgfqpoint{5.314295in}{3.115592in}}%
\pgfpathlineto{\pgfqpoint{5.317506in}{3.115318in}}%
\pgfpathlineto{\pgfqpoint{5.320717in}{3.117590in}}%
\pgfpathlineto{\pgfqpoint{5.325534in}{3.124347in}}%
\pgfpathlineto{\pgfqpoint{5.332760in}{3.134061in}}%
\pgfpathlineto{\pgfqpoint{5.336239in}{3.135293in}}%
\pgfpathlineto{\pgfqpoint{5.339182in}{3.133955in}}%
\pgfpathlineto{\pgfqpoint{5.343196in}{3.129198in}}%
\pgfpathlineto{\pgfqpoint{5.353633in}{3.115592in}}%
\pgfpathlineto{\pgfqpoint{5.356844in}{3.115318in}}%
\pgfpathlineto{\pgfqpoint{5.360056in}{3.117590in}}%
\pgfpathlineto{\pgfqpoint{5.364873in}{3.124347in}}%
\pgfpathlineto{\pgfqpoint{5.372098in}{3.134061in}}%
\pgfpathlineto{\pgfqpoint{5.375577in}{3.135293in}}%
\pgfpathlineto{\pgfqpoint{5.378521in}{3.133955in}}%
\pgfpathlineto{\pgfqpoint{5.382535in}{3.129198in}}%
\pgfpathlineto{\pgfqpoint{5.392971in}{3.115592in}}%
\pgfpathlineto{\pgfqpoint{5.396183in}{3.115318in}}%
\pgfpathlineto{\pgfqpoint{5.399394in}{3.117590in}}%
\pgfpathlineto{\pgfqpoint{5.404211in}{3.124347in}}%
\pgfpathlineto{\pgfqpoint{5.411436in}{3.134061in}}%
\pgfpathlineto{\pgfqpoint{5.414915in}{3.135293in}}%
\pgfpathlineto{\pgfqpoint{5.417859in}{3.133955in}}%
\pgfpathlineto{\pgfqpoint{5.421873in}{3.129198in}}%
\pgfpathlineto{\pgfqpoint{5.432310in}{3.115592in}}%
\pgfpathlineto{\pgfqpoint{5.435521in}{3.115318in}}%
\pgfpathlineto{\pgfqpoint{5.438732in}{3.117590in}}%
\pgfpathlineto{\pgfqpoint{5.443549in}{3.124347in}}%
\pgfpathlineto{\pgfqpoint{5.450775in}{3.134061in}}%
\pgfpathlineto{\pgfqpoint{5.454254in}{3.135293in}}%
\pgfpathlineto{\pgfqpoint{5.457197in}{3.133955in}}%
\pgfpathlineto{\pgfqpoint{5.461211in}{3.129198in}}%
\pgfpathlineto{\pgfqpoint{5.471648in}{3.115592in}}%
\pgfpathlineto{\pgfqpoint{5.474859in}{3.115318in}}%
\pgfpathlineto{\pgfqpoint{5.478071in}{3.117590in}}%
\pgfpathlineto{\pgfqpoint{5.482888in}{3.124347in}}%
\pgfpathlineto{\pgfqpoint{5.490113in}{3.134061in}}%
\pgfpathlineto{\pgfqpoint{5.493592in}{3.135293in}}%
\pgfpathlineto{\pgfqpoint{5.496536in}{3.133955in}}%
\pgfpathlineto{\pgfqpoint{5.500550in}{3.129198in}}%
\pgfpathlineto{\pgfqpoint{5.510986in}{3.115592in}}%
\pgfpathlineto{\pgfqpoint{5.514198in}{3.115318in}}%
\pgfpathlineto{\pgfqpoint{5.517409in}{3.117590in}}%
\pgfpathlineto{\pgfqpoint{5.522226in}{3.124347in}}%
\pgfpathlineto{\pgfqpoint{5.529451in}{3.134061in}}%
\pgfpathlineto{\pgfqpoint{5.532930in}{3.135293in}}%
\pgfpathlineto{\pgfqpoint{5.535874in}{3.133955in}}%
\pgfpathlineto{\pgfqpoint{5.539888in}{3.129198in}}%
\pgfpathlineto{\pgfqpoint{5.550325in}{3.115592in}}%
\pgfpathlineto{\pgfqpoint{5.553536in}{3.115318in}}%
\pgfpathlineto{\pgfqpoint{5.556747in}{3.117590in}}%
\pgfpathlineto{\pgfqpoint{5.561564in}{3.124347in}}%
\pgfpathlineto{\pgfqpoint{5.568790in}{3.134061in}}%
\pgfpathlineto{\pgfqpoint{5.572269in}{3.135293in}}%
\pgfpathlineto{\pgfqpoint{5.575212in}{3.133955in}}%
\pgfpathlineto{\pgfqpoint{5.579226in}{3.129198in}}%
\pgfpathlineto{\pgfqpoint{5.589663in}{3.115592in}}%
\pgfpathlineto{\pgfqpoint{5.592874in}{3.115318in}}%
\pgfpathlineto{\pgfqpoint{5.596086in}{3.117590in}}%
\pgfpathlineto{\pgfqpoint{5.600903in}{3.124347in}}%
\pgfpathlineto{\pgfqpoint{5.608128in}{3.134061in}}%
\pgfpathlineto{\pgfqpoint{5.611607in}{3.135293in}}%
\pgfpathlineto{\pgfqpoint{5.614551in}{3.133955in}}%
\pgfpathlineto{\pgfqpoint{5.618565in}{3.129198in}}%
\pgfpathlineto{\pgfqpoint{5.629002in}{3.115592in}}%
\pgfpathlineto{\pgfqpoint{5.632213in}{3.115318in}}%
\pgfpathlineto{\pgfqpoint{5.635424in}{3.117590in}}%
\pgfpathlineto{\pgfqpoint{5.640241in}{3.124347in}}%
\pgfpathlineto{\pgfqpoint{5.647466in}{3.134061in}}%
\pgfpathlineto{\pgfqpoint{5.650945in}{3.135293in}}%
\pgfpathlineto{\pgfqpoint{5.653889in}{3.133955in}}%
\pgfpathlineto{\pgfqpoint{5.657903in}{3.129198in}}%
\pgfpathlineto{\pgfqpoint{5.668340in}{3.115592in}}%
\pgfpathlineto{\pgfqpoint{5.671551in}{3.115318in}}%
\pgfpathlineto{\pgfqpoint{5.674762in}{3.117590in}}%
\pgfpathlineto{\pgfqpoint{5.679579in}{3.124347in}}%
\pgfpathlineto{\pgfqpoint{5.686805in}{3.134061in}}%
\pgfpathlineto{\pgfqpoint{5.690284in}{3.135293in}}%
\pgfpathlineto{\pgfqpoint{5.693227in}{3.133955in}}%
\pgfpathlineto{\pgfqpoint{5.697242in}{3.129198in}}%
\pgfpathlineto{\pgfqpoint{5.707678in}{3.115592in}}%
\pgfpathlineto{\pgfqpoint{5.710890in}{3.115318in}}%
\pgfpathlineto{\pgfqpoint{5.714101in}{3.117590in}}%
\pgfpathlineto{\pgfqpoint{5.718918in}{3.124347in}}%
\pgfpathlineto{\pgfqpoint{5.726143in}{3.134061in}}%
\pgfpathlineto{\pgfqpoint{5.729622in}{3.135293in}}%
\pgfpathlineto{\pgfqpoint{5.732566in}{3.133955in}}%
\pgfpathlineto{\pgfqpoint{5.736580in}{3.129198in}}%
\pgfpathlineto{\pgfqpoint{5.747017in}{3.115592in}}%
\pgfpathlineto{\pgfqpoint{5.750228in}{3.115318in}}%
\pgfpathlineto{\pgfqpoint{5.753439in}{3.117590in}}%
\pgfpathlineto{\pgfqpoint{5.758256in}{3.124347in}}%
\pgfpathlineto{\pgfqpoint{5.765482in}{3.134061in}}%
\pgfpathlineto{\pgfqpoint{5.768960in}{3.135293in}}%
\pgfpathlineto{\pgfqpoint{5.771904in}{3.133955in}}%
\pgfpathlineto{\pgfqpoint{5.775918in}{3.129198in}}%
\pgfpathlineto{\pgfqpoint{5.786355in}{3.115592in}}%
\pgfpathlineto{\pgfqpoint{5.789566in}{3.115318in}}%
\pgfpathlineto{\pgfqpoint{5.792778in}{3.117590in}}%
\pgfpathlineto{\pgfqpoint{5.797594in}{3.124347in}}%
\pgfpathlineto{\pgfqpoint{5.804820in}{3.134061in}}%
\pgfpathlineto{\pgfqpoint{5.808299in}{3.135293in}}%
\pgfpathlineto{\pgfqpoint{5.811242in}{3.133955in}}%
\pgfpathlineto{\pgfqpoint{5.815257in}{3.129198in}}%
\pgfpathlineto{\pgfqpoint{5.825693in}{3.115592in}}%
\pgfpathlineto{\pgfqpoint{5.828905in}{3.115318in}}%
\pgfpathlineto{\pgfqpoint{5.832116in}{3.117590in}}%
\pgfpathlineto{\pgfqpoint{5.836933in}{3.124347in}}%
\pgfpathlineto{\pgfqpoint{5.844158in}{3.134061in}}%
\pgfpathlineto{\pgfqpoint{5.847637in}{3.135293in}}%
\pgfpathlineto{\pgfqpoint{5.850581in}{3.133955in}}%
\pgfpathlineto{\pgfqpoint{5.854595in}{3.129198in}}%
\pgfpathlineto{\pgfqpoint{5.865032in}{3.115592in}}%
\pgfpathlineto{\pgfqpoint{5.868243in}{3.115318in}}%
\pgfpathlineto{\pgfqpoint{5.871454in}{3.117590in}}%
\pgfpathlineto{\pgfqpoint{5.876271in}{3.124347in}}%
\pgfpathlineto{\pgfqpoint{5.883497in}{3.134061in}}%
\pgfpathlineto{\pgfqpoint{5.886975in}{3.135293in}}%
\pgfpathlineto{\pgfqpoint{5.889919in}{3.133955in}}%
\pgfpathlineto{\pgfqpoint{5.893933in}{3.129198in}}%
\pgfpathlineto{\pgfqpoint{5.904370in}{3.115592in}}%
\pgfpathlineto{\pgfqpoint{5.907581in}{3.115318in}}%
\pgfpathlineto{\pgfqpoint{5.910793in}{3.117590in}}%
\pgfpathlineto{\pgfqpoint{5.915610in}{3.124347in}}%
\pgfpathlineto{\pgfqpoint{5.922835in}{3.134061in}}%
\pgfpathlineto{\pgfqpoint{5.926314in}{3.135293in}}%
\pgfpathlineto{\pgfqpoint{5.929258in}{3.133955in}}%
\pgfpathlineto{\pgfqpoint{5.933272in}{3.129198in}}%
\pgfpathlineto{\pgfqpoint{5.943708in}{3.115592in}}%
\pgfpathlineto{\pgfqpoint{5.946920in}{3.115318in}}%
\pgfpathlineto{\pgfqpoint{5.950131in}{3.117590in}}%
\pgfpathlineto{\pgfqpoint{5.954948in}{3.124347in}}%
\pgfpathlineto{\pgfqpoint{5.962173in}{3.134061in}}%
\pgfpathlineto{\pgfqpoint{5.965652in}{3.135293in}}%
\pgfpathlineto{\pgfqpoint{5.968596in}{3.133955in}}%
\pgfpathlineto{\pgfqpoint{5.972610in}{3.129198in}}%
\pgfpathlineto{\pgfqpoint{5.983047in}{3.115592in}}%
\pgfpathlineto{\pgfqpoint{5.986258in}{3.115318in}}%
\pgfpathlineto{\pgfqpoint{5.989469in}{3.117590in}}%
\pgfpathlineto{\pgfqpoint{5.994286in}{3.124347in}}%
\pgfpathlineto{\pgfqpoint{6.001512in}{3.134061in}}%
\pgfpathlineto{\pgfqpoint{6.004991in}{3.135293in}}%
\pgfpathlineto{\pgfqpoint{6.007934in}{3.133955in}}%
\pgfpathlineto{\pgfqpoint{6.011948in}{3.129198in}}%
\pgfpathlineto{\pgfqpoint{6.022385in}{3.115592in}}%
\pgfpathlineto{\pgfqpoint{6.025596in}{3.115318in}}%
\pgfpathlineto{\pgfqpoint{6.028808in}{3.117590in}}%
\pgfpathlineto{\pgfqpoint{6.033625in}{3.124347in}}%
\pgfpathlineto{\pgfqpoint{6.040850in}{3.134061in}}%
\pgfpathlineto{\pgfqpoint{6.044329in}{3.135293in}}%
\pgfpathlineto{\pgfqpoint{6.047273in}{3.133955in}}%
\pgfpathlineto{\pgfqpoint{6.051287in}{3.129198in}}%
\pgfpathlineto{\pgfqpoint{6.061723in}{3.115592in}}%
\pgfpathlineto{\pgfqpoint{6.064935in}{3.115318in}}%
\pgfpathlineto{\pgfqpoint{6.068146in}{3.117590in}}%
\pgfpathlineto{\pgfqpoint{6.072963in}{3.124347in}}%
\pgfpathlineto{\pgfqpoint{6.080188in}{3.134061in}}%
\pgfpathlineto{\pgfqpoint{6.083667in}{3.135293in}}%
\pgfpathlineto{\pgfqpoint{6.086611in}{3.133955in}}%
\pgfpathlineto{\pgfqpoint{6.090625in}{3.129198in}}%
\pgfpathlineto{\pgfqpoint{6.101062in}{3.115592in}}%
\pgfpathlineto{\pgfqpoint{6.104273in}{3.115318in}}%
\pgfpathlineto{\pgfqpoint{6.107484in}{3.117590in}}%
\pgfpathlineto{\pgfqpoint{6.112301in}{3.124347in}}%
\pgfpathlineto{\pgfqpoint{6.119527in}{3.134061in}}%
\pgfpathlineto{\pgfqpoint{6.123006in}{3.135293in}}%
\pgfpathlineto{\pgfqpoint{6.125949in}{3.133955in}}%
\pgfpathlineto{\pgfqpoint{6.129963in}{3.129198in}}%
\pgfpathlineto{\pgfqpoint{6.140400in}{3.115592in}}%
\pgfpathlineto{\pgfqpoint{6.143611in}{3.115318in}}%
\pgfpathlineto{\pgfqpoint{6.146823in}{3.117590in}}%
\pgfpathlineto{\pgfqpoint{6.151640in}{3.124347in}}%
\pgfpathlineto{\pgfqpoint{6.158865in}{3.134061in}}%
\pgfpathlineto{\pgfqpoint{6.162344in}{3.135293in}}%
\pgfpathlineto{\pgfqpoint{6.165288in}{3.133955in}}%
\pgfpathlineto{\pgfqpoint{6.169302in}{3.129198in}}%
\pgfpathlineto{\pgfqpoint{6.179738in}{3.115592in}}%
\pgfpathlineto{\pgfqpoint{6.182950in}{3.115318in}}%
\pgfpathlineto{\pgfqpoint{6.186161in}{3.117590in}}%
\pgfpathlineto{\pgfqpoint{6.190978in}{3.124347in}}%
\pgfpathlineto{\pgfqpoint{6.198203in}{3.134061in}}%
\pgfpathlineto{\pgfqpoint{6.201682in}{3.135293in}}%
\pgfpathlineto{\pgfqpoint{6.204626in}{3.133955in}}%
\pgfpathlineto{\pgfqpoint{6.208640in}{3.129198in}}%
\pgfpathlineto{\pgfqpoint{6.219077in}{3.115592in}}%
\pgfpathlineto{\pgfqpoint{6.222288in}{3.115318in}}%
\pgfpathlineto{\pgfqpoint{6.225499in}{3.117590in}}%
\pgfpathlineto{\pgfqpoint{6.230316in}{3.124347in}}%
\pgfpathlineto{\pgfqpoint{6.237542in}{3.134061in}}%
\pgfpathlineto{\pgfqpoint{6.241021in}{3.135293in}}%
\pgfpathlineto{\pgfqpoint{6.243964in}{3.133955in}}%
\pgfpathlineto{\pgfqpoint{6.247978in}{3.129198in}}%
\pgfpathlineto{\pgfqpoint{6.258415in}{3.115592in}}%
\pgfpathlineto{\pgfqpoint{6.261626in}{3.115318in}}%
\pgfpathlineto{\pgfqpoint{6.264838in}{3.117590in}}%
\pgfpathlineto{\pgfqpoint{6.269655in}{3.124347in}}%
\pgfpathlineto{\pgfqpoint{6.276880in}{3.134061in}}%
\pgfpathlineto{\pgfqpoint{6.280359in}{3.135293in}}%
\pgfpathlineto{\pgfqpoint{6.283303in}{3.133955in}}%
\pgfpathlineto{\pgfqpoint{6.287317in}{3.129198in}}%
\pgfpathlineto{\pgfqpoint{6.297753in}{3.115592in}}%
\pgfpathlineto{\pgfqpoint{6.300965in}{3.115318in}}%
\pgfpathlineto{\pgfqpoint{6.304176in}{3.117590in}}%
\pgfpathlineto{\pgfqpoint{6.308993in}{3.124347in}}%
\pgfpathlineto{\pgfqpoint{6.316218in}{3.134061in}}%
\pgfpathlineto{\pgfqpoint{6.319697in}{3.135293in}}%
\pgfpathlineto{\pgfqpoint{6.322641in}{3.133955in}}%
\pgfpathlineto{\pgfqpoint{6.326655in}{3.129198in}}%
\pgfpathlineto{\pgfqpoint{6.337092in}{3.115592in}}%
\pgfpathlineto{\pgfqpoint{6.340303in}{3.115318in}}%
\pgfpathlineto{\pgfqpoint{6.343514in}{3.117590in}}%
\pgfpathlineto{\pgfqpoint{6.348331in}{3.124347in}}%
\pgfpathlineto{\pgfqpoint{6.355557in}{3.134061in}}%
\pgfpathlineto{\pgfqpoint{6.359036in}{3.135293in}}%
\pgfpathlineto{\pgfqpoint{6.361979in}{3.133955in}}%
\pgfpathlineto{\pgfqpoint{6.365993in}{3.129198in}}%
\pgfpathlineto{\pgfqpoint{6.376430in}{3.115592in}}%
\pgfpathlineto{\pgfqpoint{6.379641in}{3.115318in}}%
\pgfpathlineto{\pgfqpoint{6.382853in}{3.117590in}}%
\pgfpathlineto{\pgfqpoint{6.387670in}{3.124347in}}%
\pgfpathlineto{\pgfqpoint{6.394895in}{3.134061in}}%
\pgfpathlineto{\pgfqpoint{6.398374in}{3.135293in}}%
\pgfpathlineto{\pgfqpoint{6.401318in}{3.133955in}}%
\pgfpathlineto{\pgfqpoint{6.405332in}{3.129198in}}%
\pgfpathlineto{\pgfqpoint{6.415769in}{3.115592in}}%
\pgfpathlineto{\pgfqpoint{6.418980in}{3.115318in}}%
\pgfpathlineto{\pgfqpoint{6.422191in}{3.117590in}}%
\pgfpathlineto{\pgfqpoint{6.427008in}{3.124347in}}%
\pgfpathlineto{\pgfqpoint{6.434233in}{3.134061in}}%
\pgfpathlineto{\pgfqpoint{6.437712in}{3.135293in}}%
\pgfpathlineto{\pgfqpoint{6.440656in}{3.133955in}}%
\pgfpathlineto{\pgfqpoint{6.444670in}{3.129198in}}%
\pgfpathlineto{\pgfqpoint{6.455107in}{3.115592in}}%
\pgfpathlineto{\pgfqpoint{6.458318in}{3.115318in}}%
\pgfpathlineto{\pgfqpoint{6.461529in}{3.117590in}}%
\pgfpathlineto{\pgfqpoint{6.466346in}{3.124347in}}%
\pgfpathlineto{\pgfqpoint{6.473572in}{3.134061in}}%
\pgfpathlineto{\pgfqpoint{6.477051in}{3.135293in}}%
\pgfpathlineto{\pgfqpoint{6.479994in}{3.133955in}}%
\pgfpathlineto{\pgfqpoint{6.484009in}{3.129198in}}%
\pgfpathlineto{\pgfqpoint{6.494445in}{3.115592in}}%
\pgfpathlineto{\pgfqpoint{6.497657in}{3.115318in}}%
\pgfpathlineto{\pgfqpoint{6.500868in}{3.117590in}}%
\pgfpathlineto{\pgfqpoint{6.505685in}{3.124347in}}%
\pgfpathlineto{\pgfqpoint{6.512910in}{3.134061in}}%
\pgfpathlineto{\pgfqpoint{6.516389in}{3.135293in}}%
\pgfpathlineto{\pgfqpoint{6.519333in}{3.133955in}}%
\pgfpathlineto{\pgfqpoint{6.523347in}{3.129198in}}%
\pgfpathlineto{\pgfqpoint{6.533784in}{3.115592in}}%
\pgfpathlineto{\pgfqpoint{6.536995in}{3.115318in}}%
\pgfpathlineto{\pgfqpoint{6.540206in}{3.117590in}}%
\pgfpathlineto{\pgfqpoint{6.545023in}{3.124347in}}%
\pgfpathlineto{\pgfqpoint{6.552249in}{3.134061in}}%
\pgfpathlineto{\pgfqpoint{6.555727in}{3.135293in}}%
\pgfpathlineto{\pgfqpoint{6.558671in}{3.133955in}}%
\pgfpathlineto{\pgfqpoint{6.562685in}{3.129198in}}%
\pgfpathlineto{\pgfqpoint{6.573122in}{3.115592in}}%
\pgfpathlineto{\pgfqpoint{6.576333in}{3.115318in}}%
\pgfpathlineto{\pgfqpoint{6.579545in}{3.117590in}}%
\pgfpathlineto{\pgfqpoint{6.584361in}{3.124347in}}%
\pgfpathlineto{\pgfqpoint{6.591587in}{3.134061in}}%
\pgfpathlineto{\pgfqpoint{6.595066in}{3.135293in}}%
\pgfpathlineto{\pgfqpoint{6.598009in}{3.133955in}}%
\pgfpathlineto{\pgfqpoint{6.602024in}{3.129198in}}%
\pgfpathlineto{\pgfqpoint{6.612460in}{3.115592in}}%
\pgfpathlineto{\pgfqpoint{6.615672in}{3.115318in}}%
\pgfpathlineto{\pgfqpoint{6.618883in}{3.117590in}}%
\pgfpathlineto{\pgfqpoint{6.623700in}{3.124347in}}%
\pgfpathlineto{\pgfqpoint{6.630925in}{3.134061in}}%
\pgfpathlineto{\pgfqpoint{6.634404in}{3.135293in}}%
\pgfpathlineto{\pgfqpoint{6.637348in}{3.133955in}}%
\pgfpathlineto{\pgfqpoint{6.641362in}{3.129198in}}%
\pgfpathlineto{\pgfqpoint{6.651799in}{3.115592in}}%
\pgfpathlineto{\pgfqpoint{6.655010in}{3.115318in}}%
\pgfpathlineto{\pgfqpoint{6.658221in}{3.117590in}}%
\pgfpathlineto{\pgfqpoint{6.663038in}{3.124347in}}%
\pgfpathlineto{\pgfqpoint{6.663306in}{3.124778in}}%
\pgfpathlineto{\pgfqpoint{6.663306in}{3.124778in}}%
\pgfusepath{stroke}%
\end{pgfscope}%
\begin{pgfscope}%
\pgfpathrectangle{\pgfqpoint{0.467797in}{2.292089in}}{\pgfqpoint{6.490533in}{1.666241in}}%
\pgfusepath{clip}%
\pgfsetrectcap%
\pgfsetroundjoin%
\pgfsetlinewidth{1.505625pt}%
\definecolor{currentstroke}{rgb}{0.737255,0.741176,0.133333}%
\pgfsetstrokecolor{currentstroke}%
\pgfsetdash{}{0pt}%
\pgfpathmoveto{\pgfqpoint{0.762821in}{3.125209in}}%
\pgfpathlineto{\pgfqpoint{0.769511in}{3.133962in}}%
\pgfpathlineto{\pgfqpoint{0.772722in}{3.135032in}}%
\pgfpathlineto{\pgfqpoint{0.775666in}{3.133668in}}%
\pgfpathlineto{\pgfqpoint{0.779680in}{3.128830in}}%
\pgfpathlineto{\pgfqpoint{0.789314in}{3.116032in}}%
\pgfpathlineto{\pgfqpoint{0.792525in}{3.115497in}}%
\pgfpathlineto{\pgfqpoint{0.795469in}{3.117327in}}%
\pgfpathlineto{\pgfqpoint{0.799751in}{3.122992in}}%
\pgfpathlineto{\pgfqpoint{0.808047in}{3.134118in}}%
\pgfpathlineto{\pgfqpoint{0.811258in}{3.135007in}}%
\pgfpathlineto{\pgfqpoint{0.814202in}{3.133483in}}%
\pgfpathlineto{\pgfqpoint{0.818216in}{3.128500in}}%
\pgfpathlineto{\pgfqpoint{0.827582in}{3.116060in}}%
\pgfpathlineto{\pgfqpoint{0.830793in}{3.115485in}}%
\pgfpathlineto{\pgfqpoint{0.833737in}{3.117281in}}%
\pgfpathlineto{\pgfqpoint{0.838019in}{3.122916in}}%
\pgfpathlineto{\pgfqpoint{0.846315in}{3.134084in}}%
\pgfpathlineto{\pgfqpoint{0.849526in}{3.135013in}}%
\pgfpathlineto{\pgfqpoint{0.852470in}{3.133525in}}%
\pgfpathlineto{\pgfqpoint{0.856484in}{3.128574in}}%
\pgfpathlineto{\pgfqpoint{0.865850in}{3.116090in}}%
\pgfpathlineto{\pgfqpoint{0.869061in}{3.115473in}}%
\pgfpathlineto{\pgfqpoint{0.872005in}{3.117235in}}%
\pgfpathlineto{\pgfqpoint{0.876287in}{3.122839in}}%
\pgfpathlineto{\pgfqpoint{0.884582in}{3.134050in}}%
\pgfpathlineto{\pgfqpoint{0.887794in}{3.135020in}}%
\pgfpathlineto{\pgfqpoint{0.890737in}{3.133567in}}%
\pgfpathlineto{\pgfqpoint{0.894752in}{3.128647in}}%
\pgfpathlineto{\pgfqpoint{0.904385in}{3.115963in}}%
\pgfpathlineto{\pgfqpoint{0.907597in}{3.115530in}}%
\pgfpathlineto{\pgfqpoint{0.910808in}{3.117719in}}%
\pgfpathlineto{\pgfqpoint{0.915625in}{3.124465in}}%
\pgfpathlineto{\pgfqpoint{0.922850in}{3.134015in}}%
\pgfpathlineto{\pgfqpoint{0.926062in}{3.135025in}}%
\pgfpathlineto{\pgfqpoint{0.929005in}{3.133608in}}%
\pgfpathlineto{\pgfqpoint{0.933019in}{3.128721in}}%
\pgfpathlineto{\pgfqpoint{0.942653in}{3.115990in}}%
\pgfpathlineto{\pgfqpoint{0.945865in}{3.115516in}}%
\pgfpathlineto{\pgfqpoint{0.949076in}{3.117668in}}%
\pgfpathlineto{\pgfqpoint{0.953893in}{3.124386in}}%
\pgfpathlineto{\pgfqpoint{0.961118in}{3.133980in}}%
\pgfpathlineto{\pgfqpoint{0.964330in}{3.135030in}}%
\pgfpathlineto{\pgfqpoint{0.967273in}{3.133648in}}%
\pgfpathlineto{\pgfqpoint{0.971287in}{3.128794in}}%
\pgfpathlineto{\pgfqpoint{0.980921in}{3.116018in}}%
\pgfpathlineto{\pgfqpoint{0.984133in}{3.115503in}}%
\pgfpathlineto{\pgfqpoint{0.987076in}{3.117351in}}%
\pgfpathlineto{\pgfqpoint{0.991626in}{3.123453in}}%
\pgfpathlineto{\pgfqpoint{0.999386in}{3.133944in}}%
\pgfpathlineto{\pgfqpoint{1.002598in}{3.135034in}}%
\pgfpathlineto{\pgfqpoint{1.005541in}{3.133688in}}%
\pgfpathlineto{\pgfqpoint{1.009555in}{3.128867in}}%
\pgfpathlineto{\pgfqpoint{1.019457in}{3.115898in}}%
\pgfpathlineto{\pgfqpoint{1.022668in}{3.115566in}}%
\pgfpathlineto{\pgfqpoint{1.025879in}{3.117847in}}%
\pgfpathlineto{\pgfqpoint{1.030696in}{3.124660in}}%
\pgfpathlineto{\pgfqpoint{1.037654in}{3.133908in}}%
\pgfpathlineto{\pgfqpoint{1.040865in}{3.135037in}}%
\pgfpathlineto{\pgfqpoint{1.043809in}{3.133728in}}%
\pgfpathlineto{\pgfqpoint{1.047823in}{3.128940in}}%
\pgfpathlineto{\pgfqpoint{1.057725in}{3.115924in}}%
\pgfpathlineto{\pgfqpoint{1.060936in}{3.115551in}}%
\pgfpathlineto{\pgfqpoint{1.064147in}{3.117796in}}%
\pgfpathlineto{\pgfqpoint{1.068964in}{3.124582in}}%
\pgfpathlineto{\pgfqpoint{1.075922in}{3.133871in}}%
\pgfpathlineto{\pgfqpoint{1.079133in}{3.135040in}}%
\pgfpathlineto{\pgfqpoint{1.082077in}{3.133767in}}%
\pgfpathlineto{\pgfqpoint{1.086091in}{3.129012in}}%
\pgfpathlineto{\pgfqpoint{1.095993in}{3.115950in}}%
\pgfpathlineto{\pgfqpoint{1.099204in}{3.115537in}}%
\pgfpathlineto{\pgfqpoint{1.102415in}{3.117744in}}%
\pgfpathlineto{\pgfqpoint{1.107232in}{3.124504in}}%
\pgfpathlineto{\pgfqpoint{1.114190in}{3.133834in}}%
\pgfpathlineto{\pgfqpoint{1.117401in}{3.135043in}}%
\pgfpathlineto{\pgfqpoint{1.120345in}{3.133805in}}%
\pgfpathlineto{\pgfqpoint{1.124091in}{3.129477in}}%
\pgfpathlineto{\pgfqpoint{1.134796in}{3.115715in}}%
\pgfpathlineto{\pgfqpoint{1.137739in}{3.115607in}}%
\pgfpathlineto{\pgfqpoint{1.140951in}{3.117979in}}%
\pgfpathlineto{\pgfqpoint{1.146035in}{3.125288in}}%
\pgfpathlineto{\pgfqpoint{1.152726in}{3.133998in}}%
\pgfpathlineto{\pgfqpoint{1.155937in}{3.135027in}}%
\pgfpathlineto{\pgfqpoint{1.158880in}{3.133628in}}%
\pgfpathlineto{\pgfqpoint{1.162895in}{3.128757in}}%
\pgfpathlineto{\pgfqpoint{1.172528in}{3.116004in}}%
\pgfpathlineto{\pgfqpoint{1.175740in}{3.115509in}}%
\pgfpathlineto{\pgfqpoint{1.178683in}{3.117375in}}%
\pgfpathlineto{\pgfqpoint{1.183233in}{3.123492in}}%
\pgfpathlineto{\pgfqpoint{1.190993in}{3.133962in}}%
\pgfpathlineto{\pgfqpoint{1.194205in}{3.135032in}}%
\pgfpathlineto{\pgfqpoint{1.197148in}{3.133668in}}%
\pgfpathlineto{\pgfqpoint{1.201163in}{3.128830in}}%
\pgfpathlineto{\pgfqpoint{1.210796in}{3.116032in}}%
\pgfpathlineto{\pgfqpoint{1.214008in}{3.115497in}}%
\pgfpathlineto{\pgfqpoint{1.216951in}{3.117327in}}%
\pgfpathlineto{\pgfqpoint{1.221233in}{3.122992in}}%
\pgfpathlineto{\pgfqpoint{1.229529in}{3.134118in}}%
\pgfpathlineto{\pgfqpoint{1.232740in}{3.135007in}}%
\pgfpathlineto{\pgfqpoint{1.235684in}{3.133483in}}%
\pgfpathlineto{\pgfqpoint{1.239698in}{3.128500in}}%
\pgfpathlineto{\pgfqpoint{1.249064in}{3.116060in}}%
\pgfpathlineto{\pgfqpoint{1.252276in}{3.115485in}}%
\pgfpathlineto{\pgfqpoint{1.255219in}{3.117281in}}%
\pgfpathlineto{\pgfqpoint{1.259501in}{3.122916in}}%
\pgfpathlineto{\pgfqpoint{1.267797in}{3.134084in}}%
\pgfpathlineto{\pgfqpoint{1.271008in}{3.135013in}}%
\pgfpathlineto{\pgfqpoint{1.273952in}{3.133525in}}%
\pgfpathlineto{\pgfqpoint{1.277966in}{3.128574in}}%
\pgfpathlineto{\pgfqpoint{1.287332in}{3.116090in}}%
\pgfpathlineto{\pgfqpoint{1.290544in}{3.115473in}}%
\pgfpathlineto{\pgfqpoint{1.293487in}{3.117235in}}%
\pgfpathlineto{\pgfqpoint{1.297769in}{3.122839in}}%
\pgfpathlineto{\pgfqpoint{1.306065in}{3.134050in}}%
\pgfpathlineto{\pgfqpoint{1.309276in}{3.135020in}}%
\pgfpathlineto{\pgfqpoint{1.312220in}{3.133567in}}%
\pgfpathlineto{\pgfqpoint{1.316234in}{3.128647in}}%
\pgfpathlineto{\pgfqpoint{1.325868in}{3.115963in}}%
\pgfpathlineto{\pgfqpoint{1.329079in}{3.115530in}}%
\pgfpathlineto{\pgfqpoint{1.332290in}{3.117719in}}%
\pgfpathlineto{\pgfqpoint{1.337107in}{3.124465in}}%
\pgfpathlineto{\pgfqpoint{1.344333in}{3.134015in}}%
\pgfpathlineto{\pgfqpoint{1.347544in}{3.135025in}}%
\pgfpathlineto{\pgfqpoint{1.350488in}{3.133608in}}%
\pgfpathlineto{\pgfqpoint{1.354502in}{3.128721in}}%
\pgfpathlineto{\pgfqpoint{1.364136in}{3.115990in}}%
\pgfpathlineto{\pgfqpoint{1.367347in}{3.115516in}}%
\pgfpathlineto{\pgfqpoint{1.370558in}{3.117668in}}%
\pgfpathlineto{\pgfqpoint{1.375375in}{3.124386in}}%
\pgfpathlineto{\pgfqpoint{1.382601in}{3.133980in}}%
\pgfpathlineto{\pgfqpoint{1.385812in}{3.135030in}}%
\pgfpathlineto{\pgfqpoint{1.388756in}{3.133648in}}%
\pgfpathlineto{\pgfqpoint{1.392770in}{3.128794in}}%
\pgfpathlineto{\pgfqpoint{1.402404in}{3.116018in}}%
\pgfpathlineto{\pgfqpoint{1.405615in}{3.115503in}}%
\pgfpathlineto{\pgfqpoint{1.408559in}{3.117351in}}%
\pgfpathlineto{\pgfqpoint{1.413108in}{3.123453in}}%
\pgfpathlineto{\pgfqpoint{1.420869in}{3.133944in}}%
\pgfpathlineto{\pgfqpoint{1.424080in}{3.135034in}}%
\pgfpathlineto{\pgfqpoint{1.427024in}{3.133688in}}%
\pgfpathlineto{\pgfqpoint{1.431038in}{3.128867in}}%
\pgfpathlineto{\pgfqpoint{1.440939in}{3.115898in}}%
\pgfpathlineto{\pgfqpoint{1.444150in}{3.115566in}}%
\pgfpathlineto{\pgfqpoint{1.447362in}{3.117847in}}%
\pgfpathlineto{\pgfqpoint{1.452179in}{3.124660in}}%
\pgfpathlineto{\pgfqpoint{1.459136in}{3.133908in}}%
\pgfpathlineto{\pgfqpoint{1.462348in}{3.135037in}}%
\pgfpathlineto{\pgfqpoint{1.465291in}{3.133728in}}%
\pgfpathlineto{\pgfqpoint{1.469306in}{3.128940in}}%
\pgfpathlineto{\pgfqpoint{1.479207in}{3.115924in}}%
\pgfpathlineto{\pgfqpoint{1.482418in}{3.115551in}}%
\pgfpathlineto{\pgfqpoint{1.485630in}{3.117796in}}%
\pgfpathlineto{\pgfqpoint{1.490447in}{3.124582in}}%
\pgfpathlineto{\pgfqpoint{1.497404in}{3.133871in}}%
\pgfpathlineto{\pgfqpoint{1.500616in}{3.135040in}}%
\pgfpathlineto{\pgfqpoint{1.503559in}{3.133767in}}%
\pgfpathlineto{\pgfqpoint{1.507573in}{3.129012in}}%
\pgfpathlineto{\pgfqpoint{1.517475in}{3.115950in}}%
\pgfpathlineto{\pgfqpoint{1.520686in}{3.115537in}}%
\pgfpathlineto{\pgfqpoint{1.523898in}{3.117744in}}%
\pgfpathlineto{\pgfqpoint{1.528714in}{3.124504in}}%
\pgfpathlineto{\pgfqpoint{1.535672in}{3.133834in}}%
\pgfpathlineto{\pgfqpoint{1.538884in}{3.135043in}}%
\pgfpathlineto{\pgfqpoint{1.541827in}{3.133805in}}%
\pgfpathlineto{\pgfqpoint{1.545574in}{3.129477in}}%
\pgfpathlineto{\pgfqpoint{1.556278in}{3.115715in}}%
\pgfpathlineto{\pgfqpoint{1.559222in}{3.115607in}}%
\pgfpathlineto{\pgfqpoint{1.562433in}{3.117979in}}%
\pgfpathlineto{\pgfqpoint{1.567518in}{3.125288in}}%
\pgfpathlineto{\pgfqpoint{1.574208in}{3.133998in}}%
\pgfpathlineto{\pgfqpoint{1.577419in}{3.135027in}}%
\pgfpathlineto{\pgfqpoint{1.580363in}{3.133628in}}%
\pgfpathlineto{\pgfqpoint{1.584377in}{3.128757in}}%
\pgfpathlineto{\pgfqpoint{1.594011in}{3.116004in}}%
\pgfpathlineto{\pgfqpoint{1.597222in}{3.115509in}}%
\pgfpathlineto{\pgfqpoint{1.600166in}{3.117375in}}%
\pgfpathlineto{\pgfqpoint{1.604715in}{3.123492in}}%
\pgfpathlineto{\pgfqpoint{1.612476in}{3.133962in}}%
\pgfpathlineto{\pgfqpoint{1.615687in}{3.135032in}}%
\pgfpathlineto{\pgfqpoint{1.618631in}{3.133668in}}%
\pgfpathlineto{\pgfqpoint{1.622645in}{3.128830in}}%
\pgfpathlineto{\pgfqpoint{1.632279in}{3.116032in}}%
\pgfpathlineto{\pgfqpoint{1.635490in}{3.115497in}}%
\pgfpathlineto{\pgfqpoint{1.638434in}{3.117327in}}%
\pgfpathlineto{\pgfqpoint{1.642715in}{3.122992in}}%
\pgfpathlineto{\pgfqpoint{1.651011in}{3.134118in}}%
\pgfpathlineto{\pgfqpoint{1.654223in}{3.135007in}}%
\pgfpathlineto{\pgfqpoint{1.657166in}{3.133483in}}%
\pgfpathlineto{\pgfqpoint{1.661180in}{3.128500in}}%
\pgfpathlineto{\pgfqpoint{1.670547in}{3.116060in}}%
\pgfpathlineto{\pgfqpoint{1.673758in}{3.115485in}}%
\pgfpathlineto{\pgfqpoint{1.676702in}{3.117281in}}%
\pgfpathlineto{\pgfqpoint{1.680983in}{3.122916in}}%
\pgfpathlineto{\pgfqpoint{1.689279in}{3.134084in}}%
\pgfpathlineto{\pgfqpoint{1.692490in}{3.135013in}}%
\pgfpathlineto{\pgfqpoint{1.695434in}{3.133525in}}%
\pgfpathlineto{\pgfqpoint{1.699448in}{3.128574in}}%
\pgfpathlineto{\pgfqpoint{1.708815in}{3.116090in}}%
\pgfpathlineto{\pgfqpoint{1.712026in}{3.115473in}}%
\pgfpathlineto{\pgfqpoint{1.714970in}{3.117235in}}%
\pgfpathlineto{\pgfqpoint{1.719251in}{3.122839in}}%
\pgfpathlineto{\pgfqpoint{1.727547in}{3.134050in}}%
\pgfpathlineto{\pgfqpoint{1.730758in}{3.135020in}}%
\pgfpathlineto{\pgfqpoint{1.733702in}{3.133567in}}%
\pgfpathlineto{\pgfqpoint{1.737716in}{3.128647in}}%
\pgfpathlineto{\pgfqpoint{1.747350in}{3.115963in}}%
\pgfpathlineto{\pgfqpoint{1.750561in}{3.115530in}}%
\pgfpathlineto{\pgfqpoint{1.753773in}{3.117719in}}%
\pgfpathlineto{\pgfqpoint{1.758590in}{3.124465in}}%
\pgfpathlineto{\pgfqpoint{1.765815in}{3.134015in}}%
\pgfpathlineto{\pgfqpoint{1.769026in}{3.135025in}}%
\pgfpathlineto{\pgfqpoint{1.771970in}{3.133608in}}%
\pgfpathlineto{\pgfqpoint{1.775984in}{3.128721in}}%
\pgfpathlineto{\pgfqpoint{1.785618in}{3.115990in}}%
\pgfpathlineto{\pgfqpoint{1.788829in}{3.115516in}}%
\pgfpathlineto{\pgfqpoint{1.792041in}{3.117668in}}%
\pgfpathlineto{\pgfqpoint{1.796858in}{3.124386in}}%
\pgfpathlineto{\pgfqpoint{1.804083in}{3.133980in}}%
\pgfpathlineto{\pgfqpoint{1.807294in}{3.135030in}}%
\pgfpathlineto{\pgfqpoint{1.810238in}{3.133648in}}%
\pgfpathlineto{\pgfqpoint{1.814252in}{3.128794in}}%
\pgfpathlineto{\pgfqpoint{1.823886in}{3.116018in}}%
\pgfpathlineto{\pgfqpoint{1.827097in}{3.115503in}}%
\pgfpathlineto{\pgfqpoint{1.830041in}{3.117351in}}%
\pgfpathlineto{\pgfqpoint{1.834590in}{3.123453in}}%
\pgfpathlineto{\pgfqpoint{1.842351in}{3.133944in}}%
\pgfpathlineto{\pgfqpoint{1.845562in}{3.135034in}}%
\pgfpathlineto{\pgfqpoint{1.848506in}{3.133688in}}%
\pgfpathlineto{\pgfqpoint{1.852520in}{3.128867in}}%
\pgfpathlineto{\pgfqpoint{1.862421in}{3.115898in}}%
\pgfpathlineto{\pgfqpoint{1.865633in}{3.115566in}}%
\pgfpathlineto{\pgfqpoint{1.868844in}{3.117847in}}%
\pgfpathlineto{\pgfqpoint{1.873661in}{3.124660in}}%
\pgfpathlineto{\pgfqpoint{1.880619in}{3.133908in}}%
\pgfpathlineto{\pgfqpoint{1.883830in}{3.135037in}}%
\pgfpathlineto{\pgfqpoint{1.886774in}{3.133728in}}%
\pgfpathlineto{\pgfqpoint{1.890788in}{3.128940in}}%
\pgfpathlineto{\pgfqpoint{1.900689in}{3.115924in}}%
\pgfpathlineto{\pgfqpoint{1.903901in}{3.115551in}}%
\pgfpathlineto{\pgfqpoint{1.907112in}{3.117796in}}%
\pgfpathlineto{\pgfqpoint{1.911929in}{3.124582in}}%
\pgfpathlineto{\pgfqpoint{1.918887in}{3.133871in}}%
\pgfpathlineto{\pgfqpoint{1.922098in}{3.135040in}}%
\pgfpathlineto{\pgfqpoint{1.925042in}{3.133767in}}%
\pgfpathlineto{\pgfqpoint{1.929056in}{3.129012in}}%
\pgfpathlineto{\pgfqpoint{1.938957in}{3.115950in}}%
\pgfpathlineto{\pgfqpoint{1.942169in}{3.115537in}}%
\pgfpathlineto{\pgfqpoint{1.945380in}{3.117744in}}%
\pgfpathlineto{\pgfqpoint{1.950197in}{3.124504in}}%
\pgfpathlineto{\pgfqpoint{1.957155in}{3.133834in}}%
\pgfpathlineto{\pgfqpoint{1.960366in}{3.135043in}}%
\pgfpathlineto{\pgfqpoint{1.963310in}{3.133805in}}%
\pgfpathlineto{\pgfqpoint{1.967056in}{3.129477in}}%
\pgfpathlineto{\pgfqpoint{1.977760in}{3.115715in}}%
\pgfpathlineto{\pgfqpoint{1.980704in}{3.115607in}}%
\pgfpathlineto{\pgfqpoint{1.983915in}{3.117979in}}%
\pgfpathlineto{\pgfqpoint{1.989000in}{3.125288in}}%
\pgfpathlineto{\pgfqpoint{1.995690in}{3.133998in}}%
\pgfpathlineto{\pgfqpoint{1.998901in}{3.135027in}}%
\pgfpathlineto{\pgfqpoint{2.001845in}{3.133628in}}%
\pgfpathlineto{\pgfqpoint{2.005859in}{3.128757in}}%
\pgfpathlineto{\pgfqpoint{2.015493in}{3.116004in}}%
\pgfpathlineto{\pgfqpoint{2.018704in}{3.115509in}}%
\pgfpathlineto{\pgfqpoint{2.021648in}{3.117375in}}%
\pgfpathlineto{\pgfqpoint{2.026197in}{3.123492in}}%
\pgfpathlineto{\pgfqpoint{2.033958in}{3.133962in}}%
\pgfpathlineto{\pgfqpoint{2.037169in}{3.135032in}}%
\pgfpathlineto{\pgfqpoint{2.040113in}{3.133668in}}%
\pgfpathlineto{\pgfqpoint{2.044127in}{3.128830in}}%
\pgfpathlineto{\pgfqpoint{2.053761in}{3.116032in}}%
\pgfpathlineto{\pgfqpoint{2.056972in}{3.115497in}}%
\pgfpathlineto{\pgfqpoint{2.059916in}{3.117327in}}%
\pgfpathlineto{\pgfqpoint{2.064198in}{3.122992in}}%
\pgfpathlineto{\pgfqpoint{2.072494in}{3.134118in}}%
\pgfpathlineto{\pgfqpoint{2.075705in}{3.135007in}}%
\pgfpathlineto{\pgfqpoint{2.078649in}{3.133483in}}%
\pgfpathlineto{\pgfqpoint{2.082663in}{3.128500in}}%
\pgfpathlineto{\pgfqpoint{2.092029in}{3.116060in}}%
\pgfpathlineto{\pgfqpoint{2.095240in}{3.115485in}}%
\pgfpathlineto{\pgfqpoint{2.098184in}{3.117281in}}%
\pgfpathlineto{\pgfqpoint{2.102466in}{3.122916in}}%
\pgfpathlineto{\pgfqpoint{2.110762in}{3.134084in}}%
\pgfpathlineto{\pgfqpoint{2.113973in}{3.135013in}}%
\pgfpathlineto{\pgfqpoint{2.116916in}{3.133525in}}%
\pgfpathlineto{\pgfqpoint{2.120931in}{3.128574in}}%
\pgfpathlineto{\pgfqpoint{2.130297in}{3.116090in}}%
\pgfpathlineto{\pgfqpoint{2.133508in}{3.115473in}}%
\pgfpathlineto{\pgfqpoint{2.136452in}{3.117235in}}%
\pgfpathlineto{\pgfqpoint{2.140734in}{3.122839in}}%
\pgfpathlineto{\pgfqpoint{2.149029in}{3.134050in}}%
\pgfpathlineto{\pgfqpoint{2.152241in}{3.135020in}}%
\pgfpathlineto{\pgfqpoint{2.155184in}{3.133567in}}%
\pgfpathlineto{\pgfqpoint{2.159199in}{3.128647in}}%
\pgfpathlineto{\pgfqpoint{2.168832in}{3.115963in}}%
\pgfpathlineto{\pgfqpoint{2.172044in}{3.115530in}}%
\pgfpathlineto{\pgfqpoint{2.175255in}{3.117719in}}%
\pgfpathlineto{\pgfqpoint{2.180072in}{3.124465in}}%
\pgfpathlineto{\pgfqpoint{2.187297in}{3.134015in}}%
\pgfpathlineto{\pgfqpoint{2.190509in}{3.135025in}}%
\pgfpathlineto{\pgfqpoint{2.193452in}{3.133608in}}%
\pgfpathlineto{\pgfqpoint{2.197466in}{3.128721in}}%
\pgfpathlineto{\pgfqpoint{2.207100in}{3.115990in}}%
\pgfpathlineto{\pgfqpoint{2.210312in}{3.115516in}}%
\pgfpathlineto{\pgfqpoint{2.213523in}{3.117668in}}%
\pgfpathlineto{\pgfqpoint{2.218340in}{3.124386in}}%
\pgfpathlineto{\pgfqpoint{2.225565in}{3.133980in}}%
\pgfpathlineto{\pgfqpoint{2.228777in}{3.135030in}}%
\pgfpathlineto{\pgfqpoint{2.231720in}{3.133648in}}%
\pgfpathlineto{\pgfqpoint{2.235734in}{3.128794in}}%
\pgfpathlineto{\pgfqpoint{2.245368in}{3.116018in}}%
\pgfpathlineto{\pgfqpoint{2.248580in}{3.115503in}}%
\pgfpathlineto{\pgfqpoint{2.251523in}{3.117351in}}%
\pgfpathlineto{\pgfqpoint{2.256073in}{3.123453in}}%
\pgfpathlineto{\pgfqpoint{2.263833in}{3.133944in}}%
\pgfpathlineto{\pgfqpoint{2.267044in}{3.135034in}}%
\pgfpathlineto{\pgfqpoint{2.269988in}{3.133688in}}%
\pgfpathlineto{\pgfqpoint{2.274002in}{3.128867in}}%
\pgfpathlineto{\pgfqpoint{2.283904in}{3.115898in}}%
\pgfpathlineto{\pgfqpoint{2.287115in}{3.115566in}}%
\pgfpathlineto{\pgfqpoint{2.290326in}{3.117847in}}%
\pgfpathlineto{\pgfqpoint{2.295143in}{3.124660in}}%
\pgfpathlineto{\pgfqpoint{2.302101in}{3.133908in}}%
\pgfpathlineto{\pgfqpoint{2.305312in}{3.135037in}}%
\pgfpathlineto{\pgfqpoint{2.308256in}{3.133728in}}%
\pgfpathlineto{\pgfqpoint{2.312270in}{3.128940in}}%
\pgfpathlineto{\pgfqpoint{2.322172in}{3.115924in}}%
\pgfpathlineto{\pgfqpoint{2.325383in}{3.115551in}}%
\pgfpathlineto{\pgfqpoint{2.328594in}{3.117796in}}%
\pgfpathlineto{\pgfqpoint{2.333411in}{3.124582in}}%
\pgfpathlineto{\pgfqpoint{2.340369in}{3.133871in}}%
\pgfpathlineto{\pgfqpoint{2.343580in}{3.135040in}}%
\pgfpathlineto{\pgfqpoint{2.346524in}{3.133767in}}%
\pgfpathlineto{\pgfqpoint{2.350538in}{3.129012in}}%
\pgfpathlineto{\pgfqpoint{2.360440in}{3.115950in}}%
\pgfpathlineto{\pgfqpoint{2.363651in}{3.115537in}}%
\pgfpathlineto{\pgfqpoint{2.366862in}{3.117744in}}%
\pgfpathlineto{\pgfqpoint{2.371679in}{3.124504in}}%
\pgfpathlineto{\pgfqpoint{2.378637in}{3.133834in}}%
\pgfpathlineto{\pgfqpoint{2.381848in}{3.135043in}}%
\pgfpathlineto{\pgfqpoint{2.384792in}{3.133805in}}%
\pgfpathlineto{\pgfqpoint{2.388538in}{3.129477in}}%
\pgfpathlineto{\pgfqpoint{2.399243in}{3.115715in}}%
\pgfpathlineto{\pgfqpoint{2.402186in}{3.115607in}}%
\pgfpathlineto{\pgfqpoint{2.405398in}{3.117979in}}%
\pgfpathlineto{\pgfqpoint{2.410482in}{3.125288in}}%
\pgfpathlineto{\pgfqpoint{2.417172in}{3.133998in}}%
\pgfpathlineto{\pgfqpoint{2.420384in}{3.135027in}}%
\pgfpathlineto{\pgfqpoint{2.423327in}{3.133628in}}%
\pgfpathlineto{\pgfqpoint{2.427342in}{3.128757in}}%
\pgfpathlineto{\pgfqpoint{2.436975in}{3.116004in}}%
\pgfpathlineto{\pgfqpoint{2.440187in}{3.115509in}}%
\pgfpathlineto{\pgfqpoint{2.443130in}{3.117375in}}%
\pgfpathlineto{\pgfqpoint{2.447680in}{3.123492in}}%
\pgfpathlineto{\pgfqpoint{2.455440in}{3.133962in}}%
\pgfpathlineto{\pgfqpoint{2.458652in}{3.135032in}}%
\pgfpathlineto{\pgfqpoint{2.461595in}{3.133668in}}%
\pgfpathlineto{\pgfqpoint{2.465609in}{3.128830in}}%
\pgfpathlineto{\pgfqpoint{2.475243in}{3.116032in}}%
\pgfpathlineto{\pgfqpoint{2.478455in}{3.115497in}}%
\pgfpathlineto{\pgfqpoint{2.481398in}{3.117327in}}%
\pgfpathlineto{\pgfqpoint{2.485680in}{3.122992in}}%
\pgfpathlineto{\pgfqpoint{2.493976in}{3.134118in}}%
\pgfpathlineto{\pgfqpoint{2.497187in}{3.135007in}}%
\pgfpathlineto{\pgfqpoint{2.500131in}{3.133483in}}%
\pgfpathlineto{\pgfqpoint{2.504145in}{3.128500in}}%
\pgfpathlineto{\pgfqpoint{2.513511in}{3.116060in}}%
\pgfpathlineto{\pgfqpoint{2.516723in}{3.115485in}}%
\pgfpathlineto{\pgfqpoint{2.519666in}{3.117281in}}%
\pgfpathlineto{\pgfqpoint{2.523948in}{3.122916in}}%
\pgfpathlineto{\pgfqpoint{2.532244in}{3.134084in}}%
\pgfpathlineto{\pgfqpoint{2.535455in}{3.135013in}}%
\pgfpathlineto{\pgfqpoint{2.538399in}{3.133525in}}%
\pgfpathlineto{\pgfqpoint{2.542413in}{3.128574in}}%
\pgfpathlineto{\pgfqpoint{2.551779in}{3.116090in}}%
\pgfpathlineto{\pgfqpoint{2.554990in}{3.115473in}}%
\pgfpathlineto{\pgfqpoint{2.557934in}{3.117235in}}%
\pgfpathlineto{\pgfqpoint{2.562216in}{3.122839in}}%
\pgfpathlineto{\pgfqpoint{2.570512in}{3.134050in}}%
\pgfpathlineto{\pgfqpoint{2.573723in}{3.135020in}}%
\pgfpathlineto{\pgfqpoint{2.576667in}{3.133567in}}%
\pgfpathlineto{\pgfqpoint{2.580681in}{3.128647in}}%
\pgfpathlineto{\pgfqpoint{2.590315in}{3.115963in}}%
\pgfpathlineto{\pgfqpoint{2.593526in}{3.115530in}}%
\pgfpathlineto{\pgfqpoint{2.596737in}{3.117719in}}%
\pgfpathlineto{\pgfqpoint{2.601554in}{3.124465in}}%
\pgfpathlineto{\pgfqpoint{2.608780in}{3.134015in}}%
\pgfpathlineto{\pgfqpoint{2.611991in}{3.135025in}}%
\pgfpathlineto{\pgfqpoint{2.614935in}{3.133608in}}%
\pgfpathlineto{\pgfqpoint{2.618949in}{3.128721in}}%
\pgfpathlineto{\pgfqpoint{2.628583in}{3.115990in}}%
\pgfpathlineto{\pgfqpoint{2.631794in}{3.115516in}}%
\pgfpathlineto{\pgfqpoint{2.635005in}{3.117668in}}%
\pgfpathlineto{\pgfqpoint{2.639822in}{3.124386in}}%
\pgfpathlineto{\pgfqpoint{2.647048in}{3.133980in}}%
\pgfpathlineto{\pgfqpoint{2.650259in}{3.135030in}}%
\pgfpathlineto{\pgfqpoint{2.653203in}{3.133648in}}%
\pgfpathlineto{\pgfqpoint{2.657217in}{3.128794in}}%
\pgfpathlineto{\pgfqpoint{2.666851in}{3.116018in}}%
\pgfpathlineto{\pgfqpoint{2.670062in}{3.115503in}}%
\pgfpathlineto{\pgfqpoint{2.673006in}{3.117351in}}%
\pgfpathlineto{\pgfqpoint{2.677555in}{3.123453in}}%
\pgfpathlineto{\pgfqpoint{2.685315in}{3.133944in}}%
\pgfpathlineto{\pgfqpoint{2.688527in}{3.135034in}}%
\pgfpathlineto{\pgfqpoint{2.691470in}{3.133688in}}%
\pgfpathlineto{\pgfqpoint{2.695485in}{3.128867in}}%
\pgfpathlineto{\pgfqpoint{2.705386in}{3.115898in}}%
\pgfpathlineto{\pgfqpoint{2.708597in}{3.115566in}}%
\pgfpathlineto{\pgfqpoint{2.711809in}{3.117847in}}%
\pgfpathlineto{\pgfqpoint{2.716626in}{3.124660in}}%
\pgfpathlineto{\pgfqpoint{2.723583in}{3.133908in}}%
\pgfpathlineto{\pgfqpoint{2.726795in}{3.135037in}}%
\pgfpathlineto{\pgfqpoint{2.729738in}{3.133728in}}%
\pgfpathlineto{\pgfqpoint{2.733753in}{3.128940in}}%
\pgfpathlineto{\pgfqpoint{2.743654in}{3.115924in}}%
\pgfpathlineto{\pgfqpoint{2.746865in}{3.115551in}}%
\pgfpathlineto{\pgfqpoint{2.750077in}{3.117796in}}%
\pgfpathlineto{\pgfqpoint{2.754894in}{3.124582in}}%
\pgfpathlineto{\pgfqpoint{2.761851in}{3.133871in}}%
\pgfpathlineto{\pgfqpoint{2.765063in}{3.135040in}}%
\pgfpathlineto{\pgfqpoint{2.768006in}{3.133767in}}%
\pgfpathlineto{\pgfqpoint{2.772020in}{3.129012in}}%
\pgfpathlineto{\pgfqpoint{2.781922in}{3.115950in}}%
\pgfpathlineto{\pgfqpoint{2.785133in}{3.115537in}}%
\pgfpathlineto{\pgfqpoint{2.788345in}{3.117744in}}%
\pgfpathlineto{\pgfqpoint{2.793161in}{3.124504in}}%
\pgfpathlineto{\pgfqpoint{2.800119in}{3.133834in}}%
\pgfpathlineto{\pgfqpoint{2.803331in}{3.135043in}}%
\pgfpathlineto{\pgfqpoint{2.806274in}{3.133805in}}%
\pgfpathlineto{\pgfqpoint{2.810021in}{3.129477in}}%
\pgfpathlineto{\pgfqpoint{2.820725in}{3.115715in}}%
\pgfpathlineto{\pgfqpoint{2.823669in}{3.115607in}}%
\pgfpathlineto{\pgfqpoint{2.826880in}{3.117979in}}%
\pgfpathlineto{\pgfqpoint{2.831965in}{3.125288in}}%
\pgfpathlineto{\pgfqpoint{2.838655in}{3.133998in}}%
\pgfpathlineto{\pgfqpoint{2.841866in}{3.135027in}}%
\pgfpathlineto{\pgfqpoint{2.844810in}{3.133628in}}%
\pgfpathlineto{\pgfqpoint{2.848824in}{3.128757in}}%
\pgfpathlineto{\pgfqpoint{2.858458in}{3.116004in}}%
\pgfpathlineto{\pgfqpoint{2.861669in}{3.115509in}}%
\pgfpathlineto{\pgfqpoint{2.864613in}{3.117375in}}%
\pgfpathlineto{\pgfqpoint{2.869162in}{3.123492in}}%
\pgfpathlineto{\pgfqpoint{2.876923in}{3.133962in}}%
\pgfpathlineto{\pgfqpoint{2.880134in}{3.135032in}}%
\pgfpathlineto{\pgfqpoint{2.883078in}{3.133668in}}%
\pgfpathlineto{\pgfqpoint{2.887092in}{3.128830in}}%
\pgfpathlineto{\pgfqpoint{2.896726in}{3.116032in}}%
\pgfpathlineto{\pgfqpoint{2.899937in}{3.115497in}}%
\pgfpathlineto{\pgfqpoint{2.902881in}{3.117327in}}%
\pgfpathlineto{\pgfqpoint{2.907162in}{3.122992in}}%
\pgfpathlineto{\pgfqpoint{2.915458in}{3.134118in}}%
\pgfpathlineto{\pgfqpoint{2.918670in}{3.135007in}}%
\pgfpathlineto{\pgfqpoint{2.921613in}{3.133483in}}%
\pgfpathlineto{\pgfqpoint{2.925627in}{3.128500in}}%
\pgfpathlineto{\pgfqpoint{2.934994in}{3.116060in}}%
\pgfpathlineto{\pgfqpoint{2.938205in}{3.115485in}}%
\pgfpathlineto{\pgfqpoint{2.941149in}{3.117281in}}%
\pgfpathlineto{\pgfqpoint{2.945430in}{3.122916in}}%
\pgfpathlineto{\pgfqpoint{2.953726in}{3.134084in}}%
\pgfpathlineto{\pgfqpoint{2.956937in}{3.135013in}}%
\pgfpathlineto{\pgfqpoint{2.959881in}{3.133525in}}%
\pgfpathlineto{\pgfqpoint{2.963895in}{3.128574in}}%
\pgfpathlineto{\pgfqpoint{2.973262in}{3.116090in}}%
\pgfpathlineto{\pgfqpoint{2.976473in}{3.115473in}}%
\pgfpathlineto{\pgfqpoint{2.979416in}{3.117235in}}%
\pgfpathlineto{\pgfqpoint{2.983698in}{3.122839in}}%
\pgfpathlineto{\pgfqpoint{2.991994in}{3.134050in}}%
\pgfpathlineto{\pgfqpoint{2.995205in}{3.135020in}}%
\pgfpathlineto{\pgfqpoint{2.998149in}{3.133567in}}%
\pgfpathlineto{\pgfqpoint{3.002163in}{3.128647in}}%
\pgfpathlineto{\pgfqpoint{3.011797in}{3.115963in}}%
\pgfpathlineto{\pgfqpoint{3.015008in}{3.115530in}}%
\pgfpathlineto{\pgfqpoint{3.018220in}{3.117719in}}%
\pgfpathlineto{\pgfqpoint{3.023037in}{3.124465in}}%
\pgfpathlineto{\pgfqpoint{3.030262in}{3.134015in}}%
\pgfpathlineto{\pgfqpoint{3.033473in}{3.135025in}}%
\pgfpathlineto{\pgfqpoint{3.036417in}{3.133608in}}%
\pgfpathlineto{\pgfqpoint{3.040431in}{3.128721in}}%
\pgfpathlineto{\pgfqpoint{3.050065in}{3.115990in}}%
\pgfpathlineto{\pgfqpoint{3.053276in}{3.115516in}}%
\pgfpathlineto{\pgfqpoint{3.056488in}{3.117668in}}%
\pgfpathlineto{\pgfqpoint{3.061304in}{3.124386in}}%
\pgfpathlineto{\pgfqpoint{3.068530in}{3.133980in}}%
\pgfpathlineto{\pgfqpoint{3.071741in}{3.135030in}}%
\pgfpathlineto{\pgfqpoint{3.074685in}{3.133648in}}%
\pgfpathlineto{\pgfqpoint{3.078699in}{3.128794in}}%
\pgfpathlineto{\pgfqpoint{3.088333in}{3.116018in}}%
\pgfpathlineto{\pgfqpoint{3.091544in}{3.115503in}}%
\pgfpathlineto{\pgfqpoint{3.094488in}{3.117351in}}%
\pgfpathlineto{\pgfqpoint{3.099037in}{3.123453in}}%
\pgfpathlineto{\pgfqpoint{3.106798in}{3.133944in}}%
\pgfpathlineto{\pgfqpoint{3.110009in}{3.135034in}}%
\pgfpathlineto{\pgfqpoint{3.112953in}{3.133688in}}%
\pgfpathlineto{\pgfqpoint{3.116967in}{3.128867in}}%
\pgfpathlineto{\pgfqpoint{3.126868in}{3.115898in}}%
\pgfpathlineto{\pgfqpoint{3.130080in}{3.115566in}}%
\pgfpathlineto{\pgfqpoint{3.133291in}{3.117847in}}%
\pgfpathlineto{\pgfqpoint{3.138108in}{3.124660in}}%
\pgfpathlineto{\pgfqpoint{3.145066in}{3.133908in}}%
\pgfpathlineto{\pgfqpoint{3.148277in}{3.135037in}}%
\pgfpathlineto{\pgfqpoint{3.151221in}{3.133728in}}%
\pgfpathlineto{\pgfqpoint{3.155235in}{3.128940in}}%
\pgfpathlineto{\pgfqpoint{3.165136in}{3.115924in}}%
\pgfpathlineto{\pgfqpoint{3.168348in}{3.115551in}}%
\pgfpathlineto{\pgfqpoint{3.171559in}{3.117796in}}%
\pgfpathlineto{\pgfqpoint{3.176376in}{3.124582in}}%
\pgfpathlineto{\pgfqpoint{3.183334in}{3.133871in}}%
\pgfpathlineto{\pgfqpoint{3.186545in}{3.135040in}}%
\pgfpathlineto{\pgfqpoint{3.189489in}{3.133767in}}%
\pgfpathlineto{\pgfqpoint{3.193503in}{3.129012in}}%
\pgfpathlineto{\pgfqpoint{3.203404in}{3.115950in}}%
\pgfpathlineto{\pgfqpoint{3.206616in}{3.115537in}}%
\pgfpathlineto{\pgfqpoint{3.209827in}{3.117744in}}%
\pgfpathlineto{\pgfqpoint{3.214644in}{3.124504in}}%
\pgfpathlineto{\pgfqpoint{3.221602in}{3.133834in}}%
\pgfpathlineto{\pgfqpoint{3.224813in}{3.135043in}}%
\pgfpathlineto{\pgfqpoint{3.227757in}{3.133805in}}%
\pgfpathlineto{\pgfqpoint{3.231503in}{3.129477in}}%
\pgfpathlineto{\pgfqpoint{3.242207in}{3.115715in}}%
\pgfpathlineto{\pgfqpoint{3.245151in}{3.115607in}}%
\pgfpathlineto{\pgfqpoint{3.248362in}{3.117979in}}%
\pgfpathlineto{\pgfqpoint{3.253447in}{3.125288in}}%
\pgfpathlineto{\pgfqpoint{3.260137in}{3.133998in}}%
\pgfpathlineto{\pgfqpoint{3.263348in}{3.135027in}}%
\pgfpathlineto{\pgfqpoint{3.266292in}{3.133628in}}%
\pgfpathlineto{\pgfqpoint{3.270306in}{3.128757in}}%
\pgfpathlineto{\pgfqpoint{3.279940in}{3.116004in}}%
\pgfpathlineto{\pgfqpoint{3.283151in}{3.115509in}}%
\pgfpathlineto{\pgfqpoint{3.286095in}{3.117375in}}%
\pgfpathlineto{\pgfqpoint{3.290644in}{3.123492in}}%
\pgfpathlineto{\pgfqpoint{3.298405in}{3.133962in}}%
\pgfpathlineto{\pgfqpoint{3.301616in}{3.135032in}}%
\pgfpathlineto{\pgfqpoint{3.304560in}{3.133668in}}%
\pgfpathlineto{\pgfqpoint{3.308574in}{3.128830in}}%
\pgfpathlineto{\pgfqpoint{3.318208in}{3.116032in}}%
\pgfpathlineto{\pgfqpoint{3.321419in}{3.115497in}}%
\pgfpathlineto{\pgfqpoint{3.324363in}{3.117327in}}%
\pgfpathlineto{\pgfqpoint{3.328645in}{3.122992in}}%
\pgfpathlineto{\pgfqpoint{3.336941in}{3.134118in}}%
\pgfpathlineto{\pgfqpoint{3.340152in}{3.135007in}}%
\pgfpathlineto{\pgfqpoint{3.343096in}{3.133483in}}%
\pgfpathlineto{\pgfqpoint{3.347110in}{3.128500in}}%
\pgfpathlineto{\pgfqpoint{3.356476in}{3.116060in}}%
\pgfpathlineto{\pgfqpoint{3.359687in}{3.115485in}}%
\pgfpathlineto{\pgfqpoint{3.362631in}{3.117281in}}%
\pgfpathlineto{\pgfqpoint{3.366913in}{3.122916in}}%
\pgfpathlineto{\pgfqpoint{3.375208in}{3.134084in}}%
\pgfpathlineto{\pgfqpoint{3.378420in}{3.135013in}}%
\pgfpathlineto{\pgfqpoint{3.381363in}{3.133525in}}%
\pgfpathlineto{\pgfqpoint{3.385378in}{3.128574in}}%
\pgfpathlineto{\pgfqpoint{3.394744in}{3.116090in}}%
\pgfpathlineto{\pgfqpoint{3.397955in}{3.115473in}}%
\pgfpathlineto{\pgfqpoint{3.400899in}{3.117235in}}%
\pgfpathlineto{\pgfqpoint{3.405181in}{3.122839in}}%
\pgfpathlineto{\pgfqpoint{3.413476in}{3.134050in}}%
\pgfpathlineto{\pgfqpoint{3.416688in}{3.135020in}}%
\pgfpathlineto{\pgfqpoint{3.419631in}{3.133567in}}%
\pgfpathlineto{\pgfqpoint{3.423645in}{3.128647in}}%
\pgfpathlineto{\pgfqpoint{3.433279in}{3.115963in}}%
\pgfpathlineto{\pgfqpoint{3.436491in}{3.115530in}}%
\pgfpathlineto{\pgfqpoint{3.439702in}{3.117719in}}%
\pgfpathlineto{\pgfqpoint{3.444519in}{3.124465in}}%
\pgfpathlineto{\pgfqpoint{3.451744in}{3.134015in}}%
\pgfpathlineto{\pgfqpoint{3.454956in}{3.135025in}}%
\pgfpathlineto{\pgfqpoint{3.457899in}{3.133608in}}%
\pgfpathlineto{\pgfqpoint{3.461913in}{3.128721in}}%
\pgfpathlineto{\pgfqpoint{3.471547in}{3.115990in}}%
\pgfpathlineto{\pgfqpoint{3.474759in}{3.115516in}}%
\pgfpathlineto{\pgfqpoint{3.477970in}{3.117668in}}%
\pgfpathlineto{\pgfqpoint{3.482787in}{3.124386in}}%
\pgfpathlineto{\pgfqpoint{3.490012in}{3.133980in}}%
\pgfpathlineto{\pgfqpoint{3.493224in}{3.135030in}}%
\pgfpathlineto{\pgfqpoint{3.496167in}{3.133648in}}%
\pgfpathlineto{\pgfqpoint{3.500181in}{3.128794in}}%
\pgfpathlineto{\pgfqpoint{3.509815in}{3.116018in}}%
\pgfpathlineto{\pgfqpoint{3.513026in}{3.115503in}}%
\pgfpathlineto{\pgfqpoint{3.515970in}{3.117351in}}%
\pgfpathlineto{\pgfqpoint{3.520520in}{3.123453in}}%
\pgfpathlineto{\pgfqpoint{3.528280in}{3.133944in}}%
\pgfpathlineto{\pgfqpoint{3.531491in}{3.135034in}}%
\pgfpathlineto{\pgfqpoint{3.534435in}{3.133688in}}%
\pgfpathlineto{\pgfqpoint{3.538449in}{3.128867in}}%
\pgfpathlineto{\pgfqpoint{3.548351in}{3.115898in}}%
\pgfpathlineto{\pgfqpoint{3.551562in}{3.115566in}}%
\pgfpathlineto{\pgfqpoint{3.554773in}{3.117847in}}%
\pgfpathlineto{\pgfqpoint{3.559590in}{3.124660in}}%
\pgfpathlineto{\pgfqpoint{3.566548in}{3.133908in}}%
\pgfpathlineto{\pgfqpoint{3.569759in}{3.135037in}}%
\pgfpathlineto{\pgfqpoint{3.572703in}{3.133728in}}%
\pgfpathlineto{\pgfqpoint{3.576717in}{3.128940in}}%
\pgfpathlineto{\pgfqpoint{3.586619in}{3.115924in}}%
\pgfpathlineto{\pgfqpoint{3.589830in}{3.115551in}}%
\pgfpathlineto{\pgfqpoint{3.593041in}{3.117796in}}%
\pgfpathlineto{\pgfqpoint{3.597858in}{3.124582in}}%
\pgfpathlineto{\pgfqpoint{3.604816in}{3.133871in}}%
\pgfpathlineto{\pgfqpoint{3.608027in}{3.135040in}}%
\pgfpathlineto{\pgfqpoint{3.610971in}{3.133767in}}%
\pgfpathlineto{\pgfqpoint{3.614985in}{3.129012in}}%
\pgfpathlineto{\pgfqpoint{3.624887in}{3.115950in}}%
\pgfpathlineto{\pgfqpoint{3.628098in}{3.115537in}}%
\pgfpathlineto{\pgfqpoint{3.631309in}{3.117744in}}%
\pgfpathlineto{\pgfqpoint{3.636126in}{3.124504in}}%
\pgfpathlineto{\pgfqpoint{3.643084in}{3.133834in}}%
\pgfpathlineto{\pgfqpoint{3.646295in}{3.135043in}}%
\pgfpathlineto{\pgfqpoint{3.649239in}{3.133805in}}%
\pgfpathlineto{\pgfqpoint{3.652985in}{3.129477in}}%
\pgfpathlineto{\pgfqpoint{3.663690in}{3.115715in}}%
\pgfpathlineto{\pgfqpoint{3.666633in}{3.115607in}}%
\pgfpathlineto{\pgfqpoint{3.669845in}{3.117979in}}%
\pgfpathlineto{\pgfqpoint{3.674929in}{3.125288in}}%
\pgfpathlineto{\pgfqpoint{3.681619in}{3.133998in}}%
\pgfpathlineto{\pgfqpoint{3.684831in}{3.135027in}}%
\pgfpathlineto{\pgfqpoint{3.687774in}{3.133628in}}%
\pgfpathlineto{\pgfqpoint{3.691789in}{3.128757in}}%
\pgfpathlineto{\pgfqpoint{3.701422in}{3.116004in}}%
\pgfpathlineto{\pgfqpoint{3.704634in}{3.115509in}}%
\pgfpathlineto{\pgfqpoint{3.707577in}{3.117375in}}%
\pgfpathlineto{\pgfqpoint{3.712127in}{3.123492in}}%
\pgfpathlineto{\pgfqpoint{3.719887in}{3.133962in}}%
\pgfpathlineto{\pgfqpoint{3.723099in}{3.135032in}}%
\pgfpathlineto{\pgfqpoint{3.726042in}{3.133668in}}%
\pgfpathlineto{\pgfqpoint{3.730056in}{3.128830in}}%
\pgfpathlineto{\pgfqpoint{3.739690in}{3.116032in}}%
\pgfpathlineto{\pgfqpoint{3.742902in}{3.115497in}}%
\pgfpathlineto{\pgfqpoint{3.745845in}{3.117327in}}%
\pgfpathlineto{\pgfqpoint{3.750127in}{3.122992in}}%
\pgfpathlineto{\pgfqpoint{3.758423in}{3.134118in}}%
\pgfpathlineto{\pgfqpoint{3.761634in}{3.135007in}}%
\pgfpathlineto{\pgfqpoint{3.764578in}{3.133483in}}%
\pgfpathlineto{\pgfqpoint{3.768592in}{3.128500in}}%
\pgfpathlineto{\pgfqpoint{3.777958in}{3.116060in}}%
\pgfpathlineto{\pgfqpoint{3.781170in}{3.115485in}}%
\pgfpathlineto{\pgfqpoint{3.784113in}{3.117281in}}%
\pgfpathlineto{\pgfqpoint{3.788395in}{3.122916in}}%
\pgfpathlineto{\pgfqpoint{3.796691in}{3.134084in}}%
\pgfpathlineto{\pgfqpoint{3.799902in}{3.135013in}}%
\pgfpathlineto{\pgfqpoint{3.802846in}{3.133525in}}%
\pgfpathlineto{\pgfqpoint{3.806860in}{3.128574in}}%
\pgfpathlineto{\pgfqpoint{3.816226in}{3.116090in}}%
\pgfpathlineto{\pgfqpoint{3.819437in}{3.115473in}}%
\pgfpathlineto{\pgfqpoint{3.822381in}{3.117235in}}%
\pgfpathlineto{\pgfqpoint{3.826663in}{3.122839in}}%
\pgfpathlineto{\pgfqpoint{3.834959in}{3.134050in}}%
\pgfpathlineto{\pgfqpoint{3.838170in}{3.135020in}}%
\pgfpathlineto{\pgfqpoint{3.841114in}{3.133567in}}%
\pgfpathlineto{\pgfqpoint{3.845128in}{3.128647in}}%
\pgfpathlineto{\pgfqpoint{3.854762in}{3.115963in}}%
\pgfpathlineto{\pgfqpoint{3.857973in}{3.115530in}}%
\pgfpathlineto{\pgfqpoint{3.861184in}{3.117719in}}%
\pgfpathlineto{\pgfqpoint{3.866001in}{3.124465in}}%
\pgfpathlineto{\pgfqpoint{3.873227in}{3.134015in}}%
\pgfpathlineto{\pgfqpoint{3.876438in}{3.135025in}}%
\pgfpathlineto{\pgfqpoint{3.879382in}{3.133608in}}%
\pgfpathlineto{\pgfqpoint{3.883396in}{3.128721in}}%
\pgfpathlineto{\pgfqpoint{3.893030in}{3.115990in}}%
\pgfpathlineto{\pgfqpoint{3.896241in}{3.115516in}}%
\pgfpathlineto{\pgfqpoint{3.899452in}{3.117668in}}%
\pgfpathlineto{\pgfqpoint{3.904269in}{3.124386in}}%
\pgfpathlineto{\pgfqpoint{3.911495in}{3.133980in}}%
\pgfpathlineto{\pgfqpoint{3.914706in}{3.135030in}}%
\pgfpathlineto{\pgfqpoint{3.917650in}{3.133648in}}%
\pgfpathlineto{\pgfqpoint{3.921664in}{3.128794in}}%
\pgfpathlineto{\pgfqpoint{3.931298in}{3.116018in}}%
\pgfpathlineto{\pgfqpoint{3.934509in}{3.115503in}}%
\pgfpathlineto{\pgfqpoint{3.937452in}{3.117351in}}%
\pgfpathlineto{\pgfqpoint{3.942002in}{3.123453in}}%
\pgfpathlineto{\pgfqpoint{3.949762in}{3.133944in}}%
\pgfpathlineto{\pgfqpoint{3.952974in}{3.135034in}}%
\pgfpathlineto{\pgfqpoint{3.955917in}{3.133688in}}%
\pgfpathlineto{\pgfqpoint{3.959932in}{3.128867in}}%
\pgfpathlineto{\pgfqpoint{3.969833in}{3.115898in}}%
\pgfpathlineto{\pgfqpoint{3.973044in}{3.115566in}}%
\pgfpathlineto{\pgfqpoint{3.976256in}{3.117847in}}%
\pgfpathlineto{\pgfqpoint{3.981073in}{3.124660in}}%
\pgfpathlineto{\pgfqpoint{3.988030in}{3.133908in}}%
\pgfpathlineto{\pgfqpoint{3.991242in}{3.135037in}}%
\pgfpathlineto{\pgfqpoint{3.994185in}{3.133728in}}%
\pgfpathlineto{\pgfqpoint{3.998199in}{3.128940in}}%
\pgfpathlineto{\pgfqpoint{4.008101in}{3.115924in}}%
\pgfpathlineto{\pgfqpoint{4.011312in}{3.115551in}}%
\pgfpathlineto{\pgfqpoint{4.014524in}{3.117796in}}%
\pgfpathlineto{\pgfqpoint{4.019340in}{3.124582in}}%
\pgfpathlineto{\pgfqpoint{4.026298in}{3.133871in}}%
\pgfpathlineto{\pgfqpoint{4.029510in}{3.135040in}}%
\pgfpathlineto{\pgfqpoint{4.032453in}{3.133767in}}%
\pgfpathlineto{\pgfqpoint{4.036467in}{3.129012in}}%
\pgfpathlineto{\pgfqpoint{4.046369in}{3.115950in}}%
\pgfpathlineto{\pgfqpoint{4.049580in}{3.115537in}}%
\pgfpathlineto{\pgfqpoint{4.052791in}{3.117744in}}%
\pgfpathlineto{\pgfqpoint{4.057608in}{3.124504in}}%
\pgfpathlineto{\pgfqpoint{4.064566in}{3.133834in}}%
\pgfpathlineto{\pgfqpoint{4.067778in}{3.135043in}}%
\pgfpathlineto{\pgfqpoint{4.070721in}{3.133805in}}%
\pgfpathlineto{\pgfqpoint{4.074468in}{3.129477in}}%
\pgfpathlineto{\pgfqpoint{4.085172in}{3.115715in}}%
\pgfpathlineto{\pgfqpoint{4.088116in}{3.115607in}}%
\pgfpathlineto{\pgfqpoint{4.091327in}{3.117979in}}%
\pgfpathlineto{\pgfqpoint{4.096412in}{3.125288in}}%
\pgfpathlineto{\pgfqpoint{4.103102in}{3.133998in}}%
\pgfpathlineto{\pgfqpoint{4.106313in}{3.135027in}}%
\pgfpathlineto{\pgfqpoint{4.109257in}{3.133628in}}%
\pgfpathlineto{\pgfqpoint{4.113271in}{3.128757in}}%
\pgfpathlineto{\pgfqpoint{4.122905in}{3.116004in}}%
\pgfpathlineto{\pgfqpoint{4.126116in}{3.115509in}}%
\pgfpathlineto{\pgfqpoint{4.129060in}{3.117375in}}%
\pgfpathlineto{\pgfqpoint{4.133609in}{3.123492in}}%
\pgfpathlineto{\pgfqpoint{4.141370in}{3.133962in}}%
\pgfpathlineto{\pgfqpoint{4.144581in}{3.135032in}}%
\pgfpathlineto{\pgfqpoint{4.147525in}{3.133668in}}%
\pgfpathlineto{\pgfqpoint{4.151539in}{3.128830in}}%
\pgfpathlineto{\pgfqpoint{4.161173in}{3.116032in}}%
\pgfpathlineto{\pgfqpoint{4.164384in}{3.115497in}}%
\pgfpathlineto{\pgfqpoint{4.167328in}{3.117327in}}%
\pgfpathlineto{\pgfqpoint{4.171609in}{3.122992in}}%
\pgfpathlineto{\pgfqpoint{4.179905in}{3.134118in}}%
\pgfpathlineto{\pgfqpoint{4.183116in}{3.135007in}}%
\pgfpathlineto{\pgfqpoint{4.186060in}{3.133483in}}%
\pgfpathlineto{\pgfqpoint{4.190074in}{3.128500in}}%
\pgfpathlineto{\pgfqpoint{4.199441in}{3.116060in}}%
\pgfpathlineto{\pgfqpoint{4.202652in}{3.115485in}}%
\pgfpathlineto{\pgfqpoint{4.205596in}{3.117281in}}%
\pgfpathlineto{\pgfqpoint{4.209877in}{3.122916in}}%
\pgfpathlineto{\pgfqpoint{4.218173in}{3.134084in}}%
\pgfpathlineto{\pgfqpoint{4.221384in}{3.135013in}}%
\pgfpathlineto{\pgfqpoint{4.224328in}{3.133525in}}%
\pgfpathlineto{\pgfqpoint{4.228342in}{3.128574in}}%
\pgfpathlineto{\pgfqpoint{4.237708in}{3.116090in}}%
\pgfpathlineto{\pgfqpoint{4.240920in}{3.115473in}}%
\pgfpathlineto{\pgfqpoint{4.243863in}{3.117235in}}%
\pgfpathlineto{\pgfqpoint{4.248145in}{3.122839in}}%
\pgfpathlineto{\pgfqpoint{4.256441in}{3.134050in}}%
\pgfpathlineto{\pgfqpoint{4.259652in}{3.135020in}}%
\pgfpathlineto{\pgfqpoint{4.262596in}{3.133567in}}%
\pgfpathlineto{\pgfqpoint{4.266610in}{3.128647in}}%
\pgfpathlineto{\pgfqpoint{4.276244in}{3.115963in}}%
\pgfpathlineto{\pgfqpoint{4.279455in}{3.115530in}}%
\pgfpathlineto{\pgfqpoint{4.282667in}{3.117719in}}%
\pgfpathlineto{\pgfqpoint{4.287484in}{3.124465in}}%
\pgfpathlineto{\pgfqpoint{4.294709in}{3.134015in}}%
\pgfpathlineto{\pgfqpoint{4.297920in}{3.135025in}}%
\pgfpathlineto{\pgfqpoint{4.300864in}{3.133608in}}%
\pgfpathlineto{\pgfqpoint{4.304878in}{3.128721in}}%
\pgfpathlineto{\pgfqpoint{4.314512in}{3.115990in}}%
\pgfpathlineto{\pgfqpoint{4.317723in}{3.115516in}}%
\pgfpathlineto{\pgfqpoint{4.320934in}{3.117668in}}%
\pgfpathlineto{\pgfqpoint{4.325751in}{3.124386in}}%
\pgfpathlineto{\pgfqpoint{4.332977in}{3.133980in}}%
\pgfpathlineto{\pgfqpoint{4.336188in}{3.135030in}}%
\pgfpathlineto{\pgfqpoint{4.339132in}{3.133648in}}%
\pgfpathlineto{\pgfqpoint{4.343146in}{3.128794in}}%
\pgfpathlineto{\pgfqpoint{4.352780in}{3.116018in}}%
\pgfpathlineto{\pgfqpoint{4.355991in}{3.115503in}}%
\pgfpathlineto{\pgfqpoint{4.358935in}{3.117351in}}%
\pgfpathlineto{\pgfqpoint{4.363484in}{3.123453in}}%
\pgfpathlineto{\pgfqpoint{4.371245in}{3.133944in}}%
\pgfpathlineto{\pgfqpoint{4.374456in}{3.135034in}}%
\pgfpathlineto{\pgfqpoint{4.377400in}{3.133688in}}%
\pgfpathlineto{\pgfqpoint{4.381414in}{3.128867in}}%
\pgfpathlineto{\pgfqpoint{4.391315in}{3.115898in}}%
\pgfpathlineto{\pgfqpoint{4.394527in}{3.115566in}}%
\pgfpathlineto{\pgfqpoint{4.397738in}{3.117847in}}%
\pgfpathlineto{\pgfqpoint{4.402555in}{3.124660in}}%
\pgfpathlineto{\pgfqpoint{4.409513in}{3.133908in}}%
\pgfpathlineto{\pgfqpoint{4.412724in}{3.135037in}}%
\pgfpathlineto{\pgfqpoint{4.415668in}{3.133728in}}%
\pgfpathlineto{\pgfqpoint{4.419682in}{3.128940in}}%
\pgfpathlineto{\pgfqpoint{4.429583in}{3.115924in}}%
\pgfpathlineto{\pgfqpoint{4.432795in}{3.115551in}}%
\pgfpathlineto{\pgfqpoint{4.436006in}{3.117796in}}%
\pgfpathlineto{\pgfqpoint{4.440823in}{3.124582in}}%
\pgfpathlineto{\pgfqpoint{4.447781in}{3.133871in}}%
\pgfpathlineto{\pgfqpoint{4.450992in}{3.135040in}}%
\pgfpathlineto{\pgfqpoint{4.453936in}{3.133767in}}%
\pgfpathlineto{\pgfqpoint{4.457950in}{3.129012in}}%
\pgfpathlineto{\pgfqpoint{4.467851in}{3.115950in}}%
\pgfpathlineto{\pgfqpoint{4.471062in}{3.115537in}}%
\pgfpathlineto{\pgfqpoint{4.474274in}{3.117744in}}%
\pgfpathlineto{\pgfqpoint{4.479091in}{3.124504in}}%
\pgfpathlineto{\pgfqpoint{4.486049in}{3.133834in}}%
\pgfpathlineto{\pgfqpoint{4.489260in}{3.135043in}}%
\pgfpathlineto{\pgfqpoint{4.492204in}{3.133805in}}%
\pgfpathlineto{\pgfqpoint{4.495950in}{3.129477in}}%
\pgfpathlineto{\pgfqpoint{4.506654in}{3.115715in}}%
\pgfpathlineto{\pgfqpoint{4.509598in}{3.115607in}}%
\pgfpathlineto{\pgfqpoint{4.512809in}{3.117979in}}%
\pgfpathlineto{\pgfqpoint{4.517894in}{3.125288in}}%
\pgfpathlineto{\pgfqpoint{4.524584in}{3.133998in}}%
\pgfpathlineto{\pgfqpoint{4.527795in}{3.135027in}}%
\pgfpathlineto{\pgfqpoint{4.530739in}{3.133628in}}%
\pgfpathlineto{\pgfqpoint{4.534753in}{3.128757in}}%
\pgfpathlineto{\pgfqpoint{4.544387in}{3.116004in}}%
\pgfpathlineto{\pgfqpoint{4.547598in}{3.115509in}}%
\pgfpathlineto{\pgfqpoint{4.550542in}{3.117375in}}%
\pgfpathlineto{\pgfqpoint{4.555091in}{3.123492in}}%
\pgfpathlineto{\pgfqpoint{4.562852in}{3.133962in}}%
\pgfpathlineto{\pgfqpoint{4.566063in}{3.135032in}}%
\pgfpathlineto{\pgfqpoint{4.569007in}{3.133668in}}%
\pgfpathlineto{\pgfqpoint{4.573021in}{3.128830in}}%
\pgfpathlineto{\pgfqpoint{4.582655in}{3.116032in}}%
\pgfpathlineto{\pgfqpoint{4.585866in}{3.115497in}}%
\pgfpathlineto{\pgfqpoint{4.588810in}{3.117327in}}%
\pgfpathlineto{\pgfqpoint{4.593092in}{3.122992in}}%
\pgfpathlineto{\pgfqpoint{4.601387in}{3.134118in}}%
\pgfpathlineto{\pgfqpoint{4.604599in}{3.135007in}}%
\pgfpathlineto{\pgfqpoint{4.607542in}{3.133483in}}%
\pgfpathlineto{\pgfqpoint{4.611557in}{3.128500in}}%
\pgfpathlineto{\pgfqpoint{4.620923in}{3.116060in}}%
\pgfpathlineto{\pgfqpoint{4.624134in}{3.115485in}}%
\pgfpathlineto{\pgfqpoint{4.627078in}{3.117281in}}%
\pgfpathlineto{\pgfqpoint{4.631360in}{3.122916in}}%
\pgfpathlineto{\pgfqpoint{4.639655in}{3.134084in}}%
\pgfpathlineto{\pgfqpoint{4.642867in}{3.135013in}}%
\pgfpathlineto{\pgfqpoint{4.645810in}{3.133525in}}%
\pgfpathlineto{\pgfqpoint{4.649825in}{3.128574in}}%
\pgfpathlineto{\pgfqpoint{4.659191in}{3.116090in}}%
\pgfpathlineto{\pgfqpoint{4.662402in}{3.115473in}}%
\pgfpathlineto{\pgfqpoint{4.665346in}{3.117235in}}%
\pgfpathlineto{\pgfqpoint{4.669627in}{3.122839in}}%
\pgfpathlineto{\pgfqpoint{4.677923in}{3.134050in}}%
\pgfpathlineto{\pgfqpoint{4.681135in}{3.135020in}}%
\pgfpathlineto{\pgfqpoint{4.684078in}{3.133567in}}%
\pgfpathlineto{\pgfqpoint{4.688092in}{3.128647in}}%
\pgfpathlineto{\pgfqpoint{4.697726in}{3.115963in}}%
\pgfpathlineto{\pgfqpoint{4.700938in}{3.115530in}}%
\pgfpathlineto{\pgfqpoint{4.704149in}{3.117719in}}%
\pgfpathlineto{\pgfqpoint{4.708966in}{3.124465in}}%
\pgfpathlineto{\pgfqpoint{4.716191in}{3.134015in}}%
\pgfpathlineto{\pgfqpoint{4.719403in}{3.135025in}}%
\pgfpathlineto{\pgfqpoint{4.722346in}{3.133608in}}%
\pgfpathlineto{\pgfqpoint{4.726360in}{3.128721in}}%
\pgfpathlineto{\pgfqpoint{4.735994in}{3.115990in}}%
\pgfpathlineto{\pgfqpoint{4.739206in}{3.115516in}}%
\pgfpathlineto{\pgfqpoint{4.742417in}{3.117668in}}%
\pgfpathlineto{\pgfqpoint{4.747234in}{3.124386in}}%
\pgfpathlineto{\pgfqpoint{4.754459in}{3.133980in}}%
\pgfpathlineto{\pgfqpoint{4.757670in}{3.135030in}}%
\pgfpathlineto{\pgfqpoint{4.760614in}{3.133648in}}%
\pgfpathlineto{\pgfqpoint{4.764628in}{3.128794in}}%
\pgfpathlineto{\pgfqpoint{4.774262in}{3.116018in}}%
\pgfpathlineto{\pgfqpoint{4.777473in}{3.115503in}}%
\pgfpathlineto{\pgfqpoint{4.780417in}{3.117351in}}%
\pgfpathlineto{\pgfqpoint{4.784966in}{3.123453in}}%
\pgfpathlineto{\pgfqpoint{4.792727in}{3.133944in}}%
\pgfpathlineto{\pgfqpoint{4.795938in}{3.135034in}}%
\pgfpathlineto{\pgfqpoint{4.798882in}{3.133688in}}%
\pgfpathlineto{\pgfqpoint{4.802896in}{3.128867in}}%
\pgfpathlineto{\pgfqpoint{4.812798in}{3.115898in}}%
\pgfpathlineto{\pgfqpoint{4.816009in}{3.115566in}}%
\pgfpathlineto{\pgfqpoint{4.819220in}{3.117847in}}%
\pgfpathlineto{\pgfqpoint{4.824037in}{3.124660in}}%
\pgfpathlineto{\pgfqpoint{4.830995in}{3.133908in}}%
\pgfpathlineto{\pgfqpoint{4.834206in}{3.135037in}}%
\pgfpathlineto{\pgfqpoint{4.837150in}{3.133728in}}%
\pgfpathlineto{\pgfqpoint{4.841164in}{3.128940in}}%
\pgfpathlineto{\pgfqpoint{4.851066in}{3.115924in}}%
\pgfpathlineto{\pgfqpoint{4.854277in}{3.115551in}}%
\pgfpathlineto{\pgfqpoint{4.857488in}{3.117796in}}%
\pgfpathlineto{\pgfqpoint{4.862305in}{3.124582in}}%
\pgfpathlineto{\pgfqpoint{4.869263in}{3.133871in}}%
\pgfpathlineto{\pgfqpoint{4.872474in}{3.135040in}}%
\pgfpathlineto{\pgfqpoint{4.875418in}{3.133767in}}%
\pgfpathlineto{\pgfqpoint{4.879432in}{3.129012in}}%
\pgfpathlineto{\pgfqpoint{4.889334in}{3.115950in}}%
\pgfpathlineto{\pgfqpoint{4.892545in}{3.115537in}}%
\pgfpathlineto{\pgfqpoint{4.895756in}{3.117744in}}%
\pgfpathlineto{\pgfqpoint{4.900573in}{3.124504in}}%
\pgfpathlineto{\pgfqpoint{4.907531in}{3.133834in}}%
\pgfpathlineto{\pgfqpoint{4.910742in}{3.135043in}}%
\pgfpathlineto{\pgfqpoint{4.913686in}{3.133805in}}%
\pgfpathlineto{\pgfqpoint{4.917432in}{3.129477in}}%
\pgfpathlineto{\pgfqpoint{4.928137in}{3.115715in}}%
\pgfpathlineto{\pgfqpoint{4.931080in}{3.115607in}}%
\pgfpathlineto{\pgfqpoint{4.934292in}{3.117979in}}%
\pgfpathlineto{\pgfqpoint{4.939376in}{3.125288in}}%
\pgfpathlineto{\pgfqpoint{4.946066in}{3.133998in}}%
\pgfpathlineto{\pgfqpoint{4.949278in}{3.135027in}}%
\pgfpathlineto{\pgfqpoint{4.952221in}{3.133628in}}%
\pgfpathlineto{\pgfqpoint{4.956235in}{3.128757in}}%
\pgfpathlineto{\pgfqpoint{4.965869in}{3.116004in}}%
\pgfpathlineto{\pgfqpoint{4.969081in}{3.115509in}}%
\pgfpathlineto{\pgfqpoint{4.972024in}{3.117375in}}%
\pgfpathlineto{\pgfqpoint{4.976574in}{3.123492in}}%
\pgfpathlineto{\pgfqpoint{4.984334in}{3.133962in}}%
\pgfpathlineto{\pgfqpoint{4.987546in}{3.135032in}}%
\pgfpathlineto{\pgfqpoint{4.990489in}{3.133668in}}%
\pgfpathlineto{\pgfqpoint{4.994503in}{3.128830in}}%
\pgfpathlineto{\pgfqpoint{5.004137in}{3.116032in}}%
\pgfpathlineto{\pgfqpoint{5.007349in}{3.115497in}}%
\pgfpathlineto{\pgfqpoint{5.010292in}{3.117327in}}%
\pgfpathlineto{\pgfqpoint{5.014574in}{3.122992in}}%
\pgfpathlineto{\pgfqpoint{5.022870in}{3.134118in}}%
\pgfpathlineto{\pgfqpoint{5.026081in}{3.135007in}}%
\pgfpathlineto{\pgfqpoint{5.029025in}{3.133483in}}%
\pgfpathlineto{\pgfqpoint{5.033039in}{3.128500in}}%
\pgfpathlineto{\pgfqpoint{5.042405in}{3.116060in}}%
\pgfpathlineto{\pgfqpoint{5.045616in}{3.115485in}}%
\pgfpathlineto{\pgfqpoint{5.048560in}{3.117281in}}%
\pgfpathlineto{\pgfqpoint{5.052842in}{3.122916in}}%
\pgfpathlineto{\pgfqpoint{5.061138in}{3.134084in}}%
\pgfpathlineto{\pgfqpoint{5.064349in}{3.135013in}}%
\pgfpathlineto{\pgfqpoint{5.067293in}{3.133525in}}%
\pgfpathlineto{\pgfqpoint{5.071307in}{3.128574in}}%
\pgfpathlineto{\pgfqpoint{5.080673in}{3.116090in}}%
\pgfpathlineto{\pgfqpoint{5.083884in}{3.115473in}}%
\pgfpathlineto{\pgfqpoint{5.086828in}{3.117235in}}%
\pgfpathlineto{\pgfqpoint{5.091110in}{3.122839in}}%
\pgfpathlineto{\pgfqpoint{5.099406in}{3.134050in}}%
\pgfpathlineto{\pgfqpoint{5.102617in}{3.135020in}}%
\pgfpathlineto{\pgfqpoint{5.105561in}{3.133567in}}%
\pgfpathlineto{\pgfqpoint{5.109575in}{3.128647in}}%
\pgfpathlineto{\pgfqpoint{5.119209in}{3.115963in}}%
\pgfpathlineto{\pgfqpoint{5.122420in}{3.115530in}}%
\pgfpathlineto{\pgfqpoint{5.125631in}{3.117719in}}%
\pgfpathlineto{\pgfqpoint{5.130448in}{3.124465in}}%
\pgfpathlineto{\pgfqpoint{5.137674in}{3.134015in}}%
\pgfpathlineto{\pgfqpoint{5.140885in}{3.135025in}}%
\pgfpathlineto{\pgfqpoint{5.143829in}{3.133608in}}%
\pgfpathlineto{\pgfqpoint{5.147843in}{3.128721in}}%
\pgfpathlineto{\pgfqpoint{5.157477in}{3.115990in}}%
\pgfpathlineto{\pgfqpoint{5.160688in}{3.115516in}}%
\pgfpathlineto{\pgfqpoint{5.163899in}{3.117668in}}%
\pgfpathlineto{\pgfqpoint{5.168716in}{3.124386in}}%
\pgfpathlineto{\pgfqpoint{5.175941in}{3.133980in}}%
\pgfpathlineto{\pgfqpoint{5.179153in}{3.135030in}}%
\pgfpathlineto{\pgfqpoint{5.182096in}{3.133648in}}%
\pgfpathlineto{\pgfqpoint{5.186111in}{3.128794in}}%
\pgfpathlineto{\pgfqpoint{5.195744in}{3.116018in}}%
\pgfpathlineto{\pgfqpoint{5.198956in}{3.115503in}}%
\pgfpathlineto{\pgfqpoint{5.201899in}{3.117351in}}%
\pgfpathlineto{\pgfqpoint{5.206449in}{3.123453in}}%
\pgfpathlineto{\pgfqpoint{5.214209in}{3.133944in}}%
\pgfpathlineto{\pgfqpoint{5.217421in}{3.135034in}}%
\pgfpathlineto{\pgfqpoint{5.220364in}{3.133688in}}%
\pgfpathlineto{\pgfqpoint{5.224379in}{3.128867in}}%
\pgfpathlineto{\pgfqpoint{5.234280in}{3.115898in}}%
\pgfpathlineto{\pgfqpoint{5.237491in}{3.115566in}}%
\pgfpathlineto{\pgfqpoint{5.240703in}{3.117847in}}%
\pgfpathlineto{\pgfqpoint{5.245520in}{3.124660in}}%
\pgfpathlineto{\pgfqpoint{5.252477in}{3.133908in}}%
\pgfpathlineto{\pgfqpoint{5.255689in}{3.135037in}}%
\pgfpathlineto{\pgfqpoint{5.258632in}{3.133728in}}%
\pgfpathlineto{\pgfqpoint{5.262646in}{3.128940in}}%
\pgfpathlineto{\pgfqpoint{5.272548in}{3.115924in}}%
\pgfpathlineto{\pgfqpoint{5.275759in}{3.115551in}}%
\pgfpathlineto{\pgfqpoint{5.278970in}{3.117796in}}%
\pgfpathlineto{\pgfqpoint{5.283787in}{3.124582in}}%
\pgfpathlineto{\pgfqpoint{5.290745in}{3.133871in}}%
\pgfpathlineto{\pgfqpoint{5.293957in}{3.135040in}}%
\pgfpathlineto{\pgfqpoint{5.296900in}{3.133767in}}%
\pgfpathlineto{\pgfqpoint{5.300914in}{3.129012in}}%
\pgfpathlineto{\pgfqpoint{5.310816in}{3.115950in}}%
\pgfpathlineto{\pgfqpoint{5.314027in}{3.115537in}}%
\pgfpathlineto{\pgfqpoint{5.317238in}{3.117744in}}%
\pgfpathlineto{\pgfqpoint{5.322055in}{3.124504in}}%
\pgfpathlineto{\pgfqpoint{5.329013in}{3.133834in}}%
\pgfpathlineto{\pgfqpoint{5.332224in}{3.135043in}}%
\pgfpathlineto{\pgfqpoint{5.335168in}{3.133805in}}%
\pgfpathlineto{\pgfqpoint{5.338915in}{3.129477in}}%
\pgfpathlineto{\pgfqpoint{5.349619in}{3.115715in}}%
\pgfpathlineto{\pgfqpoint{5.352563in}{3.115607in}}%
\pgfpathlineto{\pgfqpoint{5.355774in}{3.117979in}}%
\pgfpathlineto{\pgfqpoint{5.360858in}{3.125288in}}%
\pgfpathlineto{\pgfqpoint{5.367549in}{3.133998in}}%
\pgfpathlineto{\pgfqpoint{5.370760in}{3.135027in}}%
\pgfpathlineto{\pgfqpoint{5.373704in}{3.133628in}}%
\pgfpathlineto{\pgfqpoint{5.377718in}{3.128757in}}%
\pgfpathlineto{\pgfqpoint{5.387352in}{3.116004in}}%
\pgfpathlineto{\pgfqpoint{5.390563in}{3.115509in}}%
\pgfpathlineto{\pgfqpoint{5.393507in}{3.117375in}}%
\pgfpathlineto{\pgfqpoint{5.398056in}{3.123492in}}%
\pgfpathlineto{\pgfqpoint{5.405817in}{3.133962in}}%
\pgfpathlineto{\pgfqpoint{5.409028in}{3.135032in}}%
\pgfpathlineto{\pgfqpoint{5.411972in}{3.133668in}}%
\pgfpathlineto{\pgfqpoint{5.415986in}{3.128830in}}%
\pgfpathlineto{\pgfqpoint{5.425620in}{3.116032in}}%
\pgfpathlineto{\pgfqpoint{5.428831in}{3.115497in}}%
\pgfpathlineto{\pgfqpoint{5.431775in}{3.117327in}}%
\pgfpathlineto{\pgfqpoint{5.436056in}{3.122992in}}%
\pgfpathlineto{\pgfqpoint{5.444352in}{3.134118in}}%
\pgfpathlineto{\pgfqpoint{5.447563in}{3.135007in}}%
\pgfpathlineto{\pgfqpoint{5.450507in}{3.133483in}}%
\pgfpathlineto{\pgfqpoint{5.454521in}{3.128500in}}%
\pgfpathlineto{\pgfqpoint{5.463888in}{3.116060in}}%
\pgfpathlineto{\pgfqpoint{5.467099in}{3.115485in}}%
\pgfpathlineto{\pgfqpoint{5.470042in}{3.117281in}}%
\pgfpathlineto{\pgfqpoint{5.474324in}{3.122916in}}%
\pgfpathlineto{\pgfqpoint{5.482620in}{3.134084in}}%
\pgfpathlineto{\pgfqpoint{5.485831in}{3.135013in}}%
\pgfpathlineto{\pgfqpoint{5.488775in}{3.133525in}}%
\pgfpathlineto{\pgfqpoint{5.492789in}{3.128574in}}%
\pgfpathlineto{\pgfqpoint{5.502155in}{3.116090in}}%
\pgfpathlineto{\pgfqpoint{5.505367in}{3.115473in}}%
\pgfpathlineto{\pgfqpoint{5.508310in}{3.117235in}}%
\pgfpathlineto{\pgfqpoint{5.512592in}{3.122839in}}%
\pgfpathlineto{\pgfqpoint{5.520888in}{3.134050in}}%
\pgfpathlineto{\pgfqpoint{5.524099in}{3.135020in}}%
\pgfpathlineto{\pgfqpoint{5.527043in}{3.133567in}}%
\pgfpathlineto{\pgfqpoint{5.531057in}{3.128647in}}%
\pgfpathlineto{\pgfqpoint{5.540691in}{3.115963in}}%
\pgfpathlineto{\pgfqpoint{5.543902in}{3.115530in}}%
\pgfpathlineto{\pgfqpoint{5.547114in}{3.117719in}}%
\pgfpathlineto{\pgfqpoint{5.551930in}{3.124465in}}%
\pgfpathlineto{\pgfqpoint{5.559156in}{3.134015in}}%
\pgfpathlineto{\pgfqpoint{5.562367in}{3.135025in}}%
\pgfpathlineto{\pgfqpoint{5.565311in}{3.133608in}}%
\pgfpathlineto{\pgfqpoint{5.569325in}{3.128721in}}%
\pgfpathlineto{\pgfqpoint{5.578959in}{3.115990in}}%
\pgfpathlineto{\pgfqpoint{5.582170in}{3.115516in}}%
\pgfpathlineto{\pgfqpoint{5.585381in}{3.117668in}}%
\pgfpathlineto{\pgfqpoint{5.590198in}{3.124386in}}%
\pgfpathlineto{\pgfqpoint{5.597424in}{3.133980in}}%
\pgfpathlineto{\pgfqpoint{5.600635in}{3.135030in}}%
\pgfpathlineto{\pgfqpoint{5.603579in}{3.133648in}}%
\pgfpathlineto{\pgfqpoint{5.607593in}{3.128794in}}%
\pgfpathlineto{\pgfqpoint{5.617227in}{3.116018in}}%
\pgfpathlineto{\pgfqpoint{5.620438in}{3.115503in}}%
\pgfpathlineto{\pgfqpoint{5.623382in}{3.117351in}}%
\pgfpathlineto{\pgfqpoint{5.627931in}{3.123453in}}%
\pgfpathlineto{\pgfqpoint{5.635692in}{3.133944in}}%
\pgfpathlineto{\pgfqpoint{5.638903in}{3.135034in}}%
\pgfpathlineto{\pgfqpoint{5.641847in}{3.133688in}}%
\pgfpathlineto{\pgfqpoint{5.645861in}{3.128867in}}%
\pgfpathlineto{\pgfqpoint{5.655762in}{3.115898in}}%
\pgfpathlineto{\pgfqpoint{5.658974in}{3.115566in}}%
\pgfpathlineto{\pgfqpoint{5.662185in}{3.117847in}}%
\pgfpathlineto{\pgfqpoint{5.667002in}{3.124660in}}%
\pgfpathlineto{\pgfqpoint{5.673960in}{3.133908in}}%
\pgfpathlineto{\pgfqpoint{5.677171in}{3.135037in}}%
\pgfpathlineto{\pgfqpoint{5.680115in}{3.133728in}}%
\pgfpathlineto{\pgfqpoint{5.684129in}{3.128940in}}%
\pgfpathlineto{\pgfqpoint{5.694030in}{3.115924in}}%
\pgfpathlineto{\pgfqpoint{5.697242in}{3.115551in}}%
\pgfpathlineto{\pgfqpoint{5.700453in}{3.117796in}}%
\pgfpathlineto{\pgfqpoint{5.705270in}{3.124582in}}%
\pgfpathlineto{\pgfqpoint{5.712228in}{3.133871in}}%
\pgfpathlineto{\pgfqpoint{5.715439in}{3.135040in}}%
\pgfpathlineto{\pgfqpoint{5.718383in}{3.133767in}}%
\pgfpathlineto{\pgfqpoint{5.722397in}{3.129012in}}%
\pgfpathlineto{\pgfqpoint{5.732298in}{3.115950in}}%
\pgfpathlineto{\pgfqpoint{5.735509in}{3.115537in}}%
\pgfpathlineto{\pgfqpoint{5.738721in}{3.117744in}}%
\pgfpathlineto{\pgfqpoint{5.743538in}{3.124504in}}%
\pgfpathlineto{\pgfqpoint{5.750495in}{3.133834in}}%
\pgfpathlineto{\pgfqpoint{5.753707in}{3.135043in}}%
\pgfpathlineto{\pgfqpoint{5.756650in}{3.133805in}}%
\pgfpathlineto{\pgfqpoint{5.760397in}{3.129477in}}%
\pgfpathlineto{\pgfqpoint{5.771101in}{3.115715in}}%
\pgfpathlineto{\pgfqpoint{5.774045in}{3.115607in}}%
\pgfpathlineto{\pgfqpoint{5.777256in}{3.117979in}}%
\pgfpathlineto{\pgfqpoint{5.782341in}{3.125288in}}%
\pgfpathlineto{\pgfqpoint{5.789031in}{3.133998in}}%
\pgfpathlineto{\pgfqpoint{5.792242in}{3.135027in}}%
\pgfpathlineto{\pgfqpoint{5.795186in}{3.133628in}}%
\pgfpathlineto{\pgfqpoint{5.799200in}{3.128757in}}%
\pgfpathlineto{\pgfqpoint{5.808834in}{3.116004in}}%
\pgfpathlineto{\pgfqpoint{5.812045in}{3.115509in}}%
\pgfpathlineto{\pgfqpoint{5.814989in}{3.117375in}}%
\pgfpathlineto{\pgfqpoint{5.819538in}{3.123492in}}%
\pgfpathlineto{\pgfqpoint{5.827299in}{3.133962in}}%
\pgfpathlineto{\pgfqpoint{5.830510in}{3.135032in}}%
\pgfpathlineto{\pgfqpoint{5.833454in}{3.133668in}}%
\pgfpathlineto{\pgfqpoint{5.837468in}{3.128830in}}%
\pgfpathlineto{\pgfqpoint{5.847102in}{3.116032in}}%
\pgfpathlineto{\pgfqpoint{5.850313in}{3.115497in}}%
\pgfpathlineto{\pgfqpoint{5.853257in}{3.117327in}}%
\pgfpathlineto{\pgfqpoint{5.857539in}{3.122992in}}%
\pgfpathlineto{\pgfqpoint{5.865834in}{3.134118in}}%
\pgfpathlineto{\pgfqpoint{5.869046in}{3.135007in}}%
\pgfpathlineto{\pgfqpoint{5.871989in}{3.133483in}}%
\pgfpathlineto{\pgfqpoint{5.876004in}{3.128500in}}%
\pgfpathlineto{\pgfqpoint{5.885370in}{3.116060in}}%
\pgfpathlineto{\pgfqpoint{5.888581in}{3.115485in}}%
\pgfpathlineto{\pgfqpoint{5.891525in}{3.117281in}}%
\pgfpathlineto{\pgfqpoint{5.895807in}{3.122916in}}%
\pgfpathlineto{\pgfqpoint{5.904102in}{3.134084in}}%
\pgfpathlineto{\pgfqpoint{5.907314in}{3.135013in}}%
\pgfpathlineto{\pgfqpoint{5.910257in}{3.133525in}}%
\pgfpathlineto{\pgfqpoint{5.914271in}{3.128574in}}%
\pgfpathlineto{\pgfqpoint{5.923638in}{3.116090in}}%
\pgfpathlineto{\pgfqpoint{5.926849in}{3.115473in}}%
\pgfpathlineto{\pgfqpoint{5.929793in}{3.117235in}}%
\pgfpathlineto{\pgfqpoint{5.934074in}{3.122839in}}%
\pgfpathlineto{\pgfqpoint{5.942370in}{3.134050in}}%
\pgfpathlineto{\pgfqpoint{5.945582in}{3.135020in}}%
\pgfpathlineto{\pgfqpoint{5.948525in}{3.133567in}}%
\pgfpathlineto{\pgfqpoint{5.952539in}{3.128647in}}%
\pgfpathlineto{\pgfqpoint{5.962173in}{3.115963in}}%
\pgfpathlineto{\pgfqpoint{5.965385in}{3.115530in}}%
\pgfpathlineto{\pgfqpoint{5.968596in}{3.117719in}}%
\pgfpathlineto{\pgfqpoint{5.973413in}{3.124465in}}%
\pgfpathlineto{\pgfqpoint{5.980638in}{3.134015in}}%
\pgfpathlineto{\pgfqpoint{5.983849in}{3.135025in}}%
\pgfpathlineto{\pgfqpoint{5.986793in}{3.133608in}}%
\pgfpathlineto{\pgfqpoint{5.990807in}{3.128721in}}%
\pgfpathlineto{\pgfqpoint{6.000441in}{3.115990in}}%
\pgfpathlineto{\pgfqpoint{6.003652in}{3.115516in}}%
\pgfpathlineto{\pgfqpoint{6.006864in}{3.117668in}}%
\pgfpathlineto{\pgfqpoint{6.011681in}{3.124386in}}%
\pgfpathlineto{\pgfqpoint{6.018906in}{3.133980in}}%
\pgfpathlineto{\pgfqpoint{6.022117in}{3.135030in}}%
\pgfpathlineto{\pgfqpoint{6.025061in}{3.133648in}}%
\pgfpathlineto{\pgfqpoint{6.029075in}{3.128794in}}%
\pgfpathlineto{\pgfqpoint{6.038709in}{3.116018in}}%
\pgfpathlineto{\pgfqpoint{6.041920in}{3.115503in}}%
\pgfpathlineto{\pgfqpoint{6.044864in}{3.117351in}}%
\pgfpathlineto{\pgfqpoint{6.049413in}{3.123453in}}%
\pgfpathlineto{\pgfqpoint{6.057174in}{3.133944in}}%
\pgfpathlineto{\pgfqpoint{6.060385in}{3.135034in}}%
\pgfpathlineto{\pgfqpoint{6.063329in}{3.133688in}}%
\pgfpathlineto{\pgfqpoint{6.067343in}{3.128867in}}%
\pgfpathlineto{\pgfqpoint{6.077245in}{3.115898in}}%
\pgfpathlineto{\pgfqpoint{6.080456in}{3.115566in}}%
\pgfpathlineto{\pgfqpoint{6.083667in}{3.117847in}}%
\pgfpathlineto{\pgfqpoint{6.088484in}{3.124660in}}%
\pgfpathlineto{\pgfqpoint{6.095442in}{3.133908in}}%
\pgfpathlineto{\pgfqpoint{6.098653in}{3.135037in}}%
\pgfpathlineto{\pgfqpoint{6.101597in}{3.133728in}}%
\pgfpathlineto{\pgfqpoint{6.105611in}{3.128940in}}%
\pgfpathlineto{\pgfqpoint{6.115513in}{3.115924in}}%
\pgfpathlineto{\pgfqpoint{6.118724in}{3.115551in}}%
\pgfpathlineto{\pgfqpoint{6.121935in}{3.117796in}}%
\pgfpathlineto{\pgfqpoint{6.126752in}{3.124582in}}%
\pgfpathlineto{\pgfqpoint{6.133710in}{3.133871in}}%
\pgfpathlineto{\pgfqpoint{6.136921in}{3.135040in}}%
\pgfpathlineto{\pgfqpoint{6.139865in}{3.133767in}}%
\pgfpathlineto{\pgfqpoint{6.143879in}{3.129012in}}%
\pgfpathlineto{\pgfqpoint{6.153780in}{3.115950in}}%
\pgfpathlineto{\pgfqpoint{6.156992in}{3.115537in}}%
\pgfpathlineto{\pgfqpoint{6.160203in}{3.117744in}}%
\pgfpathlineto{\pgfqpoint{6.165020in}{3.124504in}}%
\pgfpathlineto{\pgfqpoint{6.171978in}{3.133834in}}%
\pgfpathlineto{\pgfqpoint{6.175189in}{3.135043in}}%
\pgfpathlineto{\pgfqpoint{6.178133in}{3.133805in}}%
\pgfpathlineto{\pgfqpoint{6.181879in}{3.129477in}}%
\pgfpathlineto{\pgfqpoint{6.192584in}{3.115715in}}%
\pgfpathlineto{\pgfqpoint{6.195527in}{3.115607in}}%
\pgfpathlineto{\pgfqpoint{6.198739in}{3.117979in}}%
\pgfpathlineto{\pgfqpoint{6.203823in}{3.125288in}}%
\pgfpathlineto{\pgfqpoint{6.210513in}{3.133998in}}%
\pgfpathlineto{\pgfqpoint{6.213725in}{3.135027in}}%
\pgfpathlineto{\pgfqpoint{6.216668in}{3.133628in}}%
\pgfpathlineto{\pgfqpoint{6.220682in}{3.128757in}}%
\pgfpathlineto{\pgfqpoint{6.230316in}{3.116004in}}%
\pgfpathlineto{\pgfqpoint{6.233528in}{3.115509in}}%
\pgfpathlineto{\pgfqpoint{6.236471in}{3.117375in}}%
\pgfpathlineto{\pgfqpoint{6.241021in}{3.123492in}}%
\pgfpathlineto{\pgfqpoint{6.248781in}{3.133962in}}%
\pgfpathlineto{\pgfqpoint{6.251993in}{3.135032in}}%
\pgfpathlineto{\pgfqpoint{6.254936in}{3.133668in}}%
\pgfpathlineto{\pgfqpoint{6.258950in}{3.128830in}}%
\pgfpathlineto{\pgfqpoint{6.268584in}{3.116032in}}%
\pgfpathlineto{\pgfqpoint{6.271796in}{3.115497in}}%
\pgfpathlineto{\pgfqpoint{6.274739in}{3.117327in}}%
\pgfpathlineto{\pgfqpoint{6.279021in}{3.122992in}}%
\pgfpathlineto{\pgfqpoint{6.287317in}{3.134118in}}%
\pgfpathlineto{\pgfqpoint{6.290528in}{3.135007in}}%
\pgfpathlineto{\pgfqpoint{6.293472in}{3.133483in}}%
\pgfpathlineto{\pgfqpoint{6.297486in}{3.128500in}}%
\pgfpathlineto{\pgfqpoint{6.306852in}{3.116060in}}%
\pgfpathlineto{\pgfqpoint{6.310063in}{3.115485in}}%
\pgfpathlineto{\pgfqpoint{6.313007in}{3.117281in}}%
\pgfpathlineto{\pgfqpoint{6.317289in}{3.122916in}}%
\pgfpathlineto{\pgfqpoint{6.325585in}{3.134084in}}%
\pgfpathlineto{\pgfqpoint{6.328796in}{3.135013in}}%
\pgfpathlineto{\pgfqpoint{6.331740in}{3.133525in}}%
\pgfpathlineto{\pgfqpoint{6.335754in}{3.128574in}}%
\pgfpathlineto{\pgfqpoint{6.345120in}{3.116090in}}%
\pgfpathlineto{\pgfqpoint{6.348331in}{3.115473in}}%
\pgfpathlineto{\pgfqpoint{6.351275in}{3.117235in}}%
\pgfpathlineto{\pgfqpoint{6.355557in}{3.122839in}}%
\pgfpathlineto{\pgfqpoint{6.363853in}{3.134050in}}%
\pgfpathlineto{\pgfqpoint{6.367064in}{3.135020in}}%
\pgfpathlineto{\pgfqpoint{6.370008in}{3.133567in}}%
\pgfpathlineto{\pgfqpoint{6.374022in}{3.128647in}}%
\pgfpathlineto{\pgfqpoint{6.383656in}{3.115963in}}%
\pgfpathlineto{\pgfqpoint{6.386867in}{3.115530in}}%
\pgfpathlineto{\pgfqpoint{6.390078in}{3.117719in}}%
\pgfpathlineto{\pgfqpoint{6.394895in}{3.124465in}}%
\pgfpathlineto{\pgfqpoint{6.402121in}{3.134015in}}%
\pgfpathlineto{\pgfqpoint{6.405332in}{3.135025in}}%
\pgfpathlineto{\pgfqpoint{6.408276in}{3.133608in}}%
\pgfpathlineto{\pgfqpoint{6.412290in}{3.128721in}}%
\pgfpathlineto{\pgfqpoint{6.421924in}{3.115990in}}%
\pgfpathlineto{\pgfqpoint{6.425135in}{3.115516in}}%
\pgfpathlineto{\pgfqpoint{6.428346in}{3.117668in}}%
\pgfpathlineto{\pgfqpoint{6.433163in}{3.124386in}}%
\pgfpathlineto{\pgfqpoint{6.440388in}{3.133980in}}%
\pgfpathlineto{\pgfqpoint{6.443600in}{3.135030in}}%
\pgfpathlineto{\pgfqpoint{6.446543in}{3.133648in}}%
\pgfpathlineto{\pgfqpoint{6.450558in}{3.128794in}}%
\pgfpathlineto{\pgfqpoint{6.460191in}{3.116018in}}%
\pgfpathlineto{\pgfqpoint{6.463403in}{3.115503in}}%
\pgfpathlineto{\pgfqpoint{6.466346in}{3.117351in}}%
\pgfpathlineto{\pgfqpoint{6.470896in}{3.123453in}}%
\pgfpathlineto{\pgfqpoint{6.478656in}{3.133944in}}%
\pgfpathlineto{\pgfqpoint{6.481868in}{3.135034in}}%
\pgfpathlineto{\pgfqpoint{6.484811in}{3.133688in}}%
\pgfpathlineto{\pgfqpoint{6.488825in}{3.128867in}}%
\pgfpathlineto{\pgfqpoint{6.498727in}{3.115898in}}%
\pgfpathlineto{\pgfqpoint{6.501938in}{3.115566in}}%
\pgfpathlineto{\pgfqpoint{6.505150in}{3.117847in}}%
\pgfpathlineto{\pgfqpoint{6.509966in}{3.124660in}}%
\pgfpathlineto{\pgfqpoint{6.516924in}{3.133908in}}%
\pgfpathlineto{\pgfqpoint{6.520136in}{3.135037in}}%
\pgfpathlineto{\pgfqpoint{6.523079in}{3.133728in}}%
\pgfpathlineto{\pgfqpoint{6.527093in}{3.128940in}}%
\pgfpathlineto{\pgfqpoint{6.536995in}{3.115924in}}%
\pgfpathlineto{\pgfqpoint{6.540206in}{3.115551in}}%
\pgfpathlineto{\pgfqpoint{6.543417in}{3.117796in}}%
\pgfpathlineto{\pgfqpoint{6.548234in}{3.124582in}}%
\pgfpathlineto{\pgfqpoint{6.555192in}{3.133871in}}%
\pgfpathlineto{\pgfqpoint{6.558403in}{3.135040in}}%
\pgfpathlineto{\pgfqpoint{6.561347in}{3.133767in}}%
\pgfpathlineto{\pgfqpoint{6.565361in}{3.129012in}}%
\pgfpathlineto{\pgfqpoint{6.575263in}{3.115950in}}%
\pgfpathlineto{\pgfqpoint{6.578474in}{3.115537in}}%
\pgfpathlineto{\pgfqpoint{6.581685in}{3.117744in}}%
\pgfpathlineto{\pgfqpoint{6.586502in}{3.124504in}}%
\pgfpathlineto{\pgfqpoint{6.593460in}{3.133834in}}%
\pgfpathlineto{\pgfqpoint{6.596671in}{3.135043in}}%
\pgfpathlineto{\pgfqpoint{6.599615in}{3.133805in}}%
\pgfpathlineto{\pgfqpoint{6.603362in}{3.129477in}}%
\pgfpathlineto{\pgfqpoint{6.614066in}{3.115715in}}%
\pgfpathlineto{\pgfqpoint{6.617010in}{3.115607in}}%
\pgfpathlineto{\pgfqpoint{6.620221in}{3.117979in}}%
\pgfpathlineto{\pgfqpoint{6.625305in}{3.125288in}}%
\pgfpathlineto{\pgfqpoint{6.631996in}{3.133998in}}%
\pgfpathlineto{\pgfqpoint{6.635207in}{3.135027in}}%
\pgfpathlineto{\pgfqpoint{6.638151in}{3.133628in}}%
\pgfpathlineto{\pgfqpoint{6.642165in}{3.128757in}}%
\pgfpathlineto{\pgfqpoint{6.651799in}{3.116004in}}%
\pgfpathlineto{\pgfqpoint{6.655010in}{3.115509in}}%
\pgfpathlineto{\pgfqpoint{6.657954in}{3.117375in}}%
\pgfpathlineto{\pgfqpoint{6.662503in}{3.123492in}}%
\pgfpathlineto{\pgfqpoint{6.663306in}{3.124778in}}%
\pgfpathlineto{\pgfqpoint{6.663306in}{3.124778in}}%
\pgfusepath{stroke}%
\end{pgfscope}%
\begin{pgfscope}%
\pgfpathrectangle{\pgfqpoint{0.467797in}{2.292089in}}{\pgfqpoint{6.490533in}{1.666241in}}%
\pgfusepath{clip}%
\pgfsetrectcap%
\pgfsetroundjoin%
\pgfsetlinewidth{1.505625pt}%
\definecolor{currentstroke}{rgb}{0.090196,0.745098,0.811765}%
\pgfsetstrokecolor{currentstroke}%
\pgfsetdash{}{0pt}%
\pgfpathmoveto{\pgfqpoint{0.762821in}{3.125209in}}%
\pgfpathlineto{\pgfqpoint{0.769243in}{3.133667in}}%
\pgfpathlineto{\pgfqpoint{0.772455in}{3.134785in}}%
\pgfpathlineto{\pgfqpoint{0.775398in}{3.133406in}}%
\pgfpathlineto{\pgfqpoint{0.779413in}{3.128499in}}%
\pgfpathlineto{\pgfqpoint{0.788779in}{3.116188in}}%
\pgfpathlineto{\pgfqpoint{0.791723in}{3.115730in}}%
\pgfpathlineto{\pgfqpoint{0.794666in}{3.117550in}}%
\pgfpathlineto{\pgfqpoint{0.798948in}{3.123256in}}%
\pgfpathlineto{\pgfqpoint{0.806976in}{3.133943in}}%
\pgfpathlineto{\pgfqpoint{0.810187in}{3.134733in}}%
\pgfpathlineto{\pgfqpoint{0.813131in}{3.133066in}}%
\pgfpathlineto{\pgfqpoint{0.817413in}{3.127490in}}%
\pgfpathlineto{\pgfqpoint{0.825709in}{3.116437in}}%
\pgfpathlineto{\pgfqpoint{0.828920in}{3.115697in}}%
\pgfpathlineto{\pgfqpoint{0.831864in}{3.117408in}}%
\pgfpathlineto{\pgfqpoint{0.836145in}{3.123022in}}%
\pgfpathlineto{\pgfqpoint{0.844174in}{3.133841in}}%
\pgfpathlineto{\pgfqpoint{0.847385in}{3.134758in}}%
\pgfpathlineto{\pgfqpoint{0.850329in}{3.133202in}}%
\pgfpathlineto{\pgfqpoint{0.854343in}{3.128137in}}%
\pgfpathlineto{\pgfqpoint{0.863174in}{3.116362in}}%
\pgfpathlineto{\pgfqpoint{0.866385in}{3.115723in}}%
\pgfpathlineto{\pgfqpoint{0.869329in}{3.117520in}}%
\pgfpathlineto{\pgfqpoint{0.873611in}{3.123208in}}%
\pgfpathlineto{\pgfqpoint{0.881639in}{3.133922in}}%
\pgfpathlineto{\pgfqpoint{0.884850in}{3.134739in}}%
\pgfpathlineto{\pgfqpoint{0.887794in}{3.133094in}}%
\pgfpathlineto{\pgfqpoint{0.892075in}{3.127538in}}%
\pgfpathlineto{\pgfqpoint{0.900371in}{3.116457in}}%
\pgfpathlineto{\pgfqpoint{0.903583in}{3.115691in}}%
\pgfpathlineto{\pgfqpoint{0.906526in}{3.117379in}}%
\pgfpathlineto{\pgfqpoint{0.910808in}{3.122974in}}%
\pgfpathlineto{\pgfqpoint{0.919104in}{3.134000in}}%
\pgfpathlineto{\pgfqpoint{0.922315in}{3.134716in}}%
\pgfpathlineto{\pgfqpoint{0.925259in}{3.132984in}}%
\pgfpathlineto{\pgfqpoint{0.929541in}{3.127352in}}%
\pgfpathlineto{\pgfqpoint{0.937569in}{3.116558in}}%
\pgfpathlineto{\pgfqpoint{0.940780in}{3.115665in}}%
\pgfpathlineto{\pgfqpoint{0.943724in}{3.117243in}}%
\pgfpathlineto{\pgfqpoint{0.947738in}{3.122326in}}%
\pgfpathlineto{\pgfqpoint{0.956569in}{3.134075in}}%
\pgfpathlineto{\pgfqpoint{0.959780in}{3.134689in}}%
\pgfpathlineto{\pgfqpoint{0.962724in}{3.132871in}}%
\pgfpathlineto{\pgfqpoint{0.967006in}{3.127165in}}%
\pgfpathlineto{\pgfqpoint{0.975034in}{3.116477in}}%
\pgfpathlineto{\pgfqpoint{0.978245in}{3.115685in}}%
\pgfpathlineto{\pgfqpoint{0.981189in}{3.117351in}}%
\pgfpathlineto{\pgfqpoint{0.985471in}{3.122926in}}%
\pgfpathlineto{\pgfqpoint{0.993766in}{3.133980in}}%
\pgfpathlineto{\pgfqpoint{0.996978in}{3.134722in}}%
\pgfpathlineto{\pgfqpoint{0.999921in}{3.133013in}}%
\pgfpathlineto{\pgfqpoint{1.004203in}{3.127400in}}%
\pgfpathlineto{\pgfqpoint{1.012231in}{3.116579in}}%
\pgfpathlineto{\pgfqpoint{1.015443in}{3.115661in}}%
\pgfpathlineto{\pgfqpoint{1.018386in}{3.117216in}}%
\pgfpathlineto{\pgfqpoint{1.022400in}{3.122279in}}%
\pgfpathlineto{\pgfqpoint{1.031232in}{3.134056in}}%
\pgfpathlineto{\pgfqpoint{1.034443in}{3.134696in}}%
\pgfpathlineto{\pgfqpoint{1.037387in}{3.132900in}}%
\pgfpathlineto{\pgfqpoint{1.041668in}{3.127213in}}%
\pgfpathlineto{\pgfqpoint{1.049696in}{3.116498in}}%
\pgfpathlineto{\pgfqpoint{1.052908in}{3.115680in}}%
\pgfpathlineto{\pgfqpoint{1.055851in}{3.117323in}}%
\pgfpathlineto{\pgfqpoint{1.060133in}{3.122878in}}%
\pgfpathlineto{\pgfqpoint{1.068429in}{3.133960in}}%
\pgfpathlineto{\pgfqpoint{1.071640in}{3.134728in}}%
\pgfpathlineto{\pgfqpoint{1.074584in}{3.133041in}}%
\pgfpathlineto{\pgfqpoint{1.078866in}{3.127448in}}%
\pgfpathlineto{\pgfqpoint{1.087162in}{3.116420in}}%
\pgfpathlineto{\pgfqpoint{1.090373in}{3.115703in}}%
\pgfpathlineto{\pgfqpoint{1.093317in}{3.117433in}}%
\pgfpathlineto{\pgfqpoint{1.097598in}{3.123064in}}%
\pgfpathlineto{\pgfqpoint{1.105627in}{3.133860in}}%
\pgfpathlineto{\pgfqpoint{1.108838in}{3.134754in}}%
\pgfpathlineto{\pgfqpoint{1.111782in}{3.133177in}}%
\pgfpathlineto{\pgfqpoint{1.115796in}{3.128095in}}%
\pgfpathlineto{\pgfqpoint{1.124627in}{3.116345in}}%
\pgfpathlineto{\pgfqpoint{1.127838in}{3.115729in}}%
\pgfpathlineto{\pgfqpoint{1.130782in}{3.117547in}}%
\pgfpathlineto{\pgfqpoint{1.135063in}{3.123251in}}%
\pgfpathlineto{\pgfqpoint{1.143092in}{3.133940in}}%
\pgfpathlineto{\pgfqpoint{1.146303in}{3.134734in}}%
\pgfpathlineto{\pgfqpoint{1.149247in}{3.133069in}}%
\pgfpathlineto{\pgfqpoint{1.153528in}{3.127496in}}%
\pgfpathlineto{\pgfqpoint{1.161824in}{3.116440in}}%
\pgfpathlineto{\pgfqpoint{1.165035in}{3.115696in}}%
\pgfpathlineto{\pgfqpoint{1.167979in}{3.117405in}}%
\pgfpathlineto{\pgfqpoint{1.172261in}{3.123016in}}%
\pgfpathlineto{\pgfqpoint{1.180289in}{3.133838in}}%
\pgfpathlineto{\pgfqpoint{1.183500in}{3.134758in}}%
\pgfpathlineto{\pgfqpoint{1.186444in}{3.133205in}}%
\pgfpathlineto{\pgfqpoint{1.190458in}{3.128142in}}%
\pgfpathlineto{\pgfqpoint{1.199289in}{3.116364in}}%
\pgfpathlineto{\pgfqpoint{1.202501in}{3.115722in}}%
\pgfpathlineto{\pgfqpoint{1.205444in}{3.117517in}}%
\pgfpathlineto{\pgfqpoint{1.209726in}{3.123203in}}%
\pgfpathlineto{\pgfqpoint{1.217754in}{3.133920in}}%
\pgfpathlineto{\pgfqpoint{1.220965in}{3.134739in}}%
\pgfpathlineto{\pgfqpoint{1.223909in}{3.133097in}}%
\pgfpathlineto{\pgfqpoint{1.228191in}{3.127543in}}%
\pgfpathlineto{\pgfqpoint{1.236487in}{3.116460in}}%
\pgfpathlineto{\pgfqpoint{1.239698in}{3.115690in}}%
\pgfpathlineto{\pgfqpoint{1.242642in}{3.117376in}}%
\pgfpathlineto{\pgfqpoint{1.246923in}{3.122968in}}%
\pgfpathlineto{\pgfqpoint{1.255219in}{3.133998in}}%
\pgfpathlineto{\pgfqpoint{1.258431in}{3.134717in}}%
\pgfpathlineto{\pgfqpoint{1.261374in}{3.132987in}}%
\pgfpathlineto{\pgfqpoint{1.265656in}{3.127357in}}%
\pgfpathlineto{\pgfqpoint{1.273684in}{3.116561in}}%
\pgfpathlineto{\pgfqpoint{1.276896in}{3.115665in}}%
\pgfpathlineto{\pgfqpoint{1.279839in}{3.117240in}}%
\pgfpathlineto{\pgfqpoint{1.283853in}{3.122321in}}%
\pgfpathlineto{\pgfqpoint{1.292684in}{3.134073in}}%
\pgfpathlineto{\pgfqpoint{1.295896in}{3.134690in}}%
\pgfpathlineto{\pgfqpoint{1.298839in}{3.132874in}}%
\pgfpathlineto{\pgfqpoint{1.303121in}{3.127171in}}%
\pgfpathlineto{\pgfqpoint{1.311149in}{3.116480in}}%
\pgfpathlineto{\pgfqpoint{1.314361in}{3.115685in}}%
\pgfpathlineto{\pgfqpoint{1.317304in}{3.117348in}}%
\pgfpathlineto{\pgfqpoint{1.321586in}{3.122921in}}%
\pgfpathlineto{\pgfqpoint{1.329882in}{3.133978in}}%
\pgfpathlineto{\pgfqpoint{1.333093in}{3.134723in}}%
\pgfpathlineto{\pgfqpoint{1.336037in}{3.133016in}}%
\pgfpathlineto{\pgfqpoint{1.340319in}{3.127405in}}%
\pgfpathlineto{\pgfqpoint{1.348347in}{3.116582in}}%
\pgfpathlineto{\pgfqpoint{1.351558in}{3.115660in}}%
\pgfpathlineto{\pgfqpoint{1.354502in}{3.117213in}}%
\pgfpathlineto{\pgfqpoint{1.358516in}{3.122274in}}%
\pgfpathlineto{\pgfqpoint{1.367347in}{3.134054in}}%
\pgfpathlineto{\pgfqpoint{1.370558in}{3.134697in}}%
\pgfpathlineto{\pgfqpoint{1.373502in}{3.132903in}}%
\pgfpathlineto{\pgfqpoint{1.377784in}{3.127219in}}%
\pgfpathlineto{\pgfqpoint{1.385812in}{3.116500in}}%
\pgfpathlineto{\pgfqpoint{1.389023in}{3.115679in}}%
\pgfpathlineto{\pgfqpoint{1.391967in}{3.117320in}}%
\pgfpathlineto{\pgfqpoint{1.396249in}{3.122873in}}%
\pgfpathlineto{\pgfqpoint{1.404544in}{3.133958in}}%
\pgfpathlineto{\pgfqpoint{1.407756in}{3.134729in}}%
\pgfpathlineto{\pgfqpoint{1.410699in}{3.133044in}}%
\pgfpathlineto{\pgfqpoint{1.414981in}{3.127453in}}%
\pgfpathlineto{\pgfqpoint{1.423277in}{3.116422in}}%
\pgfpathlineto{\pgfqpoint{1.426488in}{3.115702in}}%
\pgfpathlineto{\pgfqpoint{1.429432in}{3.117430in}}%
\pgfpathlineto{\pgfqpoint{1.433714in}{3.123059in}}%
\pgfpathlineto{\pgfqpoint{1.441742in}{3.133857in}}%
\pgfpathlineto{\pgfqpoint{1.444953in}{3.134754in}}%
\pgfpathlineto{\pgfqpoint{1.447897in}{3.133180in}}%
\pgfpathlineto{\pgfqpoint{1.451911in}{3.128100in}}%
\pgfpathlineto{\pgfqpoint{1.460742in}{3.116347in}}%
\pgfpathlineto{\pgfqpoint{1.463953in}{3.115728in}}%
\pgfpathlineto{\pgfqpoint{1.466897in}{3.117543in}}%
\pgfpathlineto{\pgfqpoint{1.471179in}{3.123246in}}%
\pgfpathlineto{\pgfqpoint{1.479207in}{3.133938in}}%
\pgfpathlineto{\pgfqpoint{1.482418in}{3.134734in}}%
\pgfpathlineto{\pgfqpoint{1.485362in}{3.133073in}}%
\pgfpathlineto{\pgfqpoint{1.489644in}{3.127501in}}%
\pgfpathlineto{\pgfqpoint{1.497940in}{3.116442in}}%
\pgfpathlineto{\pgfqpoint{1.501151in}{3.115696in}}%
\pgfpathlineto{\pgfqpoint{1.504095in}{3.117401in}}%
\pgfpathlineto{\pgfqpoint{1.508376in}{3.123011in}}%
\pgfpathlineto{\pgfqpoint{1.516405in}{3.133836in}}%
\pgfpathlineto{\pgfqpoint{1.519616in}{3.134759in}}%
\pgfpathlineto{\pgfqpoint{1.522560in}{3.133208in}}%
\pgfpathlineto{\pgfqpoint{1.526574in}{3.128147in}}%
\pgfpathlineto{\pgfqpoint{1.535405in}{3.116366in}}%
\pgfpathlineto{\pgfqpoint{1.538616in}{3.115721in}}%
\pgfpathlineto{\pgfqpoint{1.541560in}{3.117514in}}%
\pgfpathlineto{\pgfqpoint{1.545841in}{3.123197in}}%
\pgfpathlineto{\pgfqpoint{1.553870in}{3.133918in}}%
\pgfpathlineto{\pgfqpoint{1.557081in}{3.134740in}}%
\pgfpathlineto{\pgfqpoint{1.560025in}{3.133101in}}%
\pgfpathlineto{\pgfqpoint{1.564306in}{3.127549in}}%
\pgfpathlineto{\pgfqpoint{1.572602in}{3.116462in}}%
\pgfpathlineto{\pgfqpoint{1.575813in}{3.115690in}}%
\pgfpathlineto{\pgfqpoint{1.578757in}{3.117373in}}%
\pgfpathlineto{\pgfqpoint{1.583039in}{3.122963in}}%
\pgfpathlineto{\pgfqpoint{1.591335in}{3.133996in}}%
\pgfpathlineto{\pgfqpoint{1.594546in}{3.134717in}}%
\pgfpathlineto{\pgfqpoint{1.597490in}{3.132990in}}%
\pgfpathlineto{\pgfqpoint{1.601771in}{3.127363in}}%
\pgfpathlineto{\pgfqpoint{1.609800in}{3.116563in}}%
\pgfpathlineto{\pgfqpoint{1.613011in}{3.115664in}}%
\pgfpathlineto{\pgfqpoint{1.615955in}{3.117237in}}%
\pgfpathlineto{\pgfqpoint{1.619969in}{3.122316in}}%
\pgfpathlineto{\pgfqpoint{1.628800in}{3.134071in}}%
\pgfpathlineto{\pgfqpoint{1.632011in}{3.134691in}}%
\pgfpathlineto{\pgfqpoint{1.634955in}{3.132877in}}%
\pgfpathlineto{\pgfqpoint{1.639237in}{3.127176in}}%
\pgfpathlineto{\pgfqpoint{1.647265in}{3.116482in}}%
\pgfpathlineto{\pgfqpoint{1.650476in}{3.115684in}}%
\pgfpathlineto{\pgfqpoint{1.653420in}{3.117345in}}%
\pgfpathlineto{\pgfqpoint{1.657701in}{3.122915in}}%
\pgfpathlineto{\pgfqpoint{1.665997in}{3.133976in}}%
\pgfpathlineto{\pgfqpoint{1.669209in}{3.134723in}}%
\pgfpathlineto{\pgfqpoint{1.672152in}{3.133019in}}%
\pgfpathlineto{\pgfqpoint{1.676434in}{3.127411in}}%
\pgfpathlineto{\pgfqpoint{1.684462in}{3.116584in}}%
\pgfpathlineto{\pgfqpoint{1.687674in}{3.115660in}}%
\pgfpathlineto{\pgfqpoint{1.690617in}{3.117210in}}%
\pgfpathlineto{\pgfqpoint{1.694631in}{3.122269in}}%
\pgfpathlineto{\pgfqpoint{1.703462in}{3.134052in}}%
\pgfpathlineto{\pgfqpoint{1.706674in}{3.134698in}}%
\pgfpathlineto{\pgfqpoint{1.709617in}{3.132907in}}%
\pgfpathlineto{\pgfqpoint{1.713899in}{3.127224in}}%
\pgfpathlineto{\pgfqpoint{1.721927in}{3.116502in}}%
\pgfpathlineto{\pgfqpoint{1.725139in}{3.115679in}}%
\pgfpathlineto{\pgfqpoint{1.728082in}{3.117317in}}%
\pgfpathlineto{\pgfqpoint{1.732364in}{3.122868in}}%
\pgfpathlineto{\pgfqpoint{1.740660in}{3.133956in}}%
\pgfpathlineto{\pgfqpoint{1.743871in}{3.134729in}}%
\pgfpathlineto{\pgfqpoint{1.746815in}{3.133047in}}%
\pgfpathlineto{\pgfqpoint{1.751097in}{3.127458in}}%
\pgfpathlineto{\pgfqpoint{1.759392in}{3.116424in}}%
\pgfpathlineto{\pgfqpoint{1.762604in}{3.115701in}}%
\pgfpathlineto{\pgfqpoint{1.765547in}{3.117427in}}%
\pgfpathlineto{\pgfqpoint{1.769829in}{3.123053in}}%
\pgfpathlineto{\pgfqpoint{1.777857in}{3.133855in}}%
\pgfpathlineto{\pgfqpoint{1.781069in}{3.134755in}}%
\pgfpathlineto{\pgfqpoint{1.784012in}{3.133183in}}%
\pgfpathlineto{\pgfqpoint{1.788026in}{3.128106in}}%
\pgfpathlineto{\pgfqpoint{1.796858in}{3.116349in}}%
\pgfpathlineto{\pgfqpoint{1.800069in}{3.115728in}}%
\pgfpathlineto{\pgfqpoint{1.803013in}{3.117540in}}%
\pgfpathlineto{\pgfqpoint{1.807294in}{3.123240in}}%
\pgfpathlineto{\pgfqpoint{1.815322in}{3.133936in}}%
\pgfpathlineto{\pgfqpoint{1.818534in}{3.134735in}}%
\pgfpathlineto{\pgfqpoint{1.821477in}{3.133076in}}%
\pgfpathlineto{\pgfqpoint{1.825759in}{3.127506in}}%
\pgfpathlineto{\pgfqpoint{1.834055in}{3.116444in}}%
\pgfpathlineto{\pgfqpoint{1.837266in}{3.115695in}}%
\pgfpathlineto{\pgfqpoint{1.840210in}{3.117398in}}%
\pgfpathlineto{\pgfqpoint{1.844492in}{3.123006in}}%
\pgfpathlineto{\pgfqpoint{1.852520in}{3.133833in}}%
\pgfpathlineto{\pgfqpoint{1.855731in}{3.134759in}}%
\pgfpathlineto{\pgfqpoint{1.858675in}{3.133211in}}%
\pgfpathlineto{\pgfqpoint{1.862689in}{3.128153in}}%
\pgfpathlineto{\pgfqpoint{1.871520in}{3.116368in}}%
\pgfpathlineto{\pgfqpoint{1.874731in}{3.115720in}}%
\pgfpathlineto{\pgfqpoint{1.877675in}{3.117511in}}%
\pgfpathlineto{\pgfqpoint{1.881957in}{3.123192in}}%
\pgfpathlineto{\pgfqpoint{1.889985in}{3.133915in}}%
\pgfpathlineto{\pgfqpoint{1.893196in}{3.134741in}}%
\pgfpathlineto{\pgfqpoint{1.896140in}{3.133104in}}%
\pgfpathlineto{\pgfqpoint{1.900422in}{3.127554in}}%
\pgfpathlineto{\pgfqpoint{1.908718in}{3.116464in}}%
\pgfpathlineto{\pgfqpoint{1.911929in}{3.115689in}}%
\pgfpathlineto{\pgfqpoint{1.914873in}{3.117370in}}%
\pgfpathlineto{\pgfqpoint{1.919154in}{3.122958in}}%
\pgfpathlineto{\pgfqpoint{1.927450in}{3.133994in}}%
\pgfpathlineto{\pgfqpoint{1.930661in}{3.134718in}}%
\pgfpathlineto{\pgfqpoint{1.933605in}{3.132994in}}%
\pgfpathlineto{\pgfqpoint{1.937887in}{3.127368in}}%
\pgfpathlineto{\pgfqpoint{1.945915in}{3.116565in}}%
\pgfpathlineto{\pgfqpoint{1.949126in}{3.115664in}}%
\pgfpathlineto{\pgfqpoint{1.952070in}{3.117234in}}%
\pgfpathlineto{\pgfqpoint{1.956084in}{3.122311in}}%
\pgfpathlineto{\pgfqpoint{1.964915in}{3.134068in}}%
\pgfpathlineto{\pgfqpoint{1.968127in}{3.134692in}}%
\pgfpathlineto{\pgfqpoint{1.971070in}{3.132880in}}%
\pgfpathlineto{\pgfqpoint{1.975352in}{3.127181in}}%
\pgfpathlineto{\pgfqpoint{1.983380in}{3.116484in}}%
\pgfpathlineto{\pgfqpoint{1.986591in}{3.115683in}}%
\pgfpathlineto{\pgfqpoint{1.989535in}{3.117342in}}%
\pgfpathlineto{\pgfqpoint{1.993817in}{3.122910in}}%
\pgfpathlineto{\pgfqpoint{2.002113in}{3.133974in}}%
\pgfpathlineto{\pgfqpoint{2.005324in}{3.134724in}}%
\pgfpathlineto{\pgfqpoint{2.008268in}{3.133022in}}%
\pgfpathlineto{\pgfqpoint{2.012549in}{3.127416in}}%
\pgfpathlineto{\pgfqpoint{2.020578in}{3.116587in}}%
\pgfpathlineto{\pgfqpoint{2.023789in}{3.115659in}}%
\pgfpathlineto{\pgfqpoint{2.026733in}{3.117207in}}%
\pgfpathlineto{\pgfqpoint{2.030747in}{3.122264in}}%
\pgfpathlineto{\pgfqpoint{2.039578in}{3.134050in}}%
\pgfpathlineto{\pgfqpoint{2.042789in}{3.134699in}}%
\pgfpathlineto{\pgfqpoint{2.045733in}{3.132910in}}%
\pgfpathlineto{\pgfqpoint{2.050015in}{3.127229in}}%
\pgfpathlineto{\pgfqpoint{2.058043in}{3.116505in}}%
\pgfpathlineto{\pgfqpoint{2.061254in}{3.115678in}}%
\pgfpathlineto{\pgfqpoint{2.064198in}{3.117314in}}%
\pgfpathlineto{\pgfqpoint{2.068479in}{3.122862in}}%
\pgfpathlineto{\pgfqpoint{2.076775in}{3.133954in}}%
\pgfpathlineto{\pgfqpoint{2.079987in}{3.134730in}}%
\pgfpathlineto{\pgfqpoint{2.082930in}{3.133051in}}%
\pgfpathlineto{\pgfqpoint{2.087212in}{3.127464in}}%
\pgfpathlineto{\pgfqpoint{2.095508in}{3.116426in}}%
\pgfpathlineto{\pgfqpoint{2.098719in}{3.115701in}}%
\pgfpathlineto{\pgfqpoint{2.101663in}{3.117424in}}%
\pgfpathlineto{\pgfqpoint{2.105945in}{3.123048in}}%
\pgfpathlineto{\pgfqpoint{2.113973in}{3.133852in}}%
\pgfpathlineto{\pgfqpoint{2.117184in}{3.134755in}}%
\pgfpathlineto{\pgfqpoint{2.120128in}{3.133186in}}%
\pgfpathlineto{\pgfqpoint{2.124142in}{3.128111in}}%
\pgfpathlineto{\pgfqpoint{2.132973in}{3.116351in}}%
\pgfpathlineto{\pgfqpoint{2.136184in}{3.115727in}}%
\pgfpathlineto{\pgfqpoint{2.139128in}{3.117537in}}%
\pgfpathlineto{\pgfqpoint{2.143410in}{3.123235in}}%
\pgfpathlineto{\pgfqpoint{2.151438in}{3.133933in}}%
\pgfpathlineto{\pgfqpoint{2.154649in}{3.134736in}}%
\pgfpathlineto{\pgfqpoint{2.157593in}{3.133079in}}%
\pgfpathlineto{\pgfqpoint{2.161875in}{3.127512in}}%
\pgfpathlineto{\pgfqpoint{2.170170in}{3.116446in}}%
\pgfpathlineto{\pgfqpoint{2.173382in}{3.115694in}}%
\pgfpathlineto{\pgfqpoint{2.176325in}{3.117395in}}%
\pgfpathlineto{\pgfqpoint{2.180607in}{3.123000in}}%
\pgfpathlineto{\pgfqpoint{2.188903in}{3.134011in}}%
\pgfpathlineto{\pgfqpoint{2.192114in}{3.134712in}}%
\pgfpathlineto{\pgfqpoint{2.195058in}{3.132968in}}%
\pgfpathlineto{\pgfqpoint{2.199340in}{3.127325in}}%
\pgfpathlineto{\pgfqpoint{2.207368in}{3.116546in}}%
\pgfpathlineto{\pgfqpoint{2.210579in}{3.115668in}}%
\pgfpathlineto{\pgfqpoint{2.213523in}{3.117258in}}%
\pgfpathlineto{\pgfqpoint{2.217537in}{3.122352in}}%
\pgfpathlineto{\pgfqpoint{2.226368in}{3.134085in}}%
\pgfpathlineto{\pgfqpoint{2.229579in}{3.134685in}}%
\pgfpathlineto{\pgfqpoint{2.232523in}{3.132854in}}%
\pgfpathlineto{\pgfqpoint{2.236805in}{3.127139in}}%
\pgfpathlineto{\pgfqpoint{2.244833in}{3.116466in}}%
\pgfpathlineto{\pgfqpoint{2.248044in}{3.115688in}}%
\pgfpathlineto{\pgfqpoint{2.250988in}{3.117367in}}%
\pgfpathlineto{\pgfqpoint{2.255270in}{3.122952in}}%
\pgfpathlineto{\pgfqpoint{2.263566in}{3.133991in}}%
\pgfpathlineto{\pgfqpoint{2.266777in}{3.134719in}}%
\pgfpathlineto{\pgfqpoint{2.269721in}{3.132997in}}%
\pgfpathlineto{\pgfqpoint{2.274002in}{3.127373in}}%
\pgfpathlineto{\pgfqpoint{2.282031in}{3.116568in}}%
\pgfpathlineto{\pgfqpoint{2.285242in}{3.115663in}}%
\pgfpathlineto{\pgfqpoint{2.288185in}{3.117231in}}%
\pgfpathlineto{\pgfqpoint{2.292200in}{3.122305in}}%
\pgfpathlineto{\pgfqpoint{2.301031in}{3.134066in}}%
\pgfpathlineto{\pgfqpoint{2.304242in}{3.134693in}}%
\pgfpathlineto{\pgfqpoint{2.307186in}{3.132884in}}%
\pgfpathlineto{\pgfqpoint{2.311467in}{3.127187in}}%
\pgfpathlineto{\pgfqpoint{2.319496in}{3.116487in}}%
\pgfpathlineto{\pgfqpoint{2.322707in}{3.115683in}}%
\pgfpathlineto{\pgfqpoint{2.325651in}{3.117339in}}%
\pgfpathlineto{\pgfqpoint{2.329932in}{3.122905in}}%
\pgfpathlineto{\pgfqpoint{2.338228in}{3.133972in}}%
\pgfpathlineto{\pgfqpoint{2.341439in}{3.134725in}}%
\pgfpathlineto{\pgfqpoint{2.344383in}{3.133025in}}%
\pgfpathlineto{\pgfqpoint{2.348665in}{3.127421in}}%
\pgfpathlineto{\pgfqpoint{2.356961in}{3.116409in}}%
\pgfpathlineto{\pgfqpoint{2.360172in}{3.115706in}}%
\pgfpathlineto{\pgfqpoint{2.363116in}{3.117449in}}%
\pgfpathlineto{\pgfqpoint{2.367397in}{3.123091in}}%
\pgfpathlineto{\pgfqpoint{2.375426in}{3.133871in}}%
\pgfpathlineto{\pgfqpoint{2.378637in}{3.134751in}}%
\pgfpathlineto{\pgfqpoint{2.381581in}{3.133162in}}%
\pgfpathlineto{\pgfqpoint{2.385595in}{3.128069in}}%
\pgfpathlineto{\pgfqpoint{2.394426in}{3.116335in}}%
\pgfpathlineto{\pgfqpoint{2.397637in}{3.115733in}}%
\pgfpathlineto{\pgfqpoint{2.400581in}{3.117563in}}%
\pgfpathlineto{\pgfqpoint{2.404862in}{3.123278in}}%
\pgfpathlineto{\pgfqpoint{2.412891in}{3.133952in}}%
\pgfpathlineto{\pgfqpoint{2.416102in}{3.134731in}}%
\pgfpathlineto{\pgfqpoint{2.419046in}{3.133054in}}%
\pgfpathlineto{\pgfqpoint{2.423327in}{3.127469in}}%
\pgfpathlineto{\pgfqpoint{2.431623in}{3.116429in}}%
\pgfpathlineto{\pgfqpoint{2.434835in}{3.115700in}}%
\pgfpathlineto{\pgfqpoint{2.437778in}{3.117421in}}%
\pgfpathlineto{\pgfqpoint{2.442060in}{3.123043in}}%
\pgfpathlineto{\pgfqpoint{2.450088in}{3.133850in}}%
\pgfpathlineto{\pgfqpoint{2.453300in}{3.134756in}}%
\pgfpathlineto{\pgfqpoint{2.456243in}{3.133189in}}%
\pgfpathlineto{\pgfqpoint{2.460257in}{3.128116in}}%
\pgfpathlineto{\pgfqpoint{2.469088in}{3.116354in}}%
\pgfpathlineto{\pgfqpoint{2.472300in}{3.115726in}}%
\pgfpathlineto{\pgfqpoint{2.475243in}{3.117534in}}%
\pgfpathlineto{\pgfqpoint{2.479525in}{3.123230in}}%
\pgfpathlineto{\pgfqpoint{2.487553in}{3.133931in}}%
\pgfpathlineto{\pgfqpoint{2.490765in}{3.134736in}}%
\pgfpathlineto{\pgfqpoint{2.493708in}{3.133082in}}%
\pgfpathlineto{\pgfqpoint{2.497990in}{3.127517in}}%
\pgfpathlineto{\pgfqpoint{2.506286in}{3.116448in}}%
\pgfpathlineto{\pgfqpoint{2.509497in}{3.115694in}}%
\pgfpathlineto{\pgfqpoint{2.512441in}{3.117392in}}%
\pgfpathlineto{\pgfqpoint{2.516723in}{3.122995in}}%
\pgfpathlineto{\pgfqpoint{2.525018in}{3.134009in}}%
\pgfpathlineto{\pgfqpoint{2.528230in}{3.134713in}}%
\pgfpathlineto{\pgfqpoint{2.531173in}{3.132971in}}%
\pgfpathlineto{\pgfqpoint{2.535455in}{3.127331in}}%
\pgfpathlineto{\pgfqpoint{2.543483in}{3.116549in}}%
\pgfpathlineto{\pgfqpoint{2.546695in}{3.115667in}}%
\pgfpathlineto{\pgfqpoint{2.549638in}{3.117255in}}%
\pgfpathlineto{\pgfqpoint{2.553652in}{3.122347in}}%
\pgfpathlineto{\pgfqpoint{2.562484in}{3.134083in}}%
\pgfpathlineto{\pgfqpoint{2.565695in}{3.134686in}}%
\pgfpathlineto{\pgfqpoint{2.568638in}{3.132857in}}%
\pgfpathlineto{\pgfqpoint{2.572920in}{3.127144in}}%
\pgfpathlineto{\pgfqpoint{2.580948in}{3.116468in}}%
\pgfpathlineto{\pgfqpoint{2.584160in}{3.115688in}}%
\pgfpathlineto{\pgfqpoint{2.587103in}{3.117364in}}%
\pgfpathlineto{\pgfqpoint{2.591385in}{3.122947in}}%
\pgfpathlineto{\pgfqpoint{2.599681in}{3.133989in}}%
\pgfpathlineto{\pgfqpoint{2.602892in}{3.134719in}}%
\pgfpathlineto{\pgfqpoint{2.605836in}{3.133000in}}%
\pgfpathlineto{\pgfqpoint{2.610118in}{3.127379in}}%
\pgfpathlineto{\pgfqpoint{2.618146in}{3.116570in}}%
\pgfpathlineto{\pgfqpoint{2.621357in}{3.115663in}}%
\pgfpathlineto{\pgfqpoint{2.624301in}{3.117228in}}%
\pgfpathlineto{\pgfqpoint{2.628315in}{3.122300in}}%
\pgfpathlineto{\pgfqpoint{2.637146in}{3.134064in}}%
\pgfpathlineto{\pgfqpoint{2.640357in}{3.134693in}}%
\pgfpathlineto{\pgfqpoint{2.643301in}{3.132887in}}%
\pgfpathlineto{\pgfqpoint{2.647583in}{3.127192in}}%
\pgfpathlineto{\pgfqpoint{2.655611in}{3.116489in}}%
\pgfpathlineto{\pgfqpoint{2.658822in}{3.115682in}}%
\pgfpathlineto{\pgfqpoint{2.661766in}{3.117335in}}%
\pgfpathlineto{\pgfqpoint{2.666048in}{3.122899in}}%
\pgfpathlineto{\pgfqpoint{2.674344in}{3.133969in}}%
\pgfpathlineto{\pgfqpoint{2.677555in}{3.134726in}}%
\pgfpathlineto{\pgfqpoint{2.680499in}{3.133029in}}%
\pgfpathlineto{\pgfqpoint{2.684780in}{3.127427in}}%
\pgfpathlineto{\pgfqpoint{2.693076in}{3.116411in}}%
\pgfpathlineto{\pgfqpoint{2.696287in}{3.115705in}}%
\pgfpathlineto{\pgfqpoint{2.699231in}{3.117446in}}%
\pgfpathlineto{\pgfqpoint{2.703513in}{3.123085in}}%
\pgfpathlineto{\pgfqpoint{2.711541in}{3.133869in}}%
\pgfpathlineto{\pgfqpoint{2.714752in}{3.134752in}}%
\pgfpathlineto{\pgfqpoint{2.717696in}{3.133165in}}%
\pgfpathlineto{\pgfqpoint{2.721710in}{3.128074in}}%
\pgfpathlineto{\pgfqpoint{2.730541in}{3.116337in}}%
\pgfpathlineto{\pgfqpoint{2.733753in}{3.115732in}}%
\pgfpathlineto{\pgfqpoint{2.736696in}{3.117560in}}%
\pgfpathlineto{\pgfqpoint{2.740978in}{3.123272in}}%
\pgfpathlineto{\pgfqpoint{2.749006in}{3.133949in}}%
\pgfpathlineto{\pgfqpoint{2.752217in}{3.134731in}}%
\pgfpathlineto{\pgfqpoint{2.755161in}{3.133057in}}%
\pgfpathlineto{\pgfqpoint{2.759443in}{3.127474in}}%
\pgfpathlineto{\pgfqpoint{2.767739in}{3.116431in}}%
\pgfpathlineto{\pgfqpoint{2.770950in}{3.115699in}}%
\pgfpathlineto{\pgfqpoint{2.773894in}{3.117417in}}%
\pgfpathlineto{\pgfqpoint{2.778175in}{3.123037in}}%
\pgfpathlineto{\pgfqpoint{2.786204in}{3.133848in}}%
\pgfpathlineto{\pgfqpoint{2.789415in}{3.134756in}}%
\pgfpathlineto{\pgfqpoint{2.792359in}{3.133192in}}%
\pgfpathlineto{\pgfqpoint{2.796373in}{3.128121in}}%
\pgfpathlineto{\pgfqpoint{2.805204in}{3.116356in}}%
\pgfpathlineto{\pgfqpoint{2.808415in}{3.115725in}}%
\pgfpathlineto{\pgfqpoint{2.811359in}{3.117530in}}%
\pgfpathlineto{\pgfqpoint{2.815641in}{3.123224in}}%
\pgfpathlineto{\pgfqpoint{2.823669in}{3.133929in}}%
\pgfpathlineto{\pgfqpoint{2.826880in}{3.134737in}}%
\pgfpathlineto{\pgfqpoint{2.829824in}{3.133085in}}%
\pgfpathlineto{\pgfqpoint{2.834105in}{3.127522in}}%
\pgfpathlineto{\pgfqpoint{2.842401in}{3.116451in}}%
\pgfpathlineto{\pgfqpoint{2.845613in}{3.115693in}}%
\pgfpathlineto{\pgfqpoint{2.848556in}{3.117389in}}%
\pgfpathlineto{\pgfqpoint{2.852838in}{3.122990in}}%
\pgfpathlineto{\pgfqpoint{2.861134in}{3.134007in}}%
\pgfpathlineto{\pgfqpoint{2.864345in}{3.134714in}}%
\pgfpathlineto{\pgfqpoint{2.867289in}{3.132974in}}%
\pgfpathlineto{\pgfqpoint{2.871571in}{3.127336in}}%
\pgfpathlineto{\pgfqpoint{2.879599in}{3.116551in}}%
\pgfpathlineto{\pgfqpoint{2.882810in}{3.115667in}}%
\pgfpathlineto{\pgfqpoint{2.885754in}{3.117252in}}%
\pgfpathlineto{\pgfqpoint{2.889768in}{3.122342in}}%
\pgfpathlineto{\pgfqpoint{2.898599in}{3.134081in}}%
\pgfpathlineto{\pgfqpoint{2.901810in}{3.134687in}}%
\pgfpathlineto{\pgfqpoint{2.904754in}{3.132861in}}%
\pgfpathlineto{\pgfqpoint{2.909036in}{3.127149in}}%
\pgfpathlineto{\pgfqpoint{2.917064in}{3.116471in}}%
\pgfpathlineto{\pgfqpoint{2.920275in}{3.115687in}}%
\pgfpathlineto{\pgfqpoint{2.923219in}{3.117360in}}%
\pgfpathlineto{\pgfqpoint{2.927501in}{3.122942in}}%
\pgfpathlineto{\pgfqpoint{2.935796in}{3.133987in}}%
\pgfpathlineto{\pgfqpoint{2.939008in}{3.134720in}}%
\pgfpathlineto{\pgfqpoint{2.941951in}{3.133003in}}%
\pgfpathlineto{\pgfqpoint{2.946233in}{3.127384in}}%
\pgfpathlineto{\pgfqpoint{2.954261in}{3.116572in}}%
\pgfpathlineto{\pgfqpoint{2.957473in}{3.115662in}}%
\pgfpathlineto{\pgfqpoint{2.960416in}{3.117225in}}%
\pgfpathlineto{\pgfqpoint{2.964430in}{3.122295in}}%
\pgfpathlineto{\pgfqpoint{2.973262in}{3.134062in}}%
\pgfpathlineto{\pgfqpoint{2.976473in}{3.134694in}}%
\pgfpathlineto{\pgfqpoint{2.979416in}{3.132890in}}%
\pgfpathlineto{\pgfqpoint{2.983698in}{3.127197in}}%
\pgfpathlineto{\pgfqpoint{2.991726in}{3.116491in}}%
\pgfpathlineto{\pgfqpoint{2.994938in}{3.115682in}}%
\pgfpathlineto{\pgfqpoint{2.997881in}{3.117332in}}%
\pgfpathlineto{\pgfqpoint{3.002163in}{3.122894in}}%
\pgfpathlineto{\pgfqpoint{3.010459in}{3.133967in}}%
\pgfpathlineto{\pgfqpoint{3.013670in}{3.134726in}}%
\pgfpathlineto{\pgfqpoint{3.016614in}{3.133032in}}%
\pgfpathlineto{\pgfqpoint{3.020896in}{3.127432in}}%
\pgfpathlineto{\pgfqpoint{3.029192in}{3.116413in}}%
\pgfpathlineto{\pgfqpoint{3.032403in}{3.115705in}}%
\pgfpathlineto{\pgfqpoint{3.035347in}{3.117443in}}%
\pgfpathlineto{\pgfqpoint{3.039628in}{3.123080in}}%
\pgfpathlineto{\pgfqpoint{3.047656in}{3.133867in}}%
\pgfpathlineto{\pgfqpoint{3.050868in}{3.134752in}}%
\pgfpathlineto{\pgfqpoint{3.053811in}{3.133168in}}%
\pgfpathlineto{\pgfqpoint{3.057826in}{3.128080in}}%
\pgfpathlineto{\pgfqpoint{3.066657in}{3.116339in}}%
\pgfpathlineto{\pgfqpoint{3.069868in}{3.115732in}}%
\pgfpathlineto{\pgfqpoint{3.072812in}{3.117557in}}%
\pgfpathlineto{\pgfqpoint{3.077093in}{3.123267in}}%
\pgfpathlineto{\pgfqpoint{3.085122in}{3.133947in}}%
\pgfpathlineto{\pgfqpoint{3.088333in}{3.134732in}}%
\pgfpathlineto{\pgfqpoint{3.091277in}{3.133060in}}%
\pgfpathlineto{\pgfqpoint{3.095558in}{3.127480in}}%
\pgfpathlineto{\pgfqpoint{3.103854in}{3.116433in}}%
\pgfpathlineto{\pgfqpoint{3.107065in}{3.115698in}}%
\pgfpathlineto{\pgfqpoint{3.110009in}{3.117414in}}%
\pgfpathlineto{\pgfqpoint{3.114291in}{3.123032in}}%
\pgfpathlineto{\pgfqpoint{3.122319in}{3.133845in}}%
\pgfpathlineto{\pgfqpoint{3.125530in}{3.134757in}}%
\pgfpathlineto{\pgfqpoint{3.128474in}{3.133196in}}%
\pgfpathlineto{\pgfqpoint{3.132488in}{3.128127in}}%
\pgfpathlineto{\pgfqpoint{3.141319in}{3.116358in}}%
\pgfpathlineto{\pgfqpoint{3.144531in}{3.115724in}}%
\pgfpathlineto{\pgfqpoint{3.147474in}{3.117527in}}%
\pgfpathlineto{\pgfqpoint{3.151756in}{3.123219in}}%
\pgfpathlineto{\pgfqpoint{3.159784in}{3.133927in}}%
\pgfpathlineto{\pgfqpoint{3.162995in}{3.134738in}}%
\pgfpathlineto{\pgfqpoint{3.165939in}{3.133088in}}%
\pgfpathlineto{\pgfqpoint{3.170221in}{3.127527in}}%
\pgfpathlineto{\pgfqpoint{3.178517in}{3.116453in}}%
\pgfpathlineto{\pgfqpoint{3.181728in}{3.115692in}}%
\pgfpathlineto{\pgfqpoint{3.184672in}{3.117386in}}%
\pgfpathlineto{\pgfqpoint{3.188953in}{3.122984in}}%
\pgfpathlineto{\pgfqpoint{3.197249in}{3.134005in}}%
\pgfpathlineto{\pgfqpoint{3.200461in}{3.134714in}}%
\pgfpathlineto{\pgfqpoint{3.203404in}{3.132978in}}%
\pgfpathlineto{\pgfqpoint{3.207686in}{3.127341in}}%
\pgfpathlineto{\pgfqpoint{3.215714in}{3.116553in}}%
\pgfpathlineto{\pgfqpoint{3.218925in}{3.115666in}}%
\pgfpathlineto{\pgfqpoint{3.221869in}{3.117249in}}%
\pgfpathlineto{\pgfqpoint{3.225883in}{3.122337in}}%
\pgfpathlineto{\pgfqpoint{3.234714in}{3.134079in}}%
\pgfpathlineto{\pgfqpoint{3.237926in}{3.134688in}}%
\pgfpathlineto{\pgfqpoint{3.240869in}{3.132864in}}%
\pgfpathlineto{\pgfqpoint{3.245151in}{3.127155in}}%
\pgfpathlineto{\pgfqpoint{3.253179in}{3.116473in}}%
\pgfpathlineto{\pgfqpoint{3.256391in}{3.115687in}}%
\pgfpathlineto{\pgfqpoint{3.259334in}{3.117357in}}%
\pgfpathlineto{\pgfqpoint{3.263616in}{3.122937in}}%
\pgfpathlineto{\pgfqpoint{3.271912in}{3.133985in}}%
\pgfpathlineto{\pgfqpoint{3.275123in}{3.134721in}}%
\pgfpathlineto{\pgfqpoint{3.278067in}{3.133006in}}%
\pgfpathlineto{\pgfqpoint{3.282349in}{3.127389in}}%
\pgfpathlineto{\pgfqpoint{3.290377in}{3.116575in}}%
\pgfpathlineto{\pgfqpoint{3.293588in}{3.115662in}}%
\pgfpathlineto{\pgfqpoint{3.296532in}{3.117222in}}%
\pgfpathlineto{\pgfqpoint{3.300546in}{3.122290in}}%
\pgfpathlineto{\pgfqpoint{3.309377in}{3.134060in}}%
\pgfpathlineto{\pgfqpoint{3.312588in}{3.134695in}}%
\pgfpathlineto{\pgfqpoint{3.315532in}{3.132893in}}%
\pgfpathlineto{\pgfqpoint{3.319814in}{3.127203in}}%
\pgfpathlineto{\pgfqpoint{3.327842in}{3.116493in}}%
\pgfpathlineto{\pgfqpoint{3.331053in}{3.115681in}}%
\pgfpathlineto{\pgfqpoint{3.333997in}{3.117329in}}%
\pgfpathlineto{\pgfqpoint{3.338279in}{3.122889in}}%
\pgfpathlineto{\pgfqpoint{3.346574in}{3.133965in}}%
\pgfpathlineto{\pgfqpoint{3.349786in}{3.134727in}}%
\pgfpathlineto{\pgfqpoint{3.352729in}{3.133035in}}%
\pgfpathlineto{\pgfqpoint{3.357011in}{3.127437in}}%
\pgfpathlineto{\pgfqpoint{3.365307in}{3.116415in}}%
\pgfpathlineto{\pgfqpoint{3.368518in}{3.115704in}}%
\pgfpathlineto{\pgfqpoint{3.371462in}{3.117440in}}%
\pgfpathlineto{\pgfqpoint{3.375744in}{3.123075in}}%
\pgfpathlineto{\pgfqpoint{3.383772in}{3.133864in}}%
\pgfpathlineto{\pgfqpoint{3.386983in}{3.134753in}}%
\pgfpathlineto{\pgfqpoint{3.389927in}{3.133171in}}%
\pgfpathlineto{\pgfqpoint{3.393941in}{3.128085in}}%
\pgfpathlineto{\pgfqpoint{3.402772in}{3.116341in}}%
\pgfpathlineto{\pgfqpoint{3.405983in}{3.115731in}}%
\pgfpathlineto{\pgfqpoint{3.408927in}{3.117553in}}%
\pgfpathlineto{\pgfqpoint{3.413209in}{3.123262in}}%
\pgfpathlineto{\pgfqpoint{3.421237in}{3.133945in}}%
\pgfpathlineto{\pgfqpoint{3.424448in}{3.134733in}}%
\pgfpathlineto{\pgfqpoint{3.427392in}{3.133063in}}%
\pgfpathlineto{\pgfqpoint{3.431674in}{3.127485in}}%
\pgfpathlineto{\pgfqpoint{3.439970in}{3.116435in}}%
\pgfpathlineto{\pgfqpoint{3.443181in}{3.115698in}}%
\pgfpathlineto{\pgfqpoint{3.446125in}{3.117411in}}%
\pgfpathlineto{\pgfqpoint{3.450406in}{3.123027in}}%
\pgfpathlineto{\pgfqpoint{3.458434in}{3.133843in}}%
\pgfpathlineto{\pgfqpoint{3.461646in}{3.134757in}}%
\pgfpathlineto{\pgfqpoint{3.464589in}{3.133199in}}%
\pgfpathlineto{\pgfqpoint{3.468604in}{3.128132in}}%
\pgfpathlineto{\pgfqpoint{3.477435in}{3.116360in}}%
\pgfpathlineto{\pgfqpoint{3.480646in}{3.115724in}}%
\pgfpathlineto{\pgfqpoint{3.483590in}{3.117524in}}%
\pgfpathlineto{\pgfqpoint{3.487871in}{3.123213in}}%
\pgfpathlineto{\pgfqpoint{3.495900in}{3.133924in}}%
\pgfpathlineto{\pgfqpoint{3.499111in}{3.134738in}}%
\pgfpathlineto{\pgfqpoint{3.502055in}{3.133091in}}%
\pgfpathlineto{\pgfqpoint{3.506336in}{3.127533in}}%
\pgfpathlineto{\pgfqpoint{3.514632in}{3.116455in}}%
\pgfpathlineto{\pgfqpoint{3.517843in}{3.115692in}}%
\pgfpathlineto{\pgfqpoint{3.520787in}{3.117382in}}%
\pgfpathlineto{\pgfqpoint{3.525069in}{3.122979in}}%
\pgfpathlineto{\pgfqpoint{3.533365in}{3.134002in}}%
\pgfpathlineto{\pgfqpoint{3.536576in}{3.134715in}}%
\pgfpathlineto{\pgfqpoint{3.539520in}{3.132981in}}%
\pgfpathlineto{\pgfqpoint{3.543801in}{3.127347in}}%
\pgfpathlineto{\pgfqpoint{3.551830in}{3.116556in}}%
\pgfpathlineto{\pgfqpoint{3.555041in}{3.115666in}}%
\pgfpathlineto{\pgfqpoint{3.557985in}{3.117246in}}%
\pgfpathlineto{\pgfqpoint{3.561999in}{3.122331in}}%
\pgfpathlineto{\pgfqpoint{3.570830in}{3.134077in}}%
\pgfpathlineto{\pgfqpoint{3.574041in}{3.134688in}}%
\pgfpathlineto{\pgfqpoint{3.576985in}{3.132867in}}%
\pgfpathlineto{\pgfqpoint{3.581266in}{3.127160in}}%
\pgfpathlineto{\pgfqpoint{3.589295in}{3.116475in}}%
\pgfpathlineto{\pgfqpoint{3.592506in}{3.115686in}}%
\pgfpathlineto{\pgfqpoint{3.595450in}{3.117354in}}%
\pgfpathlineto{\pgfqpoint{3.599731in}{3.122931in}}%
\pgfpathlineto{\pgfqpoint{3.608027in}{3.133983in}}%
\pgfpathlineto{\pgfqpoint{3.611239in}{3.134721in}}%
\pgfpathlineto{\pgfqpoint{3.614182in}{3.133009in}}%
\pgfpathlineto{\pgfqpoint{3.618464in}{3.127395in}}%
\pgfpathlineto{\pgfqpoint{3.626492in}{3.116577in}}%
\pgfpathlineto{\pgfqpoint{3.629703in}{3.115661in}}%
\pgfpathlineto{\pgfqpoint{3.632647in}{3.117219in}}%
\pgfpathlineto{\pgfqpoint{3.636661in}{3.122285in}}%
\pgfpathlineto{\pgfqpoint{3.645492in}{3.134058in}}%
\pgfpathlineto{\pgfqpoint{3.648704in}{3.134696in}}%
\pgfpathlineto{\pgfqpoint{3.651647in}{3.132897in}}%
\pgfpathlineto{\pgfqpoint{3.655929in}{3.127208in}}%
\pgfpathlineto{\pgfqpoint{3.663957in}{3.116496in}}%
\pgfpathlineto{\pgfqpoint{3.667169in}{3.115680in}}%
\pgfpathlineto{\pgfqpoint{3.670112in}{3.117326in}}%
\pgfpathlineto{\pgfqpoint{3.674394in}{3.122883in}}%
\pgfpathlineto{\pgfqpoint{3.682690in}{3.133963in}}%
\pgfpathlineto{\pgfqpoint{3.685901in}{3.134727in}}%
\pgfpathlineto{\pgfqpoint{3.688845in}{3.133038in}}%
\pgfpathlineto{\pgfqpoint{3.693127in}{3.127443in}}%
\pgfpathlineto{\pgfqpoint{3.701422in}{3.116418in}}%
\pgfpathlineto{\pgfqpoint{3.704634in}{3.115703in}}%
\pgfpathlineto{\pgfqpoint{3.707577in}{3.117436in}}%
\pgfpathlineto{\pgfqpoint{3.711859in}{3.123069in}}%
\pgfpathlineto{\pgfqpoint{3.719887in}{3.133862in}}%
\pgfpathlineto{\pgfqpoint{3.723099in}{3.134753in}}%
\pgfpathlineto{\pgfqpoint{3.726042in}{3.133174in}}%
\pgfpathlineto{\pgfqpoint{3.730056in}{3.128090in}}%
\pgfpathlineto{\pgfqpoint{3.738887in}{3.116343in}}%
\pgfpathlineto{\pgfqpoint{3.742099in}{3.115730in}}%
\pgfpathlineto{\pgfqpoint{3.745042in}{3.117550in}}%
\pgfpathlineto{\pgfqpoint{3.749324in}{3.123256in}}%
\pgfpathlineto{\pgfqpoint{3.757352in}{3.133943in}}%
\pgfpathlineto{\pgfqpoint{3.760564in}{3.134733in}}%
\pgfpathlineto{\pgfqpoint{3.763507in}{3.133066in}}%
\pgfpathlineto{\pgfqpoint{3.767789in}{3.127490in}}%
\pgfpathlineto{\pgfqpoint{3.776085in}{3.116437in}}%
\pgfpathlineto{\pgfqpoint{3.779296in}{3.115697in}}%
\pgfpathlineto{\pgfqpoint{3.782240in}{3.117408in}}%
\pgfpathlineto{\pgfqpoint{3.786522in}{3.123022in}}%
\pgfpathlineto{\pgfqpoint{3.794550in}{3.133841in}}%
\pgfpathlineto{\pgfqpoint{3.797761in}{3.134758in}}%
\pgfpathlineto{\pgfqpoint{3.800705in}{3.133202in}}%
\pgfpathlineto{\pgfqpoint{3.804719in}{3.128137in}}%
\pgfpathlineto{\pgfqpoint{3.813550in}{3.116362in}}%
\pgfpathlineto{\pgfqpoint{3.816761in}{3.115723in}}%
\pgfpathlineto{\pgfqpoint{3.819705in}{3.117520in}}%
\pgfpathlineto{\pgfqpoint{3.823987in}{3.123208in}}%
\pgfpathlineto{\pgfqpoint{3.832015in}{3.133922in}}%
\pgfpathlineto{\pgfqpoint{3.835226in}{3.134739in}}%
\pgfpathlineto{\pgfqpoint{3.838170in}{3.133094in}}%
\pgfpathlineto{\pgfqpoint{3.842452in}{3.127538in}}%
\pgfpathlineto{\pgfqpoint{3.850748in}{3.116457in}}%
\pgfpathlineto{\pgfqpoint{3.853959in}{3.115691in}}%
\pgfpathlineto{\pgfqpoint{3.856903in}{3.117379in}}%
\pgfpathlineto{\pgfqpoint{3.861184in}{3.122974in}}%
\pgfpathlineto{\pgfqpoint{3.869480in}{3.134000in}}%
\pgfpathlineto{\pgfqpoint{3.872691in}{3.134716in}}%
\pgfpathlineto{\pgfqpoint{3.875635in}{3.132984in}}%
\pgfpathlineto{\pgfqpoint{3.879917in}{3.127352in}}%
\pgfpathlineto{\pgfqpoint{3.887945in}{3.116558in}}%
\pgfpathlineto{\pgfqpoint{3.891156in}{3.115665in}}%
\pgfpathlineto{\pgfqpoint{3.894100in}{3.117243in}}%
\pgfpathlineto{\pgfqpoint{3.898114in}{3.122326in}}%
\pgfpathlineto{\pgfqpoint{3.906945in}{3.134075in}}%
\pgfpathlineto{\pgfqpoint{3.910156in}{3.134689in}}%
\pgfpathlineto{\pgfqpoint{3.913100in}{3.132871in}}%
\pgfpathlineto{\pgfqpoint{3.917382in}{3.127165in}}%
\pgfpathlineto{\pgfqpoint{3.925410in}{3.116477in}}%
\pgfpathlineto{\pgfqpoint{3.928621in}{3.115685in}}%
\pgfpathlineto{\pgfqpoint{3.931565in}{3.117351in}}%
\pgfpathlineto{\pgfqpoint{3.935847in}{3.122926in}}%
\pgfpathlineto{\pgfqpoint{3.944143in}{3.133980in}}%
\pgfpathlineto{\pgfqpoint{3.947354in}{3.134722in}}%
\pgfpathlineto{\pgfqpoint{3.950298in}{3.133013in}}%
\pgfpathlineto{\pgfqpoint{3.954579in}{3.127400in}}%
\pgfpathlineto{\pgfqpoint{3.962608in}{3.116579in}}%
\pgfpathlineto{\pgfqpoint{3.965819in}{3.115661in}}%
\pgfpathlineto{\pgfqpoint{3.968763in}{3.117216in}}%
\pgfpathlineto{\pgfqpoint{3.972777in}{3.122279in}}%
\pgfpathlineto{\pgfqpoint{3.981608in}{3.134056in}}%
\pgfpathlineto{\pgfqpoint{3.984819in}{3.134696in}}%
\pgfpathlineto{\pgfqpoint{3.987763in}{3.132900in}}%
\pgfpathlineto{\pgfqpoint{3.992044in}{3.127213in}}%
\pgfpathlineto{\pgfqpoint{4.000073in}{3.116498in}}%
\pgfpathlineto{\pgfqpoint{4.003284in}{3.115680in}}%
\pgfpathlineto{\pgfqpoint{4.006228in}{3.117323in}}%
\pgfpathlineto{\pgfqpoint{4.010509in}{3.122878in}}%
\pgfpathlineto{\pgfqpoint{4.018805in}{3.133960in}}%
\pgfpathlineto{\pgfqpoint{4.022017in}{3.134728in}}%
\pgfpathlineto{\pgfqpoint{4.024960in}{3.133041in}}%
\pgfpathlineto{\pgfqpoint{4.029242in}{3.127448in}}%
\pgfpathlineto{\pgfqpoint{4.037538in}{3.116420in}}%
\pgfpathlineto{\pgfqpoint{4.040749in}{3.115703in}}%
\pgfpathlineto{\pgfqpoint{4.043693in}{3.117433in}}%
\pgfpathlineto{\pgfqpoint{4.047975in}{3.123064in}}%
\pgfpathlineto{\pgfqpoint{4.056003in}{3.133860in}}%
\pgfpathlineto{\pgfqpoint{4.059214in}{3.134754in}}%
\pgfpathlineto{\pgfqpoint{4.062158in}{3.133177in}}%
\pgfpathlineto{\pgfqpoint{4.066172in}{3.128095in}}%
\pgfpathlineto{\pgfqpoint{4.075003in}{3.116345in}}%
\pgfpathlineto{\pgfqpoint{4.078214in}{3.115729in}}%
\pgfpathlineto{\pgfqpoint{4.081158in}{3.117547in}}%
\pgfpathlineto{\pgfqpoint{4.085440in}{3.123251in}}%
\pgfpathlineto{\pgfqpoint{4.093468in}{3.133940in}}%
\pgfpathlineto{\pgfqpoint{4.096679in}{3.134734in}}%
\pgfpathlineto{\pgfqpoint{4.099623in}{3.133069in}}%
\pgfpathlineto{\pgfqpoint{4.103905in}{3.127496in}}%
\pgfpathlineto{\pgfqpoint{4.112200in}{3.116440in}}%
\pgfpathlineto{\pgfqpoint{4.115412in}{3.115696in}}%
\pgfpathlineto{\pgfqpoint{4.118355in}{3.117405in}}%
\pgfpathlineto{\pgfqpoint{4.122637in}{3.123016in}}%
\pgfpathlineto{\pgfqpoint{4.130665in}{3.133838in}}%
\pgfpathlineto{\pgfqpoint{4.133877in}{3.134758in}}%
\pgfpathlineto{\pgfqpoint{4.136820in}{3.133205in}}%
\pgfpathlineto{\pgfqpoint{4.140834in}{3.128142in}}%
\pgfpathlineto{\pgfqpoint{4.149665in}{3.116364in}}%
\pgfpathlineto{\pgfqpoint{4.152877in}{3.115722in}}%
\pgfpathlineto{\pgfqpoint{4.155820in}{3.117517in}}%
\pgfpathlineto{\pgfqpoint{4.160102in}{3.123203in}}%
\pgfpathlineto{\pgfqpoint{4.168130in}{3.133920in}}%
\pgfpathlineto{\pgfqpoint{4.171342in}{3.134739in}}%
\pgfpathlineto{\pgfqpoint{4.174285in}{3.133097in}}%
\pgfpathlineto{\pgfqpoint{4.178567in}{3.127543in}}%
\pgfpathlineto{\pgfqpoint{4.186863in}{3.116460in}}%
\pgfpathlineto{\pgfqpoint{4.190074in}{3.115690in}}%
\pgfpathlineto{\pgfqpoint{4.193018in}{3.117376in}}%
\pgfpathlineto{\pgfqpoint{4.197300in}{3.122968in}}%
\pgfpathlineto{\pgfqpoint{4.205596in}{3.133998in}}%
\pgfpathlineto{\pgfqpoint{4.208807in}{3.134717in}}%
\pgfpathlineto{\pgfqpoint{4.211751in}{3.132987in}}%
\pgfpathlineto{\pgfqpoint{4.216032in}{3.127357in}}%
\pgfpathlineto{\pgfqpoint{4.224060in}{3.116561in}}%
\pgfpathlineto{\pgfqpoint{4.227272in}{3.115665in}}%
\pgfpathlineto{\pgfqpoint{4.230215in}{3.117240in}}%
\pgfpathlineto{\pgfqpoint{4.234230in}{3.122321in}}%
\pgfpathlineto{\pgfqpoint{4.243061in}{3.134073in}}%
\pgfpathlineto{\pgfqpoint{4.246272in}{3.134690in}}%
\pgfpathlineto{\pgfqpoint{4.249216in}{3.132874in}}%
\pgfpathlineto{\pgfqpoint{4.253497in}{3.127171in}}%
\pgfpathlineto{\pgfqpoint{4.261526in}{3.116480in}}%
\pgfpathlineto{\pgfqpoint{4.264737in}{3.115685in}}%
\pgfpathlineto{\pgfqpoint{4.267681in}{3.117348in}}%
\pgfpathlineto{\pgfqpoint{4.271962in}{3.122921in}}%
\pgfpathlineto{\pgfqpoint{4.280258in}{3.133978in}}%
\pgfpathlineto{\pgfqpoint{4.283469in}{3.134723in}}%
\pgfpathlineto{\pgfqpoint{4.286413in}{3.133016in}}%
\pgfpathlineto{\pgfqpoint{4.290695in}{3.127405in}}%
\pgfpathlineto{\pgfqpoint{4.298723in}{3.116582in}}%
\pgfpathlineto{\pgfqpoint{4.301934in}{3.115660in}}%
\pgfpathlineto{\pgfqpoint{4.304878in}{3.117213in}}%
\pgfpathlineto{\pgfqpoint{4.308892in}{3.122274in}}%
\pgfpathlineto{\pgfqpoint{4.317723in}{3.134054in}}%
\pgfpathlineto{\pgfqpoint{4.320934in}{3.134697in}}%
\pgfpathlineto{\pgfqpoint{4.323878in}{3.132903in}}%
\pgfpathlineto{\pgfqpoint{4.328160in}{3.127219in}}%
\pgfpathlineto{\pgfqpoint{4.336188in}{3.116500in}}%
\pgfpathlineto{\pgfqpoint{4.339399in}{3.115679in}}%
\pgfpathlineto{\pgfqpoint{4.342343in}{3.117320in}}%
\pgfpathlineto{\pgfqpoint{4.346625in}{3.122873in}}%
\pgfpathlineto{\pgfqpoint{4.354921in}{3.133958in}}%
\pgfpathlineto{\pgfqpoint{4.358132in}{3.134729in}}%
\pgfpathlineto{\pgfqpoint{4.361076in}{3.133044in}}%
\pgfpathlineto{\pgfqpoint{4.365357in}{3.127453in}}%
\pgfpathlineto{\pgfqpoint{4.373653in}{3.116422in}}%
\pgfpathlineto{\pgfqpoint{4.376865in}{3.115702in}}%
\pgfpathlineto{\pgfqpoint{4.379808in}{3.117430in}}%
\pgfpathlineto{\pgfqpoint{4.384090in}{3.123059in}}%
\pgfpathlineto{\pgfqpoint{4.392118in}{3.133857in}}%
\pgfpathlineto{\pgfqpoint{4.395329in}{3.134754in}}%
\pgfpathlineto{\pgfqpoint{4.398273in}{3.133180in}}%
\pgfpathlineto{\pgfqpoint{4.402287in}{3.128100in}}%
\pgfpathlineto{\pgfqpoint{4.411118in}{3.116347in}}%
\pgfpathlineto{\pgfqpoint{4.414330in}{3.115728in}}%
\pgfpathlineto{\pgfqpoint{4.417273in}{3.117543in}}%
\pgfpathlineto{\pgfqpoint{4.421555in}{3.123246in}}%
\pgfpathlineto{\pgfqpoint{4.429583in}{3.133938in}}%
\pgfpathlineto{\pgfqpoint{4.432795in}{3.134734in}}%
\pgfpathlineto{\pgfqpoint{4.435738in}{3.133073in}}%
\pgfpathlineto{\pgfqpoint{4.440020in}{3.127501in}}%
\pgfpathlineto{\pgfqpoint{4.448316in}{3.116442in}}%
\pgfpathlineto{\pgfqpoint{4.451527in}{3.115696in}}%
\pgfpathlineto{\pgfqpoint{4.454471in}{3.117401in}}%
\pgfpathlineto{\pgfqpoint{4.458753in}{3.123011in}}%
\pgfpathlineto{\pgfqpoint{4.466781in}{3.133836in}}%
\pgfpathlineto{\pgfqpoint{4.469992in}{3.134759in}}%
\pgfpathlineto{\pgfqpoint{4.472936in}{3.133208in}}%
\pgfpathlineto{\pgfqpoint{4.476950in}{3.128147in}}%
\pgfpathlineto{\pgfqpoint{4.485781in}{3.116366in}}%
\pgfpathlineto{\pgfqpoint{4.488992in}{3.115721in}}%
\pgfpathlineto{\pgfqpoint{4.491936in}{3.117514in}}%
\pgfpathlineto{\pgfqpoint{4.496218in}{3.123197in}}%
\pgfpathlineto{\pgfqpoint{4.504246in}{3.133918in}}%
\pgfpathlineto{\pgfqpoint{4.507457in}{3.134740in}}%
\pgfpathlineto{\pgfqpoint{4.510401in}{3.133101in}}%
\pgfpathlineto{\pgfqpoint{4.514683in}{3.127549in}}%
\pgfpathlineto{\pgfqpoint{4.522978in}{3.116462in}}%
\pgfpathlineto{\pgfqpoint{4.526190in}{3.115690in}}%
\pgfpathlineto{\pgfqpoint{4.529133in}{3.117373in}}%
\pgfpathlineto{\pgfqpoint{4.533415in}{3.122963in}}%
\pgfpathlineto{\pgfqpoint{4.541711in}{3.133996in}}%
\pgfpathlineto{\pgfqpoint{4.544922in}{3.134717in}}%
\pgfpathlineto{\pgfqpoint{4.547866in}{3.132990in}}%
\pgfpathlineto{\pgfqpoint{4.552148in}{3.127363in}}%
\pgfpathlineto{\pgfqpoint{4.560176in}{3.116563in}}%
\pgfpathlineto{\pgfqpoint{4.563387in}{3.115664in}}%
\pgfpathlineto{\pgfqpoint{4.566331in}{3.117237in}}%
\pgfpathlineto{\pgfqpoint{4.570345in}{3.122316in}}%
\pgfpathlineto{\pgfqpoint{4.579176in}{3.134071in}}%
\pgfpathlineto{\pgfqpoint{4.582387in}{3.134691in}}%
\pgfpathlineto{\pgfqpoint{4.585331in}{3.132877in}}%
\pgfpathlineto{\pgfqpoint{4.589613in}{3.127176in}}%
\pgfpathlineto{\pgfqpoint{4.597641in}{3.116482in}}%
\pgfpathlineto{\pgfqpoint{4.600852in}{3.115684in}}%
\pgfpathlineto{\pgfqpoint{4.603796in}{3.117345in}}%
\pgfpathlineto{\pgfqpoint{4.608078in}{3.122915in}}%
\pgfpathlineto{\pgfqpoint{4.616374in}{3.133976in}}%
\pgfpathlineto{\pgfqpoint{4.619585in}{3.134723in}}%
\pgfpathlineto{\pgfqpoint{4.622529in}{3.133019in}}%
\pgfpathlineto{\pgfqpoint{4.626810in}{3.127411in}}%
\pgfpathlineto{\pgfqpoint{4.634838in}{3.116584in}}%
\pgfpathlineto{\pgfqpoint{4.638050in}{3.115660in}}%
\pgfpathlineto{\pgfqpoint{4.640993in}{3.117210in}}%
\pgfpathlineto{\pgfqpoint{4.645008in}{3.122269in}}%
\pgfpathlineto{\pgfqpoint{4.653839in}{3.134052in}}%
\pgfpathlineto{\pgfqpoint{4.657050in}{3.134698in}}%
\pgfpathlineto{\pgfqpoint{4.659994in}{3.132907in}}%
\pgfpathlineto{\pgfqpoint{4.664275in}{3.127224in}}%
\pgfpathlineto{\pgfqpoint{4.672304in}{3.116502in}}%
\pgfpathlineto{\pgfqpoint{4.675515in}{3.115679in}}%
\pgfpathlineto{\pgfqpoint{4.678459in}{3.117317in}}%
\pgfpathlineto{\pgfqpoint{4.682740in}{3.122868in}}%
\pgfpathlineto{\pgfqpoint{4.691036in}{3.133956in}}%
\pgfpathlineto{\pgfqpoint{4.694247in}{3.134729in}}%
\pgfpathlineto{\pgfqpoint{4.697191in}{3.133047in}}%
\pgfpathlineto{\pgfqpoint{4.701473in}{3.127458in}}%
\pgfpathlineto{\pgfqpoint{4.709769in}{3.116424in}}%
\pgfpathlineto{\pgfqpoint{4.712980in}{3.115701in}}%
\pgfpathlineto{\pgfqpoint{4.715924in}{3.117427in}}%
\pgfpathlineto{\pgfqpoint{4.720205in}{3.123053in}}%
\pgfpathlineto{\pgfqpoint{4.728234in}{3.133855in}}%
\pgfpathlineto{\pgfqpoint{4.731445in}{3.134755in}}%
\pgfpathlineto{\pgfqpoint{4.734389in}{3.133183in}}%
\pgfpathlineto{\pgfqpoint{4.738403in}{3.128106in}}%
\pgfpathlineto{\pgfqpoint{4.747234in}{3.116349in}}%
\pgfpathlineto{\pgfqpoint{4.750445in}{3.115728in}}%
\pgfpathlineto{\pgfqpoint{4.753389in}{3.117540in}}%
\pgfpathlineto{\pgfqpoint{4.757670in}{3.123240in}}%
\pgfpathlineto{\pgfqpoint{4.765699in}{3.133936in}}%
\pgfpathlineto{\pgfqpoint{4.768910in}{3.134735in}}%
\pgfpathlineto{\pgfqpoint{4.771854in}{3.133076in}}%
\pgfpathlineto{\pgfqpoint{4.776135in}{3.127506in}}%
\pgfpathlineto{\pgfqpoint{4.784431in}{3.116444in}}%
\pgfpathlineto{\pgfqpoint{4.787643in}{3.115695in}}%
\pgfpathlineto{\pgfqpoint{4.790586in}{3.117398in}}%
\pgfpathlineto{\pgfqpoint{4.794868in}{3.123006in}}%
\pgfpathlineto{\pgfqpoint{4.802896in}{3.133833in}}%
\pgfpathlineto{\pgfqpoint{4.806107in}{3.134759in}}%
\pgfpathlineto{\pgfqpoint{4.809051in}{3.133211in}}%
\pgfpathlineto{\pgfqpoint{4.813065in}{3.128153in}}%
\pgfpathlineto{\pgfqpoint{4.821896in}{3.116368in}}%
\pgfpathlineto{\pgfqpoint{4.825108in}{3.115720in}}%
\pgfpathlineto{\pgfqpoint{4.828051in}{3.117511in}}%
\pgfpathlineto{\pgfqpoint{4.832333in}{3.123192in}}%
\pgfpathlineto{\pgfqpoint{4.840361in}{3.133915in}}%
\pgfpathlineto{\pgfqpoint{4.843573in}{3.134741in}}%
\pgfpathlineto{\pgfqpoint{4.846516in}{3.133104in}}%
\pgfpathlineto{\pgfqpoint{4.850798in}{3.127554in}}%
\pgfpathlineto{\pgfqpoint{4.859094in}{3.116464in}}%
\pgfpathlineto{\pgfqpoint{4.862305in}{3.115689in}}%
\pgfpathlineto{\pgfqpoint{4.865249in}{3.117370in}}%
\pgfpathlineto{\pgfqpoint{4.869531in}{3.122958in}}%
\pgfpathlineto{\pgfqpoint{4.877826in}{3.133994in}}%
\pgfpathlineto{\pgfqpoint{4.881038in}{3.134718in}}%
\pgfpathlineto{\pgfqpoint{4.883981in}{3.132994in}}%
\pgfpathlineto{\pgfqpoint{4.888263in}{3.127368in}}%
\pgfpathlineto{\pgfqpoint{4.896291in}{3.116565in}}%
\pgfpathlineto{\pgfqpoint{4.899503in}{3.115664in}}%
\pgfpathlineto{\pgfqpoint{4.902446in}{3.117234in}}%
\pgfpathlineto{\pgfqpoint{4.906460in}{3.122311in}}%
\pgfpathlineto{\pgfqpoint{4.915291in}{3.134068in}}%
\pgfpathlineto{\pgfqpoint{4.918503in}{3.134692in}}%
\pgfpathlineto{\pgfqpoint{4.921446in}{3.132880in}}%
\pgfpathlineto{\pgfqpoint{4.925728in}{3.127181in}}%
\pgfpathlineto{\pgfqpoint{4.933756in}{3.116484in}}%
\pgfpathlineto{\pgfqpoint{4.936968in}{3.115683in}}%
\pgfpathlineto{\pgfqpoint{4.939911in}{3.117342in}}%
\pgfpathlineto{\pgfqpoint{4.944193in}{3.122910in}}%
\pgfpathlineto{\pgfqpoint{4.952489in}{3.133974in}}%
\pgfpathlineto{\pgfqpoint{4.955700in}{3.134724in}}%
\pgfpathlineto{\pgfqpoint{4.958644in}{3.133022in}}%
\pgfpathlineto{\pgfqpoint{4.962926in}{3.127416in}}%
\pgfpathlineto{\pgfqpoint{4.970954in}{3.116587in}}%
\pgfpathlineto{\pgfqpoint{4.974165in}{3.115659in}}%
\pgfpathlineto{\pgfqpoint{4.977109in}{3.117207in}}%
\pgfpathlineto{\pgfqpoint{4.981123in}{3.122264in}}%
\pgfpathlineto{\pgfqpoint{4.989954in}{3.134050in}}%
\pgfpathlineto{\pgfqpoint{4.993165in}{3.134699in}}%
\pgfpathlineto{\pgfqpoint{4.996109in}{3.132910in}}%
\pgfpathlineto{\pgfqpoint{5.000391in}{3.127229in}}%
\pgfpathlineto{\pgfqpoint{5.008419in}{3.116505in}}%
\pgfpathlineto{\pgfqpoint{5.011630in}{3.115678in}}%
\pgfpathlineto{\pgfqpoint{5.014574in}{3.117314in}}%
\pgfpathlineto{\pgfqpoint{5.018856in}{3.122862in}}%
\pgfpathlineto{\pgfqpoint{5.027152in}{3.133954in}}%
\pgfpathlineto{\pgfqpoint{5.030363in}{3.134730in}}%
\pgfpathlineto{\pgfqpoint{5.033307in}{3.133051in}}%
\pgfpathlineto{\pgfqpoint{5.037588in}{3.127464in}}%
\pgfpathlineto{\pgfqpoint{5.045884in}{3.116426in}}%
\pgfpathlineto{\pgfqpoint{5.049095in}{3.115701in}}%
\pgfpathlineto{\pgfqpoint{5.052039in}{3.117424in}}%
\pgfpathlineto{\pgfqpoint{5.056321in}{3.123048in}}%
\pgfpathlineto{\pgfqpoint{5.064349in}{3.133852in}}%
\pgfpathlineto{\pgfqpoint{5.067560in}{3.134755in}}%
\pgfpathlineto{\pgfqpoint{5.070504in}{3.133186in}}%
\pgfpathlineto{\pgfqpoint{5.074518in}{3.128111in}}%
\pgfpathlineto{\pgfqpoint{5.083349in}{3.116351in}}%
\pgfpathlineto{\pgfqpoint{5.086560in}{3.115727in}}%
\pgfpathlineto{\pgfqpoint{5.089504in}{3.117537in}}%
\pgfpathlineto{\pgfqpoint{5.093786in}{3.123235in}}%
\pgfpathlineto{\pgfqpoint{5.101814in}{3.133933in}}%
\pgfpathlineto{\pgfqpoint{5.105025in}{3.134736in}}%
\pgfpathlineto{\pgfqpoint{5.107969in}{3.133079in}}%
\pgfpathlineto{\pgfqpoint{5.112251in}{3.127512in}}%
\pgfpathlineto{\pgfqpoint{5.120547in}{3.116446in}}%
\pgfpathlineto{\pgfqpoint{5.123758in}{3.115694in}}%
\pgfpathlineto{\pgfqpoint{5.126702in}{3.117395in}}%
\pgfpathlineto{\pgfqpoint{5.130983in}{3.123000in}}%
\pgfpathlineto{\pgfqpoint{5.139279in}{3.134011in}}%
\pgfpathlineto{\pgfqpoint{5.142491in}{3.134712in}}%
\pgfpathlineto{\pgfqpoint{5.145434in}{3.132968in}}%
\pgfpathlineto{\pgfqpoint{5.149716in}{3.127325in}}%
\pgfpathlineto{\pgfqpoint{5.157744in}{3.116546in}}%
\pgfpathlineto{\pgfqpoint{5.160955in}{3.115668in}}%
\pgfpathlineto{\pgfqpoint{5.163899in}{3.117258in}}%
\pgfpathlineto{\pgfqpoint{5.167913in}{3.122352in}}%
\pgfpathlineto{\pgfqpoint{5.176744in}{3.134085in}}%
\pgfpathlineto{\pgfqpoint{5.179956in}{3.134685in}}%
\pgfpathlineto{\pgfqpoint{5.182899in}{3.132854in}}%
\pgfpathlineto{\pgfqpoint{5.187181in}{3.127139in}}%
\pgfpathlineto{\pgfqpoint{5.195209in}{3.116466in}}%
\pgfpathlineto{\pgfqpoint{5.198421in}{3.115688in}}%
\pgfpathlineto{\pgfqpoint{5.201364in}{3.117367in}}%
\pgfpathlineto{\pgfqpoint{5.205646in}{3.122952in}}%
\pgfpathlineto{\pgfqpoint{5.213942in}{3.133991in}}%
\pgfpathlineto{\pgfqpoint{5.217153in}{3.134719in}}%
\pgfpathlineto{\pgfqpoint{5.220097in}{3.132997in}}%
\pgfpathlineto{\pgfqpoint{5.224379in}{3.127373in}}%
\pgfpathlineto{\pgfqpoint{5.232407in}{3.116568in}}%
\pgfpathlineto{\pgfqpoint{5.235618in}{3.115663in}}%
\pgfpathlineto{\pgfqpoint{5.238562in}{3.117231in}}%
\pgfpathlineto{\pgfqpoint{5.242576in}{3.122305in}}%
\pgfpathlineto{\pgfqpoint{5.251407in}{3.134066in}}%
\pgfpathlineto{\pgfqpoint{5.254618in}{3.134693in}}%
\pgfpathlineto{\pgfqpoint{5.257562in}{3.132884in}}%
\pgfpathlineto{\pgfqpoint{5.261844in}{3.127187in}}%
\pgfpathlineto{\pgfqpoint{5.269872in}{3.116487in}}%
\pgfpathlineto{\pgfqpoint{5.273083in}{3.115683in}}%
\pgfpathlineto{\pgfqpoint{5.276027in}{3.117339in}}%
\pgfpathlineto{\pgfqpoint{5.280309in}{3.122905in}}%
\pgfpathlineto{\pgfqpoint{5.288604in}{3.133972in}}%
\pgfpathlineto{\pgfqpoint{5.291816in}{3.134725in}}%
\pgfpathlineto{\pgfqpoint{5.294759in}{3.133025in}}%
\pgfpathlineto{\pgfqpoint{5.299041in}{3.127421in}}%
\pgfpathlineto{\pgfqpoint{5.307337in}{3.116409in}}%
\pgfpathlineto{\pgfqpoint{5.310548in}{3.115706in}}%
\pgfpathlineto{\pgfqpoint{5.313492in}{3.117449in}}%
\pgfpathlineto{\pgfqpoint{5.317774in}{3.123091in}}%
\pgfpathlineto{\pgfqpoint{5.325802in}{3.133871in}}%
\pgfpathlineto{\pgfqpoint{5.329013in}{3.134751in}}%
\pgfpathlineto{\pgfqpoint{5.331957in}{3.133162in}}%
\pgfpathlineto{\pgfqpoint{5.335971in}{3.128069in}}%
\pgfpathlineto{\pgfqpoint{5.344802in}{3.116335in}}%
\pgfpathlineto{\pgfqpoint{5.348013in}{3.115733in}}%
\pgfpathlineto{\pgfqpoint{5.350957in}{3.117563in}}%
\pgfpathlineto{\pgfqpoint{5.355239in}{3.123278in}}%
\pgfpathlineto{\pgfqpoint{5.363267in}{3.133952in}}%
\pgfpathlineto{\pgfqpoint{5.366478in}{3.134731in}}%
\pgfpathlineto{\pgfqpoint{5.369422in}{3.133054in}}%
\pgfpathlineto{\pgfqpoint{5.373704in}{3.127469in}}%
\pgfpathlineto{\pgfqpoint{5.382000in}{3.116429in}}%
\pgfpathlineto{\pgfqpoint{5.385211in}{3.115700in}}%
\pgfpathlineto{\pgfqpoint{5.388154in}{3.117421in}}%
\pgfpathlineto{\pgfqpoint{5.392436in}{3.123043in}}%
\pgfpathlineto{\pgfqpoint{5.400464in}{3.133850in}}%
\pgfpathlineto{\pgfqpoint{5.403676in}{3.134756in}}%
\pgfpathlineto{\pgfqpoint{5.406619in}{3.133189in}}%
\pgfpathlineto{\pgfqpoint{5.410634in}{3.128116in}}%
\pgfpathlineto{\pgfqpoint{5.419465in}{3.116354in}}%
\pgfpathlineto{\pgfqpoint{5.422676in}{3.115726in}}%
\pgfpathlineto{\pgfqpoint{5.425620in}{3.117534in}}%
\pgfpathlineto{\pgfqpoint{5.429901in}{3.123230in}}%
\pgfpathlineto{\pgfqpoint{5.437930in}{3.133931in}}%
\pgfpathlineto{\pgfqpoint{5.441141in}{3.134736in}}%
\pgfpathlineto{\pgfqpoint{5.444085in}{3.133082in}}%
\pgfpathlineto{\pgfqpoint{5.448366in}{3.127517in}}%
\pgfpathlineto{\pgfqpoint{5.456662in}{3.116448in}}%
\pgfpathlineto{\pgfqpoint{5.459873in}{3.115694in}}%
\pgfpathlineto{\pgfqpoint{5.462817in}{3.117392in}}%
\pgfpathlineto{\pgfqpoint{5.467099in}{3.122995in}}%
\pgfpathlineto{\pgfqpoint{5.475395in}{3.134009in}}%
\pgfpathlineto{\pgfqpoint{5.478606in}{3.134713in}}%
\pgfpathlineto{\pgfqpoint{5.481550in}{3.132971in}}%
\pgfpathlineto{\pgfqpoint{5.485831in}{3.127331in}}%
\pgfpathlineto{\pgfqpoint{5.493860in}{3.116549in}}%
\pgfpathlineto{\pgfqpoint{5.497071in}{3.115667in}}%
\pgfpathlineto{\pgfqpoint{5.500015in}{3.117255in}}%
\pgfpathlineto{\pgfqpoint{5.504029in}{3.122347in}}%
\pgfpathlineto{\pgfqpoint{5.512860in}{3.134083in}}%
\pgfpathlineto{\pgfqpoint{5.516071in}{3.134686in}}%
\pgfpathlineto{\pgfqpoint{5.519015in}{3.132857in}}%
\pgfpathlineto{\pgfqpoint{5.523296in}{3.127144in}}%
\pgfpathlineto{\pgfqpoint{5.531325in}{3.116468in}}%
\pgfpathlineto{\pgfqpoint{5.534536in}{3.115688in}}%
\pgfpathlineto{\pgfqpoint{5.537480in}{3.117364in}}%
\pgfpathlineto{\pgfqpoint{5.541761in}{3.122947in}}%
\pgfpathlineto{\pgfqpoint{5.550057in}{3.133989in}}%
\pgfpathlineto{\pgfqpoint{5.553269in}{3.134719in}}%
\pgfpathlineto{\pgfqpoint{5.556212in}{3.133000in}}%
\pgfpathlineto{\pgfqpoint{5.560494in}{3.127379in}}%
\pgfpathlineto{\pgfqpoint{5.568522in}{3.116570in}}%
\pgfpathlineto{\pgfqpoint{5.571733in}{3.115663in}}%
\pgfpathlineto{\pgfqpoint{5.574677in}{3.117228in}}%
\pgfpathlineto{\pgfqpoint{5.578691in}{3.122300in}}%
\pgfpathlineto{\pgfqpoint{5.587522in}{3.134064in}}%
\pgfpathlineto{\pgfqpoint{5.590734in}{3.134693in}}%
\pgfpathlineto{\pgfqpoint{5.593677in}{3.132887in}}%
\pgfpathlineto{\pgfqpoint{5.597959in}{3.127192in}}%
\pgfpathlineto{\pgfqpoint{5.605987in}{3.116489in}}%
\pgfpathlineto{\pgfqpoint{5.609199in}{3.115682in}}%
\pgfpathlineto{\pgfqpoint{5.612142in}{3.117335in}}%
\pgfpathlineto{\pgfqpoint{5.616424in}{3.122899in}}%
\pgfpathlineto{\pgfqpoint{5.624720in}{3.133969in}}%
\pgfpathlineto{\pgfqpoint{5.627931in}{3.134726in}}%
\pgfpathlineto{\pgfqpoint{5.630875in}{3.133029in}}%
\pgfpathlineto{\pgfqpoint{5.635157in}{3.127427in}}%
\pgfpathlineto{\pgfqpoint{5.643452in}{3.116411in}}%
\pgfpathlineto{\pgfqpoint{5.646664in}{3.115705in}}%
\pgfpathlineto{\pgfqpoint{5.649607in}{3.117446in}}%
\pgfpathlineto{\pgfqpoint{5.653889in}{3.123085in}}%
\pgfpathlineto{\pgfqpoint{5.661917in}{3.133869in}}%
\pgfpathlineto{\pgfqpoint{5.665129in}{3.134752in}}%
\pgfpathlineto{\pgfqpoint{5.668072in}{3.133165in}}%
\pgfpathlineto{\pgfqpoint{5.672086in}{3.128074in}}%
\pgfpathlineto{\pgfqpoint{5.680917in}{3.116337in}}%
\pgfpathlineto{\pgfqpoint{5.684129in}{3.115732in}}%
\pgfpathlineto{\pgfqpoint{5.687072in}{3.117560in}}%
\pgfpathlineto{\pgfqpoint{5.691354in}{3.123272in}}%
\pgfpathlineto{\pgfqpoint{5.699382in}{3.133949in}}%
\pgfpathlineto{\pgfqpoint{5.702594in}{3.134731in}}%
\pgfpathlineto{\pgfqpoint{5.705537in}{3.133057in}}%
\pgfpathlineto{\pgfqpoint{5.709819in}{3.127474in}}%
\pgfpathlineto{\pgfqpoint{5.718115in}{3.116431in}}%
\pgfpathlineto{\pgfqpoint{5.721326in}{3.115699in}}%
\pgfpathlineto{\pgfqpoint{5.724270in}{3.117417in}}%
\pgfpathlineto{\pgfqpoint{5.728552in}{3.123037in}}%
\pgfpathlineto{\pgfqpoint{5.736580in}{3.133848in}}%
\pgfpathlineto{\pgfqpoint{5.739791in}{3.134756in}}%
\pgfpathlineto{\pgfqpoint{5.742735in}{3.133192in}}%
\pgfpathlineto{\pgfqpoint{5.746749in}{3.128121in}}%
\pgfpathlineto{\pgfqpoint{5.755580in}{3.116356in}}%
\pgfpathlineto{\pgfqpoint{5.758791in}{3.115725in}}%
\pgfpathlineto{\pgfqpoint{5.761735in}{3.117530in}}%
\pgfpathlineto{\pgfqpoint{5.766017in}{3.123224in}}%
\pgfpathlineto{\pgfqpoint{5.774045in}{3.133929in}}%
\pgfpathlineto{\pgfqpoint{5.777256in}{3.134737in}}%
\pgfpathlineto{\pgfqpoint{5.780200in}{3.133085in}}%
\pgfpathlineto{\pgfqpoint{5.784482in}{3.127522in}}%
\pgfpathlineto{\pgfqpoint{5.792778in}{3.116451in}}%
\pgfpathlineto{\pgfqpoint{5.795989in}{3.115693in}}%
\pgfpathlineto{\pgfqpoint{5.798932in}{3.117389in}}%
\pgfpathlineto{\pgfqpoint{5.803214in}{3.122990in}}%
\pgfpathlineto{\pgfqpoint{5.811510in}{3.134007in}}%
\pgfpathlineto{\pgfqpoint{5.814721in}{3.134714in}}%
\pgfpathlineto{\pgfqpoint{5.817665in}{3.132974in}}%
\pgfpathlineto{\pgfqpoint{5.821947in}{3.127336in}}%
\pgfpathlineto{\pgfqpoint{5.829975in}{3.116551in}}%
\pgfpathlineto{\pgfqpoint{5.833186in}{3.115667in}}%
\pgfpathlineto{\pgfqpoint{5.836130in}{3.117252in}}%
\pgfpathlineto{\pgfqpoint{5.840144in}{3.122342in}}%
\pgfpathlineto{\pgfqpoint{5.848975in}{3.134081in}}%
\pgfpathlineto{\pgfqpoint{5.852186in}{3.134687in}}%
\pgfpathlineto{\pgfqpoint{5.855130in}{3.132861in}}%
\pgfpathlineto{\pgfqpoint{5.859412in}{3.127149in}}%
\pgfpathlineto{\pgfqpoint{5.867440in}{3.116471in}}%
\pgfpathlineto{\pgfqpoint{5.870651in}{3.115687in}}%
\pgfpathlineto{\pgfqpoint{5.873595in}{3.117360in}}%
\pgfpathlineto{\pgfqpoint{5.877877in}{3.122942in}}%
\pgfpathlineto{\pgfqpoint{5.886173in}{3.133987in}}%
\pgfpathlineto{\pgfqpoint{5.889384in}{3.134720in}}%
\pgfpathlineto{\pgfqpoint{5.892328in}{3.133003in}}%
\pgfpathlineto{\pgfqpoint{5.896609in}{3.127384in}}%
\pgfpathlineto{\pgfqpoint{5.904638in}{3.116572in}}%
\pgfpathlineto{\pgfqpoint{5.907849in}{3.115662in}}%
\pgfpathlineto{\pgfqpoint{5.910793in}{3.117225in}}%
\pgfpathlineto{\pgfqpoint{5.914807in}{3.122295in}}%
\pgfpathlineto{\pgfqpoint{5.923638in}{3.134062in}}%
\pgfpathlineto{\pgfqpoint{5.926849in}{3.134694in}}%
\pgfpathlineto{\pgfqpoint{5.929793in}{3.132890in}}%
\pgfpathlineto{\pgfqpoint{5.934074in}{3.127197in}}%
\pgfpathlineto{\pgfqpoint{5.942103in}{3.116491in}}%
\pgfpathlineto{\pgfqpoint{5.945314in}{3.115682in}}%
\pgfpathlineto{\pgfqpoint{5.948258in}{3.117332in}}%
\pgfpathlineto{\pgfqpoint{5.952539in}{3.122894in}}%
\pgfpathlineto{\pgfqpoint{5.960835in}{3.133967in}}%
\pgfpathlineto{\pgfqpoint{5.964047in}{3.134726in}}%
\pgfpathlineto{\pgfqpoint{5.966990in}{3.133032in}}%
\pgfpathlineto{\pgfqpoint{5.971272in}{3.127432in}}%
\pgfpathlineto{\pgfqpoint{5.979568in}{3.116413in}}%
\pgfpathlineto{\pgfqpoint{5.982779in}{3.115705in}}%
\pgfpathlineto{\pgfqpoint{5.985723in}{3.117443in}}%
\pgfpathlineto{\pgfqpoint{5.990004in}{3.123080in}}%
\pgfpathlineto{\pgfqpoint{5.998033in}{3.133867in}}%
\pgfpathlineto{\pgfqpoint{6.001244in}{3.134752in}}%
\pgfpathlineto{\pgfqpoint{6.004188in}{3.133168in}}%
\pgfpathlineto{\pgfqpoint{6.008202in}{3.128080in}}%
\pgfpathlineto{\pgfqpoint{6.017033in}{3.116339in}}%
\pgfpathlineto{\pgfqpoint{6.020244in}{3.115732in}}%
\pgfpathlineto{\pgfqpoint{6.023188in}{3.117557in}}%
\pgfpathlineto{\pgfqpoint{6.027470in}{3.123267in}}%
\pgfpathlineto{\pgfqpoint{6.035498in}{3.133947in}}%
\pgfpathlineto{\pgfqpoint{6.038709in}{3.134732in}}%
\pgfpathlineto{\pgfqpoint{6.041653in}{3.133060in}}%
\pgfpathlineto{\pgfqpoint{6.045935in}{3.127480in}}%
\pgfpathlineto{\pgfqpoint{6.054230in}{3.116433in}}%
\pgfpathlineto{\pgfqpoint{6.057442in}{3.115698in}}%
\pgfpathlineto{\pgfqpoint{6.060385in}{3.117414in}}%
\pgfpathlineto{\pgfqpoint{6.064667in}{3.123032in}}%
\pgfpathlineto{\pgfqpoint{6.072695in}{3.133845in}}%
\pgfpathlineto{\pgfqpoint{6.075907in}{3.134757in}}%
\pgfpathlineto{\pgfqpoint{6.078850in}{3.133196in}}%
\pgfpathlineto{\pgfqpoint{6.082864in}{3.128127in}}%
\pgfpathlineto{\pgfqpoint{6.091695in}{3.116358in}}%
\pgfpathlineto{\pgfqpoint{6.094907in}{3.115724in}}%
\pgfpathlineto{\pgfqpoint{6.097850in}{3.117527in}}%
\pgfpathlineto{\pgfqpoint{6.102132in}{3.123219in}}%
\pgfpathlineto{\pgfqpoint{6.110160in}{3.133927in}}%
\pgfpathlineto{\pgfqpoint{6.113372in}{3.134738in}}%
\pgfpathlineto{\pgfqpoint{6.116315in}{3.133088in}}%
\pgfpathlineto{\pgfqpoint{6.120597in}{3.127527in}}%
\pgfpathlineto{\pgfqpoint{6.128893in}{3.116453in}}%
\pgfpathlineto{\pgfqpoint{6.132104in}{3.115692in}}%
\pgfpathlineto{\pgfqpoint{6.135048in}{3.117386in}}%
\pgfpathlineto{\pgfqpoint{6.139330in}{3.122984in}}%
\pgfpathlineto{\pgfqpoint{6.147625in}{3.134005in}}%
\pgfpathlineto{\pgfqpoint{6.150837in}{3.134714in}}%
\pgfpathlineto{\pgfqpoint{6.153780in}{3.132978in}}%
\pgfpathlineto{\pgfqpoint{6.158062in}{3.127341in}}%
\pgfpathlineto{\pgfqpoint{6.166090in}{3.116553in}}%
\pgfpathlineto{\pgfqpoint{6.169302in}{3.115666in}}%
\pgfpathlineto{\pgfqpoint{6.172245in}{3.117249in}}%
\pgfpathlineto{\pgfqpoint{6.176260in}{3.122337in}}%
\pgfpathlineto{\pgfqpoint{6.185091in}{3.134079in}}%
\pgfpathlineto{\pgfqpoint{6.188302in}{3.134688in}}%
\pgfpathlineto{\pgfqpoint{6.191246in}{3.132864in}}%
\pgfpathlineto{\pgfqpoint{6.195527in}{3.127155in}}%
\pgfpathlineto{\pgfqpoint{6.203556in}{3.116473in}}%
\pgfpathlineto{\pgfqpoint{6.206767in}{3.115687in}}%
\pgfpathlineto{\pgfqpoint{6.209711in}{3.117357in}}%
\pgfpathlineto{\pgfqpoint{6.213992in}{3.122937in}}%
\pgfpathlineto{\pgfqpoint{6.222288in}{3.133985in}}%
\pgfpathlineto{\pgfqpoint{6.225499in}{3.134721in}}%
\pgfpathlineto{\pgfqpoint{6.228443in}{3.133006in}}%
\pgfpathlineto{\pgfqpoint{6.232725in}{3.127389in}}%
\pgfpathlineto{\pgfqpoint{6.240753in}{3.116575in}}%
\pgfpathlineto{\pgfqpoint{6.243964in}{3.115662in}}%
\pgfpathlineto{\pgfqpoint{6.246908in}{3.117222in}}%
\pgfpathlineto{\pgfqpoint{6.250922in}{3.122290in}}%
\pgfpathlineto{\pgfqpoint{6.259753in}{3.134060in}}%
\pgfpathlineto{\pgfqpoint{6.262964in}{3.134695in}}%
\pgfpathlineto{\pgfqpoint{6.265908in}{3.132893in}}%
\pgfpathlineto{\pgfqpoint{6.270190in}{3.127203in}}%
\pgfpathlineto{\pgfqpoint{6.278218in}{3.116493in}}%
\pgfpathlineto{\pgfqpoint{6.281429in}{3.115681in}}%
\pgfpathlineto{\pgfqpoint{6.284373in}{3.117329in}}%
\pgfpathlineto{\pgfqpoint{6.288655in}{3.122889in}}%
\pgfpathlineto{\pgfqpoint{6.296951in}{3.133965in}}%
\pgfpathlineto{\pgfqpoint{6.300162in}{3.134727in}}%
\pgfpathlineto{\pgfqpoint{6.303106in}{3.133035in}}%
\pgfpathlineto{\pgfqpoint{6.307387in}{3.127437in}}%
\pgfpathlineto{\pgfqpoint{6.315683in}{3.116415in}}%
\pgfpathlineto{\pgfqpoint{6.318894in}{3.115704in}}%
\pgfpathlineto{\pgfqpoint{6.321838in}{3.117440in}}%
\pgfpathlineto{\pgfqpoint{6.326120in}{3.123075in}}%
\pgfpathlineto{\pgfqpoint{6.334148in}{3.133864in}}%
\pgfpathlineto{\pgfqpoint{6.337359in}{3.134753in}}%
\pgfpathlineto{\pgfqpoint{6.340303in}{3.133171in}}%
\pgfpathlineto{\pgfqpoint{6.344317in}{3.128085in}}%
\pgfpathlineto{\pgfqpoint{6.353148in}{3.116341in}}%
\pgfpathlineto{\pgfqpoint{6.356360in}{3.115731in}}%
\pgfpathlineto{\pgfqpoint{6.359303in}{3.117553in}}%
\pgfpathlineto{\pgfqpoint{6.363585in}{3.123262in}}%
\pgfpathlineto{\pgfqpoint{6.371613in}{3.133945in}}%
\pgfpathlineto{\pgfqpoint{6.374825in}{3.134733in}}%
\pgfpathlineto{\pgfqpoint{6.377768in}{3.133063in}}%
\pgfpathlineto{\pgfqpoint{6.382050in}{3.127485in}}%
\pgfpathlineto{\pgfqpoint{6.390346in}{3.116435in}}%
\pgfpathlineto{\pgfqpoint{6.393557in}{3.115698in}}%
\pgfpathlineto{\pgfqpoint{6.396501in}{3.117411in}}%
\pgfpathlineto{\pgfqpoint{6.400782in}{3.123027in}}%
\pgfpathlineto{\pgfqpoint{6.408811in}{3.133843in}}%
\pgfpathlineto{\pgfqpoint{6.412022in}{3.134757in}}%
\pgfpathlineto{\pgfqpoint{6.414966in}{3.133199in}}%
\pgfpathlineto{\pgfqpoint{6.418980in}{3.128132in}}%
\pgfpathlineto{\pgfqpoint{6.427811in}{3.116360in}}%
\pgfpathlineto{\pgfqpoint{6.431022in}{3.115724in}}%
\pgfpathlineto{\pgfqpoint{6.433966in}{3.117524in}}%
\pgfpathlineto{\pgfqpoint{6.438248in}{3.123213in}}%
\pgfpathlineto{\pgfqpoint{6.446276in}{3.133924in}}%
\pgfpathlineto{\pgfqpoint{6.449487in}{3.134738in}}%
\pgfpathlineto{\pgfqpoint{6.452431in}{3.133091in}}%
\pgfpathlineto{\pgfqpoint{6.456713in}{3.127533in}}%
\pgfpathlineto{\pgfqpoint{6.465008in}{3.116455in}}%
\pgfpathlineto{\pgfqpoint{6.468220in}{3.115692in}}%
\pgfpathlineto{\pgfqpoint{6.471163in}{3.117382in}}%
\pgfpathlineto{\pgfqpoint{6.475445in}{3.122979in}}%
\pgfpathlineto{\pgfqpoint{6.483741in}{3.134002in}}%
\pgfpathlineto{\pgfqpoint{6.486952in}{3.134715in}}%
\pgfpathlineto{\pgfqpoint{6.489896in}{3.132981in}}%
\pgfpathlineto{\pgfqpoint{6.494178in}{3.127347in}}%
\pgfpathlineto{\pgfqpoint{6.502206in}{3.116556in}}%
\pgfpathlineto{\pgfqpoint{6.505417in}{3.115666in}}%
\pgfpathlineto{\pgfqpoint{6.508361in}{3.117246in}}%
\pgfpathlineto{\pgfqpoint{6.512375in}{3.122331in}}%
\pgfpathlineto{\pgfqpoint{6.521206in}{3.134077in}}%
\pgfpathlineto{\pgfqpoint{6.524417in}{3.134688in}}%
\pgfpathlineto{\pgfqpoint{6.527361in}{3.132867in}}%
\pgfpathlineto{\pgfqpoint{6.531643in}{3.127160in}}%
\pgfpathlineto{\pgfqpoint{6.539671in}{3.116475in}}%
\pgfpathlineto{\pgfqpoint{6.542882in}{3.115686in}}%
\pgfpathlineto{\pgfqpoint{6.545826in}{3.117354in}}%
\pgfpathlineto{\pgfqpoint{6.550108in}{3.122931in}}%
\pgfpathlineto{\pgfqpoint{6.558403in}{3.133983in}}%
\pgfpathlineto{\pgfqpoint{6.561615in}{3.134721in}}%
\pgfpathlineto{\pgfqpoint{6.564558in}{3.133009in}}%
\pgfpathlineto{\pgfqpoint{6.568840in}{3.127395in}}%
\pgfpathlineto{\pgfqpoint{6.576868in}{3.116577in}}%
\pgfpathlineto{\pgfqpoint{6.580080in}{3.115661in}}%
\pgfpathlineto{\pgfqpoint{6.583023in}{3.117219in}}%
\pgfpathlineto{\pgfqpoint{6.587038in}{3.122285in}}%
\pgfpathlineto{\pgfqpoint{6.595869in}{3.134058in}}%
\pgfpathlineto{\pgfqpoint{6.599080in}{3.134696in}}%
\pgfpathlineto{\pgfqpoint{6.602024in}{3.132897in}}%
\pgfpathlineto{\pgfqpoint{6.606305in}{3.127208in}}%
\pgfpathlineto{\pgfqpoint{6.614334in}{3.116496in}}%
\pgfpathlineto{\pgfqpoint{6.617545in}{3.115680in}}%
\pgfpathlineto{\pgfqpoint{6.620489in}{3.117326in}}%
\pgfpathlineto{\pgfqpoint{6.624770in}{3.122883in}}%
\pgfpathlineto{\pgfqpoint{6.633066in}{3.133963in}}%
\pgfpathlineto{\pgfqpoint{6.636277in}{3.134727in}}%
\pgfpathlineto{\pgfqpoint{6.639221in}{3.133038in}}%
\pgfpathlineto{\pgfqpoint{6.643503in}{3.127443in}}%
\pgfpathlineto{\pgfqpoint{6.651799in}{3.116418in}}%
\pgfpathlineto{\pgfqpoint{6.655010in}{3.115703in}}%
\pgfpathlineto{\pgfqpoint{6.657954in}{3.117436in}}%
\pgfpathlineto{\pgfqpoint{6.662235in}{3.123069in}}%
\pgfpathlineto{\pgfqpoint{6.663306in}{3.124778in}}%
\pgfpathlineto{\pgfqpoint{6.663306in}{3.124778in}}%
\pgfusepath{stroke}%
\end{pgfscope}%
\begin{pgfscope}%
\pgfpathrectangle{\pgfqpoint{0.467797in}{2.292089in}}{\pgfqpoint{6.490533in}{1.666241in}}%
\pgfusepath{clip}%
\pgfsetrectcap%
\pgfsetroundjoin%
\pgfsetlinewidth{1.505625pt}%
\definecolor{currentstroke}{rgb}{0.121569,0.466667,0.705882}%
\pgfsetstrokecolor{currentstroke}%
\pgfsetdash{}{0pt}%
\pgfpathmoveto{\pgfqpoint{0.762821in}{3.125209in}}%
\pgfpathlineto{\pgfqpoint{0.769243in}{3.133576in}}%
\pgfpathlineto{\pgfqpoint{0.772455in}{3.134521in}}%
\pgfpathlineto{\pgfqpoint{0.775398in}{3.132933in}}%
\pgfpathlineto{\pgfqpoint{0.779413in}{3.127789in}}%
\pgfpathlineto{\pgfqpoint{0.787708in}{3.116669in}}%
\pgfpathlineto{\pgfqpoint{0.790920in}{3.115944in}}%
\pgfpathlineto{\pgfqpoint{0.793863in}{3.117723in}}%
\pgfpathlineto{\pgfqpoint{0.798145in}{3.123446in}}%
\pgfpathlineto{\pgfqpoint{0.805906in}{3.133740in}}%
\pgfpathlineto{\pgfqpoint{0.809117in}{3.134478in}}%
\pgfpathlineto{\pgfqpoint{0.812061in}{3.132710in}}%
\pgfpathlineto{\pgfqpoint{0.816342in}{3.126997in}}%
\pgfpathlineto{\pgfqpoint{0.824103in}{3.116689in}}%
\pgfpathlineto{\pgfqpoint{0.827314in}{3.115937in}}%
\pgfpathlineto{\pgfqpoint{0.830258in}{3.117694in}}%
\pgfpathlineto{\pgfqpoint{0.834540in}{3.123398in}}%
\pgfpathlineto{\pgfqpoint{0.842300in}{3.133720in}}%
\pgfpathlineto{\pgfqpoint{0.845512in}{3.134485in}}%
\pgfpathlineto{\pgfqpoint{0.848455in}{3.132739in}}%
\pgfpathlineto{\pgfqpoint{0.852737in}{3.127044in}}%
\pgfpathlineto{\pgfqpoint{0.860498in}{3.116709in}}%
\pgfpathlineto{\pgfqpoint{0.863709in}{3.115931in}}%
\pgfpathlineto{\pgfqpoint{0.866653in}{3.117666in}}%
\pgfpathlineto{\pgfqpoint{0.870934in}{3.123351in}}%
\pgfpathlineto{\pgfqpoint{0.878695in}{3.133700in}}%
\pgfpathlineto{\pgfqpoint{0.881906in}{3.134491in}}%
\pgfpathlineto{\pgfqpoint{0.884850in}{3.132767in}}%
\pgfpathlineto{\pgfqpoint{0.889132in}{3.127091in}}%
\pgfpathlineto{\pgfqpoint{0.896892in}{3.116729in}}%
\pgfpathlineto{\pgfqpoint{0.900104in}{3.115925in}}%
\pgfpathlineto{\pgfqpoint{0.903047in}{3.117638in}}%
\pgfpathlineto{\pgfqpoint{0.907329in}{3.123304in}}%
\pgfpathlineto{\pgfqpoint{0.915090in}{3.133680in}}%
\pgfpathlineto{\pgfqpoint{0.918301in}{3.134496in}}%
\pgfpathlineto{\pgfqpoint{0.921245in}{3.132795in}}%
\pgfpathlineto{\pgfqpoint{0.925526in}{3.127138in}}%
\pgfpathlineto{\pgfqpoint{0.933287in}{3.116749in}}%
\pgfpathlineto{\pgfqpoint{0.936498in}{3.115920in}}%
\pgfpathlineto{\pgfqpoint{0.939442in}{3.117610in}}%
\pgfpathlineto{\pgfqpoint{0.943724in}{3.123258in}}%
\pgfpathlineto{\pgfqpoint{0.951484in}{3.133660in}}%
\pgfpathlineto{\pgfqpoint{0.954696in}{3.134502in}}%
\pgfpathlineto{\pgfqpoint{0.957639in}{3.132823in}}%
\pgfpathlineto{\pgfqpoint{0.961921in}{3.127185in}}%
\pgfpathlineto{\pgfqpoint{0.969682in}{3.116769in}}%
\pgfpathlineto{\pgfqpoint{0.972893in}{3.115915in}}%
\pgfpathlineto{\pgfqpoint{0.975837in}{3.117582in}}%
\pgfpathlineto{\pgfqpoint{0.980118in}{3.123211in}}%
\pgfpathlineto{\pgfqpoint{0.987879in}{3.133639in}}%
\pgfpathlineto{\pgfqpoint{0.991090in}{3.134507in}}%
\pgfpathlineto{\pgfqpoint{0.994034in}{3.132851in}}%
\pgfpathlineto{\pgfqpoint{0.998316in}{3.127232in}}%
\pgfpathlineto{\pgfqpoint{1.006076in}{3.116790in}}%
\pgfpathlineto{\pgfqpoint{1.009288in}{3.115909in}}%
\pgfpathlineto{\pgfqpoint{1.012231in}{3.117554in}}%
\pgfpathlineto{\pgfqpoint{1.016513in}{3.123164in}}%
\pgfpathlineto{\pgfqpoint{1.024541in}{3.133798in}}%
\pgfpathlineto{\pgfqpoint{1.027753in}{3.134458in}}%
\pgfpathlineto{\pgfqpoint{1.030696in}{3.132623in}}%
\pgfpathlineto{\pgfqpoint{1.034978in}{3.126855in}}%
\pgfpathlineto{\pgfqpoint{1.042471in}{3.116811in}}%
\pgfpathlineto{\pgfqpoint{1.045682in}{3.115905in}}%
\pgfpathlineto{\pgfqpoint{1.048626in}{3.117527in}}%
\pgfpathlineto{\pgfqpoint{1.052908in}{3.123117in}}%
\pgfpathlineto{\pgfqpoint{1.060936in}{3.133779in}}%
\pgfpathlineto{\pgfqpoint{1.064147in}{3.134465in}}%
\pgfpathlineto{\pgfqpoint{1.067091in}{3.132652in}}%
\pgfpathlineto{\pgfqpoint{1.071373in}{3.126903in}}%
\pgfpathlineto{\pgfqpoint{1.078866in}{3.116832in}}%
\pgfpathlineto{\pgfqpoint{1.082077in}{3.115900in}}%
\pgfpathlineto{\pgfqpoint{1.085021in}{3.117500in}}%
\pgfpathlineto{\pgfqpoint{1.089035in}{3.122653in}}%
\pgfpathlineto{\pgfqpoint{1.097331in}{3.133760in}}%
\pgfpathlineto{\pgfqpoint{1.100542in}{3.134472in}}%
\pgfpathlineto{\pgfqpoint{1.103486in}{3.132681in}}%
\pgfpathlineto{\pgfqpoint{1.107767in}{3.126950in}}%
\pgfpathlineto{\pgfqpoint{1.115528in}{3.116669in}}%
\pgfpathlineto{\pgfqpoint{1.118739in}{3.115944in}}%
\pgfpathlineto{\pgfqpoint{1.121683in}{3.117723in}}%
\pgfpathlineto{\pgfqpoint{1.125965in}{3.123446in}}%
\pgfpathlineto{\pgfqpoint{1.133725in}{3.133740in}}%
\pgfpathlineto{\pgfqpoint{1.136937in}{3.134478in}}%
\pgfpathlineto{\pgfqpoint{1.139880in}{3.132710in}}%
\pgfpathlineto{\pgfqpoint{1.144162in}{3.126997in}}%
\pgfpathlineto{\pgfqpoint{1.151923in}{3.116689in}}%
\pgfpathlineto{\pgfqpoint{1.155134in}{3.115937in}}%
\pgfpathlineto{\pgfqpoint{1.158078in}{3.117694in}}%
\pgfpathlineto{\pgfqpoint{1.162359in}{3.123398in}}%
\pgfpathlineto{\pgfqpoint{1.170120in}{3.133720in}}%
\pgfpathlineto{\pgfqpoint{1.173331in}{3.134485in}}%
\pgfpathlineto{\pgfqpoint{1.176275in}{3.132739in}}%
\pgfpathlineto{\pgfqpoint{1.180557in}{3.127044in}}%
\pgfpathlineto{\pgfqpoint{1.188317in}{3.116709in}}%
\pgfpathlineto{\pgfqpoint{1.191529in}{3.115931in}}%
\pgfpathlineto{\pgfqpoint{1.194472in}{3.117666in}}%
\pgfpathlineto{\pgfqpoint{1.198754in}{3.123351in}}%
\pgfpathlineto{\pgfqpoint{1.206515in}{3.133700in}}%
\pgfpathlineto{\pgfqpoint{1.209726in}{3.134491in}}%
\pgfpathlineto{\pgfqpoint{1.212670in}{3.132767in}}%
\pgfpathlineto{\pgfqpoint{1.216951in}{3.127091in}}%
\pgfpathlineto{\pgfqpoint{1.224712in}{3.116729in}}%
\pgfpathlineto{\pgfqpoint{1.227923in}{3.115925in}}%
\pgfpathlineto{\pgfqpoint{1.230867in}{3.117638in}}%
\pgfpathlineto{\pgfqpoint{1.235149in}{3.123304in}}%
\pgfpathlineto{\pgfqpoint{1.242909in}{3.133680in}}%
\pgfpathlineto{\pgfqpoint{1.246121in}{3.134496in}}%
\pgfpathlineto{\pgfqpoint{1.249064in}{3.132795in}}%
\pgfpathlineto{\pgfqpoint{1.253346in}{3.127138in}}%
\pgfpathlineto{\pgfqpoint{1.261107in}{3.116749in}}%
\pgfpathlineto{\pgfqpoint{1.264318in}{3.115920in}}%
\pgfpathlineto{\pgfqpoint{1.267262in}{3.117610in}}%
\pgfpathlineto{\pgfqpoint{1.271543in}{3.123258in}}%
\pgfpathlineto{\pgfqpoint{1.279304in}{3.133660in}}%
\pgfpathlineto{\pgfqpoint{1.282515in}{3.134502in}}%
\pgfpathlineto{\pgfqpoint{1.285459in}{3.132823in}}%
\pgfpathlineto{\pgfqpoint{1.289741in}{3.127185in}}%
\pgfpathlineto{\pgfqpoint{1.297501in}{3.116769in}}%
\pgfpathlineto{\pgfqpoint{1.300713in}{3.115915in}}%
\pgfpathlineto{\pgfqpoint{1.303656in}{3.117582in}}%
\pgfpathlineto{\pgfqpoint{1.307938in}{3.123211in}}%
\pgfpathlineto{\pgfqpoint{1.315699in}{3.133639in}}%
\pgfpathlineto{\pgfqpoint{1.318910in}{3.134507in}}%
\pgfpathlineto{\pgfqpoint{1.321854in}{3.132851in}}%
\pgfpathlineto{\pgfqpoint{1.326135in}{3.127232in}}%
\pgfpathlineto{\pgfqpoint{1.333896in}{3.116790in}}%
\pgfpathlineto{\pgfqpoint{1.337107in}{3.115909in}}%
\pgfpathlineto{\pgfqpoint{1.340051in}{3.117554in}}%
\pgfpathlineto{\pgfqpoint{1.344333in}{3.123164in}}%
\pgfpathlineto{\pgfqpoint{1.352361in}{3.133798in}}%
\pgfpathlineto{\pgfqpoint{1.355572in}{3.134458in}}%
\pgfpathlineto{\pgfqpoint{1.358516in}{3.132623in}}%
\pgfpathlineto{\pgfqpoint{1.362798in}{3.126855in}}%
\pgfpathlineto{\pgfqpoint{1.370291in}{3.116811in}}%
\pgfpathlineto{\pgfqpoint{1.373502in}{3.115905in}}%
\pgfpathlineto{\pgfqpoint{1.376446in}{3.117527in}}%
\pgfpathlineto{\pgfqpoint{1.380727in}{3.123117in}}%
\pgfpathlineto{\pgfqpoint{1.388756in}{3.133779in}}%
\pgfpathlineto{\pgfqpoint{1.391967in}{3.134465in}}%
\pgfpathlineto{\pgfqpoint{1.394911in}{3.132652in}}%
\pgfpathlineto{\pgfqpoint{1.399192in}{3.126903in}}%
\pgfpathlineto{\pgfqpoint{1.406685in}{3.116832in}}%
\pgfpathlineto{\pgfqpoint{1.409897in}{3.115900in}}%
\pgfpathlineto{\pgfqpoint{1.412840in}{3.117500in}}%
\pgfpathlineto{\pgfqpoint{1.416854in}{3.122653in}}%
\pgfpathlineto{\pgfqpoint{1.425150in}{3.133760in}}%
\pgfpathlineto{\pgfqpoint{1.428362in}{3.134472in}}%
\pgfpathlineto{\pgfqpoint{1.431305in}{3.132681in}}%
\pgfpathlineto{\pgfqpoint{1.435587in}{3.126950in}}%
\pgfpathlineto{\pgfqpoint{1.443348in}{3.116669in}}%
\pgfpathlineto{\pgfqpoint{1.446559in}{3.115944in}}%
\pgfpathlineto{\pgfqpoint{1.449503in}{3.117723in}}%
\pgfpathlineto{\pgfqpoint{1.453784in}{3.123446in}}%
\pgfpathlineto{\pgfqpoint{1.461545in}{3.133740in}}%
\pgfpathlineto{\pgfqpoint{1.464756in}{3.134478in}}%
\pgfpathlineto{\pgfqpoint{1.467700in}{3.132710in}}%
\pgfpathlineto{\pgfqpoint{1.471982in}{3.126997in}}%
\pgfpathlineto{\pgfqpoint{1.479742in}{3.116689in}}%
\pgfpathlineto{\pgfqpoint{1.482954in}{3.115937in}}%
\pgfpathlineto{\pgfqpoint{1.485897in}{3.117694in}}%
\pgfpathlineto{\pgfqpoint{1.490179in}{3.123398in}}%
\pgfpathlineto{\pgfqpoint{1.497940in}{3.133720in}}%
\pgfpathlineto{\pgfqpoint{1.501151in}{3.134485in}}%
\pgfpathlineto{\pgfqpoint{1.504095in}{3.132739in}}%
\pgfpathlineto{\pgfqpoint{1.508376in}{3.127044in}}%
\pgfpathlineto{\pgfqpoint{1.516137in}{3.116709in}}%
\pgfpathlineto{\pgfqpoint{1.519348in}{3.115931in}}%
\pgfpathlineto{\pgfqpoint{1.522292in}{3.117666in}}%
\pgfpathlineto{\pgfqpoint{1.526574in}{3.123351in}}%
\pgfpathlineto{\pgfqpoint{1.534334in}{3.133700in}}%
\pgfpathlineto{\pgfqpoint{1.537546in}{3.134491in}}%
\pgfpathlineto{\pgfqpoint{1.540489in}{3.132767in}}%
\pgfpathlineto{\pgfqpoint{1.544771in}{3.127091in}}%
\pgfpathlineto{\pgfqpoint{1.552532in}{3.116729in}}%
\pgfpathlineto{\pgfqpoint{1.555743in}{3.115925in}}%
\pgfpathlineto{\pgfqpoint{1.558687in}{3.117638in}}%
\pgfpathlineto{\pgfqpoint{1.562968in}{3.123304in}}%
\pgfpathlineto{\pgfqpoint{1.570729in}{3.133680in}}%
\pgfpathlineto{\pgfqpoint{1.573940in}{3.134496in}}%
\pgfpathlineto{\pgfqpoint{1.576884in}{3.132795in}}%
\pgfpathlineto{\pgfqpoint{1.581166in}{3.127138in}}%
\pgfpathlineto{\pgfqpoint{1.588926in}{3.116749in}}%
\pgfpathlineto{\pgfqpoint{1.592138in}{3.115920in}}%
\pgfpathlineto{\pgfqpoint{1.595081in}{3.117610in}}%
\pgfpathlineto{\pgfqpoint{1.599363in}{3.123258in}}%
\pgfpathlineto{\pgfqpoint{1.607124in}{3.133660in}}%
\pgfpathlineto{\pgfqpoint{1.610335in}{3.134502in}}%
\pgfpathlineto{\pgfqpoint{1.613279in}{3.132823in}}%
\pgfpathlineto{\pgfqpoint{1.617560in}{3.127185in}}%
\pgfpathlineto{\pgfqpoint{1.625321in}{3.116769in}}%
\pgfpathlineto{\pgfqpoint{1.628532in}{3.115915in}}%
\pgfpathlineto{\pgfqpoint{1.631476in}{3.117582in}}%
\pgfpathlineto{\pgfqpoint{1.635758in}{3.123211in}}%
\pgfpathlineto{\pgfqpoint{1.643518in}{3.133639in}}%
\pgfpathlineto{\pgfqpoint{1.646730in}{3.134507in}}%
\pgfpathlineto{\pgfqpoint{1.649673in}{3.132851in}}%
\pgfpathlineto{\pgfqpoint{1.653955in}{3.127232in}}%
\pgfpathlineto{\pgfqpoint{1.661716in}{3.116790in}}%
\pgfpathlineto{\pgfqpoint{1.664927in}{3.115909in}}%
\pgfpathlineto{\pgfqpoint{1.667871in}{3.117554in}}%
\pgfpathlineto{\pgfqpoint{1.672152in}{3.123164in}}%
\pgfpathlineto{\pgfqpoint{1.680181in}{3.133798in}}%
\pgfpathlineto{\pgfqpoint{1.683392in}{3.134458in}}%
\pgfpathlineto{\pgfqpoint{1.686335in}{3.132623in}}%
\pgfpathlineto{\pgfqpoint{1.690617in}{3.126855in}}%
\pgfpathlineto{\pgfqpoint{1.698110in}{3.116811in}}%
\pgfpathlineto{\pgfqpoint{1.701322in}{3.115905in}}%
\pgfpathlineto{\pgfqpoint{1.704265in}{3.117527in}}%
\pgfpathlineto{\pgfqpoint{1.708547in}{3.123117in}}%
\pgfpathlineto{\pgfqpoint{1.716575in}{3.133779in}}%
\pgfpathlineto{\pgfqpoint{1.719786in}{3.134465in}}%
\pgfpathlineto{\pgfqpoint{1.722730in}{3.132652in}}%
\pgfpathlineto{\pgfqpoint{1.727012in}{3.126903in}}%
\pgfpathlineto{\pgfqpoint{1.734505in}{3.116832in}}%
\pgfpathlineto{\pgfqpoint{1.737716in}{3.115900in}}%
\pgfpathlineto{\pgfqpoint{1.740660in}{3.117500in}}%
\pgfpathlineto{\pgfqpoint{1.744674in}{3.122653in}}%
\pgfpathlineto{\pgfqpoint{1.752970in}{3.133760in}}%
\pgfpathlineto{\pgfqpoint{1.756181in}{3.134472in}}%
\pgfpathlineto{\pgfqpoint{1.759125in}{3.132681in}}%
\pgfpathlineto{\pgfqpoint{1.763407in}{3.126950in}}%
\pgfpathlineto{\pgfqpoint{1.771167in}{3.116669in}}%
\pgfpathlineto{\pgfqpoint{1.774378in}{3.115944in}}%
\pgfpathlineto{\pgfqpoint{1.777322in}{3.117723in}}%
\pgfpathlineto{\pgfqpoint{1.781604in}{3.123446in}}%
\pgfpathlineto{\pgfqpoint{1.789365in}{3.133740in}}%
\pgfpathlineto{\pgfqpoint{1.792576in}{3.134478in}}%
\pgfpathlineto{\pgfqpoint{1.795519in}{3.132710in}}%
\pgfpathlineto{\pgfqpoint{1.799801in}{3.126997in}}%
\pgfpathlineto{\pgfqpoint{1.807562in}{3.116689in}}%
\pgfpathlineto{\pgfqpoint{1.810773in}{3.115937in}}%
\pgfpathlineto{\pgfqpoint{1.813717in}{3.117694in}}%
\pgfpathlineto{\pgfqpoint{1.817999in}{3.123398in}}%
\pgfpathlineto{\pgfqpoint{1.825759in}{3.133720in}}%
\pgfpathlineto{\pgfqpoint{1.828970in}{3.134485in}}%
\pgfpathlineto{\pgfqpoint{1.831914in}{3.132739in}}%
\pgfpathlineto{\pgfqpoint{1.836196in}{3.127044in}}%
\pgfpathlineto{\pgfqpoint{1.843957in}{3.116709in}}%
\pgfpathlineto{\pgfqpoint{1.847168in}{3.115931in}}%
\pgfpathlineto{\pgfqpoint{1.850111in}{3.117666in}}%
\pgfpathlineto{\pgfqpoint{1.854393in}{3.123351in}}%
\pgfpathlineto{\pgfqpoint{1.862154in}{3.133700in}}%
\pgfpathlineto{\pgfqpoint{1.865365in}{3.134491in}}%
\pgfpathlineto{\pgfqpoint{1.868309in}{3.132767in}}%
\pgfpathlineto{\pgfqpoint{1.872591in}{3.127091in}}%
\pgfpathlineto{\pgfqpoint{1.880351in}{3.116729in}}%
\pgfpathlineto{\pgfqpoint{1.883562in}{3.115925in}}%
\pgfpathlineto{\pgfqpoint{1.886506in}{3.117638in}}%
\pgfpathlineto{\pgfqpoint{1.890788in}{3.123304in}}%
\pgfpathlineto{\pgfqpoint{1.898549in}{3.133680in}}%
\pgfpathlineto{\pgfqpoint{1.901760in}{3.134496in}}%
\pgfpathlineto{\pgfqpoint{1.904703in}{3.132795in}}%
\pgfpathlineto{\pgfqpoint{1.908985in}{3.127138in}}%
\pgfpathlineto{\pgfqpoint{1.916746in}{3.116749in}}%
\pgfpathlineto{\pgfqpoint{1.919957in}{3.115920in}}%
\pgfpathlineto{\pgfqpoint{1.922901in}{3.117610in}}%
\pgfpathlineto{\pgfqpoint{1.927183in}{3.123258in}}%
\pgfpathlineto{\pgfqpoint{1.934943in}{3.133660in}}%
\pgfpathlineto{\pgfqpoint{1.938154in}{3.134502in}}%
\pgfpathlineto{\pgfqpoint{1.941098in}{3.132823in}}%
\pgfpathlineto{\pgfqpoint{1.945380in}{3.127185in}}%
\pgfpathlineto{\pgfqpoint{1.953140in}{3.116769in}}%
\pgfpathlineto{\pgfqpoint{1.956352in}{3.115915in}}%
\pgfpathlineto{\pgfqpoint{1.959295in}{3.117582in}}%
\pgfpathlineto{\pgfqpoint{1.963577in}{3.123211in}}%
\pgfpathlineto{\pgfqpoint{1.971338in}{3.133639in}}%
\pgfpathlineto{\pgfqpoint{1.974549in}{3.134507in}}%
\pgfpathlineto{\pgfqpoint{1.977493in}{3.132851in}}%
\pgfpathlineto{\pgfqpoint{1.981775in}{3.127232in}}%
\pgfpathlineto{\pgfqpoint{1.989535in}{3.116790in}}%
\pgfpathlineto{\pgfqpoint{1.992746in}{3.115909in}}%
\pgfpathlineto{\pgfqpoint{1.995690in}{3.117554in}}%
\pgfpathlineto{\pgfqpoint{1.999972in}{3.123164in}}%
\pgfpathlineto{\pgfqpoint{2.008000in}{3.133798in}}%
\pgfpathlineto{\pgfqpoint{2.011211in}{3.134458in}}%
\pgfpathlineto{\pgfqpoint{2.014155in}{3.132623in}}%
\pgfpathlineto{\pgfqpoint{2.018437in}{3.126855in}}%
\pgfpathlineto{\pgfqpoint{2.025930in}{3.116811in}}%
\pgfpathlineto{\pgfqpoint{2.029141in}{3.115905in}}%
\pgfpathlineto{\pgfqpoint{2.032085in}{3.117527in}}%
\pgfpathlineto{\pgfqpoint{2.036367in}{3.123117in}}%
\pgfpathlineto{\pgfqpoint{2.044395in}{3.133779in}}%
\pgfpathlineto{\pgfqpoint{2.047606in}{3.134465in}}%
\pgfpathlineto{\pgfqpoint{2.050550in}{3.132652in}}%
\pgfpathlineto{\pgfqpoint{2.054831in}{3.126903in}}%
\pgfpathlineto{\pgfqpoint{2.062324in}{3.116832in}}%
\pgfpathlineto{\pgfqpoint{2.065536in}{3.115900in}}%
\pgfpathlineto{\pgfqpoint{2.068479in}{3.117500in}}%
\pgfpathlineto{\pgfqpoint{2.072494in}{3.122653in}}%
\pgfpathlineto{\pgfqpoint{2.080789in}{3.133760in}}%
\pgfpathlineto{\pgfqpoint{2.084001in}{3.134472in}}%
\pgfpathlineto{\pgfqpoint{2.086944in}{3.132681in}}%
\pgfpathlineto{\pgfqpoint{2.091226in}{3.126950in}}%
\pgfpathlineto{\pgfqpoint{2.098987in}{3.116669in}}%
\pgfpathlineto{\pgfqpoint{2.102198in}{3.115944in}}%
\pgfpathlineto{\pgfqpoint{2.105142in}{3.117723in}}%
\pgfpathlineto{\pgfqpoint{2.109423in}{3.123446in}}%
\pgfpathlineto{\pgfqpoint{2.117184in}{3.133740in}}%
\pgfpathlineto{\pgfqpoint{2.120395in}{3.134478in}}%
\pgfpathlineto{\pgfqpoint{2.123339in}{3.132710in}}%
\pgfpathlineto{\pgfqpoint{2.127621in}{3.126997in}}%
\pgfpathlineto{\pgfqpoint{2.135381in}{3.116689in}}%
\pgfpathlineto{\pgfqpoint{2.138593in}{3.115937in}}%
\pgfpathlineto{\pgfqpoint{2.141536in}{3.117694in}}%
\pgfpathlineto{\pgfqpoint{2.145818in}{3.123398in}}%
\pgfpathlineto{\pgfqpoint{2.153579in}{3.133720in}}%
\pgfpathlineto{\pgfqpoint{2.156790in}{3.134485in}}%
\pgfpathlineto{\pgfqpoint{2.159734in}{3.132739in}}%
\pgfpathlineto{\pgfqpoint{2.164015in}{3.127044in}}%
\pgfpathlineto{\pgfqpoint{2.171776in}{3.116709in}}%
\pgfpathlineto{\pgfqpoint{2.174987in}{3.115931in}}%
\pgfpathlineto{\pgfqpoint{2.177931in}{3.117666in}}%
\pgfpathlineto{\pgfqpoint{2.182213in}{3.123351in}}%
\pgfpathlineto{\pgfqpoint{2.189973in}{3.133700in}}%
\pgfpathlineto{\pgfqpoint{2.193185in}{3.134491in}}%
\pgfpathlineto{\pgfqpoint{2.196128in}{3.132767in}}%
\pgfpathlineto{\pgfqpoint{2.200410in}{3.127091in}}%
\pgfpathlineto{\pgfqpoint{2.208171in}{3.116729in}}%
\pgfpathlineto{\pgfqpoint{2.211382in}{3.115925in}}%
\pgfpathlineto{\pgfqpoint{2.214326in}{3.117638in}}%
\pgfpathlineto{\pgfqpoint{2.218607in}{3.123304in}}%
\pgfpathlineto{\pgfqpoint{2.226368in}{3.133680in}}%
\pgfpathlineto{\pgfqpoint{2.229579in}{3.134496in}}%
\pgfpathlineto{\pgfqpoint{2.232523in}{3.132795in}}%
\pgfpathlineto{\pgfqpoint{2.236805in}{3.127138in}}%
\pgfpathlineto{\pgfqpoint{2.244565in}{3.116749in}}%
\pgfpathlineto{\pgfqpoint{2.247777in}{3.115920in}}%
\pgfpathlineto{\pgfqpoint{2.250720in}{3.117610in}}%
\pgfpathlineto{\pgfqpoint{2.255002in}{3.123258in}}%
\pgfpathlineto{\pgfqpoint{2.262763in}{3.133660in}}%
\pgfpathlineto{\pgfqpoint{2.265974in}{3.134502in}}%
\pgfpathlineto{\pgfqpoint{2.268918in}{3.132823in}}%
\pgfpathlineto{\pgfqpoint{2.273199in}{3.127185in}}%
\pgfpathlineto{\pgfqpoint{2.280960in}{3.116769in}}%
\pgfpathlineto{\pgfqpoint{2.284171in}{3.115915in}}%
\pgfpathlineto{\pgfqpoint{2.287115in}{3.117582in}}%
\pgfpathlineto{\pgfqpoint{2.291397in}{3.123211in}}%
\pgfpathlineto{\pgfqpoint{2.299157in}{3.133639in}}%
\pgfpathlineto{\pgfqpoint{2.302369in}{3.134507in}}%
\pgfpathlineto{\pgfqpoint{2.305312in}{3.132851in}}%
\pgfpathlineto{\pgfqpoint{2.309594in}{3.127232in}}%
\pgfpathlineto{\pgfqpoint{2.317355in}{3.116790in}}%
\pgfpathlineto{\pgfqpoint{2.320566in}{3.115909in}}%
\pgfpathlineto{\pgfqpoint{2.323510in}{3.117554in}}%
\pgfpathlineto{\pgfqpoint{2.327791in}{3.123164in}}%
\pgfpathlineto{\pgfqpoint{2.335820in}{3.133798in}}%
\pgfpathlineto{\pgfqpoint{2.339031in}{3.134458in}}%
\pgfpathlineto{\pgfqpoint{2.341975in}{3.132623in}}%
\pgfpathlineto{\pgfqpoint{2.346256in}{3.126855in}}%
\pgfpathlineto{\pgfqpoint{2.353749in}{3.116811in}}%
\pgfpathlineto{\pgfqpoint{2.356961in}{3.115905in}}%
\pgfpathlineto{\pgfqpoint{2.359904in}{3.117527in}}%
\pgfpathlineto{\pgfqpoint{2.364186in}{3.123117in}}%
\pgfpathlineto{\pgfqpoint{2.372214in}{3.133779in}}%
\pgfpathlineto{\pgfqpoint{2.375426in}{3.134465in}}%
\pgfpathlineto{\pgfqpoint{2.378369in}{3.132652in}}%
\pgfpathlineto{\pgfqpoint{2.382651in}{3.126903in}}%
\pgfpathlineto{\pgfqpoint{2.390144in}{3.116832in}}%
\pgfpathlineto{\pgfqpoint{2.393355in}{3.115900in}}%
\pgfpathlineto{\pgfqpoint{2.396299in}{3.117500in}}%
\pgfpathlineto{\pgfqpoint{2.400313in}{3.122653in}}%
\pgfpathlineto{\pgfqpoint{2.408609in}{3.133760in}}%
\pgfpathlineto{\pgfqpoint{2.411820in}{3.134472in}}%
\pgfpathlineto{\pgfqpoint{2.414764in}{3.132681in}}%
\pgfpathlineto{\pgfqpoint{2.419046in}{3.126950in}}%
\pgfpathlineto{\pgfqpoint{2.426806in}{3.116669in}}%
\pgfpathlineto{\pgfqpoint{2.430018in}{3.115944in}}%
\pgfpathlineto{\pgfqpoint{2.432961in}{3.117723in}}%
\pgfpathlineto{\pgfqpoint{2.437243in}{3.123446in}}%
\pgfpathlineto{\pgfqpoint{2.445004in}{3.133740in}}%
\pgfpathlineto{\pgfqpoint{2.448215in}{3.134478in}}%
\pgfpathlineto{\pgfqpoint{2.451159in}{3.132710in}}%
\pgfpathlineto{\pgfqpoint{2.455440in}{3.126997in}}%
\pgfpathlineto{\pgfqpoint{2.463201in}{3.116689in}}%
\pgfpathlineto{\pgfqpoint{2.466412in}{3.115937in}}%
\pgfpathlineto{\pgfqpoint{2.469356in}{3.117694in}}%
\pgfpathlineto{\pgfqpoint{2.473638in}{3.123398in}}%
\pgfpathlineto{\pgfqpoint{2.481398in}{3.133720in}}%
\pgfpathlineto{\pgfqpoint{2.484610in}{3.134485in}}%
\pgfpathlineto{\pgfqpoint{2.487553in}{3.132739in}}%
\pgfpathlineto{\pgfqpoint{2.491835in}{3.127044in}}%
\pgfpathlineto{\pgfqpoint{2.499596in}{3.116709in}}%
\pgfpathlineto{\pgfqpoint{2.502807in}{3.115931in}}%
\pgfpathlineto{\pgfqpoint{2.505751in}{3.117666in}}%
\pgfpathlineto{\pgfqpoint{2.510032in}{3.123351in}}%
\pgfpathlineto{\pgfqpoint{2.517793in}{3.133700in}}%
\pgfpathlineto{\pgfqpoint{2.521004in}{3.134491in}}%
\pgfpathlineto{\pgfqpoint{2.523948in}{3.132767in}}%
\pgfpathlineto{\pgfqpoint{2.528230in}{3.127091in}}%
\pgfpathlineto{\pgfqpoint{2.535990in}{3.116729in}}%
\pgfpathlineto{\pgfqpoint{2.539202in}{3.115925in}}%
\pgfpathlineto{\pgfqpoint{2.542145in}{3.117638in}}%
\pgfpathlineto{\pgfqpoint{2.546427in}{3.123304in}}%
\pgfpathlineto{\pgfqpoint{2.554188in}{3.133680in}}%
\pgfpathlineto{\pgfqpoint{2.557399in}{3.134496in}}%
\pgfpathlineto{\pgfqpoint{2.560343in}{3.132795in}}%
\pgfpathlineto{\pgfqpoint{2.564624in}{3.127138in}}%
\pgfpathlineto{\pgfqpoint{2.572385in}{3.116749in}}%
\pgfpathlineto{\pgfqpoint{2.575596in}{3.115920in}}%
\pgfpathlineto{\pgfqpoint{2.578540in}{3.117610in}}%
\pgfpathlineto{\pgfqpoint{2.582822in}{3.123258in}}%
\pgfpathlineto{\pgfqpoint{2.590582in}{3.133660in}}%
\pgfpathlineto{\pgfqpoint{2.593794in}{3.134502in}}%
\pgfpathlineto{\pgfqpoint{2.596737in}{3.132823in}}%
\pgfpathlineto{\pgfqpoint{2.601019in}{3.127185in}}%
\pgfpathlineto{\pgfqpoint{2.608780in}{3.116769in}}%
\pgfpathlineto{\pgfqpoint{2.611991in}{3.115915in}}%
\pgfpathlineto{\pgfqpoint{2.614935in}{3.117582in}}%
\pgfpathlineto{\pgfqpoint{2.619216in}{3.123211in}}%
\pgfpathlineto{\pgfqpoint{2.626977in}{3.133639in}}%
\pgfpathlineto{\pgfqpoint{2.630188in}{3.134507in}}%
\pgfpathlineto{\pgfqpoint{2.633132in}{3.132851in}}%
\pgfpathlineto{\pgfqpoint{2.637414in}{3.127232in}}%
\pgfpathlineto{\pgfqpoint{2.645174in}{3.116790in}}%
\pgfpathlineto{\pgfqpoint{2.648386in}{3.115909in}}%
\pgfpathlineto{\pgfqpoint{2.651329in}{3.117554in}}%
\pgfpathlineto{\pgfqpoint{2.655611in}{3.123164in}}%
\pgfpathlineto{\pgfqpoint{2.663639in}{3.133798in}}%
\pgfpathlineto{\pgfqpoint{2.666851in}{3.134458in}}%
\pgfpathlineto{\pgfqpoint{2.669794in}{3.132623in}}%
\pgfpathlineto{\pgfqpoint{2.674076in}{3.126855in}}%
\pgfpathlineto{\pgfqpoint{2.681569in}{3.116811in}}%
\pgfpathlineto{\pgfqpoint{2.684780in}{3.115905in}}%
\pgfpathlineto{\pgfqpoint{2.687724in}{3.117527in}}%
\pgfpathlineto{\pgfqpoint{2.692006in}{3.123117in}}%
\pgfpathlineto{\pgfqpoint{2.700034in}{3.133779in}}%
\pgfpathlineto{\pgfqpoint{2.703245in}{3.134465in}}%
\pgfpathlineto{\pgfqpoint{2.706189in}{3.132652in}}%
\pgfpathlineto{\pgfqpoint{2.710471in}{3.126903in}}%
\pgfpathlineto{\pgfqpoint{2.717964in}{3.116832in}}%
\pgfpathlineto{\pgfqpoint{2.721175in}{3.115900in}}%
\pgfpathlineto{\pgfqpoint{2.724119in}{3.117500in}}%
\pgfpathlineto{\pgfqpoint{2.728133in}{3.122653in}}%
\pgfpathlineto{\pgfqpoint{2.736429in}{3.133760in}}%
\pgfpathlineto{\pgfqpoint{2.739640in}{3.134472in}}%
\pgfpathlineto{\pgfqpoint{2.742584in}{3.132681in}}%
\pgfpathlineto{\pgfqpoint{2.746865in}{3.126950in}}%
\pgfpathlineto{\pgfqpoint{2.754626in}{3.116669in}}%
\pgfpathlineto{\pgfqpoint{2.757837in}{3.115944in}}%
\pgfpathlineto{\pgfqpoint{2.760781in}{3.117723in}}%
\pgfpathlineto{\pgfqpoint{2.765063in}{3.123446in}}%
\pgfpathlineto{\pgfqpoint{2.772823in}{3.133740in}}%
\pgfpathlineto{\pgfqpoint{2.776035in}{3.134478in}}%
\pgfpathlineto{\pgfqpoint{2.778978in}{3.132710in}}%
\pgfpathlineto{\pgfqpoint{2.783260in}{3.126997in}}%
\pgfpathlineto{\pgfqpoint{2.791021in}{3.116689in}}%
\pgfpathlineto{\pgfqpoint{2.794232in}{3.115937in}}%
\pgfpathlineto{\pgfqpoint{2.797176in}{3.117694in}}%
\pgfpathlineto{\pgfqpoint{2.801457in}{3.123398in}}%
\pgfpathlineto{\pgfqpoint{2.809218in}{3.133720in}}%
\pgfpathlineto{\pgfqpoint{2.812429in}{3.134485in}}%
\pgfpathlineto{\pgfqpoint{2.815373in}{3.132739in}}%
\pgfpathlineto{\pgfqpoint{2.819655in}{3.127044in}}%
\pgfpathlineto{\pgfqpoint{2.827415in}{3.116709in}}%
\pgfpathlineto{\pgfqpoint{2.830627in}{3.115931in}}%
\pgfpathlineto{\pgfqpoint{2.833570in}{3.117666in}}%
\pgfpathlineto{\pgfqpoint{2.837852in}{3.123351in}}%
\pgfpathlineto{\pgfqpoint{2.845613in}{3.133700in}}%
\pgfpathlineto{\pgfqpoint{2.848824in}{3.134491in}}%
\pgfpathlineto{\pgfqpoint{2.851768in}{3.132767in}}%
\pgfpathlineto{\pgfqpoint{2.856049in}{3.127091in}}%
\pgfpathlineto{\pgfqpoint{2.863810in}{3.116729in}}%
\pgfpathlineto{\pgfqpoint{2.867021in}{3.115925in}}%
\pgfpathlineto{\pgfqpoint{2.869965in}{3.117638in}}%
\pgfpathlineto{\pgfqpoint{2.874247in}{3.123304in}}%
\pgfpathlineto{\pgfqpoint{2.882007in}{3.133680in}}%
\pgfpathlineto{\pgfqpoint{2.885219in}{3.134496in}}%
\pgfpathlineto{\pgfqpoint{2.888162in}{3.132795in}}%
\pgfpathlineto{\pgfqpoint{2.892444in}{3.127138in}}%
\pgfpathlineto{\pgfqpoint{2.900205in}{3.116749in}}%
\pgfpathlineto{\pgfqpoint{2.903416in}{3.115920in}}%
\pgfpathlineto{\pgfqpoint{2.906360in}{3.117610in}}%
\pgfpathlineto{\pgfqpoint{2.910641in}{3.123258in}}%
\pgfpathlineto{\pgfqpoint{2.918402in}{3.133660in}}%
\pgfpathlineto{\pgfqpoint{2.921613in}{3.134502in}}%
\pgfpathlineto{\pgfqpoint{2.924557in}{3.132823in}}%
\pgfpathlineto{\pgfqpoint{2.928839in}{3.127185in}}%
\pgfpathlineto{\pgfqpoint{2.936599in}{3.116769in}}%
\pgfpathlineto{\pgfqpoint{2.939811in}{3.115915in}}%
\pgfpathlineto{\pgfqpoint{2.942754in}{3.117582in}}%
\pgfpathlineto{\pgfqpoint{2.947036in}{3.123211in}}%
\pgfpathlineto{\pgfqpoint{2.954797in}{3.133639in}}%
\pgfpathlineto{\pgfqpoint{2.958008in}{3.134507in}}%
\pgfpathlineto{\pgfqpoint{2.960952in}{3.132851in}}%
\pgfpathlineto{\pgfqpoint{2.965233in}{3.127232in}}%
\pgfpathlineto{\pgfqpoint{2.972994in}{3.116790in}}%
\pgfpathlineto{\pgfqpoint{2.976205in}{3.115909in}}%
\pgfpathlineto{\pgfqpoint{2.979149in}{3.117554in}}%
\pgfpathlineto{\pgfqpoint{2.983431in}{3.123164in}}%
\pgfpathlineto{\pgfqpoint{2.991459in}{3.133798in}}%
\pgfpathlineto{\pgfqpoint{2.994670in}{3.134458in}}%
\pgfpathlineto{\pgfqpoint{2.997614in}{3.132623in}}%
\pgfpathlineto{\pgfqpoint{3.001896in}{3.126855in}}%
\pgfpathlineto{\pgfqpoint{3.009389in}{3.116811in}}%
\pgfpathlineto{\pgfqpoint{3.012600in}{3.115905in}}%
\pgfpathlineto{\pgfqpoint{3.015544in}{3.117527in}}%
\pgfpathlineto{\pgfqpoint{3.019825in}{3.123117in}}%
\pgfpathlineto{\pgfqpoint{3.027854in}{3.133779in}}%
\pgfpathlineto{\pgfqpoint{3.031065in}{3.134465in}}%
\pgfpathlineto{\pgfqpoint{3.034008in}{3.132652in}}%
\pgfpathlineto{\pgfqpoint{3.038290in}{3.126903in}}%
\pgfpathlineto{\pgfqpoint{3.045783in}{3.116832in}}%
\pgfpathlineto{\pgfqpoint{3.048995in}{3.115900in}}%
\pgfpathlineto{\pgfqpoint{3.051938in}{3.117500in}}%
\pgfpathlineto{\pgfqpoint{3.055952in}{3.122653in}}%
\pgfpathlineto{\pgfqpoint{3.064248in}{3.133760in}}%
\pgfpathlineto{\pgfqpoint{3.067459in}{3.134472in}}%
\pgfpathlineto{\pgfqpoint{3.070403in}{3.132681in}}%
\pgfpathlineto{\pgfqpoint{3.074685in}{3.126950in}}%
\pgfpathlineto{\pgfqpoint{3.082446in}{3.116669in}}%
\pgfpathlineto{\pgfqpoint{3.085657in}{3.115944in}}%
\pgfpathlineto{\pgfqpoint{3.088600in}{3.117723in}}%
\pgfpathlineto{\pgfqpoint{3.092882in}{3.123446in}}%
\pgfpathlineto{\pgfqpoint{3.100643in}{3.133740in}}%
\pgfpathlineto{\pgfqpoint{3.103854in}{3.134478in}}%
\pgfpathlineto{\pgfqpoint{3.106798in}{3.132710in}}%
\pgfpathlineto{\pgfqpoint{3.111080in}{3.126997in}}%
\pgfpathlineto{\pgfqpoint{3.118840in}{3.116689in}}%
\pgfpathlineto{\pgfqpoint{3.122051in}{3.115937in}}%
\pgfpathlineto{\pgfqpoint{3.124995in}{3.117694in}}%
\pgfpathlineto{\pgfqpoint{3.129277in}{3.123398in}}%
\pgfpathlineto{\pgfqpoint{3.137037in}{3.133720in}}%
\pgfpathlineto{\pgfqpoint{3.140249in}{3.134485in}}%
\pgfpathlineto{\pgfqpoint{3.143192in}{3.132739in}}%
\pgfpathlineto{\pgfqpoint{3.147474in}{3.127044in}}%
\pgfpathlineto{\pgfqpoint{3.155235in}{3.116709in}}%
\pgfpathlineto{\pgfqpoint{3.158446in}{3.115931in}}%
\pgfpathlineto{\pgfqpoint{3.161390in}{3.117666in}}%
\pgfpathlineto{\pgfqpoint{3.165672in}{3.123351in}}%
\pgfpathlineto{\pgfqpoint{3.173432in}{3.133700in}}%
\pgfpathlineto{\pgfqpoint{3.176643in}{3.134491in}}%
\pgfpathlineto{\pgfqpoint{3.179587in}{3.132767in}}%
\pgfpathlineto{\pgfqpoint{3.183869in}{3.127091in}}%
\pgfpathlineto{\pgfqpoint{3.191629in}{3.116729in}}%
\pgfpathlineto{\pgfqpoint{3.194841in}{3.115925in}}%
\pgfpathlineto{\pgfqpoint{3.197784in}{3.117638in}}%
\pgfpathlineto{\pgfqpoint{3.202066in}{3.123304in}}%
\pgfpathlineto{\pgfqpoint{3.209827in}{3.133680in}}%
\pgfpathlineto{\pgfqpoint{3.213038in}{3.134496in}}%
\pgfpathlineto{\pgfqpoint{3.215982in}{3.132795in}}%
\pgfpathlineto{\pgfqpoint{3.220264in}{3.127138in}}%
\pgfpathlineto{\pgfqpoint{3.228024in}{3.116749in}}%
\pgfpathlineto{\pgfqpoint{3.231235in}{3.115920in}}%
\pgfpathlineto{\pgfqpoint{3.234179in}{3.117610in}}%
\pgfpathlineto{\pgfqpoint{3.238461in}{3.123258in}}%
\pgfpathlineto{\pgfqpoint{3.246221in}{3.133660in}}%
\pgfpathlineto{\pgfqpoint{3.249433in}{3.134502in}}%
\pgfpathlineto{\pgfqpoint{3.252376in}{3.132823in}}%
\pgfpathlineto{\pgfqpoint{3.256658in}{3.127185in}}%
\pgfpathlineto{\pgfqpoint{3.264419in}{3.116769in}}%
\pgfpathlineto{\pgfqpoint{3.267630in}{3.115915in}}%
\pgfpathlineto{\pgfqpoint{3.270574in}{3.117582in}}%
\pgfpathlineto{\pgfqpoint{3.274856in}{3.123211in}}%
\pgfpathlineto{\pgfqpoint{3.282616in}{3.133639in}}%
\pgfpathlineto{\pgfqpoint{3.285827in}{3.134507in}}%
\pgfpathlineto{\pgfqpoint{3.288771in}{3.132851in}}%
\pgfpathlineto{\pgfqpoint{3.293053in}{3.127232in}}%
\pgfpathlineto{\pgfqpoint{3.300813in}{3.116790in}}%
\pgfpathlineto{\pgfqpoint{3.304025in}{3.115909in}}%
\pgfpathlineto{\pgfqpoint{3.306968in}{3.117554in}}%
\pgfpathlineto{\pgfqpoint{3.311250in}{3.123164in}}%
\pgfpathlineto{\pgfqpoint{3.319278in}{3.133798in}}%
\pgfpathlineto{\pgfqpoint{3.322490in}{3.134458in}}%
\pgfpathlineto{\pgfqpoint{3.325433in}{3.132623in}}%
\pgfpathlineto{\pgfqpoint{3.329715in}{3.126855in}}%
\pgfpathlineto{\pgfqpoint{3.337208in}{3.116811in}}%
\pgfpathlineto{\pgfqpoint{3.340419in}{3.115905in}}%
\pgfpathlineto{\pgfqpoint{3.343363in}{3.117527in}}%
\pgfpathlineto{\pgfqpoint{3.347645in}{3.123117in}}%
\pgfpathlineto{\pgfqpoint{3.355673in}{3.133779in}}%
\pgfpathlineto{\pgfqpoint{3.358884in}{3.134465in}}%
\pgfpathlineto{\pgfqpoint{3.361828in}{3.132652in}}%
\pgfpathlineto{\pgfqpoint{3.366110in}{3.126903in}}%
\pgfpathlineto{\pgfqpoint{3.373603in}{3.116832in}}%
\pgfpathlineto{\pgfqpoint{3.376814in}{3.115900in}}%
\pgfpathlineto{\pgfqpoint{3.379758in}{3.117500in}}%
\pgfpathlineto{\pgfqpoint{3.383772in}{3.122653in}}%
\pgfpathlineto{\pgfqpoint{3.392068in}{3.133760in}}%
\pgfpathlineto{\pgfqpoint{3.395279in}{3.134472in}}%
\pgfpathlineto{\pgfqpoint{3.398223in}{3.132681in}}%
\pgfpathlineto{\pgfqpoint{3.402504in}{3.126950in}}%
\pgfpathlineto{\pgfqpoint{3.410265in}{3.116669in}}%
\pgfpathlineto{\pgfqpoint{3.413476in}{3.115944in}}%
\pgfpathlineto{\pgfqpoint{3.416420in}{3.117723in}}%
\pgfpathlineto{\pgfqpoint{3.420702in}{3.123446in}}%
\pgfpathlineto{\pgfqpoint{3.428462in}{3.133740in}}%
\pgfpathlineto{\pgfqpoint{3.431674in}{3.134478in}}%
\pgfpathlineto{\pgfqpoint{3.434617in}{3.132710in}}%
\pgfpathlineto{\pgfqpoint{3.438899in}{3.126997in}}%
\pgfpathlineto{\pgfqpoint{3.446660in}{3.116689in}}%
\pgfpathlineto{\pgfqpoint{3.449871in}{3.115937in}}%
\pgfpathlineto{\pgfqpoint{3.452815in}{3.117694in}}%
\pgfpathlineto{\pgfqpoint{3.457096in}{3.123398in}}%
\pgfpathlineto{\pgfqpoint{3.464857in}{3.133720in}}%
\pgfpathlineto{\pgfqpoint{3.468068in}{3.134485in}}%
\pgfpathlineto{\pgfqpoint{3.471012in}{3.132739in}}%
\pgfpathlineto{\pgfqpoint{3.475294in}{3.127044in}}%
\pgfpathlineto{\pgfqpoint{3.483054in}{3.116709in}}%
\pgfpathlineto{\pgfqpoint{3.486266in}{3.115931in}}%
\pgfpathlineto{\pgfqpoint{3.489209in}{3.117666in}}%
\pgfpathlineto{\pgfqpoint{3.493491in}{3.123351in}}%
\pgfpathlineto{\pgfqpoint{3.501252in}{3.133700in}}%
\pgfpathlineto{\pgfqpoint{3.504463in}{3.134491in}}%
\pgfpathlineto{\pgfqpoint{3.507407in}{3.132767in}}%
\pgfpathlineto{\pgfqpoint{3.511688in}{3.127091in}}%
\pgfpathlineto{\pgfqpoint{3.519449in}{3.116729in}}%
\pgfpathlineto{\pgfqpoint{3.522660in}{3.115925in}}%
\pgfpathlineto{\pgfqpoint{3.525604in}{3.117638in}}%
\pgfpathlineto{\pgfqpoint{3.529886in}{3.123304in}}%
\pgfpathlineto{\pgfqpoint{3.537646in}{3.133680in}}%
\pgfpathlineto{\pgfqpoint{3.540858in}{3.134496in}}%
\pgfpathlineto{\pgfqpoint{3.543801in}{3.132795in}}%
\pgfpathlineto{\pgfqpoint{3.548083in}{3.127138in}}%
\pgfpathlineto{\pgfqpoint{3.555844in}{3.116749in}}%
\pgfpathlineto{\pgfqpoint{3.559055in}{3.115920in}}%
\pgfpathlineto{\pgfqpoint{3.561999in}{3.117610in}}%
\pgfpathlineto{\pgfqpoint{3.566280in}{3.123258in}}%
\pgfpathlineto{\pgfqpoint{3.574041in}{3.133660in}}%
\pgfpathlineto{\pgfqpoint{3.577252in}{3.134502in}}%
\pgfpathlineto{\pgfqpoint{3.580196in}{3.132823in}}%
\pgfpathlineto{\pgfqpoint{3.584478in}{3.127185in}}%
\pgfpathlineto{\pgfqpoint{3.592238in}{3.116769in}}%
\pgfpathlineto{\pgfqpoint{3.595450in}{3.115915in}}%
\pgfpathlineto{\pgfqpoint{3.598393in}{3.117582in}}%
\pgfpathlineto{\pgfqpoint{3.602675in}{3.123211in}}%
\pgfpathlineto{\pgfqpoint{3.610436in}{3.133639in}}%
\pgfpathlineto{\pgfqpoint{3.613647in}{3.134507in}}%
\pgfpathlineto{\pgfqpoint{3.616591in}{3.132851in}}%
\pgfpathlineto{\pgfqpoint{3.620872in}{3.127232in}}%
\pgfpathlineto{\pgfqpoint{3.628633in}{3.116790in}}%
\pgfpathlineto{\pgfqpoint{3.631844in}{3.115909in}}%
\pgfpathlineto{\pgfqpoint{3.634788in}{3.117554in}}%
\pgfpathlineto{\pgfqpoint{3.639070in}{3.123164in}}%
\pgfpathlineto{\pgfqpoint{3.647098in}{3.133798in}}%
\pgfpathlineto{\pgfqpoint{3.650309in}{3.134458in}}%
\pgfpathlineto{\pgfqpoint{3.653253in}{3.132623in}}%
\pgfpathlineto{\pgfqpoint{3.657535in}{3.126855in}}%
\pgfpathlineto{\pgfqpoint{3.665028in}{3.116811in}}%
\pgfpathlineto{\pgfqpoint{3.668239in}{3.115905in}}%
\pgfpathlineto{\pgfqpoint{3.671183in}{3.117527in}}%
\pgfpathlineto{\pgfqpoint{3.675464in}{3.123117in}}%
\pgfpathlineto{\pgfqpoint{3.683493in}{3.133779in}}%
\pgfpathlineto{\pgfqpoint{3.686704in}{3.134465in}}%
\pgfpathlineto{\pgfqpoint{3.689648in}{3.132652in}}%
\pgfpathlineto{\pgfqpoint{3.693929in}{3.126903in}}%
\pgfpathlineto{\pgfqpoint{3.701422in}{3.116832in}}%
\pgfpathlineto{\pgfqpoint{3.704634in}{3.115900in}}%
\pgfpathlineto{\pgfqpoint{3.707577in}{3.117500in}}%
\pgfpathlineto{\pgfqpoint{3.711591in}{3.122653in}}%
\pgfpathlineto{\pgfqpoint{3.719887in}{3.133760in}}%
\pgfpathlineto{\pgfqpoint{3.723099in}{3.134472in}}%
\pgfpathlineto{\pgfqpoint{3.726042in}{3.132681in}}%
\pgfpathlineto{\pgfqpoint{3.730324in}{3.126950in}}%
\pgfpathlineto{\pgfqpoint{3.738085in}{3.116669in}}%
\pgfpathlineto{\pgfqpoint{3.741296in}{3.115944in}}%
\pgfpathlineto{\pgfqpoint{3.744240in}{3.117723in}}%
\pgfpathlineto{\pgfqpoint{3.748521in}{3.123446in}}%
\pgfpathlineto{\pgfqpoint{3.756282in}{3.133740in}}%
\pgfpathlineto{\pgfqpoint{3.759493in}{3.134478in}}%
\pgfpathlineto{\pgfqpoint{3.762437in}{3.132710in}}%
\pgfpathlineto{\pgfqpoint{3.766719in}{3.126997in}}%
\pgfpathlineto{\pgfqpoint{3.774479in}{3.116689in}}%
\pgfpathlineto{\pgfqpoint{3.777691in}{3.115937in}}%
\pgfpathlineto{\pgfqpoint{3.780634in}{3.117694in}}%
\pgfpathlineto{\pgfqpoint{3.784916in}{3.123398in}}%
\pgfpathlineto{\pgfqpoint{3.792677in}{3.133720in}}%
\pgfpathlineto{\pgfqpoint{3.795888in}{3.134485in}}%
\pgfpathlineto{\pgfqpoint{3.798832in}{3.132739in}}%
\pgfpathlineto{\pgfqpoint{3.803113in}{3.127044in}}%
\pgfpathlineto{\pgfqpoint{3.810874in}{3.116709in}}%
\pgfpathlineto{\pgfqpoint{3.814085in}{3.115931in}}%
\pgfpathlineto{\pgfqpoint{3.817029in}{3.117666in}}%
\pgfpathlineto{\pgfqpoint{3.821311in}{3.123351in}}%
\pgfpathlineto{\pgfqpoint{3.829071in}{3.133700in}}%
\pgfpathlineto{\pgfqpoint{3.832283in}{3.134491in}}%
\pgfpathlineto{\pgfqpoint{3.835226in}{3.132767in}}%
\pgfpathlineto{\pgfqpoint{3.839508in}{3.127091in}}%
\pgfpathlineto{\pgfqpoint{3.847269in}{3.116729in}}%
\pgfpathlineto{\pgfqpoint{3.850480in}{3.115925in}}%
\pgfpathlineto{\pgfqpoint{3.853424in}{3.117638in}}%
\pgfpathlineto{\pgfqpoint{3.857705in}{3.123304in}}%
\pgfpathlineto{\pgfqpoint{3.865466in}{3.133680in}}%
\pgfpathlineto{\pgfqpoint{3.868677in}{3.134496in}}%
\pgfpathlineto{\pgfqpoint{3.871621in}{3.132795in}}%
\pgfpathlineto{\pgfqpoint{3.875903in}{3.127138in}}%
\pgfpathlineto{\pgfqpoint{3.883663in}{3.116749in}}%
\pgfpathlineto{\pgfqpoint{3.886875in}{3.115920in}}%
\pgfpathlineto{\pgfqpoint{3.889818in}{3.117610in}}%
\pgfpathlineto{\pgfqpoint{3.894100in}{3.123258in}}%
\pgfpathlineto{\pgfqpoint{3.901861in}{3.133660in}}%
\pgfpathlineto{\pgfqpoint{3.905072in}{3.134502in}}%
\pgfpathlineto{\pgfqpoint{3.908016in}{3.132823in}}%
\pgfpathlineto{\pgfqpoint{3.912297in}{3.127185in}}%
\pgfpathlineto{\pgfqpoint{3.920058in}{3.116769in}}%
\pgfpathlineto{\pgfqpoint{3.923269in}{3.115915in}}%
\pgfpathlineto{\pgfqpoint{3.926213in}{3.117582in}}%
\pgfpathlineto{\pgfqpoint{3.930495in}{3.123211in}}%
\pgfpathlineto{\pgfqpoint{3.938255in}{3.133639in}}%
\pgfpathlineto{\pgfqpoint{3.941467in}{3.134507in}}%
\pgfpathlineto{\pgfqpoint{3.944410in}{3.132851in}}%
\pgfpathlineto{\pgfqpoint{3.948692in}{3.127232in}}%
\pgfpathlineto{\pgfqpoint{3.956453in}{3.116790in}}%
\pgfpathlineto{\pgfqpoint{3.959664in}{3.115909in}}%
\pgfpathlineto{\pgfqpoint{3.962608in}{3.117554in}}%
\pgfpathlineto{\pgfqpoint{3.966889in}{3.123164in}}%
\pgfpathlineto{\pgfqpoint{3.974918in}{3.133798in}}%
\pgfpathlineto{\pgfqpoint{3.978129in}{3.134458in}}%
\pgfpathlineto{\pgfqpoint{3.981073in}{3.132623in}}%
\pgfpathlineto{\pgfqpoint{3.985354in}{3.126855in}}%
\pgfpathlineto{\pgfqpoint{3.992847in}{3.116811in}}%
\pgfpathlineto{\pgfqpoint{3.996059in}{3.115905in}}%
\pgfpathlineto{\pgfqpoint{3.999002in}{3.117527in}}%
\pgfpathlineto{\pgfqpoint{4.003284in}{3.123117in}}%
\pgfpathlineto{\pgfqpoint{4.011312in}{3.133779in}}%
\pgfpathlineto{\pgfqpoint{4.014524in}{3.134465in}}%
\pgfpathlineto{\pgfqpoint{4.017467in}{3.132652in}}%
\pgfpathlineto{\pgfqpoint{4.021749in}{3.126903in}}%
\pgfpathlineto{\pgfqpoint{4.029242in}{3.116832in}}%
\pgfpathlineto{\pgfqpoint{4.032453in}{3.115900in}}%
\pgfpathlineto{\pgfqpoint{4.035397in}{3.117500in}}%
\pgfpathlineto{\pgfqpoint{4.039411in}{3.122653in}}%
\pgfpathlineto{\pgfqpoint{4.047707in}{3.133760in}}%
\pgfpathlineto{\pgfqpoint{4.050918in}{3.134472in}}%
\pgfpathlineto{\pgfqpoint{4.053862in}{3.132681in}}%
\pgfpathlineto{\pgfqpoint{4.058144in}{3.126950in}}%
\pgfpathlineto{\pgfqpoint{4.065904in}{3.116669in}}%
\pgfpathlineto{\pgfqpoint{4.069116in}{3.115944in}}%
\pgfpathlineto{\pgfqpoint{4.072059in}{3.117723in}}%
\pgfpathlineto{\pgfqpoint{4.076341in}{3.123446in}}%
\pgfpathlineto{\pgfqpoint{4.084102in}{3.133740in}}%
\pgfpathlineto{\pgfqpoint{4.087313in}{3.134478in}}%
\pgfpathlineto{\pgfqpoint{4.090257in}{3.132710in}}%
\pgfpathlineto{\pgfqpoint{4.094538in}{3.126997in}}%
\pgfpathlineto{\pgfqpoint{4.102299in}{3.116689in}}%
\pgfpathlineto{\pgfqpoint{4.105510in}{3.115937in}}%
\pgfpathlineto{\pgfqpoint{4.108454in}{3.117694in}}%
\pgfpathlineto{\pgfqpoint{4.112736in}{3.123398in}}%
\pgfpathlineto{\pgfqpoint{4.120496in}{3.133720in}}%
\pgfpathlineto{\pgfqpoint{4.123708in}{3.134485in}}%
\pgfpathlineto{\pgfqpoint{4.126651in}{3.132739in}}%
\pgfpathlineto{\pgfqpoint{4.130933in}{3.127044in}}%
\pgfpathlineto{\pgfqpoint{4.138694in}{3.116709in}}%
\pgfpathlineto{\pgfqpoint{4.141905in}{3.115931in}}%
\pgfpathlineto{\pgfqpoint{4.144849in}{3.117666in}}%
\pgfpathlineto{\pgfqpoint{4.149130in}{3.123351in}}%
\pgfpathlineto{\pgfqpoint{4.156891in}{3.133700in}}%
\pgfpathlineto{\pgfqpoint{4.160102in}{3.134491in}}%
\pgfpathlineto{\pgfqpoint{4.163046in}{3.132767in}}%
\pgfpathlineto{\pgfqpoint{4.167328in}{3.127091in}}%
\pgfpathlineto{\pgfqpoint{4.175088in}{3.116729in}}%
\pgfpathlineto{\pgfqpoint{4.178300in}{3.115925in}}%
\pgfpathlineto{\pgfqpoint{4.181243in}{3.117638in}}%
\pgfpathlineto{\pgfqpoint{4.185525in}{3.123304in}}%
\pgfpathlineto{\pgfqpoint{4.193286in}{3.133680in}}%
\pgfpathlineto{\pgfqpoint{4.196497in}{3.134496in}}%
\pgfpathlineto{\pgfqpoint{4.199441in}{3.132795in}}%
\pgfpathlineto{\pgfqpoint{4.203722in}{3.127138in}}%
\pgfpathlineto{\pgfqpoint{4.211483in}{3.116749in}}%
\pgfpathlineto{\pgfqpoint{4.214694in}{3.115920in}}%
\pgfpathlineto{\pgfqpoint{4.217638in}{3.117610in}}%
\pgfpathlineto{\pgfqpoint{4.221920in}{3.123258in}}%
\pgfpathlineto{\pgfqpoint{4.229680in}{3.133660in}}%
\pgfpathlineto{\pgfqpoint{4.232892in}{3.134502in}}%
\pgfpathlineto{\pgfqpoint{4.235835in}{3.132823in}}%
\pgfpathlineto{\pgfqpoint{4.240117in}{3.127185in}}%
\pgfpathlineto{\pgfqpoint{4.247878in}{3.116769in}}%
\pgfpathlineto{\pgfqpoint{4.251089in}{3.115915in}}%
\pgfpathlineto{\pgfqpoint{4.254033in}{3.117582in}}%
\pgfpathlineto{\pgfqpoint{4.258314in}{3.123211in}}%
\pgfpathlineto{\pgfqpoint{4.266075in}{3.133639in}}%
\pgfpathlineto{\pgfqpoint{4.269286in}{3.134507in}}%
\pgfpathlineto{\pgfqpoint{4.272230in}{3.132851in}}%
\pgfpathlineto{\pgfqpoint{4.276512in}{3.127232in}}%
\pgfpathlineto{\pgfqpoint{4.284272in}{3.116790in}}%
\pgfpathlineto{\pgfqpoint{4.287484in}{3.115909in}}%
\pgfpathlineto{\pgfqpoint{4.290427in}{3.117554in}}%
\pgfpathlineto{\pgfqpoint{4.294709in}{3.123164in}}%
\pgfpathlineto{\pgfqpoint{4.302737in}{3.133798in}}%
\pgfpathlineto{\pgfqpoint{4.305948in}{3.134458in}}%
\pgfpathlineto{\pgfqpoint{4.308892in}{3.132623in}}%
\pgfpathlineto{\pgfqpoint{4.313174in}{3.126855in}}%
\pgfpathlineto{\pgfqpoint{4.320667in}{3.116811in}}%
\pgfpathlineto{\pgfqpoint{4.323878in}{3.115905in}}%
\pgfpathlineto{\pgfqpoint{4.326822in}{3.117527in}}%
\pgfpathlineto{\pgfqpoint{4.331104in}{3.123117in}}%
\pgfpathlineto{\pgfqpoint{4.339132in}{3.133779in}}%
\pgfpathlineto{\pgfqpoint{4.342343in}{3.134465in}}%
\pgfpathlineto{\pgfqpoint{4.345287in}{3.132652in}}%
\pgfpathlineto{\pgfqpoint{4.349569in}{3.126903in}}%
\pgfpathlineto{\pgfqpoint{4.357062in}{3.116832in}}%
\pgfpathlineto{\pgfqpoint{4.360273in}{3.115900in}}%
\pgfpathlineto{\pgfqpoint{4.363217in}{3.117500in}}%
\pgfpathlineto{\pgfqpoint{4.367231in}{3.122653in}}%
\pgfpathlineto{\pgfqpoint{4.375526in}{3.133760in}}%
\pgfpathlineto{\pgfqpoint{4.378738in}{3.134472in}}%
\pgfpathlineto{\pgfqpoint{4.381681in}{3.132681in}}%
\pgfpathlineto{\pgfqpoint{4.385963in}{3.126950in}}%
\pgfpathlineto{\pgfqpoint{4.393724in}{3.116669in}}%
\pgfpathlineto{\pgfqpoint{4.396935in}{3.115944in}}%
\pgfpathlineto{\pgfqpoint{4.399879in}{3.117723in}}%
\pgfpathlineto{\pgfqpoint{4.404161in}{3.123446in}}%
\pgfpathlineto{\pgfqpoint{4.411921in}{3.133740in}}%
\pgfpathlineto{\pgfqpoint{4.415132in}{3.134478in}}%
\pgfpathlineto{\pgfqpoint{4.418076in}{3.132710in}}%
\pgfpathlineto{\pgfqpoint{4.422358in}{3.126997in}}%
\pgfpathlineto{\pgfqpoint{4.430118in}{3.116689in}}%
\pgfpathlineto{\pgfqpoint{4.433330in}{3.115937in}}%
\pgfpathlineto{\pgfqpoint{4.436273in}{3.117694in}}%
\pgfpathlineto{\pgfqpoint{4.440555in}{3.123398in}}%
\pgfpathlineto{\pgfqpoint{4.448316in}{3.133720in}}%
\pgfpathlineto{\pgfqpoint{4.451527in}{3.134485in}}%
\pgfpathlineto{\pgfqpoint{4.454471in}{3.132739in}}%
\pgfpathlineto{\pgfqpoint{4.458753in}{3.127044in}}%
\pgfpathlineto{\pgfqpoint{4.466513in}{3.116709in}}%
\pgfpathlineto{\pgfqpoint{4.469724in}{3.115931in}}%
\pgfpathlineto{\pgfqpoint{4.472668in}{3.117666in}}%
\pgfpathlineto{\pgfqpoint{4.476950in}{3.123351in}}%
\pgfpathlineto{\pgfqpoint{4.484710in}{3.133700in}}%
\pgfpathlineto{\pgfqpoint{4.487922in}{3.134491in}}%
\pgfpathlineto{\pgfqpoint{4.490865in}{3.132767in}}%
\pgfpathlineto{\pgfqpoint{4.495147in}{3.127091in}}%
\pgfpathlineto{\pgfqpoint{4.502908in}{3.116729in}}%
\pgfpathlineto{\pgfqpoint{4.506119in}{3.115925in}}%
\pgfpathlineto{\pgfqpoint{4.509063in}{3.117638in}}%
\pgfpathlineto{\pgfqpoint{4.513345in}{3.123304in}}%
\pgfpathlineto{\pgfqpoint{4.521105in}{3.133680in}}%
\pgfpathlineto{\pgfqpoint{4.524316in}{3.134496in}}%
\pgfpathlineto{\pgfqpoint{4.527260in}{3.132795in}}%
\pgfpathlineto{\pgfqpoint{4.531542in}{3.127138in}}%
\pgfpathlineto{\pgfqpoint{4.539302in}{3.116749in}}%
\pgfpathlineto{\pgfqpoint{4.542514in}{3.115920in}}%
\pgfpathlineto{\pgfqpoint{4.545457in}{3.117610in}}%
\pgfpathlineto{\pgfqpoint{4.549739in}{3.123258in}}%
\pgfpathlineto{\pgfqpoint{4.557500in}{3.133660in}}%
\pgfpathlineto{\pgfqpoint{4.560711in}{3.134502in}}%
\pgfpathlineto{\pgfqpoint{4.563655in}{3.132823in}}%
\pgfpathlineto{\pgfqpoint{4.567937in}{3.127185in}}%
\pgfpathlineto{\pgfqpoint{4.575697in}{3.116769in}}%
\pgfpathlineto{\pgfqpoint{4.578908in}{3.115915in}}%
\pgfpathlineto{\pgfqpoint{4.581852in}{3.117582in}}%
\pgfpathlineto{\pgfqpoint{4.586134in}{3.123211in}}%
\pgfpathlineto{\pgfqpoint{4.593894in}{3.133639in}}%
\pgfpathlineto{\pgfqpoint{4.597106in}{3.134507in}}%
\pgfpathlineto{\pgfqpoint{4.600049in}{3.132851in}}%
\pgfpathlineto{\pgfqpoint{4.604331in}{3.127232in}}%
\pgfpathlineto{\pgfqpoint{4.612092in}{3.116790in}}%
\pgfpathlineto{\pgfqpoint{4.615303in}{3.115909in}}%
\pgfpathlineto{\pgfqpoint{4.618247in}{3.117554in}}%
\pgfpathlineto{\pgfqpoint{4.622529in}{3.123164in}}%
\pgfpathlineto{\pgfqpoint{4.630557in}{3.133798in}}%
\pgfpathlineto{\pgfqpoint{4.633768in}{3.134458in}}%
\pgfpathlineto{\pgfqpoint{4.636712in}{3.132623in}}%
\pgfpathlineto{\pgfqpoint{4.640993in}{3.126855in}}%
\pgfpathlineto{\pgfqpoint{4.648486in}{3.116811in}}%
\pgfpathlineto{\pgfqpoint{4.651698in}{3.115905in}}%
\pgfpathlineto{\pgfqpoint{4.654641in}{3.117527in}}%
\pgfpathlineto{\pgfqpoint{4.658923in}{3.123117in}}%
\pgfpathlineto{\pgfqpoint{4.666951in}{3.133779in}}%
\pgfpathlineto{\pgfqpoint{4.670163in}{3.134465in}}%
\pgfpathlineto{\pgfqpoint{4.673106in}{3.132652in}}%
\pgfpathlineto{\pgfqpoint{4.677388in}{3.126903in}}%
\pgfpathlineto{\pgfqpoint{4.684881in}{3.116832in}}%
\pgfpathlineto{\pgfqpoint{4.688092in}{3.115900in}}%
\pgfpathlineto{\pgfqpoint{4.691036in}{3.117500in}}%
\pgfpathlineto{\pgfqpoint{4.695050in}{3.122653in}}%
\pgfpathlineto{\pgfqpoint{4.703346in}{3.133760in}}%
\pgfpathlineto{\pgfqpoint{4.706557in}{3.134472in}}%
\pgfpathlineto{\pgfqpoint{4.709501in}{3.132681in}}%
\pgfpathlineto{\pgfqpoint{4.713783in}{3.126950in}}%
\pgfpathlineto{\pgfqpoint{4.721543in}{3.116669in}}%
\pgfpathlineto{\pgfqpoint{4.724755in}{3.115944in}}%
\pgfpathlineto{\pgfqpoint{4.727698in}{3.117723in}}%
\pgfpathlineto{\pgfqpoint{4.731980in}{3.123446in}}%
\pgfpathlineto{\pgfqpoint{4.739741in}{3.133740in}}%
\pgfpathlineto{\pgfqpoint{4.742952in}{3.134478in}}%
\pgfpathlineto{\pgfqpoint{4.745896in}{3.132710in}}%
\pgfpathlineto{\pgfqpoint{4.750177in}{3.126997in}}%
\pgfpathlineto{\pgfqpoint{4.757938in}{3.116689in}}%
\pgfpathlineto{\pgfqpoint{4.761149in}{3.115937in}}%
\pgfpathlineto{\pgfqpoint{4.764093in}{3.117694in}}%
\pgfpathlineto{\pgfqpoint{4.768375in}{3.123398in}}%
\pgfpathlineto{\pgfqpoint{4.776135in}{3.133720in}}%
\pgfpathlineto{\pgfqpoint{4.779347in}{3.134485in}}%
\pgfpathlineto{\pgfqpoint{4.782290in}{3.132739in}}%
\pgfpathlineto{\pgfqpoint{4.786572in}{3.127044in}}%
\pgfpathlineto{\pgfqpoint{4.794333in}{3.116709in}}%
\pgfpathlineto{\pgfqpoint{4.797544in}{3.115931in}}%
\pgfpathlineto{\pgfqpoint{4.800488in}{3.117666in}}%
\pgfpathlineto{\pgfqpoint{4.804769in}{3.123351in}}%
\pgfpathlineto{\pgfqpoint{4.812530in}{3.133700in}}%
\pgfpathlineto{\pgfqpoint{4.815741in}{3.134491in}}%
\pgfpathlineto{\pgfqpoint{4.818685in}{3.132767in}}%
\pgfpathlineto{\pgfqpoint{4.822967in}{3.127091in}}%
\pgfpathlineto{\pgfqpoint{4.830727in}{3.116729in}}%
\pgfpathlineto{\pgfqpoint{4.833939in}{3.115925in}}%
\pgfpathlineto{\pgfqpoint{4.836882in}{3.117638in}}%
\pgfpathlineto{\pgfqpoint{4.841164in}{3.123304in}}%
\pgfpathlineto{\pgfqpoint{4.848925in}{3.133680in}}%
\pgfpathlineto{\pgfqpoint{4.852136in}{3.134496in}}%
\pgfpathlineto{\pgfqpoint{4.855080in}{3.132795in}}%
\pgfpathlineto{\pgfqpoint{4.859361in}{3.127138in}}%
\pgfpathlineto{\pgfqpoint{4.867122in}{3.116749in}}%
\pgfpathlineto{\pgfqpoint{4.870333in}{3.115920in}}%
\pgfpathlineto{\pgfqpoint{4.873277in}{3.117610in}}%
\pgfpathlineto{\pgfqpoint{4.877559in}{3.123258in}}%
\pgfpathlineto{\pgfqpoint{4.885319in}{3.133660in}}%
\pgfpathlineto{\pgfqpoint{4.888531in}{3.134502in}}%
\pgfpathlineto{\pgfqpoint{4.891474in}{3.132823in}}%
\pgfpathlineto{\pgfqpoint{4.895756in}{3.127185in}}%
\pgfpathlineto{\pgfqpoint{4.903517in}{3.116769in}}%
\pgfpathlineto{\pgfqpoint{4.906728in}{3.115915in}}%
\pgfpathlineto{\pgfqpoint{4.909672in}{3.117582in}}%
\pgfpathlineto{\pgfqpoint{4.913953in}{3.123211in}}%
\pgfpathlineto{\pgfqpoint{4.921714in}{3.133639in}}%
\pgfpathlineto{\pgfqpoint{4.924925in}{3.134507in}}%
\pgfpathlineto{\pgfqpoint{4.927869in}{3.132851in}}%
\pgfpathlineto{\pgfqpoint{4.932151in}{3.127232in}}%
\pgfpathlineto{\pgfqpoint{4.939911in}{3.116790in}}%
\pgfpathlineto{\pgfqpoint{4.943123in}{3.115909in}}%
\pgfpathlineto{\pgfqpoint{4.946066in}{3.117554in}}%
\pgfpathlineto{\pgfqpoint{4.950348in}{3.123164in}}%
\pgfpathlineto{\pgfqpoint{4.958376in}{3.133798in}}%
\pgfpathlineto{\pgfqpoint{4.961588in}{3.134458in}}%
\pgfpathlineto{\pgfqpoint{4.964531in}{3.132623in}}%
\pgfpathlineto{\pgfqpoint{4.968813in}{3.126855in}}%
\pgfpathlineto{\pgfqpoint{4.976306in}{3.116811in}}%
\pgfpathlineto{\pgfqpoint{4.979517in}{3.115905in}}%
\pgfpathlineto{\pgfqpoint{4.982461in}{3.117527in}}%
\pgfpathlineto{\pgfqpoint{4.986743in}{3.123117in}}%
\pgfpathlineto{\pgfqpoint{4.994771in}{3.133779in}}%
\pgfpathlineto{\pgfqpoint{4.997982in}{3.134465in}}%
\pgfpathlineto{\pgfqpoint{5.000926in}{3.132652in}}%
\pgfpathlineto{\pgfqpoint{5.005208in}{3.126903in}}%
\pgfpathlineto{\pgfqpoint{5.012701in}{3.116832in}}%
\pgfpathlineto{\pgfqpoint{5.015912in}{3.115900in}}%
\pgfpathlineto{\pgfqpoint{5.018856in}{3.117500in}}%
\pgfpathlineto{\pgfqpoint{5.022870in}{3.122653in}}%
\pgfpathlineto{\pgfqpoint{5.031166in}{3.133760in}}%
\pgfpathlineto{\pgfqpoint{5.034377in}{3.134472in}}%
\pgfpathlineto{\pgfqpoint{5.037321in}{3.132681in}}%
\pgfpathlineto{\pgfqpoint{5.041602in}{3.126950in}}%
\pgfpathlineto{\pgfqpoint{5.049363in}{3.116669in}}%
\pgfpathlineto{\pgfqpoint{5.052574in}{3.115944in}}%
\pgfpathlineto{\pgfqpoint{5.055518in}{3.117723in}}%
\pgfpathlineto{\pgfqpoint{5.059800in}{3.123446in}}%
\pgfpathlineto{\pgfqpoint{5.067560in}{3.133740in}}%
\pgfpathlineto{\pgfqpoint{5.070772in}{3.134478in}}%
\pgfpathlineto{\pgfqpoint{5.073715in}{3.132710in}}%
\pgfpathlineto{\pgfqpoint{5.077997in}{3.126997in}}%
\pgfpathlineto{\pgfqpoint{5.085758in}{3.116689in}}%
\pgfpathlineto{\pgfqpoint{5.088969in}{3.115937in}}%
\pgfpathlineto{\pgfqpoint{5.091913in}{3.117694in}}%
\pgfpathlineto{\pgfqpoint{5.096194in}{3.123398in}}%
\pgfpathlineto{\pgfqpoint{5.103955in}{3.133720in}}%
\pgfpathlineto{\pgfqpoint{5.107166in}{3.134485in}}%
\pgfpathlineto{\pgfqpoint{5.110110in}{3.132739in}}%
\pgfpathlineto{\pgfqpoint{5.114392in}{3.127044in}}%
\pgfpathlineto{\pgfqpoint{5.122152in}{3.116709in}}%
\pgfpathlineto{\pgfqpoint{5.125364in}{3.115931in}}%
\pgfpathlineto{\pgfqpoint{5.128307in}{3.117666in}}%
\pgfpathlineto{\pgfqpoint{5.132589in}{3.123351in}}%
\pgfpathlineto{\pgfqpoint{5.140350in}{3.133700in}}%
\pgfpathlineto{\pgfqpoint{5.143561in}{3.134491in}}%
\pgfpathlineto{\pgfqpoint{5.146505in}{3.132767in}}%
\pgfpathlineto{\pgfqpoint{5.150786in}{3.127091in}}%
\pgfpathlineto{\pgfqpoint{5.158547in}{3.116729in}}%
\pgfpathlineto{\pgfqpoint{5.161758in}{3.115925in}}%
\pgfpathlineto{\pgfqpoint{5.164702in}{3.117638in}}%
\pgfpathlineto{\pgfqpoint{5.168984in}{3.123304in}}%
\pgfpathlineto{\pgfqpoint{5.176744in}{3.133680in}}%
\pgfpathlineto{\pgfqpoint{5.179956in}{3.134496in}}%
\pgfpathlineto{\pgfqpoint{5.182899in}{3.132795in}}%
\pgfpathlineto{\pgfqpoint{5.187181in}{3.127138in}}%
\pgfpathlineto{\pgfqpoint{5.194942in}{3.116749in}}%
\pgfpathlineto{\pgfqpoint{5.198153in}{3.115920in}}%
\pgfpathlineto{\pgfqpoint{5.201097in}{3.117610in}}%
\pgfpathlineto{\pgfqpoint{5.205378in}{3.123258in}}%
\pgfpathlineto{\pgfqpoint{5.213139in}{3.133660in}}%
\pgfpathlineto{\pgfqpoint{5.216350in}{3.134502in}}%
\pgfpathlineto{\pgfqpoint{5.219294in}{3.132823in}}%
\pgfpathlineto{\pgfqpoint{5.223576in}{3.127185in}}%
\pgfpathlineto{\pgfqpoint{5.231336in}{3.116769in}}%
\pgfpathlineto{\pgfqpoint{5.234548in}{3.115915in}}%
\pgfpathlineto{\pgfqpoint{5.237491in}{3.117582in}}%
\pgfpathlineto{\pgfqpoint{5.241773in}{3.123211in}}%
\pgfpathlineto{\pgfqpoint{5.249534in}{3.133639in}}%
\pgfpathlineto{\pgfqpoint{5.252745in}{3.134507in}}%
\pgfpathlineto{\pgfqpoint{5.255689in}{3.132851in}}%
\pgfpathlineto{\pgfqpoint{5.259970in}{3.127232in}}%
\pgfpathlineto{\pgfqpoint{5.267731in}{3.116790in}}%
\pgfpathlineto{\pgfqpoint{5.270942in}{3.115909in}}%
\pgfpathlineto{\pgfqpoint{5.273886in}{3.117554in}}%
\pgfpathlineto{\pgfqpoint{5.278168in}{3.123164in}}%
\pgfpathlineto{\pgfqpoint{5.286196in}{3.133798in}}%
\pgfpathlineto{\pgfqpoint{5.289407in}{3.134458in}}%
\pgfpathlineto{\pgfqpoint{5.292351in}{3.132623in}}%
\pgfpathlineto{\pgfqpoint{5.296633in}{3.126855in}}%
\pgfpathlineto{\pgfqpoint{5.304126in}{3.116811in}}%
\pgfpathlineto{\pgfqpoint{5.307337in}{3.115905in}}%
\pgfpathlineto{\pgfqpoint{5.310281in}{3.117527in}}%
\pgfpathlineto{\pgfqpoint{5.314562in}{3.123117in}}%
\pgfpathlineto{\pgfqpoint{5.322591in}{3.133779in}}%
\pgfpathlineto{\pgfqpoint{5.325802in}{3.134465in}}%
\pgfpathlineto{\pgfqpoint{5.328746in}{3.132652in}}%
\pgfpathlineto{\pgfqpoint{5.333027in}{3.126903in}}%
\pgfpathlineto{\pgfqpoint{5.340520in}{3.116832in}}%
\pgfpathlineto{\pgfqpoint{5.343732in}{3.115900in}}%
\pgfpathlineto{\pgfqpoint{5.346675in}{3.117500in}}%
\pgfpathlineto{\pgfqpoint{5.350689in}{3.122653in}}%
\pgfpathlineto{\pgfqpoint{5.358985in}{3.133760in}}%
\pgfpathlineto{\pgfqpoint{5.362197in}{3.134472in}}%
\pgfpathlineto{\pgfqpoint{5.365140in}{3.132681in}}%
\pgfpathlineto{\pgfqpoint{5.369422in}{3.126950in}}%
\pgfpathlineto{\pgfqpoint{5.377183in}{3.116669in}}%
\pgfpathlineto{\pgfqpoint{5.380394in}{3.115944in}}%
\pgfpathlineto{\pgfqpoint{5.383338in}{3.117723in}}%
\pgfpathlineto{\pgfqpoint{5.387619in}{3.123446in}}%
\pgfpathlineto{\pgfqpoint{5.395380in}{3.133740in}}%
\pgfpathlineto{\pgfqpoint{5.398591in}{3.134478in}}%
\pgfpathlineto{\pgfqpoint{5.401535in}{3.132710in}}%
\pgfpathlineto{\pgfqpoint{5.405817in}{3.126997in}}%
\pgfpathlineto{\pgfqpoint{5.413577in}{3.116689in}}%
\pgfpathlineto{\pgfqpoint{5.416789in}{3.115937in}}%
\pgfpathlineto{\pgfqpoint{5.419732in}{3.117694in}}%
\pgfpathlineto{\pgfqpoint{5.424014in}{3.123398in}}%
\pgfpathlineto{\pgfqpoint{5.431775in}{3.133720in}}%
\pgfpathlineto{\pgfqpoint{5.434986in}{3.134485in}}%
\pgfpathlineto{\pgfqpoint{5.437930in}{3.132739in}}%
\pgfpathlineto{\pgfqpoint{5.442211in}{3.127044in}}%
\pgfpathlineto{\pgfqpoint{5.449972in}{3.116709in}}%
\pgfpathlineto{\pgfqpoint{5.453183in}{3.115931in}}%
\pgfpathlineto{\pgfqpoint{5.456127in}{3.117666in}}%
\pgfpathlineto{\pgfqpoint{5.460409in}{3.123351in}}%
\pgfpathlineto{\pgfqpoint{5.468169in}{3.133700in}}%
\pgfpathlineto{\pgfqpoint{5.471381in}{3.134491in}}%
\pgfpathlineto{\pgfqpoint{5.474324in}{3.132767in}}%
\pgfpathlineto{\pgfqpoint{5.478606in}{3.127091in}}%
\pgfpathlineto{\pgfqpoint{5.486367in}{3.116729in}}%
\pgfpathlineto{\pgfqpoint{5.489578in}{3.115925in}}%
\pgfpathlineto{\pgfqpoint{5.492522in}{3.117638in}}%
\pgfpathlineto{\pgfqpoint{5.496803in}{3.123304in}}%
\pgfpathlineto{\pgfqpoint{5.504564in}{3.133680in}}%
\pgfpathlineto{\pgfqpoint{5.507775in}{3.134496in}}%
\pgfpathlineto{\pgfqpoint{5.510719in}{3.132795in}}%
\pgfpathlineto{\pgfqpoint{5.515001in}{3.127138in}}%
\pgfpathlineto{\pgfqpoint{5.522761in}{3.116749in}}%
\pgfpathlineto{\pgfqpoint{5.525973in}{3.115920in}}%
\pgfpathlineto{\pgfqpoint{5.528916in}{3.117610in}}%
\pgfpathlineto{\pgfqpoint{5.533198in}{3.123258in}}%
\pgfpathlineto{\pgfqpoint{5.540959in}{3.133660in}}%
\pgfpathlineto{\pgfqpoint{5.544170in}{3.134502in}}%
\pgfpathlineto{\pgfqpoint{5.547114in}{3.132823in}}%
\pgfpathlineto{\pgfqpoint{5.551395in}{3.127185in}}%
\pgfpathlineto{\pgfqpoint{5.559156in}{3.116769in}}%
\pgfpathlineto{\pgfqpoint{5.562367in}{3.115915in}}%
\pgfpathlineto{\pgfqpoint{5.565311in}{3.117582in}}%
\pgfpathlineto{\pgfqpoint{5.569593in}{3.123211in}}%
\pgfpathlineto{\pgfqpoint{5.577353in}{3.133639in}}%
\pgfpathlineto{\pgfqpoint{5.580565in}{3.134507in}}%
\pgfpathlineto{\pgfqpoint{5.583508in}{3.132851in}}%
\pgfpathlineto{\pgfqpoint{5.587790in}{3.127232in}}%
\pgfpathlineto{\pgfqpoint{5.595551in}{3.116790in}}%
\pgfpathlineto{\pgfqpoint{5.598762in}{3.115909in}}%
\pgfpathlineto{\pgfqpoint{5.601706in}{3.117554in}}%
\pgfpathlineto{\pgfqpoint{5.605987in}{3.123164in}}%
\pgfpathlineto{\pgfqpoint{5.614015in}{3.133798in}}%
\pgfpathlineto{\pgfqpoint{5.617227in}{3.134458in}}%
\pgfpathlineto{\pgfqpoint{5.620170in}{3.132623in}}%
\pgfpathlineto{\pgfqpoint{5.624452in}{3.126855in}}%
\pgfpathlineto{\pgfqpoint{5.631945in}{3.116811in}}%
\pgfpathlineto{\pgfqpoint{5.635157in}{3.115905in}}%
\pgfpathlineto{\pgfqpoint{5.638100in}{3.117527in}}%
\pgfpathlineto{\pgfqpoint{5.642382in}{3.123117in}}%
\pgfpathlineto{\pgfqpoint{5.650410in}{3.133779in}}%
\pgfpathlineto{\pgfqpoint{5.653621in}{3.134465in}}%
\pgfpathlineto{\pgfqpoint{5.656565in}{3.132652in}}%
\pgfpathlineto{\pgfqpoint{5.660847in}{3.126903in}}%
\pgfpathlineto{\pgfqpoint{5.668340in}{3.116832in}}%
\pgfpathlineto{\pgfqpoint{5.671551in}{3.115900in}}%
\pgfpathlineto{\pgfqpoint{5.674495in}{3.117500in}}%
\pgfpathlineto{\pgfqpoint{5.678509in}{3.122653in}}%
\pgfpathlineto{\pgfqpoint{5.686805in}{3.133760in}}%
\pgfpathlineto{\pgfqpoint{5.690016in}{3.134472in}}%
\pgfpathlineto{\pgfqpoint{5.692960in}{3.132681in}}%
\pgfpathlineto{\pgfqpoint{5.697242in}{3.126950in}}%
\pgfpathlineto{\pgfqpoint{5.705002in}{3.116669in}}%
\pgfpathlineto{\pgfqpoint{5.708213in}{3.115944in}}%
\pgfpathlineto{\pgfqpoint{5.711157in}{3.117723in}}%
\pgfpathlineto{\pgfqpoint{5.715439in}{3.123446in}}%
\pgfpathlineto{\pgfqpoint{5.723199in}{3.133740in}}%
\pgfpathlineto{\pgfqpoint{5.726411in}{3.134478in}}%
\pgfpathlineto{\pgfqpoint{5.729354in}{3.132710in}}%
\pgfpathlineto{\pgfqpoint{5.733636in}{3.126997in}}%
\pgfpathlineto{\pgfqpoint{5.741397in}{3.116689in}}%
\pgfpathlineto{\pgfqpoint{5.744608in}{3.115937in}}%
\pgfpathlineto{\pgfqpoint{5.747552in}{3.117694in}}%
\pgfpathlineto{\pgfqpoint{5.751834in}{3.123398in}}%
\pgfpathlineto{\pgfqpoint{5.759594in}{3.133720in}}%
\pgfpathlineto{\pgfqpoint{5.762805in}{3.134485in}}%
\pgfpathlineto{\pgfqpoint{5.765749in}{3.132739in}}%
\pgfpathlineto{\pgfqpoint{5.770031in}{3.127044in}}%
\pgfpathlineto{\pgfqpoint{5.777791in}{3.116709in}}%
\pgfpathlineto{\pgfqpoint{5.781003in}{3.115931in}}%
\pgfpathlineto{\pgfqpoint{5.783946in}{3.117666in}}%
\pgfpathlineto{\pgfqpoint{5.788228in}{3.123351in}}%
\pgfpathlineto{\pgfqpoint{5.795989in}{3.133700in}}%
\pgfpathlineto{\pgfqpoint{5.799200in}{3.134491in}}%
\pgfpathlineto{\pgfqpoint{5.802144in}{3.132767in}}%
\pgfpathlineto{\pgfqpoint{5.806426in}{3.127091in}}%
\pgfpathlineto{\pgfqpoint{5.814186in}{3.116729in}}%
\pgfpathlineto{\pgfqpoint{5.817397in}{3.115925in}}%
\pgfpathlineto{\pgfqpoint{5.820341in}{3.117638in}}%
\pgfpathlineto{\pgfqpoint{5.824623in}{3.123304in}}%
\pgfpathlineto{\pgfqpoint{5.832383in}{3.133680in}}%
\pgfpathlineto{\pgfqpoint{5.835595in}{3.134496in}}%
\pgfpathlineto{\pgfqpoint{5.838538in}{3.132795in}}%
\pgfpathlineto{\pgfqpoint{5.842820in}{3.127138in}}%
\pgfpathlineto{\pgfqpoint{5.850581in}{3.116749in}}%
\pgfpathlineto{\pgfqpoint{5.853792in}{3.115920in}}%
\pgfpathlineto{\pgfqpoint{5.856736in}{3.117610in}}%
\pgfpathlineto{\pgfqpoint{5.861018in}{3.123258in}}%
\pgfpathlineto{\pgfqpoint{5.868778in}{3.133660in}}%
\pgfpathlineto{\pgfqpoint{5.871989in}{3.134502in}}%
\pgfpathlineto{\pgfqpoint{5.874933in}{3.132823in}}%
\pgfpathlineto{\pgfqpoint{5.879215in}{3.127185in}}%
\pgfpathlineto{\pgfqpoint{5.886975in}{3.116769in}}%
\pgfpathlineto{\pgfqpoint{5.890187in}{3.115915in}}%
\pgfpathlineto{\pgfqpoint{5.893130in}{3.117582in}}%
\pgfpathlineto{\pgfqpoint{5.897412in}{3.123211in}}%
\pgfpathlineto{\pgfqpoint{5.905173in}{3.133639in}}%
\pgfpathlineto{\pgfqpoint{5.908384in}{3.134507in}}%
\pgfpathlineto{\pgfqpoint{5.911328in}{3.132851in}}%
\pgfpathlineto{\pgfqpoint{5.915610in}{3.127232in}}%
\pgfpathlineto{\pgfqpoint{5.923370in}{3.116790in}}%
\pgfpathlineto{\pgfqpoint{5.926581in}{3.115909in}}%
\pgfpathlineto{\pgfqpoint{5.929525in}{3.117554in}}%
\pgfpathlineto{\pgfqpoint{5.933807in}{3.123164in}}%
\pgfpathlineto{\pgfqpoint{5.941835in}{3.133798in}}%
\pgfpathlineto{\pgfqpoint{5.945046in}{3.134458in}}%
\pgfpathlineto{\pgfqpoint{5.947990in}{3.132623in}}%
\pgfpathlineto{\pgfqpoint{5.952272in}{3.126855in}}%
\pgfpathlineto{\pgfqpoint{5.959765in}{3.116811in}}%
\pgfpathlineto{\pgfqpoint{5.962976in}{3.115905in}}%
\pgfpathlineto{\pgfqpoint{5.965920in}{3.117527in}}%
\pgfpathlineto{\pgfqpoint{5.970202in}{3.123117in}}%
\pgfpathlineto{\pgfqpoint{5.978230in}{3.133779in}}%
\pgfpathlineto{\pgfqpoint{5.981441in}{3.134465in}}%
\pgfpathlineto{\pgfqpoint{5.984385in}{3.132652in}}%
\pgfpathlineto{\pgfqpoint{5.988666in}{3.126903in}}%
\pgfpathlineto{\pgfqpoint{5.996159in}{3.116832in}}%
\pgfpathlineto{\pgfqpoint{5.999371in}{3.115900in}}%
\pgfpathlineto{\pgfqpoint{6.002314in}{3.117500in}}%
\pgfpathlineto{\pgfqpoint{6.006329in}{3.122653in}}%
\pgfpathlineto{\pgfqpoint{6.014624in}{3.133760in}}%
\pgfpathlineto{\pgfqpoint{6.017836in}{3.134472in}}%
\pgfpathlineto{\pgfqpoint{6.020779in}{3.132681in}}%
\pgfpathlineto{\pgfqpoint{6.025061in}{3.126950in}}%
\pgfpathlineto{\pgfqpoint{6.032822in}{3.116669in}}%
\pgfpathlineto{\pgfqpoint{6.036033in}{3.115944in}}%
\pgfpathlineto{\pgfqpoint{6.038977in}{3.117723in}}%
\pgfpathlineto{\pgfqpoint{6.043258in}{3.123446in}}%
\pgfpathlineto{\pgfqpoint{6.051019in}{3.133740in}}%
\pgfpathlineto{\pgfqpoint{6.054230in}{3.134478in}}%
\pgfpathlineto{\pgfqpoint{6.057174in}{3.132710in}}%
\pgfpathlineto{\pgfqpoint{6.061456in}{3.126997in}}%
\pgfpathlineto{\pgfqpoint{6.069216in}{3.116689in}}%
\pgfpathlineto{\pgfqpoint{6.072428in}{3.115937in}}%
\pgfpathlineto{\pgfqpoint{6.075371in}{3.117694in}}%
\pgfpathlineto{\pgfqpoint{6.079653in}{3.123398in}}%
\pgfpathlineto{\pgfqpoint{6.087414in}{3.133720in}}%
\pgfpathlineto{\pgfqpoint{6.090625in}{3.134485in}}%
\pgfpathlineto{\pgfqpoint{6.093569in}{3.132739in}}%
\pgfpathlineto{\pgfqpoint{6.097850in}{3.127044in}}%
\pgfpathlineto{\pgfqpoint{6.105611in}{3.116709in}}%
\pgfpathlineto{\pgfqpoint{6.108822in}{3.115931in}}%
\pgfpathlineto{\pgfqpoint{6.111766in}{3.117666in}}%
\pgfpathlineto{\pgfqpoint{6.116048in}{3.123351in}}%
\pgfpathlineto{\pgfqpoint{6.123808in}{3.133700in}}%
\pgfpathlineto{\pgfqpoint{6.127020in}{3.134491in}}%
\pgfpathlineto{\pgfqpoint{6.129963in}{3.132767in}}%
\pgfpathlineto{\pgfqpoint{6.134245in}{3.127091in}}%
\pgfpathlineto{\pgfqpoint{6.142006in}{3.116729in}}%
\pgfpathlineto{\pgfqpoint{6.145217in}{3.115925in}}%
\pgfpathlineto{\pgfqpoint{6.148161in}{3.117638in}}%
\pgfpathlineto{\pgfqpoint{6.152442in}{3.123304in}}%
\pgfpathlineto{\pgfqpoint{6.160203in}{3.133680in}}%
\pgfpathlineto{\pgfqpoint{6.163414in}{3.134496in}}%
\pgfpathlineto{\pgfqpoint{6.166358in}{3.132795in}}%
\pgfpathlineto{\pgfqpoint{6.170640in}{3.127138in}}%
\pgfpathlineto{\pgfqpoint{6.178400in}{3.116749in}}%
\pgfpathlineto{\pgfqpoint{6.181612in}{3.115920in}}%
\pgfpathlineto{\pgfqpoint{6.184555in}{3.117610in}}%
\pgfpathlineto{\pgfqpoint{6.188837in}{3.123258in}}%
\pgfpathlineto{\pgfqpoint{6.196598in}{3.133660in}}%
\pgfpathlineto{\pgfqpoint{6.199809in}{3.134502in}}%
\pgfpathlineto{\pgfqpoint{6.202753in}{3.132823in}}%
\pgfpathlineto{\pgfqpoint{6.207034in}{3.127185in}}%
\pgfpathlineto{\pgfqpoint{6.214795in}{3.116769in}}%
\pgfpathlineto{\pgfqpoint{6.218006in}{3.115915in}}%
\pgfpathlineto{\pgfqpoint{6.220950in}{3.117582in}}%
\pgfpathlineto{\pgfqpoint{6.225232in}{3.123211in}}%
\pgfpathlineto{\pgfqpoint{6.232992in}{3.133639in}}%
\pgfpathlineto{\pgfqpoint{6.236204in}{3.134507in}}%
\pgfpathlineto{\pgfqpoint{6.239147in}{3.132851in}}%
\pgfpathlineto{\pgfqpoint{6.243429in}{3.127232in}}%
\pgfpathlineto{\pgfqpoint{6.251190in}{3.116790in}}%
\pgfpathlineto{\pgfqpoint{6.254401in}{3.115909in}}%
\pgfpathlineto{\pgfqpoint{6.257345in}{3.117554in}}%
\pgfpathlineto{\pgfqpoint{6.261626in}{3.123164in}}%
\pgfpathlineto{\pgfqpoint{6.269655in}{3.133798in}}%
\pgfpathlineto{\pgfqpoint{6.272866in}{3.134458in}}%
\pgfpathlineto{\pgfqpoint{6.275810in}{3.132623in}}%
\pgfpathlineto{\pgfqpoint{6.280091in}{3.126855in}}%
\pgfpathlineto{\pgfqpoint{6.287584in}{3.116811in}}%
\pgfpathlineto{\pgfqpoint{6.290796in}{3.115905in}}%
\pgfpathlineto{\pgfqpoint{6.293739in}{3.117527in}}%
\pgfpathlineto{\pgfqpoint{6.298021in}{3.123117in}}%
\pgfpathlineto{\pgfqpoint{6.306049in}{3.133779in}}%
\pgfpathlineto{\pgfqpoint{6.309261in}{3.134465in}}%
\pgfpathlineto{\pgfqpoint{6.312204in}{3.132652in}}%
\pgfpathlineto{\pgfqpoint{6.316486in}{3.126903in}}%
\pgfpathlineto{\pgfqpoint{6.323979in}{3.116832in}}%
\pgfpathlineto{\pgfqpoint{6.327190in}{3.115900in}}%
\pgfpathlineto{\pgfqpoint{6.330134in}{3.117500in}}%
\pgfpathlineto{\pgfqpoint{6.334148in}{3.122653in}}%
\pgfpathlineto{\pgfqpoint{6.342444in}{3.133760in}}%
\pgfpathlineto{\pgfqpoint{6.345655in}{3.134472in}}%
\pgfpathlineto{\pgfqpoint{6.348599in}{3.132681in}}%
\pgfpathlineto{\pgfqpoint{6.352881in}{3.126950in}}%
\pgfpathlineto{\pgfqpoint{6.360641in}{3.116669in}}%
\pgfpathlineto{\pgfqpoint{6.363853in}{3.115944in}}%
\pgfpathlineto{\pgfqpoint{6.366796in}{3.117723in}}%
\pgfpathlineto{\pgfqpoint{6.371078in}{3.123446in}}%
\pgfpathlineto{\pgfqpoint{6.378839in}{3.133740in}}%
\pgfpathlineto{\pgfqpoint{6.382050in}{3.134478in}}%
\pgfpathlineto{\pgfqpoint{6.384994in}{3.132710in}}%
\pgfpathlineto{\pgfqpoint{6.389275in}{3.126997in}}%
\pgfpathlineto{\pgfqpoint{6.397036in}{3.116689in}}%
\pgfpathlineto{\pgfqpoint{6.400247in}{3.115937in}}%
\pgfpathlineto{\pgfqpoint{6.403191in}{3.117694in}}%
\pgfpathlineto{\pgfqpoint{6.407473in}{3.123398in}}%
\pgfpathlineto{\pgfqpoint{6.415233in}{3.133720in}}%
\pgfpathlineto{\pgfqpoint{6.418445in}{3.134485in}}%
\pgfpathlineto{\pgfqpoint{6.421388in}{3.132739in}}%
\pgfpathlineto{\pgfqpoint{6.425670in}{3.127044in}}%
\pgfpathlineto{\pgfqpoint{6.433431in}{3.116709in}}%
\pgfpathlineto{\pgfqpoint{6.436642in}{3.115931in}}%
\pgfpathlineto{\pgfqpoint{6.439586in}{3.117666in}}%
\pgfpathlineto{\pgfqpoint{6.443867in}{3.123351in}}%
\pgfpathlineto{\pgfqpoint{6.451628in}{3.133700in}}%
\pgfpathlineto{\pgfqpoint{6.454839in}{3.134491in}}%
\pgfpathlineto{\pgfqpoint{6.457783in}{3.132767in}}%
\pgfpathlineto{\pgfqpoint{6.462065in}{3.127091in}}%
\pgfpathlineto{\pgfqpoint{6.469825in}{3.116729in}}%
\pgfpathlineto{\pgfqpoint{6.473037in}{3.115925in}}%
\pgfpathlineto{\pgfqpoint{6.475980in}{3.117638in}}%
\pgfpathlineto{\pgfqpoint{6.480262in}{3.123304in}}%
\pgfpathlineto{\pgfqpoint{6.488023in}{3.133680in}}%
\pgfpathlineto{\pgfqpoint{6.491234in}{3.134496in}}%
\pgfpathlineto{\pgfqpoint{6.494178in}{3.132795in}}%
\pgfpathlineto{\pgfqpoint{6.498459in}{3.127138in}}%
\pgfpathlineto{\pgfqpoint{6.506220in}{3.116749in}}%
\pgfpathlineto{\pgfqpoint{6.509431in}{3.115920in}}%
\pgfpathlineto{\pgfqpoint{6.512375in}{3.117610in}}%
\pgfpathlineto{\pgfqpoint{6.516657in}{3.123258in}}%
\pgfpathlineto{\pgfqpoint{6.524417in}{3.133660in}}%
\pgfpathlineto{\pgfqpoint{6.527629in}{3.134502in}}%
\pgfpathlineto{\pgfqpoint{6.530572in}{3.132823in}}%
\pgfpathlineto{\pgfqpoint{6.534854in}{3.127185in}}%
\pgfpathlineto{\pgfqpoint{6.542615in}{3.116769in}}%
\pgfpathlineto{\pgfqpoint{6.545826in}{3.115915in}}%
\pgfpathlineto{\pgfqpoint{6.548770in}{3.117582in}}%
\pgfpathlineto{\pgfqpoint{6.553051in}{3.123211in}}%
\pgfpathlineto{\pgfqpoint{6.560812in}{3.133639in}}%
\pgfpathlineto{\pgfqpoint{6.564023in}{3.134507in}}%
\pgfpathlineto{\pgfqpoint{6.566967in}{3.132851in}}%
\pgfpathlineto{\pgfqpoint{6.571249in}{3.127232in}}%
\pgfpathlineto{\pgfqpoint{6.579009in}{3.116790in}}%
\pgfpathlineto{\pgfqpoint{6.582221in}{3.115909in}}%
\pgfpathlineto{\pgfqpoint{6.585164in}{3.117554in}}%
\pgfpathlineto{\pgfqpoint{6.589446in}{3.123164in}}%
\pgfpathlineto{\pgfqpoint{6.597474in}{3.133798in}}%
\pgfpathlineto{\pgfqpoint{6.600686in}{3.134458in}}%
\pgfpathlineto{\pgfqpoint{6.603629in}{3.132623in}}%
\pgfpathlineto{\pgfqpoint{6.607911in}{3.126855in}}%
\pgfpathlineto{\pgfqpoint{6.615404in}{3.116811in}}%
\pgfpathlineto{\pgfqpoint{6.618615in}{3.115905in}}%
\pgfpathlineto{\pgfqpoint{6.621559in}{3.117527in}}%
\pgfpathlineto{\pgfqpoint{6.625841in}{3.123117in}}%
\pgfpathlineto{\pgfqpoint{6.633869in}{3.133779in}}%
\pgfpathlineto{\pgfqpoint{6.637080in}{3.134465in}}%
\pgfpathlineto{\pgfqpoint{6.640024in}{3.132652in}}%
\pgfpathlineto{\pgfqpoint{6.644306in}{3.126903in}}%
\pgfpathlineto{\pgfqpoint{6.651799in}{3.116832in}}%
\pgfpathlineto{\pgfqpoint{6.655010in}{3.115900in}}%
\pgfpathlineto{\pgfqpoint{6.657954in}{3.117500in}}%
\pgfpathlineto{\pgfqpoint{6.661968in}{3.122653in}}%
\pgfpathlineto{\pgfqpoint{6.663306in}{3.124778in}}%
\pgfpathlineto{\pgfqpoint{6.663306in}{3.124778in}}%
\pgfusepath{stroke}%
\end{pgfscope}%
\begin{pgfscope}%
\pgfpathrectangle{\pgfqpoint{0.467797in}{2.292089in}}{\pgfqpoint{6.490533in}{1.666241in}}%
\pgfusepath{clip}%
\pgfsetrectcap%
\pgfsetroundjoin%
\pgfsetlinewidth{1.505625pt}%
\definecolor{currentstroke}{rgb}{1.000000,0.498039,0.054902}%
\pgfsetstrokecolor{currentstroke}%
\pgfsetdash{}{0pt}%
\pgfpathmoveto{\pgfqpoint{0.762821in}{3.125209in}}%
\pgfpathlineto{\pgfqpoint{0.769243in}{3.133483in}}%
\pgfpathlineto{\pgfqpoint{0.772455in}{3.134255in}}%
\pgfpathlineto{\pgfqpoint{0.775398in}{3.132461in}}%
\pgfpathlineto{\pgfqpoint{0.779680in}{3.126677in}}%
\pgfpathlineto{\pgfqpoint{0.786906in}{3.117014in}}%
\pgfpathlineto{\pgfqpoint{0.790117in}{3.116142in}}%
\pgfpathlineto{\pgfqpoint{0.793061in}{3.117851in}}%
\pgfpathlineto{\pgfqpoint{0.797342in}{3.123565in}}%
\pgfpathlineto{\pgfqpoint{0.804835in}{3.133513in}}%
\pgfpathlineto{\pgfqpoint{0.808047in}{3.134245in}}%
\pgfpathlineto{\pgfqpoint{0.810990in}{3.132416in}}%
\pgfpathlineto{\pgfqpoint{0.815272in}{3.126605in}}%
\pgfpathlineto{\pgfqpoint{0.822497in}{3.116982in}}%
\pgfpathlineto{\pgfqpoint{0.825709in}{3.116150in}}%
\pgfpathlineto{\pgfqpoint{0.828652in}{3.117894in}}%
\pgfpathlineto{\pgfqpoint{0.832934in}{3.123636in}}%
\pgfpathlineto{\pgfqpoint{0.840427in}{3.133543in}}%
\pgfpathlineto{\pgfqpoint{0.843638in}{3.134235in}}%
\pgfpathlineto{\pgfqpoint{0.846582in}{3.132371in}}%
\pgfpathlineto{\pgfqpoint{0.851131in}{3.126105in}}%
\pgfpathlineto{\pgfqpoint{0.858089in}{3.116951in}}%
\pgfpathlineto{\pgfqpoint{0.861301in}{3.116159in}}%
\pgfpathlineto{\pgfqpoint{0.864244in}{3.117938in}}%
\pgfpathlineto{\pgfqpoint{0.868526in}{3.123708in}}%
\pgfpathlineto{\pgfqpoint{0.875751in}{3.133390in}}%
\pgfpathlineto{\pgfqpoint{0.878963in}{3.134281in}}%
\pgfpathlineto{\pgfqpoint{0.881906in}{3.132588in}}%
\pgfpathlineto{\pgfqpoint{0.886188in}{3.126887in}}%
\pgfpathlineto{\pgfqpoint{0.893681in}{3.116920in}}%
\pgfpathlineto{\pgfqpoint{0.896892in}{3.116169in}}%
\pgfpathlineto{\pgfqpoint{0.899836in}{3.117982in}}%
\pgfpathlineto{\pgfqpoint{0.904118in}{3.123780in}}%
\pgfpathlineto{\pgfqpoint{0.911343in}{3.133422in}}%
\pgfpathlineto{\pgfqpoint{0.914555in}{3.134272in}}%
\pgfpathlineto{\pgfqpoint{0.917498in}{3.132545in}}%
\pgfpathlineto{\pgfqpoint{0.921780in}{3.126816in}}%
\pgfpathlineto{\pgfqpoint{0.929273in}{3.116890in}}%
\pgfpathlineto{\pgfqpoint{0.932484in}{3.116179in}}%
\pgfpathlineto{\pgfqpoint{0.935428in}{3.118027in}}%
\pgfpathlineto{\pgfqpoint{0.939710in}{3.123852in}}%
\pgfpathlineto{\pgfqpoint{0.946935in}{3.133454in}}%
\pgfpathlineto{\pgfqpoint{0.950146in}{3.134264in}}%
\pgfpathlineto{\pgfqpoint{0.953090in}{3.132501in}}%
\pgfpathlineto{\pgfqpoint{0.957372in}{3.126744in}}%
\pgfpathlineto{\pgfqpoint{0.964865in}{3.116860in}}%
\pgfpathlineto{\pgfqpoint{0.967809in}{3.116135in}}%
\pgfpathlineto{\pgfqpoint{0.970752in}{3.117811in}}%
\pgfpathlineto{\pgfqpoint{0.975034in}{3.123498in}}%
\pgfpathlineto{\pgfqpoint{0.982527in}{3.133485in}}%
\pgfpathlineto{\pgfqpoint{0.985738in}{3.134254in}}%
\pgfpathlineto{\pgfqpoint{0.988682in}{3.132457in}}%
\pgfpathlineto{\pgfqpoint{0.992964in}{3.126672in}}%
\pgfpathlineto{\pgfqpoint{1.000189in}{3.117011in}}%
\pgfpathlineto{\pgfqpoint{1.003400in}{3.116143in}}%
\pgfpathlineto{\pgfqpoint{1.006344in}{3.117854in}}%
\pgfpathlineto{\pgfqpoint{1.010626in}{3.123570in}}%
\pgfpathlineto{\pgfqpoint{1.018119in}{3.133515in}}%
\pgfpathlineto{\pgfqpoint{1.021330in}{3.134244in}}%
\pgfpathlineto{\pgfqpoint{1.024274in}{3.132413in}}%
\pgfpathlineto{\pgfqpoint{1.028555in}{3.126600in}}%
\pgfpathlineto{\pgfqpoint{1.035781in}{3.116980in}}%
\pgfpathlineto{\pgfqpoint{1.038992in}{3.116151in}}%
\pgfpathlineto{\pgfqpoint{1.041936in}{3.117897in}}%
\pgfpathlineto{\pgfqpoint{1.046218in}{3.123642in}}%
\pgfpathlineto{\pgfqpoint{1.053711in}{3.133545in}}%
\pgfpathlineto{\pgfqpoint{1.056922in}{3.134234in}}%
\pgfpathlineto{\pgfqpoint{1.059866in}{3.132368in}}%
\pgfpathlineto{\pgfqpoint{1.064415in}{3.126100in}}%
\pgfpathlineto{\pgfqpoint{1.071373in}{3.116948in}}%
\pgfpathlineto{\pgfqpoint{1.074584in}{3.116160in}}%
\pgfpathlineto{\pgfqpoint{1.077528in}{3.117941in}}%
\pgfpathlineto{\pgfqpoint{1.081809in}{3.123713in}}%
\pgfpathlineto{\pgfqpoint{1.089035in}{3.133393in}}%
\pgfpathlineto{\pgfqpoint{1.092246in}{3.134280in}}%
\pgfpathlineto{\pgfqpoint{1.095190in}{3.132585in}}%
\pgfpathlineto{\pgfqpoint{1.099472in}{3.126882in}}%
\pgfpathlineto{\pgfqpoint{1.106965in}{3.116918in}}%
\pgfpathlineto{\pgfqpoint{1.110176in}{3.116170in}}%
\pgfpathlineto{\pgfqpoint{1.113120in}{3.117985in}}%
\pgfpathlineto{\pgfqpoint{1.117401in}{3.123785in}}%
\pgfpathlineto{\pgfqpoint{1.124627in}{3.133425in}}%
\pgfpathlineto{\pgfqpoint{1.127838in}{3.134272in}}%
\pgfpathlineto{\pgfqpoint{1.130782in}{3.132542in}}%
\pgfpathlineto{\pgfqpoint{1.135063in}{3.126811in}}%
\pgfpathlineto{\pgfqpoint{1.142556in}{3.116888in}}%
\pgfpathlineto{\pgfqpoint{1.145768in}{3.116180in}}%
\pgfpathlineto{\pgfqpoint{1.148711in}{3.118030in}}%
\pgfpathlineto{\pgfqpoint{1.152993in}{3.123857in}}%
\pgfpathlineto{\pgfqpoint{1.160219in}{3.133456in}}%
\pgfpathlineto{\pgfqpoint{1.163430in}{3.134263in}}%
\pgfpathlineto{\pgfqpoint{1.166374in}{3.132498in}}%
\pgfpathlineto{\pgfqpoint{1.170655in}{3.126739in}}%
\pgfpathlineto{\pgfqpoint{1.178148in}{3.116858in}}%
\pgfpathlineto{\pgfqpoint{1.181092in}{3.116135in}}%
\pgfpathlineto{\pgfqpoint{1.184036in}{3.117814in}}%
\pgfpathlineto{\pgfqpoint{1.188317in}{3.123503in}}%
\pgfpathlineto{\pgfqpoint{1.195810in}{3.133487in}}%
\pgfpathlineto{\pgfqpoint{1.199022in}{3.134254in}}%
\pgfpathlineto{\pgfqpoint{1.201965in}{3.132454in}}%
\pgfpathlineto{\pgfqpoint{1.206247in}{3.126667in}}%
\pgfpathlineto{\pgfqpoint{1.213472in}{3.117009in}}%
\pgfpathlineto{\pgfqpoint{1.216684in}{3.116143in}}%
\pgfpathlineto{\pgfqpoint{1.219627in}{3.117857in}}%
\pgfpathlineto{\pgfqpoint{1.223909in}{3.123575in}}%
\pgfpathlineto{\pgfqpoint{1.231402in}{3.133517in}}%
\pgfpathlineto{\pgfqpoint{1.234613in}{3.134244in}}%
\pgfpathlineto{\pgfqpoint{1.237557in}{3.132410in}}%
\pgfpathlineto{\pgfqpoint{1.241839in}{3.126595in}}%
\pgfpathlineto{\pgfqpoint{1.249064in}{3.116977in}}%
\pgfpathlineto{\pgfqpoint{1.252276in}{3.116152in}}%
\pgfpathlineto{\pgfqpoint{1.255219in}{3.117900in}}%
\pgfpathlineto{\pgfqpoint{1.259501in}{3.123647in}}%
\pgfpathlineto{\pgfqpoint{1.266994in}{3.133547in}}%
\pgfpathlineto{\pgfqpoint{1.270205in}{3.134233in}}%
\pgfpathlineto{\pgfqpoint{1.273149in}{3.132365in}}%
\pgfpathlineto{\pgfqpoint{1.277698in}{3.126095in}}%
\pgfpathlineto{\pgfqpoint{1.284656in}{3.116946in}}%
\pgfpathlineto{\pgfqpoint{1.287867in}{3.116161in}}%
\pgfpathlineto{\pgfqpoint{1.290811in}{3.117944in}}%
\pgfpathlineto{\pgfqpoint{1.295093in}{3.123718in}}%
\pgfpathlineto{\pgfqpoint{1.302318in}{3.133395in}}%
\pgfpathlineto{\pgfqpoint{1.305530in}{3.134280in}}%
\pgfpathlineto{\pgfqpoint{1.308473in}{3.132582in}}%
\pgfpathlineto{\pgfqpoint{1.312755in}{3.126877in}}%
\pgfpathlineto{\pgfqpoint{1.320248in}{3.116916in}}%
\pgfpathlineto{\pgfqpoint{1.323459in}{3.116170in}}%
\pgfpathlineto{\pgfqpoint{1.326403in}{3.117988in}}%
\pgfpathlineto{\pgfqpoint{1.330685in}{3.123790in}}%
\pgfpathlineto{\pgfqpoint{1.337910in}{3.133427in}}%
\pgfpathlineto{\pgfqpoint{1.341121in}{3.134271in}}%
\pgfpathlineto{\pgfqpoint{1.344065in}{3.132539in}}%
\pgfpathlineto{\pgfqpoint{1.348347in}{3.126806in}}%
\pgfpathlineto{\pgfqpoint{1.355840in}{3.116885in}}%
\pgfpathlineto{\pgfqpoint{1.359051in}{3.116181in}}%
\pgfpathlineto{\pgfqpoint{1.361995in}{3.118033in}}%
\pgfpathlineto{\pgfqpoint{1.366544in}{3.124291in}}%
\pgfpathlineto{\pgfqpoint{1.373502in}{3.133458in}}%
\pgfpathlineto{\pgfqpoint{1.376713in}{3.134262in}}%
\pgfpathlineto{\pgfqpoint{1.379657in}{3.132495in}}%
\pgfpathlineto{\pgfqpoint{1.383939in}{3.126734in}}%
\pgfpathlineto{\pgfqpoint{1.391432in}{3.116856in}}%
\pgfpathlineto{\pgfqpoint{1.394375in}{3.116136in}}%
\pgfpathlineto{\pgfqpoint{1.397319in}{3.117817in}}%
\pgfpathlineto{\pgfqpoint{1.401601in}{3.123509in}}%
\pgfpathlineto{\pgfqpoint{1.409094in}{3.133489in}}%
\pgfpathlineto{\pgfqpoint{1.412305in}{3.134253in}}%
\pgfpathlineto{\pgfqpoint{1.415249in}{3.132451in}}%
\pgfpathlineto{\pgfqpoint{1.419531in}{3.126662in}}%
\pgfpathlineto{\pgfqpoint{1.426756in}{3.117007in}}%
\pgfpathlineto{\pgfqpoint{1.429967in}{3.116144in}}%
\pgfpathlineto{\pgfqpoint{1.432911in}{3.117860in}}%
\pgfpathlineto{\pgfqpoint{1.437193in}{3.123580in}}%
\pgfpathlineto{\pgfqpoint{1.444686in}{3.133520in}}%
\pgfpathlineto{\pgfqpoint{1.447897in}{3.134243in}}%
\pgfpathlineto{\pgfqpoint{1.450841in}{3.132406in}}%
\pgfpathlineto{\pgfqpoint{1.455122in}{3.126590in}}%
\pgfpathlineto{\pgfqpoint{1.462348in}{3.116975in}}%
\pgfpathlineto{\pgfqpoint{1.465559in}{3.116152in}}%
\pgfpathlineto{\pgfqpoint{1.468503in}{3.117903in}}%
\pgfpathlineto{\pgfqpoint{1.472784in}{3.123652in}}%
\pgfpathlineto{\pgfqpoint{1.480277in}{3.133549in}}%
\pgfpathlineto{\pgfqpoint{1.483489in}{3.134232in}}%
\pgfpathlineto{\pgfqpoint{1.486432in}{3.132362in}}%
\pgfpathlineto{\pgfqpoint{1.490982in}{3.126090in}}%
\pgfpathlineto{\pgfqpoint{1.497940in}{3.116944in}}%
\pgfpathlineto{\pgfqpoint{1.501151in}{3.116161in}}%
\pgfpathlineto{\pgfqpoint{1.504095in}{3.117947in}}%
\pgfpathlineto{\pgfqpoint{1.508376in}{3.123724in}}%
\pgfpathlineto{\pgfqpoint{1.515602in}{3.133397in}}%
\pgfpathlineto{\pgfqpoint{1.518813in}{3.134279in}}%
\pgfpathlineto{\pgfqpoint{1.521757in}{3.132579in}}%
\pgfpathlineto{\pgfqpoint{1.526038in}{3.126872in}}%
\pgfpathlineto{\pgfqpoint{1.533531in}{3.116913in}}%
\pgfpathlineto{\pgfqpoint{1.536743in}{3.116171in}}%
\pgfpathlineto{\pgfqpoint{1.539686in}{3.117992in}}%
\pgfpathlineto{\pgfqpoint{1.543968in}{3.123795in}}%
\pgfpathlineto{\pgfqpoint{1.551194in}{3.133429in}}%
\pgfpathlineto{\pgfqpoint{1.554405in}{3.134271in}}%
\pgfpathlineto{\pgfqpoint{1.557349in}{3.132536in}}%
\pgfpathlineto{\pgfqpoint{1.561630in}{3.126800in}}%
\pgfpathlineto{\pgfqpoint{1.569123in}{3.116883in}}%
\pgfpathlineto{\pgfqpoint{1.572335in}{3.116181in}}%
\pgfpathlineto{\pgfqpoint{1.575278in}{3.118036in}}%
\pgfpathlineto{\pgfqpoint{1.579828in}{3.124296in}}%
\pgfpathlineto{\pgfqpoint{1.586785in}{3.133461in}}%
\pgfpathlineto{\pgfqpoint{1.589997in}{3.134262in}}%
\pgfpathlineto{\pgfqpoint{1.592940in}{3.132492in}}%
\pgfpathlineto{\pgfqpoint{1.597222in}{3.126729in}}%
\pgfpathlineto{\pgfqpoint{1.604715in}{3.116854in}}%
\pgfpathlineto{\pgfqpoint{1.607659in}{3.116136in}}%
\pgfpathlineto{\pgfqpoint{1.610602in}{3.117820in}}%
\pgfpathlineto{\pgfqpoint{1.614884in}{3.123514in}}%
\pgfpathlineto{\pgfqpoint{1.622377in}{3.133491in}}%
\pgfpathlineto{\pgfqpoint{1.625589in}{3.134252in}}%
\pgfpathlineto{\pgfqpoint{1.628532in}{3.132448in}}%
\pgfpathlineto{\pgfqpoint{1.632814in}{3.126657in}}%
\pgfpathlineto{\pgfqpoint{1.640039in}{3.117004in}}%
\pgfpathlineto{\pgfqpoint{1.643251in}{3.116144in}}%
\pgfpathlineto{\pgfqpoint{1.646194in}{3.117863in}}%
\pgfpathlineto{\pgfqpoint{1.650476in}{3.123585in}}%
\pgfpathlineto{\pgfqpoint{1.657969in}{3.133522in}}%
\pgfpathlineto{\pgfqpoint{1.661180in}{3.134242in}}%
\pgfpathlineto{\pgfqpoint{1.664124in}{3.132403in}}%
\pgfpathlineto{\pgfqpoint{1.668406in}{3.126585in}}%
\pgfpathlineto{\pgfqpoint{1.675631in}{3.116973in}}%
\pgfpathlineto{\pgfqpoint{1.678842in}{3.116153in}}%
\pgfpathlineto{\pgfqpoint{1.681786in}{3.117907in}}%
\pgfpathlineto{\pgfqpoint{1.686068in}{3.123657in}}%
\pgfpathlineto{\pgfqpoint{1.693561in}{3.133551in}}%
\pgfpathlineto{\pgfqpoint{1.696772in}{3.134232in}}%
\pgfpathlineto{\pgfqpoint{1.699716in}{3.132358in}}%
\pgfpathlineto{\pgfqpoint{1.704265in}{3.126084in}}%
\pgfpathlineto{\pgfqpoint{1.711223in}{3.116942in}}%
\pgfpathlineto{\pgfqpoint{1.714434in}{3.116162in}}%
\pgfpathlineto{\pgfqpoint{1.717378in}{3.117950in}}%
\pgfpathlineto{\pgfqpoint{1.721660in}{3.123729in}}%
\pgfpathlineto{\pgfqpoint{1.728885in}{3.133400in}}%
\pgfpathlineto{\pgfqpoint{1.732096in}{3.134278in}}%
\pgfpathlineto{\pgfqpoint{1.735040in}{3.132576in}}%
\pgfpathlineto{\pgfqpoint{1.739322in}{3.126867in}}%
\pgfpathlineto{\pgfqpoint{1.746815in}{3.116911in}}%
\pgfpathlineto{\pgfqpoint{1.750026in}{3.116172in}}%
\pgfpathlineto{\pgfqpoint{1.752970in}{3.117995in}}%
\pgfpathlineto{\pgfqpoint{1.757252in}{3.123801in}}%
\pgfpathlineto{\pgfqpoint{1.764477in}{3.133431in}}%
\pgfpathlineto{\pgfqpoint{1.767688in}{3.134270in}}%
\pgfpathlineto{\pgfqpoint{1.770632in}{3.132532in}}%
\pgfpathlineto{\pgfqpoint{1.774914in}{3.126795in}}%
\pgfpathlineto{\pgfqpoint{1.782407in}{3.116881in}}%
\pgfpathlineto{\pgfqpoint{1.785618in}{3.116182in}}%
\pgfpathlineto{\pgfqpoint{1.788562in}{3.118040in}}%
\pgfpathlineto{\pgfqpoint{1.793111in}{3.124301in}}%
\pgfpathlineto{\pgfqpoint{1.800069in}{3.133463in}}%
\pgfpathlineto{\pgfqpoint{1.803280in}{3.134261in}}%
\pgfpathlineto{\pgfqpoint{1.806224in}{3.132489in}}%
\pgfpathlineto{\pgfqpoint{1.810506in}{3.126724in}}%
\pgfpathlineto{\pgfqpoint{1.817999in}{3.116852in}}%
\pgfpathlineto{\pgfqpoint{1.820942in}{3.116137in}}%
\pgfpathlineto{\pgfqpoint{1.823886in}{3.117823in}}%
\pgfpathlineto{\pgfqpoint{1.828168in}{3.123519in}}%
\pgfpathlineto{\pgfqpoint{1.835661in}{3.133494in}}%
\pgfpathlineto{\pgfqpoint{1.838872in}{3.134252in}}%
\pgfpathlineto{\pgfqpoint{1.841816in}{3.132445in}}%
\pgfpathlineto{\pgfqpoint{1.846097in}{3.126652in}}%
\pgfpathlineto{\pgfqpoint{1.853323in}{3.117002in}}%
\pgfpathlineto{\pgfqpoint{1.856534in}{3.116145in}}%
\pgfpathlineto{\pgfqpoint{1.859478in}{3.117866in}}%
\pgfpathlineto{\pgfqpoint{1.863759in}{3.123590in}}%
\pgfpathlineto{\pgfqpoint{1.871253in}{3.133524in}}%
\pgfpathlineto{\pgfqpoint{1.874464in}{3.134242in}}%
\pgfpathlineto{\pgfqpoint{1.877407in}{3.132400in}}%
\pgfpathlineto{\pgfqpoint{1.881689in}{3.126580in}}%
\pgfpathlineto{\pgfqpoint{1.888915in}{3.116971in}}%
\pgfpathlineto{\pgfqpoint{1.892126in}{3.116154in}}%
\pgfpathlineto{\pgfqpoint{1.895070in}{3.117910in}}%
\pgfpathlineto{\pgfqpoint{1.899351in}{3.123662in}}%
\pgfpathlineto{\pgfqpoint{1.906844in}{3.133554in}}%
\pgfpathlineto{\pgfqpoint{1.909788in}{3.134286in}}%
\pgfpathlineto{\pgfqpoint{1.912732in}{3.132615in}}%
\pgfpathlineto{\pgfqpoint{1.917013in}{3.126933in}}%
\pgfpathlineto{\pgfqpoint{1.924506in}{3.116940in}}%
\pgfpathlineto{\pgfqpoint{1.927718in}{3.116163in}}%
\pgfpathlineto{\pgfqpoint{1.930661in}{3.117954in}}%
\pgfpathlineto{\pgfqpoint{1.934943in}{3.123734in}}%
\pgfpathlineto{\pgfqpoint{1.942169in}{3.133402in}}%
\pgfpathlineto{\pgfqpoint{1.945380in}{3.134278in}}%
\pgfpathlineto{\pgfqpoint{1.948324in}{3.132573in}}%
\pgfpathlineto{\pgfqpoint{1.952605in}{3.126862in}}%
\pgfpathlineto{\pgfqpoint{1.960098in}{3.116909in}}%
\pgfpathlineto{\pgfqpoint{1.963310in}{3.116173in}}%
\pgfpathlineto{\pgfqpoint{1.966253in}{3.117998in}}%
\pgfpathlineto{\pgfqpoint{1.970535in}{3.123806in}}%
\pgfpathlineto{\pgfqpoint{1.977760in}{3.133434in}}%
\pgfpathlineto{\pgfqpoint{1.980972in}{3.134269in}}%
\pgfpathlineto{\pgfqpoint{1.983915in}{3.132529in}}%
\pgfpathlineto{\pgfqpoint{1.988197in}{3.126790in}}%
\pgfpathlineto{\pgfqpoint{1.995690in}{3.116879in}}%
\pgfpathlineto{\pgfqpoint{1.998901in}{3.116183in}}%
\pgfpathlineto{\pgfqpoint{2.001845in}{3.118043in}}%
\pgfpathlineto{\pgfqpoint{2.006394in}{3.124306in}}%
\pgfpathlineto{\pgfqpoint{2.013352in}{3.133465in}}%
\pgfpathlineto{\pgfqpoint{2.016564in}{3.134260in}}%
\pgfpathlineto{\pgfqpoint{2.019507in}{3.132486in}}%
\pgfpathlineto{\pgfqpoint{2.023789in}{3.126718in}}%
\pgfpathlineto{\pgfqpoint{2.031014in}{3.117032in}}%
\pgfpathlineto{\pgfqpoint{2.034226in}{3.116137in}}%
\pgfpathlineto{\pgfqpoint{2.037169in}{3.117826in}}%
\pgfpathlineto{\pgfqpoint{2.041451in}{3.123524in}}%
\pgfpathlineto{\pgfqpoint{2.048944in}{3.133496in}}%
\pgfpathlineto{\pgfqpoint{2.052155in}{3.134251in}}%
\pgfpathlineto{\pgfqpoint{2.055099in}{3.132442in}}%
\pgfpathlineto{\pgfqpoint{2.059381in}{3.126647in}}%
\pgfpathlineto{\pgfqpoint{2.066606in}{3.117000in}}%
\pgfpathlineto{\pgfqpoint{2.069818in}{3.116145in}}%
\pgfpathlineto{\pgfqpoint{2.072761in}{3.117869in}}%
\pgfpathlineto{\pgfqpoint{2.077043in}{3.123595in}}%
\pgfpathlineto{\pgfqpoint{2.084536in}{3.133526in}}%
\pgfpathlineto{\pgfqpoint{2.087747in}{3.134241in}}%
\pgfpathlineto{\pgfqpoint{2.090691in}{3.132397in}}%
\pgfpathlineto{\pgfqpoint{2.094973in}{3.126575in}}%
\pgfpathlineto{\pgfqpoint{2.102198in}{3.116968in}}%
\pgfpathlineto{\pgfqpoint{2.105409in}{3.116154in}}%
\pgfpathlineto{\pgfqpoint{2.108353in}{3.117913in}}%
\pgfpathlineto{\pgfqpoint{2.112635in}{3.123667in}}%
\pgfpathlineto{\pgfqpoint{2.120128in}{3.133556in}}%
\pgfpathlineto{\pgfqpoint{2.123071in}{3.134285in}}%
\pgfpathlineto{\pgfqpoint{2.126015in}{3.132612in}}%
\pgfpathlineto{\pgfqpoint{2.130297in}{3.126928in}}%
\pgfpathlineto{\pgfqpoint{2.137790in}{3.116937in}}%
\pgfpathlineto{\pgfqpoint{2.141001in}{3.116163in}}%
\pgfpathlineto{\pgfqpoint{2.143945in}{3.117957in}}%
\pgfpathlineto{\pgfqpoint{2.148227in}{3.123739in}}%
\pgfpathlineto{\pgfqpoint{2.155452in}{3.133404in}}%
\pgfpathlineto{\pgfqpoint{2.158663in}{3.134277in}}%
\pgfpathlineto{\pgfqpoint{2.161607in}{3.132570in}}%
\pgfpathlineto{\pgfqpoint{2.165889in}{3.126857in}}%
\pgfpathlineto{\pgfqpoint{2.173382in}{3.116907in}}%
\pgfpathlineto{\pgfqpoint{2.176593in}{3.116173in}}%
\pgfpathlineto{\pgfqpoint{2.179537in}{3.118001in}}%
\pgfpathlineto{\pgfqpoint{2.183818in}{3.123811in}}%
\pgfpathlineto{\pgfqpoint{2.191044in}{3.133436in}}%
\pgfpathlineto{\pgfqpoint{2.194255in}{3.134269in}}%
\pgfpathlineto{\pgfqpoint{2.197199in}{3.132526in}}%
\pgfpathlineto{\pgfqpoint{2.201481in}{3.126785in}}%
\pgfpathlineto{\pgfqpoint{2.208974in}{3.116877in}}%
\pgfpathlineto{\pgfqpoint{2.212185in}{3.116184in}}%
\pgfpathlineto{\pgfqpoint{2.215129in}{3.118046in}}%
\pgfpathlineto{\pgfqpoint{2.219678in}{3.124311in}}%
\pgfpathlineto{\pgfqpoint{2.226636in}{3.133467in}}%
\pgfpathlineto{\pgfqpoint{2.229847in}{3.134260in}}%
\pgfpathlineto{\pgfqpoint{2.232791in}{3.132483in}}%
\pgfpathlineto{\pgfqpoint{2.237072in}{3.126713in}}%
\pgfpathlineto{\pgfqpoint{2.244298in}{3.117030in}}%
\pgfpathlineto{\pgfqpoint{2.247509in}{3.116138in}}%
\pgfpathlineto{\pgfqpoint{2.250453in}{3.117829in}}%
\pgfpathlineto{\pgfqpoint{2.254735in}{3.123529in}}%
\pgfpathlineto{\pgfqpoint{2.262228in}{3.133498in}}%
\pgfpathlineto{\pgfqpoint{2.265439in}{3.134250in}}%
\pgfpathlineto{\pgfqpoint{2.268383in}{3.132438in}}%
\pgfpathlineto{\pgfqpoint{2.272664in}{3.126641in}}%
\pgfpathlineto{\pgfqpoint{2.279890in}{3.116998in}}%
\pgfpathlineto{\pgfqpoint{2.283101in}{3.116146in}}%
\pgfpathlineto{\pgfqpoint{2.286045in}{3.117872in}}%
\pgfpathlineto{\pgfqpoint{2.290326in}{3.123601in}}%
\pgfpathlineto{\pgfqpoint{2.297819in}{3.133528in}}%
\pgfpathlineto{\pgfqpoint{2.301031in}{3.134240in}}%
\pgfpathlineto{\pgfqpoint{2.303974in}{3.132394in}}%
\pgfpathlineto{\pgfqpoint{2.308256in}{3.126569in}}%
\pgfpathlineto{\pgfqpoint{2.315481in}{3.116966in}}%
\pgfpathlineto{\pgfqpoint{2.318693in}{3.116155in}}%
\pgfpathlineto{\pgfqpoint{2.321636in}{3.117916in}}%
\pgfpathlineto{\pgfqpoint{2.325918in}{3.123672in}}%
\pgfpathlineto{\pgfqpoint{2.333411in}{3.133558in}}%
\pgfpathlineto{\pgfqpoint{2.336355in}{3.134285in}}%
\pgfpathlineto{\pgfqpoint{2.339299in}{3.132609in}}%
\pgfpathlineto{\pgfqpoint{2.343580in}{3.126923in}}%
\pgfpathlineto{\pgfqpoint{2.351073in}{3.116935in}}%
\pgfpathlineto{\pgfqpoint{2.354285in}{3.116164in}}%
\pgfpathlineto{\pgfqpoint{2.357228in}{3.117960in}}%
\pgfpathlineto{\pgfqpoint{2.361510in}{3.123744in}}%
\pgfpathlineto{\pgfqpoint{2.368735in}{3.133406in}}%
\pgfpathlineto{\pgfqpoint{2.371947in}{3.134277in}}%
\pgfpathlineto{\pgfqpoint{2.374890in}{3.132566in}}%
\pgfpathlineto{\pgfqpoint{2.379172in}{3.126852in}}%
\pgfpathlineto{\pgfqpoint{2.386665in}{3.116905in}}%
\pgfpathlineto{\pgfqpoint{2.389876in}{3.116174in}}%
\pgfpathlineto{\pgfqpoint{2.392820in}{3.118004in}}%
\pgfpathlineto{\pgfqpoint{2.397102in}{3.123816in}}%
\pgfpathlineto{\pgfqpoint{2.404327in}{3.133438in}}%
\pgfpathlineto{\pgfqpoint{2.407539in}{3.134268in}}%
\pgfpathlineto{\pgfqpoint{2.410482in}{3.132523in}}%
\pgfpathlineto{\pgfqpoint{2.414764in}{3.126780in}}%
\pgfpathlineto{\pgfqpoint{2.422257in}{3.116875in}}%
\pgfpathlineto{\pgfqpoint{2.425468in}{3.116185in}}%
\pgfpathlineto{\pgfqpoint{2.428412in}{3.118049in}}%
\pgfpathlineto{\pgfqpoint{2.432961in}{3.124316in}}%
\pgfpathlineto{\pgfqpoint{2.439919in}{3.133469in}}%
\pgfpathlineto{\pgfqpoint{2.443130in}{3.134259in}}%
\pgfpathlineto{\pgfqpoint{2.446074in}{3.132479in}}%
\pgfpathlineto{\pgfqpoint{2.450356in}{3.126708in}}%
\pgfpathlineto{\pgfqpoint{2.457581in}{3.117027in}}%
\pgfpathlineto{\pgfqpoint{2.460793in}{3.116138in}}%
\pgfpathlineto{\pgfqpoint{2.463736in}{3.117832in}}%
\pgfpathlineto{\pgfqpoint{2.468018in}{3.123534in}}%
\pgfpathlineto{\pgfqpoint{2.475511in}{3.133500in}}%
\pgfpathlineto{\pgfqpoint{2.478722in}{3.134249in}}%
\pgfpathlineto{\pgfqpoint{2.481666in}{3.132435in}}%
\pgfpathlineto{\pgfqpoint{2.485948in}{3.126636in}}%
\pgfpathlineto{\pgfqpoint{2.493173in}{3.116995in}}%
\pgfpathlineto{\pgfqpoint{2.496384in}{3.116147in}}%
\pgfpathlineto{\pgfqpoint{2.499328in}{3.117876in}}%
\pgfpathlineto{\pgfqpoint{2.503610in}{3.123606in}}%
\pgfpathlineto{\pgfqpoint{2.511103in}{3.133530in}}%
\pgfpathlineto{\pgfqpoint{2.514314in}{3.134239in}}%
\pgfpathlineto{\pgfqpoint{2.517258in}{3.132390in}}%
\pgfpathlineto{\pgfqpoint{2.521540in}{3.126564in}}%
\pgfpathlineto{\pgfqpoint{2.528765in}{3.116964in}}%
\pgfpathlineto{\pgfqpoint{2.531976in}{3.116155in}}%
\pgfpathlineto{\pgfqpoint{2.534920in}{3.117919in}}%
\pgfpathlineto{\pgfqpoint{2.539202in}{3.123677in}}%
\pgfpathlineto{\pgfqpoint{2.546695in}{3.133560in}}%
\pgfpathlineto{\pgfqpoint{2.549638in}{3.134284in}}%
\pgfpathlineto{\pgfqpoint{2.552582in}{3.132606in}}%
\pgfpathlineto{\pgfqpoint{2.556864in}{3.126918in}}%
\pgfpathlineto{\pgfqpoint{2.564357in}{3.116933in}}%
\pgfpathlineto{\pgfqpoint{2.567568in}{3.116165in}}%
\pgfpathlineto{\pgfqpoint{2.570512in}{3.117963in}}%
\pgfpathlineto{\pgfqpoint{2.574793in}{3.123749in}}%
\pgfpathlineto{\pgfqpoint{2.582019in}{3.133409in}}%
\pgfpathlineto{\pgfqpoint{2.585230in}{3.134276in}}%
\pgfpathlineto{\pgfqpoint{2.588174in}{3.132563in}}%
\pgfpathlineto{\pgfqpoint{2.592456in}{3.126846in}}%
\pgfpathlineto{\pgfqpoint{2.599949in}{3.116903in}}%
\pgfpathlineto{\pgfqpoint{2.603160in}{3.116175in}}%
\pgfpathlineto{\pgfqpoint{2.606104in}{3.118008in}}%
\pgfpathlineto{\pgfqpoint{2.610385in}{3.123821in}}%
\pgfpathlineto{\pgfqpoint{2.617611in}{3.133440in}}%
\pgfpathlineto{\pgfqpoint{2.620822in}{3.134268in}}%
\pgfpathlineto{\pgfqpoint{2.623766in}{3.132520in}}%
\pgfpathlineto{\pgfqpoint{2.628047in}{3.126775in}}%
\pgfpathlineto{\pgfqpoint{2.635540in}{3.116873in}}%
\pgfpathlineto{\pgfqpoint{2.638752in}{3.116185in}}%
\pgfpathlineto{\pgfqpoint{2.641695in}{3.118053in}}%
\pgfpathlineto{\pgfqpoint{2.646245in}{3.124322in}}%
\pgfpathlineto{\pgfqpoint{2.653203in}{3.133472in}}%
\pgfpathlineto{\pgfqpoint{2.656414in}{3.134258in}}%
\pgfpathlineto{\pgfqpoint{2.659358in}{3.132476in}}%
\pgfpathlineto{\pgfqpoint{2.663639in}{3.126703in}}%
\pgfpathlineto{\pgfqpoint{2.670865in}{3.117025in}}%
\pgfpathlineto{\pgfqpoint{2.674076in}{3.116139in}}%
\pgfpathlineto{\pgfqpoint{2.677020in}{3.117835in}}%
\pgfpathlineto{\pgfqpoint{2.681301in}{3.123539in}}%
\pgfpathlineto{\pgfqpoint{2.688794in}{3.133502in}}%
\pgfpathlineto{\pgfqpoint{2.692006in}{3.134249in}}%
\pgfpathlineto{\pgfqpoint{2.694949in}{3.132432in}}%
\pgfpathlineto{\pgfqpoint{2.699231in}{3.126631in}}%
\pgfpathlineto{\pgfqpoint{2.706457in}{3.116993in}}%
\pgfpathlineto{\pgfqpoint{2.709668in}{3.116147in}}%
\pgfpathlineto{\pgfqpoint{2.712611in}{3.117879in}}%
\pgfpathlineto{\pgfqpoint{2.716893in}{3.123611in}}%
\pgfpathlineto{\pgfqpoint{2.724386in}{3.133532in}}%
\pgfpathlineto{\pgfqpoint{2.727598in}{3.134239in}}%
\pgfpathlineto{\pgfqpoint{2.730541in}{3.132387in}}%
\pgfpathlineto{\pgfqpoint{2.735091in}{3.126131in}}%
\pgfpathlineto{\pgfqpoint{2.742048in}{3.116962in}}%
\pgfpathlineto{\pgfqpoint{2.745260in}{3.116156in}}%
\pgfpathlineto{\pgfqpoint{2.748203in}{3.117922in}}%
\pgfpathlineto{\pgfqpoint{2.752485in}{3.123683in}}%
\pgfpathlineto{\pgfqpoint{2.759978in}{3.133562in}}%
\pgfpathlineto{\pgfqpoint{2.762922in}{3.134283in}}%
\pgfpathlineto{\pgfqpoint{2.765865in}{3.132603in}}%
\pgfpathlineto{\pgfqpoint{2.770147in}{3.126913in}}%
\pgfpathlineto{\pgfqpoint{2.777640in}{3.116931in}}%
\pgfpathlineto{\pgfqpoint{2.780851in}{3.116166in}}%
\pgfpathlineto{\pgfqpoint{2.783795in}{3.117966in}}%
\pgfpathlineto{\pgfqpoint{2.788077in}{3.123754in}}%
\pgfpathlineto{\pgfqpoint{2.795302in}{3.133411in}}%
\pgfpathlineto{\pgfqpoint{2.798514in}{3.134275in}}%
\pgfpathlineto{\pgfqpoint{2.801457in}{3.132560in}}%
\pgfpathlineto{\pgfqpoint{2.805739in}{3.126841in}}%
\pgfpathlineto{\pgfqpoint{2.813232in}{3.116900in}}%
\pgfpathlineto{\pgfqpoint{2.816443in}{3.116176in}}%
\pgfpathlineto{\pgfqpoint{2.819387in}{3.118011in}}%
\pgfpathlineto{\pgfqpoint{2.823669in}{3.123826in}}%
\pgfpathlineto{\pgfqpoint{2.830894in}{3.133443in}}%
\pgfpathlineto{\pgfqpoint{2.834105in}{3.134267in}}%
\pgfpathlineto{\pgfqpoint{2.837049in}{3.132517in}}%
\pgfpathlineto{\pgfqpoint{2.841331in}{3.126770in}}%
\pgfpathlineto{\pgfqpoint{2.848824in}{3.116871in}}%
\pgfpathlineto{\pgfqpoint{2.852035in}{3.116186in}}%
\pgfpathlineto{\pgfqpoint{2.854979in}{3.118056in}}%
\pgfpathlineto{\pgfqpoint{2.859528in}{3.124327in}}%
\pgfpathlineto{\pgfqpoint{2.866486in}{3.133474in}}%
\pgfpathlineto{\pgfqpoint{2.869697in}{3.134258in}}%
\pgfpathlineto{\pgfqpoint{2.872641in}{3.132473in}}%
\pgfpathlineto{\pgfqpoint{2.876923in}{3.126698in}}%
\pgfpathlineto{\pgfqpoint{2.884148in}{3.117023in}}%
\pgfpathlineto{\pgfqpoint{2.887359in}{3.116140in}}%
\pgfpathlineto{\pgfqpoint{2.890303in}{3.117839in}}%
\pgfpathlineto{\pgfqpoint{2.894585in}{3.123544in}}%
\pgfpathlineto{\pgfqpoint{2.902078in}{3.133504in}}%
\pgfpathlineto{\pgfqpoint{2.905289in}{3.134248in}}%
\pgfpathlineto{\pgfqpoint{2.908233in}{3.132429in}}%
\pgfpathlineto{\pgfqpoint{2.912515in}{3.126626in}}%
\pgfpathlineto{\pgfqpoint{2.919740in}{3.116991in}}%
\pgfpathlineto{\pgfqpoint{2.922951in}{3.116148in}}%
\pgfpathlineto{\pgfqpoint{2.925895in}{3.117882in}}%
\pgfpathlineto{\pgfqpoint{2.930177in}{3.123616in}}%
\pgfpathlineto{\pgfqpoint{2.937670in}{3.133534in}}%
\pgfpathlineto{\pgfqpoint{2.940881in}{3.134238in}}%
\pgfpathlineto{\pgfqpoint{2.943825in}{3.132384in}}%
\pgfpathlineto{\pgfqpoint{2.948374in}{3.126126in}}%
\pgfpathlineto{\pgfqpoint{2.955332in}{3.116959in}}%
\pgfpathlineto{\pgfqpoint{2.958543in}{3.116157in}}%
\pgfpathlineto{\pgfqpoint{2.961487in}{3.117925in}}%
\pgfpathlineto{\pgfqpoint{2.965768in}{3.123688in}}%
\pgfpathlineto{\pgfqpoint{2.973262in}{3.133564in}}%
\pgfpathlineto{\pgfqpoint{2.976205in}{3.134283in}}%
\pgfpathlineto{\pgfqpoint{2.979149in}{3.132600in}}%
\pgfpathlineto{\pgfqpoint{2.983431in}{3.126908in}}%
\pgfpathlineto{\pgfqpoint{2.990924in}{3.116929in}}%
\pgfpathlineto{\pgfqpoint{2.994135in}{3.116166in}}%
\pgfpathlineto{\pgfqpoint{2.997079in}{3.117969in}}%
\pgfpathlineto{\pgfqpoint{3.001360in}{3.123759in}}%
\pgfpathlineto{\pgfqpoint{3.008586in}{3.133413in}}%
\pgfpathlineto{\pgfqpoint{3.011797in}{3.134275in}}%
\pgfpathlineto{\pgfqpoint{3.014741in}{3.132557in}}%
\pgfpathlineto{\pgfqpoint{3.019022in}{3.126836in}}%
\pgfpathlineto{\pgfqpoint{3.026515in}{3.116898in}}%
\pgfpathlineto{\pgfqpoint{3.029727in}{3.116176in}}%
\pgfpathlineto{\pgfqpoint{3.032670in}{3.118014in}}%
\pgfpathlineto{\pgfqpoint{3.036952in}{3.123831in}}%
\pgfpathlineto{\pgfqpoint{3.044178in}{3.133445in}}%
\pgfpathlineto{\pgfqpoint{3.047389in}{3.134266in}}%
\pgfpathlineto{\pgfqpoint{3.050333in}{3.132514in}}%
\pgfpathlineto{\pgfqpoint{3.054614in}{3.126765in}}%
\pgfpathlineto{\pgfqpoint{3.062107in}{3.116868in}}%
\pgfpathlineto{\pgfqpoint{3.065319in}{3.116187in}}%
\pgfpathlineto{\pgfqpoint{3.068262in}{3.118059in}}%
\pgfpathlineto{\pgfqpoint{3.072812in}{3.124332in}}%
\pgfpathlineto{\pgfqpoint{3.079769in}{3.133476in}}%
\pgfpathlineto{\pgfqpoint{3.082981in}{3.134257in}}%
\pgfpathlineto{\pgfqpoint{3.085924in}{3.132470in}}%
\pgfpathlineto{\pgfqpoint{3.090206in}{3.126693in}}%
\pgfpathlineto{\pgfqpoint{3.097432in}{3.117020in}}%
\pgfpathlineto{\pgfqpoint{3.100643in}{3.116140in}}%
\pgfpathlineto{\pgfqpoint{3.103587in}{3.117842in}}%
\pgfpathlineto{\pgfqpoint{3.107868in}{3.123549in}}%
\pgfpathlineto{\pgfqpoint{3.115361in}{3.133507in}}%
\pgfpathlineto{\pgfqpoint{3.118573in}{3.134247in}}%
\pgfpathlineto{\pgfqpoint{3.121516in}{3.132426in}}%
\pgfpathlineto{\pgfqpoint{3.125798in}{3.126621in}}%
\pgfpathlineto{\pgfqpoint{3.133023in}{3.116989in}}%
\pgfpathlineto{\pgfqpoint{3.136235in}{3.116149in}}%
\pgfpathlineto{\pgfqpoint{3.139178in}{3.117885in}}%
\pgfpathlineto{\pgfqpoint{3.143460in}{3.123621in}}%
\pgfpathlineto{\pgfqpoint{3.150953in}{3.133537in}}%
\pgfpathlineto{\pgfqpoint{3.154164in}{3.134237in}}%
\pgfpathlineto{\pgfqpoint{3.157108in}{3.132381in}}%
\pgfpathlineto{\pgfqpoint{3.161657in}{3.126121in}}%
\pgfpathlineto{\pgfqpoint{3.168615in}{3.116957in}}%
\pgfpathlineto{\pgfqpoint{3.171827in}{3.116157in}}%
\pgfpathlineto{\pgfqpoint{3.174770in}{3.117928in}}%
\pgfpathlineto{\pgfqpoint{3.179052in}{3.123693in}}%
\pgfpathlineto{\pgfqpoint{3.186545in}{3.133566in}}%
\pgfpathlineto{\pgfqpoint{3.189489in}{3.134282in}}%
\pgfpathlineto{\pgfqpoint{3.192432in}{3.132597in}}%
\pgfpathlineto{\pgfqpoint{3.196714in}{3.126903in}}%
\pgfpathlineto{\pgfqpoint{3.204207in}{3.116926in}}%
\pgfpathlineto{\pgfqpoint{3.207418in}{3.116167in}}%
\pgfpathlineto{\pgfqpoint{3.210362in}{3.117973in}}%
\pgfpathlineto{\pgfqpoint{3.214644in}{3.123765in}}%
\pgfpathlineto{\pgfqpoint{3.221869in}{3.133416in}}%
\pgfpathlineto{\pgfqpoint{3.225080in}{3.134274in}}%
\pgfpathlineto{\pgfqpoint{3.228024in}{3.132554in}}%
\pgfpathlineto{\pgfqpoint{3.232306in}{3.126831in}}%
\pgfpathlineto{\pgfqpoint{3.239799in}{3.116896in}}%
\pgfpathlineto{\pgfqpoint{3.243010in}{3.116177in}}%
\pgfpathlineto{\pgfqpoint{3.245954in}{3.118017in}}%
\pgfpathlineto{\pgfqpoint{3.250236in}{3.123837in}}%
\pgfpathlineto{\pgfqpoint{3.257461in}{3.133447in}}%
\pgfpathlineto{\pgfqpoint{3.260672in}{3.134266in}}%
\pgfpathlineto{\pgfqpoint{3.263616in}{3.132511in}}%
\pgfpathlineto{\pgfqpoint{3.267898in}{3.126759in}}%
\pgfpathlineto{\pgfqpoint{3.275391in}{3.116866in}}%
\pgfpathlineto{\pgfqpoint{3.278334in}{3.116133in}}%
\pgfpathlineto{\pgfqpoint{3.281278in}{3.117802in}}%
\pgfpathlineto{\pgfqpoint{3.285560in}{3.123483in}}%
\pgfpathlineto{\pgfqpoint{3.293053in}{3.133478in}}%
\pgfpathlineto{\pgfqpoint{3.296264in}{3.134256in}}%
\pgfpathlineto{\pgfqpoint{3.299208in}{3.132467in}}%
\pgfpathlineto{\pgfqpoint{3.303490in}{3.126688in}}%
\pgfpathlineto{\pgfqpoint{3.310715in}{3.117018in}}%
\pgfpathlineto{\pgfqpoint{3.313926in}{3.116141in}}%
\pgfpathlineto{\pgfqpoint{3.316870in}{3.117845in}}%
\pgfpathlineto{\pgfqpoint{3.321152in}{3.123555in}}%
\pgfpathlineto{\pgfqpoint{3.328645in}{3.133509in}}%
\pgfpathlineto{\pgfqpoint{3.331856in}{3.134247in}}%
\pgfpathlineto{\pgfqpoint{3.334800in}{3.132422in}}%
\pgfpathlineto{\pgfqpoint{3.339081in}{3.126616in}}%
\pgfpathlineto{\pgfqpoint{3.346307in}{3.116986in}}%
\pgfpathlineto{\pgfqpoint{3.349518in}{3.116149in}}%
\pgfpathlineto{\pgfqpoint{3.352462in}{3.117888in}}%
\pgfpathlineto{\pgfqpoint{3.356744in}{3.123626in}}%
\pgfpathlineto{\pgfqpoint{3.364237in}{3.133539in}}%
\pgfpathlineto{\pgfqpoint{3.367448in}{3.134236in}}%
\pgfpathlineto{\pgfqpoint{3.370392in}{3.132378in}}%
\pgfpathlineto{\pgfqpoint{3.374941in}{3.126115in}}%
\pgfpathlineto{\pgfqpoint{3.381899in}{3.116955in}}%
\pgfpathlineto{\pgfqpoint{3.385110in}{3.116158in}}%
\pgfpathlineto{\pgfqpoint{3.388054in}{3.117932in}}%
\pgfpathlineto{\pgfqpoint{3.392335in}{3.123698in}}%
\pgfpathlineto{\pgfqpoint{3.399828in}{3.133568in}}%
\pgfpathlineto{\pgfqpoint{3.402772in}{3.134282in}}%
\pgfpathlineto{\pgfqpoint{3.405716in}{3.132594in}}%
\pgfpathlineto{\pgfqpoint{3.409997in}{3.126898in}}%
\pgfpathlineto{\pgfqpoint{3.417490in}{3.116924in}}%
\pgfpathlineto{\pgfqpoint{3.420702in}{3.116168in}}%
\pgfpathlineto{\pgfqpoint{3.423645in}{3.117976in}}%
\pgfpathlineto{\pgfqpoint{3.427927in}{3.123770in}}%
\pgfpathlineto{\pgfqpoint{3.435153in}{3.133418in}}%
\pgfpathlineto{\pgfqpoint{3.438364in}{3.134274in}}%
\pgfpathlineto{\pgfqpoint{3.441308in}{3.132551in}}%
\pgfpathlineto{\pgfqpoint{3.445589in}{3.126826in}}%
\pgfpathlineto{\pgfqpoint{3.453082in}{3.116894in}}%
\pgfpathlineto{\pgfqpoint{3.456294in}{3.116178in}}%
\pgfpathlineto{\pgfqpoint{3.459237in}{3.118020in}}%
\pgfpathlineto{\pgfqpoint{3.463519in}{3.123842in}}%
\pgfpathlineto{\pgfqpoint{3.470744in}{3.133449in}}%
\pgfpathlineto{\pgfqpoint{3.473956in}{3.134265in}}%
\pgfpathlineto{\pgfqpoint{3.476899in}{3.132508in}}%
\pgfpathlineto{\pgfqpoint{3.481181in}{3.126754in}}%
\pgfpathlineto{\pgfqpoint{3.488674in}{3.116864in}}%
\pgfpathlineto{\pgfqpoint{3.491618in}{3.116134in}}%
\pgfpathlineto{\pgfqpoint{3.494562in}{3.117805in}}%
\pgfpathlineto{\pgfqpoint{3.498843in}{3.123488in}}%
\pgfpathlineto{\pgfqpoint{3.506336in}{3.133480in}}%
\pgfpathlineto{\pgfqpoint{3.509548in}{3.134256in}}%
\pgfpathlineto{\pgfqpoint{3.512491in}{3.132464in}}%
\pgfpathlineto{\pgfqpoint{3.516773in}{3.126682in}}%
\pgfpathlineto{\pgfqpoint{3.523998in}{3.117016in}}%
\pgfpathlineto{\pgfqpoint{3.527210in}{3.116141in}}%
\pgfpathlineto{\pgfqpoint{3.530153in}{3.117848in}}%
\pgfpathlineto{\pgfqpoint{3.534435in}{3.123560in}}%
\pgfpathlineto{\pgfqpoint{3.541928in}{3.133511in}}%
\pgfpathlineto{\pgfqpoint{3.545139in}{3.134246in}}%
\pgfpathlineto{\pgfqpoint{3.548083in}{3.132419in}}%
\pgfpathlineto{\pgfqpoint{3.552365in}{3.126611in}}%
\pgfpathlineto{\pgfqpoint{3.559590in}{3.116984in}}%
\pgfpathlineto{\pgfqpoint{3.562802in}{3.116150in}}%
\pgfpathlineto{\pgfqpoint{3.565745in}{3.117891in}}%
\pgfpathlineto{\pgfqpoint{3.570027in}{3.123631in}}%
\pgfpathlineto{\pgfqpoint{3.577520in}{3.133541in}}%
\pgfpathlineto{\pgfqpoint{3.580731in}{3.134235in}}%
\pgfpathlineto{\pgfqpoint{3.583675in}{3.132374in}}%
\pgfpathlineto{\pgfqpoint{3.588224in}{3.126110in}}%
\pgfpathlineto{\pgfqpoint{3.595182in}{3.116953in}}%
\pgfpathlineto{\pgfqpoint{3.598393in}{3.116159in}}%
\pgfpathlineto{\pgfqpoint{3.601337in}{3.117935in}}%
\pgfpathlineto{\pgfqpoint{3.605619in}{3.123703in}}%
\pgfpathlineto{\pgfqpoint{3.612844in}{3.133388in}}%
\pgfpathlineto{\pgfqpoint{3.616055in}{3.134281in}}%
\pgfpathlineto{\pgfqpoint{3.618999in}{3.132591in}}%
\pgfpathlineto{\pgfqpoint{3.623281in}{3.126892in}}%
\pgfpathlineto{\pgfqpoint{3.630774in}{3.116922in}}%
\pgfpathlineto{\pgfqpoint{3.633985in}{3.116168in}}%
\pgfpathlineto{\pgfqpoint{3.636929in}{3.117979in}}%
\pgfpathlineto{\pgfqpoint{3.641211in}{3.123775in}}%
\pgfpathlineto{\pgfqpoint{3.648436in}{3.133420in}}%
\pgfpathlineto{\pgfqpoint{3.651647in}{3.134273in}}%
\pgfpathlineto{\pgfqpoint{3.654591in}{3.132548in}}%
\pgfpathlineto{\pgfqpoint{3.658873in}{3.126821in}}%
\pgfpathlineto{\pgfqpoint{3.666366in}{3.116892in}}%
\pgfpathlineto{\pgfqpoint{3.669577in}{3.116178in}}%
\pgfpathlineto{\pgfqpoint{3.672521in}{3.118024in}}%
\pgfpathlineto{\pgfqpoint{3.676802in}{3.123847in}}%
\pgfpathlineto{\pgfqpoint{3.684028in}{3.133452in}}%
\pgfpathlineto{\pgfqpoint{3.687239in}{3.134264in}}%
\pgfpathlineto{\pgfqpoint{3.690183in}{3.132504in}}%
\pgfpathlineto{\pgfqpoint{3.694465in}{3.126749in}}%
\pgfpathlineto{\pgfqpoint{3.701958in}{3.116862in}}%
\pgfpathlineto{\pgfqpoint{3.704901in}{3.116134in}}%
\pgfpathlineto{\pgfqpoint{3.707845in}{3.117808in}}%
\pgfpathlineto{\pgfqpoint{3.712127in}{3.123493in}}%
\pgfpathlineto{\pgfqpoint{3.719620in}{3.133483in}}%
\pgfpathlineto{\pgfqpoint{3.722831in}{3.134255in}}%
\pgfpathlineto{\pgfqpoint{3.725775in}{3.132461in}}%
\pgfpathlineto{\pgfqpoint{3.730056in}{3.126677in}}%
\pgfpathlineto{\pgfqpoint{3.737282in}{3.117014in}}%
\pgfpathlineto{\pgfqpoint{3.740493in}{3.116142in}}%
\pgfpathlineto{\pgfqpoint{3.743437in}{3.117851in}}%
\pgfpathlineto{\pgfqpoint{3.747719in}{3.123565in}}%
\pgfpathlineto{\pgfqpoint{3.755212in}{3.133513in}}%
\pgfpathlineto{\pgfqpoint{3.758423in}{3.134245in}}%
\pgfpathlineto{\pgfqpoint{3.761367in}{3.132416in}}%
\pgfpathlineto{\pgfqpoint{3.765648in}{3.126605in}}%
\pgfpathlineto{\pgfqpoint{3.772874in}{3.116982in}}%
\pgfpathlineto{\pgfqpoint{3.776085in}{3.116150in}}%
\pgfpathlineto{\pgfqpoint{3.779029in}{3.117894in}}%
\pgfpathlineto{\pgfqpoint{3.783310in}{3.123636in}}%
\pgfpathlineto{\pgfqpoint{3.790803in}{3.133543in}}%
\pgfpathlineto{\pgfqpoint{3.794015in}{3.134235in}}%
\pgfpathlineto{\pgfqpoint{3.796958in}{3.132371in}}%
\pgfpathlineto{\pgfqpoint{3.801508in}{3.126105in}}%
\pgfpathlineto{\pgfqpoint{3.808466in}{3.116951in}}%
\pgfpathlineto{\pgfqpoint{3.811677in}{3.116159in}}%
\pgfpathlineto{\pgfqpoint{3.814621in}{3.117938in}}%
\pgfpathlineto{\pgfqpoint{3.818902in}{3.123708in}}%
\pgfpathlineto{\pgfqpoint{3.826128in}{3.133390in}}%
\pgfpathlineto{\pgfqpoint{3.829339in}{3.134281in}}%
\pgfpathlineto{\pgfqpoint{3.832283in}{3.132588in}}%
\pgfpathlineto{\pgfqpoint{3.836564in}{3.126887in}}%
\pgfpathlineto{\pgfqpoint{3.844057in}{3.116920in}}%
\pgfpathlineto{\pgfqpoint{3.847269in}{3.116169in}}%
\pgfpathlineto{\pgfqpoint{3.850212in}{3.117982in}}%
\pgfpathlineto{\pgfqpoint{3.854494in}{3.123780in}}%
\pgfpathlineto{\pgfqpoint{3.861719in}{3.133422in}}%
\pgfpathlineto{\pgfqpoint{3.864931in}{3.134272in}}%
\pgfpathlineto{\pgfqpoint{3.867874in}{3.132545in}}%
\pgfpathlineto{\pgfqpoint{3.872156in}{3.126816in}}%
\pgfpathlineto{\pgfqpoint{3.879649in}{3.116890in}}%
\pgfpathlineto{\pgfqpoint{3.882860in}{3.116179in}}%
\pgfpathlineto{\pgfqpoint{3.885804in}{3.118027in}}%
\pgfpathlineto{\pgfqpoint{3.890086in}{3.123852in}}%
\pgfpathlineto{\pgfqpoint{3.897311in}{3.133454in}}%
\pgfpathlineto{\pgfqpoint{3.900523in}{3.134264in}}%
\pgfpathlineto{\pgfqpoint{3.903466in}{3.132501in}}%
\pgfpathlineto{\pgfqpoint{3.907748in}{3.126744in}}%
\pgfpathlineto{\pgfqpoint{3.915241in}{3.116860in}}%
\pgfpathlineto{\pgfqpoint{3.918185in}{3.116135in}}%
\pgfpathlineto{\pgfqpoint{3.921128in}{3.117811in}}%
\pgfpathlineto{\pgfqpoint{3.925410in}{3.123498in}}%
\pgfpathlineto{\pgfqpoint{3.932903in}{3.133485in}}%
\pgfpathlineto{\pgfqpoint{3.936114in}{3.134254in}}%
\pgfpathlineto{\pgfqpoint{3.939058in}{3.132457in}}%
\pgfpathlineto{\pgfqpoint{3.943340in}{3.126672in}}%
\pgfpathlineto{\pgfqpoint{3.950565in}{3.117011in}}%
\pgfpathlineto{\pgfqpoint{3.953777in}{3.116143in}}%
\pgfpathlineto{\pgfqpoint{3.956720in}{3.117854in}}%
\pgfpathlineto{\pgfqpoint{3.961002in}{3.123570in}}%
\pgfpathlineto{\pgfqpoint{3.968495in}{3.133515in}}%
\pgfpathlineto{\pgfqpoint{3.971706in}{3.134244in}}%
\pgfpathlineto{\pgfqpoint{3.974650in}{3.132413in}}%
\pgfpathlineto{\pgfqpoint{3.978932in}{3.126600in}}%
\pgfpathlineto{\pgfqpoint{3.986157in}{3.116980in}}%
\pgfpathlineto{\pgfqpoint{3.989368in}{3.116151in}}%
\pgfpathlineto{\pgfqpoint{3.992312in}{3.117897in}}%
\pgfpathlineto{\pgfqpoint{3.996594in}{3.123642in}}%
\pgfpathlineto{\pgfqpoint{4.004087in}{3.133545in}}%
\pgfpathlineto{\pgfqpoint{4.007298in}{3.134234in}}%
\pgfpathlineto{\pgfqpoint{4.010242in}{3.132368in}}%
\pgfpathlineto{\pgfqpoint{4.014791in}{3.126100in}}%
\pgfpathlineto{\pgfqpoint{4.021749in}{3.116948in}}%
\pgfpathlineto{\pgfqpoint{4.024960in}{3.116160in}}%
\pgfpathlineto{\pgfqpoint{4.027904in}{3.117941in}}%
\pgfpathlineto{\pgfqpoint{4.032186in}{3.123713in}}%
\pgfpathlineto{\pgfqpoint{4.039411in}{3.133393in}}%
\pgfpathlineto{\pgfqpoint{4.042622in}{3.134280in}}%
\pgfpathlineto{\pgfqpoint{4.045566in}{3.132585in}}%
\pgfpathlineto{\pgfqpoint{4.049848in}{3.126882in}}%
\pgfpathlineto{\pgfqpoint{4.057341in}{3.116918in}}%
\pgfpathlineto{\pgfqpoint{4.060552in}{3.116170in}}%
\pgfpathlineto{\pgfqpoint{4.063496in}{3.117985in}}%
\pgfpathlineto{\pgfqpoint{4.067778in}{3.123785in}}%
\pgfpathlineto{\pgfqpoint{4.075003in}{3.133425in}}%
\pgfpathlineto{\pgfqpoint{4.078214in}{3.134272in}}%
\pgfpathlineto{\pgfqpoint{4.081158in}{3.132542in}}%
\pgfpathlineto{\pgfqpoint{4.085440in}{3.126811in}}%
\pgfpathlineto{\pgfqpoint{4.092933in}{3.116888in}}%
\pgfpathlineto{\pgfqpoint{4.096144in}{3.116180in}}%
\pgfpathlineto{\pgfqpoint{4.099088in}{3.118030in}}%
\pgfpathlineto{\pgfqpoint{4.103369in}{3.123857in}}%
\pgfpathlineto{\pgfqpoint{4.110595in}{3.133456in}}%
\pgfpathlineto{\pgfqpoint{4.113806in}{3.134263in}}%
\pgfpathlineto{\pgfqpoint{4.116750in}{3.132498in}}%
\pgfpathlineto{\pgfqpoint{4.121031in}{3.126739in}}%
\pgfpathlineto{\pgfqpoint{4.128524in}{3.116858in}}%
\pgfpathlineto{\pgfqpoint{4.131468in}{3.116135in}}%
\pgfpathlineto{\pgfqpoint{4.134412in}{3.117814in}}%
\pgfpathlineto{\pgfqpoint{4.138694in}{3.123503in}}%
\pgfpathlineto{\pgfqpoint{4.146187in}{3.133487in}}%
\pgfpathlineto{\pgfqpoint{4.149398in}{3.134254in}}%
\pgfpathlineto{\pgfqpoint{4.152342in}{3.132454in}}%
\pgfpathlineto{\pgfqpoint{4.156623in}{3.126667in}}%
\pgfpathlineto{\pgfqpoint{4.163849in}{3.117009in}}%
\pgfpathlineto{\pgfqpoint{4.167060in}{3.116143in}}%
\pgfpathlineto{\pgfqpoint{4.170004in}{3.117857in}}%
\pgfpathlineto{\pgfqpoint{4.174285in}{3.123575in}}%
\pgfpathlineto{\pgfqpoint{4.181778in}{3.133517in}}%
\pgfpathlineto{\pgfqpoint{4.184990in}{3.134244in}}%
\pgfpathlineto{\pgfqpoint{4.187933in}{3.132410in}}%
\pgfpathlineto{\pgfqpoint{4.192215in}{3.126595in}}%
\pgfpathlineto{\pgfqpoint{4.199441in}{3.116977in}}%
\pgfpathlineto{\pgfqpoint{4.202652in}{3.116152in}}%
\pgfpathlineto{\pgfqpoint{4.205596in}{3.117900in}}%
\pgfpathlineto{\pgfqpoint{4.209877in}{3.123647in}}%
\pgfpathlineto{\pgfqpoint{4.217370in}{3.133547in}}%
\pgfpathlineto{\pgfqpoint{4.220582in}{3.134233in}}%
\pgfpathlineto{\pgfqpoint{4.223525in}{3.132365in}}%
\pgfpathlineto{\pgfqpoint{4.228075in}{3.126095in}}%
\pgfpathlineto{\pgfqpoint{4.235032in}{3.116946in}}%
\pgfpathlineto{\pgfqpoint{4.238244in}{3.116161in}}%
\pgfpathlineto{\pgfqpoint{4.241187in}{3.117944in}}%
\pgfpathlineto{\pgfqpoint{4.245469in}{3.123718in}}%
\pgfpathlineto{\pgfqpoint{4.252695in}{3.133395in}}%
\pgfpathlineto{\pgfqpoint{4.255906in}{3.134280in}}%
\pgfpathlineto{\pgfqpoint{4.258849in}{3.132582in}}%
\pgfpathlineto{\pgfqpoint{4.263131in}{3.126877in}}%
\pgfpathlineto{\pgfqpoint{4.270624in}{3.116916in}}%
\pgfpathlineto{\pgfqpoint{4.273836in}{3.116170in}}%
\pgfpathlineto{\pgfqpoint{4.276779in}{3.117988in}}%
\pgfpathlineto{\pgfqpoint{4.281061in}{3.123790in}}%
\pgfpathlineto{\pgfqpoint{4.288286in}{3.133427in}}%
\pgfpathlineto{\pgfqpoint{4.291498in}{3.134271in}}%
\pgfpathlineto{\pgfqpoint{4.294441in}{3.132539in}}%
\pgfpathlineto{\pgfqpoint{4.298723in}{3.126806in}}%
\pgfpathlineto{\pgfqpoint{4.306216in}{3.116885in}}%
\pgfpathlineto{\pgfqpoint{4.309427in}{3.116181in}}%
\pgfpathlineto{\pgfqpoint{4.312371in}{3.118033in}}%
\pgfpathlineto{\pgfqpoint{4.316920in}{3.124291in}}%
\pgfpathlineto{\pgfqpoint{4.323878in}{3.133458in}}%
\pgfpathlineto{\pgfqpoint{4.327089in}{3.134262in}}%
\pgfpathlineto{\pgfqpoint{4.330033in}{3.132495in}}%
\pgfpathlineto{\pgfqpoint{4.334315in}{3.126734in}}%
\pgfpathlineto{\pgfqpoint{4.341808in}{3.116856in}}%
\pgfpathlineto{\pgfqpoint{4.344752in}{3.116136in}}%
\pgfpathlineto{\pgfqpoint{4.347695in}{3.117817in}}%
\pgfpathlineto{\pgfqpoint{4.351977in}{3.123509in}}%
\pgfpathlineto{\pgfqpoint{4.359470in}{3.133489in}}%
\pgfpathlineto{\pgfqpoint{4.362681in}{3.134253in}}%
\pgfpathlineto{\pgfqpoint{4.365625in}{3.132451in}}%
\pgfpathlineto{\pgfqpoint{4.369907in}{3.126662in}}%
\pgfpathlineto{\pgfqpoint{4.377132in}{3.117007in}}%
\pgfpathlineto{\pgfqpoint{4.380343in}{3.116144in}}%
\pgfpathlineto{\pgfqpoint{4.383287in}{3.117860in}}%
\pgfpathlineto{\pgfqpoint{4.387569in}{3.123580in}}%
\pgfpathlineto{\pgfqpoint{4.395062in}{3.133520in}}%
\pgfpathlineto{\pgfqpoint{4.398273in}{3.134243in}}%
\pgfpathlineto{\pgfqpoint{4.401217in}{3.132406in}}%
\pgfpathlineto{\pgfqpoint{4.405499in}{3.126590in}}%
\pgfpathlineto{\pgfqpoint{4.412724in}{3.116975in}}%
\pgfpathlineto{\pgfqpoint{4.415935in}{3.116152in}}%
\pgfpathlineto{\pgfqpoint{4.418879in}{3.117903in}}%
\pgfpathlineto{\pgfqpoint{4.423161in}{3.123652in}}%
\pgfpathlineto{\pgfqpoint{4.430654in}{3.133549in}}%
\pgfpathlineto{\pgfqpoint{4.433865in}{3.134232in}}%
\pgfpathlineto{\pgfqpoint{4.436809in}{3.132362in}}%
\pgfpathlineto{\pgfqpoint{4.441358in}{3.126090in}}%
\pgfpathlineto{\pgfqpoint{4.448316in}{3.116944in}}%
\pgfpathlineto{\pgfqpoint{4.451527in}{3.116161in}}%
\pgfpathlineto{\pgfqpoint{4.454471in}{3.117947in}}%
\pgfpathlineto{\pgfqpoint{4.458753in}{3.123724in}}%
\pgfpathlineto{\pgfqpoint{4.465978in}{3.133397in}}%
\pgfpathlineto{\pgfqpoint{4.469189in}{3.134279in}}%
\pgfpathlineto{\pgfqpoint{4.472133in}{3.132579in}}%
\pgfpathlineto{\pgfqpoint{4.476415in}{3.126872in}}%
\pgfpathlineto{\pgfqpoint{4.483908in}{3.116913in}}%
\pgfpathlineto{\pgfqpoint{4.487119in}{3.116171in}}%
\pgfpathlineto{\pgfqpoint{4.490063in}{3.117992in}}%
\pgfpathlineto{\pgfqpoint{4.494344in}{3.123795in}}%
\pgfpathlineto{\pgfqpoint{4.501570in}{3.133429in}}%
\pgfpathlineto{\pgfqpoint{4.504781in}{3.134271in}}%
\pgfpathlineto{\pgfqpoint{4.507725in}{3.132536in}}%
\pgfpathlineto{\pgfqpoint{4.512006in}{3.126800in}}%
\pgfpathlineto{\pgfqpoint{4.519500in}{3.116883in}}%
\pgfpathlineto{\pgfqpoint{4.522711in}{3.116181in}}%
\pgfpathlineto{\pgfqpoint{4.525654in}{3.118036in}}%
\pgfpathlineto{\pgfqpoint{4.530204in}{3.124296in}}%
\pgfpathlineto{\pgfqpoint{4.537162in}{3.133461in}}%
\pgfpathlineto{\pgfqpoint{4.540373in}{3.134262in}}%
\pgfpathlineto{\pgfqpoint{4.543317in}{3.132492in}}%
\pgfpathlineto{\pgfqpoint{4.547598in}{3.126729in}}%
\pgfpathlineto{\pgfqpoint{4.555091in}{3.116854in}}%
\pgfpathlineto{\pgfqpoint{4.558035in}{3.116136in}}%
\pgfpathlineto{\pgfqpoint{4.560979in}{3.117820in}}%
\pgfpathlineto{\pgfqpoint{4.565260in}{3.123514in}}%
\pgfpathlineto{\pgfqpoint{4.572753in}{3.133491in}}%
\pgfpathlineto{\pgfqpoint{4.575965in}{3.134252in}}%
\pgfpathlineto{\pgfqpoint{4.578908in}{3.132448in}}%
\pgfpathlineto{\pgfqpoint{4.583190in}{3.126657in}}%
\pgfpathlineto{\pgfqpoint{4.590416in}{3.117004in}}%
\pgfpathlineto{\pgfqpoint{4.593627in}{3.116144in}}%
\pgfpathlineto{\pgfqpoint{4.596571in}{3.117863in}}%
\pgfpathlineto{\pgfqpoint{4.600852in}{3.123585in}}%
\pgfpathlineto{\pgfqpoint{4.608345in}{3.133522in}}%
\pgfpathlineto{\pgfqpoint{4.611557in}{3.134242in}}%
\pgfpathlineto{\pgfqpoint{4.614500in}{3.132403in}}%
\pgfpathlineto{\pgfqpoint{4.618782in}{3.126585in}}%
\pgfpathlineto{\pgfqpoint{4.626007in}{3.116973in}}%
\pgfpathlineto{\pgfqpoint{4.629219in}{3.116153in}}%
\pgfpathlineto{\pgfqpoint{4.632162in}{3.117907in}}%
\pgfpathlineto{\pgfqpoint{4.636444in}{3.123657in}}%
\pgfpathlineto{\pgfqpoint{4.643937in}{3.133551in}}%
\pgfpathlineto{\pgfqpoint{4.647148in}{3.134232in}}%
\pgfpathlineto{\pgfqpoint{4.650092in}{3.132358in}}%
\pgfpathlineto{\pgfqpoint{4.654641in}{3.126084in}}%
\pgfpathlineto{\pgfqpoint{4.661599in}{3.116942in}}%
\pgfpathlineto{\pgfqpoint{4.664811in}{3.116162in}}%
\pgfpathlineto{\pgfqpoint{4.667754in}{3.117950in}}%
\pgfpathlineto{\pgfqpoint{4.672036in}{3.123729in}}%
\pgfpathlineto{\pgfqpoint{4.679261in}{3.133400in}}%
\pgfpathlineto{\pgfqpoint{4.682473in}{3.134278in}}%
\pgfpathlineto{\pgfqpoint{4.685416in}{3.132576in}}%
\pgfpathlineto{\pgfqpoint{4.689698in}{3.126867in}}%
\pgfpathlineto{\pgfqpoint{4.697191in}{3.116911in}}%
\pgfpathlineto{\pgfqpoint{4.700402in}{3.116172in}}%
\pgfpathlineto{\pgfqpoint{4.703346in}{3.117995in}}%
\pgfpathlineto{\pgfqpoint{4.707628in}{3.123801in}}%
\pgfpathlineto{\pgfqpoint{4.714853in}{3.133431in}}%
\pgfpathlineto{\pgfqpoint{4.718065in}{3.134270in}}%
\pgfpathlineto{\pgfqpoint{4.721008in}{3.132532in}}%
\pgfpathlineto{\pgfqpoint{4.725290in}{3.126795in}}%
\pgfpathlineto{\pgfqpoint{4.732783in}{3.116881in}}%
\pgfpathlineto{\pgfqpoint{4.735994in}{3.116182in}}%
\pgfpathlineto{\pgfqpoint{4.738938in}{3.118040in}}%
\pgfpathlineto{\pgfqpoint{4.743487in}{3.124301in}}%
\pgfpathlineto{\pgfqpoint{4.750445in}{3.133463in}}%
\pgfpathlineto{\pgfqpoint{4.753656in}{3.134261in}}%
\pgfpathlineto{\pgfqpoint{4.756600in}{3.132489in}}%
\pgfpathlineto{\pgfqpoint{4.760882in}{3.126724in}}%
\pgfpathlineto{\pgfqpoint{4.768375in}{3.116852in}}%
\pgfpathlineto{\pgfqpoint{4.771318in}{3.116137in}}%
\pgfpathlineto{\pgfqpoint{4.774262in}{3.117823in}}%
\pgfpathlineto{\pgfqpoint{4.778544in}{3.123519in}}%
\pgfpathlineto{\pgfqpoint{4.786037in}{3.133494in}}%
\pgfpathlineto{\pgfqpoint{4.789248in}{3.134252in}}%
\pgfpathlineto{\pgfqpoint{4.792192in}{3.132445in}}%
\pgfpathlineto{\pgfqpoint{4.796474in}{3.126652in}}%
\pgfpathlineto{\pgfqpoint{4.803699in}{3.117002in}}%
\pgfpathlineto{\pgfqpoint{4.806910in}{3.116145in}}%
\pgfpathlineto{\pgfqpoint{4.809854in}{3.117866in}}%
\pgfpathlineto{\pgfqpoint{4.814136in}{3.123590in}}%
\pgfpathlineto{\pgfqpoint{4.821629in}{3.133524in}}%
\pgfpathlineto{\pgfqpoint{4.824840in}{3.134242in}}%
\pgfpathlineto{\pgfqpoint{4.827784in}{3.132400in}}%
\pgfpathlineto{\pgfqpoint{4.832065in}{3.126580in}}%
\pgfpathlineto{\pgfqpoint{4.839291in}{3.116971in}}%
\pgfpathlineto{\pgfqpoint{4.842502in}{3.116154in}}%
\pgfpathlineto{\pgfqpoint{4.845446in}{3.117910in}}%
\pgfpathlineto{\pgfqpoint{4.849728in}{3.123662in}}%
\pgfpathlineto{\pgfqpoint{4.857221in}{3.133554in}}%
\pgfpathlineto{\pgfqpoint{4.860164in}{3.134286in}}%
\pgfpathlineto{\pgfqpoint{4.863108in}{3.132615in}}%
\pgfpathlineto{\pgfqpoint{4.867390in}{3.126933in}}%
\pgfpathlineto{\pgfqpoint{4.874883in}{3.116940in}}%
\pgfpathlineto{\pgfqpoint{4.878094in}{3.116163in}}%
\pgfpathlineto{\pgfqpoint{4.881038in}{3.117954in}}%
\pgfpathlineto{\pgfqpoint{4.885319in}{3.123734in}}%
\pgfpathlineto{\pgfqpoint{4.892545in}{3.133402in}}%
\pgfpathlineto{\pgfqpoint{4.895756in}{3.134278in}}%
\pgfpathlineto{\pgfqpoint{4.898700in}{3.132573in}}%
\pgfpathlineto{\pgfqpoint{4.902982in}{3.126862in}}%
\pgfpathlineto{\pgfqpoint{4.910475in}{3.116909in}}%
\pgfpathlineto{\pgfqpoint{4.913686in}{3.116173in}}%
\pgfpathlineto{\pgfqpoint{4.916630in}{3.117998in}}%
\pgfpathlineto{\pgfqpoint{4.920911in}{3.123806in}}%
\pgfpathlineto{\pgfqpoint{4.928137in}{3.133434in}}%
\pgfpathlineto{\pgfqpoint{4.931348in}{3.134269in}}%
\pgfpathlineto{\pgfqpoint{4.934292in}{3.132529in}}%
\pgfpathlineto{\pgfqpoint{4.938573in}{3.126790in}}%
\pgfpathlineto{\pgfqpoint{4.946066in}{3.116879in}}%
\pgfpathlineto{\pgfqpoint{4.949278in}{3.116183in}}%
\pgfpathlineto{\pgfqpoint{4.952221in}{3.118043in}}%
\pgfpathlineto{\pgfqpoint{4.956771in}{3.124306in}}%
\pgfpathlineto{\pgfqpoint{4.963728in}{3.133465in}}%
\pgfpathlineto{\pgfqpoint{4.966940in}{3.134260in}}%
\pgfpathlineto{\pgfqpoint{4.969883in}{3.132486in}}%
\pgfpathlineto{\pgfqpoint{4.974165in}{3.126718in}}%
\pgfpathlineto{\pgfqpoint{4.981391in}{3.117032in}}%
\pgfpathlineto{\pgfqpoint{4.984602in}{3.116137in}}%
\pgfpathlineto{\pgfqpoint{4.987546in}{3.117826in}}%
\pgfpathlineto{\pgfqpoint{4.991827in}{3.123524in}}%
\pgfpathlineto{\pgfqpoint{4.999320in}{3.133496in}}%
\pgfpathlineto{\pgfqpoint{5.002532in}{3.134251in}}%
\pgfpathlineto{\pgfqpoint{5.005475in}{3.132442in}}%
\pgfpathlineto{\pgfqpoint{5.009757in}{3.126647in}}%
\pgfpathlineto{\pgfqpoint{5.016982in}{3.117000in}}%
\pgfpathlineto{\pgfqpoint{5.020194in}{3.116145in}}%
\pgfpathlineto{\pgfqpoint{5.023137in}{3.117869in}}%
\pgfpathlineto{\pgfqpoint{5.027419in}{3.123595in}}%
\pgfpathlineto{\pgfqpoint{5.034912in}{3.133526in}}%
\pgfpathlineto{\pgfqpoint{5.038123in}{3.134241in}}%
\pgfpathlineto{\pgfqpoint{5.041067in}{3.132397in}}%
\pgfpathlineto{\pgfqpoint{5.045349in}{3.126575in}}%
\pgfpathlineto{\pgfqpoint{5.052574in}{3.116968in}}%
\pgfpathlineto{\pgfqpoint{5.055786in}{3.116154in}}%
\pgfpathlineto{\pgfqpoint{5.058729in}{3.117913in}}%
\pgfpathlineto{\pgfqpoint{5.063011in}{3.123667in}}%
\pgfpathlineto{\pgfqpoint{5.070504in}{3.133556in}}%
\pgfpathlineto{\pgfqpoint{5.073448in}{3.134285in}}%
\pgfpathlineto{\pgfqpoint{5.076391in}{3.132612in}}%
\pgfpathlineto{\pgfqpoint{5.080673in}{3.126928in}}%
\pgfpathlineto{\pgfqpoint{5.088166in}{3.116937in}}%
\pgfpathlineto{\pgfqpoint{5.091377in}{3.116163in}}%
\pgfpathlineto{\pgfqpoint{5.094321in}{3.117957in}}%
\pgfpathlineto{\pgfqpoint{5.098603in}{3.123739in}}%
\pgfpathlineto{\pgfqpoint{5.105828in}{3.133404in}}%
\pgfpathlineto{\pgfqpoint{5.109040in}{3.134277in}}%
\pgfpathlineto{\pgfqpoint{5.111983in}{3.132570in}}%
\pgfpathlineto{\pgfqpoint{5.116265in}{3.126857in}}%
\pgfpathlineto{\pgfqpoint{5.123758in}{3.116907in}}%
\pgfpathlineto{\pgfqpoint{5.126969in}{3.116173in}}%
\pgfpathlineto{\pgfqpoint{5.129913in}{3.118001in}}%
\pgfpathlineto{\pgfqpoint{5.134195in}{3.123811in}}%
\pgfpathlineto{\pgfqpoint{5.141420in}{3.133436in}}%
\pgfpathlineto{\pgfqpoint{5.144631in}{3.134269in}}%
\pgfpathlineto{\pgfqpoint{5.147575in}{3.132526in}}%
\pgfpathlineto{\pgfqpoint{5.151857in}{3.126785in}}%
\pgfpathlineto{\pgfqpoint{5.159350in}{3.116877in}}%
\pgfpathlineto{\pgfqpoint{5.162561in}{3.116184in}}%
\pgfpathlineto{\pgfqpoint{5.165505in}{3.118046in}}%
\pgfpathlineto{\pgfqpoint{5.170054in}{3.124311in}}%
\pgfpathlineto{\pgfqpoint{5.177012in}{3.133467in}}%
\pgfpathlineto{\pgfqpoint{5.180223in}{3.134260in}}%
\pgfpathlineto{\pgfqpoint{5.183167in}{3.132483in}}%
\pgfpathlineto{\pgfqpoint{5.187449in}{3.126713in}}%
\pgfpathlineto{\pgfqpoint{5.194674in}{3.117030in}}%
\pgfpathlineto{\pgfqpoint{5.197885in}{3.116138in}}%
\pgfpathlineto{\pgfqpoint{5.200829in}{3.117829in}}%
\pgfpathlineto{\pgfqpoint{5.205111in}{3.123529in}}%
\pgfpathlineto{\pgfqpoint{5.212604in}{3.133498in}}%
\pgfpathlineto{\pgfqpoint{5.215815in}{3.134250in}}%
\pgfpathlineto{\pgfqpoint{5.218759in}{3.132438in}}%
\pgfpathlineto{\pgfqpoint{5.223040in}{3.126641in}}%
\pgfpathlineto{\pgfqpoint{5.230266in}{3.116998in}}%
\pgfpathlineto{\pgfqpoint{5.233477in}{3.116146in}}%
\pgfpathlineto{\pgfqpoint{5.236421in}{3.117872in}}%
\pgfpathlineto{\pgfqpoint{5.240703in}{3.123601in}}%
\pgfpathlineto{\pgfqpoint{5.248196in}{3.133528in}}%
\pgfpathlineto{\pgfqpoint{5.251407in}{3.134240in}}%
\pgfpathlineto{\pgfqpoint{5.254351in}{3.132394in}}%
\pgfpathlineto{\pgfqpoint{5.258632in}{3.126569in}}%
\pgfpathlineto{\pgfqpoint{5.265858in}{3.116966in}}%
\pgfpathlineto{\pgfqpoint{5.269069in}{3.116155in}}%
\pgfpathlineto{\pgfqpoint{5.272013in}{3.117916in}}%
\pgfpathlineto{\pgfqpoint{5.276294in}{3.123672in}}%
\pgfpathlineto{\pgfqpoint{5.283787in}{3.133558in}}%
\pgfpathlineto{\pgfqpoint{5.286731in}{3.134285in}}%
\pgfpathlineto{\pgfqpoint{5.289675in}{3.132609in}}%
\pgfpathlineto{\pgfqpoint{5.293957in}{3.126923in}}%
\pgfpathlineto{\pgfqpoint{5.301450in}{3.116935in}}%
\pgfpathlineto{\pgfqpoint{5.304661in}{3.116164in}}%
\pgfpathlineto{\pgfqpoint{5.307605in}{3.117960in}}%
\pgfpathlineto{\pgfqpoint{5.311886in}{3.123744in}}%
\pgfpathlineto{\pgfqpoint{5.319112in}{3.133406in}}%
\pgfpathlineto{\pgfqpoint{5.322323in}{3.134277in}}%
\pgfpathlineto{\pgfqpoint{5.325267in}{3.132566in}}%
\pgfpathlineto{\pgfqpoint{5.329548in}{3.126852in}}%
\pgfpathlineto{\pgfqpoint{5.337041in}{3.116905in}}%
\pgfpathlineto{\pgfqpoint{5.340253in}{3.116174in}}%
\pgfpathlineto{\pgfqpoint{5.343196in}{3.118004in}}%
\pgfpathlineto{\pgfqpoint{5.347478in}{3.123816in}}%
\pgfpathlineto{\pgfqpoint{5.354704in}{3.133438in}}%
\pgfpathlineto{\pgfqpoint{5.357915in}{3.134268in}}%
\pgfpathlineto{\pgfqpoint{5.360858in}{3.132523in}}%
\pgfpathlineto{\pgfqpoint{5.365140in}{3.126780in}}%
\pgfpathlineto{\pgfqpoint{5.372633in}{3.116875in}}%
\pgfpathlineto{\pgfqpoint{5.375845in}{3.116185in}}%
\pgfpathlineto{\pgfqpoint{5.378788in}{3.118049in}}%
\pgfpathlineto{\pgfqpoint{5.383338in}{3.124316in}}%
\pgfpathlineto{\pgfqpoint{5.390295in}{3.133469in}}%
\pgfpathlineto{\pgfqpoint{5.393507in}{3.134259in}}%
\pgfpathlineto{\pgfqpoint{5.396450in}{3.132479in}}%
\pgfpathlineto{\pgfqpoint{5.400732in}{3.126708in}}%
\pgfpathlineto{\pgfqpoint{5.407957in}{3.117027in}}%
\pgfpathlineto{\pgfqpoint{5.411169in}{3.116138in}}%
\pgfpathlineto{\pgfqpoint{5.414112in}{3.117832in}}%
\pgfpathlineto{\pgfqpoint{5.418394in}{3.123534in}}%
\pgfpathlineto{\pgfqpoint{5.425887in}{3.133500in}}%
\pgfpathlineto{\pgfqpoint{5.429098in}{3.134249in}}%
\pgfpathlineto{\pgfqpoint{5.432042in}{3.132435in}}%
\pgfpathlineto{\pgfqpoint{5.436324in}{3.126636in}}%
\pgfpathlineto{\pgfqpoint{5.443549in}{3.116995in}}%
\pgfpathlineto{\pgfqpoint{5.446761in}{3.116147in}}%
\pgfpathlineto{\pgfqpoint{5.449704in}{3.117876in}}%
\pgfpathlineto{\pgfqpoint{5.453986in}{3.123606in}}%
\pgfpathlineto{\pgfqpoint{5.461479in}{3.133530in}}%
\pgfpathlineto{\pgfqpoint{5.464690in}{3.134239in}}%
\pgfpathlineto{\pgfqpoint{5.467634in}{3.132390in}}%
\pgfpathlineto{\pgfqpoint{5.471916in}{3.126564in}}%
\pgfpathlineto{\pgfqpoint{5.479141in}{3.116964in}}%
\pgfpathlineto{\pgfqpoint{5.482352in}{3.116155in}}%
\pgfpathlineto{\pgfqpoint{5.485296in}{3.117919in}}%
\pgfpathlineto{\pgfqpoint{5.489578in}{3.123677in}}%
\pgfpathlineto{\pgfqpoint{5.497071in}{3.133560in}}%
\pgfpathlineto{\pgfqpoint{5.500015in}{3.134284in}}%
\pgfpathlineto{\pgfqpoint{5.502958in}{3.132606in}}%
\pgfpathlineto{\pgfqpoint{5.507240in}{3.126918in}}%
\pgfpathlineto{\pgfqpoint{5.514733in}{3.116933in}}%
\pgfpathlineto{\pgfqpoint{5.517944in}{3.116165in}}%
\pgfpathlineto{\pgfqpoint{5.520888in}{3.117963in}}%
\pgfpathlineto{\pgfqpoint{5.525170in}{3.123749in}}%
\pgfpathlineto{\pgfqpoint{5.532395in}{3.133409in}}%
\pgfpathlineto{\pgfqpoint{5.535606in}{3.134276in}}%
\pgfpathlineto{\pgfqpoint{5.538550in}{3.132563in}}%
\pgfpathlineto{\pgfqpoint{5.542832in}{3.126846in}}%
\pgfpathlineto{\pgfqpoint{5.550325in}{3.116903in}}%
\pgfpathlineto{\pgfqpoint{5.553536in}{3.116175in}}%
\pgfpathlineto{\pgfqpoint{5.556480in}{3.118008in}}%
\pgfpathlineto{\pgfqpoint{5.560762in}{3.123821in}}%
\pgfpathlineto{\pgfqpoint{5.567987in}{3.133440in}}%
\pgfpathlineto{\pgfqpoint{5.571198in}{3.134268in}}%
\pgfpathlineto{\pgfqpoint{5.574142in}{3.132520in}}%
\pgfpathlineto{\pgfqpoint{5.578424in}{3.126775in}}%
\pgfpathlineto{\pgfqpoint{5.585917in}{3.116873in}}%
\pgfpathlineto{\pgfqpoint{5.589128in}{3.116185in}}%
\pgfpathlineto{\pgfqpoint{5.592072in}{3.118053in}}%
\pgfpathlineto{\pgfqpoint{5.596621in}{3.124322in}}%
\pgfpathlineto{\pgfqpoint{5.603579in}{3.133472in}}%
\pgfpathlineto{\pgfqpoint{5.606790in}{3.134258in}}%
\pgfpathlineto{\pgfqpoint{5.609734in}{3.132476in}}%
\pgfpathlineto{\pgfqpoint{5.614015in}{3.126703in}}%
\pgfpathlineto{\pgfqpoint{5.621241in}{3.117025in}}%
\pgfpathlineto{\pgfqpoint{5.624452in}{3.116139in}}%
\pgfpathlineto{\pgfqpoint{5.627396in}{3.117835in}}%
\pgfpathlineto{\pgfqpoint{5.631678in}{3.123539in}}%
\pgfpathlineto{\pgfqpoint{5.639171in}{3.133502in}}%
\pgfpathlineto{\pgfqpoint{5.642382in}{3.134249in}}%
\pgfpathlineto{\pgfqpoint{5.645326in}{3.132432in}}%
\pgfpathlineto{\pgfqpoint{5.649607in}{3.126631in}}%
\pgfpathlineto{\pgfqpoint{5.656833in}{3.116993in}}%
\pgfpathlineto{\pgfqpoint{5.660044in}{3.116147in}}%
\pgfpathlineto{\pgfqpoint{5.662988in}{3.117879in}}%
\pgfpathlineto{\pgfqpoint{5.667269in}{3.123611in}}%
\pgfpathlineto{\pgfqpoint{5.674762in}{3.133532in}}%
\pgfpathlineto{\pgfqpoint{5.677974in}{3.134239in}}%
\pgfpathlineto{\pgfqpoint{5.680917in}{3.132387in}}%
\pgfpathlineto{\pgfqpoint{5.685467in}{3.126131in}}%
\pgfpathlineto{\pgfqpoint{5.692425in}{3.116962in}}%
\pgfpathlineto{\pgfqpoint{5.695636in}{3.116156in}}%
\pgfpathlineto{\pgfqpoint{5.698580in}{3.117922in}}%
\pgfpathlineto{\pgfqpoint{5.702861in}{3.123683in}}%
\pgfpathlineto{\pgfqpoint{5.710354in}{3.133562in}}%
\pgfpathlineto{\pgfqpoint{5.713298in}{3.134283in}}%
\pgfpathlineto{\pgfqpoint{5.716242in}{3.132603in}}%
\pgfpathlineto{\pgfqpoint{5.720523in}{3.126913in}}%
\pgfpathlineto{\pgfqpoint{5.728016in}{3.116931in}}%
\pgfpathlineto{\pgfqpoint{5.731228in}{3.116166in}}%
\pgfpathlineto{\pgfqpoint{5.734171in}{3.117966in}}%
\pgfpathlineto{\pgfqpoint{5.738453in}{3.123754in}}%
\pgfpathlineto{\pgfqpoint{5.745679in}{3.133411in}}%
\pgfpathlineto{\pgfqpoint{5.748890in}{3.134275in}}%
\pgfpathlineto{\pgfqpoint{5.751834in}{3.132560in}}%
\pgfpathlineto{\pgfqpoint{5.756115in}{3.126841in}}%
\pgfpathlineto{\pgfqpoint{5.763608in}{3.116900in}}%
\pgfpathlineto{\pgfqpoint{5.766820in}{3.116176in}}%
\pgfpathlineto{\pgfqpoint{5.769763in}{3.118011in}}%
\pgfpathlineto{\pgfqpoint{5.774045in}{3.123826in}}%
\pgfpathlineto{\pgfqpoint{5.781270in}{3.133443in}}%
\pgfpathlineto{\pgfqpoint{5.784482in}{3.134267in}}%
\pgfpathlineto{\pgfqpoint{5.787425in}{3.132517in}}%
\pgfpathlineto{\pgfqpoint{5.791707in}{3.126770in}}%
\pgfpathlineto{\pgfqpoint{5.799200in}{3.116871in}}%
\pgfpathlineto{\pgfqpoint{5.802411in}{3.116186in}}%
\pgfpathlineto{\pgfqpoint{5.805355in}{3.118056in}}%
\pgfpathlineto{\pgfqpoint{5.809904in}{3.124327in}}%
\pgfpathlineto{\pgfqpoint{5.816862in}{3.133474in}}%
\pgfpathlineto{\pgfqpoint{5.820074in}{3.134258in}}%
\pgfpathlineto{\pgfqpoint{5.823017in}{3.132473in}}%
\pgfpathlineto{\pgfqpoint{5.827299in}{3.126698in}}%
\pgfpathlineto{\pgfqpoint{5.834524in}{3.117023in}}%
\pgfpathlineto{\pgfqpoint{5.837736in}{3.116140in}}%
\pgfpathlineto{\pgfqpoint{5.840679in}{3.117839in}}%
\pgfpathlineto{\pgfqpoint{5.844961in}{3.123544in}}%
\pgfpathlineto{\pgfqpoint{5.852454in}{3.133504in}}%
\pgfpathlineto{\pgfqpoint{5.855665in}{3.134248in}}%
\pgfpathlineto{\pgfqpoint{5.858609in}{3.132429in}}%
\pgfpathlineto{\pgfqpoint{5.862891in}{3.126626in}}%
\pgfpathlineto{\pgfqpoint{5.870116in}{3.116991in}}%
\pgfpathlineto{\pgfqpoint{5.873327in}{3.116148in}}%
\pgfpathlineto{\pgfqpoint{5.876271in}{3.117882in}}%
\pgfpathlineto{\pgfqpoint{5.880553in}{3.123616in}}%
\pgfpathlineto{\pgfqpoint{5.888046in}{3.133534in}}%
\pgfpathlineto{\pgfqpoint{5.891257in}{3.134238in}}%
\pgfpathlineto{\pgfqpoint{5.894201in}{3.132384in}}%
\pgfpathlineto{\pgfqpoint{5.898750in}{3.126126in}}%
\pgfpathlineto{\pgfqpoint{5.905708in}{3.116959in}}%
\pgfpathlineto{\pgfqpoint{5.908919in}{3.116157in}}%
\pgfpathlineto{\pgfqpoint{5.911863in}{3.117925in}}%
\pgfpathlineto{\pgfqpoint{5.916145in}{3.123688in}}%
\pgfpathlineto{\pgfqpoint{5.923638in}{3.133564in}}%
\pgfpathlineto{\pgfqpoint{5.926581in}{3.134283in}}%
\pgfpathlineto{\pgfqpoint{5.929525in}{3.132600in}}%
\pgfpathlineto{\pgfqpoint{5.933807in}{3.126908in}}%
\pgfpathlineto{\pgfqpoint{5.941300in}{3.116929in}}%
\pgfpathlineto{\pgfqpoint{5.944511in}{3.116166in}}%
\pgfpathlineto{\pgfqpoint{5.947455in}{3.117969in}}%
\pgfpathlineto{\pgfqpoint{5.951737in}{3.123759in}}%
\pgfpathlineto{\pgfqpoint{5.958962in}{3.133413in}}%
\pgfpathlineto{\pgfqpoint{5.962173in}{3.134275in}}%
\pgfpathlineto{\pgfqpoint{5.965117in}{3.132557in}}%
\pgfpathlineto{\pgfqpoint{5.969399in}{3.126836in}}%
\pgfpathlineto{\pgfqpoint{5.976892in}{3.116898in}}%
\pgfpathlineto{\pgfqpoint{5.980103in}{3.116176in}}%
\pgfpathlineto{\pgfqpoint{5.983047in}{3.118014in}}%
\pgfpathlineto{\pgfqpoint{5.987328in}{3.123831in}}%
\pgfpathlineto{\pgfqpoint{5.994554in}{3.133445in}}%
\pgfpathlineto{\pgfqpoint{5.997765in}{3.134266in}}%
\pgfpathlineto{\pgfqpoint{6.000709in}{3.132514in}}%
\pgfpathlineto{\pgfqpoint{6.004991in}{3.126765in}}%
\pgfpathlineto{\pgfqpoint{6.012484in}{3.116868in}}%
\pgfpathlineto{\pgfqpoint{6.015695in}{3.116187in}}%
\pgfpathlineto{\pgfqpoint{6.018639in}{3.118059in}}%
\pgfpathlineto{\pgfqpoint{6.023188in}{3.124332in}}%
\pgfpathlineto{\pgfqpoint{6.030146in}{3.133476in}}%
\pgfpathlineto{\pgfqpoint{6.033357in}{3.134257in}}%
\pgfpathlineto{\pgfqpoint{6.036301in}{3.132470in}}%
\pgfpathlineto{\pgfqpoint{6.040582in}{3.126693in}}%
\pgfpathlineto{\pgfqpoint{6.047808in}{3.117020in}}%
\pgfpathlineto{\pgfqpoint{6.051019in}{3.116140in}}%
\pgfpathlineto{\pgfqpoint{6.053963in}{3.117842in}}%
\pgfpathlineto{\pgfqpoint{6.058244in}{3.123549in}}%
\pgfpathlineto{\pgfqpoint{6.065737in}{3.133507in}}%
\pgfpathlineto{\pgfqpoint{6.068949in}{3.134247in}}%
\pgfpathlineto{\pgfqpoint{6.071892in}{3.132426in}}%
\pgfpathlineto{\pgfqpoint{6.076174in}{3.126621in}}%
\pgfpathlineto{\pgfqpoint{6.083400in}{3.116989in}}%
\pgfpathlineto{\pgfqpoint{6.086611in}{3.116149in}}%
\pgfpathlineto{\pgfqpoint{6.089555in}{3.117885in}}%
\pgfpathlineto{\pgfqpoint{6.093836in}{3.123621in}}%
\pgfpathlineto{\pgfqpoint{6.101329in}{3.133537in}}%
\pgfpathlineto{\pgfqpoint{6.104541in}{3.134237in}}%
\pgfpathlineto{\pgfqpoint{6.107484in}{3.132381in}}%
\pgfpathlineto{\pgfqpoint{6.112034in}{3.126121in}}%
\pgfpathlineto{\pgfqpoint{6.118991in}{3.116957in}}%
\pgfpathlineto{\pgfqpoint{6.122203in}{3.116157in}}%
\pgfpathlineto{\pgfqpoint{6.125146in}{3.117928in}}%
\pgfpathlineto{\pgfqpoint{6.129428in}{3.123693in}}%
\pgfpathlineto{\pgfqpoint{6.136921in}{3.133566in}}%
\pgfpathlineto{\pgfqpoint{6.139865in}{3.134282in}}%
\pgfpathlineto{\pgfqpoint{6.142809in}{3.132597in}}%
\pgfpathlineto{\pgfqpoint{6.147090in}{3.126903in}}%
\pgfpathlineto{\pgfqpoint{6.154583in}{3.116926in}}%
\pgfpathlineto{\pgfqpoint{6.157795in}{3.116167in}}%
\pgfpathlineto{\pgfqpoint{6.160738in}{3.117973in}}%
\pgfpathlineto{\pgfqpoint{6.165020in}{3.123765in}}%
\pgfpathlineto{\pgfqpoint{6.172245in}{3.133416in}}%
\pgfpathlineto{\pgfqpoint{6.175457in}{3.134274in}}%
\pgfpathlineto{\pgfqpoint{6.178400in}{3.132554in}}%
\pgfpathlineto{\pgfqpoint{6.182682in}{3.126831in}}%
\pgfpathlineto{\pgfqpoint{6.190175in}{3.116896in}}%
\pgfpathlineto{\pgfqpoint{6.193386in}{3.116177in}}%
\pgfpathlineto{\pgfqpoint{6.196330in}{3.118017in}}%
\pgfpathlineto{\pgfqpoint{6.200612in}{3.123837in}}%
\pgfpathlineto{\pgfqpoint{6.207837in}{3.133447in}}%
\pgfpathlineto{\pgfqpoint{6.211049in}{3.134266in}}%
\pgfpathlineto{\pgfqpoint{6.213992in}{3.132511in}}%
\pgfpathlineto{\pgfqpoint{6.218274in}{3.126759in}}%
\pgfpathlineto{\pgfqpoint{6.225767in}{3.116866in}}%
\pgfpathlineto{\pgfqpoint{6.228711in}{3.116133in}}%
\pgfpathlineto{\pgfqpoint{6.231654in}{3.117802in}}%
\pgfpathlineto{\pgfqpoint{6.235936in}{3.123483in}}%
\pgfpathlineto{\pgfqpoint{6.243429in}{3.133478in}}%
\pgfpathlineto{\pgfqpoint{6.246640in}{3.134256in}}%
\pgfpathlineto{\pgfqpoint{6.249584in}{3.132467in}}%
\pgfpathlineto{\pgfqpoint{6.253866in}{3.126688in}}%
\pgfpathlineto{\pgfqpoint{6.261091in}{3.117018in}}%
\pgfpathlineto{\pgfqpoint{6.264302in}{3.116141in}}%
\pgfpathlineto{\pgfqpoint{6.267246in}{3.117845in}}%
\pgfpathlineto{\pgfqpoint{6.271528in}{3.123555in}}%
\pgfpathlineto{\pgfqpoint{6.279021in}{3.133509in}}%
\pgfpathlineto{\pgfqpoint{6.282232in}{3.134247in}}%
\pgfpathlineto{\pgfqpoint{6.285176in}{3.132422in}}%
\pgfpathlineto{\pgfqpoint{6.289458in}{3.126616in}}%
\pgfpathlineto{\pgfqpoint{6.296683in}{3.116986in}}%
\pgfpathlineto{\pgfqpoint{6.299894in}{3.116149in}}%
\pgfpathlineto{\pgfqpoint{6.302838in}{3.117888in}}%
\pgfpathlineto{\pgfqpoint{6.307120in}{3.123626in}}%
\pgfpathlineto{\pgfqpoint{6.314613in}{3.133539in}}%
\pgfpathlineto{\pgfqpoint{6.317824in}{3.134236in}}%
\pgfpathlineto{\pgfqpoint{6.320768in}{3.132378in}}%
\pgfpathlineto{\pgfqpoint{6.325317in}{3.126115in}}%
\pgfpathlineto{\pgfqpoint{6.332275in}{3.116955in}}%
\pgfpathlineto{\pgfqpoint{6.335486in}{3.116158in}}%
\pgfpathlineto{\pgfqpoint{6.338430in}{3.117932in}}%
\pgfpathlineto{\pgfqpoint{6.342712in}{3.123698in}}%
\pgfpathlineto{\pgfqpoint{6.350205in}{3.133568in}}%
\pgfpathlineto{\pgfqpoint{6.353148in}{3.134282in}}%
\pgfpathlineto{\pgfqpoint{6.356092in}{3.132594in}}%
\pgfpathlineto{\pgfqpoint{6.360374in}{3.126898in}}%
\pgfpathlineto{\pgfqpoint{6.367867in}{3.116924in}}%
\pgfpathlineto{\pgfqpoint{6.371078in}{3.116168in}}%
\pgfpathlineto{\pgfqpoint{6.374022in}{3.117976in}}%
\pgfpathlineto{\pgfqpoint{6.378303in}{3.123770in}}%
\pgfpathlineto{\pgfqpoint{6.385529in}{3.133418in}}%
\pgfpathlineto{\pgfqpoint{6.388740in}{3.134274in}}%
\pgfpathlineto{\pgfqpoint{6.391684in}{3.132551in}}%
\pgfpathlineto{\pgfqpoint{6.395966in}{3.126826in}}%
\pgfpathlineto{\pgfqpoint{6.403459in}{3.116894in}}%
\pgfpathlineto{\pgfqpoint{6.406670in}{3.116178in}}%
\pgfpathlineto{\pgfqpoint{6.409614in}{3.118020in}}%
\pgfpathlineto{\pgfqpoint{6.413895in}{3.123842in}}%
\pgfpathlineto{\pgfqpoint{6.421121in}{3.133449in}}%
\pgfpathlineto{\pgfqpoint{6.424332in}{3.134265in}}%
\pgfpathlineto{\pgfqpoint{6.427276in}{3.132508in}}%
\pgfpathlineto{\pgfqpoint{6.431557in}{3.126754in}}%
\pgfpathlineto{\pgfqpoint{6.439050in}{3.116864in}}%
\pgfpathlineto{\pgfqpoint{6.441994in}{3.116134in}}%
\pgfpathlineto{\pgfqpoint{6.444938in}{3.117805in}}%
\pgfpathlineto{\pgfqpoint{6.449220in}{3.123488in}}%
\pgfpathlineto{\pgfqpoint{6.456713in}{3.133480in}}%
\pgfpathlineto{\pgfqpoint{6.459924in}{3.134256in}}%
\pgfpathlineto{\pgfqpoint{6.462867in}{3.132464in}}%
\pgfpathlineto{\pgfqpoint{6.467149in}{3.126682in}}%
\pgfpathlineto{\pgfqpoint{6.474375in}{3.117016in}}%
\pgfpathlineto{\pgfqpoint{6.477586in}{3.116141in}}%
\pgfpathlineto{\pgfqpoint{6.480530in}{3.117848in}}%
\pgfpathlineto{\pgfqpoint{6.484811in}{3.123560in}}%
\pgfpathlineto{\pgfqpoint{6.492304in}{3.133511in}}%
\pgfpathlineto{\pgfqpoint{6.495516in}{3.134246in}}%
\pgfpathlineto{\pgfqpoint{6.498459in}{3.132419in}}%
\pgfpathlineto{\pgfqpoint{6.502741in}{3.126611in}}%
\pgfpathlineto{\pgfqpoint{6.509966in}{3.116984in}}%
\pgfpathlineto{\pgfqpoint{6.513178in}{3.116150in}}%
\pgfpathlineto{\pgfqpoint{6.516121in}{3.117891in}}%
\pgfpathlineto{\pgfqpoint{6.520403in}{3.123631in}}%
\pgfpathlineto{\pgfqpoint{6.527896in}{3.133541in}}%
\pgfpathlineto{\pgfqpoint{6.531107in}{3.134235in}}%
\pgfpathlineto{\pgfqpoint{6.534051in}{3.132374in}}%
\pgfpathlineto{\pgfqpoint{6.538601in}{3.126110in}}%
\pgfpathlineto{\pgfqpoint{6.545558in}{3.116953in}}%
\pgfpathlineto{\pgfqpoint{6.548770in}{3.116159in}}%
\pgfpathlineto{\pgfqpoint{6.551713in}{3.117935in}}%
\pgfpathlineto{\pgfqpoint{6.555995in}{3.123703in}}%
\pgfpathlineto{\pgfqpoint{6.563220in}{3.133388in}}%
\pgfpathlineto{\pgfqpoint{6.566432in}{3.134281in}}%
\pgfpathlineto{\pgfqpoint{6.569375in}{3.132591in}}%
\pgfpathlineto{\pgfqpoint{6.573657in}{3.126892in}}%
\pgfpathlineto{\pgfqpoint{6.581150in}{3.116922in}}%
\pgfpathlineto{\pgfqpoint{6.584361in}{3.116168in}}%
\pgfpathlineto{\pgfqpoint{6.587305in}{3.117979in}}%
\pgfpathlineto{\pgfqpoint{6.591587in}{3.123775in}}%
\pgfpathlineto{\pgfqpoint{6.598812in}{3.133420in}}%
\pgfpathlineto{\pgfqpoint{6.602024in}{3.134273in}}%
\pgfpathlineto{\pgfqpoint{6.604967in}{3.132548in}}%
\pgfpathlineto{\pgfqpoint{6.609249in}{3.126821in}}%
\pgfpathlineto{\pgfqpoint{6.616742in}{3.116892in}}%
\pgfpathlineto{\pgfqpoint{6.619953in}{3.116178in}}%
\pgfpathlineto{\pgfqpoint{6.622897in}{3.118024in}}%
\pgfpathlineto{\pgfqpoint{6.627179in}{3.123847in}}%
\pgfpathlineto{\pgfqpoint{6.634404in}{3.133452in}}%
\pgfpathlineto{\pgfqpoint{6.637615in}{3.134264in}}%
\pgfpathlineto{\pgfqpoint{6.640559in}{3.132504in}}%
\pgfpathlineto{\pgfqpoint{6.644841in}{3.126749in}}%
\pgfpathlineto{\pgfqpoint{6.652334in}{3.116862in}}%
\pgfpathlineto{\pgfqpoint{6.655278in}{3.116134in}}%
\pgfpathlineto{\pgfqpoint{6.658221in}{3.117808in}}%
\pgfpathlineto{\pgfqpoint{6.662503in}{3.123493in}}%
\pgfpathlineto{\pgfqpoint{6.663306in}{3.124778in}}%
\pgfpathlineto{\pgfqpoint{6.663306in}{3.124778in}}%
\pgfusepath{stroke}%
\end{pgfscope}%
\begin{pgfscope}%
\pgfpathrectangle{\pgfqpoint{0.467797in}{2.292089in}}{\pgfqpoint{6.490533in}{1.666241in}}%
\pgfusepath{clip}%
\pgfsetrectcap%
\pgfsetroundjoin%
\pgfsetlinewidth{1.505625pt}%
\definecolor{currentstroke}{rgb}{0.172549,0.627451,0.172549}%
\pgfsetstrokecolor{currentstroke}%
\pgfsetdash{}{0pt}%
\pgfpathmoveto{\pgfqpoint{0.762821in}{3.125209in}}%
\pgfpathlineto{\pgfqpoint{0.768976in}{3.133207in}}%
\pgfpathlineto{\pgfqpoint{0.772187in}{3.134051in}}%
\pgfpathlineto{\pgfqpoint{0.775131in}{3.132262in}}%
\pgfpathlineto{\pgfqpoint{0.779413in}{3.126437in}}%
\pgfpathlineto{\pgfqpoint{0.786370in}{3.117184in}}%
\pgfpathlineto{\pgfqpoint{0.789582in}{3.116376in}}%
\pgfpathlineto{\pgfqpoint{0.792525in}{3.118196in}}%
\pgfpathlineto{\pgfqpoint{0.796807in}{3.124045in}}%
\pgfpathlineto{\pgfqpoint{0.803765in}{3.133262in}}%
\pgfpathlineto{\pgfqpoint{0.806709in}{3.134083in}}%
\pgfpathlineto{\pgfqpoint{0.809652in}{3.132444in}}%
\pgfpathlineto{\pgfqpoint{0.813934in}{3.126738in}}%
\pgfpathlineto{\pgfqpoint{0.821159in}{3.117130in}}%
\pgfpathlineto{\pgfqpoint{0.824103in}{3.116341in}}%
\pgfpathlineto{\pgfqpoint{0.827047in}{3.118012in}}%
\pgfpathlineto{\pgfqpoint{0.831329in}{3.123743in}}%
\pgfpathlineto{\pgfqpoint{0.838554in}{3.133316in}}%
\pgfpathlineto{\pgfqpoint{0.841498in}{3.134071in}}%
\pgfpathlineto{\pgfqpoint{0.844441in}{3.132369in}}%
\pgfpathlineto{\pgfqpoint{0.848723in}{3.126613in}}%
\pgfpathlineto{\pgfqpoint{0.855948in}{3.117077in}}%
\pgfpathlineto{\pgfqpoint{0.858892in}{3.116355in}}%
\pgfpathlineto{\pgfqpoint{0.861836in}{3.118088in}}%
\pgfpathlineto{\pgfqpoint{0.866118in}{3.123869in}}%
\pgfpathlineto{\pgfqpoint{0.873343in}{3.133367in}}%
\pgfpathlineto{\pgfqpoint{0.876287in}{3.134057in}}%
\pgfpathlineto{\pgfqpoint{0.879230in}{3.132293in}}%
\pgfpathlineto{\pgfqpoint{0.883512in}{3.126487in}}%
\pgfpathlineto{\pgfqpoint{0.890470in}{3.117206in}}%
\pgfpathlineto{\pgfqpoint{0.893681in}{3.116370in}}%
\pgfpathlineto{\pgfqpoint{0.896625in}{3.118165in}}%
\pgfpathlineto{\pgfqpoint{0.900907in}{3.123994in}}%
\pgfpathlineto{\pgfqpoint{0.907864in}{3.133240in}}%
\pgfpathlineto{\pgfqpoint{0.911076in}{3.134041in}}%
\pgfpathlineto{\pgfqpoint{0.914019in}{3.132215in}}%
\pgfpathlineto{\pgfqpoint{0.918301in}{3.126361in}}%
\pgfpathlineto{\pgfqpoint{0.925259in}{3.117151in}}%
\pgfpathlineto{\pgfqpoint{0.928203in}{3.116337in}}%
\pgfpathlineto{\pgfqpoint{0.931146in}{3.117982in}}%
\pgfpathlineto{\pgfqpoint{0.935428in}{3.123693in}}%
\pgfpathlineto{\pgfqpoint{0.942653in}{3.133294in}}%
\pgfpathlineto{\pgfqpoint{0.945597in}{3.134076in}}%
\pgfpathlineto{\pgfqpoint{0.948541in}{3.132399in}}%
\pgfpathlineto{\pgfqpoint{0.952822in}{3.126663in}}%
\pgfpathlineto{\pgfqpoint{0.960048in}{3.117098in}}%
\pgfpathlineto{\pgfqpoint{0.962992in}{3.116349in}}%
\pgfpathlineto{\pgfqpoint{0.965935in}{3.118057in}}%
\pgfpathlineto{\pgfqpoint{0.970217in}{3.123819in}}%
\pgfpathlineto{\pgfqpoint{0.977442in}{3.133347in}}%
\pgfpathlineto{\pgfqpoint{0.980386in}{3.134063in}}%
\pgfpathlineto{\pgfqpoint{0.983330in}{3.132324in}}%
\pgfpathlineto{\pgfqpoint{0.987611in}{3.126537in}}%
\pgfpathlineto{\pgfqpoint{0.994837in}{3.117046in}}%
\pgfpathlineto{\pgfqpoint{0.997781in}{3.116364in}}%
\pgfpathlineto{\pgfqpoint{1.000724in}{3.118134in}}%
\pgfpathlineto{\pgfqpoint{1.005006in}{3.123944in}}%
\pgfpathlineto{\pgfqpoint{1.011964in}{3.133218in}}%
\pgfpathlineto{\pgfqpoint{1.015175in}{3.134047in}}%
\pgfpathlineto{\pgfqpoint{1.018119in}{3.132247in}}%
\pgfpathlineto{\pgfqpoint{1.022400in}{3.126412in}}%
\pgfpathlineto{\pgfqpoint{1.029358in}{3.117173in}}%
\pgfpathlineto{\pgfqpoint{1.032570in}{3.116380in}}%
\pgfpathlineto{\pgfqpoint{1.035513in}{3.118211in}}%
\pgfpathlineto{\pgfqpoint{1.039795in}{3.124070in}}%
\pgfpathlineto{\pgfqpoint{1.046753in}{3.133273in}}%
\pgfpathlineto{\pgfqpoint{1.049696in}{3.134081in}}%
\pgfpathlineto{\pgfqpoint{1.052640in}{3.132429in}}%
\pgfpathlineto{\pgfqpoint{1.056922in}{3.126713in}}%
\pgfpathlineto{\pgfqpoint{1.064147in}{3.117119in}}%
\pgfpathlineto{\pgfqpoint{1.067091in}{3.116344in}}%
\pgfpathlineto{\pgfqpoint{1.070035in}{3.118027in}}%
\pgfpathlineto{\pgfqpoint{1.074316in}{3.123769in}}%
\pgfpathlineto{\pgfqpoint{1.081542in}{3.133326in}}%
\pgfpathlineto{\pgfqpoint{1.084486in}{3.134068in}}%
\pgfpathlineto{\pgfqpoint{1.087429in}{3.132354in}}%
\pgfpathlineto{\pgfqpoint{1.091711in}{3.126588in}}%
\pgfpathlineto{\pgfqpoint{1.098936in}{3.117067in}}%
\pgfpathlineto{\pgfqpoint{1.101880in}{3.116358in}}%
\pgfpathlineto{\pgfqpoint{1.104824in}{3.118103in}}%
\pgfpathlineto{\pgfqpoint{1.109105in}{3.123894in}}%
\pgfpathlineto{\pgfqpoint{1.116331in}{3.133378in}}%
\pgfpathlineto{\pgfqpoint{1.119275in}{3.134054in}}%
\pgfpathlineto{\pgfqpoint{1.122218in}{3.132278in}}%
\pgfpathlineto{\pgfqpoint{1.126500in}{3.126462in}}%
\pgfpathlineto{\pgfqpoint{1.133458in}{3.117195in}}%
\pgfpathlineto{\pgfqpoint{1.136669in}{3.116373in}}%
\pgfpathlineto{\pgfqpoint{1.139613in}{3.118180in}}%
\pgfpathlineto{\pgfqpoint{1.143894in}{3.124020in}}%
\pgfpathlineto{\pgfqpoint{1.150852in}{3.133251in}}%
\pgfpathlineto{\pgfqpoint{1.154064in}{3.134037in}}%
\pgfpathlineto{\pgfqpoint{1.157007in}{3.132200in}}%
\pgfpathlineto{\pgfqpoint{1.161289in}{3.126336in}}%
\pgfpathlineto{\pgfqpoint{1.168247in}{3.117141in}}%
\pgfpathlineto{\pgfqpoint{1.171190in}{3.116339in}}%
\pgfpathlineto{\pgfqpoint{1.174134in}{3.117997in}}%
\pgfpathlineto{\pgfqpoint{1.178416in}{3.123718in}}%
\pgfpathlineto{\pgfqpoint{1.185641in}{3.133305in}}%
\pgfpathlineto{\pgfqpoint{1.188585in}{3.134074in}}%
\pgfpathlineto{\pgfqpoint{1.191529in}{3.132384in}}%
\pgfpathlineto{\pgfqpoint{1.195810in}{3.126638in}}%
\pgfpathlineto{\pgfqpoint{1.203036in}{3.117088in}}%
\pgfpathlineto{\pgfqpoint{1.205979in}{3.116352in}}%
\pgfpathlineto{\pgfqpoint{1.208923in}{3.118072in}}%
\pgfpathlineto{\pgfqpoint{1.213205in}{3.123844in}}%
\pgfpathlineto{\pgfqpoint{1.220430in}{3.133357in}}%
\pgfpathlineto{\pgfqpoint{1.223374in}{3.134060in}}%
\pgfpathlineto{\pgfqpoint{1.226318in}{3.132308in}}%
\pgfpathlineto{\pgfqpoint{1.230599in}{3.126512in}}%
\pgfpathlineto{\pgfqpoint{1.237825in}{3.117036in}}%
\pgfpathlineto{\pgfqpoint{1.240768in}{3.116367in}}%
\pgfpathlineto{\pgfqpoint{1.243712in}{3.118149in}}%
\pgfpathlineto{\pgfqpoint{1.247994in}{3.123969in}}%
\pgfpathlineto{\pgfqpoint{1.254952in}{3.133229in}}%
\pgfpathlineto{\pgfqpoint{1.258163in}{3.134044in}}%
\pgfpathlineto{\pgfqpoint{1.261107in}{3.132231in}}%
\pgfpathlineto{\pgfqpoint{1.265388in}{3.126387in}}%
\pgfpathlineto{\pgfqpoint{1.272346in}{3.117162in}}%
\pgfpathlineto{\pgfqpoint{1.275290in}{3.116334in}}%
\pgfpathlineto{\pgfqpoint{1.278234in}{3.117967in}}%
\pgfpathlineto{\pgfqpoint{1.282515in}{3.123668in}}%
\pgfpathlineto{\pgfqpoint{1.289741in}{3.133284in}}%
\pgfpathlineto{\pgfqpoint{1.292684in}{3.134079in}}%
\pgfpathlineto{\pgfqpoint{1.295628in}{3.132414in}}%
\pgfpathlineto{\pgfqpoint{1.299910in}{3.126688in}}%
\pgfpathlineto{\pgfqpoint{1.307135in}{3.117109in}}%
\pgfpathlineto{\pgfqpoint{1.310079in}{3.116347in}}%
\pgfpathlineto{\pgfqpoint{1.313023in}{3.118042in}}%
\pgfpathlineto{\pgfqpoint{1.317304in}{3.123794in}}%
\pgfpathlineto{\pgfqpoint{1.324530in}{3.133337in}}%
\pgfpathlineto{\pgfqpoint{1.327473in}{3.134066in}}%
\pgfpathlineto{\pgfqpoint{1.330417in}{3.132339in}}%
\pgfpathlineto{\pgfqpoint{1.334699in}{3.126563in}}%
\pgfpathlineto{\pgfqpoint{1.341924in}{3.117057in}}%
\pgfpathlineto{\pgfqpoint{1.344868in}{3.116361in}}%
\pgfpathlineto{\pgfqpoint{1.347812in}{3.118118in}}%
\pgfpathlineto{\pgfqpoint{1.352093in}{3.123919in}}%
\pgfpathlineto{\pgfqpoint{1.359051in}{3.133207in}}%
\pgfpathlineto{\pgfqpoint{1.362262in}{3.134051in}}%
\pgfpathlineto{\pgfqpoint{1.365206in}{3.132262in}}%
\pgfpathlineto{\pgfqpoint{1.369488in}{3.126437in}}%
\pgfpathlineto{\pgfqpoint{1.376446in}{3.117184in}}%
\pgfpathlineto{\pgfqpoint{1.379657in}{3.116376in}}%
\pgfpathlineto{\pgfqpoint{1.382601in}{3.118196in}}%
\pgfpathlineto{\pgfqpoint{1.386882in}{3.124045in}}%
\pgfpathlineto{\pgfqpoint{1.393840in}{3.133262in}}%
\pgfpathlineto{\pgfqpoint{1.396784in}{3.134083in}}%
\pgfpathlineto{\pgfqpoint{1.399728in}{3.132444in}}%
\pgfpathlineto{\pgfqpoint{1.404009in}{3.126738in}}%
\pgfpathlineto{\pgfqpoint{1.411235in}{3.117130in}}%
\pgfpathlineto{\pgfqpoint{1.414178in}{3.116341in}}%
\pgfpathlineto{\pgfqpoint{1.417122in}{3.118012in}}%
\pgfpathlineto{\pgfqpoint{1.421404in}{3.123743in}}%
\pgfpathlineto{\pgfqpoint{1.428629in}{3.133316in}}%
\pgfpathlineto{\pgfqpoint{1.431573in}{3.134071in}}%
\pgfpathlineto{\pgfqpoint{1.434517in}{3.132369in}}%
\pgfpathlineto{\pgfqpoint{1.438798in}{3.126613in}}%
\pgfpathlineto{\pgfqpoint{1.446024in}{3.117077in}}%
\pgfpathlineto{\pgfqpoint{1.448967in}{3.116355in}}%
\pgfpathlineto{\pgfqpoint{1.451911in}{3.118088in}}%
\pgfpathlineto{\pgfqpoint{1.456193in}{3.123869in}}%
\pgfpathlineto{\pgfqpoint{1.463418in}{3.133367in}}%
\pgfpathlineto{\pgfqpoint{1.466362in}{3.134057in}}%
\pgfpathlineto{\pgfqpoint{1.469306in}{3.132293in}}%
\pgfpathlineto{\pgfqpoint{1.473587in}{3.126487in}}%
\pgfpathlineto{\pgfqpoint{1.480545in}{3.117206in}}%
\pgfpathlineto{\pgfqpoint{1.483756in}{3.116370in}}%
\pgfpathlineto{\pgfqpoint{1.486700in}{3.118165in}}%
\pgfpathlineto{\pgfqpoint{1.490982in}{3.123994in}}%
\pgfpathlineto{\pgfqpoint{1.497940in}{3.133240in}}%
\pgfpathlineto{\pgfqpoint{1.501151in}{3.134041in}}%
\pgfpathlineto{\pgfqpoint{1.504095in}{3.132215in}}%
\pgfpathlineto{\pgfqpoint{1.508376in}{3.126361in}}%
\pgfpathlineto{\pgfqpoint{1.515334in}{3.117151in}}%
\pgfpathlineto{\pgfqpoint{1.518278in}{3.116337in}}%
\pgfpathlineto{\pgfqpoint{1.521221in}{3.117982in}}%
\pgfpathlineto{\pgfqpoint{1.525503in}{3.123693in}}%
\pgfpathlineto{\pgfqpoint{1.532729in}{3.133294in}}%
\pgfpathlineto{\pgfqpoint{1.535672in}{3.134076in}}%
\pgfpathlineto{\pgfqpoint{1.538616in}{3.132399in}}%
\pgfpathlineto{\pgfqpoint{1.542898in}{3.126663in}}%
\pgfpathlineto{\pgfqpoint{1.550123in}{3.117098in}}%
\pgfpathlineto{\pgfqpoint{1.553067in}{3.116349in}}%
\pgfpathlineto{\pgfqpoint{1.556010in}{3.118057in}}%
\pgfpathlineto{\pgfqpoint{1.560292in}{3.123819in}}%
\pgfpathlineto{\pgfqpoint{1.567518in}{3.133347in}}%
\pgfpathlineto{\pgfqpoint{1.570461in}{3.134063in}}%
\pgfpathlineto{\pgfqpoint{1.573405in}{3.132324in}}%
\pgfpathlineto{\pgfqpoint{1.577687in}{3.126537in}}%
\pgfpathlineto{\pgfqpoint{1.584912in}{3.117046in}}%
\pgfpathlineto{\pgfqpoint{1.587856in}{3.116364in}}%
\pgfpathlineto{\pgfqpoint{1.590800in}{3.118134in}}%
\pgfpathlineto{\pgfqpoint{1.595081in}{3.123944in}}%
\pgfpathlineto{\pgfqpoint{1.602039in}{3.133218in}}%
\pgfpathlineto{\pgfqpoint{1.605250in}{3.134047in}}%
\pgfpathlineto{\pgfqpoint{1.608194in}{3.132247in}}%
\pgfpathlineto{\pgfqpoint{1.612476in}{3.126412in}}%
\pgfpathlineto{\pgfqpoint{1.619434in}{3.117173in}}%
\pgfpathlineto{\pgfqpoint{1.622645in}{3.116380in}}%
\pgfpathlineto{\pgfqpoint{1.625589in}{3.118211in}}%
\pgfpathlineto{\pgfqpoint{1.629870in}{3.124070in}}%
\pgfpathlineto{\pgfqpoint{1.636828in}{3.133273in}}%
\pgfpathlineto{\pgfqpoint{1.639772in}{3.134081in}}%
\pgfpathlineto{\pgfqpoint{1.642715in}{3.132429in}}%
\pgfpathlineto{\pgfqpoint{1.646997in}{3.126713in}}%
\pgfpathlineto{\pgfqpoint{1.654223in}{3.117119in}}%
\pgfpathlineto{\pgfqpoint{1.657166in}{3.116344in}}%
\pgfpathlineto{\pgfqpoint{1.660110in}{3.118027in}}%
\pgfpathlineto{\pgfqpoint{1.664392in}{3.123769in}}%
\pgfpathlineto{\pgfqpoint{1.671617in}{3.133326in}}%
\pgfpathlineto{\pgfqpoint{1.674561in}{3.134068in}}%
\pgfpathlineto{\pgfqpoint{1.677504in}{3.132354in}}%
\pgfpathlineto{\pgfqpoint{1.681786in}{3.126588in}}%
\pgfpathlineto{\pgfqpoint{1.689012in}{3.117067in}}%
\pgfpathlineto{\pgfqpoint{1.691955in}{3.116358in}}%
\pgfpathlineto{\pgfqpoint{1.694899in}{3.118103in}}%
\pgfpathlineto{\pgfqpoint{1.699181in}{3.123894in}}%
\pgfpathlineto{\pgfqpoint{1.706406in}{3.133378in}}%
\pgfpathlineto{\pgfqpoint{1.709350in}{3.134054in}}%
\pgfpathlineto{\pgfqpoint{1.712293in}{3.132278in}}%
\pgfpathlineto{\pgfqpoint{1.716575in}{3.126462in}}%
\pgfpathlineto{\pgfqpoint{1.723533in}{3.117195in}}%
\pgfpathlineto{\pgfqpoint{1.726744in}{3.116373in}}%
\pgfpathlineto{\pgfqpoint{1.729688in}{3.118180in}}%
\pgfpathlineto{\pgfqpoint{1.733970in}{3.124020in}}%
\pgfpathlineto{\pgfqpoint{1.740927in}{3.133251in}}%
\pgfpathlineto{\pgfqpoint{1.744139in}{3.134037in}}%
\pgfpathlineto{\pgfqpoint{1.747082in}{3.132200in}}%
\pgfpathlineto{\pgfqpoint{1.751364in}{3.126336in}}%
\pgfpathlineto{\pgfqpoint{1.758322in}{3.117141in}}%
\pgfpathlineto{\pgfqpoint{1.761266in}{3.116339in}}%
\pgfpathlineto{\pgfqpoint{1.764209in}{3.117997in}}%
\pgfpathlineto{\pgfqpoint{1.768491in}{3.123718in}}%
\pgfpathlineto{\pgfqpoint{1.775717in}{3.133305in}}%
\pgfpathlineto{\pgfqpoint{1.778660in}{3.134074in}}%
\pgfpathlineto{\pgfqpoint{1.781604in}{3.132384in}}%
\pgfpathlineto{\pgfqpoint{1.785886in}{3.126638in}}%
\pgfpathlineto{\pgfqpoint{1.793111in}{3.117088in}}%
\pgfpathlineto{\pgfqpoint{1.796055in}{3.116352in}}%
\pgfpathlineto{\pgfqpoint{1.798998in}{3.118072in}}%
\pgfpathlineto{\pgfqpoint{1.803280in}{3.123844in}}%
\pgfpathlineto{\pgfqpoint{1.810506in}{3.133357in}}%
\pgfpathlineto{\pgfqpoint{1.813449in}{3.134060in}}%
\pgfpathlineto{\pgfqpoint{1.816393in}{3.132308in}}%
\pgfpathlineto{\pgfqpoint{1.820675in}{3.126512in}}%
\pgfpathlineto{\pgfqpoint{1.827900in}{3.117036in}}%
\pgfpathlineto{\pgfqpoint{1.830844in}{3.116367in}}%
\pgfpathlineto{\pgfqpoint{1.833787in}{3.118149in}}%
\pgfpathlineto{\pgfqpoint{1.838069in}{3.123969in}}%
\pgfpathlineto{\pgfqpoint{1.845027in}{3.133229in}}%
\pgfpathlineto{\pgfqpoint{1.848238in}{3.134044in}}%
\pgfpathlineto{\pgfqpoint{1.851182in}{3.132231in}}%
\pgfpathlineto{\pgfqpoint{1.855464in}{3.126387in}}%
\pgfpathlineto{\pgfqpoint{1.862421in}{3.117162in}}%
\pgfpathlineto{\pgfqpoint{1.865365in}{3.116334in}}%
\pgfpathlineto{\pgfqpoint{1.868309in}{3.117967in}}%
\pgfpathlineto{\pgfqpoint{1.872591in}{3.123668in}}%
\pgfpathlineto{\pgfqpoint{1.879816in}{3.133284in}}%
\pgfpathlineto{\pgfqpoint{1.882760in}{3.134079in}}%
\pgfpathlineto{\pgfqpoint{1.885703in}{3.132414in}}%
\pgfpathlineto{\pgfqpoint{1.889985in}{3.126688in}}%
\pgfpathlineto{\pgfqpoint{1.897210in}{3.117109in}}%
\pgfpathlineto{\pgfqpoint{1.900154in}{3.116347in}}%
\pgfpathlineto{\pgfqpoint{1.903098in}{3.118042in}}%
\pgfpathlineto{\pgfqpoint{1.907380in}{3.123794in}}%
\pgfpathlineto{\pgfqpoint{1.914605in}{3.133337in}}%
\pgfpathlineto{\pgfqpoint{1.917549in}{3.134066in}}%
\pgfpathlineto{\pgfqpoint{1.920492in}{3.132339in}}%
\pgfpathlineto{\pgfqpoint{1.924774in}{3.126563in}}%
\pgfpathlineto{\pgfqpoint{1.931999in}{3.117057in}}%
\pgfpathlineto{\pgfqpoint{1.934943in}{3.116361in}}%
\pgfpathlineto{\pgfqpoint{1.937887in}{3.118118in}}%
\pgfpathlineto{\pgfqpoint{1.942169in}{3.123919in}}%
\pgfpathlineto{\pgfqpoint{1.949126in}{3.133207in}}%
\pgfpathlineto{\pgfqpoint{1.952338in}{3.134051in}}%
\pgfpathlineto{\pgfqpoint{1.955281in}{3.132262in}}%
\pgfpathlineto{\pgfqpoint{1.959563in}{3.126437in}}%
\pgfpathlineto{\pgfqpoint{1.966521in}{3.117184in}}%
\pgfpathlineto{\pgfqpoint{1.969732in}{3.116376in}}%
\pgfpathlineto{\pgfqpoint{1.972676in}{3.118196in}}%
\pgfpathlineto{\pgfqpoint{1.976958in}{3.124045in}}%
\pgfpathlineto{\pgfqpoint{1.983915in}{3.133262in}}%
\pgfpathlineto{\pgfqpoint{1.986859in}{3.134083in}}%
\pgfpathlineto{\pgfqpoint{1.989803in}{3.132444in}}%
\pgfpathlineto{\pgfqpoint{1.994084in}{3.126738in}}%
\pgfpathlineto{\pgfqpoint{2.001310in}{3.117130in}}%
\pgfpathlineto{\pgfqpoint{2.004254in}{3.116341in}}%
\pgfpathlineto{\pgfqpoint{2.007197in}{3.118012in}}%
\pgfpathlineto{\pgfqpoint{2.011479in}{3.123743in}}%
\pgfpathlineto{\pgfqpoint{2.018704in}{3.133316in}}%
\pgfpathlineto{\pgfqpoint{2.021648in}{3.134071in}}%
\pgfpathlineto{\pgfqpoint{2.024592in}{3.132369in}}%
\pgfpathlineto{\pgfqpoint{2.028874in}{3.126613in}}%
\pgfpathlineto{\pgfqpoint{2.036099in}{3.117077in}}%
\pgfpathlineto{\pgfqpoint{2.039043in}{3.116355in}}%
\pgfpathlineto{\pgfqpoint{2.041986in}{3.118088in}}%
\pgfpathlineto{\pgfqpoint{2.046268in}{3.123869in}}%
\pgfpathlineto{\pgfqpoint{2.053493in}{3.133367in}}%
\pgfpathlineto{\pgfqpoint{2.056437in}{3.134057in}}%
\pgfpathlineto{\pgfqpoint{2.059381in}{3.132293in}}%
\pgfpathlineto{\pgfqpoint{2.063663in}{3.126487in}}%
\pgfpathlineto{\pgfqpoint{2.070620in}{3.117206in}}%
\pgfpathlineto{\pgfqpoint{2.073832in}{3.116370in}}%
\pgfpathlineto{\pgfqpoint{2.076775in}{3.118165in}}%
\pgfpathlineto{\pgfqpoint{2.081057in}{3.123994in}}%
\pgfpathlineto{\pgfqpoint{2.088015in}{3.133240in}}%
\pgfpathlineto{\pgfqpoint{2.091226in}{3.134041in}}%
\pgfpathlineto{\pgfqpoint{2.094170in}{3.132215in}}%
\pgfpathlineto{\pgfqpoint{2.098452in}{3.126361in}}%
\pgfpathlineto{\pgfqpoint{2.105409in}{3.117151in}}%
\pgfpathlineto{\pgfqpoint{2.108353in}{3.116337in}}%
\pgfpathlineto{\pgfqpoint{2.111297in}{3.117982in}}%
\pgfpathlineto{\pgfqpoint{2.115578in}{3.123693in}}%
\pgfpathlineto{\pgfqpoint{2.122804in}{3.133294in}}%
\pgfpathlineto{\pgfqpoint{2.125748in}{3.134076in}}%
\pgfpathlineto{\pgfqpoint{2.128691in}{3.132399in}}%
\pgfpathlineto{\pgfqpoint{2.132973in}{3.126663in}}%
\pgfpathlineto{\pgfqpoint{2.140198in}{3.117098in}}%
\pgfpathlineto{\pgfqpoint{2.143142in}{3.116349in}}%
\pgfpathlineto{\pgfqpoint{2.146086in}{3.118057in}}%
\pgfpathlineto{\pgfqpoint{2.150367in}{3.123819in}}%
\pgfpathlineto{\pgfqpoint{2.157593in}{3.133347in}}%
\pgfpathlineto{\pgfqpoint{2.160537in}{3.134063in}}%
\pgfpathlineto{\pgfqpoint{2.163480in}{3.132324in}}%
\pgfpathlineto{\pgfqpoint{2.167762in}{3.126537in}}%
\pgfpathlineto{\pgfqpoint{2.174987in}{3.117046in}}%
\pgfpathlineto{\pgfqpoint{2.177931in}{3.116364in}}%
\pgfpathlineto{\pgfqpoint{2.180875in}{3.118134in}}%
\pgfpathlineto{\pgfqpoint{2.185156in}{3.123944in}}%
\pgfpathlineto{\pgfqpoint{2.192114in}{3.133218in}}%
\pgfpathlineto{\pgfqpoint{2.195326in}{3.134047in}}%
\pgfpathlineto{\pgfqpoint{2.198269in}{3.132247in}}%
\pgfpathlineto{\pgfqpoint{2.202551in}{3.126412in}}%
\pgfpathlineto{\pgfqpoint{2.209509in}{3.117173in}}%
\pgfpathlineto{\pgfqpoint{2.212720in}{3.116380in}}%
\pgfpathlineto{\pgfqpoint{2.215664in}{3.118211in}}%
\pgfpathlineto{\pgfqpoint{2.219945in}{3.124070in}}%
\pgfpathlineto{\pgfqpoint{2.226903in}{3.133273in}}%
\pgfpathlineto{\pgfqpoint{2.229847in}{3.134081in}}%
\pgfpathlineto{\pgfqpoint{2.232791in}{3.132429in}}%
\pgfpathlineto{\pgfqpoint{2.237072in}{3.126713in}}%
\pgfpathlineto{\pgfqpoint{2.244298in}{3.117119in}}%
\pgfpathlineto{\pgfqpoint{2.247241in}{3.116344in}}%
\pgfpathlineto{\pgfqpoint{2.250185in}{3.118027in}}%
\pgfpathlineto{\pgfqpoint{2.254467in}{3.123769in}}%
\pgfpathlineto{\pgfqpoint{2.261692in}{3.133326in}}%
\pgfpathlineto{\pgfqpoint{2.264636in}{3.134068in}}%
\pgfpathlineto{\pgfqpoint{2.267580in}{3.132354in}}%
\pgfpathlineto{\pgfqpoint{2.271861in}{3.126588in}}%
\pgfpathlineto{\pgfqpoint{2.279087in}{3.117067in}}%
\pgfpathlineto{\pgfqpoint{2.282031in}{3.116358in}}%
\pgfpathlineto{\pgfqpoint{2.284974in}{3.118103in}}%
\pgfpathlineto{\pgfqpoint{2.289256in}{3.123894in}}%
\pgfpathlineto{\pgfqpoint{2.296481in}{3.133378in}}%
\pgfpathlineto{\pgfqpoint{2.299425in}{3.134054in}}%
\pgfpathlineto{\pgfqpoint{2.302369in}{3.132278in}}%
\pgfpathlineto{\pgfqpoint{2.306650in}{3.126462in}}%
\pgfpathlineto{\pgfqpoint{2.313608in}{3.117195in}}%
\pgfpathlineto{\pgfqpoint{2.316820in}{3.116373in}}%
\pgfpathlineto{\pgfqpoint{2.319763in}{3.118180in}}%
\pgfpathlineto{\pgfqpoint{2.324045in}{3.124020in}}%
\pgfpathlineto{\pgfqpoint{2.331003in}{3.133251in}}%
\pgfpathlineto{\pgfqpoint{2.334214in}{3.134037in}}%
\pgfpathlineto{\pgfqpoint{2.337158in}{3.132200in}}%
\pgfpathlineto{\pgfqpoint{2.341439in}{3.126336in}}%
\pgfpathlineto{\pgfqpoint{2.348397in}{3.117141in}}%
\pgfpathlineto{\pgfqpoint{2.351341in}{3.116339in}}%
\pgfpathlineto{\pgfqpoint{2.354285in}{3.117997in}}%
\pgfpathlineto{\pgfqpoint{2.358566in}{3.123718in}}%
\pgfpathlineto{\pgfqpoint{2.365792in}{3.133305in}}%
\pgfpathlineto{\pgfqpoint{2.368735in}{3.134074in}}%
\pgfpathlineto{\pgfqpoint{2.371679in}{3.132384in}}%
\pgfpathlineto{\pgfqpoint{2.375961in}{3.126638in}}%
\pgfpathlineto{\pgfqpoint{2.383186in}{3.117088in}}%
\pgfpathlineto{\pgfqpoint{2.386130in}{3.116352in}}%
\pgfpathlineto{\pgfqpoint{2.389074in}{3.118072in}}%
\pgfpathlineto{\pgfqpoint{2.393355in}{3.123844in}}%
\pgfpathlineto{\pgfqpoint{2.400581in}{3.133357in}}%
\pgfpathlineto{\pgfqpoint{2.403524in}{3.134060in}}%
\pgfpathlineto{\pgfqpoint{2.406468in}{3.132308in}}%
\pgfpathlineto{\pgfqpoint{2.410750in}{3.126512in}}%
\pgfpathlineto{\pgfqpoint{2.417975in}{3.117036in}}%
\pgfpathlineto{\pgfqpoint{2.420919in}{3.116367in}}%
\pgfpathlineto{\pgfqpoint{2.423863in}{3.118149in}}%
\pgfpathlineto{\pgfqpoint{2.428144in}{3.123969in}}%
\pgfpathlineto{\pgfqpoint{2.435102in}{3.133229in}}%
\pgfpathlineto{\pgfqpoint{2.438313in}{3.134044in}}%
\pgfpathlineto{\pgfqpoint{2.441257in}{3.132231in}}%
\pgfpathlineto{\pgfqpoint{2.445539in}{3.126387in}}%
\pgfpathlineto{\pgfqpoint{2.452497in}{3.117162in}}%
\pgfpathlineto{\pgfqpoint{2.455440in}{3.116334in}}%
\pgfpathlineto{\pgfqpoint{2.458384in}{3.117967in}}%
\pgfpathlineto{\pgfqpoint{2.462666in}{3.123668in}}%
\pgfpathlineto{\pgfqpoint{2.469891in}{3.133284in}}%
\pgfpathlineto{\pgfqpoint{2.472835in}{3.134079in}}%
\pgfpathlineto{\pgfqpoint{2.475779in}{3.132414in}}%
\pgfpathlineto{\pgfqpoint{2.480060in}{3.126688in}}%
\pgfpathlineto{\pgfqpoint{2.487286in}{3.117109in}}%
\pgfpathlineto{\pgfqpoint{2.490229in}{3.116347in}}%
\pgfpathlineto{\pgfqpoint{2.493173in}{3.118042in}}%
\pgfpathlineto{\pgfqpoint{2.497455in}{3.123794in}}%
\pgfpathlineto{\pgfqpoint{2.504680in}{3.133337in}}%
\pgfpathlineto{\pgfqpoint{2.507624in}{3.134066in}}%
\pgfpathlineto{\pgfqpoint{2.510568in}{3.132339in}}%
\pgfpathlineto{\pgfqpoint{2.514849in}{3.126563in}}%
\pgfpathlineto{\pgfqpoint{2.522075in}{3.117057in}}%
\pgfpathlineto{\pgfqpoint{2.525018in}{3.116361in}}%
\pgfpathlineto{\pgfqpoint{2.527962in}{3.118118in}}%
\pgfpathlineto{\pgfqpoint{2.532244in}{3.123919in}}%
\pgfpathlineto{\pgfqpoint{2.539202in}{3.133207in}}%
\pgfpathlineto{\pgfqpoint{2.542413in}{3.134051in}}%
\pgfpathlineto{\pgfqpoint{2.545357in}{3.132262in}}%
\pgfpathlineto{\pgfqpoint{2.549638in}{3.126437in}}%
\pgfpathlineto{\pgfqpoint{2.556596in}{3.117184in}}%
\pgfpathlineto{\pgfqpoint{2.559807in}{3.116376in}}%
\pgfpathlineto{\pgfqpoint{2.562751in}{3.118196in}}%
\pgfpathlineto{\pgfqpoint{2.567033in}{3.124045in}}%
\pgfpathlineto{\pgfqpoint{2.573991in}{3.133262in}}%
\pgfpathlineto{\pgfqpoint{2.576934in}{3.134083in}}%
\pgfpathlineto{\pgfqpoint{2.579878in}{3.132444in}}%
\pgfpathlineto{\pgfqpoint{2.584160in}{3.126738in}}%
\pgfpathlineto{\pgfqpoint{2.591385in}{3.117130in}}%
\pgfpathlineto{\pgfqpoint{2.594329in}{3.116341in}}%
\pgfpathlineto{\pgfqpoint{2.597273in}{3.118012in}}%
\pgfpathlineto{\pgfqpoint{2.601554in}{3.123743in}}%
\pgfpathlineto{\pgfqpoint{2.608780in}{3.133316in}}%
\pgfpathlineto{\pgfqpoint{2.611723in}{3.134071in}}%
\pgfpathlineto{\pgfqpoint{2.614667in}{3.132369in}}%
\pgfpathlineto{\pgfqpoint{2.618949in}{3.126613in}}%
\pgfpathlineto{\pgfqpoint{2.626174in}{3.117077in}}%
\pgfpathlineto{\pgfqpoint{2.629118in}{3.116355in}}%
\pgfpathlineto{\pgfqpoint{2.632062in}{3.118088in}}%
\pgfpathlineto{\pgfqpoint{2.636343in}{3.123869in}}%
\pgfpathlineto{\pgfqpoint{2.643569in}{3.133367in}}%
\pgfpathlineto{\pgfqpoint{2.646512in}{3.134057in}}%
\pgfpathlineto{\pgfqpoint{2.649456in}{3.132293in}}%
\pgfpathlineto{\pgfqpoint{2.653738in}{3.126487in}}%
\pgfpathlineto{\pgfqpoint{2.660696in}{3.117206in}}%
\pgfpathlineto{\pgfqpoint{2.663907in}{3.116370in}}%
\pgfpathlineto{\pgfqpoint{2.666851in}{3.118165in}}%
\pgfpathlineto{\pgfqpoint{2.671132in}{3.123994in}}%
\pgfpathlineto{\pgfqpoint{2.678090in}{3.133240in}}%
\pgfpathlineto{\pgfqpoint{2.681301in}{3.134041in}}%
\pgfpathlineto{\pgfqpoint{2.684245in}{3.132215in}}%
\pgfpathlineto{\pgfqpoint{2.688527in}{3.126361in}}%
\pgfpathlineto{\pgfqpoint{2.695485in}{3.117151in}}%
\pgfpathlineto{\pgfqpoint{2.698428in}{3.116337in}}%
\pgfpathlineto{\pgfqpoint{2.701372in}{3.117982in}}%
\pgfpathlineto{\pgfqpoint{2.705654in}{3.123693in}}%
\pgfpathlineto{\pgfqpoint{2.712879in}{3.133294in}}%
\pgfpathlineto{\pgfqpoint{2.715823in}{3.134076in}}%
\pgfpathlineto{\pgfqpoint{2.718766in}{3.132399in}}%
\pgfpathlineto{\pgfqpoint{2.723048in}{3.126663in}}%
\pgfpathlineto{\pgfqpoint{2.730274in}{3.117098in}}%
\pgfpathlineto{\pgfqpoint{2.733217in}{3.116349in}}%
\pgfpathlineto{\pgfqpoint{2.736161in}{3.118057in}}%
\pgfpathlineto{\pgfqpoint{2.740443in}{3.123819in}}%
\pgfpathlineto{\pgfqpoint{2.747668in}{3.133347in}}%
\pgfpathlineto{\pgfqpoint{2.750612in}{3.134063in}}%
\pgfpathlineto{\pgfqpoint{2.753555in}{3.132324in}}%
\pgfpathlineto{\pgfqpoint{2.757837in}{3.126537in}}%
\pgfpathlineto{\pgfqpoint{2.765063in}{3.117046in}}%
\pgfpathlineto{\pgfqpoint{2.768006in}{3.116364in}}%
\pgfpathlineto{\pgfqpoint{2.770950in}{3.118134in}}%
\pgfpathlineto{\pgfqpoint{2.775232in}{3.123944in}}%
\pgfpathlineto{\pgfqpoint{2.782190in}{3.133218in}}%
\pgfpathlineto{\pgfqpoint{2.785401in}{3.134047in}}%
\pgfpathlineto{\pgfqpoint{2.788345in}{3.132247in}}%
\pgfpathlineto{\pgfqpoint{2.792626in}{3.126412in}}%
\pgfpathlineto{\pgfqpoint{2.799584in}{3.117173in}}%
\pgfpathlineto{\pgfqpoint{2.802795in}{3.116380in}}%
\pgfpathlineto{\pgfqpoint{2.805739in}{3.118211in}}%
\pgfpathlineto{\pgfqpoint{2.810021in}{3.124070in}}%
\pgfpathlineto{\pgfqpoint{2.816979in}{3.133273in}}%
\pgfpathlineto{\pgfqpoint{2.819922in}{3.134081in}}%
\pgfpathlineto{\pgfqpoint{2.822866in}{3.132429in}}%
\pgfpathlineto{\pgfqpoint{2.827148in}{3.126713in}}%
\pgfpathlineto{\pgfqpoint{2.834373in}{3.117119in}}%
\pgfpathlineto{\pgfqpoint{2.837317in}{3.116344in}}%
\pgfpathlineto{\pgfqpoint{2.840260in}{3.118027in}}%
\pgfpathlineto{\pgfqpoint{2.844542in}{3.123769in}}%
\pgfpathlineto{\pgfqpoint{2.851768in}{3.133326in}}%
\pgfpathlineto{\pgfqpoint{2.854711in}{3.134068in}}%
\pgfpathlineto{\pgfqpoint{2.857655in}{3.132354in}}%
\pgfpathlineto{\pgfqpoint{2.861937in}{3.126588in}}%
\pgfpathlineto{\pgfqpoint{2.869162in}{3.117067in}}%
\pgfpathlineto{\pgfqpoint{2.872106in}{3.116358in}}%
\pgfpathlineto{\pgfqpoint{2.875049in}{3.118103in}}%
\pgfpathlineto{\pgfqpoint{2.879331in}{3.123894in}}%
\pgfpathlineto{\pgfqpoint{2.886557in}{3.133378in}}%
\pgfpathlineto{\pgfqpoint{2.889500in}{3.134054in}}%
\pgfpathlineto{\pgfqpoint{2.892444in}{3.132278in}}%
\pgfpathlineto{\pgfqpoint{2.896726in}{3.126462in}}%
\pgfpathlineto{\pgfqpoint{2.903683in}{3.117195in}}%
\pgfpathlineto{\pgfqpoint{2.906895in}{3.116373in}}%
\pgfpathlineto{\pgfqpoint{2.909838in}{3.118180in}}%
\pgfpathlineto{\pgfqpoint{2.914120in}{3.124020in}}%
\pgfpathlineto{\pgfqpoint{2.921078in}{3.133251in}}%
\pgfpathlineto{\pgfqpoint{2.924289in}{3.134037in}}%
\pgfpathlineto{\pgfqpoint{2.927233in}{3.132200in}}%
\pgfpathlineto{\pgfqpoint{2.931515in}{3.126336in}}%
\pgfpathlineto{\pgfqpoint{2.938472in}{3.117141in}}%
\pgfpathlineto{\pgfqpoint{2.941416in}{3.116339in}}%
\pgfpathlineto{\pgfqpoint{2.944360in}{3.117997in}}%
\pgfpathlineto{\pgfqpoint{2.948642in}{3.123718in}}%
\pgfpathlineto{\pgfqpoint{2.955867in}{3.133305in}}%
\pgfpathlineto{\pgfqpoint{2.958811in}{3.134074in}}%
\pgfpathlineto{\pgfqpoint{2.961754in}{3.132384in}}%
\pgfpathlineto{\pgfqpoint{2.966036in}{3.126638in}}%
\pgfpathlineto{\pgfqpoint{2.973262in}{3.117088in}}%
\pgfpathlineto{\pgfqpoint{2.976205in}{3.116352in}}%
\pgfpathlineto{\pgfqpoint{2.979149in}{3.118072in}}%
\pgfpathlineto{\pgfqpoint{2.983431in}{3.123844in}}%
\pgfpathlineto{\pgfqpoint{2.990656in}{3.133357in}}%
\pgfpathlineto{\pgfqpoint{2.993600in}{3.134060in}}%
\pgfpathlineto{\pgfqpoint{2.996543in}{3.132308in}}%
\pgfpathlineto{\pgfqpoint{3.000825in}{3.126512in}}%
\pgfpathlineto{\pgfqpoint{3.008051in}{3.117036in}}%
\pgfpathlineto{\pgfqpoint{3.010994in}{3.116367in}}%
\pgfpathlineto{\pgfqpoint{3.013938in}{3.118149in}}%
\pgfpathlineto{\pgfqpoint{3.018220in}{3.123969in}}%
\pgfpathlineto{\pgfqpoint{3.025177in}{3.133229in}}%
\pgfpathlineto{\pgfqpoint{3.028389in}{3.134044in}}%
\pgfpathlineto{\pgfqpoint{3.031332in}{3.132231in}}%
\pgfpathlineto{\pgfqpoint{3.035614in}{3.126387in}}%
\pgfpathlineto{\pgfqpoint{3.042572in}{3.117162in}}%
\pgfpathlineto{\pgfqpoint{3.045516in}{3.116334in}}%
\pgfpathlineto{\pgfqpoint{3.048459in}{3.117967in}}%
\pgfpathlineto{\pgfqpoint{3.052741in}{3.123668in}}%
\pgfpathlineto{\pgfqpoint{3.059966in}{3.133284in}}%
\pgfpathlineto{\pgfqpoint{3.062910in}{3.134079in}}%
\pgfpathlineto{\pgfqpoint{3.065854in}{3.132414in}}%
\pgfpathlineto{\pgfqpoint{3.070136in}{3.126688in}}%
\pgfpathlineto{\pgfqpoint{3.077361in}{3.117109in}}%
\pgfpathlineto{\pgfqpoint{3.080305in}{3.116347in}}%
\pgfpathlineto{\pgfqpoint{3.083248in}{3.118042in}}%
\pgfpathlineto{\pgfqpoint{3.087530in}{3.123794in}}%
\pgfpathlineto{\pgfqpoint{3.094755in}{3.133337in}}%
\pgfpathlineto{\pgfqpoint{3.097699in}{3.134066in}}%
\pgfpathlineto{\pgfqpoint{3.100643in}{3.132339in}}%
\pgfpathlineto{\pgfqpoint{3.104925in}{3.126563in}}%
\pgfpathlineto{\pgfqpoint{3.112150in}{3.117057in}}%
\pgfpathlineto{\pgfqpoint{3.115094in}{3.116361in}}%
\pgfpathlineto{\pgfqpoint{3.118037in}{3.118118in}}%
\pgfpathlineto{\pgfqpoint{3.122319in}{3.123919in}}%
\pgfpathlineto{\pgfqpoint{3.129277in}{3.133207in}}%
\pgfpathlineto{\pgfqpoint{3.132488in}{3.134051in}}%
\pgfpathlineto{\pgfqpoint{3.135432in}{3.132262in}}%
\pgfpathlineto{\pgfqpoint{3.139714in}{3.126437in}}%
\pgfpathlineto{\pgfqpoint{3.146671in}{3.117184in}}%
\pgfpathlineto{\pgfqpoint{3.149883in}{3.116376in}}%
\pgfpathlineto{\pgfqpoint{3.152826in}{3.118196in}}%
\pgfpathlineto{\pgfqpoint{3.157108in}{3.124045in}}%
\pgfpathlineto{\pgfqpoint{3.164066in}{3.133262in}}%
\pgfpathlineto{\pgfqpoint{3.167010in}{3.134083in}}%
\pgfpathlineto{\pgfqpoint{3.169953in}{3.132444in}}%
\pgfpathlineto{\pgfqpoint{3.174235in}{3.126738in}}%
\pgfpathlineto{\pgfqpoint{3.181460in}{3.117130in}}%
\pgfpathlineto{\pgfqpoint{3.184404in}{3.116341in}}%
\pgfpathlineto{\pgfqpoint{3.187348in}{3.118012in}}%
\pgfpathlineto{\pgfqpoint{3.191629in}{3.123743in}}%
\pgfpathlineto{\pgfqpoint{3.198855in}{3.133316in}}%
\pgfpathlineto{\pgfqpoint{3.201799in}{3.134071in}}%
\pgfpathlineto{\pgfqpoint{3.204742in}{3.132369in}}%
\pgfpathlineto{\pgfqpoint{3.209024in}{3.126613in}}%
\pgfpathlineto{\pgfqpoint{3.216249in}{3.117077in}}%
\pgfpathlineto{\pgfqpoint{3.219193in}{3.116355in}}%
\pgfpathlineto{\pgfqpoint{3.222137in}{3.118088in}}%
\pgfpathlineto{\pgfqpoint{3.226419in}{3.123869in}}%
\pgfpathlineto{\pgfqpoint{3.233644in}{3.133367in}}%
\pgfpathlineto{\pgfqpoint{3.236588in}{3.134057in}}%
\pgfpathlineto{\pgfqpoint{3.239531in}{3.132293in}}%
\pgfpathlineto{\pgfqpoint{3.243813in}{3.126487in}}%
\pgfpathlineto{\pgfqpoint{3.250771in}{3.117206in}}%
\pgfpathlineto{\pgfqpoint{3.253982in}{3.116370in}}%
\pgfpathlineto{\pgfqpoint{3.256926in}{3.118165in}}%
\pgfpathlineto{\pgfqpoint{3.261208in}{3.123994in}}%
\pgfpathlineto{\pgfqpoint{3.268165in}{3.133240in}}%
\pgfpathlineto{\pgfqpoint{3.271377in}{3.134041in}}%
\pgfpathlineto{\pgfqpoint{3.274320in}{3.132215in}}%
\pgfpathlineto{\pgfqpoint{3.278602in}{3.126361in}}%
\pgfpathlineto{\pgfqpoint{3.285560in}{3.117151in}}%
\pgfpathlineto{\pgfqpoint{3.288504in}{3.116337in}}%
\pgfpathlineto{\pgfqpoint{3.291447in}{3.117982in}}%
\pgfpathlineto{\pgfqpoint{3.295729in}{3.123693in}}%
\pgfpathlineto{\pgfqpoint{3.302954in}{3.133294in}}%
\pgfpathlineto{\pgfqpoint{3.305898in}{3.134076in}}%
\pgfpathlineto{\pgfqpoint{3.308842in}{3.132399in}}%
\pgfpathlineto{\pgfqpoint{3.313123in}{3.126663in}}%
\pgfpathlineto{\pgfqpoint{3.320349in}{3.117098in}}%
\pgfpathlineto{\pgfqpoint{3.323293in}{3.116349in}}%
\pgfpathlineto{\pgfqpoint{3.326236in}{3.118057in}}%
\pgfpathlineto{\pgfqpoint{3.330518in}{3.123819in}}%
\pgfpathlineto{\pgfqpoint{3.337743in}{3.133347in}}%
\pgfpathlineto{\pgfqpoint{3.340687in}{3.134063in}}%
\pgfpathlineto{\pgfqpoint{3.343631in}{3.132324in}}%
\pgfpathlineto{\pgfqpoint{3.347912in}{3.126537in}}%
\pgfpathlineto{\pgfqpoint{3.355138in}{3.117046in}}%
\pgfpathlineto{\pgfqpoint{3.358082in}{3.116364in}}%
\pgfpathlineto{\pgfqpoint{3.361025in}{3.118134in}}%
\pgfpathlineto{\pgfqpoint{3.365307in}{3.123944in}}%
\pgfpathlineto{\pgfqpoint{3.372265in}{3.133218in}}%
\pgfpathlineto{\pgfqpoint{3.375476in}{3.134047in}}%
\pgfpathlineto{\pgfqpoint{3.378420in}{3.132247in}}%
\pgfpathlineto{\pgfqpoint{3.382701in}{3.126412in}}%
\pgfpathlineto{\pgfqpoint{3.389659in}{3.117173in}}%
\pgfpathlineto{\pgfqpoint{3.392871in}{3.116380in}}%
\pgfpathlineto{\pgfqpoint{3.395814in}{3.118211in}}%
\pgfpathlineto{\pgfqpoint{3.400096in}{3.124070in}}%
\pgfpathlineto{\pgfqpoint{3.407054in}{3.133273in}}%
\pgfpathlineto{\pgfqpoint{3.409997in}{3.134081in}}%
\pgfpathlineto{\pgfqpoint{3.412941in}{3.132429in}}%
\pgfpathlineto{\pgfqpoint{3.417223in}{3.126713in}}%
\pgfpathlineto{\pgfqpoint{3.424448in}{3.117119in}}%
\pgfpathlineto{\pgfqpoint{3.427392in}{3.116344in}}%
\pgfpathlineto{\pgfqpoint{3.430336in}{3.118027in}}%
\pgfpathlineto{\pgfqpoint{3.434617in}{3.123769in}}%
\pgfpathlineto{\pgfqpoint{3.441843in}{3.133326in}}%
\pgfpathlineto{\pgfqpoint{3.444786in}{3.134068in}}%
\pgfpathlineto{\pgfqpoint{3.447730in}{3.132354in}}%
\pgfpathlineto{\pgfqpoint{3.452012in}{3.126588in}}%
\pgfpathlineto{\pgfqpoint{3.459237in}{3.117067in}}%
\pgfpathlineto{\pgfqpoint{3.462181in}{3.116358in}}%
\pgfpathlineto{\pgfqpoint{3.465125in}{3.118103in}}%
\pgfpathlineto{\pgfqpoint{3.469406in}{3.123894in}}%
\pgfpathlineto{\pgfqpoint{3.476632in}{3.133378in}}%
\pgfpathlineto{\pgfqpoint{3.479576in}{3.134054in}}%
\pgfpathlineto{\pgfqpoint{3.482519in}{3.132278in}}%
\pgfpathlineto{\pgfqpoint{3.486801in}{3.126462in}}%
\pgfpathlineto{\pgfqpoint{3.493759in}{3.117195in}}%
\pgfpathlineto{\pgfqpoint{3.496970in}{3.116373in}}%
\pgfpathlineto{\pgfqpoint{3.499914in}{3.118180in}}%
\pgfpathlineto{\pgfqpoint{3.504195in}{3.124020in}}%
\pgfpathlineto{\pgfqpoint{3.511153in}{3.133251in}}%
\pgfpathlineto{\pgfqpoint{3.514365in}{3.134037in}}%
\pgfpathlineto{\pgfqpoint{3.517308in}{3.132200in}}%
\pgfpathlineto{\pgfqpoint{3.521590in}{3.126336in}}%
\pgfpathlineto{\pgfqpoint{3.528548in}{3.117141in}}%
\pgfpathlineto{\pgfqpoint{3.531491in}{3.116339in}}%
\pgfpathlineto{\pgfqpoint{3.534435in}{3.117997in}}%
\pgfpathlineto{\pgfqpoint{3.538717in}{3.123718in}}%
\pgfpathlineto{\pgfqpoint{3.545942in}{3.133305in}}%
\pgfpathlineto{\pgfqpoint{3.548886in}{3.134074in}}%
\pgfpathlineto{\pgfqpoint{3.551830in}{3.132384in}}%
\pgfpathlineto{\pgfqpoint{3.556111in}{3.126638in}}%
\pgfpathlineto{\pgfqpoint{3.563337in}{3.117088in}}%
\pgfpathlineto{\pgfqpoint{3.566280in}{3.116352in}}%
\pgfpathlineto{\pgfqpoint{3.569224in}{3.118072in}}%
\pgfpathlineto{\pgfqpoint{3.573506in}{3.123844in}}%
\pgfpathlineto{\pgfqpoint{3.580731in}{3.133357in}}%
\pgfpathlineto{\pgfqpoint{3.583675in}{3.134060in}}%
\pgfpathlineto{\pgfqpoint{3.586619in}{3.132308in}}%
\pgfpathlineto{\pgfqpoint{3.590900in}{3.126512in}}%
\pgfpathlineto{\pgfqpoint{3.598126in}{3.117036in}}%
\pgfpathlineto{\pgfqpoint{3.601069in}{3.116367in}}%
\pgfpathlineto{\pgfqpoint{3.604013in}{3.118149in}}%
\pgfpathlineto{\pgfqpoint{3.608295in}{3.123969in}}%
\pgfpathlineto{\pgfqpoint{3.615253in}{3.133229in}}%
\pgfpathlineto{\pgfqpoint{3.618464in}{3.134044in}}%
\pgfpathlineto{\pgfqpoint{3.621408in}{3.132231in}}%
\pgfpathlineto{\pgfqpoint{3.625689in}{3.126387in}}%
\pgfpathlineto{\pgfqpoint{3.632647in}{3.117162in}}%
\pgfpathlineto{\pgfqpoint{3.635591in}{3.116334in}}%
\pgfpathlineto{\pgfqpoint{3.638535in}{3.117967in}}%
\pgfpathlineto{\pgfqpoint{3.642816in}{3.123668in}}%
\pgfpathlineto{\pgfqpoint{3.650042in}{3.133284in}}%
\pgfpathlineto{\pgfqpoint{3.652985in}{3.134079in}}%
\pgfpathlineto{\pgfqpoint{3.655929in}{3.132414in}}%
\pgfpathlineto{\pgfqpoint{3.660211in}{3.126688in}}%
\pgfpathlineto{\pgfqpoint{3.667436in}{3.117109in}}%
\pgfpathlineto{\pgfqpoint{3.670380in}{3.116347in}}%
\pgfpathlineto{\pgfqpoint{3.673324in}{3.118042in}}%
\pgfpathlineto{\pgfqpoint{3.677605in}{3.123794in}}%
\pgfpathlineto{\pgfqpoint{3.684831in}{3.133337in}}%
\pgfpathlineto{\pgfqpoint{3.687774in}{3.134066in}}%
\pgfpathlineto{\pgfqpoint{3.690718in}{3.132339in}}%
\pgfpathlineto{\pgfqpoint{3.695000in}{3.126563in}}%
\pgfpathlineto{\pgfqpoint{3.702225in}{3.117057in}}%
\pgfpathlineto{\pgfqpoint{3.705169in}{3.116361in}}%
\pgfpathlineto{\pgfqpoint{3.708113in}{3.118118in}}%
\pgfpathlineto{\pgfqpoint{3.712394in}{3.123919in}}%
\pgfpathlineto{\pgfqpoint{3.719352in}{3.133207in}}%
\pgfpathlineto{\pgfqpoint{3.722563in}{3.134051in}}%
\pgfpathlineto{\pgfqpoint{3.725507in}{3.132262in}}%
\pgfpathlineto{\pgfqpoint{3.729789in}{3.126437in}}%
\pgfpathlineto{\pgfqpoint{3.736747in}{3.117184in}}%
\pgfpathlineto{\pgfqpoint{3.739958in}{3.116376in}}%
\pgfpathlineto{\pgfqpoint{3.742902in}{3.118196in}}%
\pgfpathlineto{\pgfqpoint{3.747183in}{3.124045in}}%
\pgfpathlineto{\pgfqpoint{3.754141in}{3.133262in}}%
\pgfpathlineto{\pgfqpoint{3.757085in}{3.134083in}}%
\pgfpathlineto{\pgfqpoint{3.760029in}{3.132444in}}%
\pgfpathlineto{\pgfqpoint{3.764310in}{3.126738in}}%
\pgfpathlineto{\pgfqpoint{3.771536in}{3.117130in}}%
\pgfpathlineto{\pgfqpoint{3.774479in}{3.116341in}}%
\pgfpathlineto{\pgfqpoint{3.777423in}{3.118012in}}%
\pgfpathlineto{\pgfqpoint{3.781705in}{3.123743in}}%
\pgfpathlineto{\pgfqpoint{3.788930in}{3.133316in}}%
\pgfpathlineto{\pgfqpoint{3.791874in}{3.134071in}}%
\pgfpathlineto{\pgfqpoint{3.794818in}{3.132369in}}%
\pgfpathlineto{\pgfqpoint{3.799099in}{3.126613in}}%
\pgfpathlineto{\pgfqpoint{3.806325in}{3.117077in}}%
\pgfpathlineto{\pgfqpoint{3.809268in}{3.116355in}}%
\pgfpathlineto{\pgfqpoint{3.812212in}{3.118088in}}%
\pgfpathlineto{\pgfqpoint{3.816494in}{3.123869in}}%
\pgfpathlineto{\pgfqpoint{3.823719in}{3.133367in}}%
\pgfpathlineto{\pgfqpoint{3.826663in}{3.134057in}}%
\pgfpathlineto{\pgfqpoint{3.829607in}{3.132293in}}%
\pgfpathlineto{\pgfqpoint{3.833888in}{3.126487in}}%
\pgfpathlineto{\pgfqpoint{3.840846in}{3.117206in}}%
\pgfpathlineto{\pgfqpoint{3.844057in}{3.116370in}}%
\pgfpathlineto{\pgfqpoint{3.847001in}{3.118165in}}%
\pgfpathlineto{\pgfqpoint{3.851283in}{3.123994in}}%
\pgfpathlineto{\pgfqpoint{3.858241in}{3.133240in}}%
\pgfpathlineto{\pgfqpoint{3.861452in}{3.134041in}}%
\pgfpathlineto{\pgfqpoint{3.864396in}{3.132215in}}%
\pgfpathlineto{\pgfqpoint{3.868677in}{3.126361in}}%
\pgfpathlineto{\pgfqpoint{3.875635in}{3.117151in}}%
\pgfpathlineto{\pgfqpoint{3.878579in}{3.116337in}}%
\pgfpathlineto{\pgfqpoint{3.881522in}{3.117982in}}%
\pgfpathlineto{\pgfqpoint{3.885804in}{3.123693in}}%
\pgfpathlineto{\pgfqpoint{3.893030in}{3.133294in}}%
\pgfpathlineto{\pgfqpoint{3.895973in}{3.134076in}}%
\pgfpathlineto{\pgfqpoint{3.898917in}{3.132399in}}%
\pgfpathlineto{\pgfqpoint{3.903199in}{3.126663in}}%
\pgfpathlineto{\pgfqpoint{3.910424in}{3.117098in}}%
\pgfpathlineto{\pgfqpoint{3.913368in}{3.116349in}}%
\pgfpathlineto{\pgfqpoint{3.916311in}{3.118057in}}%
\pgfpathlineto{\pgfqpoint{3.920593in}{3.123819in}}%
\pgfpathlineto{\pgfqpoint{3.927819in}{3.133347in}}%
\pgfpathlineto{\pgfqpoint{3.930762in}{3.134063in}}%
\pgfpathlineto{\pgfqpoint{3.933706in}{3.132324in}}%
\pgfpathlineto{\pgfqpoint{3.937988in}{3.126537in}}%
\pgfpathlineto{\pgfqpoint{3.945213in}{3.117046in}}%
\pgfpathlineto{\pgfqpoint{3.948157in}{3.116364in}}%
\pgfpathlineto{\pgfqpoint{3.951100in}{3.118134in}}%
\pgfpathlineto{\pgfqpoint{3.955382in}{3.123944in}}%
\pgfpathlineto{\pgfqpoint{3.962340in}{3.133218in}}%
\pgfpathlineto{\pgfqpoint{3.965551in}{3.134047in}}%
\pgfpathlineto{\pgfqpoint{3.968495in}{3.132247in}}%
\pgfpathlineto{\pgfqpoint{3.972777in}{3.126412in}}%
\pgfpathlineto{\pgfqpoint{3.979735in}{3.117173in}}%
\pgfpathlineto{\pgfqpoint{3.982946in}{3.116380in}}%
\pgfpathlineto{\pgfqpoint{3.985890in}{3.118211in}}%
\pgfpathlineto{\pgfqpoint{3.990171in}{3.124070in}}%
\pgfpathlineto{\pgfqpoint{3.997129in}{3.133273in}}%
\pgfpathlineto{\pgfqpoint{4.000073in}{3.134081in}}%
\pgfpathlineto{\pgfqpoint{4.003016in}{3.132429in}}%
\pgfpathlineto{\pgfqpoint{4.007298in}{3.126713in}}%
\pgfpathlineto{\pgfqpoint{4.014524in}{3.117119in}}%
\pgfpathlineto{\pgfqpoint{4.017467in}{3.116344in}}%
\pgfpathlineto{\pgfqpoint{4.020411in}{3.118027in}}%
\pgfpathlineto{\pgfqpoint{4.024693in}{3.123769in}}%
\pgfpathlineto{\pgfqpoint{4.031918in}{3.133326in}}%
\pgfpathlineto{\pgfqpoint{4.034862in}{3.134068in}}%
\pgfpathlineto{\pgfqpoint{4.037805in}{3.132354in}}%
\pgfpathlineto{\pgfqpoint{4.042087in}{3.126588in}}%
\pgfpathlineto{\pgfqpoint{4.049313in}{3.117067in}}%
\pgfpathlineto{\pgfqpoint{4.052256in}{3.116358in}}%
\pgfpathlineto{\pgfqpoint{4.055200in}{3.118103in}}%
\pgfpathlineto{\pgfqpoint{4.059482in}{3.123894in}}%
\pgfpathlineto{\pgfqpoint{4.066707in}{3.133378in}}%
\pgfpathlineto{\pgfqpoint{4.069651in}{3.134054in}}%
\pgfpathlineto{\pgfqpoint{4.072594in}{3.132278in}}%
\pgfpathlineto{\pgfqpoint{4.076876in}{3.126462in}}%
\pgfpathlineto{\pgfqpoint{4.083834in}{3.117195in}}%
\pgfpathlineto{\pgfqpoint{4.087045in}{3.116373in}}%
\pgfpathlineto{\pgfqpoint{4.089989in}{3.118180in}}%
\pgfpathlineto{\pgfqpoint{4.094271in}{3.124020in}}%
\pgfpathlineto{\pgfqpoint{4.101228in}{3.133251in}}%
\pgfpathlineto{\pgfqpoint{4.104440in}{3.134037in}}%
\pgfpathlineto{\pgfqpoint{4.107383in}{3.132200in}}%
\pgfpathlineto{\pgfqpoint{4.111665in}{3.126336in}}%
\pgfpathlineto{\pgfqpoint{4.118623in}{3.117141in}}%
\pgfpathlineto{\pgfqpoint{4.121567in}{3.116339in}}%
\pgfpathlineto{\pgfqpoint{4.124510in}{3.117997in}}%
\pgfpathlineto{\pgfqpoint{4.128792in}{3.123718in}}%
\pgfpathlineto{\pgfqpoint{4.136017in}{3.133305in}}%
\pgfpathlineto{\pgfqpoint{4.138961in}{3.134074in}}%
\pgfpathlineto{\pgfqpoint{4.141905in}{3.132384in}}%
\pgfpathlineto{\pgfqpoint{4.146187in}{3.126638in}}%
\pgfpathlineto{\pgfqpoint{4.153412in}{3.117088in}}%
\pgfpathlineto{\pgfqpoint{4.156356in}{3.116352in}}%
\pgfpathlineto{\pgfqpoint{4.159299in}{3.118072in}}%
\pgfpathlineto{\pgfqpoint{4.163581in}{3.123844in}}%
\pgfpathlineto{\pgfqpoint{4.170807in}{3.133357in}}%
\pgfpathlineto{\pgfqpoint{4.173750in}{3.134060in}}%
\pgfpathlineto{\pgfqpoint{4.176694in}{3.132308in}}%
\pgfpathlineto{\pgfqpoint{4.180976in}{3.126512in}}%
\pgfpathlineto{\pgfqpoint{4.188201in}{3.117036in}}%
\pgfpathlineto{\pgfqpoint{4.191145in}{3.116367in}}%
\pgfpathlineto{\pgfqpoint{4.194088in}{3.118149in}}%
\pgfpathlineto{\pgfqpoint{4.198370in}{3.123969in}}%
\pgfpathlineto{\pgfqpoint{4.205328in}{3.133229in}}%
\pgfpathlineto{\pgfqpoint{4.208539in}{3.134044in}}%
\pgfpathlineto{\pgfqpoint{4.211483in}{3.132231in}}%
\pgfpathlineto{\pgfqpoint{4.215765in}{3.126387in}}%
\pgfpathlineto{\pgfqpoint{4.222722in}{3.117162in}}%
\pgfpathlineto{\pgfqpoint{4.225666in}{3.116334in}}%
\pgfpathlineto{\pgfqpoint{4.228610in}{3.117967in}}%
\pgfpathlineto{\pgfqpoint{4.232892in}{3.123668in}}%
\pgfpathlineto{\pgfqpoint{4.240117in}{3.133284in}}%
\pgfpathlineto{\pgfqpoint{4.243061in}{3.134079in}}%
\pgfpathlineto{\pgfqpoint{4.246004in}{3.132414in}}%
\pgfpathlineto{\pgfqpoint{4.250286in}{3.126688in}}%
\pgfpathlineto{\pgfqpoint{4.257511in}{3.117109in}}%
\pgfpathlineto{\pgfqpoint{4.260455in}{3.116347in}}%
\pgfpathlineto{\pgfqpoint{4.263399in}{3.118042in}}%
\pgfpathlineto{\pgfqpoint{4.267681in}{3.123794in}}%
\pgfpathlineto{\pgfqpoint{4.274906in}{3.133337in}}%
\pgfpathlineto{\pgfqpoint{4.277850in}{3.134066in}}%
\pgfpathlineto{\pgfqpoint{4.280793in}{3.132339in}}%
\pgfpathlineto{\pgfqpoint{4.285075in}{3.126563in}}%
\pgfpathlineto{\pgfqpoint{4.292300in}{3.117057in}}%
\pgfpathlineto{\pgfqpoint{4.295244in}{3.116361in}}%
\pgfpathlineto{\pgfqpoint{4.298188in}{3.118118in}}%
\pgfpathlineto{\pgfqpoint{4.302470in}{3.123919in}}%
\pgfpathlineto{\pgfqpoint{4.309427in}{3.133207in}}%
\pgfpathlineto{\pgfqpoint{4.312639in}{3.134051in}}%
\pgfpathlineto{\pgfqpoint{4.315582in}{3.132262in}}%
\pgfpathlineto{\pgfqpoint{4.319864in}{3.126437in}}%
\pgfpathlineto{\pgfqpoint{4.326822in}{3.117184in}}%
\pgfpathlineto{\pgfqpoint{4.330033in}{3.116376in}}%
\pgfpathlineto{\pgfqpoint{4.332977in}{3.118196in}}%
\pgfpathlineto{\pgfqpoint{4.337259in}{3.124045in}}%
\pgfpathlineto{\pgfqpoint{4.344216in}{3.133262in}}%
\pgfpathlineto{\pgfqpoint{4.347160in}{3.134083in}}%
\pgfpathlineto{\pgfqpoint{4.350104in}{3.132444in}}%
\pgfpathlineto{\pgfqpoint{4.354385in}{3.126738in}}%
\pgfpathlineto{\pgfqpoint{4.361611in}{3.117130in}}%
\pgfpathlineto{\pgfqpoint{4.364555in}{3.116341in}}%
\pgfpathlineto{\pgfqpoint{4.367498in}{3.118012in}}%
\pgfpathlineto{\pgfqpoint{4.371780in}{3.123743in}}%
\pgfpathlineto{\pgfqpoint{4.379005in}{3.133316in}}%
\pgfpathlineto{\pgfqpoint{4.381949in}{3.134071in}}%
\pgfpathlineto{\pgfqpoint{4.384893in}{3.132369in}}%
\pgfpathlineto{\pgfqpoint{4.389174in}{3.126613in}}%
\pgfpathlineto{\pgfqpoint{4.396400in}{3.117077in}}%
\pgfpathlineto{\pgfqpoint{4.399344in}{3.116355in}}%
\pgfpathlineto{\pgfqpoint{4.402287in}{3.118088in}}%
\pgfpathlineto{\pgfqpoint{4.406569in}{3.123869in}}%
\pgfpathlineto{\pgfqpoint{4.413794in}{3.133367in}}%
\pgfpathlineto{\pgfqpoint{4.416738in}{3.134057in}}%
\pgfpathlineto{\pgfqpoint{4.419682in}{3.132293in}}%
\pgfpathlineto{\pgfqpoint{4.423964in}{3.126487in}}%
\pgfpathlineto{\pgfqpoint{4.430921in}{3.117206in}}%
\pgfpathlineto{\pgfqpoint{4.434133in}{3.116370in}}%
\pgfpathlineto{\pgfqpoint{4.437076in}{3.118165in}}%
\pgfpathlineto{\pgfqpoint{4.441358in}{3.123994in}}%
\pgfpathlineto{\pgfqpoint{4.448316in}{3.133240in}}%
\pgfpathlineto{\pgfqpoint{4.451527in}{3.134041in}}%
\pgfpathlineto{\pgfqpoint{4.454471in}{3.132215in}}%
\pgfpathlineto{\pgfqpoint{4.458753in}{3.126361in}}%
\pgfpathlineto{\pgfqpoint{4.465710in}{3.117151in}}%
\pgfpathlineto{\pgfqpoint{4.468654in}{3.116337in}}%
\pgfpathlineto{\pgfqpoint{4.471598in}{3.117982in}}%
\pgfpathlineto{\pgfqpoint{4.475879in}{3.123693in}}%
\pgfpathlineto{\pgfqpoint{4.483105in}{3.133294in}}%
\pgfpathlineto{\pgfqpoint{4.486049in}{3.134076in}}%
\pgfpathlineto{\pgfqpoint{4.488992in}{3.132399in}}%
\pgfpathlineto{\pgfqpoint{4.493274in}{3.126663in}}%
\pgfpathlineto{\pgfqpoint{4.500499in}{3.117098in}}%
\pgfpathlineto{\pgfqpoint{4.503443in}{3.116349in}}%
\pgfpathlineto{\pgfqpoint{4.506387in}{3.118057in}}%
\pgfpathlineto{\pgfqpoint{4.510668in}{3.123819in}}%
\pgfpathlineto{\pgfqpoint{4.517894in}{3.133347in}}%
\pgfpathlineto{\pgfqpoint{4.520838in}{3.134063in}}%
\pgfpathlineto{\pgfqpoint{4.523781in}{3.132324in}}%
\pgfpathlineto{\pgfqpoint{4.528063in}{3.126537in}}%
\pgfpathlineto{\pgfqpoint{4.535288in}{3.117046in}}%
\pgfpathlineto{\pgfqpoint{4.538232in}{3.116364in}}%
\pgfpathlineto{\pgfqpoint{4.541176in}{3.118134in}}%
\pgfpathlineto{\pgfqpoint{4.545457in}{3.123944in}}%
\pgfpathlineto{\pgfqpoint{4.552415in}{3.133218in}}%
\pgfpathlineto{\pgfqpoint{4.555627in}{3.134047in}}%
\pgfpathlineto{\pgfqpoint{4.558570in}{3.132247in}}%
\pgfpathlineto{\pgfqpoint{4.562852in}{3.126412in}}%
\pgfpathlineto{\pgfqpoint{4.569810in}{3.117173in}}%
\pgfpathlineto{\pgfqpoint{4.573021in}{3.116380in}}%
\pgfpathlineto{\pgfqpoint{4.575965in}{3.118211in}}%
\pgfpathlineto{\pgfqpoint{4.580246in}{3.124070in}}%
\pgfpathlineto{\pgfqpoint{4.587204in}{3.133273in}}%
\pgfpathlineto{\pgfqpoint{4.590148in}{3.134081in}}%
\pgfpathlineto{\pgfqpoint{4.593092in}{3.132429in}}%
\pgfpathlineto{\pgfqpoint{4.597373in}{3.126713in}}%
\pgfpathlineto{\pgfqpoint{4.604599in}{3.117119in}}%
\pgfpathlineto{\pgfqpoint{4.607542in}{3.116344in}}%
\pgfpathlineto{\pgfqpoint{4.610486in}{3.118027in}}%
\pgfpathlineto{\pgfqpoint{4.614768in}{3.123769in}}%
\pgfpathlineto{\pgfqpoint{4.621993in}{3.133326in}}%
\pgfpathlineto{\pgfqpoint{4.624937in}{3.134068in}}%
\pgfpathlineto{\pgfqpoint{4.627881in}{3.132354in}}%
\pgfpathlineto{\pgfqpoint{4.632162in}{3.126588in}}%
\pgfpathlineto{\pgfqpoint{4.639388in}{3.117067in}}%
\pgfpathlineto{\pgfqpoint{4.642331in}{3.116358in}}%
\pgfpathlineto{\pgfqpoint{4.645275in}{3.118103in}}%
\pgfpathlineto{\pgfqpoint{4.649557in}{3.123894in}}%
\pgfpathlineto{\pgfqpoint{4.656782in}{3.133378in}}%
\pgfpathlineto{\pgfqpoint{4.659726in}{3.134054in}}%
\pgfpathlineto{\pgfqpoint{4.662670in}{3.132278in}}%
\pgfpathlineto{\pgfqpoint{4.666951in}{3.126462in}}%
\pgfpathlineto{\pgfqpoint{4.673909in}{3.117195in}}%
\pgfpathlineto{\pgfqpoint{4.677121in}{3.116373in}}%
\pgfpathlineto{\pgfqpoint{4.680064in}{3.118180in}}%
\pgfpathlineto{\pgfqpoint{4.684346in}{3.124020in}}%
\pgfpathlineto{\pgfqpoint{4.691304in}{3.133251in}}%
\pgfpathlineto{\pgfqpoint{4.694515in}{3.134037in}}%
\pgfpathlineto{\pgfqpoint{4.697459in}{3.132200in}}%
\pgfpathlineto{\pgfqpoint{4.701740in}{3.126336in}}%
\pgfpathlineto{\pgfqpoint{4.708698in}{3.117141in}}%
\pgfpathlineto{\pgfqpoint{4.711642in}{3.116339in}}%
\pgfpathlineto{\pgfqpoint{4.714586in}{3.117997in}}%
\pgfpathlineto{\pgfqpoint{4.718867in}{3.123718in}}%
\pgfpathlineto{\pgfqpoint{4.726093in}{3.133305in}}%
\pgfpathlineto{\pgfqpoint{4.729036in}{3.134074in}}%
\pgfpathlineto{\pgfqpoint{4.731980in}{3.132384in}}%
\pgfpathlineto{\pgfqpoint{4.736262in}{3.126638in}}%
\pgfpathlineto{\pgfqpoint{4.743487in}{3.117088in}}%
\pgfpathlineto{\pgfqpoint{4.746431in}{3.116352in}}%
\pgfpathlineto{\pgfqpoint{4.749375in}{3.118072in}}%
\pgfpathlineto{\pgfqpoint{4.753656in}{3.123844in}}%
\pgfpathlineto{\pgfqpoint{4.760882in}{3.133357in}}%
\pgfpathlineto{\pgfqpoint{4.763825in}{3.134060in}}%
\pgfpathlineto{\pgfqpoint{4.766769in}{3.132308in}}%
\pgfpathlineto{\pgfqpoint{4.771051in}{3.126512in}}%
\pgfpathlineto{\pgfqpoint{4.778276in}{3.117036in}}%
\pgfpathlineto{\pgfqpoint{4.781220in}{3.116367in}}%
\pgfpathlineto{\pgfqpoint{4.784164in}{3.118149in}}%
\pgfpathlineto{\pgfqpoint{4.788445in}{3.123969in}}%
\pgfpathlineto{\pgfqpoint{4.795403in}{3.133229in}}%
\pgfpathlineto{\pgfqpoint{4.798614in}{3.134044in}}%
\pgfpathlineto{\pgfqpoint{4.801558in}{3.132231in}}%
\pgfpathlineto{\pgfqpoint{4.805840in}{3.126387in}}%
\pgfpathlineto{\pgfqpoint{4.812798in}{3.117162in}}%
\pgfpathlineto{\pgfqpoint{4.815741in}{3.116334in}}%
\pgfpathlineto{\pgfqpoint{4.818685in}{3.117967in}}%
\pgfpathlineto{\pgfqpoint{4.822967in}{3.123668in}}%
\pgfpathlineto{\pgfqpoint{4.830192in}{3.133284in}}%
\pgfpathlineto{\pgfqpoint{4.833136in}{3.134079in}}%
\pgfpathlineto{\pgfqpoint{4.836080in}{3.132414in}}%
\pgfpathlineto{\pgfqpoint{4.840361in}{3.126688in}}%
\pgfpathlineto{\pgfqpoint{4.847587in}{3.117109in}}%
\pgfpathlineto{\pgfqpoint{4.850530in}{3.116347in}}%
\pgfpathlineto{\pgfqpoint{4.853474in}{3.118042in}}%
\pgfpathlineto{\pgfqpoint{4.857756in}{3.123794in}}%
\pgfpathlineto{\pgfqpoint{4.864981in}{3.133337in}}%
\pgfpathlineto{\pgfqpoint{4.867925in}{3.134066in}}%
\pgfpathlineto{\pgfqpoint{4.870869in}{3.132339in}}%
\pgfpathlineto{\pgfqpoint{4.875150in}{3.126563in}}%
\pgfpathlineto{\pgfqpoint{4.882376in}{3.117057in}}%
\pgfpathlineto{\pgfqpoint{4.885319in}{3.116361in}}%
\pgfpathlineto{\pgfqpoint{4.888263in}{3.118118in}}%
\pgfpathlineto{\pgfqpoint{4.892545in}{3.123919in}}%
\pgfpathlineto{\pgfqpoint{4.899503in}{3.133207in}}%
\pgfpathlineto{\pgfqpoint{4.902714in}{3.134051in}}%
\pgfpathlineto{\pgfqpoint{4.905658in}{3.132262in}}%
\pgfpathlineto{\pgfqpoint{4.909939in}{3.126437in}}%
\pgfpathlineto{\pgfqpoint{4.916897in}{3.117184in}}%
\pgfpathlineto{\pgfqpoint{4.920108in}{3.116376in}}%
\pgfpathlineto{\pgfqpoint{4.923052in}{3.118196in}}%
\pgfpathlineto{\pgfqpoint{4.927334in}{3.124045in}}%
\pgfpathlineto{\pgfqpoint{4.934292in}{3.133262in}}%
\pgfpathlineto{\pgfqpoint{4.937235in}{3.134083in}}%
\pgfpathlineto{\pgfqpoint{4.940179in}{3.132444in}}%
\pgfpathlineto{\pgfqpoint{4.944461in}{3.126738in}}%
\pgfpathlineto{\pgfqpoint{4.951686in}{3.117130in}}%
\pgfpathlineto{\pgfqpoint{4.954630in}{3.116341in}}%
\pgfpathlineto{\pgfqpoint{4.957574in}{3.118012in}}%
\pgfpathlineto{\pgfqpoint{4.961855in}{3.123743in}}%
\pgfpathlineto{\pgfqpoint{4.969081in}{3.133316in}}%
\pgfpathlineto{\pgfqpoint{4.972024in}{3.134071in}}%
\pgfpathlineto{\pgfqpoint{4.974968in}{3.132369in}}%
\pgfpathlineto{\pgfqpoint{4.979250in}{3.126613in}}%
\pgfpathlineto{\pgfqpoint{4.986475in}{3.117077in}}%
\pgfpathlineto{\pgfqpoint{4.989419in}{3.116355in}}%
\pgfpathlineto{\pgfqpoint{4.992363in}{3.118088in}}%
\pgfpathlineto{\pgfqpoint{4.996644in}{3.123869in}}%
\pgfpathlineto{\pgfqpoint{5.003870in}{3.133367in}}%
\pgfpathlineto{\pgfqpoint{5.006813in}{3.134057in}}%
\pgfpathlineto{\pgfqpoint{5.009757in}{3.132293in}}%
\pgfpathlineto{\pgfqpoint{5.014039in}{3.126487in}}%
\pgfpathlineto{\pgfqpoint{5.020997in}{3.117206in}}%
\pgfpathlineto{\pgfqpoint{5.024208in}{3.116370in}}%
\pgfpathlineto{\pgfqpoint{5.027152in}{3.118165in}}%
\pgfpathlineto{\pgfqpoint{5.031433in}{3.123994in}}%
\pgfpathlineto{\pgfqpoint{5.038391in}{3.133240in}}%
\pgfpathlineto{\pgfqpoint{5.041602in}{3.134041in}}%
\pgfpathlineto{\pgfqpoint{5.044546in}{3.132215in}}%
\pgfpathlineto{\pgfqpoint{5.048828in}{3.126361in}}%
\pgfpathlineto{\pgfqpoint{5.055786in}{3.117151in}}%
\pgfpathlineto{\pgfqpoint{5.058729in}{3.116337in}}%
\pgfpathlineto{\pgfqpoint{5.061673in}{3.117982in}}%
\pgfpathlineto{\pgfqpoint{5.065955in}{3.123693in}}%
\pgfpathlineto{\pgfqpoint{5.073180in}{3.133294in}}%
\pgfpathlineto{\pgfqpoint{5.076124in}{3.134076in}}%
\pgfpathlineto{\pgfqpoint{5.079067in}{3.132399in}}%
\pgfpathlineto{\pgfqpoint{5.083349in}{3.126663in}}%
\pgfpathlineto{\pgfqpoint{5.090575in}{3.117098in}}%
\pgfpathlineto{\pgfqpoint{5.093518in}{3.116349in}}%
\pgfpathlineto{\pgfqpoint{5.096462in}{3.118057in}}%
\pgfpathlineto{\pgfqpoint{5.100744in}{3.123819in}}%
\pgfpathlineto{\pgfqpoint{5.107969in}{3.133347in}}%
\pgfpathlineto{\pgfqpoint{5.110913in}{3.134063in}}%
\pgfpathlineto{\pgfqpoint{5.113856in}{3.132324in}}%
\pgfpathlineto{\pgfqpoint{5.118138in}{3.126537in}}%
\pgfpathlineto{\pgfqpoint{5.125364in}{3.117046in}}%
\pgfpathlineto{\pgfqpoint{5.128307in}{3.116364in}}%
\pgfpathlineto{\pgfqpoint{5.131251in}{3.118134in}}%
\pgfpathlineto{\pgfqpoint{5.135533in}{3.123944in}}%
\pgfpathlineto{\pgfqpoint{5.142491in}{3.133218in}}%
\pgfpathlineto{\pgfqpoint{5.145702in}{3.134047in}}%
\pgfpathlineto{\pgfqpoint{5.148645in}{3.132247in}}%
\pgfpathlineto{\pgfqpoint{5.152927in}{3.126412in}}%
\pgfpathlineto{\pgfqpoint{5.159885in}{3.117173in}}%
\pgfpathlineto{\pgfqpoint{5.163096in}{3.116380in}}%
\pgfpathlineto{\pgfqpoint{5.166040in}{3.118211in}}%
\pgfpathlineto{\pgfqpoint{5.170322in}{3.124070in}}%
\pgfpathlineto{\pgfqpoint{5.177280in}{3.133273in}}%
\pgfpathlineto{\pgfqpoint{5.180223in}{3.134081in}}%
\pgfpathlineto{\pgfqpoint{5.183167in}{3.132429in}}%
\pgfpathlineto{\pgfqpoint{5.187449in}{3.126713in}}%
\pgfpathlineto{\pgfqpoint{5.194674in}{3.117119in}}%
\pgfpathlineto{\pgfqpoint{5.197618in}{3.116344in}}%
\pgfpathlineto{\pgfqpoint{5.200561in}{3.118027in}}%
\pgfpathlineto{\pgfqpoint{5.204843in}{3.123769in}}%
\pgfpathlineto{\pgfqpoint{5.212069in}{3.133326in}}%
\pgfpathlineto{\pgfqpoint{5.215012in}{3.134068in}}%
\pgfpathlineto{\pgfqpoint{5.217956in}{3.132354in}}%
\pgfpathlineto{\pgfqpoint{5.222238in}{3.126588in}}%
\pgfpathlineto{\pgfqpoint{5.229463in}{3.117067in}}%
\pgfpathlineto{\pgfqpoint{5.232407in}{3.116358in}}%
\pgfpathlineto{\pgfqpoint{5.235350in}{3.118103in}}%
\pgfpathlineto{\pgfqpoint{5.239632in}{3.123894in}}%
\pgfpathlineto{\pgfqpoint{5.246858in}{3.133378in}}%
\pgfpathlineto{\pgfqpoint{5.249801in}{3.134054in}}%
\pgfpathlineto{\pgfqpoint{5.252745in}{3.132278in}}%
\pgfpathlineto{\pgfqpoint{5.257027in}{3.126462in}}%
\pgfpathlineto{\pgfqpoint{5.263984in}{3.117195in}}%
\pgfpathlineto{\pgfqpoint{5.267196in}{3.116373in}}%
\pgfpathlineto{\pgfqpoint{5.270139in}{3.118180in}}%
\pgfpathlineto{\pgfqpoint{5.274421in}{3.124020in}}%
\pgfpathlineto{\pgfqpoint{5.281379in}{3.133251in}}%
\pgfpathlineto{\pgfqpoint{5.284590in}{3.134037in}}%
\pgfpathlineto{\pgfqpoint{5.287534in}{3.132200in}}%
\pgfpathlineto{\pgfqpoint{5.291816in}{3.126336in}}%
\pgfpathlineto{\pgfqpoint{5.298773in}{3.117141in}}%
\pgfpathlineto{\pgfqpoint{5.301717in}{3.116339in}}%
\pgfpathlineto{\pgfqpoint{5.304661in}{3.117997in}}%
\pgfpathlineto{\pgfqpoint{5.308943in}{3.123718in}}%
\pgfpathlineto{\pgfqpoint{5.316168in}{3.133305in}}%
\pgfpathlineto{\pgfqpoint{5.319112in}{3.134074in}}%
\pgfpathlineto{\pgfqpoint{5.322055in}{3.132384in}}%
\pgfpathlineto{\pgfqpoint{5.326337in}{3.126638in}}%
\pgfpathlineto{\pgfqpoint{5.333562in}{3.117088in}}%
\pgfpathlineto{\pgfqpoint{5.336506in}{3.116352in}}%
\pgfpathlineto{\pgfqpoint{5.339450in}{3.118072in}}%
\pgfpathlineto{\pgfqpoint{5.343732in}{3.123844in}}%
\pgfpathlineto{\pgfqpoint{5.350957in}{3.133357in}}%
\pgfpathlineto{\pgfqpoint{5.353901in}{3.134060in}}%
\pgfpathlineto{\pgfqpoint{5.356844in}{3.132308in}}%
\pgfpathlineto{\pgfqpoint{5.361126in}{3.126512in}}%
\pgfpathlineto{\pgfqpoint{5.368352in}{3.117036in}}%
\pgfpathlineto{\pgfqpoint{5.371295in}{3.116367in}}%
\pgfpathlineto{\pgfqpoint{5.374239in}{3.118149in}}%
\pgfpathlineto{\pgfqpoint{5.378521in}{3.123969in}}%
\pgfpathlineto{\pgfqpoint{5.385478in}{3.133229in}}%
\pgfpathlineto{\pgfqpoint{5.388690in}{3.134044in}}%
\pgfpathlineto{\pgfqpoint{5.391633in}{3.132231in}}%
\pgfpathlineto{\pgfqpoint{5.395915in}{3.126387in}}%
\pgfpathlineto{\pgfqpoint{5.402873in}{3.117162in}}%
\pgfpathlineto{\pgfqpoint{5.405817in}{3.116334in}}%
\pgfpathlineto{\pgfqpoint{5.408760in}{3.117967in}}%
\pgfpathlineto{\pgfqpoint{5.413042in}{3.123668in}}%
\pgfpathlineto{\pgfqpoint{5.420267in}{3.133284in}}%
\pgfpathlineto{\pgfqpoint{5.423211in}{3.134079in}}%
\pgfpathlineto{\pgfqpoint{5.426155in}{3.132414in}}%
\pgfpathlineto{\pgfqpoint{5.430437in}{3.126688in}}%
\pgfpathlineto{\pgfqpoint{5.437662in}{3.117109in}}%
\pgfpathlineto{\pgfqpoint{5.440606in}{3.116347in}}%
\pgfpathlineto{\pgfqpoint{5.443549in}{3.118042in}}%
\pgfpathlineto{\pgfqpoint{5.447831in}{3.123794in}}%
\pgfpathlineto{\pgfqpoint{5.455056in}{3.133337in}}%
\pgfpathlineto{\pgfqpoint{5.458000in}{3.134066in}}%
\pgfpathlineto{\pgfqpoint{5.460944in}{3.132339in}}%
\pgfpathlineto{\pgfqpoint{5.465226in}{3.126563in}}%
\pgfpathlineto{\pgfqpoint{5.472451in}{3.117057in}}%
\pgfpathlineto{\pgfqpoint{5.475395in}{3.116361in}}%
\pgfpathlineto{\pgfqpoint{5.478338in}{3.118118in}}%
\pgfpathlineto{\pgfqpoint{5.482620in}{3.123919in}}%
\pgfpathlineto{\pgfqpoint{5.489578in}{3.133207in}}%
\pgfpathlineto{\pgfqpoint{5.492789in}{3.134051in}}%
\pgfpathlineto{\pgfqpoint{5.495733in}{3.132262in}}%
\pgfpathlineto{\pgfqpoint{5.500015in}{3.126437in}}%
\pgfpathlineto{\pgfqpoint{5.506972in}{3.117184in}}%
\pgfpathlineto{\pgfqpoint{5.510184in}{3.116376in}}%
\pgfpathlineto{\pgfqpoint{5.513127in}{3.118196in}}%
\pgfpathlineto{\pgfqpoint{5.517409in}{3.124045in}}%
\pgfpathlineto{\pgfqpoint{5.524367in}{3.133262in}}%
\pgfpathlineto{\pgfqpoint{5.527311in}{3.134083in}}%
\pgfpathlineto{\pgfqpoint{5.530254in}{3.132444in}}%
\pgfpathlineto{\pgfqpoint{5.534536in}{3.126738in}}%
\pgfpathlineto{\pgfqpoint{5.541761in}{3.117130in}}%
\pgfpathlineto{\pgfqpoint{5.544705in}{3.116341in}}%
\pgfpathlineto{\pgfqpoint{5.547649in}{3.118012in}}%
\pgfpathlineto{\pgfqpoint{5.551930in}{3.123743in}}%
\pgfpathlineto{\pgfqpoint{5.559156in}{3.133316in}}%
\pgfpathlineto{\pgfqpoint{5.562100in}{3.134071in}}%
\pgfpathlineto{\pgfqpoint{5.565043in}{3.132369in}}%
\pgfpathlineto{\pgfqpoint{5.569325in}{3.126613in}}%
\pgfpathlineto{\pgfqpoint{5.576550in}{3.117077in}}%
\pgfpathlineto{\pgfqpoint{5.579494in}{3.116355in}}%
\pgfpathlineto{\pgfqpoint{5.582438in}{3.118088in}}%
\pgfpathlineto{\pgfqpoint{5.586719in}{3.123869in}}%
\pgfpathlineto{\pgfqpoint{5.593945in}{3.133367in}}%
\pgfpathlineto{\pgfqpoint{5.596889in}{3.134057in}}%
\pgfpathlineto{\pgfqpoint{5.599832in}{3.132293in}}%
\pgfpathlineto{\pgfqpoint{5.604114in}{3.126487in}}%
\pgfpathlineto{\pgfqpoint{5.611072in}{3.117206in}}%
\pgfpathlineto{\pgfqpoint{5.614283in}{3.116370in}}%
\pgfpathlineto{\pgfqpoint{5.617227in}{3.118165in}}%
\pgfpathlineto{\pgfqpoint{5.621509in}{3.123994in}}%
\pgfpathlineto{\pgfqpoint{5.628466in}{3.133240in}}%
\pgfpathlineto{\pgfqpoint{5.631678in}{3.134041in}}%
\pgfpathlineto{\pgfqpoint{5.634621in}{3.132215in}}%
\pgfpathlineto{\pgfqpoint{5.638903in}{3.126361in}}%
\pgfpathlineto{\pgfqpoint{5.645861in}{3.117151in}}%
\pgfpathlineto{\pgfqpoint{5.648805in}{3.116337in}}%
\pgfpathlineto{\pgfqpoint{5.651748in}{3.117982in}}%
\pgfpathlineto{\pgfqpoint{5.656030in}{3.123693in}}%
\pgfpathlineto{\pgfqpoint{5.663255in}{3.133294in}}%
\pgfpathlineto{\pgfqpoint{5.666199in}{3.134076in}}%
\pgfpathlineto{\pgfqpoint{5.669143in}{3.132399in}}%
\pgfpathlineto{\pgfqpoint{5.673424in}{3.126663in}}%
\pgfpathlineto{\pgfqpoint{5.680650in}{3.117098in}}%
\pgfpathlineto{\pgfqpoint{5.683594in}{3.116349in}}%
\pgfpathlineto{\pgfqpoint{5.686537in}{3.118057in}}%
\pgfpathlineto{\pgfqpoint{5.690819in}{3.123819in}}%
\pgfpathlineto{\pgfqpoint{5.698044in}{3.133347in}}%
\pgfpathlineto{\pgfqpoint{5.700988in}{3.134063in}}%
\pgfpathlineto{\pgfqpoint{5.703932in}{3.132324in}}%
\pgfpathlineto{\pgfqpoint{5.708213in}{3.126537in}}%
\pgfpathlineto{\pgfqpoint{5.715439in}{3.117046in}}%
\pgfpathlineto{\pgfqpoint{5.718383in}{3.116364in}}%
\pgfpathlineto{\pgfqpoint{5.721326in}{3.118134in}}%
\pgfpathlineto{\pgfqpoint{5.725608in}{3.123944in}}%
\pgfpathlineto{\pgfqpoint{5.732566in}{3.133218in}}%
\pgfpathlineto{\pgfqpoint{5.735777in}{3.134047in}}%
\pgfpathlineto{\pgfqpoint{5.738721in}{3.132247in}}%
\pgfpathlineto{\pgfqpoint{5.743002in}{3.126412in}}%
\pgfpathlineto{\pgfqpoint{5.749960in}{3.117173in}}%
\pgfpathlineto{\pgfqpoint{5.753172in}{3.116380in}}%
\pgfpathlineto{\pgfqpoint{5.756115in}{3.118211in}}%
\pgfpathlineto{\pgfqpoint{5.760397in}{3.124070in}}%
\pgfpathlineto{\pgfqpoint{5.767355in}{3.133273in}}%
\pgfpathlineto{\pgfqpoint{5.770298in}{3.134081in}}%
\pgfpathlineto{\pgfqpoint{5.773242in}{3.132429in}}%
\pgfpathlineto{\pgfqpoint{5.777524in}{3.126713in}}%
\pgfpathlineto{\pgfqpoint{5.784749in}{3.117119in}}%
\pgfpathlineto{\pgfqpoint{5.787693in}{3.116344in}}%
\pgfpathlineto{\pgfqpoint{5.790637in}{3.118027in}}%
\pgfpathlineto{\pgfqpoint{5.794918in}{3.123769in}}%
\pgfpathlineto{\pgfqpoint{5.802144in}{3.133326in}}%
\pgfpathlineto{\pgfqpoint{5.805087in}{3.134068in}}%
\pgfpathlineto{\pgfqpoint{5.808031in}{3.132354in}}%
\pgfpathlineto{\pgfqpoint{5.812313in}{3.126588in}}%
\pgfpathlineto{\pgfqpoint{5.819538in}{3.117067in}}%
\pgfpathlineto{\pgfqpoint{5.822482in}{3.116358in}}%
\pgfpathlineto{\pgfqpoint{5.825426in}{3.118103in}}%
\pgfpathlineto{\pgfqpoint{5.829707in}{3.123894in}}%
\pgfpathlineto{\pgfqpoint{5.836933in}{3.133378in}}%
\pgfpathlineto{\pgfqpoint{5.839876in}{3.134054in}}%
\pgfpathlineto{\pgfqpoint{5.842820in}{3.132278in}}%
\pgfpathlineto{\pgfqpoint{5.847102in}{3.126462in}}%
\pgfpathlineto{\pgfqpoint{5.854060in}{3.117195in}}%
\pgfpathlineto{\pgfqpoint{5.857271in}{3.116373in}}%
\pgfpathlineto{\pgfqpoint{5.860215in}{3.118180in}}%
\pgfpathlineto{\pgfqpoint{5.864496in}{3.124020in}}%
\pgfpathlineto{\pgfqpoint{5.871454in}{3.133251in}}%
\pgfpathlineto{\pgfqpoint{5.874666in}{3.134037in}}%
\pgfpathlineto{\pgfqpoint{5.877609in}{3.132200in}}%
\pgfpathlineto{\pgfqpoint{5.881891in}{3.126336in}}%
\pgfpathlineto{\pgfqpoint{5.888849in}{3.117141in}}%
\pgfpathlineto{\pgfqpoint{5.891792in}{3.116339in}}%
\pgfpathlineto{\pgfqpoint{5.894736in}{3.117997in}}%
\pgfpathlineto{\pgfqpoint{5.899018in}{3.123718in}}%
\pgfpathlineto{\pgfqpoint{5.906243in}{3.133305in}}%
\pgfpathlineto{\pgfqpoint{5.909187in}{3.134074in}}%
\pgfpathlineto{\pgfqpoint{5.912131in}{3.132384in}}%
\pgfpathlineto{\pgfqpoint{5.916412in}{3.126638in}}%
\pgfpathlineto{\pgfqpoint{5.923638in}{3.117088in}}%
\pgfpathlineto{\pgfqpoint{5.926581in}{3.116352in}}%
\pgfpathlineto{\pgfqpoint{5.929525in}{3.118072in}}%
\pgfpathlineto{\pgfqpoint{5.933807in}{3.123844in}}%
\pgfpathlineto{\pgfqpoint{5.941032in}{3.133357in}}%
\pgfpathlineto{\pgfqpoint{5.943976in}{3.134060in}}%
\pgfpathlineto{\pgfqpoint{5.946920in}{3.132308in}}%
\pgfpathlineto{\pgfqpoint{5.951201in}{3.126512in}}%
\pgfpathlineto{\pgfqpoint{5.958427in}{3.117036in}}%
\pgfpathlineto{\pgfqpoint{5.961370in}{3.116367in}}%
\pgfpathlineto{\pgfqpoint{5.964314in}{3.118149in}}%
\pgfpathlineto{\pgfqpoint{5.968596in}{3.123969in}}%
\pgfpathlineto{\pgfqpoint{5.975554in}{3.133229in}}%
\pgfpathlineto{\pgfqpoint{5.978765in}{3.134044in}}%
\pgfpathlineto{\pgfqpoint{5.981709in}{3.132231in}}%
\pgfpathlineto{\pgfqpoint{5.985990in}{3.126387in}}%
\pgfpathlineto{\pgfqpoint{5.992948in}{3.117162in}}%
\pgfpathlineto{\pgfqpoint{5.995892in}{3.116334in}}%
\pgfpathlineto{\pgfqpoint{5.998836in}{3.117967in}}%
\pgfpathlineto{\pgfqpoint{6.003117in}{3.123668in}}%
\pgfpathlineto{\pgfqpoint{6.010343in}{3.133284in}}%
\pgfpathlineto{\pgfqpoint{6.013286in}{3.134079in}}%
\pgfpathlineto{\pgfqpoint{6.016230in}{3.132414in}}%
\pgfpathlineto{\pgfqpoint{6.020512in}{3.126688in}}%
\pgfpathlineto{\pgfqpoint{6.027737in}{3.117109in}}%
\pgfpathlineto{\pgfqpoint{6.030681in}{3.116347in}}%
\pgfpathlineto{\pgfqpoint{6.033625in}{3.118042in}}%
\pgfpathlineto{\pgfqpoint{6.037906in}{3.123794in}}%
\pgfpathlineto{\pgfqpoint{6.045132in}{3.133337in}}%
\pgfpathlineto{\pgfqpoint{6.048075in}{3.134066in}}%
\pgfpathlineto{\pgfqpoint{6.051019in}{3.132339in}}%
\pgfpathlineto{\pgfqpoint{6.055301in}{3.126563in}}%
\pgfpathlineto{\pgfqpoint{6.062526in}{3.117057in}}%
\pgfpathlineto{\pgfqpoint{6.065470in}{3.116361in}}%
\pgfpathlineto{\pgfqpoint{6.068414in}{3.118118in}}%
\pgfpathlineto{\pgfqpoint{6.072695in}{3.123919in}}%
\pgfpathlineto{\pgfqpoint{6.079653in}{3.133207in}}%
\pgfpathlineto{\pgfqpoint{6.082864in}{3.134051in}}%
\pgfpathlineto{\pgfqpoint{6.085808in}{3.132262in}}%
\pgfpathlineto{\pgfqpoint{6.090090in}{3.126437in}}%
\pgfpathlineto{\pgfqpoint{6.097048in}{3.117184in}}%
\pgfpathlineto{\pgfqpoint{6.100259in}{3.116376in}}%
\pgfpathlineto{\pgfqpoint{6.103203in}{3.118196in}}%
\pgfpathlineto{\pgfqpoint{6.107484in}{3.124045in}}%
\pgfpathlineto{\pgfqpoint{6.114442in}{3.133262in}}%
\pgfpathlineto{\pgfqpoint{6.117386in}{3.134083in}}%
\pgfpathlineto{\pgfqpoint{6.120329in}{3.132444in}}%
\pgfpathlineto{\pgfqpoint{6.124611in}{3.126738in}}%
\pgfpathlineto{\pgfqpoint{6.131837in}{3.117130in}}%
\pgfpathlineto{\pgfqpoint{6.134780in}{3.116341in}}%
\pgfpathlineto{\pgfqpoint{6.137724in}{3.118012in}}%
\pgfpathlineto{\pgfqpoint{6.142006in}{3.123743in}}%
\pgfpathlineto{\pgfqpoint{6.149231in}{3.133316in}}%
\pgfpathlineto{\pgfqpoint{6.152175in}{3.134071in}}%
\pgfpathlineto{\pgfqpoint{6.155119in}{3.132369in}}%
\pgfpathlineto{\pgfqpoint{6.159400in}{3.126613in}}%
\pgfpathlineto{\pgfqpoint{6.166626in}{3.117077in}}%
\pgfpathlineto{\pgfqpoint{6.169569in}{3.116355in}}%
\pgfpathlineto{\pgfqpoint{6.172513in}{3.118088in}}%
\pgfpathlineto{\pgfqpoint{6.176795in}{3.123869in}}%
\pgfpathlineto{\pgfqpoint{6.184020in}{3.133367in}}%
\pgfpathlineto{\pgfqpoint{6.186964in}{3.134057in}}%
\pgfpathlineto{\pgfqpoint{6.189908in}{3.132293in}}%
\pgfpathlineto{\pgfqpoint{6.194189in}{3.126487in}}%
\pgfpathlineto{\pgfqpoint{6.201147in}{3.117206in}}%
\pgfpathlineto{\pgfqpoint{6.204358in}{3.116370in}}%
\pgfpathlineto{\pgfqpoint{6.207302in}{3.118165in}}%
\pgfpathlineto{\pgfqpoint{6.211584in}{3.123994in}}%
\pgfpathlineto{\pgfqpoint{6.218542in}{3.133240in}}%
\pgfpathlineto{\pgfqpoint{6.221753in}{3.134041in}}%
\pgfpathlineto{\pgfqpoint{6.224697in}{3.132215in}}%
\pgfpathlineto{\pgfqpoint{6.228978in}{3.126361in}}%
\pgfpathlineto{\pgfqpoint{6.235936in}{3.117151in}}%
\pgfpathlineto{\pgfqpoint{6.238880in}{3.116337in}}%
\pgfpathlineto{\pgfqpoint{6.241823in}{3.117982in}}%
\pgfpathlineto{\pgfqpoint{6.246105in}{3.123693in}}%
\pgfpathlineto{\pgfqpoint{6.253331in}{3.133294in}}%
\pgfpathlineto{\pgfqpoint{6.256274in}{3.134076in}}%
\pgfpathlineto{\pgfqpoint{6.259218in}{3.132399in}}%
\pgfpathlineto{\pgfqpoint{6.263500in}{3.126663in}}%
\pgfpathlineto{\pgfqpoint{6.270725in}{3.117098in}}%
\pgfpathlineto{\pgfqpoint{6.273669in}{3.116349in}}%
\pgfpathlineto{\pgfqpoint{6.276612in}{3.118057in}}%
\pgfpathlineto{\pgfqpoint{6.280894in}{3.123819in}}%
\pgfpathlineto{\pgfqpoint{6.288120in}{3.133347in}}%
\pgfpathlineto{\pgfqpoint{6.291063in}{3.134063in}}%
\pgfpathlineto{\pgfqpoint{6.294007in}{3.132324in}}%
\pgfpathlineto{\pgfqpoint{6.298289in}{3.126537in}}%
\pgfpathlineto{\pgfqpoint{6.305514in}{3.117046in}}%
\pgfpathlineto{\pgfqpoint{6.308458in}{3.116364in}}%
\pgfpathlineto{\pgfqpoint{6.311401in}{3.118134in}}%
\pgfpathlineto{\pgfqpoint{6.315683in}{3.123944in}}%
\pgfpathlineto{\pgfqpoint{6.322641in}{3.133218in}}%
\pgfpathlineto{\pgfqpoint{6.325852in}{3.134047in}}%
\pgfpathlineto{\pgfqpoint{6.328796in}{3.132247in}}%
\pgfpathlineto{\pgfqpoint{6.333078in}{3.126412in}}%
\pgfpathlineto{\pgfqpoint{6.340036in}{3.117173in}}%
\pgfpathlineto{\pgfqpoint{6.343247in}{3.116380in}}%
\pgfpathlineto{\pgfqpoint{6.346190in}{3.118211in}}%
\pgfpathlineto{\pgfqpoint{6.350472in}{3.124070in}}%
\pgfpathlineto{\pgfqpoint{6.357430in}{3.133273in}}%
\pgfpathlineto{\pgfqpoint{6.360374in}{3.134081in}}%
\pgfpathlineto{\pgfqpoint{6.363317in}{3.132429in}}%
\pgfpathlineto{\pgfqpoint{6.367599in}{3.126713in}}%
\pgfpathlineto{\pgfqpoint{6.374825in}{3.117119in}}%
\pgfpathlineto{\pgfqpoint{6.377768in}{3.116344in}}%
\pgfpathlineto{\pgfqpoint{6.380712in}{3.118027in}}%
\pgfpathlineto{\pgfqpoint{6.384994in}{3.123769in}}%
\pgfpathlineto{\pgfqpoint{6.392219in}{3.133326in}}%
\pgfpathlineto{\pgfqpoint{6.395163in}{3.134068in}}%
\pgfpathlineto{\pgfqpoint{6.398106in}{3.132354in}}%
\pgfpathlineto{\pgfqpoint{6.402388in}{3.126588in}}%
\pgfpathlineto{\pgfqpoint{6.409614in}{3.117067in}}%
\pgfpathlineto{\pgfqpoint{6.412557in}{3.116358in}}%
\pgfpathlineto{\pgfqpoint{6.415501in}{3.118103in}}%
\pgfpathlineto{\pgfqpoint{6.419783in}{3.123894in}}%
\pgfpathlineto{\pgfqpoint{6.427008in}{3.133378in}}%
\pgfpathlineto{\pgfqpoint{6.429952in}{3.134054in}}%
\pgfpathlineto{\pgfqpoint{6.432895in}{3.132278in}}%
\pgfpathlineto{\pgfqpoint{6.437177in}{3.126462in}}%
\pgfpathlineto{\pgfqpoint{6.444135in}{3.117195in}}%
\pgfpathlineto{\pgfqpoint{6.447346in}{3.116373in}}%
\pgfpathlineto{\pgfqpoint{6.450290in}{3.118180in}}%
\pgfpathlineto{\pgfqpoint{6.454572in}{3.124020in}}%
\pgfpathlineto{\pgfqpoint{6.461529in}{3.133251in}}%
\pgfpathlineto{\pgfqpoint{6.464741in}{3.134037in}}%
\pgfpathlineto{\pgfqpoint{6.467684in}{3.132200in}}%
\pgfpathlineto{\pgfqpoint{6.471966in}{3.126336in}}%
\pgfpathlineto{\pgfqpoint{6.478924in}{3.117141in}}%
\pgfpathlineto{\pgfqpoint{6.481868in}{3.116339in}}%
\pgfpathlineto{\pgfqpoint{6.484811in}{3.117997in}}%
\pgfpathlineto{\pgfqpoint{6.489093in}{3.123718in}}%
\pgfpathlineto{\pgfqpoint{6.496318in}{3.133305in}}%
\pgfpathlineto{\pgfqpoint{6.499262in}{3.134074in}}%
\pgfpathlineto{\pgfqpoint{6.502206in}{3.132384in}}%
\pgfpathlineto{\pgfqpoint{6.506488in}{3.126638in}}%
\pgfpathlineto{\pgfqpoint{6.513713in}{3.117088in}}%
\pgfpathlineto{\pgfqpoint{6.516657in}{3.116352in}}%
\pgfpathlineto{\pgfqpoint{6.519600in}{3.118072in}}%
\pgfpathlineto{\pgfqpoint{6.523882in}{3.123844in}}%
\pgfpathlineto{\pgfqpoint{6.531107in}{3.133357in}}%
\pgfpathlineto{\pgfqpoint{6.534051in}{3.134060in}}%
\pgfpathlineto{\pgfqpoint{6.536995in}{3.132308in}}%
\pgfpathlineto{\pgfqpoint{6.541277in}{3.126512in}}%
\pgfpathlineto{\pgfqpoint{6.548502in}{3.117036in}}%
\pgfpathlineto{\pgfqpoint{6.551446in}{3.116367in}}%
\pgfpathlineto{\pgfqpoint{6.554389in}{3.118149in}}%
\pgfpathlineto{\pgfqpoint{6.558671in}{3.123969in}}%
\pgfpathlineto{\pgfqpoint{6.565629in}{3.133229in}}%
\pgfpathlineto{\pgfqpoint{6.568840in}{3.134044in}}%
\pgfpathlineto{\pgfqpoint{6.571784in}{3.132231in}}%
\pgfpathlineto{\pgfqpoint{6.576066in}{3.126387in}}%
\pgfpathlineto{\pgfqpoint{6.583023in}{3.117162in}}%
\pgfpathlineto{\pgfqpoint{6.585967in}{3.116334in}}%
\pgfpathlineto{\pgfqpoint{6.588911in}{3.117967in}}%
\pgfpathlineto{\pgfqpoint{6.593193in}{3.123668in}}%
\pgfpathlineto{\pgfqpoint{6.600418in}{3.133284in}}%
\pgfpathlineto{\pgfqpoint{6.603362in}{3.134079in}}%
\pgfpathlineto{\pgfqpoint{6.606305in}{3.132414in}}%
\pgfpathlineto{\pgfqpoint{6.610587in}{3.126688in}}%
\pgfpathlineto{\pgfqpoint{6.617812in}{3.117109in}}%
\pgfpathlineto{\pgfqpoint{6.620756in}{3.116347in}}%
\pgfpathlineto{\pgfqpoint{6.623700in}{3.118042in}}%
\pgfpathlineto{\pgfqpoint{6.627982in}{3.123794in}}%
\pgfpathlineto{\pgfqpoint{6.635207in}{3.133337in}}%
\pgfpathlineto{\pgfqpoint{6.638151in}{3.134066in}}%
\pgfpathlineto{\pgfqpoint{6.641094in}{3.132339in}}%
\pgfpathlineto{\pgfqpoint{6.645376in}{3.126563in}}%
\pgfpathlineto{\pgfqpoint{6.652601in}{3.117057in}}%
\pgfpathlineto{\pgfqpoint{6.655545in}{3.116361in}}%
\pgfpathlineto{\pgfqpoint{6.658489in}{3.118118in}}%
\pgfpathlineto{\pgfqpoint{6.662771in}{3.123919in}}%
\pgfpathlineto{\pgfqpoint{6.663306in}{3.124778in}}%
\pgfpathlineto{\pgfqpoint{6.663306in}{3.124778in}}%
\pgfusepath{stroke}%
\end{pgfscope}%
\begin{pgfscope}%
\pgfpathrectangle{\pgfqpoint{0.467797in}{2.292089in}}{\pgfqpoint{6.490533in}{1.666241in}}%
\pgfusepath{clip}%
\pgfsetrectcap%
\pgfsetroundjoin%
\pgfsetlinewidth{1.505625pt}%
\definecolor{currentstroke}{rgb}{0.839216,0.152941,0.156863}%
\pgfsetstrokecolor{currentstroke}%
\pgfsetdash{}{0pt}%
\pgfpathmoveto{\pgfqpoint{0.762821in}{3.125209in}}%
\pgfpathlineto{\pgfqpoint{0.768976in}{3.133121in}}%
\pgfpathlineto{\pgfqpoint{0.771920in}{3.133857in}}%
\pgfpathlineto{\pgfqpoint{0.774863in}{3.132085in}}%
\pgfpathlineto{\pgfqpoint{0.779145in}{3.126227in}}%
\pgfpathlineto{\pgfqpoint{0.785835in}{3.117364in}}%
\pgfpathlineto{\pgfqpoint{0.788779in}{3.116545in}}%
\pgfpathlineto{\pgfqpoint{0.791723in}{3.118239in}}%
\pgfpathlineto{\pgfqpoint{0.796004in}{3.124037in}}%
\pgfpathlineto{\pgfqpoint{0.802962in}{3.133170in}}%
\pgfpathlineto{\pgfqpoint{0.805906in}{3.133843in}}%
\pgfpathlineto{\pgfqpoint{0.808849in}{3.132011in}}%
\pgfpathlineto{\pgfqpoint{0.813131in}{3.126108in}}%
\pgfpathlineto{\pgfqpoint{0.819821in}{3.117314in}}%
\pgfpathlineto{\pgfqpoint{0.822765in}{3.116557in}}%
\pgfpathlineto{\pgfqpoint{0.825709in}{3.118311in}}%
\pgfpathlineto{\pgfqpoint{0.829990in}{3.124155in}}%
\pgfpathlineto{\pgfqpoint{0.836681in}{3.133038in}}%
\pgfpathlineto{\pgfqpoint{0.839624in}{3.133878in}}%
\pgfpathlineto{\pgfqpoint{0.842568in}{3.132202in}}%
\pgfpathlineto{\pgfqpoint{0.846850in}{3.126419in}}%
\pgfpathlineto{\pgfqpoint{0.853808in}{3.117264in}}%
\pgfpathlineto{\pgfqpoint{0.856751in}{3.116571in}}%
\pgfpathlineto{\pgfqpoint{0.859695in}{3.118385in}}%
\pgfpathlineto{\pgfqpoint{0.863977in}{3.124274in}}%
\pgfpathlineto{\pgfqpoint{0.870667in}{3.133090in}}%
\pgfpathlineto{\pgfqpoint{0.873611in}{3.133866in}}%
\pgfpathlineto{\pgfqpoint{0.876554in}{3.132130in}}%
\pgfpathlineto{\pgfqpoint{0.880836in}{3.126301in}}%
\pgfpathlineto{\pgfqpoint{0.887794in}{3.117216in}}%
\pgfpathlineto{\pgfqpoint{0.890737in}{3.116587in}}%
\pgfpathlineto{\pgfqpoint{0.893681in}{3.118459in}}%
\pgfpathlineto{\pgfqpoint{0.898230in}{3.124823in}}%
\pgfpathlineto{\pgfqpoint{0.904653in}{3.133139in}}%
\pgfpathlineto{\pgfqpoint{0.907597in}{3.133852in}}%
\pgfpathlineto{\pgfqpoint{0.910540in}{3.132057in}}%
\pgfpathlineto{\pgfqpoint{0.914822in}{3.126182in}}%
\pgfpathlineto{\pgfqpoint{0.921512in}{3.117345in}}%
\pgfpathlineto{\pgfqpoint{0.924456in}{3.116549in}}%
\pgfpathlineto{\pgfqpoint{0.927400in}{3.118266in}}%
\pgfpathlineto{\pgfqpoint{0.931681in}{3.124081in}}%
\pgfpathlineto{\pgfqpoint{0.938639in}{3.133188in}}%
\pgfpathlineto{\pgfqpoint{0.941583in}{3.133837in}}%
\pgfpathlineto{\pgfqpoint{0.944527in}{3.131983in}}%
\pgfpathlineto{\pgfqpoint{0.949076in}{3.125633in}}%
\pgfpathlineto{\pgfqpoint{0.955499in}{3.117295in}}%
\pgfpathlineto{\pgfqpoint{0.958442in}{3.116562in}}%
\pgfpathlineto{\pgfqpoint{0.961386in}{3.118339in}}%
\pgfpathlineto{\pgfqpoint{0.965668in}{3.124200in}}%
\pgfpathlineto{\pgfqpoint{0.972358in}{3.133058in}}%
\pgfpathlineto{\pgfqpoint{0.975302in}{3.133873in}}%
\pgfpathlineto{\pgfqpoint{0.978245in}{3.132175in}}%
\pgfpathlineto{\pgfqpoint{0.982527in}{3.126374in}}%
\pgfpathlineto{\pgfqpoint{0.989485in}{3.117246in}}%
\pgfpathlineto{\pgfqpoint{0.992428in}{3.116577in}}%
\pgfpathlineto{\pgfqpoint{0.995372in}{3.118412in}}%
\pgfpathlineto{\pgfqpoint{0.999654in}{3.124318in}}%
\pgfpathlineto{\pgfqpoint{1.006344in}{3.133108in}}%
\pgfpathlineto{\pgfqpoint{1.009288in}{3.133861in}}%
\pgfpathlineto{\pgfqpoint{1.012231in}{3.132103in}}%
\pgfpathlineto{\pgfqpoint{1.016513in}{3.126256in}}%
\pgfpathlineto{\pgfqpoint{1.023203in}{3.117377in}}%
\pgfpathlineto{\pgfqpoint{1.026147in}{3.116542in}}%
\pgfpathlineto{\pgfqpoint{1.029091in}{3.118222in}}%
\pgfpathlineto{\pgfqpoint{1.033372in}{3.124008in}}%
\pgfpathlineto{\pgfqpoint{1.040330in}{3.133158in}}%
\pgfpathlineto{\pgfqpoint{1.043274in}{3.133847in}}%
\pgfpathlineto{\pgfqpoint{1.046218in}{3.132030in}}%
\pgfpathlineto{\pgfqpoint{1.050499in}{3.126138in}}%
\pgfpathlineto{\pgfqpoint{1.057190in}{3.117326in}}%
\pgfpathlineto{\pgfqpoint{1.060133in}{3.116554in}}%
\pgfpathlineto{\pgfqpoint{1.063077in}{3.118293in}}%
\pgfpathlineto{\pgfqpoint{1.067359in}{3.124126in}}%
\pgfpathlineto{\pgfqpoint{1.074316in}{3.133206in}}%
\pgfpathlineto{\pgfqpoint{1.077260in}{3.133831in}}%
\pgfpathlineto{\pgfqpoint{1.080204in}{3.131955in}}%
\pgfpathlineto{\pgfqpoint{1.084753in}{3.125589in}}%
\pgfpathlineto{\pgfqpoint{1.091176in}{3.117276in}}%
\pgfpathlineto{\pgfqpoint{1.094119in}{3.116567in}}%
\pgfpathlineto{\pgfqpoint{1.097063in}{3.118366in}}%
\pgfpathlineto{\pgfqpoint{1.101345in}{3.124244in}}%
\pgfpathlineto{\pgfqpoint{1.108035in}{3.133077in}}%
\pgfpathlineto{\pgfqpoint{1.110979in}{3.133869in}}%
\pgfpathlineto{\pgfqpoint{1.113922in}{3.132148in}}%
\pgfpathlineto{\pgfqpoint{1.118204in}{3.126330in}}%
\pgfpathlineto{\pgfqpoint{1.125162in}{3.117228in}}%
\pgfpathlineto{\pgfqpoint{1.128106in}{3.116583in}}%
\pgfpathlineto{\pgfqpoint{1.131049in}{3.118440in}}%
\pgfpathlineto{\pgfqpoint{1.135599in}{3.124793in}}%
\pgfpathlineto{\pgfqpoint{1.142021in}{3.133127in}}%
\pgfpathlineto{\pgfqpoint{1.144965in}{3.133856in}}%
\pgfpathlineto{\pgfqpoint{1.147909in}{3.132076in}}%
\pgfpathlineto{\pgfqpoint{1.152190in}{3.126212in}}%
\pgfpathlineto{\pgfqpoint{1.158880in}{3.117358in}}%
\pgfpathlineto{\pgfqpoint{1.161824in}{3.116546in}}%
\pgfpathlineto{\pgfqpoint{1.164768in}{3.118248in}}%
\pgfpathlineto{\pgfqpoint{1.169050in}{3.124052in}}%
\pgfpathlineto{\pgfqpoint{1.176007in}{3.133176in}}%
\pgfpathlineto{\pgfqpoint{1.178951in}{3.133841in}}%
\pgfpathlineto{\pgfqpoint{1.181895in}{3.132002in}}%
\pgfpathlineto{\pgfqpoint{1.186176in}{3.126094in}}%
\pgfpathlineto{\pgfqpoint{1.192867in}{3.117307in}}%
\pgfpathlineto{\pgfqpoint{1.195810in}{3.116559in}}%
\pgfpathlineto{\pgfqpoint{1.198754in}{3.118320in}}%
\pgfpathlineto{\pgfqpoint{1.203036in}{3.124170in}}%
\pgfpathlineto{\pgfqpoint{1.209726in}{3.133045in}}%
\pgfpathlineto{\pgfqpoint{1.212670in}{3.133876in}}%
\pgfpathlineto{\pgfqpoint{1.215613in}{3.132193in}}%
\pgfpathlineto{\pgfqpoint{1.219895in}{3.126404in}}%
\pgfpathlineto{\pgfqpoint{1.226853in}{3.117258in}}%
\pgfpathlineto{\pgfqpoint{1.229797in}{3.116573in}}%
\pgfpathlineto{\pgfqpoint{1.232740in}{3.118394in}}%
\pgfpathlineto{\pgfqpoint{1.237022in}{3.124288in}}%
\pgfpathlineto{\pgfqpoint{1.243712in}{3.133096in}}%
\pgfpathlineto{\pgfqpoint{1.246656in}{3.133864in}}%
\pgfpathlineto{\pgfqpoint{1.249600in}{3.132121in}}%
\pgfpathlineto{\pgfqpoint{1.253881in}{3.126286in}}%
\pgfpathlineto{\pgfqpoint{1.260839in}{3.117210in}}%
\pgfpathlineto{\pgfqpoint{1.263783in}{3.116589in}}%
\pgfpathlineto{\pgfqpoint{1.266726in}{3.118469in}}%
\pgfpathlineto{\pgfqpoint{1.271276in}{3.124837in}}%
\pgfpathlineto{\pgfqpoint{1.277698in}{3.133146in}}%
\pgfpathlineto{\pgfqpoint{1.280642in}{3.133850in}}%
\pgfpathlineto{\pgfqpoint{1.283586in}{3.132048in}}%
\pgfpathlineto{\pgfqpoint{1.287867in}{3.126168in}}%
\pgfpathlineto{\pgfqpoint{1.294558in}{3.117339in}}%
\pgfpathlineto{\pgfqpoint{1.297501in}{3.116551in}}%
\pgfpathlineto{\pgfqpoint{1.300445in}{3.118275in}}%
\pgfpathlineto{\pgfqpoint{1.304727in}{3.124096in}}%
\pgfpathlineto{\pgfqpoint{1.311685in}{3.133194in}}%
\pgfpathlineto{\pgfqpoint{1.314628in}{3.133835in}}%
\pgfpathlineto{\pgfqpoint{1.317572in}{3.131974in}}%
\pgfpathlineto{\pgfqpoint{1.322121in}{3.125619in}}%
\pgfpathlineto{\pgfqpoint{1.328544in}{3.117289in}}%
\pgfpathlineto{\pgfqpoint{1.331488in}{3.116564in}}%
\pgfpathlineto{\pgfqpoint{1.334431in}{3.118348in}}%
\pgfpathlineto{\pgfqpoint{1.338713in}{3.124214in}}%
\pgfpathlineto{\pgfqpoint{1.345403in}{3.133064in}}%
\pgfpathlineto{\pgfqpoint{1.348347in}{3.133872in}}%
\pgfpathlineto{\pgfqpoint{1.351291in}{3.132166in}}%
\pgfpathlineto{\pgfqpoint{1.355572in}{3.126360in}}%
\pgfpathlineto{\pgfqpoint{1.362530in}{3.117240in}}%
\pgfpathlineto{\pgfqpoint{1.365474in}{3.116579in}}%
\pgfpathlineto{\pgfqpoint{1.368417in}{3.118422in}}%
\pgfpathlineto{\pgfqpoint{1.372699in}{3.124333in}}%
\pgfpathlineto{\pgfqpoint{1.379389in}{3.133115in}}%
\pgfpathlineto{\pgfqpoint{1.382333in}{3.133859in}}%
\pgfpathlineto{\pgfqpoint{1.385277in}{3.132094in}}%
\pgfpathlineto{\pgfqpoint{1.389558in}{3.126241in}}%
\pgfpathlineto{\pgfqpoint{1.396249in}{3.117371in}}%
\pgfpathlineto{\pgfqpoint{1.399192in}{3.116543in}}%
\pgfpathlineto{\pgfqpoint{1.402136in}{3.118230in}}%
\pgfpathlineto{\pgfqpoint{1.406418in}{3.124022in}}%
\pgfpathlineto{\pgfqpoint{1.413376in}{3.133164in}}%
\pgfpathlineto{\pgfqpoint{1.416319in}{3.133845in}}%
\pgfpathlineto{\pgfqpoint{1.419263in}{3.132020in}}%
\pgfpathlineto{\pgfqpoint{1.423545in}{3.126123in}}%
\pgfpathlineto{\pgfqpoint{1.430235in}{3.117320in}}%
\pgfpathlineto{\pgfqpoint{1.433179in}{3.116556in}}%
\pgfpathlineto{\pgfqpoint{1.436122in}{3.118302in}}%
\pgfpathlineto{\pgfqpoint{1.440404in}{3.124140in}}%
\pgfpathlineto{\pgfqpoint{1.447362in}{3.133211in}}%
\pgfpathlineto{\pgfqpoint{1.450305in}{3.133829in}}%
\pgfpathlineto{\pgfqpoint{1.453249in}{3.131946in}}%
\pgfpathlineto{\pgfqpoint{1.457798in}{3.125574in}}%
\pgfpathlineto{\pgfqpoint{1.464221in}{3.117270in}}%
\pgfpathlineto{\pgfqpoint{1.467165in}{3.116569in}}%
\pgfpathlineto{\pgfqpoint{1.470108in}{3.118375in}}%
\pgfpathlineto{\pgfqpoint{1.474390in}{3.124259in}}%
\pgfpathlineto{\pgfqpoint{1.481080in}{3.133083in}}%
\pgfpathlineto{\pgfqpoint{1.484024in}{3.133867in}}%
\pgfpathlineto{\pgfqpoint{1.486968in}{3.132139in}}%
\pgfpathlineto{\pgfqpoint{1.491249in}{3.126315in}}%
\pgfpathlineto{\pgfqpoint{1.498207in}{3.117222in}}%
\pgfpathlineto{\pgfqpoint{1.501151in}{3.116585in}}%
\pgfpathlineto{\pgfqpoint{1.504095in}{3.118450in}}%
\pgfpathlineto{\pgfqpoint{1.508644in}{3.124808in}}%
\pgfpathlineto{\pgfqpoint{1.515066in}{3.133133in}}%
\pgfpathlineto{\pgfqpoint{1.518010in}{3.133854in}}%
\pgfpathlineto{\pgfqpoint{1.520954in}{3.132066in}}%
\pgfpathlineto{\pgfqpoint{1.525236in}{3.126197in}}%
\pgfpathlineto{\pgfqpoint{1.531926in}{3.117352in}}%
\pgfpathlineto{\pgfqpoint{1.534869in}{3.116548in}}%
\pgfpathlineto{\pgfqpoint{1.537813in}{3.118257in}}%
\pgfpathlineto{\pgfqpoint{1.542095in}{3.124067in}}%
\pgfpathlineto{\pgfqpoint{1.549053in}{3.133182in}}%
\pgfpathlineto{\pgfqpoint{1.551996in}{3.133839in}}%
\pgfpathlineto{\pgfqpoint{1.554940in}{3.131992in}}%
\pgfpathlineto{\pgfqpoint{1.559222in}{3.126079in}}%
\pgfpathlineto{\pgfqpoint{1.565912in}{3.117301in}}%
\pgfpathlineto{\pgfqpoint{1.568856in}{3.116561in}}%
\pgfpathlineto{\pgfqpoint{1.571799in}{3.118330in}}%
\pgfpathlineto{\pgfqpoint{1.576081in}{3.124185in}}%
\pgfpathlineto{\pgfqpoint{1.582771in}{3.133051in}}%
\pgfpathlineto{\pgfqpoint{1.585715in}{3.133875in}}%
\pgfpathlineto{\pgfqpoint{1.588659in}{3.132184in}}%
\pgfpathlineto{\pgfqpoint{1.592940in}{3.126389in}}%
\pgfpathlineto{\pgfqpoint{1.599898in}{3.117252in}}%
\pgfpathlineto{\pgfqpoint{1.602842in}{3.116575in}}%
\pgfpathlineto{\pgfqpoint{1.605786in}{3.118403in}}%
\pgfpathlineto{\pgfqpoint{1.610067in}{3.124303in}}%
\pgfpathlineto{\pgfqpoint{1.616757in}{3.133102in}}%
\pgfpathlineto{\pgfqpoint{1.619701in}{3.133863in}}%
\pgfpathlineto{\pgfqpoint{1.622645in}{3.132112in}}%
\pgfpathlineto{\pgfqpoint{1.626927in}{3.126271in}}%
\pgfpathlineto{\pgfqpoint{1.633617in}{3.117384in}}%
\pgfpathlineto{\pgfqpoint{1.636560in}{3.116541in}}%
\pgfpathlineto{\pgfqpoint{1.639504in}{3.118213in}}%
\pgfpathlineto{\pgfqpoint{1.643786in}{3.123993in}}%
\pgfpathlineto{\pgfqpoint{1.650744in}{3.133152in}}%
\pgfpathlineto{\pgfqpoint{1.653687in}{3.133849in}}%
\pgfpathlineto{\pgfqpoint{1.656631in}{3.132039in}}%
\pgfpathlineto{\pgfqpoint{1.660913in}{3.126153in}}%
\pgfpathlineto{\pgfqpoint{1.667603in}{3.117332in}}%
\pgfpathlineto{\pgfqpoint{1.670547in}{3.116552in}}%
\pgfpathlineto{\pgfqpoint{1.673490in}{3.118284in}}%
\pgfpathlineto{\pgfqpoint{1.677772in}{3.124111in}}%
\pgfpathlineto{\pgfqpoint{1.684730in}{3.133200in}}%
\pgfpathlineto{\pgfqpoint{1.687674in}{3.133833in}}%
\pgfpathlineto{\pgfqpoint{1.690617in}{3.131964in}}%
\pgfpathlineto{\pgfqpoint{1.695167in}{3.125604in}}%
\pgfpathlineto{\pgfqpoint{1.701589in}{3.117282in}}%
\pgfpathlineto{\pgfqpoint{1.704533in}{3.116566in}}%
\pgfpathlineto{\pgfqpoint{1.707477in}{3.118357in}}%
\pgfpathlineto{\pgfqpoint{1.711758in}{3.124229in}}%
\pgfpathlineto{\pgfqpoint{1.718448in}{3.133071in}}%
\pgfpathlineto{\pgfqpoint{1.721392in}{3.133870in}}%
\pgfpathlineto{\pgfqpoint{1.724336in}{3.132157in}}%
\pgfpathlineto{\pgfqpoint{1.728618in}{3.126345in}}%
\pgfpathlineto{\pgfqpoint{1.735575in}{3.117234in}}%
\pgfpathlineto{\pgfqpoint{1.738519in}{3.116581in}}%
\pgfpathlineto{\pgfqpoint{1.741463in}{3.118431in}}%
\pgfpathlineto{\pgfqpoint{1.746012in}{3.124778in}}%
\pgfpathlineto{\pgfqpoint{1.752435in}{3.133121in}}%
\pgfpathlineto{\pgfqpoint{1.755378in}{3.133857in}}%
\pgfpathlineto{\pgfqpoint{1.758322in}{3.132085in}}%
\pgfpathlineto{\pgfqpoint{1.762604in}{3.126227in}}%
\pgfpathlineto{\pgfqpoint{1.769294in}{3.117364in}}%
\pgfpathlineto{\pgfqpoint{1.772238in}{3.116545in}}%
\pgfpathlineto{\pgfqpoint{1.775181in}{3.118239in}}%
\pgfpathlineto{\pgfqpoint{1.779463in}{3.124037in}}%
\pgfpathlineto{\pgfqpoint{1.786421in}{3.133170in}}%
\pgfpathlineto{\pgfqpoint{1.789365in}{3.133843in}}%
\pgfpathlineto{\pgfqpoint{1.792308in}{3.132011in}}%
\pgfpathlineto{\pgfqpoint{1.796590in}{3.126108in}}%
\pgfpathlineto{\pgfqpoint{1.803280in}{3.117314in}}%
\pgfpathlineto{\pgfqpoint{1.806224in}{3.116557in}}%
\pgfpathlineto{\pgfqpoint{1.809167in}{3.118311in}}%
\pgfpathlineto{\pgfqpoint{1.813449in}{3.124155in}}%
\pgfpathlineto{\pgfqpoint{1.820139in}{3.133038in}}%
\pgfpathlineto{\pgfqpoint{1.823083in}{3.133878in}}%
\pgfpathlineto{\pgfqpoint{1.826027in}{3.132202in}}%
\pgfpathlineto{\pgfqpoint{1.830309in}{3.126419in}}%
\pgfpathlineto{\pgfqpoint{1.837266in}{3.117264in}}%
\pgfpathlineto{\pgfqpoint{1.840210in}{3.116571in}}%
\pgfpathlineto{\pgfqpoint{1.843154in}{3.118385in}}%
\pgfpathlineto{\pgfqpoint{1.847435in}{3.124274in}}%
\pgfpathlineto{\pgfqpoint{1.854126in}{3.133090in}}%
\pgfpathlineto{\pgfqpoint{1.857069in}{3.133866in}}%
\pgfpathlineto{\pgfqpoint{1.860013in}{3.132130in}}%
\pgfpathlineto{\pgfqpoint{1.864295in}{3.126301in}}%
\pgfpathlineto{\pgfqpoint{1.871253in}{3.117216in}}%
\pgfpathlineto{\pgfqpoint{1.874196in}{3.116587in}}%
\pgfpathlineto{\pgfqpoint{1.877140in}{3.118459in}}%
\pgfpathlineto{\pgfqpoint{1.881689in}{3.124823in}}%
\pgfpathlineto{\pgfqpoint{1.888112in}{3.133139in}}%
\pgfpathlineto{\pgfqpoint{1.891055in}{3.133852in}}%
\pgfpathlineto{\pgfqpoint{1.893999in}{3.132057in}}%
\pgfpathlineto{\pgfqpoint{1.898281in}{3.126182in}}%
\pgfpathlineto{\pgfqpoint{1.904971in}{3.117345in}}%
\pgfpathlineto{\pgfqpoint{1.907915in}{3.116549in}}%
\pgfpathlineto{\pgfqpoint{1.910858in}{3.118266in}}%
\pgfpathlineto{\pgfqpoint{1.915140in}{3.124081in}}%
\pgfpathlineto{\pgfqpoint{1.922098in}{3.133188in}}%
\pgfpathlineto{\pgfqpoint{1.925042in}{3.133837in}}%
\pgfpathlineto{\pgfqpoint{1.927985in}{3.131983in}}%
\pgfpathlineto{\pgfqpoint{1.932535in}{3.125633in}}%
\pgfpathlineto{\pgfqpoint{1.938957in}{3.117295in}}%
\pgfpathlineto{\pgfqpoint{1.941901in}{3.116562in}}%
\pgfpathlineto{\pgfqpoint{1.944845in}{3.118339in}}%
\pgfpathlineto{\pgfqpoint{1.949126in}{3.124200in}}%
\pgfpathlineto{\pgfqpoint{1.955817in}{3.133058in}}%
\pgfpathlineto{\pgfqpoint{1.958760in}{3.133873in}}%
\pgfpathlineto{\pgfqpoint{1.961704in}{3.132175in}}%
\pgfpathlineto{\pgfqpoint{1.965986in}{3.126374in}}%
\pgfpathlineto{\pgfqpoint{1.972943in}{3.117246in}}%
\pgfpathlineto{\pgfqpoint{1.975887in}{3.116577in}}%
\pgfpathlineto{\pgfqpoint{1.978831in}{3.118412in}}%
\pgfpathlineto{\pgfqpoint{1.983113in}{3.124318in}}%
\pgfpathlineto{\pgfqpoint{1.989803in}{3.133108in}}%
\pgfpathlineto{\pgfqpoint{1.992746in}{3.133861in}}%
\pgfpathlineto{\pgfqpoint{1.995690in}{3.132103in}}%
\pgfpathlineto{\pgfqpoint{1.999972in}{3.126256in}}%
\pgfpathlineto{\pgfqpoint{2.006662in}{3.117377in}}%
\pgfpathlineto{\pgfqpoint{2.009606in}{3.116542in}}%
\pgfpathlineto{\pgfqpoint{2.012549in}{3.118222in}}%
\pgfpathlineto{\pgfqpoint{2.016831in}{3.124008in}}%
\pgfpathlineto{\pgfqpoint{2.023789in}{3.133158in}}%
\pgfpathlineto{\pgfqpoint{2.026733in}{3.133847in}}%
\pgfpathlineto{\pgfqpoint{2.029676in}{3.132030in}}%
\pgfpathlineto{\pgfqpoint{2.033958in}{3.126138in}}%
\pgfpathlineto{\pgfqpoint{2.040648in}{3.117326in}}%
\pgfpathlineto{\pgfqpoint{2.043592in}{3.116554in}}%
\pgfpathlineto{\pgfqpoint{2.046536in}{3.118293in}}%
\pgfpathlineto{\pgfqpoint{2.050817in}{3.124126in}}%
\pgfpathlineto{\pgfqpoint{2.057775in}{3.133206in}}%
\pgfpathlineto{\pgfqpoint{2.060719in}{3.133831in}}%
\pgfpathlineto{\pgfqpoint{2.063663in}{3.131955in}}%
\pgfpathlineto{\pgfqpoint{2.068212in}{3.125589in}}%
\pgfpathlineto{\pgfqpoint{2.074634in}{3.117276in}}%
\pgfpathlineto{\pgfqpoint{2.077578in}{3.116567in}}%
\pgfpathlineto{\pgfqpoint{2.080522in}{3.118366in}}%
\pgfpathlineto{\pgfqpoint{2.084804in}{3.124244in}}%
\pgfpathlineto{\pgfqpoint{2.091494in}{3.133077in}}%
\pgfpathlineto{\pgfqpoint{2.094437in}{3.133869in}}%
\pgfpathlineto{\pgfqpoint{2.097381in}{3.132148in}}%
\pgfpathlineto{\pgfqpoint{2.101663in}{3.126330in}}%
\pgfpathlineto{\pgfqpoint{2.108621in}{3.117228in}}%
\pgfpathlineto{\pgfqpoint{2.111564in}{3.116583in}}%
\pgfpathlineto{\pgfqpoint{2.114508in}{3.118440in}}%
\pgfpathlineto{\pgfqpoint{2.119057in}{3.124793in}}%
\pgfpathlineto{\pgfqpoint{2.125480in}{3.133127in}}%
\pgfpathlineto{\pgfqpoint{2.128424in}{3.133856in}}%
\pgfpathlineto{\pgfqpoint{2.131367in}{3.132076in}}%
\pgfpathlineto{\pgfqpoint{2.135649in}{3.126212in}}%
\pgfpathlineto{\pgfqpoint{2.142339in}{3.117358in}}%
\pgfpathlineto{\pgfqpoint{2.145283in}{3.116546in}}%
\pgfpathlineto{\pgfqpoint{2.148227in}{3.118248in}}%
\pgfpathlineto{\pgfqpoint{2.152508in}{3.124052in}}%
\pgfpathlineto{\pgfqpoint{2.159466in}{3.133176in}}%
\pgfpathlineto{\pgfqpoint{2.162410in}{3.133841in}}%
\pgfpathlineto{\pgfqpoint{2.165353in}{3.132002in}}%
\pgfpathlineto{\pgfqpoint{2.169635in}{3.126094in}}%
\pgfpathlineto{\pgfqpoint{2.176325in}{3.117307in}}%
\pgfpathlineto{\pgfqpoint{2.179269in}{3.116559in}}%
\pgfpathlineto{\pgfqpoint{2.182213in}{3.118320in}}%
\pgfpathlineto{\pgfqpoint{2.186495in}{3.124170in}}%
\pgfpathlineto{\pgfqpoint{2.193185in}{3.133045in}}%
\pgfpathlineto{\pgfqpoint{2.196128in}{3.133876in}}%
\pgfpathlineto{\pgfqpoint{2.199072in}{3.132193in}}%
\pgfpathlineto{\pgfqpoint{2.203354in}{3.126404in}}%
\pgfpathlineto{\pgfqpoint{2.210312in}{3.117258in}}%
\pgfpathlineto{\pgfqpoint{2.213255in}{3.116573in}}%
\pgfpathlineto{\pgfqpoint{2.216199in}{3.118394in}}%
\pgfpathlineto{\pgfqpoint{2.220481in}{3.124288in}}%
\pgfpathlineto{\pgfqpoint{2.227171in}{3.133096in}}%
\pgfpathlineto{\pgfqpoint{2.230115in}{3.133864in}}%
\pgfpathlineto{\pgfqpoint{2.233058in}{3.132121in}}%
\pgfpathlineto{\pgfqpoint{2.237340in}{3.126286in}}%
\pgfpathlineto{\pgfqpoint{2.244298in}{3.117210in}}%
\pgfpathlineto{\pgfqpoint{2.247241in}{3.116589in}}%
\pgfpathlineto{\pgfqpoint{2.250185in}{3.118469in}}%
\pgfpathlineto{\pgfqpoint{2.254735in}{3.124837in}}%
\pgfpathlineto{\pgfqpoint{2.261157in}{3.133146in}}%
\pgfpathlineto{\pgfqpoint{2.264101in}{3.133850in}}%
\pgfpathlineto{\pgfqpoint{2.267044in}{3.132048in}}%
\pgfpathlineto{\pgfqpoint{2.271326in}{3.126168in}}%
\pgfpathlineto{\pgfqpoint{2.278016in}{3.117339in}}%
\pgfpathlineto{\pgfqpoint{2.280960in}{3.116551in}}%
\pgfpathlineto{\pgfqpoint{2.283904in}{3.118275in}}%
\pgfpathlineto{\pgfqpoint{2.288185in}{3.124096in}}%
\pgfpathlineto{\pgfqpoint{2.295143in}{3.133194in}}%
\pgfpathlineto{\pgfqpoint{2.298087in}{3.133835in}}%
\pgfpathlineto{\pgfqpoint{2.301031in}{3.131974in}}%
\pgfpathlineto{\pgfqpoint{2.305580in}{3.125619in}}%
\pgfpathlineto{\pgfqpoint{2.312003in}{3.117289in}}%
\pgfpathlineto{\pgfqpoint{2.314946in}{3.116564in}}%
\pgfpathlineto{\pgfqpoint{2.317890in}{3.118348in}}%
\pgfpathlineto{\pgfqpoint{2.322172in}{3.124214in}}%
\pgfpathlineto{\pgfqpoint{2.328862in}{3.133064in}}%
\pgfpathlineto{\pgfqpoint{2.331806in}{3.133872in}}%
\pgfpathlineto{\pgfqpoint{2.334749in}{3.132166in}}%
\pgfpathlineto{\pgfqpoint{2.339031in}{3.126360in}}%
\pgfpathlineto{\pgfqpoint{2.345989in}{3.117240in}}%
\pgfpathlineto{\pgfqpoint{2.348932in}{3.116579in}}%
\pgfpathlineto{\pgfqpoint{2.351876in}{3.118422in}}%
\pgfpathlineto{\pgfqpoint{2.356158in}{3.124333in}}%
\pgfpathlineto{\pgfqpoint{2.362848in}{3.133115in}}%
\pgfpathlineto{\pgfqpoint{2.365792in}{3.133859in}}%
\pgfpathlineto{\pgfqpoint{2.368735in}{3.132094in}}%
\pgfpathlineto{\pgfqpoint{2.373017in}{3.126241in}}%
\pgfpathlineto{\pgfqpoint{2.379707in}{3.117371in}}%
\pgfpathlineto{\pgfqpoint{2.382651in}{3.116543in}}%
\pgfpathlineto{\pgfqpoint{2.385595in}{3.118230in}}%
\pgfpathlineto{\pgfqpoint{2.389876in}{3.124022in}}%
\pgfpathlineto{\pgfqpoint{2.396834in}{3.133164in}}%
\pgfpathlineto{\pgfqpoint{2.399778in}{3.133845in}}%
\pgfpathlineto{\pgfqpoint{2.402722in}{3.132020in}}%
\pgfpathlineto{\pgfqpoint{2.407003in}{3.126123in}}%
\pgfpathlineto{\pgfqpoint{2.413694in}{3.117320in}}%
\pgfpathlineto{\pgfqpoint{2.416637in}{3.116556in}}%
\pgfpathlineto{\pgfqpoint{2.419581in}{3.118302in}}%
\pgfpathlineto{\pgfqpoint{2.423863in}{3.124140in}}%
\pgfpathlineto{\pgfqpoint{2.430820in}{3.133211in}}%
\pgfpathlineto{\pgfqpoint{2.433764in}{3.133829in}}%
\pgfpathlineto{\pgfqpoint{2.436708in}{3.131946in}}%
\pgfpathlineto{\pgfqpoint{2.441257in}{3.125574in}}%
\pgfpathlineto{\pgfqpoint{2.447680in}{3.117270in}}%
\pgfpathlineto{\pgfqpoint{2.450623in}{3.116569in}}%
\pgfpathlineto{\pgfqpoint{2.453567in}{3.118375in}}%
\pgfpathlineto{\pgfqpoint{2.457849in}{3.124259in}}%
\pgfpathlineto{\pgfqpoint{2.464539in}{3.133083in}}%
\pgfpathlineto{\pgfqpoint{2.467483in}{3.133867in}}%
\pgfpathlineto{\pgfqpoint{2.470426in}{3.132139in}}%
\pgfpathlineto{\pgfqpoint{2.474708in}{3.126315in}}%
\pgfpathlineto{\pgfqpoint{2.481666in}{3.117222in}}%
\pgfpathlineto{\pgfqpoint{2.484610in}{3.116585in}}%
\pgfpathlineto{\pgfqpoint{2.487553in}{3.118450in}}%
\pgfpathlineto{\pgfqpoint{2.492103in}{3.124808in}}%
\pgfpathlineto{\pgfqpoint{2.498525in}{3.133133in}}%
\pgfpathlineto{\pgfqpoint{2.501469in}{3.133854in}}%
\pgfpathlineto{\pgfqpoint{2.504413in}{3.132066in}}%
\pgfpathlineto{\pgfqpoint{2.508694in}{3.126197in}}%
\pgfpathlineto{\pgfqpoint{2.515385in}{3.117352in}}%
\pgfpathlineto{\pgfqpoint{2.518328in}{3.116548in}}%
\pgfpathlineto{\pgfqpoint{2.521272in}{3.118257in}}%
\pgfpathlineto{\pgfqpoint{2.525554in}{3.124067in}}%
\pgfpathlineto{\pgfqpoint{2.532511in}{3.133182in}}%
\pgfpathlineto{\pgfqpoint{2.535455in}{3.133839in}}%
\pgfpathlineto{\pgfqpoint{2.538399in}{3.131992in}}%
\pgfpathlineto{\pgfqpoint{2.542681in}{3.126079in}}%
\pgfpathlineto{\pgfqpoint{2.549371in}{3.117301in}}%
\pgfpathlineto{\pgfqpoint{2.552314in}{3.116561in}}%
\pgfpathlineto{\pgfqpoint{2.555258in}{3.118330in}}%
\pgfpathlineto{\pgfqpoint{2.559540in}{3.124185in}}%
\pgfpathlineto{\pgfqpoint{2.566230in}{3.133051in}}%
\pgfpathlineto{\pgfqpoint{2.569174in}{3.133875in}}%
\pgfpathlineto{\pgfqpoint{2.572117in}{3.132184in}}%
\pgfpathlineto{\pgfqpoint{2.576399in}{3.126389in}}%
\pgfpathlineto{\pgfqpoint{2.583357in}{3.117252in}}%
\pgfpathlineto{\pgfqpoint{2.586301in}{3.116575in}}%
\pgfpathlineto{\pgfqpoint{2.589244in}{3.118403in}}%
\pgfpathlineto{\pgfqpoint{2.593526in}{3.124303in}}%
\pgfpathlineto{\pgfqpoint{2.600216in}{3.133102in}}%
\pgfpathlineto{\pgfqpoint{2.603160in}{3.133863in}}%
\pgfpathlineto{\pgfqpoint{2.606104in}{3.132112in}}%
\pgfpathlineto{\pgfqpoint{2.610385in}{3.126271in}}%
\pgfpathlineto{\pgfqpoint{2.617076in}{3.117384in}}%
\pgfpathlineto{\pgfqpoint{2.620019in}{3.116541in}}%
\pgfpathlineto{\pgfqpoint{2.622963in}{3.118213in}}%
\pgfpathlineto{\pgfqpoint{2.627245in}{3.123993in}}%
\pgfpathlineto{\pgfqpoint{2.634202in}{3.133152in}}%
\pgfpathlineto{\pgfqpoint{2.637146in}{3.133849in}}%
\pgfpathlineto{\pgfqpoint{2.640090in}{3.132039in}}%
\pgfpathlineto{\pgfqpoint{2.644371in}{3.126153in}}%
\pgfpathlineto{\pgfqpoint{2.651062in}{3.117332in}}%
\pgfpathlineto{\pgfqpoint{2.654005in}{3.116552in}}%
\pgfpathlineto{\pgfqpoint{2.656949in}{3.118284in}}%
\pgfpathlineto{\pgfqpoint{2.661231in}{3.124111in}}%
\pgfpathlineto{\pgfqpoint{2.668189in}{3.133200in}}%
\pgfpathlineto{\pgfqpoint{2.671132in}{3.133833in}}%
\pgfpathlineto{\pgfqpoint{2.674076in}{3.131964in}}%
\pgfpathlineto{\pgfqpoint{2.678625in}{3.125604in}}%
\pgfpathlineto{\pgfqpoint{2.685048in}{3.117282in}}%
\pgfpathlineto{\pgfqpoint{2.687992in}{3.116566in}}%
\pgfpathlineto{\pgfqpoint{2.690935in}{3.118357in}}%
\pgfpathlineto{\pgfqpoint{2.695217in}{3.124229in}}%
\pgfpathlineto{\pgfqpoint{2.701907in}{3.133071in}}%
\pgfpathlineto{\pgfqpoint{2.704851in}{3.133870in}}%
\pgfpathlineto{\pgfqpoint{2.707795in}{3.132157in}}%
\pgfpathlineto{\pgfqpoint{2.712076in}{3.126345in}}%
\pgfpathlineto{\pgfqpoint{2.719034in}{3.117234in}}%
\pgfpathlineto{\pgfqpoint{2.721978in}{3.116581in}}%
\pgfpathlineto{\pgfqpoint{2.724921in}{3.118431in}}%
\pgfpathlineto{\pgfqpoint{2.729471in}{3.124778in}}%
\pgfpathlineto{\pgfqpoint{2.735893in}{3.133121in}}%
\pgfpathlineto{\pgfqpoint{2.738837in}{3.133857in}}%
\pgfpathlineto{\pgfqpoint{2.741781in}{3.132085in}}%
\pgfpathlineto{\pgfqpoint{2.746062in}{3.126227in}}%
\pgfpathlineto{\pgfqpoint{2.752753in}{3.117364in}}%
\pgfpathlineto{\pgfqpoint{2.755696in}{3.116545in}}%
\pgfpathlineto{\pgfqpoint{2.758640in}{3.118239in}}%
\pgfpathlineto{\pgfqpoint{2.762922in}{3.124037in}}%
\pgfpathlineto{\pgfqpoint{2.769880in}{3.133170in}}%
\pgfpathlineto{\pgfqpoint{2.772823in}{3.133843in}}%
\pgfpathlineto{\pgfqpoint{2.775767in}{3.132011in}}%
\pgfpathlineto{\pgfqpoint{2.780049in}{3.126108in}}%
\pgfpathlineto{\pgfqpoint{2.786739in}{3.117314in}}%
\pgfpathlineto{\pgfqpoint{2.789683in}{3.116557in}}%
\pgfpathlineto{\pgfqpoint{2.792626in}{3.118311in}}%
\pgfpathlineto{\pgfqpoint{2.796908in}{3.124155in}}%
\pgfpathlineto{\pgfqpoint{2.803598in}{3.133038in}}%
\pgfpathlineto{\pgfqpoint{2.806542in}{3.133878in}}%
\pgfpathlineto{\pgfqpoint{2.809486in}{3.132202in}}%
\pgfpathlineto{\pgfqpoint{2.813767in}{3.126419in}}%
\pgfpathlineto{\pgfqpoint{2.820725in}{3.117264in}}%
\pgfpathlineto{\pgfqpoint{2.823669in}{3.116571in}}%
\pgfpathlineto{\pgfqpoint{2.826612in}{3.118385in}}%
\pgfpathlineto{\pgfqpoint{2.830894in}{3.124274in}}%
\pgfpathlineto{\pgfqpoint{2.837584in}{3.133090in}}%
\pgfpathlineto{\pgfqpoint{2.840528in}{3.133866in}}%
\pgfpathlineto{\pgfqpoint{2.843472in}{3.132130in}}%
\pgfpathlineto{\pgfqpoint{2.847753in}{3.126301in}}%
\pgfpathlineto{\pgfqpoint{2.854711in}{3.117216in}}%
\pgfpathlineto{\pgfqpoint{2.857655in}{3.116587in}}%
\pgfpathlineto{\pgfqpoint{2.860599in}{3.118459in}}%
\pgfpathlineto{\pgfqpoint{2.865148in}{3.124823in}}%
\pgfpathlineto{\pgfqpoint{2.871571in}{3.133139in}}%
\pgfpathlineto{\pgfqpoint{2.874514in}{3.133852in}}%
\pgfpathlineto{\pgfqpoint{2.877458in}{3.132057in}}%
\pgfpathlineto{\pgfqpoint{2.881740in}{3.126182in}}%
\pgfpathlineto{\pgfqpoint{2.888430in}{3.117345in}}%
\pgfpathlineto{\pgfqpoint{2.891374in}{3.116549in}}%
\pgfpathlineto{\pgfqpoint{2.894317in}{3.118266in}}%
\pgfpathlineto{\pgfqpoint{2.898599in}{3.124081in}}%
\pgfpathlineto{\pgfqpoint{2.905557in}{3.133188in}}%
\pgfpathlineto{\pgfqpoint{2.908500in}{3.133837in}}%
\pgfpathlineto{\pgfqpoint{2.911444in}{3.131983in}}%
\pgfpathlineto{\pgfqpoint{2.915993in}{3.125633in}}%
\pgfpathlineto{\pgfqpoint{2.922416in}{3.117295in}}%
\pgfpathlineto{\pgfqpoint{2.925360in}{3.116562in}}%
\pgfpathlineto{\pgfqpoint{2.928303in}{3.118339in}}%
\pgfpathlineto{\pgfqpoint{2.932585in}{3.124200in}}%
\pgfpathlineto{\pgfqpoint{2.939275in}{3.133058in}}%
\pgfpathlineto{\pgfqpoint{2.942219in}{3.133873in}}%
\pgfpathlineto{\pgfqpoint{2.945163in}{3.132175in}}%
\pgfpathlineto{\pgfqpoint{2.949444in}{3.126374in}}%
\pgfpathlineto{\pgfqpoint{2.956402in}{3.117246in}}%
\pgfpathlineto{\pgfqpoint{2.959346in}{3.116577in}}%
\pgfpathlineto{\pgfqpoint{2.962290in}{3.118412in}}%
\pgfpathlineto{\pgfqpoint{2.966571in}{3.124318in}}%
\pgfpathlineto{\pgfqpoint{2.973262in}{3.133108in}}%
\pgfpathlineto{\pgfqpoint{2.976205in}{3.133861in}}%
\pgfpathlineto{\pgfqpoint{2.979149in}{3.132103in}}%
\pgfpathlineto{\pgfqpoint{2.983431in}{3.126256in}}%
\pgfpathlineto{\pgfqpoint{2.990121in}{3.117377in}}%
\pgfpathlineto{\pgfqpoint{2.993064in}{3.116542in}}%
\pgfpathlineto{\pgfqpoint{2.996008in}{3.118222in}}%
\pgfpathlineto{\pgfqpoint{3.000290in}{3.124008in}}%
\pgfpathlineto{\pgfqpoint{3.007248in}{3.133158in}}%
\pgfpathlineto{\pgfqpoint{3.010191in}{3.133847in}}%
\pgfpathlineto{\pgfqpoint{3.013135in}{3.132030in}}%
\pgfpathlineto{\pgfqpoint{3.017417in}{3.126138in}}%
\pgfpathlineto{\pgfqpoint{3.024107in}{3.117326in}}%
\pgfpathlineto{\pgfqpoint{3.027051in}{3.116554in}}%
\pgfpathlineto{\pgfqpoint{3.029994in}{3.118293in}}%
\pgfpathlineto{\pgfqpoint{3.034276in}{3.124126in}}%
\pgfpathlineto{\pgfqpoint{3.041234in}{3.133206in}}%
\pgfpathlineto{\pgfqpoint{3.044178in}{3.133831in}}%
\pgfpathlineto{\pgfqpoint{3.047121in}{3.131955in}}%
\pgfpathlineto{\pgfqpoint{3.051671in}{3.125589in}}%
\pgfpathlineto{\pgfqpoint{3.058093in}{3.117276in}}%
\pgfpathlineto{\pgfqpoint{3.061037in}{3.116567in}}%
\pgfpathlineto{\pgfqpoint{3.063981in}{3.118366in}}%
\pgfpathlineto{\pgfqpoint{3.068262in}{3.124244in}}%
\pgfpathlineto{\pgfqpoint{3.074952in}{3.133077in}}%
\pgfpathlineto{\pgfqpoint{3.077896in}{3.133869in}}%
\pgfpathlineto{\pgfqpoint{3.080840in}{3.132148in}}%
\pgfpathlineto{\pgfqpoint{3.085122in}{3.126330in}}%
\pgfpathlineto{\pgfqpoint{3.092079in}{3.117228in}}%
\pgfpathlineto{\pgfqpoint{3.095023in}{3.116583in}}%
\pgfpathlineto{\pgfqpoint{3.097967in}{3.118440in}}%
\pgfpathlineto{\pgfqpoint{3.102516in}{3.124793in}}%
\pgfpathlineto{\pgfqpoint{3.108939in}{3.133127in}}%
\pgfpathlineto{\pgfqpoint{3.111882in}{3.133856in}}%
\pgfpathlineto{\pgfqpoint{3.114826in}{3.132076in}}%
\pgfpathlineto{\pgfqpoint{3.119108in}{3.126212in}}%
\pgfpathlineto{\pgfqpoint{3.125798in}{3.117358in}}%
\pgfpathlineto{\pgfqpoint{3.128742in}{3.116546in}}%
\pgfpathlineto{\pgfqpoint{3.131685in}{3.118248in}}%
\pgfpathlineto{\pgfqpoint{3.135967in}{3.124052in}}%
\pgfpathlineto{\pgfqpoint{3.142925in}{3.133176in}}%
\pgfpathlineto{\pgfqpoint{3.145869in}{3.133841in}}%
\pgfpathlineto{\pgfqpoint{3.148812in}{3.132002in}}%
\pgfpathlineto{\pgfqpoint{3.153094in}{3.126094in}}%
\pgfpathlineto{\pgfqpoint{3.159784in}{3.117307in}}%
\pgfpathlineto{\pgfqpoint{3.162728in}{3.116559in}}%
\pgfpathlineto{\pgfqpoint{3.165672in}{3.118320in}}%
\pgfpathlineto{\pgfqpoint{3.169953in}{3.124170in}}%
\pgfpathlineto{\pgfqpoint{3.176643in}{3.133045in}}%
\pgfpathlineto{\pgfqpoint{3.179587in}{3.133876in}}%
\pgfpathlineto{\pgfqpoint{3.182531in}{3.132193in}}%
\pgfpathlineto{\pgfqpoint{3.186813in}{3.126404in}}%
\pgfpathlineto{\pgfqpoint{3.193770in}{3.117258in}}%
\pgfpathlineto{\pgfqpoint{3.196714in}{3.116573in}}%
\pgfpathlineto{\pgfqpoint{3.199658in}{3.118394in}}%
\pgfpathlineto{\pgfqpoint{3.203939in}{3.124288in}}%
\pgfpathlineto{\pgfqpoint{3.210630in}{3.133096in}}%
\pgfpathlineto{\pgfqpoint{3.213573in}{3.133864in}}%
\pgfpathlineto{\pgfqpoint{3.216517in}{3.132121in}}%
\pgfpathlineto{\pgfqpoint{3.220799in}{3.126286in}}%
\pgfpathlineto{\pgfqpoint{3.227757in}{3.117210in}}%
\pgfpathlineto{\pgfqpoint{3.230700in}{3.116589in}}%
\pgfpathlineto{\pgfqpoint{3.233644in}{3.118469in}}%
\pgfpathlineto{\pgfqpoint{3.238193in}{3.124837in}}%
\pgfpathlineto{\pgfqpoint{3.244616in}{3.133146in}}%
\pgfpathlineto{\pgfqpoint{3.247560in}{3.133850in}}%
\pgfpathlineto{\pgfqpoint{3.250503in}{3.132048in}}%
\pgfpathlineto{\pgfqpoint{3.254785in}{3.126168in}}%
\pgfpathlineto{\pgfqpoint{3.261475in}{3.117339in}}%
\pgfpathlineto{\pgfqpoint{3.264419in}{3.116551in}}%
\pgfpathlineto{\pgfqpoint{3.267363in}{3.118275in}}%
\pgfpathlineto{\pgfqpoint{3.271644in}{3.124096in}}%
\pgfpathlineto{\pgfqpoint{3.278602in}{3.133194in}}%
\pgfpathlineto{\pgfqpoint{3.281546in}{3.133835in}}%
\pgfpathlineto{\pgfqpoint{3.284489in}{3.131974in}}%
\pgfpathlineto{\pgfqpoint{3.289039in}{3.125619in}}%
\pgfpathlineto{\pgfqpoint{3.295461in}{3.117289in}}%
\pgfpathlineto{\pgfqpoint{3.298405in}{3.116564in}}%
\pgfpathlineto{\pgfqpoint{3.301349in}{3.118348in}}%
\pgfpathlineto{\pgfqpoint{3.305630in}{3.124214in}}%
\pgfpathlineto{\pgfqpoint{3.312321in}{3.133064in}}%
\pgfpathlineto{\pgfqpoint{3.315264in}{3.133872in}}%
\pgfpathlineto{\pgfqpoint{3.318208in}{3.132166in}}%
\pgfpathlineto{\pgfqpoint{3.322490in}{3.126360in}}%
\pgfpathlineto{\pgfqpoint{3.329448in}{3.117240in}}%
\pgfpathlineto{\pgfqpoint{3.332391in}{3.116579in}}%
\pgfpathlineto{\pgfqpoint{3.335335in}{3.118422in}}%
\pgfpathlineto{\pgfqpoint{3.339617in}{3.124333in}}%
\pgfpathlineto{\pgfqpoint{3.346307in}{3.133115in}}%
\pgfpathlineto{\pgfqpoint{3.349251in}{3.133859in}}%
\pgfpathlineto{\pgfqpoint{3.352194in}{3.132094in}}%
\pgfpathlineto{\pgfqpoint{3.356476in}{3.126241in}}%
\pgfpathlineto{\pgfqpoint{3.363166in}{3.117371in}}%
\pgfpathlineto{\pgfqpoint{3.366110in}{3.116543in}}%
\pgfpathlineto{\pgfqpoint{3.369053in}{3.118230in}}%
\pgfpathlineto{\pgfqpoint{3.373335in}{3.124022in}}%
\pgfpathlineto{\pgfqpoint{3.380293in}{3.133164in}}%
\pgfpathlineto{\pgfqpoint{3.383237in}{3.133845in}}%
\pgfpathlineto{\pgfqpoint{3.386180in}{3.132020in}}%
\pgfpathlineto{\pgfqpoint{3.390462in}{3.126123in}}%
\pgfpathlineto{\pgfqpoint{3.397152in}{3.117320in}}%
\pgfpathlineto{\pgfqpoint{3.400096in}{3.116556in}}%
\pgfpathlineto{\pgfqpoint{3.403040in}{3.118302in}}%
\pgfpathlineto{\pgfqpoint{3.407321in}{3.124140in}}%
\pgfpathlineto{\pgfqpoint{3.414279in}{3.133211in}}%
\pgfpathlineto{\pgfqpoint{3.417223in}{3.133829in}}%
\pgfpathlineto{\pgfqpoint{3.420167in}{3.131946in}}%
\pgfpathlineto{\pgfqpoint{3.424716in}{3.125574in}}%
\pgfpathlineto{\pgfqpoint{3.431138in}{3.117270in}}%
\pgfpathlineto{\pgfqpoint{3.434082in}{3.116569in}}%
\pgfpathlineto{\pgfqpoint{3.437026in}{3.118375in}}%
\pgfpathlineto{\pgfqpoint{3.441308in}{3.124259in}}%
\pgfpathlineto{\pgfqpoint{3.447998in}{3.133083in}}%
\pgfpathlineto{\pgfqpoint{3.450941in}{3.133867in}}%
\pgfpathlineto{\pgfqpoint{3.453885in}{3.132139in}}%
\pgfpathlineto{\pgfqpoint{3.458167in}{3.126315in}}%
\pgfpathlineto{\pgfqpoint{3.465125in}{3.117222in}}%
\pgfpathlineto{\pgfqpoint{3.468068in}{3.116585in}}%
\pgfpathlineto{\pgfqpoint{3.471012in}{3.118450in}}%
\pgfpathlineto{\pgfqpoint{3.475561in}{3.124808in}}%
\pgfpathlineto{\pgfqpoint{3.481984in}{3.133133in}}%
\pgfpathlineto{\pgfqpoint{3.484928in}{3.133854in}}%
\pgfpathlineto{\pgfqpoint{3.487871in}{3.132066in}}%
\pgfpathlineto{\pgfqpoint{3.492153in}{3.126197in}}%
\pgfpathlineto{\pgfqpoint{3.498843in}{3.117352in}}%
\pgfpathlineto{\pgfqpoint{3.501787in}{3.116548in}}%
\pgfpathlineto{\pgfqpoint{3.504731in}{3.118257in}}%
\pgfpathlineto{\pgfqpoint{3.509012in}{3.124067in}}%
\pgfpathlineto{\pgfqpoint{3.515970in}{3.133182in}}%
\pgfpathlineto{\pgfqpoint{3.518914in}{3.133839in}}%
\pgfpathlineto{\pgfqpoint{3.521858in}{3.131992in}}%
\pgfpathlineto{\pgfqpoint{3.526139in}{3.126079in}}%
\pgfpathlineto{\pgfqpoint{3.532829in}{3.117301in}}%
\pgfpathlineto{\pgfqpoint{3.535773in}{3.116561in}}%
\pgfpathlineto{\pgfqpoint{3.538717in}{3.118330in}}%
\pgfpathlineto{\pgfqpoint{3.542999in}{3.124185in}}%
\pgfpathlineto{\pgfqpoint{3.549689in}{3.133051in}}%
\pgfpathlineto{\pgfqpoint{3.552632in}{3.133875in}}%
\pgfpathlineto{\pgfqpoint{3.555576in}{3.132184in}}%
\pgfpathlineto{\pgfqpoint{3.559858in}{3.126389in}}%
\pgfpathlineto{\pgfqpoint{3.566816in}{3.117252in}}%
\pgfpathlineto{\pgfqpoint{3.569759in}{3.116575in}}%
\pgfpathlineto{\pgfqpoint{3.572703in}{3.118403in}}%
\pgfpathlineto{\pgfqpoint{3.576985in}{3.124303in}}%
\pgfpathlineto{\pgfqpoint{3.583675in}{3.133102in}}%
\pgfpathlineto{\pgfqpoint{3.586619in}{3.133863in}}%
\pgfpathlineto{\pgfqpoint{3.589562in}{3.132112in}}%
\pgfpathlineto{\pgfqpoint{3.593844in}{3.126271in}}%
\pgfpathlineto{\pgfqpoint{3.600534in}{3.117384in}}%
\pgfpathlineto{\pgfqpoint{3.603478in}{3.116541in}}%
\pgfpathlineto{\pgfqpoint{3.606422in}{3.118213in}}%
\pgfpathlineto{\pgfqpoint{3.610703in}{3.123993in}}%
\pgfpathlineto{\pgfqpoint{3.617661in}{3.133152in}}%
\pgfpathlineto{\pgfqpoint{3.620605in}{3.133849in}}%
\pgfpathlineto{\pgfqpoint{3.623549in}{3.132039in}}%
\pgfpathlineto{\pgfqpoint{3.627830in}{3.126153in}}%
\pgfpathlineto{\pgfqpoint{3.634520in}{3.117332in}}%
\pgfpathlineto{\pgfqpoint{3.637464in}{3.116552in}}%
\pgfpathlineto{\pgfqpoint{3.640408in}{3.118284in}}%
\pgfpathlineto{\pgfqpoint{3.644690in}{3.124111in}}%
\pgfpathlineto{\pgfqpoint{3.651647in}{3.133200in}}%
\pgfpathlineto{\pgfqpoint{3.654591in}{3.133833in}}%
\pgfpathlineto{\pgfqpoint{3.657535in}{3.131964in}}%
\pgfpathlineto{\pgfqpoint{3.662084in}{3.125604in}}%
\pgfpathlineto{\pgfqpoint{3.668507in}{3.117282in}}%
\pgfpathlineto{\pgfqpoint{3.671450in}{3.116566in}}%
\pgfpathlineto{\pgfqpoint{3.674394in}{3.118357in}}%
\pgfpathlineto{\pgfqpoint{3.678676in}{3.124229in}}%
\pgfpathlineto{\pgfqpoint{3.685366in}{3.133071in}}%
\pgfpathlineto{\pgfqpoint{3.688310in}{3.133870in}}%
\pgfpathlineto{\pgfqpoint{3.691253in}{3.132157in}}%
\pgfpathlineto{\pgfqpoint{3.695535in}{3.126345in}}%
\pgfpathlineto{\pgfqpoint{3.702493in}{3.117234in}}%
\pgfpathlineto{\pgfqpoint{3.705437in}{3.116581in}}%
\pgfpathlineto{\pgfqpoint{3.708380in}{3.118431in}}%
\pgfpathlineto{\pgfqpoint{3.712930in}{3.124778in}}%
\pgfpathlineto{\pgfqpoint{3.719352in}{3.133121in}}%
\pgfpathlineto{\pgfqpoint{3.722296in}{3.133857in}}%
\pgfpathlineto{\pgfqpoint{3.725239in}{3.132085in}}%
\pgfpathlineto{\pgfqpoint{3.729521in}{3.126227in}}%
\pgfpathlineto{\pgfqpoint{3.736211in}{3.117364in}}%
\pgfpathlineto{\pgfqpoint{3.739155in}{3.116545in}}%
\pgfpathlineto{\pgfqpoint{3.742099in}{3.118239in}}%
\pgfpathlineto{\pgfqpoint{3.746381in}{3.124037in}}%
\pgfpathlineto{\pgfqpoint{3.753338in}{3.133170in}}%
\pgfpathlineto{\pgfqpoint{3.756282in}{3.133843in}}%
\pgfpathlineto{\pgfqpoint{3.759226in}{3.132011in}}%
\pgfpathlineto{\pgfqpoint{3.763507in}{3.126108in}}%
\pgfpathlineto{\pgfqpoint{3.770198in}{3.117314in}}%
\pgfpathlineto{\pgfqpoint{3.773141in}{3.116557in}}%
\pgfpathlineto{\pgfqpoint{3.776085in}{3.118311in}}%
\pgfpathlineto{\pgfqpoint{3.780367in}{3.124155in}}%
\pgfpathlineto{\pgfqpoint{3.787057in}{3.133038in}}%
\pgfpathlineto{\pgfqpoint{3.790001in}{3.133878in}}%
\pgfpathlineto{\pgfqpoint{3.792944in}{3.132202in}}%
\pgfpathlineto{\pgfqpoint{3.797226in}{3.126419in}}%
\pgfpathlineto{\pgfqpoint{3.804184in}{3.117264in}}%
\pgfpathlineto{\pgfqpoint{3.807127in}{3.116571in}}%
\pgfpathlineto{\pgfqpoint{3.810071in}{3.118385in}}%
\pgfpathlineto{\pgfqpoint{3.814353in}{3.124274in}}%
\pgfpathlineto{\pgfqpoint{3.821043in}{3.133090in}}%
\pgfpathlineto{\pgfqpoint{3.823987in}{3.133866in}}%
\pgfpathlineto{\pgfqpoint{3.826930in}{3.132130in}}%
\pgfpathlineto{\pgfqpoint{3.831212in}{3.126301in}}%
\pgfpathlineto{\pgfqpoint{3.838170in}{3.117216in}}%
\pgfpathlineto{\pgfqpoint{3.841114in}{3.116587in}}%
\pgfpathlineto{\pgfqpoint{3.844057in}{3.118459in}}%
\pgfpathlineto{\pgfqpoint{3.848607in}{3.124823in}}%
\pgfpathlineto{\pgfqpoint{3.855029in}{3.133139in}}%
\pgfpathlineto{\pgfqpoint{3.857973in}{3.133852in}}%
\pgfpathlineto{\pgfqpoint{3.860917in}{3.132057in}}%
\pgfpathlineto{\pgfqpoint{3.865198in}{3.126182in}}%
\pgfpathlineto{\pgfqpoint{3.871889in}{3.117345in}}%
\pgfpathlineto{\pgfqpoint{3.874832in}{3.116549in}}%
\pgfpathlineto{\pgfqpoint{3.877776in}{3.118266in}}%
\pgfpathlineto{\pgfqpoint{3.882058in}{3.124081in}}%
\pgfpathlineto{\pgfqpoint{3.889015in}{3.133188in}}%
\pgfpathlineto{\pgfqpoint{3.891959in}{3.133837in}}%
\pgfpathlineto{\pgfqpoint{3.894903in}{3.131983in}}%
\pgfpathlineto{\pgfqpoint{3.899452in}{3.125633in}}%
\pgfpathlineto{\pgfqpoint{3.905875in}{3.117295in}}%
\pgfpathlineto{\pgfqpoint{3.908818in}{3.116562in}}%
\pgfpathlineto{\pgfqpoint{3.911762in}{3.118339in}}%
\pgfpathlineto{\pgfqpoint{3.916044in}{3.124200in}}%
\pgfpathlineto{\pgfqpoint{3.922734in}{3.133058in}}%
\pgfpathlineto{\pgfqpoint{3.925678in}{3.133873in}}%
\pgfpathlineto{\pgfqpoint{3.928621in}{3.132175in}}%
\pgfpathlineto{\pgfqpoint{3.932903in}{3.126374in}}%
\pgfpathlineto{\pgfqpoint{3.939861in}{3.117246in}}%
\pgfpathlineto{\pgfqpoint{3.942805in}{3.116577in}}%
\pgfpathlineto{\pgfqpoint{3.945748in}{3.118412in}}%
\pgfpathlineto{\pgfqpoint{3.950030in}{3.124318in}}%
\pgfpathlineto{\pgfqpoint{3.956720in}{3.133108in}}%
\pgfpathlineto{\pgfqpoint{3.959664in}{3.133861in}}%
\pgfpathlineto{\pgfqpoint{3.962608in}{3.132103in}}%
\pgfpathlineto{\pgfqpoint{3.966889in}{3.126256in}}%
\pgfpathlineto{\pgfqpoint{3.973580in}{3.117377in}}%
\pgfpathlineto{\pgfqpoint{3.976523in}{3.116542in}}%
\pgfpathlineto{\pgfqpoint{3.979467in}{3.118222in}}%
\pgfpathlineto{\pgfqpoint{3.983749in}{3.124008in}}%
\pgfpathlineto{\pgfqpoint{3.990706in}{3.133158in}}%
\pgfpathlineto{\pgfqpoint{3.993650in}{3.133847in}}%
\pgfpathlineto{\pgfqpoint{3.996594in}{3.132030in}}%
\pgfpathlineto{\pgfqpoint{4.000876in}{3.126138in}}%
\pgfpathlineto{\pgfqpoint{4.007566in}{3.117326in}}%
\pgfpathlineto{\pgfqpoint{4.010509in}{3.116554in}}%
\pgfpathlineto{\pgfqpoint{4.013453in}{3.118293in}}%
\pgfpathlineto{\pgfqpoint{4.017735in}{3.124126in}}%
\pgfpathlineto{\pgfqpoint{4.024693in}{3.133206in}}%
\pgfpathlineto{\pgfqpoint{4.027636in}{3.133831in}}%
\pgfpathlineto{\pgfqpoint{4.030580in}{3.131955in}}%
\pgfpathlineto{\pgfqpoint{4.035129in}{3.125589in}}%
\pgfpathlineto{\pgfqpoint{4.041552in}{3.117276in}}%
\pgfpathlineto{\pgfqpoint{4.044496in}{3.116567in}}%
\pgfpathlineto{\pgfqpoint{4.047439in}{3.118366in}}%
\pgfpathlineto{\pgfqpoint{4.051721in}{3.124244in}}%
\pgfpathlineto{\pgfqpoint{4.058411in}{3.133077in}}%
\pgfpathlineto{\pgfqpoint{4.061355in}{3.133869in}}%
\pgfpathlineto{\pgfqpoint{4.064299in}{3.132148in}}%
\pgfpathlineto{\pgfqpoint{4.068580in}{3.126330in}}%
\pgfpathlineto{\pgfqpoint{4.075538in}{3.117228in}}%
\pgfpathlineto{\pgfqpoint{4.078482in}{3.116583in}}%
\pgfpathlineto{\pgfqpoint{4.081425in}{3.118440in}}%
\pgfpathlineto{\pgfqpoint{4.085975in}{3.124793in}}%
\pgfpathlineto{\pgfqpoint{4.092397in}{3.133127in}}%
\pgfpathlineto{\pgfqpoint{4.095341in}{3.133856in}}%
\pgfpathlineto{\pgfqpoint{4.098285in}{3.132076in}}%
\pgfpathlineto{\pgfqpoint{4.102567in}{3.126212in}}%
\pgfpathlineto{\pgfqpoint{4.109257in}{3.117358in}}%
\pgfpathlineto{\pgfqpoint{4.112200in}{3.116546in}}%
\pgfpathlineto{\pgfqpoint{4.115144in}{3.118248in}}%
\pgfpathlineto{\pgfqpoint{4.119426in}{3.124052in}}%
\pgfpathlineto{\pgfqpoint{4.126384in}{3.133176in}}%
\pgfpathlineto{\pgfqpoint{4.129327in}{3.133841in}}%
\pgfpathlineto{\pgfqpoint{4.132271in}{3.132002in}}%
\pgfpathlineto{\pgfqpoint{4.136553in}{3.126094in}}%
\pgfpathlineto{\pgfqpoint{4.143243in}{3.117307in}}%
\pgfpathlineto{\pgfqpoint{4.146187in}{3.116559in}}%
\pgfpathlineto{\pgfqpoint{4.149130in}{3.118320in}}%
\pgfpathlineto{\pgfqpoint{4.153412in}{3.124170in}}%
\pgfpathlineto{\pgfqpoint{4.160102in}{3.133045in}}%
\pgfpathlineto{\pgfqpoint{4.163046in}{3.133876in}}%
\pgfpathlineto{\pgfqpoint{4.165990in}{3.132193in}}%
\pgfpathlineto{\pgfqpoint{4.170271in}{3.126404in}}%
\pgfpathlineto{\pgfqpoint{4.177229in}{3.117258in}}%
\pgfpathlineto{\pgfqpoint{4.180173in}{3.116573in}}%
\pgfpathlineto{\pgfqpoint{4.183116in}{3.118394in}}%
\pgfpathlineto{\pgfqpoint{4.187398in}{3.124288in}}%
\pgfpathlineto{\pgfqpoint{4.194088in}{3.133096in}}%
\pgfpathlineto{\pgfqpoint{4.197032in}{3.133864in}}%
\pgfpathlineto{\pgfqpoint{4.199976in}{3.132121in}}%
\pgfpathlineto{\pgfqpoint{4.204257in}{3.126286in}}%
\pgfpathlineto{\pgfqpoint{4.211215in}{3.117210in}}%
\pgfpathlineto{\pgfqpoint{4.214159in}{3.116589in}}%
\pgfpathlineto{\pgfqpoint{4.217103in}{3.118469in}}%
\pgfpathlineto{\pgfqpoint{4.221652in}{3.124837in}}%
\pgfpathlineto{\pgfqpoint{4.228075in}{3.133146in}}%
\pgfpathlineto{\pgfqpoint{4.231018in}{3.133850in}}%
\pgfpathlineto{\pgfqpoint{4.233962in}{3.132048in}}%
\pgfpathlineto{\pgfqpoint{4.238244in}{3.126168in}}%
\pgfpathlineto{\pgfqpoint{4.244934in}{3.117339in}}%
\pgfpathlineto{\pgfqpoint{4.247878in}{3.116551in}}%
\pgfpathlineto{\pgfqpoint{4.250821in}{3.118275in}}%
\pgfpathlineto{\pgfqpoint{4.255103in}{3.124096in}}%
\pgfpathlineto{\pgfqpoint{4.262061in}{3.133194in}}%
\pgfpathlineto{\pgfqpoint{4.265004in}{3.133835in}}%
\pgfpathlineto{\pgfqpoint{4.267948in}{3.131974in}}%
\pgfpathlineto{\pgfqpoint{4.272497in}{3.125619in}}%
\pgfpathlineto{\pgfqpoint{4.278920in}{3.117289in}}%
\pgfpathlineto{\pgfqpoint{4.281864in}{3.116564in}}%
\pgfpathlineto{\pgfqpoint{4.284807in}{3.118348in}}%
\pgfpathlineto{\pgfqpoint{4.289089in}{3.124214in}}%
\pgfpathlineto{\pgfqpoint{4.295779in}{3.133064in}}%
\pgfpathlineto{\pgfqpoint{4.298723in}{3.133872in}}%
\pgfpathlineto{\pgfqpoint{4.301667in}{3.132166in}}%
\pgfpathlineto{\pgfqpoint{4.305948in}{3.126360in}}%
\pgfpathlineto{\pgfqpoint{4.312906in}{3.117240in}}%
\pgfpathlineto{\pgfqpoint{4.315850in}{3.116579in}}%
\pgfpathlineto{\pgfqpoint{4.318794in}{3.118422in}}%
\pgfpathlineto{\pgfqpoint{4.323075in}{3.124333in}}%
\pgfpathlineto{\pgfqpoint{4.329766in}{3.133115in}}%
\pgfpathlineto{\pgfqpoint{4.332709in}{3.133859in}}%
\pgfpathlineto{\pgfqpoint{4.335653in}{3.132094in}}%
\pgfpathlineto{\pgfqpoint{4.339935in}{3.126241in}}%
\pgfpathlineto{\pgfqpoint{4.346625in}{3.117371in}}%
\pgfpathlineto{\pgfqpoint{4.349569in}{3.116543in}}%
\pgfpathlineto{\pgfqpoint{4.352512in}{3.118230in}}%
\pgfpathlineto{\pgfqpoint{4.356794in}{3.124022in}}%
\pgfpathlineto{\pgfqpoint{4.363752in}{3.133164in}}%
\pgfpathlineto{\pgfqpoint{4.366695in}{3.133845in}}%
\pgfpathlineto{\pgfqpoint{4.369639in}{3.132020in}}%
\pgfpathlineto{\pgfqpoint{4.373921in}{3.126123in}}%
\pgfpathlineto{\pgfqpoint{4.380611in}{3.117320in}}%
\pgfpathlineto{\pgfqpoint{4.383555in}{3.116556in}}%
\pgfpathlineto{\pgfqpoint{4.386498in}{3.118302in}}%
\pgfpathlineto{\pgfqpoint{4.390780in}{3.124140in}}%
\pgfpathlineto{\pgfqpoint{4.397738in}{3.133211in}}%
\pgfpathlineto{\pgfqpoint{4.400682in}{3.133829in}}%
\pgfpathlineto{\pgfqpoint{4.403625in}{3.131946in}}%
\pgfpathlineto{\pgfqpoint{4.408175in}{3.125574in}}%
\pgfpathlineto{\pgfqpoint{4.414597in}{3.117270in}}%
\pgfpathlineto{\pgfqpoint{4.417541in}{3.116569in}}%
\pgfpathlineto{\pgfqpoint{4.420485in}{3.118375in}}%
\pgfpathlineto{\pgfqpoint{4.424766in}{3.124259in}}%
\pgfpathlineto{\pgfqpoint{4.431457in}{3.133083in}}%
\pgfpathlineto{\pgfqpoint{4.434400in}{3.133867in}}%
\pgfpathlineto{\pgfqpoint{4.437344in}{3.132139in}}%
\pgfpathlineto{\pgfqpoint{4.441626in}{3.126315in}}%
\pgfpathlineto{\pgfqpoint{4.448583in}{3.117222in}}%
\pgfpathlineto{\pgfqpoint{4.451527in}{3.116585in}}%
\pgfpathlineto{\pgfqpoint{4.454471in}{3.118450in}}%
\pgfpathlineto{\pgfqpoint{4.459020in}{3.124808in}}%
\pgfpathlineto{\pgfqpoint{4.465443in}{3.133133in}}%
\pgfpathlineto{\pgfqpoint{4.468386in}{3.133854in}}%
\pgfpathlineto{\pgfqpoint{4.471330in}{3.132066in}}%
\pgfpathlineto{\pgfqpoint{4.475612in}{3.126197in}}%
\pgfpathlineto{\pgfqpoint{4.482302in}{3.117352in}}%
\pgfpathlineto{\pgfqpoint{4.485246in}{3.116548in}}%
\pgfpathlineto{\pgfqpoint{4.488189in}{3.118257in}}%
\pgfpathlineto{\pgfqpoint{4.492471in}{3.124067in}}%
\pgfpathlineto{\pgfqpoint{4.499429in}{3.133182in}}%
\pgfpathlineto{\pgfqpoint{4.502373in}{3.133839in}}%
\pgfpathlineto{\pgfqpoint{4.505316in}{3.131992in}}%
\pgfpathlineto{\pgfqpoint{4.509598in}{3.126079in}}%
\pgfpathlineto{\pgfqpoint{4.516288in}{3.117301in}}%
\pgfpathlineto{\pgfqpoint{4.519232in}{3.116561in}}%
\pgfpathlineto{\pgfqpoint{4.522176in}{3.118330in}}%
\pgfpathlineto{\pgfqpoint{4.526457in}{3.124185in}}%
\pgfpathlineto{\pgfqpoint{4.533148in}{3.133051in}}%
\pgfpathlineto{\pgfqpoint{4.536091in}{3.133875in}}%
\pgfpathlineto{\pgfqpoint{4.539035in}{3.132184in}}%
\pgfpathlineto{\pgfqpoint{4.543317in}{3.126389in}}%
\pgfpathlineto{\pgfqpoint{4.550274in}{3.117252in}}%
\pgfpathlineto{\pgfqpoint{4.553218in}{3.116575in}}%
\pgfpathlineto{\pgfqpoint{4.556162in}{3.118403in}}%
\pgfpathlineto{\pgfqpoint{4.560443in}{3.124303in}}%
\pgfpathlineto{\pgfqpoint{4.567134in}{3.133102in}}%
\pgfpathlineto{\pgfqpoint{4.570077in}{3.133863in}}%
\pgfpathlineto{\pgfqpoint{4.573021in}{3.132112in}}%
\pgfpathlineto{\pgfqpoint{4.577303in}{3.126271in}}%
\pgfpathlineto{\pgfqpoint{4.583993in}{3.117384in}}%
\pgfpathlineto{\pgfqpoint{4.586937in}{3.116541in}}%
\pgfpathlineto{\pgfqpoint{4.589880in}{3.118213in}}%
\pgfpathlineto{\pgfqpoint{4.594162in}{3.123993in}}%
\pgfpathlineto{\pgfqpoint{4.601120in}{3.133152in}}%
\pgfpathlineto{\pgfqpoint{4.604064in}{3.133849in}}%
\pgfpathlineto{\pgfqpoint{4.607007in}{3.132039in}}%
\pgfpathlineto{\pgfqpoint{4.611289in}{3.126153in}}%
\pgfpathlineto{\pgfqpoint{4.617979in}{3.117332in}}%
\pgfpathlineto{\pgfqpoint{4.620923in}{3.116552in}}%
\pgfpathlineto{\pgfqpoint{4.623867in}{3.118284in}}%
\pgfpathlineto{\pgfqpoint{4.628148in}{3.124111in}}%
\pgfpathlineto{\pgfqpoint{4.635106in}{3.133200in}}%
\pgfpathlineto{\pgfqpoint{4.638050in}{3.133833in}}%
\pgfpathlineto{\pgfqpoint{4.640993in}{3.131964in}}%
\pgfpathlineto{\pgfqpoint{4.645543in}{3.125604in}}%
\pgfpathlineto{\pgfqpoint{4.651965in}{3.117282in}}%
\pgfpathlineto{\pgfqpoint{4.654909in}{3.116566in}}%
\pgfpathlineto{\pgfqpoint{4.657853in}{3.118357in}}%
\pgfpathlineto{\pgfqpoint{4.662134in}{3.124229in}}%
\pgfpathlineto{\pgfqpoint{4.668825in}{3.133071in}}%
\pgfpathlineto{\pgfqpoint{4.671768in}{3.133870in}}%
\pgfpathlineto{\pgfqpoint{4.674712in}{3.132157in}}%
\pgfpathlineto{\pgfqpoint{4.678994in}{3.126345in}}%
\pgfpathlineto{\pgfqpoint{4.685952in}{3.117234in}}%
\pgfpathlineto{\pgfqpoint{4.688895in}{3.116581in}}%
\pgfpathlineto{\pgfqpoint{4.691839in}{3.118431in}}%
\pgfpathlineto{\pgfqpoint{4.696388in}{3.124778in}}%
\pgfpathlineto{\pgfqpoint{4.702811in}{3.133121in}}%
\pgfpathlineto{\pgfqpoint{4.705755in}{3.133857in}}%
\pgfpathlineto{\pgfqpoint{4.708698in}{3.132085in}}%
\pgfpathlineto{\pgfqpoint{4.712980in}{3.126227in}}%
\pgfpathlineto{\pgfqpoint{4.719670in}{3.117364in}}%
\pgfpathlineto{\pgfqpoint{4.722614in}{3.116545in}}%
\pgfpathlineto{\pgfqpoint{4.725558in}{3.118239in}}%
\pgfpathlineto{\pgfqpoint{4.729839in}{3.124037in}}%
\pgfpathlineto{\pgfqpoint{4.736797in}{3.133170in}}%
\pgfpathlineto{\pgfqpoint{4.739741in}{3.133843in}}%
\pgfpathlineto{\pgfqpoint{4.742684in}{3.132011in}}%
\pgfpathlineto{\pgfqpoint{4.746966in}{3.126108in}}%
\pgfpathlineto{\pgfqpoint{4.753656in}{3.117314in}}%
\pgfpathlineto{\pgfqpoint{4.756600in}{3.116557in}}%
\pgfpathlineto{\pgfqpoint{4.759544in}{3.118311in}}%
\pgfpathlineto{\pgfqpoint{4.763825in}{3.124155in}}%
\pgfpathlineto{\pgfqpoint{4.770516in}{3.133038in}}%
\pgfpathlineto{\pgfqpoint{4.773459in}{3.133878in}}%
\pgfpathlineto{\pgfqpoint{4.776403in}{3.132202in}}%
\pgfpathlineto{\pgfqpoint{4.780685in}{3.126419in}}%
\pgfpathlineto{\pgfqpoint{4.787643in}{3.117264in}}%
\pgfpathlineto{\pgfqpoint{4.790586in}{3.116571in}}%
\pgfpathlineto{\pgfqpoint{4.793530in}{3.118385in}}%
\pgfpathlineto{\pgfqpoint{4.797812in}{3.124274in}}%
\pgfpathlineto{\pgfqpoint{4.804502in}{3.133090in}}%
\pgfpathlineto{\pgfqpoint{4.807446in}{3.133866in}}%
\pgfpathlineto{\pgfqpoint{4.810389in}{3.132130in}}%
\pgfpathlineto{\pgfqpoint{4.814671in}{3.126301in}}%
\pgfpathlineto{\pgfqpoint{4.821629in}{3.117216in}}%
\pgfpathlineto{\pgfqpoint{4.824572in}{3.116587in}}%
\pgfpathlineto{\pgfqpoint{4.827516in}{3.118459in}}%
\pgfpathlineto{\pgfqpoint{4.832065in}{3.124823in}}%
\pgfpathlineto{\pgfqpoint{4.838488in}{3.133139in}}%
\pgfpathlineto{\pgfqpoint{4.841432in}{3.133852in}}%
\pgfpathlineto{\pgfqpoint{4.844375in}{3.132057in}}%
\pgfpathlineto{\pgfqpoint{4.848657in}{3.126182in}}%
\pgfpathlineto{\pgfqpoint{4.855347in}{3.117345in}}%
\pgfpathlineto{\pgfqpoint{4.858291in}{3.116549in}}%
\pgfpathlineto{\pgfqpoint{4.861235in}{3.118266in}}%
\pgfpathlineto{\pgfqpoint{4.865516in}{3.124081in}}%
\pgfpathlineto{\pgfqpoint{4.872474in}{3.133188in}}%
\pgfpathlineto{\pgfqpoint{4.875418in}{3.133837in}}%
\pgfpathlineto{\pgfqpoint{4.878362in}{3.131983in}}%
\pgfpathlineto{\pgfqpoint{4.882911in}{3.125633in}}%
\pgfpathlineto{\pgfqpoint{4.889334in}{3.117295in}}%
\pgfpathlineto{\pgfqpoint{4.892277in}{3.116562in}}%
\pgfpathlineto{\pgfqpoint{4.895221in}{3.118339in}}%
\pgfpathlineto{\pgfqpoint{4.899503in}{3.124200in}}%
\pgfpathlineto{\pgfqpoint{4.906193in}{3.133058in}}%
\pgfpathlineto{\pgfqpoint{4.909136in}{3.133873in}}%
\pgfpathlineto{\pgfqpoint{4.912080in}{3.132175in}}%
\pgfpathlineto{\pgfqpoint{4.916362in}{3.126374in}}%
\pgfpathlineto{\pgfqpoint{4.923320in}{3.117246in}}%
\pgfpathlineto{\pgfqpoint{4.926263in}{3.116577in}}%
\pgfpathlineto{\pgfqpoint{4.929207in}{3.118412in}}%
\pgfpathlineto{\pgfqpoint{4.933489in}{3.124318in}}%
\pgfpathlineto{\pgfqpoint{4.940179in}{3.133108in}}%
\pgfpathlineto{\pgfqpoint{4.943123in}{3.133861in}}%
\pgfpathlineto{\pgfqpoint{4.946066in}{3.132103in}}%
\pgfpathlineto{\pgfqpoint{4.950348in}{3.126256in}}%
\pgfpathlineto{\pgfqpoint{4.957038in}{3.117377in}}%
\pgfpathlineto{\pgfqpoint{4.959982in}{3.116542in}}%
\pgfpathlineto{\pgfqpoint{4.962926in}{3.118222in}}%
\pgfpathlineto{\pgfqpoint{4.967207in}{3.124008in}}%
\pgfpathlineto{\pgfqpoint{4.974165in}{3.133158in}}%
\pgfpathlineto{\pgfqpoint{4.977109in}{3.133847in}}%
\pgfpathlineto{\pgfqpoint{4.980053in}{3.132030in}}%
\pgfpathlineto{\pgfqpoint{4.984334in}{3.126138in}}%
\pgfpathlineto{\pgfqpoint{4.991024in}{3.117326in}}%
\pgfpathlineto{\pgfqpoint{4.993968in}{3.116554in}}%
\pgfpathlineto{\pgfqpoint{4.996912in}{3.118293in}}%
\pgfpathlineto{\pgfqpoint{5.001194in}{3.124126in}}%
\pgfpathlineto{\pgfqpoint{5.008151in}{3.133206in}}%
\pgfpathlineto{\pgfqpoint{5.011095in}{3.133831in}}%
\pgfpathlineto{\pgfqpoint{5.014039in}{3.131955in}}%
\pgfpathlineto{\pgfqpoint{5.018588in}{3.125589in}}%
\pgfpathlineto{\pgfqpoint{5.025011in}{3.117276in}}%
\pgfpathlineto{\pgfqpoint{5.027954in}{3.116567in}}%
\pgfpathlineto{\pgfqpoint{5.030898in}{3.118366in}}%
\pgfpathlineto{\pgfqpoint{5.035180in}{3.124244in}}%
\pgfpathlineto{\pgfqpoint{5.041870in}{3.133077in}}%
\pgfpathlineto{\pgfqpoint{5.044814in}{3.133869in}}%
\pgfpathlineto{\pgfqpoint{5.047757in}{3.132148in}}%
\pgfpathlineto{\pgfqpoint{5.052039in}{3.126330in}}%
\pgfpathlineto{\pgfqpoint{5.058997in}{3.117228in}}%
\pgfpathlineto{\pgfqpoint{5.061941in}{3.116583in}}%
\pgfpathlineto{\pgfqpoint{5.064884in}{3.118440in}}%
\pgfpathlineto{\pgfqpoint{5.069434in}{3.124793in}}%
\pgfpathlineto{\pgfqpoint{5.075856in}{3.133127in}}%
\pgfpathlineto{\pgfqpoint{5.078800in}{3.133856in}}%
\pgfpathlineto{\pgfqpoint{5.081744in}{3.132076in}}%
\pgfpathlineto{\pgfqpoint{5.086025in}{3.126212in}}%
\pgfpathlineto{\pgfqpoint{5.092715in}{3.117358in}}%
\pgfpathlineto{\pgfqpoint{5.095659in}{3.116546in}}%
\pgfpathlineto{\pgfqpoint{5.098603in}{3.118248in}}%
\pgfpathlineto{\pgfqpoint{5.102885in}{3.124052in}}%
\pgfpathlineto{\pgfqpoint{5.109842in}{3.133176in}}%
\pgfpathlineto{\pgfqpoint{5.112786in}{3.133841in}}%
\pgfpathlineto{\pgfqpoint{5.115730in}{3.132002in}}%
\pgfpathlineto{\pgfqpoint{5.120011in}{3.126094in}}%
\pgfpathlineto{\pgfqpoint{5.126702in}{3.117307in}}%
\pgfpathlineto{\pgfqpoint{5.129645in}{3.116559in}}%
\pgfpathlineto{\pgfqpoint{5.132589in}{3.118320in}}%
\pgfpathlineto{\pgfqpoint{5.136871in}{3.124170in}}%
\pgfpathlineto{\pgfqpoint{5.143561in}{3.133045in}}%
\pgfpathlineto{\pgfqpoint{5.146505in}{3.133876in}}%
\pgfpathlineto{\pgfqpoint{5.149448in}{3.132193in}}%
\pgfpathlineto{\pgfqpoint{5.153730in}{3.126404in}}%
\pgfpathlineto{\pgfqpoint{5.160688in}{3.117258in}}%
\pgfpathlineto{\pgfqpoint{5.163632in}{3.116573in}}%
\pgfpathlineto{\pgfqpoint{5.166575in}{3.118394in}}%
\pgfpathlineto{\pgfqpoint{5.170857in}{3.124288in}}%
\pgfpathlineto{\pgfqpoint{5.177547in}{3.133096in}}%
\pgfpathlineto{\pgfqpoint{5.180491in}{3.133864in}}%
\pgfpathlineto{\pgfqpoint{5.183435in}{3.132121in}}%
\pgfpathlineto{\pgfqpoint{5.187716in}{3.126286in}}%
\pgfpathlineto{\pgfqpoint{5.194674in}{3.117210in}}%
\pgfpathlineto{\pgfqpoint{5.197618in}{3.116589in}}%
\pgfpathlineto{\pgfqpoint{5.200561in}{3.118469in}}%
\pgfpathlineto{\pgfqpoint{5.205111in}{3.124837in}}%
\pgfpathlineto{\pgfqpoint{5.211533in}{3.133146in}}%
\pgfpathlineto{\pgfqpoint{5.214477in}{3.133850in}}%
\pgfpathlineto{\pgfqpoint{5.217421in}{3.132048in}}%
\pgfpathlineto{\pgfqpoint{5.221702in}{3.126168in}}%
\pgfpathlineto{\pgfqpoint{5.228393in}{3.117339in}}%
\pgfpathlineto{\pgfqpoint{5.231336in}{3.116551in}}%
\pgfpathlineto{\pgfqpoint{5.234280in}{3.118275in}}%
\pgfpathlineto{\pgfqpoint{5.238562in}{3.124096in}}%
\pgfpathlineto{\pgfqpoint{5.245520in}{3.133194in}}%
\pgfpathlineto{\pgfqpoint{5.248463in}{3.133835in}}%
\pgfpathlineto{\pgfqpoint{5.251407in}{3.131974in}}%
\pgfpathlineto{\pgfqpoint{5.255956in}{3.125619in}}%
\pgfpathlineto{\pgfqpoint{5.262379in}{3.117289in}}%
\pgfpathlineto{\pgfqpoint{5.265322in}{3.116564in}}%
\pgfpathlineto{\pgfqpoint{5.268266in}{3.118348in}}%
\pgfpathlineto{\pgfqpoint{5.272548in}{3.124214in}}%
\pgfpathlineto{\pgfqpoint{5.279238in}{3.133064in}}%
\pgfpathlineto{\pgfqpoint{5.282182in}{3.133872in}}%
\pgfpathlineto{\pgfqpoint{5.285125in}{3.132166in}}%
\pgfpathlineto{\pgfqpoint{5.289407in}{3.126360in}}%
\pgfpathlineto{\pgfqpoint{5.296365in}{3.117240in}}%
\pgfpathlineto{\pgfqpoint{5.299309in}{3.116579in}}%
\pgfpathlineto{\pgfqpoint{5.302252in}{3.118422in}}%
\pgfpathlineto{\pgfqpoint{5.306534in}{3.124333in}}%
\pgfpathlineto{\pgfqpoint{5.313224in}{3.133115in}}%
\pgfpathlineto{\pgfqpoint{5.316168in}{3.133859in}}%
\pgfpathlineto{\pgfqpoint{5.319112in}{3.132094in}}%
\pgfpathlineto{\pgfqpoint{5.323393in}{3.126241in}}%
\pgfpathlineto{\pgfqpoint{5.330084in}{3.117371in}}%
\pgfpathlineto{\pgfqpoint{5.333027in}{3.116543in}}%
\pgfpathlineto{\pgfqpoint{5.335971in}{3.118230in}}%
\pgfpathlineto{\pgfqpoint{5.340253in}{3.124022in}}%
\pgfpathlineto{\pgfqpoint{5.347210in}{3.133164in}}%
\pgfpathlineto{\pgfqpoint{5.350154in}{3.133845in}}%
\pgfpathlineto{\pgfqpoint{5.353098in}{3.132020in}}%
\pgfpathlineto{\pgfqpoint{5.357380in}{3.126123in}}%
\pgfpathlineto{\pgfqpoint{5.364070in}{3.117320in}}%
\pgfpathlineto{\pgfqpoint{5.367013in}{3.116556in}}%
\pgfpathlineto{\pgfqpoint{5.369957in}{3.118302in}}%
\pgfpathlineto{\pgfqpoint{5.374239in}{3.124140in}}%
\pgfpathlineto{\pgfqpoint{5.381197in}{3.133211in}}%
\pgfpathlineto{\pgfqpoint{5.384140in}{3.133829in}}%
\pgfpathlineto{\pgfqpoint{5.387084in}{3.131946in}}%
\pgfpathlineto{\pgfqpoint{5.391633in}{3.125574in}}%
\pgfpathlineto{\pgfqpoint{5.398056in}{3.117270in}}%
\pgfpathlineto{\pgfqpoint{5.401000in}{3.116569in}}%
\pgfpathlineto{\pgfqpoint{5.403943in}{3.118375in}}%
\pgfpathlineto{\pgfqpoint{5.408225in}{3.124259in}}%
\pgfpathlineto{\pgfqpoint{5.414915in}{3.133083in}}%
\pgfpathlineto{\pgfqpoint{5.417859in}{3.133867in}}%
\pgfpathlineto{\pgfqpoint{5.420803in}{3.132139in}}%
\pgfpathlineto{\pgfqpoint{5.425084in}{3.126315in}}%
\pgfpathlineto{\pgfqpoint{5.432042in}{3.117222in}}%
\pgfpathlineto{\pgfqpoint{5.434986in}{3.116585in}}%
\pgfpathlineto{\pgfqpoint{5.437930in}{3.118450in}}%
\pgfpathlineto{\pgfqpoint{5.442479in}{3.124808in}}%
\pgfpathlineto{\pgfqpoint{5.448901in}{3.133133in}}%
\pgfpathlineto{\pgfqpoint{5.451845in}{3.133854in}}%
\pgfpathlineto{\pgfqpoint{5.454789in}{3.132066in}}%
\pgfpathlineto{\pgfqpoint{5.459071in}{3.126197in}}%
\pgfpathlineto{\pgfqpoint{5.465761in}{3.117352in}}%
\pgfpathlineto{\pgfqpoint{5.468704in}{3.116548in}}%
\pgfpathlineto{\pgfqpoint{5.471648in}{3.118257in}}%
\pgfpathlineto{\pgfqpoint{5.475930in}{3.124067in}}%
\pgfpathlineto{\pgfqpoint{5.482888in}{3.133182in}}%
\pgfpathlineto{\pgfqpoint{5.485831in}{3.133839in}}%
\pgfpathlineto{\pgfqpoint{5.488775in}{3.131992in}}%
\pgfpathlineto{\pgfqpoint{5.493057in}{3.126079in}}%
\pgfpathlineto{\pgfqpoint{5.499747in}{3.117301in}}%
\pgfpathlineto{\pgfqpoint{5.502691in}{3.116561in}}%
\pgfpathlineto{\pgfqpoint{5.505634in}{3.118330in}}%
\pgfpathlineto{\pgfqpoint{5.509916in}{3.124185in}}%
\pgfpathlineto{\pgfqpoint{5.516606in}{3.133051in}}%
\pgfpathlineto{\pgfqpoint{5.519550in}{3.133875in}}%
\pgfpathlineto{\pgfqpoint{5.522494in}{3.132184in}}%
\pgfpathlineto{\pgfqpoint{5.526775in}{3.126389in}}%
\pgfpathlineto{\pgfqpoint{5.533733in}{3.117252in}}%
\pgfpathlineto{\pgfqpoint{5.536677in}{3.116575in}}%
\pgfpathlineto{\pgfqpoint{5.539621in}{3.118403in}}%
\pgfpathlineto{\pgfqpoint{5.543902in}{3.124303in}}%
\pgfpathlineto{\pgfqpoint{5.550592in}{3.133102in}}%
\pgfpathlineto{\pgfqpoint{5.553536in}{3.133863in}}%
\pgfpathlineto{\pgfqpoint{5.556480in}{3.132112in}}%
\pgfpathlineto{\pgfqpoint{5.560762in}{3.126271in}}%
\pgfpathlineto{\pgfqpoint{5.567452in}{3.117384in}}%
\pgfpathlineto{\pgfqpoint{5.570395in}{3.116541in}}%
\pgfpathlineto{\pgfqpoint{5.573339in}{3.118213in}}%
\pgfpathlineto{\pgfqpoint{5.577621in}{3.123993in}}%
\pgfpathlineto{\pgfqpoint{5.584579in}{3.133152in}}%
\pgfpathlineto{\pgfqpoint{5.587522in}{3.133849in}}%
\pgfpathlineto{\pgfqpoint{5.590466in}{3.132039in}}%
\pgfpathlineto{\pgfqpoint{5.594748in}{3.126153in}}%
\pgfpathlineto{\pgfqpoint{5.601438in}{3.117332in}}%
\pgfpathlineto{\pgfqpoint{5.604382in}{3.116552in}}%
\pgfpathlineto{\pgfqpoint{5.607325in}{3.118284in}}%
\pgfpathlineto{\pgfqpoint{5.611607in}{3.124111in}}%
\pgfpathlineto{\pgfqpoint{5.618565in}{3.133200in}}%
\pgfpathlineto{\pgfqpoint{5.621509in}{3.133833in}}%
\pgfpathlineto{\pgfqpoint{5.624452in}{3.131964in}}%
\pgfpathlineto{\pgfqpoint{5.629002in}{3.125604in}}%
\pgfpathlineto{\pgfqpoint{5.635424in}{3.117282in}}%
\pgfpathlineto{\pgfqpoint{5.638368in}{3.116566in}}%
\pgfpathlineto{\pgfqpoint{5.641311in}{3.118357in}}%
\pgfpathlineto{\pgfqpoint{5.645593in}{3.124229in}}%
\pgfpathlineto{\pgfqpoint{5.652283in}{3.133071in}}%
\pgfpathlineto{\pgfqpoint{5.655227in}{3.133870in}}%
\pgfpathlineto{\pgfqpoint{5.658171in}{3.132157in}}%
\pgfpathlineto{\pgfqpoint{5.662453in}{3.126345in}}%
\pgfpathlineto{\pgfqpoint{5.669410in}{3.117234in}}%
\pgfpathlineto{\pgfqpoint{5.672354in}{3.116581in}}%
\pgfpathlineto{\pgfqpoint{5.675298in}{3.118431in}}%
\pgfpathlineto{\pgfqpoint{5.679847in}{3.124778in}}%
\pgfpathlineto{\pgfqpoint{5.686270in}{3.133121in}}%
\pgfpathlineto{\pgfqpoint{5.689213in}{3.133857in}}%
\pgfpathlineto{\pgfqpoint{5.692157in}{3.132085in}}%
\pgfpathlineto{\pgfqpoint{5.696439in}{3.126227in}}%
\pgfpathlineto{\pgfqpoint{5.703129in}{3.117364in}}%
\pgfpathlineto{\pgfqpoint{5.706073in}{3.116545in}}%
\pgfpathlineto{\pgfqpoint{5.709016in}{3.118239in}}%
\pgfpathlineto{\pgfqpoint{5.713298in}{3.124037in}}%
\pgfpathlineto{\pgfqpoint{5.720256in}{3.133170in}}%
\pgfpathlineto{\pgfqpoint{5.723199in}{3.133843in}}%
\pgfpathlineto{\pgfqpoint{5.726143in}{3.132011in}}%
\pgfpathlineto{\pgfqpoint{5.730425in}{3.126108in}}%
\pgfpathlineto{\pgfqpoint{5.737115in}{3.117314in}}%
\pgfpathlineto{\pgfqpoint{5.740059in}{3.116557in}}%
\pgfpathlineto{\pgfqpoint{5.743002in}{3.118311in}}%
\pgfpathlineto{\pgfqpoint{5.747284in}{3.124155in}}%
\pgfpathlineto{\pgfqpoint{5.753974in}{3.133038in}}%
\pgfpathlineto{\pgfqpoint{5.756918in}{3.133878in}}%
\pgfpathlineto{\pgfqpoint{5.759862in}{3.132202in}}%
\pgfpathlineto{\pgfqpoint{5.764143in}{3.126419in}}%
\pgfpathlineto{\pgfqpoint{5.771101in}{3.117264in}}%
\pgfpathlineto{\pgfqpoint{5.774045in}{3.116571in}}%
\pgfpathlineto{\pgfqpoint{5.776989in}{3.118385in}}%
\pgfpathlineto{\pgfqpoint{5.781270in}{3.124274in}}%
\pgfpathlineto{\pgfqpoint{5.787961in}{3.133090in}}%
\pgfpathlineto{\pgfqpoint{5.790904in}{3.133866in}}%
\pgfpathlineto{\pgfqpoint{5.793848in}{3.132130in}}%
\pgfpathlineto{\pgfqpoint{5.798130in}{3.126301in}}%
\pgfpathlineto{\pgfqpoint{5.805087in}{3.117216in}}%
\pgfpathlineto{\pgfqpoint{5.808031in}{3.116587in}}%
\pgfpathlineto{\pgfqpoint{5.810975in}{3.118459in}}%
\pgfpathlineto{\pgfqpoint{5.815524in}{3.124823in}}%
\pgfpathlineto{\pgfqpoint{5.821947in}{3.133139in}}%
\pgfpathlineto{\pgfqpoint{5.824890in}{3.133852in}}%
\pgfpathlineto{\pgfqpoint{5.827834in}{3.132057in}}%
\pgfpathlineto{\pgfqpoint{5.832116in}{3.126182in}}%
\pgfpathlineto{\pgfqpoint{5.838806in}{3.117345in}}%
\pgfpathlineto{\pgfqpoint{5.841750in}{3.116549in}}%
\pgfpathlineto{\pgfqpoint{5.844693in}{3.118266in}}%
\pgfpathlineto{\pgfqpoint{5.848975in}{3.124081in}}%
\pgfpathlineto{\pgfqpoint{5.855933in}{3.133188in}}%
\pgfpathlineto{\pgfqpoint{5.858877in}{3.133837in}}%
\pgfpathlineto{\pgfqpoint{5.861820in}{3.131983in}}%
\pgfpathlineto{\pgfqpoint{5.866370in}{3.125633in}}%
\pgfpathlineto{\pgfqpoint{5.872792in}{3.117295in}}%
\pgfpathlineto{\pgfqpoint{5.875736in}{3.116562in}}%
\pgfpathlineto{\pgfqpoint{5.878680in}{3.118339in}}%
\pgfpathlineto{\pgfqpoint{5.882961in}{3.124200in}}%
\pgfpathlineto{\pgfqpoint{5.889652in}{3.133058in}}%
\pgfpathlineto{\pgfqpoint{5.892595in}{3.133873in}}%
\pgfpathlineto{\pgfqpoint{5.895539in}{3.132175in}}%
\pgfpathlineto{\pgfqpoint{5.899821in}{3.126374in}}%
\pgfpathlineto{\pgfqpoint{5.906778in}{3.117246in}}%
\pgfpathlineto{\pgfqpoint{5.909722in}{3.116577in}}%
\pgfpathlineto{\pgfqpoint{5.912666in}{3.118412in}}%
\pgfpathlineto{\pgfqpoint{5.916948in}{3.124318in}}%
\pgfpathlineto{\pgfqpoint{5.923638in}{3.133108in}}%
\pgfpathlineto{\pgfqpoint{5.926581in}{3.133861in}}%
\pgfpathlineto{\pgfqpoint{5.929525in}{3.132103in}}%
\pgfpathlineto{\pgfqpoint{5.933807in}{3.126256in}}%
\pgfpathlineto{\pgfqpoint{5.940497in}{3.117377in}}%
\pgfpathlineto{\pgfqpoint{5.943441in}{3.116542in}}%
\pgfpathlineto{\pgfqpoint{5.946384in}{3.118222in}}%
\pgfpathlineto{\pgfqpoint{5.950666in}{3.124008in}}%
\pgfpathlineto{\pgfqpoint{5.957624in}{3.133158in}}%
\pgfpathlineto{\pgfqpoint{5.960568in}{3.133847in}}%
\pgfpathlineto{\pgfqpoint{5.963511in}{3.132030in}}%
\pgfpathlineto{\pgfqpoint{5.967793in}{3.126138in}}%
\pgfpathlineto{\pgfqpoint{5.974483in}{3.117326in}}%
\pgfpathlineto{\pgfqpoint{5.977427in}{3.116554in}}%
\pgfpathlineto{\pgfqpoint{5.980371in}{3.118293in}}%
\pgfpathlineto{\pgfqpoint{5.984652in}{3.124126in}}%
\pgfpathlineto{\pgfqpoint{5.991610in}{3.133206in}}%
\pgfpathlineto{\pgfqpoint{5.994554in}{3.133831in}}%
\pgfpathlineto{\pgfqpoint{5.997497in}{3.131955in}}%
\pgfpathlineto{\pgfqpoint{6.002047in}{3.125589in}}%
\pgfpathlineto{\pgfqpoint{6.008469in}{3.117276in}}%
\pgfpathlineto{\pgfqpoint{6.011413in}{3.116567in}}%
\pgfpathlineto{\pgfqpoint{6.014357in}{3.118366in}}%
\pgfpathlineto{\pgfqpoint{6.018639in}{3.124244in}}%
\pgfpathlineto{\pgfqpoint{6.025329in}{3.133077in}}%
\pgfpathlineto{\pgfqpoint{6.028272in}{3.133869in}}%
\pgfpathlineto{\pgfqpoint{6.031216in}{3.132148in}}%
\pgfpathlineto{\pgfqpoint{6.035498in}{3.126330in}}%
\pgfpathlineto{\pgfqpoint{6.042456in}{3.117228in}}%
\pgfpathlineto{\pgfqpoint{6.045399in}{3.116583in}}%
\pgfpathlineto{\pgfqpoint{6.048343in}{3.118440in}}%
\pgfpathlineto{\pgfqpoint{6.052892in}{3.124793in}}%
\pgfpathlineto{\pgfqpoint{6.059315in}{3.133127in}}%
\pgfpathlineto{\pgfqpoint{6.062259in}{3.133856in}}%
\pgfpathlineto{\pgfqpoint{6.065202in}{3.132076in}}%
\pgfpathlineto{\pgfqpoint{6.069484in}{3.126212in}}%
\pgfpathlineto{\pgfqpoint{6.076174in}{3.117358in}}%
\pgfpathlineto{\pgfqpoint{6.079118in}{3.116546in}}%
\pgfpathlineto{\pgfqpoint{6.082062in}{3.118248in}}%
\pgfpathlineto{\pgfqpoint{6.086343in}{3.124052in}}%
\pgfpathlineto{\pgfqpoint{6.093301in}{3.133176in}}%
\pgfpathlineto{\pgfqpoint{6.096245in}{3.133841in}}%
\pgfpathlineto{\pgfqpoint{6.099188in}{3.132002in}}%
\pgfpathlineto{\pgfqpoint{6.103470in}{3.126094in}}%
\pgfpathlineto{\pgfqpoint{6.110160in}{3.117307in}}%
\pgfpathlineto{\pgfqpoint{6.113104in}{3.116559in}}%
\pgfpathlineto{\pgfqpoint{6.116048in}{3.118320in}}%
\pgfpathlineto{\pgfqpoint{6.120329in}{3.124170in}}%
\pgfpathlineto{\pgfqpoint{6.127020in}{3.133045in}}%
\pgfpathlineto{\pgfqpoint{6.129963in}{3.133876in}}%
\pgfpathlineto{\pgfqpoint{6.132907in}{3.132193in}}%
\pgfpathlineto{\pgfqpoint{6.137189in}{3.126404in}}%
\pgfpathlineto{\pgfqpoint{6.144147in}{3.117258in}}%
\pgfpathlineto{\pgfqpoint{6.147090in}{3.116573in}}%
\pgfpathlineto{\pgfqpoint{6.150034in}{3.118394in}}%
\pgfpathlineto{\pgfqpoint{6.154316in}{3.124288in}}%
\pgfpathlineto{\pgfqpoint{6.161006in}{3.133096in}}%
\pgfpathlineto{\pgfqpoint{6.163950in}{3.133864in}}%
\pgfpathlineto{\pgfqpoint{6.166893in}{3.132121in}}%
\pgfpathlineto{\pgfqpoint{6.171175in}{3.126286in}}%
\pgfpathlineto{\pgfqpoint{6.178133in}{3.117210in}}%
\pgfpathlineto{\pgfqpoint{6.181076in}{3.116589in}}%
\pgfpathlineto{\pgfqpoint{6.184020in}{3.118469in}}%
\pgfpathlineto{\pgfqpoint{6.188569in}{3.124837in}}%
\pgfpathlineto{\pgfqpoint{6.194992in}{3.133146in}}%
\pgfpathlineto{\pgfqpoint{6.197936in}{3.133850in}}%
\pgfpathlineto{\pgfqpoint{6.200879in}{3.132048in}}%
\pgfpathlineto{\pgfqpoint{6.205161in}{3.126168in}}%
\pgfpathlineto{\pgfqpoint{6.211851in}{3.117339in}}%
\pgfpathlineto{\pgfqpoint{6.214795in}{3.116551in}}%
\pgfpathlineto{\pgfqpoint{6.217739in}{3.118275in}}%
\pgfpathlineto{\pgfqpoint{6.222020in}{3.124096in}}%
\pgfpathlineto{\pgfqpoint{6.228978in}{3.133194in}}%
\pgfpathlineto{\pgfqpoint{6.231922in}{3.133835in}}%
\pgfpathlineto{\pgfqpoint{6.234866in}{3.131974in}}%
\pgfpathlineto{\pgfqpoint{6.239415in}{3.125619in}}%
\pgfpathlineto{\pgfqpoint{6.245838in}{3.117289in}}%
\pgfpathlineto{\pgfqpoint{6.248781in}{3.116564in}}%
\pgfpathlineto{\pgfqpoint{6.251725in}{3.118348in}}%
\pgfpathlineto{\pgfqpoint{6.256007in}{3.124214in}}%
\pgfpathlineto{\pgfqpoint{6.262697in}{3.133064in}}%
\pgfpathlineto{\pgfqpoint{6.265641in}{3.133872in}}%
\pgfpathlineto{\pgfqpoint{6.268584in}{3.132166in}}%
\pgfpathlineto{\pgfqpoint{6.272866in}{3.126360in}}%
\pgfpathlineto{\pgfqpoint{6.279824in}{3.117240in}}%
\pgfpathlineto{\pgfqpoint{6.282767in}{3.116579in}}%
\pgfpathlineto{\pgfqpoint{6.285711in}{3.118422in}}%
\pgfpathlineto{\pgfqpoint{6.289993in}{3.124333in}}%
\pgfpathlineto{\pgfqpoint{6.296683in}{3.133115in}}%
\pgfpathlineto{\pgfqpoint{6.299627in}{3.133859in}}%
\pgfpathlineto{\pgfqpoint{6.302570in}{3.132094in}}%
\pgfpathlineto{\pgfqpoint{6.306852in}{3.126241in}}%
\pgfpathlineto{\pgfqpoint{6.313542in}{3.117371in}}%
\pgfpathlineto{\pgfqpoint{6.316486in}{3.116543in}}%
\pgfpathlineto{\pgfqpoint{6.319430in}{3.118230in}}%
\pgfpathlineto{\pgfqpoint{6.323711in}{3.124022in}}%
\pgfpathlineto{\pgfqpoint{6.330669in}{3.133164in}}%
\pgfpathlineto{\pgfqpoint{6.333613in}{3.133845in}}%
\pgfpathlineto{\pgfqpoint{6.336557in}{3.132020in}}%
\pgfpathlineto{\pgfqpoint{6.340838in}{3.126123in}}%
\pgfpathlineto{\pgfqpoint{6.347529in}{3.117320in}}%
\pgfpathlineto{\pgfqpoint{6.350472in}{3.116556in}}%
\pgfpathlineto{\pgfqpoint{6.353416in}{3.118302in}}%
\pgfpathlineto{\pgfqpoint{6.357698in}{3.124140in}}%
\pgfpathlineto{\pgfqpoint{6.364655in}{3.133211in}}%
\pgfpathlineto{\pgfqpoint{6.367599in}{3.133829in}}%
\pgfpathlineto{\pgfqpoint{6.370543in}{3.131946in}}%
\pgfpathlineto{\pgfqpoint{6.375092in}{3.125574in}}%
\pgfpathlineto{\pgfqpoint{6.381515in}{3.117270in}}%
\pgfpathlineto{\pgfqpoint{6.384458in}{3.116569in}}%
\pgfpathlineto{\pgfqpoint{6.387402in}{3.118375in}}%
\pgfpathlineto{\pgfqpoint{6.391684in}{3.124259in}}%
\pgfpathlineto{\pgfqpoint{6.398374in}{3.133083in}}%
\pgfpathlineto{\pgfqpoint{6.401318in}{3.133867in}}%
\pgfpathlineto{\pgfqpoint{6.404261in}{3.132139in}}%
\pgfpathlineto{\pgfqpoint{6.408543in}{3.126315in}}%
\pgfpathlineto{\pgfqpoint{6.415501in}{3.117222in}}%
\pgfpathlineto{\pgfqpoint{6.418445in}{3.116585in}}%
\pgfpathlineto{\pgfqpoint{6.421388in}{3.118450in}}%
\pgfpathlineto{\pgfqpoint{6.425938in}{3.124808in}}%
\pgfpathlineto{\pgfqpoint{6.432360in}{3.133133in}}%
\pgfpathlineto{\pgfqpoint{6.435304in}{3.133854in}}%
\pgfpathlineto{\pgfqpoint{6.438248in}{3.132066in}}%
\pgfpathlineto{\pgfqpoint{6.442529in}{3.126197in}}%
\pgfpathlineto{\pgfqpoint{6.449220in}{3.117352in}}%
\pgfpathlineto{\pgfqpoint{6.452163in}{3.116548in}}%
\pgfpathlineto{\pgfqpoint{6.455107in}{3.118257in}}%
\pgfpathlineto{\pgfqpoint{6.459389in}{3.124067in}}%
\pgfpathlineto{\pgfqpoint{6.466346in}{3.133182in}}%
\pgfpathlineto{\pgfqpoint{6.469290in}{3.133839in}}%
\pgfpathlineto{\pgfqpoint{6.472234in}{3.131992in}}%
\pgfpathlineto{\pgfqpoint{6.476515in}{3.126079in}}%
\pgfpathlineto{\pgfqpoint{6.483206in}{3.117301in}}%
\pgfpathlineto{\pgfqpoint{6.486149in}{3.116561in}}%
\pgfpathlineto{\pgfqpoint{6.489093in}{3.118330in}}%
\pgfpathlineto{\pgfqpoint{6.493375in}{3.124185in}}%
\pgfpathlineto{\pgfqpoint{6.500065in}{3.133051in}}%
\pgfpathlineto{\pgfqpoint{6.503009in}{3.133875in}}%
\pgfpathlineto{\pgfqpoint{6.505952in}{3.132184in}}%
\pgfpathlineto{\pgfqpoint{6.510234in}{3.126389in}}%
\pgfpathlineto{\pgfqpoint{6.517192in}{3.117252in}}%
\pgfpathlineto{\pgfqpoint{6.520136in}{3.116575in}}%
\pgfpathlineto{\pgfqpoint{6.523079in}{3.118403in}}%
\pgfpathlineto{\pgfqpoint{6.527361in}{3.124303in}}%
\pgfpathlineto{\pgfqpoint{6.534051in}{3.133102in}}%
\pgfpathlineto{\pgfqpoint{6.536995in}{3.133863in}}%
\pgfpathlineto{\pgfqpoint{6.539939in}{3.132112in}}%
\pgfpathlineto{\pgfqpoint{6.544220in}{3.126271in}}%
\pgfpathlineto{\pgfqpoint{6.550910in}{3.117384in}}%
\pgfpathlineto{\pgfqpoint{6.553854in}{3.116541in}}%
\pgfpathlineto{\pgfqpoint{6.556798in}{3.118213in}}%
\pgfpathlineto{\pgfqpoint{6.561080in}{3.123993in}}%
\pgfpathlineto{\pgfqpoint{6.568037in}{3.133152in}}%
\pgfpathlineto{\pgfqpoint{6.570981in}{3.133849in}}%
\pgfpathlineto{\pgfqpoint{6.573925in}{3.132039in}}%
\pgfpathlineto{\pgfqpoint{6.578206in}{3.126153in}}%
\pgfpathlineto{\pgfqpoint{6.584897in}{3.117332in}}%
\pgfpathlineto{\pgfqpoint{6.587840in}{3.116552in}}%
\pgfpathlineto{\pgfqpoint{6.590784in}{3.118284in}}%
\pgfpathlineto{\pgfqpoint{6.595066in}{3.124111in}}%
\pgfpathlineto{\pgfqpoint{6.602024in}{3.133200in}}%
\pgfpathlineto{\pgfqpoint{6.604967in}{3.133833in}}%
\pgfpathlineto{\pgfqpoint{6.607911in}{3.131964in}}%
\pgfpathlineto{\pgfqpoint{6.612460in}{3.125604in}}%
\pgfpathlineto{\pgfqpoint{6.618883in}{3.117282in}}%
\pgfpathlineto{\pgfqpoint{6.621827in}{3.116566in}}%
\pgfpathlineto{\pgfqpoint{6.624770in}{3.118357in}}%
\pgfpathlineto{\pgfqpoint{6.629052in}{3.124229in}}%
\pgfpathlineto{\pgfqpoint{6.635742in}{3.133071in}}%
\pgfpathlineto{\pgfqpoint{6.638686in}{3.133870in}}%
\pgfpathlineto{\pgfqpoint{6.641630in}{3.132157in}}%
\pgfpathlineto{\pgfqpoint{6.645911in}{3.126345in}}%
\pgfpathlineto{\pgfqpoint{6.652869in}{3.117234in}}%
\pgfpathlineto{\pgfqpoint{6.655813in}{3.116581in}}%
\pgfpathlineto{\pgfqpoint{6.658756in}{3.118431in}}%
\pgfpathlineto{\pgfqpoint{6.663306in}{3.124778in}}%
\pgfpathlineto{\pgfqpoint{6.663306in}{3.124778in}}%
\pgfusepath{stroke}%
\end{pgfscope}%
\begin{pgfscope}%
\pgfpathrectangle{\pgfqpoint{0.467797in}{2.292089in}}{\pgfqpoint{6.490533in}{1.666241in}}%
\pgfusepath{clip}%
\pgfsetrectcap%
\pgfsetroundjoin%
\pgfsetlinewidth{1.505625pt}%
\definecolor{currentstroke}{rgb}{0.580392,0.403922,0.741176}%
\pgfsetstrokecolor{currentstroke}%
\pgfsetdash{}{0pt}%
\pgfpathmoveto{\pgfqpoint{0.762821in}{3.125209in}}%
\pgfpathlineto{\pgfqpoint{0.768976in}{3.133034in}}%
\pgfpathlineto{\pgfqpoint{0.771920in}{3.133619in}}%
\pgfpathlineto{\pgfqpoint{0.774863in}{3.131654in}}%
\pgfpathlineto{\pgfqpoint{0.779413in}{3.125183in}}%
\pgfpathlineto{\pgfqpoint{0.785300in}{3.117553in}}%
\pgfpathlineto{\pgfqpoint{0.788244in}{3.116747in}}%
\pgfpathlineto{\pgfqpoint{0.790920in}{3.118251in}}%
\pgfpathlineto{\pgfqpoint{0.794934in}{3.123547in}}%
\pgfpathlineto{\pgfqpoint{0.802159in}{3.133054in}}%
\pgfpathlineto{\pgfqpoint{0.805103in}{3.133611in}}%
\pgfpathlineto{\pgfqpoint{0.808047in}{3.131619in}}%
\pgfpathlineto{\pgfqpoint{0.812596in}{3.125129in}}%
\pgfpathlineto{\pgfqpoint{0.818483in}{3.117530in}}%
\pgfpathlineto{\pgfqpoint{0.821427in}{3.116753in}}%
\pgfpathlineto{\pgfqpoint{0.824371in}{3.118541in}}%
\pgfpathlineto{\pgfqpoint{0.828652in}{3.124454in}}%
\pgfpathlineto{\pgfqpoint{0.835075in}{3.132900in}}%
\pgfpathlineto{\pgfqpoint{0.838019in}{3.133663in}}%
\pgfpathlineto{\pgfqpoint{0.840962in}{3.131861in}}%
\pgfpathlineto{\pgfqpoint{0.845244in}{3.125938in}}%
\pgfpathlineto{\pgfqpoint{0.851667in}{3.117507in}}%
\pgfpathlineto{\pgfqpoint{0.854610in}{3.116759in}}%
\pgfpathlineto{\pgfqpoint{0.857554in}{3.118574in}}%
\pgfpathlineto{\pgfqpoint{0.861836in}{3.124507in}}%
\pgfpathlineto{\pgfqpoint{0.868258in}{3.132923in}}%
\pgfpathlineto{\pgfqpoint{0.871202in}{3.133656in}}%
\pgfpathlineto{\pgfqpoint{0.874146in}{3.131828in}}%
\pgfpathlineto{\pgfqpoint{0.878427in}{3.125885in}}%
\pgfpathlineto{\pgfqpoint{0.884850in}{3.117485in}}%
\pgfpathlineto{\pgfqpoint{0.887794in}{3.116766in}}%
\pgfpathlineto{\pgfqpoint{0.890737in}{3.118608in}}%
\pgfpathlineto{\pgfqpoint{0.895019in}{3.124560in}}%
\pgfpathlineto{\pgfqpoint{0.901442in}{3.132945in}}%
\pgfpathlineto{\pgfqpoint{0.904385in}{3.133650in}}%
\pgfpathlineto{\pgfqpoint{0.907329in}{3.131794in}}%
\pgfpathlineto{\pgfqpoint{0.911878in}{3.125401in}}%
\pgfpathlineto{\pgfqpoint{0.918033in}{3.117463in}}%
\pgfpathlineto{\pgfqpoint{0.920977in}{3.116773in}}%
\pgfpathlineto{\pgfqpoint{0.923921in}{3.118642in}}%
\pgfpathlineto{\pgfqpoint{0.928470in}{3.125045in}}%
\pgfpathlineto{\pgfqpoint{0.934625in}{3.132967in}}%
\pgfpathlineto{\pgfqpoint{0.937569in}{3.133643in}}%
\pgfpathlineto{\pgfqpoint{0.940513in}{3.131760in}}%
\pgfpathlineto{\pgfqpoint{0.945062in}{3.125348in}}%
\pgfpathlineto{\pgfqpoint{0.951217in}{3.117441in}}%
\pgfpathlineto{\pgfqpoint{0.954161in}{3.116780in}}%
\pgfpathlineto{\pgfqpoint{0.957104in}{3.118676in}}%
\pgfpathlineto{\pgfqpoint{0.961654in}{3.125098in}}%
\pgfpathlineto{\pgfqpoint{0.967809in}{3.132989in}}%
\pgfpathlineto{\pgfqpoint{0.970752in}{3.133635in}}%
\pgfpathlineto{\pgfqpoint{0.973696in}{3.131726in}}%
\pgfpathlineto{\pgfqpoint{0.978245in}{3.125294in}}%
\pgfpathlineto{\pgfqpoint{0.984400in}{3.117419in}}%
\pgfpathlineto{\pgfqpoint{0.987344in}{3.116787in}}%
\pgfpathlineto{\pgfqpoint{0.990288in}{3.118710in}}%
\pgfpathlineto{\pgfqpoint{0.994837in}{3.125151in}}%
\pgfpathlineto{\pgfqpoint{1.000992in}{3.133011in}}%
\pgfpathlineto{\pgfqpoint{1.003936in}{3.133628in}}%
\pgfpathlineto{\pgfqpoint{1.006879in}{3.131692in}}%
\pgfpathlineto{\pgfqpoint{1.011429in}{3.125241in}}%
\pgfpathlineto{\pgfqpoint{1.017584in}{3.117398in}}%
\pgfpathlineto{\pgfqpoint{1.020527in}{3.116795in}}%
\pgfpathlineto{\pgfqpoint{1.023471in}{3.118745in}}%
\pgfpathlineto{\pgfqpoint{1.028020in}{3.125205in}}%
\pgfpathlineto{\pgfqpoint{1.034175in}{3.133032in}}%
\pgfpathlineto{\pgfqpoint{1.037119in}{3.133620in}}%
\pgfpathlineto{\pgfqpoint{1.040063in}{3.131657in}}%
\pgfpathlineto{\pgfqpoint{1.044612in}{3.125188in}}%
\pgfpathlineto{\pgfqpoint{1.050499in}{3.117555in}}%
\pgfpathlineto{\pgfqpoint{1.053443in}{3.116747in}}%
\pgfpathlineto{\pgfqpoint{1.056119in}{3.118248in}}%
\pgfpathlineto{\pgfqpoint{1.060133in}{3.123542in}}%
\pgfpathlineto{\pgfqpoint{1.067359in}{3.133053in}}%
\pgfpathlineto{\pgfqpoint{1.070302in}{3.133611in}}%
\pgfpathlineto{\pgfqpoint{1.073246in}{3.131622in}}%
\pgfpathlineto{\pgfqpoint{1.077795in}{3.125134in}}%
\pgfpathlineto{\pgfqpoint{1.083683in}{3.117532in}}%
\pgfpathlineto{\pgfqpoint{1.086626in}{3.116753in}}%
\pgfpathlineto{\pgfqpoint{1.089570in}{3.118538in}}%
\pgfpathlineto{\pgfqpoint{1.093852in}{3.124449in}}%
\pgfpathlineto{\pgfqpoint{1.100274in}{3.132898in}}%
\pgfpathlineto{\pgfqpoint{1.103218in}{3.133663in}}%
\pgfpathlineto{\pgfqpoint{1.106162in}{3.131864in}}%
\pgfpathlineto{\pgfqpoint{1.110443in}{3.125943in}}%
\pgfpathlineto{\pgfqpoint{1.116866in}{3.117509in}}%
\pgfpathlineto{\pgfqpoint{1.119810in}{3.116759in}}%
\pgfpathlineto{\pgfqpoint{1.122753in}{3.118571in}}%
\pgfpathlineto{\pgfqpoint{1.127035in}{3.124502in}}%
\pgfpathlineto{\pgfqpoint{1.133458in}{3.132921in}}%
\pgfpathlineto{\pgfqpoint{1.136401in}{3.133657in}}%
\pgfpathlineto{\pgfqpoint{1.139345in}{3.131831in}}%
\pgfpathlineto{\pgfqpoint{1.143627in}{3.125890in}}%
\pgfpathlineto{\pgfqpoint{1.150049in}{3.117487in}}%
\pgfpathlineto{\pgfqpoint{1.152993in}{3.116765in}}%
\pgfpathlineto{\pgfqpoint{1.155937in}{3.118605in}}%
\pgfpathlineto{\pgfqpoint{1.160219in}{3.124555in}}%
\pgfpathlineto{\pgfqpoint{1.166641in}{3.132943in}}%
\pgfpathlineto{\pgfqpoint{1.169585in}{3.133650in}}%
\pgfpathlineto{\pgfqpoint{1.172528in}{3.131797in}}%
\pgfpathlineto{\pgfqpoint{1.177078in}{3.125406in}}%
\pgfpathlineto{\pgfqpoint{1.183233in}{3.117465in}}%
\pgfpathlineto{\pgfqpoint{1.186176in}{3.116772in}}%
\pgfpathlineto{\pgfqpoint{1.189120in}{3.118639in}}%
\pgfpathlineto{\pgfqpoint{1.193670in}{3.125040in}}%
\pgfpathlineto{\pgfqpoint{1.199824in}{3.132965in}}%
\pgfpathlineto{\pgfqpoint{1.202768in}{3.133643in}}%
\pgfpathlineto{\pgfqpoint{1.205712in}{3.131763in}}%
\pgfpathlineto{\pgfqpoint{1.210261in}{3.125352in}}%
\pgfpathlineto{\pgfqpoint{1.216416in}{3.117443in}}%
\pgfpathlineto{\pgfqpoint{1.219360in}{3.116779in}}%
\pgfpathlineto{\pgfqpoint{1.222304in}{3.118673in}}%
\pgfpathlineto{\pgfqpoint{1.226853in}{3.125093in}}%
\pgfpathlineto{\pgfqpoint{1.233008in}{3.132987in}}%
\pgfpathlineto{\pgfqpoint{1.235952in}{3.133636in}}%
\pgfpathlineto{\pgfqpoint{1.238895in}{3.131729in}}%
\pgfpathlineto{\pgfqpoint{1.243445in}{3.125299in}}%
\pgfpathlineto{\pgfqpoint{1.249600in}{3.117421in}}%
\pgfpathlineto{\pgfqpoint{1.252543in}{3.116787in}}%
\pgfpathlineto{\pgfqpoint{1.255487in}{3.118707in}}%
\pgfpathlineto{\pgfqpoint{1.260036in}{3.125146in}}%
\pgfpathlineto{\pgfqpoint{1.266191in}{3.133009in}}%
\pgfpathlineto{\pgfqpoint{1.269135in}{3.133628in}}%
\pgfpathlineto{\pgfqpoint{1.272079in}{3.131695in}}%
\pgfpathlineto{\pgfqpoint{1.276628in}{3.125246in}}%
\pgfpathlineto{\pgfqpoint{1.282783in}{3.117400in}}%
\pgfpathlineto{\pgfqpoint{1.285727in}{3.116794in}}%
\pgfpathlineto{\pgfqpoint{1.288670in}{3.118741in}}%
\pgfpathlineto{\pgfqpoint{1.293220in}{3.125200in}}%
\pgfpathlineto{\pgfqpoint{1.299375in}{3.133030in}}%
\pgfpathlineto{\pgfqpoint{1.302318in}{3.133620in}}%
\pgfpathlineto{\pgfqpoint{1.305262in}{3.131660in}}%
\pgfpathlineto{\pgfqpoint{1.309811in}{3.125192in}}%
\pgfpathlineto{\pgfqpoint{1.315699in}{3.117558in}}%
\pgfpathlineto{\pgfqpoint{1.318642in}{3.116746in}}%
\pgfpathlineto{\pgfqpoint{1.321318in}{3.118245in}}%
\pgfpathlineto{\pgfqpoint{1.325333in}{3.123537in}}%
\pgfpathlineto{\pgfqpoint{1.332558in}{3.133051in}}%
\pgfpathlineto{\pgfqpoint{1.335502in}{3.133612in}}%
\pgfpathlineto{\pgfqpoint{1.338445in}{3.131625in}}%
\pgfpathlineto{\pgfqpoint{1.342995in}{3.125139in}}%
\pgfpathlineto{\pgfqpoint{1.348882in}{3.117534in}}%
\pgfpathlineto{\pgfqpoint{1.351826in}{3.116752in}}%
\pgfpathlineto{\pgfqpoint{1.354769in}{3.118535in}}%
\pgfpathlineto{\pgfqpoint{1.359051in}{3.124444in}}%
\pgfpathlineto{\pgfqpoint{1.365474in}{3.132896in}}%
\pgfpathlineto{\pgfqpoint{1.368417in}{3.133664in}}%
\pgfpathlineto{\pgfqpoint{1.371361in}{3.131867in}}%
\pgfpathlineto{\pgfqpoint{1.375643in}{3.125948in}}%
\pgfpathlineto{\pgfqpoint{1.382065in}{3.117511in}}%
\pgfpathlineto{\pgfqpoint{1.385009in}{3.116758in}}%
\pgfpathlineto{\pgfqpoint{1.387953in}{3.118568in}}%
\pgfpathlineto{\pgfqpoint{1.392235in}{3.124497in}}%
\pgfpathlineto{\pgfqpoint{1.398657in}{3.132919in}}%
\pgfpathlineto{\pgfqpoint{1.401601in}{3.133657in}}%
\pgfpathlineto{\pgfqpoint{1.404544in}{3.131834in}}%
\pgfpathlineto{\pgfqpoint{1.408826in}{3.125895in}}%
\pgfpathlineto{\pgfqpoint{1.415249in}{3.117489in}}%
\pgfpathlineto{\pgfqpoint{1.418192in}{3.116765in}}%
\pgfpathlineto{\pgfqpoint{1.421136in}{3.118602in}}%
\pgfpathlineto{\pgfqpoint{1.425418in}{3.124551in}}%
\pgfpathlineto{\pgfqpoint{1.431840in}{3.132941in}}%
\pgfpathlineto{\pgfqpoint{1.434784in}{3.133651in}}%
\pgfpathlineto{\pgfqpoint{1.437728in}{3.131800in}}%
\pgfpathlineto{\pgfqpoint{1.442277in}{3.125411in}}%
\pgfpathlineto{\pgfqpoint{1.448432in}{3.117467in}}%
\pgfpathlineto{\pgfqpoint{1.451376in}{3.116771in}}%
\pgfpathlineto{\pgfqpoint{1.454320in}{3.118635in}}%
\pgfpathlineto{\pgfqpoint{1.458869in}{3.125035in}}%
\pgfpathlineto{\pgfqpoint{1.465024in}{3.132963in}}%
\pgfpathlineto{\pgfqpoint{1.467968in}{3.133644in}}%
\pgfpathlineto{\pgfqpoint{1.470911in}{3.131766in}}%
\pgfpathlineto{\pgfqpoint{1.475461in}{3.125357in}}%
\pgfpathlineto{\pgfqpoint{1.481616in}{3.117445in}}%
\pgfpathlineto{\pgfqpoint{1.484559in}{3.116779in}}%
\pgfpathlineto{\pgfqpoint{1.487503in}{3.118669in}}%
\pgfpathlineto{\pgfqpoint{1.492052in}{3.125088in}}%
\pgfpathlineto{\pgfqpoint{1.498207in}{3.132985in}}%
\pgfpathlineto{\pgfqpoint{1.501151in}{3.133637in}}%
\pgfpathlineto{\pgfqpoint{1.504095in}{3.131732in}}%
\pgfpathlineto{\pgfqpoint{1.508644in}{3.125304in}}%
\pgfpathlineto{\pgfqpoint{1.514799in}{3.117423in}}%
\pgfpathlineto{\pgfqpoint{1.517743in}{3.116786in}}%
\pgfpathlineto{\pgfqpoint{1.520686in}{3.118704in}}%
\pgfpathlineto{\pgfqpoint{1.525236in}{3.125142in}}%
\pgfpathlineto{\pgfqpoint{1.531391in}{3.133007in}}%
\pgfpathlineto{\pgfqpoint{1.534334in}{3.133629in}}%
\pgfpathlineto{\pgfqpoint{1.537278in}{3.131698in}}%
\pgfpathlineto{\pgfqpoint{1.541827in}{3.125251in}}%
\pgfpathlineto{\pgfqpoint{1.547982in}{3.117402in}}%
\pgfpathlineto{\pgfqpoint{1.550926in}{3.116794in}}%
\pgfpathlineto{\pgfqpoint{1.553870in}{3.118738in}}%
\pgfpathlineto{\pgfqpoint{1.558419in}{3.125195in}}%
\pgfpathlineto{\pgfqpoint{1.564574in}{3.133028in}}%
\pgfpathlineto{\pgfqpoint{1.567518in}{3.133621in}}%
\pgfpathlineto{\pgfqpoint{1.570461in}{3.131663in}}%
\pgfpathlineto{\pgfqpoint{1.575011in}{3.125197in}}%
\pgfpathlineto{\pgfqpoint{1.581166in}{3.117381in}}%
\pgfpathlineto{\pgfqpoint{1.584109in}{3.116802in}}%
\pgfpathlineto{\pgfqpoint{1.587053in}{3.118773in}}%
\pgfpathlineto{\pgfqpoint{1.591602in}{3.125248in}}%
\pgfpathlineto{\pgfqpoint{1.597490in}{3.132871in}}%
\pgfpathlineto{\pgfqpoint{1.600433in}{3.133670in}}%
\pgfpathlineto{\pgfqpoint{1.603109in}{3.132161in}}%
\pgfpathlineto{\pgfqpoint{1.607124in}{3.126860in}}%
\pgfpathlineto{\pgfqpoint{1.614349in}{3.117360in}}%
\pgfpathlineto{\pgfqpoint{1.617293in}{3.116810in}}%
\pgfpathlineto{\pgfqpoint{1.620236in}{3.118808in}}%
\pgfpathlineto{\pgfqpoint{1.624786in}{3.125302in}}%
\pgfpathlineto{\pgfqpoint{1.630673in}{3.132894in}}%
\pgfpathlineto{\pgfqpoint{1.633617in}{3.133664in}}%
\pgfpathlineto{\pgfqpoint{1.636560in}{3.131870in}}%
\pgfpathlineto{\pgfqpoint{1.640842in}{3.125953in}}%
\pgfpathlineto{\pgfqpoint{1.647265in}{3.117514in}}%
\pgfpathlineto{\pgfqpoint{1.650208in}{3.116758in}}%
\pgfpathlineto{\pgfqpoint{1.653152in}{3.118565in}}%
\pgfpathlineto{\pgfqpoint{1.657434in}{3.124493in}}%
\pgfpathlineto{\pgfqpoint{1.663856in}{3.132917in}}%
\pgfpathlineto{\pgfqpoint{1.666800in}{3.133658in}}%
\pgfpathlineto{\pgfqpoint{1.669744in}{3.131837in}}%
\pgfpathlineto{\pgfqpoint{1.674026in}{3.125900in}}%
\pgfpathlineto{\pgfqpoint{1.680448in}{3.117491in}}%
\pgfpathlineto{\pgfqpoint{1.683392in}{3.116764in}}%
\pgfpathlineto{\pgfqpoint{1.686335in}{3.118599in}}%
\pgfpathlineto{\pgfqpoint{1.690617in}{3.124546in}}%
\pgfpathlineto{\pgfqpoint{1.697040in}{3.132939in}}%
\pgfpathlineto{\pgfqpoint{1.699983in}{3.133651in}}%
\pgfpathlineto{\pgfqpoint{1.702927in}{3.131803in}}%
\pgfpathlineto{\pgfqpoint{1.707477in}{3.125416in}}%
\pgfpathlineto{\pgfqpoint{1.713631in}{3.117469in}}%
\pgfpathlineto{\pgfqpoint{1.716575in}{3.116771in}}%
\pgfpathlineto{\pgfqpoint{1.719519in}{3.118632in}}%
\pgfpathlineto{\pgfqpoint{1.724068in}{3.125030in}}%
\pgfpathlineto{\pgfqpoint{1.730223in}{3.132961in}}%
\pgfpathlineto{\pgfqpoint{1.733167in}{3.133645in}}%
\pgfpathlineto{\pgfqpoint{1.736111in}{3.131770in}}%
\pgfpathlineto{\pgfqpoint{1.740660in}{3.125362in}}%
\pgfpathlineto{\pgfqpoint{1.746815in}{3.117447in}}%
\pgfpathlineto{\pgfqpoint{1.749759in}{3.116778in}}%
\pgfpathlineto{\pgfqpoint{1.752702in}{3.118666in}}%
\pgfpathlineto{\pgfqpoint{1.757252in}{3.125083in}}%
\pgfpathlineto{\pgfqpoint{1.763407in}{3.132983in}}%
\pgfpathlineto{\pgfqpoint{1.766350in}{3.133637in}}%
\pgfpathlineto{\pgfqpoint{1.769294in}{3.131735in}}%
\pgfpathlineto{\pgfqpoint{1.773843in}{3.125309in}}%
\pgfpathlineto{\pgfqpoint{1.779998in}{3.117425in}}%
\pgfpathlineto{\pgfqpoint{1.782942in}{3.116785in}}%
\pgfpathlineto{\pgfqpoint{1.785886in}{3.118701in}}%
\pgfpathlineto{\pgfqpoint{1.790435in}{3.125137in}}%
\pgfpathlineto{\pgfqpoint{1.796590in}{3.133005in}}%
\pgfpathlineto{\pgfqpoint{1.799534in}{3.133630in}}%
\pgfpathlineto{\pgfqpoint{1.802477in}{3.131701in}}%
\pgfpathlineto{\pgfqpoint{1.807027in}{3.125256in}}%
\pgfpathlineto{\pgfqpoint{1.813182in}{3.117404in}}%
\pgfpathlineto{\pgfqpoint{1.816125in}{3.116793in}}%
\pgfpathlineto{\pgfqpoint{1.819069in}{3.118735in}}%
\pgfpathlineto{\pgfqpoint{1.823618in}{3.125190in}}%
\pgfpathlineto{\pgfqpoint{1.829773in}{3.133026in}}%
\pgfpathlineto{\pgfqpoint{1.832717in}{3.133622in}}%
\pgfpathlineto{\pgfqpoint{1.835661in}{3.131666in}}%
\pgfpathlineto{\pgfqpoint{1.840210in}{3.125202in}}%
\pgfpathlineto{\pgfqpoint{1.846365in}{3.117382in}}%
\pgfpathlineto{\pgfqpoint{1.849309in}{3.116801in}}%
\pgfpathlineto{\pgfqpoint{1.852252in}{3.118770in}}%
\pgfpathlineto{\pgfqpoint{1.856802in}{3.125243in}}%
\pgfpathlineto{\pgfqpoint{1.862689in}{3.132869in}}%
\pgfpathlineto{\pgfqpoint{1.865633in}{3.133671in}}%
\pgfpathlineto{\pgfqpoint{1.868309in}{3.132164in}}%
\pgfpathlineto{\pgfqpoint{1.872323in}{3.126865in}}%
\pgfpathlineto{\pgfqpoint{1.879548in}{3.117362in}}%
\pgfpathlineto{\pgfqpoint{1.882492in}{3.116809in}}%
\pgfpathlineto{\pgfqpoint{1.885436in}{3.118805in}}%
\pgfpathlineto{\pgfqpoint{1.889985in}{3.125297in}}%
\pgfpathlineto{\pgfqpoint{1.895872in}{3.132892in}}%
\pgfpathlineto{\pgfqpoint{1.898816in}{3.133665in}}%
\pgfpathlineto{\pgfqpoint{1.901760in}{3.131873in}}%
\pgfpathlineto{\pgfqpoint{1.906042in}{3.125958in}}%
\pgfpathlineto{\pgfqpoint{1.912464in}{3.117516in}}%
\pgfpathlineto{\pgfqpoint{1.915408in}{3.116757in}}%
\pgfpathlineto{\pgfqpoint{1.918351in}{3.118562in}}%
\pgfpathlineto{\pgfqpoint{1.922633in}{3.124488in}}%
\pgfpathlineto{\pgfqpoint{1.929056in}{3.132915in}}%
\pgfpathlineto{\pgfqpoint{1.931999in}{3.133659in}}%
\pgfpathlineto{\pgfqpoint{1.934943in}{3.131840in}}%
\pgfpathlineto{\pgfqpoint{1.939225in}{3.125905in}}%
\pgfpathlineto{\pgfqpoint{1.945647in}{3.117493in}}%
\pgfpathlineto{\pgfqpoint{1.948591in}{3.116763in}}%
\pgfpathlineto{\pgfqpoint{1.951535in}{3.118596in}}%
\pgfpathlineto{\pgfqpoint{1.955817in}{3.124541in}}%
\pgfpathlineto{\pgfqpoint{1.962239in}{3.132937in}}%
\pgfpathlineto{\pgfqpoint{1.965183in}{3.133652in}}%
\pgfpathlineto{\pgfqpoint{1.968127in}{3.131806in}}%
\pgfpathlineto{\pgfqpoint{1.972676in}{3.125420in}}%
\pgfpathlineto{\pgfqpoint{1.978831in}{3.117471in}}%
\pgfpathlineto{\pgfqpoint{1.981775in}{3.116770in}}%
\pgfpathlineto{\pgfqpoint{1.984718in}{3.118629in}}%
\pgfpathlineto{\pgfqpoint{1.989268in}{3.125025in}}%
\pgfpathlineto{\pgfqpoint{1.995423in}{3.132959in}}%
\pgfpathlineto{\pgfqpoint{1.998366in}{3.133645in}}%
\pgfpathlineto{\pgfqpoint{2.001310in}{3.131773in}}%
\pgfpathlineto{\pgfqpoint{2.005859in}{3.125367in}}%
\pgfpathlineto{\pgfqpoint{2.012014in}{3.117449in}}%
\pgfpathlineto{\pgfqpoint{2.014958in}{3.116777in}}%
\pgfpathlineto{\pgfqpoint{2.017902in}{3.118663in}}%
\pgfpathlineto{\pgfqpoint{2.022451in}{3.125078in}}%
\pgfpathlineto{\pgfqpoint{2.028606in}{3.132981in}}%
\pgfpathlineto{\pgfqpoint{2.031550in}{3.133638in}}%
\pgfpathlineto{\pgfqpoint{2.034493in}{3.131739in}}%
\pgfpathlineto{\pgfqpoint{2.039043in}{3.125314in}}%
\pgfpathlineto{\pgfqpoint{2.045198in}{3.117427in}}%
\pgfpathlineto{\pgfqpoint{2.048141in}{3.116785in}}%
\pgfpathlineto{\pgfqpoint{2.051085in}{3.118697in}}%
\pgfpathlineto{\pgfqpoint{2.055634in}{3.125132in}}%
\pgfpathlineto{\pgfqpoint{2.061789in}{3.133003in}}%
\pgfpathlineto{\pgfqpoint{2.064733in}{3.133631in}}%
\pgfpathlineto{\pgfqpoint{2.067677in}{3.131704in}}%
\pgfpathlineto{\pgfqpoint{2.072226in}{3.125260in}}%
\pgfpathlineto{\pgfqpoint{2.078381in}{3.117405in}}%
\pgfpathlineto{\pgfqpoint{2.081325in}{3.116792in}}%
\pgfpathlineto{\pgfqpoint{2.084268in}{3.118732in}}%
\pgfpathlineto{\pgfqpoint{2.088818in}{3.125185in}}%
\pgfpathlineto{\pgfqpoint{2.094973in}{3.133024in}}%
\pgfpathlineto{\pgfqpoint{2.097916in}{3.133623in}}%
\pgfpathlineto{\pgfqpoint{2.100860in}{3.131670in}}%
\pgfpathlineto{\pgfqpoint{2.105409in}{3.125207in}}%
\pgfpathlineto{\pgfqpoint{2.111564in}{3.117384in}}%
\pgfpathlineto{\pgfqpoint{2.114508in}{3.116800in}}%
\pgfpathlineto{\pgfqpoint{2.117452in}{3.118767in}}%
\pgfpathlineto{\pgfqpoint{2.122001in}{3.125239in}}%
\pgfpathlineto{\pgfqpoint{2.127888in}{3.132867in}}%
\pgfpathlineto{\pgfqpoint{2.130832in}{3.133671in}}%
\pgfpathlineto{\pgfqpoint{2.133508in}{3.132167in}}%
\pgfpathlineto{\pgfqpoint{2.137522in}{3.126869in}}%
\pgfpathlineto{\pgfqpoint{2.144748in}{3.117364in}}%
\pgfpathlineto{\pgfqpoint{2.147691in}{3.116809in}}%
\pgfpathlineto{\pgfqpoint{2.150635in}{3.118802in}}%
\pgfpathlineto{\pgfqpoint{2.155184in}{3.125292in}}%
\pgfpathlineto{\pgfqpoint{2.161072in}{3.132890in}}%
\pgfpathlineto{\pgfqpoint{2.164015in}{3.133665in}}%
\pgfpathlineto{\pgfqpoint{2.166959in}{3.131876in}}%
\pgfpathlineto{\pgfqpoint{2.171241in}{3.125963in}}%
\pgfpathlineto{\pgfqpoint{2.177663in}{3.117518in}}%
\pgfpathlineto{\pgfqpoint{2.180607in}{3.116757in}}%
\pgfpathlineto{\pgfqpoint{2.183551in}{3.118559in}}%
\pgfpathlineto{\pgfqpoint{2.187833in}{3.124483in}}%
\pgfpathlineto{\pgfqpoint{2.194255in}{3.132913in}}%
\pgfpathlineto{\pgfqpoint{2.197199in}{3.133659in}}%
\pgfpathlineto{\pgfqpoint{2.200143in}{3.131843in}}%
\pgfpathlineto{\pgfqpoint{2.204424in}{3.125909in}}%
\pgfpathlineto{\pgfqpoint{2.210847in}{3.117495in}}%
\pgfpathlineto{\pgfqpoint{2.213791in}{3.116763in}}%
\pgfpathlineto{\pgfqpoint{2.216734in}{3.118593in}}%
\pgfpathlineto{\pgfqpoint{2.221016in}{3.124536in}}%
\pgfpathlineto{\pgfqpoint{2.227439in}{3.132935in}}%
\pgfpathlineto{\pgfqpoint{2.230382in}{3.133653in}}%
\pgfpathlineto{\pgfqpoint{2.233326in}{3.131810in}}%
\pgfpathlineto{\pgfqpoint{2.237875in}{3.125425in}}%
\pgfpathlineto{\pgfqpoint{2.244030in}{3.117473in}}%
\pgfpathlineto{\pgfqpoint{2.246974in}{3.116770in}}%
\pgfpathlineto{\pgfqpoint{2.249918in}{3.118626in}}%
\pgfpathlineto{\pgfqpoint{2.254467in}{3.125020in}}%
\pgfpathlineto{\pgfqpoint{2.260622in}{3.132957in}}%
\pgfpathlineto{\pgfqpoint{2.263566in}{3.133646in}}%
\pgfpathlineto{\pgfqpoint{2.266509in}{3.131776in}}%
\pgfpathlineto{\pgfqpoint{2.271059in}{3.125372in}}%
\pgfpathlineto{\pgfqpoint{2.277214in}{3.117451in}}%
\pgfpathlineto{\pgfqpoint{2.280157in}{3.116777in}}%
\pgfpathlineto{\pgfqpoint{2.283101in}{3.118660in}}%
\pgfpathlineto{\pgfqpoint{2.287650in}{3.125074in}}%
\pgfpathlineto{\pgfqpoint{2.293805in}{3.132979in}}%
\pgfpathlineto{\pgfqpoint{2.296749in}{3.133639in}}%
\pgfpathlineto{\pgfqpoint{2.299693in}{3.131742in}}%
\pgfpathlineto{\pgfqpoint{2.304242in}{3.125319in}}%
\pgfpathlineto{\pgfqpoint{2.310397in}{3.117429in}}%
\pgfpathlineto{\pgfqpoint{2.313341in}{3.116784in}}%
\pgfpathlineto{\pgfqpoint{2.316284in}{3.118694in}}%
\pgfpathlineto{\pgfqpoint{2.320834in}{3.125127in}}%
\pgfpathlineto{\pgfqpoint{2.326989in}{3.133001in}}%
\pgfpathlineto{\pgfqpoint{2.329932in}{3.133631in}}%
\pgfpathlineto{\pgfqpoint{2.332876in}{3.131707in}}%
\pgfpathlineto{\pgfqpoint{2.337425in}{3.125265in}}%
\pgfpathlineto{\pgfqpoint{2.343580in}{3.117407in}}%
\pgfpathlineto{\pgfqpoint{2.346524in}{3.116792in}}%
\pgfpathlineto{\pgfqpoint{2.349468in}{3.118729in}}%
\pgfpathlineto{\pgfqpoint{2.354017in}{3.125180in}}%
\pgfpathlineto{\pgfqpoint{2.360172in}{3.133022in}}%
\pgfpathlineto{\pgfqpoint{2.363116in}{3.133623in}}%
\pgfpathlineto{\pgfqpoint{2.366059in}{3.131673in}}%
\pgfpathlineto{\pgfqpoint{2.370609in}{3.125212in}}%
\pgfpathlineto{\pgfqpoint{2.376764in}{3.117386in}}%
\pgfpathlineto{\pgfqpoint{2.379707in}{3.116800in}}%
\pgfpathlineto{\pgfqpoint{2.382651in}{3.118764in}}%
\pgfpathlineto{\pgfqpoint{2.387200in}{3.125234in}}%
\pgfpathlineto{\pgfqpoint{2.393088in}{3.132864in}}%
\pgfpathlineto{\pgfqpoint{2.396031in}{3.133672in}}%
\pgfpathlineto{\pgfqpoint{2.398708in}{3.132170in}}%
\pgfpathlineto{\pgfqpoint{2.402722in}{3.126874in}}%
\pgfpathlineto{\pgfqpoint{2.409947in}{3.117365in}}%
\pgfpathlineto{\pgfqpoint{2.412891in}{3.116808in}}%
\pgfpathlineto{\pgfqpoint{2.415834in}{3.118798in}}%
\pgfpathlineto{\pgfqpoint{2.420384in}{3.125287in}}%
\pgfpathlineto{\pgfqpoint{2.426271in}{3.132888in}}%
\pgfpathlineto{\pgfqpoint{2.429215in}{3.133666in}}%
\pgfpathlineto{\pgfqpoint{2.432158in}{3.131879in}}%
\pgfpathlineto{\pgfqpoint{2.436440in}{3.125967in}}%
\pgfpathlineto{\pgfqpoint{2.442863in}{3.117520in}}%
\pgfpathlineto{\pgfqpoint{2.445806in}{3.116756in}}%
\pgfpathlineto{\pgfqpoint{2.448750in}{3.118556in}}%
\pgfpathlineto{\pgfqpoint{2.453032in}{3.124478in}}%
\pgfpathlineto{\pgfqpoint{2.459454in}{3.132910in}}%
\pgfpathlineto{\pgfqpoint{2.462398in}{3.133660in}}%
\pgfpathlineto{\pgfqpoint{2.465342in}{3.131846in}}%
\pgfpathlineto{\pgfqpoint{2.469624in}{3.125914in}}%
\pgfpathlineto{\pgfqpoint{2.476046in}{3.117497in}}%
\pgfpathlineto{\pgfqpoint{2.478990in}{3.116762in}}%
\pgfpathlineto{\pgfqpoint{2.481934in}{3.118590in}}%
\pgfpathlineto{\pgfqpoint{2.486215in}{3.124531in}}%
\pgfpathlineto{\pgfqpoint{2.492638in}{3.132933in}}%
\pgfpathlineto{\pgfqpoint{2.495582in}{3.133653in}}%
\pgfpathlineto{\pgfqpoint{2.498525in}{3.131813in}}%
\pgfpathlineto{\pgfqpoint{2.502807in}{3.125861in}}%
\pgfpathlineto{\pgfqpoint{2.509230in}{3.117475in}}%
\pgfpathlineto{\pgfqpoint{2.512173in}{3.116769in}}%
\pgfpathlineto{\pgfqpoint{2.515117in}{3.118623in}}%
\pgfpathlineto{\pgfqpoint{2.519666in}{3.125015in}}%
\pgfpathlineto{\pgfqpoint{2.525821in}{3.132955in}}%
\pgfpathlineto{\pgfqpoint{2.528765in}{3.133647in}}%
\pgfpathlineto{\pgfqpoint{2.531709in}{3.131779in}}%
\pgfpathlineto{\pgfqpoint{2.536258in}{3.125377in}}%
\pgfpathlineto{\pgfqpoint{2.542413in}{3.117453in}}%
\pgfpathlineto{\pgfqpoint{2.545357in}{3.116776in}}%
\pgfpathlineto{\pgfqpoint{2.548300in}{3.118657in}}%
\pgfpathlineto{\pgfqpoint{2.552850in}{3.125069in}}%
\pgfpathlineto{\pgfqpoint{2.559005in}{3.132977in}}%
\pgfpathlineto{\pgfqpoint{2.561948in}{3.133639in}}%
\pgfpathlineto{\pgfqpoint{2.564892in}{3.131745in}}%
\pgfpathlineto{\pgfqpoint{2.569441in}{3.125323in}}%
\pgfpathlineto{\pgfqpoint{2.575596in}{3.117431in}}%
\pgfpathlineto{\pgfqpoint{2.578540in}{3.116783in}}%
\pgfpathlineto{\pgfqpoint{2.581484in}{3.118691in}}%
\pgfpathlineto{\pgfqpoint{2.586033in}{3.125122in}}%
\pgfpathlineto{\pgfqpoint{2.592188in}{3.132999in}}%
\pgfpathlineto{\pgfqpoint{2.595132in}{3.133632in}}%
\pgfpathlineto{\pgfqpoint{2.598075in}{3.131710in}}%
\pgfpathlineto{\pgfqpoint{2.602625in}{3.125270in}}%
\pgfpathlineto{\pgfqpoint{2.608780in}{3.117409in}}%
\pgfpathlineto{\pgfqpoint{2.611723in}{3.116791in}}%
\pgfpathlineto{\pgfqpoint{2.614667in}{3.118726in}}%
\pgfpathlineto{\pgfqpoint{2.619216in}{3.125175in}}%
\pgfpathlineto{\pgfqpoint{2.625371in}{3.133020in}}%
\pgfpathlineto{\pgfqpoint{2.628315in}{3.133624in}}%
\pgfpathlineto{\pgfqpoint{2.631259in}{3.131676in}}%
\pgfpathlineto{\pgfqpoint{2.635808in}{3.125217in}}%
\pgfpathlineto{\pgfqpoint{2.641963in}{3.117388in}}%
\pgfpathlineto{\pgfqpoint{2.644907in}{3.116799in}}%
\pgfpathlineto{\pgfqpoint{2.647850in}{3.118760in}}%
\pgfpathlineto{\pgfqpoint{2.652400in}{3.125229in}}%
\pgfpathlineto{\pgfqpoint{2.658287in}{3.132862in}}%
\pgfpathlineto{\pgfqpoint{2.661231in}{3.133672in}}%
\pgfpathlineto{\pgfqpoint{2.663907in}{3.132172in}}%
\pgfpathlineto{\pgfqpoint{2.667921in}{3.126879in}}%
\pgfpathlineto{\pgfqpoint{2.675146in}{3.117367in}}%
\pgfpathlineto{\pgfqpoint{2.678090in}{3.116807in}}%
\pgfpathlineto{\pgfqpoint{2.681034in}{3.118795in}}%
\pgfpathlineto{\pgfqpoint{2.685583in}{3.125282in}}%
\pgfpathlineto{\pgfqpoint{2.691470in}{3.132886in}}%
\pgfpathlineto{\pgfqpoint{2.694414in}{3.133666in}}%
\pgfpathlineto{\pgfqpoint{2.697358in}{3.131882in}}%
\pgfpathlineto{\pgfqpoint{2.701640in}{3.125972in}}%
\pgfpathlineto{\pgfqpoint{2.708062in}{3.117522in}}%
\pgfpathlineto{\pgfqpoint{2.711006in}{3.116755in}}%
\pgfpathlineto{\pgfqpoint{2.713950in}{3.118553in}}%
\pgfpathlineto{\pgfqpoint{2.718231in}{3.124473in}}%
\pgfpathlineto{\pgfqpoint{2.724654in}{3.132908in}}%
\pgfpathlineto{\pgfqpoint{2.727598in}{3.133660in}}%
\pgfpathlineto{\pgfqpoint{2.730541in}{3.131849in}}%
\pgfpathlineto{\pgfqpoint{2.734823in}{3.125919in}}%
\pgfpathlineto{\pgfqpoint{2.741246in}{3.117499in}}%
\pgfpathlineto{\pgfqpoint{2.744189in}{3.116762in}}%
\pgfpathlineto{\pgfqpoint{2.747133in}{3.118586in}}%
\pgfpathlineto{\pgfqpoint{2.751415in}{3.124526in}}%
\pgfpathlineto{\pgfqpoint{2.757837in}{3.132931in}}%
\pgfpathlineto{\pgfqpoint{2.760781in}{3.133654in}}%
\pgfpathlineto{\pgfqpoint{2.763725in}{3.131816in}}%
\pgfpathlineto{\pgfqpoint{2.768006in}{3.125866in}}%
\pgfpathlineto{\pgfqpoint{2.774429in}{3.117477in}}%
\pgfpathlineto{\pgfqpoint{2.777373in}{3.116768in}}%
\pgfpathlineto{\pgfqpoint{2.780316in}{3.118620in}}%
\pgfpathlineto{\pgfqpoint{2.784866in}{3.125011in}}%
\pgfpathlineto{\pgfqpoint{2.791021in}{3.132953in}}%
\pgfpathlineto{\pgfqpoint{2.793964in}{3.133647in}}%
\pgfpathlineto{\pgfqpoint{2.796908in}{3.131782in}}%
\pgfpathlineto{\pgfqpoint{2.801457in}{3.125382in}}%
\pgfpathlineto{\pgfqpoint{2.807612in}{3.117455in}}%
\pgfpathlineto{\pgfqpoint{2.810556in}{3.116775in}}%
\pgfpathlineto{\pgfqpoint{2.813500in}{3.118654in}}%
\pgfpathlineto{\pgfqpoint{2.818049in}{3.125064in}}%
\pgfpathlineto{\pgfqpoint{2.824204in}{3.132975in}}%
\pgfpathlineto{\pgfqpoint{2.827148in}{3.133640in}}%
\pgfpathlineto{\pgfqpoint{2.830091in}{3.131748in}}%
\pgfpathlineto{\pgfqpoint{2.834641in}{3.125328in}}%
\pgfpathlineto{\pgfqpoint{2.840796in}{3.117433in}}%
\pgfpathlineto{\pgfqpoint{2.843739in}{3.116783in}}%
\pgfpathlineto{\pgfqpoint{2.846683in}{3.118688in}}%
\pgfpathlineto{\pgfqpoint{2.851232in}{3.125117in}}%
\pgfpathlineto{\pgfqpoint{2.857387in}{3.132997in}}%
\pgfpathlineto{\pgfqpoint{2.860331in}{3.133633in}}%
\pgfpathlineto{\pgfqpoint{2.863275in}{3.131714in}}%
\pgfpathlineto{\pgfqpoint{2.867824in}{3.125275in}}%
\pgfpathlineto{\pgfqpoint{2.873979in}{3.117411in}}%
\pgfpathlineto{\pgfqpoint{2.876923in}{3.116790in}}%
\pgfpathlineto{\pgfqpoint{2.879866in}{3.118723in}}%
\pgfpathlineto{\pgfqpoint{2.884416in}{3.125171in}}%
\pgfpathlineto{\pgfqpoint{2.890571in}{3.133018in}}%
\pgfpathlineto{\pgfqpoint{2.893514in}{3.133625in}}%
\pgfpathlineto{\pgfqpoint{2.896458in}{3.131679in}}%
\pgfpathlineto{\pgfqpoint{2.901007in}{3.125222in}}%
\pgfpathlineto{\pgfqpoint{2.907162in}{3.117390in}}%
\pgfpathlineto{\pgfqpoint{2.910106in}{3.116798in}}%
\pgfpathlineto{\pgfqpoint{2.913050in}{3.118757in}}%
\pgfpathlineto{\pgfqpoint{2.917599in}{3.125224in}}%
\pgfpathlineto{\pgfqpoint{2.923754in}{3.133039in}}%
\pgfpathlineto{\pgfqpoint{2.926698in}{3.133617in}}%
\pgfpathlineto{\pgfqpoint{2.929641in}{3.131644in}}%
\pgfpathlineto{\pgfqpoint{2.934191in}{3.125168in}}%
\pgfpathlineto{\pgfqpoint{2.940078in}{3.117547in}}%
\pgfpathlineto{\pgfqpoint{2.943022in}{3.116749in}}%
\pgfpathlineto{\pgfqpoint{2.945698in}{3.118259in}}%
\pgfpathlineto{\pgfqpoint{2.949712in}{3.123561in}}%
\pgfpathlineto{\pgfqpoint{2.956937in}{3.133060in}}%
\pgfpathlineto{\pgfqpoint{2.959881in}{3.133608in}}%
\pgfpathlineto{\pgfqpoint{2.962825in}{3.131609in}}%
\pgfpathlineto{\pgfqpoint{2.967374in}{3.125115in}}%
\pgfpathlineto{\pgfqpoint{2.973262in}{3.117524in}}%
\pgfpathlineto{\pgfqpoint{2.976205in}{3.116755in}}%
\pgfpathlineto{\pgfqpoint{2.979149in}{3.118550in}}%
\pgfpathlineto{\pgfqpoint{2.983431in}{3.124468in}}%
\pgfpathlineto{\pgfqpoint{2.989853in}{3.132906in}}%
\pgfpathlineto{\pgfqpoint{2.992797in}{3.133661in}}%
\pgfpathlineto{\pgfqpoint{2.995741in}{3.131852in}}%
\pgfpathlineto{\pgfqpoint{3.000022in}{3.125924in}}%
\pgfpathlineto{\pgfqpoint{3.006445in}{3.117501in}}%
\pgfpathlineto{\pgfqpoint{3.009389in}{3.116761in}}%
\pgfpathlineto{\pgfqpoint{3.012332in}{3.118583in}}%
\pgfpathlineto{\pgfqpoint{3.016614in}{3.124522in}}%
\pgfpathlineto{\pgfqpoint{3.023037in}{3.132929in}}%
\pgfpathlineto{\pgfqpoint{3.025980in}{3.133655in}}%
\pgfpathlineto{\pgfqpoint{3.028924in}{3.131819in}}%
\pgfpathlineto{\pgfqpoint{3.033206in}{3.125871in}}%
\pgfpathlineto{\pgfqpoint{3.039628in}{3.117479in}}%
\pgfpathlineto{\pgfqpoint{3.042572in}{3.116768in}}%
\pgfpathlineto{\pgfqpoint{3.045516in}{3.118617in}}%
\pgfpathlineto{\pgfqpoint{3.050065in}{3.125006in}}%
\pgfpathlineto{\pgfqpoint{3.056220in}{3.132951in}}%
\pgfpathlineto{\pgfqpoint{3.059164in}{3.133648in}}%
\pgfpathlineto{\pgfqpoint{3.062107in}{3.131785in}}%
\pgfpathlineto{\pgfqpoint{3.066657in}{3.125386in}}%
\pgfpathlineto{\pgfqpoint{3.072812in}{3.117457in}}%
\pgfpathlineto{\pgfqpoint{3.075755in}{3.116775in}}%
\pgfpathlineto{\pgfqpoint{3.078699in}{3.118651in}}%
\pgfpathlineto{\pgfqpoint{3.083248in}{3.125059in}}%
\pgfpathlineto{\pgfqpoint{3.089403in}{3.132973in}}%
\pgfpathlineto{\pgfqpoint{3.092347in}{3.133641in}}%
\pgfpathlineto{\pgfqpoint{3.095291in}{3.131751in}}%
\pgfpathlineto{\pgfqpoint{3.099840in}{3.125333in}}%
\pgfpathlineto{\pgfqpoint{3.105995in}{3.117435in}}%
\pgfpathlineto{\pgfqpoint{3.108939in}{3.116782in}}%
\pgfpathlineto{\pgfqpoint{3.111882in}{3.118685in}}%
\pgfpathlineto{\pgfqpoint{3.116432in}{3.125112in}}%
\pgfpathlineto{\pgfqpoint{3.122587in}{3.132995in}}%
\pgfpathlineto{\pgfqpoint{3.125530in}{3.133633in}}%
\pgfpathlineto{\pgfqpoint{3.128474in}{3.131717in}}%
\pgfpathlineto{\pgfqpoint{3.133023in}{3.125280in}}%
\pgfpathlineto{\pgfqpoint{3.139178in}{3.117413in}}%
\pgfpathlineto{\pgfqpoint{3.142122in}{3.116789in}}%
\pgfpathlineto{\pgfqpoint{3.145066in}{3.118719in}}%
\pgfpathlineto{\pgfqpoint{3.149615in}{3.125166in}}%
\pgfpathlineto{\pgfqpoint{3.155770in}{3.133016in}}%
\pgfpathlineto{\pgfqpoint{3.158714in}{3.133626in}}%
\pgfpathlineto{\pgfqpoint{3.161657in}{3.131682in}}%
\pgfpathlineto{\pgfqpoint{3.166207in}{3.125226in}}%
\pgfpathlineto{\pgfqpoint{3.172362in}{3.117392in}}%
\pgfpathlineto{\pgfqpoint{3.175305in}{3.116797in}}%
\pgfpathlineto{\pgfqpoint{3.178249in}{3.118754in}}%
\pgfpathlineto{\pgfqpoint{3.182798in}{3.125219in}}%
\pgfpathlineto{\pgfqpoint{3.188953in}{3.133037in}}%
\pgfpathlineto{\pgfqpoint{3.191897in}{3.133617in}}%
\pgfpathlineto{\pgfqpoint{3.194841in}{3.131647in}}%
\pgfpathlineto{\pgfqpoint{3.199390in}{3.125173in}}%
\pgfpathlineto{\pgfqpoint{3.205277in}{3.117549in}}%
\pgfpathlineto{\pgfqpoint{3.208221in}{3.116748in}}%
\pgfpathlineto{\pgfqpoint{3.210897in}{3.118256in}}%
\pgfpathlineto{\pgfqpoint{3.214911in}{3.123557in}}%
\pgfpathlineto{\pgfqpoint{3.222137in}{3.133058in}}%
\pgfpathlineto{\pgfqpoint{3.225080in}{3.133609in}}%
\pgfpathlineto{\pgfqpoint{3.228024in}{3.131612in}}%
\pgfpathlineto{\pgfqpoint{3.232573in}{3.125120in}}%
\pgfpathlineto{\pgfqpoint{3.238461in}{3.117526in}}%
\pgfpathlineto{\pgfqpoint{3.241405in}{3.116754in}}%
\pgfpathlineto{\pgfqpoint{3.244348in}{3.118547in}}%
\pgfpathlineto{\pgfqpoint{3.248630in}{3.124464in}}%
\pgfpathlineto{\pgfqpoint{3.255053in}{3.132904in}}%
\pgfpathlineto{\pgfqpoint{3.257996in}{3.133661in}}%
\pgfpathlineto{\pgfqpoint{3.260940in}{3.131855in}}%
\pgfpathlineto{\pgfqpoint{3.265222in}{3.125929in}}%
\pgfpathlineto{\pgfqpoint{3.271644in}{3.117503in}}%
\pgfpathlineto{\pgfqpoint{3.274588in}{3.116761in}}%
\pgfpathlineto{\pgfqpoint{3.277532in}{3.118580in}}%
\pgfpathlineto{\pgfqpoint{3.281813in}{3.124517in}}%
\pgfpathlineto{\pgfqpoint{3.288236in}{3.132927in}}%
\pgfpathlineto{\pgfqpoint{3.291180in}{3.133655in}}%
\pgfpathlineto{\pgfqpoint{3.294123in}{3.131822in}}%
\pgfpathlineto{\pgfqpoint{3.298405in}{3.125876in}}%
\pgfpathlineto{\pgfqpoint{3.304828in}{3.117481in}}%
\pgfpathlineto{\pgfqpoint{3.307771in}{3.116767in}}%
\pgfpathlineto{\pgfqpoint{3.310715in}{3.118614in}}%
\pgfpathlineto{\pgfqpoint{3.315264in}{3.125001in}}%
\pgfpathlineto{\pgfqpoint{3.321419in}{3.132949in}}%
\pgfpathlineto{\pgfqpoint{3.324363in}{3.133648in}}%
\pgfpathlineto{\pgfqpoint{3.327307in}{3.131788in}}%
\pgfpathlineto{\pgfqpoint{3.331856in}{3.125391in}}%
\pgfpathlineto{\pgfqpoint{3.338011in}{3.117459in}}%
\pgfpathlineto{\pgfqpoint{3.340955in}{3.116774in}}%
\pgfpathlineto{\pgfqpoint{3.343898in}{3.118648in}}%
\pgfpathlineto{\pgfqpoint{3.348448in}{3.125054in}}%
\pgfpathlineto{\pgfqpoint{3.354603in}{3.132971in}}%
\pgfpathlineto{\pgfqpoint{3.357546in}{3.133641in}}%
\pgfpathlineto{\pgfqpoint{3.360490in}{3.131754in}}%
\pgfpathlineto{\pgfqpoint{3.365039in}{3.125338in}}%
\pgfpathlineto{\pgfqpoint{3.371194in}{3.117437in}}%
\pgfpathlineto{\pgfqpoint{3.374138in}{3.116781in}}%
\pgfpathlineto{\pgfqpoint{3.377082in}{3.118682in}}%
\pgfpathlineto{\pgfqpoint{3.381631in}{3.125108in}}%
\pgfpathlineto{\pgfqpoint{3.387786in}{3.132993in}}%
\pgfpathlineto{\pgfqpoint{3.390730in}{3.133634in}}%
\pgfpathlineto{\pgfqpoint{3.393673in}{3.131720in}}%
\pgfpathlineto{\pgfqpoint{3.398223in}{3.125285in}}%
\pgfpathlineto{\pgfqpoint{3.404378in}{3.117415in}}%
\pgfpathlineto{\pgfqpoint{3.407321in}{3.116789in}}%
\pgfpathlineto{\pgfqpoint{3.410265in}{3.118716in}}%
\pgfpathlineto{\pgfqpoint{3.414814in}{3.125161in}}%
\pgfpathlineto{\pgfqpoint{3.420969in}{3.133014in}}%
\pgfpathlineto{\pgfqpoint{3.423913in}{3.133626in}}%
\pgfpathlineto{\pgfqpoint{3.426857in}{3.131685in}}%
\pgfpathlineto{\pgfqpoint{3.431406in}{3.125231in}}%
\pgfpathlineto{\pgfqpoint{3.437561in}{3.117394in}}%
\pgfpathlineto{\pgfqpoint{3.440505in}{3.116797in}}%
\pgfpathlineto{\pgfqpoint{3.443448in}{3.118751in}}%
\pgfpathlineto{\pgfqpoint{3.447998in}{3.125214in}}%
\pgfpathlineto{\pgfqpoint{3.454153in}{3.133035in}}%
\pgfpathlineto{\pgfqpoint{3.457096in}{3.133618in}}%
\pgfpathlineto{\pgfqpoint{3.460040in}{3.131651in}}%
\pgfpathlineto{\pgfqpoint{3.464589in}{3.125178in}}%
\pgfpathlineto{\pgfqpoint{3.470477in}{3.117551in}}%
\pgfpathlineto{\pgfqpoint{3.473421in}{3.116748in}}%
\pgfpathlineto{\pgfqpoint{3.476097in}{3.118254in}}%
\pgfpathlineto{\pgfqpoint{3.480111in}{3.123552in}}%
\pgfpathlineto{\pgfqpoint{3.487336in}{3.133056in}}%
\pgfpathlineto{\pgfqpoint{3.490280in}{3.133610in}}%
\pgfpathlineto{\pgfqpoint{3.493224in}{3.131616in}}%
\pgfpathlineto{\pgfqpoint{3.497773in}{3.125125in}}%
\pgfpathlineto{\pgfqpoint{3.503660in}{3.117528in}}%
\pgfpathlineto{\pgfqpoint{3.506604in}{3.116754in}}%
\pgfpathlineto{\pgfqpoint{3.509548in}{3.118544in}}%
\pgfpathlineto{\pgfqpoint{3.513829in}{3.124459in}}%
\pgfpathlineto{\pgfqpoint{3.520252in}{3.132902in}}%
\pgfpathlineto{\pgfqpoint{3.523196in}{3.133662in}}%
\pgfpathlineto{\pgfqpoint{3.526139in}{3.131858in}}%
\pgfpathlineto{\pgfqpoint{3.530421in}{3.125934in}}%
\pgfpathlineto{\pgfqpoint{3.536844in}{3.117505in}}%
\pgfpathlineto{\pgfqpoint{3.539787in}{3.116760in}}%
\pgfpathlineto{\pgfqpoint{3.542731in}{3.118577in}}%
\pgfpathlineto{\pgfqpoint{3.547013in}{3.124512in}}%
\pgfpathlineto{\pgfqpoint{3.553435in}{3.132925in}}%
\pgfpathlineto{\pgfqpoint{3.556379in}{3.133656in}}%
\pgfpathlineto{\pgfqpoint{3.559323in}{3.131825in}}%
\pgfpathlineto{\pgfqpoint{3.563604in}{3.125880in}}%
\pgfpathlineto{\pgfqpoint{3.570027in}{3.117483in}}%
\pgfpathlineto{\pgfqpoint{3.572971in}{3.116766in}}%
\pgfpathlineto{\pgfqpoint{3.575914in}{3.118611in}}%
\pgfpathlineto{\pgfqpoint{3.580464in}{3.124996in}}%
\pgfpathlineto{\pgfqpoint{3.586619in}{3.132947in}}%
\pgfpathlineto{\pgfqpoint{3.589562in}{3.133649in}}%
\pgfpathlineto{\pgfqpoint{3.592506in}{3.131791in}}%
\pgfpathlineto{\pgfqpoint{3.597055in}{3.125396in}}%
\pgfpathlineto{\pgfqpoint{3.603210in}{3.117461in}}%
\pgfpathlineto{\pgfqpoint{3.606154in}{3.116773in}}%
\pgfpathlineto{\pgfqpoint{3.609098in}{3.118645in}}%
\pgfpathlineto{\pgfqpoint{3.613647in}{3.125049in}}%
\pgfpathlineto{\pgfqpoint{3.619802in}{3.132969in}}%
\pgfpathlineto{\pgfqpoint{3.622746in}{3.133642in}}%
\pgfpathlineto{\pgfqpoint{3.625689in}{3.131757in}}%
\pgfpathlineto{\pgfqpoint{3.630239in}{3.125343in}}%
\pgfpathlineto{\pgfqpoint{3.636394in}{3.117439in}}%
\pgfpathlineto{\pgfqpoint{3.639337in}{3.116780in}}%
\pgfpathlineto{\pgfqpoint{3.642281in}{3.118679in}}%
\pgfpathlineto{\pgfqpoint{3.646830in}{3.125103in}}%
\pgfpathlineto{\pgfqpoint{3.652985in}{3.132991in}}%
\pgfpathlineto{\pgfqpoint{3.655929in}{3.133635in}}%
\pgfpathlineto{\pgfqpoint{3.658873in}{3.131723in}}%
\pgfpathlineto{\pgfqpoint{3.663422in}{3.125289in}}%
\pgfpathlineto{\pgfqpoint{3.669577in}{3.117417in}}%
\pgfpathlineto{\pgfqpoint{3.672521in}{3.116788in}}%
\pgfpathlineto{\pgfqpoint{3.675464in}{3.118713in}}%
\pgfpathlineto{\pgfqpoint{3.680014in}{3.125156in}}%
\pgfpathlineto{\pgfqpoint{3.686169in}{3.133012in}}%
\pgfpathlineto{\pgfqpoint{3.689112in}{3.133627in}}%
\pgfpathlineto{\pgfqpoint{3.692056in}{3.131688in}}%
\pgfpathlineto{\pgfqpoint{3.696605in}{3.125236in}}%
\pgfpathlineto{\pgfqpoint{3.702760in}{3.117396in}}%
\pgfpathlineto{\pgfqpoint{3.705704in}{3.116796in}}%
\pgfpathlineto{\pgfqpoint{3.708648in}{3.118748in}}%
\pgfpathlineto{\pgfqpoint{3.713197in}{3.125209in}}%
\pgfpathlineto{\pgfqpoint{3.719352in}{3.133034in}}%
\pgfpathlineto{\pgfqpoint{3.722296in}{3.133619in}}%
\pgfpathlineto{\pgfqpoint{3.725239in}{3.131654in}}%
\pgfpathlineto{\pgfqpoint{3.729789in}{3.125183in}}%
\pgfpathlineto{\pgfqpoint{3.735676in}{3.117553in}}%
\pgfpathlineto{\pgfqpoint{3.738620in}{3.116747in}}%
\pgfpathlineto{\pgfqpoint{3.741296in}{3.118251in}}%
\pgfpathlineto{\pgfqpoint{3.745310in}{3.123547in}}%
\pgfpathlineto{\pgfqpoint{3.752535in}{3.133054in}}%
\pgfpathlineto{\pgfqpoint{3.755479in}{3.133611in}}%
\pgfpathlineto{\pgfqpoint{3.758423in}{3.131619in}}%
\pgfpathlineto{\pgfqpoint{3.762972in}{3.125129in}}%
\pgfpathlineto{\pgfqpoint{3.768860in}{3.117530in}}%
\pgfpathlineto{\pgfqpoint{3.771803in}{3.116753in}}%
\pgfpathlineto{\pgfqpoint{3.774747in}{3.118541in}}%
\pgfpathlineto{\pgfqpoint{3.779029in}{3.124454in}}%
\pgfpathlineto{\pgfqpoint{3.785451in}{3.132900in}}%
\pgfpathlineto{\pgfqpoint{3.788395in}{3.133663in}}%
\pgfpathlineto{\pgfqpoint{3.791339in}{3.131861in}}%
\pgfpathlineto{\pgfqpoint{3.795620in}{3.125938in}}%
\pgfpathlineto{\pgfqpoint{3.802043in}{3.117507in}}%
\pgfpathlineto{\pgfqpoint{3.804987in}{3.116759in}}%
\pgfpathlineto{\pgfqpoint{3.807930in}{3.118574in}}%
\pgfpathlineto{\pgfqpoint{3.812212in}{3.124507in}}%
\pgfpathlineto{\pgfqpoint{3.818635in}{3.132923in}}%
\pgfpathlineto{\pgfqpoint{3.821578in}{3.133656in}}%
\pgfpathlineto{\pgfqpoint{3.824522in}{3.131828in}}%
\pgfpathlineto{\pgfqpoint{3.828804in}{3.125885in}}%
\pgfpathlineto{\pgfqpoint{3.835226in}{3.117485in}}%
\pgfpathlineto{\pgfqpoint{3.838170in}{3.116766in}}%
\pgfpathlineto{\pgfqpoint{3.841114in}{3.118608in}}%
\pgfpathlineto{\pgfqpoint{3.845395in}{3.124560in}}%
\pgfpathlineto{\pgfqpoint{3.851818in}{3.132945in}}%
\pgfpathlineto{\pgfqpoint{3.854762in}{3.133650in}}%
\pgfpathlineto{\pgfqpoint{3.857705in}{3.131794in}}%
\pgfpathlineto{\pgfqpoint{3.862255in}{3.125401in}}%
\pgfpathlineto{\pgfqpoint{3.868410in}{3.117463in}}%
\pgfpathlineto{\pgfqpoint{3.871353in}{3.116773in}}%
\pgfpathlineto{\pgfqpoint{3.874297in}{3.118642in}}%
\pgfpathlineto{\pgfqpoint{3.878846in}{3.125045in}}%
\pgfpathlineto{\pgfqpoint{3.885001in}{3.132967in}}%
\pgfpathlineto{\pgfqpoint{3.887945in}{3.133643in}}%
\pgfpathlineto{\pgfqpoint{3.890889in}{3.131760in}}%
\pgfpathlineto{\pgfqpoint{3.895438in}{3.125348in}}%
\pgfpathlineto{\pgfqpoint{3.901593in}{3.117441in}}%
\pgfpathlineto{\pgfqpoint{3.904537in}{3.116780in}}%
\pgfpathlineto{\pgfqpoint{3.907480in}{3.118676in}}%
\pgfpathlineto{\pgfqpoint{3.912030in}{3.125098in}}%
\pgfpathlineto{\pgfqpoint{3.918185in}{3.132989in}}%
\pgfpathlineto{\pgfqpoint{3.921128in}{3.133635in}}%
\pgfpathlineto{\pgfqpoint{3.924072in}{3.131726in}}%
\pgfpathlineto{\pgfqpoint{3.928621in}{3.125294in}}%
\pgfpathlineto{\pgfqpoint{3.934776in}{3.117419in}}%
\pgfpathlineto{\pgfqpoint{3.937720in}{3.116787in}}%
\pgfpathlineto{\pgfqpoint{3.940664in}{3.118710in}}%
\pgfpathlineto{\pgfqpoint{3.945213in}{3.125151in}}%
\pgfpathlineto{\pgfqpoint{3.951368in}{3.133011in}}%
\pgfpathlineto{\pgfqpoint{3.954312in}{3.133628in}}%
\pgfpathlineto{\pgfqpoint{3.957255in}{3.131692in}}%
\pgfpathlineto{\pgfqpoint{3.961805in}{3.125241in}}%
\pgfpathlineto{\pgfqpoint{3.967960in}{3.117398in}}%
\pgfpathlineto{\pgfqpoint{3.970903in}{3.116795in}}%
\pgfpathlineto{\pgfqpoint{3.973847in}{3.118745in}}%
\pgfpathlineto{\pgfqpoint{3.978396in}{3.125205in}}%
\pgfpathlineto{\pgfqpoint{3.984551in}{3.133032in}}%
\pgfpathlineto{\pgfqpoint{3.987495in}{3.133620in}}%
\pgfpathlineto{\pgfqpoint{3.990439in}{3.131657in}}%
\pgfpathlineto{\pgfqpoint{3.994988in}{3.125188in}}%
\pgfpathlineto{\pgfqpoint{4.000876in}{3.117555in}}%
\pgfpathlineto{\pgfqpoint{4.003819in}{3.116747in}}%
\pgfpathlineto{\pgfqpoint{4.006495in}{3.118248in}}%
\pgfpathlineto{\pgfqpoint{4.010509in}{3.123542in}}%
\pgfpathlineto{\pgfqpoint{4.017735in}{3.133053in}}%
\pgfpathlineto{\pgfqpoint{4.020679in}{3.133611in}}%
\pgfpathlineto{\pgfqpoint{4.023622in}{3.131622in}}%
\pgfpathlineto{\pgfqpoint{4.028172in}{3.125134in}}%
\pgfpathlineto{\pgfqpoint{4.034059in}{3.117532in}}%
\pgfpathlineto{\pgfqpoint{4.037003in}{3.116753in}}%
\pgfpathlineto{\pgfqpoint{4.039946in}{3.118538in}}%
\pgfpathlineto{\pgfqpoint{4.044228in}{3.124449in}}%
\pgfpathlineto{\pgfqpoint{4.050651in}{3.132898in}}%
\pgfpathlineto{\pgfqpoint{4.053594in}{3.133663in}}%
\pgfpathlineto{\pgfqpoint{4.056538in}{3.131864in}}%
\pgfpathlineto{\pgfqpoint{4.060820in}{3.125943in}}%
\pgfpathlineto{\pgfqpoint{4.067242in}{3.117509in}}%
\pgfpathlineto{\pgfqpoint{4.070186in}{3.116759in}}%
\pgfpathlineto{\pgfqpoint{4.073130in}{3.118571in}}%
\pgfpathlineto{\pgfqpoint{4.077411in}{3.124502in}}%
\pgfpathlineto{\pgfqpoint{4.083834in}{3.132921in}}%
\pgfpathlineto{\pgfqpoint{4.086778in}{3.133657in}}%
\pgfpathlineto{\pgfqpoint{4.089721in}{3.131831in}}%
\pgfpathlineto{\pgfqpoint{4.094003in}{3.125890in}}%
\pgfpathlineto{\pgfqpoint{4.100426in}{3.117487in}}%
\pgfpathlineto{\pgfqpoint{4.103369in}{3.116765in}}%
\pgfpathlineto{\pgfqpoint{4.106313in}{3.118605in}}%
\pgfpathlineto{\pgfqpoint{4.110595in}{3.124555in}}%
\pgfpathlineto{\pgfqpoint{4.117017in}{3.132943in}}%
\pgfpathlineto{\pgfqpoint{4.119961in}{3.133650in}}%
\pgfpathlineto{\pgfqpoint{4.122905in}{3.131797in}}%
\pgfpathlineto{\pgfqpoint{4.127454in}{3.125406in}}%
\pgfpathlineto{\pgfqpoint{4.133609in}{3.117465in}}%
\pgfpathlineto{\pgfqpoint{4.136553in}{3.116772in}}%
\pgfpathlineto{\pgfqpoint{4.139496in}{3.118639in}}%
\pgfpathlineto{\pgfqpoint{4.144046in}{3.125040in}}%
\pgfpathlineto{\pgfqpoint{4.150201in}{3.132965in}}%
\pgfpathlineto{\pgfqpoint{4.153144in}{3.133643in}}%
\pgfpathlineto{\pgfqpoint{4.156088in}{3.131763in}}%
\pgfpathlineto{\pgfqpoint{4.160637in}{3.125352in}}%
\pgfpathlineto{\pgfqpoint{4.166792in}{3.117443in}}%
\pgfpathlineto{\pgfqpoint{4.169736in}{3.116779in}}%
\pgfpathlineto{\pgfqpoint{4.172680in}{3.118673in}}%
\pgfpathlineto{\pgfqpoint{4.177229in}{3.125093in}}%
\pgfpathlineto{\pgfqpoint{4.183384in}{3.132987in}}%
\pgfpathlineto{\pgfqpoint{4.186328in}{3.133636in}}%
\pgfpathlineto{\pgfqpoint{4.189271in}{3.131729in}}%
\pgfpathlineto{\pgfqpoint{4.193821in}{3.125299in}}%
\pgfpathlineto{\pgfqpoint{4.199976in}{3.117421in}}%
\pgfpathlineto{\pgfqpoint{4.202919in}{3.116787in}}%
\pgfpathlineto{\pgfqpoint{4.205863in}{3.118707in}}%
\pgfpathlineto{\pgfqpoint{4.210412in}{3.125146in}}%
\pgfpathlineto{\pgfqpoint{4.216567in}{3.133009in}}%
\pgfpathlineto{\pgfqpoint{4.219511in}{3.133628in}}%
\pgfpathlineto{\pgfqpoint{4.222455in}{3.131695in}}%
\pgfpathlineto{\pgfqpoint{4.227004in}{3.125246in}}%
\pgfpathlineto{\pgfqpoint{4.233159in}{3.117400in}}%
\pgfpathlineto{\pgfqpoint{4.236103in}{3.116794in}}%
\pgfpathlineto{\pgfqpoint{4.239047in}{3.118741in}}%
\pgfpathlineto{\pgfqpoint{4.243596in}{3.125200in}}%
\pgfpathlineto{\pgfqpoint{4.249751in}{3.133030in}}%
\pgfpathlineto{\pgfqpoint{4.252695in}{3.133620in}}%
\pgfpathlineto{\pgfqpoint{4.255638in}{3.131660in}}%
\pgfpathlineto{\pgfqpoint{4.260188in}{3.125192in}}%
\pgfpathlineto{\pgfqpoint{4.266075in}{3.117558in}}%
\pgfpathlineto{\pgfqpoint{4.269019in}{3.116746in}}%
\pgfpathlineto{\pgfqpoint{4.271695in}{3.118245in}}%
\pgfpathlineto{\pgfqpoint{4.275709in}{3.123537in}}%
\pgfpathlineto{\pgfqpoint{4.282934in}{3.133051in}}%
\pgfpathlineto{\pgfqpoint{4.285878in}{3.133612in}}%
\pgfpathlineto{\pgfqpoint{4.288822in}{3.131625in}}%
\pgfpathlineto{\pgfqpoint{4.293371in}{3.125139in}}%
\pgfpathlineto{\pgfqpoint{4.299258in}{3.117534in}}%
\pgfpathlineto{\pgfqpoint{4.302202in}{3.116752in}}%
\pgfpathlineto{\pgfqpoint{4.305146in}{3.118535in}}%
\pgfpathlineto{\pgfqpoint{4.309427in}{3.124444in}}%
\pgfpathlineto{\pgfqpoint{4.315850in}{3.132896in}}%
\pgfpathlineto{\pgfqpoint{4.318794in}{3.133664in}}%
\pgfpathlineto{\pgfqpoint{4.321737in}{3.131867in}}%
\pgfpathlineto{\pgfqpoint{4.326019in}{3.125948in}}%
\pgfpathlineto{\pgfqpoint{4.332442in}{3.117511in}}%
\pgfpathlineto{\pgfqpoint{4.335385in}{3.116758in}}%
\pgfpathlineto{\pgfqpoint{4.338329in}{3.118568in}}%
\pgfpathlineto{\pgfqpoint{4.342611in}{3.124497in}}%
\pgfpathlineto{\pgfqpoint{4.349033in}{3.132919in}}%
\pgfpathlineto{\pgfqpoint{4.351977in}{3.133657in}}%
\pgfpathlineto{\pgfqpoint{4.354921in}{3.131834in}}%
\pgfpathlineto{\pgfqpoint{4.359202in}{3.125895in}}%
\pgfpathlineto{\pgfqpoint{4.365625in}{3.117489in}}%
\pgfpathlineto{\pgfqpoint{4.368569in}{3.116765in}}%
\pgfpathlineto{\pgfqpoint{4.371512in}{3.118602in}}%
\pgfpathlineto{\pgfqpoint{4.375794in}{3.124551in}}%
\pgfpathlineto{\pgfqpoint{4.382217in}{3.132941in}}%
\pgfpathlineto{\pgfqpoint{4.385160in}{3.133651in}}%
\pgfpathlineto{\pgfqpoint{4.388104in}{3.131800in}}%
\pgfpathlineto{\pgfqpoint{4.392653in}{3.125411in}}%
\pgfpathlineto{\pgfqpoint{4.398808in}{3.117467in}}%
\pgfpathlineto{\pgfqpoint{4.401752in}{3.116771in}}%
\pgfpathlineto{\pgfqpoint{4.404696in}{3.118635in}}%
\pgfpathlineto{\pgfqpoint{4.409245in}{3.125035in}}%
\pgfpathlineto{\pgfqpoint{4.415400in}{3.132963in}}%
\pgfpathlineto{\pgfqpoint{4.418344in}{3.133644in}}%
\pgfpathlineto{\pgfqpoint{4.421287in}{3.131766in}}%
\pgfpathlineto{\pgfqpoint{4.425837in}{3.125357in}}%
\pgfpathlineto{\pgfqpoint{4.431992in}{3.117445in}}%
\pgfpathlineto{\pgfqpoint{4.434935in}{3.116779in}}%
\pgfpathlineto{\pgfqpoint{4.437879in}{3.118669in}}%
\pgfpathlineto{\pgfqpoint{4.442428in}{3.125088in}}%
\pgfpathlineto{\pgfqpoint{4.448583in}{3.132985in}}%
\pgfpathlineto{\pgfqpoint{4.451527in}{3.133637in}}%
\pgfpathlineto{\pgfqpoint{4.454471in}{3.131732in}}%
\pgfpathlineto{\pgfqpoint{4.459020in}{3.125304in}}%
\pgfpathlineto{\pgfqpoint{4.465175in}{3.117423in}}%
\pgfpathlineto{\pgfqpoint{4.468119in}{3.116786in}}%
\pgfpathlineto{\pgfqpoint{4.471062in}{3.118704in}}%
\pgfpathlineto{\pgfqpoint{4.475612in}{3.125142in}}%
\pgfpathlineto{\pgfqpoint{4.481767in}{3.133007in}}%
\pgfpathlineto{\pgfqpoint{4.484710in}{3.133629in}}%
\pgfpathlineto{\pgfqpoint{4.487654in}{3.131698in}}%
\pgfpathlineto{\pgfqpoint{4.492204in}{3.125251in}}%
\pgfpathlineto{\pgfqpoint{4.498358in}{3.117402in}}%
\pgfpathlineto{\pgfqpoint{4.501302in}{3.116794in}}%
\pgfpathlineto{\pgfqpoint{4.504246in}{3.118738in}}%
\pgfpathlineto{\pgfqpoint{4.508795in}{3.125195in}}%
\pgfpathlineto{\pgfqpoint{4.514950in}{3.133028in}}%
\pgfpathlineto{\pgfqpoint{4.517894in}{3.133621in}}%
\pgfpathlineto{\pgfqpoint{4.520838in}{3.131663in}}%
\pgfpathlineto{\pgfqpoint{4.525387in}{3.125197in}}%
\pgfpathlineto{\pgfqpoint{4.531542in}{3.117381in}}%
\pgfpathlineto{\pgfqpoint{4.534486in}{3.116802in}}%
\pgfpathlineto{\pgfqpoint{4.537429in}{3.118773in}}%
\pgfpathlineto{\pgfqpoint{4.541979in}{3.125248in}}%
\pgfpathlineto{\pgfqpoint{4.547866in}{3.132871in}}%
\pgfpathlineto{\pgfqpoint{4.550810in}{3.133670in}}%
\pgfpathlineto{\pgfqpoint{4.553486in}{3.132161in}}%
\pgfpathlineto{\pgfqpoint{4.557500in}{3.126860in}}%
\pgfpathlineto{\pgfqpoint{4.564725in}{3.117360in}}%
\pgfpathlineto{\pgfqpoint{4.567669in}{3.116810in}}%
\pgfpathlineto{\pgfqpoint{4.570613in}{3.118808in}}%
\pgfpathlineto{\pgfqpoint{4.575162in}{3.125302in}}%
\pgfpathlineto{\pgfqpoint{4.581049in}{3.132894in}}%
\pgfpathlineto{\pgfqpoint{4.583993in}{3.133664in}}%
\pgfpathlineto{\pgfqpoint{4.586937in}{3.131870in}}%
\pgfpathlineto{\pgfqpoint{4.591218in}{3.125953in}}%
\pgfpathlineto{\pgfqpoint{4.597641in}{3.117514in}}%
\pgfpathlineto{\pgfqpoint{4.600585in}{3.116758in}}%
\pgfpathlineto{\pgfqpoint{4.603528in}{3.118565in}}%
\pgfpathlineto{\pgfqpoint{4.607810in}{3.124493in}}%
\pgfpathlineto{\pgfqpoint{4.614233in}{3.132917in}}%
\pgfpathlineto{\pgfqpoint{4.617176in}{3.133658in}}%
\pgfpathlineto{\pgfqpoint{4.620120in}{3.131837in}}%
\pgfpathlineto{\pgfqpoint{4.624402in}{3.125900in}}%
\pgfpathlineto{\pgfqpoint{4.630824in}{3.117491in}}%
\pgfpathlineto{\pgfqpoint{4.633768in}{3.116764in}}%
\pgfpathlineto{\pgfqpoint{4.636712in}{3.118599in}}%
\pgfpathlineto{\pgfqpoint{4.640993in}{3.124546in}}%
\pgfpathlineto{\pgfqpoint{4.647416in}{3.132939in}}%
\pgfpathlineto{\pgfqpoint{4.650360in}{3.133651in}}%
\pgfpathlineto{\pgfqpoint{4.653303in}{3.131803in}}%
\pgfpathlineto{\pgfqpoint{4.657853in}{3.125416in}}%
\pgfpathlineto{\pgfqpoint{4.664008in}{3.117469in}}%
\pgfpathlineto{\pgfqpoint{4.666951in}{3.116771in}}%
\pgfpathlineto{\pgfqpoint{4.669895in}{3.118632in}}%
\pgfpathlineto{\pgfqpoint{4.674444in}{3.125030in}}%
\pgfpathlineto{\pgfqpoint{4.680599in}{3.132961in}}%
\pgfpathlineto{\pgfqpoint{4.683543in}{3.133645in}}%
\pgfpathlineto{\pgfqpoint{4.686487in}{3.131770in}}%
\pgfpathlineto{\pgfqpoint{4.691036in}{3.125362in}}%
\pgfpathlineto{\pgfqpoint{4.697191in}{3.117447in}}%
\pgfpathlineto{\pgfqpoint{4.700135in}{3.116778in}}%
\pgfpathlineto{\pgfqpoint{4.703078in}{3.118666in}}%
\pgfpathlineto{\pgfqpoint{4.707628in}{3.125083in}}%
\pgfpathlineto{\pgfqpoint{4.713783in}{3.132983in}}%
\pgfpathlineto{\pgfqpoint{4.716726in}{3.133637in}}%
\pgfpathlineto{\pgfqpoint{4.719670in}{3.131735in}}%
\pgfpathlineto{\pgfqpoint{4.724219in}{3.125309in}}%
\pgfpathlineto{\pgfqpoint{4.730374in}{3.117425in}}%
\pgfpathlineto{\pgfqpoint{4.733318in}{3.116785in}}%
\pgfpathlineto{\pgfqpoint{4.736262in}{3.118701in}}%
\pgfpathlineto{\pgfqpoint{4.740811in}{3.125137in}}%
\pgfpathlineto{\pgfqpoint{4.746966in}{3.133005in}}%
\pgfpathlineto{\pgfqpoint{4.749910in}{3.133630in}}%
\pgfpathlineto{\pgfqpoint{4.752854in}{3.131701in}}%
\pgfpathlineto{\pgfqpoint{4.757403in}{3.125256in}}%
\pgfpathlineto{\pgfqpoint{4.763558in}{3.117404in}}%
\pgfpathlineto{\pgfqpoint{4.766502in}{3.116793in}}%
\pgfpathlineto{\pgfqpoint{4.769445in}{3.118735in}}%
\pgfpathlineto{\pgfqpoint{4.773995in}{3.125190in}}%
\pgfpathlineto{\pgfqpoint{4.780150in}{3.133026in}}%
\pgfpathlineto{\pgfqpoint{4.783093in}{3.133622in}}%
\pgfpathlineto{\pgfqpoint{4.786037in}{3.131666in}}%
\pgfpathlineto{\pgfqpoint{4.790586in}{3.125202in}}%
\pgfpathlineto{\pgfqpoint{4.796741in}{3.117382in}}%
\pgfpathlineto{\pgfqpoint{4.799685in}{3.116801in}}%
\pgfpathlineto{\pgfqpoint{4.802629in}{3.118770in}}%
\pgfpathlineto{\pgfqpoint{4.807178in}{3.125243in}}%
\pgfpathlineto{\pgfqpoint{4.813065in}{3.132869in}}%
\pgfpathlineto{\pgfqpoint{4.816009in}{3.133671in}}%
\pgfpathlineto{\pgfqpoint{4.818685in}{3.132164in}}%
\pgfpathlineto{\pgfqpoint{4.822699in}{3.126865in}}%
\pgfpathlineto{\pgfqpoint{4.829925in}{3.117362in}}%
\pgfpathlineto{\pgfqpoint{4.832868in}{3.116809in}}%
\pgfpathlineto{\pgfqpoint{4.835812in}{3.118805in}}%
\pgfpathlineto{\pgfqpoint{4.840361in}{3.125297in}}%
\pgfpathlineto{\pgfqpoint{4.846249in}{3.132892in}}%
\pgfpathlineto{\pgfqpoint{4.849192in}{3.133665in}}%
\pgfpathlineto{\pgfqpoint{4.852136in}{3.131873in}}%
\pgfpathlineto{\pgfqpoint{4.856418in}{3.125958in}}%
\pgfpathlineto{\pgfqpoint{4.862840in}{3.117516in}}%
\pgfpathlineto{\pgfqpoint{4.865784in}{3.116757in}}%
\pgfpathlineto{\pgfqpoint{4.868728in}{3.118562in}}%
\pgfpathlineto{\pgfqpoint{4.873009in}{3.124488in}}%
\pgfpathlineto{\pgfqpoint{4.879432in}{3.132915in}}%
\pgfpathlineto{\pgfqpoint{4.882376in}{3.133659in}}%
\pgfpathlineto{\pgfqpoint{4.885319in}{3.131840in}}%
\pgfpathlineto{\pgfqpoint{4.889601in}{3.125905in}}%
\pgfpathlineto{\pgfqpoint{4.896024in}{3.117493in}}%
\pgfpathlineto{\pgfqpoint{4.898967in}{3.116763in}}%
\pgfpathlineto{\pgfqpoint{4.901911in}{3.118596in}}%
\pgfpathlineto{\pgfqpoint{4.906193in}{3.124541in}}%
\pgfpathlineto{\pgfqpoint{4.912615in}{3.132937in}}%
\pgfpathlineto{\pgfqpoint{4.915559in}{3.133652in}}%
\pgfpathlineto{\pgfqpoint{4.918503in}{3.131806in}}%
\pgfpathlineto{\pgfqpoint{4.923052in}{3.125420in}}%
\pgfpathlineto{\pgfqpoint{4.929207in}{3.117471in}}%
\pgfpathlineto{\pgfqpoint{4.932151in}{3.116770in}}%
\pgfpathlineto{\pgfqpoint{4.935094in}{3.118629in}}%
\pgfpathlineto{\pgfqpoint{4.939644in}{3.125025in}}%
\pgfpathlineto{\pgfqpoint{4.945799in}{3.132959in}}%
\pgfpathlineto{\pgfqpoint{4.948742in}{3.133645in}}%
\pgfpathlineto{\pgfqpoint{4.951686in}{3.131773in}}%
\pgfpathlineto{\pgfqpoint{4.956235in}{3.125367in}}%
\pgfpathlineto{\pgfqpoint{4.962390in}{3.117449in}}%
\pgfpathlineto{\pgfqpoint{4.965334in}{3.116777in}}%
\pgfpathlineto{\pgfqpoint{4.968278in}{3.118663in}}%
\pgfpathlineto{\pgfqpoint{4.972827in}{3.125078in}}%
\pgfpathlineto{\pgfqpoint{4.978982in}{3.132981in}}%
\pgfpathlineto{\pgfqpoint{4.981926in}{3.133638in}}%
\pgfpathlineto{\pgfqpoint{4.984870in}{3.131739in}}%
\pgfpathlineto{\pgfqpoint{4.989419in}{3.125314in}}%
\pgfpathlineto{\pgfqpoint{4.995574in}{3.117427in}}%
\pgfpathlineto{\pgfqpoint{4.998518in}{3.116785in}}%
\pgfpathlineto{\pgfqpoint{5.001461in}{3.118697in}}%
\pgfpathlineto{\pgfqpoint{5.006011in}{3.125132in}}%
\pgfpathlineto{\pgfqpoint{5.012166in}{3.133003in}}%
\pgfpathlineto{\pgfqpoint{5.015109in}{3.133631in}}%
\pgfpathlineto{\pgfqpoint{5.018053in}{3.131704in}}%
\pgfpathlineto{\pgfqpoint{5.022602in}{3.125260in}}%
\pgfpathlineto{\pgfqpoint{5.028757in}{3.117405in}}%
\pgfpathlineto{\pgfqpoint{5.031701in}{3.116792in}}%
\pgfpathlineto{\pgfqpoint{5.034645in}{3.118732in}}%
\pgfpathlineto{\pgfqpoint{5.039194in}{3.125185in}}%
\pgfpathlineto{\pgfqpoint{5.045349in}{3.133024in}}%
\pgfpathlineto{\pgfqpoint{5.048293in}{3.133623in}}%
\pgfpathlineto{\pgfqpoint{5.051236in}{3.131670in}}%
\pgfpathlineto{\pgfqpoint{5.055786in}{3.125207in}}%
\pgfpathlineto{\pgfqpoint{5.061941in}{3.117384in}}%
\pgfpathlineto{\pgfqpoint{5.064884in}{3.116800in}}%
\pgfpathlineto{\pgfqpoint{5.067828in}{3.118767in}}%
\pgfpathlineto{\pgfqpoint{5.072377in}{3.125239in}}%
\pgfpathlineto{\pgfqpoint{5.078265in}{3.132867in}}%
\pgfpathlineto{\pgfqpoint{5.081208in}{3.133671in}}%
\pgfpathlineto{\pgfqpoint{5.083884in}{3.132167in}}%
\pgfpathlineto{\pgfqpoint{5.087899in}{3.126869in}}%
\pgfpathlineto{\pgfqpoint{5.095124in}{3.117364in}}%
\pgfpathlineto{\pgfqpoint{5.098068in}{3.116809in}}%
\pgfpathlineto{\pgfqpoint{5.101011in}{3.118802in}}%
\pgfpathlineto{\pgfqpoint{5.105561in}{3.125292in}}%
\pgfpathlineto{\pgfqpoint{5.111448in}{3.132890in}}%
\pgfpathlineto{\pgfqpoint{5.114392in}{3.133665in}}%
\pgfpathlineto{\pgfqpoint{5.117335in}{3.131876in}}%
\pgfpathlineto{\pgfqpoint{5.121617in}{3.125963in}}%
\pgfpathlineto{\pgfqpoint{5.128040in}{3.117518in}}%
\pgfpathlineto{\pgfqpoint{5.130983in}{3.116757in}}%
\pgfpathlineto{\pgfqpoint{5.133927in}{3.118559in}}%
\pgfpathlineto{\pgfqpoint{5.138209in}{3.124483in}}%
\pgfpathlineto{\pgfqpoint{5.144631in}{3.132913in}}%
\pgfpathlineto{\pgfqpoint{5.147575in}{3.133659in}}%
\pgfpathlineto{\pgfqpoint{5.150519in}{3.131843in}}%
\pgfpathlineto{\pgfqpoint{5.154800in}{3.125909in}}%
\pgfpathlineto{\pgfqpoint{5.161223in}{3.117495in}}%
\pgfpathlineto{\pgfqpoint{5.164167in}{3.116763in}}%
\pgfpathlineto{\pgfqpoint{5.167110in}{3.118593in}}%
\pgfpathlineto{\pgfqpoint{5.171392in}{3.124536in}}%
\pgfpathlineto{\pgfqpoint{5.177815in}{3.132935in}}%
\pgfpathlineto{\pgfqpoint{5.180758in}{3.133653in}}%
\pgfpathlineto{\pgfqpoint{5.183702in}{3.131810in}}%
\pgfpathlineto{\pgfqpoint{5.188251in}{3.125425in}}%
\pgfpathlineto{\pgfqpoint{5.194406in}{3.117473in}}%
\pgfpathlineto{\pgfqpoint{5.197350in}{3.116770in}}%
\pgfpathlineto{\pgfqpoint{5.200294in}{3.118626in}}%
\pgfpathlineto{\pgfqpoint{5.204843in}{3.125020in}}%
\pgfpathlineto{\pgfqpoint{5.210998in}{3.132957in}}%
\pgfpathlineto{\pgfqpoint{5.213942in}{3.133646in}}%
\pgfpathlineto{\pgfqpoint{5.216885in}{3.131776in}}%
\pgfpathlineto{\pgfqpoint{5.221435in}{3.125372in}}%
\pgfpathlineto{\pgfqpoint{5.227590in}{3.117451in}}%
\pgfpathlineto{\pgfqpoint{5.230533in}{3.116777in}}%
\pgfpathlineto{\pgfqpoint{5.233477in}{3.118660in}}%
\pgfpathlineto{\pgfqpoint{5.238027in}{3.125074in}}%
\pgfpathlineto{\pgfqpoint{5.244181in}{3.132979in}}%
\pgfpathlineto{\pgfqpoint{5.247125in}{3.133639in}}%
\pgfpathlineto{\pgfqpoint{5.250069in}{3.131742in}}%
\pgfpathlineto{\pgfqpoint{5.254618in}{3.125319in}}%
\pgfpathlineto{\pgfqpoint{5.260773in}{3.117429in}}%
\pgfpathlineto{\pgfqpoint{5.263717in}{3.116784in}}%
\pgfpathlineto{\pgfqpoint{5.266661in}{3.118694in}}%
\pgfpathlineto{\pgfqpoint{5.271210in}{3.125127in}}%
\pgfpathlineto{\pgfqpoint{5.277365in}{3.133001in}}%
\pgfpathlineto{\pgfqpoint{5.280309in}{3.133631in}}%
\pgfpathlineto{\pgfqpoint{5.283252in}{3.131707in}}%
\pgfpathlineto{\pgfqpoint{5.287802in}{3.125265in}}%
\pgfpathlineto{\pgfqpoint{5.293957in}{3.117407in}}%
\pgfpathlineto{\pgfqpoint{5.296900in}{3.116792in}}%
\pgfpathlineto{\pgfqpoint{5.299844in}{3.118729in}}%
\pgfpathlineto{\pgfqpoint{5.304393in}{3.125180in}}%
\pgfpathlineto{\pgfqpoint{5.310548in}{3.133022in}}%
\pgfpathlineto{\pgfqpoint{5.313492in}{3.133623in}}%
\pgfpathlineto{\pgfqpoint{5.316436in}{3.131673in}}%
\pgfpathlineto{\pgfqpoint{5.320985in}{3.125212in}}%
\pgfpathlineto{\pgfqpoint{5.327140in}{3.117386in}}%
\pgfpathlineto{\pgfqpoint{5.330084in}{3.116800in}}%
\pgfpathlineto{\pgfqpoint{5.333027in}{3.118764in}}%
\pgfpathlineto{\pgfqpoint{5.337577in}{3.125234in}}%
\pgfpathlineto{\pgfqpoint{5.343464in}{3.132864in}}%
\pgfpathlineto{\pgfqpoint{5.346408in}{3.133672in}}%
\pgfpathlineto{\pgfqpoint{5.349084in}{3.132170in}}%
\pgfpathlineto{\pgfqpoint{5.353098in}{3.126874in}}%
\pgfpathlineto{\pgfqpoint{5.360323in}{3.117365in}}%
\pgfpathlineto{\pgfqpoint{5.363267in}{3.116808in}}%
\pgfpathlineto{\pgfqpoint{5.366211in}{3.118798in}}%
\pgfpathlineto{\pgfqpoint{5.370760in}{3.125287in}}%
\pgfpathlineto{\pgfqpoint{5.376647in}{3.132888in}}%
\pgfpathlineto{\pgfqpoint{5.379591in}{3.133666in}}%
\pgfpathlineto{\pgfqpoint{5.382535in}{3.131879in}}%
\pgfpathlineto{\pgfqpoint{5.386816in}{3.125967in}}%
\pgfpathlineto{\pgfqpoint{5.393239in}{3.117520in}}%
\pgfpathlineto{\pgfqpoint{5.396183in}{3.116756in}}%
\pgfpathlineto{\pgfqpoint{5.399126in}{3.118556in}}%
\pgfpathlineto{\pgfqpoint{5.403408in}{3.124478in}}%
\pgfpathlineto{\pgfqpoint{5.409831in}{3.132910in}}%
\pgfpathlineto{\pgfqpoint{5.412774in}{3.133660in}}%
\pgfpathlineto{\pgfqpoint{5.415718in}{3.131846in}}%
\pgfpathlineto{\pgfqpoint{5.420000in}{3.125914in}}%
\pgfpathlineto{\pgfqpoint{5.426422in}{3.117497in}}%
\pgfpathlineto{\pgfqpoint{5.429366in}{3.116762in}}%
\pgfpathlineto{\pgfqpoint{5.432310in}{3.118590in}}%
\pgfpathlineto{\pgfqpoint{5.436592in}{3.124531in}}%
\pgfpathlineto{\pgfqpoint{5.443014in}{3.132933in}}%
\pgfpathlineto{\pgfqpoint{5.445958in}{3.133653in}}%
\pgfpathlineto{\pgfqpoint{5.448901in}{3.131813in}}%
\pgfpathlineto{\pgfqpoint{5.453183in}{3.125861in}}%
\pgfpathlineto{\pgfqpoint{5.459606in}{3.117475in}}%
\pgfpathlineto{\pgfqpoint{5.462549in}{3.116769in}}%
\pgfpathlineto{\pgfqpoint{5.465493in}{3.118623in}}%
\pgfpathlineto{\pgfqpoint{5.470042in}{3.125015in}}%
\pgfpathlineto{\pgfqpoint{5.476197in}{3.132955in}}%
\pgfpathlineto{\pgfqpoint{5.479141in}{3.133647in}}%
\pgfpathlineto{\pgfqpoint{5.482085in}{3.131779in}}%
\pgfpathlineto{\pgfqpoint{5.486634in}{3.125377in}}%
\pgfpathlineto{\pgfqpoint{5.492789in}{3.117453in}}%
\pgfpathlineto{\pgfqpoint{5.495733in}{3.116776in}}%
\pgfpathlineto{\pgfqpoint{5.498677in}{3.118657in}}%
\pgfpathlineto{\pgfqpoint{5.503226in}{3.125069in}}%
\pgfpathlineto{\pgfqpoint{5.509381in}{3.132977in}}%
\pgfpathlineto{\pgfqpoint{5.512325in}{3.133639in}}%
\pgfpathlineto{\pgfqpoint{5.515268in}{3.131745in}}%
\pgfpathlineto{\pgfqpoint{5.519818in}{3.125323in}}%
\pgfpathlineto{\pgfqpoint{5.525973in}{3.117431in}}%
\pgfpathlineto{\pgfqpoint{5.528916in}{3.116783in}}%
\pgfpathlineto{\pgfqpoint{5.531860in}{3.118691in}}%
\pgfpathlineto{\pgfqpoint{5.536409in}{3.125122in}}%
\pgfpathlineto{\pgfqpoint{5.542564in}{3.132999in}}%
\pgfpathlineto{\pgfqpoint{5.545508in}{3.133632in}}%
\pgfpathlineto{\pgfqpoint{5.548452in}{3.131710in}}%
\pgfpathlineto{\pgfqpoint{5.553001in}{3.125270in}}%
\pgfpathlineto{\pgfqpoint{5.559156in}{3.117409in}}%
\pgfpathlineto{\pgfqpoint{5.562100in}{3.116791in}}%
\pgfpathlineto{\pgfqpoint{5.565043in}{3.118726in}}%
\pgfpathlineto{\pgfqpoint{5.569593in}{3.125175in}}%
\pgfpathlineto{\pgfqpoint{5.575748in}{3.133020in}}%
\pgfpathlineto{\pgfqpoint{5.578691in}{3.133624in}}%
\pgfpathlineto{\pgfqpoint{5.581635in}{3.131676in}}%
\pgfpathlineto{\pgfqpoint{5.586184in}{3.125217in}}%
\pgfpathlineto{\pgfqpoint{5.592339in}{3.117388in}}%
\pgfpathlineto{\pgfqpoint{5.595283in}{3.116799in}}%
\pgfpathlineto{\pgfqpoint{5.598227in}{3.118760in}}%
\pgfpathlineto{\pgfqpoint{5.602776in}{3.125229in}}%
\pgfpathlineto{\pgfqpoint{5.608663in}{3.132862in}}%
\pgfpathlineto{\pgfqpoint{5.611607in}{3.133672in}}%
\pgfpathlineto{\pgfqpoint{5.614283in}{3.132172in}}%
\pgfpathlineto{\pgfqpoint{5.618297in}{3.126879in}}%
\pgfpathlineto{\pgfqpoint{5.625523in}{3.117367in}}%
\pgfpathlineto{\pgfqpoint{5.628466in}{3.116807in}}%
\pgfpathlineto{\pgfqpoint{5.631410in}{3.118795in}}%
\pgfpathlineto{\pgfqpoint{5.635959in}{3.125282in}}%
\pgfpathlineto{\pgfqpoint{5.641847in}{3.132886in}}%
\pgfpathlineto{\pgfqpoint{5.644790in}{3.133666in}}%
\pgfpathlineto{\pgfqpoint{5.647734in}{3.131882in}}%
\pgfpathlineto{\pgfqpoint{5.652016in}{3.125972in}}%
\pgfpathlineto{\pgfqpoint{5.658438in}{3.117522in}}%
\pgfpathlineto{\pgfqpoint{5.661382in}{3.116755in}}%
\pgfpathlineto{\pgfqpoint{5.664326in}{3.118553in}}%
\pgfpathlineto{\pgfqpoint{5.668607in}{3.124473in}}%
\pgfpathlineto{\pgfqpoint{5.675030in}{3.132908in}}%
\pgfpathlineto{\pgfqpoint{5.677974in}{3.133660in}}%
\pgfpathlineto{\pgfqpoint{5.680917in}{3.131849in}}%
\pgfpathlineto{\pgfqpoint{5.685199in}{3.125919in}}%
\pgfpathlineto{\pgfqpoint{5.691622in}{3.117499in}}%
\pgfpathlineto{\pgfqpoint{5.694565in}{3.116762in}}%
\pgfpathlineto{\pgfqpoint{5.697509in}{3.118586in}}%
\pgfpathlineto{\pgfqpoint{5.701791in}{3.124526in}}%
\pgfpathlineto{\pgfqpoint{5.708213in}{3.132931in}}%
\pgfpathlineto{\pgfqpoint{5.711157in}{3.133654in}}%
\pgfpathlineto{\pgfqpoint{5.714101in}{3.131816in}}%
\pgfpathlineto{\pgfqpoint{5.718383in}{3.125866in}}%
\pgfpathlineto{\pgfqpoint{5.724805in}{3.117477in}}%
\pgfpathlineto{\pgfqpoint{5.727749in}{3.116768in}}%
\pgfpathlineto{\pgfqpoint{5.730693in}{3.118620in}}%
\pgfpathlineto{\pgfqpoint{5.735242in}{3.125011in}}%
\pgfpathlineto{\pgfqpoint{5.741397in}{3.132953in}}%
\pgfpathlineto{\pgfqpoint{5.744340in}{3.133647in}}%
\pgfpathlineto{\pgfqpoint{5.747284in}{3.131782in}}%
\pgfpathlineto{\pgfqpoint{5.751834in}{3.125382in}}%
\pgfpathlineto{\pgfqpoint{5.757988in}{3.117455in}}%
\pgfpathlineto{\pgfqpoint{5.760932in}{3.116775in}}%
\pgfpathlineto{\pgfqpoint{5.763876in}{3.118654in}}%
\pgfpathlineto{\pgfqpoint{5.768425in}{3.125064in}}%
\pgfpathlineto{\pgfqpoint{5.774580in}{3.132975in}}%
\pgfpathlineto{\pgfqpoint{5.777524in}{3.133640in}}%
\pgfpathlineto{\pgfqpoint{5.780468in}{3.131748in}}%
\pgfpathlineto{\pgfqpoint{5.785017in}{3.125328in}}%
\pgfpathlineto{\pgfqpoint{5.791172in}{3.117433in}}%
\pgfpathlineto{\pgfqpoint{5.794116in}{3.116783in}}%
\pgfpathlineto{\pgfqpoint{5.797059in}{3.118688in}}%
\pgfpathlineto{\pgfqpoint{5.801609in}{3.125117in}}%
\pgfpathlineto{\pgfqpoint{5.807764in}{3.132997in}}%
\pgfpathlineto{\pgfqpoint{5.810707in}{3.133633in}}%
\pgfpathlineto{\pgfqpoint{5.813651in}{3.131714in}}%
\pgfpathlineto{\pgfqpoint{5.818200in}{3.125275in}}%
\pgfpathlineto{\pgfqpoint{5.824355in}{3.117411in}}%
\pgfpathlineto{\pgfqpoint{5.827299in}{3.116790in}}%
\pgfpathlineto{\pgfqpoint{5.830243in}{3.118723in}}%
\pgfpathlineto{\pgfqpoint{5.834792in}{3.125171in}}%
\pgfpathlineto{\pgfqpoint{5.840947in}{3.133018in}}%
\pgfpathlineto{\pgfqpoint{5.843891in}{3.133625in}}%
\pgfpathlineto{\pgfqpoint{5.846834in}{3.131679in}}%
\pgfpathlineto{\pgfqpoint{5.851384in}{3.125222in}}%
\pgfpathlineto{\pgfqpoint{5.857539in}{3.117390in}}%
\pgfpathlineto{\pgfqpoint{5.860482in}{3.116798in}}%
\pgfpathlineto{\pgfqpoint{5.863426in}{3.118757in}}%
\pgfpathlineto{\pgfqpoint{5.867975in}{3.125224in}}%
\pgfpathlineto{\pgfqpoint{5.874130in}{3.133039in}}%
\pgfpathlineto{\pgfqpoint{5.877074in}{3.133617in}}%
\pgfpathlineto{\pgfqpoint{5.880018in}{3.131644in}}%
\pgfpathlineto{\pgfqpoint{5.884567in}{3.125168in}}%
\pgfpathlineto{\pgfqpoint{5.890454in}{3.117547in}}%
\pgfpathlineto{\pgfqpoint{5.893398in}{3.116749in}}%
\pgfpathlineto{\pgfqpoint{5.896074in}{3.118259in}}%
\pgfpathlineto{\pgfqpoint{5.900088in}{3.123561in}}%
\pgfpathlineto{\pgfqpoint{5.907314in}{3.133060in}}%
\pgfpathlineto{\pgfqpoint{5.910257in}{3.133608in}}%
\pgfpathlineto{\pgfqpoint{5.913201in}{3.131609in}}%
\pgfpathlineto{\pgfqpoint{5.917750in}{3.125115in}}%
\pgfpathlineto{\pgfqpoint{5.923638in}{3.117524in}}%
\pgfpathlineto{\pgfqpoint{5.926581in}{3.116755in}}%
\pgfpathlineto{\pgfqpoint{5.929525in}{3.118550in}}%
\pgfpathlineto{\pgfqpoint{5.933807in}{3.124468in}}%
\pgfpathlineto{\pgfqpoint{5.940229in}{3.132906in}}%
\pgfpathlineto{\pgfqpoint{5.943173in}{3.133661in}}%
\pgfpathlineto{\pgfqpoint{5.946117in}{3.131852in}}%
\pgfpathlineto{\pgfqpoint{5.950399in}{3.125924in}}%
\pgfpathlineto{\pgfqpoint{5.956821in}{3.117501in}}%
\pgfpathlineto{\pgfqpoint{5.959765in}{3.116761in}}%
\pgfpathlineto{\pgfqpoint{5.962708in}{3.118583in}}%
\pgfpathlineto{\pgfqpoint{5.966990in}{3.124522in}}%
\pgfpathlineto{\pgfqpoint{5.973413in}{3.132929in}}%
\pgfpathlineto{\pgfqpoint{5.976356in}{3.133655in}}%
\pgfpathlineto{\pgfqpoint{5.979300in}{3.131819in}}%
\pgfpathlineto{\pgfqpoint{5.983582in}{3.125871in}}%
\pgfpathlineto{\pgfqpoint{5.990004in}{3.117479in}}%
\pgfpathlineto{\pgfqpoint{5.992948in}{3.116768in}}%
\pgfpathlineto{\pgfqpoint{5.995892in}{3.118617in}}%
\pgfpathlineto{\pgfqpoint{6.000441in}{3.125006in}}%
\pgfpathlineto{\pgfqpoint{6.006596in}{3.132951in}}%
\pgfpathlineto{\pgfqpoint{6.009540in}{3.133648in}}%
\pgfpathlineto{\pgfqpoint{6.012484in}{3.131785in}}%
\pgfpathlineto{\pgfqpoint{6.017033in}{3.125386in}}%
\pgfpathlineto{\pgfqpoint{6.023188in}{3.117457in}}%
\pgfpathlineto{\pgfqpoint{6.026132in}{3.116775in}}%
\pgfpathlineto{\pgfqpoint{6.029075in}{3.118651in}}%
\pgfpathlineto{\pgfqpoint{6.033625in}{3.125059in}}%
\pgfpathlineto{\pgfqpoint{6.039780in}{3.132973in}}%
\pgfpathlineto{\pgfqpoint{6.042723in}{3.133641in}}%
\pgfpathlineto{\pgfqpoint{6.045667in}{3.131751in}}%
\pgfpathlineto{\pgfqpoint{6.050216in}{3.125333in}}%
\pgfpathlineto{\pgfqpoint{6.056371in}{3.117435in}}%
\pgfpathlineto{\pgfqpoint{6.059315in}{3.116782in}}%
\pgfpathlineto{\pgfqpoint{6.062259in}{3.118685in}}%
\pgfpathlineto{\pgfqpoint{6.066808in}{3.125112in}}%
\pgfpathlineto{\pgfqpoint{6.072963in}{3.132995in}}%
\pgfpathlineto{\pgfqpoint{6.075907in}{3.133633in}}%
\pgfpathlineto{\pgfqpoint{6.078850in}{3.131717in}}%
\pgfpathlineto{\pgfqpoint{6.083400in}{3.125280in}}%
\pgfpathlineto{\pgfqpoint{6.089555in}{3.117413in}}%
\pgfpathlineto{\pgfqpoint{6.092498in}{3.116789in}}%
\pgfpathlineto{\pgfqpoint{6.095442in}{3.118719in}}%
\pgfpathlineto{\pgfqpoint{6.099991in}{3.125166in}}%
\pgfpathlineto{\pgfqpoint{6.106146in}{3.133016in}}%
\pgfpathlineto{\pgfqpoint{6.109090in}{3.133626in}}%
\pgfpathlineto{\pgfqpoint{6.112034in}{3.131682in}}%
\pgfpathlineto{\pgfqpoint{6.116583in}{3.125226in}}%
\pgfpathlineto{\pgfqpoint{6.122738in}{3.117392in}}%
\pgfpathlineto{\pgfqpoint{6.125682in}{3.116797in}}%
\pgfpathlineto{\pgfqpoint{6.128625in}{3.118754in}}%
\pgfpathlineto{\pgfqpoint{6.133175in}{3.125219in}}%
\pgfpathlineto{\pgfqpoint{6.139330in}{3.133037in}}%
\pgfpathlineto{\pgfqpoint{6.142273in}{3.133617in}}%
\pgfpathlineto{\pgfqpoint{6.145217in}{3.131647in}}%
\pgfpathlineto{\pgfqpoint{6.149766in}{3.125173in}}%
\pgfpathlineto{\pgfqpoint{6.155654in}{3.117549in}}%
\pgfpathlineto{\pgfqpoint{6.158597in}{3.116748in}}%
\pgfpathlineto{\pgfqpoint{6.161273in}{3.118256in}}%
\pgfpathlineto{\pgfqpoint{6.165288in}{3.123557in}}%
\pgfpathlineto{\pgfqpoint{6.172513in}{3.133058in}}%
\pgfpathlineto{\pgfqpoint{6.175457in}{3.133609in}}%
\pgfpathlineto{\pgfqpoint{6.178400in}{3.131612in}}%
\pgfpathlineto{\pgfqpoint{6.182950in}{3.125120in}}%
\pgfpathlineto{\pgfqpoint{6.188837in}{3.117526in}}%
\pgfpathlineto{\pgfqpoint{6.191781in}{3.116754in}}%
\pgfpathlineto{\pgfqpoint{6.194724in}{3.118547in}}%
\pgfpathlineto{\pgfqpoint{6.199006in}{3.124464in}}%
\pgfpathlineto{\pgfqpoint{6.205429in}{3.132904in}}%
\pgfpathlineto{\pgfqpoint{6.208372in}{3.133661in}}%
\pgfpathlineto{\pgfqpoint{6.211316in}{3.131855in}}%
\pgfpathlineto{\pgfqpoint{6.215598in}{3.125929in}}%
\pgfpathlineto{\pgfqpoint{6.222020in}{3.117503in}}%
\pgfpathlineto{\pgfqpoint{6.224964in}{3.116761in}}%
\pgfpathlineto{\pgfqpoint{6.227908in}{3.118580in}}%
\pgfpathlineto{\pgfqpoint{6.232190in}{3.124517in}}%
\pgfpathlineto{\pgfqpoint{6.238612in}{3.132927in}}%
\pgfpathlineto{\pgfqpoint{6.241556in}{3.133655in}}%
\pgfpathlineto{\pgfqpoint{6.244500in}{3.131822in}}%
\pgfpathlineto{\pgfqpoint{6.248781in}{3.125876in}}%
\pgfpathlineto{\pgfqpoint{6.255204in}{3.117481in}}%
\pgfpathlineto{\pgfqpoint{6.258148in}{3.116767in}}%
\pgfpathlineto{\pgfqpoint{6.261091in}{3.118614in}}%
\pgfpathlineto{\pgfqpoint{6.265641in}{3.125001in}}%
\pgfpathlineto{\pgfqpoint{6.271796in}{3.132949in}}%
\pgfpathlineto{\pgfqpoint{6.274739in}{3.133648in}}%
\pgfpathlineto{\pgfqpoint{6.277683in}{3.131788in}}%
\pgfpathlineto{\pgfqpoint{6.282232in}{3.125391in}}%
\pgfpathlineto{\pgfqpoint{6.288387in}{3.117459in}}%
\pgfpathlineto{\pgfqpoint{6.291331in}{3.116774in}}%
\pgfpathlineto{\pgfqpoint{6.294275in}{3.118648in}}%
\pgfpathlineto{\pgfqpoint{6.298824in}{3.125054in}}%
\pgfpathlineto{\pgfqpoint{6.304979in}{3.132971in}}%
\pgfpathlineto{\pgfqpoint{6.307923in}{3.133641in}}%
\pgfpathlineto{\pgfqpoint{6.310866in}{3.131754in}}%
\pgfpathlineto{\pgfqpoint{6.315416in}{3.125338in}}%
\pgfpathlineto{\pgfqpoint{6.321571in}{3.117437in}}%
\pgfpathlineto{\pgfqpoint{6.324514in}{3.116781in}}%
\pgfpathlineto{\pgfqpoint{6.327458in}{3.118682in}}%
\pgfpathlineto{\pgfqpoint{6.332007in}{3.125108in}}%
\pgfpathlineto{\pgfqpoint{6.338162in}{3.132993in}}%
\pgfpathlineto{\pgfqpoint{6.341106in}{3.133634in}}%
\pgfpathlineto{\pgfqpoint{6.344050in}{3.131720in}}%
\pgfpathlineto{\pgfqpoint{6.348599in}{3.125285in}}%
\pgfpathlineto{\pgfqpoint{6.354754in}{3.117415in}}%
\pgfpathlineto{\pgfqpoint{6.357698in}{3.116789in}}%
\pgfpathlineto{\pgfqpoint{6.360641in}{3.118716in}}%
\pgfpathlineto{\pgfqpoint{6.365191in}{3.125161in}}%
\pgfpathlineto{\pgfqpoint{6.371346in}{3.133014in}}%
\pgfpathlineto{\pgfqpoint{6.374289in}{3.133626in}}%
\pgfpathlineto{\pgfqpoint{6.377233in}{3.131685in}}%
\pgfpathlineto{\pgfqpoint{6.381782in}{3.125231in}}%
\pgfpathlineto{\pgfqpoint{6.387937in}{3.117394in}}%
\pgfpathlineto{\pgfqpoint{6.390881in}{3.116797in}}%
\pgfpathlineto{\pgfqpoint{6.393825in}{3.118751in}}%
\pgfpathlineto{\pgfqpoint{6.398374in}{3.125214in}}%
\pgfpathlineto{\pgfqpoint{6.404529in}{3.133035in}}%
\pgfpathlineto{\pgfqpoint{6.407473in}{3.133618in}}%
\pgfpathlineto{\pgfqpoint{6.410416in}{3.131651in}}%
\pgfpathlineto{\pgfqpoint{6.414966in}{3.125178in}}%
\pgfpathlineto{\pgfqpoint{6.420853in}{3.117551in}}%
\pgfpathlineto{\pgfqpoint{6.423797in}{3.116748in}}%
\pgfpathlineto{\pgfqpoint{6.426473in}{3.118254in}}%
\pgfpathlineto{\pgfqpoint{6.430487in}{3.123552in}}%
\pgfpathlineto{\pgfqpoint{6.437712in}{3.133056in}}%
\pgfpathlineto{\pgfqpoint{6.440656in}{3.133610in}}%
\pgfpathlineto{\pgfqpoint{6.443600in}{3.131616in}}%
\pgfpathlineto{\pgfqpoint{6.448149in}{3.125125in}}%
\pgfpathlineto{\pgfqpoint{6.454036in}{3.117528in}}%
\pgfpathlineto{\pgfqpoint{6.456980in}{3.116754in}}%
\pgfpathlineto{\pgfqpoint{6.459924in}{3.118544in}}%
\pgfpathlineto{\pgfqpoint{6.464206in}{3.124459in}}%
\pgfpathlineto{\pgfqpoint{6.470628in}{3.132902in}}%
\pgfpathlineto{\pgfqpoint{6.473572in}{3.133662in}}%
\pgfpathlineto{\pgfqpoint{6.476515in}{3.131858in}}%
\pgfpathlineto{\pgfqpoint{6.480797in}{3.125934in}}%
\pgfpathlineto{\pgfqpoint{6.487220in}{3.117505in}}%
\pgfpathlineto{\pgfqpoint{6.490163in}{3.116760in}}%
\pgfpathlineto{\pgfqpoint{6.493107in}{3.118577in}}%
\pgfpathlineto{\pgfqpoint{6.497389in}{3.124512in}}%
\pgfpathlineto{\pgfqpoint{6.503811in}{3.132925in}}%
\pgfpathlineto{\pgfqpoint{6.506755in}{3.133656in}}%
\pgfpathlineto{\pgfqpoint{6.509699in}{3.131825in}}%
\pgfpathlineto{\pgfqpoint{6.513981in}{3.125880in}}%
\pgfpathlineto{\pgfqpoint{6.520403in}{3.117483in}}%
\pgfpathlineto{\pgfqpoint{6.523347in}{3.116766in}}%
\pgfpathlineto{\pgfqpoint{6.526291in}{3.118611in}}%
\pgfpathlineto{\pgfqpoint{6.530840in}{3.124996in}}%
\pgfpathlineto{\pgfqpoint{6.536995in}{3.132947in}}%
\pgfpathlineto{\pgfqpoint{6.539939in}{3.133649in}}%
\pgfpathlineto{\pgfqpoint{6.542882in}{3.131791in}}%
\pgfpathlineto{\pgfqpoint{6.547432in}{3.125396in}}%
\pgfpathlineto{\pgfqpoint{6.553587in}{3.117461in}}%
\pgfpathlineto{\pgfqpoint{6.556530in}{3.116773in}}%
\pgfpathlineto{\pgfqpoint{6.559474in}{3.118645in}}%
\pgfpathlineto{\pgfqpoint{6.564023in}{3.125049in}}%
\pgfpathlineto{\pgfqpoint{6.570178in}{3.132969in}}%
\pgfpathlineto{\pgfqpoint{6.573122in}{3.133642in}}%
\pgfpathlineto{\pgfqpoint{6.576066in}{3.131757in}}%
\pgfpathlineto{\pgfqpoint{6.580615in}{3.125343in}}%
\pgfpathlineto{\pgfqpoint{6.586770in}{3.117439in}}%
\pgfpathlineto{\pgfqpoint{6.589714in}{3.116780in}}%
\pgfpathlineto{\pgfqpoint{6.592657in}{3.118679in}}%
\pgfpathlineto{\pgfqpoint{6.597207in}{3.125103in}}%
\pgfpathlineto{\pgfqpoint{6.603362in}{3.132991in}}%
\pgfpathlineto{\pgfqpoint{6.606305in}{3.133635in}}%
\pgfpathlineto{\pgfqpoint{6.609249in}{3.131723in}}%
\pgfpathlineto{\pgfqpoint{6.613798in}{3.125289in}}%
\pgfpathlineto{\pgfqpoint{6.619953in}{3.117417in}}%
\pgfpathlineto{\pgfqpoint{6.622897in}{3.116788in}}%
\pgfpathlineto{\pgfqpoint{6.625841in}{3.118713in}}%
\pgfpathlineto{\pgfqpoint{6.630390in}{3.125156in}}%
\pgfpathlineto{\pgfqpoint{6.636545in}{3.133012in}}%
\pgfpathlineto{\pgfqpoint{6.639489in}{3.133627in}}%
\pgfpathlineto{\pgfqpoint{6.642432in}{3.131688in}}%
\pgfpathlineto{\pgfqpoint{6.646982in}{3.125236in}}%
\pgfpathlineto{\pgfqpoint{6.653137in}{3.117396in}}%
\pgfpathlineto{\pgfqpoint{6.656080in}{3.116796in}}%
\pgfpathlineto{\pgfqpoint{6.659024in}{3.118748in}}%
\pgfpathlineto{\pgfqpoint{6.663306in}{3.124778in}}%
\pgfpathlineto{\pgfqpoint{6.663306in}{3.124778in}}%
\pgfusepath{stroke}%
\end{pgfscope}%
\begin{pgfscope}%
\pgfpathrectangle{\pgfqpoint{0.467797in}{2.292089in}}{\pgfqpoint{6.490533in}{1.666241in}}%
\pgfusepath{clip}%
\pgfsetrectcap%
\pgfsetroundjoin%
\pgfsetlinewidth{1.505625pt}%
\definecolor{currentstroke}{rgb}{0.549020,0.337255,0.294118}%
\pgfsetstrokecolor{currentstroke}%
\pgfsetdash{}{0pt}%
\pgfpathmoveto{\pgfqpoint{0.762821in}{3.125209in}}%
\pgfpathlineto{\pgfqpoint{0.768708in}{3.132775in}}%
\pgfpathlineto{\pgfqpoint{0.771652in}{3.133450in}}%
\pgfpathlineto{\pgfqpoint{0.774596in}{3.131515in}}%
\pgfpathlineto{\pgfqpoint{0.779145in}{3.125027in}}%
\pgfpathlineto{\pgfqpoint{0.785032in}{3.117569in}}%
\pgfpathlineto{\pgfqpoint{0.787976in}{3.116996in}}%
\pgfpathlineto{\pgfqpoint{0.790920in}{3.119024in}}%
\pgfpathlineto{\pgfqpoint{0.795737in}{3.126005in}}%
\pgfpathlineto{\pgfqpoint{0.801356in}{3.132920in}}%
\pgfpathlineto{\pgfqpoint{0.804300in}{3.133391in}}%
\pgfpathlineto{\pgfqpoint{0.807244in}{3.131271in}}%
\pgfpathlineto{\pgfqpoint{0.812061in}{3.124232in}}%
\pgfpathlineto{\pgfqpoint{0.817413in}{3.117596in}}%
\pgfpathlineto{\pgfqpoint{0.820357in}{3.116986in}}%
\pgfpathlineto{\pgfqpoint{0.823300in}{3.118980in}}%
\pgfpathlineto{\pgfqpoint{0.827850in}{3.125508in}}%
\pgfpathlineto{\pgfqpoint{0.833737in}{3.132895in}}%
\pgfpathlineto{\pgfqpoint{0.836681in}{3.133403in}}%
\pgfpathlineto{\pgfqpoint{0.839624in}{3.131316in}}%
\pgfpathlineto{\pgfqpoint{0.844441in}{3.124298in}}%
\pgfpathlineto{\pgfqpoint{0.849793in}{3.117623in}}%
\pgfpathlineto{\pgfqpoint{0.852737in}{3.116976in}}%
\pgfpathlineto{\pgfqpoint{0.855681in}{3.118936in}}%
\pgfpathlineto{\pgfqpoint{0.860230in}{3.125442in}}%
\pgfpathlineto{\pgfqpoint{0.866118in}{3.132869in}}%
\pgfpathlineto{\pgfqpoint{0.869061in}{3.133414in}}%
\pgfpathlineto{\pgfqpoint{0.872005in}{3.131361in}}%
\pgfpathlineto{\pgfqpoint{0.876822in}{3.124364in}}%
\pgfpathlineto{\pgfqpoint{0.882174in}{3.117651in}}%
\pgfpathlineto{\pgfqpoint{0.885118in}{3.116966in}}%
\pgfpathlineto{\pgfqpoint{0.888061in}{3.118893in}}%
\pgfpathlineto{\pgfqpoint{0.892611in}{3.125375in}}%
\pgfpathlineto{\pgfqpoint{0.898498in}{3.132843in}}%
\pgfpathlineto{\pgfqpoint{0.901442in}{3.133425in}}%
\pgfpathlineto{\pgfqpoint{0.904385in}{3.131406in}}%
\pgfpathlineto{\pgfqpoint{0.909202in}{3.124430in}}%
\pgfpathlineto{\pgfqpoint{0.914822in}{3.117505in}}%
\pgfpathlineto{\pgfqpoint{0.917766in}{3.117025in}}%
\pgfpathlineto{\pgfqpoint{0.920710in}{3.119137in}}%
\pgfpathlineto{\pgfqpoint{0.925526in}{3.126170in}}%
\pgfpathlineto{\pgfqpoint{0.930879in}{3.132816in}}%
\pgfpathlineto{\pgfqpoint{0.933822in}{3.133436in}}%
\pgfpathlineto{\pgfqpoint{0.936766in}{3.131450in}}%
\pgfpathlineto{\pgfqpoint{0.941315in}{3.124927in}}%
\pgfpathlineto{\pgfqpoint{0.947203in}{3.117530in}}%
\pgfpathlineto{\pgfqpoint{0.950146in}{3.117013in}}%
\pgfpathlineto{\pgfqpoint{0.953090in}{3.119091in}}%
\pgfpathlineto{\pgfqpoint{0.957907in}{3.126104in}}%
\pgfpathlineto{\pgfqpoint{0.963259in}{3.132789in}}%
\pgfpathlineto{\pgfqpoint{0.966203in}{3.133445in}}%
\pgfpathlineto{\pgfqpoint{0.969147in}{3.131494in}}%
\pgfpathlineto{\pgfqpoint{0.973696in}{3.124994in}}%
\pgfpathlineto{\pgfqpoint{0.979583in}{3.117556in}}%
\pgfpathlineto{\pgfqpoint{0.982527in}{3.117002in}}%
\pgfpathlineto{\pgfqpoint{0.985471in}{3.119047in}}%
\pgfpathlineto{\pgfqpoint{0.990288in}{3.126038in}}%
\pgfpathlineto{\pgfqpoint{0.995640in}{3.132761in}}%
\pgfpathlineto{\pgfqpoint{0.998583in}{3.133455in}}%
\pgfpathlineto{\pgfqpoint{1.001259in}{3.131809in}}%
\pgfpathlineto{\pgfqpoint{1.005541in}{3.125922in}}%
\pgfpathlineto{\pgfqpoint{1.011964in}{3.117583in}}%
\pgfpathlineto{\pgfqpoint{1.014907in}{3.116991in}}%
\pgfpathlineto{\pgfqpoint{1.017851in}{3.119002in}}%
\pgfpathlineto{\pgfqpoint{1.022668in}{3.125972in}}%
\pgfpathlineto{\pgfqpoint{1.028288in}{3.132907in}}%
\pgfpathlineto{\pgfqpoint{1.031232in}{3.133397in}}%
\pgfpathlineto{\pgfqpoint{1.034175in}{3.131294in}}%
\pgfpathlineto{\pgfqpoint{1.038992in}{3.124265in}}%
\pgfpathlineto{\pgfqpoint{1.044344in}{3.117609in}}%
\pgfpathlineto{\pgfqpoint{1.047288in}{3.116981in}}%
\pgfpathlineto{\pgfqpoint{1.050232in}{3.118958in}}%
\pgfpathlineto{\pgfqpoint{1.054781in}{3.125475in}}%
\pgfpathlineto{\pgfqpoint{1.060668in}{3.132882in}}%
\pgfpathlineto{\pgfqpoint{1.063612in}{3.133409in}}%
\pgfpathlineto{\pgfqpoint{1.066556in}{3.131339in}}%
\pgfpathlineto{\pgfqpoint{1.071373in}{3.124331in}}%
\pgfpathlineto{\pgfqpoint{1.076725in}{3.117637in}}%
\pgfpathlineto{\pgfqpoint{1.079669in}{3.116971in}}%
\pgfpathlineto{\pgfqpoint{1.082612in}{3.118914in}}%
\pgfpathlineto{\pgfqpoint{1.087162in}{3.125409in}}%
\pgfpathlineto{\pgfqpoint{1.093049in}{3.132856in}}%
\pgfpathlineto{\pgfqpoint{1.095993in}{3.133420in}}%
\pgfpathlineto{\pgfqpoint{1.098936in}{3.131383in}}%
\pgfpathlineto{\pgfqpoint{1.103753in}{3.124397in}}%
\pgfpathlineto{\pgfqpoint{1.109373in}{3.117493in}}%
\pgfpathlineto{\pgfqpoint{1.112317in}{3.117031in}}%
\pgfpathlineto{\pgfqpoint{1.115260in}{3.119159in}}%
\pgfpathlineto{\pgfqpoint{1.120077in}{3.126203in}}%
\pgfpathlineto{\pgfqpoint{1.125430in}{3.132830in}}%
\pgfpathlineto{\pgfqpoint{1.128373in}{3.133430in}}%
\pgfpathlineto{\pgfqpoint{1.131317in}{3.131428in}}%
\pgfpathlineto{\pgfqpoint{1.136134in}{3.124463in}}%
\pgfpathlineto{\pgfqpoint{1.141754in}{3.117518in}}%
\pgfpathlineto{\pgfqpoint{1.144697in}{3.117019in}}%
\pgfpathlineto{\pgfqpoint{1.147641in}{3.119114in}}%
\pgfpathlineto{\pgfqpoint{1.152458in}{3.126137in}}%
\pgfpathlineto{\pgfqpoint{1.157810in}{3.132803in}}%
\pgfpathlineto{\pgfqpoint{1.160754in}{3.133441in}}%
\pgfpathlineto{\pgfqpoint{1.163697in}{3.131472in}}%
\pgfpathlineto{\pgfqpoint{1.168247in}{3.124960in}}%
\pgfpathlineto{\pgfqpoint{1.174134in}{3.117543in}}%
\pgfpathlineto{\pgfqpoint{1.177078in}{3.117007in}}%
\pgfpathlineto{\pgfqpoint{1.180022in}{3.119069in}}%
\pgfpathlineto{\pgfqpoint{1.184838in}{3.126071in}}%
\pgfpathlineto{\pgfqpoint{1.190191in}{3.132775in}}%
\pgfpathlineto{\pgfqpoint{1.193134in}{3.133450in}}%
\pgfpathlineto{\pgfqpoint{1.196078in}{3.131515in}}%
\pgfpathlineto{\pgfqpoint{1.200627in}{3.125027in}}%
\pgfpathlineto{\pgfqpoint{1.206515in}{3.117569in}}%
\pgfpathlineto{\pgfqpoint{1.209458in}{3.116996in}}%
\pgfpathlineto{\pgfqpoint{1.212402in}{3.119024in}}%
\pgfpathlineto{\pgfqpoint{1.217219in}{3.126005in}}%
\pgfpathlineto{\pgfqpoint{1.222839in}{3.132920in}}%
\pgfpathlineto{\pgfqpoint{1.225782in}{3.133391in}}%
\pgfpathlineto{\pgfqpoint{1.228726in}{3.131271in}}%
\pgfpathlineto{\pgfqpoint{1.233543in}{3.124232in}}%
\pgfpathlineto{\pgfqpoint{1.238895in}{3.117596in}}%
\pgfpathlineto{\pgfqpoint{1.241839in}{3.116986in}}%
\pgfpathlineto{\pgfqpoint{1.244783in}{3.118980in}}%
\pgfpathlineto{\pgfqpoint{1.249332in}{3.125508in}}%
\pgfpathlineto{\pgfqpoint{1.255219in}{3.132895in}}%
\pgfpathlineto{\pgfqpoint{1.258163in}{3.133403in}}%
\pgfpathlineto{\pgfqpoint{1.261107in}{3.131316in}}%
\pgfpathlineto{\pgfqpoint{1.265924in}{3.124298in}}%
\pgfpathlineto{\pgfqpoint{1.271276in}{3.117623in}}%
\pgfpathlineto{\pgfqpoint{1.274219in}{3.116976in}}%
\pgfpathlineto{\pgfqpoint{1.277163in}{3.118936in}}%
\pgfpathlineto{\pgfqpoint{1.281712in}{3.125442in}}%
\pgfpathlineto{\pgfqpoint{1.287600in}{3.132869in}}%
\pgfpathlineto{\pgfqpoint{1.290544in}{3.133414in}}%
\pgfpathlineto{\pgfqpoint{1.293487in}{3.131361in}}%
\pgfpathlineto{\pgfqpoint{1.298304in}{3.124364in}}%
\pgfpathlineto{\pgfqpoint{1.303656in}{3.117651in}}%
\pgfpathlineto{\pgfqpoint{1.306600in}{3.116966in}}%
\pgfpathlineto{\pgfqpoint{1.309544in}{3.118893in}}%
\pgfpathlineto{\pgfqpoint{1.314093in}{3.125375in}}%
\pgfpathlineto{\pgfqpoint{1.319980in}{3.132843in}}%
\pgfpathlineto{\pgfqpoint{1.322924in}{3.133425in}}%
\pgfpathlineto{\pgfqpoint{1.325868in}{3.131406in}}%
\pgfpathlineto{\pgfqpoint{1.330685in}{3.124430in}}%
\pgfpathlineto{\pgfqpoint{1.336304in}{3.117505in}}%
\pgfpathlineto{\pgfqpoint{1.339248in}{3.117025in}}%
\pgfpathlineto{\pgfqpoint{1.342192in}{3.119137in}}%
\pgfpathlineto{\pgfqpoint{1.347009in}{3.126170in}}%
\pgfpathlineto{\pgfqpoint{1.352361in}{3.132816in}}%
\pgfpathlineto{\pgfqpoint{1.355305in}{3.133436in}}%
\pgfpathlineto{\pgfqpoint{1.358248in}{3.131450in}}%
\pgfpathlineto{\pgfqpoint{1.362798in}{3.124927in}}%
\pgfpathlineto{\pgfqpoint{1.368685in}{3.117530in}}%
\pgfpathlineto{\pgfqpoint{1.371629in}{3.117013in}}%
\pgfpathlineto{\pgfqpoint{1.374572in}{3.119091in}}%
\pgfpathlineto{\pgfqpoint{1.379389in}{3.126104in}}%
\pgfpathlineto{\pgfqpoint{1.384741in}{3.132789in}}%
\pgfpathlineto{\pgfqpoint{1.387685in}{3.133445in}}%
\pgfpathlineto{\pgfqpoint{1.390629in}{3.131494in}}%
\pgfpathlineto{\pgfqpoint{1.395178in}{3.124994in}}%
\pgfpathlineto{\pgfqpoint{1.401066in}{3.117556in}}%
\pgfpathlineto{\pgfqpoint{1.404009in}{3.117002in}}%
\pgfpathlineto{\pgfqpoint{1.406953in}{3.119047in}}%
\pgfpathlineto{\pgfqpoint{1.411770in}{3.126038in}}%
\pgfpathlineto{\pgfqpoint{1.417122in}{3.132761in}}%
\pgfpathlineto{\pgfqpoint{1.420066in}{3.133455in}}%
\pgfpathlineto{\pgfqpoint{1.422742in}{3.131809in}}%
\pgfpathlineto{\pgfqpoint{1.427024in}{3.125922in}}%
\pgfpathlineto{\pgfqpoint{1.433446in}{3.117583in}}%
\pgfpathlineto{\pgfqpoint{1.436390in}{3.116991in}}%
\pgfpathlineto{\pgfqpoint{1.439333in}{3.119002in}}%
\pgfpathlineto{\pgfqpoint{1.444150in}{3.125972in}}%
\pgfpathlineto{\pgfqpoint{1.449770in}{3.132907in}}%
\pgfpathlineto{\pgfqpoint{1.452714in}{3.133397in}}%
\pgfpathlineto{\pgfqpoint{1.455658in}{3.131294in}}%
\pgfpathlineto{\pgfqpoint{1.460474in}{3.124265in}}%
\pgfpathlineto{\pgfqpoint{1.465827in}{3.117609in}}%
\pgfpathlineto{\pgfqpoint{1.468770in}{3.116981in}}%
\pgfpathlineto{\pgfqpoint{1.471714in}{3.118958in}}%
\pgfpathlineto{\pgfqpoint{1.476263in}{3.125475in}}%
\pgfpathlineto{\pgfqpoint{1.482151in}{3.132882in}}%
\pgfpathlineto{\pgfqpoint{1.485094in}{3.133409in}}%
\pgfpathlineto{\pgfqpoint{1.488038in}{3.131339in}}%
\pgfpathlineto{\pgfqpoint{1.492855in}{3.124331in}}%
\pgfpathlineto{\pgfqpoint{1.498207in}{3.117637in}}%
\pgfpathlineto{\pgfqpoint{1.501151in}{3.116971in}}%
\pgfpathlineto{\pgfqpoint{1.504095in}{3.118914in}}%
\pgfpathlineto{\pgfqpoint{1.508644in}{3.125409in}}%
\pgfpathlineto{\pgfqpoint{1.514531in}{3.132856in}}%
\pgfpathlineto{\pgfqpoint{1.517475in}{3.133420in}}%
\pgfpathlineto{\pgfqpoint{1.520419in}{3.131383in}}%
\pgfpathlineto{\pgfqpoint{1.525236in}{3.124397in}}%
\pgfpathlineto{\pgfqpoint{1.530855in}{3.117493in}}%
\pgfpathlineto{\pgfqpoint{1.533799in}{3.117031in}}%
\pgfpathlineto{\pgfqpoint{1.536743in}{3.119159in}}%
\pgfpathlineto{\pgfqpoint{1.541560in}{3.126203in}}%
\pgfpathlineto{\pgfqpoint{1.546912in}{3.132830in}}%
\pgfpathlineto{\pgfqpoint{1.549856in}{3.133430in}}%
\pgfpathlineto{\pgfqpoint{1.552799in}{3.131428in}}%
\pgfpathlineto{\pgfqpoint{1.557616in}{3.124463in}}%
\pgfpathlineto{\pgfqpoint{1.563236in}{3.117518in}}%
\pgfpathlineto{\pgfqpoint{1.566180in}{3.117019in}}%
\pgfpathlineto{\pgfqpoint{1.569123in}{3.119114in}}%
\pgfpathlineto{\pgfqpoint{1.573940in}{3.126137in}}%
\pgfpathlineto{\pgfqpoint{1.579292in}{3.132803in}}%
\pgfpathlineto{\pgfqpoint{1.582236in}{3.133441in}}%
\pgfpathlineto{\pgfqpoint{1.585180in}{3.131472in}}%
\pgfpathlineto{\pgfqpoint{1.589729in}{3.124960in}}%
\pgfpathlineto{\pgfqpoint{1.595616in}{3.117543in}}%
\pgfpathlineto{\pgfqpoint{1.598560in}{3.117007in}}%
\pgfpathlineto{\pgfqpoint{1.601504in}{3.119069in}}%
\pgfpathlineto{\pgfqpoint{1.606321in}{3.126071in}}%
\pgfpathlineto{\pgfqpoint{1.611673in}{3.132775in}}%
\pgfpathlineto{\pgfqpoint{1.614617in}{3.133450in}}%
\pgfpathlineto{\pgfqpoint{1.617560in}{3.131515in}}%
\pgfpathlineto{\pgfqpoint{1.622110in}{3.125027in}}%
\pgfpathlineto{\pgfqpoint{1.627997in}{3.117569in}}%
\pgfpathlineto{\pgfqpoint{1.630941in}{3.116996in}}%
\pgfpathlineto{\pgfqpoint{1.633884in}{3.119024in}}%
\pgfpathlineto{\pgfqpoint{1.638701in}{3.126005in}}%
\pgfpathlineto{\pgfqpoint{1.644321in}{3.132920in}}%
\pgfpathlineto{\pgfqpoint{1.647265in}{3.133391in}}%
\pgfpathlineto{\pgfqpoint{1.650208in}{3.131271in}}%
\pgfpathlineto{\pgfqpoint{1.655025in}{3.124232in}}%
\pgfpathlineto{\pgfqpoint{1.660378in}{3.117596in}}%
\pgfpathlineto{\pgfqpoint{1.663321in}{3.116986in}}%
\pgfpathlineto{\pgfqpoint{1.666265in}{3.118980in}}%
\pgfpathlineto{\pgfqpoint{1.670814in}{3.125508in}}%
\pgfpathlineto{\pgfqpoint{1.676702in}{3.132895in}}%
\pgfpathlineto{\pgfqpoint{1.679645in}{3.133403in}}%
\pgfpathlineto{\pgfqpoint{1.682589in}{3.131316in}}%
\pgfpathlineto{\pgfqpoint{1.687406in}{3.124298in}}%
\pgfpathlineto{\pgfqpoint{1.692758in}{3.117623in}}%
\pgfpathlineto{\pgfqpoint{1.695702in}{3.116976in}}%
\pgfpathlineto{\pgfqpoint{1.698645in}{3.118936in}}%
\pgfpathlineto{\pgfqpoint{1.703195in}{3.125442in}}%
\pgfpathlineto{\pgfqpoint{1.709082in}{3.132869in}}%
\pgfpathlineto{\pgfqpoint{1.712026in}{3.133414in}}%
\pgfpathlineto{\pgfqpoint{1.714970in}{3.131361in}}%
\pgfpathlineto{\pgfqpoint{1.719786in}{3.124364in}}%
\pgfpathlineto{\pgfqpoint{1.725139in}{3.117651in}}%
\pgfpathlineto{\pgfqpoint{1.728082in}{3.116966in}}%
\pgfpathlineto{\pgfqpoint{1.731026in}{3.118893in}}%
\pgfpathlineto{\pgfqpoint{1.735575in}{3.125375in}}%
\pgfpathlineto{\pgfqpoint{1.741463in}{3.132843in}}%
\pgfpathlineto{\pgfqpoint{1.744406in}{3.133425in}}%
\pgfpathlineto{\pgfqpoint{1.747350in}{3.131406in}}%
\pgfpathlineto{\pgfqpoint{1.752167in}{3.124430in}}%
\pgfpathlineto{\pgfqpoint{1.757787in}{3.117505in}}%
\pgfpathlineto{\pgfqpoint{1.760730in}{3.117025in}}%
\pgfpathlineto{\pgfqpoint{1.763674in}{3.119137in}}%
\pgfpathlineto{\pgfqpoint{1.768491in}{3.126170in}}%
\pgfpathlineto{\pgfqpoint{1.773843in}{3.132816in}}%
\pgfpathlineto{\pgfqpoint{1.776787in}{3.133436in}}%
\pgfpathlineto{\pgfqpoint{1.779731in}{3.131450in}}%
\pgfpathlineto{\pgfqpoint{1.784280in}{3.124927in}}%
\pgfpathlineto{\pgfqpoint{1.790167in}{3.117530in}}%
\pgfpathlineto{\pgfqpoint{1.793111in}{3.117013in}}%
\pgfpathlineto{\pgfqpoint{1.796055in}{3.119091in}}%
\pgfpathlineto{\pgfqpoint{1.800872in}{3.126104in}}%
\pgfpathlineto{\pgfqpoint{1.806224in}{3.132789in}}%
\pgfpathlineto{\pgfqpoint{1.809167in}{3.133445in}}%
\pgfpathlineto{\pgfqpoint{1.812111in}{3.131494in}}%
\pgfpathlineto{\pgfqpoint{1.816661in}{3.124994in}}%
\pgfpathlineto{\pgfqpoint{1.822548in}{3.117556in}}%
\pgfpathlineto{\pgfqpoint{1.825492in}{3.117002in}}%
\pgfpathlineto{\pgfqpoint{1.828435in}{3.119047in}}%
\pgfpathlineto{\pgfqpoint{1.833252in}{3.126038in}}%
\pgfpathlineto{\pgfqpoint{1.838604in}{3.132761in}}%
\pgfpathlineto{\pgfqpoint{1.841548in}{3.133455in}}%
\pgfpathlineto{\pgfqpoint{1.844224in}{3.131809in}}%
\pgfpathlineto{\pgfqpoint{1.848506in}{3.125922in}}%
\pgfpathlineto{\pgfqpoint{1.854928in}{3.117583in}}%
\pgfpathlineto{\pgfqpoint{1.857872in}{3.116991in}}%
\pgfpathlineto{\pgfqpoint{1.860816in}{3.119002in}}%
\pgfpathlineto{\pgfqpoint{1.865633in}{3.125972in}}%
\pgfpathlineto{\pgfqpoint{1.871253in}{3.132907in}}%
\pgfpathlineto{\pgfqpoint{1.874196in}{3.133397in}}%
\pgfpathlineto{\pgfqpoint{1.877140in}{3.131294in}}%
\pgfpathlineto{\pgfqpoint{1.881957in}{3.124265in}}%
\pgfpathlineto{\pgfqpoint{1.887309in}{3.117609in}}%
\pgfpathlineto{\pgfqpoint{1.890253in}{3.116981in}}%
\pgfpathlineto{\pgfqpoint{1.893196in}{3.118958in}}%
\pgfpathlineto{\pgfqpoint{1.897746in}{3.125475in}}%
\pgfpathlineto{\pgfqpoint{1.903633in}{3.132882in}}%
\pgfpathlineto{\pgfqpoint{1.906577in}{3.133409in}}%
\pgfpathlineto{\pgfqpoint{1.909520in}{3.131339in}}%
\pgfpathlineto{\pgfqpoint{1.914337in}{3.124331in}}%
\pgfpathlineto{\pgfqpoint{1.919690in}{3.117637in}}%
\pgfpathlineto{\pgfqpoint{1.922633in}{3.116971in}}%
\pgfpathlineto{\pgfqpoint{1.925577in}{3.118914in}}%
\pgfpathlineto{\pgfqpoint{1.930126in}{3.125409in}}%
\pgfpathlineto{\pgfqpoint{1.936014in}{3.132856in}}%
\pgfpathlineto{\pgfqpoint{1.938957in}{3.133420in}}%
\pgfpathlineto{\pgfqpoint{1.941901in}{3.131383in}}%
\pgfpathlineto{\pgfqpoint{1.946718in}{3.124397in}}%
\pgfpathlineto{\pgfqpoint{1.952338in}{3.117493in}}%
\pgfpathlineto{\pgfqpoint{1.955281in}{3.117031in}}%
\pgfpathlineto{\pgfqpoint{1.958225in}{3.119159in}}%
\pgfpathlineto{\pgfqpoint{1.963042in}{3.126203in}}%
\pgfpathlineto{\pgfqpoint{1.968394in}{3.132830in}}%
\pgfpathlineto{\pgfqpoint{1.971338in}{3.133430in}}%
\pgfpathlineto{\pgfqpoint{1.974282in}{3.131428in}}%
\pgfpathlineto{\pgfqpoint{1.979098in}{3.124463in}}%
\pgfpathlineto{\pgfqpoint{1.984718in}{3.117518in}}%
\pgfpathlineto{\pgfqpoint{1.987662in}{3.117019in}}%
\pgfpathlineto{\pgfqpoint{1.990606in}{3.119114in}}%
\pgfpathlineto{\pgfqpoint{1.995423in}{3.126137in}}%
\pgfpathlineto{\pgfqpoint{2.000775in}{3.132803in}}%
\pgfpathlineto{\pgfqpoint{2.003718in}{3.133441in}}%
\pgfpathlineto{\pgfqpoint{2.006662in}{3.131472in}}%
\pgfpathlineto{\pgfqpoint{2.011211in}{3.124960in}}%
\pgfpathlineto{\pgfqpoint{2.017099in}{3.117543in}}%
\pgfpathlineto{\pgfqpoint{2.020042in}{3.117007in}}%
\pgfpathlineto{\pgfqpoint{2.022986in}{3.119069in}}%
\pgfpathlineto{\pgfqpoint{2.027803in}{3.126071in}}%
\pgfpathlineto{\pgfqpoint{2.033155in}{3.132775in}}%
\pgfpathlineto{\pgfqpoint{2.036099in}{3.133450in}}%
\pgfpathlineto{\pgfqpoint{2.039043in}{3.131515in}}%
\pgfpathlineto{\pgfqpoint{2.043592in}{3.125027in}}%
\pgfpathlineto{\pgfqpoint{2.049479in}{3.117569in}}%
\pgfpathlineto{\pgfqpoint{2.052423in}{3.116996in}}%
\pgfpathlineto{\pgfqpoint{2.055367in}{3.119024in}}%
\pgfpathlineto{\pgfqpoint{2.060184in}{3.126005in}}%
\pgfpathlineto{\pgfqpoint{2.065803in}{3.132920in}}%
\pgfpathlineto{\pgfqpoint{2.068747in}{3.133391in}}%
\pgfpathlineto{\pgfqpoint{2.071691in}{3.131271in}}%
\pgfpathlineto{\pgfqpoint{2.076508in}{3.124232in}}%
\pgfpathlineto{\pgfqpoint{2.081860in}{3.117596in}}%
\pgfpathlineto{\pgfqpoint{2.084804in}{3.116986in}}%
\pgfpathlineto{\pgfqpoint{2.087747in}{3.118980in}}%
\pgfpathlineto{\pgfqpoint{2.092297in}{3.125508in}}%
\pgfpathlineto{\pgfqpoint{2.098184in}{3.132895in}}%
\pgfpathlineto{\pgfqpoint{2.101128in}{3.133403in}}%
\pgfpathlineto{\pgfqpoint{2.104071in}{3.131316in}}%
\pgfpathlineto{\pgfqpoint{2.108888in}{3.124298in}}%
\pgfpathlineto{\pgfqpoint{2.114240in}{3.117623in}}%
\pgfpathlineto{\pgfqpoint{2.117184in}{3.116976in}}%
\pgfpathlineto{\pgfqpoint{2.120128in}{3.118936in}}%
\pgfpathlineto{\pgfqpoint{2.124677in}{3.125442in}}%
\pgfpathlineto{\pgfqpoint{2.130564in}{3.132869in}}%
\pgfpathlineto{\pgfqpoint{2.133508in}{3.133414in}}%
\pgfpathlineto{\pgfqpoint{2.136452in}{3.131361in}}%
\pgfpathlineto{\pgfqpoint{2.141269in}{3.124364in}}%
\pgfpathlineto{\pgfqpoint{2.146621in}{3.117651in}}%
\pgfpathlineto{\pgfqpoint{2.149565in}{3.116966in}}%
\pgfpathlineto{\pgfqpoint{2.152508in}{3.118893in}}%
\pgfpathlineto{\pgfqpoint{2.157058in}{3.125375in}}%
\pgfpathlineto{\pgfqpoint{2.162945in}{3.132843in}}%
\pgfpathlineto{\pgfqpoint{2.165889in}{3.133425in}}%
\pgfpathlineto{\pgfqpoint{2.168832in}{3.131406in}}%
\pgfpathlineto{\pgfqpoint{2.173649in}{3.124430in}}%
\pgfpathlineto{\pgfqpoint{2.179269in}{3.117505in}}%
\pgfpathlineto{\pgfqpoint{2.182213in}{3.117025in}}%
\pgfpathlineto{\pgfqpoint{2.185156in}{3.119137in}}%
\pgfpathlineto{\pgfqpoint{2.189973in}{3.126170in}}%
\pgfpathlineto{\pgfqpoint{2.195326in}{3.132816in}}%
\pgfpathlineto{\pgfqpoint{2.198269in}{3.133436in}}%
\pgfpathlineto{\pgfqpoint{2.201213in}{3.131450in}}%
\pgfpathlineto{\pgfqpoint{2.205762in}{3.124927in}}%
\pgfpathlineto{\pgfqpoint{2.211650in}{3.117530in}}%
\pgfpathlineto{\pgfqpoint{2.214593in}{3.117013in}}%
\pgfpathlineto{\pgfqpoint{2.217537in}{3.119091in}}%
\pgfpathlineto{\pgfqpoint{2.222354in}{3.126104in}}%
\pgfpathlineto{\pgfqpoint{2.227706in}{3.132789in}}%
\pgfpathlineto{\pgfqpoint{2.230650in}{3.133445in}}%
\pgfpathlineto{\pgfqpoint{2.233593in}{3.131494in}}%
\pgfpathlineto{\pgfqpoint{2.238143in}{3.124994in}}%
\pgfpathlineto{\pgfqpoint{2.244030in}{3.117556in}}%
\pgfpathlineto{\pgfqpoint{2.246974in}{3.117002in}}%
\pgfpathlineto{\pgfqpoint{2.249918in}{3.119047in}}%
\pgfpathlineto{\pgfqpoint{2.254735in}{3.126038in}}%
\pgfpathlineto{\pgfqpoint{2.260087in}{3.132761in}}%
\pgfpathlineto{\pgfqpoint{2.263030in}{3.133455in}}%
\pgfpathlineto{\pgfqpoint{2.265706in}{3.131809in}}%
\pgfpathlineto{\pgfqpoint{2.269988in}{3.125922in}}%
\pgfpathlineto{\pgfqpoint{2.276411in}{3.117583in}}%
\pgfpathlineto{\pgfqpoint{2.279354in}{3.116991in}}%
\pgfpathlineto{\pgfqpoint{2.282298in}{3.119002in}}%
\pgfpathlineto{\pgfqpoint{2.287115in}{3.125972in}}%
\pgfpathlineto{\pgfqpoint{2.292735in}{3.132907in}}%
\pgfpathlineto{\pgfqpoint{2.295679in}{3.133397in}}%
\pgfpathlineto{\pgfqpoint{2.298622in}{3.131294in}}%
\pgfpathlineto{\pgfqpoint{2.303439in}{3.124265in}}%
\pgfpathlineto{\pgfqpoint{2.308791in}{3.117609in}}%
\pgfpathlineto{\pgfqpoint{2.311735in}{3.116981in}}%
\pgfpathlineto{\pgfqpoint{2.314679in}{3.118958in}}%
\pgfpathlineto{\pgfqpoint{2.319228in}{3.125475in}}%
\pgfpathlineto{\pgfqpoint{2.325115in}{3.132882in}}%
\pgfpathlineto{\pgfqpoint{2.328059in}{3.133409in}}%
\pgfpathlineto{\pgfqpoint{2.331003in}{3.131339in}}%
\pgfpathlineto{\pgfqpoint{2.335820in}{3.124331in}}%
\pgfpathlineto{\pgfqpoint{2.341172in}{3.117637in}}%
\pgfpathlineto{\pgfqpoint{2.344116in}{3.116971in}}%
\pgfpathlineto{\pgfqpoint{2.347059in}{3.118914in}}%
\pgfpathlineto{\pgfqpoint{2.351609in}{3.125409in}}%
\pgfpathlineto{\pgfqpoint{2.357496in}{3.132856in}}%
\pgfpathlineto{\pgfqpoint{2.360440in}{3.133420in}}%
\pgfpathlineto{\pgfqpoint{2.363383in}{3.131383in}}%
\pgfpathlineto{\pgfqpoint{2.368200in}{3.124397in}}%
\pgfpathlineto{\pgfqpoint{2.373820in}{3.117493in}}%
\pgfpathlineto{\pgfqpoint{2.376764in}{3.117031in}}%
\pgfpathlineto{\pgfqpoint{2.379707in}{3.119159in}}%
\pgfpathlineto{\pgfqpoint{2.384524in}{3.126203in}}%
\pgfpathlineto{\pgfqpoint{2.389876in}{3.132830in}}%
\pgfpathlineto{\pgfqpoint{2.392820in}{3.133430in}}%
\pgfpathlineto{\pgfqpoint{2.395764in}{3.131428in}}%
\pgfpathlineto{\pgfqpoint{2.400581in}{3.124463in}}%
\pgfpathlineto{\pgfqpoint{2.406201in}{3.117518in}}%
\pgfpathlineto{\pgfqpoint{2.409144in}{3.117019in}}%
\pgfpathlineto{\pgfqpoint{2.412088in}{3.119114in}}%
\pgfpathlineto{\pgfqpoint{2.416905in}{3.126137in}}%
\pgfpathlineto{\pgfqpoint{2.422257in}{3.132803in}}%
\pgfpathlineto{\pgfqpoint{2.425201in}{3.133441in}}%
\pgfpathlineto{\pgfqpoint{2.428144in}{3.131472in}}%
\pgfpathlineto{\pgfqpoint{2.432694in}{3.124960in}}%
\pgfpathlineto{\pgfqpoint{2.438581in}{3.117543in}}%
\pgfpathlineto{\pgfqpoint{2.441525in}{3.117007in}}%
\pgfpathlineto{\pgfqpoint{2.444468in}{3.119069in}}%
\pgfpathlineto{\pgfqpoint{2.449285in}{3.126071in}}%
\pgfpathlineto{\pgfqpoint{2.454638in}{3.132775in}}%
\pgfpathlineto{\pgfqpoint{2.457581in}{3.133450in}}%
\pgfpathlineto{\pgfqpoint{2.460525in}{3.131515in}}%
\pgfpathlineto{\pgfqpoint{2.465074in}{3.125027in}}%
\pgfpathlineto{\pgfqpoint{2.470962in}{3.117569in}}%
\pgfpathlineto{\pgfqpoint{2.473905in}{3.116996in}}%
\pgfpathlineto{\pgfqpoint{2.476849in}{3.119024in}}%
\pgfpathlineto{\pgfqpoint{2.481666in}{3.126005in}}%
\pgfpathlineto{\pgfqpoint{2.487286in}{3.132920in}}%
\pgfpathlineto{\pgfqpoint{2.490229in}{3.133391in}}%
\pgfpathlineto{\pgfqpoint{2.493173in}{3.131271in}}%
\pgfpathlineto{\pgfqpoint{2.497990in}{3.124232in}}%
\pgfpathlineto{\pgfqpoint{2.503342in}{3.117596in}}%
\pgfpathlineto{\pgfqpoint{2.506286in}{3.116986in}}%
\pgfpathlineto{\pgfqpoint{2.509230in}{3.118980in}}%
\pgfpathlineto{\pgfqpoint{2.513779in}{3.125508in}}%
\pgfpathlineto{\pgfqpoint{2.519666in}{3.132895in}}%
\pgfpathlineto{\pgfqpoint{2.522610in}{3.133403in}}%
\pgfpathlineto{\pgfqpoint{2.525554in}{3.131316in}}%
\pgfpathlineto{\pgfqpoint{2.530371in}{3.124298in}}%
\pgfpathlineto{\pgfqpoint{2.535723in}{3.117623in}}%
\pgfpathlineto{\pgfqpoint{2.538666in}{3.116976in}}%
\pgfpathlineto{\pgfqpoint{2.541610in}{3.118936in}}%
\pgfpathlineto{\pgfqpoint{2.546159in}{3.125442in}}%
\pgfpathlineto{\pgfqpoint{2.552047in}{3.132869in}}%
\pgfpathlineto{\pgfqpoint{2.554990in}{3.133414in}}%
\pgfpathlineto{\pgfqpoint{2.557934in}{3.131361in}}%
\pgfpathlineto{\pgfqpoint{2.562751in}{3.124364in}}%
\pgfpathlineto{\pgfqpoint{2.568103in}{3.117651in}}%
\pgfpathlineto{\pgfqpoint{2.571047in}{3.116966in}}%
\pgfpathlineto{\pgfqpoint{2.573991in}{3.118893in}}%
\pgfpathlineto{\pgfqpoint{2.578540in}{3.125375in}}%
\pgfpathlineto{\pgfqpoint{2.584427in}{3.132843in}}%
\pgfpathlineto{\pgfqpoint{2.587371in}{3.133425in}}%
\pgfpathlineto{\pgfqpoint{2.590315in}{3.131406in}}%
\pgfpathlineto{\pgfqpoint{2.595132in}{3.124430in}}%
\pgfpathlineto{\pgfqpoint{2.600751in}{3.117505in}}%
\pgfpathlineto{\pgfqpoint{2.603695in}{3.117025in}}%
\pgfpathlineto{\pgfqpoint{2.606639in}{3.119137in}}%
\pgfpathlineto{\pgfqpoint{2.611456in}{3.126170in}}%
\pgfpathlineto{\pgfqpoint{2.616808in}{3.132816in}}%
\pgfpathlineto{\pgfqpoint{2.619752in}{3.133436in}}%
\pgfpathlineto{\pgfqpoint{2.622695in}{3.131450in}}%
\pgfpathlineto{\pgfqpoint{2.627245in}{3.124927in}}%
\pgfpathlineto{\pgfqpoint{2.633132in}{3.117530in}}%
\pgfpathlineto{\pgfqpoint{2.636076in}{3.117013in}}%
\pgfpathlineto{\pgfqpoint{2.639019in}{3.119091in}}%
\pgfpathlineto{\pgfqpoint{2.643836in}{3.126104in}}%
\pgfpathlineto{\pgfqpoint{2.649188in}{3.132789in}}%
\pgfpathlineto{\pgfqpoint{2.652132in}{3.133445in}}%
\pgfpathlineto{\pgfqpoint{2.655076in}{3.131494in}}%
\pgfpathlineto{\pgfqpoint{2.659625in}{3.124994in}}%
\pgfpathlineto{\pgfqpoint{2.665513in}{3.117556in}}%
\pgfpathlineto{\pgfqpoint{2.668456in}{3.117002in}}%
\pgfpathlineto{\pgfqpoint{2.671400in}{3.119047in}}%
\pgfpathlineto{\pgfqpoint{2.676217in}{3.126038in}}%
\pgfpathlineto{\pgfqpoint{2.681569in}{3.132761in}}%
\pgfpathlineto{\pgfqpoint{2.684513in}{3.133455in}}%
\pgfpathlineto{\pgfqpoint{2.687189in}{3.131809in}}%
\pgfpathlineto{\pgfqpoint{2.691470in}{3.125922in}}%
\pgfpathlineto{\pgfqpoint{2.697893in}{3.117583in}}%
\pgfpathlineto{\pgfqpoint{2.700837in}{3.116991in}}%
\pgfpathlineto{\pgfqpoint{2.703780in}{3.119002in}}%
\pgfpathlineto{\pgfqpoint{2.708597in}{3.125972in}}%
\pgfpathlineto{\pgfqpoint{2.714217in}{3.132907in}}%
\pgfpathlineto{\pgfqpoint{2.717161in}{3.133397in}}%
\pgfpathlineto{\pgfqpoint{2.720105in}{3.131294in}}%
\pgfpathlineto{\pgfqpoint{2.724921in}{3.124265in}}%
\pgfpathlineto{\pgfqpoint{2.730274in}{3.117609in}}%
\pgfpathlineto{\pgfqpoint{2.733217in}{3.116981in}}%
\pgfpathlineto{\pgfqpoint{2.736161in}{3.118958in}}%
\pgfpathlineto{\pgfqpoint{2.740710in}{3.125475in}}%
\pgfpathlineto{\pgfqpoint{2.746598in}{3.132882in}}%
\pgfpathlineto{\pgfqpoint{2.749541in}{3.133409in}}%
\pgfpathlineto{\pgfqpoint{2.752485in}{3.131339in}}%
\pgfpathlineto{\pgfqpoint{2.757302in}{3.124331in}}%
\pgfpathlineto{\pgfqpoint{2.762654in}{3.117637in}}%
\pgfpathlineto{\pgfqpoint{2.765598in}{3.116971in}}%
\pgfpathlineto{\pgfqpoint{2.768542in}{3.118914in}}%
\pgfpathlineto{\pgfqpoint{2.773091in}{3.125409in}}%
\pgfpathlineto{\pgfqpoint{2.778978in}{3.132856in}}%
\pgfpathlineto{\pgfqpoint{2.781922in}{3.133420in}}%
\pgfpathlineto{\pgfqpoint{2.784866in}{3.131383in}}%
\pgfpathlineto{\pgfqpoint{2.789683in}{3.124397in}}%
\pgfpathlineto{\pgfqpoint{2.795302in}{3.117493in}}%
\pgfpathlineto{\pgfqpoint{2.798246in}{3.117031in}}%
\pgfpathlineto{\pgfqpoint{2.801190in}{3.119159in}}%
\pgfpathlineto{\pgfqpoint{2.806007in}{3.126203in}}%
\pgfpathlineto{\pgfqpoint{2.811359in}{3.132830in}}%
\pgfpathlineto{\pgfqpoint{2.814302in}{3.133430in}}%
\pgfpathlineto{\pgfqpoint{2.817246in}{3.131428in}}%
\pgfpathlineto{\pgfqpoint{2.822063in}{3.124463in}}%
\pgfpathlineto{\pgfqpoint{2.827683in}{3.117518in}}%
\pgfpathlineto{\pgfqpoint{2.830627in}{3.117019in}}%
\pgfpathlineto{\pgfqpoint{2.833570in}{3.119114in}}%
\pgfpathlineto{\pgfqpoint{2.838387in}{3.126137in}}%
\pgfpathlineto{\pgfqpoint{2.843739in}{3.132803in}}%
\pgfpathlineto{\pgfqpoint{2.846683in}{3.133441in}}%
\pgfpathlineto{\pgfqpoint{2.849627in}{3.131472in}}%
\pgfpathlineto{\pgfqpoint{2.854176in}{3.124960in}}%
\pgfpathlineto{\pgfqpoint{2.860063in}{3.117543in}}%
\pgfpathlineto{\pgfqpoint{2.863007in}{3.117007in}}%
\pgfpathlineto{\pgfqpoint{2.865951in}{3.119069in}}%
\pgfpathlineto{\pgfqpoint{2.870768in}{3.126071in}}%
\pgfpathlineto{\pgfqpoint{2.876120in}{3.132775in}}%
\pgfpathlineto{\pgfqpoint{2.879064in}{3.133450in}}%
\pgfpathlineto{\pgfqpoint{2.882007in}{3.131515in}}%
\pgfpathlineto{\pgfqpoint{2.886557in}{3.125027in}}%
\pgfpathlineto{\pgfqpoint{2.892444in}{3.117569in}}%
\pgfpathlineto{\pgfqpoint{2.895388in}{3.116996in}}%
\pgfpathlineto{\pgfqpoint{2.898331in}{3.119024in}}%
\pgfpathlineto{\pgfqpoint{2.903148in}{3.126005in}}%
\pgfpathlineto{\pgfqpoint{2.908768in}{3.132920in}}%
\pgfpathlineto{\pgfqpoint{2.911712in}{3.133391in}}%
\pgfpathlineto{\pgfqpoint{2.914655in}{3.131271in}}%
\pgfpathlineto{\pgfqpoint{2.919472in}{3.124232in}}%
\pgfpathlineto{\pgfqpoint{2.924824in}{3.117596in}}%
\pgfpathlineto{\pgfqpoint{2.927768in}{3.116986in}}%
\pgfpathlineto{\pgfqpoint{2.930712in}{3.118980in}}%
\pgfpathlineto{\pgfqpoint{2.935261in}{3.125508in}}%
\pgfpathlineto{\pgfqpoint{2.941149in}{3.132895in}}%
\pgfpathlineto{\pgfqpoint{2.944092in}{3.133403in}}%
\pgfpathlineto{\pgfqpoint{2.947036in}{3.131316in}}%
\pgfpathlineto{\pgfqpoint{2.951853in}{3.124298in}}%
\pgfpathlineto{\pgfqpoint{2.957205in}{3.117623in}}%
\pgfpathlineto{\pgfqpoint{2.960149in}{3.116976in}}%
\pgfpathlineto{\pgfqpoint{2.963092in}{3.118936in}}%
\pgfpathlineto{\pgfqpoint{2.967642in}{3.125442in}}%
\pgfpathlineto{\pgfqpoint{2.973529in}{3.132869in}}%
\pgfpathlineto{\pgfqpoint{2.976473in}{3.133414in}}%
\pgfpathlineto{\pgfqpoint{2.979416in}{3.131361in}}%
\pgfpathlineto{\pgfqpoint{2.984233in}{3.124364in}}%
\pgfpathlineto{\pgfqpoint{2.989586in}{3.117651in}}%
\pgfpathlineto{\pgfqpoint{2.992529in}{3.116966in}}%
\pgfpathlineto{\pgfqpoint{2.995473in}{3.118893in}}%
\pgfpathlineto{\pgfqpoint{3.000022in}{3.125375in}}%
\pgfpathlineto{\pgfqpoint{3.005910in}{3.132843in}}%
\pgfpathlineto{\pgfqpoint{3.008853in}{3.133425in}}%
\pgfpathlineto{\pgfqpoint{3.011797in}{3.131406in}}%
\pgfpathlineto{\pgfqpoint{3.016614in}{3.124430in}}%
\pgfpathlineto{\pgfqpoint{3.022234in}{3.117505in}}%
\pgfpathlineto{\pgfqpoint{3.025177in}{3.117025in}}%
\pgfpathlineto{\pgfqpoint{3.028121in}{3.119137in}}%
\pgfpathlineto{\pgfqpoint{3.032938in}{3.126170in}}%
\pgfpathlineto{\pgfqpoint{3.038290in}{3.132816in}}%
\pgfpathlineto{\pgfqpoint{3.041234in}{3.133436in}}%
\pgfpathlineto{\pgfqpoint{3.044178in}{3.131450in}}%
\pgfpathlineto{\pgfqpoint{3.048727in}{3.124927in}}%
\pgfpathlineto{\pgfqpoint{3.054614in}{3.117530in}}%
\pgfpathlineto{\pgfqpoint{3.057558in}{3.117013in}}%
\pgfpathlineto{\pgfqpoint{3.060502in}{3.119091in}}%
\pgfpathlineto{\pgfqpoint{3.065319in}{3.126104in}}%
\pgfpathlineto{\pgfqpoint{3.070671in}{3.132789in}}%
\pgfpathlineto{\pgfqpoint{3.073614in}{3.133445in}}%
\pgfpathlineto{\pgfqpoint{3.076558in}{3.131494in}}%
\pgfpathlineto{\pgfqpoint{3.081107in}{3.124994in}}%
\pgfpathlineto{\pgfqpoint{3.086995in}{3.117556in}}%
\pgfpathlineto{\pgfqpoint{3.089939in}{3.117002in}}%
\pgfpathlineto{\pgfqpoint{3.092882in}{3.119047in}}%
\pgfpathlineto{\pgfqpoint{3.097699in}{3.126038in}}%
\pgfpathlineto{\pgfqpoint{3.103051in}{3.132761in}}%
\pgfpathlineto{\pgfqpoint{3.105995in}{3.133455in}}%
\pgfpathlineto{\pgfqpoint{3.108671in}{3.131809in}}%
\pgfpathlineto{\pgfqpoint{3.112953in}{3.125922in}}%
\pgfpathlineto{\pgfqpoint{3.119375in}{3.117583in}}%
\pgfpathlineto{\pgfqpoint{3.122319in}{3.116991in}}%
\pgfpathlineto{\pgfqpoint{3.125263in}{3.119002in}}%
\pgfpathlineto{\pgfqpoint{3.130080in}{3.125972in}}%
\pgfpathlineto{\pgfqpoint{3.135699in}{3.132907in}}%
\pgfpathlineto{\pgfqpoint{3.138643in}{3.133397in}}%
\pgfpathlineto{\pgfqpoint{3.141587in}{3.131294in}}%
\pgfpathlineto{\pgfqpoint{3.146404in}{3.124265in}}%
\pgfpathlineto{\pgfqpoint{3.151756in}{3.117609in}}%
\pgfpathlineto{\pgfqpoint{3.154700in}{3.116981in}}%
\pgfpathlineto{\pgfqpoint{3.157643in}{3.118958in}}%
\pgfpathlineto{\pgfqpoint{3.162193in}{3.125475in}}%
\pgfpathlineto{\pgfqpoint{3.168080in}{3.132882in}}%
\pgfpathlineto{\pgfqpoint{3.171024in}{3.133409in}}%
\pgfpathlineto{\pgfqpoint{3.173967in}{3.131339in}}%
\pgfpathlineto{\pgfqpoint{3.178784in}{3.124331in}}%
\pgfpathlineto{\pgfqpoint{3.184136in}{3.117637in}}%
\pgfpathlineto{\pgfqpoint{3.187080in}{3.116971in}}%
\pgfpathlineto{\pgfqpoint{3.190024in}{3.118914in}}%
\pgfpathlineto{\pgfqpoint{3.194573in}{3.125409in}}%
\pgfpathlineto{\pgfqpoint{3.200461in}{3.132856in}}%
\pgfpathlineto{\pgfqpoint{3.203404in}{3.133420in}}%
\pgfpathlineto{\pgfqpoint{3.206348in}{3.131383in}}%
\pgfpathlineto{\pgfqpoint{3.211165in}{3.124397in}}%
\pgfpathlineto{\pgfqpoint{3.216785in}{3.117493in}}%
\pgfpathlineto{\pgfqpoint{3.219728in}{3.117031in}}%
\pgfpathlineto{\pgfqpoint{3.222672in}{3.119159in}}%
\pgfpathlineto{\pgfqpoint{3.227489in}{3.126203in}}%
\pgfpathlineto{\pgfqpoint{3.232841in}{3.132830in}}%
\pgfpathlineto{\pgfqpoint{3.235785in}{3.133430in}}%
\pgfpathlineto{\pgfqpoint{3.238728in}{3.131428in}}%
\pgfpathlineto{\pgfqpoint{3.243545in}{3.124463in}}%
\pgfpathlineto{\pgfqpoint{3.249165in}{3.117518in}}%
\pgfpathlineto{\pgfqpoint{3.252109in}{3.117019in}}%
\pgfpathlineto{\pgfqpoint{3.255053in}{3.119114in}}%
\pgfpathlineto{\pgfqpoint{3.259869in}{3.126137in}}%
\pgfpathlineto{\pgfqpoint{3.265222in}{3.132803in}}%
\pgfpathlineto{\pgfqpoint{3.268165in}{3.133441in}}%
\pgfpathlineto{\pgfqpoint{3.271109in}{3.131472in}}%
\pgfpathlineto{\pgfqpoint{3.275658in}{3.124960in}}%
\pgfpathlineto{\pgfqpoint{3.281546in}{3.117543in}}%
\pgfpathlineto{\pgfqpoint{3.284489in}{3.117007in}}%
\pgfpathlineto{\pgfqpoint{3.287433in}{3.119069in}}%
\pgfpathlineto{\pgfqpoint{3.292250in}{3.126071in}}%
\pgfpathlineto{\pgfqpoint{3.297602in}{3.132775in}}%
\pgfpathlineto{\pgfqpoint{3.300546in}{3.133450in}}%
\pgfpathlineto{\pgfqpoint{3.303490in}{3.131515in}}%
\pgfpathlineto{\pgfqpoint{3.308039in}{3.125027in}}%
\pgfpathlineto{\pgfqpoint{3.313926in}{3.117569in}}%
\pgfpathlineto{\pgfqpoint{3.316870in}{3.116996in}}%
\pgfpathlineto{\pgfqpoint{3.319814in}{3.119024in}}%
\pgfpathlineto{\pgfqpoint{3.324631in}{3.126005in}}%
\pgfpathlineto{\pgfqpoint{3.330250in}{3.132920in}}%
\pgfpathlineto{\pgfqpoint{3.333194in}{3.133391in}}%
\pgfpathlineto{\pgfqpoint{3.336138in}{3.131271in}}%
\pgfpathlineto{\pgfqpoint{3.340955in}{3.124232in}}%
\pgfpathlineto{\pgfqpoint{3.346307in}{3.117596in}}%
\pgfpathlineto{\pgfqpoint{3.349251in}{3.116986in}}%
\pgfpathlineto{\pgfqpoint{3.352194in}{3.118980in}}%
\pgfpathlineto{\pgfqpoint{3.356744in}{3.125508in}}%
\pgfpathlineto{\pgfqpoint{3.362631in}{3.132895in}}%
\pgfpathlineto{\pgfqpoint{3.365575in}{3.133403in}}%
\pgfpathlineto{\pgfqpoint{3.368518in}{3.131316in}}%
\pgfpathlineto{\pgfqpoint{3.373335in}{3.124298in}}%
\pgfpathlineto{\pgfqpoint{3.378687in}{3.117623in}}%
\pgfpathlineto{\pgfqpoint{3.381631in}{3.116976in}}%
\pgfpathlineto{\pgfqpoint{3.384575in}{3.118936in}}%
\pgfpathlineto{\pgfqpoint{3.389124in}{3.125442in}}%
\pgfpathlineto{\pgfqpoint{3.395011in}{3.132869in}}%
\pgfpathlineto{\pgfqpoint{3.397955in}{3.133414in}}%
\pgfpathlineto{\pgfqpoint{3.400899in}{3.131361in}}%
\pgfpathlineto{\pgfqpoint{3.405716in}{3.124364in}}%
\pgfpathlineto{\pgfqpoint{3.411068in}{3.117651in}}%
\pgfpathlineto{\pgfqpoint{3.414012in}{3.116966in}}%
\pgfpathlineto{\pgfqpoint{3.416955in}{3.118893in}}%
\pgfpathlineto{\pgfqpoint{3.421505in}{3.125375in}}%
\pgfpathlineto{\pgfqpoint{3.427392in}{3.132843in}}%
\pgfpathlineto{\pgfqpoint{3.430336in}{3.133425in}}%
\pgfpathlineto{\pgfqpoint{3.433279in}{3.131406in}}%
\pgfpathlineto{\pgfqpoint{3.438096in}{3.124430in}}%
\pgfpathlineto{\pgfqpoint{3.443716in}{3.117505in}}%
\pgfpathlineto{\pgfqpoint{3.446660in}{3.117025in}}%
\pgfpathlineto{\pgfqpoint{3.449603in}{3.119137in}}%
\pgfpathlineto{\pgfqpoint{3.454420in}{3.126170in}}%
\pgfpathlineto{\pgfqpoint{3.459773in}{3.132816in}}%
\pgfpathlineto{\pgfqpoint{3.462716in}{3.133436in}}%
\pgfpathlineto{\pgfqpoint{3.465660in}{3.131450in}}%
\pgfpathlineto{\pgfqpoint{3.470209in}{3.124927in}}%
\pgfpathlineto{\pgfqpoint{3.476097in}{3.117530in}}%
\pgfpathlineto{\pgfqpoint{3.479040in}{3.117013in}}%
\pgfpathlineto{\pgfqpoint{3.481984in}{3.119091in}}%
\pgfpathlineto{\pgfqpoint{3.486801in}{3.126104in}}%
\pgfpathlineto{\pgfqpoint{3.492153in}{3.132789in}}%
\pgfpathlineto{\pgfqpoint{3.495097in}{3.133445in}}%
\pgfpathlineto{\pgfqpoint{3.498040in}{3.131494in}}%
\pgfpathlineto{\pgfqpoint{3.502590in}{3.124994in}}%
\pgfpathlineto{\pgfqpoint{3.508477in}{3.117556in}}%
\pgfpathlineto{\pgfqpoint{3.511421in}{3.117002in}}%
\pgfpathlineto{\pgfqpoint{3.514365in}{3.119047in}}%
\pgfpathlineto{\pgfqpoint{3.519181in}{3.126038in}}%
\pgfpathlineto{\pgfqpoint{3.524534in}{3.132761in}}%
\pgfpathlineto{\pgfqpoint{3.527477in}{3.133455in}}%
\pgfpathlineto{\pgfqpoint{3.530153in}{3.131809in}}%
\pgfpathlineto{\pgfqpoint{3.534435in}{3.125922in}}%
\pgfpathlineto{\pgfqpoint{3.540858in}{3.117583in}}%
\pgfpathlineto{\pgfqpoint{3.543801in}{3.116991in}}%
\pgfpathlineto{\pgfqpoint{3.546745in}{3.119002in}}%
\pgfpathlineto{\pgfqpoint{3.551562in}{3.125972in}}%
\pgfpathlineto{\pgfqpoint{3.557182in}{3.132907in}}%
\pgfpathlineto{\pgfqpoint{3.560125in}{3.133397in}}%
\pgfpathlineto{\pgfqpoint{3.563069in}{3.131294in}}%
\pgfpathlineto{\pgfqpoint{3.567886in}{3.124265in}}%
\pgfpathlineto{\pgfqpoint{3.573238in}{3.117609in}}%
\pgfpathlineto{\pgfqpoint{3.576182in}{3.116981in}}%
\pgfpathlineto{\pgfqpoint{3.579126in}{3.118958in}}%
\pgfpathlineto{\pgfqpoint{3.583675in}{3.125475in}}%
\pgfpathlineto{\pgfqpoint{3.589562in}{3.132882in}}%
\pgfpathlineto{\pgfqpoint{3.592506in}{3.133409in}}%
\pgfpathlineto{\pgfqpoint{3.595450in}{3.131339in}}%
\pgfpathlineto{\pgfqpoint{3.600267in}{3.124331in}}%
\pgfpathlineto{\pgfqpoint{3.605619in}{3.117637in}}%
\pgfpathlineto{\pgfqpoint{3.608562in}{3.116971in}}%
\pgfpathlineto{\pgfqpoint{3.611506in}{3.118914in}}%
\pgfpathlineto{\pgfqpoint{3.616055in}{3.125409in}}%
\pgfpathlineto{\pgfqpoint{3.621943in}{3.132856in}}%
\pgfpathlineto{\pgfqpoint{3.624887in}{3.133420in}}%
\pgfpathlineto{\pgfqpoint{3.627830in}{3.131383in}}%
\pgfpathlineto{\pgfqpoint{3.632647in}{3.124397in}}%
\pgfpathlineto{\pgfqpoint{3.638267in}{3.117493in}}%
\pgfpathlineto{\pgfqpoint{3.641211in}{3.117031in}}%
\pgfpathlineto{\pgfqpoint{3.644154in}{3.119159in}}%
\pgfpathlineto{\pgfqpoint{3.648971in}{3.126203in}}%
\pgfpathlineto{\pgfqpoint{3.654323in}{3.132830in}}%
\pgfpathlineto{\pgfqpoint{3.657267in}{3.133430in}}%
\pgfpathlineto{\pgfqpoint{3.660211in}{3.131428in}}%
\pgfpathlineto{\pgfqpoint{3.665028in}{3.124463in}}%
\pgfpathlineto{\pgfqpoint{3.670647in}{3.117518in}}%
\pgfpathlineto{\pgfqpoint{3.673591in}{3.117019in}}%
\pgfpathlineto{\pgfqpoint{3.676535in}{3.119114in}}%
\pgfpathlineto{\pgfqpoint{3.681352in}{3.126137in}}%
\pgfpathlineto{\pgfqpoint{3.686704in}{3.132803in}}%
\pgfpathlineto{\pgfqpoint{3.689648in}{3.133441in}}%
\pgfpathlineto{\pgfqpoint{3.692591in}{3.131472in}}%
\pgfpathlineto{\pgfqpoint{3.697141in}{3.124960in}}%
\pgfpathlineto{\pgfqpoint{3.703028in}{3.117543in}}%
\pgfpathlineto{\pgfqpoint{3.705972in}{3.117007in}}%
\pgfpathlineto{\pgfqpoint{3.708915in}{3.119069in}}%
\pgfpathlineto{\pgfqpoint{3.713732in}{3.126071in}}%
\pgfpathlineto{\pgfqpoint{3.719085in}{3.132775in}}%
\pgfpathlineto{\pgfqpoint{3.722028in}{3.133450in}}%
\pgfpathlineto{\pgfqpoint{3.724972in}{3.131515in}}%
\pgfpathlineto{\pgfqpoint{3.729521in}{3.125027in}}%
\pgfpathlineto{\pgfqpoint{3.735409in}{3.117569in}}%
\pgfpathlineto{\pgfqpoint{3.738352in}{3.116996in}}%
\pgfpathlineto{\pgfqpoint{3.741296in}{3.119024in}}%
\pgfpathlineto{\pgfqpoint{3.746113in}{3.126005in}}%
\pgfpathlineto{\pgfqpoint{3.751733in}{3.132920in}}%
\pgfpathlineto{\pgfqpoint{3.754676in}{3.133391in}}%
\pgfpathlineto{\pgfqpoint{3.757620in}{3.131271in}}%
\pgfpathlineto{\pgfqpoint{3.762437in}{3.124232in}}%
\pgfpathlineto{\pgfqpoint{3.767789in}{3.117596in}}%
\pgfpathlineto{\pgfqpoint{3.770733in}{3.116986in}}%
\pgfpathlineto{\pgfqpoint{3.773677in}{3.118980in}}%
\pgfpathlineto{\pgfqpoint{3.778226in}{3.125508in}}%
\pgfpathlineto{\pgfqpoint{3.784113in}{3.132895in}}%
\pgfpathlineto{\pgfqpoint{3.787057in}{3.133403in}}%
\pgfpathlineto{\pgfqpoint{3.790001in}{3.131316in}}%
\pgfpathlineto{\pgfqpoint{3.794818in}{3.124298in}}%
\pgfpathlineto{\pgfqpoint{3.800170in}{3.117623in}}%
\pgfpathlineto{\pgfqpoint{3.803113in}{3.116976in}}%
\pgfpathlineto{\pgfqpoint{3.806057in}{3.118936in}}%
\pgfpathlineto{\pgfqpoint{3.810606in}{3.125442in}}%
\pgfpathlineto{\pgfqpoint{3.816494in}{3.132869in}}%
\pgfpathlineto{\pgfqpoint{3.819437in}{3.133414in}}%
\pgfpathlineto{\pgfqpoint{3.822381in}{3.131361in}}%
\pgfpathlineto{\pgfqpoint{3.827198in}{3.124364in}}%
\pgfpathlineto{\pgfqpoint{3.832550in}{3.117651in}}%
\pgfpathlineto{\pgfqpoint{3.835494in}{3.116966in}}%
\pgfpathlineto{\pgfqpoint{3.838438in}{3.118893in}}%
\pgfpathlineto{\pgfqpoint{3.842987in}{3.125375in}}%
\pgfpathlineto{\pgfqpoint{3.848874in}{3.132843in}}%
\pgfpathlineto{\pgfqpoint{3.851818in}{3.133425in}}%
\pgfpathlineto{\pgfqpoint{3.854762in}{3.131406in}}%
\pgfpathlineto{\pgfqpoint{3.859579in}{3.124430in}}%
\pgfpathlineto{\pgfqpoint{3.865198in}{3.117505in}}%
\pgfpathlineto{\pgfqpoint{3.868142in}{3.117025in}}%
\pgfpathlineto{\pgfqpoint{3.871086in}{3.119137in}}%
\pgfpathlineto{\pgfqpoint{3.875903in}{3.126170in}}%
\pgfpathlineto{\pgfqpoint{3.881255in}{3.132816in}}%
\pgfpathlineto{\pgfqpoint{3.884199in}{3.133436in}}%
\pgfpathlineto{\pgfqpoint{3.887142in}{3.131450in}}%
\pgfpathlineto{\pgfqpoint{3.891692in}{3.124927in}}%
\pgfpathlineto{\pgfqpoint{3.897579in}{3.117530in}}%
\pgfpathlineto{\pgfqpoint{3.900523in}{3.117013in}}%
\pgfpathlineto{\pgfqpoint{3.903466in}{3.119091in}}%
\pgfpathlineto{\pgfqpoint{3.908283in}{3.126104in}}%
\pgfpathlineto{\pgfqpoint{3.913635in}{3.132789in}}%
\pgfpathlineto{\pgfqpoint{3.916579in}{3.133445in}}%
\pgfpathlineto{\pgfqpoint{3.919523in}{3.131494in}}%
\pgfpathlineto{\pgfqpoint{3.924072in}{3.124994in}}%
\pgfpathlineto{\pgfqpoint{3.929959in}{3.117556in}}%
\pgfpathlineto{\pgfqpoint{3.932903in}{3.117002in}}%
\pgfpathlineto{\pgfqpoint{3.935847in}{3.119047in}}%
\pgfpathlineto{\pgfqpoint{3.940664in}{3.126038in}}%
\pgfpathlineto{\pgfqpoint{3.946016in}{3.132761in}}%
\pgfpathlineto{\pgfqpoint{3.948960in}{3.133455in}}%
\pgfpathlineto{\pgfqpoint{3.951636in}{3.131809in}}%
\pgfpathlineto{\pgfqpoint{3.955917in}{3.125922in}}%
\pgfpathlineto{\pgfqpoint{3.962340in}{3.117583in}}%
\pgfpathlineto{\pgfqpoint{3.965284in}{3.116991in}}%
\pgfpathlineto{\pgfqpoint{3.968227in}{3.119002in}}%
\pgfpathlineto{\pgfqpoint{3.973044in}{3.125972in}}%
\pgfpathlineto{\pgfqpoint{3.978664in}{3.132907in}}%
\pgfpathlineto{\pgfqpoint{3.981608in}{3.133397in}}%
\pgfpathlineto{\pgfqpoint{3.984551in}{3.131294in}}%
\pgfpathlineto{\pgfqpoint{3.989368in}{3.124265in}}%
\pgfpathlineto{\pgfqpoint{3.994721in}{3.117609in}}%
\pgfpathlineto{\pgfqpoint{3.997664in}{3.116981in}}%
\pgfpathlineto{\pgfqpoint{4.000608in}{3.118958in}}%
\pgfpathlineto{\pgfqpoint{4.005157in}{3.125475in}}%
\pgfpathlineto{\pgfqpoint{4.011045in}{3.132882in}}%
\pgfpathlineto{\pgfqpoint{4.013988in}{3.133409in}}%
\pgfpathlineto{\pgfqpoint{4.016932in}{3.131339in}}%
\pgfpathlineto{\pgfqpoint{4.021749in}{3.124331in}}%
\pgfpathlineto{\pgfqpoint{4.027101in}{3.117637in}}%
\pgfpathlineto{\pgfqpoint{4.030045in}{3.116971in}}%
\pgfpathlineto{\pgfqpoint{4.032988in}{3.118914in}}%
\pgfpathlineto{\pgfqpoint{4.037538in}{3.125409in}}%
\pgfpathlineto{\pgfqpoint{4.043425in}{3.132856in}}%
\pgfpathlineto{\pgfqpoint{4.046369in}{3.133420in}}%
\pgfpathlineto{\pgfqpoint{4.049313in}{3.131383in}}%
\pgfpathlineto{\pgfqpoint{4.054130in}{3.124397in}}%
\pgfpathlineto{\pgfqpoint{4.059749in}{3.117493in}}%
\pgfpathlineto{\pgfqpoint{4.062693in}{3.117031in}}%
\pgfpathlineto{\pgfqpoint{4.065637in}{3.119159in}}%
\pgfpathlineto{\pgfqpoint{4.070454in}{3.126203in}}%
\pgfpathlineto{\pgfqpoint{4.075806in}{3.132830in}}%
\pgfpathlineto{\pgfqpoint{4.078749in}{3.133430in}}%
\pgfpathlineto{\pgfqpoint{4.081693in}{3.131428in}}%
\pgfpathlineto{\pgfqpoint{4.086510in}{3.124463in}}%
\pgfpathlineto{\pgfqpoint{4.092130in}{3.117518in}}%
\pgfpathlineto{\pgfqpoint{4.095073in}{3.117019in}}%
\pgfpathlineto{\pgfqpoint{4.098017in}{3.119114in}}%
\pgfpathlineto{\pgfqpoint{4.102834in}{3.126137in}}%
\pgfpathlineto{\pgfqpoint{4.108186in}{3.132803in}}%
\pgfpathlineto{\pgfqpoint{4.111130in}{3.133441in}}%
\pgfpathlineto{\pgfqpoint{4.114074in}{3.131472in}}%
\pgfpathlineto{\pgfqpoint{4.118623in}{3.124960in}}%
\pgfpathlineto{\pgfqpoint{4.124510in}{3.117543in}}%
\pgfpathlineto{\pgfqpoint{4.127454in}{3.117007in}}%
\pgfpathlineto{\pgfqpoint{4.130398in}{3.119069in}}%
\pgfpathlineto{\pgfqpoint{4.135215in}{3.126071in}}%
\pgfpathlineto{\pgfqpoint{4.140567in}{3.132775in}}%
\pgfpathlineto{\pgfqpoint{4.143511in}{3.133450in}}%
\pgfpathlineto{\pgfqpoint{4.146454in}{3.131515in}}%
\pgfpathlineto{\pgfqpoint{4.151004in}{3.125027in}}%
\pgfpathlineto{\pgfqpoint{4.156891in}{3.117569in}}%
\pgfpathlineto{\pgfqpoint{4.159835in}{3.116996in}}%
\pgfpathlineto{\pgfqpoint{4.162778in}{3.119024in}}%
\pgfpathlineto{\pgfqpoint{4.167595in}{3.126005in}}%
\pgfpathlineto{\pgfqpoint{4.173215in}{3.132920in}}%
\pgfpathlineto{\pgfqpoint{4.176159in}{3.133391in}}%
\pgfpathlineto{\pgfqpoint{4.179102in}{3.131271in}}%
\pgfpathlineto{\pgfqpoint{4.183919in}{3.124232in}}%
\pgfpathlineto{\pgfqpoint{4.189271in}{3.117596in}}%
\pgfpathlineto{\pgfqpoint{4.192215in}{3.116986in}}%
\pgfpathlineto{\pgfqpoint{4.195159in}{3.118980in}}%
\pgfpathlineto{\pgfqpoint{4.199708in}{3.125508in}}%
\pgfpathlineto{\pgfqpoint{4.205596in}{3.132895in}}%
\pgfpathlineto{\pgfqpoint{4.208539in}{3.133403in}}%
\pgfpathlineto{\pgfqpoint{4.211483in}{3.131316in}}%
\pgfpathlineto{\pgfqpoint{4.216300in}{3.124298in}}%
\pgfpathlineto{\pgfqpoint{4.221652in}{3.117623in}}%
\pgfpathlineto{\pgfqpoint{4.224596in}{3.116976in}}%
\pgfpathlineto{\pgfqpoint{4.227539in}{3.118936in}}%
\pgfpathlineto{\pgfqpoint{4.232089in}{3.125442in}}%
\pgfpathlineto{\pgfqpoint{4.237976in}{3.132869in}}%
\pgfpathlineto{\pgfqpoint{4.240920in}{3.133414in}}%
\pgfpathlineto{\pgfqpoint{4.243863in}{3.131361in}}%
\pgfpathlineto{\pgfqpoint{4.248680in}{3.124364in}}%
\pgfpathlineto{\pgfqpoint{4.254033in}{3.117651in}}%
\pgfpathlineto{\pgfqpoint{4.256976in}{3.116966in}}%
\pgfpathlineto{\pgfqpoint{4.259920in}{3.118893in}}%
\pgfpathlineto{\pgfqpoint{4.264469in}{3.125375in}}%
\pgfpathlineto{\pgfqpoint{4.270357in}{3.132843in}}%
\pgfpathlineto{\pgfqpoint{4.273300in}{3.133425in}}%
\pgfpathlineto{\pgfqpoint{4.276244in}{3.131406in}}%
\pgfpathlineto{\pgfqpoint{4.281061in}{3.124430in}}%
\pgfpathlineto{\pgfqpoint{4.286681in}{3.117505in}}%
\pgfpathlineto{\pgfqpoint{4.289624in}{3.117025in}}%
\pgfpathlineto{\pgfqpoint{4.292568in}{3.119137in}}%
\pgfpathlineto{\pgfqpoint{4.297385in}{3.126170in}}%
\pgfpathlineto{\pgfqpoint{4.302737in}{3.132816in}}%
\pgfpathlineto{\pgfqpoint{4.305681in}{3.133436in}}%
\pgfpathlineto{\pgfqpoint{4.308625in}{3.131450in}}%
\pgfpathlineto{\pgfqpoint{4.313174in}{3.124927in}}%
\pgfpathlineto{\pgfqpoint{4.319061in}{3.117530in}}%
\pgfpathlineto{\pgfqpoint{4.322005in}{3.117013in}}%
\pgfpathlineto{\pgfqpoint{4.324949in}{3.119091in}}%
\pgfpathlineto{\pgfqpoint{4.329766in}{3.126104in}}%
\pgfpathlineto{\pgfqpoint{4.335118in}{3.132789in}}%
\pgfpathlineto{\pgfqpoint{4.338061in}{3.133445in}}%
\pgfpathlineto{\pgfqpoint{4.341005in}{3.131494in}}%
\pgfpathlineto{\pgfqpoint{4.345554in}{3.124994in}}%
\pgfpathlineto{\pgfqpoint{4.351442in}{3.117556in}}%
\pgfpathlineto{\pgfqpoint{4.354385in}{3.117002in}}%
\pgfpathlineto{\pgfqpoint{4.357329in}{3.119047in}}%
\pgfpathlineto{\pgfqpoint{4.362146in}{3.126038in}}%
\pgfpathlineto{\pgfqpoint{4.367498in}{3.132761in}}%
\pgfpathlineto{\pgfqpoint{4.370442in}{3.133455in}}%
\pgfpathlineto{\pgfqpoint{4.373118in}{3.131809in}}%
\pgfpathlineto{\pgfqpoint{4.377400in}{3.125922in}}%
\pgfpathlineto{\pgfqpoint{4.383822in}{3.117583in}}%
\pgfpathlineto{\pgfqpoint{4.386766in}{3.116991in}}%
\pgfpathlineto{\pgfqpoint{4.389710in}{3.119002in}}%
\pgfpathlineto{\pgfqpoint{4.394527in}{3.125972in}}%
\pgfpathlineto{\pgfqpoint{4.400146in}{3.132907in}}%
\pgfpathlineto{\pgfqpoint{4.403090in}{3.133397in}}%
\pgfpathlineto{\pgfqpoint{4.406034in}{3.131294in}}%
\pgfpathlineto{\pgfqpoint{4.410851in}{3.124265in}}%
\pgfpathlineto{\pgfqpoint{4.416203in}{3.117609in}}%
\pgfpathlineto{\pgfqpoint{4.419147in}{3.116981in}}%
\pgfpathlineto{\pgfqpoint{4.422090in}{3.118958in}}%
\pgfpathlineto{\pgfqpoint{4.426640in}{3.125475in}}%
\pgfpathlineto{\pgfqpoint{4.432527in}{3.132882in}}%
\pgfpathlineto{\pgfqpoint{4.435471in}{3.133409in}}%
\pgfpathlineto{\pgfqpoint{4.438414in}{3.131339in}}%
\pgfpathlineto{\pgfqpoint{4.443231in}{3.124331in}}%
\pgfpathlineto{\pgfqpoint{4.448583in}{3.117637in}}%
\pgfpathlineto{\pgfqpoint{4.451527in}{3.116971in}}%
\pgfpathlineto{\pgfqpoint{4.454471in}{3.118914in}}%
\pgfpathlineto{\pgfqpoint{4.459020in}{3.125409in}}%
\pgfpathlineto{\pgfqpoint{4.464908in}{3.132856in}}%
\pgfpathlineto{\pgfqpoint{4.467851in}{3.133420in}}%
\pgfpathlineto{\pgfqpoint{4.470795in}{3.131383in}}%
\pgfpathlineto{\pgfqpoint{4.475612in}{3.124397in}}%
\pgfpathlineto{\pgfqpoint{4.481232in}{3.117493in}}%
\pgfpathlineto{\pgfqpoint{4.484175in}{3.117031in}}%
\pgfpathlineto{\pgfqpoint{4.487119in}{3.119159in}}%
\pgfpathlineto{\pgfqpoint{4.491936in}{3.126203in}}%
\pgfpathlineto{\pgfqpoint{4.497288in}{3.132830in}}%
\pgfpathlineto{\pgfqpoint{4.500232in}{3.133430in}}%
\pgfpathlineto{\pgfqpoint{4.503175in}{3.131428in}}%
\pgfpathlineto{\pgfqpoint{4.507992in}{3.124463in}}%
\pgfpathlineto{\pgfqpoint{4.513612in}{3.117518in}}%
\pgfpathlineto{\pgfqpoint{4.516556in}{3.117019in}}%
\pgfpathlineto{\pgfqpoint{4.519500in}{3.119114in}}%
\pgfpathlineto{\pgfqpoint{4.524316in}{3.126137in}}%
\pgfpathlineto{\pgfqpoint{4.529669in}{3.132803in}}%
\pgfpathlineto{\pgfqpoint{4.532612in}{3.133441in}}%
\pgfpathlineto{\pgfqpoint{4.535556in}{3.131472in}}%
\pgfpathlineto{\pgfqpoint{4.540105in}{3.124960in}}%
\pgfpathlineto{\pgfqpoint{4.545993in}{3.117543in}}%
\pgfpathlineto{\pgfqpoint{4.548936in}{3.117007in}}%
\pgfpathlineto{\pgfqpoint{4.551880in}{3.119069in}}%
\pgfpathlineto{\pgfqpoint{4.556697in}{3.126071in}}%
\pgfpathlineto{\pgfqpoint{4.562049in}{3.132775in}}%
\pgfpathlineto{\pgfqpoint{4.564993in}{3.133450in}}%
\pgfpathlineto{\pgfqpoint{4.567937in}{3.131515in}}%
\pgfpathlineto{\pgfqpoint{4.572486in}{3.125027in}}%
\pgfpathlineto{\pgfqpoint{4.578373in}{3.117569in}}%
\pgfpathlineto{\pgfqpoint{4.581317in}{3.116996in}}%
\pgfpathlineto{\pgfqpoint{4.584261in}{3.119024in}}%
\pgfpathlineto{\pgfqpoint{4.589078in}{3.126005in}}%
\pgfpathlineto{\pgfqpoint{4.594697in}{3.132920in}}%
\pgfpathlineto{\pgfqpoint{4.597641in}{3.133391in}}%
\pgfpathlineto{\pgfqpoint{4.600585in}{3.131271in}}%
\pgfpathlineto{\pgfqpoint{4.605402in}{3.124232in}}%
\pgfpathlineto{\pgfqpoint{4.610754in}{3.117596in}}%
\pgfpathlineto{\pgfqpoint{4.613697in}{3.116986in}}%
\pgfpathlineto{\pgfqpoint{4.616641in}{3.118980in}}%
\pgfpathlineto{\pgfqpoint{4.621190in}{3.125508in}}%
\pgfpathlineto{\pgfqpoint{4.627078in}{3.132895in}}%
\pgfpathlineto{\pgfqpoint{4.630022in}{3.133403in}}%
\pgfpathlineto{\pgfqpoint{4.632965in}{3.131316in}}%
\pgfpathlineto{\pgfqpoint{4.637782in}{3.124298in}}%
\pgfpathlineto{\pgfqpoint{4.643134in}{3.117623in}}%
\pgfpathlineto{\pgfqpoint{4.646078in}{3.116976in}}%
\pgfpathlineto{\pgfqpoint{4.649022in}{3.118936in}}%
\pgfpathlineto{\pgfqpoint{4.653571in}{3.125442in}}%
\pgfpathlineto{\pgfqpoint{4.659458in}{3.132869in}}%
\pgfpathlineto{\pgfqpoint{4.662402in}{3.133414in}}%
\pgfpathlineto{\pgfqpoint{4.665346in}{3.131361in}}%
\pgfpathlineto{\pgfqpoint{4.670163in}{3.124364in}}%
\pgfpathlineto{\pgfqpoint{4.675515in}{3.117651in}}%
\pgfpathlineto{\pgfqpoint{4.678459in}{3.116966in}}%
\pgfpathlineto{\pgfqpoint{4.681402in}{3.118893in}}%
\pgfpathlineto{\pgfqpoint{4.685952in}{3.125375in}}%
\pgfpathlineto{\pgfqpoint{4.691839in}{3.132843in}}%
\pgfpathlineto{\pgfqpoint{4.694783in}{3.133425in}}%
\pgfpathlineto{\pgfqpoint{4.697726in}{3.131406in}}%
\pgfpathlineto{\pgfqpoint{4.702543in}{3.124430in}}%
\pgfpathlineto{\pgfqpoint{4.708163in}{3.117505in}}%
\pgfpathlineto{\pgfqpoint{4.711107in}{3.117025in}}%
\pgfpathlineto{\pgfqpoint{4.714050in}{3.119137in}}%
\pgfpathlineto{\pgfqpoint{4.718867in}{3.126170in}}%
\pgfpathlineto{\pgfqpoint{4.724219in}{3.132816in}}%
\pgfpathlineto{\pgfqpoint{4.727163in}{3.133436in}}%
\pgfpathlineto{\pgfqpoint{4.730107in}{3.131450in}}%
\pgfpathlineto{\pgfqpoint{4.734656in}{3.124927in}}%
\pgfpathlineto{\pgfqpoint{4.740544in}{3.117530in}}%
\pgfpathlineto{\pgfqpoint{4.743487in}{3.117013in}}%
\pgfpathlineto{\pgfqpoint{4.746431in}{3.119091in}}%
\pgfpathlineto{\pgfqpoint{4.751248in}{3.126104in}}%
\pgfpathlineto{\pgfqpoint{4.756600in}{3.132789in}}%
\pgfpathlineto{\pgfqpoint{4.759544in}{3.133445in}}%
\pgfpathlineto{\pgfqpoint{4.762487in}{3.131494in}}%
\pgfpathlineto{\pgfqpoint{4.767037in}{3.124994in}}%
\pgfpathlineto{\pgfqpoint{4.772924in}{3.117556in}}%
\pgfpathlineto{\pgfqpoint{4.775868in}{3.117002in}}%
\pgfpathlineto{\pgfqpoint{4.778811in}{3.119047in}}%
\pgfpathlineto{\pgfqpoint{4.783628in}{3.126038in}}%
\pgfpathlineto{\pgfqpoint{4.788981in}{3.132761in}}%
\pgfpathlineto{\pgfqpoint{4.791924in}{3.133455in}}%
\pgfpathlineto{\pgfqpoint{4.794600in}{3.131809in}}%
\pgfpathlineto{\pgfqpoint{4.798882in}{3.125922in}}%
\pgfpathlineto{\pgfqpoint{4.805305in}{3.117583in}}%
\pgfpathlineto{\pgfqpoint{4.808248in}{3.116991in}}%
\pgfpathlineto{\pgfqpoint{4.811192in}{3.119002in}}%
\pgfpathlineto{\pgfqpoint{4.816009in}{3.125972in}}%
\pgfpathlineto{\pgfqpoint{4.821629in}{3.132907in}}%
\pgfpathlineto{\pgfqpoint{4.824572in}{3.133397in}}%
\pgfpathlineto{\pgfqpoint{4.827516in}{3.131294in}}%
\pgfpathlineto{\pgfqpoint{4.832333in}{3.124265in}}%
\pgfpathlineto{\pgfqpoint{4.837685in}{3.117609in}}%
\pgfpathlineto{\pgfqpoint{4.840629in}{3.116981in}}%
\pgfpathlineto{\pgfqpoint{4.843573in}{3.118958in}}%
\pgfpathlineto{\pgfqpoint{4.848122in}{3.125475in}}%
\pgfpathlineto{\pgfqpoint{4.854009in}{3.132882in}}%
\pgfpathlineto{\pgfqpoint{4.856953in}{3.133409in}}%
\pgfpathlineto{\pgfqpoint{4.859897in}{3.131339in}}%
\pgfpathlineto{\pgfqpoint{4.864714in}{3.124331in}}%
\pgfpathlineto{\pgfqpoint{4.870066in}{3.117637in}}%
\pgfpathlineto{\pgfqpoint{4.873009in}{3.116971in}}%
\pgfpathlineto{\pgfqpoint{4.875953in}{3.118914in}}%
\pgfpathlineto{\pgfqpoint{4.880502in}{3.125409in}}%
\pgfpathlineto{\pgfqpoint{4.886390in}{3.132856in}}%
\pgfpathlineto{\pgfqpoint{4.889334in}{3.133420in}}%
\pgfpathlineto{\pgfqpoint{4.892277in}{3.131383in}}%
\pgfpathlineto{\pgfqpoint{4.897094in}{3.124397in}}%
\pgfpathlineto{\pgfqpoint{4.902714in}{3.117493in}}%
\pgfpathlineto{\pgfqpoint{4.905658in}{3.117031in}}%
\pgfpathlineto{\pgfqpoint{4.908601in}{3.119159in}}%
\pgfpathlineto{\pgfqpoint{4.913418in}{3.126203in}}%
\pgfpathlineto{\pgfqpoint{4.918770in}{3.132830in}}%
\pgfpathlineto{\pgfqpoint{4.921714in}{3.133430in}}%
\pgfpathlineto{\pgfqpoint{4.924658in}{3.131428in}}%
\pgfpathlineto{\pgfqpoint{4.929475in}{3.124463in}}%
\pgfpathlineto{\pgfqpoint{4.935094in}{3.117518in}}%
\pgfpathlineto{\pgfqpoint{4.938038in}{3.117019in}}%
\pgfpathlineto{\pgfqpoint{4.940982in}{3.119114in}}%
\pgfpathlineto{\pgfqpoint{4.945799in}{3.126137in}}%
\pgfpathlineto{\pgfqpoint{4.951151in}{3.132803in}}%
\pgfpathlineto{\pgfqpoint{4.954095in}{3.133441in}}%
\pgfpathlineto{\pgfqpoint{4.957038in}{3.131472in}}%
\pgfpathlineto{\pgfqpoint{4.961588in}{3.124960in}}%
\pgfpathlineto{\pgfqpoint{4.967475in}{3.117543in}}%
\pgfpathlineto{\pgfqpoint{4.970419in}{3.117007in}}%
\pgfpathlineto{\pgfqpoint{4.973362in}{3.119069in}}%
\pgfpathlineto{\pgfqpoint{4.978179in}{3.126071in}}%
\pgfpathlineto{\pgfqpoint{4.983531in}{3.132775in}}%
\pgfpathlineto{\pgfqpoint{4.986475in}{3.133450in}}%
\pgfpathlineto{\pgfqpoint{4.989419in}{3.131515in}}%
\pgfpathlineto{\pgfqpoint{4.993968in}{3.125027in}}%
\pgfpathlineto{\pgfqpoint{4.999856in}{3.117569in}}%
\pgfpathlineto{\pgfqpoint{5.002799in}{3.116996in}}%
\pgfpathlineto{\pgfqpoint{5.005743in}{3.119024in}}%
\pgfpathlineto{\pgfqpoint{5.010560in}{3.126005in}}%
\pgfpathlineto{\pgfqpoint{5.016180in}{3.132920in}}%
\pgfpathlineto{\pgfqpoint{5.019123in}{3.133391in}}%
\pgfpathlineto{\pgfqpoint{5.022067in}{3.131271in}}%
\pgfpathlineto{\pgfqpoint{5.026884in}{3.124232in}}%
\pgfpathlineto{\pgfqpoint{5.032236in}{3.117596in}}%
\pgfpathlineto{\pgfqpoint{5.035180in}{3.116986in}}%
\pgfpathlineto{\pgfqpoint{5.038123in}{3.118980in}}%
\pgfpathlineto{\pgfqpoint{5.042673in}{3.125508in}}%
\pgfpathlineto{\pgfqpoint{5.048560in}{3.132895in}}%
\pgfpathlineto{\pgfqpoint{5.051504in}{3.133403in}}%
\pgfpathlineto{\pgfqpoint{5.054448in}{3.131316in}}%
\pgfpathlineto{\pgfqpoint{5.059264in}{3.124298in}}%
\pgfpathlineto{\pgfqpoint{5.064617in}{3.117623in}}%
\pgfpathlineto{\pgfqpoint{5.067560in}{3.116976in}}%
\pgfpathlineto{\pgfqpoint{5.070504in}{3.118936in}}%
\pgfpathlineto{\pgfqpoint{5.075053in}{3.125442in}}%
\pgfpathlineto{\pgfqpoint{5.080941in}{3.132869in}}%
\pgfpathlineto{\pgfqpoint{5.083884in}{3.133414in}}%
\pgfpathlineto{\pgfqpoint{5.086828in}{3.131361in}}%
\pgfpathlineto{\pgfqpoint{5.091645in}{3.124364in}}%
\pgfpathlineto{\pgfqpoint{5.096997in}{3.117651in}}%
\pgfpathlineto{\pgfqpoint{5.099941in}{3.116966in}}%
\pgfpathlineto{\pgfqpoint{5.102885in}{3.118893in}}%
\pgfpathlineto{\pgfqpoint{5.107434in}{3.125375in}}%
\pgfpathlineto{\pgfqpoint{5.113321in}{3.132843in}}%
\pgfpathlineto{\pgfqpoint{5.116265in}{3.133425in}}%
\pgfpathlineto{\pgfqpoint{5.119209in}{3.131406in}}%
\pgfpathlineto{\pgfqpoint{5.124026in}{3.124430in}}%
\pgfpathlineto{\pgfqpoint{5.129645in}{3.117505in}}%
\pgfpathlineto{\pgfqpoint{5.132589in}{3.117025in}}%
\pgfpathlineto{\pgfqpoint{5.135533in}{3.119137in}}%
\pgfpathlineto{\pgfqpoint{5.140350in}{3.126170in}}%
\pgfpathlineto{\pgfqpoint{5.145702in}{3.132816in}}%
\pgfpathlineto{\pgfqpoint{5.148645in}{3.133436in}}%
\pgfpathlineto{\pgfqpoint{5.151589in}{3.131450in}}%
\pgfpathlineto{\pgfqpoint{5.156139in}{3.124927in}}%
\pgfpathlineto{\pgfqpoint{5.162026in}{3.117530in}}%
\pgfpathlineto{\pgfqpoint{5.164970in}{3.117013in}}%
\pgfpathlineto{\pgfqpoint{5.167913in}{3.119091in}}%
\pgfpathlineto{\pgfqpoint{5.172730in}{3.126104in}}%
\pgfpathlineto{\pgfqpoint{5.178082in}{3.132789in}}%
\pgfpathlineto{\pgfqpoint{5.181026in}{3.133445in}}%
\pgfpathlineto{\pgfqpoint{5.183970in}{3.131494in}}%
\pgfpathlineto{\pgfqpoint{5.188519in}{3.124994in}}%
\pgfpathlineto{\pgfqpoint{5.194406in}{3.117556in}}%
\pgfpathlineto{\pgfqpoint{5.197350in}{3.117002in}}%
\pgfpathlineto{\pgfqpoint{5.200294in}{3.119047in}}%
\pgfpathlineto{\pgfqpoint{5.205111in}{3.126038in}}%
\pgfpathlineto{\pgfqpoint{5.210463in}{3.132761in}}%
\pgfpathlineto{\pgfqpoint{5.213407in}{3.133455in}}%
\pgfpathlineto{\pgfqpoint{5.216083in}{3.131809in}}%
\pgfpathlineto{\pgfqpoint{5.220364in}{3.125922in}}%
\pgfpathlineto{\pgfqpoint{5.226787in}{3.117583in}}%
\pgfpathlineto{\pgfqpoint{5.229731in}{3.116991in}}%
\pgfpathlineto{\pgfqpoint{5.232674in}{3.119002in}}%
\pgfpathlineto{\pgfqpoint{5.237491in}{3.125972in}}%
\pgfpathlineto{\pgfqpoint{5.243111in}{3.132907in}}%
\pgfpathlineto{\pgfqpoint{5.246055in}{3.133397in}}%
\pgfpathlineto{\pgfqpoint{5.248998in}{3.131294in}}%
\pgfpathlineto{\pgfqpoint{5.253815in}{3.124265in}}%
\pgfpathlineto{\pgfqpoint{5.259168in}{3.117609in}}%
\pgfpathlineto{\pgfqpoint{5.262111in}{3.116981in}}%
\pgfpathlineto{\pgfqpoint{5.265055in}{3.118958in}}%
\pgfpathlineto{\pgfqpoint{5.269604in}{3.125475in}}%
\pgfpathlineto{\pgfqpoint{5.275492in}{3.132882in}}%
\pgfpathlineto{\pgfqpoint{5.278435in}{3.133409in}}%
\pgfpathlineto{\pgfqpoint{5.281379in}{3.131339in}}%
\pgfpathlineto{\pgfqpoint{5.286196in}{3.124331in}}%
\pgfpathlineto{\pgfqpoint{5.291548in}{3.117637in}}%
\pgfpathlineto{\pgfqpoint{5.294492in}{3.116971in}}%
\pgfpathlineto{\pgfqpoint{5.297435in}{3.118914in}}%
\pgfpathlineto{\pgfqpoint{5.301985in}{3.125409in}}%
\pgfpathlineto{\pgfqpoint{5.307872in}{3.132856in}}%
\pgfpathlineto{\pgfqpoint{5.310816in}{3.133420in}}%
\pgfpathlineto{\pgfqpoint{5.313760in}{3.131383in}}%
\pgfpathlineto{\pgfqpoint{5.318576in}{3.124397in}}%
\pgfpathlineto{\pgfqpoint{5.324196in}{3.117493in}}%
\pgfpathlineto{\pgfqpoint{5.327140in}{3.117031in}}%
\pgfpathlineto{\pgfqpoint{5.330084in}{3.119159in}}%
\pgfpathlineto{\pgfqpoint{5.334901in}{3.126203in}}%
\pgfpathlineto{\pgfqpoint{5.340253in}{3.132830in}}%
\pgfpathlineto{\pgfqpoint{5.343196in}{3.133430in}}%
\pgfpathlineto{\pgfqpoint{5.346140in}{3.131428in}}%
\pgfpathlineto{\pgfqpoint{5.350957in}{3.124463in}}%
\pgfpathlineto{\pgfqpoint{5.356577in}{3.117518in}}%
\pgfpathlineto{\pgfqpoint{5.359520in}{3.117019in}}%
\pgfpathlineto{\pgfqpoint{5.362464in}{3.119114in}}%
\pgfpathlineto{\pgfqpoint{5.367281in}{3.126137in}}%
\pgfpathlineto{\pgfqpoint{5.372633in}{3.132803in}}%
\pgfpathlineto{\pgfqpoint{5.375577in}{3.133441in}}%
\pgfpathlineto{\pgfqpoint{5.378521in}{3.131472in}}%
\pgfpathlineto{\pgfqpoint{5.383070in}{3.124960in}}%
\pgfpathlineto{\pgfqpoint{5.388957in}{3.117543in}}%
\pgfpathlineto{\pgfqpoint{5.391901in}{3.117007in}}%
\pgfpathlineto{\pgfqpoint{5.394845in}{3.119069in}}%
\pgfpathlineto{\pgfqpoint{5.399662in}{3.126071in}}%
\pgfpathlineto{\pgfqpoint{5.405014in}{3.132775in}}%
\pgfpathlineto{\pgfqpoint{5.407957in}{3.133450in}}%
\pgfpathlineto{\pgfqpoint{5.410901in}{3.131515in}}%
\pgfpathlineto{\pgfqpoint{5.415450in}{3.125027in}}%
\pgfpathlineto{\pgfqpoint{5.421338in}{3.117569in}}%
\pgfpathlineto{\pgfqpoint{5.424282in}{3.116996in}}%
\pgfpathlineto{\pgfqpoint{5.427225in}{3.119024in}}%
\pgfpathlineto{\pgfqpoint{5.432042in}{3.126005in}}%
\pgfpathlineto{\pgfqpoint{5.437662in}{3.132920in}}%
\pgfpathlineto{\pgfqpoint{5.440606in}{3.133391in}}%
\pgfpathlineto{\pgfqpoint{5.443549in}{3.131271in}}%
\pgfpathlineto{\pgfqpoint{5.448366in}{3.124232in}}%
\pgfpathlineto{\pgfqpoint{5.453718in}{3.117596in}}%
\pgfpathlineto{\pgfqpoint{5.456662in}{3.116986in}}%
\pgfpathlineto{\pgfqpoint{5.459606in}{3.118980in}}%
\pgfpathlineto{\pgfqpoint{5.464155in}{3.125508in}}%
\pgfpathlineto{\pgfqpoint{5.470042in}{3.132895in}}%
\pgfpathlineto{\pgfqpoint{5.472986in}{3.133403in}}%
\pgfpathlineto{\pgfqpoint{5.475930in}{3.131316in}}%
\pgfpathlineto{\pgfqpoint{5.480747in}{3.124298in}}%
\pgfpathlineto{\pgfqpoint{5.486099in}{3.117623in}}%
\pgfpathlineto{\pgfqpoint{5.489043in}{3.116976in}}%
\pgfpathlineto{\pgfqpoint{5.491986in}{3.118936in}}%
\pgfpathlineto{\pgfqpoint{5.496536in}{3.125442in}}%
\pgfpathlineto{\pgfqpoint{5.502423in}{3.132869in}}%
\pgfpathlineto{\pgfqpoint{5.505367in}{3.133414in}}%
\pgfpathlineto{\pgfqpoint{5.508310in}{3.131361in}}%
\pgfpathlineto{\pgfqpoint{5.513127in}{3.124364in}}%
\pgfpathlineto{\pgfqpoint{5.518479in}{3.117651in}}%
\pgfpathlineto{\pgfqpoint{5.521423in}{3.116966in}}%
\pgfpathlineto{\pgfqpoint{5.524367in}{3.118893in}}%
\pgfpathlineto{\pgfqpoint{5.528916in}{3.125375in}}%
\pgfpathlineto{\pgfqpoint{5.534804in}{3.132843in}}%
\pgfpathlineto{\pgfqpoint{5.537747in}{3.133425in}}%
\pgfpathlineto{\pgfqpoint{5.540691in}{3.131406in}}%
\pgfpathlineto{\pgfqpoint{5.545508in}{3.124430in}}%
\pgfpathlineto{\pgfqpoint{5.551128in}{3.117505in}}%
\pgfpathlineto{\pgfqpoint{5.554071in}{3.117025in}}%
\pgfpathlineto{\pgfqpoint{5.557015in}{3.119137in}}%
\pgfpathlineto{\pgfqpoint{5.561832in}{3.126170in}}%
\pgfpathlineto{\pgfqpoint{5.567184in}{3.132816in}}%
\pgfpathlineto{\pgfqpoint{5.570128in}{3.133436in}}%
\pgfpathlineto{\pgfqpoint{5.573071in}{3.131450in}}%
\pgfpathlineto{\pgfqpoint{5.577621in}{3.124927in}}%
\pgfpathlineto{\pgfqpoint{5.583508in}{3.117530in}}%
\pgfpathlineto{\pgfqpoint{5.586452in}{3.117013in}}%
\pgfpathlineto{\pgfqpoint{5.589396in}{3.119091in}}%
\pgfpathlineto{\pgfqpoint{5.594213in}{3.126104in}}%
\pgfpathlineto{\pgfqpoint{5.599565in}{3.132789in}}%
\pgfpathlineto{\pgfqpoint{5.602508in}{3.133445in}}%
\pgfpathlineto{\pgfqpoint{5.605452in}{3.131494in}}%
\pgfpathlineto{\pgfqpoint{5.610001in}{3.124994in}}%
\pgfpathlineto{\pgfqpoint{5.615889in}{3.117556in}}%
\pgfpathlineto{\pgfqpoint{5.618832in}{3.117002in}}%
\pgfpathlineto{\pgfqpoint{5.621776in}{3.119047in}}%
\pgfpathlineto{\pgfqpoint{5.626593in}{3.126038in}}%
\pgfpathlineto{\pgfqpoint{5.631945in}{3.132761in}}%
\pgfpathlineto{\pgfqpoint{5.634889in}{3.133455in}}%
\pgfpathlineto{\pgfqpoint{5.637565in}{3.131809in}}%
\pgfpathlineto{\pgfqpoint{5.641847in}{3.125922in}}%
\pgfpathlineto{\pgfqpoint{5.648269in}{3.117583in}}%
\pgfpathlineto{\pgfqpoint{5.651213in}{3.116991in}}%
\pgfpathlineto{\pgfqpoint{5.654157in}{3.119002in}}%
\pgfpathlineto{\pgfqpoint{5.658974in}{3.125972in}}%
\pgfpathlineto{\pgfqpoint{5.664593in}{3.132907in}}%
\pgfpathlineto{\pgfqpoint{5.667537in}{3.133397in}}%
\pgfpathlineto{\pgfqpoint{5.670481in}{3.131294in}}%
\pgfpathlineto{\pgfqpoint{5.675298in}{3.124265in}}%
\pgfpathlineto{\pgfqpoint{5.680650in}{3.117609in}}%
\pgfpathlineto{\pgfqpoint{5.683594in}{3.116981in}}%
\pgfpathlineto{\pgfqpoint{5.686537in}{3.118958in}}%
\pgfpathlineto{\pgfqpoint{5.691087in}{3.125475in}}%
\pgfpathlineto{\pgfqpoint{5.696974in}{3.132882in}}%
\pgfpathlineto{\pgfqpoint{5.699918in}{3.133409in}}%
\pgfpathlineto{\pgfqpoint{5.702861in}{3.131339in}}%
\pgfpathlineto{\pgfqpoint{5.707678in}{3.124331in}}%
\pgfpathlineto{\pgfqpoint{5.713030in}{3.117637in}}%
\pgfpathlineto{\pgfqpoint{5.715974in}{3.116971in}}%
\pgfpathlineto{\pgfqpoint{5.718918in}{3.118914in}}%
\pgfpathlineto{\pgfqpoint{5.723467in}{3.125409in}}%
\pgfpathlineto{\pgfqpoint{5.729354in}{3.132856in}}%
\pgfpathlineto{\pgfqpoint{5.732298in}{3.133420in}}%
\pgfpathlineto{\pgfqpoint{5.735242in}{3.131383in}}%
\pgfpathlineto{\pgfqpoint{5.740059in}{3.124397in}}%
\pgfpathlineto{\pgfqpoint{5.745679in}{3.117493in}}%
\pgfpathlineto{\pgfqpoint{5.748622in}{3.117031in}}%
\pgfpathlineto{\pgfqpoint{5.751566in}{3.119159in}}%
\pgfpathlineto{\pgfqpoint{5.756383in}{3.126203in}}%
\pgfpathlineto{\pgfqpoint{5.761735in}{3.132830in}}%
\pgfpathlineto{\pgfqpoint{5.764679in}{3.133430in}}%
\pgfpathlineto{\pgfqpoint{5.767622in}{3.131428in}}%
\pgfpathlineto{\pgfqpoint{5.772439in}{3.124463in}}%
\pgfpathlineto{\pgfqpoint{5.778059in}{3.117518in}}%
\pgfpathlineto{\pgfqpoint{5.781003in}{3.117019in}}%
\pgfpathlineto{\pgfqpoint{5.783946in}{3.119114in}}%
\pgfpathlineto{\pgfqpoint{5.788763in}{3.126137in}}%
\pgfpathlineto{\pgfqpoint{5.794116in}{3.132803in}}%
\pgfpathlineto{\pgfqpoint{5.797059in}{3.133441in}}%
\pgfpathlineto{\pgfqpoint{5.800003in}{3.131472in}}%
\pgfpathlineto{\pgfqpoint{5.804552in}{3.124960in}}%
\pgfpathlineto{\pgfqpoint{5.810440in}{3.117543in}}%
\pgfpathlineto{\pgfqpoint{5.813383in}{3.117007in}}%
\pgfpathlineto{\pgfqpoint{5.816327in}{3.119069in}}%
\pgfpathlineto{\pgfqpoint{5.821144in}{3.126071in}}%
\pgfpathlineto{\pgfqpoint{5.826496in}{3.132775in}}%
\pgfpathlineto{\pgfqpoint{5.829440in}{3.133450in}}%
\pgfpathlineto{\pgfqpoint{5.832383in}{3.131515in}}%
\pgfpathlineto{\pgfqpoint{5.836933in}{3.125027in}}%
\pgfpathlineto{\pgfqpoint{5.842820in}{3.117569in}}%
\pgfpathlineto{\pgfqpoint{5.845764in}{3.116996in}}%
\pgfpathlineto{\pgfqpoint{5.848708in}{3.119024in}}%
\pgfpathlineto{\pgfqpoint{5.853524in}{3.126005in}}%
\pgfpathlineto{\pgfqpoint{5.859144in}{3.132920in}}%
\pgfpathlineto{\pgfqpoint{5.862088in}{3.133391in}}%
\pgfpathlineto{\pgfqpoint{5.865032in}{3.131271in}}%
\pgfpathlineto{\pgfqpoint{5.869849in}{3.124232in}}%
\pgfpathlineto{\pgfqpoint{5.875201in}{3.117596in}}%
\pgfpathlineto{\pgfqpoint{5.878144in}{3.116986in}}%
\pgfpathlineto{\pgfqpoint{5.881088in}{3.118980in}}%
\pgfpathlineto{\pgfqpoint{5.885637in}{3.125508in}}%
\pgfpathlineto{\pgfqpoint{5.891525in}{3.132895in}}%
\pgfpathlineto{\pgfqpoint{5.894468in}{3.133403in}}%
\pgfpathlineto{\pgfqpoint{5.897412in}{3.131316in}}%
\pgfpathlineto{\pgfqpoint{5.902229in}{3.124298in}}%
\pgfpathlineto{\pgfqpoint{5.907581in}{3.117623in}}%
\pgfpathlineto{\pgfqpoint{5.910525in}{3.116976in}}%
\pgfpathlineto{\pgfqpoint{5.913469in}{3.118936in}}%
\pgfpathlineto{\pgfqpoint{5.918018in}{3.125442in}}%
\pgfpathlineto{\pgfqpoint{5.923905in}{3.132869in}}%
\pgfpathlineto{\pgfqpoint{5.926849in}{3.133414in}}%
\pgfpathlineto{\pgfqpoint{5.929793in}{3.131361in}}%
\pgfpathlineto{\pgfqpoint{5.934610in}{3.124364in}}%
\pgfpathlineto{\pgfqpoint{5.939962in}{3.117651in}}%
\pgfpathlineto{\pgfqpoint{5.942906in}{3.116966in}}%
\pgfpathlineto{\pgfqpoint{5.945849in}{3.118893in}}%
\pgfpathlineto{\pgfqpoint{5.950399in}{3.125375in}}%
\pgfpathlineto{\pgfqpoint{5.956286in}{3.132843in}}%
\pgfpathlineto{\pgfqpoint{5.959230in}{3.133425in}}%
\pgfpathlineto{\pgfqpoint{5.962173in}{3.131406in}}%
\pgfpathlineto{\pgfqpoint{5.966990in}{3.124430in}}%
\pgfpathlineto{\pgfqpoint{5.972610in}{3.117505in}}%
\pgfpathlineto{\pgfqpoint{5.975554in}{3.117025in}}%
\pgfpathlineto{\pgfqpoint{5.978497in}{3.119137in}}%
\pgfpathlineto{\pgfqpoint{5.983314in}{3.126170in}}%
\pgfpathlineto{\pgfqpoint{5.988666in}{3.132816in}}%
\pgfpathlineto{\pgfqpoint{5.991610in}{3.133436in}}%
\pgfpathlineto{\pgfqpoint{5.994554in}{3.131450in}}%
\pgfpathlineto{\pgfqpoint{5.999103in}{3.124927in}}%
\pgfpathlineto{\pgfqpoint{6.004991in}{3.117530in}}%
\pgfpathlineto{\pgfqpoint{6.007934in}{3.117013in}}%
\pgfpathlineto{\pgfqpoint{6.010878in}{3.119091in}}%
\pgfpathlineto{\pgfqpoint{6.015695in}{3.126104in}}%
\pgfpathlineto{\pgfqpoint{6.021047in}{3.132789in}}%
\pgfpathlineto{\pgfqpoint{6.023991in}{3.133445in}}%
\pgfpathlineto{\pgfqpoint{6.026934in}{3.131494in}}%
\pgfpathlineto{\pgfqpoint{6.031484in}{3.124994in}}%
\pgfpathlineto{\pgfqpoint{6.037371in}{3.117556in}}%
\pgfpathlineto{\pgfqpoint{6.040315in}{3.117002in}}%
\pgfpathlineto{\pgfqpoint{6.043258in}{3.119047in}}%
\pgfpathlineto{\pgfqpoint{6.048075in}{3.126038in}}%
\pgfpathlineto{\pgfqpoint{6.053428in}{3.132761in}}%
\pgfpathlineto{\pgfqpoint{6.056371in}{3.133455in}}%
\pgfpathlineto{\pgfqpoint{6.059047in}{3.131809in}}%
\pgfpathlineto{\pgfqpoint{6.063329in}{3.125922in}}%
\pgfpathlineto{\pgfqpoint{6.069752in}{3.117583in}}%
\pgfpathlineto{\pgfqpoint{6.072695in}{3.116991in}}%
\pgfpathlineto{\pgfqpoint{6.075639in}{3.119002in}}%
\pgfpathlineto{\pgfqpoint{6.080456in}{3.125972in}}%
\pgfpathlineto{\pgfqpoint{6.086076in}{3.132907in}}%
\pgfpathlineto{\pgfqpoint{6.089019in}{3.133397in}}%
\pgfpathlineto{\pgfqpoint{6.091963in}{3.131294in}}%
\pgfpathlineto{\pgfqpoint{6.096780in}{3.124265in}}%
\pgfpathlineto{\pgfqpoint{6.102132in}{3.117609in}}%
\pgfpathlineto{\pgfqpoint{6.105076in}{3.116981in}}%
\pgfpathlineto{\pgfqpoint{6.108020in}{3.118958in}}%
\pgfpathlineto{\pgfqpoint{6.112569in}{3.125475in}}%
\pgfpathlineto{\pgfqpoint{6.118456in}{3.132882in}}%
\pgfpathlineto{\pgfqpoint{6.121400in}{3.133409in}}%
\pgfpathlineto{\pgfqpoint{6.124344in}{3.131339in}}%
\pgfpathlineto{\pgfqpoint{6.129161in}{3.124331in}}%
\pgfpathlineto{\pgfqpoint{6.134513in}{3.117637in}}%
\pgfpathlineto{\pgfqpoint{6.137456in}{3.116971in}}%
\pgfpathlineto{\pgfqpoint{6.140400in}{3.118914in}}%
\pgfpathlineto{\pgfqpoint{6.144949in}{3.125409in}}%
\pgfpathlineto{\pgfqpoint{6.150837in}{3.132856in}}%
\pgfpathlineto{\pgfqpoint{6.153780in}{3.133420in}}%
\pgfpathlineto{\pgfqpoint{6.156724in}{3.131383in}}%
\pgfpathlineto{\pgfqpoint{6.161541in}{3.124397in}}%
\pgfpathlineto{\pgfqpoint{6.167161in}{3.117493in}}%
\pgfpathlineto{\pgfqpoint{6.170105in}{3.117031in}}%
\pgfpathlineto{\pgfqpoint{6.173048in}{3.119159in}}%
\pgfpathlineto{\pgfqpoint{6.177865in}{3.126203in}}%
\pgfpathlineto{\pgfqpoint{6.183217in}{3.132830in}}%
\pgfpathlineto{\pgfqpoint{6.186161in}{3.133430in}}%
\pgfpathlineto{\pgfqpoint{6.189105in}{3.131428in}}%
\pgfpathlineto{\pgfqpoint{6.193922in}{3.124463in}}%
\pgfpathlineto{\pgfqpoint{6.199541in}{3.117518in}}%
\pgfpathlineto{\pgfqpoint{6.202485in}{3.117019in}}%
\pgfpathlineto{\pgfqpoint{6.205429in}{3.119114in}}%
\pgfpathlineto{\pgfqpoint{6.210246in}{3.126137in}}%
\pgfpathlineto{\pgfqpoint{6.215598in}{3.132803in}}%
\pgfpathlineto{\pgfqpoint{6.218542in}{3.133441in}}%
\pgfpathlineto{\pgfqpoint{6.221485in}{3.131472in}}%
\pgfpathlineto{\pgfqpoint{6.226035in}{3.124960in}}%
\pgfpathlineto{\pgfqpoint{6.231922in}{3.117543in}}%
\pgfpathlineto{\pgfqpoint{6.234866in}{3.117007in}}%
\pgfpathlineto{\pgfqpoint{6.237809in}{3.119069in}}%
\pgfpathlineto{\pgfqpoint{6.242626in}{3.126071in}}%
\pgfpathlineto{\pgfqpoint{6.247978in}{3.132775in}}%
\pgfpathlineto{\pgfqpoint{6.250922in}{3.133450in}}%
\pgfpathlineto{\pgfqpoint{6.253866in}{3.131515in}}%
\pgfpathlineto{\pgfqpoint{6.258415in}{3.125027in}}%
\pgfpathlineto{\pgfqpoint{6.264302in}{3.117569in}}%
\pgfpathlineto{\pgfqpoint{6.267246in}{3.116996in}}%
\pgfpathlineto{\pgfqpoint{6.270190in}{3.119024in}}%
\pgfpathlineto{\pgfqpoint{6.275007in}{3.126005in}}%
\pgfpathlineto{\pgfqpoint{6.280627in}{3.132920in}}%
\pgfpathlineto{\pgfqpoint{6.283570in}{3.133391in}}%
\pgfpathlineto{\pgfqpoint{6.286514in}{3.131271in}}%
\pgfpathlineto{\pgfqpoint{6.291331in}{3.124232in}}%
\pgfpathlineto{\pgfqpoint{6.296683in}{3.117596in}}%
\pgfpathlineto{\pgfqpoint{6.299627in}{3.116986in}}%
\pgfpathlineto{\pgfqpoint{6.302570in}{3.118980in}}%
\pgfpathlineto{\pgfqpoint{6.307120in}{3.125508in}}%
\pgfpathlineto{\pgfqpoint{6.313007in}{3.132895in}}%
\pgfpathlineto{\pgfqpoint{6.315951in}{3.133403in}}%
\pgfpathlineto{\pgfqpoint{6.318894in}{3.131316in}}%
\pgfpathlineto{\pgfqpoint{6.323711in}{3.124298in}}%
\pgfpathlineto{\pgfqpoint{6.329064in}{3.117623in}}%
\pgfpathlineto{\pgfqpoint{6.332007in}{3.116976in}}%
\pgfpathlineto{\pgfqpoint{6.334951in}{3.118936in}}%
\pgfpathlineto{\pgfqpoint{6.339500in}{3.125442in}}%
\pgfpathlineto{\pgfqpoint{6.345388in}{3.132869in}}%
\pgfpathlineto{\pgfqpoint{6.348331in}{3.133414in}}%
\pgfpathlineto{\pgfqpoint{6.351275in}{3.131361in}}%
\pgfpathlineto{\pgfqpoint{6.356092in}{3.124364in}}%
\pgfpathlineto{\pgfqpoint{6.361444in}{3.117651in}}%
\pgfpathlineto{\pgfqpoint{6.364388in}{3.116966in}}%
\pgfpathlineto{\pgfqpoint{6.367332in}{3.118893in}}%
\pgfpathlineto{\pgfqpoint{6.371881in}{3.125375in}}%
\pgfpathlineto{\pgfqpoint{6.377768in}{3.132843in}}%
\pgfpathlineto{\pgfqpoint{6.380712in}{3.133425in}}%
\pgfpathlineto{\pgfqpoint{6.383656in}{3.131406in}}%
\pgfpathlineto{\pgfqpoint{6.388473in}{3.124430in}}%
\pgfpathlineto{\pgfqpoint{6.394092in}{3.117505in}}%
\pgfpathlineto{\pgfqpoint{6.397036in}{3.117025in}}%
\pgfpathlineto{\pgfqpoint{6.399980in}{3.119137in}}%
\pgfpathlineto{\pgfqpoint{6.404797in}{3.126170in}}%
\pgfpathlineto{\pgfqpoint{6.410149in}{3.132816in}}%
\pgfpathlineto{\pgfqpoint{6.413092in}{3.133436in}}%
\pgfpathlineto{\pgfqpoint{6.416036in}{3.131450in}}%
\pgfpathlineto{\pgfqpoint{6.420585in}{3.124927in}}%
\pgfpathlineto{\pgfqpoint{6.426473in}{3.117530in}}%
\pgfpathlineto{\pgfqpoint{6.429417in}{3.117013in}}%
\pgfpathlineto{\pgfqpoint{6.432360in}{3.119091in}}%
\pgfpathlineto{\pgfqpoint{6.437177in}{3.126104in}}%
\pgfpathlineto{\pgfqpoint{6.442529in}{3.132789in}}%
\pgfpathlineto{\pgfqpoint{6.445473in}{3.133445in}}%
\pgfpathlineto{\pgfqpoint{6.448417in}{3.131494in}}%
\pgfpathlineto{\pgfqpoint{6.452966in}{3.124994in}}%
\pgfpathlineto{\pgfqpoint{6.458853in}{3.117556in}}%
\pgfpathlineto{\pgfqpoint{6.461797in}{3.117002in}}%
\pgfpathlineto{\pgfqpoint{6.464741in}{3.119047in}}%
\pgfpathlineto{\pgfqpoint{6.469558in}{3.126038in}}%
\pgfpathlineto{\pgfqpoint{6.474910in}{3.132761in}}%
\pgfpathlineto{\pgfqpoint{6.477854in}{3.133455in}}%
\pgfpathlineto{\pgfqpoint{6.480530in}{3.131809in}}%
\pgfpathlineto{\pgfqpoint{6.484811in}{3.125922in}}%
\pgfpathlineto{\pgfqpoint{6.491234in}{3.117583in}}%
\pgfpathlineto{\pgfqpoint{6.494178in}{3.116991in}}%
\pgfpathlineto{\pgfqpoint{6.497121in}{3.119002in}}%
\pgfpathlineto{\pgfqpoint{6.501938in}{3.125972in}}%
\pgfpathlineto{\pgfqpoint{6.507558in}{3.132907in}}%
\pgfpathlineto{\pgfqpoint{6.510502in}{3.133397in}}%
\pgfpathlineto{\pgfqpoint{6.513445in}{3.131294in}}%
\pgfpathlineto{\pgfqpoint{6.518262in}{3.124265in}}%
\pgfpathlineto{\pgfqpoint{6.523614in}{3.117609in}}%
\pgfpathlineto{\pgfqpoint{6.526558in}{3.116981in}}%
\pgfpathlineto{\pgfqpoint{6.529502in}{3.118958in}}%
\pgfpathlineto{\pgfqpoint{6.534051in}{3.125475in}}%
\pgfpathlineto{\pgfqpoint{6.539939in}{3.132882in}}%
\pgfpathlineto{\pgfqpoint{6.542882in}{3.133409in}}%
\pgfpathlineto{\pgfqpoint{6.545826in}{3.131339in}}%
\pgfpathlineto{\pgfqpoint{6.550643in}{3.124331in}}%
\pgfpathlineto{\pgfqpoint{6.555995in}{3.117637in}}%
\pgfpathlineto{\pgfqpoint{6.558939in}{3.116971in}}%
\pgfpathlineto{\pgfqpoint{6.561882in}{3.118914in}}%
\pgfpathlineto{\pgfqpoint{6.566432in}{3.125409in}}%
\pgfpathlineto{\pgfqpoint{6.572319in}{3.132856in}}%
\pgfpathlineto{\pgfqpoint{6.575263in}{3.133420in}}%
\pgfpathlineto{\pgfqpoint{6.578206in}{3.131383in}}%
\pgfpathlineto{\pgfqpoint{6.583023in}{3.124397in}}%
\pgfpathlineto{\pgfqpoint{6.588643in}{3.117493in}}%
\pgfpathlineto{\pgfqpoint{6.591587in}{3.117031in}}%
\pgfpathlineto{\pgfqpoint{6.594531in}{3.119159in}}%
\pgfpathlineto{\pgfqpoint{6.599347in}{3.126203in}}%
\pgfpathlineto{\pgfqpoint{6.604700in}{3.132830in}}%
\pgfpathlineto{\pgfqpoint{6.607643in}{3.133430in}}%
\pgfpathlineto{\pgfqpoint{6.610587in}{3.131428in}}%
\pgfpathlineto{\pgfqpoint{6.615404in}{3.124463in}}%
\pgfpathlineto{\pgfqpoint{6.621024in}{3.117518in}}%
\pgfpathlineto{\pgfqpoint{6.623967in}{3.117019in}}%
\pgfpathlineto{\pgfqpoint{6.626911in}{3.119114in}}%
\pgfpathlineto{\pgfqpoint{6.631728in}{3.126137in}}%
\pgfpathlineto{\pgfqpoint{6.637080in}{3.132803in}}%
\pgfpathlineto{\pgfqpoint{6.640024in}{3.133441in}}%
\pgfpathlineto{\pgfqpoint{6.642968in}{3.131472in}}%
\pgfpathlineto{\pgfqpoint{6.647517in}{3.124960in}}%
\pgfpathlineto{\pgfqpoint{6.653404in}{3.117543in}}%
\pgfpathlineto{\pgfqpoint{6.656348in}{3.117007in}}%
\pgfpathlineto{\pgfqpoint{6.659292in}{3.119069in}}%
\pgfpathlineto{\pgfqpoint{6.663306in}{3.124778in}}%
\pgfpathlineto{\pgfqpoint{6.663306in}{3.124778in}}%
\pgfusepath{stroke}%
\end{pgfscope}%
\begin{pgfscope}%
\pgfpathrectangle{\pgfqpoint{0.467797in}{2.292089in}}{\pgfqpoint{6.490533in}{1.666241in}}%
\pgfusepath{clip}%
\pgfsetrectcap%
\pgfsetroundjoin%
\pgfsetlinewidth{1.505625pt}%
\definecolor{currentstroke}{rgb}{0.890196,0.466667,0.760784}%
\pgfsetstrokecolor{currentstroke}%
\pgfsetdash{}{0pt}%
\pgfpathmoveto{\pgfqpoint{0.762821in}{3.125209in}}%
\pgfpathlineto{\pgfqpoint{0.768708in}{3.132695in}}%
\pgfpathlineto{\pgfqpoint{0.771652in}{3.133224in}}%
\pgfpathlineto{\pgfqpoint{0.774596in}{3.131105in}}%
\pgfpathlineto{\pgfqpoint{0.779680in}{3.123612in}}%
\pgfpathlineto{\pgfqpoint{0.784765in}{3.117606in}}%
\pgfpathlineto{\pgfqpoint{0.787708in}{3.117256in}}%
\pgfpathlineto{\pgfqpoint{0.790652in}{3.119534in}}%
\pgfpathlineto{\pgfqpoint{0.796004in}{3.127530in}}%
\pgfpathlineto{\pgfqpoint{0.800821in}{3.132920in}}%
\pgfpathlineto{\pgfqpoint{0.803497in}{3.133187in}}%
\pgfpathlineto{\pgfqpoint{0.806441in}{3.130969in}}%
\pgfpathlineto{\pgfqpoint{0.811793in}{3.123003in}}%
\pgfpathlineto{\pgfqpoint{0.816610in}{3.117538in}}%
\pgfpathlineto{\pgfqpoint{0.819554in}{3.117300in}}%
\pgfpathlineto{\pgfqpoint{0.822497in}{3.119675in}}%
\pgfpathlineto{\pgfqpoint{0.828385in}{3.128522in}}%
\pgfpathlineto{\pgfqpoint{0.832667in}{3.132980in}}%
\pgfpathlineto{\pgfqpoint{0.835343in}{3.133146in}}%
\pgfpathlineto{\pgfqpoint{0.838286in}{3.130830in}}%
\pgfpathlineto{\pgfqpoint{0.843906in}{3.122407in}}%
\pgfpathlineto{\pgfqpoint{0.848455in}{3.117475in}}%
\pgfpathlineto{\pgfqpoint{0.851131in}{3.117247in}}%
\pgfpathlineto{\pgfqpoint{0.854075in}{3.119504in}}%
\pgfpathlineto{\pgfqpoint{0.859427in}{3.127490in}}%
\pgfpathlineto{\pgfqpoint{0.864244in}{3.132906in}}%
\pgfpathlineto{\pgfqpoint{0.866920in}{3.133196in}}%
\pgfpathlineto{\pgfqpoint{0.869864in}{3.130999in}}%
\pgfpathlineto{\pgfqpoint{0.874949in}{3.123462in}}%
\pgfpathlineto{\pgfqpoint{0.880033in}{3.117552in}}%
\pgfpathlineto{\pgfqpoint{0.882977in}{3.117290in}}%
\pgfpathlineto{\pgfqpoint{0.885921in}{3.119645in}}%
\pgfpathlineto{\pgfqpoint{0.891808in}{3.128484in}}%
\pgfpathlineto{\pgfqpoint{0.896090in}{3.132968in}}%
\pgfpathlineto{\pgfqpoint{0.898766in}{3.133155in}}%
\pgfpathlineto{\pgfqpoint{0.901709in}{3.130860in}}%
\pgfpathlineto{\pgfqpoint{0.907329in}{3.122446in}}%
\pgfpathlineto{\pgfqpoint{0.911878in}{3.117488in}}%
\pgfpathlineto{\pgfqpoint{0.914555in}{3.117238in}}%
\pgfpathlineto{\pgfqpoint{0.917498in}{3.119474in}}%
\pgfpathlineto{\pgfqpoint{0.922850in}{3.127450in}}%
\pgfpathlineto{\pgfqpoint{0.927667in}{3.132892in}}%
\pgfpathlineto{\pgfqpoint{0.930611in}{3.133110in}}%
\pgfpathlineto{\pgfqpoint{0.933555in}{3.130718in}}%
\pgfpathlineto{\pgfqpoint{0.939442in}{3.121865in}}%
\pgfpathlineto{\pgfqpoint{0.943724in}{3.117428in}}%
\pgfpathlineto{\pgfqpoint{0.946400in}{3.117281in}}%
\pgfpathlineto{\pgfqpoint{0.949344in}{3.119614in}}%
\pgfpathlineto{\pgfqpoint{0.954963in}{3.128045in}}%
\pgfpathlineto{\pgfqpoint{0.959513in}{3.132955in}}%
\pgfpathlineto{\pgfqpoint{0.962189in}{3.133164in}}%
\pgfpathlineto{\pgfqpoint{0.965132in}{3.130890in}}%
\pgfpathlineto{\pgfqpoint{0.970485in}{3.122896in}}%
\pgfpathlineto{\pgfqpoint{0.975302in}{3.117502in}}%
\pgfpathlineto{\pgfqpoint{0.977978in}{3.117230in}}%
\pgfpathlineto{\pgfqpoint{0.980921in}{3.119445in}}%
\pgfpathlineto{\pgfqpoint{0.986273in}{3.127410in}}%
\pgfpathlineto{\pgfqpoint{0.991090in}{3.132878in}}%
\pgfpathlineto{\pgfqpoint{0.994034in}{3.133120in}}%
\pgfpathlineto{\pgfqpoint{0.996978in}{3.130749in}}%
\pgfpathlineto{\pgfqpoint{1.002865in}{3.121903in}}%
\pgfpathlineto{\pgfqpoint{1.007147in}{3.117441in}}%
\pgfpathlineto{\pgfqpoint{1.009823in}{3.117271in}}%
\pgfpathlineto{\pgfqpoint{1.012767in}{3.119584in}}%
\pgfpathlineto{\pgfqpoint{1.018386in}{3.128006in}}%
\pgfpathlineto{\pgfqpoint{1.022936in}{3.132942in}}%
\pgfpathlineto{\pgfqpoint{1.025612in}{3.133173in}}%
\pgfpathlineto{\pgfqpoint{1.028555in}{3.130920in}}%
\pgfpathlineto{\pgfqpoint{1.033908in}{3.122936in}}%
\pgfpathlineto{\pgfqpoint{1.038725in}{3.117515in}}%
\pgfpathlineto{\pgfqpoint{1.041401in}{3.117222in}}%
\pgfpathlineto{\pgfqpoint{1.044344in}{3.119415in}}%
\pgfpathlineto{\pgfqpoint{1.049429in}{3.126950in}}%
\pgfpathlineto{\pgfqpoint{1.054513in}{3.132864in}}%
\pgfpathlineto{\pgfqpoint{1.057457in}{3.133130in}}%
\pgfpathlineto{\pgfqpoint{1.060401in}{3.130779in}}%
\pgfpathlineto{\pgfqpoint{1.066021in}{3.122341in}}%
\pgfpathlineto{\pgfqpoint{1.070570in}{3.117453in}}%
\pgfpathlineto{\pgfqpoint{1.073246in}{3.117262in}}%
\pgfpathlineto{\pgfqpoint{1.076190in}{3.119554in}}%
\pgfpathlineto{\pgfqpoint{1.081809in}{3.127966in}}%
\pgfpathlineto{\pgfqpoint{1.086359in}{3.132929in}}%
\pgfpathlineto{\pgfqpoint{1.089035in}{3.133182in}}%
\pgfpathlineto{\pgfqpoint{1.091979in}{3.130950in}}%
\pgfpathlineto{\pgfqpoint{1.097331in}{3.122976in}}%
\pgfpathlineto{\pgfqpoint{1.102148in}{3.117529in}}%
\pgfpathlineto{\pgfqpoint{1.105091in}{3.117307in}}%
\pgfpathlineto{\pgfqpoint{1.108035in}{3.119696in}}%
\pgfpathlineto{\pgfqpoint{1.113922in}{3.128547in}}%
\pgfpathlineto{\pgfqpoint{1.118204in}{3.132988in}}%
\pgfpathlineto{\pgfqpoint{1.120880in}{3.133140in}}%
\pgfpathlineto{\pgfqpoint{1.123824in}{3.130810in}}%
\pgfpathlineto{\pgfqpoint{1.129444in}{3.122380in}}%
\pgfpathlineto{\pgfqpoint{1.133993in}{3.117466in}}%
\pgfpathlineto{\pgfqpoint{1.136669in}{3.117253in}}%
\pgfpathlineto{\pgfqpoint{1.139613in}{3.119524in}}%
\pgfpathlineto{\pgfqpoint{1.144965in}{3.127517in}}%
\pgfpathlineto{\pgfqpoint{1.149782in}{3.132915in}}%
\pgfpathlineto{\pgfqpoint{1.152458in}{3.133190in}}%
\pgfpathlineto{\pgfqpoint{1.155402in}{3.130979in}}%
\pgfpathlineto{\pgfqpoint{1.160486in}{3.123435in}}%
\pgfpathlineto{\pgfqpoint{1.165571in}{3.117543in}}%
\pgfpathlineto{\pgfqpoint{1.168514in}{3.117297in}}%
\pgfpathlineto{\pgfqpoint{1.171458in}{3.119665in}}%
\pgfpathlineto{\pgfqpoint{1.177345in}{3.128509in}}%
\pgfpathlineto{\pgfqpoint{1.181627in}{3.132976in}}%
\pgfpathlineto{\pgfqpoint{1.184303in}{3.133149in}}%
\pgfpathlineto{\pgfqpoint{1.187247in}{3.130840in}}%
\pgfpathlineto{\pgfqpoint{1.192867in}{3.122420in}}%
\pgfpathlineto{\pgfqpoint{1.197416in}{3.117479in}}%
\pgfpathlineto{\pgfqpoint{1.200092in}{3.117244in}}%
\pgfpathlineto{\pgfqpoint{1.203036in}{3.119494in}}%
\pgfpathlineto{\pgfqpoint{1.208388in}{3.127476in}}%
\pgfpathlineto{\pgfqpoint{1.213205in}{3.132901in}}%
\pgfpathlineto{\pgfqpoint{1.215881in}{3.133198in}}%
\pgfpathlineto{\pgfqpoint{1.218825in}{3.131009in}}%
\pgfpathlineto{\pgfqpoint{1.223909in}{3.123475in}}%
\pgfpathlineto{\pgfqpoint{1.228994in}{3.117557in}}%
\pgfpathlineto{\pgfqpoint{1.231937in}{3.117287in}}%
\pgfpathlineto{\pgfqpoint{1.234881in}{3.119635in}}%
\pgfpathlineto{\pgfqpoint{1.240501in}{3.128071in}}%
\pgfpathlineto{\pgfqpoint{1.245050in}{3.132963in}}%
\pgfpathlineto{\pgfqpoint{1.247726in}{3.133158in}}%
\pgfpathlineto{\pgfqpoint{1.250670in}{3.130870in}}%
\pgfpathlineto{\pgfqpoint{1.256022in}{3.122869in}}%
\pgfpathlineto{\pgfqpoint{1.260839in}{3.117493in}}%
\pgfpathlineto{\pgfqpoint{1.263515in}{3.117236in}}%
\pgfpathlineto{\pgfqpoint{1.266459in}{3.119464in}}%
\pgfpathlineto{\pgfqpoint{1.271811in}{3.127436in}}%
\pgfpathlineto{\pgfqpoint{1.276628in}{3.132888in}}%
\pgfpathlineto{\pgfqpoint{1.279572in}{3.133114in}}%
\pgfpathlineto{\pgfqpoint{1.282515in}{3.130728in}}%
\pgfpathlineto{\pgfqpoint{1.288403in}{3.121878in}}%
\pgfpathlineto{\pgfqpoint{1.292684in}{3.117432in}}%
\pgfpathlineto{\pgfqpoint{1.295360in}{3.117278in}}%
\pgfpathlineto{\pgfqpoint{1.298304in}{3.119604in}}%
\pgfpathlineto{\pgfqpoint{1.303924in}{3.128032in}}%
\pgfpathlineto{\pgfqpoint{1.308473in}{3.132950in}}%
\pgfpathlineto{\pgfqpoint{1.311149in}{3.133167in}}%
\pgfpathlineto{\pgfqpoint{1.314093in}{3.130900in}}%
\pgfpathlineto{\pgfqpoint{1.319445in}{3.122909in}}%
\pgfpathlineto{\pgfqpoint{1.324262in}{3.117506in}}%
\pgfpathlineto{\pgfqpoint{1.326938in}{3.117227in}}%
\pgfpathlineto{\pgfqpoint{1.329882in}{3.119435in}}%
\pgfpathlineto{\pgfqpoint{1.334966in}{3.126977in}}%
\pgfpathlineto{\pgfqpoint{1.340051in}{3.132874in}}%
\pgfpathlineto{\pgfqpoint{1.342995in}{3.133124in}}%
\pgfpathlineto{\pgfqpoint{1.345938in}{3.130759in}}%
\pgfpathlineto{\pgfqpoint{1.351826in}{3.121916in}}%
\pgfpathlineto{\pgfqpoint{1.356107in}{3.117445in}}%
\pgfpathlineto{\pgfqpoint{1.358784in}{3.117268in}}%
\pgfpathlineto{\pgfqpoint{1.361727in}{3.119574in}}%
\pgfpathlineto{\pgfqpoint{1.367347in}{3.127993in}}%
\pgfpathlineto{\pgfqpoint{1.371896in}{3.132937in}}%
\pgfpathlineto{\pgfqpoint{1.374572in}{3.133176in}}%
\pgfpathlineto{\pgfqpoint{1.377516in}{3.130930in}}%
\pgfpathlineto{\pgfqpoint{1.382868in}{3.122949in}}%
\pgfpathlineto{\pgfqpoint{1.387685in}{3.117520in}}%
\pgfpathlineto{\pgfqpoint{1.390361in}{3.117219in}}%
\pgfpathlineto{\pgfqpoint{1.393305in}{3.119405in}}%
\pgfpathlineto{\pgfqpoint{1.398389in}{3.126937in}}%
\pgfpathlineto{\pgfqpoint{1.403474in}{3.132859in}}%
\pgfpathlineto{\pgfqpoint{1.406418in}{3.133133in}}%
\pgfpathlineto{\pgfqpoint{1.409361in}{3.130789in}}%
\pgfpathlineto{\pgfqpoint{1.414981in}{3.122354in}}%
\pgfpathlineto{\pgfqpoint{1.419531in}{3.117458in}}%
\pgfpathlineto{\pgfqpoint{1.422207in}{3.117259in}}%
\pgfpathlineto{\pgfqpoint{1.425150in}{3.119544in}}%
\pgfpathlineto{\pgfqpoint{1.430502in}{3.127543in}}%
\pgfpathlineto{\pgfqpoint{1.435319in}{3.132924in}}%
\pgfpathlineto{\pgfqpoint{1.437995in}{3.133185in}}%
\pgfpathlineto{\pgfqpoint{1.440939in}{3.130959in}}%
\pgfpathlineto{\pgfqpoint{1.446291in}{3.122989in}}%
\pgfpathlineto{\pgfqpoint{1.451108in}{3.117534in}}%
\pgfpathlineto{\pgfqpoint{1.454052in}{3.117304in}}%
\pgfpathlineto{\pgfqpoint{1.456996in}{3.119686in}}%
\pgfpathlineto{\pgfqpoint{1.462883in}{3.128535in}}%
\pgfpathlineto{\pgfqpoint{1.467165in}{3.132984in}}%
\pgfpathlineto{\pgfqpoint{1.469841in}{3.133143in}}%
\pgfpathlineto{\pgfqpoint{1.472784in}{3.130820in}}%
\pgfpathlineto{\pgfqpoint{1.478404in}{3.122394in}}%
\pgfpathlineto{\pgfqpoint{1.482954in}{3.117471in}}%
\pgfpathlineto{\pgfqpoint{1.485630in}{3.117250in}}%
\pgfpathlineto{\pgfqpoint{1.488573in}{3.119514in}}%
\pgfpathlineto{\pgfqpoint{1.493925in}{3.127503in}}%
\pgfpathlineto{\pgfqpoint{1.498742in}{3.132911in}}%
\pgfpathlineto{\pgfqpoint{1.501418in}{3.133193in}}%
\pgfpathlineto{\pgfqpoint{1.504362in}{3.130989in}}%
\pgfpathlineto{\pgfqpoint{1.509447in}{3.123448in}}%
\pgfpathlineto{\pgfqpoint{1.514531in}{3.117548in}}%
\pgfpathlineto{\pgfqpoint{1.517475in}{3.117294in}}%
\pgfpathlineto{\pgfqpoint{1.520419in}{3.119655in}}%
\pgfpathlineto{\pgfqpoint{1.526306in}{3.128496in}}%
\pgfpathlineto{\pgfqpoint{1.530588in}{3.132972in}}%
\pgfpathlineto{\pgfqpoint{1.533264in}{3.133152in}}%
\pgfpathlineto{\pgfqpoint{1.536208in}{3.130850in}}%
\pgfpathlineto{\pgfqpoint{1.541827in}{3.122433in}}%
\pgfpathlineto{\pgfqpoint{1.546377in}{3.117484in}}%
\pgfpathlineto{\pgfqpoint{1.549053in}{3.117241in}}%
\pgfpathlineto{\pgfqpoint{1.551996in}{3.119484in}}%
\pgfpathlineto{\pgfqpoint{1.557349in}{3.127463in}}%
\pgfpathlineto{\pgfqpoint{1.562165in}{3.132897in}}%
\pgfpathlineto{\pgfqpoint{1.565109in}{3.133107in}}%
\pgfpathlineto{\pgfqpoint{1.568053in}{3.130708in}}%
\pgfpathlineto{\pgfqpoint{1.574208in}{3.121464in}}%
\pgfpathlineto{\pgfqpoint{1.578490in}{3.117309in}}%
\pgfpathlineto{\pgfqpoint{1.581166in}{3.117394in}}%
\pgfpathlineto{\pgfqpoint{1.584109in}{3.119946in}}%
\pgfpathlineto{\pgfqpoint{1.595884in}{3.133339in}}%
\pgfpathlineto{\pgfqpoint{1.598560in}{3.131983in}}%
\pgfpathlineto{\pgfqpoint{1.602307in}{3.127147in}}%
\pgfpathlineto{\pgfqpoint{1.609532in}{3.117646in}}%
\pgfpathlineto{\pgfqpoint{1.612476in}{3.117233in}}%
\pgfpathlineto{\pgfqpoint{1.615419in}{3.119455in}}%
\pgfpathlineto{\pgfqpoint{1.620772in}{3.127423in}}%
\pgfpathlineto{\pgfqpoint{1.625589in}{3.132883in}}%
\pgfpathlineto{\pgfqpoint{1.628532in}{3.133117in}}%
\pgfpathlineto{\pgfqpoint{1.631476in}{3.130738in}}%
\pgfpathlineto{\pgfqpoint{1.637363in}{3.121891in}}%
\pgfpathlineto{\pgfqpoint{1.641645in}{3.117437in}}%
\pgfpathlineto{\pgfqpoint{1.644321in}{3.117274in}}%
\pgfpathlineto{\pgfqpoint{1.647265in}{3.119594in}}%
\pgfpathlineto{\pgfqpoint{1.652885in}{3.128019in}}%
\pgfpathlineto{\pgfqpoint{1.657434in}{3.132946in}}%
\pgfpathlineto{\pgfqpoint{1.660110in}{3.133170in}}%
\pgfpathlineto{\pgfqpoint{1.663054in}{3.130910in}}%
\pgfpathlineto{\pgfqpoint{1.668406in}{3.122922in}}%
\pgfpathlineto{\pgfqpoint{1.673223in}{3.117511in}}%
\pgfpathlineto{\pgfqpoint{1.675899in}{3.117224in}}%
\pgfpathlineto{\pgfqpoint{1.678842in}{3.119425in}}%
\pgfpathlineto{\pgfqpoint{1.683927in}{3.126964in}}%
\pgfpathlineto{\pgfqpoint{1.689012in}{3.132869in}}%
\pgfpathlineto{\pgfqpoint{1.691955in}{3.133127in}}%
\pgfpathlineto{\pgfqpoint{1.694899in}{3.130769in}}%
\pgfpathlineto{\pgfqpoint{1.700786in}{3.121929in}}%
\pgfpathlineto{\pgfqpoint{1.705068in}{3.117449in}}%
\pgfpathlineto{\pgfqpoint{1.707744in}{3.117265in}}%
\pgfpathlineto{\pgfqpoint{1.710688in}{3.119564in}}%
\pgfpathlineto{\pgfqpoint{1.716308in}{3.127980in}}%
\pgfpathlineto{\pgfqpoint{1.720857in}{3.132933in}}%
\pgfpathlineto{\pgfqpoint{1.723533in}{3.133179in}}%
\pgfpathlineto{\pgfqpoint{1.726477in}{3.130940in}}%
\pgfpathlineto{\pgfqpoint{1.731829in}{3.122962in}}%
\pgfpathlineto{\pgfqpoint{1.736646in}{3.117524in}}%
\pgfpathlineto{\pgfqpoint{1.739589in}{3.117310in}}%
\pgfpathlineto{\pgfqpoint{1.742533in}{3.119706in}}%
\pgfpathlineto{\pgfqpoint{1.748421in}{3.128560in}}%
\pgfpathlineto{\pgfqpoint{1.752702in}{3.132993in}}%
\pgfpathlineto{\pgfqpoint{1.755378in}{3.133137in}}%
\pgfpathlineto{\pgfqpoint{1.758322in}{3.130800in}}%
\pgfpathlineto{\pgfqpoint{1.763942in}{3.122367in}}%
\pgfpathlineto{\pgfqpoint{1.768491in}{3.117462in}}%
\pgfpathlineto{\pgfqpoint{1.771167in}{3.117256in}}%
\pgfpathlineto{\pgfqpoint{1.774111in}{3.119534in}}%
\pgfpathlineto{\pgfqpoint{1.779463in}{3.127530in}}%
\pgfpathlineto{\pgfqpoint{1.784280in}{3.132920in}}%
\pgfpathlineto{\pgfqpoint{1.786956in}{3.133187in}}%
\pgfpathlineto{\pgfqpoint{1.789900in}{3.130969in}}%
\pgfpathlineto{\pgfqpoint{1.795252in}{3.123003in}}%
\pgfpathlineto{\pgfqpoint{1.800069in}{3.117538in}}%
\pgfpathlineto{\pgfqpoint{1.803013in}{3.117300in}}%
\pgfpathlineto{\pgfqpoint{1.805956in}{3.119675in}}%
\pgfpathlineto{\pgfqpoint{1.811844in}{3.128522in}}%
\pgfpathlineto{\pgfqpoint{1.816125in}{3.132980in}}%
\pgfpathlineto{\pgfqpoint{1.818801in}{3.133146in}}%
\pgfpathlineto{\pgfqpoint{1.821745in}{3.130830in}}%
\pgfpathlineto{\pgfqpoint{1.827365in}{3.122407in}}%
\pgfpathlineto{\pgfqpoint{1.831914in}{3.117475in}}%
\pgfpathlineto{\pgfqpoint{1.834590in}{3.117247in}}%
\pgfpathlineto{\pgfqpoint{1.837534in}{3.119504in}}%
\pgfpathlineto{\pgfqpoint{1.842886in}{3.127490in}}%
\pgfpathlineto{\pgfqpoint{1.847703in}{3.132906in}}%
\pgfpathlineto{\pgfqpoint{1.850379in}{3.133196in}}%
\pgfpathlineto{\pgfqpoint{1.853323in}{3.130999in}}%
\pgfpathlineto{\pgfqpoint{1.858407in}{3.123462in}}%
\pgfpathlineto{\pgfqpoint{1.863492in}{3.117552in}}%
\pgfpathlineto{\pgfqpoint{1.866436in}{3.117290in}}%
\pgfpathlineto{\pgfqpoint{1.869379in}{3.119645in}}%
\pgfpathlineto{\pgfqpoint{1.875267in}{3.128484in}}%
\pgfpathlineto{\pgfqpoint{1.879548in}{3.132968in}}%
\pgfpathlineto{\pgfqpoint{1.882224in}{3.133155in}}%
\pgfpathlineto{\pgfqpoint{1.885168in}{3.130860in}}%
\pgfpathlineto{\pgfqpoint{1.890788in}{3.122446in}}%
\pgfpathlineto{\pgfqpoint{1.895337in}{3.117488in}}%
\pgfpathlineto{\pgfqpoint{1.898013in}{3.117238in}}%
\pgfpathlineto{\pgfqpoint{1.900957in}{3.119474in}}%
\pgfpathlineto{\pgfqpoint{1.906309in}{3.127450in}}%
\pgfpathlineto{\pgfqpoint{1.911126in}{3.132892in}}%
\pgfpathlineto{\pgfqpoint{1.914070in}{3.133110in}}%
\pgfpathlineto{\pgfqpoint{1.917013in}{3.130718in}}%
\pgfpathlineto{\pgfqpoint{1.922901in}{3.121865in}}%
\pgfpathlineto{\pgfqpoint{1.927183in}{3.117428in}}%
\pgfpathlineto{\pgfqpoint{1.929859in}{3.117281in}}%
\pgfpathlineto{\pgfqpoint{1.932802in}{3.119614in}}%
\pgfpathlineto{\pgfqpoint{1.938422in}{3.128045in}}%
\pgfpathlineto{\pgfqpoint{1.942971in}{3.132955in}}%
\pgfpathlineto{\pgfqpoint{1.945647in}{3.133164in}}%
\pgfpathlineto{\pgfqpoint{1.948591in}{3.130890in}}%
\pgfpathlineto{\pgfqpoint{1.953943in}{3.122896in}}%
\pgfpathlineto{\pgfqpoint{1.958760in}{3.117502in}}%
\pgfpathlineto{\pgfqpoint{1.961436in}{3.117230in}}%
\pgfpathlineto{\pgfqpoint{1.964380in}{3.119445in}}%
\pgfpathlineto{\pgfqpoint{1.969732in}{3.127410in}}%
\pgfpathlineto{\pgfqpoint{1.974549in}{3.132878in}}%
\pgfpathlineto{\pgfqpoint{1.977493in}{3.133120in}}%
\pgfpathlineto{\pgfqpoint{1.980436in}{3.130749in}}%
\pgfpathlineto{\pgfqpoint{1.986324in}{3.121903in}}%
\pgfpathlineto{\pgfqpoint{1.990606in}{3.117441in}}%
\pgfpathlineto{\pgfqpoint{1.993282in}{3.117271in}}%
\pgfpathlineto{\pgfqpoint{1.996225in}{3.119584in}}%
\pgfpathlineto{\pgfqpoint{2.001845in}{3.128006in}}%
\pgfpathlineto{\pgfqpoint{2.006394in}{3.132942in}}%
\pgfpathlineto{\pgfqpoint{2.009071in}{3.133173in}}%
\pgfpathlineto{\pgfqpoint{2.012014in}{3.130920in}}%
\pgfpathlineto{\pgfqpoint{2.017366in}{3.122936in}}%
\pgfpathlineto{\pgfqpoint{2.022183in}{3.117515in}}%
\pgfpathlineto{\pgfqpoint{2.024859in}{3.117222in}}%
\pgfpathlineto{\pgfqpoint{2.027803in}{3.119415in}}%
\pgfpathlineto{\pgfqpoint{2.032888in}{3.126950in}}%
\pgfpathlineto{\pgfqpoint{2.037972in}{3.132864in}}%
\pgfpathlineto{\pgfqpoint{2.040916in}{3.133130in}}%
\pgfpathlineto{\pgfqpoint{2.043860in}{3.130779in}}%
\pgfpathlineto{\pgfqpoint{2.049479in}{3.122341in}}%
\pgfpathlineto{\pgfqpoint{2.054029in}{3.117453in}}%
\pgfpathlineto{\pgfqpoint{2.056705in}{3.117262in}}%
\pgfpathlineto{\pgfqpoint{2.059648in}{3.119554in}}%
\pgfpathlineto{\pgfqpoint{2.065268in}{3.127966in}}%
\pgfpathlineto{\pgfqpoint{2.069818in}{3.132929in}}%
\pgfpathlineto{\pgfqpoint{2.072494in}{3.133182in}}%
\pgfpathlineto{\pgfqpoint{2.075437in}{3.130950in}}%
\pgfpathlineto{\pgfqpoint{2.080789in}{3.122976in}}%
\pgfpathlineto{\pgfqpoint{2.085606in}{3.117529in}}%
\pgfpathlineto{\pgfqpoint{2.088550in}{3.117307in}}%
\pgfpathlineto{\pgfqpoint{2.091494in}{3.119696in}}%
\pgfpathlineto{\pgfqpoint{2.097381in}{3.128547in}}%
\pgfpathlineto{\pgfqpoint{2.101663in}{3.132988in}}%
\pgfpathlineto{\pgfqpoint{2.104339in}{3.133140in}}%
\pgfpathlineto{\pgfqpoint{2.107283in}{3.130810in}}%
\pgfpathlineto{\pgfqpoint{2.112902in}{3.122380in}}%
\pgfpathlineto{\pgfqpoint{2.117452in}{3.117466in}}%
\pgfpathlineto{\pgfqpoint{2.120128in}{3.117253in}}%
\pgfpathlineto{\pgfqpoint{2.123071in}{3.119524in}}%
\pgfpathlineto{\pgfqpoint{2.128424in}{3.127517in}}%
\pgfpathlineto{\pgfqpoint{2.133241in}{3.132915in}}%
\pgfpathlineto{\pgfqpoint{2.135917in}{3.133190in}}%
\pgfpathlineto{\pgfqpoint{2.138860in}{3.130979in}}%
\pgfpathlineto{\pgfqpoint{2.143945in}{3.123435in}}%
\pgfpathlineto{\pgfqpoint{2.149029in}{3.117543in}}%
\pgfpathlineto{\pgfqpoint{2.151973in}{3.117297in}}%
\pgfpathlineto{\pgfqpoint{2.154917in}{3.119665in}}%
\pgfpathlineto{\pgfqpoint{2.160804in}{3.128509in}}%
\pgfpathlineto{\pgfqpoint{2.165086in}{3.132976in}}%
\pgfpathlineto{\pgfqpoint{2.167762in}{3.133149in}}%
\pgfpathlineto{\pgfqpoint{2.170706in}{3.130840in}}%
\pgfpathlineto{\pgfqpoint{2.176325in}{3.122420in}}%
\pgfpathlineto{\pgfqpoint{2.180875in}{3.117479in}}%
\pgfpathlineto{\pgfqpoint{2.183551in}{3.117244in}}%
\pgfpathlineto{\pgfqpoint{2.186495in}{3.119494in}}%
\pgfpathlineto{\pgfqpoint{2.191847in}{3.127476in}}%
\pgfpathlineto{\pgfqpoint{2.196664in}{3.132901in}}%
\pgfpathlineto{\pgfqpoint{2.199340in}{3.133198in}}%
\pgfpathlineto{\pgfqpoint{2.202283in}{3.131009in}}%
\pgfpathlineto{\pgfqpoint{2.207368in}{3.123475in}}%
\pgfpathlineto{\pgfqpoint{2.212452in}{3.117557in}}%
\pgfpathlineto{\pgfqpoint{2.215396in}{3.117287in}}%
\pgfpathlineto{\pgfqpoint{2.218340in}{3.119635in}}%
\pgfpathlineto{\pgfqpoint{2.223960in}{3.128071in}}%
\pgfpathlineto{\pgfqpoint{2.228509in}{3.132963in}}%
\pgfpathlineto{\pgfqpoint{2.231185in}{3.133158in}}%
\pgfpathlineto{\pgfqpoint{2.234129in}{3.130870in}}%
\pgfpathlineto{\pgfqpoint{2.239481in}{3.122869in}}%
\pgfpathlineto{\pgfqpoint{2.244298in}{3.117493in}}%
\pgfpathlineto{\pgfqpoint{2.246974in}{3.117236in}}%
\pgfpathlineto{\pgfqpoint{2.249918in}{3.119464in}}%
\pgfpathlineto{\pgfqpoint{2.255270in}{3.127436in}}%
\pgfpathlineto{\pgfqpoint{2.260087in}{3.132888in}}%
\pgfpathlineto{\pgfqpoint{2.263030in}{3.133114in}}%
\pgfpathlineto{\pgfqpoint{2.265974in}{3.130728in}}%
\pgfpathlineto{\pgfqpoint{2.271861in}{3.121878in}}%
\pgfpathlineto{\pgfqpoint{2.276143in}{3.117432in}}%
\pgfpathlineto{\pgfqpoint{2.278819in}{3.117278in}}%
\pgfpathlineto{\pgfqpoint{2.281763in}{3.119604in}}%
\pgfpathlineto{\pgfqpoint{2.287383in}{3.128032in}}%
\pgfpathlineto{\pgfqpoint{2.291932in}{3.132950in}}%
\pgfpathlineto{\pgfqpoint{2.294608in}{3.133167in}}%
\pgfpathlineto{\pgfqpoint{2.297552in}{3.130900in}}%
\pgfpathlineto{\pgfqpoint{2.302904in}{3.122909in}}%
\pgfpathlineto{\pgfqpoint{2.307721in}{3.117506in}}%
\pgfpathlineto{\pgfqpoint{2.310397in}{3.117227in}}%
\pgfpathlineto{\pgfqpoint{2.313341in}{3.119435in}}%
\pgfpathlineto{\pgfqpoint{2.318425in}{3.126977in}}%
\pgfpathlineto{\pgfqpoint{2.323510in}{3.132874in}}%
\pgfpathlineto{\pgfqpoint{2.326453in}{3.133124in}}%
\pgfpathlineto{\pgfqpoint{2.329397in}{3.130759in}}%
\pgfpathlineto{\pgfqpoint{2.335284in}{3.121916in}}%
\pgfpathlineto{\pgfqpoint{2.339566in}{3.117445in}}%
\pgfpathlineto{\pgfqpoint{2.342242in}{3.117268in}}%
\pgfpathlineto{\pgfqpoint{2.345186in}{3.119574in}}%
\pgfpathlineto{\pgfqpoint{2.350806in}{3.127993in}}%
\pgfpathlineto{\pgfqpoint{2.355355in}{3.132937in}}%
\pgfpathlineto{\pgfqpoint{2.358031in}{3.133176in}}%
\pgfpathlineto{\pgfqpoint{2.360975in}{3.130930in}}%
\pgfpathlineto{\pgfqpoint{2.366327in}{3.122949in}}%
\pgfpathlineto{\pgfqpoint{2.371144in}{3.117520in}}%
\pgfpathlineto{\pgfqpoint{2.373820in}{3.117219in}}%
\pgfpathlineto{\pgfqpoint{2.376764in}{3.119405in}}%
\pgfpathlineto{\pgfqpoint{2.381848in}{3.126937in}}%
\pgfpathlineto{\pgfqpoint{2.386933in}{3.132859in}}%
\pgfpathlineto{\pgfqpoint{2.389876in}{3.133133in}}%
\pgfpathlineto{\pgfqpoint{2.392820in}{3.130789in}}%
\pgfpathlineto{\pgfqpoint{2.398440in}{3.122354in}}%
\pgfpathlineto{\pgfqpoint{2.402989in}{3.117458in}}%
\pgfpathlineto{\pgfqpoint{2.405665in}{3.117259in}}%
\pgfpathlineto{\pgfqpoint{2.408609in}{3.119544in}}%
\pgfpathlineto{\pgfqpoint{2.413961in}{3.127543in}}%
\pgfpathlineto{\pgfqpoint{2.418778in}{3.132924in}}%
\pgfpathlineto{\pgfqpoint{2.421454in}{3.133185in}}%
\pgfpathlineto{\pgfqpoint{2.424398in}{3.130959in}}%
\pgfpathlineto{\pgfqpoint{2.429750in}{3.122989in}}%
\pgfpathlineto{\pgfqpoint{2.434567in}{3.117534in}}%
\pgfpathlineto{\pgfqpoint{2.437511in}{3.117304in}}%
\pgfpathlineto{\pgfqpoint{2.440454in}{3.119686in}}%
\pgfpathlineto{\pgfqpoint{2.446342in}{3.128535in}}%
\pgfpathlineto{\pgfqpoint{2.450623in}{3.132984in}}%
\pgfpathlineto{\pgfqpoint{2.453300in}{3.133143in}}%
\pgfpathlineto{\pgfqpoint{2.456243in}{3.130820in}}%
\pgfpathlineto{\pgfqpoint{2.461863in}{3.122394in}}%
\pgfpathlineto{\pgfqpoint{2.466412in}{3.117471in}}%
\pgfpathlineto{\pgfqpoint{2.469088in}{3.117250in}}%
\pgfpathlineto{\pgfqpoint{2.472032in}{3.119514in}}%
\pgfpathlineto{\pgfqpoint{2.477384in}{3.127503in}}%
\pgfpathlineto{\pgfqpoint{2.482201in}{3.132911in}}%
\pgfpathlineto{\pgfqpoint{2.484877in}{3.133193in}}%
\pgfpathlineto{\pgfqpoint{2.487821in}{3.130989in}}%
\pgfpathlineto{\pgfqpoint{2.492905in}{3.123448in}}%
\pgfpathlineto{\pgfqpoint{2.497990in}{3.117548in}}%
\pgfpathlineto{\pgfqpoint{2.500934in}{3.117294in}}%
\pgfpathlineto{\pgfqpoint{2.503877in}{3.119655in}}%
\pgfpathlineto{\pgfqpoint{2.509765in}{3.128496in}}%
\pgfpathlineto{\pgfqpoint{2.514046in}{3.132972in}}%
\pgfpathlineto{\pgfqpoint{2.516723in}{3.133152in}}%
\pgfpathlineto{\pgfqpoint{2.519666in}{3.130850in}}%
\pgfpathlineto{\pgfqpoint{2.525286in}{3.122433in}}%
\pgfpathlineto{\pgfqpoint{2.529835in}{3.117484in}}%
\pgfpathlineto{\pgfqpoint{2.532511in}{3.117241in}}%
\pgfpathlineto{\pgfqpoint{2.535455in}{3.119484in}}%
\pgfpathlineto{\pgfqpoint{2.540807in}{3.127463in}}%
\pgfpathlineto{\pgfqpoint{2.545624in}{3.132897in}}%
\pgfpathlineto{\pgfqpoint{2.548568in}{3.133107in}}%
\pgfpathlineto{\pgfqpoint{2.551512in}{3.130708in}}%
\pgfpathlineto{\pgfqpoint{2.557667in}{3.121464in}}%
\pgfpathlineto{\pgfqpoint{2.561948in}{3.117309in}}%
\pgfpathlineto{\pgfqpoint{2.564624in}{3.117394in}}%
\pgfpathlineto{\pgfqpoint{2.567568in}{3.119946in}}%
\pgfpathlineto{\pgfqpoint{2.579343in}{3.133339in}}%
\pgfpathlineto{\pgfqpoint{2.582019in}{3.131983in}}%
\pgfpathlineto{\pgfqpoint{2.585765in}{3.127147in}}%
\pgfpathlineto{\pgfqpoint{2.592991in}{3.117646in}}%
\pgfpathlineto{\pgfqpoint{2.595934in}{3.117233in}}%
\pgfpathlineto{\pgfqpoint{2.598878in}{3.119455in}}%
\pgfpathlineto{\pgfqpoint{2.604230in}{3.127423in}}%
\pgfpathlineto{\pgfqpoint{2.609047in}{3.132883in}}%
\pgfpathlineto{\pgfqpoint{2.611991in}{3.133117in}}%
\pgfpathlineto{\pgfqpoint{2.614935in}{3.130738in}}%
\pgfpathlineto{\pgfqpoint{2.620822in}{3.121891in}}%
\pgfpathlineto{\pgfqpoint{2.625104in}{3.117437in}}%
\pgfpathlineto{\pgfqpoint{2.627780in}{3.117274in}}%
\pgfpathlineto{\pgfqpoint{2.630724in}{3.119594in}}%
\pgfpathlineto{\pgfqpoint{2.636343in}{3.128019in}}%
\pgfpathlineto{\pgfqpoint{2.640893in}{3.132946in}}%
\pgfpathlineto{\pgfqpoint{2.643569in}{3.133170in}}%
\pgfpathlineto{\pgfqpoint{2.646512in}{3.130910in}}%
\pgfpathlineto{\pgfqpoint{2.651865in}{3.122922in}}%
\pgfpathlineto{\pgfqpoint{2.656681in}{3.117511in}}%
\pgfpathlineto{\pgfqpoint{2.659358in}{3.117224in}}%
\pgfpathlineto{\pgfqpoint{2.662301in}{3.119425in}}%
\pgfpathlineto{\pgfqpoint{2.667386in}{3.126964in}}%
\pgfpathlineto{\pgfqpoint{2.672470in}{3.132869in}}%
\pgfpathlineto{\pgfqpoint{2.675414in}{3.133127in}}%
\pgfpathlineto{\pgfqpoint{2.678358in}{3.130769in}}%
\pgfpathlineto{\pgfqpoint{2.684245in}{3.121929in}}%
\pgfpathlineto{\pgfqpoint{2.688527in}{3.117449in}}%
\pgfpathlineto{\pgfqpoint{2.691203in}{3.117265in}}%
\pgfpathlineto{\pgfqpoint{2.694147in}{3.119564in}}%
\pgfpathlineto{\pgfqpoint{2.699766in}{3.127980in}}%
\pgfpathlineto{\pgfqpoint{2.704316in}{3.132933in}}%
\pgfpathlineto{\pgfqpoint{2.706992in}{3.133179in}}%
\pgfpathlineto{\pgfqpoint{2.709935in}{3.130940in}}%
\pgfpathlineto{\pgfqpoint{2.715288in}{3.122962in}}%
\pgfpathlineto{\pgfqpoint{2.720105in}{3.117524in}}%
\pgfpathlineto{\pgfqpoint{2.723048in}{3.117310in}}%
\pgfpathlineto{\pgfqpoint{2.725992in}{3.119706in}}%
\pgfpathlineto{\pgfqpoint{2.731879in}{3.128560in}}%
\pgfpathlineto{\pgfqpoint{2.736161in}{3.132993in}}%
\pgfpathlineto{\pgfqpoint{2.738837in}{3.133137in}}%
\pgfpathlineto{\pgfqpoint{2.741781in}{3.130800in}}%
\pgfpathlineto{\pgfqpoint{2.747401in}{3.122367in}}%
\pgfpathlineto{\pgfqpoint{2.751950in}{3.117462in}}%
\pgfpathlineto{\pgfqpoint{2.754626in}{3.117256in}}%
\pgfpathlineto{\pgfqpoint{2.757570in}{3.119534in}}%
\pgfpathlineto{\pgfqpoint{2.762922in}{3.127530in}}%
\pgfpathlineto{\pgfqpoint{2.767739in}{3.132920in}}%
\pgfpathlineto{\pgfqpoint{2.770415in}{3.133187in}}%
\pgfpathlineto{\pgfqpoint{2.773358in}{3.130969in}}%
\pgfpathlineto{\pgfqpoint{2.778711in}{3.123003in}}%
\pgfpathlineto{\pgfqpoint{2.783528in}{3.117538in}}%
\pgfpathlineto{\pgfqpoint{2.786471in}{3.117300in}}%
\pgfpathlineto{\pgfqpoint{2.789415in}{3.119675in}}%
\pgfpathlineto{\pgfqpoint{2.795302in}{3.128522in}}%
\pgfpathlineto{\pgfqpoint{2.799584in}{3.132980in}}%
\pgfpathlineto{\pgfqpoint{2.802260in}{3.133146in}}%
\pgfpathlineto{\pgfqpoint{2.805204in}{3.130830in}}%
\pgfpathlineto{\pgfqpoint{2.810824in}{3.122407in}}%
\pgfpathlineto{\pgfqpoint{2.815373in}{3.117475in}}%
\pgfpathlineto{\pgfqpoint{2.818049in}{3.117247in}}%
\pgfpathlineto{\pgfqpoint{2.820993in}{3.119504in}}%
\pgfpathlineto{\pgfqpoint{2.826345in}{3.127490in}}%
\pgfpathlineto{\pgfqpoint{2.831162in}{3.132906in}}%
\pgfpathlineto{\pgfqpoint{2.833838in}{3.133196in}}%
\pgfpathlineto{\pgfqpoint{2.836782in}{3.130999in}}%
\pgfpathlineto{\pgfqpoint{2.841866in}{3.123462in}}%
\pgfpathlineto{\pgfqpoint{2.846951in}{3.117552in}}%
\pgfpathlineto{\pgfqpoint{2.849894in}{3.117290in}}%
\pgfpathlineto{\pgfqpoint{2.852838in}{3.119645in}}%
\pgfpathlineto{\pgfqpoint{2.858725in}{3.128484in}}%
\pgfpathlineto{\pgfqpoint{2.863007in}{3.132968in}}%
\pgfpathlineto{\pgfqpoint{2.865683in}{3.133155in}}%
\pgfpathlineto{\pgfqpoint{2.868627in}{3.130860in}}%
\pgfpathlineto{\pgfqpoint{2.874247in}{3.122446in}}%
\pgfpathlineto{\pgfqpoint{2.878796in}{3.117488in}}%
\pgfpathlineto{\pgfqpoint{2.881472in}{3.117238in}}%
\pgfpathlineto{\pgfqpoint{2.884416in}{3.119474in}}%
\pgfpathlineto{\pgfqpoint{2.889768in}{3.127450in}}%
\pgfpathlineto{\pgfqpoint{2.894585in}{3.132892in}}%
\pgfpathlineto{\pgfqpoint{2.897528in}{3.133110in}}%
\pgfpathlineto{\pgfqpoint{2.900472in}{3.130718in}}%
\pgfpathlineto{\pgfqpoint{2.906360in}{3.121865in}}%
\pgfpathlineto{\pgfqpoint{2.910641in}{3.117428in}}%
\pgfpathlineto{\pgfqpoint{2.913317in}{3.117281in}}%
\pgfpathlineto{\pgfqpoint{2.916261in}{3.119614in}}%
\pgfpathlineto{\pgfqpoint{2.921881in}{3.128045in}}%
\pgfpathlineto{\pgfqpoint{2.926430in}{3.132955in}}%
\pgfpathlineto{\pgfqpoint{2.929106in}{3.133164in}}%
\pgfpathlineto{\pgfqpoint{2.932050in}{3.130890in}}%
\pgfpathlineto{\pgfqpoint{2.937402in}{3.122896in}}%
\pgfpathlineto{\pgfqpoint{2.942219in}{3.117502in}}%
\pgfpathlineto{\pgfqpoint{2.944895in}{3.117230in}}%
\pgfpathlineto{\pgfqpoint{2.947839in}{3.119445in}}%
\pgfpathlineto{\pgfqpoint{2.953191in}{3.127410in}}%
\pgfpathlineto{\pgfqpoint{2.958008in}{3.132878in}}%
\pgfpathlineto{\pgfqpoint{2.960952in}{3.133120in}}%
\pgfpathlineto{\pgfqpoint{2.963895in}{3.130749in}}%
\pgfpathlineto{\pgfqpoint{2.969783in}{3.121903in}}%
\pgfpathlineto{\pgfqpoint{2.974064in}{3.117441in}}%
\pgfpathlineto{\pgfqpoint{2.976740in}{3.117271in}}%
\pgfpathlineto{\pgfqpoint{2.979684in}{3.119584in}}%
\pgfpathlineto{\pgfqpoint{2.985304in}{3.128006in}}%
\pgfpathlineto{\pgfqpoint{2.989853in}{3.132942in}}%
\pgfpathlineto{\pgfqpoint{2.992529in}{3.133173in}}%
\pgfpathlineto{\pgfqpoint{2.995473in}{3.130920in}}%
\pgfpathlineto{\pgfqpoint{3.000825in}{3.122936in}}%
\pgfpathlineto{\pgfqpoint{3.005642in}{3.117515in}}%
\pgfpathlineto{\pgfqpoint{3.008318in}{3.117222in}}%
\pgfpathlineto{\pgfqpoint{3.011262in}{3.119415in}}%
\pgfpathlineto{\pgfqpoint{3.016346in}{3.126950in}}%
\pgfpathlineto{\pgfqpoint{3.021431in}{3.132864in}}%
\pgfpathlineto{\pgfqpoint{3.024375in}{3.133130in}}%
\pgfpathlineto{\pgfqpoint{3.027318in}{3.130779in}}%
\pgfpathlineto{\pgfqpoint{3.032938in}{3.122341in}}%
\pgfpathlineto{\pgfqpoint{3.037487in}{3.117453in}}%
\pgfpathlineto{\pgfqpoint{3.040163in}{3.117262in}}%
\pgfpathlineto{\pgfqpoint{3.043107in}{3.119554in}}%
\pgfpathlineto{\pgfqpoint{3.048727in}{3.127966in}}%
\pgfpathlineto{\pgfqpoint{3.053276in}{3.132929in}}%
\pgfpathlineto{\pgfqpoint{3.055952in}{3.133182in}}%
\pgfpathlineto{\pgfqpoint{3.058896in}{3.130950in}}%
\pgfpathlineto{\pgfqpoint{3.064248in}{3.122976in}}%
\pgfpathlineto{\pgfqpoint{3.069065in}{3.117529in}}%
\pgfpathlineto{\pgfqpoint{3.072009in}{3.117307in}}%
\pgfpathlineto{\pgfqpoint{3.074952in}{3.119696in}}%
\pgfpathlineto{\pgfqpoint{3.080840in}{3.128547in}}%
\pgfpathlineto{\pgfqpoint{3.085122in}{3.132988in}}%
\pgfpathlineto{\pgfqpoint{3.087798in}{3.133140in}}%
\pgfpathlineto{\pgfqpoint{3.090741in}{3.130810in}}%
\pgfpathlineto{\pgfqpoint{3.096361in}{3.122380in}}%
\pgfpathlineto{\pgfqpoint{3.100910in}{3.117466in}}%
\pgfpathlineto{\pgfqpoint{3.103587in}{3.117253in}}%
\pgfpathlineto{\pgfqpoint{3.106530in}{3.119524in}}%
\pgfpathlineto{\pgfqpoint{3.111882in}{3.127517in}}%
\pgfpathlineto{\pgfqpoint{3.116699in}{3.132915in}}%
\pgfpathlineto{\pgfqpoint{3.119375in}{3.133190in}}%
\pgfpathlineto{\pgfqpoint{3.122319in}{3.130979in}}%
\pgfpathlineto{\pgfqpoint{3.127404in}{3.123435in}}%
\pgfpathlineto{\pgfqpoint{3.132488in}{3.117543in}}%
\pgfpathlineto{\pgfqpoint{3.135432in}{3.117297in}}%
\pgfpathlineto{\pgfqpoint{3.138376in}{3.119665in}}%
\pgfpathlineto{\pgfqpoint{3.144263in}{3.128509in}}%
\pgfpathlineto{\pgfqpoint{3.148545in}{3.132976in}}%
\pgfpathlineto{\pgfqpoint{3.151221in}{3.133149in}}%
\pgfpathlineto{\pgfqpoint{3.154164in}{3.130840in}}%
\pgfpathlineto{\pgfqpoint{3.159784in}{3.122420in}}%
\pgfpathlineto{\pgfqpoint{3.164333in}{3.117479in}}%
\pgfpathlineto{\pgfqpoint{3.167010in}{3.117244in}}%
\pgfpathlineto{\pgfqpoint{3.169953in}{3.119494in}}%
\pgfpathlineto{\pgfqpoint{3.175305in}{3.127476in}}%
\pgfpathlineto{\pgfqpoint{3.180122in}{3.132901in}}%
\pgfpathlineto{\pgfqpoint{3.182798in}{3.133198in}}%
\pgfpathlineto{\pgfqpoint{3.185742in}{3.131009in}}%
\pgfpathlineto{\pgfqpoint{3.190827in}{3.123475in}}%
\pgfpathlineto{\pgfqpoint{3.195911in}{3.117557in}}%
\pgfpathlineto{\pgfqpoint{3.198855in}{3.117287in}}%
\pgfpathlineto{\pgfqpoint{3.201799in}{3.119635in}}%
\pgfpathlineto{\pgfqpoint{3.207418in}{3.128071in}}%
\pgfpathlineto{\pgfqpoint{3.211968in}{3.132963in}}%
\pgfpathlineto{\pgfqpoint{3.214644in}{3.133158in}}%
\pgfpathlineto{\pgfqpoint{3.217587in}{3.130870in}}%
\pgfpathlineto{\pgfqpoint{3.222940in}{3.122869in}}%
\pgfpathlineto{\pgfqpoint{3.227757in}{3.117493in}}%
\pgfpathlineto{\pgfqpoint{3.230433in}{3.117236in}}%
\pgfpathlineto{\pgfqpoint{3.233376in}{3.119464in}}%
\pgfpathlineto{\pgfqpoint{3.238728in}{3.127436in}}%
\pgfpathlineto{\pgfqpoint{3.243545in}{3.132888in}}%
\pgfpathlineto{\pgfqpoint{3.246489in}{3.133114in}}%
\pgfpathlineto{\pgfqpoint{3.249433in}{3.130728in}}%
\pgfpathlineto{\pgfqpoint{3.255320in}{3.121878in}}%
\pgfpathlineto{\pgfqpoint{3.259602in}{3.117432in}}%
\pgfpathlineto{\pgfqpoint{3.262278in}{3.117278in}}%
\pgfpathlineto{\pgfqpoint{3.265222in}{3.119604in}}%
\pgfpathlineto{\pgfqpoint{3.270841in}{3.128032in}}%
\pgfpathlineto{\pgfqpoint{3.275391in}{3.132950in}}%
\pgfpathlineto{\pgfqpoint{3.278067in}{3.133167in}}%
\pgfpathlineto{\pgfqpoint{3.281011in}{3.130900in}}%
\pgfpathlineto{\pgfqpoint{3.286363in}{3.122909in}}%
\pgfpathlineto{\pgfqpoint{3.291180in}{3.117506in}}%
\pgfpathlineto{\pgfqpoint{3.293856in}{3.117227in}}%
\pgfpathlineto{\pgfqpoint{3.296799in}{3.119435in}}%
\pgfpathlineto{\pgfqpoint{3.301884in}{3.126977in}}%
\pgfpathlineto{\pgfqpoint{3.306968in}{3.132874in}}%
\pgfpathlineto{\pgfqpoint{3.309912in}{3.133124in}}%
\pgfpathlineto{\pgfqpoint{3.312856in}{3.130759in}}%
\pgfpathlineto{\pgfqpoint{3.318743in}{3.121916in}}%
\pgfpathlineto{\pgfqpoint{3.323025in}{3.117445in}}%
\pgfpathlineto{\pgfqpoint{3.325701in}{3.117268in}}%
\pgfpathlineto{\pgfqpoint{3.328645in}{3.119574in}}%
\pgfpathlineto{\pgfqpoint{3.334264in}{3.127993in}}%
\pgfpathlineto{\pgfqpoint{3.338814in}{3.132937in}}%
\pgfpathlineto{\pgfqpoint{3.341490in}{3.133176in}}%
\pgfpathlineto{\pgfqpoint{3.344434in}{3.130930in}}%
\pgfpathlineto{\pgfqpoint{3.349786in}{3.122949in}}%
\pgfpathlineto{\pgfqpoint{3.354603in}{3.117520in}}%
\pgfpathlineto{\pgfqpoint{3.357279in}{3.117219in}}%
\pgfpathlineto{\pgfqpoint{3.360222in}{3.119405in}}%
\pgfpathlineto{\pgfqpoint{3.365307in}{3.126937in}}%
\pgfpathlineto{\pgfqpoint{3.370392in}{3.132859in}}%
\pgfpathlineto{\pgfqpoint{3.373335in}{3.133133in}}%
\pgfpathlineto{\pgfqpoint{3.376279in}{3.130789in}}%
\pgfpathlineto{\pgfqpoint{3.381899in}{3.122354in}}%
\pgfpathlineto{\pgfqpoint{3.386448in}{3.117458in}}%
\pgfpathlineto{\pgfqpoint{3.389124in}{3.117259in}}%
\pgfpathlineto{\pgfqpoint{3.392068in}{3.119544in}}%
\pgfpathlineto{\pgfqpoint{3.397420in}{3.127543in}}%
\pgfpathlineto{\pgfqpoint{3.402237in}{3.132924in}}%
\pgfpathlineto{\pgfqpoint{3.404913in}{3.133185in}}%
\pgfpathlineto{\pgfqpoint{3.407857in}{3.130959in}}%
\pgfpathlineto{\pgfqpoint{3.413209in}{3.122989in}}%
\pgfpathlineto{\pgfqpoint{3.418026in}{3.117534in}}%
\pgfpathlineto{\pgfqpoint{3.420969in}{3.117304in}}%
\pgfpathlineto{\pgfqpoint{3.423913in}{3.119686in}}%
\pgfpathlineto{\pgfqpoint{3.429800in}{3.128535in}}%
\pgfpathlineto{\pgfqpoint{3.434082in}{3.132984in}}%
\pgfpathlineto{\pgfqpoint{3.436758in}{3.133143in}}%
\pgfpathlineto{\pgfqpoint{3.439702in}{3.130820in}}%
\pgfpathlineto{\pgfqpoint{3.445322in}{3.122394in}}%
\pgfpathlineto{\pgfqpoint{3.449871in}{3.117471in}}%
\pgfpathlineto{\pgfqpoint{3.452547in}{3.117250in}}%
\pgfpathlineto{\pgfqpoint{3.455491in}{3.119514in}}%
\pgfpathlineto{\pgfqpoint{3.460843in}{3.127503in}}%
\pgfpathlineto{\pgfqpoint{3.465660in}{3.132911in}}%
\pgfpathlineto{\pgfqpoint{3.468336in}{3.133193in}}%
\pgfpathlineto{\pgfqpoint{3.471280in}{3.130989in}}%
\pgfpathlineto{\pgfqpoint{3.476364in}{3.123448in}}%
\pgfpathlineto{\pgfqpoint{3.481449in}{3.117548in}}%
\pgfpathlineto{\pgfqpoint{3.484392in}{3.117294in}}%
\pgfpathlineto{\pgfqpoint{3.487336in}{3.119655in}}%
\pgfpathlineto{\pgfqpoint{3.493224in}{3.128496in}}%
\pgfpathlineto{\pgfqpoint{3.497505in}{3.132972in}}%
\pgfpathlineto{\pgfqpoint{3.500181in}{3.133152in}}%
\pgfpathlineto{\pgfqpoint{3.503125in}{3.130850in}}%
\pgfpathlineto{\pgfqpoint{3.508745in}{3.122433in}}%
\pgfpathlineto{\pgfqpoint{3.513294in}{3.117484in}}%
\pgfpathlineto{\pgfqpoint{3.515970in}{3.117241in}}%
\pgfpathlineto{\pgfqpoint{3.518914in}{3.119484in}}%
\pgfpathlineto{\pgfqpoint{3.524266in}{3.127463in}}%
\pgfpathlineto{\pgfqpoint{3.529083in}{3.132897in}}%
\pgfpathlineto{\pgfqpoint{3.532027in}{3.133107in}}%
\pgfpathlineto{\pgfqpoint{3.534970in}{3.130708in}}%
\pgfpathlineto{\pgfqpoint{3.541125in}{3.121464in}}%
\pgfpathlineto{\pgfqpoint{3.545407in}{3.117309in}}%
\pgfpathlineto{\pgfqpoint{3.548083in}{3.117394in}}%
\pgfpathlineto{\pgfqpoint{3.551027in}{3.119946in}}%
\pgfpathlineto{\pgfqpoint{3.562802in}{3.133339in}}%
\pgfpathlineto{\pgfqpoint{3.565478in}{3.131983in}}%
\pgfpathlineto{\pgfqpoint{3.569224in}{3.127147in}}%
\pgfpathlineto{\pgfqpoint{3.576450in}{3.117646in}}%
\pgfpathlineto{\pgfqpoint{3.579393in}{3.117233in}}%
\pgfpathlineto{\pgfqpoint{3.582337in}{3.119455in}}%
\pgfpathlineto{\pgfqpoint{3.587689in}{3.127423in}}%
\pgfpathlineto{\pgfqpoint{3.592506in}{3.132883in}}%
\pgfpathlineto{\pgfqpoint{3.595450in}{3.133117in}}%
\pgfpathlineto{\pgfqpoint{3.598393in}{3.130738in}}%
\pgfpathlineto{\pgfqpoint{3.604281in}{3.121891in}}%
\pgfpathlineto{\pgfqpoint{3.608562in}{3.117437in}}%
\pgfpathlineto{\pgfqpoint{3.611239in}{3.117274in}}%
\pgfpathlineto{\pgfqpoint{3.614182in}{3.119594in}}%
\pgfpathlineto{\pgfqpoint{3.619802in}{3.128019in}}%
\pgfpathlineto{\pgfqpoint{3.624351in}{3.132946in}}%
\pgfpathlineto{\pgfqpoint{3.627027in}{3.133170in}}%
\pgfpathlineto{\pgfqpoint{3.629971in}{3.130910in}}%
\pgfpathlineto{\pgfqpoint{3.635323in}{3.122922in}}%
\pgfpathlineto{\pgfqpoint{3.640140in}{3.117511in}}%
\pgfpathlineto{\pgfqpoint{3.642816in}{3.117224in}}%
\pgfpathlineto{\pgfqpoint{3.645760in}{3.119425in}}%
\pgfpathlineto{\pgfqpoint{3.650845in}{3.126964in}}%
\pgfpathlineto{\pgfqpoint{3.655929in}{3.132869in}}%
\pgfpathlineto{\pgfqpoint{3.658873in}{3.133127in}}%
\pgfpathlineto{\pgfqpoint{3.661816in}{3.130769in}}%
\pgfpathlineto{\pgfqpoint{3.667704in}{3.121929in}}%
\pgfpathlineto{\pgfqpoint{3.671986in}{3.117449in}}%
\pgfpathlineto{\pgfqpoint{3.674662in}{3.117265in}}%
\pgfpathlineto{\pgfqpoint{3.677605in}{3.119564in}}%
\pgfpathlineto{\pgfqpoint{3.683225in}{3.127980in}}%
\pgfpathlineto{\pgfqpoint{3.687774in}{3.132933in}}%
\pgfpathlineto{\pgfqpoint{3.690450in}{3.133179in}}%
\pgfpathlineto{\pgfqpoint{3.693394in}{3.130940in}}%
\pgfpathlineto{\pgfqpoint{3.698746in}{3.122962in}}%
\pgfpathlineto{\pgfqpoint{3.703563in}{3.117524in}}%
\pgfpathlineto{\pgfqpoint{3.706507in}{3.117310in}}%
\pgfpathlineto{\pgfqpoint{3.709451in}{3.119706in}}%
\pgfpathlineto{\pgfqpoint{3.715338in}{3.128560in}}%
\pgfpathlineto{\pgfqpoint{3.719620in}{3.132993in}}%
\pgfpathlineto{\pgfqpoint{3.722296in}{3.133137in}}%
\pgfpathlineto{\pgfqpoint{3.725239in}{3.130800in}}%
\pgfpathlineto{\pgfqpoint{3.730859in}{3.122367in}}%
\pgfpathlineto{\pgfqpoint{3.735409in}{3.117462in}}%
\pgfpathlineto{\pgfqpoint{3.738085in}{3.117256in}}%
\pgfpathlineto{\pgfqpoint{3.741028in}{3.119534in}}%
\pgfpathlineto{\pgfqpoint{3.746381in}{3.127530in}}%
\pgfpathlineto{\pgfqpoint{3.751197in}{3.132920in}}%
\pgfpathlineto{\pgfqpoint{3.753874in}{3.133187in}}%
\pgfpathlineto{\pgfqpoint{3.756817in}{3.130969in}}%
\pgfpathlineto{\pgfqpoint{3.762169in}{3.123003in}}%
\pgfpathlineto{\pgfqpoint{3.766986in}{3.117538in}}%
\pgfpathlineto{\pgfqpoint{3.769930in}{3.117300in}}%
\pgfpathlineto{\pgfqpoint{3.772874in}{3.119675in}}%
\pgfpathlineto{\pgfqpoint{3.778761in}{3.128522in}}%
\pgfpathlineto{\pgfqpoint{3.783043in}{3.132980in}}%
\pgfpathlineto{\pgfqpoint{3.785719in}{3.133146in}}%
\pgfpathlineto{\pgfqpoint{3.788663in}{3.130830in}}%
\pgfpathlineto{\pgfqpoint{3.794282in}{3.122407in}}%
\pgfpathlineto{\pgfqpoint{3.798832in}{3.117475in}}%
\pgfpathlineto{\pgfqpoint{3.801508in}{3.117247in}}%
\pgfpathlineto{\pgfqpoint{3.804451in}{3.119504in}}%
\pgfpathlineto{\pgfqpoint{3.809804in}{3.127490in}}%
\pgfpathlineto{\pgfqpoint{3.814621in}{3.132906in}}%
\pgfpathlineto{\pgfqpoint{3.817297in}{3.133196in}}%
\pgfpathlineto{\pgfqpoint{3.820240in}{3.130999in}}%
\pgfpathlineto{\pgfqpoint{3.825325in}{3.123462in}}%
\pgfpathlineto{\pgfqpoint{3.830409in}{3.117552in}}%
\pgfpathlineto{\pgfqpoint{3.833353in}{3.117290in}}%
\pgfpathlineto{\pgfqpoint{3.836297in}{3.119645in}}%
\pgfpathlineto{\pgfqpoint{3.842184in}{3.128484in}}%
\pgfpathlineto{\pgfqpoint{3.846466in}{3.132968in}}%
\pgfpathlineto{\pgfqpoint{3.849142in}{3.133155in}}%
\pgfpathlineto{\pgfqpoint{3.852086in}{3.130860in}}%
\pgfpathlineto{\pgfqpoint{3.857705in}{3.122446in}}%
\pgfpathlineto{\pgfqpoint{3.862255in}{3.117488in}}%
\pgfpathlineto{\pgfqpoint{3.864931in}{3.117238in}}%
\pgfpathlineto{\pgfqpoint{3.867874in}{3.119474in}}%
\pgfpathlineto{\pgfqpoint{3.873227in}{3.127450in}}%
\pgfpathlineto{\pgfqpoint{3.878044in}{3.132892in}}%
\pgfpathlineto{\pgfqpoint{3.880987in}{3.133110in}}%
\pgfpathlineto{\pgfqpoint{3.883931in}{3.130718in}}%
\pgfpathlineto{\pgfqpoint{3.889818in}{3.121865in}}%
\pgfpathlineto{\pgfqpoint{3.894100in}{3.117428in}}%
\pgfpathlineto{\pgfqpoint{3.896776in}{3.117281in}}%
\pgfpathlineto{\pgfqpoint{3.899720in}{3.119614in}}%
\pgfpathlineto{\pgfqpoint{3.905340in}{3.128045in}}%
\pgfpathlineto{\pgfqpoint{3.909889in}{3.132955in}}%
\pgfpathlineto{\pgfqpoint{3.912565in}{3.133164in}}%
\pgfpathlineto{\pgfqpoint{3.915509in}{3.130890in}}%
\pgfpathlineto{\pgfqpoint{3.920861in}{3.122896in}}%
\pgfpathlineto{\pgfqpoint{3.925678in}{3.117502in}}%
\pgfpathlineto{\pgfqpoint{3.928354in}{3.117230in}}%
\pgfpathlineto{\pgfqpoint{3.931298in}{3.119445in}}%
\pgfpathlineto{\pgfqpoint{3.936650in}{3.127410in}}%
\pgfpathlineto{\pgfqpoint{3.941467in}{3.132878in}}%
\pgfpathlineto{\pgfqpoint{3.944410in}{3.133120in}}%
\pgfpathlineto{\pgfqpoint{3.947354in}{3.130749in}}%
\pgfpathlineto{\pgfqpoint{3.953241in}{3.121903in}}%
\pgfpathlineto{\pgfqpoint{3.957523in}{3.117441in}}%
\pgfpathlineto{\pgfqpoint{3.960199in}{3.117271in}}%
\pgfpathlineto{\pgfqpoint{3.963143in}{3.119584in}}%
\pgfpathlineto{\pgfqpoint{3.968763in}{3.128006in}}%
\pgfpathlineto{\pgfqpoint{3.973312in}{3.132942in}}%
\pgfpathlineto{\pgfqpoint{3.975988in}{3.133173in}}%
\pgfpathlineto{\pgfqpoint{3.978932in}{3.130920in}}%
\pgfpathlineto{\pgfqpoint{3.984284in}{3.122936in}}%
\pgfpathlineto{\pgfqpoint{3.989101in}{3.117515in}}%
\pgfpathlineto{\pgfqpoint{3.991777in}{3.117222in}}%
\pgfpathlineto{\pgfqpoint{3.994721in}{3.119415in}}%
\pgfpathlineto{\pgfqpoint{3.999805in}{3.126950in}}%
\pgfpathlineto{\pgfqpoint{4.004890in}{3.132864in}}%
\pgfpathlineto{\pgfqpoint{4.007833in}{3.133130in}}%
\pgfpathlineto{\pgfqpoint{4.010777in}{3.130779in}}%
\pgfpathlineto{\pgfqpoint{4.016397in}{3.122341in}}%
\pgfpathlineto{\pgfqpoint{4.020946in}{3.117453in}}%
\pgfpathlineto{\pgfqpoint{4.023622in}{3.117262in}}%
\pgfpathlineto{\pgfqpoint{4.026566in}{3.119554in}}%
\pgfpathlineto{\pgfqpoint{4.032186in}{3.127966in}}%
\pgfpathlineto{\pgfqpoint{4.036735in}{3.132929in}}%
\pgfpathlineto{\pgfqpoint{4.039411in}{3.133182in}}%
\pgfpathlineto{\pgfqpoint{4.042355in}{3.130950in}}%
\pgfpathlineto{\pgfqpoint{4.047707in}{3.122976in}}%
\pgfpathlineto{\pgfqpoint{4.052524in}{3.117529in}}%
\pgfpathlineto{\pgfqpoint{4.055468in}{3.117307in}}%
\pgfpathlineto{\pgfqpoint{4.058411in}{3.119696in}}%
\pgfpathlineto{\pgfqpoint{4.064299in}{3.128547in}}%
\pgfpathlineto{\pgfqpoint{4.068580in}{3.132988in}}%
\pgfpathlineto{\pgfqpoint{4.071256in}{3.133140in}}%
\pgfpathlineto{\pgfqpoint{4.074200in}{3.130810in}}%
\pgfpathlineto{\pgfqpoint{4.079820in}{3.122380in}}%
\pgfpathlineto{\pgfqpoint{4.084369in}{3.117466in}}%
\pgfpathlineto{\pgfqpoint{4.087045in}{3.117253in}}%
\pgfpathlineto{\pgfqpoint{4.089989in}{3.119524in}}%
\pgfpathlineto{\pgfqpoint{4.095341in}{3.127517in}}%
\pgfpathlineto{\pgfqpoint{4.100158in}{3.132915in}}%
\pgfpathlineto{\pgfqpoint{4.102834in}{3.133190in}}%
\pgfpathlineto{\pgfqpoint{4.105778in}{3.130979in}}%
\pgfpathlineto{\pgfqpoint{4.110862in}{3.123435in}}%
\pgfpathlineto{\pgfqpoint{4.115947in}{3.117543in}}%
\pgfpathlineto{\pgfqpoint{4.118891in}{3.117297in}}%
\pgfpathlineto{\pgfqpoint{4.121834in}{3.119665in}}%
\pgfpathlineto{\pgfqpoint{4.127722in}{3.128509in}}%
\pgfpathlineto{\pgfqpoint{4.132003in}{3.132976in}}%
\pgfpathlineto{\pgfqpoint{4.134679in}{3.133149in}}%
\pgfpathlineto{\pgfqpoint{4.137623in}{3.130840in}}%
\pgfpathlineto{\pgfqpoint{4.143243in}{3.122420in}}%
\pgfpathlineto{\pgfqpoint{4.147792in}{3.117479in}}%
\pgfpathlineto{\pgfqpoint{4.150468in}{3.117244in}}%
\pgfpathlineto{\pgfqpoint{4.153412in}{3.119494in}}%
\pgfpathlineto{\pgfqpoint{4.158764in}{3.127476in}}%
\pgfpathlineto{\pgfqpoint{4.163581in}{3.132901in}}%
\pgfpathlineto{\pgfqpoint{4.166257in}{3.133198in}}%
\pgfpathlineto{\pgfqpoint{4.169201in}{3.131009in}}%
\pgfpathlineto{\pgfqpoint{4.174285in}{3.123475in}}%
\pgfpathlineto{\pgfqpoint{4.179370in}{3.117557in}}%
\pgfpathlineto{\pgfqpoint{4.182314in}{3.117287in}}%
\pgfpathlineto{\pgfqpoint{4.185257in}{3.119635in}}%
\pgfpathlineto{\pgfqpoint{4.190877in}{3.128071in}}%
\pgfpathlineto{\pgfqpoint{4.195426in}{3.132963in}}%
\pgfpathlineto{\pgfqpoint{4.198103in}{3.133158in}}%
\pgfpathlineto{\pgfqpoint{4.201046in}{3.130870in}}%
\pgfpathlineto{\pgfqpoint{4.206398in}{3.122869in}}%
\pgfpathlineto{\pgfqpoint{4.211215in}{3.117493in}}%
\pgfpathlineto{\pgfqpoint{4.213891in}{3.117236in}}%
\pgfpathlineto{\pgfqpoint{4.216835in}{3.119464in}}%
\pgfpathlineto{\pgfqpoint{4.222187in}{3.127436in}}%
\pgfpathlineto{\pgfqpoint{4.227004in}{3.132888in}}%
\pgfpathlineto{\pgfqpoint{4.229948in}{3.133114in}}%
\pgfpathlineto{\pgfqpoint{4.232892in}{3.130728in}}%
\pgfpathlineto{\pgfqpoint{4.238779in}{3.121878in}}%
\pgfpathlineto{\pgfqpoint{4.243061in}{3.117432in}}%
\pgfpathlineto{\pgfqpoint{4.245737in}{3.117278in}}%
\pgfpathlineto{\pgfqpoint{4.248680in}{3.119604in}}%
\pgfpathlineto{\pgfqpoint{4.254300in}{3.128032in}}%
\pgfpathlineto{\pgfqpoint{4.258849in}{3.132950in}}%
\pgfpathlineto{\pgfqpoint{4.261526in}{3.133167in}}%
\pgfpathlineto{\pgfqpoint{4.264469in}{3.130900in}}%
\pgfpathlineto{\pgfqpoint{4.269821in}{3.122909in}}%
\pgfpathlineto{\pgfqpoint{4.274638in}{3.117506in}}%
\pgfpathlineto{\pgfqpoint{4.277314in}{3.117227in}}%
\pgfpathlineto{\pgfqpoint{4.280258in}{3.119435in}}%
\pgfpathlineto{\pgfqpoint{4.285343in}{3.126977in}}%
\pgfpathlineto{\pgfqpoint{4.290427in}{3.132874in}}%
\pgfpathlineto{\pgfqpoint{4.293371in}{3.133124in}}%
\pgfpathlineto{\pgfqpoint{4.296315in}{3.130759in}}%
\pgfpathlineto{\pgfqpoint{4.302202in}{3.121916in}}%
\pgfpathlineto{\pgfqpoint{4.306484in}{3.117445in}}%
\pgfpathlineto{\pgfqpoint{4.309160in}{3.117268in}}%
\pgfpathlineto{\pgfqpoint{4.312103in}{3.119574in}}%
\pgfpathlineto{\pgfqpoint{4.317723in}{3.127993in}}%
\pgfpathlineto{\pgfqpoint{4.322273in}{3.132937in}}%
\pgfpathlineto{\pgfqpoint{4.324949in}{3.133176in}}%
\pgfpathlineto{\pgfqpoint{4.327892in}{3.130930in}}%
\pgfpathlineto{\pgfqpoint{4.333244in}{3.122949in}}%
\pgfpathlineto{\pgfqpoint{4.338061in}{3.117520in}}%
\pgfpathlineto{\pgfqpoint{4.340737in}{3.117219in}}%
\pgfpathlineto{\pgfqpoint{4.343681in}{3.119405in}}%
\pgfpathlineto{\pgfqpoint{4.348766in}{3.126937in}}%
\pgfpathlineto{\pgfqpoint{4.353850in}{3.132859in}}%
\pgfpathlineto{\pgfqpoint{4.356794in}{3.133133in}}%
\pgfpathlineto{\pgfqpoint{4.359738in}{3.130789in}}%
\pgfpathlineto{\pgfqpoint{4.365357in}{3.122354in}}%
\pgfpathlineto{\pgfqpoint{4.369907in}{3.117458in}}%
\pgfpathlineto{\pgfqpoint{4.372583in}{3.117259in}}%
\pgfpathlineto{\pgfqpoint{4.375526in}{3.119544in}}%
\pgfpathlineto{\pgfqpoint{4.380879in}{3.127543in}}%
\pgfpathlineto{\pgfqpoint{4.385696in}{3.132924in}}%
\pgfpathlineto{\pgfqpoint{4.388372in}{3.133185in}}%
\pgfpathlineto{\pgfqpoint{4.391315in}{3.130959in}}%
\pgfpathlineto{\pgfqpoint{4.396668in}{3.122989in}}%
\pgfpathlineto{\pgfqpoint{4.401484in}{3.117534in}}%
\pgfpathlineto{\pgfqpoint{4.404428in}{3.117304in}}%
\pgfpathlineto{\pgfqpoint{4.407372in}{3.119686in}}%
\pgfpathlineto{\pgfqpoint{4.413259in}{3.128535in}}%
\pgfpathlineto{\pgfqpoint{4.417541in}{3.132984in}}%
\pgfpathlineto{\pgfqpoint{4.420217in}{3.133143in}}%
\pgfpathlineto{\pgfqpoint{4.423161in}{3.130820in}}%
\pgfpathlineto{\pgfqpoint{4.428780in}{3.122394in}}%
\pgfpathlineto{\pgfqpoint{4.433330in}{3.117471in}}%
\pgfpathlineto{\pgfqpoint{4.436006in}{3.117250in}}%
\pgfpathlineto{\pgfqpoint{4.438950in}{3.119514in}}%
\pgfpathlineto{\pgfqpoint{4.444302in}{3.127503in}}%
\pgfpathlineto{\pgfqpoint{4.449119in}{3.132911in}}%
\pgfpathlineto{\pgfqpoint{4.451795in}{3.133193in}}%
\pgfpathlineto{\pgfqpoint{4.454738in}{3.130989in}}%
\pgfpathlineto{\pgfqpoint{4.459823in}{3.123448in}}%
\pgfpathlineto{\pgfqpoint{4.464908in}{3.117548in}}%
\pgfpathlineto{\pgfqpoint{4.467851in}{3.117294in}}%
\pgfpathlineto{\pgfqpoint{4.470795in}{3.119655in}}%
\pgfpathlineto{\pgfqpoint{4.476682in}{3.128496in}}%
\pgfpathlineto{\pgfqpoint{4.480964in}{3.132972in}}%
\pgfpathlineto{\pgfqpoint{4.483640in}{3.133152in}}%
\pgfpathlineto{\pgfqpoint{4.486584in}{3.130850in}}%
\pgfpathlineto{\pgfqpoint{4.492204in}{3.122433in}}%
\pgfpathlineto{\pgfqpoint{4.496753in}{3.117484in}}%
\pgfpathlineto{\pgfqpoint{4.499429in}{3.117241in}}%
\pgfpathlineto{\pgfqpoint{4.502373in}{3.119484in}}%
\pgfpathlineto{\pgfqpoint{4.507725in}{3.127463in}}%
\pgfpathlineto{\pgfqpoint{4.512542in}{3.132897in}}%
\pgfpathlineto{\pgfqpoint{4.515485in}{3.133107in}}%
\pgfpathlineto{\pgfqpoint{4.518429in}{3.130708in}}%
\pgfpathlineto{\pgfqpoint{4.524584in}{3.121464in}}%
\pgfpathlineto{\pgfqpoint{4.528866in}{3.117309in}}%
\pgfpathlineto{\pgfqpoint{4.531542in}{3.117394in}}%
\pgfpathlineto{\pgfqpoint{4.534486in}{3.119946in}}%
\pgfpathlineto{\pgfqpoint{4.546260in}{3.133339in}}%
\pgfpathlineto{\pgfqpoint{4.548936in}{3.131983in}}%
\pgfpathlineto{\pgfqpoint{4.552683in}{3.127147in}}%
\pgfpathlineto{\pgfqpoint{4.559908in}{3.117646in}}%
\pgfpathlineto{\pgfqpoint{4.562852in}{3.117233in}}%
\pgfpathlineto{\pgfqpoint{4.565796in}{3.119455in}}%
\pgfpathlineto{\pgfqpoint{4.571148in}{3.127423in}}%
\pgfpathlineto{\pgfqpoint{4.575965in}{3.132883in}}%
\pgfpathlineto{\pgfqpoint{4.578908in}{3.133117in}}%
\pgfpathlineto{\pgfqpoint{4.581852in}{3.130738in}}%
\pgfpathlineto{\pgfqpoint{4.587739in}{3.121891in}}%
\pgfpathlineto{\pgfqpoint{4.592021in}{3.117437in}}%
\pgfpathlineto{\pgfqpoint{4.594697in}{3.117274in}}%
\pgfpathlineto{\pgfqpoint{4.597641in}{3.119594in}}%
\pgfpathlineto{\pgfqpoint{4.603261in}{3.128019in}}%
\pgfpathlineto{\pgfqpoint{4.607810in}{3.132946in}}%
\pgfpathlineto{\pgfqpoint{4.610486in}{3.133170in}}%
\pgfpathlineto{\pgfqpoint{4.613430in}{3.130910in}}%
\pgfpathlineto{\pgfqpoint{4.618782in}{3.122922in}}%
\pgfpathlineto{\pgfqpoint{4.623599in}{3.117511in}}%
\pgfpathlineto{\pgfqpoint{4.626275in}{3.117224in}}%
\pgfpathlineto{\pgfqpoint{4.629219in}{3.119425in}}%
\pgfpathlineto{\pgfqpoint{4.634303in}{3.126964in}}%
\pgfpathlineto{\pgfqpoint{4.639388in}{3.132869in}}%
\pgfpathlineto{\pgfqpoint{4.642331in}{3.133127in}}%
\pgfpathlineto{\pgfqpoint{4.645275in}{3.130769in}}%
\pgfpathlineto{\pgfqpoint{4.651163in}{3.121929in}}%
\pgfpathlineto{\pgfqpoint{4.655444in}{3.117449in}}%
\pgfpathlineto{\pgfqpoint{4.658120in}{3.117265in}}%
\pgfpathlineto{\pgfqpoint{4.661064in}{3.119564in}}%
\pgfpathlineto{\pgfqpoint{4.666684in}{3.127980in}}%
\pgfpathlineto{\pgfqpoint{4.671233in}{3.132933in}}%
\pgfpathlineto{\pgfqpoint{4.673909in}{3.133179in}}%
\pgfpathlineto{\pgfqpoint{4.676853in}{3.130940in}}%
\pgfpathlineto{\pgfqpoint{4.682205in}{3.122962in}}%
\pgfpathlineto{\pgfqpoint{4.687022in}{3.117524in}}%
\pgfpathlineto{\pgfqpoint{4.689966in}{3.117310in}}%
\pgfpathlineto{\pgfqpoint{4.692909in}{3.119706in}}%
\pgfpathlineto{\pgfqpoint{4.698797in}{3.128560in}}%
\pgfpathlineto{\pgfqpoint{4.703078in}{3.132993in}}%
\pgfpathlineto{\pgfqpoint{4.705755in}{3.133137in}}%
\pgfpathlineto{\pgfqpoint{4.708698in}{3.130800in}}%
\pgfpathlineto{\pgfqpoint{4.714318in}{3.122367in}}%
\pgfpathlineto{\pgfqpoint{4.718867in}{3.117462in}}%
\pgfpathlineto{\pgfqpoint{4.721543in}{3.117256in}}%
\pgfpathlineto{\pgfqpoint{4.724487in}{3.119534in}}%
\pgfpathlineto{\pgfqpoint{4.729839in}{3.127530in}}%
\pgfpathlineto{\pgfqpoint{4.734656in}{3.132920in}}%
\pgfpathlineto{\pgfqpoint{4.737332in}{3.133187in}}%
\pgfpathlineto{\pgfqpoint{4.740276in}{3.130969in}}%
\pgfpathlineto{\pgfqpoint{4.745628in}{3.123003in}}%
\pgfpathlineto{\pgfqpoint{4.750445in}{3.117538in}}%
\pgfpathlineto{\pgfqpoint{4.753389in}{3.117300in}}%
\pgfpathlineto{\pgfqpoint{4.756332in}{3.119675in}}%
\pgfpathlineto{\pgfqpoint{4.762220in}{3.128522in}}%
\pgfpathlineto{\pgfqpoint{4.766502in}{3.132980in}}%
\pgfpathlineto{\pgfqpoint{4.769178in}{3.133146in}}%
\pgfpathlineto{\pgfqpoint{4.772121in}{3.130830in}}%
\pgfpathlineto{\pgfqpoint{4.777741in}{3.122407in}}%
\pgfpathlineto{\pgfqpoint{4.782290in}{3.117475in}}%
\pgfpathlineto{\pgfqpoint{4.784966in}{3.117247in}}%
\pgfpathlineto{\pgfqpoint{4.787910in}{3.119504in}}%
\pgfpathlineto{\pgfqpoint{4.793262in}{3.127490in}}%
\pgfpathlineto{\pgfqpoint{4.798079in}{3.132906in}}%
\pgfpathlineto{\pgfqpoint{4.800755in}{3.133196in}}%
\pgfpathlineto{\pgfqpoint{4.803699in}{3.130999in}}%
\pgfpathlineto{\pgfqpoint{4.808784in}{3.123462in}}%
\pgfpathlineto{\pgfqpoint{4.813868in}{3.117552in}}%
\pgfpathlineto{\pgfqpoint{4.816812in}{3.117290in}}%
\pgfpathlineto{\pgfqpoint{4.819755in}{3.119645in}}%
\pgfpathlineto{\pgfqpoint{4.825643in}{3.128484in}}%
\pgfpathlineto{\pgfqpoint{4.829925in}{3.132968in}}%
\pgfpathlineto{\pgfqpoint{4.832601in}{3.133155in}}%
\pgfpathlineto{\pgfqpoint{4.835544in}{3.130860in}}%
\pgfpathlineto{\pgfqpoint{4.841164in}{3.122446in}}%
\pgfpathlineto{\pgfqpoint{4.845713in}{3.117488in}}%
\pgfpathlineto{\pgfqpoint{4.848390in}{3.117238in}}%
\pgfpathlineto{\pgfqpoint{4.851333in}{3.119474in}}%
\pgfpathlineto{\pgfqpoint{4.856685in}{3.127450in}}%
\pgfpathlineto{\pgfqpoint{4.861502in}{3.132892in}}%
\pgfpathlineto{\pgfqpoint{4.864446in}{3.133110in}}%
\pgfpathlineto{\pgfqpoint{4.867390in}{3.130718in}}%
\pgfpathlineto{\pgfqpoint{4.873277in}{3.121865in}}%
\pgfpathlineto{\pgfqpoint{4.877559in}{3.117428in}}%
\pgfpathlineto{\pgfqpoint{4.880235in}{3.117281in}}%
\pgfpathlineto{\pgfqpoint{4.883179in}{3.119614in}}%
\pgfpathlineto{\pgfqpoint{4.888798in}{3.128045in}}%
\pgfpathlineto{\pgfqpoint{4.893348in}{3.132955in}}%
\pgfpathlineto{\pgfqpoint{4.896024in}{3.133164in}}%
\pgfpathlineto{\pgfqpoint{4.898967in}{3.130890in}}%
\pgfpathlineto{\pgfqpoint{4.904320in}{3.122896in}}%
\pgfpathlineto{\pgfqpoint{4.909136in}{3.117502in}}%
\pgfpathlineto{\pgfqpoint{4.911813in}{3.117230in}}%
\pgfpathlineto{\pgfqpoint{4.914756in}{3.119445in}}%
\pgfpathlineto{\pgfqpoint{4.920108in}{3.127410in}}%
\pgfpathlineto{\pgfqpoint{4.924925in}{3.132878in}}%
\pgfpathlineto{\pgfqpoint{4.927869in}{3.133120in}}%
\pgfpathlineto{\pgfqpoint{4.930813in}{3.130749in}}%
\pgfpathlineto{\pgfqpoint{4.936700in}{3.121903in}}%
\pgfpathlineto{\pgfqpoint{4.940982in}{3.117441in}}%
\pgfpathlineto{\pgfqpoint{4.943658in}{3.117271in}}%
\pgfpathlineto{\pgfqpoint{4.946602in}{3.119584in}}%
\pgfpathlineto{\pgfqpoint{4.952221in}{3.128006in}}%
\pgfpathlineto{\pgfqpoint{4.956771in}{3.132942in}}%
\pgfpathlineto{\pgfqpoint{4.959447in}{3.133173in}}%
\pgfpathlineto{\pgfqpoint{4.962390in}{3.130920in}}%
\pgfpathlineto{\pgfqpoint{4.967743in}{3.122936in}}%
\pgfpathlineto{\pgfqpoint{4.972560in}{3.117515in}}%
\pgfpathlineto{\pgfqpoint{4.975236in}{3.117222in}}%
\pgfpathlineto{\pgfqpoint{4.978179in}{3.119415in}}%
\pgfpathlineto{\pgfqpoint{4.983264in}{3.126950in}}%
\pgfpathlineto{\pgfqpoint{4.988348in}{3.132864in}}%
\pgfpathlineto{\pgfqpoint{4.991292in}{3.133130in}}%
\pgfpathlineto{\pgfqpoint{4.994236in}{3.130779in}}%
\pgfpathlineto{\pgfqpoint{4.999856in}{3.122341in}}%
\pgfpathlineto{\pgfqpoint{5.004405in}{3.117453in}}%
\pgfpathlineto{\pgfqpoint{5.007081in}{3.117262in}}%
\pgfpathlineto{\pgfqpoint{5.010025in}{3.119554in}}%
\pgfpathlineto{\pgfqpoint{5.015644in}{3.127966in}}%
\pgfpathlineto{\pgfqpoint{5.020194in}{3.132929in}}%
\pgfpathlineto{\pgfqpoint{5.022870in}{3.133182in}}%
\pgfpathlineto{\pgfqpoint{5.025813in}{3.130950in}}%
\pgfpathlineto{\pgfqpoint{5.031166in}{3.122976in}}%
\pgfpathlineto{\pgfqpoint{5.035983in}{3.117529in}}%
\pgfpathlineto{\pgfqpoint{5.038926in}{3.117307in}}%
\pgfpathlineto{\pgfqpoint{5.041870in}{3.119696in}}%
\pgfpathlineto{\pgfqpoint{5.047757in}{3.128547in}}%
\pgfpathlineto{\pgfqpoint{5.052039in}{3.132988in}}%
\pgfpathlineto{\pgfqpoint{5.054715in}{3.133140in}}%
\pgfpathlineto{\pgfqpoint{5.057659in}{3.130810in}}%
\pgfpathlineto{\pgfqpoint{5.063279in}{3.122380in}}%
\pgfpathlineto{\pgfqpoint{5.067828in}{3.117466in}}%
\pgfpathlineto{\pgfqpoint{5.070504in}{3.117253in}}%
\pgfpathlineto{\pgfqpoint{5.073448in}{3.119524in}}%
\pgfpathlineto{\pgfqpoint{5.078800in}{3.127517in}}%
\pgfpathlineto{\pgfqpoint{5.083617in}{3.132915in}}%
\pgfpathlineto{\pgfqpoint{5.086293in}{3.133190in}}%
\pgfpathlineto{\pgfqpoint{5.089237in}{3.130979in}}%
\pgfpathlineto{\pgfqpoint{5.094321in}{3.123435in}}%
\pgfpathlineto{\pgfqpoint{5.099406in}{3.117543in}}%
\pgfpathlineto{\pgfqpoint{5.102349in}{3.117297in}}%
\pgfpathlineto{\pgfqpoint{5.105293in}{3.119665in}}%
\pgfpathlineto{\pgfqpoint{5.111180in}{3.128509in}}%
\pgfpathlineto{\pgfqpoint{5.115462in}{3.132976in}}%
\pgfpathlineto{\pgfqpoint{5.118138in}{3.133149in}}%
\pgfpathlineto{\pgfqpoint{5.121082in}{3.130840in}}%
\pgfpathlineto{\pgfqpoint{5.126702in}{3.122420in}}%
\pgfpathlineto{\pgfqpoint{5.131251in}{3.117479in}}%
\pgfpathlineto{\pgfqpoint{5.133927in}{3.117244in}}%
\pgfpathlineto{\pgfqpoint{5.136871in}{3.119494in}}%
\pgfpathlineto{\pgfqpoint{5.142223in}{3.127476in}}%
\pgfpathlineto{\pgfqpoint{5.147040in}{3.132901in}}%
\pgfpathlineto{\pgfqpoint{5.149716in}{3.133198in}}%
\pgfpathlineto{\pgfqpoint{5.152660in}{3.131009in}}%
\pgfpathlineto{\pgfqpoint{5.157744in}{3.123475in}}%
\pgfpathlineto{\pgfqpoint{5.162829in}{3.117557in}}%
\pgfpathlineto{\pgfqpoint{5.165772in}{3.117287in}}%
\pgfpathlineto{\pgfqpoint{5.168716in}{3.119635in}}%
\pgfpathlineto{\pgfqpoint{5.174336in}{3.128071in}}%
\pgfpathlineto{\pgfqpoint{5.178885in}{3.132963in}}%
\pgfpathlineto{\pgfqpoint{5.181561in}{3.133158in}}%
\pgfpathlineto{\pgfqpoint{5.184505in}{3.130870in}}%
\pgfpathlineto{\pgfqpoint{5.189857in}{3.122869in}}%
\pgfpathlineto{\pgfqpoint{5.194674in}{3.117493in}}%
\pgfpathlineto{\pgfqpoint{5.197350in}{3.117236in}}%
\pgfpathlineto{\pgfqpoint{5.200294in}{3.119464in}}%
\pgfpathlineto{\pgfqpoint{5.205646in}{3.127436in}}%
\pgfpathlineto{\pgfqpoint{5.210463in}{3.132888in}}%
\pgfpathlineto{\pgfqpoint{5.213407in}{3.133114in}}%
\pgfpathlineto{\pgfqpoint{5.216350in}{3.130728in}}%
\pgfpathlineto{\pgfqpoint{5.222238in}{3.121878in}}%
\pgfpathlineto{\pgfqpoint{5.226519in}{3.117432in}}%
\pgfpathlineto{\pgfqpoint{5.229195in}{3.117278in}}%
\pgfpathlineto{\pgfqpoint{5.232139in}{3.119604in}}%
\pgfpathlineto{\pgfqpoint{5.237759in}{3.128032in}}%
\pgfpathlineto{\pgfqpoint{5.242308in}{3.132950in}}%
\pgfpathlineto{\pgfqpoint{5.244984in}{3.133167in}}%
\pgfpathlineto{\pgfqpoint{5.247928in}{3.130900in}}%
\pgfpathlineto{\pgfqpoint{5.253280in}{3.122909in}}%
\pgfpathlineto{\pgfqpoint{5.258097in}{3.117506in}}%
\pgfpathlineto{\pgfqpoint{5.260773in}{3.117227in}}%
\pgfpathlineto{\pgfqpoint{5.263717in}{3.119435in}}%
\pgfpathlineto{\pgfqpoint{5.268801in}{3.126977in}}%
\pgfpathlineto{\pgfqpoint{5.273886in}{3.132874in}}%
\pgfpathlineto{\pgfqpoint{5.276830in}{3.133124in}}%
\pgfpathlineto{\pgfqpoint{5.279773in}{3.130759in}}%
\pgfpathlineto{\pgfqpoint{5.285661in}{3.121916in}}%
\pgfpathlineto{\pgfqpoint{5.289942in}{3.117445in}}%
\pgfpathlineto{\pgfqpoint{5.292618in}{3.117268in}}%
\pgfpathlineto{\pgfqpoint{5.295562in}{3.119574in}}%
\pgfpathlineto{\pgfqpoint{5.301182in}{3.127993in}}%
\pgfpathlineto{\pgfqpoint{5.305731in}{3.132937in}}%
\pgfpathlineto{\pgfqpoint{5.308407in}{3.133176in}}%
\pgfpathlineto{\pgfqpoint{5.311351in}{3.130930in}}%
\pgfpathlineto{\pgfqpoint{5.316703in}{3.122949in}}%
\pgfpathlineto{\pgfqpoint{5.321520in}{3.117520in}}%
\pgfpathlineto{\pgfqpoint{5.324196in}{3.117219in}}%
\pgfpathlineto{\pgfqpoint{5.327140in}{3.119405in}}%
\pgfpathlineto{\pgfqpoint{5.332224in}{3.126937in}}%
\pgfpathlineto{\pgfqpoint{5.337309in}{3.132859in}}%
\pgfpathlineto{\pgfqpoint{5.340253in}{3.133133in}}%
\pgfpathlineto{\pgfqpoint{5.343196in}{3.130789in}}%
\pgfpathlineto{\pgfqpoint{5.348816in}{3.122354in}}%
\pgfpathlineto{\pgfqpoint{5.353365in}{3.117458in}}%
\pgfpathlineto{\pgfqpoint{5.356042in}{3.117259in}}%
\pgfpathlineto{\pgfqpoint{5.358985in}{3.119544in}}%
\pgfpathlineto{\pgfqpoint{5.364337in}{3.127543in}}%
\pgfpathlineto{\pgfqpoint{5.369154in}{3.132924in}}%
\pgfpathlineto{\pgfqpoint{5.371830in}{3.133185in}}%
\pgfpathlineto{\pgfqpoint{5.374774in}{3.130959in}}%
\pgfpathlineto{\pgfqpoint{5.380126in}{3.122989in}}%
\pgfpathlineto{\pgfqpoint{5.384943in}{3.117534in}}%
\pgfpathlineto{\pgfqpoint{5.387887in}{3.117304in}}%
\pgfpathlineto{\pgfqpoint{5.390831in}{3.119686in}}%
\pgfpathlineto{\pgfqpoint{5.396718in}{3.128535in}}%
\pgfpathlineto{\pgfqpoint{5.401000in}{3.132984in}}%
\pgfpathlineto{\pgfqpoint{5.403676in}{3.133143in}}%
\pgfpathlineto{\pgfqpoint{5.406619in}{3.130820in}}%
\pgfpathlineto{\pgfqpoint{5.412239in}{3.122394in}}%
\pgfpathlineto{\pgfqpoint{5.416789in}{3.117471in}}%
\pgfpathlineto{\pgfqpoint{5.419465in}{3.117250in}}%
\pgfpathlineto{\pgfqpoint{5.422408in}{3.119514in}}%
\pgfpathlineto{\pgfqpoint{5.427760in}{3.127503in}}%
\pgfpathlineto{\pgfqpoint{5.432577in}{3.132911in}}%
\pgfpathlineto{\pgfqpoint{5.435253in}{3.133193in}}%
\pgfpathlineto{\pgfqpoint{5.438197in}{3.130989in}}%
\pgfpathlineto{\pgfqpoint{5.443282in}{3.123448in}}%
\pgfpathlineto{\pgfqpoint{5.448366in}{3.117548in}}%
\pgfpathlineto{\pgfqpoint{5.451310in}{3.117294in}}%
\pgfpathlineto{\pgfqpoint{5.454254in}{3.119655in}}%
\pgfpathlineto{\pgfqpoint{5.460141in}{3.128496in}}%
\pgfpathlineto{\pgfqpoint{5.464423in}{3.132972in}}%
\pgfpathlineto{\pgfqpoint{5.467099in}{3.133152in}}%
\pgfpathlineto{\pgfqpoint{5.470042in}{3.130850in}}%
\pgfpathlineto{\pgfqpoint{5.475662in}{3.122433in}}%
\pgfpathlineto{\pgfqpoint{5.480212in}{3.117484in}}%
\pgfpathlineto{\pgfqpoint{5.482888in}{3.117241in}}%
\pgfpathlineto{\pgfqpoint{5.485831in}{3.119484in}}%
\pgfpathlineto{\pgfqpoint{5.491184in}{3.127463in}}%
\pgfpathlineto{\pgfqpoint{5.496000in}{3.132897in}}%
\pgfpathlineto{\pgfqpoint{5.498944in}{3.133107in}}%
\pgfpathlineto{\pgfqpoint{5.501888in}{3.130708in}}%
\pgfpathlineto{\pgfqpoint{5.508043in}{3.121464in}}%
\pgfpathlineto{\pgfqpoint{5.512325in}{3.117309in}}%
\pgfpathlineto{\pgfqpoint{5.515001in}{3.117394in}}%
\pgfpathlineto{\pgfqpoint{5.517944in}{3.119946in}}%
\pgfpathlineto{\pgfqpoint{5.529719in}{3.133339in}}%
\pgfpathlineto{\pgfqpoint{5.532395in}{3.131983in}}%
\pgfpathlineto{\pgfqpoint{5.536142in}{3.127147in}}%
\pgfpathlineto{\pgfqpoint{5.543367in}{3.117646in}}%
\pgfpathlineto{\pgfqpoint{5.546311in}{3.117233in}}%
\pgfpathlineto{\pgfqpoint{5.549254in}{3.119455in}}%
\pgfpathlineto{\pgfqpoint{5.554607in}{3.127423in}}%
\pgfpathlineto{\pgfqpoint{5.559423in}{3.132883in}}%
\pgfpathlineto{\pgfqpoint{5.562367in}{3.133117in}}%
\pgfpathlineto{\pgfqpoint{5.565311in}{3.130738in}}%
\pgfpathlineto{\pgfqpoint{5.571198in}{3.121891in}}%
\pgfpathlineto{\pgfqpoint{5.575480in}{3.117437in}}%
\pgfpathlineto{\pgfqpoint{5.578156in}{3.117274in}}%
\pgfpathlineto{\pgfqpoint{5.581100in}{3.119594in}}%
\pgfpathlineto{\pgfqpoint{5.586719in}{3.128019in}}%
\pgfpathlineto{\pgfqpoint{5.591269in}{3.132946in}}%
\pgfpathlineto{\pgfqpoint{5.593945in}{3.133170in}}%
\pgfpathlineto{\pgfqpoint{5.596889in}{3.130910in}}%
\pgfpathlineto{\pgfqpoint{5.602241in}{3.122922in}}%
\pgfpathlineto{\pgfqpoint{5.607058in}{3.117511in}}%
\pgfpathlineto{\pgfqpoint{5.609734in}{3.117224in}}%
\pgfpathlineto{\pgfqpoint{5.612677in}{3.119425in}}%
\pgfpathlineto{\pgfqpoint{5.617762in}{3.126964in}}%
\pgfpathlineto{\pgfqpoint{5.622847in}{3.132869in}}%
\pgfpathlineto{\pgfqpoint{5.625790in}{3.133127in}}%
\pgfpathlineto{\pgfqpoint{5.628734in}{3.130769in}}%
\pgfpathlineto{\pgfqpoint{5.634621in}{3.121929in}}%
\pgfpathlineto{\pgfqpoint{5.638903in}{3.117449in}}%
\pgfpathlineto{\pgfqpoint{5.641579in}{3.117265in}}%
\pgfpathlineto{\pgfqpoint{5.644523in}{3.119564in}}%
\pgfpathlineto{\pgfqpoint{5.650143in}{3.127980in}}%
\pgfpathlineto{\pgfqpoint{5.654692in}{3.132933in}}%
\pgfpathlineto{\pgfqpoint{5.657368in}{3.133179in}}%
\pgfpathlineto{\pgfqpoint{5.660312in}{3.130940in}}%
\pgfpathlineto{\pgfqpoint{5.665664in}{3.122962in}}%
\pgfpathlineto{\pgfqpoint{5.670481in}{3.117524in}}%
\pgfpathlineto{\pgfqpoint{5.673424in}{3.117310in}}%
\pgfpathlineto{\pgfqpoint{5.676368in}{3.119706in}}%
\pgfpathlineto{\pgfqpoint{5.682255in}{3.128560in}}%
\pgfpathlineto{\pgfqpoint{5.686537in}{3.132993in}}%
\pgfpathlineto{\pgfqpoint{5.689213in}{3.133137in}}%
\pgfpathlineto{\pgfqpoint{5.692157in}{3.130800in}}%
\pgfpathlineto{\pgfqpoint{5.697777in}{3.122367in}}%
\pgfpathlineto{\pgfqpoint{5.702326in}{3.117462in}}%
\pgfpathlineto{\pgfqpoint{5.705002in}{3.117256in}}%
\pgfpathlineto{\pgfqpoint{5.707946in}{3.119534in}}%
\pgfpathlineto{\pgfqpoint{5.713298in}{3.127530in}}%
\pgfpathlineto{\pgfqpoint{5.718115in}{3.132920in}}%
\pgfpathlineto{\pgfqpoint{5.720791in}{3.133187in}}%
\pgfpathlineto{\pgfqpoint{5.723735in}{3.130969in}}%
\pgfpathlineto{\pgfqpoint{5.729087in}{3.123003in}}%
\pgfpathlineto{\pgfqpoint{5.733904in}{3.117538in}}%
\pgfpathlineto{\pgfqpoint{5.736847in}{3.117300in}}%
\pgfpathlineto{\pgfqpoint{5.739791in}{3.119675in}}%
\pgfpathlineto{\pgfqpoint{5.745679in}{3.128522in}}%
\pgfpathlineto{\pgfqpoint{5.749960in}{3.132980in}}%
\pgfpathlineto{\pgfqpoint{5.752636in}{3.133146in}}%
\pgfpathlineto{\pgfqpoint{5.755580in}{3.130830in}}%
\pgfpathlineto{\pgfqpoint{5.761200in}{3.122407in}}%
\pgfpathlineto{\pgfqpoint{5.765749in}{3.117475in}}%
\pgfpathlineto{\pgfqpoint{5.768425in}{3.117247in}}%
\pgfpathlineto{\pgfqpoint{5.771369in}{3.119504in}}%
\pgfpathlineto{\pgfqpoint{5.776721in}{3.127490in}}%
\pgfpathlineto{\pgfqpoint{5.781538in}{3.132906in}}%
\pgfpathlineto{\pgfqpoint{5.784214in}{3.133196in}}%
\pgfpathlineto{\pgfqpoint{5.787158in}{3.130999in}}%
\pgfpathlineto{\pgfqpoint{5.792242in}{3.123462in}}%
\pgfpathlineto{\pgfqpoint{5.797327in}{3.117552in}}%
\pgfpathlineto{\pgfqpoint{5.800271in}{3.117290in}}%
\pgfpathlineto{\pgfqpoint{5.803214in}{3.119645in}}%
\pgfpathlineto{\pgfqpoint{5.809102in}{3.128484in}}%
\pgfpathlineto{\pgfqpoint{5.813383in}{3.132968in}}%
\pgfpathlineto{\pgfqpoint{5.816059in}{3.133155in}}%
\pgfpathlineto{\pgfqpoint{5.819003in}{3.130860in}}%
\pgfpathlineto{\pgfqpoint{5.824623in}{3.122446in}}%
\pgfpathlineto{\pgfqpoint{5.829172in}{3.117488in}}%
\pgfpathlineto{\pgfqpoint{5.831848in}{3.117238in}}%
\pgfpathlineto{\pgfqpoint{5.834792in}{3.119474in}}%
\pgfpathlineto{\pgfqpoint{5.840144in}{3.127450in}}%
\pgfpathlineto{\pgfqpoint{5.844961in}{3.132892in}}%
\pgfpathlineto{\pgfqpoint{5.847905in}{3.133110in}}%
\pgfpathlineto{\pgfqpoint{5.850848in}{3.130718in}}%
\pgfpathlineto{\pgfqpoint{5.856736in}{3.121865in}}%
\pgfpathlineto{\pgfqpoint{5.861018in}{3.117428in}}%
\pgfpathlineto{\pgfqpoint{5.863694in}{3.117281in}}%
\pgfpathlineto{\pgfqpoint{5.866637in}{3.119614in}}%
\pgfpathlineto{\pgfqpoint{5.872257in}{3.128045in}}%
\pgfpathlineto{\pgfqpoint{5.876806in}{3.132955in}}%
\pgfpathlineto{\pgfqpoint{5.879482in}{3.133164in}}%
\pgfpathlineto{\pgfqpoint{5.882426in}{3.130890in}}%
\pgfpathlineto{\pgfqpoint{5.887778in}{3.122896in}}%
\pgfpathlineto{\pgfqpoint{5.892595in}{3.117502in}}%
\pgfpathlineto{\pgfqpoint{5.895271in}{3.117230in}}%
\pgfpathlineto{\pgfqpoint{5.898215in}{3.119445in}}%
\pgfpathlineto{\pgfqpoint{5.903567in}{3.127410in}}%
\pgfpathlineto{\pgfqpoint{5.908384in}{3.132878in}}%
\pgfpathlineto{\pgfqpoint{5.911328in}{3.133120in}}%
\pgfpathlineto{\pgfqpoint{5.914271in}{3.130749in}}%
\pgfpathlineto{\pgfqpoint{5.920159in}{3.121903in}}%
\pgfpathlineto{\pgfqpoint{5.924441in}{3.117441in}}%
\pgfpathlineto{\pgfqpoint{5.927117in}{3.117271in}}%
\pgfpathlineto{\pgfqpoint{5.930060in}{3.119584in}}%
\pgfpathlineto{\pgfqpoint{5.935680in}{3.128006in}}%
\pgfpathlineto{\pgfqpoint{5.940229in}{3.132942in}}%
\pgfpathlineto{\pgfqpoint{5.942906in}{3.133173in}}%
\pgfpathlineto{\pgfqpoint{5.945849in}{3.130920in}}%
\pgfpathlineto{\pgfqpoint{5.951201in}{3.122936in}}%
\pgfpathlineto{\pgfqpoint{5.956018in}{3.117515in}}%
\pgfpathlineto{\pgfqpoint{5.958694in}{3.117222in}}%
\pgfpathlineto{\pgfqpoint{5.961638in}{3.119415in}}%
\pgfpathlineto{\pgfqpoint{5.966723in}{3.126950in}}%
\pgfpathlineto{\pgfqpoint{5.971807in}{3.132864in}}%
\pgfpathlineto{\pgfqpoint{5.974751in}{3.133130in}}%
\pgfpathlineto{\pgfqpoint{5.977695in}{3.130779in}}%
\pgfpathlineto{\pgfqpoint{5.983314in}{3.122341in}}%
\pgfpathlineto{\pgfqpoint{5.987864in}{3.117453in}}%
\pgfpathlineto{\pgfqpoint{5.990540in}{3.117262in}}%
\pgfpathlineto{\pgfqpoint{5.993483in}{3.119554in}}%
\pgfpathlineto{\pgfqpoint{5.999103in}{3.127966in}}%
\pgfpathlineto{\pgfqpoint{6.003652in}{3.132929in}}%
\pgfpathlineto{\pgfqpoint{6.006329in}{3.133182in}}%
\pgfpathlineto{\pgfqpoint{6.009272in}{3.130950in}}%
\pgfpathlineto{\pgfqpoint{6.014624in}{3.122976in}}%
\pgfpathlineto{\pgfqpoint{6.019441in}{3.117529in}}%
\pgfpathlineto{\pgfqpoint{6.022385in}{3.117307in}}%
\pgfpathlineto{\pgfqpoint{6.025329in}{3.119696in}}%
\pgfpathlineto{\pgfqpoint{6.031216in}{3.128547in}}%
\pgfpathlineto{\pgfqpoint{6.035498in}{3.132988in}}%
\pgfpathlineto{\pgfqpoint{6.038174in}{3.133140in}}%
\pgfpathlineto{\pgfqpoint{6.041118in}{3.130810in}}%
\pgfpathlineto{\pgfqpoint{6.046737in}{3.122380in}}%
\pgfpathlineto{\pgfqpoint{6.051287in}{3.117466in}}%
\pgfpathlineto{\pgfqpoint{6.053963in}{3.117253in}}%
\pgfpathlineto{\pgfqpoint{6.056906in}{3.119524in}}%
\pgfpathlineto{\pgfqpoint{6.062259in}{3.127517in}}%
\pgfpathlineto{\pgfqpoint{6.067076in}{3.132915in}}%
\pgfpathlineto{\pgfqpoint{6.069752in}{3.133190in}}%
\pgfpathlineto{\pgfqpoint{6.072695in}{3.130979in}}%
\pgfpathlineto{\pgfqpoint{6.077780in}{3.123435in}}%
\pgfpathlineto{\pgfqpoint{6.082864in}{3.117543in}}%
\pgfpathlineto{\pgfqpoint{6.085808in}{3.117297in}}%
\pgfpathlineto{\pgfqpoint{6.088752in}{3.119665in}}%
\pgfpathlineto{\pgfqpoint{6.094639in}{3.128509in}}%
\pgfpathlineto{\pgfqpoint{6.098921in}{3.132976in}}%
\pgfpathlineto{\pgfqpoint{6.101597in}{3.133149in}}%
\pgfpathlineto{\pgfqpoint{6.104541in}{3.130840in}}%
\pgfpathlineto{\pgfqpoint{6.110160in}{3.122420in}}%
\pgfpathlineto{\pgfqpoint{6.114710in}{3.117479in}}%
\pgfpathlineto{\pgfqpoint{6.117386in}{3.117244in}}%
\pgfpathlineto{\pgfqpoint{6.120329in}{3.119494in}}%
\pgfpathlineto{\pgfqpoint{6.125682in}{3.127476in}}%
\pgfpathlineto{\pgfqpoint{6.130499in}{3.132901in}}%
\pgfpathlineto{\pgfqpoint{6.133175in}{3.133198in}}%
\pgfpathlineto{\pgfqpoint{6.136118in}{3.131009in}}%
\pgfpathlineto{\pgfqpoint{6.141203in}{3.123475in}}%
\pgfpathlineto{\pgfqpoint{6.146287in}{3.117557in}}%
\pgfpathlineto{\pgfqpoint{6.149231in}{3.117287in}}%
\pgfpathlineto{\pgfqpoint{6.152175in}{3.119635in}}%
\pgfpathlineto{\pgfqpoint{6.157795in}{3.128071in}}%
\pgfpathlineto{\pgfqpoint{6.162344in}{3.132963in}}%
\pgfpathlineto{\pgfqpoint{6.165020in}{3.133158in}}%
\pgfpathlineto{\pgfqpoint{6.167964in}{3.130870in}}%
\pgfpathlineto{\pgfqpoint{6.173316in}{3.122869in}}%
\pgfpathlineto{\pgfqpoint{6.178133in}{3.117493in}}%
\pgfpathlineto{\pgfqpoint{6.180809in}{3.117236in}}%
\pgfpathlineto{\pgfqpoint{6.183753in}{3.119464in}}%
\pgfpathlineto{\pgfqpoint{6.189105in}{3.127436in}}%
\pgfpathlineto{\pgfqpoint{6.193922in}{3.132888in}}%
\pgfpathlineto{\pgfqpoint{6.196865in}{3.133114in}}%
\pgfpathlineto{\pgfqpoint{6.199809in}{3.130728in}}%
\pgfpathlineto{\pgfqpoint{6.205696in}{3.121878in}}%
\pgfpathlineto{\pgfqpoint{6.209978in}{3.117432in}}%
\pgfpathlineto{\pgfqpoint{6.212654in}{3.117278in}}%
\pgfpathlineto{\pgfqpoint{6.215598in}{3.119604in}}%
\pgfpathlineto{\pgfqpoint{6.221218in}{3.128032in}}%
\pgfpathlineto{\pgfqpoint{6.225767in}{3.132950in}}%
\pgfpathlineto{\pgfqpoint{6.228443in}{3.133167in}}%
\pgfpathlineto{\pgfqpoint{6.231387in}{3.130900in}}%
\pgfpathlineto{\pgfqpoint{6.236739in}{3.122909in}}%
\pgfpathlineto{\pgfqpoint{6.241556in}{3.117506in}}%
\pgfpathlineto{\pgfqpoint{6.244232in}{3.117227in}}%
\pgfpathlineto{\pgfqpoint{6.247176in}{3.119435in}}%
\pgfpathlineto{\pgfqpoint{6.252260in}{3.126977in}}%
\pgfpathlineto{\pgfqpoint{6.257345in}{3.132874in}}%
\pgfpathlineto{\pgfqpoint{6.260288in}{3.133124in}}%
\pgfpathlineto{\pgfqpoint{6.263232in}{3.130759in}}%
\pgfpathlineto{\pgfqpoint{6.269119in}{3.121916in}}%
\pgfpathlineto{\pgfqpoint{6.273401in}{3.117445in}}%
\pgfpathlineto{\pgfqpoint{6.276077in}{3.117268in}}%
\pgfpathlineto{\pgfqpoint{6.279021in}{3.119574in}}%
\pgfpathlineto{\pgfqpoint{6.284641in}{3.127993in}}%
\pgfpathlineto{\pgfqpoint{6.289190in}{3.132937in}}%
\pgfpathlineto{\pgfqpoint{6.291866in}{3.133176in}}%
\pgfpathlineto{\pgfqpoint{6.294810in}{3.130930in}}%
\pgfpathlineto{\pgfqpoint{6.300162in}{3.122949in}}%
\pgfpathlineto{\pgfqpoint{6.304979in}{3.117520in}}%
\pgfpathlineto{\pgfqpoint{6.307655in}{3.117219in}}%
\pgfpathlineto{\pgfqpoint{6.310599in}{3.119405in}}%
\pgfpathlineto{\pgfqpoint{6.315683in}{3.126937in}}%
\pgfpathlineto{\pgfqpoint{6.320768in}{3.132859in}}%
\pgfpathlineto{\pgfqpoint{6.323711in}{3.133133in}}%
\pgfpathlineto{\pgfqpoint{6.326655in}{3.130789in}}%
\pgfpathlineto{\pgfqpoint{6.332275in}{3.122354in}}%
\pgfpathlineto{\pgfqpoint{6.336824in}{3.117458in}}%
\pgfpathlineto{\pgfqpoint{6.339500in}{3.117259in}}%
\pgfpathlineto{\pgfqpoint{6.342444in}{3.119544in}}%
\pgfpathlineto{\pgfqpoint{6.347796in}{3.127543in}}%
\pgfpathlineto{\pgfqpoint{6.352613in}{3.132924in}}%
\pgfpathlineto{\pgfqpoint{6.355289in}{3.133185in}}%
\pgfpathlineto{\pgfqpoint{6.358233in}{3.130959in}}%
\pgfpathlineto{\pgfqpoint{6.363585in}{3.122989in}}%
\pgfpathlineto{\pgfqpoint{6.368402in}{3.117534in}}%
\pgfpathlineto{\pgfqpoint{6.371346in}{3.117304in}}%
\pgfpathlineto{\pgfqpoint{6.374289in}{3.119686in}}%
\pgfpathlineto{\pgfqpoint{6.380177in}{3.128535in}}%
\pgfpathlineto{\pgfqpoint{6.384458in}{3.132984in}}%
\pgfpathlineto{\pgfqpoint{6.387134in}{3.133143in}}%
\pgfpathlineto{\pgfqpoint{6.390078in}{3.130820in}}%
\pgfpathlineto{\pgfqpoint{6.395698in}{3.122394in}}%
\pgfpathlineto{\pgfqpoint{6.400247in}{3.117471in}}%
\pgfpathlineto{\pgfqpoint{6.402923in}{3.117250in}}%
\pgfpathlineto{\pgfqpoint{6.405867in}{3.119514in}}%
\pgfpathlineto{\pgfqpoint{6.411219in}{3.127503in}}%
\pgfpathlineto{\pgfqpoint{6.416036in}{3.132911in}}%
\pgfpathlineto{\pgfqpoint{6.418712in}{3.133193in}}%
\pgfpathlineto{\pgfqpoint{6.421656in}{3.130989in}}%
\pgfpathlineto{\pgfqpoint{6.426740in}{3.123448in}}%
\pgfpathlineto{\pgfqpoint{6.431825in}{3.117548in}}%
\pgfpathlineto{\pgfqpoint{6.434769in}{3.117294in}}%
\pgfpathlineto{\pgfqpoint{6.437712in}{3.119655in}}%
\pgfpathlineto{\pgfqpoint{6.443600in}{3.128496in}}%
\pgfpathlineto{\pgfqpoint{6.447881in}{3.132972in}}%
\pgfpathlineto{\pgfqpoint{6.450558in}{3.133152in}}%
\pgfpathlineto{\pgfqpoint{6.453501in}{3.130850in}}%
\pgfpathlineto{\pgfqpoint{6.459121in}{3.122433in}}%
\pgfpathlineto{\pgfqpoint{6.463670in}{3.117484in}}%
\pgfpathlineto{\pgfqpoint{6.466346in}{3.117241in}}%
\pgfpathlineto{\pgfqpoint{6.469290in}{3.119484in}}%
\pgfpathlineto{\pgfqpoint{6.474642in}{3.127463in}}%
\pgfpathlineto{\pgfqpoint{6.479459in}{3.132897in}}%
\pgfpathlineto{\pgfqpoint{6.482403in}{3.133107in}}%
\pgfpathlineto{\pgfqpoint{6.485347in}{3.130708in}}%
\pgfpathlineto{\pgfqpoint{6.491502in}{3.121464in}}%
\pgfpathlineto{\pgfqpoint{6.495783in}{3.117309in}}%
\pgfpathlineto{\pgfqpoint{6.498459in}{3.117394in}}%
\pgfpathlineto{\pgfqpoint{6.501403in}{3.119946in}}%
\pgfpathlineto{\pgfqpoint{6.513178in}{3.133339in}}%
\pgfpathlineto{\pgfqpoint{6.515854in}{3.131983in}}%
\pgfpathlineto{\pgfqpoint{6.519600in}{3.127147in}}%
\pgfpathlineto{\pgfqpoint{6.526826in}{3.117646in}}%
\pgfpathlineto{\pgfqpoint{6.529769in}{3.117233in}}%
\pgfpathlineto{\pgfqpoint{6.532713in}{3.119455in}}%
\pgfpathlineto{\pgfqpoint{6.538065in}{3.127423in}}%
\pgfpathlineto{\pgfqpoint{6.542882in}{3.132883in}}%
\pgfpathlineto{\pgfqpoint{6.545826in}{3.133117in}}%
\pgfpathlineto{\pgfqpoint{6.548770in}{3.130738in}}%
\pgfpathlineto{\pgfqpoint{6.554657in}{3.121891in}}%
\pgfpathlineto{\pgfqpoint{6.558939in}{3.117437in}}%
\pgfpathlineto{\pgfqpoint{6.561615in}{3.117274in}}%
\pgfpathlineto{\pgfqpoint{6.564558in}{3.119594in}}%
\pgfpathlineto{\pgfqpoint{6.570178in}{3.128019in}}%
\pgfpathlineto{\pgfqpoint{6.574728in}{3.132946in}}%
\pgfpathlineto{\pgfqpoint{6.577404in}{3.133170in}}%
\pgfpathlineto{\pgfqpoint{6.580347in}{3.130910in}}%
\pgfpathlineto{\pgfqpoint{6.585699in}{3.122922in}}%
\pgfpathlineto{\pgfqpoint{6.590516in}{3.117511in}}%
\pgfpathlineto{\pgfqpoint{6.593193in}{3.117224in}}%
\pgfpathlineto{\pgfqpoint{6.596136in}{3.119425in}}%
\pgfpathlineto{\pgfqpoint{6.601221in}{3.126964in}}%
\pgfpathlineto{\pgfqpoint{6.606305in}{3.132869in}}%
\pgfpathlineto{\pgfqpoint{6.609249in}{3.133127in}}%
\pgfpathlineto{\pgfqpoint{6.612193in}{3.130769in}}%
\pgfpathlineto{\pgfqpoint{6.618080in}{3.121929in}}%
\pgfpathlineto{\pgfqpoint{6.622362in}{3.117449in}}%
\pgfpathlineto{\pgfqpoint{6.625038in}{3.117265in}}%
\pgfpathlineto{\pgfqpoint{6.627982in}{3.119564in}}%
\pgfpathlineto{\pgfqpoint{6.633601in}{3.127980in}}%
\pgfpathlineto{\pgfqpoint{6.638151in}{3.132933in}}%
\pgfpathlineto{\pgfqpoint{6.640827in}{3.133179in}}%
\pgfpathlineto{\pgfqpoint{6.643770in}{3.130940in}}%
\pgfpathlineto{\pgfqpoint{6.649123in}{3.122962in}}%
\pgfpathlineto{\pgfqpoint{6.653939in}{3.117524in}}%
\pgfpathlineto{\pgfqpoint{6.656883in}{3.117310in}}%
\pgfpathlineto{\pgfqpoint{6.659827in}{3.119706in}}%
\pgfpathlineto{\pgfqpoint{6.663306in}{3.124778in}}%
\pgfpathlineto{\pgfqpoint{6.663306in}{3.124778in}}%
\pgfusepath{stroke}%
\end{pgfscope}%
\begin{pgfscope}%
\pgfpathrectangle{\pgfqpoint{0.467797in}{2.292089in}}{\pgfqpoint{6.490533in}{1.666241in}}%
\pgfusepath{clip}%
\pgfsetrectcap%
\pgfsetroundjoin%
\pgfsetlinewidth{1.505625pt}%
\definecolor{currentstroke}{rgb}{0.498039,0.498039,0.498039}%
\pgfsetstrokecolor{currentstroke}%
\pgfsetdash{}{0pt}%
\pgfpathmoveto{\pgfqpoint{0.762821in}{3.125209in}}%
\pgfpathlineto{\pgfqpoint{0.768708in}{3.132614in}}%
\pgfpathlineto{\pgfqpoint{0.771652in}{3.132997in}}%
\pgfpathlineto{\pgfqpoint{0.774596in}{3.130698in}}%
\pgfpathlineto{\pgfqpoint{0.780215in}{3.122270in}}%
\pgfpathlineto{\pgfqpoint{0.784497in}{3.117659in}}%
\pgfpathlineto{\pgfqpoint{0.787173in}{3.117420in}}%
\pgfpathlineto{\pgfqpoint{0.790117in}{3.119712in}}%
\pgfpathlineto{\pgfqpoint{0.795737in}{3.128138in}}%
\pgfpathlineto{\pgfqpoint{0.800018in}{3.132756in}}%
\pgfpathlineto{\pgfqpoint{0.802694in}{3.133002in}}%
\pgfpathlineto{\pgfqpoint{0.805638in}{3.130715in}}%
\pgfpathlineto{\pgfqpoint{0.811258in}{3.122291in}}%
\pgfpathlineto{\pgfqpoint{0.815540in}{3.117667in}}%
\pgfpathlineto{\pgfqpoint{0.818216in}{3.117415in}}%
\pgfpathlineto{\pgfqpoint{0.821159in}{3.119696in}}%
\pgfpathlineto{\pgfqpoint{0.826779in}{3.128117in}}%
\pgfpathlineto{\pgfqpoint{0.831061in}{3.132748in}}%
\pgfpathlineto{\pgfqpoint{0.833737in}{3.133006in}}%
\pgfpathlineto{\pgfqpoint{0.836681in}{3.130731in}}%
\pgfpathlineto{\pgfqpoint{0.842300in}{3.122313in}}%
\pgfpathlineto{\pgfqpoint{0.846582in}{3.117674in}}%
\pgfpathlineto{\pgfqpoint{0.849258in}{3.117410in}}%
\pgfpathlineto{\pgfqpoint{0.852202in}{3.119679in}}%
\pgfpathlineto{\pgfqpoint{0.857822in}{3.128096in}}%
\pgfpathlineto{\pgfqpoint{0.862371in}{3.132871in}}%
\pgfpathlineto{\pgfqpoint{0.865047in}{3.132911in}}%
\pgfpathlineto{\pgfqpoint{0.867991in}{3.130429in}}%
\pgfpathlineto{\pgfqpoint{0.879766in}{3.117275in}}%
\pgfpathlineto{\pgfqpoint{0.882442in}{3.118810in}}%
\pgfpathlineto{\pgfqpoint{0.886456in}{3.124258in}}%
\pgfpathlineto{\pgfqpoint{0.892878in}{3.132580in}}%
\pgfpathlineto{\pgfqpoint{0.895822in}{3.133016in}}%
\pgfpathlineto{\pgfqpoint{0.898766in}{3.130764in}}%
\pgfpathlineto{\pgfqpoint{0.904118in}{3.122762in}}%
\pgfpathlineto{\pgfqpoint{0.908667in}{3.117689in}}%
\pgfpathlineto{\pgfqpoint{0.911343in}{3.117401in}}%
\pgfpathlineto{\pgfqpoint{0.914287in}{3.119647in}}%
\pgfpathlineto{\pgfqpoint{0.919639in}{3.127646in}}%
\pgfpathlineto{\pgfqpoint{0.924188in}{3.132726in}}%
\pgfpathlineto{\pgfqpoint{0.926865in}{3.133020in}}%
\pgfpathlineto{\pgfqpoint{0.929808in}{3.130780in}}%
\pgfpathlineto{\pgfqpoint{0.935160in}{3.122784in}}%
\pgfpathlineto{\pgfqpoint{0.939710in}{3.117697in}}%
\pgfpathlineto{\pgfqpoint{0.942386in}{3.117396in}}%
\pgfpathlineto{\pgfqpoint{0.945329in}{3.119631in}}%
\pgfpathlineto{\pgfqpoint{0.950682in}{3.127624in}}%
\pgfpathlineto{\pgfqpoint{0.955231in}{3.132718in}}%
\pgfpathlineto{\pgfqpoint{0.957907in}{3.133025in}}%
\pgfpathlineto{\pgfqpoint{0.960851in}{3.130796in}}%
\pgfpathlineto{\pgfqpoint{0.966203in}{3.122805in}}%
\pgfpathlineto{\pgfqpoint{0.970752in}{3.117704in}}%
\pgfpathlineto{\pgfqpoint{0.973428in}{3.117392in}}%
\pgfpathlineto{\pgfqpoint{0.976372in}{3.119614in}}%
\pgfpathlineto{\pgfqpoint{0.981724in}{3.127603in}}%
\pgfpathlineto{\pgfqpoint{0.986273in}{3.132711in}}%
\pgfpathlineto{\pgfqpoint{0.988950in}{3.133029in}}%
\pgfpathlineto{\pgfqpoint{0.991893in}{3.130813in}}%
\pgfpathlineto{\pgfqpoint{0.997245in}{3.122827in}}%
\pgfpathlineto{\pgfqpoint{1.001795in}{3.117712in}}%
\pgfpathlineto{\pgfqpoint{1.004471in}{3.117387in}}%
\pgfpathlineto{\pgfqpoint{1.007414in}{3.119598in}}%
\pgfpathlineto{\pgfqpoint{1.012767in}{3.127581in}}%
\pgfpathlineto{\pgfqpoint{1.017316in}{3.132703in}}%
\pgfpathlineto{\pgfqpoint{1.019992in}{3.133034in}}%
\pgfpathlineto{\pgfqpoint{1.022936in}{3.130829in}}%
\pgfpathlineto{\pgfqpoint{1.028288in}{3.122849in}}%
\pgfpathlineto{\pgfqpoint{1.032837in}{3.117720in}}%
\pgfpathlineto{\pgfqpoint{1.035513in}{3.117383in}}%
\pgfpathlineto{\pgfqpoint{1.038457in}{3.119582in}}%
\pgfpathlineto{\pgfqpoint{1.043809in}{3.127559in}}%
\pgfpathlineto{\pgfqpoint{1.048358in}{3.132695in}}%
\pgfpathlineto{\pgfqpoint{1.051035in}{3.133038in}}%
\pgfpathlineto{\pgfqpoint{1.053978in}{3.130845in}}%
\pgfpathlineto{\pgfqpoint{1.059330in}{3.122870in}}%
\pgfpathlineto{\pgfqpoint{1.063880in}{3.117728in}}%
\pgfpathlineto{\pgfqpoint{1.066556in}{3.117379in}}%
\pgfpathlineto{\pgfqpoint{1.069499in}{3.119566in}}%
\pgfpathlineto{\pgfqpoint{1.074584in}{3.127122in}}%
\pgfpathlineto{\pgfqpoint{1.079401in}{3.132687in}}%
\pgfpathlineto{\pgfqpoint{1.082077in}{3.133042in}}%
\pgfpathlineto{\pgfqpoint{1.085021in}{3.130861in}}%
\pgfpathlineto{\pgfqpoint{1.090105in}{3.123308in}}%
\pgfpathlineto{\pgfqpoint{1.094922in}{3.117735in}}%
\pgfpathlineto{\pgfqpoint{1.097598in}{3.117374in}}%
\pgfpathlineto{\pgfqpoint{1.100542in}{3.119550in}}%
\pgfpathlineto{\pgfqpoint{1.105627in}{3.127100in}}%
\pgfpathlineto{\pgfqpoint{1.110443in}{3.132679in}}%
\pgfpathlineto{\pgfqpoint{1.113120in}{3.133047in}}%
\pgfpathlineto{\pgfqpoint{1.116063in}{3.130877in}}%
\pgfpathlineto{\pgfqpoint{1.121148in}{3.123330in}}%
\pgfpathlineto{\pgfqpoint{1.125965in}{3.117743in}}%
\pgfpathlineto{\pgfqpoint{1.128641in}{3.117370in}}%
\pgfpathlineto{\pgfqpoint{1.131317in}{3.119239in}}%
\pgfpathlineto{\pgfqpoint{1.135866in}{3.125800in}}%
\pgfpathlineto{\pgfqpoint{1.141218in}{3.132509in}}%
\pgfpathlineto{\pgfqpoint{1.144162in}{3.133051in}}%
\pgfpathlineto{\pgfqpoint{1.146838in}{3.131187in}}%
\pgfpathlineto{\pgfqpoint{1.151387in}{3.124631in}}%
\pgfpathlineto{\pgfqpoint{1.156740in}{3.117915in}}%
\pgfpathlineto{\pgfqpoint{1.159683in}{3.117366in}}%
\pgfpathlineto{\pgfqpoint{1.162359in}{3.119224in}}%
\pgfpathlineto{\pgfqpoint{1.166909in}{3.125777in}}%
\pgfpathlineto{\pgfqpoint{1.172261in}{3.132500in}}%
\pgfpathlineto{\pgfqpoint{1.175205in}{3.133055in}}%
\pgfpathlineto{\pgfqpoint{1.177881in}{3.131202in}}%
\pgfpathlineto{\pgfqpoint{1.182430in}{3.124653in}}%
\pgfpathlineto{\pgfqpoint{1.187782in}{3.117924in}}%
\pgfpathlineto{\pgfqpoint{1.190726in}{3.117362in}}%
\pgfpathlineto{\pgfqpoint{1.193402in}{3.119209in}}%
\pgfpathlineto{\pgfqpoint{1.197951in}{3.125754in}}%
\pgfpathlineto{\pgfqpoint{1.203303in}{3.132490in}}%
\pgfpathlineto{\pgfqpoint{1.206247in}{3.133059in}}%
\pgfpathlineto{\pgfqpoint{1.208923in}{3.131217in}}%
\pgfpathlineto{\pgfqpoint{1.213472in}{3.124676in}}%
\pgfpathlineto{\pgfqpoint{1.218825in}{3.117933in}}%
\pgfpathlineto{\pgfqpoint{1.221768in}{3.117358in}}%
\pgfpathlineto{\pgfqpoint{1.224444in}{3.119194in}}%
\pgfpathlineto{\pgfqpoint{1.228994in}{3.125732in}}%
\pgfpathlineto{\pgfqpoint{1.234346in}{3.132481in}}%
\pgfpathlineto{\pgfqpoint{1.237290in}{3.133063in}}%
\pgfpathlineto{\pgfqpoint{1.239966in}{3.131232in}}%
\pgfpathlineto{\pgfqpoint{1.244515in}{3.124699in}}%
\pgfpathlineto{\pgfqpoint{1.249867in}{3.117943in}}%
\pgfpathlineto{\pgfqpoint{1.252811in}{3.117354in}}%
\pgfpathlineto{\pgfqpoint{1.255487in}{3.119180in}}%
\pgfpathlineto{\pgfqpoint{1.260036in}{3.125709in}}%
\pgfpathlineto{\pgfqpoint{1.265656in}{3.132639in}}%
\pgfpathlineto{\pgfqpoint{1.268600in}{3.132982in}}%
\pgfpathlineto{\pgfqpoint{1.271543in}{3.130649in}}%
\pgfpathlineto{\pgfqpoint{1.277431in}{3.121812in}}%
\pgfpathlineto{\pgfqpoint{1.281712in}{3.117514in}}%
\pgfpathlineto{\pgfqpoint{1.284389in}{3.117541in}}%
\pgfpathlineto{\pgfqpoint{1.287332in}{3.120085in}}%
\pgfpathlineto{\pgfqpoint{1.298839in}{3.133168in}}%
\pgfpathlineto{\pgfqpoint{1.301515in}{3.131787in}}%
\pgfpathlineto{\pgfqpoint{1.305530in}{3.126465in}}%
\pgfpathlineto{\pgfqpoint{1.312220in}{3.117792in}}%
\pgfpathlineto{\pgfqpoint{1.315163in}{3.117429in}}%
\pgfpathlineto{\pgfqpoint{1.318107in}{3.119745in}}%
\pgfpathlineto{\pgfqpoint{1.323727in}{3.128180in}}%
\pgfpathlineto{\pgfqpoint{1.328009in}{3.132770in}}%
\pgfpathlineto{\pgfqpoint{1.330685in}{3.132992in}}%
\pgfpathlineto{\pgfqpoint{1.333628in}{3.130682in}}%
\pgfpathlineto{\pgfqpoint{1.339248in}{3.122249in}}%
\pgfpathlineto{\pgfqpoint{1.343530in}{3.117652in}}%
\pgfpathlineto{\pgfqpoint{1.346206in}{3.117424in}}%
\pgfpathlineto{\pgfqpoint{1.349150in}{3.119729in}}%
\pgfpathlineto{\pgfqpoint{1.354769in}{3.128159in}}%
\pgfpathlineto{\pgfqpoint{1.359051in}{3.132763in}}%
\pgfpathlineto{\pgfqpoint{1.361727in}{3.132997in}}%
\pgfpathlineto{\pgfqpoint{1.364671in}{3.130698in}}%
\pgfpathlineto{\pgfqpoint{1.370291in}{3.122270in}}%
\pgfpathlineto{\pgfqpoint{1.374572in}{3.117659in}}%
\pgfpathlineto{\pgfqpoint{1.377248in}{3.117420in}}%
\pgfpathlineto{\pgfqpoint{1.380192in}{3.119712in}}%
\pgfpathlineto{\pgfqpoint{1.385812in}{3.128138in}}%
\pgfpathlineto{\pgfqpoint{1.390094in}{3.132756in}}%
\pgfpathlineto{\pgfqpoint{1.392770in}{3.133002in}}%
\pgfpathlineto{\pgfqpoint{1.395713in}{3.130715in}}%
\pgfpathlineto{\pgfqpoint{1.401333in}{3.122291in}}%
\pgfpathlineto{\pgfqpoint{1.405615in}{3.117667in}}%
\pgfpathlineto{\pgfqpoint{1.408291in}{3.117415in}}%
\pgfpathlineto{\pgfqpoint{1.411235in}{3.119696in}}%
\pgfpathlineto{\pgfqpoint{1.416854in}{3.128117in}}%
\pgfpathlineto{\pgfqpoint{1.421136in}{3.132748in}}%
\pgfpathlineto{\pgfqpoint{1.423812in}{3.133006in}}%
\pgfpathlineto{\pgfqpoint{1.426756in}{3.130731in}}%
\pgfpathlineto{\pgfqpoint{1.432376in}{3.122313in}}%
\pgfpathlineto{\pgfqpoint{1.436657in}{3.117674in}}%
\pgfpathlineto{\pgfqpoint{1.439333in}{3.117410in}}%
\pgfpathlineto{\pgfqpoint{1.442277in}{3.119679in}}%
\pgfpathlineto{\pgfqpoint{1.447897in}{3.128096in}}%
\pgfpathlineto{\pgfqpoint{1.452446in}{3.132871in}}%
\pgfpathlineto{\pgfqpoint{1.455122in}{3.132911in}}%
\pgfpathlineto{\pgfqpoint{1.458066in}{3.130429in}}%
\pgfpathlineto{\pgfqpoint{1.469841in}{3.117275in}}%
\pgfpathlineto{\pgfqpoint{1.472517in}{3.118810in}}%
\pgfpathlineto{\pgfqpoint{1.476531in}{3.124258in}}%
\pgfpathlineto{\pgfqpoint{1.482954in}{3.132580in}}%
\pgfpathlineto{\pgfqpoint{1.485897in}{3.133016in}}%
\pgfpathlineto{\pgfqpoint{1.488841in}{3.130764in}}%
\pgfpathlineto{\pgfqpoint{1.494193in}{3.122762in}}%
\pgfpathlineto{\pgfqpoint{1.498742in}{3.117689in}}%
\pgfpathlineto{\pgfqpoint{1.501418in}{3.117401in}}%
\pgfpathlineto{\pgfqpoint{1.504362in}{3.119647in}}%
\pgfpathlineto{\pgfqpoint{1.509714in}{3.127646in}}%
\pgfpathlineto{\pgfqpoint{1.514264in}{3.132726in}}%
\pgfpathlineto{\pgfqpoint{1.516940in}{3.133020in}}%
\pgfpathlineto{\pgfqpoint{1.519883in}{3.130780in}}%
\pgfpathlineto{\pgfqpoint{1.525236in}{3.122784in}}%
\pgfpathlineto{\pgfqpoint{1.529785in}{3.117697in}}%
\pgfpathlineto{\pgfqpoint{1.532461in}{3.117396in}}%
\pgfpathlineto{\pgfqpoint{1.535405in}{3.119631in}}%
\pgfpathlineto{\pgfqpoint{1.540757in}{3.127624in}}%
\pgfpathlineto{\pgfqpoint{1.545306in}{3.132718in}}%
\pgfpathlineto{\pgfqpoint{1.547982in}{3.133025in}}%
\pgfpathlineto{\pgfqpoint{1.550926in}{3.130796in}}%
\pgfpathlineto{\pgfqpoint{1.556278in}{3.122805in}}%
\pgfpathlineto{\pgfqpoint{1.560827in}{3.117704in}}%
\pgfpathlineto{\pgfqpoint{1.563504in}{3.117392in}}%
\pgfpathlineto{\pgfqpoint{1.566447in}{3.119614in}}%
\pgfpathlineto{\pgfqpoint{1.571799in}{3.127603in}}%
\pgfpathlineto{\pgfqpoint{1.576349in}{3.132711in}}%
\pgfpathlineto{\pgfqpoint{1.579025in}{3.133029in}}%
\pgfpathlineto{\pgfqpoint{1.581968in}{3.130813in}}%
\pgfpathlineto{\pgfqpoint{1.587321in}{3.122827in}}%
\pgfpathlineto{\pgfqpoint{1.591870in}{3.117712in}}%
\pgfpathlineto{\pgfqpoint{1.594546in}{3.117387in}}%
\pgfpathlineto{\pgfqpoint{1.597490in}{3.119598in}}%
\pgfpathlineto{\pgfqpoint{1.602842in}{3.127581in}}%
\pgfpathlineto{\pgfqpoint{1.607391in}{3.132703in}}%
\pgfpathlineto{\pgfqpoint{1.610067in}{3.133034in}}%
\pgfpathlineto{\pgfqpoint{1.613011in}{3.130829in}}%
\pgfpathlineto{\pgfqpoint{1.618363in}{3.122849in}}%
\pgfpathlineto{\pgfqpoint{1.622912in}{3.117720in}}%
\pgfpathlineto{\pgfqpoint{1.625589in}{3.117383in}}%
\pgfpathlineto{\pgfqpoint{1.628532in}{3.119582in}}%
\pgfpathlineto{\pgfqpoint{1.633884in}{3.127559in}}%
\pgfpathlineto{\pgfqpoint{1.638434in}{3.132695in}}%
\pgfpathlineto{\pgfqpoint{1.641110in}{3.133038in}}%
\pgfpathlineto{\pgfqpoint{1.644053in}{3.130845in}}%
\pgfpathlineto{\pgfqpoint{1.649406in}{3.122870in}}%
\pgfpathlineto{\pgfqpoint{1.653955in}{3.117728in}}%
\pgfpathlineto{\pgfqpoint{1.656631in}{3.117379in}}%
\pgfpathlineto{\pgfqpoint{1.659575in}{3.119566in}}%
\pgfpathlineto{\pgfqpoint{1.664659in}{3.127122in}}%
\pgfpathlineto{\pgfqpoint{1.669476in}{3.132687in}}%
\pgfpathlineto{\pgfqpoint{1.672152in}{3.133042in}}%
\pgfpathlineto{\pgfqpoint{1.675096in}{3.130861in}}%
\pgfpathlineto{\pgfqpoint{1.680181in}{3.123308in}}%
\pgfpathlineto{\pgfqpoint{1.684997in}{3.117735in}}%
\pgfpathlineto{\pgfqpoint{1.687674in}{3.117374in}}%
\pgfpathlineto{\pgfqpoint{1.690617in}{3.119550in}}%
\pgfpathlineto{\pgfqpoint{1.695702in}{3.127100in}}%
\pgfpathlineto{\pgfqpoint{1.700519in}{3.132679in}}%
\pgfpathlineto{\pgfqpoint{1.703195in}{3.133047in}}%
\pgfpathlineto{\pgfqpoint{1.706138in}{3.130877in}}%
\pgfpathlineto{\pgfqpoint{1.711223in}{3.123330in}}%
\pgfpathlineto{\pgfqpoint{1.716040in}{3.117743in}}%
\pgfpathlineto{\pgfqpoint{1.718716in}{3.117370in}}%
\pgfpathlineto{\pgfqpoint{1.721392in}{3.119239in}}%
\pgfpathlineto{\pgfqpoint{1.725941in}{3.125800in}}%
\pgfpathlineto{\pgfqpoint{1.731294in}{3.132509in}}%
\pgfpathlineto{\pgfqpoint{1.734237in}{3.133051in}}%
\pgfpathlineto{\pgfqpoint{1.736913in}{3.131187in}}%
\pgfpathlineto{\pgfqpoint{1.741463in}{3.124631in}}%
\pgfpathlineto{\pgfqpoint{1.746815in}{3.117915in}}%
\pgfpathlineto{\pgfqpoint{1.749759in}{3.117366in}}%
\pgfpathlineto{\pgfqpoint{1.752435in}{3.119224in}}%
\pgfpathlineto{\pgfqpoint{1.756984in}{3.125777in}}%
\pgfpathlineto{\pgfqpoint{1.762336in}{3.132500in}}%
\pgfpathlineto{\pgfqpoint{1.765280in}{3.133055in}}%
\pgfpathlineto{\pgfqpoint{1.767956in}{3.131202in}}%
\pgfpathlineto{\pgfqpoint{1.772505in}{3.124653in}}%
\pgfpathlineto{\pgfqpoint{1.777857in}{3.117924in}}%
\pgfpathlineto{\pgfqpoint{1.780801in}{3.117362in}}%
\pgfpathlineto{\pgfqpoint{1.783477in}{3.119209in}}%
\pgfpathlineto{\pgfqpoint{1.788026in}{3.125754in}}%
\pgfpathlineto{\pgfqpoint{1.793379in}{3.132490in}}%
\pgfpathlineto{\pgfqpoint{1.796322in}{3.133059in}}%
\pgfpathlineto{\pgfqpoint{1.798998in}{3.131217in}}%
\pgfpathlineto{\pgfqpoint{1.803548in}{3.124676in}}%
\pgfpathlineto{\pgfqpoint{1.808900in}{3.117933in}}%
\pgfpathlineto{\pgfqpoint{1.811844in}{3.117358in}}%
\pgfpathlineto{\pgfqpoint{1.814520in}{3.119194in}}%
\pgfpathlineto{\pgfqpoint{1.819069in}{3.125732in}}%
\pgfpathlineto{\pgfqpoint{1.824421in}{3.132481in}}%
\pgfpathlineto{\pgfqpoint{1.827365in}{3.133063in}}%
\pgfpathlineto{\pgfqpoint{1.830041in}{3.131232in}}%
\pgfpathlineto{\pgfqpoint{1.834590in}{3.124699in}}%
\pgfpathlineto{\pgfqpoint{1.839942in}{3.117943in}}%
\pgfpathlineto{\pgfqpoint{1.842886in}{3.117354in}}%
\pgfpathlineto{\pgfqpoint{1.845562in}{3.119180in}}%
\pgfpathlineto{\pgfqpoint{1.850111in}{3.125709in}}%
\pgfpathlineto{\pgfqpoint{1.855731in}{3.132639in}}%
\pgfpathlineto{\pgfqpoint{1.858675in}{3.132982in}}%
\pgfpathlineto{\pgfqpoint{1.861619in}{3.130649in}}%
\pgfpathlineto{\pgfqpoint{1.867506in}{3.121812in}}%
\pgfpathlineto{\pgfqpoint{1.871788in}{3.117514in}}%
\pgfpathlineto{\pgfqpoint{1.874464in}{3.117541in}}%
\pgfpathlineto{\pgfqpoint{1.877407in}{3.120085in}}%
\pgfpathlineto{\pgfqpoint{1.888915in}{3.133168in}}%
\pgfpathlineto{\pgfqpoint{1.891591in}{3.131787in}}%
\pgfpathlineto{\pgfqpoint{1.895605in}{3.126465in}}%
\pgfpathlineto{\pgfqpoint{1.902295in}{3.117792in}}%
\pgfpathlineto{\pgfqpoint{1.905239in}{3.117429in}}%
\pgfpathlineto{\pgfqpoint{1.908182in}{3.119745in}}%
\pgfpathlineto{\pgfqpoint{1.913802in}{3.128180in}}%
\pgfpathlineto{\pgfqpoint{1.918084in}{3.132770in}}%
\pgfpathlineto{\pgfqpoint{1.920760in}{3.132992in}}%
\pgfpathlineto{\pgfqpoint{1.923704in}{3.130682in}}%
\pgfpathlineto{\pgfqpoint{1.929323in}{3.122249in}}%
\pgfpathlineto{\pgfqpoint{1.933605in}{3.117652in}}%
\pgfpathlineto{\pgfqpoint{1.936281in}{3.117424in}}%
\pgfpathlineto{\pgfqpoint{1.939225in}{3.119729in}}%
\pgfpathlineto{\pgfqpoint{1.944845in}{3.128159in}}%
\pgfpathlineto{\pgfqpoint{1.949126in}{3.132763in}}%
\pgfpathlineto{\pgfqpoint{1.951802in}{3.132997in}}%
\pgfpathlineto{\pgfqpoint{1.954746in}{3.130698in}}%
\pgfpathlineto{\pgfqpoint{1.960366in}{3.122270in}}%
\pgfpathlineto{\pgfqpoint{1.964648in}{3.117659in}}%
\pgfpathlineto{\pgfqpoint{1.967324in}{3.117420in}}%
\pgfpathlineto{\pgfqpoint{1.970267in}{3.119712in}}%
\pgfpathlineto{\pgfqpoint{1.975887in}{3.128138in}}%
\pgfpathlineto{\pgfqpoint{1.980169in}{3.132756in}}%
\pgfpathlineto{\pgfqpoint{1.982845in}{3.133002in}}%
\pgfpathlineto{\pgfqpoint{1.985789in}{3.130715in}}%
\pgfpathlineto{\pgfqpoint{1.991408in}{3.122291in}}%
\pgfpathlineto{\pgfqpoint{1.995690in}{3.117667in}}%
\pgfpathlineto{\pgfqpoint{1.998366in}{3.117415in}}%
\pgfpathlineto{\pgfqpoint{2.001310in}{3.119696in}}%
\pgfpathlineto{\pgfqpoint{2.006930in}{3.128117in}}%
\pgfpathlineto{\pgfqpoint{2.011211in}{3.132748in}}%
\pgfpathlineto{\pgfqpoint{2.013887in}{3.133006in}}%
\pgfpathlineto{\pgfqpoint{2.016831in}{3.130731in}}%
\pgfpathlineto{\pgfqpoint{2.022451in}{3.122313in}}%
\pgfpathlineto{\pgfqpoint{2.026733in}{3.117674in}}%
\pgfpathlineto{\pgfqpoint{2.029409in}{3.117410in}}%
\pgfpathlineto{\pgfqpoint{2.032352in}{3.119679in}}%
\pgfpathlineto{\pgfqpoint{2.037972in}{3.128096in}}%
\pgfpathlineto{\pgfqpoint{2.042522in}{3.132871in}}%
\pgfpathlineto{\pgfqpoint{2.045198in}{3.132911in}}%
\pgfpathlineto{\pgfqpoint{2.048141in}{3.130429in}}%
\pgfpathlineto{\pgfqpoint{2.059916in}{3.117275in}}%
\pgfpathlineto{\pgfqpoint{2.062592in}{3.118810in}}%
\pgfpathlineto{\pgfqpoint{2.066606in}{3.124258in}}%
\pgfpathlineto{\pgfqpoint{2.073029in}{3.132580in}}%
\pgfpathlineto{\pgfqpoint{2.075972in}{3.133016in}}%
\pgfpathlineto{\pgfqpoint{2.078916in}{3.130764in}}%
\pgfpathlineto{\pgfqpoint{2.084268in}{3.122762in}}%
\pgfpathlineto{\pgfqpoint{2.088818in}{3.117689in}}%
\pgfpathlineto{\pgfqpoint{2.091494in}{3.117401in}}%
\pgfpathlineto{\pgfqpoint{2.094437in}{3.119647in}}%
\pgfpathlineto{\pgfqpoint{2.099790in}{3.127646in}}%
\pgfpathlineto{\pgfqpoint{2.104339in}{3.132726in}}%
\pgfpathlineto{\pgfqpoint{2.107015in}{3.133020in}}%
\pgfpathlineto{\pgfqpoint{2.109959in}{3.130780in}}%
\pgfpathlineto{\pgfqpoint{2.115311in}{3.122784in}}%
\pgfpathlineto{\pgfqpoint{2.119860in}{3.117697in}}%
\pgfpathlineto{\pgfqpoint{2.122536in}{3.117396in}}%
\pgfpathlineto{\pgfqpoint{2.125480in}{3.119631in}}%
\pgfpathlineto{\pgfqpoint{2.130832in}{3.127624in}}%
\pgfpathlineto{\pgfqpoint{2.135381in}{3.132718in}}%
\pgfpathlineto{\pgfqpoint{2.138058in}{3.133025in}}%
\pgfpathlineto{\pgfqpoint{2.141001in}{3.130796in}}%
\pgfpathlineto{\pgfqpoint{2.146353in}{3.122805in}}%
\pgfpathlineto{\pgfqpoint{2.150903in}{3.117704in}}%
\pgfpathlineto{\pgfqpoint{2.153579in}{3.117392in}}%
\pgfpathlineto{\pgfqpoint{2.156522in}{3.119614in}}%
\pgfpathlineto{\pgfqpoint{2.161875in}{3.127603in}}%
\pgfpathlineto{\pgfqpoint{2.166424in}{3.132711in}}%
\pgfpathlineto{\pgfqpoint{2.169100in}{3.133029in}}%
\pgfpathlineto{\pgfqpoint{2.172044in}{3.130813in}}%
\pgfpathlineto{\pgfqpoint{2.177396in}{3.122827in}}%
\pgfpathlineto{\pgfqpoint{2.181945in}{3.117712in}}%
\pgfpathlineto{\pgfqpoint{2.184621in}{3.117387in}}%
\pgfpathlineto{\pgfqpoint{2.187565in}{3.119598in}}%
\pgfpathlineto{\pgfqpoint{2.192917in}{3.127581in}}%
\pgfpathlineto{\pgfqpoint{2.197466in}{3.132703in}}%
\pgfpathlineto{\pgfqpoint{2.200143in}{3.133034in}}%
\pgfpathlineto{\pgfqpoint{2.203086in}{3.130829in}}%
\pgfpathlineto{\pgfqpoint{2.208438in}{3.122849in}}%
\pgfpathlineto{\pgfqpoint{2.212988in}{3.117720in}}%
\pgfpathlineto{\pgfqpoint{2.215664in}{3.117383in}}%
\pgfpathlineto{\pgfqpoint{2.218607in}{3.119582in}}%
\pgfpathlineto{\pgfqpoint{2.223960in}{3.127559in}}%
\pgfpathlineto{\pgfqpoint{2.228509in}{3.132695in}}%
\pgfpathlineto{\pgfqpoint{2.231185in}{3.133038in}}%
\pgfpathlineto{\pgfqpoint{2.234129in}{3.130845in}}%
\pgfpathlineto{\pgfqpoint{2.239481in}{3.122870in}}%
\pgfpathlineto{\pgfqpoint{2.244030in}{3.117728in}}%
\pgfpathlineto{\pgfqpoint{2.246706in}{3.117379in}}%
\pgfpathlineto{\pgfqpoint{2.249650in}{3.119566in}}%
\pgfpathlineto{\pgfqpoint{2.254735in}{3.127122in}}%
\pgfpathlineto{\pgfqpoint{2.259551in}{3.132687in}}%
\pgfpathlineto{\pgfqpoint{2.262228in}{3.133042in}}%
\pgfpathlineto{\pgfqpoint{2.265171in}{3.130861in}}%
\pgfpathlineto{\pgfqpoint{2.270256in}{3.123308in}}%
\pgfpathlineto{\pgfqpoint{2.275073in}{3.117735in}}%
\pgfpathlineto{\pgfqpoint{2.277749in}{3.117374in}}%
\pgfpathlineto{\pgfqpoint{2.280692in}{3.119550in}}%
\pgfpathlineto{\pgfqpoint{2.285777in}{3.127100in}}%
\pgfpathlineto{\pgfqpoint{2.290594in}{3.132679in}}%
\pgfpathlineto{\pgfqpoint{2.293270in}{3.133047in}}%
\pgfpathlineto{\pgfqpoint{2.296214in}{3.130877in}}%
\pgfpathlineto{\pgfqpoint{2.301298in}{3.123330in}}%
\pgfpathlineto{\pgfqpoint{2.306115in}{3.117743in}}%
\pgfpathlineto{\pgfqpoint{2.308791in}{3.117370in}}%
\pgfpathlineto{\pgfqpoint{2.311467in}{3.119239in}}%
\pgfpathlineto{\pgfqpoint{2.316017in}{3.125800in}}%
\pgfpathlineto{\pgfqpoint{2.321369in}{3.132509in}}%
\pgfpathlineto{\pgfqpoint{2.324313in}{3.133051in}}%
\pgfpathlineto{\pgfqpoint{2.326989in}{3.131187in}}%
\pgfpathlineto{\pgfqpoint{2.331538in}{3.124631in}}%
\pgfpathlineto{\pgfqpoint{2.336890in}{3.117915in}}%
\pgfpathlineto{\pgfqpoint{2.339834in}{3.117366in}}%
\pgfpathlineto{\pgfqpoint{2.342510in}{3.119224in}}%
\pgfpathlineto{\pgfqpoint{2.347059in}{3.125777in}}%
\pgfpathlineto{\pgfqpoint{2.352411in}{3.132500in}}%
\pgfpathlineto{\pgfqpoint{2.355355in}{3.133055in}}%
\pgfpathlineto{\pgfqpoint{2.358031in}{3.131202in}}%
\pgfpathlineto{\pgfqpoint{2.362580in}{3.124653in}}%
\pgfpathlineto{\pgfqpoint{2.367933in}{3.117924in}}%
\pgfpathlineto{\pgfqpoint{2.370876in}{3.117362in}}%
\pgfpathlineto{\pgfqpoint{2.373552in}{3.119209in}}%
\pgfpathlineto{\pgfqpoint{2.378102in}{3.125754in}}%
\pgfpathlineto{\pgfqpoint{2.383454in}{3.132490in}}%
\pgfpathlineto{\pgfqpoint{2.386398in}{3.133059in}}%
\pgfpathlineto{\pgfqpoint{2.389074in}{3.131217in}}%
\pgfpathlineto{\pgfqpoint{2.393623in}{3.124676in}}%
\pgfpathlineto{\pgfqpoint{2.398975in}{3.117933in}}%
\pgfpathlineto{\pgfqpoint{2.401919in}{3.117358in}}%
\pgfpathlineto{\pgfqpoint{2.404595in}{3.119194in}}%
\pgfpathlineto{\pgfqpoint{2.409144in}{3.125732in}}%
\pgfpathlineto{\pgfqpoint{2.414496in}{3.132481in}}%
\pgfpathlineto{\pgfqpoint{2.417440in}{3.133063in}}%
\pgfpathlineto{\pgfqpoint{2.420116in}{3.131232in}}%
\pgfpathlineto{\pgfqpoint{2.424665in}{3.124699in}}%
\pgfpathlineto{\pgfqpoint{2.430018in}{3.117943in}}%
\pgfpathlineto{\pgfqpoint{2.432961in}{3.117354in}}%
\pgfpathlineto{\pgfqpoint{2.435637in}{3.119180in}}%
\pgfpathlineto{\pgfqpoint{2.440187in}{3.125709in}}%
\pgfpathlineto{\pgfqpoint{2.445806in}{3.132639in}}%
\pgfpathlineto{\pgfqpoint{2.448750in}{3.132982in}}%
\pgfpathlineto{\pgfqpoint{2.451694in}{3.130649in}}%
\pgfpathlineto{\pgfqpoint{2.457581in}{3.121812in}}%
\pgfpathlineto{\pgfqpoint{2.461863in}{3.117514in}}%
\pgfpathlineto{\pgfqpoint{2.464539in}{3.117541in}}%
\pgfpathlineto{\pgfqpoint{2.467483in}{3.120085in}}%
\pgfpathlineto{\pgfqpoint{2.478990in}{3.133168in}}%
\pgfpathlineto{\pgfqpoint{2.481666in}{3.131787in}}%
\pgfpathlineto{\pgfqpoint{2.485680in}{3.126465in}}%
\pgfpathlineto{\pgfqpoint{2.492370in}{3.117792in}}%
\pgfpathlineto{\pgfqpoint{2.495314in}{3.117429in}}%
\pgfpathlineto{\pgfqpoint{2.498258in}{3.119745in}}%
\pgfpathlineto{\pgfqpoint{2.503877in}{3.128180in}}%
\pgfpathlineto{\pgfqpoint{2.508159in}{3.132770in}}%
\pgfpathlineto{\pgfqpoint{2.510835in}{3.132992in}}%
\pgfpathlineto{\pgfqpoint{2.513779in}{3.130682in}}%
\pgfpathlineto{\pgfqpoint{2.519399in}{3.122249in}}%
\pgfpathlineto{\pgfqpoint{2.523680in}{3.117652in}}%
\pgfpathlineto{\pgfqpoint{2.526356in}{3.117424in}}%
\pgfpathlineto{\pgfqpoint{2.529300in}{3.119729in}}%
\pgfpathlineto{\pgfqpoint{2.534920in}{3.128159in}}%
\pgfpathlineto{\pgfqpoint{2.539202in}{3.132763in}}%
\pgfpathlineto{\pgfqpoint{2.541878in}{3.132997in}}%
\pgfpathlineto{\pgfqpoint{2.544821in}{3.130698in}}%
\pgfpathlineto{\pgfqpoint{2.550441in}{3.122270in}}%
\pgfpathlineto{\pgfqpoint{2.554723in}{3.117659in}}%
\pgfpathlineto{\pgfqpoint{2.557399in}{3.117420in}}%
\pgfpathlineto{\pgfqpoint{2.560343in}{3.119712in}}%
\pgfpathlineto{\pgfqpoint{2.565962in}{3.128138in}}%
\pgfpathlineto{\pgfqpoint{2.570244in}{3.132756in}}%
\pgfpathlineto{\pgfqpoint{2.572920in}{3.133002in}}%
\pgfpathlineto{\pgfqpoint{2.575864in}{3.130715in}}%
\pgfpathlineto{\pgfqpoint{2.581484in}{3.122291in}}%
\pgfpathlineto{\pgfqpoint{2.585765in}{3.117667in}}%
\pgfpathlineto{\pgfqpoint{2.588441in}{3.117415in}}%
\pgfpathlineto{\pgfqpoint{2.591385in}{3.119696in}}%
\pgfpathlineto{\pgfqpoint{2.597005in}{3.128117in}}%
\pgfpathlineto{\pgfqpoint{2.601287in}{3.132748in}}%
\pgfpathlineto{\pgfqpoint{2.603963in}{3.133006in}}%
\pgfpathlineto{\pgfqpoint{2.606906in}{3.130731in}}%
\pgfpathlineto{\pgfqpoint{2.612526in}{3.122313in}}%
\pgfpathlineto{\pgfqpoint{2.616808in}{3.117674in}}%
\pgfpathlineto{\pgfqpoint{2.619484in}{3.117410in}}%
\pgfpathlineto{\pgfqpoint{2.622428in}{3.119679in}}%
\pgfpathlineto{\pgfqpoint{2.628047in}{3.128096in}}%
\pgfpathlineto{\pgfqpoint{2.632597in}{3.132871in}}%
\pgfpathlineto{\pgfqpoint{2.635273in}{3.132911in}}%
\pgfpathlineto{\pgfqpoint{2.638217in}{3.130429in}}%
\pgfpathlineto{\pgfqpoint{2.649991in}{3.117275in}}%
\pgfpathlineto{\pgfqpoint{2.652667in}{3.118810in}}%
\pgfpathlineto{\pgfqpoint{2.656681in}{3.124258in}}%
\pgfpathlineto{\pgfqpoint{2.663104in}{3.132580in}}%
\pgfpathlineto{\pgfqpoint{2.666048in}{3.133016in}}%
\pgfpathlineto{\pgfqpoint{2.668991in}{3.130764in}}%
\pgfpathlineto{\pgfqpoint{2.674344in}{3.122762in}}%
\pgfpathlineto{\pgfqpoint{2.678893in}{3.117689in}}%
\pgfpathlineto{\pgfqpoint{2.681569in}{3.117401in}}%
\pgfpathlineto{\pgfqpoint{2.684513in}{3.119647in}}%
\pgfpathlineto{\pgfqpoint{2.689865in}{3.127646in}}%
\pgfpathlineto{\pgfqpoint{2.694414in}{3.132726in}}%
\pgfpathlineto{\pgfqpoint{2.697090in}{3.133020in}}%
\pgfpathlineto{\pgfqpoint{2.700034in}{3.130780in}}%
\pgfpathlineto{\pgfqpoint{2.705386in}{3.122784in}}%
\pgfpathlineto{\pgfqpoint{2.709935in}{3.117697in}}%
\pgfpathlineto{\pgfqpoint{2.712611in}{3.117396in}}%
\pgfpathlineto{\pgfqpoint{2.715555in}{3.119631in}}%
\pgfpathlineto{\pgfqpoint{2.720907in}{3.127624in}}%
\pgfpathlineto{\pgfqpoint{2.725457in}{3.132718in}}%
\pgfpathlineto{\pgfqpoint{2.728133in}{3.133025in}}%
\pgfpathlineto{\pgfqpoint{2.731076in}{3.130796in}}%
\pgfpathlineto{\pgfqpoint{2.736429in}{3.122805in}}%
\pgfpathlineto{\pgfqpoint{2.740978in}{3.117704in}}%
\pgfpathlineto{\pgfqpoint{2.743654in}{3.117392in}}%
\pgfpathlineto{\pgfqpoint{2.746598in}{3.119614in}}%
\pgfpathlineto{\pgfqpoint{2.751950in}{3.127603in}}%
\pgfpathlineto{\pgfqpoint{2.756499in}{3.132711in}}%
\pgfpathlineto{\pgfqpoint{2.759175in}{3.133029in}}%
\pgfpathlineto{\pgfqpoint{2.762119in}{3.130813in}}%
\pgfpathlineto{\pgfqpoint{2.767471in}{3.122827in}}%
\pgfpathlineto{\pgfqpoint{2.772020in}{3.117712in}}%
\pgfpathlineto{\pgfqpoint{2.774697in}{3.117387in}}%
\pgfpathlineto{\pgfqpoint{2.777640in}{3.119598in}}%
\pgfpathlineto{\pgfqpoint{2.782992in}{3.127581in}}%
\pgfpathlineto{\pgfqpoint{2.787542in}{3.132703in}}%
\pgfpathlineto{\pgfqpoint{2.790218in}{3.133034in}}%
\pgfpathlineto{\pgfqpoint{2.793161in}{3.130829in}}%
\pgfpathlineto{\pgfqpoint{2.798514in}{3.122849in}}%
\pgfpathlineto{\pgfqpoint{2.803063in}{3.117720in}}%
\pgfpathlineto{\pgfqpoint{2.805739in}{3.117383in}}%
\pgfpathlineto{\pgfqpoint{2.808683in}{3.119582in}}%
\pgfpathlineto{\pgfqpoint{2.814035in}{3.127559in}}%
\pgfpathlineto{\pgfqpoint{2.818584in}{3.132695in}}%
\pgfpathlineto{\pgfqpoint{2.821260in}{3.133038in}}%
\pgfpathlineto{\pgfqpoint{2.824204in}{3.130845in}}%
\pgfpathlineto{\pgfqpoint{2.829556in}{3.122870in}}%
\pgfpathlineto{\pgfqpoint{2.834105in}{3.117728in}}%
\pgfpathlineto{\pgfqpoint{2.836782in}{3.117379in}}%
\pgfpathlineto{\pgfqpoint{2.839725in}{3.119566in}}%
\pgfpathlineto{\pgfqpoint{2.844810in}{3.127122in}}%
\pgfpathlineto{\pgfqpoint{2.849627in}{3.132687in}}%
\pgfpathlineto{\pgfqpoint{2.852303in}{3.133042in}}%
\pgfpathlineto{\pgfqpoint{2.855246in}{3.130861in}}%
\pgfpathlineto{\pgfqpoint{2.860331in}{3.123308in}}%
\pgfpathlineto{\pgfqpoint{2.865148in}{3.117735in}}%
\pgfpathlineto{\pgfqpoint{2.867824in}{3.117374in}}%
\pgfpathlineto{\pgfqpoint{2.870768in}{3.119550in}}%
\pgfpathlineto{\pgfqpoint{2.875852in}{3.127100in}}%
\pgfpathlineto{\pgfqpoint{2.880669in}{3.132679in}}%
\pgfpathlineto{\pgfqpoint{2.883345in}{3.133047in}}%
\pgfpathlineto{\pgfqpoint{2.886289in}{3.130877in}}%
\pgfpathlineto{\pgfqpoint{2.891374in}{3.123330in}}%
\pgfpathlineto{\pgfqpoint{2.896190in}{3.117743in}}%
\pgfpathlineto{\pgfqpoint{2.898867in}{3.117370in}}%
\pgfpathlineto{\pgfqpoint{2.901543in}{3.119239in}}%
\pgfpathlineto{\pgfqpoint{2.906092in}{3.125800in}}%
\pgfpathlineto{\pgfqpoint{2.911444in}{3.132509in}}%
\pgfpathlineto{\pgfqpoint{2.914388in}{3.133051in}}%
\pgfpathlineto{\pgfqpoint{2.917064in}{3.131187in}}%
\pgfpathlineto{\pgfqpoint{2.921613in}{3.124631in}}%
\pgfpathlineto{\pgfqpoint{2.926965in}{3.117915in}}%
\pgfpathlineto{\pgfqpoint{2.929909in}{3.117366in}}%
\pgfpathlineto{\pgfqpoint{2.932585in}{3.119224in}}%
\pgfpathlineto{\pgfqpoint{2.937134in}{3.125777in}}%
\pgfpathlineto{\pgfqpoint{2.942487in}{3.132500in}}%
\pgfpathlineto{\pgfqpoint{2.945430in}{3.133055in}}%
\pgfpathlineto{\pgfqpoint{2.948106in}{3.131202in}}%
\pgfpathlineto{\pgfqpoint{2.952656in}{3.124653in}}%
\pgfpathlineto{\pgfqpoint{2.958008in}{3.117924in}}%
\pgfpathlineto{\pgfqpoint{2.960952in}{3.117362in}}%
\pgfpathlineto{\pgfqpoint{2.963628in}{3.119209in}}%
\pgfpathlineto{\pgfqpoint{2.968177in}{3.125754in}}%
\pgfpathlineto{\pgfqpoint{2.973529in}{3.132490in}}%
\pgfpathlineto{\pgfqpoint{2.976473in}{3.133059in}}%
\pgfpathlineto{\pgfqpoint{2.979149in}{3.131217in}}%
\pgfpathlineto{\pgfqpoint{2.983698in}{3.124676in}}%
\pgfpathlineto{\pgfqpoint{2.989050in}{3.117933in}}%
\pgfpathlineto{\pgfqpoint{2.991994in}{3.117358in}}%
\pgfpathlineto{\pgfqpoint{2.994670in}{3.119194in}}%
\pgfpathlineto{\pgfqpoint{2.999219in}{3.125732in}}%
\pgfpathlineto{\pgfqpoint{3.004572in}{3.132481in}}%
\pgfpathlineto{\pgfqpoint{3.007515in}{3.133063in}}%
\pgfpathlineto{\pgfqpoint{3.010191in}{3.131232in}}%
\pgfpathlineto{\pgfqpoint{3.014741in}{3.124699in}}%
\pgfpathlineto{\pgfqpoint{3.020093in}{3.117943in}}%
\pgfpathlineto{\pgfqpoint{3.023037in}{3.117354in}}%
\pgfpathlineto{\pgfqpoint{3.025713in}{3.119180in}}%
\pgfpathlineto{\pgfqpoint{3.030262in}{3.125709in}}%
\pgfpathlineto{\pgfqpoint{3.035882in}{3.132639in}}%
\pgfpathlineto{\pgfqpoint{3.038825in}{3.132982in}}%
\pgfpathlineto{\pgfqpoint{3.041769in}{3.130649in}}%
\pgfpathlineto{\pgfqpoint{3.047656in}{3.121812in}}%
\pgfpathlineto{\pgfqpoint{3.051938in}{3.117514in}}%
\pgfpathlineto{\pgfqpoint{3.054614in}{3.117541in}}%
\pgfpathlineto{\pgfqpoint{3.057558in}{3.120085in}}%
\pgfpathlineto{\pgfqpoint{3.069065in}{3.133168in}}%
\pgfpathlineto{\pgfqpoint{3.071741in}{3.131787in}}%
\pgfpathlineto{\pgfqpoint{3.075755in}{3.126465in}}%
\pgfpathlineto{\pgfqpoint{3.082446in}{3.117792in}}%
\pgfpathlineto{\pgfqpoint{3.085389in}{3.117429in}}%
\pgfpathlineto{\pgfqpoint{3.088333in}{3.119745in}}%
\pgfpathlineto{\pgfqpoint{3.093953in}{3.128180in}}%
\pgfpathlineto{\pgfqpoint{3.098234in}{3.132770in}}%
\pgfpathlineto{\pgfqpoint{3.100910in}{3.132992in}}%
\pgfpathlineto{\pgfqpoint{3.103854in}{3.130682in}}%
\pgfpathlineto{\pgfqpoint{3.109474in}{3.122249in}}%
\pgfpathlineto{\pgfqpoint{3.113756in}{3.117652in}}%
\pgfpathlineto{\pgfqpoint{3.116432in}{3.117424in}}%
\pgfpathlineto{\pgfqpoint{3.119375in}{3.119729in}}%
\pgfpathlineto{\pgfqpoint{3.124995in}{3.128159in}}%
\pgfpathlineto{\pgfqpoint{3.129277in}{3.132763in}}%
\pgfpathlineto{\pgfqpoint{3.131953in}{3.132997in}}%
\pgfpathlineto{\pgfqpoint{3.134897in}{3.130698in}}%
\pgfpathlineto{\pgfqpoint{3.140516in}{3.122270in}}%
\pgfpathlineto{\pgfqpoint{3.144798in}{3.117659in}}%
\pgfpathlineto{\pgfqpoint{3.147474in}{3.117420in}}%
\pgfpathlineto{\pgfqpoint{3.150418in}{3.119712in}}%
\pgfpathlineto{\pgfqpoint{3.156038in}{3.128138in}}%
\pgfpathlineto{\pgfqpoint{3.160319in}{3.132756in}}%
\pgfpathlineto{\pgfqpoint{3.162995in}{3.133002in}}%
\pgfpathlineto{\pgfqpoint{3.165939in}{3.130715in}}%
\pgfpathlineto{\pgfqpoint{3.171559in}{3.122291in}}%
\pgfpathlineto{\pgfqpoint{3.175841in}{3.117667in}}%
\pgfpathlineto{\pgfqpoint{3.178517in}{3.117415in}}%
\pgfpathlineto{\pgfqpoint{3.181460in}{3.119696in}}%
\pgfpathlineto{\pgfqpoint{3.187080in}{3.128117in}}%
\pgfpathlineto{\pgfqpoint{3.191362in}{3.132748in}}%
\pgfpathlineto{\pgfqpoint{3.194038in}{3.133006in}}%
\pgfpathlineto{\pgfqpoint{3.196982in}{3.130731in}}%
\pgfpathlineto{\pgfqpoint{3.202601in}{3.122313in}}%
\pgfpathlineto{\pgfqpoint{3.206883in}{3.117674in}}%
\pgfpathlineto{\pgfqpoint{3.209559in}{3.117410in}}%
\pgfpathlineto{\pgfqpoint{3.212503in}{3.119679in}}%
\pgfpathlineto{\pgfqpoint{3.218123in}{3.128096in}}%
\pgfpathlineto{\pgfqpoint{3.222672in}{3.132871in}}%
\pgfpathlineto{\pgfqpoint{3.225348in}{3.132911in}}%
\pgfpathlineto{\pgfqpoint{3.228292in}{3.130429in}}%
\pgfpathlineto{\pgfqpoint{3.240067in}{3.117275in}}%
\pgfpathlineto{\pgfqpoint{3.242743in}{3.118810in}}%
\pgfpathlineto{\pgfqpoint{3.246757in}{3.124258in}}%
\pgfpathlineto{\pgfqpoint{3.253179in}{3.132580in}}%
\pgfpathlineto{\pgfqpoint{3.256123in}{3.133016in}}%
\pgfpathlineto{\pgfqpoint{3.259067in}{3.130764in}}%
\pgfpathlineto{\pgfqpoint{3.264419in}{3.122762in}}%
\pgfpathlineto{\pgfqpoint{3.268968in}{3.117689in}}%
\pgfpathlineto{\pgfqpoint{3.271644in}{3.117401in}}%
\pgfpathlineto{\pgfqpoint{3.274588in}{3.119647in}}%
\pgfpathlineto{\pgfqpoint{3.279940in}{3.127646in}}%
\pgfpathlineto{\pgfqpoint{3.284489in}{3.132726in}}%
\pgfpathlineto{\pgfqpoint{3.287165in}{3.133020in}}%
\pgfpathlineto{\pgfqpoint{3.290109in}{3.130780in}}%
\pgfpathlineto{\pgfqpoint{3.295461in}{3.122784in}}%
\pgfpathlineto{\pgfqpoint{3.300011in}{3.117697in}}%
\pgfpathlineto{\pgfqpoint{3.302687in}{3.117396in}}%
\pgfpathlineto{\pgfqpoint{3.305630in}{3.119631in}}%
\pgfpathlineto{\pgfqpoint{3.310983in}{3.127624in}}%
\pgfpathlineto{\pgfqpoint{3.315532in}{3.132718in}}%
\pgfpathlineto{\pgfqpoint{3.318208in}{3.133025in}}%
\pgfpathlineto{\pgfqpoint{3.321152in}{3.130796in}}%
\pgfpathlineto{\pgfqpoint{3.326504in}{3.122805in}}%
\pgfpathlineto{\pgfqpoint{3.331053in}{3.117704in}}%
\pgfpathlineto{\pgfqpoint{3.333729in}{3.117392in}}%
\pgfpathlineto{\pgfqpoint{3.336673in}{3.119614in}}%
\pgfpathlineto{\pgfqpoint{3.342025in}{3.127603in}}%
\pgfpathlineto{\pgfqpoint{3.346574in}{3.132711in}}%
\pgfpathlineto{\pgfqpoint{3.349251in}{3.133029in}}%
\pgfpathlineto{\pgfqpoint{3.352194in}{3.130813in}}%
\pgfpathlineto{\pgfqpoint{3.357546in}{3.122827in}}%
\pgfpathlineto{\pgfqpoint{3.362096in}{3.117712in}}%
\pgfpathlineto{\pgfqpoint{3.364772in}{3.117387in}}%
\pgfpathlineto{\pgfqpoint{3.367715in}{3.119598in}}%
\pgfpathlineto{\pgfqpoint{3.373068in}{3.127581in}}%
\pgfpathlineto{\pgfqpoint{3.377617in}{3.132703in}}%
\pgfpathlineto{\pgfqpoint{3.380293in}{3.133034in}}%
\pgfpathlineto{\pgfqpoint{3.383237in}{3.130829in}}%
\pgfpathlineto{\pgfqpoint{3.388589in}{3.122849in}}%
\pgfpathlineto{\pgfqpoint{3.393138in}{3.117720in}}%
\pgfpathlineto{\pgfqpoint{3.395814in}{3.117383in}}%
\pgfpathlineto{\pgfqpoint{3.398758in}{3.119582in}}%
\pgfpathlineto{\pgfqpoint{3.404110in}{3.127559in}}%
\pgfpathlineto{\pgfqpoint{3.408659in}{3.132695in}}%
\pgfpathlineto{\pgfqpoint{3.411336in}{3.133038in}}%
\pgfpathlineto{\pgfqpoint{3.414279in}{3.130845in}}%
\pgfpathlineto{\pgfqpoint{3.419631in}{3.122870in}}%
\pgfpathlineto{\pgfqpoint{3.424181in}{3.117728in}}%
\pgfpathlineto{\pgfqpoint{3.426857in}{3.117379in}}%
\pgfpathlineto{\pgfqpoint{3.429800in}{3.119566in}}%
\pgfpathlineto{\pgfqpoint{3.434885in}{3.127122in}}%
\pgfpathlineto{\pgfqpoint{3.439702in}{3.132687in}}%
\pgfpathlineto{\pgfqpoint{3.442378in}{3.133042in}}%
\pgfpathlineto{\pgfqpoint{3.445322in}{3.130861in}}%
\pgfpathlineto{\pgfqpoint{3.450406in}{3.123308in}}%
\pgfpathlineto{\pgfqpoint{3.455223in}{3.117735in}}%
\pgfpathlineto{\pgfqpoint{3.457899in}{3.117374in}}%
\pgfpathlineto{\pgfqpoint{3.460843in}{3.119550in}}%
\pgfpathlineto{\pgfqpoint{3.465928in}{3.127100in}}%
\pgfpathlineto{\pgfqpoint{3.470744in}{3.132679in}}%
\pgfpathlineto{\pgfqpoint{3.473421in}{3.133047in}}%
\pgfpathlineto{\pgfqpoint{3.476364in}{3.130877in}}%
\pgfpathlineto{\pgfqpoint{3.481449in}{3.123330in}}%
\pgfpathlineto{\pgfqpoint{3.486266in}{3.117743in}}%
\pgfpathlineto{\pgfqpoint{3.488942in}{3.117370in}}%
\pgfpathlineto{\pgfqpoint{3.491618in}{3.119239in}}%
\pgfpathlineto{\pgfqpoint{3.496167in}{3.125800in}}%
\pgfpathlineto{\pgfqpoint{3.501519in}{3.132509in}}%
\pgfpathlineto{\pgfqpoint{3.504463in}{3.133051in}}%
\pgfpathlineto{\pgfqpoint{3.507139in}{3.131187in}}%
\pgfpathlineto{\pgfqpoint{3.511688in}{3.124631in}}%
\pgfpathlineto{\pgfqpoint{3.517041in}{3.117915in}}%
\pgfpathlineto{\pgfqpoint{3.519984in}{3.117366in}}%
\pgfpathlineto{\pgfqpoint{3.522660in}{3.119224in}}%
\pgfpathlineto{\pgfqpoint{3.527210in}{3.125777in}}%
\pgfpathlineto{\pgfqpoint{3.532562in}{3.132500in}}%
\pgfpathlineto{\pgfqpoint{3.535506in}{3.133055in}}%
\pgfpathlineto{\pgfqpoint{3.538182in}{3.131202in}}%
\pgfpathlineto{\pgfqpoint{3.542731in}{3.124653in}}%
\pgfpathlineto{\pgfqpoint{3.548083in}{3.117924in}}%
\pgfpathlineto{\pgfqpoint{3.551027in}{3.117362in}}%
\pgfpathlineto{\pgfqpoint{3.553703in}{3.119209in}}%
\pgfpathlineto{\pgfqpoint{3.558252in}{3.125754in}}%
\pgfpathlineto{\pgfqpoint{3.563604in}{3.132490in}}%
\pgfpathlineto{\pgfqpoint{3.566548in}{3.133059in}}%
\pgfpathlineto{\pgfqpoint{3.569224in}{3.131217in}}%
\pgfpathlineto{\pgfqpoint{3.573773in}{3.124676in}}%
\pgfpathlineto{\pgfqpoint{3.579126in}{3.117933in}}%
\pgfpathlineto{\pgfqpoint{3.582069in}{3.117358in}}%
\pgfpathlineto{\pgfqpoint{3.584745in}{3.119194in}}%
\pgfpathlineto{\pgfqpoint{3.589295in}{3.125732in}}%
\pgfpathlineto{\pgfqpoint{3.594647in}{3.132481in}}%
\pgfpathlineto{\pgfqpoint{3.597591in}{3.133063in}}%
\pgfpathlineto{\pgfqpoint{3.600267in}{3.131232in}}%
\pgfpathlineto{\pgfqpoint{3.604816in}{3.124699in}}%
\pgfpathlineto{\pgfqpoint{3.610168in}{3.117943in}}%
\pgfpathlineto{\pgfqpoint{3.613112in}{3.117354in}}%
\pgfpathlineto{\pgfqpoint{3.615788in}{3.119180in}}%
\pgfpathlineto{\pgfqpoint{3.620337in}{3.125709in}}%
\pgfpathlineto{\pgfqpoint{3.625957in}{3.132639in}}%
\pgfpathlineto{\pgfqpoint{3.628901in}{3.132982in}}%
\pgfpathlineto{\pgfqpoint{3.631844in}{3.130649in}}%
\pgfpathlineto{\pgfqpoint{3.637732in}{3.121812in}}%
\pgfpathlineto{\pgfqpoint{3.642013in}{3.117514in}}%
\pgfpathlineto{\pgfqpoint{3.644690in}{3.117541in}}%
\pgfpathlineto{\pgfqpoint{3.647633in}{3.120085in}}%
\pgfpathlineto{\pgfqpoint{3.659140in}{3.133168in}}%
\pgfpathlineto{\pgfqpoint{3.661816in}{3.131787in}}%
\pgfpathlineto{\pgfqpoint{3.665831in}{3.126465in}}%
\pgfpathlineto{\pgfqpoint{3.672521in}{3.117792in}}%
\pgfpathlineto{\pgfqpoint{3.675464in}{3.117429in}}%
\pgfpathlineto{\pgfqpoint{3.678408in}{3.119745in}}%
\pgfpathlineto{\pgfqpoint{3.684028in}{3.128180in}}%
\pgfpathlineto{\pgfqpoint{3.688310in}{3.132770in}}%
\pgfpathlineto{\pgfqpoint{3.690986in}{3.132992in}}%
\pgfpathlineto{\pgfqpoint{3.693929in}{3.130682in}}%
\pgfpathlineto{\pgfqpoint{3.699549in}{3.122249in}}%
\pgfpathlineto{\pgfqpoint{3.703831in}{3.117652in}}%
\pgfpathlineto{\pgfqpoint{3.706507in}{3.117424in}}%
\pgfpathlineto{\pgfqpoint{3.709451in}{3.119729in}}%
\pgfpathlineto{\pgfqpoint{3.715070in}{3.128159in}}%
\pgfpathlineto{\pgfqpoint{3.719352in}{3.132763in}}%
\pgfpathlineto{\pgfqpoint{3.722028in}{3.132997in}}%
\pgfpathlineto{\pgfqpoint{3.724972in}{3.130698in}}%
\pgfpathlineto{\pgfqpoint{3.730592in}{3.122270in}}%
\pgfpathlineto{\pgfqpoint{3.734873in}{3.117659in}}%
\pgfpathlineto{\pgfqpoint{3.737549in}{3.117420in}}%
\pgfpathlineto{\pgfqpoint{3.740493in}{3.119712in}}%
\pgfpathlineto{\pgfqpoint{3.746113in}{3.128138in}}%
\pgfpathlineto{\pgfqpoint{3.750395in}{3.132756in}}%
\pgfpathlineto{\pgfqpoint{3.753071in}{3.133002in}}%
\pgfpathlineto{\pgfqpoint{3.756014in}{3.130715in}}%
\pgfpathlineto{\pgfqpoint{3.761634in}{3.122291in}}%
\pgfpathlineto{\pgfqpoint{3.765916in}{3.117667in}}%
\pgfpathlineto{\pgfqpoint{3.768592in}{3.117415in}}%
\pgfpathlineto{\pgfqpoint{3.771536in}{3.119696in}}%
\pgfpathlineto{\pgfqpoint{3.777155in}{3.128117in}}%
\pgfpathlineto{\pgfqpoint{3.781437in}{3.132748in}}%
\pgfpathlineto{\pgfqpoint{3.784113in}{3.133006in}}%
\pgfpathlineto{\pgfqpoint{3.787057in}{3.130731in}}%
\pgfpathlineto{\pgfqpoint{3.792677in}{3.122313in}}%
\pgfpathlineto{\pgfqpoint{3.796958in}{3.117674in}}%
\pgfpathlineto{\pgfqpoint{3.799634in}{3.117410in}}%
\pgfpathlineto{\pgfqpoint{3.802578in}{3.119679in}}%
\pgfpathlineto{\pgfqpoint{3.808198in}{3.128096in}}%
\pgfpathlineto{\pgfqpoint{3.812747in}{3.132871in}}%
\pgfpathlineto{\pgfqpoint{3.815423in}{3.132911in}}%
\pgfpathlineto{\pgfqpoint{3.818367in}{3.130429in}}%
\pgfpathlineto{\pgfqpoint{3.830142in}{3.117275in}}%
\pgfpathlineto{\pgfqpoint{3.832818in}{3.118810in}}%
\pgfpathlineto{\pgfqpoint{3.836832in}{3.124258in}}%
\pgfpathlineto{\pgfqpoint{3.843255in}{3.132580in}}%
\pgfpathlineto{\pgfqpoint{3.846198in}{3.133016in}}%
\pgfpathlineto{\pgfqpoint{3.849142in}{3.130764in}}%
\pgfpathlineto{\pgfqpoint{3.854494in}{3.122762in}}%
\pgfpathlineto{\pgfqpoint{3.859043in}{3.117689in}}%
\pgfpathlineto{\pgfqpoint{3.861719in}{3.117401in}}%
\pgfpathlineto{\pgfqpoint{3.864663in}{3.119647in}}%
\pgfpathlineto{\pgfqpoint{3.870015in}{3.127646in}}%
\pgfpathlineto{\pgfqpoint{3.874565in}{3.132726in}}%
\pgfpathlineto{\pgfqpoint{3.877241in}{3.133020in}}%
\pgfpathlineto{\pgfqpoint{3.880184in}{3.130780in}}%
\pgfpathlineto{\pgfqpoint{3.885537in}{3.122784in}}%
\pgfpathlineto{\pgfqpoint{3.890086in}{3.117697in}}%
\pgfpathlineto{\pgfqpoint{3.892762in}{3.117396in}}%
\pgfpathlineto{\pgfqpoint{3.895706in}{3.119631in}}%
\pgfpathlineto{\pgfqpoint{3.901058in}{3.127624in}}%
\pgfpathlineto{\pgfqpoint{3.905607in}{3.132718in}}%
\pgfpathlineto{\pgfqpoint{3.908283in}{3.133025in}}%
\pgfpathlineto{\pgfqpoint{3.911227in}{3.130796in}}%
\pgfpathlineto{\pgfqpoint{3.916579in}{3.122805in}}%
\pgfpathlineto{\pgfqpoint{3.921128in}{3.117704in}}%
\pgfpathlineto{\pgfqpoint{3.923804in}{3.117392in}}%
\pgfpathlineto{\pgfqpoint{3.926748in}{3.119614in}}%
\pgfpathlineto{\pgfqpoint{3.932100in}{3.127603in}}%
\pgfpathlineto{\pgfqpoint{3.936650in}{3.132711in}}%
\pgfpathlineto{\pgfqpoint{3.939326in}{3.133029in}}%
\pgfpathlineto{\pgfqpoint{3.942269in}{3.130813in}}%
\pgfpathlineto{\pgfqpoint{3.947622in}{3.122827in}}%
\pgfpathlineto{\pgfqpoint{3.952171in}{3.117712in}}%
\pgfpathlineto{\pgfqpoint{3.954847in}{3.117387in}}%
\pgfpathlineto{\pgfqpoint{3.957791in}{3.119598in}}%
\pgfpathlineto{\pgfqpoint{3.963143in}{3.127581in}}%
\pgfpathlineto{\pgfqpoint{3.967692in}{3.132703in}}%
\pgfpathlineto{\pgfqpoint{3.970368in}{3.133034in}}%
\pgfpathlineto{\pgfqpoint{3.973312in}{3.130829in}}%
\pgfpathlineto{\pgfqpoint{3.978664in}{3.122849in}}%
\pgfpathlineto{\pgfqpoint{3.983213in}{3.117720in}}%
\pgfpathlineto{\pgfqpoint{3.985890in}{3.117383in}}%
\pgfpathlineto{\pgfqpoint{3.988833in}{3.119582in}}%
\pgfpathlineto{\pgfqpoint{3.994185in}{3.127559in}}%
\pgfpathlineto{\pgfqpoint{3.998735in}{3.132695in}}%
\pgfpathlineto{\pgfqpoint{4.001411in}{3.133038in}}%
\pgfpathlineto{\pgfqpoint{4.004354in}{3.130845in}}%
\pgfpathlineto{\pgfqpoint{4.009707in}{3.122870in}}%
\pgfpathlineto{\pgfqpoint{4.014256in}{3.117728in}}%
\pgfpathlineto{\pgfqpoint{4.016932in}{3.117379in}}%
\pgfpathlineto{\pgfqpoint{4.019876in}{3.119566in}}%
\pgfpathlineto{\pgfqpoint{4.024960in}{3.127122in}}%
\pgfpathlineto{\pgfqpoint{4.029777in}{3.132687in}}%
\pgfpathlineto{\pgfqpoint{4.032453in}{3.133042in}}%
\pgfpathlineto{\pgfqpoint{4.035397in}{3.130861in}}%
\pgfpathlineto{\pgfqpoint{4.040482in}{3.123308in}}%
\pgfpathlineto{\pgfqpoint{4.045298in}{3.117735in}}%
\pgfpathlineto{\pgfqpoint{4.047975in}{3.117374in}}%
\pgfpathlineto{\pgfqpoint{4.050918in}{3.119550in}}%
\pgfpathlineto{\pgfqpoint{4.056003in}{3.127100in}}%
\pgfpathlineto{\pgfqpoint{4.060820in}{3.132679in}}%
\pgfpathlineto{\pgfqpoint{4.063496in}{3.133047in}}%
\pgfpathlineto{\pgfqpoint{4.066439in}{3.130877in}}%
\pgfpathlineto{\pgfqpoint{4.071524in}{3.123330in}}%
\pgfpathlineto{\pgfqpoint{4.076341in}{3.117743in}}%
\pgfpathlineto{\pgfqpoint{4.079017in}{3.117370in}}%
\pgfpathlineto{\pgfqpoint{4.081693in}{3.119239in}}%
\pgfpathlineto{\pgfqpoint{4.086242in}{3.125800in}}%
\pgfpathlineto{\pgfqpoint{4.091595in}{3.132509in}}%
\pgfpathlineto{\pgfqpoint{4.094538in}{3.133051in}}%
\pgfpathlineto{\pgfqpoint{4.097214in}{3.131187in}}%
\pgfpathlineto{\pgfqpoint{4.101764in}{3.124631in}}%
\pgfpathlineto{\pgfqpoint{4.107116in}{3.117915in}}%
\pgfpathlineto{\pgfqpoint{4.110060in}{3.117366in}}%
\pgfpathlineto{\pgfqpoint{4.112736in}{3.119224in}}%
\pgfpathlineto{\pgfqpoint{4.117285in}{3.125777in}}%
\pgfpathlineto{\pgfqpoint{4.122637in}{3.132500in}}%
\pgfpathlineto{\pgfqpoint{4.125581in}{3.133055in}}%
\pgfpathlineto{\pgfqpoint{4.128257in}{3.131202in}}%
\pgfpathlineto{\pgfqpoint{4.132806in}{3.124653in}}%
\pgfpathlineto{\pgfqpoint{4.138158in}{3.117924in}}%
\pgfpathlineto{\pgfqpoint{4.141102in}{3.117362in}}%
\pgfpathlineto{\pgfqpoint{4.143778in}{3.119209in}}%
\pgfpathlineto{\pgfqpoint{4.148327in}{3.125754in}}%
\pgfpathlineto{\pgfqpoint{4.153680in}{3.132490in}}%
\pgfpathlineto{\pgfqpoint{4.156623in}{3.133059in}}%
\pgfpathlineto{\pgfqpoint{4.159299in}{3.131217in}}%
\pgfpathlineto{\pgfqpoint{4.163849in}{3.124676in}}%
\pgfpathlineto{\pgfqpoint{4.169201in}{3.117933in}}%
\pgfpathlineto{\pgfqpoint{4.172145in}{3.117358in}}%
\pgfpathlineto{\pgfqpoint{4.174821in}{3.119194in}}%
\pgfpathlineto{\pgfqpoint{4.179370in}{3.125732in}}%
\pgfpathlineto{\pgfqpoint{4.184722in}{3.132481in}}%
\pgfpathlineto{\pgfqpoint{4.187666in}{3.133063in}}%
\pgfpathlineto{\pgfqpoint{4.190342in}{3.131232in}}%
\pgfpathlineto{\pgfqpoint{4.194891in}{3.124699in}}%
\pgfpathlineto{\pgfqpoint{4.200243in}{3.117943in}}%
\pgfpathlineto{\pgfqpoint{4.203187in}{3.117354in}}%
\pgfpathlineto{\pgfqpoint{4.205863in}{3.119180in}}%
\pgfpathlineto{\pgfqpoint{4.210412in}{3.125709in}}%
\pgfpathlineto{\pgfqpoint{4.216032in}{3.132639in}}%
\pgfpathlineto{\pgfqpoint{4.218976in}{3.132982in}}%
\pgfpathlineto{\pgfqpoint{4.221920in}{3.130649in}}%
\pgfpathlineto{\pgfqpoint{4.227807in}{3.121812in}}%
\pgfpathlineto{\pgfqpoint{4.232089in}{3.117514in}}%
\pgfpathlineto{\pgfqpoint{4.234765in}{3.117541in}}%
\pgfpathlineto{\pgfqpoint{4.237708in}{3.120085in}}%
\pgfpathlineto{\pgfqpoint{4.249216in}{3.133168in}}%
\pgfpathlineto{\pgfqpoint{4.251892in}{3.131787in}}%
\pgfpathlineto{\pgfqpoint{4.255906in}{3.126465in}}%
\pgfpathlineto{\pgfqpoint{4.262596in}{3.117792in}}%
\pgfpathlineto{\pgfqpoint{4.265540in}{3.117429in}}%
\pgfpathlineto{\pgfqpoint{4.268483in}{3.119745in}}%
\pgfpathlineto{\pgfqpoint{4.274103in}{3.128180in}}%
\pgfpathlineto{\pgfqpoint{4.278385in}{3.132770in}}%
\pgfpathlineto{\pgfqpoint{4.281061in}{3.132992in}}%
\pgfpathlineto{\pgfqpoint{4.284005in}{3.130682in}}%
\pgfpathlineto{\pgfqpoint{4.289624in}{3.122249in}}%
\pgfpathlineto{\pgfqpoint{4.293906in}{3.117652in}}%
\pgfpathlineto{\pgfqpoint{4.296582in}{3.117424in}}%
\pgfpathlineto{\pgfqpoint{4.299526in}{3.119729in}}%
\pgfpathlineto{\pgfqpoint{4.305146in}{3.128159in}}%
\pgfpathlineto{\pgfqpoint{4.309427in}{3.132763in}}%
\pgfpathlineto{\pgfqpoint{4.312103in}{3.132997in}}%
\pgfpathlineto{\pgfqpoint{4.315047in}{3.130698in}}%
\pgfpathlineto{\pgfqpoint{4.320667in}{3.122270in}}%
\pgfpathlineto{\pgfqpoint{4.324949in}{3.117659in}}%
\pgfpathlineto{\pgfqpoint{4.327625in}{3.117420in}}%
\pgfpathlineto{\pgfqpoint{4.330568in}{3.119712in}}%
\pgfpathlineto{\pgfqpoint{4.336188in}{3.128138in}}%
\pgfpathlineto{\pgfqpoint{4.340470in}{3.132756in}}%
\pgfpathlineto{\pgfqpoint{4.343146in}{3.133002in}}%
\pgfpathlineto{\pgfqpoint{4.346090in}{3.130715in}}%
\pgfpathlineto{\pgfqpoint{4.351709in}{3.122291in}}%
\pgfpathlineto{\pgfqpoint{4.355991in}{3.117667in}}%
\pgfpathlineto{\pgfqpoint{4.358667in}{3.117415in}}%
\pgfpathlineto{\pgfqpoint{4.361611in}{3.119696in}}%
\pgfpathlineto{\pgfqpoint{4.367231in}{3.128117in}}%
\pgfpathlineto{\pgfqpoint{4.371512in}{3.132748in}}%
\pgfpathlineto{\pgfqpoint{4.374188in}{3.133006in}}%
\pgfpathlineto{\pgfqpoint{4.377132in}{3.130731in}}%
\pgfpathlineto{\pgfqpoint{4.382752in}{3.122313in}}%
\pgfpathlineto{\pgfqpoint{4.387034in}{3.117674in}}%
\pgfpathlineto{\pgfqpoint{4.389710in}{3.117410in}}%
\pgfpathlineto{\pgfqpoint{4.392653in}{3.119679in}}%
\pgfpathlineto{\pgfqpoint{4.398273in}{3.128096in}}%
\pgfpathlineto{\pgfqpoint{4.402822in}{3.132871in}}%
\pgfpathlineto{\pgfqpoint{4.405499in}{3.132911in}}%
\pgfpathlineto{\pgfqpoint{4.408442in}{3.130429in}}%
\pgfpathlineto{\pgfqpoint{4.420217in}{3.117275in}}%
\pgfpathlineto{\pgfqpoint{4.422893in}{3.118810in}}%
\pgfpathlineto{\pgfqpoint{4.426907in}{3.124258in}}%
\pgfpathlineto{\pgfqpoint{4.433330in}{3.132580in}}%
\pgfpathlineto{\pgfqpoint{4.436273in}{3.133016in}}%
\pgfpathlineto{\pgfqpoint{4.439217in}{3.130764in}}%
\pgfpathlineto{\pgfqpoint{4.444569in}{3.122762in}}%
\pgfpathlineto{\pgfqpoint{4.449119in}{3.117689in}}%
\pgfpathlineto{\pgfqpoint{4.451795in}{3.117401in}}%
\pgfpathlineto{\pgfqpoint{4.454738in}{3.119647in}}%
\pgfpathlineto{\pgfqpoint{4.460091in}{3.127646in}}%
\pgfpathlineto{\pgfqpoint{4.464640in}{3.132726in}}%
\pgfpathlineto{\pgfqpoint{4.467316in}{3.133020in}}%
\pgfpathlineto{\pgfqpoint{4.470260in}{3.130780in}}%
\pgfpathlineto{\pgfqpoint{4.475612in}{3.122784in}}%
\pgfpathlineto{\pgfqpoint{4.480161in}{3.117697in}}%
\pgfpathlineto{\pgfqpoint{4.482837in}{3.117396in}}%
\pgfpathlineto{\pgfqpoint{4.485781in}{3.119631in}}%
\pgfpathlineto{\pgfqpoint{4.491133in}{3.127624in}}%
\pgfpathlineto{\pgfqpoint{4.495682in}{3.132718in}}%
\pgfpathlineto{\pgfqpoint{4.498358in}{3.133025in}}%
\pgfpathlineto{\pgfqpoint{4.501302in}{3.130796in}}%
\pgfpathlineto{\pgfqpoint{4.506654in}{3.122805in}}%
\pgfpathlineto{\pgfqpoint{4.511204in}{3.117704in}}%
\pgfpathlineto{\pgfqpoint{4.513880in}{3.117392in}}%
\pgfpathlineto{\pgfqpoint{4.516823in}{3.119614in}}%
\pgfpathlineto{\pgfqpoint{4.522176in}{3.127603in}}%
\pgfpathlineto{\pgfqpoint{4.526725in}{3.132711in}}%
\pgfpathlineto{\pgfqpoint{4.529401in}{3.133029in}}%
\pgfpathlineto{\pgfqpoint{4.532345in}{3.130813in}}%
\pgfpathlineto{\pgfqpoint{4.537697in}{3.122827in}}%
\pgfpathlineto{\pgfqpoint{4.542246in}{3.117712in}}%
\pgfpathlineto{\pgfqpoint{4.544922in}{3.117387in}}%
\pgfpathlineto{\pgfqpoint{4.547866in}{3.119598in}}%
\pgfpathlineto{\pgfqpoint{4.553218in}{3.127581in}}%
\pgfpathlineto{\pgfqpoint{4.557767in}{3.132703in}}%
\pgfpathlineto{\pgfqpoint{4.560443in}{3.133034in}}%
\pgfpathlineto{\pgfqpoint{4.563387in}{3.130829in}}%
\pgfpathlineto{\pgfqpoint{4.568739in}{3.122849in}}%
\pgfpathlineto{\pgfqpoint{4.573289in}{3.117720in}}%
\pgfpathlineto{\pgfqpoint{4.575965in}{3.117383in}}%
\pgfpathlineto{\pgfqpoint{4.578908in}{3.119582in}}%
\pgfpathlineto{\pgfqpoint{4.584261in}{3.127559in}}%
\pgfpathlineto{\pgfqpoint{4.588810in}{3.132695in}}%
\pgfpathlineto{\pgfqpoint{4.591486in}{3.133038in}}%
\pgfpathlineto{\pgfqpoint{4.594430in}{3.130845in}}%
\pgfpathlineto{\pgfqpoint{4.599782in}{3.122870in}}%
\pgfpathlineto{\pgfqpoint{4.604331in}{3.117728in}}%
\pgfpathlineto{\pgfqpoint{4.607007in}{3.117379in}}%
\pgfpathlineto{\pgfqpoint{4.609951in}{3.119566in}}%
\pgfpathlineto{\pgfqpoint{4.615035in}{3.127122in}}%
\pgfpathlineto{\pgfqpoint{4.619852in}{3.132687in}}%
\pgfpathlineto{\pgfqpoint{4.622529in}{3.133042in}}%
\pgfpathlineto{\pgfqpoint{4.625472in}{3.130861in}}%
\pgfpathlineto{\pgfqpoint{4.630557in}{3.123308in}}%
\pgfpathlineto{\pgfqpoint{4.635374in}{3.117735in}}%
\pgfpathlineto{\pgfqpoint{4.638050in}{3.117374in}}%
\pgfpathlineto{\pgfqpoint{4.640993in}{3.119550in}}%
\pgfpathlineto{\pgfqpoint{4.646078in}{3.127100in}}%
\pgfpathlineto{\pgfqpoint{4.650895in}{3.132679in}}%
\pgfpathlineto{\pgfqpoint{4.653571in}{3.133047in}}%
\pgfpathlineto{\pgfqpoint{4.656515in}{3.130877in}}%
\pgfpathlineto{\pgfqpoint{4.661599in}{3.123330in}}%
\pgfpathlineto{\pgfqpoint{4.666416in}{3.117743in}}%
\pgfpathlineto{\pgfqpoint{4.669092in}{3.117370in}}%
\pgfpathlineto{\pgfqpoint{4.671768in}{3.119239in}}%
\pgfpathlineto{\pgfqpoint{4.676318in}{3.125800in}}%
\pgfpathlineto{\pgfqpoint{4.681670in}{3.132509in}}%
\pgfpathlineto{\pgfqpoint{4.684614in}{3.133051in}}%
\pgfpathlineto{\pgfqpoint{4.687290in}{3.131187in}}%
\pgfpathlineto{\pgfqpoint{4.691839in}{3.124631in}}%
\pgfpathlineto{\pgfqpoint{4.697191in}{3.117915in}}%
\pgfpathlineto{\pgfqpoint{4.700135in}{3.117366in}}%
\pgfpathlineto{\pgfqpoint{4.702811in}{3.119224in}}%
\pgfpathlineto{\pgfqpoint{4.707360in}{3.125777in}}%
\pgfpathlineto{\pgfqpoint{4.712712in}{3.132500in}}%
\pgfpathlineto{\pgfqpoint{4.715656in}{3.133055in}}%
\pgfpathlineto{\pgfqpoint{4.718332in}{3.131202in}}%
\pgfpathlineto{\pgfqpoint{4.722881in}{3.124653in}}%
\pgfpathlineto{\pgfqpoint{4.728234in}{3.117924in}}%
\pgfpathlineto{\pgfqpoint{4.731177in}{3.117362in}}%
\pgfpathlineto{\pgfqpoint{4.733853in}{3.119209in}}%
\pgfpathlineto{\pgfqpoint{4.738403in}{3.125754in}}%
\pgfpathlineto{\pgfqpoint{4.743755in}{3.132490in}}%
\pgfpathlineto{\pgfqpoint{4.746699in}{3.133059in}}%
\pgfpathlineto{\pgfqpoint{4.749375in}{3.131217in}}%
\pgfpathlineto{\pgfqpoint{4.753924in}{3.124676in}}%
\pgfpathlineto{\pgfqpoint{4.759276in}{3.117933in}}%
\pgfpathlineto{\pgfqpoint{4.762220in}{3.117358in}}%
\pgfpathlineto{\pgfqpoint{4.764896in}{3.119194in}}%
\pgfpathlineto{\pgfqpoint{4.769445in}{3.125732in}}%
\pgfpathlineto{\pgfqpoint{4.774797in}{3.132481in}}%
\pgfpathlineto{\pgfqpoint{4.777741in}{3.133063in}}%
\pgfpathlineto{\pgfqpoint{4.780417in}{3.131232in}}%
\pgfpathlineto{\pgfqpoint{4.784966in}{3.124699in}}%
\pgfpathlineto{\pgfqpoint{4.790319in}{3.117943in}}%
\pgfpathlineto{\pgfqpoint{4.793262in}{3.117354in}}%
\pgfpathlineto{\pgfqpoint{4.795938in}{3.119180in}}%
\pgfpathlineto{\pgfqpoint{4.800488in}{3.125709in}}%
\pgfpathlineto{\pgfqpoint{4.806107in}{3.132639in}}%
\pgfpathlineto{\pgfqpoint{4.809051in}{3.132982in}}%
\pgfpathlineto{\pgfqpoint{4.811995in}{3.130649in}}%
\pgfpathlineto{\pgfqpoint{4.817882in}{3.121812in}}%
\pgfpathlineto{\pgfqpoint{4.822164in}{3.117514in}}%
\pgfpathlineto{\pgfqpoint{4.824840in}{3.117541in}}%
\pgfpathlineto{\pgfqpoint{4.827784in}{3.120085in}}%
\pgfpathlineto{\pgfqpoint{4.839291in}{3.133168in}}%
\pgfpathlineto{\pgfqpoint{4.841967in}{3.131787in}}%
\pgfpathlineto{\pgfqpoint{4.845981in}{3.126465in}}%
\pgfpathlineto{\pgfqpoint{4.852671in}{3.117792in}}%
\pgfpathlineto{\pgfqpoint{4.855615in}{3.117429in}}%
\pgfpathlineto{\pgfqpoint{4.858559in}{3.119745in}}%
\pgfpathlineto{\pgfqpoint{4.864178in}{3.128180in}}%
\pgfpathlineto{\pgfqpoint{4.868460in}{3.132770in}}%
\pgfpathlineto{\pgfqpoint{4.871136in}{3.132992in}}%
\pgfpathlineto{\pgfqpoint{4.874080in}{3.130682in}}%
\pgfpathlineto{\pgfqpoint{4.879700in}{3.122249in}}%
\pgfpathlineto{\pgfqpoint{4.883981in}{3.117652in}}%
\pgfpathlineto{\pgfqpoint{4.886657in}{3.117424in}}%
\pgfpathlineto{\pgfqpoint{4.889601in}{3.119729in}}%
\pgfpathlineto{\pgfqpoint{4.895221in}{3.128159in}}%
\pgfpathlineto{\pgfqpoint{4.899503in}{3.132763in}}%
\pgfpathlineto{\pgfqpoint{4.902179in}{3.132997in}}%
\pgfpathlineto{\pgfqpoint{4.905122in}{3.130698in}}%
\pgfpathlineto{\pgfqpoint{4.910742in}{3.122270in}}%
\pgfpathlineto{\pgfqpoint{4.915024in}{3.117659in}}%
\pgfpathlineto{\pgfqpoint{4.917700in}{3.117420in}}%
\pgfpathlineto{\pgfqpoint{4.920644in}{3.119712in}}%
\pgfpathlineto{\pgfqpoint{4.926263in}{3.128138in}}%
\pgfpathlineto{\pgfqpoint{4.930545in}{3.132756in}}%
\pgfpathlineto{\pgfqpoint{4.933221in}{3.133002in}}%
\pgfpathlineto{\pgfqpoint{4.936165in}{3.130715in}}%
\pgfpathlineto{\pgfqpoint{4.941785in}{3.122291in}}%
\pgfpathlineto{\pgfqpoint{4.946066in}{3.117667in}}%
\pgfpathlineto{\pgfqpoint{4.948742in}{3.117415in}}%
\pgfpathlineto{\pgfqpoint{4.951686in}{3.119696in}}%
\pgfpathlineto{\pgfqpoint{4.957306in}{3.128117in}}%
\pgfpathlineto{\pgfqpoint{4.961588in}{3.132748in}}%
\pgfpathlineto{\pgfqpoint{4.964264in}{3.133006in}}%
\pgfpathlineto{\pgfqpoint{4.967207in}{3.130731in}}%
\pgfpathlineto{\pgfqpoint{4.972827in}{3.122313in}}%
\pgfpathlineto{\pgfqpoint{4.977109in}{3.117674in}}%
\pgfpathlineto{\pgfqpoint{4.979785in}{3.117410in}}%
\pgfpathlineto{\pgfqpoint{4.982729in}{3.119679in}}%
\pgfpathlineto{\pgfqpoint{4.988348in}{3.128096in}}%
\pgfpathlineto{\pgfqpoint{4.992898in}{3.132871in}}%
\pgfpathlineto{\pgfqpoint{4.995574in}{3.132911in}}%
\pgfpathlineto{\pgfqpoint{4.998518in}{3.130429in}}%
\pgfpathlineto{\pgfqpoint{5.010292in}{3.117275in}}%
\pgfpathlineto{\pgfqpoint{5.012968in}{3.118810in}}%
\pgfpathlineto{\pgfqpoint{5.016982in}{3.124258in}}%
\pgfpathlineto{\pgfqpoint{5.023405in}{3.132580in}}%
\pgfpathlineto{\pgfqpoint{5.026349in}{3.133016in}}%
\pgfpathlineto{\pgfqpoint{5.029292in}{3.130764in}}%
\pgfpathlineto{\pgfqpoint{5.034645in}{3.122762in}}%
\pgfpathlineto{\pgfqpoint{5.039194in}{3.117689in}}%
\pgfpathlineto{\pgfqpoint{5.041870in}{3.117401in}}%
\pgfpathlineto{\pgfqpoint{5.044814in}{3.119647in}}%
\pgfpathlineto{\pgfqpoint{5.050166in}{3.127646in}}%
\pgfpathlineto{\pgfqpoint{5.054715in}{3.132726in}}%
\pgfpathlineto{\pgfqpoint{5.057391in}{3.133020in}}%
\pgfpathlineto{\pgfqpoint{5.060335in}{3.130780in}}%
\pgfpathlineto{\pgfqpoint{5.065687in}{3.122784in}}%
\pgfpathlineto{\pgfqpoint{5.070236in}{3.117697in}}%
\pgfpathlineto{\pgfqpoint{5.072912in}{3.117396in}}%
\pgfpathlineto{\pgfqpoint{5.075856in}{3.119631in}}%
\pgfpathlineto{\pgfqpoint{5.081208in}{3.127624in}}%
\pgfpathlineto{\pgfqpoint{5.085758in}{3.132718in}}%
\pgfpathlineto{\pgfqpoint{5.088434in}{3.133025in}}%
\pgfpathlineto{\pgfqpoint{5.091377in}{3.130796in}}%
\pgfpathlineto{\pgfqpoint{5.096730in}{3.122805in}}%
\pgfpathlineto{\pgfqpoint{5.101279in}{3.117704in}}%
\pgfpathlineto{\pgfqpoint{5.103955in}{3.117392in}}%
\pgfpathlineto{\pgfqpoint{5.106899in}{3.119614in}}%
\pgfpathlineto{\pgfqpoint{5.112251in}{3.127603in}}%
\pgfpathlineto{\pgfqpoint{5.116800in}{3.132711in}}%
\pgfpathlineto{\pgfqpoint{5.119476in}{3.133029in}}%
\pgfpathlineto{\pgfqpoint{5.122420in}{3.130813in}}%
\pgfpathlineto{\pgfqpoint{5.127772in}{3.122827in}}%
\pgfpathlineto{\pgfqpoint{5.132321in}{3.117712in}}%
\pgfpathlineto{\pgfqpoint{5.134997in}{3.117387in}}%
\pgfpathlineto{\pgfqpoint{5.137941in}{3.119598in}}%
\pgfpathlineto{\pgfqpoint{5.143293in}{3.127581in}}%
\pgfpathlineto{\pgfqpoint{5.147843in}{3.132703in}}%
\pgfpathlineto{\pgfqpoint{5.150519in}{3.133034in}}%
\pgfpathlineto{\pgfqpoint{5.153462in}{3.130829in}}%
\pgfpathlineto{\pgfqpoint{5.158815in}{3.122849in}}%
\pgfpathlineto{\pgfqpoint{5.163364in}{3.117720in}}%
\pgfpathlineto{\pgfqpoint{5.166040in}{3.117383in}}%
\pgfpathlineto{\pgfqpoint{5.168984in}{3.119582in}}%
\pgfpathlineto{\pgfqpoint{5.174336in}{3.127559in}}%
\pgfpathlineto{\pgfqpoint{5.178885in}{3.132695in}}%
\pgfpathlineto{\pgfqpoint{5.181561in}{3.133038in}}%
\pgfpathlineto{\pgfqpoint{5.184505in}{3.130845in}}%
\pgfpathlineto{\pgfqpoint{5.189857in}{3.122870in}}%
\pgfpathlineto{\pgfqpoint{5.194406in}{3.117728in}}%
\pgfpathlineto{\pgfqpoint{5.197083in}{3.117379in}}%
\pgfpathlineto{\pgfqpoint{5.200026in}{3.119566in}}%
\pgfpathlineto{\pgfqpoint{5.205111in}{3.127122in}}%
\pgfpathlineto{\pgfqpoint{5.209928in}{3.132687in}}%
\pgfpathlineto{\pgfqpoint{5.212604in}{3.133042in}}%
\pgfpathlineto{\pgfqpoint{5.215547in}{3.130861in}}%
\pgfpathlineto{\pgfqpoint{5.220632in}{3.123308in}}%
\pgfpathlineto{\pgfqpoint{5.225449in}{3.117735in}}%
\pgfpathlineto{\pgfqpoint{5.228125in}{3.117374in}}%
\pgfpathlineto{\pgfqpoint{5.231069in}{3.119550in}}%
\pgfpathlineto{\pgfqpoint{5.236153in}{3.127100in}}%
\pgfpathlineto{\pgfqpoint{5.240970in}{3.132679in}}%
\pgfpathlineto{\pgfqpoint{5.243646in}{3.133047in}}%
\pgfpathlineto{\pgfqpoint{5.246590in}{3.130877in}}%
\pgfpathlineto{\pgfqpoint{5.251675in}{3.123330in}}%
\pgfpathlineto{\pgfqpoint{5.256491in}{3.117743in}}%
\pgfpathlineto{\pgfqpoint{5.259168in}{3.117370in}}%
\pgfpathlineto{\pgfqpoint{5.261844in}{3.119239in}}%
\pgfpathlineto{\pgfqpoint{5.266393in}{3.125800in}}%
\pgfpathlineto{\pgfqpoint{5.271745in}{3.132509in}}%
\pgfpathlineto{\pgfqpoint{5.274689in}{3.133051in}}%
\pgfpathlineto{\pgfqpoint{5.277365in}{3.131187in}}%
\pgfpathlineto{\pgfqpoint{5.281914in}{3.124631in}}%
\pgfpathlineto{\pgfqpoint{5.287266in}{3.117915in}}%
\pgfpathlineto{\pgfqpoint{5.290210in}{3.117366in}}%
\pgfpathlineto{\pgfqpoint{5.292886in}{3.119224in}}%
\pgfpathlineto{\pgfqpoint{5.297435in}{3.125777in}}%
\pgfpathlineto{\pgfqpoint{5.302788in}{3.132500in}}%
\pgfpathlineto{\pgfqpoint{5.305731in}{3.133055in}}%
\pgfpathlineto{\pgfqpoint{5.308407in}{3.131202in}}%
\pgfpathlineto{\pgfqpoint{5.312957in}{3.124653in}}%
\pgfpathlineto{\pgfqpoint{5.318309in}{3.117924in}}%
\pgfpathlineto{\pgfqpoint{5.321253in}{3.117362in}}%
\pgfpathlineto{\pgfqpoint{5.323929in}{3.119209in}}%
\pgfpathlineto{\pgfqpoint{5.328478in}{3.125754in}}%
\pgfpathlineto{\pgfqpoint{5.333830in}{3.132490in}}%
\pgfpathlineto{\pgfqpoint{5.336774in}{3.133059in}}%
\pgfpathlineto{\pgfqpoint{5.339450in}{3.131217in}}%
\pgfpathlineto{\pgfqpoint{5.343999in}{3.124676in}}%
\pgfpathlineto{\pgfqpoint{5.349351in}{3.117933in}}%
\pgfpathlineto{\pgfqpoint{5.352295in}{3.117358in}}%
\pgfpathlineto{\pgfqpoint{5.354971in}{3.119194in}}%
\pgfpathlineto{\pgfqpoint{5.359520in}{3.125732in}}%
\pgfpathlineto{\pgfqpoint{5.364873in}{3.132481in}}%
\pgfpathlineto{\pgfqpoint{5.367816in}{3.133063in}}%
\pgfpathlineto{\pgfqpoint{5.370492in}{3.131232in}}%
\pgfpathlineto{\pgfqpoint{5.375042in}{3.124699in}}%
\pgfpathlineto{\pgfqpoint{5.380394in}{3.117943in}}%
\pgfpathlineto{\pgfqpoint{5.383338in}{3.117354in}}%
\pgfpathlineto{\pgfqpoint{5.386014in}{3.119180in}}%
\pgfpathlineto{\pgfqpoint{5.390563in}{3.125709in}}%
\pgfpathlineto{\pgfqpoint{5.396183in}{3.132639in}}%
\pgfpathlineto{\pgfqpoint{5.399126in}{3.132982in}}%
\pgfpathlineto{\pgfqpoint{5.402070in}{3.130649in}}%
\pgfpathlineto{\pgfqpoint{5.407957in}{3.121812in}}%
\pgfpathlineto{\pgfqpoint{5.412239in}{3.117514in}}%
\pgfpathlineto{\pgfqpoint{5.414915in}{3.117541in}}%
\pgfpathlineto{\pgfqpoint{5.417859in}{3.120085in}}%
\pgfpathlineto{\pgfqpoint{5.429366in}{3.133168in}}%
\pgfpathlineto{\pgfqpoint{5.432042in}{3.131787in}}%
\pgfpathlineto{\pgfqpoint{5.436056in}{3.126465in}}%
\pgfpathlineto{\pgfqpoint{5.442746in}{3.117792in}}%
\pgfpathlineto{\pgfqpoint{5.445690in}{3.117429in}}%
\pgfpathlineto{\pgfqpoint{5.448634in}{3.119745in}}%
\pgfpathlineto{\pgfqpoint{5.454254in}{3.128180in}}%
\pgfpathlineto{\pgfqpoint{5.458535in}{3.132770in}}%
\pgfpathlineto{\pgfqpoint{5.461211in}{3.132992in}}%
\pgfpathlineto{\pgfqpoint{5.464155in}{3.130682in}}%
\pgfpathlineto{\pgfqpoint{5.469775in}{3.122249in}}%
\pgfpathlineto{\pgfqpoint{5.474057in}{3.117652in}}%
\pgfpathlineto{\pgfqpoint{5.476733in}{3.117424in}}%
\pgfpathlineto{\pgfqpoint{5.479676in}{3.119729in}}%
\pgfpathlineto{\pgfqpoint{5.485296in}{3.128159in}}%
\pgfpathlineto{\pgfqpoint{5.489578in}{3.132763in}}%
\pgfpathlineto{\pgfqpoint{5.492254in}{3.132997in}}%
\pgfpathlineto{\pgfqpoint{5.495198in}{3.130698in}}%
\pgfpathlineto{\pgfqpoint{5.500817in}{3.122270in}}%
\pgfpathlineto{\pgfqpoint{5.505099in}{3.117659in}}%
\pgfpathlineto{\pgfqpoint{5.507775in}{3.117420in}}%
\pgfpathlineto{\pgfqpoint{5.510719in}{3.119712in}}%
\pgfpathlineto{\pgfqpoint{5.516339in}{3.128138in}}%
\pgfpathlineto{\pgfqpoint{5.520620in}{3.132756in}}%
\pgfpathlineto{\pgfqpoint{5.523296in}{3.133002in}}%
\pgfpathlineto{\pgfqpoint{5.526240in}{3.130715in}}%
\pgfpathlineto{\pgfqpoint{5.531860in}{3.122291in}}%
\pgfpathlineto{\pgfqpoint{5.536142in}{3.117667in}}%
\pgfpathlineto{\pgfqpoint{5.538818in}{3.117415in}}%
\pgfpathlineto{\pgfqpoint{5.541761in}{3.119696in}}%
\pgfpathlineto{\pgfqpoint{5.547381in}{3.128117in}}%
\pgfpathlineto{\pgfqpoint{5.551663in}{3.132748in}}%
\pgfpathlineto{\pgfqpoint{5.554339in}{3.133006in}}%
\pgfpathlineto{\pgfqpoint{5.557283in}{3.130731in}}%
\pgfpathlineto{\pgfqpoint{5.562902in}{3.122313in}}%
\pgfpathlineto{\pgfqpoint{5.567184in}{3.117674in}}%
\pgfpathlineto{\pgfqpoint{5.569860in}{3.117410in}}%
\pgfpathlineto{\pgfqpoint{5.572804in}{3.119679in}}%
\pgfpathlineto{\pgfqpoint{5.578424in}{3.128096in}}%
\pgfpathlineto{\pgfqpoint{5.582973in}{3.132871in}}%
\pgfpathlineto{\pgfqpoint{5.585649in}{3.132911in}}%
\pgfpathlineto{\pgfqpoint{5.588593in}{3.130429in}}%
\pgfpathlineto{\pgfqpoint{5.600367in}{3.117275in}}%
\pgfpathlineto{\pgfqpoint{5.603044in}{3.118810in}}%
\pgfpathlineto{\pgfqpoint{5.607058in}{3.124258in}}%
\pgfpathlineto{\pgfqpoint{5.613480in}{3.132580in}}%
\pgfpathlineto{\pgfqpoint{5.616424in}{3.133016in}}%
\pgfpathlineto{\pgfqpoint{5.619368in}{3.130764in}}%
\pgfpathlineto{\pgfqpoint{5.624720in}{3.122762in}}%
\pgfpathlineto{\pgfqpoint{5.629269in}{3.117689in}}%
\pgfpathlineto{\pgfqpoint{5.631945in}{3.117401in}}%
\pgfpathlineto{\pgfqpoint{5.634889in}{3.119647in}}%
\pgfpathlineto{\pgfqpoint{5.640241in}{3.127646in}}%
\pgfpathlineto{\pgfqpoint{5.644790in}{3.132726in}}%
\pgfpathlineto{\pgfqpoint{5.647466in}{3.133020in}}%
\pgfpathlineto{\pgfqpoint{5.650410in}{3.130780in}}%
\pgfpathlineto{\pgfqpoint{5.655762in}{3.122784in}}%
\pgfpathlineto{\pgfqpoint{5.660312in}{3.117697in}}%
\pgfpathlineto{\pgfqpoint{5.662988in}{3.117396in}}%
\pgfpathlineto{\pgfqpoint{5.665931in}{3.119631in}}%
\pgfpathlineto{\pgfqpoint{5.671284in}{3.127624in}}%
\pgfpathlineto{\pgfqpoint{5.675833in}{3.132718in}}%
\pgfpathlineto{\pgfqpoint{5.678509in}{3.133025in}}%
\pgfpathlineto{\pgfqpoint{5.681453in}{3.130796in}}%
\pgfpathlineto{\pgfqpoint{5.686805in}{3.122805in}}%
\pgfpathlineto{\pgfqpoint{5.691354in}{3.117704in}}%
\pgfpathlineto{\pgfqpoint{5.694030in}{3.117392in}}%
\pgfpathlineto{\pgfqpoint{5.696974in}{3.119614in}}%
\pgfpathlineto{\pgfqpoint{5.702326in}{3.127603in}}%
\pgfpathlineto{\pgfqpoint{5.706875in}{3.132711in}}%
\pgfpathlineto{\pgfqpoint{5.709551in}{3.133029in}}%
\pgfpathlineto{\pgfqpoint{5.712495in}{3.130813in}}%
\pgfpathlineto{\pgfqpoint{5.717847in}{3.122827in}}%
\pgfpathlineto{\pgfqpoint{5.722397in}{3.117712in}}%
\pgfpathlineto{\pgfqpoint{5.725073in}{3.117387in}}%
\pgfpathlineto{\pgfqpoint{5.728016in}{3.119598in}}%
\pgfpathlineto{\pgfqpoint{5.733369in}{3.127581in}}%
\pgfpathlineto{\pgfqpoint{5.737918in}{3.132703in}}%
\pgfpathlineto{\pgfqpoint{5.740594in}{3.133034in}}%
\pgfpathlineto{\pgfqpoint{5.743538in}{3.130829in}}%
\pgfpathlineto{\pgfqpoint{5.748890in}{3.122849in}}%
\pgfpathlineto{\pgfqpoint{5.753439in}{3.117720in}}%
\pgfpathlineto{\pgfqpoint{5.756115in}{3.117383in}}%
\pgfpathlineto{\pgfqpoint{5.759059in}{3.119582in}}%
\pgfpathlineto{\pgfqpoint{5.764411in}{3.127559in}}%
\pgfpathlineto{\pgfqpoint{5.768960in}{3.132695in}}%
\pgfpathlineto{\pgfqpoint{5.771636in}{3.133038in}}%
\pgfpathlineto{\pgfqpoint{5.774580in}{3.130845in}}%
\pgfpathlineto{\pgfqpoint{5.779932in}{3.122870in}}%
\pgfpathlineto{\pgfqpoint{5.784482in}{3.117728in}}%
\pgfpathlineto{\pgfqpoint{5.787158in}{3.117379in}}%
\pgfpathlineto{\pgfqpoint{5.790101in}{3.119566in}}%
\pgfpathlineto{\pgfqpoint{5.795186in}{3.127122in}}%
\pgfpathlineto{\pgfqpoint{5.800003in}{3.132687in}}%
\pgfpathlineto{\pgfqpoint{5.802679in}{3.133042in}}%
\pgfpathlineto{\pgfqpoint{5.805623in}{3.130861in}}%
\pgfpathlineto{\pgfqpoint{5.810707in}{3.123308in}}%
\pgfpathlineto{\pgfqpoint{5.815524in}{3.117735in}}%
\pgfpathlineto{\pgfqpoint{5.818200in}{3.117374in}}%
\pgfpathlineto{\pgfqpoint{5.821144in}{3.119550in}}%
\pgfpathlineto{\pgfqpoint{5.826228in}{3.127100in}}%
\pgfpathlineto{\pgfqpoint{5.831045in}{3.132679in}}%
\pgfpathlineto{\pgfqpoint{5.833722in}{3.133047in}}%
\pgfpathlineto{\pgfqpoint{5.836665in}{3.130877in}}%
\pgfpathlineto{\pgfqpoint{5.841750in}{3.123330in}}%
\pgfpathlineto{\pgfqpoint{5.846567in}{3.117743in}}%
\pgfpathlineto{\pgfqpoint{5.849243in}{3.117370in}}%
\pgfpathlineto{\pgfqpoint{5.851919in}{3.119239in}}%
\pgfpathlineto{\pgfqpoint{5.856468in}{3.125800in}}%
\pgfpathlineto{\pgfqpoint{5.861820in}{3.132509in}}%
\pgfpathlineto{\pgfqpoint{5.864764in}{3.133051in}}%
\pgfpathlineto{\pgfqpoint{5.867440in}{3.131187in}}%
\pgfpathlineto{\pgfqpoint{5.871989in}{3.124631in}}%
\pgfpathlineto{\pgfqpoint{5.877342in}{3.117915in}}%
\pgfpathlineto{\pgfqpoint{5.880285in}{3.117366in}}%
\pgfpathlineto{\pgfqpoint{5.882961in}{3.119224in}}%
\pgfpathlineto{\pgfqpoint{5.887511in}{3.125777in}}%
\pgfpathlineto{\pgfqpoint{5.892863in}{3.132500in}}%
\pgfpathlineto{\pgfqpoint{5.895807in}{3.133055in}}%
\pgfpathlineto{\pgfqpoint{5.898483in}{3.131202in}}%
\pgfpathlineto{\pgfqpoint{5.903032in}{3.124653in}}%
\pgfpathlineto{\pgfqpoint{5.908384in}{3.117924in}}%
\pgfpathlineto{\pgfqpoint{5.911328in}{3.117362in}}%
\pgfpathlineto{\pgfqpoint{5.914004in}{3.119209in}}%
\pgfpathlineto{\pgfqpoint{5.918553in}{3.125754in}}%
\pgfpathlineto{\pgfqpoint{5.923905in}{3.132490in}}%
\pgfpathlineto{\pgfqpoint{5.926849in}{3.133059in}}%
\pgfpathlineto{\pgfqpoint{5.929525in}{3.131217in}}%
\pgfpathlineto{\pgfqpoint{5.934074in}{3.124676in}}%
\pgfpathlineto{\pgfqpoint{5.939427in}{3.117933in}}%
\pgfpathlineto{\pgfqpoint{5.942370in}{3.117358in}}%
\pgfpathlineto{\pgfqpoint{5.945046in}{3.119194in}}%
\pgfpathlineto{\pgfqpoint{5.949596in}{3.125732in}}%
\pgfpathlineto{\pgfqpoint{5.954948in}{3.132481in}}%
\pgfpathlineto{\pgfqpoint{5.957892in}{3.133063in}}%
\pgfpathlineto{\pgfqpoint{5.960568in}{3.131232in}}%
\pgfpathlineto{\pgfqpoint{5.965117in}{3.124699in}}%
\pgfpathlineto{\pgfqpoint{5.970469in}{3.117943in}}%
\pgfpathlineto{\pgfqpoint{5.973413in}{3.117354in}}%
\pgfpathlineto{\pgfqpoint{5.976089in}{3.119180in}}%
\pgfpathlineto{\pgfqpoint{5.980638in}{3.125709in}}%
\pgfpathlineto{\pgfqpoint{5.986258in}{3.132639in}}%
\pgfpathlineto{\pgfqpoint{5.989202in}{3.132982in}}%
\pgfpathlineto{\pgfqpoint{5.992145in}{3.130649in}}%
\pgfpathlineto{\pgfqpoint{5.998033in}{3.121812in}}%
\pgfpathlineto{\pgfqpoint{6.002314in}{3.117514in}}%
\pgfpathlineto{\pgfqpoint{6.004991in}{3.117541in}}%
\pgfpathlineto{\pgfqpoint{6.007934in}{3.120085in}}%
\pgfpathlineto{\pgfqpoint{6.019441in}{3.133168in}}%
\pgfpathlineto{\pgfqpoint{6.022117in}{3.131787in}}%
\pgfpathlineto{\pgfqpoint{6.026132in}{3.126465in}}%
\pgfpathlineto{\pgfqpoint{6.032822in}{3.117792in}}%
\pgfpathlineto{\pgfqpoint{6.035765in}{3.117429in}}%
\pgfpathlineto{\pgfqpoint{6.038709in}{3.119745in}}%
\pgfpathlineto{\pgfqpoint{6.044329in}{3.128180in}}%
\pgfpathlineto{\pgfqpoint{6.048611in}{3.132770in}}%
\pgfpathlineto{\pgfqpoint{6.051287in}{3.132992in}}%
\pgfpathlineto{\pgfqpoint{6.054230in}{3.130682in}}%
\pgfpathlineto{\pgfqpoint{6.059850in}{3.122249in}}%
\pgfpathlineto{\pgfqpoint{6.064132in}{3.117652in}}%
\pgfpathlineto{\pgfqpoint{6.066808in}{3.117424in}}%
\pgfpathlineto{\pgfqpoint{6.069752in}{3.119729in}}%
\pgfpathlineto{\pgfqpoint{6.075371in}{3.128159in}}%
\pgfpathlineto{\pgfqpoint{6.079653in}{3.132763in}}%
\pgfpathlineto{\pgfqpoint{6.082329in}{3.132997in}}%
\pgfpathlineto{\pgfqpoint{6.085273in}{3.130698in}}%
\pgfpathlineto{\pgfqpoint{6.090893in}{3.122270in}}%
\pgfpathlineto{\pgfqpoint{6.095174in}{3.117659in}}%
\pgfpathlineto{\pgfqpoint{6.097850in}{3.117420in}}%
\pgfpathlineto{\pgfqpoint{6.100794in}{3.119712in}}%
\pgfpathlineto{\pgfqpoint{6.106414in}{3.128138in}}%
\pgfpathlineto{\pgfqpoint{6.110696in}{3.132756in}}%
\pgfpathlineto{\pgfqpoint{6.113372in}{3.133002in}}%
\pgfpathlineto{\pgfqpoint{6.116315in}{3.130715in}}%
\pgfpathlineto{\pgfqpoint{6.121935in}{3.122291in}}%
\pgfpathlineto{\pgfqpoint{6.126217in}{3.117667in}}%
\pgfpathlineto{\pgfqpoint{6.128893in}{3.117415in}}%
\pgfpathlineto{\pgfqpoint{6.131837in}{3.119696in}}%
\pgfpathlineto{\pgfqpoint{6.137456in}{3.128117in}}%
\pgfpathlineto{\pgfqpoint{6.141738in}{3.132748in}}%
\pgfpathlineto{\pgfqpoint{6.144414in}{3.133006in}}%
\pgfpathlineto{\pgfqpoint{6.147358in}{3.130731in}}%
\pgfpathlineto{\pgfqpoint{6.152978in}{3.122313in}}%
\pgfpathlineto{\pgfqpoint{6.157259in}{3.117674in}}%
\pgfpathlineto{\pgfqpoint{6.159935in}{3.117410in}}%
\pgfpathlineto{\pgfqpoint{6.162879in}{3.119679in}}%
\pgfpathlineto{\pgfqpoint{6.168499in}{3.128096in}}%
\pgfpathlineto{\pgfqpoint{6.173048in}{3.132871in}}%
\pgfpathlineto{\pgfqpoint{6.175724in}{3.132911in}}%
\pgfpathlineto{\pgfqpoint{6.178668in}{3.130429in}}%
\pgfpathlineto{\pgfqpoint{6.190443in}{3.117275in}}%
\pgfpathlineto{\pgfqpoint{6.193119in}{3.118810in}}%
\pgfpathlineto{\pgfqpoint{6.197133in}{3.124258in}}%
\pgfpathlineto{\pgfqpoint{6.203556in}{3.132580in}}%
\pgfpathlineto{\pgfqpoint{6.206499in}{3.133016in}}%
\pgfpathlineto{\pgfqpoint{6.209443in}{3.130764in}}%
\pgfpathlineto{\pgfqpoint{6.214795in}{3.122762in}}%
\pgfpathlineto{\pgfqpoint{6.219344in}{3.117689in}}%
\pgfpathlineto{\pgfqpoint{6.222020in}{3.117401in}}%
\pgfpathlineto{\pgfqpoint{6.224964in}{3.119647in}}%
\pgfpathlineto{\pgfqpoint{6.230316in}{3.127646in}}%
\pgfpathlineto{\pgfqpoint{6.234866in}{3.132726in}}%
\pgfpathlineto{\pgfqpoint{6.237542in}{3.133020in}}%
\pgfpathlineto{\pgfqpoint{6.240485in}{3.130780in}}%
\pgfpathlineto{\pgfqpoint{6.245838in}{3.122784in}}%
\pgfpathlineto{\pgfqpoint{6.250387in}{3.117697in}}%
\pgfpathlineto{\pgfqpoint{6.253063in}{3.117396in}}%
\pgfpathlineto{\pgfqpoint{6.256007in}{3.119631in}}%
\pgfpathlineto{\pgfqpoint{6.261359in}{3.127624in}}%
\pgfpathlineto{\pgfqpoint{6.265908in}{3.132718in}}%
\pgfpathlineto{\pgfqpoint{6.268584in}{3.133025in}}%
\pgfpathlineto{\pgfqpoint{6.271528in}{3.130796in}}%
\pgfpathlineto{\pgfqpoint{6.276880in}{3.122805in}}%
\pgfpathlineto{\pgfqpoint{6.281429in}{3.117704in}}%
\pgfpathlineto{\pgfqpoint{6.284105in}{3.117392in}}%
\pgfpathlineto{\pgfqpoint{6.287049in}{3.119614in}}%
\pgfpathlineto{\pgfqpoint{6.292401in}{3.127603in}}%
\pgfpathlineto{\pgfqpoint{6.296951in}{3.132711in}}%
\pgfpathlineto{\pgfqpoint{6.299627in}{3.133029in}}%
\pgfpathlineto{\pgfqpoint{6.302570in}{3.130813in}}%
\pgfpathlineto{\pgfqpoint{6.307923in}{3.122827in}}%
\pgfpathlineto{\pgfqpoint{6.312472in}{3.117712in}}%
\pgfpathlineto{\pgfqpoint{6.315148in}{3.117387in}}%
\pgfpathlineto{\pgfqpoint{6.318092in}{3.119598in}}%
\pgfpathlineto{\pgfqpoint{6.323444in}{3.127581in}}%
\pgfpathlineto{\pgfqpoint{6.327993in}{3.132703in}}%
\pgfpathlineto{\pgfqpoint{6.330669in}{3.133034in}}%
\pgfpathlineto{\pgfqpoint{6.333613in}{3.130829in}}%
\pgfpathlineto{\pgfqpoint{6.338965in}{3.122849in}}%
\pgfpathlineto{\pgfqpoint{6.343514in}{3.117720in}}%
\pgfpathlineto{\pgfqpoint{6.346190in}{3.117383in}}%
\pgfpathlineto{\pgfqpoint{6.349134in}{3.119582in}}%
\pgfpathlineto{\pgfqpoint{6.354486in}{3.127559in}}%
\pgfpathlineto{\pgfqpoint{6.359036in}{3.132695in}}%
\pgfpathlineto{\pgfqpoint{6.361712in}{3.133038in}}%
\pgfpathlineto{\pgfqpoint{6.364655in}{3.130845in}}%
\pgfpathlineto{\pgfqpoint{6.370008in}{3.122870in}}%
\pgfpathlineto{\pgfqpoint{6.374557in}{3.117728in}}%
\pgfpathlineto{\pgfqpoint{6.377233in}{3.117379in}}%
\pgfpathlineto{\pgfqpoint{6.380177in}{3.119566in}}%
\pgfpathlineto{\pgfqpoint{6.385261in}{3.127122in}}%
\pgfpathlineto{\pgfqpoint{6.390078in}{3.132687in}}%
\pgfpathlineto{\pgfqpoint{6.392754in}{3.133042in}}%
\pgfpathlineto{\pgfqpoint{6.395698in}{3.130861in}}%
\pgfpathlineto{\pgfqpoint{6.400782in}{3.123308in}}%
\pgfpathlineto{\pgfqpoint{6.405599in}{3.117735in}}%
\pgfpathlineto{\pgfqpoint{6.408276in}{3.117374in}}%
\pgfpathlineto{\pgfqpoint{6.411219in}{3.119550in}}%
\pgfpathlineto{\pgfqpoint{6.416304in}{3.127100in}}%
\pgfpathlineto{\pgfqpoint{6.421121in}{3.132679in}}%
\pgfpathlineto{\pgfqpoint{6.423797in}{3.133047in}}%
\pgfpathlineto{\pgfqpoint{6.426740in}{3.130877in}}%
\pgfpathlineto{\pgfqpoint{6.431825in}{3.123330in}}%
\pgfpathlineto{\pgfqpoint{6.436642in}{3.117743in}}%
\pgfpathlineto{\pgfqpoint{6.439318in}{3.117370in}}%
\pgfpathlineto{\pgfqpoint{6.441994in}{3.119239in}}%
\pgfpathlineto{\pgfqpoint{6.446543in}{3.125800in}}%
\pgfpathlineto{\pgfqpoint{6.451896in}{3.132509in}}%
\pgfpathlineto{\pgfqpoint{6.454839in}{3.133051in}}%
\pgfpathlineto{\pgfqpoint{6.457515in}{3.131187in}}%
\pgfpathlineto{\pgfqpoint{6.462065in}{3.124631in}}%
\pgfpathlineto{\pgfqpoint{6.467417in}{3.117915in}}%
\pgfpathlineto{\pgfqpoint{6.470361in}{3.117366in}}%
\pgfpathlineto{\pgfqpoint{6.473037in}{3.119224in}}%
\pgfpathlineto{\pgfqpoint{6.477586in}{3.125777in}}%
\pgfpathlineto{\pgfqpoint{6.482938in}{3.132500in}}%
\pgfpathlineto{\pgfqpoint{6.485882in}{3.133055in}}%
\pgfpathlineto{\pgfqpoint{6.488558in}{3.131202in}}%
\pgfpathlineto{\pgfqpoint{6.493107in}{3.124653in}}%
\pgfpathlineto{\pgfqpoint{6.498459in}{3.117924in}}%
\pgfpathlineto{\pgfqpoint{6.501403in}{3.117362in}}%
\pgfpathlineto{\pgfqpoint{6.504079in}{3.119209in}}%
\pgfpathlineto{\pgfqpoint{6.508628in}{3.125754in}}%
\pgfpathlineto{\pgfqpoint{6.513981in}{3.132490in}}%
\pgfpathlineto{\pgfqpoint{6.516924in}{3.133059in}}%
\pgfpathlineto{\pgfqpoint{6.519600in}{3.131217in}}%
\pgfpathlineto{\pgfqpoint{6.524150in}{3.124676in}}%
\pgfpathlineto{\pgfqpoint{6.529502in}{3.117933in}}%
\pgfpathlineto{\pgfqpoint{6.532446in}{3.117358in}}%
\pgfpathlineto{\pgfqpoint{6.535122in}{3.119194in}}%
\pgfpathlineto{\pgfqpoint{6.539671in}{3.125732in}}%
\pgfpathlineto{\pgfqpoint{6.545023in}{3.132481in}}%
\pgfpathlineto{\pgfqpoint{6.547967in}{3.133063in}}%
\pgfpathlineto{\pgfqpoint{6.550643in}{3.131232in}}%
\pgfpathlineto{\pgfqpoint{6.555192in}{3.124699in}}%
\pgfpathlineto{\pgfqpoint{6.560544in}{3.117943in}}%
\pgfpathlineto{\pgfqpoint{6.563488in}{3.117354in}}%
\pgfpathlineto{\pgfqpoint{6.566164in}{3.119180in}}%
\pgfpathlineto{\pgfqpoint{6.570713in}{3.125709in}}%
\pgfpathlineto{\pgfqpoint{6.576333in}{3.132639in}}%
\pgfpathlineto{\pgfqpoint{6.579277in}{3.132982in}}%
\pgfpathlineto{\pgfqpoint{6.582221in}{3.130649in}}%
\pgfpathlineto{\pgfqpoint{6.588108in}{3.121812in}}%
\pgfpathlineto{\pgfqpoint{6.592390in}{3.117514in}}%
\pgfpathlineto{\pgfqpoint{6.595066in}{3.117541in}}%
\pgfpathlineto{\pgfqpoint{6.598009in}{3.120085in}}%
\pgfpathlineto{\pgfqpoint{6.609517in}{3.133168in}}%
\pgfpathlineto{\pgfqpoint{6.612193in}{3.131787in}}%
\pgfpathlineto{\pgfqpoint{6.616207in}{3.126465in}}%
\pgfpathlineto{\pgfqpoint{6.622897in}{3.117792in}}%
\pgfpathlineto{\pgfqpoint{6.625841in}{3.117429in}}%
\pgfpathlineto{\pgfqpoint{6.628784in}{3.119745in}}%
\pgfpathlineto{\pgfqpoint{6.634404in}{3.128180in}}%
\pgfpathlineto{\pgfqpoint{6.638686in}{3.132770in}}%
\pgfpathlineto{\pgfqpoint{6.641362in}{3.132992in}}%
\pgfpathlineto{\pgfqpoint{6.644306in}{3.130682in}}%
\pgfpathlineto{\pgfqpoint{6.649925in}{3.122249in}}%
\pgfpathlineto{\pgfqpoint{6.654207in}{3.117652in}}%
\pgfpathlineto{\pgfqpoint{6.656883in}{3.117424in}}%
\pgfpathlineto{\pgfqpoint{6.659827in}{3.119729in}}%
\pgfpathlineto{\pgfqpoint{6.663306in}{3.124778in}}%
\pgfpathlineto{\pgfqpoint{6.663306in}{3.124778in}}%
\pgfusepath{stroke}%
\end{pgfscope}%
\begin{pgfscope}%
\pgfpathrectangle{\pgfqpoint{0.467797in}{2.292089in}}{\pgfqpoint{6.490533in}{1.666241in}}%
\pgfusepath{clip}%
\pgfsetrectcap%
\pgfsetroundjoin%
\pgfsetlinewidth{1.505625pt}%
\definecolor{currentstroke}{rgb}{0.737255,0.741176,0.133333}%
\pgfsetstrokecolor{currentstroke}%
\pgfsetdash{}{0pt}%
\pgfpathmoveto{\pgfqpoint{0.762821in}{3.125209in}}%
\pgfpathlineto{\pgfqpoint{0.768441in}{3.132371in}}%
\pgfpathlineto{\pgfqpoint{0.771384in}{3.132865in}}%
\pgfpathlineto{\pgfqpoint{0.774060in}{3.130917in}}%
\pgfpathlineto{\pgfqpoint{0.778877in}{3.123848in}}%
\pgfpathlineto{\pgfqpoint{0.783962in}{3.117862in}}%
\pgfpathlineto{\pgfqpoint{0.786638in}{3.117569in}}%
\pgfpathlineto{\pgfqpoint{0.789582in}{3.119858in}}%
\pgfpathlineto{\pgfqpoint{0.795469in}{3.128683in}}%
\pgfpathlineto{\pgfqpoint{0.799483in}{3.132712in}}%
\pgfpathlineto{\pgfqpoint{0.802159in}{3.132730in}}%
\pgfpathlineto{\pgfqpoint{0.805103in}{3.130182in}}%
\pgfpathlineto{\pgfqpoint{0.816342in}{3.117416in}}%
\pgfpathlineto{\pgfqpoint{0.819019in}{3.118825in}}%
\pgfpathlineto{\pgfqpoint{0.823033in}{3.124211in}}%
\pgfpathlineto{\pgfqpoint{0.829455in}{3.132483in}}%
\pgfpathlineto{\pgfqpoint{0.832131in}{3.132891in}}%
\pgfpathlineto{\pgfqpoint{0.834807in}{3.131010in}}%
\pgfpathlineto{\pgfqpoint{0.839357in}{3.124412in}}%
\pgfpathlineto{\pgfqpoint{0.844709in}{3.117910in}}%
\pgfpathlineto{\pgfqpoint{0.847385in}{3.117542in}}%
\pgfpathlineto{\pgfqpoint{0.850061in}{3.119458in}}%
\pgfpathlineto{\pgfqpoint{0.854878in}{3.126507in}}%
\pgfpathlineto{\pgfqpoint{0.859963in}{3.132535in}}%
\pgfpathlineto{\pgfqpoint{0.862639in}{3.132863in}}%
\pgfpathlineto{\pgfqpoint{0.865315in}{3.130911in}}%
\pgfpathlineto{\pgfqpoint{0.870132in}{3.123839in}}%
\pgfpathlineto{\pgfqpoint{0.875216in}{3.117859in}}%
\pgfpathlineto{\pgfqpoint{0.877892in}{3.117571in}}%
\pgfpathlineto{\pgfqpoint{0.880836in}{3.119864in}}%
\pgfpathlineto{\pgfqpoint{0.886723in}{3.128691in}}%
\pgfpathlineto{\pgfqpoint{0.890737in}{3.132715in}}%
\pgfpathlineto{\pgfqpoint{0.893414in}{3.132727in}}%
\pgfpathlineto{\pgfqpoint{0.896357in}{3.130175in}}%
\pgfpathlineto{\pgfqpoint{0.907597in}{3.117416in}}%
\pgfpathlineto{\pgfqpoint{0.910273in}{3.118830in}}%
\pgfpathlineto{\pgfqpoint{0.914287in}{3.124220in}}%
\pgfpathlineto{\pgfqpoint{0.920710in}{3.132486in}}%
\pgfpathlineto{\pgfqpoint{0.923386in}{3.132889in}}%
\pgfpathlineto{\pgfqpoint{0.926062in}{3.131005in}}%
\pgfpathlineto{\pgfqpoint{0.930611in}{3.124403in}}%
\pgfpathlineto{\pgfqpoint{0.935963in}{3.117907in}}%
\pgfpathlineto{\pgfqpoint{0.938639in}{3.117543in}}%
\pgfpathlineto{\pgfqpoint{0.941315in}{3.119464in}}%
\pgfpathlineto{\pgfqpoint{0.946132in}{3.126516in}}%
\pgfpathlineto{\pgfqpoint{0.951217in}{3.132538in}}%
\pgfpathlineto{\pgfqpoint{0.953893in}{3.132861in}}%
\pgfpathlineto{\pgfqpoint{0.956569in}{3.130905in}}%
\pgfpathlineto{\pgfqpoint{0.961386in}{3.123831in}}%
\pgfpathlineto{\pgfqpoint{0.966470in}{3.117856in}}%
\pgfpathlineto{\pgfqpoint{0.969147in}{3.117573in}}%
\pgfpathlineto{\pgfqpoint{0.972090in}{3.119871in}}%
\pgfpathlineto{\pgfqpoint{0.977978in}{3.128699in}}%
\pgfpathlineto{\pgfqpoint{0.981992in}{3.132717in}}%
\pgfpathlineto{\pgfqpoint{0.984668in}{3.132725in}}%
\pgfpathlineto{\pgfqpoint{0.987611in}{3.130168in}}%
\pgfpathlineto{\pgfqpoint{0.998851in}{3.117417in}}%
\pgfpathlineto{\pgfqpoint{1.001527in}{3.118835in}}%
\pgfpathlineto{\pgfqpoint{1.005541in}{3.124229in}}%
\pgfpathlineto{\pgfqpoint{1.011964in}{3.132489in}}%
\pgfpathlineto{\pgfqpoint{1.014640in}{3.132887in}}%
\pgfpathlineto{\pgfqpoint{1.017316in}{3.130999in}}%
\pgfpathlineto{\pgfqpoint{1.022133in}{3.123967in}}%
\pgfpathlineto{\pgfqpoint{1.027217in}{3.117904in}}%
\pgfpathlineto{\pgfqpoint{1.029894in}{3.117545in}}%
\pgfpathlineto{\pgfqpoint{1.032570in}{3.119470in}}%
\pgfpathlineto{\pgfqpoint{1.037387in}{3.126525in}}%
\pgfpathlineto{\pgfqpoint{1.042471in}{3.132541in}}%
\pgfpathlineto{\pgfqpoint{1.045147in}{3.132859in}}%
\pgfpathlineto{\pgfqpoint{1.047823in}{3.130899in}}%
\pgfpathlineto{\pgfqpoint{1.052640in}{3.123822in}}%
\pgfpathlineto{\pgfqpoint{1.057725in}{3.117853in}}%
\pgfpathlineto{\pgfqpoint{1.060401in}{3.117575in}}%
\pgfpathlineto{\pgfqpoint{1.063344in}{3.119877in}}%
\pgfpathlineto{\pgfqpoint{1.069232in}{3.128707in}}%
\pgfpathlineto{\pgfqpoint{1.073246in}{3.132720in}}%
\pgfpathlineto{\pgfqpoint{1.075922in}{3.132723in}}%
\pgfpathlineto{\pgfqpoint{1.078866in}{3.130162in}}%
\pgfpathlineto{\pgfqpoint{1.090105in}{3.117417in}}%
\pgfpathlineto{\pgfqpoint{1.092781in}{3.118840in}}%
\pgfpathlineto{\pgfqpoint{1.096795in}{3.124237in}}%
\pgfpathlineto{\pgfqpoint{1.103218in}{3.132492in}}%
\pgfpathlineto{\pgfqpoint{1.105894in}{3.132886in}}%
\pgfpathlineto{\pgfqpoint{1.108570in}{3.130993in}}%
\pgfpathlineto{\pgfqpoint{1.113387in}{3.123958in}}%
\pgfpathlineto{\pgfqpoint{1.118472in}{3.117900in}}%
\pgfpathlineto{\pgfqpoint{1.121148in}{3.117547in}}%
\pgfpathlineto{\pgfqpoint{1.123824in}{3.119476in}}%
\pgfpathlineto{\pgfqpoint{1.128641in}{3.126534in}}%
\pgfpathlineto{\pgfqpoint{1.133725in}{3.132544in}}%
\pgfpathlineto{\pgfqpoint{1.136401in}{3.132858in}}%
\pgfpathlineto{\pgfqpoint{1.139078in}{3.130893in}}%
\pgfpathlineto{\pgfqpoint{1.143894in}{3.123813in}}%
\pgfpathlineto{\pgfqpoint{1.148979in}{3.117850in}}%
\pgfpathlineto{\pgfqpoint{1.151655in}{3.117576in}}%
\pgfpathlineto{\pgfqpoint{1.154599in}{3.119884in}}%
\pgfpathlineto{\pgfqpoint{1.160486in}{3.128715in}}%
\pgfpathlineto{\pgfqpoint{1.164500in}{3.132722in}}%
\pgfpathlineto{\pgfqpoint{1.167176in}{3.132720in}}%
\pgfpathlineto{\pgfqpoint{1.170120in}{3.130155in}}%
\pgfpathlineto{\pgfqpoint{1.181360in}{3.117418in}}%
\pgfpathlineto{\pgfqpoint{1.184036in}{3.118845in}}%
\pgfpathlineto{\pgfqpoint{1.188050in}{3.124246in}}%
\pgfpathlineto{\pgfqpoint{1.194472in}{3.132495in}}%
\pgfpathlineto{\pgfqpoint{1.197148in}{3.132884in}}%
\pgfpathlineto{\pgfqpoint{1.199824in}{3.130987in}}%
\pgfpathlineto{\pgfqpoint{1.204641in}{3.123949in}}%
\pgfpathlineto{\pgfqpoint{1.209726in}{3.117897in}}%
\pgfpathlineto{\pgfqpoint{1.212402in}{3.117549in}}%
\pgfpathlineto{\pgfqpoint{1.215078in}{3.119482in}}%
\pgfpathlineto{\pgfqpoint{1.219895in}{3.126542in}}%
\pgfpathlineto{\pgfqpoint{1.224980in}{3.132547in}}%
\pgfpathlineto{\pgfqpoint{1.227656in}{3.132856in}}%
\pgfpathlineto{\pgfqpoint{1.230599in}{3.130582in}}%
\pgfpathlineto{\pgfqpoint{1.236219in}{3.122154in}}%
\pgfpathlineto{\pgfqpoint{1.240501in}{3.117715in}}%
\pgfpathlineto{\pgfqpoint{1.243177in}{3.117681in}}%
\pgfpathlineto{\pgfqpoint{1.246121in}{3.120214in}}%
\pgfpathlineto{\pgfqpoint{1.257360in}{3.133005in}}%
\pgfpathlineto{\pgfqpoint{1.260036in}{3.131611in}}%
\pgfpathlineto{\pgfqpoint{1.264050in}{3.126237in}}%
\pgfpathlineto{\pgfqpoint{1.270473in}{3.117947in}}%
\pgfpathlineto{\pgfqpoint{1.273149in}{3.117523in}}%
\pgfpathlineto{\pgfqpoint{1.275825in}{3.119389in}}%
\pgfpathlineto{\pgfqpoint{1.280374in}{3.125978in}}%
\pgfpathlineto{\pgfqpoint{1.285727in}{3.132499in}}%
\pgfpathlineto{\pgfqpoint{1.288403in}{3.132883in}}%
\pgfpathlineto{\pgfqpoint{1.291079in}{3.130981in}}%
\pgfpathlineto{\pgfqpoint{1.295896in}{3.123940in}}%
\pgfpathlineto{\pgfqpoint{1.300980in}{3.117894in}}%
\pgfpathlineto{\pgfqpoint{1.303656in}{3.117550in}}%
\pgfpathlineto{\pgfqpoint{1.306332in}{3.119488in}}%
\pgfpathlineto{\pgfqpoint{1.311149in}{3.126551in}}%
\pgfpathlineto{\pgfqpoint{1.316234in}{3.132550in}}%
\pgfpathlineto{\pgfqpoint{1.318910in}{3.132854in}}%
\pgfpathlineto{\pgfqpoint{1.321854in}{3.130575in}}%
\pgfpathlineto{\pgfqpoint{1.327473in}{3.122146in}}%
\pgfpathlineto{\pgfqpoint{1.331755in}{3.117712in}}%
\pgfpathlineto{\pgfqpoint{1.334431in}{3.117684in}}%
\pgfpathlineto{\pgfqpoint{1.337375in}{3.120221in}}%
\pgfpathlineto{\pgfqpoint{1.348614in}{3.133004in}}%
\pgfpathlineto{\pgfqpoint{1.351291in}{3.131606in}}%
\pgfpathlineto{\pgfqpoint{1.355305in}{3.126228in}}%
\pgfpathlineto{\pgfqpoint{1.361727in}{3.117944in}}%
\pgfpathlineto{\pgfqpoint{1.364403in}{3.117525in}}%
\pgfpathlineto{\pgfqpoint{1.367079in}{3.119395in}}%
\pgfpathlineto{\pgfqpoint{1.371629in}{3.125987in}}%
\pgfpathlineto{\pgfqpoint{1.376981in}{3.132502in}}%
\pgfpathlineto{\pgfqpoint{1.379657in}{3.132881in}}%
\pgfpathlineto{\pgfqpoint{1.382333in}{3.130975in}}%
\pgfpathlineto{\pgfqpoint{1.387150in}{3.123931in}}%
\pgfpathlineto{\pgfqpoint{1.392235in}{3.117891in}}%
\pgfpathlineto{\pgfqpoint{1.394911in}{3.117552in}}%
\pgfpathlineto{\pgfqpoint{1.397587in}{3.119494in}}%
\pgfpathlineto{\pgfqpoint{1.402404in}{3.126560in}}%
\pgfpathlineto{\pgfqpoint{1.407488in}{3.132553in}}%
\pgfpathlineto{\pgfqpoint{1.410164in}{3.132852in}}%
\pgfpathlineto{\pgfqpoint{1.413108in}{3.130569in}}%
\pgfpathlineto{\pgfqpoint{1.418728in}{3.122138in}}%
\pgfpathlineto{\pgfqpoint{1.423009in}{3.117710in}}%
\pgfpathlineto{\pgfqpoint{1.425685in}{3.117686in}}%
\pgfpathlineto{\pgfqpoint{1.428629in}{3.120228in}}%
\pgfpathlineto{\pgfqpoint{1.439869in}{3.133004in}}%
\pgfpathlineto{\pgfqpoint{1.442545in}{3.131601in}}%
\pgfpathlineto{\pgfqpoint{1.446559in}{3.126219in}}%
\pgfpathlineto{\pgfqpoint{1.452981in}{3.117940in}}%
\pgfpathlineto{\pgfqpoint{1.455658in}{3.117526in}}%
\pgfpathlineto{\pgfqpoint{1.458334in}{3.119401in}}%
\pgfpathlineto{\pgfqpoint{1.462883in}{3.125996in}}%
\pgfpathlineto{\pgfqpoint{1.468235in}{3.132505in}}%
\pgfpathlineto{\pgfqpoint{1.470911in}{3.132879in}}%
\pgfpathlineto{\pgfqpoint{1.473587in}{3.130969in}}%
\pgfpathlineto{\pgfqpoint{1.478404in}{3.123923in}}%
\pgfpathlineto{\pgfqpoint{1.483489in}{3.117888in}}%
\pgfpathlineto{\pgfqpoint{1.486165in}{3.117554in}}%
\pgfpathlineto{\pgfqpoint{1.488841in}{3.119500in}}%
\pgfpathlineto{\pgfqpoint{1.493658in}{3.126569in}}%
\pgfpathlineto{\pgfqpoint{1.498742in}{3.132556in}}%
\pgfpathlineto{\pgfqpoint{1.501418in}{3.132850in}}%
\pgfpathlineto{\pgfqpoint{1.504362in}{3.130563in}}%
\pgfpathlineto{\pgfqpoint{1.510250in}{3.121738in}}%
\pgfpathlineto{\pgfqpoint{1.514264in}{3.117707in}}%
\pgfpathlineto{\pgfqpoint{1.516940in}{3.117688in}}%
\pgfpathlineto{\pgfqpoint{1.519883in}{3.120235in}}%
\pgfpathlineto{\pgfqpoint{1.531123in}{3.133003in}}%
\pgfpathlineto{\pgfqpoint{1.533799in}{3.131596in}}%
\pgfpathlineto{\pgfqpoint{1.537813in}{3.126210in}}%
\pgfpathlineto{\pgfqpoint{1.544236in}{3.117937in}}%
\pgfpathlineto{\pgfqpoint{1.546912in}{3.117528in}}%
\pgfpathlineto{\pgfqpoint{1.549588in}{3.119407in}}%
\pgfpathlineto{\pgfqpoint{1.554137in}{3.126005in}}%
\pgfpathlineto{\pgfqpoint{1.559489in}{3.132508in}}%
\pgfpathlineto{\pgfqpoint{1.562165in}{3.132878in}}%
\pgfpathlineto{\pgfqpoint{1.564842in}{3.130963in}}%
\pgfpathlineto{\pgfqpoint{1.569658in}{3.123914in}}%
\pgfpathlineto{\pgfqpoint{1.574743in}{3.117885in}}%
\pgfpathlineto{\pgfqpoint{1.577419in}{3.117556in}}%
\pgfpathlineto{\pgfqpoint{1.580095in}{3.119506in}}%
\pgfpathlineto{\pgfqpoint{1.584912in}{3.126577in}}%
\pgfpathlineto{\pgfqpoint{1.589997in}{3.132559in}}%
\pgfpathlineto{\pgfqpoint{1.592673in}{3.132848in}}%
\pgfpathlineto{\pgfqpoint{1.595616in}{3.130556in}}%
\pgfpathlineto{\pgfqpoint{1.601504in}{3.121730in}}%
\pgfpathlineto{\pgfqpoint{1.605518in}{3.117705in}}%
\pgfpathlineto{\pgfqpoint{1.608194in}{3.117691in}}%
\pgfpathlineto{\pgfqpoint{1.611138in}{3.120242in}}%
\pgfpathlineto{\pgfqpoint{1.622377in}{3.133003in}}%
\pgfpathlineto{\pgfqpoint{1.625053in}{3.131590in}}%
\pgfpathlineto{\pgfqpoint{1.629067in}{3.126201in}}%
\pgfpathlineto{\pgfqpoint{1.635490in}{3.117934in}}%
\pgfpathlineto{\pgfqpoint{1.638166in}{3.117529in}}%
\pgfpathlineto{\pgfqpoint{1.640842in}{3.119413in}}%
\pgfpathlineto{\pgfqpoint{1.645392in}{3.126013in}}%
\pgfpathlineto{\pgfqpoint{1.650744in}{3.132511in}}%
\pgfpathlineto{\pgfqpoint{1.653420in}{3.132876in}}%
\pgfpathlineto{\pgfqpoint{1.656096in}{3.130957in}}%
\pgfpathlineto{\pgfqpoint{1.660913in}{3.123905in}}%
\pgfpathlineto{\pgfqpoint{1.665997in}{3.117882in}}%
\pgfpathlineto{\pgfqpoint{1.668673in}{3.117557in}}%
\pgfpathlineto{\pgfqpoint{1.671349in}{3.119512in}}%
\pgfpathlineto{\pgfqpoint{1.676166in}{3.126586in}}%
\pgfpathlineto{\pgfqpoint{1.681251in}{3.132562in}}%
\pgfpathlineto{\pgfqpoint{1.683927in}{3.132847in}}%
\pgfpathlineto{\pgfqpoint{1.686871in}{3.130550in}}%
\pgfpathlineto{\pgfqpoint{1.692758in}{3.121722in}}%
\pgfpathlineto{\pgfqpoint{1.696772in}{3.117702in}}%
\pgfpathlineto{\pgfqpoint{1.699448in}{3.117693in}}%
\pgfpathlineto{\pgfqpoint{1.702392in}{3.120249in}}%
\pgfpathlineto{\pgfqpoint{1.713631in}{3.133002in}}%
\pgfpathlineto{\pgfqpoint{1.716308in}{3.131585in}}%
\pgfpathlineto{\pgfqpoint{1.720322in}{3.126192in}}%
\pgfpathlineto{\pgfqpoint{1.726744in}{3.117931in}}%
\pgfpathlineto{\pgfqpoint{1.729420in}{3.117531in}}%
\pgfpathlineto{\pgfqpoint{1.732096in}{3.119419in}}%
\pgfpathlineto{\pgfqpoint{1.736913in}{3.126450in}}%
\pgfpathlineto{\pgfqpoint{1.741998in}{3.132515in}}%
\pgfpathlineto{\pgfqpoint{1.744674in}{3.132874in}}%
\pgfpathlineto{\pgfqpoint{1.747350in}{3.130951in}}%
\pgfpathlineto{\pgfqpoint{1.752167in}{3.123896in}}%
\pgfpathlineto{\pgfqpoint{1.757252in}{3.117879in}}%
\pgfpathlineto{\pgfqpoint{1.759928in}{3.117559in}}%
\pgfpathlineto{\pgfqpoint{1.762604in}{3.119518in}}%
\pgfpathlineto{\pgfqpoint{1.767421in}{3.126595in}}%
\pgfpathlineto{\pgfqpoint{1.772505in}{3.132565in}}%
\pgfpathlineto{\pgfqpoint{1.775181in}{3.132845in}}%
\pgfpathlineto{\pgfqpoint{1.778125in}{3.130543in}}%
\pgfpathlineto{\pgfqpoint{1.784012in}{3.121714in}}%
\pgfpathlineto{\pgfqpoint{1.788026in}{3.117700in}}%
\pgfpathlineto{\pgfqpoint{1.790703in}{3.117696in}}%
\pgfpathlineto{\pgfqpoint{1.793646in}{3.120256in}}%
\pgfpathlineto{\pgfqpoint{1.804886in}{3.133002in}}%
\pgfpathlineto{\pgfqpoint{1.807562in}{3.131580in}}%
\pgfpathlineto{\pgfqpoint{1.811576in}{3.126184in}}%
\pgfpathlineto{\pgfqpoint{1.817999in}{3.117927in}}%
\pgfpathlineto{\pgfqpoint{1.820675in}{3.117533in}}%
\pgfpathlineto{\pgfqpoint{1.823351in}{3.119425in}}%
\pgfpathlineto{\pgfqpoint{1.828168in}{3.126459in}}%
\pgfpathlineto{\pgfqpoint{1.833252in}{3.132518in}}%
\pgfpathlineto{\pgfqpoint{1.835928in}{3.132872in}}%
\pgfpathlineto{\pgfqpoint{1.838604in}{3.130944in}}%
\pgfpathlineto{\pgfqpoint{1.843421in}{3.123888in}}%
\pgfpathlineto{\pgfqpoint{1.848506in}{3.117876in}}%
\pgfpathlineto{\pgfqpoint{1.851182in}{3.117561in}}%
\pgfpathlineto{\pgfqpoint{1.853858in}{3.119524in}}%
\pgfpathlineto{\pgfqpoint{1.858675in}{3.126604in}}%
\pgfpathlineto{\pgfqpoint{1.863759in}{3.132568in}}%
\pgfpathlineto{\pgfqpoint{1.866436in}{3.132843in}}%
\pgfpathlineto{\pgfqpoint{1.869379in}{3.130537in}}%
\pgfpathlineto{\pgfqpoint{1.875267in}{3.121706in}}%
\pgfpathlineto{\pgfqpoint{1.879281in}{3.117697in}}%
\pgfpathlineto{\pgfqpoint{1.881957in}{3.117698in}}%
\pgfpathlineto{\pgfqpoint{1.884901in}{3.120262in}}%
\pgfpathlineto{\pgfqpoint{1.896140in}{3.133001in}}%
\pgfpathlineto{\pgfqpoint{1.898816in}{3.131575in}}%
\pgfpathlineto{\pgfqpoint{1.902830in}{3.126175in}}%
\pgfpathlineto{\pgfqpoint{1.909253in}{3.117924in}}%
\pgfpathlineto{\pgfqpoint{1.911929in}{3.117534in}}%
\pgfpathlineto{\pgfqpoint{1.914605in}{3.119431in}}%
\pgfpathlineto{\pgfqpoint{1.919422in}{3.126468in}}%
\pgfpathlineto{\pgfqpoint{1.924506in}{3.132521in}}%
\pgfpathlineto{\pgfqpoint{1.927183in}{3.132871in}}%
\pgfpathlineto{\pgfqpoint{1.929859in}{3.130938in}}%
\pgfpathlineto{\pgfqpoint{1.934676in}{3.123879in}}%
\pgfpathlineto{\pgfqpoint{1.939760in}{3.117873in}}%
\pgfpathlineto{\pgfqpoint{1.942436in}{3.117563in}}%
\pgfpathlineto{\pgfqpoint{1.945112in}{3.119531in}}%
\pgfpathlineto{\pgfqpoint{1.949929in}{3.126612in}}%
\pgfpathlineto{\pgfqpoint{1.955014in}{3.132571in}}%
\pgfpathlineto{\pgfqpoint{1.957690in}{3.132841in}}%
\pgfpathlineto{\pgfqpoint{1.960634in}{3.130530in}}%
\pgfpathlineto{\pgfqpoint{1.966521in}{3.121698in}}%
\pgfpathlineto{\pgfqpoint{1.970535in}{3.117695in}}%
\pgfpathlineto{\pgfqpoint{1.973211in}{3.117700in}}%
\pgfpathlineto{\pgfqpoint{1.976155in}{3.120269in}}%
\pgfpathlineto{\pgfqpoint{1.987394in}{3.133001in}}%
\pgfpathlineto{\pgfqpoint{1.990070in}{3.131570in}}%
\pgfpathlineto{\pgfqpoint{1.994084in}{3.126166in}}%
\pgfpathlineto{\pgfqpoint{2.000507in}{3.117921in}}%
\pgfpathlineto{\pgfqpoint{2.003183in}{3.117536in}}%
\pgfpathlineto{\pgfqpoint{2.005859in}{3.119437in}}%
\pgfpathlineto{\pgfqpoint{2.010676in}{3.126476in}}%
\pgfpathlineto{\pgfqpoint{2.015761in}{3.132524in}}%
\pgfpathlineto{\pgfqpoint{2.018437in}{3.132869in}}%
\pgfpathlineto{\pgfqpoint{2.021113in}{3.130932in}}%
\pgfpathlineto{\pgfqpoint{2.025930in}{3.123870in}}%
\pgfpathlineto{\pgfqpoint{2.031014in}{3.117870in}}%
\pgfpathlineto{\pgfqpoint{2.033690in}{3.117564in}}%
\pgfpathlineto{\pgfqpoint{2.036634in}{3.119842in}}%
\pgfpathlineto{\pgfqpoint{2.042254in}{3.128271in}}%
\pgfpathlineto{\pgfqpoint{2.046536in}{3.132706in}}%
\pgfpathlineto{\pgfqpoint{2.049212in}{3.132736in}}%
\pgfpathlineto{\pgfqpoint{2.052155in}{3.130199in}}%
\pgfpathlineto{\pgfqpoint{2.063395in}{3.117414in}}%
\pgfpathlineto{\pgfqpoint{2.066071in}{3.118812in}}%
\pgfpathlineto{\pgfqpoint{2.070085in}{3.124189in}}%
\pgfpathlineto{\pgfqpoint{2.076508in}{3.132474in}}%
\pgfpathlineto{\pgfqpoint{2.079184in}{3.132895in}}%
\pgfpathlineto{\pgfqpoint{2.081860in}{3.131025in}}%
\pgfpathlineto{\pgfqpoint{2.086409in}{3.124434in}}%
\pgfpathlineto{\pgfqpoint{2.091761in}{3.117918in}}%
\pgfpathlineto{\pgfqpoint{2.094437in}{3.117538in}}%
\pgfpathlineto{\pgfqpoint{2.097114in}{3.119443in}}%
\pgfpathlineto{\pgfqpoint{2.101930in}{3.126485in}}%
\pgfpathlineto{\pgfqpoint{2.107015in}{3.132527in}}%
\pgfpathlineto{\pgfqpoint{2.109691in}{3.132867in}}%
\pgfpathlineto{\pgfqpoint{2.112367in}{3.130926in}}%
\pgfpathlineto{\pgfqpoint{2.117184in}{3.123861in}}%
\pgfpathlineto{\pgfqpoint{2.122269in}{3.117867in}}%
\pgfpathlineto{\pgfqpoint{2.124945in}{3.117566in}}%
\pgfpathlineto{\pgfqpoint{2.127888in}{3.119848in}}%
\pgfpathlineto{\pgfqpoint{2.133508in}{3.128279in}}%
\pgfpathlineto{\pgfqpoint{2.137790in}{3.132709in}}%
\pgfpathlineto{\pgfqpoint{2.140466in}{3.132733in}}%
\pgfpathlineto{\pgfqpoint{2.143410in}{3.130192in}}%
\pgfpathlineto{\pgfqpoint{2.154649in}{3.117415in}}%
\pgfpathlineto{\pgfqpoint{2.157325in}{3.118817in}}%
\pgfpathlineto{\pgfqpoint{2.161339in}{3.124198in}}%
\pgfpathlineto{\pgfqpoint{2.167762in}{3.132478in}}%
\pgfpathlineto{\pgfqpoint{2.170438in}{3.132893in}}%
\pgfpathlineto{\pgfqpoint{2.173114in}{3.131019in}}%
\pgfpathlineto{\pgfqpoint{2.177663in}{3.124425in}}%
\pgfpathlineto{\pgfqpoint{2.183016in}{3.117915in}}%
\pgfpathlineto{\pgfqpoint{2.185692in}{3.117539in}}%
\pgfpathlineto{\pgfqpoint{2.188368in}{3.119449in}}%
\pgfpathlineto{\pgfqpoint{2.193185in}{3.126494in}}%
\pgfpathlineto{\pgfqpoint{2.198269in}{3.132530in}}%
\pgfpathlineto{\pgfqpoint{2.200945in}{3.132866in}}%
\pgfpathlineto{\pgfqpoint{2.203621in}{3.130920in}}%
\pgfpathlineto{\pgfqpoint{2.208438in}{3.123852in}}%
\pgfpathlineto{\pgfqpoint{2.213523in}{3.117864in}}%
\pgfpathlineto{\pgfqpoint{2.216199in}{3.117568in}}%
\pgfpathlineto{\pgfqpoint{2.219143in}{3.119855in}}%
\pgfpathlineto{\pgfqpoint{2.225030in}{3.128679in}}%
\pgfpathlineto{\pgfqpoint{2.229044in}{3.132711in}}%
\pgfpathlineto{\pgfqpoint{2.231720in}{3.132731in}}%
\pgfpathlineto{\pgfqpoint{2.234664in}{3.130186in}}%
\pgfpathlineto{\pgfqpoint{2.245903in}{3.117415in}}%
\pgfpathlineto{\pgfqpoint{2.248580in}{3.118822in}}%
\pgfpathlineto{\pgfqpoint{2.252594in}{3.124207in}}%
\pgfpathlineto{\pgfqpoint{2.259016in}{3.132481in}}%
\pgfpathlineto{\pgfqpoint{2.261692in}{3.132891in}}%
\pgfpathlineto{\pgfqpoint{2.264368in}{3.131013in}}%
\pgfpathlineto{\pgfqpoint{2.268918in}{3.124417in}}%
\pgfpathlineto{\pgfqpoint{2.274270in}{3.117911in}}%
\pgfpathlineto{\pgfqpoint{2.276946in}{3.117541in}}%
\pgfpathlineto{\pgfqpoint{2.279622in}{3.119455in}}%
\pgfpathlineto{\pgfqpoint{2.284439in}{3.126503in}}%
\pgfpathlineto{\pgfqpoint{2.289524in}{3.132533in}}%
\pgfpathlineto{\pgfqpoint{2.292200in}{3.132864in}}%
\pgfpathlineto{\pgfqpoint{2.294876in}{3.130914in}}%
\pgfpathlineto{\pgfqpoint{2.299693in}{3.123844in}}%
\pgfpathlineto{\pgfqpoint{2.304777in}{3.117861in}}%
\pgfpathlineto{\pgfqpoint{2.307453in}{3.117570in}}%
\pgfpathlineto{\pgfqpoint{2.310397in}{3.119861in}}%
\pgfpathlineto{\pgfqpoint{2.316284in}{3.128687in}}%
\pgfpathlineto{\pgfqpoint{2.320298in}{3.132713in}}%
\pgfpathlineto{\pgfqpoint{2.322975in}{3.132729in}}%
\pgfpathlineto{\pgfqpoint{2.325918in}{3.130179in}}%
\pgfpathlineto{\pgfqpoint{2.337158in}{3.117416in}}%
\pgfpathlineto{\pgfqpoint{2.339834in}{3.118827in}}%
\pgfpathlineto{\pgfqpoint{2.343848in}{3.124215in}}%
\pgfpathlineto{\pgfqpoint{2.350271in}{3.132484in}}%
\pgfpathlineto{\pgfqpoint{2.352947in}{3.132890in}}%
\pgfpathlineto{\pgfqpoint{2.355623in}{3.131007in}}%
\pgfpathlineto{\pgfqpoint{2.360172in}{3.124408in}}%
\pgfpathlineto{\pgfqpoint{2.365524in}{3.117908in}}%
\pgfpathlineto{\pgfqpoint{2.368200in}{3.117543in}}%
\pgfpathlineto{\pgfqpoint{2.370876in}{3.119461in}}%
\pgfpathlineto{\pgfqpoint{2.375693in}{3.126512in}}%
\pgfpathlineto{\pgfqpoint{2.380778in}{3.132536in}}%
\pgfpathlineto{\pgfqpoint{2.383454in}{3.132862in}}%
\pgfpathlineto{\pgfqpoint{2.386130in}{3.130908in}}%
\pgfpathlineto{\pgfqpoint{2.390947in}{3.123835in}}%
\pgfpathlineto{\pgfqpoint{2.396031in}{3.117858in}}%
\pgfpathlineto{\pgfqpoint{2.398708in}{3.117572in}}%
\pgfpathlineto{\pgfqpoint{2.401651in}{3.119868in}}%
\pgfpathlineto{\pgfqpoint{2.407539in}{3.128695in}}%
\pgfpathlineto{\pgfqpoint{2.411553in}{3.132716in}}%
\pgfpathlineto{\pgfqpoint{2.414229in}{3.132726in}}%
\pgfpathlineto{\pgfqpoint{2.417172in}{3.130172in}}%
\pgfpathlineto{\pgfqpoint{2.428412in}{3.117416in}}%
\pgfpathlineto{\pgfqpoint{2.431088in}{3.118832in}}%
\pgfpathlineto{\pgfqpoint{2.435102in}{3.124224in}}%
\pgfpathlineto{\pgfqpoint{2.441525in}{3.132487in}}%
\pgfpathlineto{\pgfqpoint{2.444201in}{3.132888in}}%
\pgfpathlineto{\pgfqpoint{2.446877in}{3.131002in}}%
\pgfpathlineto{\pgfqpoint{2.451426in}{3.124399in}}%
\pgfpathlineto{\pgfqpoint{2.456778in}{3.117905in}}%
\pgfpathlineto{\pgfqpoint{2.459454in}{3.117544in}}%
\pgfpathlineto{\pgfqpoint{2.462131in}{3.119467in}}%
\pgfpathlineto{\pgfqpoint{2.466948in}{3.126520in}}%
\pgfpathlineto{\pgfqpoint{2.472032in}{3.132539in}}%
\pgfpathlineto{\pgfqpoint{2.474708in}{3.132860in}}%
\pgfpathlineto{\pgfqpoint{2.477384in}{3.130902in}}%
\pgfpathlineto{\pgfqpoint{2.482201in}{3.123826in}}%
\pgfpathlineto{\pgfqpoint{2.487286in}{3.117855in}}%
\pgfpathlineto{\pgfqpoint{2.489962in}{3.117574in}}%
\pgfpathlineto{\pgfqpoint{2.492905in}{3.119874in}}%
\pgfpathlineto{\pgfqpoint{2.498793in}{3.128703in}}%
\pgfpathlineto{\pgfqpoint{2.502807in}{3.132718in}}%
\pgfpathlineto{\pgfqpoint{2.505483in}{3.132724in}}%
\pgfpathlineto{\pgfqpoint{2.508427in}{3.130165in}}%
\pgfpathlineto{\pgfqpoint{2.519666in}{3.117417in}}%
\pgfpathlineto{\pgfqpoint{2.522342in}{3.118837in}}%
\pgfpathlineto{\pgfqpoint{2.526356in}{3.124233in}}%
\pgfpathlineto{\pgfqpoint{2.532779in}{3.132491in}}%
\pgfpathlineto{\pgfqpoint{2.535455in}{3.132887in}}%
\pgfpathlineto{\pgfqpoint{2.538131in}{3.130996in}}%
\pgfpathlineto{\pgfqpoint{2.542948in}{3.123962in}}%
\pgfpathlineto{\pgfqpoint{2.548033in}{3.117902in}}%
\pgfpathlineto{\pgfqpoint{2.550709in}{3.117546in}}%
\pgfpathlineto{\pgfqpoint{2.553385in}{3.119473in}}%
\pgfpathlineto{\pgfqpoint{2.558202in}{3.126529in}}%
\pgfpathlineto{\pgfqpoint{2.563286in}{3.132542in}}%
\pgfpathlineto{\pgfqpoint{2.565962in}{3.132858in}}%
\pgfpathlineto{\pgfqpoint{2.568638in}{3.130896in}}%
\pgfpathlineto{\pgfqpoint{2.573455in}{3.123817in}}%
\pgfpathlineto{\pgfqpoint{2.578540in}{3.117852in}}%
\pgfpathlineto{\pgfqpoint{2.581216in}{3.117575in}}%
\pgfpathlineto{\pgfqpoint{2.584160in}{3.119881in}}%
\pgfpathlineto{\pgfqpoint{2.590047in}{3.128711in}}%
\pgfpathlineto{\pgfqpoint{2.594061in}{3.132721in}}%
\pgfpathlineto{\pgfqpoint{2.596737in}{3.132721in}}%
\pgfpathlineto{\pgfqpoint{2.599681in}{3.130158in}}%
\pgfpathlineto{\pgfqpoint{2.610921in}{3.117418in}}%
\pgfpathlineto{\pgfqpoint{2.613597in}{3.118843in}}%
\pgfpathlineto{\pgfqpoint{2.617611in}{3.124242in}}%
\pgfpathlineto{\pgfqpoint{2.624033in}{3.132494in}}%
\pgfpathlineto{\pgfqpoint{2.626709in}{3.132885in}}%
\pgfpathlineto{\pgfqpoint{2.629385in}{3.130990in}}%
\pgfpathlineto{\pgfqpoint{2.634202in}{3.123953in}}%
\pgfpathlineto{\pgfqpoint{2.639287in}{3.117899in}}%
\pgfpathlineto{\pgfqpoint{2.641963in}{3.117548in}}%
\pgfpathlineto{\pgfqpoint{2.644639in}{3.119479in}}%
\pgfpathlineto{\pgfqpoint{2.649456in}{3.126538in}}%
\pgfpathlineto{\pgfqpoint{2.654541in}{3.132545in}}%
\pgfpathlineto{\pgfqpoint{2.657217in}{3.132857in}}%
\pgfpathlineto{\pgfqpoint{2.659893in}{3.130890in}}%
\pgfpathlineto{\pgfqpoint{2.664710in}{3.123809in}}%
\pgfpathlineto{\pgfqpoint{2.669794in}{3.117849in}}%
\pgfpathlineto{\pgfqpoint{2.672470in}{3.117577in}}%
\pgfpathlineto{\pgfqpoint{2.675414in}{3.119887in}}%
\pgfpathlineto{\pgfqpoint{2.681301in}{3.128719in}}%
\pgfpathlineto{\pgfqpoint{2.685315in}{3.132723in}}%
\pgfpathlineto{\pgfqpoint{2.687992in}{3.132719in}}%
\pgfpathlineto{\pgfqpoint{2.690935in}{3.130151in}}%
\pgfpathlineto{\pgfqpoint{2.702175in}{3.117418in}}%
\pgfpathlineto{\pgfqpoint{2.704851in}{3.118848in}}%
\pgfpathlineto{\pgfqpoint{2.708865in}{3.124251in}}%
\pgfpathlineto{\pgfqpoint{2.715288in}{3.132497in}}%
\pgfpathlineto{\pgfqpoint{2.717964in}{3.132883in}}%
\pgfpathlineto{\pgfqpoint{2.720640in}{3.130984in}}%
\pgfpathlineto{\pgfqpoint{2.725457in}{3.123945in}}%
\pgfpathlineto{\pgfqpoint{2.730541in}{3.117896in}}%
\pgfpathlineto{\pgfqpoint{2.733217in}{3.117549in}}%
\pgfpathlineto{\pgfqpoint{2.735893in}{3.119485in}}%
\pgfpathlineto{\pgfqpoint{2.740710in}{3.126547in}}%
\pgfpathlineto{\pgfqpoint{2.745795in}{3.132549in}}%
\pgfpathlineto{\pgfqpoint{2.748471in}{3.132855in}}%
\pgfpathlineto{\pgfqpoint{2.751415in}{3.130579in}}%
\pgfpathlineto{\pgfqpoint{2.757034in}{3.122150in}}%
\pgfpathlineto{\pgfqpoint{2.761316in}{3.117713in}}%
\pgfpathlineto{\pgfqpoint{2.763992in}{3.117682in}}%
\pgfpathlineto{\pgfqpoint{2.766936in}{3.120218in}}%
\pgfpathlineto{\pgfqpoint{2.778175in}{3.133005in}}%
\pgfpathlineto{\pgfqpoint{2.780851in}{3.131608in}}%
\pgfpathlineto{\pgfqpoint{2.784866in}{3.126232in}}%
\pgfpathlineto{\pgfqpoint{2.791288in}{3.117945in}}%
\pgfpathlineto{\pgfqpoint{2.793964in}{3.117524in}}%
\pgfpathlineto{\pgfqpoint{2.796640in}{3.119392in}}%
\pgfpathlineto{\pgfqpoint{2.801190in}{3.125982in}}%
\pgfpathlineto{\pgfqpoint{2.806542in}{3.132500in}}%
\pgfpathlineto{\pgfqpoint{2.809218in}{3.132882in}}%
\pgfpathlineto{\pgfqpoint{2.811894in}{3.130978in}}%
\pgfpathlineto{\pgfqpoint{2.816711in}{3.123936in}}%
\pgfpathlineto{\pgfqpoint{2.821795in}{3.117893in}}%
\pgfpathlineto{\pgfqpoint{2.824472in}{3.117551in}}%
\pgfpathlineto{\pgfqpoint{2.827148in}{3.119491in}}%
\pgfpathlineto{\pgfqpoint{2.831965in}{3.126555in}}%
\pgfpathlineto{\pgfqpoint{2.837049in}{3.132552in}}%
\pgfpathlineto{\pgfqpoint{2.839725in}{3.132853in}}%
\pgfpathlineto{\pgfqpoint{2.842669in}{3.130572in}}%
\pgfpathlineto{\pgfqpoint{2.848289in}{3.122142in}}%
\pgfpathlineto{\pgfqpoint{2.852570in}{3.117711in}}%
\pgfpathlineto{\pgfqpoint{2.855246in}{3.117685in}}%
\pgfpathlineto{\pgfqpoint{2.858190in}{3.120225in}}%
\pgfpathlineto{\pgfqpoint{2.869430in}{3.133004in}}%
\pgfpathlineto{\pgfqpoint{2.872106in}{3.131603in}}%
\pgfpathlineto{\pgfqpoint{2.876120in}{3.126223in}}%
\pgfpathlineto{\pgfqpoint{2.882542in}{3.117942in}}%
\pgfpathlineto{\pgfqpoint{2.885219in}{3.117525in}}%
\pgfpathlineto{\pgfqpoint{2.887895in}{3.119398in}}%
\pgfpathlineto{\pgfqpoint{2.892444in}{3.125991in}}%
\pgfpathlineto{\pgfqpoint{2.897796in}{3.132503in}}%
\pgfpathlineto{\pgfqpoint{2.900472in}{3.132880in}}%
\pgfpathlineto{\pgfqpoint{2.903148in}{3.130972in}}%
\pgfpathlineto{\pgfqpoint{2.907965in}{3.123927in}}%
\pgfpathlineto{\pgfqpoint{2.913050in}{3.117890in}}%
\pgfpathlineto{\pgfqpoint{2.915726in}{3.117553in}}%
\pgfpathlineto{\pgfqpoint{2.918402in}{3.119497in}}%
\pgfpathlineto{\pgfqpoint{2.923219in}{3.126564in}}%
\pgfpathlineto{\pgfqpoint{2.928303in}{3.132555in}}%
\pgfpathlineto{\pgfqpoint{2.930979in}{3.132851in}}%
\pgfpathlineto{\pgfqpoint{2.933923in}{3.130566in}}%
\pgfpathlineto{\pgfqpoint{2.939811in}{3.121742in}}%
\pgfpathlineto{\pgfqpoint{2.943825in}{3.117708in}}%
\pgfpathlineto{\pgfqpoint{2.946501in}{3.117687in}}%
\pgfpathlineto{\pgfqpoint{2.949444in}{3.120232in}}%
\pgfpathlineto{\pgfqpoint{2.960684in}{3.133004in}}%
\pgfpathlineto{\pgfqpoint{2.963360in}{3.131598in}}%
\pgfpathlineto{\pgfqpoint{2.967374in}{3.126215in}}%
\pgfpathlineto{\pgfqpoint{2.973797in}{3.117939in}}%
\pgfpathlineto{\pgfqpoint{2.976473in}{3.117527in}}%
\pgfpathlineto{\pgfqpoint{2.979149in}{3.119404in}}%
\pgfpathlineto{\pgfqpoint{2.983698in}{3.126000in}}%
\pgfpathlineto{\pgfqpoint{2.989050in}{3.132507in}}%
\pgfpathlineto{\pgfqpoint{2.991726in}{3.132878in}}%
\pgfpathlineto{\pgfqpoint{2.994403in}{3.130966in}}%
\pgfpathlineto{\pgfqpoint{2.999219in}{3.123918in}}%
\pgfpathlineto{\pgfqpoint{3.004304in}{3.117886in}}%
\pgfpathlineto{\pgfqpoint{3.006980in}{3.117555in}}%
\pgfpathlineto{\pgfqpoint{3.009656in}{3.119503in}}%
\pgfpathlineto{\pgfqpoint{3.014473in}{3.126573in}}%
\pgfpathlineto{\pgfqpoint{3.019558in}{3.132558in}}%
\pgfpathlineto{\pgfqpoint{3.022234in}{3.132849in}}%
\pgfpathlineto{\pgfqpoint{3.025177in}{3.130559in}}%
\pgfpathlineto{\pgfqpoint{3.031065in}{3.121734in}}%
\pgfpathlineto{\pgfqpoint{3.035079in}{3.117706in}}%
\pgfpathlineto{\pgfqpoint{3.037755in}{3.117690in}}%
\pgfpathlineto{\pgfqpoint{3.040699in}{3.120238in}}%
\pgfpathlineto{\pgfqpoint{3.051938in}{3.133003in}}%
\pgfpathlineto{\pgfqpoint{3.054614in}{3.131593in}}%
\pgfpathlineto{\pgfqpoint{3.058628in}{3.126206in}}%
\pgfpathlineto{\pgfqpoint{3.065051in}{3.117935in}}%
\pgfpathlineto{\pgfqpoint{3.067727in}{3.117529in}}%
\pgfpathlineto{\pgfqpoint{3.070403in}{3.119410in}}%
\pgfpathlineto{\pgfqpoint{3.074952in}{3.126009in}}%
\pgfpathlineto{\pgfqpoint{3.080305in}{3.132510in}}%
\pgfpathlineto{\pgfqpoint{3.082981in}{3.132877in}}%
\pgfpathlineto{\pgfqpoint{3.085657in}{3.130960in}}%
\pgfpathlineto{\pgfqpoint{3.090474in}{3.123909in}}%
\pgfpathlineto{\pgfqpoint{3.095558in}{3.117883in}}%
\pgfpathlineto{\pgfqpoint{3.098234in}{3.117556in}}%
\pgfpathlineto{\pgfqpoint{3.100910in}{3.119509in}}%
\pgfpathlineto{\pgfqpoint{3.105727in}{3.126582in}}%
\pgfpathlineto{\pgfqpoint{3.110812in}{3.132561in}}%
\pgfpathlineto{\pgfqpoint{3.113488in}{3.132848in}}%
\pgfpathlineto{\pgfqpoint{3.116432in}{3.130553in}}%
\pgfpathlineto{\pgfqpoint{3.122319in}{3.121726in}}%
\pgfpathlineto{\pgfqpoint{3.126333in}{3.117704in}}%
\pgfpathlineto{\pgfqpoint{3.129009in}{3.117692in}}%
\pgfpathlineto{\pgfqpoint{3.131953in}{3.120245in}}%
\pgfpathlineto{\pgfqpoint{3.143192in}{3.133003in}}%
\pgfpathlineto{\pgfqpoint{3.145869in}{3.131588in}}%
\pgfpathlineto{\pgfqpoint{3.149883in}{3.126197in}}%
\pgfpathlineto{\pgfqpoint{3.156305in}{3.117932in}}%
\pgfpathlineto{\pgfqpoint{3.158981in}{3.117530in}}%
\pgfpathlineto{\pgfqpoint{3.161657in}{3.119416in}}%
\pgfpathlineto{\pgfqpoint{3.166207in}{3.126018in}}%
\pgfpathlineto{\pgfqpoint{3.171559in}{3.132513in}}%
\pgfpathlineto{\pgfqpoint{3.174235in}{3.132875in}}%
\pgfpathlineto{\pgfqpoint{3.176911in}{3.130954in}}%
\pgfpathlineto{\pgfqpoint{3.181728in}{3.123901in}}%
\pgfpathlineto{\pgfqpoint{3.186813in}{3.117880in}}%
\pgfpathlineto{\pgfqpoint{3.189489in}{3.117558in}}%
\pgfpathlineto{\pgfqpoint{3.192165in}{3.119515in}}%
\pgfpathlineto{\pgfqpoint{3.196982in}{3.126590in}}%
\pgfpathlineto{\pgfqpoint{3.202066in}{3.132564in}}%
\pgfpathlineto{\pgfqpoint{3.204742in}{3.132846in}}%
\pgfpathlineto{\pgfqpoint{3.207686in}{3.130546in}}%
\pgfpathlineto{\pgfqpoint{3.213573in}{3.121718in}}%
\pgfpathlineto{\pgfqpoint{3.217587in}{3.117701in}}%
\pgfpathlineto{\pgfqpoint{3.220264in}{3.117694in}}%
\pgfpathlineto{\pgfqpoint{3.223207in}{3.120252in}}%
\pgfpathlineto{\pgfqpoint{3.234447in}{3.133002in}}%
\pgfpathlineto{\pgfqpoint{3.237123in}{3.131583in}}%
\pgfpathlineto{\pgfqpoint{3.241137in}{3.126188in}}%
\pgfpathlineto{\pgfqpoint{3.247560in}{3.117929in}}%
\pgfpathlineto{\pgfqpoint{3.250236in}{3.117532in}}%
\pgfpathlineto{\pgfqpoint{3.252912in}{3.119422in}}%
\pgfpathlineto{\pgfqpoint{3.257729in}{3.126455in}}%
\pgfpathlineto{\pgfqpoint{3.262813in}{3.132516in}}%
\pgfpathlineto{\pgfqpoint{3.265489in}{3.132873in}}%
\pgfpathlineto{\pgfqpoint{3.268165in}{3.130948in}}%
\pgfpathlineto{\pgfqpoint{3.272982in}{3.123892in}}%
\pgfpathlineto{\pgfqpoint{3.278067in}{3.117877in}}%
\pgfpathlineto{\pgfqpoint{3.280743in}{3.117560in}}%
\pgfpathlineto{\pgfqpoint{3.283419in}{3.119521in}}%
\pgfpathlineto{\pgfqpoint{3.288236in}{3.126599in}}%
\pgfpathlineto{\pgfqpoint{3.293320in}{3.132567in}}%
\pgfpathlineto{\pgfqpoint{3.295997in}{3.132844in}}%
\pgfpathlineto{\pgfqpoint{3.298940in}{3.130540in}}%
\pgfpathlineto{\pgfqpoint{3.304828in}{3.121710in}}%
\pgfpathlineto{\pgfqpoint{3.308842in}{3.117699in}}%
\pgfpathlineto{\pgfqpoint{3.311518in}{3.117697in}}%
\pgfpathlineto{\pgfqpoint{3.314461in}{3.120259in}}%
\pgfpathlineto{\pgfqpoint{3.325701in}{3.133001in}}%
\pgfpathlineto{\pgfqpoint{3.328377in}{3.131578in}}%
\pgfpathlineto{\pgfqpoint{3.332391in}{3.126179in}}%
\pgfpathlineto{\pgfqpoint{3.338814in}{3.117926in}}%
\pgfpathlineto{\pgfqpoint{3.341490in}{3.117533in}}%
\pgfpathlineto{\pgfqpoint{3.344166in}{3.119428in}}%
\pgfpathlineto{\pgfqpoint{3.348983in}{3.126463in}}%
\pgfpathlineto{\pgfqpoint{3.354067in}{3.132519in}}%
\pgfpathlineto{\pgfqpoint{3.356744in}{3.132872in}}%
\pgfpathlineto{\pgfqpoint{3.359420in}{3.130941in}}%
\pgfpathlineto{\pgfqpoint{3.364237in}{3.123883in}}%
\pgfpathlineto{\pgfqpoint{3.369321in}{3.117874in}}%
\pgfpathlineto{\pgfqpoint{3.371997in}{3.117562in}}%
\pgfpathlineto{\pgfqpoint{3.374673in}{3.119528in}}%
\pgfpathlineto{\pgfqpoint{3.379490in}{3.126608in}}%
\pgfpathlineto{\pgfqpoint{3.384575in}{3.132570in}}%
\pgfpathlineto{\pgfqpoint{3.387251in}{3.132842in}}%
\pgfpathlineto{\pgfqpoint{3.390194in}{3.130533in}}%
\pgfpathlineto{\pgfqpoint{3.396082in}{3.121702in}}%
\pgfpathlineto{\pgfqpoint{3.400096in}{3.117696in}}%
\pgfpathlineto{\pgfqpoint{3.402772in}{3.117699in}}%
\pgfpathlineto{\pgfqpoint{3.405716in}{3.120266in}}%
\pgfpathlineto{\pgfqpoint{3.416955in}{3.133001in}}%
\pgfpathlineto{\pgfqpoint{3.419631in}{3.131572in}}%
\pgfpathlineto{\pgfqpoint{3.423645in}{3.126170in}}%
\pgfpathlineto{\pgfqpoint{3.430068in}{3.117923in}}%
\pgfpathlineto{\pgfqpoint{3.432744in}{3.117535in}}%
\pgfpathlineto{\pgfqpoint{3.435420in}{3.119434in}}%
\pgfpathlineto{\pgfqpoint{3.440237in}{3.126472in}}%
\pgfpathlineto{\pgfqpoint{3.445322in}{3.132522in}}%
\pgfpathlineto{\pgfqpoint{3.447998in}{3.132870in}}%
\pgfpathlineto{\pgfqpoint{3.450674in}{3.130935in}}%
\pgfpathlineto{\pgfqpoint{3.455491in}{3.123874in}}%
\pgfpathlineto{\pgfqpoint{3.460575in}{3.117871in}}%
\pgfpathlineto{\pgfqpoint{3.463251in}{3.117564in}}%
\pgfpathlineto{\pgfqpoint{3.466195in}{3.119839in}}%
\pgfpathlineto{\pgfqpoint{3.471815in}{3.128267in}}%
\pgfpathlineto{\pgfqpoint{3.476097in}{3.132705in}}%
\pgfpathlineto{\pgfqpoint{3.478773in}{3.132737in}}%
\pgfpathlineto{\pgfqpoint{3.481716in}{3.130203in}}%
\pgfpathlineto{\pgfqpoint{3.492956in}{3.117414in}}%
\pgfpathlineto{\pgfqpoint{3.495632in}{3.118809in}}%
\pgfpathlineto{\pgfqpoint{3.499646in}{3.124184in}}%
\pgfpathlineto{\pgfqpoint{3.506069in}{3.132473in}}%
\pgfpathlineto{\pgfqpoint{3.508745in}{3.132895in}}%
\pgfpathlineto{\pgfqpoint{3.511421in}{3.131028in}}%
\pgfpathlineto{\pgfqpoint{3.515970in}{3.124439in}}%
\pgfpathlineto{\pgfqpoint{3.521322in}{3.117919in}}%
\pgfpathlineto{\pgfqpoint{3.523998in}{3.117537in}}%
\pgfpathlineto{\pgfqpoint{3.526674in}{3.119440in}}%
\pgfpathlineto{\pgfqpoint{3.531491in}{3.126481in}}%
\pgfpathlineto{\pgfqpoint{3.536576in}{3.132525in}}%
\pgfpathlineto{\pgfqpoint{3.539252in}{3.132868in}}%
\pgfpathlineto{\pgfqpoint{3.541928in}{3.130929in}}%
\pgfpathlineto{\pgfqpoint{3.546745in}{3.123866in}}%
\pgfpathlineto{\pgfqpoint{3.551830in}{3.117868in}}%
\pgfpathlineto{\pgfqpoint{3.554506in}{3.117565in}}%
\pgfpathlineto{\pgfqpoint{3.557449in}{3.119845in}}%
\pgfpathlineto{\pgfqpoint{3.563069in}{3.128275in}}%
\pgfpathlineto{\pgfqpoint{3.567351in}{3.132707in}}%
\pgfpathlineto{\pgfqpoint{3.570027in}{3.132735in}}%
\pgfpathlineto{\pgfqpoint{3.572971in}{3.130196in}}%
\pgfpathlineto{\pgfqpoint{3.584210in}{3.117415in}}%
\pgfpathlineto{\pgfqpoint{3.586886in}{3.118814in}}%
\pgfpathlineto{\pgfqpoint{3.590900in}{3.124193in}}%
\pgfpathlineto{\pgfqpoint{3.597323in}{3.132476in}}%
\pgfpathlineto{\pgfqpoint{3.599999in}{3.132894in}}%
\pgfpathlineto{\pgfqpoint{3.602675in}{3.131022in}}%
\pgfpathlineto{\pgfqpoint{3.607224in}{3.124430in}}%
\pgfpathlineto{\pgfqpoint{3.612577in}{3.117916in}}%
\pgfpathlineto{\pgfqpoint{3.615253in}{3.117538in}}%
\pgfpathlineto{\pgfqpoint{3.617929in}{3.119446in}}%
\pgfpathlineto{\pgfqpoint{3.622746in}{3.126490in}}%
\pgfpathlineto{\pgfqpoint{3.627830in}{3.132529in}}%
\pgfpathlineto{\pgfqpoint{3.630506in}{3.132866in}}%
\pgfpathlineto{\pgfqpoint{3.633182in}{3.130923in}}%
\pgfpathlineto{\pgfqpoint{3.637999in}{3.123857in}}%
\pgfpathlineto{\pgfqpoint{3.643084in}{3.117865in}}%
\pgfpathlineto{\pgfqpoint{3.645760in}{3.117567in}}%
\pgfpathlineto{\pgfqpoint{3.648704in}{3.119851in}}%
\pgfpathlineto{\pgfqpoint{3.654323in}{3.128283in}}%
\pgfpathlineto{\pgfqpoint{3.658605in}{3.132710in}}%
\pgfpathlineto{\pgfqpoint{3.661281in}{3.132732in}}%
\pgfpathlineto{\pgfqpoint{3.664225in}{3.130189in}}%
\pgfpathlineto{\pgfqpoint{3.675464in}{3.117415in}}%
\pgfpathlineto{\pgfqpoint{3.678141in}{3.118820in}}%
\pgfpathlineto{\pgfqpoint{3.682155in}{3.124202in}}%
\pgfpathlineto{\pgfqpoint{3.688577in}{3.132479in}}%
\pgfpathlineto{\pgfqpoint{3.691253in}{3.132892in}}%
\pgfpathlineto{\pgfqpoint{3.693929in}{3.131016in}}%
\pgfpathlineto{\pgfqpoint{3.698479in}{3.124421in}}%
\pgfpathlineto{\pgfqpoint{3.703831in}{3.117913in}}%
\pgfpathlineto{\pgfqpoint{3.706507in}{3.117540in}}%
\pgfpathlineto{\pgfqpoint{3.709183in}{3.119452in}}%
\pgfpathlineto{\pgfqpoint{3.714000in}{3.126498in}}%
\pgfpathlineto{\pgfqpoint{3.719085in}{3.132532in}}%
\pgfpathlineto{\pgfqpoint{3.721761in}{3.132865in}}%
\pgfpathlineto{\pgfqpoint{3.724437in}{3.130917in}}%
\pgfpathlineto{\pgfqpoint{3.729254in}{3.123848in}}%
\pgfpathlineto{\pgfqpoint{3.734338in}{3.117862in}}%
\pgfpathlineto{\pgfqpoint{3.737014in}{3.117569in}}%
\pgfpathlineto{\pgfqpoint{3.739958in}{3.119858in}}%
\pgfpathlineto{\pgfqpoint{3.745845in}{3.128683in}}%
\pgfpathlineto{\pgfqpoint{3.749859in}{3.132712in}}%
\pgfpathlineto{\pgfqpoint{3.752535in}{3.132730in}}%
\pgfpathlineto{\pgfqpoint{3.755479in}{3.130182in}}%
\pgfpathlineto{\pgfqpoint{3.766719in}{3.117416in}}%
\pgfpathlineto{\pgfqpoint{3.769395in}{3.118825in}}%
\pgfpathlineto{\pgfqpoint{3.773409in}{3.124211in}}%
\pgfpathlineto{\pgfqpoint{3.779831in}{3.132483in}}%
\pgfpathlineto{\pgfqpoint{3.782508in}{3.132891in}}%
\pgfpathlineto{\pgfqpoint{3.785184in}{3.131010in}}%
\pgfpathlineto{\pgfqpoint{3.789733in}{3.124412in}}%
\pgfpathlineto{\pgfqpoint{3.795085in}{3.117910in}}%
\pgfpathlineto{\pgfqpoint{3.797761in}{3.117542in}}%
\pgfpathlineto{\pgfqpoint{3.800437in}{3.119458in}}%
\pgfpathlineto{\pgfqpoint{3.805254in}{3.126507in}}%
\pgfpathlineto{\pgfqpoint{3.810339in}{3.132535in}}%
\pgfpathlineto{\pgfqpoint{3.813015in}{3.132863in}}%
\pgfpathlineto{\pgfqpoint{3.815691in}{3.130911in}}%
\pgfpathlineto{\pgfqpoint{3.820508in}{3.123839in}}%
\pgfpathlineto{\pgfqpoint{3.825592in}{3.117859in}}%
\pgfpathlineto{\pgfqpoint{3.828269in}{3.117571in}}%
\pgfpathlineto{\pgfqpoint{3.831212in}{3.119864in}}%
\pgfpathlineto{\pgfqpoint{3.837100in}{3.128691in}}%
\pgfpathlineto{\pgfqpoint{3.841114in}{3.132715in}}%
\pgfpathlineto{\pgfqpoint{3.843790in}{3.132727in}}%
\pgfpathlineto{\pgfqpoint{3.846733in}{3.130175in}}%
\pgfpathlineto{\pgfqpoint{3.857973in}{3.117416in}}%
\pgfpathlineto{\pgfqpoint{3.860649in}{3.118830in}}%
\pgfpathlineto{\pgfqpoint{3.864663in}{3.124220in}}%
\pgfpathlineto{\pgfqpoint{3.871086in}{3.132486in}}%
\pgfpathlineto{\pgfqpoint{3.873762in}{3.132889in}}%
\pgfpathlineto{\pgfqpoint{3.876438in}{3.131005in}}%
\pgfpathlineto{\pgfqpoint{3.880987in}{3.124403in}}%
\pgfpathlineto{\pgfqpoint{3.886339in}{3.117907in}}%
\pgfpathlineto{\pgfqpoint{3.889015in}{3.117543in}}%
\pgfpathlineto{\pgfqpoint{3.891692in}{3.119464in}}%
\pgfpathlineto{\pgfqpoint{3.896508in}{3.126516in}}%
\pgfpathlineto{\pgfqpoint{3.901593in}{3.132538in}}%
\pgfpathlineto{\pgfqpoint{3.904269in}{3.132861in}}%
\pgfpathlineto{\pgfqpoint{3.906945in}{3.130905in}}%
\pgfpathlineto{\pgfqpoint{3.911762in}{3.123831in}}%
\pgfpathlineto{\pgfqpoint{3.916847in}{3.117856in}}%
\pgfpathlineto{\pgfqpoint{3.919523in}{3.117573in}}%
\pgfpathlineto{\pgfqpoint{3.922466in}{3.119871in}}%
\pgfpathlineto{\pgfqpoint{3.928354in}{3.128699in}}%
\pgfpathlineto{\pgfqpoint{3.932368in}{3.132717in}}%
\pgfpathlineto{\pgfqpoint{3.935044in}{3.132725in}}%
\pgfpathlineto{\pgfqpoint{3.937988in}{3.130168in}}%
\pgfpathlineto{\pgfqpoint{3.949227in}{3.117417in}}%
\pgfpathlineto{\pgfqpoint{3.951903in}{3.118835in}}%
\pgfpathlineto{\pgfqpoint{3.955917in}{3.124229in}}%
\pgfpathlineto{\pgfqpoint{3.962340in}{3.132489in}}%
\pgfpathlineto{\pgfqpoint{3.965016in}{3.132887in}}%
\pgfpathlineto{\pgfqpoint{3.967692in}{3.130999in}}%
\pgfpathlineto{\pgfqpoint{3.972509in}{3.123967in}}%
\pgfpathlineto{\pgfqpoint{3.977594in}{3.117904in}}%
\pgfpathlineto{\pgfqpoint{3.980270in}{3.117545in}}%
\pgfpathlineto{\pgfqpoint{3.982946in}{3.119470in}}%
\pgfpathlineto{\pgfqpoint{3.987763in}{3.126525in}}%
\pgfpathlineto{\pgfqpoint{3.992847in}{3.132541in}}%
\pgfpathlineto{\pgfqpoint{3.995523in}{3.132859in}}%
\pgfpathlineto{\pgfqpoint{3.998199in}{3.130899in}}%
\pgfpathlineto{\pgfqpoint{4.003016in}{3.123822in}}%
\pgfpathlineto{\pgfqpoint{4.008101in}{3.117853in}}%
\pgfpathlineto{\pgfqpoint{4.010777in}{3.117575in}}%
\pgfpathlineto{\pgfqpoint{4.013721in}{3.119877in}}%
\pgfpathlineto{\pgfqpoint{4.019608in}{3.128707in}}%
\pgfpathlineto{\pgfqpoint{4.023622in}{3.132720in}}%
\pgfpathlineto{\pgfqpoint{4.026298in}{3.132723in}}%
\pgfpathlineto{\pgfqpoint{4.029242in}{3.130162in}}%
\pgfpathlineto{\pgfqpoint{4.040482in}{3.117417in}}%
\pgfpathlineto{\pgfqpoint{4.043158in}{3.118840in}}%
\pgfpathlineto{\pgfqpoint{4.047172in}{3.124237in}}%
\pgfpathlineto{\pgfqpoint{4.053594in}{3.132492in}}%
\pgfpathlineto{\pgfqpoint{4.056270in}{3.132886in}}%
\pgfpathlineto{\pgfqpoint{4.058946in}{3.130993in}}%
\pgfpathlineto{\pgfqpoint{4.063763in}{3.123958in}}%
\pgfpathlineto{\pgfqpoint{4.068848in}{3.117900in}}%
\pgfpathlineto{\pgfqpoint{4.071524in}{3.117547in}}%
\pgfpathlineto{\pgfqpoint{4.074200in}{3.119476in}}%
\pgfpathlineto{\pgfqpoint{4.079017in}{3.126534in}}%
\pgfpathlineto{\pgfqpoint{4.084102in}{3.132544in}}%
\pgfpathlineto{\pgfqpoint{4.086778in}{3.132858in}}%
\pgfpathlineto{\pgfqpoint{4.089454in}{3.130893in}}%
\pgfpathlineto{\pgfqpoint{4.094271in}{3.123813in}}%
\pgfpathlineto{\pgfqpoint{4.099355in}{3.117850in}}%
\pgfpathlineto{\pgfqpoint{4.102031in}{3.117576in}}%
\pgfpathlineto{\pgfqpoint{4.104975in}{3.119884in}}%
\pgfpathlineto{\pgfqpoint{4.110862in}{3.128715in}}%
\pgfpathlineto{\pgfqpoint{4.114876in}{3.132722in}}%
\pgfpathlineto{\pgfqpoint{4.117553in}{3.132720in}}%
\pgfpathlineto{\pgfqpoint{4.120496in}{3.130155in}}%
\pgfpathlineto{\pgfqpoint{4.131736in}{3.117418in}}%
\pgfpathlineto{\pgfqpoint{4.134412in}{3.118845in}}%
\pgfpathlineto{\pgfqpoint{4.138426in}{3.124246in}}%
\pgfpathlineto{\pgfqpoint{4.144849in}{3.132495in}}%
\pgfpathlineto{\pgfqpoint{4.147525in}{3.132884in}}%
\pgfpathlineto{\pgfqpoint{4.150201in}{3.130987in}}%
\pgfpathlineto{\pgfqpoint{4.155018in}{3.123949in}}%
\pgfpathlineto{\pgfqpoint{4.160102in}{3.117897in}}%
\pgfpathlineto{\pgfqpoint{4.162778in}{3.117549in}}%
\pgfpathlineto{\pgfqpoint{4.165454in}{3.119482in}}%
\pgfpathlineto{\pgfqpoint{4.170271in}{3.126542in}}%
\pgfpathlineto{\pgfqpoint{4.175356in}{3.132547in}}%
\pgfpathlineto{\pgfqpoint{4.178032in}{3.132856in}}%
\pgfpathlineto{\pgfqpoint{4.180976in}{3.130582in}}%
\pgfpathlineto{\pgfqpoint{4.186595in}{3.122154in}}%
\pgfpathlineto{\pgfqpoint{4.190877in}{3.117715in}}%
\pgfpathlineto{\pgfqpoint{4.193553in}{3.117681in}}%
\pgfpathlineto{\pgfqpoint{4.196497in}{3.120214in}}%
\pgfpathlineto{\pgfqpoint{4.207736in}{3.133005in}}%
\pgfpathlineto{\pgfqpoint{4.210412in}{3.131611in}}%
\pgfpathlineto{\pgfqpoint{4.214427in}{3.126237in}}%
\pgfpathlineto{\pgfqpoint{4.220849in}{3.117947in}}%
\pgfpathlineto{\pgfqpoint{4.223525in}{3.117523in}}%
\pgfpathlineto{\pgfqpoint{4.226201in}{3.119389in}}%
\pgfpathlineto{\pgfqpoint{4.230751in}{3.125978in}}%
\pgfpathlineto{\pgfqpoint{4.236103in}{3.132499in}}%
\pgfpathlineto{\pgfqpoint{4.238779in}{3.132883in}}%
\pgfpathlineto{\pgfqpoint{4.241455in}{3.130981in}}%
\pgfpathlineto{\pgfqpoint{4.246272in}{3.123940in}}%
\pgfpathlineto{\pgfqpoint{4.251356in}{3.117894in}}%
\pgfpathlineto{\pgfqpoint{4.254033in}{3.117550in}}%
\pgfpathlineto{\pgfqpoint{4.256709in}{3.119488in}}%
\pgfpathlineto{\pgfqpoint{4.261526in}{3.126551in}}%
\pgfpathlineto{\pgfqpoint{4.266610in}{3.132550in}}%
\pgfpathlineto{\pgfqpoint{4.269286in}{3.132854in}}%
\pgfpathlineto{\pgfqpoint{4.272230in}{3.130575in}}%
\pgfpathlineto{\pgfqpoint{4.277850in}{3.122146in}}%
\pgfpathlineto{\pgfqpoint{4.282131in}{3.117712in}}%
\pgfpathlineto{\pgfqpoint{4.284807in}{3.117684in}}%
\pgfpathlineto{\pgfqpoint{4.287751in}{3.120221in}}%
\pgfpathlineto{\pgfqpoint{4.298991in}{3.133004in}}%
\pgfpathlineto{\pgfqpoint{4.301667in}{3.131606in}}%
\pgfpathlineto{\pgfqpoint{4.305681in}{3.126228in}}%
\pgfpathlineto{\pgfqpoint{4.312103in}{3.117944in}}%
\pgfpathlineto{\pgfqpoint{4.314780in}{3.117525in}}%
\pgfpathlineto{\pgfqpoint{4.317456in}{3.119395in}}%
\pgfpathlineto{\pgfqpoint{4.322005in}{3.125987in}}%
\pgfpathlineto{\pgfqpoint{4.327357in}{3.132502in}}%
\pgfpathlineto{\pgfqpoint{4.330033in}{3.132881in}}%
\pgfpathlineto{\pgfqpoint{4.332709in}{3.130975in}}%
\pgfpathlineto{\pgfqpoint{4.337526in}{3.123931in}}%
\pgfpathlineto{\pgfqpoint{4.342611in}{3.117891in}}%
\pgfpathlineto{\pgfqpoint{4.345287in}{3.117552in}}%
\pgfpathlineto{\pgfqpoint{4.347963in}{3.119494in}}%
\pgfpathlineto{\pgfqpoint{4.352780in}{3.126560in}}%
\pgfpathlineto{\pgfqpoint{4.357864in}{3.132553in}}%
\pgfpathlineto{\pgfqpoint{4.360540in}{3.132852in}}%
\pgfpathlineto{\pgfqpoint{4.363484in}{3.130569in}}%
\pgfpathlineto{\pgfqpoint{4.369104in}{3.122138in}}%
\pgfpathlineto{\pgfqpoint{4.373386in}{3.117710in}}%
\pgfpathlineto{\pgfqpoint{4.376062in}{3.117686in}}%
\pgfpathlineto{\pgfqpoint{4.379005in}{3.120228in}}%
\pgfpathlineto{\pgfqpoint{4.390245in}{3.133004in}}%
\pgfpathlineto{\pgfqpoint{4.392921in}{3.131601in}}%
\pgfpathlineto{\pgfqpoint{4.396935in}{3.126219in}}%
\pgfpathlineto{\pgfqpoint{4.403358in}{3.117940in}}%
\pgfpathlineto{\pgfqpoint{4.406034in}{3.117526in}}%
\pgfpathlineto{\pgfqpoint{4.408710in}{3.119401in}}%
\pgfpathlineto{\pgfqpoint{4.413259in}{3.125996in}}%
\pgfpathlineto{\pgfqpoint{4.418611in}{3.132505in}}%
\pgfpathlineto{\pgfqpoint{4.421287in}{3.132879in}}%
\pgfpathlineto{\pgfqpoint{4.423964in}{3.130969in}}%
\pgfpathlineto{\pgfqpoint{4.428780in}{3.123923in}}%
\pgfpathlineto{\pgfqpoint{4.433865in}{3.117888in}}%
\pgfpathlineto{\pgfqpoint{4.436541in}{3.117554in}}%
\pgfpathlineto{\pgfqpoint{4.439217in}{3.119500in}}%
\pgfpathlineto{\pgfqpoint{4.444034in}{3.126569in}}%
\pgfpathlineto{\pgfqpoint{4.449119in}{3.132556in}}%
\pgfpathlineto{\pgfqpoint{4.451795in}{3.132850in}}%
\pgfpathlineto{\pgfqpoint{4.454738in}{3.130563in}}%
\pgfpathlineto{\pgfqpoint{4.460626in}{3.121738in}}%
\pgfpathlineto{\pgfqpoint{4.464640in}{3.117707in}}%
\pgfpathlineto{\pgfqpoint{4.467316in}{3.117688in}}%
\pgfpathlineto{\pgfqpoint{4.470260in}{3.120235in}}%
\pgfpathlineto{\pgfqpoint{4.481499in}{3.133003in}}%
\pgfpathlineto{\pgfqpoint{4.484175in}{3.131596in}}%
\pgfpathlineto{\pgfqpoint{4.488189in}{3.126210in}}%
\pgfpathlineto{\pgfqpoint{4.494612in}{3.117937in}}%
\pgfpathlineto{\pgfqpoint{4.497288in}{3.117528in}}%
\pgfpathlineto{\pgfqpoint{4.499964in}{3.119407in}}%
\pgfpathlineto{\pgfqpoint{4.504513in}{3.126005in}}%
\pgfpathlineto{\pgfqpoint{4.509866in}{3.132508in}}%
\pgfpathlineto{\pgfqpoint{4.512542in}{3.132878in}}%
\pgfpathlineto{\pgfqpoint{4.515218in}{3.130963in}}%
\pgfpathlineto{\pgfqpoint{4.520035in}{3.123914in}}%
\pgfpathlineto{\pgfqpoint{4.525119in}{3.117885in}}%
\pgfpathlineto{\pgfqpoint{4.527795in}{3.117556in}}%
\pgfpathlineto{\pgfqpoint{4.530471in}{3.119506in}}%
\pgfpathlineto{\pgfqpoint{4.535288in}{3.126577in}}%
\pgfpathlineto{\pgfqpoint{4.540373in}{3.132559in}}%
\pgfpathlineto{\pgfqpoint{4.543049in}{3.132848in}}%
\pgfpathlineto{\pgfqpoint{4.545993in}{3.130556in}}%
\pgfpathlineto{\pgfqpoint{4.551880in}{3.121730in}}%
\pgfpathlineto{\pgfqpoint{4.555894in}{3.117705in}}%
\pgfpathlineto{\pgfqpoint{4.558570in}{3.117691in}}%
\pgfpathlineto{\pgfqpoint{4.561514in}{3.120242in}}%
\pgfpathlineto{\pgfqpoint{4.572753in}{3.133003in}}%
\pgfpathlineto{\pgfqpoint{4.575430in}{3.131590in}}%
\pgfpathlineto{\pgfqpoint{4.579444in}{3.126201in}}%
\pgfpathlineto{\pgfqpoint{4.585866in}{3.117934in}}%
\pgfpathlineto{\pgfqpoint{4.588542in}{3.117529in}}%
\pgfpathlineto{\pgfqpoint{4.591218in}{3.119413in}}%
\pgfpathlineto{\pgfqpoint{4.595768in}{3.126013in}}%
\pgfpathlineto{\pgfqpoint{4.601120in}{3.132511in}}%
\pgfpathlineto{\pgfqpoint{4.603796in}{3.132876in}}%
\pgfpathlineto{\pgfqpoint{4.606472in}{3.130957in}}%
\pgfpathlineto{\pgfqpoint{4.611289in}{3.123905in}}%
\pgfpathlineto{\pgfqpoint{4.616374in}{3.117882in}}%
\pgfpathlineto{\pgfqpoint{4.619050in}{3.117557in}}%
\pgfpathlineto{\pgfqpoint{4.621726in}{3.119512in}}%
\pgfpathlineto{\pgfqpoint{4.626543in}{3.126586in}}%
\pgfpathlineto{\pgfqpoint{4.631627in}{3.132562in}}%
\pgfpathlineto{\pgfqpoint{4.634303in}{3.132847in}}%
\pgfpathlineto{\pgfqpoint{4.637247in}{3.130550in}}%
\pgfpathlineto{\pgfqpoint{4.643134in}{3.121722in}}%
\pgfpathlineto{\pgfqpoint{4.647148in}{3.117702in}}%
\pgfpathlineto{\pgfqpoint{4.649825in}{3.117693in}}%
\pgfpathlineto{\pgfqpoint{4.652768in}{3.120249in}}%
\pgfpathlineto{\pgfqpoint{4.664008in}{3.133002in}}%
\pgfpathlineto{\pgfqpoint{4.666684in}{3.131585in}}%
\pgfpathlineto{\pgfqpoint{4.670698in}{3.126192in}}%
\pgfpathlineto{\pgfqpoint{4.677121in}{3.117931in}}%
\pgfpathlineto{\pgfqpoint{4.679797in}{3.117531in}}%
\pgfpathlineto{\pgfqpoint{4.682473in}{3.119419in}}%
\pgfpathlineto{\pgfqpoint{4.687290in}{3.126450in}}%
\pgfpathlineto{\pgfqpoint{4.692374in}{3.132515in}}%
\pgfpathlineto{\pgfqpoint{4.695050in}{3.132874in}}%
\pgfpathlineto{\pgfqpoint{4.697726in}{3.130951in}}%
\pgfpathlineto{\pgfqpoint{4.702543in}{3.123896in}}%
\pgfpathlineto{\pgfqpoint{4.707628in}{3.117879in}}%
\pgfpathlineto{\pgfqpoint{4.710304in}{3.117559in}}%
\pgfpathlineto{\pgfqpoint{4.712980in}{3.119518in}}%
\pgfpathlineto{\pgfqpoint{4.717797in}{3.126595in}}%
\pgfpathlineto{\pgfqpoint{4.722881in}{3.132565in}}%
\pgfpathlineto{\pgfqpoint{4.725558in}{3.132845in}}%
\pgfpathlineto{\pgfqpoint{4.728501in}{3.130543in}}%
\pgfpathlineto{\pgfqpoint{4.734389in}{3.121714in}}%
\pgfpathlineto{\pgfqpoint{4.738403in}{3.117700in}}%
\pgfpathlineto{\pgfqpoint{4.741079in}{3.117696in}}%
\pgfpathlineto{\pgfqpoint{4.744022in}{3.120256in}}%
\pgfpathlineto{\pgfqpoint{4.755262in}{3.133002in}}%
\pgfpathlineto{\pgfqpoint{4.757938in}{3.131580in}}%
\pgfpathlineto{\pgfqpoint{4.761952in}{3.126184in}}%
\pgfpathlineto{\pgfqpoint{4.768375in}{3.117927in}}%
\pgfpathlineto{\pgfqpoint{4.771051in}{3.117533in}}%
\pgfpathlineto{\pgfqpoint{4.773727in}{3.119425in}}%
\pgfpathlineto{\pgfqpoint{4.778544in}{3.126459in}}%
\pgfpathlineto{\pgfqpoint{4.783628in}{3.132518in}}%
\pgfpathlineto{\pgfqpoint{4.786305in}{3.132872in}}%
\pgfpathlineto{\pgfqpoint{4.788981in}{3.130944in}}%
\pgfpathlineto{\pgfqpoint{4.793798in}{3.123888in}}%
\pgfpathlineto{\pgfqpoint{4.798882in}{3.117876in}}%
\pgfpathlineto{\pgfqpoint{4.801558in}{3.117561in}}%
\pgfpathlineto{\pgfqpoint{4.804234in}{3.119524in}}%
\pgfpathlineto{\pgfqpoint{4.809051in}{3.126604in}}%
\pgfpathlineto{\pgfqpoint{4.814136in}{3.132568in}}%
\pgfpathlineto{\pgfqpoint{4.816812in}{3.132843in}}%
\pgfpathlineto{\pgfqpoint{4.819755in}{3.130537in}}%
\pgfpathlineto{\pgfqpoint{4.825643in}{3.121706in}}%
\pgfpathlineto{\pgfqpoint{4.829657in}{3.117697in}}%
\pgfpathlineto{\pgfqpoint{4.832333in}{3.117698in}}%
\pgfpathlineto{\pgfqpoint{4.835277in}{3.120262in}}%
\pgfpathlineto{\pgfqpoint{4.846516in}{3.133001in}}%
\pgfpathlineto{\pgfqpoint{4.849192in}{3.131575in}}%
\pgfpathlineto{\pgfqpoint{4.853206in}{3.126175in}}%
\pgfpathlineto{\pgfqpoint{4.859629in}{3.117924in}}%
\pgfpathlineto{\pgfqpoint{4.862305in}{3.117534in}}%
\pgfpathlineto{\pgfqpoint{4.864981in}{3.119431in}}%
\pgfpathlineto{\pgfqpoint{4.869798in}{3.126468in}}%
\pgfpathlineto{\pgfqpoint{4.874883in}{3.132521in}}%
\pgfpathlineto{\pgfqpoint{4.877559in}{3.132871in}}%
\pgfpathlineto{\pgfqpoint{4.880235in}{3.130938in}}%
\pgfpathlineto{\pgfqpoint{4.885052in}{3.123879in}}%
\pgfpathlineto{\pgfqpoint{4.890136in}{3.117873in}}%
\pgfpathlineto{\pgfqpoint{4.892812in}{3.117563in}}%
\pgfpathlineto{\pgfqpoint{4.895488in}{3.119531in}}%
\pgfpathlineto{\pgfqpoint{4.900305in}{3.126612in}}%
\pgfpathlineto{\pgfqpoint{4.905390in}{3.132571in}}%
\pgfpathlineto{\pgfqpoint{4.908066in}{3.132841in}}%
\pgfpathlineto{\pgfqpoint{4.911010in}{3.130530in}}%
\pgfpathlineto{\pgfqpoint{4.916897in}{3.121698in}}%
\pgfpathlineto{\pgfqpoint{4.920911in}{3.117695in}}%
\pgfpathlineto{\pgfqpoint{4.923587in}{3.117700in}}%
\pgfpathlineto{\pgfqpoint{4.926531in}{3.120269in}}%
\pgfpathlineto{\pgfqpoint{4.937771in}{3.133001in}}%
\pgfpathlineto{\pgfqpoint{4.940447in}{3.131570in}}%
\pgfpathlineto{\pgfqpoint{4.944461in}{3.126166in}}%
\pgfpathlineto{\pgfqpoint{4.950883in}{3.117921in}}%
\pgfpathlineto{\pgfqpoint{4.953559in}{3.117536in}}%
\pgfpathlineto{\pgfqpoint{4.956235in}{3.119437in}}%
\pgfpathlineto{\pgfqpoint{4.961052in}{3.126476in}}%
\pgfpathlineto{\pgfqpoint{4.966137in}{3.132524in}}%
\pgfpathlineto{\pgfqpoint{4.968813in}{3.132869in}}%
\pgfpathlineto{\pgfqpoint{4.971489in}{3.130932in}}%
\pgfpathlineto{\pgfqpoint{4.976306in}{3.123870in}}%
\pgfpathlineto{\pgfqpoint{4.981391in}{3.117870in}}%
\pgfpathlineto{\pgfqpoint{4.984067in}{3.117564in}}%
\pgfpathlineto{\pgfqpoint{4.987010in}{3.119842in}}%
\pgfpathlineto{\pgfqpoint{4.992630in}{3.128271in}}%
\pgfpathlineto{\pgfqpoint{4.996912in}{3.132706in}}%
\pgfpathlineto{\pgfqpoint{4.999588in}{3.132736in}}%
\pgfpathlineto{\pgfqpoint{5.002532in}{3.130199in}}%
\pgfpathlineto{\pgfqpoint{5.013771in}{3.117414in}}%
\pgfpathlineto{\pgfqpoint{5.016447in}{3.118812in}}%
\pgfpathlineto{\pgfqpoint{5.020461in}{3.124189in}}%
\pgfpathlineto{\pgfqpoint{5.026884in}{3.132474in}}%
\pgfpathlineto{\pgfqpoint{5.029560in}{3.132895in}}%
\pgfpathlineto{\pgfqpoint{5.032236in}{3.131025in}}%
\pgfpathlineto{\pgfqpoint{5.036785in}{3.124434in}}%
\pgfpathlineto{\pgfqpoint{5.042138in}{3.117918in}}%
\pgfpathlineto{\pgfqpoint{5.044814in}{3.117538in}}%
\pgfpathlineto{\pgfqpoint{5.047490in}{3.119443in}}%
\pgfpathlineto{\pgfqpoint{5.052307in}{3.126485in}}%
\pgfpathlineto{\pgfqpoint{5.057391in}{3.132527in}}%
\pgfpathlineto{\pgfqpoint{5.060067in}{3.132867in}}%
\pgfpathlineto{\pgfqpoint{5.062743in}{3.130926in}}%
\pgfpathlineto{\pgfqpoint{5.067560in}{3.123861in}}%
\pgfpathlineto{\pgfqpoint{5.072645in}{3.117867in}}%
\pgfpathlineto{\pgfqpoint{5.075321in}{3.117566in}}%
\pgfpathlineto{\pgfqpoint{5.078265in}{3.119848in}}%
\pgfpathlineto{\pgfqpoint{5.083884in}{3.128279in}}%
\pgfpathlineto{\pgfqpoint{5.088166in}{3.132709in}}%
\pgfpathlineto{\pgfqpoint{5.090842in}{3.132733in}}%
\pgfpathlineto{\pgfqpoint{5.093786in}{3.130192in}}%
\pgfpathlineto{\pgfqpoint{5.105025in}{3.117415in}}%
\pgfpathlineto{\pgfqpoint{5.107701in}{3.118817in}}%
\pgfpathlineto{\pgfqpoint{5.111716in}{3.124198in}}%
\pgfpathlineto{\pgfqpoint{5.118138in}{3.132478in}}%
\pgfpathlineto{\pgfqpoint{5.120814in}{3.132893in}}%
\pgfpathlineto{\pgfqpoint{5.123490in}{3.131019in}}%
\pgfpathlineto{\pgfqpoint{5.128040in}{3.124425in}}%
\pgfpathlineto{\pgfqpoint{5.133392in}{3.117915in}}%
\pgfpathlineto{\pgfqpoint{5.136068in}{3.117539in}}%
\pgfpathlineto{\pgfqpoint{5.138744in}{3.119449in}}%
\pgfpathlineto{\pgfqpoint{5.143561in}{3.126494in}}%
\pgfpathlineto{\pgfqpoint{5.148645in}{3.132530in}}%
\pgfpathlineto{\pgfqpoint{5.151322in}{3.132866in}}%
\pgfpathlineto{\pgfqpoint{5.153998in}{3.130920in}}%
\pgfpathlineto{\pgfqpoint{5.158815in}{3.123852in}}%
\pgfpathlineto{\pgfqpoint{5.163899in}{3.117864in}}%
\pgfpathlineto{\pgfqpoint{5.166575in}{3.117568in}}%
\pgfpathlineto{\pgfqpoint{5.169519in}{3.119855in}}%
\pgfpathlineto{\pgfqpoint{5.175406in}{3.128679in}}%
\pgfpathlineto{\pgfqpoint{5.179420in}{3.132711in}}%
\pgfpathlineto{\pgfqpoint{5.182096in}{3.132731in}}%
\pgfpathlineto{\pgfqpoint{5.185040in}{3.130186in}}%
\pgfpathlineto{\pgfqpoint{5.196280in}{3.117415in}}%
\pgfpathlineto{\pgfqpoint{5.198956in}{3.118822in}}%
\pgfpathlineto{\pgfqpoint{5.202970in}{3.124207in}}%
\pgfpathlineto{\pgfqpoint{5.209392in}{3.132481in}}%
\pgfpathlineto{\pgfqpoint{5.212069in}{3.132891in}}%
\pgfpathlineto{\pgfqpoint{5.214745in}{3.131013in}}%
\pgfpathlineto{\pgfqpoint{5.219294in}{3.124417in}}%
\pgfpathlineto{\pgfqpoint{5.224646in}{3.117911in}}%
\pgfpathlineto{\pgfqpoint{5.227322in}{3.117541in}}%
\pgfpathlineto{\pgfqpoint{5.229998in}{3.119455in}}%
\pgfpathlineto{\pgfqpoint{5.234815in}{3.126503in}}%
\pgfpathlineto{\pgfqpoint{5.239900in}{3.132533in}}%
\pgfpathlineto{\pgfqpoint{5.242576in}{3.132864in}}%
\pgfpathlineto{\pgfqpoint{5.245252in}{3.130914in}}%
\pgfpathlineto{\pgfqpoint{5.250069in}{3.123844in}}%
\pgfpathlineto{\pgfqpoint{5.255153in}{3.117861in}}%
\pgfpathlineto{\pgfqpoint{5.257829in}{3.117570in}}%
\pgfpathlineto{\pgfqpoint{5.260773in}{3.119861in}}%
\pgfpathlineto{\pgfqpoint{5.266661in}{3.128687in}}%
\pgfpathlineto{\pgfqpoint{5.270675in}{3.132713in}}%
\pgfpathlineto{\pgfqpoint{5.273351in}{3.132729in}}%
\pgfpathlineto{\pgfqpoint{5.276294in}{3.130179in}}%
\pgfpathlineto{\pgfqpoint{5.287534in}{3.117416in}}%
\pgfpathlineto{\pgfqpoint{5.290210in}{3.118827in}}%
\pgfpathlineto{\pgfqpoint{5.294224in}{3.124215in}}%
\pgfpathlineto{\pgfqpoint{5.300647in}{3.132484in}}%
\pgfpathlineto{\pgfqpoint{5.303323in}{3.132890in}}%
\pgfpathlineto{\pgfqpoint{5.305999in}{3.131007in}}%
\pgfpathlineto{\pgfqpoint{5.310548in}{3.124408in}}%
\pgfpathlineto{\pgfqpoint{5.315900in}{3.117908in}}%
\pgfpathlineto{\pgfqpoint{5.318576in}{3.117543in}}%
\pgfpathlineto{\pgfqpoint{5.321253in}{3.119461in}}%
\pgfpathlineto{\pgfqpoint{5.326069in}{3.126512in}}%
\pgfpathlineto{\pgfqpoint{5.331154in}{3.132536in}}%
\pgfpathlineto{\pgfqpoint{5.333830in}{3.132862in}}%
\pgfpathlineto{\pgfqpoint{5.336506in}{3.130908in}}%
\pgfpathlineto{\pgfqpoint{5.341323in}{3.123835in}}%
\pgfpathlineto{\pgfqpoint{5.346408in}{3.117858in}}%
\pgfpathlineto{\pgfqpoint{5.349084in}{3.117572in}}%
\pgfpathlineto{\pgfqpoint{5.352027in}{3.119868in}}%
\pgfpathlineto{\pgfqpoint{5.357915in}{3.128695in}}%
\pgfpathlineto{\pgfqpoint{5.361929in}{3.132716in}}%
\pgfpathlineto{\pgfqpoint{5.364605in}{3.132726in}}%
\pgfpathlineto{\pgfqpoint{5.367549in}{3.130172in}}%
\pgfpathlineto{\pgfqpoint{5.378788in}{3.117416in}}%
\pgfpathlineto{\pgfqpoint{5.381464in}{3.118832in}}%
\pgfpathlineto{\pgfqpoint{5.385478in}{3.124224in}}%
\pgfpathlineto{\pgfqpoint{5.391901in}{3.132487in}}%
\pgfpathlineto{\pgfqpoint{5.394577in}{3.132888in}}%
\pgfpathlineto{\pgfqpoint{5.397253in}{3.131002in}}%
\pgfpathlineto{\pgfqpoint{5.401802in}{3.124399in}}%
\pgfpathlineto{\pgfqpoint{5.407155in}{3.117905in}}%
\pgfpathlineto{\pgfqpoint{5.409831in}{3.117544in}}%
\pgfpathlineto{\pgfqpoint{5.412507in}{3.119467in}}%
\pgfpathlineto{\pgfqpoint{5.417324in}{3.126520in}}%
\pgfpathlineto{\pgfqpoint{5.422408in}{3.132539in}}%
\pgfpathlineto{\pgfqpoint{5.425084in}{3.132860in}}%
\pgfpathlineto{\pgfqpoint{5.427760in}{3.130902in}}%
\pgfpathlineto{\pgfqpoint{5.432577in}{3.123826in}}%
\pgfpathlineto{\pgfqpoint{5.437662in}{3.117855in}}%
\pgfpathlineto{\pgfqpoint{5.440338in}{3.117574in}}%
\pgfpathlineto{\pgfqpoint{5.443282in}{3.119874in}}%
\pgfpathlineto{\pgfqpoint{5.449169in}{3.128703in}}%
\pgfpathlineto{\pgfqpoint{5.453183in}{3.132718in}}%
\pgfpathlineto{\pgfqpoint{5.455859in}{3.132724in}}%
\pgfpathlineto{\pgfqpoint{5.458803in}{3.130165in}}%
\pgfpathlineto{\pgfqpoint{5.470042in}{3.117417in}}%
\pgfpathlineto{\pgfqpoint{5.472719in}{3.118837in}}%
\pgfpathlineto{\pgfqpoint{5.476733in}{3.124233in}}%
\pgfpathlineto{\pgfqpoint{5.483155in}{3.132491in}}%
\pgfpathlineto{\pgfqpoint{5.485831in}{3.132887in}}%
\pgfpathlineto{\pgfqpoint{5.488507in}{3.130996in}}%
\pgfpathlineto{\pgfqpoint{5.493324in}{3.123962in}}%
\pgfpathlineto{\pgfqpoint{5.498409in}{3.117902in}}%
\pgfpathlineto{\pgfqpoint{5.501085in}{3.117546in}}%
\pgfpathlineto{\pgfqpoint{5.503761in}{3.119473in}}%
\pgfpathlineto{\pgfqpoint{5.508578in}{3.126529in}}%
\pgfpathlineto{\pgfqpoint{5.513663in}{3.132542in}}%
\pgfpathlineto{\pgfqpoint{5.516339in}{3.132858in}}%
\pgfpathlineto{\pgfqpoint{5.519015in}{3.130896in}}%
\pgfpathlineto{\pgfqpoint{5.523832in}{3.123817in}}%
\pgfpathlineto{\pgfqpoint{5.528916in}{3.117852in}}%
\pgfpathlineto{\pgfqpoint{5.531592in}{3.117575in}}%
\pgfpathlineto{\pgfqpoint{5.534536in}{3.119881in}}%
\pgfpathlineto{\pgfqpoint{5.540423in}{3.128711in}}%
\pgfpathlineto{\pgfqpoint{5.544437in}{3.132721in}}%
\pgfpathlineto{\pgfqpoint{5.547114in}{3.132721in}}%
\pgfpathlineto{\pgfqpoint{5.550057in}{3.130158in}}%
\pgfpathlineto{\pgfqpoint{5.561297in}{3.117418in}}%
\pgfpathlineto{\pgfqpoint{5.563973in}{3.118843in}}%
\pgfpathlineto{\pgfqpoint{5.567987in}{3.124242in}}%
\pgfpathlineto{\pgfqpoint{5.574410in}{3.132494in}}%
\pgfpathlineto{\pgfqpoint{5.577086in}{3.132885in}}%
\pgfpathlineto{\pgfqpoint{5.579762in}{3.130990in}}%
\pgfpathlineto{\pgfqpoint{5.584579in}{3.123953in}}%
\pgfpathlineto{\pgfqpoint{5.589663in}{3.117899in}}%
\pgfpathlineto{\pgfqpoint{5.592339in}{3.117548in}}%
\pgfpathlineto{\pgfqpoint{5.595015in}{3.119479in}}%
\pgfpathlineto{\pgfqpoint{5.599832in}{3.126538in}}%
\pgfpathlineto{\pgfqpoint{5.604917in}{3.132545in}}%
\pgfpathlineto{\pgfqpoint{5.607593in}{3.132857in}}%
\pgfpathlineto{\pgfqpoint{5.610269in}{3.130890in}}%
\pgfpathlineto{\pgfqpoint{5.615086in}{3.123809in}}%
\pgfpathlineto{\pgfqpoint{5.620170in}{3.117849in}}%
\pgfpathlineto{\pgfqpoint{5.622847in}{3.117577in}}%
\pgfpathlineto{\pgfqpoint{5.625790in}{3.119887in}}%
\pgfpathlineto{\pgfqpoint{5.631678in}{3.128719in}}%
\pgfpathlineto{\pgfqpoint{5.635692in}{3.132723in}}%
\pgfpathlineto{\pgfqpoint{5.638368in}{3.132719in}}%
\pgfpathlineto{\pgfqpoint{5.641311in}{3.130151in}}%
\pgfpathlineto{\pgfqpoint{5.652551in}{3.117418in}}%
\pgfpathlineto{\pgfqpoint{5.655227in}{3.118848in}}%
\pgfpathlineto{\pgfqpoint{5.659241in}{3.124251in}}%
\pgfpathlineto{\pgfqpoint{5.665664in}{3.132497in}}%
\pgfpathlineto{\pgfqpoint{5.668340in}{3.132883in}}%
\pgfpathlineto{\pgfqpoint{5.671016in}{3.130984in}}%
\pgfpathlineto{\pgfqpoint{5.675833in}{3.123945in}}%
\pgfpathlineto{\pgfqpoint{5.680917in}{3.117896in}}%
\pgfpathlineto{\pgfqpoint{5.683594in}{3.117549in}}%
\pgfpathlineto{\pgfqpoint{5.686270in}{3.119485in}}%
\pgfpathlineto{\pgfqpoint{5.691087in}{3.126547in}}%
\pgfpathlineto{\pgfqpoint{5.696171in}{3.132549in}}%
\pgfpathlineto{\pgfqpoint{5.698847in}{3.132855in}}%
\pgfpathlineto{\pgfqpoint{5.701791in}{3.130579in}}%
\pgfpathlineto{\pgfqpoint{5.707411in}{3.122150in}}%
\pgfpathlineto{\pgfqpoint{5.711692in}{3.117713in}}%
\pgfpathlineto{\pgfqpoint{5.714368in}{3.117682in}}%
\pgfpathlineto{\pgfqpoint{5.717312in}{3.120218in}}%
\pgfpathlineto{\pgfqpoint{5.728552in}{3.133005in}}%
\pgfpathlineto{\pgfqpoint{5.731228in}{3.131608in}}%
\pgfpathlineto{\pgfqpoint{5.735242in}{3.126232in}}%
\pgfpathlineto{\pgfqpoint{5.741664in}{3.117945in}}%
\pgfpathlineto{\pgfqpoint{5.744340in}{3.117524in}}%
\pgfpathlineto{\pgfqpoint{5.747017in}{3.119392in}}%
\pgfpathlineto{\pgfqpoint{5.751566in}{3.125982in}}%
\pgfpathlineto{\pgfqpoint{5.756918in}{3.132500in}}%
\pgfpathlineto{\pgfqpoint{5.759594in}{3.132882in}}%
\pgfpathlineto{\pgfqpoint{5.762270in}{3.130978in}}%
\pgfpathlineto{\pgfqpoint{5.767087in}{3.123936in}}%
\pgfpathlineto{\pgfqpoint{5.772172in}{3.117893in}}%
\pgfpathlineto{\pgfqpoint{5.774848in}{3.117551in}}%
\pgfpathlineto{\pgfqpoint{5.777524in}{3.119491in}}%
\pgfpathlineto{\pgfqpoint{5.782341in}{3.126555in}}%
\pgfpathlineto{\pgfqpoint{5.787425in}{3.132552in}}%
\pgfpathlineto{\pgfqpoint{5.790101in}{3.132853in}}%
\pgfpathlineto{\pgfqpoint{5.793045in}{3.130572in}}%
\pgfpathlineto{\pgfqpoint{5.798665in}{3.122142in}}%
\pgfpathlineto{\pgfqpoint{5.802947in}{3.117711in}}%
\pgfpathlineto{\pgfqpoint{5.805623in}{3.117685in}}%
\pgfpathlineto{\pgfqpoint{5.808566in}{3.120225in}}%
\pgfpathlineto{\pgfqpoint{5.819806in}{3.133004in}}%
\pgfpathlineto{\pgfqpoint{5.822482in}{3.131603in}}%
\pgfpathlineto{\pgfqpoint{5.826496in}{3.126223in}}%
\pgfpathlineto{\pgfqpoint{5.832919in}{3.117942in}}%
\pgfpathlineto{\pgfqpoint{5.835595in}{3.117525in}}%
\pgfpathlineto{\pgfqpoint{5.838271in}{3.119398in}}%
\pgfpathlineto{\pgfqpoint{5.842820in}{3.125991in}}%
\pgfpathlineto{\pgfqpoint{5.848172in}{3.132503in}}%
\pgfpathlineto{\pgfqpoint{5.850848in}{3.132880in}}%
\pgfpathlineto{\pgfqpoint{5.853524in}{3.130972in}}%
\pgfpathlineto{\pgfqpoint{5.858341in}{3.123927in}}%
\pgfpathlineto{\pgfqpoint{5.863426in}{3.117890in}}%
\pgfpathlineto{\pgfqpoint{5.866102in}{3.117553in}}%
\pgfpathlineto{\pgfqpoint{5.868778in}{3.119497in}}%
\pgfpathlineto{\pgfqpoint{5.873595in}{3.126564in}}%
\pgfpathlineto{\pgfqpoint{5.878680in}{3.132555in}}%
\pgfpathlineto{\pgfqpoint{5.881356in}{3.132851in}}%
\pgfpathlineto{\pgfqpoint{5.884299in}{3.130566in}}%
\pgfpathlineto{\pgfqpoint{5.890187in}{3.121742in}}%
\pgfpathlineto{\pgfqpoint{5.894201in}{3.117708in}}%
\pgfpathlineto{\pgfqpoint{5.896877in}{3.117687in}}%
\pgfpathlineto{\pgfqpoint{5.899821in}{3.120232in}}%
\pgfpathlineto{\pgfqpoint{5.911060in}{3.133004in}}%
\pgfpathlineto{\pgfqpoint{5.913736in}{3.131598in}}%
\pgfpathlineto{\pgfqpoint{5.917750in}{3.126215in}}%
\pgfpathlineto{\pgfqpoint{5.924173in}{3.117939in}}%
\pgfpathlineto{\pgfqpoint{5.926849in}{3.117527in}}%
\pgfpathlineto{\pgfqpoint{5.929525in}{3.119404in}}%
\pgfpathlineto{\pgfqpoint{5.934074in}{3.126000in}}%
\pgfpathlineto{\pgfqpoint{5.939427in}{3.132507in}}%
\pgfpathlineto{\pgfqpoint{5.942103in}{3.132878in}}%
\pgfpathlineto{\pgfqpoint{5.944779in}{3.130966in}}%
\pgfpathlineto{\pgfqpoint{5.949596in}{3.123918in}}%
\pgfpathlineto{\pgfqpoint{5.954680in}{3.117886in}}%
\pgfpathlineto{\pgfqpoint{5.957356in}{3.117555in}}%
\pgfpathlineto{\pgfqpoint{5.960032in}{3.119503in}}%
\pgfpathlineto{\pgfqpoint{5.964849in}{3.126573in}}%
\pgfpathlineto{\pgfqpoint{5.969934in}{3.132558in}}%
\pgfpathlineto{\pgfqpoint{5.972610in}{3.132849in}}%
\pgfpathlineto{\pgfqpoint{5.975554in}{3.130559in}}%
\pgfpathlineto{\pgfqpoint{5.981441in}{3.121734in}}%
\pgfpathlineto{\pgfqpoint{5.985455in}{3.117706in}}%
\pgfpathlineto{\pgfqpoint{5.988131in}{3.117690in}}%
\pgfpathlineto{\pgfqpoint{5.991075in}{3.120238in}}%
\pgfpathlineto{\pgfqpoint{6.002314in}{3.133003in}}%
\pgfpathlineto{\pgfqpoint{6.004991in}{3.131593in}}%
\pgfpathlineto{\pgfqpoint{6.009005in}{3.126206in}}%
\pgfpathlineto{\pgfqpoint{6.015427in}{3.117935in}}%
\pgfpathlineto{\pgfqpoint{6.018103in}{3.117529in}}%
\pgfpathlineto{\pgfqpoint{6.020779in}{3.119410in}}%
\pgfpathlineto{\pgfqpoint{6.025329in}{3.126009in}}%
\pgfpathlineto{\pgfqpoint{6.030681in}{3.132510in}}%
\pgfpathlineto{\pgfqpoint{6.033357in}{3.132877in}}%
\pgfpathlineto{\pgfqpoint{6.036033in}{3.130960in}}%
\pgfpathlineto{\pgfqpoint{6.040850in}{3.123909in}}%
\pgfpathlineto{\pgfqpoint{6.045935in}{3.117883in}}%
\pgfpathlineto{\pgfqpoint{6.048611in}{3.117556in}}%
\pgfpathlineto{\pgfqpoint{6.051287in}{3.119509in}}%
\pgfpathlineto{\pgfqpoint{6.056104in}{3.126582in}}%
\pgfpathlineto{\pgfqpoint{6.061188in}{3.132561in}}%
\pgfpathlineto{\pgfqpoint{6.063864in}{3.132848in}}%
\pgfpathlineto{\pgfqpoint{6.066808in}{3.130553in}}%
\pgfpathlineto{\pgfqpoint{6.072695in}{3.121726in}}%
\pgfpathlineto{\pgfqpoint{6.076709in}{3.117704in}}%
\pgfpathlineto{\pgfqpoint{6.079385in}{3.117692in}}%
\pgfpathlineto{\pgfqpoint{6.082329in}{3.120245in}}%
\pgfpathlineto{\pgfqpoint{6.093569in}{3.133003in}}%
\pgfpathlineto{\pgfqpoint{6.096245in}{3.131588in}}%
\pgfpathlineto{\pgfqpoint{6.100259in}{3.126197in}}%
\pgfpathlineto{\pgfqpoint{6.106681in}{3.117932in}}%
\pgfpathlineto{\pgfqpoint{6.109358in}{3.117530in}}%
\pgfpathlineto{\pgfqpoint{6.112034in}{3.119416in}}%
\pgfpathlineto{\pgfqpoint{6.116583in}{3.126018in}}%
\pgfpathlineto{\pgfqpoint{6.121935in}{3.132513in}}%
\pgfpathlineto{\pgfqpoint{6.124611in}{3.132875in}}%
\pgfpathlineto{\pgfqpoint{6.127287in}{3.130954in}}%
\pgfpathlineto{\pgfqpoint{6.132104in}{3.123901in}}%
\pgfpathlineto{\pgfqpoint{6.137189in}{3.117880in}}%
\pgfpathlineto{\pgfqpoint{6.139865in}{3.117558in}}%
\pgfpathlineto{\pgfqpoint{6.142541in}{3.119515in}}%
\pgfpathlineto{\pgfqpoint{6.147358in}{3.126590in}}%
\pgfpathlineto{\pgfqpoint{6.152442in}{3.132564in}}%
\pgfpathlineto{\pgfqpoint{6.155119in}{3.132846in}}%
\pgfpathlineto{\pgfqpoint{6.158062in}{3.130546in}}%
\pgfpathlineto{\pgfqpoint{6.163950in}{3.121718in}}%
\pgfpathlineto{\pgfqpoint{6.167964in}{3.117701in}}%
\pgfpathlineto{\pgfqpoint{6.170640in}{3.117694in}}%
\pgfpathlineto{\pgfqpoint{6.173583in}{3.120252in}}%
\pgfpathlineto{\pgfqpoint{6.184823in}{3.133002in}}%
\pgfpathlineto{\pgfqpoint{6.187499in}{3.131583in}}%
\pgfpathlineto{\pgfqpoint{6.191513in}{3.126188in}}%
\pgfpathlineto{\pgfqpoint{6.197936in}{3.117929in}}%
\pgfpathlineto{\pgfqpoint{6.200612in}{3.117532in}}%
\pgfpathlineto{\pgfqpoint{6.203288in}{3.119422in}}%
\pgfpathlineto{\pgfqpoint{6.208105in}{3.126455in}}%
\pgfpathlineto{\pgfqpoint{6.213189in}{3.132516in}}%
\pgfpathlineto{\pgfqpoint{6.215865in}{3.132873in}}%
\pgfpathlineto{\pgfqpoint{6.218542in}{3.130948in}}%
\pgfpathlineto{\pgfqpoint{6.223358in}{3.123892in}}%
\pgfpathlineto{\pgfqpoint{6.228443in}{3.117877in}}%
\pgfpathlineto{\pgfqpoint{6.231119in}{3.117560in}}%
\pgfpathlineto{\pgfqpoint{6.233795in}{3.119521in}}%
\pgfpathlineto{\pgfqpoint{6.238612in}{3.126599in}}%
\pgfpathlineto{\pgfqpoint{6.243697in}{3.132567in}}%
\pgfpathlineto{\pgfqpoint{6.246373in}{3.132844in}}%
\pgfpathlineto{\pgfqpoint{6.249316in}{3.130540in}}%
\pgfpathlineto{\pgfqpoint{6.255204in}{3.121710in}}%
\pgfpathlineto{\pgfqpoint{6.259218in}{3.117699in}}%
\pgfpathlineto{\pgfqpoint{6.261894in}{3.117697in}}%
\pgfpathlineto{\pgfqpoint{6.264838in}{3.120259in}}%
\pgfpathlineto{\pgfqpoint{6.276077in}{3.133001in}}%
\pgfpathlineto{\pgfqpoint{6.278753in}{3.131578in}}%
\pgfpathlineto{\pgfqpoint{6.282767in}{3.126179in}}%
\pgfpathlineto{\pgfqpoint{6.289190in}{3.117926in}}%
\pgfpathlineto{\pgfqpoint{6.291866in}{3.117533in}}%
\pgfpathlineto{\pgfqpoint{6.294542in}{3.119428in}}%
\pgfpathlineto{\pgfqpoint{6.299359in}{3.126463in}}%
\pgfpathlineto{\pgfqpoint{6.304444in}{3.132519in}}%
\pgfpathlineto{\pgfqpoint{6.307120in}{3.132872in}}%
\pgfpathlineto{\pgfqpoint{6.309796in}{3.130941in}}%
\pgfpathlineto{\pgfqpoint{6.314613in}{3.123883in}}%
\pgfpathlineto{\pgfqpoint{6.319697in}{3.117874in}}%
\pgfpathlineto{\pgfqpoint{6.322373in}{3.117562in}}%
\pgfpathlineto{\pgfqpoint{6.325049in}{3.119528in}}%
\pgfpathlineto{\pgfqpoint{6.329866in}{3.126608in}}%
\pgfpathlineto{\pgfqpoint{6.334951in}{3.132570in}}%
\pgfpathlineto{\pgfqpoint{6.337627in}{3.132842in}}%
\pgfpathlineto{\pgfqpoint{6.340571in}{3.130533in}}%
\pgfpathlineto{\pgfqpoint{6.346458in}{3.121702in}}%
\pgfpathlineto{\pgfqpoint{6.350472in}{3.117696in}}%
\pgfpathlineto{\pgfqpoint{6.353148in}{3.117699in}}%
\pgfpathlineto{\pgfqpoint{6.356092in}{3.120266in}}%
\pgfpathlineto{\pgfqpoint{6.367332in}{3.133001in}}%
\pgfpathlineto{\pgfqpoint{6.370008in}{3.131572in}}%
\pgfpathlineto{\pgfqpoint{6.374022in}{3.126170in}}%
\pgfpathlineto{\pgfqpoint{6.380444in}{3.117923in}}%
\pgfpathlineto{\pgfqpoint{6.383120in}{3.117535in}}%
\pgfpathlineto{\pgfqpoint{6.385796in}{3.119434in}}%
\pgfpathlineto{\pgfqpoint{6.390613in}{3.126472in}}%
\pgfpathlineto{\pgfqpoint{6.395698in}{3.132522in}}%
\pgfpathlineto{\pgfqpoint{6.398374in}{3.132870in}}%
\pgfpathlineto{\pgfqpoint{6.401050in}{3.130935in}}%
\pgfpathlineto{\pgfqpoint{6.405867in}{3.123874in}}%
\pgfpathlineto{\pgfqpoint{6.410952in}{3.117871in}}%
\pgfpathlineto{\pgfqpoint{6.413628in}{3.117564in}}%
\pgfpathlineto{\pgfqpoint{6.416571in}{3.119839in}}%
\pgfpathlineto{\pgfqpoint{6.422191in}{3.128267in}}%
\pgfpathlineto{\pgfqpoint{6.426473in}{3.132705in}}%
\pgfpathlineto{\pgfqpoint{6.429149in}{3.132737in}}%
\pgfpathlineto{\pgfqpoint{6.432093in}{3.130203in}}%
\pgfpathlineto{\pgfqpoint{6.443332in}{3.117414in}}%
\pgfpathlineto{\pgfqpoint{6.446008in}{3.118809in}}%
\pgfpathlineto{\pgfqpoint{6.450022in}{3.124184in}}%
\pgfpathlineto{\pgfqpoint{6.456445in}{3.132473in}}%
\pgfpathlineto{\pgfqpoint{6.459121in}{3.132895in}}%
\pgfpathlineto{\pgfqpoint{6.461797in}{3.131028in}}%
\pgfpathlineto{\pgfqpoint{6.466346in}{3.124439in}}%
\pgfpathlineto{\pgfqpoint{6.471699in}{3.117919in}}%
\pgfpathlineto{\pgfqpoint{6.474375in}{3.117537in}}%
\pgfpathlineto{\pgfqpoint{6.477051in}{3.119440in}}%
\pgfpathlineto{\pgfqpoint{6.481868in}{3.126481in}}%
\pgfpathlineto{\pgfqpoint{6.486952in}{3.132525in}}%
\pgfpathlineto{\pgfqpoint{6.489628in}{3.132868in}}%
\pgfpathlineto{\pgfqpoint{6.492304in}{3.130929in}}%
\pgfpathlineto{\pgfqpoint{6.497121in}{3.123866in}}%
\pgfpathlineto{\pgfqpoint{6.502206in}{3.117868in}}%
\pgfpathlineto{\pgfqpoint{6.504882in}{3.117565in}}%
\pgfpathlineto{\pgfqpoint{6.507826in}{3.119845in}}%
\pgfpathlineto{\pgfqpoint{6.513445in}{3.128275in}}%
\pgfpathlineto{\pgfqpoint{6.517727in}{3.132707in}}%
\pgfpathlineto{\pgfqpoint{6.520403in}{3.132735in}}%
\pgfpathlineto{\pgfqpoint{6.523347in}{3.130196in}}%
\pgfpathlineto{\pgfqpoint{6.534586in}{3.117415in}}%
\pgfpathlineto{\pgfqpoint{6.537262in}{3.118814in}}%
\pgfpathlineto{\pgfqpoint{6.541277in}{3.124193in}}%
\pgfpathlineto{\pgfqpoint{6.547699in}{3.132476in}}%
\pgfpathlineto{\pgfqpoint{6.550375in}{3.132894in}}%
\pgfpathlineto{\pgfqpoint{6.553051in}{3.131022in}}%
\pgfpathlineto{\pgfqpoint{6.557601in}{3.124430in}}%
\pgfpathlineto{\pgfqpoint{6.562953in}{3.117916in}}%
\pgfpathlineto{\pgfqpoint{6.565629in}{3.117538in}}%
\pgfpathlineto{\pgfqpoint{6.568305in}{3.119446in}}%
\pgfpathlineto{\pgfqpoint{6.573122in}{3.126490in}}%
\pgfpathlineto{\pgfqpoint{6.578206in}{3.132529in}}%
\pgfpathlineto{\pgfqpoint{6.580883in}{3.132866in}}%
\pgfpathlineto{\pgfqpoint{6.583559in}{3.130923in}}%
\pgfpathlineto{\pgfqpoint{6.588376in}{3.123857in}}%
\pgfpathlineto{\pgfqpoint{6.593460in}{3.117865in}}%
\pgfpathlineto{\pgfqpoint{6.596136in}{3.117567in}}%
\pgfpathlineto{\pgfqpoint{6.599080in}{3.119851in}}%
\pgfpathlineto{\pgfqpoint{6.604700in}{3.128283in}}%
\pgfpathlineto{\pgfqpoint{6.608981in}{3.132710in}}%
\pgfpathlineto{\pgfqpoint{6.611657in}{3.132732in}}%
\pgfpathlineto{\pgfqpoint{6.614601in}{3.130189in}}%
\pgfpathlineto{\pgfqpoint{6.625841in}{3.117415in}}%
\pgfpathlineto{\pgfqpoint{6.628517in}{3.118820in}}%
\pgfpathlineto{\pgfqpoint{6.632531in}{3.124202in}}%
\pgfpathlineto{\pgfqpoint{6.638953in}{3.132479in}}%
\pgfpathlineto{\pgfqpoint{6.641630in}{3.132892in}}%
\pgfpathlineto{\pgfqpoint{6.644306in}{3.131016in}}%
\pgfpathlineto{\pgfqpoint{6.648855in}{3.124421in}}%
\pgfpathlineto{\pgfqpoint{6.654207in}{3.117913in}}%
\pgfpathlineto{\pgfqpoint{6.656883in}{3.117540in}}%
\pgfpathlineto{\pgfqpoint{6.659559in}{3.119452in}}%
\pgfpathlineto{\pgfqpoint{6.663306in}{3.124778in}}%
\pgfpathlineto{\pgfqpoint{6.663306in}{3.124778in}}%
\pgfusepath{stroke}%
\end{pgfscope}%
\begin{pgfscope}%
\pgfpathrectangle{\pgfqpoint{0.467797in}{2.292089in}}{\pgfqpoint{6.490533in}{1.666241in}}%
\pgfusepath{clip}%
\pgfsetrectcap%
\pgfsetroundjoin%
\pgfsetlinewidth{1.505625pt}%
\definecolor{currentstroke}{rgb}{0.090196,0.745098,0.811765}%
\pgfsetstrokecolor{currentstroke}%
\pgfsetdash{}{0pt}%
\pgfpathmoveto{\pgfqpoint{0.762821in}{3.125209in}}%
\pgfpathlineto{\pgfqpoint{0.768441in}{3.132297in}}%
\pgfpathlineto{\pgfqpoint{0.771117in}{3.132739in}}%
\pgfpathlineto{\pgfqpoint{0.773793in}{3.130846in}}%
\pgfpathlineto{\pgfqpoint{0.778610in}{3.123786in}}%
\pgfpathlineto{\pgfqpoint{0.783427in}{3.118071in}}%
\pgfpathlineto{\pgfqpoint{0.786103in}{3.117706in}}%
\pgfpathlineto{\pgfqpoint{0.788779in}{3.119666in}}%
\pgfpathlineto{\pgfqpoint{0.793596in}{3.126768in}}%
\pgfpathlineto{\pgfqpoint{0.798413in}{3.132396in}}%
\pgfpathlineto{\pgfqpoint{0.801089in}{3.132685in}}%
\pgfpathlineto{\pgfqpoint{0.803765in}{3.130657in}}%
\pgfpathlineto{\pgfqpoint{0.808849in}{3.123099in}}%
\pgfpathlineto{\pgfqpoint{0.813399in}{3.117977in}}%
\pgfpathlineto{\pgfqpoint{0.816075in}{3.117764in}}%
\pgfpathlineto{\pgfqpoint{0.819019in}{3.120176in}}%
\pgfpathlineto{\pgfqpoint{0.830526in}{3.132788in}}%
\pgfpathlineto{\pgfqpoint{0.833202in}{3.131054in}}%
\pgfpathlineto{\pgfqpoint{0.837483in}{3.124954in}}%
\pgfpathlineto{\pgfqpoint{0.842836in}{3.118190in}}%
\pgfpathlineto{\pgfqpoint{0.845512in}{3.117651in}}%
\pgfpathlineto{\pgfqpoint{0.848188in}{3.119455in}}%
\pgfpathlineto{\pgfqpoint{0.852737in}{3.126032in}}%
\pgfpathlineto{\pgfqpoint{0.857822in}{3.132282in}}%
\pgfpathlineto{\pgfqpoint{0.860498in}{3.132745in}}%
\pgfpathlineto{\pgfqpoint{0.863174in}{3.130873in}}%
\pgfpathlineto{\pgfqpoint{0.867991in}{3.123824in}}%
\pgfpathlineto{\pgfqpoint{0.872808in}{3.118086in}}%
\pgfpathlineto{\pgfqpoint{0.875484in}{3.117698in}}%
\pgfpathlineto{\pgfqpoint{0.878160in}{3.119639in}}%
\pgfpathlineto{\pgfqpoint{0.882977in}{3.126730in}}%
\pgfpathlineto{\pgfqpoint{0.887794in}{3.132382in}}%
\pgfpathlineto{\pgfqpoint{0.890470in}{3.132693in}}%
\pgfpathlineto{\pgfqpoint{0.893146in}{3.130685in}}%
\pgfpathlineto{\pgfqpoint{0.898230in}{3.123136in}}%
\pgfpathlineto{\pgfqpoint{0.902780in}{3.117990in}}%
\pgfpathlineto{\pgfqpoint{0.905456in}{3.117755in}}%
\pgfpathlineto{\pgfqpoint{0.908400in}{3.120147in}}%
\pgfpathlineto{\pgfqpoint{0.919907in}{3.132793in}}%
\pgfpathlineto{\pgfqpoint{0.922583in}{3.131079in}}%
\pgfpathlineto{\pgfqpoint{0.926865in}{3.124994in}}%
\pgfpathlineto{\pgfqpoint{0.932484in}{3.118043in}}%
\pgfpathlineto{\pgfqpoint{0.935160in}{3.117722in}}%
\pgfpathlineto{\pgfqpoint{0.937836in}{3.119721in}}%
\pgfpathlineto{\pgfqpoint{0.942921in}{3.127264in}}%
\pgfpathlineto{\pgfqpoint{0.947470in}{3.132422in}}%
\pgfpathlineto{\pgfqpoint{0.950146in}{3.132668in}}%
\pgfpathlineto{\pgfqpoint{0.953090in}{3.130287in}}%
\pgfpathlineto{\pgfqpoint{0.964597in}{3.117623in}}%
\pgfpathlineto{\pgfqpoint{0.967273in}{3.119327in}}%
\pgfpathlineto{\pgfqpoint{0.971555in}{3.125406in}}%
\pgfpathlineto{\pgfqpoint{0.977175in}{3.132368in}}%
\pgfpathlineto{\pgfqpoint{0.979851in}{3.132701in}}%
\pgfpathlineto{\pgfqpoint{0.982527in}{3.130712in}}%
\pgfpathlineto{\pgfqpoint{0.987611in}{3.123174in}}%
\pgfpathlineto{\pgfqpoint{0.992161in}{3.118003in}}%
\pgfpathlineto{\pgfqpoint{0.994837in}{3.117747in}}%
\pgfpathlineto{\pgfqpoint{0.997781in}{3.120117in}}%
\pgfpathlineto{\pgfqpoint{1.009288in}{3.132798in}}%
\pgfpathlineto{\pgfqpoint{1.011964in}{3.131104in}}%
\pgfpathlineto{\pgfqpoint{1.016246in}{3.125033in}}%
\pgfpathlineto{\pgfqpoint{1.021865in}{3.118057in}}%
\pgfpathlineto{\pgfqpoint{1.024541in}{3.117714in}}%
\pgfpathlineto{\pgfqpoint{1.027217in}{3.119693in}}%
\pgfpathlineto{\pgfqpoint{1.032302in}{3.127226in}}%
\pgfpathlineto{\pgfqpoint{1.036851in}{3.132409in}}%
\pgfpathlineto{\pgfqpoint{1.039527in}{3.132677in}}%
\pgfpathlineto{\pgfqpoint{1.042203in}{3.130629in}}%
\pgfpathlineto{\pgfqpoint{1.047288in}{3.123061in}}%
\pgfpathlineto{\pgfqpoint{1.051837in}{3.117964in}}%
\pgfpathlineto{\pgfqpoint{1.054513in}{3.117773in}}%
\pgfpathlineto{\pgfqpoint{1.057457in}{3.120206in}}%
\pgfpathlineto{\pgfqpoint{1.068697in}{3.132832in}}%
\pgfpathlineto{\pgfqpoint{1.071373in}{3.131299in}}%
\pgfpathlineto{\pgfqpoint{1.075387in}{3.125778in}}%
\pgfpathlineto{\pgfqpoint{1.081274in}{3.118175in}}%
\pgfpathlineto{\pgfqpoint{1.083950in}{3.117657in}}%
\pgfpathlineto{\pgfqpoint{1.086626in}{3.119481in}}%
\pgfpathlineto{\pgfqpoint{1.091176in}{3.126071in}}%
\pgfpathlineto{\pgfqpoint{1.096260in}{3.132297in}}%
\pgfpathlineto{\pgfqpoint{1.098936in}{3.132739in}}%
\pgfpathlineto{\pgfqpoint{1.101612in}{3.130846in}}%
\pgfpathlineto{\pgfqpoint{1.106429in}{3.123786in}}%
\pgfpathlineto{\pgfqpoint{1.111246in}{3.118071in}}%
\pgfpathlineto{\pgfqpoint{1.113922in}{3.117706in}}%
\pgfpathlineto{\pgfqpoint{1.116598in}{3.119666in}}%
\pgfpathlineto{\pgfqpoint{1.121415in}{3.126768in}}%
\pgfpathlineto{\pgfqpoint{1.126232in}{3.132396in}}%
\pgfpathlineto{\pgfqpoint{1.128908in}{3.132685in}}%
\pgfpathlineto{\pgfqpoint{1.131584in}{3.130657in}}%
\pgfpathlineto{\pgfqpoint{1.136669in}{3.123099in}}%
\pgfpathlineto{\pgfqpoint{1.141218in}{3.117977in}}%
\pgfpathlineto{\pgfqpoint{1.143894in}{3.117764in}}%
\pgfpathlineto{\pgfqpoint{1.146838in}{3.120176in}}%
\pgfpathlineto{\pgfqpoint{1.158345in}{3.132788in}}%
\pgfpathlineto{\pgfqpoint{1.161021in}{3.131054in}}%
\pgfpathlineto{\pgfqpoint{1.165303in}{3.124954in}}%
\pgfpathlineto{\pgfqpoint{1.170655in}{3.118190in}}%
\pgfpathlineto{\pgfqpoint{1.173331in}{3.117651in}}%
\pgfpathlineto{\pgfqpoint{1.176007in}{3.119455in}}%
\pgfpathlineto{\pgfqpoint{1.180557in}{3.126032in}}%
\pgfpathlineto{\pgfqpoint{1.185641in}{3.132282in}}%
\pgfpathlineto{\pgfqpoint{1.188317in}{3.132745in}}%
\pgfpathlineto{\pgfqpoint{1.190993in}{3.130873in}}%
\pgfpathlineto{\pgfqpoint{1.195810in}{3.123824in}}%
\pgfpathlineto{\pgfqpoint{1.200627in}{3.118086in}}%
\pgfpathlineto{\pgfqpoint{1.203303in}{3.117698in}}%
\pgfpathlineto{\pgfqpoint{1.205979in}{3.119639in}}%
\pgfpathlineto{\pgfqpoint{1.210796in}{3.126730in}}%
\pgfpathlineto{\pgfqpoint{1.215613in}{3.132382in}}%
\pgfpathlineto{\pgfqpoint{1.218289in}{3.132693in}}%
\pgfpathlineto{\pgfqpoint{1.220965in}{3.130685in}}%
\pgfpathlineto{\pgfqpoint{1.226050in}{3.123136in}}%
\pgfpathlineto{\pgfqpoint{1.230599in}{3.117990in}}%
\pgfpathlineto{\pgfqpoint{1.233275in}{3.117755in}}%
\pgfpathlineto{\pgfqpoint{1.236219in}{3.120147in}}%
\pgfpathlineto{\pgfqpoint{1.247726in}{3.132793in}}%
\pgfpathlineto{\pgfqpoint{1.250402in}{3.131079in}}%
\pgfpathlineto{\pgfqpoint{1.254684in}{3.124994in}}%
\pgfpathlineto{\pgfqpoint{1.260304in}{3.118043in}}%
\pgfpathlineto{\pgfqpoint{1.262980in}{3.117722in}}%
\pgfpathlineto{\pgfqpoint{1.265656in}{3.119721in}}%
\pgfpathlineto{\pgfqpoint{1.270741in}{3.127264in}}%
\pgfpathlineto{\pgfqpoint{1.275290in}{3.132422in}}%
\pgfpathlineto{\pgfqpoint{1.277966in}{3.132668in}}%
\pgfpathlineto{\pgfqpoint{1.280910in}{3.130287in}}%
\pgfpathlineto{\pgfqpoint{1.292417in}{3.117623in}}%
\pgfpathlineto{\pgfqpoint{1.295093in}{3.119327in}}%
\pgfpathlineto{\pgfqpoint{1.299375in}{3.125406in}}%
\pgfpathlineto{\pgfqpoint{1.304994in}{3.132368in}}%
\pgfpathlineto{\pgfqpoint{1.307670in}{3.132701in}}%
\pgfpathlineto{\pgfqpoint{1.310347in}{3.130712in}}%
\pgfpathlineto{\pgfqpoint{1.315431in}{3.123174in}}%
\pgfpathlineto{\pgfqpoint{1.319980in}{3.118003in}}%
\pgfpathlineto{\pgfqpoint{1.322656in}{3.117747in}}%
\pgfpathlineto{\pgfqpoint{1.325600in}{3.120117in}}%
\pgfpathlineto{\pgfqpoint{1.337107in}{3.132798in}}%
\pgfpathlineto{\pgfqpoint{1.339783in}{3.131104in}}%
\pgfpathlineto{\pgfqpoint{1.344065in}{3.125033in}}%
\pgfpathlineto{\pgfqpoint{1.349685in}{3.118057in}}%
\pgfpathlineto{\pgfqpoint{1.352361in}{3.117714in}}%
\pgfpathlineto{\pgfqpoint{1.355037in}{3.119693in}}%
\pgfpathlineto{\pgfqpoint{1.360122in}{3.127226in}}%
\pgfpathlineto{\pgfqpoint{1.364671in}{3.132409in}}%
\pgfpathlineto{\pgfqpoint{1.367347in}{3.132677in}}%
\pgfpathlineto{\pgfqpoint{1.370023in}{3.130629in}}%
\pgfpathlineto{\pgfqpoint{1.375108in}{3.123061in}}%
\pgfpathlineto{\pgfqpoint{1.379657in}{3.117964in}}%
\pgfpathlineto{\pgfqpoint{1.382333in}{3.117773in}}%
\pgfpathlineto{\pgfqpoint{1.385277in}{3.120206in}}%
\pgfpathlineto{\pgfqpoint{1.396516in}{3.132832in}}%
\pgfpathlineto{\pgfqpoint{1.399192in}{3.131299in}}%
\pgfpathlineto{\pgfqpoint{1.403206in}{3.125778in}}%
\pgfpathlineto{\pgfqpoint{1.409094in}{3.118175in}}%
\pgfpathlineto{\pgfqpoint{1.411770in}{3.117657in}}%
\pgfpathlineto{\pgfqpoint{1.414446in}{3.119481in}}%
\pgfpathlineto{\pgfqpoint{1.418995in}{3.126071in}}%
\pgfpathlineto{\pgfqpoint{1.424080in}{3.132297in}}%
\pgfpathlineto{\pgfqpoint{1.426756in}{3.132739in}}%
\pgfpathlineto{\pgfqpoint{1.429432in}{3.130846in}}%
\pgfpathlineto{\pgfqpoint{1.434249in}{3.123786in}}%
\pgfpathlineto{\pgfqpoint{1.439066in}{3.118071in}}%
\pgfpathlineto{\pgfqpoint{1.441742in}{3.117706in}}%
\pgfpathlineto{\pgfqpoint{1.444418in}{3.119666in}}%
\pgfpathlineto{\pgfqpoint{1.449235in}{3.126768in}}%
\pgfpathlineto{\pgfqpoint{1.454052in}{3.132396in}}%
\pgfpathlineto{\pgfqpoint{1.456728in}{3.132685in}}%
\pgfpathlineto{\pgfqpoint{1.459404in}{3.130657in}}%
\pgfpathlineto{\pgfqpoint{1.464489in}{3.123099in}}%
\pgfpathlineto{\pgfqpoint{1.469038in}{3.117977in}}%
\pgfpathlineto{\pgfqpoint{1.471714in}{3.117764in}}%
\pgfpathlineto{\pgfqpoint{1.474658in}{3.120176in}}%
\pgfpathlineto{\pgfqpoint{1.486165in}{3.132788in}}%
\pgfpathlineto{\pgfqpoint{1.488841in}{3.131054in}}%
\pgfpathlineto{\pgfqpoint{1.493123in}{3.124954in}}%
\pgfpathlineto{\pgfqpoint{1.498475in}{3.118190in}}%
\pgfpathlineto{\pgfqpoint{1.501151in}{3.117651in}}%
\pgfpathlineto{\pgfqpoint{1.503827in}{3.119455in}}%
\pgfpathlineto{\pgfqpoint{1.508376in}{3.126032in}}%
\pgfpathlineto{\pgfqpoint{1.513461in}{3.132282in}}%
\pgfpathlineto{\pgfqpoint{1.516137in}{3.132745in}}%
\pgfpathlineto{\pgfqpoint{1.518813in}{3.130873in}}%
\pgfpathlineto{\pgfqpoint{1.523630in}{3.123824in}}%
\pgfpathlineto{\pgfqpoint{1.528447in}{3.118086in}}%
\pgfpathlineto{\pgfqpoint{1.531123in}{3.117698in}}%
\pgfpathlineto{\pgfqpoint{1.533799in}{3.119639in}}%
\pgfpathlineto{\pgfqpoint{1.538616in}{3.126730in}}%
\pgfpathlineto{\pgfqpoint{1.543433in}{3.132382in}}%
\pgfpathlineto{\pgfqpoint{1.546109in}{3.132693in}}%
\pgfpathlineto{\pgfqpoint{1.548785in}{3.130685in}}%
\pgfpathlineto{\pgfqpoint{1.553870in}{3.123136in}}%
\pgfpathlineto{\pgfqpoint{1.558419in}{3.117990in}}%
\pgfpathlineto{\pgfqpoint{1.561095in}{3.117755in}}%
\pgfpathlineto{\pgfqpoint{1.564039in}{3.120147in}}%
\pgfpathlineto{\pgfqpoint{1.575546in}{3.132793in}}%
\pgfpathlineto{\pgfqpoint{1.578222in}{3.131079in}}%
\pgfpathlineto{\pgfqpoint{1.582504in}{3.124994in}}%
\pgfpathlineto{\pgfqpoint{1.588123in}{3.118043in}}%
\pgfpathlineto{\pgfqpoint{1.590800in}{3.117722in}}%
\pgfpathlineto{\pgfqpoint{1.593476in}{3.119721in}}%
\pgfpathlineto{\pgfqpoint{1.598560in}{3.127264in}}%
\pgfpathlineto{\pgfqpoint{1.603109in}{3.132422in}}%
\pgfpathlineto{\pgfqpoint{1.605786in}{3.132668in}}%
\pgfpathlineto{\pgfqpoint{1.608729in}{3.130287in}}%
\pgfpathlineto{\pgfqpoint{1.620236in}{3.117623in}}%
\pgfpathlineto{\pgfqpoint{1.622912in}{3.119327in}}%
\pgfpathlineto{\pgfqpoint{1.627194in}{3.125406in}}%
\pgfpathlineto{\pgfqpoint{1.632814in}{3.132368in}}%
\pgfpathlineto{\pgfqpoint{1.635490in}{3.132701in}}%
\pgfpathlineto{\pgfqpoint{1.638166in}{3.130712in}}%
\pgfpathlineto{\pgfqpoint{1.643251in}{3.123174in}}%
\pgfpathlineto{\pgfqpoint{1.647800in}{3.118003in}}%
\pgfpathlineto{\pgfqpoint{1.650476in}{3.117747in}}%
\pgfpathlineto{\pgfqpoint{1.653420in}{3.120117in}}%
\pgfpathlineto{\pgfqpoint{1.664927in}{3.132798in}}%
\pgfpathlineto{\pgfqpoint{1.667603in}{3.131104in}}%
\pgfpathlineto{\pgfqpoint{1.671885in}{3.125033in}}%
\pgfpathlineto{\pgfqpoint{1.677504in}{3.118057in}}%
\pgfpathlineto{\pgfqpoint{1.680181in}{3.117714in}}%
\pgfpathlineto{\pgfqpoint{1.682857in}{3.119693in}}%
\pgfpathlineto{\pgfqpoint{1.687941in}{3.127226in}}%
\pgfpathlineto{\pgfqpoint{1.692490in}{3.132409in}}%
\pgfpathlineto{\pgfqpoint{1.695167in}{3.132677in}}%
\pgfpathlineto{\pgfqpoint{1.697843in}{3.130629in}}%
\pgfpathlineto{\pgfqpoint{1.702927in}{3.123061in}}%
\pgfpathlineto{\pgfqpoint{1.707477in}{3.117964in}}%
\pgfpathlineto{\pgfqpoint{1.710153in}{3.117773in}}%
\pgfpathlineto{\pgfqpoint{1.713096in}{3.120206in}}%
\pgfpathlineto{\pgfqpoint{1.724336in}{3.132832in}}%
\pgfpathlineto{\pgfqpoint{1.727012in}{3.131299in}}%
\pgfpathlineto{\pgfqpoint{1.731026in}{3.125778in}}%
\pgfpathlineto{\pgfqpoint{1.736913in}{3.118175in}}%
\pgfpathlineto{\pgfqpoint{1.739589in}{3.117657in}}%
\pgfpathlineto{\pgfqpoint{1.742266in}{3.119481in}}%
\pgfpathlineto{\pgfqpoint{1.746815in}{3.126071in}}%
\pgfpathlineto{\pgfqpoint{1.751899in}{3.132297in}}%
\pgfpathlineto{\pgfqpoint{1.754575in}{3.132739in}}%
\pgfpathlineto{\pgfqpoint{1.757252in}{3.130846in}}%
\pgfpathlineto{\pgfqpoint{1.762069in}{3.123786in}}%
\pgfpathlineto{\pgfqpoint{1.766885in}{3.118071in}}%
\pgfpathlineto{\pgfqpoint{1.769562in}{3.117706in}}%
\pgfpathlineto{\pgfqpoint{1.772238in}{3.119666in}}%
\pgfpathlineto{\pgfqpoint{1.777055in}{3.126768in}}%
\pgfpathlineto{\pgfqpoint{1.781871in}{3.132396in}}%
\pgfpathlineto{\pgfqpoint{1.784548in}{3.132685in}}%
\pgfpathlineto{\pgfqpoint{1.787224in}{3.130657in}}%
\pgfpathlineto{\pgfqpoint{1.792308in}{3.123099in}}%
\pgfpathlineto{\pgfqpoint{1.796858in}{3.117977in}}%
\pgfpathlineto{\pgfqpoint{1.799534in}{3.117764in}}%
\pgfpathlineto{\pgfqpoint{1.802477in}{3.120176in}}%
\pgfpathlineto{\pgfqpoint{1.813984in}{3.132788in}}%
\pgfpathlineto{\pgfqpoint{1.816661in}{3.131054in}}%
\pgfpathlineto{\pgfqpoint{1.820942in}{3.124954in}}%
\pgfpathlineto{\pgfqpoint{1.826294in}{3.118190in}}%
\pgfpathlineto{\pgfqpoint{1.828970in}{3.117651in}}%
\pgfpathlineto{\pgfqpoint{1.831647in}{3.119455in}}%
\pgfpathlineto{\pgfqpoint{1.836196in}{3.126032in}}%
\pgfpathlineto{\pgfqpoint{1.841280in}{3.132282in}}%
\pgfpathlineto{\pgfqpoint{1.843957in}{3.132745in}}%
\pgfpathlineto{\pgfqpoint{1.846633in}{3.130873in}}%
\pgfpathlineto{\pgfqpoint{1.851450in}{3.123824in}}%
\pgfpathlineto{\pgfqpoint{1.856266in}{3.118086in}}%
\pgfpathlineto{\pgfqpoint{1.858943in}{3.117698in}}%
\pgfpathlineto{\pgfqpoint{1.861619in}{3.119639in}}%
\pgfpathlineto{\pgfqpoint{1.866436in}{3.126730in}}%
\pgfpathlineto{\pgfqpoint{1.871253in}{3.132382in}}%
\pgfpathlineto{\pgfqpoint{1.873929in}{3.132693in}}%
\pgfpathlineto{\pgfqpoint{1.876605in}{3.130685in}}%
\pgfpathlineto{\pgfqpoint{1.881689in}{3.123136in}}%
\pgfpathlineto{\pgfqpoint{1.886239in}{3.117990in}}%
\pgfpathlineto{\pgfqpoint{1.888915in}{3.117755in}}%
\pgfpathlineto{\pgfqpoint{1.891858in}{3.120147in}}%
\pgfpathlineto{\pgfqpoint{1.903365in}{3.132793in}}%
\pgfpathlineto{\pgfqpoint{1.906042in}{3.131079in}}%
\pgfpathlineto{\pgfqpoint{1.910323in}{3.124994in}}%
\pgfpathlineto{\pgfqpoint{1.915943in}{3.118043in}}%
\pgfpathlineto{\pgfqpoint{1.918619in}{3.117722in}}%
\pgfpathlineto{\pgfqpoint{1.921295in}{3.119721in}}%
\pgfpathlineto{\pgfqpoint{1.926380in}{3.127264in}}%
\pgfpathlineto{\pgfqpoint{1.930929in}{3.132422in}}%
\pgfpathlineto{\pgfqpoint{1.933605in}{3.132668in}}%
\pgfpathlineto{\pgfqpoint{1.936549in}{3.130287in}}%
\pgfpathlineto{\pgfqpoint{1.948056in}{3.117623in}}%
\pgfpathlineto{\pgfqpoint{1.950732in}{3.119327in}}%
\pgfpathlineto{\pgfqpoint{1.955014in}{3.125406in}}%
\pgfpathlineto{\pgfqpoint{1.960634in}{3.132368in}}%
\pgfpathlineto{\pgfqpoint{1.963310in}{3.132701in}}%
\pgfpathlineto{\pgfqpoint{1.965986in}{3.130712in}}%
\pgfpathlineto{\pgfqpoint{1.971070in}{3.123174in}}%
\pgfpathlineto{\pgfqpoint{1.975620in}{3.118003in}}%
\pgfpathlineto{\pgfqpoint{1.978296in}{3.117747in}}%
\pgfpathlineto{\pgfqpoint{1.981239in}{3.120117in}}%
\pgfpathlineto{\pgfqpoint{1.992746in}{3.132798in}}%
\pgfpathlineto{\pgfqpoint{1.995423in}{3.131104in}}%
\pgfpathlineto{\pgfqpoint{1.999704in}{3.125033in}}%
\pgfpathlineto{\pgfqpoint{2.005324in}{3.118057in}}%
\pgfpathlineto{\pgfqpoint{2.008000in}{3.117714in}}%
\pgfpathlineto{\pgfqpoint{2.010676in}{3.119693in}}%
\pgfpathlineto{\pgfqpoint{2.015761in}{3.127226in}}%
\pgfpathlineto{\pgfqpoint{2.020310in}{3.132409in}}%
\pgfpathlineto{\pgfqpoint{2.022986in}{3.132677in}}%
\pgfpathlineto{\pgfqpoint{2.025662in}{3.130629in}}%
\pgfpathlineto{\pgfqpoint{2.030747in}{3.123061in}}%
\pgfpathlineto{\pgfqpoint{2.035296in}{3.117964in}}%
\pgfpathlineto{\pgfqpoint{2.037972in}{3.117773in}}%
\pgfpathlineto{\pgfqpoint{2.040916in}{3.120206in}}%
\pgfpathlineto{\pgfqpoint{2.052155in}{3.132832in}}%
\pgfpathlineto{\pgfqpoint{2.054831in}{3.131299in}}%
\pgfpathlineto{\pgfqpoint{2.058846in}{3.125778in}}%
\pgfpathlineto{\pgfqpoint{2.064733in}{3.118175in}}%
\pgfpathlineto{\pgfqpoint{2.067409in}{3.117657in}}%
\pgfpathlineto{\pgfqpoint{2.070085in}{3.119481in}}%
\pgfpathlineto{\pgfqpoint{2.074634in}{3.126071in}}%
\pgfpathlineto{\pgfqpoint{2.079719in}{3.132297in}}%
\pgfpathlineto{\pgfqpoint{2.082395in}{3.132739in}}%
\pgfpathlineto{\pgfqpoint{2.085071in}{3.130846in}}%
\pgfpathlineto{\pgfqpoint{2.089888in}{3.123786in}}%
\pgfpathlineto{\pgfqpoint{2.094705in}{3.118071in}}%
\pgfpathlineto{\pgfqpoint{2.097381in}{3.117706in}}%
\pgfpathlineto{\pgfqpoint{2.100057in}{3.119666in}}%
\pgfpathlineto{\pgfqpoint{2.104874in}{3.126768in}}%
\pgfpathlineto{\pgfqpoint{2.109691in}{3.132396in}}%
\pgfpathlineto{\pgfqpoint{2.112367in}{3.132685in}}%
\pgfpathlineto{\pgfqpoint{2.115043in}{3.130657in}}%
\pgfpathlineto{\pgfqpoint{2.120128in}{3.123099in}}%
\pgfpathlineto{\pgfqpoint{2.124677in}{3.117977in}}%
\pgfpathlineto{\pgfqpoint{2.127353in}{3.117764in}}%
\pgfpathlineto{\pgfqpoint{2.130297in}{3.120176in}}%
\pgfpathlineto{\pgfqpoint{2.141804in}{3.132788in}}%
\pgfpathlineto{\pgfqpoint{2.144480in}{3.131054in}}%
\pgfpathlineto{\pgfqpoint{2.148762in}{3.124954in}}%
\pgfpathlineto{\pgfqpoint{2.154114in}{3.118190in}}%
\pgfpathlineto{\pgfqpoint{2.156790in}{3.117651in}}%
\pgfpathlineto{\pgfqpoint{2.159466in}{3.119455in}}%
\pgfpathlineto{\pgfqpoint{2.164015in}{3.126032in}}%
\pgfpathlineto{\pgfqpoint{2.169100in}{3.132282in}}%
\pgfpathlineto{\pgfqpoint{2.171776in}{3.132745in}}%
\pgfpathlineto{\pgfqpoint{2.174452in}{3.130873in}}%
\pgfpathlineto{\pgfqpoint{2.179269in}{3.123824in}}%
\pgfpathlineto{\pgfqpoint{2.184086in}{3.118086in}}%
\pgfpathlineto{\pgfqpoint{2.186762in}{3.117698in}}%
\pgfpathlineto{\pgfqpoint{2.189438in}{3.119639in}}%
\pgfpathlineto{\pgfqpoint{2.194255in}{3.126730in}}%
\pgfpathlineto{\pgfqpoint{2.199072in}{3.132382in}}%
\pgfpathlineto{\pgfqpoint{2.201748in}{3.132693in}}%
\pgfpathlineto{\pgfqpoint{2.204424in}{3.130685in}}%
\pgfpathlineto{\pgfqpoint{2.209509in}{3.123136in}}%
\pgfpathlineto{\pgfqpoint{2.214058in}{3.117990in}}%
\pgfpathlineto{\pgfqpoint{2.216734in}{3.117755in}}%
\pgfpathlineto{\pgfqpoint{2.219678in}{3.120147in}}%
\pgfpathlineto{\pgfqpoint{2.231185in}{3.132793in}}%
\pgfpathlineto{\pgfqpoint{2.233861in}{3.131079in}}%
\pgfpathlineto{\pgfqpoint{2.238143in}{3.124994in}}%
\pgfpathlineto{\pgfqpoint{2.243763in}{3.118043in}}%
\pgfpathlineto{\pgfqpoint{2.246439in}{3.117722in}}%
\pgfpathlineto{\pgfqpoint{2.249115in}{3.119721in}}%
\pgfpathlineto{\pgfqpoint{2.254199in}{3.127264in}}%
\pgfpathlineto{\pgfqpoint{2.258749in}{3.132422in}}%
\pgfpathlineto{\pgfqpoint{2.261425in}{3.132668in}}%
\pgfpathlineto{\pgfqpoint{2.264368in}{3.130287in}}%
\pgfpathlineto{\pgfqpoint{2.275876in}{3.117623in}}%
\pgfpathlineto{\pgfqpoint{2.278552in}{3.119327in}}%
\pgfpathlineto{\pgfqpoint{2.282833in}{3.125406in}}%
\pgfpathlineto{\pgfqpoint{2.288453in}{3.132368in}}%
\pgfpathlineto{\pgfqpoint{2.291129in}{3.132701in}}%
\pgfpathlineto{\pgfqpoint{2.293805in}{3.130712in}}%
\pgfpathlineto{\pgfqpoint{2.298890in}{3.123174in}}%
\pgfpathlineto{\pgfqpoint{2.303439in}{3.118003in}}%
\pgfpathlineto{\pgfqpoint{2.306115in}{3.117747in}}%
\pgfpathlineto{\pgfqpoint{2.309059in}{3.120117in}}%
\pgfpathlineto{\pgfqpoint{2.320566in}{3.132798in}}%
\pgfpathlineto{\pgfqpoint{2.323242in}{3.131104in}}%
\pgfpathlineto{\pgfqpoint{2.327524in}{3.125033in}}%
\pgfpathlineto{\pgfqpoint{2.333144in}{3.118057in}}%
\pgfpathlineto{\pgfqpoint{2.335820in}{3.117714in}}%
\pgfpathlineto{\pgfqpoint{2.338496in}{3.119693in}}%
\pgfpathlineto{\pgfqpoint{2.343580in}{3.127226in}}%
\pgfpathlineto{\pgfqpoint{2.348130in}{3.132409in}}%
\pgfpathlineto{\pgfqpoint{2.350806in}{3.132677in}}%
\pgfpathlineto{\pgfqpoint{2.353482in}{3.130629in}}%
\pgfpathlineto{\pgfqpoint{2.358566in}{3.123061in}}%
\pgfpathlineto{\pgfqpoint{2.363116in}{3.117964in}}%
\pgfpathlineto{\pgfqpoint{2.365792in}{3.117773in}}%
\pgfpathlineto{\pgfqpoint{2.368735in}{3.120206in}}%
\pgfpathlineto{\pgfqpoint{2.379975in}{3.132832in}}%
\pgfpathlineto{\pgfqpoint{2.382651in}{3.131299in}}%
\pgfpathlineto{\pgfqpoint{2.386665in}{3.125778in}}%
\pgfpathlineto{\pgfqpoint{2.392553in}{3.118175in}}%
\pgfpathlineto{\pgfqpoint{2.395229in}{3.117657in}}%
\pgfpathlineto{\pgfqpoint{2.397905in}{3.119481in}}%
\pgfpathlineto{\pgfqpoint{2.402454in}{3.126071in}}%
\pgfpathlineto{\pgfqpoint{2.407539in}{3.132297in}}%
\pgfpathlineto{\pgfqpoint{2.410215in}{3.132739in}}%
\pgfpathlineto{\pgfqpoint{2.412891in}{3.130846in}}%
\pgfpathlineto{\pgfqpoint{2.417708in}{3.123786in}}%
\pgfpathlineto{\pgfqpoint{2.422525in}{3.118071in}}%
\pgfpathlineto{\pgfqpoint{2.425201in}{3.117706in}}%
\pgfpathlineto{\pgfqpoint{2.427877in}{3.119666in}}%
\pgfpathlineto{\pgfqpoint{2.432694in}{3.126768in}}%
\pgfpathlineto{\pgfqpoint{2.437511in}{3.132396in}}%
\pgfpathlineto{\pgfqpoint{2.440187in}{3.132685in}}%
\pgfpathlineto{\pgfqpoint{2.442863in}{3.130657in}}%
\pgfpathlineto{\pgfqpoint{2.447947in}{3.123099in}}%
\pgfpathlineto{\pgfqpoint{2.452497in}{3.117977in}}%
\pgfpathlineto{\pgfqpoint{2.455173in}{3.117764in}}%
\pgfpathlineto{\pgfqpoint{2.458116in}{3.120176in}}%
\pgfpathlineto{\pgfqpoint{2.469624in}{3.132788in}}%
\pgfpathlineto{\pgfqpoint{2.472300in}{3.131054in}}%
\pgfpathlineto{\pgfqpoint{2.476581in}{3.124954in}}%
\pgfpathlineto{\pgfqpoint{2.481934in}{3.118190in}}%
\pgfpathlineto{\pgfqpoint{2.484610in}{3.117651in}}%
\pgfpathlineto{\pgfqpoint{2.487286in}{3.119455in}}%
\pgfpathlineto{\pgfqpoint{2.491835in}{3.126032in}}%
\pgfpathlineto{\pgfqpoint{2.496920in}{3.132282in}}%
\pgfpathlineto{\pgfqpoint{2.499596in}{3.132745in}}%
\pgfpathlineto{\pgfqpoint{2.502272in}{3.130873in}}%
\pgfpathlineto{\pgfqpoint{2.507089in}{3.123824in}}%
\pgfpathlineto{\pgfqpoint{2.511906in}{3.118086in}}%
\pgfpathlineto{\pgfqpoint{2.514582in}{3.117698in}}%
\pgfpathlineto{\pgfqpoint{2.517258in}{3.119639in}}%
\pgfpathlineto{\pgfqpoint{2.522075in}{3.126730in}}%
\pgfpathlineto{\pgfqpoint{2.526892in}{3.132382in}}%
\pgfpathlineto{\pgfqpoint{2.529568in}{3.132693in}}%
\pgfpathlineto{\pgfqpoint{2.532244in}{3.130685in}}%
\pgfpathlineto{\pgfqpoint{2.537328in}{3.123136in}}%
\pgfpathlineto{\pgfqpoint{2.541878in}{3.117990in}}%
\pgfpathlineto{\pgfqpoint{2.544554in}{3.117755in}}%
\pgfpathlineto{\pgfqpoint{2.547497in}{3.120147in}}%
\pgfpathlineto{\pgfqpoint{2.559005in}{3.132793in}}%
\pgfpathlineto{\pgfqpoint{2.561681in}{3.131079in}}%
\pgfpathlineto{\pgfqpoint{2.565962in}{3.124994in}}%
\pgfpathlineto{\pgfqpoint{2.571582in}{3.118043in}}%
\pgfpathlineto{\pgfqpoint{2.574258in}{3.117722in}}%
\pgfpathlineto{\pgfqpoint{2.576934in}{3.119721in}}%
\pgfpathlineto{\pgfqpoint{2.582019in}{3.127264in}}%
\pgfpathlineto{\pgfqpoint{2.586568in}{3.132422in}}%
\pgfpathlineto{\pgfqpoint{2.589244in}{3.132668in}}%
\pgfpathlineto{\pgfqpoint{2.592188in}{3.130287in}}%
\pgfpathlineto{\pgfqpoint{2.603695in}{3.117623in}}%
\pgfpathlineto{\pgfqpoint{2.606371in}{3.119327in}}%
\pgfpathlineto{\pgfqpoint{2.610653in}{3.125406in}}%
\pgfpathlineto{\pgfqpoint{2.616273in}{3.132368in}}%
\pgfpathlineto{\pgfqpoint{2.618949in}{3.132701in}}%
\pgfpathlineto{\pgfqpoint{2.621625in}{3.130712in}}%
\pgfpathlineto{\pgfqpoint{2.626709in}{3.123174in}}%
\pgfpathlineto{\pgfqpoint{2.631259in}{3.118003in}}%
\pgfpathlineto{\pgfqpoint{2.633935in}{3.117747in}}%
\pgfpathlineto{\pgfqpoint{2.636878in}{3.120117in}}%
\pgfpathlineto{\pgfqpoint{2.648386in}{3.132798in}}%
\pgfpathlineto{\pgfqpoint{2.651062in}{3.131104in}}%
\pgfpathlineto{\pgfqpoint{2.655343in}{3.125033in}}%
\pgfpathlineto{\pgfqpoint{2.660963in}{3.118057in}}%
\pgfpathlineto{\pgfqpoint{2.663639in}{3.117714in}}%
\pgfpathlineto{\pgfqpoint{2.666315in}{3.119693in}}%
\pgfpathlineto{\pgfqpoint{2.671400in}{3.127226in}}%
\pgfpathlineto{\pgfqpoint{2.675949in}{3.132409in}}%
\pgfpathlineto{\pgfqpoint{2.678625in}{3.132677in}}%
\pgfpathlineto{\pgfqpoint{2.681301in}{3.130629in}}%
\pgfpathlineto{\pgfqpoint{2.686386in}{3.123061in}}%
\pgfpathlineto{\pgfqpoint{2.690935in}{3.117964in}}%
\pgfpathlineto{\pgfqpoint{2.693611in}{3.117773in}}%
\pgfpathlineto{\pgfqpoint{2.696555in}{3.120206in}}%
\pgfpathlineto{\pgfqpoint{2.707795in}{3.132832in}}%
\pgfpathlineto{\pgfqpoint{2.710471in}{3.131299in}}%
\pgfpathlineto{\pgfqpoint{2.714485in}{3.125778in}}%
\pgfpathlineto{\pgfqpoint{2.720372in}{3.118175in}}%
\pgfpathlineto{\pgfqpoint{2.723048in}{3.117657in}}%
\pgfpathlineto{\pgfqpoint{2.725724in}{3.119481in}}%
\pgfpathlineto{\pgfqpoint{2.730274in}{3.126071in}}%
\pgfpathlineto{\pgfqpoint{2.735358in}{3.132297in}}%
\pgfpathlineto{\pgfqpoint{2.738034in}{3.132739in}}%
\pgfpathlineto{\pgfqpoint{2.740710in}{3.130846in}}%
\pgfpathlineto{\pgfqpoint{2.745527in}{3.123786in}}%
\pgfpathlineto{\pgfqpoint{2.750344in}{3.118071in}}%
\pgfpathlineto{\pgfqpoint{2.753020in}{3.117706in}}%
\pgfpathlineto{\pgfqpoint{2.755696in}{3.119666in}}%
\pgfpathlineto{\pgfqpoint{2.760513in}{3.126768in}}%
\pgfpathlineto{\pgfqpoint{2.765330in}{3.132396in}}%
\pgfpathlineto{\pgfqpoint{2.768006in}{3.132685in}}%
\pgfpathlineto{\pgfqpoint{2.770682in}{3.130657in}}%
\pgfpathlineto{\pgfqpoint{2.775767in}{3.123099in}}%
\pgfpathlineto{\pgfqpoint{2.780316in}{3.117977in}}%
\pgfpathlineto{\pgfqpoint{2.782992in}{3.117764in}}%
\pgfpathlineto{\pgfqpoint{2.785936in}{3.120176in}}%
\pgfpathlineto{\pgfqpoint{2.797443in}{3.132788in}}%
\pgfpathlineto{\pgfqpoint{2.800119in}{3.131054in}}%
\pgfpathlineto{\pgfqpoint{2.804401in}{3.124954in}}%
\pgfpathlineto{\pgfqpoint{2.809753in}{3.118190in}}%
\pgfpathlineto{\pgfqpoint{2.812429in}{3.117651in}}%
\pgfpathlineto{\pgfqpoint{2.815105in}{3.119455in}}%
\pgfpathlineto{\pgfqpoint{2.819655in}{3.126032in}}%
\pgfpathlineto{\pgfqpoint{2.824739in}{3.132282in}}%
\pgfpathlineto{\pgfqpoint{2.827415in}{3.132745in}}%
\pgfpathlineto{\pgfqpoint{2.830091in}{3.130873in}}%
\pgfpathlineto{\pgfqpoint{2.834908in}{3.123824in}}%
\pgfpathlineto{\pgfqpoint{2.839725in}{3.118086in}}%
\pgfpathlineto{\pgfqpoint{2.842401in}{3.117698in}}%
\pgfpathlineto{\pgfqpoint{2.845077in}{3.119639in}}%
\pgfpathlineto{\pgfqpoint{2.849894in}{3.126730in}}%
\pgfpathlineto{\pgfqpoint{2.854711in}{3.132382in}}%
\pgfpathlineto{\pgfqpoint{2.857387in}{3.132693in}}%
\pgfpathlineto{\pgfqpoint{2.860063in}{3.130685in}}%
\pgfpathlineto{\pgfqpoint{2.865148in}{3.123136in}}%
\pgfpathlineto{\pgfqpoint{2.869697in}{3.117990in}}%
\pgfpathlineto{\pgfqpoint{2.872373in}{3.117755in}}%
\pgfpathlineto{\pgfqpoint{2.875317in}{3.120147in}}%
\pgfpathlineto{\pgfqpoint{2.886824in}{3.132793in}}%
\pgfpathlineto{\pgfqpoint{2.889500in}{3.131079in}}%
\pgfpathlineto{\pgfqpoint{2.893782in}{3.124994in}}%
\pgfpathlineto{\pgfqpoint{2.899402in}{3.118043in}}%
\pgfpathlineto{\pgfqpoint{2.902078in}{3.117722in}}%
\pgfpathlineto{\pgfqpoint{2.904754in}{3.119721in}}%
\pgfpathlineto{\pgfqpoint{2.909838in}{3.127264in}}%
\pgfpathlineto{\pgfqpoint{2.914388in}{3.132422in}}%
\pgfpathlineto{\pgfqpoint{2.917064in}{3.132668in}}%
\pgfpathlineto{\pgfqpoint{2.920008in}{3.130287in}}%
\pgfpathlineto{\pgfqpoint{2.931515in}{3.117623in}}%
\pgfpathlineto{\pgfqpoint{2.934191in}{3.119327in}}%
\pgfpathlineto{\pgfqpoint{2.938472in}{3.125406in}}%
\pgfpathlineto{\pgfqpoint{2.944092in}{3.132368in}}%
\pgfpathlineto{\pgfqpoint{2.946768in}{3.132701in}}%
\pgfpathlineto{\pgfqpoint{2.949444in}{3.130712in}}%
\pgfpathlineto{\pgfqpoint{2.954529in}{3.123174in}}%
\pgfpathlineto{\pgfqpoint{2.959078in}{3.118003in}}%
\pgfpathlineto{\pgfqpoint{2.961754in}{3.117747in}}%
\pgfpathlineto{\pgfqpoint{2.964698in}{3.120117in}}%
\pgfpathlineto{\pgfqpoint{2.976205in}{3.132798in}}%
\pgfpathlineto{\pgfqpoint{2.978881in}{3.131104in}}%
\pgfpathlineto{\pgfqpoint{2.983163in}{3.125033in}}%
\pgfpathlineto{\pgfqpoint{2.988783in}{3.118057in}}%
\pgfpathlineto{\pgfqpoint{2.991459in}{3.117714in}}%
\pgfpathlineto{\pgfqpoint{2.994135in}{3.119693in}}%
\pgfpathlineto{\pgfqpoint{2.999219in}{3.127226in}}%
\pgfpathlineto{\pgfqpoint{3.003769in}{3.132409in}}%
\pgfpathlineto{\pgfqpoint{3.006445in}{3.132677in}}%
\pgfpathlineto{\pgfqpoint{3.009121in}{3.130629in}}%
\pgfpathlineto{\pgfqpoint{3.014206in}{3.123061in}}%
\pgfpathlineto{\pgfqpoint{3.018755in}{3.117964in}}%
\pgfpathlineto{\pgfqpoint{3.021431in}{3.117773in}}%
\pgfpathlineto{\pgfqpoint{3.024375in}{3.120206in}}%
\pgfpathlineto{\pgfqpoint{3.035614in}{3.132832in}}%
\pgfpathlineto{\pgfqpoint{3.038290in}{3.131299in}}%
\pgfpathlineto{\pgfqpoint{3.042304in}{3.125778in}}%
\pgfpathlineto{\pgfqpoint{3.048192in}{3.118175in}}%
\pgfpathlineto{\pgfqpoint{3.050868in}{3.117657in}}%
\pgfpathlineto{\pgfqpoint{3.053544in}{3.119481in}}%
\pgfpathlineto{\pgfqpoint{3.058093in}{3.126071in}}%
\pgfpathlineto{\pgfqpoint{3.063178in}{3.132297in}}%
\pgfpathlineto{\pgfqpoint{3.065854in}{3.132739in}}%
\pgfpathlineto{\pgfqpoint{3.068530in}{3.130846in}}%
\pgfpathlineto{\pgfqpoint{3.073347in}{3.123786in}}%
\pgfpathlineto{\pgfqpoint{3.078164in}{3.118071in}}%
\pgfpathlineto{\pgfqpoint{3.080840in}{3.117706in}}%
\pgfpathlineto{\pgfqpoint{3.083516in}{3.119666in}}%
\pgfpathlineto{\pgfqpoint{3.088333in}{3.126768in}}%
\pgfpathlineto{\pgfqpoint{3.093150in}{3.132396in}}%
\pgfpathlineto{\pgfqpoint{3.095826in}{3.132685in}}%
\pgfpathlineto{\pgfqpoint{3.098502in}{3.130657in}}%
\pgfpathlineto{\pgfqpoint{3.103587in}{3.123099in}}%
\pgfpathlineto{\pgfqpoint{3.108136in}{3.117977in}}%
\pgfpathlineto{\pgfqpoint{3.110812in}{3.117764in}}%
\pgfpathlineto{\pgfqpoint{3.113756in}{3.120176in}}%
\pgfpathlineto{\pgfqpoint{3.125263in}{3.132788in}}%
\pgfpathlineto{\pgfqpoint{3.127939in}{3.131054in}}%
\pgfpathlineto{\pgfqpoint{3.132221in}{3.124954in}}%
\pgfpathlineto{\pgfqpoint{3.137573in}{3.118190in}}%
\pgfpathlineto{\pgfqpoint{3.140249in}{3.117651in}}%
\pgfpathlineto{\pgfqpoint{3.142925in}{3.119455in}}%
\pgfpathlineto{\pgfqpoint{3.147474in}{3.126032in}}%
\pgfpathlineto{\pgfqpoint{3.152559in}{3.132282in}}%
\pgfpathlineto{\pgfqpoint{3.155235in}{3.132745in}}%
\pgfpathlineto{\pgfqpoint{3.157911in}{3.130873in}}%
\pgfpathlineto{\pgfqpoint{3.162728in}{3.123824in}}%
\pgfpathlineto{\pgfqpoint{3.167545in}{3.118086in}}%
\pgfpathlineto{\pgfqpoint{3.170221in}{3.117698in}}%
\pgfpathlineto{\pgfqpoint{3.172897in}{3.119639in}}%
\pgfpathlineto{\pgfqpoint{3.177714in}{3.126730in}}%
\pgfpathlineto{\pgfqpoint{3.182531in}{3.132382in}}%
\pgfpathlineto{\pgfqpoint{3.185207in}{3.132693in}}%
\pgfpathlineto{\pgfqpoint{3.187883in}{3.130685in}}%
\pgfpathlineto{\pgfqpoint{3.192968in}{3.123136in}}%
\pgfpathlineto{\pgfqpoint{3.197517in}{3.117990in}}%
\pgfpathlineto{\pgfqpoint{3.200193in}{3.117755in}}%
\pgfpathlineto{\pgfqpoint{3.203137in}{3.120147in}}%
\pgfpathlineto{\pgfqpoint{3.214644in}{3.132793in}}%
\pgfpathlineto{\pgfqpoint{3.217320in}{3.131079in}}%
\pgfpathlineto{\pgfqpoint{3.221602in}{3.124994in}}%
\pgfpathlineto{\pgfqpoint{3.227221in}{3.118043in}}%
\pgfpathlineto{\pgfqpoint{3.229897in}{3.117722in}}%
\pgfpathlineto{\pgfqpoint{3.232573in}{3.119721in}}%
\pgfpathlineto{\pgfqpoint{3.237658in}{3.127264in}}%
\pgfpathlineto{\pgfqpoint{3.242207in}{3.132422in}}%
\pgfpathlineto{\pgfqpoint{3.244883in}{3.132668in}}%
\pgfpathlineto{\pgfqpoint{3.247827in}{3.130287in}}%
\pgfpathlineto{\pgfqpoint{3.259334in}{3.117623in}}%
\pgfpathlineto{\pgfqpoint{3.262010in}{3.119327in}}%
\pgfpathlineto{\pgfqpoint{3.266292in}{3.125406in}}%
\pgfpathlineto{\pgfqpoint{3.271912in}{3.132368in}}%
\pgfpathlineto{\pgfqpoint{3.274588in}{3.132701in}}%
\pgfpathlineto{\pgfqpoint{3.277264in}{3.130712in}}%
\pgfpathlineto{\pgfqpoint{3.282349in}{3.123174in}}%
\pgfpathlineto{\pgfqpoint{3.286898in}{3.118003in}}%
\pgfpathlineto{\pgfqpoint{3.289574in}{3.117747in}}%
\pgfpathlineto{\pgfqpoint{3.292518in}{3.120117in}}%
\pgfpathlineto{\pgfqpoint{3.304025in}{3.132798in}}%
\pgfpathlineto{\pgfqpoint{3.306701in}{3.131104in}}%
\pgfpathlineto{\pgfqpoint{3.310983in}{3.125033in}}%
\pgfpathlineto{\pgfqpoint{3.316602in}{3.118057in}}%
\pgfpathlineto{\pgfqpoint{3.319278in}{3.117714in}}%
\pgfpathlineto{\pgfqpoint{3.321955in}{3.119693in}}%
\pgfpathlineto{\pgfqpoint{3.327039in}{3.127226in}}%
\pgfpathlineto{\pgfqpoint{3.331588in}{3.132409in}}%
\pgfpathlineto{\pgfqpoint{3.334264in}{3.132677in}}%
\pgfpathlineto{\pgfqpoint{3.336941in}{3.130629in}}%
\pgfpathlineto{\pgfqpoint{3.342025in}{3.123061in}}%
\pgfpathlineto{\pgfqpoint{3.346574in}{3.117964in}}%
\pgfpathlineto{\pgfqpoint{3.349251in}{3.117773in}}%
\pgfpathlineto{\pgfqpoint{3.352194in}{3.120206in}}%
\pgfpathlineto{\pgfqpoint{3.363434in}{3.132832in}}%
\pgfpathlineto{\pgfqpoint{3.366110in}{3.131299in}}%
\pgfpathlineto{\pgfqpoint{3.370124in}{3.125778in}}%
\pgfpathlineto{\pgfqpoint{3.376011in}{3.118175in}}%
\pgfpathlineto{\pgfqpoint{3.378687in}{3.117657in}}%
\pgfpathlineto{\pgfqpoint{3.381363in}{3.119481in}}%
\pgfpathlineto{\pgfqpoint{3.385913in}{3.126071in}}%
\pgfpathlineto{\pgfqpoint{3.390997in}{3.132297in}}%
\pgfpathlineto{\pgfqpoint{3.393673in}{3.132739in}}%
\pgfpathlineto{\pgfqpoint{3.396349in}{3.130846in}}%
\pgfpathlineto{\pgfqpoint{3.401166in}{3.123786in}}%
\pgfpathlineto{\pgfqpoint{3.405983in}{3.118071in}}%
\pgfpathlineto{\pgfqpoint{3.408659in}{3.117706in}}%
\pgfpathlineto{\pgfqpoint{3.411336in}{3.119666in}}%
\pgfpathlineto{\pgfqpoint{3.416152in}{3.126768in}}%
\pgfpathlineto{\pgfqpoint{3.420969in}{3.132396in}}%
\pgfpathlineto{\pgfqpoint{3.423645in}{3.132685in}}%
\pgfpathlineto{\pgfqpoint{3.426322in}{3.130657in}}%
\pgfpathlineto{\pgfqpoint{3.431406in}{3.123099in}}%
\pgfpathlineto{\pgfqpoint{3.435955in}{3.117977in}}%
\pgfpathlineto{\pgfqpoint{3.438632in}{3.117764in}}%
\pgfpathlineto{\pgfqpoint{3.441575in}{3.120176in}}%
\pgfpathlineto{\pgfqpoint{3.453082in}{3.132788in}}%
\pgfpathlineto{\pgfqpoint{3.455758in}{3.131054in}}%
\pgfpathlineto{\pgfqpoint{3.460040in}{3.124954in}}%
\pgfpathlineto{\pgfqpoint{3.465392in}{3.118190in}}%
\pgfpathlineto{\pgfqpoint{3.468068in}{3.117651in}}%
\pgfpathlineto{\pgfqpoint{3.470744in}{3.119455in}}%
\pgfpathlineto{\pgfqpoint{3.475294in}{3.126032in}}%
\pgfpathlineto{\pgfqpoint{3.480378in}{3.132282in}}%
\pgfpathlineto{\pgfqpoint{3.483054in}{3.132745in}}%
\pgfpathlineto{\pgfqpoint{3.485730in}{3.130873in}}%
\pgfpathlineto{\pgfqpoint{3.490547in}{3.123824in}}%
\pgfpathlineto{\pgfqpoint{3.495364in}{3.118086in}}%
\pgfpathlineto{\pgfqpoint{3.498040in}{3.117698in}}%
\pgfpathlineto{\pgfqpoint{3.500717in}{3.119639in}}%
\pgfpathlineto{\pgfqpoint{3.505533in}{3.126730in}}%
\pgfpathlineto{\pgfqpoint{3.510350in}{3.132382in}}%
\pgfpathlineto{\pgfqpoint{3.513026in}{3.132693in}}%
\pgfpathlineto{\pgfqpoint{3.515703in}{3.130685in}}%
\pgfpathlineto{\pgfqpoint{3.520787in}{3.123136in}}%
\pgfpathlineto{\pgfqpoint{3.525336in}{3.117990in}}%
\pgfpathlineto{\pgfqpoint{3.528013in}{3.117755in}}%
\pgfpathlineto{\pgfqpoint{3.530956in}{3.120147in}}%
\pgfpathlineto{\pgfqpoint{3.542463in}{3.132793in}}%
\pgfpathlineto{\pgfqpoint{3.545139in}{3.131079in}}%
\pgfpathlineto{\pgfqpoint{3.549421in}{3.124994in}}%
\pgfpathlineto{\pgfqpoint{3.555041in}{3.118043in}}%
\pgfpathlineto{\pgfqpoint{3.557717in}{3.117722in}}%
\pgfpathlineto{\pgfqpoint{3.560393in}{3.119721in}}%
\pgfpathlineto{\pgfqpoint{3.565478in}{3.127264in}}%
\pgfpathlineto{\pgfqpoint{3.570027in}{3.132422in}}%
\pgfpathlineto{\pgfqpoint{3.572703in}{3.132668in}}%
\pgfpathlineto{\pgfqpoint{3.575647in}{3.130287in}}%
\pgfpathlineto{\pgfqpoint{3.587154in}{3.117623in}}%
\pgfpathlineto{\pgfqpoint{3.589830in}{3.119327in}}%
\pgfpathlineto{\pgfqpoint{3.594112in}{3.125406in}}%
\pgfpathlineto{\pgfqpoint{3.599731in}{3.132368in}}%
\pgfpathlineto{\pgfqpoint{3.602407in}{3.132701in}}%
\pgfpathlineto{\pgfqpoint{3.605084in}{3.130712in}}%
\pgfpathlineto{\pgfqpoint{3.610168in}{3.123174in}}%
\pgfpathlineto{\pgfqpoint{3.614717in}{3.118003in}}%
\pgfpathlineto{\pgfqpoint{3.617394in}{3.117747in}}%
\pgfpathlineto{\pgfqpoint{3.620337in}{3.120117in}}%
\pgfpathlineto{\pgfqpoint{3.631844in}{3.132798in}}%
\pgfpathlineto{\pgfqpoint{3.634520in}{3.131104in}}%
\pgfpathlineto{\pgfqpoint{3.638802in}{3.125033in}}%
\pgfpathlineto{\pgfqpoint{3.644422in}{3.118057in}}%
\pgfpathlineto{\pgfqpoint{3.647098in}{3.117714in}}%
\pgfpathlineto{\pgfqpoint{3.649774in}{3.119693in}}%
\pgfpathlineto{\pgfqpoint{3.654859in}{3.127226in}}%
\pgfpathlineto{\pgfqpoint{3.659408in}{3.132409in}}%
\pgfpathlineto{\pgfqpoint{3.662084in}{3.132677in}}%
\pgfpathlineto{\pgfqpoint{3.664760in}{3.130629in}}%
\pgfpathlineto{\pgfqpoint{3.669845in}{3.123061in}}%
\pgfpathlineto{\pgfqpoint{3.674394in}{3.117964in}}%
\pgfpathlineto{\pgfqpoint{3.677070in}{3.117773in}}%
\pgfpathlineto{\pgfqpoint{3.680014in}{3.120206in}}%
\pgfpathlineto{\pgfqpoint{3.691253in}{3.132832in}}%
\pgfpathlineto{\pgfqpoint{3.693929in}{3.131299in}}%
\pgfpathlineto{\pgfqpoint{3.697943in}{3.125778in}}%
\pgfpathlineto{\pgfqpoint{3.703831in}{3.118175in}}%
\pgfpathlineto{\pgfqpoint{3.706507in}{3.117657in}}%
\pgfpathlineto{\pgfqpoint{3.709183in}{3.119481in}}%
\pgfpathlineto{\pgfqpoint{3.713732in}{3.126071in}}%
\pgfpathlineto{\pgfqpoint{3.718817in}{3.132297in}}%
\pgfpathlineto{\pgfqpoint{3.721493in}{3.132739in}}%
\pgfpathlineto{\pgfqpoint{3.724169in}{3.130846in}}%
\pgfpathlineto{\pgfqpoint{3.728986in}{3.123786in}}%
\pgfpathlineto{\pgfqpoint{3.733803in}{3.118071in}}%
\pgfpathlineto{\pgfqpoint{3.736479in}{3.117706in}}%
\pgfpathlineto{\pgfqpoint{3.739155in}{3.119666in}}%
\pgfpathlineto{\pgfqpoint{3.743972in}{3.126768in}}%
\pgfpathlineto{\pgfqpoint{3.748789in}{3.132396in}}%
\pgfpathlineto{\pgfqpoint{3.751465in}{3.132685in}}%
\pgfpathlineto{\pgfqpoint{3.754141in}{3.130657in}}%
\pgfpathlineto{\pgfqpoint{3.759226in}{3.123099in}}%
\pgfpathlineto{\pgfqpoint{3.763775in}{3.117977in}}%
\pgfpathlineto{\pgfqpoint{3.766451in}{3.117764in}}%
\pgfpathlineto{\pgfqpoint{3.769395in}{3.120176in}}%
\pgfpathlineto{\pgfqpoint{3.780902in}{3.132788in}}%
\pgfpathlineto{\pgfqpoint{3.783578in}{3.131054in}}%
\pgfpathlineto{\pgfqpoint{3.787860in}{3.124954in}}%
\pgfpathlineto{\pgfqpoint{3.793212in}{3.118190in}}%
\pgfpathlineto{\pgfqpoint{3.795888in}{3.117651in}}%
\pgfpathlineto{\pgfqpoint{3.798564in}{3.119455in}}%
\pgfpathlineto{\pgfqpoint{3.803113in}{3.126032in}}%
\pgfpathlineto{\pgfqpoint{3.808198in}{3.132282in}}%
\pgfpathlineto{\pgfqpoint{3.810874in}{3.132745in}}%
\pgfpathlineto{\pgfqpoint{3.813550in}{3.130873in}}%
\pgfpathlineto{\pgfqpoint{3.818367in}{3.123824in}}%
\pgfpathlineto{\pgfqpoint{3.823184in}{3.118086in}}%
\pgfpathlineto{\pgfqpoint{3.825860in}{3.117698in}}%
\pgfpathlineto{\pgfqpoint{3.828536in}{3.119639in}}%
\pgfpathlineto{\pgfqpoint{3.833353in}{3.126730in}}%
\pgfpathlineto{\pgfqpoint{3.838170in}{3.132382in}}%
\pgfpathlineto{\pgfqpoint{3.840846in}{3.132693in}}%
\pgfpathlineto{\pgfqpoint{3.843522in}{3.130685in}}%
\pgfpathlineto{\pgfqpoint{3.848607in}{3.123136in}}%
\pgfpathlineto{\pgfqpoint{3.853156in}{3.117990in}}%
\pgfpathlineto{\pgfqpoint{3.855832in}{3.117755in}}%
\pgfpathlineto{\pgfqpoint{3.858776in}{3.120147in}}%
\pgfpathlineto{\pgfqpoint{3.870283in}{3.132793in}}%
\pgfpathlineto{\pgfqpoint{3.872959in}{3.131079in}}%
\pgfpathlineto{\pgfqpoint{3.877241in}{3.124994in}}%
\pgfpathlineto{\pgfqpoint{3.882860in}{3.118043in}}%
\pgfpathlineto{\pgfqpoint{3.885537in}{3.117722in}}%
\pgfpathlineto{\pgfqpoint{3.888213in}{3.119721in}}%
\pgfpathlineto{\pgfqpoint{3.893297in}{3.127264in}}%
\pgfpathlineto{\pgfqpoint{3.897847in}{3.132422in}}%
\pgfpathlineto{\pgfqpoint{3.900523in}{3.132668in}}%
\pgfpathlineto{\pgfqpoint{3.903466in}{3.130287in}}%
\pgfpathlineto{\pgfqpoint{3.914973in}{3.117623in}}%
\pgfpathlineto{\pgfqpoint{3.917650in}{3.119327in}}%
\pgfpathlineto{\pgfqpoint{3.921931in}{3.125406in}}%
\pgfpathlineto{\pgfqpoint{3.927551in}{3.132368in}}%
\pgfpathlineto{\pgfqpoint{3.930227in}{3.132701in}}%
\pgfpathlineto{\pgfqpoint{3.932903in}{3.130712in}}%
\pgfpathlineto{\pgfqpoint{3.937988in}{3.123174in}}%
\pgfpathlineto{\pgfqpoint{3.942537in}{3.118003in}}%
\pgfpathlineto{\pgfqpoint{3.945213in}{3.117747in}}%
\pgfpathlineto{\pgfqpoint{3.948157in}{3.120117in}}%
\pgfpathlineto{\pgfqpoint{3.959664in}{3.132798in}}%
\pgfpathlineto{\pgfqpoint{3.962340in}{3.131104in}}%
\pgfpathlineto{\pgfqpoint{3.966622in}{3.125033in}}%
\pgfpathlineto{\pgfqpoint{3.972242in}{3.118057in}}%
\pgfpathlineto{\pgfqpoint{3.974918in}{3.117714in}}%
\pgfpathlineto{\pgfqpoint{3.977594in}{3.119693in}}%
\pgfpathlineto{\pgfqpoint{3.982678in}{3.127226in}}%
\pgfpathlineto{\pgfqpoint{3.987228in}{3.132409in}}%
\pgfpathlineto{\pgfqpoint{3.989904in}{3.132677in}}%
\pgfpathlineto{\pgfqpoint{3.992580in}{3.130629in}}%
\pgfpathlineto{\pgfqpoint{3.997664in}{3.123061in}}%
\pgfpathlineto{\pgfqpoint{4.002214in}{3.117964in}}%
\pgfpathlineto{\pgfqpoint{4.004890in}{3.117773in}}%
\pgfpathlineto{\pgfqpoint{4.007833in}{3.120206in}}%
\pgfpathlineto{\pgfqpoint{4.019073in}{3.132832in}}%
\pgfpathlineto{\pgfqpoint{4.021749in}{3.131299in}}%
\pgfpathlineto{\pgfqpoint{4.025763in}{3.125778in}}%
\pgfpathlineto{\pgfqpoint{4.031650in}{3.118175in}}%
\pgfpathlineto{\pgfqpoint{4.034327in}{3.117657in}}%
\pgfpathlineto{\pgfqpoint{4.037003in}{3.119481in}}%
\pgfpathlineto{\pgfqpoint{4.041552in}{3.126071in}}%
\pgfpathlineto{\pgfqpoint{4.046636in}{3.132297in}}%
\pgfpathlineto{\pgfqpoint{4.049313in}{3.132739in}}%
\pgfpathlineto{\pgfqpoint{4.051989in}{3.130846in}}%
\pgfpathlineto{\pgfqpoint{4.056806in}{3.123786in}}%
\pgfpathlineto{\pgfqpoint{4.061623in}{3.118071in}}%
\pgfpathlineto{\pgfqpoint{4.064299in}{3.117706in}}%
\pgfpathlineto{\pgfqpoint{4.066975in}{3.119666in}}%
\pgfpathlineto{\pgfqpoint{4.071792in}{3.126768in}}%
\pgfpathlineto{\pgfqpoint{4.076609in}{3.132396in}}%
\pgfpathlineto{\pgfqpoint{4.079285in}{3.132685in}}%
\pgfpathlineto{\pgfqpoint{4.081961in}{3.130657in}}%
\pgfpathlineto{\pgfqpoint{4.087045in}{3.123099in}}%
\pgfpathlineto{\pgfqpoint{4.091595in}{3.117977in}}%
\pgfpathlineto{\pgfqpoint{4.094271in}{3.117764in}}%
\pgfpathlineto{\pgfqpoint{4.097214in}{3.120176in}}%
\pgfpathlineto{\pgfqpoint{4.108721in}{3.132788in}}%
\pgfpathlineto{\pgfqpoint{4.111398in}{3.131054in}}%
\pgfpathlineto{\pgfqpoint{4.115679in}{3.124954in}}%
\pgfpathlineto{\pgfqpoint{4.121031in}{3.118190in}}%
\pgfpathlineto{\pgfqpoint{4.123708in}{3.117651in}}%
\pgfpathlineto{\pgfqpoint{4.126384in}{3.119455in}}%
\pgfpathlineto{\pgfqpoint{4.130933in}{3.126032in}}%
\pgfpathlineto{\pgfqpoint{4.136017in}{3.132282in}}%
\pgfpathlineto{\pgfqpoint{4.138694in}{3.132745in}}%
\pgfpathlineto{\pgfqpoint{4.141370in}{3.130873in}}%
\pgfpathlineto{\pgfqpoint{4.146187in}{3.123824in}}%
\pgfpathlineto{\pgfqpoint{4.151004in}{3.118086in}}%
\pgfpathlineto{\pgfqpoint{4.153680in}{3.117698in}}%
\pgfpathlineto{\pgfqpoint{4.156356in}{3.119639in}}%
\pgfpathlineto{\pgfqpoint{4.161173in}{3.126730in}}%
\pgfpathlineto{\pgfqpoint{4.165990in}{3.132382in}}%
\pgfpathlineto{\pgfqpoint{4.168666in}{3.132693in}}%
\pgfpathlineto{\pgfqpoint{4.171342in}{3.130685in}}%
\pgfpathlineto{\pgfqpoint{4.176426in}{3.123136in}}%
\pgfpathlineto{\pgfqpoint{4.180976in}{3.117990in}}%
\pgfpathlineto{\pgfqpoint{4.183652in}{3.117755in}}%
\pgfpathlineto{\pgfqpoint{4.186595in}{3.120147in}}%
\pgfpathlineto{\pgfqpoint{4.198103in}{3.132793in}}%
\pgfpathlineto{\pgfqpoint{4.200779in}{3.131079in}}%
\pgfpathlineto{\pgfqpoint{4.205060in}{3.124994in}}%
\pgfpathlineto{\pgfqpoint{4.210680in}{3.118043in}}%
\pgfpathlineto{\pgfqpoint{4.213356in}{3.117722in}}%
\pgfpathlineto{\pgfqpoint{4.216032in}{3.119721in}}%
\pgfpathlineto{\pgfqpoint{4.221117in}{3.127264in}}%
\pgfpathlineto{\pgfqpoint{4.225666in}{3.132422in}}%
\pgfpathlineto{\pgfqpoint{4.228342in}{3.132668in}}%
\pgfpathlineto{\pgfqpoint{4.231286in}{3.130287in}}%
\pgfpathlineto{\pgfqpoint{4.242793in}{3.117623in}}%
\pgfpathlineto{\pgfqpoint{4.245469in}{3.119327in}}%
\pgfpathlineto{\pgfqpoint{4.249751in}{3.125406in}}%
\pgfpathlineto{\pgfqpoint{4.255371in}{3.132368in}}%
\pgfpathlineto{\pgfqpoint{4.258047in}{3.132701in}}%
\pgfpathlineto{\pgfqpoint{4.260723in}{3.130712in}}%
\pgfpathlineto{\pgfqpoint{4.265807in}{3.123174in}}%
\pgfpathlineto{\pgfqpoint{4.270357in}{3.118003in}}%
\pgfpathlineto{\pgfqpoint{4.273033in}{3.117747in}}%
\pgfpathlineto{\pgfqpoint{4.275976in}{3.120117in}}%
\pgfpathlineto{\pgfqpoint{4.287484in}{3.132798in}}%
\pgfpathlineto{\pgfqpoint{4.290160in}{3.131104in}}%
\pgfpathlineto{\pgfqpoint{4.294441in}{3.125033in}}%
\pgfpathlineto{\pgfqpoint{4.300061in}{3.118057in}}%
\pgfpathlineto{\pgfqpoint{4.302737in}{3.117714in}}%
\pgfpathlineto{\pgfqpoint{4.305413in}{3.119693in}}%
\pgfpathlineto{\pgfqpoint{4.310498in}{3.127226in}}%
\pgfpathlineto{\pgfqpoint{4.315047in}{3.132409in}}%
\pgfpathlineto{\pgfqpoint{4.317723in}{3.132677in}}%
\pgfpathlineto{\pgfqpoint{4.320399in}{3.130629in}}%
\pgfpathlineto{\pgfqpoint{4.325484in}{3.123061in}}%
\pgfpathlineto{\pgfqpoint{4.330033in}{3.117964in}}%
\pgfpathlineto{\pgfqpoint{4.332709in}{3.117773in}}%
\pgfpathlineto{\pgfqpoint{4.335653in}{3.120206in}}%
\pgfpathlineto{\pgfqpoint{4.346892in}{3.132832in}}%
\pgfpathlineto{\pgfqpoint{4.349569in}{3.131299in}}%
\pgfpathlineto{\pgfqpoint{4.353583in}{3.125778in}}%
\pgfpathlineto{\pgfqpoint{4.359470in}{3.118175in}}%
\pgfpathlineto{\pgfqpoint{4.362146in}{3.117657in}}%
\pgfpathlineto{\pgfqpoint{4.364822in}{3.119481in}}%
\pgfpathlineto{\pgfqpoint{4.369372in}{3.126071in}}%
\pgfpathlineto{\pgfqpoint{4.374456in}{3.132297in}}%
\pgfpathlineto{\pgfqpoint{4.377132in}{3.132739in}}%
\pgfpathlineto{\pgfqpoint{4.379808in}{3.130846in}}%
\pgfpathlineto{\pgfqpoint{4.384625in}{3.123786in}}%
\pgfpathlineto{\pgfqpoint{4.389442in}{3.118071in}}%
\pgfpathlineto{\pgfqpoint{4.392118in}{3.117706in}}%
\pgfpathlineto{\pgfqpoint{4.394794in}{3.119666in}}%
\pgfpathlineto{\pgfqpoint{4.399611in}{3.126768in}}%
\pgfpathlineto{\pgfqpoint{4.404428in}{3.132396in}}%
\pgfpathlineto{\pgfqpoint{4.407104in}{3.132685in}}%
\pgfpathlineto{\pgfqpoint{4.409780in}{3.130657in}}%
\pgfpathlineto{\pgfqpoint{4.414865in}{3.123099in}}%
\pgfpathlineto{\pgfqpoint{4.419414in}{3.117977in}}%
\pgfpathlineto{\pgfqpoint{4.422090in}{3.117764in}}%
\pgfpathlineto{\pgfqpoint{4.425034in}{3.120176in}}%
\pgfpathlineto{\pgfqpoint{4.436541in}{3.132788in}}%
\pgfpathlineto{\pgfqpoint{4.439217in}{3.131054in}}%
\pgfpathlineto{\pgfqpoint{4.443499in}{3.124954in}}%
\pgfpathlineto{\pgfqpoint{4.448851in}{3.118190in}}%
\pgfpathlineto{\pgfqpoint{4.451527in}{3.117651in}}%
\pgfpathlineto{\pgfqpoint{4.454203in}{3.119455in}}%
\pgfpathlineto{\pgfqpoint{4.458753in}{3.126032in}}%
\pgfpathlineto{\pgfqpoint{4.463837in}{3.132282in}}%
\pgfpathlineto{\pgfqpoint{4.466513in}{3.132745in}}%
\pgfpathlineto{\pgfqpoint{4.469189in}{3.130873in}}%
\pgfpathlineto{\pgfqpoint{4.474006in}{3.123824in}}%
\pgfpathlineto{\pgfqpoint{4.478823in}{3.118086in}}%
\pgfpathlineto{\pgfqpoint{4.481499in}{3.117698in}}%
\pgfpathlineto{\pgfqpoint{4.484175in}{3.119639in}}%
\pgfpathlineto{\pgfqpoint{4.488992in}{3.126730in}}%
\pgfpathlineto{\pgfqpoint{4.493809in}{3.132382in}}%
\pgfpathlineto{\pgfqpoint{4.496485in}{3.132693in}}%
\pgfpathlineto{\pgfqpoint{4.499161in}{3.130685in}}%
\pgfpathlineto{\pgfqpoint{4.504246in}{3.123136in}}%
\pgfpathlineto{\pgfqpoint{4.508795in}{3.117990in}}%
\pgfpathlineto{\pgfqpoint{4.511471in}{3.117755in}}%
\pgfpathlineto{\pgfqpoint{4.514415in}{3.120147in}}%
\pgfpathlineto{\pgfqpoint{4.525922in}{3.132793in}}%
\pgfpathlineto{\pgfqpoint{4.528598in}{3.131079in}}%
\pgfpathlineto{\pgfqpoint{4.532880in}{3.124994in}}%
\pgfpathlineto{\pgfqpoint{4.538500in}{3.118043in}}%
\pgfpathlineto{\pgfqpoint{4.541176in}{3.117722in}}%
\pgfpathlineto{\pgfqpoint{4.543852in}{3.119721in}}%
\pgfpathlineto{\pgfqpoint{4.548936in}{3.127264in}}%
\pgfpathlineto{\pgfqpoint{4.553486in}{3.132422in}}%
\pgfpathlineto{\pgfqpoint{4.556162in}{3.132668in}}%
\pgfpathlineto{\pgfqpoint{4.559105in}{3.130287in}}%
\pgfpathlineto{\pgfqpoint{4.570613in}{3.117623in}}%
\pgfpathlineto{\pgfqpoint{4.573289in}{3.119327in}}%
\pgfpathlineto{\pgfqpoint{4.577570in}{3.125406in}}%
\pgfpathlineto{\pgfqpoint{4.583190in}{3.132368in}}%
\pgfpathlineto{\pgfqpoint{4.585866in}{3.132701in}}%
\pgfpathlineto{\pgfqpoint{4.588542in}{3.130712in}}%
\pgfpathlineto{\pgfqpoint{4.593627in}{3.123174in}}%
\pgfpathlineto{\pgfqpoint{4.598176in}{3.118003in}}%
\pgfpathlineto{\pgfqpoint{4.600852in}{3.117747in}}%
\pgfpathlineto{\pgfqpoint{4.603796in}{3.120117in}}%
\pgfpathlineto{\pgfqpoint{4.615303in}{3.132798in}}%
\pgfpathlineto{\pgfqpoint{4.617979in}{3.131104in}}%
\pgfpathlineto{\pgfqpoint{4.622261in}{3.125033in}}%
\pgfpathlineto{\pgfqpoint{4.627881in}{3.118057in}}%
\pgfpathlineto{\pgfqpoint{4.630557in}{3.117714in}}%
\pgfpathlineto{\pgfqpoint{4.633233in}{3.119693in}}%
\pgfpathlineto{\pgfqpoint{4.638317in}{3.127226in}}%
\pgfpathlineto{\pgfqpoint{4.642867in}{3.132409in}}%
\pgfpathlineto{\pgfqpoint{4.645543in}{3.132677in}}%
\pgfpathlineto{\pgfqpoint{4.648219in}{3.130629in}}%
\pgfpathlineto{\pgfqpoint{4.653303in}{3.123061in}}%
\pgfpathlineto{\pgfqpoint{4.657853in}{3.117964in}}%
\pgfpathlineto{\pgfqpoint{4.660529in}{3.117773in}}%
\pgfpathlineto{\pgfqpoint{4.663473in}{3.120206in}}%
\pgfpathlineto{\pgfqpoint{4.674712in}{3.132832in}}%
\pgfpathlineto{\pgfqpoint{4.677388in}{3.131299in}}%
\pgfpathlineto{\pgfqpoint{4.681402in}{3.125778in}}%
\pgfpathlineto{\pgfqpoint{4.687290in}{3.118175in}}%
\pgfpathlineto{\pgfqpoint{4.689966in}{3.117657in}}%
\pgfpathlineto{\pgfqpoint{4.692642in}{3.119481in}}%
\pgfpathlineto{\pgfqpoint{4.697191in}{3.126071in}}%
\pgfpathlineto{\pgfqpoint{4.702276in}{3.132297in}}%
\pgfpathlineto{\pgfqpoint{4.704952in}{3.132739in}}%
\pgfpathlineto{\pgfqpoint{4.707628in}{3.130846in}}%
\pgfpathlineto{\pgfqpoint{4.712445in}{3.123786in}}%
\pgfpathlineto{\pgfqpoint{4.717262in}{3.118071in}}%
\pgfpathlineto{\pgfqpoint{4.719938in}{3.117706in}}%
\pgfpathlineto{\pgfqpoint{4.722614in}{3.119666in}}%
\pgfpathlineto{\pgfqpoint{4.727431in}{3.126768in}}%
\pgfpathlineto{\pgfqpoint{4.732248in}{3.132396in}}%
\pgfpathlineto{\pgfqpoint{4.734924in}{3.132685in}}%
\pgfpathlineto{\pgfqpoint{4.737600in}{3.130657in}}%
\pgfpathlineto{\pgfqpoint{4.742684in}{3.123099in}}%
\pgfpathlineto{\pgfqpoint{4.747234in}{3.117977in}}%
\pgfpathlineto{\pgfqpoint{4.749910in}{3.117764in}}%
\pgfpathlineto{\pgfqpoint{4.752854in}{3.120176in}}%
\pgfpathlineto{\pgfqpoint{4.764361in}{3.132788in}}%
\pgfpathlineto{\pgfqpoint{4.767037in}{3.131054in}}%
\pgfpathlineto{\pgfqpoint{4.771318in}{3.124954in}}%
\pgfpathlineto{\pgfqpoint{4.776671in}{3.118190in}}%
\pgfpathlineto{\pgfqpoint{4.779347in}{3.117651in}}%
\pgfpathlineto{\pgfqpoint{4.782023in}{3.119455in}}%
\pgfpathlineto{\pgfqpoint{4.786572in}{3.126032in}}%
\pgfpathlineto{\pgfqpoint{4.791657in}{3.132282in}}%
\pgfpathlineto{\pgfqpoint{4.794333in}{3.132745in}}%
\pgfpathlineto{\pgfqpoint{4.797009in}{3.130873in}}%
\pgfpathlineto{\pgfqpoint{4.801826in}{3.123824in}}%
\pgfpathlineto{\pgfqpoint{4.806643in}{3.118086in}}%
\pgfpathlineto{\pgfqpoint{4.809319in}{3.117698in}}%
\pgfpathlineto{\pgfqpoint{4.811995in}{3.119639in}}%
\pgfpathlineto{\pgfqpoint{4.816812in}{3.126730in}}%
\pgfpathlineto{\pgfqpoint{4.821629in}{3.132382in}}%
\pgfpathlineto{\pgfqpoint{4.824305in}{3.132693in}}%
\pgfpathlineto{\pgfqpoint{4.826981in}{3.130685in}}%
\pgfpathlineto{\pgfqpoint{4.832065in}{3.123136in}}%
\pgfpathlineto{\pgfqpoint{4.836615in}{3.117990in}}%
\pgfpathlineto{\pgfqpoint{4.839291in}{3.117755in}}%
\pgfpathlineto{\pgfqpoint{4.842235in}{3.120147in}}%
\pgfpathlineto{\pgfqpoint{4.853742in}{3.132793in}}%
\pgfpathlineto{\pgfqpoint{4.856418in}{3.131079in}}%
\pgfpathlineto{\pgfqpoint{4.860699in}{3.124994in}}%
\pgfpathlineto{\pgfqpoint{4.866319in}{3.118043in}}%
\pgfpathlineto{\pgfqpoint{4.868995in}{3.117722in}}%
\pgfpathlineto{\pgfqpoint{4.871671in}{3.119721in}}%
\pgfpathlineto{\pgfqpoint{4.876756in}{3.127264in}}%
\pgfpathlineto{\pgfqpoint{4.881305in}{3.132422in}}%
\pgfpathlineto{\pgfqpoint{4.883981in}{3.132668in}}%
\pgfpathlineto{\pgfqpoint{4.886925in}{3.130287in}}%
\pgfpathlineto{\pgfqpoint{4.898432in}{3.117623in}}%
\pgfpathlineto{\pgfqpoint{4.901108in}{3.119327in}}%
\pgfpathlineto{\pgfqpoint{4.905390in}{3.125406in}}%
\pgfpathlineto{\pgfqpoint{4.911010in}{3.132368in}}%
\pgfpathlineto{\pgfqpoint{4.913686in}{3.132701in}}%
\pgfpathlineto{\pgfqpoint{4.916362in}{3.130712in}}%
\pgfpathlineto{\pgfqpoint{4.921446in}{3.123174in}}%
\pgfpathlineto{\pgfqpoint{4.925996in}{3.118003in}}%
\pgfpathlineto{\pgfqpoint{4.928672in}{3.117747in}}%
\pgfpathlineto{\pgfqpoint{4.931616in}{3.120117in}}%
\pgfpathlineto{\pgfqpoint{4.943123in}{3.132798in}}%
\pgfpathlineto{\pgfqpoint{4.945799in}{3.131104in}}%
\pgfpathlineto{\pgfqpoint{4.950080in}{3.125033in}}%
\pgfpathlineto{\pgfqpoint{4.955700in}{3.118057in}}%
\pgfpathlineto{\pgfqpoint{4.958376in}{3.117714in}}%
\pgfpathlineto{\pgfqpoint{4.961052in}{3.119693in}}%
\pgfpathlineto{\pgfqpoint{4.966137in}{3.127226in}}%
\pgfpathlineto{\pgfqpoint{4.970686in}{3.132409in}}%
\pgfpathlineto{\pgfqpoint{4.973362in}{3.132677in}}%
\pgfpathlineto{\pgfqpoint{4.976038in}{3.130629in}}%
\pgfpathlineto{\pgfqpoint{4.981123in}{3.123061in}}%
\pgfpathlineto{\pgfqpoint{4.985672in}{3.117964in}}%
\pgfpathlineto{\pgfqpoint{4.988348in}{3.117773in}}%
\pgfpathlineto{\pgfqpoint{4.991292in}{3.120206in}}%
\pgfpathlineto{\pgfqpoint{5.002532in}{3.132832in}}%
\pgfpathlineto{\pgfqpoint{5.005208in}{3.131299in}}%
\pgfpathlineto{\pgfqpoint{5.009222in}{3.125778in}}%
\pgfpathlineto{\pgfqpoint{5.015109in}{3.118175in}}%
\pgfpathlineto{\pgfqpoint{5.017785in}{3.117657in}}%
\pgfpathlineto{\pgfqpoint{5.020461in}{3.119481in}}%
\pgfpathlineto{\pgfqpoint{5.025011in}{3.126071in}}%
\pgfpathlineto{\pgfqpoint{5.030095in}{3.132297in}}%
\pgfpathlineto{\pgfqpoint{5.032771in}{3.132739in}}%
\pgfpathlineto{\pgfqpoint{5.035447in}{3.130846in}}%
\pgfpathlineto{\pgfqpoint{5.040264in}{3.123786in}}%
\pgfpathlineto{\pgfqpoint{5.045081in}{3.118071in}}%
\pgfpathlineto{\pgfqpoint{5.047757in}{3.117706in}}%
\pgfpathlineto{\pgfqpoint{5.050433in}{3.119666in}}%
\pgfpathlineto{\pgfqpoint{5.055250in}{3.126768in}}%
\pgfpathlineto{\pgfqpoint{5.060067in}{3.132396in}}%
\pgfpathlineto{\pgfqpoint{5.062743in}{3.132685in}}%
\pgfpathlineto{\pgfqpoint{5.065419in}{3.130657in}}%
\pgfpathlineto{\pgfqpoint{5.070504in}{3.123099in}}%
\pgfpathlineto{\pgfqpoint{5.075053in}{3.117977in}}%
\pgfpathlineto{\pgfqpoint{5.077729in}{3.117764in}}%
\pgfpathlineto{\pgfqpoint{5.080673in}{3.120176in}}%
\pgfpathlineto{\pgfqpoint{5.092180in}{3.132788in}}%
\pgfpathlineto{\pgfqpoint{5.094856in}{3.131054in}}%
\pgfpathlineto{\pgfqpoint{5.099138in}{3.124954in}}%
\pgfpathlineto{\pgfqpoint{5.104490in}{3.118190in}}%
\pgfpathlineto{\pgfqpoint{5.107166in}{3.117651in}}%
\pgfpathlineto{\pgfqpoint{5.109842in}{3.119455in}}%
\pgfpathlineto{\pgfqpoint{5.114392in}{3.126032in}}%
\pgfpathlineto{\pgfqpoint{5.119476in}{3.132282in}}%
\pgfpathlineto{\pgfqpoint{5.122152in}{3.132745in}}%
\pgfpathlineto{\pgfqpoint{5.124828in}{3.130873in}}%
\pgfpathlineto{\pgfqpoint{5.129645in}{3.123824in}}%
\pgfpathlineto{\pgfqpoint{5.134462in}{3.118086in}}%
\pgfpathlineto{\pgfqpoint{5.137138in}{3.117698in}}%
\pgfpathlineto{\pgfqpoint{5.139814in}{3.119639in}}%
\pgfpathlineto{\pgfqpoint{5.144631in}{3.126730in}}%
\pgfpathlineto{\pgfqpoint{5.149448in}{3.132382in}}%
\pgfpathlineto{\pgfqpoint{5.152124in}{3.132693in}}%
\pgfpathlineto{\pgfqpoint{5.154800in}{3.130685in}}%
\pgfpathlineto{\pgfqpoint{5.159885in}{3.123136in}}%
\pgfpathlineto{\pgfqpoint{5.164434in}{3.117990in}}%
\pgfpathlineto{\pgfqpoint{5.167110in}{3.117755in}}%
\pgfpathlineto{\pgfqpoint{5.170054in}{3.120147in}}%
\pgfpathlineto{\pgfqpoint{5.181561in}{3.132793in}}%
\pgfpathlineto{\pgfqpoint{5.184237in}{3.131079in}}%
\pgfpathlineto{\pgfqpoint{5.188519in}{3.124994in}}%
\pgfpathlineto{\pgfqpoint{5.194139in}{3.118043in}}%
\pgfpathlineto{\pgfqpoint{5.196815in}{3.117722in}}%
\pgfpathlineto{\pgfqpoint{5.199491in}{3.119721in}}%
\pgfpathlineto{\pgfqpoint{5.204576in}{3.127264in}}%
\pgfpathlineto{\pgfqpoint{5.209125in}{3.132422in}}%
\pgfpathlineto{\pgfqpoint{5.211801in}{3.132668in}}%
\pgfpathlineto{\pgfqpoint{5.214745in}{3.130287in}}%
\pgfpathlineto{\pgfqpoint{5.226252in}{3.117623in}}%
\pgfpathlineto{\pgfqpoint{5.228928in}{3.119327in}}%
\pgfpathlineto{\pgfqpoint{5.233210in}{3.125406in}}%
\pgfpathlineto{\pgfqpoint{5.238829in}{3.132368in}}%
\pgfpathlineto{\pgfqpoint{5.241505in}{3.132701in}}%
\pgfpathlineto{\pgfqpoint{5.244181in}{3.130712in}}%
\pgfpathlineto{\pgfqpoint{5.249266in}{3.123174in}}%
\pgfpathlineto{\pgfqpoint{5.253815in}{3.118003in}}%
\pgfpathlineto{\pgfqpoint{5.256491in}{3.117747in}}%
\pgfpathlineto{\pgfqpoint{5.259435in}{3.120117in}}%
\pgfpathlineto{\pgfqpoint{5.270942in}{3.132798in}}%
\pgfpathlineto{\pgfqpoint{5.273618in}{3.131104in}}%
\pgfpathlineto{\pgfqpoint{5.277900in}{3.125033in}}%
\pgfpathlineto{\pgfqpoint{5.283520in}{3.118057in}}%
\pgfpathlineto{\pgfqpoint{5.286196in}{3.117714in}}%
\pgfpathlineto{\pgfqpoint{5.288872in}{3.119693in}}%
\pgfpathlineto{\pgfqpoint{5.293957in}{3.127226in}}%
\pgfpathlineto{\pgfqpoint{5.298506in}{3.132409in}}%
\pgfpathlineto{\pgfqpoint{5.301182in}{3.132677in}}%
\pgfpathlineto{\pgfqpoint{5.303858in}{3.130629in}}%
\pgfpathlineto{\pgfqpoint{5.308943in}{3.123061in}}%
\pgfpathlineto{\pgfqpoint{5.313492in}{3.117964in}}%
\pgfpathlineto{\pgfqpoint{5.316168in}{3.117773in}}%
\pgfpathlineto{\pgfqpoint{5.319112in}{3.120206in}}%
\pgfpathlineto{\pgfqpoint{5.330351in}{3.132832in}}%
\pgfpathlineto{\pgfqpoint{5.333027in}{3.131299in}}%
\pgfpathlineto{\pgfqpoint{5.337041in}{3.125778in}}%
\pgfpathlineto{\pgfqpoint{5.342929in}{3.118175in}}%
\pgfpathlineto{\pgfqpoint{5.345605in}{3.117657in}}%
\pgfpathlineto{\pgfqpoint{5.348281in}{3.119481in}}%
\pgfpathlineto{\pgfqpoint{5.352830in}{3.126071in}}%
\pgfpathlineto{\pgfqpoint{5.357915in}{3.132297in}}%
\pgfpathlineto{\pgfqpoint{5.360591in}{3.132739in}}%
\pgfpathlineto{\pgfqpoint{5.363267in}{3.130846in}}%
\pgfpathlineto{\pgfqpoint{5.368084in}{3.123786in}}%
\pgfpathlineto{\pgfqpoint{5.372901in}{3.118071in}}%
\pgfpathlineto{\pgfqpoint{5.375577in}{3.117706in}}%
\pgfpathlineto{\pgfqpoint{5.378253in}{3.119666in}}%
\pgfpathlineto{\pgfqpoint{5.383070in}{3.126768in}}%
\pgfpathlineto{\pgfqpoint{5.387887in}{3.132396in}}%
\pgfpathlineto{\pgfqpoint{5.390563in}{3.132685in}}%
\pgfpathlineto{\pgfqpoint{5.393239in}{3.130657in}}%
\pgfpathlineto{\pgfqpoint{5.398324in}{3.123099in}}%
\pgfpathlineto{\pgfqpoint{5.402873in}{3.117977in}}%
\pgfpathlineto{\pgfqpoint{5.405549in}{3.117764in}}%
\pgfpathlineto{\pgfqpoint{5.408493in}{3.120176in}}%
\pgfpathlineto{\pgfqpoint{5.420000in}{3.132788in}}%
\pgfpathlineto{\pgfqpoint{5.422676in}{3.131054in}}%
\pgfpathlineto{\pgfqpoint{5.426958in}{3.124954in}}%
\pgfpathlineto{\pgfqpoint{5.432310in}{3.118190in}}%
\pgfpathlineto{\pgfqpoint{5.434986in}{3.117651in}}%
\pgfpathlineto{\pgfqpoint{5.437662in}{3.119455in}}%
\pgfpathlineto{\pgfqpoint{5.442211in}{3.126032in}}%
\pgfpathlineto{\pgfqpoint{5.447296in}{3.132282in}}%
\pgfpathlineto{\pgfqpoint{5.449972in}{3.132745in}}%
\pgfpathlineto{\pgfqpoint{5.452648in}{3.130873in}}%
\pgfpathlineto{\pgfqpoint{5.457465in}{3.123824in}}%
\pgfpathlineto{\pgfqpoint{5.462282in}{3.118086in}}%
\pgfpathlineto{\pgfqpoint{5.464958in}{3.117698in}}%
\pgfpathlineto{\pgfqpoint{5.467634in}{3.119639in}}%
\pgfpathlineto{\pgfqpoint{5.472451in}{3.126730in}}%
\pgfpathlineto{\pgfqpoint{5.477268in}{3.132382in}}%
\pgfpathlineto{\pgfqpoint{5.479944in}{3.132693in}}%
\pgfpathlineto{\pgfqpoint{5.482620in}{3.130685in}}%
\pgfpathlineto{\pgfqpoint{5.487705in}{3.123136in}}%
\pgfpathlineto{\pgfqpoint{5.492254in}{3.117990in}}%
\pgfpathlineto{\pgfqpoint{5.494930in}{3.117755in}}%
\pgfpathlineto{\pgfqpoint{5.497874in}{3.120147in}}%
\pgfpathlineto{\pgfqpoint{5.509381in}{3.132793in}}%
\pgfpathlineto{\pgfqpoint{5.512057in}{3.131079in}}%
\pgfpathlineto{\pgfqpoint{5.516339in}{3.124994in}}%
\pgfpathlineto{\pgfqpoint{5.521958in}{3.118043in}}%
\pgfpathlineto{\pgfqpoint{5.524634in}{3.117722in}}%
\pgfpathlineto{\pgfqpoint{5.527311in}{3.119721in}}%
\pgfpathlineto{\pgfqpoint{5.532395in}{3.127264in}}%
\pgfpathlineto{\pgfqpoint{5.536944in}{3.132422in}}%
\pgfpathlineto{\pgfqpoint{5.539621in}{3.132668in}}%
\pgfpathlineto{\pgfqpoint{5.542564in}{3.130287in}}%
\pgfpathlineto{\pgfqpoint{5.554071in}{3.117623in}}%
\pgfpathlineto{\pgfqpoint{5.556747in}{3.119327in}}%
\pgfpathlineto{\pgfqpoint{5.561029in}{3.125406in}}%
\pgfpathlineto{\pgfqpoint{5.566649in}{3.132368in}}%
\pgfpathlineto{\pgfqpoint{5.569325in}{3.132701in}}%
\pgfpathlineto{\pgfqpoint{5.572001in}{3.130712in}}%
\pgfpathlineto{\pgfqpoint{5.577086in}{3.123174in}}%
\pgfpathlineto{\pgfqpoint{5.581635in}{3.118003in}}%
\pgfpathlineto{\pgfqpoint{5.584311in}{3.117747in}}%
\pgfpathlineto{\pgfqpoint{5.587255in}{3.120117in}}%
\pgfpathlineto{\pgfqpoint{5.598762in}{3.132798in}}%
\pgfpathlineto{\pgfqpoint{5.601438in}{3.131104in}}%
\pgfpathlineto{\pgfqpoint{5.605720in}{3.125033in}}%
\pgfpathlineto{\pgfqpoint{5.611339in}{3.118057in}}%
\pgfpathlineto{\pgfqpoint{5.614015in}{3.117714in}}%
\pgfpathlineto{\pgfqpoint{5.616692in}{3.119693in}}%
\pgfpathlineto{\pgfqpoint{5.621776in}{3.127226in}}%
\pgfpathlineto{\pgfqpoint{5.626325in}{3.132409in}}%
\pgfpathlineto{\pgfqpoint{5.629002in}{3.132677in}}%
\pgfpathlineto{\pgfqpoint{5.631678in}{3.130629in}}%
\pgfpathlineto{\pgfqpoint{5.636762in}{3.123061in}}%
\pgfpathlineto{\pgfqpoint{5.641311in}{3.117964in}}%
\pgfpathlineto{\pgfqpoint{5.643988in}{3.117773in}}%
\pgfpathlineto{\pgfqpoint{5.646931in}{3.120206in}}%
\pgfpathlineto{\pgfqpoint{5.658171in}{3.132832in}}%
\pgfpathlineto{\pgfqpoint{5.660847in}{3.131299in}}%
\pgfpathlineto{\pgfqpoint{5.664861in}{3.125778in}}%
\pgfpathlineto{\pgfqpoint{5.670748in}{3.118175in}}%
\pgfpathlineto{\pgfqpoint{5.673424in}{3.117657in}}%
\pgfpathlineto{\pgfqpoint{5.676101in}{3.119481in}}%
\pgfpathlineto{\pgfqpoint{5.680650in}{3.126071in}}%
\pgfpathlineto{\pgfqpoint{5.685734in}{3.132297in}}%
\pgfpathlineto{\pgfqpoint{5.688410in}{3.132739in}}%
\pgfpathlineto{\pgfqpoint{5.691087in}{3.130846in}}%
\pgfpathlineto{\pgfqpoint{5.695903in}{3.123786in}}%
\pgfpathlineto{\pgfqpoint{5.700720in}{3.118071in}}%
\pgfpathlineto{\pgfqpoint{5.703397in}{3.117706in}}%
\pgfpathlineto{\pgfqpoint{5.706073in}{3.119666in}}%
\pgfpathlineto{\pgfqpoint{5.710890in}{3.126768in}}%
\pgfpathlineto{\pgfqpoint{5.715706in}{3.132396in}}%
\pgfpathlineto{\pgfqpoint{5.718383in}{3.132685in}}%
\pgfpathlineto{\pgfqpoint{5.721059in}{3.130657in}}%
\pgfpathlineto{\pgfqpoint{5.726143in}{3.123099in}}%
\pgfpathlineto{\pgfqpoint{5.730693in}{3.117977in}}%
\pgfpathlineto{\pgfqpoint{5.733369in}{3.117764in}}%
\pgfpathlineto{\pgfqpoint{5.736312in}{3.120176in}}%
\pgfpathlineto{\pgfqpoint{5.747819in}{3.132788in}}%
\pgfpathlineto{\pgfqpoint{5.750495in}{3.131054in}}%
\pgfpathlineto{\pgfqpoint{5.754777in}{3.124954in}}%
\pgfpathlineto{\pgfqpoint{5.760129in}{3.118190in}}%
\pgfpathlineto{\pgfqpoint{5.762805in}{3.117651in}}%
\pgfpathlineto{\pgfqpoint{5.765482in}{3.119455in}}%
\pgfpathlineto{\pgfqpoint{5.770031in}{3.126032in}}%
\pgfpathlineto{\pgfqpoint{5.775115in}{3.132282in}}%
\pgfpathlineto{\pgfqpoint{5.777791in}{3.132745in}}%
\pgfpathlineto{\pgfqpoint{5.780468in}{3.130873in}}%
\pgfpathlineto{\pgfqpoint{5.785284in}{3.123824in}}%
\pgfpathlineto{\pgfqpoint{5.790101in}{3.118086in}}%
\pgfpathlineto{\pgfqpoint{5.792778in}{3.117698in}}%
\pgfpathlineto{\pgfqpoint{5.795454in}{3.119639in}}%
\pgfpathlineto{\pgfqpoint{5.800271in}{3.126730in}}%
\pgfpathlineto{\pgfqpoint{5.805087in}{3.132382in}}%
\pgfpathlineto{\pgfqpoint{5.807764in}{3.132693in}}%
\pgfpathlineto{\pgfqpoint{5.810440in}{3.130685in}}%
\pgfpathlineto{\pgfqpoint{5.815524in}{3.123136in}}%
\pgfpathlineto{\pgfqpoint{5.820074in}{3.117990in}}%
\pgfpathlineto{\pgfqpoint{5.822750in}{3.117755in}}%
\pgfpathlineto{\pgfqpoint{5.825693in}{3.120147in}}%
\pgfpathlineto{\pgfqpoint{5.837200in}{3.132793in}}%
\pgfpathlineto{\pgfqpoint{5.839876in}{3.131079in}}%
\pgfpathlineto{\pgfqpoint{5.844158in}{3.124994in}}%
\pgfpathlineto{\pgfqpoint{5.849778in}{3.118043in}}%
\pgfpathlineto{\pgfqpoint{5.852454in}{3.117722in}}%
\pgfpathlineto{\pgfqpoint{5.855130in}{3.119721in}}%
\pgfpathlineto{\pgfqpoint{5.860215in}{3.127264in}}%
\pgfpathlineto{\pgfqpoint{5.864764in}{3.132422in}}%
\pgfpathlineto{\pgfqpoint{5.867440in}{3.132668in}}%
\pgfpathlineto{\pgfqpoint{5.870384in}{3.130287in}}%
\pgfpathlineto{\pgfqpoint{5.881891in}{3.117623in}}%
\pgfpathlineto{\pgfqpoint{5.884567in}{3.119327in}}%
\pgfpathlineto{\pgfqpoint{5.888849in}{3.125406in}}%
\pgfpathlineto{\pgfqpoint{5.894468in}{3.132368in}}%
\pgfpathlineto{\pgfqpoint{5.897145in}{3.132701in}}%
\pgfpathlineto{\pgfqpoint{5.899821in}{3.130712in}}%
\pgfpathlineto{\pgfqpoint{5.904905in}{3.123174in}}%
\pgfpathlineto{\pgfqpoint{5.909455in}{3.118003in}}%
\pgfpathlineto{\pgfqpoint{5.912131in}{3.117747in}}%
\pgfpathlineto{\pgfqpoint{5.915074in}{3.120117in}}%
\pgfpathlineto{\pgfqpoint{5.926581in}{3.132798in}}%
\pgfpathlineto{\pgfqpoint{5.929258in}{3.131104in}}%
\pgfpathlineto{\pgfqpoint{5.933539in}{3.125033in}}%
\pgfpathlineto{\pgfqpoint{5.939159in}{3.118057in}}%
\pgfpathlineto{\pgfqpoint{5.941835in}{3.117714in}}%
\pgfpathlineto{\pgfqpoint{5.944511in}{3.119693in}}%
\pgfpathlineto{\pgfqpoint{5.949596in}{3.127226in}}%
\pgfpathlineto{\pgfqpoint{5.954145in}{3.132409in}}%
\pgfpathlineto{\pgfqpoint{5.956821in}{3.132677in}}%
\pgfpathlineto{\pgfqpoint{5.959497in}{3.130629in}}%
\pgfpathlineto{\pgfqpoint{5.964582in}{3.123061in}}%
\pgfpathlineto{\pgfqpoint{5.969131in}{3.117964in}}%
\pgfpathlineto{\pgfqpoint{5.971807in}{3.117773in}}%
\pgfpathlineto{\pgfqpoint{5.974751in}{3.120206in}}%
\pgfpathlineto{\pgfqpoint{5.985990in}{3.132832in}}%
\pgfpathlineto{\pgfqpoint{5.988666in}{3.131299in}}%
\pgfpathlineto{\pgfqpoint{5.992681in}{3.125778in}}%
\pgfpathlineto{\pgfqpoint{5.998568in}{3.118175in}}%
\pgfpathlineto{\pgfqpoint{6.001244in}{3.117657in}}%
\pgfpathlineto{\pgfqpoint{6.003920in}{3.119481in}}%
\pgfpathlineto{\pgfqpoint{6.008469in}{3.126071in}}%
\pgfpathlineto{\pgfqpoint{6.013554in}{3.132297in}}%
\pgfpathlineto{\pgfqpoint{6.016230in}{3.132739in}}%
\pgfpathlineto{\pgfqpoint{6.018906in}{3.130846in}}%
\pgfpathlineto{\pgfqpoint{6.023723in}{3.123786in}}%
\pgfpathlineto{\pgfqpoint{6.028540in}{3.118071in}}%
\pgfpathlineto{\pgfqpoint{6.031216in}{3.117706in}}%
\pgfpathlineto{\pgfqpoint{6.033892in}{3.119666in}}%
\pgfpathlineto{\pgfqpoint{6.038709in}{3.126768in}}%
\pgfpathlineto{\pgfqpoint{6.043526in}{3.132396in}}%
\pgfpathlineto{\pgfqpoint{6.046202in}{3.132685in}}%
\pgfpathlineto{\pgfqpoint{6.048878in}{3.130657in}}%
\pgfpathlineto{\pgfqpoint{6.053963in}{3.123099in}}%
\pgfpathlineto{\pgfqpoint{6.058512in}{3.117977in}}%
\pgfpathlineto{\pgfqpoint{6.061188in}{3.117764in}}%
\pgfpathlineto{\pgfqpoint{6.064132in}{3.120176in}}%
\pgfpathlineto{\pgfqpoint{6.075639in}{3.132788in}}%
\pgfpathlineto{\pgfqpoint{6.078315in}{3.131054in}}%
\pgfpathlineto{\pgfqpoint{6.082597in}{3.124954in}}%
\pgfpathlineto{\pgfqpoint{6.087949in}{3.118190in}}%
\pgfpathlineto{\pgfqpoint{6.090625in}{3.117651in}}%
\pgfpathlineto{\pgfqpoint{6.093301in}{3.119455in}}%
\pgfpathlineto{\pgfqpoint{6.097850in}{3.126032in}}%
\pgfpathlineto{\pgfqpoint{6.102935in}{3.132282in}}%
\pgfpathlineto{\pgfqpoint{6.105611in}{3.132745in}}%
\pgfpathlineto{\pgfqpoint{6.108287in}{3.130873in}}%
\pgfpathlineto{\pgfqpoint{6.113104in}{3.123824in}}%
\pgfpathlineto{\pgfqpoint{6.117921in}{3.118086in}}%
\pgfpathlineto{\pgfqpoint{6.120597in}{3.117698in}}%
\pgfpathlineto{\pgfqpoint{6.123273in}{3.119639in}}%
\pgfpathlineto{\pgfqpoint{6.128090in}{3.126730in}}%
\pgfpathlineto{\pgfqpoint{6.132907in}{3.132382in}}%
\pgfpathlineto{\pgfqpoint{6.135583in}{3.132693in}}%
\pgfpathlineto{\pgfqpoint{6.138259in}{3.130685in}}%
\pgfpathlineto{\pgfqpoint{6.143344in}{3.123136in}}%
\pgfpathlineto{\pgfqpoint{6.147893in}{3.117990in}}%
\pgfpathlineto{\pgfqpoint{6.150569in}{3.117755in}}%
\pgfpathlineto{\pgfqpoint{6.153513in}{3.120147in}}%
\pgfpathlineto{\pgfqpoint{6.165020in}{3.132793in}}%
\pgfpathlineto{\pgfqpoint{6.167696in}{3.131079in}}%
\pgfpathlineto{\pgfqpoint{6.171978in}{3.124994in}}%
\pgfpathlineto{\pgfqpoint{6.177598in}{3.118043in}}%
\pgfpathlineto{\pgfqpoint{6.180274in}{3.117722in}}%
\pgfpathlineto{\pgfqpoint{6.182950in}{3.119721in}}%
\pgfpathlineto{\pgfqpoint{6.188034in}{3.127264in}}%
\pgfpathlineto{\pgfqpoint{6.192584in}{3.132422in}}%
\pgfpathlineto{\pgfqpoint{6.195260in}{3.132668in}}%
\pgfpathlineto{\pgfqpoint{6.198203in}{3.130287in}}%
\pgfpathlineto{\pgfqpoint{6.209711in}{3.117623in}}%
\pgfpathlineto{\pgfqpoint{6.212387in}{3.119327in}}%
\pgfpathlineto{\pgfqpoint{6.216668in}{3.125406in}}%
\pgfpathlineto{\pgfqpoint{6.222288in}{3.132368in}}%
\pgfpathlineto{\pgfqpoint{6.224964in}{3.132701in}}%
\pgfpathlineto{\pgfqpoint{6.227640in}{3.130712in}}%
\pgfpathlineto{\pgfqpoint{6.232725in}{3.123174in}}%
\pgfpathlineto{\pgfqpoint{6.237274in}{3.118003in}}%
\pgfpathlineto{\pgfqpoint{6.239950in}{3.117747in}}%
\pgfpathlineto{\pgfqpoint{6.242894in}{3.120117in}}%
\pgfpathlineto{\pgfqpoint{6.254401in}{3.132798in}}%
\pgfpathlineto{\pgfqpoint{6.257077in}{3.131104in}}%
\pgfpathlineto{\pgfqpoint{6.261359in}{3.125033in}}%
\pgfpathlineto{\pgfqpoint{6.266979in}{3.118057in}}%
\pgfpathlineto{\pgfqpoint{6.269655in}{3.117714in}}%
\pgfpathlineto{\pgfqpoint{6.272331in}{3.119693in}}%
\pgfpathlineto{\pgfqpoint{6.277415in}{3.127226in}}%
\pgfpathlineto{\pgfqpoint{6.281965in}{3.132409in}}%
\pgfpathlineto{\pgfqpoint{6.284641in}{3.132677in}}%
\pgfpathlineto{\pgfqpoint{6.287317in}{3.130629in}}%
\pgfpathlineto{\pgfqpoint{6.292401in}{3.123061in}}%
\pgfpathlineto{\pgfqpoint{6.296951in}{3.117964in}}%
\pgfpathlineto{\pgfqpoint{6.299627in}{3.117773in}}%
\pgfpathlineto{\pgfqpoint{6.302570in}{3.120206in}}%
\pgfpathlineto{\pgfqpoint{6.313810in}{3.132832in}}%
\pgfpathlineto{\pgfqpoint{6.316486in}{3.131299in}}%
\pgfpathlineto{\pgfqpoint{6.320500in}{3.125778in}}%
\pgfpathlineto{\pgfqpoint{6.326388in}{3.118175in}}%
\pgfpathlineto{\pgfqpoint{6.329064in}{3.117657in}}%
\pgfpathlineto{\pgfqpoint{6.331740in}{3.119481in}}%
\pgfpathlineto{\pgfqpoint{6.336289in}{3.126071in}}%
\pgfpathlineto{\pgfqpoint{6.341374in}{3.132297in}}%
\pgfpathlineto{\pgfqpoint{6.344050in}{3.132739in}}%
\pgfpathlineto{\pgfqpoint{6.346726in}{3.130846in}}%
\pgfpathlineto{\pgfqpoint{6.351543in}{3.123786in}}%
\pgfpathlineto{\pgfqpoint{6.356360in}{3.118071in}}%
\pgfpathlineto{\pgfqpoint{6.359036in}{3.117706in}}%
\pgfpathlineto{\pgfqpoint{6.361712in}{3.119666in}}%
\pgfpathlineto{\pgfqpoint{6.366529in}{3.126768in}}%
\pgfpathlineto{\pgfqpoint{6.371346in}{3.132396in}}%
\pgfpathlineto{\pgfqpoint{6.374022in}{3.132685in}}%
\pgfpathlineto{\pgfqpoint{6.376698in}{3.130657in}}%
\pgfpathlineto{\pgfqpoint{6.381782in}{3.123099in}}%
\pgfpathlineto{\pgfqpoint{6.386332in}{3.117977in}}%
\pgfpathlineto{\pgfqpoint{6.389008in}{3.117764in}}%
\pgfpathlineto{\pgfqpoint{6.391951in}{3.120176in}}%
\pgfpathlineto{\pgfqpoint{6.403459in}{3.132788in}}%
\pgfpathlineto{\pgfqpoint{6.406135in}{3.131054in}}%
\pgfpathlineto{\pgfqpoint{6.410416in}{3.124954in}}%
\pgfpathlineto{\pgfqpoint{6.415769in}{3.118190in}}%
\pgfpathlineto{\pgfqpoint{6.418445in}{3.117651in}}%
\pgfpathlineto{\pgfqpoint{6.421121in}{3.119455in}}%
\pgfpathlineto{\pgfqpoint{6.425670in}{3.126032in}}%
\pgfpathlineto{\pgfqpoint{6.430755in}{3.132282in}}%
\pgfpathlineto{\pgfqpoint{6.433431in}{3.132745in}}%
\pgfpathlineto{\pgfqpoint{6.436107in}{3.130873in}}%
\pgfpathlineto{\pgfqpoint{6.440924in}{3.123824in}}%
\pgfpathlineto{\pgfqpoint{6.445741in}{3.118086in}}%
\pgfpathlineto{\pgfqpoint{6.448417in}{3.117698in}}%
\pgfpathlineto{\pgfqpoint{6.451093in}{3.119639in}}%
\pgfpathlineto{\pgfqpoint{6.455910in}{3.126730in}}%
\pgfpathlineto{\pgfqpoint{6.460727in}{3.132382in}}%
\pgfpathlineto{\pgfqpoint{6.463403in}{3.132693in}}%
\pgfpathlineto{\pgfqpoint{6.466079in}{3.130685in}}%
\pgfpathlineto{\pgfqpoint{6.471163in}{3.123136in}}%
\pgfpathlineto{\pgfqpoint{6.475713in}{3.117990in}}%
\pgfpathlineto{\pgfqpoint{6.478389in}{3.117755in}}%
\pgfpathlineto{\pgfqpoint{6.481332in}{3.120147in}}%
\pgfpathlineto{\pgfqpoint{6.492840in}{3.132793in}}%
\pgfpathlineto{\pgfqpoint{6.495516in}{3.131079in}}%
\pgfpathlineto{\pgfqpoint{6.499797in}{3.124994in}}%
\pgfpathlineto{\pgfqpoint{6.505417in}{3.118043in}}%
\pgfpathlineto{\pgfqpoint{6.508093in}{3.117722in}}%
\pgfpathlineto{\pgfqpoint{6.510769in}{3.119721in}}%
\pgfpathlineto{\pgfqpoint{6.515854in}{3.127264in}}%
\pgfpathlineto{\pgfqpoint{6.520403in}{3.132422in}}%
\pgfpathlineto{\pgfqpoint{6.523079in}{3.132668in}}%
\pgfpathlineto{\pgfqpoint{6.526023in}{3.130287in}}%
\pgfpathlineto{\pgfqpoint{6.537530in}{3.117623in}}%
\pgfpathlineto{\pgfqpoint{6.540206in}{3.119327in}}%
\pgfpathlineto{\pgfqpoint{6.544488in}{3.125406in}}%
\pgfpathlineto{\pgfqpoint{6.550108in}{3.132368in}}%
\pgfpathlineto{\pgfqpoint{6.552784in}{3.132701in}}%
\pgfpathlineto{\pgfqpoint{6.555460in}{3.130712in}}%
\pgfpathlineto{\pgfqpoint{6.560544in}{3.123174in}}%
\pgfpathlineto{\pgfqpoint{6.565094in}{3.118003in}}%
\pgfpathlineto{\pgfqpoint{6.567770in}{3.117747in}}%
\pgfpathlineto{\pgfqpoint{6.570713in}{3.120117in}}%
\pgfpathlineto{\pgfqpoint{6.582221in}{3.132798in}}%
\pgfpathlineto{\pgfqpoint{6.584897in}{3.131104in}}%
\pgfpathlineto{\pgfqpoint{6.589178in}{3.125033in}}%
\pgfpathlineto{\pgfqpoint{6.594798in}{3.118057in}}%
\pgfpathlineto{\pgfqpoint{6.597474in}{3.117714in}}%
\pgfpathlineto{\pgfqpoint{6.600150in}{3.119693in}}%
\pgfpathlineto{\pgfqpoint{6.605235in}{3.127226in}}%
\pgfpathlineto{\pgfqpoint{6.609784in}{3.132409in}}%
\pgfpathlineto{\pgfqpoint{6.612460in}{3.132677in}}%
\pgfpathlineto{\pgfqpoint{6.615136in}{3.130629in}}%
\pgfpathlineto{\pgfqpoint{6.620221in}{3.123061in}}%
\pgfpathlineto{\pgfqpoint{6.624770in}{3.117964in}}%
\pgfpathlineto{\pgfqpoint{6.627446in}{3.117773in}}%
\pgfpathlineto{\pgfqpoint{6.630390in}{3.120206in}}%
\pgfpathlineto{\pgfqpoint{6.641630in}{3.132832in}}%
\pgfpathlineto{\pgfqpoint{6.644306in}{3.131299in}}%
\pgfpathlineto{\pgfqpoint{6.648320in}{3.125778in}}%
\pgfpathlineto{\pgfqpoint{6.654207in}{3.118175in}}%
\pgfpathlineto{\pgfqpoint{6.656883in}{3.117657in}}%
\pgfpathlineto{\pgfqpoint{6.659559in}{3.119481in}}%
\pgfpathlineto{\pgfqpoint{6.663306in}{3.124778in}}%
\pgfpathlineto{\pgfqpoint{6.663306in}{3.124778in}}%
\pgfusepath{stroke}%
\end{pgfscope}%
\begin{pgfscope}%
\pgfsetrectcap%
\pgfsetmiterjoin%
\pgfsetlinewidth{0.803000pt}%
\definecolor{currentstroke}{rgb}{0.000000,0.000000,0.000000}%
\pgfsetstrokecolor{currentstroke}%
\pgfsetdash{}{0pt}%
\pgfpathmoveto{\pgfqpoint{0.467797in}{2.292089in}}%
\pgfpathlineto{\pgfqpoint{0.467797in}{3.958330in}}%
\pgfusepath{stroke}%
\end{pgfscope}%
\begin{pgfscope}%
\pgfsetrectcap%
\pgfsetmiterjoin%
\pgfsetlinewidth{0.803000pt}%
\definecolor{currentstroke}{rgb}{0.000000,0.000000,0.000000}%
\pgfsetstrokecolor{currentstroke}%
\pgfsetdash{}{0pt}%
\pgfpathmoveto{\pgfqpoint{6.958330in}{2.292089in}}%
\pgfpathlineto{\pgfqpoint{6.958330in}{3.958330in}}%
\pgfusepath{stroke}%
\end{pgfscope}%
\begin{pgfscope}%
\pgfsetrectcap%
\pgfsetmiterjoin%
\pgfsetlinewidth{0.803000pt}%
\definecolor{currentstroke}{rgb}{0.000000,0.000000,0.000000}%
\pgfsetstrokecolor{currentstroke}%
\pgfsetdash{}{0pt}%
\pgfpathmoveto{\pgfqpoint{0.467797in}{2.292089in}}%
\pgfpathlineto{\pgfqpoint{6.958330in}{2.292089in}}%
\pgfusepath{stroke}%
\end{pgfscope}%
\begin{pgfscope}%
\pgfsetrectcap%
\pgfsetmiterjoin%
\pgfsetlinewidth{0.803000pt}%
\definecolor{currentstroke}{rgb}{0.000000,0.000000,0.000000}%
\pgfsetstrokecolor{currentstroke}%
\pgfsetdash{}{0pt}%
\pgfpathmoveto{\pgfqpoint{0.467797in}{3.958330in}}%
\pgfpathlineto{\pgfqpoint{6.958330in}{3.958330in}}%
\pgfusepath{stroke}%
\end{pgfscope}%
\begin{pgfscope}%
\pgfsetbuttcap%
\pgfsetmiterjoin%
\definecolor{currentfill}{rgb}{1.000000,1.000000,1.000000}%
\pgfsetfillcolor{currentfill}%
\pgfsetlinewidth{0.000000pt}%
\definecolor{currentstroke}{rgb}{0.000000,0.000000,0.000000}%
\pgfsetstrokecolor{currentstroke}%
\pgfsetstrokeopacity{0.000000}%
\pgfsetdash{}{0pt}%
\pgfpathmoveto{\pgfqpoint{0.467797in}{0.273305in}}%
\pgfpathlineto{\pgfqpoint{6.958330in}{0.273305in}}%
\pgfpathlineto{\pgfqpoint{6.958330in}{1.939546in}}%
\pgfpathlineto{\pgfqpoint{0.467797in}{1.939546in}}%
\pgfpathlineto{\pgfqpoint{0.467797in}{0.273305in}}%
\pgfpathclose%
\pgfusepath{fill}%
\end{pgfscope}%
\begin{pgfscope}%
\pgfsetbuttcap%
\pgfsetroundjoin%
\definecolor{currentfill}{rgb}{0.000000,0.000000,0.000000}%
\pgfsetfillcolor{currentfill}%
\pgfsetlinewidth{0.803000pt}%
\definecolor{currentstroke}{rgb}{0.000000,0.000000,0.000000}%
\pgfsetstrokecolor{currentstroke}%
\pgfsetdash{}{0pt}%
\pgfsys@defobject{currentmarker}{\pgfqpoint{0.000000in}{-0.048611in}}{\pgfqpoint{0.000000in}{0.000000in}}{%
\pgfpathmoveto{\pgfqpoint{0.000000in}{0.000000in}}%
\pgfpathlineto{\pgfqpoint{0.000000in}{-0.048611in}}%
\pgfusepath{stroke,fill}%
}%
\begin{pgfscope}%
\pgfsys@transformshift{0.762821in}{0.273305in}%
\pgfsys@useobject{currentmarker}{}%
\end{pgfscope}%
\end{pgfscope}%
\begin{pgfscope}%
\definecolor{textcolor}{rgb}{0.000000,0.000000,0.000000}%
\pgfsetstrokecolor{textcolor}%
\pgfsetfillcolor{textcolor}%
\pgftext[x=0.762821in,y=0.176083in,,top]{\color{textcolor}\sffamily\fontsize{10.000000}{12.000000}\selectfont 0}%
\end{pgfscope}%
\begin{pgfscope}%
\pgfsetbuttcap%
\pgfsetroundjoin%
\definecolor{currentfill}{rgb}{0.000000,0.000000,0.000000}%
\pgfsetfillcolor{currentfill}%
\pgfsetlinewidth{0.803000pt}%
\definecolor{currentstroke}{rgb}{0.000000,0.000000,0.000000}%
\pgfsetstrokecolor{currentstroke}%
\pgfsetdash{}{0pt}%
\pgfsys@defobject{currentmarker}{\pgfqpoint{0.000000in}{-0.048611in}}{\pgfqpoint{0.000000in}{0.000000in}}{%
\pgfpathmoveto{\pgfqpoint{0.000000in}{0.000000in}}%
\pgfpathlineto{\pgfqpoint{0.000000in}{-0.048611in}}%
\pgfusepath{stroke,fill}%
}%
\begin{pgfscope}%
\pgfsys@transformshift{2.100860in}{0.273305in}%
\pgfsys@useobject{currentmarker}{}%
\end{pgfscope}%
\end{pgfscope}%
\begin{pgfscope}%
\definecolor{textcolor}{rgb}{0.000000,0.000000,0.000000}%
\pgfsetstrokecolor{textcolor}%
\pgfsetfillcolor{textcolor}%
\pgftext[x=2.100860in,y=0.176083in,,top]{\color{textcolor}\sffamily\fontsize{10.000000}{12.000000}\selectfont 5000}%
\end{pgfscope}%
\begin{pgfscope}%
\pgfsetbuttcap%
\pgfsetroundjoin%
\definecolor{currentfill}{rgb}{0.000000,0.000000,0.000000}%
\pgfsetfillcolor{currentfill}%
\pgfsetlinewidth{0.803000pt}%
\definecolor{currentstroke}{rgb}{0.000000,0.000000,0.000000}%
\pgfsetstrokecolor{currentstroke}%
\pgfsetdash{}{0pt}%
\pgfsys@defobject{currentmarker}{\pgfqpoint{0.000000in}{-0.048611in}}{\pgfqpoint{0.000000in}{0.000000in}}{%
\pgfpathmoveto{\pgfqpoint{0.000000in}{0.000000in}}%
\pgfpathlineto{\pgfqpoint{0.000000in}{-0.048611in}}%
\pgfusepath{stroke,fill}%
}%
\begin{pgfscope}%
\pgfsys@transformshift{3.438899in}{0.273305in}%
\pgfsys@useobject{currentmarker}{}%
\end{pgfscope}%
\end{pgfscope}%
\begin{pgfscope}%
\definecolor{textcolor}{rgb}{0.000000,0.000000,0.000000}%
\pgfsetstrokecolor{textcolor}%
\pgfsetfillcolor{textcolor}%
\pgftext[x=3.438899in,y=0.176083in,,top]{\color{textcolor}\sffamily\fontsize{10.000000}{12.000000}\selectfont 10000}%
\end{pgfscope}%
\begin{pgfscope}%
\pgfsetbuttcap%
\pgfsetroundjoin%
\definecolor{currentfill}{rgb}{0.000000,0.000000,0.000000}%
\pgfsetfillcolor{currentfill}%
\pgfsetlinewidth{0.803000pt}%
\definecolor{currentstroke}{rgb}{0.000000,0.000000,0.000000}%
\pgfsetstrokecolor{currentstroke}%
\pgfsetdash{}{0pt}%
\pgfsys@defobject{currentmarker}{\pgfqpoint{0.000000in}{-0.048611in}}{\pgfqpoint{0.000000in}{0.000000in}}{%
\pgfpathmoveto{\pgfqpoint{0.000000in}{0.000000in}}%
\pgfpathlineto{\pgfqpoint{0.000000in}{-0.048611in}}%
\pgfusepath{stroke,fill}%
}%
\begin{pgfscope}%
\pgfsys@transformshift{4.776938in}{0.273305in}%
\pgfsys@useobject{currentmarker}{}%
\end{pgfscope}%
\end{pgfscope}%
\begin{pgfscope}%
\definecolor{textcolor}{rgb}{0.000000,0.000000,0.000000}%
\pgfsetstrokecolor{textcolor}%
\pgfsetfillcolor{textcolor}%
\pgftext[x=4.776938in,y=0.176083in,,top]{\color{textcolor}\sffamily\fontsize{10.000000}{12.000000}\selectfont 15000}%
\end{pgfscope}%
\begin{pgfscope}%
\pgfsetbuttcap%
\pgfsetroundjoin%
\definecolor{currentfill}{rgb}{0.000000,0.000000,0.000000}%
\pgfsetfillcolor{currentfill}%
\pgfsetlinewidth{0.803000pt}%
\definecolor{currentstroke}{rgb}{0.000000,0.000000,0.000000}%
\pgfsetstrokecolor{currentstroke}%
\pgfsetdash{}{0pt}%
\pgfsys@defobject{currentmarker}{\pgfqpoint{0.000000in}{-0.048611in}}{\pgfqpoint{0.000000in}{0.000000in}}{%
\pgfpathmoveto{\pgfqpoint{0.000000in}{0.000000in}}%
\pgfpathlineto{\pgfqpoint{0.000000in}{-0.048611in}}%
\pgfusepath{stroke,fill}%
}%
\begin{pgfscope}%
\pgfsys@transformshift{6.114977in}{0.273305in}%
\pgfsys@useobject{currentmarker}{}%
\end{pgfscope}%
\end{pgfscope}%
\begin{pgfscope}%
\definecolor{textcolor}{rgb}{0.000000,0.000000,0.000000}%
\pgfsetstrokecolor{textcolor}%
\pgfsetfillcolor{textcolor}%
\pgftext[x=6.114977in,y=0.176083in,,top]{\color{textcolor}\sffamily\fontsize{10.000000}{12.000000}\selectfont 20000}%
\end{pgfscope}%
\begin{pgfscope}%
\pgfsetbuttcap%
\pgfsetroundjoin%
\definecolor{currentfill}{rgb}{0.000000,0.000000,0.000000}%
\pgfsetfillcolor{currentfill}%
\pgfsetlinewidth{0.803000pt}%
\definecolor{currentstroke}{rgb}{0.000000,0.000000,0.000000}%
\pgfsetstrokecolor{currentstroke}%
\pgfsetdash{}{0pt}%
\pgfsys@defobject{currentmarker}{\pgfqpoint{-0.048611in}{0.000000in}}{\pgfqpoint{-0.000000in}{0.000000in}}{%
\pgfpathmoveto{\pgfqpoint{-0.000000in}{0.000000in}}%
\pgfpathlineto{\pgfqpoint{-0.048611in}{0.000000in}}%
\pgfusepath{stroke,fill}%
}%
\begin{pgfscope}%
\pgfsys@transformshift{0.467797in}{0.288510in}%
\pgfsys@useobject{currentmarker}{}%
\end{pgfscope}%
\end{pgfscope}%
\begin{pgfscope}%
\definecolor{textcolor}{rgb}{0.000000,0.000000,0.000000}%
\pgfsetstrokecolor{textcolor}%
\pgfsetfillcolor{textcolor}%
\pgftext[x=0.041670in, y=0.235748in, left, base]{\color{textcolor}\sffamily\fontsize{10.000000}{12.000000}\selectfont \ensuremath{-}1.0}%
\end{pgfscope}%
\begin{pgfscope}%
\pgfsetbuttcap%
\pgfsetroundjoin%
\definecolor{currentfill}{rgb}{0.000000,0.000000,0.000000}%
\pgfsetfillcolor{currentfill}%
\pgfsetlinewidth{0.803000pt}%
\definecolor{currentstroke}{rgb}{0.000000,0.000000,0.000000}%
\pgfsetstrokecolor{currentstroke}%
\pgfsetdash{}{0pt}%
\pgfsys@defobject{currentmarker}{\pgfqpoint{-0.048611in}{0.000000in}}{\pgfqpoint{-0.000000in}{0.000000in}}{%
\pgfpathmoveto{\pgfqpoint{-0.000000in}{0.000000in}}%
\pgfpathlineto{\pgfqpoint{-0.048611in}{0.000000in}}%
\pgfusepath{stroke,fill}%
}%
\begin{pgfscope}%
\pgfsys@transformshift{0.467797in}{0.697468in}%
\pgfsys@useobject{currentmarker}{}%
\end{pgfscope}%
\end{pgfscope}%
\begin{pgfscope}%
\definecolor{textcolor}{rgb}{0.000000,0.000000,0.000000}%
\pgfsetstrokecolor{textcolor}%
\pgfsetfillcolor{textcolor}%
\pgftext[x=0.041670in, y=0.644706in, left, base]{\color{textcolor}\sffamily\fontsize{10.000000}{12.000000}\selectfont \ensuremath{-}0.5}%
\end{pgfscope}%
\begin{pgfscope}%
\pgfsetbuttcap%
\pgfsetroundjoin%
\definecolor{currentfill}{rgb}{0.000000,0.000000,0.000000}%
\pgfsetfillcolor{currentfill}%
\pgfsetlinewidth{0.803000pt}%
\definecolor{currentstroke}{rgb}{0.000000,0.000000,0.000000}%
\pgfsetstrokecolor{currentstroke}%
\pgfsetdash{}{0pt}%
\pgfsys@defobject{currentmarker}{\pgfqpoint{-0.048611in}{0.000000in}}{\pgfqpoint{-0.000000in}{0.000000in}}{%
\pgfpathmoveto{\pgfqpoint{-0.000000in}{0.000000in}}%
\pgfpathlineto{\pgfqpoint{-0.048611in}{0.000000in}}%
\pgfusepath{stroke,fill}%
}%
\begin{pgfscope}%
\pgfsys@transformshift{0.467797in}{1.106426in}%
\pgfsys@useobject{currentmarker}{}%
\end{pgfscope}%
\end{pgfscope}%
\begin{pgfscope}%
\definecolor{textcolor}{rgb}{0.000000,0.000000,0.000000}%
\pgfsetstrokecolor{textcolor}%
\pgfsetfillcolor{textcolor}%
\pgftext[x=0.149695in, y=1.053664in, left, base]{\color{textcolor}\sffamily\fontsize{10.000000}{12.000000}\selectfont 0.0}%
\end{pgfscope}%
\begin{pgfscope}%
\pgfsetbuttcap%
\pgfsetroundjoin%
\definecolor{currentfill}{rgb}{0.000000,0.000000,0.000000}%
\pgfsetfillcolor{currentfill}%
\pgfsetlinewidth{0.803000pt}%
\definecolor{currentstroke}{rgb}{0.000000,0.000000,0.000000}%
\pgfsetstrokecolor{currentstroke}%
\pgfsetdash{}{0pt}%
\pgfsys@defobject{currentmarker}{\pgfqpoint{-0.048611in}{0.000000in}}{\pgfqpoint{-0.000000in}{0.000000in}}{%
\pgfpathmoveto{\pgfqpoint{-0.000000in}{0.000000in}}%
\pgfpathlineto{\pgfqpoint{-0.048611in}{0.000000in}}%
\pgfusepath{stroke,fill}%
}%
\begin{pgfscope}%
\pgfsys@transformshift{0.467797in}{1.515383in}%
\pgfsys@useobject{currentmarker}{}%
\end{pgfscope}%
\end{pgfscope}%
\begin{pgfscope}%
\definecolor{textcolor}{rgb}{0.000000,0.000000,0.000000}%
\pgfsetstrokecolor{textcolor}%
\pgfsetfillcolor{textcolor}%
\pgftext[x=0.149695in, y=1.462622in, left, base]{\color{textcolor}\sffamily\fontsize{10.000000}{12.000000}\selectfont 0.5}%
\end{pgfscope}%
\begin{pgfscope}%
\pgfsetbuttcap%
\pgfsetroundjoin%
\definecolor{currentfill}{rgb}{0.000000,0.000000,0.000000}%
\pgfsetfillcolor{currentfill}%
\pgfsetlinewidth{0.803000pt}%
\definecolor{currentstroke}{rgb}{0.000000,0.000000,0.000000}%
\pgfsetstrokecolor{currentstroke}%
\pgfsetdash{}{0pt}%
\pgfsys@defobject{currentmarker}{\pgfqpoint{-0.048611in}{0.000000in}}{\pgfqpoint{-0.000000in}{0.000000in}}{%
\pgfpathmoveto{\pgfqpoint{-0.000000in}{0.000000in}}%
\pgfpathlineto{\pgfqpoint{-0.048611in}{0.000000in}}%
\pgfusepath{stroke,fill}%
}%
\begin{pgfscope}%
\pgfsys@transformshift{0.467797in}{1.924341in}%
\pgfsys@useobject{currentmarker}{}%
\end{pgfscope}%
\end{pgfscope}%
\begin{pgfscope}%
\definecolor{textcolor}{rgb}{0.000000,0.000000,0.000000}%
\pgfsetstrokecolor{textcolor}%
\pgfsetfillcolor{textcolor}%
\pgftext[x=0.149695in, y=1.871580in, left, base]{\color{textcolor}\sffamily\fontsize{10.000000}{12.000000}\selectfont 1.0}%
\end{pgfscope}%
\begin{pgfscope}%
\pgfpathrectangle{\pgfqpoint{0.467797in}{0.273305in}}{\pgfqpoint{6.490533in}{1.666241in}}%
\pgfusepath{clip}%
\pgfsetrectcap%
\pgfsetroundjoin%
\pgfsetlinewidth{1.505625pt}%
\definecolor{currentstroke}{rgb}{0.121569,0.466667,0.705882}%
\pgfsetstrokecolor{currentstroke}%
\pgfsetdash{}{0pt}%
\pgfpathmoveto{\pgfqpoint{0.762821in}{1.106426in}}%
\pgfpathlineto{\pgfqpoint{0.770582in}{1.687786in}}%
\pgfpathlineto{\pgfqpoint{0.774328in}{1.828736in}}%
\pgfpathlineto{\pgfqpoint{0.777004in}{1.862832in}}%
\pgfpathlineto{\pgfqpoint{0.777539in}{1.863808in}}%
\pgfpathlineto{\pgfqpoint{0.778075in}{1.863085in}}%
\pgfpathlineto{\pgfqpoint{0.779413in}{1.854693in}}%
\pgfpathlineto{\pgfqpoint{0.782089in}{1.817239in}}%
\pgfpathlineto{\pgfqpoint{0.790117in}{1.693786in}}%
\pgfpathlineto{\pgfqpoint{0.792258in}{1.686354in}}%
\pgfpathlineto{\pgfqpoint{0.793061in}{1.687132in}}%
\pgfpathlineto{\pgfqpoint{0.794666in}{1.693884in}}%
\pgfpathlineto{\pgfqpoint{0.797610in}{1.719605in}}%
\pgfpathlineto{\pgfqpoint{0.804835in}{1.786371in}}%
\pgfpathlineto{\pgfqpoint{0.806976in}{1.791387in}}%
\pgfpathlineto{\pgfqpoint{0.808047in}{1.790486in}}%
\pgfpathlineto{\pgfqpoint{0.809920in}{1.783984in}}%
\pgfpathlineto{\pgfqpoint{0.813399in}{1.760422in}}%
\pgfpathlineto{\pgfqpoint{0.819554in}{1.720423in}}%
\pgfpathlineto{\pgfqpoint{0.821695in}{1.716584in}}%
\pgfpathlineto{\pgfqpoint{0.823033in}{1.717624in}}%
\pgfpathlineto{\pgfqpoint{0.824906in}{1.723108in}}%
\pgfpathlineto{\pgfqpoint{0.828652in}{1.743414in}}%
\pgfpathlineto{\pgfqpoint{0.834272in}{1.771595in}}%
\pgfpathlineto{\pgfqpoint{0.836681in}{1.774750in}}%
\pgfpathlineto{\pgfqpoint{0.838019in}{1.773554in}}%
\pgfpathlineto{\pgfqpoint{0.840160in}{1.767744in}}%
\pgfpathlineto{\pgfqpoint{0.844709in}{1.746373in}}%
\pgfpathlineto{\pgfqpoint{0.849258in}{1.729236in}}%
\pgfpathlineto{\pgfqpoint{0.851399in}{1.727111in}}%
\pgfpathlineto{\pgfqpoint{0.853005in}{1.728462in}}%
\pgfpathlineto{\pgfqpoint{0.855413in}{1.734621in}}%
\pgfpathlineto{\pgfqpoint{0.865850in}{1.767464in}}%
\pgfpathlineto{\pgfqpoint{0.867456in}{1.766700in}}%
\pgfpathlineto{\pgfqpoint{0.869596in}{1.762590in}}%
\pgfpathlineto{\pgfqpoint{0.873878in}{1.747982in}}%
\pgfpathlineto{\pgfqpoint{0.878695in}{1.734121in}}%
\pgfpathlineto{\pgfqpoint{0.881104in}{1.732448in}}%
\pgfpathlineto{\pgfqpoint{0.882977in}{1.734081in}}%
\pgfpathlineto{\pgfqpoint{0.885653in}{1.740140in}}%
\pgfpathlineto{\pgfqpoint{0.894484in}{1.763032in}}%
\pgfpathlineto{\pgfqpoint{0.896357in}{1.763242in}}%
\pgfpathlineto{\pgfqpoint{0.898498in}{1.760749in}}%
\pgfpathlineto{\pgfqpoint{0.901977in}{1.752175in}}%
\pgfpathlineto{\pgfqpoint{0.908400in}{1.736730in}}%
\pgfpathlineto{\pgfqpoint{0.910808in}{1.735673in}}%
\pgfpathlineto{\pgfqpoint{0.912949in}{1.737576in}}%
\pgfpathlineto{\pgfqpoint{0.916160in}{1.744322in}}%
\pgfpathlineto{\pgfqpoint{0.923386in}{1.760075in}}%
\pgfpathlineto{\pgfqpoint{0.925794in}{1.760734in}}%
\pgfpathlineto{\pgfqpoint{0.927935in}{1.758746in}}%
\pgfpathlineto{\pgfqpoint{0.931414in}{1.751702in}}%
\pgfpathlineto{\pgfqpoint{0.937836in}{1.738728in}}%
\pgfpathlineto{\pgfqpoint{0.940245in}{1.737762in}}%
\pgfpathlineto{\pgfqpoint{0.942386in}{1.739296in}}%
\pgfpathlineto{\pgfqpoint{0.945597in}{1.744924in}}%
\pgfpathlineto{\pgfqpoint{0.952822in}{1.758371in}}%
\pgfpathlineto{\pgfqpoint{0.955231in}{1.759007in}}%
\pgfpathlineto{\pgfqpoint{0.957639in}{1.757032in}}%
\pgfpathlineto{\pgfqpoint{0.961386in}{1.750272in}}%
\pgfpathlineto{\pgfqpoint{0.967273in}{1.740147in}}%
\pgfpathlineto{\pgfqpoint{0.969949in}{1.739298in}}%
\pgfpathlineto{\pgfqpoint{0.972358in}{1.741110in}}%
\pgfpathlineto{\pgfqpoint{0.976104in}{1.747403in}}%
\pgfpathlineto{\pgfqpoint{0.982259in}{1.757129in}}%
\pgfpathlineto{\pgfqpoint{0.984935in}{1.757673in}}%
\pgfpathlineto{\pgfqpoint{0.987611in}{1.755434in}}%
\pgfpathlineto{\pgfqpoint{0.992428in}{1.747221in}}%
\pgfpathlineto{\pgfqpoint{0.996978in}{1.741002in}}%
\pgfpathlineto{\pgfqpoint{0.999654in}{1.740456in}}%
\pgfpathlineto{\pgfqpoint{1.002330in}{1.742538in}}%
\pgfpathlineto{\pgfqpoint{1.007147in}{1.750275in}}%
\pgfpathlineto{\pgfqpoint{1.011964in}{1.756357in}}%
\pgfpathlineto{\pgfqpoint{1.014640in}{1.756649in}}%
\pgfpathlineto{\pgfqpoint{1.017316in}{1.754481in}}%
\pgfpathlineto{\pgfqpoint{1.022668in}{1.746195in}}%
\pgfpathlineto{\pgfqpoint{1.026950in}{1.741527in}}%
\pgfpathlineto{\pgfqpoint{1.029626in}{1.741465in}}%
\pgfpathlineto{\pgfqpoint{1.032570in}{1.744035in}}%
\pgfpathlineto{\pgfqpoint{1.043274in}{1.756082in}}%
\pgfpathlineto{\pgfqpoint{1.045950in}{1.754801in}}%
\pgfpathlineto{\pgfqpoint{1.049964in}{1.749700in}}%
\pgfpathlineto{\pgfqpoint{1.056119in}{1.742334in}}%
\pgfpathlineto{\pgfqpoint{1.059063in}{1.742088in}}%
\pgfpathlineto{\pgfqpoint{1.062006in}{1.744383in}}%
\pgfpathlineto{\pgfqpoint{1.072978in}{1.755464in}}%
\pgfpathlineto{\pgfqpoint{1.075922in}{1.753902in}}%
\pgfpathlineto{\pgfqpoint{1.080739in}{1.747854in}}%
\pgfpathlineto{\pgfqpoint{1.085824in}{1.742744in}}%
\pgfpathlineto{\pgfqpoint{1.088767in}{1.742686in}}%
\pgfpathlineto{\pgfqpoint{1.091979in}{1.745238in}}%
\pgfpathlineto{\pgfqpoint{1.101880in}{1.754989in}}%
\pgfpathlineto{\pgfqpoint{1.104824in}{1.754000in}}%
\pgfpathlineto{\pgfqpoint{1.108838in}{1.749724in}}%
\pgfpathlineto{\pgfqpoint{1.115260in}{1.743190in}}%
\pgfpathlineto{\pgfqpoint{1.118204in}{1.743083in}}%
\pgfpathlineto{\pgfqpoint{1.121415in}{1.745417in}}%
\pgfpathlineto{\pgfqpoint{1.131584in}{1.754597in}}%
\pgfpathlineto{\pgfqpoint{1.134528in}{1.753526in}}%
\pgfpathlineto{\pgfqpoint{1.138810in}{1.749093in}}%
\pgfpathlineto{\pgfqpoint{1.144697in}{1.743559in}}%
\pgfpathlineto{\pgfqpoint{1.147909in}{1.743504in}}%
\pgfpathlineto{\pgfqpoint{1.151387in}{1.746097in}}%
\pgfpathlineto{\pgfqpoint{1.160486in}{1.754216in}}%
\pgfpathlineto{\pgfqpoint{1.163430in}{1.753622in}}%
\pgfpathlineto{\pgfqpoint{1.167444in}{1.750059in}}%
\pgfpathlineto{\pgfqpoint{1.174402in}{1.743768in}}%
\pgfpathlineto{\pgfqpoint{1.177613in}{1.743868in}}%
\pgfpathlineto{\pgfqpoint{1.181360in}{1.746717in}}%
\pgfpathlineto{\pgfqpoint{1.189655in}{1.753879in}}%
\pgfpathlineto{\pgfqpoint{1.192867in}{1.753413in}}%
\pgfpathlineto{\pgfqpoint{1.196881in}{1.750063in}}%
\pgfpathlineto{\pgfqpoint{1.204106in}{1.743945in}}%
\pgfpathlineto{\pgfqpoint{1.207318in}{1.744186in}}%
\pgfpathlineto{\pgfqpoint{1.211064in}{1.747018in}}%
\pgfpathlineto{\pgfqpoint{1.219092in}{1.753639in}}%
\pgfpathlineto{\pgfqpoint{1.222304in}{1.753239in}}%
\pgfpathlineto{\pgfqpoint{1.226318in}{1.750072in}}%
\pgfpathlineto{\pgfqpoint{1.233543in}{1.744168in}}%
\pgfpathlineto{\pgfqpoint{1.236754in}{1.744352in}}%
\pgfpathlineto{\pgfqpoint{1.240501in}{1.747031in}}%
\pgfpathlineto{\pgfqpoint{1.248797in}{1.753490in}}%
\pgfpathlineto{\pgfqpoint{1.252008in}{1.752972in}}%
\pgfpathlineto{\pgfqpoint{1.256290in}{1.749561in}}%
\pgfpathlineto{\pgfqpoint{1.262980in}{1.744357in}}%
\pgfpathlineto{\pgfqpoint{1.266191in}{1.744490in}}%
\pgfpathlineto{\pgfqpoint{1.269938in}{1.747037in}}%
\pgfpathlineto{\pgfqpoint{1.278234in}{1.753320in}}%
\pgfpathlineto{\pgfqpoint{1.281445in}{1.752862in}}%
\pgfpathlineto{\pgfqpoint{1.285727in}{1.749597in}}%
\pgfpathlineto{\pgfqpoint{1.292684in}{1.744444in}}%
\pgfpathlineto{\pgfqpoint{1.296163in}{1.744815in}}%
\pgfpathlineto{\pgfqpoint{1.300445in}{1.748015in}}%
\pgfpathlineto{\pgfqpoint{1.307403in}{1.753117in}}%
\pgfpathlineto{\pgfqpoint{1.310882in}{1.752773in}}%
\pgfpathlineto{\pgfqpoint{1.315163in}{1.749633in}}%
\pgfpathlineto{\pgfqpoint{1.322121in}{1.744578in}}%
\pgfpathlineto{\pgfqpoint{1.325600in}{1.744894in}}%
\pgfpathlineto{\pgfqpoint{1.329882in}{1.747978in}}%
\pgfpathlineto{\pgfqpoint{1.336840in}{1.752994in}}%
\pgfpathlineto{\pgfqpoint{1.340319in}{1.752704in}}%
\pgfpathlineto{\pgfqpoint{1.344600in}{1.749672in}}%
\pgfpathlineto{\pgfqpoint{1.351558in}{1.744689in}}%
\pgfpathlineto{\pgfqpoint{1.355037in}{1.744954in}}%
\pgfpathlineto{\pgfqpoint{1.359319in}{1.747939in}}%
\pgfpathlineto{\pgfqpoint{1.366544in}{1.752956in}}%
\pgfpathlineto{\pgfqpoint{1.370023in}{1.752541in}}%
\pgfpathlineto{\pgfqpoint{1.374572in}{1.749234in}}%
\pgfpathlineto{\pgfqpoint{1.380995in}{1.744781in}}%
\pgfpathlineto{\pgfqpoint{1.384474in}{1.744998in}}%
\pgfpathlineto{\pgfqpoint{1.388756in}{1.747898in}}%
\pgfpathlineto{\pgfqpoint{1.395981in}{1.752875in}}%
\pgfpathlineto{\pgfqpoint{1.399460in}{1.752509in}}%
\pgfpathlineto{\pgfqpoint{1.404009in}{1.749284in}}%
\pgfpathlineto{\pgfqpoint{1.410699in}{1.744788in}}%
\pgfpathlineto{\pgfqpoint{1.414178in}{1.745131in}}%
\pgfpathlineto{\pgfqpoint{1.418460in}{1.748087in}}%
\pgfpathlineto{\pgfqpoint{1.425418in}{1.752812in}}%
\pgfpathlineto{\pgfqpoint{1.428897in}{1.752493in}}%
\pgfpathlineto{\pgfqpoint{1.433179in}{1.749567in}}%
\pgfpathlineto{\pgfqpoint{1.440136in}{1.744844in}}%
\pgfpathlineto{\pgfqpoint{1.443615in}{1.745141in}}%
\pgfpathlineto{\pgfqpoint{1.447897in}{1.748038in}}%
\pgfpathlineto{\pgfqpoint{1.455122in}{1.752823in}}%
\pgfpathlineto{\pgfqpoint{1.458601in}{1.752380in}}%
\pgfpathlineto{\pgfqpoint{1.463151in}{1.749154in}}%
\pgfpathlineto{\pgfqpoint{1.469573in}{1.744883in}}%
\pgfpathlineto{\pgfqpoint{1.473052in}{1.745136in}}%
\pgfpathlineto{\pgfqpoint{1.477334in}{1.747988in}}%
\pgfpathlineto{\pgfqpoint{1.484559in}{1.752794in}}%
\pgfpathlineto{\pgfqpoint{1.488038in}{1.752394in}}%
\pgfpathlineto{\pgfqpoint{1.492587in}{1.749208in}}%
\pgfpathlineto{\pgfqpoint{1.499278in}{1.744845in}}%
\pgfpathlineto{\pgfqpoint{1.502757in}{1.745224in}}%
\pgfpathlineto{\pgfqpoint{1.507306in}{1.748394in}}%
\pgfpathlineto{\pgfqpoint{1.513996in}{1.752780in}}%
\pgfpathlineto{\pgfqpoint{1.517475in}{1.752422in}}%
\pgfpathlineto{\pgfqpoint{1.522024in}{1.749264in}}%
\pgfpathlineto{\pgfqpoint{1.528714in}{1.744852in}}%
\pgfpathlineto{\pgfqpoint{1.532193in}{1.745189in}}%
\pgfpathlineto{\pgfqpoint{1.536743in}{1.748337in}}%
\pgfpathlineto{\pgfqpoint{1.543433in}{1.752781in}}%
\pgfpathlineto{\pgfqpoint{1.546912in}{1.752463in}}%
\pgfpathlineto{\pgfqpoint{1.551194in}{1.749554in}}%
\pgfpathlineto{\pgfqpoint{1.558151in}{1.744844in}}%
\pgfpathlineto{\pgfqpoint{1.561630in}{1.745141in}}%
\pgfpathlineto{\pgfqpoint{1.565912in}{1.748046in}}%
\pgfpathlineto{\pgfqpoint{1.573137in}{1.752855in}}%
\pgfpathlineto{\pgfqpoint{1.576616in}{1.752408in}}%
\pgfpathlineto{\pgfqpoint{1.581166in}{1.749148in}}%
\pgfpathlineto{\pgfqpoint{1.587588in}{1.744820in}}%
\pgfpathlineto{\pgfqpoint{1.591067in}{1.745078in}}%
\pgfpathlineto{\pgfqpoint{1.595349in}{1.747982in}}%
\pgfpathlineto{\pgfqpoint{1.602574in}{1.752890in}}%
\pgfpathlineto{\pgfqpoint{1.606053in}{1.752480in}}%
\pgfpathlineto{\pgfqpoint{1.610602in}{1.749209in}}%
\pgfpathlineto{\pgfqpoint{1.617293in}{1.744717in}}%
\pgfpathlineto{\pgfqpoint{1.620772in}{1.745108in}}%
\pgfpathlineto{\pgfqpoint{1.625053in}{1.748150in}}%
\pgfpathlineto{\pgfqpoint{1.632011in}{1.752942in}}%
\pgfpathlineto{\pgfqpoint{1.635490in}{1.752569in}}%
\pgfpathlineto{\pgfqpoint{1.639772in}{1.749515in}}%
\pgfpathlineto{\pgfqpoint{1.646730in}{1.744656in}}%
\pgfpathlineto{\pgfqpoint{1.650208in}{1.745011in}}%
\pgfpathlineto{\pgfqpoint{1.654490in}{1.748079in}}%
\pgfpathlineto{\pgfqpoint{1.661448in}{1.753011in}}%
\pgfpathlineto{\pgfqpoint{1.664927in}{1.752675in}}%
\pgfpathlineto{\pgfqpoint{1.669209in}{1.749588in}}%
\pgfpathlineto{\pgfqpoint{1.676166in}{1.744577in}}%
\pgfpathlineto{\pgfqpoint{1.679645in}{1.744895in}}%
\pgfpathlineto{\pgfqpoint{1.683927in}{1.748003in}}%
\pgfpathlineto{\pgfqpoint{1.690885in}{1.753100in}}%
\pgfpathlineto{\pgfqpoint{1.694364in}{1.752801in}}%
\pgfpathlineto{\pgfqpoint{1.698645in}{1.749668in}}%
\pgfpathlineto{\pgfqpoint{1.705603in}{1.744478in}}%
\pgfpathlineto{\pgfqpoint{1.709082in}{1.744758in}}%
\pgfpathlineto{\pgfqpoint{1.713096in}{1.747667in}}%
\pgfpathlineto{\pgfqpoint{1.720589in}{1.753278in}}%
\pgfpathlineto{\pgfqpoint{1.723801in}{1.752949in}}%
\pgfpathlineto{\pgfqpoint{1.727815in}{1.750011in}}%
\pgfpathlineto{\pgfqpoint{1.735308in}{1.744285in}}%
\pgfpathlineto{\pgfqpoint{1.738519in}{1.744598in}}%
\pgfpathlineto{\pgfqpoint{1.742533in}{1.747568in}}%
\pgfpathlineto{\pgfqpoint{1.750026in}{1.753419in}}%
\pgfpathlineto{\pgfqpoint{1.753237in}{1.753123in}}%
\pgfpathlineto{\pgfqpoint{1.757252in}{1.750116in}}%
\pgfpathlineto{\pgfqpoint{1.764745in}{1.744131in}}%
\pgfpathlineto{\pgfqpoint{1.767956in}{1.744409in}}%
\pgfpathlineto{\pgfqpoint{1.771970in}{1.747456in}}%
\pgfpathlineto{\pgfqpoint{1.779463in}{1.753588in}}%
\pgfpathlineto{\pgfqpoint{1.782674in}{1.753328in}}%
\pgfpathlineto{\pgfqpoint{1.786421in}{1.750504in}}%
\pgfpathlineto{\pgfqpoint{1.794449in}{1.743876in}}%
\pgfpathlineto{\pgfqpoint{1.797660in}{1.744311in}}%
\pgfpathlineto{\pgfqpoint{1.801674in}{1.747610in}}%
\pgfpathlineto{\pgfqpoint{1.808900in}{1.753791in}}%
\pgfpathlineto{\pgfqpoint{1.812111in}{1.753568in}}%
\pgfpathlineto{\pgfqpoint{1.815858in}{1.750650in}}%
\pgfpathlineto{\pgfqpoint{1.823886in}{1.743646in}}%
\pgfpathlineto{\pgfqpoint{1.827097in}{1.744052in}}%
\pgfpathlineto{\pgfqpoint{1.830844in}{1.747187in}}%
\pgfpathlineto{\pgfqpoint{1.838604in}{1.754116in}}%
\pgfpathlineto{\pgfqpoint{1.841548in}{1.753852in}}%
\pgfpathlineto{\pgfqpoint{1.845027in}{1.751106in}}%
\pgfpathlineto{\pgfqpoint{1.853590in}{1.743302in}}%
\pgfpathlineto{\pgfqpoint{1.856534in}{1.743746in}}%
\pgfpathlineto{\pgfqpoint{1.860281in}{1.747021in}}%
\pgfpathlineto{\pgfqpoint{1.868041in}{1.754418in}}%
\pgfpathlineto{\pgfqpoint{1.870985in}{1.754189in}}%
\pgfpathlineto{\pgfqpoint{1.874464in}{1.751317in}}%
\pgfpathlineto{\pgfqpoint{1.883295in}{1.742903in}}%
\pgfpathlineto{\pgfqpoint{1.886239in}{1.743533in}}%
\pgfpathlineto{\pgfqpoint{1.889985in}{1.747156in}}%
\pgfpathlineto{\pgfqpoint{1.897478in}{1.754784in}}%
\pgfpathlineto{\pgfqpoint{1.900422in}{1.754592in}}%
\pgfpathlineto{\pgfqpoint{1.903633in}{1.751881in}}%
\pgfpathlineto{\pgfqpoint{1.912999in}{1.742434in}}%
\pgfpathlineto{\pgfqpoint{1.915675in}{1.743101in}}%
\pgfpathlineto{\pgfqpoint{1.919154in}{1.746599in}}%
\pgfpathlineto{\pgfqpoint{1.927183in}{1.755323in}}%
\pgfpathlineto{\pgfqpoint{1.929859in}{1.755081in}}%
\pgfpathlineto{\pgfqpoint{1.933070in}{1.752202in}}%
\pgfpathlineto{\pgfqpoint{1.942436in}{1.741906in}}%
\pgfpathlineto{\pgfqpoint{1.945112in}{1.742574in}}%
\pgfpathlineto{\pgfqpoint{1.948591in}{1.746325in}}%
\pgfpathlineto{\pgfqpoint{1.956619in}{1.755887in}}%
\pgfpathlineto{\pgfqpoint{1.959295in}{1.755682in}}%
\pgfpathlineto{\pgfqpoint{1.962239in}{1.752945in}}%
\pgfpathlineto{\pgfqpoint{1.972408in}{1.741188in}}%
\pgfpathlineto{\pgfqpoint{1.975084in}{1.742342in}}%
\pgfpathlineto{\pgfqpoint{1.978563in}{1.746827in}}%
\pgfpathlineto{\pgfqpoint{1.985789in}{1.756471in}}%
\pgfpathlineto{\pgfqpoint{1.988465in}{1.756565in}}%
\pgfpathlineto{\pgfqpoint{1.991141in}{1.754186in}}%
\pgfpathlineto{\pgfqpoint{1.996761in}{1.744996in}}%
\pgfpathlineto{\pgfqpoint{2.000775in}{1.740609in}}%
\pgfpathlineto{\pgfqpoint{2.003183in}{1.740611in}}%
\pgfpathlineto{\pgfqpoint{2.005859in}{1.743097in}}%
\pgfpathlineto{\pgfqpoint{2.011211in}{1.752358in}}%
\pgfpathlineto{\pgfqpoint{2.015493in}{1.757505in}}%
\pgfpathlineto{\pgfqpoint{2.017902in}{1.757536in}}%
\pgfpathlineto{\pgfqpoint{2.020578in}{1.754927in}}%
\pgfpathlineto{\pgfqpoint{2.025662in}{1.745592in}}%
\pgfpathlineto{\pgfqpoint{2.029944in}{1.739728in}}%
\pgfpathlineto{\pgfqpoint{2.032352in}{1.739384in}}%
\pgfpathlineto{\pgfqpoint{2.034761in}{1.741489in}}%
\pgfpathlineto{\pgfqpoint{2.038775in}{1.748857in}}%
\pgfpathlineto{\pgfqpoint{2.044395in}{1.758325in}}%
\pgfpathlineto{\pgfqpoint{2.046803in}{1.759025in}}%
\pgfpathlineto{\pgfqpoint{2.049212in}{1.757080in}}%
\pgfpathlineto{\pgfqpoint{2.052691in}{1.750633in}}%
\pgfpathlineto{\pgfqpoint{2.059381in}{1.738361in}}%
\pgfpathlineto{\pgfqpoint{2.061789in}{1.737883in}}%
\pgfpathlineto{\pgfqpoint{2.063930in}{1.739856in}}%
\pgfpathlineto{\pgfqpoint{2.067409in}{1.746793in}}%
\pgfpathlineto{\pgfqpoint{2.074099in}{1.760145in}}%
\pgfpathlineto{\pgfqpoint{2.076508in}{1.760706in}}%
\pgfpathlineto{\pgfqpoint{2.078649in}{1.758597in}}%
\pgfpathlineto{\pgfqpoint{2.082127in}{1.751074in}}%
\pgfpathlineto{\pgfqpoint{2.088818in}{1.736432in}}%
\pgfpathlineto{\pgfqpoint{2.090959in}{1.735673in}}%
\pgfpathlineto{\pgfqpoint{2.093099in}{1.737621in}}%
\pgfpathlineto{\pgfqpoint{2.096311in}{1.744765in}}%
\pgfpathlineto{\pgfqpoint{2.103804in}{1.762760in}}%
\pgfpathlineto{\pgfqpoint{2.105945in}{1.763274in}}%
\pgfpathlineto{\pgfqpoint{2.107818in}{1.761260in}}%
\pgfpathlineto{\pgfqpoint{2.110762in}{1.754180in}}%
\pgfpathlineto{\pgfqpoint{2.118790in}{1.732955in}}%
\pgfpathlineto{\pgfqpoint{2.120663in}{1.732588in}}%
\pgfpathlineto{\pgfqpoint{2.122536in}{1.734806in}}%
\pgfpathlineto{\pgfqpoint{2.125480in}{1.742731in}}%
\pgfpathlineto{\pgfqpoint{2.133508in}{1.766845in}}%
\pgfpathlineto{\pgfqpoint{2.135381in}{1.767315in}}%
\pgfpathlineto{\pgfqpoint{2.137255in}{1.764832in}}%
\pgfpathlineto{\pgfqpoint{2.140198in}{1.755809in}}%
\pgfpathlineto{\pgfqpoint{2.148494in}{1.727601in}}%
\pgfpathlineto{\pgfqpoint{2.150100in}{1.727288in}}%
\pgfpathlineto{\pgfqpoint{2.151706in}{1.729516in}}%
\pgfpathlineto{\pgfqpoint{2.154382in}{1.738288in}}%
\pgfpathlineto{\pgfqpoint{2.163748in}{1.774607in}}%
\pgfpathlineto{\pgfqpoint{2.165086in}{1.774339in}}%
\pgfpathlineto{\pgfqpoint{2.166692in}{1.771222in}}%
\pgfpathlineto{\pgfqpoint{2.169368in}{1.760039in}}%
\pgfpathlineto{\pgfqpoint{2.178466in}{1.716781in}}%
\pgfpathlineto{\pgfqpoint{2.179804in}{1.717051in}}%
\pgfpathlineto{\pgfqpoint{2.181410in}{1.720886in}}%
\pgfpathlineto{\pgfqpoint{2.184086in}{1.734939in}}%
\pgfpathlineto{\pgfqpoint{2.193452in}{1.791308in}}%
\pgfpathlineto{\pgfqpoint{2.194523in}{1.790808in}}%
\pgfpathlineto{\pgfqpoint{2.196128in}{1.785754in}}%
\pgfpathlineto{\pgfqpoint{2.198804in}{1.766734in}}%
\pgfpathlineto{\pgfqpoint{2.208438in}{1.686354in}}%
\pgfpathlineto{\pgfqpoint{2.209241in}{1.687159in}}%
\pgfpathlineto{\pgfqpoint{2.210579in}{1.692905in}}%
\pgfpathlineto{\pgfqpoint{2.212720in}{1.713445in}}%
\pgfpathlineto{\pgfqpoint{2.216467in}{1.775115in}}%
\pgfpathlineto{\pgfqpoint{2.221819in}{1.858117in}}%
\pgfpathlineto{\pgfqpoint{2.223157in}{1.863782in}}%
\pgfpathlineto{\pgfqpoint{2.223424in}{1.863728in}}%
\pgfpathlineto{\pgfqpoint{2.224227in}{1.860915in}}%
\pgfpathlineto{\pgfqpoint{2.225833in}{1.842163in}}%
\pgfpathlineto{\pgfqpoint{2.228241in}{1.777314in}}%
\pgfpathlineto{\pgfqpoint{2.231453in}{1.619098in}}%
\pgfpathlineto{\pgfqpoint{2.236537in}{1.233915in}}%
\pgfpathlineto{\pgfqpoint{2.245903in}{0.518107in}}%
\pgfpathlineto{\pgfqpoint{2.249650in}{0.381154in}}%
\pgfpathlineto{\pgfqpoint{2.252058in}{0.350542in}}%
\pgfpathlineto{\pgfqpoint{2.252861in}{0.349070in}}%
\pgfpathlineto{\pgfqpoint{2.253129in}{0.349433in}}%
\pgfpathlineto{\pgfqpoint{2.254199in}{0.354734in}}%
\pgfpathlineto{\pgfqpoint{2.256340in}{0.380174in}}%
\pgfpathlineto{\pgfqpoint{2.266509in}{0.524988in}}%
\pgfpathlineto{\pgfqpoint{2.267580in}{0.526498in}}%
\pgfpathlineto{\pgfqpoint{2.267847in}{0.526340in}}%
\pgfpathlineto{\pgfqpoint{2.268918in}{0.523699in}}%
\pgfpathlineto{\pgfqpoint{2.271059in}{0.510089in}}%
\pgfpathlineto{\pgfqpoint{2.276411in}{0.453879in}}%
\pgfpathlineto{\pgfqpoint{2.280425in}{0.424841in}}%
\pgfpathlineto{\pgfqpoint{2.282298in}{0.421455in}}%
\pgfpathlineto{\pgfqpoint{2.283369in}{0.422630in}}%
\pgfpathlineto{\pgfqpoint{2.285242in}{0.429543in}}%
\pgfpathlineto{\pgfqpoint{2.288721in}{0.453488in}}%
\pgfpathlineto{\pgfqpoint{2.294608in}{0.491965in}}%
\pgfpathlineto{\pgfqpoint{2.297017in}{0.496280in}}%
\pgfpathlineto{\pgfqpoint{2.298355in}{0.494984in}}%
\pgfpathlineto{\pgfqpoint{2.300495in}{0.488045in}}%
\pgfpathlineto{\pgfqpoint{2.304777in}{0.463651in}}%
\pgfpathlineto{\pgfqpoint{2.309594in}{0.440903in}}%
\pgfpathlineto{\pgfqpoint{2.311735in}{0.438096in}}%
\pgfpathlineto{\pgfqpoint{2.313073in}{0.439086in}}%
\pgfpathlineto{\pgfqpoint{2.315214in}{0.444624in}}%
\pgfpathlineto{\pgfqpoint{2.319496in}{0.464473in}}%
\pgfpathlineto{\pgfqpoint{2.324313in}{0.483335in}}%
\pgfpathlineto{\pgfqpoint{2.326721in}{0.485723in}}%
\pgfpathlineto{\pgfqpoint{2.328327in}{0.484169in}}%
\pgfpathlineto{\pgfqpoint{2.330735in}{0.477771in}}%
\pgfpathlineto{\pgfqpoint{2.340904in}{0.445422in}}%
\pgfpathlineto{\pgfqpoint{2.342510in}{0.446006in}}%
\pgfpathlineto{\pgfqpoint{2.344651in}{0.449914in}}%
\pgfpathlineto{\pgfqpoint{2.348665in}{0.463376in}}%
\pgfpathlineto{\pgfqpoint{2.353749in}{0.478514in}}%
\pgfpathlineto{\pgfqpoint{2.356158in}{0.480423in}}%
\pgfpathlineto{\pgfqpoint{2.358031in}{0.478967in}}%
\pgfpathlineto{\pgfqpoint{2.360707in}{0.473093in}}%
\pgfpathlineto{\pgfqpoint{2.369806in}{0.449726in}}%
\pgfpathlineto{\pgfqpoint{2.371679in}{0.449682in}}%
\pgfpathlineto{\pgfqpoint{2.373820in}{0.452344in}}%
\pgfpathlineto{\pgfqpoint{2.377567in}{0.461854in}}%
\pgfpathlineto{\pgfqpoint{2.383454in}{0.475963in}}%
\pgfpathlineto{\pgfqpoint{2.385862in}{0.477211in}}%
\pgfpathlineto{\pgfqpoint{2.388003in}{0.475467in}}%
\pgfpathlineto{\pgfqpoint{2.391215in}{0.468871in}}%
\pgfpathlineto{\pgfqpoint{2.398708in}{0.452657in}}%
\pgfpathlineto{\pgfqpoint{2.401116in}{0.452173in}}%
\pgfpathlineto{\pgfqpoint{2.403257in}{0.454302in}}%
\pgfpathlineto{\pgfqpoint{2.407003in}{0.462126in}}%
\pgfpathlineto{\pgfqpoint{2.412891in}{0.473987in}}%
\pgfpathlineto{\pgfqpoint{2.415299in}{0.475112in}}%
\pgfpathlineto{\pgfqpoint{2.417440in}{0.473713in}}%
\pgfpathlineto{\pgfqpoint{2.420651in}{0.468215in}}%
\pgfpathlineto{\pgfqpoint{2.428144in}{0.454375in}}%
\pgfpathlineto{\pgfqpoint{2.430553in}{0.453888in}}%
\pgfpathlineto{\pgfqpoint{2.432961in}{0.455997in}}%
\pgfpathlineto{\pgfqpoint{2.436975in}{0.463427in}}%
\pgfpathlineto{\pgfqpoint{2.442595in}{0.472818in}}%
\pgfpathlineto{\pgfqpoint{2.445271in}{0.473514in}}%
\pgfpathlineto{\pgfqpoint{2.447680in}{0.471576in}}%
\pgfpathlineto{\pgfqpoint{2.451694in}{0.464656in}}%
\pgfpathlineto{\pgfqpoint{2.457314in}{0.455824in}}%
\pgfpathlineto{\pgfqpoint{2.459990in}{0.455136in}}%
\pgfpathlineto{\pgfqpoint{2.462398in}{0.456925in}}%
\pgfpathlineto{\pgfqpoint{2.466412in}{0.463404in}}%
\pgfpathlineto{\pgfqpoint{2.472032in}{0.471751in}}%
\pgfpathlineto{\pgfqpoint{2.474708in}{0.472433in}}%
\pgfpathlineto{\pgfqpoint{2.477384in}{0.470472in}}%
\pgfpathlineto{\pgfqpoint{2.481934in}{0.463272in}}%
\pgfpathlineto{\pgfqpoint{2.487018in}{0.456576in}}%
\pgfpathlineto{\pgfqpoint{2.489694in}{0.456155in}}%
\pgfpathlineto{\pgfqpoint{2.492370in}{0.458212in}}%
\pgfpathlineto{\pgfqpoint{2.497455in}{0.466024in}}%
\pgfpathlineto{\pgfqpoint{2.502004in}{0.471256in}}%
\pgfpathlineto{\pgfqpoint{2.504680in}{0.471442in}}%
\pgfpathlineto{\pgfqpoint{2.507624in}{0.468982in}}%
\pgfpathlineto{\pgfqpoint{2.518596in}{0.456779in}}%
\pgfpathlineto{\pgfqpoint{2.521272in}{0.458170in}}%
\pgfpathlineto{\pgfqpoint{2.525286in}{0.463353in}}%
\pgfpathlineto{\pgfqpoint{2.531173in}{0.470442in}}%
\pgfpathlineto{\pgfqpoint{2.534117in}{0.470811in}}%
\pgfpathlineto{\pgfqpoint{2.537061in}{0.468618in}}%
\pgfpathlineto{\pgfqpoint{2.548300in}{0.457405in}}%
\pgfpathlineto{\pgfqpoint{2.551244in}{0.459073in}}%
\pgfpathlineto{\pgfqpoint{2.556329in}{0.465544in}}%
\pgfpathlineto{\pgfqpoint{2.561145in}{0.470164in}}%
\pgfpathlineto{\pgfqpoint{2.564089in}{0.470109in}}%
\pgfpathlineto{\pgfqpoint{2.567300in}{0.467461in}}%
\pgfpathlineto{\pgfqpoint{2.576934in}{0.457873in}}%
\pgfpathlineto{\pgfqpoint{2.579878in}{0.458758in}}%
\pgfpathlineto{\pgfqpoint{2.583892in}{0.462956in}}%
\pgfpathlineto{\pgfqpoint{2.590315in}{0.469602in}}%
\pgfpathlineto{\pgfqpoint{2.593258in}{0.469813in}}%
\pgfpathlineto{\pgfqpoint{2.596470in}{0.467572in}}%
\pgfpathlineto{\pgfqpoint{2.606906in}{0.458256in}}%
\pgfpathlineto{\pgfqpoint{2.609850in}{0.459423in}}%
\pgfpathlineto{\pgfqpoint{2.614399in}{0.464244in}}%
\pgfpathlineto{\pgfqpoint{2.620019in}{0.469345in}}%
\pgfpathlineto{\pgfqpoint{2.622963in}{0.469396in}}%
\pgfpathlineto{\pgfqpoint{2.626442in}{0.466892in}}%
\pgfpathlineto{\pgfqpoint{2.635808in}{0.458617in}}%
\pgfpathlineto{\pgfqpoint{2.638752in}{0.459304in}}%
\pgfpathlineto{\pgfqpoint{2.642766in}{0.462942in}}%
\pgfpathlineto{\pgfqpoint{2.649456in}{0.469035in}}%
\pgfpathlineto{\pgfqpoint{2.652667in}{0.469036in}}%
\pgfpathlineto{\pgfqpoint{2.656146in}{0.466540in}}%
\pgfpathlineto{\pgfqpoint{2.665245in}{0.458899in}}%
\pgfpathlineto{\pgfqpoint{2.668456in}{0.459657in}}%
\pgfpathlineto{\pgfqpoint{2.672738in}{0.463506in}}%
\pgfpathlineto{\pgfqpoint{2.679161in}{0.468865in}}%
\pgfpathlineto{\pgfqpoint{2.682372in}{0.468721in}}%
\pgfpathlineto{\pgfqpoint{2.686118in}{0.465966in}}%
\pgfpathlineto{\pgfqpoint{2.694414in}{0.459184in}}%
\pgfpathlineto{\pgfqpoint{2.697625in}{0.459677in}}%
\pgfpathlineto{\pgfqpoint{2.701640in}{0.462914in}}%
\pgfpathlineto{\pgfqpoint{2.708865in}{0.468720in}}%
\pgfpathlineto{\pgfqpoint{2.712076in}{0.468444in}}%
\pgfpathlineto{\pgfqpoint{2.716090in}{0.465431in}}%
\pgfpathlineto{\pgfqpoint{2.723583in}{0.459450in}}%
\pgfpathlineto{\pgfqpoint{2.726795in}{0.459698in}}%
\pgfpathlineto{\pgfqpoint{2.730809in}{0.462636in}}%
\pgfpathlineto{\pgfqpoint{2.738302in}{0.468532in}}%
\pgfpathlineto{\pgfqpoint{2.741513in}{0.468310in}}%
\pgfpathlineto{\pgfqpoint{2.745527in}{0.465441in}}%
\pgfpathlineto{\pgfqpoint{2.753288in}{0.459558in}}%
\pgfpathlineto{\pgfqpoint{2.756499in}{0.459929in}}%
\pgfpathlineto{\pgfqpoint{2.760513in}{0.462874in}}%
\pgfpathlineto{\pgfqpoint{2.768006in}{0.468438in}}%
\pgfpathlineto{\pgfqpoint{2.771485in}{0.467974in}}%
\pgfpathlineto{\pgfqpoint{2.775767in}{0.464709in}}%
\pgfpathlineto{\pgfqpoint{2.782457in}{0.459769in}}%
\pgfpathlineto{\pgfqpoint{2.785936in}{0.460021in}}%
\pgfpathlineto{\pgfqpoint{2.790218in}{0.463094in}}%
\pgfpathlineto{\pgfqpoint{2.797443in}{0.468305in}}%
\pgfpathlineto{\pgfqpoint{2.800922in}{0.467899in}}%
\pgfpathlineto{\pgfqpoint{2.805204in}{0.464751in}}%
\pgfpathlineto{\pgfqpoint{2.812162in}{0.459825in}}%
\pgfpathlineto{\pgfqpoint{2.815641in}{0.460204in}}%
\pgfpathlineto{\pgfqpoint{2.819922in}{0.463301in}}%
\pgfpathlineto{\pgfqpoint{2.826880in}{0.468195in}}%
\pgfpathlineto{\pgfqpoint{2.830359in}{0.467841in}}%
\pgfpathlineto{\pgfqpoint{2.834641in}{0.464792in}}%
\pgfpathlineto{\pgfqpoint{2.841598in}{0.459925in}}%
\pgfpathlineto{\pgfqpoint{2.845077in}{0.460253in}}%
\pgfpathlineto{\pgfqpoint{2.849359in}{0.463258in}}%
\pgfpathlineto{\pgfqpoint{2.856317in}{0.468104in}}%
\pgfpathlineto{\pgfqpoint{2.859796in}{0.467800in}}%
\pgfpathlineto{\pgfqpoint{2.864078in}{0.464836in}}%
\pgfpathlineto{\pgfqpoint{2.871035in}{0.460007in}}%
\pgfpathlineto{\pgfqpoint{2.874514in}{0.460287in}}%
\pgfpathlineto{\pgfqpoint{2.878796in}{0.463214in}}%
\pgfpathlineto{\pgfqpoint{2.886021in}{0.468091in}}%
\pgfpathlineto{\pgfqpoint{2.889500in}{0.467664in}}%
\pgfpathlineto{\pgfqpoint{2.894050in}{0.464413in}}%
\pgfpathlineto{\pgfqpoint{2.900472in}{0.460071in}}%
\pgfpathlineto{\pgfqpoint{2.903951in}{0.460306in}}%
\pgfpathlineto{\pgfqpoint{2.908233in}{0.463168in}}%
\pgfpathlineto{\pgfqpoint{2.915458in}{0.468037in}}%
\pgfpathlineto{\pgfqpoint{2.918937in}{0.467656in}}%
\pgfpathlineto{\pgfqpoint{2.923486in}{0.464465in}}%
\pgfpathlineto{\pgfqpoint{2.930177in}{0.460056in}}%
\pgfpathlineto{\pgfqpoint{2.933656in}{0.460415in}}%
\pgfpathlineto{\pgfqpoint{2.938205in}{0.463581in}}%
\pgfpathlineto{\pgfqpoint{2.944895in}{0.467999in}}%
\pgfpathlineto{\pgfqpoint{2.948374in}{0.467662in}}%
\pgfpathlineto{\pgfqpoint{2.952923in}{0.464518in}}%
\pgfpathlineto{\pgfqpoint{2.959614in}{0.460086in}}%
\pgfpathlineto{\pgfqpoint{2.963092in}{0.460402in}}%
\pgfpathlineto{\pgfqpoint{2.967642in}{0.463527in}}%
\pgfpathlineto{\pgfqpoint{2.974332in}{0.467976in}}%
\pgfpathlineto{\pgfqpoint{2.977811in}{0.467681in}}%
\pgfpathlineto{\pgfqpoint{2.982093in}{0.464803in}}%
\pgfpathlineto{\pgfqpoint{2.989318in}{0.460044in}}%
\pgfpathlineto{\pgfqpoint{2.992797in}{0.460486in}}%
\pgfpathlineto{\pgfqpoint{2.997346in}{0.463703in}}%
\pgfpathlineto{\pgfqpoint{3.003769in}{0.467968in}}%
\pgfpathlineto{\pgfqpoint{3.007248in}{0.467715in}}%
\pgfpathlineto{\pgfqpoint{3.011529in}{0.464860in}}%
\pgfpathlineto{\pgfqpoint{3.018755in}{0.460042in}}%
\pgfpathlineto{\pgfqpoint{3.022234in}{0.460443in}}%
\pgfpathlineto{\pgfqpoint{3.026783in}{0.463646in}}%
\pgfpathlineto{\pgfqpoint{3.033473in}{0.468037in}}%
\pgfpathlineto{\pgfqpoint{3.036952in}{0.467656in}}%
\pgfpathlineto{\pgfqpoint{3.041502in}{0.464456in}}%
\pgfpathlineto{\pgfqpoint{3.048192in}{0.460024in}}%
\pgfpathlineto{\pgfqpoint{3.051671in}{0.460386in}}%
\pgfpathlineto{\pgfqpoint{3.056220in}{0.463586in}}%
\pgfpathlineto{\pgfqpoint{3.062910in}{0.468063in}}%
\pgfpathlineto{\pgfqpoint{3.066389in}{0.467720in}}%
\pgfpathlineto{\pgfqpoint{3.070671in}{0.464752in}}%
\pgfpathlineto{\pgfqpoint{3.077629in}{0.459991in}}%
\pgfpathlineto{\pgfqpoint{3.081107in}{0.460315in}}%
\pgfpathlineto{\pgfqpoint{3.085389in}{0.463286in}}%
\pgfpathlineto{\pgfqpoint{3.092347in}{0.468104in}}%
\pgfpathlineto{\pgfqpoint{3.095826in}{0.467799in}}%
\pgfpathlineto{\pgfqpoint{3.100108in}{0.464820in}}%
\pgfpathlineto{\pgfqpoint{3.107065in}{0.459941in}}%
\pgfpathlineto{\pgfqpoint{3.110544in}{0.460227in}}%
\pgfpathlineto{\pgfqpoint{3.114826in}{0.463217in}}%
\pgfpathlineto{\pgfqpoint{3.122051in}{0.468224in}}%
\pgfpathlineto{\pgfqpoint{3.125530in}{0.467782in}}%
\pgfpathlineto{\pgfqpoint{3.130080in}{0.464406in}}%
\pgfpathlineto{\pgfqpoint{3.136502in}{0.459874in}}%
\pgfpathlineto{\pgfqpoint{3.139981in}{0.460121in}}%
\pgfpathlineto{\pgfqpoint{3.144263in}{0.463141in}}%
\pgfpathlineto{\pgfqpoint{3.151488in}{0.468306in}}%
\pgfpathlineto{\pgfqpoint{3.154967in}{0.467898in}}%
\pgfpathlineto{\pgfqpoint{3.159249in}{0.464724in}}%
\pgfpathlineto{\pgfqpoint{3.166207in}{0.459718in}}%
\pgfpathlineto{\pgfqpoint{3.169686in}{0.460109in}}%
\pgfpathlineto{\pgfqpoint{3.173967in}{0.463309in}}%
\pgfpathlineto{\pgfqpoint{3.180925in}{0.468407in}}%
\pgfpathlineto{\pgfqpoint{3.184404in}{0.468034in}}%
\pgfpathlineto{\pgfqpoint{3.188686in}{0.464804in}}%
\pgfpathlineto{\pgfqpoint{3.195644in}{0.459605in}}%
\pgfpathlineto{\pgfqpoint{3.198855in}{0.459847in}}%
\pgfpathlineto{\pgfqpoint{3.202869in}{0.462713in}}%
\pgfpathlineto{\pgfqpoint{3.210630in}{0.468596in}}%
\pgfpathlineto{\pgfqpoint{3.213841in}{0.468195in}}%
\pgfpathlineto{\pgfqpoint{3.217855in}{0.465154in}}%
\pgfpathlineto{\pgfqpoint{3.225080in}{0.459468in}}%
\pgfpathlineto{\pgfqpoint{3.228292in}{0.459673in}}%
\pgfpathlineto{\pgfqpoint{3.232038in}{0.462347in}}%
\pgfpathlineto{\pgfqpoint{3.240334in}{0.468808in}}%
\pgfpathlineto{\pgfqpoint{3.243545in}{0.468254in}}%
\pgfpathlineto{\pgfqpoint{3.247827in}{0.464716in}}%
\pgfpathlineto{\pgfqpoint{3.254517in}{0.459302in}}%
\pgfpathlineto{\pgfqpoint{3.257729in}{0.459468in}}%
\pgfpathlineto{\pgfqpoint{3.261475in}{0.462214in}}%
\pgfpathlineto{\pgfqpoint{3.269771in}{0.469004in}}%
\pgfpathlineto{\pgfqpoint{3.272982in}{0.468473in}}%
\pgfpathlineto{\pgfqpoint{3.276996in}{0.465099in}}%
\pgfpathlineto{\pgfqpoint{3.283954in}{0.459102in}}%
\pgfpathlineto{\pgfqpoint{3.287165in}{0.459227in}}%
\pgfpathlineto{\pgfqpoint{3.290912in}{0.462063in}}%
\pgfpathlineto{\pgfqpoint{3.299208in}{0.469237in}}%
\pgfpathlineto{\pgfqpoint{3.302419in}{0.468730in}}%
\pgfpathlineto{\pgfqpoint{3.306433in}{0.465223in}}%
\pgfpathlineto{\pgfqpoint{3.313659in}{0.458773in}}%
\pgfpathlineto{\pgfqpoint{3.316602in}{0.458942in}}%
\pgfpathlineto{\pgfqpoint{3.320081in}{0.461604in}}%
\pgfpathlineto{\pgfqpoint{3.328912in}{0.469576in}}%
\pgfpathlineto{\pgfqpoint{3.331856in}{0.469034in}}%
\pgfpathlineto{\pgfqpoint{3.335603in}{0.465676in}}%
\pgfpathlineto{\pgfqpoint{3.343096in}{0.458476in}}%
\pgfpathlineto{\pgfqpoint{3.346039in}{0.458604in}}%
\pgfpathlineto{\pgfqpoint{3.349518in}{0.461385in}}%
\pgfpathlineto{\pgfqpoint{3.358617in}{0.469970in}}%
\pgfpathlineto{\pgfqpoint{3.361560in}{0.469237in}}%
\pgfpathlineto{\pgfqpoint{3.365307in}{0.465529in}}%
\pgfpathlineto{\pgfqpoint{3.372532in}{0.458116in}}%
\pgfpathlineto{\pgfqpoint{3.375476in}{0.458198in}}%
\pgfpathlineto{\pgfqpoint{3.378687in}{0.460817in}}%
\pgfpathlineto{\pgfqpoint{3.388321in}{0.470433in}}%
\pgfpathlineto{\pgfqpoint{3.390997in}{0.469665in}}%
\pgfpathlineto{\pgfqpoint{3.394744in}{0.465723in}}%
\pgfpathlineto{\pgfqpoint{3.401969in}{0.457676in}}%
\pgfpathlineto{\pgfqpoint{3.404913in}{0.457707in}}%
\pgfpathlineto{\pgfqpoint{3.408124in}{0.460485in}}%
\pgfpathlineto{\pgfqpoint{3.418026in}{0.470988in}}%
\pgfpathlineto{\pgfqpoint{3.420702in}{0.469990in}}%
\pgfpathlineto{\pgfqpoint{3.424448in}{0.465564in}}%
\pgfpathlineto{\pgfqpoint{3.431406in}{0.457134in}}%
\pgfpathlineto{\pgfqpoint{3.434082in}{0.456990in}}%
\pgfpathlineto{\pgfqpoint{3.437026in}{0.459406in}}%
\pgfpathlineto{\pgfqpoint{3.448265in}{0.471601in}}%
\pgfpathlineto{\pgfqpoint{3.450941in}{0.469865in}}%
\pgfpathlineto{\pgfqpoint{3.455223in}{0.463602in}}%
\pgfpathlineto{\pgfqpoint{3.460575in}{0.456614in}}%
\pgfpathlineto{\pgfqpoint{3.463251in}{0.456135in}}%
\pgfpathlineto{\pgfqpoint{3.465928in}{0.458170in}}%
\pgfpathlineto{\pgfqpoint{3.470477in}{0.465383in}}%
\pgfpathlineto{\pgfqpoint{3.475561in}{0.472029in}}%
\pgfpathlineto{\pgfqpoint{3.478237in}{0.472298in}}%
\pgfpathlineto{\pgfqpoint{3.480914in}{0.469932in}}%
\pgfpathlineto{\pgfqpoint{3.485730in}{0.461679in}}%
\pgfpathlineto{\pgfqpoint{3.490280in}{0.455577in}}%
\pgfpathlineto{\pgfqpoint{3.492956in}{0.455255in}}%
\pgfpathlineto{\pgfqpoint{3.495632in}{0.457736in}}%
\pgfpathlineto{\pgfqpoint{3.500449in}{0.466493in}}%
\pgfpathlineto{\pgfqpoint{3.504998in}{0.473032in}}%
\pgfpathlineto{\pgfqpoint{3.507407in}{0.473513in}}%
\pgfpathlineto{\pgfqpoint{3.509815in}{0.471537in}}%
\pgfpathlineto{\pgfqpoint{3.513562in}{0.464840in}}%
\pgfpathlineto{\pgfqpoint{3.519717in}{0.454419in}}%
\pgfpathlineto{\pgfqpoint{3.522125in}{0.453866in}}%
\pgfpathlineto{\pgfqpoint{3.524534in}{0.455951in}}%
\pgfpathlineto{\pgfqpoint{3.528280in}{0.463117in}}%
\pgfpathlineto{\pgfqpoint{3.534435in}{0.474381in}}%
\pgfpathlineto{\pgfqpoint{3.536844in}{0.475017in}}%
\pgfpathlineto{\pgfqpoint{3.538984in}{0.473180in}}%
\pgfpathlineto{\pgfqpoint{3.542463in}{0.466378in}}%
\pgfpathlineto{\pgfqpoint{3.549154in}{0.452827in}}%
\pgfpathlineto{\pgfqpoint{3.551562in}{0.452093in}}%
\pgfpathlineto{\pgfqpoint{3.553703in}{0.454055in}}%
\pgfpathlineto{\pgfqpoint{3.556914in}{0.460740in}}%
\pgfpathlineto{\pgfqpoint{3.564140in}{0.476545in}}%
\pgfpathlineto{\pgfqpoint{3.566280in}{0.477135in}}%
\pgfpathlineto{\pgfqpoint{3.568421in}{0.475023in}}%
\pgfpathlineto{\pgfqpoint{3.571633in}{0.467714in}}%
\pgfpathlineto{\pgfqpoint{3.578858in}{0.450221in}}%
\pgfpathlineto{\pgfqpoint{3.580999in}{0.449521in}}%
\pgfpathlineto{\pgfqpoint{3.582872in}{0.451374in}}%
\pgfpathlineto{\pgfqpoint{3.585816in}{0.458272in}}%
\pgfpathlineto{\pgfqpoint{3.594112in}{0.480007in}}%
\pgfpathlineto{\pgfqpoint{3.595985in}{0.480189in}}%
\pgfpathlineto{\pgfqpoint{3.597858in}{0.477792in}}%
\pgfpathlineto{\pgfqpoint{3.600802in}{0.469665in}}%
\pgfpathlineto{\pgfqpoint{3.608830in}{0.445876in}}%
\pgfpathlineto{\pgfqpoint{3.610703in}{0.445616in}}%
\pgfpathlineto{\pgfqpoint{3.612577in}{0.448304in}}%
\pgfpathlineto{\pgfqpoint{3.615520in}{0.457562in}}%
\pgfpathlineto{\pgfqpoint{3.623549in}{0.485111in}}%
\pgfpathlineto{\pgfqpoint{3.625422in}{0.485474in}}%
\pgfpathlineto{\pgfqpoint{3.627295in}{0.482394in}}%
\pgfpathlineto{\pgfqpoint{3.630239in}{0.471606in}}%
\pgfpathlineto{\pgfqpoint{3.638267in}{0.438895in}}%
\pgfpathlineto{\pgfqpoint{3.639873in}{0.438208in}}%
\pgfpathlineto{\pgfqpoint{3.641478in}{0.440570in}}%
\pgfpathlineto{\pgfqpoint{3.643887in}{0.449425in}}%
\pgfpathlineto{\pgfqpoint{3.654323in}{0.496267in}}%
\pgfpathlineto{\pgfqpoint{3.655661in}{0.494664in}}%
\pgfpathlineto{\pgfqpoint{3.657802in}{0.486703in}}%
\pgfpathlineto{\pgfqpoint{3.661549in}{0.460940in}}%
\pgfpathlineto{\pgfqpoint{3.667169in}{0.424436in}}%
\pgfpathlineto{\pgfqpoint{3.669042in}{0.421464in}}%
\pgfpathlineto{\pgfqpoint{3.670112in}{0.422822in}}%
\pgfpathlineto{\pgfqpoint{3.671986in}{0.430692in}}%
\pgfpathlineto{\pgfqpoint{3.674929in}{0.455239in}}%
\pgfpathlineto{\pgfqpoint{3.682690in}{0.525055in}}%
\pgfpathlineto{\pgfqpoint{3.683760in}{0.526498in}}%
\pgfpathlineto{\pgfqpoint{3.684028in}{0.526338in}}%
\pgfpathlineto{\pgfqpoint{3.685098in}{0.523516in}}%
\pgfpathlineto{\pgfqpoint{3.686972in}{0.510047in}}%
\pgfpathlineto{\pgfqpoint{3.689915in}{0.469269in}}%
\pgfpathlineto{\pgfqpoint{3.697943in}{0.349766in}}%
\pgfpathlineto{\pgfqpoint{3.698479in}{0.349043in}}%
\pgfpathlineto{\pgfqpoint{3.699014in}{0.350019in}}%
\pgfpathlineto{\pgfqpoint{3.700352in}{0.360634in}}%
\pgfpathlineto{\pgfqpoint{3.702225in}{0.397364in}}%
\pgfpathlineto{\pgfqpoint{3.704901in}{0.498065in}}%
\pgfpathlineto{\pgfqpoint{3.708915in}{0.750319in}}%
\pgfpathlineto{\pgfqpoint{3.723901in}{1.808011in}}%
\pgfpathlineto{\pgfqpoint{3.726845in}{1.860041in}}%
\pgfpathlineto{\pgfqpoint{3.727916in}{1.863808in}}%
\pgfpathlineto{\pgfqpoint{3.728451in}{1.863085in}}%
\pgfpathlineto{\pgfqpoint{3.729789in}{1.854693in}}%
\pgfpathlineto{\pgfqpoint{3.732465in}{1.817239in}}%
\pgfpathlineto{\pgfqpoint{3.740493in}{1.693786in}}%
\pgfpathlineto{\pgfqpoint{3.742634in}{1.686354in}}%
\pgfpathlineto{\pgfqpoint{3.743437in}{1.687132in}}%
\pgfpathlineto{\pgfqpoint{3.745042in}{1.693884in}}%
\pgfpathlineto{\pgfqpoint{3.747986in}{1.719605in}}%
\pgfpathlineto{\pgfqpoint{3.755212in}{1.786371in}}%
\pgfpathlineto{\pgfqpoint{3.757352in}{1.791387in}}%
\pgfpathlineto{\pgfqpoint{3.758423in}{1.790486in}}%
\pgfpathlineto{\pgfqpoint{3.760296in}{1.783984in}}%
\pgfpathlineto{\pgfqpoint{3.763775in}{1.760422in}}%
\pgfpathlineto{\pgfqpoint{3.769930in}{1.720423in}}%
\pgfpathlineto{\pgfqpoint{3.772071in}{1.716584in}}%
\pgfpathlineto{\pgfqpoint{3.773409in}{1.717624in}}%
\pgfpathlineto{\pgfqpoint{3.775282in}{1.723108in}}%
\pgfpathlineto{\pgfqpoint{3.779029in}{1.743414in}}%
\pgfpathlineto{\pgfqpoint{3.784648in}{1.771595in}}%
\pgfpathlineto{\pgfqpoint{3.787057in}{1.774750in}}%
\pgfpathlineto{\pgfqpoint{3.788395in}{1.773554in}}%
\pgfpathlineto{\pgfqpoint{3.790536in}{1.767744in}}%
\pgfpathlineto{\pgfqpoint{3.795085in}{1.746373in}}%
\pgfpathlineto{\pgfqpoint{3.799634in}{1.729236in}}%
\pgfpathlineto{\pgfqpoint{3.801775in}{1.727111in}}%
\pgfpathlineto{\pgfqpoint{3.803381in}{1.728462in}}%
\pgfpathlineto{\pgfqpoint{3.805789in}{1.734621in}}%
\pgfpathlineto{\pgfqpoint{3.816226in}{1.767464in}}%
\pgfpathlineto{\pgfqpoint{3.817832in}{1.766700in}}%
\pgfpathlineto{\pgfqpoint{3.819973in}{1.762590in}}%
\pgfpathlineto{\pgfqpoint{3.824254in}{1.747982in}}%
\pgfpathlineto{\pgfqpoint{3.829071in}{1.734121in}}%
\pgfpathlineto{\pgfqpoint{3.831480in}{1.732448in}}%
\pgfpathlineto{\pgfqpoint{3.833353in}{1.734081in}}%
\pgfpathlineto{\pgfqpoint{3.836029in}{1.740140in}}%
\pgfpathlineto{\pgfqpoint{3.844860in}{1.763032in}}%
\pgfpathlineto{\pgfqpoint{3.846733in}{1.763242in}}%
\pgfpathlineto{\pgfqpoint{3.848874in}{1.760749in}}%
\pgfpathlineto{\pgfqpoint{3.852353in}{1.752175in}}%
\pgfpathlineto{\pgfqpoint{3.858776in}{1.736730in}}%
\pgfpathlineto{\pgfqpoint{3.861184in}{1.735673in}}%
\pgfpathlineto{\pgfqpoint{3.863325in}{1.737576in}}%
\pgfpathlineto{\pgfqpoint{3.866536in}{1.744322in}}%
\pgfpathlineto{\pgfqpoint{3.873762in}{1.760075in}}%
\pgfpathlineto{\pgfqpoint{3.876170in}{1.760734in}}%
\pgfpathlineto{\pgfqpoint{3.878311in}{1.758746in}}%
\pgfpathlineto{\pgfqpoint{3.881790in}{1.751702in}}%
\pgfpathlineto{\pgfqpoint{3.888213in}{1.738728in}}%
\pgfpathlineto{\pgfqpoint{3.890621in}{1.737762in}}%
\pgfpathlineto{\pgfqpoint{3.892762in}{1.739296in}}%
\pgfpathlineto{\pgfqpoint{3.895973in}{1.744924in}}%
\pgfpathlineto{\pgfqpoint{3.903199in}{1.758371in}}%
\pgfpathlineto{\pgfqpoint{3.905607in}{1.759007in}}%
\pgfpathlineto{\pgfqpoint{3.908016in}{1.757032in}}%
\pgfpathlineto{\pgfqpoint{3.911762in}{1.750272in}}%
\pgfpathlineto{\pgfqpoint{3.917650in}{1.740147in}}%
\pgfpathlineto{\pgfqpoint{3.920326in}{1.739298in}}%
\pgfpathlineto{\pgfqpoint{3.922734in}{1.741110in}}%
\pgfpathlineto{\pgfqpoint{3.926481in}{1.747403in}}%
\pgfpathlineto{\pgfqpoint{3.932636in}{1.757129in}}%
\pgfpathlineto{\pgfqpoint{3.935312in}{1.757673in}}%
\pgfpathlineto{\pgfqpoint{3.937988in}{1.755434in}}%
\pgfpathlineto{\pgfqpoint{3.942805in}{1.747221in}}%
\pgfpathlineto{\pgfqpoint{3.947354in}{1.741002in}}%
\pgfpathlineto{\pgfqpoint{3.950030in}{1.740456in}}%
\pgfpathlineto{\pgfqpoint{3.952706in}{1.742538in}}%
\pgfpathlineto{\pgfqpoint{3.957523in}{1.750275in}}%
\pgfpathlineto{\pgfqpoint{3.962340in}{1.756357in}}%
\pgfpathlineto{\pgfqpoint{3.965016in}{1.756649in}}%
\pgfpathlineto{\pgfqpoint{3.967692in}{1.754481in}}%
\pgfpathlineto{\pgfqpoint{3.973044in}{1.746195in}}%
\pgfpathlineto{\pgfqpoint{3.977326in}{1.741527in}}%
\pgfpathlineto{\pgfqpoint{3.980002in}{1.741465in}}%
\pgfpathlineto{\pgfqpoint{3.982946in}{1.744035in}}%
\pgfpathlineto{\pgfqpoint{3.993650in}{1.756082in}}%
\pgfpathlineto{\pgfqpoint{3.996326in}{1.754801in}}%
\pgfpathlineto{\pgfqpoint{4.000340in}{1.749700in}}%
\pgfpathlineto{\pgfqpoint{4.006495in}{1.742334in}}%
\pgfpathlineto{\pgfqpoint{4.009439in}{1.742088in}}%
\pgfpathlineto{\pgfqpoint{4.012383in}{1.744383in}}%
\pgfpathlineto{\pgfqpoint{4.023355in}{1.755464in}}%
\pgfpathlineto{\pgfqpoint{4.026298in}{1.753902in}}%
\pgfpathlineto{\pgfqpoint{4.031115in}{1.747854in}}%
\pgfpathlineto{\pgfqpoint{4.036200in}{1.742744in}}%
\pgfpathlineto{\pgfqpoint{4.039143in}{1.742686in}}%
\pgfpathlineto{\pgfqpoint{4.042355in}{1.745238in}}%
\pgfpathlineto{\pgfqpoint{4.052256in}{1.754989in}}%
\pgfpathlineto{\pgfqpoint{4.055200in}{1.754000in}}%
\pgfpathlineto{\pgfqpoint{4.059214in}{1.749724in}}%
\pgfpathlineto{\pgfqpoint{4.065637in}{1.743190in}}%
\pgfpathlineto{\pgfqpoint{4.068580in}{1.743083in}}%
\pgfpathlineto{\pgfqpoint{4.071792in}{1.745417in}}%
\pgfpathlineto{\pgfqpoint{4.081961in}{1.754597in}}%
\pgfpathlineto{\pgfqpoint{4.084904in}{1.753526in}}%
\pgfpathlineto{\pgfqpoint{4.089186in}{1.749093in}}%
\pgfpathlineto{\pgfqpoint{4.095073in}{1.743559in}}%
\pgfpathlineto{\pgfqpoint{4.098285in}{1.743504in}}%
\pgfpathlineto{\pgfqpoint{4.101764in}{1.746097in}}%
\pgfpathlineto{\pgfqpoint{4.110862in}{1.754216in}}%
\pgfpathlineto{\pgfqpoint{4.113806in}{1.753622in}}%
\pgfpathlineto{\pgfqpoint{4.117820in}{1.750059in}}%
\pgfpathlineto{\pgfqpoint{4.124778in}{1.743768in}}%
\pgfpathlineto{\pgfqpoint{4.127989in}{1.743868in}}%
\pgfpathlineto{\pgfqpoint{4.131736in}{1.746717in}}%
\pgfpathlineto{\pgfqpoint{4.140032in}{1.753879in}}%
\pgfpathlineto{\pgfqpoint{4.143243in}{1.753413in}}%
\pgfpathlineto{\pgfqpoint{4.147257in}{1.750063in}}%
\pgfpathlineto{\pgfqpoint{4.154482in}{1.743945in}}%
\pgfpathlineto{\pgfqpoint{4.157694in}{1.744186in}}%
\pgfpathlineto{\pgfqpoint{4.161440in}{1.747018in}}%
\pgfpathlineto{\pgfqpoint{4.169468in}{1.753639in}}%
\pgfpathlineto{\pgfqpoint{4.172680in}{1.753239in}}%
\pgfpathlineto{\pgfqpoint{4.176694in}{1.750072in}}%
\pgfpathlineto{\pgfqpoint{4.183919in}{1.744168in}}%
\pgfpathlineto{\pgfqpoint{4.187131in}{1.744352in}}%
\pgfpathlineto{\pgfqpoint{4.190877in}{1.747031in}}%
\pgfpathlineto{\pgfqpoint{4.199173in}{1.753490in}}%
\pgfpathlineto{\pgfqpoint{4.202384in}{1.752972in}}%
\pgfpathlineto{\pgfqpoint{4.206666in}{1.749561in}}%
\pgfpathlineto{\pgfqpoint{4.213356in}{1.744357in}}%
\pgfpathlineto{\pgfqpoint{4.216567in}{1.744490in}}%
\pgfpathlineto{\pgfqpoint{4.220314in}{1.747037in}}%
\pgfpathlineto{\pgfqpoint{4.228610in}{1.753320in}}%
\pgfpathlineto{\pgfqpoint{4.231821in}{1.752862in}}%
\pgfpathlineto{\pgfqpoint{4.236103in}{1.749597in}}%
\pgfpathlineto{\pgfqpoint{4.243061in}{1.744444in}}%
\pgfpathlineto{\pgfqpoint{4.246540in}{1.744815in}}%
\pgfpathlineto{\pgfqpoint{4.250821in}{1.748015in}}%
\pgfpathlineto{\pgfqpoint{4.257779in}{1.753117in}}%
\pgfpathlineto{\pgfqpoint{4.261258in}{1.752773in}}%
\pgfpathlineto{\pgfqpoint{4.265540in}{1.749633in}}%
\pgfpathlineto{\pgfqpoint{4.272497in}{1.744578in}}%
\pgfpathlineto{\pgfqpoint{4.275976in}{1.744894in}}%
\pgfpathlineto{\pgfqpoint{4.280258in}{1.747978in}}%
\pgfpathlineto{\pgfqpoint{4.287216in}{1.752994in}}%
\pgfpathlineto{\pgfqpoint{4.290695in}{1.752704in}}%
\pgfpathlineto{\pgfqpoint{4.294977in}{1.749672in}}%
\pgfpathlineto{\pgfqpoint{4.301934in}{1.744689in}}%
\pgfpathlineto{\pgfqpoint{4.305413in}{1.744954in}}%
\pgfpathlineto{\pgfqpoint{4.309695in}{1.747939in}}%
\pgfpathlineto{\pgfqpoint{4.316920in}{1.752956in}}%
\pgfpathlineto{\pgfqpoint{4.320399in}{1.752541in}}%
\pgfpathlineto{\pgfqpoint{4.324949in}{1.749234in}}%
\pgfpathlineto{\pgfqpoint{4.331371in}{1.744781in}}%
\pgfpathlineto{\pgfqpoint{4.334850in}{1.744998in}}%
\pgfpathlineto{\pgfqpoint{4.339132in}{1.747898in}}%
\pgfpathlineto{\pgfqpoint{4.346357in}{1.752875in}}%
\pgfpathlineto{\pgfqpoint{4.349836in}{1.752509in}}%
\pgfpathlineto{\pgfqpoint{4.354385in}{1.749284in}}%
\pgfpathlineto{\pgfqpoint{4.361076in}{1.744788in}}%
\pgfpathlineto{\pgfqpoint{4.364555in}{1.745131in}}%
\pgfpathlineto{\pgfqpoint{4.368836in}{1.748087in}}%
\pgfpathlineto{\pgfqpoint{4.375794in}{1.752812in}}%
\pgfpathlineto{\pgfqpoint{4.379273in}{1.752493in}}%
\pgfpathlineto{\pgfqpoint{4.383555in}{1.749567in}}%
\pgfpathlineto{\pgfqpoint{4.390513in}{1.744844in}}%
\pgfpathlineto{\pgfqpoint{4.393991in}{1.745141in}}%
\pgfpathlineto{\pgfqpoint{4.398273in}{1.748038in}}%
\pgfpathlineto{\pgfqpoint{4.405499in}{1.752823in}}%
\pgfpathlineto{\pgfqpoint{4.408977in}{1.752380in}}%
\pgfpathlineto{\pgfqpoint{4.413527in}{1.749154in}}%
\pgfpathlineto{\pgfqpoint{4.419949in}{1.744883in}}%
\pgfpathlineto{\pgfqpoint{4.423428in}{1.745136in}}%
\pgfpathlineto{\pgfqpoint{4.427710in}{1.747988in}}%
\pgfpathlineto{\pgfqpoint{4.434935in}{1.752794in}}%
\pgfpathlineto{\pgfqpoint{4.438414in}{1.752394in}}%
\pgfpathlineto{\pgfqpoint{4.442964in}{1.749208in}}%
\pgfpathlineto{\pgfqpoint{4.449654in}{1.744845in}}%
\pgfpathlineto{\pgfqpoint{4.453133in}{1.745224in}}%
\pgfpathlineto{\pgfqpoint{4.457682in}{1.748394in}}%
\pgfpathlineto{\pgfqpoint{4.464372in}{1.752780in}}%
\pgfpathlineto{\pgfqpoint{4.467851in}{1.752422in}}%
\pgfpathlineto{\pgfqpoint{4.472401in}{1.749264in}}%
\pgfpathlineto{\pgfqpoint{4.479091in}{1.744852in}}%
\pgfpathlineto{\pgfqpoint{4.482570in}{1.745189in}}%
\pgfpathlineto{\pgfqpoint{4.487119in}{1.748337in}}%
\pgfpathlineto{\pgfqpoint{4.493809in}{1.752781in}}%
\pgfpathlineto{\pgfqpoint{4.497288in}{1.752463in}}%
\pgfpathlineto{\pgfqpoint{4.501570in}{1.749554in}}%
\pgfpathlineto{\pgfqpoint{4.508528in}{1.744844in}}%
\pgfpathlineto{\pgfqpoint{4.512006in}{1.745141in}}%
\pgfpathlineto{\pgfqpoint{4.516288in}{1.748046in}}%
\pgfpathlineto{\pgfqpoint{4.523514in}{1.752855in}}%
\pgfpathlineto{\pgfqpoint{4.526993in}{1.752408in}}%
\pgfpathlineto{\pgfqpoint{4.531542in}{1.749148in}}%
\pgfpathlineto{\pgfqpoint{4.537964in}{1.744820in}}%
\pgfpathlineto{\pgfqpoint{4.541443in}{1.745078in}}%
\pgfpathlineto{\pgfqpoint{4.545725in}{1.747982in}}%
\pgfpathlineto{\pgfqpoint{4.552950in}{1.752890in}}%
\pgfpathlineto{\pgfqpoint{4.556429in}{1.752480in}}%
\pgfpathlineto{\pgfqpoint{4.560979in}{1.749209in}}%
\pgfpathlineto{\pgfqpoint{4.567669in}{1.744717in}}%
\pgfpathlineto{\pgfqpoint{4.571148in}{1.745108in}}%
\pgfpathlineto{\pgfqpoint{4.575430in}{1.748150in}}%
\pgfpathlineto{\pgfqpoint{4.582387in}{1.752942in}}%
\pgfpathlineto{\pgfqpoint{4.585866in}{1.752569in}}%
\pgfpathlineto{\pgfqpoint{4.590148in}{1.749515in}}%
\pgfpathlineto{\pgfqpoint{4.597106in}{1.744656in}}%
\pgfpathlineto{\pgfqpoint{4.600585in}{1.745011in}}%
\pgfpathlineto{\pgfqpoint{4.604866in}{1.748079in}}%
\pgfpathlineto{\pgfqpoint{4.611824in}{1.753011in}}%
\pgfpathlineto{\pgfqpoint{4.615303in}{1.752675in}}%
\pgfpathlineto{\pgfqpoint{4.619585in}{1.749588in}}%
\pgfpathlineto{\pgfqpoint{4.626543in}{1.744577in}}%
\pgfpathlineto{\pgfqpoint{4.630022in}{1.744895in}}%
\pgfpathlineto{\pgfqpoint{4.634303in}{1.748003in}}%
\pgfpathlineto{\pgfqpoint{4.641261in}{1.753100in}}%
\pgfpathlineto{\pgfqpoint{4.644740in}{1.752801in}}%
\pgfpathlineto{\pgfqpoint{4.649022in}{1.749668in}}%
\pgfpathlineto{\pgfqpoint{4.655979in}{1.744478in}}%
\pgfpathlineto{\pgfqpoint{4.659458in}{1.744758in}}%
\pgfpathlineto{\pgfqpoint{4.663473in}{1.747667in}}%
\pgfpathlineto{\pgfqpoint{4.670966in}{1.753278in}}%
\pgfpathlineto{\pgfqpoint{4.674177in}{1.752949in}}%
\pgfpathlineto{\pgfqpoint{4.678191in}{1.750011in}}%
\pgfpathlineto{\pgfqpoint{4.685684in}{1.744285in}}%
\pgfpathlineto{\pgfqpoint{4.688895in}{1.744598in}}%
\pgfpathlineto{\pgfqpoint{4.692909in}{1.747568in}}%
\pgfpathlineto{\pgfqpoint{4.700402in}{1.753419in}}%
\pgfpathlineto{\pgfqpoint{4.703614in}{1.753123in}}%
\pgfpathlineto{\pgfqpoint{4.707628in}{1.750116in}}%
\pgfpathlineto{\pgfqpoint{4.715121in}{1.744131in}}%
\pgfpathlineto{\pgfqpoint{4.718332in}{1.744409in}}%
\pgfpathlineto{\pgfqpoint{4.722346in}{1.747456in}}%
\pgfpathlineto{\pgfqpoint{4.729839in}{1.753588in}}%
\pgfpathlineto{\pgfqpoint{4.733051in}{1.753328in}}%
\pgfpathlineto{\pgfqpoint{4.736797in}{1.750504in}}%
\pgfpathlineto{\pgfqpoint{4.744825in}{1.743876in}}%
\pgfpathlineto{\pgfqpoint{4.748037in}{1.744311in}}%
\pgfpathlineto{\pgfqpoint{4.752051in}{1.747610in}}%
\pgfpathlineto{\pgfqpoint{4.759276in}{1.753791in}}%
\pgfpathlineto{\pgfqpoint{4.762487in}{1.753568in}}%
\pgfpathlineto{\pgfqpoint{4.766234in}{1.750650in}}%
\pgfpathlineto{\pgfqpoint{4.774262in}{1.743646in}}%
\pgfpathlineto{\pgfqpoint{4.777473in}{1.744052in}}%
\pgfpathlineto{\pgfqpoint{4.781220in}{1.747187in}}%
\pgfpathlineto{\pgfqpoint{4.788981in}{1.754116in}}%
\pgfpathlineto{\pgfqpoint{4.791924in}{1.753852in}}%
\pgfpathlineto{\pgfqpoint{4.795403in}{1.751106in}}%
\pgfpathlineto{\pgfqpoint{4.803967in}{1.743302in}}%
\pgfpathlineto{\pgfqpoint{4.806910in}{1.743746in}}%
\pgfpathlineto{\pgfqpoint{4.810657in}{1.747021in}}%
\pgfpathlineto{\pgfqpoint{4.818417in}{1.754418in}}%
\pgfpathlineto{\pgfqpoint{4.821361in}{1.754189in}}%
\pgfpathlineto{\pgfqpoint{4.824840in}{1.751317in}}%
\pgfpathlineto{\pgfqpoint{4.833671in}{1.742903in}}%
\pgfpathlineto{\pgfqpoint{4.836615in}{1.743533in}}%
\pgfpathlineto{\pgfqpoint{4.840361in}{1.747156in}}%
\pgfpathlineto{\pgfqpoint{4.847854in}{1.754784in}}%
\pgfpathlineto{\pgfqpoint{4.850798in}{1.754592in}}%
\pgfpathlineto{\pgfqpoint{4.854009in}{1.751881in}}%
\pgfpathlineto{\pgfqpoint{4.863376in}{1.742434in}}%
\pgfpathlineto{\pgfqpoint{4.866052in}{1.743101in}}%
\pgfpathlineto{\pgfqpoint{4.869531in}{1.746599in}}%
\pgfpathlineto{\pgfqpoint{4.877559in}{1.755323in}}%
\pgfpathlineto{\pgfqpoint{4.880235in}{1.755081in}}%
\pgfpathlineto{\pgfqpoint{4.883446in}{1.752202in}}%
\pgfpathlineto{\pgfqpoint{4.892812in}{1.741906in}}%
\pgfpathlineto{\pgfqpoint{4.895488in}{1.742574in}}%
\pgfpathlineto{\pgfqpoint{4.898967in}{1.746325in}}%
\pgfpathlineto{\pgfqpoint{4.906996in}{1.755887in}}%
\pgfpathlineto{\pgfqpoint{4.909672in}{1.755682in}}%
\pgfpathlineto{\pgfqpoint{4.912615in}{1.752945in}}%
\pgfpathlineto{\pgfqpoint{4.922784in}{1.741188in}}%
\pgfpathlineto{\pgfqpoint{4.925461in}{1.742342in}}%
\pgfpathlineto{\pgfqpoint{4.928939in}{1.746827in}}%
\pgfpathlineto{\pgfqpoint{4.936165in}{1.756471in}}%
\pgfpathlineto{\pgfqpoint{4.938841in}{1.756565in}}%
\pgfpathlineto{\pgfqpoint{4.941517in}{1.754186in}}%
\pgfpathlineto{\pgfqpoint{4.947137in}{1.744996in}}%
\pgfpathlineto{\pgfqpoint{4.951151in}{1.740609in}}%
\pgfpathlineto{\pgfqpoint{4.953559in}{1.740611in}}%
\pgfpathlineto{\pgfqpoint{4.956235in}{1.743097in}}%
\pgfpathlineto{\pgfqpoint{4.961588in}{1.752358in}}%
\pgfpathlineto{\pgfqpoint{4.965869in}{1.757505in}}%
\pgfpathlineto{\pgfqpoint{4.968278in}{1.757536in}}%
\pgfpathlineto{\pgfqpoint{4.970954in}{1.754927in}}%
\pgfpathlineto{\pgfqpoint{4.976038in}{1.745592in}}%
\pgfpathlineto{\pgfqpoint{4.980320in}{1.739728in}}%
\pgfpathlineto{\pgfqpoint{4.982729in}{1.739384in}}%
\pgfpathlineto{\pgfqpoint{4.985137in}{1.741489in}}%
\pgfpathlineto{\pgfqpoint{4.989151in}{1.748857in}}%
\pgfpathlineto{\pgfqpoint{4.994771in}{1.758325in}}%
\pgfpathlineto{\pgfqpoint{4.997179in}{1.759025in}}%
\pgfpathlineto{\pgfqpoint{4.999588in}{1.757080in}}%
\pgfpathlineto{\pgfqpoint{5.003067in}{1.750633in}}%
\pgfpathlineto{\pgfqpoint{5.009757in}{1.738361in}}%
\pgfpathlineto{\pgfqpoint{5.012166in}{1.737883in}}%
\pgfpathlineto{\pgfqpoint{5.014306in}{1.739856in}}%
\pgfpathlineto{\pgfqpoint{5.017785in}{1.746793in}}%
\pgfpathlineto{\pgfqpoint{5.024475in}{1.760145in}}%
\pgfpathlineto{\pgfqpoint{5.026884in}{1.760706in}}%
\pgfpathlineto{\pgfqpoint{5.029025in}{1.758597in}}%
\pgfpathlineto{\pgfqpoint{5.032504in}{1.751074in}}%
\pgfpathlineto{\pgfqpoint{5.039194in}{1.736432in}}%
\pgfpathlineto{\pgfqpoint{5.041335in}{1.735673in}}%
\pgfpathlineto{\pgfqpoint{5.043476in}{1.737621in}}%
\pgfpathlineto{\pgfqpoint{5.046687in}{1.744765in}}%
\pgfpathlineto{\pgfqpoint{5.054180in}{1.762760in}}%
\pgfpathlineto{\pgfqpoint{5.056321in}{1.763274in}}%
\pgfpathlineto{\pgfqpoint{5.058194in}{1.761260in}}%
\pgfpathlineto{\pgfqpoint{5.061138in}{1.754180in}}%
\pgfpathlineto{\pgfqpoint{5.069166in}{1.732955in}}%
\pgfpathlineto{\pgfqpoint{5.071039in}{1.732588in}}%
\pgfpathlineto{\pgfqpoint{5.072912in}{1.734806in}}%
\pgfpathlineto{\pgfqpoint{5.075856in}{1.742731in}}%
\pgfpathlineto{\pgfqpoint{5.083884in}{1.766845in}}%
\pgfpathlineto{\pgfqpoint{5.085758in}{1.767315in}}%
\pgfpathlineto{\pgfqpoint{5.087631in}{1.764832in}}%
\pgfpathlineto{\pgfqpoint{5.090575in}{1.755809in}}%
\pgfpathlineto{\pgfqpoint{5.098870in}{1.727601in}}%
\pgfpathlineto{\pgfqpoint{5.100476in}{1.727288in}}%
\pgfpathlineto{\pgfqpoint{5.102082in}{1.729516in}}%
\pgfpathlineto{\pgfqpoint{5.104758in}{1.738288in}}%
\pgfpathlineto{\pgfqpoint{5.114124in}{1.774607in}}%
\pgfpathlineto{\pgfqpoint{5.115462in}{1.774339in}}%
\pgfpathlineto{\pgfqpoint{5.117068in}{1.771222in}}%
\pgfpathlineto{\pgfqpoint{5.119744in}{1.760039in}}%
\pgfpathlineto{\pgfqpoint{5.128843in}{1.716781in}}%
\pgfpathlineto{\pgfqpoint{5.130181in}{1.717051in}}%
\pgfpathlineto{\pgfqpoint{5.131786in}{1.720886in}}%
\pgfpathlineto{\pgfqpoint{5.134462in}{1.734939in}}%
\pgfpathlineto{\pgfqpoint{5.143829in}{1.791308in}}%
\pgfpathlineto{\pgfqpoint{5.144899in}{1.790808in}}%
\pgfpathlineto{\pgfqpoint{5.146505in}{1.785754in}}%
\pgfpathlineto{\pgfqpoint{5.149181in}{1.766734in}}%
\pgfpathlineto{\pgfqpoint{5.158815in}{1.686354in}}%
\pgfpathlineto{\pgfqpoint{5.159617in}{1.687159in}}%
\pgfpathlineto{\pgfqpoint{5.160955in}{1.692905in}}%
\pgfpathlineto{\pgfqpoint{5.163096in}{1.713445in}}%
\pgfpathlineto{\pgfqpoint{5.166843in}{1.775115in}}%
\pgfpathlineto{\pgfqpoint{5.172195in}{1.858117in}}%
\pgfpathlineto{\pgfqpoint{5.173533in}{1.863782in}}%
\pgfpathlineto{\pgfqpoint{5.173801in}{1.863728in}}%
\pgfpathlineto{\pgfqpoint{5.174603in}{1.860915in}}%
\pgfpathlineto{\pgfqpoint{5.176209in}{1.842163in}}%
\pgfpathlineto{\pgfqpoint{5.178618in}{1.777314in}}%
\pgfpathlineto{\pgfqpoint{5.181829in}{1.619098in}}%
\pgfpathlineto{\pgfqpoint{5.186913in}{1.233915in}}%
\pgfpathlineto{\pgfqpoint{5.196280in}{0.518107in}}%
\pgfpathlineto{\pgfqpoint{5.200026in}{0.381154in}}%
\pgfpathlineto{\pgfqpoint{5.202435in}{0.350542in}}%
\pgfpathlineto{\pgfqpoint{5.203237in}{0.349070in}}%
\pgfpathlineto{\pgfqpoint{5.203505in}{0.349433in}}%
\pgfpathlineto{\pgfqpoint{5.204576in}{0.354734in}}%
\pgfpathlineto{\pgfqpoint{5.206716in}{0.380174in}}%
\pgfpathlineto{\pgfqpoint{5.216885in}{0.524988in}}%
\pgfpathlineto{\pgfqpoint{5.217956in}{0.526498in}}%
\pgfpathlineto{\pgfqpoint{5.218224in}{0.526340in}}%
\pgfpathlineto{\pgfqpoint{5.219294in}{0.523699in}}%
\pgfpathlineto{\pgfqpoint{5.221435in}{0.510089in}}%
\pgfpathlineto{\pgfqpoint{5.226787in}{0.453879in}}%
\pgfpathlineto{\pgfqpoint{5.230801in}{0.424841in}}%
\pgfpathlineto{\pgfqpoint{5.232674in}{0.421455in}}%
\pgfpathlineto{\pgfqpoint{5.233745in}{0.422630in}}%
\pgfpathlineto{\pgfqpoint{5.235618in}{0.429543in}}%
\pgfpathlineto{\pgfqpoint{5.239097in}{0.453488in}}%
\pgfpathlineto{\pgfqpoint{5.244984in}{0.491965in}}%
\pgfpathlineto{\pgfqpoint{5.247393in}{0.496280in}}%
\pgfpathlineto{\pgfqpoint{5.248731in}{0.494984in}}%
\pgfpathlineto{\pgfqpoint{5.250872in}{0.488045in}}%
\pgfpathlineto{\pgfqpoint{5.255153in}{0.463651in}}%
\pgfpathlineto{\pgfqpoint{5.259970in}{0.440903in}}%
\pgfpathlineto{\pgfqpoint{5.262111in}{0.438096in}}%
\pgfpathlineto{\pgfqpoint{5.263449in}{0.439086in}}%
\pgfpathlineto{\pgfqpoint{5.265590in}{0.444624in}}%
\pgfpathlineto{\pgfqpoint{5.269872in}{0.464473in}}%
\pgfpathlineto{\pgfqpoint{5.274689in}{0.483335in}}%
\pgfpathlineto{\pgfqpoint{5.277097in}{0.485723in}}%
\pgfpathlineto{\pgfqpoint{5.278703in}{0.484169in}}%
\pgfpathlineto{\pgfqpoint{5.281111in}{0.477771in}}%
\pgfpathlineto{\pgfqpoint{5.291280in}{0.445422in}}%
\pgfpathlineto{\pgfqpoint{5.292886in}{0.446006in}}%
\pgfpathlineto{\pgfqpoint{5.295027in}{0.449914in}}%
\pgfpathlineto{\pgfqpoint{5.299041in}{0.463376in}}%
\pgfpathlineto{\pgfqpoint{5.304126in}{0.478514in}}%
\pgfpathlineto{\pgfqpoint{5.306534in}{0.480423in}}%
\pgfpathlineto{\pgfqpoint{5.308407in}{0.478967in}}%
\pgfpathlineto{\pgfqpoint{5.311083in}{0.473093in}}%
\pgfpathlineto{\pgfqpoint{5.320182in}{0.449726in}}%
\pgfpathlineto{\pgfqpoint{5.322055in}{0.449682in}}%
\pgfpathlineto{\pgfqpoint{5.324196in}{0.452344in}}%
\pgfpathlineto{\pgfqpoint{5.327943in}{0.461854in}}%
\pgfpathlineto{\pgfqpoint{5.333830in}{0.475963in}}%
\pgfpathlineto{\pgfqpoint{5.336239in}{0.477211in}}%
\pgfpathlineto{\pgfqpoint{5.338379in}{0.475467in}}%
\pgfpathlineto{\pgfqpoint{5.341591in}{0.468871in}}%
\pgfpathlineto{\pgfqpoint{5.349084in}{0.452657in}}%
\pgfpathlineto{\pgfqpoint{5.351492in}{0.452173in}}%
\pgfpathlineto{\pgfqpoint{5.353633in}{0.454302in}}%
\pgfpathlineto{\pgfqpoint{5.357380in}{0.462126in}}%
\pgfpathlineto{\pgfqpoint{5.363267in}{0.473987in}}%
\pgfpathlineto{\pgfqpoint{5.365675in}{0.475112in}}%
\pgfpathlineto{\pgfqpoint{5.367816in}{0.473713in}}%
\pgfpathlineto{\pgfqpoint{5.371028in}{0.468215in}}%
\pgfpathlineto{\pgfqpoint{5.378521in}{0.454375in}}%
\pgfpathlineto{\pgfqpoint{5.380929in}{0.453888in}}%
\pgfpathlineto{\pgfqpoint{5.383338in}{0.455997in}}%
\pgfpathlineto{\pgfqpoint{5.387352in}{0.463427in}}%
\pgfpathlineto{\pgfqpoint{5.392971in}{0.472818in}}%
\pgfpathlineto{\pgfqpoint{5.395648in}{0.473514in}}%
\pgfpathlineto{\pgfqpoint{5.398056in}{0.471576in}}%
\pgfpathlineto{\pgfqpoint{5.402070in}{0.464656in}}%
\pgfpathlineto{\pgfqpoint{5.407690in}{0.455824in}}%
\pgfpathlineto{\pgfqpoint{5.410366in}{0.455136in}}%
\pgfpathlineto{\pgfqpoint{5.412774in}{0.456925in}}%
\pgfpathlineto{\pgfqpoint{5.416789in}{0.463404in}}%
\pgfpathlineto{\pgfqpoint{5.422408in}{0.471751in}}%
\pgfpathlineto{\pgfqpoint{5.425084in}{0.472433in}}%
\pgfpathlineto{\pgfqpoint{5.427760in}{0.470472in}}%
\pgfpathlineto{\pgfqpoint{5.432310in}{0.463272in}}%
\pgfpathlineto{\pgfqpoint{5.437394in}{0.456576in}}%
\pgfpathlineto{\pgfqpoint{5.440070in}{0.456155in}}%
\pgfpathlineto{\pgfqpoint{5.442746in}{0.458212in}}%
\pgfpathlineto{\pgfqpoint{5.447831in}{0.466024in}}%
\pgfpathlineto{\pgfqpoint{5.452380in}{0.471256in}}%
\pgfpathlineto{\pgfqpoint{5.455056in}{0.471442in}}%
\pgfpathlineto{\pgfqpoint{5.458000in}{0.468982in}}%
\pgfpathlineto{\pgfqpoint{5.468972in}{0.456779in}}%
\pgfpathlineto{\pgfqpoint{5.471648in}{0.458170in}}%
\pgfpathlineto{\pgfqpoint{5.475662in}{0.463353in}}%
\pgfpathlineto{\pgfqpoint{5.481550in}{0.470442in}}%
\pgfpathlineto{\pgfqpoint{5.484493in}{0.470811in}}%
\pgfpathlineto{\pgfqpoint{5.487437in}{0.468618in}}%
\pgfpathlineto{\pgfqpoint{5.498677in}{0.457405in}}%
\pgfpathlineto{\pgfqpoint{5.501620in}{0.459073in}}%
\pgfpathlineto{\pgfqpoint{5.506705in}{0.465544in}}%
\pgfpathlineto{\pgfqpoint{5.511522in}{0.470164in}}%
\pgfpathlineto{\pgfqpoint{5.514465in}{0.470109in}}%
\pgfpathlineto{\pgfqpoint{5.517677in}{0.467461in}}%
\pgfpathlineto{\pgfqpoint{5.527311in}{0.457873in}}%
\pgfpathlineto{\pgfqpoint{5.530254in}{0.458758in}}%
\pgfpathlineto{\pgfqpoint{5.534268in}{0.462956in}}%
\pgfpathlineto{\pgfqpoint{5.540691in}{0.469602in}}%
\pgfpathlineto{\pgfqpoint{5.543635in}{0.469813in}}%
\pgfpathlineto{\pgfqpoint{5.546846in}{0.467572in}}%
\pgfpathlineto{\pgfqpoint{5.557283in}{0.458256in}}%
\pgfpathlineto{\pgfqpoint{5.560226in}{0.459423in}}%
\pgfpathlineto{\pgfqpoint{5.564776in}{0.464244in}}%
\pgfpathlineto{\pgfqpoint{5.570395in}{0.469345in}}%
\pgfpathlineto{\pgfqpoint{5.573339in}{0.469396in}}%
\pgfpathlineto{\pgfqpoint{5.576818in}{0.466892in}}%
\pgfpathlineto{\pgfqpoint{5.586184in}{0.458617in}}%
\pgfpathlineto{\pgfqpoint{5.589128in}{0.459304in}}%
\pgfpathlineto{\pgfqpoint{5.593142in}{0.462942in}}%
\pgfpathlineto{\pgfqpoint{5.599832in}{0.469035in}}%
\pgfpathlineto{\pgfqpoint{5.603044in}{0.469036in}}%
\pgfpathlineto{\pgfqpoint{5.606522in}{0.466540in}}%
\pgfpathlineto{\pgfqpoint{5.615621in}{0.458899in}}%
\pgfpathlineto{\pgfqpoint{5.618832in}{0.459657in}}%
\pgfpathlineto{\pgfqpoint{5.623114in}{0.463506in}}%
\pgfpathlineto{\pgfqpoint{5.629537in}{0.468865in}}%
\pgfpathlineto{\pgfqpoint{5.632748in}{0.468721in}}%
\pgfpathlineto{\pgfqpoint{5.636495in}{0.465966in}}%
\pgfpathlineto{\pgfqpoint{5.644790in}{0.459184in}}%
\pgfpathlineto{\pgfqpoint{5.648002in}{0.459677in}}%
\pgfpathlineto{\pgfqpoint{5.652016in}{0.462914in}}%
\pgfpathlineto{\pgfqpoint{5.659241in}{0.468720in}}%
\pgfpathlineto{\pgfqpoint{5.662453in}{0.468444in}}%
\pgfpathlineto{\pgfqpoint{5.666467in}{0.465431in}}%
\pgfpathlineto{\pgfqpoint{5.673960in}{0.459450in}}%
\pgfpathlineto{\pgfqpoint{5.677171in}{0.459698in}}%
\pgfpathlineto{\pgfqpoint{5.681185in}{0.462636in}}%
\pgfpathlineto{\pgfqpoint{5.688678in}{0.468532in}}%
\pgfpathlineto{\pgfqpoint{5.691889in}{0.468310in}}%
\pgfpathlineto{\pgfqpoint{5.695903in}{0.465441in}}%
\pgfpathlineto{\pgfqpoint{5.703664in}{0.459558in}}%
\pgfpathlineto{\pgfqpoint{5.706875in}{0.459929in}}%
\pgfpathlineto{\pgfqpoint{5.710890in}{0.462874in}}%
\pgfpathlineto{\pgfqpoint{5.718383in}{0.468438in}}%
\pgfpathlineto{\pgfqpoint{5.721861in}{0.467974in}}%
\pgfpathlineto{\pgfqpoint{5.726143in}{0.464709in}}%
\pgfpathlineto{\pgfqpoint{5.732833in}{0.459769in}}%
\pgfpathlineto{\pgfqpoint{5.736312in}{0.460021in}}%
\pgfpathlineto{\pgfqpoint{5.740594in}{0.463094in}}%
\pgfpathlineto{\pgfqpoint{5.747819in}{0.468305in}}%
\pgfpathlineto{\pgfqpoint{5.751298in}{0.467899in}}%
\pgfpathlineto{\pgfqpoint{5.755580in}{0.464751in}}%
\pgfpathlineto{\pgfqpoint{5.762538in}{0.459825in}}%
\pgfpathlineto{\pgfqpoint{5.766017in}{0.460204in}}%
\pgfpathlineto{\pgfqpoint{5.770298in}{0.463301in}}%
\pgfpathlineto{\pgfqpoint{5.777256in}{0.468195in}}%
\pgfpathlineto{\pgfqpoint{5.780735in}{0.467841in}}%
\pgfpathlineto{\pgfqpoint{5.785017in}{0.464792in}}%
\pgfpathlineto{\pgfqpoint{5.791975in}{0.459925in}}%
\pgfpathlineto{\pgfqpoint{5.795454in}{0.460253in}}%
\pgfpathlineto{\pgfqpoint{5.799735in}{0.463258in}}%
\pgfpathlineto{\pgfqpoint{5.806693in}{0.468104in}}%
\pgfpathlineto{\pgfqpoint{5.810172in}{0.467800in}}%
\pgfpathlineto{\pgfqpoint{5.814454in}{0.464836in}}%
\pgfpathlineto{\pgfqpoint{5.821412in}{0.460007in}}%
\pgfpathlineto{\pgfqpoint{5.824890in}{0.460287in}}%
\pgfpathlineto{\pgfqpoint{5.829172in}{0.463214in}}%
\pgfpathlineto{\pgfqpoint{5.836398in}{0.468091in}}%
\pgfpathlineto{\pgfqpoint{5.839876in}{0.467664in}}%
\pgfpathlineto{\pgfqpoint{5.844426in}{0.464413in}}%
\pgfpathlineto{\pgfqpoint{5.850848in}{0.460071in}}%
\pgfpathlineto{\pgfqpoint{5.854327in}{0.460306in}}%
\pgfpathlineto{\pgfqpoint{5.858609in}{0.463168in}}%
\pgfpathlineto{\pgfqpoint{5.865834in}{0.468037in}}%
\pgfpathlineto{\pgfqpoint{5.869313in}{0.467656in}}%
\pgfpathlineto{\pgfqpoint{5.873863in}{0.464465in}}%
\pgfpathlineto{\pgfqpoint{5.880553in}{0.460056in}}%
\pgfpathlineto{\pgfqpoint{5.884032in}{0.460415in}}%
\pgfpathlineto{\pgfqpoint{5.888581in}{0.463581in}}%
\pgfpathlineto{\pgfqpoint{5.895271in}{0.467999in}}%
\pgfpathlineto{\pgfqpoint{5.898750in}{0.467662in}}%
\pgfpathlineto{\pgfqpoint{5.903300in}{0.464518in}}%
\pgfpathlineto{\pgfqpoint{5.909990in}{0.460086in}}%
\pgfpathlineto{\pgfqpoint{5.913469in}{0.460402in}}%
\pgfpathlineto{\pgfqpoint{5.918018in}{0.463527in}}%
\pgfpathlineto{\pgfqpoint{5.924708in}{0.467976in}}%
\pgfpathlineto{\pgfqpoint{5.928187in}{0.467681in}}%
\pgfpathlineto{\pgfqpoint{5.932469in}{0.464803in}}%
\pgfpathlineto{\pgfqpoint{5.939694in}{0.460044in}}%
\pgfpathlineto{\pgfqpoint{5.943173in}{0.460486in}}%
\pgfpathlineto{\pgfqpoint{5.947722in}{0.463703in}}%
\pgfpathlineto{\pgfqpoint{5.954145in}{0.467968in}}%
\pgfpathlineto{\pgfqpoint{5.957624in}{0.467715in}}%
\pgfpathlineto{\pgfqpoint{5.961906in}{0.464860in}}%
\pgfpathlineto{\pgfqpoint{5.969131in}{0.460042in}}%
\pgfpathlineto{\pgfqpoint{5.972610in}{0.460443in}}%
\pgfpathlineto{\pgfqpoint{5.977159in}{0.463646in}}%
\pgfpathlineto{\pgfqpoint{5.983849in}{0.468037in}}%
\pgfpathlineto{\pgfqpoint{5.987328in}{0.467656in}}%
\pgfpathlineto{\pgfqpoint{5.991878in}{0.464456in}}%
\pgfpathlineto{\pgfqpoint{5.998568in}{0.460024in}}%
\pgfpathlineto{\pgfqpoint{6.002047in}{0.460386in}}%
\pgfpathlineto{\pgfqpoint{6.006596in}{0.463586in}}%
\pgfpathlineto{\pgfqpoint{6.013286in}{0.468063in}}%
\pgfpathlineto{\pgfqpoint{6.016765in}{0.467720in}}%
\pgfpathlineto{\pgfqpoint{6.021047in}{0.464752in}}%
\pgfpathlineto{\pgfqpoint{6.028005in}{0.459991in}}%
\pgfpathlineto{\pgfqpoint{6.031484in}{0.460315in}}%
\pgfpathlineto{\pgfqpoint{6.035765in}{0.463286in}}%
\pgfpathlineto{\pgfqpoint{6.042723in}{0.468104in}}%
\pgfpathlineto{\pgfqpoint{6.046202in}{0.467799in}}%
\pgfpathlineto{\pgfqpoint{6.050484in}{0.464820in}}%
\pgfpathlineto{\pgfqpoint{6.057442in}{0.459941in}}%
\pgfpathlineto{\pgfqpoint{6.060921in}{0.460227in}}%
\pgfpathlineto{\pgfqpoint{6.065202in}{0.463217in}}%
\pgfpathlineto{\pgfqpoint{6.072428in}{0.468224in}}%
\pgfpathlineto{\pgfqpoint{6.075907in}{0.467782in}}%
\pgfpathlineto{\pgfqpoint{6.080456in}{0.464406in}}%
\pgfpathlineto{\pgfqpoint{6.086879in}{0.459874in}}%
\pgfpathlineto{\pgfqpoint{6.090357in}{0.460121in}}%
\pgfpathlineto{\pgfqpoint{6.094639in}{0.463141in}}%
\pgfpathlineto{\pgfqpoint{6.101865in}{0.468306in}}%
\pgfpathlineto{\pgfqpoint{6.105343in}{0.467898in}}%
\pgfpathlineto{\pgfqpoint{6.109625in}{0.464724in}}%
\pgfpathlineto{\pgfqpoint{6.116583in}{0.459718in}}%
\pgfpathlineto{\pgfqpoint{6.120062in}{0.460109in}}%
\pgfpathlineto{\pgfqpoint{6.124344in}{0.463309in}}%
\pgfpathlineto{\pgfqpoint{6.131301in}{0.468407in}}%
\pgfpathlineto{\pgfqpoint{6.134780in}{0.468034in}}%
\pgfpathlineto{\pgfqpoint{6.139062in}{0.464804in}}%
\pgfpathlineto{\pgfqpoint{6.146020in}{0.459605in}}%
\pgfpathlineto{\pgfqpoint{6.149231in}{0.459847in}}%
\pgfpathlineto{\pgfqpoint{6.153245in}{0.462713in}}%
\pgfpathlineto{\pgfqpoint{6.161006in}{0.468596in}}%
\pgfpathlineto{\pgfqpoint{6.164217in}{0.468195in}}%
\pgfpathlineto{\pgfqpoint{6.168231in}{0.465154in}}%
\pgfpathlineto{\pgfqpoint{6.175457in}{0.459468in}}%
\pgfpathlineto{\pgfqpoint{6.178668in}{0.459673in}}%
\pgfpathlineto{\pgfqpoint{6.182415in}{0.462347in}}%
\pgfpathlineto{\pgfqpoint{6.190710in}{0.468808in}}%
\pgfpathlineto{\pgfqpoint{6.193922in}{0.468254in}}%
\pgfpathlineto{\pgfqpoint{6.198203in}{0.464716in}}%
\pgfpathlineto{\pgfqpoint{6.204894in}{0.459302in}}%
\pgfpathlineto{\pgfqpoint{6.208105in}{0.459468in}}%
\pgfpathlineto{\pgfqpoint{6.211851in}{0.462214in}}%
\pgfpathlineto{\pgfqpoint{6.220147in}{0.469004in}}%
\pgfpathlineto{\pgfqpoint{6.223358in}{0.468473in}}%
\pgfpathlineto{\pgfqpoint{6.227373in}{0.465099in}}%
\pgfpathlineto{\pgfqpoint{6.234330in}{0.459102in}}%
\pgfpathlineto{\pgfqpoint{6.237542in}{0.459227in}}%
\pgfpathlineto{\pgfqpoint{6.241288in}{0.462063in}}%
\pgfpathlineto{\pgfqpoint{6.249584in}{0.469237in}}%
\pgfpathlineto{\pgfqpoint{6.252795in}{0.468730in}}%
\pgfpathlineto{\pgfqpoint{6.256809in}{0.465223in}}%
\pgfpathlineto{\pgfqpoint{6.264035in}{0.458773in}}%
\pgfpathlineto{\pgfqpoint{6.266979in}{0.458942in}}%
\pgfpathlineto{\pgfqpoint{6.270457in}{0.461604in}}%
\pgfpathlineto{\pgfqpoint{6.279289in}{0.469576in}}%
\pgfpathlineto{\pgfqpoint{6.282232in}{0.469034in}}%
\pgfpathlineto{\pgfqpoint{6.285979in}{0.465676in}}%
\pgfpathlineto{\pgfqpoint{6.293472in}{0.458476in}}%
\pgfpathlineto{\pgfqpoint{6.296415in}{0.458604in}}%
\pgfpathlineto{\pgfqpoint{6.299894in}{0.461385in}}%
\pgfpathlineto{\pgfqpoint{6.308993in}{0.469970in}}%
\pgfpathlineto{\pgfqpoint{6.311937in}{0.469237in}}%
\pgfpathlineto{\pgfqpoint{6.315683in}{0.465529in}}%
\pgfpathlineto{\pgfqpoint{6.322909in}{0.458116in}}%
\pgfpathlineto{\pgfqpoint{6.325852in}{0.458198in}}%
\pgfpathlineto{\pgfqpoint{6.329064in}{0.460817in}}%
\pgfpathlineto{\pgfqpoint{6.338697in}{0.470433in}}%
\pgfpathlineto{\pgfqpoint{6.341374in}{0.469665in}}%
\pgfpathlineto{\pgfqpoint{6.345120in}{0.465723in}}%
\pgfpathlineto{\pgfqpoint{6.352345in}{0.457676in}}%
\pgfpathlineto{\pgfqpoint{6.355289in}{0.457707in}}%
\pgfpathlineto{\pgfqpoint{6.358500in}{0.460485in}}%
\pgfpathlineto{\pgfqpoint{6.368402in}{0.470988in}}%
\pgfpathlineto{\pgfqpoint{6.371078in}{0.469990in}}%
\pgfpathlineto{\pgfqpoint{6.374825in}{0.465564in}}%
\pgfpathlineto{\pgfqpoint{6.381782in}{0.457134in}}%
\pgfpathlineto{\pgfqpoint{6.384458in}{0.456990in}}%
\pgfpathlineto{\pgfqpoint{6.387402in}{0.459406in}}%
\pgfpathlineto{\pgfqpoint{6.398642in}{0.471601in}}%
\pgfpathlineto{\pgfqpoint{6.401318in}{0.469865in}}%
\pgfpathlineto{\pgfqpoint{6.405599in}{0.463602in}}%
\pgfpathlineto{\pgfqpoint{6.410952in}{0.456614in}}%
\pgfpathlineto{\pgfqpoint{6.413628in}{0.456135in}}%
\pgfpathlineto{\pgfqpoint{6.416304in}{0.458170in}}%
\pgfpathlineto{\pgfqpoint{6.420853in}{0.465383in}}%
\pgfpathlineto{\pgfqpoint{6.425938in}{0.472029in}}%
\pgfpathlineto{\pgfqpoint{6.428614in}{0.472298in}}%
\pgfpathlineto{\pgfqpoint{6.431290in}{0.469932in}}%
\pgfpathlineto{\pgfqpoint{6.436107in}{0.461679in}}%
\pgfpathlineto{\pgfqpoint{6.440656in}{0.455577in}}%
\pgfpathlineto{\pgfqpoint{6.443332in}{0.455255in}}%
\pgfpathlineto{\pgfqpoint{6.446008in}{0.457736in}}%
\pgfpathlineto{\pgfqpoint{6.450825in}{0.466493in}}%
\pgfpathlineto{\pgfqpoint{6.455374in}{0.473032in}}%
\pgfpathlineto{\pgfqpoint{6.457783in}{0.473513in}}%
\pgfpathlineto{\pgfqpoint{6.460191in}{0.471537in}}%
\pgfpathlineto{\pgfqpoint{6.463938in}{0.464840in}}%
\pgfpathlineto{\pgfqpoint{6.470093in}{0.454419in}}%
\pgfpathlineto{\pgfqpoint{6.472501in}{0.453866in}}%
\pgfpathlineto{\pgfqpoint{6.474910in}{0.455951in}}%
\pgfpathlineto{\pgfqpoint{6.478656in}{0.463117in}}%
\pgfpathlineto{\pgfqpoint{6.484811in}{0.474381in}}%
\pgfpathlineto{\pgfqpoint{6.487220in}{0.475017in}}%
\pgfpathlineto{\pgfqpoint{6.489361in}{0.473180in}}%
\pgfpathlineto{\pgfqpoint{6.492840in}{0.466378in}}%
\pgfpathlineto{\pgfqpoint{6.499530in}{0.452827in}}%
\pgfpathlineto{\pgfqpoint{6.501938in}{0.452093in}}%
\pgfpathlineto{\pgfqpoint{6.504079in}{0.454055in}}%
\pgfpathlineto{\pgfqpoint{6.507290in}{0.460740in}}%
\pgfpathlineto{\pgfqpoint{6.514516in}{0.476545in}}%
\pgfpathlineto{\pgfqpoint{6.516657in}{0.477135in}}%
\pgfpathlineto{\pgfqpoint{6.518798in}{0.475023in}}%
\pgfpathlineto{\pgfqpoint{6.522009in}{0.467714in}}%
\pgfpathlineto{\pgfqpoint{6.529234in}{0.450221in}}%
\pgfpathlineto{\pgfqpoint{6.531375in}{0.449521in}}%
\pgfpathlineto{\pgfqpoint{6.533248in}{0.451374in}}%
\pgfpathlineto{\pgfqpoint{6.536192in}{0.458272in}}%
\pgfpathlineto{\pgfqpoint{6.544488in}{0.480007in}}%
\pgfpathlineto{\pgfqpoint{6.546361in}{0.480189in}}%
\pgfpathlineto{\pgfqpoint{6.548234in}{0.477792in}}%
\pgfpathlineto{\pgfqpoint{6.551178in}{0.469665in}}%
\pgfpathlineto{\pgfqpoint{6.559206in}{0.445876in}}%
\pgfpathlineto{\pgfqpoint{6.561080in}{0.445616in}}%
\pgfpathlineto{\pgfqpoint{6.562953in}{0.448304in}}%
\pgfpathlineto{\pgfqpoint{6.565897in}{0.457562in}}%
\pgfpathlineto{\pgfqpoint{6.573925in}{0.485111in}}%
\pgfpathlineto{\pgfqpoint{6.575798in}{0.485474in}}%
\pgfpathlineto{\pgfqpoint{6.577671in}{0.482394in}}%
\pgfpathlineto{\pgfqpoint{6.580615in}{0.471606in}}%
\pgfpathlineto{\pgfqpoint{6.588643in}{0.438895in}}%
\pgfpathlineto{\pgfqpoint{6.590249in}{0.438208in}}%
\pgfpathlineto{\pgfqpoint{6.591854in}{0.440570in}}%
\pgfpathlineto{\pgfqpoint{6.594263in}{0.449425in}}%
\pgfpathlineto{\pgfqpoint{6.604700in}{0.496267in}}%
\pgfpathlineto{\pgfqpoint{6.606038in}{0.494664in}}%
\pgfpathlineto{\pgfqpoint{6.608179in}{0.486703in}}%
\pgfpathlineto{\pgfqpoint{6.611925in}{0.460940in}}%
\pgfpathlineto{\pgfqpoint{6.617545in}{0.424436in}}%
\pgfpathlineto{\pgfqpoint{6.619418in}{0.421464in}}%
\pgfpathlineto{\pgfqpoint{6.620489in}{0.422822in}}%
\pgfpathlineto{\pgfqpoint{6.622362in}{0.430692in}}%
\pgfpathlineto{\pgfqpoint{6.625305in}{0.455239in}}%
\pgfpathlineto{\pgfqpoint{6.633066in}{0.525055in}}%
\pgfpathlineto{\pgfqpoint{6.634137in}{0.526498in}}%
\pgfpathlineto{\pgfqpoint{6.634404in}{0.526338in}}%
\pgfpathlineto{\pgfqpoint{6.635475in}{0.523516in}}%
\pgfpathlineto{\pgfqpoint{6.637348in}{0.510047in}}%
\pgfpathlineto{\pgfqpoint{6.640291in}{0.469269in}}%
\pgfpathlineto{\pgfqpoint{6.648320in}{0.349766in}}%
\pgfpathlineto{\pgfqpoint{6.648855in}{0.349043in}}%
\pgfpathlineto{\pgfqpoint{6.649390in}{0.350019in}}%
\pgfpathlineto{\pgfqpoint{6.650728in}{0.360634in}}%
\pgfpathlineto{\pgfqpoint{6.652601in}{0.397364in}}%
\pgfpathlineto{\pgfqpoint{6.655278in}{0.498065in}}%
\pgfpathlineto{\pgfqpoint{6.659292in}{0.750319in}}%
\pgfpathlineto{\pgfqpoint{6.663306in}{1.083123in}}%
\pgfpathlineto{\pgfqpoint{6.663306in}{1.083123in}}%
\pgfusepath{stroke}%
\end{pgfscope}%
\begin{pgfscope}%
\pgfsetrectcap%
\pgfsetmiterjoin%
\pgfsetlinewidth{0.803000pt}%
\definecolor{currentstroke}{rgb}{0.000000,0.000000,0.000000}%
\pgfsetstrokecolor{currentstroke}%
\pgfsetdash{}{0pt}%
\pgfpathmoveto{\pgfqpoint{0.467797in}{0.273305in}}%
\pgfpathlineto{\pgfqpoint{0.467797in}{1.939546in}}%
\pgfusepath{stroke}%
\end{pgfscope}%
\begin{pgfscope}%
\pgfsetrectcap%
\pgfsetmiterjoin%
\pgfsetlinewidth{0.803000pt}%
\definecolor{currentstroke}{rgb}{0.000000,0.000000,0.000000}%
\pgfsetstrokecolor{currentstroke}%
\pgfsetdash{}{0pt}%
\pgfpathmoveto{\pgfqpoint{6.958330in}{0.273305in}}%
\pgfpathlineto{\pgfqpoint{6.958330in}{1.939546in}}%
\pgfusepath{stroke}%
\end{pgfscope}%
\begin{pgfscope}%
\pgfsetrectcap%
\pgfsetmiterjoin%
\pgfsetlinewidth{0.803000pt}%
\definecolor{currentstroke}{rgb}{0.000000,0.000000,0.000000}%
\pgfsetstrokecolor{currentstroke}%
\pgfsetdash{}{0pt}%
\pgfpathmoveto{\pgfqpoint{0.467797in}{0.273305in}}%
\pgfpathlineto{\pgfqpoint{6.958330in}{0.273305in}}%
\pgfusepath{stroke}%
\end{pgfscope}%
\begin{pgfscope}%
\pgfsetrectcap%
\pgfsetmiterjoin%
\pgfsetlinewidth{0.803000pt}%
\definecolor{currentstroke}{rgb}{0.000000,0.000000,0.000000}%
\pgfsetstrokecolor{currentstroke}%
\pgfsetdash{}{0pt}%
\pgfpathmoveto{\pgfqpoint{0.467797in}{1.939546in}}%
\pgfpathlineto{\pgfqpoint{6.958330in}{1.939546in}}%
\pgfusepath{stroke}%
\end{pgfscope}%
\end{pgfpicture}%
\makeatother%
\endgroup%
}
    \end{center}
    \caption{\emph{Onda quadra} ottenuta sommando 100 sinusoidi armoniche dispari.}
\end{figure}

\item Nell'onda triangolare, l'ampiezza di ciascuna armonica di ordine dispari è inversamente proporzionale al quadrato del suo ordine; le armoniche di ordine pari hanno ampiezza nulla. Quindi, se la fondamentale (armonica 1) ha ampiezza 1, allora la terza armonica (di frequenza tripla) ha ampiezza $\frac{1}{9}$, la quinta armonica (di frequenza quintupla rispetto alla fondamentale) ha ampiezza $\frac{1}{25}$ e così via. Le armoniche di ordine 2, 4, 6 eccetera non sono presenti. Le armoniche di ordine dispari hanno termini di fase alternati: se la fondamentale ha fase nulla, allora la terza armonica ha fase $\pi$ (cioè fase opposta rispetto alla fondamentale), la quinta ha fase nulla, la settima fase $\pi$ e così via.

\begin{figure}
    \begin{center}
       \scalebox{0.6} {%% Creator: Matplotlib, PGF backend
%%
%% To include the figure in your LaTeX document, write
%%   \input{<filename>.pgf}
%%
%% Make sure the required packages are loaded in your preamble
%%   \usepackage{pgf}
%%
%% Also ensure that all the required font packages are loaded; for instance,
%% the lmodern package is sometimes necessary when using math font.
%%   \usepackage{lmodern}
%%
%% Figures using additional raster images can only be included by \input if
%% they are in the same directory as the main LaTeX file. For loading figures
%% from other directories you can use the `import` package
%%   \usepackage{import}
%%
%% and then include the figures with
%%   \import{<path to file>}{<filename>.pgf}
%%
%% Matplotlib used the following preamble
%%   
%%   \makeatletter\@ifpackageloaded{underscore}{}{\usepackage[strings]{underscore}}\makeatother
%%
\begingroup%
\makeatletter%
\begin{pgfpicture}%
\pgfpathrectangle{\pgfpointorigin}{\pgfqpoint{6.400000in}{4.800000in}}%
\pgfusepath{use as bounding box, clip}%
\begin{pgfscope}%
\pgfsetbuttcap%
\pgfsetmiterjoin%
\definecolor{currentfill}{rgb}{1.000000,1.000000,1.000000}%
\pgfsetfillcolor{currentfill}%
\pgfsetlinewidth{0.000000pt}%
\definecolor{currentstroke}{rgb}{1.000000,1.000000,1.000000}%
\pgfsetstrokecolor{currentstroke}%
\pgfsetdash{}{0pt}%
\pgfpathmoveto{\pgfqpoint{0.000000in}{0.000000in}}%
\pgfpathlineto{\pgfqpoint{6.400000in}{0.000000in}}%
\pgfpathlineto{\pgfqpoint{6.400000in}{4.800000in}}%
\pgfpathlineto{\pgfqpoint{0.000000in}{4.800000in}}%
\pgfpathlineto{\pgfqpoint{0.000000in}{0.000000in}}%
\pgfpathclose%
\pgfusepath{fill}%
\end{pgfscope}%
\begin{pgfscope}%
\pgfsetbuttcap%
\pgfsetmiterjoin%
\definecolor{currentfill}{rgb}{1.000000,1.000000,1.000000}%
\pgfsetfillcolor{currentfill}%
\pgfsetlinewidth{0.000000pt}%
\definecolor{currentstroke}{rgb}{0.000000,0.000000,0.000000}%
\pgfsetstrokecolor{currentstroke}%
\pgfsetstrokeopacity{0.000000}%
\pgfsetdash{}{0pt}%
\pgfpathmoveto{\pgfqpoint{0.800000in}{0.528000in}}%
\pgfpathlineto{\pgfqpoint{5.760000in}{0.528000in}}%
\pgfpathlineto{\pgfqpoint{5.760000in}{4.224000in}}%
\pgfpathlineto{\pgfqpoint{0.800000in}{4.224000in}}%
\pgfpathlineto{\pgfqpoint{0.800000in}{0.528000in}}%
\pgfpathclose%
\pgfusepath{fill}%
\end{pgfscope}%
\begin{pgfscope}%
\pgfsetbuttcap%
\pgfsetroundjoin%
\definecolor{currentfill}{rgb}{0.000000,0.000000,0.000000}%
\pgfsetfillcolor{currentfill}%
\pgfsetlinewidth{0.803000pt}%
\definecolor{currentstroke}{rgb}{0.000000,0.000000,0.000000}%
\pgfsetstrokecolor{currentstroke}%
\pgfsetdash{}{0pt}%
\pgfsys@defobject{currentmarker}{\pgfqpoint{0.000000in}{-0.048611in}}{\pgfqpoint{0.000000in}{0.000000in}}{%
\pgfpathmoveto{\pgfqpoint{0.000000in}{0.000000in}}%
\pgfpathlineto{\pgfqpoint{0.000000in}{-0.048611in}}%
\pgfusepath{stroke,fill}%
}%
\begin{pgfscope}%
\pgfsys@transformshift{1.025455in}{0.528000in}%
\pgfsys@useobject{currentmarker}{}%
\end{pgfscope}%
\end{pgfscope}%
\begin{pgfscope}%
\definecolor{textcolor}{rgb}{0.000000,0.000000,0.000000}%
\pgfsetstrokecolor{textcolor}%
\pgfsetfillcolor{textcolor}%
\pgftext[x=1.025455in,y=0.430778in,,top]{\color{textcolor}\rmfamily\fontsize{10.000000}{12.000000}\selectfont \(\displaystyle {0}\)}%
\end{pgfscope}%
\begin{pgfscope}%
\pgfsetbuttcap%
\pgfsetroundjoin%
\definecolor{currentfill}{rgb}{0.000000,0.000000,0.000000}%
\pgfsetfillcolor{currentfill}%
\pgfsetlinewidth{0.803000pt}%
\definecolor{currentstroke}{rgb}{0.000000,0.000000,0.000000}%
\pgfsetstrokecolor{currentstroke}%
\pgfsetdash{}{0pt}%
\pgfsys@defobject{currentmarker}{\pgfqpoint{0.000000in}{-0.048611in}}{\pgfqpoint{0.000000in}{0.000000in}}{%
\pgfpathmoveto{\pgfqpoint{0.000000in}{0.000000in}}%
\pgfpathlineto{\pgfqpoint{0.000000in}{-0.048611in}}%
\pgfusepath{stroke,fill}%
}%
\begin{pgfscope}%
\pgfsys@transformshift{2.047971in}{0.528000in}%
\pgfsys@useobject{currentmarker}{}%
\end{pgfscope}%
\end{pgfscope}%
\begin{pgfscope}%
\definecolor{textcolor}{rgb}{0.000000,0.000000,0.000000}%
\pgfsetstrokecolor{textcolor}%
\pgfsetfillcolor{textcolor}%
\pgftext[x=2.047971in,y=0.430778in,,top]{\color{textcolor}\rmfamily\fontsize{10.000000}{12.000000}\selectfont \(\displaystyle {5000}\)}%
\end{pgfscope}%
\begin{pgfscope}%
\pgfsetbuttcap%
\pgfsetroundjoin%
\definecolor{currentfill}{rgb}{0.000000,0.000000,0.000000}%
\pgfsetfillcolor{currentfill}%
\pgfsetlinewidth{0.803000pt}%
\definecolor{currentstroke}{rgb}{0.000000,0.000000,0.000000}%
\pgfsetstrokecolor{currentstroke}%
\pgfsetdash{}{0pt}%
\pgfsys@defobject{currentmarker}{\pgfqpoint{0.000000in}{-0.048611in}}{\pgfqpoint{0.000000in}{0.000000in}}{%
\pgfpathmoveto{\pgfqpoint{0.000000in}{0.000000in}}%
\pgfpathlineto{\pgfqpoint{0.000000in}{-0.048611in}}%
\pgfusepath{stroke,fill}%
}%
\begin{pgfscope}%
\pgfsys@transformshift{3.070486in}{0.528000in}%
\pgfsys@useobject{currentmarker}{}%
\end{pgfscope}%
\end{pgfscope}%
\begin{pgfscope}%
\definecolor{textcolor}{rgb}{0.000000,0.000000,0.000000}%
\pgfsetstrokecolor{textcolor}%
\pgfsetfillcolor{textcolor}%
\pgftext[x=3.070486in,y=0.430778in,,top]{\color{textcolor}\rmfamily\fontsize{10.000000}{12.000000}\selectfont \(\displaystyle {10000}\)}%
\end{pgfscope}%
\begin{pgfscope}%
\pgfsetbuttcap%
\pgfsetroundjoin%
\definecolor{currentfill}{rgb}{0.000000,0.000000,0.000000}%
\pgfsetfillcolor{currentfill}%
\pgfsetlinewidth{0.803000pt}%
\definecolor{currentstroke}{rgb}{0.000000,0.000000,0.000000}%
\pgfsetstrokecolor{currentstroke}%
\pgfsetdash{}{0pt}%
\pgfsys@defobject{currentmarker}{\pgfqpoint{0.000000in}{-0.048611in}}{\pgfqpoint{0.000000in}{0.000000in}}{%
\pgfpathmoveto{\pgfqpoint{0.000000in}{0.000000in}}%
\pgfpathlineto{\pgfqpoint{0.000000in}{-0.048611in}}%
\pgfusepath{stroke,fill}%
}%
\begin{pgfscope}%
\pgfsys@transformshift{4.093002in}{0.528000in}%
\pgfsys@useobject{currentmarker}{}%
\end{pgfscope}%
\end{pgfscope}%
\begin{pgfscope}%
\definecolor{textcolor}{rgb}{0.000000,0.000000,0.000000}%
\pgfsetstrokecolor{textcolor}%
\pgfsetfillcolor{textcolor}%
\pgftext[x=4.093002in,y=0.430778in,,top]{\color{textcolor}\rmfamily\fontsize{10.000000}{12.000000}\selectfont \(\displaystyle {15000}\)}%
\end{pgfscope}%
\begin{pgfscope}%
\pgfsetbuttcap%
\pgfsetroundjoin%
\definecolor{currentfill}{rgb}{0.000000,0.000000,0.000000}%
\pgfsetfillcolor{currentfill}%
\pgfsetlinewidth{0.803000pt}%
\definecolor{currentstroke}{rgb}{0.000000,0.000000,0.000000}%
\pgfsetstrokecolor{currentstroke}%
\pgfsetdash{}{0pt}%
\pgfsys@defobject{currentmarker}{\pgfqpoint{0.000000in}{-0.048611in}}{\pgfqpoint{0.000000in}{0.000000in}}{%
\pgfpathmoveto{\pgfqpoint{0.000000in}{0.000000in}}%
\pgfpathlineto{\pgfqpoint{0.000000in}{-0.048611in}}%
\pgfusepath{stroke,fill}%
}%
\begin{pgfscope}%
\pgfsys@transformshift{5.115518in}{0.528000in}%
\pgfsys@useobject{currentmarker}{}%
\end{pgfscope}%
\end{pgfscope}%
\begin{pgfscope}%
\definecolor{textcolor}{rgb}{0.000000,0.000000,0.000000}%
\pgfsetstrokecolor{textcolor}%
\pgfsetfillcolor{textcolor}%
\pgftext[x=5.115518in,y=0.430778in,,top]{\color{textcolor}\rmfamily\fontsize{10.000000}{12.000000}\selectfont \(\displaystyle {20000}\)}%
\end{pgfscope}%
\begin{pgfscope}%
\pgfsetbuttcap%
\pgfsetroundjoin%
\definecolor{currentfill}{rgb}{0.000000,0.000000,0.000000}%
\pgfsetfillcolor{currentfill}%
\pgfsetlinewidth{0.803000pt}%
\definecolor{currentstroke}{rgb}{0.000000,0.000000,0.000000}%
\pgfsetstrokecolor{currentstroke}%
\pgfsetdash{}{0pt}%
\pgfsys@defobject{currentmarker}{\pgfqpoint{-0.048611in}{0.000000in}}{\pgfqpoint{-0.000000in}{0.000000in}}{%
\pgfpathmoveto{\pgfqpoint{-0.000000in}{0.000000in}}%
\pgfpathlineto{\pgfqpoint{-0.048611in}{0.000000in}}%
\pgfusepath{stroke,fill}%
}%
\begin{pgfscope}%
\pgfsys@transformshift{0.800000in}{0.956919in}%
\pgfsys@useobject{currentmarker}{}%
\end{pgfscope}%
\end{pgfscope}%
\begin{pgfscope}%
\definecolor{textcolor}{rgb}{0.000000,0.000000,0.000000}%
\pgfsetstrokecolor{textcolor}%
\pgfsetfillcolor{textcolor}%
\pgftext[x=0.417283in, y=0.908694in, left, base]{\color{textcolor}\rmfamily\fontsize{10.000000}{12.000000}\selectfont \(\displaystyle {\ensuremath{-}1.0}\)}%
\end{pgfscope}%
\begin{pgfscope}%
\pgfsetbuttcap%
\pgfsetroundjoin%
\definecolor{currentfill}{rgb}{0.000000,0.000000,0.000000}%
\pgfsetfillcolor{currentfill}%
\pgfsetlinewidth{0.803000pt}%
\definecolor{currentstroke}{rgb}{0.000000,0.000000,0.000000}%
\pgfsetstrokecolor{currentstroke}%
\pgfsetdash{}{0pt}%
\pgfsys@defobject{currentmarker}{\pgfqpoint{-0.048611in}{0.000000in}}{\pgfqpoint{-0.000000in}{0.000000in}}{%
\pgfpathmoveto{\pgfqpoint{-0.000000in}{0.000000in}}%
\pgfpathlineto{\pgfqpoint{-0.048611in}{0.000000in}}%
\pgfusepath{stroke,fill}%
}%
\begin{pgfscope}%
\pgfsys@transformshift{0.800000in}{1.666460in}%
\pgfsys@useobject{currentmarker}{}%
\end{pgfscope}%
\end{pgfscope}%
\begin{pgfscope}%
\definecolor{textcolor}{rgb}{0.000000,0.000000,0.000000}%
\pgfsetstrokecolor{textcolor}%
\pgfsetfillcolor{textcolor}%
\pgftext[x=0.417283in, y=1.618234in, left, base]{\color{textcolor}\rmfamily\fontsize{10.000000}{12.000000}\selectfont \(\displaystyle {\ensuremath{-}0.5}\)}%
\end{pgfscope}%
\begin{pgfscope}%
\pgfsetbuttcap%
\pgfsetroundjoin%
\definecolor{currentfill}{rgb}{0.000000,0.000000,0.000000}%
\pgfsetfillcolor{currentfill}%
\pgfsetlinewidth{0.803000pt}%
\definecolor{currentstroke}{rgb}{0.000000,0.000000,0.000000}%
\pgfsetstrokecolor{currentstroke}%
\pgfsetdash{}{0pt}%
\pgfsys@defobject{currentmarker}{\pgfqpoint{-0.048611in}{0.000000in}}{\pgfqpoint{-0.000000in}{0.000000in}}{%
\pgfpathmoveto{\pgfqpoint{-0.000000in}{0.000000in}}%
\pgfpathlineto{\pgfqpoint{-0.048611in}{0.000000in}}%
\pgfusepath{stroke,fill}%
}%
\begin{pgfscope}%
\pgfsys@transformshift{0.800000in}{2.376000in}%
\pgfsys@useobject{currentmarker}{}%
\end{pgfscope}%
\end{pgfscope}%
\begin{pgfscope}%
\definecolor{textcolor}{rgb}{0.000000,0.000000,0.000000}%
\pgfsetstrokecolor{textcolor}%
\pgfsetfillcolor{textcolor}%
\pgftext[x=0.525308in, y=2.327775in, left, base]{\color{textcolor}\rmfamily\fontsize{10.000000}{12.000000}\selectfont \(\displaystyle {0.0}\)}%
\end{pgfscope}%
\begin{pgfscope}%
\pgfsetbuttcap%
\pgfsetroundjoin%
\definecolor{currentfill}{rgb}{0.000000,0.000000,0.000000}%
\pgfsetfillcolor{currentfill}%
\pgfsetlinewidth{0.803000pt}%
\definecolor{currentstroke}{rgb}{0.000000,0.000000,0.000000}%
\pgfsetstrokecolor{currentstroke}%
\pgfsetdash{}{0pt}%
\pgfsys@defobject{currentmarker}{\pgfqpoint{-0.048611in}{0.000000in}}{\pgfqpoint{-0.000000in}{0.000000in}}{%
\pgfpathmoveto{\pgfqpoint{-0.000000in}{0.000000in}}%
\pgfpathlineto{\pgfqpoint{-0.048611in}{0.000000in}}%
\pgfusepath{stroke,fill}%
}%
\begin{pgfscope}%
\pgfsys@transformshift{0.800000in}{3.085541in}%
\pgfsys@useobject{currentmarker}{}%
\end{pgfscope}%
\end{pgfscope}%
\begin{pgfscope}%
\definecolor{textcolor}{rgb}{0.000000,0.000000,0.000000}%
\pgfsetstrokecolor{textcolor}%
\pgfsetfillcolor{textcolor}%
\pgftext[x=0.525308in, y=3.037315in, left, base]{\color{textcolor}\rmfamily\fontsize{10.000000}{12.000000}\selectfont \(\displaystyle {0.5}\)}%
\end{pgfscope}%
\begin{pgfscope}%
\pgfsetbuttcap%
\pgfsetroundjoin%
\definecolor{currentfill}{rgb}{0.000000,0.000000,0.000000}%
\pgfsetfillcolor{currentfill}%
\pgfsetlinewidth{0.803000pt}%
\definecolor{currentstroke}{rgb}{0.000000,0.000000,0.000000}%
\pgfsetstrokecolor{currentstroke}%
\pgfsetdash{}{0pt}%
\pgfsys@defobject{currentmarker}{\pgfqpoint{-0.048611in}{0.000000in}}{\pgfqpoint{-0.000000in}{0.000000in}}{%
\pgfpathmoveto{\pgfqpoint{-0.000000in}{0.000000in}}%
\pgfpathlineto{\pgfqpoint{-0.048611in}{0.000000in}}%
\pgfusepath{stroke,fill}%
}%
\begin{pgfscope}%
\pgfsys@transformshift{0.800000in}{3.795081in}%
\pgfsys@useobject{currentmarker}{}%
\end{pgfscope}%
\end{pgfscope}%
\begin{pgfscope}%
\definecolor{textcolor}{rgb}{0.000000,0.000000,0.000000}%
\pgfsetstrokecolor{textcolor}%
\pgfsetfillcolor{textcolor}%
\pgftext[x=0.525308in, y=3.746856in, left, base]{\color{textcolor}\rmfamily\fontsize{10.000000}{12.000000}\selectfont \(\displaystyle {1.0}\)}%
\end{pgfscope}%
\begin{pgfscope}%
\pgfpathrectangle{\pgfqpoint{0.800000in}{0.528000in}}{\pgfqpoint{4.960000in}{3.696000in}}%
\pgfusepath{clip}%
\pgfsetrectcap%
\pgfsetroundjoin%
\pgfsetlinewidth{1.505625pt}%
\definecolor{currentstroke}{rgb}{0.121569,0.466667,0.705882}%
\pgfsetstrokecolor{currentstroke}%
\pgfsetdash{}{0pt}%
\pgfpathmoveto{\pgfqpoint{1.025455in}{0.696000in}}%
\pgfpathlineto{\pgfqpoint{1.031794in}{0.697106in}}%
\pgfpathlineto{\pgfqpoint{1.038338in}{0.700557in}}%
\pgfpathlineto{\pgfqpoint{1.045496in}{0.706972in}}%
\pgfpathlineto{\pgfqpoint{1.053471in}{0.717268in}}%
\pgfpathlineto{\pgfqpoint{1.062470in}{0.732667in}}%
\pgfpathlineto{\pgfqpoint{1.072899in}{0.755137in}}%
\pgfpathlineto{\pgfqpoint{1.085169in}{0.787155in}}%
\pgfpathlineto{\pgfqpoint{1.100098in}{0.832708in}}%
\pgfpathlineto{\pgfqpoint{1.120140in}{0.901626in}}%
\pgfpathlineto{\pgfqpoint{1.192943in}{1.158488in}}%
\pgfpathlineto{\pgfqpoint{1.218506in}{1.236488in}}%
\pgfpathlineto{\pgfqpoint{1.254089in}{1.337025in}}%
\pgfpathlineto{\pgfqpoint{1.295603in}{1.455870in}}%
\pgfpathlineto{\pgfqpoint{1.325870in}{1.549865in}}%
\pgfpathlineto{\pgfqpoint{1.367588in}{1.687768in}}%
\pgfpathlineto{\pgfqpoint{1.416260in}{1.846672in}}%
\pgfpathlineto{\pgfqpoint{1.451844in}{1.954724in}}%
\pgfpathlineto{\pgfqpoint{1.546529in}{2.237229in}}%
\pgfpathlineto{\pgfqpoint{1.591315in}{2.383259in}}%
\pgfpathlineto{\pgfqpoint{1.644281in}{2.554315in}}%
\pgfpathlineto{\pgfqpoint{1.681705in}{2.667065in}}%
\pgfpathlineto{\pgfqpoint{1.764120in}{2.912085in}}%
\pgfpathlineto{\pgfqpoint{1.801135in}{3.032489in}}%
\pgfpathlineto{\pgfqpoint{1.869235in}{3.255390in}}%
\pgfpathlineto{\pgfqpoint{1.900115in}{3.347303in}}%
\pgfpathlineto{\pgfqpoint{1.989278in}{3.606357in}}%
\pgfpathlineto{\pgfqpoint{2.014432in}{3.691786in}}%
\pgfpathlineto{\pgfqpoint{2.056151in}{3.843393in}}%
\pgfpathlineto{\pgfqpoint{2.080282in}{3.926216in}}%
\pgfpathlineto{\pgfqpoint{2.096438in}{3.974195in}}%
\pgfpathlineto{\pgfqpoint{2.109526in}{4.006459in}}%
\pgfpathlineto{\pgfqpoint{2.120569in}{4.028041in}}%
\pgfpathlineto{\pgfqpoint{2.129976in}{4.041833in}}%
\pgfpathlineto{\pgfqpoint{2.138156in}{4.050136in}}%
\pgfpathlineto{\pgfqpoint{2.145314in}{4.054467in}}%
\pgfpathlineto{\pgfqpoint{2.151858in}{4.055977in}}%
\pgfpathlineto{\pgfqpoint{2.158198in}{4.055192in}}%
\pgfpathlineto{\pgfqpoint{2.164742in}{4.052069in}}%
\pgfpathlineto{\pgfqpoint{2.171695in}{4.046216in}}%
\pgfpathlineto{\pgfqpoint{2.179466in}{4.036672in}}%
\pgfpathlineto{\pgfqpoint{2.188464in}{4.021850in}}%
\pgfpathlineto{\pgfqpoint{2.198689in}{4.000459in}}%
\pgfpathlineto{\pgfqpoint{2.210755in}{3.969710in}}%
\pgfpathlineto{\pgfqpoint{2.225275in}{3.926216in}}%
\pgfpathlineto{\pgfqpoint{2.244294in}{3.861680in}}%
\pgfpathlineto{\pgfqpoint{2.326708in}{3.573173in}}%
\pgfpathlineto{\pgfqpoint{2.353294in}{3.493915in}}%
\pgfpathlineto{\pgfqpoint{2.446547in}{3.223359in}}%
\pgfpathlineto{\pgfqpoint{2.482949in}{3.104242in}}%
\pgfpathlineto{\pgfqpoint{2.544913in}{2.901160in}}%
\pgfpathlineto{\pgfqpoint{2.580497in}{2.793361in}}%
\pgfpathlineto{\pgfqpoint{2.671910in}{2.520935in}}%
\pgfpathlineto{\pgfqpoint{2.715060in}{2.380558in}}%
\pgfpathlineto{\pgfqpoint{2.771503in}{2.198003in}}%
\pgfpathlineto{\pgfqpoint{2.808927in}{2.085235in}}%
\pgfpathlineto{\pgfqpoint{2.891546in}{1.839593in}}%
\pgfpathlineto{\pgfqpoint{2.928561in}{1.719171in}}%
\pgfpathlineto{\pgfqpoint{2.996456in}{1.496927in}}%
\pgfpathlineto{\pgfqpoint{3.027336in}{1.404991in}}%
\pgfpathlineto{\pgfqpoint{3.116909in}{1.144647in}}%
\pgfpathlineto{\pgfqpoint{3.142063in}{1.059130in}}%
\pgfpathlineto{\pgfqpoint{3.186031in}{0.899429in}}%
\pgfpathlineto{\pgfqpoint{3.209140in}{0.820908in}}%
\pgfpathlineto{\pgfqpoint{3.224886in}{0.774776in}}%
\pgfpathlineto{\pgfqpoint{3.237770in}{0.743534in}}%
\pgfpathlineto{\pgfqpoint{3.248609in}{0.722757in}}%
\pgfpathlineto{\pgfqpoint{3.258016in}{0.709301in}}%
\pgfpathlineto{\pgfqpoint{3.266196in}{0.701306in}}%
\pgfpathlineto{\pgfqpoint{3.273354in}{0.697254in}}%
\pgfpathlineto{\pgfqpoint{3.279898in}{0.696001in}}%
\pgfpathlineto{\pgfqpoint{3.286237in}{0.697036in}}%
\pgfpathlineto{\pgfqpoint{3.292781in}{0.700414in}}%
\pgfpathlineto{\pgfqpoint{3.299939in}{0.706751in}}%
\pgfpathlineto{\pgfqpoint{3.307915in}{0.716964in}}%
\pgfpathlineto{\pgfqpoint{3.316913in}{0.732275in}}%
\pgfpathlineto{\pgfqpoint{3.327342in}{0.754652in}}%
\pgfpathlineto{\pgfqpoint{3.339613in}{0.786578in}}%
\pgfpathlineto{\pgfqpoint{3.354541in}{0.832044in}}%
\pgfpathlineto{\pgfqpoint{3.374583in}{0.900893in}}%
\pgfpathlineto{\pgfqpoint{3.447999in}{1.159794in}}%
\pgfpathlineto{\pgfqpoint{3.473562in}{1.237683in}}%
\pgfpathlineto{\pgfqpoint{3.509555in}{1.339302in}}%
\pgfpathlineto{\pgfqpoint{3.550455in}{1.456481in}}%
\pgfpathlineto{\pgfqpoint{3.580722in}{1.550523in}}%
\pgfpathlineto{\pgfqpoint{3.622645in}{1.689136in}}%
\pgfpathlineto{\pgfqpoint{3.670908in}{1.846672in}}%
\pgfpathlineto{\pgfqpoint{3.706491in}{1.954724in}}%
\pgfpathlineto{\pgfqpoint{3.801176in}{2.237229in}}%
\pgfpathlineto{\pgfqpoint{3.845963in}{2.383259in}}%
\pgfpathlineto{\pgfqpoint{3.898929in}{2.554315in}}%
\pgfpathlineto{\pgfqpoint{3.936353in}{2.667065in}}%
\pgfpathlineto{\pgfqpoint{4.018768in}{2.912085in}}%
\pgfpathlineto{\pgfqpoint{4.055783in}{3.032489in}}%
\pgfpathlineto{\pgfqpoint{4.123882in}{3.255390in}}%
\pgfpathlineto{\pgfqpoint{4.154762in}{3.347303in}}%
\pgfpathlineto{\pgfqpoint{4.243926in}{3.606357in}}%
\pgfpathlineto{\pgfqpoint{4.269080in}{3.691786in}}%
\pgfpathlineto{\pgfqpoint{4.310798in}{3.843393in}}%
\pgfpathlineto{\pgfqpoint{4.334930in}{3.926216in}}%
\pgfpathlineto{\pgfqpoint{4.351085in}{3.974195in}}%
\pgfpathlineto{\pgfqpoint{4.364174in}{4.006459in}}%
\pgfpathlineto{\pgfqpoint{4.375217in}{4.028041in}}%
\pgfpathlineto{\pgfqpoint{4.384624in}{4.041833in}}%
\pgfpathlineto{\pgfqpoint{4.392804in}{4.050136in}}%
\pgfpathlineto{\pgfqpoint{4.399962in}{4.054467in}}%
\pgfpathlineto{\pgfqpoint{4.406506in}{4.055977in}}%
\pgfpathlineto{\pgfqpoint{4.412845in}{4.055192in}}%
\pgfpathlineto{\pgfqpoint{4.419390in}{4.052069in}}%
\pgfpathlineto{\pgfqpoint{4.426343in}{4.046216in}}%
\pgfpathlineto{\pgfqpoint{4.434114in}{4.036672in}}%
\pgfpathlineto{\pgfqpoint{4.443112in}{4.021850in}}%
\pgfpathlineto{\pgfqpoint{4.453337in}{4.000459in}}%
\pgfpathlineto{\pgfqpoint{4.465403in}{3.969710in}}%
\pgfpathlineto{\pgfqpoint{4.479922in}{3.926216in}}%
\pgfpathlineto{\pgfqpoint{4.498941in}{3.861680in}}%
\pgfpathlineto{\pgfqpoint{4.581356in}{3.573173in}}%
\pgfpathlineto{\pgfqpoint{4.607941in}{3.493915in}}%
\pgfpathlineto{\pgfqpoint{4.701195in}{3.223359in}}%
\pgfpathlineto{\pgfqpoint{4.737597in}{3.104242in}}%
\pgfpathlineto{\pgfqpoint{4.799561in}{2.901160in}}%
\pgfpathlineto{\pgfqpoint{4.835145in}{2.793361in}}%
\pgfpathlineto{\pgfqpoint{4.926557in}{2.520935in}}%
\pgfpathlineto{\pgfqpoint{4.969708in}{2.380558in}}%
\pgfpathlineto{\pgfqpoint{5.026151in}{2.198003in}}%
\pgfpathlineto{\pgfqpoint{5.063575in}{2.085235in}}%
\pgfpathlineto{\pgfqpoint{5.146194in}{1.839593in}}%
\pgfpathlineto{\pgfqpoint{5.183209in}{1.719171in}}%
\pgfpathlineto{\pgfqpoint{5.251104in}{1.496927in}}%
\pgfpathlineto{\pgfqpoint{5.281984in}{1.404991in}}%
\pgfpathlineto{\pgfqpoint{5.371556in}{1.144647in}}%
\pgfpathlineto{\pgfqpoint{5.396710in}{1.059130in}}%
\pgfpathlineto{\pgfqpoint{5.440678in}{0.899429in}}%
\pgfpathlineto{\pgfqpoint{5.463787in}{0.820908in}}%
\pgfpathlineto{\pgfqpoint{5.479534in}{0.774776in}}%
\pgfpathlineto{\pgfqpoint{5.492418in}{0.743534in}}%
\pgfpathlineto{\pgfqpoint{5.503256in}{0.722757in}}%
\pgfpathlineto{\pgfqpoint{5.512664in}{0.709301in}}%
\pgfpathlineto{\pgfqpoint{5.520844in}{0.701306in}}%
\pgfpathlineto{\pgfqpoint{5.528001in}{0.697254in}}%
\pgfpathlineto{\pgfqpoint{5.534545in}{0.696001in}}%
\pgfpathlineto{\pgfqpoint{5.534545in}{0.696001in}}%
\pgfusepath{stroke}%
\end{pgfscope}%
\begin{pgfscope}%
\pgfsetrectcap%
\pgfsetmiterjoin%
\pgfsetlinewidth{0.803000pt}%
\definecolor{currentstroke}{rgb}{0.000000,0.000000,0.000000}%
\pgfsetstrokecolor{currentstroke}%
\pgfsetdash{}{0pt}%
\pgfpathmoveto{\pgfqpoint{0.800000in}{0.528000in}}%
\pgfpathlineto{\pgfqpoint{0.800000in}{4.224000in}}%
\pgfusepath{stroke}%
\end{pgfscope}%
\begin{pgfscope}%
\pgfsetrectcap%
\pgfsetmiterjoin%
\pgfsetlinewidth{0.803000pt}%
\definecolor{currentstroke}{rgb}{0.000000,0.000000,0.000000}%
\pgfsetstrokecolor{currentstroke}%
\pgfsetdash{}{0pt}%
\pgfpathmoveto{\pgfqpoint{5.760000in}{0.528000in}}%
\pgfpathlineto{\pgfqpoint{5.760000in}{4.224000in}}%
\pgfusepath{stroke}%
\end{pgfscope}%
\begin{pgfscope}%
\pgfsetrectcap%
\pgfsetmiterjoin%
\pgfsetlinewidth{0.803000pt}%
\definecolor{currentstroke}{rgb}{0.000000,0.000000,0.000000}%
\pgfsetstrokecolor{currentstroke}%
\pgfsetdash{}{0pt}%
\pgfpathmoveto{\pgfqpoint{0.800000in}{0.528000in}}%
\pgfpathlineto{\pgfqpoint{5.760000in}{0.528000in}}%
\pgfusepath{stroke}%
\end{pgfscope}%
\begin{pgfscope}%
\pgfsetrectcap%
\pgfsetmiterjoin%
\pgfsetlinewidth{0.803000pt}%
\definecolor{currentstroke}{rgb}{0.000000,0.000000,0.000000}%
\pgfsetstrokecolor{currentstroke}%
\pgfsetdash{}{0pt}%
\pgfpathmoveto{\pgfqpoint{0.800000in}{4.224000in}}%
\pgfpathlineto{\pgfqpoint{5.760000in}{4.224000in}}%
\pgfusepath{stroke}%
\end{pgfscope}%
\end{pgfpicture}%
\makeatother%
\endgroup%
}
    \end{center}
    \caption{\emph{Onda triangolare} ottenuta sommando 10 sinusoidi armoniche dispari con fasi alternate.}
\end{figure}

\item Il treno d'impulsi è composto da una somma di armoniche tutte alla stessa ampiezza, con termine di fase nullo. Questo produce un segnale che percettivamente è estremamente, innaturalmente brillante, dal momento che in linea generale gli spettri che troviamo nel mondo fisico tendono ad avere ampiezze digradanti al crescere della frequenza. Si può fare una considerazione interessante sul treno d'impulsi: se allontaniamo gli impulsi tra loro nel tempo, stiamo aumentando il loro periodo, quindi abbassando la frequenza fondamentale, quindi ``avvicinando'' tra loro le armoniche. Più gli impulsi sono radi, più lo spettro è ``fitto'' di armoniche di pari ampiezza. Se gli impulsi diventano infinitamente radi --- cioè, se la frequenza diventa infinitesima, cioè se si considera un singolo impulso --- allora lo spettro sarà costituito da armoniche infinitamente vicine. Questo caso particolare è il cosiddetto \emph{delta di Dirac}, una funzione costituita da un singolo impulso non nullo di durata infinitesima, che contiene tutte le frequenze dello spettro, in un dominio infinito, in eguale quantità. Quando battiamo le mani per saggiare la risposta acustica di una stanza, o pizzichiamo una corda con un plettro, stiamo approssimando il comportamento di un delta di Dirac: idealmente, immettiamo nel sistema (che sia la stanza o la corda) ogni possibile frequenza attraverso un impulso estremamente breve, in maniera che il sistema risuoni nella maniera più ricca possibile producendo quella che viene chiamata la sua \emph{risposta d'impulso}.

\end{itemize}

Segnali con la stessa distribuzione spettrale di ampiezze ma fasi diverse rispetto a quelle indicate sopra avranno, in generale, lo stesso ``suono'' ma forme d'onda diverse. Questa considerazione non vale solo per i tipi di segnali che abbiamo descritto, ma per qualsiasi altro segnale.

Osserviamo infine che sinusoide, onda quadra e onda triangolare condividono la caratteristica di essere \emph{antisimmetriche}: la parte negativa del segnale, come rappresentato nel dominio del tempo, è speculare rispetto alla sua parte positiva. Rispetto al periodo della fondamentale, le sue armoniche dispari sono antisimmetriche, quelle pari no: per questa ragione i segnali antisimmetrici contengono solo armoniche di ordine dispari (questo è vero anche per la sinusoide, la cui unica componente è di ordine 1) e sono quindi detti essi stessi \emph{dispari}.



\subsection{Spettri inarmonici a componenti discrete}

Esiste una categoria di spettri che non consideriamo armonici, e che sono costituiti da componenti sinusoidali individuali organizzate in maniera non regolare. Da un punto di vista puramente teorico bisogna fare una distinzione di una certa sottigliezza: se le frequenze delle componenti hanno rapporti razionali tra loro, in senso proprio il segnale sarà sempre periodico, con un periodo pari al minimo comune multiplo dei periodi delle componenti (e quindi con una frequenza pari al massimo comun divisore delle frequenze delle componenti). Per esempio, se un segnale è composto da una componente a 100 Hz, una a 212 Hz e una a 213.5 Hz ci sarà comunque una periodicità di 0.5 Hz. Solo rapporti di frequenza irrazionali (per esempio, tra $100 \sqrt{2}$ Hz e $200 \pi$ Hz) non definiscono un massimo comun divisore e quindi produrranno un segnale puramente inarmonico.

\begin{figure}
    \begin{center}
       \scalebox{0.6} {%% Creator: Matplotlib, PGF backend
%%
%% To include the figure in your LaTeX document, write
%%   \input{<filename>.pgf}
%%
%% Make sure the required packages are loaded in your preamble
%%   \usepackage{pgf}
%%
%% Also ensure that all the required font packages are loaded; for instance,
%% the lmodern package is sometimes necessary when using math font.
%%   \usepackage{lmodern}
%%
%% Figures using additional raster images can only be included by \input if
%% they are in the same directory as the main LaTeX file. For loading figures
%% from other directories you can use the `import` package
%%   \usepackage{import}
%%
%% and then include the figures with
%%   \import{<path to file>}{<filename>.pgf}
%%
%% Matplotlib used the following preamble
%%   
%%   \usepackage{fontspec}
%%   \setmainfont{DejaVuSerif.ttf}[Path=\detokenize{/opt/homebrew/Caskroom/miniconda/base/envs/label-studio/lib/python3.9/site-packages/matplotlib/mpl-data/fonts/ttf/}]
%%   \setsansfont{DejaVuSans.ttf}[Path=\detokenize{/opt/homebrew/Caskroom/miniconda/base/envs/label-studio/lib/python3.9/site-packages/matplotlib/mpl-data/fonts/ttf/}]
%%   \setmonofont{DejaVuSansMono.ttf}[Path=\detokenize{/opt/homebrew/Caskroom/miniconda/base/envs/label-studio/lib/python3.9/site-packages/matplotlib/mpl-data/fonts/ttf/}]
%%   \makeatletter\@ifpackageloaded{underscore}{}{\usepackage[strings]{underscore}}\makeatother
%%
\begingroup%
\makeatletter%
\begin{pgfpicture}%
\pgfpathrectangle{\pgfpointorigin}{\pgfqpoint{6.400000in}{4.800000in}}%
\pgfusepath{use as bounding box, clip}%
\begin{pgfscope}%
\pgfsetbuttcap%
\pgfsetmiterjoin%
\definecolor{currentfill}{rgb}{1.000000,1.000000,1.000000}%
\pgfsetfillcolor{currentfill}%
\pgfsetlinewidth{0.000000pt}%
\definecolor{currentstroke}{rgb}{1.000000,1.000000,1.000000}%
\pgfsetstrokecolor{currentstroke}%
\pgfsetdash{}{0pt}%
\pgfpathmoveto{\pgfqpoint{0.000000in}{0.000000in}}%
\pgfpathlineto{\pgfqpoint{6.400000in}{0.000000in}}%
\pgfpathlineto{\pgfqpoint{6.400000in}{4.800000in}}%
\pgfpathlineto{\pgfqpoint{0.000000in}{4.800000in}}%
\pgfpathlineto{\pgfqpoint{0.000000in}{0.000000in}}%
\pgfpathclose%
\pgfusepath{fill}%
\end{pgfscope}%
\begin{pgfscope}%
\pgfsetbuttcap%
\pgfsetmiterjoin%
\definecolor{currentfill}{rgb}{1.000000,1.000000,1.000000}%
\pgfsetfillcolor{currentfill}%
\pgfsetlinewidth{0.000000pt}%
\definecolor{currentstroke}{rgb}{0.000000,0.000000,0.000000}%
\pgfsetstrokecolor{currentstroke}%
\pgfsetstrokeopacity{0.000000}%
\pgfsetdash{}{0pt}%
\pgfpathmoveto{\pgfqpoint{0.800000in}{2.544000in}}%
\pgfpathlineto{\pgfqpoint{5.760000in}{2.544000in}}%
\pgfpathlineto{\pgfqpoint{5.760000in}{4.224000in}}%
\pgfpathlineto{\pgfqpoint{0.800000in}{4.224000in}}%
\pgfpathlineto{\pgfqpoint{0.800000in}{2.544000in}}%
\pgfpathclose%
\pgfusepath{fill}%
\end{pgfscope}%
\begin{pgfscope}%
\pgfsetbuttcap%
\pgfsetroundjoin%
\definecolor{currentfill}{rgb}{0.000000,0.000000,0.000000}%
\pgfsetfillcolor{currentfill}%
\pgfsetlinewidth{0.803000pt}%
\definecolor{currentstroke}{rgb}{0.000000,0.000000,0.000000}%
\pgfsetstrokecolor{currentstroke}%
\pgfsetdash{}{0pt}%
\pgfsys@defobject{currentmarker}{\pgfqpoint{0.000000in}{-0.048611in}}{\pgfqpoint{0.000000in}{0.000000in}}{%
\pgfpathmoveto{\pgfqpoint{0.000000in}{0.000000in}}%
\pgfpathlineto{\pgfqpoint{0.000000in}{-0.048611in}}%
\pgfusepath{stroke,fill}%
}%
\begin{pgfscope}%
\pgfsys@transformshift{1.025455in}{2.544000in}%
\pgfsys@useobject{currentmarker}{}%
\end{pgfscope}%
\end{pgfscope}%
\begin{pgfscope}%
\definecolor{textcolor}{rgb}{0.000000,0.000000,0.000000}%
\pgfsetstrokecolor{textcolor}%
\pgfsetfillcolor{textcolor}%
\pgftext[x=1.025455in,y=2.446778in,,top]{\color{textcolor}\sffamily\fontsize{10.000000}{12.000000}\selectfont 0}%
\end{pgfscope}%
\begin{pgfscope}%
\pgfsetbuttcap%
\pgfsetroundjoin%
\definecolor{currentfill}{rgb}{0.000000,0.000000,0.000000}%
\pgfsetfillcolor{currentfill}%
\pgfsetlinewidth{0.803000pt}%
\definecolor{currentstroke}{rgb}{0.000000,0.000000,0.000000}%
\pgfsetstrokecolor{currentstroke}%
\pgfsetdash{}{0pt}%
\pgfsys@defobject{currentmarker}{\pgfqpoint{0.000000in}{-0.048611in}}{\pgfqpoint{0.000000in}{0.000000in}}{%
\pgfpathmoveto{\pgfqpoint{0.000000in}{0.000000in}}%
\pgfpathlineto{\pgfqpoint{0.000000in}{-0.048611in}}%
\pgfusepath{stroke,fill}%
}%
\begin{pgfscope}%
\pgfsys@transformshift{2.049085in}{2.544000in}%
\pgfsys@useobject{currentmarker}{}%
\end{pgfscope}%
\end{pgfscope}%
\begin{pgfscope}%
\definecolor{textcolor}{rgb}{0.000000,0.000000,0.000000}%
\pgfsetstrokecolor{textcolor}%
\pgfsetfillcolor{textcolor}%
\pgftext[x=2.049085in,y=2.446778in,,top]{\color{textcolor}\sffamily\fontsize{10.000000}{12.000000}\selectfont 200}%
\end{pgfscope}%
\begin{pgfscope}%
\pgfsetbuttcap%
\pgfsetroundjoin%
\definecolor{currentfill}{rgb}{0.000000,0.000000,0.000000}%
\pgfsetfillcolor{currentfill}%
\pgfsetlinewidth{0.803000pt}%
\definecolor{currentstroke}{rgb}{0.000000,0.000000,0.000000}%
\pgfsetstrokecolor{currentstroke}%
\pgfsetdash{}{0pt}%
\pgfsys@defobject{currentmarker}{\pgfqpoint{0.000000in}{-0.048611in}}{\pgfqpoint{0.000000in}{0.000000in}}{%
\pgfpathmoveto{\pgfqpoint{0.000000in}{0.000000in}}%
\pgfpathlineto{\pgfqpoint{0.000000in}{-0.048611in}}%
\pgfusepath{stroke,fill}%
}%
\begin{pgfscope}%
\pgfsys@transformshift{3.072715in}{2.544000in}%
\pgfsys@useobject{currentmarker}{}%
\end{pgfscope}%
\end{pgfscope}%
\begin{pgfscope}%
\definecolor{textcolor}{rgb}{0.000000,0.000000,0.000000}%
\pgfsetstrokecolor{textcolor}%
\pgfsetfillcolor{textcolor}%
\pgftext[x=3.072715in,y=2.446778in,,top]{\color{textcolor}\sffamily\fontsize{10.000000}{12.000000}\selectfont 400}%
\end{pgfscope}%
\begin{pgfscope}%
\pgfsetbuttcap%
\pgfsetroundjoin%
\definecolor{currentfill}{rgb}{0.000000,0.000000,0.000000}%
\pgfsetfillcolor{currentfill}%
\pgfsetlinewidth{0.803000pt}%
\definecolor{currentstroke}{rgb}{0.000000,0.000000,0.000000}%
\pgfsetstrokecolor{currentstroke}%
\pgfsetdash{}{0pt}%
\pgfsys@defobject{currentmarker}{\pgfqpoint{0.000000in}{-0.048611in}}{\pgfqpoint{0.000000in}{0.000000in}}{%
\pgfpathmoveto{\pgfqpoint{0.000000in}{0.000000in}}%
\pgfpathlineto{\pgfqpoint{0.000000in}{-0.048611in}}%
\pgfusepath{stroke,fill}%
}%
\begin{pgfscope}%
\pgfsys@transformshift{4.096345in}{2.544000in}%
\pgfsys@useobject{currentmarker}{}%
\end{pgfscope}%
\end{pgfscope}%
\begin{pgfscope}%
\definecolor{textcolor}{rgb}{0.000000,0.000000,0.000000}%
\pgfsetstrokecolor{textcolor}%
\pgfsetfillcolor{textcolor}%
\pgftext[x=4.096345in,y=2.446778in,,top]{\color{textcolor}\sffamily\fontsize{10.000000}{12.000000}\selectfont 600}%
\end{pgfscope}%
\begin{pgfscope}%
\pgfsetbuttcap%
\pgfsetroundjoin%
\definecolor{currentfill}{rgb}{0.000000,0.000000,0.000000}%
\pgfsetfillcolor{currentfill}%
\pgfsetlinewidth{0.803000pt}%
\definecolor{currentstroke}{rgb}{0.000000,0.000000,0.000000}%
\pgfsetstrokecolor{currentstroke}%
\pgfsetdash{}{0pt}%
\pgfsys@defobject{currentmarker}{\pgfqpoint{0.000000in}{-0.048611in}}{\pgfqpoint{0.000000in}{0.000000in}}{%
\pgfpathmoveto{\pgfqpoint{0.000000in}{0.000000in}}%
\pgfpathlineto{\pgfqpoint{0.000000in}{-0.048611in}}%
\pgfusepath{stroke,fill}%
}%
\begin{pgfscope}%
\pgfsys@transformshift{5.119975in}{2.544000in}%
\pgfsys@useobject{currentmarker}{}%
\end{pgfscope}%
\end{pgfscope}%
\begin{pgfscope}%
\definecolor{textcolor}{rgb}{0.000000,0.000000,0.000000}%
\pgfsetstrokecolor{textcolor}%
\pgfsetfillcolor{textcolor}%
\pgftext[x=5.119975in,y=2.446778in,,top]{\color{textcolor}\sffamily\fontsize{10.000000}{12.000000}\selectfont 800}%
\end{pgfscope}%
\begin{pgfscope}%
\pgfsetbuttcap%
\pgfsetroundjoin%
\definecolor{currentfill}{rgb}{0.000000,0.000000,0.000000}%
\pgfsetfillcolor{currentfill}%
\pgfsetlinewidth{0.803000pt}%
\definecolor{currentstroke}{rgb}{0.000000,0.000000,0.000000}%
\pgfsetstrokecolor{currentstroke}%
\pgfsetdash{}{0pt}%
\pgfsys@defobject{currentmarker}{\pgfqpoint{-0.048611in}{0.000000in}}{\pgfqpoint{-0.000000in}{0.000000in}}{%
\pgfpathmoveto{\pgfqpoint{-0.000000in}{0.000000in}}%
\pgfpathlineto{\pgfqpoint{-0.048611in}{0.000000in}}%
\pgfusepath{stroke,fill}%
}%
\begin{pgfscope}%
\pgfsys@transformshift{0.800000in}{2.620364in}%
\pgfsys@useobject{currentmarker}{}%
\end{pgfscope}%
\end{pgfscope}%
\begin{pgfscope}%
\definecolor{textcolor}{rgb}{0.000000,0.000000,0.000000}%
\pgfsetstrokecolor{textcolor}%
\pgfsetfillcolor{textcolor}%
\pgftext[x=0.373873in, y=2.567602in, left, base]{\color{textcolor}\sffamily\fontsize{10.000000}{12.000000}\selectfont \ensuremath{-}1.0}%
\end{pgfscope}%
\begin{pgfscope}%
\pgfsetbuttcap%
\pgfsetroundjoin%
\definecolor{currentfill}{rgb}{0.000000,0.000000,0.000000}%
\pgfsetfillcolor{currentfill}%
\pgfsetlinewidth{0.803000pt}%
\definecolor{currentstroke}{rgb}{0.000000,0.000000,0.000000}%
\pgfsetstrokecolor{currentstroke}%
\pgfsetdash{}{0pt}%
\pgfsys@defobject{currentmarker}{\pgfqpoint{-0.048611in}{0.000000in}}{\pgfqpoint{-0.000000in}{0.000000in}}{%
\pgfpathmoveto{\pgfqpoint{-0.000000in}{0.000000in}}%
\pgfpathlineto{\pgfqpoint{-0.048611in}{0.000000in}}%
\pgfusepath{stroke,fill}%
}%
\begin{pgfscope}%
\pgfsys@transformshift{0.800000in}{3.002182in}%
\pgfsys@useobject{currentmarker}{}%
\end{pgfscope}%
\end{pgfscope}%
\begin{pgfscope}%
\definecolor{textcolor}{rgb}{0.000000,0.000000,0.000000}%
\pgfsetstrokecolor{textcolor}%
\pgfsetfillcolor{textcolor}%
\pgftext[x=0.373873in, y=2.949420in, left, base]{\color{textcolor}\sffamily\fontsize{10.000000}{12.000000}\selectfont \ensuremath{-}0.5}%
\end{pgfscope}%
\begin{pgfscope}%
\pgfsetbuttcap%
\pgfsetroundjoin%
\definecolor{currentfill}{rgb}{0.000000,0.000000,0.000000}%
\pgfsetfillcolor{currentfill}%
\pgfsetlinewidth{0.803000pt}%
\definecolor{currentstroke}{rgb}{0.000000,0.000000,0.000000}%
\pgfsetstrokecolor{currentstroke}%
\pgfsetdash{}{0pt}%
\pgfsys@defobject{currentmarker}{\pgfqpoint{-0.048611in}{0.000000in}}{\pgfqpoint{-0.000000in}{0.000000in}}{%
\pgfpathmoveto{\pgfqpoint{-0.000000in}{0.000000in}}%
\pgfpathlineto{\pgfqpoint{-0.048611in}{0.000000in}}%
\pgfusepath{stroke,fill}%
}%
\begin{pgfscope}%
\pgfsys@transformshift{0.800000in}{3.384000in}%
\pgfsys@useobject{currentmarker}{}%
\end{pgfscope}%
\end{pgfscope}%
\begin{pgfscope}%
\definecolor{textcolor}{rgb}{0.000000,0.000000,0.000000}%
\pgfsetstrokecolor{textcolor}%
\pgfsetfillcolor{textcolor}%
\pgftext[x=0.481898in, y=3.331238in, left, base]{\color{textcolor}\sffamily\fontsize{10.000000}{12.000000}\selectfont 0.0}%
\end{pgfscope}%
\begin{pgfscope}%
\pgfsetbuttcap%
\pgfsetroundjoin%
\definecolor{currentfill}{rgb}{0.000000,0.000000,0.000000}%
\pgfsetfillcolor{currentfill}%
\pgfsetlinewidth{0.803000pt}%
\definecolor{currentstroke}{rgb}{0.000000,0.000000,0.000000}%
\pgfsetstrokecolor{currentstroke}%
\pgfsetdash{}{0pt}%
\pgfsys@defobject{currentmarker}{\pgfqpoint{-0.048611in}{0.000000in}}{\pgfqpoint{-0.000000in}{0.000000in}}{%
\pgfpathmoveto{\pgfqpoint{-0.000000in}{0.000000in}}%
\pgfpathlineto{\pgfqpoint{-0.048611in}{0.000000in}}%
\pgfusepath{stroke,fill}%
}%
\begin{pgfscope}%
\pgfsys@transformshift{0.800000in}{3.765818in}%
\pgfsys@useobject{currentmarker}{}%
\end{pgfscope}%
\end{pgfscope}%
\begin{pgfscope}%
\definecolor{textcolor}{rgb}{0.000000,0.000000,0.000000}%
\pgfsetstrokecolor{textcolor}%
\pgfsetfillcolor{textcolor}%
\pgftext[x=0.481898in, y=3.713057in, left, base]{\color{textcolor}\sffamily\fontsize{10.000000}{12.000000}\selectfont 0.5}%
\end{pgfscope}%
\begin{pgfscope}%
\pgfsetbuttcap%
\pgfsetroundjoin%
\definecolor{currentfill}{rgb}{0.000000,0.000000,0.000000}%
\pgfsetfillcolor{currentfill}%
\pgfsetlinewidth{0.803000pt}%
\definecolor{currentstroke}{rgb}{0.000000,0.000000,0.000000}%
\pgfsetstrokecolor{currentstroke}%
\pgfsetdash{}{0pt}%
\pgfsys@defobject{currentmarker}{\pgfqpoint{-0.048611in}{0.000000in}}{\pgfqpoint{-0.000000in}{0.000000in}}{%
\pgfpathmoveto{\pgfqpoint{-0.000000in}{0.000000in}}%
\pgfpathlineto{\pgfqpoint{-0.048611in}{0.000000in}}%
\pgfusepath{stroke,fill}%
}%
\begin{pgfscope}%
\pgfsys@transformshift{0.800000in}{4.147636in}%
\pgfsys@useobject{currentmarker}{}%
\end{pgfscope}%
\end{pgfscope}%
\begin{pgfscope}%
\definecolor{textcolor}{rgb}{0.000000,0.000000,0.000000}%
\pgfsetstrokecolor{textcolor}%
\pgfsetfillcolor{textcolor}%
\pgftext[x=0.481898in, y=4.094875in, left, base]{\color{textcolor}\sffamily\fontsize{10.000000}{12.000000}\selectfont 1.0}%
\end{pgfscope}%
\begin{pgfscope}%
\pgfpathrectangle{\pgfqpoint{0.800000in}{2.544000in}}{\pgfqpoint{4.960000in}{1.680000in}}%
\pgfusepath{clip}%
\pgfsetrectcap%
\pgfsetroundjoin%
\pgfsetlinewidth{1.505625pt}%
\definecolor{currentstroke}{rgb}{0.121569,0.466667,0.705882}%
\pgfsetstrokecolor{currentstroke}%
\pgfsetdash{}{0pt}%
\pgfpathmoveto{\pgfqpoint{1.025455in}{3.384000in}}%
\pgfpathlineto{\pgfqpoint{1.081754in}{3.619459in}}%
\pgfpathlineto{\pgfqpoint{1.112463in}{3.739620in}}%
\pgfpathlineto{\pgfqpoint{1.138054in}{3.831974in}}%
\pgfpathlineto{\pgfqpoint{1.163645in}{3.915248in}}%
\pgfpathlineto{\pgfqpoint{1.184117in}{3.974191in}}%
\pgfpathlineto{\pgfqpoint{1.204590in}{4.025474in}}%
\pgfpathlineto{\pgfqpoint{1.219944in}{4.058506in}}%
\pgfpathlineto{\pgfqpoint{1.235299in}{4.086612in}}%
\pgfpathlineto{\pgfqpoint{1.250653in}{4.109586in}}%
\pgfpathlineto{\pgfqpoint{1.266008in}{4.127261in}}%
\pgfpathlineto{\pgfqpoint{1.276244in}{4.136035in}}%
\pgfpathlineto{\pgfqpoint{1.286480in}{4.142367in}}%
\pgfpathlineto{\pgfqpoint{1.296717in}{4.146237in}}%
\pgfpathlineto{\pgfqpoint{1.306953in}{4.147632in}}%
\pgfpathlineto{\pgfqpoint{1.317189in}{4.146547in}}%
\pgfpathlineto{\pgfqpoint{1.327425in}{4.142986in}}%
\pgfpathlineto{\pgfqpoint{1.337662in}{4.136961in}}%
\pgfpathlineto{\pgfqpoint{1.347898in}{4.128490in}}%
\pgfpathlineto{\pgfqpoint{1.358134in}{4.117603in}}%
\pgfpathlineto{\pgfqpoint{1.373489in}{4.096818in}}%
\pgfpathlineto{\pgfqpoint{1.388843in}{4.070828in}}%
\pgfpathlineto{\pgfqpoint{1.404198in}{4.039822in}}%
\pgfpathlineto{\pgfqpoint{1.419552in}{4.004025in}}%
\pgfpathlineto{\pgfqpoint{1.440025in}{3.949303in}}%
\pgfpathlineto{\pgfqpoint{1.460497in}{3.887245in}}%
\pgfpathlineto{\pgfqpoint{1.480970in}{3.818655in}}%
\pgfpathlineto{\pgfqpoint{1.506561in}{3.725098in}}%
\pgfpathlineto{\pgfqpoint{1.537270in}{3.603882in}}%
\pgfpathlineto{\pgfqpoint{1.583333in}{3.411194in}}%
\pgfpathlineto{\pgfqpoint{1.649869in}{3.133071in}}%
\pgfpathlineto{\pgfqpoint{1.680578in}{3.014020in}}%
\pgfpathlineto{\pgfqpoint{1.706169in}{2.922913in}}%
\pgfpathlineto{\pgfqpoint{1.726641in}{2.856673in}}%
\pgfpathlineto{\pgfqpoint{1.747114in}{2.797276in}}%
\pgfpathlineto{\pgfqpoint{1.767586in}{2.745493in}}%
\pgfpathlineto{\pgfqpoint{1.782941in}{2.712061in}}%
\pgfpathlineto{\pgfqpoint{1.798295in}{2.683537in}}%
\pgfpathlineto{\pgfqpoint{1.813650in}{2.660128in}}%
\pgfpathlineto{\pgfqpoint{1.829004in}{2.642006in}}%
\pgfpathlineto{\pgfqpoint{1.839241in}{2.632929in}}%
\pgfpathlineto{\pgfqpoint{1.849477in}{2.626290in}}%
\pgfpathlineto{\pgfqpoint{1.859713in}{2.622112in}}%
\pgfpathlineto{\pgfqpoint{1.869949in}{2.620407in}}%
\pgfpathlineto{\pgfqpoint{1.880186in}{2.621182in}}%
\pgfpathlineto{\pgfqpoint{1.890422in}{2.624434in}}%
\pgfpathlineto{\pgfqpoint{1.900658in}{2.630152in}}%
\pgfpathlineto{\pgfqpoint{1.910895in}{2.638318in}}%
\pgfpathlineto{\pgfqpoint{1.921131in}{2.648905in}}%
\pgfpathlineto{\pgfqpoint{1.931367in}{2.661879in}}%
\pgfpathlineto{\pgfqpoint{1.946722in}{2.685721in}}%
\pgfpathlineto{\pgfqpoint{1.962076in}{2.714663in}}%
\pgfpathlineto{\pgfqpoint{1.977431in}{2.748493in}}%
\pgfpathlineto{\pgfqpoint{1.992785in}{2.786965in}}%
\pgfpathlineto{\pgfqpoint{2.013258in}{2.844990in}}%
\pgfpathlineto{\pgfqpoint{2.033730in}{2.910010in}}%
\pgfpathlineto{\pgfqpoint{2.059321in}{2.999829in}}%
\pgfpathlineto{\pgfqpoint{2.090030in}{3.117716in}}%
\pgfpathlineto{\pgfqpoint{2.125857in}{3.264810in}}%
\pgfpathlineto{\pgfqpoint{2.223102in}{3.670568in}}%
\pgfpathlineto{\pgfqpoint{2.253811in}{3.786819in}}%
\pgfpathlineto{\pgfqpoint{2.279402in}{3.874856in}}%
\pgfpathlineto{\pgfqpoint{2.299874in}{3.938203in}}%
\pgfpathlineto{\pgfqpoint{2.320347in}{3.994358in}}%
\pgfpathlineto{\pgfqpoint{2.335701in}{4.031312in}}%
\pgfpathlineto{\pgfqpoint{2.351056in}{4.063538in}}%
\pgfpathlineto{\pgfqpoint{2.366410in}{4.090802in}}%
\pgfpathlineto{\pgfqpoint{2.381765in}{4.112904in}}%
\pgfpathlineto{\pgfqpoint{2.392001in}{4.124689in}}%
\pgfpathlineto{\pgfqpoint{2.402237in}{4.134069in}}%
\pgfpathlineto{\pgfqpoint{2.412473in}{4.141014in}}%
\pgfpathlineto{\pgfqpoint{2.422710in}{4.145501in}}%
\pgfpathlineto{\pgfqpoint{2.432946in}{4.147515in}}%
\pgfpathlineto{\pgfqpoint{2.443182in}{4.147050in}}%
\pgfpathlineto{\pgfqpoint{2.453419in}{4.144108in}}%
\pgfpathlineto{\pgfqpoint{2.463655in}{4.138697in}}%
\pgfpathlineto{\pgfqpoint{2.473891in}{4.130836in}}%
\pgfpathlineto{\pgfqpoint{2.484128in}{4.120550in}}%
\pgfpathlineto{\pgfqpoint{2.494364in}{4.107872in}}%
\pgfpathlineto{\pgfqpoint{2.509718in}{4.084463in}}%
\pgfpathlineto{\pgfqpoint{2.525073in}{4.055939in}}%
\pgfpathlineto{\pgfqpoint{2.540427in}{4.022507in}}%
\pgfpathlineto{\pgfqpoint{2.555782in}{3.984412in}}%
\pgfpathlineto{\pgfqpoint{2.576254in}{3.926850in}}%
\pgfpathlineto{\pgfqpoint{2.596727in}{3.862243in}}%
\pgfpathlineto{\pgfqpoint{2.622318in}{3.772863in}}%
\pgfpathlineto{\pgfqpoint{2.653027in}{3.655376in}}%
\pgfpathlineto{\pgfqpoint{2.688854in}{3.508561in}}%
\pgfpathlineto{\pgfqpoint{2.791217in}{3.082372in}}%
\pgfpathlineto{\pgfqpoint{2.821925in}{2.967410in}}%
\pgfpathlineto{\pgfqpoint{2.847516in}{2.880755in}}%
\pgfpathlineto{\pgfqpoint{2.867989in}{2.818697in}}%
\pgfpathlineto{\pgfqpoint{2.888461in}{2.763975in}}%
\pgfpathlineto{\pgfqpoint{2.903816in}{2.728178in}}%
\pgfpathlineto{\pgfqpoint{2.919170in}{2.697172in}}%
\pgfpathlineto{\pgfqpoint{2.934525in}{2.671181in}}%
\pgfpathlineto{\pgfqpoint{2.949879in}{2.650397in}}%
\pgfpathlineto{\pgfqpoint{2.960116in}{2.639510in}}%
\pgfpathlineto{\pgfqpoint{2.970352in}{2.631039in}}%
\pgfpathlineto{\pgfqpoint{2.980588in}{2.625014in}}%
\pgfpathlineto{\pgfqpoint{2.990824in}{2.621453in}}%
\pgfpathlineto{\pgfqpoint{3.001061in}{2.620368in}}%
\pgfpathlineto{\pgfqpoint{3.011297in}{2.621763in}}%
\pgfpathlineto{\pgfqpoint{3.021533in}{2.625633in}}%
\pgfpathlineto{\pgfqpoint{3.031770in}{2.631965in}}%
\pgfpathlineto{\pgfqpoint{3.042006in}{2.640739in}}%
\pgfpathlineto{\pgfqpoint{3.052242in}{2.651926in}}%
\pgfpathlineto{\pgfqpoint{3.067597in}{2.673151in}}%
\pgfpathlineto{\pgfqpoint{3.082951in}{2.699567in}}%
\pgfpathlineto{\pgfqpoint{3.098306in}{2.730982in}}%
\pgfpathlineto{\pgfqpoint{3.113660in}{2.767166in}}%
\pgfpathlineto{\pgfqpoint{3.134133in}{2.822368in}}%
\pgfpathlineto{\pgfqpoint{3.154605in}{2.884860in}}%
\pgfpathlineto{\pgfqpoint{3.180196in}{2.971979in}}%
\pgfpathlineto{\pgfqpoint{3.205787in}{3.067449in}}%
\pgfpathlineto{\pgfqpoint{3.236496in}{3.190300in}}%
\pgfpathlineto{\pgfqpoint{3.287677in}{3.405757in}}%
\pgfpathlineto{\pgfqpoint{3.343977in}{3.640061in}}%
\pgfpathlineto{\pgfqpoint{3.374686in}{3.758730in}}%
\pgfpathlineto{\pgfqpoint{3.400277in}{3.849412in}}%
\pgfpathlineto{\pgfqpoint{3.420749in}{3.915248in}}%
\pgfpathlineto{\pgfqpoint{3.441222in}{3.974191in}}%
\pgfpathlineto{\pgfqpoint{3.461694in}{4.025474in}}%
\pgfpathlineto{\pgfqpoint{3.477049in}{4.058506in}}%
\pgfpathlineto{\pgfqpoint{3.492403in}{4.086612in}}%
\pgfpathlineto{\pgfqpoint{3.507758in}{4.109586in}}%
\pgfpathlineto{\pgfqpoint{3.523112in}{4.127261in}}%
\pgfpathlineto{\pgfqpoint{3.533348in}{4.136035in}}%
\pgfpathlineto{\pgfqpoint{3.543585in}{4.142367in}}%
\pgfpathlineto{\pgfqpoint{3.553821in}{4.146237in}}%
\pgfpathlineto{\pgfqpoint{3.564057in}{4.147632in}}%
\pgfpathlineto{\pgfqpoint{3.574294in}{4.146547in}}%
\pgfpathlineto{\pgfqpoint{3.584530in}{4.142986in}}%
\pgfpathlineto{\pgfqpoint{3.594766in}{4.136961in}}%
\pgfpathlineto{\pgfqpoint{3.605003in}{4.128490in}}%
\pgfpathlineto{\pgfqpoint{3.615239in}{4.117603in}}%
\pgfpathlineto{\pgfqpoint{3.630593in}{4.096818in}}%
\pgfpathlineto{\pgfqpoint{3.645948in}{4.070828in}}%
\pgfpathlineto{\pgfqpoint{3.661302in}{4.039822in}}%
\pgfpathlineto{\pgfqpoint{3.676657in}{4.004025in}}%
\pgfpathlineto{\pgfqpoint{3.697129in}{3.949303in}}%
\pgfpathlineto{\pgfqpoint{3.717602in}{3.887245in}}%
\pgfpathlineto{\pgfqpoint{3.738075in}{3.818655in}}%
\pgfpathlineto{\pgfqpoint{3.763665in}{3.725098in}}%
\pgfpathlineto{\pgfqpoint{3.794374in}{3.603882in}}%
\pgfpathlineto{\pgfqpoint{3.840438in}{3.411194in}}%
\pgfpathlineto{\pgfqpoint{3.906973in}{3.133071in}}%
\pgfpathlineto{\pgfqpoint{3.937682in}{3.014020in}}%
\pgfpathlineto{\pgfqpoint{3.963273in}{2.922913in}}%
\pgfpathlineto{\pgfqpoint{3.983746in}{2.856673in}}%
\pgfpathlineto{\pgfqpoint{4.004218in}{2.797276in}}%
\pgfpathlineto{\pgfqpoint{4.024691in}{2.745493in}}%
\pgfpathlineto{\pgfqpoint{4.040045in}{2.712061in}}%
\pgfpathlineto{\pgfqpoint{4.055400in}{2.683537in}}%
\pgfpathlineto{\pgfqpoint{4.070754in}{2.660128in}}%
\pgfpathlineto{\pgfqpoint{4.086109in}{2.642006in}}%
\pgfpathlineto{\pgfqpoint{4.096345in}{2.632929in}}%
\pgfpathlineto{\pgfqpoint{4.106581in}{2.626290in}}%
\pgfpathlineto{\pgfqpoint{4.116818in}{2.622112in}}%
\pgfpathlineto{\pgfqpoint{4.127054in}{2.620407in}}%
\pgfpathlineto{\pgfqpoint{4.137290in}{2.621182in}}%
\pgfpathlineto{\pgfqpoint{4.147527in}{2.624434in}}%
\pgfpathlineto{\pgfqpoint{4.157763in}{2.630152in}}%
\pgfpathlineto{\pgfqpoint{4.167999in}{2.638318in}}%
\pgfpathlineto{\pgfqpoint{4.178235in}{2.648905in}}%
\pgfpathlineto{\pgfqpoint{4.188472in}{2.661879in}}%
\pgfpathlineto{\pgfqpoint{4.203826in}{2.685721in}}%
\pgfpathlineto{\pgfqpoint{4.219181in}{2.714663in}}%
\pgfpathlineto{\pgfqpoint{4.234535in}{2.748493in}}%
\pgfpathlineto{\pgfqpoint{4.249890in}{2.786965in}}%
\pgfpathlineto{\pgfqpoint{4.270362in}{2.844990in}}%
\pgfpathlineto{\pgfqpoint{4.290835in}{2.910010in}}%
\pgfpathlineto{\pgfqpoint{4.316426in}{2.999829in}}%
\pgfpathlineto{\pgfqpoint{4.347134in}{3.117716in}}%
\pgfpathlineto{\pgfqpoint{4.382962in}{3.264810in}}%
\pgfpathlineto{\pgfqpoint{4.480206in}{3.670568in}}%
\pgfpathlineto{\pgfqpoint{4.510915in}{3.786819in}}%
\pgfpathlineto{\pgfqpoint{4.536506in}{3.874856in}}%
\pgfpathlineto{\pgfqpoint{4.556979in}{3.938203in}}%
\pgfpathlineto{\pgfqpoint{4.577451in}{3.994358in}}%
\pgfpathlineto{\pgfqpoint{4.592806in}{4.031312in}}%
\pgfpathlineto{\pgfqpoint{4.608160in}{4.063538in}}%
\pgfpathlineto{\pgfqpoint{4.623515in}{4.090802in}}%
\pgfpathlineto{\pgfqpoint{4.638869in}{4.112904in}}%
\pgfpathlineto{\pgfqpoint{4.649105in}{4.124689in}}%
\pgfpathlineto{\pgfqpoint{4.659342in}{4.134069in}}%
\pgfpathlineto{\pgfqpoint{4.669578in}{4.141014in}}%
\pgfpathlineto{\pgfqpoint{4.679814in}{4.145501in}}%
\pgfpathlineto{\pgfqpoint{4.690051in}{4.147515in}}%
\pgfpathlineto{\pgfqpoint{4.700287in}{4.147050in}}%
\pgfpathlineto{\pgfqpoint{4.710523in}{4.144108in}}%
\pgfpathlineto{\pgfqpoint{4.720759in}{4.138697in}}%
\pgfpathlineto{\pgfqpoint{4.730996in}{4.130836in}}%
\pgfpathlineto{\pgfqpoint{4.741232in}{4.120550in}}%
\pgfpathlineto{\pgfqpoint{4.751468in}{4.107872in}}%
\pgfpathlineto{\pgfqpoint{4.766823in}{4.084463in}}%
\pgfpathlineto{\pgfqpoint{4.782177in}{4.055939in}}%
\pgfpathlineto{\pgfqpoint{4.797532in}{4.022507in}}%
\pgfpathlineto{\pgfqpoint{4.812886in}{3.984412in}}%
\pgfpathlineto{\pgfqpoint{4.833359in}{3.926850in}}%
\pgfpathlineto{\pgfqpoint{4.853831in}{3.862243in}}%
\pgfpathlineto{\pgfqpoint{4.879422in}{3.772863in}}%
\pgfpathlineto{\pgfqpoint{4.910131in}{3.655376in}}%
\pgfpathlineto{\pgfqpoint{4.945958in}{3.508561in}}%
\pgfpathlineto{\pgfqpoint{5.048321in}{3.082372in}}%
\pgfpathlineto{\pgfqpoint{5.079030in}{2.967410in}}%
\pgfpathlineto{\pgfqpoint{5.104621in}{2.880755in}}%
\pgfpathlineto{\pgfqpoint{5.125093in}{2.818697in}}%
\pgfpathlineto{\pgfqpoint{5.145566in}{2.763975in}}%
\pgfpathlineto{\pgfqpoint{5.160920in}{2.728178in}}%
\pgfpathlineto{\pgfqpoint{5.176275in}{2.697172in}}%
\pgfpathlineto{\pgfqpoint{5.191629in}{2.671181in}}%
\pgfpathlineto{\pgfqpoint{5.206984in}{2.650397in}}%
\pgfpathlineto{\pgfqpoint{5.217220in}{2.639510in}}%
\pgfpathlineto{\pgfqpoint{5.227456in}{2.631039in}}%
\pgfpathlineto{\pgfqpoint{5.237693in}{2.625014in}}%
\pgfpathlineto{\pgfqpoint{5.247929in}{2.621453in}}%
\pgfpathlineto{\pgfqpoint{5.258165in}{2.620368in}}%
\pgfpathlineto{\pgfqpoint{5.268402in}{2.621763in}}%
\pgfpathlineto{\pgfqpoint{5.278638in}{2.625633in}}%
\pgfpathlineto{\pgfqpoint{5.288874in}{2.631965in}}%
\pgfpathlineto{\pgfqpoint{5.299111in}{2.640739in}}%
\pgfpathlineto{\pgfqpoint{5.309347in}{2.651926in}}%
\pgfpathlineto{\pgfqpoint{5.324701in}{2.673151in}}%
\pgfpathlineto{\pgfqpoint{5.340056in}{2.699567in}}%
\pgfpathlineto{\pgfqpoint{5.355410in}{2.730982in}}%
\pgfpathlineto{\pgfqpoint{5.370765in}{2.767166in}}%
\pgfpathlineto{\pgfqpoint{5.391237in}{2.822368in}}%
\pgfpathlineto{\pgfqpoint{5.411710in}{2.884860in}}%
\pgfpathlineto{\pgfqpoint{5.437301in}{2.971979in}}%
\pgfpathlineto{\pgfqpoint{5.462891in}{3.067449in}}%
\pgfpathlineto{\pgfqpoint{5.493600in}{3.190300in}}%
\pgfpathlineto{\pgfqpoint{5.534545in}{3.362243in}}%
\pgfpathlineto{\pgfqpoint{5.534545in}{3.362243in}}%
\pgfusepath{stroke}%
\end{pgfscope}%
\begin{pgfscope}%
\pgfpathrectangle{\pgfqpoint{0.800000in}{2.544000in}}{\pgfqpoint{4.960000in}{1.680000in}}%
\pgfusepath{clip}%
\pgfsetrectcap%
\pgfsetroundjoin%
\pgfsetlinewidth{1.505625pt}%
\definecolor{currentstroke}{rgb}{1.000000,0.498039,0.054902}%
\pgfsetstrokecolor{currentstroke}%
\pgfsetdash{}{0pt}%
\pgfpathmoveto{\pgfqpoint{1.025455in}{3.384000in}}%
\pgfpathlineto{\pgfqpoint{1.061282in}{3.697379in}}%
\pgfpathlineto{\pgfqpoint{1.081754in}{3.854913in}}%
\pgfpathlineto{\pgfqpoint{1.097109in}{3.955551in}}%
\pgfpathlineto{\pgfqpoint{1.107345in}{4.012422in}}%
\pgfpathlineto{\pgfqpoint{1.117581in}{4.060131in}}%
\pgfpathlineto{\pgfqpoint{1.127818in}{4.097983in}}%
\pgfpathlineto{\pgfqpoint{1.138054in}{4.125424in}}%
\pgfpathlineto{\pgfqpoint{1.143172in}{4.135111in}}%
\pgfpathlineto{\pgfqpoint{1.148290in}{4.142056in}}%
\pgfpathlineto{\pgfqpoint{1.153408in}{4.146237in}}%
\pgfpathlineto{\pgfqpoint{1.158526in}{4.147636in}}%
\pgfpathlineto{\pgfqpoint{1.163645in}{4.146250in}}%
\pgfpathlineto{\pgfqpoint{1.168763in}{4.142083in}}%
\pgfpathlineto{\pgfqpoint{1.173881in}{4.135150in}}%
\pgfpathlineto{\pgfqpoint{1.178999in}{4.125477in}}%
\pgfpathlineto{\pgfqpoint{1.189235in}{4.098060in}}%
\pgfpathlineto{\pgfqpoint{1.199472in}{4.060233in}}%
\pgfpathlineto{\pgfqpoint{1.209708in}{4.012546in}}%
\pgfpathlineto{\pgfqpoint{1.219944in}{3.955695in}}%
\pgfpathlineto{\pgfqpoint{1.235299in}{3.855084in}}%
\pgfpathlineto{\pgfqpoint{1.250653in}{3.739043in}}%
\pgfpathlineto{\pgfqpoint{1.271126in}{3.566945in}}%
\pgfpathlineto{\pgfqpoint{1.337662in}{2.989160in}}%
\pgfpathlineto{\pgfqpoint{1.353016in}{2.877816in}}%
\pgfpathlineto{\pgfqpoint{1.368371in}{2.783051in}}%
\pgfpathlineto{\pgfqpoint{1.378607in}{2.730644in}}%
\pgfpathlineto{\pgfqpoint{1.388843in}{2.687762in}}%
\pgfpathlineto{\pgfqpoint{1.399080in}{2.655031in}}%
\pgfpathlineto{\pgfqpoint{1.409316in}{2.632929in}}%
\pgfpathlineto{\pgfqpoint{1.414434in}{2.625970in}}%
\pgfpathlineto{\pgfqpoint{1.419552in}{2.621776in}}%
\pgfpathlineto{\pgfqpoint{1.424670in}{2.620364in}}%
\pgfpathlineto{\pgfqpoint{1.429788in}{2.621737in}}%
\pgfpathlineto{\pgfqpoint{1.434907in}{2.625891in}}%
\pgfpathlineto{\pgfqpoint{1.440025in}{2.632811in}}%
\pgfpathlineto{\pgfqpoint{1.445143in}{2.642471in}}%
\pgfpathlineto{\pgfqpoint{1.455379in}{2.669863in}}%
\pgfpathlineto{\pgfqpoint{1.465616in}{2.707666in}}%
\pgfpathlineto{\pgfqpoint{1.475852in}{2.755330in}}%
\pgfpathlineto{\pgfqpoint{1.486088in}{2.812160in}}%
\pgfpathlineto{\pgfqpoint{1.501443in}{2.912745in}}%
\pgfpathlineto{\pgfqpoint{1.516797in}{3.028765in}}%
\pgfpathlineto{\pgfqpoint{1.537270in}{3.200844in}}%
\pgfpathlineto{\pgfqpoint{1.603806in}{3.778653in}}%
\pgfpathlineto{\pgfqpoint{1.619160in}{3.890021in}}%
\pgfpathlineto{\pgfqpoint{1.634514in}{3.984815in}}%
\pgfpathlineto{\pgfqpoint{1.644751in}{4.037244in}}%
\pgfpathlineto{\pgfqpoint{1.654987in}{4.080148in}}%
\pgfpathlineto{\pgfqpoint{1.665223in}{4.112904in}}%
\pgfpathlineto{\pgfqpoint{1.675460in}{4.135032in}}%
\pgfpathlineto{\pgfqpoint{1.680578in}{4.142004in}}%
\pgfpathlineto{\pgfqpoint{1.685696in}{4.146210in}}%
\pgfpathlineto{\pgfqpoint{1.690814in}{4.147636in}}%
\pgfpathlineto{\pgfqpoint{1.695932in}{4.146276in}}%
\pgfpathlineto{\pgfqpoint{1.701050in}{4.142135in}}%
\pgfpathlineto{\pgfqpoint{1.706169in}{4.135228in}}%
\pgfpathlineto{\pgfqpoint{1.711287in}{4.125580in}}%
\pgfpathlineto{\pgfqpoint{1.721523in}{4.098214in}}%
\pgfpathlineto{\pgfqpoint{1.731759in}{4.060435in}}%
\pgfpathlineto{\pgfqpoint{1.741996in}{4.012793in}}%
\pgfpathlineto{\pgfqpoint{1.752232in}{3.955984in}}%
\pgfpathlineto{\pgfqpoint{1.767586in}{3.855426in}}%
\pgfpathlineto{\pgfqpoint{1.782941in}{3.739428in}}%
\pgfpathlineto{\pgfqpoint{1.803413in}{3.567368in}}%
\pgfpathlineto{\pgfqpoint{1.869949in}{2.989533in}}%
\pgfpathlineto{\pgfqpoint{1.885304in}{2.878142in}}%
\pgfpathlineto{\pgfqpoint{1.900658in}{2.783320in}}%
\pgfpathlineto{\pgfqpoint{1.910895in}{2.730869in}}%
\pgfpathlineto{\pgfqpoint{1.921131in}{2.687941in}}%
\pgfpathlineto{\pgfqpoint{1.931367in}{2.655161in}}%
\pgfpathlineto{\pgfqpoint{1.941604in}{2.633007in}}%
\pgfpathlineto{\pgfqpoint{1.946722in}{2.626022in}}%
\pgfpathlineto{\pgfqpoint{1.951840in}{2.621803in}}%
\pgfpathlineto{\pgfqpoint{1.956958in}{2.620364in}}%
\pgfpathlineto{\pgfqpoint{1.962076in}{2.621711in}}%
\pgfpathlineto{\pgfqpoint{1.967194in}{2.625839in}}%
\pgfpathlineto{\pgfqpoint{1.972312in}{2.632733in}}%
\pgfpathlineto{\pgfqpoint{1.977431in}{2.642368in}}%
\pgfpathlineto{\pgfqpoint{1.987667in}{2.669709in}}%
\pgfpathlineto{\pgfqpoint{1.997903in}{2.707464in}}%
\pgfpathlineto{\pgfqpoint{2.008140in}{2.755083in}}%
\pgfpathlineto{\pgfqpoint{2.018376in}{2.811872in}}%
\pgfpathlineto{\pgfqpoint{2.033730in}{2.912402in}}%
\pgfpathlineto{\pgfqpoint{2.049085in}{3.028380in}}%
\pgfpathlineto{\pgfqpoint{2.069557in}{3.200421in}}%
\pgfpathlineto{\pgfqpoint{2.136093in}{3.778281in}}%
\pgfpathlineto{\pgfqpoint{2.151448in}{3.889695in}}%
\pgfpathlineto{\pgfqpoint{2.166802in}{3.984546in}}%
\pgfpathlineto{\pgfqpoint{2.177038in}{4.037018in}}%
\pgfpathlineto{\pgfqpoint{2.187275in}{4.079969in}}%
\pgfpathlineto{\pgfqpoint{2.197511in}{4.112774in}}%
\pgfpathlineto{\pgfqpoint{2.207747in}{4.134953in}}%
\pgfpathlineto{\pgfqpoint{2.212866in}{4.141951in}}%
\pgfpathlineto{\pgfqpoint{2.217984in}{4.146184in}}%
\pgfpathlineto{\pgfqpoint{2.223102in}{4.147636in}}%
\pgfpathlineto{\pgfqpoint{2.228220in}{4.146302in}}%
\pgfpathlineto{\pgfqpoint{2.233338in}{4.142187in}}%
\pgfpathlineto{\pgfqpoint{2.238456in}{4.135306in}}%
\pgfpathlineto{\pgfqpoint{2.243574in}{4.125684in}}%
\pgfpathlineto{\pgfqpoint{2.253811in}{4.098368in}}%
\pgfpathlineto{\pgfqpoint{2.264047in}{4.060636in}}%
\pgfpathlineto{\pgfqpoint{2.274283in}{4.013040in}}%
\pgfpathlineto{\pgfqpoint{2.284520in}{3.956272in}}%
\pgfpathlineto{\pgfqpoint{2.299874in}{3.855769in}}%
\pgfpathlineto{\pgfqpoint{2.315229in}{3.739813in}}%
\pgfpathlineto{\pgfqpoint{2.335701in}{3.567790in}}%
\pgfpathlineto{\pgfqpoint{2.402237in}{2.989906in}}%
\pgfpathlineto{\pgfqpoint{2.417592in}{2.878468in}}%
\pgfpathlineto{\pgfqpoint{2.432946in}{2.783588in}}%
\pgfpathlineto{\pgfqpoint{2.443182in}{2.731095in}}%
\pgfpathlineto{\pgfqpoint{2.453419in}{2.688120in}}%
\pgfpathlineto{\pgfqpoint{2.463655in}{2.655291in}}%
\pgfpathlineto{\pgfqpoint{2.473891in}{2.633086in}}%
\pgfpathlineto{\pgfqpoint{2.479009in}{2.626076in}}%
\pgfpathlineto{\pgfqpoint{2.484128in}{2.621830in}}%
\pgfpathlineto{\pgfqpoint{2.489246in}{2.620365in}}%
\pgfpathlineto{\pgfqpoint{2.494364in}{2.621685in}}%
\pgfpathlineto{\pgfqpoint{2.499482in}{2.625787in}}%
\pgfpathlineto{\pgfqpoint{2.504600in}{2.632655in}}%
\pgfpathlineto{\pgfqpoint{2.509718in}{2.642264in}}%
\pgfpathlineto{\pgfqpoint{2.519955in}{2.669555in}}%
\pgfpathlineto{\pgfqpoint{2.530191in}{2.707263in}}%
\pgfpathlineto{\pgfqpoint{2.540427in}{2.754837in}}%
\pgfpathlineto{\pgfqpoint{2.550664in}{2.811584in}}%
\pgfpathlineto{\pgfqpoint{2.566018in}{2.912060in}}%
\pgfpathlineto{\pgfqpoint{2.581372in}{3.027995in}}%
\pgfpathlineto{\pgfqpoint{2.601845in}{3.199999in}}%
\pgfpathlineto{\pgfqpoint{2.668381in}{3.777908in}}%
\pgfpathlineto{\pgfqpoint{2.683735in}{3.889369in}}%
\pgfpathlineto{\pgfqpoint{2.699090in}{3.984277in}}%
\pgfpathlineto{\pgfqpoint{2.709326in}{4.036792in}}%
\pgfpathlineto{\pgfqpoint{2.719562in}{4.079790in}}%
\pgfpathlineto{\pgfqpoint{2.729799in}{4.112644in}}%
\pgfpathlineto{\pgfqpoint{2.740035in}{4.134874in}}%
\pgfpathlineto{\pgfqpoint{2.745153in}{4.141898in}}%
\pgfpathlineto{\pgfqpoint{2.750271in}{4.146157in}}%
\pgfpathlineto{\pgfqpoint{2.755390in}{4.147635in}}%
\pgfpathlineto{\pgfqpoint{2.760508in}{4.146327in}}%
\pgfpathlineto{\pgfqpoint{2.765626in}{4.142239in}}%
\pgfpathlineto{\pgfqpoint{2.770744in}{4.135384in}}%
\pgfpathlineto{\pgfqpoint{2.775862in}{4.125788in}}%
\pgfpathlineto{\pgfqpoint{2.786098in}{4.098522in}}%
\pgfpathlineto{\pgfqpoint{2.796335in}{4.060838in}}%
\pgfpathlineto{\pgfqpoint{2.806571in}{4.013287in}}%
\pgfpathlineto{\pgfqpoint{2.816807in}{3.956560in}}%
\pgfpathlineto{\pgfqpoint{2.832162in}{3.856111in}}%
\pgfpathlineto{\pgfqpoint{2.847516in}{3.740198in}}%
\pgfpathlineto{\pgfqpoint{2.867989in}{3.568212in}}%
\pgfpathlineto{\pgfqpoint{2.934525in}{2.990279in}}%
\pgfpathlineto{\pgfqpoint{2.949879in}{2.878794in}}%
\pgfpathlineto{\pgfqpoint{2.965234in}{2.783857in}}%
\pgfpathlineto{\pgfqpoint{2.975470in}{2.731320in}}%
\pgfpathlineto{\pgfqpoint{2.985706in}{2.688299in}}%
\pgfpathlineto{\pgfqpoint{2.995943in}{2.655421in}}%
\pgfpathlineto{\pgfqpoint{3.006179in}{2.633166in}}%
\pgfpathlineto{\pgfqpoint{3.011297in}{2.626129in}}%
\pgfpathlineto{\pgfqpoint{3.016415in}{2.621857in}}%
\pgfpathlineto{\pgfqpoint{3.021533in}{2.620365in}}%
\pgfpathlineto{\pgfqpoint{3.026652in}{2.621660in}}%
\pgfpathlineto{\pgfqpoint{3.031770in}{2.625735in}}%
\pgfpathlineto{\pgfqpoint{3.036888in}{2.632577in}}%
\pgfpathlineto{\pgfqpoint{3.042006in}{2.642161in}}%
\pgfpathlineto{\pgfqpoint{3.052242in}{2.669402in}}%
\pgfpathlineto{\pgfqpoint{3.062479in}{2.707061in}}%
\pgfpathlineto{\pgfqpoint{3.072715in}{2.754590in}}%
\pgfpathlineto{\pgfqpoint{3.082951in}{2.811296in}}%
\pgfpathlineto{\pgfqpoint{3.098306in}{2.911718in}}%
\pgfpathlineto{\pgfqpoint{3.113660in}{3.027610in}}%
\pgfpathlineto{\pgfqpoint{3.134133in}{3.199576in}}%
\pgfpathlineto{\pgfqpoint{3.200669in}{3.777535in}}%
\pgfpathlineto{\pgfqpoint{3.216023in}{3.889043in}}%
\pgfpathlineto{\pgfqpoint{3.231378in}{3.984008in}}%
\pgfpathlineto{\pgfqpoint{3.241614in}{4.036566in}}%
\pgfpathlineto{\pgfqpoint{3.251850in}{4.079611in}}%
\pgfpathlineto{\pgfqpoint{3.262086in}{4.112513in}}%
\pgfpathlineto{\pgfqpoint{3.272323in}{4.134795in}}%
\pgfpathlineto{\pgfqpoint{3.277441in}{4.141844in}}%
\pgfpathlineto{\pgfqpoint{3.282559in}{4.146130in}}%
\pgfpathlineto{\pgfqpoint{3.287677in}{4.147634in}}%
\pgfpathlineto{\pgfqpoint{3.292795in}{4.146353in}}%
\pgfpathlineto{\pgfqpoint{3.297914in}{4.142290in}}%
\pgfpathlineto{\pgfqpoint{3.303032in}{4.135461in}}%
\pgfpathlineto{\pgfqpoint{3.308150in}{4.125891in}}%
\pgfpathlineto{\pgfqpoint{3.318386in}{4.098675in}}%
\pgfpathlineto{\pgfqpoint{3.328622in}{4.061039in}}%
\pgfpathlineto{\pgfqpoint{3.338859in}{4.013533in}}%
\pgfpathlineto{\pgfqpoint{3.349095in}{3.956848in}}%
\pgfpathlineto{\pgfqpoint{3.364449in}{3.856453in}}%
\pgfpathlineto{\pgfqpoint{3.379804in}{3.740583in}}%
\pgfpathlineto{\pgfqpoint{3.400277in}{3.568635in}}%
\pgfpathlineto{\pgfqpoint{3.466813in}{2.990652in}}%
\pgfpathlineto{\pgfqpoint{3.482167in}{2.879121in}}%
\pgfpathlineto{\pgfqpoint{3.497521in}{2.784127in}}%
\pgfpathlineto{\pgfqpoint{3.507758in}{2.731546in}}%
\pgfpathlineto{\pgfqpoint{3.517994in}{2.688479in}}%
\pgfpathlineto{\pgfqpoint{3.528230in}{2.655552in}}%
\pgfpathlineto{\pgfqpoint{3.538467in}{2.633245in}}%
\pgfpathlineto{\pgfqpoint{3.543585in}{2.626182in}}%
\pgfpathlineto{\pgfqpoint{3.548703in}{2.621884in}}%
\pgfpathlineto{\pgfqpoint{3.553821in}{2.620366in}}%
\pgfpathlineto{\pgfqpoint{3.558939in}{2.621635in}}%
\pgfpathlineto{\pgfqpoint{3.564057in}{2.625684in}}%
\pgfpathlineto{\pgfqpoint{3.569176in}{2.632500in}}%
\pgfpathlineto{\pgfqpoint{3.574294in}{2.642057in}}%
\pgfpathlineto{\pgfqpoint{3.584530in}{2.669248in}}%
\pgfpathlineto{\pgfqpoint{3.594766in}{2.706860in}}%
\pgfpathlineto{\pgfqpoint{3.605003in}{2.754344in}}%
\pgfpathlineto{\pgfqpoint{3.615239in}{2.811008in}}%
\pgfpathlineto{\pgfqpoint{3.630593in}{2.911376in}}%
\pgfpathlineto{\pgfqpoint{3.645948in}{3.027225in}}%
\pgfpathlineto{\pgfqpoint{3.666420in}{3.199154in}}%
\pgfpathlineto{\pgfqpoint{3.732956in}{3.777162in}}%
\pgfpathlineto{\pgfqpoint{3.748311in}{3.888716in}}%
\pgfpathlineto{\pgfqpoint{3.763665in}{3.983739in}}%
\pgfpathlineto{\pgfqpoint{3.773902in}{4.036340in}}%
\pgfpathlineto{\pgfqpoint{3.784138in}{4.079431in}}%
\pgfpathlineto{\pgfqpoint{3.794374in}{4.112383in}}%
\pgfpathlineto{\pgfqpoint{3.804610in}{4.134715in}}%
\pgfpathlineto{\pgfqpoint{3.809729in}{4.141791in}}%
\pgfpathlineto{\pgfqpoint{3.814847in}{4.146102in}}%
\pgfpathlineto{\pgfqpoint{3.819965in}{4.147633in}}%
\pgfpathlineto{\pgfqpoint{3.825083in}{4.146378in}}%
\pgfpathlineto{\pgfqpoint{3.830201in}{4.142342in}}%
\pgfpathlineto{\pgfqpoint{3.835319in}{4.135539in}}%
\pgfpathlineto{\pgfqpoint{3.840438in}{4.125994in}}%
\pgfpathlineto{\pgfqpoint{3.850674in}{4.098828in}}%
\pgfpathlineto{\pgfqpoint{3.860910in}{4.061241in}}%
\pgfpathlineto{\pgfqpoint{3.871146in}{4.013779in}}%
\pgfpathlineto{\pgfqpoint{3.881383in}{3.957136in}}%
\pgfpathlineto{\pgfqpoint{3.896737in}{3.856795in}}%
\pgfpathlineto{\pgfqpoint{3.912092in}{3.740968in}}%
\pgfpathlineto{\pgfqpoint{3.932564in}{3.569057in}}%
\pgfpathlineto{\pgfqpoint{3.999100in}{2.991025in}}%
\pgfpathlineto{\pgfqpoint{4.014455in}{2.879447in}}%
\pgfpathlineto{\pgfqpoint{4.029809in}{2.784396in}}%
\pgfpathlineto{\pgfqpoint{4.040045in}{2.731773in}}%
\pgfpathlineto{\pgfqpoint{4.050282in}{2.688659in}}%
\pgfpathlineto{\pgfqpoint{4.060518in}{2.655682in}}%
\pgfpathlineto{\pgfqpoint{4.070754in}{2.633325in}}%
\pgfpathlineto{\pgfqpoint{4.075872in}{2.626236in}}%
\pgfpathlineto{\pgfqpoint{4.080991in}{2.621912in}}%
\pgfpathlineto{\pgfqpoint{4.086109in}{2.620368in}}%
\pgfpathlineto{\pgfqpoint{4.091227in}{2.621610in}}%
\pgfpathlineto{\pgfqpoint{4.096345in}{2.625633in}}%
\pgfpathlineto{\pgfqpoint{4.101463in}{2.632423in}}%
\pgfpathlineto{\pgfqpoint{4.106581in}{2.641955in}}%
\pgfpathlineto{\pgfqpoint{4.116818in}{2.669095in}}%
\pgfpathlineto{\pgfqpoint{4.127054in}{2.706659in}}%
\pgfpathlineto{\pgfqpoint{4.137290in}{2.754098in}}%
\pgfpathlineto{\pgfqpoint{4.147527in}{2.810720in}}%
\pgfpathlineto{\pgfqpoint{4.162881in}{2.911034in}}%
\pgfpathlineto{\pgfqpoint{4.178235in}{3.026840in}}%
\pgfpathlineto{\pgfqpoint{4.198708in}{3.198732in}}%
\pgfpathlineto{\pgfqpoint{4.265244in}{3.776789in}}%
\pgfpathlineto{\pgfqpoint{4.280598in}{3.888389in}}%
\pgfpathlineto{\pgfqpoint{4.295953in}{3.983469in}}%
\pgfpathlineto{\pgfqpoint{4.306189in}{4.036114in}}%
\pgfpathlineto{\pgfqpoint{4.316426in}{4.079251in}}%
\pgfpathlineto{\pgfqpoint{4.326662in}{4.112252in}}%
\pgfpathlineto{\pgfqpoint{4.336898in}{4.134635in}}%
\pgfpathlineto{\pgfqpoint{4.342016in}{4.141737in}}%
\pgfpathlineto{\pgfqpoint{4.347134in}{4.146074in}}%
\pgfpathlineto{\pgfqpoint{4.352253in}{4.147632in}}%
\pgfpathlineto{\pgfqpoint{4.357371in}{4.146403in}}%
\pgfpathlineto{\pgfqpoint{4.362489in}{4.142393in}}%
\pgfpathlineto{\pgfqpoint{4.367607in}{4.135616in}}%
\pgfpathlineto{\pgfqpoint{4.372725in}{4.126097in}}%
\pgfpathlineto{\pgfqpoint{4.382962in}{4.098981in}}%
\pgfpathlineto{\pgfqpoint{4.393198in}{4.061442in}}%
\pgfpathlineto{\pgfqpoint{4.403434in}{4.014025in}}%
\pgfpathlineto{\pgfqpoint{4.413670in}{3.957423in}}%
\pgfpathlineto{\pgfqpoint{4.429025in}{3.857136in}}%
\pgfpathlineto{\pgfqpoint{4.444379in}{3.741352in}}%
\pgfpathlineto{\pgfqpoint{4.464852in}{3.569479in}}%
\pgfpathlineto{\pgfqpoint{4.531388in}{2.991398in}}%
\pgfpathlineto{\pgfqpoint{4.546742in}{2.879774in}}%
\pgfpathlineto{\pgfqpoint{4.562097in}{2.784666in}}%
\pgfpathlineto{\pgfqpoint{4.572333in}{2.731999in}}%
\pgfpathlineto{\pgfqpoint{4.582569in}{2.688839in}}%
\pgfpathlineto{\pgfqpoint{4.592806in}{2.655813in}}%
\pgfpathlineto{\pgfqpoint{4.603042in}{2.633405in}}%
\pgfpathlineto{\pgfqpoint{4.608160in}{2.626290in}}%
\pgfpathlineto{\pgfqpoint{4.613278in}{2.621940in}}%
\pgfpathlineto{\pgfqpoint{4.618396in}{2.620369in}}%
\pgfpathlineto{\pgfqpoint{4.623515in}{2.621585in}}%
\pgfpathlineto{\pgfqpoint{4.628633in}{2.625582in}}%
\pgfpathlineto{\pgfqpoint{4.633751in}{2.632346in}}%
\pgfpathlineto{\pgfqpoint{4.638869in}{2.641852in}}%
\pgfpathlineto{\pgfqpoint{4.649105in}{2.668942in}}%
\pgfpathlineto{\pgfqpoint{4.659342in}{2.706458in}}%
\pgfpathlineto{\pgfqpoint{4.669578in}{2.753852in}}%
\pgfpathlineto{\pgfqpoint{4.679814in}{2.810433in}}%
\pgfpathlineto{\pgfqpoint{4.695169in}{2.910693in}}%
\pgfpathlineto{\pgfqpoint{4.710523in}{3.026455in}}%
\pgfpathlineto{\pgfqpoint{4.730996in}{3.198310in}}%
\pgfpathlineto{\pgfqpoint{4.797532in}{3.776415in}}%
\pgfpathlineto{\pgfqpoint{4.812886in}{3.888063in}}%
\pgfpathlineto{\pgfqpoint{4.828241in}{3.983199in}}%
\pgfpathlineto{\pgfqpoint{4.838477in}{4.035887in}}%
\pgfpathlineto{\pgfqpoint{4.848713in}{4.079071in}}%
\pgfpathlineto{\pgfqpoint{4.858950in}{4.112121in}}%
\pgfpathlineto{\pgfqpoint{4.869186in}{4.134555in}}%
\pgfpathlineto{\pgfqpoint{4.874304in}{4.141683in}}%
\pgfpathlineto{\pgfqpoint{4.879422in}{4.146046in}}%
\pgfpathlineto{\pgfqpoint{4.884540in}{4.147630in}}%
\pgfpathlineto{\pgfqpoint{4.889658in}{4.146427in}}%
\pgfpathlineto{\pgfqpoint{4.894777in}{4.142443in}}%
\pgfpathlineto{\pgfqpoint{4.899895in}{4.135692in}}%
\pgfpathlineto{\pgfqpoint{4.905013in}{4.126199in}}%
\pgfpathlineto{\pgfqpoint{4.915249in}{4.099134in}}%
\pgfpathlineto{\pgfqpoint{4.925486in}{4.061642in}}%
\pgfpathlineto{\pgfqpoint{4.935722in}{4.014271in}}%
\pgfpathlineto{\pgfqpoint{4.945958in}{3.957711in}}%
\pgfpathlineto{\pgfqpoint{4.961313in}{3.857478in}}%
\pgfpathlineto{\pgfqpoint{4.976667in}{3.741737in}}%
\pgfpathlineto{\pgfqpoint{4.997140in}{3.569901in}}%
\pgfpathlineto{\pgfqpoint{5.063676in}{2.991771in}}%
\pgfpathlineto{\pgfqpoint{5.079030in}{2.880101in}}%
\pgfpathlineto{\pgfqpoint{5.094384in}{2.784935in}}%
\pgfpathlineto{\pgfqpoint{5.104621in}{2.732226in}}%
\pgfpathlineto{\pgfqpoint{5.114857in}{2.689019in}}%
\pgfpathlineto{\pgfqpoint{5.125093in}{2.655945in}}%
\pgfpathlineto{\pgfqpoint{5.135330in}{2.633485in}}%
\pgfpathlineto{\pgfqpoint{5.140448in}{2.626344in}}%
\pgfpathlineto{\pgfqpoint{5.145566in}{2.621968in}}%
\pgfpathlineto{\pgfqpoint{5.150684in}{2.620371in}}%
\pgfpathlineto{\pgfqpoint{5.155802in}{2.621560in}}%
\pgfpathlineto{\pgfqpoint{5.160920in}{2.625531in}}%
\pgfpathlineto{\pgfqpoint{5.166039in}{2.632269in}}%
\pgfpathlineto{\pgfqpoint{5.171157in}{2.641750in}}%
\pgfpathlineto{\pgfqpoint{5.181393in}{2.668790in}}%
\pgfpathlineto{\pgfqpoint{5.191629in}{2.706257in}}%
\pgfpathlineto{\pgfqpoint{5.201866in}{2.753606in}}%
\pgfpathlineto{\pgfqpoint{5.212102in}{2.810146in}}%
\pgfpathlineto{\pgfqpoint{5.227456in}{2.910351in}}%
\pgfpathlineto{\pgfqpoint{5.242811in}{3.026071in}}%
\pgfpathlineto{\pgfqpoint{5.263283in}{3.197888in}}%
\pgfpathlineto{\pgfqpoint{5.329819in}{3.776042in}}%
\pgfpathlineto{\pgfqpoint{5.345174in}{3.887736in}}%
\pgfpathlineto{\pgfqpoint{5.360528in}{3.982930in}}%
\pgfpathlineto{\pgfqpoint{5.370765in}{4.035661in}}%
\pgfpathlineto{\pgfqpoint{5.381001in}{4.078891in}}%
\pgfpathlineto{\pgfqpoint{5.391237in}{4.111990in}}%
\pgfpathlineto{\pgfqpoint{5.401474in}{4.134475in}}%
\pgfpathlineto{\pgfqpoint{5.406592in}{4.141628in}}%
\pgfpathlineto{\pgfqpoint{5.411710in}{4.146018in}}%
\pgfpathlineto{\pgfqpoint{5.416828in}{4.147628in}}%
\pgfpathlineto{\pgfqpoint{5.421946in}{4.146452in}}%
\pgfpathlineto{\pgfqpoint{5.427064in}{4.142494in}}%
\pgfpathlineto{\pgfqpoint{5.432182in}{4.135769in}}%
\pgfpathlineto{\pgfqpoint{5.437301in}{4.126301in}}%
\pgfpathlineto{\pgfqpoint{5.447537in}{4.099286in}}%
\pgfpathlineto{\pgfqpoint{5.457773in}{4.061843in}}%
\pgfpathlineto{\pgfqpoint{5.468009in}{4.014517in}}%
\pgfpathlineto{\pgfqpoint{5.478246in}{3.957998in}}%
\pgfpathlineto{\pgfqpoint{5.493600in}{3.857819in}}%
\pgfpathlineto{\pgfqpoint{5.508955in}{3.742121in}}%
\pgfpathlineto{\pgfqpoint{5.529427in}{3.570323in}}%
\pgfpathlineto{\pgfqpoint{5.534545in}{3.525274in}}%
\pgfpathlineto{\pgfqpoint{5.534545in}{3.525274in}}%
\pgfusepath{stroke}%
\end{pgfscope}%
\begin{pgfscope}%
\pgfpathrectangle{\pgfqpoint{0.800000in}{2.544000in}}{\pgfqpoint{4.960000in}{1.680000in}}%
\pgfusepath{clip}%
\pgfsetrectcap%
\pgfsetroundjoin%
\pgfsetlinewidth{1.505625pt}%
\definecolor{currentstroke}{rgb}{0.172549,0.627451,0.172549}%
\pgfsetstrokecolor{currentstroke}%
\pgfsetdash{}{0pt}%
\pgfpathmoveto{\pgfqpoint{1.025455in}{3.384000in}}%
\pgfpathlineto{\pgfqpoint{1.061282in}{3.699462in}}%
\pgfpathlineto{\pgfqpoint{1.081754in}{3.857734in}}%
\pgfpathlineto{\pgfqpoint{1.097109in}{3.958571in}}%
\pgfpathlineto{\pgfqpoint{1.107345in}{4.015375in}}%
\pgfpathlineto{\pgfqpoint{1.117581in}{4.062842in}}%
\pgfpathlineto{\pgfqpoint{1.127818in}{4.100272in}}%
\pgfpathlineto{\pgfqpoint{1.138054in}{4.127111in}}%
\pgfpathlineto{\pgfqpoint{1.143172in}{4.136428in}}%
\pgfpathlineto{\pgfqpoint{1.148290in}{4.142962in}}%
\pgfpathlineto{\pgfqpoint{1.153408in}{4.146687in}}%
\pgfpathlineto{\pgfqpoint{1.158526in}{4.147590in}}%
\pgfpathlineto{\pgfqpoint{1.163645in}{4.145668in}}%
\pgfpathlineto{\pgfqpoint{1.168763in}{4.140928in}}%
\pgfpathlineto{\pgfqpoint{1.173881in}{4.133387in}}%
\pgfpathlineto{\pgfqpoint{1.178999in}{4.123074in}}%
\pgfpathlineto{\pgfqpoint{1.189235in}{4.094291in}}%
\pgfpathlineto{\pgfqpoint{1.199472in}{4.055006in}}%
\pgfpathlineto{\pgfqpoint{1.209708in}{4.005799in}}%
\pgfpathlineto{\pgfqpoint{1.219944in}{3.947398in}}%
\pgfpathlineto{\pgfqpoint{1.235299in}{3.844480in}}%
\pgfpathlineto{\pgfqpoint{1.250653in}{3.726266in}}%
\pgfpathlineto{\pgfqpoint{1.271126in}{3.551697in}}%
\pgfpathlineto{\pgfqpoint{1.332544in}{3.012118in}}%
\pgfpathlineto{\pgfqpoint{1.347898in}{2.897240in}}%
\pgfpathlineto{\pgfqpoint{1.363253in}{2.798531in}}%
\pgfpathlineto{\pgfqpoint{1.373489in}{2.743353in}}%
\pgfpathlineto{\pgfqpoint{1.383725in}{2.697648in}}%
\pgfpathlineto{\pgfqpoint{1.393961in}{2.662092in}}%
\pgfpathlineto{\pgfqpoint{1.404198in}{2.637210in}}%
\pgfpathlineto{\pgfqpoint{1.409316in}{2.628893in}}%
\pgfpathlineto{\pgfqpoint{1.414434in}{2.623370in}}%
\pgfpathlineto{\pgfqpoint{1.419552in}{2.620661in}}%
\pgfpathlineto{\pgfqpoint{1.424670in}{2.620777in}}%
\pgfpathlineto{\pgfqpoint{1.429788in}{2.623717in}}%
\pgfpathlineto{\pgfqpoint{1.434907in}{2.629470in}}%
\pgfpathlineto{\pgfqpoint{1.440025in}{2.638014in}}%
\pgfpathlineto{\pgfqpoint{1.445143in}{2.649319in}}%
\pgfpathlineto{\pgfqpoint{1.455379in}{2.680032in}}%
\pgfpathlineto{\pgfqpoint{1.465616in}{2.721154in}}%
\pgfpathlineto{\pgfqpoint{1.475852in}{2.772076in}}%
\pgfpathlineto{\pgfqpoint{1.491206in}{2.865154in}}%
\pgfpathlineto{\pgfqpoint{1.506561in}{2.975466in}}%
\pgfpathlineto{\pgfqpoint{1.527033in}{3.142959in}}%
\pgfpathlineto{\pgfqpoint{1.562860in}{3.463822in}}%
\pgfpathlineto{\pgfqpoint{1.588451in}{3.687626in}}%
\pgfpathlineto{\pgfqpoint{1.608924in}{3.847512in}}%
\pgfpathlineto{\pgfqpoint{1.624278in}{3.949961in}}%
\pgfpathlineto{\pgfqpoint{1.634514in}{4.008002in}}%
\pgfpathlineto{\pgfqpoint{1.644751in}{4.056815in}}%
\pgfpathlineto{\pgfqpoint{1.654987in}{4.095681in}}%
\pgfpathlineto{\pgfqpoint{1.665223in}{4.124023in}}%
\pgfpathlineto{\pgfqpoint{1.670342in}{4.134110in}}%
\pgfpathlineto{\pgfqpoint{1.675460in}{4.141422in}}%
\pgfpathlineto{\pgfqpoint{1.680578in}{4.145932in}}%
\pgfpathlineto{\pgfqpoint{1.685696in}{4.147623in}}%
\pgfpathlineto{\pgfqpoint{1.690814in}{4.146488in}}%
\pgfpathlineto{\pgfqpoint{1.695932in}{4.142532in}}%
\pgfpathlineto{\pgfqpoint{1.701050in}{4.135769in}}%
\pgfpathlineto{\pgfqpoint{1.706169in}{4.126225in}}%
\pgfpathlineto{\pgfqpoint{1.716405in}{4.098943in}}%
\pgfpathlineto{\pgfqpoint{1.726641in}{4.061090in}}%
\pgfpathlineto{\pgfqpoint{1.736878in}{4.013225in}}%
\pgfpathlineto{\pgfqpoint{1.747114in}{3.956056in}}%
\pgfpathlineto{\pgfqpoint{1.762468in}{3.854741in}}%
\pgfpathlineto{\pgfqpoint{1.777823in}{3.737790in}}%
\pgfpathlineto{\pgfqpoint{1.798295in}{3.564303in}}%
\pgfpathlineto{\pgfqpoint{1.864831in}{2.983217in}}%
\pgfpathlineto{\pgfqpoint{1.880186in}{2.871897in}}%
\pgfpathlineto{\pgfqpoint{1.895540in}{2.777587in}}%
\pgfpathlineto{\pgfqpoint{1.905776in}{2.725739in}}%
\pgfpathlineto{\pgfqpoint{1.916013in}{2.683624in}}%
\pgfpathlineto{\pgfqpoint{1.926249in}{2.651864in}}%
\pgfpathlineto{\pgfqpoint{1.936485in}{2.630931in}}%
\pgfpathlineto{\pgfqpoint{1.941604in}{2.624627in}}%
\pgfpathlineto{\pgfqpoint{1.946722in}{2.621132in}}%
\pgfpathlineto{\pgfqpoint{1.951840in}{2.620461in}}%
\pgfpathlineto{\pgfqpoint{1.956958in}{2.622614in}}%
\pgfpathlineto{\pgfqpoint{1.962076in}{2.627585in}}%
\pgfpathlineto{\pgfqpoint{1.967194in}{2.635354in}}%
\pgfpathlineto{\pgfqpoint{1.972312in}{2.645893in}}%
\pgfpathlineto{\pgfqpoint{1.982549in}{2.675116in}}%
\pgfpathlineto{\pgfqpoint{1.992785in}{2.714820in}}%
\pgfpathlineto{\pgfqpoint{2.003021in}{2.764419in}}%
\pgfpathlineto{\pgfqpoint{2.013258in}{2.823180in}}%
\pgfpathlineto{\pgfqpoint{2.028612in}{2.926564in}}%
\pgfpathlineto{\pgfqpoint{2.043967in}{3.045143in}}%
\pgfpathlineto{\pgfqpoint{2.064439in}{3.220020in}}%
\pgfpathlineto{\pgfqpoint{2.125857in}{3.759204in}}%
\pgfpathlineto{\pgfqpoint{2.141211in}{3.873688in}}%
\pgfpathlineto{\pgfqpoint{2.156566in}{3.971906in}}%
\pgfpathlineto{\pgfqpoint{2.166802in}{4.026711in}}%
\pgfpathlineto{\pgfqpoint{2.177038in}{4.072013in}}%
\pgfpathlineto{\pgfqpoint{2.187275in}{4.107141in}}%
\pgfpathlineto{\pgfqpoint{2.197511in}{4.131577in}}%
\pgfpathlineto{\pgfqpoint{2.202629in}{4.139665in}}%
\pgfpathlineto{\pgfqpoint{2.207747in}{4.144958in}}%
\pgfpathlineto{\pgfqpoint{2.212866in}{4.147435in}}%
\pgfpathlineto{\pgfqpoint{2.217984in}{4.147088in}}%
\pgfpathlineto{\pgfqpoint{2.223102in}{4.143917in}}%
\pgfpathlineto{\pgfqpoint{2.228220in}{4.137935in}}%
\pgfpathlineto{\pgfqpoint{2.233338in}{4.129162in}}%
\pgfpathlineto{\pgfqpoint{2.243574in}{4.103389in}}%
\pgfpathlineto{\pgfqpoint{2.253811in}{4.066979in}}%
\pgfpathlineto{\pgfqpoint{2.264047in}{4.020470in}}%
\pgfpathlineto{\pgfqpoint{2.274283in}{3.964550in}}%
\pgfpathlineto{\pgfqpoint{2.289638in}{3.864868in}}%
\pgfpathlineto{\pgfqpoint{2.304992in}{3.749212in}}%
\pgfpathlineto{\pgfqpoint{2.325465in}{3.576858in}}%
\pgfpathlineto{\pgfqpoint{2.392001in}{2.994295in}}%
\pgfpathlineto{\pgfqpoint{2.407355in}{2.881574in}}%
\pgfpathlineto{\pgfqpoint{2.422710in}{2.785543in}}%
\pgfpathlineto{\pgfqpoint{2.432946in}{2.732396in}}%
\pgfpathlineto{\pgfqpoint{2.443182in}{2.688884in}}%
\pgfpathlineto{\pgfqpoint{2.453419in}{2.655650in}}%
\pgfpathlineto{\pgfqpoint{2.463655in}{2.633185in}}%
\pgfpathlineto{\pgfqpoint{2.468773in}{2.626102in}}%
\pgfpathlineto{\pgfqpoint{2.473891in}{2.621823in}}%
\pgfpathlineto{\pgfqpoint{2.479009in}{2.620364in}}%
\pgfpathlineto{\pgfqpoint{2.484128in}{2.621730in}}%
\pgfpathlineto{\pgfqpoint{2.489246in}{2.625917in}}%
\pgfpathlineto{\pgfqpoint{2.494364in}{2.632909in}}%
\pgfpathlineto{\pgfqpoint{2.499482in}{2.642680in}}%
\pgfpathlineto{\pgfqpoint{2.509718in}{2.670404in}}%
\pgfpathlineto{\pgfqpoint{2.519955in}{2.708679in}}%
\pgfpathlineto{\pgfqpoint{2.530191in}{2.756941in}}%
\pgfpathlineto{\pgfqpoint{2.540427in}{2.814474in}}%
\pgfpathlineto{\pgfqpoint{2.555782in}{2.916263in}}%
\pgfpathlineto{\pgfqpoint{2.571136in}{3.033589in}}%
\pgfpathlineto{\pgfqpoint{2.591609in}{3.207399in}}%
\pgfpathlineto{\pgfqpoint{2.653027in}{3.747874in}}%
\pgfpathlineto{\pgfqpoint{2.668381in}{3.863684in}}%
\pgfpathlineto{\pgfqpoint{2.683735in}{3.963559in}}%
\pgfpathlineto{\pgfqpoint{2.693972in}{4.019627in}}%
\pgfpathlineto{\pgfqpoint{2.704208in}{4.066296in}}%
\pgfpathlineto{\pgfqpoint{2.714444in}{4.102877in}}%
\pgfpathlineto{\pgfqpoint{2.724681in}{4.128828in}}%
\pgfpathlineto{\pgfqpoint{2.729799in}{4.137691in}}%
\pgfpathlineto{\pgfqpoint{2.734917in}{4.143765in}}%
\pgfpathlineto{\pgfqpoint{2.740035in}{4.147029in}}%
\pgfpathlineto{\pgfqpoint{2.745153in}{4.147469in}}%
\pgfpathlineto{\pgfqpoint{2.750271in}{4.145084in}}%
\pgfpathlineto{\pgfqpoint{2.755390in}{4.139883in}}%
\pgfpathlineto{\pgfqpoint{2.760508in}{4.131886in}}%
\pgfpathlineto{\pgfqpoint{2.765626in}{4.121121in}}%
\pgfpathlineto{\pgfqpoint{2.775862in}{4.091459in}}%
\pgfpathlineto{\pgfqpoint{2.786098in}{4.051337in}}%
\pgfpathlineto{\pgfqpoint{2.796335in}{4.001347in}}%
\pgfpathlineto{\pgfqpoint{2.806571in}{3.942229in}}%
\pgfpathlineto{\pgfqpoint{2.821925in}{3.838381in}}%
\pgfpathlineto{\pgfqpoint{2.837280in}{3.719440in}}%
\pgfpathlineto{\pgfqpoint{2.862871in}{3.498567in}}%
\pgfpathlineto{\pgfqpoint{2.914052in}{3.046508in}}%
\pgfpathlineto{\pgfqpoint{2.929407in}{2.927784in}}%
\pgfpathlineto{\pgfqpoint{2.944761in}{2.824215in}}%
\pgfpathlineto{\pgfqpoint{2.960116in}{2.739240in}}%
\pgfpathlineto{\pgfqpoint{2.970352in}{2.694344in}}%
\pgfpathlineto{\pgfqpoint{2.980588in}{2.659644in}}%
\pgfpathlineto{\pgfqpoint{2.990824in}{2.635656in}}%
\pgfpathlineto{\pgfqpoint{2.995943in}{2.627795in}}%
\pgfpathlineto{\pgfqpoint{3.001061in}{2.622733in}}%
\pgfpathlineto{\pgfqpoint{3.006179in}{2.620487in}}%
\pgfpathlineto{\pgfqpoint{3.011297in}{2.621066in}}%
\pgfpathlineto{\pgfqpoint{3.016415in}{2.624468in}}%
\pgfpathlineto{\pgfqpoint{3.021533in}{2.630680in}}%
\pgfpathlineto{\pgfqpoint{3.026652in}{2.639679in}}%
\pgfpathlineto{\pgfqpoint{3.036888in}{2.665897in}}%
\pgfpathlineto{\pgfqpoint{3.047124in}{2.702733in}}%
\pgfpathlineto{\pgfqpoint{3.057360in}{2.749642in}}%
\pgfpathlineto{\pgfqpoint{3.067597in}{2.805931in}}%
\pgfpathlineto{\pgfqpoint{3.082951in}{2.906096in}}%
\pgfpathlineto{\pgfqpoint{3.098306in}{3.022137in}}%
\pgfpathlineto{\pgfqpoint{3.118778in}{3.194829in}}%
\pgfpathlineto{\pgfqpoint{3.185314in}{3.776975in}}%
\pgfpathlineto{\pgfqpoint{3.200669in}{3.889287in}}%
\pgfpathlineto{\pgfqpoint{3.216023in}{3.984815in}}%
\pgfpathlineto{\pgfqpoint{3.226259in}{4.037581in}}%
\pgfpathlineto{\pgfqpoint{3.236496in}{4.080684in}}%
\pgfpathlineto{\pgfqpoint{3.246732in}{4.113485in}}%
\pgfpathlineto{\pgfqpoint{3.256968in}{4.135500in}}%
\pgfpathlineto{\pgfqpoint{3.262086in}{4.142354in}}%
\pgfpathlineto{\pgfqpoint{3.267205in}{4.146403in}}%
\pgfpathlineto{\pgfqpoint{3.272323in}{4.147630in}}%
\pgfpathlineto{\pgfqpoint{3.277441in}{4.146032in}}%
\pgfpathlineto{\pgfqpoint{3.282559in}{4.141615in}}%
\pgfpathlineto{\pgfqpoint{3.287677in}{4.134394in}}%
\pgfpathlineto{\pgfqpoint{3.292795in}{4.124397in}}%
\pgfpathlineto{\pgfqpoint{3.303032in}{4.096231in}}%
\pgfpathlineto{\pgfqpoint{3.313268in}{4.057534in}}%
\pgfpathlineto{\pgfqpoint{3.323504in}{4.008878in}}%
\pgfpathlineto{\pgfqpoint{3.333741in}{3.950983in}}%
\pgfpathlineto{\pgfqpoint{3.349095in}{3.848721in}}%
\pgfpathlineto{\pgfqpoint{3.364449in}{3.731023in}}%
\pgfpathlineto{\pgfqpoint{3.384922in}{3.556894in}}%
\pgfpathlineto{\pgfqpoint{3.446340in}{3.016783in}}%
\pgfpathlineto{\pgfqpoint{3.461694in}{2.901359in}}%
\pgfpathlineto{\pgfqpoint{3.477049in}{2.801968in}}%
\pgfpathlineto{\pgfqpoint{3.487285in}{2.746270in}}%
\pgfpathlineto{\pgfqpoint{3.497521in}{2.700002in}}%
\pgfpathlineto{\pgfqpoint{3.507758in}{2.663847in}}%
\pgfpathlineto{\pgfqpoint{3.517994in}{2.638341in}}%
\pgfpathlineto{\pgfqpoint{3.523112in}{2.629706in}}%
\pgfpathlineto{\pgfqpoint{3.528230in}{2.623861in}}%
\pgfpathlineto{\pgfqpoint{3.533348in}{2.620829in}}%
\pgfpathlineto{\pgfqpoint{3.538467in}{2.620620in}}%
\pgfpathlineto{\pgfqpoint{3.543585in}{2.623236in}}%
\pgfpathlineto{\pgfqpoint{3.548703in}{2.628667in}}%
\pgfpathlineto{\pgfqpoint{3.553821in}{2.636893in}}%
\pgfpathlineto{\pgfqpoint{3.558939in}{2.647883in}}%
\pgfpathlineto{\pgfqpoint{3.569176in}{2.677983in}}%
\pgfpathlineto{\pgfqpoint{3.579412in}{2.718523in}}%
\pgfpathlineto{\pgfqpoint{3.589648in}{2.768902in}}%
\pgfpathlineto{\pgfqpoint{3.599884in}{2.828377in}}%
\pgfpathlineto{\pgfqpoint{3.615239in}{2.932685in}}%
\pgfpathlineto{\pgfqpoint{3.630593in}{3.051985in}}%
\pgfpathlineto{\pgfqpoint{3.656184in}{3.273199in}}%
\pgfpathlineto{\pgfqpoint{3.707366in}{3.724903in}}%
\pgfpathlineto{\pgfqpoint{3.722720in}{3.843264in}}%
\pgfpathlineto{\pgfqpoint{3.738075in}{3.946368in}}%
\pgfpathlineto{\pgfqpoint{3.753429in}{4.030792in}}%
\pgfpathlineto{\pgfqpoint{3.763665in}{4.075283in}}%
\pgfpathlineto{\pgfqpoint{3.773902in}{4.109552in}}%
\pgfpathlineto{\pgfqpoint{3.784138in}{4.133093in}}%
\pgfpathlineto{\pgfqpoint{3.789256in}{4.140725in}}%
\pgfpathlineto{\pgfqpoint{3.794374in}{4.145558in}}%
\pgfpathlineto{\pgfqpoint{3.799492in}{4.147572in}}%
\pgfpathlineto{\pgfqpoint{3.804610in}{4.146762in}}%
\pgfpathlineto{\pgfqpoint{3.809729in}{4.143129in}}%
\pgfpathlineto{\pgfqpoint{3.814847in}{4.136687in}}%
\pgfpathlineto{\pgfqpoint{3.819965in}{4.127460in}}%
\pgfpathlineto{\pgfqpoint{3.830201in}{4.100799in}}%
\pgfpathlineto{\pgfqpoint{3.840438in}{4.063538in}}%
\pgfpathlineto{\pgfqpoint{3.850674in}{4.016230in}}%
\pgfpathlineto{\pgfqpoint{3.860910in}{3.959574in}}%
\pgfpathlineto{\pgfqpoint{3.876265in}{3.858928in}}%
\pgfpathlineto{\pgfqpoint{3.891619in}{3.742506in}}%
\pgfpathlineto{\pgfqpoint{3.912092in}{3.569479in}}%
\pgfpathlineto{\pgfqpoint{3.978628in}{2.987764in}}%
\pgfpathlineto{\pgfqpoint{3.993982in}{2.875864in}}%
\pgfpathlineto{\pgfqpoint{4.009336in}{2.780842in}}%
\pgfpathlineto{\pgfqpoint{4.019573in}{2.728457in}}%
\pgfpathlineto{\pgfqpoint{4.029809in}{2.685765in}}%
\pgfpathlineto{\pgfqpoint{4.040045in}{2.653398in}}%
\pgfpathlineto{\pgfqpoint{4.050282in}{2.631833in}}%
\pgfpathlineto{\pgfqpoint{4.055400in}{2.625208in}}%
\pgfpathlineto{\pgfqpoint{4.060518in}{2.621390in}}%
\pgfpathlineto{\pgfqpoint{4.065636in}{2.620394in}}%
\pgfpathlineto{\pgfqpoint{4.070754in}{2.622224in}}%
\pgfpathlineto{\pgfqpoint{4.075872in}{2.626872in}}%
\pgfpathlineto{\pgfqpoint{4.080991in}{2.634321in}}%
\pgfpathlineto{\pgfqpoint{4.086109in}{2.644544in}}%
\pgfpathlineto{\pgfqpoint{4.096345in}{2.673151in}}%
\pgfpathlineto{\pgfqpoint{4.106581in}{2.712268in}}%
\pgfpathlineto{\pgfqpoint{4.116818in}{2.761318in}}%
\pgfpathlineto{\pgfqpoint{4.127054in}{2.819575in}}%
\pgfpathlineto{\pgfqpoint{4.142408in}{2.922306in}}%
\pgfpathlineto{\pgfqpoint{4.157763in}{3.040373in}}%
\pgfpathlineto{\pgfqpoint{4.178235in}{3.214817in}}%
\pgfpathlineto{\pgfqpoint{4.239653in}{3.754551in}}%
\pgfpathlineto{\pgfqpoint{4.255008in}{3.869586in}}%
\pgfpathlineto{\pgfqpoint{4.270362in}{3.968490in}}%
\pgfpathlineto{\pgfqpoint{4.280598in}{4.023816in}}%
\pgfpathlineto{\pgfqpoint{4.290835in}{4.069683in}}%
\pgfpathlineto{\pgfqpoint{4.301071in}{4.105410in}}%
\pgfpathlineto{\pgfqpoint{4.311307in}{4.130471in}}%
\pgfpathlineto{\pgfqpoint{4.316426in}{4.138879in}}%
\pgfpathlineto{\pgfqpoint{4.321544in}{4.144494in}}%
\pgfpathlineto{\pgfqpoint{4.326662in}{4.147295in}}%
\pgfpathlineto{\pgfqpoint{4.331780in}{4.147271in}}%
\pgfpathlineto{\pgfqpoint{4.336898in}{4.144424in}}%
\pgfpathlineto{\pgfqpoint{4.342016in}{4.138763in}}%
\pgfpathlineto{\pgfqpoint{4.347134in}{4.130310in}}%
\pgfpathlineto{\pgfqpoint{4.352253in}{4.119095in}}%
\pgfpathlineto{\pgfqpoint{4.362489in}{4.088557in}}%
\pgfpathlineto{\pgfqpoint{4.372725in}{4.047601in}}%
\pgfpathlineto{\pgfqpoint{4.382962in}{3.996833in}}%
\pgfpathlineto{\pgfqpoint{4.393198in}{3.937004in}}%
\pgfpathlineto{\pgfqpoint{4.408552in}{3.832238in}}%
\pgfpathlineto{\pgfqpoint{4.423907in}{3.712582in}}%
\pgfpathlineto{\pgfqpoint{4.449497in}{3.491032in}}%
\pgfpathlineto{\pgfqpoint{4.495561in}{3.081772in}}%
\pgfpathlineto{\pgfqpoint{4.516033in}{2.921700in}}%
\pgfpathlineto{\pgfqpoint{4.531388in}{2.819063in}}%
\pgfpathlineto{\pgfqpoint{4.541624in}{2.760878in}}%
\pgfpathlineto{\pgfqpoint{4.551860in}{2.711906in}}%
\pgfpathlineto{\pgfqpoint{4.562097in}{2.672873in}}%
\pgfpathlineto{\pgfqpoint{4.572333in}{2.644355in}}%
\pgfpathlineto{\pgfqpoint{4.577451in}{2.634177in}}%
\pgfpathlineto{\pgfqpoint{4.582569in}{2.626773in}}%
\pgfpathlineto{\pgfqpoint{4.587688in}{2.622171in}}%
\pgfpathlineto{\pgfqpoint{4.592806in}{2.620388in}}%
\pgfpathlineto{\pgfqpoint{4.597924in}{2.621430in}}%
\pgfpathlineto{\pgfqpoint{4.603042in}{2.625294in}}%
\pgfpathlineto{\pgfqpoint{4.608160in}{2.631965in}}%
\pgfpathlineto{\pgfqpoint{4.613278in}{2.641418in}}%
\pgfpathlineto{\pgfqpoint{4.623515in}{2.668523in}}%
\pgfpathlineto{\pgfqpoint{4.633751in}{2.706207in}}%
\pgfpathlineto{\pgfqpoint{4.643987in}{2.753913in}}%
\pgfpathlineto{\pgfqpoint{4.654224in}{2.810936in}}%
\pgfpathlineto{\pgfqpoint{4.669578in}{2.912060in}}%
\pgfpathlineto{\pgfqpoint{4.684932in}{3.028861in}}%
\pgfpathlineto{\pgfqpoint{4.705405in}{3.202217in}}%
\pgfpathlineto{\pgfqpoint{4.771941in}{3.783486in}}%
\pgfpathlineto{\pgfqpoint{4.787295in}{3.894972in}}%
\pgfpathlineto{\pgfqpoint{4.802650in}{3.989486in}}%
\pgfpathlineto{\pgfqpoint{4.812886in}{4.041488in}}%
\pgfpathlineto{\pgfqpoint{4.823122in}{4.083768in}}%
\pgfpathlineto{\pgfqpoint{4.833359in}{4.115701in}}%
\pgfpathlineto{\pgfqpoint{4.843595in}{4.136815in}}%
\pgfpathlineto{\pgfqpoint{4.848713in}{4.143211in}}%
\pgfpathlineto{\pgfqpoint{4.853831in}{4.146798in}}%
\pgfpathlineto{\pgfqpoint{4.858950in}{4.147562in}}%
\pgfpathlineto{\pgfqpoint{4.864068in}{4.145501in}}%
\pgfpathlineto{\pgfqpoint{4.869186in}{4.140623in}}%
\pgfpathlineto{\pgfqpoint{4.874304in}{4.132945in}}%
\pgfpathlineto{\pgfqpoint{4.879422in}{4.122496in}}%
\pgfpathlineto{\pgfqpoint{4.889658in}{4.093449in}}%
\pgfpathlineto{\pgfqpoint{4.899895in}{4.053912in}}%
\pgfpathlineto{\pgfqpoint{4.910131in}{4.004470in}}%
\pgfpathlineto{\pgfqpoint{4.920367in}{3.945853in}}%
\pgfpathlineto{\pgfqpoint{4.935722in}{3.842655in}}%
\pgfpathlineto{\pgfqpoint{4.951076in}{3.724222in}}%
\pgfpathlineto{\pgfqpoint{4.971549in}{3.549467in}}%
\pgfpathlineto{\pgfqpoint{5.032967in}{3.010124in}}%
\pgfpathlineto{\pgfqpoint{5.048321in}{2.895482in}}%
\pgfpathlineto{\pgfqpoint{5.063676in}{2.797067in}}%
\pgfpathlineto{\pgfqpoint{5.073912in}{2.742113in}}%
\pgfpathlineto{\pgfqpoint{5.084148in}{2.696650in}}%
\pgfpathlineto{\pgfqpoint{5.094384in}{2.661350in}}%
\pgfpathlineto{\pgfqpoint{5.104621in}{2.636736in}}%
\pgfpathlineto{\pgfqpoint{5.109739in}{2.628556in}}%
\pgfpathlineto{\pgfqpoint{5.114857in}{2.623171in}}%
\pgfpathlineto{\pgfqpoint{5.119975in}{2.620601in}}%
\pgfpathlineto{\pgfqpoint{5.125093in}{2.620856in}}%
\pgfpathlineto{\pgfqpoint{5.130212in}{2.623934in}}%
\pgfpathlineto{\pgfqpoint{5.135330in}{2.629825in}}%
\pgfpathlineto{\pgfqpoint{5.140448in}{2.638506in}}%
\pgfpathlineto{\pgfqpoint{5.150684in}{2.664101in}}%
\pgfpathlineto{\pgfqpoint{5.160920in}{2.700341in}}%
\pgfpathlineto{\pgfqpoint{5.171157in}{2.746689in}}%
\pgfpathlineto{\pgfqpoint{5.181393in}{2.802462in}}%
\pgfpathlineto{\pgfqpoint{5.196747in}{2.901950in}}%
\pgfpathlineto{\pgfqpoint{5.212102in}{3.017451in}}%
\pgfpathlineto{\pgfqpoint{5.232575in}{3.189669in}}%
\pgfpathlineto{\pgfqpoint{5.299111in}{3.772395in}}%
\pgfpathlineto{\pgfqpoint{5.314465in}{3.885278in}}%
\pgfpathlineto{\pgfqpoint{5.329819in}{3.981510in}}%
\pgfpathlineto{\pgfqpoint{5.340056in}{4.034808in}}%
\pgfpathlineto{\pgfqpoint{5.350292in}{4.078484in}}%
\pgfpathlineto{\pgfqpoint{5.360528in}{4.111891in}}%
\pgfpathlineto{\pgfqpoint{5.370765in}{4.134535in}}%
\pgfpathlineto{\pgfqpoint{5.375883in}{4.141710in}}%
\pgfpathlineto{\pgfqpoint{5.381001in}{4.146081in}}%
\pgfpathlineto{\pgfqpoint{5.386119in}{4.147633in}}%
\pgfpathlineto{\pgfqpoint{5.391237in}{4.146359in}}%
\pgfpathlineto{\pgfqpoint{5.396355in}{4.142265in}}%
\pgfpathlineto{\pgfqpoint{5.401474in}{4.135364in}}%
\pgfpathlineto{\pgfqpoint{5.406592in}{4.125684in}}%
\pgfpathlineto{\pgfqpoint{5.416828in}{4.098137in}}%
\pgfpathlineto{\pgfqpoint{5.427064in}{4.060030in}}%
\pgfpathlineto{\pgfqpoint{5.437301in}{4.011927in}}%
\pgfpathlineto{\pgfqpoint{5.447537in}{3.954540in}}%
\pgfpathlineto{\pgfqpoint{5.462891in}{3.852940in}}%
\pgfpathlineto{\pgfqpoint{5.478246in}{3.735763in}}%
\pgfpathlineto{\pgfqpoint{5.498718in}{3.562082in}}%
\pgfpathlineto{\pgfqpoint{5.534545in}{3.239413in}}%
\pgfpathlineto{\pgfqpoint{5.534545in}{3.239413in}}%
\pgfusepath{stroke}%
\end{pgfscope}%
\begin{pgfscope}%
\pgfsetrectcap%
\pgfsetmiterjoin%
\pgfsetlinewidth{0.803000pt}%
\definecolor{currentstroke}{rgb}{0.000000,0.000000,0.000000}%
\pgfsetstrokecolor{currentstroke}%
\pgfsetdash{}{0pt}%
\pgfpathmoveto{\pgfqpoint{0.800000in}{2.544000in}}%
\pgfpathlineto{\pgfqpoint{0.800000in}{4.224000in}}%
\pgfusepath{stroke}%
\end{pgfscope}%
\begin{pgfscope}%
\pgfsetrectcap%
\pgfsetmiterjoin%
\pgfsetlinewidth{0.803000pt}%
\definecolor{currentstroke}{rgb}{0.000000,0.000000,0.000000}%
\pgfsetstrokecolor{currentstroke}%
\pgfsetdash{}{0pt}%
\pgfpathmoveto{\pgfqpoint{5.760000in}{2.544000in}}%
\pgfpathlineto{\pgfqpoint{5.760000in}{4.224000in}}%
\pgfusepath{stroke}%
\end{pgfscope}%
\begin{pgfscope}%
\pgfsetrectcap%
\pgfsetmiterjoin%
\pgfsetlinewidth{0.803000pt}%
\definecolor{currentstroke}{rgb}{0.000000,0.000000,0.000000}%
\pgfsetstrokecolor{currentstroke}%
\pgfsetdash{}{0pt}%
\pgfpathmoveto{\pgfqpoint{0.800000in}{2.544000in}}%
\pgfpathlineto{\pgfqpoint{5.760000in}{2.544000in}}%
\pgfusepath{stroke}%
\end{pgfscope}%
\begin{pgfscope}%
\pgfsetrectcap%
\pgfsetmiterjoin%
\pgfsetlinewidth{0.803000pt}%
\definecolor{currentstroke}{rgb}{0.000000,0.000000,0.000000}%
\pgfsetstrokecolor{currentstroke}%
\pgfsetdash{}{0pt}%
\pgfpathmoveto{\pgfqpoint{0.800000in}{4.224000in}}%
\pgfpathlineto{\pgfqpoint{5.760000in}{4.224000in}}%
\pgfusepath{stroke}%
\end{pgfscope}%
\begin{pgfscope}%
\pgfsetbuttcap%
\pgfsetmiterjoin%
\definecolor{currentfill}{rgb}{1.000000,1.000000,1.000000}%
\pgfsetfillcolor{currentfill}%
\pgfsetfillopacity{0.800000}%
\pgfsetlinewidth{1.003750pt}%
\definecolor{currentstroke}{rgb}{0.800000,0.800000,0.800000}%
\pgfsetstrokecolor{currentstroke}%
\pgfsetstrokeopacity{0.800000}%
\pgfsetdash{}{0pt}%
\pgfpathmoveto{\pgfqpoint{4.599233in}{3.057381in}}%
\pgfpathlineto{\pgfqpoint{5.662778in}{3.057381in}}%
\pgfpathquadraticcurveto{\pgfqpoint{5.690556in}{3.057381in}}{\pgfqpoint{5.690556in}{3.085159in}}%
\pgfpathlineto{\pgfqpoint{5.690556in}{3.682841in}}%
\pgfpathquadraticcurveto{\pgfqpoint{5.690556in}{3.710619in}}{\pgfqpoint{5.662778in}{3.710619in}}%
\pgfpathlineto{\pgfqpoint{4.599233in}{3.710619in}}%
\pgfpathquadraticcurveto{\pgfqpoint{4.571456in}{3.710619in}}{\pgfqpoint{4.571456in}{3.682841in}}%
\pgfpathlineto{\pgfqpoint{4.571456in}{3.085159in}}%
\pgfpathquadraticcurveto{\pgfqpoint{4.571456in}{3.057381in}}{\pgfqpoint{4.599233in}{3.057381in}}%
\pgfpathlineto{\pgfqpoint{4.599233in}{3.057381in}}%
\pgfpathclose%
\pgfusepath{stroke,fill}%
\end{pgfscope}%
\begin{pgfscope}%
\pgfsetrectcap%
\pgfsetroundjoin%
\pgfsetlinewidth{1.505625pt}%
\definecolor{currentstroke}{rgb}{0.121569,0.466667,0.705882}%
\pgfsetstrokecolor{currentstroke}%
\pgfsetdash{}{0pt}%
\pgfpathmoveto{\pgfqpoint{4.627011in}{3.598152in}}%
\pgfpathlineto{\pgfqpoint{4.765900in}{3.598152in}}%
\pgfpathlineto{\pgfqpoint{4.904789in}{3.598152in}}%
\pgfusepath{stroke}%
\end{pgfscope}%
\begin{pgfscope}%
\definecolor{textcolor}{rgb}{0.000000,0.000000,0.000000}%
\pgfsetstrokecolor{textcolor}%
\pgfsetfillcolor{textcolor}%
\pgftext[x=5.015900in,y=3.549541in,left,base]{\color{textcolor}\sffamily\fontsize{10.000000}{12.000000}\selectfont 100 Hz}%
\end{pgfscope}%
\begin{pgfscope}%
\pgfsetrectcap%
\pgfsetroundjoin%
\pgfsetlinewidth{1.505625pt}%
\definecolor{currentstroke}{rgb}{1.000000,0.498039,0.054902}%
\pgfsetstrokecolor{currentstroke}%
\pgfsetdash{}{0pt}%
\pgfpathmoveto{\pgfqpoint{4.627011in}{3.394294in}}%
\pgfpathlineto{\pgfqpoint{4.765900in}{3.394294in}}%
\pgfpathlineto{\pgfqpoint{4.904789in}{3.394294in}}%
\pgfusepath{stroke}%
\end{pgfscope}%
\begin{pgfscope}%
\definecolor{textcolor}{rgb}{0.000000,0.000000,0.000000}%
\pgfsetstrokecolor{textcolor}%
\pgfsetfillcolor{textcolor}%
\pgftext[x=5.015900in,y=3.345683in,left,base]{\color{textcolor}\sffamily\fontsize{10.000000}{12.000000}\selectfont 212 Hz}%
\end{pgfscope}%
\begin{pgfscope}%
\pgfsetrectcap%
\pgfsetroundjoin%
\pgfsetlinewidth{1.505625pt}%
\definecolor{currentstroke}{rgb}{0.172549,0.627451,0.172549}%
\pgfsetstrokecolor{currentstroke}%
\pgfsetdash{}{0pt}%
\pgfpathmoveto{\pgfqpoint{4.627011in}{3.190437in}}%
\pgfpathlineto{\pgfqpoint{4.765900in}{3.190437in}}%
\pgfpathlineto{\pgfqpoint{4.904789in}{3.190437in}}%
\pgfusepath{stroke}%
\end{pgfscope}%
\begin{pgfscope}%
\definecolor{textcolor}{rgb}{0.000000,0.000000,0.000000}%
\pgfsetstrokecolor{textcolor}%
\pgfsetfillcolor{textcolor}%
\pgftext[x=5.015900in,y=3.141826in,left,base]{\color{textcolor}\sffamily\fontsize{10.000000}{12.000000}\selectfont 213.5 Hz}%
\end{pgfscope}%
\begin{pgfscope}%
\pgfsetbuttcap%
\pgfsetmiterjoin%
\definecolor{currentfill}{rgb}{1.000000,1.000000,1.000000}%
\pgfsetfillcolor{currentfill}%
\pgfsetlinewidth{0.000000pt}%
\definecolor{currentstroke}{rgb}{0.000000,0.000000,0.000000}%
\pgfsetstrokecolor{currentstroke}%
\pgfsetstrokeopacity{0.000000}%
\pgfsetdash{}{0pt}%
\pgfpathmoveto{\pgfqpoint{0.800000in}{0.528000in}}%
\pgfpathlineto{\pgfqpoint{5.760000in}{0.528000in}}%
\pgfpathlineto{\pgfqpoint{5.760000in}{2.208000in}}%
\pgfpathlineto{\pgfqpoint{0.800000in}{2.208000in}}%
\pgfpathlineto{\pgfqpoint{0.800000in}{0.528000in}}%
\pgfpathclose%
\pgfusepath{fill}%
\end{pgfscope}%
\begin{pgfscope}%
\pgfsetbuttcap%
\pgfsetroundjoin%
\definecolor{currentfill}{rgb}{0.000000,0.000000,0.000000}%
\pgfsetfillcolor{currentfill}%
\pgfsetlinewidth{0.803000pt}%
\definecolor{currentstroke}{rgb}{0.000000,0.000000,0.000000}%
\pgfsetstrokecolor{currentstroke}%
\pgfsetdash{}{0pt}%
\pgfsys@defobject{currentmarker}{\pgfqpoint{0.000000in}{-0.048611in}}{\pgfqpoint{0.000000in}{0.000000in}}{%
\pgfpathmoveto{\pgfqpoint{0.000000in}{0.000000in}}%
\pgfpathlineto{\pgfqpoint{0.000000in}{-0.048611in}}%
\pgfusepath{stroke,fill}%
}%
\begin{pgfscope}%
\pgfsys@transformshift{1.025455in}{0.528000in}%
\pgfsys@useobject{currentmarker}{}%
\end{pgfscope}%
\end{pgfscope}%
\begin{pgfscope}%
\definecolor{textcolor}{rgb}{0.000000,0.000000,0.000000}%
\pgfsetstrokecolor{textcolor}%
\pgfsetfillcolor{textcolor}%
\pgftext[x=1.025455in,y=0.430778in,,top]{\color{textcolor}\sffamily\fontsize{10.000000}{12.000000}\selectfont 0}%
\end{pgfscope}%
\begin{pgfscope}%
\pgfsetbuttcap%
\pgfsetroundjoin%
\definecolor{currentfill}{rgb}{0.000000,0.000000,0.000000}%
\pgfsetfillcolor{currentfill}%
\pgfsetlinewidth{0.803000pt}%
\definecolor{currentstroke}{rgb}{0.000000,0.000000,0.000000}%
\pgfsetstrokecolor{currentstroke}%
\pgfsetdash{}{0pt}%
\pgfsys@defobject{currentmarker}{\pgfqpoint{0.000000in}{-0.048611in}}{\pgfqpoint{0.000000in}{0.000000in}}{%
\pgfpathmoveto{\pgfqpoint{0.000000in}{0.000000in}}%
\pgfpathlineto{\pgfqpoint{0.000000in}{-0.048611in}}%
\pgfusepath{stroke,fill}%
}%
\begin{pgfscope}%
\pgfsys@transformshift{2.049085in}{0.528000in}%
\pgfsys@useobject{currentmarker}{}%
\end{pgfscope}%
\end{pgfscope}%
\begin{pgfscope}%
\definecolor{textcolor}{rgb}{0.000000,0.000000,0.000000}%
\pgfsetstrokecolor{textcolor}%
\pgfsetfillcolor{textcolor}%
\pgftext[x=2.049085in,y=0.430778in,,top]{\color{textcolor}\sffamily\fontsize{10.000000}{12.000000}\selectfont 200}%
\end{pgfscope}%
\begin{pgfscope}%
\pgfsetbuttcap%
\pgfsetroundjoin%
\definecolor{currentfill}{rgb}{0.000000,0.000000,0.000000}%
\pgfsetfillcolor{currentfill}%
\pgfsetlinewidth{0.803000pt}%
\definecolor{currentstroke}{rgb}{0.000000,0.000000,0.000000}%
\pgfsetstrokecolor{currentstroke}%
\pgfsetdash{}{0pt}%
\pgfsys@defobject{currentmarker}{\pgfqpoint{0.000000in}{-0.048611in}}{\pgfqpoint{0.000000in}{0.000000in}}{%
\pgfpathmoveto{\pgfqpoint{0.000000in}{0.000000in}}%
\pgfpathlineto{\pgfqpoint{0.000000in}{-0.048611in}}%
\pgfusepath{stroke,fill}%
}%
\begin{pgfscope}%
\pgfsys@transformshift{3.072715in}{0.528000in}%
\pgfsys@useobject{currentmarker}{}%
\end{pgfscope}%
\end{pgfscope}%
\begin{pgfscope}%
\definecolor{textcolor}{rgb}{0.000000,0.000000,0.000000}%
\pgfsetstrokecolor{textcolor}%
\pgfsetfillcolor{textcolor}%
\pgftext[x=3.072715in,y=0.430778in,,top]{\color{textcolor}\sffamily\fontsize{10.000000}{12.000000}\selectfont 400}%
\end{pgfscope}%
\begin{pgfscope}%
\pgfsetbuttcap%
\pgfsetroundjoin%
\definecolor{currentfill}{rgb}{0.000000,0.000000,0.000000}%
\pgfsetfillcolor{currentfill}%
\pgfsetlinewidth{0.803000pt}%
\definecolor{currentstroke}{rgb}{0.000000,0.000000,0.000000}%
\pgfsetstrokecolor{currentstroke}%
\pgfsetdash{}{0pt}%
\pgfsys@defobject{currentmarker}{\pgfqpoint{0.000000in}{-0.048611in}}{\pgfqpoint{0.000000in}{0.000000in}}{%
\pgfpathmoveto{\pgfqpoint{0.000000in}{0.000000in}}%
\pgfpathlineto{\pgfqpoint{0.000000in}{-0.048611in}}%
\pgfusepath{stroke,fill}%
}%
\begin{pgfscope}%
\pgfsys@transformshift{4.096345in}{0.528000in}%
\pgfsys@useobject{currentmarker}{}%
\end{pgfscope}%
\end{pgfscope}%
\begin{pgfscope}%
\definecolor{textcolor}{rgb}{0.000000,0.000000,0.000000}%
\pgfsetstrokecolor{textcolor}%
\pgfsetfillcolor{textcolor}%
\pgftext[x=4.096345in,y=0.430778in,,top]{\color{textcolor}\sffamily\fontsize{10.000000}{12.000000}\selectfont 600}%
\end{pgfscope}%
\begin{pgfscope}%
\pgfsetbuttcap%
\pgfsetroundjoin%
\definecolor{currentfill}{rgb}{0.000000,0.000000,0.000000}%
\pgfsetfillcolor{currentfill}%
\pgfsetlinewidth{0.803000pt}%
\definecolor{currentstroke}{rgb}{0.000000,0.000000,0.000000}%
\pgfsetstrokecolor{currentstroke}%
\pgfsetdash{}{0pt}%
\pgfsys@defobject{currentmarker}{\pgfqpoint{0.000000in}{-0.048611in}}{\pgfqpoint{0.000000in}{0.000000in}}{%
\pgfpathmoveto{\pgfqpoint{0.000000in}{0.000000in}}%
\pgfpathlineto{\pgfqpoint{0.000000in}{-0.048611in}}%
\pgfusepath{stroke,fill}%
}%
\begin{pgfscope}%
\pgfsys@transformshift{5.119975in}{0.528000in}%
\pgfsys@useobject{currentmarker}{}%
\end{pgfscope}%
\end{pgfscope}%
\begin{pgfscope}%
\definecolor{textcolor}{rgb}{0.000000,0.000000,0.000000}%
\pgfsetstrokecolor{textcolor}%
\pgfsetfillcolor{textcolor}%
\pgftext[x=5.119975in,y=0.430778in,,top]{\color{textcolor}\sffamily\fontsize{10.000000}{12.000000}\selectfont 800}%
\end{pgfscope}%
\begin{pgfscope}%
\pgfsetbuttcap%
\pgfsetroundjoin%
\definecolor{currentfill}{rgb}{0.000000,0.000000,0.000000}%
\pgfsetfillcolor{currentfill}%
\pgfsetlinewidth{0.803000pt}%
\definecolor{currentstroke}{rgb}{0.000000,0.000000,0.000000}%
\pgfsetstrokecolor{currentstroke}%
\pgfsetdash{}{0pt}%
\pgfsys@defobject{currentmarker}{\pgfqpoint{-0.048611in}{0.000000in}}{\pgfqpoint{-0.000000in}{0.000000in}}{%
\pgfpathmoveto{\pgfqpoint{-0.000000in}{0.000000in}}%
\pgfpathlineto{\pgfqpoint{-0.048611in}{0.000000in}}%
\pgfusepath{stroke,fill}%
}%
\begin{pgfscope}%
\pgfsys@transformshift{0.800000in}{0.870005in}%
\pgfsys@useobject{currentmarker}{}%
\end{pgfscope}%
\end{pgfscope}%
\begin{pgfscope}%
\definecolor{textcolor}{rgb}{0.000000,0.000000,0.000000}%
\pgfsetstrokecolor{textcolor}%
\pgfsetfillcolor{textcolor}%
\pgftext[x=0.506387in, y=0.817243in, left, base]{\color{textcolor}\sffamily\fontsize{10.000000}{12.000000}\selectfont \ensuremath{-}2}%
\end{pgfscope}%
\begin{pgfscope}%
\pgfsetbuttcap%
\pgfsetroundjoin%
\definecolor{currentfill}{rgb}{0.000000,0.000000,0.000000}%
\pgfsetfillcolor{currentfill}%
\pgfsetlinewidth{0.803000pt}%
\definecolor{currentstroke}{rgb}{0.000000,0.000000,0.000000}%
\pgfsetstrokecolor{currentstroke}%
\pgfsetdash{}{0pt}%
\pgfsys@defobject{currentmarker}{\pgfqpoint{-0.048611in}{0.000000in}}{\pgfqpoint{-0.000000in}{0.000000in}}{%
\pgfpathmoveto{\pgfqpoint{-0.000000in}{0.000000in}}%
\pgfpathlineto{\pgfqpoint{-0.048611in}{0.000000in}}%
\pgfusepath{stroke,fill}%
}%
\begin{pgfscope}%
\pgfsys@transformshift{0.800000in}{1.406817in}%
\pgfsys@useobject{currentmarker}{}%
\end{pgfscope}%
\end{pgfscope}%
\begin{pgfscope}%
\definecolor{textcolor}{rgb}{0.000000,0.000000,0.000000}%
\pgfsetstrokecolor{textcolor}%
\pgfsetfillcolor{textcolor}%
\pgftext[x=0.614412in, y=1.354055in, left, base]{\color{textcolor}\sffamily\fontsize{10.000000}{12.000000}\selectfont 0}%
\end{pgfscope}%
\begin{pgfscope}%
\pgfsetbuttcap%
\pgfsetroundjoin%
\definecolor{currentfill}{rgb}{0.000000,0.000000,0.000000}%
\pgfsetfillcolor{currentfill}%
\pgfsetlinewidth{0.803000pt}%
\definecolor{currentstroke}{rgb}{0.000000,0.000000,0.000000}%
\pgfsetstrokecolor{currentstroke}%
\pgfsetdash{}{0pt}%
\pgfsys@defobject{currentmarker}{\pgfqpoint{-0.048611in}{0.000000in}}{\pgfqpoint{-0.000000in}{0.000000in}}{%
\pgfpathmoveto{\pgfqpoint{-0.000000in}{0.000000in}}%
\pgfpathlineto{\pgfqpoint{-0.048611in}{0.000000in}}%
\pgfusepath{stroke,fill}%
}%
\begin{pgfscope}%
\pgfsys@transformshift{0.800000in}{1.943629in}%
\pgfsys@useobject{currentmarker}{}%
\end{pgfscope}%
\end{pgfscope}%
\begin{pgfscope}%
\definecolor{textcolor}{rgb}{0.000000,0.000000,0.000000}%
\pgfsetstrokecolor{textcolor}%
\pgfsetfillcolor{textcolor}%
\pgftext[x=0.614412in, y=1.890867in, left, base]{\color{textcolor}\sffamily\fontsize{10.000000}{12.000000}\selectfont 2}%
\end{pgfscope}%
\begin{pgfscope}%
\pgfpathrectangle{\pgfqpoint{0.800000in}{0.528000in}}{\pgfqpoint{4.960000in}{1.680000in}}%
\pgfusepath{clip}%
\pgfsetrectcap%
\pgfsetroundjoin%
\pgfsetlinewidth{1.505625pt}%
\definecolor{currentstroke}{rgb}{0.121569,0.466667,0.705882}%
\pgfsetstrokecolor{currentstroke}%
\pgfsetdash{}{0pt}%
\pgfpathmoveto{\pgfqpoint{1.025455in}{1.406817in}}%
\pgfpathlineto{\pgfqpoint{1.061282in}{1.681028in}}%
\pgfpathlineto{\pgfqpoint{1.081754in}{1.821605in}}%
\pgfpathlineto{\pgfqpoint{1.097109in}{1.913919in}}%
\pgfpathlineto{\pgfqpoint{1.112463in}{1.992183in}}%
\pgfpathlineto{\pgfqpoint{1.122699in}{2.035521in}}%
\pgfpathlineto{\pgfqpoint{1.132936in}{2.071209in}}%
\pgfpathlineto{\pgfqpoint{1.143172in}{2.098872in}}%
\pgfpathlineto{\pgfqpoint{1.153408in}{2.118243in}}%
\pgfpathlineto{\pgfqpoint{1.158526in}{2.124769in}}%
\pgfpathlineto{\pgfqpoint{1.163645in}{2.129175in}}%
\pgfpathlineto{\pgfqpoint{1.168763in}{2.131462in}}%
\pgfpathlineto{\pgfqpoint{1.173881in}{2.131636in}}%
\pgfpathlineto{\pgfqpoint{1.178999in}{2.129713in}}%
\pgfpathlineto{\pgfqpoint{1.184117in}{2.125712in}}%
\pgfpathlineto{\pgfqpoint{1.189235in}{2.119665in}}%
\pgfpathlineto{\pgfqpoint{1.199472in}{2.101577in}}%
\pgfpathlineto{\pgfqpoint{1.209708in}{2.075819in}}%
\pgfpathlineto{\pgfqpoint{1.219944in}{2.042862in}}%
\pgfpathlineto{\pgfqpoint{1.230181in}{2.003278in}}%
\pgfpathlineto{\pgfqpoint{1.245535in}{1.932938in}}%
\pgfpathlineto{\pgfqpoint{1.260889in}{1.851731in}}%
\pgfpathlineto{\pgfqpoint{1.281362in}{1.731516in}}%
\pgfpathlineto{\pgfqpoint{1.342780in}{1.359494in}}%
\pgfpathlineto{\pgfqpoint{1.358134in}{1.280017in}}%
\pgfpathlineto{\pgfqpoint{1.373489in}{1.211347in}}%
\pgfpathlineto{\pgfqpoint{1.383725in}{1.172615in}}%
\pgfpathlineto{\pgfqpoint{1.393961in}{1.140120in}}%
\pgfpathlineto{\pgfqpoint{1.404198in}{1.114262in}}%
\pgfpathlineto{\pgfqpoint{1.414434in}{1.095336in}}%
\pgfpathlineto{\pgfqpoint{1.424670in}{1.083527in}}%
\pgfpathlineto{\pgfqpoint{1.429788in}{1.080317in}}%
\pgfpathlineto{\pgfqpoint{1.434907in}{1.078904in}}%
\pgfpathlineto{\pgfqpoint{1.440025in}{1.079279in}}%
\pgfpathlineto{\pgfqpoint{1.445143in}{1.081426in}}%
\pgfpathlineto{\pgfqpoint{1.450261in}{1.085322in}}%
\pgfpathlineto{\pgfqpoint{1.455379in}{1.090938in}}%
\pgfpathlineto{\pgfqpoint{1.465616in}{1.107175in}}%
\pgfpathlineto{\pgfqpoint{1.475852in}{1.129768in}}%
\pgfpathlineto{\pgfqpoint{1.486088in}{1.158247in}}%
\pgfpathlineto{\pgfqpoint{1.501443in}{1.210747in}}%
\pgfpathlineto{\pgfqpoint{1.516797in}{1.272967in}}%
\pgfpathlineto{\pgfqpoint{1.537270in}{1.366542in}}%
\pgfpathlineto{\pgfqpoint{1.593569in}{1.632585in}}%
\pgfpathlineto{\pgfqpoint{1.608924in}{1.693441in}}%
\pgfpathlineto{\pgfqpoint{1.624278in}{1.744105in}}%
\pgfpathlineto{\pgfqpoint{1.634514in}{1.771089in}}%
\pgfpathlineto{\pgfqpoint{1.644751in}{1.791966in}}%
\pgfpathlineto{\pgfqpoint{1.654987in}{1.806262in}}%
\pgfpathlineto{\pgfqpoint{1.660105in}{1.810821in}}%
\pgfpathlineto{\pgfqpoint{1.665223in}{1.813601in}}%
\pgfpathlineto{\pgfqpoint{1.670342in}{1.814574in}}%
\pgfpathlineto{\pgfqpoint{1.675460in}{1.813715in}}%
\pgfpathlineto{\pgfqpoint{1.680578in}{1.811008in}}%
\pgfpathlineto{\pgfqpoint{1.685696in}{1.806444in}}%
\pgfpathlineto{\pgfqpoint{1.690814in}{1.800020in}}%
\pgfpathlineto{\pgfqpoint{1.701050in}{1.781621in}}%
\pgfpathlineto{\pgfqpoint{1.711287in}{1.755936in}}%
\pgfpathlineto{\pgfqpoint{1.721523in}{1.723203in}}%
\pgfpathlineto{\pgfqpoint{1.731759in}{1.683770in}}%
\pgfpathlineto{\pgfqpoint{1.747114in}{1.613074in}}%
\pgfpathlineto{\pgfqpoint{1.762468in}{1.530280in}}%
\pgfpathlineto{\pgfqpoint{1.782941in}{1.405228in}}%
\pgfpathlineto{\pgfqpoint{1.813650in}{1.199948in}}%
\pgfpathlineto{\pgfqpoint{1.849477in}{0.963806in}}%
\pgfpathlineto{\pgfqpoint{1.869949in}{0.845278in}}%
\pgfpathlineto{\pgfqpoint{1.885304in}{0.769591in}}%
\pgfpathlineto{\pgfqpoint{1.895540in}{0.726855in}}%
\pgfpathlineto{\pgfqpoint{1.905776in}{0.691061in}}%
\pgfpathlineto{\pgfqpoint{1.916013in}{0.662762in}}%
\pgfpathlineto{\pgfqpoint{1.926249in}{0.642413in}}%
\pgfpathlineto{\pgfqpoint{1.931367in}{0.635331in}}%
\pgfpathlineto{\pgfqpoint{1.936485in}{0.630357in}}%
\pgfpathlineto{\pgfqpoint{1.941604in}{0.627517in}}%
\pgfpathlineto{\pgfqpoint{1.946722in}{0.626830in}}%
\pgfpathlineto{\pgfqpoint{1.951840in}{0.628306in}}%
\pgfpathlineto{\pgfqpoint{1.956958in}{0.631949in}}%
\pgfpathlineto{\pgfqpoint{1.962076in}{0.637755in}}%
\pgfpathlineto{\pgfqpoint{1.967194in}{0.645713in}}%
\pgfpathlineto{\pgfqpoint{1.977431in}{0.668006in}}%
\pgfpathlineto{\pgfqpoint{1.987667in}{0.698591in}}%
\pgfpathlineto{\pgfqpoint{1.997903in}{0.737121in}}%
\pgfpathlineto{\pgfqpoint{2.008140in}{0.783139in}}%
\pgfpathlineto{\pgfqpoint{2.023494in}{0.864955in}}%
\pgfpathlineto{\pgfqpoint{2.038848in}{0.960062in}}%
\pgfpathlineto{\pgfqpoint{2.059321in}{1.102863in}}%
\pgfpathlineto{\pgfqpoint{2.090030in}{1.336443in}}%
\pgfpathlineto{\pgfqpoint{2.125857in}{1.606670in}}%
\pgfpathlineto{\pgfqpoint{2.146330in}{1.744946in}}%
\pgfpathlineto{\pgfqpoint{2.161684in}{1.835646in}}%
\pgfpathlineto{\pgfqpoint{2.177038in}{1.912491in}}%
\pgfpathlineto{\pgfqpoint{2.187275in}{1.955042in}}%
\pgfpathlineto{\pgfqpoint{2.197511in}{1.990107in}}%
\pgfpathlineto{\pgfqpoint{2.207747in}{2.017344in}}%
\pgfpathlineto{\pgfqpoint{2.217984in}{2.036520in}}%
\pgfpathlineto{\pgfqpoint{2.223102in}{2.043045in}}%
\pgfpathlineto{\pgfqpoint{2.228220in}{2.047521in}}%
\pgfpathlineto{\pgfqpoint{2.233338in}{2.049952in}}%
\pgfpathlineto{\pgfqpoint{2.238456in}{2.050347in}}%
\pgfpathlineto{\pgfqpoint{2.243574in}{2.048727in}}%
\pgfpathlineto{\pgfqpoint{2.248693in}{2.045116in}}%
\pgfpathlineto{\pgfqpoint{2.253811in}{2.039546in}}%
\pgfpathlineto{\pgfqpoint{2.264047in}{2.022696in}}%
\pgfpathlineto{\pgfqpoint{2.274283in}{1.998568in}}%
\pgfpathlineto{\pgfqpoint{2.284520in}{1.967659in}}%
\pgfpathlineto{\pgfqpoint{2.294756in}{1.930557in}}%
\pgfpathlineto{\pgfqpoint{2.310110in}{1.864781in}}%
\pgfpathlineto{\pgfqpoint{2.325465in}{1.789170in}}%
\pgfpathlineto{\pgfqpoint{2.351056in}{1.649138in}}%
\pgfpathlineto{\pgfqpoint{2.392001in}{1.420383in}}%
\pgfpathlineto{\pgfqpoint{2.412473in}{1.319121in}}%
\pgfpathlineto{\pgfqpoint{2.427828in}{1.254202in}}%
\pgfpathlineto{\pgfqpoint{2.443182in}{1.201209in}}%
\pgfpathlineto{\pgfqpoint{2.453419in}{1.173388in}}%
\pgfpathlineto{\pgfqpoint{2.463655in}{1.152052in}}%
\pgfpathlineto{\pgfqpoint{2.473891in}{1.137490in}}%
\pgfpathlineto{\pgfqpoint{2.479009in}{1.132812in}}%
\pgfpathlineto{\pgfqpoint{2.484128in}{1.129886in}}%
\pgfpathlineto{\pgfqpoint{2.489246in}{1.128719in}}%
\pgfpathlineto{\pgfqpoint{2.494364in}{1.129308in}}%
\pgfpathlineto{\pgfqpoint{2.499482in}{1.131645in}}%
\pgfpathlineto{\pgfqpoint{2.504600in}{1.135714in}}%
\pgfpathlineto{\pgfqpoint{2.509718in}{1.141492in}}%
\pgfpathlineto{\pgfqpoint{2.519955in}{1.158050in}}%
\pgfpathlineto{\pgfqpoint{2.530191in}{1.180999in}}%
\pgfpathlineto{\pgfqpoint{2.540427in}{1.209921in}}%
\pgfpathlineto{\pgfqpoint{2.550664in}{1.244299in}}%
\pgfpathlineto{\pgfqpoint{2.566018in}{1.304748in}}%
\pgfpathlineto{\pgfqpoint{2.586491in}{1.398157in}}%
\pgfpathlineto{\pgfqpoint{2.622318in}{1.578270in}}%
\pgfpathlineto{\pgfqpoint{2.647908in}{1.702029in}}%
\pgfpathlineto{\pgfqpoint{2.663263in}{1.767665in}}%
\pgfpathlineto{\pgfqpoint{2.678617in}{1.823470in}}%
\pgfpathlineto{\pgfqpoint{2.688854in}{1.854006in}}%
\pgfpathlineto{\pgfqpoint{2.699090in}{1.878475in}}%
\pgfpathlineto{\pgfqpoint{2.709326in}{1.896343in}}%
\pgfpathlineto{\pgfqpoint{2.714444in}{1.902662in}}%
\pgfpathlineto{\pgfqpoint{2.719562in}{1.907175in}}%
\pgfpathlineto{\pgfqpoint{2.724681in}{1.909844in}}%
\pgfpathlineto{\pgfqpoint{2.729799in}{1.910637in}}%
\pgfpathlineto{\pgfqpoint{2.734917in}{1.909530in}}%
\pgfpathlineto{\pgfqpoint{2.740035in}{1.906504in}}%
\pgfpathlineto{\pgfqpoint{2.745153in}{1.901549in}}%
\pgfpathlineto{\pgfqpoint{2.750271in}{1.894662in}}%
\pgfpathlineto{\pgfqpoint{2.760508in}{1.875114in}}%
\pgfpathlineto{\pgfqpoint{2.770744in}{1.847975in}}%
\pgfpathlineto{\pgfqpoint{2.780980in}{1.813476in}}%
\pgfpathlineto{\pgfqpoint{2.791217in}{1.771958in}}%
\pgfpathlineto{\pgfqpoint{2.806571in}{1.697526in}}%
\pgfpathlineto{\pgfqpoint{2.821925in}{1.610256in}}%
\pgfpathlineto{\pgfqpoint{2.842398in}{1.478085in}}%
\pgfpathlineto{\pgfqpoint{2.873107in}{1.259727in}}%
\pgfpathlineto{\pgfqpoint{2.908934in}{1.005009in}}%
\pgfpathlineto{\pgfqpoint{2.929407in}{0.874515in}}%
\pgfpathlineto{\pgfqpoint{2.944761in}{0.789343in}}%
\pgfpathlineto{\pgfqpoint{2.960116in}{0.717982in}}%
\pgfpathlineto{\pgfqpoint{2.970352in}{0.679184in}}%
\pgfpathlineto{\pgfqpoint{2.980588in}{0.648046in}}%
\pgfpathlineto{\pgfqpoint{2.990824in}{0.624999in}}%
\pgfpathlineto{\pgfqpoint{2.995943in}{0.616616in}}%
\pgfpathlineto{\pgfqpoint{3.001061in}{0.610369in}}%
\pgfpathlineto{\pgfqpoint{3.006179in}{0.606279in}}%
\pgfpathlineto{\pgfqpoint{3.011297in}{0.604364in}}%
\pgfpathlineto{\pgfqpoint{3.016415in}{0.604629in}}%
\pgfpathlineto{\pgfqpoint{3.021533in}{0.607077in}}%
\pgfpathlineto{\pgfqpoint{3.026652in}{0.611700in}}%
\pgfpathlineto{\pgfqpoint{3.031770in}{0.618485in}}%
\pgfpathlineto{\pgfqpoint{3.042006in}{0.638444in}}%
\pgfpathlineto{\pgfqpoint{3.052242in}{0.666695in}}%
\pgfpathlineto{\pgfqpoint{3.062479in}{0.702870in}}%
\pgfpathlineto{\pgfqpoint{3.072715in}{0.746489in}}%
\pgfpathlineto{\pgfqpoint{3.088069in}{0.824592in}}%
\pgfpathlineto{\pgfqpoint{3.103424in}{0.915802in}}%
\pgfpathlineto{\pgfqpoint{3.123896in}{1.053075in}}%
\pgfpathlineto{\pgfqpoint{3.154605in}{1.277637in}}%
\pgfpathlineto{\pgfqpoint{3.190432in}{1.536556in}}%
\pgfpathlineto{\pgfqpoint{3.210905in}{1.668308in}}%
\pgfpathlineto{\pgfqpoint{3.226259in}{1.754265in}}%
\pgfpathlineto{\pgfqpoint{3.241614in}{1.826628in}}%
\pgfpathlineto{\pgfqpoint{3.251850in}{1.866400in}}%
\pgfpathlineto{\pgfqpoint{3.262086in}{1.898900in}}%
\pgfpathlineto{\pgfqpoint{3.272323in}{1.923825in}}%
\pgfpathlineto{\pgfqpoint{3.282559in}{1.940983in}}%
\pgfpathlineto{\pgfqpoint{3.287677in}{1.946621in}}%
\pgfpathlineto{\pgfqpoint{3.292795in}{1.950298in}}%
\pgfpathlineto{\pgfqpoint{3.297914in}{1.952022in}}%
\pgfpathlineto{\pgfqpoint{3.303032in}{1.951808in}}%
\pgfpathlineto{\pgfqpoint{3.308150in}{1.949681in}}%
\pgfpathlineto{\pgfqpoint{3.313268in}{1.945668in}}%
\pgfpathlineto{\pgfqpoint{3.318386in}{1.939806in}}%
\pgfpathlineto{\pgfqpoint{3.328622in}{1.922716in}}%
\pgfpathlineto{\pgfqpoint{3.338859in}{1.898833in}}%
\pgfpathlineto{\pgfqpoint{3.349095in}{1.868675in}}%
\pgfpathlineto{\pgfqpoint{3.359331in}{1.832851in}}%
\pgfpathlineto{\pgfqpoint{3.374686in}{1.770014in}}%
\pgfpathlineto{\pgfqpoint{3.395158in}{1.673362in}}%
\pgfpathlineto{\pgfqpoint{3.466813in}{1.316114in}}%
\pgfpathlineto{\pgfqpoint{3.482167in}{1.255166in}}%
\pgfpathlineto{\pgfqpoint{3.497521in}{1.205408in}}%
\pgfpathlineto{\pgfqpoint{3.507758in}{1.179399in}}%
\pgfpathlineto{\pgfqpoint{3.517994in}{1.159649in}}%
\pgfpathlineto{\pgfqpoint{3.528230in}{1.146495in}}%
\pgfpathlineto{\pgfqpoint{3.533348in}{1.142469in}}%
\pgfpathlineto{\pgfqpoint{3.538467in}{1.140172in}}%
\pgfpathlineto{\pgfqpoint{3.543585in}{1.139614in}}%
\pgfpathlineto{\pgfqpoint{3.548703in}{1.140800in}}%
\pgfpathlineto{\pgfqpoint{3.553821in}{1.143730in}}%
\pgfpathlineto{\pgfqpoint{3.558939in}{1.148392in}}%
\pgfpathlineto{\pgfqpoint{3.569176in}{1.162845in}}%
\pgfpathlineto{\pgfqpoint{3.579412in}{1.183948in}}%
\pgfpathlineto{\pgfqpoint{3.589648in}{1.211379in}}%
\pgfpathlineto{\pgfqpoint{3.599884in}{1.244717in}}%
\pgfpathlineto{\pgfqpoint{3.615239in}{1.304639in}}%
\pgfpathlineto{\pgfqpoint{3.630593in}{1.374543in}}%
\pgfpathlineto{\pgfqpoint{3.651066in}{1.478921in}}%
\pgfpathlineto{\pgfqpoint{3.712484in}{1.802696in}}%
\pgfpathlineto{\pgfqpoint{3.727838in}{1.870123in}}%
\pgfpathlineto{\pgfqpoint{3.743193in}{1.926495in}}%
\pgfpathlineto{\pgfqpoint{3.753429in}{1.956765in}}%
\pgfpathlineto{\pgfqpoint{3.763665in}{1.980481in}}%
\pgfpathlineto{\pgfqpoint{3.773902in}{1.997142in}}%
\pgfpathlineto{\pgfqpoint{3.779020in}{2.002696in}}%
\pgfpathlineto{\pgfqpoint{3.784138in}{2.006344in}}%
\pgfpathlineto{\pgfqpoint{3.789256in}{2.008052in}}%
\pgfpathlineto{\pgfqpoint{3.794374in}{2.007792in}}%
\pgfpathlineto{\pgfqpoint{3.799492in}{2.005545in}}%
\pgfpathlineto{\pgfqpoint{3.804610in}{2.001298in}}%
\pgfpathlineto{\pgfqpoint{3.809729in}{1.995046in}}%
\pgfpathlineto{\pgfqpoint{3.819965in}{1.976540in}}%
\pgfpathlineto{\pgfqpoint{3.830201in}{1.950127in}}%
\pgfpathlineto{\pgfqpoint{3.840438in}{1.916021in}}%
\pgfpathlineto{\pgfqpoint{3.850674in}{1.874551in}}%
\pgfpathlineto{\pgfqpoint{3.866028in}{1.799514in}}%
\pgfpathlineto{\pgfqpoint{3.881383in}{1.710798in}}%
\pgfpathlineto{\pgfqpoint{3.901855in}{1.575342in}}%
\pgfpathlineto{\pgfqpoint{3.927446in}{1.387861in}}%
\pgfpathlineto{\pgfqpoint{3.973509in}{1.044355in}}%
\pgfpathlineto{\pgfqpoint{3.993982in}{0.908132in}}%
\pgfpathlineto{\pgfqpoint{4.009336in}{0.818926in}}%
\pgfpathlineto{\pgfqpoint{4.024691in}{0.743703in}}%
\pgfpathlineto{\pgfqpoint{4.034927in}{0.702397in}}%
\pgfpathlineto{\pgfqpoint{4.045164in}{0.668778in}}%
\pgfpathlineto{\pgfqpoint{4.055400in}{0.643258in}}%
\pgfpathlineto{\pgfqpoint{4.065636in}{0.626135in}}%
\pgfpathlineto{\pgfqpoint{4.070754in}{0.620785in}}%
\pgfpathlineto{\pgfqpoint{4.075872in}{0.617595in}}%
\pgfpathlineto{\pgfqpoint{4.080991in}{0.616569in}}%
\pgfpathlineto{\pgfqpoint{4.086109in}{0.617706in}}%
\pgfpathlineto{\pgfqpoint{4.091227in}{0.620996in}}%
\pgfpathlineto{\pgfqpoint{4.096345in}{0.626421in}}%
\pgfpathlineto{\pgfqpoint{4.101463in}{0.633957in}}%
\pgfpathlineto{\pgfqpoint{4.111700in}{0.655231in}}%
\pgfpathlineto{\pgfqpoint{4.121936in}{0.684482in}}%
\pgfpathlineto{\pgfqpoint{4.132172in}{0.721268in}}%
\pgfpathlineto{\pgfqpoint{4.142408in}{0.765047in}}%
\pgfpathlineto{\pgfqpoint{4.157763in}{0.842413in}}%
\pgfpathlineto{\pgfqpoint{4.173117in}{0.931578in}}%
\pgfpathlineto{\pgfqpoint{4.193590in}{1.063900in}}%
\pgfpathlineto{\pgfqpoint{4.265244in}{1.544639in}}%
\pgfpathlineto{\pgfqpoint{4.280598in}{1.630671in}}%
\pgfpathlineto{\pgfqpoint{4.295953in}{1.704728in}}%
\pgfpathlineto{\pgfqpoint{4.306189in}{1.746447in}}%
\pgfpathlineto{\pgfqpoint{4.316426in}{1.781484in}}%
\pgfpathlineto{\pgfqpoint{4.326662in}{1.809474in}}%
\pgfpathlineto{\pgfqpoint{4.336898in}{1.830159in}}%
\pgfpathlineto{\pgfqpoint{4.347134in}{1.843395in}}%
\pgfpathlineto{\pgfqpoint{4.352253in}{1.847206in}}%
\pgfpathlineto{\pgfqpoint{4.357371in}{1.849152in}}%
\pgfpathlineto{\pgfqpoint{4.362489in}{1.849246in}}%
\pgfpathlineto{\pgfqpoint{4.367607in}{1.847510in}}%
\pgfpathlineto{\pgfqpoint{4.372725in}{1.843971in}}%
\pgfpathlineto{\pgfqpoint{4.377843in}{1.838664in}}%
\pgfpathlineto{\pgfqpoint{4.388080in}{1.822915in}}%
\pgfpathlineto{\pgfqpoint{4.398316in}{1.800668in}}%
\pgfpathlineto{\pgfqpoint{4.408552in}{1.772424in}}%
\pgfpathlineto{\pgfqpoint{4.418789in}{1.738774in}}%
\pgfpathlineto{\pgfqpoint{4.434143in}{1.679647in}}%
\pgfpathlineto{\pgfqpoint{4.454616in}{1.588685in}}%
\pgfpathlineto{\pgfqpoint{4.521152in}{1.276740in}}%
\pgfpathlineto{\pgfqpoint{4.536506in}{1.218525in}}%
\pgfpathlineto{\pgfqpoint{4.551860in}{1.170966in}}%
\pgfpathlineto{\pgfqpoint{4.562097in}{1.146186in}}%
\pgfpathlineto{\pgfqpoint{4.572333in}{1.127524in}}%
\pgfpathlineto{\pgfqpoint{4.582569in}{1.115365in}}%
\pgfpathlineto{\pgfqpoint{4.587688in}{1.111817in}}%
\pgfpathlineto{\pgfqpoint{4.592806in}{1.109993in}}%
\pgfpathlineto{\pgfqpoint{4.597924in}{1.109911in}}%
\pgfpathlineto{\pgfqpoint{4.603042in}{1.111583in}}%
\pgfpathlineto{\pgfqpoint{4.608160in}{1.115012in}}%
\pgfpathlineto{\pgfqpoint{4.613278in}{1.120198in}}%
\pgfpathlineto{\pgfqpoint{4.623515in}{1.135791in}}%
\pgfpathlineto{\pgfqpoint{4.633751in}{1.158203in}}%
\pgfpathlineto{\pgfqpoint{4.643987in}{1.187165in}}%
\pgfpathlineto{\pgfqpoint{4.654224in}{1.222304in}}%
\pgfpathlineto{\pgfqpoint{4.669578in}{1.285529in}}%
\pgfpathlineto{\pgfqpoint{4.684932in}{1.359576in}}%
\pgfpathlineto{\pgfqpoint{4.705405in}{1.470954in}}%
\pgfpathlineto{\pgfqpoint{4.777059in}{1.879413in}}%
\pgfpathlineto{\pgfqpoint{4.792414in}{1.950514in}}%
\pgfpathlineto{\pgfqpoint{4.807768in}{2.009535in}}%
\pgfpathlineto{\pgfqpoint{4.818004in}{2.041029in}}%
\pgfpathlineto{\pgfqpoint{4.828241in}{2.065566in}}%
\pgfpathlineto{\pgfqpoint{4.838477in}{2.082667in}}%
\pgfpathlineto{\pgfqpoint{4.843595in}{2.088310in}}%
\pgfpathlineto{\pgfqpoint{4.848713in}{2.091962in}}%
\pgfpathlineto{\pgfqpoint{4.853831in}{2.093595in}}%
\pgfpathlineto{\pgfqpoint{4.858950in}{2.093184in}}%
\pgfpathlineto{\pgfqpoint{4.864068in}{2.090716in}}%
\pgfpathlineto{\pgfqpoint{4.869186in}{2.086181in}}%
\pgfpathlineto{\pgfqpoint{4.874304in}{2.079578in}}%
\pgfpathlineto{\pgfqpoint{4.884540in}{2.060199in}}%
\pgfpathlineto{\pgfqpoint{4.894777in}{2.032716in}}%
\pgfpathlineto{\pgfqpoint{4.905013in}{1.997376in}}%
\pgfpathlineto{\pgfqpoint{4.915249in}{1.954541in}}%
\pgfpathlineto{\pgfqpoint{4.930604in}{1.877283in}}%
\pgfpathlineto{\pgfqpoint{4.945958in}{1.786219in}}%
\pgfpathlineto{\pgfqpoint{4.966431in}{1.647581in}}%
\pgfpathlineto{\pgfqpoint{4.992021in}{1.456243in}}%
\pgfpathlineto{\pgfqpoint{5.038085in}{1.106505in}}%
\pgfpathlineto{\pgfqpoint{5.058557in}{0.967743in}}%
\pgfpathlineto{\pgfqpoint{5.073912in}{0.876634in}}%
\pgfpathlineto{\pgfqpoint{5.089266in}{0.799433in}}%
\pgfpathlineto{\pgfqpoint{5.099503in}{0.756727in}}%
\pgfpathlineto{\pgfqpoint{5.109739in}{0.721614in}}%
\pgfpathlineto{\pgfqpoint{5.119975in}{0.694477in}}%
\pgfpathlineto{\pgfqpoint{5.130212in}{0.675588in}}%
\pgfpathlineto{\pgfqpoint{5.135330in}{0.669291in}}%
\pgfpathlineto{\pgfqpoint{5.140448in}{0.665107in}}%
\pgfpathlineto{\pgfqpoint{5.145566in}{0.663037in}}%
\pgfpathlineto{\pgfqpoint{5.150684in}{0.663076in}}%
\pgfpathlineto{\pgfqpoint{5.155802in}{0.665210in}}%
\pgfpathlineto{\pgfqpoint{5.160920in}{0.669421in}}%
\pgfpathlineto{\pgfqpoint{5.166039in}{0.675680in}}%
\pgfpathlineto{\pgfqpoint{5.176275in}{0.694202in}}%
\pgfpathlineto{\pgfqpoint{5.186511in}{0.720423in}}%
\pgfpathlineto{\pgfqpoint{5.196747in}{0.753883in}}%
\pgfpathlineto{\pgfqpoint{5.206984in}{0.794029in}}%
\pgfpathlineto{\pgfqpoint{5.222338in}{0.865353in}}%
\pgfpathlineto{\pgfqpoint{5.237693in}{0.947758in}}%
\pgfpathlineto{\pgfqpoint{5.258165in}{1.069977in}}%
\pgfpathlineto{\pgfqpoint{5.324701in}{1.480068in}}%
\pgfpathlineto{\pgfqpoint{5.340056in}{1.559552in}}%
\pgfpathlineto{\pgfqpoint{5.355410in}{1.627803in}}%
\pgfpathlineto{\pgfqpoint{5.365646in}{1.666075in}}%
\pgfpathlineto{\pgfqpoint{5.375883in}{1.698007in}}%
\pgfpathlineto{\pgfqpoint{5.386119in}{1.723233in}}%
\pgfpathlineto{\pgfqpoint{5.396355in}{1.741495in}}%
\pgfpathlineto{\pgfqpoint{5.406592in}{1.752645in}}%
\pgfpathlineto{\pgfqpoint{5.411710in}{1.755537in}}%
\pgfpathlineto{\pgfqpoint{5.416828in}{1.756646in}}%
\pgfpathlineto{\pgfqpoint{5.421946in}{1.755985in}}%
\pgfpathlineto{\pgfqpoint{5.427064in}{1.753574in}}%
\pgfpathlineto{\pgfqpoint{5.432182in}{1.749438in}}%
\pgfpathlineto{\pgfqpoint{5.442419in}{1.736135in}}%
\pgfpathlineto{\pgfqpoint{5.452655in}{1.716419in}}%
\pgfpathlineto{\pgfqpoint{5.462891in}{1.690733in}}%
\pgfpathlineto{\pgfqpoint{5.473128in}{1.659613in}}%
\pgfpathlineto{\pgfqpoint{5.488482in}{1.604119in}}%
\pgfpathlineto{\pgfqpoint{5.503837in}{1.540198in}}%
\pgfpathlineto{\pgfqpoint{5.529427in}{1.422316in}}%
\pgfpathlineto{\pgfqpoint{5.534545in}{1.398005in}}%
\pgfpathlineto{\pgfqpoint{5.534545in}{1.398005in}}%
\pgfusepath{stroke}%
\end{pgfscope}%
\begin{pgfscope}%
\pgfsetrectcap%
\pgfsetmiterjoin%
\pgfsetlinewidth{0.803000pt}%
\definecolor{currentstroke}{rgb}{0.000000,0.000000,0.000000}%
\pgfsetstrokecolor{currentstroke}%
\pgfsetdash{}{0pt}%
\pgfpathmoveto{\pgfqpoint{0.800000in}{0.528000in}}%
\pgfpathlineto{\pgfqpoint{0.800000in}{2.208000in}}%
\pgfusepath{stroke}%
\end{pgfscope}%
\begin{pgfscope}%
\pgfsetrectcap%
\pgfsetmiterjoin%
\pgfsetlinewidth{0.803000pt}%
\definecolor{currentstroke}{rgb}{0.000000,0.000000,0.000000}%
\pgfsetstrokecolor{currentstroke}%
\pgfsetdash{}{0pt}%
\pgfpathmoveto{\pgfqpoint{5.760000in}{0.528000in}}%
\pgfpathlineto{\pgfqpoint{5.760000in}{2.208000in}}%
\pgfusepath{stroke}%
\end{pgfscope}%
\begin{pgfscope}%
\pgfsetrectcap%
\pgfsetmiterjoin%
\pgfsetlinewidth{0.803000pt}%
\definecolor{currentstroke}{rgb}{0.000000,0.000000,0.000000}%
\pgfsetstrokecolor{currentstroke}%
\pgfsetdash{}{0pt}%
\pgfpathmoveto{\pgfqpoint{0.800000in}{0.528000in}}%
\pgfpathlineto{\pgfqpoint{5.760000in}{0.528000in}}%
\pgfusepath{stroke}%
\end{pgfscope}%
\begin{pgfscope}%
\pgfsetrectcap%
\pgfsetmiterjoin%
\pgfsetlinewidth{0.803000pt}%
\definecolor{currentstroke}{rgb}{0.000000,0.000000,0.000000}%
\pgfsetstrokecolor{currentstroke}%
\pgfsetdash{}{0pt}%
\pgfpathmoveto{\pgfqpoint{0.800000in}{2.208000in}}%
\pgfpathlineto{\pgfqpoint{5.760000in}{2.208000in}}%
\pgfusepath{stroke}%
\end{pgfscope}%
\end{pgfpicture}%
\makeatother%
\endgroup%
}
    \end{center}
    \caption{\emph{Spettro inarmonico} ottenuto sommando una componente a 100 Hz, una a 212 Hz e una a 213.5 Hz.}
\end{figure}


La nostra percezione di armonicità (e con essa la capacità di individuare percettivamente una frequenza fondamentale) tende però ad attenuarsi e a svanire in almeno due circostanze: quando la frequenza fondamentale scende al di sotto dei 20 o 30 Hz e quando il segnale non contiene armoniche di ordine basso. Nel primo esempio entrambe le condizioni sono vere: la fondamentale sarebbe estremamente grave, ampiamente al di fuori della nostra banda di percezione frequenziale, e le tre componenti ne costituirebbero rispettivamente la 200esima, 424esima e 427esima armonica. Nella pratica, quindi, quando una di queste due circostanze si avvera consideriamo inarmonico lo spettro del segnale. È facile riconoscere questo tipo di spettri guardando un loro grafico nel dominio della frequenza: se l'occhio non distingue l'equidistanza delle componenti, possiamo assumere che la qualità sonora dello spettro sarà di tipo inarmonico.

Esistono alcuni oggetti (come le campane e le piastre di metallo), oltre ad alcune specifiche tecniche di sintesi e trattamento sonora (come la modulazione ad anello, il \emph{frequency shifting}, la sintesi per modulazione di frequenza) che tendono a produrre spettri ascrivibili a questa tipologia.




\subsection{Spettri rumorosi (a componenti continue)}

Non tutti gli spettri sono scomponibili in singole componenti sinusoidali a frequenze precisamente individuate: è infatti possibile che l'ampiezza sia non nulla per componenti a frequenze infinitamente vicine. Un caso che abbiamo già incontrato è il delta di Dirac. Questo equivale a dire che esistono regioni dello spazio delle frequenze in cui l'energia è distribuita in maniera continua. In termini di rappresentazione nel dominio del tempo, questo corrisponde a segnali caotici, il cui comportamento può essere descritto esclusivamente in maniera statistica. In termini di percezione sonora, questo produce suoni inarmonici che classifichiamo intuitivamente come ``rumorosi'': molte consonanti, gli strumenti a percussione ad altezza indeterminata come i piatti, il vento tra le foglie, una cascata.

Se in un suono a componenti continue emergono alcune bande di frequenza relativamente strette con ampiezze relativamente alte, possiamo attribuirgli un'intonazione: un esempio classico è il suono del flauto, soprattutto quando prodotto con molta ``aria''. In senso proprio, dovremmo dire che nessun suono del mondo reale ha componenti precisamente sinusoidali: a un'analisi abbastanza accurata, ciascuna delle cosiddette sinusoidi si rivelerebbe come una componente continua a banda molto stretta.

Tra i segnali a componenti continue, ricordiamo almeno:

\begin{itemize}

\item Il \emph{rumore bianco}, il cui spettro contiene energia a tutte le frequenze, e la stessa quantità di energia per banda di frequenza: per esempio, la quantità di energia nella banda tra 100 e 200 Hz è uguale a quella nella banda tra 1100 e 1200 Hz. Questo rende il rumore bianco estremamente brillante e aspro. Nel dominio del tempo, il rumore bianco è modellizzabile come un segnale completamente casuale nel dominio del tempo.

\item Il \emph{rumore rosa}, il cui spettro contiene energia a tutte le frequenze, e la stessa quantità di energia per ottava (o, più generalmente, per intervallo nello spazio delle altezze): per esempio, la quantità di energia nella banda tra 100 e 200 Hz è uguale a quella nella banda tra 1000 e 2000 Hz. 

\begin{figure}
    \begin{center}
       \scalebox{0.6} {%% Creator: Matplotlib, PGF backend
%%
%% To include the figure in your LaTeX document, write
%%   \input{<filename>.pgf}
%%
%% Make sure the required packages are loaded in your preamble
%%   \usepackage{pgf}
%%
%% Also ensure that all the required font packages are loaded; for instance,
%% the lmodern package is sometimes necessary when using math font.
%%   \usepackage{lmodern}
%%
%% Figures using additional raster images can only be included by \input if
%% they are in the same directory as the main LaTeX file. For loading figures
%% from other directories you can use the `import` package
%%   \usepackage{import}
%%
%% and then include the figures with
%%   \import{<path to file>}{<filename>.pgf}
%%
%% Matplotlib used the following preamble
%%   
%%   \makeatletter\@ifpackageloaded{underscore}{}{\usepackage[strings]{underscore}}\makeatother
%%
\begingroup%
\makeatletter%
\begin{pgfpicture}%
\pgfpathrectangle{\pgfpointorigin}{\pgfqpoint{7.000000in}{4.000000in}}%
\pgfusepath{use as bounding box, clip}%
\begin{pgfscope}%
\pgfsetbuttcap%
\pgfsetmiterjoin%
\definecolor{currentfill}{rgb}{1.000000,1.000000,1.000000}%
\pgfsetfillcolor{currentfill}%
\pgfsetlinewidth{0.000000pt}%
\definecolor{currentstroke}{rgb}{1.000000,1.000000,1.000000}%
\pgfsetstrokecolor{currentstroke}%
\pgfsetdash{}{0pt}%
\pgfpathmoveto{\pgfqpoint{0.000000in}{0.000000in}}%
\pgfpathlineto{\pgfqpoint{7.000000in}{0.000000in}}%
\pgfpathlineto{\pgfqpoint{7.000000in}{4.000000in}}%
\pgfpathlineto{\pgfqpoint{0.000000in}{4.000000in}}%
\pgfpathlineto{\pgfqpoint{0.000000in}{0.000000in}}%
\pgfpathclose%
\pgfusepath{fill}%
\end{pgfscope}%
\begin{pgfscope}%
\pgfsetbuttcap%
\pgfsetmiterjoin%
\definecolor{currentfill}{rgb}{1.000000,1.000000,1.000000}%
\pgfsetfillcolor{currentfill}%
\pgfsetlinewidth{0.000000pt}%
\definecolor{currentstroke}{rgb}{0.000000,0.000000,0.000000}%
\pgfsetstrokecolor{currentstroke}%
\pgfsetstrokeopacity{0.000000}%
\pgfsetdash{}{0pt}%
\pgfpathmoveto{\pgfqpoint{0.535225in}{0.370679in}}%
\pgfpathlineto{\pgfqpoint{6.850000in}{0.370679in}}%
\pgfpathlineto{\pgfqpoint{6.850000in}{3.551852in}}%
\pgfpathlineto{\pgfqpoint{0.535225in}{3.551852in}}%
\pgfpathlineto{\pgfqpoint{0.535225in}{0.370679in}}%
\pgfpathclose%
\pgfusepath{fill}%
\end{pgfscope}%
\begin{pgfscope}%
\pgfsetbuttcap%
\pgfsetroundjoin%
\definecolor{currentfill}{rgb}{0.000000,0.000000,0.000000}%
\pgfsetfillcolor{currentfill}%
\pgfsetlinewidth{0.803000pt}%
\definecolor{currentstroke}{rgb}{0.000000,0.000000,0.000000}%
\pgfsetstrokecolor{currentstroke}%
\pgfsetdash{}{0pt}%
\pgfsys@defobject{currentmarker}{\pgfqpoint{0.000000in}{-0.048611in}}{\pgfqpoint{0.000000in}{0.000000in}}{%
\pgfpathmoveto{\pgfqpoint{0.000000in}{0.000000in}}%
\pgfpathlineto{\pgfqpoint{0.000000in}{-0.048611in}}%
\pgfusepath{stroke,fill}%
}%
\begin{pgfscope}%
\pgfsys@transformshift{0.994324in}{0.370679in}%
\pgfsys@useobject{currentmarker}{}%
\end{pgfscope}%
\end{pgfscope}%
\begin{pgfscope}%
\definecolor{textcolor}{rgb}{0.000000,0.000000,0.000000}%
\pgfsetstrokecolor{textcolor}%
\pgfsetfillcolor{textcolor}%
\pgftext[x=0.994324in,y=0.273457in,,top]{\color{textcolor}\rmfamily\fontsize{10.000000}{12.000000}\selectfont \(\displaystyle {10^{-6}}\)}%
\end{pgfscope}%
\begin{pgfscope}%
\pgfsetbuttcap%
\pgfsetroundjoin%
\definecolor{currentfill}{rgb}{0.000000,0.000000,0.000000}%
\pgfsetfillcolor{currentfill}%
\pgfsetlinewidth{0.803000pt}%
\definecolor{currentstroke}{rgb}{0.000000,0.000000,0.000000}%
\pgfsetstrokecolor{currentstroke}%
\pgfsetdash{}{0pt}%
\pgfsys@defobject{currentmarker}{\pgfqpoint{0.000000in}{-0.048611in}}{\pgfqpoint{0.000000in}{0.000000in}}{%
\pgfpathmoveto{\pgfqpoint{0.000000in}{0.000000in}}%
\pgfpathlineto{\pgfqpoint{0.000000in}{-0.048611in}}%
\pgfusepath{stroke,fill}%
}%
\begin{pgfscope}%
\pgfsys@transformshift{1.971455in}{0.370679in}%
\pgfsys@useobject{currentmarker}{}%
\end{pgfscope}%
\end{pgfscope}%
\begin{pgfscope}%
\definecolor{textcolor}{rgb}{0.000000,0.000000,0.000000}%
\pgfsetstrokecolor{textcolor}%
\pgfsetfillcolor{textcolor}%
\pgftext[x=1.971455in,y=0.273457in,,top]{\color{textcolor}\rmfamily\fontsize{10.000000}{12.000000}\selectfont \(\displaystyle {10^{-5}}\)}%
\end{pgfscope}%
\begin{pgfscope}%
\pgfsetbuttcap%
\pgfsetroundjoin%
\definecolor{currentfill}{rgb}{0.000000,0.000000,0.000000}%
\pgfsetfillcolor{currentfill}%
\pgfsetlinewidth{0.803000pt}%
\definecolor{currentstroke}{rgb}{0.000000,0.000000,0.000000}%
\pgfsetstrokecolor{currentstroke}%
\pgfsetdash{}{0pt}%
\pgfsys@defobject{currentmarker}{\pgfqpoint{0.000000in}{-0.048611in}}{\pgfqpoint{0.000000in}{0.000000in}}{%
\pgfpathmoveto{\pgfqpoint{0.000000in}{0.000000in}}%
\pgfpathlineto{\pgfqpoint{0.000000in}{-0.048611in}}%
\pgfusepath{stroke,fill}%
}%
\begin{pgfscope}%
\pgfsys@transformshift{2.948586in}{0.370679in}%
\pgfsys@useobject{currentmarker}{}%
\end{pgfscope}%
\end{pgfscope}%
\begin{pgfscope}%
\definecolor{textcolor}{rgb}{0.000000,0.000000,0.000000}%
\pgfsetstrokecolor{textcolor}%
\pgfsetfillcolor{textcolor}%
\pgftext[x=2.948586in,y=0.273457in,,top]{\color{textcolor}\rmfamily\fontsize{10.000000}{12.000000}\selectfont \(\displaystyle {10^{-4}}\)}%
\end{pgfscope}%
\begin{pgfscope}%
\pgfsetbuttcap%
\pgfsetroundjoin%
\definecolor{currentfill}{rgb}{0.000000,0.000000,0.000000}%
\pgfsetfillcolor{currentfill}%
\pgfsetlinewidth{0.803000pt}%
\definecolor{currentstroke}{rgb}{0.000000,0.000000,0.000000}%
\pgfsetstrokecolor{currentstroke}%
\pgfsetdash{}{0pt}%
\pgfsys@defobject{currentmarker}{\pgfqpoint{0.000000in}{-0.048611in}}{\pgfqpoint{0.000000in}{0.000000in}}{%
\pgfpathmoveto{\pgfqpoint{0.000000in}{0.000000in}}%
\pgfpathlineto{\pgfqpoint{0.000000in}{-0.048611in}}%
\pgfusepath{stroke,fill}%
}%
\begin{pgfscope}%
\pgfsys@transformshift{3.925717in}{0.370679in}%
\pgfsys@useobject{currentmarker}{}%
\end{pgfscope}%
\end{pgfscope}%
\begin{pgfscope}%
\definecolor{textcolor}{rgb}{0.000000,0.000000,0.000000}%
\pgfsetstrokecolor{textcolor}%
\pgfsetfillcolor{textcolor}%
\pgftext[x=3.925717in,y=0.273457in,,top]{\color{textcolor}\rmfamily\fontsize{10.000000}{12.000000}\selectfont \(\displaystyle {10^{-3}}\)}%
\end{pgfscope}%
\begin{pgfscope}%
\pgfsetbuttcap%
\pgfsetroundjoin%
\definecolor{currentfill}{rgb}{0.000000,0.000000,0.000000}%
\pgfsetfillcolor{currentfill}%
\pgfsetlinewidth{0.803000pt}%
\definecolor{currentstroke}{rgb}{0.000000,0.000000,0.000000}%
\pgfsetstrokecolor{currentstroke}%
\pgfsetdash{}{0pt}%
\pgfsys@defobject{currentmarker}{\pgfqpoint{0.000000in}{-0.048611in}}{\pgfqpoint{0.000000in}{0.000000in}}{%
\pgfpathmoveto{\pgfqpoint{0.000000in}{0.000000in}}%
\pgfpathlineto{\pgfqpoint{0.000000in}{-0.048611in}}%
\pgfusepath{stroke,fill}%
}%
\begin{pgfscope}%
\pgfsys@transformshift{4.902848in}{0.370679in}%
\pgfsys@useobject{currentmarker}{}%
\end{pgfscope}%
\end{pgfscope}%
\begin{pgfscope}%
\definecolor{textcolor}{rgb}{0.000000,0.000000,0.000000}%
\pgfsetstrokecolor{textcolor}%
\pgfsetfillcolor{textcolor}%
\pgftext[x=4.902848in,y=0.273457in,,top]{\color{textcolor}\rmfamily\fontsize{10.000000}{12.000000}\selectfont \(\displaystyle {10^{-2}}\)}%
\end{pgfscope}%
\begin{pgfscope}%
\pgfsetbuttcap%
\pgfsetroundjoin%
\definecolor{currentfill}{rgb}{0.000000,0.000000,0.000000}%
\pgfsetfillcolor{currentfill}%
\pgfsetlinewidth{0.803000pt}%
\definecolor{currentstroke}{rgb}{0.000000,0.000000,0.000000}%
\pgfsetstrokecolor{currentstroke}%
\pgfsetdash{}{0pt}%
\pgfsys@defobject{currentmarker}{\pgfqpoint{0.000000in}{-0.048611in}}{\pgfqpoint{0.000000in}{0.000000in}}{%
\pgfpathmoveto{\pgfqpoint{0.000000in}{0.000000in}}%
\pgfpathlineto{\pgfqpoint{0.000000in}{-0.048611in}}%
\pgfusepath{stroke,fill}%
}%
\begin{pgfscope}%
\pgfsys@transformshift{5.879979in}{0.370679in}%
\pgfsys@useobject{currentmarker}{}%
\end{pgfscope}%
\end{pgfscope}%
\begin{pgfscope}%
\definecolor{textcolor}{rgb}{0.000000,0.000000,0.000000}%
\pgfsetstrokecolor{textcolor}%
\pgfsetfillcolor{textcolor}%
\pgftext[x=5.879979in,y=0.273457in,,top]{\color{textcolor}\rmfamily\fontsize{10.000000}{12.000000}\selectfont \(\displaystyle {10^{-1}}\)}%
\end{pgfscope}%
\begin{pgfscope}%
\pgfsetbuttcap%
\pgfsetroundjoin%
\definecolor{currentfill}{rgb}{0.000000,0.000000,0.000000}%
\pgfsetfillcolor{currentfill}%
\pgfsetlinewidth{0.602250pt}%
\definecolor{currentstroke}{rgb}{0.000000,0.000000,0.000000}%
\pgfsetstrokecolor{currentstroke}%
\pgfsetdash{}{0pt}%
\pgfsys@defobject{currentmarker}{\pgfqpoint{0.000000in}{-0.027778in}}{\pgfqpoint{0.000000in}{0.000000in}}{%
\pgfpathmoveto{\pgfqpoint{0.000000in}{0.000000in}}%
\pgfpathlineto{\pgfqpoint{0.000000in}{-0.027778in}}%
\pgfusepath{stroke,fill}%
}%
\begin{pgfscope}%
\pgfsys@transformshift{0.605485in}{0.370679in}%
\pgfsys@useobject{currentmarker}{}%
\end{pgfscope}%
\end{pgfscope}%
\begin{pgfscope}%
\pgfsetbuttcap%
\pgfsetroundjoin%
\definecolor{currentfill}{rgb}{0.000000,0.000000,0.000000}%
\pgfsetfillcolor{currentfill}%
\pgfsetlinewidth{0.602250pt}%
\definecolor{currentstroke}{rgb}{0.000000,0.000000,0.000000}%
\pgfsetstrokecolor{currentstroke}%
\pgfsetdash{}{0pt}%
\pgfsys@defobject{currentmarker}{\pgfqpoint{0.000000in}{-0.027778in}}{\pgfqpoint{0.000000in}{0.000000in}}{%
\pgfpathmoveto{\pgfqpoint{0.000000in}{0.000000in}}%
\pgfpathlineto{\pgfqpoint{0.000000in}{-0.027778in}}%
\pgfusepath{stroke,fill}%
}%
\begin{pgfscope}%
\pgfsys@transformshift{0.700178in}{0.370679in}%
\pgfsys@useobject{currentmarker}{}%
\end{pgfscope}%
\end{pgfscope}%
\begin{pgfscope}%
\pgfsetbuttcap%
\pgfsetroundjoin%
\definecolor{currentfill}{rgb}{0.000000,0.000000,0.000000}%
\pgfsetfillcolor{currentfill}%
\pgfsetlinewidth{0.602250pt}%
\definecolor{currentstroke}{rgb}{0.000000,0.000000,0.000000}%
\pgfsetstrokecolor{currentstroke}%
\pgfsetdash{}{0pt}%
\pgfsys@defobject{currentmarker}{\pgfqpoint{0.000000in}{-0.027778in}}{\pgfqpoint{0.000000in}{0.000000in}}{%
\pgfpathmoveto{\pgfqpoint{0.000000in}{0.000000in}}%
\pgfpathlineto{\pgfqpoint{0.000000in}{-0.027778in}}%
\pgfusepath{stroke,fill}%
}%
\begin{pgfscope}%
\pgfsys@transformshift{0.777549in}{0.370679in}%
\pgfsys@useobject{currentmarker}{}%
\end{pgfscope}%
\end{pgfscope}%
\begin{pgfscope}%
\pgfsetbuttcap%
\pgfsetroundjoin%
\definecolor{currentfill}{rgb}{0.000000,0.000000,0.000000}%
\pgfsetfillcolor{currentfill}%
\pgfsetlinewidth{0.602250pt}%
\definecolor{currentstroke}{rgb}{0.000000,0.000000,0.000000}%
\pgfsetstrokecolor{currentstroke}%
\pgfsetdash{}{0pt}%
\pgfsys@defobject{currentmarker}{\pgfqpoint{0.000000in}{-0.027778in}}{\pgfqpoint{0.000000in}{0.000000in}}{%
\pgfpathmoveto{\pgfqpoint{0.000000in}{0.000000in}}%
\pgfpathlineto{\pgfqpoint{0.000000in}{-0.027778in}}%
\pgfusepath{stroke,fill}%
}%
\begin{pgfscope}%
\pgfsys@transformshift{0.842965in}{0.370679in}%
\pgfsys@useobject{currentmarker}{}%
\end{pgfscope}%
\end{pgfscope}%
\begin{pgfscope}%
\pgfsetbuttcap%
\pgfsetroundjoin%
\definecolor{currentfill}{rgb}{0.000000,0.000000,0.000000}%
\pgfsetfillcolor{currentfill}%
\pgfsetlinewidth{0.602250pt}%
\definecolor{currentstroke}{rgb}{0.000000,0.000000,0.000000}%
\pgfsetstrokecolor{currentstroke}%
\pgfsetdash{}{0pt}%
\pgfsys@defobject{currentmarker}{\pgfqpoint{0.000000in}{-0.027778in}}{\pgfqpoint{0.000000in}{0.000000in}}{%
\pgfpathmoveto{\pgfqpoint{0.000000in}{0.000000in}}%
\pgfpathlineto{\pgfqpoint{0.000000in}{-0.027778in}}%
\pgfusepath{stroke,fill}%
}%
\begin{pgfscope}%
\pgfsys@transformshift{0.899630in}{0.370679in}%
\pgfsys@useobject{currentmarker}{}%
\end{pgfscope}%
\end{pgfscope}%
\begin{pgfscope}%
\pgfsetbuttcap%
\pgfsetroundjoin%
\definecolor{currentfill}{rgb}{0.000000,0.000000,0.000000}%
\pgfsetfillcolor{currentfill}%
\pgfsetlinewidth{0.602250pt}%
\definecolor{currentstroke}{rgb}{0.000000,0.000000,0.000000}%
\pgfsetstrokecolor{currentstroke}%
\pgfsetdash{}{0pt}%
\pgfsys@defobject{currentmarker}{\pgfqpoint{0.000000in}{-0.027778in}}{\pgfqpoint{0.000000in}{0.000000in}}{%
\pgfpathmoveto{\pgfqpoint{0.000000in}{0.000000in}}%
\pgfpathlineto{\pgfqpoint{0.000000in}{-0.027778in}}%
\pgfusepath{stroke,fill}%
}%
\begin{pgfscope}%
\pgfsys@transformshift{0.949613in}{0.370679in}%
\pgfsys@useobject{currentmarker}{}%
\end{pgfscope}%
\end{pgfscope}%
\begin{pgfscope}%
\pgfsetbuttcap%
\pgfsetroundjoin%
\definecolor{currentfill}{rgb}{0.000000,0.000000,0.000000}%
\pgfsetfillcolor{currentfill}%
\pgfsetlinewidth{0.602250pt}%
\definecolor{currentstroke}{rgb}{0.000000,0.000000,0.000000}%
\pgfsetstrokecolor{currentstroke}%
\pgfsetdash{}{0pt}%
\pgfsys@defobject{currentmarker}{\pgfqpoint{0.000000in}{-0.027778in}}{\pgfqpoint{0.000000in}{0.000000in}}{%
\pgfpathmoveto{\pgfqpoint{0.000000in}{0.000000in}}%
\pgfpathlineto{\pgfqpoint{0.000000in}{-0.027778in}}%
\pgfusepath{stroke,fill}%
}%
\begin{pgfscope}%
\pgfsys@transformshift{1.288470in}{0.370679in}%
\pgfsys@useobject{currentmarker}{}%
\end{pgfscope}%
\end{pgfscope}%
\begin{pgfscope}%
\pgfsetbuttcap%
\pgfsetroundjoin%
\definecolor{currentfill}{rgb}{0.000000,0.000000,0.000000}%
\pgfsetfillcolor{currentfill}%
\pgfsetlinewidth{0.602250pt}%
\definecolor{currentstroke}{rgb}{0.000000,0.000000,0.000000}%
\pgfsetstrokecolor{currentstroke}%
\pgfsetdash{}{0pt}%
\pgfsys@defobject{currentmarker}{\pgfqpoint{0.000000in}{-0.027778in}}{\pgfqpoint{0.000000in}{0.000000in}}{%
\pgfpathmoveto{\pgfqpoint{0.000000in}{0.000000in}}%
\pgfpathlineto{\pgfqpoint{0.000000in}{-0.027778in}}%
\pgfusepath{stroke,fill}%
}%
\begin{pgfscope}%
\pgfsys@transformshift{1.460534in}{0.370679in}%
\pgfsys@useobject{currentmarker}{}%
\end{pgfscope}%
\end{pgfscope}%
\begin{pgfscope}%
\pgfsetbuttcap%
\pgfsetroundjoin%
\definecolor{currentfill}{rgb}{0.000000,0.000000,0.000000}%
\pgfsetfillcolor{currentfill}%
\pgfsetlinewidth{0.602250pt}%
\definecolor{currentstroke}{rgb}{0.000000,0.000000,0.000000}%
\pgfsetstrokecolor{currentstroke}%
\pgfsetdash{}{0pt}%
\pgfsys@defobject{currentmarker}{\pgfqpoint{0.000000in}{-0.027778in}}{\pgfqpoint{0.000000in}{0.000000in}}{%
\pgfpathmoveto{\pgfqpoint{0.000000in}{0.000000in}}%
\pgfpathlineto{\pgfqpoint{0.000000in}{-0.027778in}}%
\pgfusepath{stroke,fill}%
}%
\begin{pgfscope}%
\pgfsys@transformshift{1.582616in}{0.370679in}%
\pgfsys@useobject{currentmarker}{}%
\end{pgfscope}%
\end{pgfscope}%
\begin{pgfscope}%
\pgfsetbuttcap%
\pgfsetroundjoin%
\definecolor{currentfill}{rgb}{0.000000,0.000000,0.000000}%
\pgfsetfillcolor{currentfill}%
\pgfsetlinewidth{0.602250pt}%
\definecolor{currentstroke}{rgb}{0.000000,0.000000,0.000000}%
\pgfsetstrokecolor{currentstroke}%
\pgfsetdash{}{0pt}%
\pgfsys@defobject{currentmarker}{\pgfqpoint{0.000000in}{-0.027778in}}{\pgfqpoint{0.000000in}{0.000000in}}{%
\pgfpathmoveto{\pgfqpoint{0.000000in}{0.000000in}}%
\pgfpathlineto{\pgfqpoint{0.000000in}{-0.027778in}}%
\pgfusepath{stroke,fill}%
}%
\begin{pgfscope}%
\pgfsys@transformshift{1.677309in}{0.370679in}%
\pgfsys@useobject{currentmarker}{}%
\end{pgfscope}%
\end{pgfscope}%
\begin{pgfscope}%
\pgfsetbuttcap%
\pgfsetroundjoin%
\definecolor{currentfill}{rgb}{0.000000,0.000000,0.000000}%
\pgfsetfillcolor{currentfill}%
\pgfsetlinewidth{0.602250pt}%
\definecolor{currentstroke}{rgb}{0.000000,0.000000,0.000000}%
\pgfsetstrokecolor{currentstroke}%
\pgfsetdash{}{0pt}%
\pgfsys@defobject{currentmarker}{\pgfqpoint{0.000000in}{-0.027778in}}{\pgfqpoint{0.000000in}{0.000000in}}{%
\pgfpathmoveto{\pgfqpoint{0.000000in}{0.000000in}}%
\pgfpathlineto{\pgfqpoint{0.000000in}{-0.027778in}}%
\pgfusepath{stroke,fill}%
}%
\begin{pgfscope}%
\pgfsys@transformshift{1.754680in}{0.370679in}%
\pgfsys@useobject{currentmarker}{}%
\end{pgfscope}%
\end{pgfscope}%
\begin{pgfscope}%
\pgfsetbuttcap%
\pgfsetroundjoin%
\definecolor{currentfill}{rgb}{0.000000,0.000000,0.000000}%
\pgfsetfillcolor{currentfill}%
\pgfsetlinewidth{0.602250pt}%
\definecolor{currentstroke}{rgb}{0.000000,0.000000,0.000000}%
\pgfsetstrokecolor{currentstroke}%
\pgfsetdash{}{0pt}%
\pgfsys@defobject{currentmarker}{\pgfqpoint{0.000000in}{-0.027778in}}{\pgfqpoint{0.000000in}{0.000000in}}{%
\pgfpathmoveto{\pgfqpoint{0.000000in}{0.000000in}}%
\pgfpathlineto{\pgfqpoint{0.000000in}{-0.027778in}}%
\pgfusepath{stroke,fill}%
}%
\begin{pgfscope}%
\pgfsys@transformshift{1.820096in}{0.370679in}%
\pgfsys@useobject{currentmarker}{}%
\end{pgfscope}%
\end{pgfscope}%
\begin{pgfscope}%
\pgfsetbuttcap%
\pgfsetroundjoin%
\definecolor{currentfill}{rgb}{0.000000,0.000000,0.000000}%
\pgfsetfillcolor{currentfill}%
\pgfsetlinewidth{0.602250pt}%
\definecolor{currentstroke}{rgb}{0.000000,0.000000,0.000000}%
\pgfsetstrokecolor{currentstroke}%
\pgfsetdash{}{0pt}%
\pgfsys@defobject{currentmarker}{\pgfqpoint{0.000000in}{-0.027778in}}{\pgfqpoint{0.000000in}{0.000000in}}{%
\pgfpathmoveto{\pgfqpoint{0.000000in}{0.000000in}}%
\pgfpathlineto{\pgfqpoint{0.000000in}{-0.027778in}}%
\pgfusepath{stroke,fill}%
}%
\begin{pgfscope}%
\pgfsys@transformshift{1.876761in}{0.370679in}%
\pgfsys@useobject{currentmarker}{}%
\end{pgfscope}%
\end{pgfscope}%
\begin{pgfscope}%
\pgfsetbuttcap%
\pgfsetroundjoin%
\definecolor{currentfill}{rgb}{0.000000,0.000000,0.000000}%
\pgfsetfillcolor{currentfill}%
\pgfsetlinewidth{0.602250pt}%
\definecolor{currentstroke}{rgb}{0.000000,0.000000,0.000000}%
\pgfsetstrokecolor{currentstroke}%
\pgfsetdash{}{0pt}%
\pgfsys@defobject{currentmarker}{\pgfqpoint{0.000000in}{-0.027778in}}{\pgfqpoint{0.000000in}{0.000000in}}{%
\pgfpathmoveto{\pgfqpoint{0.000000in}{0.000000in}}%
\pgfpathlineto{\pgfqpoint{0.000000in}{-0.027778in}}%
\pgfusepath{stroke,fill}%
}%
\begin{pgfscope}%
\pgfsys@transformshift{1.926744in}{0.370679in}%
\pgfsys@useobject{currentmarker}{}%
\end{pgfscope}%
\end{pgfscope}%
\begin{pgfscope}%
\pgfsetbuttcap%
\pgfsetroundjoin%
\definecolor{currentfill}{rgb}{0.000000,0.000000,0.000000}%
\pgfsetfillcolor{currentfill}%
\pgfsetlinewidth{0.602250pt}%
\definecolor{currentstroke}{rgb}{0.000000,0.000000,0.000000}%
\pgfsetstrokecolor{currentstroke}%
\pgfsetdash{}{0pt}%
\pgfsys@defobject{currentmarker}{\pgfqpoint{0.000000in}{-0.027778in}}{\pgfqpoint{0.000000in}{0.000000in}}{%
\pgfpathmoveto{\pgfqpoint{0.000000in}{0.000000in}}%
\pgfpathlineto{\pgfqpoint{0.000000in}{-0.027778in}}%
\pgfusepath{stroke,fill}%
}%
\begin{pgfscope}%
\pgfsys@transformshift{2.265601in}{0.370679in}%
\pgfsys@useobject{currentmarker}{}%
\end{pgfscope}%
\end{pgfscope}%
\begin{pgfscope}%
\pgfsetbuttcap%
\pgfsetroundjoin%
\definecolor{currentfill}{rgb}{0.000000,0.000000,0.000000}%
\pgfsetfillcolor{currentfill}%
\pgfsetlinewidth{0.602250pt}%
\definecolor{currentstroke}{rgb}{0.000000,0.000000,0.000000}%
\pgfsetstrokecolor{currentstroke}%
\pgfsetdash{}{0pt}%
\pgfsys@defobject{currentmarker}{\pgfqpoint{0.000000in}{-0.027778in}}{\pgfqpoint{0.000000in}{0.000000in}}{%
\pgfpathmoveto{\pgfqpoint{0.000000in}{0.000000in}}%
\pgfpathlineto{\pgfqpoint{0.000000in}{-0.027778in}}%
\pgfusepath{stroke,fill}%
}%
\begin{pgfscope}%
\pgfsys@transformshift{2.437665in}{0.370679in}%
\pgfsys@useobject{currentmarker}{}%
\end{pgfscope}%
\end{pgfscope}%
\begin{pgfscope}%
\pgfsetbuttcap%
\pgfsetroundjoin%
\definecolor{currentfill}{rgb}{0.000000,0.000000,0.000000}%
\pgfsetfillcolor{currentfill}%
\pgfsetlinewidth{0.602250pt}%
\definecolor{currentstroke}{rgb}{0.000000,0.000000,0.000000}%
\pgfsetstrokecolor{currentstroke}%
\pgfsetdash{}{0pt}%
\pgfsys@defobject{currentmarker}{\pgfqpoint{0.000000in}{-0.027778in}}{\pgfqpoint{0.000000in}{0.000000in}}{%
\pgfpathmoveto{\pgfqpoint{0.000000in}{0.000000in}}%
\pgfpathlineto{\pgfqpoint{0.000000in}{-0.027778in}}%
\pgfusepath{stroke,fill}%
}%
\begin{pgfscope}%
\pgfsys@transformshift{2.559747in}{0.370679in}%
\pgfsys@useobject{currentmarker}{}%
\end{pgfscope}%
\end{pgfscope}%
\begin{pgfscope}%
\pgfsetbuttcap%
\pgfsetroundjoin%
\definecolor{currentfill}{rgb}{0.000000,0.000000,0.000000}%
\pgfsetfillcolor{currentfill}%
\pgfsetlinewidth{0.602250pt}%
\definecolor{currentstroke}{rgb}{0.000000,0.000000,0.000000}%
\pgfsetstrokecolor{currentstroke}%
\pgfsetdash{}{0pt}%
\pgfsys@defobject{currentmarker}{\pgfqpoint{0.000000in}{-0.027778in}}{\pgfqpoint{0.000000in}{0.000000in}}{%
\pgfpathmoveto{\pgfqpoint{0.000000in}{0.000000in}}%
\pgfpathlineto{\pgfqpoint{0.000000in}{-0.027778in}}%
\pgfusepath{stroke,fill}%
}%
\begin{pgfscope}%
\pgfsys@transformshift{2.654441in}{0.370679in}%
\pgfsys@useobject{currentmarker}{}%
\end{pgfscope}%
\end{pgfscope}%
\begin{pgfscope}%
\pgfsetbuttcap%
\pgfsetroundjoin%
\definecolor{currentfill}{rgb}{0.000000,0.000000,0.000000}%
\pgfsetfillcolor{currentfill}%
\pgfsetlinewidth{0.602250pt}%
\definecolor{currentstroke}{rgb}{0.000000,0.000000,0.000000}%
\pgfsetstrokecolor{currentstroke}%
\pgfsetdash{}{0pt}%
\pgfsys@defobject{currentmarker}{\pgfqpoint{0.000000in}{-0.027778in}}{\pgfqpoint{0.000000in}{0.000000in}}{%
\pgfpathmoveto{\pgfqpoint{0.000000in}{0.000000in}}%
\pgfpathlineto{\pgfqpoint{0.000000in}{-0.027778in}}%
\pgfusepath{stroke,fill}%
}%
\begin{pgfscope}%
\pgfsys@transformshift{2.731811in}{0.370679in}%
\pgfsys@useobject{currentmarker}{}%
\end{pgfscope}%
\end{pgfscope}%
\begin{pgfscope}%
\pgfsetbuttcap%
\pgfsetroundjoin%
\definecolor{currentfill}{rgb}{0.000000,0.000000,0.000000}%
\pgfsetfillcolor{currentfill}%
\pgfsetlinewidth{0.602250pt}%
\definecolor{currentstroke}{rgb}{0.000000,0.000000,0.000000}%
\pgfsetstrokecolor{currentstroke}%
\pgfsetdash{}{0pt}%
\pgfsys@defobject{currentmarker}{\pgfqpoint{0.000000in}{-0.027778in}}{\pgfqpoint{0.000000in}{0.000000in}}{%
\pgfpathmoveto{\pgfqpoint{0.000000in}{0.000000in}}%
\pgfpathlineto{\pgfqpoint{0.000000in}{-0.027778in}}%
\pgfusepath{stroke,fill}%
}%
\begin{pgfscope}%
\pgfsys@transformshift{2.797227in}{0.370679in}%
\pgfsys@useobject{currentmarker}{}%
\end{pgfscope}%
\end{pgfscope}%
\begin{pgfscope}%
\pgfsetbuttcap%
\pgfsetroundjoin%
\definecolor{currentfill}{rgb}{0.000000,0.000000,0.000000}%
\pgfsetfillcolor{currentfill}%
\pgfsetlinewidth{0.602250pt}%
\definecolor{currentstroke}{rgb}{0.000000,0.000000,0.000000}%
\pgfsetstrokecolor{currentstroke}%
\pgfsetdash{}{0pt}%
\pgfsys@defobject{currentmarker}{\pgfqpoint{0.000000in}{-0.027778in}}{\pgfqpoint{0.000000in}{0.000000in}}{%
\pgfpathmoveto{\pgfqpoint{0.000000in}{0.000000in}}%
\pgfpathlineto{\pgfqpoint{0.000000in}{-0.027778in}}%
\pgfusepath{stroke,fill}%
}%
\begin{pgfscope}%
\pgfsys@transformshift{2.853893in}{0.370679in}%
\pgfsys@useobject{currentmarker}{}%
\end{pgfscope}%
\end{pgfscope}%
\begin{pgfscope}%
\pgfsetbuttcap%
\pgfsetroundjoin%
\definecolor{currentfill}{rgb}{0.000000,0.000000,0.000000}%
\pgfsetfillcolor{currentfill}%
\pgfsetlinewidth{0.602250pt}%
\definecolor{currentstroke}{rgb}{0.000000,0.000000,0.000000}%
\pgfsetstrokecolor{currentstroke}%
\pgfsetdash{}{0pt}%
\pgfsys@defobject{currentmarker}{\pgfqpoint{0.000000in}{-0.027778in}}{\pgfqpoint{0.000000in}{0.000000in}}{%
\pgfpathmoveto{\pgfqpoint{0.000000in}{0.000000in}}%
\pgfpathlineto{\pgfqpoint{0.000000in}{-0.027778in}}%
\pgfusepath{stroke,fill}%
}%
\begin{pgfscope}%
\pgfsys@transformshift{2.903875in}{0.370679in}%
\pgfsys@useobject{currentmarker}{}%
\end{pgfscope}%
\end{pgfscope}%
\begin{pgfscope}%
\pgfsetbuttcap%
\pgfsetroundjoin%
\definecolor{currentfill}{rgb}{0.000000,0.000000,0.000000}%
\pgfsetfillcolor{currentfill}%
\pgfsetlinewidth{0.602250pt}%
\definecolor{currentstroke}{rgb}{0.000000,0.000000,0.000000}%
\pgfsetstrokecolor{currentstroke}%
\pgfsetdash{}{0pt}%
\pgfsys@defobject{currentmarker}{\pgfqpoint{0.000000in}{-0.027778in}}{\pgfqpoint{0.000000in}{0.000000in}}{%
\pgfpathmoveto{\pgfqpoint{0.000000in}{0.000000in}}%
\pgfpathlineto{\pgfqpoint{0.000000in}{-0.027778in}}%
\pgfusepath{stroke,fill}%
}%
\begin{pgfscope}%
\pgfsys@transformshift{3.242732in}{0.370679in}%
\pgfsys@useobject{currentmarker}{}%
\end{pgfscope}%
\end{pgfscope}%
\begin{pgfscope}%
\pgfsetbuttcap%
\pgfsetroundjoin%
\definecolor{currentfill}{rgb}{0.000000,0.000000,0.000000}%
\pgfsetfillcolor{currentfill}%
\pgfsetlinewidth{0.602250pt}%
\definecolor{currentstroke}{rgb}{0.000000,0.000000,0.000000}%
\pgfsetstrokecolor{currentstroke}%
\pgfsetdash{}{0pt}%
\pgfsys@defobject{currentmarker}{\pgfqpoint{0.000000in}{-0.027778in}}{\pgfqpoint{0.000000in}{0.000000in}}{%
\pgfpathmoveto{\pgfqpoint{0.000000in}{0.000000in}}%
\pgfpathlineto{\pgfqpoint{0.000000in}{-0.027778in}}%
\pgfusepath{stroke,fill}%
}%
\begin{pgfscope}%
\pgfsys@transformshift{3.414796in}{0.370679in}%
\pgfsys@useobject{currentmarker}{}%
\end{pgfscope}%
\end{pgfscope}%
\begin{pgfscope}%
\pgfsetbuttcap%
\pgfsetroundjoin%
\definecolor{currentfill}{rgb}{0.000000,0.000000,0.000000}%
\pgfsetfillcolor{currentfill}%
\pgfsetlinewidth{0.602250pt}%
\definecolor{currentstroke}{rgb}{0.000000,0.000000,0.000000}%
\pgfsetstrokecolor{currentstroke}%
\pgfsetdash{}{0pt}%
\pgfsys@defobject{currentmarker}{\pgfqpoint{0.000000in}{-0.027778in}}{\pgfqpoint{0.000000in}{0.000000in}}{%
\pgfpathmoveto{\pgfqpoint{0.000000in}{0.000000in}}%
\pgfpathlineto{\pgfqpoint{0.000000in}{-0.027778in}}%
\pgfusepath{stroke,fill}%
}%
\begin{pgfscope}%
\pgfsys@transformshift{3.536878in}{0.370679in}%
\pgfsys@useobject{currentmarker}{}%
\end{pgfscope}%
\end{pgfscope}%
\begin{pgfscope}%
\pgfsetbuttcap%
\pgfsetroundjoin%
\definecolor{currentfill}{rgb}{0.000000,0.000000,0.000000}%
\pgfsetfillcolor{currentfill}%
\pgfsetlinewidth{0.602250pt}%
\definecolor{currentstroke}{rgb}{0.000000,0.000000,0.000000}%
\pgfsetstrokecolor{currentstroke}%
\pgfsetdash{}{0pt}%
\pgfsys@defobject{currentmarker}{\pgfqpoint{0.000000in}{-0.027778in}}{\pgfqpoint{0.000000in}{0.000000in}}{%
\pgfpathmoveto{\pgfqpoint{0.000000in}{0.000000in}}%
\pgfpathlineto{\pgfqpoint{0.000000in}{-0.027778in}}%
\pgfusepath{stroke,fill}%
}%
\begin{pgfscope}%
\pgfsys@transformshift{3.631572in}{0.370679in}%
\pgfsys@useobject{currentmarker}{}%
\end{pgfscope}%
\end{pgfscope}%
\begin{pgfscope}%
\pgfsetbuttcap%
\pgfsetroundjoin%
\definecolor{currentfill}{rgb}{0.000000,0.000000,0.000000}%
\pgfsetfillcolor{currentfill}%
\pgfsetlinewidth{0.602250pt}%
\definecolor{currentstroke}{rgb}{0.000000,0.000000,0.000000}%
\pgfsetstrokecolor{currentstroke}%
\pgfsetdash{}{0pt}%
\pgfsys@defobject{currentmarker}{\pgfqpoint{0.000000in}{-0.027778in}}{\pgfqpoint{0.000000in}{0.000000in}}{%
\pgfpathmoveto{\pgfqpoint{0.000000in}{0.000000in}}%
\pgfpathlineto{\pgfqpoint{0.000000in}{-0.027778in}}%
\pgfusepath{stroke,fill}%
}%
\begin{pgfscope}%
\pgfsys@transformshift{3.708942in}{0.370679in}%
\pgfsys@useobject{currentmarker}{}%
\end{pgfscope}%
\end{pgfscope}%
\begin{pgfscope}%
\pgfsetbuttcap%
\pgfsetroundjoin%
\definecolor{currentfill}{rgb}{0.000000,0.000000,0.000000}%
\pgfsetfillcolor{currentfill}%
\pgfsetlinewidth{0.602250pt}%
\definecolor{currentstroke}{rgb}{0.000000,0.000000,0.000000}%
\pgfsetstrokecolor{currentstroke}%
\pgfsetdash{}{0pt}%
\pgfsys@defobject{currentmarker}{\pgfqpoint{0.000000in}{-0.027778in}}{\pgfqpoint{0.000000in}{0.000000in}}{%
\pgfpathmoveto{\pgfqpoint{0.000000in}{0.000000in}}%
\pgfpathlineto{\pgfqpoint{0.000000in}{-0.027778in}}%
\pgfusepath{stroke,fill}%
}%
\begin{pgfscope}%
\pgfsys@transformshift{3.774358in}{0.370679in}%
\pgfsys@useobject{currentmarker}{}%
\end{pgfscope}%
\end{pgfscope}%
\begin{pgfscope}%
\pgfsetbuttcap%
\pgfsetroundjoin%
\definecolor{currentfill}{rgb}{0.000000,0.000000,0.000000}%
\pgfsetfillcolor{currentfill}%
\pgfsetlinewidth{0.602250pt}%
\definecolor{currentstroke}{rgb}{0.000000,0.000000,0.000000}%
\pgfsetstrokecolor{currentstroke}%
\pgfsetdash{}{0pt}%
\pgfsys@defobject{currentmarker}{\pgfqpoint{0.000000in}{-0.027778in}}{\pgfqpoint{0.000000in}{0.000000in}}{%
\pgfpathmoveto{\pgfqpoint{0.000000in}{0.000000in}}%
\pgfpathlineto{\pgfqpoint{0.000000in}{-0.027778in}}%
\pgfusepath{stroke,fill}%
}%
\begin{pgfscope}%
\pgfsys@transformshift{3.831024in}{0.370679in}%
\pgfsys@useobject{currentmarker}{}%
\end{pgfscope}%
\end{pgfscope}%
\begin{pgfscope}%
\pgfsetbuttcap%
\pgfsetroundjoin%
\definecolor{currentfill}{rgb}{0.000000,0.000000,0.000000}%
\pgfsetfillcolor{currentfill}%
\pgfsetlinewidth{0.602250pt}%
\definecolor{currentstroke}{rgb}{0.000000,0.000000,0.000000}%
\pgfsetstrokecolor{currentstroke}%
\pgfsetdash{}{0pt}%
\pgfsys@defobject{currentmarker}{\pgfqpoint{0.000000in}{-0.027778in}}{\pgfqpoint{0.000000in}{0.000000in}}{%
\pgfpathmoveto{\pgfqpoint{0.000000in}{0.000000in}}%
\pgfpathlineto{\pgfqpoint{0.000000in}{-0.027778in}}%
\pgfusepath{stroke,fill}%
}%
\begin{pgfscope}%
\pgfsys@transformshift{3.881006in}{0.370679in}%
\pgfsys@useobject{currentmarker}{}%
\end{pgfscope}%
\end{pgfscope}%
\begin{pgfscope}%
\pgfsetbuttcap%
\pgfsetroundjoin%
\definecolor{currentfill}{rgb}{0.000000,0.000000,0.000000}%
\pgfsetfillcolor{currentfill}%
\pgfsetlinewidth{0.602250pt}%
\definecolor{currentstroke}{rgb}{0.000000,0.000000,0.000000}%
\pgfsetstrokecolor{currentstroke}%
\pgfsetdash{}{0pt}%
\pgfsys@defobject{currentmarker}{\pgfqpoint{0.000000in}{-0.027778in}}{\pgfqpoint{0.000000in}{0.000000in}}{%
\pgfpathmoveto{\pgfqpoint{0.000000in}{0.000000in}}%
\pgfpathlineto{\pgfqpoint{0.000000in}{-0.027778in}}%
\pgfusepath{stroke,fill}%
}%
\begin{pgfscope}%
\pgfsys@transformshift{4.219863in}{0.370679in}%
\pgfsys@useobject{currentmarker}{}%
\end{pgfscope}%
\end{pgfscope}%
\begin{pgfscope}%
\pgfsetbuttcap%
\pgfsetroundjoin%
\definecolor{currentfill}{rgb}{0.000000,0.000000,0.000000}%
\pgfsetfillcolor{currentfill}%
\pgfsetlinewidth{0.602250pt}%
\definecolor{currentstroke}{rgb}{0.000000,0.000000,0.000000}%
\pgfsetstrokecolor{currentstroke}%
\pgfsetdash{}{0pt}%
\pgfsys@defobject{currentmarker}{\pgfqpoint{0.000000in}{-0.027778in}}{\pgfqpoint{0.000000in}{0.000000in}}{%
\pgfpathmoveto{\pgfqpoint{0.000000in}{0.000000in}}%
\pgfpathlineto{\pgfqpoint{0.000000in}{-0.027778in}}%
\pgfusepath{stroke,fill}%
}%
\begin{pgfscope}%
\pgfsys@transformshift{4.391927in}{0.370679in}%
\pgfsys@useobject{currentmarker}{}%
\end{pgfscope}%
\end{pgfscope}%
\begin{pgfscope}%
\pgfsetbuttcap%
\pgfsetroundjoin%
\definecolor{currentfill}{rgb}{0.000000,0.000000,0.000000}%
\pgfsetfillcolor{currentfill}%
\pgfsetlinewidth{0.602250pt}%
\definecolor{currentstroke}{rgb}{0.000000,0.000000,0.000000}%
\pgfsetstrokecolor{currentstroke}%
\pgfsetdash{}{0pt}%
\pgfsys@defobject{currentmarker}{\pgfqpoint{0.000000in}{-0.027778in}}{\pgfqpoint{0.000000in}{0.000000in}}{%
\pgfpathmoveto{\pgfqpoint{0.000000in}{0.000000in}}%
\pgfpathlineto{\pgfqpoint{0.000000in}{-0.027778in}}%
\pgfusepath{stroke,fill}%
}%
\begin{pgfscope}%
\pgfsys@transformshift{4.514009in}{0.370679in}%
\pgfsys@useobject{currentmarker}{}%
\end{pgfscope}%
\end{pgfscope}%
\begin{pgfscope}%
\pgfsetbuttcap%
\pgfsetroundjoin%
\definecolor{currentfill}{rgb}{0.000000,0.000000,0.000000}%
\pgfsetfillcolor{currentfill}%
\pgfsetlinewidth{0.602250pt}%
\definecolor{currentstroke}{rgb}{0.000000,0.000000,0.000000}%
\pgfsetstrokecolor{currentstroke}%
\pgfsetdash{}{0pt}%
\pgfsys@defobject{currentmarker}{\pgfqpoint{0.000000in}{-0.027778in}}{\pgfqpoint{0.000000in}{0.000000in}}{%
\pgfpathmoveto{\pgfqpoint{0.000000in}{0.000000in}}%
\pgfpathlineto{\pgfqpoint{0.000000in}{-0.027778in}}%
\pgfusepath{stroke,fill}%
}%
\begin{pgfscope}%
\pgfsys@transformshift{4.608703in}{0.370679in}%
\pgfsys@useobject{currentmarker}{}%
\end{pgfscope}%
\end{pgfscope}%
\begin{pgfscope}%
\pgfsetbuttcap%
\pgfsetroundjoin%
\definecolor{currentfill}{rgb}{0.000000,0.000000,0.000000}%
\pgfsetfillcolor{currentfill}%
\pgfsetlinewidth{0.602250pt}%
\definecolor{currentstroke}{rgb}{0.000000,0.000000,0.000000}%
\pgfsetstrokecolor{currentstroke}%
\pgfsetdash{}{0pt}%
\pgfsys@defobject{currentmarker}{\pgfqpoint{0.000000in}{-0.027778in}}{\pgfqpoint{0.000000in}{0.000000in}}{%
\pgfpathmoveto{\pgfqpoint{0.000000in}{0.000000in}}%
\pgfpathlineto{\pgfqpoint{0.000000in}{-0.027778in}}%
\pgfusepath{stroke,fill}%
}%
\begin{pgfscope}%
\pgfsys@transformshift{4.686073in}{0.370679in}%
\pgfsys@useobject{currentmarker}{}%
\end{pgfscope}%
\end{pgfscope}%
\begin{pgfscope}%
\pgfsetbuttcap%
\pgfsetroundjoin%
\definecolor{currentfill}{rgb}{0.000000,0.000000,0.000000}%
\pgfsetfillcolor{currentfill}%
\pgfsetlinewidth{0.602250pt}%
\definecolor{currentstroke}{rgb}{0.000000,0.000000,0.000000}%
\pgfsetstrokecolor{currentstroke}%
\pgfsetdash{}{0pt}%
\pgfsys@defobject{currentmarker}{\pgfqpoint{0.000000in}{-0.027778in}}{\pgfqpoint{0.000000in}{0.000000in}}{%
\pgfpathmoveto{\pgfqpoint{0.000000in}{0.000000in}}%
\pgfpathlineto{\pgfqpoint{0.000000in}{-0.027778in}}%
\pgfusepath{stroke,fill}%
}%
\begin{pgfscope}%
\pgfsys@transformshift{4.751489in}{0.370679in}%
\pgfsys@useobject{currentmarker}{}%
\end{pgfscope}%
\end{pgfscope}%
\begin{pgfscope}%
\pgfsetbuttcap%
\pgfsetroundjoin%
\definecolor{currentfill}{rgb}{0.000000,0.000000,0.000000}%
\pgfsetfillcolor{currentfill}%
\pgfsetlinewidth{0.602250pt}%
\definecolor{currentstroke}{rgb}{0.000000,0.000000,0.000000}%
\pgfsetstrokecolor{currentstroke}%
\pgfsetdash{}{0pt}%
\pgfsys@defobject{currentmarker}{\pgfqpoint{0.000000in}{-0.027778in}}{\pgfqpoint{0.000000in}{0.000000in}}{%
\pgfpathmoveto{\pgfqpoint{0.000000in}{0.000000in}}%
\pgfpathlineto{\pgfqpoint{0.000000in}{-0.027778in}}%
\pgfusepath{stroke,fill}%
}%
\begin{pgfscope}%
\pgfsys@transformshift{4.808155in}{0.370679in}%
\pgfsys@useobject{currentmarker}{}%
\end{pgfscope}%
\end{pgfscope}%
\begin{pgfscope}%
\pgfsetbuttcap%
\pgfsetroundjoin%
\definecolor{currentfill}{rgb}{0.000000,0.000000,0.000000}%
\pgfsetfillcolor{currentfill}%
\pgfsetlinewidth{0.602250pt}%
\definecolor{currentstroke}{rgb}{0.000000,0.000000,0.000000}%
\pgfsetstrokecolor{currentstroke}%
\pgfsetdash{}{0pt}%
\pgfsys@defobject{currentmarker}{\pgfqpoint{0.000000in}{-0.027778in}}{\pgfqpoint{0.000000in}{0.000000in}}{%
\pgfpathmoveto{\pgfqpoint{0.000000in}{0.000000in}}%
\pgfpathlineto{\pgfqpoint{0.000000in}{-0.027778in}}%
\pgfusepath{stroke,fill}%
}%
\begin{pgfscope}%
\pgfsys@transformshift{4.858137in}{0.370679in}%
\pgfsys@useobject{currentmarker}{}%
\end{pgfscope}%
\end{pgfscope}%
\begin{pgfscope}%
\pgfsetbuttcap%
\pgfsetroundjoin%
\definecolor{currentfill}{rgb}{0.000000,0.000000,0.000000}%
\pgfsetfillcolor{currentfill}%
\pgfsetlinewidth{0.602250pt}%
\definecolor{currentstroke}{rgb}{0.000000,0.000000,0.000000}%
\pgfsetstrokecolor{currentstroke}%
\pgfsetdash{}{0pt}%
\pgfsys@defobject{currentmarker}{\pgfqpoint{0.000000in}{-0.027778in}}{\pgfqpoint{0.000000in}{0.000000in}}{%
\pgfpathmoveto{\pgfqpoint{0.000000in}{0.000000in}}%
\pgfpathlineto{\pgfqpoint{0.000000in}{-0.027778in}}%
\pgfusepath{stroke,fill}%
}%
\begin{pgfscope}%
\pgfsys@transformshift{5.196994in}{0.370679in}%
\pgfsys@useobject{currentmarker}{}%
\end{pgfscope}%
\end{pgfscope}%
\begin{pgfscope}%
\pgfsetbuttcap%
\pgfsetroundjoin%
\definecolor{currentfill}{rgb}{0.000000,0.000000,0.000000}%
\pgfsetfillcolor{currentfill}%
\pgfsetlinewidth{0.602250pt}%
\definecolor{currentstroke}{rgb}{0.000000,0.000000,0.000000}%
\pgfsetstrokecolor{currentstroke}%
\pgfsetdash{}{0pt}%
\pgfsys@defobject{currentmarker}{\pgfqpoint{0.000000in}{-0.027778in}}{\pgfqpoint{0.000000in}{0.000000in}}{%
\pgfpathmoveto{\pgfqpoint{0.000000in}{0.000000in}}%
\pgfpathlineto{\pgfqpoint{0.000000in}{-0.027778in}}%
\pgfusepath{stroke,fill}%
}%
\begin{pgfscope}%
\pgfsys@transformshift{5.369058in}{0.370679in}%
\pgfsys@useobject{currentmarker}{}%
\end{pgfscope}%
\end{pgfscope}%
\begin{pgfscope}%
\pgfsetbuttcap%
\pgfsetroundjoin%
\definecolor{currentfill}{rgb}{0.000000,0.000000,0.000000}%
\pgfsetfillcolor{currentfill}%
\pgfsetlinewidth{0.602250pt}%
\definecolor{currentstroke}{rgb}{0.000000,0.000000,0.000000}%
\pgfsetstrokecolor{currentstroke}%
\pgfsetdash{}{0pt}%
\pgfsys@defobject{currentmarker}{\pgfqpoint{0.000000in}{-0.027778in}}{\pgfqpoint{0.000000in}{0.000000in}}{%
\pgfpathmoveto{\pgfqpoint{0.000000in}{0.000000in}}%
\pgfpathlineto{\pgfqpoint{0.000000in}{-0.027778in}}%
\pgfusepath{stroke,fill}%
}%
\begin{pgfscope}%
\pgfsys@transformshift{5.491140in}{0.370679in}%
\pgfsys@useobject{currentmarker}{}%
\end{pgfscope}%
\end{pgfscope}%
\begin{pgfscope}%
\pgfsetbuttcap%
\pgfsetroundjoin%
\definecolor{currentfill}{rgb}{0.000000,0.000000,0.000000}%
\pgfsetfillcolor{currentfill}%
\pgfsetlinewidth{0.602250pt}%
\definecolor{currentstroke}{rgb}{0.000000,0.000000,0.000000}%
\pgfsetstrokecolor{currentstroke}%
\pgfsetdash{}{0pt}%
\pgfsys@defobject{currentmarker}{\pgfqpoint{0.000000in}{-0.027778in}}{\pgfqpoint{0.000000in}{0.000000in}}{%
\pgfpathmoveto{\pgfqpoint{0.000000in}{0.000000in}}%
\pgfpathlineto{\pgfqpoint{0.000000in}{-0.027778in}}%
\pgfusepath{stroke,fill}%
}%
\begin{pgfscope}%
\pgfsys@transformshift{5.585834in}{0.370679in}%
\pgfsys@useobject{currentmarker}{}%
\end{pgfscope}%
\end{pgfscope}%
\begin{pgfscope}%
\pgfsetbuttcap%
\pgfsetroundjoin%
\definecolor{currentfill}{rgb}{0.000000,0.000000,0.000000}%
\pgfsetfillcolor{currentfill}%
\pgfsetlinewidth{0.602250pt}%
\definecolor{currentstroke}{rgb}{0.000000,0.000000,0.000000}%
\pgfsetstrokecolor{currentstroke}%
\pgfsetdash{}{0pt}%
\pgfsys@defobject{currentmarker}{\pgfqpoint{0.000000in}{-0.027778in}}{\pgfqpoint{0.000000in}{0.000000in}}{%
\pgfpathmoveto{\pgfqpoint{0.000000in}{0.000000in}}%
\pgfpathlineto{\pgfqpoint{0.000000in}{-0.027778in}}%
\pgfusepath{stroke,fill}%
}%
\begin{pgfscope}%
\pgfsys@transformshift{5.663204in}{0.370679in}%
\pgfsys@useobject{currentmarker}{}%
\end{pgfscope}%
\end{pgfscope}%
\begin{pgfscope}%
\pgfsetbuttcap%
\pgfsetroundjoin%
\definecolor{currentfill}{rgb}{0.000000,0.000000,0.000000}%
\pgfsetfillcolor{currentfill}%
\pgfsetlinewidth{0.602250pt}%
\definecolor{currentstroke}{rgb}{0.000000,0.000000,0.000000}%
\pgfsetstrokecolor{currentstroke}%
\pgfsetdash{}{0pt}%
\pgfsys@defobject{currentmarker}{\pgfqpoint{0.000000in}{-0.027778in}}{\pgfqpoint{0.000000in}{0.000000in}}{%
\pgfpathmoveto{\pgfqpoint{0.000000in}{0.000000in}}%
\pgfpathlineto{\pgfqpoint{0.000000in}{-0.027778in}}%
\pgfusepath{stroke,fill}%
}%
\begin{pgfscope}%
\pgfsys@transformshift{5.728620in}{0.370679in}%
\pgfsys@useobject{currentmarker}{}%
\end{pgfscope}%
\end{pgfscope}%
\begin{pgfscope}%
\pgfsetbuttcap%
\pgfsetroundjoin%
\definecolor{currentfill}{rgb}{0.000000,0.000000,0.000000}%
\pgfsetfillcolor{currentfill}%
\pgfsetlinewidth{0.602250pt}%
\definecolor{currentstroke}{rgb}{0.000000,0.000000,0.000000}%
\pgfsetstrokecolor{currentstroke}%
\pgfsetdash{}{0pt}%
\pgfsys@defobject{currentmarker}{\pgfqpoint{0.000000in}{-0.027778in}}{\pgfqpoint{0.000000in}{0.000000in}}{%
\pgfpathmoveto{\pgfqpoint{0.000000in}{0.000000in}}%
\pgfpathlineto{\pgfqpoint{0.000000in}{-0.027778in}}%
\pgfusepath{stroke,fill}%
}%
\begin{pgfscope}%
\pgfsys@transformshift{5.785286in}{0.370679in}%
\pgfsys@useobject{currentmarker}{}%
\end{pgfscope}%
\end{pgfscope}%
\begin{pgfscope}%
\pgfsetbuttcap%
\pgfsetroundjoin%
\definecolor{currentfill}{rgb}{0.000000,0.000000,0.000000}%
\pgfsetfillcolor{currentfill}%
\pgfsetlinewidth{0.602250pt}%
\definecolor{currentstroke}{rgb}{0.000000,0.000000,0.000000}%
\pgfsetstrokecolor{currentstroke}%
\pgfsetdash{}{0pt}%
\pgfsys@defobject{currentmarker}{\pgfqpoint{0.000000in}{-0.027778in}}{\pgfqpoint{0.000000in}{0.000000in}}{%
\pgfpathmoveto{\pgfqpoint{0.000000in}{0.000000in}}%
\pgfpathlineto{\pgfqpoint{0.000000in}{-0.027778in}}%
\pgfusepath{stroke,fill}%
}%
\begin{pgfscope}%
\pgfsys@transformshift{5.835268in}{0.370679in}%
\pgfsys@useobject{currentmarker}{}%
\end{pgfscope}%
\end{pgfscope}%
\begin{pgfscope}%
\pgfsetbuttcap%
\pgfsetroundjoin%
\definecolor{currentfill}{rgb}{0.000000,0.000000,0.000000}%
\pgfsetfillcolor{currentfill}%
\pgfsetlinewidth{0.602250pt}%
\definecolor{currentstroke}{rgb}{0.000000,0.000000,0.000000}%
\pgfsetstrokecolor{currentstroke}%
\pgfsetdash{}{0pt}%
\pgfsys@defobject{currentmarker}{\pgfqpoint{0.000000in}{-0.027778in}}{\pgfqpoint{0.000000in}{0.000000in}}{%
\pgfpathmoveto{\pgfqpoint{0.000000in}{0.000000in}}%
\pgfpathlineto{\pgfqpoint{0.000000in}{-0.027778in}}%
\pgfusepath{stroke,fill}%
}%
\begin{pgfscope}%
\pgfsys@transformshift{6.174125in}{0.370679in}%
\pgfsys@useobject{currentmarker}{}%
\end{pgfscope}%
\end{pgfscope}%
\begin{pgfscope}%
\pgfsetbuttcap%
\pgfsetroundjoin%
\definecolor{currentfill}{rgb}{0.000000,0.000000,0.000000}%
\pgfsetfillcolor{currentfill}%
\pgfsetlinewidth{0.602250pt}%
\definecolor{currentstroke}{rgb}{0.000000,0.000000,0.000000}%
\pgfsetstrokecolor{currentstroke}%
\pgfsetdash{}{0pt}%
\pgfsys@defobject{currentmarker}{\pgfqpoint{0.000000in}{-0.027778in}}{\pgfqpoint{0.000000in}{0.000000in}}{%
\pgfpathmoveto{\pgfqpoint{0.000000in}{0.000000in}}%
\pgfpathlineto{\pgfqpoint{0.000000in}{-0.027778in}}%
\pgfusepath{stroke,fill}%
}%
\begin{pgfscope}%
\pgfsys@transformshift{6.346189in}{0.370679in}%
\pgfsys@useobject{currentmarker}{}%
\end{pgfscope}%
\end{pgfscope}%
\begin{pgfscope}%
\pgfsetbuttcap%
\pgfsetroundjoin%
\definecolor{currentfill}{rgb}{0.000000,0.000000,0.000000}%
\pgfsetfillcolor{currentfill}%
\pgfsetlinewidth{0.602250pt}%
\definecolor{currentstroke}{rgb}{0.000000,0.000000,0.000000}%
\pgfsetstrokecolor{currentstroke}%
\pgfsetdash{}{0pt}%
\pgfsys@defobject{currentmarker}{\pgfqpoint{0.000000in}{-0.027778in}}{\pgfqpoint{0.000000in}{0.000000in}}{%
\pgfpathmoveto{\pgfqpoint{0.000000in}{0.000000in}}%
\pgfpathlineto{\pgfqpoint{0.000000in}{-0.027778in}}%
\pgfusepath{stroke,fill}%
}%
\begin{pgfscope}%
\pgfsys@transformshift{6.468271in}{0.370679in}%
\pgfsys@useobject{currentmarker}{}%
\end{pgfscope}%
\end{pgfscope}%
\begin{pgfscope}%
\pgfsetbuttcap%
\pgfsetroundjoin%
\definecolor{currentfill}{rgb}{0.000000,0.000000,0.000000}%
\pgfsetfillcolor{currentfill}%
\pgfsetlinewidth{0.602250pt}%
\definecolor{currentstroke}{rgb}{0.000000,0.000000,0.000000}%
\pgfsetstrokecolor{currentstroke}%
\pgfsetdash{}{0pt}%
\pgfsys@defobject{currentmarker}{\pgfqpoint{0.000000in}{-0.027778in}}{\pgfqpoint{0.000000in}{0.000000in}}{%
\pgfpathmoveto{\pgfqpoint{0.000000in}{0.000000in}}%
\pgfpathlineto{\pgfqpoint{0.000000in}{-0.027778in}}%
\pgfusepath{stroke,fill}%
}%
\begin{pgfscope}%
\pgfsys@transformshift{6.562965in}{0.370679in}%
\pgfsys@useobject{currentmarker}{}%
\end{pgfscope}%
\end{pgfscope}%
\begin{pgfscope}%
\pgfsetbuttcap%
\pgfsetroundjoin%
\definecolor{currentfill}{rgb}{0.000000,0.000000,0.000000}%
\pgfsetfillcolor{currentfill}%
\pgfsetlinewidth{0.602250pt}%
\definecolor{currentstroke}{rgb}{0.000000,0.000000,0.000000}%
\pgfsetstrokecolor{currentstroke}%
\pgfsetdash{}{0pt}%
\pgfsys@defobject{currentmarker}{\pgfqpoint{0.000000in}{-0.027778in}}{\pgfqpoint{0.000000in}{0.000000in}}{%
\pgfpathmoveto{\pgfqpoint{0.000000in}{0.000000in}}%
\pgfpathlineto{\pgfqpoint{0.000000in}{-0.027778in}}%
\pgfusepath{stroke,fill}%
}%
\begin{pgfscope}%
\pgfsys@transformshift{6.640335in}{0.370679in}%
\pgfsys@useobject{currentmarker}{}%
\end{pgfscope}%
\end{pgfscope}%
\begin{pgfscope}%
\pgfsetbuttcap%
\pgfsetroundjoin%
\definecolor{currentfill}{rgb}{0.000000,0.000000,0.000000}%
\pgfsetfillcolor{currentfill}%
\pgfsetlinewidth{0.602250pt}%
\definecolor{currentstroke}{rgb}{0.000000,0.000000,0.000000}%
\pgfsetstrokecolor{currentstroke}%
\pgfsetdash{}{0pt}%
\pgfsys@defobject{currentmarker}{\pgfqpoint{0.000000in}{-0.027778in}}{\pgfqpoint{0.000000in}{0.000000in}}{%
\pgfpathmoveto{\pgfqpoint{0.000000in}{0.000000in}}%
\pgfpathlineto{\pgfqpoint{0.000000in}{-0.027778in}}%
\pgfusepath{stroke,fill}%
}%
\begin{pgfscope}%
\pgfsys@transformshift{6.705751in}{0.370679in}%
\pgfsys@useobject{currentmarker}{}%
\end{pgfscope}%
\end{pgfscope}%
\begin{pgfscope}%
\pgfsetbuttcap%
\pgfsetroundjoin%
\definecolor{currentfill}{rgb}{0.000000,0.000000,0.000000}%
\pgfsetfillcolor{currentfill}%
\pgfsetlinewidth{0.602250pt}%
\definecolor{currentstroke}{rgb}{0.000000,0.000000,0.000000}%
\pgfsetstrokecolor{currentstroke}%
\pgfsetdash{}{0pt}%
\pgfsys@defobject{currentmarker}{\pgfqpoint{0.000000in}{-0.027778in}}{\pgfqpoint{0.000000in}{0.000000in}}{%
\pgfpathmoveto{\pgfqpoint{0.000000in}{0.000000in}}%
\pgfpathlineto{\pgfqpoint{0.000000in}{-0.027778in}}%
\pgfusepath{stroke,fill}%
}%
\begin{pgfscope}%
\pgfsys@transformshift{6.762417in}{0.370679in}%
\pgfsys@useobject{currentmarker}{}%
\end{pgfscope}%
\end{pgfscope}%
\begin{pgfscope}%
\pgfsetbuttcap%
\pgfsetroundjoin%
\definecolor{currentfill}{rgb}{0.000000,0.000000,0.000000}%
\pgfsetfillcolor{currentfill}%
\pgfsetlinewidth{0.602250pt}%
\definecolor{currentstroke}{rgb}{0.000000,0.000000,0.000000}%
\pgfsetstrokecolor{currentstroke}%
\pgfsetdash{}{0pt}%
\pgfsys@defobject{currentmarker}{\pgfqpoint{0.000000in}{-0.027778in}}{\pgfqpoint{0.000000in}{0.000000in}}{%
\pgfpathmoveto{\pgfqpoint{0.000000in}{0.000000in}}%
\pgfpathlineto{\pgfqpoint{0.000000in}{-0.027778in}}%
\pgfusepath{stroke,fill}%
}%
\begin{pgfscope}%
\pgfsys@transformshift{6.812399in}{0.370679in}%
\pgfsys@useobject{currentmarker}{}%
\end{pgfscope}%
\end{pgfscope}%
\begin{pgfscope}%
\pgfsetbuttcap%
\pgfsetroundjoin%
\definecolor{currentfill}{rgb}{0.000000,0.000000,0.000000}%
\pgfsetfillcolor{currentfill}%
\pgfsetlinewidth{0.803000pt}%
\definecolor{currentstroke}{rgb}{0.000000,0.000000,0.000000}%
\pgfsetstrokecolor{currentstroke}%
\pgfsetdash{}{0pt}%
\pgfsys@defobject{currentmarker}{\pgfqpoint{-0.048611in}{0.000000in}}{\pgfqpoint{-0.000000in}{0.000000in}}{%
\pgfpathmoveto{\pgfqpoint{-0.000000in}{0.000000in}}%
\pgfpathlineto{\pgfqpoint{-0.048611in}{0.000000in}}%
\pgfusepath{stroke,fill}%
}%
\begin{pgfscope}%
\pgfsys@transformshift{0.535225in}{0.370679in}%
\pgfsys@useobject{currentmarker}{}%
\end{pgfscope}%
\end{pgfscope}%
\begin{pgfscope}%
\definecolor{textcolor}{rgb}{0.000000,0.000000,0.000000}%
\pgfsetstrokecolor{textcolor}%
\pgfsetfillcolor{textcolor}%
\pgftext[x=0.150000in, y=0.322454in, left, base]{\color{textcolor}\rmfamily\fontsize{10.000000}{12.000000}\selectfont \(\displaystyle {10^{-3}}\)}%
\end{pgfscope}%
\begin{pgfscope}%
\pgfsetbuttcap%
\pgfsetroundjoin%
\definecolor{currentfill}{rgb}{0.000000,0.000000,0.000000}%
\pgfsetfillcolor{currentfill}%
\pgfsetlinewidth{0.803000pt}%
\definecolor{currentstroke}{rgb}{0.000000,0.000000,0.000000}%
\pgfsetstrokecolor{currentstroke}%
\pgfsetdash{}{0pt}%
\pgfsys@defobject{currentmarker}{\pgfqpoint{-0.048611in}{0.000000in}}{\pgfqpoint{-0.000000in}{0.000000in}}{%
\pgfpathmoveto{\pgfqpoint{-0.000000in}{0.000000in}}%
\pgfpathlineto{\pgfqpoint{-0.048611in}{0.000000in}}%
\pgfusepath{stroke,fill}%
}%
\begin{pgfscope}%
\pgfsys@transformshift{0.535225in}{1.044243in}%
\pgfsys@useobject{currentmarker}{}%
\end{pgfscope}%
\end{pgfscope}%
\begin{pgfscope}%
\definecolor{textcolor}{rgb}{0.000000,0.000000,0.000000}%
\pgfsetstrokecolor{textcolor}%
\pgfsetfillcolor{textcolor}%
\pgftext[x=0.150000in, y=0.996017in, left, base]{\color{textcolor}\rmfamily\fontsize{10.000000}{12.000000}\selectfont \(\displaystyle {10^{-1}}\)}%
\end{pgfscope}%
\begin{pgfscope}%
\pgfsetbuttcap%
\pgfsetroundjoin%
\definecolor{currentfill}{rgb}{0.000000,0.000000,0.000000}%
\pgfsetfillcolor{currentfill}%
\pgfsetlinewidth{0.803000pt}%
\definecolor{currentstroke}{rgb}{0.000000,0.000000,0.000000}%
\pgfsetstrokecolor{currentstroke}%
\pgfsetdash{}{0pt}%
\pgfsys@defobject{currentmarker}{\pgfqpoint{-0.048611in}{0.000000in}}{\pgfqpoint{-0.000000in}{0.000000in}}{%
\pgfpathmoveto{\pgfqpoint{-0.000000in}{0.000000in}}%
\pgfpathlineto{\pgfqpoint{-0.048611in}{0.000000in}}%
\pgfusepath{stroke,fill}%
}%
\begin{pgfscope}%
\pgfsys@transformshift{0.535225in}{1.717807in}%
\pgfsys@useobject{currentmarker}{}%
\end{pgfscope}%
\end{pgfscope}%
\begin{pgfscope}%
\definecolor{textcolor}{rgb}{0.000000,0.000000,0.000000}%
\pgfsetstrokecolor{textcolor}%
\pgfsetfillcolor{textcolor}%
\pgftext[x=0.236806in, y=1.669581in, left, base]{\color{textcolor}\rmfamily\fontsize{10.000000}{12.000000}\selectfont \(\displaystyle {10^{1}}\)}%
\end{pgfscope}%
\begin{pgfscope}%
\pgfsetbuttcap%
\pgfsetroundjoin%
\definecolor{currentfill}{rgb}{0.000000,0.000000,0.000000}%
\pgfsetfillcolor{currentfill}%
\pgfsetlinewidth{0.803000pt}%
\definecolor{currentstroke}{rgb}{0.000000,0.000000,0.000000}%
\pgfsetstrokecolor{currentstroke}%
\pgfsetdash{}{0pt}%
\pgfsys@defobject{currentmarker}{\pgfqpoint{-0.048611in}{0.000000in}}{\pgfqpoint{-0.000000in}{0.000000in}}{%
\pgfpathmoveto{\pgfqpoint{-0.000000in}{0.000000in}}%
\pgfpathlineto{\pgfqpoint{-0.048611in}{0.000000in}}%
\pgfusepath{stroke,fill}%
}%
\begin{pgfscope}%
\pgfsys@transformshift{0.535225in}{2.391370in}%
\pgfsys@useobject{currentmarker}{}%
\end{pgfscope}%
\end{pgfscope}%
\begin{pgfscope}%
\definecolor{textcolor}{rgb}{0.000000,0.000000,0.000000}%
\pgfsetstrokecolor{textcolor}%
\pgfsetfillcolor{textcolor}%
\pgftext[x=0.236806in, y=2.343145in, left, base]{\color{textcolor}\rmfamily\fontsize{10.000000}{12.000000}\selectfont \(\displaystyle {10^{3}}\)}%
\end{pgfscope}%
\begin{pgfscope}%
\pgfsetbuttcap%
\pgfsetroundjoin%
\definecolor{currentfill}{rgb}{0.000000,0.000000,0.000000}%
\pgfsetfillcolor{currentfill}%
\pgfsetlinewidth{0.803000pt}%
\definecolor{currentstroke}{rgb}{0.000000,0.000000,0.000000}%
\pgfsetstrokecolor{currentstroke}%
\pgfsetdash{}{0pt}%
\pgfsys@defobject{currentmarker}{\pgfqpoint{-0.048611in}{0.000000in}}{\pgfqpoint{-0.000000in}{0.000000in}}{%
\pgfpathmoveto{\pgfqpoint{-0.000000in}{0.000000in}}%
\pgfpathlineto{\pgfqpoint{-0.048611in}{0.000000in}}%
\pgfusepath{stroke,fill}%
}%
\begin{pgfscope}%
\pgfsys@transformshift{0.535225in}{3.064934in}%
\pgfsys@useobject{currentmarker}{}%
\end{pgfscope}%
\end{pgfscope}%
\begin{pgfscope}%
\definecolor{textcolor}{rgb}{0.000000,0.000000,0.000000}%
\pgfsetstrokecolor{textcolor}%
\pgfsetfillcolor{textcolor}%
\pgftext[x=0.236806in, y=3.016709in, left, base]{\color{textcolor}\rmfamily\fontsize{10.000000}{12.000000}\selectfont \(\displaystyle {10^{5}}\)}%
\end{pgfscope}%
\begin{pgfscope}%
\pgfpathrectangle{\pgfqpoint{0.535225in}{0.370679in}}{\pgfqpoint{6.314775in}{3.181174in}}%
\pgfusepath{clip}%
\pgfsetrectcap%
\pgfsetroundjoin%
\pgfsetlinewidth{3.011250pt}%
\definecolor{currentstroke}{rgb}{0.647059,0.164706,0.164706}%
\pgfsetstrokecolor{currentstroke}%
\pgfsetdash{}{0pt}%
\pgfpathmoveto{\pgfqpoint{0.525225in}{3.183079in}}%
\pgfpathlineto{\pgfqpoint{0.822260in}{3.184982in}}%
\pgfpathlineto{\pgfqpoint{1.116406in}{3.238099in}}%
\pgfpathlineto{\pgfqpoint{1.288470in}{3.192155in}}%
\pgfpathlineto{\pgfqpoint{1.410551in}{3.138180in}}%
\pgfpathlineto{\pgfqpoint{1.505245in}{3.039883in}}%
\pgfpathlineto{\pgfqpoint{1.582616in}{3.068757in}}%
\pgfpathlineto{\pgfqpoint{1.648031in}{2.877575in}}%
\pgfpathlineto{\pgfqpoint{1.704697in}{3.123703in}}%
\pgfpathlineto{\pgfqpoint{1.754680in}{3.015596in}}%
\pgfpathlineto{\pgfqpoint{1.799391in}{3.019863in}}%
\pgfpathlineto{\pgfqpoint{1.839837in}{3.043778in}}%
\pgfpathlineto{\pgfqpoint{1.876761in}{2.977863in}}%
\pgfpathlineto{\pgfqpoint{1.942177in}{3.098622in}}%
\pgfpathlineto{\pgfqpoint{1.971455in}{2.900839in}}%
\pgfpathlineto{\pgfqpoint{1.998843in}{3.013074in}}%
\pgfpathlineto{\pgfqpoint{2.024570in}{2.906145in}}%
\pgfpathlineto{\pgfqpoint{2.048826in}{2.966057in}}%
\pgfpathlineto{\pgfqpoint{2.071770in}{2.870468in}}%
\pgfpathlineto{\pgfqpoint{2.093537in}{2.913393in}}%
\pgfpathlineto{\pgfqpoint{2.114241in}{2.806050in}}%
\pgfpathlineto{\pgfqpoint{2.133983in}{2.954615in}}%
\pgfpathlineto{\pgfqpoint{2.152847in}{2.693923in}}%
\pgfpathlineto{\pgfqpoint{2.170907in}{2.682714in}}%
\pgfpathlineto{\pgfqpoint{2.188231in}{2.826904in}}%
\pgfpathlineto{\pgfqpoint{2.204874in}{2.842307in}}%
\pgfpathlineto{\pgfqpoint{2.220890in}{2.833597in}}%
\pgfpathlineto{\pgfqpoint{2.236323in}{2.947360in}}%
\pgfpathlineto{\pgfqpoint{2.251214in}{2.857789in}}%
\pgfpathlineto{\pgfqpoint{2.265601in}{2.898263in}}%
\pgfpathlineto{\pgfqpoint{2.279516in}{2.776041in}}%
\pgfpathlineto{\pgfqpoint{2.292989in}{2.835146in}}%
\pgfpathlineto{\pgfqpoint{2.306047in}{2.812752in}}%
\pgfpathlineto{\pgfqpoint{2.318716in}{2.851845in}}%
\pgfpathlineto{\pgfqpoint{2.331017in}{2.768144in}}%
\pgfpathlineto{\pgfqpoint{2.342971in}{2.885926in}}%
\pgfpathlineto{\pgfqpoint{2.354599in}{2.806745in}}%
\pgfpathlineto{\pgfqpoint{2.365916in}{2.858717in}}%
\pgfpathlineto{\pgfqpoint{2.376939in}{2.783611in}}%
\pgfpathlineto{\pgfqpoint{2.387683in}{2.892012in}}%
\pgfpathlineto{\pgfqpoint{2.398161in}{2.884724in}}%
\pgfpathlineto{\pgfqpoint{2.408387in}{2.829296in}}%
\pgfpathlineto{\pgfqpoint{2.418373in}{2.802414in}}%
\pgfpathlineto{\pgfqpoint{2.428129in}{2.853610in}}%
\pgfpathlineto{\pgfqpoint{2.437665in}{2.877111in}}%
\pgfpathlineto{\pgfqpoint{2.446992in}{2.868040in}}%
\pgfpathlineto{\pgfqpoint{2.456119in}{2.902597in}}%
\pgfpathlineto{\pgfqpoint{2.465053in}{2.766961in}}%
\pgfpathlineto{\pgfqpoint{2.473803in}{2.781608in}}%
\pgfpathlineto{\pgfqpoint{2.482376in}{2.826658in}}%
\pgfpathlineto{\pgfqpoint{2.490780in}{2.818432in}}%
\pgfpathlineto{\pgfqpoint{2.499020in}{2.280165in}}%
\pgfpathlineto{\pgfqpoint{2.507103in}{2.745922in}}%
\pgfpathlineto{\pgfqpoint{2.515036in}{2.840059in}}%
\pgfpathlineto{\pgfqpoint{2.522822in}{2.768087in}}%
\pgfpathlineto{\pgfqpoint{2.530469in}{2.559709in}}%
\pgfpathlineto{\pgfqpoint{2.537980in}{2.819059in}}%
\pgfpathlineto{\pgfqpoint{2.545360in}{2.725955in}}%
\pgfpathlineto{\pgfqpoint{2.552614in}{2.843821in}}%
\pgfpathlineto{\pgfqpoint{2.559747in}{2.709809in}}%
\pgfpathlineto{\pgfqpoint{2.566761in}{2.209948in}}%
\pgfpathlineto{\pgfqpoint{2.573662in}{2.674658in}}%
\pgfpathlineto{\pgfqpoint{2.580451in}{2.749673in}}%
\pgfpathlineto{\pgfqpoint{2.587135in}{2.745050in}}%
\pgfpathlineto{\pgfqpoint{2.593714in}{2.747126in}}%
\pgfpathlineto{\pgfqpoint{2.600193in}{2.775782in}}%
\pgfpathlineto{\pgfqpoint{2.606574in}{2.749221in}}%
\pgfpathlineto{\pgfqpoint{2.612861in}{2.755351in}}%
\pgfpathlineto{\pgfqpoint{2.619057in}{2.728136in}}%
\pgfpathlineto{\pgfqpoint{2.625163in}{2.655386in}}%
\pgfpathlineto{\pgfqpoint{2.631182in}{2.821680in}}%
\pgfpathlineto{\pgfqpoint{2.637117in}{2.762214in}}%
\pgfpathlineto{\pgfqpoint{2.642971in}{2.759291in}}%
\pgfpathlineto{\pgfqpoint{2.648744in}{2.768266in}}%
\pgfpathlineto{\pgfqpoint{2.654441in}{2.715684in}}%
\pgfpathlineto{\pgfqpoint{2.671084in}{2.637352in}}%
\pgfpathlineto{\pgfqpoint{2.676490in}{2.761092in}}%
\pgfpathlineto{\pgfqpoint{2.681828in}{2.527861in}}%
\pgfpathlineto{\pgfqpoint{2.687100in}{2.702612in}}%
\pgfpathlineto{\pgfqpoint{2.692307in}{2.658153in}}%
\pgfpathlineto{\pgfqpoint{2.697451in}{2.535342in}}%
\pgfpathlineto{\pgfqpoint{2.702533in}{2.705479in}}%
\pgfpathlineto{\pgfqpoint{2.707555in}{2.660000in}}%
\pgfpathlineto{\pgfqpoint{2.712518in}{2.703905in}}%
\pgfpathlineto{\pgfqpoint{2.717424in}{2.594934in}}%
\pgfpathlineto{\pgfqpoint{2.722274in}{2.323197in}}%
\pgfpathlineto{\pgfqpoint{2.727069in}{2.706263in}}%
\pgfpathlineto{\pgfqpoint{2.731811in}{2.777915in}}%
\pgfpathlineto{\pgfqpoint{2.736500in}{2.788861in}}%
\pgfpathlineto{\pgfqpoint{2.741138in}{2.697567in}}%
\pgfpathlineto{\pgfqpoint{2.745726in}{2.768376in}}%
\pgfpathlineto{\pgfqpoint{2.754755in}{2.794425in}}%
\pgfpathlineto{\pgfqpoint{2.759199in}{2.685205in}}%
\pgfpathlineto{\pgfqpoint{2.763596in}{2.348863in}}%
\pgfpathlineto{\pgfqpoint{2.767949in}{2.760012in}}%
\pgfpathlineto{\pgfqpoint{2.780745in}{2.668682in}}%
\pgfpathlineto{\pgfqpoint{2.784926in}{2.734040in}}%
\pgfpathlineto{\pgfqpoint{2.789066in}{2.524431in}}%
\pgfpathlineto{\pgfqpoint{2.793166in}{2.741290in}}%
\pgfpathlineto{\pgfqpoint{2.797227in}{2.758815in}}%
\pgfpathlineto{\pgfqpoint{2.801249in}{2.625066in}}%
\pgfpathlineto{\pgfqpoint{2.805234in}{2.731297in}}%
\pgfpathlineto{\pgfqpoint{2.809181in}{2.709054in}}%
\pgfpathlineto{\pgfqpoint{2.813093in}{2.651557in}}%
\pgfpathlineto{\pgfqpoint{2.816968in}{2.558166in}}%
\pgfpathlineto{\pgfqpoint{2.820809in}{2.642202in}}%
\pgfpathlineto{\pgfqpoint{2.824615in}{2.682043in}}%
\pgfpathlineto{\pgfqpoint{2.828387in}{2.666942in}}%
\pgfpathlineto{\pgfqpoint{2.832126in}{2.556210in}}%
\pgfpathlineto{\pgfqpoint{2.835832in}{2.579540in}}%
\pgfpathlineto{\pgfqpoint{2.839506in}{2.641368in}}%
\pgfpathlineto{\pgfqpoint{2.843149in}{2.750607in}}%
\pgfpathlineto{\pgfqpoint{2.846760in}{2.546434in}}%
\pgfpathlineto{\pgfqpoint{2.850341in}{2.538157in}}%
\pgfpathlineto{\pgfqpoint{2.853893in}{2.552318in}}%
\pgfpathlineto{\pgfqpoint{2.857414in}{2.762320in}}%
\pgfpathlineto{\pgfqpoint{2.860907in}{2.531784in}}%
\pgfpathlineto{\pgfqpoint{2.864371in}{2.595811in}}%
\pgfpathlineto{\pgfqpoint{2.867807in}{2.696173in}}%
\pgfpathlineto{\pgfqpoint{2.871216in}{2.642135in}}%
\pgfpathlineto{\pgfqpoint{2.874597in}{2.622539in}}%
\pgfpathlineto{\pgfqpoint{2.881280in}{2.704594in}}%
\pgfpathlineto{\pgfqpoint{2.884583in}{2.604926in}}%
\pgfpathlineto{\pgfqpoint{2.887860in}{2.418156in}}%
\pgfpathlineto{\pgfqpoint{2.891111in}{2.540655in}}%
\pgfpathlineto{\pgfqpoint{2.894339in}{2.615987in}}%
\pgfpathlineto{\pgfqpoint{2.897541in}{2.554312in}}%
\pgfpathlineto{\pgfqpoint{2.900720in}{2.668791in}}%
\pgfpathlineto{\pgfqpoint{2.903875in}{2.695229in}}%
\pgfpathlineto{\pgfqpoint{2.907007in}{2.627428in}}%
\pgfpathlineto{\pgfqpoint{2.910116in}{2.680563in}}%
\pgfpathlineto{\pgfqpoint{2.913202in}{2.623988in}}%
\pgfpathlineto{\pgfqpoint{2.916266in}{2.707262in}}%
\pgfpathlineto{\pgfqpoint{2.919308in}{2.622612in}}%
\pgfpathlineto{\pgfqpoint{2.922329in}{2.729582in}}%
\pgfpathlineto{\pgfqpoint{2.925328in}{2.622506in}}%
\pgfpathlineto{\pgfqpoint{2.931263in}{2.707031in}}%
\pgfpathlineto{\pgfqpoint{2.934200in}{2.704098in}}%
\pgfpathlineto{\pgfqpoint{2.937116in}{2.679518in}}%
\pgfpathlineto{\pgfqpoint{2.940013in}{2.595741in}}%
\pgfpathlineto{\pgfqpoint{2.942890in}{2.577510in}}%
\pgfpathlineto{\pgfqpoint{2.945748in}{2.603891in}}%
\pgfpathlineto{\pgfqpoint{2.951406in}{2.634883in}}%
\pgfpathlineto{\pgfqpoint{2.954207in}{2.631489in}}%
\pgfpathlineto{\pgfqpoint{2.956990in}{2.593730in}}%
\pgfpathlineto{\pgfqpoint{2.959754in}{2.530126in}}%
\pgfpathlineto{\pgfqpoint{2.962501in}{2.567304in}}%
\pgfpathlineto{\pgfqpoint{2.965230in}{2.504975in}}%
\pgfpathlineto{\pgfqpoint{2.967942in}{2.664064in}}%
\pgfpathlineto{\pgfqpoint{2.970636in}{2.505832in}}%
\pgfpathlineto{\pgfqpoint{2.973313in}{2.632573in}}%
\pgfpathlineto{\pgfqpoint{2.975974in}{2.684826in}}%
\pgfpathlineto{\pgfqpoint{2.978618in}{2.635984in}}%
\pgfpathlineto{\pgfqpoint{2.981246in}{2.631359in}}%
\pgfpathlineto{\pgfqpoint{2.983857in}{2.439258in}}%
\pgfpathlineto{\pgfqpoint{2.986453in}{2.480639in}}%
\pgfpathlineto{\pgfqpoint{2.989032in}{2.672581in}}%
\pgfpathlineto{\pgfqpoint{2.991597in}{2.546805in}}%
\pgfpathlineto{\pgfqpoint{2.994145in}{2.681186in}}%
\pgfpathlineto{\pgfqpoint{2.996679in}{2.552051in}}%
\pgfpathlineto{\pgfqpoint{2.999197in}{2.610296in}}%
\pgfpathlineto{\pgfqpoint{3.001701in}{2.572211in}}%
\pgfpathlineto{\pgfqpoint{3.004190in}{2.640231in}}%
\pgfpathlineto{\pgfqpoint{3.006664in}{2.449848in}}%
\pgfpathlineto{\pgfqpoint{3.009124in}{2.712908in}}%
\pgfpathlineto{\pgfqpoint{3.011570in}{2.693812in}}%
\pgfpathlineto{\pgfqpoint{3.014002in}{2.625203in}}%
\pgfpathlineto{\pgfqpoint{3.016420in}{2.453645in}}%
\pgfpathlineto{\pgfqpoint{3.018824in}{2.613653in}}%
\pgfpathlineto{\pgfqpoint{3.021215in}{2.573699in}}%
\pgfpathlineto{\pgfqpoint{3.023593in}{2.653504in}}%
\pgfpathlineto{\pgfqpoint{3.025957in}{2.695352in}}%
\pgfpathlineto{\pgfqpoint{3.028308in}{2.679513in}}%
\pgfpathlineto{\pgfqpoint{3.030646in}{2.561013in}}%
\pgfpathlineto{\pgfqpoint{3.032971in}{2.526871in}}%
\pgfpathlineto{\pgfqpoint{3.035284in}{2.569075in}}%
\pgfpathlineto{\pgfqpoint{3.037584in}{2.545600in}}%
\pgfpathlineto{\pgfqpoint{3.039872in}{2.629464in}}%
\pgfpathlineto{\pgfqpoint{3.042147in}{2.563520in}}%
\pgfpathlineto{\pgfqpoint{3.044410in}{2.424576in}}%
\pgfpathlineto{\pgfqpoint{3.048901in}{2.577617in}}%
\pgfpathlineto{\pgfqpoint{3.051128in}{2.517433in}}%
\pgfpathlineto{\pgfqpoint{3.053344in}{2.300750in}}%
\pgfpathlineto{\pgfqpoint{3.055549in}{2.596971in}}%
\pgfpathlineto{\pgfqpoint{3.057742in}{2.640648in}}%
\pgfpathlineto{\pgfqpoint{3.059924in}{2.381455in}}%
\pgfpathlineto{\pgfqpoint{3.064254in}{2.592433in}}%
\pgfpathlineto{\pgfqpoint{3.066403in}{2.308869in}}%
\pgfpathlineto{\pgfqpoint{3.068541in}{2.632630in}}%
\pgfpathlineto{\pgfqpoint{3.070668in}{2.647223in}}%
\pgfpathlineto{\pgfqpoint{3.072784in}{2.587724in}}%
\pgfpathlineto{\pgfqpoint{3.074890in}{2.636517in}}%
\pgfpathlineto{\pgfqpoint{3.076986in}{2.648613in}}%
\pgfpathlineto{\pgfqpoint{3.079071in}{2.627714in}}%
\pgfpathlineto{\pgfqpoint{3.081146in}{2.631910in}}%
\pgfpathlineto{\pgfqpoint{3.083211in}{2.555005in}}%
\pgfpathlineto{\pgfqpoint{3.085267in}{2.576088in}}%
\pgfpathlineto{\pgfqpoint{3.087312in}{2.506939in}}%
\pgfpathlineto{\pgfqpoint{3.089347in}{2.662566in}}%
\pgfpathlineto{\pgfqpoint{3.091373in}{2.406372in}}%
\pgfpathlineto{\pgfqpoint{3.093389in}{2.367640in}}%
\pgfpathlineto{\pgfqpoint{3.095395in}{2.577713in}}%
\pgfpathlineto{\pgfqpoint{3.097392in}{2.566340in}}%
\pgfpathlineto{\pgfqpoint{3.099380in}{2.539764in}}%
\pgfpathlineto{\pgfqpoint{3.101358in}{2.682887in}}%
\pgfpathlineto{\pgfqpoint{3.103327in}{2.527247in}}%
\pgfpathlineto{\pgfqpoint{3.105287in}{2.565197in}}%
\pgfpathlineto{\pgfqpoint{3.107238in}{2.562446in}}%
\pgfpathlineto{\pgfqpoint{3.109181in}{2.409419in}}%
\pgfpathlineto{\pgfqpoint{3.111114in}{2.447321in}}%
\pgfpathlineto{\pgfqpoint{3.113038in}{2.648900in}}%
\pgfpathlineto{\pgfqpoint{3.114954in}{2.570039in}}%
\pgfpathlineto{\pgfqpoint{3.116862in}{2.274583in}}%
\pgfpathlineto{\pgfqpoint{3.118760in}{2.692564in}}%
\pgfpathlineto{\pgfqpoint{3.120651in}{2.579232in}}%
\pgfpathlineto{\pgfqpoint{3.122532in}{2.611790in}}%
\pgfpathlineto{\pgfqpoint{3.124406in}{2.619183in}}%
\pgfpathlineto{\pgfqpoint{3.126271in}{2.614910in}}%
\pgfpathlineto{\pgfqpoint{3.128128in}{2.536622in}}%
\pgfpathlineto{\pgfqpoint{3.131819in}{2.585790in}}%
\pgfpathlineto{\pgfqpoint{3.133652in}{2.621952in}}%
\pgfpathlineto{\pgfqpoint{3.135477in}{2.562745in}}%
\pgfpathlineto{\pgfqpoint{3.137294in}{2.425611in}}%
\pgfpathlineto{\pgfqpoint{3.139104in}{2.543257in}}%
\pgfpathlineto{\pgfqpoint{3.140906in}{2.390536in}}%
\pgfpathlineto{\pgfqpoint{3.144487in}{2.554953in}}%
\pgfpathlineto{\pgfqpoint{3.146266in}{2.271314in}}%
\pgfpathlineto{\pgfqpoint{3.148038in}{2.553900in}}%
\pgfpathlineto{\pgfqpoint{3.149803in}{2.550436in}}%
\pgfpathlineto{\pgfqpoint{3.151560in}{2.552883in}}%
\pgfpathlineto{\pgfqpoint{3.153310in}{2.605014in}}%
\pgfpathlineto{\pgfqpoint{3.155053in}{2.622201in}}%
\pgfpathlineto{\pgfqpoint{3.156788in}{2.571656in}}%
\pgfpathlineto{\pgfqpoint{3.158517in}{2.578824in}}%
\pgfpathlineto{\pgfqpoint{3.160238in}{2.464627in}}%
\pgfpathlineto{\pgfqpoint{3.161953in}{2.598452in}}%
\pgfpathlineto{\pgfqpoint{3.163661in}{2.319198in}}%
\pgfpathlineto{\pgfqpoint{3.165362in}{2.596410in}}%
\pgfpathlineto{\pgfqpoint{3.167056in}{2.534068in}}%
\pgfpathlineto{\pgfqpoint{3.168743in}{2.536447in}}%
\pgfpathlineto{\pgfqpoint{3.170424in}{2.519084in}}%
\pgfpathlineto{\pgfqpoint{3.173765in}{2.599505in}}%
\pgfpathlineto{\pgfqpoint{3.175426in}{2.555866in}}%
\pgfpathlineto{\pgfqpoint{3.177080in}{2.464403in}}%
\pgfpathlineto{\pgfqpoint{3.178728in}{2.669802in}}%
\pgfpathlineto{\pgfqpoint{3.180370in}{2.526881in}}%
\pgfpathlineto{\pgfqpoint{3.182005in}{2.570114in}}%
\pgfpathlineto{\pgfqpoint{3.183634in}{2.580351in}}%
\pgfpathlineto{\pgfqpoint{3.186874in}{2.567791in}}%
\pgfpathlineto{\pgfqpoint{3.188484in}{2.457049in}}%
\pgfpathlineto{\pgfqpoint{3.190089in}{2.458698in}}%
\pgfpathlineto{\pgfqpoint{3.191687in}{2.620029in}}%
\pgfpathlineto{\pgfqpoint{3.193279in}{2.621008in}}%
\pgfpathlineto{\pgfqpoint{3.198021in}{2.516697in}}%
\pgfpathlineto{\pgfqpoint{3.199590in}{2.592988in}}%
\pgfpathlineto{\pgfqpoint{3.201153in}{2.360928in}}%
\pgfpathlineto{\pgfqpoint{3.204262in}{2.609084in}}%
\pgfpathlineto{\pgfqpoint{3.205808in}{2.321788in}}%
\pgfpathlineto{\pgfqpoint{3.210412in}{2.550062in}}%
\pgfpathlineto{\pgfqpoint{3.211936in}{2.516262in}}%
\pgfpathlineto{\pgfqpoint{3.213454in}{2.566614in}}%
\pgfpathlineto{\pgfqpoint{3.214967in}{2.569115in}}%
\pgfpathlineto{\pgfqpoint{3.216474in}{2.492396in}}%
\pgfpathlineto{\pgfqpoint{3.217977in}{2.599964in}}%
\pgfpathlineto{\pgfqpoint{3.219473in}{2.263734in}}%
\pgfpathlineto{\pgfqpoint{3.220965in}{2.603802in}}%
\pgfpathlineto{\pgfqpoint{3.222451in}{2.274351in}}%
\pgfpathlineto{\pgfqpoint{3.223933in}{2.476739in}}%
\pgfpathlineto{\pgfqpoint{3.226880in}{2.543063in}}%
\pgfpathlineto{\pgfqpoint{3.228346in}{2.459137in}}%
\pgfpathlineto{\pgfqpoint{3.229806in}{2.294852in}}%
\pgfpathlineto{\pgfqpoint{3.234159in}{2.555157in}}%
\pgfpathlineto{\pgfqpoint{3.235600in}{2.447869in}}%
\pgfpathlineto{\pgfqpoint{3.237036in}{2.438218in}}%
\pgfpathlineto{\pgfqpoint{3.238467in}{2.527663in}}%
\pgfpathlineto{\pgfqpoint{3.239894in}{2.513884in}}%
\pgfpathlineto{\pgfqpoint{3.241315in}{2.594816in}}%
\pgfpathlineto{\pgfqpoint{3.242732in}{2.390761in}}%
\pgfpathlineto{\pgfqpoint{3.244144in}{2.568161in}}%
\pgfpathlineto{\pgfqpoint{3.245552in}{2.503429in}}%
\pgfpathlineto{\pgfqpoint{3.246955in}{2.496888in}}%
\pgfpathlineto{\pgfqpoint{3.248353in}{2.513340in}}%
\pgfpathlineto{\pgfqpoint{3.249746in}{2.463039in}}%
\pgfpathlineto{\pgfqpoint{3.253900in}{2.548157in}}%
\pgfpathlineto{\pgfqpoint{3.255276in}{2.476401in}}%
\pgfpathlineto{\pgfqpoint{3.258014in}{2.600244in}}%
\pgfpathlineto{\pgfqpoint{3.259376in}{2.450324in}}%
\pgfpathlineto{\pgfqpoint{3.260734in}{2.444052in}}%
\pgfpathlineto{\pgfqpoint{3.262087in}{2.573789in}}%
\pgfpathlineto{\pgfqpoint{3.263437in}{2.539271in}}%
\pgfpathlineto{\pgfqpoint{3.264782in}{2.589621in}}%
\pgfpathlineto{\pgfqpoint{3.266123in}{2.591465in}}%
\pgfpathlineto{\pgfqpoint{3.267459in}{2.482321in}}%
\pgfpathlineto{\pgfqpoint{3.268792in}{2.242068in}}%
\pgfpathlineto{\pgfqpoint{3.270120in}{2.585003in}}%
\pgfpathlineto{\pgfqpoint{3.271444in}{2.558917in}}%
\pgfpathlineto{\pgfqpoint{3.274080in}{2.403786in}}%
\pgfpathlineto{\pgfqpoint{3.275391in}{2.573698in}}%
\pgfpathlineto{\pgfqpoint{3.276699in}{2.466528in}}%
\pgfpathlineto{\pgfqpoint{3.278003in}{2.481074in}}%
\pgfpathlineto{\pgfqpoint{3.279303in}{2.474804in}}%
\pgfpathlineto{\pgfqpoint{3.280598in}{2.578767in}}%
\pgfpathlineto{\pgfqpoint{3.281890in}{2.563403in}}%
\pgfpathlineto{\pgfqpoint{3.283178in}{2.580921in}}%
\pgfpathlineto{\pgfqpoint{3.284462in}{2.307409in}}%
\pgfpathlineto{\pgfqpoint{3.285742in}{2.554227in}}%
\pgfpathlineto{\pgfqpoint{3.287019in}{2.447492in}}%
\pgfpathlineto{\pgfqpoint{3.288291in}{2.243356in}}%
\pgfpathlineto{\pgfqpoint{3.292086in}{2.595406in}}%
\pgfpathlineto{\pgfqpoint{3.293343in}{2.509283in}}%
\pgfpathlineto{\pgfqpoint{3.294597in}{2.566275in}}%
\pgfpathlineto{\pgfqpoint{3.295847in}{2.543584in}}%
\pgfpathlineto{\pgfqpoint{3.297093in}{2.377118in}}%
\pgfpathlineto{\pgfqpoint{3.298336in}{2.495978in}}%
\pgfpathlineto{\pgfqpoint{3.299575in}{2.304920in}}%
\pgfpathlineto{\pgfqpoint{3.300810in}{2.574809in}}%
\pgfpathlineto{\pgfqpoint{3.303270in}{2.431779in}}%
\pgfpathlineto{\pgfqpoint{3.304495in}{2.508354in}}%
\pgfpathlineto{\pgfqpoint{3.306934in}{2.486333in}}%
\pgfpathlineto{\pgfqpoint{3.308148in}{2.482014in}}%
\pgfpathlineto{\pgfqpoint{3.309359in}{2.491124in}}%
\pgfpathlineto{\pgfqpoint{3.310566in}{2.551435in}}%
\pgfpathlineto{\pgfqpoint{3.311770in}{2.484893in}}%
\pgfpathlineto{\pgfqpoint{3.312970in}{2.515314in}}%
\pgfpathlineto{\pgfqpoint{3.314167in}{2.292671in}}%
\pgfpathlineto{\pgfqpoint{3.315361in}{2.570813in}}%
\pgfpathlineto{\pgfqpoint{3.316551in}{2.397545in}}%
\pgfpathlineto{\pgfqpoint{3.317738in}{2.528768in}}%
\pgfpathlineto{\pgfqpoint{3.320103in}{2.473241in}}%
\pgfpathlineto{\pgfqpoint{3.321280in}{2.485441in}}%
\pgfpathlineto{\pgfqpoint{3.323624in}{2.395603in}}%
\pgfpathlineto{\pgfqpoint{3.324792in}{2.559220in}}%
\pgfpathlineto{\pgfqpoint{3.327117in}{2.332621in}}%
\pgfpathlineto{\pgfqpoint{3.328275in}{2.489687in}}%
\pgfpathlineto{\pgfqpoint{3.329430in}{2.451644in}}%
\pgfpathlineto{\pgfqpoint{3.330581in}{2.520598in}}%
\pgfpathlineto{\pgfqpoint{3.331730in}{2.359165in}}%
\pgfpathlineto{\pgfqpoint{3.332875in}{2.428974in}}%
\pgfpathlineto{\pgfqpoint{3.334017in}{2.280156in}}%
\pgfpathlineto{\pgfqpoint{3.335157in}{2.571488in}}%
\pgfpathlineto{\pgfqpoint{3.336293in}{2.378940in}}%
\pgfpathlineto{\pgfqpoint{3.337426in}{2.489020in}}%
\pgfpathlineto{\pgfqpoint{3.338556in}{2.433971in}}%
\pgfpathlineto{\pgfqpoint{3.339683in}{2.538859in}}%
\pgfpathlineto{\pgfqpoint{3.341928in}{2.461619in}}%
\pgfpathlineto{\pgfqpoint{3.343047in}{2.368697in}}%
\pgfpathlineto{\pgfqpoint{3.345274in}{2.597755in}}%
\pgfpathlineto{\pgfqpoint{3.347490in}{2.324405in}}%
\pgfpathlineto{\pgfqpoint{3.348594in}{2.351077in}}%
\pgfpathlineto{\pgfqpoint{3.350793in}{2.500167in}}%
\pgfpathlineto{\pgfqpoint{3.351888in}{2.455692in}}%
\pgfpathlineto{\pgfqpoint{3.355156in}{2.545842in}}%
\pgfpathlineto{\pgfqpoint{3.356240in}{2.477921in}}%
\pgfpathlineto{\pgfqpoint{3.357321in}{2.309332in}}%
\pgfpathlineto{\pgfqpoint{3.360549in}{2.534908in}}%
\pgfpathlineto{\pgfqpoint{3.361619in}{2.235894in}}%
\pgfpathlineto{\pgfqpoint{3.362686in}{2.549543in}}%
\pgfpathlineto{\pgfqpoint{3.363751in}{2.436780in}}%
\pgfpathlineto{\pgfqpoint{3.364814in}{2.507032in}}%
\pgfpathlineto{\pgfqpoint{3.365873in}{2.445776in}}%
\pgfpathlineto{\pgfqpoint{3.366930in}{2.483763in}}%
\pgfpathlineto{\pgfqpoint{3.367984in}{2.431399in}}%
\pgfpathlineto{\pgfqpoint{3.369036in}{2.122100in}}%
\pgfpathlineto{\pgfqpoint{3.371132in}{2.514609in}}%
\pgfpathlineto{\pgfqpoint{3.372176in}{2.413269in}}%
\pgfpathlineto{\pgfqpoint{3.373217in}{2.454254in}}%
\pgfpathlineto{\pgfqpoint{3.374256in}{2.452488in}}%
\pgfpathlineto{\pgfqpoint{3.375292in}{2.474775in}}%
\pgfpathlineto{\pgfqpoint{3.376326in}{2.396686in}}%
\pgfpathlineto{\pgfqpoint{3.378386in}{2.601003in}}%
\pgfpathlineto{\pgfqpoint{3.379412in}{2.283156in}}%
\pgfpathlineto{\pgfqpoint{3.383493in}{2.585182in}}%
\pgfpathlineto{\pgfqpoint{3.384507in}{2.500943in}}%
\pgfpathlineto{\pgfqpoint{3.385518in}{2.586284in}}%
\pgfpathlineto{\pgfqpoint{3.388539in}{2.339551in}}%
\pgfpathlineto{\pgfqpoint{3.389541in}{2.557505in}}%
\pgfpathlineto{\pgfqpoint{3.390540in}{2.428149in}}%
\pgfpathlineto{\pgfqpoint{3.391538in}{2.437382in}}%
\pgfpathlineto{\pgfqpoint{3.392533in}{2.383630in}}%
\pgfpathlineto{\pgfqpoint{3.394516in}{2.582697in}}%
\pgfpathlineto{\pgfqpoint{3.395504in}{2.470221in}}%
\pgfpathlineto{\pgfqpoint{3.396490in}{2.539378in}}%
\pgfpathlineto{\pgfqpoint{3.397473in}{2.364375in}}%
\pgfpathlineto{\pgfqpoint{3.399433in}{2.473362in}}%
\pgfpathlineto{\pgfqpoint{3.400410in}{2.283183in}}%
\pgfpathlineto{\pgfqpoint{3.403326in}{2.491017in}}%
\pgfpathlineto{\pgfqpoint{3.404294in}{2.546179in}}%
\pgfpathlineto{\pgfqpoint{3.405260in}{2.399990in}}%
\pgfpathlineto{\pgfqpoint{3.406223in}{2.527104in}}%
\pgfpathlineto{\pgfqpoint{3.408143in}{2.431952in}}%
\pgfpathlineto{\pgfqpoint{3.410055in}{2.543481in}}%
\pgfpathlineto{\pgfqpoint{3.411958in}{2.321290in}}%
\pgfpathlineto{\pgfqpoint{3.412906in}{2.528811in}}%
\pgfpathlineto{\pgfqpoint{3.413852in}{2.361123in}}%
\pgfpathlineto{\pgfqpoint{3.415738in}{2.506920in}}%
\pgfpathlineto{\pgfqpoint{3.417616in}{2.317767in}}%
\pgfpathlineto{\pgfqpoint{3.418552in}{2.435714in}}%
\pgfpathlineto{\pgfqpoint{3.419485in}{2.413995in}}%
\pgfpathlineto{\pgfqpoint{3.420417in}{2.505969in}}%
\pgfpathlineto{\pgfqpoint{3.422274in}{2.354674in}}%
\pgfpathlineto{\pgfqpoint{3.423200in}{2.523876in}}%
\pgfpathlineto{\pgfqpoint{3.424123in}{2.505019in}}%
\pgfpathlineto{\pgfqpoint{3.425045in}{2.337028in}}%
\pgfpathlineto{\pgfqpoint{3.425964in}{2.418531in}}%
\pgfpathlineto{\pgfqpoint{3.426882in}{2.325287in}}%
\pgfpathlineto{\pgfqpoint{3.427797in}{2.494018in}}%
\pgfpathlineto{\pgfqpoint{3.428711in}{2.469335in}}%
\pgfpathlineto{\pgfqpoint{3.430532in}{2.525092in}}%
\pgfpathlineto{\pgfqpoint{3.431440in}{2.229869in}}%
\pgfpathlineto{\pgfqpoint{3.433250in}{2.597630in}}%
\pgfpathlineto{\pgfqpoint{3.434152in}{2.351353in}}%
\pgfpathlineto{\pgfqpoint{3.435052in}{2.485300in}}%
\pgfpathlineto{\pgfqpoint{3.435950in}{2.391827in}}%
\pgfpathlineto{\pgfqpoint{3.436846in}{2.403865in}}%
\pgfpathlineto{\pgfqpoint{3.438633in}{2.491635in}}%
\pgfpathlineto{\pgfqpoint{3.439523in}{2.494720in}}%
\pgfpathlineto{\pgfqpoint{3.441299in}{2.397219in}}%
\pgfpathlineto{\pgfqpoint{3.443949in}{2.503029in}}%
\pgfpathlineto{\pgfqpoint{3.444828in}{2.408686in}}%
\pgfpathlineto{\pgfqpoint{3.447456in}{2.560542in}}%
\pgfpathlineto{\pgfqpoint{3.449198in}{2.284673in}}%
\pgfpathlineto{\pgfqpoint{3.450067in}{2.425313in}}%
\pgfpathlineto{\pgfqpoint{3.450934in}{2.276870in}}%
\pgfpathlineto{\pgfqpoint{3.453524in}{2.581168in}}%
\pgfpathlineto{\pgfqpoint{3.454384in}{2.448683in}}%
\pgfpathlineto{\pgfqpoint{3.455242in}{2.470393in}}%
\pgfpathlineto{\pgfqpoint{3.456099in}{2.505015in}}%
\pgfpathlineto{\pgfqpoint{3.457807in}{2.406156in}}%
\pgfpathlineto{\pgfqpoint{3.458658in}{2.480299in}}%
\pgfpathlineto{\pgfqpoint{3.459507in}{2.361780in}}%
\pgfpathlineto{\pgfqpoint{3.460355in}{2.418040in}}%
\pgfpathlineto{\pgfqpoint{3.461201in}{2.336831in}}%
\pgfpathlineto{\pgfqpoint{3.463730in}{2.524553in}}%
\pgfpathlineto{\pgfqpoint{3.464569in}{2.259972in}}%
\pgfpathlineto{\pgfqpoint{3.465407in}{2.487016in}}%
\pgfpathlineto{\pgfqpoint{3.466243in}{2.461943in}}%
\pgfpathlineto{\pgfqpoint{3.467078in}{2.473951in}}%
\pgfpathlineto{\pgfqpoint{3.467911in}{2.438365in}}%
\pgfpathlineto{\pgfqpoint{3.468742in}{2.458507in}}%
\pgfpathlineto{\pgfqpoint{3.469572in}{2.547561in}}%
\pgfpathlineto{\pgfqpoint{3.470400in}{2.430285in}}%
\pgfpathlineto{\pgfqpoint{3.471226in}{2.480800in}}%
\pgfpathlineto{\pgfqpoint{3.472051in}{2.314798in}}%
\pgfpathlineto{\pgfqpoint{3.473696in}{2.502037in}}%
\pgfpathlineto{\pgfqpoint{3.474516in}{2.539329in}}%
\pgfpathlineto{\pgfqpoint{3.476966in}{2.355915in}}%
\pgfpathlineto{\pgfqpoint{3.477780in}{2.437034in}}%
\pgfpathlineto{\pgfqpoint{3.479403in}{2.380002in}}%
\pgfpathlineto{\pgfqpoint{3.480212in}{2.442863in}}%
\pgfpathlineto{\pgfqpoint{3.481020in}{2.367489in}}%
\pgfpathlineto{\pgfqpoint{3.482630in}{2.480565in}}%
\pgfpathlineto{\pgfqpoint{3.483433in}{2.403220in}}%
\pgfpathlineto{\pgfqpoint{3.484235in}{2.441599in}}%
\pgfpathlineto{\pgfqpoint{3.485034in}{2.259705in}}%
\pgfpathlineto{\pgfqpoint{3.485833in}{2.367304in}}%
\pgfpathlineto{\pgfqpoint{3.486630in}{2.312716in}}%
\pgfpathlineto{\pgfqpoint{3.488219in}{2.530513in}}%
\pgfpathlineto{\pgfqpoint{3.489803in}{2.226390in}}%
\pgfpathlineto{\pgfqpoint{3.490592in}{2.536916in}}%
\pgfpathlineto{\pgfqpoint{3.491380in}{2.430292in}}%
\pgfpathlineto{\pgfqpoint{3.492167in}{2.417905in}}%
\pgfpathlineto{\pgfqpoint{3.492952in}{2.476765in}}%
\pgfpathlineto{\pgfqpoint{3.495299in}{2.345454in}}%
\pgfpathlineto{\pgfqpoint{3.496856in}{2.527368in}}%
\pgfpathlineto{\pgfqpoint{3.497632in}{2.311233in}}%
\pgfpathlineto{\pgfqpoint{3.499953in}{2.574733in}}%
\pgfpathlineto{\pgfqpoint{3.501494in}{2.316765in}}%
\pgfpathlineto{\pgfqpoint{3.502262in}{2.496200in}}%
\pgfpathlineto{\pgfqpoint{3.503029in}{2.424180in}}%
\pgfpathlineto{\pgfqpoint{3.503794in}{2.341359in}}%
\pgfpathlineto{\pgfqpoint{3.504558in}{2.523951in}}%
\pgfpathlineto{\pgfqpoint{3.506082in}{2.292213in}}%
\pgfpathlineto{\pgfqpoint{3.506841in}{2.331611in}}%
\pgfpathlineto{\pgfqpoint{3.507600in}{2.475173in}}%
\pgfpathlineto{\pgfqpoint{3.508357in}{2.324662in}}%
\pgfpathlineto{\pgfqpoint{3.509113in}{2.395993in}}%
\pgfpathlineto{\pgfqpoint{3.509867in}{2.173258in}}%
\pgfpathlineto{\pgfqpoint{3.511372in}{2.450022in}}%
\pgfpathlineto{\pgfqpoint{3.512871in}{2.300764in}}%
\pgfpathlineto{\pgfqpoint{3.513619in}{2.346464in}}%
\pgfpathlineto{\pgfqpoint{3.514366in}{2.487185in}}%
\pgfpathlineto{\pgfqpoint{3.515855in}{2.328140in}}%
\pgfpathlineto{\pgfqpoint{3.516597in}{2.436326in}}%
\pgfpathlineto{\pgfqpoint{3.517338in}{2.324785in}}%
\pgfpathlineto{\pgfqpoint{3.518078in}{2.373082in}}%
\pgfpathlineto{\pgfqpoint{3.518817in}{2.470342in}}%
\pgfpathlineto{\pgfqpoint{3.519554in}{2.407098in}}%
\pgfpathlineto{\pgfqpoint{3.521759in}{2.556713in}}%
\pgfpathlineto{\pgfqpoint{3.522491in}{2.402733in}}%
\pgfpathlineto{\pgfqpoint{3.523222in}{2.492164in}}%
\pgfpathlineto{\pgfqpoint{3.523952in}{2.424830in}}%
\pgfpathlineto{\pgfqpoint{3.524681in}{2.461663in}}%
\pgfpathlineto{\pgfqpoint{3.525408in}{2.149764in}}%
\pgfpathlineto{\pgfqpoint{3.526859in}{2.497267in}}%
\pgfpathlineto{\pgfqpoint{3.529026in}{2.298467in}}%
\pgfpathlineto{\pgfqpoint{3.529746in}{2.097120in}}%
\pgfpathlineto{\pgfqpoint{3.531182in}{2.423404in}}%
\pgfpathlineto{\pgfqpoint{3.531898in}{2.295004in}}%
\pgfpathlineto{\pgfqpoint{3.532613in}{2.300310in}}%
\pgfpathlineto{\pgfqpoint{3.534751in}{2.488430in}}%
\pgfpathlineto{\pgfqpoint{3.536170in}{2.310074in}}%
\pgfpathlineto{\pgfqpoint{3.536878in}{2.472544in}}%
\pgfpathlineto{\pgfqpoint{3.538290in}{2.245429in}}%
\pgfpathlineto{\pgfqpoint{3.538994in}{2.424646in}}%
\pgfpathlineto{\pgfqpoint{3.539698in}{2.389257in}}%
\pgfpathlineto{\pgfqpoint{3.540400in}{2.235980in}}%
\pgfpathlineto{\pgfqpoint{3.541100in}{2.283205in}}%
\pgfpathlineto{\pgfqpoint{3.541800in}{2.512834in}}%
\pgfpathlineto{\pgfqpoint{3.543196in}{2.209431in}}%
\pgfpathlineto{\pgfqpoint{3.544587in}{2.500911in}}%
\pgfpathlineto{\pgfqpoint{3.545281in}{2.332259in}}%
\pgfpathlineto{\pgfqpoint{3.545974in}{2.425930in}}%
\pgfpathlineto{\pgfqpoint{3.546666in}{2.468453in}}%
\pgfpathlineto{\pgfqpoint{3.547356in}{2.448605in}}%
\pgfpathlineto{\pgfqpoint{3.548046in}{2.387081in}}%
\pgfpathlineto{\pgfqpoint{3.548734in}{2.408667in}}%
\pgfpathlineto{\pgfqpoint{3.549421in}{2.478458in}}%
\pgfpathlineto{\pgfqpoint{3.550108in}{2.036504in}}%
\pgfpathlineto{\pgfqpoint{3.550793in}{2.454886in}}%
\pgfpathlineto{\pgfqpoint{3.551476in}{2.442974in}}%
\pgfpathlineto{\pgfqpoint{3.552159in}{2.447503in}}%
\pgfpathlineto{\pgfqpoint{3.552841in}{2.428956in}}%
\pgfpathlineto{\pgfqpoint{3.554201in}{2.505921in}}%
\pgfpathlineto{\pgfqpoint{3.555557in}{2.318022in}}%
\pgfpathlineto{\pgfqpoint{3.556233in}{2.441835in}}%
\pgfpathlineto{\pgfqpoint{3.556908in}{2.216111in}}%
\pgfpathlineto{\pgfqpoint{3.558256in}{2.482417in}}%
\pgfpathlineto{\pgfqpoint{3.559599in}{2.210541in}}%
\pgfpathlineto{\pgfqpoint{3.560937in}{2.464966in}}%
\pgfpathlineto{\pgfqpoint{3.562937in}{2.336426in}}%
\pgfpathlineto{\pgfqpoint{3.563602in}{2.371896in}}%
\pgfpathlineto{\pgfqpoint{3.564266in}{2.365730in}}%
\pgfpathlineto{\pgfqpoint{3.564928in}{2.337815in}}%
\pgfpathlineto{\pgfqpoint{3.565590in}{2.450504in}}%
\pgfpathlineto{\pgfqpoint{3.566250in}{2.285730in}}%
\pgfpathlineto{\pgfqpoint{3.567568in}{2.433584in}}%
\pgfpathlineto{\pgfqpoint{3.568225in}{2.459802in}}%
\pgfpathlineto{\pgfqpoint{3.568882in}{2.395846in}}%
\pgfpathlineto{\pgfqpoint{3.569537in}{2.417982in}}%
\pgfpathlineto{\pgfqpoint{3.570192in}{2.484371in}}%
\pgfpathlineto{\pgfqpoint{3.571497in}{2.196139in}}%
\pgfpathlineto{\pgfqpoint{3.572799in}{2.464969in}}%
\pgfpathlineto{\pgfqpoint{3.573448in}{2.456554in}}%
\pgfpathlineto{\pgfqpoint{3.574097in}{2.485933in}}%
\pgfpathlineto{\pgfqpoint{3.575391in}{2.437373in}}%
\pgfpathlineto{\pgfqpoint{3.576036in}{2.479453in}}%
\pgfpathlineto{\pgfqpoint{3.577966in}{2.355988in}}%
\pgfpathlineto{\pgfqpoint{3.578608in}{2.394843in}}%
\pgfpathlineto{\pgfqpoint{3.579888in}{2.316911in}}%
\pgfpathlineto{\pgfqpoint{3.580527in}{2.396191in}}%
\pgfpathlineto{\pgfqpoint{3.581801in}{2.283306in}}%
\pgfpathlineto{\pgfqpoint{3.582437in}{2.412472in}}%
\pgfpathlineto{\pgfqpoint{3.583072in}{2.365430in}}%
\pgfpathlineto{\pgfqpoint{3.583705in}{2.349572in}}%
\pgfpathlineto{\pgfqpoint{3.584338in}{2.491180in}}%
\pgfpathlineto{\pgfqpoint{3.585601in}{2.304496in}}%
\pgfpathlineto{\pgfqpoint{3.586231in}{2.491017in}}%
\pgfpathlineto{\pgfqpoint{3.586861in}{2.447777in}}%
\pgfpathlineto{\pgfqpoint{3.587489in}{2.230122in}}%
\pgfpathlineto{\pgfqpoint{3.588116in}{2.383112in}}%
\pgfpathlineto{\pgfqpoint{3.588742in}{2.277187in}}%
\pgfpathlineto{\pgfqpoint{3.589368in}{2.468877in}}%
\pgfpathlineto{\pgfqpoint{3.589992in}{2.231833in}}%
\pgfpathlineto{\pgfqpoint{3.590616in}{2.430556in}}%
\pgfpathlineto{\pgfqpoint{3.591239in}{2.429281in}}%
\pgfpathlineto{\pgfqpoint{3.591860in}{2.260455in}}%
\pgfpathlineto{\pgfqpoint{3.592481in}{2.314256in}}%
\pgfpathlineto{\pgfqpoint{3.594338in}{2.509042in}}%
\pgfpathlineto{\pgfqpoint{3.595572in}{2.474593in}}%
\pgfpathlineto{\pgfqpoint{3.597416in}{2.339034in}}%
\pgfpathlineto{\pgfqpoint{3.598029in}{2.068727in}}%
\pgfpathlineto{\pgfqpoint{3.598641in}{2.262634in}}%
\pgfpathlineto{\pgfqpoint{3.599252in}{2.422297in}}%
\pgfpathlineto{\pgfqpoint{3.599862in}{2.418843in}}%
\pgfpathlineto{\pgfqpoint{3.600471in}{2.276884in}}%
\pgfpathlineto{\pgfqpoint{3.601079in}{2.372453in}}%
\pgfpathlineto{\pgfqpoint{3.601687in}{2.289672in}}%
\pgfpathlineto{\pgfqpoint{3.602899in}{2.399431in}}%
\pgfpathlineto{\pgfqpoint{3.603504in}{2.152289in}}%
\pgfpathlineto{\pgfqpoint{3.604712in}{2.467370in}}%
\pgfpathlineto{\pgfqpoint{3.605314in}{2.305300in}}%
\pgfpathlineto{\pgfqpoint{3.605915in}{2.382119in}}%
\pgfpathlineto{\pgfqpoint{3.606516in}{2.374054in}}%
\pgfpathlineto{\pgfqpoint{3.607116in}{2.250019in}}%
\pgfpathlineto{\pgfqpoint{3.608313in}{2.430421in}}%
\pgfpathlineto{\pgfqpoint{3.608910in}{2.400974in}}%
\pgfpathlineto{\pgfqpoint{3.610697in}{2.482481in}}%
\pgfpathlineto{\pgfqpoint{3.611884in}{2.273827in}}%
\pgfpathlineto{\pgfqpoint{3.612476in}{2.473273in}}%
\pgfpathlineto{\pgfqpoint{3.613068in}{2.388424in}}%
\pgfpathlineto{\pgfqpoint{3.613658in}{2.437323in}}%
\pgfpathlineto{\pgfqpoint{3.614248in}{2.254587in}}%
\pgfpathlineto{\pgfqpoint{3.614837in}{2.393285in}}%
\pgfpathlineto{\pgfqpoint{3.615425in}{2.294470in}}%
\pgfpathlineto{\pgfqpoint{3.616013in}{2.486255in}}%
\pgfpathlineto{\pgfqpoint{3.616599in}{2.285146in}}%
\pgfpathlineto{\pgfqpoint{3.617185in}{2.390808in}}%
\pgfpathlineto{\pgfqpoint{3.617770in}{2.458403in}}%
\pgfpathlineto{\pgfqpoint{3.618354in}{2.458120in}}%
\pgfpathlineto{\pgfqpoint{3.618937in}{2.391245in}}%
\pgfpathlineto{\pgfqpoint{3.619520in}{2.457004in}}%
\pgfpathlineto{\pgfqpoint{3.620683in}{2.189202in}}%
\pgfpathlineto{\pgfqpoint{3.621263in}{2.285525in}}%
\pgfpathlineto{\pgfqpoint{3.622421in}{2.455701in}}%
\pgfpathlineto{\pgfqpoint{3.622998in}{2.329504in}}%
\pgfpathlineto{\pgfqpoint{3.623575in}{2.331804in}}%
\pgfpathlineto{\pgfqpoint{3.625302in}{2.423113in}}%
\pgfpathlineto{\pgfqpoint{3.627021in}{2.015238in}}%
\pgfpathlineto{\pgfqpoint{3.627592in}{2.429154in}}%
\pgfpathlineto{\pgfqpoint{3.628163in}{2.259863in}}%
\pgfpathlineto{\pgfqpoint{3.629871in}{2.442248in}}%
\pgfpathlineto{\pgfqpoint{3.630438in}{2.174071in}}%
\pgfpathlineto{\pgfqpoint{3.631005in}{2.327717in}}%
\pgfpathlineto{\pgfqpoint{3.631572in}{2.245361in}}%
\pgfpathlineto{\pgfqpoint{3.632702in}{2.451569in}}%
\pgfpathlineto{\pgfqpoint{3.633266in}{2.400022in}}%
\pgfpathlineto{\pgfqpoint{3.633829in}{2.361308in}}%
\pgfpathlineto{\pgfqpoint{3.634391in}{2.379234in}}%
\pgfpathlineto{\pgfqpoint{3.634953in}{2.121731in}}%
\pgfpathlineto{\pgfqpoint{3.635514in}{2.349339in}}%
\pgfpathlineto{\pgfqpoint{3.636074in}{2.316396in}}%
\pgfpathlineto{\pgfqpoint{3.636634in}{2.342896in}}%
\pgfpathlineto{\pgfqpoint{3.637750in}{2.463582in}}%
\pgfpathlineto{\pgfqpoint{3.638308in}{2.255065in}}%
\pgfpathlineto{\pgfqpoint{3.638864in}{2.303932in}}%
\pgfpathlineto{\pgfqpoint{3.639420in}{2.325567in}}%
\pgfpathlineto{\pgfqpoint{3.639975in}{2.435675in}}%
\pgfpathlineto{\pgfqpoint{3.641083in}{2.205295in}}%
\pgfpathlineto{\pgfqpoint{3.642740in}{2.425633in}}%
\pgfpathlineto{\pgfqpoint{3.644390in}{2.450054in}}%
\pgfpathlineto{\pgfqpoint{3.644938in}{2.441092in}}%
\pgfpathlineto{\pgfqpoint{3.645486in}{2.148951in}}%
\pgfpathlineto{\pgfqpoint{3.646034in}{2.308939in}}%
\pgfpathlineto{\pgfqpoint{3.648759in}{2.534784in}}%
\pgfpathlineto{\pgfqpoint{3.649302in}{2.268692in}}%
\pgfpathlineto{\pgfqpoint{3.650386in}{2.270027in}}%
\pgfpathlineto{\pgfqpoint{3.650927in}{2.421828in}}%
\pgfpathlineto{\pgfqpoint{3.651467in}{2.417831in}}%
\pgfpathlineto{\pgfqpoint{3.652007in}{2.330571in}}%
\pgfpathlineto{\pgfqpoint{3.653621in}{2.486289in}}%
\pgfpathlineto{\pgfqpoint{3.654158in}{2.329424in}}%
\pgfpathlineto{\pgfqpoint{3.654694in}{2.361675in}}%
\pgfpathlineto{\pgfqpoint{3.655230in}{2.376722in}}%
\pgfpathlineto{\pgfqpoint{3.655765in}{2.479238in}}%
\pgfpathlineto{\pgfqpoint{3.657365in}{2.238789in}}%
\pgfpathlineto{\pgfqpoint{3.657897in}{2.467463in}}%
\pgfpathlineto{\pgfqpoint{3.658959in}{2.449696in}}%
\pgfpathlineto{\pgfqpoint{3.660548in}{2.324489in}}%
\pgfpathlineto{\pgfqpoint{3.662130in}{2.475743in}}%
\pgfpathlineto{\pgfqpoint{3.663707in}{2.153792in}}%
\pgfpathlineto{\pgfqpoint{3.664755in}{2.394783in}}%
\pgfpathlineto{\pgfqpoint{3.665278in}{2.318588in}}%
\pgfpathlineto{\pgfqpoint{3.666321in}{2.468031in}}%
\pgfpathlineto{\pgfqpoint{3.666842in}{2.204019in}}%
\pgfpathlineto{\pgfqpoint{3.667363in}{2.432434in}}%
\pgfpathlineto{\pgfqpoint{3.667883in}{2.449373in}}%
\pgfpathlineto{\pgfqpoint{3.668402in}{2.289158in}}%
\pgfpathlineto{\pgfqpoint{3.668920in}{2.386357in}}%
\pgfpathlineto{\pgfqpoint{3.669438in}{2.342767in}}%
\pgfpathlineto{\pgfqpoint{3.669955in}{2.457421in}}%
\pgfpathlineto{\pgfqpoint{3.670472in}{2.273711in}}%
\pgfpathlineto{\pgfqpoint{3.670988in}{2.291099in}}%
\pgfpathlineto{\pgfqpoint{3.672532in}{2.433473in}}%
\pgfpathlineto{\pgfqpoint{3.674070in}{2.304491in}}%
\pgfpathlineto{\pgfqpoint{3.676113in}{2.456911in}}%
\pgfpathlineto{\pgfqpoint{3.676622in}{2.431357in}}%
\pgfpathlineto{\pgfqpoint{3.678146in}{2.204664in}}%
\pgfpathlineto{\pgfqpoint{3.679159in}{2.355120in}}%
\pgfpathlineto{\pgfqpoint{3.679664in}{2.292854in}}%
\pgfpathlineto{\pgfqpoint{3.680169in}{2.367097in}}%
\pgfpathlineto{\pgfqpoint{3.680673in}{2.441448in}}%
\pgfpathlineto{\pgfqpoint{3.681680in}{2.273272in}}%
\pgfpathlineto{\pgfqpoint{3.682183in}{2.418473in}}%
\pgfpathlineto{\pgfqpoint{3.682684in}{2.313431in}}%
\pgfpathlineto{\pgfqpoint{3.683686in}{2.475683in}}%
\pgfpathlineto{\pgfqpoint{3.685185in}{2.117628in}}%
\pgfpathlineto{\pgfqpoint{3.686678in}{2.426743in}}%
\pgfpathlineto{\pgfqpoint{3.689650in}{2.294369in}}%
\pgfpathlineto{\pgfqpoint{3.690143in}{2.478927in}}%
\pgfpathlineto{\pgfqpoint{3.690635in}{2.307919in}}%
\pgfpathlineto{\pgfqpoint{3.691127in}{2.353543in}}%
\pgfpathlineto{\pgfqpoint{3.691619in}{2.330342in}}%
\pgfpathlineto{\pgfqpoint{3.692110in}{2.198050in}}%
\pgfpathlineto{\pgfqpoint{3.692600in}{2.358009in}}%
\pgfpathlineto{\pgfqpoint{3.693090in}{2.377743in}}%
\pgfpathlineto{\pgfqpoint{3.694067in}{2.482885in}}%
\pgfpathlineto{\pgfqpoint{3.695043in}{2.312519in}}%
\pgfpathlineto{\pgfqpoint{3.696502in}{2.454788in}}%
\pgfpathlineto{\pgfqpoint{3.696987in}{2.276072in}}%
\pgfpathlineto{\pgfqpoint{3.697472in}{2.317295in}}%
\pgfpathlineto{\pgfqpoint{3.697956in}{2.428273in}}%
\pgfpathlineto{\pgfqpoint{3.698440in}{2.417886in}}%
\pgfpathlineto{\pgfqpoint{3.698923in}{2.279783in}}%
\pgfpathlineto{\pgfqpoint{3.699887in}{2.301192in}}%
\pgfpathlineto{\pgfqpoint{3.701330in}{2.501853in}}%
\pgfpathlineto{\pgfqpoint{3.702768in}{2.241084in}}%
\pgfpathlineto{\pgfqpoint{3.703246in}{2.338705in}}%
\pgfpathlineto{\pgfqpoint{3.703723in}{2.239756in}}%
\pgfpathlineto{\pgfqpoint{3.704201in}{2.143819in}}%
\pgfpathlineto{\pgfqpoint{3.704677in}{2.257742in}}%
\pgfpathlineto{\pgfqpoint{3.706578in}{2.476233in}}%
\pgfpathlineto{\pgfqpoint{3.707525in}{2.312936in}}%
\pgfpathlineto{\pgfqpoint{3.708470in}{2.365000in}}%
\pgfpathlineto{\pgfqpoint{3.709884in}{2.437112in}}%
\pgfpathlineto{\pgfqpoint{3.710354in}{2.310124in}}%
\pgfpathlineto{\pgfqpoint{3.711293in}{2.336265in}}%
\pgfpathlineto{\pgfqpoint{3.711762in}{2.401509in}}%
\pgfpathlineto{\pgfqpoint{3.712698in}{2.386228in}}%
\pgfpathlineto{\pgfqpoint{3.713165in}{2.352465in}}%
\pgfpathlineto{\pgfqpoint{3.713631in}{2.420049in}}%
\pgfpathlineto{\pgfqpoint{3.714097in}{2.211301in}}%
\pgfpathlineto{\pgfqpoint{3.714563in}{2.367349in}}%
\pgfpathlineto{\pgfqpoint{3.715028in}{2.432883in}}%
\pgfpathlineto{\pgfqpoint{3.715492in}{2.210553in}}%
\pgfpathlineto{\pgfqpoint{3.715956in}{2.396887in}}%
\pgfpathlineto{\pgfqpoint{3.716883in}{2.378821in}}%
\pgfpathlineto{\pgfqpoint{3.717346in}{2.089212in}}%
\pgfpathlineto{\pgfqpoint{3.717808in}{2.386389in}}%
\pgfpathlineto{\pgfqpoint{3.718269in}{2.424639in}}%
\pgfpathlineto{\pgfqpoint{3.718730in}{2.374570in}}%
\pgfpathlineto{\pgfqpoint{3.719191in}{2.354315in}}%
\pgfpathlineto{\pgfqpoint{3.720110in}{2.379442in}}%
\pgfpathlineto{\pgfqpoint{3.720569in}{2.104559in}}%
\pgfpathlineto{\pgfqpoint{3.721028in}{2.336563in}}%
\pgfpathlineto{\pgfqpoint{3.721943in}{2.245087in}}%
\pgfpathlineto{\pgfqpoint{3.722400in}{2.355796in}}%
\pgfpathlineto{\pgfqpoint{3.722857in}{2.106362in}}%
\pgfpathlineto{\pgfqpoint{3.723313in}{2.274618in}}%
\pgfpathlineto{\pgfqpoint{3.723768in}{2.289411in}}%
\pgfpathlineto{\pgfqpoint{3.724224in}{2.243085in}}%
\pgfpathlineto{\pgfqpoint{3.724678in}{2.336844in}}%
\pgfpathlineto{\pgfqpoint{3.725132in}{2.335699in}}%
\pgfpathlineto{\pgfqpoint{3.726039in}{2.140199in}}%
\pgfpathlineto{\pgfqpoint{3.727396in}{2.471632in}}%
\pgfpathlineto{\pgfqpoint{3.727847in}{2.416513in}}%
\pgfpathlineto{\pgfqpoint{3.728297in}{2.209878in}}%
\pgfpathlineto{\pgfqpoint{3.728748in}{2.422565in}}%
\pgfpathlineto{\pgfqpoint{3.729647in}{2.203665in}}%
\pgfpathlineto{\pgfqpoint{3.730096in}{2.413273in}}%
\pgfpathlineto{\pgfqpoint{3.730544in}{2.177820in}}%
\pgfpathlineto{\pgfqpoint{3.731439in}{2.316088in}}%
\pgfpathlineto{\pgfqpoint{3.731886in}{2.179943in}}%
\pgfpathlineto{\pgfqpoint{3.732333in}{2.365553in}}%
\pgfpathlineto{\pgfqpoint{3.732779in}{2.429840in}}%
\pgfpathlineto{\pgfqpoint{3.733224in}{2.303224in}}%
\pgfpathlineto{\pgfqpoint{3.734114in}{2.333111in}}%
\pgfpathlineto{\pgfqpoint{3.735002in}{2.354335in}}%
\pgfpathlineto{\pgfqpoint{3.735445in}{2.333235in}}%
\pgfpathlineto{\pgfqpoint{3.735888in}{2.375017in}}%
\pgfpathlineto{\pgfqpoint{3.736330in}{2.359768in}}%
\pgfpathlineto{\pgfqpoint{3.736772in}{2.181790in}}%
\pgfpathlineto{\pgfqpoint{3.737213in}{2.405386in}}%
\pgfpathlineto{\pgfqpoint{3.737654in}{2.472643in}}%
\pgfpathlineto{\pgfqpoint{3.739851in}{2.151123in}}%
\pgfpathlineto{\pgfqpoint{3.740727in}{2.366481in}}%
\pgfpathlineto{\pgfqpoint{3.741165in}{2.363019in}}%
\pgfpathlineto{\pgfqpoint{3.741601in}{2.211745in}}%
\pgfpathlineto{\pgfqpoint{3.742038in}{2.288312in}}%
\pgfpathlineto{\pgfqpoint{3.742474in}{2.394720in}}%
\pgfpathlineto{\pgfqpoint{3.742909in}{2.244126in}}%
\pgfpathlineto{\pgfqpoint{3.743344in}{2.380885in}}%
\pgfpathlineto{\pgfqpoint{3.743779in}{2.402429in}}%
\pgfpathlineto{\pgfqpoint{3.744213in}{2.263477in}}%
\pgfpathlineto{\pgfqpoint{3.745080in}{2.277200in}}%
\pgfpathlineto{\pgfqpoint{3.745513in}{2.290979in}}%
\pgfpathlineto{\pgfqpoint{3.745945in}{2.194890in}}%
\pgfpathlineto{\pgfqpoint{3.746377in}{2.303625in}}%
\pgfpathlineto{\pgfqpoint{3.746808in}{2.296382in}}%
\pgfpathlineto{\pgfqpoint{3.747239in}{2.390655in}}%
\pgfpathlineto{\pgfqpoint{3.748100in}{2.374334in}}%
\pgfpathlineto{\pgfqpoint{3.748530in}{2.384482in}}%
\pgfpathlineto{\pgfqpoint{3.748959in}{2.170807in}}%
\pgfpathlineto{\pgfqpoint{3.749817in}{2.277693in}}%
\pgfpathlineto{\pgfqpoint{3.750245in}{2.412910in}}%
\pgfpathlineto{\pgfqpoint{3.750672in}{2.212975in}}%
\pgfpathlineto{\pgfqpoint{3.751526in}{2.319139in}}%
\pgfpathlineto{\pgfqpoint{3.751952in}{2.314804in}}%
\pgfpathlineto{\pgfqpoint{3.752378in}{2.210654in}}%
\pgfpathlineto{\pgfqpoint{3.752804in}{2.281593in}}%
\pgfpathlineto{\pgfqpoint{3.753653in}{2.409235in}}%
\pgfpathlineto{\pgfqpoint{3.754501in}{2.390746in}}%
\pgfpathlineto{\pgfqpoint{3.754924in}{2.249388in}}%
\pgfpathlineto{\pgfqpoint{3.755347in}{2.426341in}}%
\pgfpathlineto{\pgfqpoint{3.756192in}{2.435203in}}%
\pgfpathlineto{\pgfqpoint{3.756613in}{2.374552in}}%
\pgfpathlineto{\pgfqpoint{3.757455in}{2.443505in}}%
\pgfpathlineto{\pgfqpoint{3.757876in}{2.237556in}}%
\pgfpathlineto{\pgfqpoint{3.758296in}{2.460855in}}%
\pgfpathlineto{\pgfqpoint{3.758715in}{2.340017in}}%
\pgfpathlineto{\pgfqpoint{3.759134in}{2.374352in}}%
\pgfpathlineto{\pgfqpoint{3.759553in}{2.328394in}}%
\pgfpathlineto{\pgfqpoint{3.759971in}{2.277224in}}%
\pgfpathlineto{\pgfqpoint{3.760389in}{2.393014in}}%
\pgfpathlineto{\pgfqpoint{3.760807in}{2.252035in}}%
\pgfpathlineto{\pgfqpoint{3.761224in}{2.292228in}}%
\pgfpathlineto{\pgfqpoint{3.761640in}{2.196323in}}%
\pgfpathlineto{\pgfqpoint{3.762057in}{2.336809in}}%
\pgfpathlineto{\pgfqpoint{3.762472in}{2.363081in}}%
\pgfpathlineto{\pgfqpoint{3.762888in}{2.338696in}}%
\pgfpathlineto{\pgfqpoint{3.763303in}{2.292731in}}%
\pgfpathlineto{\pgfqpoint{3.763718in}{2.426486in}}%
\pgfpathlineto{\pgfqpoint{3.764132in}{2.222237in}}%
\pgfpathlineto{\pgfqpoint{3.764959in}{2.395621in}}%
\pgfpathlineto{\pgfqpoint{3.765372in}{2.288489in}}%
\pgfpathlineto{\pgfqpoint{3.765785in}{2.342672in}}%
\pgfpathlineto{\pgfqpoint{3.766197in}{2.098225in}}%
\pgfpathlineto{\pgfqpoint{3.766609in}{2.294908in}}%
\pgfpathlineto{\pgfqpoint{3.767020in}{2.313804in}}%
\pgfpathlineto{\pgfqpoint{3.767431in}{2.290592in}}%
\pgfpathlineto{\pgfqpoint{3.767842in}{2.113563in}}%
\pgfpathlineto{\pgfqpoint{3.768252in}{2.372205in}}%
\pgfpathlineto{\pgfqpoint{3.768662in}{2.290518in}}%
\pgfpathlineto{\pgfqpoint{3.769071in}{2.381335in}}%
\pgfpathlineto{\pgfqpoint{3.769889in}{2.382365in}}%
\pgfpathlineto{\pgfqpoint{3.770297in}{2.327340in}}%
\pgfpathlineto{\pgfqpoint{3.771112in}{2.385284in}}%
\pgfpathlineto{\pgfqpoint{3.771926in}{2.083950in}}%
\pgfpathlineto{\pgfqpoint{3.772332in}{2.351575in}}%
\pgfpathlineto{\pgfqpoint{3.773144in}{2.236098in}}%
\pgfpathlineto{\pgfqpoint{3.773549in}{2.165520in}}%
\pgfpathlineto{\pgfqpoint{3.773953in}{2.260603in}}%
\pgfpathlineto{\pgfqpoint{3.774762in}{2.457062in}}%
\pgfpathlineto{\pgfqpoint{3.775569in}{2.393442in}}%
\pgfpathlineto{\pgfqpoint{3.776374in}{2.410360in}}%
\pgfpathlineto{\pgfqpoint{3.777178in}{2.241741in}}%
\pgfpathlineto{\pgfqpoint{3.777579in}{2.347006in}}%
\pgfpathlineto{\pgfqpoint{3.777980in}{2.196455in}}%
\pgfpathlineto{\pgfqpoint{3.778380in}{2.295758in}}%
\pgfpathlineto{\pgfqpoint{3.778780in}{2.280515in}}%
\pgfpathlineto{\pgfqpoint{3.779580in}{2.441363in}}%
\pgfpathlineto{\pgfqpoint{3.779979in}{2.289850in}}%
\pgfpathlineto{\pgfqpoint{3.780377in}{2.037271in}}%
\pgfpathlineto{\pgfqpoint{3.781173in}{2.133913in}}%
\pgfpathlineto{\pgfqpoint{3.781571in}{2.426146in}}%
\pgfpathlineto{\pgfqpoint{3.782365in}{2.284119in}}%
\pgfpathlineto{\pgfqpoint{3.783157in}{2.260969in}}%
\pgfpathlineto{\pgfqpoint{3.783553in}{2.333355in}}%
\pgfpathlineto{\pgfqpoint{3.783948in}{2.223487in}}%
\pgfpathlineto{\pgfqpoint{3.784343in}{2.308170in}}%
\pgfpathlineto{\pgfqpoint{3.784738in}{2.322847in}}%
\pgfpathlineto{\pgfqpoint{3.785526in}{2.276397in}}%
\pgfpathlineto{\pgfqpoint{3.785919in}{2.366213in}}%
\pgfpathlineto{\pgfqpoint{3.786313in}{2.264859in}}%
\pgfpathlineto{\pgfqpoint{3.786705in}{2.335314in}}%
\pgfpathlineto{\pgfqpoint{3.787098in}{2.333875in}}%
\pgfpathlineto{\pgfqpoint{3.787490in}{2.310339in}}%
\pgfpathlineto{\pgfqpoint{3.787881in}{2.114861in}}%
\pgfpathlineto{\pgfqpoint{3.788273in}{2.311520in}}%
\pgfpathlineto{\pgfqpoint{3.788664in}{2.368972in}}%
\pgfpathlineto{\pgfqpoint{3.789054in}{2.214704in}}%
\pgfpathlineto{\pgfqpoint{3.789444in}{2.414976in}}%
\pgfpathlineto{\pgfqpoint{3.789834in}{2.232783in}}%
\pgfpathlineto{\pgfqpoint{3.790224in}{2.298488in}}%
\pgfpathlineto{\pgfqpoint{3.791002in}{2.262061in}}%
\pgfpathlineto{\pgfqpoint{3.791390in}{2.213905in}}%
\pgfpathlineto{\pgfqpoint{3.791778in}{2.254572in}}%
\pgfpathlineto{\pgfqpoint{3.792553in}{2.411774in}}%
\pgfpathlineto{\pgfqpoint{3.792940in}{2.349860in}}%
\pgfpathlineto{\pgfqpoint{3.793327in}{2.186271in}}%
\pgfpathlineto{\pgfqpoint{3.793713in}{2.277049in}}%
\pgfpathlineto{\pgfqpoint{3.794099in}{2.392037in}}%
\pgfpathlineto{\pgfqpoint{3.794485in}{2.342005in}}%
\pgfpathlineto{\pgfqpoint{3.794870in}{2.286880in}}%
\pgfpathlineto{\pgfqpoint{3.795255in}{2.325472in}}%
\pgfpathlineto{\pgfqpoint{3.795640in}{2.379752in}}%
\pgfpathlineto{\pgfqpoint{3.796024in}{2.301825in}}%
\pgfpathlineto{\pgfqpoint{3.796408in}{2.355024in}}%
\pgfpathlineto{\pgfqpoint{3.796791in}{2.142871in}}%
\pgfpathlineto{\pgfqpoint{3.797557in}{2.288364in}}%
\pgfpathlineto{\pgfqpoint{3.797940in}{2.229955in}}%
\pgfpathlineto{\pgfqpoint{3.798322in}{2.427530in}}%
\pgfpathlineto{\pgfqpoint{3.799085in}{2.358547in}}%
\pgfpathlineto{\pgfqpoint{3.799466in}{2.062208in}}%
\pgfpathlineto{\pgfqpoint{3.799847in}{2.277209in}}%
\pgfpathlineto{\pgfqpoint{3.800607in}{2.198631in}}%
\pgfpathlineto{\pgfqpoint{3.800987in}{2.456474in}}%
\pgfpathlineto{\pgfqpoint{3.801367in}{2.188198in}}%
\pgfpathlineto{\pgfqpoint{3.802124in}{2.335633in}}%
\pgfpathlineto{\pgfqpoint{3.802503in}{2.251047in}}%
\pgfpathlineto{\pgfqpoint{3.802881in}{2.409780in}}%
\pgfpathlineto{\pgfqpoint{3.804013in}{2.222352in}}%
\pgfpathlineto{\pgfqpoint{3.804390in}{2.431572in}}%
\pgfpathlineto{\pgfqpoint{3.805142in}{2.428325in}}%
\pgfpathlineto{\pgfqpoint{3.806643in}{2.099719in}}%
\pgfpathlineto{\pgfqpoint{3.808138in}{2.441175in}}%
\pgfpathlineto{\pgfqpoint{3.808511in}{1.976938in}}%
\pgfpathlineto{\pgfqpoint{3.808884in}{2.348335in}}%
\pgfpathlineto{\pgfqpoint{3.809257in}{2.438966in}}%
\pgfpathlineto{\pgfqpoint{3.809629in}{2.426274in}}%
\pgfpathlineto{\pgfqpoint{3.810000in}{2.322114in}}%
\pgfpathlineto{\pgfqpoint{3.810743in}{2.336771in}}%
\pgfpathlineto{\pgfqpoint{3.811484in}{2.397234in}}%
\pgfpathlineto{\pgfqpoint{3.811854in}{2.281183in}}%
\pgfpathlineto{\pgfqpoint{3.812224in}{2.303578in}}%
\pgfpathlineto{\pgfqpoint{3.812594in}{2.444110in}}%
\pgfpathlineto{\pgfqpoint{3.812963in}{2.250541in}}%
\pgfpathlineto{\pgfqpoint{3.813332in}{2.427692in}}%
\pgfpathlineto{\pgfqpoint{3.813700in}{2.250394in}}%
\pgfpathlineto{\pgfqpoint{3.814436in}{2.350053in}}%
\pgfpathlineto{\pgfqpoint{3.815905in}{2.266425in}}%
\pgfpathlineto{\pgfqpoint{3.816271in}{2.318750in}}%
\pgfpathlineto{\pgfqpoint{3.816637in}{2.286655in}}%
\pgfpathlineto{\pgfqpoint{3.817003in}{2.232542in}}%
\pgfpathlineto{\pgfqpoint{3.817368in}{2.288210in}}%
\pgfpathlineto{\pgfqpoint{3.819190in}{2.432449in}}%
\pgfpathlineto{\pgfqpoint{3.820280in}{2.129711in}}%
\pgfpathlineto{\pgfqpoint{3.820642in}{2.276001in}}%
\pgfpathlineto{\pgfqpoint{3.821004in}{2.361173in}}%
\pgfpathlineto{\pgfqpoint{3.821366in}{2.234934in}}%
\pgfpathlineto{\pgfqpoint{3.821728in}{2.354224in}}%
\pgfpathlineto{\pgfqpoint{3.822089in}{2.201138in}}%
\pgfpathlineto{\pgfqpoint{3.822811in}{2.319934in}}%
\pgfpathlineto{\pgfqpoint{3.823171in}{2.419510in}}%
\pgfpathlineto{\pgfqpoint{3.823531in}{2.231023in}}%
\pgfpathlineto{\pgfqpoint{3.824969in}{2.337987in}}%
\pgfpathlineto{\pgfqpoint{3.825327in}{2.346127in}}%
\pgfpathlineto{\pgfqpoint{3.825686in}{2.217810in}}%
\pgfpathlineto{\pgfqpoint{3.826044in}{2.377455in}}%
\pgfpathlineto{\pgfqpoint{3.826401in}{2.432755in}}%
\pgfpathlineto{\pgfqpoint{3.826759in}{2.372806in}}%
\pgfpathlineto{\pgfqpoint{3.827829in}{2.166485in}}%
\pgfpathlineto{\pgfqpoint{3.827472in}{2.440074in}}%
\pgfpathlineto{\pgfqpoint{3.828185in}{2.224661in}}%
\pgfpathlineto{\pgfqpoint{3.828541in}{2.271663in}}%
\pgfpathlineto{\pgfqpoint{3.828896in}{1.935005in}}%
\pgfpathlineto{\pgfqpoint{3.829252in}{2.411991in}}%
\pgfpathlineto{\pgfqpoint{3.829607in}{2.217669in}}%
\pgfpathlineto{\pgfqpoint{3.829961in}{2.179440in}}%
\pgfpathlineto{\pgfqpoint{3.830316in}{2.018766in}}%
\pgfpathlineto{\pgfqpoint{3.830670in}{2.400504in}}%
\pgfpathlineto{\pgfqpoint{3.831377in}{2.363440in}}%
\pgfpathlineto{\pgfqpoint{3.831730in}{2.195188in}}%
\pgfpathlineto{\pgfqpoint{3.832436in}{2.288121in}}%
\pgfpathlineto{\pgfqpoint{3.833140in}{2.383648in}}%
\pgfpathlineto{\pgfqpoint{3.833492in}{2.326221in}}%
\pgfpathlineto{\pgfqpoint{3.833843in}{2.326867in}}%
\pgfpathlineto{\pgfqpoint{3.834194in}{2.375803in}}%
\pgfpathlineto{\pgfqpoint{3.834545in}{2.231040in}}%
\pgfpathlineto{\pgfqpoint{3.835246in}{2.388015in}}%
\pgfpathlineto{\pgfqpoint{3.835596in}{2.272902in}}%
\pgfpathlineto{\pgfqpoint{3.836295in}{2.393410in}}%
\pgfpathlineto{\pgfqpoint{3.836644in}{2.297246in}}%
\pgfpathlineto{\pgfqpoint{3.837342in}{2.387767in}}%
\pgfpathlineto{\pgfqpoint{3.839080in}{2.179254in}}%
\pgfpathlineto{\pgfqpoint{3.841157in}{2.412376in}}%
\pgfpathlineto{\pgfqpoint{3.842536in}{2.283990in}}%
\pgfpathlineto{\pgfqpoint{3.842880in}{2.318153in}}%
\pgfpathlineto{\pgfqpoint{3.843224in}{2.279856in}}%
\pgfpathlineto{\pgfqpoint{3.843567in}{2.283041in}}%
\pgfpathlineto{\pgfqpoint{3.844253in}{2.194576in}}%
\pgfpathlineto{\pgfqpoint{3.845280in}{2.354162in}}%
\pgfpathlineto{\pgfqpoint{3.845622in}{2.352241in}}%
\pgfpathlineto{\pgfqpoint{3.845964in}{2.347136in}}%
\pgfpathlineto{\pgfqpoint{3.846646in}{2.375531in}}%
\pgfpathlineto{\pgfqpoint{3.846987in}{2.283593in}}%
\pgfpathlineto{\pgfqpoint{3.848007in}{2.341702in}}%
\pgfpathlineto{\pgfqpoint{3.848686in}{2.250352in}}%
\pgfpathlineto{\pgfqpoint{3.849025in}{2.141967in}}%
\pgfpathlineto{\pgfqpoint{3.849364in}{2.392061in}}%
\pgfpathlineto{\pgfqpoint{3.850041in}{2.193434in}}%
\pgfpathlineto{\pgfqpoint{3.851054in}{2.396866in}}%
\pgfpathlineto{\pgfqpoint{3.850717in}{2.095551in}}%
\pgfpathlineto{\pgfqpoint{3.851391in}{2.309571in}}%
\pgfpathlineto{\pgfqpoint{3.852065in}{2.340273in}}%
\pgfpathlineto{\pgfqpoint{3.852401in}{2.180820in}}%
\pgfpathlineto{\pgfqpoint{3.852737in}{2.375166in}}%
\pgfpathlineto{\pgfqpoint{3.853073in}{2.035200in}}%
\pgfpathlineto{\pgfqpoint{3.853409in}{2.300688in}}%
\pgfpathlineto{\pgfqpoint{3.854079in}{2.312532in}}%
\pgfpathlineto{\pgfqpoint{3.854749in}{2.140541in}}%
\pgfpathlineto{\pgfqpoint{3.856084in}{2.317915in}}%
\pgfpathlineto{\pgfqpoint{3.856417in}{2.148278in}}%
\pgfpathlineto{\pgfqpoint{3.857083in}{2.298288in}}%
\pgfpathlineto{\pgfqpoint{3.857416in}{2.296475in}}%
\pgfpathlineto{\pgfqpoint{3.857748in}{2.165646in}}%
\pgfpathlineto{\pgfqpoint{3.858411in}{2.277552in}}%
\pgfpathlineto{\pgfqpoint{3.858743in}{2.347739in}}%
\pgfpathlineto{\pgfqpoint{3.859405in}{2.298024in}}%
\pgfpathlineto{\pgfqpoint{3.859735in}{2.015264in}}%
\pgfpathlineto{\pgfqpoint{3.860396in}{2.157633in}}%
\pgfpathlineto{\pgfqpoint{3.860726in}{2.382534in}}%
\pgfpathlineto{\pgfqpoint{3.861385in}{2.346841in}}%
\pgfpathlineto{\pgfqpoint{3.862700in}{2.255794in}}%
\pgfpathlineto{\pgfqpoint{3.863028in}{2.390729in}}%
\pgfpathlineto{\pgfqpoint{3.863683in}{2.228887in}}%
\pgfpathlineto{\pgfqpoint{3.864010in}{2.111823in}}%
\pgfpathlineto{\pgfqpoint{3.864337in}{2.243747in}}%
\pgfpathlineto{\pgfqpoint{3.864664in}{2.388585in}}%
\pgfpathlineto{\pgfqpoint{3.864991in}{2.215386in}}%
\pgfpathlineto{\pgfqpoint{3.865317in}{2.352515in}}%
\pgfpathlineto{\pgfqpoint{3.865969in}{2.229325in}}%
\pgfpathlineto{\pgfqpoint{3.866620in}{2.259699in}}%
\pgfpathlineto{\pgfqpoint{3.866945in}{2.272302in}}%
\pgfpathlineto{\pgfqpoint{3.867918in}{2.401671in}}%
\pgfpathlineto{\pgfqpoint{3.868566in}{2.133355in}}%
\pgfpathlineto{\pgfqpoint{3.868890in}{2.353812in}}%
\pgfpathlineto{\pgfqpoint{3.869213in}{2.365839in}}%
\pgfpathlineto{\pgfqpoint{3.870182in}{2.267468in}}%
\pgfpathlineto{\pgfqpoint{3.870826in}{2.240756in}}%
\pgfpathlineto{\pgfqpoint{3.871470in}{2.406331in}}%
\pgfpathlineto{\pgfqpoint{3.871791in}{2.216052in}}%
\pgfpathlineto{\pgfqpoint{3.872433in}{2.406827in}}%
\pgfpathlineto{\pgfqpoint{3.873714in}{2.228007in}}%
\pgfpathlineto{\pgfqpoint{3.874672in}{2.279973in}}%
\pgfpathlineto{\pgfqpoint{3.874991in}{2.303568in}}%
\pgfpathlineto{\pgfqpoint{3.875310in}{2.433433in}}%
\pgfpathlineto{\pgfqpoint{3.875947in}{2.364428in}}%
\pgfpathlineto{\pgfqpoint{3.876265in}{2.219885in}}%
\pgfpathlineto{\pgfqpoint{3.876583in}{2.408008in}}%
\pgfpathlineto{\pgfqpoint{3.876900in}{2.293135in}}%
\pgfpathlineto{\pgfqpoint{3.877534in}{2.345853in}}%
\pgfpathlineto{\pgfqpoint{3.877851in}{2.311269in}}%
\pgfpathlineto{\pgfqpoint{3.878168in}{2.294697in}}%
\pgfpathlineto{\pgfqpoint{3.878484in}{2.348346in}}%
\pgfpathlineto{\pgfqpoint{3.879432in}{2.322877in}}%
\pgfpathlineto{\pgfqpoint{3.880377in}{2.142968in}}%
\pgfpathlineto{\pgfqpoint{3.881321in}{2.424684in}}%
\pgfpathlineto{\pgfqpoint{3.881635in}{2.300753in}}%
\pgfpathlineto{\pgfqpoint{3.881948in}{2.192037in}}%
\pgfpathlineto{\pgfqpoint{3.882262in}{2.353838in}}%
\pgfpathlineto{\pgfqpoint{3.882575in}{2.285352in}}%
\pgfpathlineto{\pgfqpoint{3.883201in}{2.359722in}}%
\pgfpathlineto{\pgfqpoint{3.883826in}{2.327144in}}%
\pgfpathlineto{\pgfqpoint{3.884138in}{2.285761in}}%
\pgfpathlineto{\pgfqpoint{3.884450in}{2.324929in}}%
\pgfpathlineto{\pgfqpoint{3.884762in}{2.008841in}}%
\pgfpathlineto{\pgfqpoint{3.885384in}{2.382919in}}%
\pgfpathlineto{\pgfqpoint{3.885695in}{2.108073in}}%
\pgfpathlineto{\pgfqpoint{3.886006in}{2.427966in}}%
\pgfpathlineto{\pgfqpoint{3.886937in}{2.317321in}}%
\pgfpathlineto{\pgfqpoint{3.887247in}{2.320446in}}%
\pgfpathlineto{\pgfqpoint{3.887866in}{2.057270in}}%
\pgfpathlineto{\pgfqpoint{3.888484in}{2.256090in}}%
\pgfpathlineto{\pgfqpoint{3.888793in}{2.167363in}}%
\pgfpathlineto{\pgfqpoint{3.889101in}{2.214020in}}%
\pgfpathlineto{\pgfqpoint{3.889410in}{2.321460in}}%
\pgfpathlineto{\pgfqpoint{3.890026in}{2.301604in}}%
\pgfpathlineto{\pgfqpoint{3.890333in}{1.965606in}}%
\pgfpathlineto{\pgfqpoint{3.890641in}{2.330697in}}%
\pgfpathlineto{\pgfqpoint{3.890948in}{2.176970in}}%
\pgfpathlineto{\pgfqpoint{3.891562in}{2.382299in}}%
\pgfpathlineto{\pgfqpoint{3.891868in}{2.304499in}}%
\pgfpathlineto{\pgfqpoint{3.892480in}{2.216334in}}%
\pgfpathlineto{\pgfqpoint{3.893092in}{2.217816in}}%
\pgfpathlineto{\pgfqpoint{3.893703in}{2.362957in}}%
\pgfpathlineto{\pgfqpoint{3.894312in}{2.293782in}}%
\pgfpathlineto{\pgfqpoint{3.894921in}{2.192108in}}%
\pgfpathlineto{\pgfqpoint{3.895529in}{2.250467in}}%
\pgfpathlineto{\pgfqpoint{3.896136in}{2.370382in}}%
\pgfpathlineto{\pgfqpoint{3.896439in}{2.184170in}}%
\pgfpathlineto{\pgfqpoint{3.897045in}{2.264690in}}%
\pgfpathlineto{\pgfqpoint{3.897348in}{2.358399in}}%
\pgfpathlineto{\pgfqpoint{3.897650in}{2.331305in}}%
\pgfpathlineto{\pgfqpoint{3.897952in}{2.012964in}}%
\pgfpathlineto{\pgfqpoint{3.898857in}{2.201171in}}%
\pgfpathlineto{\pgfqpoint{3.899159in}{2.352827in}}%
\pgfpathlineto{\pgfqpoint{3.900061in}{2.332057in}}%
\pgfpathlineto{\pgfqpoint{3.901262in}{2.176271in}}%
\pgfpathlineto{\pgfqpoint{3.902160in}{2.194574in}}%
\pgfpathlineto{\pgfqpoint{3.902758in}{2.376803in}}%
\pgfpathlineto{\pgfqpoint{3.903354in}{2.359978in}}%
\pgfpathlineto{\pgfqpoint{3.903950in}{2.114776in}}%
\pgfpathlineto{\pgfqpoint{3.904546in}{2.298611in}}%
\pgfpathlineto{\pgfqpoint{3.906030in}{1.967222in}}%
\pgfpathlineto{\pgfqpoint{3.907214in}{2.380677in}}%
\pgfpathlineto{\pgfqpoint{3.907804in}{2.137133in}}%
\pgfpathlineto{\pgfqpoint{3.908099in}{2.387900in}}%
\pgfpathlineto{\pgfqpoint{3.908689in}{2.216594in}}%
\pgfpathlineto{\pgfqpoint{3.909277in}{2.341442in}}%
\pgfpathlineto{\pgfqpoint{3.909571in}{2.303300in}}%
\pgfpathlineto{\pgfqpoint{3.909865in}{2.143649in}}%
\pgfpathlineto{\pgfqpoint{3.910452in}{2.170499in}}%
\pgfpathlineto{\pgfqpoint{3.911038in}{2.303704in}}%
\pgfpathlineto{\pgfqpoint{3.911331in}{2.070549in}}%
\pgfpathlineto{\pgfqpoint{3.911916in}{2.200121in}}%
\pgfpathlineto{\pgfqpoint{3.912208in}{2.428448in}}%
\pgfpathlineto{\pgfqpoint{3.913083in}{2.377828in}}%
\pgfpathlineto{\pgfqpoint{3.913375in}{2.174298in}}%
\pgfpathlineto{\pgfqpoint{3.913666in}{2.407690in}}%
\pgfpathlineto{\pgfqpoint{3.913957in}{2.245337in}}%
\pgfpathlineto{\pgfqpoint{3.914247in}{2.349395in}}%
\pgfpathlineto{\pgfqpoint{3.914538in}{2.129713in}}%
\pgfpathlineto{\pgfqpoint{3.915118in}{2.332239in}}%
\pgfpathlineto{\pgfqpoint{3.915698in}{2.022532in}}%
\pgfpathlineto{\pgfqpoint{3.916277in}{2.244171in}}%
\pgfpathlineto{\pgfqpoint{3.917144in}{2.277102in}}%
\pgfpathlineto{\pgfqpoint{3.917433in}{2.155459in}}%
\pgfpathlineto{\pgfqpoint{3.917721in}{2.352873in}}%
\pgfpathlineto{\pgfqpoint{3.918297in}{2.193096in}}%
\pgfpathlineto{\pgfqpoint{3.918585in}{2.388735in}}%
\pgfpathlineto{\pgfqpoint{3.918873in}{2.173532in}}%
\pgfpathlineto{\pgfqpoint{3.919447in}{2.253724in}}%
\pgfpathlineto{\pgfqpoint{3.919734in}{2.068474in}}%
\pgfpathlineto{\pgfqpoint{3.920308in}{2.397869in}}%
\pgfpathlineto{\pgfqpoint{3.920880in}{2.201198in}}%
\pgfpathlineto{\pgfqpoint{3.921738in}{2.348637in}}%
\pgfpathlineto{\pgfqpoint{3.922024in}{2.356928in}}%
\pgfpathlineto{\pgfqpoint{3.922879in}{2.174083in}}%
\pgfpathlineto{\pgfqpoint{3.923448in}{2.216654in}}%
\pgfpathlineto{\pgfqpoint{3.924584in}{2.300101in}}%
\pgfpathlineto{\pgfqpoint{3.924868in}{2.179400in}}%
\pgfpathlineto{\pgfqpoint{3.925717in}{2.211928in}}%
\pgfpathlineto{\pgfqpoint{3.926000in}{2.348867in}}%
\pgfpathlineto{\pgfqpoint{3.926565in}{2.229747in}}%
\pgfpathlineto{\pgfqpoint{3.926847in}{2.218934in}}%
\pgfpathlineto{\pgfqpoint{3.927130in}{2.391003in}}%
\pgfpathlineto{\pgfqpoint{3.927411in}{2.191127in}}%
\pgfpathlineto{\pgfqpoint{3.927693in}{2.264492in}}%
\pgfpathlineto{\pgfqpoint{3.927975in}{1.945256in}}%
\pgfpathlineto{\pgfqpoint{3.928537in}{2.329286in}}%
\pgfpathlineto{\pgfqpoint{3.929099in}{2.256733in}}%
\pgfpathlineto{\pgfqpoint{3.929660in}{2.308176in}}%
\pgfpathlineto{\pgfqpoint{3.929940in}{2.360488in}}%
\pgfpathlineto{\pgfqpoint{3.930500in}{2.276708in}}%
\pgfpathlineto{\pgfqpoint{3.930779in}{2.219299in}}%
\pgfpathlineto{\pgfqpoint{3.931059in}{2.306950in}}%
\pgfpathlineto{\pgfqpoint{3.931338in}{2.335859in}}%
\pgfpathlineto{\pgfqpoint{3.932453in}{2.196355in}}%
\pgfpathlineto{\pgfqpoint{3.933288in}{2.176573in}}%
\pgfpathlineto{\pgfqpoint{3.933566in}{2.333125in}}%
\pgfpathlineto{\pgfqpoint{3.934121in}{2.163765in}}%
\pgfpathlineto{\pgfqpoint{3.934398in}{2.304907in}}%
\pgfpathlineto{\pgfqpoint{3.934675in}{2.358130in}}%
\pgfpathlineto{\pgfqpoint{3.934952in}{2.287164in}}%
\pgfpathlineto{\pgfqpoint{3.935782in}{2.116406in}}%
\pgfpathlineto{\pgfqpoint{3.936058in}{2.293198in}}%
\pgfpathlineto{\pgfqpoint{3.936885in}{2.149385in}}%
\pgfpathlineto{\pgfqpoint{3.937161in}{2.156653in}}%
\pgfpathlineto{\pgfqpoint{3.937711in}{2.154171in}}%
\pgfpathlineto{\pgfqpoint{3.937986in}{2.368802in}}%
\pgfpathlineto{\pgfqpoint{3.938261in}{2.127638in}}%
\pgfpathlineto{\pgfqpoint{3.939084in}{2.158664in}}%
\pgfpathlineto{\pgfqpoint{3.940726in}{2.401842in}}%
\pgfpathlineto{\pgfqpoint{3.941544in}{2.202713in}}%
\pgfpathlineto{\pgfqpoint{3.942089in}{2.230438in}}%
\pgfpathlineto{\pgfqpoint{3.942361in}{2.256680in}}%
\pgfpathlineto{\pgfqpoint{3.942633in}{2.164616in}}%
\pgfpathlineto{\pgfqpoint{3.943176in}{2.343879in}}%
\pgfpathlineto{\pgfqpoint{3.943719in}{2.248712in}}%
\pgfpathlineto{\pgfqpoint{3.943990in}{2.387880in}}%
\pgfpathlineto{\pgfqpoint{3.944532in}{2.225706in}}%
\pgfpathlineto{\pgfqpoint{3.944802in}{2.284304in}}%
\pgfpathlineto{\pgfqpoint{3.945073in}{2.261785in}}%
\pgfpathlineto{\pgfqpoint{3.945343in}{2.008817in}}%
\pgfpathlineto{\pgfqpoint{3.945883in}{2.340383in}}%
\pgfpathlineto{\pgfqpoint{3.946153in}{2.227810in}}%
\pgfpathlineto{\pgfqpoint{3.947230in}{2.087059in}}%
\pgfpathlineto{\pgfqpoint{3.947498in}{2.299095in}}%
\pgfpathlineto{\pgfqpoint{3.948036in}{2.210352in}}%
\pgfpathlineto{\pgfqpoint{3.948840in}{1.997364in}}%
\pgfpathlineto{\pgfqpoint{3.948572in}{2.351904in}}%
\pgfpathlineto{\pgfqpoint{3.949108in}{2.227258in}}%
\pgfpathlineto{\pgfqpoint{3.949376in}{2.314852in}}%
\pgfpathlineto{\pgfqpoint{3.950178in}{2.219457in}}%
\pgfpathlineto{\pgfqpoint{3.950978in}{2.345644in}}%
\pgfpathlineto{\pgfqpoint{3.951777in}{2.143933in}}%
\pgfpathlineto{\pgfqpoint{3.952043in}{2.328086in}}%
\pgfpathlineto{\pgfqpoint{3.952574in}{2.253891in}}%
\pgfpathlineto{\pgfqpoint{3.952840in}{2.100067in}}%
\pgfpathlineto{\pgfqpoint{3.953370in}{2.341905in}}%
\pgfpathlineto{\pgfqpoint{3.953635in}{2.268526in}}%
\pgfpathlineto{\pgfqpoint{3.953900in}{2.283466in}}%
\pgfpathlineto{\pgfqpoint{3.954165in}{2.255082in}}%
\pgfpathlineto{\pgfqpoint{3.954693in}{2.220200in}}%
\pgfpathlineto{\pgfqpoint{3.954958in}{2.294102in}}%
\pgfpathlineto{\pgfqpoint{3.955222in}{1.960444in}}%
\pgfpathlineto{\pgfqpoint{3.956013in}{2.336041in}}%
\pgfpathlineto{\pgfqpoint{3.956276in}{2.237177in}}%
\pgfpathlineto{\pgfqpoint{3.956539in}{2.351710in}}%
\pgfpathlineto{\pgfqpoint{3.956802in}{2.349651in}}%
\pgfpathlineto{\pgfqpoint{3.957065in}{2.392010in}}%
\pgfpathlineto{\pgfqpoint{3.957853in}{2.040265in}}%
\pgfpathlineto{\pgfqpoint{3.958115in}{2.232620in}}%
\pgfpathlineto{\pgfqpoint{3.958377in}{2.252779in}}%
\pgfpathlineto{\pgfqpoint{3.958639in}{2.116672in}}%
\pgfpathlineto{\pgfqpoint{3.958900in}{2.291499in}}%
\pgfpathlineto{\pgfqpoint{3.959162in}{2.250409in}}%
\pgfpathlineto{\pgfqpoint{3.959423in}{2.320337in}}%
\pgfpathlineto{\pgfqpoint{3.959684in}{2.307713in}}%
\pgfpathlineto{\pgfqpoint{3.959946in}{2.154217in}}%
\pgfpathlineto{\pgfqpoint{3.960728in}{2.292426in}}%
\pgfpathlineto{\pgfqpoint{3.960988in}{2.319282in}}%
\pgfpathlineto{\pgfqpoint{3.961248in}{2.264525in}}%
\pgfpathlineto{\pgfqpoint{3.961509in}{2.168901in}}%
\pgfpathlineto{\pgfqpoint{3.962288in}{2.256702in}}%
\pgfpathlineto{\pgfqpoint{3.962547in}{2.264490in}}%
\pgfpathlineto{\pgfqpoint{3.963066in}{2.195881in}}%
\pgfpathlineto{\pgfqpoint{3.963325in}{2.373566in}}%
\pgfpathlineto{\pgfqpoint{3.963584in}{2.176461in}}%
\pgfpathlineto{\pgfqpoint{3.963842in}{2.294984in}}%
\pgfpathlineto{\pgfqpoint{3.964101in}{1.961852in}}%
\pgfpathlineto{\pgfqpoint{3.964359in}{2.384162in}}%
\pgfpathlineto{\pgfqpoint{3.964876in}{2.255753in}}%
\pgfpathlineto{\pgfqpoint{3.965649in}{2.400461in}}%
\pgfpathlineto{\pgfqpoint{3.966163in}{2.354436in}}%
\pgfpathlineto{\pgfqpoint{3.966678in}{2.118738in}}%
\pgfpathlineto{\pgfqpoint{3.967191in}{2.234686in}}%
\pgfpathlineto{\pgfqpoint{3.967960in}{2.302059in}}%
\pgfpathlineto{\pgfqpoint{3.967704in}{2.187796in}}%
\pgfpathlineto{\pgfqpoint{3.968216in}{2.242548in}}%
\pgfpathlineto{\pgfqpoint{3.968472in}{2.213400in}}%
\pgfpathlineto{\pgfqpoint{3.968983in}{2.254749in}}%
\pgfpathlineto{\pgfqpoint{3.969239in}{2.388541in}}%
\pgfpathlineto{\pgfqpoint{3.969494in}{2.150780in}}%
\pgfpathlineto{\pgfqpoint{3.969749in}{2.285621in}}%
\pgfpathlineto{\pgfqpoint{3.970004in}{1.962326in}}%
\pgfpathlineto{\pgfqpoint{3.970513in}{2.377176in}}%
\pgfpathlineto{\pgfqpoint{3.970768in}{2.260611in}}%
\pgfpathlineto{\pgfqpoint{3.971022in}{2.096411in}}%
\pgfpathlineto{\pgfqpoint{3.971530in}{2.308180in}}%
\pgfpathlineto{\pgfqpoint{3.971784in}{2.197394in}}%
\pgfpathlineto{\pgfqpoint{3.972292in}{2.196059in}}%
\pgfpathlineto{\pgfqpoint{3.972798in}{2.292125in}}%
\pgfpathlineto{\pgfqpoint{3.973051in}{2.043645in}}%
\pgfpathlineto{\pgfqpoint{3.973557in}{2.355255in}}%
\pgfpathlineto{\pgfqpoint{3.973810in}{2.305721in}}%
\pgfpathlineto{\pgfqpoint{3.975071in}{2.180585in}}%
\pgfpathlineto{\pgfqpoint{3.975826in}{2.278519in}}%
\pgfpathlineto{\pgfqpoint{3.976077in}{2.187183in}}%
\pgfpathlineto{\pgfqpoint{3.976328in}{2.094595in}}%
\pgfpathlineto{\pgfqpoint{3.976830in}{2.271543in}}%
\pgfpathlineto{\pgfqpoint{3.977582in}{2.250104in}}%
\pgfpathlineto{\pgfqpoint{3.977832in}{2.367277in}}%
\pgfpathlineto{\pgfqpoint{3.978332in}{2.114425in}}%
\pgfpathlineto{\pgfqpoint{3.978832in}{2.286506in}}%
\pgfpathlineto{\pgfqpoint{3.979331in}{2.290941in}}%
\pgfpathlineto{\pgfqpoint{3.979829in}{2.303578in}}%
\pgfpathlineto{\pgfqpoint{3.980327in}{2.151277in}}%
\pgfpathlineto{\pgfqpoint{3.980576in}{2.354446in}}%
\pgfpathlineto{\pgfqpoint{3.981321in}{2.123910in}}%
\pgfpathlineto{\pgfqpoint{3.981569in}{2.310440in}}%
\pgfpathlineto{\pgfqpoint{3.982560in}{2.186108in}}%
\pgfpathlineto{\pgfqpoint{3.983302in}{2.216631in}}%
\pgfpathlineto{\pgfqpoint{3.983548in}{2.199164in}}%
\pgfpathlineto{\pgfqpoint{3.983795in}{2.072181in}}%
\pgfpathlineto{\pgfqpoint{3.984042in}{2.300659in}}%
\pgfpathlineto{\pgfqpoint{3.984535in}{2.166982in}}%
\pgfpathlineto{\pgfqpoint{3.984781in}{2.332297in}}%
\pgfpathlineto{\pgfqpoint{3.985519in}{2.328891in}}%
\pgfpathlineto{\pgfqpoint{3.985764in}{1.967737in}}%
\pgfpathlineto{\pgfqpoint{3.986010in}{2.354006in}}%
\pgfpathlineto{\pgfqpoint{3.986501in}{2.187628in}}%
\pgfpathlineto{\pgfqpoint{3.986746in}{2.196400in}}%
\pgfpathlineto{\pgfqpoint{3.986991in}{2.165291in}}%
\pgfpathlineto{\pgfqpoint{3.987235in}{2.264279in}}%
\pgfpathlineto{\pgfqpoint{3.987480in}{2.334866in}}%
\pgfpathlineto{\pgfqpoint{3.987725in}{2.236225in}}%
\pgfpathlineto{\pgfqpoint{3.988213in}{2.273388in}}%
\pgfpathlineto{\pgfqpoint{3.988701in}{2.085388in}}%
\pgfpathlineto{\pgfqpoint{3.988945in}{2.299661in}}%
\pgfpathlineto{\pgfqpoint{3.989189in}{2.245724in}}%
\pgfpathlineto{\pgfqpoint{3.989919in}{2.365610in}}%
\pgfpathlineto{\pgfqpoint{3.990405in}{2.339530in}}%
\pgfpathlineto{\pgfqpoint{3.990648in}{2.359017in}}%
\pgfpathlineto{\pgfqpoint{3.990891in}{2.350755in}}%
\pgfpathlineto{\pgfqpoint{3.991133in}{2.260294in}}%
\pgfpathlineto{\pgfqpoint{3.991860in}{2.296793in}}%
\pgfpathlineto{\pgfqpoint{3.992102in}{2.348161in}}%
\pgfpathlineto{\pgfqpoint{3.992586in}{2.257447in}}%
\pgfpathlineto{\pgfqpoint{3.992827in}{2.301517in}}%
\pgfpathlineto{\pgfqpoint{3.993069in}{2.293511in}}%
\pgfpathlineto{\pgfqpoint{3.993310in}{2.190664in}}%
\pgfpathlineto{\pgfqpoint{3.993551in}{2.315642in}}%
\pgfpathlineto{\pgfqpoint{3.994033in}{2.228736in}}%
\pgfpathlineto{\pgfqpoint{3.994515in}{2.360688in}}%
\pgfpathlineto{\pgfqpoint{3.994755in}{2.176435in}}%
\pgfpathlineto{\pgfqpoint{3.995236in}{2.311163in}}%
\pgfpathlineto{\pgfqpoint{3.995716in}{2.134631in}}%
\pgfpathlineto{\pgfqpoint{3.995955in}{2.324766in}}%
\pgfpathlineto{\pgfqpoint{3.996674in}{2.207455in}}%
\pgfpathlineto{\pgfqpoint{3.996913in}{2.398167in}}%
\pgfpathlineto{\pgfqpoint{3.997630in}{2.317888in}}%
\pgfpathlineto{\pgfqpoint{3.997869in}{2.082729in}}%
\pgfpathlineto{\pgfqpoint{3.998346in}{2.339569in}}%
\pgfpathlineto{\pgfqpoint{3.998585in}{2.255468in}}%
\pgfpathlineto{\pgfqpoint{3.998823in}{2.306487in}}%
\pgfpathlineto{\pgfqpoint{3.999299in}{2.252412in}}%
\pgfpathlineto{\pgfqpoint{3.999537in}{2.213282in}}%
\pgfpathlineto{\pgfqpoint{3.999774in}{2.272599in}}%
\pgfpathlineto{\pgfqpoint{4.000012in}{2.319525in}}%
\pgfpathlineto{\pgfqpoint{4.000249in}{2.089282in}}%
\pgfpathlineto{\pgfqpoint{4.000724in}{2.366560in}}%
\pgfpathlineto{\pgfqpoint{4.000961in}{2.295550in}}%
\pgfpathlineto{\pgfqpoint{4.001198in}{2.319967in}}%
\pgfpathlineto{\pgfqpoint{4.002144in}{2.152238in}}%
\pgfpathlineto{\pgfqpoint{4.002380in}{2.172274in}}%
\pgfpathlineto{\pgfqpoint{4.002616in}{2.360105in}}%
\pgfpathlineto{\pgfqpoint{4.002852in}{2.083173in}}%
\pgfpathlineto{\pgfqpoint{4.003559in}{2.276905in}}%
\pgfpathlineto{\pgfqpoint{4.004030in}{2.273236in}}%
\pgfpathlineto{\pgfqpoint{4.004265in}{2.330943in}}%
\pgfpathlineto{\pgfqpoint{4.004970in}{2.133660in}}%
\pgfpathlineto{\pgfqpoint{4.005204in}{2.244497in}}%
\pgfpathlineto{\pgfqpoint{4.005439in}{2.335396in}}%
\pgfpathlineto{\pgfqpoint{4.005673in}{1.822087in}}%
\pgfpathlineto{\pgfqpoint{4.006376in}{2.216251in}}%
\pgfpathlineto{\pgfqpoint{4.006610in}{2.319423in}}%
\pgfpathlineto{\pgfqpoint{4.007310in}{2.172839in}}%
\pgfpathlineto{\pgfqpoint{4.008010in}{2.166519in}}%
\pgfpathlineto{\pgfqpoint{4.008709in}{2.317661in}}%
\pgfpathlineto{\pgfqpoint{4.008941in}{1.991441in}}%
\pgfpathlineto{\pgfqpoint{4.009406in}{2.339209in}}%
\pgfpathlineto{\pgfqpoint{4.009870in}{2.183528in}}%
\pgfpathlineto{\pgfqpoint{4.010102in}{2.149345in}}%
\pgfpathlineto{\pgfqpoint{4.010566in}{2.193831in}}%
\pgfpathlineto{\pgfqpoint{4.010797in}{2.311022in}}%
\pgfpathlineto{\pgfqpoint{4.011029in}{2.185153in}}%
\pgfpathlineto{\pgfqpoint{4.011722in}{2.238081in}}%
\pgfpathlineto{\pgfqpoint{4.012415in}{2.295111in}}%
\pgfpathlineto{\pgfqpoint{4.013106in}{1.932244in}}%
\pgfpathlineto{\pgfqpoint{4.013566in}{2.316955in}}%
\pgfpathlineto{\pgfqpoint{4.014256in}{2.280456in}}%
\pgfpathlineto{\pgfqpoint{4.014944in}{2.397470in}}%
\pgfpathlineto{\pgfqpoint{4.015631in}{1.951851in}}%
\pgfpathlineto{\pgfqpoint{4.016318in}{2.105353in}}%
\pgfpathlineto{\pgfqpoint{4.017459in}{2.334147in}}%
\pgfpathlineto{\pgfqpoint{4.017231in}{2.087127in}}%
\pgfpathlineto{\pgfqpoint{4.017686in}{2.244924in}}%
\pgfpathlineto{\pgfqpoint{4.018142in}{2.149132in}}%
\pgfpathlineto{\pgfqpoint{4.018369in}{2.152540in}}%
\pgfpathlineto{\pgfqpoint{4.018824in}{2.366444in}}%
\pgfpathlineto{\pgfqpoint{4.019278in}{1.896892in}}%
\pgfpathlineto{\pgfqpoint{4.019732in}{2.390916in}}%
\pgfpathlineto{\pgfqpoint{4.020411in}{2.287157in}}%
\pgfpathlineto{\pgfqpoint{4.020637in}{2.196530in}}%
\pgfpathlineto{\pgfqpoint{4.021315in}{2.303829in}}%
\pgfpathlineto{\pgfqpoint{4.021541in}{2.259482in}}%
\pgfpathlineto{\pgfqpoint{4.021992in}{2.288363in}}%
\pgfpathlineto{\pgfqpoint{4.022893in}{2.355811in}}%
\pgfpathlineto{\pgfqpoint{4.022668in}{2.221891in}}%
\pgfpathlineto{\pgfqpoint{4.023118in}{2.307550in}}%
\pgfpathlineto{\pgfqpoint{4.024466in}{1.948057in}}%
\pgfpathlineto{\pgfqpoint{4.025585in}{2.340142in}}%
\pgfpathlineto{\pgfqpoint{4.026255in}{2.195303in}}%
\pgfpathlineto{\pgfqpoint{4.026478in}{2.365572in}}%
\pgfpathlineto{\pgfqpoint{4.026701in}{2.247205in}}%
\pgfpathlineto{\pgfqpoint{4.026924in}{2.321075in}}%
\pgfpathlineto{\pgfqpoint{4.027815in}{2.270727in}}%
\pgfpathlineto{\pgfqpoint{4.028260in}{2.340752in}}%
\pgfpathlineto{\pgfqpoint{4.029147in}{2.080585in}}%
\pgfpathlineto{\pgfqpoint{4.029591in}{2.290327in}}%
\pgfpathlineto{\pgfqpoint{4.030476in}{2.278885in}}%
\pgfpathlineto{\pgfqpoint{4.030697in}{2.275324in}}%
\pgfpathlineto{\pgfqpoint{4.031800in}{2.149011in}}%
\pgfpathlineto{\pgfqpoint{4.031359in}{2.343119in}}%
\pgfpathlineto{\pgfqpoint{4.032020in}{2.168188in}}%
\pgfpathlineto{\pgfqpoint{4.032240in}{2.235818in}}%
\pgfpathlineto{\pgfqpoint{4.032460in}{2.174375in}}%
\pgfpathlineto{\pgfqpoint{4.032680in}{2.018067in}}%
\pgfpathlineto{\pgfqpoint{4.032900in}{2.286268in}}%
\pgfpathlineto{\pgfqpoint{4.033339in}{2.176856in}}%
\pgfpathlineto{\pgfqpoint{4.033997in}{2.315281in}}%
\pgfpathlineto{\pgfqpoint{4.034435in}{2.255740in}}%
\pgfpathlineto{\pgfqpoint{4.035092in}{2.214366in}}%
\pgfpathlineto{\pgfqpoint{4.035529in}{2.364708in}}%
\pgfpathlineto{\pgfqpoint{4.035965in}{2.162919in}}%
\pgfpathlineto{\pgfqpoint{4.036184in}{2.099236in}}%
\pgfpathlineto{\pgfqpoint{4.036402in}{2.206896in}}%
\pgfpathlineto{\pgfqpoint{4.037273in}{2.157311in}}%
\pgfpathlineto{\pgfqpoint{4.037490in}{2.385986in}}%
\pgfpathlineto{\pgfqpoint{4.038576in}{2.074653in}}%
\pgfpathlineto{\pgfqpoint{4.038792in}{2.280230in}}%
\pgfpathlineto{\pgfqpoint{4.039009in}{2.313651in}}%
\pgfpathlineto{\pgfqpoint{4.039226in}{2.210798in}}%
\pgfpathlineto{\pgfqpoint{4.039442in}{2.246514in}}%
\pgfpathlineto{\pgfqpoint{4.039875in}{2.044514in}}%
\pgfpathlineto{\pgfqpoint{4.040523in}{2.222316in}}%
\pgfpathlineto{\pgfqpoint{4.041170in}{2.384573in}}%
\pgfpathlineto{\pgfqpoint{4.040954in}{2.203754in}}%
\pgfpathlineto{\pgfqpoint{4.041601in}{2.261431in}}%
\pgfpathlineto{\pgfqpoint{4.042246in}{2.117230in}}%
\pgfpathlineto{\pgfqpoint{4.042890in}{2.173097in}}%
\pgfpathlineto{\pgfqpoint{4.043748in}{2.335046in}}%
\pgfpathlineto{\pgfqpoint{4.043320in}{2.095911in}}%
\pgfpathlineto{\pgfqpoint{4.043962in}{2.204342in}}%
\pgfpathlineto{\pgfqpoint{4.044176in}{2.236520in}}%
\pgfpathlineto{\pgfqpoint{4.044390in}{2.195590in}}%
\pgfpathlineto{\pgfqpoint{4.044604in}{2.091850in}}%
\pgfpathlineto{\pgfqpoint{4.044818in}{2.313085in}}%
\pgfpathlineto{\pgfqpoint{4.045458in}{2.175847in}}%
\pgfpathlineto{\pgfqpoint{4.046737in}{2.327618in}}%
\pgfpathlineto{\pgfqpoint{4.048223in}{1.935091in}}%
\pgfpathlineto{\pgfqpoint{4.048435in}{2.304152in}}%
\pgfpathlineto{\pgfqpoint{4.049493in}{2.251777in}}%
\pgfpathlineto{\pgfqpoint{4.049704in}{2.035141in}}%
\pgfpathlineto{\pgfqpoint{4.050337in}{2.159648in}}%
\pgfpathlineto{\pgfqpoint{4.050970in}{2.326567in}}%
\pgfpathlineto{\pgfqpoint{4.051601in}{2.290824in}}%
\pgfpathlineto{\pgfqpoint{4.052441in}{2.218109in}}%
\pgfpathlineto{\pgfqpoint{4.052651in}{2.361136in}}%
\pgfpathlineto{\pgfqpoint{4.053489in}{2.191924in}}%
\pgfpathlineto{\pgfqpoint{4.053699in}{2.191275in}}%
\pgfpathlineto{\pgfqpoint{4.054326in}{2.147837in}}%
\pgfpathlineto{\pgfqpoint{4.054952in}{2.317392in}}%
\pgfpathlineto{\pgfqpoint{4.055578in}{2.084345in}}%
\pgfpathlineto{\pgfqpoint{4.056202in}{2.193853in}}%
\pgfpathlineto{\pgfqpoint{4.056410in}{2.194034in}}%
\pgfpathlineto{\pgfqpoint{4.056618in}{2.140808in}}%
\pgfpathlineto{\pgfqpoint{4.057034in}{2.296401in}}%
\pgfpathlineto{\pgfqpoint{4.057241in}{2.225498in}}%
\pgfpathlineto{\pgfqpoint{4.057449in}{2.307094in}}%
\pgfpathlineto{\pgfqpoint{4.058070in}{2.118897in}}%
\pgfpathlineto{\pgfqpoint{4.058277in}{2.290343in}}%
\pgfpathlineto{\pgfqpoint{4.058691in}{2.318726in}}%
\pgfpathlineto{\pgfqpoint{4.059311in}{2.084013in}}%
\pgfpathlineto{\pgfqpoint{4.059930in}{2.329531in}}%
\pgfpathlineto{\pgfqpoint{4.060548in}{2.237809in}}%
\pgfpathlineto{\pgfqpoint{4.060754in}{2.261150in}}%
\pgfpathlineto{\pgfqpoint{4.061371in}{1.893302in}}%
\pgfpathlineto{\pgfqpoint{4.061782in}{2.328082in}}%
\pgfpathlineto{\pgfqpoint{4.062807in}{1.661624in}}%
\pgfpathlineto{\pgfqpoint{4.063830in}{2.313922in}}%
\pgfpathlineto{\pgfqpoint{4.064034in}{2.297593in}}%
\pgfpathlineto{\pgfqpoint{4.064851in}{2.156645in}}%
\pgfpathlineto{\pgfqpoint{4.065054in}{2.241142in}}%
\pgfpathlineto{\pgfqpoint{4.065258in}{2.314288in}}%
\pgfpathlineto{\pgfqpoint{4.065462in}{2.020129in}}%
\pgfpathlineto{\pgfqpoint{4.066072in}{2.297290in}}%
\pgfpathlineto{\pgfqpoint{4.067087in}{2.120807in}}%
\pgfpathlineto{\pgfqpoint{4.067289in}{2.255156in}}%
\pgfpathlineto{\pgfqpoint{4.067695in}{2.052701in}}%
\pgfpathlineto{\pgfqpoint{4.067897in}{2.120161in}}%
\pgfpathlineto{\pgfqpoint{4.068099in}{2.040839in}}%
\pgfpathlineto{\pgfqpoint{4.068504in}{2.263659in}}%
\pgfpathlineto{\pgfqpoint{4.068706in}{2.255331in}}%
\pgfpathlineto{\pgfqpoint{4.069513in}{2.087550in}}%
\pgfpathlineto{\pgfqpoint{4.069109in}{2.360547in}}%
\pgfpathlineto{\pgfqpoint{4.069916in}{2.160841in}}%
\pgfpathlineto{\pgfqpoint{4.070721in}{2.319050in}}%
\pgfpathlineto{\pgfqpoint{4.070318in}{1.933281in}}%
\pgfpathlineto{\pgfqpoint{4.071123in}{2.311653in}}%
\pgfpathlineto{\pgfqpoint{4.071725in}{2.076281in}}%
\pgfpathlineto{\pgfqpoint{4.072125in}{2.257287in}}%
\pgfpathlineto{\pgfqpoint{4.072526in}{2.221955in}}%
\pgfpathlineto{\pgfqpoint{4.072726in}{2.226210in}}%
\pgfpathlineto{\pgfqpoint{4.073126in}{2.266006in}}%
\pgfpathlineto{\pgfqpoint{4.073725in}{2.050328in}}%
\pgfpathlineto{\pgfqpoint{4.073925in}{2.311739in}}%
\pgfpathlineto{\pgfqpoint{4.074921in}{2.257659in}}%
\pgfpathlineto{\pgfqpoint{4.075120in}{2.277613in}}%
\pgfpathlineto{\pgfqpoint{4.075319in}{2.216310in}}%
\pgfpathlineto{\pgfqpoint{4.076312in}{2.040878in}}%
\pgfpathlineto{\pgfqpoint{4.075717in}{2.270620in}}%
\pgfpathlineto{\pgfqpoint{4.077105in}{2.060041in}}%
\pgfpathlineto{\pgfqpoint{4.077303in}{2.375101in}}%
\pgfpathlineto{\pgfqpoint{4.078292in}{2.218523in}}%
\pgfpathlineto{\pgfqpoint{4.078489in}{2.163587in}}%
\pgfpathlineto{\pgfqpoint{4.078884in}{2.231602in}}%
\pgfpathlineto{\pgfqpoint{4.079081in}{2.331008in}}%
\pgfpathlineto{\pgfqpoint{4.079278in}{2.071837in}}%
\pgfpathlineto{\pgfqpoint{4.079868in}{2.235780in}}%
\pgfpathlineto{\pgfqpoint{4.080065in}{2.228989in}}%
\pgfpathlineto{\pgfqpoint{4.080655in}{2.281859in}}%
\pgfpathlineto{\pgfqpoint{4.081243in}{2.103522in}}%
\pgfpathlineto{\pgfqpoint{4.081635in}{1.891000in}}%
\pgfpathlineto{\pgfqpoint{4.082418in}{2.338268in}}%
\pgfpathlineto{\pgfqpoint{4.082614in}{2.157475in}}%
\pgfpathlineto{\pgfqpoint{4.083005in}{2.342120in}}%
\pgfpathlineto{\pgfqpoint{4.083395in}{2.271209in}}%
\pgfpathlineto{\pgfqpoint{4.083590in}{2.345364in}}%
\pgfpathlineto{\pgfqpoint{4.083785in}{2.143021in}}%
\pgfpathlineto{\pgfqpoint{4.084175in}{2.244136in}}%
\pgfpathlineto{\pgfqpoint{4.085147in}{2.294555in}}%
\pgfpathlineto{\pgfqpoint{4.085342in}{2.104573in}}%
\pgfpathlineto{\pgfqpoint{4.085924in}{2.312555in}}%
\pgfpathlineto{\pgfqpoint{4.086699in}{2.306968in}}%
\pgfpathlineto{\pgfqpoint{4.087473in}{2.173839in}}%
\pgfpathlineto{\pgfqpoint{4.087666in}{2.204786in}}%
\pgfpathlineto{\pgfqpoint{4.087859in}{2.338129in}}%
\pgfpathlineto{\pgfqpoint{4.088245in}{2.193652in}}%
\pgfpathlineto{\pgfqpoint{4.088438in}{1.912802in}}%
\pgfpathlineto{\pgfqpoint{4.088823in}{2.280923in}}%
\pgfpathlineto{\pgfqpoint{4.089401in}{2.035733in}}%
\pgfpathlineto{\pgfqpoint{4.090170in}{1.964789in}}%
\pgfpathlineto{\pgfqpoint{4.090553in}{2.324714in}}%
\pgfpathlineto{\pgfqpoint{4.090937in}{2.185202in}}%
\pgfpathlineto{\pgfqpoint{4.091512in}{2.197431in}}%
\pgfpathlineto{\pgfqpoint{4.091703in}{2.343172in}}%
\pgfpathlineto{\pgfqpoint{4.092276in}{2.014599in}}%
\pgfpathlineto{\pgfqpoint{4.092468in}{2.314978in}}%
\pgfpathlineto{\pgfqpoint{4.092658in}{2.151542in}}%
\pgfpathlineto{\pgfqpoint{4.093612in}{2.205294in}}%
\pgfpathlineto{\pgfqpoint{4.093802in}{2.287414in}}%
\pgfpathlineto{\pgfqpoint{4.093993in}{2.061437in}}%
\pgfpathlineto{\pgfqpoint{4.094183in}{2.242126in}}%
\pgfpathlineto{\pgfqpoint{4.094753in}{2.309009in}}%
\pgfpathlineto{\pgfqpoint{4.095512in}{1.802977in}}%
\pgfpathlineto{\pgfqpoint{4.096081in}{2.324010in}}%
\pgfpathlineto{\pgfqpoint{4.096838in}{2.249051in}}%
\pgfpathlineto{\pgfqpoint{4.097215in}{2.019097in}}%
\pgfpathlineto{\pgfqpoint{4.097970in}{2.136891in}}%
\pgfpathlineto{\pgfqpoint{4.099476in}{2.365334in}}%
\pgfpathlineto{\pgfqpoint{4.100226in}{2.005917in}}%
\pgfpathlineto{\pgfqpoint{4.100789in}{2.098844in}}%
\pgfpathlineto{\pgfqpoint{4.102098in}{2.338913in}}%
\pgfpathlineto{\pgfqpoint{4.102471in}{2.082610in}}%
\pgfpathlineto{\pgfqpoint{4.103402in}{2.185447in}}%
\pgfpathlineto{\pgfqpoint{4.104332in}{2.326490in}}%
\pgfpathlineto{\pgfqpoint{4.104889in}{2.111367in}}%
\pgfpathlineto{\pgfqpoint{4.104703in}{2.331822in}}%
\pgfpathlineto{\pgfqpoint{4.105445in}{2.285440in}}%
\pgfpathlineto{\pgfqpoint{4.106555in}{2.096507in}}%
\pgfpathlineto{\pgfqpoint{4.106370in}{2.322512in}}%
\pgfpathlineto{\pgfqpoint{4.106739in}{2.229853in}}%
\pgfpathlineto{\pgfqpoint{4.106924in}{2.233513in}}%
\pgfpathlineto{\pgfqpoint{4.107477in}{2.027190in}}%
\pgfpathlineto{\pgfqpoint{4.107662in}{2.356272in}}%
\pgfpathlineto{\pgfqpoint{4.108582in}{2.316044in}}%
\pgfpathlineto{\pgfqpoint{4.109867in}{2.119459in}}%
\pgfpathlineto{\pgfqpoint{4.110234in}{2.291770in}}%
\pgfpathlineto{\pgfqpoint{4.110966in}{2.226268in}}%
\pgfpathlineto{\pgfqpoint{4.111148in}{2.185372in}}%
\pgfpathlineto{\pgfqpoint{4.111696in}{2.309825in}}%
\pgfpathlineto{\pgfqpoint{4.111879in}{2.306805in}}%
\pgfpathlineto{\pgfqpoint{4.112244in}{2.330359in}}%
\pgfpathlineto{\pgfqpoint{4.112972in}{2.122939in}}%
\pgfpathlineto{\pgfqpoint{4.113336in}{2.336220in}}%
\pgfpathlineto{\pgfqpoint{4.113881in}{2.095841in}}%
\pgfpathlineto{\pgfqpoint{4.114244in}{2.277133in}}%
\pgfpathlineto{\pgfqpoint{4.114425in}{2.075671in}}%
\pgfpathlineto{\pgfqpoint{4.115150in}{2.378256in}}%
\pgfpathlineto{\pgfqpoint{4.115331in}{2.183355in}}%
\pgfpathlineto{\pgfqpoint{4.115512in}{2.321210in}}%
\pgfpathlineto{\pgfqpoint{4.115693in}{1.990231in}}%
\pgfpathlineto{\pgfqpoint{4.116235in}{2.258499in}}%
\pgfpathlineto{\pgfqpoint{4.116416in}{1.772768in}}%
\pgfpathlineto{\pgfqpoint{4.117317in}{2.164209in}}%
\pgfpathlineto{\pgfqpoint{4.117857in}{2.039990in}}%
\pgfpathlineto{\pgfqpoint{4.117677in}{2.202910in}}%
\pgfpathlineto{\pgfqpoint{4.118037in}{2.199555in}}%
\pgfpathlineto{\pgfqpoint{4.118576in}{2.342044in}}%
\pgfpathlineto{\pgfqpoint{4.118756in}{2.170839in}}%
\pgfpathlineto{\pgfqpoint{4.118935in}{2.175521in}}%
\pgfpathlineto{\pgfqpoint{4.119294in}{2.034196in}}%
\pgfpathlineto{\pgfqpoint{4.119473in}{2.208456in}}%
\pgfpathlineto{\pgfqpoint{4.119831in}{2.273031in}}%
\pgfpathlineto{\pgfqpoint{4.120010in}{2.141697in}}%
\pgfpathlineto{\pgfqpoint{4.120368in}{2.172553in}}%
\pgfpathlineto{\pgfqpoint{4.120726in}{2.150296in}}%
\pgfpathlineto{\pgfqpoint{4.120904in}{1.957564in}}%
\pgfpathlineto{\pgfqpoint{4.121261in}{2.277071in}}%
\pgfpathlineto{\pgfqpoint{4.121618in}{2.260096in}}%
\pgfpathlineto{\pgfqpoint{4.122153in}{2.326554in}}%
\pgfpathlineto{\pgfqpoint{4.122331in}{2.139322in}}%
\pgfpathlineto{\pgfqpoint{4.122509in}{2.262514in}}%
\pgfpathlineto{\pgfqpoint{4.123220in}{2.086430in}}%
\pgfpathlineto{\pgfqpoint{4.123397in}{2.349790in}}%
\pgfpathlineto{\pgfqpoint{4.124107in}{2.076474in}}%
\pgfpathlineto{\pgfqpoint{4.124284in}{2.155016in}}%
\pgfpathlineto{\pgfqpoint{4.124639in}{2.216859in}}%
\pgfpathlineto{\pgfqpoint{4.124816in}{2.154992in}}%
\pgfpathlineto{\pgfqpoint{4.125346in}{2.159939in}}%
\pgfpathlineto{\pgfqpoint{4.126053in}{2.282106in}}%
\pgfpathlineto{\pgfqpoint{4.126405in}{1.984096in}}%
\pgfpathlineto{\pgfqpoint{4.126934in}{2.274869in}}%
\pgfpathlineto{\pgfqpoint{4.127638in}{2.195108in}}%
\pgfpathlineto{\pgfqpoint{4.128516in}{2.197360in}}%
\pgfpathlineto{\pgfqpoint{4.128691in}{2.142042in}}%
\pgfpathlineto{\pgfqpoint{4.129392in}{2.337696in}}%
\pgfpathlineto{\pgfqpoint{4.129742in}{2.280695in}}%
\pgfpathlineto{\pgfqpoint{4.130266in}{2.093635in}}%
\pgfpathlineto{\pgfqpoint{4.130441in}{2.284427in}}%
\pgfpathlineto{\pgfqpoint{4.130965in}{2.209378in}}%
\pgfpathlineto{\pgfqpoint{4.131139in}{2.285015in}}%
\pgfpathlineto{\pgfqpoint{4.131488in}{2.136088in}}%
\pgfpathlineto{\pgfqpoint{4.132184in}{2.266822in}}%
\pgfpathlineto{\pgfqpoint{4.132358in}{2.157903in}}%
\pgfpathlineto{\pgfqpoint{4.132879in}{2.303292in}}%
\pgfpathlineto{\pgfqpoint{4.133226in}{2.185276in}}%
\pgfpathlineto{\pgfqpoint{4.134093in}{2.257245in}}%
\pgfpathlineto{\pgfqpoint{4.133573in}{1.999092in}}%
\pgfpathlineto{\pgfqpoint{4.134266in}{2.221798in}}%
\pgfpathlineto{\pgfqpoint{4.134439in}{2.025053in}}%
\pgfpathlineto{\pgfqpoint{4.134612in}{2.229811in}}%
\pgfpathlineto{\pgfqpoint{4.135303in}{2.101751in}}%
\pgfpathlineto{\pgfqpoint{4.135475in}{2.323829in}}%
\pgfpathlineto{\pgfqpoint{4.135820in}{1.893544in}}%
\pgfpathlineto{\pgfqpoint{4.136337in}{2.281331in}}%
\pgfpathlineto{\pgfqpoint{4.137026in}{2.298978in}}%
\pgfpathlineto{\pgfqpoint{4.137541in}{2.143193in}}%
\pgfpathlineto{\pgfqpoint{4.137713in}{2.342186in}}%
\pgfpathlineto{\pgfqpoint{4.138056in}{2.055585in}}%
\pgfpathlineto{\pgfqpoint{4.138742in}{2.289873in}}%
\pgfpathlineto{\pgfqpoint{4.138913in}{2.191048in}}%
\pgfpathlineto{\pgfqpoint{4.139084in}{2.310927in}}%
\pgfpathlineto{\pgfqpoint{4.139768in}{2.308624in}}%
\pgfpathlineto{\pgfqpoint{4.141473in}{2.089294in}}%
\pgfpathlineto{\pgfqpoint{4.141643in}{2.305299in}}%
\pgfpathlineto{\pgfqpoint{4.142662in}{2.225624in}}%
\pgfpathlineto{\pgfqpoint{4.143341in}{2.330840in}}%
\pgfpathlineto{\pgfqpoint{4.143171in}{2.139911in}}%
\pgfpathlineto{\pgfqpoint{4.143679in}{2.237356in}}%
\pgfpathlineto{\pgfqpoint{4.144862in}{1.991182in}}%
\pgfpathlineto{\pgfqpoint{4.144694in}{2.297458in}}%
\pgfpathlineto{\pgfqpoint{4.145031in}{1.996537in}}%
\pgfpathlineto{\pgfqpoint{4.146379in}{2.264379in}}%
\pgfpathlineto{\pgfqpoint{4.147387in}{2.064105in}}%
\pgfpathlineto{\pgfqpoint{4.147555in}{2.233747in}}%
\pgfpathlineto{\pgfqpoint{4.147722in}{2.264639in}}%
\pgfpathlineto{\pgfqpoint{4.148058in}{2.204796in}}%
\pgfpathlineto{\pgfqpoint{4.148225in}{2.081156in}}%
\pgfpathlineto{\pgfqpoint{4.148894in}{2.265893in}}%
\pgfpathlineto{\pgfqpoint{4.149062in}{2.358031in}}%
\pgfpathlineto{\pgfqpoint{4.149563in}{2.154888in}}%
\pgfpathlineto{\pgfqpoint{4.149730in}{2.265481in}}%
\pgfpathlineto{\pgfqpoint{4.150397in}{2.125179in}}%
\pgfpathlineto{\pgfqpoint{4.150063in}{2.305221in}}%
\pgfpathlineto{\pgfqpoint{4.150730in}{2.291088in}}%
\pgfpathlineto{\pgfqpoint{4.150896in}{2.287992in}}%
\pgfpathlineto{\pgfqpoint{4.151727in}{1.949503in}}%
\pgfpathlineto{\pgfqpoint{4.151893in}{2.313616in}}%
\pgfpathlineto{\pgfqpoint{4.152059in}{2.247722in}}%
\pgfpathlineto{\pgfqpoint{4.152225in}{2.270891in}}%
\pgfpathlineto{\pgfqpoint{4.152391in}{2.259169in}}%
\pgfpathlineto{\pgfqpoint{4.152723in}{2.296075in}}%
\pgfpathlineto{\pgfqpoint{4.153716in}{1.876361in}}%
\pgfpathlineto{\pgfqpoint{4.154212in}{2.303710in}}%
\pgfpathlineto{\pgfqpoint{4.154871in}{2.267923in}}%
\pgfpathlineto{\pgfqpoint{4.155695in}{1.995885in}}%
\pgfpathlineto{\pgfqpoint{4.155860in}{2.234597in}}%
\pgfpathlineto{\pgfqpoint{4.156024in}{2.350349in}}%
\pgfpathlineto{\pgfqpoint{4.156353in}{2.143669in}}%
\pgfpathlineto{\pgfqpoint{4.156845in}{2.214350in}}%
\pgfpathlineto{\pgfqpoint{4.157501in}{2.061291in}}%
\pgfpathlineto{\pgfqpoint{4.157665in}{2.251596in}}%
\pgfpathlineto{\pgfqpoint{4.157829in}{2.185141in}}%
\pgfpathlineto{\pgfqpoint{4.158320in}{2.306686in}}%
\pgfpathlineto{\pgfqpoint{4.158483in}{2.126458in}}%
\pgfpathlineto{\pgfqpoint{4.158647in}{2.181638in}}%
\pgfpathlineto{\pgfqpoint{4.158810in}{2.103759in}}%
\pgfpathlineto{\pgfqpoint{4.159463in}{2.246951in}}%
\pgfpathlineto{\pgfqpoint{4.159626in}{2.228346in}}%
\pgfpathlineto{\pgfqpoint{4.159789in}{2.235220in}}%
\pgfpathlineto{\pgfqpoint{4.160115in}{2.314802in}}%
\pgfpathlineto{\pgfqpoint{4.160765in}{2.030856in}}%
\pgfpathlineto{\pgfqpoint{4.161091in}{2.268028in}}%
\pgfpathlineto{\pgfqpoint{4.161740in}{2.080958in}}%
\pgfpathlineto{\pgfqpoint{4.161902in}{1.941786in}}%
\pgfpathlineto{\pgfqpoint{4.162226in}{2.280849in}}%
\pgfpathlineto{\pgfqpoint{4.162712in}{1.974906in}}%
\pgfpathlineto{\pgfqpoint{4.163682in}{2.340114in}}%
\pgfpathlineto{\pgfqpoint{4.163197in}{1.879476in}}%
\pgfpathlineto{\pgfqpoint{4.163844in}{2.198235in}}%
\pgfpathlineto{\pgfqpoint{4.164166in}{2.064204in}}%
\pgfpathlineto{\pgfqpoint{4.164972in}{2.276988in}}%
\pgfpathlineto{\pgfqpoint{4.165133in}{2.050156in}}%
\pgfpathlineto{\pgfqpoint{4.165615in}{2.286012in}}%
\pgfpathlineto{\pgfqpoint{4.166097in}{2.148266in}}%
\pgfpathlineto{\pgfqpoint{4.166739in}{2.300400in}}%
\pgfpathlineto{\pgfqpoint{4.166899in}{2.124076in}}%
\pgfpathlineto{\pgfqpoint{4.167060in}{2.022992in}}%
\pgfpathlineto{\pgfqpoint{4.167540in}{2.368709in}}%
\pgfpathlineto{\pgfqpoint{4.167860in}{2.099329in}}%
\pgfpathlineto{\pgfqpoint{4.168020in}{2.243565in}}%
\pgfpathlineto{\pgfqpoint{4.168978in}{2.209039in}}%
\pgfpathlineto{\pgfqpoint{4.169296in}{2.030499in}}%
\pgfpathlineto{\pgfqpoint{4.169456in}{2.358884in}}%
\pgfpathlineto{\pgfqpoint{4.170093in}{2.187862in}}%
\pgfpathlineto{\pgfqpoint{4.170252in}{2.136144in}}%
\pgfpathlineto{\pgfqpoint{4.170411in}{2.246037in}}%
\pgfpathlineto{\pgfqpoint{4.170728in}{2.145282in}}%
\pgfpathlineto{\pgfqpoint{4.170887in}{2.300413in}}%
\pgfpathlineto{\pgfqpoint{4.171363in}{2.113704in}}%
\pgfpathlineto{\pgfqpoint{4.171839in}{2.151362in}}%
\pgfpathlineto{\pgfqpoint{4.171997in}{2.164930in}}%
\pgfpathlineto{\pgfqpoint{4.172155in}{2.324663in}}%
\pgfpathlineto{\pgfqpoint{4.172788in}{2.044830in}}%
\pgfpathlineto{\pgfqpoint{4.173104in}{2.197798in}}%
\pgfpathlineto{\pgfqpoint{4.173893in}{2.127528in}}%
\pgfpathlineto{\pgfqpoint{4.174523in}{2.237513in}}%
\pgfpathlineto{\pgfqpoint{4.174208in}{2.080301in}}%
\pgfpathlineto{\pgfqpoint{4.174838in}{2.207536in}}%
\pgfpathlineto{\pgfqpoint{4.174995in}{2.008970in}}%
\pgfpathlineto{\pgfqpoint{4.175623in}{2.243652in}}%
\pgfpathlineto{\pgfqpoint{4.175780in}{2.123694in}}%
\pgfpathlineto{\pgfqpoint{4.176721in}{2.228129in}}%
\pgfpathlineto{\pgfqpoint{4.176408in}{2.005712in}}%
\pgfpathlineto{\pgfqpoint{4.176877in}{2.184802in}}%
\pgfpathlineto{\pgfqpoint{4.177034in}{2.074932in}}%
\pgfpathlineto{\pgfqpoint{4.177503in}{2.265516in}}%
\pgfpathlineto{\pgfqpoint{4.177816in}{2.119452in}}%
\pgfpathlineto{\pgfqpoint{4.177972in}{2.329913in}}%
\pgfpathlineto{\pgfqpoint{4.178440in}{2.000949in}}%
\pgfpathlineto{\pgfqpoint{4.178907in}{2.217143in}}%
\pgfpathlineto{\pgfqpoint{4.179375in}{2.028610in}}%
\pgfpathlineto{\pgfqpoint{4.179841in}{2.305677in}}%
\pgfpathlineto{\pgfqpoint{4.179997in}{2.223227in}}%
\pgfpathlineto{\pgfqpoint{4.180152in}{2.279575in}}%
\pgfpathlineto{\pgfqpoint{4.180618in}{2.187206in}}%
\pgfpathlineto{\pgfqpoint{4.180773in}{2.220592in}}%
\pgfpathlineto{\pgfqpoint{4.180928in}{1.791210in}}%
\pgfpathlineto{\pgfqpoint{4.181702in}{2.305769in}}%
\pgfpathlineto{\pgfqpoint{4.181857in}{2.146412in}}%
\pgfpathlineto{\pgfqpoint{4.182166in}{2.333097in}}%
\pgfpathlineto{\pgfqpoint{4.182630in}{2.058673in}}%
\pgfpathlineto{\pgfqpoint{4.183093in}{2.292187in}}%
\pgfpathlineto{\pgfqpoint{4.183556in}{1.684545in}}%
\pgfpathlineto{\pgfqpoint{4.183864in}{2.327980in}}%
\pgfpathlineto{\pgfqpoint{4.184786in}{2.278798in}}%
\pgfpathlineto{\pgfqpoint{4.185401in}{2.287541in}}%
\pgfpathlineto{\pgfqpoint{4.186014in}{1.753854in}}%
\pgfpathlineto{\pgfqpoint{4.186473in}{2.241170in}}%
\pgfpathlineto{\pgfqpoint{4.187238in}{2.146014in}}%
\pgfpathlineto{\pgfqpoint{4.188153in}{2.067873in}}%
\pgfpathlineto{\pgfqpoint{4.188458in}{2.258706in}}%
\pgfpathlineto{\pgfqpoint{4.189067in}{2.047015in}}%
\pgfpathlineto{\pgfqpoint{4.188915in}{2.304447in}}%
\pgfpathlineto{\pgfqpoint{4.189371in}{2.245446in}}%
\pgfpathlineto{\pgfqpoint{4.189523in}{2.340854in}}%
\pgfpathlineto{\pgfqpoint{4.189827in}{2.045867in}}%
\pgfpathlineto{\pgfqpoint{4.190130in}{2.147666in}}%
\pgfpathlineto{\pgfqpoint{4.190585in}{2.254632in}}%
\pgfpathlineto{\pgfqpoint{4.191040in}{2.003330in}}%
\pgfpathlineto{\pgfqpoint{4.191645in}{2.280839in}}%
\pgfpathlineto{\pgfqpoint{4.192249in}{2.262356in}}%
\pgfpathlineto{\pgfqpoint{4.192400in}{2.334681in}}%
\pgfpathlineto{\pgfqpoint{4.192852in}{2.061313in}}%
\pgfpathlineto{\pgfqpoint{4.193154in}{2.201354in}}%
\pgfpathlineto{\pgfqpoint{4.193455in}{2.135766in}}%
\pgfpathlineto{\pgfqpoint{4.193756in}{2.232881in}}%
\pgfpathlineto{\pgfqpoint{4.193906in}{2.211844in}}%
\pgfpathlineto{\pgfqpoint{4.194057in}{1.932834in}}%
\pgfpathlineto{\pgfqpoint{4.194207in}{2.279908in}}%
\pgfpathlineto{\pgfqpoint{4.194958in}{2.196944in}}%
\pgfpathlineto{\pgfqpoint{4.195108in}{2.265982in}}%
\pgfpathlineto{\pgfqpoint{4.195707in}{2.097970in}}%
\pgfpathlineto{\pgfqpoint{4.195857in}{2.090214in}}%
\pgfpathlineto{\pgfqpoint{4.196006in}{2.110067in}}%
\pgfpathlineto{\pgfqpoint{4.196156in}{2.141194in}}%
\pgfpathlineto{\pgfqpoint{4.196306in}{1.850103in}}%
\pgfpathlineto{\pgfqpoint{4.196754in}{2.267091in}}%
\pgfpathlineto{\pgfqpoint{4.197202in}{2.166391in}}%
\pgfpathlineto{\pgfqpoint{4.197351in}{2.082833in}}%
\pgfpathlineto{\pgfqpoint{4.198096in}{2.216972in}}%
\pgfpathlineto{\pgfqpoint{4.198245in}{2.131270in}}%
\pgfpathlineto{\pgfqpoint{4.198394in}{2.133063in}}%
\pgfpathlineto{\pgfqpoint{4.198840in}{2.220652in}}%
\pgfpathlineto{\pgfqpoint{4.199286in}{2.154764in}}%
\pgfpathlineto{\pgfqpoint{4.199434in}{1.953211in}}%
\pgfpathlineto{\pgfqpoint{4.200176in}{2.316070in}}%
\pgfpathlineto{\pgfqpoint{4.200324in}{2.239204in}}%
\pgfpathlineto{\pgfqpoint{4.201212in}{2.078773in}}%
\pgfpathlineto{\pgfqpoint{4.201359in}{2.255624in}}%
\pgfpathlineto{\pgfqpoint{4.201507in}{2.308019in}}%
\pgfpathlineto{\pgfqpoint{4.201655in}{1.908859in}}%
\pgfpathlineto{\pgfqpoint{4.202540in}{2.234015in}}%
\pgfpathlineto{\pgfqpoint{4.203129in}{2.236625in}}%
\pgfpathlineto{\pgfqpoint{4.203423in}{1.938157in}}%
\pgfpathlineto{\pgfqpoint{4.204598in}{2.324819in}}%
\pgfpathlineto{\pgfqpoint{4.206061in}{1.894482in}}%
\pgfpathlineto{\pgfqpoint{4.207083in}{2.294250in}}%
\pgfpathlineto{\pgfqpoint{4.207229in}{2.252241in}}%
\pgfpathlineto{\pgfqpoint{4.208248in}{2.070439in}}%
\pgfpathlineto{\pgfqpoint{4.209119in}{2.264560in}}%
\pgfpathlineto{\pgfqpoint{4.208974in}{2.067503in}}%
\pgfpathlineto{\pgfqpoint{4.209264in}{2.263662in}}%
\pgfpathlineto{\pgfqpoint{4.209409in}{2.087047in}}%
\pgfpathlineto{\pgfqpoint{4.209989in}{2.299711in}}%
\pgfpathlineto{\pgfqpoint{4.210423in}{2.174712in}}%
\pgfpathlineto{\pgfqpoint{4.210568in}{2.150043in}}%
\pgfpathlineto{\pgfqpoint{4.210712in}{1.884138in}}%
\pgfpathlineto{\pgfqpoint{4.211145in}{2.251024in}}%
\pgfpathlineto{\pgfqpoint{4.211578in}{2.200514in}}%
\pgfpathlineto{\pgfqpoint{4.212155in}{2.040508in}}%
\pgfpathlineto{\pgfqpoint{4.212443in}{2.151140in}}%
\pgfpathlineto{\pgfqpoint{4.212587in}{2.248780in}}%
\pgfpathlineto{\pgfqpoint{4.213018in}{2.080039in}}%
\pgfpathlineto{\pgfqpoint{4.213449in}{2.154450in}}%
\pgfpathlineto{\pgfqpoint{4.213593in}{2.019392in}}%
\pgfpathlineto{\pgfqpoint{4.214167in}{2.254258in}}%
\pgfpathlineto{\pgfqpoint{4.214454in}{2.242928in}}%
\pgfpathlineto{\pgfqpoint{4.215026in}{2.036050in}}%
\pgfpathlineto{\pgfqpoint{4.215455in}{2.291915in}}%
\pgfpathlineto{\pgfqpoint{4.215598in}{2.112111in}}%
\pgfpathlineto{\pgfqpoint{4.215884in}{2.313960in}}%
\pgfpathlineto{\pgfqpoint{4.216455in}{2.098545in}}%
\pgfpathlineto{\pgfqpoint{4.216597in}{2.231772in}}%
\pgfpathlineto{\pgfqpoint{4.216882in}{2.302022in}}%
\pgfpathlineto{\pgfqpoint{4.217878in}{2.014649in}}%
\pgfpathlineto{\pgfqpoint{4.218020in}{1.983259in}}%
\pgfpathlineto{\pgfqpoint{4.218162in}{2.126614in}}%
\pgfpathlineto{\pgfqpoint{4.219155in}{2.124972in}}%
\pgfpathlineto{\pgfqpoint{4.219297in}{2.268449in}}%
\pgfpathlineto{\pgfqpoint{4.220005in}{2.054459in}}%
\pgfpathlineto{\pgfqpoint{4.219863in}{2.317413in}}%
\pgfpathlineto{\pgfqpoint{4.220570in}{2.156089in}}%
\pgfpathlineto{\pgfqpoint{4.221557in}{2.131260in}}%
\pgfpathlineto{\pgfqpoint{4.221698in}{2.218094in}}%
\pgfpathlineto{\pgfqpoint{4.222402in}{1.884202in}}%
\pgfpathlineto{\pgfqpoint{4.222261in}{2.245613in}}%
\pgfpathlineto{\pgfqpoint{4.222542in}{2.189487in}}%
\pgfpathlineto{\pgfqpoint{4.222683in}{2.314677in}}%
\pgfpathlineto{\pgfqpoint{4.223385in}{2.091402in}}%
\pgfpathlineto{\pgfqpoint{4.223525in}{2.110978in}}%
\pgfpathlineto{\pgfqpoint{4.223805in}{2.048966in}}%
\pgfpathlineto{\pgfqpoint{4.224226in}{2.014404in}}%
\pgfpathlineto{\pgfqpoint{4.224925in}{2.294252in}}%
\pgfpathlineto{\pgfqpoint{4.225065in}{1.946991in}}%
\pgfpathlineto{\pgfqpoint{4.226042in}{2.167165in}}%
\pgfpathlineto{\pgfqpoint{4.226181in}{2.016806in}}%
\pgfpathlineto{\pgfqpoint{4.226599in}{2.276445in}}%
\pgfpathlineto{\pgfqpoint{4.227156in}{2.110607in}}%
\pgfpathlineto{\pgfqpoint{4.227434in}{2.229295in}}%
\pgfpathlineto{\pgfqpoint{4.227712in}{1.983937in}}%
\pgfpathlineto{\pgfqpoint{4.228128in}{2.186228in}}%
\pgfpathlineto{\pgfqpoint{4.228959in}{2.023303in}}%
\pgfpathlineto{\pgfqpoint{4.228544in}{2.200987in}}%
\pgfpathlineto{\pgfqpoint{4.229236in}{2.128446in}}%
\pgfpathlineto{\pgfqpoint{4.229789in}{2.310305in}}%
\pgfpathlineto{\pgfqpoint{4.230204in}{2.117576in}}%
\pgfpathlineto{\pgfqpoint{4.230480in}{2.275436in}}%
\pgfpathlineto{\pgfqpoint{4.230618in}{2.056610in}}%
\pgfpathlineto{\pgfqpoint{4.230893in}{2.278882in}}%
\pgfpathlineto{\pgfqpoint{4.231582in}{2.142962in}}%
\pgfpathlineto{\pgfqpoint{4.231720in}{2.251131in}}%
\pgfpathlineto{\pgfqpoint{4.232681in}{2.220116in}}%
\pgfpathlineto{\pgfqpoint{4.232956in}{2.335265in}}%
\pgfpathlineto{\pgfqpoint{4.233641in}{2.180089in}}%
\pgfpathlineto{\pgfqpoint{4.234052in}{2.286673in}}%
\pgfpathlineto{\pgfqpoint{4.234188in}{2.107248in}}%
\pgfpathlineto{\pgfqpoint{4.234598in}{2.185955in}}%
\pgfpathlineto{\pgfqpoint{4.235417in}{2.022379in}}%
\pgfpathlineto{\pgfqpoint{4.236371in}{2.265782in}}%
\pgfpathlineto{\pgfqpoint{4.236643in}{2.216925in}}%
\pgfpathlineto{\pgfqpoint{4.237458in}{2.047818in}}%
\pgfpathlineto{\pgfqpoint{4.237051in}{2.242371in}}%
\pgfpathlineto{\pgfqpoint{4.237729in}{2.162398in}}%
\pgfpathlineto{\pgfqpoint{4.238271in}{2.081989in}}%
\pgfpathlineto{\pgfqpoint{4.238678in}{2.243285in}}%
\pgfpathlineto{\pgfqpoint{4.239489in}{1.975174in}}%
\pgfpathlineto{\pgfqpoint{4.239624in}{2.304340in}}%
\pgfpathlineto{\pgfqpoint{4.239894in}{2.076352in}}%
\pgfpathlineto{\pgfqpoint{4.240837in}{2.289615in}}%
\pgfpathlineto{\pgfqpoint{4.240568in}{1.944583in}}%
\pgfpathlineto{\pgfqpoint{4.241106in}{2.211268in}}%
\pgfpathlineto{\pgfqpoint{4.241913in}{1.982550in}}%
\pgfpathlineto{\pgfqpoint{4.241375in}{2.277626in}}%
\pgfpathlineto{\pgfqpoint{4.242316in}{2.040988in}}%
\pgfpathlineto{\pgfqpoint{4.243254in}{2.317706in}}%
\pgfpathlineto{\pgfqpoint{4.242986in}{1.959864in}}%
\pgfpathlineto{\pgfqpoint{4.243521in}{2.165472in}}%
\pgfpathlineto{\pgfqpoint{4.244323in}{2.229889in}}%
\pgfpathlineto{\pgfqpoint{4.244590in}{1.990122in}}%
\pgfpathlineto{\pgfqpoint{4.245790in}{2.306523in}}%
\pgfpathlineto{\pgfqpoint{4.246056in}{2.111799in}}%
\pgfpathlineto{\pgfqpoint{4.246853in}{2.285572in}}%
\pgfpathlineto{\pgfqpoint{4.248178in}{1.939443in}}%
\pgfpathlineto{\pgfqpoint{4.249235in}{2.247137in}}%
\pgfpathlineto{\pgfqpoint{4.249367in}{2.008144in}}%
\pgfpathlineto{\pgfqpoint{4.249763in}{2.253750in}}%
\pgfpathlineto{\pgfqpoint{4.250290in}{2.214171in}}%
\pgfpathlineto{\pgfqpoint{4.250422in}{2.242145in}}%
\pgfpathlineto{\pgfqpoint{4.250553in}{2.223841in}}%
\pgfpathlineto{\pgfqpoint{4.250685in}{1.949337in}}%
\pgfpathlineto{\pgfqpoint{4.251211in}{2.259555in}}%
\pgfpathlineto{\pgfqpoint{4.251605in}{2.212369in}}%
\pgfpathlineto{\pgfqpoint{4.251736in}{2.279869in}}%
\pgfpathlineto{\pgfqpoint{4.251998in}{2.106311in}}%
\pgfpathlineto{\pgfqpoint{4.252129in}{2.159901in}}%
\pgfpathlineto{\pgfqpoint{4.252260in}{1.592548in}}%
\pgfpathlineto{\pgfqpoint{4.253046in}{2.305677in}}%
\pgfpathlineto{\pgfqpoint{4.253177in}{2.191041in}}%
\pgfpathlineto{\pgfqpoint{4.253308in}{2.202660in}}%
\pgfpathlineto{\pgfqpoint{4.253438in}{1.994907in}}%
\pgfpathlineto{\pgfqpoint{4.253569in}{2.271296in}}%
\pgfpathlineto{\pgfqpoint{4.254483in}{2.037795in}}%
\pgfpathlineto{\pgfqpoint{4.254874in}{2.030272in}}%
\pgfpathlineto{\pgfqpoint{4.255654in}{2.264913in}}%
\pgfpathlineto{\pgfqpoint{4.256434in}{2.046832in}}%
\pgfpathlineto{\pgfqpoint{4.256823in}{2.135981in}}%
\pgfpathlineto{\pgfqpoint{4.257729in}{2.222088in}}%
\pgfpathlineto{\pgfqpoint{4.257082in}{1.863463in}}%
\pgfpathlineto{\pgfqpoint{4.257988in}{2.221448in}}%
\pgfpathlineto{\pgfqpoint{4.258634in}{1.841746in}}%
\pgfpathlineto{\pgfqpoint{4.258505in}{2.227893in}}%
\pgfpathlineto{\pgfqpoint{4.259021in}{2.177850in}}%
\pgfpathlineto{\pgfqpoint{4.259666in}{2.208044in}}%
\pgfpathlineto{\pgfqpoint{4.259279in}{2.111925in}}%
\pgfpathlineto{\pgfqpoint{4.259795in}{2.134328in}}%
\pgfpathlineto{\pgfqpoint{4.260052in}{1.979818in}}%
\pgfpathlineto{\pgfqpoint{4.260181in}{1.521463in}}%
\pgfpathlineto{\pgfqpoint{4.260566in}{2.271719in}}%
\pgfpathlineto{\pgfqpoint{4.261080in}{2.151673in}}%
\pgfpathlineto{\pgfqpoint{4.261593in}{2.221476in}}%
\pgfpathlineto{\pgfqpoint{4.261337in}{2.122123in}}%
\pgfpathlineto{\pgfqpoint{4.261850in}{2.132457in}}%
\pgfpathlineto{\pgfqpoint{4.261978in}{2.130841in}}%
\pgfpathlineto{\pgfqpoint{4.262490in}{2.036871in}}%
\pgfpathlineto{\pgfqpoint{4.262873in}{2.309410in}}%
\pgfpathlineto{\pgfqpoint{4.263001in}{1.989269in}}%
\pgfpathlineto{\pgfqpoint{4.264022in}{2.152985in}}%
\pgfpathlineto{\pgfqpoint{4.264277in}{2.264859in}}%
\pgfpathlineto{\pgfqpoint{4.264659in}{2.028472in}}%
\pgfpathlineto{\pgfqpoint{4.265041in}{2.222029in}}%
\pgfpathlineto{\pgfqpoint{4.266057in}{1.980466in}}%
\pgfpathlineto{\pgfqpoint{4.265803in}{2.258417in}}%
\pgfpathlineto{\pgfqpoint{4.266184in}{2.078636in}}%
\pgfpathlineto{\pgfqpoint{4.266437in}{2.255301in}}%
\pgfpathlineto{\pgfqpoint{4.267197in}{2.046081in}}%
\pgfpathlineto{\pgfqpoint{4.267324in}{1.867438in}}%
\pgfpathlineto{\pgfqpoint{4.267577in}{2.265655in}}%
\pgfpathlineto{\pgfqpoint{4.268082in}{2.213018in}}%
\pgfpathlineto{\pgfqpoint{4.268208in}{2.260080in}}%
\pgfpathlineto{\pgfqpoint{4.268460in}{2.170878in}}%
\pgfpathlineto{\pgfqpoint{4.268965in}{2.186667in}}%
\pgfpathlineto{\pgfqpoint{4.269468in}{1.977038in}}%
\pgfpathlineto{\pgfqpoint{4.269217in}{2.213538in}}%
\pgfpathlineto{\pgfqpoint{4.270097in}{2.092847in}}%
\pgfpathlineto{\pgfqpoint{4.270223in}{2.218762in}}%
\pgfpathlineto{\pgfqpoint{4.270851in}{2.025912in}}%
\pgfpathlineto{\pgfqpoint{4.271227in}{2.119369in}}%
\pgfpathlineto{\pgfqpoint{4.271603in}{1.990559in}}%
\pgfpathlineto{\pgfqpoint{4.272353in}{2.254595in}}%
\pgfpathlineto{\pgfqpoint{4.272853in}{1.964310in}}%
\pgfpathlineto{\pgfqpoint{4.273477in}{2.036490in}}%
\pgfpathlineto{\pgfqpoint{4.274348in}{2.222050in}}%
\pgfpathlineto{\pgfqpoint{4.274100in}{2.028609in}}%
\pgfpathlineto{\pgfqpoint{4.274597in}{2.165616in}}%
\pgfpathlineto{\pgfqpoint{4.275343in}{1.983511in}}%
\pgfpathlineto{\pgfqpoint{4.275467in}{2.184489in}}%
\pgfpathlineto{\pgfqpoint{4.275591in}{2.190451in}}%
\pgfpathlineto{\pgfqpoint{4.276087in}{2.217322in}}%
\pgfpathlineto{\pgfqpoint{4.276582in}{1.985600in}}%
\pgfpathlineto{\pgfqpoint{4.276706in}{2.284238in}}%
\pgfpathlineto{\pgfqpoint{4.277324in}{1.964919in}}%
\pgfpathlineto{\pgfqpoint{4.277694in}{2.063765in}}%
\pgfpathlineto{\pgfqpoint{4.277818in}{2.249288in}}%
\pgfpathlineto{\pgfqpoint{4.278557in}{1.994894in}}%
\pgfpathlineto{\pgfqpoint{4.278804in}{2.143917in}}%
\pgfpathlineto{\pgfqpoint{4.278927in}{2.010991in}}%
\pgfpathlineto{\pgfqpoint{4.279419in}{2.246888in}}%
\pgfpathlineto{\pgfqpoint{4.279910in}{2.117338in}}%
\pgfpathlineto{\pgfqpoint{4.280401in}{2.080621in}}%
\pgfpathlineto{\pgfqpoint{4.281014in}{2.234721in}}%
\pgfpathlineto{\pgfqpoint{4.282115in}{2.047698in}}%
\pgfpathlineto{\pgfqpoint{4.282359in}{2.257985in}}%
\pgfpathlineto{\pgfqpoint{4.282481in}{2.047155in}}%
\pgfpathlineto{\pgfqpoint{4.283213in}{2.190520in}}%
\pgfpathlineto{\pgfqpoint{4.283578in}{2.057679in}}%
\pgfpathlineto{\pgfqpoint{4.284065in}{2.286110in}}%
\pgfpathlineto{\pgfqpoint{4.284308in}{2.188143in}}%
\pgfpathlineto{\pgfqpoint{4.284429in}{2.236025in}}%
\pgfpathlineto{\pgfqpoint{4.285036in}{2.040663in}}%
\pgfpathlineto{\pgfqpoint{4.285521in}{1.949281in}}%
\pgfpathlineto{\pgfqpoint{4.285642in}{2.247692in}}%
\pgfpathlineto{\pgfqpoint{4.286006in}{2.082968in}}%
\pgfpathlineto{\pgfqpoint{4.286611in}{2.241656in}}%
\pgfpathlineto{\pgfqpoint{4.286248in}{2.030427in}}%
\pgfpathlineto{\pgfqpoint{4.287094in}{2.161053in}}%
\pgfpathlineto{\pgfqpoint{4.287456in}{2.284099in}}%
\pgfpathlineto{\pgfqpoint{4.287938in}{1.740257in}}%
\pgfpathlineto{\pgfqpoint{4.288058in}{2.273060in}}%
\pgfpathlineto{\pgfqpoint{4.289021in}{2.149727in}}%
\pgfpathlineto{\pgfqpoint{4.289981in}{1.779342in}}%
\pgfpathlineto{\pgfqpoint{4.289381in}{2.273756in}}%
\pgfpathlineto{\pgfqpoint{4.290101in}{2.147921in}}%
\pgfpathlineto{\pgfqpoint{4.290461in}{2.069811in}}%
\pgfpathlineto{\pgfqpoint{4.290820in}{2.249460in}}%
\pgfpathlineto{\pgfqpoint{4.291059in}{2.219262in}}%
\pgfpathlineto{\pgfqpoint{4.291179in}{2.276595in}}%
\pgfpathlineto{\pgfqpoint{4.291537in}{2.129520in}}%
\pgfpathlineto{\pgfqpoint{4.291776in}{2.214000in}}%
\pgfpathlineto{\pgfqpoint{4.292373in}{2.019931in}}%
\pgfpathlineto{\pgfqpoint{4.292254in}{2.267747in}}%
\pgfpathlineto{\pgfqpoint{4.292969in}{2.062598in}}%
\pgfpathlineto{\pgfqpoint{4.293682in}{2.291222in}}%
\pgfpathlineto{\pgfqpoint{4.293920in}{1.886028in}}%
\pgfpathlineto{\pgfqpoint{4.294039in}{2.244575in}}%
\pgfpathlineto{\pgfqpoint{4.294869in}{1.932149in}}%
\pgfpathlineto{\pgfqpoint{4.295106in}{2.301611in}}%
\pgfpathlineto{\pgfqpoint{4.295462in}{2.097317in}}%
\pgfpathlineto{\pgfqpoint{4.296171in}{2.178911in}}%
\pgfpathlineto{\pgfqpoint{4.296289in}{2.239700in}}%
\pgfpathlineto{\pgfqpoint{4.296644in}{1.853150in}}%
\pgfpathlineto{\pgfqpoint{4.297116in}{2.153581in}}%
\pgfpathlineto{\pgfqpoint{4.297351in}{2.100194in}}%
\pgfpathlineto{\pgfqpoint{4.297823in}{2.221868in}}%
\pgfpathlineto{\pgfqpoint{4.297940in}{2.125254in}}%
\pgfpathlineto{\pgfqpoint{4.298528in}{2.295188in}}%
\pgfpathlineto{\pgfqpoint{4.298176in}{2.084969in}}%
\pgfpathlineto{\pgfqpoint{4.299115in}{2.275328in}}%
\pgfpathlineto{\pgfqpoint{4.300053in}{1.977954in}}%
\pgfpathlineto{\pgfqpoint{4.300170in}{2.237900in}}%
\pgfpathlineto{\pgfqpoint{4.300287in}{2.238294in}}%
\pgfpathlineto{\pgfqpoint{4.300404in}{2.056142in}}%
\pgfpathlineto{\pgfqpoint{4.301339in}{2.174329in}}%
\pgfpathlineto{\pgfqpoint{4.301456in}{2.169555in}}%
\pgfpathlineto{\pgfqpoint{4.302039in}{1.840154in}}%
\pgfpathlineto{\pgfqpoint{4.302622in}{2.265397in}}%
\pgfpathlineto{\pgfqpoint{4.303900in}{1.958552in}}%
\pgfpathlineto{\pgfqpoint{4.304016in}{2.205460in}}%
\pgfpathlineto{\pgfqpoint{4.305059in}{2.119184in}}%
\pgfpathlineto{\pgfqpoint{4.305406in}{2.236205in}}%
\pgfpathlineto{\pgfqpoint{4.305521in}{2.047979in}}%
\pgfpathlineto{\pgfqpoint{4.306215in}{2.210508in}}%
\pgfpathlineto{\pgfqpoint{4.306330in}{2.200863in}}%
\pgfpathlineto{\pgfqpoint{4.307022in}{1.894456in}}%
\pgfpathlineto{\pgfqpoint{4.307367in}{2.249363in}}%
\pgfpathlineto{\pgfqpoint{4.307482in}{1.993467in}}%
\pgfpathlineto{\pgfqpoint{4.307597in}{1.962656in}}%
\pgfpathlineto{\pgfqpoint{4.307942in}{2.163193in}}%
\pgfpathlineto{\pgfqpoint{4.308402in}{2.065712in}}%
\pgfpathlineto{\pgfqpoint{4.308287in}{2.217654in}}%
\pgfpathlineto{\pgfqpoint{4.309090in}{2.081618in}}%
\pgfpathlineto{\pgfqpoint{4.309205in}{2.232487in}}%
\pgfpathlineto{\pgfqpoint{4.309434in}{1.973992in}}%
\pgfpathlineto{\pgfqpoint{4.310235in}{2.165382in}}%
\pgfpathlineto{\pgfqpoint{4.310349in}{1.940532in}}%
\pgfpathlineto{\pgfqpoint{4.311034in}{2.295630in}}%
\pgfpathlineto{\pgfqpoint{4.311262in}{2.213974in}}%
\pgfpathlineto{\pgfqpoint{4.312174in}{2.047091in}}%
\pgfpathlineto{\pgfqpoint{4.311946in}{2.218768in}}%
\pgfpathlineto{\pgfqpoint{4.312401in}{2.098285in}}%
\pgfpathlineto{\pgfqpoint{4.312856in}{2.256112in}}%
\pgfpathlineto{\pgfqpoint{4.313310in}{2.056621in}}%
\pgfpathlineto{\pgfqpoint{4.313537in}{2.205033in}}%
\pgfpathlineto{\pgfqpoint{4.314444in}{2.208879in}}%
\pgfpathlineto{\pgfqpoint{4.314783in}{1.915041in}}%
\pgfpathlineto{\pgfqpoint{4.315913in}{2.179976in}}%
\pgfpathlineto{\pgfqpoint{4.316702in}{2.274777in}}%
\pgfpathlineto{\pgfqpoint{4.317039in}{2.030769in}}%
\pgfpathlineto{\pgfqpoint{4.317152in}{2.214579in}}%
\pgfpathlineto{\pgfqpoint{4.317264in}{2.010101in}}%
\pgfpathlineto{\pgfqpoint{4.318163in}{2.127168in}}%
\pgfpathlineto{\pgfqpoint{4.318275in}{2.232781in}}%
\pgfpathlineto{\pgfqpoint{4.318835in}{1.957194in}}%
\pgfpathlineto{\pgfqpoint{4.319171in}{2.177756in}}%
\pgfpathlineto{\pgfqpoint{4.319954in}{2.095795in}}%
\pgfpathlineto{\pgfqpoint{4.319619in}{2.229713in}}%
\pgfpathlineto{\pgfqpoint{4.320178in}{2.110093in}}%
\pgfpathlineto{\pgfqpoint{4.320401in}{2.241674in}}%
\pgfpathlineto{\pgfqpoint{4.320959in}{2.085442in}}%
\pgfpathlineto{\pgfqpoint{4.321182in}{2.225568in}}%
\pgfpathlineto{\pgfqpoint{4.321404in}{1.872150in}}%
\pgfpathlineto{\pgfqpoint{4.321849in}{2.294840in}}%
\pgfpathlineto{\pgfqpoint{4.322405in}{2.022406in}}%
\pgfpathlineto{\pgfqpoint{4.322738in}{2.261933in}}%
\pgfpathlineto{\pgfqpoint{4.323626in}{2.168435in}}%
\pgfpathlineto{\pgfqpoint{4.323958in}{2.191615in}}%
\pgfpathlineto{\pgfqpoint{4.323847in}{2.124387in}}%
\pgfpathlineto{\pgfqpoint{4.324068in}{2.131630in}}%
\pgfpathlineto{\pgfqpoint{4.324400in}{1.923761in}}%
\pgfpathlineto{\pgfqpoint{4.324732in}{2.206144in}}%
\pgfpathlineto{\pgfqpoint{4.324842in}{2.131548in}}%
\pgfpathlineto{\pgfqpoint{4.324953in}{2.263125in}}%
\pgfpathlineto{\pgfqpoint{4.325725in}{2.013001in}}%
\pgfpathlineto{\pgfqpoint{4.325835in}{2.070882in}}%
\pgfpathlineto{\pgfqpoint{4.326056in}{2.214647in}}%
\pgfpathlineto{\pgfqpoint{4.326606in}{2.196284in}}%
\pgfpathlineto{\pgfqpoint{4.327265in}{1.854269in}}%
\pgfpathlineto{\pgfqpoint{4.327155in}{2.262579in}}%
\pgfpathlineto{\pgfqpoint{4.327704in}{2.105547in}}%
\pgfpathlineto{\pgfqpoint{4.328143in}{2.201409in}}%
\pgfpathlineto{\pgfqpoint{4.328581in}{2.096110in}}%
\pgfpathlineto{\pgfqpoint{4.328691in}{2.136876in}}%
\pgfpathlineto{\pgfqpoint{4.328909in}{2.266404in}}%
\pgfpathlineto{\pgfqpoint{4.329784in}{1.894405in}}%
\pgfpathlineto{\pgfqpoint{4.330220in}{2.227470in}}%
\pgfpathlineto{\pgfqpoint{4.330983in}{2.185706in}}%
\pgfpathlineto{\pgfqpoint{4.331744in}{1.985316in}}%
\pgfpathlineto{\pgfqpoint{4.331418in}{2.292369in}}%
\pgfpathlineto{\pgfqpoint{4.332070in}{2.197442in}}%
\pgfpathlineto{\pgfqpoint{4.332287in}{2.236741in}}%
\pgfpathlineto{\pgfqpoint{4.333155in}{1.922332in}}%
\pgfpathlineto{\pgfqpoint{4.333371in}{2.239805in}}%
\pgfpathlineto{\pgfqpoint{4.334237in}{2.108263in}}%
\pgfpathlineto{\pgfqpoint{4.334453in}{2.217105in}}%
\pgfpathlineto{\pgfqpoint{4.335316in}{1.970342in}}%
\pgfpathlineto{\pgfqpoint{4.335962in}{1.874587in}}%
\pgfpathlineto{\pgfqpoint{4.336607in}{2.247508in}}%
\pgfpathlineto{\pgfqpoint{4.337143in}{1.987514in}}%
\pgfpathlineto{\pgfqpoint{4.337680in}{2.054953in}}%
\pgfpathlineto{\pgfqpoint{4.338001in}{2.227450in}}%
\pgfpathlineto{\pgfqpoint{4.338857in}{2.159310in}}%
\pgfpathlineto{\pgfqpoint{4.339604in}{2.271475in}}%
\pgfpathlineto{\pgfqpoint{4.340031in}{1.991441in}}%
\pgfpathlineto{\pgfqpoint{4.340457in}{2.250519in}}%
\pgfpathlineto{\pgfqpoint{4.341201in}{2.058773in}}%
\pgfpathlineto{\pgfqpoint{4.341945in}{1.980657in}}%
\pgfpathlineto{\pgfqpoint{4.341520in}{2.163720in}}%
\pgfpathlineto{\pgfqpoint{4.342051in}{2.060323in}}%
\pgfpathlineto{\pgfqpoint{4.342898in}{2.223110in}}%
\pgfpathlineto{\pgfqpoint{4.342263in}{1.896645in}}%
\pgfpathlineto{\pgfqpoint{4.343004in}{2.123714in}}%
\pgfpathlineto{\pgfqpoint{4.343110in}{1.949579in}}%
\pgfpathlineto{\pgfqpoint{4.343322in}{2.205833in}}%
\pgfpathlineto{\pgfqpoint{4.344061in}{2.126457in}}%
\pgfpathlineto{\pgfqpoint{4.344799in}{2.254352in}}%
\pgfpathlineto{\pgfqpoint{4.344589in}{2.066451in}}%
\pgfpathlineto{\pgfqpoint{4.345010in}{2.142697in}}%
\pgfpathlineto{\pgfqpoint{4.345221in}{1.929784in}}%
\pgfpathlineto{\pgfqpoint{4.345326in}{2.280586in}}%
\pgfpathlineto{\pgfqpoint{4.346062in}{2.115481in}}%
\pgfpathlineto{\pgfqpoint{4.346167in}{2.154920in}}%
\pgfpathlineto{\pgfqpoint{4.346587in}{2.142121in}}%
\pgfpathlineto{\pgfqpoint{4.346692in}{1.855739in}}%
\pgfpathlineto{\pgfqpoint{4.346902in}{2.251562in}}%
\pgfpathlineto{\pgfqpoint{4.347740in}{1.987565in}}%
\pgfpathlineto{\pgfqpoint{4.347845in}{1.988367in}}%
\pgfpathlineto{\pgfqpoint{4.348681in}{2.232432in}}%
\pgfpathlineto{\pgfqpoint{4.348472in}{1.935525in}}%
\pgfpathlineto{\pgfqpoint{4.348994in}{2.181972in}}%
\pgfpathlineto{\pgfqpoint{4.349724in}{2.233460in}}%
\pgfpathlineto{\pgfqpoint{4.350036in}{1.783465in}}%
\pgfpathlineto{\pgfqpoint{4.350140in}{2.205582in}}%
\pgfpathlineto{\pgfqpoint{4.351179in}{2.164146in}}%
\pgfpathlineto{\pgfqpoint{4.351283in}{2.235653in}}%
\pgfpathlineto{\pgfqpoint{4.351387in}{1.901314in}}%
\pgfpathlineto{\pgfqpoint{4.352009in}{2.155677in}}%
\pgfpathlineto{\pgfqpoint{4.352423in}{1.920257in}}%
\pgfpathlineto{\pgfqpoint{4.352216in}{2.232971in}}%
\pgfpathlineto{\pgfqpoint{4.353147in}{2.090624in}}%
\pgfpathlineto{\pgfqpoint{4.353457in}{2.149571in}}%
\pgfpathlineto{\pgfqpoint{4.353560in}{2.047404in}}%
\pgfpathlineto{\pgfqpoint{4.353767in}{2.048445in}}%
\pgfpathlineto{\pgfqpoint{4.353870in}{1.909857in}}%
\pgfpathlineto{\pgfqpoint{4.354076in}{2.128913in}}%
\pgfpathlineto{\pgfqpoint{4.354694in}{2.110361in}}%
\pgfpathlineto{\pgfqpoint{4.354797in}{2.224202in}}%
\pgfpathlineto{\pgfqpoint{4.355311in}{1.793931in}}%
\pgfpathlineto{\pgfqpoint{4.355722in}{2.064244in}}%
\pgfpathlineto{\pgfqpoint{4.355928in}{2.115976in}}%
\pgfpathlineto{\pgfqpoint{4.356851in}{2.256277in}}%
\pgfpathlineto{\pgfqpoint{4.356646in}{1.969160in}}%
\pgfpathlineto{\pgfqpoint{4.356953in}{2.155131in}}%
\pgfpathlineto{\pgfqpoint{4.357158in}{1.950433in}}%
\pgfpathlineto{\pgfqpoint{4.357363in}{2.232909in}}%
\pgfpathlineto{\pgfqpoint{4.357874in}{2.168540in}}%
\pgfpathlineto{\pgfqpoint{4.358792in}{2.251995in}}%
\pgfpathlineto{\pgfqpoint{4.358282in}{2.003600in}}%
\pgfpathlineto{\pgfqpoint{4.358894in}{2.251063in}}%
\pgfpathlineto{\pgfqpoint{4.359607in}{1.852341in}}%
\pgfpathlineto{\pgfqpoint{4.360014in}{2.125532in}}%
\pgfpathlineto{\pgfqpoint{4.360522in}{2.241866in}}%
\pgfpathlineto{\pgfqpoint{4.360217in}{2.016030in}}%
\pgfpathlineto{\pgfqpoint{4.361030in}{2.196090in}}%
\pgfpathlineto{\pgfqpoint{4.361739in}{2.206993in}}%
\pgfpathlineto{\pgfqpoint{4.362144in}{1.914582in}}%
\pgfpathlineto{\pgfqpoint{4.363255in}{2.262378in}}%
\pgfpathlineto{\pgfqpoint{4.363860in}{1.720892in}}%
\pgfpathlineto{\pgfqpoint{4.364263in}{2.272267in}}%
\pgfpathlineto{\pgfqpoint{4.364464in}{1.781394in}}%
\pgfpathlineto{\pgfqpoint{4.365469in}{2.233881in}}%
\pgfpathlineto{\pgfqpoint{4.365569in}{2.039953in}}%
\pgfpathlineto{\pgfqpoint{4.365670in}{2.028325in}}%
\pgfpathlineto{\pgfqpoint{4.365971in}{2.112115in}}%
\pgfpathlineto{\pgfqpoint{4.366472in}{2.197174in}}%
\pgfpathlineto{\pgfqpoint{4.366572in}{2.028251in}}%
\pgfpathlineto{\pgfqpoint{4.366672in}{2.143036in}}%
\pgfpathlineto{\pgfqpoint{4.366772in}{1.715915in}}%
\pgfpathlineto{\pgfqpoint{4.367572in}{2.215851in}}%
\pgfpathlineto{\pgfqpoint{4.367771in}{2.132166in}}%
\pgfpathlineto{\pgfqpoint{4.368170in}{2.039543in}}%
\pgfpathlineto{\pgfqpoint{4.368270in}{2.286221in}}%
\pgfpathlineto{\pgfqpoint{4.368370in}{1.827245in}}%
\pgfpathlineto{\pgfqpoint{4.369366in}{2.123993in}}%
\pgfpathlineto{\pgfqpoint{4.369465in}{2.123716in}}%
\pgfpathlineto{\pgfqpoint{4.369664in}{2.086716in}}%
\pgfpathlineto{\pgfqpoint{4.369862in}{2.213377in}}%
\pgfpathlineto{\pgfqpoint{4.370260in}{1.946643in}}%
\pgfpathlineto{\pgfqpoint{4.370954in}{2.143156in}}%
\pgfpathlineto{\pgfqpoint{4.371548in}{2.168312in}}%
\pgfpathlineto{\pgfqpoint{4.371350in}{2.049300in}}%
\pgfpathlineto{\pgfqpoint{4.372042in}{2.153500in}}%
\pgfpathlineto{\pgfqpoint{4.372339in}{2.035622in}}%
\pgfpathlineto{\pgfqpoint{4.372437in}{2.237131in}}%
\pgfpathlineto{\pgfqpoint{4.373128in}{2.044649in}}%
\pgfpathlineto{\pgfqpoint{4.373522in}{2.249250in}}%
\pgfpathlineto{\pgfqpoint{4.374014in}{1.888730in}}%
\pgfpathlineto{\pgfqpoint{4.374211in}{2.011947in}}%
\pgfpathlineto{\pgfqpoint{4.374506in}{2.201135in}}%
\pgfpathlineto{\pgfqpoint{4.375193in}{1.929287in}}%
\pgfpathlineto{\pgfqpoint{4.375389in}{2.178176in}}%
\pgfpathlineto{\pgfqpoint{4.375977in}{1.875527in}}%
\pgfpathlineto{\pgfqpoint{4.375585in}{2.213235in}}%
\pgfpathlineto{\pgfqpoint{4.376564in}{2.104950in}}%
\pgfpathlineto{\pgfqpoint{4.376662in}{2.218459in}}%
\pgfpathlineto{\pgfqpoint{4.376760in}{2.002215in}}%
\pgfpathlineto{\pgfqpoint{4.377541in}{2.204217in}}%
\pgfpathlineto{\pgfqpoint{4.378223in}{1.968260in}}%
\pgfpathlineto{\pgfqpoint{4.378515in}{2.236443in}}%
\pgfpathlineto{\pgfqpoint{4.378613in}{2.066344in}}%
\pgfpathlineto{\pgfqpoint{4.378904in}{2.241671in}}%
\pgfpathlineto{\pgfqpoint{4.379196in}{1.990129in}}%
\pgfpathlineto{\pgfqpoint{4.379682in}{2.163177in}}%
\pgfpathlineto{\pgfqpoint{4.380070in}{1.798853in}}%
\pgfpathlineto{\pgfqpoint{4.380748in}{2.203937in}}%
\pgfpathlineto{\pgfqpoint{4.380845in}{2.053999in}}%
\pgfpathlineto{\pgfqpoint{4.381522in}{2.190223in}}%
\pgfpathlineto{\pgfqpoint{4.381425in}{2.022784in}}%
\pgfpathlineto{\pgfqpoint{4.381812in}{2.152589in}}%
\pgfpathlineto{\pgfqpoint{4.381908in}{1.905565in}}%
\pgfpathlineto{\pgfqpoint{4.382680in}{2.217477in}}%
\pgfpathlineto{\pgfqpoint{4.382873in}{2.040484in}}%
\pgfpathlineto{\pgfqpoint{4.383931in}{2.272640in}}%
\pgfpathlineto{\pgfqpoint{4.384795in}{1.863804in}}%
\pgfpathlineto{\pgfqpoint{4.385083in}{1.915725in}}%
\pgfpathlineto{\pgfqpoint{4.386136in}{2.222934in}}%
\pgfpathlineto{\pgfqpoint{4.386327in}{2.121423in}}%
\pgfpathlineto{\pgfqpoint{4.386900in}{2.252129in}}%
\pgfpathlineto{\pgfqpoint{4.386995in}{1.890246in}}%
\pgfpathlineto{\pgfqpoint{4.387281in}{2.121696in}}%
\pgfpathlineto{\pgfqpoint{4.387853in}{1.900249in}}%
\pgfpathlineto{\pgfqpoint{4.387472in}{2.175155in}}%
\pgfpathlineto{\pgfqpoint{4.388424in}{2.021201in}}%
\pgfpathlineto{\pgfqpoint{4.388709in}{2.203763in}}%
\pgfpathlineto{\pgfqpoint{4.389279in}{2.049569in}}%
\pgfpathlineto{\pgfqpoint{4.389374in}{1.923375in}}%
\pgfpathlineto{\pgfqpoint{4.389468in}{2.212652in}}%
\pgfpathlineto{\pgfqpoint{4.390321in}{1.928359in}}%
\pgfpathlineto{\pgfqpoint{4.390983in}{2.228524in}}%
\pgfpathlineto{\pgfqpoint{4.391361in}{2.085632in}}%
\pgfpathlineto{\pgfqpoint{4.391456in}{1.880471in}}%
\pgfpathlineto{\pgfqpoint{4.391550in}{2.205642in}}%
\pgfpathlineto{\pgfqpoint{4.392399in}{2.094252in}}%
\pgfpathlineto{\pgfqpoint{4.393246in}{2.243458in}}%
\pgfpathlineto{\pgfqpoint{4.393152in}{1.993094in}}%
\pgfpathlineto{\pgfqpoint{4.393434in}{2.128921in}}%
\pgfpathlineto{\pgfqpoint{4.394185in}{1.960447in}}%
\pgfpathlineto{\pgfqpoint{4.393809in}{2.134060in}}%
\pgfpathlineto{\pgfqpoint{4.394466in}{2.044671in}}%
\pgfpathlineto{\pgfqpoint{4.395309in}{1.900216in}}%
\pgfpathlineto{\pgfqpoint{4.395589in}{2.233250in}}%
\pgfpathlineto{\pgfqpoint{4.395963in}{2.010314in}}%
\pgfpathlineto{\pgfqpoint{4.396710in}{2.130332in}}%
\pgfpathlineto{\pgfqpoint{4.396803in}{2.131389in}}%
\pgfpathlineto{\pgfqpoint{4.397548in}{2.255027in}}%
\pgfpathlineto{\pgfqpoint{4.397920in}{1.819144in}}%
\pgfpathlineto{\pgfqpoint{4.398199in}{2.265383in}}%
\pgfpathlineto{\pgfqpoint{4.399127in}{2.198263in}}%
\pgfpathlineto{\pgfqpoint{4.399776in}{1.869086in}}%
\pgfpathlineto{\pgfqpoint{4.400053in}{2.208935in}}%
\pgfpathlineto{\pgfqpoint{4.400238in}{2.031343in}}%
\pgfpathlineto{\pgfqpoint{4.400978in}{2.195865in}}%
\pgfpathlineto{\pgfqpoint{4.400793in}{1.990217in}}%
\pgfpathlineto{\pgfqpoint{4.401347in}{2.147684in}}%
\pgfpathlineto{\pgfqpoint{4.401623in}{1.943220in}}%
\pgfpathlineto{\pgfqpoint{4.401808in}{2.219165in}}%
\pgfpathlineto{\pgfqpoint{4.402452in}{2.010027in}}%
\pgfpathlineto{\pgfqpoint{4.402912in}{2.243537in}}%
\pgfpathlineto{\pgfqpoint{4.402636in}{1.862707in}}%
\pgfpathlineto{\pgfqpoint{4.403554in}{2.140220in}}%
\pgfpathlineto{\pgfqpoint{4.404471in}{1.953749in}}%
\pgfpathlineto{\pgfqpoint{4.403738in}{2.188825in}}%
\pgfpathlineto{\pgfqpoint{4.404563in}{2.080784in}}%
\pgfpathlineto{\pgfqpoint{4.405111in}{2.267087in}}%
\pgfpathlineto{\pgfqpoint{4.404746in}{1.899095in}}%
\pgfpathlineto{\pgfqpoint{4.405660in}{2.086386in}}%
\pgfpathlineto{\pgfqpoint{4.405751in}{2.085345in}}%
\pgfpathlineto{\pgfqpoint{4.406389in}{2.005485in}}%
\pgfpathlineto{\pgfqpoint{4.406207in}{2.198434in}}%
\pgfpathlineto{\pgfqpoint{4.406480in}{2.012569in}}%
\pgfpathlineto{\pgfqpoint{4.407573in}{2.204051in}}%
\pgfpathlineto{\pgfqpoint{4.408752in}{1.890186in}}%
\pgfpathlineto{\pgfqpoint{4.409296in}{2.185255in}}%
\pgfpathlineto{\pgfqpoint{4.409929in}{2.082661in}}%
\pgfpathlineto{\pgfqpoint{4.410200in}{2.227750in}}%
\pgfpathlineto{\pgfqpoint{4.410291in}{1.884895in}}%
\pgfpathlineto{\pgfqpoint{4.410742in}{2.064795in}}%
\pgfpathlineto{\pgfqpoint{4.410832in}{1.745944in}}%
\pgfpathlineto{\pgfqpoint{4.411733in}{2.239257in}}%
\pgfpathlineto{\pgfqpoint{4.412452in}{1.935038in}}%
\pgfpathlineto{\pgfqpoint{4.412901in}{2.186183in}}%
\pgfpathlineto{\pgfqpoint{4.413708in}{1.969278in}}%
\pgfpathlineto{\pgfqpoint{4.413260in}{2.249775in}}%
\pgfpathlineto{\pgfqpoint{4.413977in}{2.070927in}}%
\pgfpathlineto{\pgfqpoint{4.414961in}{1.937852in}}%
\pgfpathlineto{\pgfqpoint{4.415139in}{2.195615in}}%
\pgfpathlineto{\pgfqpoint{4.415764in}{2.229155in}}%
\pgfpathlineto{\pgfqpoint{4.416388in}{1.867496in}}%
\pgfpathlineto{\pgfqpoint{4.417277in}{2.216062in}}%
\pgfpathlineto{\pgfqpoint{4.416743in}{1.846062in}}%
\pgfpathlineto{\pgfqpoint{4.417543in}{2.033735in}}%
\pgfpathlineto{\pgfqpoint{4.417721in}{2.130443in}}%
\pgfpathlineto{\pgfqpoint{4.418341in}{1.965171in}}%
\pgfpathlineto{\pgfqpoint{4.418607in}{2.227502in}}%
\pgfpathlineto{\pgfqpoint{4.419492in}{2.162667in}}%
\pgfpathlineto{\pgfqpoint{4.419580in}{1.818831in}}%
\pgfpathlineto{\pgfqpoint{4.419933in}{2.192896in}}%
\pgfpathlineto{\pgfqpoint{4.420551in}{2.052194in}}%
\pgfpathlineto{\pgfqpoint{4.421168in}{2.207639in}}%
\pgfpathlineto{\pgfqpoint{4.421256in}{1.844713in}}%
\pgfpathlineto{\pgfqpoint{4.421695in}{2.184631in}}%
\pgfpathlineto{\pgfqpoint{4.422398in}{1.940108in}}%
\pgfpathlineto{\pgfqpoint{4.422574in}{2.217056in}}%
\pgfpathlineto{\pgfqpoint{4.422837in}{2.131421in}}%
\pgfpathlineto{\pgfqpoint{4.423450in}{1.834657in}}%
\pgfpathlineto{\pgfqpoint{4.423275in}{2.203097in}}%
\pgfpathlineto{\pgfqpoint{4.423975in}{2.076427in}}%
\pgfpathlineto{\pgfqpoint{4.424325in}{2.218948in}}%
\pgfpathlineto{\pgfqpoint{4.424237in}{2.032315in}}%
\pgfpathlineto{\pgfqpoint{4.424936in}{2.188482in}}%
\pgfpathlineto{\pgfqpoint{4.425285in}{1.812859in}}%
\pgfpathlineto{\pgfqpoint{4.425894in}{2.206321in}}%
\pgfpathlineto{\pgfqpoint{4.426069in}{2.084409in}}%
\pgfpathlineto{\pgfqpoint{4.426503in}{2.178110in}}%
\pgfpathlineto{\pgfqpoint{4.426330in}{1.987168in}}%
\pgfpathlineto{\pgfqpoint{4.426677in}{2.084030in}}%
\pgfpathlineto{\pgfqpoint{4.426764in}{1.891320in}}%
\pgfpathlineto{\pgfqpoint{4.427458in}{2.245578in}}%
\pgfpathlineto{\pgfqpoint{4.427805in}{1.965819in}}%
\pgfpathlineto{\pgfqpoint{4.428065in}{2.198722in}}%
\pgfpathlineto{\pgfqpoint{4.428152in}{2.057845in}}%
\pgfpathlineto{\pgfqpoint{4.428238in}{1.790588in}}%
\pgfpathlineto{\pgfqpoint{4.429190in}{2.200093in}}%
\pgfpathlineto{\pgfqpoint{4.429449in}{1.921737in}}%
\pgfpathlineto{\pgfqpoint{4.429707in}{2.231586in}}%
\pgfpathlineto{\pgfqpoint{4.430397in}{2.032612in}}%
\pgfpathlineto{\pgfqpoint{4.430741in}{2.208731in}}%
\pgfpathlineto{\pgfqpoint{4.431601in}{2.167891in}}%
\pgfpathlineto{\pgfqpoint{4.431687in}{2.215587in}}%
\pgfpathlineto{\pgfqpoint{4.432116in}{1.861735in}}%
\pgfpathlineto{\pgfqpoint{4.432202in}{2.117178in}}%
\pgfpathlineto{\pgfqpoint{4.432288in}{1.813679in}}%
\pgfpathlineto{\pgfqpoint{4.432545in}{2.196202in}}%
\pgfpathlineto{\pgfqpoint{4.433315in}{2.023454in}}%
\pgfpathlineto{\pgfqpoint{4.433401in}{1.995962in}}%
\pgfpathlineto{\pgfqpoint{4.433914in}{2.172900in}}%
\pgfpathlineto{\pgfqpoint{4.434085in}{2.116212in}}%
\pgfpathlineto{\pgfqpoint{4.434852in}{2.254553in}}%
\pgfpathlineto{\pgfqpoint{4.434682in}{1.971060in}}%
\pgfpathlineto{\pgfqpoint{4.434938in}{2.100364in}}%
\pgfpathlineto{\pgfqpoint{4.435534in}{1.976416in}}%
\pgfpathlineto{\pgfqpoint{4.435449in}{2.242851in}}%
\pgfpathlineto{\pgfqpoint{4.435959in}{2.139546in}}%
\pgfpathlineto{\pgfqpoint{4.436044in}{2.189147in}}%
\pgfpathlineto{\pgfqpoint{4.436299in}{1.900173in}}%
\pgfpathlineto{\pgfqpoint{4.436638in}{2.158028in}}%
\pgfpathlineto{\pgfqpoint{4.436723in}{1.912012in}}%
\pgfpathlineto{\pgfqpoint{4.436893in}{2.200769in}}%
\pgfpathlineto{\pgfqpoint{4.437740in}{2.133097in}}%
\pgfpathlineto{\pgfqpoint{4.437825in}{2.227209in}}%
\pgfpathlineto{\pgfqpoint{4.438163in}{1.777121in}}%
\pgfpathlineto{\pgfqpoint{4.438755in}{2.048124in}}%
\pgfpathlineto{\pgfqpoint{4.439177in}{2.193008in}}%
\pgfpathlineto{\pgfqpoint{4.439008in}{1.902747in}}%
\pgfpathlineto{\pgfqpoint{4.440020in}{2.133559in}}%
\pgfpathlineto{\pgfqpoint{4.440188in}{1.808395in}}%
\pgfpathlineto{\pgfqpoint{4.440861in}{2.200361in}}%
\pgfpathlineto{\pgfqpoint{4.441113in}{2.060449in}}%
\pgfpathlineto{\pgfqpoint{4.441365in}{2.196790in}}%
\pgfpathlineto{\pgfqpoint{4.442120in}{2.007062in}}%
\pgfpathlineto{\pgfqpoint{4.442203in}{2.065343in}}%
\pgfpathlineto{\pgfqpoint{4.442538in}{2.207021in}}%
\pgfpathlineto{\pgfqpoint{4.442873in}{1.916110in}}%
\pgfpathlineto{\pgfqpoint{4.443374in}{2.129367in}}%
\pgfpathlineto{\pgfqpoint{4.443708in}{1.840228in}}%
\pgfpathlineto{\pgfqpoint{4.443792in}{2.176462in}}%
\pgfpathlineto{\pgfqpoint{4.444542in}{1.918789in}}%
\pgfpathlineto{\pgfqpoint{4.444792in}{2.190770in}}%
\pgfpathlineto{\pgfqpoint{4.445624in}{2.113217in}}%
\pgfpathlineto{\pgfqpoint{4.446537in}{1.990370in}}%
\pgfpathlineto{\pgfqpoint{4.445873in}{2.156233in}}%
\pgfpathlineto{\pgfqpoint{4.446703in}{2.120936in}}%
\pgfpathlineto{\pgfqpoint{4.447034in}{2.035153in}}%
\pgfpathlineto{\pgfqpoint{4.446951in}{2.193546in}}%
\pgfpathlineto{\pgfqpoint{4.447779in}{2.091147in}}%
\pgfpathlineto{\pgfqpoint{4.448192in}{2.206191in}}%
\pgfpathlineto{\pgfqpoint{4.448357in}{1.963402in}}%
\pgfpathlineto{\pgfqpoint{4.448440in}{2.065961in}}%
\pgfpathlineto{\pgfqpoint{4.448522in}{1.845003in}}%
\pgfpathlineto{\pgfqpoint{4.449100in}{2.209629in}}%
\pgfpathlineto{\pgfqpoint{4.449512in}{2.149947in}}%
\pgfpathlineto{\pgfqpoint{4.449758in}{1.957489in}}%
\pgfpathlineto{\pgfqpoint{4.450170in}{2.177368in}}%
\pgfpathlineto{\pgfqpoint{4.450663in}{2.120398in}}%
\pgfpathlineto{\pgfqpoint{4.450745in}{2.116476in}}%
\pgfpathlineto{\pgfqpoint{4.450827in}{2.128743in}}%
\pgfpathlineto{\pgfqpoint{4.451565in}{2.254079in}}%
\pgfpathlineto{\pgfqpoint{4.451319in}{1.994585in}}%
\pgfpathlineto{\pgfqpoint{4.451974in}{2.178550in}}%
\pgfpathlineto{\pgfqpoint{4.452629in}{1.916064in}}%
\pgfpathlineto{\pgfqpoint{4.452465in}{2.199362in}}%
\pgfpathlineto{\pgfqpoint{4.453037in}{2.184715in}}%
\pgfpathlineto{\pgfqpoint{4.453527in}{1.659150in}}%
\pgfpathlineto{\pgfqpoint{4.453935in}{2.262433in}}%
\pgfpathlineto{\pgfqpoint{4.454179in}{2.143140in}}%
\pgfpathlineto{\pgfqpoint{4.455074in}{2.215057in}}%
\pgfpathlineto{\pgfqpoint{4.454586in}{1.910445in}}%
\pgfpathlineto{\pgfqpoint{4.455236in}{2.163702in}}%
\pgfpathlineto{\pgfqpoint{4.455318in}{2.154235in}}%
\pgfpathlineto{\pgfqpoint{4.455642in}{2.197258in}}%
\pgfpathlineto{\pgfqpoint{4.456372in}{1.941061in}}%
\pgfpathlineto{\pgfqpoint{4.456453in}{2.210307in}}%
\pgfpathlineto{\pgfqpoint{4.456615in}{1.840902in}}%
\pgfpathlineto{\pgfqpoint{4.457505in}{2.035846in}}%
\pgfpathlineto{\pgfqpoint{4.457666in}{2.252227in}}%
\pgfpathlineto{\pgfqpoint{4.457909in}{1.993617in}}%
\pgfpathlineto{\pgfqpoint{4.458473in}{2.060875in}}%
\pgfpathlineto{\pgfqpoint{4.458554in}{1.761821in}}%
\pgfpathlineto{\pgfqpoint{4.458876in}{2.222781in}}%
\pgfpathlineto{\pgfqpoint{4.459520in}{1.966196in}}%
\pgfpathlineto{\pgfqpoint{4.459681in}{2.182937in}}%
\pgfpathlineto{\pgfqpoint{4.460404in}{1.965589in}}%
\pgfpathlineto{\pgfqpoint{4.460564in}{2.066030in}}%
\pgfpathlineto{\pgfqpoint{4.460725in}{1.863425in}}%
\pgfpathlineto{\pgfqpoint{4.461045in}{2.186031in}}%
\pgfpathlineto{\pgfqpoint{4.461526in}{2.003926in}}%
\pgfpathlineto{\pgfqpoint{4.462006in}{2.198867in}}%
\pgfpathlineto{\pgfqpoint{4.462165in}{1.826555in}}%
\pgfpathlineto{\pgfqpoint{4.462645in}{2.033236in}}%
\pgfpathlineto{\pgfqpoint{4.462725in}{2.031944in}}%
\pgfpathlineto{\pgfqpoint{4.463203in}{1.980663in}}%
\pgfpathlineto{\pgfqpoint{4.463602in}{2.175251in}}%
\pgfpathlineto{\pgfqpoint{4.463920in}{1.815549in}}%
\pgfpathlineto{\pgfqpoint{4.464636in}{2.117788in}}%
\pgfpathlineto{\pgfqpoint{4.464715in}{2.196497in}}%
\pgfpathlineto{\pgfqpoint{4.464874in}{1.937249in}}%
\pgfpathlineto{\pgfqpoint{4.465588in}{2.113173in}}%
\pgfpathlineto{\pgfqpoint{4.466459in}{1.849856in}}%
\pgfpathlineto{\pgfqpoint{4.465826in}{2.241776in}}%
\pgfpathlineto{\pgfqpoint{4.466776in}{1.954780in}}%
\pgfpathlineto{\pgfqpoint{4.467171in}{1.846339in}}%
\pgfpathlineto{\pgfqpoint{4.466934in}{2.152357in}}%
\pgfpathlineto{\pgfqpoint{4.467250in}{2.074536in}}%
\pgfpathlineto{\pgfqpoint{4.467408in}{2.199202in}}%
\pgfpathlineto{\pgfqpoint{4.467644in}{1.767539in}}%
\pgfpathlineto{\pgfqpoint{4.468196in}{2.045051in}}%
\pgfpathlineto{\pgfqpoint{4.468747in}{1.958385in}}%
\pgfpathlineto{\pgfqpoint{4.468590in}{2.199594in}}%
\pgfpathlineto{\pgfqpoint{4.469298in}{2.016805in}}%
\pgfpathlineto{\pgfqpoint{4.469376in}{2.195948in}}%
\pgfpathlineto{\pgfqpoint{4.470004in}{1.939507in}}%
\pgfpathlineto{\pgfqpoint{4.470397in}{2.171955in}}%
\pgfpathlineto{\pgfqpoint{4.471258in}{1.864130in}}%
\pgfpathlineto{\pgfqpoint{4.470553in}{2.211031in}}%
\pgfpathlineto{\pgfqpoint{4.471805in}{1.871038in}}%
\pgfpathlineto{\pgfqpoint{4.472118in}{2.202460in}}%
\pgfpathlineto{\pgfqpoint{4.472039in}{1.863748in}}%
\pgfpathlineto{\pgfqpoint{4.472975in}{2.071394in}}%
\pgfpathlineto{\pgfqpoint{4.473053in}{1.986069in}}%
\pgfpathlineto{\pgfqpoint{4.473831in}{2.231947in}}%
\pgfpathlineto{\pgfqpoint{4.474065in}{2.095262in}}%
\pgfpathlineto{\pgfqpoint{4.474686in}{2.219395in}}%
\pgfpathlineto{\pgfqpoint{4.474763in}{1.771513in}}%
\pgfpathlineto{\pgfqpoint{4.475074in}{2.143410in}}%
\pgfpathlineto{\pgfqpoint{4.475384in}{1.695683in}}%
\pgfpathlineto{\pgfqpoint{4.476080in}{2.154162in}}%
\pgfpathlineto{\pgfqpoint{4.476158in}{2.189070in}}%
\pgfpathlineto{\pgfqpoint{4.476235in}{1.954354in}}%
\pgfpathlineto{\pgfqpoint{4.476467in}{2.033057in}}%
\pgfpathlineto{\pgfqpoint{4.476544in}{1.844211in}}%
\pgfpathlineto{\pgfqpoint{4.476930in}{2.156989in}}%
\pgfpathlineto{\pgfqpoint{4.477470in}{2.034048in}}%
\pgfpathlineto{\pgfqpoint{4.477701in}{2.195362in}}%
\pgfpathlineto{\pgfqpoint{4.477778in}{1.968311in}}%
\pgfpathlineto{\pgfqpoint{4.478625in}{2.140549in}}%
\pgfpathlineto{\pgfqpoint{4.478702in}{2.138753in}}%
\pgfpathlineto{\pgfqpoint{4.478779in}{2.178739in}}%
\pgfpathlineto{\pgfqpoint{4.479239in}{2.017909in}}%
\pgfpathlineto{\pgfqpoint{4.479776in}{2.174119in}}%
\pgfpathlineto{\pgfqpoint{4.480160in}{1.852834in}}%
\pgfpathlineto{\pgfqpoint{4.480772in}{2.178057in}}%
\pgfpathlineto{\pgfqpoint{4.480848in}{2.224963in}}%
\pgfpathlineto{\pgfqpoint{4.481231in}{1.956819in}}%
\pgfpathlineto{\pgfqpoint{4.481689in}{2.107660in}}%
\pgfpathlineto{\pgfqpoint{4.482680in}{1.892275in}}%
\pgfpathlineto{\pgfqpoint{4.481994in}{2.181954in}}%
\pgfpathlineto{\pgfqpoint{4.482832in}{1.986813in}}%
\pgfpathlineto{\pgfqpoint{4.483896in}{2.197504in}}%
\pgfpathlineto{\pgfqpoint{4.483213in}{1.898980in}}%
\pgfpathlineto{\pgfqpoint{4.483972in}{2.099046in}}%
\pgfpathlineto{\pgfqpoint{4.484579in}{1.911055in}}%
\pgfpathlineto{\pgfqpoint{4.484807in}{2.220648in}}%
\pgfpathlineto{\pgfqpoint{4.485034in}{2.054141in}}%
\pgfpathlineto{\pgfqpoint{4.485185in}{2.165916in}}%
\pgfpathlineto{\pgfqpoint{4.485261in}{1.879824in}}%
\pgfpathlineto{\pgfqpoint{4.486244in}{2.136222in}}%
\pgfpathlineto{\pgfqpoint{4.486319in}{1.791691in}}%
\pgfpathlineto{\pgfqpoint{4.486847in}{2.180403in}}%
\pgfpathlineto{\pgfqpoint{4.487300in}{2.132470in}}%
\pgfpathlineto{\pgfqpoint{4.488127in}{1.923234in}}%
\pgfpathlineto{\pgfqpoint{4.487902in}{2.141366in}}%
\pgfpathlineto{\pgfqpoint{4.488428in}{2.087305in}}%
\pgfpathlineto{\pgfqpoint{4.489178in}{1.906451in}}%
\pgfpathlineto{\pgfqpoint{4.489553in}{2.251130in}}%
\pgfpathlineto{\pgfqpoint{4.490451in}{1.876723in}}%
\pgfpathlineto{\pgfqpoint{4.490750in}{1.882061in}}%
\pgfpathlineto{\pgfqpoint{4.491198in}{2.161789in}}%
\pgfpathlineto{\pgfqpoint{4.490974in}{1.752609in}}%
\pgfpathlineto{\pgfqpoint{4.491870in}{1.973010in}}%
\pgfpathlineto{\pgfqpoint{4.492019in}{2.163053in}}%
\pgfpathlineto{\pgfqpoint{4.492540in}{1.906866in}}%
\pgfpathlineto{\pgfqpoint{4.492763in}{2.124945in}}%
\pgfpathlineto{\pgfqpoint{4.492837in}{1.754215in}}%
\pgfpathlineto{\pgfqpoint{4.493209in}{2.157228in}}%
\pgfpathlineto{\pgfqpoint{4.493877in}{1.873483in}}%
\pgfpathlineto{\pgfqpoint{4.494395in}{2.188216in}}%
\pgfpathlineto{\pgfqpoint{4.494840in}{1.774884in}}%
\pgfpathlineto{\pgfqpoint{4.494988in}{2.115958in}}%
\pgfpathlineto{\pgfqpoint{4.495062in}{1.866023in}}%
\pgfpathlineto{\pgfqpoint{4.495653in}{2.166606in}}%
\pgfpathlineto{\pgfqpoint{4.496096in}{2.016799in}}%
\pgfpathlineto{\pgfqpoint{4.496980in}{2.189912in}}%
\pgfpathlineto{\pgfqpoint{4.496612in}{1.952287in}}%
\pgfpathlineto{\pgfqpoint{4.497201in}{2.128328in}}%
\pgfpathlineto{\pgfqpoint{4.497863in}{2.202662in}}%
\pgfpathlineto{\pgfqpoint{4.498303in}{1.997354in}}%
\pgfpathlineto{\pgfqpoint{4.499110in}{2.239358in}}%
\pgfpathlineto{\pgfqpoint{4.498963in}{1.977679in}}%
\pgfpathlineto{\pgfqpoint{4.499476in}{2.181895in}}%
\pgfpathlineto{\pgfqpoint{4.500645in}{1.763162in}}%
\pgfpathlineto{\pgfqpoint{4.499695in}{2.219242in}}%
\pgfpathlineto{\pgfqpoint{4.500718in}{1.907684in}}%
\pgfpathlineto{\pgfqpoint{4.500864in}{2.218927in}}%
\pgfpathlineto{\pgfqpoint{4.501666in}{1.832155in}}%
\pgfpathlineto{\pgfqpoint{4.501884in}{2.087077in}}%
\pgfpathlineto{\pgfqpoint{4.502394in}{1.972235in}}%
\pgfpathlineto{\pgfqpoint{4.502466in}{2.206267in}}%
\pgfpathlineto{\pgfqpoint{4.503047in}{2.029377in}}%
\pgfpathlineto{\pgfqpoint{4.503337in}{2.210585in}}%
\pgfpathlineto{\pgfqpoint{4.503192in}{1.937421in}}%
\pgfpathlineto{\pgfqpoint{4.503917in}{2.209581in}}%
\pgfpathlineto{\pgfqpoint{4.504279in}{1.604237in}}%
\pgfpathlineto{\pgfqpoint{4.505002in}{2.128645in}}%
\pgfpathlineto{\pgfqpoint{4.505075in}{2.122236in}}%
\pgfpathlineto{\pgfqpoint{4.505147in}{1.612610in}}%
\pgfpathlineto{\pgfqpoint{4.505940in}{2.182395in}}%
\pgfpathlineto{\pgfqpoint{4.506157in}{2.056406in}}%
\pgfpathlineto{\pgfqpoint{4.506229in}{2.183643in}}%
\pgfpathlineto{\pgfqpoint{4.506733in}{1.818613in}}%
\pgfpathlineto{\pgfqpoint{4.507236in}{2.177300in}}%
\pgfpathlineto{\pgfqpoint{4.507452in}{1.929385in}}%
\pgfpathlineto{\pgfqpoint{4.508313in}{2.256885in}}%
\pgfpathlineto{\pgfqpoint{4.509315in}{1.789425in}}%
\pgfpathlineto{\pgfqpoint{4.509458in}{2.142249in}}%
\pgfpathlineto{\pgfqpoint{4.509530in}{2.159128in}}%
\pgfpathlineto{\pgfqpoint{4.509744in}{1.992296in}}%
\pgfpathlineto{\pgfqpoint{4.509887in}{2.122166in}}%
\pgfpathlineto{\pgfqpoint{4.510885in}{1.855600in}}%
\pgfpathlineto{\pgfqpoint{4.510672in}{2.221155in}}%
\pgfpathlineto{\pgfqpoint{4.510957in}{2.155284in}}%
\pgfpathlineto{\pgfqpoint{4.511028in}{2.196971in}}%
\pgfpathlineto{\pgfqpoint{4.511597in}{1.908019in}}%
\pgfpathlineto{\pgfqpoint{4.511740in}{2.115647in}}%
\pgfpathlineto{\pgfqpoint{4.511953in}{2.182905in}}%
\pgfpathlineto{\pgfqpoint{4.512876in}{1.860293in}}%
\pgfpathlineto{\pgfqpoint{4.513584in}{2.190930in}}%
\pgfpathlineto{\pgfqpoint{4.513655in}{1.774033in}}%
\pgfpathlineto{\pgfqpoint{4.514009in}{1.994627in}}%
\pgfpathlineto{\pgfqpoint{4.514292in}{2.212018in}}%
\pgfpathlineto{\pgfqpoint{4.514645in}{1.891850in}}%
\pgfpathlineto{\pgfqpoint{4.515068in}{2.094054in}}%
\pgfpathlineto{\pgfqpoint{4.515562in}{1.696816in}}%
\pgfpathlineto{\pgfqpoint{4.515492in}{2.194819in}}%
\pgfpathlineto{\pgfqpoint{4.516125in}{2.082100in}}%
\pgfpathlineto{\pgfqpoint{4.516618in}{2.193590in}}%
\pgfpathlineto{\pgfqpoint{4.516758in}{1.951906in}}%
\pgfpathlineto{\pgfqpoint{4.517180in}{2.096973in}}%
\pgfpathlineto{\pgfqpoint{4.517951in}{1.786976in}}%
\pgfpathlineto{\pgfqpoint{4.517741in}{2.187088in}}%
\pgfpathlineto{\pgfqpoint{4.518301in}{1.985155in}}%
\pgfpathlineto{\pgfqpoint{4.519141in}{2.211583in}}%
\pgfpathlineto{\pgfqpoint{4.519001in}{1.768429in}}%
\pgfpathlineto{\pgfqpoint{4.519420in}{2.044774in}}%
\pgfpathlineto{\pgfqpoint{4.519630in}{1.975644in}}%
\pgfpathlineto{\pgfqpoint{4.519909in}{2.211433in}}%
\pgfpathlineto{\pgfqpoint{4.520397in}{1.997462in}}%
\pgfpathlineto{\pgfqpoint{4.520675in}{2.175778in}}%
\pgfpathlineto{\pgfqpoint{4.520954in}{1.944846in}}%
\pgfpathlineto{\pgfqpoint{4.521579in}{2.150409in}}%
\pgfpathlineto{\pgfqpoint{4.521649in}{1.813858in}}%
\pgfpathlineto{\pgfqpoint{4.521718in}{2.209802in}}%
\pgfpathlineto{\pgfqpoint{4.522620in}{1.994172in}}%
\pgfpathlineto{\pgfqpoint{4.523244in}{2.198779in}}%
\pgfpathlineto{\pgfqpoint{4.523520in}{1.881356in}}%
\pgfpathlineto{\pgfqpoint{4.523728in}{2.062969in}}%
\pgfpathlineto{\pgfqpoint{4.524211in}{1.896908in}}%
\pgfpathlineto{\pgfqpoint{4.524418in}{2.243915in}}%
\pgfpathlineto{\pgfqpoint{4.524763in}{2.034300in}}%
\pgfpathlineto{\pgfqpoint{4.525108in}{2.172018in}}%
\pgfpathlineto{\pgfqpoint{4.525384in}{1.899930in}}%
\pgfpathlineto{\pgfqpoint{4.525865in}{2.093344in}}%
\pgfpathlineto{\pgfqpoint{4.526415in}{1.769557in}}%
\pgfpathlineto{\pgfqpoint{4.526072in}{2.177644in}}%
\pgfpathlineto{\pgfqpoint{4.527102in}{1.897981in}}%
\pgfpathlineto{\pgfqpoint{4.527444in}{2.140474in}}%
\pgfpathlineto{\pgfqpoint{4.527924in}{1.876793in}}%
\pgfpathlineto{\pgfqpoint{4.528197in}{2.056034in}}%
\pgfpathlineto{\pgfqpoint{4.528266in}{1.738925in}}%
\pgfpathlineto{\pgfqpoint{4.528539in}{2.158720in}}%
\pgfpathlineto{\pgfqpoint{4.529290in}{2.052587in}}%
\pgfpathlineto{\pgfqpoint{4.529495in}{1.956269in}}%
\pgfpathlineto{\pgfqpoint{4.530244in}{2.155951in}}%
\pgfpathlineto{\pgfqpoint{4.530313in}{2.208453in}}%
\pgfpathlineto{\pgfqpoint{4.530789in}{1.887890in}}%
\pgfpathlineto{\pgfqpoint{4.531061in}{1.968159in}}%
\pgfpathlineto{\pgfqpoint{4.531536in}{1.833946in}}%
\pgfpathlineto{\pgfqpoint{4.531739in}{2.184825in}}%
\pgfpathlineto{\pgfqpoint{4.532011in}{2.006995in}}%
\pgfpathlineto{\pgfqpoint{4.532620in}{2.194001in}}%
\pgfpathlineto{\pgfqpoint{4.532214in}{1.934446in}}%
\pgfpathlineto{\pgfqpoint{4.533026in}{2.079589in}}%
\pgfpathlineto{\pgfqpoint{4.533364in}{1.760652in}}%
\pgfpathlineto{\pgfqpoint{4.533297in}{2.196256in}}%
\pgfpathlineto{\pgfqpoint{4.534107in}{2.108989in}}%
\pgfpathlineto{\pgfqpoint{4.534242in}{1.791443in}}%
\pgfpathlineto{\pgfqpoint{4.534444in}{2.162418in}}%
\pgfpathlineto{\pgfqpoint{4.535050in}{2.068216in}}%
\pgfpathlineto{\pgfqpoint{4.535924in}{2.203960in}}%
\pgfpathlineto{\pgfqpoint{4.535185in}{1.515875in}}%
\pgfpathlineto{\pgfqpoint{4.536126in}{2.171279in}}%
\pgfpathlineto{\pgfqpoint{4.536193in}{1.940539in}}%
\pgfpathlineto{\pgfqpoint{4.536998in}{2.206996in}}%
\pgfpathlineto{\pgfqpoint{4.537266in}{2.044295in}}%
\pgfpathlineto{\pgfqpoint{4.537935in}{2.183131in}}%
\pgfpathlineto{\pgfqpoint{4.538269in}{1.897321in}}%
\pgfpathlineto{\pgfqpoint{4.539003in}{2.201151in}}%
\pgfpathlineto{\pgfqpoint{4.539603in}{2.113298in}}%
\pgfpathlineto{\pgfqpoint{4.540334in}{1.879646in}}%
\pgfpathlineto{\pgfqpoint{4.539935in}{2.160837in}}%
\pgfpathlineto{\pgfqpoint{4.540600in}{2.038308in}}%
\pgfpathlineto{\pgfqpoint{4.540667in}{2.185483in}}%
\pgfpathlineto{\pgfqpoint{4.541529in}{1.863887in}}%
\pgfpathlineto{\pgfqpoint{4.541662in}{2.070492in}}%
\pgfpathlineto{\pgfqpoint{4.542192in}{1.824094in}}%
\pgfpathlineto{\pgfqpoint{4.542125in}{2.153509in}}%
\pgfpathlineto{\pgfqpoint{4.542721in}{2.077318in}}%
\pgfpathlineto{\pgfqpoint{4.542985in}{1.859553in}}%
\pgfpathlineto{\pgfqpoint{4.543183in}{2.123596in}}%
\pgfpathlineto{\pgfqpoint{4.543711in}{1.916934in}}%
\pgfpathlineto{\pgfqpoint{4.544370in}{2.172310in}}%
\pgfpathlineto{\pgfqpoint{4.544172in}{1.908977in}}%
\pgfpathlineto{\pgfqpoint{4.544831in}{2.047606in}}%
\pgfpathlineto{\pgfqpoint{4.545028in}{2.211449in}}%
\pgfpathlineto{\pgfqpoint{4.545619in}{1.918974in}}%
\pgfpathlineto{\pgfqpoint{4.545882in}{2.076898in}}%
\pgfpathlineto{\pgfqpoint{4.546406in}{1.836576in}}%
\pgfpathlineto{\pgfqpoint{4.546930in}{2.141871in}}%
\pgfpathlineto{\pgfqpoint{4.547649in}{1.922323in}}%
\pgfpathlineto{\pgfqpoint{4.547323in}{2.152378in}}%
\pgfpathlineto{\pgfqpoint{4.548041in}{1.996626in}}%
\pgfpathlineto{\pgfqpoint{4.548368in}{2.201412in}}%
\pgfpathlineto{\pgfqpoint{4.548563in}{1.929757in}}%
\pgfpathlineto{\pgfqpoint{4.549150in}{2.133595in}}%
\pgfpathlineto{\pgfqpoint{4.549215in}{1.911903in}}%
\pgfpathlineto{\pgfqpoint{4.549345in}{2.204228in}}%
\pgfpathlineto{\pgfqpoint{4.550255in}{1.960196in}}%
\pgfpathlineto{\pgfqpoint{4.551163in}{2.144954in}}%
\pgfpathlineto{\pgfqpoint{4.551293in}{1.844764in}}%
\pgfpathlineto{\pgfqpoint{4.551357in}{2.002126in}}%
\pgfpathlineto{\pgfqpoint{4.551811in}{2.199086in}}%
\pgfpathlineto{\pgfqpoint{4.551616in}{1.841125in}}%
\pgfpathlineto{\pgfqpoint{4.552457in}{2.125962in}}%
\pgfpathlineto{\pgfqpoint{4.553232in}{1.772329in}}%
\pgfpathlineto{\pgfqpoint{4.552586in}{2.176510in}}%
\pgfpathlineto{\pgfqpoint{4.553618in}{1.965215in}}%
\pgfpathlineto{\pgfqpoint{4.553940in}{2.159833in}}%
\pgfpathlineto{\pgfqpoint{4.553811in}{1.849467in}}%
\pgfpathlineto{\pgfqpoint{4.554712in}{2.090887in}}%
\pgfpathlineto{\pgfqpoint{4.555097in}{1.703018in}}%
\pgfpathlineto{\pgfqpoint{4.555290in}{2.152506in}}%
\pgfpathlineto{\pgfqpoint{4.555867in}{1.998001in}}%
\pgfpathlineto{\pgfqpoint{4.556763in}{2.211863in}}%
\pgfpathlineto{\pgfqpoint{4.556187in}{1.765362in}}%
\pgfpathlineto{\pgfqpoint{4.556955in}{2.138553in}}%
\pgfpathlineto{\pgfqpoint{4.557849in}{2.159284in}}%
\pgfpathlineto{\pgfqpoint{4.558104in}{1.932867in}}%
\pgfpathlineto{\pgfqpoint{4.558359in}{2.197551in}}%
\pgfpathlineto{\pgfqpoint{4.558677in}{1.812042in}}%
\pgfpathlineto{\pgfqpoint{4.558868in}{2.012063in}}%
\pgfpathlineto{\pgfqpoint{4.559441in}{2.188791in}}%
\pgfpathlineto{\pgfqpoint{4.559949in}{1.471519in}}%
\pgfpathlineto{\pgfqpoint{4.560139in}{2.208192in}}%
\pgfpathlineto{\pgfqpoint{4.561090in}{2.118537in}}%
\pgfpathlineto{\pgfqpoint{4.562101in}{1.915344in}}%
\pgfpathlineto{\pgfqpoint{4.562164in}{2.185650in}}%
\pgfpathlineto{\pgfqpoint{4.562228in}{1.940356in}}%
\pgfpathlineto{\pgfqpoint{4.563111in}{2.207262in}}%
\pgfpathlineto{\pgfqpoint{4.563048in}{1.831694in}}%
\pgfpathlineto{\pgfqpoint{4.563362in}{2.119607in}}%
\pgfpathlineto{\pgfqpoint{4.563425in}{1.822838in}}%
\pgfpathlineto{\pgfqpoint{4.563677in}{2.204852in}}%
\pgfpathlineto{\pgfqpoint{4.564431in}{2.148287in}}%
\pgfpathlineto{\pgfqpoint{4.565498in}{1.994294in}}%
\pgfpathlineto{\pgfqpoint{4.565560in}{2.120027in}}%
\pgfpathlineto{\pgfqpoint{4.565873in}{1.877502in}}%
\pgfpathlineto{\pgfqpoint{4.566249in}{2.151328in}}%
\pgfpathlineto{\pgfqpoint{4.566624in}{2.019351in}}%
\pgfpathlineto{\pgfqpoint{4.567186in}{1.745849in}}%
\pgfpathlineto{\pgfqpoint{4.567809in}{2.153143in}}%
\pgfpathlineto{\pgfqpoint{4.568805in}{1.906820in}}%
\pgfpathlineto{\pgfqpoint{4.568992in}{2.067567in}}%
\pgfpathlineto{\pgfqpoint{4.569550in}{2.168650in}}%
\pgfpathlineto{\pgfqpoint{4.569488in}{1.984500in}}%
\pgfpathlineto{\pgfqpoint{4.569984in}{2.162749in}}%
\pgfpathlineto{\pgfqpoint{4.570108in}{1.619877in}}%
\pgfpathlineto{\pgfqpoint{4.570604in}{2.175497in}}%
\pgfpathlineto{\pgfqpoint{4.571099in}{2.141841in}}%
\pgfpathlineto{\pgfqpoint{4.571963in}{1.843947in}}%
\pgfpathlineto{\pgfqpoint{4.571902in}{2.164491in}}%
\pgfpathlineto{\pgfqpoint{4.572272in}{2.008982in}}%
\pgfpathlineto{\pgfqpoint{4.572765in}{1.594031in}}%
\pgfpathlineto{\pgfqpoint{4.572457in}{2.108405in}}%
\pgfpathlineto{\pgfqpoint{4.573319in}{1.970513in}}%
\pgfpathlineto{\pgfqpoint{4.573442in}{2.172030in}}%
\pgfpathlineto{\pgfqpoint{4.573503in}{1.955034in}}%
\pgfpathlineto{\pgfqpoint{4.574424in}{2.091364in}}%
\pgfpathlineto{\pgfqpoint{4.574976in}{1.858597in}}%
\pgfpathlineto{\pgfqpoint{4.574731in}{2.184720in}}%
\pgfpathlineto{\pgfqpoint{4.575527in}{1.902687in}}%
\pgfpathlineto{\pgfqpoint{4.576322in}{2.160274in}}%
\pgfpathlineto{\pgfqpoint{4.576077in}{1.845220in}}%
\pgfpathlineto{\pgfqpoint{4.576566in}{2.095570in}}%
\pgfpathlineto{\pgfqpoint{4.576627in}{1.791912in}}%
\pgfpathlineto{\pgfqpoint{4.576932in}{2.172691in}}%
\pgfpathlineto{\pgfqpoint{4.577663in}{2.017216in}}%
\pgfpathlineto{\pgfqpoint{4.577785in}{2.148722in}}%
\pgfpathlineto{\pgfqpoint{4.578636in}{1.913498in}}%
\pgfpathlineto{\pgfqpoint{4.578757in}{2.085947in}}%
\pgfpathlineto{\pgfqpoint{4.579425in}{1.782685in}}%
\pgfpathlineto{\pgfqpoint{4.579546in}{2.200650in}}%
\pgfpathlineto{\pgfqpoint{4.579849in}{2.075600in}}%
\pgfpathlineto{\pgfqpoint{4.580212in}{1.862003in}}%
\pgfpathlineto{\pgfqpoint{4.580575in}{2.132124in}}%
\pgfpathlineto{\pgfqpoint{4.580938in}{1.985286in}}%
\pgfpathlineto{\pgfqpoint{4.581662in}{2.178084in}}%
\pgfpathlineto{\pgfqpoint{4.581722in}{1.691836in}}%
\pgfpathlineto{\pgfqpoint{4.582084in}{2.157968in}}%
\pgfpathlineto{\pgfqpoint{4.582926in}{1.640061in}}%
\pgfpathlineto{\pgfqpoint{4.582806in}{2.166469in}}%
\pgfpathlineto{\pgfqpoint{4.583287in}{2.023367in}}%
\pgfpathlineto{\pgfqpoint{4.583527in}{2.193956in}}%
\pgfpathlineto{\pgfqpoint{4.583647in}{1.860658in}}%
\pgfpathlineto{\pgfqpoint{4.584307in}{2.011488in}}%
\pgfpathlineto{\pgfqpoint{4.584666in}{1.834348in}}%
\pgfpathlineto{\pgfqpoint{4.584786in}{2.180626in}}%
\pgfpathlineto{\pgfqpoint{4.585384in}{2.047911in}}%
\pgfpathlineto{\pgfqpoint{4.585683in}{1.949812in}}%
\pgfpathlineto{\pgfqpoint{4.585743in}{2.049443in}}%
\pgfpathlineto{\pgfqpoint{4.585803in}{2.026186in}}%
\pgfpathlineto{\pgfqpoint{4.586161in}{2.211075in}}%
\pgfpathlineto{\pgfqpoint{4.586280in}{1.847886in}}%
\pgfpathlineto{\pgfqpoint{4.586876in}{2.027236in}}%
\pgfpathlineto{\pgfqpoint{4.587293in}{1.756615in}}%
\pgfpathlineto{\pgfqpoint{4.587650in}{2.125629in}}%
\pgfpathlineto{\pgfqpoint{4.588006in}{1.822502in}}%
\pgfpathlineto{\pgfqpoint{4.588956in}{2.135144in}}%
\pgfpathlineto{\pgfqpoint{4.588837in}{1.745584in}}%
\pgfpathlineto{\pgfqpoint{4.589193in}{2.072548in}}%
\pgfpathlineto{\pgfqpoint{4.589430in}{1.889312in}}%
\pgfpathlineto{\pgfqpoint{4.589903in}{2.134141in}}%
\pgfpathlineto{\pgfqpoint{4.590317in}{2.049859in}}%
\pgfpathlineto{\pgfqpoint{4.591261in}{2.205806in}}%
\pgfpathlineto{\pgfqpoint{4.591026in}{1.739360in}}%
\pgfpathlineto{\pgfqpoint{4.591320in}{2.025946in}}%
\pgfpathlineto{\pgfqpoint{4.592204in}{1.843705in}}%
\pgfpathlineto{\pgfqpoint{4.592027in}{2.184374in}}%
\pgfpathlineto{\pgfqpoint{4.592321in}{1.927255in}}%
\pgfpathlineto{\pgfqpoint{4.592733in}{2.204178in}}%
\pgfpathlineto{\pgfqpoint{4.593144in}{1.869051in}}%
\pgfpathlineto{\pgfqpoint{4.593437in}{1.949075in}}%
\pgfpathlineto{\pgfqpoint{4.593496in}{2.173786in}}%
\pgfpathlineto{\pgfqpoint{4.594199in}{1.941355in}}%
\pgfpathlineto{\pgfqpoint{4.594550in}{2.161846in}}%
\pgfpathlineto{\pgfqpoint{4.595660in}{1.914943in}}%
\pgfpathlineto{\pgfqpoint{4.596535in}{2.179771in}}%
\pgfpathlineto{\pgfqpoint{4.596243in}{1.823707in}}%
\pgfpathlineto{\pgfqpoint{4.596826in}{2.164440in}}%
\pgfpathlineto{\pgfqpoint{4.597756in}{1.770170in}}%
\pgfpathlineto{\pgfqpoint{4.596942in}{2.185282in}}%
\pgfpathlineto{\pgfqpoint{4.597930in}{1.932987in}}%
\pgfpathlineto{\pgfqpoint{4.598741in}{2.129315in}}%
\pgfpathlineto{\pgfqpoint{4.598336in}{1.855848in}}%
\pgfpathlineto{\pgfqpoint{4.599031in}{1.924411in}}%
\pgfpathlineto{\pgfqpoint{4.599378in}{2.162895in}}%
\pgfpathlineto{\pgfqpoint{4.599841in}{1.885473in}}%
\pgfpathlineto{\pgfqpoint{4.600072in}{2.044913in}}%
\pgfpathlineto{\pgfqpoint{4.600649in}{1.901943in}}%
\pgfpathlineto{\pgfqpoint{4.600591in}{2.196852in}}%
\pgfpathlineto{\pgfqpoint{4.601167in}{2.041243in}}%
\pgfpathlineto{\pgfqpoint{4.601973in}{1.868578in}}%
\pgfpathlineto{\pgfqpoint{4.601858in}{2.158264in}}%
\pgfpathlineto{\pgfqpoint{4.602145in}{2.065584in}}%
\pgfpathlineto{\pgfqpoint{4.602203in}{2.194271in}}%
\pgfpathlineto{\pgfqpoint{4.602318in}{1.980108in}}%
\pgfpathlineto{\pgfqpoint{4.603236in}{2.052394in}}%
\pgfpathlineto{\pgfqpoint{4.603522in}{2.165879in}}%
\pgfpathlineto{\pgfqpoint{4.603809in}{1.962987in}}%
\pgfpathlineto{\pgfqpoint{4.604095in}{2.165671in}}%
\pgfpathlineto{\pgfqpoint{4.604780in}{1.778516in}}%
\pgfpathlineto{\pgfqpoint{4.605180in}{2.204829in}}%
\pgfpathlineto{\pgfqpoint{4.605237in}{2.059427in}}%
\pgfpathlineto{\pgfqpoint{4.606092in}{2.156554in}}%
\pgfpathlineto{\pgfqpoint{4.605693in}{1.887050in}}%
\pgfpathlineto{\pgfqpoint{4.606206in}{1.988377in}}%
\pgfpathlineto{\pgfqpoint{4.606263in}{1.667477in}}%
\pgfpathlineto{\pgfqpoint{4.606547in}{2.148688in}}%
\pgfpathlineto{\pgfqpoint{4.607286in}{2.107676in}}%
\pgfpathlineto{\pgfqpoint{4.607513in}{1.888898in}}%
\pgfpathlineto{\pgfqpoint{4.607456in}{2.128900in}}%
\pgfpathlineto{\pgfqpoint{4.608420in}{1.969500in}}%
\pgfpathlineto{\pgfqpoint{4.609268in}{2.183275in}}%
\pgfpathlineto{\pgfqpoint{4.609381in}{1.775660in}}%
\pgfpathlineto{\pgfqpoint{4.609438in}{2.052227in}}%
\pgfpathlineto{\pgfqpoint{4.609720in}{1.675824in}}%
\pgfpathlineto{\pgfqpoint{4.609663in}{2.158297in}}%
\pgfpathlineto{\pgfqpoint{4.610566in}{1.976264in}}%
\pgfpathlineto{\pgfqpoint{4.610622in}{1.972370in}}%
\pgfpathlineto{\pgfqpoint{4.610847in}{2.155632in}}%
\pgfpathlineto{\pgfqpoint{4.611635in}{1.697421in}}%
\pgfpathlineto{\pgfqpoint{4.611859in}{2.180249in}}%
\pgfpathlineto{\pgfqpoint{4.612757in}{2.078219in}}%
\pgfpathlineto{\pgfqpoint{4.613373in}{2.167735in}}%
\pgfpathlineto{\pgfqpoint{4.613149in}{1.816868in}}%
\pgfpathlineto{\pgfqpoint{4.613485in}{1.909539in}}%
\pgfpathlineto{\pgfqpoint{4.614044in}{1.844943in}}%
\pgfpathlineto{\pgfqpoint{4.613932in}{2.128045in}}%
\pgfpathlineto{\pgfqpoint{4.614100in}{2.060556in}}%
\pgfpathlineto{\pgfqpoint{4.614435in}{2.183519in}}%
\pgfpathlineto{\pgfqpoint{4.614770in}{1.747307in}}%
\pgfpathlineto{\pgfqpoint{4.615104in}{1.912667in}}%
\pgfpathlineto{\pgfqpoint{4.615327in}{1.838434in}}%
\pgfpathlineto{\pgfqpoint{4.615272in}{2.078448in}}%
\pgfpathlineto{\pgfqpoint{4.615383in}{1.996656in}}%
\pgfpathlineto{\pgfqpoint{4.615828in}{2.153351in}}%
\pgfpathlineto{\pgfqpoint{4.616051in}{1.864509in}}%
\pgfpathlineto{\pgfqpoint{4.616440in}{1.933734in}}%
\pgfpathlineto{\pgfqpoint{4.616884in}{1.507132in}}%
\pgfpathlineto{\pgfqpoint{4.616829in}{2.136001in}}%
\pgfpathlineto{\pgfqpoint{4.617383in}{2.057975in}}%
\pgfpathlineto{\pgfqpoint{4.618325in}{2.174500in}}%
\pgfpathlineto{\pgfqpoint{4.617993in}{1.903884in}}%
\pgfpathlineto{\pgfqpoint{4.618491in}{2.081657in}}%
\pgfpathlineto{\pgfqpoint{4.618878in}{2.182595in}}%
\pgfpathlineto{\pgfqpoint{4.619650in}{1.867107in}}%
\pgfpathlineto{\pgfqpoint{4.620091in}{1.791975in}}%
\pgfpathlineto{\pgfqpoint{4.620752in}{2.165933in}}%
\pgfpathlineto{\pgfqpoint{4.621301in}{1.841822in}}%
\pgfpathlineto{\pgfqpoint{4.621576in}{2.191523in}}%
\pgfpathlineto{\pgfqpoint{4.621905in}{1.910475in}}%
\pgfpathlineto{\pgfqpoint{4.621960in}{2.167229in}}%
\pgfpathlineto{\pgfqpoint{4.622563in}{1.897164in}}%
\pgfpathlineto{\pgfqpoint{4.623001in}{2.039704in}}%
\pgfpathlineto{\pgfqpoint{4.623383in}{2.134433in}}%
\pgfpathlineto{\pgfqpoint{4.623110in}{1.779770in}}%
\pgfpathlineto{\pgfqpoint{4.623438in}{1.927900in}}%
\pgfpathlineto{\pgfqpoint{4.623493in}{1.794594in}}%
\pgfpathlineto{\pgfqpoint{4.624093in}{2.112389in}}%
\pgfpathlineto{\pgfqpoint{4.624475in}{2.030995in}}%
\pgfpathlineto{\pgfqpoint{4.625292in}{2.140743in}}%
\pgfpathlineto{\pgfqpoint{4.624693in}{1.683349in}}%
\pgfpathlineto{\pgfqpoint{4.625564in}{2.063817in}}%
\pgfpathlineto{\pgfqpoint{4.626270in}{1.855230in}}%
\pgfpathlineto{\pgfqpoint{4.626596in}{2.151455in}}%
\pgfpathlineto{\pgfqpoint{4.626650in}{2.005381in}}%
\pgfpathlineto{\pgfqpoint{4.627246in}{2.183049in}}%
\pgfpathlineto{\pgfqpoint{4.627571in}{1.816161in}}%
\pgfpathlineto{\pgfqpoint{4.627734in}{2.067228in}}%
\pgfpathlineto{\pgfqpoint{4.627842in}{1.830378in}}%
\pgfpathlineto{\pgfqpoint{4.628598in}{2.138328in}}%
\pgfpathlineto{\pgfqpoint{4.628814in}{2.029089in}}%
\pgfpathlineto{\pgfqpoint{4.629138in}{2.139305in}}%
\pgfpathlineto{\pgfqpoint{4.629353in}{1.902844in}}%
\pgfpathlineto{\pgfqpoint{4.629892in}{2.063146in}}%
\pgfpathlineto{\pgfqpoint{4.630914in}{1.821915in}}%
\pgfpathlineto{\pgfqpoint{4.630484in}{2.182857in}}%
\pgfpathlineto{\pgfqpoint{4.630967in}{2.068190in}}%
\pgfpathlineto{\pgfqpoint{4.631557in}{1.911744in}}%
\pgfpathlineto{\pgfqpoint{4.631236in}{2.158727in}}%
\pgfpathlineto{\pgfqpoint{4.631986in}{1.975139in}}%
\pgfpathlineto{\pgfqpoint{4.633056in}{2.198041in}}%
\pgfpathlineto{\pgfqpoint{4.633003in}{1.854683in}}%
\pgfpathlineto{\pgfqpoint{4.633109in}{2.103093in}}%
\pgfpathlineto{\pgfqpoint{4.633216in}{2.021064in}}%
\pgfpathlineto{\pgfqpoint{4.633270in}{2.126935in}}%
\pgfpathlineto{\pgfqpoint{4.633376in}{1.715881in}}%
\pgfpathlineto{\pgfqpoint{4.634123in}{2.127474in}}%
\pgfpathlineto{\pgfqpoint{4.634443in}{1.934487in}}%
\pgfpathlineto{\pgfqpoint{4.634656in}{2.105173in}}%
\pgfpathlineto{\pgfqpoint{4.634762in}{1.814108in}}%
\pgfpathlineto{\pgfqpoint{4.635613in}{2.075582in}}%
\pgfpathlineto{\pgfqpoint{4.636196in}{1.852378in}}%
\pgfpathlineto{\pgfqpoint{4.635825in}{2.156538in}}%
\pgfpathlineto{\pgfqpoint{4.636568in}{2.009156in}}%
\pgfpathlineto{\pgfqpoint{4.636673in}{1.806277in}}%
\pgfpathlineto{\pgfqpoint{4.637679in}{2.146749in}}%
\pgfpathlineto{\pgfqpoint{4.637784in}{1.784900in}}%
\pgfpathlineto{\pgfqpoint{4.638787in}{2.042965in}}%
\pgfpathlineto{\pgfqpoint{4.638945in}{2.140728in}}%
\pgfpathlineto{\pgfqpoint{4.639524in}{1.819874in}}%
\pgfpathlineto{\pgfqpoint{4.639840in}{2.064948in}}%
\pgfpathlineto{\pgfqpoint{4.640260in}{1.721040in}}%
\pgfpathlineto{\pgfqpoint{4.640680in}{2.110473in}}%
\pgfpathlineto{\pgfqpoint{4.640890in}{1.933904in}}%
\pgfpathlineto{\pgfqpoint{4.641414in}{2.115725in}}%
\pgfpathlineto{\pgfqpoint{4.641519in}{1.713308in}}%
\pgfpathlineto{\pgfqpoint{4.642043in}{2.089621in}}%
\pgfpathlineto{\pgfqpoint{4.642565in}{1.750236in}}%
\pgfpathlineto{\pgfqpoint{4.642774in}{2.109132in}}%
\pgfpathlineto{\pgfqpoint{4.643140in}{1.947230in}}%
\pgfpathlineto{\pgfqpoint{4.643505in}{2.172997in}}%
\pgfpathlineto{\pgfqpoint{4.643713in}{1.844126in}}%
\pgfpathlineto{\pgfqpoint{4.644286in}{2.071721in}}%
\pgfpathlineto{\pgfqpoint{4.645014in}{1.773029in}}%
\pgfpathlineto{\pgfqpoint{4.645221in}{2.161862in}}%
\pgfpathlineto{\pgfqpoint{4.645481in}{1.922606in}}%
\pgfpathlineto{\pgfqpoint{4.645688in}{2.130753in}}%
\pgfpathlineto{\pgfqpoint{4.645740in}{1.797654in}}%
\pgfpathlineto{\pgfqpoint{4.646828in}{2.035970in}}%
\pgfpathlineto{\pgfqpoint{4.647190in}{1.699309in}}%
\pgfpathlineto{\pgfqpoint{4.647861in}{2.136059in}}%
\pgfpathlineto{\pgfqpoint{4.647912in}{2.030379in}}%
\pgfpathlineto{\pgfqpoint{4.648840in}{1.609355in}}%
\pgfpathlineto{\pgfqpoint{4.648170in}{2.168729in}}%
\pgfpathlineto{\pgfqpoint{4.649046in}{1.952671in}}%
\pgfpathlineto{\pgfqpoint{4.649611in}{2.153522in}}%
\pgfpathlineto{\pgfqpoint{4.649457in}{1.884282in}}%
\pgfpathlineto{\pgfqpoint{4.650022in}{2.124963in}}%
\pgfpathlineto{\pgfqpoint{4.650996in}{1.744647in}}%
\pgfpathlineto{\pgfqpoint{4.650279in}{2.134594in}}%
\pgfpathlineto{\pgfqpoint{4.651099in}{1.761202in}}%
\pgfpathlineto{\pgfqpoint{4.651713in}{2.123200in}}%
\pgfpathlineto{\pgfqpoint{4.651815in}{1.515297in}}%
\pgfpathlineto{\pgfqpoint{4.652224in}{2.044469in}}%
\pgfpathlineto{\pgfqpoint{4.652683in}{2.118875in}}%
\pgfpathlineto{\pgfqpoint{4.653244in}{1.709189in}}%
\pgfpathlineto{\pgfqpoint{4.653397in}{2.137565in}}%
\pgfpathlineto{\pgfqpoint{4.654363in}{1.989964in}}%
\pgfpathlineto{\pgfqpoint{4.654668in}{1.732029in}}%
\pgfpathlineto{\pgfqpoint{4.654770in}{2.078317in}}%
\pgfpathlineto{\pgfqpoint{4.654922in}{2.071488in}}%
\pgfpathlineto{\pgfqpoint{4.654973in}{2.179628in}}%
\pgfpathlineto{\pgfqpoint{4.655834in}{1.755542in}}%
\pgfpathlineto{\pgfqpoint{4.655986in}{2.032783in}}%
\pgfpathlineto{\pgfqpoint{4.656138in}{2.123487in}}%
\pgfpathlineto{\pgfqpoint{4.656087in}{1.886396in}}%
\pgfpathlineto{\pgfqpoint{4.656441in}{2.068225in}}%
\pgfpathlineto{\pgfqpoint{4.656694in}{1.554292in}}%
\pgfpathlineto{\pgfqpoint{4.657300in}{2.134102in}}%
\pgfpathlineto{\pgfqpoint{4.657502in}{1.985514in}}%
\pgfpathlineto{\pgfqpoint{4.657855in}{2.174157in}}%
\pgfpathlineto{\pgfqpoint{4.658157in}{1.710512in}}%
\pgfpathlineto{\pgfqpoint{4.658610in}{2.047197in}}%
\pgfpathlineto{\pgfqpoint{4.658761in}{1.874247in}}%
\pgfpathlineto{\pgfqpoint{4.659012in}{2.096609in}}%
\pgfpathlineto{\pgfqpoint{4.659715in}{1.980237in}}%
\pgfpathlineto{\pgfqpoint{4.660016in}{2.126047in}}%
\pgfpathlineto{\pgfqpoint{4.660517in}{1.789709in}}%
\pgfpathlineto{\pgfqpoint{4.660818in}{1.988368in}}%
\pgfpathlineto{\pgfqpoint{4.660968in}{2.030250in}}%
\pgfpathlineto{\pgfqpoint{4.661018in}{1.742992in}}%
\pgfpathlineto{\pgfqpoint{4.661767in}{2.137990in}}%
\pgfpathlineto{\pgfqpoint{4.662067in}{2.001504in}}%
\pgfpathlineto{\pgfqpoint{4.662117in}{1.820287in}}%
\pgfpathlineto{\pgfqpoint{4.663014in}{2.112144in}}%
\pgfpathlineto{\pgfqpoint{4.663113in}{1.991391in}}%
\pgfpathlineto{\pgfqpoint{4.664058in}{2.114568in}}%
\pgfpathlineto{\pgfqpoint{4.663710in}{1.780854in}}%
\pgfpathlineto{\pgfqpoint{4.664257in}{2.068161in}}%
\pgfpathlineto{\pgfqpoint{4.665149in}{1.825529in}}%
\pgfpathlineto{\pgfqpoint{4.665199in}{2.122071in}}%
\pgfpathlineto{\pgfqpoint{4.665397in}{1.924369in}}%
\pgfpathlineto{\pgfqpoint{4.665743in}{2.139511in}}%
\pgfpathlineto{\pgfqpoint{4.666435in}{1.976143in}}%
\pgfpathlineto{\pgfqpoint{4.667471in}{1.824148in}}%
\pgfpathlineto{\pgfqpoint{4.666583in}{2.135016in}}%
\pgfpathlineto{\pgfqpoint{4.667520in}{2.043400in}}%
\pgfpathlineto{\pgfqpoint{4.668455in}{1.916559in}}%
\pgfpathlineto{\pgfqpoint{4.668160in}{2.132658in}}%
\pgfpathlineto{\pgfqpoint{4.668602in}{1.975038in}}%
\pgfpathlineto{\pgfqpoint{4.669241in}{2.129369in}}%
\pgfpathlineto{\pgfqpoint{4.669584in}{1.839991in}}%
\pgfpathlineto{\pgfqpoint{4.669682in}{2.007798in}}%
\pgfpathlineto{\pgfqpoint{4.669927in}{1.781912in}}%
\pgfpathlineto{\pgfqpoint{4.670074in}{2.155676in}}%
\pgfpathlineto{\pgfqpoint{4.670661in}{1.956845in}}%
\pgfpathlineto{\pgfqpoint{4.671638in}{2.126141in}}%
\pgfpathlineto{\pgfqpoint{4.670905in}{1.835380in}}%
\pgfpathlineto{\pgfqpoint{4.671735in}{2.077976in}}%
\pgfpathlineto{\pgfqpoint{4.671930in}{2.146551in}}%
\pgfpathlineto{\pgfqpoint{4.672904in}{1.859770in}}%
\pgfpathlineto{\pgfqpoint{4.673779in}{2.138411in}}%
\pgfpathlineto{\pgfqpoint{4.673827in}{1.815000in}}%
\pgfpathlineto{\pgfqpoint{4.673973in}{2.124469in}}%
\pgfpathlineto{\pgfqpoint{4.674021in}{1.627315in}}%
\pgfpathlineto{\pgfqpoint{4.674652in}{2.125165in}}%
\pgfpathlineto{\pgfqpoint{4.675087in}{1.918426in}}%
\pgfpathlineto{\pgfqpoint{4.676054in}{2.109198in}}%
\pgfpathlineto{\pgfqpoint{4.675619in}{1.854574in}}%
\pgfpathlineto{\pgfqpoint{4.676199in}{1.966994in}}%
\pgfpathlineto{\pgfqpoint{4.676922in}{2.138265in}}%
\pgfpathlineto{\pgfqpoint{4.676585in}{1.878097in}}%
\pgfpathlineto{\pgfqpoint{4.677355in}{2.101313in}}%
\pgfpathlineto{\pgfqpoint{4.677788in}{1.592701in}}%
\pgfpathlineto{\pgfqpoint{4.677500in}{2.142070in}}%
\pgfpathlineto{\pgfqpoint{4.678461in}{2.063812in}}%
\pgfpathlineto{\pgfqpoint{4.679037in}{1.794051in}}%
\pgfpathlineto{\pgfqpoint{4.679612in}{2.128272in}}%
\pgfpathlineto{\pgfqpoint{4.679707in}{1.896143in}}%
\pgfpathlineto{\pgfqpoint{4.680711in}{2.006082in}}%
\pgfpathlineto{\pgfqpoint{4.681189in}{2.137059in}}%
\pgfpathlineto{\pgfqpoint{4.681236in}{1.797023in}}%
\pgfpathlineto{\pgfqpoint{4.681856in}{2.129318in}}%
\pgfpathlineto{\pgfqpoint{4.682855in}{1.793587in}}%
\pgfpathlineto{\pgfqpoint{4.682950in}{2.158934in}}%
\pgfpathlineto{\pgfqpoint{4.684135in}{1.754814in}}%
\pgfpathlineto{\pgfqpoint{4.684183in}{1.878359in}}%
\pgfpathlineto{\pgfqpoint{4.684940in}{2.127706in}}%
\pgfpathlineto{\pgfqpoint{4.685035in}{1.835312in}}%
\pgfpathlineto{\pgfqpoint{4.685271in}{1.914319in}}%
\pgfpathlineto{\pgfqpoint{4.685743in}{1.744795in}}%
\pgfpathlineto{\pgfqpoint{4.686026in}{2.133236in}}%
\pgfpathlineto{\pgfqpoint{4.686215in}{2.058143in}}%
\pgfpathlineto{\pgfqpoint{4.686591in}{1.841814in}}%
\pgfpathlineto{\pgfqpoint{4.686497in}{2.100183in}}%
\pgfpathlineto{\pgfqpoint{4.687391in}{2.003544in}}%
\pgfpathlineto{\pgfqpoint{4.688049in}{2.143643in}}%
\pgfpathlineto{\pgfqpoint{4.687814in}{1.756466in}}%
\pgfpathlineto{\pgfqpoint{4.688518in}{2.056769in}}%
\pgfpathlineto{\pgfqpoint{4.689127in}{2.120373in}}%
\pgfpathlineto{\pgfqpoint{4.688986in}{1.824957in}}%
\pgfpathlineto{\pgfqpoint{4.689548in}{1.961851in}}%
\pgfpathlineto{\pgfqpoint{4.689922in}{2.144542in}}%
\pgfpathlineto{\pgfqpoint{4.689688in}{1.661631in}}%
\pgfpathlineto{\pgfqpoint{4.690622in}{1.940408in}}%
\pgfpathlineto{\pgfqpoint{4.691042in}{2.132337in}}%
\pgfpathlineto{\pgfqpoint{4.691461in}{1.774531in}}%
\pgfpathlineto{\pgfqpoint{4.691508in}{2.150138in}}%
\pgfpathlineto{\pgfqpoint{4.692577in}{2.092411in}}%
\pgfpathlineto{\pgfqpoint{4.692902in}{1.674758in}}%
\pgfpathlineto{\pgfqpoint{4.693041in}{2.139467in}}%
\pgfpathlineto{\pgfqpoint{4.693644in}{2.058928in}}%
\pgfpathlineto{\pgfqpoint{4.693829in}{2.116551in}}%
\pgfpathlineto{\pgfqpoint{4.694107in}{1.612273in}}%
\pgfpathlineto{\pgfqpoint{4.694569in}{2.026818in}}%
\pgfpathlineto{\pgfqpoint{4.694754in}{1.578289in}}%
\pgfpathlineto{\pgfqpoint{4.695031in}{2.127130in}}%
\pgfpathlineto{\pgfqpoint{4.695723in}{1.830562in}}%
\pgfpathlineto{\pgfqpoint{4.696414in}{2.127154in}}%
\pgfpathlineto{\pgfqpoint{4.696874in}{2.106235in}}%
\pgfpathlineto{\pgfqpoint{4.697287in}{1.804765in}}%
\pgfpathlineto{\pgfqpoint{4.697425in}{2.135024in}}%
\pgfpathlineto{\pgfqpoint{4.698021in}{1.994792in}}%
\pgfpathlineto{\pgfqpoint{4.698434in}{1.909095in}}%
\pgfpathlineto{\pgfqpoint{4.698525in}{2.108858in}}%
\pgfpathlineto{\pgfqpoint{4.699029in}{1.806324in}}%
\pgfpathlineto{\pgfqpoint{4.699577in}{2.110944in}}%
\pgfpathlineto{\pgfqpoint{4.699668in}{1.855319in}}%
\pgfpathlineto{\pgfqpoint{4.700262in}{2.143600in}}%
\pgfpathlineto{\pgfqpoint{4.700034in}{1.698628in}}%
\pgfpathlineto{\pgfqpoint{4.700763in}{1.874392in}}%
\pgfpathlineto{\pgfqpoint{4.701173in}{2.165819in}}%
\pgfpathlineto{\pgfqpoint{4.701627in}{1.790778in}}%
\pgfpathlineto{\pgfqpoint{4.701855in}{1.956708in}}%
\pgfpathlineto{\pgfqpoint{4.702581in}{1.842744in}}%
\pgfpathlineto{\pgfqpoint{4.702399in}{2.085131in}}%
\pgfpathlineto{\pgfqpoint{4.702944in}{1.912258in}}%
\pgfpathlineto{\pgfqpoint{4.703351in}{2.121662in}}%
\pgfpathlineto{\pgfqpoint{4.703713in}{1.802064in}}%
\pgfpathlineto{\pgfqpoint{4.704075in}{2.006134in}}%
\pgfpathlineto{\pgfqpoint{4.704888in}{2.106754in}}%
\pgfpathlineto{\pgfqpoint{4.704527in}{1.823745in}}%
\pgfpathlineto{\pgfqpoint{4.704978in}{2.057711in}}%
\pgfpathlineto{\pgfqpoint{4.705879in}{1.802727in}}%
\pgfpathlineto{\pgfqpoint{4.705338in}{2.174296in}}%
\pgfpathlineto{\pgfqpoint{4.706104in}{1.933894in}}%
\pgfpathlineto{\pgfqpoint{4.706913in}{2.106272in}}%
\pgfpathlineto{\pgfqpoint{4.706643in}{1.671925in}}%
\pgfpathlineto{\pgfqpoint{4.707182in}{2.057482in}}%
\pgfpathlineto{\pgfqpoint{4.707496in}{1.884259in}}%
\pgfpathlineto{\pgfqpoint{4.708123in}{2.102811in}}%
\pgfpathlineto{\pgfqpoint{4.708302in}{2.014358in}}%
\pgfpathlineto{\pgfqpoint{4.708481in}{2.081676in}}%
\pgfpathlineto{\pgfqpoint{4.708526in}{1.764103in}}%
\pgfpathlineto{\pgfqpoint{4.709374in}{2.056220in}}%
\pgfpathlineto{\pgfqpoint{4.709865in}{1.760596in}}%
\pgfpathlineto{\pgfqpoint{4.710132in}{2.110676in}}%
\pgfpathlineto{\pgfqpoint{4.710489in}{1.844814in}}%
\pgfpathlineto{\pgfqpoint{4.710889in}{2.115883in}}%
\pgfpathlineto{\pgfqpoint{4.710845in}{1.806310in}}%
\pgfpathlineto{\pgfqpoint{4.711600in}{2.078461in}}%
\pgfpathlineto{\pgfqpoint{4.712088in}{1.648887in}}%
\pgfpathlineto{\pgfqpoint{4.712399in}{2.122140in}}%
\pgfpathlineto{\pgfqpoint{4.712709in}{1.994633in}}%
\pgfpathlineto{\pgfqpoint{4.713505in}{2.090878in}}%
\pgfpathlineto{\pgfqpoint{4.713417in}{1.673282in}}%
\pgfpathlineto{\pgfqpoint{4.713682in}{1.976436in}}%
\pgfpathlineto{\pgfqpoint{4.713726in}{1.757731in}}%
\pgfpathlineto{\pgfqpoint{4.714256in}{2.081705in}}%
\pgfpathlineto{\pgfqpoint{4.714785in}{1.940715in}}%
\pgfpathlineto{\pgfqpoint{4.714829in}{2.118378in}}%
\pgfpathlineto{\pgfqpoint{4.715841in}{1.855803in}}%
\pgfpathlineto{\pgfqpoint{4.715885in}{2.007936in}}%
\pgfpathlineto{\pgfqpoint{4.715929in}{1.715756in}}%
\pgfpathlineto{\pgfqpoint{4.716851in}{2.115850in}}%
\pgfpathlineto{\pgfqpoint{4.716983in}{1.948973in}}%
\pgfpathlineto{\pgfqpoint{4.717070in}{2.157870in}}%
\pgfpathlineto{\pgfqpoint{4.717640in}{1.769420in}}%
\pgfpathlineto{\pgfqpoint{4.718033in}{1.968820in}}%
\pgfpathlineto{\pgfqpoint{4.718208in}{1.670058in}}%
\pgfpathlineto{\pgfqpoint{4.718252in}{2.106529in}}%
\pgfpathlineto{\pgfqpoint{4.719125in}{1.927782in}}%
\pgfpathlineto{\pgfqpoint{4.719823in}{2.163172in}}%
\pgfpathlineto{\pgfqpoint{4.719387in}{1.764156in}}%
\pgfpathlineto{\pgfqpoint{4.720258in}{2.059574in}}%
\pgfpathlineto{\pgfqpoint{4.721170in}{1.831883in}}%
\pgfpathlineto{\pgfqpoint{4.720388in}{2.117440in}}%
\pgfpathlineto{\pgfqpoint{4.721387in}{1.988531in}}%
\pgfpathlineto{\pgfqpoint{4.722341in}{2.158294in}}%
\pgfpathlineto{\pgfqpoint{4.721864in}{1.772246in}}%
\pgfpathlineto{\pgfqpoint{4.722471in}{2.029076in}}%
\pgfpathlineto{\pgfqpoint{4.722514in}{1.817295in}}%
\pgfpathlineto{\pgfqpoint{4.723033in}{2.133667in}}%
\pgfpathlineto{\pgfqpoint{4.723551in}{1.984146in}}%
\pgfpathlineto{\pgfqpoint{4.724241in}{2.103682in}}%
\pgfpathlineto{\pgfqpoint{4.724069in}{1.774803in}}%
\pgfpathlineto{\pgfqpoint{4.724586in}{1.996774in}}%
\pgfpathlineto{\pgfqpoint{4.724930in}{1.613306in}}%
\pgfpathlineto{\pgfqpoint{4.725059in}{2.101854in}}%
\pgfpathlineto{\pgfqpoint{4.725661in}{2.066630in}}%
\pgfpathlineto{\pgfqpoint{4.726819in}{1.758702in}}%
\pgfpathlineto{\pgfqpoint{4.725919in}{2.105995in}}%
\pgfpathlineto{\pgfqpoint{4.726905in}{1.924448in}}%
\pgfpathlineto{\pgfqpoint{4.727376in}{2.123751in}}%
\pgfpathlineto{\pgfqpoint{4.726990in}{1.766792in}}%
\pgfpathlineto{\pgfqpoint{4.728017in}{1.996173in}}%
\pgfpathlineto{\pgfqpoint{4.728188in}{2.086343in}}%
\pgfpathlineto{\pgfqpoint{4.729126in}{1.574602in}}%
\pgfpathlineto{\pgfqpoint{4.729296in}{2.110920in}}%
\pgfpathlineto{\pgfqpoint{4.730232in}{1.944198in}}%
\pgfpathlineto{\pgfqpoint{4.730954in}{2.157624in}}%
\pgfpathlineto{\pgfqpoint{4.730317in}{1.758970in}}%
\pgfpathlineto{\pgfqpoint{4.731251in}{2.102064in}}%
\pgfpathlineto{\pgfqpoint{4.732055in}{1.832878in}}%
\pgfpathlineto{\pgfqpoint{4.732351in}{2.101948in}}%
\pgfpathlineto{\pgfqpoint{4.732985in}{1.705359in}}%
\pgfpathlineto{\pgfqpoint{4.733281in}{2.112232in}}%
\pgfpathlineto{\pgfqpoint{4.733534in}{1.877510in}}%
\pgfpathlineto{\pgfqpoint{4.734502in}{2.155700in}}%
\pgfpathlineto{\pgfqpoint{4.733787in}{1.650339in}}%
\pgfpathlineto{\pgfqpoint{4.734670in}{2.000206in}}%
\pgfpathlineto{\pgfqpoint{4.734755in}{1.995757in}}%
\pgfpathlineto{\pgfqpoint{4.735469in}{1.512795in}}%
\pgfpathlineto{\pgfqpoint{4.734881in}{2.136412in}}%
\pgfpathlineto{\pgfqpoint{4.735846in}{2.031746in}}%
\pgfpathlineto{\pgfqpoint{4.735888in}{2.036150in}}%
\pgfpathlineto{\pgfqpoint{4.735972in}{2.003220in}}%
\pgfpathlineto{\pgfqpoint{4.736558in}{1.760170in}}%
\pgfpathlineto{\pgfqpoint{4.736935in}{2.102691in}}%
\pgfpathlineto{\pgfqpoint{4.737019in}{2.003337in}}%
\pgfpathlineto{\pgfqpoint{4.737938in}{2.125429in}}%
\pgfpathlineto{\pgfqpoint{4.737228in}{1.722636in}}%
\pgfpathlineto{\pgfqpoint{4.737979in}{1.968784in}}%
\pgfpathlineto{\pgfqpoint{4.738396in}{1.760437in}}%
\pgfpathlineto{\pgfqpoint{4.738605in}{2.093676in}}%
\pgfpathlineto{\pgfqpoint{4.739063in}{1.887907in}}%
\pgfpathlineto{\pgfqpoint{4.739604in}{2.143176in}}%
\pgfpathlineto{\pgfqpoint{4.739562in}{1.703931in}}%
\pgfpathlineto{\pgfqpoint{4.740185in}{2.019854in}}%
\pgfpathlineto{\pgfqpoint{4.740392in}{2.115577in}}%
\pgfpathlineto{\pgfqpoint{4.740683in}{1.896505in}}%
\pgfpathlineto{\pgfqpoint{4.741221in}{2.064009in}}%
\pgfpathlineto{\pgfqpoint{4.741925in}{1.842910in}}%
\pgfpathlineto{\pgfqpoint{4.741842in}{2.118138in}}%
\pgfpathlineto{\pgfqpoint{4.742338in}{2.036129in}}%
\pgfpathlineto{\pgfqpoint{4.742420in}{1.536116in}}%
\pgfpathlineto{\pgfqpoint{4.742751in}{2.120747in}}%
\pgfpathlineto{\pgfqpoint{4.743493in}{1.992836in}}%
\pgfpathlineto{\pgfqpoint{4.744274in}{1.832426in}}%
\pgfpathlineto{\pgfqpoint{4.744439in}{2.087206in}}%
\pgfpathlineto{\pgfqpoint{4.745424in}{1.835472in}}%
\pgfpathlineto{\pgfqpoint{4.745137in}{2.155954in}}%
\pgfpathlineto{\pgfqpoint{4.745547in}{1.925205in}}%
\pgfpathlineto{\pgfqpoint{4.745997in}{2.105016in}}%
\pgfpathlineto{\pgfqpoint{4.746079in}{1.799881in}}%
\pgfpathlineto{\pgfqpoint{4.746652in}{2.027186in}}%
\pgfpathlineto{\pgfqpoint{4.746938in}{2.116113in}}%
\pgfpathlineto{\pgfqpoint{4.746815in}{1.723417in}}%
\pgfpathlineto{\pgfqpoint{4.746979in}{2.081012in}}%
\pgfpathlineto{\pgfqpoint{4.747183in}{1.660981in}}%
\pgfpathlineto{\pgfqpoint{4.747673in}{2.161057in}}%
\pgfpathlineto{\pgfqpoint{4.748080in}{2.051459in}}%
\pgfpathlineto{\pgfqpoint{4.748162in}{2.139110in}}%
\pgfpathlineto{\pgfqpoint{4.748406in}{1.844195in}}%
\pgfpathlineto{\pgfqpoint{4.748976in}{1.856855in}}%
\pgfpathlineto{\pgfqpoint{4.749016in}{1.769607in}}%
\pgfpathlineto{\pgfqpoint{4.749057in}{2.108194in}}%
\pgfpathlineto{\pgfqpoint{4.749991in}{1.884898in}}%
\pgfpathlineto{\pgfqpoint{4.750437in}{2.124861in}}%
\pgfpathlineto{\pgfqpoint{4.750518in}{1.700219in}}%
\pgfpathlineto{\pgfqpoint{4.751125in}{1.996354in}}%
\pgfpathlineto{\pgfqpoint{4.752014in}{2.130134in}}%
\pgfpathlineto{\pgfqpoint{4.751772in}{1.856418in}}%
\pgfpathlineto{\pgfqpoint{4.752296in}{2.115390in}}%
\pgfpathlineto{\pgfqpoint{4.752861in}{1.810904in}}%
\pgfpathlineto{\pgfqpoint{4.753465in}{1.958518in}}%
\pgfpathlineto{\pgfqpoint{4.754429in}{2.111435in}}%
\pgfpathlineto{\pgfqpoint{4.753827in}{1.639856in}}%
\pgfpathlineto{\pgfqpoint{4.754549in}{2.068985in}}%
\pgfpathlineto{\pgfqpoint{4.755391in}{1.712509in}}%
\pgfpathlineto{\pgfqpoint{4.754750in}{2.178652in}}%
\pgfpathlineto{\pgfqpoint{4.755671in}{1.754220in}}%
\pgfpathlineto{\pgfqpoint{4.756311in}{2.139399in}}%
\pgfpathlineto{\pgfqpoint{4.756790in}{2.029086in}}%
\pgfpathlineto{\pgfqpoint{4.756830in}{1.704444in}}%
\pgfpathlineto{\pgfqpoint{4.756950in}{2.083243in}}%
\pgfpathlineto{\pgfqpoint{4.757907in}{1.910500in}}%
\pgfpathlineto{\pgfqpoint{4.758583in}{2.153187in}}%
\pgfpathlineto{\pgfqpoint{4.758026in}{1.782692in}}%
\pgfpathlineto{\pgfqpoint{4.759059in}{2.005787in}}%
\pgfpathlineto{\pgfqpoint{4.759179in}{1.888005in}}%
\pgfpathlineto{\pgfqpoint{4.759615in}{2.097487in}}%
\pgfpathlineto{\pgfqpoint{4.759813in}{2.012996in}}%
\pgfpathlineto{\pgfqpoint{4.759853in}{2.113455in}}%
\pgfpathlineto{\pgfqpoint{4.759932in}{1.752708in}}%
\pgfpathlineto{\pgfqpoint{4.760921in}{2.037675in}}%
\pgfpathlineto{\pgfqpoint{4.761474in}{1.808119in}}%
\pgfpathlineto{\pgfqpoint{4.761869in}{2.128229in}}%
\pgfpathlineto{\pgfqpoint{4.762066in}{1.945550in}}%
\pgfpathlineto{\pgfqpoint{4.762303in}{1.816925in}}%
\pgfpathlineto{\pgfqpoint{4.762263in}{2.107225in}}%
\pgfpathlineto{\pgfqpoint{4.763011in}{2.026460in}}%
\pgfpathlineto{\pgfqpoint{4.763050in}{2.133087in}}%
\pgfpathlineto{\pgfqpoint{4.763915in}{1.681153in}}%
\pgfpathlineto{\pgfqpoint{4.764072in}{2.071868in}}%
\pgfpathlineto{\pgfqpoint{4.765012in}{1.869198in}}%
\pgfpathlineto{\pgfqpoint{4.765208in}{1.943863in}}%
\pgfpathlineto{\pgfqpoint{4.765912in}{2.107756in}}%
\pgfpathlineto{\pgfqpoint{4.765951in}{1.843859in}}%
\pgfpathlineto{\pgfqpoint{4.766263in}{2.103117in}}%
\pgfpathlineto{\pgfqpoint{4.767160in}{1.810632in}}%
\pgfpathlineto{\pgfqpoint{4.766965in}{2.128322in}}%
\pgfpathlineto{\pgfqpoint{4.767394in}{1.994412in}}%
\pgfpathlineto{\pgfqpoint{4.767977in}{2.106751in}}%
\pgfpathlineto{\pgfqpoint{4.767705in}{1.681455in}}%
\pgfpathlineto{\pgfqpoint{4.768443in}{2.058676in}}%
\pgfpathlineto{\pgfqpoint{4.768870in}{1.836801in}}%
\pgfpathlineto{\pgfqpoint{4.768754in}{2.154615in}}%
\pgfpathlineto{\pgfqpoint{4.769568in}{1.932475in}}%
\pgfpathlineto{\pgfqpoint{4.769762in}{2.117816in}}%
\pgfpathlineto{\pgfqpoint{4.770265in}{1.746674in}}%
\pgfpathlineto{\pgfqpoint{4.770690in}{2.039898in}}%
\pgfpathlineto{\pgfqpoint{4.770806in}{1.786490in}}%
\pgfpathlineto{\pgfqpoint{4.770922in}{2.132963in}}%
\pgfpathlineto{\pgfqpoint{4.771886in}{1.890237in}}%
\pgfpathlineto{\pgfqpoint{4.772194in}{2.117327in}}%
\pgfpathlineto{\pgfqpoint{4.772771in}{1.557488in}}%
\pgfpathlineto{\pgfqpoint{4.773001in}{1.958793in}}%
\pgfpathlineto{\pgfqpoint{4.773846in}{1.716276in}}%
\pgfpathlineto{\pgfqpoint{4.773462in}{2.132303in}}%
\pgfpathlineto{\pgfqpoint{4.774037in}{1.824323in}}%
\pgfpathlineto{\pgfqpoint{4.775071in}{2.112461in}}%
\pgfpathlineto{\pgfqpoint{4.774803in}{1.774338in}}%
\pgfpathlineto{\pgfqpoint{4.775185in}{2.054985in}}%
\pgfpathlineto{\pgfqpoint{4.775720in}{1.593722in}}%
\pgfpathlineto{\pgfqpoint{4.776025in}{2.089292in}}%
\pgfpathlineto{\pgfqpoint{4.776407in}{2.019809in}}%
\pgfpathlineto{\pgfqpoint{4.776940in}{1.865738in}}%
\pgfpathlineto{\pgfqpoint{4.776749in}{2.040394in}}%
\pgfpathlineto{\pgfqpoint{4.777244in}{2.017016in}}%
\pgfpathlineto{\pgfqpoint{4.777434in}{2.097014in}}%
\pgfpathlineto{\pgfqpoint{4.778042in}{1.659555in}}%
\pgfpathlineto{\pgfqpoint{4.778156in}{1.961406in}}%
\pgfpathlineto{\pgfqpoint{4.778194in}{1.735739in}}%
\pgfpathlineto{\pgfqpoint{4.778801in}{2.068874in}}%
\pgfpathlineto{\pgfqpoint{4.779255in}{1.986958in}}%
\pgfpathlineto{\pgfqpoint{4.780087in}{2.117520in}}%
\pgfpathlineto{\pgfqpoint{4.779634in}{1.781036in}}%
\pgfpathlineto{\pgfqpoint{4.780352in}{1.959293in}}%
\pgfpathlineto{\pgfqpoint{4.780918in}{1.714470in}}%
\pgfpathlineto{\pgfqpoint{4.780805in}{2.106557in}}%
\pgfpathlineto{\pgfqpoint{4.781370in}{1.937844in}}%
\pgfpathlineto{\pgfqpoint{4.781483in}{1.743782in}}%
\pgfpathlineto{\pgfqpoint{4.782499in}{2.098421in}}%
\pgfpathlineto{\pgfqpoint{4.782724in}{1.553739in}}%
\pgfpathlineto{\pgfqpoint{4.783699in}{1.869167in}}%
\pgfpathlineto{\pgfqpoint{4.784634in}{2.129333in}}%
\pgfpathlineto{\pgfqpoint{4.784298in}{1.792463in}}%
\pgfpathlineto{\pgfqpoint{4.784821in}{2.003538in}}%
\pgfpathlineto{\pgfqpoint{4.785381in}{2.088324in}}%
\pgfpathlineto{\pgfqpoint{4.784933in}{1.658745in}}%
\pgfpathlineto{\pgfqpoint{4.785941in}{2.063482in}}%
\pgfpathlineto{\pgfqpoint{4.786871in}{1.776586in}}%
\pgfpathlineto{\pgfqpoint{4.786350in}{2.122351in}}%
\pgfpathlineto{\pgfqpoint{4.787057in}{2.019057in}}%
\pgfpathlineto{\pgfqpoint{4.787800in}{1.768831in}}%
\pgfpathlineto{\pgfqpoint{4.788097in}{2.101470in}}%
\pgfpathlineto{\pgfqpoint{4.789133in}{1.687751in}}%
\pgfpathlineto{\pgfqpoint{4.788652in}{2.109841in}}%
\pgfpathlineto{\pgfqpoint{4.789207in}{1.938857in}}%
\pgfpathlineto{\pgfqpoint{4.789799in}{1.709937in}}%
\pgfpathlineto{\pgfqpoint{4.790315in}{2.064907in}}%
\pgfpathlineto{\pgfqpoint{4.791089in}{1.557537in}}%
\pgfpathlineto{\pgfqpoint{4.790463in}{2.113694in}}%
\pgfpathlineto{\pgfqpoint{4.791420in}{2.022881in}}%
\pgfpathlineto{\pgfqpoint{4.792045in}{1.831847in}}%
\pgfpathlineto{\pgfqpoint{4.791861in}{2.077940in}}%
\pgfpathlineto{\pgfqpoint{4.792339in}{1.985409in}}%
\pgfpathlineto{\pgfqpoint{4.792559in}{2.137515in}}%
\pgfpathlineto{\pgfqpoint{4.793182in}{1.550297in}}%
\pgfpathlineto{\pgfqpoint{4.793439in}{1.983270in}}%
\pgfpathlineto{\pgfqpoint{4.794207in}{1.801594in}}%
\pgfpathlineto{\pgfqpoint{4.794024in}{2.119462in}}%
\pgfpathlineto{\pgfqpoint{4.794353in}{2.009858in}}%
\pgfpathlineto{\pgfqpoint{4.794390in}{2.110749in}}%
\pgfpathlineto{\pgfqpoint{4.794937in}{1.799027in}}%
\pgfpathlineto{\pgfqpoint{4.795338in}{1.960276in}}%
\pgfpathlineto{\pgfqpoint{4.795375in}{1.668518in}}%
\pgfpathlineto{\pgfqpoint{4.795666in}{2.100383in}}%
\pgfpathlineto{\pgfqpoint{4.796430in}{1.972477in}}%
\pgfpathlineto{\pgfqpoint{4.797411in}{1.717961in}}%
\pgfpathlineto{\pgfqpoint{4.797157in}{2.075523in}}%
\pgfpathlineto{\pgfqpoint{4.797519in}{1.847607in}}%
\pgfpathlineto{\pgfqpoint{4.798425in}{2.133016in}}%
\pgfpathlineto{\pgfqpoint{4.798208in}{1.703123in}}%
\pgfpathlineto{\pgfqpoint{4.798606in}{1.827666in}}%
\pgfpathlineto{\pgfqpoint{4.798642in}{1.687479in}}%
\pgfpathlineto{\pgfqpoint{4.799401in}{2.132004in}}%
\pgfpathlineto{\pgfqpoint{4.799690in}{1.854404in}}%
\pgfpathlineto{\pgfqpoint{4.800374in}{1.609567in}}%
\pgfpathlineto{\pgfqpoint{4.800555in}{2.064242in}}%
\pgfpathlineto{\pgfqpoint{4.800663in}{2.002361in}}%
\pgfpathlineto{\pgfqpoint{4.801022in}{2.103107in}}%
\pgfpathlineto{\pgfqpoint{4.800735in}{1.819621in}}%
\pgfpathlineto{\pgfqpoint{4.801633in}{2.056453in}}%
\pgfpathlineto{\pgfqpoint{4.802423in}{1.615681in}}%
\pgfpathlineto{\pgfqpoint{4.801777in}{2.101641in}}%
\pgfpathlineto{\pgfqpoint{4.802745in}{2.023640in}}%
\pgfpathlineto{\pgfqpoint{4.803461in}{1.730996in}}%
\pgfpathlineto{\pgfqpoint{4.803675in}{2.084357in}}%
\pgfpathlineto{\pgfqpoint{4.803854in}{1.940224in}}%
\pgfpathlineto{\pgfqpoint{4.804389in}{1.754163in}}%
\pgfpathlineto{\pgfqpoint{4.804996in}{2.155871in}}%
\pgfpathlineto{\pgfqpoint{4.805885in}{1.727313in}}%
\pgfpathlineto{\pgfqpoint{4.806134in}{1.793144in}}%
\pgfpathlineto{\pgfqpoint{4.807057in}{2.111055in}}%
\pgfpathlineto{\pgfqpoint{4.806418in}{1.330659in}}%
\pgfpathlineto{\pgfqpoint{4.807199in}{1.803722in}}%
\pgfpathlineto{\pgfqpoint{4.807836in}{1.551485in}}%
\pgfpathlineto{\pgfqpoint{4.807411in}{2.091613in}}%
\pgfpathlineto{\pgfqpoint{4.808190in}{1.888584in}}%
\pgfpathlineto{\pgfqpoint{4.808755in}{2.125098in}}%
\pgfpathlineto{\pgfqpoint{4.808649in}{1.605414in}}%
\pgfpathlineto{\pgfqpoint{4.809285in}{2.047155in}}%
\pgfpathlineto{\pgfqpoint{4.809567in}{1.715398in}}%
\pgfpathlineto{\pgfqpoint{4.810236in}{2.081604in}}%
\pgfpathlineto{\pgfqpoint{4.810412in}{1.841463in}}%
\pgfpathlineto{\pgfqpoint{4.810447in}{2.083087in}}%
\pgfpathlineto{\pgfqpoint{4.810588in}{1.366659in}}%
\pgfpathlineto{\pgfqpoint{4.811536in}{2.014001in}}%
\pgfpathlineto{\pgfqpoint{4.812517in}{1.718007in}}%
\pgfpathlineto{\pgfqpoint{4.812482in}{2.071507in}}%
\pgfpathlineto{\pgfqpoint{4.812657in}{1.835643in}}%
\pgfpathlineto{\pgfqpoint{4.813182in}{2.047458in}}%
\pgfpathlineto{\pgfqpoint{4.813531in}{1.679494in}}%
\pgfpathlineto{\pgfqpoint{4.813741in}{1.870425in}}%
\pgfpathlineto{\pgfqpoint{4.813810in}{1.465394in}}%
\pgfpathlineto{\pgfqpoint{4.814194in}{2.094899in}}%
\pgfpathlineto{\pgfqpoint{4.814821in}{1.921748in}}%
\pgfpathlineto{\pgfqpoint{4.814891in}{2.042135in}}%
\pgfpathlineto{\pgfqpoint{4.815099in}{1.676722in}}%
\pgfpathlineto{\pgfqpoint{4.815134in}{1.528855in}}%
\pgfpathlineto{\pgfqpoint{4.815656in}{2.086898in}}%
\pgfpathlineto{\pgfqpoint{4.816107in}{1.933145in}}%
\pgfpathlineto{\pgfqpoint{4.816419in}{2.075087in}}%
\pgfpathlineto{\pgfqpoint{4.817043in}{1.773200in}}%
\pgfpathlineto{\pgfqpoint{4.817182in}{1.902963in}}%
\pgfpathlineto{\pgfqpoint{4.817943in}{2.137755in}}%
\pgfpathlineto{\pgfqpoint{4.817874in}{1.777357in}}%
\pgfpathlineto{\pgfqpoint{4.818323in}{2.038822in}}%
\pgfpathlineto{\pgfqpoint{4.818806in}{1.738841in}}%
\pgfpathlineto{\pgfqpoint{4.818426in}{2.088751in}}%
\pgfpathlineto{\pgfqpoint{4.819460in}{1.842579in}}%
\pgfpathlineto{\pgfqpoint{4.820561in}{2.115869in}}%
\pgfpathlineto{\pgfqpoint{4.820080in}{1.659721in}}%
\pgfpathlineto{\pgfqpoint{4.820595in}{2.086918in}}%
\pgfpathlineto{\pgfqpoint{4.821213in}{1.672441in}}%
\pgfpathlineto{\pgfqpoint{4.821693in}{1.926577in}}%
\pgfpathlineto{\pgfqpoint{4.822275in}{2.091365in}}%
\pgfpathlineto{\pgfqpoint{4.822035in}{1.667840in}}%
\pgfpathlineto{\pgfqpoint{4.822753in}{2.018094in}}%
\pgfpathlineto{\pgfqpoint{4.822822in}{1.836068in}}%
\pgfpathlineto{\pgfqpoint{4.823300in}{2.091861in}}%
\pgfpathlineto{\pgfqpoint{4.823845in}{1.959584in}}%
\pgfpathlineto{\pgfqpoint{4.824288in}{2.087626in}}%
\pgfpathlineto{\pgfqpoint{4.824594in}{1.603164in}}%
\pgfpathlineto{\pgfqpoint{4.824968in}{2.042910in}}%
\pgfpathlineto{\pgfqpoint{4.825206in}{1.683534in}}%
\pgfpathlineto{\pgfqpoint{4.825376in}{2.120989in}}%
\pgfpathlineto{\pgfqpoint{4.826055in}{1.768210in}}%
\pgfpathlineto{\pgfqpoint{4.826258in}{2.096279in}}%
\pgfpathlineto{\pgfqpoint{4.826732in}{1.623558in}}%
\pgfpathlineto{\pgfqpoint{4.827172in}{1.879110in}}%
\pgfpathlineto{\pgfqpoint{4.828286in}{2.087085in}}%
\pgfpathlineto{\pgfqpoint{4.827240in}{1.717278in}}%
\pgfpathlineto{\pgfqpoint{4.828320in}{2.023921in}}%
\pgfpathlineto{\pgfqpoint{4.828354in}{2.016441in}}%
\pgfpathlineto{\pgfqpoint{4.828388in}{1.365675in}}%
\pgfpathlineto{\pgfqpoint{4.828758in}{2.089437in}}%
\pgfpathlineto{\pgfqpoint{4.829465in}{1.935072in}}%
\pgfpathlineto{\pgfqpoint{4.830104in}{2.105125in}}%
\pgfpathlineto{\pgfqpoint{4.829801in}{1.794250in}}%
\pgfpathlineto{\pgfqpoint{4.830506in}{2.047822in}}%
\pgfpathlineto{\pgfqpoint{4.830540in}{1.715709in}}%
\pgfpathlineto{\pgfqpoint{4.830574in}{2.130105in}}%
\pgfpathlineto{\pgfqpoint{4.831612in}{2.047416in}}%
\pgfpathlineto{\pgfqpoint{4.831947in}{1.769659in}}%
\pgfpathlineto{\pgfqpoint{4.832013in}{2.086391in}}%
\pgfpathlineto{\pgfqpoint{4.832782in}{1.865511in}}%
\pgfpathlineto{\pgfqpoint{4.833215in}{2.093571in}}%
\pgfpathlineto{\pgfqpoint{4.832915in}{1.731762in}}%
\pgfpathlineto{\pgfqpoint{4.833915in}{1.996245in}}%
\pgfpathlineto{\pgfqpoint{4.833948in}{2.113477in}}%
\pgfpathlineto{\pgfqpoint{4.834680in}{1.826596in}}%
\pgfpathlineto{\pgfqpoint{4.834978in}{2.070780in}}%
\pgfpathlineto{\pgfqpoint{4.835144in}{1.731735in}}%
\pgfpathlineto{\pgfqpoint{4.835940in}{2.088135in}}%
\pgfpathlineto{\pgfqpoint{4.836106in}{1.940296in}}%
\pgfpathlineto{\pgfqpoint{4.836800in}{2.070683in}}%
\pgfpathlineto{\pgfqpoint{4.836899in}{1.804754in}}%
\pgfpathlineto{\pgfqpoint{4.837164in}{1.996775in}}%
\pgfpathlineto{\pgfqpoint{4.837626in}{1.555507in}}%
\pgfpathlineto{\pgfqpoint{4.838120in}{2.096184in}}%
\pgfpathlineto{\pgfqpoint{4.838285in}{1.899618in}}%
\pgfpathlineto{\pgfqpoint{4.838450in}{2.116716in}}%
\pgfpathlineto{\pgfqpoint{4.838483in}{1.820491in}}%
\pgfpathlineto{\pgfqpoint{4.838516in}{1.601605in}}%
\pgfpathlineto{\pgfqpoint{4.838614in}{2.080664in}}%
\pgfpathlineto{\pgfqpoint{4.839568in}{1.864599in}}%
\pgfpathlineto{\pgfqpoint{4.840060in}{2.111770in}}%
\pgfpathlineto{\pgfqpoint{4.839995in}{1.559023in}}%
\pgfpathlineto{\pgfqpoint{4.840716in}{2.000610in}}%
\pgfpathlineto{\pgfqpoint{4.840749in}{2.042496in}}%
\pgfpathlineto{\pgfqpoint{4.841338in}{1.759869in}}%
\pgfpathlineto{\pgfqpoint{4.841665in}{1.945360in}}%
\pgfpathlineto{\pgfqpoint{4.842513in}{1.748665in}}%
\pgfpathlineto{\pgfqpoint{4.842415in}{2.091351in}}%
\pgfpathlineto{\pgfqpoint{4.842774in}{1.929607in}}%
\pgfpathlineto{\pgfqpoint{4.843393in}{2.128897in}}%
\pgfpathlineto{\pgfqpoint{4.843588in}{1.442489in}}%
\pgfpathlineto{\pgfqpoint{4.843783in}{1.965037in}}%
\pgfpathlineto{\pgfqpoint{4.843816in}{1.737159in}}%
\pgfpathlineto{\pgfqpoint{4.844725in}{2.066815in}}%
\pgfpathlineto{\pgfqpoint{4.844887in}{1.891247in}}%
\pgfpathlineto{\pgfqpoint{4.845244in}{1.700404in}}%
\pgfpathlineto{\pgfqpoint{4.845082in}{2.089003in}}%
\pgfpathlineto{\pgfqpoint{4.845956in}{1.978045in}}%
\pgfpathlineto{\pgfqpoint{4.846086in}{1.892935in}}%
\pgfpathlineto{\pgfqpoint{4.846118in}{1.704403in}}%
\pgfpathlineto{\pgfqpoint{4.846797in}{2.099738in}}%
\pgfpathlineto{\pgfqpoint{4.847152in}{1.967035in}}%
\pgfpathlineto{\pgfqpoint{4.847925in}{2.113031in}}%
\pgfpathlineto{\pgfqpoint{4.847345in}{1.757299in}}%
\pgfpathlineto{\pgfqpoint{4.848247in}{1.945970in}}%
\pgfpathlineto{\pgfqpoint{4.848311in}{1.770149in}}%
\pgfpathlineto{\pgfqpoint{4.848986in}{2.115262in}}%
\pgfpathlineto{\pgfqpoint{4.849275in}{1.983737in}}%
\pgfpathlineto{\pgfqpoint{4.849436in}{1.862186in}}%
\pgfpathlineto{\pgfqpoint{4.849500in}{2.000968in}}%
\pgfpathlineto{\pgfqpoint{4.849885in}{1.566178in}}%
\pgfpathlineto{\pgfqpoint{4.849981in}{2.073540in}}%
\pgfpathlineto{\pgfqpoint{4.850621in}{1.871394in}}%
\pgfpathlineto{\pgfqpoint{4.850781in}{1.781144in}}%
\pgfpathlineto{\pgfqpoint{4.851740in}{2.093014in}}%
\pgfpathlineto{\pgfqpoint{4.852027in}{1.538547in}}%
\pgfpathlineto{\pgfqpoint{4.852887in}{1.822299in}}%
\pgfpathlineto{\pgfqpoint{4.853523in}{2.094203in}}%
\pgfpathlineto{\pgfqpoint{4.853936in}{1.716634in}}%
\pgfpathlineto{\pgfqpoint{4.854031in}{2.001522in}}%
\pgfpathlineto{\pgfqpoint{4.854444in}{2.038115in}}%
\pgfpathlineto{\pgfqpoint{4.854602in}{1.779872in}}%
\pgfpathlineto{\pgfqpoint{4.854919in}{1.982632in}}%
\pgfpathlineto{\pgfqpoint{4.854951in}{1.497050in}}%
\pgfpathlineto{\pgfqpoint{4.855204in}{2.098070in}}%
\pgfpathlineto{\pgfqpoint{4.856026in}{1.977792in}}%
\pgfpathlineto{\pgfqpoint{4.856878in}{1.776059in}}%
\pgfpathlineto{\pgfqpoint{4.856500in}{2.103858in}}%
\pgfpathlineto{\pgfqpoint{4.857130in}{1.993469in}}%
\pgfpathlineto{\pgfqpoint{4.857791in}{1.725269in}}%
\pgfpathlineto{\pgfqpoint{4.857666in}{2.075731in}}%
\pgfpathlineto{\pgfqpoint{4.858263in}{1.783632in}}%
\pgfpathlineto{\pgfqpoint{4.859330in}{2.096618in}}%
\pgfpathlineto{\pgfqpoint{4.858546in}{1.780549in}}%
\pgfpathlineto{\pgfqpoint{4.859393in}{2.037381in}}%
\pgfpathlineto{\pgfqpoint{4.860051in}{1.636446in}}%
\pgfpathlineto{\pgfqpoint{4.859769in}{2.094430in}}%
\pgfpathlineto{\pgfqpoint{4.860520in}{1.914400in}}%
\pgfpathlineto{\pgfqpoint{4.860738in}{2.077320in}}%
\pgfpathlineto{\pgfqpoint{4.861082in}{1.700973in}}%
\pgfpathlineto{\pgfqpoint{4.861612in}{1.964569in}}%
\pgfpathlineto{\pgfqpoint{4.861830in}{1.698878in}}%
\pgfpathlineto{\pgfqpoint{4.862049in}{2.080994in}}%
\pgfpathlineto{\pgfqpoint{4.862764in}{1.800828in}}%
\pgfpathlineto{\pgfqpoint{4.863603in}{2.081351in}}%
\pgfpathlineto{\pgfqpoint{4.863044in}{1.756244in}}%
\pgfpathlineto{\pgfqpoint{4.863913in}{1.981940in}}%
\pgfpathlineto{\pgfqpoint{4.864316in}{2.115287in}}%
\pgfpathlineto{\pgfqpoint{4.864285in}{1.756353in}}%
\pgfpathlineto{\pgfqpoint{4.864719in}{2.017748in}}%
\pgfpathlineto{\pgfqpoint{4.865090in}{1.643162in}}%
\pgfpathlineto{\pgfqpoint{4.865121in}{2.079004in}}%
\pgfpathlineto{\pgfqpoint{4.865801in}{1.809324in}}%
\pgfpathlineto{\pgfqpoint{4.866017in}{2.080392in}}%
\pgfpathlineto{\pgfqpoint{4.866880in}{1.643649in}}%
\pgfpathlineto{\pgfqpoint{4.866910in}{1.862225in}}%
\pgfpathlineto{\pgfqpoint{4.867526in}{2.066482in}}%
\pgfpathlineto{\pgfqpoint{4.867895in}{1.632097in}}%
\pgfpathlineto{\pgfqpoint{4.868018in}{1.990478in}}%
\pgfpathlineto{\pgfqpoint{4.868417in}{1.616015in}}%
\pgfpathlineto{\pgfqpoint{4.868509in}{2.063904in}}%
\pgfpathlineto{\pgfqpoint{4.869152in}{1.693252in}}%
\pgfpathlineto{\pgfqpoint{4.869673in}{2.076406in}}%
\pgfpathlineto{\pgfqpoint{4.869887in}{1.691237in}}%
\pgfpathlineto{\pgfqpoint{4.870254in}{1.826351in}}%
\pgfpathlineto{\pgfqpoint{4.870681in}{2.094932in}}%
\pgfpathlineto{\pgfqpoint{4.870315in}{1.594339in}}%
\pgfpathlineto{\pgfqpoint{4.871382in}{2.039645in}}%
\pgfpathlineto{\pgfqpoint{4.871443in}{1.625229in}}%
\pgfpathlineto{\pgfqpoint{4.871748in}{2.090116in}}%
\pgfpathlineto{\pgfqpoint{4.872508in}{1.911093in}}%
\pgfpathlineto{\pgfqpoint{4.873510in}{1.302844in}}%
\pgfpathlineto{\pgfqpoint{4.872782in}{2.068378in}}%
\pgfpathlineto{\pgfqpoint{4.873570in}{1.718378in}}%
\pgfpathlineto{\pgfqpoint{4.874570in}{2.066116in}}%
\pgfpathlineto{\pgfqpoint{4.874690in}{1.821227in}}%
\pgfpathlineto{\pgfqpoint{4.875295in}{2.062372in}}%
\pgfpathlineto{\pgfqpoint{4.875446in}{1.724790in}}%
\pgfpathlineto{\pgfqpoint{4.875808in}{1.865457in}}%
\pgfpathlineto{\pgfqpoint{4.876290in}{2.076438in}}%
\pgfpathlineto{\pgfqpoint{4.876862in}{1.954201in}}%
\pgfpathlineto{\pgfqpoint{4.876892in}{1.643492in}}%
\pgfpathlineto{\pgfqpoint{4.876982in}{2.091154in}}%
\pgfpathlineto{\pgfqpoint{4.877973in}{1.838862in}}%
\pgfpathlineto{\pgfqpoint{4.878603in}{1.704788in}}%
\pgfpathlineto{\pgfqpoint{4.878153in}{2.046292in}}%
\pgfpathlineto{\pgfqpoint{4.878782in}{1.921610in}}%
\pgfpathlineto{\pgfqpoint{4.878812in}{2.075909in}}%
\pgfpathlineto{\pgfqpoint{4.878932in}{1.721573in}}%
\pgfpathlineto{\pgfqpoint{4.879889in}{1.902159in}}%
\pgfpathlineto{\pgfqpoint{4.880784in}{1.567208in}}%
\pgfpathlineto{\pgfqpoint{4.880068in}{2.044308in}}%
\pgfpathlineto{\pgfqpoint{4.880813in}{1.939520in}}%
\pgfpathlineto{\pgfqpoint{4.881796in}{2.082909in}}%
\pgfpathlineto{\pgfqpoint{4.881260in}{1.672983in}}%
\pgfpathlineto{\pgfqpoint{4.881914in}{2.032344in}}%
\pgfpathlineto{\pgfqpoint{4.882449in}{1.757767in}}%
\pgfpathlineto{\pgfqpoint{4.882390in}{2.075350in}}%
\pgfpathlineto{\pgfqpoint{4.883072in}{1.779171in}}%
\pgfpathlineto{\pgfqpoint{4.883368in}{2.022173in}}%
\pgfpathlineto{\pgfqpoint{4.884167in}{1.766001in}}%
\pgfpathlineto{\pgfqpoint{4.884197in}{1.982881in}}%
\pgfpathlineto{\pgfqpoint{4.884758in}{1.799048in}}%
\pgfpathlineto{\pgfqpoint{4.884906in}{2.064728in}}%
\pgfpathlineto{\pgfqpoint{4.885260in}{1.960724in}}%
\pgfpathlineto{\pgfqpoint{4.886085in}{2.069482in}}%
\pgfpathlineto{\pgfqpoint{4.886173in}{1.687820in}}%
\pgfpathlineto{\pgfqpoint{4.886320in}{1.850196in}}%
\pgfpathlineto{\pgfqpoint{4.887407in}{2.075078in}}%
\pgfpathlineto{\pgfqpoint{4.887055in}{1.606234in}}%
\pgfpathlineto{\pgfqpoint{4.887466in}{1.998755in}}%
\pgfpathlineto{\pgfqpoint{4.887905in}{1.446649in}}%
\pgfpathlineto{\pgfqpoint{4.888052in}{2.099845in}}%
\pgfpathlineto{\pgfqpoint{4.888550in}{1.797961in}}%
\pgfpathlineto{\pgfqpoint{4.888579in}{2.058722in}}%
\pgfpathlineto{\pgfqpoint{4.889281in}{1.625119in}}%
\pgfpathlineto{\pgfqpoint{4.889660in}{1.910025in}}%
\pgfpathlineto{\pgfqpoint{4.890127in}{2.028428in}}%
\pgfpathlineto{\pgfqpoint{4.890447in}{1.729172in}}%
\pgfpathlineto{\pgfqpoint{4.890739in}{2.019607in}}%
\pgfpathlineto{\pgfqpoint{4.891727in}{1.703192in}}%
\pgfpathlineto{\pgfqpoint{4.891262in}{2.049848in}}%
\pgfpathlineto{\pgfqpoint{4.891843in}{1.982811in}}%
\pgfpathlineto{\pgfqpoint{4.892540in}{2.067340in}}%
\pgfpathlineto{\pgfqpoint{4.892104in}{1.420346in}}%
\pgfpathlineto{\pgfqpoint{4.892800in}{1.933675in}}%
\pgfpathlineto{\pgfqpoint{4.893032in}{1.674149in}}%
\pgfpathlineto{\pgfqpoint{4.893755in}{2.045941in}}%
\pgfpathlineto{\pgfqpoint{4.893871in}{1.929114in}}%
\pgfpathlineto{\pgfqpoint{4.894881in}{2.110472in}}%
\pgfpathlineto{\pgfqpoint{4.893986in}{1.774225in}}%
\pgfpathlineto{\pgfqpoint{4.894996in}{1.990712in}}%
\pgfpathlineto{\pgfqpoint{4.895889in}{1.669783in}}%
\pgfpathlineto{\pgfqpoint{4.895860in}{2.071041in}}%
\pgfpathlineto{\pgfqpoint{4.896090in}{1.970685in}}%
\pgfpathlineto{\pgfqpoint{4.896406in}{2.073729in}}%
\pgfpathlineto{\pgfqpoint{4.896435in}{1.643099in}}%
\pgfpathlineto{\pgfqpoint{4.897124in}{1.911755in}}%
\pgfpathlineto{\pgfqpoint{4.897210in}{1.626127in}}%
\pgfpathlineto{\pgfqpoint{4.897181in}{2.062590in}}%
\pgfpathlineto{\pgfqpoint{4.898212in}{1.895740in}}%
\pgfpathlineto{\pgfqpoint{4.898526in}{2.067410in}}%
\pgfpathlineto{\pgfqpoint{4.898898in}{1.768448in}}%
\pgfpathlineto{\pgfqpoint{4.899297in}{1.978274in}}%
\pgfpathlineto{\pgfqpoint{4.899896in}{1.650553in}}%
\pgfpathlineto{\pgfqpoint{4.899639in}{2.085231in}}%
\pgfpathlineto{\pgfqpoint{4.900380in}{1.813217in}}%
\pgfpathlineto{\pgfqpoint{4.900920in}{2.042326in}}%
\pgfpathlineto{\pgfqpoint{4.901006in}{1.707729in}}%
\pgfpathlineto{\pgfqpoint{4.901488in}{1.994030in}}%
\pgfpathlineto{\pgfqpoint{4.901857in}{2.106957in}}%
\pgfpathlineto{\pgfqpoint{4.902594in}{1.536812in}}%
\pgfpathlineto{\pgfqpoint{4.902877in}{2.057920in}}%
\pgfpathlineto{\pgfqpoint{4.903696in}{1.903918in}}%
\pgfpathlineto{\pgfqpoint{4.904430in}{1.678733in}}%
\pgfpathlineto{\pgfqpoint{4.904486in}{2.061099in}}%
\pgfpathlineto{\pgfqpoint{4.904740in}{1.947538in}}%
\pgfpathlineto{\pgfqpoint{4.905134in}{2.076745in}}%
\pgfpathlineto{\pgfqpoint{4.905218in}{1.727013in}}%
\pgfpathlineto{\pgfqpoint{4.905865in}{2.027842in}}%
\pgfpathlineto{\pgfqpoint{4.907043in}{1.416383in}}%
\pgfpathlineto{\pgfqpoint{4.905921in}{2.082686in}}%
\pgfpathlineto{\pgfqpoint{4.907071in}{1.753519in}}%
\pgfpathlineto{\pgfqpoint{4.907938in}{2.063434in}}%
\pgfpathlineto{\pgfqpoint{4.907575in}{1.617086in}}%
\pgfpathlineto{\pgfqpoint{4.908218in}{2.009200in}}%
\pgfpathlineto{\pgfqpoint{4.909278in}{1.628986in}}%
\pgfpathlineto{\pgfqpoint{4.908358in}{2.055288in}}%
\pgfpathlineto{\pgfqpoint{4.909334in}{1.875070in}}%
\pgfpathlineto{\pgfqpoint{4.909752in}{2.072012in}}%
\pgfpathlineto{\pgfqpoint{4.909974in}{1.698785in}}%
\pgfpathlineto{\pgfqpoint{4.910447in}{2.040336in}}%
\pgfpathlineto{\pgfqpoint{4.910919in}{1.689210in}}%
\pgfpathlineto{\pgfqpoint{4.911557in}{1.792143in}}%
\pgfpathlineto{\pgfqpoint{4.912194in}{2.063698in}}%
\pgfpathlineto{\pgfqpoint{4.911723in}{1.722845in}}%
\pgfpathlineto{\pgfqpoint{4.912692in}{2.001145in}}%
\pgfpathlineto{\pgfqpoint{4.912940in}{1.685837in}}%
\pgfpathlineto{\pgfqpoint{4.913023in}{2.075355in}}%
\pgfpathlineto{\pgfqpoint{4.913796in}{1.981571in}}%
\pgfpathlineto{\pgfqpoint{4.914540in}{1.677977in}}%
\pgfpathlineto{\pgfqpoint{4.914567in}{2.060046in}}%
\pgfpathlineto{\pgfqpoint{4.914925in}{1.844707in}}%
\pgfpathlineto{\pgfqpoint{4.915502in}{2.083292in}}%
\pgfpathlineto{\pgfqpoint{4.915749in}{1.774250in}}%
\pgfpathlineto{\pgfqpoint{4.916051in}{2.009433in}}%
\pgfpathlineto{\pgfqpoint{4.917146in}{1.627010in}}%
\pgfpathlineto{\pgfqpoint{4.916298in}{2.044262in}}%
\pgfpathlineto{\pgfqpoint{4.917228in}{1.800695in}}%
\pgfpathlineto{\pgfqpoint{4.917256in}{2.065518in}}%
\pgfpathlineto{\pgfqpoint{4.917556in}{1.631877in}}%
\pgfpathlineto{\pgfqpoint{4.918321in}{1.968893in}}%
\pgfpathlineto{\pgfqpoint{4.919030in}{1.597449in}}%
\pgfpathlineto{\pgfqpoint{4.919111in}{2.060527in}}%
\pgfpathlineto{\pgfqpoint{4.919438in}{1.838750in}}%
\pgfpathlineto{\pgfqpoint{4.920172in}{2.091392in}}%
\pgfpathlineto{\pgfqpoint{4.919710in}{1.633490in}}%
\pgfpathlineto{\pgfqpoint{4.920552in}{1.857616in}}%
\pgfpathlineto{\pgfqpoint{4.920959in}{2.055352in}}%
\pgfpathlineto{\pgfqpoint{4.921311in}{1.758725in}}%
\pgfpathlineto{\pgfqpoint{4.921636in}{1.983439in}}%
\pgfpathlineto{\pgfqpoint{4.921960in}{1.723090in}}%
\pgfpathlineto{\pgfqpoint{4.922366in}{2.063797in}}%
\pgfpathlineto{\pgfqpoint{4.922744in}{1.772041in}}%
\pgfpathlineto{\pgfqpoint{4.923284in}{1.688135in}}%
\pgfpathlineto{\pgfqpoint{4.923903in}{2.097504in}}%
\pgfpathlineto{\pgfqpoint{4.924979in}{1.396693in}}%
\pgfpathlineto{\pgfqpoint{4.925006in}{1.784640in}}%
\pgfpathlineto{\pgfqpoint{4.925032in}{2.052389in}}%
\pgfpathlineto{\pgfqpoint{4.925676in}{1.462765in}}%
\pgfpathlineto{\pgfqpoint{4.926105in}{2.003974in}}%
\pgfpathlineto{\pgfqpoint{4.926614in}{1.743426in}}%
\pgfpathlineto{\pgfqpoint{4.926426in}{2.057033in}}%
\pgfpathlineto{\pgfqpoint{4.927228in}{1.947849in}}%
\pgfpathlineto{\pgfqpoint{4.927389in}{2.043202in}}%
\pgfpathlineto{\pgfqpoint{4.928002in}{1.781406in}}%
\pgfpathlineto{\pgfqpoint{4.928296in}{1.925912in}}%
\pgfpathlineto{\pgfqpoint{4.929307in}{1.523098in}}%
\pgfpathlineto{\pgfqpoint{4.928429in}{2.052311in}}%
\pgfpathlineto{\pgfqpoint{4.929413in}{1.750962in}}%
\pgfpathlineto{\pgfqpoint{4.929812in}{2.075202in}}%
\pgfpathlineto{\pgfqpoint{4.930104in}{1.528055in}}%
\pgfpathlineto{\pgfqpoint{4.930501in}{1.900187in}}%
\pgfpathlineto{\pgfqpoint{4.930528in}{1.665094in}}%
\pgfpathlineto{\pgfqpoint{4.930819in}{2.074594in}}%
\pgfpathlineto{\pgfqpoint{4.931587in}{1.965733in}}%
\pgfpathlineto{\pgfqpoint{4.931957in}{1.699701in}}%
\pgfpathlineto{\pgfqpoint{4.931772in}{2.040218in}}%
\pgfpathlineto{\pgfqpoint{4.932669in}{1.994638in}}%
\pgfpathlineto{\pgfqpoint{4.933065in}{2.069220in}}%
\pgfpathlineto{\pgfqpoint{4.933381in}{1.735224in}}%
\pgfpathlineto{\pgfqpoint{4.933512in}{1.834247in}}%
\pgfpathlineto{\pgfqpoint{4.933539in}{1.637409in}}%
\pgfpathlineto{\pgfqpoint{4.934485in}{2.046998in}}%
\pgfpathlineto{\pgfqpoint{4.934590in}{1.918474in}}%
\pgfpathlineto{\pgfqpoint{4.934642in}{2.078564in}}%
\pgfpathlineto{\pgfqpoint{4.935482in}{1.770762in}}%
\pgfpathlineto{\pgfqpoint{4.935508in}{2.031210in}}%
\pgfpathlineto{\pgfqpoint{4.936424in}{1.456641in}}%
\pgfpathlineto{\pgfqpoint{4.936058in}{2.061492in}}%
\pgfpathlineto{\pgfqpoint{4.936607in}{1.924324in}}%
\pgfpathlineto{\pgfqpoint{4.936763in}{2.018580in}}%
\pgfpathlineto{\pgfqpoint{4.937311in}{1.738195in}}%
\pgfpathlineto{\pgfqpoint{4.937729in}{1.961772in}}%
\pgfpathlineto{\pgfqpoint{4.937989in}{1.718392in}}%
\pgfpathlineto{\pgfqpoint{4.938171in}{2.078393in}}%
\pgfpathlineto{\pgfqpoint{4.938822in}{1.921660in}}%
\pgfpathlineto{\pgfqpoint{4.939549in}{2.042266in}}%
\pgfpathlineto{\pgfqpoint{4.939133in}{1.811075in}}%
\pgfpathlineto{\pgfqpoint{4.939886in}{2.029121in}}%
\pgfpathlineto{\pgfqpoint{4.940430in}{1.736596in}}%
\pgfpathlineto{\pgfqpoint{4.940352in}{2.062104in}}%
\pgfpathlineto{\pgfqpoint{4.940999in}{1.951556in}}%
\pgfpathlineto{\pgfqpoint{4.941103in}{1.768131in}}%
\pgfpathlineto{\pgfqpoint{4.941619in}{2.071698in}}%
\pgfpathlineto{\pgfqpoint{4.941852in}{1.795850in}}%
\pgfpathlineto{\pgfqpoint{4.942883in}{2.061140in}}%
\pgfpathlineto{\pgfqpoint{4.942574in}{1.653485in}}%
\pgfpathlineto{\pgfqpoint{4.942960in}{1.931873in}}%
\pgfpathlineto{\pgfqpoint{4.943474in}{2.076556in}}%
\pgfpathlineto{\pgfqpoint{4.943372in}{1.734253in}}%
\pgfpathlineto{\pgfqpoint{4.943937in}{1.777563in}}%
\pgfpathlineto{\pgfqpoint{4.943963in}{1.637312in}}%
\pgfpathlineto{\pgfqpoint{4.944963in}{2.060597in}}%
\pgfpathlineto{\pgfqpoint{4.945014in}{1.751995in}}%
\pgfpathlineto{\pgfqpoint{4.945756in}{2.049860in}}%
\pgfpathlineto{\pgfqpoint{4.945091in}{1.659096in}}%
\pgfpathlineto{\pgfqpoint{4.946140in}{1.982132in}}%
\pgfpathlineto{\pgfqpoint{4.946574in}{2.118859in}}%
\pgfpathlineto{\pgfqpoint{4.946446in}{1.585337in}}%
\pgfpathlineto{\pgfqpoint{4.947109in}{1.994086in}}%
\pgfpathlineto{\pgfqpoint{4.947211in}{1.663388in}}%
\pgfpathlineto{\pgfqpoint{4.948051in}{2.067101in}}%
\pgfpathlineto{\pgfqpoint{4.948229in}{1.935493in}}%
\pgfpathlineto{\pgfqpoint{4.949042in}{2.099251in}}%
\pgfpathlineto{\pgfqpoint{4.948433in}{1.449145in}}%
\pgfpathlineto{\pgfqpoint{4.949321in}{1.973417in}}%
\pgfpathlineto{\pgfqpoint{4.949600in}{1.708722in}}%
\pgfpathlineto{\pgfqpoint{4.950182in}{2.065698in}}%
\pgfpathlineto{\pgfqpoint{4.950435in}{1.943493in}}%
\pgfpathlineto{\pgfqpoint{4.950713in}{2.026046in}}%
\pgfpathlineto{\pgfqpoint{4.950537in}{1.656821in}}%
\pgfpathlineto{\pgfqpoint{4.951572in}{2.015592in}}%
\pgfpathlineto{\pgfqpoint{4.951774in}{1.692019in}}%
\pgfpathlineto{\pgfqpoint{4.952278in}{2.059645in}}%
\pgfpathlineto{\pgfqpoint{4.952705in}{1.836696in}}%
\pgfpathlineto{\pgfqpoint{4.953284in}{2.050904in}}%
\pgfpathlineto{\pgfqpoint{4.953585in}{1.662074in}}%
\pgfpathlineto{\pgfqpoint{4.953786in}{2.003279in}}%
\pgfpathlineto{\pgfqpoint{4.953886in}{1.688802in}}%
\pgfpathlineto{\pgfqpoint{4.954412in}{2.059627in}}%
\pgfpathlineto{\pgfqpoint{4.954913in}{1.731953in}}%
\pgfpathlineto{\pgfqpoint{4.955438in}{2.063825in}}%
\pgfpathlineto{\pgfqpoint{4.955263in}{1.562410in}}%
\pgfpathlineto{\pgfqpoint{4.956038in}{1.906660in}}%
\pgfpathlineto{\pgfqpoint{4.956188in}{2.046603in}}%
\pgfpathlineto{\pgfqpoint{4.957085in}{1.685608in}}%
\pgfpathlineto{\pgfqpoint{4.957135in}{1.930250in}}%
\pgfpathlineto{\pgfqpoint{4.957856in}{1.548297in}}%
\pgfpathlineto{\pgfqpoint{4.958129in}{2.082079in}}%
\pgfpathlineto{\pgfqpoint{4.958253in}{1.570784in}}%
\pgfpathlineto{\pgfqpoint{4.958477in}{2.099819in}}%
\pgfpathlineto{\pgfqpoint{4.959369in}{1.939712in}}%
\pgfpathlineto{\pgfqpoint{4.960087in}{1.520801in}}%
\pgfpathlineto{\pgfqpoint{4.959716in}{2.029302in}}%
\pgfpathlineto{\pgfqpoint{4.960482in}{1.878653in}}%
\pgfpathlineto{\pgfqpoint{4.960655in}{2.050611in}}%
\pgfpathlineto{\pgfqpoint{4.960877in}{1.578410in}}%
\pgfpathlineto{\pgfqpoint{4.961617in}{2.027341in}}%
\pgfpathlineto{\pgfqpoint{4.961666in}{1.670702in}}%
\pgfpathlineto{\pgfqpoint{4.961863in}{2.117218in}}%
\pgfpathlineto{\pgfqpoint{4.962724in}{1.938215in}}%
\pgfpathlineto{\pgfqpoint{4.963215in}{2.051728in}}%
\pgfpathlineto{\pgfqpoint{4.963607in}{1.651766in}}%
\pgfpathlineto{\pgfqpoint{4.963828in}{1.992528in}}%
\pgfpathlineto{\pgfqpoint{4.964758in}{1.733184in}}%
\pgfpathlineto{\pgfqpoint{4.964097in}{2.054343in}}%
\pgfpathlineto{\pgfqpoint{4.964953in}{1.932980in}}%
\pgfpathlineto{\pgfqpoint{4.965027in}{1.707050in}}%
\pgfpathlineto{\pgfqpoint{4.965051in}{2.039399in}}%
\pgfpathlineto{\pgfqpoint{4.966052in}{1.793904in}}%
\pgfpathlineto{\pgfqpoint{4.966661in}{2.036900in}}%
\pgfpathlineto{\pgfqpoint{4.966101in}{1.535503in}}%
\pgfpathlineto{\pgfqpoint{4.967172in}{1.858079in}}%
\pgfpathlineto{\pgfqpoint{4.968094in}{2.058035in}}%
\pgfpathlineto{\pgfqpoint{4.967317in}{1.718276in}}%
\pgfpathlineto{\pgfqpoint{4.968288in}{1.908885in}}%
\pgfpathlineto{\pgfqpoint{4.968507in}{2.000658in}}%
\pgfpathlineto{\pgfqpoint{4.968604in}{1.693685in}}%
\pgfpathlineto{\pgfqpoint{4.969160in}{1.876188in}}%
\pgfpathlineto{\pgfqpoint{4.969354in}{1.730075in}}%
\pgfpathlineto{\pgfqpoint{4.969451in}{2.054685in}}%
\pgfpathlineto{\pgfqpoint{4.970272in}{1.873229in}}%
\pgfpathlineto{\pgfqpoint{4.971164in}{2.045230in}}%
\pgfpathlineto{\pgfqpoint{4.970755in}{1.455808in}}%
\pgfpathlineto{\pgfqpoint{4.971357in}{1.916976in}}%
\pgfpathlineto{\pgfqpoint{4.971862in}{1.640380in}}%
\pgfpathlineto{\pgfqpoint{4.972295in}{2.044896in}}%
\pgfpathlineto{\pgfqpoint{4.972439in}{1.916523in}}%
\pgfpathlineto{\pgfqpoint{4.972871in}{2.045994in}}%
\pgfpathlineto{\pgfqpoint{4.973206in}{1.658654in}}%
\pgfpathlineto{\pgfqpoint{4.973518in}{1.958113in}}%
\pgfpathlineto{\pgfqpoint{4.973709in}{1.700187in}}%
\pgfpathlineto{\pgfqpoint{4.973949in}{2.048291in}}%
\pgfpathlineto{\pgfqpoint{4.974618in}{1.907651in}}%
\pgfpathlineto{\pgfqpoint{4.974714in}{1.686637in}}%
\pgfpathlineto{\pgfqpoint{4.975716in}{2.045960in}}%
\pgfpathlineto{\pgfqpoint{4.976382in}{1.720583in}}%
\pgfpathlineto{\pgfqpoint{4.976834in}{1.837617in}}%
\pgfpathlineto{\pgfqpoint{4.977570in}{2.033285in}}%
\pgfpathlineto{\pgfqpoint{4.976977in}{1.718514in}}%
\pgfpathlineto{\pgfqpoint{4.977831in}{1.879633in}}%
\pgfpathlineto{\pgfqpoint{4.977997in}{2.048087in}}%
\pgfpathlineto{\pgfqpoint{4.978944in}{1.460239in}}%
\pgfpathlineto{\pgfqpoint{4.979015in}{2.041685in}}%
\pgfpathlineto{\pgfqpoint{4.980054in}{1.853925in}}%
\pgfpathlineto{\pgfqpoint{4.980902in}{1.444299in}}%
\pgfpathlineto{\pgfqpoint{4.980996in}{2.021139in}}%
\pgfpathlineto{\pgfqpoint{4.981090in}{1.797612in}}%
\pgfpathlineto{\pgfqpoint{4.981184in}{2.052507in}}%
\pgfpathlineto{\pgfqpoint{4.981396in}{1.568149in}}%
\pgfpathlineto{\pgfqpoint{4.982218in}{1.981689in}}%
\pgfpathlineto{\pgfqpoint{4.982242in}{1.660157in}}%
\pgfpathlineto{\pgfqpoint{4.983273in}{2.066407in}}%
\pgfpathlineto{\pgfqpoint{4.983320in}{1.867482in}}%
\pgfpathlineto{\pgfqpoint{4.984395in}{2.039117in}}%
\pgfpathlineto{\pgfqpoint{4.983483in}{1.743450in}}%
\pgfpathlineto{\pgfqpoint{4.984441in}{1.951346in}}%
\pgfpathlineto{\pgfqpoint{4.984465in}{1.954613in}}%
\pgfpathlineto{\pgfqpoint{4.984511in}{1.659191in}}%
\pgfpathlineto{\pgfqpoint{4.985001in}{2.059942in}}%
\pgfpathlineto{\pgfqpoint{4.985584in}{1.764038in}}%
\pgfpathlineto{\pgfqpoint{4.986095in}{1.673064in}}%
\pgfpathlineto{\pgfqpoint{4.986723in}{2.075930in}}%
\pgfpathlineto{\pgfqpoint{4.987674in}{1.756608in}}%
\pgfpathlineto{\pgfqpoint{4.987372in}{2.082522in}}%
\pgfpathlineto{\pgfqpoint{4.987859in}{1.818392in}}%
\pgfpathlineto{\pgfqpoint{4.988830in}{2.033719in}}%
\pgfpathlineto{\pgfqpoint{4.988622in}{1.701313in}}%
\pgfpathlineto{\pgfqpoint{4.988946in}{1.713935in}}%
\pgfpathlineto{\pgfqpoint{4.988969in}{1.546280in}}%
\pgfpathlineto{\pgfqpoint{4.989500in}{2.047767in}}%
\pgfpathlineto{\pgfqpoint{4.990007in}{1.925810in}}%
\pgfpathlineto{\pgfqpoint{4.990996in}{2.055818in}}%
\pgfpathlineto{\pgfqpoint{4.990513in}{1.656407in}}%
\pgfpathlineto{\pgfqpoint{4.991088in}{1.887452in}}%
\pgfpathlineto{\pgfqpoint{4.991548in}{2.020319in}}%
\pgfpathlineto{\pgfqpoint{4.992098in}{1.672515in}}%
\pgfpathlineto{\pgfqpoint{4.993083in}{2.009952in}}%
\pgfpathlineto{\pgfqpoint{4.993037in}{1.638947in}}%
\pgfpathlineto{\pgfqpoint{4.993220in}{1.927270in}}%
\pgfpathlineto{\pgfqpoint{4.993334in}{1.988926in}}%
\pgfpathlineto{\pgfqpoint{4.993426in}{1.809724in}}%
\pgfpathlineto{\pgfqpoint{4.994407in}{1.518833in}}%
\pgfpathlineto{\pgfqpoint{4.994453in}{2.039751in}}%
\pgfpathlineto{\pgfqpoint{4.994521in}{1.848616in}}%
\pgfpathlineto{\pgfqpoint{4.994658in}{2.057395in}}%
\pgfpathlineto{\pgfqpoint{4.994863in}{1.653154in}}%
\pgfpathlineto{\pgfqpoint{4.995614in}{1.966135in}}%
\pgfpathlineto{\pgfqpoint{4.995932in}{1.489563in}}%
\pgfpathlineto{\pgfqpoint{4.995978in}{2.035620in}}%
\pgfpathlineto{\pgfqpoint{4.996727in}{1.933029in}}%
\pgfpathlineto{\pgfqpoint{4.996795in}{1.623796in}}%
\pgfpathlineto{\pgfqpoint{4.997338in}{2.008604in}}%
\pgfpathlineto{\pgfqpoint{4.997836in}{1.927612in}}%
\pgfpathlineto{\pgfqpoint{4.998672in}{2.033059in}}%
\pgfpathlineto{\pgfqpoint{4.998356in}{1.632572in}}%
\pgfpathlineto{\pgfqpoint{4.998785in}{2.013141in}}%
\pgfpathlineto{\pgfqpoint{4.999799in}{1.622489in}}%
\pgfpathlineto{\pgfqpoint{4.999844in}{2.036054in}}%
\pgfpathlineto{\pgfqpoint{4.999889in}{1.930598in}}%
\pgfpathlineto{\pgfqpoint{5.000339in}{1.992247in}}%
\pgfpathlineto{\pgfqpoint{5.000070in}{1.690586in}}%
\pgfpathlineto{\pgfqpoint{5.000901in}{1.911415in}}%
\pgfpathlineto{\pgfqpoint{5.001933in}{1.594327in}}%
\pgfpathlineto{\pgfqpoint{5.001417in}{2.036440in}}%
\pgfpathlineto{\pgfqpoint{5.002000in}{1.906832in}}%
\pgfpathlineto{\pgfqpoint{5.002716in}{2.056155in}}%
\pgfpathlineto{\pgfqpoint{5.002336in}{1.671971in}}%
\pgfpathlineto{\pgfqpoint{5.002984in}{1.998503in}}%
\pgfpathlineto{\pgfqpoint{5.003810in}{1.485189in}}%
\pgfpathlineto{\pgfqpoint{5.003386in}{2.011986in}}%
\pgfpathlineto{\pgfqpoint{5.004100in}{1.771008in}}%
\pgfpathlineto{\pgfqpoint{5.005013in}{2.041726in}}%
\pgfpathlineto{\pgfqpoint{5.004724in}{1.634398in}}%
\pgfpathlineto{\pgfqpoint{5.005191in}{2.006022in}}%
\pgfpathlineto{\pgfqpoint{5.005368in}{1.565398in}}%
\pgfpathlineto{\pgfqpoint{5.006079in}{2.015374in}}%
\pgfpathlineto{\pgfqpoint{5.006301in}{1.928967in}}%
\pgfpathlineto{\pgfqpoint{5.007120in}{1.514262in}}%
\pgfpathlineto{\pgfqpoint{5.007341in}{2.071260in}}%
\pgfpathlineto{\pgfqpoint{5.007430in}{1.829483in}}%
\pgfpathlineto{\pgfqpoint{5.008247in}{2.035127in}}%
\pgfpathlineto{\pgfqpoint{5.007805in}{1.668074in}}%
\pgfpathlineto{\pgfqpoint{5.008556in}{1.852947in}}%
\pgfpathlineto{\pgfqpoint{5.008578in}{1.852512in}}%
\pgfpathlineto{\pgfqpoint{5.009195in}{2.049836in}}%
\pgfpathlineto{\pgfqpoint{5.009525in}{1.574780in}}%
\pgfpathlineto{\pgfqpoint{5.009679in}{1.986294in}}%
\pgfpathlineto{\pgfqpoint{5.010777in}{1.655701in}}%
\pgfpathlineto{\pgfqpoint{5.010141in}{2.016379in}}%
\pgfpathlineto{\pgfqpoint{5.010821in}{1.849918in}}%
\pgfpathlineto{\pgfqpoint{5.010975in}{2.053495in}}%
\pgfpathlineto{\pgfqpoint{5.010931in}{1.520273in}}%
\pgfpathlineto{\pgfqpoint{5.011939in}{2.000060in}}%
\pgfpathlineto{\pgfqpoint{5.012420in}{1.571690in}}%
\pgfpathlineto{\pgfqpoint{5.012398in}{2.048813in}}%
\pgfpathlineto{\pgfqpoint{5.013053in}{1.750088in}}%
\pgfpathlineto{\pgfqpoint{5.014099in}{2.043933in}}%
\pgfpathlineto{\pgfqpoint{5.013729in}{1.505054in}}%
\pgfpathlineto{\pgfqpoint{5.014142in}{1.941234in}}%
\pgfpathlineto{\pgfqpoint{5.014164in}{1.581719in}}%
\pgfpathlineto{\pgfqpoint{5.014730in}{2.065033in}}%
\pgfpathlineto{\pgfqpoint{5.015251in}{1.836398in}}%
\pgfpathlineto{\pgfqpoint{5.015468in}{2.030303in}}%
\pgfpathlineto{\pgfqpoint{5.016053in}{1.648787in}}%
\pgfpathlineto{\pgfqpoint{5.016313in}{1.949505in}}%
\pgfpathlineto{\pgfqpoint{5.016984in}{1.579324in}}%
\pgfpathlineto{\pgfqpoint{5.016898in}{2.064597in}}%
\pgfpathlineto{\pgfqpoint{5.017416in}{1.946427in}}%
\pgfpathlineto{\pgfqpoint{5.018193in}{1.579912in}}%
\pgfpathlineto{\pgfqpoint{5.017546in}{2.037955in}}%
\pgfpathlineto{\pgfqpoint{5.018559in}{1.782159in}}%
\pgfpathlineto{\pgfqpoint{5.018947in}{2.019818in}}%
\pgfpathlineto{\pgfqpoint{5.019184in}{1.514243in}}%
\pgfpathlineto{\pgfqpoint{5.019678in}{1.905366in}}%
\pgfpathlineto{\pgfqpoint{5.020708in}{1.536903in}}%
\pgfpathlineto{\pgfqpoint{5.020279in}{2.032709in}}%
\pgfpathlineto{\pgfqpoint{5.020794in}{1.799074in}}%
\pgfpathlineto{\pgfqpoint{5.021628in}{2.044398in}}%
\pgfpathlineto{\pgfqpoint{5.021714in}{1.616158in}}%
\pgfpathlineto{\pgfqpoint{5.021906in}{1.900709in}}%
\pgfpathlineto{\pgfqpoint{5.021928in}{1.735749in}}%
\pgfpathlineto{\pgfqpoint{5.022909in}{2.040274in}}%
\pgfpathlineto{\pgfqpoint{5.023016in}{1.887397in}}%
\pgfpathlineto{\pgfqpoint{5.023250in}{2.030896in}}%
\pgfpathlineto{\pgfqpoint{5.023825in}{1.470015in}}%
\pgfpathlineto{\pgfqpoint{5.024123in}{1.977986in}}%
\pgfpathlineto{\pgfqpoint{5.024972in}{1.609067in}}%
\pgfpathlineto{\pgfqpoint{5.024675in}{2.001184in}}%
\pgfpathlineto{\pgfqpoint{5.025248in}{1.877968in}}%
\pgfpathlineto{\pgfqpoint{5.025524in}{2.035603in}}%
\pgfpathlineto{\pgfqpoint{5.025502in}{1.627999in}}%
\pgfpathlineto{\pgfqpoint{5.026349in}{1.972775in}}%
\pgfpathlineto{\pgfqpoint{5.026624in}{1.620712in}}%
\pgfpathlineto{\pgfqpoint{5.027025in}{2.011071in}}%
\pgfpathlineto{\pgfqpoint{5.027447in}{1.947429in}}%
\pgfpathlineto{\pgfqpoint{5.027848in}{2.027830in}}%
\pgfpathlineto{\pgfqpoint{5.027743in}{1.657652in}}%
\pgfpathlineto{\pgfqpoint{5.028374in}{1.894884in}}%
\pgfpathlineto{\pgfqpoint{5.028795in}{1.650084in}}%
\pgfpathlineto{\pgfqpoint{5.028459in}{2.045193in}}%
\pgfpathlineto{\pgfqpoint{5.029467in}{1.916662in}}%
\pgfpathlineto{\pgfqpoint{5.030034in}{1.697571in}}%
\pgfpathlineto{\pgfqpoint{5.029551in}{2.018401in}}%
\pgfpathlineto{\pgfqpoint{5.030558in}{1.909609in}}%
\pgfpathlineto{\pgfqpoint{5.030704in}{2.029129in}}%
\pgfpathlineto{\pgfqpoint{5.030600in}{1.633864in}}%
\pgfpathlineto{\pgfqpoint{5.031666in}{1.950103in}}%
\pgfpathlineto{\pgfqpoint{5.032501in}{1.602819in}}%
\pgfpathlineto{\pgfqpoint{5.032229in}{2.022999in}}%
\pgfpathlineto{\pgfqpoint{5.032730in}{1.843711in}}%
\pgfpathlineto{\pgfqpoint{5.032917in}{1.575695in}}%
\pgfpathlineto{\pgfqpoint{5.033832in}{2.031516in}}%
\pgfpathlineto{\pgfqpoint{5.034310in}{1.529269in}}%
\pgfpathlineto{\pgfqpoint{5.034953in}{1.928969in}}%
\pgfpathlineto{\pgfqpoint{5.035160in}{2.012932in}}%
\pgfpathlineto{\pgfqpoint{5.036091in}{1.674784in}}%
\pgfpathlineto{\pgfqpoint{5.037082in}{1.544802in}}%
\pgfpathlineto{\pgfqpoint{5.037185in}{2.057221in}}%
\pgfpathlineto{\pgfqpoint{5.037618in}{1.650325in}}%
\pgfpathlineto{\pgfqpoint{5.038297in}{1.846510in}}%
\pgfpathlineto{\pgfqpoint{5.038728in}{1.990346in}}%
\pgfpathlineto{\pgfqpoint{5.038502in}{1.627620in}}%
\pgfpathlineto{\pgfqpoint{5.038954in}{1.893762in}}%
\pgfpathlineto{\pgfqpoint{5.038975in}{1.252367in}}%
\pgfpathlineto{\pgfqpoint{5.039180in}{2.005758in}}%
\pgfpathlineto{\pgfqpoint{5.040041in}{1.849901in}}%
\pgfpathlineto{\pgfqpoint{5.040061in}{2.041074in}}%
\pgfpathlineto{\pgfqpoint{5.041063in}{1.649677in}}%
\pgfpathlineto{\pgfqpoint{5.041145in}{1.921366in}}%
\pgfpathlineto{\pgfqpoint{5.042206in}{1.675895in}}%
\pgfpathlineto{\pgfqpoint{5.042022in}{2.020264in}}%
\pgfpathlineto{\pgfqpoint{5.042246in}{1.919412in}}%
\pgfpathlineto{\pgfqpoint{5.043142in}{1.349750in}}%
\pgfpathlineto{\pgfqpoint{5.042877in}{2.025133in}}%
\pgfpathlineto{\pgfqpoint{5.043264in}{1.934078in}}%
\pgfpathlineto{\pgfqpoint{5.043508in}{2.003019in}}%
\pgfpathlineto{\pgfqpoint{5.043487in}{1.475376in}}%
\pgfpathlineto{\pgfqpoint{5.044339in}{1.958423in}}%
\pgfpathlineto{\pgfqpoint{5.044927in}{1.623955in}}%
\pgfpathlineto{\pgfqpoint{5.044583in}{2.018980in}}%
\pgfpathlineto{\pgfqpoint{5.045453in}{1.801354in}}%
\pgfpathlineto{\pgfqpoint{5.046220in}{2.035883in}}%
\pgfpathlineto{\pgfqpoint{5.046200in}{1.371241in}}%
\pgfpathlineto{\pgfqpoint{5.046563in}{1.882282in}}%
\pgfpathlineto{\pgfqpoint{5.047107in}{2.048550in}}%
\pgfpathlineto{\pgfqpoint{5.046906in}{1.699511in}}%
\pgfpathlineto{\pgfqpoint{5.047610in}{1.967726in}}%
\pgfpathlineto{\pgfqpoint{5.047932in}{1.442752in}}%
\pgfpathlineto{\pgfqpoint{5.048193in}{2.021701in}}%
\pgfpathlineto{\pgfqpoint{5.048715in}{1.813488in}}%
\pgfpathlineto{\pgfqpoint{5.048856in}{2.052876in}}%
\pgfpathlineto{\pgfqpoint{5.048976in}{1.662370in}}%
\pgfpathlineto{\pgfqpoint{5.049797in}{1.994692in}}%
\pgfpathlineto{\pgfqpoint{5.050497in}{1.416850in}}%
\pgfpathlineto{\pgfqpoint{5.050617in}{2.023039in}}%
\pgfpathlineto{\pgfqpoint{5.050916in}{1.848379in}}%
\pgfpathlineto{\pgfqpoint{5.051535in}{2.012500in}}%
\pgfpathlineto{\pgfqpoint{5.051674in}{1.607844in}}%
\pgfpathlineto{\pgfqpoint{5.052013in}{1.970548in}}%
\pgfpathlineto{\pgfqpoint{5.052609in}{1.440416in}}%
\pgfpathlineto{\pgfqpoint{5.052430in}{2.028215in}}%
\pgfpathlineto{\pgfqpoint{5.053146in}{1.724798in}}%
\pgfpathlineto{\pgfqpoint{5.053741in}{2.058163in}}%
\pgfpathlineto{\pgfqpoint{5.054256in}{1.942255in}}%
\pgfpathlineto{\pgfqpoint{5.055008in}{1.636169in}}%
\pgfpathlineto{\pgfqpoint{5.055284in}{2.040027in}}%
\pgfpathlineto{\pgfqpoint{5.055363in}{1.894122in}}%
\pgfpathlineto{\pgfqpoint{5.056074in}{2.022737in}}%
\pgfpathlineto{\pgfqpoint{5.055719in}{1.611292in}}%
\pgfpathlineto{\pgfqpoint{5.056448in}{1.804778in}}%
\pgfpathlineto{\pgfqpoint{5.056488in}{1.998720in}}%
\pgfpathlineto{\pgfqpoint{5.057118in}{1.436105in}}%
\pgfpathlineto{\pgfqpoint{5.057570in}{1.939935in}}%
\pgfpathlineto{\pgfqpoint{5.057805in}{1.631490in}}%
\pgfpathlineto{\pgfqpoint{5.058257in}{2.039226in}}%
\pgfpathlineto{\pgfqpoint{5.058688in}{1.858050in}}%
\pgfpathlineto{\pgfqpoint{5.058982in}{2.048367in}}%
\pgfpathlineto{\pgfqpoint{5.059686in}{1.671972in}}%
\pgfpathlineto{\pgfqpoint{5.059804in}{1.951432in}}%
\pgfpathlineto{\pgfqpoint{5.060331in}{1.493876in}}%
\pgfpathlineto{\pgfqpoint{5.059843in}{2.023183in}}%
\pgfpathlineto{\pgfqpoint{5.060897in}{1.955501in}}%
\pgfpathlineto{\pgfqpoint{5.061656in}{1.594384in}}%
\pgfpathlineto{\pgfqpoint{5.061520in}{2.002511in}}%
\pgfpathlineto{\pgfqpoint{5.062006in}{1.954002in}}%
\pgfpathlineto{\pgfqpoint{5.062026in}{2.030177in}}%
\pgfpathlineto{\pgfqpoint{5.062356in}{1.638726in}}%
\pgfpathlineto{\pgfqpoint{5.063113in}{1.977048in}}%
\pgfpathlineto{\pgfqpoint{5.063675in}{2.010849in}}%
\pgfpathlineto{\pgfqpoint{5.064217in}{1.645348in}}%
\pgfpathlineto{\pgfqpoint{5.064526in}{2.047812in}}%
\pgfpathlineto{\pgfqpoint{5.064372in}{1.633151in}}%
\pgfpathlineto{\pgfqpoint{5.065337in}{1.963143in}}%
\pgfpathlineto{\pgfqpoint{5.066243in}{1.598182in}}%
\pgfpathlineto{\pgfqpoint{5.065434in}{2.014042in}}%
\pgfpathlineto{\pgfqpoint{5.066436in}{1.906767in}}%
\pgfpathlineto{\pgfqpoint{5.066859in}{1.990475in}}%
\pgfpathlineto{\pgfqpoint{5.066493in}{1.574544in}}%
\pgfpathlineto{\pgfqpoint{5.067473in}{1.810251in}}%
\pgfpathlineto{\pgfqpoint{5.067608in}{1.525311in}}%
\pgfpathlineto{\pgfqpoint{5.068279in}{1.993955in}}%
\pgfpathlineto{\pgfqpoint{5.068585in}{1.806459in}}%
\pgfpathlineto{\pgfqpoint{5.068987in}{1.543182in}}%
\pgfpathlineto{\pgfqpoint{5.069618in}{2.028720in}}%
\pgfpathlineto{\pgfqpoint{5.070057in}{1.499621in}}%
\pgfpathlineto{\pgfqpoint{5.069847in}{2.046819in}}%
\pgfpathlineto{\pgfqpoint{5.070724in}{1.802139in}}%
\pgfpathlineto{\pgfqpoint{5.070990in}{1.988065in}}%
\pgfpathlineto{\pgfqpoint{5.071713in}{1.605660in}}%
\pgfpathlineto{\pgfqpoint{5.071846in}{1.901729in}}%
\pgfpathlineto{\pgfqpoint{5.072757in}{2.025097in}}%
\pgfpathlineto{\pgfqpoint{5.072169in}{1.685410in}}%
\pgfpathlineto{\pgfqpoint{5.072947in}{1.945495in}}%
\pgfpathlineto{\pgfqpoint{5.073174in}{1.451382in}}%
\pgfpathlineto{\pgfqpoint{5.073325in}{2.027773in}}%
\pgfpathlineto{\pgfqpoint{5.074082in}{1.801274in}}%
\pgfpathlineto{\pgfqpoint{5.074686in}{1.675815in}}%
\pgfpathlineto{\pgfqpoint{5.074384in}{2.030721in}}%
\pgfpathlineto{\pgfqpoint{5.075064in}{1.903579in}}%
\pgfpathlineto{\pgfqpoint{5.075214in}{2.030966in}}%
\pgfpathlineto{\pgfqpoint{5.075648in}{1.571261in}}%
\pgfpathlineto{\pgfqpoint{5.076193in}{2.003391in}}%
\pgfpathlineto{\pgfqpoint{5.077057in}{1.512380in}}%
\pgfpathlineto{\pgfqpoint{5.076963in}{2.033209in}}%
\pgfpathlineto{\pgfqpoint{5.077320in}{1.899980in}}%
\pgfpathlineto{\pgfqpoint{5.077526in}{1.500670in}}%
\pgfpathlineto{\pgfqpoint{5.077545in}{1.998706in}}%
\pgfpathlineto{\pgfqpoint{5.078313in}{1.930104in}}%
\pgfpathlineto{\pgfqpoint{5.078705in}{1.996866in}}%
\pgfpathlineto{\pgfqpoint{5.078500in}{1.470333in}}%
\pgfpathlineto{\pgfqpoint{5.079415in}{1.920885in}}%
\pgfpathlineto{\pgfqpoint{5.079676in}{1.969197in}}%
\pgfpathlineto{\pgfqpoint{5.079639in}{1.700006in}}%
\pgfpathlineto{\pgfqpoint{5.079714in}{1.936796in}}%
\pgfpathlineto{\pgfqpoint{5.080552in}{1.379116in}}%
\pgfpathlineto{\pgfqpoint{5.080738in}{2.026838in}}%
\pgfpathlineto{\pgfqpoint{5.080813in}{1.979584in}}%
\pgfpathlineto{\pgfqpoint{5.081574in}{1.614837in}}%
\pgfpathlineto{\pgfqpoint{5.080961in}{2.038438in}}%
\pgfpathlineto{\pgfqpoint{5.081927in}{1.812292in}}%
\pgfpathlineto{\pgfqpoint{5.082724in}{2.009960in}}%
\pgfpathlineto{\pgfqpoint{5.082928in}{1.646220in}}%
\pgfpathlineto{\pgfqpoint{5.083020in}{1.900463in}}%
\pgfpathlineto{\pgfqpoint{5.084074in}{1.651572in}}%
\pgfpathlineto{\pgfqpoint{5.083797in}{2.017879in}}%
\pgfpathlineto{\pgfqpoint{5.084129in}{1.798590in}}%
\pgfpathlineto{\pgfqpoint{5.084350in}{2.027255in}}%
\pgfpathlineto{\pgfqpoint{5.084313in}{1.549396in}}%
\pgfpathlineto{\pgfqpoint{5.085272in}{1.982637in}}%
\pgfpathlineto{\pgfqpoint{5.085566in}{1.684310in}}%
\pgfpathlineto{\pgfqpoint{5.085934in}{1.999943in}}%
\pgfpathlineto{\pgfqpoint{5.086375in}{1.905074in}}%
\pgfpathlineto{\pgfqpoint{5.087218in}{1.997096in}}%
\pgfpathlineto{\pgfqpoint{5.086466in}{1.489369in}}%
\pgfpathlineto{\pgfqpoint{5.087401in}{1.918904in}}%
\pgfpathlineto{\pgfqpoint{5.088042in}{1.588448in}}%
\pgfpathlineto{\pgfqpoint{5.087676in}{1.987273in}}%
\pgfpathlineto{\pgfqpoint{5.088499in}{1.976892in}}%
\pgfpathlineto{\pgfqpoint{5.088846in}{1.653277in}}%
\pgfpathlineto{\pgfqpoint{5.088992in}{2.022309in}}%
\pgfpathlineto{\pgfqpoint{5.089630in}{1.885987in}}%
\pgfpathlineto{\pgfqpoint{5.090503in}{1.991468in}}%
\pgfpathlineto{\pgfqpoint{5.090049in}{1.361249in}}%
\pgfpathlineto{\pgfqpoint{5.090612in}{1.897539in}}%
\pgfpathlineto{\pgfqpoint{5.091339in}{1.559466in}}%
\pgfpathlineto{\pgfqpoint{5.090685in}{1.986112in}}%
\pgfpathlineto{\pgfqpoint{5.091720in}{1.886232in}}%
\pgfpathlineto{\pgfqpoint{5.092625in}{1.982161in}}%
\pgfpathlineto{\pgfqpoint{5.092010in}{1.710275in}}%
\pgfpathlineto{\pgfqpoint{5.092697in}{1.910603in}}%
\pgfpathlineto{\pgfqpoint{5.093781in}{1.612715in}}%
\pgfpathlineto{\pgfqpoint{5.093438in}{1.992919in}}%
\pgfpathlineto{\pgfqpoint{5.093799in}{1.868698in}}%
\pgfpathlineto{\pgfqpoint{5.094844in}{1.982439in}}%
\pgfpathlineto{\pgfqpoint{5.094322in}{1.638594in}}%
\pgfpathlineto{\pgfqpoint{5.094898in}{1.893715in}}%
\pgfpathlineto{\pgfqpoint{5.095869in}{1.529928in}}%
\pgfpathlineto{\pgfqpoint{5.095761in}{2.032019in}}%
\pgfpathlineto{\pgfqpoint{5.096030in}{1.776544in}}%
\pgfpathlineto{\pgfqpoint{5.096855in}{1.608806in}}%
\pgfpathlineto{\pgfqpoint{5.097124in}{2.017607in}}%
\pgfpathlineto{\pgfqpoint{5.097535in}{1.645088in}}%
\pgfpathlineto{\pgfqpoint{5.098232in}{1.965524in}}%
\pgfpathlineto{\pgfqpoint{5.098607in}{2.019541in}}%
\pgfpathlineto{\pgfqpoint{5.098571in}{1.556180in}}%
\pgfpathlineto{\pgfqpoint{5.099302in}{1.926804in}}%
\pgfpathlineto{\pgfqpoint{5.100067in}{1.387529in}}%
\pgfpathlineto{\pgfqpoint{5.099782in}{2.022344in}}%
\pgfpathlineto{\pgfqpoint{5.100404in}{1.752926in}}%
\pgfpathlineto{\pgfqpoint{5.100919in}{2.033377in}}%
\pgfpathlineto{\pgfqpoint{5.100600in}{1.698633in}}%
\pgfpathlineto{\pgfqpoint{5.101504in}{1.893556in}}%
\pgfpathlineto{\pgfqpoint{5.101805in}{1.604300in}}%
\pgfpathlineto{\pgfqpoint{5.102583in}{2.010078in}}%
\pgfpathlineto{\pgfqpoint{5.102619in}{1.701538in}}%
\pgfpathlineto{\pgfqpoint{5.103642in}{2.027571in}}%
\pgfpathlineto{\pgfqpoint{5.103289in}{1.566882in}}%
\pgfpathlineto{\pgfqpoint{5.103748in}{1.920778in}}%
\pgfpathlineto{\pgfqpoint{5.104188in}{1.459265in}}%
\pgfpathlineto{\pgfqpoint{5.104100in}{2.032387in}}%
\pgfpathlineto{\pgfqpoint{5.104857in}{1.694884in}}%
\pgfpathlineto{\pgfqpoint{5.105734in}{1.990632in}}%
\pgfpathlineto{\pgfqpoint{5.104944in}{1.464862in}}%
\pgfpathlineto{\pgfqpoint{5.105980in}{1.940597in}}%
\pgfpathlineto{\pgfqpoint{5.106908in}{1.739721in}}%
\pgfpathlineto{\pgfqpoint{5.106330in}{2.028994in}}%
\pgfpathlineto{\pgfqpoint{5.107083in}{1.790438in}}%
\pgfpathlineto{\pgfqpoint{5.107886in}{1.998215in}}%
\pgfpathlineto{\pgfqpoint{5.107345in}{1.489404in}}%
\pgfpathlineto{\pgfqpoint{5.108200in}{1.852793in}}%
\pgfpathlineto{\pgfqpoint{5.108758in}{1.614980in}}%
\pgfpathlineto{\pgfqpoint{5.108915in}{1.979830in}}%
\pgfpathlineto{\pgfqpoint{5.109263in}{1.885512in}}%
\pgfpathlineto{\pgfqpoint{5.109280in}{2.017742in}}%
\pgfpathlineto{\pgfqpoint{5.109402in}{1.602232in}}%
\pgfpathlineto{\pgfqpoint{5.110374in}{1.989377in}}%
\pgfpathlineto{\pgfqpoint{5.110877in}{1.401653in}}%
\pgfpathlineto{\pgfqpoint{5.111050in}{2.015227in}}%
\pgfpathlineto{\pgfqpoint{5.111483in}{1.946082in}}%
\pgfpathlineto{\pgfqpoint{5.112555in}{1.620569in}}%
\pgfpathlineto{\pgfqpoint{5.111552in}{1.999945in}}%
\pgfpathlineto{\pgfqpoint{5.112572in}{1.837728in}}%
\pgfpathlineto{\pgfqpoint{5.113193in}{1.978312in}}%
\pgfpathlineto{\pgfqpoint{5.112693in}{1.620698in}}%
\pgfpathlineto{\pgfqpoint{5.113675in}{1.922053in}}%
\pgfpathlineto{\pgfqpoint{5.114415in}{1.369022in}}%
\pgfpathlineto{\pgfqpoint{5.114019in}{2.010722in}}%
\pgfpathlineto{\pgfqpoint{5.114775in}{1.846258in}}%
\pgfpathlineto{\pgfqpoint{5.115650in}{2.004172in}}%
\pgfpathlineto{\pgfqpoint{5.115256in}{1.636918in}}%
\pgfpathlineto{\pgfqpoint{5.115873in}{1.965403in}}%
\pgfpathlineto{\pgfqpoint{5.116420in}{1.548573in}}%
\pgfpathlineto{\pgfqpoint{5.116010in}{2.009224in}}%
\pgfpathlineto{\pgfqpoint{5.116967in}{1.904920in}}%
\pgfpathlineto{\pgfqpoint{5.117616in}{2.013428in}}%
\pgfpathlineto{\pgfqpoint{5.117462in}{1.478235in}}%
\pgfpathlineto{\pgfqpoint{5.117957in}{1.586047in}}%
\pgfpathlineto{\pgfqpoint{5.117974in}{1.512538in}}%
\pgfpathlineto{\pgfqpoint{5.118195in}{1.994612in}}%
\pgfpathlineto{\pgfqpoint{5.119012in}{1.823767in}}%
\pgfpathlineto{\pgfqpoint{5.119505in}{1.468749in}}%
\pgfpathlineto{\pgfqpoint{5.119658in}{1.988673in}}%
\pgfpathlineto{\pgfqpoint{5.120082in}{1.879312in}}%
\pgfpathlineto{\pgfqpoint{5.120759in}{2.000016in}}%
\pgfpathlineto{\pgfqpoint{5.120776in}{1.217342in}}%
\pgfpathlineto{\pgfqpoint{5.121149in}{1.971179in}}%
\pgfpathlineto{\pgfqpoint{5.121267in}{1.653466in}}%
\pgfpathlineto{\pgfqpoint{5.121825in}{2.037707in}}%
\pgfpathlineto{\pgfqpoint{5.122264in}{1.696540in}}%
\pgfpathlineto{\pgfqpoint{5.122533in}{1.979410in}}%
\pgfpathlineto{\pgfqpoint{5.122988in}{1.515840in}}%
\pgfpathlineto{\pgfqpoint{5.123375in}{1.762399in}}%
\pgfpathlineto{\pgfqpoint{5.124182in}{1.648541in}}%
\pgfpathlineto{\pgfqpoint{5.123611in}{1.990914in}}%
\pgfpathlineto{\pgfqpoint{5.124451in}{1.776548in}}%
\pgfpathlineto{\pgfqpoint{5.125507in}{2.010409in}}%
\pgfpathlineto{\pgfqpoint{5.125306in}{1.490262in}}%
\pgfpathlineto{\pgfqpoint{5.125524in}{1.908985in}}%
\pgfpathlineto{\pgfqpoint{5.126176in}{1.484137in}}%
\pgfpathlineto{\pgfqpoint{5.126276in}{1.992411in}}%
\pgfpathlineto{\pgfqpoint{5.126627in}{1.614363in}}%
\pgfpathlineto{\pgfqpoint{5.126677in}{1.992275in}}%
\pgfpathlineto{\pgfqpoint{5.126944in}{1.098569in}}%
\pgfpathlineto{\pgfqpoint{5.127744in}{1.771768in}}%
\pgfpathlineto{\pgfqpoint{5.127894in}{2.004717in}}%
\pgfpathlineto{\pgfqpoint{5.127811in}{1.652636in}}%
\pgfpathlineto{\pgfqpoint{5.128825in}{1.775366in}}%
\pgfpathlineto{\pgfqpoint{5.128842in}{1.537534in}}%
\pgfpathlineto{\pgfqpoint{5.129555in}{2.015011in}}%
\pgfpathlineto{\pgfqpoint{5.129920in}{1.840495in}}%
\pgfpathlineto{\pgfqpoint{5.130748in}{1.573269in}}%
\pgfpathlineto{\pgfqpoint{5.130830in}{2.007686in}}%
\pgfpathlineto{\pgfqpoint{5.131012in}{1.750124in}}%
\pgfpathlineto{\pgfqpoint{5.131194in}{1.989551in}}%
\pgfpathlineto{\pgfqpoint{5.131557in}{1.505380in}}%
\pgfpathlineto{\pgfqpoint{5.132101in}{1.806612in}}%
\pgfpathlineto{\pgfqpoint{5.133122in}{1.480230in}}%
\pgfpathlineto{\pgfqpoint{5.133139in}{1.988749in}}%
\pgfpathlineto{\pgfqpoint{5.133188in}{1.817272in}}%
\pgfpathlineto{\pgfqpoint{5.133599in}{1.994350in}}%
\pgfpathlineto{\pgfqpoint{5.134157in}{1.648022in}}%
\pgfpathlineto{\pgfqpoint{5.134321in}{1.953175in}}%
\pgfpathlineto{\pgfqpoint{5.134337in}{1.951388in}}%
\pgfpathlineto{\pgfqpoint{5.134550in}{1.500107in}}%
\pgfpathlineto{\pgfqpoint{5.134616in}{1.987859in}}%
\pgfpathlineto{\pgfqpoint{5.135467in}{1.710037in}}%
\pgfpathlineto{\pgfqpoint{5.135680in}{2.027220in}}%
\pgfpathlineto{\pgfqpoint{5.136186in}{1.397256in}}%
\pgfpathlineto{\pgfqpoint{5.136578in}{1.961940in}}%
\pgfpathlineto{\pgfqpoint{5.137115in}{1.574934in}}%
\pgfpathlineto{\pgfqpoint{5.136855in}{2.001511in}}%
\pgfpathlineto{\pgfqpoint{5.137734in}{1.698434in}}%
\pgfpathlineto{\pgfqpoint{5.137962in}{1.988131in}}%
\pgfpathlineto{\pgfqpoint{5.138059in}{1.596750in}}%
\pgfpathlineto{\pgfqpoint{5.138855in}{1.801384in}}%
\pgfpathlineto{\pgfqpoint{5.139390in}{1.982391in}}%
\pgfpathlineto{\pgfqpoint{5.139001in}{1.602422in}}%
\pgfpathlineto{\pgfqpoint{5.139940in}{1.923840in}}%
\pgfpathlineto{\pgfqpoint{5.140458in}{1.557688in}}%
\pgfpathlineto{\pgfqpoint{5.140991in}{1.982924in}}%
\pgfpathlineto{\pgfqpoint{5.141071in}{1.695255in}}%
\pgfpathlineto{\pgfqpoint{5.141297in}{2.017129in}}%
\pgfpathlineto{\pgfqpoint{5.142055in}{1.577902in}}%
\pgfpathlineto{\pgfqpoint{5.142200in}{1.807836in}}%
\pgfpathlineto{\pgfqpoint{5.143084in}{1.656032in}}%
\pgfpathlineto{\pgfqpoint{5.142779in}{1.978310in}}%
\pgfpathlineto{\pgfqpoint{5.143293in}{1.808239in}}%
\pgfpathlineto{\pgfqpoint{5.143774in}{1.997883in}}%
\pgfpathlineto{\pgfqpoint{5.143758in}{1.422099in}}%
\pgfpathlineto{\pgfqpoint{5.144271in}{1.820726in}}%
\pgfpathlineto{\pgfqpoint{5.144463in}{1.665906in}}%
\pgfpathlineto{\pgfqpoint{5.145359in}{2.020097in}}%
\pgfpathlineto{\pgfqpoint{5.146204in}{1.636442in}}%
\pgfpathlineto{\pgfqpoint{5.146491in}{1.804366in}}%
\pgfpathlineto{\pgfqpoint{5.147176in}{1.967588in}}%
\pgfpathlineto{\pgfqpoint{5.146921in}{1.465558in}}%
\pgfpathlineto{\pgfqpoint{5.147621in}{1.939666in}}%
\pgfpathlineto{\pgfqpoint{5.148002in}{1.603879in}}%
\pgfpathlineto{\pgfqpoint{5.147732in}{2.005042in}}%
\pgfpathlineto{\pgfqpoint{5.148779in}{1.747223in}}%
\pgfpathlineto{\pgfqpoint{5.149270in}{1.968839in}}%
\pgfpathlineto{\pgfqpoint{5.149698in}{1.509170in}}%
\pgfpathlineto{\pgfqpoint{5.149887in}{1.761863in}}%
\pgfpathlineto{\pgfqpoint{5.150503in}{1.584644in}}%
\pgfpathlineto{\pgfqpoint{5.150140in}{1.970763in}}%
\pgfpathlineto{\pgfqpoint{5.150850in}{1.883398in}}%
\pgfpathlineto{\pgfqpoint{5.150914in}{2.031577in}}%
\pgfpathlineto{\pgfqpoint{5.151607in}{1.543814in}}%
\pgfpathlineto{\pgfqpoint{5.151921in}{1.873172in}}%
\pgfpathlineto{\pgfqpoint{5.152205in}{1.610057in}}%
\pgfpathlineto{\pgfqpoint{5.152723in}{1.977465in}}%
\pgfpathlineto{\pgfqpoint{5.153021in}{1.927463in}}%
\pgfpathlineto{\pgfqpoint{5.153962in}{1.507487in}}%
\pgfpathlineto{\pgfqpoint{5.153523in}{2.029125in}}%
\pgfpathlineto{\pgfqpoint{5.154118in}{1.873578in}}%
\pgfpathlineto{\pgfqpoint{5.155196in}{1.971044in}}%
\pgfpathlineto{\pgfqpoint{5.154697in}{1.374825in}}%
\pgfpathlineto{\pgfqpoint{5.155212in}{1.894910in}}%
\pgfpathlineto{\pgfqpoint{5.155742in}{1.510413in}}%
\pgfpathlineto{\pgfqpoint{5.155633in}{2.004381in}}%
\pgfpathlineto{\pgfqpoint{5.156303in}{1.866018in}}%
\pgfpathlineto{\pgfqpoint{5.157205in}{1.988711in}}%
\pgfpathlineto{\pgfqpoint{5.156443in}{1.596613in}}%
\pgfpathlineto{\pgfqpoint{5.157407in}{1.900862in}}%
\pgfpathlineto{\pgfqpoint{5.158508in}{1.573389in}}%
\pgfpathlineto{\pgfqpoint{5.158074in}{1.978159in}}%
\pgfpathlineto{\pgfqpoint{5.158524in}{1.732210in}}%
\pgfpathlineto{\pgfqpoint{5.158663in}{1.989514in}}%
\pgfpathlineto{\pgfqpoint{5.158973in}{1.376210in}}%
\pgfpathlineto{\pgfqpoint{5.159653in}{1.888347in}}%
\pgfpathlineto{\pgfqpoint{5.160455in}{1.603575in}}%
\pgfpathlineto{\pgfqpoint{5.160039in}{1.997168in}}%
\pgfpathlineto{\pgfqpoint{5.160764in}{1.742751in}}%
\pgfpathlineto{\pgfqpoint{5.160779in}{2.002798in}}%
\pgfpathlineto{\pgfqpoint{5.161564in}{1.548723in}}%
\pgfpathlineto{\pgfqpoint{5.161871in}{1.713498in}}%
\pgfpathlineto{\pgfqpoint{5.162394in}{1.994210in}}%
\pgfpathlineto{\pgfqpoint{5.162163in}{1.649482in}}%
\pgfpathlineto{\pgfqpoint{5.162992in}{1.867493in}}%
\pgfpathlineto{\pgfqpoint{5.163665in}{1.402456in}}%
\pgfpathlineto{\pgfqpoint{5.163834in}{1.963038in}}%
\pgfpathlineto{\pgfqpoint{5.164078in}{1.735380in}}%
\pgfpathlineto{\pgfqpoint{5.164598in}{1.995561in}}%
\pgfpathlineto{\pgfqpoint{5.165132in}{1.556934in}}%
\pgfpathlineto{\pgfqpoint{5.165178in}{1.856867in}}%
\pgfpathlineto{\pgfqpoint{5.165604in}{1.575907in}}%
\pgfpathlineto{\pgfqpoint{5.166076in}{1.988084in}}%
\pgfpathlineto{\pgfqpoint{5.166289in}{1.814524in}}%
\pgfpathlineto{\pgfqpoint{5.166654in}{1.345383in}}%
\pgfpathlineto{\pgfqpoint{5.167003in}{1.958668in}}%
\pgfpathlineto{\pgfqpoint{5.167322in}{1.864401in}}%
\pgfpathlineto{\pgfqpoint{5.167928in}{1.985907in}}%
\pgfpathlineto{\pgfqpoint{5.167807in}{1.551147in}}%
\pgfpathlineto{\pgfqpoint{5.168428in}{1.908432in}}%
\pgfpathlineto{\pgfqpoint{5.168549in}{1.471852in}}%
\pgfpathlineto{\pgfqpoint{5.168715in}{1.990991in}}%
\pgfpathlineto{\pgfqpoint{5.169531in}{1.829542in}}%
\pgfpathlineto{\pgfqpoint{5.170616in}{1.987176in}}%
\pgfpathlineto{\pgfqpoint{5.170089in}{1.553861in}}%
\pgfpathlineto{\pgfqpoint{5.170631in}{1.817005in}}%
\pgfpathlineto{\pgfqpoint{5.171684in}{2.009443in}}%
\pgfpathlineto{\pgfqpoint{5.171067in}{1.616404in}}%
\pgfpathlineto{\pgfqpoint{5.171699in}{1.921271in}}%
\pgfpathlineto{\pgfqpoint{5.171744in}{1.484863in}}%
\pgfpathlineto{\pgfqpoint{5.172628in}{2.014081in}}%
\pgfpathlineto{\pgfqpoint{5.172808in}{1.927554in}}%
\pgfpathlineto{\pgfqpoint{5.173780in}{1.480341in}}%
\pgfpathlineto{\pgfqpoint{5.173407in}{2.026165in}}%
\pgfpathlineto{\pgfqpoint{5.173900in}{1.806804in}}%
\pgfpathlineto{\pgfqpoint{5.173915in}{1.984421in}}%
\pgfpathlineto{\pgfqpoint{5.174795in}{1.493345in}}%
\pgfpathlineto{\pgfqpoint{5.175004in}{1.825516in}}%
\pgfpathlineto{\pgfqpoint{5.175138in}{2.007170in}}%
\pgfpathlineto{\pgfqpoint{5.175376in}{1.592785in}}%
\pgfpathlineto{\pgfqpoint{5.176135in}{1.960613in}}%
\pgfpathlineto{\pgfqpoint{5.176149in}{1.542341in}}%
\pgfpathlineto{\pgfqpoint{5.176892in}{1.981939in}}%
\pgfpathlineto{\pgfqpoint{5.177247in}{1.647744in}}%
\pgfpathlineto{\pgfqpoint{5.178047in}{1.996055in}}%
\pgfpathlineto{\pgfqpoint{5.178136in}{1.598703in}}%
\pgfpathlineto{\pgfqpoint{5.178357in}{1.829451in}}%
\pgfpathlineto{\pgfqpoint{5.178948in}{1.474213in}}%
\pgfpathlineto{\pgfqpoint{5.178697in}{1.985681in}}%
\pgfpathlineto{\pgfqpoint{5.179450in}{1.889552in}}%
\pgfpathlineto{\pgfqpoint{5.180245in}{1.585550in}}%
\pgfpathlineto{\pgfqpoint{5.180068in}{1.982185in}}%
\pgfpathlineto{\pgfqpoint{5.180525in}{1.772670in}}%
\pgfpathlineto{\pgfqpoint{5.181083in}{1.974240in}}%
\pgfpathlineto{\pgfqpoint{5.181465in}{1.563745in}}%
\pgfpathlineto{\pgfqpoint{5.181641in}{1.911373in}}%
\pgfpathlineto{\pgfqpoint{5.182417in}{1.543356in}}%
\pgfpathlineto{\pgfqpoint{5.182637in}{1.964526in}}%
\pgfpathlineto{\pgfqpoint{5.182739in}{1.823349in}}%
\pgfpathlineto{\pgfqpoint{5.183558in}{2.013283in}}%
\pgfpathlineto{\pgfqpoint{5.183631in}{1.488168in}}%
\pgfpathlineto{\pgfqpoint{5.183835in}{1.884957in}}%
\pgfpathlineto{\pgfqpoint{5.184229in}{1.615322in}}%
\pgfpathlineto{\pgfqpoint{5.184404in}{1.988671in}}%
\pgfpathlineto{\pgfqpoint{5.184942in}{1.805814in}}%
\pgfpathlineto{\pgfqpoint{5.185306in}{2.010208in}}%
\pgfpathlineto{\pgfqpoint{5.185611in}{1.421363in}}%
\pgfpathlineto{\pgfqpoint{5.186047in}{1.878515in}}%
\pgfpathlineto{\pgfqpoint{5.187120in}{1.336458in}}%
\pgfpathlineto{\pgfqpoint{5.186323in}{1.976222in}}%
\pgfpathlineto{\pgfqpoint{5.187149in}{1.874830in}}%
\pgfpathlineto{\pgfqpoint{5.187756in}{1.390472in}}%
\pgfpathlineto{\pgfqpoint{5.187453in}{2.000647in}}%
\pgfpathlineto{\pgfqpoint{5.188233in}{1.838893in}}%
\pgfpathlineto{\pgfqpoint{5.188421in}{1.966917in}}%
\pgfpathlineto{\pgfqpoint{5.188753in}{1.588108in}}%
\pgfpathlineto{\pgfqpoint{5.189344in}{1.860086in}}%
\pgfpathlineto{\pgfqpoint{5.189862in}{1.611011in}}%
\pgfpathlineto{\pgfqpoint{5.189991in}{2.007769in}}%
\pgfpathlineto{\pgfqpoint{5.190451in}{1.767009in}}%
\pgfpathlineto{\pgfqpoint{5.191298in}{2.008872in}}%
\pgfpathlineto{\pgfqpoint{5.191456in}{1.463887in}}%
\pgfpathlineto{\pgfqpoint{5.191570in}{1.839694in}}%
\pgfpathlineto{\pgfqpoint{5.192243in}{1.533885in}}%
\pgfpathlineto{\pgfqpoint{5.192100in}{2.015609in}}%
\pgfpathlineto{\pgfqpoint{5.192672in}{1.778474in}}%
\pgfpathlineto{\pgfqpoint{5.193343in}{2.005468in}}%
\pgfpathlineto{\pgfqpoint{5.193543in}{1.633428in}}%
\pgfpathlineto{\pgfqpoint{5.193785in}{1.955272in}}%
\pgfpathlineto{\pgfqpoint{5.194227in}{1.529195in}}%
\pgfpathlineto{\pgfqpoint{5.194682in}{1.991411in}}%
\pgfpathlineto{\pgfqpoint{5.194910in}{1.801284in}}%
\pgfpathlineto{\pgfqpoint{5.195265in}{2.047397in}}%
\pgfpathlineto{\pgfqpoint{5.195733in}{1.499748in}}%
\pgfpathlineto{\pgfqpoint{5.196017in}{1.873611in}}%
\pgfpathlineto{\pgfqpoint{5.197051in}{1.542405in}}%
\pgfpathlineto{\pgfqpoint{5.196584in}{1.978614in}}%
\pgfpathlineto{\pgfqpoint{5.197121in}{1.791247in}}%
\pgfpathlineto{\pgfqpoint{5.198195in}{2.015102in}}%
\pgfpathlineto{\pgfqpoint{5.197277in}{1.469403in}}%
\pgfpathlineto{\pgfqpoint{5.198223in}{1.829859in}}%
\pgfpathlineto{\pgfqpoint{5.199167in}{1.607174in}}%
\pgfpathlineto{\pgfqpoint{5.198519in}{1.997520in}}%
\pgfpathlineto{\pgfqpoint{5.199308in}{1.867302in}}%
\pgfpathlineto{\pgfqpoint{5.199842in}{1.981216in}}%
\pgfpathlineto{\pgfqpoint{5.200123in}{1.436839in}}%
\pgfpathlineto{\pgfqpoint{5.200347in}{1.881010in}}%
\pgfpathlineto{\pgfqpoint{5.200614in}{1.488714in}}%
\pgfpathlineto{\pgfqpoint{5.200488in}{1.974632in}}%
\pgfpathlineto{\pgfqpoint{5.201455in}{1.770389in}}%
\pgfpathlineto{\pgfqpoint{5.202433in}{2.018807in}}%
\pgfpathlineto{\pgfqpoint{5.202014in}{1.329251in}}%
\pgfpathlineto{\pgfqpoint{5.202573in}{1.943799in}}%
\pgfpathlineto{\pgfqpoint{5.203396in}{1.413442in}}%
\pgfpathlineto{\pgfqpoint{5.202601in}{1.996738in}}%
\pgfpathlineto{\pgfqpoint{5.203702in}{1.619828in}}%
\pgfpathlineto{\pgfqpoint{5.204551in}{1.960635in}}%
\pgfpathlineto{\pgfqpoint{5.204676in}{1.587568in}}%
\pgfpathlineto{\pgfqpoint{5.204829in}{1.951852in}}%
\pgfpathlineto{\pgfqpoint{5.205328in}{1.674243in}}%
\pgfpathlineto{\pgfqpoint{5.205148in}{1.968877in}}%
\pgfpathlineto{\pgfqpoint{5.205952in}{1.845092in}}%
\pgfpathlineto{\pgfqpoint{5.206270in}{2.000350in}}%
\pgfpathlineto{\pgfqpoint{5.206865in}{1.543036in}}%
\pgfpathlineto{\pgfqpoint{5.207017in}{1.845245in}}%
\pgfpathlineto{\pgfqpoint{5.207431in}{1.412548in}}%
\pgfpathlineto{\pgfqpoint{5.207749in}{1.979422in}}%
\pgfpathlineto{\pgfqpoint{5.208135in}{1.728271in}}%
\pgfpathlineto{\pgfqpoint{5.208603in}{1.964028in}}%
\pgfpathlineto{\pgfqpoint{5.208424in}{1.218469in}}%
\pgfpathlineto{\pgfqpoint{5.209277in}{1.825075in}}%
\pgfpathlineto{\pgfqpoint{5.209538in}{1.293339in}}%
\pgfpathlineto{\pgfqpoint{5.209785in}{1.982393in}}%
\pgfpathlineto{\pgfqpoint{5.210388in}{1.800156in}}%
\pgfpathlineto{\pgfqpoint{5.210936in}{1.958444in}}%
\pgfpathlineto{\pgfqpoint{5.211429in}{1.499359in}}%
\pgfpathlineto{\pgfqpoint{5.211497in}{1.812459in}}%
\pgfpathlineto{\pgfqpoint{5.212016in}{1.997897in}}%
\pgfpathlineto{\pgfqpoint{5.212057in}{1.558093in}}%
\pgfpathlineto{\pgfqpoint{5.212589in}{1.804769in}}%
\pgfpathlineto{\pgfqpoint{5.213039in}{1.561774in}}%
\pgfpathlineto{\pgfqpoint{5.212821in}{1.969765in}}%
\pgfpathlineto{\pgfqpoint{5.213665in}{1.784521in}}%
\pgfpathlineto{\pgfqpoint{5.213991in}{2.018341in}}%
\pgfpathlineto{\pgfqpoint{5.213828in}{1.652047in}}%
\pgfpathlineto{\pgfqpoint{5.214765in}{1.770579in}}%
\pgfpathlineto{\pgfqpoint{5.215579in}{1.425297in}}%
\pgfpathlineto{\pgfqpoint{5.215240in}{1.949575in}}%
\pgfpathlineto{\pgfqpoint{5.215822in}{1.844026in}}%
\pgfpathlineto{\pgfqpoint{5.216728in}{2.002168in}}%
\pgfpathlineto{\pgfqpoint{5.216268in}{1.603279in}}%
\pgfpathlineto{\pgfqpoint{5.216890in}{1.905652in}}%
\pgfpathlineto{\pgfqpoint{5.217537in}{1.527607in}}%
\pgfpathlineto{\pgfqpoint{5.217712in}{1.955803in}}%
\pgfpathlineto{\pgfqpoint{5.217995in}{1.914398in}}%
\pgfpathlineto{\pgfqpoint{5.218143in}{1.363667in}}%
\pgfpathlineto{\pgfqpoint{5.218587in}{1.954113in}}%
\pgfpathlineto{\pgfqpoint{5.219165in}{1.726385in}}%
\pgfpathlineto{\pgfqpoint{5.220077in}{1.979873in}}%
\pgfpathlineto{\pgfqpoint{5.219849in}{1.264394in}}%
\pgfpathlineto{\pgfqpoint{5.220278in}{1.861028in}}%
\pgfpathlineto{\pgfqpoint{5.220412in}{1.948008in}}%
\pgfpathlineto{\pgfqpoint{5.220545in}{1.589725in}}%
\pgfpathlineto{\pgfqpoint{5.220853in}{1.888330in}}%
\pgfpathlineto{\pgfqpoint{5.221574in}{1.526589in}}%
\pgfpathlineto{\pgfqpoint{5.221013in}{1.986809in}}%
\pgfpathlineto{\pgfqpoint{5.221961in}{1.848639in}}%
\pgfpathlineto{\pgfqpoint{5.222135in}{1.998548in}}%
\pgfpathlineto{\pgfqpoint{5.222068in}{1.551457in}}%
\pgfpathlineto{\pgfqpoint{5.222228in}{1.783650in}}%
\pgfpathlineto{\pgfqpoint{5.222241in}{1.590601in}}%
\pgfpathlineto{\pgfqpoint{5.223000in}{1.953440in}}%
\pgfpathlineto{\pgfqpoint{5.223333in}{1.829299in}}%
\pgfpathlineto{\pgfqpoint{5.223745in}{1.969683in}}%
\pgfpathlineto{\pgfqpoint{5.223904in}{1.539063in}}%
\pgfpathlineto{\pgfqpoint{5.224422in}{1.910063in}}%
\pgfpathlineto{\pgfqpoint{5.224859in}{1.636583in}}%
\pgfpathlineto{\pgfqpoint{5.225389in}{2.009368in}}%
\pgfpathlineto{\pgfqpoint{5.225534in}{1.861916in}}%
\pgfpathlineto{\pgfqpoint{5.225666in}{1.974960in}}%
\pgfpathlineto{\pgfqpoint{5.226591in}{1.539341in}}%
\pgfpathlineto{\pgfqpoint{5.226630in}{1.886567in}}%
\pgfpathlineto{\pgfqpoint{5.226828in}{1.939961in}}%
\pgfpathlineto{\pgfqpoint{5.227724in}{1.474933in}}%
\pgfpathlineto{\pgfqpoint{5.227737in}{1.942872in}}%
\pgfpathlineto{\pgfqpoint{5.228814in}{1.331856in}}%
\pgfpathlineto{\pgfqpoint{5.228841in}{1.752143in}}%
\pgfpathlineto{\pgfqpoint{5.229365in}{1.949408in}}%
\pgfpathlineto{\pgfqpoint{5.228906in}{1.524860in}}%
\pgfpathlineto{\pgfqpoint{5.229942in}{1.917787in}}%
\pgfpathlineto{\pgfqpoint{5.230987in}{1.279588in}}%
\pgfpathlineto{\pgfqpoint{5.230413in}{1.954263in}}%
\pgfpathlineto{\pgfqpoint{5.231053in}{1.573396in}}%
\pgfpathlineto{\pgfqpoint{5.231640in}{1.950244in}}%
\pgfpathlineto{\pgfqpoint{5.232005in}{1.462115in}}%
\pgfpathlineto{\pgfqpoint{5.232161in}{1.851563in}}%
\pgfpathlineto{\pgfqpoint{5.232525in}{1.483269in}}%
\pgfpathlineto{\pgfqpoint{5.232785in}{1.964463in}}%
\pgfpathlineto{\pgfqpoint{5.233266in}{1.818497in}}%
\pgfpathlineto{\pgfqpoint{5.234071in}{1.956989in}}%
\pgfpathlineto{\pgfqpoint{5.234317in}{1.492651in}}%
\pgfpathlineto{\pgfqpoint{5.234369in}{1.911714in}}%
\pgfpathlineto{\pgfqpoint{5.234705in}{1.590487in}}%
\pgfpathlineto{\pgfqpoint{5.234395in}{1.962464in}}%
\pgfpathlineto{\pgfqpoint{5.235494in}{1.723597in}}%
\pgfpathlineto{\pgfqpoint{5.236152in}{1.970727in}}%
\pgfpathlineto{\pgfqpoint{5.236475in}{1.502032in}}%
\pgfpathlineto{\pgfqpoint{5.236578in}{1.861840in}}%
\pgfpathlineto{\pgfqpoint{5.236604in}{1.500252in}}%
\pgfpathlineto{\pgfqpoint{5.236668in}{1.938131in}}%
\pgfpathlineto{\pgfqpoint{5.237685in}{1.888117in}}%
\pgfpathlineto{\pgfqpoint{5.238583in}{1.977497in}}%
\pgfpathlineto{\pgfqpoint{5.238404in}{1.537087in}}%
\pgfpathlineto{\pgfqpoint{5.238750in}{1.851434in}}%
\pgfpathlineto{\pgfqpoint{5.239774in}{1.334493in}}%
\pgfpathlineto{\pgfqpoint{5.239493in}{1.955392in}}%
\pgfpathlineto{\pgfqpoint{5.239851in}{1.822828in}}%
\pgfpathlineto{\pgfqpoint{5.240081in}{1.950982in}}%
\pgfpathlineto{\pgfqpoint{5.240209in}{1.624786in}}%
\pgfpathlineto{\pgfqpoint{5.240898in}{1.828639in}}%
\pgfpathlineto{\pgfqpoint{5.240911in}{1.418527in}}%
\pgfpathlineto{\pgfqpoint{5.241574in}{1.948628in}}%
\pgfpathlineto{\pgfqpoint{5.242006in}{1.851234in}}%
\pgfpathlineto{\pgfqpoint{5.242947in}{1.430151in}}%
\pgfpathlineto{\pgfqpoint{5.242604in}{1.938594in}}%
\pgfpathlineto{\pgfqpoint{5.243124in}{1.795843in}}%
\pgfpathlineto{\pgfqpoint{5.244176in}{1.996715in}}%
\pgfpathlineto{\pgfqpoint{5.243733in}{1.571753in}}%
\pgfpathlineto{\pgfqpoint{5.244202in}{1.874516in}}%
\pgfpathlineto{\pgfqpoint{5.244493in}{1.485845in}}%
\pgfpathlineto{\pgfqpoint{5.244783in}{1.942031in}}%
\pgfpathlineto{\pgfqpoint{5.245314in}{1.724345in}}%
\pgfpathlineto{\pgfqpoint{5.245793in}{1.966136in}}%
\pgfpathlineto{\pgfqpoint{5.246083in}{1.414593in}}%
\pgfpathlineto{\pgfqpoint{5.246411in}{1.792656in}}%
\pgfpathlineto{\pgfqpoint{5.246423in}{1.337825in}}%
\pgfpathlineto{\pgfqpoint{5.246851in}{1.941824in}}%
\pgfpathlineto{\pgfqpoint{5.247517in}{1.708994in}}%
\pgfpathlineto{\pgfqpoint{5.247580in}{2.021871in}}%
\pgfpathlineto{\pgfqpoint{5.247906in}{1.541390in}}%
\pgfpathlineto{\pgfqpoint{5.248646in}{1.923801in}}%
\pgfpathlineto{\pgfqpoint{5.249609in}{1.536068in}}%
\pgfpathlineto{\pgfqpoint{5.248971in}{1.975980in}}%
\pgfpathlineto{\pgfqpoint{5.249759in}{1.842055in}}%
\pgfpathlineto{\pgfqpoint{5.250171in}{1.976266in}}%
\pgfpathlineto{\pgfqpoint{5.250271in}{1.380444in}}%
\pgfpathlineto{\pgfqpoint{5.250795in}{1.769408in}}%
\pgfpathlineto{\pgfqpoint{5.251119in}{1.359223in}}%
\pgfpathlineto{\pgfqpoint{5.251629in}{1.957579in}}%
\pgfpathlineto{\pgfqpoint{5.251890in}{1.751887in}}%
\pgfpathlineto{\pgfqpoint{5.251977in}{2.003327in}}%
\pgfpathlineto{\pgfqpoint{5.252858in}{1.545391in}}%
\pgfpathlineto{\pgfqpoint{5.253019in}{1.902794in}}%
\pgfpathlineto{\pgfqpoint{5.253701in}{1.535666in}}%
\pgfpathlineto{\pgfqpoint{5.253589in}{1.962795in}}%
\pgfpathlineto{\pgfqpoint{5.254134in}{1.873740in}}%
\pgfpathlineto{\pgfqpoint{5.254838in}{1.588789in}}%
\pgfpathlineto{\pgfqpoint{5.255010in}{1.949656in}}%
\pgfpathlineto{\pgfqpoint{5.255257in}{1.810806in}}%
\pgfpathlineto{\pgfqpoint{5.255553in}{1.958199in}}%
\pgfpathlineto{\pgfqpoint{5.255688in}{1.530812in}}%
\pgfpathlineto{\pgfqpoint{5.256353in}{1.750055in}}%
\pgfpathlineto{\pgfqpoint{5.256685in}{1.462000in}}%
\pgfpathlineto{\pgfqpoint{5.257164in}{1.939626in}}%
\pgfpathlineto{\pgfqpoint{5.257446in}{1.772619in}}%
\pgfpathlineto{\pgfqpoint{5.257643in}{2.003006in}}%
\pgfpathlineto{\pgfqpoint{5.258378in}{1.438780in}}%
\pgfpathlineto{\pgfqpoint{5.258549in}{1.758941in}}%
\pgfpathlineto{\pgfqpoint{5.258647in}{1.471989in}}%
\pgfpathlineto{\pgfqpoint{5.259026in}{1.960077in}}%
\pgfpathlineto{\pgfqpoint{5.259624in}{1.862250in}}%
\pgfpathlineto{\pgfqpoint{5.259746in}{1.935673in}}%
\pgfpathlineto{\pgfqpoint{5.260149in}{1.417216in}}%
\pgfpathlineto{\pgfqpoint{5.260733in}{1.916013in}}%
\pgfpathlineto{\pgfqpoint{5.260758in}{1.578934in}}%
\pgfpathlineto{\pgfqpoint{5.261244in}{1.952661in}}%
\pgfpathlineto{\pgfqpoint{5.261840in}{1.876855in}}%
\pgfpathlineto{\pgfqpoint{5.262870in}{1.954343in}}%
\pgfpathlineto{\pgfqpoint{5.262798in}{1.422036in}}%
\pgfpathlineto{\pgfqpoint{5.262931in}{1.812627in}}%
\pgfpathlineto{\pgfqpoint{5.263488in}{1.944993in}}%
\pgfpathlineto{\pgfqpoint{5.263705in}{1.516768in}}%
\pgfpathlineto{\pgfqpoint{5.263935in}{1.855331in}}%
\pgfpathlineto{\pgfqpoint{5.264068in}{1.329166in}}%
\pgfpathlineto{\pgfqpoint{5.264019in}{1.970947in}}%
\pgfpathlineto{\pgfqpoint{5.265057in}{1.640468in}}%
\pgfpathlineto{\pgfqpoint{5.265069in}{1.636312in}}%
\pgfpathlineto{\pgfqpoint{5.265202in}{1.959831in}}%
\pgfpathlineto{\pgfqpoint{5.265442in}{1.726736in}}%
\pgfpathlineto{\pgfqpoint{5.265936in}{1.976204in}}%
\pgfpathlineto{\pgfqpoint{5.266092in}{1.544540in}}%
\pgfpathlineto{\pgfqpoint{5.266548in}{1.794388in}}%
\pgfpathlineto{\pgfqpoint{5.267628in}{1.528047in}}%
\pgfpathlineto{\pgfqpoint{5.266608in}{1.980830in}}%
\pgfpathlineto{\pgfqpoint{5.267652in}{1.864895in}}%
\pgfpathlineto{\pgfqpoint{5.268202in}{1.982136in}}%
\pgfpathlineto{\pgfqpoint{5.267724in}{1.424634in}}%
\pgfpathlineto{\pgfqpoint{5.268740in}{1.860775in}}%
\pgfpathlineto{\pgfqpoint{5.269647in}{1.516912in}}%
\pgfpathlineto{\pgfqpoint{5.268991in}{2.005398in}}%
\pgfpathlineto{\pgfqpoint{5.269850in}{1.859318in}}%
\pgfpathlineto{\pgfqpoint{5.270421in}{1.475872in}}%
\pgfpathlineto{\pgfqpoint{5.270813in}{1.939809in}}%
\pgfpathlineto{\pgfqpoint{5.270992in}{1.700973in}}%
\pgfpathlineto{\pgfqpoint{5.271834in}{1.929784in}}%
\pgfpathlineto{\pgfqpoint{5.271989in}{1.485904in}}%
\pgfpathlineto{\pgfqpoint{5.272107in}{1.837406in}}%
\pgfpathlineto{\pgfqpoint{5.273042in}{1.957205in}}%
\pgfpathlineto{\pgfqpoint{5.272605in}{1.555215in}}%
\pgfpathlineto{\pgfqpoint{5.273196in}{1.915582in}}%
\pgfpathlineto{\pgfqpoint{5.273243in}{1.438695in}}%
\pgfpathlineto{\pgfqpoint{5.273940in}{1.982421in}}%
\pgfpathlineto{\pgfqpoint{5.274306in}{1.843644in}}%
\pgfpathlineto{\pgfqpoint{5.275154in}{1.979242in}}%
\pgfpathlineto{\pgfqpoint{5.274801in}{1.517914in}}%
\pgfpathlineto{\pgfqpoint{5.275354in}{1.890760in}}%
\pgfpathlineto{\pgfqpoint{5.276329in}{1.459925in}}%
\pgfpathlineto{\pgfqpoint{5.276059in}{1.935370in}}%
\pgfpathlineto{\pgfqpoint{5.276469in}{1.746796in}}%
\pgfpathlineto{\pgfqpoint{5.277255in}{1.526776in}}%
\pgfpathlineto{\pgfqpoint{5.277091in}{1.987569in}}%
\pgfpathlineto{\pgfqpoint{5.277571in}{1.674335in}}%
\pgfpathlineto{\pgfqpoint{5.278552in}{1.954073in}}%
\pgfpathlineto{\pgfqpoint{5.278634in}{1.443139in}}%
\pgfpathlineto{\pgfqpoint{5.278681in}{1.877336in}}%
\pgfpathlineto{\pgfqpoint{5.279368in}{1.464257in}}%
\pgfpathlineto{\pgfqpoint{5.279065in}{1.938789in}}%
\pgfpathlineto{\pgfqpoint{5.279788in}{1.801662in}}%
\pgfpathlineto{\pgfqpoint{5.279974in}{1.937605in}}%
\pgfpathlineto{\pgfqpoint{5.280369in}{1.552483in}}%
\pgfpathlineto{\pgfqpoint{5.280752in}{1.800192in}}%
\pgfpathlineto{\pgfqpoint{5.281680in}{1.505944in}}%
\pgfpathlineto{\pgfqpoint{5.281611in}{1.984696in}}%
\pgfpathlineto{\pgfqpoint{5.281854in}{1.841236in}}%
\pgfpathlineto{\pgfqpoint{5.282433in}{1.394941in}}%
\pgfpathlineto{\pgfqpoint{5.282837in}{1.942575in}}%
\pgfpathlineto{\pgfqpoint{5.282965in}{1.646318in}}%
\pgfpathlineto{\pgfqpoint{5.283807in}{1.945680in}}%
\pgfpathlineto{\pgfqpoint{5.283126in}{1.521339in}}%
\pgfpathlineto{\pgfqpoint{5.284095in}{1.922732in}}%
\pgfpathlineto{\pgfqpoint{5.284268in}{1.469488in}}%
\pgfpathlineto{\pgfqpoint{5.285165in}{1.977834in}}%
\pgfpathlineto{\pgfqpoint{5.285211in}{1.545682in}}%
\pgfpathlineto{\pgfqpoint{5.285923in}{1.952799in}}%
\pgfpathlineto{\pgfqpoint{5.285670in}{1.537918in}}%
\pgfpathlineto{\pgfqpoint{5.286324in}{1.861918in}}%
\pgfpathlineto{\pgfqpoint{5.286679in}{1.986332in}}%
\pgfpathlineto{\pgfqpoint{5.287194in}{1.518260in}}%
\pgfpathlineto{\pgfqpoint{5.287377in}{1.882401in}}%
\pgfpathlineto{\pgfqpoint{5.287800in}{1.958869in}}%
\pgfpathlineto{\pgfqpoint{5.288496in}{1.415092in}}%
\pgfpathlineto{\pgfqpoint{5.289100in}{1.925987in}}%
\pgfpathlineto{\pgfqpoint{5.288735in}{1.361948in}}%
\pgfpathlineto{\pgfqpoint{5.289623in}{1.816554in}}%
\pgfpathlineto{\pgfqpoint{5.289692in}{1.982275in}}%
\pgfpathlineto{\pgfqpoint{5.289703in}{1.553928in}}%
\pgfpathlineto{\pgfqpoint{5.290691in}{1.811782in}}%
\pgfpathlineto{\pgfqpoint{5.290827in}{1.466770in}}%
\pgfpathlineto{\pgfqpoint{5.290714in}{1.951532in}}%
\pgfpathlineto{\pgfqpoint{5.291790in}{1.804325in}}%
\pgfpathlineto{\pgfqpoint{5.292637in}{1.923806in}}%
\pgfpathlineto{\pgfqpoint{5.292784in}{1.566407in}}%
\pgfpathlineto{\pgfqpoint{5.292908in}{1.831556in}}%
\pgfpathlineto{\pgfqpoint{5.293066in}{1.941928in}}%
\pgfpathlineto{\pgfqpoint{5.293044in}{1.584592in}}%
\pgfpathlineto{\pgfqpoint{5.293844in}{1.785931in}}%
\pgfpathlineto{\pgfqpoint{5.294811in}{1.335496in}}%
\pgfpathlineto{\pgfqpoint{5.294361in}{1.946946in}}%
\pgfpathlineto{\pgfqpoint{5.294957in}{1.700958in}}%
\pgfpathlineto{\pgfqpoint{5.295159in}{1.950007in}}%
\pgfpathlineto{\pgfqpoint{5.295653in}{1.609010in}}%
\pgfpathlineto{\pgfqpoint{5.296067in}{1.740082in}}%
\pgfpathlineto{\pgfqpoint{5.296940in}{1.935939in}}%
\pgfpathlineto{\pgfqpoint{5.296224in}{1.593061in}}%
\pgfpathlineto{\pgfqpoint{5.297197in}{1.825671in}}%
\pgfpathlineto{\pgfqpoint{5.297599in}{1.472588in}}%
\pgfpathlineto{\pgfqpoint{5.297677in}{1.960575in}}%
\pgfpathlineto{\pgfqpoint{5.298123in}{1.635992in}}%
\pgfpathlineto{\pgfqpoint{5.298313in}{1.959543in}}%
\pgfpathlineto{\pgfqpoint{5.298535in}{1.491816in}}%
\pgfpathlineto{\pgfqpoint{5.299236in}{1.761188in}}%
\pgfpathlineto{\pgfqpoint{5.300003in}{1.964867in}}%
\pgfpathlineto{\pgfqpoint{5.299503in}{1.292821in}}%
\pgfpathlineto{\pgfqpoint{5.300347in}{1.756540in}}%
\pgfpathlineto{\pgfqpoint{5.301133in}{1.934641in}}%
\pgfpathlineto{\pgfqpoint{5.301100in}{1.495810in}}%
\pgfpathlineto{\pgfqpoint{5.301421in}{1.781382in}}%
\pgfpathlineto{\pgfqpoint{5.301973in}{1.441499in}}%
\pgfpathlineto{\pgfqpoint{5.302161in}{1.937429in}}%
\pgfpathlineto{\pgfqpoint{5.302525in}{1.775943in}}%
\pgfpathlineto{\pgfqpoint{5.303451in}{1.975430in}}%
\pgfpathlineto{\pgfqpoint{5.303418in}{1.255118in}}%
\pgfpathlineto{\pgfqpoint{5.303638in}{1.813216in}}%
\pgfpathlineto{\pgfqpoint{5.304012in}{1.390373in}}%
\pgfpathlineto{\pgfqpoint{5.303726in}{1.942040in}}%
\pgfpathlineto{\pgfqpoint{5.304726in}{1.735477in}}%
\pgfpathlineto{\pgfqpoint{5.305811in}{1.943934in}}%
\pgfpathlineto{\pgfqpoint{5.305329in}{1.344643in}}%
\pgfpathlineto{\pgfqpoint{5.305833in}{1.832076in}}%
\pgfpathlineto{\pgfqpoint{5.306489in}{1.585810in}}%
\pgfpathlineto{\pgfqpoint{5.306401in}{1.977746in}}%
\pgfpathlineto{\pgfqpoint{5.306948in}{1.785191in}}%
\pgfpathlineto{\pgfqpoint{5.307406in}{1.921365in}}%
\pgfpathlineto{\pgfqpoint{5.307766in}{1.496585in}}%
\pgfpathlineto{\pgfqpoint{5.307983in}{1.865421in}}%
\pgfpathlineto{\pgfqpoint{5.308484in}{1.340925in}}%
\pgfpathlineto{\pgfqpoint{5.309006in}{1.909920in}}%
\pgfpathlineto{\pgfqpoint{5.309093in}{1.832604in}}%
\pgfpathlineto{\pgfqpoint{5.309614in}{1.470245in}}%
\pgfpathlineto{\pgfqpoint{5.309581in}{1.927533in}}%
\pgfpathlineto{\pgfqpoint{5.310210in}{1.800034in}}%
\pgfpathlineto{\pgfqpoint{5.310513in}{1.907276in}}%
\pgfpathlineto{\pgfqpoint{5.310308in}{1.539322in}}%
\pgfpathlineto{\pgfqpoint{5.311141in}{1.839102in}}%
\pgfpathlineto{\pgfqpoint{5.311432in}{1.528792in}}%
\pgfpathlineto{\pgfqpoint{5.312026in}{1.938799in}}%
\pgfpathlineto{\pgfqpoint{5.312242in}{1.792237in}}%
\pgfpathlineto{\pgfqpoint{5.312425in}{1.952291in}}%
\pgfpathlineto{\pgfqpoint{5.312920in}{1.490705in}}%
\pgfpathlineto{\pgfqpoint{5.313351in}{1.819608in}}%
\pgfpathlineto{\pgfqpoint{5.313620in}{1.929185in}}%
\pgfpathlineto{\pgfqpoint{5.314457in}{1.491961in}}%
\pgfpathlineto{\pgfqpoint{5.315496in}{1.925044in}}%
\pgfpathlineto{\pgfqpoint{5.315560in}{1.771282in}}%
\pgfpathlineto{\pgfqpoint{5.316404in}{1.257778in}}%
\pgfpathlineto{\pgfqpoint{5.315699in}{1.954887in}}%
\pgfpathlineto{\pgfqpoint{5.316661in}{1.713684in}}%
\pgfpathlineto{\pgfqpoint{5.317279in}{1.429000in}}%
\pgfpathlineto{\pgfqpoint{5.317769in}{1.957375in}}%
\pgfpathlineto{\pgfqpoint{5.318609in}{1.558235in}}%
\pgfpathlineto{\pgfqpoint{5.318874in}{1.750242in}}%
\pgfpathlineto{\pgfqpoint{5.319648in}{1.949673in}}%
\pgfpathlineto{\pgfqpoint{5.319510in}{1.349406in}}%
\pgfpathlineto{\pgfqpoint{5.319987in}{1.781716in}}%
\pgfpathlineto{\pgfqpoint{5.320706in}{1.953177in}}%
\pgfpathlineto{\pgfqpoint{5.320051in}{1.448859in}}%
\pgfpathlineto{\pgfqpoint{5.321087in}{1.835703in}}%
\pgfpathlineto{\pgfqpoint{5.321846in}{1.500844in}}%
\pgfpathlineto{\pgfqpoint{5.321129in}{1.938434in}}%
\pgfpathlineto{\pgfqpoint{5.322204in}{1.708210in}}%
\pgfpathlineto{\pgfqpoint{5.322404in}{1.969088in}}%
\pgfpathlineto{\pgfqpoint{5.322951in}{1.463011in}}%
\pgfpathlineto{\pgfqpoint{5.323330in}{1.889659in}}%
\pgfpathlineto{\pgfqpoint{5.323540in}{1.923631in}}%
\pgfpathlineto{\pgfqpoint{5.323561in}{1.536813in}}%
\pgfpathlineto{\pgfqpoint{5.324211in}{1.788591in}}%
\pgfpathlineto{\pgfqpoint{5.324222in}{1.274366in}}%
\pgfpathlineto{\pgfqpoint{5.324232in}{1.934869in}}%
\pgfpathlineto{\pgfqpoint{5.325321in}{1.598454in}}%
\pgfpathlineto{\pgfqpoint{5.326240in}{1.955530in}}%
\pgfpathlineto{\pgfqpoint{5.326062in}{1.306610in}}%
\pgfpathlineto{\pgfqpoint{5.326448in}{1.924598in}}%
\pgfpathlineto{\pgfqpoint{5.327417in}{1.418564in}}%
\pgfpathlineto{\pgfqpoint{5.327250in}{1.956350in}}%
\pgfpathlineto{\pgfqpoint{5.327583in}{1.664984in}}%
\pgfpathlineto{\pgfqpoint{5.328300in}{1.921249in}}%
\pgfpathlineto{\pgfqpoint{5.328175in}{1.462398in}}%
\pgfpathlineto{\pgfqpoint{5.328694in}{1.766354in}}%
\pgfpathlineto{\pgfqpoint{5.329182in}{1.329820in}}%
\pgfpathlineto{\pgfqpoint{5.329047in}{1.936330in}}%
\pgfpathlineto{\pgfqpoint{5.329803in}{1.822123in}}%
\pgfpathlineto{\pgfqpoint{5.330082in}{1.539766in}}%
\pgfpathlineto{\pgfqpoint{5.330764in}{1.941885in}}%
\pgfpathlineto{\pgfqpoint{5.330918in}{1.710030in}}%
\pgfpathlineto{\pgfqpoint{5.331671in}{1.931949in}}%
\pgfpathlineto{\pgfqpoint{5.331959in}{1.409834in}}%
\pgfpathlineto{\pgfqpoint{5.332031in}{1.788691in}}%
\pgfpathlineto{\pgfqpoint{5.333008in}{1.475349in}}%
\pgfpathlineto{\pgfqpoint{5.332679in}{1.929179in}}%
\pgfpathlineto{\pgfqpoint{5.333141in}{1.697081in}}%
\pgfpathlineto{\pgfqpoint{5.333849in}{1.955252in}}%
\pgfpathlineto{\pgfqpoint{5.334043in}{1.543718in}}%
\pgfpathlineto{\pgfqpoint{5.334248in}{1.686747in}}%
\pgfpathlineto{\pgfqpoint{5.334463in}{1.929396in}}%
\pgfpathlineto{\pgfqpoint{5.334944in}{1.459446in}}%
\pgfpathlineto{\pgfqpoint{5.335362in}{1.832466in}}%
\pgfpathlineto{\pgfqpoint{5.336188in}{1.433749in}}%
\pgfpathlineto{\pgfqpoint{5.336321in}{1.922674in}}%
\pgfpathlineto{\pgfqpoint{5.336464in}{1.743174in}}%
\pgfpathlineto{\pgfqpoint{5.337328in}{1.935005in}}%
\pgfpathlineto{\pgfqpoint{5.337491in}{1.318678in}}%
\pgfpathlineto{\pgfqpoint{5.337572in}{1.824661in}}%
\pgfpathlineto{\pgfqpoint{5.338404in}{1.325075in}}%
\pgfpathlineto{\pgfqpoint{5.337927in}{1.940792in}}%
\pgfpathlineto{\pgfqpoint{5.338678in}{1.765138in}}%
\pgfpathlineto{\pgfqpoint{5.339376in}{1.969173in}}%
\pgfpathlineto{\pgfqpoint{5.339720in}{1.433889in}}%
\pgfpathlineto{\pgfqpoint{5.339780in}{1.822410in}}%
\pgfpathlineto{\pgfqpoint{5.340669in}{1.464698in}}%
\pgfpathlineto{\pgfqpoint{5.339952in}{1.915334in}}%
\pgfpathlineto{\pgfqpoint{5.340890in}{1.795204in}}%
\pgfpathlineto{\pgfqpoint{5.341364in}{1.947304in}}%
\pgfpathlineto{\pgfqpoint{5.341907in}{1.485166in}}%
\pgfpathlineto{\pgfqpoint{5.341977in}{1.750552in}}%
\pgfpathlineto{\pgfqpoint{5.342640in}{1.470711in}}%
\pgfpathlineto{\pgfqpoint{5.342018in}{1.917633in}}%
\pgfpathlineto{\pgfqpoint{5.343082in}{1.745715in}}%
\pgfpathlineto{\pgfqpoint{5.343322in}{1.432344in}}%
\pgfpathlineto{\pgfqpoint{5.343783in}{1.938238in}}%
\pgfpathlineto{\pgfqpoint{5.344153in}{1.733818in}}%
\pgfpathlineto{\pgfqpoint{5.344433in}{1.938464in}}%
\pgfpathlineto{\pgfqpoint{5.344842in}{1.479792in}}%
\pgfpathlineto{\pgfqpoint{5.345272in}{1.851516in}}%
\pgfpathlineto{\pgfqpoint{5.346208in}{1.407542in}}%
\pgfpathlineto{\pgfqpoint{5.345481in}{1.951218in}}%
\pgfpathlineto{\pgfqpoint{5.346377in}{1.793649in}}%
\pgfpathlineto{\pgfqpoint{5.346656in}{1.941973in}}%
\pgfpathlineto{\pgfqpoint{5.347381in}{1.385468in}}%
\pgfpathlineto{\pgfqpoint{5.347470in}{1.736416in}}%
\pgfpathlineto{\pgfqpoint{5.347480in}{1.735779in}}%
\pgfpathlineto{\pgfqpoint{5.347490in}{1.762178in}}%
\pgfpathlineto{\pgfqpoint{5.348283in}{1.940651in}}%
\pgfpathlineto{\pgfqpoint{5.347758in}{1.492503in}}%
\pgfpathlineto{\pgfqpoint{5.348560in}{1.811638in}}%
\pgfpathlineto{\pgfqpoint{5.348570in}{1.381492in}}%
\pgfpathlineto{\pgfqpoint{5.348738in}{1.929857in}}%
\pgfpathlineto{\pgfqpoint{5.349667in}{1.796512in}}%
\pgfpathlineto{\pgfqpoint{5.349865in}{1.450964in}}%
\pgfpathlineto{\pgfqpoint{5.349736in}{1.928983in}}%
\pgfpathlineto{\pgfqpoint{5.350761in}{1.723561in}}%
\pgfpathlineto{\pgfqpoint{5.351253in}{1.912982in}}%
\pgfpathlineto{\pgfqpoint{5.351175in}{1.485307in}}%
\pgfpathlineto{\pgfqpoint{5.351863in}{1.879544in}}%
\pgfpathlineto{\pgfqpoint{5.351951in}{1.297113in}}%
\pgfpathlineto{\pgfqpoint{5.352285in}{1.930183in}}%
\pgfpathlineto{\pgfqpoint{5.352971in}{1.813862in}}%
\pgfpathlineto{\pgfqpoint{5.353314in}{1.393852in}}%
\pgfpathlineto{\pgfqpoint{5.353030in}{1.933704in}}%
\pgfpathlineto{\pgfqpoint{5.354076in}{1.753954in}}%
\pgfpathlineto{\pgfqpoint{5.354096in}{1.910556in}}%
\pgfpathlineto{\pgfqpoint{5.354379in}{1.525841in}}%
\pgfpathlineto{\pgfqpoint{5.355189in}{1.818516in}}%
\pgfpathlineto{\pgfqpoint{5.356162in}{1.055700in}}%
\pgfpathlineto{\pgfqpoint{5.355374in}{1.926056in}}%
\pgfpathlineto{\pgfqpoint{5.356298in}{1.738078in}}%
\pgfpathlineto{\pgfqpoint{5.356696in}{1.936902in}}%
\pgfpathlineto{\pgfqpoint{5.357210in}{1.465696in}}%
\pgfpathlineto{\pgfqpoint{5.357414in}{1.827200in}}%
\pgfpathlineto{\pgfqpoint{5.357860in}{1.495163in}}%
\pgfpathlineto{\pgfqpoint{5.358276in}{1.900358in}}%
\pgfpathlineto{\pgfqpoint{5.358527in}{1.709673in}}%
\pgfpathlineto{\pgfqpoint{5.359261in}{1.917594in}}%
\pgfpathlineto{\pgfqpoint{5.359367in}{1.432659in}}%
\pgfpathlineto{\pgfqpoint{5.359637in}{1.737413in}}%
\pgfpathlineto{\pgfqpoint{5.360562in}{1.163479in}}%
\pgfpathlineto{\pgfqpoint{5.359772in}{1.912166in}}%
\pgfpathlineto{\pgfqpoint{5.360726in}{1.872896in}}%
\pgfpathlineto{\pgfqpoint{5.361446in}{1.936607in}}%
\pgfpathlineto{\pgfqpoint{5.360918in}{1.382755in}}%
\pgfpathlineto{\pgfqpoint{5.361792in}{1.669081in}}%
\pgfpathlineto{\pgfqpoint{5.362012in}{1.330314in}}%
\pgfpathlineto{\pgfqpoint{5.362865in}{1.930597in}}%
\pgfpathlineto{\pgfqpoint{5.362884in}{1.482000in}}%
\pgfpathlineto{\pgfqpoint{5.363725in}{1.948646in}}%
\pgfpathlineto{\pgfqpoint{5.362941in}{1.325966in}}%
\pgfpathlineto{\pgfqpoint{5.364002in}{1.850981in}}%
\pgfpathlineto{\pgfqpoint{5.364107in}{1.355946in}}%
\pgfpathlineto{\pgfqpoint{5.365012in}{1.942812in}}%
\pgfpathlineto{\pgfqpoint{5.365108in}{1.642369in}}%
\pgfpathlineto{\pgfqpoint{5.365735in}{1.942489in}}%
\pgfpathlineto{\pgfqpoint{5.365498in}{1.270508in}}%
\pgfpathlineto{\pgfqpoint{5.366220in}{1.849058in}}%
\pgfpathlineto{\pgfqpoint{5.366410in}{1.360987in}}%
\pgfpathlineto{\pgfqpoint{5.366362in}{1.919628in}}%
\pgfpathlineto{\pgfqpoint{5.367329in}{1.834065in}}%
\pgfpathlineto{\pgfqpoint{5.368077in}{1.922826in}}%
\pgfpathlineto{\pgfqpoint{5.367433in}{1.446052in}}%
\pgfpathlineto{\pgfqpoint{5.368417in}{1.903160in}}%
\pgfpathlineto{\pgfqpoint{5.369153in}{1.300451in}}%
\pgfpathlineto{\pgfqpoint{5.369011in}{1.920231in}}%
\pgfpathlineto{\pgfqpoint{5.369539in}{1.678812in}}%
\pgfpathlineto{\pgfqpoint{5.370047in}{1.908177in}}%
\pgfpathlineto{\pgfqpoint{5.369652in}{1.484642in}}%
\pgfpathlineto{\pgfqpoint{5.370649in}{1.888440in}}%
\pgfpathlineto{\pgfqpoint{5.370837in}{1.591725in}}%
\pgfpathlineto{\pgfqpoint{5.371728in}{1.899044in}}%
\pgfpathlineto{\pgfqpoint{5.371766in}{1.800363in}}%
\pgfpathlineto{\pgfqpoint{5.372786in}{1.511101in}}%
\pgfpathlineto{\pgfqpoint{5.372655in}{1.910855in}}%
\pgfpathlineto{\pgfqpoint{5.372851in}{1.716145in}}%
\pgfpathlineto{\pgfqpoint{5.372973in}{1.935640in}}%
\pgfpathlineto{\pgfqpoint{5.373393in}{1.282287in}}%
\pgfpathlineto{\pgfqpoint{5.373962in}{1.862033in}}%
\pgfpathlineto{\pgfqpoint{5.374195in}{1.510803in}}%
\pgfpathlineto{\pgfqpoint{5.374586in}{1.927179in}}%
\pgfpathlineto{\pgfqpoint{5.375070in}{1.849946in}}%
\pgfpathlineto{\pgfqpoint{5.375692in}{1.926741in}}%
\pgfpathlineto{\pgfqpoint{5.375739in}{1.518765in}}%
\pgfpathlineto{\pgfqpoint{5.376110in}{1.760765in}}%
\pgfpathlineto{\pgfqpoint{5.377157in}{1.470366in}}%
\pgfpathlineto{\pgfqpoint{5.377110in}{1.924995in}}%
\pgfpathlineto{\pgfqpoint{5.377221in}{1.628073in}}%
\pgfpathlineto{\pgfqpoint{5.377970in}{1.931943in}}%
\pgfpathlineto{\pgfqpoint{5.377841in}{1.452343in}}%
\pgfpathlineto{\pgfqpoint{5.378339in}{1.824398in}}%
\pgfpathlineto{\pgfqpoint{5.379215in}{1.193239in}}%
\pgfpathlineto{\pgfqpoint{5.378846in}{1.915724in}}%
\pgfpathlineto{\pgfqpoint{5.379454in}{1.719018in}}%
\pgfpathlineto{\pgfqpoint{5.379693in}{1.902968in}}%
\pgfpathlineto{\pgfqpoint{5.379758in}{1.491134in}}%
\pgfpathlineto{\pgfqpoint{5.380566in}{1.829504in}}%
\pgfpathlineto{\pgfqpoint{5.381474in}{1.082415in}}%
\pgfpathlineto{\pgfqpoint{5.381291in}{1.926983in}}%
\pgfpathlineto{\pgfqpoint{5.381666in}{1.666510in}}%
\pgfpathlineto{\pgfqpoint{5.381831in}{1.920431in}}%
\pgfpathlineto{\pgfqpoint{5.382736in}{1.508453in}}%
\pgfpathlineto{\pgfqpoint{5.382781in}{1.819921in}}%
\pgfpathlineto{\pgfqpoint{5.382864in}{1.919286in}}%
\pgfpathlineto{\pgfqpoint{5.383210in}{1.454028in}}%
\pgfpathlineto{\pgfqpoint{5.383885in}{1.906050in}}%
\pgfpathlineto{\pgfqpoint{5.384931in}{1.439574in}}%
\pgfpathlineto{\pgfqpoint{5.384340in}{1.944482in}}%
\pgfpathlineto{\pgfqpoint{5.384994in}{1.820262in}}%
\pgfpathlineto{\pgfqpoint{5.385639in}{1.488496in}}%
\pgfpathlineto{\pgfqpoint{5.385720in}{1.952849in}}%
\pgfpathlineto{\pgfqpoint{5.386119in}{1.726624in}}%
\pgfpathlineto{\pgfqpoint{5.386861in}{1.900550in}}%
\pgfpathlineto{\pgfqpoint{5.386671in}{1.439995in}}%
\pgfpathlineto{\pgfqpoint{5.387214in}{1.708086in}}%
\pgfpathlineto{\pgfqpoint{5.388252in}{1.452814in}}%
\pgfpathlineto{\pgfqpoint{5.387981in}{1.922150in}}%
\pgfpathlineto{\pgfqpoint{5.388306in}{1.767455in}}%
\pgfpathlineto{\pgfqpoint{5.388387in}{1.914436in}}%
\pgfpathlineto{\pgfqpoint{5.389233in}{1.302195in}}%
\pgfpathlineto{\pgfqpoint{5.389395in}{1.775920in}}%
\pgfpathlineto{\pgfqpoint{5.390391in}{1.492271in}}%
\pgfpathlineto{\pgfqpoint{5.389583in}{1.925535in}}%
\pgfpathlineto{\pgfqpoint{5.390499in}{1.809028in}}%
\pgfpathlineto{\pgfqpoint{5.390517in}{1.880149in}}%
\pgfpathlineto{\pgfqpoint{5.390544in}{1.765484in}}%
\pgfpathlineto{\pgfqpoint{5.390553in}{1.777239in}}%
\pgfpathlineto{\pgfqpoint{5.390750in}{1.443980in}}%
\pgfpathlineto{\pgfqpoint{5.391350in}{1.921712in}}%
\pgfpathlineto{\pgfqpoint{5.391663in}{1.606797in}}%
\pgfpathlineto{\pgfqpoint{5.392690in}{1.899884in}}%
\pgfpathlineto{\pgfqpoint{5.391913in}{1.451664in}}%
\pgfpathlineto{\pgfqpoint{5.392779in}{1.779979in}}%
\pgfpathlineto{\pgfqpoint{5.393545in}{1.468783in}}%
\pgfpathlineto{\pgfqpoint{5.393599in}{1.931426in}}%
\pgfpathlineto{\pgfqpoint{5.393866in}{1.632222in}}%
\pgfpathlineto{\pgfqpoint{5.394914in}{1.926307in}}%
\pgfpathlineto{\pgfqpoint{5.394266in}{1.190430in}}%
\pgfpathlineto{\pgfqpoint{5.394985in}{1.879263in}}%
\pgfpathlineto{\pgfqpoint{5.396022in}{1.450600in}}%
\pgfpathlineto{\pgfqpoint{5.396066in}{1.937314in}}%
\pgfpathlineto{\pgfqpoint{5.396092in}{1.639901in}}%
\pgfpathlineto{\pgfqpoint{5.396941in}{1.930636in}}%
\pgfpathlineto{\pgfqpoint{5.397020in}{1.467105in}}%
\pgfpathlineto{\pgfqpoint{5.397197in}{1.873259in}}%
\pgfpathlineto{\pgfqpoint{5.398026in}{1.300175in}}%
\pgfpathlineto{\pgfqpoint{5.397920in}{1.897835in}}%
\pgfpathlineto{\pgfqpoint{5.398307in}{1.817991in}}%
\pgfpathlineto{\pgfqpoint{5.398642in}{1.482163in}}%
\pgfpathlineto{\pgfqpoint{5.399336in}{1.927207in}}%
\pgfpathlineto{\pgfqpoint{5.399415in}{1.732955in}}%
\pgfpathlineto{\pgfqpoint{5.399722in}{1.902084in}}%
\pgfpathlineto{\pgfqpoint{5.399468in}{1.431763in}}%
\pgfpathlineto{\pgfqpoint{5.400511in}{1.750515in}}%
\pgfpathlineto{\pgfqpoint{5.401272in}{1.474828in}}%
\pgfpathlineto{\pgfqpoint{5.400791in}{1.926457in}}%
\pgfpathlineto{\pgfqpoint{5.401613in}{1.734778in}}%
\pgfpathlineto{\pgfqpoint{5.402703in}{1.902581in}}%
\pgfpathlineto{\pgfqpoint{5.402224in}{1.437856in}}%
\pgfpathlineto{\pgfqpoint{5.402712in}{1.771079in}}%
\pgfpathlineto{\pgfqpoint{5.403704in}{1.926276in}}%
\pgfpathlineto{\pgfqpoint{5.403817in}{1.229211in}}%
\pgfpathlineto{\pgfqpoint{5.403965in}{1.910448in}}%
\pgfpathlineto{\pgfqpoint{5.404936in}{1.825698in}}%
\pgfpathlineto{\pgfqpoint{5.405317in}{1.398855in}}%
\pgfpathlineto{\pgfqpoint{5.405560in}{1.925641in}}%
\pgfpathlineto{\pgfqpoint{5.406035in}{1.818563in}}%
\pgfpathlineto{\pgfqpoint{5.406493in}{1.944298in}}%
\pgfpathlineto{\pgfqpoint{5.406321in}{1.426258in}}%
\pgfpathlineto{\pgfqpoint{5.407140in}{1.828478in}}%
\pgfpathlineto{\pgfqpoint{5.407829in}{1.432472in}}%
\pgfpathlineto{\pgfqpoint{5.408096in}{1.896001in}}%
\pgfpathlineto{\pgfqpoint{5.408260in}{1.662889in}}%
\pgfpathlineto{\pgfqpoint{5.409127in}{1.885545in}}%
\pgfpathlineto{\pgfqpoint{5.408689in}{1.311772in}}%
\pgfpathlineto{\pgfqpoint{5.409376in}{1.811736in}}%
\pgfpathlineto{\pgfqpoint{5.409907in}{1.484916in}}%
\pgfpathlineto{\pgfqpoint{5.409556in}{1.918291in}}%
\pgfpathlineto{\pgfqpoint{5.410489in}{1.779314in}}%
\pgfpathlineto{\pgfqpoint{5.410934in}{1.931603in}}%
\pgfpathlineto{\pgfqpoint{5.410994in}{1.390769in}}%
\pgfpathlineto{\pgfqpoint{5.411583in}{1.822188in}}%
\pgfpathlineto{\pgfqpoint{5.411924in}{1.232843in}}%
\pgfpathlineto{\pgfqpoint{5.412546in}{1.932837in}}%
\pgfpathlineto{\pgfqpoint{5.412699in}{1.671533in}}%
\pgfpathlineto{\pgfqpoint{5.413022in}{1.900141in}}%
\pgfpathlineto{\pgfqpoint{5.413302in}{1.182167in}}%
\pgfpathlineto{\pgfqpoint{5.413812in}{1.803450in}}%
\pgfpathlineto{\pgfqpoint{5.414397in}{1.469453in}}%
\pgfpathlineto{\pgfqpoint{5.414507in}{1.892076in}}%
\pgfpathlineto{\pgfqpoint{5.414795in}{1.648462in}}%
\pgfpathlineto{\pgfqpoint{5.414948in}{1.920134in}}%
\pgfpathlineto{\pgfqpoint{5.415793in}{1.387787in}}%
\pgfpathlineto{\pgfqpoint{5.415903in}{1.566996in}}%
\pgfpathlineto{\pgfqpoint{5.417008in}{1.909704in}}%
\pgfpathlineto{\pgfqpoint{5.416215in}{1.091636in}}%
\pgfpathlineto{\pgfqpoint{5.417016in}{1.630717in}}%
\pgfpathlineto{\pgfqpoint{5.417807in}{1.920798in}}%
\pgfpathlineto{\pgfqpoint{5.417799in}{1.446319in}}%
\pgfpathlineto{\pgfqpoint{5.418126in}{1.812110in}}%
\pgfpathlineto{\pgfqpoint{5.418974in}{1.428677in}}%
\pgfpathlineto{\pgfqpoint{5.418143in}{1.886242in}}%
\pgfpathlineto{\pgfqpoint{5.419242in}{1.689597in}}%
\pgfpathlineto{\pgfqpoint{5.419468in}{1.901937in}}%
\pgfpathlineto{\pgfqpoint{5.420088in}{1.481477in}}%
\pgfpathlineto{\pgfqpoint{5.420347in}{1.798746in}}%
\pgfpathlineto{\pgfqpoint{5.420522in}{1.399104in}}%
\pgfpathlineto{\pgfqpoint{5.421332in}{1.894612in}}%
\pgfpathlineto{\pgfqpoint{5.421465in}{1.592772in}}%
\pgfpathlineto{\pgfqpoint{5.422065in}{1.905161in}}%
\pgfpathlineto{\pgfqpoint{5.421832in}{1.424551in}}%
\pgfpathlineto{\pgfqpoint{5.422589in}{1.678872in}}%
\pgfpathlineto{\pgfqpoint{5.422597in}{1.437594in}}%
\pgfpathlineto{\pgfqpoint{5.423593in}{1.886230in}}%
\pgfpathlineto{\pgfqpoint{5.423693in}{1.803670in}}%
\pgfpathlineto{\pgfqpoint{5.423950in}{1.399565in}}%
\pgfpathlineto{\pgfqpoint{5.424728in}{1.908605in}}%
\pgfpathlineto{\pgfqpoint{5.424827in}{1.623141in}}%
\pgfpathlineto{\pgfqpoint{5.425108in}{1.924621in}}%
\pgfpathlineto{\pgfqpoint{5.425860in}{1.413173in}}%
\pgfpathlineto{\pgfqpoint{5.425942in}{1.765540in}}%
\pgfpathlineto{\pgfqpoint{5.427013in}{1.351692in}}%
\pgfpathlineto{\pgfqpoint{5.426676in}{1.933596in}}%
\pgfpathlineto{\pgfqpoint{5.427046in}{1.689552in}}%
\pgfpathlineto{\pgfqpoint{5.427309in}{1.894243in}}%
\pgfpathlineto{\pgfqpoint{5.427786in}{1.441005in}}%
\pgfpathlineto{\pgfqpoint{5.428155in}{1.829203in}}%
\pgfpathlineto{\pgfqpoint{5.428204in}{1.285536in}}%
\pgfpathlineto{\pgfqpoint{5.428549in}{1.902733in}}%
\pgfpathlineto{\pgfqpoint{5.429261in}{1.854116in}}%
\pgfpathlineto{\pgfqpoint{5.429294in}{1.376606in}}%
\pgfpathlineto{\pgfqpoint{5.429523in}{1.904141in}}%
\pgfpathlineto{\pgfqpoint{5.430389in}{1.583605in}}%
\pgfpathlineto{\pgfqpoint{5.431286in}{1.900584in}}%
\pgfpathlineto{\pgfqpoint{5.430968in}{1.443614in}}%
\pgfpathlineto{\pgfqpoint{5.431489in}{1.719779in}}%
\pgfpathlineto{\pgfqpoint{5.432262in}{1.345910in}}%
\pgfpathlineto{\pgfqpoint{5.432180in}{1.924310in}}%
\pgfpathlineto{\pgfqpoint{5.432595in}{1.774382in}}%
\pgfpathlineto{\pgfqpoint{5.433438in}{1.259581in}}%
\pgfpathlineto{\pgfqpoint{5.432643in}{1.913051in}}%
\pgfpathlineto{\pgfqpoint{5.433600in}{1.763108in}}%
\pgfpathlineto{\pgfqpoint{5.433778in}{1.916182in}}%
\pgfpathlineto{\pgfqpoint{5.434393in}{1.394377in}}%
\pgfpathlineto{\pgfqpoint{5.434700in}{1.813810in}}%
\pgfpathlineto{\pgfqpoint{5.435233in}{1.389225in}}%
\pgfpathlineto{\pgfqpoint{5.434741in}{1.869641in}}%
\pgfpathlineto{\pgfqpoint{5.435806in}{1.795549in}}%
\pgfpathlineto{\pgfqpoint{5.436313in}{1.913381in}}%
\pgfpathlineto{\pgfqpoint{5.436764in}{1.391844in}}%
\pgfpathlineto{\pgfqpoint{5.436876in}{1.871479in}}%
\pgfpathlineto{\pgfqpoint{5.437093in}{1.372160in}}%
\pgfpathlineto{\pgfqpoint{5.437527in}{1.892047in}}%
\pgfpathlineto{\pgfqpoint{5.437984in}{1.754644in}}%
\pgfpathlineto{\pgfqpoint{5.438625in}{1.459998in}}%
\pgfpathlineto{\pgfqpoint{5.438232in}{1.915645in}}%
\pgfpathlineto{\pgfqpoint{5.439065in}{1.708004in}}%
\pgfpathlineto{\pgfqpoint{5.439376in}{1.915803in}}%
\pgfpathlineto{\pgfqpoint{5.439664in}{1.291700in}}%
\pgfpathlineto{\pgfqpoint{5.440175in}{1.737526in}}%
\pgfpathlineto{\pgfqpoint{5.440701in}{1.389770in}}%
\pgfpathlineto{\pgfqpoint{5.440693in}{1.938949in}}%
\pgfpathlineto{\pgfqpoint{5.441282in}{1.846031in}}%
\pgfpathlineto{\pgfqpoint{5.441727in}{1.483704in}}%
\pgfpathlineto{\pgfqpoint{5.442053in}{1.903592in}}%
\pgfpathlineto{\pgfqpoint{5.442394in}{1.653177in}}%
\pgfpathlineto{\pgfqpoint{5.442711in}{1.893706in}}%
\pgfpathlineto{\pgfqpoint{5.443028in}{1.437621in}}%
\pgfpathlineto{\pgfqpoint{5.443527in}{1.758697in}}%
\pgfpathlineto{\pgfqpoint{5.443630in}{1.382487in}}%
\pgfpathlineto{\pgfqpoint{5.444104in}{1.903025in}}%
\pgfpathlineto{\pgfqpoint{5.444641in}{1.677056in}}%
\pgfpathlineto{\pgfqpoint{5.444649in}{1.908975in}}%
\pgfpathlineto{\pgfqpoint{5.445343in}{1.432500in}}%
\pgfpathlineto{\pgfqpoint{5.445752in}{1.782103in}}%
\pgfpathlineto{\pgfqpoint{5.446342in}{1.483079in}}%
\pgfpathlineto{\pgfqpoint{5.446185in}{1.895669in}}%
\pgfpathlineto{\pgfqpoint{5.446861in}{1.730654in}}%
\pgfpathlineto{\pgfqpoint{5.447637in}{1.898633in}}%
\pgfpathlineto{\pgfqpoint{5.447865in}{1.400307in}}%
\pgfpathlineto{\pgfqpoint{5.447966in}{1.752464in}}%
\pgfpathlineto{\pgfqpoint{5.448240in}{1.525816in}}%
\pgfpathlineto{\pgfqpoint{5.448811in}{1.926335in}}%
\pgfpathlineto{\pgfqpoint{5.449069in}{1.637051in}}%
\pgfpathlineto{\pgfqpoint{5.449654in}{1.896834in}}%
\pgfpathlineto{\pgfqpoint{5.449608in}{1.497049in}}%
\pgfpathlineto{\pgfqpoint{5.450184in}{1.785461in}}%
\pgfpathlineto{\pgfqpoint{5.450931in}{1.903028in}}%
\pgfpathlineto{\pgfqpoint{5.450675in}{1.392273in}}%
\pgfpathlineto{\pgfqpoint{5.451227in}{1.716556in}}%
\pgfpathlineto{\pgfqpoint{5.452166in}{1.379380in}}%
\pgfpathlineto{\pgfqpoint{5.452042in}{1.887935in}}%
\pgfpathlineto{\pgfqpoint{5.452321in}{1.794058in}}%
\pgfpathlineto{\pgfqpoint{5.453242in}{1.465243in}}%
\pgfpathlineto{\pgfqpoint{5.452910in}{1.914819in}}%
\pgfpathlineto{\pgfqpoint{5.453412in}{1.829753in}}%
\pgfpathlineto{\pgfqpoint{5.453706in}{1.884321in}}%
\pgfpathlineto{\pgfqpoint{5.453551in}{1.298991in}}%
\pgfpathlineto{\pgfqpoint{5.454501in}{1.847365in}}%
\pgfpathlineto{\pgfqpoint{5.455217in}{1.285030in}}%
\pgfpathlineto{\pgfqpoint{5.455356in}{1.902212in}}%
\pgfpathlineto{\pgfqpoint{5.455610in}{1.732530in}}%
\pgfpathlineto{\pgfqpoint{5.455764in}{1.925005in}}%
\pgfpathlineto{\pgfqpoint{5.456194in}{1.492628in}}%
\pgfpathlineto{\pgfqpoint{5.456708in}{1.674317in}}%
\pgfpathlineto{\pgfqpoint{5.457252in}{1.470930in}}%
\pgfpathlineto{\pgfqpoint{5.456792in}{1.886093in}}%
\pgfpathlineto{\pgfqpoint{5.457765in}{1.519409in}}%
\pgfpathlineto{\pgfqpoint{5.457773in}{1.925285in}}%
\pgfpathlineto{\pgfqpoint{5.458239in}{1.471257in}}%
\pgfpathlineto{\pgfqpoint{5.458873in}{1.739911in}}%
\pgfpathlineto{\pgfqpoint{5.459361in}{1.370189in}}%
\pgfpathlineto{\pgfqpoint{5.459910in}{1.906594in}}%
\pgfpathlineto{\pgfqpoint{5.459994in}{1.487029in}}%
\pgfpathlineto{\pgfqpoint{5.460557in}{1.892807in}}%
\pgfpathlineto{\pgfqpoint{5.460199in}{1.432370in}}%
\pgfpathlineto{\pgfqpoint{5.461111in}{1.833499in}}%
\pgfpathlineto{\pgfqpoint{5.461726in}{1.382252in}}%
\pgfpathlineto{\pgfqpoint{5.462051in}{1.908990in}}%
\pgfpathlineto{\pgfqpoint{5.462233in}{1.645390in}}%
\pgfpathlineto{\pgfqpoint{5.463299in}{1.887286in}}%
\pgfpathlineto{\pgfqpoint{5.462566in}{1.415022in}}%
\pgfpathlineto{\pgfqpoint{5.463337in}{1.777993in}}%
\pgfpathlineto{\pgfqpoint{5.463677in}{1.384566in}}%
\pgfpathlineto{\pgfqpoint{5.463526in}{1.910407in}}%
\pgfpathlineto{\pgfqpoint{5.464446in}{1.715204in}}%
\pgfpathlineto{\pgfqpoint{5.464897in}{1.520095in}}%
\pgfpathlineto{\pgfqpoint{5.464641in}{1.887553in}}%
\pgfpathlineto{\pgfqpoint{5.465551in}{1.772568in}}%
\pgfpathlineto{\pgfqpoint{5.466122in}{1.476465in}}%
\pgfpathlineto{\pgfqpoint{5.466384in}{1.899408in}}%
\pgfpathlineto{\pgfqpoint{5.466662in}{1.709013in}}%
\pgfpathlineto{\pgfqpoint{5.467575in}{1.902477in}}%
\pgfpathlineto{\pgfqpoint{5.467186in}{1.370833in}}%
\pgfpathlineto{\pgfqpoint{5.467754in}{1.796719in}}%
\pgfpathlineto{\pgfqpoint{5.468053in}{1.405583in}}%
\pgfpathlineto{\pgfqpoint{5.467784in}{1.891587in}}%
\pgfpathlineto{\pgfqpoint{5.468866in}{1.696181in}}%
\pgfpathlineto{\pgfqpoint{5.469001in}{1.894351in}}%
\pgfpathlineto{\pgfqpoint{5.469626in}{1.474697in}}%
\pgfpathlineto{\pgfqpoint{5.469968in}{1.659441in}}%
\pgfpathlineto{\pgfqpoint{5.470407in}{1.881300in}}%
\pgfpathlineto{\pgfqpoint{5.470280in}{1.432574in}}%
\pgfpathlineto{\pgfqpoint{5.471037in}{1.792209in}}%
\pgfpathlineto{\pgfqpoint{5.471297in}{1.437848in}}%
\pgfpathlineto{\pgfqpoint{5.471334in}{1.896959in}}%
\pgfpathlineto{\pgfqpoint{5.472148in}{1.576881in}}%
\pgfpathlineto{\pgfqpoint{5.473219in}{1.889072in}}%
\pgfpathlineto{\pgfqpoint{5.472193in}{1.378274in}}%
\pgfpathlineto{\pgfqpoint{5.473278in}{1.746781in}}%
\pgfpathlineto{\pgfqpoint{5.473949in}{1.394909in}}%
\pgfpathlineto{\pgfqpoint{5.473293in}{1.880497in}}%
\pgfpathlineto{\pgfqpoint{5.474391in}{1.517184in}}%
\pgfpathlineto{\pgfqpoint{5.474928in}{1.901793in}}%
\pgfpathlineto{\pgfqpoint{5.474663in}{1.348133in}}%
\pgfpathlineto{\pgfqpoint{5.475508in}{1.787067in}}%
\pgfpathlineto{\pgfqpoint{5.475779in}{1.133568in}}%
\pgfpathlineto{\pgfqpoint{5.476387in}{1.900736in}}%
\pgfpathlineto{\pgfqpoint{5.476614in}{1.559098in}}%
\pgfpathlineto{\pgfqpoint{5.477528in}{1.894939in}}%
\pgfpathlineto{\pgfqpoint{5.477587in}{1.422669in}}%
\pgfpathlineto{\pgfqpoint{5.477733in}{1.803494in}}%
\pgfpathlineto{\pgfqpoint{5.478302in}{1.398543in}}%
\pgfpathlineto{\pgfqpoint{5.478003in}{1.881004in}}%
\pgfpathlineto{\pgfqpoint{5.478841in}{1.664334in}}%
\pgfpathlineto{\pgfqpoint{5.479037in}{1.863129in}}%
\pgfpathlineto{\pgfqpoint{5.479670in}{1.371969in}}%
\pgfpathlineto{\pgfqpoint{5.479953in}{1.761480in}}%
\pgfpathlineto{\pgfqpoint{5.480998in}{1.420455in}}%
\pgfpathlineto{\pgfqpoint{5.480062in}{1.871269in}}%
\pgfpathlineto{\pgfqpoint{5.481056in}{1.748095in}}%
\pgfpathlineto{\pgfqpoint{5.481880in}{1.904450in}}%
\pgfpathlineto{\pgfqpoint{5.481533in}{1.311551in}}%
\pgfpathlineto{\pgfqpoint{5.482162in}{1.758969in}}%
\pgfpathlineto{\pgfqpoint{5.482458in}{1.393656in}}%
\pgfpathlineto{\pgfqpoint{5.482848in}{1.876750in}}%
\pgfpathlineto{\pgfqpoint{5.483273in}{1.506747in}}%
\pgfpathlineto{\pgfqpoint{5.483554in}{1.925372in}}%
\pgfpathlineto{\pgfqpoint{5.483482in}{1.411047in}}%
\pgfpathlineto{\pgfqpoint{5.484381in}{1.749858in}}%
\pgfpathlineto{\pgfqpoint{5.485315in}{1.453988in}}%
\pgfpathlineto{\pgfqpoint{5.485372in}{1.899099in}}%
\pgfpathlineto{\pgfqpoint{5.485487in}{1.792530in}}%
\pgfpathlineto{\pgfqpoint{5.485852in}{1.419077in}}%
\pgfpathlineto{\pgfqpoint{5.486460in}{1.903704in}}%
\pgfpathlineto{\pgfqpoint{5.486589in}{1.575628in}}%
\pgfpathlineto{\pgfqpoint{5.486946in}{1.902023in}}%
\pgfpathlineto{\pgfqpoint{5.486989in}{1.444629in}}%
\pgfpathlineto{\pgfqpoint{5.487703in}{1.836483in}}%
\pgfpathlineto{\pgfqpoint{5.488607in}{1.379995in}}%
\pgfpathlineto{\pgfqpoint{5.488230in}{1.886522in}}%
\pgfpathlineto{\pgfqpoint{5.488814in}{1.789509in}}%
\pgfpathlineto{\pgfqpoint{5.489205in}{1.898719in}}%
\pgfpathlineto{\pgfqpoint{5.489411in}{1.422858in}}%
\pgfpathlineto{\pgfqpoint{5.489751in}{1.793715in}}%
\pgfpathlineto{\pgfqpoint{5.490078in}{1.394206in}}%
\pgfpathlineto{\pgfqpoint{5.490630in}{1.871822in}}%
\pgfpathlineto{\pgfqpoint{5.490857in}{1.744591in}}%
\pgfpathlineto{\pgfqpoint{5.491868in}{1.890057in}}%
\pgfpathlineto{\pgfqpoint{5.491550in}{1.344440in}}%
\pgfpathlineto{\pgfqpoint{5.491974in}{1.875961in}}%
\pgfpathlineto{\pgfqpoint{5.492609in}{1.371094in}}%
\pgfpathlineto{\pgfqpoint{5.493088in}{1.636110in}}%
\pgfpathlineto{\pgfqpoint{5.493601in}{1.889844in}}%
\pgfpathlineto{\pgfqpoint{5.493847in}{1.423978in}}%
\pgfpathlineto{\pgfqpoint{5.494177in}{1.748602in}}%
\pgfpathlineto{\pgfqpoint{5.494184in}{1.330329in}}%
\pgfpathlineto{\pgfqpoint{5.494669in}{1.882362in}}%
\pgfpathlineto{\pgfqpoint{5.495285in}{1.578449in}}%
\pgfpathlineto{\pgfqpoint{5.495621in}{1.899158in}}%
\pgfpathlineto{\pgfqpoint{5.496370in}{1.184315in}}%
\pgfpathlineto{\pgfqpoint{5.496405in}{1.812631in}}%
\pgfpathlineto{\pgfqpoint{5.496963in}{1.388300in}}%
\pgfpathlineto{\pgfqpoint{5.496419in}{1.868144in}}%
\pgfpathlineto{\pgfqpoint{5.497507in}{1.719088in}}%
\pgfpathlineto{\pgfqpoint{5.498578in}{1.892167in}}%
\pgfpathlineto{\pgfqpoint{5.497695in}{1.325375in}}%
\pgfpathlineto{\pgfqpoint{5.498620in}{1.820610in}}%
\pgfpathlineto{\pgfqpoint{5.499557in}{1.333302in}}%
\pgfpathlineto{\pgfqpoint{5.499210in}{1.889666in}}%
\pgfpathlineto{\pgfqpoint{5.499731in}{1.745558in}}%
\pgfpathlineto{\pgfqpoint{5.500188in}{1.869359in}}%
\pgfpathlineto{\pgfqpoint{5.500561in}{1.392131in}}%
\pgfpathlineto{\pgfqpoint{5.500838in}{1.835033in}}%
\pgfpathlineto{\pgfqpoint{5.501839in}{1.351451in}}%
\pgfpathlineto{\pgfqpoint{5.501446in}{1.892319in}}%
\pgfpathlineto{\pgfqpoint{5.501963in}{1.404660in}}%
\pgfpathlineto{\pgfqpoint{5.503017in}{1.880236in}}%
\pgfpathlineto{\pgfqpoint{5.502391in}{1.330327in}}%
\pgfpathlineto{\pgfqpoint{5.503079in}{1.704882in}}%
\pgfpathlineto{\pgfqpoint{5.503560in}{1.877078in}}%
\pgfpathlineto{\pgfqpoint{5.503299in}{1.161558in}}%
\pgfpathlineto{\pgfqpoint{5.504178in}{1.654374in}}%
\pgfpathlineto{\pgfqpoint{5.505198in}{1.288398in}}%
\pgfpathlineto{\pgfqpoint{5.504829in}{1.888609in}}%
\pgfpathlineto{\pgfqpoint{5.505281in}{1.642749in}}%
\pgfpathlineto{\pgfqpoint{5.505424in}{1.900163in}}%
\pgfpathlineto{\pgfqpoint{5.505540in}{1.430595in}}%
\pgfpathlineto{\pgfqpoint{5.506374in}{1.641622in}}%
\pgfpathlineto{\pgfqpoint{5.506803in}{1.392194in}}%
\pgfpathlineto{\pgfqpoint{5.506537in}{1.862903in}}%
\pgfpathlineto{\pgfqpoint{5.507478in}{1.795742in}}%
\pgfpathlineto{\pgfqpoint{5.507586in}{1.453097in}}%
\pgfpathlineto{\pgfqpoint{5.508558in}{1.880293in}}%
\pgfpathlineto{\pgfqpoint{5.508572in}{1.672772in}}%
\pgfpathlineto{\pgfqpoint{5.509284in}{1.897905in}}%
\pgfpathlineto{\pgfqpoint{5.508735in}{1.404719in}}%
\pgfpathlineto{\pgfqpoint{5.509684in}{1.701421in}}%
\pgfpathlineto{\pgfqpoint{5.510279in}{1.872813in}}%
\pgfpathlineto{\pgfqpoint{5.510759in}{1.297239in}}%
\pgfpathlineto{\pgfqpoint{5.510799in}{1.833482in}}%
\pgfpathlineto{\pgfqpoint{5.511878in}{1.402984in}}%
\pgfpathlineto{\pgfqpoint{5.511069in}{1.864970in}}%
\pgfpathlineto{\pgfqpoint{5.511905in}{1.716102in}}%
\pgfpathlineto{\pgfqpoint{5.512464in}{1.896747in}}%
\pgfpathlineto{\pgfqpoint{5.512733in}{1.452535in}}%
\pgfpathlineto{\pgfqpoint{5.513002in}{1.700388in}}%
\pgfpathlineto{\pgfqpoint{5.513109in}{1.348699in}}%
\pgfpathlineto{\pgfqpoint{5.513579in}{1.865107in}}%
\pgfpathlineto{\pgfqpoint{5.514102in}{1.658399in}}%
\pgfpathlineto{\pgfqpoint{5.514939in}{1.893623in}}%
\pgfpathlineto{\pgfqpoint{5.515199in}{1.380912in}}%
\pgfpathlineto{\pgfqpoint{5.515213in}{1.668286in}}%
\pgfpathlineto{\pgfqpoint{5.515754in}{1.863224in}}%
\pgfpathlineto{\pgfqpoint{5.515420in}{1.477183in}}%
\pgfpathlineto{\pgfqpoint{5.516321in}{1.801136in}}%
\pgfpathlineto{\pgfqpoint{5.517293in}{1.420907in}}%
\pgfpathlineto{\pgfqpoint{5.516813in}{1.873695in}}%
\pgfpathlineto{\pgfqpoint{5.517426in}{1.641083in}}%
\pgfpathlineto{\pgfqpoint{5.517751in}{1.897214in}}%
\pgfpathlineto{\pgfqpoint{5.517977in}{1.301489in}}%
\pgfpathlineto{\pgfqpoint{5.518534in}{1.736524in}}%
\pgfpathlineto{\pgfqpoint{5.518998in}{1.447942in}}%
\pgfpathlineto{\pgfqpoint{5.519038in}{1.873358in}}%
\pgfpathlineto{\pgfqpoint{5.519634in}{1.716812in}}%
\pgfpathlineto{\pgfqpoint{5.520565in}{1.887403in}}%
\pgfpathlineto{\pgfqpoint{5.520083in}{1.397939in}}%
\pgfpathlineto{\pgfqpoint{5.520743in}{1.781705in}}%
\pgfpathlineto{\pgfqpoint{5.521580in}{1.365655in}}%
\pgfpathlineto{\pgfqpoint{5.520770in}{1.852710in}}%
\pgfpathlineto{\pgfqpoint{5.521705in}{1.712406in}}%
\pgfpathlineto{\pgfqpoint{5.522008in}{1.286033in}}%
\pgfpathlineto{\pgfqpoint{5.522809in}{1.891414in}}%
\pgfpathlineto{\pgfqpoint{5.523137in}{1.335952in}}%
\pgfpathlineto{\pgfqpoint{5.523917in}{1.823626in}}%
\pgfpathlineto{\pgfqpoint{5.524748in}{1.227810in}}%
\pgfpathlineto{\pgfqpoint{5.524421in}{1.872419in}}%
\pgfpathlineto{\pgfqpoint{5.525087in}{1.691843in}}%
\pgfpathlineto{\pgfqpoint{5.525185in}{1.915289in}}%
\pgfpathlineto{\pgfqpoint{5.525570in}{1.310363in}}%
\pgfpathlineto{\pgfqpoint{5.526189in}{1.842983in}}%
\pgfpathlineto{\pgfqpoint{5.527224in}{1.326214in}}%
\pgfpathlineto{\pgfqpoint{5.526313in}{1.881202in}}%
\pgfpathlineto{\pgfqpoint{5.527302in}{1.770249in}}%
\pgfpathlineto{\pgfqpoint{5.527457in}{1.184393in}}%
\pgfpathlineto{\pgfqpoint{5.527386in}{1.876261in}}%
\pgfpathlineto{\pgfqpoint{5.528411in}{1.635845in}}%
\pgfpathlineto{\pgfqpoint{5.528877in}{1.864061in}}%
\pgfpathlineto{\pgfqpoint{5.529278in}{1.395577in}}%
\pgfpathlineto{\pgfqpoint{5.529523in}{1.766838in}}%
\pgfpathlineto{\pgfqpoint{5.530556in}{1.877633in}}%
\pgfpathlineto{\pgfqpoint{5.530227in}{1.332248in}}%
\pgfpathlineto{\pgfqpoint{5.530601in}{1.641213in}}%
\pgfpathlineto{\pgfqpoint{5.530839in}{1.392569in}}%
\pgfpathlineto{\pgfqpoint{5.531123in}{1.885724in}}%
\pgfpathlineto{\pgfqpoint{5.531695in}{1.697243in}}%
\pgfpathlineto{\pgfqpoint{5.532434in}{1.882477in}}%
\pgfpathlineto{\pgfqpoint{5.532447in}{1.352676in}}%
\pgfpathlineto{\pgfqpoint{5.532799in}{1.661120in}}%
\pgfpathlineto{\pgfqpoint{5.533562in}{1.274807in}}%
\pgfpathlineto{\pgfqpoint{5.533235in}{1.890886in}}%
\pgfpathlineto{\pgfqpoint{5.533901in}{1.710970in}}%
\pgfpathlineto{\pgfqpoint{5.534642in}{1.888046in}}%
\pgfpathlineto{\pgfqpoint{5.534501in}{1.305934in}}%
\pgfpathlineto{\pgfqpoint{5.534916in}{1.558164in}}%
\pgfpathlineto{\pgfqpoint{5.535089in}{1.334611in}}%
\pgfpathlineto{\pgfqpoint{5.535171in}{1.898827in}}%
\pgfpathlineto{\pgfqpoint{5.536012in}{1.754370in}}%
\pgfpathlineto{\pgfqpoint{5.536019in}{1.850226in}}%
\pgfpathlineto{\pgfqpoint{5.536762in}{1.269632in}}%
\pgfpathlineto{\pgfqpoint{5.537118in}{1.754934in}}%
\pgfpathlineto{\pgfqpoint{5.537340in}{1.345493in}}%
\pgfpathlineto{\pgfqpoint{5.537441in}{1.858914in}}%
\pgfpathlineto{\pgfqpoint{5.538227in}{1.592066in}}%
\pgfpathlineto{\pgfqpoint{5.538385in}{1.866533in}}%
\pgfpathlineto{\pgfqpoint{5.538556in}{1.299864in}}%
\pgfpathlineto{\pgfqpoint{5.539346in}{1.793558in}}%
\pgfpathlineto{\pgfqpoint{5.539561in}{1.359130in}}%
\pgfpathlineto{\pgfqpoint{5.540046in}{1.860233in}}%
\pgfpathlineto{\pgfqpoint{5.540456in}{1.745872in}}%
\pgfpathlineto{\pgfqpoint{5.540846in}{1.356682in}}%
\pgfpathlineto{\pgfqpoint{5.540978in}{1.859210in}}%
\pgfpathlineto{\pgfqpoint{5.541562in}{1.733225in}}%
\pgfpathlineto{\pgfqpoint{5.541638in}{1.875672in}}%
\pgfpathlineto{\pgfqpoint{5.541606in}{1.291571in}}%
\pgfpathlineto{\pgfqpoint{5.542272in}{1.701994in}}%
\pgfpathlineto{\pgfqpoint{5.542278in}{1.297779in}}%
\pgfpathlineto{\pgfqpoint{5.543249in}{1.868003in}}%
\pgfpathlineto{\pgfqpoint{5.543380in}{1.723438in}}%
\pgfpathlineto{\pgfqpoint{5.543436in}{1.337600in}}%
\pgfpathlineto{\pgfqpoint{5.543955in}{1.880558in}}%
\pgfpathlineto{\pgfqpoint{5.544479in}{1.654289in}}%
\pgfpathlineto{\pgfqpoint{5.545376in}{1.870003in}}%
\pgfpathlineto{\pgfqpoint{5.545426in}{1.331715in}}%
\pgfpathlineto{\pgfqpoint{5.545594in}{1.739261in}}%
\pgfpathlineto{\pgfqpoint{5.546166in}{1.358380in}}%
\pgfpathlineto{\pgfqpoint{5.546632in}{1.897072in}}%
\pgfpathlineto{\pgfqpoint{5.546706in}{1.710552in}}%
\pgfpathlineto{\pgfqpoint{5.546930in}{1.435451in}}%
\pgfpathlineto{\pgfqpoint{5.547667in}{1.906778in}}%
\pgfpathlineto{\pgfqpoint{5.547809in}{1.740296in}}%
\pgfpathlineto{\pgfqpoint{5.547927in}{1.860090in}}%
\pgfpathlineto{\pgfqpoint{5.547958in}{1.349334in}}%
\pgfpathlineto{\pgfqpoint{5.548817in}{1.767834in}}%
\pgfpathlineto{\pgfqpoint{5.548823in}{1.253758in}}%
\pgfpathlineto{\pgfqpoint{5.549421in}{1.875127in}}%
\pgfpathlineto{\pgfqpoint{5.549927in}{1.668067in}}%
\pgfpathlineto{\pgfqpoint{5.550019in}{1.844687in}}%
\pgfpathlineto{\pgfqpoint{5.550708in}{1.369498in}}%
\pgfpathlineto{\pgfqpoint{5.551040in}{1.746636in}}%
\pgfpathlineto{\pgfqpoint{5.551850in}{1.313841in}}%
\pgfpathlineto{\pgfqpoint{5.552052in}{1.851630in}}%
\pgfpathlineto{\pgfqpoint{5.552144in}{1.737011in}}%
\pgfpathlineto{\pgfqpoint{5.552376in}{1.876084in}}%
\pgfpathlineto{\pgfqpoint{5.552848in}{1.390295in}}%
\pgfpathlineto{\pgfqpoint{5.553220in}{1.692169in}}%
\pgfpathlineto{\pgfqpoint{5.553581in}{1.198088in}}%
\pgfpathlineto{\pgfqpoint{5.553526in}{1.866729in}}%
\pgfpathlineto{\pgfqpoint{5.554331in}{1.486572in}}%
\pgfpathlineto{\pgfqpoint{5.554526in}{1.872921in}}%
\pgfpathlineto{\pgfqpoint{5.554422in}{1.448674in}}%
\pgfpathlineto{\pgfqpoint{5.555451in}{1.777480in}}%
\pgfpathlineto{\pgfqpoint{5.556538in}{1.420108in}}%
\pgfpathlineto{\pgfqpoint{5.555469in}{1.861271in}}%
\pgfpathlineto{\pgfqpoint{5.556556in}{1.794707in}}%
\pgfpathlineto{\pgfqpoint{5.556792in}{1.261727in}}%
\pgfpathlineto{\pgfqpoint{5.556653in}{1.873902in}}%
\pgfpathlineto{\pgfqpoint{5.557664in}{1.478293in}}%
\pgfpathlineto{\pgfqpoint{5.557954in}{1.874125in}}%
\pgfpathlineto{\pgfqpoint{5.558437in}{1.179451in}}%
\pgfpathlineto{\pgfqpoint{5.558775in}{1.509649in}}%
\pgfpathlineto{\pgfqpoint{5.558962in}{1.864883in}}%
\pgfpathlineto{\pgfqpoint{5.559088in}{0.875323in}}%
\pgfpathlineto{\pgfqpoint{5.559889in}{1.720437in}}%
\pgfpathlineto{\pgfqpoint{5.560394in}{1.425130in}}%
\pgfpathlineto{\pgfqpoint{5.560712in}{1.916614in}}%
\pgfpathlineto{\pgfqpoint{5.561000in}{1.702891in}}%
\pgfpathlineto{\pgfqpoint{5.561564in}{1.892654in}}%
\pgfpathlineto{\pgfqpoint{5.561204in}{1.354482in}}%
\pgfpathlineto{\pgfqpoint{5.562097in}{1.744750in}}%
\pgfpathlineto{\pgfqpoint{5.562545in}{1.387905in}}%
\pgfpathlineto{\pgfqpoint{5.563095in}{1.876304in}}%
\pgfpathlineto{\pgfqpoint{5.563208in}{1.657495in}}%
\pgfpathlineto{\pgfqpoint{5.564126in}{1.324197in}}%
\pgfpathlineto{\pgfqpoint{5.563226in}{1.867955in}}%
\pgfpathlineto{\pgfqpoint{5.564299in}{1.493478in}}%
\pgfpathlineto{\pgfqpoint{5.565066in}{1.850061in}}%
\pgfpathlineto{\pgfqpoint{5.564597in}{1.334395in}}%
\pgfpathlineto{\pgfqpoint{5.565411in}{1.706593in}}%
\pgfpathlineto{\pgfqpoint{5.566336in}{1.308308in}}%
\pgfpathlineto{\pgfqpoint{5.566010in}{1.865203in}}%
\pgfpathlineto{\pgfqpoint{5.566514in}{1.541824in}}%
\pgfpathlineto{\pgfqpoint{5.566727in}{1.871382in}}%
\pgfpathlineto{\pgfqpoint{5.567389in}{1.408497in}}%
\pgfpathlineto{\pgfqpoint{5.567625in}{1.661406in}}%
\pgfpathlineto{\pgfqpoint{5.567944in}{1.866709in}}%
\pgfpathlineto{\pgfqpoint{5.568646in}{1.235497in}}%
\pgfpathlineto{\pgfqpoint{5.568740in}{1.799100in}}%
\pgfpathlineto{\pgfqpoint{5.569117in}{1.260852in}}%
\pgfpathlineto{\pgfqpoint{5.568958in}{1.842274in}}%
\pgfpathlineto{\pgfqpoint{5.569858in}{1.629633in}}%
\pgfpathlineto{\pgfqpoint{5.570427in}{1.860960in}}%
\pgfpathlineto{\pgfqpoint{5.570703in}{1.268153in}}%
\pgfpathlineto{\pgfqpoint{5.570973in}{1.751789in}}%
\pgfpathlineto{\pgfqpoint{5.571348in}{1.891061in}}%
\pgfpathlineto{\pgfqpoint{5.571617in}{1.336637in}}%
\pgfpathlineto{\pgfqpoint{5.572067in}{1.617589in}}%
\pgfpathlineto{\pgfqpoint{5.572394in}{1.263789in}}%
\pgfpathlineto{\pgfqpoint{5.572669in}{1.843823in}}%
\pgfpathlineto{\pgfqpoint{5.573170in}{1.668514in}}%
\pgfpathlineto{\pgfqpoint{5.573974in}{1.866600in}}%
\pgfpathlineto{\pgfqpoint{5.573473in}{1.426802in}}%
\pgfpathlineto{\pgfqpoint{5.574259in}{1.601937in}}%
\pgfpathlineto{\pgfqpoint{5.574323in}{1.430326in}}%
\pgfpathlineto{\pgfqpoint{5.574492in}{1.869582in}}%
\pgfpathlineto{\pgfqpoint{5.575357in}{1.658918in}}%
\pgfpathlineto{\pgfqpoint{5.575896in}{1.849457in}}%
\pgfpathlineto{\pgfqpoint{5.575681in}{1.358535in}}%
\pgfpathlineto{\pgfqpoint{5.576469in}{1.740812in}}%
\pgfpathlineto{\pgfqpoint{5.576868in}{1.289544in}}%
\pgfpathlineto{\pgfqpoint{5.577526in}{1.843106in}}%
\pgfpathlineto{\pgfqpoint{5.577572in}{1.724588in}}%
\pgfpathlineto{\pgfqpoint{5.578321in}{1.891374in}}%
\pgfpathlineto{\pgfqpoint{5.577803in}{1.388328in}}%
\pgfpathlineto{\pgfqpoint{5.578678in}{1.656895in}}%
\pgfpathlineto{\pgfqpoint{5.579610in}{1.857421in}}%
\pgfpathlineto{\pgfqpoint{5.578851in}{1.362161in}}%
\pgfpathlineto{\pgfqpoint{5.579770in}{1.655668in}}%
\pgfpathlineto{\pgfqpoint{5.580579in}{1.151520in}}%
\pgfpathlineto{\pgfqpoint{5.580183in}{1.860019in}}%
\pgfpathlineto{\pgfqpoint{5.580877in}{1.705930in}}%
\pgfpathlineto{\pgfqpoint{5.580928in}{1.277233in}}%
\pgfpathlineto{\pgfqpoint{5.581191in}{1.841542in}}%
\pgfpathlineto{\pgfqpoint{5.581963in}{1.698396in}}%
\pgfpathlineto{\pgfqpoint{5.581969in}{1.877208in}}%
\pgfpathlineto{\pgfqpoint{5.582950in}{1.212831in}}%
\pgfpathlineto{\pgfqpoint{5.583075in}{1.753283in}}%
\pgfpathlineto{\pgfqpoint{5.583382in}{1.264824in}}%
\pgfpathlineto{\pgfqpoint{5.583707in}{1.887540in}}%
\pgfpathlineto{\pgfqpoint{5.584184in}{1.700578in}}%
\pgfpathlineto{\pgfqpoint{5.584190in}{1.873910in}}%
\pgfpathlineto{\pgfqpoint{5.584195in}{1.293925in}}%
\pgfpathlineto{\pgfqpoint{5.585290in}{1.630713in}}%
\pgfpathlineto{\pgfqpoint{5.585341in}{1.863566in}}%
\pgfpathlineto{\pgfqpoint{5.585613in}{1.306031in}}%
\pgfpathlineto{\pgfqpoint{5.586405in}{1.736813in}}%
\pgfpathlineto{\pgfqpoint{5.586953in}{1.340301in}}%
\pgfpathlineto{\pgfqpoint{5.587246in}{1.864392in}}%
\pgfpathlineto{\pgfqpoint{5.587511in}{1.699801in}}%
\pgfpathlineto{\pgfqpoint{5.587916in}{1.836786in}}%
\pgfpathlineto{\pgfqpoint{5.587922in}{1.270580in}}%
\pgfpathlineto{\pgfqpoint{5.588614in}{1.707880in}}%
\pgfpathlineto{\pgfqpoint{5.588755in}{1.330226in}}%
\pgfpathlineto{\pgfqpoint{5.588637in}{1.858690in}}%
\pgfpathlineto{\pgfqpoint{5.589714in}{1.738523in}}%
\pgfpathlineto{\pgfqpoint{5.590314in}{1.861858in}}%
\pgfpathlineto{\pgfqpoint{5.590398in}{1.303954in}}%
\pgfpathlineto{\pgfqpoint{5.590812in}{1.755249in}}%
\pgfpathlineto{\pgfqpoint{5.591510in}{1.136843in}}%
\pgfpathlineto{\pgfqpoint{5.591220in}{1.844780in}}%
\pgfpathlineto{\pgfqpoint{5.591923in}{1.692366in}}%
\pgfpathlineto{\pgfqpoint{5.592598in}{1.863683in}}%
\pgfpathlineto{\pgfqpoint{5.592792in}{1.404005in}}%
\pgfpathlineto{\pgfqpoint{5.592982in}{1.598526in}}%
\pgfpathlineto{\pgfqpoint{5.594060in}{1.352609in}}%
\pgfpathlineto{\pgfqpoint{5.593988in}{1.863357in}}%
\pgfpathlineto{\pgfqpoint{5.594076in}{1.636767in}}%
\pgfpathlineto{\pgfqpoint{5.594160in}{1.856012in}}%
\pgfpathlineto{\pgfqpoint{5.594415in}{1.211987in}}%
\pgfpathlineto{\pgfqpoint{5.595190in}{1.701987in}}%
\pgfpathlineto{\pgfqpoint{5.595887in}{1.889240in}}%
\pgfpathlineto{\pgfqpoint{5.596191in}{1.384696in}}%
\pgfpathlineto{\pgfqpoint{5.596246in}{1.711251in}}%
\pgfpathlineto{\pgfqpoint{5.597310in}{1.375354in}}%
\pgfpathlineto{\pgfqpoint{5.596621in}{1.867128in}}%
\pgfpathlineto{\pgfqpoint{5.597354in}{1.642819in}}%
\pgfpathlineto{\pgfqpoint{5.597415in}{1.866165in}}%
\pgfpathlineto{\pgfqpoint{5.598048in}{1.397384in}}%
\pgfpathlineto{\pgfqpoint{5.598460in}{1.651878in}}%
\pgfpathlineto{\pgfqpoint{5.599524in}{1.378851in}}%
\pgfpathlineto{\pgfqpoint{5.599431in}{1.888969in}}%
\pgfpathlineto{\pgfqpoint{5.599535in}{1.687921in}}%
\pgfpathlineto{\pgfqpoint{5.599617in}{1.852734in}}%
\pgfpathlineto{\pgfqpoint{5.600509in}{1.311549in}}%
\pgfpathlineto{\pgfqpoint{5.600646in}{1.717756in}}%
\pgfpathlineto{\pgfqpoint{5.601584in}{1.872252in}}%
\pgfpathlineto{\pgfqpoint{5.600700in}{1.315805in}}%
\pgfpathlineto{\pgfqpoint{5.601726in}{1.718270in}}%
\pgfpathlineto{\pgfqpoint{5.602086in}{1.270079in}}%
\pgfpathlineto{\pgfqpoint{5.602641in}{1.847844in}}%
\pgfpathlineto{\pgfqpoint{5.602836in}{1.660837in}}%
\pgfpathlineto{\pgfqpoint{5.603944in}{1.849620in}}%
\pgfpathlineto{\pgfqpoint{5.603635in}{1.215059in}}%
\pgfpathlineto{\pgfqpoint{5.603949in}{1.705777in}}%
\pgfpathlineto{\pgfqpoint{5.604491in}{1.352040in}}%
\pgfpathlineto{\pgfqpoint{5.604789in}{1.846672in}}%
\pgfpathlineto{\pgfqpoint{5.605059in}{1.441552in}}%
\pgfpathlineto{\pgfqpoint{5.606096in}{1.828513in}}%
\pgfpathlineto{\pgfqpoint{5.605394in}{1.018609in}}%
\pgfpathlineto{\pgfqpoint{5.606172in}{1.781253in}}%
\pgfpathlineto{\pgfqpoint{5.606824in}{0.983194in}}%
\pgfpathlineto{\pgfqpoint{5.606845in}{1.845775in}}%
\pgfpathlineto{\pgfqpoint{5.607287in}{1.711579in}}%
\pgfpathlineto{\pgfqpoint{5.607776in}{1.846698in}}%
\pgfpathlineto{\pgfqpoint{5.608302in}{1.284173in}}%
\pgfpathlineto{\pgfqpoint{5.608388in}{1.703117in}}%
\pgfpathlineto{\pgfqpoint{5.608597in}{1.223656in}}%
\pgfpathlineto{\pgfqpoint{5.609294in}{1.841445in}}%
\pgfpathlineto{\pgfqpoint{5.609503in}{1.563413in}}%
\pgfpathlineto{\pgfqpoint{5.609717in}{1.855881in}}%
\pgfpathlineto{\pgfqpoint{5.610417in}{1.272988in}}%
\pgfpathlineto{\pgfqpoint{5.610588in}{1.825277in}}%
\pgfpathlineto{\pgfqpoint{5.611659in}{1.295310in}}%
\pgfpathlineto{\pgfqpoint{5.610774in}{1.829127in}}%
\pgfpathlineto{\pgfqpoint{5.611696in}{1.524041in}}%
\pgfpathlineto{\pgfqpoint{5.612515in}{1.856957in}}%
\pgfpathlineto{\pgfqpoint{5.612223in}{1.272465in}}%
\pgfpathlineto{\pgfqpoint{5.612807in}{1.826833in}}%
\pgfpathlineto{\pgfqpoint{5.613879in}{1.350928in}}%
\pgfpathlineto{\pgfqpoint{5.612855in}{1.865089in}}%
\pgfpathlineto{\pgfqpoint{5.613916in}{1.745294in}}%
\pgfpathlineto{\pgfqpoint{5.614857in}{0.781057in}}%
\pgfpathlineto{\pgfqpoint{5.614334in}{1.874933in}}%
\pgfpathlineto{\pgfqpoint{5.615032in}{1.636403in}}%
\pgfpathlineto{\pgfqpoint{5.615269in}{1.842252in}}%
\pgfpathlineto{\pgfqpoint{5.615480in}{1.208046in}}%
\pgfpathlineto{\pgfqpoint{5.616139in}{1.738005in}}%
\pgfpathlineto{\pgfqpoint{5.616882in}{1.382746in}}%
\pgfpathlineto{\pgfqpoint{5.616829in}{1.854098in}}%
\pgfpathlineto{\pgfqpoint{5.617244in}{1.568876in}}%
\pgfpathlineto{\pgfqpoint{5.617554in}{1.833500in}}%
\pgfpathlineto{\pgfqpoint{5.618331in}{1.300047in}}%
\pgfpathlineto{\pgfqpoint{5.618357in}{1.709940in}}%
\pgfpathlineto{\pgfqpoint{5.619085in}{1.269036in}}%
\pgfpathlineto{\pgfqpoint{5.619158in}{1.850951in}}%
\pgfpathlineto{\pgfqpoint{5.619466in}{1.708250in}}%
\pgfpathlineto{\pgfqpoint{5.619702in}{1.825398in}}%
\pgfpathlineto{\pgfqpoint{5.619540in}{1.358616in}}%
\pgfpathlineto{\pgfqpoint{5.619827in}{1.621480in}}%
\pgfpathlineto{\pgfqpoint{5.619926in}{1.208720in}}%
\pgfpathlineto{\pgfqpoint{5.620177in}{1.859066in}}%
\pgfpathlineto{\pgfqpoint{5.620933in}{1.701147in}}%
\pgfpathlineto{\pgfqpoint{5.621271in}{1.835710in}}%
\pgfpathlineto{\pgfqpoint{5.621989in}{1.309222in}}%
\pgfpathlineto{\pgfqpoint{5.622036in}{1.628938in}}%
\pgfpathlineto{\pgfqpoint{5.622394in}{1.229355in}}%
\pgfpathlineto{\pgfqpoint{5.622363in}{1.839015in}}%
\pgfpathlineto{\pgfqpoint{5.623146in}{1.563595in}}%
\pgfpathlineto{\pgfqpoint{5.623798in}{1.852163in}}%
\pgfpathlineto{\pgfqpoint{5.623384in}{1.423133in}}%
\pgfpathlineto{\pgfqpoint{5.624253in}{1.736708in}}%
\pgfpathlineto{\pgfqpoint{5.624341in}{1.832452in}}%
\pgfpathlineto{\pgfqpoint{5.625363in}{1.332970in}}%
\pgfpathlineto{\pgfqpoint{5.625667in}{1.836050in}}%
\pgfpathlineto{\pgfqpoint{5.626388in}{1.268052in}}%
\pgfpathlineto{\pgfqpoint{5.626475in}{1.688904in}}%
\pgfpathlineto{\pgfqpoint{5.626717in}{1.866378in}}%
\pgfpathlineto{\pgfqpoint{5.626804in}{1.254729in}}%
\pgfpathlineto{\pgfqpoint{5.627589in}{1.791803in}}%
\pgfpathlineto{\pgfqpoint{5.627651in}{1.260043in}}%
\pgfpathlineto{\pgfqpoint{5.628532in}{1.861687in}}%
\pgfpathlineto{\pgfqpoint{5.628701in}{1.672332in}}%
\pgfpathlineto{\pgfqpoint{5.629799in}{1.858421in}}%
\pgfpathlineto{\pgfqpoint{5.629748in}{1.319422in}}%
\pgfpathlineto{\pgfqpoint{5.629814in}{1.705602in}}%
\pgfpathlineto{\pgfqpoint{5.629830in}{1.224771in}}%
\pgfpathlineto{\pgfqpoint{5.630288in}{1.868232in}}%
\pgfpathlineto{\pgfqpoint{5.630920in}{1.527318in}}%
\pgfpathlineto{\pgfqpoint{5.631321in}{1.846231in}}%
\pgfpathlineto{\pgfqpoint{5.631281in}{1.206601in}}%
\pgfpathlineto{\pgfqpoint{5.632033in}{1.747728in}}%
\pgfpathlineto{\pgfqpoint{5.632068in}{1.879445in}}%
\pgfpathlineto{\pgfqpoint{5.632798in}{1.184421in}}%
\pgfpathlineto{\pgfqpoint{5.633107in}{1.729903in}}%
\pgfpathlineto{\pgfqpoint{5.633577in}{1.371956in}}%
\pgfpathlineto{\pgfqpoint{5.633653in}{1.823211in}}%
\pgfpathlineto{\pgfqpoint{5.634219in}{1.628385in}}%
\pgfpathlineto{\pgfqpoint{5.635167in}{1.829190in}}%
\pgfpathlineto{\pgfqpoint{5.634603in}{1.301431in}}%
\pgfpathlineto{\pgfqpoint{5.635333in}{1.776097in}}%
\pgfpathlineto{\pgfqpoint{5.635429in}{1.241054in}}%
\pgfpathlineto{\pgfqpoint{5.636168in}{1.853823in}}%
\pgfpathlineto{\pgfqpoint{5.636440in}{1.681095in}}%
\pgfpathlineto{\pgfqpoint{5.637202in}{1.838697in}}%
\pgfpathlineto{\pgfqpoint{5.636656in}{1.344378in}}%
\pgfpathlineto{\pgfqpoint{5.637548in}{1.732975in}}%
\pgfpathlineto{\pgfqpoint{5.638419in}{1.303874in}}%
\pgfpathlineto{\pgfqpoint{5.637899in}{1.820360in}}%
\pgfpathlineto{\pgfqpoint{5.638664in}{1.583008in}}%
\pgfpathlineto{\pgfqpoint{5.638709in}{1.830846in}}%
\pgfpathlineto{\pgfqpoint{5.639263in}{1.387845in}}%
\pgfpathlineto{\pgfqpoint{5.639776in}{1.728712in}}%
\pgfpathlineto{\pgfqpoint{5.640045in}{1.316873in}}%
\pgfpathlineto{\pgfqpoint{5.640453in}{1.834000in}}%
\pgfpathlineto{\pgfqpoint{5.640886in}{1.675088in}}%
\pgfpathlineto{\pgfqpoint{5.641139in}{1.372982in}}%
\pgfpathlineto{\pgfqpoint{5.641968in}{1.834063in}}%
\pgfpathlineto{\pgfqpoint{5.641993in}{1.672174in}}%
\pgfpathlineto{\pgfqpoint{5.642513in}{1.877490in}}%
\pgfpathlineto{\pgfqpoint{5.642255in}{1.198030in}}%
\pgfpathlineto{\pgfqpoint{5.643082in}{1.755684in}}%
\pgfpathlineto{\pgfqpoint{5.643734in}{1.354202in}}%
\pgfpathlineto{\pgfqpoint{5.644015in}{1.849696in}}%
\pgfpathlineto{\pgfqpoint{5.644193in}{1.711755in}}%
\pgfpathlineto{\pgfqpoint{5.644720in}{1.298034in}}%
\pgfpathlineto{\pgfqpoint{5.644774in}{1.860134in}}%
\pgfpathlineto{\pgfqpoint{5.645301in}{1.719671in}}%
\pgfpathlineto{\pgfqpoint{5.646225in}{1.829372in}}%
\pgfpathlineto{\pgfqpoint{5.645724in}{1.287555in}}%
\pgfpathlineto{\pgfqpoint{5.646401in}{1.739568in}}%
\pgfpathlineto{\pgfqpoint{5.646489in}{1.330084in}}%
\pgfpathlineto{\pgfqpoint{5.646519in}{1.851032in}}%
\pgfpathlineto{\pgfqpoint{5.647513in}{1.681864in}}%
\pgfpathlineto{\pgfqpoint{5.647782in}{1.230340in}}%
\pgfpathlineto{\pgfqpoint{5.648061in}{1.828632in}}%
\pgfpathlineto{\pgfqpoint{5.648618in}{1.490928in}}%
\pgfpathlineto{\pgfqpoint{5.649076in}{1.846812in}}%
\pgfpathlineto{\pgfqpoint{5.649359in}{1.374753in}}%
\pgfpathlineto{\pgfqpoint{5.649724in}{1.666977in}}%
\pgfpathlineto{\pgfqpoint{5.649928in}{1.347208in}}%
\pgfpathlineto{\pgfqpoint{5.650069in}{1.838800in}}%
\pgfpathlineto{\pgfqpoint{5.650837in}{1.426391in}}%
\pgfpathlineto{\pgfqpoint{5.651196in}{1.840819in}}%
\pgfpathlineto{\pgfqpoint{5.651540in}{1.266031in}}%
\pgfpathlineto{\pgfqpoint{5.651947in}{1.633419in}}%
\pgfpathlineto{\pgfqpoint{5.652276in}{1.837157in}}%
\pgfpathlineto{\pgfqpoint{5.652687in}{1.173469in}}%
\pgfpathlineto{\pgfqpoint{5.653040in}{1.605653in}}%
\pgfpathlineto{\pgfqpoint{5.653836in}{1.067597in}}%
\pgfpathlineto{\pgfqpoint{5.653566in}{1.843490in}}%
\pgfpathlineto{\pgfqpoint{5.654130in}{1.457520in}}%
\pgfpathlineto{\pgfqpoint{5.654929in}{1.819446in}}%
\pgfpathlineto{\pgfqpoint{5.654636in}{1.004221in}}%
\pgfpathlineto{\pgfqpoint{5.655246in}{1.767421in}}%
\pgfpathlineto{\pgfqpoint{5.655534in}{1.240464in}}%
\pgfpathlineto{\pgfqpoint{5.655616in}{1.835422in}}%
\pgfpathlineto{\pgfqpoint{5.656359in}{1.666484in}}%
\pgfpathlineto{\pgfqpoint{5.656848in}{1.833507in}}%
\pgfpathlineto{\pgfqpoint{5.657212in}{1.159761in}}%
\pgfpathlineto{\pgfqpoint{5.657455in}{1.696486in}}%
\pgfpathlineto{\pgfqpoint{5.657508in}{1.284933in}}%
\pgfpathlineto{\pgfqpoint{5.658558in}{1.827403in}}%
\pgfpathlineto{\pgfqpoint{5.658568in}{1.609402in}}%
\pgfpathlineto{\pgfqpoint{5.659439in}{1.831180in}}%
\pgfpathlineto{\pgfqpoint{5.659501in}{1.230239in}}%
\pgfpathlineto{\pgfqpoint{5.659639in}{1.574508in}}%
\pgfpathlineto{\pgfqpoint{5.660622in}{1.091603in}}%
\pgfpathlineto{\pgfqpoint{5.660465in}{1.853708in}}%
\pgfpathlineto{\pgfqpoint{5.660745in}{1.617566in}}%
\pgfpathlineto{\pgfqpoint{5.660935in}{1.826248in}}%
\pgfpathlineto{\pgfqpoint{5.661712in}{1.337359in}}%
\pgfpathlineto{\pgfqpoint{5.661858in}{1.714650in}}%
\pgfpathlineto{\pgfqpoint{5.662614in}{1.310357in}}%
\pgfpathlineto{\pgfqpoint{5.662680in}{1.830903in}}%
\pgfpathlineto{\pgfqpoint{5.662968in}{1.682343in}}%
\pgfpathlineto{\pgfqpoint{5.663383in}{1.836660in}}%
\pgfpathlineto{\pgfqpoint{5.663600in}{1.218584in}}%
\pgfpathlineto{\pgfqpoint{5.664085in}{1.720961in}}%
\pgfpathlineto{\pgfqpoint{5.664363in}{1.278279in}}%
\pgfpathlineto{\pgfqpoint{5.665048in}{1.828014in}}%
\pgfpathlineto{\pgfqpoint{5.665194in}{1.678344in}}%
\pgfpathlineto{\pgfqpoint{5.665808in}{1.319221in}}%
\pgfpathlineto{\pgfqpoint{5.666216in}{1.836848in}}%
\pgfpathlineto{\pgfqpoint{5.666286in}{1.612562in}}%
\pgfpathlineto{\pgfqpoint{5.666469in}{1.832125in}}%
\pgfpathlineto{\pgfqpoint{5.667118in}{1.243026in}}%
\pgfpathlineto{\pgfqpoint{5.667394in}{1.683789in}}%
\pgfpathlineto{\pgfqpoint{5.668005in}{1.329996in}}%
\pgfpathlineto{\pgfqpoint{5.667800in}{1.822010in}}%
\pgfpathlineto{\pgfqpoint{5.668504in}{1.595868in}}%
\pgfpathlineto{\pgfqpoint{5.668597in}{1.827793in}}%
\pgfpathlineto{\pgfqpoint{5.669536in}{1.221786in}}%
\pgfpathlineto{\pgfqpoint{5.669615in}{1.724140in}}%
\pgfpathlineto{\pgfqpoint{5.669666in}{1.792254in}}%
\pgfpathlineto{\pgfqpoint{5.669741in}{1.439413in}}%
\pgfpathlineto{\pgfqpoint{5.669847in}{1.775040in}}%
\pgfpathlineto{\pgfqpoint{5.670654in}{1.307280in}}%
\pgfpathlineto{\pgfqpoint{5.670483in}{1.830401in}}%
\pgfpathlineto{\pgfqpoint{5.670955in}{1.573830in}}%
\pgfpathlineto{\pgfqpoint{5.671270in}{1.829862in}}%
\pgfpathlineto{\pgfqpoint{5.671113in}{1.186313in}}%
\pgfpathlineto{\pgfqpoint{5.672070in}{1.733524in}}%
\pgfpathlineto{\pgfqpoint{5.673163in}{1.097630in}}%
\pgfpathlineto{\pgfqpoint{5.673038in}{1.861764in}}%
\pgfpathlineto{\pgfqpoint{5.673181in}{1.609219in}}%
\pgfpathlineto{\pgfqpoint{5.673922in}{1.821853in}}%
\pgfpathlineto{\pgfqpoint{5.673618in}{1.167657in}}%
\pgfpathlineto{\pgfqpoint{5.674294in}{1.732503in}}%
\pgfpathlineto{\pgfqpoint{5.674308in}{1.358823in}}%
\pgfpathlineto{\pgfqpoint{5.674932in}{1.828648in}}%
\pgfpathlineto{\pgfqpoint{5.675400in}{1.660438in}}%
\pgfpathlineto{\pgfqpoint{5.676397in}{1.810283in}}%
\pgfpathlineto{\pgfqpoint{5.675816in}{1.212519in}}%
\pgfpathlineto{\pgfqpoint{5.676507in}{1.687903in}}%
\pgfpathlineto{\pgfqpoint{5.677315in}{1.239847in}}%
\pgfpathlineto{\pgfqpoint{5.677073in}{1.826938in}}%
\pgfpathlineto{\pgfqpoint{5.677616in}{1.649615in}}%
\pgfpathlineto{\pgfqpoint{5.677625in}{1.817106in}}%
\pgfpathlineto{\pgfqpoint{5.678140in}{1.302798in}}%
\pgfpathlineto{\pgfqpoint{5.678708in}{1.751027in}}%
\pgfpathlineto{\pgfqpoint{5.678713in}{1.300012in}}%
\pgfpathlineto{\pgfqpoint{5.679381in}{1.845012in}}%
\pgfpathlineto{\pgfqpoint{5.679821in}{1.596145in}}%
\pgfpathlineto{\pgfqpoint{5.680410in}{1.202302in}}%
\pgfpathlineto{\pgfqpoint{5.680722in}{1.847849in}}%
\pgfpathlineto{\pgfqpoint{5.680867in}{1.557098in}}%
\pgfpathlineto{\pgfqpoint{5.680948in}{1.841963in}}%
\pgfpathlineto{\pgfqpoint{5.681739in}{0.977201in}}%
\pgfpathlineto{\pgfqpoint{5.681983in}{1.785235in}}%
\pgfpathlineto{\pgfqpoint{5.682339in}{1.322924in}}%
\pgfpathlineto{\pgfqpoint{5.682091in}{1.837954in}}%
\pgfpathlineto{\pgfqpoint{5.683095in}{1.736172in}}%
\pgfpathlineto{\pgfqpoint{5.683365in}{1.210807in}}%
\pgfpathlineto{\pgfqpoint{5.683532in}{1.831680in}}%
\pgfpathlineto{\pgfqpoint{5.684201in}{1.761105in}}%
\pgfpathlineto{\pgfqpoint{5.684725in}{1.826229in}}%
\pgfpathlineto{\pgfqpoint{5.684945in}{1.216214in}}%
\pgfpathlineto{\pgfqpoint{5.685312in}{1.815769in}}%
\pgfpathlineto{\pgfqpoint{5.685920in}{1.200122in}}%
\pgfpathlineto{\pgfqpoint{5.685545in}{1.830343in}}%
\pgfpathlineto{\pgfqpoint{5.686425in}{1.665537in}}%
\pgfpathlineto{\pgfqpoint{5.687157in}{1.101238in}}%
\pgfpathlineto{\pgfqpoint{5.686514in}{1.815168in}}%
\pgfpathlineto{\pgfqpoint{5.687535in}{1.429407in}}%
\pgfpathlineto{\pgfqpoint{5.687744in}{1.854427in}}%
\pgfpathlineto{\pgfqpoint{5.688571in}{1.269479in}}%
\pgfpathlineto{\pgfqpoint{5.688647in}{1.643249in}}%
\pgfpathlineto{\pgfqpoint{5.689428in}{1.267730in}}%
\pgfpathlineto{\pgfqpoint{5.689108in}{1.832970in}}%
\pgfpathlineto{\pgfqpoint{5.689760in}{1.540994in}}%
\pgfpathlineto{\pgfqpoint{5.690592in}{1.800242in}}%
\pgfpathlineto{\pgfqpoint{5.690079in}{1.323044in}}%
\pgfpathlineto{\pgfqpoint{5.690870in}{1.691252in}}%
\pgfpathlineto{\pgfqpoint{5.691087in}{1.353650in}}%
\pgfpathlineto{\pgfqpoint{5.691360in}{1.820676in}}%
\pgfpathlineto{\pgfqpoint{5.691982in}{1.688491in}}%
\pgfpathlineto{\pgfqpoint{5.692858in}{1.819307in}}%
\pgfpathlineto{\pgfqpoint{5.693012in}{1.233311in}}%
\pgfpathlineto{\pgfqpoint{5.693060in}{1.668897in}}%
\pgfpathlineto{\pgfqpoint{5.693688in}{1.245433in}}%
\pgfpathlineto{\pgfqpoint{5.693126in}{1.844688in}}%
\pgfpathlineto{\pgfqpoint{5.694166in}{1.608696in}}%
\pgfpathlineto{\pgfqpoint{5.694552in}{1.841556in}}%
\pgfpathlineto{\pgfqpoint{5.694819in}{1.172034in}}%
\pgfpathlineto{\pgfqpoint{5.695278in}{1.681495in}}%
\pgfpathlineto{\pgfqpoint{5.696008in}{1.820419in}}%
\pgfpathlineto{\pgfqpoint{5.696143in}{0.984972in}}%
\pgfpathlineto{\pgfqpoint{5.696370in}{1.675570in}}%
\pgfpathlineto{\pgfqpoint{5.697324in}{1.269781in}}%
\pgfpathlineto{\pgfqpoint{5.696745in}{1.842436in}}%
\pgfpathlineto{\pgfqpoint{5.697476in}{1.734118in}}%
\pgfpathlineto{\pgfqpoint{5.697528in}{1.130322in}}%
\pgfpathlineto{\pgfqpoint{5.697502in}{1.839038in}}%
\pgfpathlineto{\pgfqpoint{5.698575in}{1.737922in}}%
\pgfpathlineto{\pgfqpoint{5.699493in}{1.829061in}}%
\pgfpathlineto{\pgfqpoint{5.699303in}{1.274097in}}%
\pgfpathlineto{\pgfqpoint{5.699680in}{1.649366in}}%
\pgfpathlineto{\pgfqpoint{5.699714in}{1.820608in}}%
\pgfpathlineto{\pgfqpoint{5.700285in}{1.273873in}}%
\pgfpathlineto{\pgfqpoint{5.700756in}{1.721927in}}%
\pgfpathlineto{\pgfqpoint{5.701096in}{1.232061in}}%
\pgfpathlineto{\pgfqpoint{5.701148in}{1.824986in}}%
\pgfpathlineto{\pgfqpoint{5.701868in}{1.355119in}}%
\pgfpathlineto{\pgfqpoint{5.702388in}{1.814631in}}%
\pgfpathlineto{\pgfqpoint{5.702405in}{1.244960in}}%
\pgfpathlineto{\pgfqpoint{5.702981in}{1.790072in}}%
\pgfpathlineto{\pgfqpoint{5.703736in}{1.262726in}}%
\pgfpathlineto{\pgfqpoint{5.703341in}{1.826011in}}%
\pgfpathlineto{\pgfqpoint{5.704092in}{1.724559in}}%
\pgfpathlineto{\pgfqpoint{5.704271in}{1.209907in}}%
\pgfpathlineto{\pgfqpoint{5.705041in}{1.817567in}}%
\pgfpathlineto{\pgfqpoint{5.705195in}{1.587454in}}%
\pgfpathlineto{\pgfqpoint{5.705199in}{1.834622in}}%
\pgfpathlineto{\pgfqpoint{5.705404in}{1.306487in}}%
\pgfpathlineto{\pgfqpoint{5.706304in}{1.585609in}}%
\pgfpathlineto{\pgfqpoint{5.707015in}{1.208918in}}%
\pgfpathlineto{\pgfqpoint{5.706542in}{1.825313in}}%
\pgfpathlineto{\pgfqpoint{5.707393in}{1.565557in}}%
\pgfpathlineto{\pgfqpoint{5.707737in}{1.810286in}}%
\pgfpathlineto{\pgfqpoint{5.708123in}{1.256283in}}%
\pgfpathlineto{\pgfqpoint{5.708505in}{1.730159in}}%
\pgfpathlineto{\pgfqpoint{5.709271in}{1.312999in}}%
\pgfpathlineto{\pgfqpoint{5.709398in}{1.817234in}}%
\pgfpathlineto{\pgfqpoint{5.709614in}{1.798656in}}%
\pgfpathlineto{\pgfqpoint{5.710407in}{1.369734in}}%
\pgfpathlineto{\pgfqpoint{5.709956in}{1.814847in}}%
\pgfpathlineto{\pgfqpoint{5.710724in}{1.625649in}}%
\pgfpathlineto{\pgfqpoint{5.711082in}{1.810360in}}%
\pgfpathlineto{\pgfqpoint{5.710762in}{1.358137in}}%
\pgfpathlineto{\pgfqpoint{5.711835in}{1.656396in}}%
\pgfpathlineto{\pgfqpoint{5.712893in}{1.839202in}}%
\pgfpathlineto{\pgfqpoint{5.712860in}{1.095391in}}%
\pgfpathlineto{\pgfqpoint{5.712906in}{1.806793in}}%
\pgfpathlineto{\pgfqpoint{5.713317in}{1.212263in}}%
\pgfpathlineto{\pgfqpoint{5.713497in}{1.815497in}}%
\pgfpathlineto{\pgfqpoint{5.714020in}{1.257704in}}%
\pgfpathlineto{\pgfqpoint{5.715019in}{1.842423in}}%
\pgfpathlineto{\pgfqpoint{5.714129in}{1.213611in}}%
\pgfpathlineto{\pgfqpoint{5.715136in}{1.649664in}}%
\pgfpathlineto{\pgfqpoint{5.715311in}{1.367850in}}%
\pgfpathlineto{\pgfqpoint{5.716190in}{1.809634in}}%
\pgfpathlineto{\pgfqpoint{5.716236in}{1.681681in}}%
\pgfpathlineto{\pgfqpoint{5.716693in}{1.831627in}}%
\pgfpathlineto{\pgfqpoint{5.716976in}{1.207096in}}%
\pgfpathlineto{\pgfqpoint{5.717337in}{1.647722in}}%
\pgfpathlineto{\pgfqpoint{5.717548in}{1.376663in}}%
\pgfpathlineto{\pgfqpoint{5.717349in}{1.828807in}}%
\pgfpathlineto{\pgfqpoint{5.718444in}{1.568637in}}%
\pgfpathlineto{\pgfqpoint{5.718981in}{1.825800in}}%
\pgfpathlineto{\pgfqpoint{5.719200in}{1.330633in}}%
\pgfpathlineto{\pgfqpoint{5.719552in}{1.612140in}}%
\pgfpathlineto{\pgfqpoint{5.719597in}{1.336599in}}%
\pgfpathlineto{\pgfqpoint{5.719634in}{1.830779in}}%
\pgfpathlineto{\pgfqpoint{5.720657in}{1.516606in}}%
\pgfpathlineto{\pgfqpoint{5.721570in}{1.804391in}}%
\pgfpathlineto{\pgfqpoint{5.721011in}{1.345285in}}%
\pgfpathlineto{\pgfqpoint{5.721771in}{1.754850in}}%
\pgfpathlineto{\pgfqpoint{5.722141in}{1.205984in}}%
\pgfpathlineto{\pgfqpoint{5.722485in}{1.794016in}}%
\pgfpathlineto{\pgfqpoint{5.722887in}{1.653170in}}%
\pgfpathlineto{\pgfqpoint{5.723403in}{1.827301in}}%
\pgfpathlineto{\pgfqpoint{5.723542in}{1.385805in}}%
\pgfpathlineto{\pgfqpoint{5.724000in}{1.708447in}}%
\pgfpathlineto{\pgfqpoint{5.724857in}{1.250938in}}%
\pgfpathlineto{\pgfqpoint{5.724281in}{1.814785in}}%
\pgfpathlineto{\pgfqpoint{5.725110in}{1.605291in}}%
\pgfpathlineto{\pgfqpoint{5.725846in}{1.798443in}}%
\pgfpathlineto{\pgfqpoint{5.726143in}{1.145656in}}%
\pgfpathlineto{\pgfqpoint{5.726217in}{1.724387in}}%
\pgfpathlineto{\pgfqpoint{5.726639in}{1.209407in}}%
\pgfpathlineto{\pgfqpoint{5.727321in}{1.816482in}}%
\pgfpathlineto{\pgfqpoint{5.727325in}{1.530071in}}%
\pgfpathlineto{\pgfqpoint{5.727446in}{1.798212in}}%
\pgfpathlineto{\pgfqpoint{5.727373in}{1.235248in}}%
\pgfpathlineto{\pgfqpoint{5.728434in}{1.558660in}}%
\pgfpathlineto{\pgfqpoint{5.728778in}{1.845136in}}%
\pgfpathlineto{\pgfqpoint{5.729427in}{1.370458in}}%
\pgfpathlineto{\pgfqpoint{5.729565in}{1.641477in}}%
\pgfpathlineto{\pgfqpoint{5.729895in}{1.355673in}}%
\pgfpathlineto{\pgfqpoint{5.729625in}{1.800066in}}%
\pgfpathlineto{\pgfqpoint{5.730672in}{1.681547in}}%
\pgfpathlineto{\pgfqpoint{5.730885in}{1.821669in}}%
\pgfpathlineto{\pgfqpoint{5.731520in}{1.239195in}}%
\pgfpathlineto{\pgfqpoint{5.731777in}{1.708113in}}%
\pgfpathlineto{\pgfqpoint{5.731965in}{1.247456in}}%
\pgfpathlineto{\pgfqpoint{5.732694in}{1.796269in}}%
\pgfpathlineto{\pgfqpoint{5.732891in}{1.581312in}}%
\pgfpathlineto{\pgfqpoint{5.733143in}{1.365104in}}%
\pgfpathlineto{\pgfqpoint{5.733206in}{1.810805in}}%
\pgfpathlineto{\pgfqpoint{5.733977in}{1.668949in}}%
\pgfpathlineto{\pgfqpoint{5.734974in}{1.814867in}}%
\pgfpathlineto{\pgfqpoint{5.734787in}{1.187544in}}%
\pgfpathlineto{\pgfqpoint{5.735081in}{1.790029in}}%
\pgfpathlineto{\pgfqpoint{5.735829in}{0.921364in}}%
\pgfpathlineto{\pgfqpoint{5.735583in}{1.844596in}}%
\pgfpathlineto{\pgfqpoint{5.736191in}{1.691619in}}%
\pgfpathlineto{\pgfqpoint{5.736655in}{1.798248in}}%
\pgfpathlineto{\pgfqpoint{5.737000in}{1.368843in}}%
\pgfpathlineto{\pgfqpoint{5.737273in}{1.525192in}}%
\pgfpathlineto{\pgfqpoint{5.737380in}{1.258145in}}%
\pgfpathlineto{\pgfqpoint{5.737704in}{1.806173in}}%
\pgfpathlineto{\pgfqpoint{5.738380in}{1.665461in}}%
\pgfpathlineto{\pgfqpoint{5.739398in}{1.253513in}}%
\pgfpathlineto{\pgfqpoint{5.738811in}{1.842153in}}%
\pgfpathlineto{\pgfqpoint{5.739489in}{1.578914in}}%
\pgfpathlineto{\pgfqpoint{5.739571in}{1.817089in}}%
\pgfpathlineto{\pgfqpoint{5.739961in}{1.259489in}}%
\pgfpathlineto{\pgfqpoint{5.740606in}{1.708145in}}%
\pgfpathlineto{\pgfqpoint{5.740834in}{1.312252in}}%
\pgfpathlineto{\pgfqpoint{5.741065in}{1.812713in}}%
\pgfpathlineto{\pgfqpoint{5.741622in}{1.632992in}}%
\pgfpathlineto{\pgfqpoint{5.742053in}{1.803890in}}%
\pgfpathlineto{\pgfqpoint{5.742237in}{1.176223in}}%
\pgfpathlineto{\pgfqpoint{5.742726in}{1.506927in}}%
\pgfpathlineto{\pgfqpoint{5.743000in}{1.804676in}}%
\pgfpathlineto{\pgfqpoint{5.743207in}{1.205945in}}%
\pgfpathlineto{\pgfqpoint{5.743859in}{1.680612in}}%
\pgfpathlineto{\pgfqpoint{5.744295in}{1.214630in}}%
\pgfpathlineto{\pgfqpoint{5.744260in}{1.781530in}}%
\pgfpathlineto{\pgfqpoint{5.744968in}{1.645604in}}%
\pgfpathlineto{\pgfqpoint{5.745128in}{1.828812in}}%
\pgfpathlineto{\pgfqpoint{5.746036in}{1.294094in}}%
\pgfpathlineto{\pgfqpoint{5.746060in}{1.655221in}}%
\pgfpathlineto{\pgfqpoint{5.747021in}{1.267152in}}%
\pgfpathlineto{\pgfqpoint{5.746831in}{1.817586in}}%
\pgfpathlineto{\pgfqpoint{5.747171in}{1.387046in}}%
\pgfpathlineto{\pgfqpoint{5.747357in}{1.821924in}}%
\pgfpathlineto{\pgfqpoint{5.747682in}{1.328176in}}%
\pgfpathlineto{\pgfqpoint{5.748284in}{1.561229in}}%
\pgfpathlineto{\pgfqpoint{5.748812in}{1.806216in}}%
\pgfpathlineto{\pgfqpoint{5.748473in}{1.312071in}}%
\pgfpathlineto{\pgfqpoint{5.749390in}{1.559720in}}%
\pgfpathlineto{\pgfqpoint{5.749725in}{1.185266in}}%
\pgfpathlineto{\pgfqpoint{5.749982in}{1.818135in}}%
\pgfpathlineto{\pgfqpoint{5.750486in}{1.706394in}}%
\pgfpathlineto{\pgfqpoint{5.750835in}{1.813680in}}%
\pgfpathlineto{\pgfqpoint{5.751138in}{1.243555in}}%
\pgfpathlineto{\pgfqpoint{5.751563in}{1.650429in}}%
\pgfpathlineto{\pgfqpoint{5.751567in}{1.189253in}}%
\pgfpathlineto{\pgfqpoint{5.752160in}{1.818984in}}%
\pgfpathlineto{\pgfqpoint{5.752672in}{1.718073in}}%
\pgfpathlineto{\pgfqpoint{5.753465in}{1.361435in}}%
\pgfpathlineto{\pgfqpoint{5.753507in}{1.802918in}}%
\pgfpathlineto{\pgfqpoint{5.753785in}{1.664968in}}%
\pgfpathlineto{\pgfqpoint{5.754733in}{1.803988in}}%
\pgfpathlineto{\pgfqpoint{5.754801in}{1.248306in}}%
\pgfpathlineto{\pgfqpoint{5.754896in}{1.711679in}}%
\pgfpathlineto{\pgfqpoint{5.755856in}{1.199950in}}%
\pgfpathlineto{\pgfqpoint{5.755230in}{1.797822in}}%
\pgfpathlineto{\pgfqpoint{5.756015in}{1.451541in}}%
\pgfpathlineto{\pgfqpoint{5.756549in}{1.801757in}}%
\pgfpathlineto{\pgfqpoint{5.756644in}{1.078951in}}%
\pgfpathlineto{\pgfqpoint{5.757124in}{1.669319in}}%
\pgfpathlineto{\pgfqpoint{5.757943in}{1.123103in}}%
\pgfpathlineto{\pgfqpoint{5.757396in}{1.797458in}}%
\pgfpathlineto{\pgfqpoint{5.758234in}{1.534715in}}%
\pgfpathlineto{\pgfqpoint{5.758580in}{1.832836in}}%
\pgfpathlineto{\pgfqpoint{5.758836in}{1.294782in}}%
\pgfpathlineto{\pgfqpoint{5.759344in}{1.687645in}}%
\pgfpathlineto{\pgfqpoint{5.759581in}{1.212132in}}%
\pgfpathlineto{\pgfqpoint{5.760313in}{1.822082in}}%
\pgfpathlineto{\pgfqpoint{5.760455in}{1.653214in}}%
\pgfpathlineto{\pgfqpoint{5.761306in}{1.218769in}}%
\pgfpathlineto{\pgfqpoint{5.761017in}{1.827681in}}%
\pgfpathlineto{\pgfqpoint{5.761567in}{1.503404in}}%
\pgfpathlineto{\pgfqpoint{5.762233in}{1.800174in}}%
\pgfpathlineto{\pgfqpoint{5.762050in}{1.369212in}}%
\pgfpathlineto{\pgfqpoint{5.762680in}{1.780974in}}%
\pgfpathlineto{\pgfqpoint{5.763347in}{1.221155in}}%
\pgfpathlineto{\pgfqpoint{5.763578in}{1.799521in}}%
\pgfpathlineto{\pgfqpoint{5.763798in}{1.592153in}}%
\pgfpathlineto{\pgfqpoint{5.764723in}{1.813775in}}%
\pgfpathlineto{\pgfqpoint{5.764846in}{1.286891in}}%
\pgfpathlineto{\pgfqpoint{5.764898in}{1.505328in}}%
\pgfpathlineto{\pgfqpoint{5.765276in}{0.876952in}}%
\pgfpathlineto{\pgfqpoint{5.765035in}{1.801531in}}%
\pgfpathlineto{\pgfqpoint{5.766002in}{1.430718in}}%
\pgfpathlineto{\pgfqpoint{5.766390in}{1.791232in}}%
\pgfpathlineto{\pgfqpoint{5.766512in}{1.147897in}}%
\pgfpathlineto{\pgfqpoint{5.767114in}{1.596539in}}%
\pgfpathlineto{\pgfqpoint{5.768143in}{1.275145in}}%
\pgfpathlineto{\pgfqpoint{5.767708in}{1.816341in}}%
\pgfpathlineto{\pgfqpoint{5.768227in}{1.544086in}}%
\pgfpathlineto{\pgfqpoint{5.768489in}{1.791563in}}%
\pgfpathlineto{\pgfqpoint{5.768279in}{1.234094in}}%
\pgfpathlineto{\pgfqpoint{5.769341in}{1.742520in}}%
\pgfpathlineto{\pgfqpoint{5.770445in}{1.249037in}}%
\pgfpathlineto{\pgfqpoint{5.769973in}{1.799800in}}%
\pgfpathlineto{\pgfqpoint{5.770453in}{1.470913in}}%
\pgfpathlineto{\pgfqpoint{5.771002in}{1.802640in}}%
\pgfpathlineto{\pgfqpoint{5.770691in}{1.258525in}}%
\pgfpathlineto{\pgfqpoint{5.771564in}{1.524788in}}%
\pgfpathlineto{\pgfqpoint{5.771992in}{1.787306in}}%
\pgfpathlineto{\pgfqpoint{5.771784in}{1.225831in}}%
\pgfpathlineto{\pgfqpoint{5.772677in}{1.655754in}}%
\pgfpathlineto{\pgfqpoint{5.773500in}{1.784799in}}%
\pgfpathlineto{\pgfqpoint{5.773416in}{1.255792in}}%
\pgfpathlineto{\pgfqpoint{5.773696in}{1.542554in}}%
\pgfpathlineto{\pgfqpoint{5.774542in}{1.281187in}}%
\pgfpathlineto{\pgfqpoint{5.774291in}{1.804908in}}%
\pgfpathlineto{\pgfqpoint{5.774799in}{1.608250in}}%
\pgfpathlineto{\pgfqpoint{5.775317in}{1.784668in}}%
\pgfpathlineto{\pgfqpoint{5.775636in}{1.193564in}}%
\pgfpathlineto{\pgfqpoint{5.775907in}{1.707318in}}%
\pgfpathlineto{\pgfqpoint{5.776590in}{1.254999in}}%
\pgfpathlineto{\pgfqpoint{5.776290in}{1.792207in}}%
\pgfpathlineto{\pgfqpoint{5.777015in}{1.661949in}}%
\pgfpathlineto{\pgfqpoint{5.777488in}{1.815707in}}%
\pgfpathlineto{\pgfqpoint{5.778035in}{1.230355in}}%
\pgfpathlineto{\pgfqpoint{5.778117in}{1.633698in}}%
\pgfpathlineto{\pgfqpoint{5.778782in}{1.117135in}}%
\pgfpathlineto{\pgfqpoint{5.778146in}{1.783231in}}%
\pgfpathlineto{\pgfqpoint{5.779224in}{1.655101in}}%
\pgfpathlineto{\pgfqpoint{5.779722in}{1.771059in}}%
\pgfpathlineto{\pgfqpoint{5.779650in}{1.236061in}}%
\pgfpathlineto{\pgfqpoint{5.780334in}{1.680562in}}%
\pgfpathlineto{\pgfqpoint{5.781017in}{1.157494in}}%
\pgfpathlineto{\pgfqpoint{5.781321in}{1.783338in}}%
\pgfpathlineto{\pgfqpoint{5.781442in}{1.589760in}}%
\pgfpathlineto{\pgfqpoint{5.782219in}{1.782457in}}%
\pgfpathlineto{\pgfqpoint{5.781478in}{1.260522in}}%
\pgfpathlineto{\pgfqpoint{5.782550in}{1.549382in}}%
\pgfpathlineto{\pgfqpoint{5.782568in}{1.798144in}}%
\pgfpathlineto{\pgfqpoint{5.783020in}{1.266043in}}%
\pgfpathlineto{\pgfqpoint{5.783663in}{1.609405in}}%
\pgfpathlineto{\pgfqpoint{5.784518in}{1.230255in}}%
\pgfpathlineto{\pgfqpoint{5.783812in}{1.781305in}}%
\pgfpathlineto{\pgfqpoint{5.784762in}{1.648263in}}%
\pgfpathlineto{\pgfqpoint{5.785833in}{1.806623in}}%
\pgfpathlineto{\pgfqpoint{5.785561in}{1.275201in}}%
\pgfpathlineto{\pgfqpoint{5.785869in}{1.627961in}}%
\pgfpathlineto{\pgfqpoint{5.786888in}{1.039947in}}%
\pgfpathlineto{\pgfqpoint{5.786254in}{1.795518in}}%
\pgfpathlineto{\pgfqpoint{5.786945in}{1.631443in}}%
\pgfpathlineto{\pgfqpoint{5.787473in}{1.799699in}}%
\pgfpathlineto{\pgfqpoint{5.787680in}{1.223552in}}%
\pgfpathlineto{\pgfqpoint{5.788053in}{1.589985in}}%
\pgfpathlineto{\pgfqpoint{5.788221in}{1.074021in}}%
\pgfpathlineto{\pgfqpoint{5.788765in}{1.792851in}}%
\pgfpathlineto{\pgfqpoint{5.789046in}{1.647614in}}%
\pgfpathlineto{\pgfqpoint{5.789774in}{1.798292in}}%
\pgfpathlineto{\pgfqpoint{5.789428in}{1.211895in}}%
\pgfpathlineto{\pgfqpoint{5.790155in}{1.698402in}}%
\pgfpathlineto{\pgfqpoint{5.790400in}{1.301756in}}%
\pgfpathlineto{\pgfqpoint{5.790218in}{1.787083in}}%
\pgfpathlineto{\pgfqpoint{5.791266in}{1.663083in}}%
\pgfpathlineto{\pgfqpoint{5.791949in}{1.175402in}}%
\pgfpathlineto{\pgfqpoint{5.792328in}{1.789284in}}%
\pgfpathlineto{\pgfqpoint{5.792370in}{1.611711in}}%
\pgfpathlineto{\pgfqpoint{5.792446in}{1.810612in}}%
\pgfpathlineto{\pgfqpoint{5.793020in}{1.131051in}}%
\pgfpathlineto{\pgfqpoint{5.793478in}{1.659660in}}%
\pgfpathlineto{\pgfqpoint{5.794448in}{1.022357in}}%
\pgfpathlineto{\pgfqpoint{5.793908in}{1.777062in}}%
\pgfpathlineto{\pgfqpoint{5.794586in}{1.497580in}}%
\pgfpathlineto{\pgfqpoint{5.795606in}{1.784568in}}%
\pgfpathlineto{\pgfqpoint{5.795419in}{1.130969in}}%
\pgfpathlineto{\pgfqpoint{5.795699in}{1.658448in}}%
\pgfpathlineto{\pgfqpoint{5.796536in}{1.288514in}}%
\pgfpathlineto{\pgfqpoint{5.796505in}{1.791852in}}%
\pgfpathlineto{\pgfqpoint{5.796795in}{1.554479in}}%
\pgfpathlineto{\pgfqpoint{5.797884in}{1.787698in}}%
\pgfpathlineto{\pgfqpoint{5.797259in}{1.236947in}}%
\pgfpathlineto{\pgfqpoint{5.797905in}{1.666023in}}%
\pgfpathlineto{\pgfqpoint{5.798577in}{1.145646in}}%
\pgfpathlineto{\pgfqpoint{5.798200in}{1.789893in}}%
\pgfpathlineto{\pgfqpoint{5.799009in}{1.551854in}}%
\pgfpathlineto{\pgfqpoint{5.799608in}{1.816478in}}%
\pgfpathlineto{\pgfqpoint{5.799197in}{1.115105in}}%
\pgfpathlineto{\pgfqpoint{5.800120in}{1.634072in}}%
\pgfpathlineto{\pgfqpoint{5.800304in}{1.265459in}}%
\pgfpathlineto{\pgfqpoint{5.800789in}{1.813724in}}%
\pgfpathlineto{\pgfqpoint{5.801228in}{1.393100in}}%
\pgfpathlineto{\pgfqpoint{5.801821in}{1.773825in}}%
\pgfpathlineto{\pgfqpoint{5.802269in}{0.879423in}}%
\pgfpathlineto{\pgfqpoint{5.802341in}{1.634241in}}%
\pgfpathlineto{\pgfqpoint{5.803226in}{1.781964in}}%
\pgfpathlineto{\pgfqpoint{5.803321in}{1.278585in}}%
\pgfpathlineto{\pgfqpoint{5.803453in}{1.678297in}}%
\pgfpathlineto{\pgfqpoint{5.803599in}{1.181494in}}%
\pgfpathlineto{\pgfqpoint{5.804482in}{1.798157in}}%
\pgfpathlineto{\pgfqpoint{5.804563in}{1.480146in}}%
\pgfpathlineto{\pgfqpoint{5.805070in}{1.782766in}}%
\pgfpathlineto{\pgfqpoint{5.805046in}{1.184221in}}%
\pgfpathlineto{\pgfqpoint{5.805674in}{1.673493in}}%
\pgfpathlineto{\pgfqpoint{5.806317in}{1.211724in}}%
\pgfpathlineto{\pgfqpoint{5.806266in}{1.797510in}}%
\pgfpathlineto{\pgfqpoint{5.806784in}{1.583668in}}%
\pgfpathlineto{\pgfqpoint{5.807215in}{1.760777in}}%
\pgfpathlineto{\pgfqpoint{5.807876in}{0.949160in}}%
\pgfpathlineto{\pgfqpoint{5.807896in}{1.591450in}}%
\pgfpathlineto{\pgfqpoint{5.808807in}{1.761475in}}%
\pgfpathlineto{\pgfqpoint{5.808556in}{1.049195in}}%
\pgfpathlineto{\pgfqpoint{5.809004in}{1.607646in}}%
\pgfpathlineto{\pgfqpoint{5.809666in}{1.279356in}}%
\pgfpathlineto{\pgfqpoint{5.809208in}{1.795841in}}%
\pgfpathlineto{\pgfqpoint{5.810093in}{1.538069in}}%
\pgfpathlineto{\pgfqpoint{5.810490in}{1.789947in}}%
\pgfpathlineto{\pgfqpoint{5.810153in}{1.165313in}}%
\pgfpathlineto{\pgfqpoint{5.811202in}{1.688713in}}%
\pgfpathlineto{\pgfqpoint{5.811255in}{1.270498in}}%
\pgfpathlineto{\pgfqpoint{5.812249in}{1.808531in}}%
\pgfpathlineto{\pgfqpoint{5.812319in}{1.369272in}}%
\pgfpathlineto{\pgfqpoint{5.812561in}{1.790608in}}%
\pgfpathlineto{\pgfqpoint{5.813349in}{1.075859in}}%
\pgfpathlineto{\pgfqpoint{5.813432in}{1.606408in}}%
\pgfpathlineto{\pgfqpoint{5.813753in}{1.782853in}}%
\pgfpathlineto{\pgfqpoint{5.813746in}{1.092178in}}%
\pgfpathlineto{\pgfqpoint{5.814539in}{1.709650in}}%
\pgfpathlineto{\pgfqpoint{5.814863in}{1.258841in}}%
\pgfpathlineto{\pgfqpoint{5.815031in}{1.799931in}}%
\pgfpathlineto{\pgfqpoint{5.815650in}{1.371288in}}%
\pgfpathlineto{\pgfqpoint{5.816525in}{1.777108in}}%
\pgfpathlineto{\pgfqpoint{5.816061in}{1.212171in}}%
\pgfpathlineto{\pgfqpoint{5.816761in}{1.659170in}}%
\pgfpathlineto{\pgfqpoint{5.817441in}{1.347516in}}%
\pgfpathlineto{\pgfqpoint{5.817604in}{1.782295in}}%
\pgfpathlineto{\pgfqpoint{5.817870in}{1.627237in}}%
\pgfpathlineto{\pgfqpoint{5.818056in}{1.767081in}}%
\pgfpathlineto{\pgfqpoint{5.818217in}{1.184370in}}%
\pgfpathlineto{\pgfqpoint{5.818972in}{1.715887in}}%
\pgfpathlineto{\pgfqpoint{5.819174in}{1.291262in}}%
\pgfpathlineto{\pgfqpoint{5.819719in}{1.768574in}}%
\pgfpathlineto{\pgfqpoint{5.820081in}{1.664834in}}%
\pgfpathlineto{\pgfqpoint{5.820992in}{1.782764in}}%
\pgfpathlineto{\pgfqpoint{5.821057in}{1.257831in}}%
\pgfpathlineto{\pgfqpoint{5.821178in}{1.543729in}}%
\pgfpathlineto{\pgfqpoint{5.821350in}{1.149764in}}%
\pgfpathlineto{\pgfqpoint{5.821217in}{1.786038in}}%
\pgfpathlineto{\pgfqpoint{5.822284in}{1.605148in}}%
\pgfpathlineto{\pgfqpoint{5.822660in}{1.790112in}}%
\pgfpathlineto{\pgfqpoint{5.822650in}{0.875030in}}%
\pgfpathlineto{\pgfqpoint{5.823395in}{1.633635in}}%
\pgfpathlineto{\pgfqpoint{5.824002in}{1.255798in}}%
\pgfpathlineto{\pgfqpoint{5.823579in}{1.781092in}}%
\pgfpathlineto{\pgfqpoint{5.824476in}{1.568148in}}%
\pgfpathlineto{\pgfqpoint{5.824840in}{1.785280in}}%
\pgfpathlineto{\pgfqpoint{5.824695in}{1.072512in}}%
\pgfpathlineto{\pgfqpoint{5.825587in}{1.646144in}}%
\pgfpathlineto{\pgfqpoint{5.826660in}{1.332525in}}%
\pgfpathlineto{\pgfqpoint{5.826480in}{1.768624in}}%
\pgfpathlineto{\pgfqpoint{5.826698in}{1.523557in}}%
\pgfpathlineto{\pgfqpoint{5.827435in}{1.828195in}}%
\pgfpathlineto{\pgfqpoint{5.826743in}{1.198265in}}%
\pgfpathlineto{\pgfqpoint{5.827810in}{1.709765in}}%
\pgfpathlineto{\pgfqpoint{5.827835in}{1.141890in}}%
\pgfpathlineto{\pgfqpoint{5.828539in}{1.773617in}}%
\pgfpathlineto{\pgfqpoint{5.828922in}{1.514561in}}%
\pgfpathlineto{\pgfqpoint{5.829282in}{1.792708in}}%
\pgfpathlineto{\pgfqpoint{5.829151in}{0.975434in}}%
\pgfpathlineto{\pgfqpoint{5.830034in}{1.647007in}}%
\pgfpathlineto{\pgfqpoint{5.830168in}{1.785219in}}%
\pgfpathlineto{\pgfqpoint{5.830679in}{0.927317in}}%
\pgfpathlineto{\pgfqpoint{5.831124in}{1.527668in}}%
\pgfpathlineto{\pgfqpoint{5.832120in}{1.223147in}}%
\pgfpathlineto{\pgfqpoint{5.831524in}{1.760683in}}%
\pgfpathlineto{\pgfqpoint{5.832218in}{1.598429in}}%
\pgfpathlineto{\pgfqpoint{5.832487in}{1.793110in}}%
\pgfpathlineto{\pgfqpoint{5.832335in}{1.216885in}}%
\pgfpathlineto{\pgfqpoint{5.833324in}{1.654628in}}%
\pgfpathlineto{\pgfqpoint{5.833801in}{1.178621in}}%
\pgfpathlineto{\pgfqpoint{5.834296in}{1.789392in}}%
\pgfpathlineto{\pgfqpoint{5.834431in}{1.720312in}}%
\pgfpathlineto{\pgfqpoint{5.835083in}{1.767764in}}%
\pgfpathlineto{\pgfqpoint{5.834567in}{1.110512in}}%
\pgfpathlineto{\pgfqpoint{5.835520in}{1.579456in}}%
\pgfpathlineto{\pgfqpoint{5.836612in}{1.316902in}}%
\pgfpathlineto{\pgfqpoint{5.836264in}{1.796149in}}%
\pgfpathlineto{\pgfqpoint{5.836627in}{1.600472in}}%
\pgfpathlineto{\pgfqpoint{5.836737in}{1.788365in}}%
\pgfpathlineto{\pgfqpoint{5.837297in}{1.188549in}}%
\pgfpathlineto{\pgfqpoint{5.837735in}{1.586534in}}%
\pgfpathlineto{\pgfqpoint{5.837963in}{1.105179in}}%
\pgfpathlineto{\pgfqpoint{5.838066in}{1.779825in}}%
\pgfpathlineto{\pgfqpoint{5.838831in}{1.623464in}}%
\pgfpathlineto{\pgfqpoint{5.839002in}{1.774530in}}%
\pgfpathlineto{\pgfqpoint{5.839643in}{1.118055in}}%
\pgfpathlineto{\pgfqpoint{5.839939in}{1.647218in}}%
\pgfpathlineto{\pgfqpoint{5.840343in}{1.124607in}}%
\pgfpathlineto{\pgfqpoint{5.840976in}{1.784709in}}%
\pgfpathlineto{\pgfqpoint{5.841047in}{1.611879in}}%
\pgfpathlineto{\pgfqpoint{5.841881in}{1.782004in}}%
\pgfpathlineto{\pgfqpoint{5.841066in}{1.103779in}}%
\pgfpathlineto{\pgfqpoint{5.842147in}{1.552751in}}%
\pgfpathlineto{\pgfqpoint{5.842224in}{1.091486in}}%
\pgfpathlineto{\pgfqpoint{5.842610in}{1.771834in}}%
\pgfpathlineto{\pgfqpoint{5.843256in}{1.624649in}}%
\pgfpathlineto{\pgfqpoint{5.843989in}{1.063142in}}%
\pgfpathlineto{\pgfqpoint{5.844045in}{1.766983in}}%
\pgfpathlineto{\pgfqpoint{5.844365in}{1.658369in}}%
\pgfpathlineto{\pgfqpoint{5.845357in}{1.791946in}}%
\pgfpathlineto{\pgfqpoint{5.845204in}{1.009656in}}%
\pgfpathlineto{\pgfqpoint{5.845465in}{1.596410in}}%
\pgfpathlineto{\pgfqpoint{5.846008in}{1.143350in}}%
\pgfpathlineto{\pgfqpoint{5.845894in}{1.765444in}}%
\pgfpathlineto{\pgfqpoint{5.846568in}{1.538630in}}%
\pgfpathlineto{\pgfqpoint{5.847088in}{1.761782in}}%
\pgfpathlineto{\pgfqpoint{5.846990in}{1.155888in}}%
\pgfpathlineto{\pgfqpoint{5.847678in}{1.670111in}}%
\pgfpathlineto{\pgfqpoint{5.847971in}{1.174053in}}%
\pgfpathlineto{\pgfqpoint{5.847959in}{1.784386in}}%
\pgfpathlineto{\pgfqpoint{5.848788in}{1.578187in}}%
\pgfpathlineto{\pgfqpoint{5.849077in}{1.771686in}}%
\pgfpathlineto{\pgfqpoint{5.849469in}{0.975972in}}%
\pgfpathlineto{\pgfqpoint{5.849897in}{1.663769in}}%
\pgfpathlineto{\pgfqpoint{5.850140in}{1.209665in}}%
\pgfpathlineto{\pgfqpoint{5.850328in}{1.788492in}}%
\pgfpathlineto{\pgfqpoint{5.851008in}{1.558853in}}%
\pgfpathlineto{\pgfqpoint{5.851280in}{1.241258in}}%
\pgfpathlineto{\pgfqpoint{5.851174in}{1.753754in}}%
\pgfpathlineto{\pgfqpoint{5.851477in}{1.625130in}}%
\pgfpathlineto{\pgfqpoint{5.852187in}{1.769379in}}%
\pgfpathlineto{\pgfqpoint{5.851951in}{0.997094in}}%
\pgfpathlineto{\pgfqpoint{5.852580in}{1.678322in}}%
\pgfpathlineto{\pgfqpoint{5.853535in}{1.133792in}}%
\pgfpathlineto{\pgfqpoint{5.853002in}{1.778741in}}%
\pgfpathlineto{\pgfqpoint{5.853692in}{1.465668in}}%
\pgfpathlineto{\pgfqpoint{5.854113in}{1.766387in}}%
\pgfpathlineto{\pgfqpoint{5.854705in}{0.926637in}}%
\pgfpathlineto{\pgfqpoint{5.854804in}{1.684823in}}%
\pgfpathlineto{\pgfqpoint{5.855251in}{1.169732in}}%
\pgfpathlineto{\pgfqpoint{5.855548in}{1.777387in}}%
\pgfpathlineto{\pgfqpoint{5.855916in}{1.590263in}}%
\pgfpathlineto{\pgfqpoint{5.856772in}{1.784476in}}%
\pgfpathlineto{\pgfqpoint{5.856739in}{1.289082in}}%
\pgfpathlineto{\pgfqpoint{5.857014in}{1.696807in}}%
\pgfpathlineto{\pgfqpoint{5.857989in}{1.268118in}}%
\pgfpathlineto{\pgfqpoint{5.857888in}{1.780547in}}%
\pgfpathlineto{\pgfqpoint{5.858126in}{1.622711in}}%
\pgfpathlineto{\pgfqpoint{5.858707in}{1.781982in}}%
\pgfpathlineto{\pgfqpoint{5.858239in}{1.116618in}}%
\pgfpathlineto{\pgfqpoint{5.859176in}{1.660743in}}%
\pgfpathlineto{\pgfqpoint{5.860138in}{0.835304in}}%
\pgfpathlineto{\pgfqpoint{5.860132in}{1.773273in}}%
\pgfpathlineto{\pgfqpoint{5.860286in}{1.451243in}}%
\pgfpathlineto{\pgfqpoint{5.860606in}{1.794626in}}%
\pgfpathlineto{\pgfqpoint{5.860600in}{1.213489in}}%
\pgfpathlineto{\pgfqpoint{5.861399in}{1.627996in}}%
\pgfpathlineto{\pgfqpoint{5.861771in}{1.077120in}}%
\pgfpathlineto{\pgfqpoint{5.862240in}{1.766282in}}%
\pgfpathlineto{\pgfqpoint{5.862512in}{1.463598in}}%
\pgfpathlineto{\pgfqpoint{5.863292in}{1.791251in}}%
\pgfpathlineto{\pgfqpoint{5.863101in}{1.182397in}}%
\pgfpathlineto{\pgfqpoint{5.863625in}{1.580408in}}%
\pgfpathlineto{\pgfqpoint{5.864192in}{1.758857in}}%
\pgfpathlineto{\pgfqpoint{5.864673in}{1.062278in}}%
\pgfpathlineto{\pgfqpoint{5.864732in}{1.615017in}}%
\pgfpathlineto{\pgfqpoint{5.865408in}{1.114559in}}%
\pgfpathlineto{\pgfqpoint{5.865315in}{1.771995in}}%
\pgfpathlineto{\pgfqpoint{5.865845in}{1.489743in}}%
\pgfpathlineto{\pgfqpoint{5.866844in}{1.074373in}}%
\pgfpathlineto{\pgfqpoint{5.866488in}{1.774112in}}%
\pgfpathlineto{\pgfqpoint{5.866943in}{1.631507in}}%
\pgfpathlineto{\pgfqpoint{5.866966in}{1.728742in}}%
\pgfpathlineto{\pgfqpoint{5.866957in}{1.493279in}}%
\pgfpathlineto{\pgfqpoint{5.867007in}{1.566805in}}%
\pgfpathlineto{\pgfqpoint{5.867054in}{1.242321in}}%
\pgfpathlineto{\pgfqpoint{5.867602in}{1.774117in}}%
\pgfpathlineto{\pgfqpoint{5.868117in}{1.490956in}}%
\pgfpathlineto{\pgfqpoint{5.868155in}{1.770361in}}%
\pgfpathlineto{\pgfqpoint{5.868443in}{1.016381in}}%
\pgfpathlineto{\pgfqpoint{5.869227in}{1.633280in}}%
\pgfpathlineto{\pgfqpoint{5.869383in}{1.158724in}}%
\pgfpathlineto{\pgfqpoint{5.870296in}{1.799727in}}%
\pgfpathlineto{\pgfqpoint{5.870337in}{1.563958in}}%
\pgfpathlineto{\pgfqpoint{5.870629in}{1.755258in}}%
\pgfpathlineto{\pgfqpoint{5.870898in}{0.917487in}}%
\pgfpathlineto{\pgfqpoint{5.871426in}{1.614303in}}%
\pgfpathlineto{\pgfqpoint{5.871608in}{1.198464in}}%
\pgfpathlineto{\pgfqpoint{5.871432in}{1.769812in}}%
\pgfpathlineto{\pgfqpoint{5.872536in}{1.624615in}}%
\pgfpathlineto{\pgfqpoint{5.873333in}{1.746907in}}%
\pgfpathlineto{\pgfqpoint{5.873643in}{1.289865in}}%
\pgfpathlineto{\pgfqpoint{5.874151in}{1.772970in}}%
\pgfpathlineto{\pgfqpoint{5.874696in}{1.112631in}}%
\pgfpathlineto{\pgfqpoint{5.874756in}{1.733195in}}%
\pgfpathlineto{\pgfqpoint{5.875274in}{1.233746in}}%
\pgfpathlineto{\pgfqpoint{5.875263in}{1.769344in}}%
\pgfpathlineto{\pgfqpoint{5.875869in}{1.585751in}}%
\pgfpathlineto{\pgfqpoint{5.876306in}{1.068956in}}%
\pgfpathlineto{\pgfqpoint{5.876577in}{1.758649in}}%
\pgfpathlineto{\pgfqpoint{5.876970in}{1.577961in}}%
\pgfpathlineto{\pgfqpoint{5.877878in}{1.776814in}}%
\pgfpathlineto{\pgfqpoint{5.877263in}{1.116703in}}%
\pgfpathlineto{\pgfqpoint{5.878083in}{1.663271in}}%
\pgfpathlineto{\pgfqpoint{5.878310in}{1.252344in}}%
\pgfpathlineto{\pgfqpoint{5.878449in}{1.771865in}}%
\pgfpathlineto{\pgfqpoint{5.879195in}{1.574336in}}%
\pgfpathlineto{\pgfqpoint{5.880192in}{1.175614in}}%
\pgfpathlineto{\pgfqpoint{5.879365in}{1.760566in}}%
\pgfpathlineto{\pgfqpoint{5.880293in}{1.361736in}}%
\pgfpathlineto{\pgfqpoint{5.880341in}{1.756294in}}%
\pgfpathlineto{\pgfqpoint{5.881143in}{1.161841in}}%
\pgfpathlineto{\pgfqpoint{5.881406in}{1.584641in}}%
\pgfpathlineto{\pgfqpoint{5.881848in}{1.769950in}}%
\pgfpathlineto{\pgfqpoint{5.881451in}{1.209321in}}%
\pgfpathlineto{\pgfqpoint{5.882515in}{1.606844in}}%
\pgfpathlineto{\pgfqpoint{5.883389in}{1.038527in}}%
\pgfpathlineto{\pgfqpoint{5.882855in}{1.746942in}}%
\pgfpathlineto{\pgfqpoint{5.883585in}{1.655170in}}%
\pgfpathlineto{\pgfqpoint{5.884544in}{1.778370in}}%
\pgfpathlineto{\pgfqpoint{5.883692in}{1.270344in}}%
\pgfpathlineto{\pgfqpoint{5.884692in}{1.628848in}}%
\pgfpathlineto{\pgfqpoint{5.885606in}{1.330378in}}%
\pgfpathlineto{\pgfqpoint{5.885388in}{1.762343in}}%
\pgfpathlineto{\pgfqpoint{5.885790in}{1.528684in}}%
\pgfpathlineto{\pgfqpoint{5.886066in}{1.775789in}}%
\pgfpathlineto{\pgfqpoint{5.886200in}{1.086172in}}%
\pgfpathlineto{\pgfqpoint{5.886902in}{1.614451in}}%
\pgfpathlineto{\pgfqpoint{5.887700in}{1.752887in}}%
\pgfpathlineto{\pgfqpoint{5.887494in}{1.206395in}}%
\pgfpathlineto{\pgfqpoint{5.887922in}{1.700638in}}%
\pgfpathlineto{\pgfqpoint{5.888211in}{1.293161in}}%
\pgfpathlineto{\pgfqpoint{5.888402in}{1.748764in}}%
\pgfpathlineto{\pgfqpoint{5.889034in}{1.621178in}}%
\pgfpathlineto{\pgfqpoint{5.889123in}{1.295072in}}%
\pgfpathlineto{\pgfqpoint{5.889657in}{1.769831in}}%
\pgfpathlineto{\pgfqpoint{5.890146in}{1.514500in}}%
\pgfpathlineto{\pgfqpoint{5.891054in}{1.754835in}}%
\pgfpathlineto{\pgfqpoint{5.890996in}{1.113667in}}%
\pgfpathlineto{\pgfqpoint{5.891255in}{1.636552in}}%
\pgfpathlineto{\pgfqpoint{5.891913in}{1.244161in}}%
\pgfpathlineto{\pgfqpoint{5.891280in}{1.756827in}}%
\pgfpathlineto{\pgfqpoint{5.892367in}{1.551435in}}%
\pgfpathlineto{\pgfqpoint{5.892960in}{1.760878in}}%
\pgfpathlineto{\pgfqpoint{5.892853in}{1.097476in}}%
\pgfpathlineto{\pgfqpoint{5.893475in}{1.447409in}}%
\pgfpathlineto{\pgfqpoint{5.894121in}{1.782365in}}%
\pgfpathlineto{\pgfqpoint{5.893703in}{1.049089in}}%
\pgfpathlineto{\pgfqpoint{5.894586in}{1.701902in}}%
\pgfpathlineto{\pgfqpoint{5.895116in}{1.229112in}}%
\pgfpathlineto{\pgfqpoint{5.894663in}{1.748469in}}%
\pgfpathlineto{\pgfqpoint{5.895697in}{1.659553in}}%
\pgfpathlineto{\pgfqpoint{5.896640in}{1.741359in}}%
\pgfpathlineto{\pgfqpoint{5.896038in}{1.170825in}}%
\pgfpathlineto{\pgfqpoint{5.896667in}{1.488747in}}%
\pgfpathlineto{\pgfqpoint{5.896757in}{1.156582in}}%
\pgfpathlineto{\pgfqpoint{5.897341in}{1.785222in}}%
\pgfpathlineto{\pgfqpoint{5.897778in}{1.396728in}}%
\pgfpathlineto{\pgfqpoint{5.898179in}{1.753423in}}%
\pgfpathlineto{\pgfqpoint{5.898063in}{1.265723in}}%
\pgfpathlineto{\pgfqpoint{5.898889in}{1.425805in}}%
\pgfpathlineto{\pgfqpoint{5.899327in}{1.759841in}}%
\pgfpathlineto{\pgfqpoint{5.899067in}{1.278324in}}%
\pgfpathlineto{\pgfqpoint{5.899997in}{1.698134in}}%
\pgfpathlineto{\pgfqpoint{5.899999in}{1.018209in}}%
\pgfpathlineto{\pgfqpoint{5.900975in}{1.773030in}}%
\pgfpathlineto{\pgfqpoint{5.901107in}{1.511983in}}%
\pgfpathlineto{\pgfqpoint{5.901193in}{1.768130in}}%
\pgfpathlineto{\pgfqpoint{5.902180in}{1.216263in}}%
\pgfpathlineto{\pgfqpoint{5.902217in}{1.626492in}}%
\pgfpathlineto{\pgfqpoint{5.902628in}{1.127833in}}%
\pgfpathlineto{\pgfqpoint{5.903065in}{1.739245in}}%
\pgfpathlineto{\pgfqpoint{5.903327in}{1.563421in}}%
\pgfpathlineto{\pgfqpoint{5.903330in}{1.563387in}}%
\pgfpathlineto{\pgfqpoint{5.903630in}{1.253614in}}%
\pgfpathlineto{\pgfqpoint{5.903349in}{1.783966in}}%
\pgfpathlineto{\pgfqpoint{5.904437in}{1.501006in}}%
\pgfpathlineto{\pgfqpoint{5.904907in}{1.780840in}}%
\pgfpathlineto{\pgfqpoint{5.904752in}{1.176820in}}%
\pgfpathlineto{\pgfqpoint{5.905546in}{1.652669in}}%
\pgfpathlineto{\pgfqpoint{5.906116in}{1.201251in}}%
\pgfpathlineto{\pgfqpoint{5.905733in}{1.787652in}}%
\pgfpathlineto{\pgfqpoint{5.906656in}{1.686082in}}%
\pgfpathlineto{\pgfqpoint{5.907205in}{1.120174in}}%
\pgfpathlineto{\pgfqpoint{5.907383in}{1.750510in}}%
\pgfpathlineto{\pgfqpoint{5.907770in}{1.358273in}}%
\pgfpathlineto{\pgfqpoint{5.908604in}{1.759674in}}%
\pgfpathlineto{\pgfqpoint{5.908541in}{1.176380in}}%
\pgfpathlineto{\pgfqpoint{5.908882in}{1.664696in}}%
\pgfpathlineto{\pgfqpoint{5.909982in}{1.126608in}}%
\pgfpathlineto{\pgfqpoint{5.909800in}{1.756861in}}%
\pgfpathlineto{\pgfqpoint{5.909990in}{1.559408in}}%
\pgfpathlineto{\pgfqpoint{5.910504in}{1.752755in}}%
\pgfpathlineto{\pgfqpoint{5.910464in}{1.084063in}}%
\pgfpathlineto{\pgfqpoint{5.911098in}{1.619624in}}%
\pgfpathlineto{\pgfqpoint{5.911288in}{1.067403in}}%
\pgfpathlineto{\pgfqpoint{5.911348in}{1.766206in}}%
\pgfpathlineto{\pgfqpoint{5.912209in}{1.602850in}}%
\pgfpathlineto{\pgfqpoint{5.913257in}{1.743500in}}%
\pgfpathlineto{\pgfqpoint{5.913330in}{1.226666in}}%
\pgfpathlineto{\pgfqpoint{5.914375in}{1.754411in}}%
\pgfpathlineto{\pgfqpoint{5.914182in}{1.212301in}}%
\pgfpathlineto{\pgfqpoint{5.914442in}{1.551322in}}%
\pgfpathlineto{\pgfqpoint{5.915485in}{1.734330in}}%
\pgfpathlineto{\pgfqpoint{5.914664in}{1.083595in}}%
\pgfpathlineto{\pgfqpoint{5.915555in}{1.633333in}}%
\pgfpathlineto{\pgfqpoint{5.916407in}{1.029341in}}%
\pgfpathlineto{\pgfqpoint{5.916044in}{1.745871in}}%
\pgfpathlineto{\pgfqpoint{5.916664in}{1.530197in}}%
\pgfpathlineto{\pgfqpoint{5.917401in}{1.745820in}}%
\pgfpathlineto{\pgfqpoint{5.916833in}{1.156187in}}%
\pgfpathlineto{\pgfqpoint{5.917773in}{1.627669in}}%
\pgfpathlineto{\pgfqpoint{5.917786in}{1.220982in}}%
\pgfpathlineto{\pgfqpoint{5.918288in}{1.769122in}}%
\pgfpathlineto{\pgfqpoint{5.918885in}{1.540771in}}%
\pgfpathlineto{\pgfqpoint{5.919434in}{1.787792in}}%
\pgfpathlineto{\pgfqpoint{5.919664in}{1.191908in}}%
\pgfpathlineto{\pgfqpoint{5.919998in}{1.635218in}}%
\pgfpathlineto{\pgfqpoint{5.920654in}{1.167440in}}%
\pgfpathlineto{\pgfqpoint{5.920965in}{1.749043in}}%
\pgfpathlineto{\pgfqpoint{5.921107in}{1.606583in}}%
\pgfpathlineto{\pgfqpoint{5.922125in}{1.751905in}}%
\pgfpathlineto{\pgfqpoint{5.921178in}{1.141293in}}%
\pgfpathlineto{\pgfqpoint{5.922209in}{1.509813in}}%
\pgfpathlineto{\pgfqpoint{5.922550in}{1.207181in}}%
\pgfpathlineto{\pgfqpoint{5.923041in}{1.749687in}}%
\pgfpathlineto{\pgfqpoint{5.923317in}{1.571386in}}%
\pgfpathlineto{\pgfqpoint{5.923901in}{1.742765in}}%
\pgfpathlineto{\pgfqpoint{5.923853in}{1.077405in}}%
\pgfpathlineto{\pgfqpoint{5.924426in}{1.686012in}}%
\pgfpathlineto{\pgfqpoint{5.924816in}{1.143400in}}%
\pgfpathlineto{\pgfqpoint{5.925353in}{1.768042in}}%
\pgfpathlineto{\pgfqpoint{5.925536in}{1.555515in}}%
\pgfpathlineto{\pgfqpoint{5.926419in}{1.742573in}}%
\pgfpathlineto{\pgfqpoint{5.926549in}{1.065645in}}%
\pgfpathlineto{\pgfqpoint{5.926645in}{1.559752in}}%
\pgfpathlineto{\pgfqpoint{5.927387in}{1.169215in}}%
\pgfpathlineto{\pgfqpoint{5.927235in}{1.750115in}}%
\pgfpathlineto{\pgfqpoint{5.927743in}{1.616354in}}%
\pgfpathlineto{\pgfqpoint{5.928170in}{1.747484in}}%
\pgfpathlineto{\pgfqpoint{5.927981in}{1.103449in}}%
\pgfpathlineto{\pgfqpoint{5.928847in}{1.487354in}}%
\pgfpathlineto{\pgfqpoint{5.928857in}{1.175177in}}%
\pgfpathlineto{\pgfqpoint{5.928864in}{1.766274in}}%
\pgfpathlineto{\pgfqpoint{5.929955in}{1.508565in}}%
\pgfpathlineto{\pgfqpoint{5.930307in}{1.754863in}}%
\pgfpathlineto{\pgfqpoint{5.930791in}{1.207425in}}%
\pgfpathlineto{\pgfqpoint{5.931065in}{1.559085in}}%
\pgfpathlineto{\pgfqpoint{5.931764in}{1.045696in}}%
\pgfpathlineto{\pgfqpoint{5.931536in}{1.755125in}}%
\pgfpathlineto{\pgfqpoint{5.932172in}{1.548860in}}%
\pgfpathlineto{\pgfqpoint{5.932344in}{1.764190in}}%
\pgfpathlineto{\pgfqpoint{5.933221in}{1.194018in}}%
\pgfpathlineto{\pgfqpoint{5.933281in}{1.547614in}}%
\pgfpathlineto{\pgfqpoint{5.934074in}{1.167328in}}%
\pgfpathlineto{\pgfqpoint{5.933543in}{1.744532in}}%
\pgfpathlineto{\pgfqpoint{5.934393in}{1.523361in}}%
\pgfpathlineto{\pgfqpoint{5.934970in}{1.736503in}}%
\pgfpathlineto{\pgfqpoint{5.934979in}{1.232171in}}%
\pgfpathlineto{\pgfqpoint{5.935504in}{1.601682in}}%
\pgfpathlineto{\pgfqpoint{5.936047in}{1.731827in}}%
\pgfpathlineto{\pgfqpoint{5.936349in}{1.088227in}}%
\pgfpathlineto{\pgfqpoint{5.936609in}{1.704381in}}%
\pgfpathlineto{\pgfqpoint{5.937376in}{1.236589in}}%
\pgfpathlineto{\pgfqpoint{5.937312in}{1.747353in}}%
\pgfpathlineto{\pgfqpoint{5.937719in}{1.521752in}}%
\pgfpathlineto{\pgfqpoint{5.937729in}{1.739832in}}%
\pgfpathlineto{\pgfqpoint{5.938560in}{1.130982in}}%
\pgfpathlineto{\pgfqpoint{5.938829in}{1.578738in}}%
\pgfpathlineto{\pgfqpoint{5.939828in}{0.956799in}}%
\pgfpathlineto{\pgfqpoint{5.939877in}{1.721973in}}%
\pgfpathlineto{\pgfqpoint{5.939936in}{1.544321in}}%
\pgfpathlineto{\pgfqpoint{5.940186in}{1.738870in}}%
\pgfpathlineto{\pgfqpoint{5.940164in}{1.185870in}}%
\pgfpathlineto{\pgfqpoint{5.941042in}{1.605588in}}%
\pgfpathlineto{\pgfqpoint{5.941218in}{1.074391in}}%
\pgfpathlineto{\pgfqpoint{5.941250in}{1.742616in}}%
\pgfpathlineto{\pgfqpoint{5.942153in}{1.549605in}}%
\pgfpathlineto{\pgfqpoint{5.942697in}{1.774555in}}%
\pgfpathlineto{\pgfqpoint{5.942602in}{1.241805in}}%
\pgfpathlineto{\pgfqpoint{5.943261in}{1.539194in}}%
\pgfpathlineto{\pgfqpoint{5.943461in}{1.757066in}}%
\pgfpathlineto{\pgfqpoint{5.943310in}{1.263175in}}%
\pgfpathlineto{\pgfqpoint{5.943697in}{1.581236in}}%
\pgfpathlineto{\pgfqpoint{5.943984in}{1.160011in}}%
\pgfpathlineto{\pgfqpoint{5.944446in}{1.731119in}}%
\pgfpathlineto{\pgfqpoint{5.944808in}{1.398772in}}%
\pgfpathlineto{\pgfqpoint{5.944883in}{1.081153in}}%
\pgfpathlineto{\pgfqpoint{5.944951in}{1.752324in}}%
\pgfpathlineto{\pgfqpoint{5.945916in}{1.426739in}}%
\pgfpathlineto{\pgfqpoint{5.946262in}{1.741341in}}%
\pgfpathlineto{\pgfqpoint{5.946584in}{1.201532in}}%
\pgfpathlineto{\pgfqpoint{5.947029in}{1.632668in}}%
\pgfpathlineto{\pgfqpoint{5.947451in}{1.753024in}}%
\pgfpathlineto{\pgfqpoint{5.947572in}{1.018401in}}%
\pgfpathlineto{\pgfqpoint{5.948112in}{1.690705in}}%
\pgfpathlineto{\pgfqpoint{5.948692in}{1.005863in}}%
\pgfpathlineto{\pgfqpoint{5.948700in}{1.758218in}}%
\pgfpathlineto{\pgfqpoint{5.949221in}{1.690393in}}%
\pgfpathlineto{\pgfqpoint{5.949457in}{1.129920in}}%
\pgfpathlineto{\pgfqpoint{5.949997in}{1.755680in}}%
\pgfpathlineto{\pgfqpoint{5.950333in}{1.621528in}}%
\pgfpathlineto{\pgfqpoint{5.951341in}{1.147981in}}%
\pgfpathlineto{\pgfqpoint{5.950371in}{1.749348in}}%
\pgfpathlineto{\pgfqpoint{5.951441in}{1.516941in}}%
\pgfpathlineto{\pgfqpoint{5.951759in}{1.754447in}}%
\pgfpathlineto{\pgfqpoint{5.952251in}{1.172749in}}%
\pgfpathlineto{\pgfqpoint{5.952551in}{1.661166in}}%
\pgfpathlineto{\pgfqpoint{5.953642in}{1.092917in}}%
\pgfpathlineto{\pgfqpoint{5.953475in}{1.779367in}}%
\pgfpathlineto{\pgfqpoint{5.953663in}{1.511465in}}%
\pgfpathlineto{\pgfqpoint{5.953789in}{1.735487in}}%
\pgfpathlineto{\pgfqpoint{5.953725in}{1.229109in}}%
\pgfpathlineto{\pgfqpoint{5.954775in}{1.637276in}}%
\pgfpathlineto{\pgfqpoint{5.955770in}{1.733930in}}%
\pgfpathlineto{\pgfqpoint{5.955206in}{1.144935in}}%
\pgfpathlineto{\pgfqpoint{5.955867in}{1.539665in}}%
\pgfpathlineto{\pgfqpoint{5.956444in}{1.140018in}}%
\pgfpathlineto{\pgfqpoint{5.956885in}{1.747624in}}%
\pgfpathlineto{\pgfqpoint{5.956977in}{1.407189in}}%
\pgfpathlineto{\pgfqpoint{5.957322in}{1.742094in}}%
\pgfpathlineto{\pgfqpoint{5.957208in}{1.019515in}}%
\pgfpathlineto{\pgfqpoint{5.958087in}{1.563197in}}%
\pgfpathlineto{\pgfqpoint{5.958421in}{1.160526in}}%
\pgfpathlineto{\pgfqpoint{5.958539in}{1.730599in}}%
\pgfpathlineto{\pgfqpoint{5.959199in}{1.489674in}}%
\pgfpathlineto{\pgfqpoint{5.960223in}{1.746279in}}%
\pgfpathlineto{\pgfqpoint{5.959750in}{1.163013in}}%
\pgfpathlineto{\pgfqpoint{5.960305in}{1.563827in}}%
\pgfpathlineto{\pgfqpoint{5.960591in}{1.146029in}}%
\pgfpathlineto{\pgfqpoint{5.961358in}{1.742810in}}%
\pgfpathlineto{\pgfqpoint{5.961414in}{1.562544in}}%
\pgfpathlineto{\pgfqpoint{5.961428in}{1.765453in}}%
\pgfpathlineto{\pgfqpoint{5.962123in}{1.106053in}}%
\pgfpathlineto{\pgfqpoint{5.962519in}{1.613654in}}%
\pgfpathlineto{\pgfqpoint{5.962629in}{1.061163in}}%
\pgfpathlineto{\pgfqpoint{5.962645in}{1.746407in}}%
\pgfpathlineto{\pgfqpoint{5.963631in}{1.413928in}}%
\pgfpathlineto{\pgfqpoint{5.964369in}{1.728127in}}%
\pgfpathlineto{\pgfqpoint{5.963638in}{1.078657in}}%
\pgfpathlineto{\pgfqpoint{5.964742in}{1.578055in}}%
\pgfpathlineto{\pgfqpoint{5.964777in}{1.745265in}}%
\pgfpathlineto{\pgfqpoint{5.965203in}{1.051009in}}%
\pgfpathlineto{\pgfqpoint{5.965850in}{1.599496in}}%
\pgfpathlineto{\pgfqpoint{5.966539in}{1.102181in}}%
\pgfpathlineto{\pgfqpoint{5.966204in}{1.726256in}}%
\pgfpathlineto{\pgfqpoint{5.966961in}{1.609462in}}%
\pgfpathlineto{\pgfqpoint{5.968063in}{1.125670in}}%
\pgfpathlineto{\pgfqpoint{5.967281in}{1.771079in}}%
\pgfpathlineto{\pgfqpoint{5.968072in}{1.406867in}}%
\pgfpathlineto{\pgfqpoint{5.968355in}{1.727720in}}%
\pgfpathlineto{\pgfqpoint{5.968828in}{1.035446in}}%
\pgfpathlineto{\pgfqpoint{5.969183in}{1.553415in}}%
\pgfpathlineto{\pgfqpoint{5.969323in}{1.081174in}}%
\pgfpathlineto{\pgfqpoint{5.970289in}{1.738820in}}%
\pgfpathlineto{\pgfqpoint{5.970294in}{1.522545in}}%
\pgfpathlineto{\pgfqpoint{5.970753in}{1.729245in}}%
\pgfpathlineto{\pgfqpoint{5.970324in}{1.172500in}}%
\pgfpathlineto{\pgfqpoint{5.971404in}{1.669721in}}%
\pgfpathlineto{\pgfqpoint{5.971816in}{1.095345in}}%
\pgfpathlineto{\pgfqpoint{5.972083in}{1.741126in}}%
\pgfpathlineto{\pgfqpoint{5.972515in}{1.567930in}}%
\pgfpathlineto{\pgfqpoint{5.973320in}{1.144442in}}%
\pgfpathlineto{\pgfqpoint{5.973402in}{1.739361in}}%
\pgfpathlineto{\pgfqpoint{5.973622in}{1.609910in}}%
\pgfpathlineto{\pgfqpoint{5.973878in}{1.718108in}}%
\pgfpathlineto{\pgfqpoint{5.974272in}{1.039058in}}%
\pgfpathlineto{\pgfqpoint{5.974707in}{1.519963in}}%
\pgfpathlineto{\pgfqpoint{5.974872in}{1.050481in}}%
\pgfpathlineto{\pgfqpoint{5.975257in}{1.725521in}}%
\pgfpathlineto{\pgfqpoint{5.975817in}{1.424227in}}%
\pgfpathlineto{\pgfqpoint{5.976043in}{1.734256in}}%
\pgfpathlineto{\pgfqpoint{5.976917in}{1.215177in}}%
\pgfpathlineto{\pgfqpoint{5.976931in}{1.570306in}}%
\pgfpathlineto{\pgfqpoint{5.977475in}{1.729517in}}%
\pgfpathlineto{\pgfqpoint{5.977495in}{0.957327in}}%
\pgfpathlineto{\pgfqpoint{5.978037in}{1.504350in}}%
\pgfpathlineto{\pgfqpoint{5.978535in}{1.019129in}}%
\pgfpathlineto{\pgfqpoint{5.978272in}{1.722907in}}%
\pgfpathlineto{\pgfqpoint{5.979133in}{1.541925in}}%
\pgfpathlineto{\pgfqpoint{5.979731in}{1.740268in}}%
\pgfpathlineto{\pgfqpoint{5.979261in}{0.869631in}}%
\pgfpathlineto{\pgfqpoint{5.980245in}{1.648800in}}%
\pgfpathlineto{\pgfqpoint{5.980627in}{1.113011in}}%
\pgfpathlineto{\pgfqpoint{5.980892in}{1.725622in}}%
\pgfpathlineto{\pgfqpoint{5.981358in}{1.542224in}}%
\pgfpathlineto{\pgfqpoint{5.982104in}{1.743969in}}%
\pgfpathlineto{\pgfqpoint{5.982444in}{1.066570in}}%
\pgfpathlineto{\pgfqpoint{5.982471in}{1.602546in}}%
\pgfpathlineto{\pgfqpoint{5.982508in}{1.742720in}}%
\pgfpathlineto{\pgfqpoint{5.982828in}{1.004722in}}%
\pgfpathlineto{\pgfqpoint{5.983571in}{1.578347in}}%
\pgfpathlineto{\pgfqpoint{5.984249in}{1.023666in}}%
\pgfpathlineto{\pgfqpoint{5.983631in}{1.744979in}}%
\pgfpathlineto{\pgfqpoint{5.984680in}{1.601732in}}%
\pgfpathlineto{\pgfqpoint{5.984718in}{1.739637in}}%
\pgfpathlineto{\pgfqpoint{5.985473in}{1.136632in}}%
\pgfpathlineto{\pgfqpoint{5.985733in}{1.640480in}}%
\pgfpathlineto{\pgfqpoint{5.986496in}{1.091321in}}%
\pgfpathlineto{\pgfqpoint{5.986031in}{1.735005in}}%
\pgfpathlineto{\pgfqpoint{5.986845in}{1.456307in}}%
\pgfpathlineto{\pgfqpoint{5.987575in}{1.731788in}}%
\pgfpathlineto{\pgfqpoint{5.987485in}{1.016168in}}%
\pgfpathlineto{\pgfqpoint{5.987957in}{1.626803in}}%
\pgfpathlineto{\pgfqpoint{5.988472in}{0.996565in}}%
\pgfpathlineto{\pgfqpoint{5.988097in}{1.730169in}}%
\pgfpathlineto{\pgfqpoint{5.989067in}{1.616613in}}%
\pgfpathlineto{\pgfqpoint{5.989225in}{1.179548in}}%
\pgfpathlineto{\pgfqpoint{5.989715in}{1.725633in}}%
\pgfpathlineto{\pgfqpoint{5.990180in}{1.559980in}}%
\pgfpathlineto{\pgfqpoint{5.990217in}{1.216154in}}%
\pgfpathlineto{\pgfqpoint{5.990614in}{1.737004in}}%
\pgfpathlineto{\pgfqpoint{5.991041in}{1.666794in}}%
\pgfpathlineto{\pgfqpoint{5.991391in}{1.708086in}}%
\pgfpathlineto{\pgfqpoint{5.991891in}{1.083871in}}%
\pgfpathlineto{\pgfqpoint{5.992135in}{1.597366in}}%
\pgfpathlineto{\pgfqpoint{5.992330in}{1.129689in}}%
\pgfpathlineto{\pgfqpoint{5.992773in}{1.731342in}}%
\pgfpathlineto{\pgfqpoint{5.993247in}{1.447991in}}%
\pgfpathlineto{\pgfqpoint{5.993265in}{1.719696in}}%
\pgfpathlineto{\pgfqpoint{5.994186in}{1.088785in}}%
\pgfpathlineto{\pgfqpoint{5.994351in}{1.474049in}}%
\pgfpathlineto{\pgfqpoint{5.994632in}{1.095517in}}%
\pgfpathlineto{\pgfqpoint{5.994452in}{1.732382in}}%
\pgfpathlineto{\pgfqpoint{5.995462in}{1.418781in}}%
\pgfpathlineto{\pgfqpoint{5.996493in}{1.101955in}}%
\pgfpathlineto{\pgfqpoint{5.995494in}{1.718707in}}%
\pgfpathlineto{\pgfqpoint{5.996570in}{1.516170in}}%
\pgfpathlineto{\pgfqpoint{5.996734in}{1.032256in}}%
\pgfpathlineto{\pgfqpoint{5.996691in}{1.714145in}}%
\pgfpathlineto{\pgfqpoint{5.997637in}{1.433791in}}%
\pgfpathlineto{\pgfqpoint{5.997657in}{1.735430in}}%
\pgfpathlineto{\pgfqpoint{5.998381in}{1.200645in}}%
\pgfpathlineto{\pgfqpoint{5.998749in}{1.586021in}}%
\pgfpathlineto{\pgfqpoint{5.999593in}{1.714994in}}%
\pgfpathlineto{\pgfqpoint{5.999524in}{1.180394in}}%
\pgfpathlineto{\pgfqpoint{5.999849in}{1.615395in}}%
\pgfpathlineto{\pgfqpoint{6.000275in}{1.239449in}}%
\pgfpathlineto{\pgfqpoint{6.000954in}{1.741667in}}%
\pgfpathlineto{\pgfqpoint{6.000958in}{1.580710in}}%
\pgfpathlineto{\pgfqpoint{6.001031in}{1.156074in}}%
\pgfpathlineto{\pgfqpoint{6.001258in}{1.738480in}}%
\pgfpathlineto{\pgfqpoint{6.002055in}{1.288157in}}%
\pgfpathlineto{\pgfqpoint{6.003154in}{1.730814in}}%
\pgfpathlineto{\pgfqpoint{6.002794in}{0.852624in}}%
\pgfpathlineto{\pgfqpoint{6.003167in}{1.637144in}}%
\pgfpathlineto{\pgfqpoint{6.004135in}{1.034924in}}%
\pgfpathlineto{\pgfqpoint{6.003482in}{1.727946in}}%
\pgfpathlineto{\pgfqpoint{6.004277in}{1.552977in}}%
\pgfpathlineto{\pgfqpoint{6.005223in}{1.729314in}}%
\pgfpathlineto{\pgfqpoint{6.005335in}{1.083948in}}%
\pgfpathlineto{\pgfqpoint{6.005383in}{1.636065in}}%
\pgfpathlineto{\pgfqpoint{6.006025in}{1.155807in}}%
\pgfpathlineto{\pgfqpoint{6.005630in}{1.706156in}}%
\pgfpathlineto{\pgfqpoint{6.006496in}{1.464931in}}%
\pgfpathlineto{\pgfqpoint{6.007316in}{1.713533in}}%
\pgfpathlineto{\pgfqpoint{6.007576in}{0.963405in}}%
\pgfpathlineto{\pgfqpoint{6.007607in}{1.662619in}}%
\pgfpathlineto{\pgfqpoint{6.008158in}{1.097823in}}%
\pgfpathlineto{\pgfqpoint{6.007886in}{1.732832in}}%
\pgfpathlineto{\pgfqpoint{6.008720in}{1.449406in}}%
\pgfpathlineto{\pgfqpoint{6.009379in}{1.721457in}}%
\pgfpathlineto{\pgfqpoint{6.009465in}{1.201241in}}%
\pgfpathlineto{\pgfqpoint{6.009834in}{1.577831in}}%
\pgfpathlineto{\pgfqpoint{6.009911in}{1.138477in}}%
\pgfpathlineto{\pgfqpoint{6.010023in}{1.710943in}}%
\pgfpathlineto{\pgfqpoint{6.010943in}{1.664572in}}%
\pgfpathlineto{\pgfqpoint{6.011071in}{1.082396in}}%
\pgfpathlineto{\pgfqpoint{6.011317in}{1.721835in}}%
\pgfpathlineto{\pgfqpoint{6.012065in}{1.373637in}}%
\pgfpathlineto{\pgfqpoint{6.013179in}{1.715418in}}%
\pgfpathlineto{\pgfqpoint{6.012345in}{1.157448in}}%
\pgfpathlineto{\pgfqpoint{6.013183in}{1.613598in}}%
\pgfpathlineto{\pgfqpoint{6.013435in}{1.740452in}}%
\pgfpathlineto{\pgfqpoint{6.013449in}{1.230341in}}%
\pgfpathlineto{\pgfqpoint{6.014258in}{1.675704in}}%
\pgfpathlineto{\pgfqpoint{6.014839in}{1.010075in}}%
\pgfpathlineto{\pgfqpoint{6.014841in}{1.705888in}}%
\pgfpathlineto{\pgfqpoint{6.015368in}{1.598556in}}%
\pgfpathlineto{\pgfqpoint{6.015473in}{1.721449in}}%
\pgfpathlineto{\pgfqpoint{6.015411in}{1.161608in}}%
\pgfpathlineto{\pgfqpoint{6.016475in}{1.592956in}}%
\pgfpathlineto{\pgfqpoint{6.016957in}{1.102610in}}%
\pgfpathlineto{\pgfqpoint{6.016977in}{1.715463in}}%
\pgfpathlineto{\pgfqpoint{6.017585in}{1.419688in}}%
\pgfpathlineto{\pgfqpoint{6.018674in}{1.738692in}}%
\pgfpathlineto{\pgfqpoint{6.018029in}{1.111002in}}%
\pgfpathlineto{\pgfqpoint{6.018697in}{1.625472in}}%
\pgfpathlineto{\pgfqpoint{6.019278in}{1.084095in}}%
\pgfpathlineto{\pgfqpoint{6.018727in}{1.719563in}}%
\pgfpathlineto{\pgfqpoint{6.019807in}{1.356500in}}%
\pgfpathlineto{\pgfqpoint{6.019903in}{1.722685in}}%
\pgfpathlineto{\pgfqpoint{6.019850in}{1.077645in}}%
\pgfpathlineto{\pgfqpoint{6.020919in}{1.628018in}}%
\pgfpathlineto{\pgfqpoint{6.020982in}{1.118213in}}%
\pgfpathlineto{\pgfqpoint{6.021043in}{1.706794in}}%
\pgfpathlineto{\pgfqpoint{6.022030in}{1.633676in}}%
\pgfpathlineto{\pgfqpoint{6.022517in}{1.177410in}}%
\pgfpathlineto{\pgfqpoint{6.022359in}{1.718650in}}%
\pgfpathlineto{\pgfqpoint{6.023133in}{1.513424in}}%
\pgfpathlineto{\pgfqpoint{6.023733in}{1.709943in}}%
\pgfpathlineto{\pgfqpoint{6.023864in}{1.206396in}}%
\pgfpathlineto{\pgfqpoint{6.024242in}{1.515910in}}%
\pgfpathlineto{\pgfqpoint{6.024808in}{1.141345in}}%
\pgfpathlineto{\pgfqpoint{6.024709in}{1.709772in}}%
\pgfpathlineto{\pgfqpoint{6.025336in}{1.611046in}}%
\pgfpathlineto{\pgfqpoint{6.025441in}{1.737189in}}%
\pgfpathlineto{\pgfqpoint{6.026333in}{1.091315in}}%
\pgfpathlineto{\pgfqpoint{6.026426in}{1.600148in}}%
\pgfpathlineto{\pgfqpoint{6.026762in}{0.939204in}}%
\pgfpathlineto{\pgfqpoint{6.026586in}{1.719009in}}%
\pgfpathlineto{\pgfqpoint{6.027536in}{1.555368in}}%
\pgfpathlineto{\pgfqpoint{6.027742in}{1.146723in}}%
\pgfpathlineto{\pgfqpoint{6.028566in}{1.734468in}}%
\pgfpathlineto{\pgfqpoint{6.028646in}{1.483424in}}%
\pgfpathlineto{\pgfqpoint{6.029452in}{1.718755in}}%
\pgfpathlineto{\pgfqpoint{6.029653in}{0.818107in}}%
\pgfpathlineto{\pgfqpoint{6.029758in}{1.627831in}}%
\pgfpathlineto{\pgfqpoint{6.030158in}{0.873922in}}%
\pgfpathlineto{\pgfqpoint{6.030598in}{1.747465in}}%
\pgfpathlineto{\pgfqpoint{6.030868in}{1.593504in}}%
\pgfpathlineto{\pgfqpoint{6.031397in}{1.718735in}}%
\pgfpathlineto{\pgfqpoint{6.031242in}{1.087667in}}%
\pgfpathlineto{\pgfqpoint{6.031945in}{1.467555in}}%
\pgfpathlineto{\pgfqpoint{6.032508in}{0.899791in}}%
\pgfpathlineto{\pgfqpoint{6.032635in}{1.727118in}}%
\pgfpathlineto{\pgfqpoint{6.033055in}{1.487603in}}%
\pgfpathlineto{\pgfqpoint{6.033690in}{1.068106in}}%
\pgfpathlineto{\pgfqpoint{6.033877in}{1.728334in}}%
\pgfpathlineto{\pgfqpoint{6.034148in}{1.584838in}}%
\pgfpathlineto{\pgfqpoint{6.034872in}{1.723120in}}%
\pgfpathlineto{\pgfqpoint{6.034881in}{1.098823in}}%
\pgfpathlineto{\pgfqpoint{6.035245in}{1.609387in}}%
\pgfpathlineto{\pgfqpoint{6.036086in}{0.830479in}}%
\pgfpathlineto{\pgfqpoint{6.035613in}{1.715988in}}%
\pgfpathlineto{\pgfqpoint{6.036354in}{1.530364in}}%
\pgfpathlineto{\pgfqpoint{6.036356in}{1.708796in}}%
\pgfpathlineto{\pgfqpoint{6.037208in}{1.098378in}}%
\pgfpathlineto{\pgfqpoint{6.037466in}{1.676321in}}%
\pgfpathlineto{\pgfqpoint{6.037926in}{1.151770in}}%
\pgfpathlineto{\pgfqpoint{6.038100in}{1.709117in}}%
\pgfpathlineto{\pgfqpoint{6.038585in}{1.526516in}}%
\pgfpathlineto{\pgfqpoint{6.039621in}{1.739210in}}%
\pgfpathlineto{\pgfqpoint{6.039077in}{1.071125in}}%
\pgfpathlineto{\pgfqpoint{6.039695in}{1.488441in}}%
\pgfpathlineto{\pgfqpoint{6.040789in}{1.733531in}}%
\pgfpathlineto{\pgfqpoint{6.039907in}{1.088371in}}%
\pgfpathlineto{\pgfqpoint{6.040806in}{1.643407in}}%
\pgfpathlineto{\pgfqpoint{6.041354in}{1.129247in}}%
\pgfpathlineto{\pgfqpoint{6.040951in}{1.698146in}}%
\pgfpathlineto{\pgfqpoint{6.041916in}{1.607702in}}%
\pgfpathlineto{\pgfqpoint{6.042549in}{0.936190in}}%
\pgfpathlineto{\pgfqpoint{6.042871in}{1.731716in}}%
\pgfpathlineto{\pgfqpoint{6.043028in}{1.513689in}}%
\pgfpathlineto{\pgfqpoint{6.043776in}{1.712281in}}%
\pgfpathlineto{\pgfqpoint{6.043599in}{1.020921in}}%
\pgfpathlineto{\pgfqpoint{6.044140in}{1.636106in}}%
\pgfpathlineto{\pgfqpoint{6.045080in}{1.092452in}}%
\pgfpathlineto{\pgfqpoint{6.044364in}{1.714688in}}%
\pgfpathlineto{\pgfqpoint{6.045251in}{1.629882in}}%
\pgfpathlineto{\pgfqpoint{6.046162in}{1.727880in}}%
\pgfpathlineto{\pgfqpoint{6.045356in}{0.949212in}}%
\pgfpathlineto{\pgfqpoint{6.046355in}{1.577834in}}%
\pgfpathlineto{\pgfqpoint{6.046751in}{1.167939in}}%
\pgfpathlineto{\pgfqpoint{6.047193in}{1.712487in}}%
\pgfpathlineto{\pgfqpoint{6.047466in}{1.567157in}}%
\pgfpathlineto{\pgfqpoint{6.048382in}{1.722132in}}%
\pgfpathlineto{\pgfqpoint{6.047548in}{1.023722in}}%
\pgfpathlineto{\pgfqpoint{6.048574in}{1.512640in}}%
\pgfpathlineto{\pgfqpoint{6.048719in}{1.185837in}}%
\pgfpathlineto{\pgfqpoint{6.049338in}{1.713589in}}%
\pgfpathlineto{\pgfqpoint{6.049681in}{1.542914in}}%
\pgfpathlineto{\pgfqpoint{6.049922in}{1.706625in}}%
\pgfpathlineto{\pgfqpoint{6.050025in}{1.154061in}}%
\pgfpathlineto{\pgfqpoint{6.050790in}{1.628743in}}%
\pgfpathlineto{\pgfqpoint{6.051591in}{0.980276in}}%
\pgfpathlineto{\pgfqpoint{6.050948in}{1.754402in}}%
\pgfpathlineto{\pgfqpoint{6.051902in}{1.439721in}}%
\pgfpathlineto{\pgfqpoint{6.052108in}{1.708364in}}%
\pgfpathlineto{\pgfqpoint{6.052709in}{1.178964in}}%
\pgfpathlineto{\pgfqpoint{6.053012in}{1.566226in}}%
\pgfpathlineto{\pgfqpoint{6.053298in}{1.051341in}}%
\pgfpathlineto{\pgfqpoint{6.053473in}{1.721315in}}%
\pgfpathlineto{\pgfqpoint{6.054123in}{1.529099in}}%
\pgfpathlineto{\pgfqpoint{6.054923in}{1.709952in}}%
\pgfpathlineto{\pgfqpoint{6.055024in}{0.987687in}}%
\pgfpathlineto{\pgfqpoint{6.055234in}{1.611324in}}%
\pgfpathlineto{\pgfqpoint{6.055722in}{1.100346in}}%
\pgfpathlineto{\pgfqpoint{6.055464in}{1.725243in}}%
\pgfpathlineto{\pgfqpoint{6.056347in}{1.528845in}}%
\pgfpathlineto{\pgfqpoint{6.056485in}{1.719936in}}%
\pgfpathlineto{\pgfqpoint{6.057150in}{1.079861in}}%
\pgfpathlineto{\pgfqpoint{6.057456in}{1.592169in}}%
\pgfpathlineto{\pgfqpoint{6.057590in}{1.685768in}}%
\pgfpathlineto{\pgfqpoint{6.058566in}{1.037474in}}%
\pgfpathlineto{\pgfqpoint{6.059262in}{1.727624in}}%
\pgfpathlineto{\pgfqpoint{6.059679in}{1.676055in}}%
\pgfpathlineto{\pgfqpoint{6.059837in}{1.047561in}}%
\pgfpathlineto{\pgfqpoint{6.060283in}{1.706621in}}%
\pgfpathlineto{\pgfqpoint{6.060791in}{1.592910in}}%
\pgfpathlineto{\pgfqpoint{6.061038in}{1.712325in}}%
\pgfpathlineto{\pgfqpoint{6.061813in}{1.123219in}}%
\pgfpathlineto{\pgfqpoint{6.061898in}{1.585308in}}%
\pgfpathlineto{\pgfqpoint{6.062370in}{1.066029in}}%
\pgfpathlineto{\pgfqpoint{6.062057in}{1.702525in}}%
\pgfpathlineto{\pgfqpoint{6.063010in}{1.449376in}}%
\pgfpathlineto{\pgfqpoint{6.063763in}{1.733302in}}%
\pgfpathlineto{\pgfqpoint{6.063627in}{0.753848in}}%
\pgfpathlineto{\pgfqpoint{6.064124in}{1.613707in}}%
\pgfpathlineto{\pgfqpoint{6.064393in}{1.084145in}}%
\pgfpathlineto{\pgfqpoint{6.064800in}{1.703644in}}%
\pgfpathlineto{\pgfqpoint{6.065237in}{1.522344in}}%
\pgfpathlineto{\pgfqpoint{6.066198in}{1.702703in}}%
\pgfpathlineto{\pgfqpoint{6.065703in}{1.106765in}}%
\pgfpathlineto{\pgfqpoint{6.066345in}{1.587506in}}%
\pgfpathlineto{\pgfqpoint{6.067018in}{0.961145in}}%
\pgfpathlineto{\pgfqpoint{6.066854in}{1.721360in}}%
\pgfpathlineto{\pgfqpoint{6.067456in}{1.327467in}}%
\pgfpathlineto{\pgfqpoint{6.068468in}{1.704348in}}%
\pgfpathlineto{\pgfqpoint{6.067852in}{1.195621in}}%
\pgfpathlineto{\pgfqpoint{6.068568in}{1.476402in}}%
\pgfpathlineto{\pgfqpoint{6.068582in}{1.142982in}}%
\pgfpathlineto{\pgfqpoint{6.068706in}{1.691169in}}%
\pgfpathlineto{\pgfqpoint{6.069677in}{1.529525in}}%
\pgfpathlineto{\pgfqpoint{6.069720in}{1.701319in}}%
\pgfpathlineto{\pgfqpoint{6.070165in}{0.964250in}}%
\pgfpathlineto{\pgfqpoint{6.070788in}{1.652434in}}%
\pgfpathlineto{\pgfqpoint{6.071752in}{1.166834in}}%
\pgfpathlineto{\pgfqpoint{6.071552in}{1.706430in}}%
\pgfpathlineto{\pgfqpoint{6.071898in}{1.442293in}}%
\pgfpathlineto{\pgfqpoint{6.071984in}{1.721085in}}%
\pgfpathlineto{\pgfqpoint{6.072139in}{1.077427in}}%
\pgfpathlineto{\pgfqpoint{6.073009in}{1.588518in}}%
\pgfpathlineto{\pgfqpoint{6.073210in}{1.081899in}}%
\pgfpathlineto{\pgfqpoint{6.073438in}{1.690400in}}%
\pgfpathlineto{\pgfqpoint{6.074117in}{1.514804in}}%
\pgfpathlineto{\pgfqpoint{6.074865in}{1.703544in}}%
\pgfpathlineto{\pgfqpoint{6.074255in}{1.123553in}}%
\pgfpathlineto{\pgfqpoint{6.075227in}{1.582250in}}%
\pgfpathlineto{\pgfqpoint{6.076078in}{1.146459in}}%
\pgfpathlineto{\pgfqpoint{6.076119in}{1.690582in}}%
\pgfpathlineto{\pgfqpoint{6.076333in}{1.318784in}}%
\pgfpathlineto{\pgfqpoint{6.077352in}{1.699359in}}%
\pgfpathlineto{\pgfqpoint{6.076869in}{1.028202in}}%
\pgfpathlineto{\pgfqpoint{6.077445in}{1.635708in}}%
\pgfpathlineto{\pgfqpoint{6.077759in}{1.085192in}}%
\pgfpathlineto{\pgfqpoint{6.078468in}{1.687111in}}%
\pgfpathlineto{\pgfqpoint{6.078555in}{1.424742in}}%
\pgfpathlineto{\pgfqpoint{6.079433in}{1.705485in}}%
\pgfpathlineto{\pgfqpoint{6.078743in}{1.093264in}}%
\pgfpathlineto{\pgfqpoint{6.079665in}{1.386815in}}%
\pgfpathlineto{\pgfqpoint{6.079767in}{0.735999in}}%
\pgfpathlineto{\pgfqpoint{6.080154in}{1.711955in}}%
\pgfpathlineto{\pgfqpoint{6.080773in}{1.526870in}}%
\pgfpathlineto{\pgfqpoint{6.081009in}{1.707798in}}%
\pgfpathlineto{\pgfqpoint{6.081351in}{1.045223in}}%
\pgfpathlineto{\pgfqpoint{6.081886in}{1.638441in}}%
\pgfpathlineto{\pgfqpoint{6.081900in}{1.114175in}}%
\pgfpathlineto{\pgfqpoint{6.082894in}{1.697560in}}%
\pgfpathlineto{\pgfqpoint{6.082999in}{1.215320in}}%
\pgfpathlineto{\pgfqpoint{6.083162in}{1.712473in}}%
\pgfpathlineto{\pgfqpoint{6.083311in}{0.887144in}}%
\pgfpathlineto{\pgfqpoint{6.084111in}{1.493827in}}%
\pgfpathlineto{\pgfqpoint{6.085166in}{1.718670in}}%
\pgfpathlineto{\pgfqpoint{6.085105in}{1.109860in}}%
\pgfpathlineto{\pgfqpoint{6.085218in}{1.522009in}}%
\pgfpathlineto{\pgfqpoint{6.085802in}{1.065238in}}%
\pgfpathlineto{\pgfqpoint{6.085753in}{1.701880in}}%
\pgfpathlineto{\pgfqpoint{6.086328in}{1.658856in}}%
\pgfpathlineto{\pgfqpoint{6.086533in}{0.985192in}}%
\pgfpathlineto{\pgfqpoint{6.087073in}{1.707094in}}%
\pgfpathlineto{\pgfqpoint{6.087440in}{1.553955in}}%
\pgfpathlineto{\pgfqpoint{6.087889in}{1.690968in}}%
\pgfpathlineto{\pgfqpoint{6.088041in}{1.053362in}}%
\pgfpathlineto{\pgfqpoint{6.088550in}{1.550167in}}%
\pgfpathlineto{\pgfqpoint{6.089078in}{1.698072in}}%
\pgfpathlineto{\pgfqpoint{6.089441in}{1.002831in}}%
\pgfpathlineto{\pgfqpoint{6.089658in}{1.575176in}}%
\pgfpathlineto{\pgfqpoint{6.090493in}{1.096730in}}%
\pgfpathlineto{\pgfqpoint{6.089763in}{1.689250in}}%
\pgfpathlineto{\pgfqpoint{6.090768in}{1.501546in}}%
\pgfpathlineto{\pgfqpoint{6.091021in}{1.691933in}}%
\pgfpathlineto{\pgfqpoint{6.091310in}{1.050799in}}%
\pgfpathlineto{\pgfqpoint{6.091874in}{1.559416in}}%
\pgfpathlineto{\pgfqpoint{6.092605in}{0.977164in}}%
\pgfpathlineto{\pgfqpoint{6.092342in}{1.692620in}}%
\pgfpathlineto{\pgfqpoint{6.092983in}{1.387623in}}%
\pgfpathlineto{\pgfqpoint{6.093911in}{1.684186in}}%
\pgfpathlineto{\pgfqpoint{6.093355in}{0.966407in}}%
\pgfpathlineto{\pgfqpoint{6.094093in}{1.581832in}}%
\pgfpathlineto{\pgfqpoint{6.094228in}{0.950037in}}%
\pgfpathlineto{\pgfqpoint{6.094663in}{1.705725in}}%
\pgfpathlineto{\pgfqpoint{6.095204in}{1.504198in}}%
\pgfpathlineto{\pgfqpoint{6.095565in}{1.702356in}}%
\pgfpathlineto{\pgfqpoint{6.095516in}{1.033470in}}%
\pgfpathlineto{\pgfqpoint{6.096315in}{1.597919in}}%
\pgfpathlineto{\pgfqpoint{6.096996in}{1.105028in}}%
\pgfpathlineto{\pgfqpoint{6.096571in}{1.686475in}}%
\pgfpathlineto{\pgfqpoint{6.097428in}{1.475230in}}%
\pgfpathlineto{\pgfqpoint{6.097577in}{1.713572in}}%
\pgfpathlineto{\pgfqpoint{6.098486in}{0.929457in}}%
\pgfpathlineto{\pgfqpoint{6.098537in}{1.524679in}}%
\pgfpathlineto{\pgfqpoint{6.098922in}{0.904401in}}%
\pgfpathlineto{\pgfqpoint{6.099109in}{1.691156in}}%
\pgfpathlineto{\pgfqpoint{6.099648in}{1.205996in}}%
\pgfpathlineto{\pgfqpoint{6.100574in}{1.698205in}}%
\pgfpathlineto{\pgfqpoint{6.099998in}{1.101759in}}%
\pgfpathlineto{\pgfqpoint{6.100759in}{1.590749in}}%
\pgfpathlineto{\pgfqpoint{6.101426in}{0.817738in}}%
\pgfpathlineto{\pgfqpoint{6.101234in}{1.690487in}}%
\pgfpathlineto{\pgfqpoint{6.101872in}{1.486826in}}%
\pgfpathlineto{\pgfqpoint{6.102052in}{1.684588in}}%
\pgfpathlineto{\pgfqpoint{6.102884in}{1.017805in}}%
\pgfpathlineto{\pgfqpoint{6.102983in}{1.627909in}}%
\pgfpathlineto{\pgfqpoint{6.103437in}{0.951048in}}%
\pgfpathlineto{\pgfqpoint{6.103016in}{1.709979in}}%
\pgfpathlineto{\pgfqpoint{6.104094in}{1.544261in}}%
\pgfpathlineto{\pgfqpoint{6.104870in}{1.690586in}}%
\pgfpathlineto{\pgfqpoint{6.104274in}{0.961284in}}%
\pgfpathlineto{\pgfqpoint{6.105205in}{1.583076in}}%
\pgfpathlineto{\pgfqpoint{6.105737in}{1.037238in}}%
\pgfpathlineto{\pgfqpoint{6.106126in}{1.693952in}}%
\pgfpathlineto{\pgfqpoint{6.106315in}{1.580836in}}%
\pgfpathlineto{\pgfqpoint{6.106935in}{1.072972in}}%
\pgfpathlineto{\pgfqpoint{6.106925in}{1.693044in}}%
\pgfpathlineto{\pgfqpoint{6.107425in}{1.512312in}}%
\pgfpathlineto{\pgfqpoint{6.108249in}{1.683800in}}%
\pgfpathlineto{\pgfqpoint{6.108455in}{1.119675in}}%
\pgfpathlineto{\pgfqpoint{6.108533in}{1.655330in}}%
\pgfpathlineto{\pgfqpoint{6.109462in}{0.992211in}}%
\pgfpathlineto{\pgfqpoint{6.109485in}{1.689699in}}%
\pgfpathlineto{\pgfqpoint{6.109644in}{1.455445in}}%
\pgfpathlineto{\pgfqpoint{6.110618in}{1.698807in}}%
\pgfpathlineto{\pgfqpoint{6.110118in}{0.995812in}}%
\pgfpathlineto{\pgfqpoint{6.110754in}{1.587725in}}%
\pgfpathlineto{\pgfqpoint{6.111360in}{1.004626in}}%
\pgfpathlineto{\pgfqpoint{6.110882in}{1.688437in}}%
\pgfpathlineto{\pgfqpoint{6.111866in}{1.493563in}}%
\pgfpathlineto{\pgfqpoint{6.112029in}{1.703811in}}%
\pgfpathlineto{\pgfqpoint{6.112156in}{1.139418in}}%
\pgfpathlineto{\pgfqpoint{6.112977in}{1.658252in}}%
\pgfpathlineto{\pgfqpoint{6.113297in}{0.819098in}}%
\pgfpathlineto{\pgfqpoint{6.112992in}{1.687146in}}%
\pgfpathlineto{\pgfqpoint{6.114088in}{1.341996in}}%
\pgfpathlineto{\pgfqpoint{6.114222in}{1.691973in}}%
\pgfpathlineto{\pgfqpoint{6.114271in}{1.190832in}}%
\pgfpathlineto{\pgfqpoint{6.115200in}{1.527826in}}%
\pgfpathlineto{\pgfqpoint{6.115242in}{1.087713in}}%
\pgfpathlineto{\pgfqpoint{6.116211in}{1.677062in}}%
\pgfpathlineto{\pgfqpoint{6.116312in}{1.259574in}}%
\pgfpathlineto{\pgfqpoint{6.116471in}{1.690369in}}%
\pgfpathlineto{\pgfqpoint{6.116969in}{1.061691in}}%
\pgfpathlineto{\pgfqpoint{6.117424in}{1.590227in}}%
\pgfpathlineto{\pgfqpoint{6.117836in}{1.089350in}}%
\pgfpathlineto{\pgfqpoint{6.117476in}{1.681898in}}%
\pgfpathlineto{\pgfqpoint{6.118535in}{1.615676in}}%
\pgfpathlineto{\pgfqpoint{6.119350in}{0.835270in}}%
\pgfpathlineto{\pgfqpoint{6.119638in}{1.682995in}}%
\pgfpathlineto{\pgfqpoint{6.119644in}{1.207721in}}%
\pgfpathlineto{\pgfqpoint{6.120655in}{1.679984in}}%
\pgfpathlineto{\pgfqpoint{6.119659in}{0.951781in}}%
\pgfpathlineto{\pgfqpoint{6.120756in}{1.516854in}}%
\pgfpathlineto{\pgfqpoint{6.121533in}{1.050345in}}%
\pgfpathlineto{\pgfqpoint{6.120831in}{1.697618in}}%
\pgfpathlineto{\pgfqpoint{6.121868in}{1.445277in}}%
\pgfpathlineto{\pgfqpoint{6.122557in}{1.682339in}}%
\pgfpathlineto{\pgfqpoint{6.122346in}{1.080099in}}%
\pgfpathlineto{\pgfqpoint{6.122980in}{1.591844in}}%
\pgfpathlineto{\pgfqpoint{6.123715in}{1.157775in}}%
\pgfpathlineto{\pgfqpoint{6.123229in}{1.697623in}}%
\pgfpathlineto{\pgfqpoint{6.124091in}{1.561149in}}%
\pgfpathlineto{\pgfqpoint{6.124636in}{1.690289in}}%
\pgfpathlineto{\pgfqpoint{6.125057in}{1.121615in}}%
\pgfpathlineto{\pgfqpoint{6.125183in}{1.393041in}}%
\pgfpathlineto{\pgfqpoint{6.126151in}{0.975718in}}%
\pgfpathlineto{\pgfqpoint{6.125841in}{1.669171in}}%
\pgfpathlineto{\pgfqpoint{6.126289in}{1.274561in}}%
\pgfpathlineto{\pgfqpoint{6.126802in}{1.686517in}}%
\pgfpathlineto{\pgfqpoint{6.126419in}{1.084126in}}%
\pgfpathlineto{\pgfqpoint{6.127399in}{1.505442in}}%
\pgfpathlineto{\pgfqpoint{6.128262in}{1.064623in}}%
\pgfpathlineto{\pgfqpoint{6.127440in}{1.688822in}}%
\pgfpathlineto{\pgfqpoint{6.128509in}{1.535767in}}%
\pgfpathlineto{\pgfqpoint{6.128946in}{1.073315in}}%
\pgfpathlineto{\pgfqpoint{6.129002in}{1.698706in}}%
\pgfpathlineto{\pgfqpoint{6.129614in}{1.522385in}}%
\pgfpathlineto{\pgfqpoint{6.130571in}{1.685611in}}%
\pgfpathlineto{\pgfqpoint{6.130624in}{0.955118in}}%
\pgfpathlineto{\pgfqpoint{6.130725in}{1.579250in}}%
\pgfpathlineto{\pgfqpoint{6.131351in}{1.686659in}}%
\pgfpathlineto{\pgfqpoint{6.131449in}{0.932346in}}%
\pgfpathlineto{\pgfqpoint{6.131831in}{1.482114in}}%
\pgfpathlineto{\pgfqpoint{6.132905in}{1.686960in}}%
\pgfpathlineto{\pgfqpoint{6.132374in}{1.033247in}}%
\pgfpathlineto{\pgfqpoint{6.132930in}{1.577620in}}%
\pgfpathlineto{\pgfqpoint{6.133001in}{0.863544in}}%
\pgfpathlineto{\pgfqpoint{6.133453in}{1.681707in}}%
\pgfpathlineto{\pgfqpoint{6.134041in}{1.505358in}}%
\pgfpathlineto{\pgfqpoint{6.134050in}{1.667608in}}%
\pgfpathlineto{\pgfqpoint{6.134279in}{0.875024in}}%
\pgfpathlineto{\pgfqpoint{6.134875in}{1.549535in}}%
\pgfpathlineto{\pgfqpoint{6.135861in}{1.081224in}}%
\pgfpathlineto{\pgfqpoint{6.135742in}{1.676250in}}%
\pgfpathlineto{\pgfqpoint{6.135986in}{1.498466in}}%
\pgfpathlineto{\pgfqpoint{6.136308in}{1.697256in}}%
\pgfpathlineto{\pgfqpoint{6.136883in}{1.119491in}}%
\pgfpathlineto{\pgfqpoint{6.137088in}{1.520368in}}%
\pgfpathlineto{\pgfqpoint{6.137685in}{1.675107in}}%
\pgfpathlineto{\pgfqpoint{6.138201in}{0.976801in}}%
\pgfpathlineto{\pgfqpoint{6.138347in}{1.703231in}}%
\pgfpathlineto{\pgfqpoint{6.139313in}{1.488404in}}%
\pgfpathlineto{\pgfqpoint{6.140015in}{0.987926in}}%
\pgfpathlineto{\pgfqpoint{6.139494in}{1.697465in}}%
\pgfpathlineto{\pgfqpoint{6.140421in}{1.493394in}}%
\pgfpathlineto{\pgfqpoint{6.140464in}{1.677974in}}%
\pgfpathlineto{\pgfqpoint{6.140651in}{1.065115in}}%
\pgfpathlineto{\pgfqpoint{6.141532in}{1.515804in}}%
\pgfpathlineto{\pgfqpoint{6.142017in}{1.084004in}}%
\pgfpathlineto{\pgfqpoint{6.141584in}{1.668414in}}%
\pgfpathlineto{\pgfqpoint{6.142621in}{1.439829in}}%
\pgfpathlineto{\pgfqpoint{6.142980in}{1.685599in}}%
\pgfpathlineto{\pgfqpoint{6.143140in}{1.029399in}}%
\pgfpathlineto{\pgfqpoint{6.143732in}{1.581107in}}%
\pgfpathlineto{\pgfqpoint{6.144471in}{1.097935in}}%
\pgfpathlineto{\pgfqpoint{6.144462in}{1.693404in}}%
\pgfpathlineto{\pgfqpoint{6.144843in}{1.435897in}}%
\pgfpathlineto{\pgfqpoint{6.145365in}{1.672911in}}%
\pgfpathlineto{\pgfqpoint{6.145665in}{1.015341in}}%
\pgfpathlineto{\pgfqpoint{6.145954in}{1.480834in}}%
\pgfpathlineto{\pgfqpoint{6.146615in}{1.146996in}}%
\pgfpathlineto{\pgfqpoint{6.146488in}{1.695422in}}%
\pgfpathlineto{\pgfqpoint{6.147057in}{1.446205in}}%
\pgfpathlineto{\pgfqpoint{6.147886in}{1.694913in}}%
\pgfpathlineto{\pgfqpoint{6.148155in}{0.894316in}}%
\pgfpathlineto{\pgfqpoint{6.148167in}{1.552816in}}%
\pgfpathlineto{\pgfqpoint{6.148423in}{0.918803in}}%
\pgfpathlineto{\pgfqpoint{6.148383in}{1.679535in}}%
\pgfpathlineto{\pgfqpoint{6.149277in}{1.521931in}}%
\pgfpathlineto{\pgfqpoint{6.149996in}{1.691527in}}%
\pgfpathlineto{\pgfqpoint{6.150017in}{1.019132in}}%
\pgfpathlineto{\pgfqpoint{6.150387in}{1.464164in}}%
\pgfpathlineto{\pgfqpoint{6.151449in}{1.705485in}}%
\pgfpathlineto{\pgfqpoint{6.151479in}{1.125192in}}%
\pgfpathlineto{\pgfqpoint{6.151497in}{1.453877in}}%
\pgfpathlineto{\pgfqpoint{6.151576in}{1.041636in}}%
\pgfpathlineto{\pgfqpoint{6.152296in}{1.703743in}}%
\pgfpathlineto{\pgfqpoint{6.152602in}{1.458816in}}%
\pgfpathlineto{\pgfqpoint{6.153294in}{1.676363in}}%
\pgfpathlineto{\pgfqpoint{6.152916in}{1.116200in}}%
\pgfpathlineto{\pgfqpoint{6.153713in}{1.505969in}}%
\pgfpathlineto{\pgfqpoint{6.154738in}{0.948592in}}%
\pgfpathlineto{\pgfqpoint{6.154021in}{1.684813in}}%
\pgfpathlineto{\pgfqpoint{6.154823in}{1.584585in}}%
\pgfpathlineto{\pgfqpoint{6.155641in}{1.058655in}}%
\pgfpathlineto{\pgfqpoint{6.155669in}{1.695549in}}%
\pgfpathlineto{\pgfqpoint{6.155936in}{1.407700in}}%
\pgfpathlineto{\pgfqpoint{6.156457in}{1.672106in}}%
\pgfpathlineto{\pgfqpoint{6.156113in}{1.016076in}}%
\pgfpathlineto{\pgfqpoint{6.157046in}{1.583787in}}%
\pgfpathlineto{\pgfqpoint{6.157764in}{0.910909in}}%
\pgfpathlineto{\pgfqpoint{6.157923in}{1.681365in}}%
\pgfpathlineto{\pgfqpoint{6.158158in}{1.394207in}}%
\pgfpathlineto{\pgfqpoint{6.158255in}{1.679634in}}%
\pgfpathlineto{\pgfqpoint{6.158732in}{1.108966in}}%
\pgfpathlineto{\pgfqpoint{6.159270in}{1.485486in}}%
\pgfpathlineto{\pgfqpoint{6.159377in}{1.105778in}}%
\pgfpathlineto{\pgfqpoint{6.159746in}{1.671412in}}%
\pgfpathlineto{\pgfqpoint{6.160381in}{1.552756in}}%
\pgfpathlineto{\pgfqpoint{6.161325in}{1.031583in}}%
\pgfpathlineto{\pgfqpoint{6.160398in}{1.678709in}}%
\pgfpathlineto{\pgfqpoint{6.161469in}{1.327464in}}%
\pgfpathlineto{\pgfqpoint{6.162338in}{1.662466in}}%
\pgfpathlineto{\pgfqpoint{6.161765in}{1.064246in}}%
\pgfpathlineto{\pgfqpoint{6.162581in}{1.517695in}}%
\pgfpathlineto{\pgfqpoint{6.163066in}{1.657951in}}%
\pgfpathlineto{\pgfqpoint{6.163342in}{1.099755in}}%
\pgfpathlineto{\pgfqpoint{6.163692in}{1.523915in}}%
\pgfpathlineto{\pgfqpoint{6.164315in}{1.001271in}}%
\pgfpathlineto{\pgfqpoint{6.164698in}{1.682207in}}%
\pgfpathlineto{\pgfqpoint{6.164802in}{1.494274in}}%
\pgfpathlineto{\pgfqpoint{6.165488in}{1.677437in}}%
\pgfpathlineto{\pgfqpoint{6.164928in}{1.048921in}}%
\pgfpathlineto{\pgfqpoint{6.165905in}{1.507173in}}%
\pgfpathlineto{\pgfqpoint{6.166601in}{0.846303in}}%
\pgfpathlineto{\pgfqpoint{6.166699in}{1.661727in}}%
\pgfpathlineto{\pgfqpoint{6.167016in}{1.553942in}}%
\pgfpathlineto{\pgfqpoint{6.167472in}{1.685981in}}%
\pgfpathlineto{\pgfqpoint{6.167881in}{0.923720in}}%
\pgfpathlineto{\pgfqpoint{6.168060in}{1.448026in}}%
\pgfpathlineto{\pgfqpoint{6.168200in}{0.910153in}}%
\pgfpathlineto{\pgfqpoint{6.168733in}{1.667093in}}%
\pgfpathlineto{\pgfqpoint{6.169171in}{1.517425in}}%
\pgfpathlineto{\pgfqpoint{6.169979in}{1.090522in}}%
\pgfpathlineto{\pgfqpoint{6.170197in}{1.648896in}}%
\pgfpathlineto{\pgfqpoint{6.170283in}{1.402937in}}%
\pgfpathlineto{\pgfqpoint{6.171114in}{1.053416in}}%
\pgfpathlineto{\pgfqpoint{6.171064in}{1.660967in}}%
\pgfpathlineto{\pgfqpoint{6.171365in}{1.349221in}}%
\pgfpathlineto{\pgfqpoint{6.172116in}{1.680057in}}%
\pgfpathlineto{\pgfqpoint{6.172353in}{1.087817in}}%
\pgfpathlineto{\pgfqpoint{6.172477in}{1.602649in}}%
\pgfpathlineto{\pgfqpoint{6.173178in}{0.999922in}}%
\pgfpathlineto{\pgfqpoint{6.173308in}{1.657399in}}%
\pgfpathlineto{\pgfqpoint{6.173590in}{1.502927in}}%
\pgfpathlineto{\pgfqpoint{6.174322in}{1.666464in}}%
\pgfpathlineto{\pgfqpoint{6.173951in}{0.951048in}}%
\pgfpathlineto{\pgfqpoint{6.174701in}{1.604103in}}%
\pgfpathlineto{\pgfqpoint{6.175408in}{1.034913in}}%
\pgfpathlineto{\pgfqpoint{6.174997in}{1.667150in}}%
\pgfpathlineto{\pgfqpoint{6.175811in}{1.416280in}}%
\pgfpathlineto{\pgfqpoint{6.176043in}{1.686619in}}%
\pgfpathlineto{\pgfqpoint{6.175981in}{0.962573in}}%
\pgfpathlineto{\pgfqpoint{6.176921in}{1.483960in}}%
\pgfpathlineto{\pgfqpoint{6.177427in}{1.028011in}}%
\pgfpathlineto{\pgfqpoint{6.176970in}{1.675472in}}%
\pgfpathlineto{\pgfqpoint{6.178030in}{1.541664in}}%
\pgfpathlineto{\pgfqpoint{6.178722in}{1.681171in}}%
\pgfpathlineto{\pgfqpoint{6.178861in}{1.036527in}}%
\pgfpathlineto{\pgfqpoint{6.179134in}{1.523316in}}%
\pgfpathlineto{\pgfqpoint{6.179189in}{1.022029in}}%
\pgfpathlineto{\pgfqpoint{6.179530in}{1.658234in}}%
\pgfpathlineto{\pgfqpoint{6.180245in}{1.398683in}}%
\pgfpathlineto{\pgfqpoint{6.180544in}{1.660349in}}%
\pgfpathlineto{\pgfqpoint{6.181064in}{0.987837in}}%
\pgfpathlineto{\pgfqpoint{6.181357in}{1.585524in}}%
\pgfpathlineto{\pgfqpoint{6.182049in}{1.066426in}}%
\pgfpathlineto{\pgfqpoint{6.182427in}{1.672561in}}%
\pgfpathlineto{\pgfqpoint{6.182466in}{1.543302in}}%
\pgfpathlineto{\pgfqpoint{6.183302in}{1.662029in}}%
\pgfpathlineto{\pgfqpoint{6.183000in}{1.033720in}}%
\pgfpathlineto{\pgfqpoint{6.183576in}{1.647271in}}%
\pgfpathlineto{\pgfqpoint{6.183692in}{0.993085in}}%
\pgfpathlineto{\pgfqpoint{6.184324in}{1.668329in}}%
\pgfpathlineto{\pgfqpoint{6.184687in}{1.364810in}}%
\pgfpathlineto{\pgfqpoint{6.185726in}{1.656739in}}%
\pgfpathlineto{\pgfqpoint{6.184918in}{1.064727in}}%
\pgfpathlineto{\pgfqpoint{6.185799in}{1.509728in}}%
\pgfpathlineto{\pgfqpoint{6.186775in}{0.993826in}}%
\pgfpathlineto{\pgfqpoint{6.186618in}{1.691731in}}%
\pgfpathlineto{\pgfqpoint{6.186909in}{1.539462in}}%
\pgfpathlineto{\pgfqpoint{6.187625in}{1.686160in}}%
\pgfpathlineto{\pgfqpoint{6.187008in}{1.028633in}}%
\pgfpathlineto{\pgfqpoint{6.188017in}{1.460355in}}%
\pgfpathlineto{\pgfqpoint{6.189048in}{0.930746in}}%
\pgfpathlineto{\pgfqpoint{6.188553in}{1.657344in}}%
\pgfpathlineto{\pgfqpoint{6.189123in}{1.515181in}}%
\pgfpathlineto{\pgfqpoint{6.189251in}{1.658773in}}%
\pgfpathlineto{\pgfqpoint{6.190215in}{1.089556in}}%
\pgfpathlineto{\pgfqpoint{6.190233in}{1.561245in}}%
\pgfpathlineto{\pgfqpoint{6.190249in}{0.890916in}}%
\pgfpathlineto{\pgfqpoint{6.191120in}{1.650567in}}%
\pgfpathlineto{\pgfqpoint{6.191344in}{1.449029in}}%
\pgfpathlineto{\pgfqpoint{6.192433in}{1.685552in}}%
\pgfpathlineto{\pgfqpoint{6.192244in}{1.131361in}}%
\pgfpathlineto{\pgfqpoint{6.192455in}{1.585466in}}%
\pgfpathlineto{\pgfqpoint{6.193537in}{0.964323in}}%
\pgfpathlineto{\pgfqpoint{6.192875in}{1.666040in}}%
\pgfpathlineto{\pgfqpoint{6.193566in}{1.510060in}}%
\pgfpathlineto{\pgfqpoint{6.194125in}{1.661111in}}%
\pgfpathlineto{\pgfqpoint{6.194552in}{0.912847in}}%
\pgfpathlineto{\pgfqpoint{6.194676in}{1.547666in}}%
\pgfpathlineto{\pgfqpoint{6.195573in}{1.041320in}}%
\pgfpathlineto{\pgfqpoint{6.195014in}{1.650428in}}%
\pgfpathlineto{\pgfqpoint{6.195789in}{1.456607in}}%
\pgfpathlineto{\pgfqpoint{6.196319in}{1.663623in}}%
\pgfpathlineto{\pgfqpoint{6.195850in}{0.900607in}}%
\pgfpathlineto{\pgfqpoint{6.196900in}{1.495733in}}%
\pgfpathlineto{\pgfqpoint{6.197189in}{1.004837in}}%
\pgfpathlineto{\pgfqpoint{6.197417in}{1.671708in}}%
\pgfpathlineto{\pgfqpoint{6.198011in}{1.468845in}}%
\pgfpathlineto{\pgfqpoint{6.199043in}{1.042623in}}%
\pgfpathlineto{\pgfqpoint{6.198115in}{1.667037in}}%
\pgfpathlineto{\pgfqpoint{6.199111in}{1.321966in}}%
\pgfpathlineto{\pgfqpoint{6.199706in}{1.652851in}}%
\pgfpathlineto{\pgfqpoint{6.199658in}{1.038180in}}%
\pgfpathlineto{\pgfqpoint{6.200223in}{1.501099in}}%
\pgfpathlineto{\pgfqpoint{6.201183in}{1.004799in}}%
\pgfpathlineto{\pgfqpoint{6.201141in}{1.663538in}}%
\pgfpathlineto{\pgfqpoint{6.201333in}{1.520144in}}%
\pgfpathlineto{\pgfqpoint{6.201923in}{1.666987in}}%
\pgfpathlineto{\pgfqpoint{6.201502in}{0.925497in}}%
\pgfpathlineto{\pgfqpoint{6.202436in}{1.437802in}}%
\pgfpathlineto{\pgfqpoint{6.203404in}{1.086444in}}%
\pgfpathlineto{\pgfqpoint{6.203200in}{1.660419in}}%
\pgfpathlineto{\pgfqpoint{6.203544in}{1.431210in}}%
\pgfpathlineto{\pgfqpoint{6.203804in}{1.666784in}}%
\pgfpathlineto{\pgfqpoint{6.204095in}{0.855033in}}%
\pgfpathlineto{\pgfqpoint{6.204654in}{1.498734in}}%
\pgfpathlineto{\pgfqpoint{6.205341in}{1.043943in}}%
\pgfpathlineto{\pgfqpoint{6.204990in}{1.664672in}}%
\pgfpathlineto{\pgfqpoint{6.205764in}{1.466779in}}%
\pgfpathlineto{\pgfqpoint{6.206766in}{1.677862in}}%
\pgfpathlineto{\pgfqpoint{6.206209in}{0.999763in}}%
\pgfpathlineto{\pgfqpoint{6.206875in}{1.535759in}}%
\pgfpathlineto{\pgfqpoint{6.206908in}{0.939806in}}%
\pgfpathlineto{\pgfqpoint{6.207039in}{1.669362in}}%
\pgfpathlineto{\pgfqpoint{6.207987in}{1.358464in}}%
\pgfpathlineto{\pgfqpoint{6.208582in}{1.653247in}}%
\pgfpathlineto{\pgfqpoint{6.208250in}{1.062195in}}%
\pgfpathlineto{\pgfqpoint{6.209097in}{1.518560in}}%
\pgfpathlineto{\pgfqpoint{6.209646in}{0.912364in}}%
\pgfpathlineto{\pgfqpoint{6.210032in}{1.674791in}}%
\pgfpathlineto{\pgfqpoint{6.210209in}{1.349784in}}%
\pgfpathlineto{\pgfqpoint{6.211042in}{1.676061in}}%
\pgfpathlineto{\pgfqpoint{6.210645in}{1.063733in}}%
\pgfpathlineto{\pgfqpoint{6.211320in}{1.461034in}}%
\pgfpathlineto{\pgfqpoint{6.211757in}{0.831204in}}%
\pgfpathlineto{\pgfqpoint{6.212089in}{1.652088in}}%
\pgfpathlineto{\pgfqpoint{6.212429in}{1.424128in}}%
\pgfpathlineto{\pgfqpoint{6.213171in}{1.672604in}}%
\pgfpathlineto{\pgfqpoint{6.213141in}{0.887517in}}%
\pgfpathlineto{\pgfqpoint{6.213537in}{1.516474in}}%
\pgfpathlineto{\pgfqpoint{6.213852in}{0.944426in}}%
\pgfpathlineto{\pgfqpoint{6.214064in}{1.653087in}}%
\pgfpathlineto{\pgfqpoint{6.214650in}{1.189612in}}%
\pgfpathlineto{\pgfqpoint{6.215008in}{1.667625in}}%
\pgfpathlineto{\pgfqpoint{6.215248in}{0.997012in}}%
\pgfpathlineto{\pgfqpoint{6.215763in}{1.610661in}}%
\pgfpathlineto{\pgfqpoint{6.215992in}{0.993702in}}%
\pgfpathlineto{\pgfqpoint{6.216802in}{1.670441in}}%
\pgfpathlineto{\pgfqpoint{6.216873in}{1.566437in}}%
\pgfpathlineto{\pgfqpoint{6.217856in}{1.082721in}}%
\pgfpathlineto{\pgfqpoint{6.217643in}{1.654341in}}%
\pgfpathlineto{\pgfqpoint{6.217986in}{1.434158in}}%
\pgfpathlineto{\pgfqpoint{6.218449in}{1.683780in}}%
\pgfpathlineto{\pgfqpoint{6.218196in}{1.082846in}}%
\pgfpathlineto{\pgfqpoint{6.219093in}{1.525204in}}%
\pgfpathlineto{\pgfqpoint{6.219570in}{1.094613in}}%
\pgfpathlineto{\pgfqpoint{6.219402in}{1.659442in}}%
\pgfpathlineto{\pgfqpoint{6.220203in}{1.502754in}}%
\pgfpathlineto{\pgfqpoint{6.220561in}{1.658599in}}%
\pgfpathlineto{\pgfqpoint{6.220412in}{1.046970in}}%
\pgfpathlineto{\pgfqpoint{6.221300in}{1.529566in}}%
\pgfpathlineto{\pgfqpoint{6.221344in}{0.929989in}}%
\pgfpathlineto{\pgfqpoint{6.221706in}{1.657288in}}%
\pgfpathlineto{\pgfqpoint{6.222410in}{1.428250in}}%
\pgfpathlineto{\pgfqpoint{6.223500in}{0.851395in}}%
\pgfpathlineto{\pgfqpoint{6.223520in}{1.674762in}}%
\pgfpathlineto{\pgfqpoint{6.223988in}{1.019268in}}%
\pgfpathlineto{\pgfqpoint{6.224634in}{1.195737in}}%
\pgfpathlineto{\pgfqpoint{6.225277in}{1.642621in}}%
\pgfpathlineto{\pgfqpoint{6.225475in}{0.986356in}}%
\pgfpathlineto{\pgfqpoint{6.225747in}{1.567916in}}%
\pgfpathlineto{\pgfqpoint{6.226303in}{0.919078in}}%
\pgfpathlineto{\pgfqpoint{6.226124in}{1.656237in}}%
\pgfpathlineto{\pgfqpoint{6.226860in}{1.462678in}}%
\pgfpathlineto{\pgfqpoint{6.227244in}{1.664275in}}%
\pgfpathlineto{\pgfqpoint{6.227157in}{0.764031in}}%
\pgfpathlineto{\pgfqpoint{6.227971in}{1.554802in}}%
\pgfpathlineto{\pgfqpoint{6.228447in}{1.036754in}}%
\pgfpathlineto{\pgfqpoint{6.228950in}{1.638565in}}%
\pgfpathlineto{\pgfqpoint{6.229083in}{1.298352in}}%
\pgfpathlineto{\pgfqpoint{6.230129in}{1.661959in}}%
\pgfpathlineto{\pgfqpoint{6.229852in}{1.081846in}}%
\pgfpathlineto{\pgfqpoint{6.230194in}{1.547244in}}%
\pgfpathlineto{\pgfqpoint{6.231184in}{0.948457in}}%
\pgfpathlineto{\pgfqpoint{6.230850in}{1.649874in}}%
\pgfpathlineto{\pgfqpoint{6.231304in}{1.532675in}}%
\pgfpathlineto{\pgfqpoint{6.232405in}{1.634065in}}%
\pgfpathlineto{\pgfqpoint{6.232079in}{0.974396in}}%
\pgfpathlineto{\pgfqpoint{6.232410in}{1.535694in}}%
\pgfpathlineto{\pgfqpoint{6.232707in}{0.931351in}}%
\pgfpathlineto{\pgfqpoint{6.232635in}{1.689660in}}%
\pgfpathlineto{\pgfqpoint{6.233520in}{1.501210in}}%
\pgfpathlineto{\pgfqpoint{6.233543in}{1.658778in}}%
\pgfpathlineto{\pgfqpoint{6.233981in}{0.890185in}}%
\pgfpathlineto{\pgfqpoint{6.234630in}{1.643200in}}%
\pgfpathlineto{\pgfqpoint{6.234709in}{0.914092in}}%
\pgfpathlineto{\pgfqpoint{6.235741in}{1.559625in}}%
\pgfpathlineto{\pgfqpoint{6.236374in}{0.882229in}}%
\pgfpathlineto{\pgfqpoint{6.236694in}{1.661476in}}%
\pgfpathlineto{\pgfqpoint{6.236853in}{1.515472in}}%
\pgfpathlineto{\pgfqpoint{6.237481in}{1.627433in}}%
\pgfpathlineto{\pgfqpoint{6.236923in}{0.984828in}}%
\pgfpathlineto{\pgfqpoint{6.237961in}{1.527065in}}%
\pgfpathlineto{\pgfqpoint{6.238713in}{0.884214in}}%
\pgfpathlineto{\pgfqpoint{6.238689in}{1.637410in}}%
\pgfpathlineto{\pgfqpoint{6.239070in}{1.493671in}}%
\pgfpathlineto{\pgfqpoint{6.239909in}{1.638351in}}%
\pgfpathlineto{\pgfqpoint{6.240055in}{1.100633in}}%
\pgfpathlineto{\pgfqpoint{6.240181in}{1.539213in}}%
\pgfpathlineto{\pgfqpoint{6.241085in}{0.904000in}}%
\pgfpathlineto{\pgfqpoint{6.240915in}{1.666663in}}%
\pgfpathlineto{\pgfqpoint{6.241292in}{1.421333in}}%
\pgfpathlineto{\pgfqpoint{6.241467in}{1.646616in}}%
\pgfpathlineto{\pgfqpoint{6.241507in}{0.852895in}}%
\pgfpathlineto{\pgfqpoint{6.242400in}{1.529074in}}%
\pgfpathlineto{\pgfqpoint{6.243174in}{1.089060in}}%
\pgfpathlineto{\pgfqpoint{6.242748in}{1.648141in}}%
\pgfpathlineto{\pgfqpoint{6.243510in}{1.562276in}}%
\pgfpathlineto{\pgfqpoint{6.243887in}{0.811073in}}%
\pgfpathlineto{\pgfqpoint{6.244486in}{1.644615in}}%
\pgfpathlineto{\pgfqpoint{6.244623in}{1.424037in}}%
\pgfpathlineto{\pgfqpoint{6.245083in}{1.652387in}}%
\pgfpathlineto{\pgfqpoint{6.245479in}{1.055719in}}%
\pgfpathlineto{\pgfqpoint{6.245734in}{1.540272in}}%
\pgfpathlineto{\pgfqpoint{6.246799in}{0.926441in}}%
\pgfpathlineto{\pgfqpoint{6.246536in}{1.653712in}}%
\pgfpathlineto{\pgfqpoint{6.246845in}{1.558082in}}%
\pgfpathlineto{\pgfqpoint{6.247843in}{0.974307in}}%
\pgfpathlineto{\pgfqpoint{6.247708in}{1.624940in}}%
\pgfpathlineto{\pgfqpoint{6.247956in}{1.213307in}}%
\pgfpathlineto{\pgfqpoint{6.248561in}{1.643752in}}%
\pgfpathlineto{\pgfqpoint{6.248575in}{0.963386in}}%
\pgfpathlineto{\pgfqpoint{6.249068in}{1.451283in}}%
\pgfpathlineto{\pgfqpoint{6.249165in}{0.954258in}}%
\pgfpathlineto{\pgfqpoint{6.250112in}{1.636764in}}%
\pgfpathlineto{\pgfqpoint{6.250176in}{1.296459in}}%
\pgfpathlineto{\pgfqpoint{6.250177in}{1.640894in}}%
\pgfpathlineto{\pgfqpoint{6.250378in}{1.018562in}}%
\pgfpathlineto{\pgfqpoint{6.251287in}{1.499354in}}%
\pgfpathlineto{\pgfqpoint{6.252345in}{0.846127in}}%
\pgfpathlineto{\pgfqpoint{6.251322in}{1.686420in}}%
\pgfpathlineto{\pgfqpoint{6.252398in}{1.513084in}}%
\pgfpathlineto{\pgfqpoint{6.252548in}{0.776531in}}%
\pgfpathlineto{\pgfqpoint{6.253220in}{1.628393in}}%
\pgfpathlineto{\pgfqpoint{6.253509in}{1.491110in}}%
\pgfpathlineto{\pgfqpoint{6.253950in}{0.914502in}}%
\pgfpathlineto{\pgfqpoint{6.253942in}{1.653502in}}%
\pgfpathlineto{\pgfqpoint{6.254583in}{1.590614in}}%
\pgfpathlineto{\pgfqpoint{6.255040in}{1.638185in}}%
\pgfpathlineto{\pgfqpoint{6.255338in}{0.851668in}}%
\pgfpathlineto{\pgfqpoint{6.255691in}{1.495835in}}%
\pgfpathlineto{\pgfqpoint{6.256438in}{0.956484in}}%
\pgfpathlineto{\pgfqpoint{6.256069in}{1.667125in}}%
\pgfpathlineto{\pgfqpoint{6.256802in}{1.407170in}}%
\pgfpathlineto{\pgfqpoint{6.257305in}{1.646328in}}%
\pgfpathlineto{\pgfqpoint{6.257496in}{0.953111in}}%
\pgfpathlineto{\pgfqpoint{6.257913in}{1.477761in}}%
\pgfpathlineto{\pgfqpoint{6.258248in}{0.958452in}}%
\pgfpathlineto{\pgfqpoint{6.258938in}{1.638510in}}%
\pgfpathlineto{\pgfqpoint{6.259023in}{1.327750in}}%
\pgfpathlineto{\pgfqpoint{6.259176in}{1.639861in}}%
\pgfpathlineto{\pgfqpoint{6.259081in}{1.024349in}}%
\pgfpathlineto{\pgfqpoint{6.260135in}{1.512817in}}%
\pgfpathlineto{\pgfqpoint{6.260710in}{1.075773in}}%
\pgfpathlineto{\pgfqpoint{6.260962in}{1.636998in}}%
\pgfpathlineto{\pgfqpoint{6.261246in}{1.488484in}}%
\pgfpathlineto{\pgfqpoint{6.261715in}{0.965181in}}%
\pgfpathlineto{\pgfqpoint{6.261517in}{1.649726in}}%
\pgfpathlineto{\pgfqpoint{6.262354in}{1.432385in}}%
\pgfpathlineto{\pgfqpoint{6.262796in}{1.650731in}}%
\pgfpathlineto{\pgfqpoint{6.263087in}{0.836389in}}%
\pgfpathlineto{\pgfqpoint{6.263464in}{1.444325in}}%
\pgfpathlineto{\pgfqpoint{6.264338in}{1.631275in}}%
\pgfpathlineto{\pgfqpoint{6.264360in}{1.009498in}}%
\pgfpathlineto{\pgfqpoint{6.264561in}{1.560711in}}%
\pgfpathlineto{\pgfqpoint{6.265139in}{0.889581in}}%
\pgfpathlineto{\pgfqpoint{6.264985in}{1.654259in}}%
\pgfpathlineto{\pgfqpoint{6.265672in}{1.398422in}}%
\pgfpathlineto{\pgfqpoint{6.266237in}{1.647594in}}%
\pgfpathlineto{\pgfqpoint{6.266759in}{0.775374in}}%
\pgfpathlineto{\pgfqpoint{6.266783in}{1.498142in}}%
\pgfpathlineto{\pgfqpoint{6.266986in}{0.769988in}}%
\pgfpathlineto{\pgfqpoint{6.267455in}{1.650818in}}%
\pgfpathlineto{\pgfqpoint{6.267895in}{1.241448in}}%
\pgfpathlineto{\pgfqpoint{6.268348in}{1.644465in}}%
\pgfpathlineto{\pgfqpoint{6.268477in}{0.978914in}}%
\pgfpathlineto{\pgfqpoint{6.269006in}{1.554655in}}%
\pgfpathlineto{\pgfqpoint{6.269127in}{0.844269in}}%
\pgfpathlineto{\pgfqpoint{6.269107in}{1.635142in}}%
\pgfpathlineto{\pgfqpoint{6.270116in}{1.391519in}}%
\pgfpathlineto{\pgfqpoint{6.270813in}{1.640731in}}%
\pgfpathlineto{\pgfqpoint{6.270267in}{0.797002in}}%
\pgfpathlineto{\pgfqpoint{6.271228in}{1.599817in}}%
\pgfpathlineto{\pgfqpoint{6.271723in}{0.978854in}}%
\pgfpathlineto{\pgfqpoint{6.272253in}{1.619315in}}%
\pgfpathlineto{\pgfqpoint{6.272340in}{1.393236in}}%
\pgfpathlineto{\pgfqpoint{6.272873in}{1.629713in}}%
\pgfpathlineto{\pgfqpoint{6.272576in}{0.918119in}}%
\pgfpathlineto{\pgfqpoint{6.273450in}{1.468211in}}%
\pgfpathlineto{\pgfqpoint{6.274258in}{1.030773in}}%
\pgfpathlineto{\pgfqpoint{6.274550in}{1.643968in}}%
\pgfpathlineto{\pgfqpoint{6.274559in}{1.501847in}}%
\pgfpathlineto{\pgfqpoint{6.275079in}{1.633528in}}%
\pgfpathlineto{\pgfqpoint{6.275036in}{0.728857in}}%
\pgfpathlineto{\pgfqpoint{6.275668in}{1.527557in}}%
\pgfpathlineto{\pgfqpoint{6.276510in}{0.955162in}}%
\pgfpathlineto{\pgfqpoint{6.276294in}{1.641920in}}%
\pgfpathlineto{\pgfqpoint{6.276778in}{1.392094in}}%
\pgfpathlineto{\pgfqpoint{6.277585in}{1.639614in}}%
\pgfpathlineto{\pgfqpoint{6.277171in}{0.965941in}}%
\pgfpathlineto{\pgfqpoint{6.277889in}{1.506788in}}%
\pgfpathlineto{\pgfqpoint{6.278506in}{0.977255in}}%
\pgfpathlineto{\pgfqpoint{6.278473in}{1.643985in}}%
\pgfpathlineto{\pgfqpoint{6.278998in}{1.598452in}}%
\pgfpathlineto{\pgfqpoint{6.279099in}{1.636708in}}%
\pgfpathlineto{\pgfqpoint{6.279083in}{1.016232in}}%
\pgfpathlineto{\pgfqpoint{6.279375in}{1.401018in}}%
\pgfpathlineto{\pgfqpoint{6.280228in}{0.997988in}}%
\pgfpathlineto{\pgfqpoint{6.279844in}{1.635278in}}%
\pgfpathlineto{\pgfqpoint{6.280485in}{1.486839in}}%
\pgfpathlineto{\pgfqpoint{6.280669in}{1.626777in}}%
\pgfpathlineto{\pgfqpoint{6.281368in}{1.044862in}}%
\pgfpathlineto{\pgfqpoint{6.281592in}{1.360145in}}%
\pgfpathlineto{\pgfqpoint{6.282446in}{0.981852in}}%
\pgfpathlineto{\pgfqpoint{6.282063in}{1.649210in}}%
\pgfpathlineto{\pgfqpoint{6.282683in}{1.187624in}}%
\pgfpathlineto{\pgfqpoint{6.282707in}{1.640975in}}%
\pgfpathlineto{\pgfqpoint{6.283226in}{1.018636in}}%
\pgfpathlineto{\pgfqpoint{6.283795in}{1.387710in}}%
\pgfpathlineto{\pgfqpoint{6.284074in}{1.628730in}}%
\pgfpathlineto{\pgfqpoint{6.283976in}{1.034941in}}%
\pgfpathlineto{\pgfqpoint{6.284908in}{1.514452in}}%
\pgfpathlineto{\pgfqpoint{6.284979in}{0.926880in}}%
\pgfpathlineto{\pgfqpoint{6.284971in}{1.635910in}}%
\pgfpathlineto{\pgfqpoint{6.286009in}{1.293446in}}%
\pgfpathlineto{\pgfqpoint{6.286200in}{1.627899in}}%
\pgfpathlineto{\pgfqpoint{6.286789in}{0.998258in}}%
\pgfpathlineto{\pgfqpoint{6.287120in}{1.407940in}}%
\pgfpathlineto{\pgfqpoint{6.288078in}{0.989830in}}%
\pgfpathlineto{\pgfqpoint{6.287573in}{1.644653in}}%
\pgfpathlineto{\pgfqpoint{6.288231in}{1.417761in}}%
\pgfpathlineto{\pgfqpoint{6.289017in}{0.924391in}}%
\pgfpathlineto{\pgfqpoint{6.288573in}{1.626683in}}%
\pgfpathlineto{\pgfqpoint{6.289291in}{1.367889in}}%
\pgfpathlineto{\pgfqpoint{6.290043in}{1.653202in}}%
\pgfpathlineto{\pgfqpoint{6.290226in}{0.938720in}}%
\pgfpathlineto{\pgfqpoint{6.290401in}{1.545496in}}%
\pgfpathlineto{\pgfqpoint{6.290703in}{0.918209in}}%
\pgfpathlineto{\pgfqpoint{6.290748in}{1.628824in}}%
\pgfpathlineto{\pgfqpoint{6.291513in}{1.306722in}}%
\pgfpathlineto{\pgfqpoint{6.291769in}{1.646181in}}%
\pgfpathlineto{\pgfqpoint{6.292544in}{0.696952in}}%
\pgfpathlineto{\pgfqpoint{6.292623in}{1.377804in}}%
\pgfpathlineto{\pgfqpoint{6.292777in}{1.054093in}}%
\pgfpathlineto{\pgfqpoint{6.293105in}{1.645343in}}%
\pgfpathlineto{\pgfqpoint{6.293733in}{1.488959in}}%
\pgfpathlineto{\pgfqpoint{6.293919in}{0.566229in}}%
\pgfpathlineto{\pgfqpoint{6.294526in}{1.646712in}}%
\pgfpathlineto{\pgfqpoint{6.294844in}{1.370655in}}%
\pgfpathlineto{\pgfqpoint{6.294980in}{1.639508in}}%
\pgfpathlineto{\pgfqpoint{6.294898in}{0.956179in}}%
\pgfpathlineto{\pgfqpoint{6.295954in}{1.511336in}}%
\pgfpathlineto{\pgfqpoint{6.296812in}{0.998476in}}%
\pgfpathlineto{\pgfqpoint{6.296850in}{1.637258in}}%
\pgfpathlineto{\pgfqpoint{6.297064in}{1.515989in}}%
\pgfpathlineto{\pgfqpoint{6.297520in}{1.624922in}}%
\pgfpathlineto{\pgfqpoint{6.298016in}{0.928367in}}%
\pgfpathlineto{\pgfqpoint{6.298173in}{1.446001in}}%
\pgfpathlineto{\pgfqpoint{6.299050in}{0.956956in}}%
\pgfpathlineto{\pgfqpoint{6.298415in}{1.636394in}}%
\pgfpathlineto{\pgfqpoint{6.299273in}{1.491149in}}%
\pgfpathlineto{\pgfqpoint{6.300266in}{1.643766in}}%
\pgfpathlineto{\pgfqpoint{6.300228in}{0.979274in}}%
\pgfpathlineto{\pgfqpoint{6.300383in}{1.463079in}}%
\pgfpathlineto{\pgfqpoint{6.301078in}{1.615245in}}%
\pgfpathlineto{\pgfqpoint{6.300981in}{0.965430in}}%
\pgfpathlineto{\pgfqpoint{6.301252in}{1.487345in}}%
\pgfpathlineto{\pgfqpoint{6.301950in}{0.896713in}}%
\pgfpathlineto{\pgfqpoint{6.301654in}{1.638886in}}%
\pgfpathlineto{\pgfqpoint{6.302363in}{1.481471in}}%
\pgfpathlineto{\pgfqpoint{6.302542in}{1.008876in}}%
\pgfpathlineto{\pgfqpoint{6.303398in}{1.621775in}}%
\pgfpathlineto{\pgfqpoint{6.303474in}{1.424066in}}%
\pgfpathlineto{\pgfqpoint{6.303551in}{1.639376in}}%
\pgfpathlineto{\pgfqpoint{6.303580in}{1.033802in}}%
\pgfpathlineto{\pgfqpoint{6.304585in}{1.438953in}}%
\pgfpathlineto{\pgfqpoint{6.304963in}{0.827441in}}%
\pgfpathlineto{\pgfqpoint{6.305420in}{1.631105in}}%
\pgfpathlineto{\pgfqpoint{6.305696in}{1.301096in}}%
\pgfpathlineto{\pgfqpoint{6.305739in}{1.632538in}}%
\pgfpathlineto{\pgfqpoint{6.306790in}{0.819536in}}%
\pgfpathlineto{\pgfqpoint{6.306806in}{1.323997in}}%
\pgfpathlineto{\pgfqpoint{6.306894in}{1.630945in}}%
\pgfpathlineto{\pgfqpoint{6.307376in}{0.771173in}}%
\pgfpathlineto{\pgfqpoint{6.307918in}{1.491320in}}%
\pgfpathlineto{\pgfqpoint{6.308998in}{0.951092in}}%
\pgfpathlineto{\pgfqpoint{6.308445in}{1.634394in}}%
\pgfpathlineto{\pgfqpoint{6.309029in}{1.142058in}}%
\pgfpathlineto{\pgfqpoint{6.309619in}{1.631427in}}%
\pgfpathlineto{\pgfqpoint{6.309612in}{0.927834in}}%
\pgfpathlineto{\pgfqpoint{6.310141in}{1.493532in}}%
\pgfpathlineto{\pgfqpoint{6.310401in}{0.754743in}}%
\pgfpathlineto{\pgfqpoint{6.310315in}{1.613529in}}%
\pgfpathlineto{\pgfqpoint{6.311252in}{1.426300in}}%
\pgfpathlineto{\pgfqpoint{6.311284in}{1.634593in}}%
\pgfpathlineto{\pgfqpoint{6.311992in}{0.980484in}}%
\pgfpathlineto{\pgfqpoint{6.312362in}{1.437353in}}%
\pgfpathlineto{\pgfqpoint{6.312925in}{1.056276in}}%
\pgfpathlineto{\pgfqpoint{6.312885in}{1.626475in}}%
\pgfpathlineto{\pgfqpoint{6.313470in}{1.519911in}}%
\pgfpathlineto{\pgfqpoint{6.314339in}{1.621208in}}%
\pgfpathlineto{\pgfqpoint{6.314545in}{1.014785in}}%
\pgfpathlineto{\pgfqpoint{6.314579in}{1.462695in}}%
\pgfpathlineto{\pgfqpoint{6.315161in}{0.672665in}}%
\pgfpathlineto{\pgfqpoint{6.314711in}{1.608856in}}%
\pgfpathlineto{\pgfqpoint{6.315690in}{1.376654in}}%
\pgfpathlineto{\pgfqpoint{6.316480in}{1.614301in}}%
\pgfpathlineto{\pgfqpoint{6.316439in}{0.847822in}}%
\pgfpathlineto{\pgfqpoint{6.316799in}{1.483773in}}%
\pgfpathlineto{\pgfqpoint{6.317750in}{0.912416in}}%
\pgfpathlineto{\pgfqpoint{6.317170in}{1.601352in}}%
\pgfpathlineto{\pgfqpoint{6.317911in}{1.385208in}}%
\pgfpathlineto{\pgfqpoint{6.318008in}{1.616813in}}%
\pgfpathlineto{\pgfqpoint{6.318778in}{0.960385in}}%
\pgfpathlineto{\pgfqpoint{6.319017in}{1.324049in}}%
\pgfpathlineto{\pgfqpoint{6.319531in}{0.923556in}}%
\pgfpathlineto{\pgfqpoint{6.320023in}{1.630325in}}%
\pgfpathlineto{\pgfqpoint{6.320126in}{1.512344in}}%
\pgfpathlineto{\pgfqpoint{6.320381in}{0.946294in}}%
\pgfpathlineto{\pgfqpoint{6.321145in}{1.629254in}}%
\pgfpathlineto{\pgfqpoint{6.321238in}{1.381791in}}%
\pgfpathlineto{\pgfqpoint{6.322002in}{1.633674in}}%
\pgfpathlineto{\pgfqpoint{6.321889in}{0.836802in}}%
\pgfpathlineto{\pgfqpoint{6.322350in}{1.515443in}}%
\pgfpathlineto{\pgfqpoint{6.323109in}{0.923743in}}%
\pgfpathlineto{\pgfqpoint{6.323453in}{1.625141in}}%
\pgfpathlineto{\pgfqpoint{6.323461in}{1.503807in}}%
\pgfpathlineto{\pgfqpoint{6.323666in}{0.938001in}}%
\pgfpathlineto{\pgfqpoint{6.324033in}{1.631543in}}%
\pgfpathlineto{\pgfqpoint{6.324571in}{1.483629in}}%
\pgfpathlineto{\pgfqpoint{6.325403in}{1.604669in}}%
\pgfpathlineto{\pgfqpoint{6.325023in}{0.724289in}}%
\pgfpathlineto{\pgfqpoint{6.325679in}{1.542883in}}%
\pgfpathlineto{\pgfqpoint{6.326312in}{1.041889in}}%
\pgfpathlineto{\pgfqpoint{6.325701in}{1.624853in}}%
\pgfpathlineto{\pgfqpoint{6.326791in}{1.260120in}}%
\pgfpathlineto{\pgfqpoint{6.327837in}{1.616618in}}%
\pgfpathlineto{\pgfqpoint{6.327056in}{0.959004in}}%
\pgfpathlineto{\pgfqpoint{6.327902in}{1.492928in}}%
\pgfpathlineto{\pgfqpoint{6.328152in}{1.023418in}}%
\pgfpathlineto{\pgfqpoint{6.327980in}{1.612760in}}%
\pgfpathlineto{\pgfqpoint{6.329012in}{1.516209in}}%
\pgfpathlineto{\pgfqpoint{6.329549in}{1.615258in}}%
\pgfpathlineto{\pgfqpoint{6.329879in}{0.881384in}}%
\pgfpathlineto{\pgfqpoint{6.330122in}{1.571993in}}%
\pgfpathlineto{\pgfqpoint{6.330961in}{0.852114in}}%
\pgfpathlineto{\pgfqpoint{6.330231in}{1.624160in}}%
\pgfpathlineto{\pgfqpoint{6.331233in}{1.514562in}}%
\pgfpathlineto{\pgfqpoint{6.331324in}{0.885518in}}%
\pgfpathlineto{\pgfqpoint{6.331914in}{1.617048in}}%
\pgfpathlineto{\pgfqpoint{6.332342in}{1.440673in}}%
\pgfpathlineto{\pgfqpoint{6.332743in}{1.619521in}}%
\pgfpathlineto{\pgfqpoint{6.333146in}{0.975973in}}%
\pgfpathlineto{\pgfqpoint{6.333452in}{1.562800in}}%
\pgfpathlineto{\pgfqpoint{6.334082in}{1.014950in}}%
\pgfpathlineto{\pgfqpoint{6.334199in}{1.612617in}}%
\pgfpathlineto{\pgfqpoint{6.334564in}{1.122331in}}%
\pgfpathlineto{\pgfqpoint{6.334946in}{1.616368in}}%
\pgfpathlineto{\pgfqpoint{6.335087in}{1.015431in}}%
\pgfpathlineto{\pgfqpoint{6.335676in}{1.503315in}}%
\pgfpathlineto{\pgfqpoint{6.336783in}{0.951473in}}%
\pgfpathlineto{\pgfqpoint{6.336266in}{1.612808in}}%
\pgfpathlineto{\pgfqpoint{6.336786in}{1.470577in}}%
\pgfpathlineto{\pgfqpoint{6.337688in}{1.617564in}}%
\pgfpathlineto{\pgfqpoint{6.337221in}{0.996216in}}%
\pgfpathlineto{\pgfqpoint{6.337897in}{1.510503in}}%
\pgfpathlineto{\pgfqpoint{6.338091in}{0.887261in}}%
\pgfpathlineto{\pgfqpoint{6.337989in}{1.620520in}}%
\pgfpathlineto{\pgfqpoint{6.339008in}{1.465414in}}%
\pgfpathlineto{\pgfqpoint{6.339189in}{0.967787in}}%
\pgfpathlineto{\pgfqpoint{6.339457in}{1.643685in}}%
\pgfpathlineto{\pgfqpoint{6.340118in}{1.420496in}}%
\pgfpathlineto{\pgfqpoint{6.340323in}{1.638031in}}%
\pgfpathlineto{\pgfqpoint{6.340660in}{0.967921in}}%
\pgfpathlineto{\pgfqpoint{6.341227in}{1.577048in}}%
\pgfpathlineto{\pgfqpoint{6.341228in}{0.871268in}}%
\pgfpathlineto{\pgfqpoint{6.342174in}{1.630294in}}%
\pgfpathlineto{\pgfqpoint{6.342338in}{1.362044in}}%
\pgfpathlineto{\pgfqpoint{6.343194in}{0.940079in}}%
\pgfpathlineto{\pgfqpoint{6.342455in}{1.609629in}}%
\pgfpathlineto{\pgfqpoint{6.343443in}{1.488676in}}%
\pgfpathlineto{\pgfqpoint{6.344325in}{1.631755in}}%
\pgfpathlineto{\pgfqpoint{6.344490in}{0.964794in}}%
\pgfpathlineto{\pgfqpoint{6.344552in}{1.559596in}}%
\pgfpathlineto{\pgfqpoint{6.344834in}{0.927351in}}%
\pgfpathlineto{\pgfqpoint{6.345432in}{1.626012in}}%
\pgfpathlineto{\pgfqpoint{6.345663in}{1.284346in}}%
\pgfpathlineto{\pgfqpoint{6.345782in}{1.617648in}}%
\pgfpathlineto{\pgfqpoint{6.346368in}{1.067766in}}%
\pgfpathlineto{\pgfqpoint{6.346775in}{1.458231in}}%
\pgfpathlineto{\pgfqpoint{6.346923in}{1.647929in}}%
\pgfpathlineto{\pgfqpoint{6.347526in}{1.056283in}}%
\pgfpathlineto{\pgfqpoint{6.347884in}{1.492439in}}%
\pgfpathlineto{\pgfqpoint{6.348955in}{0.778854in}}%
\pgfpathlineto{\pgfqpoint{6.347992in}{1.610148in}}%
\pgfpathlineto{\pgfqpoint{6.348994in}{1.514861in}}%
\pgfpathlineto{\pgfqpoint{6.349690in}{0.905265in}}%
\pgfpathlineto{\pgfqpoint{6.350044in}{1.608337in}}%
\pgfpathlineto{\pgfqpoint{6.350105in}{1.413683in}}%
\pgfpathlineto{\pgfqpoint{6.350489in}{1.611276in}}%
\pgfpathlineto{\pgfqpoint{6.350617in}{0.916036in}}%
\pgfpathlineto{\pgfqpoint{6.351214in}{1.322412in}}%
\pgfpathlineto{\pgfqpoint{6.351690in}{1.004824in}}%
\pgfpathlineto{\pgfqpoint{6.351450in}{1.604142in}}%
\pgfpathlineto{\pgfqpoint{6.352323in}{1.038146in}}%
\pgfpathlineto{\pgfqpoint{6.353005in}{1.614635in}}%
\pgfpathlineto{\pgfqpoint{6.352407in}{0.757318in}}%
\pgfpathlineto{\pgfqpoint{6.353435in}{1.504040in}}%
\pgfpathlineto{\pgfqpoint{6.354242in}{1.634798in}}%
\pgfpathlineto{\pgfqpoint{6.354057in}{0.913856in}}%
\pgfpathlineto{\pgfqpoint{6.354532in}{1.415640in}}%
\pgfpathlineto{\pgfqpoint{6.354626in}{0.930319in}}%
\pgfpathlineto{\pgfqpoint{6.355567in}{1.622061in}}%
\pgfpathlineto{\pgfqpoint{6.355642in}{1.363096in}}%
\pgfpathlineto{\pgfqpoint{6.356421in}{1.612551in}}%
\pgfpathlineto{\pgfqpoint{6.356574in}{0.852405in}}%
\pgfpathlineto{\pgfqpoint{6.356753in}{1.412580in}}%
\pgfpathlineto{\pgfqpoint{6.356982in}{0.894985in}}%
\pgfpathlineto{\pgfqpoint{6.357502in}{1.621664in}}%
\pgfpathlineto{\pgfqpoint{6.357863in}{1.290007in}}%
\pgfpathlineto{\pgfqpoint{6.358580in}{1.602007in}}%
\pgfpathlineto{\pgfqpoint{6.358928in}{0.877934in}}%
\pgfpathlineto{\pgfqpoint{6.358974in}{1.484888in}}%
\pgfpathlineto{\pgfqpoint{6.359098in}{0.975060in}}%
\pgfpathlineto{\pgfqpoint{6.359168in}{1.605645in}}%
\pgfpathlineto{\pgfqpoint{6.360085in}{1.309919in}}%
\pgfpathlineto{\pgfqpoint{6.361182in}{1.609253in}}%
\pgfpathlineto{\pgfqpoint{6.360447in}{0.997671in}}%
\pgfpathlineto{\pgfqpoint{6.361199in}{1.506759in}}%
\pgfpathlineto{\pgfqpoint{6.362210in}{0.826020in}}%
\pgfpathlineto{\pgfqpoint{6.361667in}{1.606747in}}%
\pgfpathlineto{\pgfqpoint{6.362310in}{1.406345in}}%
\pgfpathlineto{\pgfqpoint{6.362898in}{1.593510in}}%
\pgfpathlineto{\pgfqpoint{6.362880in}{0.903283in}}%
\pgfpathlineto{\pgfqpoint{6.363420in}{1.430536in}}%
\pgfpathlineto{\pgfqpoint{6.363736in}{0.955358in}}%
\pgfpathlineto{\pgfqpoint{6.363477in}{1.603766in}}%
\pgfpathlineto{\pgfqpoint{6.364531in}{1.368027in}}%
\pgfpathlineto{\pgfqpoint{6.364848in}{0.833774in}}%
\pgfpathlineto{\pgfqpoint{6.365525in}{1.602890in}}%
\pgfpathlineto{\pgfqpoint{6.365640in}{1.420752in}}%
\pgfpathlineto{\pgfqpoint{6.365673in}{1.612616in}}%
\pgfpathlineto{\pgfqpoint{6.366077in}{0.926494in}}%
\pgfpathlineto{\pgfqpoint{6.366752in}{1.562294in}}%
\pgfpathlineto{\pgfqpoint{6.367810in}{0.881138in}}%
\pgfpathlineto{\pgfqpoint{6.366990in}{1.622986in}}%
\pgfpathlineto{\pgfqpoint{6.367863in}{1.471740in}}%
\pgfpathlineto{\pgfqpoint{6.368009in}{1.615674in}}%
\pgfpathlineto{\pgfqpoint{6.368908in}{0.900958in}}%
\pgfpathlineto{\pgfqpoint{6.368969in}{1.512081in}}%
\pgfpathlineto{\pgfqpoint{6.369224in}{0.884881in}}%
\pgfpathlineto{\pgfqpoint{6.369350in}{1.603384in}}%
\pgfpathlineto{\pgfqpoint{6.370080in}{1.393392in}}%
\pgfpathlineto{\pgfqpoint{6.370086in}{0.808773in}}%
\pgfpathlineto{\pgfqpoint{6.370789in}{1.614456in}}%
\pgfpathlineto{\pgfqpoint{6.371191in}{1.449875in}}%
\pgfpathlineto{\pgfqpoint{6.372192in}{1.604991in}}%
\pgfpathlineto{\pgfqpoint{6.372072in}{1.001975in}}%
\pgfpathlineto{\pgfqpoint{6.372300in}{1.431823in}}%
\pgfpathlineto{\pgfqpoint{6.373258in}{0.900330in}}%
\pgfpathlineto{\pgfqpoint{6.373247in}{1.623504in}}%
\pgfpathlineto{\pgfqpoint{6.373412in}{1.274451in}}%
\pgfpathlineto{\pgfqpoint{6.374369in}{1.613254in}}%
\pgfpathlineto{\pgfqpoint{6.373450in}{0.783938in}}%
\pgfpathlineto{\pgfqpoint{6.374523in}{1.442721in}}%
\pgfpathlineto{\pgfqpoint{6.375405in}{0.927582in}}%
\pgfpathlineto{\pgfqpoint{6.375460in}{1.599000in}}%
\pgfpathlineto{\pgfqpoint{6.375634in}{1.291857in}}%
\pgfpathlineto{\pgfqpoint{6.376625in}{1.607020in}}%
\pgfpathlineto{\pgfqpoint{6.375736in}{0.838680in}}%
\pgfpathlineto{\pgfqpoint{6.376746in}{1.508401in}}%
\pgfpathlineto{\pgfqpoint{6.376794in}{0.715826in}}%
\pgfpathlineto{\pgfqpoint{6.377310in}{1.597757in}}%
\pgfpathlineto{\pgfqpoint{6.377859in}{1.257052in}}%
\pgfpathlineto{\pgfqpoint{6.378339in}{1.611302in}}%
\pgfpathlineto{\pgfqpoint{6.378658in}{1.004846in}}%
\pgfpathlineto{\pgfqpoint{6.378970in}{1.351963in}}%
\pgfpathlineto{\pgfqpoint{6.379626in}{1.597604in}}%
\pgfpathlineto{\pgfqpoint{6.379721in}{0.940334in}}%
\pgfpathlineto{\pgfqpoint{6.380082in}{1.416740in}}%
\pgfpathlineto{\pgfqpoint{6.380944in}{0.959489in}}%
\pgfpathlineto{\pgfqpoint{6.380650in}{1.602417in}}%
\pgfpathlineto{\pgfqpoint{6.381192in}{1.484296in}}%
\pgfpathlineto{\pgfqpoint{6.381239in}{1.626404in}}%
\pgfpathlineto{\pgfqpoint{6.381216in}{0.681947in}}%
\pgfpathlineto{\pgfqpoint{6.382302in}{1.447905in}}%
\pgfpathlineto{\pgfqpoint{6.382666in}{0.858666in}}%
\pgfpathlineto{\pgfqpoint{6.383200in}{1.613383in}}%
\pgfpathlineto{\pgfqpoint{6.383400in}{1.474989in}}%
\pgfpathlineto{\pgfqpoint{6.384045in}{1.611617in}}%
\pgfpathlineto{\pgfqpoint{6.384355in}{0.979288in}}%
\pgfpathlineto{\pgfqpoint{6.384506in}{1.357991in}}%
\pgfpathlineto{\pgfqpoint{6.385413in}{0.820139in}}%
\pgfpathlineto{\pgfqpoint{6.385141in}{1.599901in}}%
\pgfpathlineto{\pgfqpoint{6.385616in}{1.263942in}}%
\pgfpathlineto{\pgfqpoint{6.386188in}{1.602263in}}%
\pgfpathlineto{\pgfqpoint{6.386335in}{0.943044in}}%
\pgfpathlineto{\pgfqpoint{6.386727in}{1.549649in}}%
\pgfpathlineto{\pgfqpoint{6.387435in}{0.930379in}}%
\pgfpathlineto{\pgfqpoint{6.387295in}{1.593900in}}%
\pgfpathlineto{\pgfqpoint{6.387838in}{1.503213in}}%
\pgfpathlineto{\pgfqpoint{6.388349in}{1.620367in}}%
\pgfpathlineto{\pgfqpoint{6.388119in}{0.818957in}}%
\pgfpathlineto{\pgfqpoint{6.388901in}{1.347408in}}%
\pgfpathlineto{\pgfqpoint{6.388902in}{0.940770in}}%
\pgfpathlineto{\pgfqpoint{6.389988in}{1.596345in}}%
\pgfpathlineto{\pgfqpoint{6.390012in}{1.389094in}}%
\pgfpathlineto{\pgfqpoint{6.390916in}{1.601411in}}%
\pgfpathlineto{\pgfqpoint{6.390478in}{0.982389in}}%
\pgfpathlineto{\pgfqpoint{6.391122in}{1.451719in}}%
\pgfpathlineto{\pgfqpoint{6.391399in}{0.811568in}}%
\pgfpathlineto{\pgfqpoint{6.391487in}{1.600140in}}%
\pgfpathlineto{\pgfqpoint{6.392233in}{1.368389in}}%
\pgfpathlineto{\pgfqpoint{6.392412in}{1.611977in}}%
\pgfpathlineto{\pgfqpoint{6.392470in}{0.880060in}}%
\pgfpathlineto{\pgfqpoint{6.393344in}{1.473180in}}%
\pgfpathlineto{\pgfqpoint{6.393408in}{0.826455in}}%
\pgfpathlineto{\pgfqpoint{6.394366in}{1.612227in}}%
\pgfpathlineto{\pgfqpoint{6.394454in}{1.331371in}}%
\pgfpathlineto{\pgfqpoint{6.395010in}{1.586742in}}%
\pgfpathlineto{\pgfqpoint{6.395031in}{0.777787in}}%
\pgfpathlineto{\pgfqpoint{6.395565in}{1.477641in}}%
\pgfpathlineto{\pgfqpoint{6.395751in}{0.844940in}}%
\pgfpathlineto{\pgfqpoint{6.395759in}{1.613703in}}%
\pgfpathlineto{\pgfqpoint{6.396677in}{1.326108in}}%
\pgfpathlineto{\pgfqpoint{6.397415in}{1.621513in}}%
\pgfpathlineto{\pgfqpoint{6.397231in}{0.989382in}}%
\pgfpathlineto{\pgfqpoint{6.397788in}{1.366944in}}%
\pgfpathlineto{\pgfqpoint{6.398736in}{1.595364in}}%
\pgfpathlineto{\pgfqpoint{6.398562in}{0.971780in}}%
\pgfpathlineto{\pgfqpoint{6.398899in}{1.464229in}}%
\pgfpathlineto{\pgfqpoint{6.399510in}{0.788356in}}%
\pgfpathlineto{\pgfqpoint{6.399903in}{1.595095in}}%
\pgfpathlineto{\pgfqpoint{6.400010in}{1.285306in}}%
\pgfpathlineto{\pgfqpoint{6.400385in}{1.608050in}}%
\pgfpathlineto{\pgfqpoint{6.401101in}{0.912531in}}%
\pgfpathlineto{\pgfqpoint{6.401121in}{1.350848in}}%
\pgfpathlineto{\pgfqpoint{6.401338in}{1.619804in}}%
\pgfpathlineto{\pgfqpoint{6.401702in}{0.886836in}}%
\pgfpathlineto{\pgfqpoint{6.402232in}{1.552062in}}%
\pgfpathlineto{\pgfqpoint{6.403326in}{0.905296in}}%
\pgfpathlineto{\pgfqpoint{6.403129in}{1.609430in}}%
\pgfpathlineto{\pgfqpoint{6.403344in}{1.335300in}}%
\pgfpathlineto{\pgfqpoint{6.403619in}{0.853918in}}%
\pgfpathlineto{\pgfqpoint{6.403905in}{1.592996in}}%
\pgfpathlineto{\pgfqpoint{6.404449in}{1.459645in}}%
\pgfpathlineto{\pgfqpoint{6.405305in}{1.593521in}}%
\pgfpathlineto{\pgfqpoint{6.404801in}{0.883416in}}%
\pgfpathlineto{\pgfqpoint{6.405558in}{1.530925in}}%
\pgfpathlineto{\pgfqpoint{6.406073in}{0.908293in}}%
\pgfpathlineto{\pgfqpoint{6.406293in}{1.606198in}}%
\pgfpathlineto{\pgfqpoint{6.406669in}{1.487420in}}%
\pgfpathlineto{\pgfqpoint{6.407662in}{0.904476in}}%
\pgfpathlineto{\pgfqpoint{6.407671in}{1.601555in}}%
\pgfpathlineto{\pgfqpoint{6.407780in}{1.408759in}}%
\pgfpathlineto{\pgfqpoint{6.407904in}{1.627771in}}%
\pgfpathlineto{\pgfqpoint{6.408215in}{0.903012in}}%
\pgfpathlineto{\pgfqpoint{6.408889in}{1.370718in}}%
\pgfpathlineto{\pgfqpoint{6.409258in}{0.948173in}}%
\pgfpathlineto{\pgfqpoint{6.409779in}{1.602218in}}%
\pgfpathlineto{\pgfqpoint{6.409999in}{1.422741in}}%
\pgfpathlineto{\pgfqpoint{6.410369in}{0.946377in}}%
\pgfpathlineto{\pgfqpoint{6.410620in}{1.609416in}}%
\pgfpathlineto{\pgfqpoint{6.411110in}{1.395948in}}%
\pgfpathlineto{\pgfqpoint{6.411922in}{1.593102in}}%
\pgfpathlineto{\pgfqpoint{6.412161in}{0.894537in}}%
\pgfpathlineto{\pgfqpoint{6.412213in}{1.460034in}}%
\pgfpathlineto{\pgfqpoint{6.412803in}{0.779046in}}%
\pgfpathlineto{\pgfqpoint{6.413270in}{1.594395in}}%
\pgfpathlineto{\pgfqpoint{6.413323in}{1.464208in}}%
\pgfpathlineto{\pgfqpoint{6.413710in}{0.743769in}}%
\pgfpathlineto{\pgfqpoint{6.414350in}{1.581435in}}%
\pgfpathlineto{\pgfqpoint{6.414435in}{1.287427in}}%
\pgfpathlineto{\pgfqpoint{6.415051in}{1.599681in}}%
\pgfpathlineto{\pgfqpoint{6.415128in}{0.866540in}}%
\pgfpathlineto{\pgfqpoint{6.415544in}{1.497717in}}%
\pgfpathlineto{\pgfqpoint{6.416463in}{0.801737in}}%
\pgfpathlineto{\pgfqpoint{6.415973in}{1.588416in}}%
\pgfpathlineto{\pgfqpoint{6.416655in}{1.454856in}}%
\pgfpathlineto{\pgfqpoint{6.416991in}{1.587657in}}%
\pgfpathlineto{\pgfqpoint{6.417143in}{0.930989in}}%
\pgfpathlineto{\pgfqpoint{6.417759in}{1.567327in}}%
\pgfpathlineto{\pgfqpoint{6.418659in}{0.818391in}}%
\pgfpathlineto{\pgfqpoint{6.417900in}{1.586480in}}%
\pgfpathlineto{\pgfqpoint{6.418871in}{1.290793in}}%
\pgfpathlineto{\pgfqpoint{6.419803in}{1.616318in}}%
\pgfpathlineto{\pgfqpoint{6.419405in}{0.945827in}}%
\pgfpathlineto{\pgfqpoint{6.419982in}{1.272152in}}%
\pgfpathlineto{\pgfqpoint{6.420161in}{1.583766in}}%
\pgfpathlineto{\pgfqpoint{6.420063in}{0.794553in}}%
\pgfpathlineto{\pgfqpoint{6.421094in}{1.535402in}}%
\pgfpathlineto{\pgfqpoint{6.421132in}{0.935114in}}%
\pgfpathlineto{\pgfqpoint{6.422054in}{1.587211in}}%
\pgfpathlineto{\pgfqpoint{6.422205in}{1.301125in}}%
\pgfpathlineto{\pgfqpoint{6.422387in}{1.583380in}}%
\pgfpathlineto{\pgfqpoint{6.422977in}{0.926315in}}%
\pgfpathlineto{\pgfqpoint{6.423315in}{1.380503in}}%
\pgfpathlineto{\pgfqpoint{6.423547in}{0.955927in}}%
\pgfpathlineto{\pgfqpoint{6.424010in}{1.593894in}}%
\pgfpathlineto{\pgfqpoint{6.424426in}{1.289150in}}%
\pgfpathlineto{\pgfqpoint{6.424750in}{1.613720in}}%
\pgfpathlineto{\pgfqpoint{6.425195in}{0.802676in}}%
\pgfpathlineto{\pgfqpoint{6.425536in}{1.282641in}}%
\pgfpathlineto{\pgfqpoint{6.425820in}{1.588245in}}%
\pgfpathlineto{\pgfqpoint{6.426438in}{0.900156in}}%
\pgfpathlineto{\pgfqpoint{6.426647in}{1.425728in}}%
\pgfpathlineto{\pgfqpoint{6.427490in}{0.871021in}}%
\pgfpathlineto{\pgfqpoint{6.427095in}{1.600607in}}%
\pgfpathlineto{\pgfqpoint{6.427759in}{1.323131in}}%
\pgfpathlineto{\pgfqpoint{6.428253in}{1.584327in}}%
\pgfpathlineto{\pgfqpoint{6.428201in}{0.945418in}}%
\pgfpathlineto{\pgfqpoint{6.428870in}{1.446454in}}%
\pgfpathlineto{\pgfqpoint{6.429863in}{0.767297in}}%
\pgfpathlineto{\pgfqpoint{6.429637in}{1.589920in}}%
\pgfpathlineto{\pgfqpoint{6.429981in}{1.292225in}}%
\pgfpathlineto{\pgfqpoint{6.430288in}{1.582753in}}%
\pgfpathlineto{\pgfqpoint{6.430499in}{0.913438in}}%
\pgfpathlineto{\pgfqpoint{6.431093in}{1.519222in}}%
\pgfpathlineto{\pgfqpoint{6.431493in}{0.837883in}}%
\pgfpathlineto{\pgfqpoint{6.431920in}{1.576844in}}%
\pgfpathlineto{\pgfqpoint{6.432204in}{1.331768in}}%
\pgfpathlineto{\pgfqpoint{6.432752in}{1.567845in}}%
\pgfpathlineto{\pgfqpoint{6.432805in}{0.692300in}}%
\pgfpathlineto{\pgfqpoint{6.433316in}{1.455397in}}%
\pgfpathlineto{\pgfqpoint{6.433759in}{0.905621in}}%
\pgfpathlineto{\pgfqpoint{6.434343in}{1.590593in}}%
\pgfpathlineto{\pgfqpoint{6.434427in}{1.480149in}}%
\pgfpathlineto{\pgfqpoint{6.434846in}{1.602686in}}%
\pgfpathlineto{\pgfqpoint{6.434891in}{0.938298in}}%
\pgfpathlineto{\pgfqpoint{6.435509in}{1.443826in}}%
\pgfpathlineto{\pgfqpoint{6.436489in}{0.936719in}}%
\pgfpathlineto{\pgfqpoint{6.436356in}{1.580935in}}%
\pgfpathlineto{\pgfqpoint{6.436620in}{1.412689in}}%
\pgfpathlineto{\pgfqpoint{6.436931in}{0.838597in}}%
\pgfpathlineto{\pgfqpoint{6.437453in}{1.596113in}}%
\pgfpathlineto{\pgfqpoint{6.437731in}{1.348571in}}%
\pgfpathlineto{\pgfqpoint{6.437905in}{1.578417in}}%
\pgfpathlineto{\pgfqpoint{6.437935in}{0.853007in}}%
\pgfpathlineto{\pgfqpoint{6.438843in}{1.480718in}}%
\pgfpathlineto{\pgfqpoint{6.439693in}{0.833637in}}%
\pgfpathlineto{\pgfqpoint{6.439898in}{1.578392in}}%
\pgfpathlineto{\pgfqpoint{6.439954in}{1.205603in}}%
\pgfpathlineto{\pgfqpoint{6.441018in}{1.572666in}}%
\pgfpathlineto{\pgfqpoint{6.440362in}{1.009813in}}%
\pgfpathlineto{\pgfqpoint{6.441065in}{1.474049in}}%
\pgfpathlineto{\pgfqpoint{6.441376in}{0.905050in}}%
\pgfpathlineto{\pgfqpoint{6.441579in}{1.577875in}}%
\pgfpathlineto{\pgfqpoint{6.442177in}{1.203499in}}%
\pgfpathlineto{\pgfqpoint{6.443168in}{1.593493in}}%
\pgfpathlineto{\pgfqpoint{6.443062in}{0.836698in}}%
\pgfpathlineto{\pgfqpoint{6.443288in}{1.450444in}}%
\pgfpathlineto{\pgfqpoint{6.443483in}{1.007001in}}%
\pgfpathlineto{\pgfqpoint{6.443429in}{1.603944in}}%
\pgfpathlineto{\pgfqpoint{6.444397in}{1.356141in}}%
\pgfpathlineto{\pgfqpoint{6.444986in}{1.571464in}}%
\pgfpathlineto{\pgfqpoint{6.445371in}{0.836483in}}%
\pgfpathlineto{\pgfqpoint{6.445508in}{1.437251in}}%
\pgfpathlineto{\pgfqpoint{6.446566in}{0.963755in}}%
\pgfpathlineto{\pgfqpoint{6.445647in}{1.576441in}}%
\pgfpathlineto{\pgfqpoint{6.446619in}{1.251259in}}%
\pgfpathlineto{\pgfqpoint{6.447720in}{1.603160in}}%
\pgfpathlineto{\pgfqpoint{6.447581in}{0.863110in}}%
\pgfpathlineto{\pgfqpoint{6.447728in}{1.407390in}}%
\pgfpathlineto{\pgfqpoint{6.448669in}{0.884611in}}%
\pgfpathlineto{\pgfqpoint{6.447873in}{1.581886in}}%
\pgfpathlineto{\pgfqpoint{6.448839in}{1.201910in}}%
\pgfpathlineto{\pgfqpoint{6.449677in}{1.566353in}}%
\pgfpathlineto{\pgfqpoint{6.449135in}{0.738238in}}%
\pgfpathlineto{\pgfqpoint{6.449950in}{1.512090in}}%
\pgfpathlineto{\pgfqpoint{6.450620in}{0.843783in}}%
\pgfpathlineto{\pgfqpoint{6.450086in}{1.579260in}}%
\pgfpathlineto{\pgfqpoint{6.451061in}{1.304715in}}%
\pgfpathlineto{\pgfqpoint{6.452167in}{0.897019in}}%
\pgfpathlineto{\pgfqpoint{6.451451in}{1.560085in}}%
\pgfpathlineto{\pgfqpoint{6.452169in}{1.433207in}}%
\pgfpathlineto{\pgfqpoint{6.452590in}{1.577212in}}%
\pgfpathlineto{\pgfqpoint{6.453166in}{0.739850in}}%
\pgfpathlineto{\pgfqpoint{6.453279in}{1.479964in}}%
\pgfpathlineto{\pgfqpoint{6.454216in}{0.681275in}}%
\pgfpathlineto{\pgfqpoint{6.454130in}{1.588571in}}%
\pgfpathlineto{\pgfqpoint{6.454390in}{1.444369in}}%
\pgfpathlineto{\pgfqpoint{6.455025in}{1.588680in}}%
\pgfpathlineto{\pgfqpoint{6.455104in}{0.915791in}}%
\pgfpathlineto{\pgfqpoint{6.455500in}{1.367713in}}%
\pgfpathlineto{\pgfqpoint{6.456141in}{1.586779in}}%
\pgfpathlineto{\pgfqpoint{6.456324in}{0.846190in}}%
\pgfpathlineto{\pgfqpoint{6.456604in}{1.404604in}}%
\pgfpathlineto{\pgfqpoint{6.456956in}{0.912574in}}%
\pgfpathlineto{\pgfqpoint{6.457392in}{1.565587in}}%
\pgfpathlineto{\pgfqpoint{6.457716in}{1.191344in}}%
\pgfpathlineto{\pgfqpoint{6.458353in}{1.596204in}}%
\pgfpathlineto{\pgfqpoint{6.458304in}{0.728355in}}%
\pgfpathlineto{\pgfqpoint{6.458827in}{1.484793in}}%
\pgfpathlineto{\pgfqpoint{6.459866in}{0.951709in}}%
\pgfpathlineto{\pgfqpoint{6.459006in}{1.625358in}}%
\pgfpathlineto{\pgfqpoint{6.459938in}{1.453512in}}%
\pgfpathlineto{\pgfqpoint{6.460083in}{0.836872in}}%
\pgfpathlineto{\pgfqpoint{6.460921in}{1.563397in}}%
\pgfpathlineto{\pgfqpoint{6.461050in}{1.289072in}}%
\pgfpathlineto{\pgfqpoint{6.461212in}{1.565534in}}%
\pgfpathlineto{\pgfqpoint{6.461750in}{0.869158in}}%
\pgfpathlineto{\pgfqpoint{6.462161in}{1.308103in}}%
\pgfpathlineto{\pgfqpoint{6.462185in}{0.954078in}}%
\pgfpathlineto{\pgfqpoint{6.462592in}{1.586177in}}%
\pgfpathlineto{\pgfqpoint{6.463271in}{1.470895in}}%
\pgfpathlineto{\pgfqpoint{6.463456in}{0.886464in}}%
\pgfpathlineto{\pgfqpoint{6.463845in}{1.557562in}}%
\pgfpathlineto{\pgfqpoint{6.464382in}{1.343221in}}%
\pgfpathlineto{\pgfqpoint{6.464818in}{1.593551in}}%
\pgfpathlineto{\pgfqpoint{6.464578in}{0.935174in}}%
\pgfpathlineto{\pgfqpoint{6.465494in}{1.450180in}}%
\pgfpathlineto{\pgfqpoint{6.465652in}{0.807651in}}%
\pgfpathlineto{\pgfqpoint{6.465796in}{1.573594in}}%
\pgfpathlineto{\pgfqpoint{6.466606in}{1.372944in}}%
\pgfpathlineto{\pgfqpoint{6.467674in}{0.702679in}}%
\pgfpathlineto{\pgfqpoint{6.466942in}{1.571209in}}%
\pgfpathlineto{\pgfqpoint{6.467689in}{1.301124in}}%
\pgfpathlineto{\pgfqpoint{6.467970in}{1.587225in}}%
\pgfpathlineto{\pgfqpoint{6.468361in}{0.974835in}}%
\pgfpathlineto{\pgfqpoint{6.468800in}{1.505174in}}%
\pgfpathlineto{\pgfqpoint{6.469849in}{0.913482in}}%
\pgfpathlineto{\pgfqpoint{6.469875in}{1.580758in}}%
\pgfpathlineto{\pgfqpoint{6.469911in}{1.414958in}}%
\pgfpathlineto{\pgfqpoint{6.470985in}{1.573501in}}%
\pgfpathlineto{\pgfqpoint{6.470012in}{0.853911in}}%
\pgfpathlineto{\pgfqpoint{6.471015in}{1.367765in}}%
\pgfpathlineto{\pgfqpoint{6.471587in}{0.904260in}}%
\pgfpathlineto{\pgfqpoint{6.471169in}{1.582575in}}%
\pgfpathlineto{\pgfqpoint{6.472126in}{1.508183in}}%
\pgfpathlineto{\pgfqpoint{6.472662in}{0.923686in}}%
\pgfpathlineto{\pgfqpoint{6.472194in}{1.569873in}}%
\pgfpathlineto{\pgfqpoint{6.473239in}{1.390610in}}%
\pgfpathlineto{\pgfqpoint{6.473920in}{1.570051in}}%
\pgfpathlineto{\pgfqpoint{6.474220in}{0.881913in}}%
\pgfpathlineto{\pgfqpoint{6.474350in}{1.467829in}}%
\pgfpathlineto{\pgfqpoint{6.474519in}{0.639337in}}%
\pgfpathlineto{\pgfqpoint{6.474519in}{1.582406in}}%
\pgfpathlineto{\pgfqpoint{6.475461in}{1.202668in}}%
\pgfpathlineto{\pgfqpoint{6.475493in}{1.574270in}}%
\pgfpathlineto{\pgfqpoint{6.475976in}{0.827035in}}%
\pgfpathlineto{\pgfqpoint{6.476571in}{1.386412in}}%
\pgfpathlineto{\pgfqpoint{6.477324in}{0.885632in}}%
\pgfpathlineto{\pgfqpoint{6.477160in}{1.583432in}}%
\pgfpathlineto{\pgfqpoint{6.477681in}{1.392209in}}%
\pgfpathlineto{\pgfqpoint{6.477902in}{1.571491in}}%
\pgfpathlineto{\pgfqpoint{6.477899in}{0.898704in}}%
\pgfpathlineto{\pgfqpoint{6.478792in}{1.473849in}}%
\pgfpathlineto{\pgfqpoint{6.479746in}{0.821825in}}%
\pgfpathlineto{\pgfqpoint{6.479148in}{1.578579in}}%
\pgfpathlineto{\pgfqpoint{6.479907in}{1.338663in}}%
\pgfpathlineto{\pgfqpoint{6.480457in}{1.566384in}}%
\pgfpathlineto{\pgfqpoint{6.480033in}{0.914502in}}%
\pgfpathlineto{\pgfqpoint{6.481018in}{1.487103in}}%
\pgfpathlineto{\pgfqpoint{6.482052in}{0.922684in}}%
\pgfpathlineto{\pgfqpoint{6.482087in}{1.576558in}}%
\pgfpathlineto{\pgfqpoint{6.482130in}{1.269352in}}%
\pgfpathlineto{\pgfqpoint{6.482388in}{1.574008in}}%
\pgfpathlineto{\pgfqpoint{6.482767in}{0.878675in}}%
\pgfpathlineto{\pgfqpoint{6.483242in}{1.415591in}}%
\pgfpathlineto{\pgfqpoint{6.483898in}{0.942229in}}%
\pgfpathlineto{\pgfqpoint{6.483350in}{1.585531in}}%
\pgfpathlineto{\pgfqpoint{6.484351in}{1.316662in}}%
\pgfpathlineto{\pgfqpoint{6.485425in}{1.573270in}}%
\pgfpathlineto{\pgfqpoint{6.484479in}{0.741397in}}%
\pgfpathlineto{\pgfqpoint{6.485463in}{1.454586in}}%
\pgfpathlineto{\pgfqpoint{6.486223in}{0.725039in}}%
\pgfpathlineto{\pgfqpoint{6.486349in}{1.564141in}}%
\pgfpathlineto{\pgfqpoint{6.486573in}{1.465803in}}%
\pgfpathlineto{\pgfqpoint{6.486962in}{1.580414in}}%
\pgfpathlineto{\pgfqpoint{6.487164in}{0.360679in}}%
\pgfpathmoveto{\pgfqpoint{6.487164in}{0.360679in}}%
\pgfpathlineto{\pgfqpoint{6.488048in}{1.569190in}}%
\pgfpathlineto{\pgfqpoint{6.488275in}{1.469182in}}%
\pgfpathlineto{\pgfqpoint{6.488904in}{1.560307in}}%
\pgfpathlineto{\pgfqpoint{6.489089in}{0.830357in}}%
\pgfpathlineto{\pgfqpoint{6.489382in}{1.458907in}}%
\pgfpathlineto{\pgfqpoint{6.490357in}{0.941066in}}%
\pgfpathlineto{\pgfqpoint{6.490159in}{1.563961in}}%
\pgfpathlineto{\pgfqpoint{6.490493in}{1.187506in}}%
\pgfpathlineto{\pgfqpoint{6.491544in}{1.576173in}}%
\pgfpathlineto{\pgfqpoint{6.491019in}{0.976182in}}%
\pgfpathlineto{\pgfqpoint{6.491605in}{1.462010in}}%
\pgfpathlineto{\pgfqpoint{6.491771in}{0.756198in}}%
\pgfpathlineto{\pgfqpoint{6.491865in}{1.573439in}}%
\pgfpathlineto{\pgfqpoint{6.492716in}{1.265944in}}%
\pgfpathlineto{\pgfqpoint{6.493548in}{1.563251in}}%
\pgfpathlineto{\pgfqpoint{6.493369in}{0.813310in}}%
\pgfpathlineto{\pgfqpoint{6.493827in}{1.225873in}}%
\pgfpathlineto{\pgfqpoint{6.494308in}{1.552768in}}%
\pgfpathlineto{\pgfqpoint{6.494824in}{0.722385in}}%
\pgfpathlineto{\pgfqpoint{6.494939in}{1.488852in}}%
\pgfpathlineto{\pgfqpoint{6.495624in}{0.886703in}}%
\pgfpathlineto{\pgfqpoint{6.495641in}{1.554406in}}%
\pgfpathlineto{\pgfqpoint{6.496050in}{1.423742in}}%
\pgfpathlineto{\pgfqpoint{6.496339in}{1.562023in}}%
\pgfpathlineto{\pgfqpoint{6.496832in}{0.814182in}}%
\pgfpathlineto{\pgfqpoint{6.497159in}{1.432185in}}%
\pgfpathlineto{\pgfqpoint{6.497814in}{0.910161in}}%
\pgfpathlineto{\pgfqpoint{6.497496in}{1.561506in}}%
\pgfpathlineto{\pgfqpoint{6.498270in}{1.320194in}}%
\pgfpathlineto{\pgfqpoint{6.499375in}{1.564074in}}%
\pgfpathlineto{\pgfqpoint{6.498369in}{0.908807in}}%
\pgfpathlineto{\pgfqpoint{6.499381in}{1.327519in}}%
\pgfpathlineto{\pgfqpoint{6.500118in}{1.563535in}}%
\pgfpathlineto{\pgfqpoint{6.499644in}{0.862425in}}%
\pgfpathlineto{\pgfqpoint{6.500492in}{1.440533in}}%
\pgfpathlineto{\pgfqpoint{6.500856in}{0.881546in}}%
\pgfpathlineto{\pgfqpoint{6.500914in}{1.573156in}}%
\pgfpathlineto{\pgfqpoint{6.501604in}{1.384492in}}%
\pgfpathlineto{\pgfqpoint{6.502258in}{1.591622in}}%
\pgfpathlineto{\pgfqpoint{6.502621in}{0.880054in}}%
\pgfpathlineto{\pgfqpoint{6.502714in}{1.522606in}}%
\pgfpathlineto{\pgfqpoint{6.503426in}{0.915792in}}%
\pgfpathlineto{\pgfqpoint{6.503661in}{1.569217in}}%
\pgfpathlineto{\pgfqpoint{6.503826in}{1.488347in}}%
\pgfpathlineto{\pgfqpoint{6.504805in}{0.882260in}}%
\pgfpathlineto{\pgfqpoint{6.503857in}{1.588030in}}%
\pgfpathlineto{\pgfqpoint{6.504937in}{1.376226in}}%
\pgfpathlineto{\pgfqpoint{6.505614in}{1.562529in}}%
\pgfpathlineto{\pgfqpoint{6.505496in}{0.937911in}}%
\pgfpathlineto{\pgfqpoint{6.506048in}{1.490704in}}%
\pgfpathlineto{\pgfqpoint{6.506905in}{0.814163in}}%
\pgfpathlineto{\pgfqpoint{6.506971in}{1.552344in}}%
\pgfpathlineto{\pgfqpoint{6.507160in}{0.964278in}}%
\pgfpathlineto{\pgfqpoint{6.507991in}{1.561160in}}%
\pgfpathlineto{\pgfqpoint{6.508253in}{0.760046in}}%
\pgfpathlineto{\pgfqpoint{6.508272in}{1.469558in}}%
\pgfpathlineto{\pgfqpoint{6.508994in}{0.816520in}}%
\pgfpathlineto{\pgfqpoint{6.508783in}{1.550373in}}%
\pgfpathlineto{\pgfqpoint{6.509383in}{1.241263in}}%
\pgfpathlineto{\pgfqpoint{6.509975in}{1.561755in}}%
\pgfpathlineto{\pgfqpoint{6.510114in}{0.818090in}}%
\pgfpathlineto{\pgfqpoint{6.510494in}{1.407120in}}%
\pgfpathlineto{\pgfqpoint{6.511420in}{0.791645in}}%
\pgfpathlineto{\pgfqpoint{6.510778in}{1.564204in}}%
\pgfpathlineto{\pgfqpoint{6.511604in}{1.474736in}}%
\pgfpathlineto{\pgfqpoint{6.512397in}{0.891441in}}%
\pgfpathlineto{\pgfqpoint{6.511982in}{1.563085in}}%
\pgfpathlineto{\pgfqpoint{6.512716in}{1.363121in}}%
\pgfpathlineto{\pgfqpoint{6.513679in}{1.561513in}}%
\pgfpathlineto{\pgfqpoint{6.512942in}{0.828965in}}%
\pgfpathlineto{\pgfqpoint{6.513827in}{1.364077in}}%
\pgfpathlineto{\pgfqpoint{6.514725in}{1.571490in}}%
\pgfpathlineto{\pgfqpoint{6.514455in}{0.850929in}}%
\pgfpathlineto{\pgfqpoint{6.514932in}{1.373621in}}%
\pgfpathlineto{\pgfqpoint{6.514933in}{0.858193in}}%
\pgfpathlineto{\pgfqpoint{6.515712in}{1.557595in}}%
\pgfpathlineto{\pgfqpoint{6.516043in}{1.207792in}}%
\pgfpathlineto{\pgfqpoint{6.516798in}{1.557497in}}%
\pgfpathlineto{\pgfqpoint{6.517032in}{0.736316in}}%
\pgfpathlineto{\pgfqpoint{6.517155in}{1.472599in}}%
\pgfpathlineto{\pgfqpoint{6.517643in}{0.844880in}}%
\pgfpathlineto{\pgfqpoint{6.517606in}{1.558011in}}%
\pgfpathlineto{\pgfqpoint{6.518267in}{1.065541in}}%
\pgfpathlineto{\pgfqpoint{6.518980in}{1.566351in}}%
\pgfpathlineto{\pgfqpoint{6.518517in}{0.730326in}}%
\pgfpathlineto{\pgfqpoint{6.519378in}{1.272994in}}%
\pgfpathlineto{\pgfqpoint{6.520316in}{1.554724in}}%
\pgfpathlineto{\pgfqpoint{6.519649in}{0.661697in}}%
\pgfpathlineto{\pgfqpoint{6.520489in}{1.362800in}}%
\pgfpathlineto{\pgfqpoint{6.520704in}{0.949370in}}%
\pgfpathlineto{\pgfqpoint{6.521352in}{1.550810in}}%
\pgfpathlineto{\pgfqpoint{6.521598in}{1.356579in}}%
\pgfpathlineto{\pgfqpoint{6.522220in}{1.548357in}}%
\pgfpathlineto{\pgfqpoint{6.521903in}{0.726090in}}%
\pgfpathlineto{\pgfqpoint{6.522709in}{1.451626in}}%
\pgfpathlineto{\pgfqpoint{6.522817in}{0.746165in}}%
\pgfpathlineto{\pgfqpoint{6.523212in}{1.559562in}}%
\pgfpathlineto{\pgfqpoint{6.523820in}{1.180137in}}%
\pgfpathlineto{\pgfqpoint{6.524460in}{1.550417in}}%
\pgfpathlineto{\pgfqpoint{6.524800in}{0.790049in}}%
\pgfpathlineto{\pgfqpoint{6.524932in}{1.390905in}}%
\pgfpathlineto{\pgfqpoint{6.525987in}{0.814436in}}%
\pgfpathlineto{\pgfqpoint{6.525483in}{1.546619in}}%
\pgfpathlineto{\pgfqpoint{6.526019in}{1.447738in}}%
\pgfpathlineto{\pgfqpoint{6.526019in}{1.558647in}}%
\pgfpathlineto{\pgfqpoint{6.526588in}{0.471191in}}%
\pgfpathlineto{\pgfqpoint{6.527128in}{1.320745in}}%
\pgfpathlineto{\pgfqpoint{6.527734in}{0.729803in}}%
\pgfpathlineto{\pgfqpoint{6.527988in}{1.558501in}}%
\pgfpathlineto{\pgfqpoint{6.528238in}{1.360925in}}%
\pgfpathlineto{\pgfqpoint{6.528876in}{1.545329in}}%
\pgfpathlineto{\pgfqpoint{6.529204in}{0.766797in}}%
\pgfpathlineto{\pgfqpoint{6.529347in}{1.390624in}}%
\pgfpathlineto{\pgfqpoint{6.529740in}{0.615753in}}%
\pgfpathlineto{\pgfqpoint{6.529531in}{1.542542in}}%
\pgfpathlineto{\pgfqpoint{6.530458in}{1.250361in}}%
\pgfpathlineto{\pgfqpoint{6.530659in}{1.588859in}}%
\pgfpathlineto{\pgfqpoint{6.531152in}{0.927471in}}%
\pgfpathlineto{\pgfqpoint{6.531569in}{1.371374in}}%
\pgfpathlineto{\pgfqpoint{6.532131in}{1.556394in}}%
\pgfpathlineto{\pgfqpoint{6.532127in}{0.797989in}}%
\pgfpathlineto{\pgfqpoint{6.532680in}{1.412143in}}%
\pgfpathlineto{\pgfqpoint{6.533500in}{0.823639in}}%
\pgfpathlineto{\pgfqpoint{6.533743in}{1.576029in}}%
\pgfpathlineto{\pgfqpoint{6.533791in}{1.312238in}}%
\pgfpathlineto{\pgfqpoint{6.534134in}{1.565391in}}%
\pgfpathlineto{\pgfqpoint{6.534676in}{0.461740in}}%
\pgfpathlineto{\pgfqpoint{6.534902in}{1.435238in}}%
\pgfpathlineto{\pgfqpoint{6.535723in}{0.697310in}}%
\pgfpathlineto{\pgfqpoint{6.534906in}{1.543243in}}%
\pgfpathlineto{\pgfqpoint{6.536013in}{1.452866in}}%
\pgfpathlineto{\pgfqpoint{6.536567in}{0.769307in}}%
\pgfpathlineto{\pgfqpoint{6.536136in}{1.537511in}}%
\pgfpathlineto{\pgfqpoint{6.537123in}{1.395720in}}%
\pgfpathlineto{\pgfqpoint{6.537203in}{1.556781in}}%
\pgfpathlineto{\pgfqpoint{6.537991in}{0.925743in}}%
\pgfpathlineto{\pgfqpoint{6.538235in}{1.492983in}}%
\pgfpathlineto{\pgfqpoint{6.538765in}{0.729549in}}%
\pgfpathlineto{\pgfqpoint{6.539240in}{1.571565in}}%
\pgfpathlineto{\pgfqpoint{6.539346in}{1.386425in}}%
\pgfpathlineto{\pgfqpoint{6.540200in}{1.563077in}}%
\pgfpathlineto{\pgfqpoint{6.540150in}{0.851662in}}%
\pgfpathlineto{\pgfqpoint{6.540457in}{1.426123in}}%
\pgfpathlineto{\pgfqpoint{6.540924in}{1.560602in}}%
\pgfpathlineto{\pgfqpoint{6.540608in}{0.892991in}}%
\pgfpathlineto{\pgfqpoint{6.541558in}{1.452496in}}%
\pgfpathlineto{\pgfqpoint{6.541753in}{0.740508in}}%
\pgfpathlineto{\pgfqpoint{6.542564in}{1.558913in}}%
\pgfpathlineto{\pgfqpoint{6.542669in}{1.374234in}}%
\pgfpathlineto{\pgfqpoint{6.543530in}{1.556444in}}%
\pgfpathlineto{\pgfqpoint{6.543087in}{0.781247in}}%
\pgfpathlineto{\pgfqpoint{6.543775in}{1.457248in}}%
\pgfpathlineto{\pgfqpoint{6.544140in}{0.782224in}}%
\pgfpathlineto{\pgfqpoint{6.544611in}{1.566349in}}%
\pgfpathlineto{\pgfqpoint{6.544886in}{1.400839in}}%
\pgfpathlineto{\pgfqpoint{6.545345in}{0.769001in}}%
\pgfpathlineto{\pgfqpoint{6.545414in}{1.544460in}}%
\pgfpathlineto{\pgfqpoint{6.545998in}{1.272324in}}%
\pgfpathlineto{\pgfqpoint{6.546706in}{0.789648in}}%
\pgfpathlineto{\pgfqpoint{6.546626in}{1.561090in}}%
\pgfpathlineto{\pgfqpoint{6.547097in}{1.366752in}}%
\pgfpathlineto{\pgfqpoint{6.548182in}{1.549227in}}%
\pgfpathlineto{\pgfqpoint{6.547510in}{0.766087in}}%
\pgfpathlineto{\pgfqpoint{6.548208in}{1.475719in}}%
\pgfpathlineto{\pgfqpoint{6.548905in}{0.631171in}}%
\pgfpathlineto{\pgfqpoint{6.548458in}{1.545465in}}%
\pgfpathlineto{\pgfqpoint{6.549319in}{1.412690in}}%
\pgfpathlineto{\pgfqpoint{6.549949in}{1.559829in}}%
\pgfpathlineto{\pgfqpoint{6.550079in}{0.671659in}}%
\pgfpathlineto{\pgfqpoint{6.550426in}{1.344299in}}%
\pgfpathlineto{\pgfqpoint{6.550762in}{0.909579in}}%
\pgfpathlineto{\pgfqpoint{6.551373in}{1.543757in}}%
\pgfpathlineto{\pgfqpoint{6.551537in}{1.311184in}}%
\pgfpathlineto{\pgfqpoint{6.551886in}{1.567633in}}%
\pgfpathlineto{\pgfqpoint{6.552353in}{0.618666in}}%
\pgfpathlineto{\pgfqpoint{6.552649in}{1.437317in}}%
\pgfpathlineto{\pgfqpoint{6.553357in}{1.551935in}}%
\pgfpathlineto{\pgfqpoint{6.553545in}{0.847897in}}%
\pgfpathlineto{\pgfqpoint{6.553758in}{1.395408in}}%
\pgfpathlineto{\pgfqpoint{6.554705in}{0.599642in}}%
\pgfpathlineto{\pgfqpoint{6.554186in}{1.553042in}}%
\pgfpathlineto{\pgfqpoint{6.554869in}{1.360367in}}%
\pgfpathlineto{\pgfqpoint{6.555849in}{0.764243in}}%
\pgfpathlineto{\pgfqpoint{6.555663in}{1.539847in}}%
\pgfpathlineto{\pgfqpoint{6.555981in}{1.305314in}}%
\pgfpathlineto{\pgfqpoint{6.556015in}{0.835677in}}%
\pgfpathlineto{\pgfqpoint{6.556250in}{1.549600in}}%
\pgfpathlineto{\pgfqpoint{6.557083in}{1.312463in}}%
\pgfpathlineto{\pgfqpoint{6.557129in}{1.555570in}}%
\pgfpathlineto{\pgfqpoint{6.558028in}{0.779075in}}%
\pgfpathlineto{\pgfqpoint{6.558194in}{1.335873in}}%
\pgfpathlineto{\pgfqpoint{6.559238in}{1.546111in}}%
\pgfpathlineto{\pgfqpoint{6.559107in}{0.846959in}}%
\pgfpathlineto{\pgfqpoint{6.559256in}{1.355242in}}%
\pgfpathlineto{\pgfqpoint{6.559256in}{0.667555in}}%
\pgfpathlineto{\pgfqpoint{6.559408in}{1.531312in}}%
\pgfpathlineto{\pgfqpoint{6.560367in}{1.411761in}}%
\pgfpathlineto{\pgfqpoint{6.561476in}{0.653575in}}%
\pgfpathlineto{\pgfqpoint{6.561142in}{1.553121in}}%
\pgfpathlineto{\pgfqpoint{6.561477in}{1.399225in}}%
\pgfpathlineto{\pgfqpoint{6.562138in}{1.548873in}}%
\pgfpathlineto{\pgfqpoint{6.561829in}{0.800103in}}%
\pgfpathlineto{\pgfqpoint{6.562577in}{1.241628in}}%
\pgfpathlineto{\pgfqpoint{6.562927in}{0.850278in}}%
\pgfpathlineto{\pgfqpoint{6.562913in}{1.548735in}}%
\pgfpathlineto{\pgfqpoint{6.562965in}{1.360126in}}%
\pgfusepath{stroke}%
\end{pgfscope}%
\begin{pgfscope}%
\pgfpathrectangle{\pgfqpoint{0.535225in}{0.370679in}}{\pgfqpoint{6.314775in}{3.181174in}}%
\pgfusepath{clip}%
\pgfsetrectcap%
\pgfsetroundjoin%
\pgfsetlinewidth{3.011250pt}%
\definecolor{currentstroke}{rgb}{1.000000,0.411765,0.705882}%
\pgfsetstrokecolor{currentstroke}%
\pgfsetdash{}{0pt}%
\pgfpathmoveto{\pgfqpoint{0.525225in}{3.232708in}}%
\pgfpathlineto{\pgfqpoint{0.822260in}{3.234608in}}%
\pgfpathlineto{\pgfqpoint{1.116406in}{2.856413in}}%
\pgfpathlineto{\pgfqpoint{1.288470in}{3.200444in}}%
\pgfpathlineto{\pgfqpoint{1.410551in}{3.008738in}}%
\pgfpathlineto{\pgfqpoint{1.505245in}{3.131283in}}%
\pgfpathlineto{\pgfqpoint{1.582616in}{3.096680in}}%
\pgfpathlineto{\pgfqpoint{1.648031in}{3.051817in}}%
\pgfpathlineto{\pgfqpoint{1.704697in}{2.997501in}}%
\pgfpathlineto{\pgfqpoint{1.754680in}{2.972754in}}%
\pgfpathlineto{\pgfqpoint{1.799391in}{2.871035in}}%
\pgfpathlineto{\pgfqpoint{1.839837in}{2.794006in}}%
\pgfpathlineto{\pgfqpoint{1.876761in}{3.101617in}}%
\pgfpathlineto{\pgfqpoint{1.910729in}{2.765480in}}%
\pgfpathlineto{\pgfqpoint{1.942177in}{3.072035in}}%
\pgfpathlineto{\pgfqpoint{1.971455in}{3.009405in}}%
\pgfpathlineto{\pgfqpoint{1.998843in}{3.090659in}}%
\pgfpathlineto{\pgfqpoint{2.024570in}{3.019725in}}%
\pgfpathlineto{\pgfqpoint{2.048826in}{3.011985in}}%
\pgfpathlineto{\pgfqpoint{2.071770in}{2.881947in}}%
\pgfpathlineto{\pgfqpoint{2.093537in}{3.003549in}}%
\pgfpathlineto{\pgfqpoint{2.114241in}{2.951537in}}%
\pgfpathlineto{\pgfqpoint{2.133983in}{2.995507in}}%
\pgfpathlineto{\pgfqpoint{2.152847in}{3.105933in}}%
\pgfpathlineto{\pgfqpoint{2.170907in}{2.788857in}}%
\pgfpathlineto{\pgfqpoint{2.188231in}{2.958069in}}%
\pgfpathlineto{\pgfqpoint{2.204874in}{2.957812in}}%
\pgfpathlineto{\pgfqpoint{2.220890in}{2.934125in}}%
\pgfpathlineto{\pgfqpoint{2.236323in}{2.987661in}}%
\pgfpathlineto{\pgfqpoint{2.251214in}{2.967124in}}%
\pgfpathlineto{\pgfqpoint{2.265601in}{2.885321in}}%
\pgfpathlineto{\pgfqpoint{2.279516in}{2.944368in}}%
\pgfpathlineto{\pgfqpoint{2.292989in}{2.982811in}}%
\pgfpathlineto{\pgfqpoint{2.306047in}{2.960134in}}%
\pgfpathlineto{\pgfqpoint{2.318716in}{2.959440in}}%
\pgfpathlineto{\pgfqpoint{2.331017in}{2.993054in}}%
\pgfpathlineto{\pgfqpoint{2.342971in}{2.980791in}}%
\pgfpathlineto{\pgfqpoint{2.365916in}{3.042117in}}%
\pgfpathlineto{\pgfqpoint{2.376939in}{2.941032in}}%
\pgfpathlineto{\pgfqpoint{2.387683in}{2.909366in}}%
\pgfpathlineto{\pgfqpoint{2.398161in}{3.005002in}}%
\pgfpathlineto{\pgfqpoint{2.408387in}{2.832340in}}%
\pgfpathlineto{\pgfqpoint{2.418373in}{2.885832in}}%
\pgfpathlineto{\pgfqpoint{2.428129in}{2.813656in}}%
\pgfpathlineto{\pgfqpoint{2.437665in}{2.848088in}}%
\pgfpathlineto{\pgfqpoint{2.446992in}{2.781228in}}%
\pgfpathlineto{\pgfqpoint{2.456119in}{2.871839in}}%
\pgfpathlineto{\pgfqpoint{2.465053in}{2.816849in}}%
\pgfpathlineto{\pgfqpoint{2.473803in}{3.023808in}}%
\pgfpathlineto{\pgfqpoint{2.482376in}{2.918423in}}%
\pgfpathlineto{\pgfqpoint{2.490780in}{2.917006in}}%
\pgfpathlineto{\pgfqpoint{2.499020in}{2.846376in}}%
\pgfpathlineto{\pgfqpoint{2.507103in}{2.901476in}}%
\pgfpathlineto{\pgfqpoint{2.515036in}{2.874294in}}%
\pgfpathlineto{\pgfqpoint{2.522822in}{2.949269in}}%
\pgfpathlineto{\pgfqpoint{2.530469in}{3.004933in}}%
\pgfpathlineto{\pgfqpoint{2.537980in}{2.835578in}}%
\pgfpathlineto{\pgfqpoint{2.545360in}{2.860487in}}%
\pgfpathlineto{\pgfqpoint{2.552614in}{2.795225in}}%
\pgfpathlineto{\pgfqpoint{2.559747in}{2.840191in}}%
\pgfpathlineto{\pgfqpoint{2.566761in}{2.852079in}}%
\pgfpathlineto{\pgfqpoint{2.573662in}{2.926928in}}%
\pgfpathlineto{\pgfqpoint{2.580451in}{2.823171in}}%
\pgfpathlineto{\pgfqpoint{2.593714in}{2.867454in}}%
\pgfpathlineto{\pgfqpoint{2.600193in}{2.928379in}}%
\pgfpathlineto{\pgfqpoint{2.606574in}{2.931043in}}%
\pgfpathlineto{\pgfqpoint{2.612861in}{2.896748in}}%
\pgfpathlineto{\pgfqpoint{2.619057in}{2.942463in}}%
\pgfpathlineto{\pgfqpoint{2.625163in}{2.797784in}}%
\pgfpathlineto{\pgfqpoint{2.631182in}{2.834314in}}%
\pgfpathlineto{\pgfqpoint{2.637117in}{2.905403in}}%
\pgfpathlineto{\pgfqpoint{2.642971in}{2.882967in}}%
\pgfpathlineto{\pgfqpoint{2.648744in}{2.853434in}}%
\pgfpathlineto{\pgfqpoint{2.654441in}{2.881183in}}%
\pgfpathlineto{\pgfqpoint{2.660061in}{2.936833in}}%
\pgfpathlineto{\pgfqpoint{2.665609in}{2.874535in}}%
\pgfpathlineto{\pgfqpoint{2.671084in}{2.886364in}}%
\pgfpathlineto{\pgfqpoint{2.676490in}{2.775314in}}%
\pgfpathlineto{\pgfqpoint{2.681828in}{2.907627in}}%
\pgfpathlineto{\pgfqpoint{2.687100in}{2.899456in}}%
\pgfpathlineto{\pgfqpoint{2.692307in}{2.982052in}}%
\pgfpathlineto{\pgfqpoint{2.697451in}{2.678199in}}%
\pgfpathlineto{\pgfqpoint{2.702533in}{2.737774in}}%
\pgfpathlineto{\pgfqpoint{2.707555in}{2.949733in}}%
\pgfpathlineto{\pgfqpoint{2.717424in}{2.765665in}}%
\pgfpathlineto{\pgfqpoint{2.722274in}{2.704959in}}%
\pgfpathlineto{\pgfqpoint{2.727069in}{2.819838in}}%
\pgfpathlineto{\pgfqpoint{2.731811in}{2.894165in}}%
\pgfpathlineto{\pgfqpoint{2.741138in}{2.877776in}}%
\pgfpathlineto{\pgfqpoint{2.745726in}{2.787272in}}%
\pgfpathlineto{\pgfqpoint{2.750264in}{2.916061in}}%
\pgfpathlineto{\pgfqpoint{2.754755in}{2.722665in}}%
\pgfpathlineto{\pgfqpoint{2.759199in}{2.711328in}}%
\pgfpathlineto{\pgfqpoint{2.763596in}{2.729547in}}%
\pgfpathlineto{\pgfqpoint{2.767949in}{2.881177in}}%
\pgfpathlineto{\pgfqpoint{2.772257in}{2.866375in}}%
\pgfpathlineto{\pgfqpoint{2.776522in}{2.935943in}}%
\pgfpathlineto{\pgfqpoint{2.780745in}{2.815046in}}%
\pgfpathlineto{\pgfqpoint{2.784926in}{2.976288in}}%
\pgfpathlineto{\pgfqpoint{2.789066in}{2.754928in}}%
\pgfpathlineto{\pgfqpoint{2.793166in}{2.705921in}}%
\pgfpathlineto{\pgfqpoint{2.797227in}{2.743654in}}%
\pgfpathlineto{\pgfqpoint{2.801249in}{2.609526in}}%
\pgfpathlineto{\pgfqpoint{2.805234in}{2.826795in}}%
\pgfpathlineto{\pgfqpoint{2.809181in}{2.769760in}}%
\pgfpathlineto{\pgfqpoint{2.813093in}{2.692250in}}%
\pgfpathlineto{\pgfqpoint{2.816968in}{2.932500in}}%
\pgfpathlineto{\pgfqpoint{2.820809in}{2.577231in}}%
\pgfpathlineto{\pgfqpoint{2.824615in}{2.765057in}}%
\pgfpathlineto{\pgfqpoint{2.828387in}{2.796632in}}%
\pgfpathlineto{\pgfqpoint{2.832126in}{2.680405in}}%
\pgfpathlineto{\pgfqpoint{2.835832in}{2.938573in}}%
\pgfpathlineto{\pgfqpoint{2.839506in}{2.794091in}}%
\pgfpathlineto{\pgfqpoint{2.843149in}{2.805488in}}%
\pgfpathlineto{\pgfqpoint{2.846760in}{2.875057in}}%
\pgfpathlineto{\pgfqpoint{2.850341in}{2.800321in}}%
\pgfpathlineto{\pgfqpoint{2.853893in}{2.762168in}}%
\pgfpathlineto{\pgfqpoint{2.857414in}{2.745156in}}%
\pgfpathlineto{\pgfqpoint{2.860907in}{2.860764in}}%
\pgfpathlineto{\pgfqpoint{2.864371in}{2.605742in}}%
\pgfpathlineto{\pgfqpoint{2.867807in}{2.870534in}}%
\pgfpathlineto{\pgfqpoint{2.871216in}{2.677461in}}%
\pgfpathlineto{\pgfqpoint{2.874597in}{2.937084in}}%
\pgfpathlineto{\pgfqpoint{2.877952in}{2.801890in}}%
\pgfpathlineto{\pgfqpoint{2.881280in}{2.726516in}}%
\pgfpathlineto{\pgfqpoint{2.884583in}{2.892112in}}%
\pgfpathlineto{\pgfqpoint{2.887860in}{2.708144in}}%
\pgfpathlineto{\pgfqpoint{2.891111in}{2.764835in}}%
\pgfpathlineto{\pgfqpoint{2.894339in}{2.399024in}}%
\pgfpathlineto{\pgfqpoint{2.897541in}{2.783592in}}%
\pgfpathlineto{\pgfqpoint{2.900720in}{2.714843in}}%
\pgfpathlineto{\pgfqpoint{2.903875in}{2.820377in}}%
\pgfpathlineto{\pgfqpoint{2.907007in}{2.749529in}}%
\pgfpathlineto{\pgfqpoint{2.910116in}{2.897369in}}%
\pgfpathlineto{\pgfqpoint{2.913202in}{2.914421in}}%
\pgfpathlineto{\pgfqpoint{2.919308in}{2.682008in}}%
\pgfpathlineto{\pgfqpoint{2.922329in}{2.841298in}}%
\pgfpathlineto{\pgfqpoint{2.925328in}{2.686408in}}%
\pgfpathlineto{\pgfqpoint{2.928306in}{2.862885in}}%
\pgfpathlineto{\pgfqpoint{2.931263in}{2.887356in}}%
\pgfpathlineto{\pgfqpoint{2.934200in}{2.710510in}}%
\pgfpathlineto{\pgfqpoint{2.937116in}{2.869659in}}%
\pgfpathlineto{\pgfqpoint{2.940013in}{2.777012in}}%
\pgfpathlineto{\pgfqpoint{2.942890in}{2.932617in}}%
\pgfpathlineto{\pgfqpoint{2.945748in}{2.864914in}}%
\pgfpathlineto{\pgfqpoint{2.948586in}{2.835613in}}%
\pgfpathlineto{\pgfqpoint{2.951406in}{2.783403in}}%
\pgfpathlineto{\pgfqpoint{2.954207in}{2.665543in}}%
\pgfpathlineto{\pgfqpoint{2.956990in}{2.789362in}}%
\pgfpathlineto{\pgfqpoint{2.962501in}{2.859195in}}%
\pgfpathlineto{\pgfqpoint{2.965230in}{2.759061in}}%
\pgfpathlineto{\pgfqpoint{2.967942in}{2.747275in}}%
\pgfpathlineto{\pgfqpoint{2.970636in}{2.779996in}}%
\pgfpathlineto{\pgfqpoint{2.973313in}{2.914602in}}%
\pgfpathlineto{\pgfqpoint{2.975974in}{2.754963in}}%
\pgfpathlineto{\pgfqpoint{2.978618in}{2.962982in}}%
\pgfpathlineto{\pgfqpoint{2.981246in}{2.839271in}}%
\pgfpathlineto{\pgfqpoint{2.983857in}{2.825376in}}%
\pgfpathlineto{\pgfqpoint{2.986453in}{2.924144in}}%
\pgfpathlineto{\pgfqpoint{2.989032in}{2.797031in}}%
\pgfpathlineto{\pgfqpoint{2.991597in}{2.855827in}}%
\pgfpathlineto{\pgfqpoint{2.994145in}{2.848381in}}%
\pgfpathlineto{\pgfqpoint{2.996679in}{2.788925in}}%
\pgfpathlineto{\pgfqpoint{3.004190in}{2.912277in}}%
\pgfpathlineto{\pgfqpoint{3.006664in}{2.744430in}}%
\pgfpathlineto{\pgfqpoint{3.009124in}{2.816520in}}%
\pgfpathlineto{\pgfqpoint{3.011570in}{2.778753in}}%
\pgfpathlineto{\pgfqpoint{3.014002in}{2.860054in}}%
\pgfpathlineto{\pgfqpoint{3.016420in}{2.624321in}}%
\pgfpathlineto{\pgfqpoint{3.018824in}{2.587832in}}%
\pgfpathlineto{\pgfqpoint{3.023593in}{2.735850in}}%
\pgfpathlineto{\pgfqpoint{3.025957in}{2.930552in}}%
\pgfpathlineto{\pgfqpoint{3.028308in}{2.814040in}}%
\pgfpathlineto{\pgfqpoint{3.030646in}{2.908856in}}%
\pgfpathlineto{\pgfqpoint{3.032971in}{2.436615in}}%
\pgfpathlineto{\pgfqpoint{3.035284in}{2.632097in}}%
\pgfpathlineto{\pgfqpoint{3.037584in}{2.953298in}}%
\pgfpathlineto{\pgfqpoint{3.042147in}{2.758389in}}%
\pgfpathlineto{\pgfqpoint{3.044410in}{2.876620in}}%
\pgfpathlineto{\pgfqpoint{3.046661in}{2.889111in}}%
\pgfpathlineto{\pgfqpoint{3.048901in}{2.845136in}}%
\pgfpathlineto{\pgfqpoint{3.051128in}{2.833283in}}%
\pgfpathlineto{\pgfqpoint{3.053344in}{2.796798in}}%
\pgfpathlineto{\pgfqpoint{3.055549in}{2.856280in}}%
\pgfpathlineto{\pgfqpoint{3.057742in}{2.693011in}}%
\pgfpathlineto{\pgfqpoint{3.062095in}{2.927700in}}%
\pgfpathlineto{\pgfqpoint{3.064254in}{2.808175in}}%
\pgfpathlineto{\pgfqpoint{3.066403in}{2.836854in}}%
\pgfpathlineto{\pgfqpoint{3.068541in}{2.717555in}}%
\pgfpathlineto{\pgfqpoint{3.070668in}{2.694405in}}%
\pgfpathlineto{\pgfqpoint{3.072784in}{2.927770in}}%
\pgfpathlineto{\pgfqpoint{3.076986in}{2.701209in}}%
\pgfpathlineto{\pgfqpoint{3.079071in}{2.851582in}}%
\pgfpathlineto{\pgfqpoint{3.083211in}{2.777797in}}%
\pgfpathlineto{\pgfqpoint{3.085267in}{2.792065in}}%
\pgfpathlineto{\pgfqpoint{3.087312in}{2.707116in}}%
\pgfpathlineto{\pgfqpoint{3.089347in}{2.836939in}}%
\pgfpathlineto{\pgfqpoint{3.091373in}{2.781227in}}%
\pgfpathlineto{\pgfqpoint{3.093389in}{2.917364in}}%
\pgfpathlineto{\pgfqpoint{3.095395in}{2.838577in}}%
\pgfpathlineto{\pgfqpoint{3.097392in}{2.805308in}}%
\pgfpathlineto{\pgfqpoint{3.101358in}{2.637183in}}%
\pgfpathlineto{\pgfqpoint{3.103327in}{2.801737in}}%
\pgfpathlineto{\pgfqpoint{3.105287in}{2.806252in}}%
\pgfpathlineto{\pgfqpoint{3.107238in}{2.907422in}}%
\pgfpathlineto{\pgfqpoint{3.109181in}{2.483642in}}%
\pgfpathlineto{\pgfqpoint{3.111114in}{2.819626in}}%
\pgfpathlineto{\pgfqpoint{3.113038in}{2.864083in}}%
\pgfpathlineto{\pgfqpoint{3.114954in}{2.939944in}}%
\pgfpathlineto{\pgfqpoint{3.118760in}{2.776131in}}%
\pgfpathlineto{\pgfqpoint{3.120651in}{2.691318in}}%
\pgfpathlineto{\pgfqpoint{3.122532in}{2.870060in}}%
\pgfpathlineto{\pgfqpoint{3.124406in}{2.875220in}}%
\pgfpathlineto{\pgfqpoint{3.128128in}{2.782860in}}%
\pgfpathlineto{\pgfqpoint{3.129978in}{2.530586in}}%
\pgfpathlineto{\pgfqpoint{3.131819in}{2.833295in}}%
\pgfpathlineto{\pgfqpoint{3.133652in}{2.833970in}}%
\pgfpathlineto{\pgfqpoint{3.137294in}{2.563842in}}%
\pgfpathlineto{\pgfqpoint{3.139104in}{2.848801in}}%
\pgfpathlineto{\pgfqpoint{3.140906in}{2.819474in}}%
\pgfpathlineto{\pgfqpoint{3.142700in}{2.810583in}}%
\pgfpathlineto{\pgfqpoint{3.144487in}{2.581105in}}%
\pgfpathlineto{\pgfqpoint{3.146266in}{2.663079in}}%
\pgfpathlineto{\pgfqpoint{3.148038in}{2.660471in}}%
\pgfpathlineto{\pgfqpoint{3.151560in}{2.793013in}}%
\pgfpathlineto{\pgfqpoint{3.153310in}{2.861978in}}%
\pgfpathlineto{\pgfqpoint{3.155053in}{2.623633in}}%
\pgfpathlineto{\pgfqpoint{3.156788in}{2.767694in}}%
\pgfpathlineto{\pgfqpoint{3.158517in}{2.507324in}}%
\pgfpathlineto{\pgfqpoint{3.161953in}{2.865037in}}%
\pgfpathlineto{\pgfqpoint{3.165362in}{2.675195in}}%
\pgfpathlineto{\pgfqpoint{3.167056in}{2.879011in}}%
\pgfpathlineto{\pgfqpoint{3.168743in}{2.839272in}}%
\pgfpathlineto{\pgfqpoint{3.170424in}{2.715493in}}%
\pgfpathlineto{\pgfqpoint{3.172098in}{2.737882in}}%
\pgfpathlineto{\pgfqpoint{3.173765in}{2.604013in}}%
\pgfpathlineto{\pgfqpoint{3.175426in}{2.827861in}}%
\pgfpathlineto{\pgfqpoint{3.177080in}{2.493927in}}%
\pgfpathlineto{\pgfqpoint{3.178728in}{2.824553in}}%
\pgfpathlineto{\pgfqpoint{3.180370in}{2.849274in}}%
\pgfpathlineto{\pgfqpoint{3.182005in}{2.620338in}}%
\pgfpathlineto{\pgfqpoint{3.183634in}{2.905908in}}%
\pgfpathlineto{\pgfqpoint{3.185257in}{2.751743in}}%
\pgfpathlineto{\pgfqpoint{3.186874in}{2.464076in}}%
\pgfpathlineto{\pgfqpoint{3.188484in}{2.863118in}}%
\pgfpathlineto{\pgfqpoint{3.190089in}{2.683683in}}%
\pgfpathlineto{\pgfqpoint{3.191687in}{2.736440in}}%
\pgfpathlineto{\pgfqpoint{3.193279in}{2.649760in}}%
\pgfpathlineto{\pgfqpoint{3.194866in}{2.819185in}}%
\pgfpathlineto{\pgfqpoint{3.196446in}{2.702612in}}%
\pgfpathlineto{\pgfqpoint{3.198021in}{2.873757in}}%
\pgfpathlineto{\pgfqpoint{3.201153in}{2.619981in}}%
\pgfpathlineto{\pgfqpoint{3.202710in}{2.679782in}}%
\pgfpathlineto{\pgfqpoint{3.204262in}{2.683501in}}%
\pgfpathlineto{\pgfqpoint{3.205808in}{2.672250in}}%
\pgfpathlineto{\pgfqpoint{3.207348in}{2.648847in}}%
\pgfpathlineto{\pgfqpoint{3.208883in}{2.741206in}}%
\pgfpathlineto{\pgfqpoint{3.210412in}{2.539228in}}%
\pgfpathlineto{\pgfqpoint{3.211936in}{2.842510in}}%
\pgfpathlineto{\pgfqpoint{3.213454in}{2.755794in}}%
\pgfpathlineto{\pgfqpoint{3.214967in}{2.815538in}}%
\pgfpathlineto{\pgfqpoint{3.216474in}{2.763273in}}%
\pgfpathlineto{\pgfqpoint{3.217977in}{2.851291in}}%
\pgfpathlineto{\pgfqpoint{3.219473in}{2.828783in}}%
\pgfpathlineto{\pgfqpoint{3.220965in}{2.827093in}}%
\pgfpathlineto{\pgfqpoint{3.223933in}{2.799017in}}%
\pgfpathlineto{\pgfqpoint{3.226880in}{2.626576in}}%
\pgfpathlineto{\pgfqpoint{3.229806in}{2.840918in}}%
\pgfpathlineto{\pgfqpoint{3.231262in}{2.870647in}}%
\pgfpathlineto{\pgfqpoint{3.232713in}{2.715803in}}%
\pgfpathlineto{\pgfqpoint{3.234159in}{2.857200in}}%
\pgfpathlineto{\pgfqpoint{3.235600in}{2.829257in}}%
\pgfpathlineto{\pgfqpoint{3.237036in}{2.870240in}}%
\pgfpathlineto{\pgfqpoint{3.239894in}{2.687937in}}%
\pgfpathlineto{\pgfqpoint{3.241315in}{2.811893in}}%
\pgfpathlineto{\pgfqpoint{3.242732in}{2.739182in}}%
\pgfpathlineto{\pgfqpoint{3.244144in}{2.790430in}}%
\pgfpathlineto{\pgfqpoint{3.245552in}{2.597720in}}%
\pgfpathlineto{\pgfqpoint{3.246955in}{2.928043in}}%
\pgfpathlineto{\pgfqpoint{3.248353in}{2.666348in}}%
\pgfpathlineto{\pgfqpoint{3.249746in}{2.883423in}}%
\pgfpathlineto{\pgfqpoint{3.251136in}{2.911754in}}%
\pgfpathlineto{\pgfqpoint{3.253900in}{2.608802in}}%
\pgfpathlineto{\pgfqpoint{3.255276in}{2.819472in}}%
\pgfpathlineto{\pgfqpoint{3.258014in}{2.665016in}}%
\pgfpathlineto{\pgfqpoint{3.259376in}{2.725914in}}%
\pgfpathlineto{\pgfqpoint{3.260734in}{2.709648in}}%
\pgfpathlineto{\pgfqpoint{3.262087in}{2.897309in}}%
\pgfpathlineto{\pgfqpoint{3.263437in}{2.810227in}}%
\pgfpathlineto{\pgfqpoint{3.264782in}{2.852695in}}%
\pgfpathlineto{\pgfqpoint{3.266123in}{2.748110in}}%
\pgfpathlineto{\pgfqpoint{3.267459in}{2.825262in}}%
\pgfpathlineto{\pgfqpoint{3.268792in}{2.758464in}}%
\pgfpathlineto{\pgfqpoint{3.270120in}{2.863496in}}%
\pgfpathlineto{\pgfqpoint{3.271444in}{2.879517in}}%
\pgfpathlineto{\pgfqpoint{3.272764in}{2.773966in}}%
\pgfpathlineto{\pgfqpoint{3.274080in}{2.787431in}}%
\pgfpathlineto{\pgfqpoint{3.275391in}{2.627337in}}%
\pgfpathlineto{\pgfqpoint{3.276699in}{2.839974in}}%
\pgfpathlineto{\pgfqpoint{3.278003in}{2.841835in}}%
\pgfpathlineto{\pgfqpoint{3.281890in}{2.698617in}}%
\pgfpathlineto{\pgfqpoint{3.284462in}{2.840058in}}%
\pgfpathlineto{\pgfqpoint{3.285742in}{2.852384in}}%
\pgfpathlineto{\pgfqpoint{3.287019in}{2.782531in}}%
\pgfpathlineto{\pgfqpoint{3.288291in}{2.821401in}}%
\pgfpathlineto{\pgfqpoint{3.289560in}{2.806393in}}%
\pgfpathlineto{\pgfqpoint{3.290825in}{2.842967in}}%
\pgfpathlineto{\pgfqpoint{3.293343in}{2.701963in}}%
\pgfpathlineto{\pgfqpoint{3.294597in}{2.659258in}}%
\pgfpathlineto{\pgfqpoint{3.295847in}{2.697692in}}%
\pgfpathlineto{\pgfqpoint{3.298336in}{2.870732in}}%
\pgfpathlineto{\pgfqpoint{3.299575in}{2.555265in}}%
\pgfpathlineto{\pgfqpoint{3.302042in}{2.804843in}}%
\pgfpathlineto{\pgfqpoint{3.303270in}{2.648075in}}%
\pgfpathlineto{\pgfqpoint{3.304495in}{2.725080in}}%
\pgfpathlineto{\pgfqpoint{3.305716in}{2.553592in}}%
\pgfpathlineto{\pgfqpoint{3.306934in}{2.774168in}}%
\pgfpathlineto{\pgfqpoint{3.308148in}{2.779913in}}%
\pgfpathlineto{\pgfqpoint{3.309359in}{2.718594in}}%
\pgfpathlineto{\pgfqpoint{3.310566in}{2.738252in}}%
\pgfpathlineto{\pgfqpoint{3.311770in}{2.864875in}}%
\pgfpathlineto{\pgfqpoint{3.312970in}{2.609668in}}%
\pgfpathlineto{\pgfqpoint{3.315361in}{2.767343in}}%
\pgfpathlineto{\pgfqpoint{3.316551in}{2.790953in}}%
\pgfpathlineto{\pgfqpoint{3.317738in}{2.720514in}}%
\pgfpathlineto{\pgfqpoint{3.318922in}{2.728198in}}%
\pgfpathlineto{\pgfqpoint{3.321280in}{2.821737in}}%
\pgfpathlineto{\pgfqpoint{3.322454in}{2.799570in}}%
\pgfpathlineto{\pgfqpoint{3.323624in}{2.728080in}}%
\pgfpathlineto{\pgfqpoint{3.324792in}{2.741328in}}%
\pgfpathlineto{\pgfqpoint{3.325956in}{2.673762in}}%
\pgfpathlineto{\pgfqpoint{3.327117in}{2.793143in}}%
\pgfpathlineto{\pgfqpoint{3.328275in}{2.749870in}}%
\pgfpathlineto{\pgfqpoint{3.329430in}{2.759893in}}%
\pgfpathlineto{\pgfqpoint{3.330581in}{2.457827in}}%
\pgfpathlineto{\pgfqpoint{3.331730in}{2.732802in}}%
\pgfpathlineto{\pgfqpoint{3.332875in}{2.658051in}}%
\pgfpathlineto{\pgfqpoint{3.334017in}{2.681331in}}%
\pgfpathlineto{\pgfqpoint{3.335157in}{2.850222in}}%
\pgfpathlineto{\pgfqpoint{3.337426in}{2.731942in}}%
\pgfpathlineto{\pgfqpoint{3.338556in}{2.722272in}}%
\pgfpathlineto{\pgfqpoint{3.339683in}{2.580885in}}%
\pgfpathlineto{\pgfqpoint{3.341928in}{2.789049in}}%
\pgfpathlineto{\pgfqpoint{3.343047in}{2.550759in}}%
\pgfpathlineto{\pgfqpoint{3.346384in}{2.871438in}}%
\pgfpathlineto{\pgfqpoint{3.347490in}{2.657878in}}%
\pgfpathlineto{\pgfqpoint{3.350793in}{2.833898in}}%
\pgfpathlineto{\pgfqpoint{3.351888in}{2.817886in}}%
\pgfpathlineto{\pgfqpoint{3.352980in}{2.627222in}}%
\pgfpathlineto{\pgfqpoint{3.354070in}{2.694201in}}%
\pgfpathlineto{\pgfqpoint{3.355156in}{2.236113in}}%
\pgfpathlineto{\pgfqpoint{3.356240in}{2.819494in}}%
\pgfpathlineto{\pgfqpoint{3.357321in}{2.730350in}}%
\pgfpathlineto{\pgfqpoint{3.358400in}{2.544096in}}%
\pgfpathlineto{\pgfqpoint{3.359476in}{2.862975in}}%
\pgfpathlineto{\pgfqpoint{3.360549in}{2.782257in}}%
\pgfpathlineto{\pgfqpoint{3.361619in}{2.615780in}}%
\pgfpathlineto{\pgfqpoint{3.362686in}{2.722623in}}%
\pgfpathlineto{\pgfqpoint{3.363751in}{2.711543in}}%
\pgfpathlineto{\pgfqpoint{3.365873in}{2.506903in}}%
\pgfpathlineto{\pgfqpoint{3.367984in}{2.861962in}}%
\pgfpathlineto{\pgfqpoint{3.369036in}{2.645708in}}%
\pgfpathlineto{\pgfqpoint{3.370085in}{2.718029in}}%
\pgfpathlineto{\pgfqpoint{3.372176in}{2.544596in}}%
\pgfpathlineto{\pgfqpoint{3.373217in}{2.802770in}}%
\pgfpathlineto{\pgfqpoint{3.375292in}{2.617327in}}%
\pgfpathlineto{\pgfqpoint{3.376326in}{2.743324in}}%
\pgfpathlineto{\pgfqpoint{3.377357in}{2.682366in}}%
\pgfpathlineto{\pgfqpoint{3.378386in}{2.765156in}}%
\pgfpathlineto{\pgfqpoint{3.379412in}{2.556280in}}%
\pgfpathlineto{\pgfqpoint{3.381457in}{2.877673in}}%
\pgfpathlineto{\pgfqpoint{3.382476in}{2.802765in}}%
\pgfpathlineto{\pgfqpoint{3.383493in}{2.828279in}}%
\pgfpathlineto{\pgfqpoint{3.384507in}{2.757860in}}%
\pgfpathlineto{\pgfqpoint{3.385518in}{2.881751in}}%
\pgfpathlineto{\pgfqpoint{3.387534in}{2.805868in}}%
\pgfpathlineto{\pgfqpoint{3.388539in}{2.852510in}}%
\pgfpathlineto{\pgfqpoint{3.390540in}{2.763719in}}%
\pgfpathlineto{\pgfqpoint{3.391538in}{2.320584in}}%
\pgfpathlineto{\pgfqpoint{3.393525in}{2.846155in}}%
\pgfpathlineto{\pgfqpoint{3.395504in}{2.736062in}}%
\pgfpathlineto{\pgfqpoint{3.396490in}{2.711621in}}%
\pgfpathlineto{\pgfqpoint{3.398454in}{2.552094in}}%
\pgfpathlineto{\pgfqpoint{3.399433in}{2.703166in}}%
\pgfpathlineto{\pgfqpoint{3.400410in}{2.638859in}}%
\pgfpathlineto{\pgfqpoint{3.401384in}{2.743185in}}%
\pgfpathlineto{\pgfqpoint{3.402356in}{2.671028in}}%
\pgfpathlineto{\pgfqpoint{3.403326in}{2.690699in}}%
\pgfpathlineto{\pgfqpoint{3.404294in}{2.668569in}}%
\pgfpathlineto{\pgfqpoint{3.405260in}{2.731949in}}%
\pgfpathlineto{\pgfqpoint{3.406223in}{2.687325in}}%
\pgfpathlineto{\pgfqpoint{3.407184in}{2.807822in}}%
\pgfpathlineto{\pgfqpoint{3.408143in}{2.791960in}}%
\pgfpathlineto{\pgfqpoint{3.409100in}{2.603872in}}%
\pgfpathlineto{\pgfqpoint{3.410055in}{2.821154in}}%
\pgfpathlineto{\pgfqpoint{3.411007in}{2.640823in}}%
\pgfpathlineto{\pgfqpoint{3.411958in}{2.758365in}}%
\pgfpathlineto{\pgfqpoint{3.412906in}{2.589682in}}%
\pgfpathlineto{\pgfqpoint{3.413852in}{2.809432in}}%
\pgfpathlineto{\pgfqpoint{3.414796in}{2.708976in}}%
\pgfpathlineto{\pgfqpoint{3.415738in}{2.750712in}}%
\pgfpathlineto{\pgfqpoint{3.416678in}{2.466242in}}%
\pgfpathlineto{\pgfqpoint{3.418552in}{2.778178in}}%
\pgfpathlineto{\pgfqpoint{3.420417in}{2.688741in}}%
\pgfpathlineto{\pgfqpoint{3.421347in}{2.720845in}}%
\pgfpathlineto{\pgfqpoint{3.422274in}{2.606601in}}%
\pgfpathlineto{\pgfqpoint{3.423200in}{2.883169in}}%
\pgfpathlineto{\pgfqpoint{3.424123in}{2.751489in}}%
\pgfpathlineto{\pgfqpoint{3.425045in}{2.880617in}}%
\pgfpathlineto{\pgfqpoint{3.425964in}{2.774099in}}%
\pgfpathlineto{\pgfqpoint{3.426882in}{2.827951in}}%
\pgfpathlineto{\pgfqpoint{3.428711in}{2.738670in}}%
\pgfpathlineto{\pgfqpoint{3.429623in}{2.285572in}}%
\pgfpathlineto{\pgfqpoint{3.431440in}{2.778376in}}%
\pgfpathlineto{\pgfqpoint{3.432346in}{2.816277in}}%
\pgfpathlineto{\pgfqpoint{3.433250in}{2.699949in}}%
\pgfpathlineto{\pgfqpoint{3.434152in}{2.757417in}}%
\pgfpathlineto{\pgfqpoint{3.435052in}{2.711066in}}%
\pgfpathlineto{\pgfqpoint{3.436846in}{2.834899in}}%
\pgfpathlineto{\pgfqpoint{3.439523in}{2.617918in}}%
\pgfpathlineto{\pgfqpoint{3.440412in}{2.718024in}}%
\pgfpathlineto{\pgfqpoint{3.442184in}{2.533597in}}%
\pgfpathlineto{\pgfqpoint{3.443067in}{2.844087in}}%
\pgfpathlineto{\pgfqpoint{3.443949in}{2.703940in}}%
\pgfpathlineto{\pgfqpoint{3.444828in}{2.762614in}}%
\pgfpathlineto{\pgfqpoint{3.445706in}{2.627417in}}%
\pgfpathlineto{\pgfqpoint{3.446582in}{2.822574in}}%
\pgfpathlineto{\pgfqpoint{3.447456in}{2.697239in}}%
\pgfpathlineto{\pgfqpoint{3.448328in}{2.744798in}}%
\pgfpathlineto{\pgfqpoint{3.449198in}{2.672749in}}%
\pgfpathlineto{\pgfqpoint{3.450067in}{2.432073in}}%
\pgfpathlineto{\pgfqpoint{3.450934in}{2.735700in}}%
\pgfpathlineto{\pgfqpoint{3.451799in}{2.689433in}}%
\pgfpathlineto{\pgfqpoint{3.452663in}{2.598573in}}%
\pgfpathlineto{\pgfqpoint{3.453524in}{2.652072in}}%
\pgfpathlineto{\pgfqpoint{3.454384in}{2.355759in}}%
\pgfpathlineto{\pgfqpoint{3.456099in}{2.692273in}}%
\pgfpathlineto{\pgfqpoint{3.458658in}{2.861199in}}%
\pgfpathlineto{\pgfqpoint{3.461201in}{2.612765in}}%
\pgfpathlineto{\pgfqpoint{3.462046in}{2.729858in}}%
\pgfpathlineto{\pgfqpoint{3.462889in}{2.624682in}}%
\pgfpathlineto{\pgfqpoint{3.464569in}{2.855657in}}%
\pgfpathlineto{\pgfqpoint{3.465407in}{2.854799in}}%
\pgfpathlineto{\pgfqpoint{3.466243in}{2.743660in}}%
\pgfpathlineto{\pgfqpoint{3.467078in}{2.773535in}}%
\pgfpathlineto{\pgfqpoint{3.467911in}{2.797525in}}%
\pgfpathlineto{\pgfqpoint{3.468742in}{2.773968in}}%
\pgfpathlineto{\pgfqpoint{3.469572in}{2.794059in}}%
\pgfpathlineto{\pgfqpoint{3.470400in}{2.681364in}}%
\pgfpathlineto{\pgfqpoint{3.471226in}{2.702959in}}%
\pgfpathlineto{\pgfqpoint{3.472051in}{2.795234in}}%
\pgfpathlineto{\pgfqpoint{3.472874in}{2.603256in}}%
\pgfpathlineto{\pgfqpoint{3.473696in}{2.730549in}}%
\pgfpathlineto{\pgfqpoint{3.474516in}{2.707256in}}%
\pgfpathlineto{\pgfqpoint{3.475334in}{2.689823in}}%
\pgfpathlineto{\pgfqpoint{3.476151in}{2.838947in}}%
\pgfpathlineto{\pgfqpoint{3.477780in}{2.540261in}}%
\pgfpathlineto{\pgfqpoint{3.479403in}{2.734050in}}%
\pgfpathlineto{\pgfqpoint{3.480212in}{2.685344in}}%
\pgfpathlineto{\pgfqpoint{3.481020in}{2.857217in}}%
\pgfpathlineto{\pgfqpoint{3.481826in}{2.816036in}}%
\pgfpathlineto{\pgfqpoint{3.482630in}{2.779887in}}%
\pgfpathlineto{\pgfqpoint{3.483433in}{2.590280in}}%
\pgfpathlineto{\pgfqpoint{3.485034in}{2.838269in}}%
\pgfpathlineto{\pgfqpoint{3.485833in}{2.716813in}}%
\pgfpathlineto{\pgfqpoint{3.486630in}{2.819422in}}%
\pgfpathlineto{\pgfqpoint{3.487425in}{2.619430in}}%
\pgfpathlineto{\pgfqpoint{3.489012in}{2.815911in}}%
\pgfpathlineto{\pgfqpoint{3.489803in}{2.831752in}}%
\pgfpathlineto{\pgfqpoint{3.492167in}{2.460795in}}%
\pgfpathlineto{\pgfqpoint{3.492952in}{2.682787in}}%
\pgfpathlineto{\pgfqpoint{3.493736in}{2.530350in}}%
\pgfpathlineto{\pgfqpoint{3.495299in}{2.836160in}}%
\pgfpathlineto{\pgfqpoint{3.496856in}{2.737439in}}%
\pgfpathlineto{\pgfqpoint{3.497632in}{2.790909in}}%
\pgfpathlineto{\pgfqpoint{3.498407in}{2.715853in}}%
\pgfpathlineto{\pgfqpoint{3.499181in}{2.763279in}}%
\pgfpathlineto{\pgfqpoint{3.500724in}{2.580267in}}%
\pgfpathlineto{\pgfqpoint{3.501494in}{2.635490in}}%
\pgfpathlineto{\pgfqpoint{3.502262in}{2.778984in}}%
\pgfpathlineto{\pgfqpoint{3.503029in}{2.690838in}}%
\pgfpathlineto{\pgfqpoint{3.503794in}{2.758642in}}%
\pgfpathlineto{\pgfqpoint{3.504558in}{2.599806in}}%
\pgfpathlineto{\pgfqpoint{3.506082in}{2.848306in}}%
\pgfpathlineto{\pgfqpoint{3.507600in}{2.590213in}}%
\pgfpathlineto{\pgfqpoint{3.508357in}{2.832713in}}%
\pgfpathlineto{\pgfqpoint{3.509113in}{2.792866in}}%
\pgfpathlineto{\pgfqpoint{3.509867in}{2.741548in}}%
\pgfpathlineto{\pgfqpoint{3.510620in}{2.822952in}}%
\pgfpathlineto{\pgfqpoint{3.511372in}{2.786791in}}%
\pgfpathlineto{\pgfqpoint{3.512122in}{2.587840in}}%
\pgfpathlineto{\pgfqpoint{3.512871in}{2.830624in}}%
\pgfpathlineto{\pgfqpoint{3.514366in}{2.512614in}}%
\pgfpathlineto{\pgfqpoint{3.515855in}{2.789665in}}%
\pgfpathlineto{\pgfqpoint{3.516597in}{2.579333in}}%
\pgfpathlineto{\pgfqpoint{3.517338in}{2.616403in}}%
\pgfpathlineto{\pgfqpoint{3.518078in}{2.734329in}}%
\pgfpathlineto{\pgfqpoint{3.518817in}{2.625756in}}%
\pgfpathlineto{\pgfqpoint{3.519554in}{2.809656in}}%
\pgfpathlineto{\pgfqpoint{3.520291in}{2.627351in}}%
\pgfpathlineto{\pgfqpoint{3.521025in}{2.663631in}}%
\pgfpathlineto{\pgfqpoint{3.521759in}{2.713769in}}%
\pgfpathlineto{\pgfqpoint{3.522491in}{2.700425in}}%
\pgfpathlineto{\pgfqpoint{3.523952in}{2.718435in}}%
\pgfpathlineto{\pgfqpoint{3.524681in}{2.625616in}}%
\pgfpathlineto{\pgfqpoint{3.526134in}{2.847655in}}%
\pgfpathlineto{\pgfqpoint{3.526859in}{2.837260in}}%
\pgfpathlineto{\pgfqpoint{3.529026in}{2.714037in}}%
\pgfpathlineto{\pgfqpoint{3.529746in}{2.712057in}}%
\pgfpathlineto{\pgfqpoint{3.530464in}{2.645372in}}%
\pgfpathlineto{\pgfqpoint{3.531898in}{2.828810in}}%
\pgfpathlineto{\pgfqpoint{3.532613in}{2.661809in}}%
\pgfpathlineto{\pgfqpoint{3.533327in}{2.734698in}}%
\pgfpathlineto{\pgfqpoint{3.534039in}{2.817115in}}%
\pgfpathlineto{\pgfqpoint{3.535461in}{2.689721in}}%
\pgfpathlineto{\pgfqpoint{3.536170in}{2.811330in}}%
\pgfpathlineto{\pgfqpoint{3.536878in}{2.730486in}}%
\pgfpathlineto{\pgfqpoint{3.537585in}{2.792744in}}%
\pgfpathlineto{\pgfqpoint{3.538290in}{2.699969in}}%
\pgfpathlineto{\pgfqpoint{3.538994in}{2.785446in}}%
\pgfpathlineto{\pgfqpoint{3.539698in}{2.601871in}}%
\pgfpathlineto{\pgfqpoint{3.540400in}{2.809850in}}%
\pgfpathlineto{\pgfqpoint{3.541100in}{2.695843in}}%
\pgfpathlineto{\pgfqpoint{3.541800in}{2.583444in}}%
\pgfpathlineto{\pgfqpoint{3.542499in}{2.632652in}}%
\pgfpathlineto{\pgfqpoint{3.543196in}{2.624036in}}%
\pgfpathlineto{\pgfqpoint{3.545281in}{2.912618in}}%
\pgfpathlineto{\pgfqpoint{3.546666in}{2.758922in}}%
\pgfpathlineto{\pgfqpoint{3.548046in}{2.726371in}}%
\pgfpathlineto{\pgfqpoint{3.548734in}{2.827330in}}%
\pgfpathlineto{\pgfqpoint{3.549421in}{2.773588in}}%
\pgfpathlineto{\pgfqpoint{3.550108in}{2.646579in}}%
\pgfpathlineto{\pgfqpoint{3.551476in}{2.831489in}}%
\pgfpathlineto{\pgfqpoint{3.552159in}{2.694875in}}%
\pgfpathlineto{\pgfqpoint{3.552841in}{2.768208in}}%
\pgfpathlineto{\pgfqpoint{3.553522in}{2.755821in}}%
\pgfpathlineto{\pgfqpoint{3.554201in}{2.811175in}}%
\pgfpathlineto{\pgfqpoint{3.554880in}{2.670850in}}%
\pgfpathlineto{\pgfqpoint{3.555557in}{2.791809in}}%
\pgfpathlineto{\pgfqpoint{3.556233in}{2.664918in}}%
\pgfpathlineto{\pgfqpoint{3.556908in}{2.734649in}}%
\pgfpathlineto{\pgfqpoint{3.557583in}{2.842580in}}%
\pgfpathlineto{\pgfqpoint{3.558256in}{2.784218in}}%
\pgfpathlineto{\pgfqpoint{3.558928in}{2.782965in}}%
\pgfpathlineto{\pgfqpoint{3.560937in}{2.538712in}}%
\pgfpathlineto{\pgfqpoint{3.561605in}{2.843709in}}%
\pgfpathlineto{\pgfqpoint{3.562272in}{2.689817in}}%
\pgfpathlineto{\pgfqpoint{3.562937in}{2.617699in}}%
\pgfpathlineto{\pgfqpoint{3.564266in}{2.725872in}}%
\pgfpathlineto{\pgfqpoint{3.564928in}{2.696295in}}%
\pgfpathlineto{\pgfqpoint{3.565590in}{2.806257in}}%
\pgfpathlineto{\pgfqpoint{3.566250in}{2.708085in}}%
\pgfpathlineto{\pgfqpoint{3.566910in}{2.844860in}}%
\pgfpathlineto{\pgfqpoint{3.567568in}{2.746482in}}%
\pgfpathlineto{\pgfqpoint{3.568225in}{2.804649in}}%
\pgfpathlineto{\pgfqpoint{3.568882in}{2.788996in}}%
\pgfpathlineto{\pgfqpoint{3.569537in}{2.591814in}}%
\pgfpathlineto{\pgfqpoint{3.570192in}{2.712095in}}%
\pgfpathlineto{\pgfqpoint{3.570845in}{2.692520in}}%
\pgfpathlineto{\pgfqpoint{3.571497in}{2.768084in}}%
\pgfpathlineto{\pgfqpoint{3.572149in}{2.731865in}}%
\pgfpathlineto{\pgfqpoint{3.572799in}{2.736754in}}%
\pgfpathlineto{\pgfqpoint{3.574097in}{2.667872in}}%
\pgfpathlineto{\pgfqpoint{3.574744in}{2.770251in}}%
\pgfpathlineto{\pgfqpoint{3.575391in}{2.752221in}}%
\pgfpathlineto{\pgfqpoint{3.576036in}{2.636240in}}%
\pgfpathlineto{\pgfqpoint{3.576680in}{2.768955in}}%
\pgfpathlineto{\pgfqpoint{3.577324in}{2.750436in}}%
\pgfpathlineto{\pgfqpoint{3.577966in}{2.755817in}}%
\pgfpathlineto{\pgfqpoint{3.578608in}{2.697809in}}%
\pgfpathlineto{\pgfqpoint{3.579248in}{2.793527in}}%
\pgfpathlineto{\pgfqpoint{3.579888in}{2.756049in}}%
\pgfpathlineto{\pgfqpoint{3.580527in}{2.357503in}}%
\pgfpathlineto{\pgfqpoint{3.581164in}{2.586224in}}%
\pgfpathlineto{\pgfqpoint{3.583072in}{2.778312in}}%
\pgfpathlineto{\pgfqpoint{3.583705in}{2.626121in}}%
\pgfpathlineto{\pgfqpoint{3.584338in}{2.691287in}}%
\pgfpathlineto{\pgfqpoint{3.584970in}{2.808847in}}%
\pgfpathlineto{\pgfqpoint{3.585601in}{2.728964in}}%
\pgfpathlineto{\pgfqpoint{3.586231in}{2.766312in}}%
\pgfpathlineto{\pgfqpoint{3.586861in}{2.723465in}}%
\pgfpathlineto{\pgfqpoint{3.587489in}{2.810437in}}%
\pgfpathlineto{\pgfqpoint{3.588116in}{2.391982in}}%
\pgfpathlineto{\pgfqpoint{3.588742in}{2.836750in}}%
\pgfpathlineto{\pgfqpoint{3.589368in}{2.762130in}}%
\pgfpathlineto{\pgfqpoint{3.589992in}{2.692869in}}%
\pgfpathlineto{\pgfqpoint{3.590616in}{2.769242in}}%
\pgfpathlineto{\pgfqpoint{3.591860in}{2.668361in}}%
\pgfpathlineto{\pgfqpoint{3.592481in}{2.747625in}}%
\pgfpathlineto{\pgfqpoint{3.593101in}{2.591090in}}%
\pgfpathlineto{\pgfqpoint{3.593720in}{2.666641in}}%
\pgfpathlineto{\pgfqpoint{3.594338in}{2.743820in}}%
\pgfpathlineto{\pgfqpoint{3.594956in}{2.658769in}}%
\pgfpathlineto{\pgfqpoint{3.596188in}{2.736737in}}%
\pgfpathlineto{\pgfqpoint{3.597416in}{2.592854in}}%
\pgfpathlineto{\pgfqpoint{3.598029in}{2.695801in}}%
\pgfpathlineto{\pgfqpoint{3.598641in}{2.612245in}}%
\pgfpathlineto{\pgfqpoint{3.599252in}{2.685388in}}%
\pgfpathlineto{\pgfqpoint{3.599862in}{2.508672in}}%
\pgfpathlineto{\pgfqpoint{3.601687in}{2.856875in}}%
\pgfpathlineto{\pgfqpoint{3.603504in}{2.715224in}}%
\pgfpathlineto{\pgfqpoint{3.604712in}{2.745591in}}%
\pgfpathlineto{\pgfqpoint{3.605915in}{2.579918in}}%
\pgfpathlineto{\pgfqpoint{3.607116in}{2.771909in}}%
\pgfpathlineto{\pgfqpoint{3.607715in}{2.762811in}}%
\pgfpathlineto{\pgfqpoint{3.608313in}{2.646752in}}%
\pgfpathlineto{\pgfqpoint{3.608910in}{2.693802in}}%
\pgfpathlineto{\pgfqpoint{3.609507in}{2.853825in}}%
\pgfpathlineto{\pgfqpoint{3.610697in}{2.659079in}}%
\pgfpathlineto{\pgfqpoint{3.611291in}{2.789126in}}%
\pgfpathlineto{\pgfqpoint{3.611884in}{2.720356in}}%
\pgfpathlineto{\pgfqpoint{3.612476in}{2.642043in}}%
\pgfpathlineto{\pgfqpoint{3.613068in}{2.760035in}}%
\pgfpathlineto{\pgfqpoint{3.613658in}{2.599631in}}%
\pgfpathlineto{\pgfqpoint{3.614248in}{2.644440in}}%
\pgfpathlineto{\pgfqpoint{3.614837in}{2.604004in}}%
\pgfpathlineto{\pgfqpoint{3.615425in}{2.781670in}}%
\pgfpathlineto{\pgfqpoint{3.616013in}{2.685740in}}%
\pgfpathlineto{\pgfqpoint{3.616599in}{2.727988in}}%
\pgfpathlineto{\pgfqpoint{3.617185in}{2.417425in}}%
\pgfpathlineto{\pgfqpoint{3.618354in}{2.825822in}}%
\pgfpathlineto{\pgfqpoint{3.620102in}{2.642007in}}%
\pgfpathlineto{\pgfqpoint{3.622421in}{2.778885in}}%
\pgfpathlineto{\pgfqpoint{3.623575in}{2.682736in}}%
\pgfpathlineto{\pgfqpoint{3.624727in}{2.692384in}}%
\pgfpathlineto{\pgfqpoint{3.625302in}{2.686467in}}%
\pgfpathlineto{\pgfqpoint{3.625875in}{2.502248in}}%
\pgfpathlineto{\pgfqpoint{3.627592in}{2.775772in}}%
\pgfpathlineto{\pgfqpoint{3.628163in}{2.586816in}}%
\pgfpathlineto{\pgfqpoint{3.628733in}{2.643716in}}%
\pgfpathlineto{\pgfqpoint{3.629302in}{2.637325in}}%
\pgfpathlineto{\pgfqpoint{3.629871in}{2.650347in}}%
\pgfpathlineto{\pgfqpoint{3.631005in}{2.779429in}}%
\pgfpathlineto{\pgfqpoint{3.631572in}{2.683514in}}%
\pgfpathlineto{\pgfqpoint{3.632702in}{2.850329in}}%
\pgfpathlineto{\pgfqpoint{3.633266in}{2.499849in}}%
\pgfpathlineto{\pgfqpoint{3.633829in}{2.861322in}}%
\pgfpathlineto{\pgfqpoint{3.634391in}{2.793754in}}%
\pgfpathlineto{\pgfqpoint{3.634953in}{2.721570in}}%
\pgfpathlineto{\pgfqpoint{3.635514in}{2.808195in}}%
\pgfpathlineto{\pgfqpoint{3.636074in}{2.657310in}}%
\pgfpathlineto{\pgfqpoint{3.636634in}{2.716040in}}%
\pgfpathlineto{\pgfqpoint{3.637192in}{2.692140in}}%
\pgfpathlineto{\pgfqpoint{3.637750in}{2.703985in}}%
\pgfpathlineto{\pgfqpoint{3.639420in}{2.795713in}}%
\pgfpathlineto{\pgfqpoint{3.640529in}{2.697409in}}%
\pgfpathlineto{\pgfqpoint{3.641083in}{2.805424in}}%
\pgfpathlineto{\pgfqpoint{3.641636in}{2.707658in}}%
\pgfpathlineto{\pgfqpoint{3.642188in}{2.725786in}}%
\pgfpathlineto{\pgfqpoint{3.642740in}{2.536453in}}%
\pgfpathlineto{\pgfqpoint{3.643290in}{2.708177in}}%
\pgfpathlineto{\pgfqpoint{3.643840in}{2.698239in}}%
\pgfpathlineto{\pgfqpoint{3.644390in}{2.770388in}}%
\pgfpathlineto{\pgfqpoint{3.644938in}{2.721507in}}%
\pgfpathlineto{\pgfqpoint{3.645486in}{2.732466in}}%
\pgfpathlineto{\pgfqpoint{3.646580in}{2.668020in}}%
\pgfpathlineto{\pgfqpoint{3.647126in}{2.778972in}}%
\pgfpathlineto{\pgfqpoint{3.648215in}{2.505394in}}%
\pgfpathlineto{\pgfqpoint{3.649844in}{2.798124in}}%
\pgfpathlineto{\pgfqpoint{3.652007in}{2.637175in}}%
\pgfpathlineto{\pgfqpoint{3.652546in}{2.657833in}}%
\pgfpathlineto{\pgfqpoint{3.653084in}{2.640140in}}%
\pgfpathlineto{\pgfqpoint{3.653621in}{2.837309in}}%
\pgfpathlineto{\pgfqpoint{3.654158in}{2.769651in}}%
\pgfpathlineto{\pgfqpoint{3.654694in}{2.774866in}}%
\pgfpathlineto{\pgfqpoint{3.655230in}{2.548876in}}%
\pgfpathlineto{\pgfqpoint{3.655765in}{2.759216in}}%
\pgfpathlineto{\pgfqpoint{3.657365in}{2.511934in}}%
\pgfpathlineto{\pgfqpoint{3.658429in}{2.773218in}}%
\pgfpathlineto{\pgfqpoint{3.658959in}{2.761492in}}%
\pgfpathlineto{\pgfqpoint{3.659489in}{2.757071in}}%
\pgfpathlineto{\pgfqpoint{3.660019in}{2.608961in}}%
\pgfpathlineto{\pgfqpoint{3.660548in}{2.636176in}}%
\pgfpathlineto{\pgfqpoint{3.661603in}{2.695148in}}%
\pgfpathlineto{\pgfqpoint{3.662130in}{2.607161in}}%
\pgfpathlineto{\pgfqpoint{3.662656in}{2.699736in}}%
\pgfpathlineto{\pgfqpoint{3.663707in}{2.584913in}}%
\pgfpathlineto{\pgfqpoint{3.664231in}{2.727054in}}%
\pgfpathlineto{\pgfqpoint{3.664755in}{2.706620in}}%
\pgfpathlineto{\pgfqpoint{3.665278in}{2.560973in}}%
\pgfpathlineto{\pgfqpoint{3.665800in}{2.764471in}}%
\pgfpathlineto{\pgfqpoint{3.666321in}{2.656554in}}%
\pgfpathlineto{\pgfqpoint{3.666842in}{2.686773in}}%
\pgfpathlineto{\pgfqpoint{3.667363in}{2.663986in}}%
\pgfpathlineto{\pgfqpoint{3.668402in}{2.497921in}}%
\pgfpathlineto{\pgfqpoint{3.668920in}{2.756370in}}%
\pgfpathlineto{\pgfqpoint{3.669438in}{2.699643in}}%
\pgfpathlineto{\pgfqpoint{3.669955in}{2.701561in}}%
\pgfpathlineto{\pgfqpoint{3.670472in}{2.723376in}}%
\pgfpathlineto{\pgfqpoint{3.670988in}{2.703322in}}%
\pgfpathlineto{\pgfqpoint{3.671503in}{2.469387in}}%
\pgfpathlineto{\pgfqpoint{3.672018in}{2.705101in}}%
\pgfpathlineto{\pgfqpoint{3.673045in}{2.607655in}}%
\pgfpathlineto{\pgfqpoint{3.673558in}{2.675988in}}%
\pgfpathlineto{\pgfqpoint{3.674070in}{2.724040in}}%
\pgfpathlineto{\pgfqpoint{3.674582in}{2.536179in}}%
\pgfpathlineto{\pgfqpoint{3.675093in}{2.719042in}}%
\pgfpathlineto{\pgfqpoint{3.677131in}{2.595384in}}%
\pgfpathlineto{\pgfqpoint{3.678652in}{2.826063in}}%
\pgfpathlineto{\pgfqpoint{3.679159in}{2.747689in}}%
\pgfpathlineto{\pgfqpoint{3.679664in}{2.418039in}}%
\pgfpathlineto{\pgfqpoint{3.680673in}{2.434479in}}%
\pgfpathlineto{\pgfqpoint{3.681680in}{2.700557in}}%
\pgfpathlineto{\pgfqpoint{3.682684in}{2.719333in}}%
\pgfpathlineto{\pgfqpoint{3.683186in}{2.459317in}}%
\pgfpathlineto{\pgfqpoint{3.684187in}{2.779015in}}%
\pgfpathlineto{\pgfqpoint{3.684686in}{2.612403in}}%
\pgfpathlineto{\pgfqpoint{3.685683in}{2.729892in}}%
\pgfpathlineto{\pgfqpoint{3.686181in}{2.700453in}}%
\pgfpathlineto{\pgfqpoint{3.686678in}{2.631407in}}%
\pgfpathlineto{\pgfqpoint{3.688167in}{2.826154in}}%
\pgfpathlineto{\pgfqpoint{3.689156in}{2.585794in}}%
\pgfpathlineto{\pgfqpoint{3.690143in}{2.793444in}}%
\pgfpathlineto{\pgfqpoint{3.690635in}{2.535467in}}%
\pgfpathlineto{\pgfqpoint{3.691127in}{2.707908in}}%
\pgfpathlineto{\pgfqpoint{3.691619in}{2.604029in}}%
\pgfpathlineto{\pgfqpoint{3.692110in}{2.750544in}}%
\pgfpathlineto{\pgfqpoint{3.692600in}{2.745358in}}%
\pgfpathlineto{\pgfqpoint{3.693090in}{2.435421in}}%
\pgfpathlineto{\pgfqpoint{3.693579in}{2.795621in}}%
\pgfpathlineto{\pgfqpoint{3.695530in}{2.691877in}}%
\pgfpathlineto{\pgfqpoint{3.696016in}{2.753861in}}%
\pgfpathlineto{\pgfqpoint{3.696502in}{2.557393in}}%
\pgfpathlineto{\pgfqpoint{3.696987in}{2.581812in}}%
\pgfpathlineto{\pgfqpoint{3.698440in}{2.763696in}}%
\pgfpathlineto{\pgfqpoint{3.699887in}{2.579760in}}%
\pgfpathlineto{\pgfqpoint{3.700369in}{2.594782in}}%
\pgfpathlineto{\pgfqpoint{3.701330in}{2.741402in}}%
\pgfpathlineto{\pgfqpoint{3.702289in}{2.528075in}}%
\pgfpathlineto{\pgfqpoint{3.703723in}{2.745942in}}%
\pgfpathlineto{\pgfqpoint{3.704201in}{2.691467in}}%
\pgfpathlineto{\pgfqpoint{3.704677in}{2.769847in}}%
\pgfpathlineto{\pgfqpoint{3.705629in}{2.749977in}}%
\pgfpathlineto{\pgfqpoint{3.707525in}{2.567148in}}%
\pgfpathlineto{\pgfqpoint{3.708470in}{2.714382in}}%
\pgfpathlineto{\pgfqpoint{3.708942in}{2.691282in}}%
\pgfpathlineto{\pgfqpoint{3.709413in}{2.704587in}}%
\pgfpathlineto{\pgfqpoint{3.710354in}{2.396543in}}%
\pgfpathlineto{\pgfqpoint{3.711762in}{2.773187in}}%
\pgfpathlineto{\pgfqpoint{3.712230in}{2.419242in}}%
\pgfpathlineto{\pgfqpoint{3.712698in}{2.623913in}}%
\pgfpathlineto{\pgfqpoint{3.713165in}{2.697182in}}%
\pgfpathlineto{\pgfqpoint{3.713631in}{2.681174in}}%
\pgfpathlineto{\pgfqpoint{3.714097in}{2.643234in}}%
\pgfpathlineto{\pgfqpoint{3.714563in}{2.471765in}}%
\pgfpathlineto{\pgfqpoint{3.715028in}{2.657856in}}%
\pgfpathlineto{\pgfqpoint{3.715492in}{2.658314in}}%
\pgfpathlineto{\pgfqpoint{3.716420in}{2.805727in}}%
\pgfpathlineto{\pgfqpoint{3.716883in}{2.604544in}}%
\pgfpathlineto{\pgfqpoint{3.717346in}{2.655086in}}%
\pgfpathlineto{\pgfqpoint{3.717808in}{2.725853in}}%
\pgfpathlineto{\pgfqpoint{3.718269in}{2.504673in}}%
\pgfpathlineto{\pgfqpoint{3.718730in}{2.688847in}}%
\pgfpathlineto{\pgfqpoint{3.719191in}{2.673716in}}%
\pgfpathlineto{\pgfqpoint{3.720110in}{2.426854in}}%
\pgfpathlineto{\pgfqpoint{3.721028in}{2.810398in}}%
\pgfpathlineto{\pgfqpoint{3.721486in}{2.699099in}}%
\pgfpathlineto{\pgfqpoint{3.721943in}{2.744469in}}%
\pgfpathlineto{\pgfqpoint{3.722400in}{2.731111in}}%
\pgfpathlineto{\pgfqpoint{3.722857in}{2.701953in}}%
\pgfpathlineto{\pgfqpoint{3.723313in}{2.496890in}}%
\pgfpathlineto{\pgfqpoint{3.723768in}{2.668087in}}%
\pgfpathlineto{\pgfqpoint{3.725132in}{2.802938in}}%
\pgfpathlineto{\pgfqpoint{3.725586in}{2.752343in}}%
\pgfpathlineto{\pgfqpoint{3.726039in}{2.837942in}}%
\pgfpathlineto{\pgfqpoint{3.726492in}{2.739509in}}%
\pgfpathlineto{\pgfqpoint{3.727847in}{2.529265in}}%
\pgfpathlineto{\pgfqpoint{3.728297in}{2.623466in}}%
\pgfpathlineto{\pgfqpoint{3.728748in}{2.821924in}}%
\pgfpathlineto{\pgfqpoint{3.729197in}{2.703819in}}%
\pgfpathlineto{\pgfqpoint{3.729647in}{2.691851in}}%
\pgfpathlineto{\pgfqpoint{3.730096in}{2.803254in}}%
\pgfpathlineto{\pgfqpoint{3.730544in}{2.656284in}}%
\pgfpathlineto{\pgfqpoint{3.730992in}{2.746243in}}%
\pgfpathlineto{\pgfqpoint{3.731886in}{2.734351in}}%
\pgfpathlineto{\pgfqpoint{3.732333in}{2.739824in}}%
\pgfpathlineto{\pgfqpoint{3.732779in}{2.667594in}}%
\pgfpathlineto{\pgfqpoint{3.733224in}{2.404863in}}%
\pgfpathlineto{\pgfqpoint{3.733669in}{2.725342in}}%
\pgfpathlineto{\pgfqpoint{3.734558in}{2.542513in}}%
\pgfpathlineto{\pgfqpoint{3.735445in}{2.546493in}}%
\pgfpathlineto{\pgfqpoint{3.736330in}{2.698039in}}%
\pgfpathlineto{\pgfqpoint{3.737213in}{2.708297in}}%
\pgfpathlineto{\pgfqpoint{3.737654in}{2.508015in}}%
\pgfpathlineto{\pgfqpoint{3.738974in}{2.753163in}}%
\pgfpathlineto{\pgfqpoint{3.739413in}{2.658047in}}%
\pgfpathlineto{\pgfqpoint{3.739851in}{2.717942in}}%
\pgfpathlineto{\pgfqpoint{3.740290in}{2.751668in}}%
\pgfpathlineto{\pgfqpoint{3.741165in}{2.550013in}}%
\pgfpathlineto{\pgfqpoint{3.741601in}{2.676720in}}%
\pgfpathlineto{\pgfqpoint{3.742038in}{2.720921in}}%
\pgfpathlineto{\pgfqpoint{3.742909in}{2.582528in}}%
\pgfpathlineto{\pgfqpoint{3.743779in}{2.758062in}}%
\pgfpathlineto{\pgfqpoint{3.744213in}{2.744261in}}%
\pgfpathlineto{\pgfqpoint{3.744647in}{2.554795in}}%
\pgfpathlineto{\pgfqpoint{3.745513in}{2.658223in}}%
\pgfpathlineto{\pgfqpoint{3.745945in}{2.533654in}}%
\pgfpathlineto{\pgfqpoint{3.746377in}{2.630346in}}%
\pgfpathlineto{\pgfqpoint{3.746808in}{2.783920in}}%
\pgfpathlineto{\pgfqpoint{3.747239in}{2.580181in}}%
\pgfpathlineto{\pgfqpoint{3.747670in}{2.396743in}}%
\pgfpathlineto{\pgfqpoint{3.748100in}{2.580626in}}%
\pgfpathlineto{\pgfqpoint{3.748959in}{2.779217in}}%
\pgfpathlineto{\pgfqpoint{3.749388in}{2.587008in}}%
\pgfpathlineto{\pgfqpoint{3.749817in}{2.680598in}}%
\pgfpathlineto{\pgfqpoint{3.750245in}{2.799612in}}%
\pgfpathlineto{\pgfqpoint{3.750672in}{2.601708in}}%
\pgfpathlineto{\pgfqpoint{3.751099in}{2.726344in}}%
\pgfpathlineto{\pgfqpoint{3.751526in}{2.740490in}}%
\pgfpathlineto{\pgfqpoint{3.753229in}{2.465519in}}%
\pgfpathlineto{\pgfqpoint{3.754077in}{2.803793in}}%
\pgfpathlineto{\pgfqpoint{3.754501in}{2.568195in}}%
\pgfpathlineto{\pgfqpoint{3.756192in}{2.769240in}}%
\pgfpathlineto{\pgfqpoint{3.757455in}{2.335450in}}%
\pgfpathlineto{\pgfqpoint{3.758296in}{2.799930in}}%
\pgfpathlineto{\pgfqpoint{3.758715in}{2.584710in}}%
\pgfpathlineto{\pgfqpoint{3.759553in}{2.734177in}}%
\pgfpathlineto{\pgfqpoint{3.760389in}{2.457963in}}%
\pgfpathlineto{\pgfqpoint{3.760807in}{2.783970in}}%
\pgfpathlineto{\pgfqpoint{3.761640in}{2.676102in}}%
\pgfpathlineto{\pgfqpoint{3.762057in}{2.681322in}}%
\pgfpathlineto{\pgfqpoint{3.762472in}{2.679300in}}%
\pgfpathlineto{\pgfqpoint{3.764132in}{2.581030in}}%
\pgfpathlineto{\pgfqpoint{3.765785in}{2.752943in}}%
\pgfpathlineto{\pgfqpoint{3.766197in}{2.725865in}}%
\pgfpathlineto{\pgfqpoint{3.767020in}{2.815040in}}%
\pgfpathlineto{\pgfqpoint{3.767431in}{2.801468in}}%
\pgfpathlineto{\pgfqpoint{3.769071in}{2.468327in}}%
\pgfpathlineto{\pgfqpoint{3.769889in}{2.742185in}}%
\pgfpathlineto{\pgfqpoint{3.770297in}{2.613370in}}%
\pgfpathlineto{\pgfqpoint{3.771926in}{2.721697in}}%
\pgfpathlineto{\pgfqpoint{3.772332in}{2.704081in}}%
\pgfpathlineto{\pgfqpoint{3.772738in}{2.592520in}}%
\pgfpathlineto{\pgfqpoint{3.773144in}{2.657121in}}%
\pgfpathlineto{\pgfqpoint{3.773549in}{2.727244in}}%
\pgfpathlineto{\pgfqpoint{3.773953in}{2.644646in}}%
\pgfpathlineto{\pgfqpoint{3.774358in}{2.708671in}}%
\pgfpathlineto{\pgfqpoint{3.774762in}{2.601396in}}%
\pgfpathlineto{\pgfqpoint{3.775165in}{2.714188in}}%
\pgfpathlineto{\pgfqpoint{3.775569in}{2.649757in}}%
\pgfpathlineto{\pgfqpoint{3.775971in}{2.670037in}}%
\pgfpathlineto{\pgfqpoint{3.776776in}{2.667116in}}%
\pgfpathlineto{\pgfqpoint{3.777178in}{2.796808in}}%
\pgfpathlineto{\pgfqpoint{3.777980in}{2.653025in}}%
\pgfpathlineto{\pgfqpoint{3.778380in}{2.575902in}}%
\pgfpathlineto{\pgfqpoint{3.779180in}{2.608553in}}%
\pgfpathlineto{\pgfqpoint{3.779580in}{2.587272in}}%
\pgfpathlineto{\pgfqpoint{3.781173in}{2.801544in}}%
\pgfpathlineto{\pgfqpoint{3.781571in}{2.685311in}}%
\pgfpathlineto{\pgfqpoint{3.781968in}{2.668538in}}%
\pgfpathlineto{\pgfqpoint{3.782365in}{2.684195in}}%
\pgfpathlineto{\pgfqpoint{3.782761in}{2.554647in}}%
\pgfpathlineto{\pgfqpoint{3.783157in}{2.590424in}}%
\pgfpathlineto{\pgfqpoint{3.783553in}{2.762317in}}%
\pgfpathlineto{\pgfqpoint{3.783948in}{2.746408in}}%
\pgfpathlineto{\pgfqpoint{3.784343in}{2.608576in}}%
\pgfpathlineto{\pgfqpoint{3.784738in}{2.791207in}}%
\pgfpathlineto{\pgfqpoint{3.785132in}{2.678598in}}%
\pgfpathlineto{\pgfqpoint{3.785526in}{2.657697in}}%
\pgfpathlineto{\pgfqpoint{3.785919in}{2.812704in}}%
\pgfpathlineto{\pgfqpoint{3.786313in}{2.581402in}}%
\pgfpathlineto{\pgfqpoint{3.787098in}{2.768956in}}%
\pgfpathlineto{\pgfqpoint{3.787490in}{2.533635in}}%
\pgfpathlineto{\pgfqpoint{3.787881in}{2.781589in}}%
\pgfpathlineto{\pgfqpoint{3.788273in}{2.809407in}}%
\pgfpathlineto{\pgfqpoint{3.789054in}{2.461254in}}%
\pgfpathlineto{\pgfqpoint{3.789444in}{2.580780in}}%
\pgfpathlineto{\pgfqpoint{3.789834in}{2.546881in}}%
\pgfpathlineto{\pgfqpoint{3.790224in}{2.804208in}}%
\pgfpathlineto{\pgfqpoint{3.791002in}{2.647701in}}%
\pgfpathlineto{\pgfqpoint{3.791390in}{2.634862in}}%
\pgfpathlineto{\pgfqpoint{3.792553in}{2.770668in}}%
\pgfpathlineto{\pgfqpoint{3.792940in}{2.622514in}}%
\pgfpathlineto{\pgfqpoint{3.793713in}{2.684533in}}%
\pgfpathlineto{\pgfqpoint{3.794099in}{2.778584in}}%
\pgfpathlineto{\pgfqpoint{3.794485in}{2.725885in}}%
\pgfpathlineto{\pgfqpoint{3.794870in}{2.653351in}}%
\pgfpathlineto{\pgfqpoint{3.795255in}{2.798937in}}%
\pgfpathlineto{\pgfqpoint{3.796024in}{2.714985in}}%
\pgfpathlineto{\pgfqpoint{3.796408in}{2.745622in}}%
\pgfpathlineto{\pgfqpoint{3.798322in}{2.534972in}}%
\pgfpathlineto{\pgfqpoint{3.799466in}{2.705542in}}%
\pgfpathlineto{\pgfqpoint{3.799847in}{2.639240in}}%
\pgfpathlineto{\pgfqpoint{3.800227in}{2.697871in}}%
\pgfpathlineto{\pgfqpoint{3.800607in}{2.735022in}}%
\pgfpathlineto{\pgfqpoint{3.800987in}{2.380298in}}%
\pgfpathlineto{\pgfqpoint{3.801746in}{2.459117in}}%
\pgfpathlineto{\pgfqpoint{3.802124in}{2.490501in}}%
\pgfpathlineto{\pgfqpoint{3.802503in}{2.660623in}}%
\pgfpathlineto{\pgfqpoint{3.802881in}{2.518774in}}%
\pgfpathlineto{\pgfqpoint{3.803258in}{2.407211in}}%
\pgfpathlineto{\pgfqpoint{3.803636in}{2.574157in}}%
\pgfpathlineto{\pgfqpoint{3.804013in}{2.578929in}}%
\pgfpathlineto{\pgfqpoint{3.805142in}{2.779248in}}%
\pgfpathlineto{\pgfqpoint{3.805518in}{2.697133in}}%
\pgfpathlineto{\pgfqpoint{3.805893in}{2.608712in}}%
\pgfpathlineto{\pgfqpoint{3.806268in}{2.713973in}}%
\pgfpathlineto{\pgfqpoint{3.806643in}{2.677254in}}%
\pgfpathlineto{\pgfqpoint{3.807017in}{2.828986in}}%
\pgfpathlineto{\pgfqpoint{3.807391in}{2.490943in}}%
\pgfpathlineto{\pgfqpoint{3.808138in}{2.721735in}}%
\pgfpathlineto{\pgfqpoint{3.808511in}{2.545872in}}%
\pgfpathlineto{\pgfqpoint{3.808884in}{2.624494in}}%
\pgfpathlineto{\pgfqpoint{3.809629in}{2.825886in}}%
\pgfpathlineto{\pgfqpoint{3.810000in}{2.746078in}}%
\pgfpathlineto{\pgfqpoint{3.810372in}{2.531111in}}%
\pgfpathlineto{\pgfqpoint{3.810743in}{2.751176in}}%
\pgfpathlineto{\pgfqpoint{3.811114in}{2.737539in}}%
\pgfpathlineto{\pgfqpoint{3.811484in}{2.718762in}}%
\pgfpathlineto{\pgfqpoint{3.812224in}{2.787492in}}%
\pgfpathlineto{\pgfqpoint{3.814436in}{2.497364in}}%
\pgfpathlineto{\pgfqpoint{3.814804in}{2.825078in}}%
\pgfpathlineto{\pgfqpoint{3.815538in}{2.614112in}}%
\pgfpathlineto{\pgfqpoint{3.815905in}{2.718396in}}%
\pgfpathlineto{\pgfqpoint{3.816637in}{2.684925in}}%
\pgfpathlineto{\pgfqpoint{3.817003in}{2.710252in}}%
\pgfpathlineto{\pgfqpoint{3.817368in}{2.664425in}}%
\pgfpathlineto{\pgfqpoint{3.817733in}{2.658140in}}%
\pgfpathlineto{\pgfqpoint{3.818462in}{2.754006in}}%
\pgfpathlineto{\pgfqpoint{3.819190in}{2.386632in}}%
\pgfpathlineto{\pgfqpoint{3.819554in}{2.724165in}}%
\pgfpathlineto{\pgfqpoint{3.819917in}{2.751391in}}%
\pgfpathlineto{\pgfqpoint{3.820280in}{2.643425in}}%
\pgfpathlineto{\pgfqpoint{3.820642in}{2.692050in}}%
\pgfpathlineto{\pgfqpoint{3.821004in}{2.795474in}}%
\pgfpathlineto{\pgfqpoint{3.821366in}{2.674357in}}%
\pgfpathlineto{\pgfqpoint{3.821728in}{2.723248in}}%
\pgfpathlineto{\pgfqpoint{3.823171in}{2.511254in}}%
\pgfpathlineto{\pgfqpoint{3.822450in}{2.772567in}}%
\pgfpathlineto{\pgfqpoint{3.823531in}{2.572802in}}%
\pgfpathlineto{\pgfqpoint{3.824251in}{2.773309in}}%
\pgfpathlineto{\pgfqpoint{3.824969in}{2.764237in}}%
\pgfpathlineto{\pgfqpoint{3.826401in}{2.596546in}}%
\pgfpathlineto{\pgfqpoint{3.827116in}{2.513732in}}%
\pgfpathlineto{\pgfqpoint{3.827829in}{2.720990in}}%
\pgfpathlineto{\pgfqpoint{3.828185in}{2.689886in}}%
\pgfpathlineto{\pgfqpoint{3.828541in}{2.471223in}}%
\pgfpathlineto{\pgfqpoint{3.829252in}{2.713387in}}%
\pgfpathlineto{\pgfqpoint{3.829961in}{2.486716in}}%
\pgfpathlineto{\pgfqpoint{3.830316in}{2.634898in}}%
\pgfpathlineto{\pgfqpoint{3.830670in}{2.659493in}}%
\pgfpathlineto{\pgfqpoint{3.831024in}{2.528946in}}%
\pgfpathlineto{\pgfqpoint{3.831377in}{2.679991in}}%
\pgfpathlineto{\pgfqpoint{3.831730in}{2.689457in}}%
\pgfpathlineto{\pgfqpoint{3.832436in}{2.516855in}}%
\pgfpathlineto{\pgfqpoint{3.832788in}{2.710323in}}%
\pgfpathlineto{\pgfqpoint{3.833140in}{2.419010in}}%
\pgfpathlineto{\pgfqpoint{3.833843in}{2.684645in}}%
\pgfpathlineto{\pgfqpoint{3.834194in}{2.603992in}}%
\pgfpathlineto{\pgfqpoint{3.834545in}{2.726459in}}%
\pgfpathlineto{\pgfqpoint{3.834896in}{2.827312in}}%
\pgfpathlineto{\pgfqpoint{3.835246in}{2.724289in}}%
\pgfpathlineto{\pgfqpoint{3.835946in}{2.600879in}}%
\pgfpathlineto{\pgfqpoint{3.836295in}{2.762243in}}%
\pgfpathlineto{\pgfqpoint{3.836644in}{2.399531in}}%
\pgfpathlineto{\pgfqpoint{3.837342in}{2.660167in}}%
\pgfpathlineto{\pgfqpoint{3.838038in}{2.765551in}}%
\pgfpathlineto{\pgfqpoint{3.838386in}{2.600930in}}%
\pgfpathlineto{\pgfqpoint{3.839080in}{2.787691in}}%
\pgfpathlineto{\pgfqpoint{3.839427in}{2.718936in}}%
\pgfpathlineto{\pgfqpoint{3.839774in}{2.424806in}}%
\pgfpathlineto{\pgfqpoint{3.840120in}{2.636194in}}%
\pgfpathlineto{\pgfqpoint{3.840466in}{2.807522in}}%
\pgfpathlineto{\pgfqpoint{3.840812in}{2.567983in}}%
\pgfpathlineto{\pgfqpoint{3.841157in}{2.407910in}}%
\pgfpathlineto{\pgfqpoint{3.841502in}{2.661484in}}%
\pgfpathlineto{\pgfqpoint{3.841847in}{2.449885in}}%
\pgfpathlineto{\pgfqpoint{3.843224in}{2.744524in}}%
\pgfpathlineto{\pgfqpoint{3.843567in}{2.774687in}}%
\pgfpathlineto{\pgfqpoint{3.843910in}{2.732199in}}%
\pgfpathlineto{\pgfqpoint{3.844253in}{2.564067in}}%
\pgfpathlineto{\pgfqpoint{3.844938in}{2.641565in}}%
\pgfpathlineto{\pgfqpoint{3.845622in}{2.765195in}}%
\pgfpathlineto{\pgfqpoint{3.845964in}{2.669494in}}%
\pgfpathlineto{\pgfqpoint{3.846646in}{2.582856in}}%
\pgfpathlineto{\pgfqpoint{3.847327in}{2.614948in}}%
\pgfpathlineto{\pgfqpoint{3.848007in}{2.583982in}}%
\pgfpathlineto{\pgfqpoint{3.848347in}{2.676874in}}%
\pgfpathlineto{\pgfqpoint{3.848686in}{2.668303in}}%
\pgfpathlineto{\pgfqpoint{3.849025in}{2.499041in}}%
\pgfpathlineto{\pgfqpoint{3.849703in}{2.679857in}}%
\pgfpathlineto{\pgfqpoint{3.850041in}{2.540217in}}%
\pgfpathlineto{\pgfqpoint{3.850717in}{2.726242in}}%
\pgfpathlineto{\pgfqpoint{3.851054in}{2.321276in}}%
\pgfpathlineto{\pgfqpoint{3.851391in}{2.746076in}}%
\pgfpathlineto{\pgfqpoint{3.851728in}{2.576635in}}%
\pgfpathlineto{\pgfqpoint{3.852737in}{2.760410in}}%
\pgfpathlineto{\pgfqpoint{3.853073in}{2.721453in}}%
\pgfpathlineto{\pgfqpoint{3.853409in}{2.709303in}}%
\pgfpathlineto{\pgfqpoint{3.854079in}{2.767765in}}%
\pgfpathlineto{\pgfqpoint{3.854414in}{2.514885in}}%
\pgfpathlineto{\pgfqpoint{3.854749in}{2.749535in}}%
\pgfpathlineto{\pgfqpoint{3.855751in}{2.690016in}}%
\pgfpathlineto{\pgfqpoint{3.856084in}{2.690367in}}%
\pgfpathlineto{\pgfqpoint{3.856417in}{2.659290in}}%
\pgfpathlineto{\pgfqpoint{3.856750in}{2.752399in}}%
\pgfpathlineto{\pgfqpoint{3.857416in}{2.724800in}}%
\pgfpathlineto{\pgfqpoint{3.858080in}{2.636691in}}%
\pgfpathlineto{\pgfqpoint{3.858411in}{2.752646in}}%
\pgfpathlineto{\pgfqpoint{3.858743in}{2.582334in}}%
\pgfpathlineto{\pgfqpoint{3.859074in}{2.689731in}}%
\pgfpathlineto{\pgfqpoint{3.859405in}{2.665342in}}%
\pgfpathlineto{\pgfqpoint{3.859735in}{2.460491in}}%
\pgfpathlineto{\pgfqpoint{3.860066in}{2.634751in}}%
\pgfpathlineto{\pgfqpoint{3.860396in}{2.741130in}}%
\pgfpathlineto{\pgfqpoint{3.860726in}{2.705102in}}%
\pgfpathlineto{\pgfqpoint{3.861714in}{2.585838in}}%
\pgfpathlineto{\pgfqpoint{3.862371in}{2.719736in}}%
\pgfpathlineto{\pgfqpoint{3.863028in}{2.671572in}}%
\pgfpathlineto{\pgfqpoint{3.863355in}{2.653575in}}%
\pgfpathlineto{\pgfqpoint{3.863683in}{2.687748in}}%
\pgfpathlineto{\pgfqpoint{3.864010in}{2.727524in}}%
\pgfpathlineto{\pgfqpoint{3.864337in}{2.665096in}}%
\pgfpathlineto{\pgfqpoint{3.864991in}{2.715736in}}%
\pgfpathlineto{\pgfqpoint{3.866294in}{2.462591in}}%
\pgfpathlineto{\pgfqpoint{3.867270in}{2.731542in}}%
\pgfpathlineto{\pgfqpoint{3.867594in}{2.567543in}}%
\pgfpathlineto{\pgfqpoint{3.868243in}{2.678959in}}%
\pgfpathlineto{\pgfqpoint{3.868566in}{2.760614in}}%
\pgfpathlineto{\pgfqpoint{3.868890in}{2.568632in}}%
\pgfpathlineto{\pgfqpoint{3.869213in}{2.529687in}}%
\pgfpathlineto{\pgfqpoint{3.869536in}{2.621435in}}%
\pgfpathlineto{\pgfqpoint{3.869859in}{2.760998in}}%
\pgfpathlineto{\pgfqpoint{3.870504in}{2.742153in}}%
\pgfpathlineto{\pgfqpoint{3.871148in}{2.745308in}}%
\pgfpathlineto{\pgfqpoint{3.871791in}{2.548804in}}%
\pgfpathlineto{\pgfqpoint{3.872112in}{2.663276in}}%
\pgfpathlineto{\pgfqpoint{3.873074in}{2.627240in}}%
\pgfpathlineto{\pgfqpoint{3.873714in}{2.775918in}}%
\pgfpathlineto{\pgfqpoint{3.874034in}{2.677949in}}%
\pgfpathlineto{\pgfqpoint{3.874353in}{2.647238in}}%
\pgfpathlineto{\pgfqpoint{3.874672in}{2.738747in}}%
\pgfpathlineto{\pgfqpoint{3.874991in}{2.624388in}}%
\pgfpathlineto{\pgfqpoint{3.875310in}{2.260513in}}%
\pgfpathlineto{\pgfqpoint{3.875629in}{2.695337in}}%
\pgfpathlineto{\pgfqpoint{3.875947in}{2.635530in}}%
\pgfpathlineto{\pgfqpoint{3.876265in}{2.706401in}}%
\pgfpathlineto{\pgfqpoint{3.876583in}{2.700424in}}%
\pgfpathlineto{\pgfqpoint{3.876900in}{2.566438in}}%
\pgfpathlineto{\pgfqpoint{3.877851in}{2.580304in}}%
\pgfpathlineto{\pgfqpoint{3.878168in}{2.691094in}}%
\pgfpathlineto{\pgfqpoint{3.878484in}{2.346278in}}%
\pgfpathlineto{\pgfqpoint{3.879116in}{2.642577in}}%
\pgfpathlineto{\pgfqpoint{3.880062in}{2.813592in}}%
\pgfpathlineto{\pgfqpoint{3.879747in}{2.614711in}}%
\pgfpathlineto{\pgfqpoint{3.880377in}{2.666311in}}%
\pgfpathlineto{\pgfqpoint{3.880692in}{2.680553in}}%
\pgfpathlineto{\pgfqpoint{3.881321in}{2.685208in}}%
\pgfpathlineto{\pgfqpoint{3.881635in}{2.593847in}}%
\pgfpathlineto{\pgfqpoint{3.881948in}{2.727371in}}%
\pgfpathlineto{\pgfqpoint{3.882575in}{2.563020in}}%
\pgfpathlineto{\pgfqpoint{3.882888in}{2.567784in}}%
\pgfpathlineto{\pgfqpoint{3.883826in}{2.748487in}}%
\pgfpathlineto{\pgfqpoint{3.884450in}{2.422933in}}%
\pgfpathlineto{\pgfqpoint{3.884762in}{2.775934in}}%
\pgfpathlineto{\pgfqpoint{3.885073in}{2.464519in}}%
\pgfpathlineto{\pgfqpoint{3.885695in}{2.763917in}}%
\pgfpathlineto{\pgfqpoint{3.886317in}{2.624624in}}%
\pgfpathlineto{\pgfqpoint{3.887247in}{2.620312in}}%
\pgfpathlineto{\pgfqpoint{3.887557in}{2.742502in}}%
\pgfpathlineto{\pgfqpoint{3.888793in}{2.542490in}}%
\pgfpathlineto{\pgfqpoint{3.890026in}{2.781094in}}%
\pgfpathlineto{\pgfqpoint{3.890333in}{2.556828in}}%
\pgfpathlineto{\pgfqpoint{3.891255in}{2.612724in}}%
\pgfpathlineto{\pgfqpoint{3.891868in}{2.588754in}}%
\pgfpathlineto{\pgfqpoint{3.892480in}{2.777355in}}%
\pgfpathlineto{\pgfqpoint{3.892786in}{2.517542in}}%
\pgfpathlineto{\pgfqpoint{3.893092in}{2.701309in}}%
\pgfpathlineto{\pgfqpoint{3.893703in}{2.681872in}}%
\pgfpathlineto{\pgfqpoint{3.894007in}{2.781051in}}%
\pgfpathlineto{\pgfqpoint{3.894312in}{2.750189in}}%
\pgfpathlineto{\pgfqpoint{3.894921in}{2.769585in}}%
\pgfpathlineto{\pgfqpoint{3.895833in}{2.482107in}}%
\pgfpathlineto{\pgfqpoint{3.896136in}{2.692194in}}%
\pgfpathlineto{\pgfqpoint{3.896742in}{2.562812in}}%
\pgfpathlineto{\pgfqpoint{3.897045in}{2.355148in}}%
\pgfpathlineto{\pgfqpoint{3.897348in}{2.741010in}}%
\pgfpathlineto{\pgfqpoint{3.897650in}{2.629970in}}%
\pgfpathlineto{\pgfqpoint{3.897952in}{2.567240in}}%
\pgfpathlineto{\pgfqpoint{3.898556in}{2.770193in}}%
\pgfpathlineto{\pgfqpoint{3.899159in}{2.719870in}}%
\pgfpathlineto{\pgfqpoint{3.899761in}{2.443766in}}%
\pgfpathlineto{\pgfqpoint{3.900362in}{2.657123in}}%
\pgfpathlineto{\pgfqpoint{3.900662in}{2.715964in}}%
\pgfpathlineto{\pgfqpoint{3.900962in}{2.660873in}}%
\pgfpathlineto{\pgfqpoint{3.901262in}{2.566018in}}%
\pgfpathlineto{\pgfqpoint{3.901561in}{2.753790in}}%
\pgfpathlineto{\pgfqpoint{3.901861in}{2.597312in}}%
\pgfpathlineto{\pgfqpoint{3.903056in}{2.689986in}}%
\pgfpathlineto{\pgfqpoint{3.903950in}{2.614283in}}%
\pgfpathlineto{\pgfqpoint{3.903652in}{2.718864in}}%
\pgfpathlineto{\pgfqpoint{3.904248in}{2.656637in}}%
\pgfpathlineto{\pgfqpoint{3.904546in}{2.658716in}}%
\pgfpathlineto{\pgfqpoint{3.905140in}{2.615593in}}%
\pgfpathlineto{\pgfqpoint{3.905733in}{2.567780in}}%
\pgfpathlineto{\pgfqpoint{3.906030in}{2.698715in}}%
\pgfpathlineto{\pgfqpoint{3.906326in}{2.510252in}}%
\pgfpathlineto{\pgfqpoint{3.906622in}{2.706763in}}%
\pgfpathlineto{\pgfqpoint{3.906918in}{2.615087in}}%
\pgfpathlineto{\pgfqpoint{3.908394in}{2.737048in}}%
\pgfpathlineto{\pgfqpoint{3.909571in}{2.470083in}}%
\pgfpathlineto{\pgfqpoint{3.908983in}{2.827762in}}%
\pgfpathlineto{\pgfqpoint{3.909865in}{2.601456in}}%
\pgfpathlineto{\pgfqpoint{3.910159in}{2.625546in}}%
\pgfpathlineto{\pgfqpoint{3.910745in}{2.609065in}}%
\pgfpathlineto{\pgfqpoint{3.911623in}{2.791053in}}%
\pgfpathlineto{\pgfqpoint{3.912208in}{2.791978in}}%
\pgfpathlineto{\pgfqpoint{3.912792in}{2.663375in}}%
\pgfpathlineto{\pgfqpoint{3.913083in}{2.797126in}}%
\pgfpathlineto{\pgfqpoint{3.913375in}{2.505992in}}%
\pgfpathlineto{\pgfqpoint{3.913957in}{2.769654in}}%
\pgfpathlineto{\pgfqpoint{3.914828in}{2.641269in}}%
\pgfpathlineto{\pgfqpoint{3.915118in}{2.774317in}}%
\pgfpathlineto{\pgfqpoint{3.915408in}{2.488424in}}%
\pgfpathlineto{\pgfqpoint{3.915698in}{2.435379in}}%
\pgfpathlineto{\pgfqpoint{3.915988in}{2.684845in}}%
\pgfpathlineto{\pgfqpoint{3.916855in}{2.604413in}}%
\pgfpathlineto{\pgfqpoint{3.917433in}{2.633023in}}%
\pgfpathlineto{\pgfqpoint{3.917721in}{2.319809in}}%
\pgfpathlineto{\pgfqpoint{3.918009in}{2.720478in}}%
\pgfpathlineto{\pgfqpoint{3.918297in}{2.653650in}}%
\pgfpathlineto{\pgfqpoint{3.918873in}{2.759832in}}%
\pgfpathlineto{\pgfqpoint{3.919447in}{2.661861in}}%
\pgfpathlineto{\pgfqpoint{3.920594in}{2.490852in}}%
\pgfpathlineto{\pgfqpoint{3.921167in}{2.740212in}}%
\pgfpathlineto{\pgfqpoint{3.921738in}{2.560279in}}%
\pgfpathlineto{\pgfqpoint{3.922879in}{2.758400in}}%
\pgfpathlineto{\pgfqpoint{3.923164in}{2.613837in}}%
\pgfpathlineto{\pgfqpoint{3.923732in}{2.627185in}}%
\pgfpathlineto{\pgfqpoint{3.924300in}{2.476735in}}%
\pgfpathlineto{\pgfqpoint{3.924868in}{2.682195in}}%
\pgfpathlineto{\pgfqpoint{3.925434in}{2.561904in}}%
\pgfpathlineto{\pgfqpoint{3.926000in}{2.654780in}}%
\pgfpathlineto{\pgfqpoint{3.926283in}{2.308430in}}%
\pgfpathlineto{\pgfqpoint{3.926565in}{2.679966in}}%
\pgfpathlineto{\pgfqpoint{3.926847in}{2.672376in}}%
\pgfpathlineto{\pgfqpoint{3.927130in}{2.673146in}}%
\pgfpathlineto{\pgfqpoint{3.927693in}{2.707761in}}%
\pgfpathlineto{\pgfqpoint{3.927975in}{2.531627in}}%
\pgfpathlineto{\pgfqpoint{3.928256in}{2.763020in}}%
\pgfpathlineto{\pgfqpoint{3.928537in}{2.379864in}}%
\pgfpathlineto{\pgfqpoint{3.929099in}{2.670707in}}%
\pgfpathlineto{\pgfqpoint{3.929379in}{2.406107in}}%
\pgfpathlineto{\pgfqpoint{3.929660in}{2.822944in}}%
\pgfpathlineto{\pgfqpoint{3.930220in}{2.527510in}}%
\pgfpathlineto{\pgfqpoint{3.930500in}{2.543203in}}%
\pgfpathlineto{\pgfqpoint{3.930779in}{2.703945in}}%
\pgfpathlineto{\pgfqpoint{3.931338in}{2.530829in}}%
\pgfpathlineto{\pgfqpoint{3.931617in}{2.697599in}}%
\pgfpathlineto{\pgfqpoint{3.932175in}{2.641737in}}%
\pgfpathlineto{\pgfqpoint{3.932453in}{2.652292in}}%
\pgfpathlineto{\pgfqpoint{3.933288in}{2.755500in}}%
\pgfpathlineto{\pgfqpoint{3.933566in}{2.521527in}}%
\pgfpathlineto{\pgfqpoint{3.934398in}{2.682550in}}%
\pgfpathlineto{\pgfqpoint{3.935505in}{2.559689in}}%
\pgfpathlineto{\pgfqpoint{3.935782in}{2.580549in}}%
\pgfpathlineto{\pgfqpoint{3.937161in}{2.731996in}}%
\pgfpathlineto{\pgfqpoint{3.937436in}{2.759228in}}%
\pgfpathlineto{\pgfqpoint{3.937711in}{2.690978in}}%
\pgfpathlineto{\pgfqpoint{3.938536in}{2.701487in}}%
\pgfpathlineto{\pgfqpoint{3.939084in}{2.586902in}}%
\pgfpathlineto{\pgfqpoint{3.939632in}{2.704848in}}%
\pgfpathlineto{\pgfqpoint{3.939906in}{2.700559in}}%
\pgfpathlineto{\pgfqpoint{3.940179in}{2.461574in}}%
\pgfpathlineto{\pgfqpoint{3.940453in}{2.745423in}}%
\pgfpathlineto{\pgfqpoint{3.940999in}{2.677803in}}%
\pgfpathlineto{\pgfqpoint{3.941817in}{2.564651in}}%
\pgfpathlineto{\pgfqpoint{3.942089in}{2.589160in}}%
\pgfpathlineto{\pgfqpoint{3.942361in}{2.740172in}}%
\pgfpathlineto{\pgfqpoint{3.942633in}{2.503815in}}%
\pgfpathlineto{\pgfqpoint{3.942905in}{2.706858in}}%
\pgfpathlineto{\pgfqpoint{3.943448in}{2.265803in}}%
\pgfpathlineto{\pgfqpoint{3.943990in}{2.615251in}}%
\pgfpathlineto{\pgfqpoint{3.945073in}{2.675408in}}%
\pgfpathlineto{\pgfqpoint{3.945343in}{2.584857in}}%
\pgfpathlineto{\pgfqpoint{3.945883in}{2.724293in}}%
\pgfpathlineto{\pgfqpoint{3.946153in}{2.663221in}}%
\pgfpathlineto{\pgfqpoint{3.946691in}{2.734826in}}%
\pgfpathlineto{\pgfqpoint{3.946961in}{2.627479in}}%
\pgfpathlineto{\pgfqpoint{3.947230in}{2.583421in}}%
\pgfpathlineto{\pgfqpoint{3.947498in}{2.649473in}}%
\pgfpathlineto{\pgfqpoint{3.947767in}{2.773321in}}%
\pgfpathlineto{\pgfqpoint{3.948304in}{2.591340in}}%
\pgfpathlineto{\pgfqpoint{3.948572in}{2.701936in}}%
\pgfpathlineto{\pgfqpoint{3.948840in}{2.724273in}}%
\pgfpathlineto{\pgfqpoint{3.949108in}{2.428204in}}%
\pgfpathlineto{\pgfqpoint{3.949910in}{2.710451in}}%
\pgfpathlineto{\pgfqpoint{3.950711in}{2.557366in}}%
\pgfpathlineto{\pgfqpoint{3.950978in}{2.590292in}}%
\pgfpathlineto{\pgfqpoint{3.951244in}{2.723308in}}%
\pgfpathlineto{\pgfqpoint{3.951777in}{2.644328in}}%
\pgfpathlineto{\pgfqpoint{3.952043in}{2.523850in}}%
\pgfpathlineto{\pgfqpoint{3.952309in}{2.731405in}}%
\pgfpathlineto{\pgfqpoint{3.952840in}{2.593510in}}%
\pgfpathlineto{\pgfqpoint{3.953105in}{2.609734in}}%
\pgfpathlineto{\pgfqpoint{3.953370in}{2.526731in}}%
\pgfpathlineto{\pgfqpoint{3.953635in}{2.702728in}}%
\pgfpathlineto{\pgfqpoint{3.953900in}{2.652310in}}%
\pgfpathlineto{\pgfqpoint{3.954165in}{2.691936in}}%
\pgfpathlineto{\pgfqpoint{3.954429in}{2.602284in}}%
\pgfpathlineto{\pgfqpoint{3.954693in}{2.396754in}}%
\pgfpathlineto{\pgfqpoint{3.954958in}{2.623517in}}%
\pgfpathlineto{\pgfqpoint{3.955222in}{2.435285in}}%
\pgfpathlineto{\pgfqpoint{3.955749in}{2.740025in}}%
\pgfpathlineto{\pgfqpoint{3.956539in}{2.705940in}}%
\pgfpathlineto{\pgfqpoint{3.956802in}{2.728064in}}%
\pgfpathlineto{\pgfqpoint{3.957590in}{2.440060in}}%
\pgfpathlineto{\pgfqpoint{3.957853in}{2.640459in}}%
\pgfpathlineto{\pgfqpoint{3.958115in}{2.639236in}}%
\pgfpathlineto{\pgfqpoint{3.958377in}{2.502941in}}%
\pgfpathlineto{\pgfqpoint{3.958639in}{2.642362in}}%
\pgfpathlineto{\pgfqpoint{3.959162in}{2.527627in}}%
\pgfpathlineto{\pgfqpoint{3.959423in}{2.700858in}}%
\pgfpathlineto{\pgfqpoint{3.959946in}{2.621088in}}%
\pgfpathlineto{\pgfqpoint{3.960206in}{2.436801in}}%
\pgfpathlineto{\pgfqpoint{3.960728in}{2.807620in}}%
\pgfpathlineto{\pgfqpoint{3.960988in}{2.478739in}}%
\pgfpathlineto{\pgfqpoint{3.962288in}{2.602908in}}%
\pgfpathlineto{\pgfqpoint{3.963066in}{2.547766in}}%
\pgfpathlineto{\pgfqpoint{3.963325in}{2.555201in}}%
\pgfpathlineto{\pgfqpoint{3.963584in}{2.734120in}}%
\pgfpathlineto{\pgfqpoint{3.963842in}{2.508054in}}%
\pgfpathlineto{\pgfqpoint{3.964359in}{2.621251in}}%
\pgfpathlineto{\pgfqpoint{3.964876in}{2.697361in}}%
\pgfpathlineto{\pgfqpoint{3.965391in}{2.694384in}}%
\pgfpathlineto{\pgfqpoint{3.965649in}{2.618149in}}%
\pgfpathlineto{\pgfqpoint{3.966163in}{2.642231in}}%
\pgfpathlineto{\pgfqpoint{3.966421in}{2.763021in}}%
\pgfpathlineto{\pgfqpoint{3.966934in}{2.590335in}}%
\pgfpathlineto{\pgfqpoint{3.967191in}{2.547129in}}%
\pgfpathlineto{\pgfqpoint{3.967704in}{2.610086in}}%
\pgfpathlineto{\pgfqpoint{3.967960in}{2.581460in}}%
\pgfpathlineto{\pgfqpoint{3.968983in}{2.752081in}}%
\pgfpathlineto{\pgfqpoint{3.969239in}{2.648530in}}%
\pgfpathlineto{\pgfqpoint{3.969749in}{2.700727in}}%
\pgfpathlineto{\pgfqpoint{3.970004in}{2.489872in}}%
\pgfpathlineto{\pgfqpoint{3.970768in}{2.615641in}}%
\pgfpathlineto{\pgfqpoint{3.971022in}{2.667279in}}%
\pgfpathlineto{\pgfqpoint{3.971276in}{2.568419in}}%
\pgfpathlineto{\pgfqpoint{3.971530in}{2.607189in}}%
\pgfpathlineto{\pgfqpoint{3.972038in}{2.667104in}}%
\pgfpathlineto{\pgfqpoint{3.972545in}{2.503536in}}%
\pgfpathlineto{\pgfqpoint{3.973810in}{2.729745in}}%
\pgfpathlineto{\pgfqpoint{3.974062in}{2.725614in}}%
\pgfpathlineto{\pgfqpoint{3.974315in}{2.392671in}}%
\pgfpathlineto{\pgfqpoint{3.975071in}{2.578371in}}%
\pgfpathlineto{\pgfqpoint{3.976077in}{2.556663in}}%
\pgfpathlineto{\pgfqpoint{3.976328in}{2.748142in}}%
\pgfpathlineto{\pgfqpoint{3.977332in}{2.753204in}}%
\pgfpathlineto{\pgfqpoint{3.977582in}{2.574316in}}%
\pgfpathlineto{\pgfqpoint{3.977832in}{2.748447in}}%
\pgfpathlineto{\pgfqpoint{3.978332in}{2.690508in}}%
\pgfpathlineto{\pgfqpoint{3.978582in}{2.219649in}}%
\pgfpathlineto{\pgfqpoint{3.979331in}{2.561217in}}%
\pgfpathlineto{\pgfqpoint{3.979580in}{2.705530in}}%
\pgfpathlineto{\pgfqpoint{3.979829in}{2.390836in}}%
\pgfpathlineto{\pgfqpoint{3.980078in}{2.647276in}}%
\pgfpathlineto{\pgfqpoint{3.980327in}{2.419809in}}%
\pgfpathlineto{\pgfqpoint{3.981073in}{2.724198in}}%
\pgfpathlineto{\pgfqpoint{3.982312in}{2.532174in}}%
\pgfpathlineto{\pgfqpoint{3.982560in}{2.542223in}}%
\pgfpathlineto{\pgfqpoint{3.982807in}{2.656178in}}%
\pgfpathlineto{\pgfqpoint{3.983054in}{2.459166in}}%
\pgfpathlineto{\pgfqpoint{3.983795in}{2.641148in}}%
\pgfpathlineto{\pgfqpoint{3.984288in}{2.761121in}}%
\pgfpathlineto{\pgfqpoint{3.984535in}{2.531879in}}%
\pgfpathlineto{\pgfqpoint{3.985027in}{2.812647in}}%
\pgfpathlineto{\pgfqpoint{3.985519in}{2.634517in}}%
\pgfpathlineto{\pgfqpoint{3.985764in}{2.721665in}}%
\pgfpathlineto{\pgfqpoint{3.986010in}{2.601564in}}%
\pgfpathlineto{\pgfqpoint{3.986501in}{2.699956in}}%
\pgfpathlineto{\pgfqpoint{3.987725in}{2.536343in}}%
\pgfpathlineto{\pgfqpoint{3.987969in}{2.748165in}}%
\pgfpathlineto{\pgfqpoint{3.988457in}{2.429786in}}%
\pgfpathlineto{\pgfqpoint{3.988945in}{2.689512in}}%
\pgfpathlineto{\pgfqpoint{3.989676in}{2.730011in}}%
\pgfpathlineto{\pgfqpoint{3.989919in}{2.697411in}}%
\pgfpathlineto{\pgfqpoint{3.991133in}{2.537205in}}%
\pgfpathlineto{\pgfqpoint{3.990648in}{2.737946in}}%
\pgfpathlineto{\pgfqpoint{3.991376in}{2.570196in}}%
\pgfpathlineto{\pgfqpoint{3.992102in}{2.701023in}}%
\pgfpathlineto{\pgfqpoint{3.991860in}{2.521666in}}%
\pgfpathlineto{\pgfqpoint{3.992586in}{2.647870in}}%
\pgfpathlineto{\pgfqpoint{3.992827in}{2.331870in}}%
\pgfpathlineto{\pgfqpoint{3.993069in}{2.735958in}}%
\pgfpathlineto{\pgfqpoint{3.993551in}{2.491816in}}%
\pgfpathlineto{\pgfqpoint{3.995236in}{2.745486in}}%
\pgfpathlineto{\pgfqpoint{3.996195in}{2.608967in}}%
\pgfpathlineto{\pgfqpoint{3.996674in}{2.641713in}}%
\pgfpathlineto{\pgfqpoint{3.996913in}{2.647073in}}%
\pgfpathlineto{\pgfqpoint{3.997392in}{2.575479in}}%
\pgfpathlineto{\pgfqpoint{3.997630in}{2.726745in}}%
\pgfpathlineto{\pgfqpoint{3.998823in}{2.406303in}}%
\pgfpathlineto{\pgfqpoint{3.999061in}{2.706516in}}%
\pgfpathlineto{\pgfqpoint{3.999774in}{2.604464in}}%
\pgfpathlineto{\pgfqpoint{4.000012in}{2.464021in}}%
\pgfpathlineto{\pgfqpoint{4.000724in}{2.587469in}}%
\pgfpathlineto{\pgfqpoint{4.001434in}{2.573142in}}%
\pgfpathlineto{\pgfqpoint{4.001907in}{2.783117in}}%
\pgfpathlineto{\pgfqpoint{4.002616in}{2.343786in}}%
\pgfpathlineto{\pgfqpoint{4.003559in}{2.427151in}}%
\pgfpathlineto{\pgfqpoint{4.004265in}{2.417632in}}%
\pgfpathlineto{\pgfqpoint{4.004970in}{2.718691in}}%
\pgfpathlineto{\pgfqpoint{4.005204in}{2.721368in}}%
\pgfpathlineto{\pgfqpoint{4.006376in}{2.590163in}}%
\pgfpathlineto{\pgfqpoint{4.006610in}{2.670390in}}%
\pgfpathlineto{\pgfqpoint{4.007310in}{2.637423in}}%
\pgfpathlineto{\pgfqpoint{4.007544in}{2.491071in}}%
\pgfpathlineto{\pgfqpoint{4.007777in}{2.683729in}}%
\pgfpathlineto{\pgfqpoint{4.008243in}{2.658068in}}%
\pgfpathlineto{\pgfqpoint{4.008941in}{2.551138in}}%
\pgfpathlineto{\pgfqpoint{4.009174in}{2.657024in}}%
\pgfpathlineto{\pgfqpoint{4.009406in}{2.676734in}}%
\pgfpathlineto{\pgfqpoint{4.010334in}{2.797101in}}%
\pgfpathlineto{\pgfqpoint{4.010566in}{2.424425in}}%
\pgfpathlineto{\pgfqpoint{4.011491in}{2.758662in}}%
\pgfpathlineto{\pgfqpoint{4.011722in}{2.658230in}}%
\pgfpathlineto{\pgfqpoint{4.011953in}{2.646078in}}%
\pgfpathlineto{\pgfqpoint{4.012645in}{2.761257in}}%
\pgfpathlineto{\pgfqpoint{4.012876in}{2.579271in}}%
\pgfpathlineto{\pgfqpoint{4.013336in}{2.687171in}}%
\pgfpathlineto{\pgfqpoint{4.013566in}{2.664751in}}%
\pgfpathlineto{\pgfqpoint{4.013796in}{2.558647in}}%
\pgfpathlineto{\pgfqpoint{4.014026in}{2.705843in}}%
\pgfpathlineto{\pgfqpoint{4.014485in}{2.643294in}}%
\pgfpathlineto{\pgfqpoint{4.015403in}{2.586258in}}%
\pgfpathlineto{\pgfqpoint{4.015631in}{2.722093in}}%
\pgfpathlineto{\pgfqpoint{4.016089in}{2.469088in}}%
\pgfpathlineto{\pgfqpoint{4.017003in}{2.622688in}}%
\pgfpathlineto{\pgfqpoint{4.017686in}{2.667478in}}%
\pgfpathlineto{\pgfqpoint{4.017914in}{2.453946in}}%
\pgfpathlineto{\pgfqpoint{4.018369in}{2.710482in}}%
\pgfpathlineto{\pgfqpoint{4.018824in}{2.388037in}}%
\pgfpathlineto{\pgfqpoint{4.019051in}{2.580223in}}%
\pgfpathlineto{\pgfqpoint{4.019505in}{2.718497in}}%
\pgfpathlineto{\pgfqpoint{4.019732in}{2.461088in}}%
\pgfpathlineto{\pgfqpoint{4.019958in}{2.735764in}}%
\pgfpathlineto{\pgfqpoint{4.020637in}{2.676285in}}%
\pgfpathlineto{\pgfqpoint{4.020864in}{2.480125in}}%
\pgfpathlineto{\pgfqpoint{4.021315in}{2.703409in}}%
\pgfpathlineto{\pgfqpoint{4.021767in}{2.514585in}}%
\pgfpathlineto{\pgfqpoint{4.022893in}{2.692817in}}%
\pgfpathlineto{\pgfqpoint{4.022218in}{2.483763in}}%
\pgfpathlineto{\pgfqpoint{4.023118in}{2.665349in}}%
\pgfpathlineto{\pgfqpoint{4.023343in}{2.748295in}}%
\pgfpathlineto{\pgfqpoint{4.024017in}{2.596505in}}%
\pgfpathlineto{\pgfqpoint{4.024241in}{2.724791in}}%
\pgfpathlineto{\pgfqpoint{4.024690in}{2.808035in}}%
\pgfpathlineto{\pgfqpoint{4.025809in}{2.448592in}}%
\pgfpathlineto{\pgfqpoint{4.026032in}{2.627700in}}%
\pgfpathlineto{\pgfqpoint{4.026701in}{2.581000in}}%
\pgfpathlineto{\pgfqpoint{4.026924in}{2.376229in}}%
\pgfpathlineto{\pgfqpoint{4.027370in}{2.708184in}}%
\pgfpathlineto{\pgfqpoint{4.027815in}{2.547584in}}%
\pgfpathlineto{\pgfqpoint{4.028037in}{2.546441in}}%
\pgfpathlineto{\pgfqpoint{4.028482in}{2.537408in}}%
\pgfpathlineto{\pgfqpoint{4.029369in}{2.726825in}}%
\pgfpathlineto{\pgfqpoint{4.029812in}{2.689323in}}%
\pgfpathlineto{\pgfqpoint{4.030476in}{2.470321in}}%
\pgfpathlineto{\pgfqpoint{4.030254in}{2.710750in}}%
\pgfpathlineto{\pgfqpoint{4.030917in}{2.616842in}}%
\pgfpathlineto{\pgfqpoint{4.031138in}{2.773076in}}%
\pgfpathlineto{\pgfqpoint{4.031359in}{2.538989in}}%
\pgfpathlineto{\pgfqpoint{4.032020in}{2.731568in}}%
\pgfpathlineto{\pgfqpoint{4.032900in}{2.589153in}}%
\pgfpathlineto{\pgfqpoint{4.033339in}{2.666244in}}%
\pgfpathlineto{\pgfqpoint{4.033559in}{2.663593in}}%
\pgfpathlineto{\pgfqpoint{4.034873in}{2.488196in}}%
\pgfpathlineto{\pgfqpoint{4.036184in}{2.746674in}}%
\pgfpathlineto{\pgfqpoint{4.036402in}{2.621800in}}%
\pgfpathlineto{\pgfqpoint{4.037273in}{2.637485in}}%
\pgfpathlineto{\pgfqpoint{4.037490in}{2.732671in}}%
\pgfpathlineto{\pgfqpoint{4.037925in}{2.527009in}}%
\pgfpathlineto{\pgfqpoint{4.038142in}{2.642969in}}%
\pgfpathlineto{\pgfqpoint{4.038576in}{2.582743in}}%
\pgfpathlineto{\pgfqpoint{4.039442in}{2.468765in}}%
\pgfpathlineto{\pgfqpoint{4.039875in}{2.742123in}}%
\pgfpathlineto{\pgfqpoint{4.040091in}{2.386122in}}%
\pgfpathlineto{\pgfqpoint{4.040954in}{2.697525in}}%
\pgfpathlineto{\pgfqpoint{4.041816in}{2.367007in}}%
\pgfpathlineto{\pgfqpoint{4.042246in}{2.435432in}}%
\pgfpathlineto{\pgfqpoint{4.043105in}{2.694692in}}%
\pgfpathlineto{\pgfqpoint{4.043534in}{2.624050in}}%
\pgfpathlineto{\pgfqpoint{4.044176in}{2.707532in}}%
\pgfpathlineto{\pgfqpoint{4.044390in}{2.505972in}}%
\pgfpathlineto{\pgfqpoint{4.045245in}{2.643332in}}%
\pgfpathlineto{\pgfqpoint{4.045458in}{2.763032in}}%
\pgfpathlineto{\pgfqpoint{4.046098in}{2.565185in}}%
\pgfpathlineto{\pgfqpoint{4.046311in}{2.735311in}}%
\pgfpathlineto{\pgfqpoint{4.046949in}{2.579263in}}%
\pgfpathlineto{\pgfqpoint{4.047587in}{2.671057in}}%
\pgfpathlineto{\pgfqpoint{4.047799in}{2.720344in}}%
\pgfpathlineto{\pgfqpoint{4.048223in}{2.587671in}}%
\pgfpathlineto{\pgfqpoint{4.048435in}{2.598005in}}%
\pgfpathlineto{\pgfqpoint{4.049282in}{2.741355in}}%
\pgfpathlineto{\pgfqpoint{4.049493in}{2.513341in}}%
\pgfpathlineto{\pgfqpoint{4.049704in}{2.746710in}}%
\pgfpathlineto{\pgfqpoint{4.050548in}{2.558025in}}%
\pgfpathlineto{\pgfqpoint{4.051180in}{2.708106in}}%
\pgfpathlineto{\pgfqpoint{4.051811in}{2.704242in}}%
\pgfpathlineto{\pgfqpoint{4.052861in}{2.584488in}}%
\pgfpathlineto{\pgfqpoint{4.052651in}{2.733191in}}%
\pgfpathlineto{\pgfqpoint{4.053071in}{2.585346in}}%
\pgfpathlineto{\pgfqpoint{4.053280in}{2.699777in}}%
\pgfpathlineto{\pgfqpoint{4.053908in}{2.565352in}}%
\pgfpathlineto{\pgfqpoint{4.054117in}{2.421574in}}%
\pgfpathlineto{\pgfqpoint{4.054326in}{2.656158in}}%
\pgfpathlineto{\pgfqpoint{4.054744in}{2.646893in}}%
\pgfpathlineto{\pgfqpoint{4.054952in}{2.732257in}}%
\pgfpathlineto{\pgfqpoint{4.055369in}{2.459638in}}%
\pgfpathlineto{\pgfqpoint{4.055578in}{2.591195in}}%
\pgfpathlineto{\pgfqpoint{4.055994in}{2.683831in}}%
\pgfpathlineto{\pgfqpoint{4.056410in}{2.737430in}}%
\pgfpathlineto{\pgfqpoint{4.056618in}{2.663773in}}%
\pgfpathlineto{\pgfqpoint{4.056826in}{2.299624in}}%
\pgfpathlineto{\pgfqpoint{4.057656in}{2.578963in}}%
\pgfpathlineto{\pgfqpoint{4.057863in}{2.705202in}}%
\pgfpathlineto{\pgfqpoint{4.058691in}{2.701657in}}%
\pgfpathlineto{\pgfqpoint{4.059311in}{2.600566in}}%
\pgfpathlineto{\pgfqpoint{4.059724in}{2.635602in}}%
\pgfpathlineto{\pgfqpoint{4.059930in}{2.748003in}}%
\pgfpathlineto{\pgfqpoint{4.060136in}{2.464657in}}%
\pgfpathlineto{\pgfqpoint{4.060548in}{2.590900in}}%
\pgfpathlineto{\pgfqpoint{4.060754in}{2.563714in}}%
\pgfpathlineto{\pgfqpoint{4.061166in}{2.736441in}}%
\pgfpathlineto{\pgfqpoint{4.061782in}{2.648309in}}%
\pgfpathlineto{\pgfqpoint{4.063012in}{2.417868in}}%
\pgfpathlineto{\pgfqpoint{4.062398in}{2.788026in}}%
\pgfpathlineto{\pgfqpoint{4.063217in}{2.474549in}}%
\pgfpathlineto{\pgfqpoint{4.063421in}{2.648556in}}%
\pgfpathlineto{\pgfqpoint{4.063830in}{2.451760in}}%
\pgfpathlineto{\pgfqpoint{4.064443in}{2.590802in}}%
\pgfpathlineto{\pgfqpoint{4.064647in}{2.399509in}}%
\pgfpathlineto{\pgfqpoint{4.064851in}{2.716683in}}%
\pgfpathlineto{\pgfqpoint{4.065258in}{2.471003in}}%
\pgfpathlineto{\pgfqpoint{4.065868in}{2.719740in}}%
\pgfpathlineto{\pgfqpoint{4.066478in}{2.587965in}}%
\pgfpathlineto{\pgfqpoint{4.066681in}{2.389583in}}%
\pgfpathlineto{\pgfqpoint{4.067087in}{2.665752in}}%
\pgfpathlineto{\pgfqpoint{4.067492in}{2.517903in}}%
\pgfpathlineto{\pgfqpoint{4.067897in}{2.733594in}}%
\pgfpathlineto{\pgfqpoint{4.068504in}{2.492631in}}%
\pgfpathlineto{\pgfqpoint{4.068706in}{2.571354in}}%
\pgfpathlineto{\pgfqpoint{4.069513in}{2.714985in}}%
\pgfpathlineto{\pgfqpoint{4.069109in}{2.548001in}}%
\pgfpathlineto{\pgfqpoint{4.069714in}{2.621679in}}%
\pgfpathlineto{\pgfqpoint{4.070318in}{2.458200in}}%
\pgfpathlineto{\pgfqpoint{4.070520in}{2.535556in}}%
\pgfpathlineto{\pgfqpoint{4.071524in}{2.681099in}}%
\pgfpathlineto{\pgfqpoint{4.071323in}{2.516337in}}%
\pgfpathlineto{\pgfqpoint{4.071725in}{2.637795in}}%
\pgfpathlineto{\pgfqpoint{4.072125in}{2.682106in}}%
\pgfpathlineto{\pgfqpoint{4.072526in}{2.609292in}}%
\pgfpathlineto{\pgfqpoint{4.073126in}{2.438259in}}%
\pgfpathlineto{\pgfqpoint{4.073326in}{2.667504in}}%
\pgfpathlineto{\pgfqpoint{4.073526in}{2.578861in}}%
\pgfpathlineto{\pgfqpoint{4.074124in}{2.769425in}}%
\pgfpathlineto{\pgfqpoint{4.073925in}{2.537939in}}%
\pgfpathlineto{\pgfqpoint{4.074324in}{2.649922in}}%
\pgfpathlineto{\pgfqpoint{4.074523in}{2.485689in}}%
\pgfpathlineto{\pgfqpoint{4.074722in}{2.703116in}}%
\pgfpathlineto{\pgfqpoint{4.075319in}{2.628794in}}%
\pgfpathlineto{\pgfqpoint{4.075518in}{2.787863in}}%
\pgfpathlineto{\pgfqpoint{4.075915in}{2.614228in}}%
\pgfpathlineto{\pgfqpoint{4.076312in}{2.651426in}}%
\pgfpathlineto{\pgfqpoint{4.077699in}{2.486090in}}%
\pgfpathlineto{\pgfqpoint{4.078292in}{2.700847in}}%
\pgfpathlineto{\pgfqpoint{4.078686in}{2.466829in}}%
\pgfpathlineto{\pgfqpoint{4.078884in}{2.327528in}}%
\pgfpathlineto{\pgfqpoint{4.079278in}{2.677155in}}%
\pgfpathlineto{\pgfqpoint{4.079475in}{2.611763in}}%
\pgfpathlineto{\pgfqpoint{4.079672in}{2.630083in}}%
\pgfpathlineto{\pgfqpoint{4.080458in}{2.445748in}}%
\pgfpathlineto{\pgfqpoint{4.080655in}{2.652771in}}%
\pgfpathlineto{\pgfqpoint{4.081047in}{2.475218in}}%
\pgfpathlineto{\pgfqpoint{4.081243in}{2.703617in}}%
\pgfpathlineto{\pgfqpoint{4.081439in}{2.735690in}}%
\pgfpathlineto{\pgfqpoint{4.082418in}{2.498622in}}%
\pgfpathlineto{\pgfqpoint{4.082614in}{2.557105in}}%
\pgfpathlineto{\pgfqpoint{4.083200in}{2.721762in}}%
\pgfpathlineto{\pgfqpoint{4.083395in}{2.494190in}}%
\pgfpathlineto{\pgfqpoint{4.083590in}{2.627221in}}%
\pgfpathlineto{\pgfqpoint{4.083785in}{2.366671in}}%
\pgfpathlineto{\pgfqpoint{4.084175in}{2.635239in}}%
\pgfpathlineto{\pgfqpoint{4.084759in}{2.493341in}}%
\pgfpathlineto{\pgfqpoint{4.085342in}{2.647089in}}%
\pgfpathlineto{\pgfqpoint{4.085924in}{2.597929in}}%
\pgfpathlineto{\pgfqpoint{4.086118in}{2.546430in}}%
\pgfpathlineto{\pgfqpoint{4.086312in}{2.704292in}}%
\pgfpathlineto{\pgfqpoint{4.086505in}{2.675901in}}%
\pgfpathlineto{\pgfqpoint{4.086699in}{2.747262in}}%
\pgfpathlineto{\pgfqpoint{4.087086in}{2.598595in}}%
\pgfpathlineto{\pgfqpoint{4.087666in}{2.714511in}}%
\pgfpathlineto{\pgfqpoint{4.088245in}{2.466647in}}%
\pgfpathlineto{\pgfqpoint{4.089401in}{2.703166in}}%
\pgfpathlineto{\pgfqpoint{4.089593in}{2.519702in}}%
\pgfpathlineto{\pgfqpoint{4.090361in}{2.596925in}}%
\pgfpathlineto{\pgfqpoint{4.091129in}{2.547457in}}%
\pgfpathlineto{\pgfqpoint{4.091320in}{2.730389in}}%
\pgfpathlineto{\pgfqpoint{4.091894in}{2.462485in}}%
\pgfpathlineto{\pgfqpoint{4.092468in}{2.595577in}}%
\pgfpathlineto{\pgfqpoint{4.093231in}{2.718191in}}%
\pgfpathlineto{\pgfqpoint{4.093040in}{2.562566in}}%
\pgfpathlineto{\pgfqpoint{4.093612in}{2.693548in}}%
\pgfpathlineto{\pgfqpoint{4.093802in}{2.563644in}}%
\pgfpathlineto{\pgfqpoint{4.094563in}{2.715692in}}%
\pgfpathlineto{\pgfqpoint{4.094753in}{2.759607in}}%
\pgfpathlineto{\pgfqpoint{4.094943in}{2.646070in}}%
\pgfpathlineto{\pgfqpoint{4.095891in}{2.499103in}}%
\pgfpathlineto{\pgfqpoint{4.095323in}{2.654377in}}%
\pgfpathlineto{\pgfqpoint{4.096081in}{2.522301in}}%
\pgfpathlineto{\pgfqpoint{4.096270in}{2.709647in}}%
\pgfpathlineto{\pgfqpoint{4.097215in}{2.653778in}}%
\pgfpathlineto{\pgfqpoint{4.097782in}{2.690429in}}%
\pgfpathlineto{\pgfqpoint{4.097970in}{2.492753in}}%
\pgfpathlineto{\pgfqpoint{4.099100in}{2.723729in}}%
\pgfpathlineto{\pgfqpoint{4.099288in}{2.538855in}}%
\pgfpathlineto{\pgfqpoint{4.100226in}{2.698861in}}%
\pgfpathlineto{\pgfqpoint{4.100601in}{2.463435in}}%
\pgfpathlineto{\pgfqpoint{4.101537in}{2.541319in}}%
\pgfpathlineto{\pgfqpoint{4.102098in}{2.641094in}}%
\pgfpathlineto{\pgfqpoint{4.102657in}{2.292995in}}%
\pgfpathlineto{\pgfqpoint{4.103960in}{2.735449in}}%
\pgfpathlineto{\pgfqpoint{4.104146in}{2.753597in}}%
\pgfpathlineto{\pgfqpoint{4.104703in}{2.440235in}}%
\pgfpathlineto{\pgfqpoint{4.105260in}{2.679230in}}%
\pgfpathlineto{\pgfqpoint{4.106000in}{2.517554in}}%
\pgfpathlineto{\pgfqpoint{4.106185in}{2.681723in}}%
\pgfpathlineto{\pgfqpoint{4.106370in}{2.659041in}}%
\pgfpathlineto{\pgfqpoint{4.106555in}{2.710806in}}%
\pgfpathlineto{\pgfqpoint{4.107109in}{2.561889in}}%
\pgfpathlineto{\pgfqpoint{4.107293in}{2.560413in}}%
\pgfpathlineto{\pgfqpoint{4.107662in}{2.657700in}}%
\pgfpathlineto{\pgfqpoint{4.108582in}{2.654628in}}%
\pgfpathlineto{\pgfqpoint{4.108950in}{2.733359in}}%
\pgfpathlineto{\pgfqpoint{4.109317in}{2.564446in}}%
\pgfpathlineto{\pgfqpoint{4.109500in}{2.761329in}}%
\pgfpathlineto{\pgfqpoint{4.110417in}{2.687859in}}%
\pgfpathlineto{\pgfqpoint{4.111514in}{2.522445in}}%
\pgfpathlineto{\pgfqpoint{4.111148in}{2.708528in}}%
\pgfpathlineto{\pgfqpoint{4.111879in}{2.580879in}}%
\pgfpathlineto{\pgfqpoint{4.112608in}{2.539164in}}%
\pgfpathlineto{\pgfqpoint{4.113154in}{2.670352in}}%
\pgfpathlineto{\pgfqpoint{4.113336in}{2.658186in}}%
\pgfpathlineto{\pgfqpoint{4.113881in}{2.702831in}}%
\pgfpathlineto{\pgfqpoint{4.114425in}{2.495667in}}%
\pgfpathlineto{\pgfqpoint{4.114607in}{2.782073in}}%
\pgfpathlineto{\pgfqpoint{4.115331in}{2.689166in}}%
\pgfpathlineto{\pgfqpoint{4.115512in}{2.312199in}}%
\pgfpathlineto{\pgfqpoint{4.115693in}{2.706842in}}%
\pgfpathlineto{\pgfqpoint{4.116416in}{2.690447in}}%
\pgfpathlineto{\pgfqpoint{4.116776in}{2.621813in}}%
\pgfpathlineto{\pgfqpoint{4.116957in}{2.692343in}}%
\pgfpathlineto{\pgfqpoint{4.117137in}{2.667141in}}%
\pgfpathlineto{\pgfqpoint{4.117317in}{2.767140in}}%
\pgfpathlineto{\pgfqpoint{4.117677in}{2.630486in}}%
\pgfpathlineto{\pgfqpoint{4.118217in}{2.696757in}}%
\pgfpathlineto{\pgfqpoint{4.119115in}{2.735199in}}%
\pgfpathlineto{\pgfqpoint{4.119473in}{2.551682in}}%
\pgfpathlineto{\pgfqpoint{4.119652in}{2.562423in}}%
\pgfpathlineto{\pgfqpoint{4.119831in}{2.432706in}}%
\pgfpathlineto{\pgfqpoint{4.120368in}{2.653022in}}%
\pgfpathlineto{\pgfqpoint{4.120547in}{2.619296in}}%
\pgfpathlineto{\pgfqpoint{4.120904in}{2.738911in}}%
\pgfpathlineto{\pgfqpoint{4.121083in}{2.383699in}}%
\pgfpathlineto{\pgfqpoint{4.121975in}{2.601123in}}%
\pgfpathlineto{\pgfqpoint{4.122153in}{2.566022in}}%
\pgfpathlineto{\pgfqpoint{4.122509in}{2.665788in}}%
\pgfpathlineto{\pgfqpoint{4.122687in}{2.644613in}}%
\pgfpathlineto{\pgfqpoint{4.122864in}{2.677102in}}%
\pgfpathlineto{\pgfqpoint{4.123042in}{2.488404in}}%
\pgfpathlineto{\pgfqpoint{4.124107in}{2.521769in}}%
\pgfpathlineto{\pgfqpoint{4.124284in}{2.760190in}}%
\pgfpathlineto{\pgfqpoint{4.124639in}{2.471902in}}%
\pgfpathlineto{\pgfqpoint{4.125169in}{2.693294in}}%
\pgfpathlineto{\pgfqpoint{4.125346in}{2.686666in}}%
\pgfpathlineto{\pgfqpoint{4.126053in}{2.394334in}}%
\pgfpathlineto{\pgfqpoint{4.126405in}{2.688391in}}%
\pgfpathlineto{\pgfqpoint{4.126582in}{2.558043in}}%
\pgfpathlineto{\pgfqpoint{4.127110in}{2.703027in}}%
\pgfpathlineto{\pgfqpoint{4.126934in}{2.434772in}}%
\pgfpathlineto{\pgfqpoint{4.127638in}{2.611149in}}%
\pgfpathlineto{\pgfqpoint{4.127989in}{2.509291in}}%
\pgfpathlineto{\pgfqpoint{4.128165in}{2.667241in}}%
\pgfpathlineto{\pgfqpoint{4.128691in}{2.519670in}}%
\pgfpathlineto{\pgfqpoint{4.129567in}{2.674151in}}%
\pgfpathlineto{\pgfqpoint{4.129917in}{2.630658in}}%
\pgfpathlineto{\pgfqpoint{4.130092in}{2.667327in}}%
\pgfpathlineto{\pgfqpoint{4.130441in}{2.509238in}}%
\pgfpathlineto{\pgfqpoint{4.130616in}{2.481579in}}%
\pgfpathlineto{\pgfqpoint{4.131139in}{2.739100in}}%
\pgfpathlineto{\pgfqpoint{4.131836in}{2.729275in}}%
\pgfpathlineto{\pgfqpoint{4.132358in}{2.426913in}}%
\pgfpathlineto{\pgfqpoint{4.132879in}{2.575646in}}%
\pgfpathlineto{\pgfqpoint{4.133226in}{2.720630in}}%
\pgfpathlineto{\pgfqpoint{4.133399in}{2.505101in}}%
\pgfpathlineto{\pgfqpoint{4.134093in}{2.702203in}}%
\pgfpathlineto{\pgfqpoint{4.134439in}{2.542570in}}%
\pgfpathlineto{\pgfqpoint{4.135130in}{2.602820in}}%
\pgfpathlineto{\pgfqpoint{4.135303in}{2.685217in}}%
\pgfpathlineto{\pgfqpoint{4.135648in}{2.556867in}}%
\pgfpathlineto{\pgfqpoint{4.135993in}{2.567700in}}%
\pgfpathlineto{\pgfqpoint{4.136165in}{2.433307in}}%
\pgfpathlineto{\pgfqpoint{4.136337in}{2.727043in}}%
\pgfpathlineto{\pgfqpoint{4.137026in}{2.586212in}}%
\pgfpathlineto{\pgfqpoint{4.137198in}{2.764413in}}%
\pgfpathlineto{\pgfqpoint{4.138228in}{2.723476in}}%
\pgfpathlineto{\pgfqpoint{4.139255in}{2.319413in}}%
\pgfpathlineto{\pgfqpoint{4.139426in}{2.571796in}}%
\pgfpathlineto{\pgfqpoint{4.139939in}{2.697946in}}%
\pgfpathlineto{\pgfqpoint{4.140110in}{2.572388in}}%
\pgfpathlineto{\pgfqpoint{4.140280in}{2.442658in}}%
\pgfpathlineto{\pgfqpoint{4.140451in}{2.668414in}}%
\pgfpathlineto{\pgfqpoint{4.140962in}{2.514619in}}%
\pgfpathlineto{\pgfqpoint{4.141643in}{2.690207in}}%
\pgfpathlineto{\pgfqpoint{4.141983in}{2.639294in}}%
\pgfpathlineto{\pgfqpoint{4.142323in}{2.671741in}}%
\pgfpathlineto{\pgfqpoint{4.143002in}{2.529600in}}%
\pgfpathlineto{\pgfqpoint{4.143341in}{2.694865in}}%
\pgfpathlineto{\pgfqpoint{4.143848in}{2.525443in}}%
\pgfpathlineto{\pgfqpoint{4.144018in}{2.604988in}}%
\pgfpathlineto{\pgfqpoint{4.144356in}{2.475289in}}%
\pgfpathlineto{\pgfqpoint{4.144862in}{2.617212in}}%
\pgfpathlineto{\pgfqpoint{4.145031in}{2.697793in}}%
\pgfpathlineto{\pgfqpoint{4.145200in}{2.558126in}}%
\pgfpathlineto{\pgfqpoint{4.145706in}{2.657329in}}%
\pgfpathlineto{\pgfqpoint{4.145874in}{2.259327in}}%
\pgfpathlineto{\pgfqpoint{4.146715in}{2.727753in}}%
\pgfpathlineto{\pgfqpoint{4.147051in}{2.747879in}}%
\pgfpathlineto{\pgfqpoint{4.148058in}{2.485666in}}%
\pgfpathlineto{\pgfqpoint{4.148894in}{2.672003in}}%
\pgfpathlineto{\pgfqpoint{4.149229in}{2.667606in}}%
\pgfpathlineto{\pgfqpoint{4.150063in}{2.487837in}}%
\pgfpathlineto{\pgfqpoint{4.150230in}{2.647853in}}%
\pgfpathlineto{\pgfqpoint{4.151063in}{2.753403in}}%
\pgfpathlineto{\pgfqpoint{4.150563in}{2.558405in}}%
\pgfpathlineto{\pgfqpoint{4.151229in}{2.740964in}}%
\pgfpathlineto{\pgfqpoint{4.151395in}{2.296530in}}%
\pgfpathlineto{\pgfqpoint{4.152225in}{2.652442in}}%
\pgfpathlineto{\pgfqpoint{4.152888in}{2.423589in}}%
\pgfpathlineto{\pgfqpoint{4.152723in}{2.734129in}}%
\pgfpathlineto{\pgfqpoint{4.153054in}{2.654086in}}%
\pgfpathlineto{\pgfqpoint{4.153220in}{2.744353in}}%
\pgfpathlineto{\pgfqpoint{4.153881in}{2.567205in}}%
\pgfpathlineto{\pgfqpoint{4.154046in}{2.724613in}}%
\pgfpathlineto{\pgfqpoint{4.154377in}{2.592609in}}%
\pgfpathlineto{\pgfqpoint{4.155036in}{2.670948in}}%
\pgfpathlineto{\pgfqpoint{4.155201in}{2.725432in}}%
\pgfpathlineto{\pgfqpoint{4.155530in}{2.619416in}}%
\pgfpathlineto{\pgfqpoint{4.155860in}{2.627218in}}%
\pgfpathlineto{\pgfqpoint{4.156188in}{2.687453in}}%
\pgfpathlineto{\pgfqpoint{4.156845in}{2.548699in}}%
\pgfpathlineto{\pgfqpoint{4.157009in}{2.703964in}}%
\pgfpathlineto{\pgfqpoint{4.157992in}{2.626554in}}%
\pgfpathlineto{\pgfqpoint{4.158483in}{2.685636in}}%
\pgfpathlineto{\pgfqpoint{4.158810in}{2.506444in}}%
\pgfpathlineto{\pgfqpoint{4.159952in}{2.695789in}}%
\pgfpathlineto{\pgfqpoint{4.160115in}{2.677593in}}%
\pgfpathlineto{\pgfqpoint{4.160928in}{2.533230in}}%
\pgfpathlineto{\pgfqpoint{4.160765in}{2.685038in}}%
\pgfpathlineto{\pgfqpoint{4.161091in}{2.604399in}}%
\pgfpathlineto{\pgfqpoint{4.161253in}{2.704593in}}%
\pgfpathlineto{\pgfqpoint{4.161902in}{2.548033in}}%
\pgfpathlineto{\pgfqpoint{4.162226in}{2.678578in}}%
\pgfpathlineto{\pgfqpoint{4.162550in}{2.472979in}}%
\pgfpathlineto{\pgfqpoint{4.163359in}{2.637646in}}%
\pgfpathlineto{\pgfqpoint{4.163521in}{2.705947in}}%
\pgfpathlineto{\pgfqpoint{4.163682in}{2.481486in}}%
\pgfpathlineto{\pgfqpoint{4.164166in}{2.681842in}}%
\pgfpathlineto{\pgfqpoint{4.165455in}{2.494745in}}%
\pgfpathlineto{\pgfqpoint{4.165776in}{2.725716in}}%
\pgfpathlineto{\pgfqpoint{4.165937in}{2.413534in}}%
\pgfpathlineto{\pgfqpoint{4.166579in}{2.629128in}}%
\pgfpathlineto{\pgfqpoint{4.166739in}{2.448154in}}%
\pgfpathlineto{\pgfqpoint{4.167220in}{2.719873in}}%
\pgfpathlineto{\pgfqpoint{4.167700in}{2.608427in}}%
\pgfpathlineto{\pgfqpoint{4.167860in}{2.696733in}}%
\pgfpathlineto{\pgfqpoint{4.168020in}{2.414072in}}%
\pgfpathlineto{\pgfqpoint{4.168659in}{2.647840in}}%
\pgfpathlineto{\pgfqpoint{4.168978in}{2.685750in}}%
\pgfpathlineto{\pgfqpoint{4.169774in}{2.399920in}}%
\pgfpathlineto{\pgfqpoint{4.170411in}{2.713968in}}%
\pgfpathlineto{\pgfqpoint{4.170887in}{2.525771in}}%
\pgfpathlineto{\pgfqpoint{4.171204in}{2.644776in}}%
\pgfpathlineto{\pgfqpoint{4.171839in}{2.518339in}}%
\pgfpathlineto{\pgfqpoint{4.171997in}{2.631269in}}%
\pgfpathlineto{\pgfqpoint{4.172313in}{2.516547in}}%
\pgfpathlineto{\pgfqpoint{4.172472in}{2.755247in}}%
\pgfpathlineto{\pgfqpoint{4.173104in}{2.552882in}}%
\pgfpathlineto{\pgfqpoint{4.173262in}{2.540464in}}%
\pgfpathlineto{\pgfqpoint{4.173893in}{2.504257in}}%
\pgfpathlineto{\pgfqpoint{4.174680in}{2.730741in}}%
\pgfpathlineto{\pgfqpoint{4.176094in}{2.420349in}}%
\pgfpathlineto{\pgfqpoint{4.176251in}{2.733019in}}%
\pgfpathlineto{\pgfqpoint{4.177190in}{2.701473in}}%
\pgfpathlineto{\pgfqpoint{4.177816in}{2.318716in}}%
\pgfpathlineto{\pgfqpoint{4.177503in}{2.703314in}}%
\pgfpathlineto{\pgfqpoint{4.178440in}{2.470581in}}%
\pgfpathlineto{\pgfqpoint{4.178596in}{2.773291in}}%
\pgfpathlineto{\pgfqpoint{4.179375in}{2.384740in}}%
\pgfpathlineto{\pgfqpoint{4.179530in}{2.582409in}}%
\pgfpathlineto{\pgfqpoint{4.179686in}{2.711058in}}%
\pgfpathlineto{\pgfqpoint{4.180463in}{2.470133in}}%
\pgfpathlineto{\pgfqpoint{4.180618in}{2.369255in}}%
\pgfpathlineto{\pgfqpoint{4.180773in}{2.687496in}}%
\pgfpathlineto{\pgfqpoint{4.181238in}{2.584900in}}%
\pgfpathlineto{\pgfqpoint{4.181393in}{2.619687in}}%
\pgfpathlineto{\pgfqpoint{4.181702in}{2.539646in}}%
\pgfpathlineto{\pgfqpoint{4.181857in}{2.586970in}}%
\pgfpathlineto{\pgfqpoint{4.182012in}{2.401150in}}%
\pgfpathlineto{\pgfqpoint{4.182166in}{2.651507in}}%
\pgfpathlineto{\pgfqpoint{4.182939in}{2.544210in}}%
\pgfpathlineto{\pgfqpoint{4.183247in}{2.631454in}}%
\pgfpathlineto{\pgfqpoint{4.183710in}{2.390921in}}%
\pgfpathlineto{\pgfqpoint{4.183864in}{2.720881in}}%
\pgfpathlineto{\pgfqpoint{4.184786in}{2.432862in}}%
\pgfpathlineto{\pgfqpoint{4.185247in}{2.718166in}}%
\pgfpathlineto{\pgfqpoint{4.185861in}{2.430732in}}%
\pgfpathlineto{\pgfqpoint{4.186167in}{2.723705in}}%
\pgfpathlineto{\pgfqpoint{4.187238in}{2.608772in}}%
\pgfpathlineto{\pgfqpoint{4.187390in}{2.598379in}}%
\pgfpathlineto{\pgfqpoint{4.187543in}{2.732396in}}%
\pgfpathlineto{\pgfqpoint{4.188001in}{2.423569in}}%
\pgfpathlineto{\pgfqpoint{4.188306in}{2.606077in}}%
\pgfpathlineto{\pgfqpoint{4.188458in}{2.428589in}}%
\pgfpathlineto{\pgfqpoint{4.188915in}{2.683955in}}%
\pgfpathlineto{\pgfqpoint{4.189371in}{2.531548in}}%
\pgfpathlineto{\pgfqpoint{4.189675in}{2.682561in}}%
\pgfpathlineto{\pgfqpoint{4.190130in}{2.494313in}}%
\pgfpathlineto{\pgfqpoint{4.190434in}{2.580837in}}%
\pgfpathlineto{\pgfqpoint{4.190888in}{2.491850in}}%
\pgfpathlineto{\pgfqpoint{4.191342in}{2.730199in}}%
\pgfpathlineto{\pgfqpoint{4.192551in}{2.364611in}}%
\pgfpathlineto{\pgfqpoint{4.193455in}{2.680336in}}%
\pgfpathlineto{\pgfqpoint{4.193756in}{2.567713in}}%
\pgfpathlineto{\pgfqpoint{4.194057in}{2.648688in}}%
\pgfpathlineto{\pgfqpoint{4.194207in}{2.481102in}}%
\pgfpathlineto{\pgfqpoint{4.194357in}{2.726960in}}%
\pgfpathlineto{\pgfqpoint{4.195108in}{2.541714in}}%
\pgfpathlineto{\pgfqpoint{4.195258in}{2.552475in}}%
\pgfpathlineto{\pgfqpoint{4.196156in}{2.730152in}}%
\pgfpathlineto{\pgfqpoint{4.195557in}{2.499940in}}%
\pgfpathlineto{\pgfqpoint{4.196306in}{2.669016in}}%
\pgfpathlineto{\pgfqpoint{4.196754in}{2.457631in}}%
\pgfpathlineto{\pgfqpoint{4.197351in}{2.591118in}}%
\pgfpathlineto{\pgfqpoint{4.197947in}{2.625640in}}%
\pgfpathlineto{\pgfqpoint{4.197798in}{2.566373in}}%
\pgfpathlineto{\pgfqpoint{4.198245in}{2.570793in}}%
\pgfpathlineto{\pgfqpoint{4.198394in}{2.455016in}}%
\pgfpathlineto{\pgfqpoint{4.198691in}{2.708040in}}%
\pgfpathlineto{\pgfqpoint{4.199137in}{2.503245in}}%
\pgfpathlineto{\pgfqpoint{4.200176in}{2.743768in}}%
\pgfpathlineto{\pgfqpoint{4.200324in}{2.590589in}}%
\pgfpathlineto{\pgfqpoint{4.200472in}{2.711168in}}%
\pgfpathlineto{\pgfqpoint{4.201064in}{2.433572in}}%
\pgfpathlineto{\pgfqpoint{4.201212in}{2.640926in}}%
\pgfpathlineto{\pgfqpoint{4.201802in}{2.337060in}}%
\pgfpathlineto{\pgfqpoint{4.201507in}{2.683669in}}%
\pgfpathlineto{\pgfqpoint{4.202392in}{2.515316in}}%
\pgfpathlineto{\pgfqpoint{4.202834in}{2.628890in}}%
\pgfpathlineto{\pgfqpoint{4.202982in}{2.709030in}}%
\pgfpathlineto{\pgfqpoint{4.203423in}{2.428638in}}%
\pgfpathlineto{\pgfqpoint{4.203717in}{2.671149in}}%
\pgfpathlineto{\pgfqpoint{4.204011in}{2.428593in}}%
\pgfpathlineto{\pgfqpoint{4.204304in}{2.749249in}}%
\pgfpathlineto{\pgfqpoint{4.204744in}{2.664073in}}%
\pgfpathlineto{\pgfqpoint{4.205037in}{2.599148in}}%
\pgfpathlineto{\pgfqpoint{4.205477in}{2.662306in}}%
\pgfpathlineto{\pgfqpoint{4.206061in}{2.353234in}}%
\pgfpathlineto{\pgfqpoint{4.206937in}{2.687877in}}%
\pgfpathlineto{\pgfqpoint{4.207083in}{2.300840in}}%
\pgfpathlineto{\pgfqpoint{4.207229in}{2.622864in}}%
\pgfpathlineto{\pgfqpoint{4.207375in}{2.602966in}}%
\pgfpathlineto{\pgfqpoint{4.207520in}{2.680202in}}%
\pgfpathlineto{\pgfqpoint{4.207666in}{2.356993in}}%
\pgfpathlineto{\pgfqpoint{4.208684in}{2.552703in}}%
\pgfpathlineto{\pgfqpoint{4.209554in}{2.684593in}}%
\pgfpathlineto{\pgfqpoint{4.209264in}{2.541061in}}%
\pgfpathlineto{\pgfqpoint{4.209844in}{2.680809in}}%
\pgfpathlineto{\pgfqpoint{4.210857in}{2.464445in}}%
\pgfpathlineto{\pgfqpoint{4.211001in}{2.575731in}}%
\pgfpathlineto{\pgfqpoint{4.211145in}{2.664802in}}%
\pgfpathlineto{\pgfqpoint{4.211434in}{2.523813in}}%
\pgfpathlineto{\pgfqpoint{4.212155in}{2.613548in}}%
\pgfpathlineto{\pgfqpoint{4.212731in}{2.678114in}}%
\pgfpathlineto{\pgfqpoint{4.212875in}{2.512766in}}%
\pgfpathlineto{\pgfqpoint{4.213018in}{2.356260in}}%
\pgfpathlineto{\pgfqpoint{4.213162in}{2.708871in}}%
\pgfpathlineto{\pgfqpoint{4.213880in}{2.604698in}}%
\pgfpathlineto{\pgfqpoint{4.214023in}{2.701068in}}%
\pgfpathlineto{\pgfqpoint{4.214597in}{2.452760in}}%
\pgfpathlineto{\pgfqpoint{4.214740in}{2.379718in}}%
\pgfpathlineto{\pgfqpoint{4.215026in}{2.587536in}}%
\pgfpathlineto{\pgfqpoint{4.215312in}{2.585843in}}%
\pgfpathlineto{\pgfqpoint{4.215598in}{2.620424in}}%
\pgfpathlineto{\pgfqpoint{4.216169in}{2.400574in}}%
\pgfpathlineto{\pgfqpoint{4.216312in}{2.706844in}}%
\pgfpathlineto{\pgfqpoint{4.216597in}{2.496583in}}%
\pgfpathlineto{\pgfqpoint{4.217167in}{2.375732in}}%
\pgfpathlineto{\pgfqpoint{4.217878in}{2.690093in}}%
\pgfpathlineto{\pgfqpoint{4.218446in}{2.426928in}}%
\pgfpathlineto{\pgfqpoint{4.218162in}{2.722897in}}%
\pgfpathlineto{\pgfqpoint{4.219014in}{2.626078in}}%
\pgfpathlineto{\pgfqpoint{4.219155in}{2.650824in}}%
\pgfpathlineto{\pgfqpoint{4.219297in}{2.462649in}}%
\pgfpathlineto{\pgfqpoint{4.219580in}{2.715410in}}%
\pgfpathlineto{\pgfqpoint{4.220146in}{2.666945in}}%
\pgfpathlineto{\pgfqpoint{4.220287in}{2.676746in}}%
\pgfpathlineto{\pgfqpoint{4.220429in}{2.660816in}}%
\pgfpathlineto{\pgfqpoint{4.220993in}{2.688530in}}%
\pgfpathlineto{\pgfqpoint{4.221698in}{2.451137in}}%
\pgfpathlineto{\pgfqpoint{4.221980in}{2.732166in}}%
\pgfpathlineto{\pgfqpoint{4.222823in}{2.692508in}}%
\pgfpathlineto{\pgfqpoint{4.223385in}{2.179303in}}%
\pgfpathlineto{\pgfqpoint{4.223104in}{2.703308in}}%
\pgfpathlineto{\pgfqpoint{4.223946in}{2.601048in}}%
\pgfpathlineto{\pgfqpoint{4.224086in}{2.608277in}}%
\pgfpathlineto{\pgfqpoint{4.224226in}{2.691215in}}%
\pgfpathlineto{\pgfqpoint{4.224925in}{2.555926in}}%
\pgfpathlineto{\pgfqpoint{4.225065in}{2.653091in}}%
\pgfpathlineto{\pgfqpoint{4.225205in}{2.458089in}}%
\pgfpathlineto{\pgfqpoint{4.225763in}{2.673169in}}%
\pgfpathlineto{\pgfqpoint{4.226042in}{2.663063in}}%
\pgfpathlineto{\pgfqpoint{4.226181in}{2.678973in}}%
\pgfpathlineto{\pgfqpoint{4.226460in}{2.351939in}}%
\pgfpathlineto{\pgfqpoint{4.226878in}{2.692099in}}%
\pgfpathlineto{\pgfqpoint{4.227295in}{2.559058in}}%
\pgfpathlineto{\pgfqpoint{4.227434in}{2.562459in}}%
\pgfpathlineto{\pgfqpoint{4.227712in}{2.766169in}}%
\pgfpathlineto{\pgfqpoint{4.227850in}{2.302203in}}%
\pgfpathlineto{\pgfqpoint{4.228821in}{2.688953in}}%
\pgfpathlineto{\pgfqpoint{4.228959in}{2.706791in}}%
\pgfpathlineto{\pgfqpoint{4.229236in}{2.716784in}}%
\pgfpathlineto{\pgfqpoint{4.230066in}{2.516561in}}%
\pgfpathlineto{\pgfqpoint{4.230480in}{2.702364in}}%
\pgfpathlineto{\pgfqpoint{4.231169in}{2.659774in}}%
\pgfpathlineto{\pgfqpoint{4.231857in}{2.408217in}}%
\pgfpathlineto{\pgfqpoint{4.231720in}{2.682102in}}%
\pgfpathlineto{\pgfqpoint{4.232269in}{2.649221in}}%
\pgfpathlineto{\pgfqpoint{4.232407in}{2.653795in}}%
\pgfpathlineto{\pgfqpoint{4.232681in}{2.730163in}}%
\pgfpathlineto{\pgfqpoint{4.232819in}{2.663101in}}%
\pgfpathlineto{\pgfqpoint{4.232956in}{2.302991in}}%
\pgfpathlineto{\pgfqpoint{4.233915in}{2.652627in}}%
\pgfpathlineto{\pgfqpoint{4.234052in}{2.639749in}}%
\pgfpathlineto{\pgfqpoint{4.234188in}{2.650508in}}%
\pgfpathlineto{\pgfqpoint{4.234325in}{2.506479in}}%
\pgfpathlineto{\pgfqpoint{4.234735in}{2.744184in}}%
\pgfpathlineto{\pgfqpoint{4.235281in}{2.575303in}}%
\pgfpathlineto{\pgfqpoint{4.235690in}{2.660876in}}%
\pgfpathlineto{\pgfqpoint{4.236099in}{2.660327in}}%
\pgfpathlineto{\pgfqpoint{4.236235in}{2.297223in}}%
\pgfpathlineto{\pgfqpoint{4.237186in}{2.665233in}}%
\pgfpathlineto{\pgfqpoint{4.237729in}{2.463405in}}%
\pgfpathlineto{\pgfqpoint{4.237594in}{2.722582in}}%
\pgfpathlineto{\pgfqpoint{4.238542in}{2.563249in}}%
\pgfpathlineto{\pgfqpoint{4.239219in}{2.675965in}}%
\pgfpathlineto{\pgfqpoint{4.238813in}{2.477476in}}%
\pgfpathlineto{\pgfqpoint{4.239354in}{2.601856in}}%
\pgfpathlineto{\pgfqpoint{4.239489in}{2.431916in}}%
\pgfpathlineto{\pgfqpoint{4.239624in}{2.704828in}}%
\pgfpathlineto{\pgfqpoint{4.240298in}{2.547424in}}%
\pgfpathlineto{\pgfqpoint{4.241106in}{2.674590in}}%
\pgfpathlineto{\pgfqpoint{4.240568in}{2.464098in}}%
\pgfpathlineto{\pgfqpoint{4.241375in}{2.529797in}}%
\pgfpathlineto{\pgfqpoint{4.242181in}{2.360078in}}%
\pgfpathlineto{\pgfqpoint{4.241913in}{2.601737in}}%
\pgfpathlineto{\pgfqpoint{4.242316in}{2.438286in}}%
\pgfpathlineto{\pgfqpoint{4.243521in}{2.683966in}}%
\pgfpathlineto{\pgfqpoint{4.243922in}{2.314252in}}%
\pgfpathlineto{\pgfqpoint{4.244190in}{2.702337in}}%
\pgfpathlineto{\pgfqpoint{4.244590in}{2.586262in}}%
\pgfpathlineto{\pgfqpoint{4.245257in}{2.688093in}}%
\pgfpathlineto{\pgfqpoint{4.244857in}{2.527108in}}%
\pgfpathlineto{\pgfqpoint{4.245523in}{2.639443in}}%
\pgfpathlineto{\pgfqpoint{4.245656in}{2.503361in}}%
\pgfpathlineto{\pgfqpoint{4.246587in}{2.518781in}}%
\pgfpathlineto{\pgfqpoint{4.246720in}{2.647960in}}%
\pgfpathlineto{\pgfqpoint{4.247516in}{2.396089in}}%
\pgfpathlineto{\pgfqpoint{4.247781in}{2.609711in}}%
\pgfpathlineto{\pgfqpoint{4.248178in}{2.698441in}}%
\pgfpathlineto{\pgfqpoint{4.248046in}{2.500836in}}%
\pgfpathlineto{\pgfqpoint{4.248443in}{2.621029in}}%
\pgfpathlineto{\pgfqpoint{4.248839in}{2.716228in}}%
\pgfpathlineto{\pgfqpoint{4.249367in}{2.386979in}}%
\pgfpathlineto{\pgfqpoint{4.250553in}{2.722679in}}%
\pgfpathlineto{\pgfqpoint{4.250685in}{2.503629in}}%
\pgfpathlineto{\pgfqpoint{4.251605in}{2.608316in}}%
\pgfpathlineto{\pgfqpoint{4.251736in}{2.615810in}}%
\pgfpathlineto{\pgfqpoint{4.251867in}{2.585459in}}%
\pgfpathlineto{\pgfqpoint{4.252129in}{2.600312in}}%
\pgfpathlineto{\pgfqpoint{4.252915in}{2.456587in}}%
\pgfpathlineto{\pgfqpoint{4.252653in}{2.653059in}}%
\pgfpathlineto{\pgfqpoint{4.253177in}{2.571761in}}%
\pgfpathlineto{\pgfqpoint{4.253569in}{2.548433in}}%
\pgfpathlineto{\pgfqpoint{4.253700in}{2.666461in}}%
\pgfpathlineto{\pgfqpoint{4.253830in}{2.395744in}}%
\pgfpathlineto{\pgfqpoint{4.254222in}{2.732749in}}%
\pgfpathlineto{\pgfqpoint{4.254743in}{2.656864in}}%
\pgfpathlineto{\pgfqpoint{4.255654in}{2.423534in}}%
\pgfpathlineto{\pgfqpoint{4.255004in}{2.719993in}}%
\pgfpathlineto{\pgfqpoint{4.256044in}{2.589295in}}%
\pgfpathlineto{\pgfqpoint{4.256563in}{2.753642in}}%
\pgfpathlineto{\pgfqpoint{4.257082in}{2.465421in}}%
\pgfpathlineto{\pgfqpoint{4.257729in}{2.666840in}}%
\pgfpathlineto{\pgfqpoint{4.258247in}{2.526379in}}%
\pgfpathlineto{\pgfqpoint{4.258763in}{2.613684in}}%
\pgfpathlineto{\pgfqpoint{4.258634in}{2.519970in}}%
\pgfpathlineto{\pgfqpoint{4.258892in}{2.553917in}}%
\pgfpathlineto{\pgfqpoint{4.259408in}{2.414097in}}%
\pgfpathlineto{\pgfqpoint{4.260052in}{2.682380in}}%
\pgfpathlineto{\pgfqpoint{4.260566in}{2.198400in}}%
\pgfpathlineto{\pgfqpoint{4.261208in}{2.570645in}}%
\pgfpathlineto{\pgfqpoint{4.261593in}{2.519042in}}%
\pgfpathlineto{\pgfqpoint{4.261978in}{2.689631in}}%
\pgfpathlineto{\pgfqpoint{4.262106in}{2.379495in}}%
\pgfpathlineto{\pgfqpoint{4.262362in}{2.772670in}}%
\pgfpathlineto{\pgfqpoint{4.263129in}{2.457786in}}%
\pgfpathlineto{\pgfqpoint{4.264150in}{2.699144in}}%
\pgfpathlineto{\pgfqpoint{4.264277in}{2.608366in}}%
\pgfpathlineto{\pgfqpoint{4.264786in}{2.540424in}}%
\pgfpathlineto{\pgfqpoint{4.264914in}{2.714785in}}%
\pgfpathlineto{\pgfqpoint{4.265041in}{2.556353in}}%
\pgfpathlineto{\pgfqpoint{4.265168in}{2.715643in}}%
\pgfpathlineto{\pgfqpoint{4.265295in}{2.438430in}}%
\pgfpathlineto{\pgfqpoint{4.266057in}{2.689709in}}%
\pgfpathlineto{\pgfqpoint{4.267324in}{2.468161in}}%
\pgfpathlineto{\pgfqpoint{4.268334in}{2.665399in}}%
\pgfpathlineto{\pgfqpoint{4.267829in}{2.455775in}}%
\pgfpathlineto{\pgfqpoint{4.268460in}{2.662637in}}%
\pgfpathlineto{\pgfqpoint{4.268587in}{2.655598in}}%
\pgfpathlineto{\pgfqpoint{4.269343in}{2.377780in}}%
\pgfpathlineto{\pgfqpoint{4.269091in}{2.686310in}}%
\pgfpathlineto{\pgfqpoint{4.269720in}{2.584000in}}%
\pgfpathlineto{\pgfqpoint{4.269846in}{2.700739in}}%
\pgfpathlineto{\pgfqpoint{4.270600in}{2.508458in}}%
\pgfpathlineto{\pgfqpoint{4.270851in}{2.660409in}}%
\pgfpathlineto{\pgfqpoint{4.271227in}{2.434518in}}%
\pgfpathlineto{\pgfqpoint{4.271352in}{2.681443in}}%
\pgfpathlineto{\pgfqpoint{4.271853in}{2.513669in}}%
\pgfpathlineto{\pgfqpoint{4.272853in}{2.668006in}}%
\pgfpathlineto{\pgfqpoint{4.272103in}{2.368895in}}%
\pgfpathlineto{\pgfqpoint{4.272978in}{2.548425in}}%
\pgfpathlineto{\pgfqpoint{4.273726in}{2.700968in}}%
\pgfpathlineto{\pgfqpoint{4.273601in}{2.520649in}}%
\pgfpathlineto{\pgfqpoint{4.273850in}{2.567129in}}%
\pgfpathlineto{\pgfqpoint{4.273975in}{2.446439in}}%
\pgfpathlineto{\pgfqpoint{4.274473in}{2.719345in}}%
\pgfpathlineto{\pgfqpoint{4.274846in}{2.624658in}}%
\pgfpathlineto{\pgfqpoint{4.274970in}{2.631777in}}%
\pgfpathlineto{\pgfqpoint{4.275591in}{2.325862in}}%
\pgfpathlineto{\pgfqpoint{4.275467in}{2.687366in}}%
\pgfpathlineto{\pgfqpoint{4.275963in}{2.605484in}}%
\pgfpathlineto{\pgfqpoint{4.276087in}{2.634591in}}%
\pgfpathlineto{\pgfqpoint{4.276334in}{2.563127in}}%
\pgfpathlineto{\pgfqpoint{4.276829in}{2.673991in}}%
\pgfpathlineto{\pgfqpoint{4.277447in}{2.378407in}}%
\pgfpathlineto{\pgfqpoint{4.278434in}{2.711198in}}%
\pgfpathlineto{\pgfqpoint{4.278681in}{2.700662in}}%
\pgfpathlineto{\pgfqpoint{4.278804in}{2.387067in}}%
\pgfpathlineto{\pgfqpoint{4.279296in}{2.798833in}}%
\pgfpathlineto{\pgfqpoint{4.279787in}{2.487141in}}%
\pgfpathlineto{\pgfqpoint{4.279910in}{2.711154in}}%
\pgfpathlineto{\pgfqpoint{4.280278in}{2.331062in}}%
\pgfpathlineto{\pgfqpoint{4.280891in}{2.702531in}}%
\pgfpathlineto{\pgfqpoint{4.281259in}{2.297037in}}%
\pgfpathlineto{\pgfqpoint{4.281626in}{2.730970in}}%
\pgfpathlineto{\pgfqpoint{4.282115in}{2.471630in}}%
\pgfpathlineto{\pgfqpoint{4.282969in}{2.672208in}}%
\pgfpathlineto{\pgfqpoint{4.283213in}{2.596556in}}%
\pgfpathlineto{\pgfqpoint{4.283578in}{2.431390in}}%
\pgfpathlineto{\pgfqpoint{4.283700in}{2.635131in}}%
\pgfpathlineto{\pgfqpoint{4.284308in}{2.462263in}}%
\pgfpathlineto{\pgfqpoint{4.284429in}{2.715499in}}%
\pgfpathlineto{\pgfqpoint{4.285400in}{2.616503in}}%
\pgfpathlineto{\pgfqpoint{4.286006in}{2.550800in}}%
\pgfpathlineto{\pgfqpoint{4.286248in}{2.656734in}}%
\pgfpathlineto{\pgfqpoint{4.286369in}{2.696684in}}%
\pgfpathlineto{\pgfqpoint{4.286490in}{2.529810in}}%
\pgfpathlineto{\pgfqpoint{4.287094in}{2.597939in}}%
\pgfpathlineto{\pgfqpoint{4.287576in}{2.500235in}}%
\pgfpathlineto{\pgfqpoint{4.287697in}{2.685656in}}%
\pgfpathlineto{\pgfqpoint{4.288299in}{2.521712in}}%
\pgfpathlineto{\pgfqpoint{4.288420in}{2.509995in}}%
\pgfpathlineto{\pgfqpoint{4.288540in}{2.558650in}}%
\pgfpathlineto{\pgfqpoint{4.289261in}{2.694662in}}%
\pgfpathlineto{\pgfqpoint{4.289381in}{2.552903in}}%
\pgfpathlineto{\pgfqpoint{4.289621in}{2.672183in}}%
\pgfpathlineto{\pgfqpoint{4.289861in}{2.681873in}}%
\pgfpathlineto{\pgfqpoint{4.291059in}{2.306723in}}%
\pgfpathlineto{\pgfqpoint{4.292015in}{2.678823in}}%
\pgfpathlineto{\pgfqpoint{4.292254in}{2.637815in}}%
\pgfpathlineto{\pgfqpoint{4.292373in}{2.529217in}}%
\pgfpathlineto{\pgfqpoint{4.292969in}{2.687942in}}%
\pgfpathlineto{\pgfqpoint{4.293326in}{2.599860in}}%
\pgfpathlineto{\pgfqpoint{4.293564in}{2.659703in}}%
\pgfpathlineto{\pgfqpoint{4.293682in}{2.560231in}}%
\pgfpathlineto{\pgfqpoint{4.293801in}{2.614237in}}%
\pgfpathlineto{\pgfqpoint{4.294395in}{2.283402in}}%
\pgfpathlineto{\pgfqpoint{4.294869in}{2.623973in}}%
\pgfpathlineto{\pgfqpoint{4.295343in}{2.675145in}}%
\pgfpathlineto{\pgfqpoint{4.296053in}{2.472654in}}%
\pgfpathlineto{\pgfqpoint{4.296998in}{2.688330in}}%
\pgfpathlineto{\pgfqpoint{4.297116in}{2.589632in}}%
\pgfpathlineto{\pgfqpoint{4.297234in}{2.430569in}}%
\pgfpathlineto{\pgfqpoint{4.298176in}{2.642094in}}%
\pgfpathlineto{\pgfqpoint{4.298293in}{2.699388in}}%
\pgfpathlineto{\pgfqpoint{4.298646in}{2.523214in}}%
\pgfpathlineto{\pgfqpoint{4.298763in}{2.610745in}}%
\pgfpathlineto{\pgfqpoint{4.298881in}{2.373236in}}%
\pgfpathlineto{\pgfqpoint{4.299819in}{2.677300in}}%
\pgfpathlineto{\pgfqpoint{4.300404in}{2.422881in}}%
\pgfpathlineto{\pgfqpoint{4.300872in}{2.607019in}}%
\pgfpathlineto{\pgfqpoint{4.300989in}{2.631692in}}%
\pgfpathlineto{\pgfqpoint{4.301223in}{2.590529in}}%
\pgfpathlineto{\pgfqpoint{4.301339in}{2.273825in}}%
\pgfpathlineto{\pgfqpoint{4.301456in}{2.714545in}}%
\pgfpathlineto{\pgfqpoint{4.302272in}{2.604167in}}%
\pgfpathlineto{\pgfqpoint{4.302389in}{2.666107in}}%
\pgfpathlineto{\pgfqpoint{4.302854in}{2.372736in}}%
\pgfpathlineto{\pgfqpoint{4.303203in}{2.526244in}}%
\pgfpathlineto{\pgfqpoint{4.303436in}{2.651357in}}%
\pgfpathlineto{\pgfqpoint{4.304364in}{2.603090in}}%
\pgfpathlineto{\pgfqpoint{4.304596in}{2.285362in}}%
\pgfpathlineto{\pgfqpoint{4.304943in}{2.631865in}}%
\pgfpathlineto{\pgfqpoint{4.305521in}{2.442416in}}%
\pgfpathlineto{\pgfqpoint{4.306445in}{2.704862in}}%
\pgfpathlineto{\pgfqpoint{4.305868in}{2.429407in}}%
\pgfpathlineto{\pgfqpoint{4.306676in}{2.545417in}}%
\pgfpathlineto{\pgfqpoint{4.306906in}{2.449628in}}%
\pgfpathlineto{\pgfqpoint{4.307252in}{2.702207in}}%
\pgfpathlineto{\pgfqpoint{4.307482in}{2.436035in}}%
\pgfpathlineto{\pgfqpoint{4.307712in}{2.706973in}}%
\pgfpathlineto{\pgfqpoint{4.308287in}{2.607656in}}%
\pgfpathlineto{\pgfqpoint{4.308631in}{2.517300in}}%
\pgfpathlineto{\pgfqpoint{4.309434in}{2.681929in}}%
\pgfpathlineto{\pgfqpoint{4.310806in}{2.233028in}}%
\pgfpathlineto{\pgfqpoint{4.310235in}{2.718821in}}%
\pgfpathlineto{\pgfqpoint{4.310920in}{2.499150in}}%
\pgfpathlineto{\pgfqpoint{4.311946in}{2.727264in}}%
\pgfpathlineto{\pgfqpoint{4.311376in}{2.465050in}}%
\pgfpathlineto{\pgfqpoint{4.312174in}{2.599151in}}%
\pgfpathlineto{\pgfqpoint{4.312856in}{2.398921in}}%
\pgfpathlineto{\pgfqpoint{4.312401in}{2.711177in}}%
\pgfpathlineto{\pgfqpoint{4.313197in}{2.525765in}}%
\pgfpathlineto{\pgfqpoint{4.313991in}{2.664281in}}%
\pgfpathlineto{\pgfqpoint{4.313651in}{2.473676in}}%
\pgfpathlineto{\pgfqpoint{4.314331in}{2.548913in}}%
\pgfpathlineto{\pgfqpoint{4.314896in}{2.418898in}}%
\pgfpathlineto{\pgfqpoint{4.315009in}{2.673844in}}%
\pgfpathlineto{\pgfqpoint{4.315461in}{2.442223in}}%
\pgfpathlineto{\pgfqpoint{4.315687in}{2.701682in}}%
\pgfpathlineto{\pgfqpoint{4.316251in}{2.398027in}}%
\pgfpathlineto{\pgfqpoint{4.316589in}{2.544526in}}%
\pgfpathlineto{\pgfqpoint{4.317264in}{2.384316in}}%
\pgfpathlineto{\pgfqpoint{4.317377in}{2.700208in}}%
\pgfpathlineto{\pgfqpoint{4.317601in}{2.573732in}}%
\pgfpathlineto{\pgfqpoint{4.318387in}{2.545566in}}%
\pgfpathlineto{\pgfqpoint{4.318835in}{2.662217in}}%
\pgfpathlineto{\pgfqpoint{4.319619in}{2.709583in}}%
\pgfpathlineto{\pgfqpoint{4.320178in}{2.272300in}}%
\pgfpathlineto{\pgfqpoint{4.321070in}{2.724678in}}%
\pgfpathlineto{\pgfqpoint{4.321293in}{2.641900in}}%
\pgfpathlineto{\pgfqpoint{4.321738in}{2.363689in}}%
\pgfpathlineto{\pgfqpoint{4.322294in}{2.668105in}}%
\pgfpathlineto{\pgfqpoint{4.322405in}{2.713232in}}%
\pgfpathlineto{\pgfqpoint{4.322516in}{2.569480in}}%
\pgfpathlineto{\pgfqpoint{4.322960in}{2.641134in}}%
\pgfpathlineto{\pgfqpoint{4.323404in}{2.402798in}}%
\pgfpathlineto{\pgfqpoint{4.323626in}{2.722965in}}%
\pgfpathlineto{\pgfqpoint{4.323958in}{2.697409in}}%
\pgfpathlineto{\pgfqpoint{4.325394in}{2.148282in}}%
\pgfpathlineto{\pgfqpoint{4.324511in}{2.697849in}}%
\pgfpathlineto{\pgfqpoint{4.325504in}{2.474626in}}%
\pgfpathlineto{\pgfqpoint{4.326166in}{2.705676in}}%
\pgfpathlineto{\pgfqpoint{4.326716in}{2.664059in}}%
\pgfpathlineto{\pgfqpoint{4.327375in}{2.348614in}}%
\pgfpathlineto{\pgfqpoint{4.327595in}{2.680336in}}%
\pgfpathlineto{\pgfqpoint{4.327924in}{2.617519in}}%
\pgfpathlineto{\pgfqpoint{4.328143in}{2.711930in}}%
\pgfpathlineto{\pgfqpoint{4.328581in}{2.511515in}}%
\pgfpathlineto{\pgfqpoint{4.329128in}{2.249653in}}%
\pgfpathlineto{\pgfqpoint{4.329456in}{2.669240in}}%
\pgfpathlineto{\pgfqpoint{4.329565in}{2.582883in}}%
\pgfpathlineto{\pgfqpoint{4.330002in}{2.322093in}}%
\pgfpathlineto{\pgfqpoint{4.329784in}{2.688433in}}%
\pgfpathlineto{\pgfqpoint{4.330111in}{2.557280in}}%
\pgfpathlineto{\pgfqpoint{4.330220in}{2.756127in}}%
\pgfpathlineto{\pgfqpoint{4.331092in}{2.402704in}}%
\pgfpathlineto{\pgfqpoint{4.331744in}{2.395520in}}%
\pgfpathlineto{\pgfqpoint{4.332396in}{2.711002in}}%
\pgfpathlineto{\pgfqpoint{4.333696in}{2.437468in}}%
\pgfpathlineto{\pgfqpoint{4.334020in}{2.695929in}}%
\pgfpathlineto{\pgfqpoint{4.333912in}{2.425070in}}%
\pgfpathlineto{\pgfqpoint{4.334776in}{2.494982in}}%
\pgfpathlineto{\pgfqpoint{4.334884in}{2.222312in}}%
\pgfpathlineto{\pgfqpoint{4.335423in}{2.680556in}}%
\pgfpathlineto{\pgfqpoint{4.335746in}{2.504198in}}%
\pgfpathlineto{\pgfqpoint{4.336069in}{2.706418in}}%
\pgfpathlineto{\pgfqpoint{4.336284in}{2.373055in}}%
\pgfpathlineto{\pgfqpoint{4.336929in}{2.646820in}}%
\pgfpathlineto{\pgfqpoint{4.337143in}{2.437126in}}%
\pgfpathlineto{\pgfqpoint{4.338001in}{2.594995in}}%
\pgfpathlineto{\pgfqpoint{4.338108in}{2.688091in}}%
\pgfpathlineto{\pgfqpoint{4.338429in}{2.443905in}}%
\pgfpathlineto{\pgfqpoint{4.338964in}{2.595734in}}%
\pgfpathlineto{\pgfqpoint{4.339604in}{2.406826in}}%
\pgfpathlineto{\pgfqpoint{4.339498in}{2.653210in}}%
\pgfpathlineto{\pgfqpoint{4.340031in}{2.480278in}}%
\pgfpathlineto{\pgfqpoint{4.340989in}{2.717023in}}%
\pgfpathlineto{\pgfqpoint{4.340350in}{2.456878in}}%
\pgfpathlineto{\pgfqpoint{4.341095in}{2.616721in}}%
\pgfpathlineto{\pgfqpoint{4.341308in}{2.405989in}}%
\pgfpathlineto{\pgfqpoint{4.341626in}{2.642606in}}%
\pgfpathlineto{\pgfqpoint{4.342369in}{2.475457in}}%
\pgfpathlineto{\pgfqpoint{4.343427in}{2.733660in}}%
\pgfpathlineto{\pgfqpoint{4.343956in}{2.224061in}}%
\pgfpathlineto{\pgfqpoint{4.344589in}{2.557098in}}%
\pgfpathlineto{\pgfqpoint{4.344694in}{2.564206in}}%
\pgfpathlineto{\pgfqpoint{4.344799in}{2.648689in}}%
\pgfpathlineto{\pgfqpoint{4.345431in}{2.512524in}}%
\pgfpathlineto{\pgfqpoint{4.345536in}{2.581453in}}%
\pgfpathlineto{\pgfqpoint{4.346482in}{2.639377in}}%
\pgfpathlineto{\pgfqpoint{4.346587in}{2.331194in}}%
\pgfpathlineto{\pgfqpoint{4.347635in}{2.643043in}}%
\pgfpathlineto{\pgfqpoint{4.347740in}{2.556693in}}%
\pgfpathlineto{\pgfqpoint{4.348158in}{2.692688in}}%
\pgfpathlineto{\pgfqpoint{4.348263in}{2.654730in}}%
\pgfpathlineto{\pgfqpoint{4.348472in}{2.098933in}}%
\pgfpathlineto{\pgfqpoint{4.348889in}{2.663864in}}%
\pgfpathlineto{\pgfqpoint{4.349411in}{2.516069in}}%
\pgfpathlineto{\pgfqpoint{4.349619in}{2.741344in}}%
\pgfpathlineto{\pgfqpoint{4.349932in}{2.237410in}}%
\pgfpathlineto{\pgfqpoint{4.350348in}{2.588200in}}%
\pgfpathlineto{\pgfqpoint{4.350452in}{2.407407in}}%
\pgfpathlineto{\pgfqpoint{4.351076in}{2.636357in}}%
\pgfpathlineto{\pgfqpoint{4.351387in}{2.603460in}}%
\pgfpathlineto{\pgfqpoint{4.351491in}{2.725040in}}%
\pgfpathlineto{\pgfqpoint{4.352216in}{2.512132in}}%
\pgfpathlineto{\pgfqpoint{4.352527in}{2.688169in}}%
\pgfpathlineto{\pgfqpoint{4.353044in}{2.403459in}}%
\pgfpathlineto{\pgfqpoint{4.353663in}{2.663982in}}%
\pgfpathlineto{\pgfqpoint{4.353767in}{2.659789in}}%
\pgfpathlineto{\pgfqpoint{4.354591in}{2.703551in}}%
\pgfpathlineto{\pgfqpoint{4.355003in}{2.089477in}}%
\pgfpathlineto{\pgfqpoint{4.355517in}{2.642967in}}%
\pgfpathlineto{\pgfqpoint{4.356133in}{2.549431in}}%
\pgfpathlineto{\pgfqpoint{4.356851in}{2.656010in}}%
\pgfpathlineto{\pgfqpoint{4.356953in}{2.457194in}}%
\pgfpathlineto{\pgfqpoint{4.357056in}{2.700559in}}%
\pgfpathlineto{\pgfqpoint{4.357976in}{2.432394in}}%
\pgfpathlineto{\pgfqpoint{4.358078in}{2.610872in}}%
\pgfpathlineto{\pgfqpoint{4.358180in}{2.605640in}}%
\pgfpathlineto{\pgfqpoint{4.358996in}{2.608524in}}%
\pgfpathlineto{\pgfqpoint{4.359607in}{2.259387in}}%
\pgfpathlineto{\pgfqpoint{4.360116in}{2.728707in}}%
\pgfpathlineto{\pgfqpoint{4.360827in}{2.650203in}}%
\pgfpathlineto{\pgfqpoint{4.361739in}{2.333303in}}%
\pgfpathlineto{\pgfqpoint{4.361638in}{2.668313in}}%
\pgfpathlineto{\pgfqpoint{4.361941in}{2.546314in}}%
\pgfpathlineto{\pgfqpoint{4.362649in}{2.642631in}}%
\pgfpathlineto{\pgfqpoint{4.362548in}{2.415770in}}%
\pgfpathlineto{\pgfqpoint{4.362952in}{2.520034in}}%
\pgfpathlineto{\pgfqpoint{4.363053in}{2.346458in}}%
\pgfpathlineto{\pgfqpoint{4.363759in}{2.679505in}}%
\pgfpathlineto{\pgfqpoint{4.363961in}{2.568651in}}%
\pgfpathlineto{\pgfqpoint{4.364665in}{2.754196in}}%
\pgfpathlineto{\pgfqpoint{4.364464in}{2.466860in}}%
\pgfpathlineto{\pgfqpoint{4.365067in}{2.612120in}}%
\pgfpathlineto{\pgfqpoint{4.365569in}{2.460739in}}%
\pgfpathlineto{\pgfqpoint{4.365870in}{2.677068in}}%
\pgfpathlineto{\pgfqpoint{4.366171in}{2.604840in}}%
\pgfpathlineto{\pgfqpoint{4.366271in}{2.707184in}}%
\pgfpathlineto{\pgfqpoint{4.366572in}{2.217286in}}%
\pgfpathlineto{\pgfqpoint{4.367272in}{2.621033in}}%
\pgfpathlineto{\pgfqpoint{4.367871in}{2.712467in}}%
\pgfpathlineto{\pgfqpoint{4.368170in}{2.511773in}}%
\pgfpathlineto{\pgfqpoint{4.368569in}{2.670393in}}%
\pgfpathlineto{\pgfqpoint{4.369167in}{2.647112in}}%
\pgfpathlineto{\pgfqpoint{4.369266in}{2.425599in}}%
\pgfpathlineto{\pgfqpoint{4.369862in}{2.695740in}}%
\pgfpathlineto{\pgfqpoint{4.370260in}{2.574548in}}%
\pgfpathlineto{\pgfqpoint{4.370656in}{2.492464in}}%
\pgfpathlineto{\pgfqpoint{4.371251in}{2.678284in}}%
\pgfpathlineto{\pgfqpoint{4.372042in}{2.352130in}}%
\pgfpathlineto{\pgfqpoint{4.371548in}{2.713422in}}%
\pgfpathlineto{\pgfqpoint{4.372339in}{2.593920in}}%
\pgfpathlineto{\pgfqpoint{4.373029in}{2.707648in}}%
\pgfpathlineto{\pgfqpoint{4.372635in}{2.506472in}}%
\pgfpathlineto{\pgfqpoint{4.373128in}{2.564566in}}%
\pgfpathlineto{\pgfqpoint{4.373227in}{2.473275in}}%
\pgfpathlineto{\pgfqpoint{4.373621in}{2.745842in}}%
\pgfpathlineto{\pgfqpoint{4.374211in}{2.546285in}}%
\pgfpathlineto{\pgfqpoint{4.374506in}{2.742808in}}%
\pgfpathlineto{\pgfqpoint{4.374408in}{2.333079in}}%
\pgfpathlineto{\pgfqpoint{4.375291in}{2.664522in}}%
\pgfpathlineto{\pgfqpoint{4.375683in}{2.351228in}}%
\pgfpathlineto{\pgfqpoint{4.375487in}{2.694554in}}%
\pgfpathlineto{\pgfqpoint{4.376369in}{2.673547in}}%
\pgfpathlineto{\pgfqpoint{4.377248in}{2.485195in}}%
\pgfpathlineto{\pgfqpoint{4.377638in}{2.522364in}}%
\pgfpathlineto{\pgfqpoint{4.377736in}{2.690643in}}%
\pgfpathlineto{\pgfqpoint{4.378613in}{2.455963in}}%
\pgfpathlineto{\pgfqpoint{4.378807in}{2.649720in}}%
\pgfpathlineto{\pgfqpoint{4.379487in}{2.371499in}}%
\pgfpathlineto{\pgfqpoint{4.379876in}{2.444756in}}%
\pgfpathlineto{\pgfqpoint{4.380748in}{2.668799in}}%
\pgfpathlineto{\pgfqpoint{4.380360in}{2.224835in}}%
\pgfpathlineto{\pgfqpoint{4.380942in}{2.511397in}}%
\pgfpathlineto{\pgfqpoint{4.381038in}{2.315479in}}%
\pgfpathlineto{\pgfqpoint{4.381908in}{2.656982in}}%
\pgfpathlineto{\pgfqpoint{4.382005in}{2.607931in}}%
\pgfpathlineto{\pgfqpoint{4.382101in}{2.611063in}}%
\pgfpathlineto{\pgfqpoint{4.382487in}{2.320841in}}%
\pgfpathlineto{\pgfqpoint{4.382391in}{2.685107in}}%
\pgfpathlineto{\pgfqpoint{4.383258in}{2.345308in}}%
\pgfpathlineto{\pgfqpoint{4.384507in}{2.689377in}}%
\pgfpathlineto{\pgfqpoint{4.384891in}{2.455031in}}%
\pgfpathlineto{\pgfqpoint{4.385370in}{2.702300in}}%
\pgfpathlineto{\pgfqpoint{4.385657in}{2.509404in}}%
\pgfpathlineto{\pgfqpoint{4.385849in}{2.680002in}}%
\pgfpathlineto{\pgfqpoint{4.386709in}{2.424298in}}%
\pgfpathlineto{\pgfqpoint{4.386804in}{2.611762in}}%
\pgfpathlineto{\pgfqpoint{4.386900in}{2.369807in}}%
\pgfpathlineto{\pgfqpoint{4.387090in}{2.658217in}}%
\pgfpathlineto{\pgfqpoint{4.387853in}{2.387749in}}%
\pgfpathlineto{\pgfqpoint{4.388519in}{2.651709in}}%
\pgfpathlineto{\pgfqpoint{4.388899in}{2.550202in}}%
\pgfpathlineto{\pgfqpoint{4.388994in}{2.316075in}}%
\pgfpathlineto{\pgfqpoint{4.389942in}{2.652780in}}%
\pgfpathlineto{\pgfqpoint{4.390037in}{2.680201in}}%
\pgfpathlineto{\pgfqpoint{4.390226in}{2.444584in}}%
\pgfpathlineto{\pgfqpoint{4.390321in}{2.633780in}}%
\pgfpathlineto{\pgfqpoint{4.390416in}{2.342245in}}%
\pgfpathlineto{\pgfqpoint{4.390794in}{2.695457in}}%
\pgfpathlineto{\pgfqpoint{4.391456in}{2.551952in}}%
\pgfpathlineto{\pgfqpoint{4.391833in}{2.263720in}}%
\pgfpathlineto{\pgfqpoint{4.391644in}{2.718716in}}%
\pgfpathlineto{\pgfqpoint{4.392493in}{2.369565in}}%
\pgfpathlineto{\pgfqpoint{4.392963in}{2.729877in}}%
\pgfpathlineto{\pgfqpoint{4.393621in}{2.470775in}}%
\pgfpathlineto{\pgfqpoint{4.394372in}{2.652223in}}%
\pgfpathlineto{\pgfqpoint{4.394653in}{2.620322in}}%
\pgfpathlineto{\pgfqpoint{4.395589in}{2.293019in}}%
\pgfpathlineto{\pgfqpoint{4.395683in}{2.642311in}}%
\pgfpathlineto{\pgfqpoint{4.395776in}{2.429739in}}%
\pgfpathlineto{\pgfqpoint{4.396616in}{2.681125in}}%
\pgfpathlineto{\pgfqpoint{4.396710in}{2.188426in}}%
\pgfpathlineto{\pgfqpoint{4.396896in}{2.596968in}}%
\pgfpathlineto{\pgfqpoint{4.397734in}{2.469927in}}%
\pgfpathlineto{\pgfqpoint{4.397827in}{2.639534in}}%
\pgfpathlineto{\pgfqpoint{4.397920in}{2.646266in}}%
\pgfpathlineto{\pgfqpoint{4.398199in}{2.466567in}}%
\pgfpathlineto{\pgfqpoint{4.398385in}{2.667704in}}%
\pgfpathlineto{\pgfqpoint{4.399035in}{2.517225in}}%
\pgfpathlineto{\pgfqpoint{4.399405in}{2.737529in}}%
\pgfpathlineto{\pgfqpoint{4.399961in}{2.410384in}}%
\pgfpathlineto{\pgfqpoint{4.400146in}{2.554382in}}%
\pgfpathlineto{\pgfqpoint{4.400516in}{2.702282in}}%
\pgfpathlineto{\pgfqpoint{4.401254in}{2.422794in}}%
\pgfpathlineto{\pgfqpoint{4.402268in}{2.662486in}}%
\pgfpathlineto{\pgfqpoint{4.402084in}{2.351753in}}%
\pgfpathlineto{\pgfqpoint{4.402360in}{2.546020in}}%
\pgfpathlineto{\pgfqpoint{4.403095in}{2.270242in}}%
\pgfpathlineto{\pgfqpoint{4.402544in}{2.696746in}}%
\pgfpathlineto{\pgfqpoint{4.403463in}{2.538689in}}%
\pgfpathlineto{\pgfqpoint{4.404013in}{2.404356in}}%
\pgfpathlineto{\pgfqpoint{4.404288in}{2.662557in}}%
\pgfpathlineto{\pgfqpoint{4.404471in}{2.522502in}}%
\pgfpathlineto{\pgfqpoint{4.405020in}{2.662887in}}%
\pgfpathlineto{\pgfqpoint{4.405294in}{2.429416in}}%
\pgfpathlineto{\pgfqpoint{4.405660in}{2.591234in}}%
\pgfpathlineto{\pgfqpoint{4.406480in}{2.302296in}}%
\pgfpathlineto{\pgfqpoint{4.406025in}{2.708933in}}%
\pgfpathlineto{\pgfqpoint{4.406663in}{2.567004in}}%
\pgfpathlineto{\pgfqpoint{4.406754in}{2.707639in}}%
\pgfpathlineto{\pgfqpoint{4.407663in}{2.471252in}}%
\pgfpathlineto{\pgfqpoint{4.407754in}{2.411416in}}%
\pgfpathlineto{\pgfqpoint{4.408027in}{2.672219in}}%
\pgfpathlineto{\pgfqpoint{4.408390in}{2.545178in}}%
\pgfpathlineto{\pgfqpoint{4.408480in}{2.691015in}}%
\pgfpathlineto{\pgfqpoint{4.409205in}{2.445115in}}%
\pgfpathlineto{\pgfqpoint{4.409477in}{2.507334in}}%
\pgfpathlineto{\pgfqpoint{4.409839in}{2.693549in}}%
\pgfpathlineto{\pgfqpoint{4.410110in}{2.500124in}}%
\pgfpathlineto{\pgfqpoint{4.410652in}{2.593477in}}%
\pgfpathlineto{\pgfqpoint{4.411463in}{2.231767in}}%
\pgfpathlineto{\pgfqpoint{4.411012in}{2.645967in}}%
\pgfpathlineto{\pgfqpoint{4.411913in}{2.385753in}}%
\pgfpathlineto{\pgfqpoint{4.412901in}{2.705167in}}%
\pgfpathlineto{\pgfqpoint{4.412183in}{2.199371in}}%
\pgfpathlineto{\pgfqpoint{4.412991in}{2.487934in}}%
\pgfpathlineto{\pgfqpoint{4.413081in}{2.306435in}}%
\pgfpathlineto{\pgfqpoint{4.413888in}{2.704806in}}%
\pgfpathlineto{\pgfqpoint{4.413977in}{2.542422in}}%
\pgfpathlineto{\pgfqpoint{4.414871in}{2.653889in}}%
\pgfpathlineto{\pgfqpoint{4.414603in}{2.402770in}}%
\pgfpathlineto{\pgfqpoint{4.415050in}{2.529644in}}%
\pgfpathlineto{\pgfqpoint{4.415229in}{2.377798in}}%
\pgfpathlineto{\pgfqpoint{4.415942in}{2.631420in}}%
\pgfpathlineto{\pgfqpoint{4.416031in}{2.578611in}}%
\pgfpathlineto{\pgfqpoint{4.416921in}{2.666403in}}%
\pgfpathlineto{\pgfqpoint{4.416832in}{2.506566in}}%
\pgfpathlineto{\pgfqpoint{4.417010in}{2.544642in}}%
\pgfpathlineto{\pgfqpoint{4.417099in}{2.488499in}}%
\pgfpathlineto{\pgfqpoint{4.417721in}{2.699252in}}%
\pgfpathlineto{\pgfqpoint{4.417898in}{2.640390in}}%
\pgfpathlineto{\pgfqpoint{4.417987in}{2.639610in}}%
\pgfpathlineto{\pgfqpoint{4.418076in}{2.404599in}}%
\pgfpathlineto{\pgfqpoint{4.418341in}{2.697151in}}%
\pgfpathlineto{\pgfqpoint{4.419050in}{2.567090in}}%
\pgfpathlineto{\pgfqpoint{4.419669in}{2.710824in}}%
\pgfpathlineto{\pgfqpoint{4.419315in}{2.499884in}}%
\pgfpathlineto{\pgfqpoint{4.420198in}{2.673936in}}%
\pgfpathlineto{\pgfqpoint{4.420551in}{2.462304in}}%
\pgfpathlineto{\pgfqpoint{4.421344in}{2.605949in}}%
\pgfpathlineto{\pgfqpoint{4.421432in}{2.676461in}}%
\pgfpathlineto{\pgfqpoint{4.422047in}{2.396203in}}%
\pgfpathlineto{\pgfqpoint{4.422135in}{2.504836in}}%
\pgfpathlineto{\pgfqpoint{4.422223in}{2.350615in}}%
\pgfpathlineto{\pgfqpoint{4.422310in}{2.650734in}}%
\pgfpathlineto{\pgfqpoint{4.423275in}{2.445904in}}%
\pgfpathlineto{\pgfqpoint{4.423713in}{2.681006in}}%
\pgfpathlineto{\pgfqpoint{4.424150in}{2.414380in}}%
\pgfpathlineto{\pgfqpoint{4.424499in}{2.613081in}}%
\pgfpathlineto{\pgfqpoint{4.424761in}{2.704551in}}%
\pgfpathlineto{\pgfqpoint{4.425633in}{2.385526in}}%
\pgfpathlineto{\pgfqpoint{4.426677in}{2.673683in}}%
\pgfpathlineto{\pgfqpoint{4.426764in}{2.575922in}}%
\pgfpathlineto{\pgfqpoint{4.427545in}{2.397207in}}%
\pgfpathlineto{\pgfqpoint{4.426938in}{2.661850in}}%
\pgfpathlineto{\pgfqpoint{4.427892in}{2.558266in}}%
\pgfpathlineto{\pgfqpoint{4.427979in}{2.674142in}}%
\pgfpathlineto{\pgfqpoint{4.428411in}{2.415208in}}%
\pgfpathlineto{\pgfqpoint{4.428930in}{2.568379in}}%
\pgfpathlineto{\pgfqpoint{4.429794in}{2.676212in}}%
\pgfpathlineto{\pgfqpoint{4.430139in}{2.414808in}}%
\pgfpathlineto{\pgfqpoint{4.430914in}{2.633847in}}%
\pgfpathlineto{\pgfqpoint{4.431257in}{2.507559in}}%
\pgfpathlineto{\pgfqpoint{4.431343in}{2.342265in}}%
\pgfpathlineto{\pgfqpoint{4.431773in}{2.668463in}}%
\pgfpathlineto{\pgfqpoint{4.432288in}{2.579894in}}%
\pgfpathlineto{\pgfqpoint{4.432802in}{2.400729in}}%
\pgfpathlineto{\pgfqpoint{4.432545in}{2.690400in}}%
\pgfpathlineto{\pgfqpoint{4.433144in}{2.447867in}}%
\pgfpathlineto{\pgfqpoint{4.433828in}{2.664859in}}%
\pgfpathlineto{\pgfqpoint{4.433486in}{2.425831in}}%
\pgfpathlineto{\pgfqpoint{4.434170in}{2.632590in}}%
\pgfpathlineto{\pgfqpoint{4.435108in}{2.015200in}}%
\pgfpathlineto{\pgfqpoint{4.434341in}{2.705687in}}%
\pgfpathlineto{\pgfqpoint{4.435278in}{2.493492in}}%
\pgfpathlineto{\pgfqpoint{4.436214in}{2.717180in}}%
\pgfpathlineto{\pgfqpoint{4.436129in}{2.378774in}}%
\pgfpathlineto{\pgfqpoint{4.436384in}{2.517376in}}%
\pgfpathlineto{\pgfqpoint{4.436978in}{2.381340in}}%
\pgfpathlineto{\pgfqpoint{4.437147in}{2.631601in}}%
\pgfpathlineto{\pgfqpoint{4.437232in}{2.638701in}}%
\pgfpathlineto{\pgfqpoint{4.437402in}{2.574614in}}%
\pgfpathlineto{\pgfqpoint{4.438502in}{2.384369in}}%
\pgfpathlineto{\pgfqpoint{4.437656in}{2.658613in}}%
\pgfpathlineto{\pgfqpoint{4.438586in}{2.546703in}}%
\pgfpathlineto{\pgfqpoint{4.438924in}{2.702497in}}%
\pgfpathlineto{\pgfqpoint{4.439599in}{2.459181in}}%
\pgfpathlineto{\pgfqpoint{4.439683in}{2.570116in}}%
\pgfpathlineto{\pgfqpoint{4.439851in}{2.390466in}}%
\pgfpathlineto{\pgfqpoint{4.440441in}{2.720662in}}%
\pgfpathlineto{\pgfqpoint{4.440861in}{2.525987in}}%
\pgfpathlineto{\pgfqpoint{4.440945in}{2.641060in}}%
\pgfpathlineto{\pgfqpoint{4.441029in}{2.411420in}}%
\pgfpathlineto{\pgfqpoint{4.441868in}{2.545210in}}%
\pgfpathlineto{\pgfqpoint{4.441952in}{2.444709in}}%
\pgfpathlineto{\pgfqpoint{4.442455in}{2.604443in}}%
\pgfpathlineto{\pgfqpoint{4.442873in}{2.569434in}}%
\pgfpathlineto{\pgfqpoint{4.443040in}{2.668704in}}%
\pgfpathlineto{\pgfqpoint{4.443374in}{2.473801in}}%
\pgfpathlineto{\pgfqpoint{4.443875in}{2.547234in}}%
\pgfpathlineto{\pgfqpoint{4.444209in}{2.432302in}}%
\pgfpathlineto{\pgfqpoint{4.444042in}{2.629179in}}%
\pgfpathlineto{\pgfqpoint{4.444292in}{2.543275in}}%
\pgfpathlineto{\pgfqpoint{4.444376in}{2.698059in}}%
\pgfpathlineto{\pgfqpoint{4.445291in}{2.481743in}}%
\pgfpathlineto{\pgfqpoint{4.445375in}{2.529624in}}%
\pgfpathlineto{\pgfqpoint{4.446039in}{2.454477in}}%
\pgfpathlineto{\pgfqpoint{4.445624in}{2.612516in}}%
\pgfpathlineto{\pgfqpoint{4.446205in}{2.543017in}}%
\pgfpathlineto{\pgfqpoint{4.446288in}{2.647109in}}%
\pgfpathlineto{\pgfqpoint{4.446371in}{2.381583in}}%
\pgfpathlineto{\pgfqpoint{4.447283in}{2.576967in}}%
\pgfpathlineto{\pgfqpoint{4.447365in}{2.406920in}}%
\pgfpathlineto{\pgfqpoint{4.448027in}{2.700621in}}%
\pgfpathlineto{\pgfqpoint{4.448357in}{2.561693in}}%
\pgfpathlineto{\pgfqpoint{4.448605in}{2.702767in}}%
\pgfpathlineto{\pgfqpoint{4.448687in}{2.301291in}}%
\pgfpathlineto{\pgfqpoint{4.449512in}{2.643727in}}%
\pgfpathlineto{\pgfqpoint{4.450498in}{2.407041in}}%
\pgfpathlineto{\pgfqpoint{4.450745in}{2.456526in}}%
\pgfpathlineto{\pgfqpoint{4.450991in}{2.671583in}}%
\pgfpathlineto{\pgfqpoint{4.451483in}{2.344482in}}%
\pgfpathlineto{\pgfqpoint{4.451811in}{2.620428in}}%
\pgfpathlineto{\pgfqpoint{4.452629in}{2.308742in}}%
\pgfpathlineto{\pgfqpoint{4.452056in}{2.723021in}}%
\pgfpathlineto{\pgfqpoint{4.452956in}{2.410745in}}%
\pgfpathlineto{\pgfqpoint{4.453201in}{2.664390in}}%
\pgfpathlineto{\pgfqpoint{4.454098in}{2.554963in}}%
\pgfpathlineto{\pgfqpoint{4.454667in}{2.351511in}}%
\pgfpathlineto{\pgfqpoint{4.454260in}{2.661413in}}%
\pgfpathlineto{\pgfqpoint{4.455155in}{2.538924in}}%
\pgfpathlineto{\pgfqpoint{4.455480in}{2.643267in}}%
\pgfpathlineto{\pgfqpoint{4.455967in}{2.270596in}}%
\pgfpathlineto{\pgfqpoint{4.456210in}{2.541149in}}%
\pgfpathlineto{\pgfqpoint{4.456291in}{2.281997in}}%
\pgfpathlineto{\pgfqpoint{4.456696in}{2.676076in}}%
\pgfpathlineto{\pgfqpoint{4.457262in}{2.563440in}}%
\pgfpathlineto{\pgfqpoint{4.457666in}{2.616960in}}%
\pgfpathlineto{\pgfqpoint{4.458070in}{2.420219in}}%
\pgfpathlineto{\pgfqpoint{4.458231in}{2.527949in}}%
\pgfpathlineto{\pgfqpoint{4.459198in}{2.335918in}}%
\pgfpathlineto{\pgfqpoint{4.458473in}{2.688594in}}%
\pgfpathlineto{\pgfqpoint{4.459279in}{2.531419in}}%
\pgfpathlineto{\pgfqpoint{4.459359in}{2.525491in}}%
\pgfpathlineto{\pgfqpoint{4.459440in}{2.584772in}}%
\pgfpathlineto{\pgfqpoint{4.459520in}{2.609601in}}%
\pgfpathlineto{\pgfqpoint{4.459681in}{2.451165in}}%
\pgfpathlineto{\pgfqpoint{4.459842in}{2.560998in}}%
\pgfpathlineto{\pgfqpoint{4.459922in}{2.218875in}}%
\pgfpathlineto{\pgfqpoint{4.460805in}{2.689257in}}%
\pgfpathlineto{\pgfqpoint{4.460885in}{2.504403in}}%
\pgfpathlineto{\pgfqpoint{4.461606in}{2.662943in}}%
\pgfpathlineto{\pgfqpoint{4.461045in}{2.475182in}}%
\pgfpathlineto{\pgfqpoint{4.461766in}{2.553234in}}%
\pgfpathlineto{\pgfqpoint{4.461846in}{2.294598in}}%
\pgfpathlineto{\pgfqpoint{4.461926in}{2.643793in}}%
\pgfpathlineto{\pgfqpoint{4.462804in}{2.622807in}}%
\pgfpathlineto{\pgfqpoint{4.462884in}{2.638686in}}%
\pgfpathlineto{\pgfqpoint{4.462964in}{2.350561in}}%
\pgfpathlineto{\pgfqpoint{4.463442in}{2.688839in}}%
\pgfpathlineto{\pgfqpoint{4.464000in}{2.607140in}}%
\pgfpathlineto{\pgfqpoint{4.464636in}{2.345448in}}%
\pgfpathlineto{\pgfqpoint{4.464556in}{2.670094in}}%
\pgfpathlineto{\pgfqpoint{4.465192in}{2.488709in}}%
\pgfpathlineto{\pgfqpoint{4.465588in}{2.690628in}}%
\pgfpathlineto{\pgfqpoint{4.466143in}{2.219369in}}%
\pgfpathlineto{\pgfqpoint{4.466301in}{2.620232in}}%
\pgfpathlineto{\pgfqpoint{4.466934in}{2.683429in}}%
\pgfpathlineto{\pgfqpoint{4.466855in}{2.488661in}}%
\pgfpathlineto{\pgfqpoint{4.467250in}{2.644424in}}%
\pgfpathlineto{\pgfqpoint{4.467881in}{2.406305in}}%
\pgfpathlineto{\pgfqpoint{4.467960in}{2.652448in}}%
\pgfpathlineto{\pgfqpoint{4.468275in}{2.433419in}}%
\pgfpathlineto{\pgfqpoint{4.469298in}{2.676329in}}%
\pgfpathlineto{\pgfqpoint{4.468905in}{2.341914in}}%
\pgfpathlineto{\pgfqpoint{4.469376in}{2.669312in}}%
\pgfpathlineto{\pgfqpoint{4.470318in}{2.469770in}}%
\pgfpathlineto{\pgfqpoint{4.470553in}{2.608341in}}%
\pgfpathlineto{\pgfqpoint{4.470788in}{2.358090in}}%
\pgfpathlineto{\pgfqpoint{4.470710in}{2.724036in}}%
\pgfpathlineto{\pgfqpoint{4.471727in}{2.546226in}}%
\pgfpathlineto{\pgfqpoint{4.472039in}{2.644344in}}%
\pgfpathlineto{\pgfqpoint{4.472352in}{2.353621in}}%
\pgfpathlineto{\pgfqpoint{4.472820in}{2.617619in}}%
\pgfpathlineto{\pgfqpoint{4.473131in}{2.275394in}}%
\pgfpathlineto{\pgfqpoint{4.473287in}{2.682822in}}%
\pgfpathlineto{\pgfqpoint{4.473987in}{2.485893in}}%
\pgfpathlineto{\pgfqpoint{4.474919in}{2.691618in}}%
\pgfpathlineto{\pgfqpoint{4.474375in}{2.309773in}}%
\pgfpathlineto{\pgfqpoint{4.475074in}{2.495593in}}%
\pgfpathlineto{\pgfqpoint{4.475616in}{2.671897in}}%
\pgfpathlineto{\pgfqpoint{4.476003in}{2.387096in}}%
\pgfpathlineto{\pgfqpoint{4.476235in}{2.581455in}}%
\pgfpathlineto{\pgfqpoint{4.477316in}{2.324644in}}%
\pgfpathlineto{\pgfqpoint{4.476853in}{2.638606in}}%
\pgfpathlineto{\pgfqpoint{4.477393in}{2.374056in}}%
\pgfpathlineto{\pgfqpoint{4.477547in}{2.679966in}}%
\pgfpathlineto{\pgfqpoint{4.478625in}{2.655929in}}%
\pgfpathlineto{\pgfqpoint{4.478702in}{2.294045in}}%
\pgfpathlineto{\pgfqpoint{4.479776in}{2.369556in}}%
\pgfpathlineto{\pgfqpoint{4.479853in}{2.647315in}}%
\pgfpathlineto{\pgfqpoint{4.480925in}{2.643587in}}%
\pgfpathlineto{\pgfqpoint{4.481383in}{2.236632in}}%
\pgfpathlineto{\pgfqpoint{4.481307in}{2.655213in}}%
\pgfpathlineto{\pgfqpoint{4.481994in}{2.463605in}}%
\pgfpathlineto{\pgfqpoint{4.482528in}{2.623077in}}%
\pgfpathlineto{\pgfqpoint{4.482451in}{2.423546in}}%
\pgfpathlineto{\pgfqpoint{4.482984in}{2.595283in}}%
\pgfpathlineto{\pgfqpoint{4.483060in}{2.377025in}}%
\pgfpathlineto{\pgfqpoint{4.483365in}{2.700113in}}%
\pgfpathlineto{\pgfqpoint{4.484048in}{2.656147in}}%
\pgfpathlineto{\pgfqpoint{4.484807in}{2.385901in}}%
\pgfpathlineto{\pgfqpoint{4.485261in}{2.522933in}}%
\pgfpathlineto{\pgfqpoint{4.485412in}{2.479121in}}%
\pgfpathlineto{\pgfqpoint{4.485639in}{2.688277in}}%
\pgfpathlineto{\pgfqpoint{4.485715in}{2.398227in}}%
\pgfpathlineto{\pgfqpoint{4.486546in}{2.580175in}}%
\pgfpathlineto{\pgfqpoint{4.487450in}{2.649801in}}%
\pgfpathlineto{\pgfqpoint{4.486697in}{2.410605in}}%
\pgfpathlineto{\pgfqpoint{4.487601in}{2.584125in}}%
\pgfpathlineto{\pgfqpoint{4.488278in}{2.353623in}}%
\pgfpathlineto{\pgfqpoint{4.488202in}{2.669975in}}%
\pgfpathlineto{\pgfqpoint{4.488653in}{2.376329in}}%
\pgfpathlineto{\pgfqpoint{4.488953in}{2.709977in}}%
\pgfpathlineto{\pgfqpoint{4.489403in}{2.337791in}}%
\pgfpathlineto{\pgfqpoint{4.489778in}{2.673469in}}%
\pgfpathlineto{\pgfqpoint{4.490302in}{2.423161in}}%
\pgfpathlineto{\pgfqpoint{4.490900in}{2.568275in}}%
\pgfpathlineto{\pgfqpoint{4.491497in}{2.427767in}}%
\pgfpathlineto{\pgfqpoint{4.491422in}{2.656616in}}%
\pgfpathlineto{\pgfqpoint{4.491870in}{2.570544in}}%
\pgfpathlineto{\pgfqpoint{4.491944in}{2.641163in}}%
\pgfpathlineto{\pgfqpoint{4.492391in}{2.318694in}}%
\pgfpathlineto{\pgfqpoint{4.492911in}{2.517022in}}%
\pgfpathlineto{\pgfqpoint{4.493431in}{2.213963in}}%
\pgfpathlineto{\pgfqpoint{4.493580in}{2.651827in}}%
\pgfpathlineto{\pgfqpoint{4.493951in}{2.382768in}}%
\pgfpathlineto{\pgfqpoint{4.494544in}{2.699152in}}%
\pgfpathlineto{\pgfqpoint{4.494840in}{2.331562in}}%
\pgfpathlineto{\pgfqpoint{4.495062in}{2.571566in}}%
\pgfpathlineto{\pgfqpoint{4.495283in}{2.368809in}}%
\pgfpathlineto{\pgfqpoint{4.495653in}{2.696199in}}%
\pgfpathlineto{\pgfqpoint{4.496096in}{2.483214in}}%
\pgfpathlineto{\pgfqpoint{4.496686in}{2.728964in}}%
\pgfpathlineto{\pgfqpoint{4.496538in}{2.058484in}}%
\pgfpathlineto{\pgfqpoint{4.497054in}{2.435868in}}%
\pgfpathlineto{\pgfqpoint{4.497127in}{2.212459in}}%
\pgfpathlineto{\pgfqpoint{4.497936in}{2.640300in}}%
\pgfpathlineto{\pgfqpoint{4.498083in}{2.323006in}}%
\pgfpathlineto{\pgfqpoint{4.498890in}{2.717147in}}%
\pgfpathlineto{\pgfqpoint{4.498303in}{2.232904in}}%
\pgfpathlineto{\pgfqpoint{4.499183in}{2.603817in}}%
\pgfpathlineto{\pgfqpoint{4.499915in}{2.678024in}}%
\pgfpathlineto{\pgfqpoint{4.500353in}{2.369453in}}%
\pgfpathlineto{\pgfqpoint{4.500572in}{2.638839in}}%
\pgfpathlineto{\pgfqpoint{4.501448in}{2.555022in}}%
\pgfpathlineto{\pgfqpoint{4.501739in}{2.336728in}}%
\pgfpathlineto{\pgfqpoint{4.502248in}{2.663573in}}%
\pgfpathlineto{\pgfqpoint{4.502539in}{2.498556in}}%
\pgfpathlineto{\pgfqpoint{4.503192in}{2.687493in}}%
\pgfpathlineto{\pgfqpoint{4.502902in}{2.488372in}}%
\pgfpathlineto{\pgfqpoint{4.503772in}{2.681652in}}%
\pgfpathlineto{\pgfqpoint{4.504424in}{2.173203in}}%
\pgfpathlineto{\pgfqpoint{4.504930in}{2.580395in}}%
\pgfpathlineto{\pgfqpoint{4.505075in}{2.527557in}}%
\pgfpathlineto{\pgfqpoint{4.505147in}{2.651809in}}%
\pgfpathlineto{\pgfqpoint{4.505291in}{2.418191in}}%
\pgfpathlineto{\pgfqpoint{4.506301in}{2.474683in}}%
\pgfpathlineto{\pgfqpoint{4.506948in}{2.662508in}}%
\pgfpathlineto{\pgfqpoint{4.507380in}{2.557000in}}%
\pgfpathlineto{\pgfqpoint{4.508026in}{2.432181in}}%
\pgfpathlineto{\pgfqpoint{4.508169in}{2.650153in}}%
\pgfpathlineto{\pgfqpoint{4.508384in}{2.544553in}}%
\pgfpathlineto{\pgfqpoint{4.509172in}{2.628258in}}%
\pgfpathlineto{\pgfqpoint{4.508528in}{2.409056in}}%
\pgfpathlineto{\pgfqpoint{4.509315in}{2.529093in}}%
\pgfpathlineto{\pgfqpoint{4.509387in}{2.357170in}}%
\pgfpathlineto{\pgfqpoint{4.509744in}{2.640748in}}%
\pgfpathlineto{\pgfqpoint{4.510386in}{2.602762in}}%
\pgfpathlineto{\pgfqpoint{4.510529in}{2.444595in}}%
\pgfpathlineto{\pgfqpoint{4.511384in}{2.665999in}}%
\pgfpathlineto{\pgfqpoint{4.512024in}{2.353068in}}%
\pgfpathlineto{\pgfqpoint{4.512308in}{2.671034in}}%
\pgfpathlineto{\pgfqpoint{4.512450in}{2.639760in}}%
\pgfpathlineto{\pgfqpoint{4.513018in}{2.456021in}}%
\pgfpathlineto{\pgfqpoint{4.513230in}{2.608054in}}%
\pgfpathlineto{\pgfqpoint{4.513301in}{2.196279in}}%
\pgfpathlineto{\pgfqpoint{4.513655in}{2.697555in}}%
\pgfpathlineto{\pgfqpoint{4.514292in}{2.614328in}}%
\pgfpathlineto{\pgfqpoint{4.515068in}{2.328233in}}%
\pgfpathlineto{\pgfqpoint{4.515280in}{2.691065in}}%
\pgfpathlineto{\pgfqpoint{4.515351in}{2.576423in}}%
\pgfpathlineto{\pgfqpoint{4.515421in}{2.612338in}}%
\pgfpathlineto{\pgfqpoint{4.515914in}{2.388263in}}%
\pgfpathlineto{\pgfqpoint{4.516407in}{2.602940in}}%
\pgfpathlineto{\pgfqpoint{4.516547in}{2.388662in}}%
\pgfpathlineto{\pgfqpoint{4.517039in}{2.670935in}}%
\pgfpathlineto{\pgfqpoint{4.517531in}{2.456813in}}%
\pgfpathlineto{\pgfqpoint{4.517881in}{2.674007in}}%
\pgfpathlineto{\pgfqpoint{4.518091in}{2.299386in}}%
\pgfpathlineto{\pgfqpoint{4.518721in}{2.655097in}}%
\pgfpathlineto{\pgfqpoint{4.519001in}{2.421134in}}%
\pgfpathlineto{\pgfqpoint{4.519839in}{2.449842in}}%
\pgfpathlineto{\pgfqpoint{4.520257in}{2.670992in}}%
\pgfpathlineto{\pgfqpoint{4.520327in}{2.445353in}}%
\pgfpathlineto{\pgfqpoint{4.520675in}{2.587946in}}%
\pgfpathlineto{\pgfqpoint{4.520745in}{2.095137in}}%
\pgfpathlineto{\pgfqpoint{4.521232in}{2.681442in}}%
\pgfpathlineto{\pgfqpoint{4.521788in}{2.254072in}}%
\pgfpathlineto{\pgfqpoint{4.522482in}{2.665883in}}%
\pgfpathlineto{\pgfqpoint{4.521927in}{2.225664in}}%
\pgfpathlineto{\pgfqpoint{4.522967in}{2.593543in}}%
\pgfpathlineto{\pgfqpoint{4.523382in}{2.384378in}}%
\pgfpathlineto{\pgfqpoint{4.523866in}{2.657862in}}%
\pgfpathlineto{\pgfqpoint{4.524142in}{2.459780in}}%
\pgfpathlineto{\pgfqpoint{4.524487in}{2.674871in}}%
\pgfpathlineto{\pgfqpoint{4.524694in}{2.289857in}}%
\pgfpathlineto{\pgfqpoint{4.525315in}{2.623314in}}%
\pgfpathlineto{\pgfqpoint{4.525934in}{2.405309in}}%
\pgfpathlineto{\pgfqpoint{4.526278in}{2.624466in}}%
\pgfpathlineto{\pgfqpoint{4.526484in}{2.478104in}}%
\pgfpathlineto{\pgfqpoint{4.527170in}{2.692793in}}%
\pgfpathlineto{\pgfqpoint{4.527513in}{2.659072in}}%
\pgfpathlineto{\pgfqpoint{4.527581in}{2.276119in}}%
\pgfpathlineto{\pgfqpoint{4.528608in}{2.423174in}}%
\pgfpathlineto{\pgfqpoint{4.529427in}{2.650899in}}%
\pgfpathlineto{\pgfqpoint{4.529086in}{2.395042in}}%
\pgfpathlineto{\pgfqpoint{4.529700in}{2.459699in}}%
\pgfpathlineto{\pgfqpoint{4.530108in}{2.662408in}}%
\pgfpathlineto{\pgfqpoint{4.529972in}{2.269716in}}%
\pgfpathlineto{\pgfqpoint{4.531061in}{2.602094in}}%
\pgfpathlineto{\pgfqpoint{4.531807in}{2.247636in}}%
\pgfpathlineto{\pgfqpoint{4.531536in}{2.653896in}}%
\pgfpathlineto{\pgfqpoint{4.532146in}{2.492542in}}%
\pgfpathlineto{\pgfqpoint{4.532282in}{2.669898in}}%
\pgfpathlineto{\pgfqpoint{4.533026in}{2.336374in}}%
\pgfpathlineto{\pgfqpoint{4.533229in}{2.555007in}}%
\pgfpathlineto{\pgfqpoint{4.533432in}{2.432171in}}%
\pgfpathlineto{\pgfqpoint{4.533770in}{2.649824in}}%
\pgfpathlineto{\pgfqpoint{4.534242in}{2.590119in}}%
\pgfpathlineto{\pgfqpoint{4.534309in}{2.587834in}}%
\pgfpathlineto{\pgfqpoint{4.535387in}{2.375789in}}%
\pgfpathlineto{\pgfqpoint{4.534646in}{2.637930in}}%
\pgfpathlineto{\pgfqpoint{4.535454in}{2.524878in}}%
\pgfpathlineto{\pgfqpoint{4.535857in}{2.683088in}}%
\pgfpathlineto{\pgfqpoint{4.535790in}{2.377632in}}%
\pgfpathlineto{\pgfqpoint{4.536797in}{2.628140in}}%
\pgfpathlineto{\pgfqpoint{4.537600in}{2.319789in}}%
\pgfpathlineto{\pgfqpoint{4.537199in}{2.667631in}}%
\pgfpathlineto{\pgfqpoint{4.537868in}{2.374461in}}%
\pgfpathlineto{\pgfqpoint{4.537935in}{2.657207in}}%
\pgfpathlineto{\pgfqpoint{4.538402in}{2.363039in}}%
\pgfpathlineto{\pgfqpoint{4.539003in}{2.491494in}}%
\pgfpathlineto{\pgfqpoint{4.539403in}{2.679180in}}%
\pgfpathlineto{\pgfqpoint{4.539935in}{2.379815in}}%
\pgfpathlineto{\pgfqpoint{4.540068in}{2.500552in}}%
\pgfpathlineto{\pgfqpoint{4.540334in}{2.658409in}}%
\pgfpathlineto{\pgfqpoint{4.541065in}{2.321694in}}%
\pgfpathlineto{\pgfqpoint{4.541595in}{2.664348in}}%
\pgfpathlineto{\pgfqpoint{4.542192in}{2.428616in}}%
\pgfpathlineto{\pgfqpoint{4.543051in}{2.648620in}}%
\pgfpathlineto{\pgfqpoint{4.542588in}{2.374580in}}%
\pgfpathlineto{\pgfqpoint{4.543315in}{2.447932in}}%
\pgfpathlineto{\pgfqpoint{4.543777in}{2.409949in}}%
\pgfpathlineto{\pgfqpoint{4.543579in}{2.707492in}}%
\pgfpathlineto{\pgfqpoint{4.544041in}{2.525306in}}%
\pgfpathlineto{\pgfqpoint{4.545094in}{2.656662in}}%
\pgfpathlineto{\pgfqpoint{4.544436in}{2.352959in}}%
\pgfpathlineto{\pgfqpoint{4.545159in}{2.543183in}}%
\pgfpathlineto{\pgfqpoint{4.545750in}{2.686997in}}%
\pgfpathlineto{\pgfqpoint{4.545291in}{2.395864in}}%
\pgfpathlineto{\pgfqpoint{4.545947in}{2.517844in}}%
\pgfpathlineto{\pgfqpoint{4.546472in}{2.226388in}}%
\pgfpathlineto{\pgfqpoint{4.546734in}{2.638397in}}%
\pgfpathlineto{\pgfqpoint{4.547061in}{2.476015in}}%
\pgfpathlineto{\pgfqpoint{4.547323in}{2.305124in}}%
\pgfpathlineto{\pgfqpoint{4.547976in}{2.654422in}}%
\pgfpathlineto{\pgfqpoint{4.549150in}{2.242982in}}%
\pgfpathlineto{\pgfqpoint{4.549605in}{2.677479in}}%
\pgfpathlineto{\pgfqpoint{4.550320in}{2.602577in}}%
\pgfpathlineto{\pgfqpoint{4.550904in}{2.416981in}}%
\pgfpathlineto{\pgfqpoint{4.551293in}{2.610370in}}%
\pgfpathlineto{\pgfqpoint{4.551422in}{2.584569in}}%
\pgfpathlineto{\pgfqpoint{4.551681in}{2.385575in}}%
\pgfpathlineto{\pgfqpoint{4.551616in}{2.670220in}}%
\pgfpathlineto{\pgfqpoint{4.552586in}{2.470223in}}%
\pgfpathlineto{\pgfqpoint{4.553038in}{2.647762in}}%
\pgfpathlineto{\pgfqpoint{4.552909in}{2.317182in}}%
\pgfpathlineto{\pgfqpoint{4.553683in}{2.568098in}}%
\pgfpathlineto{\pgfqpoint{4.554326in}{2.431303in}}%
\pgfpathlineto{\pgfqpoint{4.554069in}{2.636571in}}%
\pgfpathlineto{\pgfqpoint{4.554391in}{2.578361in}}%
\pgfpathlineto{\pgfqpoint{4.554712in}{2.679106in}}%
\pgfpathlineto{\pgfqpoint{4.555162in}{2.424938in}}%
\pgfpathlineto{\pgfqpoint{4.555418in}{2.639879in}}%
\pgfpathlineto{\pgfqpoint{4.555547in}{2.069898in}}%
\pgfpathlineto{\pgfqpoint{4.556123in}{2.715232in}}%
\pgfpathlineto{\pgfqpoint{4.556571in}{2.363481in}}%
\pgfpathlineto{\pgfqpoint{4.557019in}{2.661530in}}%
\pgfpathlineto{\pgfqpoint{4.556763in}{2.321761in}}%
\pgfpathlineto{\pgfqpoint{4.557722in}{2.601234in}}%
\pgfpathlineto{\pgfqpoint{4.557785in}{1.960453in}}%
\pgfpathlineto{\pgfqpoint{4.558677in}{2.685769in}}%
\pgfpathlineto{\pgfqpoint{4.558805in}{2.406672in}}%
\pgfpathlineto{\pgfqpoint{4.559695in}{2.660805in}}%
\pgfpathlineto{\pgfqpoint{4.559186in}{2.355005in}}%
\pgfpathlineto{\pgfqpoint{4.559949in}{2.535717in}}%
\pgfpathlineto{\pgfqpoint{4.560266in}{2.704896in}}%
\pgfpathlineto{\pgfqpoint{4.560583in}{2.406575in}}%
\pgfpathlineto{\pgfqpoint{4.561026in}{2.628080in}}%
\pgfpathlineto{\pgfqpoint{4.561216in}{2.354848in}}%
\pgfpathlineto{\pgfqpoint{4.561785in}{2.660220in}}%
\pgfpathlineto{\pgfqpoint{4.562228in}{2.490401in}}%
\pgfpathlineto{\pgfqpoint{4.562669in}{2.653573in}}%
\pgfpathlineto{\pgfqpoint{4.563111in}{2.500934in}}%
\pgfpathlineto{\pgfqpoint{4.563174in}{2.179752in}}%
\pgfpathlineto{\pgfqpoint{4.564180in}{2.651073in}}%
\pgfpathlineto{\pgfqpoint{4.565310in}{2.370495in}}%
\pgfpathlineto{\pgfqpoint{4.565372in}{2.577902in}}%
\pgfpathlineto{\pgfqpoint{4.565498in}{2.658718in}}%
\pgfpathlineto{\pgfqpoint{4.565873in}{2.398031in}}%
\pgfpathlineto{\pgfqpoint{4.566436in}{2.600664in}}%
\pgfpathlineto{\pgfqpoint{4.567435in}{2.243855in}}%
\pgfpathlineto{\pgfqpoint{4.566936in}{2.631977in}}%
\pgfpathlineto{\pgfqpoint{4.567498in}{2.617546in}}%
\pgfpathlineto{\pgfqpoint{4.567560in}{2.632565in}}%
\pgfpathlineto{\pgfqpoint{4.567872in}{2.515802in}}%
\pgfpathlineto{\pgfqpoint{4.567996in}{2.520967in}}%
\pgfpathlineto{\pgfqpoint{4.568867in}{2.642817in}}%
\pgfpathlineto{\pgfqpoint{4.569054in}{2.299249in}}%
\pgfpathlineto{\pgfqpoint{4.569612in}{2.704367in}}%
\pgfpathlineto{\pgfqpoint{4.570170in}{2.518462in}}%
\pgfpathlineto{\pgfqpoint{4.570232in}{2.241605in}}%
\pgfpathlineto{\pgfqpoint{4.570728in}{2.618200in}}%
\pgfpathlineto{\pgfqpoint{4.571284in}{2.378517in}}%
\pgfpathlineto{\pgfqpoint{4.571346in}{2.675892in}}%
\pgfpathlineto{\pgfqpoint{4.572395in}{2.528599in}}%
\pgfpathlineto{\pgfqpoint{4.572826in}{2.238184in}}%
\pgfpathlineto{\pgfqpoint{4.573565in}{2.687129in}}%
\pgfpathlineto{\pgfqpoint{4.573687in}{2.497380in}}%
\pgfpathlineto{\pgfqpoint{4.573810in}{2.655890in}}%
\pgfpathlineto{\pgfqpoint{4.574792in}{2.331051in}}%
\pgfpathlineto{\pgfqpoint{4.575221in}{2.655287in}}%
\pgfpathlineto{\pgfqpoint{4.576016in}{2.618944in}}%
\pgfpathlineto{\pgfqpoint{4.576077in}{2.378852in}}%
\pgfpathlineto{\pgfqpoint{4.576260in}{2.639211in}}%
\pgfpathlineto{\pgfqpoint{4.577115in}{2.537846in}}%
\pgfpathlineto{\pgfqpoint{4.577358in}{2.634165in}}%
\pgfpathlineto{\pgfqpoint{4.577419in}{2.333092in}}%
\pgfpathlineto{\pgfqpoint{4.578210in}{2.570154in}}%
\pgfpathlineto{\pgfqpoint{4.579182in}{2.382701in}}%
\pgfpathlineto{\pgfqpoint{4.578939in}{2.646361in}}%
\pgfpathlineto{\pgfqpoint{4.579303in}{2.502700in}}%
\pgfpathlineto{\pgfqpoint{4.580091in}{2.643041in}}%
\pgfpathlineto{\pgfqpoint{4.579485in}{2.324402in}}%
\pgfpathlineto{\pgfqpoint{4.580273in}{2.418800in}}%
\pgfpathlineto{\pgfqpoint{4.580394in}{2.524368in}}%
\pgfpathlineto{\pgfqpoint{4.581481in}{2.672410in}}%
\pgfpathlineto{\pgfqpoint{4.580877in}{2.247788in}}%
\pgfpathlineto{\pgfqpoint{4.581541in}{2.616666in}}%
\pgfpathlineto{\pgfqpoint{4.582023in}{2.236744in}}%
\pgfpathlineto{\pgfqpoint{4.582385in}{2.644439in}}%
\pgfpathlineto{\pgfqpoint{4.582746in}{2.401157in}}%
\pgfpathlineto{\pgfqpoint{4.583167in}{2.648007in}}%
\pgfpathlineto{\pgfqpoint{4.583047in}{2.318436in}}%
\pgfpathlineto{\pgfqpoint{4.583887in}{2.621161in}}%
\pgfpathlineto{\pgfqpoint{4.584067in}{2.363692in}}%
\pgfpathlineto{\pgfqpoint{4.584127in}{2.653978in}}%
\pgfpathlineto{\pgfqpoint{4.585025in}{2.438726in}}%
\pgfpathlineto{\pgfqpoint{4.585145in}{2.702641in}}%
\pgfpathlineto{\pgfqpoint{4.585384in}{2.352946in}}%
\pgfpathlineto{\pgfqpoint{4.586101in}{2.666472in}}%
\pgfpathlineto{\pgfqpoint{4.587114in}{2.357844in}}%
\pgfpathlineto{\pgfqpoint{4.587174in}{2.599147in}}%
\pgfpathlineto{\pgfqpoint{4.587531in}{2.668453in}}%
\pgfpathlineto{\pgfqpoint{4.587293in}{2.391403in}}%
\pgfpathlineto{\pgfqpoint{4.587947in}{2.532850in}}%
\pgfpathlineto{\pgfqpoint{4.588006in}{2.319452in}}%
\pgfpathlineto{\pgfqpoint{4.588837in}{2.666123in}}%
\pgfpathlineto{\pgfqpoint{4.589015in}{2.545714in}}%
\pgfpathlineto{\pgfqpoint{4.589193in}{2.520953in}}%
\pgfpathlineto{\pgfqpoint{4.589430in}{2.606127in}}%
\pgfpathlineto{\pgfqpoint{4.589489in}{2.593053in}}%
\pgfpathlineto{\pgfqpoint{4.589726in}{2.680904in}}%
\pgfpathlineto{\pgfqpoint{4.589607in}{2.405061in}}%
\pgfpathlineto{\pgfqpoint{4.589844in}{2.679211in}}%
\pgfpathlineto{\pgfqpoint{4.589903in}{2.226409in}}%
\pgfpathlineto{\pgfqpoint{4.590376in}{2.685783in}}%
\pgfpathlineto{\pgfqpoint{4.590967in}{2.456890in}}%
\pgfpathlineto{\pgfqpoint{4.591320in}{2.671993in}}%
\pgfpathlineto{\pgfqpoint{4.591674in}{2.410797in}}%
\pgfpathlineto{\pgfqpoint{4.591968in}{2.538018in}}%
\pgfpathlineto{\pgfqpoint{4.592556in}{2.220828in}}%
\pgfpathlineto{\pgfqpoint{4.592850in}{2.628429in}}%
\pgfpathlineto{\pgfqpoint{4.593026in}{2.324664in}}%
\pgfpathlineto{\pgfqpoint{4.593085in}{2.650975in}}%
\pgfpathlineto{\pgfqpoint{4.593496in}{2.112273in}}%
\pgfpathlineto{\pgfqpoint{4.594140in}{2.646900in}}%
\pgfpathlineto{\pgfqpoint{4.594550in}{2.263162in}}%
\pgfpathlineto{\pgfqpoint{4.594959in}{2.680491in}}%
\pgfpathlineto{\pgfqpoint{4.595193in}{2.411680in}}%
\pgfpathlineto{\pgfqpoint{4.595252in}{2.669378in}}%
\pgfpathlineto{\pgfqpoint{4.595777in}{2.277874in}}%
\pgfpathlineto{\pgfqpoint{4.596302in}{2.589917in}}%
\pgfpathlineto{\pgfqpoint{4.596593in}{2.190351in}}%
\pgfpathlineto{\pgfqpoint{4.597233in}{2.647518in}}%
\pgfpathlineto{\pgfqpoint{4.597407in}{2.486381in}}%
\pgfpathlineto{\pgfqpoint{4.598278in}{2.644915in}}%
\pgfpathlineto{\pgfqpoint{4.597756in}{2.192355in}}%
\pgfpathlineto{\pgfqpoint{4.598568in}{2.590956in}}%
\pgfpathlineto{\pgfqpoint{4.599320in}{2.348013in}}%
\pgfpathlineto{\pgfqpoint{4.599552in}{2.686125in}}%
\pgfpathlineto{\pgfqpoint{4.599667in}{2.542302in}}%
\pgfpathlineto{\pgfqpoint{4.600418in}{2.319701in}}%
\pgfpathlineto{\pgfqpoint{4.600591in}{2.657313in}}%
\pgfpathlineto{\pgfqpoint{4.600649in}{2.251210in}}%
\pgfpathlineto{\pgfqpoint{4.601685in}{2.480433in}}%
\pgfpathlineto{\pgfqpoint{4.602490in}{2.678154in}}%
\pgfpathlineto{\pgfqpoint{4.602145in}{2.430235in}}%
\pgfpathlineto{\pgfqpoint{4.602777in}{2.593270in}}%
\pgfpathlineto{\pgfqpoint{4.603408in}{2.371086in}}%
\pgfpathlineto{\pgfqpoint{4.602949in}{2.652967in}}%
\pgfpathlineto{\pgfqpoint{4.603866in}{2.456560in}}%
\pgfpathlineto{\pgfqpoint{4.604495in}{2.671891in}}%
\pgfpathlineto{\pgfqpoint{4.604838in}{2.242096in}}%
\pgfpathlineto{\pgfqpoint{4.604952in}{2.561837in}}%
\pgfpathlineto{\pgfqpoint{4.605237in}{2.444707in}}%
\pgfpathlineto{\pgfqpoint{4.605066in}{2.678799in}}%
\pgfpathlineto{\pgfqpoint{4.605978in}{2.533414in}}%
\pgfpathlineto{\pgfqpoint{4.606888in}{2.638735in}}%
\pgfpathlineto{\pgfqpoint{4.606775in}{2.346343in}}%
\pgfpathlineto{\pgfqpoint{4.607002in}{2.485937in}}%
\pgfpathlineto{\pgfqpoint{4.607286in}{2.276901in}}%
\pgfpathlineto{\pgfqpoint{4.607115in}{2.690855in}}%
\pgfpathlineto{\pgfqpoint{4.608080in}{2.572312in}}%
\pgfpathlineto{\pgfqpoint{4.608703in}{2.307397in}}%
\pgfpathlineto{\pgfqpoint{4.608589in}{2.609689in}}%
\pgfpathlineto{\pgfqpoint{4.609268in}{2.486305in}}%
\pgfpathlineto{\pgfqpoint{4.609663in}{2.657941in}}%
\pgfpathlineto{\pgfqpoint{4.609833in}{2.159523in}}%
\pgfpathlineto{\pgfqpoint{4.610397in}{2.565978in}}%
\pgfpathlineto{\pgfqpoint{4.610453in}{2.215598in}}%
\pgfpathlineto{\pgfqpoint{4.610566in}{2.716359in}}%
\pgfpathlineto{\pgfqpoint{4.611466in}{2.560544in}}%
\pgfpathlineto{\pgfqpoint{4.611579in}{2.432770in}}%
\pgfpathlineto{\pgfqpoint{4.611747in}{2.619452in}}%
\pgfpathlineto{\pgfqpoint{4.612140in}{2.535885in}}%
\pgfpathlineto{\pgfqpoint{4.612196in}{2.724041in}}%
\pgfpathlineto{\pgfqpoint{4.612757in}{2.409952in}}%
\pgfpathlineto{\pgfqpoint{4.613205in}{2.518392in}}%
\pgfpathlineto{\pgfqpoint{4.613597in}{2.401461in}}%
\pgfpathlineto{\pgfqpoint{4.613485in}{2.668656in}}%
\pgfpathlineto{\pgfqpoint{4.614268in}{2.511026in}}%
\pgfpathlineto{\pgfqpoint{4.614603in}{2.633436in}}%
\pgfpathlineto{\pgfqpoint{4.614993in}{2.344280in}}%
\pgfpathlineto{\pgfqpoint{4.615327in}{2.559996in}}%
\pgfpathlineto{\pgfqpoint{4.615550in}{2.292250in}}%
\pgfpathlineto{\pgfqpoint{4.616106in}{2.635313in}}%
\pgfpathlineto{\pgfqpoint{4.616384in}{2.414938in}}%
\pgfpathlineto{\pgfqpoint{4.616440in}{2.644153in}}%
\pgfpathlineto{\pgfqpoint{4.617051in}{2.273309in}}%
\pgfpathlineto{\pgfqpoint{4.617494in}{2.516219in}}%
\pgfpathlineto{\pgfqpoint{4.618546in}{2.633137in}}%
\pgfpathlineto{\pgfqpoint{4.617993in}{2.352678in}}%
\pgfpathlineto{\pgfqpoint{4.618601in}{2.580170in}}%
\pgfpathlineto{\pgfqpoint{4.618988in}{2.309252in}}%
\pgfpathlineto{\pgfqpoint{4.619264in}{2.681383in}}%
\pgfpathlineto{\pgfqpoint{4.619705in}{2.575523in}}%
\pgfpathlineto{\pgfqpoint{4.620091in}{2.439127in}}%
\pgfpathlineto{\pgfqpoint{4.620256in}{2.668781in}}%
\pgfpathlineto{\pgfqpoint{4.620807in}{2.473043in}}%
\pgfpathlineto{\pgfqpoint{4.621686in}{2.706585in}}%
\pgfpathlineto{\pgfqpoint{4.621246in}{2.365516in}}%
\pgfpathlineto{\pgfqpoint{4.621960in}{2.643883in}}%
\pgfpathlineto{\pgfqpoint{4.622070in}{2.380744in}}%
\pgfpathlineto{\pgfqpoint{4.622946in}{2.652703in}}%
\pgfpathlineto{\pgfqpoint{4.623055in}{2.545695in}}%
\pgfpathlineto{\pgfqpoint{4.623711in}{2.673221in}}%
\pgfpathlineto{\pgfqpoint{4.623602in}{2.327425in}}%
\pgfpathlineto{\pgfqpoint{4.624039in}{2.601128in}}%
\pgfpathlineto{\pgfqpoint{4.624965in}{2.290034in}}%
\pgfpathlineto{\pgfqpoint{4.624530in}{2.614611in}}%
\pgfpathlineto{\pgfqpoint{4.625129in}{2.516103in}}%
\pgfpathlineto{\pgfqpoint{4.626107in}{2.174136in}}%
\pgfpathlineto{\pgfqpoint{4.625618in}{2.658249in}}%
\pgfpathlineto{\pgfqpoint{4.626162in}{2.542653in}}%
\pgfpathlineto{\pgfqpoint{4.626921in}{2.626761in}}%
\pgfpathlineto{\pgfqpoint{4.626433in}{2.343506in}}%
\pgfpathlineto{\pgfqpoint{4.627192in}{2.607556in}}%
\pgfpathlineto{\pgfqpoint{4.627571in}{2.338981in}}%
\pgfpathlineto{\pgfqpoint{4.627625in}{2.655835in}}%
\pgfpathlineto{\pgfqpoint{4.628328in}{2.481838in}}%
\pgfpathlineto{\pgfqpoint{4.629084in}{2.637373in}}%
\pgfpathlineto{\pgfqpoint{4.629030in}{2.385219in}}%
\pgfpathlineto{\pgfqpoint{4.629300in}{2.534893in}}%
\pgfpathlineto{\pgfqpoint{4.629353in}{2.299477in}}%
\pgfpathlineto{\pgfqpoint{4.630215in}{2.683266in}}%
\pgfpathlineto{\pgfqpoint{4.630376in}{2.408992in}}%
\pgfpathlineto{\pgfqpoint{4.631182in}{2.658078in}}%
\pgfpathlineto{\pgfqpoint{4.631343in}{2.383302in}}%
\pgfpathlineto{\pgfqpoint{4.631450in}{2.553936in}}%
\pgfpathlineto{\pgfqpoint{4.631504in}{2.236916in}}%
\pgfpathlineto{\pgfqpoint{4.631825in}{2.674923in}}%
\pgfpathlineto{\pgfqpoint{4.632521in}{2.565544in}}%
\pgfpathlineto{\pgfqpoint{4.633376in}{2.213547in}}%
\pgfpathlineto{\pgfqpoint{4.632949in}{2.660136in}}%
\pgfpathlineto{\pgfqpoint{4.633697in}{2.322692in}}%
\pgfpathlineto{\pgfqpoint{4.633803in}{2.289024in}}%
\pgfpathlineto{\pgfqpoint{4.634815in}{2.656444in}}%
\pgfpathlineto{\pgfqpoint{4.635613in}{2.261522in}}%
\pgfpathlineto{\pgfqpoint{4.635241in}{2.656496in}}%
\pgfpathlineto{\pgfqpoint{4.635931in}{2.421718in}}%
\pgfpathlineto{\pgfqpoint{4.636832in}{2.650756in}}%
\pgfpathlineto{\pgfqpoint{4.636515in}{2.347230in}}%
\pgfpathlineto{\pgfqpoint{4.637044in}{2.628628in}}%
\pgfpathlineto{\pgfqpoint{4.638154in}{2.256937in}}%
\pgfpathlineto{\pgfqpoint{4.639156in}{2.610653in}}%
\pgfpathlineto{\pgfqpoint{4.639261in}{2.565531in}}%
\pgfpathlineto{\pgfqpoint{4.639367in}{2.556167in}}%
\pgfpathlineto{\pgfqpoint{4.640208in}{2.380837in}}%
\pgfpathlineto{\pgfqpoint{4.640050in}{2.636310in}}%
\pgfpathlineto{\pgfqpoint{4.640470in}{2.443313in}}%
\pgfpathlineto{\pgfqpoint{4.640523in}{2.627755in}}%
\pgfpathlineto{\pgfqpoint{4.640733in}{2.206373in}}%
\pgfpathlineto{\pgfqpoint{4.641572in}{2.437476in}}%
\pgfpathlineto{\pgfqpoint{4.642043in}{2.598511in}}%
\pgfpathlineto{\pgfqpoint{4.642461in}{2.272638in}}%
\pgfpathlineto{\pgfqpoint{4.642618in}{2.581454in}}%
\pgfpathlineto{\pgfqpoint{4.643192in}{2.221861in}}%
\pgfpathlineto{\pgfqpoint{4.643713in}{2.386615in}}%
\pgfpathlineto{\pgfqpoint{4.644754in}{2.645498in}}%
\pgfpathlineto{\pgfqpoint{4.644234in}{2.332261in}}%
\pgfpathlineto{\pgfqpoint{4.644910in}{2.516095in}}%
\pgfpathlineto{\pgfqpoint{4.645844in}{2.199339in}}%
\pgfpathlineto{\pgfqpoint{4.645429in}{2.618906in}}%
\pgfpathlineto{\pgfqpoint{4.645896in}{2.473322in}}%
\pgfpathlineto{\pgfqpoint{4.646414in}{2.637153in}}%
\pgfpathlineto{\pgfqpoint{4.646517in}{2.269769in}}%
\pgfpathlineto{\pgfqpoint{4.647035in}{2.584061in}}%
\pgfpathlineto{\pgfqpoint{4.647345in}{2.315736in}}%
\pgfpathlineto{\pgfqpoint{4.647758in}{2.628888in}}%
\pgfpathlineto{\pgfqpoint{4.648170in}{2.480546in}}%
\pgfpathlineto{\pgfqpoint{4.648789in}{2.631949in}}%
\pgfpathlineto{\pgfqpoint{4.648273in}{2.343386in}}%
\pgfpathlineto{\pgfqpoint{4.649252in}{2.542082in}}%
\pgfpathlineto{\pgfqpoint{4.649611in}{2.339119in}}%
\pgfpathlineto{\pgfqpoint{4.649406in}{2.624933in}}%
\pgfpathlineto{\pgfqpoint{4.650330in}{2.447795in}}%
\pgfpathlineto{\pgfqpoint{4.650381in}{2.686832in}}%
\pgfpathlineto{\pgfqpoint{4.650587in}{2.267588in}}%
\pgfpathlineto{\pgfqpoint{4.651457in}{2.580502in}}%
\pgfpathlineto{\pgfqpoint{4.652020in}{2.308855in}}%
\pgfpathlineto{\pgfqpoint{4.651968in}{2.631282in}}%
\pgfpathlineto{\pgfqpoint{4.652530in}{2.486468in}}%
\pgfpathlineto{\pgfqpoint{4.652581in}{2.668157in}}%
\pgfpathlineto{\pgfqpoint{4.652734in}{2.401461in}}%
\pgfpathlineto{\pgfqpoint{4.653651in}{2.578811in}}%
\pgfpathlineto{\pgfqpoint{4.653753in}{2.307460in}}%
\pgfpathlineto{\pgfqpoint{4.654312in}{2.713599in}}%
\pgfpathlineto{\pgfqpoint{4.654719in}{2.519521in}}%
\pgfpathlineto{\pgfqpoint{4.655429in}{2.658267in}}%
\pgfpathlineto{\pgfqpoint{4.654820in}{2.297280in}}%
\pgfpathlineto{\pgfqpoint{4.655530in}{2.639707in}}%
\pgfpathlineto{\pgfqpoint{4.655581in}{2.002075in}}%
\pgfpathlineto{\pgfqpoint{4.656593in}{2.664673in}}%
\pgfpathlineto{\pgfqpoint{4.656644in}{2.540436in}}%
\pgfpathlineto{\pgfqpoint{4.657098in}{2.333686in}}%
\pgfpathlineto{\pgfqpoint{4.657149in}{2.590566in}}%
\pgfpathlineto{\pgfqpoint{4.657552in}{2.447472in}}%
\pgfpathlineto{\pgfqpoint{4.657754in}{2.650965in}}%
\pgfpathlineto{\pgfqpoint{4.657955in}{2.269543in}}%
\pgfpathlineto{\pgfqpoint{4.658660in}{2.594748in}}%
\pgfpathlineto{\pgfqpoint{4.659263in}{2.354377in}}%
\pgfpathlineto{\pgfqpoint{4.659514in}{2.642654in}}%
\pgfpathlineto{\pgfqpoint{4.659715in}{2.560130in}}%
\pgfpathlineto{\pgfqpoint{4.660066in}{2.633825in}}%
\pgfpathlineto{\pgfqpoint{4.660116in}{2.319153in}}%
\pgfpathlineto{\pgfqpoint{4.660617in}{2.456326in}}%
\pgfpathlineto{\pgfqpoint{4.661318in}{2.231496in}}%
\pgfpathlineto{\pgfqpoint{4.660717in}{2.649416in}}%
\pgfpathlineto{\pgfqpoint{4.661667in}{2.467285in}}%
\pgfpathlineto{\pgfqpoint{4.662117in}{2.665762in}}%
\pgfpathlineto{\pgfqpoint{4.661767in}{2.255001in}}%
\pgfpathlineto{\pgfqpoint{4.662516in}{2.621432in}}%
\pgfpathlineto{\pgfqpoint{4.662565in}{2.018685in}}%
\pgfpathlineto{\pgfqpoint{4.662615in}{2.633844in}}%
\pgfpathlineto{\pgfqpoint{4.663611in}{2.514321in}}%
\pgfpathlineto{\pgfqpoint{4.663660in}{2.616106in}}%
\pgfpathlineto{\pgfqpoint{4.663959in}{2.124597in}}%
\pgfpathlineto{\pgfqpoint{4.664653in}{2.492962in}}%
\pgfpathlineto{\pgfqpoint{4.665397in}{2.224595in}}%
\pgfpathlineto{\pgfqpoint{4.664802in}{2.635253in}}%
\pgfpathlineto{\pgfqpoint{4.665496in}{2.536694in}}%
\pgfpathlineto{\pgfqpoint{4.665743in}{2.040444in}}%
\pgfpathlineto{\pgfqpoint{4.666583in}{2.663783in}}%
\pgfpathlineto{\pgfqpoint{4.666978in}{2.299483in}}%
\pgfpathlineto{\pgfqpoint{4.667717in}{2.461074in}}%
\pgfpathlineto{\pgfqpoint{4.667963in}{2.657037in}}%
\pgfpathlineto{\pgfqpoint{4.668455in}{2.305051in}}%
\pgfpathlineto{\pgfqpoint{4.668995in}{2.589021in}}%
\pgfpathlineto{\pgfqpoint{4.669535in}{2.321125in}}%
\pgfpathlineto{\pgfqpoint{4.669731in}{2.617847in}}%
\pgfpathlineto{\pgfqpoint{4.670074in}{2.554511in}}%
\pgfpathlineto{\pgfqpoint{4.670661in}{2.396848in}}%
\pgfpathlineto{\pgfqpoint{4.670710in}{2.627307in}}%
\pgfpathlineto{\pgfqpoint{4.670759in}{2.178213in}}%
\pgfpathlineto{\pgfqpoint{4.670808in}{2.645850in}}%
\pgfpathlineto{\pgfqpoint{4.671784in}{2.583787in}}%
\pgfpathlineto{\pgfqpoint{4.672369in}{2.666254in}}%
\pgfpathlineto{\pgfqpoint{4.673147in}{2.029466in}}%
\pgfpathlineto{\pgfqpoint{4.673779in}{2.585069in}}%
\pgfpathlineto{\pgfqpoint{4.674264in}{2.224494in}}%
\pgfpathlineto{\pgfqpoint{4.674312in}{2.161839in}}%
\pgfpathlineto{\pgfqpoint{4.674652in}{2.671297in}}%
\pgfpathlineto{\pgfqpoint{4.675087in}{2.473176in}}%
\pgfpathlineto{\pgfqpoint{4.675523in}{2.633022in}}%
\pgfpathlineto{\pgfqpoint{4.675474in}{2.316739in}}%
\pgfpathlineto{\pgfqpoint{4.676199in}{2.485868in}}%
\pgfpathlineto{\pgfqpoint{4.676874in}{2.154701in}}%
\pgfpathlineto{\pgfqpoint{4.677115in}{2.661620in}}%
\pgfpathlineto{\pgfqpoint{4.677259in}{2.515791in}}%
\pgfpathlineto{\pgfqpoint{4.677500in}{2.611973in}}%
\pgfpathlineto{\pgfqpoint{4.677836in}{2.311690in}}%
\pgfpathlineto{\pgfqpoint{4.678365in}{2.519634in}}%
\pgfpathlineto{\pgfqpoint{4.679085in}{2.592932in}}%
\pgfpathlineto{\pgfqpoint{4.679468in}{2.010661in}}%
\pgfpathlineto{\pgfqpoint{4.680281in}{2.649238in}}%
\pgfpathlineto{\pgfqpoint{4.680568in}{2.511330in}}%
\pgfpathlineto{\pgfqpoint{4.680998in}{2.306917in}}%
\pgfpathlineto{\pgfqpoint{4.681427in}{2.697269in}}%
\pgfpathlineto{\pgfqpoint{4.681665in}{2.344544in}}%
\pgfpathlineto{\pgfqpoint{4.682617in}{2.677078in}}%
\pgfpathlineto{\pgfqpoint{4.682760in}{2.490501in}}%
\pgfpathlineto{\pgfqpoint{4.683329in}{2.322430in}}%
\pgfpathlineto{\pgfqpoint{4.683377in}{2.621161in}}%
\pgfpathlineto{\pgfqpoint{4.683756in}{2.578223in}}%
\pgfpathlineto{\pgfqpoint{4.683851in}{2.406096in}}%
\pgfpathlineto{\pgfqpoint{4.683993in}{2.602377in}}%
\pgfpathlineto{\pgfqpoint{4.684230in}{2.165428in}}%
\pgfpathlineto{\pgfqpoint{4.684325in}{2.635906in}}%
\pgfpathlineto{\pgfqpoint{4.685082in}{2.375934in}}%
\pgfpathlineto{\pgfqpoint{4.686026in}{2.665314in}}%
\pgfpathlineto{\pgfqpoint{4.685884in}{2.356559in}}%
\pgfpathlineto{\pgfqpoint{4.686215in}{2.549883in}}%
\pgfpathlineto{\pgfqpoint{4.686356in}{2.338578in}}%
\pgfpathlineto{\pgfqpoint{4.686497in}{2.622929in}}%
\pgfpathlineto{\pgfqpoint{4.687344in}{2.495510in}}%
\pgfpathlineto{\pgfqpoint{4.687626in}{2.642558in}}%
\pgfpathlineto{\pgfqpoint{4.687861in}{2.243650in}}%
\pgfpathlineto{\pgfqpoint{4.688330in}{2.465397in}}%
\pgfpathlineto{\pgfqpoint{4.689174in}{2.246508in}}%
\pgfpathlineto{\pgfqpoint{4.688471in}{2.656594in}}%
\pgfpathlineto{\pgfqpoint{4.689408in}{2.527814in}}%
\pgfpathlineto{\pgfqpoint{4.689548in}{2.643440in}}%
\pgfpathlineto{\pgfqpoint{4.689735in}{2.199697in}}%
\pgfpathlineto{\pgfqpoint{4.690436in}{2.524494in}}%
\pgfpathlineto{\pgfqpoint{4.690669in}{2.027418in}}%
\pgfpathlineto{\pgfqpoint{4.691228in}{2.643474in}}%
\pgfpathlineto{\pgfqpoint{4.691508in}{2.570373in}}%
\pgfpathlineto{\pgfqpoint{4.691787in}{2.671364in}}%
\pgfpathlineto{\pgfqpoint{4.691880in}{2.391306in}}%
\pgfpathlineto{\pgfqpoint{4.692345in}{2.463926in}}%
\pgfpathlineto{\pgfqpoint{4.692391in}{2.336768in}}%
\pgfpathlineto{\pgfqpoint{4.692577in}{2.629390in}}%
\pgfpathlineto{\pgfqpoint{4.693412in}{2.513220in}}%
\pgfpathlineto{\pgfqpoint{4.693922in}{2.618995in}}%
\pgfpathlineto{\pgfqpoint{4.693968in}{2.356160in}}%
\pgfpathlineto{\pgfqpoint{4.694523in}{2.554194in}}%
\pgfpathlineto{\pgfqpoint{4.694892in}{2.237852in}}%
\pgfpathlineto{\pgfqpoint{4.695216in}{2.708194in}}%
\pgfpathlineto{\pgfqpoint{4.695631in}{2.554585in}}%
\pgfpathlineto{\pgfqpoint{4.695815in}{2.259965in}}%
\pgfpathlineto{\pgfqpoint{4.695953in}{2.639241in}}%
\pgfpathlineto{\pgfqpoint{4.696782in}{2.500347in}}%
\pgfpathlineto{\pgfqpoint{4.697563in}{2.653029in}}%
\pgfpathlineto{\pgfqpoint{4.697792in}{2.379831in}}%
\pgfpathlineto{\pgfqpoint{4.697884in}{2.523671in}}%
\pgfpathlineto{\pgfqpoint{4.697975in}{2.586199in}}%
\pgfpathlineto{\pgfqpoint{4.698983in}{2.232064in}}%
\pgfpathlineto{\pgfqpoint{4.699303in}{2.633396in}}%
\pgfpathlineto{\pgfqpoint{4.700034in}{2.217636in}}%
\pgfpathlineto{\pgfqpoint{4.700125in}{2.539405in}}%
\pgfpathlineto{\pgfqpoint{4.700717in}{2.083901in}}%
\pgfpathlineto{\pgfqpoint{4.700353in}{2.586719in}}%
\pgfpathlineto{\pgfqpoint{4.701173in}{2.576700in}}%
\pgfpathlineto{\pgfqpoint{4.701218in}{2.582159in}}%
\pgfpathlineto{\pgfqpoint{4.701264in}{2.561192in}}%
\pgfpathlineto{\pgfqpoint{4.702218in}{2.195603in}}%
\pgfpathlineto{\pgfqpoint{4.702036in}{2.615289in}}%
\pgfpathlineto{\pgfqpoint{4.702354in}{2.310099in}}%
\pgfpathlineto{\pgfqpoint{4.703034in}{2.661886in}}%
\pgfpathlineto{\pgfqpoint{4.702944in}{2.253951in}}%
\pgfpathlineto{\pgfqpoint{4.703487in}{2.514376in}}%
\pgfpathlineto{\pgfqpoint{4.704256in}{2.245295in}}%
\pgfpathlineto{\pgfqpoint{4.704075in}{2.657251in}}%
\pgfpathlineto{\pgfqpoint{4.704527in}{2.499349in}}%
\pgfpathlineto{\pgfqpoint{4.705383in}{2.638731in}}%
\pgfpathlineto{\pgfqpoint{4.704933in}{2.383291in}}%
\pgfpathlineto{\pgfqpoint{4.705609in}{2.490205in}}%
\pgfpathlineto{\pgfqpoint{4.706014in}{2.371034in}}%
\pgfpathlineto{\pgfqpoint{4.705834in}{2.639806in}}%
\pgfpathlineto{\pgfqpoint{4.706239in}{2.565199in}}%
\pgfpathlineto{\pgfqpoint{4.707272in}{2.654464in}}%
\pgfpathlineto{\pgfqpoint{4.706688in}{2.344517in}}%
\pgfpathlineto{\pgfqpoint{4.707316in}{2.597169in}}%
\pgfpathlineto{\pgfqpoint{4.708168in}{2.318658in}}%
\pgfpathlineto{\pgfqpoint{4.707854in}{2.677783in}}%
\pgfpathlineto{\pgfqpoint{4.708481in}{2.486162in}}%
\pgfpathlineto{\pgfqpoint{4.708794in}{2.610697in}}%
\pgfpathlineto{\pgfqpoint{4.709151in}{2.303949in}}%
\pgfpathlineto{\pgfqpoint{4.709508in}{2.458629in}}%
\pgfpathlineto{\pgfqpoint{4.710222in}{2.272740in}}%
\pgfpathlineto{\pgfqpoint{4.709910in}{2.615197in}}%
\pgfpathlineto{\pgfqpoint{4.710489in}{2.511918in}}%
\pgfpathlineto{\pgfqpoint{4.711023in}{2.618615in}}%
\pgfpathlineto{\pgfqpoint{4.710889in}{2.307192in}}%
\pgfpathlineto{\pgfqpoint{4.711556in}{2.593189in}}%
\pgfpathlineto{\pgfqpoint{4.712221in}{2.081264in}}%
\pgfpathlineto{\pgfqpoint{4.712177in}{2.684257in}}%
\pgfpathlineto{\pgfqpoint{4.712664in}{2.550332in}}%
\pgfpathlineto{\pgfqpoint{4.713372in}{2.607238in}}%
\pgfpathlineto{\pgfqpoint{4.713196in}{2.120582in}}%
\pgfpathlineto{\pgfqpoint{4.713549in}{2.558003in}}%
\pgfpathlineto{\pgfqpoint{4.714565in}{2.257523in}}%
\pgfpathlineto{\pgfqpoint{4.713947in}{2.600106in}}%
\pgfpathlineto{\pgfqpoint{4.714653in}{2.450362in}}%
\pgfpathlineto{\pgfqpoint{4.715269in}{2.625990in}}%
\pgfpathlineto{\pgfqpoint{4.714741in}{2.237336in}}%
\pgfpathlineto{\pgfqpoint{4.715753in}{2.533450in}}%
\pgfpathlineto{\pgfqpoint{4.716368in}{2.204916in}}%
\pgfpathlineto{\pgfqpoint{4.716324in}{2.616354in}}%
\pgfpathlineto{\pgfqpoint{4.716851in}{2.368659in}}%
\pgfpathlineto{\pgfqpoint{4.717596in}{2.619179in}}%
\pgfpathlineto{\pgfqpoint{4.717946in}{2.346399in}}%
\pgfpathlineto{\pgfqpoint{4.718339in}{2.626409in}}%
\pgfpathlineto{\pgfqpoint{4.719038in}{2.459064in}}%
\pgfpathlineto{\pgfqpoint{4.719387in}{2.256475in}}%
\pgfpathlineto{\pgfqpoint{4.719953in}{2.639793in}}%
\pgfpathlineto{\pgfqpoint{4.720127in}{2.449408in}}%
\pgfpathlineto{\pgfqpoint{4.721040in}{2.649210in}}%
\pgfpathlineto{\pgfqpoint{4.720606in}{2.310170in}}%
\pgfpathlineto{\pgfqpoint{4.721257in}{2.537852in}}%
\pgfpathlineto{\pgfqpoint{4.721517in}{2.487570in}}%
\pgfpathlineto{\pgfqpoint{4.721561in}{2.602162in}}%
\pgfpathlineto{\pgfqpoint{4.721604in}{2.598491in}}%
\pgfpathlineto{\pgfqpoint{4.721648in}{2.342971in}}%
\pgfpathlineto{\pgfqpoint{4.722471in}{2.624895in}}%
\pgfpathlineto{\pgfqpoint{4.722730in}{2.385705in}}%
\pgfpathlineto{\pgfqpoint{4.722946in}{2.230694in}}%
\pgfpathlineto{\pgfqpoint{4.723292in}{2.664022in}}%
\pgfpathlineto{\pgfqpoint{4.723637in}{2.452582in}}%
\pgfpathlineto{\pgfqpoint{4.723853in}{2.628531in}}%
\pgfpathlineto{\pgfqpoint{4.723810in}{2.347902in}}%
\pgfpathlineto{\pgfqpoint{4.724758in}{2.497572in}}%
\pgfpathlineto{\pgfqpoint{4.724801in}{2.266509in}}%
\pgfpathlineto{\pgfqpoint{4.725403in}{2.585048in}}%
\pgfpathlineto{\pgfqpoint{4.725833in}{2.433101in}}%
\pgfpathlineto{\pgfqpoint{4.726433in}{2.604206in}}%
\pgfpathlineto{\pgfqpoint{4.726090in}{2.333816in}}%
\pgfpathlineto{\pgfqpoint{4.726948in}{2.546714in}}%
\pgfpathlineto{\pgfqpoint{4.727162in}{2.608679in}}%
\pgfpathlineto{\pgfqpoint{4.727033in}{2.465233in}}%
\pgfpathlineto{\pgfqpoint{4.727204in}{2.484756in}}%
\pgfpathlineto{\pgfqpoint{4.728145in}{2.699372in}}%
\pgfpathlineto{\pgfqpoint{4.728316in}{2.295747in}}%
\pgfpathlineto{\pgfqpoint{4.729041in}{2.624305in}}%
\pgfpathlineto{\pgfqpoint{4.728955in}{2.222506in}}%
\pgfpathlineto{\pgfqpoint{4.729509in}{2.570914in}}%
\pgfpathlineto{\pgfqpoint{4.730572in}{2.209625in}}%
\pgfpathlineto{\pgfqpoint{4.730445in}{2.634199in}}%
\pgfpathlineto{\pgfqpoint{4.730614in}{2.505943in}}%
\pgfpathlineto{\pgfqpoint{4.730827in}{2.635507in}}%
\pgfpathlineto{\pgfqpoint{4.731547in}{2.325300in}}%
\pgfpathlineto{\pgfqpoint{4.731674in}{2.530906in}}%
\pgfpathlineto{\pgfqpoint{4.732605in}{2.191112in}}%
\pgfpathlineto{\pgfqpoint{4.732690in}{2.621391in}}%
\pgfpathlineto{\pgfqpoint{4.732774in}{2.400086in}}%
\pgfpathlineto{\pgfqpoint{4.733027in}{2.652455in}}%
\pgfpathlineto{\pgfqpoint{4.733913in}{2.542833in}}%
\pgfpathlineto{\pgfqpoint{4.734502in}{1.871064in}}%
\pgfpathlineto{\pgfqpoint{4.734166in}{2.621145in}}%
\pgfpathlineto{\pgfqpoint{4.734923in}{2.471115in}}%
\pgfpathlineto{\pgfqpoint{4.736014in}{2.634056in}}%
\pgfpathlineto{\pgfqpoint{4.735720in}{2.334288in}}%
\pgfpathlineto{\pgfqpoint{4.736056in}{2.550670in}}%
\pgfpathlineto{\pgfqpoint{4.736684in}{2.325803in}}%
\pgfpathlineto{\pgfqpoint{4.736307in}{2.618736in}}%
\pgfpathlineto{\pgfqpoint{4.737102in}{2.567608in}}%
\pgfpathlineto{\pgfqpoint{4.737144in}{2.612802in}}%
\pgfpathlineto{\pgfqpoint{4.737311in}{2.374013in}}%
\pgfpathlineto{\pgfqpoint{4.738105in}{2.486085in}}%
\pgfpathlineto{\pgfqpoint{4.738563in}{2.147265in}}%
\pgfpathlineto{\pgfqpoint{4.738771in}{2.613004in}}%
\pgfpathlineto{\pgfqpoint{4.739063in}{2.560885in}}%
\pgfpathlineto{\pgfqpoint{4.739104in}{2.625037in}}%
\pgfpathlineto{\pgfqpoint{4.739479in}{2.305987in}}%
\pgfpathlineto{\pgfqpoint{4.740143in}{2.538666in}}%
\pgfpathlineto{\pgfqpoint{4.740185in}{2.539013in}}%
\pgfpathlineto{\pgfqpoint{4.740227in}{2.639799in}}%
\pgfpathlineto{\pgfqpoint{4.741056in}{2.284089in}}%
\pgfpathlineto{\pgfqpoint{4.741263in}{2.500080in}}%
\pgfpathlineto{\pgfqpoint{4.741842in}{2.291581in}}%
\pgfpathlineto{\pgfqpoint{4.741470in}{2.632774in}}%
\pgfpathlineto{\pgfqpoint{4.742338in}{2.445711in}}%
\pgfpathlineto{\pgfqpoint{4.742586in}{2.624618in}}%
\pgfpathlineto{\pgfqpoint{4.743369in}{2.291454in}}%
\pgfpathlineto{\pgfqpoint{4.743451in}{2.521561in}}%
\pgfpathlineto{\pgfqpoint{4.743575in}{2.376683in}}%
\pgfpathlineto{\pgfqpoint{4.744439in}{2.627603in}}%
\pgfpathlineto{\pgfqpoint{4.744562in}{2.455172in}}%
\pgfpathlineto{\pgfqpoint{4.744726in}{2.654633in}}%
\pgfpathlineto{\pgfqpoint{4.745055in}{2.366529in}}%
\pgfpathlineto{\pgfqpoint{4.745670in}{2.562597in}}%
\pgfpathlineto{\pgfqpoint{4.746079in}{2.662664in}}%
\pgfpathlineto{\pgfqpoint{4.746856in}{2.217995in}}%
\pgfpathlineto{\pgfqpoint{4.747142in}{2.655200in}}%
\pgfpathlineto{\pgfqpoint{4.747265in}{2.217392in}}%
\pgfpathlineto{\pgfqpoint{4.747958in}{2.521064in}}%
\pgfpathlineto{\pgfqpoint{4.748569in}{2.230257in}}%
\pgfpathlineto{\pgfqpoint{4.748650in}{2.622795in}}%
\pgfpathlineto{\pgfqpoint{4.749016in}{2.410187in}}%
\pgfpathlineto{\pgfqpoint{4.749423in}{2.647302in}}%
\pgfpathlineto{\pgfqpoint{4.749950in}{2.358845in}}%
\pgfpathlineto{\pgfqpoint{4.750153in}{2.551515in}}%
\pgfpathlineto{\pgfqpoint{4.750923in}{2.241774in}}%
\pgfpathlineto{\pgfqpoint{4.751044in}{2.630578in}}%
\pgfpathlineto{\pgfqpoint{4.751327in}{2.463846in}}%
\pgfpathlineto{\pgfqpoint{4.752337in}{2.630572in}}%
\pgfpathlineto{\pgfqpoint{4.752095in}{2.248786in}}%
\pgfpathlineto{\pgfqpoint{4.752377in}{2.423448in}}%
\pgfpathlineto{\pgfqpoint{4.752458in}{2.293808in}}%
\pgfpathlineto{\pgfqpoint{4.753102in}{2.617175in}}%
\pgfpathlineto{\pgfqpoint{4.753384in}{2.583142in}}%
\pgfpathlineto{\pgfqpoint{4.753626in}{2.644664in}}%
\pgfpathlineto{\pgfqpoint{4.753706in}{2.378583in}}%
\pgfpathlineto{\pgfqpoint{4.753867in}{2.609166in}}%
\pgfpathlineto{\pgfqpoint{4.754870in}{2.111274in}}%
\pgfpathlineto{\pgfqpoint{4.754268in}{2.623104in}}%
\pgfpathlineto{\pgfqpoint{4.754950in}{2.481803in}}%
\pgfpathlineto{\pgfqpoint{4.755951in}{2.628910in}}%
\pgfpathlineto{\pgfqpoint{4.755671in}{2.267387in}}%
\pgfpathlineto{\pgfqpoint{4.756071in}{2.518834in}}%
\pgfpathlineto{\pgfqpoint{4.756950in}{2.337341in}}%
\pgfpathlineto{\pgfqpoint{4.757150in}{2.610353in}}%
\pgfpathlineto{\pgfqpoint{4.757867in}{2.200163in}}%
\pgfpathlineto{\pgfqpoint{4.757508in}{2.622640in}}%
\pgfpathlineto{\pgfqpoint{4.758265in}{2.458992in}}%
\pgfpathlineto{\pgfqpoint{4.758344in}{2.646667in}}%
\pgfpathlineto{\pgfqpoint{4.758583in}{2.249752in}}%
\pgfpathlineto{\pgfqpoint{4.759417in}{2.589774in}}%
\pgfpathlineto{\pgfqpoint{4.760090in}{2.359680in}}%
\pgfpathlineto{\pgfqpoint{4.760209in}{2.642860in}}%
\pgfpathlineto{\pgfqpoint{4.760486in}{2.528017in}}%
\pgfpathlineto{\pgfqpoint{4.761000in}{2.631703in}}%
\pgfpathlineto{\pgfqpoint{4.761198in}{2.260900in}}%
\pgfpathlineto{\pgfqpoint{4.761593in}{2.532672in}}%
\pgfpathlineto{\pgfqpoint{4.762342in}{2.248883in}}%
\pgfpathlineto{\pgfqpoint{4.761711in}{2.625330in}}%
\pgfpathlineto{\pgfqpoint{4.762499in}{2.550105in}}%
\pgfpathlineto{\pgfqpoint{4.762736in}{2.673498in}}%
\pgfpathlineto{\pgfqpoint{4.762618in}{2.321581in}}%
\pgfpathlineto{\pgfqpoint{4.763561in}{2.571854in}}%
\pgfpathlineto{\pgfqpoint{4.763876in}{2.124987in}}%
\pgfpathlineto{\pgfqpoint{4.763719in}{2.653761in}}%
\pgfpathlineto{\pgfqpoint{4.764699in}{2.500629in}}%
\pgfpathlineto{\pgfqpoint{4.765795in}{2.679526in}}%
\pgfpathlineto{\pgfqpoint{4.765012in}{2.205916in}}%
\pgfpathlineto{\pgfqpoint{4.765834in}{2.593026in}}%
\pgfpathlineto{\pgfqpoint{4.766770in}{2.381807in}}%
\pgfpathlineto{\pgfqpoint{4.765951in}{2.637446in}}%
\pgfpathlineto{\pgfqpoint{4.767004in}{2.454530in}}%
\pgfpathlineto{\pgfqpoint{4.767277in}{2.595915in}}%
\pgfpathlineto{\pgfqpoint{4.767861in}{2.192564in}}%
\pgfpathlineto{\pgfqpoint{4.768172in}{2.579245in}}%
\pgfpathlineto{\pgfqpoint{4.768443in}{2.377161in}}%
\pgfpathlineto{\pgfqpoint{4.768482in}{2.616805in}}%
\pgfpathlineto{\pgfqpoint{4.769258in}{2.589993in}}%
\pgfpathlineto{\pgfqpoint{4.769839in}{2.605857in}}%
\pgfpathlineto{\pgfqpoint{4.769878in}{2.444730in}}%
\pgfpathlineto{\pgfqpoint{4.769917in}{2.530864in}}%
\pgfpathlineto{\pgfqpoint{4.770922in}{2.208015in}}%
\pgfpathlineto{\pgfqpoint{4.770806in}{2.623989in}}%
\pgfpathlineto{\pgfqpoint{4.770999in}{2.509008in}}%
\pgfpathlineto{\pgfqpoint{4.771076in}{2.621775in}}%
\pgfpathlineto{\pgfqpoint{4.771577in}{2.286076in}}%
\pgfpathlineto{\pgfqpoint{4.772117in}{2.601416in}}%
\pgfpathlineto{\pgfqpoint{4.773078in}{2.214832in}}%
\pgfpathlineto{\pgfqpoint{4.772617in}{2.633351in}}%
\pgfpathlineto{\pgfqpoint{4.773232in}{2.547811in}}%
\pgfpathlineto{\pgfqpoint{4.773577in}{2.139992in}}%
\pgfpathlineto{\pgfqpoint{4.774190in}{2.647769in}}%
\pgfpathlineto{\pgfqpoint{4.774497in}{2.415746in}}%
\pgfpathlineto{\pgfqpoint{4.775224in}{2.653231in}}%
\pgfpathlineto{\pgfqpoint{4.775376in}{2.339691in}}%
\pgfpathlineto{\pgfqpoint{4.775567in}{2.448925in}}%
\pgfpathlineto{\pgfqpoint{4.776064in}{2.358775in}}%
\pgfpathlineto{\pgfqpoint{4.776407in}{2.611785in}}%
\pgfpathlineto{\pgfqpoint{4.776597in}{2.578631in}}%
\pgfpathlineto{\pgfqpoint{4.777510in}{2.365426in}}%
\pgfpathlineto{\pgfqpoint{4.777282in}{2.613249in}}%
\pgfpathlineto{\pgfqpoint{4.777662in}{2.524826in}}%
\pgfpathlineto{\pgfqpoint{4.777776in}{2.635407in}}%
\pgfpathlineto{\pgfqpoint{4.777966in}{2.240891in}}%
\pgfpathlineto{\pgfqpoint{4.778725in}{2.500355in}}%
\pgfpathlineto{\pgfqpoint{4.779634in}{2.289877in}}%
\pgfpathlineto{\pgfqpoint{4.779218in}{2.607504in}}%
\pgfpathlineto{\pgfqpoint{4.779823in}{2.437186in}}%
\pgfpathlineto{\pgfqpoint{4.780050in}{2.629611in}}%
\pgfpathlineto{\pgfqpoint{4.780691in}{2.359554in}}%
\pgfpathlineto{\pgfqpoint{4.780918in}{2.509748in}}%
\pgfpathlineto{\pgfqpoint{4.781709in}{2.321569in}}%
\pgfpathlineto{\pgfqpoint{4.781106in}{2.599772in}}%
\pgfpathlineto{\pgfqpoint{4.782010in}{2.543494in}}%
\pgfpathlineto{\pgfqpoint{4.782123in}{2.248727in}}%
\pgfpathlineto{\pgfqpoint{4.782574in}{2.635247in}}%
\pgfpathlineto{\pgfqpoint{4.783174in}{2.458155in}}%
\pgfpathlineto{\pgfqpoint{4.783362in}{2.606551in}}%
\pgfpathlineto{\pgfqpoint{4.784148in}{2.174341in}}%
\pgfpathlineto{\pgfqpoint{4.784298in}{2.497118in}}%
\pgfpathlineto{\pgfqpoint{4.785269in}{2.634739in}}%
\pgfpathlineto{\pgfqpoint{4.784709in}{2.287244in}}%
\pgfpathlineto{\pgfqpoint{4.785344in}{2.542454in}}%
\pgfpathlineto{\pgfqpoint{4.785531in}{2.233962in}}%
\pgfpathlineto{\pgfqpoint{4.785419in}{2.602153in}}%
\pgfpathlineto{\pgfqpoint{4.786425in}{2.420032in}}%
\pgfpathlineto{\pgfqpoint{4.786946in}{2.660299in}}%
\pgfpathlineto{\pgfqpoint{4.786908in}{2.082362in}}%
\pgfpathlineto{\pgfqpoint{4.787540in}{2.514369in}}%
\pgfpathlineto{\pgfqpoint{4.787614in}{2.371563in}}%
\pgfpathlineto{\pgfqpoint{4.787948in}{2.616470in}}%
\pgfpathlineto{\pgfqpoint{4.788652in}{2.468705in}}%
\pgfpathlineto{\pgfqpoint{4.789577in}{2.635854in}}%
\pgfpathlineto{\pgfqpoint{4.789133in}{2.366082in}}%
\pgfpathlineto{\pgfqpoint{4.789688in}{2.424310in}}%
\pgfpathlineto{\pgfqpoint{4.790684in}{2.309869in}}%
\pgfpathlineto{\pgfqpoint{4.789872in}{2.609244in}}%
\pgfpathlineto{\pgfqpoint{4.790721in}{2.423324in}}%
\pgfpathlineto{\pgfqpoint{4.791163in}{2.650462in}}%
\pgfpathlineto{\pgfqpoint{4.790868in}{2.288218in}}%
\pgfpathlineto{\pgfqpoint{4.791788in}{2.531150in}}%
\pgfpathlineto{\pgfqpoint{4.792743in}{2.257631in}}%
\pgfpathlineto{\pgfqpoint{4.792449in}{2.650327in}}%
\pgfpathlineto{\pgfqpoint{4.792853in}{2.538842in}}%
\pgfpathlineto{\pgfqpoint{4.793951in}{2.690474in}}%
\pgfpathlineto{\pgfqpoint{4.793219in}{2.312086in}}%
\pgfpathlineto{\pgfqpoint{4.793988in}{2.623564in}}%
\pgfpathlineto{\pgfqpoint{4.794390in}{2.181544in}}%
\pgfpathlineto{\pgfqpoint{4.795119in}{2.469229in}}%
\pgfpathlineto{\pgfqpoint{4.795265in}{2.612103in}}%
\pgfpathlineto{\pgfqpoint{4.795957in}{2.313625in}}%
\pgfpathlineto{\pgfqpoint{4.796176in}{2.450193in}}%
\pgfpathlineto{\pgfqpoint{4.796757in}{2.201107in}}%
\pgfpathlineto{\pgfqpoint{4.797012in}{2.635975in}}%
\pgfpathlineto{\pgfqpoint{4.797229in}{2.543959in}}%
\pgfpathlineto{\pgfqpoint{4.797954in}{2.649360in}}%
\pgfpathlineto{\pgfqpoint{4.798172in}{2.335927in}}%
\pgfpathlineto{\pgfqpoint{4.798208in}{2.631743in}}%
\pgfpathlineto{\pgfqpoint{4.799076in}{2.300060in}}%
\pgfpathlineto{\pgfqpoint{4.799293in}{2.543270in}}%
\pgfpathlineto{\pgfqpoint{4.799942in}{2.241769in}}%
\pgfpathlineto{\pgfqpoint{4.800338in}{2.665993in}}%
\pgfpathlineto{\pgfqpoint{4.800374in}{2.473179in}}%
\pgfpathlineto{\pgfqpoint{4.800411in}{2.609738in}}%
\pgfpathlineto{\pgfqpoint{4.801310in}{2.238439in}}%
\pgfpathlineto{\pgfqpoint{4.801454in}{2.463492in}}%
\pgfpathlineto{\pgfqpoint{4.802064in}{2.081287in}}%
\pgfpathlineto{\pgfqpoint{4.801956in}{2.630983in}}%
\pgfpathlineto{\pgfqpoint{4.802423in}{2.537350in}}%
\pgfpathlineto{\pgfqpoint{4.802530in}{2.635477in}}%
\pgfpathlineto{\pgfqpoint{4.802709in}{2.251584in}}%
\pgfpathlineto{\pgfqpoint{4.803425in}{2.596942in}}%
\pgfpathlineto{\pgfqpoint{4.803890in}{2.191137in}}%
\pgfpathlineto{\pgfqpoint{4.803532in}{2.629295in}}%
\pgfpathlineto{\pgfqpoint{4.804532in}{2.503074in}}%
\pgfpathlineto{\pgfqpoint{4.805209in}{2.301579in}}%
\pgfpathlineto{\pgfqpoint{4.805316in}{2.625561in}}%
\pgfpathlineto{\pgfqpoint{4.805530in}{2.527147in}}%
\pgfpathlineto{\pgfqpoint{4.805850in}{2.624119in}}%
\pgfpathlineto{\pgfqpoint{4.805814in}{2.277624in}}%
\pgfpathlineto{\pgfqpoint{4.806525in}{2.499668in}}%
\pgfpathlineto{\pgfqpoint{4.807270in}{2.282552in}}%
\pgfpathlineto{\pgfqpoint{4.807057in}{2.647843in}}%
\pgfpathlineto{\pgfqpoint{4.807624in}{2.509388in}}%
\pgfpathlineto{\pgfqpoint{4.807659in}{2.509100in}}%
\pgfpathlineto{\pgfqpoint{4.808119in}{2.621611in}}%
\pgfpathlineto{\pgfqpoint{4.808367in}{2.273634in}}%
\pgfpathlineto{\pgfqpoint{4.808720in}{2.527425in}}%
\pgfpathlineto{\pgfqpoint{4.809285in}{2.197817in}}%
\pgfpathlineto{\pgfqpoint{4.809391in}{2.619284in}}%
\pgfpathlineto{\pgfqpoint{4.809813in}{2.567300in}}%
\pgfpathlineto{\pgfqpoint{4.810482in}{2.152872in}}%
\pgfpathlineto{\pgfqpoint{4.810412in}{2.627415in}}%
\pgfpathlineto{\pgfqpoint{4.811009in}{2.471836in}}%
\pgfpathlineto{\pgfqpoint{4.811045in}{2.672873in}}%
\pgfpathlineto{\pgfqpoint{4.811817in}{2.161882in}}%
\pgfpathlineto{\pgfqpoint{4.812097in}{2.520715in}}%
\pgfpathlineto{\pgfqpoint{4.812412in}{2.252574in}}%
\pgfpathlineto{\pgfqpoint{4.813147in}{2.617968in}}%
\pgfpathlineto{\pgfqpoint{4.813182in}{2.471000in}}%
\pgfpathlineto{\pgfqpoint{4.813461in}{2.651274in}}%
\pgfpathlineto{\pgfqpoint{4.814194in}{2.255794in}}%
\pgfpathlineto{\pgfqpoint{4.814299in}{2.549006in}}%
\pgfpathlineto{\pgfqpoint{4.815412in}{2.331014in}}%
\pgfpathlineto{\pgfqpoint{4.815099in}{2.643912in}}%
\pgfpathlineto{\pgfqpoint{4.815447in}{2.404615in}}%
\pgfpathlineto{\pgfqpoint{4.815586in}{2.634644in}}%
\pgfpathlineto{\pgfqpoint{4.815899in}{2.214281in}}%
\pgfpathlineto{\pgfqpoint{4.816558in}{2.444714in}}%
\pgfpathlineto{\pgfqpoint{4.817493in}{2.633747in}}%
\pgfpathlineto{\pgfqpoint{4.817320in}{2.281537in}}%
\pgfpathlineto{\pgfqpoint{4.817562in}{2.380108in}}%
\pgfpathlineto{\pgfqpoint{4.817597in}{2.373806in}}%
\pgfpathlineto{\pgfqpoint{4.817632in}{2.625899in}}%
\pgfpathlineto{\pgfqpoint{4.818426in}{2.347990in}}%
\pgfpathlineto{\pgfqpoint{4.818702in}{2.462335in}}%
\pgfpathlineto{\pgfqpoint{4.819116in}{2.634131in}}%
\pgfpathlineto{\pgfqpoint{4.818840in}{2.323055in}}%
\pgfpathlineto{\pgfqpoint{4.819873in}{2.630325in}}%
\pgfpathlineto{\pgfqpoint{4.820011in}{2.293461in}}%
\pgfpathlineto{\pgfqpoint{4.820973in}{2.456135in}}%
\pgfpathlineto{\pgfqpoint{4.821761in}{2.284108in}}%
\pgfpathlineto{\pgfqpoint{4.822104in}{2.624713in}}%
\pgfpathlineto{\pgfqpoint{4.822138in}{2.196798in}}%
\pgfpathlineto{\pgfqpoint{4.823197in}{2.408198in}}%
\pgfpathlineto{\pgfqpoint{4.824254in}{2.620820in}}%
\pgfpathlineto{\pgfqpoint{4.823402in}{2.074657in}}%
\pgfpathlineto{\pgfqpoint{4.824322in}{2.589085in}}%
\pgfpathlineto{\pgfqpoint{4.824560in}{2.591442in}}%
\pgfpathlineto{\pgfqpoint{4.825444in}{1.912571in}}%
\pgfpathlineto{\pgfqpoint{4.826224in}{2.638072in}}%
\pgfpathlineto{\pgfqpoint{4.826563in}{2.587458in}}%
\pgfpathlineto{\pgfqpoint{4.827104in}{2.201721in}}%
\pgfpathlineto{\pgfqpoint{4.827645in}{2.626154in}}%
\pgfpathlineto{\pgfqpoint{4.827679in}{2.429101in}}%
\pgfpathlineto{\pgfqpoint{4.827848in}{2.577920in}}%
\pgfpathlineto{\pgfqpoint{4.827915in}{2.510122in}}%
\pgfpathlineto{\pgfqpoint{4.827949in}{2.141613in}}%
\pgfpathlineto{\pgfqpoint{4.828826in}{2.615469in}}%
\pgfpathlineto{\pgfqpoint{4.829028in}{2.470921in}}%
\pgfpathlineto{\pgfqpoint{4.829095in}{2.106332in}}%
\pgfpathlineto{\pgfqpoint{4.829432in}{2.630250in}}%
\pgfpathlineto{\pgfqpoint{4.830070in}{2.421834in}}%
\pgfpathlineto{\pgfqpoint{4.830842in}{2.586621in}}%
\pgfpathlineto{\pgfqpoint{4.830808in}{2.262862in}}%
\pgfpathlineto{\pgfqpoint{4.831177in}{2.464747in}}%
\pgfpathlineto{\pgfqpoint{4.831779in}{2.038930in}}%
\pgfpathlineto{\pgfqpoint{4.832114in}{2.660469in}}%
\pgfpathlineto{\pgfqpoint{4.832247in}{2.511948in}}%
\pgfpathlineto{\pgfqpoint{4.832281in}{2.513726in}}%
\pgfpathlineto{\pgfqpoint{4.832314in}{2.494715in}}%
\pgfpathlineto{\pgfqpoint{4.832715in}{2.254460in}}%
\pgfpathlineto{\pgfqpoint{4.833182in}{2.622798in}}%
\pgfpathlineto{\pgfqpoint{4.833415in}{2.503172in}}%
\pgfpathlineto{\pgfqpoint{4.833648in}{2.590464in}}%
\pgfpathlineto{\pgfqpoint{4.833848in}{2.254651in}}%
\pgfpathlineto{\pgfqpoint{4.834480in}{2.588516in}}%
\pgfpathlineto{\pgfqpoint{4.835443in}{2.219938in}}%
\pgfpathlineto{\pgfqpoint{4.835045in}{2.616424in}}%
\pgfpathlineto{\pgfqpoint{4.835609in}{2.505356in}}%
\pgfpathlineto{\pgfqpoint{4.836304in}{2.321544in}}%
\pgfpathlineto{\pgfqpoint{4.835874in}{2.657614in}}%
\pgfpathlineto{\pgfqpoint{4.836668in}{2.532338in}}%
\pgfpathlineto{\pgfqpoint{4.837659in}{2.241426in}}%
\pgfpathlineto{\pgfqpoint{4.837131in}{2.626907in}}%
\pgfpathlineto{\pgfqpoint{4.837725in}{2.540360in}}%
\pgfpathlineto{\pgfqpoint{4.837758in}{2.544783in}}%
\pgfpathlineto{\pgfqpoint{4.837824in}{2.444863in}}%
\pgfpathlineto{\pgfqpoint{4.837890in}{2.518684in}}%
\pgfpathlineto{\pgfqpoint{4.837923in}{2.122976in}}%
\pgfpathlineto{\pgfqpoint{4.838713in}{2.642455in}}%
\pgfpathlineto{\pgfqpoint{4.838976in}{2.462176in}}%
\pgfpathlineto{\pgfqpoint{4.839009in}{2.608692in}}%
\pgfpathlineto{\pgfqpoint{4.839469in}{2.243002in}}%
\pgfpathlineto{\pgfqpoint{4.840093in}{2.499895in}}%
\pgfpathlineto{\pgfqpoint{4.840880in}{2.124470in}}%
\pgfpathlineto{\pgfqpoint{4.841141in}{2.642724in}}%
\pgfpathlineto{\pgfqpoint{4.841370in}{2.131648in}}%
\pgfpathlineto{\pgfqpoint{4.842252in}{2.555619in}}%
\pgfpathlineto{\pgfqpoint{4.842676in}{2.610352in}}%
\pgfpathlineto{\pgfqpoint{4.842904in}{2.366123in}}%
\pgfpathlineto{\pgfqpoint{4.843035in}{2.554161in}}%
\pgfpathlineto{\pgfqpoint{4.843848in}{2.624472in}}%
\pgfpathlineto{\pgfqpoint{4.844173in}{2.277953in}}%
\pgfpathlineto{\pgfqpoint{4.844238in}{2.627079in}}%
\pgfpathlineto{\pgfqpoint{4.845114in}{2.188023in}}%
\pgfpathlineto{\pgfqpoint{4.845309in}{2.506212in}}%
\pgfpathlineto{\pgfqpoint{4.845633in}{2.607251in}}%
\pgfpathlineto{\pgfqpoint{4.846150in}{2.202504in}}%
\pgfpathlineto{\pgfqpoint{4.846344in}{2.450578in}}%
\pgfpathlineto{\pgfqpoint{4.847313in}{1.898594in}}%
\pgfpathlineto{\pgfqpoint{4.846732in}{2.578338in}}%
\pgfpathlineto{\pgfqpoint{4.847377in}{2.395718in}}%
\pgfpathlineto{\pgfqpoint{4.847700in}{2.626036in}}%
\pgfpathlineto{\pgfqpoint{4.848408in}{2.182282in}}%
\pgfpathlineto{\pgfqpoint{4.848536in}{2.603582in}}%
\pgfpathlineto{\pgfqpoint{4.849211in}{2.244159in}}%
\pgfpathlineto{\pgfqpoint{4.848697in}{2.667635in}}%
\pgfpathlineto{\pgfqpoint{4.849660in}{2.500468in}}%
\pgfpathlineto{\pgfqpoint{4.850109in}{2.643709in}}%
\pgfpathlineto{\pgfqpoint{4.849821in}{2.262827in}}%
\pgfpathlineto{\pgfqpoint{4.850461in}{2.449989in}}%
\pgfpathlineto{\pgfqpoint{4.851548in}{2.061796in}}%
\pgfpathlineto{\pgfqpoint{4.851197in}{2.597316in}}%
\pgfpathlineto{\pgfqpoint{4.851580in}{2.391507in}}%
\pgfpathlineto{\pgfqpoint{4.851803in}{2.626296in}}%
\pgfpathlineto{\pgfqpoint{4.852250in}{2.290116in}}%
\pgfpathlineto{\pgfqpoint{4.852537in}{2.625692in}}%
\pgfpathlineto{\pgfqpoint{4.853205in}{2.253316in}}%
\pgfpathlineto{\pgfqpoint{4.852696in}{2.648297in}}%
\pgfpathlineto{\pgfqpoint{4.853650in}{2.565162in}}%
\pgfpathlineto{\pgfqpoint{4.854697in}{2.016968in}}%
\pgfpathlineto{\pgfqpoint{4.854634in}{2.626719in}}%
\pgfpathlineto{\pgfqpoint{4.854760in}{2.206943in}}%
\pgfpathlineto{\pgfqpoint{4.855014in}{2.638562in}}%
\pgfpathlineto{\pgfqpoint{4.855868in}{2.491804in}}%
\pgfpathlineto{\pgfqpoint{4.856152in}{2.144943in}}%
\pgfpathlineto{\pgfqpoint{4.856373in}{2.630726in}}%
\pgfpathlineto{\pgfqpoint{4.856941in}{2.464883in}}%
\pgfpathlineto{\pgfqpoint{4.857603in}{2.624927in}}%
\pgfpathlineto{\pgfqpoint{4.857791in}{2.263852in}}%
\pgfpathlineto{\pgfqpoint{4.858043in}{2.455359in}}%
\pgfpathlineto{\pgfqpoint{4.858106in}{2.604016in}}%
\pgfpathlineto{\pgfqpoint{4.858483in}{2.215396in}}%
\pgfpathlineto{\pgfqpoint{4.859111in}{2.483078in}}%
\pgfpathlineto{\pgfqpoint{4.859142in}{2.271886in}}%
\pgfpathlineto{\pgfqpoint{4.859769in}{2.630953in}}%
\pgfpathlineto{\pgfqpoint{4.860207in}{2.582277in}}%
\pgfpathlineto{\pgfqpoint{4.861051in}{2.604838in}}%
\pgfpathlineto{\pgfqpoint{4.860676in}{2.276984in}}%
\pgfpathlineto{\pgfqpoint{4.861082in}{2.478722in}}%
\pgfpathlineto{\pgfqpoint{4.861425in}{2.168500in}}%
\pgfpathlineto{\pgfqpoint{4.861456in}{2.649446in}}%
\pgfpathlineto{\pgfqpoint{4.862173in}{2.497165in}}%
\pgfpathlineto{\pgfqpoint{4.862453in}{2.314140in}}%
\pgfpathlineto{\pgfqpoint{4.862671in}{2.596882in}}%
\pgfpathlineto{\pgfqpoint{4.862795in}{2.529936in}}%
\pgfpathlineto{\pgfqpoint{4.863696in}{2.364309in}}%
\pgfpathlineto{\pgfqpoint{4.863913in}{2.657835in}}%
\pgfpathlineto{\pgfqpoint{4.864842in}{2.330265in}}%
\pgfpathlineto{\pgfqpoint{4.865028in}{2.580264in}}%
\pgfpathlineto{\pgfqpoint{4.865121in}{2.178926in}}%
\pgfpathlineto{\pgfqpoint{4.865090in}{2.601371in}}%
\pgfpathlineto{\pgfqpoint{4.865893in}{2.473552in}}%
\pgfpathlineto{\pgfqpoint{4.865924in}{2.669186in}}%
\pgfpathlineto{\pgfqpoint{4.866202in}{2.191925in}}%
\pgfpathlineto{\pgfqpoint{4.866972in}{2.654772in}}%
\pgfpathlineto{\pgfqpoint{4.867618in}{2.240148in}}%
\pgfpathlineto{\pgfqpoint{4.868110in}{2.444027in}}%
\pgfpathlineto{\pgfqpoint{4.869060in}{2.210596in}}%
\pgfpathlineto{\pgfqpoint{4.868785in}{2.597169in}}%
\pgfpathlineto{\pgfqpoint{4.869214in}{2.401624in}}%
\pgfpathlineto{\pgfqpoint{4.869795in}{2.604456in}}%
\pgfpathlineto{\pgfqpoint{4.869673in}{2.221538in}}%
\pgfpathlineto{\pgfqpoint{4.870376in}{2.553276in}}%
\pgfpathlineto{\pgfqpoint{4.870467in}{2.411684in}}%
\pgfpathlineto{\pgfqpoint{4.870528in}{2.499241in}}%
\pgfpathlineto{\pgfqpoint{4.871321in}{2.154099in}}%
\pgfpathlineto{\pgfqpoint{4.871078in}{2.617512in}}%
\pgfpathlineto{\pgfqpoint{4.871626in}{2.451372in}}%
\pgfpathlineto{\pgfqpoint{4.871717in}{2.635224in}}%
\pgfpathlineto{\pgfqpoint{4.872417in}{2.045921in}}%
\pgfpathlineto{\pgfqpoint{4.872630in}{2.590738in}}%
\pgfpathlineto{\pgfqpoint{4.872660in}{2.136007in}}%
\pgfpathlineto{\pgfqpoint{4.873328in}{2.599130in}}%
\pgfpathlineto{\pgfqpoint{4.873722in}{2.533975in}}%
\pgfpathlineto{\pgfqpoint{4.873752in}{2.600615in}}%
\pgfpathlineto{\pgfqpoint{4.874176in}{2.300417in}}%
\pgfpathlineto{\pgfqpoint{4.874751in}{2.536311in}}%
\pgfpathlineto{\pgfqpoint{4.875083in}{2.264167in}}%
\pgfpathlineto{\pgfqpoint{4.875174in}{2.621680in}}%
\pgfpathlineto{\pgfqpoint{4.875868in}{2.329295in}}%
\pgfpathlineto{\pgfqpoint{4.876832in}{2.623789in}}%
\pgfpathlineto{\pgfqpoint{4.876801in}{2.274525in}}%
\pgfpathlineto{\pgfqpoint{4.876982in}{2.402979in}}%
\pgfpathlineto{\pgfqpoint{4.877493in}{2.634714in}}%
\pgfpathlineto{\pgfqpoint{4.878093in}{2.238160in}}%
\pgfpathlineto{\pgfqpoint{4.879141in}{2.612768in}}%
\pgfpathlineto{\pgfqpoint{4.878872in}{2.208480in}}%
\pgfpathlineto{\pgfqpoint{4.879231in}{2.429938in}}%
\pgfpathlineto{\pgfqpoint{4.880217in}{2.618051in}}%
\pgfpathlineto{\pgfqpoint{4.879978in}{2.189471in}}%
\pgfpathlineto{\pgfqpoint{4.880366in}{2.588182in}}%
\pgfpathlineto{\pgfqpoint{4.880545in}{2.208840in}}%
\pgfpathlineto{\pgfqpoint{4.880485in}{2.601442in}}%
\pgfpathlineto{\pgfqpoint{4.881528in}{2.336636in}}%
\pgfpathlineto{\pgfqpoint{4.882301in}{2.648006in}}%
\pgfpathlineto{\pgfqpoint{4.882152in}{2.282835in}}%
\pgfpathlineto{\pgfqpoint{4.882657in}{2.529520in}}%
\pgfpathlineto{\pgfqpoint{4.883605in}{2.625050in}}%
\pgfpathlineto{\pgfqpoint{4.883131in}{2.319264in}}%
\pgfpathlineto{\pgfqpoint{4.883664in}{2.565922in}}%
\pgfpathlineto{\pgfqpoint{4.883872in}{2.240181in}}%
\pgfpathlineto{\pgfqpoint{4.884167in}{2.622027in}}%
\pgfpathlineto{\pgfqpoint{4.884758in}{2.542006in}}%
\pgfpathlineto{\pgfqpoint{4.884935in}{2.204253in}}%
\pgfpathlineto{\pgfqpoint{4.885142in}{2.612157in}}%
\pgfpathlineto{\pgfqpoint{4.885908in}{2.306526in}}%
\pgfpathlineto{\pgfqpoint{4.886585in}{2.643051in}}%
\pgfpathlineto{\pgfqpoint{4.886085in}{2.057150in}}%
\pgfpathlineto{\pgfqpoint{4.887055in}{2.593147in}}%
\pgfpathlineto{\pgfqpoint{4.887700in}{2.126947in}}%
\pgfpathlineto{\pgfqpoint{4.887818in}{2.638420in}}%
\pgfpathlineto{\pgfqpoint{4.888198in}{2.371903in}}%
\pgfpathlineto{\pgfqpoint{4.889134in}{2.648120in}}%
\pgfpathlineto{\pgfqpoint{4.888784in}{1.980711in}}%
\pgfpathlineto{\pgfqpoint{4.889339in}{2.550152in}}%
\pgfpathlineto{\pgfqpoint{4.890098in}{2.604344in}}%
\pgfpathlineto{\pgfqpoint{4.889602in}{2.182920in}}%
\pgfpathlineto{\pgfqpoint{4.890273in}{2.381115in}}%
\pgfpathlineto{\pgfqpoint{4.890331in}{2.215519in}}%
\pgfpathlineto{\pgfqpoint{4.890476in}{2.634455in}}%
\pgfpathlineto{\pgfqpoint{4.891233in}{2.513862in}}%
\pgfpathlineto{\pgfqpoint{4.891872in}{2.610182in}}%
\pgfpathlineto{\pgfqpoint{4.891959in}{2.078607in}}%
\pgfpathlineto{\pgfqpoint{4.892366in}{2.552946in}}%
\pgfpathlineto{\pgfqpoint{4.892945in}{2.102531in}}%
\pgfpathlineto{\pgfqpoint{4.893466in}{2.599720in}}%
\pgfpathlineto{\pgfqpoint{4.894593in}{2.257268in}}%
\pgfpathlineto{\pgfqpoint{4.894621in}{2.468800in}}%
\pgfpathlineto{\pgfqpoint{4.895486in}{2.620574in}}%
\pgfpathlineto{\pgfqpoint{4.895256in}{2.333506in}}%
\pgfpathlineto{\pgfqpoint{4.895716in}{2.500062in}}%
\pgfpathlineto{\pgfqpoint{4.896693in}{2.226992in}}%
\pgfpathlineto{\pgfqpoint{4.896607in}{2.647893in}}%
\pgfpathlineto{\pgfqpoint{4.896751in}{2.534205in}}%
\pgfpathlineto{\pgfqpoint{4.897639in}{2.602478in}}%
\pgfpathlineto{\pgfqpoint{4.897324in}{2.309570in}}%
\pgfpathlineto{\pgfqpoint{4.897725in}{2.419577in}}%
\pgfpathlineto{\pgfqpoint{4.898383in}{2.297791in}}%
\pgfpathlineto{\pgfqpoint{4.898326in}{2.639037in}}%
\pgfpathlineto{\pgfqpoint{4.898783in}{2.557365in}}%
\pgfpathlineto{\pgfqpoint{4.898869in}{2.151652in}}%
\pgfpathlineto{\pgfqpoint{4.899326in}{2.581492in}}%
\pgfpathlineto{\pgfqpoint{4.899896in}{2.379125in}}%
\pgfpathlineto{\pgfqpoint{4.900295in}{2.218152in}}%
\pgfpathlineto{\pgfqpoint{4.901034in}{2.598283in}}%
\pgfpathlineto{\pgfqpoint{4.901999in}{2.174043in}}%
\pgfpathlineto{\pgfqpoint{4.901829in}{2.611303in}}%
\pgfpathlineto{\pgfqpoint{4.902141in}{2.408008in}}%
\pgfpathlineto{\pgfqpoint{4.902905in}{2.601214in}}%
\pgfpathlineto{\pgfqpoint{4.902367in}{2.230098in}}%
\pgfpathlineto{\pgfqpoint{4.903244in}{2.388667in}}%
\pgfpathlineto{\pgfqpoint{4.903640in}{2.603525in}}%
\pgfpathlineto{\pgfqpoint{4.903837in}{2.194636in}}%
\pgfpathlineto{\pgfqpoint{4.904373in}{2.478729in}}%
\pgfpathlineto{\pgfqpoint{4.904627in}{2.303438in}}%
\pgfpathlineto{\pgfqpoint{4.904909in}{2.580324in}}%
\pgfpathlineto{\pgfqpoint{4.905387in}{2.547015in}}%
\pgfpathlineto{\pgfqpoint{4.906146in}{2.593547in}}%
\pgfpathlineto{\pgfqpoint{4.905471in}{2.352443in}}%
\pgfpathlineto{\pgfqpoint{4.906230in}{2.543945in}}%
\pgfpathlineto{\pgfqpoint{4.906595in}{2.262033in}}%
\pgfpathlineto{\pgfqpoint{4.906959in}{2.606516in}}%
\pgfpathlineto{\pgfqpoint{4.907351in}{2.320860in}}%
\pgfpathlineto{\pgfqpoint{4.908162in}{2.603745in}}%
\pgfpathlineto{\pgfqpoint{4.908246in}{2.289847in}}%
\pgfpathlineto{\pgfqpoint{4.908469in}{2.524513in}}%
\pgfpathlineto{\pgfqpoint{4.908609in}{2.593791in}}%
\pgfpathlineto{\pgfqpoint{4.908860in}{2.291097in}}%
\pgfpathlineto{\pgfqpoint{4.909362in}{2.536706in}}%
\pgfpathlineto{\pgfqpoint{4.909946in}{2.300216in}}%
\pgfpathlineto{\pgfqpoint{4.909529in}{2.622932in}}%
\pgfpathlineto{\pgfqpoint{4.910447in}{2.493356in}}%
\pgfpathlineto{\pgfqpoint{4.910475in}{2.588886in}}%
\pgfpathlineto{\pgfqpoint{4.910974in}{2.223852in}}%
\pgfpathlineto{\pgfqpoint{4.911557in}{2.523126in}}%
\pgfpathlineto{\pgfqpoint{4.912332in}{2.113090in}}%
\pgfpathlineto{\pgfqpoint{4.911834in}{2.606647in}}%
\pgfpathlineto{\pgfqpoint{4.912692in}{2.435234in}}%
\pgfpathlineto{\pgfqpoint{4.912775in}{2.623267in}}%
\pgfpathlineto{\pgfqpoint{4.913410in}{2.146702in}}%
\pgfpathlineto{\pgfqpoint{4.913631in}{2.562583in}}%
\pgfpathlineto{\pgfqpoint{4.913658in}{1.995712in}}%
\pgfpathlineto{\pgfqpoint{4.914650in}{2.593636in}}%
\pgfpathlineto{\pgfqpoint{4.914732in}{2.489301in}}%
\pgfpathlineto{\pgfqpoint{4.914760in}{2.621237in}}%
\pgfpathlineto{\pgfqpoint{4.915447in}{2.279231in}}%
\pgfpathlineto{\pgfqpoint{4.915831in}{2.446868in}}%
\pgfpathlineto{\pgfqpoint{4.915941in}{2.562140in}}%
\pgfpathlineto{\pgfqpoint{4.916654in}{2.255838in}}%
\pgfpathlineto{\pgfqpoint{4.916955in}{2.516745in}}%
\pgfpathlineto{\pgfqpoint{4.918048in}{2.214951in}}%
\pgfpathlineto{\pgfqpoint{4.917228in}{2.603323in}}%
\pgfpathlineto{\pgfqpoint{4.918075in}{2.325976in}}%
\pgfpathlineto{\pgfqpoint{4.918212in}{2.603247in}}%
\pgfpathlineto{\pgfqpoint{4.918812in}{2.246785in}}%
\pgfpathlineto{\pgfqpoint{4.919193in}{2.564429in}}%
\pgfpathlineto{\pgfqpoint{4.919438in}{2.203857in}}%
\pgfpathlineto{\pgfqpoint{4.920226in}{2.572985in}}%
\pgfpathlineto{\pgfqpoint{4.920308in}{2.476475in}}%
\pgfpathlineto{\pgfqpoint{4.920606in}{2.058623in}}%
\pgfpathlineto{\pgfqpoint{4.920579in}{2.603751in}}%
\pgfpathlineto{\pgfqpoint{4.921284in}{2.356655in}}%
\pgfpathlineto{\pgfqpoint{4.922069in}{2.624118in}}%
\pgfpathlineto{\pgfqpoint{4.921825in}{2.114887in}}%
\pgfpathlineto{\pgfqpoint{4.922420in}{2.592982in}}%
\pgfpathlineto{\pgfqpoint{4.922447in}{2.215901in}}%
\pgfpathlineto{\pgfqpoint{4.922582in}{2.631748in}}%
\pgfpathlineto{\pgfqpoint{4.923526in}{2.506515in}}%
\pgfpathlineto{\pgfqpoint{4.923957in}{2.151515in}}%
\pgfpathlineto{\pgfqpoint{4.923634in}{2.590353in}}%
\pgfpathlineto{\pgfqpoint{4.924603in}{2.492743in}}%
\pgfpathlineto{\pgfqpoint{4.924630in}{2.599295in}}%
\pgfpathlineto{\pgfqpoint{4.924818in}{2.223683in}}%
\pgfpathlineto{\pgfqpoint{4.925676in}{2.381349in}}%
\pgfpathlineto{\pgfqpoint{4.925730in}{2.178537in}}%
\pgfpathlineto{\pgfqpoint{4.926426in}{2.596404in}}%
\pgfpathlineto{\pgfqpoint{4.926828in}{2.202748in}}%
\pgfpathlineto{\pgfqpoint{4.926961in}{2.599429in}}%
\pgfpathlineto{\pgfqpoint{4.927949in}{2.395884in}}%
\pgfpathlineto{\pgfqpoint{4.928375in}{2.609925in}}%
\pgfpathlineto{\pgfqpoint{4.928748in}{1.920826in}}%
\pgfpathlineto{\pgfqpoint{4.929068in}{2.520160in}}%
\pgfpathlineto{\pgfqpoint{4.930077in}{2.251722in}}%
\pgfpathlineto{\pgfqpoint{4.929307in}{2.598507in}}%
\pgfpathlineto{\pgfqpoint{4.930183in}{2.314002in}}%
\pgfpathlineto{\pgfqpoint{4.930872in}{2.623191in}}%
\pgfpathlineto{\pgfqpoint{4.931243in}{2.150528in}}%
\pgfpathlineto{\pgfqpoint{4.931269in}{2.443101in}}%
\pgfpathlineto{\pgfqpoint{4.932221in}{2.116495in}}%
\pgfpathlineto{\pgfqpoint{4.931798in}{2.592300in}}%
\pgfpathlineto{\pgfqpoint{4.932379in}{2.308945in}}%
\pgfpathlineto{\pgfqpoint{4.933275in}{2.603398in}}%
\pgfpathlineto{\pgfqpoint{4.933249in}{2.244761in}}%
\pgfpathlineto{\pgfqpoint{4.933539in}{2.528509in}}%
\pgfpathlineto{\pgfqpoint{4.933565in}{2.308329in}}%
\pgfpathlineto{\pgfqpoint{4.934511in}{2.600749in}}%
\pgfpathlineto{\pgfqpoint{4.934669in}{2.352987in}}%
\pgfpathlineto{\pgfqpoint{4.935717in}{2.615131in}}%
\pgfpathlineto{\pgfqpoint{4.934747in}{2.278658in}}%
\pgfpathlineto{\pgfqpoint{4.935822in}{2.597564in}}%
\pgfpathlineto{\pgfqpoint{4.936659in}{2.157883in}}%
\pgfpathlineto{\pgfqpoint{4.936868in}{2.609141in}}%
\pgfpathlineto{\pgfqpoint{4.936920in}{2.497116in}}%
\pgfpathlineto{\pgfqpoint{4.937859in}{2.605490in}}%
\pgfpathlineto{\pgfqpoint{4.937051in}{2.303570in}}%
\pgfpathlineto{\pgfqpoint{4.937911in}{2.587246in}}%
\pgfpathlineto{\pgfqpoint{4.938614in}{2.179855in}}%
\pgfpathlineto{\pgfqpoint{4.938327in}{2.646201in}}%
\pgfpathlineto{\pgfqpoint{4.939004in}{2.290049in}}%
\pgfpathlineto{\pgfqpoint{4.939107in}{2.588741in}}%
\pgfpathlineto{\pgfqpoint{4.939081in}{2.230116in}}%
\pgfpathlineto{\pgfqpoint{4.940119in}{2.498536in}}%
\pgfpathlineto{\pgfqpoint{4.941077in}{2.606991in}}%
\pgfpathlineto{\pgfqpoint{4.940663in}{2.138644in}}%
\pgfpathlineto{\pgfqpoint{4.941129in}{2.457319in}}%
\pgfpathlineto{\pgfqpoint{4.941284in}{2.611743in}}%
\pgfpathlineto{\pgfqpoint{4.942239in}{2.294706in}}%
\pgfpathlineto{\pgfqpoint{4.943140in}{2.623799in}}%
\pgfpathlineto{\pgfqpoint{4.943320in}{2.592296in}}%
\pgfpathlineto{\pgfqpoint{4.943346in}{2.089046in}}%
\pgfpathlineto{\pgfqpoint{4.943731in}{2.597147in}}%
\pgfpathlineto{\pgfqpoint{4.944425in}{2.506189in}}%
\pgfpathlineto{\pgfqpoint{4.944655in}{2.104042in}}%
\pgfpathlineto{\pgfqpoint{4.944912in}{2.598908in}}%
\pgfpathlineto{\pgfqpoint{4.945526in}{2.484450in}}%
\pgfpathlineto{\pgfqpoint{4.945986in}{2.304737in}}%
\pgfpathlineto{\pgfqpoint{4.945782in}{2.559852in}}%
\pgfpathlineto{\pgfqpoint{4.946548in}{2.364286in}}%
\pgfpathlineto{\pgfqpoint{4.947364in}{2.590922in}}%
\pgfpathlineto{\pgfqpoint{4.947109in}{2.140685in}}%
\pgfpathlineto{\pgfqpoint{4.947670in}{2.475746in}}%
\pgfpathlineto{\pgfqpoint{4.948407in}{2.637512in}}%
\pgfpathlineto{\pgfqpoint{4.947746in}{2.187578in}}%
\pgfpathlineto{\pgfqpoint{4.948712in}{2.418262in}}%
\pgfpathlineto{\pgfqpoint{4.949245in}{2.171148in}}%
\pgfpathlineto{\pgfqpoint{4.949144in}{2.591979in}}%
\pgfpathlineto{\pgfqpoint{4.949549in}{2.479378in}}%
\pgfpathlineto{\pgfqpoint{4.950359in}{2.613517in}}%
\pgfpathlineto{\pgfqpoint{4.949701in}{2.080857in}}%
\pgfpathlineto{\pgfqpoint{4.950612in}{2.507862in}}%
\pgfpathlineto{\pgfqpoint{4.950815in}{2.640058in}}%
\pgfpathlineto{\pgfqpoint{4.951698in}{2.135494in}}%
\pgfpathlineto{\pgfqpoint{4.952378in}{2.591042in}}%
\pgfpathlineto{\pgfqpoint{4.952806in}{2.545737in}}%
\pgfpathlineto{\pgfqpoint{4.953083in}{2.122740in}}%
\pgfpathlineto{\pgfqpoint{4.953635in}{2.595588in}}%
\pgfpathlineto{\pgfqpoint{4.953936in}{2.453232in}}%
\pgfpathlineto{\pgfqpoint{4.954237in}{2.204322in}}%
\pgfpathlineto{\pgfqpoint{4.954488in}{2.601633in}}%
\pgfpathlineto{\pgfqpoint{4.954638in}{2.443187in}}%
\pgfpathlineto{\pgfqpoint{4.954663in}{2.625815in}}%
\pgfpathlineto{\pgfqpoint{4.955138in}{2.087510in}}%
\pgfpathlineto{\pgfqpoint{4.955738in}{2.436194in}}%
\pgfpathlineto{\pgfqpoint{4.956387in}{2.645211in}}%
\pgfpathlineto{\pgfqpoint{4.955913in}{2.137823in}}%
\pgfpathlineto{\pgfqpoint{4.956811in}{2.502284in}}%
\pgfpathlineto{\pgfqpoint{4.957483in}{2.228296in}}%
\pgfpathlineto{\pgfqpoint{4.957433in}{2.598641in}}%
\pgfpathlineto{\pgfqpoint{4.957906in}{2.482108in}}%
\pgfpathlineto{\pgfqpoint{4.957930in}{2.565568in}}%
\pgfpathlineto{\pgfqpoint{4.958229in}{2.299446in}}%
\pgfpathlineto{\pgfqpoint{4.958973in}{2.371858in}}%
\pgfpathlineto{\pgfqpoint{4.959864in}{2.202560in}}%
\pgfpathlineto{\pgfqpoint{4.959394in}{2.597807in}}%
\pgfpathlineto{\pgfqpoint{4.960012in}{2.384266in}}%
\pgfpathlineto{\pgfqpoint{4.960334in}{2.620256in}}%
\pgfpathlineto{\pgfqpoint{4.960828in}{1.973077in}}%
\pgfpathlineto{\pgfqpoint{4.961124in}{2.465753in}}%
\pgfpathlineto{\pgfqpoint{4.961296in}{2.243402in}}%
\pgfpathlineto{\pgfqpoint{4.961666in}{2.596969in}}%
\pgfpathlineto{\pgfqpoint{4.962207in}{2.553647in}}%
\pgfpathlineto{\pgfqpoint{4.962453in}{2.241563in}}%
\pgfpathlineto{\pgfqpoint{4.962330in}{2.610352in}}%
\pgfpathlineto{\pgfqpoint{4.963288in}{2.419945in}}%
\pgfpathlineto{\pgfqpoint{4.963681in}{2.617135in}}%
\pgfpathlineto{\pgfqpoint{4.963705in}{2.110345in}}%
\pgfpathlineto{\pgfqpoint{4.964415in}{2.583625in}}%
\pgfpathlineto{\pgfqpoint{4.964978in}{2.140520in}}%
\pgfpathlineto{\pgfqpoint{4.964587in}{2.636830in}}%
\pgfpathlineto{\pgfqpoint{4.965588in}{2.352659in}}%
\pgfpathlineto{\pgfqpoint{4.966247in}{2.583067in}}%
\pgfpathlineto{\pgfqpoint{4.965662in}{2.082627in}}%
\pgfpathlineto{\pgfqpoint{4.966685in}{2.378017in}}%
\pgfpathlineto{\pgfqpoint{4.967415in}{2.180263in}}%
\pgfpathlineto{\pgfqpoint{4.966758in}{2.613332in}}%
\pgfpathlineto{\pgfqpoint{4.967609in}{2.460798in}}%
\pgfpathlineto{\pgfqpoint{4.968143in}{2.613917in}}%
\pgfpathlineto{\pgfqpoint{4.967876in}{2.169191in}}%
\pgfpathlineto{\pgfqpoint{4.968725in}{2.560161in}}%
\pgfpathlineto{\pgfqpoint{4.969354in}{2.235015in}}%
\pgfpathlineto{\pgfqpoint{4.969015in}{2.586001in}}%
\pgfpathlineto{\pgfqpoint{4.969862in}{2.406306in}}%
\pgfpathlineto{\pgfqpoint{4.969934in}{2.575499in}}%
\pgfpathlineto{\pgfqpoint{4.970393in}{2.224821in}}%
\pgfpathlineto{\pgfqpoint{4.970923in}{2.535051in}}%
\pgfpathlineto{\pgfqpoint{4.971044in}{2.155531in}}%
\pgfpathlineto{\pgfqpoint{4.971694in}{2.617738in}}%
\pgfpathlineto{\pgfqpoint{4.972030in}{2.171545in}}%
\pgfpathlineto{\pgfqpoint{4.972319in}{2.613448in}}%
\pgfpathlineto{\pgfqpoint{4.972631in}{2.121203in}}%
\pgfpathlineto{\pgfqpoint{4.973135in}{2.348755in}}%
\pgfpathlineto{\pgfqpoint{4.973254in}{2.178270in}}%
\pgfpathlineto{\pgfqpoint{4.974140in}{2.625340in}}%
\pgfpathlineto{\pgfqpoint{4.974164in}{2.632776in}}%
\pgfpathlineto{\pgfqpoint{4.974331in}{2.346149in}}%
\pgfpathlineto{\pgfqpoint{4.974355in}{2.344002in}}%
\pgfpathlineto{\pgfqpoint{4.975263in}{2.626795in}}%
\pgfpathlineto{\pgfqpoint{4.974809in}{2.265484in}}%
\pgfpathlineto{\pgfqpoint{4.975477in}{2.545877in}}%
\pgfpathlineto{\pgfqpoint{4.976525in}{2.153266in}}%
\pgfpathlineto{\pgfqpoint{4.976002in}{2.585353in}}%
\pgfpathlineto{\pgfqpoint{4.976573in}{2.559426in}}%
\pgfpathlineto{\pgfqpoint{4.976834in}{1.977109in}}%
\pgfpathlineto{\pgfqpoint{4.976787in}{2.591148in}}%
\pgfpathlineto{\pgfqpoint{4.977736in}{2.341710in}}%
\pgfpathlineto{\pgfqpoint{4.978376in}{2.626040in}}%
\pgfpathlineto{\pgfqpoint{4.978826in}{2.490401in}}%
\pgfpathlineto{\pgfqpoint{4.978849in}{2.158373in}}%
\pgfpathlineto{\pgfqpoint{4.979298in}{2.584899in}}%
\pgfpathlineto{\pgfqpoint{4.979936in}{2.421523in}}%
\pgfpathlineto{\pgfqpoint{4.980172in}{2.621392in}}%
\pgfpathlineto{\pgfqpoint{4.980926in}{2.077762in}}%
\pgfpathlineto{\pgfqpoint{4.981043in}{2.607217in}}%
\pgfpathlineto{\pgfqpoint{4.981278in}{2.193952in}}%
\pgfpathlineto{\pgfqpoint{4.981842in}{2.644316in}}%
\pgfpathlineto{\pgfqpoint{4.982171in}{2.260686in}}%
\pgfpathlineto{\pgfqpoint{4.982992in}{2.605635in}}%
\pgfpathlineto{\pgfqpoint{4.983296in}{2.522864in}}%
\pgfpathlineto{\pgfqpoint{4.983904in}{2.187946in}}%
\pgfpathlineto{\pgfqpoint{4.983881in}{2.600012in}}%
\pgfpathlineto{\pgfqpoint{4.984395in}{2.435528in}}%
\pgfpathlineto{\pgfqpoint{4.984908in}{2.646317in}}%
\pgfpathlineto{\pgfqpoint{4.984465in}{2.184826in}}%
\pgfpathlineto{\pgfqpoint{4.985491in}{2.598954in}}%
\pgfpathlineto{\pgfqpoint{4.986188in}{2.188264in}}%
\pgfpathlineto{\pgfqpoint{4.986119in}{2.616362in}}%
\pgfpathlineto{\pgfqpoint{4.986607in}{2.420051in}}%
\pgfpathlineto{\pgfqpoint{4.987280in}{2.595344in}}%
\pgfpathlineto{\pgfqpoint{4.987326in}{2.064639in}}%
\pgfpathlineto{\pgfqpoint{4.987604in}{2.344231in}}%
\pgfpathlineto{\pgfqpoint{4.987627in}{2.274804in}}%
\pgfpathlineto{\pgfqpoint{4.988345in}{2.588827in}}%
\pgfpathlineto{\pgfqpoint{4.988669in}{2.355518in}}%
\pgfpathlineto{\pgfqpoint{4.988692in}{2.613174in}}%
\pgfpathlineto{\pgfqpoint{4.989384in}{2.209770in}}%
\pgfpathlineto{\pgfqpoint{4.989776in}{2.579502in}}%
\pgfpathlineto{\pgfqpoint{4.989915in}{1.864308in}}%
\pgfpathlineto{\pgfqpoint{4.990467in}{2.622776in}}%
\pgfpathlineto{\pgfqpoint{4.990881in}{2.433281in}}%
\pgfpathlineto{\pgfqpoint{4.991915in}{2.602071in}}%
\pgfpathlineto{\pgfqpoint{4.991594in}{2.251899in}}%
\pgfpathlineto{\pgfqpoint{4.991984in}{2.455830in}}%
\pgfpathlineto{\pgfqpoint{4.992785in}{2.084783in}}%
\pgfpathlineto{\pgfqpoint{4.992419in}{2.628757in}}%
\pgfpathlineto{\pgfqpoint{4.992854in}{2.400489in}}%
\pgfpathlineto{\pgfqpoint{4.993289in}{2.580535in}}%
\pgfpathlineto{\pgfqpoint{4.993449in}{2.099493in}}%
\pgfpathlineto{\pgfqpoint{4.993951in}{2.463853in}}%
\pgfpathlineto{\pgfqpoint{4.994590in}{2.183244in}}%
\pgfpathlineto{\pgfqpoint{4.994749in}{2.625219in}}%
\pgfpathlineto{\pgfqpoint{4.995045in}{2.525126in}}%
\pgfpathlineto{\pgfqpoint{4.995341in}{2.213829in}}%
\pgfpathlineto{\pgfqpoint{4.995682in}{2.569329in}}%
\pgfpathlineto{\pgfqpoint{4.996091in}{2.412016in}}%
\pgfpathlineto{\pgfqpoint{4.996114in}{2.638535in}}%
\pgfpathlineto{\pgfqpoint{4.996999in}{2.139437in}}%
\pgfpathlineto{\pgfqpoint{4.997203in}{2.509301in}}%
\pgfpathlineto{\pgfqpoint{4.997316in}{2.286049in}}%
\pgfpathlineto{\pgfqpoint{4.998040in}{2.607510in}}%
\pgfpathlineto{\pgfqpoint{4.998266in}{2.349188in}}%
\pgfpathlineto{\pgfqpoint{4.998695in}{2.612055in}}%
\pgfpathlineto{\pgfqpoint{4.999146in}{2.169651in}}%
\pgfpathlineto{\pgfqpoint{4.999371in}{2.455486in}}%
\pgfpathlineto{\pgfqpoint{5.000204in}{2.169458in}}%
\pgfpathlineto{\pgfqpoint{4.999777in}{2.584595in}}%
\pgfpathlineto{\pgfqpoint{5.000474in}{2.411367in}}%
\pgfpathlineto{\pgfqpoint{5.000901in}{2.618662in}}%
\pgfpathlineto{\pgfqpoint{5.001372in}{2.201257in}}%
\pgfpathlineto{\pgfqpoint{5.001552in}{2.502000in}}%
\pgfpathlineto{\pgfqpoint{5.002492in}{2.161977in}}%
\pgfpathlineto{\pgfqpoint{5.002179in}{2.615690in}}%
\pgfpathlineto{\pgfqpoint{5.002649in}{2.488936in}}%
\pgfpathlineto{\pgfqpoint{5.003230in}{2.648448in}}%
\pgfpathlineto{\pgfqpoint{5.002738in}{2.084959in}}%
\pgfpathlineto{\pgfqpoint{5.003743in}{2.553799in}}%
\pgfpathlineto{\pgfqpoint{5.004590in}{1.859269in}}%
\pgfpathlineto{\pgfqpoint{5.004545in}{2.644678in}}%
\pgfpathlineto{\pgfqpoint{5.004835in}{2.507592in}}%
\pgfpathlineto{\pgfqpoint{5.004857in}{2.595372in}}%
\pgfpathlineto{\pgfqpoint{5.005635in}{2.154670in}}%
\pgfpathlineto{\pgfqpoint{5.005923in}{2.432078in}}%
\pgfpathlineto{\pgfqpoint{5.006168in}{2.221742in}}%
\pgfpathlineto{\pgfqpoint{5.006456in}{2.589291in}}%
\pgfpathlineto{\pgfqpoint{5.006965in}{2.393157in}}%
\pgfpathlineto{\pgfqpoint{5.007916in}{2.589087in}}%
\pgfpathlineto{\pgfqpoint{5.007629in}{2.093828in}}%
\pgfpathlineto{\pgfqpoint{5.008070in}{2.437205in}}%
\pgfpathlineto{\pgfqpoint{5.008137in}{2.599684in}}%
\pgfpathlineto{\pgfqpoint{5.008666in}{2.214120in}}%
\pgfpathlineto{\pgfqpoint{5.009151in}{2.503682in}}%
\pgfpathlineto{\pgfqpoint{5.009547in}{2.237551in}}%
\pgfpathlineto{\pgfqpoint{5.009679in}{2.609433in}}%
\pgfpathlineto{\pgfqpoint{5.010273in}{2.374717in}}%
\pgfpathlineto{\pgfqpoint{5.010602in}{2.599757in}}%
\pgfpathlineto{\pgfqpoint{5.010668in}{2.175505in}}%
\pgfpathlineto{\pgfqpoint{5.011369in}{2.422831in}}%
\pgfpathlineto{\pgfqpoint{5.011391in}{2.085343in}}%
\pgfpathlineto{\pgfqpoint{5.011917in}{2.625572in}}%
\pgfpathlineto{\pgfqpoint{5.012463in}{2.367702in}}%
\pgfpathlineto{\pgfqpoint{5.013554in}{2.633482in}}%
\pgfpathlineto{\pgfqpoint{5.013249in}{2.221841in}}%
\pgfpathlineto{\pgfqpoint{5.013576in}{2.528059in}}%
\pgfpathlineto{\pgfqpoint{5.014664in}{2.149207in}}%
\pgfpathlineto{\pgfqpoint{5.013903in}{2.650279in}}%
\pgfpathlineto{\pgfqpoint{5.014686in}{2.429395in}}%
\pgfpathlineto{\pgfqpoint{5.015229in}{2.167531in}}%
\pgfpathlineto{\pgfqpoint{5.015251in}{2.614758in}}%
\pgfpathlineto{\pgfqpoint{5.015793in}{2.437778in}}%
\pgfpathlineto{\pgfqpoint{5.016032in}{2.022611in}}%
\pgfpathlineto{\pgfqpoint{5.016660in}{2.599798in}}%
\pgfpathlineto{\pgfqpoint{5.016789in}{2.095754in}}%
\pgfpathlineto{\pgfqpoint{5.017589in}{2.584161in}}%
\pgfpathlineto{\pgfqpoint{5.017913in}{2.408263in}}%
\pgfpathlineto{\pgfqpoint{5.018624in}{2.594643in}}%
\pgfpathlineto{\pgfqpoint{5.018279in}{2.238273in}}%
\pgfpathlineto{\pgfqpoint{5.019011in}{2.484402in}}%
\pgfpathlineto{\pgfqpoint{5.019570in}{2.060287in}}%
\pgfpathlineto{\pgfqpoint{5.019635in}{2.595039in}}%
\pgfpathlineto{\pgfqpoint{5.020086in}{2.477993in}}%
\pgfpathlineto{\pgfqpoint{5.020579in}{2.606534in}}%
\pgfpathlineto{\pgfqpoint{5.020836in}{2.202982in}}%
\pgfpathlineto{\pgfqpoint{5.021200in}{2.521789in}}%
\pgfpathlineto{\pgfqpoint{5.021607in}{2.649797in}}%
\pgfpathlineto{\pgfqpoint{5.021307in}{2.254033in}}%
\pgfpathlineto{\pgfqpoint{5.022141in}{2.412838in}}%
\pgfpathlineto{\pgfqpoint{5.023144in}{2.128707in}}%
\pgfpathlineto{\pgfqpoint{5.023123in}{2.616731in}}%
\pgfpathlineto{\pgfqpoint{5.023250in}{2.414521in}}%
\pgfpathlineto{\pgfqpoint{5.023549in}{2.621587in}}%
\pgfpathlineto{\pgfqpoint{5.023676in}{2.192961in}}%
\pgfpathlineto{\pgfqpoint{5.024357in}{2.538423in}}%
\pgfpathlineto{\pgfqpoint{5.024378in}{2.125831in}}%
\pgfpathlineto{\pgfqpoint{5.025121in}{2.607024in}}%
\pgfpathlineto{\pgfqpoint{5.025460in}{2.535801in}}%
\pgfpathlineto{\pgfqpoint{5.026307in}{2.193975in}}%
\pgfpathlineto{\pgfqpoint{5.026180in}{2.599683in}}%
\pgfpathlineto{\pgfqpoint{5.026582in}{2.377593in}}%
\pgfpathlineto{\pgfqpoint{5.027300in}{2.209115in}}%
\pgfpathlineto{\pgfqpoint{5.027679in}{2.587265in}}%
\pgfpathlineto{\pgfqpoint{5.028627in}{2.147693in}}%
\pgfpathlineto{\pgfqpoint{5.028374in}{2.606292in}}%
\pgfpathlineto{\pgfqpoint{5.028795in}{2.460459in}}%
\pgfpathlineto{\pgfqpoint{5.028984in}{2.606846in}}%
\pgfpathlineto{\pgfqpoint{5.029363in}{2.194495in}}%
\pgfpathlineto{\pgfqpoint{5.029803in}{2.496233in}}%
\pgfpathlineto{\pgfqpoint{5.029824in}{2.169028in}}%
\pgfpathlineto{\pgfqpoint{5.030160in}{2.592588in}}%
\pgfpathlineto{\pgfqpoint{5.030913in}{2.440970in}}%
\pgfpathlineto{\pgfqpoint{5.030976in}{2.249964in}}%
\pgfpathlineto{\pgfqpoint{5.031164in}{2.590657in}}%
\pgfpathlineto{\pgfqpoint{5.031937in}{2.379713in}}%
\pgfpathlineto{\pgfqpoint{5.032021in}{2.269616in}}%
\pgfpathlineto{\pgfqpoint{5.033021in}{2.572730in}}%
\pgfpathlineto{\pgfqpoint{5.033500in}{2.270993in}}%
\pgfpathlineto{\pgfqpoint{5.034144in}{2.395878in}}%
\pgfpathlineto{\pgfqpoint{5.034829in}{2.220769in}}%
\pgfpathlineto{\pgfqpoint{5.034414in}{2.609469in}}%
\pgfpathlineto{\pgfqpoint{5.035160in}{2.394491in}}%
\pgfpathlineto{\pgfqpoint{5.035740in}{2.593683in}}%
\pgfpathlineto{\pgfqpoint{5.035284in}{2.215874in}}%
\pgfpathlineto{\pgfqpoint{5.036236in}{2.568978in}}%
\pgfpathlineto{\pgfqpoint{5.036360in}{2.170246in}}%
\pgfpathlineto{\pgfqpoint{5.036958in}{2.607391in}}%
\pgfpathlineto{\pgfqpoint{5.037350in}{2.474832in}}%
\pgfpathlineto{\pgfqpoint{5.037680in}{2.611980in}}%
\pgfpathlineto{\pgfqpoint{5.038317in}{2.115729in}}%
\pgfpathlineto{\pgfqpoint{5.038461in}{2.493145in}}%
\pgfpathlineto{\pgfqpoint{5.039508in}{2.140052in}}%
\pgfpathlineto{\pgfqpoint{5.038749in}{2.592077in}}%
\pgfpathlineto{\pgfqpoint{5.039570in}{2.373283in}}%
\pgfpathlineto{\pgfqpoint{5.040389in}{2.640720in}}%
\pgfpathlineto{\pgfqpoint{5.040164in}{2.243679in}}%
\pgfpathlineto{\pgfqpoint{5.040675in}{2.468953in}}%
\pgfpathlineto{\pgfqpoint{5.041574in}{2.205661in}}%
\pgfpathlineto{\pgfqpoint{5.041431in}{2.589542in}}%
\pgfpathlineto{\pgfqpoint{5.041717in}{2.349680in}}%
\pgfpathlineto{\pgfqpoint{5.041737in}{2.595477in}}%
\pgfpathlineto{\pgfqpoint{5.042511in}{2.155132in}}%
\pgfpathlineto{\pgfqpoint{5.042816in}{2.307221in}}%
\pgfpathlineto{\pgfqpoint{5.043325in}{2.567078in}}%
\pgfpathlineto{\pgfqpoint{5.043508in}{2.055268in}}%
\pgfpathlineto{\pgfqpoint{5.043954in}{2.526938in}}%
\pgfpathlineto{\pgfqpoint{5.044562in}{2.594961in}}%
\pgfpathlineto{\pgfqpoint{5.044197in}{2.294874in}}%
\pgfpathlineto{\pgfqpoint{5.044846in}{2.422896in}}%
\pgfpathlineto{\pgfqpoint{5.045736in}{2.590114in}}%
\pgfpathlineto{\pgfqpoint{5.045958in}{2.062592in}}%
\pgfpathlineto{\pgfqpoint{5.046664in}{2.596124in}}%
\pgfpathlineto{\pgfqpoint{5.047087in}{2.474743in}}%
\pgfpathlineto{\pgfqpoint{5.047631in}{2.189540in}}%
\pgfpathlineto{\pgfqpoint{5.047912in}{2.595662in}}%
\pgfpathlineto{\pgfqpoint{5.048193in}{2.455954in}}%
\pgfpathlineto{\pgfqpoint{5.048816in}{2.610581in}}%
\pgfpathlineto{\pgfqpoint{5.048434in}{2.243285in}}%
\pgfpathlineto{\pgfqpoint{5.049277in}{2.432369in}}%
\pgfpathlineto{\pgfqpoint{5.050197in}{2.044161in}}%
\pgfpathlineto{\pgfqpoint{5.050277in}{2.583496in}}%
\pgfpathlineto{\pgfqpoint{5.050377in}{2.283490in}}%
\pgfpathlineto{\pgfqpoint{5.051315in}{2.578394in}}%
\pgfpathlineto{\pgfqpoint{5.050816in}{1.952230in}}%
\pgfpathlineto{\pgfqpoint{5.051495in}{2.401269in}}%
\pgfpathlineto{\pgfqpoint{5.052032in}{2.591594in}}%
\pgfpathlineto{\pgfqpoint{5.051853in}{2.217944in}}%
\pgfpathlineto{\pgfqpoint{5.052589in}{2.530324in}}%
\pgfpathlineto{\pgfqpoint{5.053404in}{2.159232in}}%
\pgfpathlineto{\pgfqpoint{5.053146in}{2.583644in}}%
\pgfpathlineto{\pgfqpoint{5.053701in}{2.194807in}}%
\pgfpathlineto{\pgfqpoint{5.054177in}{2.588382in}}%
\pgfpathlineto{\pgfqpoint{5.054098in}{2.184464in}}%
\pgfpathlineto{\pgfqpoint{5.054810in}{2.453321in}}%
\pgfpathlineto{\pgfqpoint{5.055087in}{2.129750in}}%
\pgfpathlineto{\pgfqpoint{5.055442in}{2.586076in}}%
\pgfpathlineto{\pgfqpoint{5.055916in}{2.306716in}}%
\pgfpathlineto{\pgfqpoint{5.056094in}{2.630626in}}%
\pgfpathlineto{\pgfqpoint{5.056763in}{2.140700in}}%
\pgfpathlineto{\pgfqpoint{5.057039in}{2.508496in}}%
\pgfpathlineto{\pgfqpoint{5.057511in}{2.600876in}}%
\pgfpathlineto{\pgfqpoint{5.057314in}{2.105099in}}%
\pgfpathlineto{\pgfqpoint{5.058139in}{2.489623in}}%
\pgfpathlineto{\pgfqpoint{5.058923in}{2.197347in}}%
\pgfpathlineto{\pgfqpoint{5.058943in}{2.573433in}}%
\pgfpathlineto{\pgfqpoint{5.059256in}{2.452906in}}%
\pgfpathlineto{\pgfqpoint{5.059530in}{2.587097in}}%
\pgfpathlineto{\pgfqpoint{5.059843in}{2.188466in}}%
\pgfpathlineto{\pgfqpoint{5.060370in}{2.550548in}}%
\pgfpathlineto{\pgfqpoint{5.061286in}{2.191093in}}%
\pgfpathlineto{\pgfqpoint{5.061247in}{2.587589in}}%
\pgfpathlineto{\pgfqpoint{5.061462in}{2.511300in}}%
\pgfpathlineto{\pgfqpoint{5.061481in}{2.594677in}}%
\pgfpathlineto{\pgfqpoint{5.061520in}{2.175894in}}%
\pgfpathlineto{\pgfqpoint{5.062550in}{2.433191in}}%
\pgfpathlineto{\pgfqpoint{5.063559in}{2.155140in}}%
\pgfpathlineto{\pgfqpoint{5.063074in}{2.617654in}}%
\pgfpathlineto{\pgfqpoint{5.063656in}{2.308706in}}%
\pgfpathlineto{\pgfqpoint{5.063849in}{2.627926in}}%
\pgfpathlineto{\pgfqpoint{5.064333in}{2.245823in}}%
\pgfpathlineto{\pgfqpoint{5.064758in}{2.365181in}}%
\pgfpathlineto{\pgfqpoint{5.065260in}{2.039820in}}%
\pgfpathlineto{\pgfqpoint{5.064951in}{2.623561in}}%
\pgfpathlineto{\pgfqpoint{5.065588in}{2.484931in}}%
\pgfpathlineto{\pgfqpoint{5.066493in}{2.608146in}}%
\pgfpathlineto{\pgfqpoint{5.066147in}{2.205958in}}%
\pgfpathlineto{\pgfqpoint{5.066686in}{2.496717in}}%
\pgfpathlineto{\pgfqpoint{5.066916in}{2.178511in}}%
\pgfpathlineto{\pgfqpoint{5.067262in}{2.585871in}}%
\pgfpathlineto{\pgfqpoint{5.067800in}{2.369057in}}%
\pgfpathlineto{\pgfqpoint{5.068528in}{2.627525in}}%
\pgfpathlineto{\pgfqpoint{5.068336in}{2.237482in}}%
\pgfpathlineto{\pgfqpoint{5.068930in}{2.540356in}}%
\pgfpathlineto{\pgfqpoint{5.069293in}{2.279369in}}%
\pgfpathlineto{\pgfqpoint{5.069350in}{2.607358in}}%
\pgfpathlineto{\pgfqpoint{5.070038in}{2.514810in}}%
\pgfpathlineto{\pgfqpoint{5.070743in}{2.603706in}}%
\pgfpathlineto{\pgfqpoint{5.070324in}{2.141017in}}%
\pgfpathlineto{\pgfqpoint{5.071086in}{2.387879in}}%
\pgfpathlineto{\pgfqpoint{5.071105in}{2.220422in}}%
\pgfpathlineto{\pgfqpoint{5.072017in}{2.580473in}}%
\pgfpathlineto{\pgfqpoint{5.072188in}{2.343672in}}%
\pgfpathlineto{\pgfqpoint{5.072283in}{2.595278in}}%
\pgfpathlineto{\pgfqpoint{5.073155in}{2.263224in}}%
\pgfpathlineto{\pgfqpoint{5.073306in}{2.456105in}}%
\pgfpathlineto{\pgfqpoint{5.073401in}{2.567539in}}%
\pgfpathlineto{\pgfqpoint{5.074176in}{2.126555in}}%
\pgfpathlineto{\pgfqpoint{5.074384in}{2.489565in}}%
\pgfpathlineto{\pgfqpoint{5.074403in}{2.211249in}}%
\pgfpathlineto{\pgfqpoint{5.074856in}{2.610833in}}%
\pgfpathlineto{\pgfqpoint{5.075478in}{2.355248in}}%
\pgfpathlineto{\pgfqpoint{5.076287in}{2.585959in}}%
\pgfpathlineto{\pgfqpoint{5.075685in}{2.172536in}}%
\pgfpathlineto{\pgfqpoint{5.076588in}{2.561040in}}%
\pgfpathlineto{\pgfqpoint{5.077489in}{2.154748in}}%
\pgfpathlineto{\pgfqpoint{5.076625in}{2.580167in}}%
\pgfpathlineto{\pgfqpoint{5.077695in}{2.462749in}}%
\pgfpathlineto{\pgfqpoint{5.078013in}{2.589949in}}%
\pgfpathlineto{\pgfqpoint{5.078500in}{2.095265in}}%
\pgfpathlineto{\pgfqpoint{5.078818in}{2.519082in}}%
\pgfpathlineto{\pgfqpoint{5.079135in}{1.983245in}}%
\pgfpathlineto{\pgfqpoint{5.079154in}{2.611132in}}%
\pgfpathlineto{\pgfqpoint{5.079900in}{2.560419in}}%
\pgfpathlineto{\pgfqpoint{5.080868in}{2.628379in}}%
\pgfpathlineto{\pgfqpoint{5.080273in}{2.253232in}}%
\pgfpathlineto{\pgfqpoint{5.080887in}{2.524840in}}%
\pgfpathlineto{\pgfqpoint{5.081630in}{2.138350in}}%
\pgfpathlineto{\pgfqpoint{5.081593in}{2.597355in}}%
\pgfpathlineto{\pgfqpoint{5.082001in}{2.493115in}}%
\pgfpathlineto{\pgfqpoint{5.082335in}{2.616127in}}%
\pgfpathlineto{\pgfqpoint{5.082465in}{2.258599in}}%
\pgfpathlineto{\pgfqpoint{5.082576in}{2.413912in}}%
\pgfpathlineto{\pgfqpoint{5.082706in}{2.089326in}}%
\pgfpathlineto{\pgfqpoint{5.083667in}{2.560861in}}%
\pgfpathlineto{\pgfqpoint{5.084424in}{2.182496in}}%
\pgfpathlineto{\pgfqpoint{5.083944in}{2.582656in}}%
\pgfpathlineto{\pgfqpoint{5.084793in}{2.415695in}}%
\pgfpathlineto{\pgfqpoint{5.085621in}{2.567687in}}%
\pgfpathlineto{\pgfqpoint{5.085529in}{2.185984in}}%
\pgfpathlineto{\pgfqpoint{5.085879in}{2.539440in}}%
\pgfpathlineto{\pgfqpoint{5.086375in}{2.255003in}}%
\pgfpathlineto{\pgfqpoint{5.086136in}{2.641220in}}%
\pgfpathlineto{\pgfqpoint{5.086998in}{2.359533in}}%
\pgfpathlineto{\pgfqpoint{5.087182in}{2.604024in}}%
\pgfpathlineto{\pgfqpoint{5.087438in}{2.172269in}}%
\pgfpathlineto{\pgfqpoint{5.088115in}{2.484255in}}%
\pgfpathlineto{\pgfqpoint{5.089083in}{2.184368in}}%
\pgfpathlineto{\pgfqpoint{5.088225in}{2.614001in}}%
\pgfpathlineto{\pgfqpoint{5.089229in}{2.309043in}}%
\pgfpathlineto{\pgfqpoint{5.090176in}{2.581706in}}%
\pgfpathlineto{\pgfqpoint{5.089539in}{2.164829in}}%
\pgfpathlineto{\pgfqpoint{5.090340in}{2.514223in}}%
\pgfpathlineto{\pgfqpoint{5.090412in}{2.590357in}}%
\pgfpathlineto{\pgfqpoint{5.091484in}{2.134162in}}%
\pgfpathlineto{\pgfqpoint{5.091520in}{2.627142in}}%
\pgfpathlineto{\pgfqpoint{5.092625in}{2.568267in}}%
\pgfpathlineto{\pgfqpoint{5.093059in}{2.165099in}}%
\pgfpathlineto{\pgfqpoint{5.093474in}{2.602939in}}%
\pgfpathlineto{\pgfqpoint{5.093727in}{2.303111in}}%
\pgfpathlineto{\pgfqpoint{5.093745in}{2.597864in}}%
\pgfpathlineto{\pgfqpoint{5.094628in}{2.116268in}}%
\pgfpathlineto{\pgfqpoint{5.094844in}{2.413270in}}%
\pgfpathlineto{\pgfqpoint{5.095581in}{2.599887in}}%
\pgfpathlineto{\pgfqpoint{5.095150in}{2.204592in}}%
\pgfpathlineto{\pgfqpoint{5.095959in}{2.560498in}}%
\pgfpathlineto{\pgfqpoint{5.096945in}{2.165982in}}%
\pgfpathlineto{\pgfqpoint{5.096855in}{2.578749in}}%
\pgfpathlineto{\pgfqpoint{5.097070in}{2.380280in}}%
\pgfpathlineto{\pgfqpoint{5.097338in}{2.550881in}}%
\pgfpathlineto{\pgfqpoint{5.098089in}{2.077721in}}%
\pgfpathlineto{\pgfqpoint{5.098196in}{2.485050in}}%
\pgfpathlineto{\pgfqpoint{5.099159in}{2.599512in}}%
\pgfpathlineto{\pgfqpoint{5.098981in}{2.101155in}}%
\pgfpathlineto{\pgfqpoint{5.099284in}{2.515015in}}%
\pgfpathlineto{\pgfqpoint{5.099426in}{2.189440in}}%
\pgfpathlineto{\pgfqpoint{5.099924in}{2.572523in}}%
\pgfpathlineto{\pgfqpoint{5.100386in}{2.446915in}}%
\pgfpathlineto{\pgfqpoint{5.100688in}{2.614408in}}%
\pgfpathlineto{\pgfqpoint{5.101167in}{2.148455in}}%
\pgfpathlineto{\pgfqpoint{5.101486in}{2.527201in}}%
\pgfpathlineto{\pgfqpoint{5.101557in}{2.246387in}}%
\pgfpathlineto{\pgfqpoint{5.101770in}{2.586684in}}%
\pgfpathlineto{\pgfqpoint{5.102601in}{2.427988in}}%
\pgfpathlineto{\pgfqpoint{5.103589in}{2.589910in}}%
\pgfpathlineto{\pgfqpoint{5.103325in}{2.179763in}}%
\pgfpathlineto{\pgfqpoint{5.103624in}{2.540546in}}%
\pgfpathlineto{\pgfqpoint{5.103642in}{2.076406in}}%
\pgfpathlineto{\pgfqpoint{5.103889in}{2.592552in}}%
\pgfpathlineto{\pgfqpoint{5.104733in}{2.325899in}}%
\pgfpathlineto{\pgfqpoint{5.104909in}{2.597383in}}%
\pgfpathlineto{\pgfqpoint{5.105717in}{2.181435in}}%
\pgfpathlineto{\pgfqpoint{5.105840in}{2.360089in}}%
\pgfpathlineto{\pgfqpoint{5.106032in}{2.276135in}}%
\pgfpathlineto{\pgfqpoint{5.105927in}{2.552169in}}%
\pgfpathlineto{\pgfqpoint{5.106260in}{2.498985in}}%
\pgfpathlineto{\pgfqpoint{5.106365in}{2.542531in}}%
\pgfpathlineto{\pgfqpoint{5.106488in}{2.067485in}}%
\pgfpathlineto{\pgfqpoint{5.107240in}{2.324499in}}%
\pgfpathlineto{\pgfqpoint{5.107258in}{2.080814in}}%
\pgfpathlineto{\pgfqpoint{5.107537in}{2.549689in}}%
\pgfpathlineto{\pgfqpoint{5.108340in}{2.453475in}}%
\pgfpathlineto{\pgfqpoint{5.108810in}{2.261524in}}%
\pgfpathlineto{\pgfqpoint{5.108706in}{2.586295in}}%
\pgfpathlineto{\pgfqpoint{5.109297in}{2.400040in}}%
\pgfpathlineto{\pgfqpoint{5.110270in}{2.591572in}}%
\pgfpathlineto{\pgfqpoint{5.109523in}{2.113559in}}%
\pgfpathlineto{\pgfqpoint{5.110409in}{2.413986in}}%
\pgfpathlineto{\pgfqpoint{5.111362in}{2.130179in}}%
\pgfpathlineto{\pgfqpoint{5.110583in}{2.594170in}}%
\pgfpathlineto{\pgfqpoint{5.111501in}{2.503961in}}%
\pgfpathlineto{\pgfqpoint{5.112296in}{1.971181in}}%
\pgfpathlineto{\pgfqpoint{5.112417in}{2.572640in}}%
\pgfpathlineto{\pgfqpoint{5.112572in}{2.382450in}}%
\pgfpathlineto{\pgfqpoint{5.113089in}{2.615326in}}%
\pgfpathlineto{\pgfqpoint{5.113210in}{2.161524in}}%
\pgfpathlineto{\pgfqpoint{5.113692in}{2.505005in}}%
\pgfpathlineto{\pgfqpoint{5.113710in}{2.504820in}}%
\pgfpathlineto{\pgfqpoint{5.113727in}{2.507426in}}%
\pgfpathlineto{\pgfqpoint{5.114088in}{2.135980in}}%
\pgfpathlineto{\pgfqpoint{5.113778in}{2.618802in}}%
\pgfpathlineto{\pgfqpoint{5.114827in}{2.498912in}}%
\pgfpathlineto{\pgfqpoint{5.114947in}{2.170180in}}%
\pgfpathlineto{\pgfqpoint{5.115496in}{2.598862in}}%
\pgfpathlineto{\pgfqpoint{5.115941in}{2.260287in}}%
\pgfpathlineto{\pgfqpoint{5.115976in}{2.606711in}}%
\pgfpathlineto{\pgfqpoint{5.116027in}{2.048976in}}%
\pgfpathlineto{\pgfqpoint{5.117053in}{2.488486in}}%
\pgfpathlineto{\pgfqpoint{5.117070in}{2.159645in}}%
\pgfpathlineto{\pgfqpoint{5.117974in}{2.591973in}}%
\pgfpathlineto{\pgfqpoint{5.118161in}{2.499302in}}%
\pgfpathlineto{\pgfqpoint{5.118195in}{2.466304in}}%
\pgfpathlineto{\pgfqpoint{5.118212in}{2.512548in}}%
\pgfpathlineto{\pgfqpoint{5.119182in}{2.137653in}}%
\pgfpathlineto{\pgfqpoint{5.118349in}{2.557483in}}%
\pgfpathlineto{\pgfqpoint{5.119318in}{2.310048in}}%
\pgfpathlineto{\pgfqpoint{5.119539in}{2.614601in}}%
\pgfpathlineto{\pgfqpoint{5.119454in}{2.168765in}}%
\pgfpathlineto{\pgfqpoint{5.120438in}{2.428595in}}%
\pgfpathlineto{\pgfqpoint{5.120489in}{2.251233in}}%
\pgfpathlineto{\pgfqpoint{5.121419in}{2.593566in}}%
\pgfpathlineto{\pgfqpoint{5.121842in}{2.168904in}}%
\pgfpathlineto{\pgfqpoint{5.122533in}{2.478016in}}%
\pgfpathlineto{\pgfqpoint{5.123476in}{2.112044in}}%
\pgfpathlineto{\pgfqpoint{5.123224in}{2.588718in}}%
\pgfpathlineto{\pgfqpoint{5.123628in}{2.459343in}}%
\pgfpathlineto{\pgfqpoint{5.123678in}{2.223798in}}%
\pgfpathlineto{\pgfqpoint{5.124736in}{2.598982in}}%
\pgfpathlineto{\pgfqpoint{5.125155in}{2.214328in}}%
\pgfpathlineto{\pgfqpoint{5.125289in}{2.622152in}}%
\pgfpathlineto{\pgfqpoint{5.125858in}{2.403572in}}%
\pgfpathlineto{\pgfqpoint{5.126310in}{2.593455in}}%
\pgfpathlineto{\pgfqpoint{5.126326in}{2.129828in}}%
\pgfpathlineto{\pgfqpoint{5.126994in}{2.533876in}}%
\pgfpathlineto{\pgfqpoint{5.128011in}{2.160100in}}%
\pgfpathlineto{\pgfqpoint{5.127794in}{2.605362in}}%
\pgfpathlineto{\pgfqpoint{5.128110in}{2.402282in}}%
\pgfpathlineto{\pgfqpoint{5.128742in}{2.581173in}}%
\pgfpathlineto{\pgfqpoint{5.128792in}{2.127448in}}%
\pgfpathlineto{\pgfqpoint{5.129224in}{2.509089in}}%
\pgfpathlineto{\pgfqpoint{5.130135in}{2.179532in}}%
\pgfpathlineto{\pgfqpoint{5.129804in}{2.555108in}}%
\pgfpathlineto{\pgfqpoint{5.130318in}{2.487382in}}%
\pgfpathlineto{\pgfqpoint{5.130830in}{2.590647in}}%
\pgfpathlineto{\pgfqpoint{5.131029in}{2.157941in}}%
\pgfpathlineto{\pgfqpoint{5.131392in}{2.505658in}}%
\pgfpathlineto{\pgfqpoint{5.132382in}{2.131322in}}%
\pgfpathlineto{\pgfqpoint{5.132316in}{2.572649in}}%
\pgfpathlineto{\pgfqpoint{5.132497in}{2.327683in}}%
\pgfpathlineto{\pgfqpoint{5.133352in}{2.565017in}}%
\pgfpathlineto{\pgfqpoint{5.132810in}{2.160246in}}%
\pgfpathlineto{\pgfqpoint{5.133615in}{2.499150in}}%
\pgfpathlineto{\pgfqpoint{5.134190in}{2.191865in}}%
\pgfpathlineto{\pgfqpoint{5.133960in}{2.576446in}}%
\pgfpathlineto{\pgfqpoint{5.134714in}{2.471437in}}%
\pgfpathlineto{\pgfqpoint{5.134911in}{2.583084in}}%
\pgfpathlineto{\pgfqpoint{5.135598in}{2.138167in}}%
\pgfpathlineto{\pgfqpoint{5.135794in}{2.497560in}}%
\pgfpathlineto{\pgfqpoint{5.136316in}{2.158733in}}%
\pgfpathlineto{\pgfqpoint{5.136643in}{2.620098in}}%
\pgfpathlineto{\pgfqpoint{5.136904in}{2.489496in}}%
\pgfpathlineto{\pgfqpoint{5.137164in}{2.201527in}}%
\pgfpathlineto{\pgfqpoint{5.137295in}{2.599630in}}%
\pgfpathlineto{\pgfqpoint{5.137994in}{2.392549in}}%
\pgfpathlineto{\pgfqpoint{5.138514in}{2.598814in}}%
\pgfpathlineto{\pgfqpoint{5.138855in}{2.246661in}}%
\pgfpathlineto{\pgfqpoint{5.139098in}{2.522933in}}%
\pgfpathlineto{\pgfqpoint{5.139357in}{2.166682in}}%
\pgfpathlineto{\pgfqpoint{5.139600in}{2.560817in}}%
\pgfpathlineto{\pgfqpoint{5.140199in}{2.496908in}}%
\pgfpathlineto{\pgfqpoint{5.140603in}{2.574862in}}%
\pgfpathlineto{\pgfqpoint{5.140797in}{1.857917in}}%
\pgfpathlineto{\pgfqpoint{5.141281in}{2.514631in}}%
\pgfpathlineto{\pgfqpoint{5.142135in}{2.148920in}}%
\pgfpathlineto{\pgfqpoint{5.142087in}{2.579483in}}%
\pgfpathlineto{\pgfqpoint{5.142393in}{2.442922in}}%
\pgfpathlineto{\pgfqpoint{5.142537in}{2.571747in}}%
\pgfpathlineto{\pgfqpoint{5.142425in}{2.320668in}}%
\pgfpathlineto{\pgfqpoint{5.142618in}{2.456276in}}%
\pgfpathlineto{\pgfqpoint{5.143646in}{2.135037in}}%
\pgfpathlineto{\pgfqpoint{5.143389in}{2.592294in}}%
\pgfpathlineto{\pgfqpoint{5.143742in}{2.168994in}}%
\pgfpathlineto{\pgfqpoint{5.143934in}{2.611995in}}%
\pgfpathlineto{\pgfqpoint{5.144063in}{2.056587in}}%
\pgfpathlineto{\pgfqpoint{5.144879in}{2.550647in}}%
\pgfpathlineto{\pgfqpoint{5.145630in}{2.072120in}}%
\pgfpathlineto{\pgfqpoint{5.145454in}{2.561028in}}%
\pgfpathlineto{\pgfqpoint{5.145965in}{2.446380in}}%
\pgfpathlineto{\pgfqpoint{5.147017in}{2.608866in}}%
\pgfpathlineto{\pgfqpoint{5.146093in}{2.124048in}}%
\pgfpathlineto{\pgfqpoint{5.147049in}{2.488766in}}%
\pgfpathlineto{\pgfqpoint{5.147526in}{2.005377in}}%
\pgfpathlineto{\pgfqpoint{5.147128in}{2.548410in}}%
\pgfpathlineto{\pgfqpoint{5.148161in}{2.396734in}}%
\pgfpathlineto{\pgfqpoint{5.148383in}{2.576869in}}%
\pgfpathlineto{\pgfqpoint{5.148193in}{2.105798in}}%
\pgfpathlineto{\pgfqpoint{5.149239in}{2.330224in}}%
\pgfpathlineto{\pgfqpoint{5.149255in}{2.072303in}}%
\pgfpathlineto{\pgfqpoint{5.149634in}{2.586677in}}%
\pgfpathlineto{\pgfqpoint{5.150330in}{2.228829in}}%
\pgfpathlineto{\pgfqpoint{5.151402in}{2.583560in}}%
\pgfpathlineto{\pgfqpoint{5.151260in}{2.167370in}}%
\pgfpathlineto{\pgfqpoint{5.151449in}{2.523320in}}%
\pgfpathlineto{\pgfqpoint{5.151890in}{2.566872in}}%
\pgfpathlineto{\pgfqpoint{5.151622in}{2.024553in}}%
\pgfpathlineto{\pgfqpoint{5.152409in}{2.381443in}}%
\pgfpathlineto{\pgfqpoint{5.153460in}{1.996162in}}%
\pgfpathlineto{\pgfqpoint{5.152896in}{2.587728in}}%
\pgfpathlineto{\pgfqpoint{5.153507in}{2.259359in}}%
\pgfpathlineto{\pgfqpoint{5.154572in}{2.566499in}}%
\pgfpathlineto{\pgfqpoint{5.153774in}{2.107966in}}%
\pgfpathlineto{\pgfqpoint{5.154634in}{2.513133in}}%
\pgfpathlineto{\pgfqpoint{5.155696in}{2.101524in}}%
\pgfpathlineto{\pgfqpoint{5.155150in}{2.565494in}}%
\pgfpathlineto{\pgfqpoint{5.155742in}{2.484885in}}%
\pgfpathlineto{\pgfqpoint{5.156817in}{2.205293in}}%
\pgfpathlineto{\pgfqpoint{5.156475in}{2.567534in}}%
\pgfpathlineto{\pgfqpoint{5.156848in}{2.461636in}}%
\pgfpathlineto{\pgfqpoint{5.157547in}{2.584177in}}%
\pgfpathlineto{\pgfqpoint{5.157174in}{2.209704in}}%
\pgfpathlineto{\pgfqpoint{5.157950in}{2.439087in}}%
\pgfpathlineto{\pgfqpoint{5.158880in}{2.067304in}}%
\pgfpathlineto{\pgfqpoint{5.158152in}{2.610135in}}%
\pgfpathlineto{\pgfqpoint{5.159050in}{2.465405in}}%
\pgfpathlineto{\pgfqpoint{5.159096in}{1.880531in}}%
\pgfpathlineto{\pgfqpoint{5.159931in}{2.590577in}}%
\pgfpathlineto{\pgfqpoint{5.160147in}{2.379827in}}%
\pgfpathlineto{\pgfqpoint{5.160764in}{2.573694in}}%
\pgfpathlineto{\pgfqpoint{5.161195in}{2.204686in}}%
\pgfpathlineto{\pgfqpoint{5.161256in}{2.522774in}}%
\pgfpathlineto{\pgfqpoint{5.161902in}{2.037680in}}%
\pgfpathlineto{\pgfqpoint{5.162225in}{2.545841in}}%
\pgfpathlineto{\pgfqpoint{5.162363in}{2.357423in}}%
\pgfpathlineto{\pgfqpoint{5.162424in}{2.549891in}}%
\pgfpathlineto{\pgfqpoint{5.162731in}{2.195939in}}%
\pgfpathlineto{\pgfqpoint{5.163466in}{2.465101in}}%
\pgfpathlineto{\pgfqpoint{5.164308in}{2.006617in}}%
\pgfpathlineto{\pgfqpoint{5.163604in}{2.592683in}}%
\pgfpathlineto{\pgfqpoint{5.164567in}{2.249719in}}%
\pgfpathlineto{\pgfqpoint{5.164628in}{2.566029in}}%
\pgfpathlineto{\pgfqpoint{5.164781in}{2.183448in}}%
\pgfpathlineto{\pgfqpoint{5.165680in}{2.446358in}}%
\pgfpathlineto{\pgfqpoint{5.166183in}{2.565158in}}%
\pgfpathlineto{\pgfqpoint{5.166244in}{2.018351in}}%
\pgfpathlineto{\pgfqpoint{5.166791in}{2.546933in}}%
\pgfpathlineto{\pgfqpoint{5.167519in}{2.218594in}}%
\pgfpathlineto{\pgfqpoint{5.167276in}{2.604424in}}%
\pgfpathlineto{\pgfqpoint{5.167898in}{2.409285in}}%
\pgfpathlineto{\pgfqpoint{5.168821in}{2.571612in}}%
\pgfpathlineto{\pgfqpoint{5.168095in}{2.106066in}}%
\pgfpathlineto{\pgfqpoint{5.169002in}{2.465068in}}%
\pgfpathlineto{\pgfqpoint{5.169833in}{1.992667in}}%
\pgfpathlineto{\pgfqpoint{5.169606in}{2.563757in}}%
\pgfpathlineto{\pgfqpoint{5.170089in}{2.486121in}}%
\pgfpathlineto{\pgfqpoint{5.170420in}{2.569272in}}%
\pgfpathlineto{\pgfqpoint{5.170737in}{2.237291in}}%
\pgfpathlineto{\pgfqpoint{5.171143in}{2.509827in}}%
\pgfpathlineto{\pgfqpoint{5.171774in}{2.210165in}}%
\pgfpathlineto{\pgfqpoint{5.171308in}{2.582172in}}%
\pgfpathlineto{\pgfqpoint{5.172269in}{2.362879in}}%
\pgfpathlineto{\pgfqpoint{5.172913in}{2.577995in}}%
\pgfpathlineto{\pgfqpoint{5.173242in}{2.243603in}}%
\pgfpathlineto{\pgfqpoint{5.173377in}{2.400914in}}%
\pgfpathlineto{\pgfqpoint{5.173691in}{2.582457in}}%
\pgfpathlineto{\pgfqpoint{5.174064in}{2.201881in}}%
\pgfpathlineto{\pgfqpoint{5.174512in}{2.509413in}}%
\pgfpathlineto{\pgfqpoint{5.174542in}{2.529773in}}%
\pgfpathlineto{\pgfqpoint{5.174601in}{2.364726in}}%
\pgfpathlineto{\pgfqpoint{5.175183in}{1.845812in}}%
\pgfpathlineto{\pgfqpoint{5.175495in}{2.597379in}}%
\pgfpathlineto{\pgfqpoint{5.175689in}{2.232503in}}%
\pgfpathlineto{\pgfqpoint{5.175971in}{2.600686in}}%
\pgfpathlineto{\pgfqpoint{5.176149in}{2.182906in}}%
\pgfpathlineto{\pgfqpoint{5.176803in}{2.486393in}}%
\pgfpathlineto{\pgfqpoint{5.177040in}{2.100532in}}%
\pgfpathlineto{\pgfqpoint{5.177455in}{2.590442in}}%
\pgfpathlineto{\pgfqpoint{5.177884in}{2.200524in}}%
\pgfpathlineto{\pgfqpoint{5.178564in}{2.565184in}}%
\pgfpathlineto{\pgfqpoint{5.178579in}{1.999934in}}%
\pgfpathlineto{\pgfqpoint{5.179007in}{2.476701in}}%
\pgfpathlineto{\pgfqpoint{5.179376in}{2.123274in}}%
\pgfpathlineto{\pgfqpoint{5.179111in}{2.586422in}}%
\pgfpathlineto{\pgfqpoint{5.180113in}{2.452302in}}%
\pgfpathlineto{\pgfqpoint{5.180392in}{2.228403in}}%
\pgfpathlineto{\pgfqpoint{5.180142in}{2.584984in}}%
\pgfpathlineto{\pgfqpoint{5.181215in}{2.447712in}}%
\pgfpathlineto{\pgfqpoint{5.181435in}{2.603552in}}%
\pgfpathlineto{\pgfqpoint{5.181743in}{2.191853in}}%
\pgfpathlineto{\pgfqpoint{5.182315in}{2.436860in}}%
\pgfpathlineto{\pgfqpoint{5.183105in}{2.023313in}}%
\pgfpathlineto{\pgfqpoint{5.183222in}{2.570677in}}%
\pgfpathlineto{\pgfqpoint{5.183426in}{2.433352in}}%
\pgfpathlineto{\pgfqpoint{5.184462in}{2.573426in}}%
\pgfpathlineto{\pgfqpoint{5.183602in}{2.236864in}}%
\pgfpathlineto{\pgfqpoint{5.184520in}{2.549709in}}%
\pgfpathlineto{\pgfqpoint{5.185568in}{1.974559in}}%
\pgfpathlineto{\pgfqpoint{5.185379in}{2.591075in}}%
\pgfpathlineto{\pgfqpoint{5.185626in}{2.511133in}}%
\pgfpathlineto{\pgfqpoint{5.186294in}{2.057290in}}%
\pgfpathlineto{\pgfqpoint{5.185815in}{2.557985in}}%
\pgfpathlineto{\pgfqpoint{5.186743in}{2.421101in}}%
\pgfpathlineto{\pgfqpoint{5.187265in}{2.564576in}}%
\pgfpathlineto{\pgfqpoint{5.187279in}{2.170574in}}%
\pgfpathlineto{\pgfqpoint{5.187843in}{2.348885in}}%
\pgfpathlineto{\pgfqpoint{5.188782in}{2.249204in}}%
\pgfpathlineto{\pgfqpoint{5.188204in}{2.585995in}}%
\pgfpathlineto{\pgfqpoint{5.188911in}{2.437773in}}%
\pgfpathlineto{\pgfqpoint{5.189315in}{2.583668in}}%
\pgfpathlineto{\pgfqpoint{5.189761in}{2.093023in}}%
\pgfpathlineto{\pgfqpoint{5.190006in}{2.426782in}}%
\pgfpathlineto{\pgfqpoint{5.190494in}{2.104941in}}%
\pgfpathlineto{\pgfqpoint{5.191068in}{2.573895in}}%
\pgfpathlineto{\pgfqpoint{5.191112in}{2.388183in}}%
\pgfpathlineto{\pgfqpoint{5.191241in}{2.558504in}}%
\pgfpathlineto{\pgfqpoint{5.191140in}{2.257207in}}%
\pgfpathlineto{\pgfqpoint{5.192229in}{2.508551in}}%
\pgfpathlineto{\pgfqpoint{5.193200in}{2.113226in}}%
\pgfpathlineto{\pgfqpoint{5.192629in}{2.566306in}}%
\pgfpathlineto{\pgfqpoint{5.193343in}{2.417042in}}%
\pgfpathlineto{\pgfqpoint{5.193828in}{2.133065in}}%
\pgfpathlineto{\pgfqpoint{5.193999in}{2.581644in}}%
\pgfpathlineto{\pgfqpoint{5.194383in}{2.444519in}}%
\pgfpathlineto{\pgfqpoint{5.195364in}{2.576609in}}%
\pgfpathlineto{\pgfqpoint{5.194483in}{2.124984in}}%
\pgfpathlineto{\pgfqpoint{5.195478in}{2.388978in}}%
\pgfpathlineto{\pgfqpoint{5.196385in}{2.101169in}}%
\pgfpathlineto{\pgfqpoint{5.196442in}{2.552283in}}%
\pgfpathlineto{\pgfqpoint{5.196555in}{2.442291in}}%
\pgfpathlineto{\pgfqpoint{5.197206in}{2.615455in}}%
\pgfpathlineto{\pgfqpoint{5.197249in}{2.052215in}}%
\pgfpathlineto{\pgfqpoint{5.197658in}{2.486436in}}%
\pgfpathlineto{\pgfqpoint{5.197941in}{2.069551in}}%
\pgfpathlineto{\pgfqpoint{5.198660in}{2.588507in}}%
\pgfpathlineto{\pgfqpoint{5.198745in}{2.435224in}}%
\pgfpathlineto{\pgfqpoint{5.199575in}{2.564412in}}%
\pgfpathlineto{\pgfqpoint{5.199223in}{2.101525in}}%
\pgfpathlineto{\pgfqpoint{5.199842in}{2.430517in}}%
\pgfpathlineto{\pgfqpoint{5.200137in}{2.590691in}}%
\pgfpathlineto{\pgfqpoint{5.200965in}{1.953098in}}%
\pgfpathlineto{\pgfqpoint{5.202000in}{2.563887in}}%
\pgfpathlineto{\pgfqpoint{5.202084in}{2.325524in}}%
\pgfpathlineto{\pgfqpoint{5.202838in}{2.584610in}}%
\pgfpathlineto{\pgfqpoint{5.202922in}{2.071582in}}%
\pgfpathlineto{\pgfqpoint{5.203173in}{2.423177in}}%
\pgfpathlineto{\pgfqpoint{5.204217in}{2.179042in}}%
\pgfpathlineto{\pgfqpoint{5.204078in}{2.567946in}}%
\pgfpathlineto{\pgfqpoint{5.204273in}{2.409633in}}%
\pgfpathlineto{\pgfqpoint{5.205092in}{2.579319in}}%
\pgfpathlineto{\pgfqpoint{5.204509in}{2.189435in}}%
\pgfpathlineto{\pgfqpoint{5.205314in}{2.476730in}}%
\pgfpathlineto{\pgfqpoint{5.205827in}{2.131131in}}%
\pgfpathlineto{\pgfqpoint{5.206077in}{2.585410in}}%
\pgfpathlineto{\pgfqpoint{5.206423in}{2.273895in}}%
\pgfpathlineto{\pgfqpoint{5.207280in}{2.563630in}}%
\pgfpathlineto{\pgfqpoint{5.206893in}{1.875744in}}%
\pgfpathlineto{\pgfqpoint{5.207542in}{2.413788in}}%
\pgfpathlineto{\pgfqpoint{5.208272in}{2.557406in}}%
\pgfpathlineto{\pgfqpoint{5.208658in}{2.154878in}}%
\pgfpathlineto{\pgfqpoint{5.209126in}{2.611522in}}%
\pgfpathlineto{\pgfqpoint{5.209648in}{2.145804in}}%
\pgfpathlineto{\pgfqpoint{5.209771in}{2.474423in}}%
\pgfpathlineto{\pgfqpoint{5.210320in}{2.097618in}}%
\pgfpathlineto{\pgfqpoint{5.210690in}{2.564524in}}%
\pgfpathlineto{\pgfqpoint{5.210868in}{2.387419in}}%
\pgfpathlineto{\pgfqpoint{5.211196in}{2.576191in}}%
\pgfpathlineto{\pgfqpoint{5.211101in}{2.187160in}}%
\pgfpathlineto{\pgfqpoint{5.211975in}{2.444579in}}%
\pgfpathlineto{\pgfqpoint{5.212467in}{2.158832in}}%
\pgfpathlineto{\pgfqpoint{5.212153in}{2.561163in}}%
\pgfpathlineto{\pgfqpoint{5.213094in}{2.291661in}}%
\pgfpathlineto{\pgfqpoint{5.213679in}{2.579260in}}%
\pgfpathlineto{\pgfqpoint{5.213134in}{2.138816in}}%
\pgfpathlineto{\pgfqpoint{5.214209in}{2.426062in}}%
\pgfpathlineto{\pgfqpoint{5.214413in}{1.844481in}}%
\pgfpathlineto{\pgfqpoint{5.214616in}{2.568030in}}%
\pgfpathlineto{\pgfqpoint{5.215321in}{2.253193in}}%
\pgfpathlineto{\pgfqpoint{5.215700in}{2.554394in}}%
\pgfpathlineto{\pgfqpoint{5.215768in}{1.977981in}}%
\pgfpathlineto{\pgfqpoint{5.216444in}{2.444694in}}%
\pgfpathlineto{\pgfqpoint{5.217348in}{2.183034in}}%
\pgfpathlineto{\pgfqpoint{5.217065in}{2.520206in}}%
\pgfpathlineto{\pgfqpoint{5.217551in}{2.291397in}}%
\pgfpathlineto{\pgfqpoint{5.218654in}{2.554888in}}%
\pgfpathlineto{\pgfqpoint{5.217807in}{2.093778in}}%
\pgfpathlineto{\pgfqpoint{5.218668in}{2.467740in}}%
\pgfpathlineto{\pgfqpoint{5.218936in}{2.003703in}}%
\pgfpathlineto{\pgfqpoint{5.219661in}{2.572468in}}%
\pgfpathlineto{\pgfqpoint{5.219795in}{2.301355in}}%
\pgfpathlineto{\pgfqpoint{5.220452in}{2.569892in}}%
\pgfpathlineto{\pgfqpoint{5.220264in}{1.991137in}}%
\pgfpathlineto{\pgfqpoint{5.220893in}{2.509064in}}%
\pgfpathlineto{\pgfqpoint{5.221961in}{2.076153in}}%
\pgfpathlineto{\pgfqpoint{5.221174in}{2.611123in}}%
\pgfpathlineto{\pgfqpoint{5.222001in}{2.468602in}}%
\pgfpathlineto{\pgfqpoint{5.222135in}{2.574802in}}%
\pgfpathlineto{\pgfqpoint{5.222654in}{2.128753in}}%
\pgfpathlineto{\pgfqpoint{5.223107in}{2.451691in}}%
\pgfpathlineto{\pgfqpoint{5.223891in}{1.608823in}}%
\pgfpathlineto{\pgfqpoint{5.223718in}{2.571211in}}%
\pgfpathlineto{\pgfqpoint{5.224223in}{2.414416in}}%
\pgfpathlineto{\pgfqpoint{5.225322in}{2.583594in}}%
\pgfpathlineto{\pgfqpoint{5.224634in}{2.184994in}}%
\pgfpathlineto{\pgfqpoint{5.225336in}{2.490728in}}%
\pgfpathlineto{\pgfqpoint{5.226314in}{2.145883in}}%
\pgfpathlineto{\pgfqpoint{5.226327in}{2.584697in}}%
\pgfpathlineto{\pgfqpoint{5.226432in}{2.417045in}}%
\pgfpathlineto{\pgfqpoint{5.226789in}{2.573625in}}%
\pgfpathlineto{\pgfqpoint{5.227197in}{2.129825in}}%
\pgfpathlineto{\pgfqpoint{5.227540in}{2.494588in}}%
\pgfpathlineto{\pgfqpoint{5.228053in}{2.006707in}}%
\pgfpathlineto{\pgfqpoint{5.227908in}{2.586279in}}%
\pgfpathlineto{\pgfqpoint{5.228644in}{2.501885in}}%
\pgfpathlineto{\pgfqpoint{5.228828in}{2.564259in}}%
\pgfpathlineto{\pgfqpoint{5.228959in}{2.048808in}}%
\pgfpathlineto{\pgfqpoint{5.229418in}{2.517393in}}%
\pgfpathlineto{\pgfqpoint{5.229431in}{2.081724in}}%
\pgfpathlineto{\pgfqpoint{5.230295in}{2.560889in}}%
\pgfpathlineto{\pgfqpoint{5.230530in}{2.468379in}}%
\pgfpathlineto{\pgfqpoint{5.230844in}{1.951477in}}%
\pgfpathlineto{\pgfqpoint{5.231575in}{2.575041in}}%
\pgfpathlineto{\pgfqpoint{5.231640in}{2.303695in}}%
\pgfpathlineto{\pgfqpoint{5.231939in}{2.586761in}}%
\pgfpathlineto{\pgfqpoint{5.232135in}{2.192430in}}%
\pgfpathlineto{\pgfqpoint{5.232746in}{2.446346in}}%
\pgfpathlineto{\pgfqpoint{5.233084in}{2.070278in}}%
\pgfpathlineto{\pgfqpoint{5.233136in}{2.578553in}}%
\pgfpathlineto{\pgfqpoint{5.233850in}{2.260494in}}%
\pgfpathlineto{\pgfqpoint{5.234951in}{2.578752in}}%
\pgfpathlineto{\pgfqpoint{5.233915in}{2.094881in}}%
\pgfpathlineto{\pgfqpoint{5.234964in}{2.442846in}}%
\pgfpathlineto{\pgfqpoint{5.235597in}{2.055619in}}%
\pgfpathlineto{\pgfqpoint{5.235145in}{2.588058in}}%
\pgfpathlineto{\pgfqpoint{5.236062in}{2.489548in}}%
\pgfpathlineto{\pgfqpoint{5.236810in}{1.981710in}}%
\pgfpathlineto{\pgfqpoint{5.236977in}{2.586884in}}%
\pgfpathlineto{\pgfqpoint{5.237170in}{2.374365in}}%
\pgfpathlineto{\pgfqpoint{5.237967in}{2.569391in}}%
\pgfpathlineto{\pgfqpoint{5.237762in}{1.944213in}}%
\pgfpathlineto{\pgfqpoint{5.238288in}{2.506280in}}%
\pgfpathlineto{\pgfqpoint{5.238468in}{2.164642in}}%
\pgfpathlineto{\pgfqpoint{5.239045in}{2.573454in}}%
\pgfpathlineto{\pgfqpoint{5.239442in}{2.355003in}}%
\pgfpathlineto{\pgfqpoint{5.240043in}{2.562248in}}%
\pgfpathlineto{\pgfqpoint{5.239544in}{2.096546in}}%
\pgfpathlineto{\pgfqpoint{5.240554in}{2.479946in}}%
\pgfpathlineto{\pgfqpoint{5.241472in}{2.114682in}}%
\pgfpathlineto{\pgfqpoint{5.241383in}{2.586457in}}%
\pgfpathlineto{\pgfqpoint{5.241663in}{2.442507in}}%
\pgfpathlineto{\pgfqpoint{5.242248in}{2.184716in}}%
\pgfpathlineto{\pgfqpoint{5.242515in}{2.573411in}}%
\pgfpathlineto{\pgfqpoint{5.242769in}{2.413412in}}%
\pgfpathlineto{\pgfqpoint{5.242947in}{2.557164in}}%
\pgfpathlineto{\pgfqpoint{5.243175in}{2.120960in}}%
\pgfpathlineto{\pgfqpoint{5.243885in}{2.449005in}}%
\pgfpathlineto{\pgfqpoint{5.244429in}{2.117467in}}%
\pgfpathlineto{\pgfqpoint{5.244834in}{2.604644in}}%
\pgfpathlineto{\pgfqpoint{5.244960in}{2.464983in}}%
\pgfpathlineto{\pgfqpoint{5.245617in}{2.556355in}}%
\pgfpathlineto{\pgfqpoint{5.245200in}{2.192997in}}%
\pgfpathlineto{\pgfqpoint{5.246058in}{2.418543in}}%
\pgfpathlineto{\pgfqpoint{5.246763in}{2.159526in}}%
\pgfpathlineto{\pgfqpoint{5.246989in}{2.589718in}}%
\pgfpathlineto{\pgfqpoint{5.247153in}{2.386900in}}%
\pgfpathlineto{\pgfqpoint{5.247429in}{2.559919in}}%
\pgfpathlineto{\pgfqpoint{5.247404in}{2.035698in}}%
\pgfpathlineto{\pgfqpoint{5.248257in}{2.499111in}}%
\pgfpathlineto{\pgfqpoint{5.248496in}{2.166500in}}%
\pgfpathlineto{\pgfqpoint{5.248821in}{2.599433in}}%
\pgfpathlineto{\pgfqpoint{5.249359in}{2.425662in}}%
\pgfpathlineto{\pgfqpoint{5.249872in}{2.592588in}}%
\pgfpathlineto{\pgfqpoint{5.249984in}{2.161920in}}%
\pgfpathlineto{\pgfqpoint{5.250446in}{2.461282in}}%
\pgfpathlineto{\pgfqpoint{5.251417in}{2.134137in}}%
\pgfpathlineto{\pgfqpoint{5.250832in}{2.560768in}}%
\pgfpathlineto{\pgfqpoint{5.251554in}{2.399321in}}%
\pgfpathlineto{\pgfqpoint{5.252449in}{2.571052in}}%
\pgfpathlineto{\pgfqpoint{5.251728in}{2.210282in}}%
\pgfpathlineto{\pgfqpoint{5.252498in}{2.415750in}}%
\pgfpathlineto{\pgfqpoint{5.252685in}{2.129015in}}%
\pgfpathlineto{\pgfqpoint{5.253280in}{2.585153in}}%
\pgfpathlineto{\pgfqpoint{5.253602in}{2.425459in}}%
\pgfpathlineto{\pgfqpoint{5.254381in}{2.568422in}}%
\pgfpathlineto{\pgfqpoint{5.254455in}{2.184595in}}%
\pgfpathlineto{\pgfqpoint{5.254652in}{2.455271in}}%
\pgfpathlineto{\pgfqpoint{5.254912in}{2.047062in}}%
\pgfpathlineto{\pgfqpoint{5.255651in}{2.565795in}}%
\pgfpathlineto{\pgfqpoint{5.255762in}{2.374978in}}%
\pgfpathlineto{\pgfqpoint{5.255787in}{2.444694in}}%
\pgfpathlineto{\pgfqpoint{5.256660in}{2.103186in}}%
\pgfpathlineto{\pgfqpoint{5.256009in}{2.572837in}}%
\pgfpathlineto{\pgfqpoint{5.256894in}{2.445481in}}%
\pgfpathlineto{\pgfqpoint{5.257741in}{2.101908in}}%
\pgfpathlineto{\pgfqpoint{5.257373in}{2.589448in}}%
\pgfpathlineto{\pgfqpoint{5.257998in}{2.402485in}}%
\pgfpathlineto{\pgfqpoint{5.258231in}{2.575261in}}%
\pgfpathlineto{\pgfqpoint{5.258708in}{2.055484in}}%
\pgfpathlineto{\pgfqpoint{5.259087in}{2.458739in}}%
\pgfpathlineto{\pgfqpoint{5.259905in}{2.090100in}}%
\pgfpathlineto{\pgfqpoint{5.259258in}{2.581934in}}%
\pgfpathlineto{\pgfqpoint{5.260185in}{2.391217in}}%
\pgfpathlineto{\pgfqpoint{5.261013in}{2.158965in}}%
\pgfpathlineto{\pgfqpoint{5.261305in}{2.594584in}}%
\pgfpathlineto{\pgfqpoint{5.262034in}{2.052352in}}%
\pgfpathlineto{\pgfqpoint{5.262410in}{2.123410in}}%
\pgfpathlineto{\pgfqpoint{5.262640in}{2.558969in}}%
\pgfpathlineto{\pgfqpoint{5.263524in}{2.461612in}}%
\pgfpathlineto{\pgfqpoint{5.264056in}{2.566746in}}%
\pgfpathlineto{\pgfqpoint{5.263995in}{2.157884in}}%
\pgfpathlineto{\pgfqpoint{5.264635in}{2.482382in}}%
\pgfpathlineto{\pgfqpoint{5.264719in}{1.989758in}}%
\pgfpathlineto{\pgfqpoint{5.265659in}{2.563540in}}%
\pgfpathlineto{\pgfqpoint{5.265755in}{2.398694in}}%
\pgfpathlineto{\pgfqpoint{5.266332in}{2.582016in}}%
\pgfpathlineto{\pgfqpoint{5.266152in}{2.236210in}}%
\pgfpathlineto{\pgfqpoint{5.266548in}{2.488014in}}%
\pgfpathlineto{\pgfqpoint{5.267472in}{2.098951in}}%
\pgfpathlineto{\pgfqpoint{5.267184in}{2.574710in}}%
\pgfpathlineto{\pgfqpoint{5.267664in}{2.263019in}}%
\pgfpathlineto{\pgfqpoint{5.267867in}{2.534957in}}%
\pgfpathlineto{\pgfqpoint{5.267891in}{2.111308in}}%
\pgfpathlineto{\pgfqpoint{5.268764in}{2.461391in}}%
\pgfpathlineto{\pgfqpoint{5.269289in}{2.147085in}}%
\pgfpathlineto{\pgfqpoint{5.268979in}{2.566966in}}%
\pgfpathlineto{\pgfqpoint{5.269873in}{2.466843in}}%
\pgfpathlineto{\pgfqpoint{5.270742in}{2.572802in}}%
\pgfpathlineto{\pgfqpoint{5.270861in}{2.135881in}}%
\pgfpathlineto{\pgfqpoint{5.270944in}{2.492895in}}%
\pgfpathlineto{\pgfqpoint{5.271728in}{1.956611in}}%
\pgfpathlineto{\pgfqpoint{5.271597in}{2.586351in}}%
\pgfpathlineto{\pgfqpoint{5.272048in}{2.386430in}}%
\pgfpathlineto{\pgfqpoint{5.272948in}{2.546299in}}%
\pgfpathlineto{\pgfqpoint{5.272569in}{1.982347in}}%
\pgfpathlineto{\pgfqpoint{5.273137in}{2.391201in}}%
\pgfpathlineto{\pgfqpoint{5.273869in}{2.175824in}}%
\pgfpathlineto{\pgfqpoint{5.273373in}{2.576808in}}%
\pgfpathlineto{\pgfqpoint{5.274235in}{2.369718in}}%
\pgfpathlineto{\pgfqpoint{5.274683in}{2.593694in}}%
\pgfpathlineto{\pgfqpoint{5.274600in}{2.006665in}}%
\pgfpathlineto{\pgfqpoint{5.275342in}{2.513722in}}%
\pgfpathlineto{\pgfqpoint{5.275389in}{2.113012in}}%
\pgfpathlineto{\pgfqpoint{5.275812in}{2.574224in}}%
\pgfpathlineto{\pgfqpoint{5.276458in}{2.314484in}}%
\pgfpathlineto{\pgfqpoint{5.276587in}{2.551082in}}%
\pgfpathlineto{\pgfqpoint{5.276540in}{2.010661in}}%
\pgfpathlineto{\pgfqpoint{5.277571in}{2.477246in}}%
\pgfpathlineto{\pgfqpoint{5.278202in}{2.110342in}}%
\pgfpathlineto{\pgfqpoint{5.278575in}{2.551493in}}%
\pgfpathlineto{\pgfqpoint{5.278692in}{2.309602in}}%
\pgfpathlineto{\pgfqpoint{5.279648in}{2.556287in}}%
\pgfpathlineto{\pgfqpoint{5.278716in}{2.040939in}}%
\pgfpathlineto{\pgfqpoint{5.279811in}{2.508670in}}%
\pgfpathlineto{\pgfqpoint{5.280741in}{2.051160in}}%
\pgfpathlineto{\pgfqpoint{5.280253in}{2.545370in}}%
\pgfpathlineto{\pgfqpoint{5.280915in}{2.448635in}}%
\pgfpathlineto{\pgfqpoint{5.281263in}{2.569504in}}%
\pgfpathlineto{\pgfqpoint{5.281877in}{2.013812in}}%
\pgfpathlineto{\pgfqpoint{5.282016in}{2.519420in}}%
\pgfpathlineto{\pgfqpoint{5.282468in}{2.102035in}}%
\pgfpathlineto{\pgfqpoint{5.283115in}{2.543926in}}%
\pgfpathlineto{\pgfqpoint{5.283126in}{2.557626in}}%
\pgfpathlineto{\pgfqpoint{5.283219in}{2.090879in}}%
\pgfpathlineto{\pgfqpoint{5.284003in}{2.452797in}}%
\pgfpathlineto{\pgfqpoint{5.284993in}{1.953947in}}%
\pgfpathlineto{\pgfqpoint{5.284682in}{2.579883in}}%
\pgfpathlineto{\pgfqpoint{5.285108in}{2.391821in}}%
\pgfpathlineto{\pgfqpoint{5.285326in}{1.870959in}}%
\pgfpathlineto{\pgfqpoint{5.285441in}{2.557607in}}%
\pgfpathlineto{\pgfqpoint{5.286152in}{2.338259in}}%
\pgfpathlineto{\pgfqpoint{5.286965in}{2.552517in}}%
\pgfpathlineto{\pgfqpoint{5.286748in}{2.093315in}}%
\pgfpathlineto{\pgfqpoint{5.287263in}{2.395898in}}%
\pgfpathlineto{\pgfqpoint{5.287937in}{2.547715in}}%
\pgfpathlineto{\pgfqpoint{5.287743in}{2.170249in}}%
\pgfpathlineto{\pgfqpoint{5.288302in}{2.492715in}}%
\pgfpathlineto{\pgfqpoint{5.288895in}{1.833695in}}%
\pgfpathlineto{\pgfqpoint{5.288758in}{2.538567in}}%
\pgfpathlineto{\pgfqpoint{5.289407in}{2.214499in}}%
\pgfpathlineto{\pgfqpoint{5.289544in}{2.583991in}}%
\pgfpathlineto{\pgfqpoint{5.290294in}{2.048375in}}%
\pgfpathlineto{\pgfqpoint{5.290521in}{2.529349in}}%
\pgfpathlineto{\pgfqpoint{5.291042in}{2.188730in}}%
\pgfpathlineto{\pgfqpoint{5.290895in}{2.566367in}}%
\pgfpathlineto{\pgfqpoint{5.291631in}{2.516952in}}%
\pgfpathlineto{\pgfqpoint{5.291948in}{1.948566in}}%
\pgfpathlineto{\pgfqpoint{5.291892in}{2.546481in}}%
\pgfpathlineto{\pgfqpoint{5.292762in}{2.442835in}}%
\pgfpathlineto{\pgfqpoint{5.293529in}{2.590095in}}%
\pgfpathlineto{\pgfqpoint{5.293359in}{2.095928in}}%
\pgfpathlineto{\pgfqpoint{5.293833in}{2.384801in}}%
\pgfpathlineto{\pgfqpoint{5.294316in}{2.184813in}}%
\pgfpathlineto{\pgfqpoint{5.293934in}{2.559015in}}%
\pgfpathlineto{\pgfqpoint{5.294923in}{2.484384in}}%
\pgfpathlineto{\pgfqpoint{5.295798in}{1.743969in}}%
\pgfpathlineto{\pgfqpoint{5.294991in}{2.573452in}}%
\pgfpathlineto{\pgfqpoint{5.296045in}{2.392483in}}%
\pgfpathlineto{\pgfqpoint{5.296459in}{2.590939in}}%
\pgfpathlineto{\pgfqpoint{5.297041in}{2.067041in}}%
\pgfpathlineto{\pgfqpoint{5.297152in}{2.432906in}}%
\pgfpathlineto{\pgfqpoint{5.297331in}{2.227300in}}%
\pgfpathlineto{\pgfqpoint{5.298034in}{2.540438in}}%
\pgfpathlineto{\pgfqpoint{5.298223in}{2.452270in}}%
\pgfpathlineto{\pgfqpoint{5.298446in}{2.601383in}}%
\pgfpathlineto{\pgfqpoint{5.298992in}{1.827915in}}%
\pgfpathlineto{\pgfqpoint{5.299314in}{2.458056in}}%
\pgfpathlineto{\pgfqpoint{5.299436in}{2.092786in}}%
\pgfpathlineto{\pgfqpoint{5.299903in}{2.550910in}}%
\pgfpathlineto{\pgfqpoint{5.300424in}{2.308155in}}%
\pgfpathlineto{\pgfqpoint{5.301288in}{2.586295in}}%
\pgfpathlineto{\pgfqpoint{5.300878in}{2.114259in}}%
\pgfpathlineto{\pgfqpoint{5.301542in}{2.472704in}}%
\pgfpathlineto{\pgfqpoint{5.302536in}{1.975004in}}%
\pgfpathlineto{\pgfqpoint{5.301896in}{2.565691in}}%
\pgfpathlineto{\pgfqpoint{5.302591in}{2.426319in}}%
\pgfpathlineto{\pgfqpoint{5.303517in}{2.573745in}}%
\pgfpathlineto{\pgfqpoint{5.302636in}{2.161823in}}%
\pgfpathlineto{\pgfqpoint{5.303693in}{2.423405in}}%
\pgfpathlineto{\pgfqpoint{5.304023in}{2.176376in}}%
\pgfpathlineto{\pgfqpoint{5.303836in}{2.582651in}}%
\pgfpathlineto{\pgfqpoint{5.304781in}{2.435595in}}%
\pgfpathlineto{\pgfqpoint{5.305756in}{2.549002in}}%
\pgfpathlineto{\pgfqpoint{5.305767in}{2.134135in}}%
\pgfpathlineto{\pgfqpoint{5.305876in}{2.501836in}}%
\pgfpathlineto{\pgfqpoint{5.306533in}{2.003920in}}%
\pgfpathlineto{\pgfqpoint{5.306511in}{2.559079in}}%
\pgfpathlineto{\pgfqpoint{5.306991in}{2.281083in}}%
\pgfpathlineto{\pgfqpoint{5.307013in}{2.093565in}}%
\pgfpathlineto{\pgfqpoint{5.307885in}{2.534425in}}%
\pgfpathlineto{\pgfqpoint{5.307951in}{2.299525in}}%
\pgfpathlineto{\pgfqpoint{5.308919in}{2.545878in}}%
\pgfpathlineto{\pgfqpoint{5.308854in}{2.128309in}}%
\pgfpathlineto{\pgfqpoint{5.309060in}{2.356601in}}%
\pgfpathlineto{\pgfqpoint{5.309473in}{2.069737in}}%
\pgfpathlineto{\pgfqpoint{5.309679in}{2.551359in}}%
\pgfpathlineto{\pgfqpoint{5.310123in}{2.473270in}}%
\pgfpathlineto{\pgfqpoint{5.310145in}{2.381255in}}%
\pgfpathlineto{\pgfqpoint{5.310167in}{2.386930in}}%
\pgfpathlineto{\pgfqpoint{5.311173in}{2.013743in}}%
\pgfpathlineto{\pgfqpoint{5.310275in}{2.554629in}}%
\pgfpathlineto{\pgfqpoint{5.311270in}{2.373360in}}%
\pgfpathlineto{\pgfqpoint{5.311530in}{2.566154in}}%
\pgfpathlineto{\pgfqpoint{5.311983in}{1.985500in}}%
\pgfpathlineto{\pgfqpoint{5.312382in}{2.542693in}}%
\pgfpathlineto{\pgfqpoint{5.312608in}{2.133726in}}%
\pgfpathlineto{\pgfqpoint{5.313189in}{2.560198in}}%
\pgfpathlineto{\pgfqpoint{5.313491in}{2.333082in}}%
\pgfpathlineto{\pgfqpoint{5.313834in}{2.572252in}}%
\pgfpathlineto{\pgfqpoint{5.313942in}{2.102830in}}%
\pgfpathlineto{\pgfqpoint{5.314607in}{2.456352in}}%
\pgfpathlineto{\pgfqpoint{5.315528in}{2.095313in}}%
\pgfpathlineto{\pgfqpoint{5.314907in}{2.577436in}}%
\pgfpathlineto{\pgfqpoint{5.315689in}{2.437328in}}%
\pgfpathlineto{\pgfqpoint{5.315699in}{2.551289in}}%
\pgfpathlineto{\pgfqpoint{5.315817in}{1.936046in}}%
\pgfpathlineto{\pgfqpoint{5.316789in}{2.432733in}}%
\pgfpathlineto{\pgfqpoint{5.316959in}{2.086694in}}%
\pgfpathlineto{\pgfqpoint{5.317151in}{2.552843in}}%
\pgfpathlineto{\pgfqpoint{5.317907in}{2.143417in}}%
\pgfpathlineto{\pgfqpoint{5.318694in}{2.571429in}}%
\pgfpathlineto{\pgfqpoint{5.319023in}{2.432816in}}%
\pgfpathlineto{\pgfqpoint{5.319394in}{2.569990in}}%
\pgfpathlineto{\pgfqpoint{5.320093in}{2.083969in}}%
\pgfpathlineto{\pgfqpoint{5.320727in}{2.588997in}}%
\pgfpathlineto{\pgfqpoint{5.320981in}{2.012330in}}%
\pgfpathlineto{\pgfqpoint{5.321213in}{2.477727in}}%
\pgfpathlineto{\pgfqpoint{5.321435in}{2.107162in}}%
\pgfpathlineto{\pgfqpoint{5.322036in}{2.588675in}}%
\pgfpathlineto{\pgfqpoint{5.322331in}{2.375148in}}%
\pgfpathlineto{\pgfqpoint{5.323298in}{2.148646in}}%
\pgfpathlineto{\pgfqpoint{5.323267in}{2.543528in}}%
\pgfpathlineto{\pgfqpoint{5.323435in}{2.351612in}}%
\pgfpathlineto{\pgfqpoint{5.324274in}{2.559638in}}%
\pgfpathlineto{\pgfqpoint{5.323813in}{2.075770in}}%
\pgfpathlineto{\pgfqpoint{5.324557in}{2.492784in}}%
\pgfpathlineto{\pgfqpoint{5.325331in}{2.013843in}}%
\pgfpathlineto{\pgfqpoint{5.325477in}{2.543682in}}%
\pgfpathlineto{\pgfqpoint{5.325666in}{2.354796in}}%
\pgfpathlineto{\pgfqpoint{5.326052in}{2.564710in}}%
\pgfpathlineto{\pgfqpoint{5.326135in}{2.155104in}}%
\pgfpathlineto{\pgfqpoint{5.326771in}{2.395822in}}%
\pgfpathlineto{\pgfqpoint{5.327552in}{2.562719in}}%
\pgfpathlineto{\pgfqpoint{5.327188in}{1.910805in}}%
\pgfpathlineto{\pgfqpoint{5.327864in}{2.470681in}}%
\pgfpathlineto{\pgfqpoint{5.328404in}{1.928060in}}%
\pgfpathlineto{\pgfqpoint{5.328663in}{2.561614in}}%
\pgfpathlineto{\pgfqpoint{5.328985in}{2.341111in}}%
\pgfpathlineto{\pgfqpoint{5.329275in}{2.562395in}}%
\pgfpathlineto{\pgfqpoint{5.329099in}{2.071730in}}%
\pgfpathlineto{\pgfqpoint{5.330103in}{2.429318in}}%
\pgfpathlineto{\pgfqpoint{5.331052in}{2.551412in}}%
\pgfpathlineto{\pgfqpoint{5.330660in}{2.039274in}}%
\pgfpathlineto{\pgfqpoint{5.331114in}{2.368217in}}%
\pgfpathlineto{\pgfqpoint{5.331825in}{2.084503in}}%
\pgfpathlineto{\pgfqpoint{5.332124in}{2.555926in}}%
\pgfpathlineto{\pgfqpoint{5.332206in}{2.330473in}}%
\pgfpathlineto{\pgfqpoint{5.332607in}{2.553695in}}%
\pgfpathlineto{\pgfqpoint{5.332884in}{2.195277in}}%
\pgfpathlineto{\pgfqpoint{5.333315in}{2.417424in}}%
\pgfpathlineto{\pgfqpoint{5.333562in}{2.544754in}}%
\pgfpathlineto{\pgfqpoint{5.333367in}{2.136596in}}%
\pgfpathlineto{\pgfqpoint{5.334217in}{2.329588in}}%
\pgfpathlineto{\pgfqpoint{5.335250in}{2.022978in}}%
\pgfpathlineto{\pgfqpoint{5.334790in}{2.570485in}}%
\pgfpathlineto{\pgfqpoint{5.335311in}{2.418310in}}%
\pgfpathlineto{\pgfqpoint{5.336087in}{2.549466in}}%
\pgfpathlineto{\pgfqpoint{5.335628in}{1.933227in}}%
\pgfpathlineto{\pgfqpoint{5.336372in}{2.503213in}}%
\pgfpathlineto{\pgfqpoint{5.336749in}{1.882230in}}%
\pgfpathlineto{\pgfqpoint{5.336637in}{2.535351in}}%
\pgfpathlineto{\pgfqpoint{5.337481in}{2.465052in}}%
\pgfpathlineto{\pgfqpoint{5.337542in}{2.517024in}}%
\pgfpathlineto{\pgfqpoint{5.337552in}{2.333231in}}%
\pgfpathlineto{\pgfqpoint{5.337582in}{2.405917in}}%
\pgfpathlineto{\pgfqpoint{5.337887in}{2.122371in}}%
\pgfpathlineto{\pgfqpoint{5.337755in}{2.581524in}}%
\pgfpathlineto{\pgfqpoint{5.338688in}{2.292556in}}%
\pgfpathlineto{\pgfqpoint{5.339265in}{2.573042in}}%
\pgfpathlineto{\pgfqpoint{5.339386in}{2.117759in}}%
\pgfpathlineto{\pgfqpoint{5.339791in}{2.265969in}}%
\pgfpathlineto{\pgfqpoint{5.340739in}{2.094735in}}%
\pgfpathlineto{\pgfqpoint{5.340154in}{2.602180in}}%
\pgfpathlineto{\pgfqpoint{5.340870in}{2.449221in}}%
\pgfpathlineto{\pgfqpoint{5.341646in}{2.542984in}}%
\pgfpathlineto{\pgfqpoint{5.340941in}{2.078455in}}%
\pgfpathlineto{\pgfqpoint{5.341877in}{2.450760in}}%
\pgfpathlineto{\pgfqpoint{5.342389in}{2.060228in}}%
\pgfpathlineto{\pgfqpoint{5.342480in}{2.520937in}}%
\pgfpathlineto{\pgfqpoint{5.342981in}{2.460737in}}%
\pgfpathlineto{\pgfqpoint{5.343863in}{2.560962in}}%
\pgfpathlineto{\pgfqpoint{5.343322in}{2.096175in}}%
\pgfpathlineto{\pgfqpoint{5.344063in}{2.497935in}}%
\pgfpathlineto{\pgfqpoint{5.344743in}{2.541448in}}%
\pgfpathlineto{\pgfqpoint{5.345172in}{2.011824in}}%
\pgfpathlineto{\pgfqpoint{5.345929in}{2.550105in}}%
\pgfpathlineto{\pgfqpoint{5.346288in}{2.418336in}}%
\pgfpathlineto{\pgfqpoint{5.347202in}{2.145842in}}%
\pgfpathlineto{\pgfqpoint{5.346367in}{2.531373in}}%
\pgfpathlineto{\pgfqpoint{5.347391in}{2.431669in}}%
\pgfpathlineto{\pgfqpoint{5.348273in}{1.978600in}}%
\pgfpathlineto{\pgfqpoint{5.347589in}{2.540777in}}%
\pgfpathlineto{\pgfqpoint{5.348481in}{2.361117in}}%
\pgfpathlineto{\pgfqpoint{5.348709in}{2.550218in}}%
\pgfpathlineto{\pgfqpoint{5.349480in}{2.094422in}}%
\pgfpathlineto{\pgfqpoint{5.349578in}{2.465954in}}%
\pgfpathlineto{\pgfqpoint{5.349894in}{2.035256in}}%
\pgfpathlineto{\pgfqpoint{5.349736in}{2.573387in}}%
\pgfpathlineto{\pgfqpoint{5.350683in}{2.343672in}}%
\pgfpathlineto{\pgfqpoint{5.350939in}{2.549175in}}%
\pgfpathlineto{\pgfqpoint{5.351715in}{1.972518in}}%
\pgfpathlineto{\pgfqpoint{5.351774in}{2.410833in}}%
\pgfpathlineto{\pgfqpoint{5.352295in}{2.021547in}}%
\pgfpathlineto{\pgfqpoint{5.351912in}{2.562294in}}%
\pgfpathlineto{\pgfqpoint{5.352883in}{2.295108in}}%
\pgfpathlineto{\pgfqpoint{5.353245in}{1.713318in}}%
\pgfpathlineto{\pgfqpoint{5.353988in}{2.554127in}}%
\pgfpathlineto{\pgfqpoint{5.354418in}{1.881895in}}%
\pgfpathlineto{\pgfqpoint{5.355101in}{2.425829in}}%
\pgfpathlineto{\pgfqpoint{5.355325in}{2.552189in}}%
\pgfpathlineto{\pgfqpoint{5.355695in}{2.027256in}}%
\pgfpathlineto{\pgfqpoint{5.355802in}{2.462008in}}%
\pgfpathlineto{\pgfqpoint{5.356327in}{1.845644in}}%
\pgfpathlineto{\pgfqpoint{5.356512in}{2.542434in}}%
\pgfpathlineto{\pgfqpoint{5.356910in}{2.387132in}}%
\pgfpathlineto{\pgfqpoint{5.357443in}{2.570928in}}%
\pgfpathlineto{\pgfqpoint{5.357046in}{1.963099in}}%
\pgfpathlineto{\pgfqpoint{5.358024in}{2.499660in}}%
\pgfpathlineto{\pgfqpoint{5.358643in}{1.985023in}}%
\pgfpathlineto{\pgfqpoint{5.358489in}{2.530330in}}%
\pgfpathlineto{\pgfqpoint{5.359145in}{2.366288in}}%
\pgfpathlineto{\pgfqpoint{5.359715in}{2.548095in}}%
\pgfpathlineto{\pgfqpoint{5.359406in}{1.924219in}}%
\pgfpathlineto{\pgfqpoint{5.360244in}{2.392192in}}%
\pgfpathlineto{\pgfqpoint{5.361101in}{2.068598in}}%
\pgfpathlineto{\pgfqpoint{5.360350in}{2.563952in}}%
\pgfpathlineto{\pgfqpoint{5.361350in}{2.441249in}}%
\pgfpathlineto{\pgfqpoint{5.361897in}{2.546505in}}%
\pgfpathlineto{\pgfqpoint{5.361600in}{2.086176in}}%
\pgfpathlineto{\pgfqpoint{5.362444in}{2.383507in}}%
\pgfpathlineto{\pgfqpoint{5.362903in}{2.228303in}}%
\pgfpathlineto{\pgfqpoint{5.362635in}{2.555782in}}%
\pgfpathlineto{\pgfqpoint{5.363448in}{2.467141in}}%
\pgfpathlineto{\pgfqpoint{5.364365in}{2.571746in}}%
\pgfpathlineto{\pgfqpoint{5.364460in}{2.158083in}}%
\pgfpathlineto{\pgfqpoint{5.364555in}{2.496136in}}%
\pgfpathlineto{\pgfqpoint{5.365365in}{1.941294in}}%
\pgfpathlineto{\pgfqpoint{5.365031in}{2.553906in}}%
\pgfpathlineto{\pgfqpoint{5.365678in}{2.159290in}}%
\pgfpathlineto{\pgfqpoint{5.366362in}{2.558066in}}%
\pgfpathlineto{\pgfqpoint{5.366419in}{1.983251in}}%
\pgfpathlineto{\pgfqpoint{5.366789in}{2.285133in}}%
\pgfpathlineto{\pgfqpoint{5.367026in}{2.574962in}}%
\pgfpathlineto{\pgfqpoint{5.367717in}{1.815988in}}%
\pgfpathlineto{\pgfqpoint{5.367906in}{2.497976in}}%
\pgfpathlineto{\pgfqpoint{5.368653in}{1.962555in}}%
\pgfpathlineto{\pgfqpoint{5.368152in}{2.557689in}}%
\pgfpathlineto{\pgfqpoint{5.369011in}{2.368790in}}%
\pgfpathlineto{\pgfqpoint{5.369887in}{2.550804in}}%
\pgfpathlineto{\pgfqpoint{5.369869in}{2.115561in}}%
\pgfpathlineto{\pgfqpoint{5.370066in}{2.250618in}}%
\pgfpathlineto{\pgfqpoint{5.370076in}{2.105115in}}%
\pgfpathlineto{\pgfqpoint{5.370931in}{2.543391in}}%
\pgfpathlineto{\pgfqpoint{5.371166in}{2.273158in}}%
\pgfpathlineto{\pgfqpoint{5.371259in}{2.567837in}}%
\pgfpathlineto{\pgfqpoint{5.372234in}{2.017537in}}%
\pgfpathlineto{\pgfqpoint{5.372271in}{2.296424in}}%
\pgfpathlineto{\pgfqpoint{5.372281in}{2.295633in}}%
\pgfpathlineto{\pgfqpoint{5.372730in}{2.540758in}}%
\pgfpathlineto{\pgfqpoint{5.373178in}{1.991956in}}%
\pgfpathlineto{\pgfqpoint{5.373393in}{2.510284in}}%
\pgfpathlineto{\pgfqpoint{5.374195in}{2.058961in}}%
\pgfpathlineto{\pgfqpoint{5.373981in}{2.546895in}}%
\pgfpathlineto{\pgfqpoint{5.374502in}{2.515767in}}%
\pgfpathlineto{\pgfqpoint{5.375302in}{2.015055in}}%
\pgfpathlineto{\pgfqpoint{5.374828in}{2.543885in}}%
\pgfpathlineto{\pgfqpoint{5.375646in}{2.037392in}}%
\pgfpathlineto{\pgfqpoint{5.375878in}{2.545719in}}%
\pgfpathlineto{\pgfqpoint{5.376768in}{2.445132in}}%
\pgfpathlineto{\pgfqpoint{5.376990in}{1.955853in}}%
\pgfpathlineto{\pgfqpoint{5.377813in}{2.527265in}}%
\pgfpathlineto{\pgfqpoint{5.377878in}{2.429153in}}%
\pgfpathlineto{\pgfqpoint{5.378385in}{2.055550in}}%
\pgfpathlineto{\pgfqpoint{5.378395in}{2.538587in}}%
\pgfpathlineto{\pgfqpoint{5.378985in}{2.358475in}}%
\pgfpathlineto{\pgfqpoint{5.379638in}{2.538700in}}%
\pgfpathlineto{\pgfqpoint{5.379417in}{2.080721in}}%
\pgfpathlineto{\pgfqpoint{5.380089in}{2.424427in}}%
\pgfpathlineto{\pgfqpoint{5.380465in}{2.025685in}}%
\pgfpathlineto{\pgfqpoint{5.381034in}{2.551164in}}%
\pgfpathlineto{\pgfqpoint{5.381190in}{2.379494in}}%
\pgfpathlineto{\pgfqpoint{5.381657in}{2.526107in}}%
\pgfpathlineto{\pgfqpoint{5.381236in}{2.011200in}}%
\pgfpathlineto{\pgfqpoint{5.382279in}{2.293434in}}%
\pgfpathlineto{\pgfqpoint{5.382754in}{1.995600in}}%
\pgfpathlineto{\pgfqpoint{5.382873in}{2.558297in}}%
\pgfpathlineto{\pgfqpoint{5.383375in}{2.262877in}}%
\pgfpathlineto{\pgfqpoint{5.383976in}{2.526790in}}%
\pgfpathlineto{\pgfqpoint{5.384240in}{2.004158in}}%
\pgfpathlineto{\pgfqpoint{5.384485in}{2.378699in}}%
\pgfpathlineto{\pgfqpoint{5.385348in}{2.080828in}}%
\pgfpathlineto{\pgfqpoint{5.385031in}{2.569662in}}%
\pgfpathlineto{\pgfqpoint{5.385593in}{2.334239in}}%
\pgfpathlineto{\pgfqpoint{5.386019in}{2.560482in}}%
\pgfpathlineto{\pgfqpoint{5.386599in}{2.165568in}}%
\pgfpathlineto{\pgfqpoint{5.386698in}{2.391826in}}%
\pgfpathlineto{\pgfqpoint{5.387485in}{2.144772in}}%
\pgfpathlineto{\pgfqpoint{5.386789in}{2.513623in}}%
\pgfpathlineto{\pgfqpoint{5.387792in}{2.299028in}}%
\pgfpathlineto{\pgfqpoint{5.388783in}{2.551845in}}%
\pgfpathlineto{\pgfqpoint{5.387873in}{2.134933in}}%
\pgfpathlineto{\pgfqpoint{5.388900in}{2.435272in}}%
\pgfpathlineto{\pgfqpoint{5.388999in}{2.022161in}}%
\pgfpathlineto{\pgfqpoint{5.389089in}{2.555959in}}%
\pgfpathlineto{\pgfqpoint{5.389988in}{2.484960in}}%
\pgfpathlineto{\pgfqpoint{5.390454in}{2.562370in}}%
\pgfpathlineto{\pgfqpoint{5.390140in}{1.994646in}}%
\pgfpathlineto{\pgfqpoint{5.391072in}{2.347876in}}%
\pgfpathlineto{\pgfqpoint{5.391904in}{2.078110in}}%
\pgfpathlineto{\pgfqpoint{5.391269in}{2.531812in}}%
\pgfpathlineto{\pgfqpoint{5.391931in}{2.448457in}}%
\pgfpathlineto{\pgfqpoint{5.392181in}{2.559000in}}%
\pgfpathlineto{\pgfqpoint{5.391994in}{1.825568in}}%
\pgfpathlineto{\pgfqpoint{5.393038in}{2.483560in}}%
\pgfpathlineto{\pgfqpoint{5.393536in}{2.039757in}}%
\pgfpathlineto{\pgfqpoint{5.393162in}{2.560056in}}%
\pgfpathlineto{\pgfqpoint{5.394159in}{2.278363in}}%
\pgfpathlineto{\pgfqpoint{5.394381in}{2.572191in}}%
\pgfpathlineto{\pgfqpoint{5.394266in}{2.134103in}}%
\pgfpathlineto{\pgfqpoint{5.395269in}{2.449451in}}%
\pgfpathlineto{\pgfqpoint{5.395703in}{2.020815in}}%
\pgfpathlineto{\pgfqpoint{5.395827in}{2.543850in}}%
\pgfpathlineto{\pgfqpoint{5.396367in}{2.467893in}}%
\pgfpathlineto{\pgfqpoint{5.396375in}{2.577611in}}%
\pgfpathlineto{\pgfqpoint{5.397215in}{2.210053in}}%
\pgfpathlineto{\pgfqpoint{5.397462in}{2.371905in}}%
\pgfpathlineto{\pgfqpoint{5.398052in}{2.038918in}}%
\pgfpathlineto{\pgfqpoint{5.397585in}{2.533964in}}%
\pgfpathlineto{\pgfqpoint{5.398571in}{2.346890in}}%
\pgfpathlineto{\pgfqpoint{5.399125in}{1.955933in}}%
\pgfpathlineto{\pgfqpoint{5.398958in}{2.550302in}}%
\pgfpathlineto{\pgfqpoint{5.399635in}{2.432375in}}%
\pgfpathlineto{\pgfqpoint{5.400310in}{2.552144in}}%
\pgfpathlineto{\pgfqpoint{5.399810in}{2.021290in}}%
\pgfpathlineto{\pgfqpoint{5.400721in}{2.360808in}}%
\pgfpathlineto{\pgfqpoint{5.401360in}{2.028375in}}%
\pgfpathlineto{\pgfqpoint{5.401430in}{2.553335in}}%
\pgfpathlineto{\pgfqpoint{5.401831in}{2.319546in}}%
\pgfpathlineto{\pgfqpoint{5.401919in}{2.529680in}}%
\pgfpathlineto{\pgfqpoint{5.402538in}{2.182491in}}%
\pgfpathlineto{\pgfqpoint{5.402947in}{2.474399in}}%
\pgfpathlineto{\pgfqpoint{5.403999in}{2.049484in}}%
\pgfpathlineto{\pgfqpoint{5.404034in}{2.552614in}}%
\pgfpathlineto{\pgfqpoint{5.404060in}{2.308087in}}%
\pgfpathlineto{\pgfqpoint{5.404095in}{2.539667in}}%
\pgfpathlineto{\pgfqpoint{5.404919in}{2.112722in}}%
\pgfpathlineto{\pgfqpoint{5.405170in}{2.455986in}}%
\pgfpathlineto{\pgfqpoint{5.405940in}{2.021189in}}%
\pgfpathlineto{\pgfqpoint{5.406243in}{2.550942in}}%
\pgfpathlineto{\pgfqpoint{5.406277in}{2.249794in}}%
\pgfpathlineto{\pgfqpoint{5.406554in}{2.558848in}}%
\pgfpathlineto{\pgfqpoint{5.407080in}{2.074508in}}%
\pgfpathlineto{\pgfqpoint{5.407390in}{2.451522in}}%
\pgfpathlineto{\pgfqpoint{5.407657in}{2.096875in}}%
\pgfpathlineto{\pgfqpoint{5.407545in}{2.553254in}}%
\pgfpathlineto{\pgfqpoint{5.408500in}{2.251766in}}%
\pgfpathlineto{\pgfqpoint{5.409376in}{2.529110in}}%
\pgfpathlineto{\pgfqpoint{5.408612in}{2.104067in}}%
\pgfpathlineto{\pgfqpoint{5.409616in}{2.480246in}}%
\pgfpathlineto{\pgfqpoint{5.410061in}{1.737843in}}%
\pgfpathlineto{\pgfqpoint{5.410344in}{2.554119in}}%
\pgfpathlineto{\pgfqpoint{5.410729in}{2.387574in}}%
\pgfpathlineto{\pgfqpoint{5.411489in}{2.569611in}}%
\pgfpathlineto{\pgfqpoint{5.411164in}{2.123534in}}%
\pgfpathlineto{\pgfqpoint{5.411804in}{2.462724in}}%
\pgfpathlineto{\pgfqpoint{5.412614in}{1.958487in}}%
\pgfpathlineto{\pgfqpoint{5.412511in}{2.563208in}}%
\pgfpathlineto{\pgfqpoint{5.412920in}{2.279230in}}%
\pgfpathlineto{\pgfqpoint{5.413200in}{2.557150in}}%
\pgfpathlineto{\pgfqpoint{5.412962in}{1.763493in}}%
\pgfpathlineto{\pgfqpoint{5.414032in}{2.489941in}}%
\pgfpathlineto{\pgfqpoint{5.414854in}{1.894330in}}%
\pgfpathlineto{\pgfqpoint{5.414092in}{2.559827in}}%
\pgfpathlineto{\pgfqpoint{5.415159in}{2.202388in}}%
\pgfpathlineto{\pgfqpoint{5.415244in}{2.553191in}}%
\pgfpathlineto{\pgfqpoint{5.415742in}{2.049180in}}%
\pgfpathlineto{\pgfqpoint{5.416257in}{2.162344in}}%
\pgfpathlineto{\pgfqpoint{5.416603in}{2.040286in}}%
\pgfpathlineto{\pgfqpoint{5.416738in}{2.537295in}}%
\pgfpathlineto{\pgfqpoint{5.417336in}{2.386813in}}%
\pgfpathlineto{\pgfqpoint{5.417504in}{2.549847in}}%
\pgfpathlineto{\pgfqpoint{5.418244in}{1.833353in}}%
\pgfpathlineto{\pgfqpoint{5.418437in}{2.490802in}}%
\pgfpathlineto{\pgfqpoint{5.418999in}{1.848681in}}%
\pgfpathlineto{\pgfqpoint{5.419167in}{2.554717in}}%
\pgfpathlineto{\pgfqpoint{5.419544in}{2.520645in}}%
\pgfpathlineto{\pgfqpoint{5.420129in}{2.045908in}}%
\pgfpathlineto{\pgfqpoint{5.420439in}{2.538579in}}%
\pgfpathlineto{\pgfqpoint{5.420706in}{2.367566in}}%
\pgfpathlineto{\pgfqpoint{5.421490in}{2.577680in}}%
\pgfpathlineto{\pgfqpoint{5.421682in}{2.067347in}}%
\pgfpathlineto{\pgfqpoint{5.421823in}{2.557738in}}%
\pgfpathlineto{\pgfqpoint{5.421840in}{2.019914in}}%
\pgfpathlineto{\pgfqpoint{5.422929in}{2.436858in}}%
\pgfpathlineto{\pgfqpoint{5.423336in}{1.842607in}}%
\pgfpathlineto{\pgfqpoint{5.423004in}{2.560537in}}%
\pgfpathlineto{\pgfqpoint{5.424041in}{2.332018in}}%
\pgfpathlineto{\pgfqpoint{5.424513in}{2.582210in}}%
\pgfpathlineto{\pgfqpoint{5.424662in}{2.114727in}}%
\pgfpathlineto{\pgfqpoint{5.425150in}{2.389380in}}%
\pgfpathlineto{\pgfqpoint{5.425736in}{2.108932in}}%
\pgfpathlineto{\pgfqpoint{5.425513in}{2.535938in}}%
\pgfpathlineto{\pgfqpoint{5.426255in}{2.312483in}}%
\pgfpathlineto{\pgfqpoint{5.426560in}{2.564945in}}%
\pgfpathlineto{\pgfqpoint{5.426725in}{2.059335in}}%
\pgfpathlineto{\pgfqpoint{5.427367in}{2.452750in}}%
\pgfpathlineto{\pgfqpoint{5.427416in}{2.074719in}}%
\pgfpathlineto{\pgfqpoint{5.427917in}{2.535777in}}%
\pgfpathlineto{\pgfqpoint{5.428475in}{2.373044in}}%
\pgfpathlineto{\pgfqpoint{5.429433in}{2.530915in}}%
\pgfpathlineto{\pgfqpoint{5.428811in}{2.103144in}}%
\pgfpathlineto{\pgfqpoint{5.429474in}{2.390269in}}%
\pgfpathlineto{\pgfqpoint{5.429735in}{1.955007in}}%
\pgfpathlineto{\pgfqpoint{5.429490in}{2.525103in}}%
\pgfpathlineto{\pgfqpoint{5.430576in}{2.261276in}}%
\pgfpathlineto{\pgfqpoint{5.430642in}{2.525403in}}%
\pgfpathlineto{\pgfqpoint{5.431367in}{2.119943in}}%
\pgfpathlineto{\pgfqpoint{5.431684in}{2.439068in}}%
\pgfpathlineto{\pgfqpoint{5.432717in}{1.811908in}}%
\pgfpathlineto{\pgfqpoint{5.432351in}{2.548990in}}%
\pgfpathlineto{\pgfqpoint{5.432798in}{2.368482in}}%
\pgfpathlineto{\pgfqpoint{5.433057in}{2.538282in}}%
\pgfpathlineto{\pgfqpoint{5.433065in}{1.904398in}}%
\pgfpathlineto{\pgfqpoint{5.433908in}{2.487629in}}%
\pgfpathlineto{\pgfqpoint{5.434498in}{2.519011in}}%
\pgfpathlineto{\pgfqpoint{5.435023in}{2.160168in}}%
\pgfpathlineto{\pgfqpoint{5.435136in}{2.528563in}}%
\pgfpathlineto{\pgfqpoint{5.435362in}{2.041297in}}%
\pgfpathlineto{\pgfqpoint{5.436136in}{2.400768in}}%
\pgfpathlineto{\pgfqpoint{5.436217in}{2.017244in}}%
\pgfpathlineto{\pgfqpoint{5.436611in}{2.553027in}}%
\pgfpathlineto{\pgfqpoint{5.437238in}{2.406040in}}%
\pgfpathlineto{\pgfqpoint{5.438336in}{2.539874in}}%
\pgfpathlineto{\pgfqpoint{5.437719in}{1.990286in}}%
\pgfpathlineto{\pgfqpoint{5.438344in}{2.355968in}}%
\pgfpathlineto{\pgfqpoint{5.438585in}{2.552609in}}%
\pgfpathlineto{\pgfqpoint{5.438633in}{2.120968in}}%
\pgfpathlineto{\pgfqpoint{5.439456in}{2.492803in}}%
\pgfpathlineto{\pgfqpoint{5.440079in}{2.009299in}}%
\pgfpathlineto{\pgfqpoint{5.440549in}{2.541405in}}%
\pgfpathlineto{\pgfqpoint{5.440565in}{2.276731in}}%
\pgfpathlineto{\pgfqpoint{5.440621in}{2.551240in}}%
\pgfpathlineto{\pgfqpoint{5.441123in}{2.081025in}}%
\pgfpathlineto{\pgfqpoint{5.441671in}{2.420879in}}%
\pgfpathlineto{\pgfqpoint{5.441902in}{1.976350in}}%
\pgfpathlineto{\pgfqpoint{5.442235in}{2.555945in}}%
\pgfpathlineto{\pgfqpoint{5.442775in}{2.227612in}}%
\pgfpathlineto{\pgfqpoint{5.443487in}{2.537094in}}%
\pgfpathlineto{\pgfqpoint{5.443574in}{2.144237in}}%
\pgfpathlineto{\pgfqpoint{5.443883in}{2.406746in}}%
\pgfpathlineto{\pgfqpoint{5.444523in}{2.025535in}}%
\pgfpathlineto{\pgfqpoint{5.443954in}{2.517315in}}%
\pgfpathlineto{\pgfqpoint{5.444988in}{2.338277in}}%
\pgfpathlineto{\pgfqpoint{5.445154in}{2.524410in}}%
\pgfpathlineto{\pgfqpoint{5.445808in}{2.053288in}}%
\pgfpathlineto{\pgfqpoint{5.446114in}{2.506033in}}%
\pgfpathlineto{\pgfqpoint{5.446610in}{2.068495in}}%
\pgfpathlineto{\pgfqpoint{5.446209in}{2.550739in}}%
\pgfpathlineto{\pgfqpoint{5.447238in}{2.084107in}}%
\pgfpathlineto{\pgfqpoint{5.448005in}{2.577213in}}%
\pgfpathlineto{\pgfqpoint{5.447473in}{2.031320in}}%
\pgfpathlineto{\pgfqpoint{5.448358in}{2.475400in}}%
\pgfpathlineto{\pgfqpoint{5.449381in}{1.865818in}}%
\pgfpathlineto{\pgfqpoint{5.448639in}{2.538653in}}%
\pgfpathlineto{\pgfqpoint{5.449475in}{2.046914in}}%
\pgfpathlineto{\pgfqpoint{5.450028in}{2.544567in}}%
\pgfpathlineto{\pgfqpoint{5.450589in}{2.360588in}}%
\pgfpathlineto{\pgfqpoint{5.451367in}{2.142152in}}%
\pgfpathlineto{\pgfqpoint{5.450791in}{2.531696in}}%
\pgfpathlineto{\pgfqpoint{5.451685in}{2.422898in}}%
\pgfpathlineto{\pgfqpoint{5.452112in}{2.539107in}}%
\pgfpathlineto{\pgfqpoint{5.452360in}{1.955832in}}%
\pgfpathlineto{\pgfqpoint{5.452778in}{2.301140in}}%
\pgfpathlineto{\pgfqpoint{5.453528in}{2.113246in}}%
\pgfpathlineto{\pgfqpoint{5.453405in}{2.532976in}}%
\pgfpathlineto{\pgfqpoint{5.453868in}{2.305196in}}%
\pgfpathlineto{\pgfqpoint{5.454655in}{2.536333in}}%
\pgfpathlineto{\pgfqpoint{5.453922in}{2.064039in}}%
\pgfpathlineto{\pgfqpoint{5.454979in}{2.469594in}}%
\pgfpathlineto{\pgfqpoint{5.455548in}{2.013358in}}%
\pgfpathlineto{\pgfqpoint{5.455325in}{2.544842in}}%
\pgfpathlineto{\pgfqpoint{5.456094in}{2.368534in}}%
\pgfpathlineto{\pgfqpoint{5.456654in}{2.532111in}}%
\pgfpathlineto{\pgfqpoint{5.456440in}{1.950622in}}%
\pgfpathlineto{\pgfqpoint{5.457176in}{2.482176in}}%
\pgfpathlineto{\pgfqpoint{5.458262in}{2.030035in}}%
\pgfpathlineto{\pgfqpoint{5.457352in}{2.514324in}}%
\pgfpathlineto{\pgfqpoint{5.458285in}{2.377406in}}%
\pgfpathlineto{\pgfqpoint{5.458820in}{2.553458in}}%
\pgfpathlineto{\pgfqpoint{5.458377in}{2.090545in}}%
\pgfpathlineto{\pgfqpoint{5.459354in}{2.420058in}}%
\pgfpathlineto{\pgfqpoint{5.459826in}{2.122613in}}%
\pgfpathlineto{\pgfqpoint{5.459879in}{2.549225in}}%
\pgfpathlineto{\pgfqpoint{5.460465in}{2.364786in}}%
\pgfpathlineto{\pgfqpoint{5.461339in}{2.579056in}}%
\pgfpathlineto{\pgfqpoint{5.460610in}{2.027465in}}%
\pgfpathlineto{\pgfqpoint{5.461581in}{2.445171in}}%
\pgfpathlineto{\pgfqpoint{5.461650in}{1.993348in}}%
\pgfpathlineto{\pgfqpoint{5.461930in}{2.550689in}}%
\pgfpathlineto{\pgfqpoint{5.462695in}{2.274826in}}%
\pgfpathlineto{\pgfqpoint{5.463692in}{2.521204in}}%
\pgfpathlineto{\pgfqpoint{5.463526in}{2.067204in}}%
\pgfpathlineto{\pgfqpoint{5.463767in}{2.323902in}}%
\pgfpathlineto{\pgfqpoint{5.464461in}{2.553623in}}%
\pgfpathlineto{\pgfqpoint{5.464875in}{1.900615in}}%
\pgfpathlineto{\pgfqpoint{5.465702in}{2.587364in}}%
\pgfpathlineto{\pgfqpoint{5.465987in}{2.459075in}}%
\pgfpathlineto{\pgfqpoint{5.466557in}{2.004287in}}%
\pgfpathlineto{\pgfqpoint{5.466744in}{2.536746in}}%
\pgfpathlineto{\pgfqpoint{5.467111in}{2.372391in}}%
\pgfpathlineto{\pgfqpoint{5.467978in}{2.545196in}}%
\pgfpathlineto{\pgfqpoint{5.467844in}{2.005790in}}%
\pgfpathlineto{\pgfqpoint{5.468173in}{2.449168in}}%
\pgfpathlineto{\pgfqpoint{5.469083in}{1.922599in}}%
\pgfpathlineto{\pgfqpoint{5.468516in}{2.552487in}}%
\pgfpathlineto{\pgfqpoint{5.469291in}{2.238510in}}%
\pgfpathlineto{\pgfqpoint{5.469410in}{2.059089in}}%
\pgfpathlineto{\pgfqpoint{5.469559in}{2.541894in}}%
\pgfpathlineto{\pgfqpoint{5.470317in}{2.351848in}}%
\pgfpathlineto{\pgfqpoint{5.470911in}{2.527068in}}%
\pgfpathlineto{\pgfqpoint{5.470592in}{1.992016in}}%
\pgfpathlineto{\pgfqpoint{5.471423in}{2.408641in}}%
\pgfpathlineto{\pgfqpoint{5.472511in}{1.749426in}}%
\pgfpathlineto{\pgfqpoint{5.471963in}{2.538388in}}%
\pgfpathlineto{\pgfqpoint{5.472533in}{2.401535in}}%
\pgfpathlineto{\pgfqpoint{5.473219in}{2.064059in}}%
\pgfpathlineto{\pgfqpoint{5.473337in}{2.543313in}}%
\pgfpathlineto{\pgfqpoint{5.473544in}{2.212063in}}%
\pgfpathlineto{\pgfqpoint{5.473669in}{2.533546in}}%
\pgfpathlineto{\pgfqpoint{5.473890in}{1.920507in}}%
\pgfpathlineto{\pgfqpoint{5.474656in}{2.419634in}}%
\pgfpathlineto{\pgfqpoint{5.475361in}{2.534437in}}%
\pgfpathlineto{\pgfqpoint{5.475207in}{2.127036in}}%
\pgfpathlineto{\pgfqpoint{5.475684in}{2.477914in}}%
\pgfpathlineto{\pgfqpoint{5.476197in}{2.046280in}}%
\pgfpathlineto{\pgfqpoint{5.476651in}{2.543340in}}%
\pgfpathlineto{\pgfqpoint{5.476790in}{2.398065in}}%
\pgfpathlineto{\pgfqpoint{5.477433in}{2.516747in}}%
\pgfpathlineto{\pgfqpoint{5.477477in}{2.089200in}}%
\pgfpathlineto{\pgfqpoint{5.477871in}{2.401830in}}%
\pgfpathlineto{\pgfqpoint{5.477879in}{2.083512in}}%
\pgfpathlineto{\pgfqpoint{5.478673in}{2.527644in}}%
\pgfpathlineto{\pgfqpoint{5.478979in}{2.383693in}}%
\pgfpathlineto{\pgfqpoint{5.479110in}{2.544402in}}%
\pgfpathlineto{\pgfqpoint{5.479917in}{2.123374in}}%
\pgfpathlineto{\pgfqpoint{5.480091in}{2.413371in}}%
\pgfpathlineto{\pgfqpoint{5.480403in}{1.983883in}}%
\pgfpathlineto{\pgfqpoint{5.480534in}{2.499452in}}%
\pgfpathlineto{\pgfqpoint{5.481200in}{2.187688in}}%
\pgfpathlineto{\pgfqpoint{5.481685in}{2.534192in}}%
\pgfpathlineto{\pgfqpoint{5.481244in}{2.081001in}}%
\pgfpathlineto{\pgfqpoint{5.482314in}{2.376696in}}%
\pgfpathlineto{\pgfqpoint{5.482733in}{2.048837in}}%
\pgfpathlineto{\pgfqpoint{5.483324in}{2.512569in}}%
\pgfpathlineto{\pgfqpoint{5.483425in}{2.380644in}}%
\pgfpathlineto{\pgfqpoint{5.484259in}{1.716863in}}%
\pgfpathlineto{\pgfqpoint{5.484087in}{2.526751in}}%
\pgfpathlineto{\pgfqpoint{5.484532in}{2.286476in}}%
\pgfpathlineto{\pgfqpoint{5.485035in}{2.534056in}}%
\pgfpathlineto{\pgfqpoint{5.485530in}{2.053385in}}%
\pgfpathlineto{\pgfqpoint{5.485637in}{2.325325in}}%
\pgfpathlineto{\pgfqpoint{5.486189in}{1.954237in}}%
\pgfpathlineto{\pgfqpoint{5.486632in}{2.513174in}}%
\pgfpathlineto{\pgfqpoint{5.486732in}{2.298208in}}%
\pgfpathlineto{\pgfqpoint{5.487318in}{2.556023in}}%
\pgfpathlineto{\pgfqpoint{5.487589in}{1.823333in}}%
\pgfpathlineto{\pgfqpoint{5.487845in}{2.451877in}}%
\pgfpathlineto{\pgfqpoint{5.488935in}{2.075740in}}%
\pgfpathlineto{\pgfqpoint{5.488693in}{2.523735in}}%
\pgfpathlineto{\pgfqpoint{5.488956in}{2.302093in}}%
\pgfpathlineto{\pgfqpoint{5.489787in}{2.548131in}}%
\pgfpathlineto{\pgfqpoint{5.489041in}{1.870771in}}%
\pgfpathlineto{\pgfqpoint{5.490064in}{2.423888in}}%
\pgfpathlineto{\pgfqpoint{5.490141in}{1.276687in}}%
\pgfpathlineto{\pgfqpoint{5.490333in}{2.543326in}}%
\pgfpathlineto{\pgfqpoint{5.491175in}{2.227236in}}%
\pgfpathlineto{\pgfqpoint{5.491182in}{2.562557in}}%
\pgfpathlineto{\pgfqpoint{5.491218in}{2.079500in}}%
\pgfpathlineto{\pgfqpoint{5.492284in}{2.401901in}}%
\pgfpathlineto{\pgfqpoint{5.493355in}{2.058200in}}%
\pgfpathlineto{\pgfqpoint{5.493313in}{2.519483in}}%
\pgfpathlineto{\pgfqpoint{5.493397in}{2.230865in}}%
\pgfpathlineto{\pgfqpoint{5.494388in}{2.519501in}}%
\pgfpathlineto{\pgfqpoint{5.494325in}{1.994702in}}%
\pgfpathlineto{\pgfqpoint{5.494507in}{2.294392in}}%
\pgfpathlineto{\pgfqpoint{5.494528in}{1.969122in}}%
\pgfpathlineto{\pgfqpoint{5.495573in}{2.510940in}}%
\pgfpathlineto{\pgfqpoint{5.495607in}{2.430940in}}%
\pgfpathlineto{\pgfqpoint{5.496621in}{1.991836in}}%
\pgfpathlineto{\pgfqpoint{5.496572in}{2.534589in}}%
\pgfpathlineto{\pgfqpoint{5.496712in}{2.420327in}}%
\pgfpathlineto{\pgfqpoint{5.497333in}{2.510401in}}%
\pgfpathlineto{\pgfqpoint{5.497340in}{1.882782in}}%
\pgfpathlineto{\pgfqpoint{5.497611in}{2.363415in}}%
\pgfpathlineto{\pgfqpoint{5.497618in}{1.932919in}}%
\pgfpathlineto{\pgfqpoint{5.498349in}{2.524027in}}%
\pgfpathlineto{\pgfqpoint{5.498717in}{2.266414in}}%
\pgfpathlineto{\pgfqpoint{5.499273in}{2.531646in}}%
\pgfpathlineto{\pgfqpoint{5.499446in}{2.083832in}}%
\pgfpathlineto{\pgfqpoint{5.499835in}{2.428800in}}%
\pgfpathlineto{\pgfqpoint{5.500755in}{2.033081in}}%
\pgfpathlineto{\pgfqpoint{5.500624in}{2.515041in}}%
\pgfpathlineto{\pgfqpoint{5.500956in}{2.314479in}}%
\pgfpathlineto{\pgfqpoint{5.501901in}{1.826861in}}%
\pgfpathlineto{\pgfqpoint{5.501239in}{2.526996in}}%
\pgfpathlineto{\pgfqpoint{5.502053in}{2.382785in}}%
\pgfpathlineto{\pgfqpoint{5.502728in}{2.522746in}}%
\pgfpathlineto{\pgfqpoint{5.502280in}{2.129606in}}%
\pgfpathlineto{\pgfqpoint{5.503155in}{2.326266in}}%
\pgfpathlineto{\pgfqpoint{5.503478in}{2.514661in}}%
\pgfpathlineto{\pgfqpoint{5.503883in}{2.112500in}}%
\pgfpathlineto{\pgfqpoint{5.504486in}{2.040673in}}%
\pgfpathlineto{\pgfqpoint{5.504281in}{2.518754in}}%
\pgfpathlineto{\pgfqpoint{5.504959in}{2.359102in}}%
\pgfpathlineto{\pgfqpoint{5.505950in}{2.542956in}}%
\pgfpathlineto{\pgfqpoint{5.505978in}{2.116428in}}%
\pgfpathlineto{\pgfqpoint{5.506066in}{2.464242in}}%
\pgfpathlineto{\pgfqpoint{5.506667in}{2.030327in}}%
\pgfpathlineto{\pgfqpoint{5.507089in}{2.531892in}}%
\pgfpathlineto{\pgfqpoint{5.507178in}{2.388949in}}%
\pgfpathlineto{\pgfqpoint{5.507552in}{2.535913in}}%
\pgfpathlineto{\pgfqpoint{5.507859in}{1.786461in}}%
\pgfpathlineto{\pgfqpoint{5.508219in}{2.351520in}}%
\pgfpathlineto{\pgfqpoint{5.508511in}{2.043969in}}%
\pgfpathlineto{\pgfqpoint{5.509148in}{2.529965in}}%
\pgfpathlineto{\pgfqpoint{5.509325in}{2.210279in}}%
\pgfpathlineto{\pgfqpoint{5.509954in}{2.525280in}}%
\pgfpathlineto{\pgfqpoint{5.509880in}{2.017606in}}%
\pgfpathlineto{\pgfqpoint{5.510434in}{2.469258in}}%
\pgfpathlineto{\pgfqpoint{5.511353in}{2.137573in}}%
\pgfpathlineto{\pgfqpoint{5.511056in}{2.522828in}}%
\pgfpathlineto{\pgfqpoint{5.511548in}{2.290322in}}%
\pgfpathlineto{\pgfqpoint{5.512605in}{2.530769in}}%
\pgfpathlineto{\pgfqpoint{5.511932in}{2.067953in}}%
\pgfpathlineto{\pgfqpoint{5.512666in}{2.478048in}}%
\pgfpathlineto{\pgfqpoint{5.513311in}{1.922516in}}%
\pgfpathlineto{\pgfqpoint{5.512766in}{2.534850in}}%
\pgfpathlineto{\pgfqpoint{5.513787in}{2.421971in}}%
\pgfpathlineto{\pgfqpoint{5.514490in}{2.565319in}}%
\pgfpathlineto{\pgfqpoint{5.514222in}{2.046655in}}%
\pgfpathlineto{\pgfqpoint{5.514898in}{2.446494in}}%
\pgfpathlineto{\pgfqpoint{5.515159in}{2.076727in}}%
\pgfpathlineto{\pgfqpoint{5.515927in}{2.514424in}}%
\pgfpathlineto{\pgfqpoint{5.516007in}{2.373115in}}%
\pgfpathlineto{\pgfqpoint{5.516187in}{2.532575in}}%
\pgfpathlineto{\pgfqpoint{5.516227in}{2.004508in}}%
\pgfpathlineto{\pgfqpoint{5.517013in}{2.466106in}}%
\pgfpathlineto{\pgfqpoint{5.517485in}{1.802275in}}%
\pgfpathlineto{\pgfqpoint{5.518023in}{2.516077in}}%
\pgfpathlineto{\pgfqpoint{5.518123in}{2.409823in}}%
\pgfpathlineto{\pgfqpoint{5.518554in}{2.525612in}}%
\pgfpathlineto{\pgfqpoint{5.518998in}{1.969424in}}%
\pgfpathlineto{\pgfqpoint{5.519203in}{2.277540in}}%
\pgfpathlineto{\pgfqpoint{5.519210in}{2.110043in}}%
\pgfpathlineto{\pgfqpoint{5.519574in}{2.541859in}}%
\pgfpathlineto{\pgfqpoint{5.520314in}{2.218262in}}%
\pgfpathlineto{\pgfqpoint{5.520948in}{2.500317in}}%
\pgfpathlineto{\pgfqpoint{5.521172in}{2.074604in}}%
\pgfpathlineto{\pgfqpoint{5.521429in}{2.338621in}}%
\pgfpathlineto{\pgfqpoint{5.522060in}{2.529765in}}%
\pgfpathlineto{\pgfqpoint{5.522330in}{1.970832in}}%
\pgfpathlineto{\pgfqpoint{5.522534in}{2.299105in}}%
\pgfpathlineto{\pgfqpoint{5.523590in}{1.972785in}}%
\pgfpathlineto{\pgfqpoint{5.522901in}{2.532369in}}%
\pgfpathlineto{\pgfqpoint{5.523616in}{2.342686in}}%
\pgfpathlineto{\pgfqpoint{5.524127in}{2.489923in}}%
\pgfpathlineto{\pgfqpoint{5.524277in}{2.076824in}}%
\pgfpathlineto{\pgfqpoint{5.524728in}{2.431610in}}%
\pgfpathlineto{\pgfqpoint{5.525557in}{2.533747in}}%
\pgfpathlineto{\pgfqpoint{5.524780in}{1.966027in}}%
\pgfpathlineto{\pgfqpoint{5.525772in}{2.398423in}}%
\pgfpathlineto{\pgfqpoint{5.526463in}{2.061880in}}%
\pgfpathlineto{\pgfqpoint{5.526111in}{2.511317in}}%
\pgfpathlineto{\pgfqpoint{5.526879in}{2.449712in}}%
\pgfpathlineto{\pgfqpoint{5.527938in}{2.517420in}}%
\pgfpathlineto{\pgfqpoint{5.527529in}{2.071415in}}%
\pgfpathlineto{\pgfqpoint{5.527951in}{2.347500in}}%
\pgfpathlineto{\pgfqpoint{5.528126in}{1.923500in}}%
\pgfpathlineto{\pgfqpoint{5.528223in}{2.548011in}}%
\pgfpathlineto{\pgfqpoint{5.529064in}{2.179762in}}%
\pgfpathlineto{\pgfqpoint{5.529827in}{2.518568in}}%
\pgfpathlineto{\pgfqpoint{5.530137in}{1.951344in}}%
\pgfpathlineto{\pgfqpoint{5.530182in}{2.369985in}}%
\pgfpathlineto{\pgfqpoint{5.530556in}{1.926131in}}%
\pgfpathlineto{\pgfqpoint{5.530337in}{2.522954in}}%
\pgfpathlineto{\pgfqpoint{5.531258in}{2.314213in}}%
\pgfpathlineto{\pgfqpoint{5.531914in}{2.532371in}}%
\pgfpathlineto{\pgfqpoint{5.532190in}{2.037020in}}%
\pgfpathlineto{\pgfqpoint{5.532357in}{2.426993in}}%
\pgfpathlineto{\pgfqpoint{5.532363in}{2.071716in}}%
\pgfpathlineto{\pgfqpoint{5.532472in}{2.534478in}}%
\pgfpathlineto{\pgfqpoint{5.533466in}{2.346654in}}%
\pgfpathlineto{\pgfqpoint{5.533472in}{2.518362in}}%
\pgfpathlineto{\pgfqpoint{5.534323in}{2.089916in}}%
\pgfpathlineto{\pgfqpoint{5.534572in}{2.373343in}}%
\pgfpathlineto{\pgfqpoint{5.535076in}{2.051639in}}%
\pgfpathlineto{\pgfqpoint{5.535152in}{2.524614in}}%
\pgfpathlineto{\pgfqpoint{5.535681in}{2.275689in}}%
\pgfpathlineto{\pgfqpoint{5.536775in}{2.530859in}}%
\pgfpathlineto{\pgfqpoint{5.536318in}{1.978279in}}%
\pgfpathlineto{\pgfqpoint{5.536794in}{2.361065in}}%
\pgfpathlineto{\pgfqpoint{5.536997in}{1.977161in}}%
\pgfpathlineto{\pgfqpoint{5.536820in}{2.536690in}}%
\pgfpathlineto{\pgfqpoint{5.537898in}{2.377825in}}%
\pgfpathlineto{\pgfqpoint{5.538973in}{2.543960in}}%
\pgfpathlineto{\pgfqpoint{5.538910in}{2.062569in}}%
\pgfpathlineto{\pgfqpoint{5.539011in}{2.449414in}}%
\pgfpathlineto{\pgfqpoint{5.539983in}{2.015357in}}%
\pgfpathlineto{\pgfqpoint{5.539220in}{2.524983in}}%
\pgfpathlineto{\pgfqpoint{5.540128in}{2.324327in}}%
\pgfpathlineto{\pgfqpoint{5.540871in}{2.523663in}}%
\pgfpathlineto{\pgfqpoint{5.540204in}{1.901070in}}%
\pgfpathlineto{\pgfqpoint{5.541236in}{2.399398in}}%
\pgfpathlineto{\pgfqpoint{5.541273in}{2.092324in}}%
\pgfpathlineto{\pgfqpoint{5.541996in}{2.535522in}}%
\pgfpathlineto{\pgfqpoint{5.542347in}{2.346522in}}%
\pgfpathlineto{\pgfqpoint{5.542917in}{1.942594in}}%
\pgfpathlineto{\pgfqpoint{5.542823in}{2.525820in}}%
\pgfpathlineto{\pgfqpoint{5.543449in}{2.261564in}}%
\pgfpathlineto{\pgfqpoint{5.543605in}{2.531037in}}%
\pgfpathlineto{\pgfqpoint{5.543580in}{1.957535in}}%
\pgfpathlineto{\pgfqpoint{5.544560in}{2.331277in}}%
\pgfpathlineto{\pgfqpoint{5.544729in}{2.530132in}}%
\pgfpathlineto{\pgfqpoint{5.545152in}{2.018956in}}%
\pgfpathlineto{\pgfqpoint{5.545663in}{2.432953in}}%
\pgfpathlineto{\pgfqpoint{5.545669in}{1.899068in}}%
\pgfpathlineto{\pgfqpoint{5.545737in}{2.522360in}}%
\pgfpathlineto{\pgfqpoint{5.546768in}{2.501101in}}%
\pgfpathlineto{\pgfqpoint{5.547723in}{2.054849in}}%
\pgfpathlineto{\pgfqpoint{5.547493in}{2.534999in}}%
\pgfpathlineto{\pgfqpoint{5.547883in}{2.422486in}}%
\pgfpathlineto{\pgfqpoint{5.548990in}{2.005990in}}%
\pgfpathlineto{\pgfqpoint{5.548557in}{2.530469in}}%
\pgfpathlineto{\pgfqpoint{5.549002in}{2.383404in}}%
\pgfpathlineto{\pgfqpoint{5.549606in}{2.538526in}}%
\pgfpathlineto{\pgfqpoint{5.549964in}{2.030887in}}%
\pgfpathlineto{\pgfqpoint{5.550111in}{2.438384in}}%
\pgfpathlineto{\pgfqpoint{5.550720in}{2.056287in}}%
\pgfpathlineto{\pgfqpoint{5.550253in}{2.530999in}}%
\pgfpathlineto{\pgfqpoint{5.551224in}{2.311713in}}%
\pgfpathlineto{\pgfqpoint{5.551752in}{2.508418in}}%
\pgfpathlineto{\pgfqpoint{5.551770in}{1.998470in}}%
\pgfpathlineto{\pgfqpoint{5.552334in}{2.364811in}}%
\pgfpathlineto{\pgfqpoint{5.553110in}{1.882465in}}%
\pgfpathlineto{\pgfqpoint{5.552646in}{2.528801in}}%
\pgfpathlineto{\pgfqpoint{5.553428in}{2.356590in}}%
\pgfpathlineto{\pgfqpoint{5.554252in}{2.513575in}}%
\pgfpathlineto{\pgfqpoint{5.553508in}{2.034417in}}%
\pgfpathlineto{\pgfqpoint{5.554538in}{2.466687in}}%
\pgfpathlineto{\pgfqpoint{5.555493in}{2.020596in}}%
\pgfpathlineto{\pgfqpoint{5.555548in}{2.509018in}}%
\pgfpathlineto{\pgfqpoint{5.555658in}{2.264281in}}%
\pgfpathlineto{\pgfqpoint{5.556350in}{2.506016in}}%
\pgfpathlineto{\pgfqpoint{5.555864in}{1.691982in}}%
\pgfpathlineto{\pgfqpoint{5.556768in}{2.422549in}}%
\pgfpathlineto{\pgfqpoint{5.557343in}{2.080757in}}%
\pgfpathlineto{\pgfqpoint{5.557107in}{2.491095in}}%
\pgfpathlineto{\pgfqpoint{5.557881in}{2.194723in}}%
\pgfpathlineto{\pgfqpoint{5.558208in}{2.516061in}}%
\pgfpathlineto{\pgfqpoint{5.558407in}{2.128167in}}%
\pgfpathlineto{\pgfqpoint{5.558992in}{2.374407in}}%
\pgfpathlineto{\pgfqpoint{5.559985in}{1.893152in}}%
\pgfpathlineto{\pgfqpoint{5.559486in}{2.511271in}}%
\pgfpathlineto{\pgfqpoint{5.560099in}{2.413322in}}%
\pgfpathlineto{\pgfqpoint{5.560376in}{2.524319in}}%
\pgfpathlineto{\pgfqpoint{5.560244in}{1.899373in}}%
\pgfpathlineto{\pgfqpoint{5.561168in}{2.328214in}}%
\pgfpathlineto{\pgfqpoint{5.561324in}{1.977832in}}%
\pgfpathlineto{\pgfqpoint{5.561306in}{2.514005in}}%
\pgfpathlineto{\pgfqpoint{5.562276in}{2.366920in}}%
\pgfpathlineto{\pgfqpoint{5.563268in}{2.512618in}}%
\pgfpathlineto{\pgfqpoint{5.563351in}{1.836095in}}%
\pgfpathlineto{\pgfqpoint{5.563381in}{2.438464in}}%
\pgfpathlineto{\pgfqpoint{5.564382in}{1.963061in}}%
\pgfpathlineto{\pgfqpoint{5.563697in}{2.525822in}}%
\pgfpathlineto{\pgfqpoint{5.564495in}{2.315919in}}%
\pgfpathlineto{\pgfqpoint{5.565393in}{2.555949in}}%
\pgfpathlineto{\pgfqpoint{5.564608in}{2.017681in}}%
\pgfpathlineto{\pgfqpoint{5.565595in}{2.418819in}}%
\pgfpathlineto{\pgfqpoint{5.566614in}{1.922197in}}%
\pgfpathlineto{\pgfqpoint{5.565707in}{2.512469in}}%
\pgfpathlineto{\pgfqpoint{5.566703in}{2.408738in}}%
\pgfpathlineto{\pgfqpoint{5.567761in}{2.052872in}}%
\pgfpathlineto{\pgfqpoint{5.567052in}{2.524708in}}%
\pgfpathlineto{\pgfqpoint{5.567797in}{2.263235in}}%
\pgfpathlineto{\pgfqpoint{5.568428in}{2.522766in}}%
\pgfpathlineto{\pgfqpoint{5.568339in}{1.977824in}}%
\pgfpathlineto{\pgfqpoint{5.568911in}{2.380719in}}%
\pgfpathlineto{\pgfqpoint{5.569899in}{2.064331in}}%
\pgfpathlineto{\pgfqpoint{5.569146in}{2.541050in}}%
\pgfpathlineto{\pgfqpoint{5.569981in}{2.419308in}}%
\pgfpathlineto{\pgfqpoint{5.570216in}{2.505352in}}%
\pgfpathlineto{\pgfqpoint{5.570427in}{2.075423in}}%
\pgfpathlineto{\pgfqpoint{5.571055in}{2.440772in}}%
\pgfpathlineto{\pgfqpoint{5.571857in}{2.016566in}}%
\pgfpathlineto{\pgfqpoint{5.571365in}{2.524042in}}%
\pgfpathlineto{\pgfqpoint{5.572166in}{2.390664in}}%
\pgfpathlineto{\pgfqpoint{5.572470in}{2.493744in}}%
\pgfpathlineto{\pgfqpoint{5.572540in}{2.068025in}}%
\pgfpathlineto{\pgfqpoint{5.572838in}{2.354910in}}%
\pgfpathlineto{\pgfqpoint{5.572844in}{1.978011in}}%
\pgfpathlineto{\pgfqpoint{5.573118in}{2.525378in}}%
\pgfpathlineto{\pgfqpoint{5.573945in}{2.409534in}}%
\pgfpathlineto{\pgfqpoint{5.574544in}{1.968838in}}%
\pgfpathlineto{\pgfqpoint{5.574410in}{2.518741in}}%
\pgfpathlineto{\pgfqpoint{5.575061in}{2.198362in}}%
\pgfpathlineto{\pgfqpoint{5.575096in}{2.529884in}}%
\pgfpathlineto{\pgfqpoint{5.575293in}{2.079543in}}%
\pgfpathlineto{\pgfqpoint{5.576174in}{2.422188in}}%
\pgfpathlineto{\pgfqpoint{5.576660in}{1.987817in}}%
\pgfpathlineto{\pgfqpoint{5.576596in}{2.528065in}}%
\pgfpathlineto{\pgfqpoint{5.577289in}{2.326383in}}%
\pgfpathlineto{\pgfqpoint{5.577370in}{2.518264in}}%
\pgfpathlineto{\pgfqpoint{5.577878in}{1.876998in}}%
\pgfpathlineto{\pgfqpoint{5.578385in}{2.341423in}}%
\pgfpathlineto{\pgfqpoint{5.578943in}{1.741020in}}%
\pgfpathlineto{\pgfqpoint{5.578862in}{2.519890in}}%
\pgfpathlineto{\pgfqpoint{5.579495in}{2.359239in}}%
\pgfpathlineto{\pgfqpoint{5.580063in}{1.814335in}}%
\pgfpathlineto{\pgfqpoint{5.579541in}{2.519232in}}%
\pgfpathlineto{\pgfqpoint{5.580596in}{2.473093in}}%
\pgfpathlineto{\pgfqpoint{5.581449in}{2.058471in}}%
\pgfpathlineto{\pgfqpoint{5.581266in}{2.539668in}}%
\pgfpathlineto{\pgfqpoint{5.581740in}{2.149012in}}%
\pgfpathlineto{\pgfqpoint{5.582288in}{2.523319in}}%
\pgfpathlineto{\pgfqpoint{5.581849in}{2.091440in}}%
\pgfpathlineto{\pgfqpoint{5.582853in}{2.365700in}}%
\pgfpathlineto{\pgfqpoint{5.583109in}{1.834896in}}%
\pgfpathlineto{\pgfqpoint{5.583468in}{2.523194in}}%
\pgfpathlineto{\pgfqpoint{5.583968in}{2.148434in}}%
\pgfpathlineto{\pgfqpoint{5.584422in}{2.508615in}}%
\pgfpathlineto{\pgfqpoint{5.585069in}{1.894538in}}%
\pgfpathlineto{\pgfqpoint{5.585081in}{2.488309in}}%
\pgfpathlineto{\pgfqpoint{5.585086in}{1.810745in}}%
\pgfpathlineto{\pgfqpoint{5.585743in}{2.534172in}}%
\pgfpathlineto{\pgfqpoint{5.586190in}{2.299883in}}%
\pgfpathlineto{\pgfqpoint{5.586919in}{2.537061in}}%
\pgfpathlineto{\pgfqpoint{5.586552in}{2.116374in}}%
\pgfpathlineto{\pgfqpoint{5.587291in}{2.479146in}}%
\pgfpathlineto{\pgfqpoint{5.587347in}{2.013831in}}%
\pgfpathlineto{\pgfqpoint{5.588372in}{2.514381in}}%
\pgfpathlineto{\pgfqpoint{5.588400in}{2.473162in}}%
\pgfpathlineto{\pgfqpoint{5.588940in}{1.954251in}}%
\pgfpathlineto{\pgfqpoint{5.589114in}{2.510107in}}%
\pgfpathlineto{\pgfqpoint{5.589512in}{2.440023in}}%
\pgfpathlineto{\pgfqpoint{5.590039in}{1.973259in}}%
\pgfpathlineto{\pgfqpoint{5.589972in}{2.503553in}}%
\pgfpathlineto{\pgfqpoint{5.590622in}{2.173824in}}%
\pgfpathlineto{\pgfqpoint{5.590806in}{2.508737in}}%
\pgfpathlineto{\pgfqpoint{5.590773in}{2.025288in}}%
\pgfpathlineto{\pgfqpoint{5.591734in}{2.388260in}}%
\pgfpathlineto{\pgfqpoint{5.592018in}{2.552383in}}%
\pgfpathlineto{\pgfqpoint{5.591962in}{2.018853in}}%
\pgfpathlineto{\pgfqpoint{5.592837in}{2.297269in}}%
\pgfpathlineto{\pgfqpoint{5.592909in}{2.219342in}}%
\pgfpathlineto{\pgfqpoint{5.592915in}{2.413085in}}%
\pgfpathlineto{\pgfqpoint{5.592932in}{2.361246in}}%
\pgfpathlineto{\pgfqpoint{5.593260in}{2.547184in}}%
\pgfpathlineto{\pgfqpoint{5.593838in}{1.984719in}}%
\pgfpathlineto{\pgfqpoint{5.594043in}{2.396783in}}%
\pgfpathlineto{\pgfqpoint{5.594381in}{1.849954in}}%
\pgfpathlineto{\pgfqpoint{5.594282in}{2.524221in}}%
\pgfpathlineto{\pgfqpoint{5.595152in}{2.368698in}}%
\pgfpathlineto{\pgfqpoint{5.595765in}{2.531502in}}%
\pgfpathlineto{\pgfqpoint{5.595810in}{1.838251in}}%
\pgfpathlineto{\pgfqpoint{5.596257in}{2.299917in}}%
\pgfpathlineto{\pgfqpoint{5.596952in}{2.037979in}}%
\pgfpathlineto{\pgfqpoint{5.597338in}{2.510285in}}%
\pgfpathlineto{\pgfqpoint{5.597360in}{2.266454in}}%
\pgfpathlineto{\pgfqpoint{5.598465in}{2.499033in}}%
\pgfpathlineto{\pgfqpoint{5.598174in}{2.103401in}}%
\pgfpathlineto{\pgfqpoint{5.598471in}{2.381547in}}%
\pgfpathlineto{\pgfqpoint{5.598553in}{2.457575in}}%
\pgfpathlineto{\pgfqpoint{5.598504in}{2.273956in}}%
\pgfpathlineto{\pgfqpoint{5.598570in}{2.303519in}}%
\pgfpathlineto{\pgfqpoint{5.598762in}{2.006141in}}%
\pgfpathlineto{\pgfqpoint{5.599546in}{2.497512in}}%
\pgfpathlineto{\pgfqpoint{5.599672in}{2.291328in}}%
\pgfpathlineto{\pgfqpoint{5.599863in}{2.500595in}}%
\pgfpathlineto{\pgfqpoint{5.600192in}{2.023919in}}%
\pgfpathlineto{\pgfqpoint{5.600766in}{2.403047in}}%
\pgfpathlineto{\pgfqpoint{5.601361in}{2.020649in}}%
\pgfpathlineto{\pgfqpoint{5.601830in}{2.524934in}}%
\pgfpathlineto{\pgfqpoint{5.601879in}{2.268867in}}%
\pgfpathlineto{\pgfqpoint{5.602347in}{2.502180in}}%
\pgfpathlineto{\pgfqpoint{5.601911in}{1.839967in}}%
\pgfpathlineto{\pgfqpoint{5.602989in}{2.372140in}}%
\pgfpathlineto{\pgfqpoint{5.603662in}{1.859961in}}%
\pgfpathlineto{\pgfqpoint{5.603684in}{2.535828in}}%
\pgfpathlineto{\pgfqpoint{5.604090in}{2.288808in}}%
\pgfpathlineto{\pgfqpoint{5.604470in}{2.533335in}}%
\pgfpathlineto{\pgfqpoint{5.604394in}{1.847322in}}%
\pgfpathlineto{\pgfqpoint{5.605200in}{2.239724in}}%
\pgfpathlineto{\pgfqpoint{5.605956in}{2.537081in}}%
\pgfpathlineto{\pgfqpoint{5.605702in}{2.023519in}}%
\pgfpathlineto{\pgfqpoint{5.606317in}{2.459036in}}%
\pgfpathlineto{\pgfqpoint{5.606366in}{2.553451in}}%
\pgfpathlineto{\pgfqpoint{5.607088in}{1.945623in}}%
\pgfpathlineto{\pgfqpoint{5.607351in}{2.400158in}}%
\pgfpathlineto{\pgfqpoint{5.607642in}{2.055167in}}%
\pgfpathlineto{\pgfqpoint{5.608163in}{2.519919in}}%
\pgfpathlineto{\pgfqpoint{5.608458in}{2.416673in}}%
\pgfpathlineto{\pgfqpoint{5.609347in}{2.507275in}}%
\pgfpathlineto{\pgfqpoint{5.609214in}{1.942871in}}%
\pgfpathlineto{\pgfqpoint{5.609561in}{2.423625in}}%
\pgfpathlineto{\pgfqpoint{5.610406in}{1.883224in}}%
\pgfpathlineto{\pgfqpoint{5.610086in}{2.485243in}}%
\pgfpathlineto{\pgfqpoint{5.610678in}{2.353691in}}%
\pgfpathlineto{\pgfqpoint{5.611212in}{2.522596in}}%
\pgfpathlineto{\pgfqpoint{5.611409in}{1.881578in}}%
\pgfpathlineto{\pgfqpoint{5.611781in}{2.355166in}}%
\pgfpathlineto{\pgfqpoint{5.611845in}{2.037037in}}%
\pgfpathlineto{\pgfqpoint{5.611819in}{2.514842in}}%
\pgfpathlineto{\pgfqpoint{5.612892in}{2.282095in}}%
\pgfpathlineto{\pgfqpoint{5.613773in}{2.522486in}}%
\pgfpathlineto{\pgfqpoint{5.613487in}{2.017765in}}%
\pgfpathlineto{\pgfqpoint{5.614006in}{2.413283in}}%
\pgfpathlineto{\pgfqpoint{5.614424in}{1.881444in}}%
\pgfpathlineto{\pgfqpoint{5.615085in}{2.538171in}}%
\pgfpathlineto{\pgfqpoint{5.615111in}{2.400554in}}%
\pgfpathlineto{\pgfqpoint{5.615174in}{2.487044in}}%
\pgfpathlineto{\pgfqpoint{5.616045in}{1.912119in}}%
\pgfpathlineto{\pgfqpoint{5.616219in}{2.409370in}}%
\pgfpathlineto{\pgfqpoint{5.616229in}{2.446413in}}%
\pgfpathlineto{\pgfqpoint{5.616234in}{2.421004in}}%
\pgfpathlineto{\pgfqpoint{5.616287in}{1.962002in}}%
\pgfpathlineto{\pgfqpoint{5.616655in}{2.531078in}}%
\pgfpathlineto{\pgfqpoint{5.617344in}{2.297025in}}%
\pgfpathlineto{\pgfqpoint{5.617979in}{2.517984in}}%
\pgfpathlineto{\pgfqpoint{5.618320in}{1.895115in}}%
\pgfpathlineto{\pgfqpoint{5.618456in}{2.351068in}}%
\pgfpathlineto{\pgfqpoint{5.618624in}{2.058409in}}%
\pgfpathlineto{\pgfqpoint{5.618645in}{2.492006in}}%
\pgfpathlineto{\pgfqpoint{5.619555in}{2.311039in}}%
\pgfpathlineto{\pgfqpoint{5.619770in}{2.524249in}}%
\pgfpathlineto{\pgfqpoint{5.620198in}{2.030950in}}%
\pgfpathlineto{\pgfqpoint{5.620662in}{2.481639in}}%
\pgfpathlineto{\pgfqpoint{5.621292in}{2.008734in}}%
\pgfpathlineto{\pgfqpoint{5.621099in}{2.517681in}}%
\pgfpathlineto{\pgfqpoint{5.621771in}{2.409331in}}%
\pgfpathlineto{\pgfqpoint{5.622700in}{2.046143in}}%
\pgfpathlineto{\pgfqpoint{5.622108in}{2.528791in}}%
\pgfpathlineto{\pgfqpoint{5.622887in}{2.364089in}}%
\pgfpathlineto{\pgfqpoint{5.623415in}{1.982861in}}%
\pgfpathlineto{\pgfqpoint{5.623793in}{2.507862in}}%
\pgfpathlineto{\pgfqpoint{5.623969in}{2.423287in}}%
\pgfpathlineto{\pgfqpoint{5.624119in}{2.511297in}}%
\pgfpathlineto{\pgfqpoint{5.624951in}{2.087323in}}%
\pgfpathlineto{\pgfqpoint{5.625064in}{2.311344in}}%
\pgfpathlineto{\pgfqpoint{5.625183in}{1.960727in}}%
\pgfpathlineto{\pgfqpoint{5.625585in}{2.539233in}}%
\pgfpathlineto{\pgfqpoint{5.626064in}{2.324487in}}%
\pgfpathlineto{\pgfqpoint{5.626665in}{2.520847in}}%
\pgfpathlineto{\pgfqpoint{5.626933in}{1.970234in}}%
\pgfpathlineto{\pgfqpoint{5.627174in}{2.335486in}}%
\pgfpathlineto{\pgfqpoint{5.627769in}{2.511532in}}%
\pgfpathlineto{\pgfqpoint{5.627953in}{1.901653in}}%
\pgfpathlineto{\pgfqpoint{5.628281in}{2.290675in}}%
\pgfpathlineto{\pgfqpoint{5.629079in}{1.810410in}}%
\pgfpathlineto{\pgfqpoint{5.628895in}{2.531154in}}%
\pgfpathlineto{\pgfqpoint{5.629304in}{2.352443in}}%
\pgfpathlineto{\pgfqpoint{5.629528in}{2.543988in}}%
\pgfpathlineto{\pgfqpoint{5.629906in}{1.832685in}}%
\pgfpathlineto{\pgfqpoint{5.630406in}{2.478399in}}%
\pgfpathlineto{\pgfqpoint{5.630451in}{1.843701in}}%
\pgfpathlineto{\pgfqpoint{5.630513in}{2.502250in}}%
\pgfpathlineto{\pgfqpoint{5.631515in}{2.189772in}}%
\pgfpathlineto{\pgfqpoint{5.631535in}{2.504176in}}%
\pgfpathlineto{\pgfqpoint{5.632195in}{1.829890in}}%
\pgfpathlineto{\pgfqpoint{5.632626in}{2.363451in}}%
\pgfpathlineto{\pgfqpoint{5.632884in}{2.519145in}}%
\pgfpathlineto{\pgfqpoint{5.633653in}{1.892081in}}%
\pgfpathlineto{\pgfqpoint{5.633729in}{2.403731in}}%
\pgfpathlineto{\pgfqpoint{5.634446in}{2.006010in}}%
\pgfpathlineto{\pgfqpoint{5.633759in}{2.503926in}}%
\pgfpathlineto{\pgfqpoint{5.634845in}{2.252130in}}%
\pgfpathlineto{\pgfqpoint{5.634850in}{2.252243in}}%
\pgfpathlineto{\pgfqpoint{5.635399in}{2.525862in}}%
\pgfpathlineto{\pgfqpoint{5.635489in}{1.863431in}}%
\pgfpathlineto{\pgfqpoint{5.635962in}{2.347508in}}%
\pgfpathlineto{\pgfqpoint{5.637002in}{2.526255in}}%
\pgfpathlineto{\pgfqpoint{5.636982in}{1.969072in}}%
\pgfpathlineto{\pgfqpoint{5.637067in}{2.419415in}}%
\pgfpathlineto{\pgfqpoint{5.637262in}{1.894970in}}%
\pgfpathlineto{\pgfqpoint{5.637603in}{2.528405in}}%
\pgfpathlineto{\pgfqpoint{5.638179in}{2.097837in}}%
\pgfpathlineto{\pgfqpoint{5.638204in}{2.514844in}}%
\pgfpathlineto{\pgfqpoint{5.638629in}{1.927229in}}%
\pgfpathlineto{\pgfqpoint{5.639308in}{2.344463in}}%
\pgfpathlineto{\pgfqpoint{5.640349in}{1.914178in}}%
\pgfpathlineto{\pgfqpoint{5.640060in}{2.502029in}}%
\pgfpathlineto{\pgfqpoint{5.640414in}{2.230198in}}%
\pgfpathlineto{\pgfqpoint{5.640881in}{2.499185in}}%
\pgfpathlineto{\pgfqpoint{5.640732in}{1.779222in}}%
\pgfpathlineto{\pgfqpoint{5.641527in}{2.420497in}}%
\pgfpathlineto{\pgfqpoint{5.642488in}{2.012506in}}%
\pgfpathlineto{\pgfqpoint{5.642439in}{2.521010in}}%
\pgfpathlineto{\pgfqpoint{5.642637in}{2.370584in}}%
\pgfpathlineto{\pgfqpoint{5.643047in}{1.667134in}}%
\pgfpathlineto{\pgfqpoint{5.643102in}{2.525154in}}%
\pgfpathlineto{\pgfqpoint{5.643744in}{2.327196in}}%
\pgfpathlineto{\pgfqpoint{5.644587in}{2.485089in}}%
\pgfpathlineto{\pgfqpoint{5.644755in}{2.009432in}}%
\pgfpathlineto{\pgfqpoint{5.644838in}{2.376881in}}%
\pgfpathlineto{\pgfqpoint{5.645645in}{1.878526in}}%
\pgfpathlineto{\pgfqpoint{5.645842in}{2.525602in}}%
\pgfpathlineto{\pgfqpoint{5.645945in}{2.270979in}}%
\pgfpathlineto{\pgfqpoint{5.646906in}{2.492841in}}%
\pgfpathlineto{\pgfqpoint{5.646107in}{1.968333in}}%
\pgfpathlineto{\pgfqpoint{5.647058in}{2.390508in}}%
\pgfpathlineto{\pgfqpoint{5.647489in}{2.061011in}}%
\pgfpathlineto{\pgfqpoint{5.647802in}{2.531007in}}%
\pgfpathlineto{\pgfqpoint{5.648168in}{2.320329in}}%
\pgfpathlineto{\pgfqpoint{5.649213in}{1.894727in}}%
\pgfpathlineto{\pgfqpoint{5.648378in}{2.521941in}}%
\pgfpathlineto{\pgfqpoint{5.649281in}{2.279803in}}%
\pgfpathlineto{\pgfqpoint{5.649714in}{2.526785in}}%
\pgfpathlineto{\pgfqpoint{5.649480in}{1.980517in}}%
\pgfpathlineto{\pgfqpoint{5.650390in}{2.463113in}}%
\pgfpathlineto{\pgfqpoint{5.651351in}{1.877585in}}%
\pgfpathlineto{\pgfqpoint{5.650536in}{2.485864in}}%
\pgfpathlineto{\pgfqpoint{5.651502in}{2.321929in}}%
\pgfpathlineto{\pgfqpoint{5.652538in}{2.487777in}}%
\pgfpathlineto{\pgfqpoint{5.652204in}{2.019183in}}%
\pgfpathlineto{\pgfqpoint{5.652610in}{2.334242in}}%
\pgfpathlineto{\pgfqpoint{5.653368in}{2.091286in}}%
\pgfpathlineto{\pgfqpoint{5.653320in}{2.506717in}}%
\pgfpathlineto{\pgfqpoint{5.653721in}{2.306877in}}%
\pgfpathlineto{\pgfqpoint{5.654814in}{2.493425in}}%
\pgfpathlineto{\pgfqpoint{5.654323in}{1.929500in}}%
\pgfpathlineto{\pgfqpoint{5.654828in}{2.296337in}}%
\pgfpathlineto{\pgfqpoint{5.655784in}{2.493731in}}%
\pgfpathlineto{\pgfqpoint{5.655328in}{1.912507in}}%
\pgfpathlineto{\pgfqpoint{5.655928in}{2.407328in}}%
\pgfpathlineto{\pgfqpoint{5.656733in}{1.952584in}}%
\pgfpathlineto{\pgfqpoint{5.656407in}{2.517215in}}%
\pgfpathlineto{\pgfqpoint{5.657035in}{2.305888in}}%
\pgfpathlineto{\pgfqpoint{5.657259in}{2.505570in}}%
\pgfpathlineto{\pgfqpoint{5.658043in}{1.984372in}}%
\pgfpathlineto{\pgfqpoint{5.658143in}{2.361261in}}%
\pgfpathlineto{\pgfqpoint{5.658758in}{2.039015in}}%
\pgfpathlineto{\pgfqpoint{5.659111in}{2.521664in}}%
\pgfpathlineto{\pgfqpoint{5.659253in}{2.127048in}}%
\pgfpathlineto{\pgfqpoint{5.660085in}{2.535010in}}%
\pgfpathlineto{\pgfqpoint{5.659929in}{1.948164in}}%
\pgfpathlineto{\pgfqpoint{5.660366in}{2.477193in}}%
\pgfpathlineto{\pgfqpoint{5.660480in}{2.498962in}}%
\pgfpathlineto{\pgfqpoint{5.661480in}{1.960950in}}%
\pgfpathlineto{\pgfqpoint{5.661655in}{2.510450in}}%
\pgfpathlineto{\pgfqpoint{5.662591in}{2.285032in}}%
\pgfpathlineto{\pgfqpoint{5.662822in}{1.825237in}}%
\pgfpathlineto{\pgfqpoint{5.663492in}{2.504780in}}%
\pgfpathlineto{\pgfqpoint{5.663694in}{2.253361in}}%
\pgfpathlineto{\pgfqpoint{5.664682in}{2.509309in}}%
\pgfpathlineto{\pgfqpoint{5.663774in}{2.037878in}}%
\pgfpathlineto{\pgfqpoint{5.664804in}{2.351044in}}%
\pgfpathlineto{\pgfqpoint{5.664828in}{2.510062in}}%
\pgfpathlineto{\pgfqpoint{5.664818in}{2.067238in}}%
\pgfpathlineto{\pgfqpoint{5.665640in}{2.330680in}}%
\pgfpathlineto{\pgfqpoint{5.665869in}{1.905343in}}%
\pgfpathlineto{\pgfqpoint{5.665729in}{2.501384in}}%
\pgfpathlineto{\pgfqpoint{5.666749in}{2.341344in}}%
\pgfpathlineto{\pgfqpoint{5.667287in}{2.538282in}}%
\pgfpathlineto{\pgfqpoint{5.667207in}{1.823603in}}%
\pgfpathlineto{\pgfqpoint{5.667856in}{2.309056in}}%
\pgfpathlineto{\pgfqpoint{5.668811in}{1.763796in}}%
\pgfpathlineto{\pgfqpoint{5.668816in}{2.512727in}}%
\pgfpathlineto{\pgfqpoint{5.668969in}{2.202262in}}%
\pgfpathlineto{\pgfqpoint{5.669690in}{2.529739in}}%
\pgfpathlineto{\pgfqpoint{5.669564in}{1.911374in}}%
\pgfpathlineto{\pgfqpoint{5.670079in}{2.266478in}}%
\pgfpathlineto{\pgfqpoint{5.670576in}{2.497628in}}%
\pgfpathlineto{\pgfqpoint{5.670830in}{1.649066in}}%
\pgfpathlineto{\pgfqpoint{5.671196in}{2.383801in}}%
\pgfpathlineto{\pgfqpoint{5.671464in}{1.854571in}}%
\pgfpathlineto{\pgfqpoint{5.671589in}{2.525319in}}%
\pgfpathlineto{\pgfqpoint{5.672305in}{2.251094in}}%
\pgfpathlineto{\pgfqpoint{5.672356in}{2.499047in}}%
\pgfpathlineto{\pgfqpoint{5.672582in}{1.963802in}}%
\pgfpathlineto{\pgfqpoint{5.673420in}{2.498316in}}%
\pgfpathlineto{\pgfqpoint{5.673540in}{2.008836in}}%
\pgfpathlineto{\pgfqpoint{5.674271in}{2.550158in}}%
\pgfpathlineto{\pgfqpoint{5.674533in}{2.422453in}}%
\pgfpathlineto{\pgfqpoint{5.675546in}{1.933260in}}%
\pgfpathlineto{\pgfqpoint{5.675271in}{2.514964in}}%
\pgfpathlineto{\pgfqpoint{5.675652in}{2.307939in}}%
\pgfpathlineto{\pgfqpoint{5.676740in}{2.513182in}}%
\pgfpathlineto{\pgfqpoint{5.676123in}{1.980291in}}%
\pgfpathlineto{\pgfqpoint{5.676763in}{2.331741in}}%
\pgfpathlineto{\pgfqpoint{5.677502in}{2.500672in}}%
\pgfpathlineto{\pgfqpoint{5.676868in}{1.899011in}}%
\pgfpathlineto{\pgfqpoint{5.677867in}{2.421851in}}%
\pgfpathlineto{\pgfqpoint{5.678358in}{1.900297in}}%
\pgfpathlineto{\pgfqpoint{5.678336in}{2.523818in}}%
\pgfpathlineto{\pgfqpoint{5.678981in}{2.245274in}}%
\pgfpathlineto{\pgfqpoint{5.679381in}{2.508733in}}%
\pgfpathlineto{\pgfqpoint{5.679022in}{1.966019in}}%
\pgfpathlineto{\pgfqpoint{5.680093in}{2.318288in}}%
\pgfpathlineto{\pgfqpoint{5.680948in}{2.036410in}}%
\pgfpathlineto{\pgfqpoint{5.680591in}{2.512467in}}%
\pgfpathlineto{\pgfqpoint{5.681192in}{2.319846in}}%
\pgfpathlineto{\pgfqpoint{5.681883in}{2.521438in}}%
\pgfpathlineto{\pgfqpoint{5.681685in}{1.845250in}}%
\pgfpathlineto{\pgfqpoint{5.682303in}{2.491835in}}%
\pgfpathlineto{\pgfqpoint{5.683068in}{1.964516in}}%
\pgfpathlineto{\pgfqpoint{5.682415in}{2.531671in}}%
\pgfpathlineto{\pgfqpoint{5.683415in}{2.450393in}}%
\pgfpathlineto{\pgfqpoint{5.683819in}{1.762517in}}%
\pgfpathlineto{\pgfqpoint{5.684344in}{2.506970in}}%
\pgfpathlineto{\pgfqpoint{5.684546in}{2.263233in}}%
\pgfpathlineto{\pgfqpoint{5.685509in}{2.501691in}}%
\pgfpathlineto{\pgfqpoint{5.684972in}{2.040765in}}%
\pgfpathlineto{\pgfqpoint{5.685661in}{2.452928in}}%
\pgfpathlineto{\pgfqpoint{5.686407in}{2.037722in}}%
\pgfpathlineto{\pgfqpoint{5.685970in}{2.522086in}}%
\pgfpathlineto{\pgfqpoint{5.686778in}{2.163350in}}%
\pgfpathlineto{\pgfqpoint{5.687873in}{2.492246in}}%
\pgfpathlineto{\pgfqpoint{5.686898in}{1.983654in}}%
\pgfpathlineto{\pgfqpoint{5.687891in}{2.454530in}}%
\pgfpathlineto{\pgfqpoint{5.688545in}{1.808675in}}%
\pgfpathlineto{\pgfqpoint{5.688638in}{2.481760in}}%
\pgfpathlineto{\pgfqpoint{5.689002in}{2.299553in}}%
\pgfpathlineto{\pgfqpoint{5.689707in}{2.490778in}}%
\pgfpathlineto{\pgfqpoint{5.689547in}{1.839981in}}%
\pgfpathlineto{\pgfqpoint{5.690114in}{2.367675in}}%
\pgfpathlineto{\pgfqpoint{5.690782in}{2.087459in}}%
\pgfpathlineto{\pgfqpoint{5.691210in}{2.506011in}}%
\pgfpathlineto{\pgfqpoint{5.691224in}{2.388592in}}%
\pgfpathlineto{\pgfqpoint{5.691563in}{1.957366in}}%
\pgfpathlineto{\pgfqpoint{5.692299in}{2.515344in}}%
\pgfpathlineto{\pgfqpoint{5.692334in}{2.375313in}}%
\pgfpathlineto{\pgfqpoint{5.692730in}{2.525943in}}%
\pgfpathlineto{\pgfqpoint{5.693447in}{1.987392in}}%
\pgfpathlineto{\pgfqpoint{5.693644in}{2.503633in}}%
\pgfpathlineto{\pgfqpoint{5.694534in}{1.949157in}}%
\pgfpathlineto{\pgfqpoint{5.694561in}{2.375389in}}%
\pgfpathlineto{\pgfqpoint{5.694635in}{1.988243in}}%
\pgfpathlineto{\pgfqpoint{5.695449in}{2.511115in}}%
\pgfpathlineto{\pgfqpoint{5.695650in}{2.394347in}}%
\pgfpathlineto{\pgfqpoint{5.696187in}{2.500152in}}%
\pgfpathlineto{\pgfqpoint{5.695881in}{1.804401in}}%
\pgfpathlineto{\pgfqpoint{5.696758in}{2.405378in}}%
\pgfpathlineto{\pgfqpoint{5.696897in}{1.913578in}}%
\pgfpathlineto{\pgfqpoint{5.697480in}{2.517905in}}%
\pgfpathlineto{\pgfqpoint{5.697867in}{2.342195in}}%
\pgfpathlineto{\pgfqpoint{5.698917in}{2.520410in}}%
\pgfpathlineto{\pgfqpoint{5.698176in}{1.971922in}}%
\pgfpathlineto{\pgfqpoint{5.698952in}{2.245261in}}%
\pgfpathlineto{\pgfqpoint{5.699251in}{2.034614in}}%
\pgfpathlineto{\pgfqpoint{5.699956in}{2.485244in}}%
\pgfpathlineto{\pgfqpoint{5.700047in}{2.242541in}}%
\pgfpathlineto{\pgfqpoint{5.700898in}{2.515783in}}%
\pgfpathlineto{\pgfqpoint{5.700354in}{1.796888in}}%
\pgfpathlineto{\pgfqpoint{5.701161in}{2.379359in}}%
\pgfpathlineto{\pgfqpoint{5.702014in}{2.491839in}}%
\pgfpathlineto{\pgfqpoint{5.701316in}{1.949586in}}%
\pgfpathlineto{\pgfqpoint{5.702259in}{2.377778in}}%
\pgfpathlineto{\pgfqpoint{5.703015in}{1.820557in}}%
\pgfpathlineto{\pgfqpoint{5.702496in}{2.511451in}}%
\pgfpathlineto{\pgfqpoint{5.703367in}{2.396431in}}%
\pgfpathlineto{\pgfqpoint{5.703577in}{1.949201in}}%
\pgfpathlineto{\pgfqpoint{5.704027in}{2.481649in}}%
\pgfpathlineto{\pgfqpoint{5.704477in}{2.355721in}}%
\pgfpathlineto{\pgfqpoint{5.705058in}{2.526213in}}%
\pgfpathlineto{\pgfqpoint{5.704742in}{2.031392in}}%
\pgfpathlineto{\pgfqpoint{5.705588in}{2.452934in}}%
\pgfpathlineto{\pgfqpoint{5.706078in}{2.019534in}}%
\pgfpathlineto{\pgfqpoint{5.706525in}{2.505990in}}%
\pgfpathlineto{\pgfqpoint{5.706708in}{2.227388in}}%
\pgfpathlineto{\pgfqpoint{5.707813in}{2.492577in}}%
\pgfpathlineto{\pgfqpoint{5.707112in}{1.928741in}}%
\pgfpathlineto{\pgfqpoint{5.707822in}{2.415420in}}%
\pgfpathlineto{\pgfqpoint{5.708492in}{2.497417in}}%
\pgfpathlineto{\pgfqpoint{5.708055in}{1.857986in}}%
\pgfpathlineto{\pgfqpoint{5.708856in}{2.364305in}}%
\pgfpathlineto{\pgfqpoint{5.708861in}{2.004242in}}%
\pgfpathlineto{\pgfqpoint{5.709597in}{2.493114in}}%
\pgfpathlineto{\pgfqpoint{5.709964in}{2.300835in}}%
\pgfpathlineto{\pgfqpoint{5.710496in}{2.510698in}}%
\pgfpathlineto{\pgfqpoint{5.710407in}{1.987063in}}%
\pgfpathlineto{\pgfqpoint{5.711073in}{2.434982in}}%
\pgfpathlineto{\pgfqpoint{5.711305in}{1.987405in}}%
\pgfpathlineto{\pgfqpoint{5.711292in}{2.500296in}}%
\pgfpathlineto{\pgfqpoint{5.712184in}{2.386792in}}%
\pgfpathlineto{\pgfqpoint{5.712310in}{2.030135in}}%
\pgfpathlineto{\pgfqpoint{5.713036in}{2.504795in}}%
\pgfpathlineto{\pgfqpoint{5.713283in}{2.228545in}}%
\pgfpathlineto{\pgfqpoint{5.713459in}{2.501999in}}%
\pgfpathlineto{\pgfqpoint{5.713417in}{2.006767in}}%
\pgfpathlineto{\pgfqpoint{5.714392in}{2.312528in}}%
\pgfpathlineto{\pgfqpoint{5.715256in}{2.030006in}}%
\pgfpathlineto{\pgfqpoint{5.715181in}{2.498917in}}%
\pgfpathlineto{\pgfqpoint{5.715502in}{2.339645in}}%
\pgfpathlineto{\pgfqpoint{5.715940in}{1.993133in}}%
\pgfpathlineto{\pgfqpoint{5.715511in}{2.516038in}}%
\pgfpathlineto{\pgfqpoint{5.716610in}{2.345561in}}%
\pgfpathlineto{\pgfqpoint{5.717312in}{2.497426in}}%
\pgfpathlineto{\pgfqpoint{5.716714in}{1.718395in}}%
\pgfpathlineto{\pgfqpoint{5.717710in}{2.341524in}}%
\pgfpathlineto{\pgfqpoint{5.718038in}{2.050162in}}%
\pgfpathlineto{\pgfqpoint{5.718650in}{2.527319in}}%
\pgfpathlineto{\pgfqpoint{5.718812in}{2.173647in}}%
\pgfpathlineto{\pgfqpoint{5.718895in}{2.503178in}}%
\pgfpathlineto{\pgfqpoint{5.719250in}{2.013430in}}%
\pgfpathlineto{\pgfqpoint{5.719923in}{2.402231in}}%
\pgfpathlineto{\pgfqpoint{5.720026in}{1.808007in}}%
\pgfpathlineto{\pgfqpoint{5.720961in}{2.513500in}}%
\pgfpathlineto{\pgfqpoint{5.721031in}{2.369841in}}%
\pgfpathlineto{\pgfqpoint{5.721767in}{2.504393in}}%
\pgfpathlineto{\pgfqpoint{5.721981in}{1.968867in}}%
\pgfpathlineto{\pgfqpoint{5.722137in}{2.391362in}}%
\pgfpathlineto{\pgfqpoint{5.722690in}{1.873073in}}%
\pgfpathlineto{\pgfqpoint{5.722748in}{2.524875in}}%
\pgfpathlineto{\pgfqpoint{5.723247in}{2.347193in}}%
\pgfpathlineto{\pgfqpoint{5.723812in}{2.509746in}}%
\pgfpathlineto{\pgfqpoint{5.724126in}{2.027296in}}%
\pgfpathlineto{\pgfqpoint{5.724269in}{2.386721in}}%
\pgfpathlineto{\pgfqpoint{5.725305in}{1.883947in}}%
\pgfpathlineto{\pgfqpoint{5.724543in}{2.489120in}}%
\pgfpathlineto{\pgfqpoint{5.725378in}{2.284156in}}%
\pgfpathlineto{\pgfqpoint{5.726371in}{2.520638in}}%
\pgfpathlineto{\pgfqpoint{5.726253in}{1.948891in}}%
\pgfpathlineto{\pgfqpoint{5.726444in}{2.436354in}}%
\pgfpathlineto{\pgfqpoint{5.726448in}{1.772786in}}%
\pgfpathlineto{\pgfqpoint{5.727248in}{2.486344in}}%
\pgfpathlineto{\pgfqpoint{5.727552in}{2.369027in}}%
\pgfpathlineto{\pgfqpoint{5.728426in}{1.567185in}}%
\pgfpathlineto{\pgfqpoint{5.727665in}{2.507794in}}%
\pgfpathlineto{\pgfqpoint{5.728672in}{2.324149in}}%
\pgfpathlineto{\pgfqpoint{5.729722in}{2.507285in}}%
\pgfpathlineto{\pgfqpoint{5.729242in}{1.901579in}}%
\pgfpathlineto{\pgfqpoint{5.729782in}{2.452005in}}%
\pgfpathlineto{\pgfqpoint{5.730217in}{1.877421in}}%
\pgfpathlineto{\pgfqpoint{5.730700in}{2.505258in}}%
\pgfpathlineto{\pgfqpoint{5.730897in}{2.198420in}}%
\pgfpathlineto{\pgfqpoint{5.731339in}{2.486348in}}%
\pgfpathlineto{\pgfqpoint{5.731801in}{1.965013in}}%
\pgfpathlineto{\pgfqpoint{5.732009in}{2.306757in}}%
\pgfpathlineto{\pgfqpoint{5.732594in}{1.928721in}}%
\pgfpathlineto{\pgfqpoint{5.732951in}{2.519515in}}%
\pgfpathlineto{\pgfqpoint{5.733063in}{2.370821in}}%
\pgfpathlineto{\pgfqpoint{5.733466in}{2.509538in}}%
\pgfpathlineto{\pgfqpoint{5.733961in}{1.865072in}}%
\pgfpathlineto{\pgfqpoint{5.734165in}{2.364119in}}%
\pgfpathlineto{\pgfqpoint{5.734691in}{1.910238in}}%
\pgfpathlineto{\pgfqpoint{5.734492in}{2.512836in}}%
\pgfpathlineto{\pgfqpoint{5.735276in}{2.279045in}}%
\pgfpathlineto{\pgfqpoint{5.735451in}{2.484689in}}%
\pgfpathlineto{\pgfqpoint{5.735583in}{1.819790in}}%
\pgfpathlineto{\pgfqpoint{5.736385in}{2.398234in}}%
\pgfpathlineto{\pgfqpoint{5.736544in}{1.812745in}}%
\pgfpathlineto{\pgfqpoint{5.736397in}{2.516366in}}%
\pgfpathlineto{\pgfqpoint{5.737499in}{2.188927in}}%
\pgfpathlineto{\pgfqpoint{5.738507in}{2.496792in}}%
\pgfpathlineto{\pgfqpoint{5.738021in}{2.037495in}}%
\pgfpathlineto{\pgfqpoint{5.738609in}{2.222734in}}%
\pgfpathlineto{\pgfqpoint{5.739670in}{2.491367in}}%
\pgfpathlineto{\pgfqpoint{5.739162in}{1.916698in}}%
\pgfpathlineto{\pgfqpoint{5.739721in}{2.357367in}}%
\pgfpathlineto{\pgfqpoint{5.740586in}{1.862908in}}%
\pgfpathlineto{\pgfqpoint{5.740524in}{2.474945in}}%
\pgfpathlineto{\pgfqpoint{5.740834in}{2.154843in}}%
\pgfpathlineto{\pgfqpoint{5.741109in}{2.493598in}}%
\pgfpathlineto{\pgfqpoint{5.741473in}{1.880058in}}%
\pgfpathlineto{\pgfqpoint{5.741944in}{2.148203in}}%
\pgfpathlineto{\pgfqpoint{5.742672in}{2.500843in}}%
\pgfpathlineto{\pgfqpoint{5.742116in}{1.875637in}}%
\pgfpathlineto{\pgfqpoint{5.743058in}{2.368721in}}%
\pgfpathlineto{\pgfqpoint{5.743074in}{1.779283in}}%
\pgfpathlineto{\pgfqpoint{5.743671in}{2.504485in}}%
\pgfpathlineto{\pgfqpoint{5.744170in}{2.173270in}}%
\pgfpathlineto{\pgfqpoint{5.744536in}{2.504523in}}%
\pgfpathlineto{\pgfqpoint{5.744455in}{2.045521in}}%
\pgfpathlineto{\pgfqpoint{5.745279in}{2.436954in}}%
\pgfpathlineto{\pgfqpoint{5.746083in}{1.954180in}}%
\pgfpathlineto{\pgfqpoint{5.745982in}{2.520067in}}%
\pgfpathlineto{\pgfqpoint{5.746389in}{2.370602in}}%
\pgfpathlineto{\pgfqpoint{5.746517in}{2.516534in}}%
\pgfpathlineto{\pgfqpoint{5.746742in}{2.044269in}}%
\pgfpathlineto{\pgfqpoint{5.747458in}{2.303144in}}%
\pgfpathlineto{\pgfqpoint{5.747960in}{1.921106in}}%
\pgfpathlineto{\pgfqpoint{5.747477in}{2.486810in}}%
\pgfpathlineto{\pgfqpoint{5.748566in}{2.364334in}}%
\pgfpathlineto{\pgfqpoint{5.748947in}{1.602476in}}%
\pgfpathlineto{\pgfqpoint{5.749463in}{2.513668in}}%
\pgfpathlineto{\pgfqpoint{5.749679in}{2.305494in}}%
\pgfpathlineto{\pgfqpoint{5.749898in}{2.507184in}}%
\pgfpathlineto{\pgfqpoint{5.750482in}{1.917304in}}%
\pgfpathlineto{\pgfqpoint{5.750754in}{2.363810in}}%
\pgfpathlineto{\pgfqpoint{5.750877in}{1.801500in}}%
\pgfpathlineto{\pgfqpoint{5.751540in}{2.515553in}}%
\pgfpathlineto{\pgfqpoint{5.751861in}{2.328490in}}%
\pgfpathlineto{\pgfqpoint{5.752213in}{2.494246in}}%
\pgfpathlineto{\pgfqpoint{5.752144in}{2.022253in}}%
\pgfpathlineto{\pgfqpoint{5.752969in}{2.380124in}}%
\pgfpathlineto{\pgfqpoint{5.753747in}{1.972670in}}%
\pgfpathlineto{\pgfqpoint{5.753423in}{2.478659in}}%
\pgfpathlineto{\pgfqpoint{5.754082in}{2.280848in}}%
\pgfpathlineto{\pgfqpoint{5.754174in}{2.504208in}}%
\pgfpathlineto{\pgfqpoint{5.754733in}{1.970224in}}%
\pgfpathlineto{\pgfqpoint{5.755196in}{2.460375in}}%
\pgfpathlineto{\pgfqpoint{5.756076in}{1.909317in}}%
\pgfpathlineto{\pgfqpoint{5.755280in}{2.493528in}}%
\pgfpathlineto{\pgfqpoint{5.756307in}{2.322768in}}%
\pgfpathlineto{\pgfqpoint{5.756420in}{2.019269in}}%
\pgfpathlineto{\pgfqpoint{5.756969in}{2.479258in}}%
\pgfpathlineto{\pgfqpoint{5.757169in}{2.130517in}}%
\pgfpathlineto{\pgfqpoint{5.757411in}{2.492004in}}%
\pgfpathlineto{\pgfqpoint{5.757234in}{1.968999in}}%
\pgfpathlineto{\pgfqpoint{5.758283in}{2.364708in}}%
\pgfpathlineto{\pgfqpoint{5.759348in}{1.833687in}}%
\pgfpathlineto{\pgfqpoint{5.758467in}{2.493783in}}%
\pgfpathlineto{\pgfqpoint{5.759393in}{2.377653in}}%
\pgfpathlineto{\pgfqpoint{5.760170in}{2.500635in}}%
\pgfpathlineto{\pgfqpoint{5.759799in}{1.929553in}}%
\pgfpathlineto{\pgfqpoint{5.760497in}{2.339268in}}%
\pgfpathlineto{\pgfqpoint{5.760770in}{1.917839in}}%
\pgfpathlineto{\pgfqpoint{5.761231in}{2.495142in}}%
\pgfpathlineto{\pgfqpoint{5.761594in}{2.321242in}}%
\pgfpathlineto{\pgfqpoint{5.762595in}{2.512200in}}%
\pgfpathlineto{\pgfqpoint{5.761919in}{1.852248in}}%
\pgfpathlineto{\pgfqpoint{5.762699in}{2.385064in}}%
\pgfpathlineto{\pgfqpoint{5.763414in}{1.903702in}}%
\pgfpathlineto{\pgfqpoint{5.762800in}{2.501559in}}%
\pgfpathlineto{\pgfqpoint{5.763809in}{2.328056in}}%
\pgfpathlineto{\pgfqpoint{5.764430in}{1.701789in}}%
\pgfpathlineto{\pgfqpoint{5.764419in}{2.493532in}}%
\pgfpathlineto{\pgfqpoint{5.764920in}{2.281358in}}%
\pgfpathlineto{\pgfqpoint{5.765372in}{2.494860in}}%
\pgfpathlineto{\pgfqpoint{5.765435in}{1.966724in}}%
\pgfpathlineto{\pgfqpoint{5.766031in}{2.454916in}}%
\pgfpathlineto{\pgfqpoint{5.766811in}{1.995837in}}%
\pgfpathlineto{\pgfqpoint{5.767118in}{2.513087in}}%
\pgfpathlineto{\pgfqpoint{5.767144in}{2.316027in}}%
\pgfpathlineto{\pgfqpoint{5.767321in}{2.505613in}}%
\pgfpathlineto{\pgfqpoint{5.768143in}{1.952398in}}%
\pgfpathlineto{\pgfqpoint{5.768253in}{2.292563in}}%
\pgfpathlineto{\pgfqpoint{5.768526in}{2.514470in}}%
\pgfpathlineto{\pgfqpoint{5.768662in}{1.938976in}}%
\pgfpathlineto{\pgfqpoint{5.769345in}{2.362015in}}%
\pgfpathlineto{\pgfqpoint{5.769532in}{2.017910in}}%
\pgfpathlineto{\pgfqpoint{5.770149in}{2.507185in}}%
\pgfpathlineto{\pgfqpoint{5.770456in}{2.187767in}}%
\pgfpathlineto{\pgfqpoint{5.770636in}{2.498290in}}%
\pgfpathlineto{\pgfqpoint{5.770617in}{1.982440in}}%
\pgfpathlineto{\pgfqpoint{5.771572in}{2.469098in}}%
\pgfpathlineto{\pgfqpoint{5.772316in}{1.704152in}}%
\pgfpathlineto{\pgfqpoint{5.772024in}{2.535055in}}%
\pgfpathlineto{\pgfqpoint{5.772684in}{2.357248in}}%
\pgfpathlineto{\pgfqpoint{5.773776in}{2.483869in}}%
\pgfpathlineto{\pgfqpoint{5.772837in}{1.895673in}}%
\pgfpathlineto{\pgfqpoint{5.773783in}{2.351291in}}%
\pgfpathlineto{\pgfqpoint{5.774636in}{1.693292in}}%
\pgfpathlineto{\pgfqpoint{5.774669in}{2.484899in}}%
\pgfpathlineto{\pgfqpoint{5.774893in}{2.269461in}}%
\pgfpathlineto{\pgfqpoint{5.775719in}{2.487027in}}%
\pgfpathlineto{\pgfqpoint{5.774966in}{1.995249in}}%
\pgfpathlineto{\pgfqpoint{5.775997in}{2.400283in}}%
\pgfpathlineto{\pgfqpoint{5.776131in}{2.032240in}}%
\pgfpathlineto{\pgfqpoint{5.776759in}{2.485834in}}%
\pgfpathlineto{\pgfqpoint{5.777109in}{2.334176in}}%
\pgfpathlineto{\pgfqpoint{5.778218in}{1.842816in}}%
\pgfpathlineto{\pgfqpoint{5.777912in}{2.489204in}}%
\pgfpathlineto{\pgfqpoint{5.778222in}{2.253660in}}%
\pgfpathlineto{\pgfqpoint{5.778434in}{2.497277in}}%
\pgfpathlineto{\pgfqpoint{5.778448in}{1.923683in}}%
\pgfpathlineto{\pgfqpoint{5.779335in}{2.398324in}}%
\pgfpathlineto{\pgfqpoint{5.779525in}{1.952150in}}%
\pgfpathlineto{\pgfqpoint{5.779432in}{2.481605in}}%
\pgfpathlineto{\pgfqpoint{5.780445in}{2.311667in}}%
\pgfpathlineto{\pgfqpoint{5.781264in}{2.489569in}}%
\pgfpathlineto{\pgfqpoint{5.780989in}{2.016489in}}%
\pgfpathlineto{\pgfqpoint{5.781545in}{2.304897in}}%
\pgfpathlineto{\pgfqpoint{5.782112in}{1.970907in}}%
\pgfpathlineto{\pgfqpoint{5.782162in}{2.477177in}}%
\pgfpathlineto{\pgfqpoint{5.782657in}{2.158318in}}%
\pgfpathlineto{\pgfqpoint{5.783084in}{2.481445in}}%
\pgfpathlineto{\pgfqpoint{5.783326in}{1.966167in}}%
\pgfpathlineto{\pgfqpoint{5.783769in}{2.439531in}}%
\pgfpathlineto{\pgfqpoint{5.783915in}{1.949374in}}%
\pgfpathlineto{\pgfqpoint{5.784603in}{2.486140in}}%
\pgfpathlineto{\pgfqpoint{5.784879in}{2.323804in}}%
\pgfpathlineto{\pgfqpoint{5.785332in}{2.469637in}}%
\pgfpathlineto{\pgfqpoint{5.785162in}{2.047912in}}%
\pgfpathlineto{\pgfqpoint{5.785982in}{2.241395in}}%
\pgfpathlineto{\pgfqpoint{5.786264in}{2.475936in}}%
\pgfpathlineto{\pgfqpoint{5.786560in}{1.779246in}}%
\pgfpathlineto{\pgfqpoint{5.787110in}{2.349042in}}%
\pgfpathlineto{\pgfqpoint{5.787406in}{2.015010in}}%
\pgfpathlineto{\pgfqpoint{5.787314in}{2.478710in}}%
\pgfpathlineto{\pgfqpoint{5.788221in}{2.262707in}}%
\pgfpathlineto{\pgfqpoint{5.788555in}{2.495782in}}%
\pgfpathlineto{\pgfqpoint{5.789014in}{1.947455in}}%
\pgfpathlineto{\pgfqpoint{5.789333in}{2.397651in}}%
\pgfpathlineto{\pgfqpoint{5.789977in}{1.839552in}}%
\pgfpathlineto{\pgfqpoint{5.789477in}{2.497007in}}%
\pgfpathlineto{\pgfqpoint{5.790446in}{2.158319in}}%
\pgfpathlineto{\pgfqpoint{5.790819in}{2.504992in}}%
\pgfpathlineto{\pgfqpoint{5.791457in}{1.902561in}}%
\pgfpathlineto{\pgfqpoint{5.791559in}{2.407890in}}%
\pgfpathlineto{\pgfqpoint{5.791611in}{2.520623in}}%
\pgfpathlineto{\pgfqpoint{5.792077in}{1.911376in}}%
\pgfpathlineto{\pgfqpoint{5.792658in}{2.429223in}}%
\pgfpathlineto{\pgfqpoint{5.792665in}{1.831783in}}%
\pgfpathlineto{\pgfqpoint{5.793065in}{2.517859in}}%
\pgfpathlineto{\pgfqpoint{5.793769in}{2.355494in}}%
\pgfpathlineto{\pgfqpoint{5.794500in}{1.821253in}}%
\pgfpathlineto{\pgfqpoint{5.794424in}{2.497642in}}%
\pgfpathlineto{\pgfqpoint{5.794877in}{2.233735in}}%
\pgfpathlineto{\pgfqpoint{5.795837in}{2.506444in}}%
\pgfpathlineto{\pgfqpoint{5.795170in}{1.862965in}}%
\pgfpathlineto{\pgfqpoint{5.795989in}{2.327700in}}%
\pgfpathlineto{\pgfqpoint{5.796643in}{1.957257in}}%
\pgfpathlineto{\pgfqpoint{5.796819in}{2.487223in}}%
\pgfpathlineto{\pgfqpoint{5.797097in}{2.262564in}}%
\pgfpathlineto{\pgfqpoint{5.797905in}{2.505785in}}%
\pgfpathlineto{\pgfqpoint{5.798135in}{1.889472in}}%
\pgfpathlineto{\pgfqpoint{5.798203in}{2.455689in}}%
\pgfpathlineto{\pgfqpoint{5.798207in}{1.865711in}}%
\pgfpathlineto{\pgfqpoint{5.798402in}{2.494792in}}%
\pgfpathlineto{\pgfqpoint{5.799317in}{2.025637in}}%
\pgfpathlineto{\pgfqpoint{5.800281in}{2.470787in}}%
\pgfpathlineto{\pgfqpoint{5.800222in}{1.968500in}}%
\pgfpathlineto{\pgfqpoint{5.800434in}{2.388172in}}%
\pgfpathlineto{\pgfqpoint{5.800949in}{2.492238in}}%
\pgfpathlineto{\pgfqpoint{5.800813in}{1.985071in}}%
\pgfpathlineto{\pgfqpoint{5.801487in}{2.217767in}}%
\pgfpathlineto{\pgfqpoint{5.802314in}{1.951055in}}%
\pgfpathlineto{\pgfqpoint{5.802093in}{2.514662in}}%
\pgfpathlineto{\pgfqpoint{5.802592in}{2.252974in}}%
\pgfpathlineto{\pgfqpoint{5.803413in}{2.509860in}}%
\pgfpathlineto{\pgfqpoint{5.802721in}{1.862547in}}%
\pgfpathlineto{\pgfqpoint{5.803701in}{2.476797in}}%
\pgfpathlineto{\pgfqpoint{5.804648in}{1.975447in}}%
\pgfpathlineto{\pgfqpoint{5.803816in}{2.494245in}}%
\pgfpathlineto{\pgfqpoint{5.804810in}{2.230133in}}%
\pgfpathlineto{\pgfqpoint{5.805701in}{2.484517in}}%
\pgfpathlineto{\pgfqpoint{5.804840in}{1.913943in}}%
\pgfpathlineto{\pgfqpoint{5.805920in}{2.377541in}}%
\pgfpathlineto{\pgfqpoint{5.806122in}{1.895316in}}%
\pgfpathlineto{\pgfqpoint{5.806858in}{2.490656in}}%
\pgfpathlineto{\pgfqpoint{5.807030in}{2.255735in}}%
\pgfpathlineto{\pgfqpoint{5.807282in}{2.529530in}}%
\pgfpathlineto{\pgfqpoint{5.807460in}{1.935609in}}%
\pgfpathlineto{\pgfqpoint{5.808140in}{2.320676in}}%
\pgfpathlineto{\pgfqpoint{5.808378in}{1.867125in}}%
\pgfpathlineto{\pgfqpoint{5.808181in}{2.484863in}}%
\pgfpathlineto{\pgfqpoint{5.809245in}{2.183019in}}%
\pgfpathlineto{\pgfqpoint{5.810310in}{2.511027in}}%
\pgfpathlineto{\pgfqpoint{5.810196in}{1.815910in}}%
\pgfpathlineto{\pgfqpoint{5.810360in}{2.419003in}}%
\pgfpathlineto{\pgfqpoint{5.811212in}{1.919557in}}%
\pgfpathlineto{\pgfqpoint{5.810393in}{2.499371in}}%
\pgfpathlineto{\pgfqpoint{5.811472in}{2.299223in}}%
\pgfpathlineto{\pgfqpoint{5.811638in}{2.484167in}}%
\pgfpathlineto{\pgfqpoint{5.811884in}{1.816156in}}%
\pgfpathlineto{\pgfqpoint{5.812537in}{2.291293in}}%
\pgfpathlineto{\pgfqpoint{5.812541in}{1.980864in}}%
\pgfpathlineto{\pgfqpoint{5.812753in}{2.518713in}}%
\pgfpathlineto{\pgfqpoint{5.813647in}{2.260919in}}%
\pgfpathlineto{\pgfqpoint{5.814552in}{2.475663in}}%
\pgfpathlineto{\pgfqpoint{5.814595in}{1.828927in}}%
\pgfpathlineto{\pgfqpoint{5.814754in}{2.266278in}}%
\pgfpathlineto{\pgfqpoint{5.815528in}{2.008970in}}%
\pgfpathlineto{\pgfqpoint{5.815439in}{2.478792in}}%
\pgfpathlineto{\pgfqpoint{5.815861in}{2.051930in}}%
\pgfpathlineto{\pgfqpoint{5.816111in}{2.491807in}}%
\pgfpathlineto{\pgfqpoint{5.815933in}{1.838845in}}%
\pgfpathlineto{\pgfqpoint{5.816971in}{2.251028in}}%
\pgfpathlineto{\pgfqpoint{5.817696in}{2.485949in}}%
\pgfpathlineto{\pgfqpoint{5.817840in}{1.875907in}}%
\pgfpathlineto{\pgfqpoint{5.818083in}{2.264217in}}%
\pgfpathlineto{\pgfqpoint{5.819011in}{2.478642in}}%
\pgfpathlineto{\pgfqpoint{5.819050in}{1.776581in}}%
\pgfpathlineto{\pgfqpoint{5.819188in}{2.406723in}}%
\pgfpathlineto{\pgfqpoint{5.819354in}{1.941611in}}%
\pgfpathlineto{\pgfqpoint{5.820065in}{2.515807in}}%
\pgfpathlineto{\pgfqpoint{5.820299in}{2.244803in}}%
\pgfpathlineto{\pgfqpoint{5.820940in}{1.903896in}}%
\pgfpathlineto{\pgfqpoint{5.820472in}{2.503701in}}%
\pgfpathlineto{\pgfqpoint{5.821399in}{2.320069in}}%
\pgfpathlineto{\pgfqpoint{5.822096in}{2.488531in}}%
\pgfpathlineto{\pgfqpoint{5.821623in}{1.965316in}}%
\pgfpathlineto{\pgfqpoint{5.822511in}{2.389633in}}%
\pgfpathlineto{\pgfqpoint{5.823123in}{1.880639in}}%
\pgfpathlineto{\pgfqpoint{5.822971in}{2.478096in}}%
\pgfpathlineto{\pgfqpoint{5.823618in}{2.369494in}}%
\pgfpathlineto{\pgfqpoint{5.824095in}{2.507090in}}%
\pgfpathlineto{\pgfqpoint{5.824463in}{1.779187in}}%
\pgfpathlineto{\pgfqpoint{5.824724in}{2.393003in}}%
\pgfpathlineto{\pgfqpoint{5.824840in}{1.976873in}}%
\pgfpathlineto{\pgfqpoint{5.825433in}{2.477578in}}%
\pgfpathlineto{\pgfqpoint{5.825844in}{2.135907in}}%
\pgfpathlineto{\pgfqpoint{5.825851in}{2.486753in}}%
\pgfpathlineto{\pgfqpoint{5.826172in}{1.936009in}}%
\pgfpathlineto{\pgfqpoint{5.826955in}{2.343525in}}%
\pgfpathlineto{\pgfqpoint{5.827768in}{1.729452in}}%
\pgfpathlineto{\pgfqpoint{5.827317in}{2.498250in}}%
\pgfpathlineto{\pgfqpoint{5.828063in}{2.182082in}}%
\pgfpathlineto{\pgfqpoint{5.828296in}{2.492664in}}%
\pgfpathlineto{\pgfqpoint{5.828398in}{1.866907in}}%
\pgfpathlineto{\pgfqpoint{5.829174in}{2.357907in}}%
\pgfpathlineto{\pgfqpoint{5.829687in}{1.797828in}}%
\pgfpathlineto{\pgfqpoint{5.829183in}{2.505906in}}%
\pgfpathlineto{\pgfqpoint{5.830276in}{2.354446in}}%
\pgfpathlineto{\pgfqpoint{5.830718in}{2.499371in}}%
\pgfpathlineto{\pgfqpoint{5.830803in}{1.907882in}}%
\pgfpathlineto{\pgfqpoint{5.831381in}{2.337215in}}%
\pgfpathlineto{\pgfqpoint{5.831498in}{1.937863in}}%
\pgfpathlineto{\pgfqpoint{5.832322in}{2.496997in}}%
\pgfpathlineto{\pgfqpoint{5.832490in}{2.230103in}}%
\pgfpathlineto{\pgfqpoint{5.832575in}{2.488996in}}%
\pgfpathlineto{\pgfqpoint{5.832911in}{1.968343in}}%
\pgfpathlineto{\pgfqpoint{5.833602in}{2.358199in}}%
\pgfpathlineto{\pgfqpoint{5.833624in}{1.815478in}}%
\pgfpathlineto{\pgfqpoint{5.834542in}{2.472131in}}%
\pgfpathlineto{\pgfqpoint{5.834715in}{2.295240in}}%
\pgfpathlineto{\pgfqpoint{5.835017in}{2.469363in}}%
\pgfpathlineto{\pgfqpoint{5.835253in}{1.766456in}}%
\pgfpathlineto{\pgfqpoint{5.835828in}{2.406258in}}%
\pgfpathlineto{\pgfqpoint{5.836840in}{1.917548in}}%
\pgfpathlineto{\pgfqpoint{5.836430in}{2.473289in}}%
\pgfpathlineto{\pgfqpoint{5.836941in}{2.196466in}}%
\pgfpathlineto{\pgfqpoint{5.837582in}{2.492310in}}%
\pgfpathlineto{\pgfqpoint{5.837991in}{2.017121in}}%
\pgfpathlineto{\pgfqpoint{5.838051in}{2.215920in}}%
\pgfpathlineto{\pgfqpoint{5.838341in}{1.954513in}}%
\pgfpathlineto{\pgfqpoint{5.838444in}{2.479158in}}%
\pgfpathlineto{\pgfqpoint{5.839161in}{2.219475in}}%
\pgfpathlineto{\pgfqpoint{5.839516in}{2.516235in}}%
\pgfpathlineto{\pgfqpoint{5.839507in}{1.774578in}}%
\pgfpathlineto{\pgfqpoint{5.840271in}{2.307890in}}%
\pgfpathlineto{\pgfqpoint{5.840759in}{1.745787in}}%
\pgfpathlineto{\pgfqpoint{5.840529in}{2.487477in}}%
\pgfpathlineto{\pgfqpoint{5.841376in}{2.280155in}}%
\pgfpathlineto{\pgfqpoint{5.842212in}{2.493739in}}%
\pgfpathlineto{\pgfqpoint{5.841766in}{1.951738in}}%
\pgfpathlineto{\pgfqpoint{5.842490in}{2.436377in}}%
\pgfpathlineto{\pgfqpoint{5.843327in}{1.859924in}}%
\pgfpathlineto{\pgfqpoint{5.842576in}{2.480731in}}%
\pgfpathlineto{\pgfqpoint{5.843604in}{2.308522in}}%
\pgfpathlineto{\pgfqpoint{5.843767in}{2.481196in}}%
\pgfpathlineto{\pgfqpoint{5.843653in}{1.808009in}}%
\pgfpathlineto{\pgfqpoint{5.844715in}{2.409068in}}%
\pgfpathlineto{\pgfqpoint{5.844934in}{1.812708in}}%
\pgfpathlineto{\pgfqpoint{5.845388in}{2.476729in}}%
\pgfpathlineto{\pgfqpoint{5.845830in}{2.134243in}}%
\pgfpathlineto{\pgfqpoint{5.846182in}{2.479748in}}%
\pgfpathlineto{\pgfqpoint{5.846773in}{1.983127in}}%
\pgfpathlineto{\pgfqpoint{5.846941in}{2.256336in}}%
\pgfpathlineto{\pgfqpoint{5.847265in}{2.493457in}}%
\pgfpathlineto{\pgfqpoint{5.847372in}{1.758235in}}%
\pgfpathlineto{\pgfqpoint{5.848047in}{2.200740in}}%
\pgfpathlineto{\pgfqpoint{5.848273in}{2.462654in}}%
\pgfpathlineto{\pgfqpoint{5.848294in}{1.855717in}}%
\pgfpathlineto{\pgfqpoint{5.848830in}{2.375959in}}%
\pgfpathlineto{\pgfqpoint{5.849897in}{1.730701in}}%
\pgfpathlineto{\pgfqpoint{5.849928in}{2.493907in}}%
\pgfpathlineto{\pgfqpoint{5.849940in}{2.215373in}}%
\pgfpathlineto{\pgfqpoint{5.850923in}{2.471097in}}%
\pgfpathlineto{\pgfqpoint{5.850016in}{1.926067in}}%
\pgfpathlineto{\pgfqpoint{5.851050in}{2.303975in}}%
\pgfpathlineto{\pgfqpoint{5.851335in}{1.876592in}}%
\pgfpathlineto{\pgfqpoint{5.851855in}{2.483399in}}%
\pgfpathlineto{\pgfqpoint{5.852160in}{2.162970in}}%
\pgfpathlineto{\pgfqpoint{5.853156in}{2.465228in}}%
\pgfpathlineto{\pgfqpoint{5.852743in}{1.858661in}}%
\pgfpathlineto{\pgfqpoint{5.853270in}{2.386200in}}%
\pgfpathlineto{\pgfqpoint{5.853526in}{1.873499in}}%
\pgfpathlineto{\pgfqpoint{5.853421in}{2.489348in}}%
\pgfpathlineto{\pgfqpoint{5.854386in}{2.260265in}}%
\pgfpathlineto{\pgfqpoint{5.854894in}{2.492930in}}%
\pgfpathlineto{\pgfqpoint{5.854669in}{1.908367in}}%
\pgfpathlineto{\pgfqpoint{5.855497in}{2.294898in}}%
\pgfpathlineto{\pgfqpoint{5.856362in}{1.949689in}}%
\pgfpathlineto{\pgfqpoint{5.855865in}{2.491736in}}%
\pgfpathlineto{\pgfqpoint{5.856607in}{2.265455in}}%
\pgfpathlineto{\pgfqpoint{5.857318in}{2.497961in}}%
\pgfpathlineto{\pgfqpoint{5.856906in}{1.912406in}}%
\pgfpathlineto{\pgfqpoint{5.857721in}{2.387614in}}%
\pgfpathlineto{\pgfqpoint{5.858567in}{1.889924in}}%
\pgfpathlineto{\pgfqpoint{5.858558in}{2.517547in}}%
\pgfpathlineto{\pgfqpoint{5.858831in}{2.251491in}}%
\pgfpathlineto{\pgfqpoint{5.858852in}{2.483180in}}%
\pgfpathlineto{\pgfqpoint{5.859660in}{2.005394in}}%
\pgfpathlineto{\pgfqpoint{5.859942in}{2.294868in}}%
\pgfpathlineto{\pgfqpoint{5.860884in}{2.489627in}}%
\pgfpathlineto{\pgfqpoint{5.860793in}{1.831534in}}%
\pgfpathlineto{\pgfqpoint{5.861044in}{2.194236in}}%
\pgfpathlineto{\pgfqpoint{5.861907in}{1.752223in}}%
\pgfpathlineto{\pgfqpoint{5.861263in}{2.473833in}}%
\pgfpathlineto{\pgfqpoint{5.862149in}{2.278902in}}%
\pgfpathlineto{\pgfqpoint{5.862181in}{2.479891in}}%
\pgfpathlineto{\pgfqpoint{5.862347in}{1.765253in}}%
\pgfpathlineto{\pgfqpoint{5.863263in}{2.375978in}}%
\pgfpathlineto{\pgfqpoint{5.863372in}{2.487798in}}%
\pgfpathlineto{\pgfqpoint{5.864062in}{1.831305in}}%
\pgfpathlineto{\pgfqpoint{5.864344in}{2.316598in}}%
\pgfpathlineto{\pgfqpoint{5.865151in}{1.979958in}}%
\pgfpathlineto{\pgfqpoint{5.864523in}{2.519504in}}%
\pgfpathlineto{\pgfqpoint{5.865458in}{2.232326in}}%
\pgfpathlineto{\pgfqpoint{5.866154in}{1.860752in}}%
\pgfpathlineto{\pgfqpoint{5.866160in}{2.468330in}}%
\pgfpathlineto{\pgfqpoint{5.866555in}{2.314411in}}%
\pgfpathlineto{\pgfqpoint{5.867275in}{2.485906in}}%
\pgfpathlineto{\pgfqpoint{5.867369in}{1.802746in}}%
\pgfpathlineto{\pgfqpoint{5.867663in}{2.354875in}}%
\pgfpathlineto{\pgfqpoint{5.867838in}{1.884195in}}%
\pgfpathlineto{\pgfqpoint{5.868350in}{2.495483in}}%
\pgfpathlineto{\pgfqpoint{5.868774in}{2.338267in}}%
\pgfpathlineto{\pgfqpoint{5.869169in}{1.758700in}}%
\pgfpathlineto{\pgfqpoint{5.868945in}{2.479414in}}%
\pgfpathlineto{\pgfqpoint{5.869888in}{2.256543in}}%
\pgfpathlineto{\pgfqpoint{5.870360in}{2.489174in}}%
\pgfpathlineto{\pgfqpoint{5.870808in}{1.947203in}}%
\pgfpathlineto{\pgfqpoint{5.871008in}{2.437641in}}%
\pgfpathlineto{\pgfqpoint{5.871533in}{1.867950in}}%
\pgfpathlineto{\pgfqpoint{5.871149in}{2.489622in}}%
\pgfpathlineto{\pgfqpoint{5.872122in}{2.259115in}}%
\pgfpathlineto{\pgfqpoint{5.873089in}{2.481501in}}%
\pgfpathlineto{\pgfqpoint{5.872415in}{1.936933in}}%
\pgfpathlineto{\pgfqpoint{5.873198in}{2.305208in}}%
\pgfpathlineto{\pgfqpoint{5.873945in}{1.728319in}}%
\pgfpathlineto{\pgfqpoint{5.873526in}{2.488976in}}%
\pgfpathlineto{\pgfqpoint{5.874306in}{2.219395in}}%
\pgfpathlineto{\pgfqpoint{5.874810in}{2.503703in}}%
\pgfpathlineto{\pgfqpoint{5.875031in}{1.920357in}}%
\pgfpathlineto{\pgfqpoint{5.875417in}{2.224048in}}%
\pgfpathlineto{\pgfqpoint{5.875654in}{2.494204in}}%
\pgfpathlineto{\pgfqpoint{5.876425in}{1.879973in}}%
\pgfpathlineto{\pgfqpoint{5.876522in}{2.408185in}}%
\pgfpathlineto{\pgfqpoint{5.877582in}{1.972169in}}%
\pgfpathlineto{\pgfqpoint{5.877451in}{2.485291in}}%
\pgfpathlineto{\pgfqpoint{5.877633in}{2.306260in}}%
\pgfpathlineto{\pgfqpoint{5.877818in}{1.924746in}}%
\pgfpathlineto{\pgfqpoint{5.877801in}{2.478892in}}%
\pgfpathlineto{\pgfqpoint{5.878744in}{2.295503in}}%
\pgfpathlineto{\pgfqpoint{5.879014in}{2.497052in}}%
\pgfpathlineto{\pgfqpoint{5.879524in}{1.744726in}}%
\pgfpathlineto{\pgfqpoint{5.879855in}{2.377955in}}%
\pgfpathlineto{\pgfqpoint{5.880112in}{1.790447in}}%
\pgfpathlineto{\pgfqpoint{5.880582in}{2.498372in}}%
\pgfpathlineto{\pgfqpoint{5.880966in}{1.970516in}}%
\pgfpathlineto{\pgfqpoint{5.881036in}{2.481094in}}%
\pgfpathlineto{\pgfqpoint{5.881767in}{1.913210in}}%
\pgfpathlineto{\pgfqpoint{5.882076in}{2.362171in}}%
\pgfpathlineto{\pgfqpoint{5.882861in}{1.786656in}}%
\pgfpathlineto{\pgfqpoint{5.882211in}{2.481391in}}%
\pgfpathlineto{\pgfqpoint{5.883187in}{2.360525in}}%
\pgfpathlineto{\pgfqpoint{5.883947in}{1.853437in}}%
\pgfpathlineto{\pgfqpoint{5.883776in}{2.482231in}}%
\pgfpathlineto{\pgfqpoint{5.884297in}{2.386599in}}%
\pgfpathlineto{\pgfqpoint{5.884762in}{2.484028in}}%
\pgfpathlineto{\pgfqpoint{5.884692in}{1.826454in}}%
\pgfpathlineto{\pgfqpoint{5.885399in}{2.337657in}}%
\pgfpathlineto{\pgfqpoint{5.885595in}{1.860174in}}%
\pgfpathlineto{\pgfqpoint{5.886362in}{2.489507in}}%
\pgfpathlineto{\pgfqpoint{5.886509in}{2.264143in}}%
\pgfpathlineto{\pgfqpoint{5.886849in}{2.471464in}}%
\pgfpathlineto{\pgfqpoint{5.887178in}{1.858569in}}%
\pgfpathlineto{\pgfqpoint{5.887625in}{2.392509in}}%
\pgfpathlineto{\pgfqpoint{5.888569in}{1.947220in}}%
\pgfpathlineto{\pgfqpoint{5.888433in}{2.483699in}}%
\pgfpathlineto{\pgfqpoint{5.888738in}{2.328342in}}%
\pgfpathlineto{\pgfqpoint{5.889762in}{1.946051in}}%
\pgfpathlineto{\pgfqpoint{5.889159in}{2.476852in}}%
\pgfpathlineto{\pgfqpoint{5.889842in}{2.210013in}}%
\pgfpathlineto{\pgfqpoint{5.890386in}{2.495298in}}%
\pgfpathlineto{\pgfqpoint{5.890726in}{1.838262in}}%
\pgfpathlineto{\pgfqpoint{5.890952in}{2.338006in}}%
\pgfpathlineto{\pgfqpoint{5.891481in}{1.738063in}}%
\pgfpathlineto{\pgfqpoint{5.891910in}{2.467805in}}%
\pgfpathlineto{\pgfqpoint{5.892059in}{2.164562in}}%
\pgfpathlineto{\pgfqpoint{5.892809in}{2.460469in}}%
\pgfpathlineto{\pgfqpoint{5.892086in}{1.932387in}}%
\pgfpathlineto{\pgfqpoint{5.893171in}{2.356169in}}%
\pgfpathlineto{\pgfqpoint{5.893798in}{2.477783in}}%
\pgfpathlineto{\pgfqpoint{5.893261in}{1.642256in}}%
\pgfpathlineto{\pgfqpoint{5.894275in}{2.376169in}}%
\pgfpathlineto{\pgfqpoint{5.894592in}{1.966146in}}%
\pgfpathlineto{\pgfqpoint{5.894958in}{2.471790in}}%
\pgfpathlineto{\pgfqpoint{5.895384in}{2.168987in}}%
\pgfpathlineto{\pgfqpoint{5.896169in}{2.464134in}}%
\pgfpathlineto{\pgfqpoint{5.895670in}{1.757358in}}%
\pgfpathlineto{\pgfqpoint{5.896495in}{2.360387in}}%
\pgfpathlineto{\pgfqpoint{5.896778in}{1.702049in}}%
\pgfpathlineto{\pgfqpoint{5.896694in}{2.492997in}}%
\pgfpathlineto{\pgfqpoint{5.897604in}{2.423880in}}%
\pgfpathlineto{\pgfqpoint{5.898572in}{1.814175in}}%
\pgfpathlineto{\pgfqpoint{5.898084in}{2.470805in}}%
\pgfpathlineto{\pgfqpoint{5.898718in}{2.359609in}}%
\pgfpathlineto{\pgfqpoint{5.899083in}{2.480347in}}%
\pgfpathlineto{\pgfqpoint{5.898862in}{1.863442in}}%
\pgfpathlineto{\pgfqpoint{5.899821in}{2.338171in}}%
\pgfpathlineto{\pgfqpoint{5.900919in}{1.819038in}}%
\pgfpathlineto{\pgfqpoint{5.900207in}{2.482756in}}%
\pgfpathlineto{\pgfqpoint{5.900932in}{2.316746in}}%
\pgfpathlineto{\pgfqpoint{5.901026in}{2.465922in}}%
\pgfpathlineto{\pgfqpoint{5.901701in}{1.859930in}}%
\pgfpathlineto{\pgfqpoint{5.902037in}{2.255388in}}%
\pgfpathlineto{\pgfqpoint{5.902212in}{1.893590in}}%
\pgfpathlineto{\pgfqpoint{5.902453in}{2.475925in}}%
\pgfpathlineto{\pgfqpoint{5.903134in}{2.377333in}}%
\pgfpathlineto{\pgfqpoint{5.903699in}{2.479633in}}%
\pgfpathlineto{\pgfqpoint{5.903563in}{1.899346in}}%
\pgfpathlineto{\pgfqpoint{5.904245in}{2.394969in}}%
\pgfpathlineto{\pgfqpoint{5.904944in}{1.671315in}}%
\pgfpathlineto{\pgfqpoint{5.905245in}{2.461869in}}%
\pgfpathlineto{\pgfqpoint{5.905355in}{2.121060in}}%
\pgfpathlineto{\pgfqpoint{5.905424in}{2.470561in}}%
\pgfpathlineto{\pgfqpoint{5.906201in}{1.918334in}}%
\pgfpathlineto{\pgfqpoint{5.906467in}{2.316432in}}%
\pgfpathlineto{\pgfqpoint{5.907086in}{2.491909in}}%
\pgfpathlineto{\pgfqpoint{5.907240in}{1.737014in}}%
\pgfpathlineto{\pgfqpoint{5.907500in}{2.390540in}}%
\pgfpathlineto{\pgfqpoint{5.907850in}{1.827443in}}%
\pgfpathlineto{\pgfqpoint{5.907860in}{2.472363in}}%
\pgfpathlineto{\pgfqpoint{5.908609in}{2.266498in}}%
\pgfpathlineto{\pgfqpoint{5.909217in}{2.490470in}}%
\pgfpathlineto{\pgfqpoint{5.909127in}{1.922191in}}%
\pgfpathlineto{\pgfqpoint{5.909721in}{2.334537in}}%
\pgfpathlineto{\pgfqpoint{5.910472in}{1.877711in}}%
\pgfpathlineto{\pgfqpoint{5.909871in}{2.471545in}}%
\pgfpathlineto{\pgfqpoint{5.910828in}{2.132679in}}%
\pgfpathlineto{\pgfqpoint{5.911881in}{2.476645in}}%
\pgfpathlineto{\pgfqpoint{5.911377in}{1.950742in}}%
\pgfpathlineto{\pgfqpoint{5.911939in}{2.205566in}}%
\pgfpathlineto{\pgfqpoint{5.912073in}{2.458483in}}%
\pgfpathlineto{\pgfqpoint{5.912911in}{1.788943in}}%
\pgfpathlineto{\pgfqpoint{5.913053in}{2.378004in}}%
\pgfpathlineto{\pgfqpoint{5.913298in}{2.467871in}}%
\pgfpathlineto{\pgfqpoint{5.913097in}{1.952421in}}%
\pgfpathlineto{\pgfqpoint{5.913414in}{2.343240in}}%
\pgfpathlineto{\pgfqpoint{5.914046in}{1.700427in}}%
\pgfpathlineto{\pgfqpoint{5.913646in}{2.449789in}}%
\pgfpathlineto{\pgfqpoint{5.914523in}{2.343254in}}%
\pgfpathlineto{\pgfqpoint{5.914940in}{1.897163in}}%
\pgfpathlineto{\pgfqpoint{5.915313in}{2.466693in}}%
\pgfpathlineto{\pgfqpoint{5.915633in}{2.165108in}}%
\pgfpathlineto{\pgfqpoint{5.915914in}{2.482791in}}%
\pgfpathlineto{\pgfqpoint{5.916412in}{1.922820in}}%
\pgfpathlineto{\pgfqpoint{5.916745in}{2.362439in}}%
\pgfpathlineto{\pgfqpoint{5.917082in}{1.884791in}}%
\pgfpathlineto{\pgfqpoint{5.917833in}{2.477195in}}%
\pgfpathlineto{\pgfqpoint{5.917851in}{2.317730in}}%
\pgfpathlineto{\pgfqpoint{5.918128in}{2.491669in}}%
\pgfpathlineto{\pgfqpoint{5.918554in}{1.781255in}}%
\pgfpathlineto{\pgfqpoint{5.918960in}{2.314177in}}%
\pgfpathlineto{\pgfqpoint{5.919795in}{1.829148in}}%
\pgfpathlineto{\pgfqpoint{5.919339in}{2.468345in}}%
\pgfpathlineto{\pgfqpoint{5.920060in}{2.226000in}}%
\pgfpathlineto{\pgfqpoint{5.920402in}{2.492181in}}%
\pgfpathlineto{\pgfqpoint{5.920161in}{1.912533in}}%
\pgfpathlineto{\pgfqpoint{5.921173in}{2.470293in}}%
\pgfpathlineto{\pgfqpoint{5.921671in}{1.787323in}}%
\pgfpathlineto{\pgfqpoint{5.922289in}{2.156958in}}%
\pgfpathlineto{\pgfqpoint{5.922435in}{2.473663in}}%
\pgfpathlineto{\pgfqpoint{5.923064in}{1.872099in}}%
\pgfpathlineto{\pgfqpoint{5.923401in}{2.281642in}}%
\pgfpathlineto{\pgfqpoint{5.924128in}{1.879622in}}%
\pgfpathlineto{\pgfqpoint{5.923741in}{2.467189in}}%
\pgfpathlineto{\pgfqpoint{5.924383in}{2.253419in}}%
\pgfpathlineto{\pgfqpoint{5.924890in}{2.494405in}}%
\pgfpathlineto{\pgfqpoint{5.925472in}{1.927925in}}%
\pgfpathlineto{\pgfqpoint{5.925495in}{2.335998in}}%
\pgfpathlineto{\pgfqpoint{5.925536in}{1.903253in}}%
\pgfpathlineto{\pgfqpoint{5.925739in}{2.497279in}}%
\pgfpathlineto{\pgfqpoint{5.926609in}{2.214307in}}%
\pgfpathlineto{\pgfqpoint{5.926675in}{2.485745in}}%
\pgfpathlineto{\pgfqpoint{5.927410in}{1.664176in}}%
\pgfpathlineto{\pgfqpoint{5.927721in}{2.258209in}}%
\pgfpathlineto{\pgfqpoint{5.928259in}{1.921637in}}%
\pgfpathlineto{\pgfqpoint{5.928183in}{2.458909in}}%
\pgfpathlineto{\pgfqpoint{5.928827in}{2.230939in}}%
\pgfpathlineto{\pgfqpoint{5.929595in}{2.473982in}}%
\pgfpathlineto{\pgfqpoint{5.929046in}{1.930179in}}%
\pgfpathlineto{\pgfqpoint{5.929937in}{2.274126in}}%
\pgfpathlineto{\pgfqpoint{5.930723in}{1.869050in}}%
\pgfpathlineto{\pgfqpoint{5.930050in}{2.460525in}}%
\pgfpathlineto{\pgfqpoint{5.931045in}{2.244870in}}%
\pgfpathlineto{\pgfqpoint{5.932047in}{2.484153in}}%
\pgfpathlineto{\pgfqpoint{5.931481in}{1.652837in}}%
\pgfpathlineto{\pgfqpoint{5.932157in}{2.377502in}}%
\pgfpathlineto{\pgfqpoint{5.932782in}{1.907608in}}%
\pgfpathlineto{\pgfqpoint{5.932425in}{2.469498in}}%
\pgfpathlineto{\pgfqpoint{5.933266in}{2.432318in}}%
\pgfpathlineto{\pgfqpoint{5.933526in}{1.794843in}}%
\pgfpathlineto{\pgfqpoint{5.934186in}{2.457296in}}%
\pgfpathlineto{\pgfqpoint{5.934390in}{2.267573in}}%
\pgfpathlineto{\pgfqpoint{5.934544in}{2.463310in}}%
\pgfpathlineto{\pgfqpoint{5.935158in}{1.772908in}}%
\pgfpathlineto{\pgfqpoint{5.935501in}{2.259455in}}%
\pgfpathlineto{\pgfqpoint{5.935647in}{2.458824in}}%
\pgfpathlineto{\pgfqpoint{5.935843in}{1.775637in}}%
\pgfpathlineto{\pgfqpoint{5.936597in}{2.399609in}}%
\pgfpathlineto{\pgfqpoint{5.936725in}{1.830098in}}%
\pgfpathlineto{\pgfqpoint{5.936965in}{2.465012in}}%
\pgfpathlineto{\pgfqpoint{5.937707in}{2.310403in}}%
\pgfpathlineto{\pgfqpoint{5.937737in}{1.760111in}}%
\pgfpathlineto{\pgfqpoint{5.937727in}{2.490508in}}%
\pgfpathlineto{\pgfqpoint{5.938812in}{2.366420in}}%
\pgfpathlineto{\pgfqpoint{5.939562in}{2.464274in}}%
\pgfpathlineto{\pgfqpoint{5.939023in}{1.805703in}}%
\pgfpathlineto{\pgfqpoint{5.939919in}{2.406267in}}%
\pgfpathlineto{\pgfqpoint{5.940466in}{1.879523in}}%
\pgfpathlineto{\pgfqpoint{5.940468in}{2.495731in}}%
\pgfpathlineto{\pgfqpoint{5.941032in}{2.252073in}}%
\pgfpathlineto{\pgfqpoint{5.941916in}{2.459546in}}%
\pgfpathlineto{\pgfqpoint{5.941688in}{1.760148in}}%
\pgfpathlineto{\pgfqpoint{5.942131in}{2.389136in}}%
\pgfpathlineto{\pgfqpoint{5.942475in}{1.717167in}}%
\pgfpathlineto{\pgfqpoint{5.942575in}{2.486591in}}%
\pgfpathlineto{\pgfqpoint{5.943241in}{2.364480in}}%
\pgfpathlineto{\pgfqpoint{5.944281in}{1.925768in}}%
\pgfpathlineto{\pgfqpoint{5.944235in}{2.525650in}}%
\pgfpathlineto{\pgfqpoint{5.944351in}{2.342848in}}%
\pgfpathlineto{\pgfqpoint{5.945247in}{2.460148in}}%
\pgfpathlineto{\pgfqpoint{5.944626in}{1.885941in}}%
\pgfpathlineto{\pgfqpoint{5.945451in}{2.439501in}}%
\pgfpathlineto{\pgfqpoint{5.945943in}{1.869268in}}%
\pgfpathlineto{\pgfqpoint{5.946301in}{2.494161in}}%
\pgfpathlineto{\pgfqpoint{5.946565in}{2.132580in}}%
\pgfpathlineto{\pgfqpoint{5.946865in}{2.490131in}}%
\pgfpathlineto{\pgfqpoint{5.947480in}{1.851095in}}%
\pgfpathlineto{\pgfqpoint{5.947678in}{2.390711in}}%
\pgfpathlineto{\pgfqpoint{5.948497in}{1.973645in}}%
\pgfpathlineto{\pgfqpoint{5.948755in}{2.453590in}}%
\pgfpathlineto{\pgfqpoint{5.948791in}{2.235981in}}%
\pgfpathlineto{\pgfqpoint{5.949613in}{2.470974in}}%
\pgfpathlineto{\pgfqpoint{5.949519in}{1.912336in}}%
\pgfpathlineto{\pgfqpoint{5.949901in}{2.437590in}}%
\pgfpathlineto{\pgfqpoint{5.950055in}{1.821514in}}%
\pgfpathlineto{\pgfqpoint{5.950266in}{2.494243in}}%
\pgfpathlineto{\pgfqpoint{5.951013in}{2.188729in}}%
\pgfpathlineto{\pgfqpoint{5.951869in}{2.458557in}}%
\pgfpathlineto{\pgfqpoint{5.951195in}{1.870099in}}%
\pgfpathlineto{\pgfqpoint{5.952127in}{2.377148in}}%
\pgfpathlineto{\pgfqpoint{5.953121in}{1.936217in}}%
\pgfpathlineto{\pgfqpoint{5.952816in}{2.474059in}}%
\pgfpathlineto{\pgfqpoint{5.953240in}{2.308079in}}%
\pgfpathlineto{\pgfqpoint{5.953887in}{2.504920in}}%
\pgfpathlineto{\pgfqpoint{5.953549in}{1.914989in}}%
\pgfpathlineto{\pgfqpoint{5.954348in}{2.360692in}}%
\pgfpathlineto{\pgfqpoint{5.954621in}{1.907142in}}%
\pgfpathlineto{\pgfqpoint{5.954613in}{2.458209in}}%
\pgfpathlineto{\pgfqpoint{5.955457in}{2.249335in}}%
\pgfpathlineto{\pgfqpoint{5.955498in}{2.493900in}}%
\pgfpathlineto{\pgfqpoint{5.955763in}{1.671344in}}%
\pgfpathlineto{\pgfqpoint{5.956569in}{2.349746in}}%
\pgfpathlineto{\pgfqpoint{5.957168in}{1.680034in}}%
\pgfpathlineto{\pgfqpoint{5.957489in}{2.467984in}}%
\pgfpathlineto{\pgfqpoint{5.957680in}{2.289889in}}%
\pgfpathlineto{\pgfqpoint{5.957821in}{2.453065in}}%
\pgfpathlineto{\pgfqpoint{5.958435in}{1.878738in}}%
\pgfpathlineto{\pgfqpoint{5.958795in}{2.397794in}}%
\pgfpathlineto{\pgfqpoint{5.959553in}{1.885741in}}%
\pgfpathlineto{\pgfqpoint{5.959408in}{2.458150in}}%
\pgfpathlineto{\pgfqpoint{5.959905in}{2.411881in}}%
\pgfpathlineto{\pgfqpoint{5.960769in}{2.456070in}}%
\pgfpathlineto{\pgfqpoint{5.960153in}{1.902395in}}%
\pgfpathlineto{\pgfqpoint{5.960897in}{2.285476in}}%
\pgfpathlineto{\pgfqpoint{5.960900in}{1.921832in}}%
\pgfpathlineto{\pgfqpoint{5.961465in}{2.474712in}}%
\pgfpathlineto{\pgfqpoint{5.962009in}{2.135023in}}%
\pgfpathlineto{\pgfqpoint{5.963043in}{2.461457in}}%
\pgfpathlineto{\pgfqpoint{5.962407in}{1.854560in}}%
\pgfpathlineto{\pgfqpoint{5.963120in}{2.191066in}}%
\pgfpathlineto{\pgfqpoint{5.964040in}{2.487538in}}%
\pgfpathlineto{\pgfqpoint{5.963749in}{1.730871in}}%
\pgfpathlineto{\pgfqpoint{5.964234in}{2.385626in}}%
\pgfpathlineto{\pgfqpoint{5.964858in}{1.946554in}}%
\pgfpathlineto{\pgfqpoint{5.965337in}{2.487886in}}%
\pgfpathlineto{\pgfqpoint{5.965349in}{2.222799in}}%
\pgfpathlineto{\pgfqpoint{5.966333in}{2.456345in}}%
\pgfpathlineto{\pgfqpoint{5.965682in}{1.731436in}}%
\pgfpathlineto{\pgfqpoint{5.966455in}{2.288394in}}%
\pgfpathlineto{\pgfqpoint{5.967417in}{1.911789in}}%
\pgfpathlineto{\pgfqpoint{5.967467in}{2.470547in}}%
\pgfpathlineto{\pgfqpoint{5.967564in}{2.290275in}}%
\pgfpathlineto{\pgfqpoint{5.967907in}{2.482981in}}%
\pgfpathlineto{\pgfqpoint{5.967649in}{1.664229in}}%
\pgfpathlineto{\pgfqpoint{5.968676in}{2.376405in}}%
\pgfpathlineto{\pgfqpoint{5.969016in}{1.978829in}}%
\pgfpathlineto{\pgfqpoint{5.969781in}{2.451446in}}%
\pgfpathlineto{\pgfqpoint{5.969788in}{2.247520in}}%
\pgfpathlineto{\pgfqpoint{5.970849in}{2.485462in}}%
\pgfpathlineto{\pgfqpoint{5.970049in}{1.701914in}}%
\pgfpathlineto{\pgfqpoint{5.970897in}{2.272317in}}%
\pgfpathlineto{\pgfqpoint{5.971659in}{1.862883in}}%
\pgfpathlineto{\pgfqpoint{5.971668in}{2.449401in}}%
\pgfpathlineto{\pgfqpoint{5.972003in}{2.229667in}}%
\pgfpathlineto{\pgfqpoint{5.973038in}{2.474194in}}%
\pgfpathlineto{\pgfqpoint{5.972938in}{1.659175in}}%
\pgfpathlineto{\pgfqpoint{5.973113in}{2.304647in}}%
\pgfpathlineto{\pgfqpoint{5.973424in}{1.926037in}}%
\pgfpathlineto{\pgfqpoint{5.973622in}{2.469122in}}%
\pgfpathlineto{\pgfqpoint{5.974216in}{2.189804in}}%
\pgfpathlineto{\pgfqpoint{5.974279in}{2.465547in}}%
\pgfpathlineto{\pgfqpoint{5.974327in}{1.733440in}}%
\pgfpathlineto{\pgfqpoint{5.975325in}{2.189929in}}%
\pgfpathlineto{\pgfqpoint{5.976216in}{1.663589in}}%
\pgfpathlineto{\pgfqpoint{5.975451in}{2.446539in}}%
\pgfpathlineto{\pgfqpoint{5.976435in}{2.214140in}}%
\pgfpathlineto{\pgfqpoint{5.977201in}{1.900891in}}%
\pgfpathlineto{\pgfqpoint{5.977115in}{2.488608in}}%
\pgfpathlineto{\pgfqpoint{5.977522in}{2.248667in}}%
\pgfpathlineto{\pgfqpoint{5.978007in}{2.471051in}}%
\pgfpathlineto{\pgfqpoint{5.977877in}{1.855127in}}%
\pgfpathlineto{\pgfqpoint{5.978634in}{2.348129in}}%
\pgfpathlineto{\pgfqpoint{5.978752in}{2.469343in}}%
\pgfpathlineto{\pgfqpoint{5.979191in}{1.800990in}}%
\pgfpathlineto{\pgfqpoint{5.979733in}{2.355019in}}%
\pgfpathlineto{\pgfqpoint{5.980316in}{1.765797in}}%
\pgfpathlineto{\pgfqpoint{5.980397in}{2.483462in}}%
\pgfpathlineto{\pgfqpoint{5.980843in}{2.381011in}}%
\pgfpathlineto{\pgfqpoint{5.981298in}{1.806973in}}%
\pgfpathlineto{\pgfqpoint{5.981242in}{2.468880in}}%
\pgfpathlineto{\pgfqpoint{5.981955in}{2.334184in}}%
\pgfpathlineto{\pgfqpoint{5.982986in}{1.828315in}}%
\pgfpathlineto{\pgfqpoint{5.982251in}{2.469386in}}%
\pgfpathlineto{\pgfqpoint{5.983057in}{2.272704in}}%
\pgfpathlineto{\pgfqpoint{5.983549in}{2.458195in}}%
\pgfpathlineto{\pgfqpoint{5.983742in}{1.680977in}}%
\pgfpathlineto{\pgfqpoint{5.984165in}{2.239328in}}%
\pgfpathlineto{\pgfqpoint{5.984364in}{1.819447in}}%
\pgfpathlineto{\pgfqpoint{5.984408in}{2.473543in}}%
\pgfpathlineto{\pgfqpoint{5.985270in}{2.375416in}}%
\pgfpathlineto{\pgfqpoint{5.985447in}{1.838564in}}%
\pgfpathlineto{\pgfqpoint{5.985460in}{2.456101in}}%
\pgfpathlineto{\pgfqpoint{5.986381in}{2.331657in}}%
\pgfpathlineto{\pgfqpoint{5.986634in}{2.493719in}}%
\pgfpathlineto{\pgfqpoint{5.987232in}{1.808556in}}%
\pgfpathlineto{\pgfqpoint{5.987489in}{2.318492in}}%
\pgfpathlineto{\pgfqpoint{5.987516in}{1.799333in}}%
\pgfpathlineto{\pgfqpoint{5.987693in}{2.456666in}}%
\pgfpathlineto{\pgfqpoint{5.988599in}{2.244811in}}%
\pgfpathlineto{\pgfqpoint{5.988842in}{2.456193in}}%
\pgfpathlineto{\pgfqpoint{5.989078in}{1.857397in}}%
\pgfpathlineto{\pgfqpoint{5.989710in}{2.331475in}}%
\pgfpathlineto{\pgfqpoint{5.989918in}{1.725373in}}%
\pgfpathlineto{\pgfqpoint{5.990265in}{2.463769in}}%
\pgfpathlineto{\pgfqpoint{5.990818in}{2.276722in}}%
\pgfpathlineto{\pgfqpoint{5.990990in}{2.448839in}}%
\pgfpathlineto{\pgfqpoint{5.990823in}{1.651572in}}%
\pgfpathlineto{\pgfqpoint{5.991926in}{2.422290in}}%
\pgfpathlineto{\pgfqpoint{5.992590in}{1.853767in}}%
\pgfpathlineto{\pgfqpoint{5.992082in}{2.475193in}}%
\pgfpathlineto{\pgfqpoint{5.993037in}{2.211974in}}%
\pgfpathlineto{\pgfqpoint{5.993070in}{2.470342in}}%
\pgfpathlineto{\pgfqpoint{5.993347in}{1.525458in}}%
\pgfpathlineto{\pgfqpoint{5.994148in}{2.368346in}}%
\pgfpathlineto{\pgfqpoint{5.995195in}{1.748610in}}%
\pgfpathlineto{\pgfqpoint{5.994979in}{2.466289in}}%
\pgfpathlineto{\pgfqpoint{5.995257in}{2.115261in}}%
\pgfpathlineto{\pgfqpoint{5.995611in}{2.462411in}}%
\pgfpathlineto{\pgfqpoint{5.995809in}{1.930526in}}%
\pgfpathlineto{\pgfqpoint{5.996368in}{2.310265in}}%
\pgfpathlineto{\pgfqpoint{5.997356in}{2.481745in}}%
\pgfpathlineto{\pgfqpoint{5.997457in}{1.978531in}}%
\pgfpathlineto{\pgfqpoint{5.997479in}{2.300788in}}%
\pgfpathlineto{\pgfqpoint{5.998184in}{1.833719in}}%
\pgfpathlineto{\pgfqpoint{5.998441in}{2.459797in}}%
\pgfpathlineto{\pgfqpoint{5.998588in}{2.242444in}}%
\pgfpathlineto{\pgfqpoint{5.998853in}{2.515128in}}%
\pgfpathlineto{\pgfqpoint{5.998838in}{1.779617in}}%
\pgfpathlineto{\pgfqpoint{5.999695in}{2.247593in}}%
\pgfpathlineto{\pgfqpoint{6.000143in}{1.933976in}}%
\pgfpathlineto{\pgfqpoint{6.000469in}{2.447886in}}%
\pgfpathlineto{\pgfqpoint{6.000805in}{2.236264in}}%
\pgfpathlineto{\pgfqpoint{6.001791in}{2.449127in}}%
\pgfpathlineto{\pgfqpoint{6.001864in}{1.862887in}}%
\pgfpathlineto{\pgfqpoint{6.001912in}{2.239860in}}%
\pgfpathlineto{\pgfqpoint{6.002436in}{1.801316in}}%
\pgfpathlineto{\pgfqpoint{6.002852in}{2.458581in}}%
\pgfpathlineto{\pgfqpoint{6.003021in}{1.951087in}}%
\pgfpathlineto{\pgfqpoint{6.003527in}{2.447234in}}%
\pgfpathlineto{\pgfqpoint{6.003926in}{1.798232in}}%
\pgfpathlineto{\pgfqpoint{6.004133in}{2.373020in}}%
\pgfpathlineto{\pgfqpoint{6.004958in}{1.903195in}}%
\pgfpathlineto{\pgfqpoint{6.005217in}{2.456230in}}%
\pgfpathlineto{\pgfqpoint{6.005242in}{2.273289in}}%
\pgfpathlineto{\pgfqpoint{6.005278in}{2.453448in}}%
\pgfpathlineto{\pgfqpoint{6.005257in}{1.792047in}}%
\pgfpathlineto{\pgfqpoint{6.006351in}{2.315168in}}%
\pgfpathlineto{\pgfqpoint{6.007010in}{1.807440in}}%
\pgfpathlineto{\pgfqpoint{6.007314in}{2.476448in}}%
\pgfpathlineto{\pgfqpoint{6.007463in}{2.199669in}}%
\pgfpathlineto{\pgfqpoint{6.007521in}{1.819741in}}%
\pgfpathlineto{\pgfqpoint{6.008214in}{2.493132in}}%
\pgfpathlineto{\pgfqpoint{6.008563in}{2.178444in}}%
\pgfpathlineto{\pgfqpoint{6.009567in}{2.464657in}}%
\pgfpathlineto{\pgfqpoint{6.008599in}{1.851389in}}%
\pgfpathlineto{\pgfqpoint{6.009675in}{2.310060in}}%
\pgfpathlineto{\pgfqpoint{6.010568in}{2.443908in}}%
\pgfpathlineto{\pgfqpoint{6.010363in}{1.839345in}}%
\pgfpathlineto{\pgfqpoint{6.010781in}{2.275394in}}%
\pgfpathlineto{\pgfqpoint{6.011632in}{1.829507in}}%
\pgfpathlineto{\pgfqpoint{6.011507in}{2.456052in}}%
\pgfpathlineto{\pgfqpoint{6.011891in}{2.295689in}}%
\pgfpathlineto{\pgfqpoint{6.012678in}{2.478897in}}%
\pgfpathlineto{\pgfqpoint{6.012024in}{1.876505in}}%
\pgfpathlineto{\pgfqpoint{6.012999in}{2.200788in}}%
\pgfpathlineto{\pgfqpoint{6.013699in}{2.445987in}}%
\pgfpathlineto{\pgfqpoint{6.013675in}{1.868821in}}%
\pgfpathlineto{\pgfqpoint{6.013926in}{2.254387in}}%
\pgfpathlineto{\pgfqpoint{6.013928in}{1.756684in}}%
\pgfpathlineto{\pgfqpoint{6.014351in}{2.464265in}}%
\pgfpathlineto{\pgfqpoint{6.015037in}{2.271939in}}%
\pgfpathlineto{\pgfqpoint{6.015084in}{1.844001in}}%
\pgfpathlineto{\pgfqpoint{6.015896in}{2.471170in}}%
\pgfpathlineto{\pgfqpoint{6.016149in}{2.153816in}}%
\pgfpathlineto{\pgfqpoint{6.016912in}{2.453462in}}%
\pgfpathlineto{\pgfqpoint{6.017133in}{1.715500in}}%
\pgfpathlineto{\pgfqpoint{6.017262in}{2.298327in}}%
\pgfpathlineto{\pgfqpoint{6.018286in}{1.570888in}}%
\pgfpathlineto{\pgfqpoint{6.017573in}{2.479722in}}%
\pgfpathlineto{\pgfqpoint{6.018372in}{2.297503in}}%
\pgfpathlineto{\pgfqpoint{6.018825in}{2.450223in}}%
\pgfpathlineto{\pgfqpoint{6.018505in}{1.929299in}}%
\pgfpathlineto{\pgfqpoint{6.019485in}{2.406073in}}%
\pgfpathlineto{\pgfqpoint{6.019899in}{1.820223in}}%
\pgfpathlineto{\pgfqpoint{6.019907in}{2.462148in}}%
\pgfpathlineto{\pgfqpoint{6.020598in}{2.095794in}}%
\pgfpathlineto{\pgfqpoint{6.021622in}{2.461693in}}%
\pgfpathlineto{\pgfqpoint{6.021284in}{1.929596in}}%
\pgfpathlineto{\pgfqpoint{6.021710in}{2.230050in}}%
\pgfpathlineto{\pgfqpoint{6.022553in}{2.462268in}}%
\pgfpathlineto{\pgfqpoint{6.022630in}{1.879988in}}%
\pgfpathlineto{\pgfqpoint{6.022820in}{2.309402in}}%
\pgfpathlineto{\pgfqpoint{6.023474in}{1.945523in}}%
\pgfpathlineto{\pgfqpoint{6.022863in}{2.458073in}}%
\pgfpathlineto{\pgfqpoint{6.023930in}{2.147685in}}%
\pgfpathlineto{\pgfqpoint{6.024256in}{2.431961in}}%
\pgfpathlineto{\pgfqpoint{6.024236in}{1.791491in}}%
\pgfpathlineto{\pgfqpoint{6.025043in}{2.374010in}}%
\pgfpathlineto{\pgfqpoint{6.025369in}{1.826224in}}%
\pgfpathlineto{\pgfqpoint{6.025810in}{2.470319in}}%
\pgfpathlineto{\pgfqpoint{6.026157in}{2.151735in}}%
\pgfpathlineto{\pgfqpoint{6.027264in}{2.467111in}}%
\pgfpathlineto{\pgfqpoint{6.026209in}{1.888474in}}%
\pgfpathlineto{\pgfqpoint{6.027270in}{2.306950in}}%
\pgfpathlineto{\pgfqpoint{6.027596in}{2.454259in}}%
\pgfpathlineto{\pgfqpoint{6.027630in}{1.813919in}}%
\pgfpathlineto{\pgfqpoint{6.028379in}{2.224228in}}%
\pgfpathlineto{\pgfqpoint{6.029398in}{1.733265in}}%
\pgfpathlineto{\pgfqpoint{6.029191in}{2.433179in}}%
\pgfpathlineto{\pgfqpoint{6.029454in}{2.359443in}}%
\pgfpathlineto{\pgfqpoint{6.030277in}{2.447984in}}%
\pgfpathlineto{\pgfqpoint{6.030257in}{1.855151in}}%
\pgfpathlineto{\pgfqpoint{6.030559in}{2.241031in}}%
\pgfpathlineto{\pgfqpoint{6.030703in}{1.937110in}}%
\pgfpathlineto{\pgfqpoint{6.031605in}{2.474329in}}%
\pgfpathlineto{\pgfqpoint{6.031668in}{2.379855in}}%
\pgfpathlineto{\pgfqpoint{6.032775in}{1.797006in}}%
\pgfpathlineto{\pgfqpoint{6.032704in}{2.466942in}}%
\pgfpathlineto{\pgfqpoint{6.032781in}{2.261310in}}%
\pgfpathlineto{\pgfqpoint{6.033615in}{2.463320in}}%
\pgfpathlineto{\pgfqpoint{6.032917in}{1.893624in}}%
\pgfpathlineto{\pgfqpoint{6.033892in}{2.376479in}}%
\pgfpathlineto{\pgfqpoint{6.033963in}{1.759746in}}%
\pgfpathlineto{\pgfqpoint{6.034716in}{2.445702in}}%
\pgfpathlineto{\pgfqpoint{6.035005in}{2.317065in}}%
\pgfpathlineto{\pgfqpoint{6.035447in}{1.937451in}}%
\pgfpathlineto{\pgfqpoint{6.035360in}{2.455940in}}%
\pgfpathlineto{\pgfqpoint{6.036117in}{2.291197in}}%
\pgfpathlineto{\pgfqpoint{6.036696in}{2.440625in}}%
\pgfpathlineto{\pgfqpoint{6.037202in}{1.890999in}}%
\pgfpathlineto{\pgfqpoint{6.037226in}{2.303548in}}%
\pgfpathlineto{\pgfqpoint{6.037606in}{1.881444in}}%
\pgfpathlineto{\pgfqpoint{6.037411in}{2.450756in}}%
\pgfpathlineto{\pgfqpoint{6.038338in}{2.197479in}}%
\pgfpathlineto{\pgfqpoint{6.038811in}{2.472289in}}%
\pgfpathlineto{\pgfqpoint{6.038669in}{1.911862in}}%
\pgfpathlineto{\pgfqpoint{6.039444in}{2.207898in}}%
\pgfpathlineto{\pgfqpoint{6.039509in}{1.701109in}}%
\pgfpathlineto{\pgfqpoint{6.039982in}{2.455147in}}%
\pgfpathlineto{\pgfqpoint{6.040554in}{2.212449in}}%
\pgfpathlineto{\pgfqpoint{6.040798in}{2.452337in}}%
\pgfpathlineto{\pgfqpoint{6.041350in}{1.800029in}}%
\pgfpathlineto{\pgfqpoint{6.041667in}{2.360318in}}%
\pgfpathlineto{\pgfqpoint{6.042482in}{1.850951in}}%
\pgfpathlineto{\pgfqpoint{6.041864in}{2.465039in}}%
\pgfpathlineto{\pgfqpoint{6.042779in}{2.248374in}}%
\pgfpathlineto{\pgfqpoint{6.043567in}{2.460919in}}%
\pgfpathlineto{\pgfqpoint{6.043613in}{1.849766in}}%
\pgfpathlineto{\pgfqpoint{6.043880in}{2.213275in}}%
\pgfpathlineto{\pgfqpoint{6.044950in}{1.515435in}}%
\pgfpathlineto{\pgfqpoint{6.044326in}{2.443542in}}%
\pgfpathlineto{\pgfqpoint{6.044990in}{2.338253in}}%
\pgfpathlineto{\pgfqpoint{6.045892in}{1.709490in}}%
\pgfpathlineto{\pgfqpoint{6.045640in}{2.450458in}}%
\pgfpathlineto{\pgfqpoint{6.046099in}{2.297491in}}%
\pgfpathlineto{\pgfqpoint{6.046749in}{2.443765in}}%
\pgfpathlineto{\pgfqpoint{6.046267in}{1.867281in}}%
\pgfpathlineto{\pgfqpoint{6.047207in}{2.387342in}}%
\pgfpathlineto{\pgfqpoint{6.048154in}{1.865397in}}%
\pgfpathlineto{\pgfqpoint{6.047683in}{2.459615in}}%
\pgfpathlineto{\pgfqpoint{6.048319in}{2.260603in}}%
\pgfpathlineto{\pgfqpoint{6.048407in}{2.457315in}}%
\pgfpathlineto{\pgfqpoint{6.048802in}{1.816304in}}%
\pgfpathlineto{\pgfqpoint{6.049427in}{2.235547in}}%
\pgfpathlineto{\pgfqpoint{6.050286in}{1.689970in}}%
\pgfpathlineto{\pgfqpoint{6.049549in}{2.469990in}}%
\pgfpathlineto{\pgfqpoint{6.050536in}{2.192581in}}%
\pgfpathlineto{\pgfqpoint{6.051024in}{2.456779in}}%
\pgfpathlineto{\pgfqpoint{6.051109in}{1.803854in}}%
\pgfpathlineto{\pgfqpoint{6.051649in}{2.305825in}}%
\pgfpathlineto{\pgfqpoint{6.052479in}{1.820737in}}%
\pgfpathlineto{\pgfqpoint{6.052268in}{2.457146in}}%
\pgfpathlineto{\pgfqpoint{6.052754in}{2.273810in}}%
\pgfpathlineto{\pgfqpoint{6.052952in}{2.442429in}}%
\pgfpathlineto{\pgfqpoint{6.053768in}{1.829377in}}%
\pgfpathlineto{\pgfqpoint{6.053862in}{2.414888in}}%
\pgfpathlineto{\pgfqpoint{6.053993in}{1.620040in}}%
\pgfpathlineto{\pgfqpoint{6.054895in}{2.450891in}}%
\pgfpathlineto{\pgfqpoint{6.054972in}{2.250403in}}%
\pgfpathlineto{\pgfqpoint{6.055403in}{2.471551in}}%
\pgfpathlineto{\pgfqpoint{6.055524in}{1.825528in}}%
\pgfpathlineto{\pgfqpoint{6.056085in}{2.346758in}}%
\pgfpathlineto{\pgfqpoint{6.056988in}{1.762133in}}%
\pgfpathlineto{\pgfqpoint{6.057001in}{2.465501in}}%
\pgfpathlineto{\pgfqpoint{6.057199in}{2.146410in}}%
\pgfpathlineto{\pgfqpoint{6.057793in}{2.444897in}}%
\pgfpathlineto{\pgfqpoint{6.057616in}{1.976918in}}%
\pgfpathlineto{\pgfqpoint{6.058310in}{2.264474in}}%
\pgfpathlineto{\pgfqpoint{6.058503in}{1.831669in}}%
\pgfpathlineto{\pgfqpoint{6.058837in}{2.478561in}}%
\pgfpathlineto{\pgfqpoint{6.059379in}{2.141279in}}%
\pgfpathlineto{\pgfqpoint{6.059918in}{2.447752in}}%
\pgfpathlineto{\pgfqpoint{6.060079in}{1.802390in}}%
\pgfpathlineto{\pgfqpoint{6.060490in}{2.104985in}}%
\pgfpathlineto{\pgfqpoint{6.060689in}{2.442256in}}%
\pgfpathlineto{\pgfqpoint{6.060558in}{1.761750in}}%
\pgfpathlineto{\pgfqpoint{6.061601in}{2.311571in}}%
\pgfpathlineto{\pgfqpoint{6.062484in}{2.437353in}}%
\pgfpathlineto{\pgfqpoint{6.062508in}{1.839074in}}%
\pgfpathlineto{\pgfqpoint{6.062695in}{2.261351in}}%
\pgfpathlineto{\pgfqpoint{6.063316in}{1.812715in}}%
\pgfpathlineto{\pgfqpoint{6.063425in}{2.448770in}}%
\pgfpathlineto{\pgfqpoint{6.063805in}{2.058653in}}%
\pgfpathlineto{\pgfqpoint{6.064862in}{2.462461in}}%
\pgfpathlineto{\pgfqpoint{6.064102in}{1.726777in}}%
\pgfpathlineto{\pgfqpoint{6.064919in}{2.394907in}}%
\pgfpathlineto{\pgfqpoint{6.066019in}{1.785346in}}%
\pgfpathlineto{\pgfqpoint{6.065204in}{2.458056in}}%
\pgfpathlineto{\pgfqpoint{6.066030in}{2.306856in}}%
\pgfpathlineto{\pgfqpoint{6.066848in}{1.827137in}}%
\pgfpathlineto{\pgfqpoint{6.066039in}{2.447136in}}%
\pgfpathlineto{\pgfqpoint{6.067140in}{2.211908in}}%
\pgfpathlineto{\pgfqpoint{6.067465in}{2.448346in}}%
\pgfpathlineto{\pgfqpoint{6.067978in}{1.712599in}}%
\pgfpathlineto{\pgfqpoint{6.068248in}{2.219257in}}%
\pgfpathlineto{\pgfqpoint{6.069157in}{1.720768in}}%
\pgfpathlineto{\pgfqpoint{6.069180in}{2.464224in}}%
\pgfpathlineto{\pgfqpoint{6.069356in}{2.282580in}}%
\pgfpathlineto{\pgfqpoint{6.069928in}{2.437896in}}%
\pgfpathlineto{\pgfqpoint{6.069964in}{1.754004in}}%
\pgfpathlineto{\pgfqpoint{6.070457in}{2.168432in}}%
\pgfpathlineto{\pgfqpoint{6.070533in}{1.749606in}}%
\pgfpathlineto{\pgfqpoint{6.070954in}{2.447056in}}%
\pgfpathlineto{\pgfqpoint{6.071565in}{2.257380in}}%
\pgfpathlineto{\pgfqpoint{6.072103in}{2.438602in}}%
\pgfpathlineto{\pgfqpoint{6.071666in}{1.789796in}}%
\pgfpathlineto{\pgfqpoint{6.072677in}{2.297918in}}%
\pgfpathlineto{\pgfqpoint{6.073253in}{1.771283in}}%
\pgfpathlineto{\pgfqpoint{6.073021in}{2.490703in}}%
\pgfpathlineto{\pgfqpoint{6.073787in}{2.277564in}}%
\pgfpathlineto{\pgfqpoint{6.074140in}{2.451113in}}%
\pgfpathlineto{\pgfqpoint{6.074657in}{1.860618in}}%
\pgfpathlineto{\pgfqpoint{6.074893in}{2.273779in}}%
\pgfpathlineto{\pgfqpoint{6.074945in}{1.835475in}}%
\pgfpathlineto{\pgfqpoint{6.075711in}{2.454991in}}%
\pgfpathlineto{\pgfqpoint{6.076003in}{2.210691in}}%
\pgfpathlineto{\pgfqpoint{6.076112in}{2.446628in}}%
\pgfpathlineto{\pgfqpoint{6.076541in}{1.813692in}}%
\pgfpathlineto{\pgfqpoint{6.077116in}{2.367265in}}%
\pgfpathlineto{\pgfqpoint{6.077402in}{1.842236in}}%
\pgfpathlineto{\pgfqpoint{6.078156in}{2.445570in}}%
\pgfpathlineto{\pgfqpoint{6.078227in}{2.319544in}}%
\pgfpathlineto{\pgfqpoint{6.078855in}{2.436270in}}%
\pgfpathlineto{\pgfqpoint{6.078807in}{1.906491in}}%
\pgfpathlineto{\pgfqpoint{6.079311in}{2.120027in}}%
\pgfpathlineto{\pgfqpoint{6.079946in}{1.860405in}}%
\pgfpathlineto{\pgfqpoint{6.079592in}{2.450977in}}%
\pgfpathlineto{\pgfqpoint{6.080419in}{2.132892in}}%
\pgfpathlineto{\pgfqpoint{6.080856in}{2.449859in}}%
\pgfpathlineto{\pgfqpoint{6.081328in}{1.840937in}}%
\pgfpathlineto{\pgfqpoint{6.081530in}{2.250868in}}%
\pgfpathlineto{\pgfqpoint{6.082060in}{2.444944in}}%
\pgfpathlineto{\pgfqpoint{6.082084in}{1.888678in}}%
\pgfpathlineto{\pgfqpoint{6.082641in}{2.237139in}}%
\pgfpathlineto{\pgfqpoint{6.082911in}{2.438400in}}%
\pgfpathlineto{\pgfqpoint{6.083524in}{1.839517in}}%
\pgfpathlineto{\pgfqpoint{6.083752in}{2.379051in}}%
\pgfpathlineto{\pgfqpoint{6.084160in}{1.915381in}}%
\pgfpathlineto{\pgfqpoint{6.084202in}{2.450534in}}%
\pgfpathlineto{\pgfqpoint{6.084862in}{2.197210in}}%
\pgfpathlineto{\pgfqpoint{6.084984in}{2.450374in}}%
\pgfpathlineto{\pgfqpoint{6.085838in}{1.902834in}}%
\pgfpathlineto{\pgfqpoint{6.085971in}{2.349529in}}%
\pgfpathlineto{\pgfqpoint{6.086314in}{1.845608in}}%
\pgfpathlineto{\pgfqpoint{6.086974in}{2.446575in}}%
\pgfpathlineto{\pgfqpoint{6.087082in}{2.317064in}}%
\pgfpathlineto{\pgfqpoint{6.087297in}{2.471608in}}%
\pgfpathlineto{\pgfqpoint{6.087363in}{1.811572in}}%
\pgfpathlineto{\pgfqpoint{6.088180in}{2.334003in}}%
\pgfpathlineto{\pgfqpoint{6.088192in}{1.834206in}}%
\pgfpathlineto{\pgfqpoint{6.088206in}{2.459957in}}%
\pgfpathlineto{\pgfqpoint{6.089290in}{2.395091in}}%
\pgfpathlineto{\pgfqpoint{6.089637in}{2.439141in}}%
\pgfpathlineto{\pgfqpoint{6.089739in}{1.798775in}}%
\pgfpathlineto{\pgfqpoint{6.090382in}{2.374295in}}%
\pgfpathlineto{\pgfqpoint{6.091178in}{1.836270in}}%
\pgfpathlineto{\pgfqpoint{6.091267in}{2.455098in}}%
\pgfpathlineto{\pgfqpoint{6.091496in}{2.214895in}}%
\pgfpathlineto{\pgfqpoint{6.091747in}{2.464156in}}%
\pgfpathlineto{\pgfqpoint{6.092467in}{1.849212in}}%
\pgfpathlineto{\pgfqpoint{6.092605in}{2.283430in}}%
\pgfpathlineto{\pgfqpoint{6.093256in}{1.748860in}}%
\pgfpathlineto{\pgfqpoint{6.092819in}{2.428158in}}%
\pgfpathlineto{\pgfqpoint{6.093712in}{2.306980in}}%
\pgfpathlineto{\pgfqpoint{6.094559in}{2.461408in}}%
\pgfpathlineto{\pgfqpoint{6.094016in}{1.806541in}}%
\pgfpathlineto{\pgfqpoint{6.094822in}{2.343797in}}%
\pgfpathlineto{\pgfqpoint{6.095212in}{1.842017in}}%
\pgfpathlineto{\pgfqpoint{6.095149in}{2.447030in}}%
\pgfpathlineto{\pgfqpoint{6.095932in}{2.205979in}}%
\pgfpathlineto{\pgfqpoint{6.096430in}{2.433096in}}%
\pgfpathlineto{\pgfqpoint{6.096957in}{1.801413in}}%
\pgfpathlineto{\pgfqpoint{6.097043in}{2.285507in}}%
\pgfpathlineto{\pgfqpoint{6.097980in}{1.783163in}}%
\pgfpathlineto{\pgfqpoint{6.097929in}{2.438977in}}%
\pgfpathlineto{\pgfqpoint{6.098153in}{2.194096in}}%
\pgfpathlineto{\pgfqpoint{6.099059in}{2.446080in}}%
\pgfpathlineto{\pgfqpoint{6.098368in}{1.856825in}}%
\pgfpathlineto{\pgfqpoint{6.099263in}{2.301706in}}%
\pgfpathlineto{\pgfqpoint{6.099983in}{1.675388in}}%
\pgfpathlineto{\pgfqpoint{6.099710in}{2.452628in}}%
\pgfpathlineto{\pgfqpoint{6.100372in}{1.977646in}}%
\pgfpathlineto{\pgfqpoint{6.100920in}{2.449148in}}%
\pgfpathlineto{\pgfqpoint{6.100501in}{1.811447in}}%
\pgfpathlineto{\pgfqpoint{6.101485in}{2.333741in}}%
\pgfpathlineto{\pgfqpoint{6.101530in}{1.791570in}}%
\pgfpathlineto{\pgfqpoint{6.101832in}{2.465114in}}%
\pgfpathlineto{\pgfqpoint{6.102598in}{2.195065in}}%
\pgfpathlineto{\pgfqpoint{6.102845in}{2.443631in}}%
\pgfpathlineto{\pgfqpoint{6.103193in}{1.660600in}}%
\pgfpathlineto{\pgfqpoint{6.103710in}{2.402737in}}%
\pgfpathlineto{\pgfqpoint{6.103773in}{1.887418in}}%
\pgfpathlineto{\pgfqpoint{6.103815in}{2.415872in}}%
\pgfpathlineto{\pgfqpoint{6.104822in}{2.090215in}}%
\pgfpathlineto{\pgfqpoint{6.104912in}{2.433634in}}%
\pgfpathlineto{\pgfqpoint{6.105684in}{1.762400in}}%
\pgfpathlineto{\pgfqpoint{6.105935in}{2.312149in}}%
\pgfpathlineto{\pgfqpoint{6.106337in}{1.857679in}}%
\pgfpathlineto{\pgfqpoint{6.106695in}{2.458510in}}%
\pgfpathlineto{\pgfqpoint{6.107045in}{2.357567in}}%
\pgfpathlineto{\pgfqpoint{6.107904in}{1.746550in}}%
\pgfpathlineto{\pgfqpoint{6.107420in}{2.446586in}}%
\pgfpathlineto{\pgfqpoint{6.108156in}{1.959145in}}%
\pgfpathlineto{\pgfqpoint{6.108277in}{2.435314in}}%
\pgfpathlineto{\pgfqpoint{6.108741in}{1.885739in}}%
\pgfpathlineto{\pgfqpoint{6.109267in}{2.038272in}}%
\pgfpathlineto{\pgfqpoint{6.110224in}{2.476637in}}%
\pgfpathlineto{\pgfqpoint{6.109555in}{1.825897in}}%
\pgfpathlineto{\pgfqpoint{6.110380in}{2.292644in}}%
\pgfpathlineto{\pgfqpoint{6.111299in}{1.847465in}}%
\pgfpathlineto{\pgfqpoint{6.111322in}{2.438379in}}%
\pgfpathlineto{\pgfqpoint{6.111491in}{2.231209in}}%
\pgfpathlineto{\pgfqpoint{6.111870in}{2.454414in}}%
\pgfpathlineto{\pgfqpoint{6.111932in}{1.898679in}}%
\pgfpathlineto{\pgfqpoint{6.112603in}{2.326773in}}%
\pgfpathlineto{\pgfqpoint{6.113131in}{1.732713in}}%
\pgfpathlineto{\pgfqpoint{6.113046in}{2.449453in}}%
\pgfpathlineto{\pgfqpoint{6.113713in}{2.247672in}}%
\pgfpathlineto{\pgfqpoint{6.114632in}{2.477643in}}%
\pgfpathlineto{\pgfqpoint{6.114645in}{1.872485in}}%
\pgfpathlineto{\pgfqpoint{6.114824in}{2.269961in}}%
\pgfpathlineto{\pgfqpoint{6.115819in}{1.936673in}}%
\pgfpathlineto{\pgfqpoint{6.115268in}{2.440282in}}%
\pgfpathlineto{\pgfqpoint{6.115934in}{2.323056in}}%
\pgfpathlineto{\pgfqpoint{6.116281in}{1.888146in}}%
\pgfpathlineto{\pgfqpoint{6.116990in}{2.460838in}}%
\pgfpathlineto{\pgfqpoint{6.117047in}{2.124697in}}%
\pgfpathlineto{\pgfqpoint{6.117434in}{2.449356in}}%
\pgfpathlineto{\pgfqpoint{6.117980in}{1.589150in}}%
\pgfpathlineto{\pgfqpoint{6.118157in}{2.305127in}}%
\pgfpathlineto{\pgfqpoint{6.118588in}{1.776077in}}%
\pgfpathlineto{\pgfqpoint{6.118859in}{2.436879in}}%
\pgfpathlineto{\pgfqpoint{6.119268in}{2.162066in}}%
\pgfpathlineto{\pgfqpoint{6.120295in}{2.436932in}}%
\pgfpathlineto{\pgfqpoint{6.119578in}{1.684417in}}%
\pgfpathlineto{\pgfqpoint{6.120380in}{2.309482in}}%
\pgfpathlineto{\pgfqpoint{6.120777in}{1.627007in}}%
\pgfpathlineto{\pgfqpoint{6.120942in}{2.435300in}}%
\pgfpathlineto{\pgfqpoint{6.121490in}{2.325938in}}%
\pgfpathlineto{\pgfqpoint{6.122352in}{2.437321in}}%
\pgfpathlineto{\pgfqpoint{6.122405in}{1.804200in}}%
\pgfpathlineto{\pgfqpoint{6.122601in}{2.377309in}}%
\pgfpathlineto{\pgfqpoint{6.122975in}{1.882893in}}%
\pgfpathlineto{\pgfqpoint{6.122803in}{2.444919in}}%
\pgfpathlineto{\pgfqpoint{6.123715in}{2.316953in}}%
\pgfpathlineto{\pgfqpoint{6.124644in}{2.439580in}}%
\pgfpathlineto{\pgfqpoint{6.124731in}{1.841527in}}%
\pgfpathlineto{\pgfqpoint{6.124824in}{2.367206in}}%
\pgfpathlineto{\pgfqpoint{6.125019in}{1.814782in}}%
\pgfpathlineto{\pgfqpoint{6.125305in}{2.445331in}}%
\pgfpathlineto{\pgfqpoint{6.125936in}{2.277299in}}%
\pgfpathlineto{\pgfqpoint{6.126716in}{2.460444in}}%
\pgfpathlineto{\pgfqpoint{6.126980in}{1.718677in}}%
\pgfpathlineto{\pgfqpoint{6.127042in}{2.342329in}}%
\pgfpathlineto{\pgfqpoint{6.127477in}{1.718605in}}%
\pgfpathlineto{\pgfqpoint{6.127505in}{2.435606in}}%
\pgfpathlineto{\pgfqpoint{6.128153in}{2.170234in}}%
\pgfpathlineto{\pgfqpoint{6.128256in}{2.441736in}}%
\pgfpathlineto{\pgfqpoint{6.128897in}{1.941274in}}%
\pgfpathlineto{\pgfqpoint{6.129265in}{2.243011in}}%
\pgfpathlineto{\pgfqpoint{6.129945in}{1.820132in}}%
\pgfpathlineto{\pgfqpoint{6.129812in}{2.427914in}}%
\pgfpathlineto{\pgfqpoint{6.130375in}{2.285809in}}%
\pgfpathlineto{\pgfqpoint{6.131190in}{1.750693in}}%
\pgfpathlineto{\pgfqpoint{6.131196in}{2.430245in}}%
\pgfpathlineto{\pgfqpoint{6.131485in}{2.254525in}}%
\pgfpathlineto{\pgfqpoint{6.132298in}{2.451141in}}%
\pgfpathlineto{\pgfqpoint{6.132360in}{1.450211in}}%
\pgfpathlineto{\pgfqpoint{6.132594in}{2.356650in}}%
\pgfpathlineto{\pgfqpoint{6.133000in}{1.722163in}}%
\pgfpathlineto{\pgfqpoint{6.133427in}{2.450955in}}%
\pgfpathlineto{\pgfqpoint{6.133705in}{2.238417in}}%
\pgfpathlineto{\pgfqpoint{6.134506in}{2.444683in}}%
\pgfpathlineto{\pgfqpoint{6.134153in}{1.681000in}}%
\pgfpathlineto{\pgfqpoint{6.134816in}{2.365667in}}%
\pgfpathlineto{\pgfqpoint{6.135777in}{1.652881in}}%
\pgfpathlineto{\pgfqpoint{6.135686in}{2.435121in}}%
\pgfpathlineto{\pgfqpoint{6.135926in}{2.356963in}}%
\pgfpathlineto{\pgfqpoint{6.136469in}{2.442240in}}%
\pgfpathlineto{\pgfqpoint{6.136105in}{1.669316in}}%
\pgfpathlineto{\pgfqpoint{6.137016in}{2.347727in}}%
\pgfpathlineto{\pgfqpoint{6.137881in}{1.716211in}}%
\pgfpathlineto{\pgfqpoint{6.137317in}{2.439794in}}%
\pgfpathlineto{\pgfqpoint{6.138126in}{2.250550in}}%
\pgfpathlineto{\pgfqpoint{6.139007in}{1.778838in}}%
\pgfpathlineto{\pgfqpoint{6.139050in}{2.441692in}}%
\pgfpathlineto{\pgfqpoint{6.139225in}{2.254234in}}%
\pgfpathlineto{\pgfqpoint{6.139336in}{2.464077in}}%
\pgfpathlineto{\pgfqpoint{6.139905in}{1.777777in}}%
\pgfpathlineto{\pgfqpoint{6.140331in}{2.273292in}}%
\pgfpathlineto{\pgfqpoint{6.140985in}{1.887381in}}%
\pgfpathlineto{\pgfqpoint{6.141433in}{2.455312in}}%
\pgfpathlineto{\pgfqpoint{6.141442in}{2.146916in}}%
\pgfpathlineto{\pgfqpoint{6.141736in}{2.452432in}}%
\pgfpathlineto{\pgfqpoint{6.141968in}{1.892845in}}%
\pgfpathlineto{\pgfqpoint{6.142554in}{2.294309in}}%
\pgfpathlineto{\pgfqpoint{6.143365in}{1.864539in}}%
\pgfpathlineto{\pgfqpoint{6.142935in}{2.440669in}}%
\pgfpathlineto{\pgfqpoint{6.143663in}{2.289460in}}%
\pgfpathlineto{\pgfqpoint{6.143773in}{2.448533in}}%
\pgfpathlineto{\pgfqpoint{6.143875in}{1.677369in}}%
\pgfpathlineto{\pgfqpoint{6.144774in}{2.286554in}}%
\pgfpathlineto{\pgfqpoint{6.145123in}{2.436268in}}%
\pgfpathlineto{\pgfqpoint{6.145279in}{1.822187in}}%
\pgfpathlineto{\pgfqpoint{6.145887in}{2.342002in}}%
\pgfpathlineto{\pgfqpoint{6.146482in}{1.876364in}}%
\pgfpathlineto{\pgfqpoint{6.146087in}{2.426878in}}%
\pgfpathlineto{\pgfqpoint{6.147000in}{2.163825in}}%
\pgfpathlineto{\pgfqpoint{6.147618in}{2.426522in}}%
\pgfpathlineto{\pgfqpoint{6.147780in}{1.740848in}}%
\pgfpathlineto{\pgfqpoint{6.148111in}{2.212897in}}%
\pgfpathlineto{\pgfqpoint{6.148453in}{2.438060in}}%
\pgfpathlineto{\pgfqpoint{6.148893in}{1.576454in}}%
\pgfpathlineto{\pgfqpoint{6.149223in}{2.322628in}}%
\pgfpathlineto{\pgfqpoint{6.149734in}{1.755042in}}%
\pgfpathlineto{\pgfqpoint{6.150194in}{2.435339in}}%
\pgfpathlineto{\pgfqpoint{6.150336in}{1.996263in}}%
\pgfpathlineto{\pgfqpoint{6.151430in}{2.428628in}}%
\pgfpathlineto{\pgfqpoint{6.150705in}{1.808858in}}%
\pgfpathlineto{\pgfqpoint{6.151448in}{2.266466in}}%
\pgfpathlineto{\pgfqpoint{6.152428in}{2.424713in}}%
\pgfpathlineto{\pgfqpoint{6.151902in}{1.770178in}}%
\pgfpathlineto{\pgfqpoint{6.152558in}{2.308462in}}%
\pgfpathlineto{\pgfqpoint{6.153340in}{1.706967in}}%
\pgfpathlineto{\pgfqpoint{6.153460in}{2.429069in}}%
\pgfpathlineto{\pgfqpoint{6.153670in}{2.071394in}}%
\pgfpathlineto{\pgfqpoint{6.154765in}{2.427689in}}%
\pgfpathlineto{\pgfqpoint{6.154018in}{1.752484in}}%
\pgfpathlineto{\pgfqpoint{6.154781in}{2.365964in}}%
\pgfpathlineto{\pgfqpoint{6.155703in}{2.451403in}}%
\pgfpathlineto{\pgfqpoint{6.155624in}{1.828177in}}%
\pgfpathlineto{\pgfqpoint{6.155808in}{2.331549in}}%
\pgfpathlineto{\pgfqpoint{6.156805in}{1.920587in}}%
\pgfpathlineto{\pgfqpoint{6.156488in}{2.426833in}}%
\pgfpathlineto{\pgfqpoint{6.156918in}{2.283152in}}%
\pgfpathlineto{\pgfqpoint{6.157989in}{2.425308in}}%
\pgfpathlineto{\pgfqpoint{6.157052in}{1.802009in}}%
\pgfpathlineto{\pgfqpoint{6.158020in}{2.200837in}}%
\pgfpathlineto{\pgfqpoint{6.158332in}{1.794591in}}%
\pgfpathlineto{\pgfqpoint{6.158304in}{2.437686in}}%
\pgfpathlineto{\pgfqpoint{6.159129in}{1.929906in}}%
\pgfpathlineto{\pgfqpoint{6.159747in}{2.424626in}}%
\pgfpathlineto{\pgfqpoint{6.159840in}{1.815943in}}%
\pgfpathlineto{\pgfqpoint{6.160242in}{2.300723in}}%
\pgfpathlineto{\pgfqpoint{6.161040in}{2.424391in}}%
\pgfpathlineto{\pgfqpoint{6.160636in}{1.938941in}}%
\pgfpathlineto{\pgfqpoint{6.161350in}{2.365610in}}%
\pgfpathlineto{\pgfqpoint{6.161551in}{1.673553in}}%
\pgfpathlineto{\pgfqpoint{6.161612in}{2.439714in}}%
\pgfpathlineto{\pgfqpoint{6.162460in}{2.308056in}}%
\pgfpathlineto{\pgfqpoint{6.163062in}{2.449813in}}%
\pgfpathlineto{\pgfqpoint{6.162777in}{1.832012in}}%
\pgfpathlineto{\pgfqpoint{6.163536in}{2.413103in}}%
\pgfpathlineto{\pgfqpoint{6.163687in}{1.715158in}}%
\pgfpathlineto{\pgfqpoint{6.164565in}{2.442499in}}%
\pgfpathlineto{\pgfqpoint{6.164647in}{2.375010in}}%
\pgfpathlineto{\pgfqpoint{6.165007in}{1.594180in}}%
\pgfpathlineto{\pgfqpoint{6.165198in}{2.435831in}}%
\pgfpathlineto{\pgfqpoint{6.165760in}{2.206686in}}%
\pgfpathlineto{\pgfqpoint{6.165826in}{2.431209in}}%
\pgfpathlineto{\pgfqpoint{6.165954in}{1.832190in}}%
\pgfpathlineto{\pgfqpoint{6.166839in}{2.070133in}}%
\pgfpathlineto{\pgfqpoint{6.167779in}{1.635793in}}%
\pgfpathlineto{\pgfqpoint{6.166862in}{2.435407in}}%
\pgfpathlineto{\pgfqpoint{6.167948in}{2.226988in}}%
\pgfpathlineto{\pgfqpoint{6.168022in}{2.436491in}}%
\pgfpathlineto{\pgfqpoint{6.168350in}{1.823981in}}%
\pgfpathlineto{\pgfqpoint{6.169058in}{2.261512in}}%
\pgfpathlineto{\pgfqpoint{6.170094in}{1.806984in}}%
\pgfpathlineto{\pgfqpoint{6.169576in}{2.444052in}}%
\pgfpathlineto{\pgfqpoint{6.170169in}{2.283943in}}%
\pgfpathlineto{\pgfqpoint{6.170321in}{2.452218in}}%
\pgfpathlineto{\pgfqpoint{6.171198in}{1.777327in}}%
\pgfpathlineto{\pgfqpoint{6.171271in}{2.277955in}}%
\pgfpathlineto{\pgfqpoint{6.171869in}{1.911797in}}%
\pgfpathlineto{\pgfqpoint{6.171476in}{2.435931in}}%
\pgfpathlineto{\pgfqpoint{6.172382in}{2.224849in}}%
\pgfpathlineto{\pgfqpoint{6.172508in}{2.435212in}}%
\pgfpathlineto{\pgfqpoint{6.173186in}{1.762112in}}%
\pgfpathlineto{\pgfqpoint{6.173491in}{2.230471in}}%
\pgfpathlineto{\pgfqpoint{6.174004in}{1.583255in}}%
\pgfpathlineto{\pgfqpoint{6.174428in}{2.444788in}}%
\pgfpathlineto{\pgfqpoint{6.174600in}{2.306190in}}%
\pgfpathlineto{\pgfqpoint{6.175488in}{2.434113in}}%
\pgfpathlineto{\pgfqpoint{6.175559in}{1.875385in}}%
\pgfpathlineto{\pgfqpoint{6.175702in}{2.172383in}}%
\pgfpathlineto{\pgfqpoint{6.175922in}{1.845858in}}%
\pgfpathlineto{\pgfqpoint{6.176366in}{2.428662in}}%
\pgfpathlineto{\pgfqpoint{6.176813in}{2.195390in}}%
\pgfpathlineto{\pgfqpoint{6.177674in}{2.432753in}}%
\pgfpathlineto{\pgfqpoint{6.177561in}{1.733150in}}%
\pgfpathlineto{\pgfqpoint{6.177923in}{2.309036in}}%
\pgfpathlineto{\pgfqpoint{6.178328in}{1.627056in}}%
\pgfpathlineto{\pgfqpoint{6.178006in}{2.433715in}}%
\pgfpathlineto{\pgfqpoint{6.179035in}{2.227669in}}%
\pgfpathlineto{\pgfqpoint{6.179881in}{1.597085in}}%
\pgfpathlineto{\pgfqpoint{6.179826in}{2.418585in}}%
\pgfpathlineto{\pgfqpoint{6.180141in}{2.258774in}}%
\pgfpathlineto{\pgfqpoint{6.180730in}{2.461799in}}%
\pgfpathlineto{\pgfqpoint{6.180886in}{1.751268in}}%
\pgfpathlineto{\pgfqpoint{6.181248in}{2.226844in}}%
\pgfpathlineto{\pgfqpoint{6.181707in}{1.644427in}}%
\pgfpathlineto{\pgfqpoint{6.182006in}{2.440064in}}%
\pgfpathlineto{\pgfqpoint{6.182358in}{2.296148in}}%
\pgfpathlineto{\pgfqpoint{6.182749in}{1.897930in}}%
\pgfpathlineto{\pgfqpoint{6.183397in}{2.425406in}}%
\pgfpathlineto{\pgfqpoint{6.183465in}{2.170500in}}%
\pgfpathlineto{\pgfqpoint{6.183584in}{2.456668in}}%
\pgfpathlineto{\pgfqpoint{6.183525in}{1.781091in}}%
\pgfpathlineto{\pgfqpoint{6.184578in}{2.435250in}}%
\pgfpathlineto{\pgfqpoint{6.185076in}{1.790769in}}%
\pgfpathlineto{\pgfqpoint{6.185554in}{2.439183in}}%
\pgfpathlineto{\pgfqpoint{6.185689in}{2.182047in}}%
\pgfpathlineto{\pgfqpoint{6.186196in}{2.459632in}}%
\pgfpathlineto{\pgfqpoint{6.186019in}{1.812818in}}%
\pgfpathlineto{\pgfqpoint{6.186798in}{2.295796in}}%
\pgfpathlineto{\pgfqpoint{6.187260in}{1.748368in}}%
\pgfpathlineto{\pgfqpoint{6.187432in}{2.439259in}}%
\pgfpathlineto{\pgfqpoint{6.187910in}{2.100476in}}%
\pgfpathlineto{\pgfqpoint{6.187935in}{2.382730in}}%
\pgfpathlineto{\pgfqpoint{6.187955in}{2.246601in}}%
\pgfpathlineto{\pgfqpoint{6.187956in}{1.702699in}}%
\pgfpathlineto{\pgfqpoint{6.188163in}{2.444235in}}%
\pgfpathlineto{\pgfqpoint{6.189065in}{2.308118in}}%
\pgfpathlineto{\pgfqpoint{6.190020in}{2.453475in}}%
\pgfpathlineto{\pgfqpoint{6.189366in}{1.760641in}}%
\pgfpathlineto{\pgfqpoint{6.190176in}{2.303369in}}%
\pgfpathlineto{\pgfqpoint{6.191060in}{1.664563in}}%
\pgfpathlineto{\pgfqpoint{6.191212in}{2.439268in}}%
\pgfpathlineto{\pgfqpoint{6.191288in}{2.147101in}}%
\pgfpathlineto{\pgfqpoint{6.192227in}{2.421839in}}%
\pgfpathlineto{\pgfqpoint{6.191811in}{1.717966in}}%
\pgfpathlineto{\pgfqpoint{6.192402in}{2.308186in}}%
\pgfpathlineto{\pgfqpoint{6.192643in}{1.670660in}}%
\pgfpathlineto{\pgfqpoint{6.192436in}{2.420866in}}%
\pgfpathlineto{\pgfqpoint{6.193513in}{2.306192in}}%
\pgfpathlineto{\pgfqpoint{6.194122in}{2.446168in}}%
\pgfpathlineto{\pgfqpoint{6.193802in}{1.795660in}}%
\pgfpathlineto{\pgfqpoint{6.194616in}{2.332374in}}%
\pgfpathlineto{\pgfqpoint{6.195444in}{1.805832in}}%
\pgfpathlineto{\pgfqpoint{6.195072in}{2.433883in}}%
\pgfpathlineto{\pgfqpoint{6.195726in}{2.203353in}}%
\pgfpathlineto{\pgfqpoint{6.195745in}{2.422330in}}%
\pgfpathlineto{\pgfqpoint{6.196639in}{1.630172in}}%
\pgfpathlineto{\pgfqpoint{6.196837in}{2.245076in}}%
\pgfpathlineto{\pgfqpoint{6.197781in}{1.872462in}}%
\pgfpathlineto{\pgfqpoint{6.197466in}{2.436311in}}%
\pgfpathlineto{\pgfqpoint{6.197944in}{2.303451in}}%
\pgfpathlineto{\pgfqpoint{6.198615in}{2.411887in}}%
\pgfpathlineto{\pgfqpoint{6.198075in}{1.770670in}}%
\pgfpathlineto{\pgfqpoint{6.199028in}{2.357978in}}%
\pgfpathlineto{\pgfqpoint{6.199030in}{1.879345in}}%
\pgfpathlineto{\pgfqpoint{6.199438in}{2.433905in}}%
\pgfpathlineto{\pgfqpoint{6.200140in}{2.229140in}}%
\pgfpathlineto{\pgfqpoint{6.200762in}{1.697958in}}%
\pgfpathlineto{\pgfqpoint{6.200441in}{2.437876in}}%
\pgfpathlineto{\pgfqpoint{6.201249in}{2.321481in}}%
\pgfpathlineto{\pgfqpoint{6.201451in}{1.825861in}}%
\pgfpathlineto{\pgfqpoint{6.201802in}{2.434103in}}%
\pgfpathlineto{\pgfqpoint{6.202358in}{2.298417in}}%
\pgfpathlineto{\pgfqpoint{6.202982in}{2.440862in}}%
\pgfpathlineto{\pgfqpoint{6.203245in}{1.829612in}}%
\pgfpathlineto{\pgfqpoint{6.203465in}{2.256705in}}%
\pgfpathlineto{\pgfqpoint{6.203475in}{1.726645in}}%
\pgfpathlineto{\pgfqpoint{6.204049in}{2.440114in}}%
\pgfpathlineto{\pgfqpoint{6.204576in}{2.208479in}}%
\pgfpathlineto{\pgfqpoint{6.204663in}{2.445257in}}%
\pgfpathlineto{\pgfqpoint{6.205642in}{1.797705in}}%
\pgfpathlineto{\pgfqpoint{6.205687in}{2.357716in}}%
\pgfpathlineto{\pgfqpoint{6.205788in}{1.821158in}}%
\pgfpathlineto{\pgfqpoint{6.206745in}{2.425426in}}%
\pgfpathlineto{\pgfqpoint{6.206799in}{2.230070in}}%
\pgfpathlineto{\pgfqpoint{6.207239in}{2.431595in}}%
\pgfpathlineto{\pgfqpoint{6.207005in}{1.666747in}}%
\pgfpathlineto{\pgfqpoint{6.207906in}{2.243467in}}%
\pgfpathlineto{\pgfqpoint{6.208338in}{1.666101in}}%
\pgfpathlineto{\pgfqpoint{6.208597in}{2.430135in}}%
\pgfpathlineto{\pgfqpoint{6.209016in}{2.343084in}}%
\pgfpathlineto{\pgfqpoint{6.209857in}{2.443704in}}%
\pgfpathlineto{\pgfqpoint{6.209272in}{1.724152in}}%
\pgfpathlineto{\pgfqpoint{6.210118in}{2.335456in}}%
\pgfpathlineto{\pgfqpoint{6.210885in}{1.843603in}}%
\pgfpathlineto{\pgfqpoint{6.210300in}{2.432889in}}%
\pgfpathlineto{\pgfqpoint{6.211229in}{2.279518in}}%
\pgfpathlineto{\pgfqpoint{6.211295in}{1.634587in}}%
\pgfpathlineto{\pgfqpoint{6.211344in}{2.433262in}}%
\pgfpathlineto{\pgfqpoint{6.212339in}{2.156635in}}%
\pgfpathlineto{\pgfqpoint{6.212776in}{2.415975in}}%
\pgfpathlineto{\pgfqpoint{6.213415in}{1.648962in}}%
\pgfpathlineto{\pgfqpoint{6.213451in}{2.346166in}}%
\pgfpathlineto{\pgfqpoint{6.213572in}{1.787802in}}%
\pgfpathlineto{\pgfqpoint{6.213738in}{2.428156in}}%
\pgfpathlineto{\pgfqpoint{6.214562in}{2.153491in}}%
\pgfpathlineto{\pgfqpoint{6.215471in}{2.425370in}}%
\pgfpathlineto{\pgfqpoint{6.214794in}{1.771772in}}%
\pgfpathlineto{\pgfqpoint{6.215674in}{2.270244in}}%
\pgfpathlineto{\pgfqpoint{6.215908in}{2.423538in}}%
\pgfpathlineto{\pgfqpoint{6.216281in}{1.791994in}}%
\pgfpathlineto{\pgfqpoint{6.216766in}{2.247038in}}%
\pgfpathlineto{\pgfqpoint{6.216926in}{1.589520in}}%
\pgfpathlineto{\pgfqpoint{6.217512in}{2.442356in}}%
\pgfpathlineto{\pgfqpoint{6.217877in}{1.821151in}}%
\pgfpathlineto{\pgfqpoint{6.218440in}{2.418248in}}%
\pgfpathlineto{\pgfqpoint{6.218217in}{1.779846in}}%
\pgfpathlineto{\pgfqpoint{6.218987in}{2.264996in}}%
\pgfpathlineto{\pgfqpoint{6.218989in}{1.837636in}}%
\pgfpathlineto{\pgfqpoint{6.219218in}{2.423555in}}%
\pgfpathlineto{\pgfqpoint{6.220098in}{2.162657in}}%
\pgfpathlineto{\pgfqpoint{6.220499in}{2.423544in}}%
\pgfpathlineto{\pgfqpoint{6.220992in}{1.788023in}}%
\pgfpathlineto{\pgfqpoint{6.221209in}{2.305610in}}%
\pgfpathlineto{\pgfqpoint{6.221823in}{1.845706in}}%
\pgfpathlineto{\pgfqpoint{6.221309in}{2.425292in}}%
\pgfpathlineto{\pgfqpoint{6.222319in}{2.305667in}}%
\pgfpathlineto{\pgfqpoint{6.222383in}{1.665577in}}%
\pgfpathlineto{\pgfqpoint{6.222939in}{2.438231in}}%
\pgfpathlineto{\pgfqpoint{6.223431in}{2.264802in}}%
\pgfpathlineto{\pgfqpoint{6.223685in}{1.774169in}}%
\pgfpathlineto{\pgfqpoint{6.224500in}{2.415976in}}%
\pgfpathlineto{\pgfqpoint{6.224536in}{2.204879in}}%
\pgfpathlineto{\pgfqpoint{6.225402in}{2.433394in}}%
\pgfpathlineto{\pgfqpoint{6.225103in}{1.812826in}}%
\pgfpathlineto{\pgfqpoint{6.225647in}{2.329420in}}%
\pgfpathlineto{\pgfqpoint{6.226605in}{1.734551in}}%
\pgfpathlineto{\pgfqpoint{6.226032in}{2.415455in}}%
\pgfpathlineto{\pgfqpoint{6.226758in}{2.154920in}}%
\pgfpathlineto{\pgfqpoint{6.227352in}{2.433997in}}%
\pgfpathlineto{\pgfqpoint{6.227014in}{1.724710in}}%
\pgfpathlineto{\pgfqpoint{6.227870in}{2.291925in}}%
\pgfpathlineto{\pgfqpoint{6.228123in}{1.786320in}}%
\pgfpathlineto{\pgfqpoint{6.227983in}{2.403232in}}%
\pgfpathlineto{\pgfqpoint{6.228981in}{2.238013in}}%
\pgfpathlineto{\pgfqpoint{6.229969in}{2.442286in}}%
\pgfpathlineto{\pgfqpoint{6.229570in}{1.667545in}}%
\pgfpathlineto{\pgfqpoint{6.230093in}{2.383883in}}%
\pgfpathlineto{\pgfqpoint{6.230159in}{1.798895in}}%
\pgfpathlineto{\pgfqpoint{6.230240in}{2.424765in}}%
\pgfpathlineto{\pgfqpoint{6.231205in}{2.243865in}}%
\pgfpathlineto{\pgfqpoint{6.232185in}{2.430689in}}%
\pgfpathlineto{\pgfqpoint{6.231688in}{1.837065in}}%
\pgfpathlineto{\pgfqpoint{6.232293in}{2.218012in}}%
\pgfpathlineto{\pgfqpoint{6.233389in}{1.570686in}}%
\pgfpathlineto{\pgfqpoint{6.232344in}{2.420023in}}%
\pgfpathlineto{\pgfqpoint{6.233403in}{2.396860in}}%
\pgfpathlineto{\pgfqpoint{6.233717in}{1.451838in}}%
\pgfpathlineto{\pgfqpoint{6.233570in}{2.419278in}}%
\pgfpathlineto{\pgfqpoint{6.234515in}{2.240837in}}%
\pgfpathlineto{\pgfqpoint{6.235179in}{2.420192in}}%
\pgfpathlineto{\pgfqpoint{6.235356in}{1.684580in}}%
\pgfpathlineto{\pgfqpoint{6.235621in}{2.350441in}}%
\pgfpathlineto{\pgfqpoint{6.236297in}{1.783687in}}%
\pgfpathlineto{\pgfqpoint{6.235723in}{2.421388in}}%
\pgfpathlineto{\pgfqpoint{6.236732in}{2.235476in}}%
\pgfpathlineto{\pgfqpoint{6.237660in}{2.418016in}}%
\pgfpathlineto{\pgfqpoint{6.236769in}{1.859615in}}%
\pgfpathlineto{\pgfqpoint{6.237834in}{2.276503in}}%
\pgfpathlineto{\pgfqpoint{6.238171in}{1.817753in}}%
\pgfpathlineto{\pgfqpoint{6.238548in}{2.440119in}}%
\pgfpathlineto{\pgfqpoint{6.238945in}{1.962550in}}%
\pgfpathlineto{\pgfqpoint{6.239278in}{2.426059in}}%
\pgfpathlineto{\pgfqpoint{6.239093in}{1.741220in}}%
\pgfpathlineto{\pgfqpoint{6.240056in}{2.342304in}}%
\pgfpathlineto{\pgfqpoint{6.240500in}{1.777786in}}%
\pgfpathlineto{\pgfqpoint{6.240772in}{2.435792in}}%
\pgfpathlineto{\pgfqpoint{6.241166in}{1.896361in}}%
\pgfpathlineto{\pgfqpoint{6.241700in}{2.442537in}}%
\pgfpathlineto{\pgfqpoint{6.241972in}{1.790057in}}%
\pgfpathlineto{\pgfqpoint{6.242278in}{2.276424in}}%
\pgfpathlineto{\pgfqpoint{6.243013in}{1.592438in}}%
\pgfpathlineto{\pgfqpoint{6.243021in}{2.427183in}}%
\pgfpathlineto{\pgfqpoint{6.243386in}{2.147286in}}%
\pgfpathlineto{\pgfqpoint{6.244248in}{2.438682in}}%
\pgfpathlineto{\pgfqpoint{6.244447in}{1.577552in}}%
\pgfpathlineto{\pgfqpoint{6.244498in}{2.292885in}}%
\pgfpathlineto{\pgfqpoint{6.244702in}{1.753383in}}%
\pgfpathlineto{\pgfqpoint{6.244729in}{2.423014in}}%
\pgfpathlineto{\pgfqpoint{6.245609in}{2.221084in}}%
\pgfpathlineto{\pgfqpoint{6.246184in}{2.431198in}}%
\pgfpathlineto{\pgfqpoint{6.245940in}{1.896919in}}%
\pgfpathlineto{\pgfqpoint{6.246721in}{2.292770in}}%
\pgfpathlineto{\pgfqpoint{6.247606in}{1.765849in}}%
\pgfpathlineto{\pgfqpoint{6.247796in}{2.414725in}}%
\pgfpathlineto{\pgfqpoint{6.247832in}{2.217546in}}%
\pgfpathlineto{\pgfqpoint{6.248676in}{2.425434in}}%
\pgfpathlineto{\pgfqpoint{6.248567in}{1.732716in}}%
\pgfpathlineto{\pgfqpoint{6.248941in}{2.211678in}}%
\pgfpathlineto{\pgfqpoint{6.249766in}{1.804785in}}%
\pgfpathlineto{\pgfqpoint{6.249117in}{2.466177in}}%
\pgfpathlineto{\pgfqpoint{6.250050in}{2.173090in}}%
\pgfpathlineto{\pgfqpoint{6.250586in}{2.451477in}}%
\pgfpathlineto{\pgfqpoint{6.250393in}{1.750389in}}%
\pgfpathlineto{\pgfqpoint{6.251162in}{2.285961in}}%
\pgfpathlineto{\pgfqpoint{6.251926in}{1.832454in}}%
\pgfpathlineto{\pgfqpoint{6.252164in}{2.438479in}}%
\pgfpathlineto{\pgfqpoint{6.252274in}{2.103176in}}%
\pgfpathlineto{\pgfqpoint{6.252836in}{2.425589in}}%
\pgfpathlineto{\pgfqpoint{6.252826in}{1.858532in}}%
\pgfpathlineto{\pgfqpoint{6.253387in}{2.288992in}}%
\pgfpathlineto{\pgfqpoint{6.253916in}{1.852488in}}%
\pgfpathlineto{\pgfqpoint{6.254464in}{2.464111in}}%
\pgfpathlineto{\pgfqpoint{6.254497in}{2.307090in}}%
\pgfpathlineto{\pgfqpoint{6.254907in}{2.416286in}}%
\pgfpathlineto{\pgfqpoint{6.254545in}{1.773067in}}%
\pgfpathlineto{\pgfqpoint{6.255570in}{2.343126in}}%
\pgfpathlineto{\pgfqpoint{6.256363in}{1.800012in}}%
\pgfpathlineto{\pgfqpoint{6.256631in}{2.422989in}}%
\pgfpathlineto{\pgfqpoint{6.256681in}{2.294646in}}%
\pgfpathlineto{\pgfqpoint{6.256695in}{1.748313in}}%
\pgfpathlineto{\pgfqpoint{6.257311in}{2.419260in}}%
\pgfpathlineto{\pgfqpoint{6.257793in}{2.202982in}}%
\pgfpathlineto{\pgfqpoint{6.258278in}{2.422957in}}%
\pgfpathlineto{\pgfqpoint{6.258224in}{1.752797in}}%
\pgfpathlineto{\pgfqpoint{6.258904in}{2.212303in}}%
\pgfpathlineto{\pgfqpoint{6.259492in}{1.538708in}}%
\pgfpathlineto{\pgfqpoint{6.259160in}{2.425674in}}%
\pgfpathlineto{\pgfqpoint{6.260015in}{2.130989in}}%
\pgfpathlineto{\pgfqpoint{6.260269in}{2.407145in}}%
\pgfpathlineto{\pgfqpoint{6.260391in}{1.795954in}}%
\pgfpathlineto{\pgfqpoint{6.261126in}{2.274832in}}%
\pgfpathlineto{\pgfqpoint{6.261978in}{1.783098in}}%
\pgfpathlineto{\pgfqpoint{6.262012in}{2.427489in}}%
\pgfpathlineto{\pgfqpoint{6.262238in}{2.192474in}}%
\pgfpathlineto{\pgfqpoint{6.262798in}{2.417640in}}%
\pgfpathlineto{\pgfqpoint{6.262315in}{1.204084in}}%
\pgfpathlineto{\pgfqpoint{6.263336in}{2.339022in}}%
\pgfpathlineto{\pgfqpoint{6.263699in}{1.754276in}}%
\pgfpathlineto{\pgfqpoint{6.264365in}{2.431830in}}%
\pgfpathlineto{\pgfqpoint{6.264447in}{2.377559in}}%
\pgfpathlineto{\pgfqpoint{6.264721in}{1.794287in}}%
\pgfpathlineto{\pgfqpoint{6.265388in}{2.437145in}}%
\pgfpathlineto{\pgfqpoint{6.265560in}{2.123110in}}%
\pgfpathlineto{\pgfqpoint{6.266173in}{2.436170in}}%
\pgfpathlineto{\pgfqpoint{6.266529in}{1.717465in}}%
\pgfpathlineto{\pgfqpoint{6.266678in}{2.364268in}}%
\pgfpathlineto{\pgfqpoint{6.267045in}{1.823036in}}%
\pgfpathlineto{\pgfqpoint{6.267155in}{2.425659in}}%
\pgfpathlineto{\pgfqpoint{6.267789in}{2.277230in}}%
\pgfpathlineto{\pgfqpoint{6.268189in}{1.597453in}}%
\pgfpathlineto{\pgfqpoint{6.268296in}{2.405688in}}%
\pgfpathlineto{\pgfqpoint{6.268900in}{2.130005in}}%
\pgfpathlineto{\pgfqpoint{6.269558in}{2.411658in}}%
\pgfpathlineto{\pgfqpoint{6.269250in}{1.743218in}}%
\pgfpathlineto{\pgfqpoint{6.270010in}{2.356723in}}%
\pgfpathlineto{\pgfqpoint{6.271013in}{1.709219in}}%
\pgfpathlineto{\pgfqpoint{6.270950in}{2.421373in}}%
\pgfpathlineto{\pgfqpoint{6.271121in}{2.217620in}}%
\pgfpathlineto{\pgfqpoint{6.271525in}{2.409692in}}%
\pgfpathlineto{\pgfqpoint{6.271978in}{1.727858in}}%
\pgfpathlineto{\pgfqpoint{6.272231in}{2.220379in}}%
\pgfpathlineto{\pgfqpoint{6.272712in}{1.554200in}}%
\pgfpathlineto{\pgfqpoint{6.272564in}{2.448452in}}%
\pgfpathlineto{\pgfqpoint{6.273341in}{2.266798in}}%
\pgfpathlineto{\pgfqpoint{6.273435in}{1.736925in}}%
\pgfpathlineto{\pgfqpoint{6.273376in}{2.432749in}}%
\pgfpathlineto{\pgfqpoint{6.274436in}{2.297694in}}%
\pgfpathlineto{\pgfqpoint{6.274689in}{2.421951in}}%
\pgfpathlineto{\pgfqpoint{6.274510in}{1.662216in}}%
\pgfpathlineto{\pgfqpoint{6.275545in}{2.237588in}}%
\pgfpathlineto{\pgfqpoint{6.276044in}{1.782822in}}%
\pgfpathlineto{\pgfqpoint{6.276030in}{2.427270in}}%
\pgfpathlineto{\pgfqpoint{6.276651in}{2.250763in}}%
\pgfpathlineto{\pgfqpoint{6.276939in}{2.423315in}}%
\pgfpathlineto{\pgfqpoint{6.277685in}{1.835698in}}%
\pgfpathlineto{\pgfqpoint{6.277760in}{2.270578in}}%
\pgfpathlineto{\pgfqpoint{6.278271in}{1.694724in}}%
\pgfpathlineto{\pgfqpoint{6.278303in}{2.428635in}}%
\pgfpathlineto{\pgfqpoint{6.278870in}{2.225262in}}%
\pgfpathlineto{\pgfqpoint{6.279167in}{2.419040in}}%
\pgfpathlineto{\pgfqpoint{6.279070in}{1.688600in}}%
\pgfpathlineto{\pgfqpoint{6.279978in}{2.283415in}}%
\pgfpathlineto{\pgfqpoint{6.280689in}{1.637124in}}%
\pgfpathlineto{\pgfqpoint{6.280856in}{2.417929in}}%
\pgfpathlineto{\pgfqpoint{6.281089in}{2.338255in}}%
\pgfpathlineto{\pgfqpoint{6.281527in}{1.820268in}}%
\pgfpathlineto{\pgfqpoint{6.281826in}{2.423481in}}%
\pgfpathlineto{\pgfqpoint{6.282201in}{2.155977in}}%
\pgfpathlineto{\pgfqpoint{6.282372in}{2.435516in}}%
\pgfpathlineto{\pgfqpoint{6.282821in}{1.739962in}}%
\pgfpathlineto{\pgfqpoint{6.283313in}{2.345530in}}%
\pgfpathlineto{\pgfqpoint{6.283996in}{1.754741in}}%
\pgfpathlineto{\pgfqpoint{6.284125in}{2.436278in}}%
\pgfpathlineto{\pgfqpoint{6.284423in}{2.305137in}}%
\pgfpathlineto{\pgfqpoint{6.285449in}{2.423844in}}%
\pgfpathlineto{\pgfqpoint{6.284561in}{1.761197in}}%
\pgfpathlineto{\pgfqpoint{6.285532in}{2.307445in}}%
\pgfpathlineto{\pgfqpoint{6.286240in}{1.656703in}}%
\pgfpathlineto{\pgfqpoint{6.285840in}{2.407832in}}%
\pgfpathlineto{\pgfqpoint{6.286643in}{2.353966in}}%
\pgfpathlineto{\pgfqpoint{6.286676in}{1.569727in}}%
\pgfpathlineto{\pgfqpoint{6.287567in}{2.419641in}}%
\pgfpathlineto{\pgfqpoint{6.287761in}{2.034164in}}%
\pgfpathlineto{\pgfqpoint{6.288649in}{2.409975in}}%
\pgfpathlineto{\pgfqpoint{6.288382in}{1.708685in}}%
\pgfpathlineto{\pgfqpoint{6.288873in}{2.297897in}}%
\pgfpathlineto{\pgfqpoint{6.288989in}{2.406425in}}%
\pgfpathlineto{\pgfqpoint{6.289513in}{1.822375in}}%
\pgfpathlineto{\pgfqpoint{6.289977in}{2.191875in}}%
\pgfpathlineto{\pgfqpoint{6.290384in}{1.611090in}}%
\pgfpathlineto{\pgfqpoint{6.290949in}{2.428421in}}%
\pgfpathlineto{\pgfqpoint{6.291088in}{2.198401in}}%
\pgfpathlineto{\pgfqpoint{6.291209in}{2.428314in}}%
\pgfpathlineto{\pgfqpoint{6.292089in}{1.769355in}}%
\pgfpathlineto{\pgfqpoint{6.292200in}{2.310912in}}%
\pgfpathlineto{\pgfqpoint{6.293116in}{1.770117in}}%
\pgfpathlineto{\pgfqpoint{6.292978in}{2.417280in}}%
\pgfpathlineto{\pgfqpoint{6.293311in}{2.222001in}}%
\pgfpathlineto{\pgfqpoint{6.293625in}{2.454339in}}%
\pgfpathlineto{\pgfqpoint{6.294065in}{1.740396in}}%
\pgfpathlineto{\pgfqpoint{6.294422in}{2.244058in}}%
\pgfpathlineto{\pgfqpoint{6.295401in}{2.427101in}}%
\pgfpathlineto{\pgfqpoint{6.294969in}{1.662712in}}%
\pgfpathlineto{\pgfqpoint{6.295495in}{2.246504in}}%
\pgfpathlineto{\pgfqpoint{6.295500in}{1.816760in}}%
\pgfpathlineto{\pgfqpoint{6.295902in}{2.434771in}}%
\pgfpathlineto{\pgfqpoint{6.296605in}{2.329668in}}%
\pgfpathlineto{\pgfqpoint{6.296791in}{1.494925in}}%
\pgfpathlineto{\pgfqpoint{6.297439in}{2.427918in}}%
\pgfpathlineto{\pgfqpoint{6.297719in}{2.188024in}}%
\pgfpathlineto{\pgfqpoint{6.297797in}{2.416510in}}%
\pgfpathlineto{\pgfqpoint{6.297814in}{1.799994in}}%
\pgfpathlineto{\pgfqpoint{6.298806in}{2.249426in}}%
\pgfpathlineto{\pgfqpoint{6.299151in}{1.783863in}}%
\pgfpathlineto{\pgfqpoint{6.299228in}{2.418501in}}%
\pgfpathlineto{\pgfqpoint{6.299917in}{2.138793in}}%
\pgfpathlineto{\pgfqpoint{6.300869in}{2.408008in}}%
\pgfpathlineto{\pgfqpoint{6.299931in}{1.784947in}}%
\pgfpathlineto{\pgfqpoint{6.301029in}{2.237076in}}%
\pgfpathlineto{\pgfqpoint{6.301546in}{1.608779in}}%
\pgfpathlineto{\pgfqpoint{6.301388in}{2.415660in}}%
\pgfpathlineto{\pgfqpoint{6.302139in}{2.230272in}}%
\pgfpathlineto{\pgfqpoint{6.302604in}{2.423713in}}%
\pgfpathlineto{\pgfqpoint{6.302278in}{1.787773in}}%
\pgfpathlineto{\pgfqpoint{6.303239in}{2.309052in}}%
\pgfpathlineto{\pgfqpoint{6.303748in}{1.567255in}}%
\pgfpathlineto{\pgfqpoint{6.303501in}{2.421060in}}%
\pgfpathlineto{\pgfqpoint{6.304350in}{2.246575in}}%
\pgfpathlineto{\pgfqpoint{6.305448in}{1.761776in}}%
\pgfpathlineto{\pgfqpoint{6.304909in}{2.412195in}}%
\pgfpathlineto{\pgfqpoint{6.305459in}{2.254403in}}%
\pgfpathlineto{\pgfqpoint{6.306433in}{2.434641in}}%
\pgfpathlineto{\pgfqpoint{6.306407in}{1.805622in}}%
\pgfpathlineto{\pgfqpoint{6.306564in}{2.269498in}}%
\pgfpathlineto{\pgfqpoint{6.306565in}{1.643486in}}%
\pgfpathlineto{\pgfqpoint{6.307318in}{2.422236in}}%
\pgfpathlineto{\pgfqpoint{6.307675in}{2.250069in}}%
\pgfpathlineto{\pgfqpoint{6.307942in}{1.682050in}}%
\pgfpathlineto{\pgfqpoint{6.308687in}{2.395713in}}%
\pgfpathlineto{\pgfqpoint{6.308785in}{2.190089in}}%
\pgfpathlineto{\pgfqpoint{6.309695in}{2.418397in}}%
\pgfpathlineto{\pgfqpoint{6.309758in}{1.669082in}}%
\pgfpathlineto{\pgfqpoint{6.309895in}{2.277770in}}%
\pgfpathlineto{\pgfqpoint{6.309984in}{1.751805in}}%
\pgfpathlineto{\pgfqpoint{6.310737in}{2.423370in}}%
\pgfpathlineto{\pgfqpoint{6.311006in}{2.274711in}}%
\pgfpathlineto{\pgfqpoint{6.311536in}{2.423331in}}%
\pgfpathlineto{\pgfqpoint{6.311677in}{1.716039in}}%
\pgfpathlineto{\pgfqpoint{6.312095in}{2.281015in}}%
\pgfpathlineto{\pgfqpoint{6.312228in}{1.747142in}}%
\pgfpathlineto{\pgfqpoint{6.312463in}{2.403132in}}%
\pgfpathlineto{\pgfqpoint{6.313205in}{2.199457in}}%
\pgfpathlineto{\pgfqpoint{6.313242in}{2.420746in}}%
\pgfpathlineto{\pgfqpoint{6.314152in}{1.818218in}}%
\pgfpathlineto{\pgfqpoint{6.314315in}{2.242907in}}%
\pgfpathlineto{\pgfqpoint{6.314846in}{1.707182in}}%
\pgfpathlineto{\pgfqpoint{6.315200in}{2.414537in}}%
\pgfpathlineto{\pgfqpoint{6.315426in}{2.217727in}}%
\pgfpathlineto{\pgfqpoint{6.315533in}{2.424898in}}%
\pgfpathlineto{\pgfqpoint{6.315816in}{1.555871in}}%
\pgfpathlineto{\pgfqpoint{6.316520in}{2.295733in}}%
\pgfpathlineto{\pgfqpoint{6.316948in}{1.648850in}}%
\pgfpathlineto{\pgfqpoint{6.317629in}{2.413982in}}%
\pgfpathlineto{\pgfqpoint{6.317630in}{2.263922in}}%
\pgfpathlineto{\pgfqpoint{6.318143in}{2.393906in}}%
\pgfpathlineto{\pgfqpoint{6.318590in}{1.657345in}}%
\pgfpathlineto{\pgfqpoint{6.318738in}{2.187456in}}%
\pgfpathlineto{\pgfqpoint{6.319093in}{1.657291in}}%
\pgfpathlineto{\pgfqpoint{6.318977in}{2.413220in}}%
\pgfpathlineto{\pgfqpoint{6.319849in}{2.242993in}}%
\pgfpathlineto{\pgfqpoint{6.320607in}{1.758080in}}%
\pgfpathlineto{\pgfqpoint{6.319970in}{2.417635in}}%
\pgfpathlineto{\pgfqpoint{6.320944in}{2.311162in}}%
\pgfpathlineto{\pgfqpoint{6.321832in}{2.424949in}}%
\pgfpathlineto{\pgfqpoint{6.321647in}{1.701345in}}%
\pgfpathlineto{\pgfqpoint{6.322053in}{2.232213in}}%
\pgfpathlineto{\pgfqpoint{6.322747in}{1.809704in}}%
\pgfpathlineto{\pgfqpoint{6.322185in}{2.410922in}}%
\pgfpathlineto{\pgfqpoint{6.323164in}{2.190423in}}%
\pgfpathlineto{\pgfqpoint{6.323662in}{2.412465in}}%
\pgfpathlineto{\pgfqpoint{6.323797in}{1.824841in}}%
\pgfpathlineto{\pgfqpoint{6.324275in}{2.267311in}}%
\pgfpathlineto{\pgfqpoint{6.324953in}{1.584172in}}%
\pgfpathlineto{\pgfqpoint{6.324383in}{2.425023in}}%
\pgfpathlineto{\pgfqpoint{6.325384in}{2.282356in}}%
\pgfpathlineto{\pgfqpoint{6.325805in}{2.422482in}}%
\pgfpathlineto{\pgfqpoint{6.325684in}{1.646419in}}%
\pgfpathlineto{\pgfqpoint{6.326492in}{2.349975in}}%
\pgfpathlineto{\pgfqpoint{6.327543in}{1.768393in}}%
\pgfpathlineto{\pgfqpoint{6.327062in}{2.426462in}}%
\pgfpathlineto{\pgfqpoint{6.327603in}{2.244398in}}%
\pgfpathlineto{\pgfqpoint{6.328344in}{2.419502in}}%
\pgfpathlineto{\pgfqpoint{6.328312in}{1.493158in}}%
\pgfpathlineto{\pgfqpoint{6.328704in}{2.328043in}}%
\pgfpathlineto{\pgfqpoint{6.329029in}{1.855926in}}%
\pgfpathlineto{\pgfqpoint{6.329588in}{2.400492in}}%
\pgfpathlineto{\pgfqpoint{6.329815in}{2.126582in}}%
\pgfpathlineto{\pgfqpoint{6.330042in}{2.425765in}}%
\pgfpathlineto{\pgfqpoint{6.330464in}{1.819894in}}%
\pgfpathlineto{\pgfqpoint{6.330926in}{2.290082in}}%
\pgfpathlineto{\pgfqpoint{6.331880in}{1.864760in}}%
\pgfpathlineto{\pgfqpoint{6.331002in}{2.421406in}}%
\pgfpathlineto{\pgfqpoint{6.332038in}{2.172648in}}%
\pgfpathlineto{\pgfqpoint{6.332728in}{1.681481in}}%
\pgfpathlineto{\pgfqpoint{6.332352in}{2.381213in}}%
\pgfpathlineto{\pgfqpoint{6.333147in}{2.152381in}}%
\pgfpathlineto{\pgfqpoint{6.333508in}{1.392290in}}%
\pgfpathlineto{\pgfqpoint{6.334259in}{2.409149in}}%
\pgfpathlineto{\pgfqpoint{6.335240in}{1.868140in}}%
\pgfpathlineto{\pgfqpoint{6.335371in}{2.162022in}}%
\pgfpathlineto{\pgfqpoint{6.336468in}{2.423534in}}%
\pgfpathlineto{\pgfqpoint{6.336466in}{1.774023in}}%
\pgfpathlineto{\pgfqpoint{6.336481in}{2.307554in}}%
\pgfpathlineto{\pgfqpoint{6.337210in}{1.767915in}}%
\pgfpathlineto{\pgfqpoint{6.337588in}{2.394805in}}%
\pgfpathlineto{\pgfqpoint{6.337591in}{2.209138in}}%
\pgfpathlineto{\pgfqpoint{6.338499in}{2.419909in}}%
\pgfpathlineto{\pgfqpoint{6.338339in}{1.712740in}}%
\pgfpathlineto{\pgfqpoint{6.338702in}{2.261252in}}%
\pgfpathlineto{\pgfqpoint{6.339627in}{1.614330in}}%
\pgfpathlineto{\pgfqpoint{6.338891in}{2.405294in}}%
\pgfpathlineto{\pgfqpoint{6.339808in}{2.211273in}}%
\pgfpathlineto{\pgfqpoint{6.340362in}{2.389022in}}%
\pgfpathlineto{\pgfqpoint{6.339899in}{1.685942in}}%
\pgfpathlineto{\pgfqpoint{6.340919in}{2.216223in}}%
\pgfpathlineto{\pgfqpoint{6.341148in}{2.408912in}}%
\pgfpathlineto{\pgfqpoint{6.341819in}{1.738691in}}%
\pgfpathlineto{\pgfqpoint{6.342023in}{2.250606in}}%
\pgfpathlineto{\pgfqpoint{6.342025in}{1.729260in}}%
\pgfpathlineto{\pgfqpoint{6.343094in}{2.420030in}}%
\pgfpathlineto{\pgfqpoint{6.343133in}{2.061519in}}%
\pgfpathlineto{\pgfqpoint{6.343647in}{2.425676in}}%
\pgfpathlineto{\pgfqpoint{6.343629in}{1.685864in}}%
\pgfpathlineto{\pgfqpoint{6.344244in}{2.231260in}}%
\pgfpathlineto{\pgfqpoint{6.344785in}{1.793437in}}%
\pgfpathlineto{\pgfqpoint{6.344632in}{2.429437in}}%
\pgfpathlineto{\pgfqpoint{6.345354in}{2.192179in}}%
\pgfpathlineto{\pgfqpoint{6.345864in}{2.437976in}}%
\pgfpathlineto{\pgfqpoint{6.345918in}{1.687233in}}%
\pgfpathlineto{\pgfqpoint{6.346466in}{2.278681in}}%
\pgfpathlineto{\pgfqpoint{6.346518in}{1.532700in}}%
\pgfpathlineto{\pgfqpoint{6.346569in}{2.420789in}}%
\pgfpathlineto{\pgfqpoint{6.347576in}{2.298487in}}%
\pgfpathlineto{\pgfqpoint{6.348249in}{1.646827in}}%
\pgfpathlineto{\pgfqpoint{6.347629in}{2.408002in}}%
\pgfpathlineto{\pgfqpoint{6.348687in}{2.285444in}}%
\pgfpathlineto{\pgfqpoint{6.348744in}{2.429425in}}%
\pgfpathlineto{\pgfqpoint{6.349209in}{1.646039in}}%
\pgfpathlineto{\pgfqpoint{6.349755in}{2.225650in}}%
\pgfpathlineto{\pgfqpoint{6.350633in}{1.712527in}}%
\pgfpathlineto{\pgfqpoint{6.350034in}{2.402822in}}%
\pgfpathlineto{\pgfqpoint{6.350866in}{2.259611in}}%
\pgfpathlineto{\pgfqpoint{6.351036in}{1.794716in}}%
\pgfpathlineto{\pgfqpoint{6.351481in}{2.399534in}}%
\pgfpathlineto{\pgfqpoint{6.351978in}{2.174394in}}%
\pgfpathlineto{\pgfqpoint{6.352695in}{2.425654in}}%
\pgfpathlineto{\pgfqpoint{6.352942in}{1.471698in}}%
\pgfpathlineto{\pgfqpoint{6.353087in}{2.295324in}}%
\pgfpathlineto{\pgfqpoint{6.353435in}{1.719948in}}%
\pgfpathlineto{\pgfqpoint{6.353299in}{2.422910in}}%
\pgfpathlineto{\pgfqpoint{6.354198in}{2.163961in}}%
\pgfpathlineto{\pgfqpoint{6.355277in}{2.413592in}}%
\pgfpathlineto{\pgfqpoint{6.355249in}{1.752646in}}%
\pgfpathlineto{\pgfqpoint{6.355311in}{2.204103in}}%
\pgfpathlineto{\pgfqpoint{6.356168in}{1.512464in}}%
\pgfpathlineto{\pgfqpoint{6.355692in}{2.414602in}}%
\pgfpathlineto{\pgfqpoint{6.356418in}{2.202616in}}%
\pgfpathlineto{\pgfqpoint{6.356979in}{2.406916in}}%
\pgfpathlineto{\pgfqpoint{6.357028in}{1.601762in}}%
\pgfpathlineto{\pgfqpoint{6.357530in}{2.304418in}}%
\pgfpathlineto{\pgfqpoint{6.357958in}{1.666566in}}%
\pgfpathlineto{\pgfqpoint{6.357838in}{2.412132in}}%
\pgfpathlineto{\pgfqpoint{6.358641in}{2.222644in}}%
\pgfpathlineto{\pgfqpoint{6.358666in}{2.400654in}}%
\pgfpathlineto{\pgfqpoint{6.358961in}{1.820814in}}%
\pgfpathlineto{\pgfqpoint{6.359752in}{2.291652in}}%
\pgfpathlineto{\pgfqpoint{6.360381in}{1.495147in}}%
\pgfpathlineto{\pgfqpoint{6.360654in}{2.413559in}}%
\pgfpathlineto{\pgfqpoint{6.360862in}{2.315665in}}%
\pgfpathlineto{\pgfqpoint{6.361816in}{2.407406in}}%
\pgfpathlineto{\pgfqpoint{6.361498in}{1.668915in}}%
\pgfpathlineto{\pgfqpoint{6.361960in}{2.211572in}}%
\pgfpathlineto{\pgfqpoint{6.362587in}{1.610916in}}%
\pgfpathlineto{\pgfqpoint{6.362724in}{2.416883in}}%
\pgfpathlineto{\pgfqpoint{6.363071in}{2.184212in}}%
\pgfpathlineto{\pgfqpoint{6.363550in}{2.405667in}}%
\pgfpathlineto{\pgfqpoint{6.363296in}{1.573186in}}%
\pgfpathlineto{\pgfqpoint{6.364181in}{2.204425in}}%
\pgfpathlineto{\pgfqpoint{6.364217in}{1.215687in}}%
\pgfpathlineto{\pgfqpoint{6.364467in}{2.397617in}}%
\pgfpathlineto{\pgfqpoint{6.365291in}{2.203342in}}%
\pgfpathlineto{\pgfqpoint{6.365480in}{2.400400in}}%
\pgfpathlineto{\pgfqpoint{6.366170in}{1.805385in}}%
\pgfpathlineto{\pgfqpoint{6.366401in}{2.205403in}}%
\pgfpathlineto{\pgfqpoint{6.366811in}{1.654086in}}%
\pgfpathlineto{\pgfqpoint{6.367157in}{2.407245in}}%
\pgfpathlineto{\pgfqpoint{6.367507in}{2.161883in}}%
\pgfpathlineto{\pgfqpoint{6.367719in}{2.404016in}}%
\pgfpathlineto{\pgfqpoint{6.368578in}{1.731781in}}%
\pgfpathlineto{\pgfqpoint{6.368618in}{2.256527in}}%
\pgfpathlineto{\pgfqpoint{6.369008in}{1.703275in}}%
\pgfpathlineto{\pgfqpoint{6.369206in}{2.413327in}}%
\pgfpathlineto{\pgfqpoint{6.369729in}{2.013310in}}%
\pgfpathlineto{\pgfqpoint{6.370807in}{2.398887in}}%
\pgfpathlineto{\pgfqpoint{6.370431in}{1.730506in}}%
\pgfpathlineto{\pgfqpoint{6.370840in}{2.170589in}}%
\pgfpathlineto{\pgfqpoint{6.371160in}{1.800168in}}%
\pgfpathlineto{\pgfqpoint{6.371094in}{2.397599in}}%
\pgfpathlineto{\pgfqpoint{6.371874in}{2.276954in}}%
\pgfpathlineto{\pgfqpoint{6.372706in}{2.400142in}}%
\pgfpathlineto{\pgfqpoint{6.372793in}{1.451141in}}%
\pgfpathlineto{\pgfqpoint{6.372983in}{2.237628in}}%
\pgfpathlineto{\pgfqpoint{6.373342in}{1.711223in}}%
\pgfpathlineto{\pgfqpoint{6.374079in}{2.408941in}}%
\pgfpathlineto{\pgfqpoint{6.374093in}{2.285810in}}%
\pgfpathlineto{\pgfqpoint{6.375170in}{2.399116in}}%
\pgfpathlineto{\pgfqpoint{6.374572in}{1.677180in}}%
\pgfpathlineto{\pgfqpoint{6.375189in}{2.273564in}}%
\pgfpathlineto{\pgfqpoint{6.375850in}{1.302330in}}%
\pgfpathlineto{\pgfqpoint{6.375857in}{2.408598in}}%
\pgfpathlineto{\pgfqpoint{6.376300in}{2.128711in}}%
\pgfpathlineto{\pgfqpoint{6.377349in}{2.461293in}}%
\pgfpathlineto{\pgfqpoint{6.377012in}{1.610103in}}%
\pgfpathlineto{\pgfqpoint{6.377411in}{2.138867in}}%
\pgfpathlineto{\pgfqpoint{6.378144in}{1.650345in}}%
\pgfpathlineto{\pgfqpoint{6.377509in}{2.415811in}}%
\pgfpathlineto{\pgfqpoint{6.378522in}{2.082190in}}%
\pgfpathlineto{\pgfqpoint{6.379249in}{2.392914in}}%
\pgfpathlineto{\pgfqpoint{6.379027in}{1.660085in}}%
\pgfpathlineto{\pgfqpoint{6.379637in}{2.333325in}}%
\pgfpathlineto{\pgfqpoint{6.380234in}{1.726637in}}%
\pgfpathlineto{\pgfqpoint{6.379899in}{2.401033in}}%
\pgfpathlineto{\pgfqpoint{6.380747in}{2.257420in}}%
\pgfpathlineto{\pgfqpoint{6.381796in}{2.414655in}}%
\pgfpathlineto{\pgfqpoint{6.381426in}{1.720459in}}%
\pgfpathlineto{\pgfqpoint{6.381857in}{2.272596in}}%
\pgfpathlineto{\pgfqpoint{6.381946in}{1.784762in}}%
\pgfpathlineto{\pgfqpoint{6.381956in}{2.396809in}}%
\pgfpathlineto{\pgfqpoint{6.382969in}{2.119545in}}%
\pgfpathlineto{\pgfqpoint{6.383427in}{2.403123in}}%
\pgfpathlineto{\pgfqpoint{6.383192in}{1.760397in}}%
\pgfpathlineto{\pgfqpoint{6.384080in}{2.202667in}}%
\pgfpathlineto{\pgfqpoint{6.384626in}{2.393681in}}%
\pgfpathlineto{\pgfqpoint{6.384134in}{1.807656in}}%
\pgfpathlineto{\pgfqpoint{6.385191in}{2.303205in}}%
\pgfpathlineto{\pgfqpoint{6.385710in}{1.777287in}}%
\pgfpathlineto{\pgfqpoint{6.385733in}{2.400937in}}%
\pgfpathlineto{\pgfqpoint{6.386303in}{1.908916in}}%
\pgfpathlineto{\pgfqpoint{6.387176in}{2.408000in}}%
\pgfpathlineto{\pgfqpoint{6.386546in}{1.754205in}}%
\pgfpathlineto{\pgfqpoint{6.387415in}{2.317116in}}%
\pgfpathlineto{\pgfqpoint{6.387926in}{1.618361in}}%
\pgfpathlineto{\pgfqpoint{6.387572in}{2.404869in}}%
\pgfpathlineto{\pgfqpoint{6.388526in}{2.209233in}}%
\pgfpathlineto{\pgfqpoint{6.389485in}{2.424489in}}%
\pgfpathlineto{\pgfqpoint{6.388924in}{1.602825in}}%
\pgfpathlineto{\pgfqpoint{6.389634in}{2.299161in}}%
\pgfpathlineto{\pgfqpoint{6.389953in}{1.777903in}}%
\pgfpathlineto{\pgfqpoint{6.389738in}{2.404117in}}%
\pgfpathlineto{\pgfqpoint{6.390745in}{2.121012in}}%
\pgfpathlineto{\pgfqpoint{6.390929in}{2.402782in}}%
\pgfpathlineto{\pgfqpoint{6.391826in}{1.626210in}}%
\pgfpathlineto{\pgfqpoint{6.391856in}{2.327349in}}%
\pgfpathlineto{\pgfqpoint{6.392525in}{1.655873in}}%
\pgfpathlineto{\pgfqpoint{6.392600in}{2.418907in}}%
\pgfpathlineto{\pgfqpoint{6.392967in}{2.264864in}}%
\pgfpathlineto{\pgfqpoint{6.393037in}{1.236111in}}%
\pgfpathlineto{\pgfqpoint{6.393001in}{2.408114in}}%
\pgfpathlineto{\pgfqpoint{6.394079in}{2.213613in}}%
\pgfpathlineto{\pgfqpoint{6.394995in}{2.417055in}}%
\pgfpathlineto{\pgfqpoint{6.394759in}{1.813288in}}%
\pgfpathlineto{\pgfqpoint{6.395189in}{2.174758in}}%
\pgfpathlineto{\pgfqpoint{6.395457in}{1.785297in}}%
\pgfpathlineto{\pgfqpoint{6.396227in}{2.396577in}}%
\pgfpathlineto{\pgfqpoint{6.396295in}{2.047878in}}%
\pgfpathlineto{\pgfqpoint{6.396662in}{2.398063in}}%
\pgfpathlineto{\pgfqpoint{6.396918in}{1.655417in}}%
\pgfpathlineto{\pgfqpoint{6.397406in}{2.316609in}}%
\pgfpathlineto{\pgfqpoint{6.397612in}{1.725580in}}%
\pgfpathlineto{\pgfqpoint{6.398356in}{2.384731in}}%
\pgfpathlineto{\pgfqpoint{6.398518in}{2.079649in}}%
\pgfpathlineto{\pgfqpoint{6.399517in}{2.393681in}}%
\pgfpathlineto{\pgfqpoint{6.398757in}{1.652859in}}%
\pgfpathlineto{\pgfqpoint{6.399628in}{2.148245in}}%
\pgfpathlineto{\pgfqpoint{6.399678in}{1.768482in}}%
\pgfpathlineto{\pgfqpoint{6.400594in}{2.396752in}}%
\pgfpathlineto{\pgfqpoint{6.400738in}{2.302323in}}%
\pgfpathlineto{\pgfqpoint{6.401402in}{2.385106in}}%
\pgfpathlineto{\pgfqpoint{6.401135in}{1.721625in}}%
\pgfpathlineto{\pgfqpoint{6.401838in}{2.255575in}}%
\pgfpathlineto{\pgfqpoint{6.402389in}{1.660503in}}%
\pgfpathlineto{\pgfqpoint{6.402510in}{2.402765in}}%
\pgfpathlineto{\pgfqpoint{6.402948in}{2.247513in}}%
\pgfpathlineto{\pgfqpoint{6.403366in}{2.431682in}}%
\pgfpathlineto{\pgfqpoint{6.403287in}{1.528981in}}%
\pgfpathlineto{\pgfqpoint{6.404048in}{2.215056in}}%
\pgfpathlineto{\pgfqpoint{6.404255in}{1.749735in}}%
\pgfpathlineto{\pgfqpoint{6.404549in}{2.416067in}}%
\pgfpathlineto{\pgfqpoint{6.405158in}{2.290323in}}%
\pgfpathlineto{\pgfqpoint{6.405452in}{2.408194in}}%
\pgfpathlineto{\pgfqpoint{6.405183in}{1.748272in}}%
\pgfpathlineto{\pgfqpoint{6.406268in}{2.304255in}}%
\pgfpathlineto{\pgfqpoint{6.406887in}{1.810016in}}%
\pgfpathlineto{\pgfqpoint{6.407086in}{2.399607in}}%
\pgfpathlineto{\pgfqpoint{6.407379in}{2.270128in}}%
\pgfpathlineto{\pgfqpoint{6.407886in}{1.707470in}}%
\pgfpathlineto{\pgfqpoint{6.408327in}{2.400555in}}%
\pgfpathlineto{\pgfqpoint{6.408490in}{2.214072in}}%
\pgfpathlineto{\pgfqpoint{6.408527in}{1.660607in}}%
\pgfpathlineto{\pgfqpoint{6.408800in}{2.389627in}}%
\pgfpathlineto{\pgfqpoint{6.409598in}{2.105838in}}%
\pgfpathlineto{\pgfqpoint{6.410149in}{2.404692in}}%
\pgfpathlineto{\pgfqpoint{6.410481in}{1.655024in}}%
\pgfpathlineto{\pgfqpoint{6.410709in}{2.220688in}}%
\pgfpathlineto{\pgfqpoint{6.410929in}{2.386248in}}%
\pgfpathlineto{\pgfqpoint{6.411292in}{1.483768in}}%
\pgfpathlineto{\pgfqpoint{6.411819in}{2.187721in}}%
\pgfpathlineto{\pgfqpoint{6.412706in}{1.623045in}}%
\pgfpathlineto{\pgfqpoint{6.412414in}{2.400866in}}%
\pgfpathlineto{\pgfqpoint{6.412922in}{2.270950in}}%
\pgfpathlineto{\pgfqpoint{6.413632in}{2.406077in}}%
\pgfpathlineto{\pgfqpoint{6.413555in}{1.701161in}}%
\pgfpathlineto{\pgfqpoint{6.414031in}{2.023583in}}%
\pgfpathlineto{\pgfqpoint{6.414728in}{2.414987in}}%
\pgfpathlineto{\pgfqpoint{6.414421in}{1.665096in}}%
\pgfpathlineto{\pgfqpoint{6.415143in}{2.345240in}}%
\pgfpathlineto{\pgfqpoint{6.415839in}{1.763531in}}%
\pgfpathlineto{\pgfqpoint{6.416069in}{2.418994in}}%
\pgfpathlineto{\pgfqpoint{6.416254in}{2.286686in}}%
\pgfpathlineto{\pgfqpoint{6.416894in}{2.418621in}}%
\pgfpathlineto{\pgfqpoint{6.416489in}{1.528527in}}%
\pgfpathlineto{\pgfqpoint{6.417323in}{2.345790in}}%
\pgfpathlineto{\pgfqpoint{6.417767in}{1.795327in}}%
\pgfpathlineto{\pgfqpoint{6.417788in}{2.387512in}}%
\pgfpathlineto{\pgfqpoint{6.418434in}{2.254363in}}%
\pgfpathlineto{\pgfqpoint{6.418808in}{1.784535in}}%
\pgfpathlineto{\pgfqpoint{6.418550in}{2.393186in}}%
\pgfpathlineto{\pgfqpoint{6.419546in}{2.182857in}}%
\pgfpathlineto{\pgfqpoint{6.420185in}{2.389123in}}%
\pgfpathlineto{\pgfqpoint{6.420269in}{1.727546in}}%
\pgfpathlineto{\pgfqpoint{6.420656in}{2.239332in}}%
\pgfpathlineto{\pgfqpoint{6.421461in}{1.509656in}}%
\pgfpathlineto{\pgfqpoint{6.420853in}{2.396140in}}%
\pgfpathlineto{\pgfqpoint{6.421767in}{2.070410in}}%
\pgfpathlineto{\pgfqpoint{6.422642in}{2.402195in}}%
\pgfpathlineto{\pgfqpoint{6.422175in}{1.568834in}}%
\pgfpathlineto{\pgfqpoint{6.422878in}{2.282668in}}%
\pgfpathlineto{\pgfqpoint{6.423622in}{1.582001in}}%
\pgfpathlineto{\pgfqpoint{6.423254in}{2.409440in}}%
\pgfpathlineto{\pgfqpoint{6.423990in}{2.246918in}}%
\pgfpathlineto{\pgfqpoint{6.424045in}{2.397648in}}%
\pgfpathlineto{\pgfqpoint{6.425014in}{1.603587in}}%
\pgfpathlineto{\pgfqpoint{6.425093in}{2.278728in}}%
\pgfpathlineto{\pgfqpoint{6.425316in}{1.825361in}}%
\pgfpathlineto{\pgfqpoint{6.425834in}{2.391615in}}%
\pgfpathlineto{\pgfqpoint{6.426204in}{2.144815in}}%
\pgfpathlineto{\pgfqpoint{6.426858in}{2.407116in}}%
\pgfpathlineto{\pgfqpoint{6.427217in}{1.531594in}}%
\pgfpathlineto{\pgfqpoint{6.427315in}{2.306782in}}%
\pgfpathlineto{\pgfqpoint{6.427943in}{1.760042in}}%
\pgfpathlineto{\pgfqpoint{6.428146in}{2.404943in}}%
\pgfpathlineto{\pgfqpoint{6.428325in}{2.294714in}}%
\pgfpathlineto{\pgfqpoint{6.428596in}{2.405127in}}%
\pgfpathlineto{\pgfqpoint{6.428721in}{1.794852in}}%
\pgfpathlineto{\pgfqpoint{6.429435in}{2.236676in}}%
\pgfpathlineto{\pgfqpoint{6.430081in}{1.697662in}}%
\pgfpathlineto{\pgfqpoint{6.429475in}{2.408765in}}%
\pgfpathlineto{\pgfqpoint{6.430548in}{2.115089in}}%
\pgfpathlineto{\pgfqpoint{6.431148in}{2.426135in}}%
\pgfpathlineto{\pgfqpoint{6.431010in}{1.767449in}}%
\pgfpathlineto{\pgfqpoint{6.431659in}{2.142001in}}%
\pgfpathlineto{\pgfqpoint{6.432152in}{1.626541in}}%
\pgfpathlineto{\pgfqpoint{6.431757in}{2.402088in}}%
\pgfpathlineto{\pgfqpoint{6.432769in}{2.259344in}}%
\pgfpathlineto{\pgfqpoint{6.433824in}{2.415287in}}%
\pgfpathlineto{\pgfqpoint{6.432944in}{1.636751in}}%
\pgfpathlineto{\pgfqpoint{6.433874in}{2.205476in}}%
\pgfpathlineto{\pgfqpoint{6.434813in}{1.758271in}}%
\pgfpathlineto{\pgfqpoint{6.434798in}{2.383110in}}%
\pgfpathlineto{\pgfqpoint{6.434985in}{2.262226in}}%
\pgfpathlineto{\pgfqpoint{6.435276in}{1.839035in}}%
\pgfpathlineto{\pgfqpoint{6.435480in}{2.403490in}}%
\pgfpathlineto{\pgfqpoint{6.436096in}{2.243113in}}%
\pgfpathlineto{\pgfqpoint{6.436415in}{1.725212in}}%
\pgfpathlineto{\pgfqpoint{6.436107in}{2.406536in}}%
\pgfpathlineto{\pgfqpoint{6.437194in}{2.045735in}}%
\pgfpathlineto{\pgfqpoint{6.437736in}{2.404228in}}%
\pgfpathlineto{\pgfqpoint{6.437285in}{1.737234in}}%
\pgfpathlineto{\pgfqpoint{6.438305in}{2.211265in}}%
\pgfpathlineto{\pgfqpoint{6.439127in}{2.397069in}}%
\pgfpathlineto{\pgfqpoint{6.438310in}{1.702820in}}%
\pgfpathlineto{\pgfqpoint{6.439416in}{2.321236in}}%
\pgfpathlineto{\pgfqpoint{6.439655in}{1.629856in}}%
\pgfpathlineto{\pgfqpoint{6.439810in}{2.398995in}}%
\pgfpathlineto{\pgfqpoint{6.440527in}{2.287901in}}%
\pgfpathlineto{\pgfqpoint{6.441615in}{2.412335in}}%
\pgfpathlineto{\pgfqpoint{6.441498in}{1.693415in}}%
\pgfpathlineto{\pgfqpoint{6.441637in}{2.276774in}}%
\pgfpathlineto{\pgfqpoint{6.442701in}{1.646902in}}%
\pgfpathlineto{\pgfqpoint{6.442399in}{2.395611in}}%
\pgfpathlineto{\pgfqpoint{6.442748in}{2.280764in}}%
\pgfpathlineto{\pgfqpoint{6.443644in}{2.381801in}}%
\pgfpathlineto{\pgfqpoint{6.442850in}{1.799354in}}%
\pgfpathlineto{\pgfqpoint{6.443853in}{2.210986in}}%
\pgfpathlineto{\pgfqpoint{6.443880in}{1.774791in}}%
\pgfpathlineto{\pgfqpoint{6.444117in}{2.392409in}}%
\pgfpathlineto{\pgfqpoint{6.444963in}{2.090866in}}%
\pgfpathlineto{\pgfqpoint{6.445798in}{2.390361in}}%
\pgfpathlineto{\pgfqpoint{6.446070in}{1.713212in}}%
\pgfpathlineto{\pgfqpoint{6.446074in}{2.214535in}}%
\pgfpathlineto{\pgfqpoint{6.446593in}{1.740383in}}%
\pgfpathlineto{\pgfqpoint{6.446540in}{2.436178in}}%
\pgfpathlineto{\pgfqpoint{6.447181in}{2.216283in}}%
\pgfpathlineto{\pgfqpoint{6.447447in}{2.379858in}}%
\pgfpathlineto{\pgfqpoint{6.447567in}{1.825148in}}%
\pgfpathlineto{\pgfqpoint{6.448292in}{2.259084in}}%
\pgfpathlineto{\pgfqpoint{6.448721in}{1.700451in}}%
\pgfpathlineto{\pgfqpoint{6.448592in}{2.397710in}}%
\pgfpathlineto{\pgfqpoint{6.449403in}{2.236549in}}%
\pgfpathlineto{\pgfqpoint{6.450432in}{2.388675in}}%
\pgfpathlineto{\pgfqpoint{6.450390in}{1.772979in}}%
\pgfpathlineto{\pgfqpoint{6.450510in}{2.311932in}}%
\pgfpathlineto{\pgfqpoint{6.450832in}{1.706465in}}%
\pgfpathlineto{\pgfqpoint{6.450930in}{2.385722in}}%
\pgfpathlineto{\pgfqpoint{6.451620in}{2.296462in}}%
\pgfpathlineto{\pgfqpoint{6.452234in}{1.494585in}}%
\pgfpathlineto{\pgfqpoint{6.452664in}{2.407885in}}%
\pgfpathlineto{\pgfqpoint{6.452733in}{2.192212in}}%
\pgfpathlineto{\pgfqpoint{6.453493in}{2.412677in}}%
\pgfpathlineto{\pgfqpoint{6.453282in}{1.579535in}}%
\pgfpathlineto{\pgfqpoint{6.453844in}{2.252899in}}%
\pgfpathlineto{\pgfqpoint{6.454231in}{1.593008in}}%
\pgfpathlineto{\pgfqpoint{6.454492in}{2.386365in}}%
\pgfpathlineto{\pgfqpoint{6.454954in}{2.227987in}}%
\pgfpathlineto{\pgfqpoint{6.455025in}{2.409986in}}%
\pgfpathlineto{\pgfqpoint{6.455557in}{1.580212in}}%
\pgfpathlineto{\pgfqpoint{6.456064in}{2.163618in}}%
\pgfpathlineto{\pgfqpoint{6.456785in}{1.603881in}}%
\pgfpathlineto{\pgfqpoint{6.456930in}{2.401855in}}%
\pgfpathlineto{\pgfqpoint{6.457174in}{2.091136in}}%
\pgfpathlineto{\pgfqpoint{6.457816in}{2.384416in}}%
\pgfpathlineto{\pgfqpoint{6.457542in}{1.743307in}}%
\pgfpathlineto{\pgfqpoint{6.458286in}{2.341222in}}%
\pgfpathlineto{\pgfqpoint{6.458672in}{1.721512in}}%
\pgfpathlineto{\pgfqpoint{6.459290in}{2.397231in}}%
\pgfpathlineto{\pgfqpoint{6.459397in}{2.239644in}}%
\pgfpathlineto{\pgfqpoint{6.459744in}{2.390805in}}%
\pgfpathlineto{\pgfqpoint{6.459555in}{1.750288in}}%
\pgfpathlineto{\pgfqpoint{6.460506in}{2.186368in}}%
\pgfpathlineto{\pgfqpoint{6.461550in}{1.735941in}}%
\pgfpathlineto{\pgfqpoint{6.461259in}{2.379525in}}%
\pgfpathlineto{\pgfqpoint{6.461617in}{2.149165in}}%
\pgfpathlineto{\pgfqpoint{6.461903in}{2.391473in}}%
\pgfpathlineto{\pgfqpoint{6.461625in}{1.718827in}}%
\pgfpathlineto{\pgfqpoint{6.462728in}{2.294155in}}%
\pgfpathlineto{\pgfqpoint{6.462858in}{1.744628in}}%
\pgfpathlineto{\pgfqpoint{6.463343in}{2.387229in}}%
\pgfpathlineto{\pgfqpoint{6.463839in}{2.113780in}}%
\pgfpathlineto{\pgfqpoint{6.464329in}{2.397005in}}%
\pgfpathlineto{\pgfqpoint{6.464611in}{1.737782in}}%
\pgfpathlineto{\pgfqpoint{6.464951in}{2.197737in}}%
\pgfpathlineto{\pgfqpoint{6.465714in}{1.767532in}}%
\pgfpathlineto{\pgfqpoint{6.465353in}{2.402393in}}%
\pgfpathlineto{\pgfqpoint{6.466061in}{2.194175in}}%
\pgfpathlineto{\pgfqpoint{6.466133in}{2.389821in}}%
\pgfpathlineto{\pgfqpoint{6.466374in}{1.791651in}}%
\pgfpathlineto{\pgfqpoint{6.467171in}{2.237089in}}%
\pgfpathlineto{\pgfqpoint{6.467258in}{1.504646in}}%
\pgfpathlineto{\pgfqpoint{6.467541in}{2.391954in}}%
\pgfpathlineto{\pgfqpoint{6.468282in}{2.181775in}}%
\pgfpathlineto{\pgfqpoint{6.468971in}{2.392655in}}%
\pgfpathlineto{\pgfqpoint{6.468601in}{1.610132in}}%
\pgfpathlineto{\pgfqpoint{6.469393in}{2.291242in}}%
\pgfpathlineto{\pgfqpoint{6.470162in}{1.694883in}}%
\pgfpathlineto{\pgfqpoint{6.469491in}{2.386471in}}%
\pgfpathlineto{\pgfqpoint{6.470505in}{2.155644in}}%
\pgfpathlineto{\pgfqpoint{6.471131in}{2.405391in}}%
\pgfpathlineto{\pgfqpoint{6.471343in}{1.634150in}}%
\pgfpathlineto{\pgfqpoint{6.471616in}{2.232501in}}%
\pgfpathlineto{\pgfqpoint{6.472694in}{1.644184in}}%
\pgfpathlineto{\pgfqpoint{6.472181in}{2.385647in}}%
\pgfpathlineto{\pgfqpoint{6.472727in}{2.207878in}}%
\pgfpathlineto{\pgfqpoint{6.473230in}{2.375087in}}%
\pgfpathlineto{\pgfqpoint{6.472960in}{1.506607in}}%
\pgfpathlineto{\pgfqpoint{6.473839in}{2.272430in}}%
\pgfpathlineto{\pgfqpoint{6.474303in}{1.617387in}}%
\pgfpathlineto{\pgfqpoint{6.474087in}{2.404414in}}%
\pgfpathlineto{\pgfqpoint{6.474949in}{2.160522in}}%
\pgfpathlineto{\pgfqpoint{6.475303in}{2.380712in}}%
\pgfpathlineto{\pgfqpoint{6.475259in}{1.763654in}}%
\pgfpathlineto{\pgfqpoint{6.476060in}{2.288898in}}%
\pgfpathlineto{\pgfqpoint{6.476860in}{1.588728in}}%
\pgfpathlineto{\pgfqpoint{6.477062in}{2.375182in}}%
\pgfpathlineto{\pgfqpoint{6.477171in}{2.221795in}}%
\pgfpathlineto{\pgfqpoint{6.477954in}{1.759818in}}%
\pgfpathlineto{\pgfqpoint{6.477919in}{2.380639in}}%
\pgfpathlineto{\pgfqpoint{6.478282in}{2.141776in}}%
\pgfpathlineto{\pgfqpoint{6.478824in}{2.385515in}}%
\pgfpathlineto{\pgfqpoint{6.479113in}{1.768921in}}%
\pgfpathlineto{\pgfqpoint{6.479394in}{2.207116in}}%
\pgfpathlineto{\pgfqpoint{6.479535in}{1.687035in}}%
\pgfpathlineto{\pgfqpoint{6.479766in}{2.384470in}}%
\pgfpathlineto{\pgfqpoint{6.480503in}{2.324286in}}%
\pgfpathlineto{\pgfqpoint{6.481047in}{2.395253in}}%
\pgfpathlineto{\pgfqpoint{6.480957in}{1.568254in}}%
\pgfpathlineto{\pgfqpoint{6.481468in}{2.146637in}}%
\pgfpathlineto{\pgfqpoint{6.482139in}{1.732493in}}%
\pgfpathlineto{\pgfqpoint{6.482214in}{2.391148in}}%
\pgfpathlineto{\pgfqpoint{6.482578in}{2.081395in}}%
\pgfpathlineto{\pgfqpoint{6.483660in}{2.399301in}}%
\pgfpathlineto{\pgfqpoint{6.483288in}{1.655324in}}%
\pgfpathlineto{\pgfqpoint{6.483690in}{2.270877in}}%
\pgfpathlineto{\pgfqpoint{6.484232in}{1.632893in}}%
\pgfpathlineto{\pgfqpoint{6.484404in}{2.381230in}}%
\pgfpathlineto{\pgfqpoint{6.484801in}{2.068092in}}%
\pgfpathlineto{\pgfqpoint{6.485436in}{1.741631in}}%
\pgfpathlineto{\pgfqpoint{6.485662in}{2.408380in}}%
\pgfpathlineto{\pgfqpoint{6.485904in}{2.191801in}}%
\pgfpathlineto{\pgfqpoint{6.486681in}{2.387705in}}%
\pgfpathlineto{\pgfqpoint{6.486376in}{1.761127in}}%
\pgfpathlineto{\pgfqpoint{6.487014in}{2.200519in}}%
\pgfpathlineto{\pgfqpoint{6.487863in}{1.787443in}}%
\pgfpathlineto{\pgfqpoint{6.487797in}{2.387093in}}%
\pgfpathlineto{\pgfqpoint{6.488125in}{2.087363in}}%
\pgfpathlineto{\pgfqpoint{6.488479in}{2.394785in}}%
\pgfpathlineto{\pgfqpoint{6.488995in}{1.576804in}}%
\pgfpathlineto{\pgfqpoint{6.489236in}{2.303477in}}%
\pgfpathlineto{\pgfqpoint{6.490316in}{1.806643in}}%
\pgfpathlineto{\pgfqpoint{6.490143in}{2.377682in}}%
\pgfpathlineto{\pgfqpoint{6.490348in}{2.080403in}}%
\pgfpathlineto{\pgfqpoint{6.491199in}{2.392149in}}%
\pgfpathlineto{\pgfqpoint{6.491348in}{1.717354in}}%
\pgfpathlineto{\pgfqpoint{6.491459in}{2.281628in}}%
\pgfpathlineto{\pgfqpoint{6.492368in}{1.706416in}}%
\pgfpathlineto{\pgfqpoint{6.492022in}{2.384677in}}%
\pgfpathlineto{\pgfqpoint{6.492570in}{2.259729in}}%
\pgfpathlineto{\pgfqpoint{6.493381in}{1.611803in}}%
\pgfpathlineto{\pgfqpoint{6.493515in}{2.389660in}}%
\pgfpathlineto{\pgfqpoint{6.493681in}{2.126850in}}%
\pgfpathlineto{\pgfqpoint{6.494010in}{2.386581in}}%
\pgfpathlineto{\pgfqpoint{6.494456in}{1.702762in}}%
\pgfpathlineto{\pgfqpoint{6.494793in}{2.290107in}}%
\pgfpathlineto{\pgfqpoint{6.495634in}{1.639264in}}%
\pgfpathlineto{\pgfqpoint{6.495529in}{2.387853in}}%
\pgfpathlineto{\pgfqpoint{6.495904in}{2.015335in}}%
\pgfpathlineto{\pgfqpoint{6.496966in}{2.371032in}}%
\pgfpathlineto{\pgfqpoint{6.496223in}{1.672025in}}%
\pgfpathlineto{\pgfqpoint{6.497015in}{2.238130in}}%
\pgfpathlineto{\pgfqpoint{6.497662in}{1.699730in}}%
\pgfpathlineto{\pgfqpoint{6.498048in}{2.371225in}}%
\pgfpathlineto{\pgfqpoint{6.498126in}{2.158733in}}%
\pgfpathlineto{\pgfqpoint{6.499048in}{1.579680in}}%
\pgfpathlineto{\pgfqpoint{6.499062in}{2.385888in}}%
\pgfpathlineto{\pgfqpoint{6.499231in}{2.250622in}}%
\pgfpathlineto{\pgfqpoint{6.500020in}{2.384414in}}%
\pgfpathlineto{\pgfqpoint{6.499233in}{1.685485in}}%
\pgfpathlineto{\pgfqpoint{6.500341in}{2.176556in}}%
\pgfpathlineto{\pgfqpoint{6.501415in}{2.393204in}}%
\pgfpathlineto{\pgfqpoint{6.501092in}{1.747885in}}%
\pgfpathlineto{\pgfqpoint{6.501451in}{2.147326in}}%
\pgfpathlineto{\pgfqpoint{6.501810in}{1.755563in}}%
\pgfpathlineto{\pgfqpoint{6.502414in}{2.391330in}}%
\pgfpathlineto{\pgfqpoint{6.502561in}{2.053961in}}%
\pgfpathlineto{\pgfqpoint{6.502771in}{2.386693in}}%
\pgfpathlineto{\pgfqpoint{6.503345in}{1.608293in}}%
\pgfpathlineto{\pgfqpoint{6.503673in}{2.190697in}}%
\pgfpathlineto{\pgfqpoint{6.503883in}{1.686904in}}%
\pgfpathlineto{\pgfqpoint{6.504553in}{2.401203in}}%
\pgfpathlineto{\pgfqpoint{6.504783in}{2.253815in}}%
\pgfpathlineto{\pgfqpoint{6.505624in}{2.384209in}}%
\pgfpathlineto{\pgfqpoint{6.505362in}{1.636954in}}%
\pgfpathlineto{\pgfqpoint{6.505834in}{2.256633in}}%
\pgfpathlineto{\pgfqpoint{6.505980in}{1.755043in}}%
\pgfpathlineto{\pgfqpoint{6.506351in}{2.386578in}}%
\pgfpathlineto{\pgfqpoint{6.506945in}{2.274323in}}%
\pgfpathlineto{\pgfqpoint{6.507065in}{1.630358in}}%
\pgfpathlineto{\pgfqpoint{6.507074in}{2.391341in}}%
\pgfpathlineto{\pgfqpoint{6.508056in}{2.241458in}}%
\pgfpathlineto{\pgfqpoint{6.508544in}{2.383852in}}%
\pgfpathlineto{\pgfqpoint{6.508935in}{1.618746in}}%
\pgfpathlineto{\pgfqpoint{6.509165in}{2.239013in}}%
\pgfpathlineto{\pgfqpoint{6.510144in}{1.715504in}}%
\pgfpathlineto{\pgfqpoint{6.510248in}{2.378142in}}%
\pgfpathlineto{\pgfqpoint{6.510276in}{2.264377in}}%
\pgfpathlineto{\pgfqpoint{6.510669in}{2.390777in}}%
\pgfpathlineto{\pgfqpoint{6.510978in}{1.551598in}}%
\pgfpathlineto{\pgfqpoint{6.511385in}{2.333903in}}%
\pgfpathlineto{\pgfqpoint{6.511773in}{1.675853in}}%
\pgfpathlineto{\pgfqpoint{6.512297in}{2.392953in}}%
\pgfpathlineto{\pgfqpoint{6.512496in}{2.017229in}}%
\pgfpathlineto{\pgfqpoint{6.513590in}{2.382790in}}%
\pgfpathlineto{\pgfqpoint{6.512998in}{1.778512in}}%
\pgfpathlineto{\pgfqpoint{6.513606in}{2.112407in}}%
\pgfpathlineto{\pgfqpoint{6.514206in}{1.773387in}}%
\pgfpathlineto{\pgfqpoint{6.514019in}{2.386231in}}%
\pgfpathlineto{\pgfqpoint{6.514716in}{2.193842in}}%
\pgfpathlineto{\pgfqpoint{6.515274in}{2.375844in}}%
\pgfpathlineto{\pgfqpoint{6.515272in}{1.746451in}}%
\pgfpathlineto{\pgfqpoint{6.515826in}{2.173021in}}%
\pgfpathlineto{\pgfqpoint{6.516895in}{1.611618in}}%
\pgfpathlineto{\pgfqpoint{6.516861in}{2.395023in}}%
\pgfpathlineto{\pgfqpoint{6.516936in}{2.086039in}}%
\pgfpathlineto{\pgfqpoint{6.517427in}{2.396655in}}%
\pgfpathlineto{\pgfqpoint{6.517698in}{1.651035in}}%
\pgfpathlineto{\pgfqpoint{6.518047in}{2.183164in}}%
\pgfpathlineto{\pgfqpoint{6.518980in}{1.746571in}}%
\pgfpathlineto{\pgfqpoint{6.518640in}{2.381721in}}%
\pgfpathlineto{\pgfqpoint{6.519154in}{2.183765in}}%
\pgfpathlineto{\pgfqpoint{6.519704in}{2.368340in}}%
\pgfpathlineto{\pgfqpoint{6.519586in}{1.617510in}}%
\pgfpathlineto{\pgfqpoint{6.520265in}{2.319044in}}%
\pgfpathlineto{\pgfqpoint{6.520895in}{1.786716in}}%
\pgfpathlineto{\pgfqpoint{6.521232in}{2.380158in}}%
\pgfpathlineto{\pgfqpoint{6.521377in}{2.114721in}}%
\pgfpathlineto{\pgfqpoint{6.521472in}{1.731368in}}%
\pgfpathlineto{\pgfqpoint{6.522101in}{2.378044in}}%
\pgfpathlineto{\pgfqpoint{6.522480in}{2.265678in}}%
\pgfpathlineto{\pgfqpoint{6.523424in}{2.390061in}}%
\pgfpathlineto{\pgfqpoint{6.523526in}{1.604444in}}%
\pgfpathlineto{\pgfqpoint{6.523573in}{2.067056in}}%
\pgfpathlineto{\pgfqpoint{6.524415in}{1.583479in}}%
\pgfpathlineto{\pgfqpoint{6.523632in}{2.384339in}}%
\pgfpathlineto{\pgfqpoint{6.524684in}{2.151756in}}%
\pgfpathlineto{\pgfqpoint{6.524983in}{2.395388in}}%
\pgfpathlineto{\pgfqpoint{6.525012in}{1.691180in}}%
\pgfpathlineto{\pgfqpoint{6.525796in}{2.284799in}}%
\pgfpathlineto{\pgfqpoint{6.525938in}{2.385275in}}%
\pgfpathlineto{\pgfqpoint{6.526085in}{1.744294in}}%
\pgfpathlineto{\pgfqpoint{6.526607in}{2.220297in}}%
\pgfpathlineto{\pgfqpoint{6.526640in}{1.755090in}}%
\pgfpathlineto{\pgfqpoint{6.526788in}{2.413776in}}%
\pgfpathlineto{\pgfqpoint{6.527718in}{2.114505in}}%
\pgfpathlineto{\pgfqpoint{6.528114in}{2.379294in}}%
\pgfpathlineto{\pgfqpoint{6.528766in}{1.533253in}}%
\pgfpathlineto{\pgfqpoint{6.528829in}{2.252955in}}%
\pgfpathlineto{\pgfqpoint{6.529713in}{1.138944in}}%
\pgfpathlineto{\pgfqpoint{6.529801in}{2.375000in}}%
\pgfpathlineto{\pgfqpoint{6.529940in}{2.224237in}}%
\pgfpathlineto{\pgfqpoint{6.530805in}{1.639833in}}%
\pgfpathlineto{\pgfqpoint{6.530393in}{2.391143in}}%
\pgfpathlineto{\pgfqpoint{6.531051in}{2.198446in}}%
\pgfpathlineto{\pgfqpoint{6.532113in}{2.385425in}}%
\pgfpathlineto{\pgfqpoint{6.531429in}{1.692998in}}%
\pgfpathlineto{\pgfqpoint{6.532162in}{2.239445in}}%
\pgfpathlineto{\pgfqpoint{6.532274in}{1.683023in}}%
\pgfpathlineto{\pgfqpoint{6.532452in}{2.398233in}}%
\pgfpathlineto{\pgfqpoint{6.533273in}{2.229142in}}%
\pgfpathlineto{\pgfqpoint{6.533410in}{2.375804in}}%
\pgfpathlineto{\pgfqpoint{6.533471in}{1.613468in}}%
\pgfpathlineto{\pgfqpoint{6.534381in}{2.173086in}}%
\pgfpathlineto{\pgfqpoint{6.534998in}{1.608362in}}%
\pgfpathlineto{\pgfqpoint{6.535441in}{2.392112in}}%
\pgfpathlineto{\pgfqpoint{6.535491in}{2.002973in}}%
\pgfpathlineto{\pgfqpoint{6.536460in}{2.385084in}}%
\pgfpathlineto{\pgfqpoint{6.535667in}{1.457694in}}%
\pgfpathlineto{\pgfqpoint{6.536603in}{2.247996in}}%
\pgfpathlineto{\pgfqpoint{6.537447in}{1.755911in}}%
\pgfpathlineto{\pgfqpoint{6.537700in}{2.389086in}}%
\pgfpathlineto{\pgfqpoint{6.537715in}{2.051049in}}%
\pgfpathlineto{\pgfqpoint{6.538303in}{2.398393in}}%
\pgfpathlineto{\pgfqpoint{6.538751in}{1.628392in}}%
\pgfpathlineto{\pgfqpoint{6.538827in}{2.309518in}}%
\pgfpathlineto{\pgfqpoint{6.539380in}{2.379824in}}%
\pgfpathlineto{\pgfqpoint{6.539278in}{1.719004in}}%
\pgfpathlineto{\pgfqpoint{6.539927in}{2.300430in}}%
\pgfpathlineto{\pgfqpoint{6.540502in}{1.684753in}}%
\pgfpathlineto{\pgfqpoint{6.540063in}{2.378062in}}%
\pgfpathlineto{\pgfqpoint{6.541038in}{1.999806in}}%
\pgfpathlineto{\pgfqpoint{6.541682in}{2.376268in}}%
\pgfpathlineto{\pgfqpoint{6.541742in}{1.349708in}}%
\pgfpathlineto{\pgfqpoint{6.542149in}{2.277082in}}%
\pgfpathlineto{\pgfqpoint{6.542393in}{1.641391in}}%
\pgfpathlineto{\pgfqpoint{6.542928in}{2.379087in}}%
\pgfpathlineto{\pgfqpoint{6.543261in}{1.914285in}}%
\pgfpathlineto{\pgfqpoint{6.544207in}{2.374688in}}%
\pgfpathlineto{\pgfqpoint{6.544335in}{1.679275in}}%
\pgfpathlineto{\pgfqpoint{6.544372in}{2.142091in}}%
\pgfpathlineto{\pgfqpoint{6.544707in}{2.377391in}}%
\pgfpathlineto{\pgfqpoint{6.544645in}{1.725982in}}%
\pgfpathlineto{\pgfqpoint{6.545471in}{2.223793in}}%
\pgfpathlineto{\pgfqpoint{6.546090in}{1.629002in}}%
\pgfpathlineto{\pgfqpoint{6.546054in}{2.361047in}}%
\pgfpathlineto{\pgfqpoint{6.546582in}{2.278514in}}%
\pgfpathlineto{\pgfqpoint{6.547334in}{1.792131in}}%
\pgfpathlineto{\pgfqpoint{6.547165in}{2.384368in}}%
\pgfpathlineto{\pgfqpoint{6.547692in}{2.183899in}}%
\pgfpathlineto{\pgfqpoint{6.548500in}{2.380752in}}%
\pgfpathlineto{\pgfqpoint{6.547827in}{1.694230in}}%
\pgfpathlineto{\pgfqpoint{6.548803in}{2.316684in}}%
\pgfpathlineto{\pgfqpoint{6.548971in}{1.747987in}}%
\pgfpathlineto{\pgfqpoint{6.549707in}{2.382823in}}%
\pgfpathlineto{\pgfqpoint{6.549914in}{2.192060in}}%
\pgfpathlineto{\pgfqpoint{6.550988in}{2.363729in}}%
\pgfpathlineto{\pgfqpoint{6.550579in}{1.580409in}}%
\pgfpathlineto{\pgfqpoint{6.551025in}{2.249207in}}%
\pgfpathlineto{\pgfqpoint{6.551735in}{1.689902in}}%
\pgfpathlineto{\pgfqpoint{6.551351in}{2.389208in}}%
\pgfpathlineto{\pgfqpoint{6.552136in}{2.197188in}}%
\pgfpathlineto{\pgfqpoint{6.552438in}{2.355618in}}%
\pgfpathlineto{\pgfqpoint{6.552424in}{1.753914in}}%
\pgfpathlineto{\pgfqpoint{6.553245in}{2.201548in}}%
\pgfpathlineto{\pgfqpoint{6.553439in}{1.647074in}}%
\pgfpathlineto{\pgfqpoint{6.554285in}{2.379554in}}%
\pgfpathlineto{\pgfqpoint{6.554356in}{2.200322in}}%
\pgfpathlineto{\pgfqpoint{6.555459in}{2.360164in}}%
\pgfpathlineto{\pgfqpoint{6.554376in}{1.766647in}}%
\pgfpathlineto{\pgfqpoint{6.555466in}{2.290160in}}%
\pgfpathlineto{\pgfqpoint{6.556383in}{1.771818in}}%
\pgfpathlineto{\pgfqpoint{6.555836in}{2.378528in}}%
\pgfpathlineto{\pgfqpoint{6.556577in}{2.311650in}}%
\pgfpathlineto{\pgfqpoint{6.556781in}{2.386538in}}%
\pgfpathlineto{\pgfqpoint{6.557014in}{1.800902in}}%
\pgfpathlineto{\pgfqpoint{6.557589in}{2.092033in}}%
\pgfpathlineto{\pgfqpoint{6.557651in}{1.654091in}}%
\pgfpathlineto{\pgfqpoint{6.557881in}{2.370866in}}%
\pgfpathlineto{\pgfqpoint{6.558700in}{1.987756in}}%
\pgfpathlineto{\pgfqpoint{6.559218in}{2.373136in}}%
\pgfpathlineto{\pgfqpoint{6.558845in}{1.710897in}}%
\pgfpathlineto{\pgfqpoint{6.559811in}{2.196947in}}%
\pgfpathlineto{\pgfqpoint{6.559891in}{2.378754in}}%
\pgfpathlineto{\pgfqpoint{6.559870in}{1.612005in}}%
\pgfpathlineto{\pgfqpoint{6.560922in}{2.281276in}}%
\pgfpathlineto{\pgfqpoint{6.561447in}{1.618360in}}%
\pgfpathlineto{\pgfqpoint{6.561659in}{2.357936in}}%
\pgfpathlineto{\pgfqpoint{6.562033in}{2.149456in}}%
\pgfpathlineto{\pgfqpoint{6.562873in}{1.635959in}}%
\pgfpathlineto{\pgfqpoint{6.562689in}{2.383948in}}%
\pgfpathlineto{\pgfqpoint{6.562965in}{2.238419in}}%
\pgfusepath{stroke}%
\end{pgfscope}%
\begin{pgfscope}%
\pgfpathrectangle{\pgfqpoint{0.535225in}{0.370679in}}{\pgfqpoint{6.314775in}{3.181174in}}%
\pgfusepath{clip}%
\pgfsetrectcap%
\pgfsetroundjoin%
\pgfsetlinewidth{3.011250pt}%
\definecolor{currentstroke}{rgb}{1.000000,1.000000,0.000000}%
\pgfsetstrokecolor{currentstroke}%
\pgfsetdash{}{0pt}%
\pgfpathmoveto{\pgfqpoint{0.525225in}{2.533535in}}%
\pgfpathlineto{\pgfqpoint{0.822260in}{2.533579in}}%
\pgfpathlineto{\pgfqpoint{1.116406in}{2.485967in}}%
\pgfpathlineto{\pgfqpoint{1.288470in}{2.483571in}}%
\pgfpathlineto{\pgfqpoint{1.410551in}{2.451989in}}%
\pgfpathlineto{\pgfqpoint{1.505245in}{2.318468in}}%
\pgfpathlineto{\pgfqpoint{1.582616in}{2.359366in}}%
\pgfpathlineto{\pgfqpoint{1.648031in}{2.416538in}}%
\pgfpathlineto{\pgfqpoint{1.704697in}{2.439628in}}%
\pgfpathlineto{\pgfqpoint{1.754680in}{2.479804in}}%
\pgfpathlineto{\pgfqpoint{1.799391in}{2.507843in}}%
\pgfpathlineto{\pgfqpoint{1.839837in}{2.397786in}}%
\pgfpathlineto{\pgfqpoint{1.876761in}{2.396336in}}%
\pgfpathlineto{\pgfqpoint{1.910729in}{2.392011in}}%
\pgfpathlineto{\pgfqpoint{1.942177in}{2.403834in}}%
\pgfpathlineto{\pgfqpoint{1.971455in}{2.496446in}}%
\pgfpathlineto{\pgfqpoint{1.998843in}{2.392834in}}%
\pgfpathlineto{\pgfqpoint{2.024570in}{2.415119in}}%
\pgfpathlineto{\pgfqpoint{2.048826in}{2.490044in}}%
\pgfpathlineto{\pgfqpoint{2.071770in}{2.388845in}}%
\pgfpathlineto{\pgfqpoint{2.093537in}{2.145522in}}%
\pgfpathlineto{\pgfqpoint{2.114241in}{2.344108in}}%
\pgfpathlineto{\pgfqpoint{2.133983in}{2.395441in}}%
\pgfpathlineto{\pgfqpoint{2.152847in}{2.434184in}}%
\pgfpathlineto{\pgfqpoint{2.170907in}{2.444428in}}%
\pgfpathlineto{\pgfqpoint{2.188231in}{2.471602in}}%
\pgfpathlineto{\pgfqpoint{2.204874in}{2.469587in}}%
\pgfpathlineto{\pgfqpoint{2.236323in}{2.362886in}}%
\pgfpathlineto{\pgfqpoint{2.251214in}{2.284622in}}%
\pgfpathlineto{\pgfqpoint{2.265601in}{2.444208in}}%
\pgfpathlineto{\pgfqpoint{2.279516in}{2.452456in}}%
\pgfpathlineto{\pgfqpoint{2.292989in}{2.357185in}}%
\pgfpathlineto{\pgfqpoint{2.306047in}{2.460019in}}%
\pgfpathlineto{\pgfqpoint{2.318716in}{2.240097in}}%
\pgfpathlineto{\pgfqpoint{2.331017in}{2.476439in}}%
\pgfpathlineto{\pgfqpoint{2.342971in}{2.343043in}}%
\pgfpathlineto{\pgfqpoint{2.354599in}{2.560136in}}%
\pgfpathlineto{\pgfqpoint{2.376939in}{2.169488in}}%
\pgfpathlineto{\pgfqpoint{2.387683in}{2.499054in}}%
\pgfpathlineto{\pgfqpoint{2.398161in}{2.444093in}}%
\pgfpathlineto{\pgfqpoint{2.408387in}{2.413422in}}%
\pgfpathlineto{\pgfqpoint{2.418373in}{2.491307in}}%
\pgfpathlineto{\pgfqpoint{2.428129in}{2.452505in}}%
\pgfpathlineto{\pgfqpoint{2.437665in}{2.359590in}}%
\pgfpathlineto{\pgfqpoint{2.446992in}{2.349804in}}%
\pgfpathlineto{\pgfqpoint{2.456119in}{2.450824in}}%
\pgfpathlineto{\pgfqpoint{2.465053in}{2.310563in}}%
\pgfpathlineto{\pgfqpoint{2.473803in}{2.397200in}}%
\pgfpathlineto{\pgfqpoint{2.482376in}{2.407862in}}%
\pgfpathlineto{\pgfqpoint{2.490780in}{2.253245in}}%
\pgfpathlineto{\pgfqpoint{2.499020in}{2.448198in}}%
\pgfpathlineto{\pgfqpoint{2.507103in}{2.360935in}}%
\pgfpathlineto{\pgfqpoint{2.515036in}{2.390590in}}%
\pgfpathlineto{\pgfqpoint{2.522822in}{2.379747in}}%
\pgfpathlineto{\pgfqpoint{2.530469in}{2.190267in}}%
\pgfpathlineto{\pgfqpoint{2.537980in}{2.166603in}}%
\pgfpathlineto{\pgfqpoint{2.545360in}{2.232218in}}%
\pgfpathlineto{\pgfqpoint{2.552614in}{2.481511in}}%
\pgfpathlineto{\pgfqpoint{2.559747in}{2.379783in}}%
\pgfpathlineto{\pgfqpoint{2.566761in}{2.178023in}}%
\pgfpathlineto{\pgfqpoint{2.573662in}{2.474875in}}%
\pgfpathlineto{\pgfqpoint{2.580451in}{2.340325in}}%
\pgfpathlineto{\pgfqpoint{2.587135in}{2.481493in}}%
\pgfpathlineto{\pgfqpoint{2.593714in}{2.402984in}}%
\pgfpathlineto{\pgfqpoint{2.600193in}{2.440467in}}%
\pgfpathlineto{\pgfqpoint{2.606574in}{2.439198in}}%
\pgfpathlineto{\pgfqpoint{2.619057in}{2.394949in}}%
\pgfpathlineto{\pgfqpoint{2.625163in}{2.293123in}}%
\pgfpathlineto{\pgfqpoint{2.631182in}{2.356933in}}%
\pgfpathlineto{\pgfqpoint{2.637117in}{2.462163in}}%
\pgfpathlineto{\pgfqpoint{2.642971in}{1.938486in}}%
\pgfpathlineto{\pgfqpoint{2.648744in}{2.015997in}}%
\pgfpathlineto{\pgfqpoint{2.654441in}{2.279805in}}%
\pgfpathlineto{\pgfqpoint{2.660061in}{2.435445in}}%
\pgfpathlineto{\pgfqpoint{2.665609in}{2.254107in}}%
\pgfpathlineto{\pgfqpoint{2.671084in}{2.499952in}}%
\pgfpathlineto{\pgfqpoint{2.676490in}{2.415327in}}%
\pgfpathlineto{\pgfqpoint{2.681828in}{2.484245in}}%
\pgfpathlineto{\pgfqpoint{2.687100in}{2.435783in}}%
\pgfpathlineto{\pgfqpoint{2.692307in}{2.292014in}}%
\pgfpathlineto{\pgfqpoint{2.697451in}{2.417899in}}%
\pgfpathlineto{\pgfqpoint{2.702533in}{2.390666in}}%
\pgfpathlineto{\pgfqpoint{2.707555in}{2.409277in}}%
\pgfpathlineto{\pgfqpoint{2.712518in}{2.446319in}}%
\pgfpathlineto{\pgfqpoint{2.717424in}{2.427475in}}%
\pgfpathlineto{\pgfqpoint{2.722274in}{2.397292in}}%
\pgfpathlineto{\pgfqpoint{2.727069in}{2.430657in}}%
\pgfpathlineto{\pgfqpoint{2.731811in}{2.384404in}}%
\pgfpathlineto{\pgfqpoint{2.736500in}{2.431236in}}%
\pgfpathlineto{\pgfqpoint{2.741138in}{2.247902in}}%
\pgfpathlineto{\pgfqpoint{2.745726in}{2.406445in}}%
\pgfpathlineto{\pgfqpoint{2.750264in}{2.356278in}}%
\pgfpathlineto{\pgfqpoint{2.754755in}{2.339767in}}%
\pgfpathlineto{\pgfqpoint{2.759199in}{2.446488in}}%
\pgfpathlineto{\pgfqpoint{2.763596in}{2.478216in}}%
\pgfpathlineto{\pgfqpoint{2.767949in}{2.342549in}}%
\pgfpathlineto{\pgfqpoint{2.776522in}{2.245633in}}%
\pgfpathlineto{\pgfqpoint{2.780745in}{2.481062in}}%
\pgfpathlineto{\pgfqpoint{2.784926in}{2.364261in}}%
\pgfpathlineto{\pgfqpoint{2.789066in}{2.409338in}}%
\pgfpathlineto{\pgfqpoint{2.793166in}{2.399943in}}%
\pgfpathlineto{\pgfqpoint{2.797227in}{2.321719in}}%
\pgfpathlineto{\pgfqpoint{2.801249in}{2.306174in}}%
\pgfpathlineto{\pgfqpoint{2.805234in}{2.510902in}}%
\pgfpathlineto{\pgfqpoint{2.809181in}{2.416131in}}%
\pgfpathlineto{\pgfqpoint{2.813093in}{2.160028in}}%
\pgfpathlineto{\pgfqpoint{2.816968in}{2.391775in}}%
\pgfpathlineto{\pgfqpoint{2.820809in}{2.514088in}}%
\pgfpathlineto{\pgfqpoint{2.824615in}{2.316404in}}%
\pgfpathlineto{\pgfqpoint{2.828387in}{2.265812in}}%
\pgfpathlineto{\pgfqpoint{2.832126in}{2.306251in}}%
\pgfpathlineto{\pgfqpoint{2.835832in}{2.261504in}}%
\pgfpathlineto{\pgfqpoint{2.839506in}{2.275286in}}%
\pgfpathlineto{\pgfqpoint{2.843149in}{2.394023in}}%
\pgfpathlineto{\pgfqpoint{2.846760in}{2.428397in}}%
\pgfpathlineto{\pgfqpoint{2.850341in}{2.407188in}}%
\pgfpathlineto{\pgfqpoint{2.853893in}{2.222813in}}%
\pgfpathlineto{\pgfqpoint{2.857414in}{2.272225in}}%
\pgfpathlineto{\pgfqpoint{2.860907in}{2.218063in}}%
\pgfpathlineto{\pgfqpoint{2.864371in}{2.507766in}}%
\pgfpathlineto{\pgfqpoint{2.867807in}{2.274426in}}%
\pgfpathlineto{\pgfqpoint{2.871216in}{2.361510in}}%
\pgfpathlineto{\pgfqpoint{2.874597in}{2.379669in}}%
\pgfpathlineto{\pgfqpoint{2.877952in}{2.353146in}}%
\pgfpathlineto{\pgfqpoint{2.881280in}{2.373191in}}%
\pgfpathlineto{\pgfqpoint{2.884583in}{2.451903in}}%
\pgfpathlineto{\pgfqpoint{2.887860in}{2.379327in}}%
\pgfpathlineto{\pgfqpoint{2.891111in}{2.222483in}}%
\pgfpathlineto{\pgfqpoint{2.900720in}{2.442621in}}%
\pgfpathlineto{\pgfqpoint{2.903875in}{2.209123in}}%
\pgfpathlineto{\pgfqpoint{2.907007in}{2.374159in}}%
\pgfpathlineto{\pgfqpoint{2.910116in}{2.465579in}}%
\pgfpathlineto{\pgfqpoint{2.913202in}{2.149550in}}%
\pgfpathlineto{\pgfqpoint{2.916266in}{2.435335in}}%
\pgfpathlineto{\pgfqpoint{2.919308in}{2.439268in}}%
\pgfpathlineto{\pgfqpoint{2.922329in}{2.309159in}}%
\pgfpathlineto{\pgfqpoint{2.928306in}{2.464540in}}%
\pgfpathlineto{\pgfqpoint{2.931263in}{2.060715in}}%
\pgfpathlineto{\pgfqpoint{2.934200in}{2.449037in}}%
\pgfpathlineto{\pgfqpoint{2.937116in}{2.418207in}}%
\pgfpathlineto{\pgfqpoint{2.940013in}{2.134173in}}%
\pgfpathlineto{\pgfqpoint{2.942890in}{2.343785in}}%
\pgfpathlineto{\pgfqpoint{2.948586in}{2.467196in}}%
\pgfpathlineto{\pgfqpoint{2.951406in}{2.351527in}}%
\pgfpathlineto{\pgfqpoint{2.954207in}{2.401864in}}%
\pgfpathlineto{\pgfqpoint{2.956990in}{2.345142in}}%
\pgfpathlineto{\pgfqpoint{2.959754in}{2.516492in}}%
\pgfpathlineto{\pgfqpoint{2.962501in}{2.494960in}}%
\pgfpathlineto{\pgfqpoint{2.965230in}{2.348040in}}%
\pgfpathlineto{\pgfqpoint{2.967942in}{2.405035in}}%
\pgfpathlineto{\pgfqpoint{2.970636in}{2.129994in}}%
\pgfpathlineto{\pgfqpoint{2.973313in}{2.338881in}}%
\pgfpathlineto{\pgfqpoint{2.975974in}{2.376688in}}%
\pgfpathlineto{\pgfqpoint{2.978618in}{2.497072in}}%
\pgfpathlineto{\pgfqpoint{2.981246in}{2.221354in}}%
\pgfpathlineto{\pgfqpoint{2.983857in}{2.429654in}}%
\pgfpathlineto{\pgfqpoint{2.986453in}{2.449285in}}%
\pgfpathlineto{\pgfqpoint{2.989032in}{2.173246in}}%
\pgfpathlineto{\pgfqpoint{2.994145in}{2.361456in}}%
\pgfpathlineto{\pgfqpoint{2.996679in}{2.441328in}}%
\pgfpathlineto{\pgfqpoint{2.999197in}{2.341842in}}%
\pgfpathlineto{\pgfqpoint{3.001701in}{2.392992in}}%
\pgfpathlineto{\pgfqpoint{3.004190in}{2.383544in}}%
\pgfpathlineto{\pgfqpoint{3.006664in}{2.410976in}}%
\pgfpathlineto{\pgfqpoint{3.009124in}{2.418615in}}%
\pgfpathlineto{\pgfqpoint{3.011570in}{2.381144in}}%
\pgfpathlineto{\pgfqpoint{3.014002in}{2.146414in}}%
\pgfpathlineto{\pgfqpoint{3.018824in}{2.405736in}}%
\pgfpathlineto{\pgfqpoint{3.021215in}{2.401880in}}%
\pgfpathlineto{\pgfqpoint{3.023593in}{2.296223in}}%
\pgfpathlineto{\pgfqpoint{3.025957in}{2.393879in}}%
\pgfpathlineto{\pgfqpoint{3.028308in}{2.413402in}}%
\pgfpathlineto{\pgfqpoint{3.030646in}{2.472906in}}%
\pgfpathlineto{\pgfqpoint{3.032971in}{2.302670in}}%
\pgfpathlineto{\pgfqpoint{3.035284in}{2.467261in}}%
\pgfpathlineto{\pgfqpoint{3.037584in}{2.501822in}}%
\pgfpathlineto{\pgfqpoint{3.039872in}{2.442357in}}%
\pgfpathlineto{\pgfqpoint{3.042147in}{2.318511in}}%
\pgfpathlineto{\pgfqpoint{3.046661in}{2.451636in}}%
\pgfpathlineto{\pgfqpoint{3.048901in}{2.426968in}}%
\pgfpathlineto{\pgfqpoint{3.051128in}{2.373754in}}%
\pgfpathlineto{\pgfqpoint{3.053344in}{2.414753in}}%
\pgfpathlineto{\pgfqpoint{3.055549in}{2.497778in}}%
\pgfpathlineto{\pgfqpoint{3.059924in}{2.445745in}}%
\pgfpathlineto{\pgfqpoint{3.062095in}{2.290045in}}%
\pgfpathlineto{\pgfqpoint{3.064254in}{2.334016in}}%
\pgfpathlineto{\pgfqpoint{3.066403in}{2.436110in}}%
\pgfpathlineto{\pgfqpoint{3.068541in}{2.399876in}}%
\pgfpathlineto{\pgfqpoint{3.070668in}{2.405205in}}%
\pgfpathlineto{\pgfqpoint{3.072784in}{2.429999in}}%
\pgfpathlineto{\pgfqpoint{3.074890in}{2.382338in}}%
\pgfpathlineto{\pgfqpoint{3.076986in}{2.232973in}}%
\pgfpathlineto{\pgfqpoint{3.079071in}{2.325829in}}%
\pgfpathlineto{\pgfqpoint{3.081146in}{2.075168in}}%
\pgfpathlineto{\pgfqpoint{3.083211in}{2.391507in}}%
\pgfpathlineto{\pgfqpoint{3.087312in}{2.471439in}}%
\pgfpathlineto{\pgfqpoint{3.089347in}{2.418888in}}%
\pgfpathlineto{\pgfqpoint{3.091373in}{2.333982in}}%
\pgfpathlineto{\pgfqpoint{3.093389in}{2.397251in}}%
\pgfpathlineto{\pgfqpoint{3.095395in}{2.398033in}}%
\pgfpathlineto{\pgfqpoint{3.097392in}{2.534878in}}%
\pgfpathlineto{\pgfqpoint{3.101358in}{2.336234in}}%
\pgfpathlineto{\pgfqpoint{3.103327in}{2.324842in}}%
\pgfpathlineto{\pgfqpoint{3.105287in}{2.256488in}}%
\pgfpathlineto{\pgfqpoint{3.107238in}{2.391583in}}%
\pgfpathlineto{\pgfqpoint{3.109181in}{2.350874in}}%
\pgfpathlineto{\pgfqpoint{3.111114in}{2.467472in}}%
\pgfpathlineto{\pgfqpoint{3.113038in}{2.430897in}}%
\pgfpathlineto{\pgfqpoint{3.114954in}{2.427103in}}%
\pgfpathlineto{\pgfqpoint{3.116862in}{2.473580in}}%
\pgfpathlineto{\pgfqpoint{3.120651in}{2.343086in}}%
\pgfpathlineto{\pgfqpoint{3.124406in}{2.487249in}}%
\pgfpathlineto{\pgfqpoint{3.126271in}{2.316642in}}%
\pgfpathlineto{\pgfqpoint{3.128128in}{2.446162in}}%
\pgfpathlineto{\pgfqpoint{3.129978in}{2.304578in}}%
\pgfpathlineto{\pgfqpoint{3.131819in}{2.473533in}}%
\pgfpathlineto{\pgfqpoint{3.133652in}{2.332859in}}%
\pgfpathlineto{\pgfqpoint{3.135477in}{2.365107in}}%
\pgfpathlineto{\pgfqpoint{3.137294in}{2.538546in}}%
\pgfpathlineto{\pgfqpoint{3.139104in}{2.479538in}}%
\pgfpathlineto{\pgfqpoint{3.140906in}{2.309726in}}%
\pgfpathlineto{\pgfqpoint{3.142700in}{2.341380in}}%
\pgfpathlineto{\pgfqpoint{3.144487in}{2.405445in}}%
\pgfpathlineto{\pgfqpoint{3.146266in}{2.411372in}}%
\pgfpathlineto{\pgfqpoint{3.148038in}{2.497425in}}%
\pgfpathlineto{\pgfqpoint{3.149803in}{2.482296in}}%
\pgfpathlineto{\pgfqpoint{3.151560in}{2.350644in}}%
\pgfpathlineto{\pgfqpoint{3.153310in}{2.306035in}}%
\pgfpathlineto{\pgfqpoint{3.155053in}{2.480253in}}%
\pgfpathlineto{\pgfqpoint{3.158517in}{2.103622in}}%
\pgfpathlineto{\pgfqpoint{3.160238in}{2.483171in}}%
\pgfpathlineto{\pgfqpoint{3.161953in}{2.357842in}}%
\pgfpathlineto{\pgfqpoint{3.163661in}{2.420473in}}%
\pgfpathlineto{\pgfqpoint{3.165362in}{2.431201in}}%
\pgfpathlineto{\pgfqpoint{3.167056in}{2.386554in}}%
\pgfpathlineto{\pgfqpoint{3.168743in}{2.477581in}}%
\pgfpathlineto{\pgfqpoint{3.170424in}{2.411064in}}%
\pgfpathlineto{\pgfqpoint{3.172098in}{2.233689in}}%
\pgfpathlineto{\pgfqpoint{3.173765in}{2.496800in}}%
\pgfpathlineto{\pgfqpoint{3.175426in}{2.372344in}}%
\pgfpathlineto{\pgfqpoint{3.180370in}{2.283132in}}%
\pgfpathlineto{\pgfqpoint{3.182005in}{2.410271in}}%
\pgfpathlineto{\pgfqpoint{3.183634in}{2.146292in}}%
\pgfpathlineto{\pgfqpoint{3.185257in}{2.500870in}}%
\pgfpathlineto{\pgfqpoint{3.186874in}{2.316104in}}%
\pgfpathlineto{\pgfqpoint{3.188484in}{2.438613in}}%
\pgfpathlineto{\pgfqpoint{3.190089in}{2.475842in}}%
\pgfpathlineto{\pgfqpoint{3.193279in}{2.335400in}}%
\pgfpathlineto{\pgfqpoint{3.194866in}{2.519757in}}%
\pgfpathlineto{\pgfqpoint{3.196446in}{2.405073in}}%
\pgfpathlineto{\pgfqpoint{3.198021in}{2.114255in}}%
\pgfpathlineto{\pgfqpoint{3.201153in}{2.363929in}}%
\pgfpathlineto{\pgfqpoint{3.202710in}{2.472872in}}%
\pgfpathlineto{\pgfqpoint{3.204262in}{2.483745in}}%
\pgfpathlineto{\pgfqpoint{3.205808in}{2.317228in}}%
\pgfpathlineto{\pgfqpoint{3.207348in}{2.472849in}}%
\pgfpathlineto{\pgfqpoint{3.210412in}{2.371999in}}%
\pgfpathlineto{\pgfqpoint{3.211936in}{2.468265in}}%
\pgfpathlineto{\pgfqpoint{3.213454in}{2.398534in}}%
\pgfpathlineto{\pgfqpoint{3.214967in}{2.202879in}}%
\pgfpathlineto{\pgfqpoint{3.216474in}{2.425145in}}%
\pgfpathlineto{\pgfqpoint{3.217977in}{2.433365in}}%
\pgfpathlineto{\pgfqpoint{3.219473in}{2.159942in}}%
\pgfpathlineto{\pgfqpoint{3.220965in}{2.426861in}}%
\pgfpathlineto{\pgfqpoint{3.222451in}{2.455556in}}%
\pgfpathlineto{\pgfqpoint{3.223933in}{2.385291in}}%
\pgfpathlineto{\pgfqpoint{3.225409in}{2.453148in}}%
\pgfpathlineto{\pgfqpoint{3.226880in}{2.457088in}}%
\pgfpathlineto{\pgfqpoint{3.229806in}{2.278077in}}%
\pgfpathlineto{\pgfqpoint{3.232713in}{2.480676in}}%
\pgfpathlineto{\pgfqpoint{3.234159in}{2.291349in}}%
\pgfpathlineto{\pgfqpoint{3.235600in}{2.356153in}}%
\pgfpathlineto{\pgfqpoint{3.237036in}{2.480679in}}%
\pgfpathlineto{\pgfqpoint{3.238467in}{2.257047in}}%
\pgfpathlineto{\pgfqpoint{3.241315in}{2.502783in}}%
\pgfpathlineto{\pgfqpoint{3.242732in}{2.362515in}}%
\pgfpathlineto{\pgfqpoint{3.245552in}{2.474926in}}%
\pgfpathlineto{\pgfqpoint{3.246955in}{2.481714in}}%
\pgfpathlineto{\pgfqpoint{3.248353in}{2.452612in}}%
\pgfpathlineto{\pgfqpoint{3.249746in}{2.268519in}}%
\pgfpathlineto{\pgfqpoint{3.251136in}{2.360696in}}%
\pgfpathlineto{\pgfqpoint{3.252520in}{2.321678in}}%
\pgfpathlineto{\pgfqpoint{3.255276in}{2.497557in}}%
\pgfpathlineto{\pgfqpoint{3.256647in}{2.454719in}}%
\pgfpathlineto{\pgfqpoint{3.258014in}{2.368250in}}%
\pgfpathlineto{\pgfqpoint{3.259376in}{2.514473in}}%
\pgfpathlineto{\pgfqpoint{3.260734in}{2.366727in}}%
\pgfpathlineto{\pgfqpoint{3.262087in}{2.527244in}}%
\pgfpathlineto{\pgfqpoint{3.263437in}{2.320165in}}%
\pgfpathlineto{\pgfqpoint{3.267459in}{2.421196in}}%
\pgfpathlineto{\pgfqpoint{3.268792in}{2.398767in}}%
\pgfpathlineto{\pgfqpoint{3.270120in}{2.449829in}}%
\pgfpathlineto{\pgfqpoint{3.271444in}{2.455963in}}%
\pgfpathlineto{\pgfqpoint{3.274080in}{2.304588in}}%
\pgfpathlineto{\pgfqpoint{3.276699in}{2.158726in}}%
\pgfpathlineto{\pgfqpoint{3.278003in}{2.420066in}}%
\pgfpathlineto{\pgfqpoint{3.279303in}{2.337757in}}%
\pgfpathlineto{\pgfqpoint{3.281890in}{2.529886in}}%
\pgfpathlineto{\pgfqpoint{3.284462in}{2.399179in}}%
\pgfpathlineto{\pgfqpoint{3.285742in}{2.352731in}}%
\pgfpathlineto{\pgfqpoint{3.287019in}{2.365667in}}%
\pgfpathlineto{\pgfqpoint{3.288291in}{2.321999in}}%
\pgfpathlineto{\pgfqpoint{3.289560in}{2.463212in}}%
\pgfpathlineto{\pgfqpoint{3.290825in}{2.372563in}}%
\pgfpathlineto{\pgfqpoint{3.292086in}{2.406024in}}%
\pgfpathlineto{\pgfqpoint{3.293343in}{2.370429in}}%
\pgfpathlineto{\pgfqpoint{3.294597in}{2.131134in}}%
\pgfpathlineto{\pgfqpoint{3.295847in}{2.464641in}}%
\pgfpathlineto{\pgfqpoint{3.297093in}{2.412932in}}%
\pgfpathlineto{\pgfqpoint{3.298336in}{2.494397in}}%
\pgfpathlineto{\pgfqpoint{3.299575in}{2.485135in}}%
\pgfpathlineto{\pgfqpoint{3.302042in}{2.195778in}}%
\pgfpathlineto{\pgfqpoint{3.303270in}{2.423861in}}%
\pgfpathlineto{\pgfqpoint{3.304495in}{2.297443in}}%
\pgfpathlineto{\pgfqpoint{3.305716in}{2.428678in}}%
\pgfpathlineto{\pgfqpoint{3.306934in}{2.289996in}}%
\pgfpathlineto{\pgfqpoint{3.308148in}{2.467591in}}%
\pgfpathlineto{\pgfqpoint{3.309359in}{2.425236in}}%
\pgfpathlineto{\pgfqpoint{3.310566in}{2.437381in}}%
\pgfpathlineto{\pgfqpoint{3.311770in}{2.256482in}}%
\pgfpathlineto{\pgfqpoint{3.312970in}{2.383434in}}%
\pgfpathlineto{\pgfqpoint{3.315361in}{2.289819in}}%
\pgfpathlineto{\pgfqpoint{3.316551in}{2.400814in}}%
\pgfpathlineto{\pgfqpoint{3.317738in}{2.359699in}}%
\pgfpathlineto{\pgfqpoint{3.318922in}{2.433782in}}%
\pgfpathlineto{\pgfqpoint{3.320103in}{2.317127in}}%
\pgfpathlineto{\pgfqpoint{3.321280in}{2.439373in}}%
\pgfpathlineto{\pgfqpoint{3.322454in}{2.445886in}}%
\pgfpathlineto{\pgfqpoint{3.324792in}{2.375449in}}%
\pgfpathlineto{\pgfqpoint{3.325956in}{2.459176in}}%
\pgfpathlineto{\pgfqpoint{3.327117in}{2.448796in}}%
\pgfpathlineto{\pgfqpoint{3.330581in}{2.348780in}}%
\pgfpathlineto{\pgfqpoint{3.331730in}{2.383248in}}%
\pgfpathlineto{\pgfqpoint{3.332875in}{2.082116in}}%
\pgfpathlineto{\pgfqpoint{3.335157in}{2.367821in}}%
\pgfpathlineto{\pgfqpoint{3.336293in}{2.305424in}}%
\pgfpathlineto{\pgfqpoint{3.337426in}{2.402399in}}%
\pgfpathlineto{\pgfqpoint{3.338556in}{2.239460in}}%
\pgfpathlineto{\pgfqpoint{3.339683in}{2.467701in}}%
\pgfpathlineto{\pgfqpoint{3.340807in}{2.427898in}}%
\pgfpathlineto{\pgfqpoint{3.341928in}{2.316811in}}%
\pgfpathlineto{\pgfqpoint{3.343047in}{2.338240in}}%
\pgfpathlineto{\pgfqpoint{3.344162in}{2.497931in}}%
\pgfpathlineto{\pgfqpoint{3.345274in}{2.489558in}}%
\pgfpathlineto{\pgfqpoint{3.347490in}{2.321727in}}%
\pgfpathlineto{\pgfqpoint{3.349695in}{2.477953in}}%
\pgfpathlineto{\pgfqpoint{3.351888in}{2.357255in}}%
\pgfpathlineto{\pgfqpoint{3.352980in}{2.378531in}}%
\pgfpathlineto{\pgfqpoint{3.354070in}{2.301242in}}%
\pgfpathlineto{\pgfqpoint{3.356240in}{2.454392in}}%
\pgfpathlineto{\pgfqpoint{3.358400in}{2.354844in}}%
\pgfpathlineto{\pgfqpoint{3.359476in}{2.449072in}}%
\pgfpathlineto{\pgfqpoint{3.360549in}{2.279665in}}%
\pgfpathlineto{\pgfqpoint{3.362686in}{2.469326in}}%
\pgfpathlineto{\pgfqpoint{3.364814in}{2.344697in}}%
\pgfpathlineto{\pgfqpoint{3.366930in}{2.538935in}}%
\pgfpathlineto{\pgfqpoint{3.367984in}{2.421732in}}%
\pgfpathlineto{\pgfqpoint{3.369036in}{2.447912in}}%
\pgfpathlineto{\pgfqpoint{3.370085in}{2.435800in}}%
\pgfpathlineto{\pgfqpoint{3.371132in}{2.539876in}}%
\pgfpathlineto{\pgfqpoint{3.372176in}{2.260444in}}%
\pgfpathlineto{\pgfqpoint{3.374256in}{2.362878in}}%
\pgfpathlineto{\pgfqpoint{3.375292in}{2.392832in}}%
\pgfpathlineto{\pgfqpoint{3.376326in}{2.502206in}}%
\pgfpathlineto{\pgfqpoint{3.377357in}{2.232143in}}%
\pgfpathlineto{\pgfqpoint{3.378386in}{2.365277in}}%
\pgfpathlineto{\pgfqpoint{3.380436in}{2.290928in}}%
\pgfpathlineto{\pgfqpoint{3.381457in}{2.551496in}}%
\pgfpathlineto{\pgfqpoint{3.382476in}{2.395246in}}%
\pgfpathlineto{\pgfqpoint{3.383493in}{2.458753in}}%
\pgfpathlineto{\pgfqpoint{3.385518in}{2.319824in}}%
\pgfpathlineto{\pgfqpoint{3.386527in}{2.447118in}}%
\pgfpathlineto{\pgfqpoint{3.389541in}{2.389149in}}%
\pgfpathlineto{\pgfqpoint{3.390540in}{2.429200in}}%
\pgfpathlineto{\pgfqpoint{3.392533in}{2.325993in}}%
\pgfpathlineto{\pgfqpoint{3.393525in}{2.518849in}}%
\pgfpathlineto{\pgfqpoint{3.394516in}{2.443902in}}%
\pgfpathlineto{\pgfqpoint{3.395504in}{2.226490in}}%
\pgfpathlineto{\pgfqpoint{3.398454in}{2.480850in}}%
\pgfpathlineto{\pgfqpoint{3.399433in}{2.371640in}}%
\pgfpathlineto{\pgfqpoint{3.400410in}{2.425261in}}%
\pgfpathlineto{\pgfqpoint{3.401384in}{2.389031in}}%
\pgfpathlineto{\pgfqpoint{3.402356in}{2.409783in}}%
\pgfpathlineto{\pgfqpoint{3.403326in}{2.259185in}}%
\pgfpathlineto{\pgfqpoint{3.405260in}{2.400737in}}%
\pgfpathlineto{\pgfqpoint{3.406223in}{2.243486in}}%
\pgfpathlineto{\pgfqpoint{3.407184in}{2.247956in}}%
\pgfpathlineto{\pgfqpoint{3.409100in}{2.397542in}}%
\pgfpathlineto{\pgfqpoint{3.411007in}{2.487960in}}%
\pgfpathlineto{\pgfqpoint{3.411958in}{2.491026in}}%
\pgfpathlineto{\pgfqpoint{3.413852in}{2.395020in}}%
\pgfpathlineto{\pgfqpoint{3.414796in}{2.378623in}}%
\pgfpathlineto{\pgfqpoint{3.415738in}{2.395677in}}%
\pgfpathlineto{\pgfqpoint{3.416678in}{2.290589in}}%
\pgfpathlineto{\pgfqpoint{3.417616in}{2.449124in}}%
\pgfpathlineto{\pgfqpoint{3.418552in}{2.366754in}}%
\pgfpathlineto{\pgfqpoint{3.419485in}{2.375678in}}%
\pgfpathlineto{\pgfqpoint{3.420417in}{1.948290in}}%
\pgfpathlineto{\pgfqpoint{3.421347in}{2.387862in}}%
\pgfpathlineto{\pgfqpoint{3.422274in}{2.380074in}}%
\pgfpathlineto{\pgfqpoint{3.423200in}{2.491747in}}%
\pgfpathlineto{\pgfqpoint{3.424123in}{2.369292in}}%
\pgfpathlineto{\pgfqpoint{3.425045in}{2.489927in}}%
\pgfpathlineto{\pgfqpoint{3.425964in}{2.236305in}}%
\pgfpathlineto{\pgfqpoint{3.426882in}{2.351300in}}%
\pgfpathlineto{\pgfqpoint{3.427797in}{2.280501in}}%
\pgfpathlineto{\pgfqpoint{3.428711in}{2.533309in}}%
\pgfpathlineto{\pgfqpoint{3.430532in}{2.299484in}}%
\pgfpathlineto{\pgfqpoint{3.431440in}{2.461269in}}%
\pgfpathlineto{\pgfqpoint{3.432346in}{2.348106in}}%
\pgfpathlineto{\pgfqpoint{3.433250in}{2.378129in}}%
\pgfpathlineto{\pgfqpoint{3.434152in}{2.161465in}}%
\pgfpathlineto{\pgfqpoint{3.435950in}{2.445567in}}%
\pgfpathlineto{\pgfqpoint{3.436846in}{2.430470in}}%
\pgfpathlineto{\pgfqpoint{3.437740in}{2.306534in}}%
\pgfpathlineto{\pgfqpoint{3.438633in}{2.409664in}}%
\pgfpathlineto{\pgfqpoint{3.439523in}{2.329690in}}%
\pgfpathlineto{\pgfqpoint{3.440412in}{2.537762in}}%
\pgfpathlineto{\pgfqpoint{3.442184in}{2.307535in}}%
\pgfpathlineto{\pgfqpoint{3.443067in}{2.446546in}}%
\pgfpathlineto{\pgfqpoint{3.443949in}{2.435955in}}%
\pgfpathlineto{\pgfqpoint{3.444828in}{2.351343in}}%
\pgfpathlineto{\pgfqpoint{3.445706in}{2.459038in}}%
\pgfpathlineto{\pgfqpoint{3.446582in}{2.199454in}}%
\pgfpathlineto{\pgfqpoint{3.447456in}{2.465213in}}%
\pgfpathlineto{\pgfqpoint{3.448328in}{2.329502in}}%
\pgfpathlineto{\pgfqpoint{3.450934in}{2.440635in}}%
\pgfpathlineto{\pgfqpoint{3.451799in}{2.454623in}}%
\pgfpathlineto{\pgfqpoint{3.452663in}{2.432539in}}%
\pgfpathlineto{\pgfqpoint{3.453524in}{2.502542in}}%
\pgfpathlineto{\pgfqpoint{3.454384in}{2.424826in}}%
\pgfpathlineto{\pgfqpoint{3.455242in}{2.444472in}}%
\pgfpathlineto{\pgfqpoint{3.456099in}{2.445094in}}%
\pgfpathlineto{\pgfqpoint{3.457807in}{2.404596in}}%
\pgfpathlineto{\pgfqpoint{3.458658in}{2.341426in}}%
\pgfpathlineto{\pgfqpoint{3.459507in}{2.113870in}}%
\pgfpathlineto{\pgfqpoint{3.461201in}{2.414643in}}%
\pgfpathlineto{\pgfqpoint{3.462046in}{2.425474in}}%
\pgfpathlineto{\pgfqpoint{3.462889in}{2.395303in}}%
\pgfpathlineto{\pgfqpoint{3.463730in}{2.396575in}}%
\pgfpathlineto{\pgfqpoint{3.464569in}{2.440720in}}%
\pgfpathlineto{\pgfqpoint{3.465407in}{2.255933in}}%
\pgfpathlineto{\pgfqpoint{3.467078in}{2.432008in}}%
\pgfpathlineto{\pgfqpoint{3.467911in}{2.434457in}}%
\pgfpathlineto{\pgfqpoint{3.468742in}{2.487802in}}%
\pgfpathlineto{\pgfqpoint{3.469572in}{2.469854in}}%
\pgfpathlineto{\pgfqpoint{3.470400in}{2.146764in}}%
\pgfpathlineto{\pgfqpoint{3.471226in}{2.408388in}}%
\pgfpathlineto{\pgfqpoint{3.472051in}{2.346007in}}%
\pgfpathlineto{\pgfqpoint{3.472874in}{2.489231in}}%
\pgfpathlineto{\pgfqpoint{3.473696in}{2.267454in}}%
\pgfpathlineto{\pgfqpoint{3.475334in}{2.426093in}}%
\pgfpathlineto{\pgfqpoint{3.476966in}{2.290460in}}%
\pgfpathlineto{\pgfqpoint{3.478592in}{2.491533in}}%
\pgfpathlineto{\pgfqpoint{3.480212in}{2.282983in}}%
\pgfpathlineto{\pgfqpoint{3.481826in}{2.372673in}}%
\pgfpathlineto{\pgfqpoint{3.482630in}{2.382225in}}%
\pgfpathlineto{\pgfqpoint{3.483433in}{2.495901in}}%
\pgfpathlineto{\pgfqpoint{3.484235in}{2.480922in}}%
\pgfpathlineto{\pgfqpoint{3.485034in}{2.359995in}}%
\pgfpathlineto{\pgfqpoint{3.485833in}{2.370538in}}%
\pgfpathlineto{\pgfqpoint{3.487425in}{2.439146in}}%
\pgfpathlineto{\pgfqpoint{3.488219in}{2.188800in}}%
\pgfpathlineto{\pgfqpoint{3.489803in}{2.529443in}}%
\pgfpathlineto{\pgfqpoint{3.490592in}{2.419563in}}%
\pgfpathlineto{\pgfqpoint{3.491380in}{2.484827in}}%
\pgfpathlineto{\pgfqpoint{3.492167in}{2.438078in}}%
\pgfpathlineto{\pgfqpoint{3.492952in}{2.307481in}}%
\pgfpathlineto{\pgfqpoint{3.493736in}{2.422025in}}%
\pgfpathlineto{\pgfqpoint{3.494518in}{2.395073in}}%
\pgfpathlineto{\pgfqpoint{3.495299in}{2.487241in}}%
\pgfpathlineto{\pgfqpoint{3.496078in}{2.443374in}}%
\pgfpathlineto{\pgfqpoint{3.496856in}{2.241037in}}%
\pgfpathlineto{\pgfqpoint{3.498407in}{2.453060in}}%
\pgfpathlineto{\pgfqpoint{3.499181in}{2.282414in}}%
\pgfpathlineto{\pgfqpoint{3.499953in}{2.347765in}}%
\pgfpathlineto{\pgfqpoint{3.500724in}{2.407235in}}%
\pgfpathlineto{\pgfqpoint{3.502262in}{2.168885in}}%
\pgfpathlineto{\pgfqpoint{3.503794in}{2.408917in}}%
\pgfpathlineto{\pgfqpoint{3.504558in}{2.383596in}}%
\pgfpathlineto{\pgfqpoint{3.505320in}{2.389719in}}%
\pgfpathlineto{\pgfqpoint{3.506082in}{2.380176in}}%
\pgfpathlineto{\pgfqpoint{3.507600in}{2.431049in}}%
\pgfpathlineto{\pgfqpoint{3.509113in}{2.354718in}}%
\pgfpathlineto{\pgfqpoint{3.509867in}{2.357602in}}%
\pgfpathlineto{\pgfqpoint{3.510620in}{2.370815in}}%
\pgfpathlineto{\pgfqpoint{3.512122in}{2.466962in}}%
\pgfpathlineto{\pgfqpoint{3.512871in}{2.478881in}}%
\pgfpathlineto{\pgfqpoint{3.514366in}{2.382649in}}%
\pgfpathlineto{\pgfqpoint{3.516597in}{2.553617in}}%
\pgfpathlineto{\pgfqpoint{3.518078in}{2.444323in}}%
\pgfpathlineto{\pgfqpoint{3.518817in}{2.306943in}}%
\pgfpathlineto{\pgfqpoint{3.520291in}{2.462672in}}%
\pgfpathlineto{\pgfqpoint{3.521025in}{2.355012in}}%
\pgfpathlineto{\pgfqpoint{3.522491in}{2.496597in}}%
\pgfpathlineto{\pgfqpoint{3.523222in}{2.297230in}}%
\pgfpathlineto{\pgfqpoint{3.523952in}{2.347803in}}%
\pgfpathlineto{\pgfqpoint{3.524681in}{2.410697in}}%
\pgfpathlineto{\pgfqpoint{3.525408in}{2.397716in}}%
\pgfpathlineto{\pgfqpoint{3.526859in}{2.344473in}}%
\pgfpathlineto{\pgfqpoint{3.527582in}{2.376501in}}%
\pgfpathlineto{\pgfqpoint{3.528305in}{2.304156in}}%
\pgfpathlineto{\pgfqpoint{3.529026in}{2.453714in}}%
\pgfpathlineto{\pgfqpoint{3.529746in}{2.366716in}}%
\pgfpathlineto{\pgfqpoint{3.530464in}{2.465533in}}%
\pgfpathlineto{\pgfqpoint{3.531182in}{2.344905in}}%
\pgfpathlineto{\pgfqpoint{3.531898in}{2.516126in}}%
\pgfpathlineto{\pgfqpoint{3.532613in}{2.406679in}}%
\pgfpathlineto{\pgfqpoint{3.533327in}{2.480893in}}%
\pgfpathlineto{\pgfqpoint{3.534751in}{2.316234in}}%
\pgfpathlineto{\pgfqpoint{3.535461in}{2.551152in}}%
\pgfpathlineto{\pgfqpoint{3.536170in}{2.475557in}}%
\pgfpathlineto{\pgfqpoint{3.536878in}{2.502393in}}%
\pgfpathlineto{\pgfqpoint{3.538290in}{2.289304in}}%
\pgfpathlineto{\pgfqpoint{3.538994in}{2.381899in}}%
\pgfpathlineto{\pgfqpoint{3.539698in}{2.375543in}}%
\pgfpathlineto{\pgfqpoint{3.541100in}{2.233753in}}%
\pgfpathlineto{\pgfqpoint{3.541800in}{2.299785in}}%
\pgfpathlineto{\pgfqpoint{3.542499in}{2.512460in}}%
\pgfpathlineto{\pgfqpoint{3.543196in}{2.461658in}}%
\pgfpathlineto{\pgfqpoint{3.543892in}{2.257069in}}%
\pgfpathlineto{\pgfqpoint{3.545974in}{2.434900in}}%
\pgfpathlineto{\pgfqpoint{3.546666in}{2.386184in}}%
\pgfpathlineto{\pgfqpoint{3.547356in}{2.467140in}}%
\pgfpathlineto{\pgfqpoint{3.548046in}{2.454235in}}%
\pgfpathlineto{\pgfqpoint{3.549421in}{2.342426in}}%
\pgfpathlineto{\pgfqpoint{3.551476in}{2.417873in}}%
\pgfpathlineto{\pgfqpoint{3.552159in}{2.315426in}}%
\pgfpathlineto{\pgfqpoint{3.552841in}{2.432555in}}%
\pgfpathlineto{\pgfqpoint{3.553522in}{2.406625in}}%
\pgfpathlineto{\pgfqpoint{3.554201in}{2.396830in}}%
\pgfpathlineto{\pgfqpoint{3.554880in}{2.283727in}}%
\pgfpathlineto{\pgfqpoint{3.555557in}{2.285617in}}%
\pgfpathlineto{\pgfqpoint{3.556233in}{2.365614in}}%
\pgfpathlineto{\pgfqpoint{3.556908in}{2.277278in}}%
\pgfpathlineto{\pgfqpoint{3.557583in}{2.484735in}}%
\pgfpathlineto{\pgfqpoint{3.558256in}{2.364402in}}%
\pgfpathlineto{\pgfqpoint{3.560268in}{2.476739in}}%
\pgfpathlineto{\pgfqpoint{3.560937in}{2.395985in}}%
\pgfpathlineto{\pgfqpoint{3.561605in}{2.475770in}}%
\pgfpathlineto{\pgfqpoint{3.562272in}{2.276373in}}%
\pgfpathlineto{\pgfqpoint{3.562937in}{2.397237in}}%
\pgfpathlineto{\pgfqpoint{3.563602in}{2.376681in}}%
\pgfpathlineto{\pgfqpoint{3.564266in}{2.478625in}}%
\pgfpathlineto{\pgfqpoint{3.564928in}{2.460579in}}%
\pgfpathlineto{\pgfqpoint{3.565590in}{2.440400in}}%
\pgfpathlineto{\pgfqpoint{3.566250in}{2.230281in}}%
\pgfpathlineto{\pgfqpoint{3.566910in}{2.368302in}}%
\pgfpathlineto{\pgfqpoint{3.567568in}{2.232265in}}%
\pgfpathlineto{\pgfqpoint{3.568882in}{2.341079in}}%
\pgfpathlineto{\pgfqpoint{3.569537in}{2.302826in}}%
\pgfpathlineto{\pgfqpoint{3.570192in}{2.378074in}}%
\pgfpathlineto{\pgfqpoint{3.570845in}{2.298717in}}%
\pgfpathlineto{\pgfqpoint{3.571497in}{2.399105in}}%
\pgfpathlineto{\pgfqpoint{3.572149in}{2.181053in}}%
\pgfpathlineto{\pgfqpoint{3.572799in}{2.296653in}}%
\pgfpathlineto{\pgfqpoint{3.573448in}{2.465412in}}%
\pgfpathlineto{\pgfqpoint{3.574097in}{2.309434in}}%
\pgfpathlineto{\pgfqpoint{3.574744in}{2.455775in}}%
\pgfpathlineto{\pgfqpoint{3.575391in}{2.455070in}}%
\pgfpathlineto{\pgfqpoint{3.576036in}{2.482424in}}%
\pgfpathlineto{\pgfqpoint{3.576680in}{2.179698in}}%
\pgfpathlineto{\pgfqpoint{3.577324in}{2.341761in}}%
\pgfpathlineto{\pgfqpoint{3.578608in}{2.426565in}}%
\pgfpathlineto{\pgfqpoint{3.579248in}{2.353606in}}%
\pgfpathlineto{\pgfqpoint{3.579888in}{2.433238in}}%
\pgfpathlineto{\pgfqpoint{3.580527in}{2.411747in}}%
\pgfpathlineto{\pgfqpoint{3.581164in}{2.387571in}}%
\pgfpathlineto{\pgfqpoint{3.581801in}{2.440218in}}%
\pgfpathlineto{\pgfqpoint{3.582437in}{2.420611in}}%
\pgfpathlineto{\pgfqpoint{3.583072in}{2.276882in}}%
\pgfpathlineto{\pgfqpoint{3.583705in}{2.471690in}}%
\pgfpathlineto{\pgfqpoint{3.584338in}{2.458836in}}%
\pgfpathlineto{\pgfqpoint{3.584970in}{2.487555in}}%
\pgfpathlineto{\pgfqpoint{3.586861in}{2.204973in}}%
\pgfpathlineto{\pgfqpoint{3.587489in}{2.430422in}}%
\pgfpathlineto{\pgfqpoint{3.588116in}{2.335882in}}%
\pgfpathlineto{\pgfqpoint{3.588742in}{2.349932in}}%
\pgfpathlineto{\pgfqpoint{3.589992in}{2.072261in}}%
\pgfpathlineto{\pgfqpoint{3.591239in}{2.392161in}}%
\pgfpathlineto{\pgfqpoint{3.591860in}{2.419257in}}%
\pgfpathlineto{\pgfqpoint{3.592481in}{2.316073in}}%
\pgfpathlineto{\pgfqpoint{3.593101in}{2.443659in}}%
\pgfpathlineto{\pgfqpoint{3.593720in}{2.371636in}}%
\pgfpathlineto{\pgfqpoint{3.594338in}{2.390670in}}%
\pgfpathlineto{\pgfqpoint{3.595572in}{2.217566in}}%
\pgfpathlineto{\pgfqpoint{3.596802in}{2.520686in}}%
\pgfpathlineto{\pgfqpoint{3.597416in}{2.407082in}}%
\pgfpathlineto{\pgfqpoint{3.598029in}{2.253709in}}%
\pgfpathlineto{\pgfqpoint{3.598641in}{2.477220in}}%
\pgfpathlineto{\pgfqpoint{3.599252in}{2.347905in}}%
\pgfpathlineto{\pgfqpoint{3.599862in}{2.424787in}}%
\pgfpathlineto{\pgfqpoint{3.600471in}{2.329157in}}%
\pgfpathlineto{\pgfqpoint{3.601079in}{2.407609in}}%
\pgfpathlineto{\pgfqpoint{3.602294in}{2.214699in}}%
\pgfpathlineto{\pgfqpoint{3.602899in}{2.465486in}}%
\pgfpathlineto{\pgfqpoint{3.603504in}{2.399306in}}%
\pgfpathlineto{\pgfqpoint{3.604108in}{2.200372in}}%
\pgfpathlineto{\pgfqpoint{3.604712in}{2.352065in}}%
\pgfpathlineto{\pgfqpoint{3.605314in}{2.339133in}}%
\pgfpathlineto{\pgfqpoint{3.605915in}{2.280564in}}%
\pgfpathlineto{\pgfqpoint{3.606516in}{2.443465in}}%
\pgfpathlineto{\pgfqpoint{3.607116in}{2.375068in}}%
\pgfpathlineto{\pgfqpoint{3.607715in}{2.369070in}}%
\pgfpathlineto{\pgfqpoint{3.608313in}{2.409509in}}%
\pgfpathlineto{\pgfqpoint{3.608910in}{2.189400in}}%
\pgfpathlineto{\pgfqpoint{3.609507in}{2.378665in}}%
\pgfpathlineto{\pgfqpoint{3.610102in}{2.487944in}}%
\pgfpathlineto{\pgfqpoint{3.610697in}{2.465434in}}%
\pgfpathlineto{\pgfqpoint{3.611291in}{2.290108in}}%
\pgfpathlineto{\pgfqpoint{3.611884in}{2.303292in}}%
\pgfpathlineto{\pgfqpoint{3.612476in}{2.374230in}}%
\pgfpathlineto{\pgfqpoint{3.613068in}{2.238667in}}%
\pgfpathlineto{\pgfqpoint{3.613658in}{2.320148in}}%
\pgfpathlineto{\pgfqpoint{3.614837in}{2.386412in}}%
\pgfpathlineto{\pgfqpoint{3.615425in}{2.335300in}}%
\pgfpathlineto{\pgfqpoint{3.616013in}{2.433168in}}%
\pgfpathlineto{\pgfqpoint{3.616599in}{2.379629in}}%
\pgfpathlineto{\pgfqpoint{3.617185in}{2.373806in}}%
\pgfpathlineto{\pgfqpoint{3.617770in}{2.398195in}}%
\pgfpathlineto{\pgfqpoint{3.618937in}{2.531063in}}%
\pgfpathlineto{\pgfqpoint{3.620683in}{2.160112in}}%
\pgfpathlineto{\pgfqpoint{3.621842in}{2.524644in}}%
\pgfpathlineto{\pgfqpoint{3.622421in}{2.493514in}}%
\pgfpathlineto{\pgfqpoint{3.622998in}{2.535667in}}%
\pgfpathlineto{\pgfqpoint{3.623575in}{2.238847in}}%
\pgfpathlineto{\pgfqpoint{3.624151in}{2.344741in}}%
\pgfpathlineto{\pgfqpoint{3.624727in}{2.495318in}}%
\pgfpathlineto{\pgfqpoint{3.625302in}{2.343687in}}%
\pgfpathlineto{\pgfqpoint{3.625875in}{2.430279in}}%
\pgfpathlineto{\pgfqpoint{3.626448in}{2.438478in}}%
\pgfpathlineto{\pgfqpoint{3.627021in}{2.471378in}}%
\pgfpathlineto{\pgfqpoint{3.627592in}{2.316201in}}%
\pgfpathlineto{\pgfqpoint{3.628163in}{2.397314in}}%
\pgfpathlineto{\pgfqpoint{3.628733in}{2.484982in}}%
\pgfpathlineto{\pgfqpoint{3.629871in}{2.342896in}}%
\pgfpathlineto{\pgfqpoint{3.630438in}{2.443174in}}%
\pgfpathlineto{\pgfqpoint{3.631005in}{2.424407in}}%
\pgfpathlineto{\pgfqpoint{3.632137in}{2.343802in}}%
\pgfpathlineto{\pgfqpoint{3.632702in}{1.977047in}}%
\pgfpathlineto{\pgfqpoint{3.633266in}{2.486512in}}%
\pgfpathlineto{\pgfqpoint{3.633829in}{2.161976in}}%
\pgfpathlineto{\pgfqpoint{3.634391in}{2.328389in}}%
\pgfpathlineto{\pgfqpoint{3.634953in}{2.098901in}}%
\pgfpathlineto{\pgfqpoint{3.635514in}{2.495406in}}%
\pgfpathlineto{\pgfqpoint{3.636074in}{2.419367in}}%
\pgfpathlineto{\pgfqpoint{3.637192in}{2.239271in}}%
\pgfpathlineto{\pgfqpoint{3.639420in}{2.462092in}}%
\pgfpathlineto{\pgfqpoint{3.640529in}{2.286377in}}%
\pgfpathlineto{\pgfqpoint{3.641636in}{2.507646in}}%
\pgfpathlineto{\pgfqpoint{3.642188in}{2.436078in}}%
\pgfpathlineto{\pgfqpoint{3.642740in}{2.315846in}}%
\pgfpathlineto{\pgfqpoint{3.643840in}{2.476992in}}%
\pgfpathlineto{\pgfqpoint{3.644390in}{2.290825in}}%
\pgfpathlineto{\pgfqpoint{3.644938in}{2.394838in}}%
\pgfpathlineto{\pgfqpoint{3.645486in}{2.389010in}}%
\pgfpathlineto{\pgfqpoint{3.646034in}{2.243222in}}%
\pgfpathlineto{\pgfqpoint{3.646580in}{2.357932in}}%
\pgfpathlineto{\pgfqpoint{3.647126in}{2.465000in}}%
\pgfpathlineto{\pgfqpoint{3.647671in}{2.170348in}}%
\pgfpathlineto{\pgfqpoint{3.648215in}{2.376325in}}%
\pgfpathlineto{\pgfqpoint{3.648759in}{2.375203in}}%
\pgfpathlineto{\pgfqpoint{3.650386in}{2.458563in}}%
\pgfpathlineto{\pgfqpoint{3.650927in}{2.267078in}}%
\pgfpathlineto{\pgfqpoint{3.651467in}{2.372942in}}%
\pgfpathlineto{\pgfqpoint{3.652546in}{2.406264in}}%
\pgfpathlineto{\pgfqpoint{3.653084in}{2.471253in}}%
\pgfpathlineto{\pgfqpoint{3.653621in}{2.439586in}}%
\pgfpathlineto{\pgfqpoint{3.654158in}{2.465018in}}%
\pgfpathlineto{\pgfqpoint{3.654694in}{2.456091in}}%
\pgfpathlineto{\pgfqpoint{3.655230in}{2.440374in}}%
\pgfpathlineto{\pgfqpoint{3.655765in}{2.475571in}}%
\pgfpathlineto{\pgfqpoint{3.657365in}{2.384782in}}%
\pgfpathlineto{\pgfqpoint{3.657897in}{2.401983in}}%
\pgfpathlineto{\pgfqpoint{3.658429in}{2.293762in}}%
\pgfpathlineto{\pgfqpoint{3.658959in}{2.424383in}}%
\pgfpathlineto{\pgfqpoint{3.659489in}{2.407259in}}%
\pgfpathlineto{\pgfqpoint{3.660019in}{2.397724in}}%
\pgfpathlineto{\pgfqpoint{3.660548in}{2.404030in}}%
\pgfpathlineto{\pgfqpoint{3.661076in}{2.357929in}}%
\pgfpathlineto{\pgfqpoint{3.661603in}{2.079843in}}%
\pgfpathlineto{\pgfqpoint{3.662130in}{2.490648in}}%
\pgfpathlineto{\pgfqpoint{3.662656in}{2.439951in}}%
\pgfpathlineto{\pgfqpoint{3.664755in}{2.357507in}}%
\pgfpathlineto{\pgfqpoint{3.665278in}{2.195292in}}%
\pgfpathlineto{\pgfqpoint{3.665800in}{2.486798in}}%
\pgfpathlineto{\pgfqpoint{3.666321in}{2.385726in}}%
\pgfpathlineto{\pgfqpoint{3.667883in}{2.272348in}}%
\pgfpathlineto{\pgfqpoint{3.668402in}{2.275252in}}%
\pgfpathlineto{\pgfqpoint{3.668920in}{2.507027in}}%
\pgfpathlineto{\pgfqpoint{3.669438in}{2.392485in}}%
\pgfpathlineto{\pgfqpoint{3.670472in}{2.271454in}}%
\pgfpathlineto{\pgfqpoint{3.671503in}{2.505975in}}%
\pgfpathlineto{\pgfqpoint{3.673045in}{2.380518in}}%
\pgfpathlineto{\pgfqpoint{3.673558in}{2.404168in}}%
\pgfpathlineto{\pgfqpoint{3.674070in}{2.200181in}}%
\pgfpathlineto{\pgfqpoint{3.674582in}{2.417759in}}%
\pgfpathlineto{\pgfqpoint{3.675093in}{2.468939in}}%
\pgfpathlineto{\pgfqpoint{3.676622in}{2.322461in}}%
\pgfpathlineto{\pgfqpoint{3.677131in}{2.464375in}}%
\pgfpathlineto{\pgfqpoint{3.677638in}{2.435999in}}%
\pgfpathlineto{\pgfqpoint{3.678146in}{2.405786in}}%
\pgfpathlineto{\pgfqpoint{3.679159in}{2.484375in}}%
\pgfpathlineto{\pgfqpoint{3.679664in}{2.345704in}}%
\pgfpathlineto{\pgfqpoint{3.680169in}{2.359845in}}%
\pgfpathlineto{\pgfqpoint{3.681680in}{2.488936in}}%
\pgfpathlineto{\pgfqpoint{3.682684in}{2.372373in}}%
\pgfpathlineto{\pgfqpoint{3.683186in}{2.433108in}}%
\pgfpathlineto{\pgfqpoint{3.683686in}{2.390837in}}%
\pgfpathlineto{\pgfqpoint{3.684187in}{2.385350in}}%
\pgfpathlineto{\pgfqpoint{3.684686in}{2.334911in}}%
\pgfpathlineto{\pgfqpoint{3.685683in}{2.447481in}}%
\pgfpathlineto{\pgfqpoint{3.686181in}{2.373424in}}%
\pgfpathlineto{\pgfqpoint{3.686678in}{2.115232in}}%
\pgfpathlineto{\pgfqpoint{3.687671in}{2.486749in}}%
\pgfpathlineto{\pgfqpoint{3.688661in}{1.974804in}}%
\pgfpathlineto{\pgfqpoint{3.689156in}{2.437606in}}%
\pgfpathlineto{\pgfqpoint{3.690143in}{2.411848in}}%
\pgfpathlineto{\pgfqpoint{3.690635in}{2.386933in}}%
\pgfpathlineto{\pgfqpoint{3.691127in}{2.452431in}}%
\pgfpathlineto{\pgfqpoint{3.691619in}{2.132202in}}%
\pgfpathlineto{\pgfqpoint{3.692110in}{2.317524in}}%
\pgfpathlineto{\pgfqpoint{3.692600in}{2.338510in}}%
\pgfpathlineto{\pgfqpoint{3.693579in}{2.330210in}}%
\pgfpathlineto{\pgfqpoint{3.694067in}{2.420165in}}%
\pgfpathlineto{\pgfqpoint{3.694556in}{2.164362in}}%
\pgfpathlineto{\pgfqpoint{3.695043in}{2.457584in}}%
\pgfpathlineto{\pgfqpoint{3.695530in}{2.479295in}}%
\pgfpathlineto{\pgfqpoint{3.696016in}{2.314062in}}%
\pgfpathlineto{\pgfqpoint{3.696502in}{2.450953in}}%
\pgfpathlineto{\pgfqpoint{3.696987in}{2.393058in}}%
\pgfpathlineto{\pgfqpoint{3.697472in}{2.442526in}}%
\pgfpathlineto{\pgfqpoint{3.697956in}{2.438166in}}%
\pgfpathlineto{\pgfqpoint{3.698440in}{2.256348in}}%
\pgfpathlineto{\pgfqpoint{3.699405in}{2.304957in}}%
\pgfpathlineto{\pgfqpoint{3.699887in}{2.471898in}}%
\pgfpathlineto{\pgfqpoint{3.700369in}{2.412484in}}%
\pgfpathlineto{\pgfqpoint{3.701330in}{2.360648in}}%
\pgfpathlineto{\pgfqpoint{3.701810in}{2.217545in}}%
\pgfpathlineto{\pgfqpoint{3.702289in}{2.365327in}}%
\pgfpathlineto{\pgfqpoint{3.703723in}{2.458883in}}%
\pgfpathlineto{\pgfqpoint{3.704201in}{2.445767in}}%
\pgfpathlineto{\pgfqpoint{3.705153in}{2.368335in}}%
\pgfpathlineto{\pgfqpoint{3.705629in}{2.464070in}}%
\pgfpathlineto{\pgfqpoint{3.706103in}{2.420201in}}%
\pgfpathlineto{\pgfqpoint{3.707052in}{2.334079in}}%
\pgfpathlineto{\pgfqpoint{3.707525in}{2.371329in}}%
\pgfpathlineto{\pgfqpoint{3.707998in}{2.356412in}}%
\pgfpathlineto{\pgfqpoint{3.708470in}{2.535622in}}%
\pgfpathlineto{\pgfqpoint{3.708942in}{2.257463in}}%
\pgfpathlineto{\pgfqpoint{3.709413in}{2.409921in}}%
\pgfpathlineto{\pgfqpoint{3.710824in}{2.525850in}}%
\pgfpathlineto{\pgfqpoint{3.711293in}{2.336295in}}%
\pgfpathlineto{\pgfqpoint{3.711762in}{2.393393in}}%
\pgfpathlineto{\pgfqpoint{3.712698in}{2.486303in}}%
\pgfpathlineto{\pgfqpoint{3.713165in}{2.264731in}}%
\pgfpathlineto{\pgfqpoint{3.713631in}{2.432936in}}%
\pgfpathlineto{\pgfqpoint{3.714097in}{2.362864in}}%
\pgfpathlineto{\pgfqpoint{3.714563in}{2.428253in}}%
\pgfpathlineto{\pgfqpoint{3.715028in}{2.413567in}}%
\pgfpathlineto{\pgfqpoint{3.715492in}{2.301938in}}%
\pgfpathlineto{\pgfqpoint{3.715956in}{2.440454in}}%
\pgfpathlineto{\pgfqpoint{3.716420in}{2.365039in}}%
\pgfpathlineto{\pgfqpoint{3.716883in}{2.470145in}}%
\pgfpathlineto{\pgfqpoint{3.717346in}{2.456245in}}%
\pgfpathlineto{\pgfqpoint{3.717808in}{2.273499in}}%
\pgfpathlineto{\pgfqpoint{3.718269in}{2.376975in}}%
\pgfpathlineto{\pgfqpoint{3.718730in}{2.421745in}}%
\pgfpathlineto{\pgfqpoint{3.719191in}{2.413423in}}%
\pgfpathlineto{\pgfqpoint{3.719651in}{2.290190in}}%
\pgfpathlineto{\pgfqpoint{3.720110in}{2.365626in}}%
\pgfpathlineto{\pgfqpoint{3.721943in}{2.485773in}}%
\pgfpathlineto{\pgfqpoint{3.722400in}{2.387573in}}%
\pgfpathlineto{\pgfqpoint{3.723313in}{2.418293in}}%
\pgfpathlineto{\pgfqpoint{3.723768in}{2.348583in}}%
\pgfpathlineto{\pgfqpoint{3.724224in}{2.431336in}}%
\pgfpathlineto{\pgfqpoint{3.724678in}{2.495177in}}%
\pgfpathlineto{\pgfqpoint{3.726039in}{2.317763in}}%
\pgfpathlineto{\pgfqpoint{3.726944in}{2.502035in}}%
\pgfpathlineto{\pgfqpoint{3.727396in}{2.450330in}}%
\pgfpathlineto{\pgfqpoint{3.727847in}{2.389170in}}%
\pgfpathlineto{\pgfqpoint{3.728297in}{2.468213in}}%
\pgfpathlineto{\pgfqpoint{3.728748in}{2.461245in}}%
\pgfpathlineto{\pgfqpoint{3.730096in}{2.226666in}}%
\pgfpathlineto{\pgfqpoint{3.731886in}{2.520944in}}%
\pgfpathlineto{\pgfqpoint{3.732333in}{2.215952in}}%
\pgfpathlineto{\pgfqpoint{3.732779in}{2.482672in}}%
\pgfpathlineto{\pgfqpoint{3.733224in}{2.506192in}}%
\pgfpathlineto{\pgfqpoint{3.733669in}{2.316340in}}%
\pgfpathlineto{\pgfqpoint{3.734558in}{2.340020in}}%
\pgfpathlineto{\pgfqpoint{3.735002in}{2.337388in}}%
\pgfpathlineto{\pgfqpoint{3.735445in}{2.371128in}}%
\pgfpathlineto{\pgfqpoint{3.735888in}{2.370215in}}%
\pgfpathlineto{\pgfqpoint{3.736772in}{2.321539in}}%
\pgfpathlineto{\pgfqpoint{3.737654in}{2.309249in}}%
\pgfpathlineto{\pgfqpoint{3.738094in}{2.488119in}}%
\pgfpathlineto{\pgfqpoint{3.738974in}{2.518432in}}%
\pgfpathlineto{\pgfqpoint{3.739413in}{2.365214in}}%
\pgfpathlineto{\pgfqpoint{3.740290in}{2.431404in}}%
\pgfpathlineto{\pgfqpoint{3.740727in}{2.390702in}}%
\pgfpathlineto{\pgfqpoint{3.741165in}{2.237715in}}%
\pgfpathlineto{\pgfqpoint{3.741601in}{2.377226in}}%
\pgfpathlineto{\pgfqpoint{3.742038in}{2.410742in}}%
\pgfpathlineto{\pgfqpoint{3.742474in}{2.233688in}}%
\pgfpathlineto{\pgfqpoint{3.742909in}{2.414144in}}%
\pgfpathlineto{\pgfqpoint{3.743344in}{2.362784in}}%
\pgfpathlineto{\pgfqpoint{3.743779in}{2.380545in}}%
\pgfpathlineto{\pgfqpoint{3.745080in}{2.445144in}}%
\pgfpathlineto{\pgfqpoint{3.745513in}{2.428579in}}%
\pgfpathlineto{\pgfqpoint{3.745945in}{2.148940in}}%
\pgfpathlineto{\pgfqpoint{3.746377in}{2.387687in}}%
\pgfpathlineto{\pgfqpoint{3.746808in}{2.373357in}}%
\pgfpathlineto{\pgfqpoint{3.747670in}{2.476394in}}%
\pgfpathlineto{\pgfqpoint{3.748959in}{2.187477in}}%
\pgfpathlineto{\pgfqpoint{3.750245in}{2.483718in}}%
\pgfpathlineto{\pgfqpoint{3.750672in}{2.320593in}}%
\pgfpathlineto{\pgfqpoint{3.751099in}{2.488185in}}%
\pgfpathlineto{\pgfqpoint{3.751952in}{2.488801in}}%
\pgfpathlineto{\pgfqpoint{3.752378in}{2.370653in}}%
\pgfpathlineto{\pgfqpoint{3.753229in}{2.455587in}}%
\pgfpathlineto{\pgfqpoint{3.754501in}{2.290901in}}%
\pgfpathlineto{\pgfqpoint{3.755347in}{2.464026in}}%
\pgfpathlineto{\pgfqpoint{3.755770in}{2.275516in}}%
\pgfpathlineto{\pgfqpoint{3.756192in}{2.473635in}}%
\pgfpathlineto{\pgfqpoint{3.756613in}{2.301307in}}%
\pgfpathlineto{\pgfqpoint{3.757035in}{2.487568in}}%
\pgfpathlineto{\pgfqpoint{3.757876in}{2.468528in}}%
\pgfpathlineto{\pgfqpoint{3.758296in}{2.469866in}}%
\pgfpathlineto{\pgfqpoint{3.759971in}{2.317908in}}%
\pgfpathlineto{\pgfqpoint{3.760389in}{2.414888in}}%
\pgfpathlineto{\pgfqpoint{3.760807in}{2.268363in}}%
\pgfpathlineto{\pgfqpoint{3.761224in}{2.409773in}}%
\pgfpathlineto{\pgfqpoint{3.761640in}{2.172512in}}%
\pgfpathlineto{\pgfqpoint{3.762057in}{2.378817in}}%
\pgfpathlineto{\pgfqpoint{3.762472in}{2.475489in}}%
\pgfpathlineto{\pgfqpoint{3.762888in}{2.341237in}}%
\pgfpathlineto{\pgfqpoint{3.764132in}{2.421712in}}%
\pgfpathlineto{\pgfqpoint{3.764546in}{2.248147in}}%
\pgfpathlineto{\pgfqpoint{3.764959in}{2.336846in}}%
\pgfpathlineto{\pgfqpoint{3.765372in}{2.483611in}}%
\pgfpathlineto{\pgfqpoint{3.765785in}{2.450938in}}%
\pgfpathlineto{\pgfqpoint{3.767020in}{2.190417in}}%
\pgfpathlineto{\pgfqpoint{3.766609in}{2.451249in}}%
\pgfpathlineto{\pgfqpoint{3.767431in}{2.274004in}}%
\pgfpathlineto{\pgfqpoint{3.768252in}{2.500181in}}%
\pgfpathlineto{\pgfqpoint{3.768662in}{2.404390in}}%
\pgfpathlineto{\pgfqpoint{3.769480in}{2.307830in}}%
\pgfpathlineto{\pgfqpoint{3.769889in}{2.376474in}}%
\pgfpathlineto{\pgfqpoint{3.770297in}{2.399554in}}%
\pgfpathlineto{\pgfqpoint{3.770705in}{2.379713in}}%
\pgfpathlineto{\pgfqpoint{3.771112in}{2.095145in}}%
\pgfpathlineto{\pgfqpoint{3.771519in}{2.320888in}}%
\pgfpathlineto{\pgfqpoint{3.771926in}{2.449113in}}%
\pgfpathlineto{\pgfqpoint{3.772332in}{2.322674in}}%
\pgfpathlineto{\pgfqpoint{3.772738in}{2.345553in}}%
\pgfpathlineto{\pgfqpoint{3.773144in}{2.440849in}}%
\pgfpathlineto{\pgfqpoint{3.773549in}{2.365724in}}%
\pgfpathlineto{\pgfqpoint{3.774358in}{2.272269in}}%
\pgfpathlineto{\pgfqpoint{3.774762in}{2.496477in}}%
\pgfpathlineto{\pgfqpoint{3.775165in}{2.435122in}}%
\pgfpathlineto{\pgfqpoint{3.775569in}{2.221458in}}%
\pgfpathlineto{\pgfqpoint{3.776374in}{2.363736in}}%
\pgfpathlineto{\pgfqpoint{3.776776in}{2.359014in}}%
\pgfpathlineto{\pgfqpoint{3.777579in}{2.319109in}}%
\pgfpathlineto{\pgfqpoint{3.777980in}{2.446836in}}%
\pgfpathlineto{\pgfqpoint{3.778780in}{2.256466in}}%
\pgfpathlineto{\pgfqpoint{3.779180in}{2.336171in}}%
\pgfpathlineto{\pgfqpoint{3.779580in}{2.488236in}}%
\pgfpathlineto{\pgfqpoint{3.779979in}{2.277925in}}%
\pgfpathlineto{\pgfqpoint{3.780377in}{2.423273in}}%
\pgfpathlineto{\pgfqpoint{3.780776in}{2.405595in}}%
\pgfpathlineto{\pgfqpoint{3.781968in}{2.485346in}}%
\pgfpathlineto{\pgfqpoint{3.782365in}{2.243894in}}%
\pgfpathlineto{\pgfqpoint{3.783157in}{2.339529in}}%
\pgfpathlineto{\pgfqpoint{3.783553in}{2.528421in}}%
\pgfpathlineto{\pgfqpoint{3.784343in}{2.512135in}}%
\pgfpathlineto{\pgfqpoint{3.785526in}{2.230450in}}%
\pgfpathlineto{\pgfqpoint{3.785919in}{2.531745in}}%
\pgfpathlineto{\pgfqpoint{3.786705in}{2.361332in}}%
\pgfpathlineto{\pgfqpoint{3.787098in}{2.493797in}}%
\pgfpathlineto{\pgfqpoint{3.787490in}{2.435658in}}%
\pgfpathlineto{\pgfqpoint{3.787881in}{2.117160in}}%
\pgfpathlineto{\pgfqpoint{3.788664in}{2.354147in}}%
\pgfpathlineto{\pgfqpoint{3.789054in}{2.479804in}}%
\pgfpathlineto{\pgfqpoint{3.789444in}{2.386821in}}%
\pgfpathlineto{\pgfqpoint{3.789834in}{2.244381in}}%
\pgfpathlineto{\pgfqpoint{3.790224in}{2.456004in}}%
\pgfpathlineto{\pgfqpoint{3.790613in}{2.365295in}}%
\pgfpathlineto{\pgfqpoint{3.791002in}{2.405811in}}%
\pgfpathlineto{\pgfqpoint{3.791390in}{1.802378in}}%
\pgfpathlineto{\pgfqpoint{3.791778in}{2.475015in}}%
\pgfpathlineto{\pgfqpoint{3.792166in}{2.256925in}}%
\pgfpathlineto{\pgfqpoint{3.792553in}{2.396862in}}%
\pgfpathlineto{\pgfqpoint{3.792940in}{2.378846in}}%
\pgfpathlineto{\pgfqpoint{3.793327in}{2.248241in}}%
\pgfpathlineto{\pgfqpoint{3.793713in}{2.449801in}}%
\pgfpathlineto{\pgfqpoint{3.794099in}{2.461208in}}%
\pgfpathlineto{\pgfqpoint{3.794485in}{2.397529in}}%
\pgfpathlineto{\pgfqpoint{3.795255in}{2.425839in}}%
\pgfpathlineto{\pgfqpoint{3.795640in}{2.480337in}}%
\pgfpathlineto{\pgfqpoint{3.796024in}{2.251031in}}%
\pgfpathlineto{\pgfqpoint{3.796408in}{2.483077in}}%
\pgfpathlineto{\pgfqpoint{3.796791in}{2.411431in}}%
\pgfpathlineto{\pgfqpoint{3.797174in}{2.377747in}}%
\pgfpathlineto{\pgfqpoint{3.797557in}{2.113273in}}%
\pgfpathlineto{\pgfqpoint{3.797940in}{2.497063in}}%
\pgfpathlineto{\pgfqpoint{3.798322in}{2.502965in}}%
\pgfpathlineto{\pgfqpoint{3.798704in}{2.177777in}}%
\pgfpathlineto{\pgfqpoint{3.799466in}{2.330950in}}%
\pgfpathlineto{\pgfqpoint{3.799847in}{2.473613in}}%
\pgfpathlineto{\pgfqpoint{3.800607in}{2.415591in}}%
\pgfpathlineto{\pgfqpoint{3.801367in}{2.353540in}}%
\pgfpathlineto{\pgfqpoint{3.801746in}{2.468536in}}%
\pgfpathlineto{\pgfqpoint{3.802503in}{2.464512in}}%
\pgfpathlineto{\pgfqpoint{3.802881in}{2.165294in}}%
\pgfpathlineto{\pgfqpoint{3.803258in}{2.393994in}}%
\pgfpathlineto{\pgfqpoint{3.803636in}{2.475682in}}%
\pgfpathlineto{\pgfqpoint{3.804013in}{2.326918in}}%
\pgfpathlineto{\pgfqpoint{3.804766in}{2.167537in}}%
\pgfpathlineto{\pgfqpoint{3.805142in}{2.506119in}}%
\pgfpathlineto{\pgfqpoint{3.805893in}{2.424604in}}%
\pgfpathlineto{\pgfqpoint{3.806268in}{2.421346in}}%
\pgfpathlineto{\pgfqpoint{3.806643in}{2.493191in}}%
\pgfpathlineto{\pgfqpoint{3.807017in}{2.404087in}}%
\pgfpathlineto{\pgfqpoint{3.808138in}{2.107442in}}%
\pgfpathlineto{\pgfqpoint{3.807765in}{2.454292in}}%
\pgfpathlineto{\pgfqpoint{3.808511in}{2.330616in}}%
\pgfpathlineto{\pgfqpoint{3.809257in}{2.303863in}}%
\pgfpathlineto{\pgfqpoint{3.810372in}{2.416505in}}%
\pgfpathlineto{\pgfqpoint{3.810743in}{2.429522in}}%
\pgfpathlineto{\pgfqpoint{3.811114in}{2.058317in}}%
\pgfpathlineto{\pgfqpoint{3.811484in}{2.490638in}}%
\pgfpathlineto{\pgfqpoint{3.811854in}{2.526627in}}%
\pgfpathlineto{\pgfqpoint{3.812224in}{2.349309in}}%
\pgfpathlineto{\pgfqpoint{3.812963in}{2.493909in}}%
\pgfpathlineto{\pgfqpoint{3.814436in}{2.346392in}}%
\pgfpathlineto{\pgfqpoint{3.814804in}{2.462609in}}%
\pgfpathlineto{\pgfqpoint{3.815171in}{2.330381in}}%
\pgfpathlineto{\pgfqpoint{3.815538in}{2.332755in}}%
\pgfpathlineto{\pgfqpoint{3.815905in}{2.497821in}}%
\pgfpathlineto{\pgfqpoint{3.816271in}{2.187172in}}%
\pgfpathlineto{\pgfqpoint{3.816637in}{2.475616in}}%
\pgfpathlineto{\pgfqpoint{3.817368in}{2.227879in}}%
\pgfpathlineto{\pgfqpoint{3.818098in}{2.358709in}}%
\pgfpathlineto{\pgfqpoint{3.819554in}{2.453161in}}%
\pgfpathlineto{\pgfqpoint{3.819917in}{2.443180in}}%
\pgfpathlineto{\pgfqpoint{3.820280in}{2.478964in}}%
\pgfpathlineto{\pgfqpoint{3.820642in}{2.437371in}}%
\pgfpathlineto{\pgfqpoint{3.821004in}{2.408326in}}%
\pgfpathlineto{\pgfqpoint{3.821366in}{2.223300in}}%
\pgfpathlineto{\pgfqpoint{3.821728in}{2.339566in}}%
\pgfpathlineto{\pgfqpoint{3.822811in}{2.447542in}}%
\pgfpathlineto{\pgfqpoint{3.823531in}{2.299188in}}%
\pgfpathlineto{\pgfqpoint{3.823891in}{2.378959in}}%
\pgfpathlineto{\pgfqpoint{3.824610in}{2.481259in}}%
\pgfpathlineto{\pgfqpoint{3.824969in}{2.238866in}}%
\pgfpathlineto{\pgfqpoint{3.825686in}{2.411406in}}%
\pgfpathlineto{\pgfqpoint{3.826044in}{2.481881in}}%
\pgfpathlineto{\pgfqpoint{3.826401in}{2.358298in}}%
\pgfpathlineto{\pgfqpoint{3.826759in}{2.448171in}}%
\pgfpathlineto{\pgfqpoint{3.827472in}{2.311587in}}%
\pgfpathlineto{\pgfqpoint{3.827829in}{2.451491in}}%
\pgfpathlineto{\pgfqpoint{3.828185in}{2.433085in}}%
\pgfpathlineto{\pgfqpoint{3.828896in}{2.426776in}}%
\pgfpathlineto{\pgfqpoint{3.829252in}{2.480539in}}%
\pgfpathlineto{\pgfqpoint{3.830316in}{2.213141in}}%
\pgfpathlineto{\pgfqpoint{3.831024in}{2.547131in}}%
\pgfpathlineto{\pgfqpoint{3.831377in}{2.255542in}}%
\pgfpathlineto{\pgfqpoint{3.833140in}{2.494941in}}%
\pgfpathlineto{\pgfqpoint{3.833492in}{2.496193in}}%
\pgfpathlineto{\pgfqpoint{3.834896in}{2.277084in}}%
\pgfpathlineto{\pgfqpoint{3.835246in}{2.300268in}}%
\pgfpathlineto{\pgfqpoint{3.835596in}{2.236582in}}%
\pgfpathlineto{\pgfqpoint{3.835946in}{2.431572in}}%
\pgfpathlineto{\pgfqpoint{3.836644in}{2.315548in}}%
\pgfpathlineto{\pgfqpoint{3.836993in}{2.095793in}}%
\pgfpathlineto{\pgfqpoint{3.837342in}{2.114542in}}%
\pgfpathlineto{\pgfqpoint{3.838733in}{2.467612in}}%
\pgfpathlineto{\pgfqpoint{3.839427in}{2.494938in}}%
\pgfpathlineto{\pgfqpoint{3.840120in}{2.412408in}}%
\pgfpathlineto{\pgfqpoint{3.840466in}{2.471443in}}%
\pgfpathlineto{\pgfqpoint{3.840812in}{2.387986in}}%
\pgfpathlineto{\pgfqpoint{3.841157in}{2.400956in}}%
\pgfpathlineto{\pgfqpoint{3.841502in}{2.336059in}}%
\pgfpathlineto{\pgfqpoint{3.841847in}{2.495225in}}%
\pgfpathlineto{\pgfqpoint{3.842880in}{2.467979in}}%
\pgfpathlineto{\pgfqpoint{3.843224in}{2.149253in}}%
\pgfpathlineto{\pgfqpoint{3.843910in}{2.335283in}}%
\pgfpathlineto{\pgfqpoint{3.844253in}{2.468538in}}%
\pgfpathlineto{\pgfqpoint{3.844596in}{2.304689in}}%
\pgfpathlineto{\pgfqpoint{3.844938in}{2.288970in}}%
\pgfpathlineto{\pgfqpoint{3.846305in}{2.488010in}}%
\pgfpathlineto{\pgfqpoint{3.847327in}{2.408398in}}%
\pgfpathlineto{\pgfqpoint{3.848347in}{2.487576in}}%
\pgfpathlineto{\pgfqpoint{3.849364in}{2.293884in}}%
\pgfpathlineto{\pgfqpoint{3.849703in}{2.368872in}}%
\pgfpathlineto{\pgfqpoint{3.850041in}{2.420840in}}%
\pgfpathlineto{\pgfqpoint{3.850717in}{2.370259in}}%
\pgfpathlineto{\pgfqpoint{3.852065in}{2.484069in}}%
\pgfpathlineto{\pgfqpoint{3.852401in}{2.342336in}}%
\pgfpathlineto{\pgfqpoint{3.853409in}{2.357551in}}%
\pgfpathlineto{\pgfqpoint{3.853744in}{2.288907in}}%
\pgfpathlineto{\pgfqpoint{3.854079in}{2.348136in}}%
\pgfpathlineto{\pgfqpoint{3.854414in}{2.458786in}}%
\pgfpathlineto{\pgfqpoint{3.854749in}{2.287228in}}%
\pgfpathlineto{\pgfqpoint{3.855083in}{2.340907in}}%
\pgfpathlineto{\pgfqpoint{3.855417in}{2.153853in}}%
\pgfpathlineto{\pgfqpoint{3.856084in}{2.389020in}}%
\pgfpathlineto{\pgfqpoint{3.856417in}{2.455840in}}%
\pgfpathlineto{\pgfqpoint{3.857083in}{2.425078in}}%
\pgfpathlineto{\pgfqpoint{3.857416in}{2.363098in}}%
\pgfpathlineto{\pgfqpoint{3.857748in}{2.559255in}}%
\pgfpathlineto{\pgfqpoint{3.858080in}{2.227820in}}%
\pgfpathlineto{\pgfqpoint{3.858411in}{2.451267in}}%
\pgfpathlineto{\pgfqpoint{3.858743in}{2.452011in}}%
\pgfpathlineto{\pgfqpoint{3.859074in}{2.428937in}}%
\pgfpathlineto{\pgfqpoint{3.860066in}{2.171968in}}%
\pgfpathlineto{\pgfqpoint{3.860396in}{2.222476in}}%
\pgfpathlineto{\pgfqpoint{3.861055in}{2.215900in}}%
\pgfpathlineto{\pgfqpoint{3.861714in}{2.539013in}}%
\pgfpathlineto{\pgfqpoint{3.862700in}{2.342931in}}%
\pgfpathlineto{\pgfqpoint{3.863028in}{2.509764in}}%
\pgfpathlineto{\pgfqpoint{3.863355in}{2.303977in}}%
\pgfpathlineto{\pgfqpoint{3.863683in}{2.452686in}}%
\pgfpathlineto{\pgfqpoint{3.864991in}{2.222381in}}%
\pgfpathlineto{\pgfqpoint{3.864337in}{2.461839in}}%
\pgfpathlineto{\pgfqpoint{3.865317in}{2.306200in}}%
\pgfpathlineto{\pgfqpoint{3.866620in}{2.460311in}}%
\pgfpathlineto{\pgfqpoint{3.866945in}{1.928933in}}%
\pgfpathlineto{\pgfqpoint{3.867594in}{2.220971in}}%
\pgfpathlineto{\pgfqpoint{3.868243in}{2.410003in}}%
\pgfpathlineto{\pgfqpoint{3.868890in}{2.341935in}}%
\pgfpathlineto{\pgfqpoint{3.870504in}{2.511741in}}%
\pgfpathlineto{\pgfqpoint{3.870826in}{2.494862in}}%
\pgfpathlineto{\pgfqpoint{3.872112in}{2.369875in}}%
\pgfpathlineto{\pgfqpoint{3.872433in}{2.391367in}}%
\pgfpathlineto{\pgfqpoint{3.872754in}{2.373350in}}%
\pgfpathlineto{\pgfqpoint{3.873074in}{2.524611in}}%
\pgfpathlineto{\pgfqpoint{3.873394in}{2.403054in}}%
\pgfpathlineto{\pgfqpoint{3.873714in}{2.122504in}}%
\pgfpathlineto{\pgfqpoint{3.874353in}{2.440989in}}%
\pgfpathlineto{\pgfqpoint{3.875629in}{2.221149in}}%
\pgfpathlineto{\pgfqpoint{3.876900in}{2.459970in}}%
\pgfpathlineto{\pgfqpoint{3.877217in}{2.331866in}}%
\pgfpathlineto{\pgfqpoint{3.877851in}{2.419112in}}%
\pgfpathlineto{\pgfqpoint{3.878484in}{2.413743in}}%
\pgfpathlineto{\pgfqpoint{3.878800in}{2.377830in}}%
\pgfpathlineto{\pgfqpoint{3.879116in}{2.452374in}}%
\pgfpathlineto{\pgfqpoint{3.879432in}{2.387442in}}%
\pgfpathlineto{\pgfqpoint{3.880377in}{2.460612in}}%
\pgfpathlineto{\pgfqpoint{3.880692in}{2.436236in}}%
\pgfpathlineto{\pgfqpoint{3.881635in}{2.206461in}}%
\pgfpathlineto{\pgfqpoint{3.881948in}{2.512666in}}%
\pgfpathlineto{\pgfqpoint{3.882575in}{2.205573in}}%
\pgfpathlineto{\pgfqpoint{3.882888in}{2.219625in}}%
\pgfpathlineto{\pgfqpoint{3.883826in}{2.452821in}}%
\pgfpathlineto{\pgfqpoint{3.883514in}{2.132832in}}%
\pgfpathlineto{\pgfqpoint{3.884138in}{2.406616in}}%
\pgfpathlineto{\pgfqpoint{3.884450in}{2.282927in}}%
\pgfpathlineto{\pgfqpoint{3.885073in}{2.429648in}}%
\pgfpathlineto{\pgfqpoint{3.885384in}{2.430580in}}%
\pgfpathlineto{\pgfqpoint{3.886317in}{2.184499in}}%
\pgfpathlineto{\pgfqpoint{3.886627in}{2.334308in}}%
\pgfpathlineto{\pgfqpoint{3.887247in}{2.456181in}}%
\pgfpathlineto{\pgfqpoint{3.887866in}{2.440961in}}%
\pgfpathlineto{\pgfqpoint{3.888175in}{2.300650in}}%
\pgfpathlineto{\pgfqpoint{3.889101in}{2.304857in}}%
\pgfpathlineto{\pgfqpoint{3.889410in}{2.460138in}}%
\pgfpathlineto{\pgfqpoint{3.890333in}{2.453439in}}%
\pgfpathlineto{\pgfqpoint{3.891562in}{2.263316in}}%
\pgfpathlineto{\pgfqpoint{3.890948in}{2.462998in}}%
\pgfpathlineto{\pgfqpoint{3.891868in}{2.368646in}}%
\pgfpathlineto{\pgfqpoint{3.892786in}{2.473387in}}%
\pgfpathlineto{\pgfqpoint{3.893092in}{2.289232in}}%
\pgfpathlineto{\pgfqpoint{3.893703in}{2.328192in}}%
\pgfpathlineto{\pgfqpoint{3.894007in}{2.520149in}}%
\pgfpathlineto{\pgfqpoint{3.894617in}{2.476017in}}%
\pgfpathlineto{\pgfqpoint{3.894921in}{2.152246in}}%
\pgfpathlineto{\pgfqpoint{3.895833in}{2.309949in}}%
\pgfpathlineto{\pgfqpoint{3.896136in}{2.193726in}}%
\pgfpathlineto{\pgfqpoint{3.896439in}{2.494441in}}%
\pgfpathlineto{\pgfqpoint{3.897348in}{2.198375in}}%
\pgfpathlineto{\pgfqpoint{3.897650in}{2.503298in}}%
\pgfpathlineto{\pgfqpoint{3.898254in}{2.147351in}}%
\pgfpathlineto{\pgfqpoint{3.898556in}{2.499730in}}%
\pgfpathlineto{\pgfqpoint{3.898857in}{2.223850in}}%
\pgfpathlineto{\pgfqpoint{3.899460in}{2.300737in}}%
\pgfpathlineto{\pgfqpoint{3.899761in}{2.462521in}}%
\pgfpathlineto{\pgfqpoint{3.900362in}{2.396920in}}%
\pgfpathlineto{\pgfqpoint{3.900962in}{2.326121in}}%
\pgfpathlineto{\pgfqpoint{3.901262in}{2.474036in}}%
\pgfpathlineto{\pgfqpoint{3.901561in}{2.316652in}}%
\pgfpathlineto{\pgfqpoint{3.902160in}{2.434791in}}%
\pgfpathlineto{\pgfqpoint{3.902758in}{2.190550in}}%
\pgfpathlineto{\pgfqpoint{3.903056in}{2.398314in}}%
\pgfpathlineto{\pgfqpoint{3.903652in}{2.492654in}}%
\pgfpathlineto{\pgfqpoint{3.904248in}{2.477917in}}%
\pgfpathlineto{\pgfqpoint{3.905140in}{2.558408in}}%
\pgfpathlineto{\pgfqpoint{3.905437in}{2.157040in}}%
\pgfpathlineto{\pgfqpoint{3.906622in}{2.471280in}}%
\pgfpathlineto{\pgfqpoint{3.906918in}{2.480252in}}%
\pgfpathlineto{\pgfqpoint{3.908394in}{2.202388in}}%
\pgfpathlineto{\pgfqpoint{3.909277in}{2.498206in}}%
\pgfpathlineto{\pgfqpoint{3.909571in}{2.406261in}}%
\pgfpathlineto{\pgfqpoint{3.909865in}{2.348839in}}%
\pgfpathlineto{\pgfqpoint{3.910159in}{2.549597in}}%
\pgfpathlineto{\pgfqpoint{3.911038in}{2.397325in}}%
\pgfpathlineto{\pgfqpoint{3.911331in}{2.486361in}}%
\pgfpathlineto{\pgfqpoint{3.911623in}{2.340518in}}%
\pgfpathlineto{\pgfqpoint{3.911916in}{2.358329in}}%
\pgfpathlineto{\pgfqpoint{3.912500in}{2.410012in}}%
\pgfpathlineto{\pgfqpoint{3.912792in}{2.378708in}}%
\pgfpathlineto{\pgfqpoint{3.913083in}{2.343068in}}%
\pgfpathlineto{\pgfqpoint{3.913957in}{2.501108in}}%
\pgfpathlineto{\pgfqpoint{3.914247in}{2.443113in}}%
\pgfpathlineto{\pgfqpoint{3.914538in}{2.429733in}}%
\pgfpathlineto{\pgfqpoint{3.914828in}{2.344683in}}%
\pgfpathlineto{\pgfqpoint{3.915118in}{2.486807in}}%
\pgfpathlineto{\pgfqpoint{3.915408in}{2.367164in}}%
\pgfpathlineto{\pgfqpoint{3.915698in}{2.453216in}}%
\pgfpathlineto{\pgfqpoint{3.915988in}{2.345456in}}%
\pgfpathlineto{\pgfqpoint{3.916277in}{2.406377in}}%
\pgfpathlineto{\pgfqpoint{3.917144in}{2.247254in}}%
\pgfpathlineto{\pgfqpoint{3.918297in}{2.454308in}}%
\pgfpathlineto{\pgfqpoint{3.918873in}{2.407038in}}%
\pgfpathlineto{\pgfqpoint{3.919447in}{2.513649in}}%
\pgfpathlineto{\pgfqpoint{3.919734in}{2.349055in}}%
\pgfpathlineto{\pgfqpoint{3.920021in}{2.458332in}}%
\pgfpathlineto{\pgfqpoint{3.920308in}{2.330057in}}%
\pgfpathlineto{\pgfqpoint{3.920594in}{2.530629in}}%
\pgfpathlineto{\pgfqpoint{3.921167in}{2.435899in}}%
\pgfpathlineto{\pgfqpoint{3.921452in}{2.543039in}}%
\pgfpathlineto{\pgfqpoint{3.922024in}{2.417652in}}%
\pgfpathlineto{\pgfqpoint{3.922309in}{2.414982in}}%
\pgfpathlineto{\pgfqpoint{3.922879in}{2.451533in}}%
\pgfpathlineto{\pgfqpoint{3.923448in}{2.381932in}}%
\pgfpathlineto{\pgfqpoint{3.923732in}{2.435949in}}%
\pgfpathlineto{\pgfqpoint{3.924017in}{2.425088in}}%
\pgfpathlineto{\pgfqpoint{3.924584in}{2.192605in}}%
\pgfpathlineto{\pgfqpoint{3.924868in}{2.482009in}}%
\pgfpathlineto{\pgfqpoint{3.925151in}{2.462364in}}%
\pgfpathlineto{\pgfqpoint{3.925434in}{2.048309in}}%
\pgfpathlineto{\pgfqpoint{3.926000in}{2.439708in}}%
\pgfpathlineto{\pgfqpoint{3.926283in}{2.529686in}}%
\pgfpathlineto{\pgfqpoint{3.926565in}{2.387630in}}%
\pgfpathlineto{\pgfqpoint{3.926847in}{2.391613in}}%
\pgfpathlineto{\pgfqpoint{3.927411in}{2.326165in}}%
\pgfpathlineto{\pgfqpoint{3.927693in}{2.224190in}}%
\pgfpathlineto{\pgfqpoint{3.928256in}{2.399395in}}%
\pgfpathlineto{\pgfqpoint{3.928537in}{2.494712in}}%
\pgfpathlineto{\pgfqpoint{3.929099in}{2.353942in}}%
\pgfpathlineto{\pgfqpoint{3.929940in}{2.193925in}}%
\pgfpathlineto{\pgfqpoint{3.930779in}{2.472970in}}%
\pgfpathlineto{\pgfqpoint{3.931059in}{2.464912in}}%
\pgfpathlineto{\pgfqpoint{3.931617in}{2.291050in}}%
\pgfpathlineto{\pgfqpoint{3.932175in}{2.295067in}}%
\pgfpathlineto{\pgfqpoint{3.932453in}{2.495040in}}%
\pgfpathlineto{\pgfqpoint{3.933288in}{2.328030in}}%
\pgfpathlineto{\pgfqpoint{3.933566in}{2.291765in}}%
\pgfpathlineto{\pgfqpoint{3.933843in}{2.468451in}}%
\pgfpathlineto{\pgfqpoint{3.934121in}{2.274421in}}%
\pgfpathlineto{\pgfqpoint{3.934675in}{2.331283in}}%
\pgfpathlineto{\pgfqpoint{3.935505in}{2.512684in}}%
\pgfpathlineto{\pgfqpoint{3.935782in}{2.408221in}}%
\pgfpathlineto{\pgfqpoint{3.936334in}{2.419369in}}%
\pgfpathlineto{\pgfqpoint{3.937161in}{2.307499in}}%
\pgfpathlineto{\pgfqpoint{3.938810in}{2.432623in}}%
\pgfpathlineto{\pgfqpoint{3.939084in}{2.278416in}}%
\pgfpathlineto{\pgfqpoint{3.939358in}{2.444648in}}%
\pgfpathlineto{\pgfqpoint{3.939906in}{2.376838in}}%
\pgfpathlineto{\pgfqpoint{3.940453in}{2.513981in}}%
\pgfpathlineto{\pgfqpoint{3.940999in}{2.450814in}}%
\pgfpathlineto{\pgfqpoint{3.942633in}{2.124811in}}%
\pgfpathlineto{\pgfqpoint{3.943990in}{2.484305in}}%
\pgfpathlineto{\pgfqpoint{3.945343in}{2.299217in}}%
\pgfpathlineto{\pgfqpoint{3.945883in}{2.495770in}}%
\pgfpathlineto{\pgfqpoint{3.946422in}{2.416823in}}%
\pgfpathlineto{\pgfqpoint{3.947498in}{2.490898in}}%
\pgfpathlineto{\pgfqpoint{3.947767in}{2.294532in}}%
\pgfpathlineto{\pgfqpoint{3.948572in}{2.497696in}}%
\pgfpathlineto{\pgfqpoint{3.948840in}{2.488512in}}%
\pgfpathlineto{\pgfqpoint{3.949910in}{2.150463in}}%
\pgfpathlineto{\pgfqpoint{3.950178in}{2.512648in}}%
\pgfpathlineto{\pgfqpoint{3.950978in}{2.485449in}}%
\pgfpathlineto{\pgfqpoint{3.951511in}{2.210993in}}%
\pgfpathlineto{\pgfqpoint{3.952043in}{2.406920in}}%
\pgfpathlineto{\pgfqpoint{3.952309in}{2.530154in}}%
\pgfpathlineto{\pgfqpoint{3.952840in}{2.291409in}}%
\pgfpathlineto{\pgfqpoint{3.953105in}{2.423355in}}%
\pgfpathlineto{\pgfqpoint{3.953900in}{2.208573in}}%
\pgfpathlineto{\pgfqpoint{3.954165in}{2.407786in}}%
\pgfpathlineto{\pgfqpoint{3.954429in}{2.406895in}}%
\pgfpathlineto{\pgfqpoint{3.954693in}{2.283562in}}%
\pgfpathlineto{\pgfqpoint{3.955485in}{2.306199in}}%
\pgfpathlineto{\pgfqpoint{3.956802in}{2.436459in}}%
\pgfpathlineto{\pgfqpoint{3.957328in}{2.481048in}}%
\pgfpathlineto{\pgfqpoint{3.958115in}{2.109879in}}%
\pgfpathlineto{\pgfqpoint{3.958639in}{2.518784in}}%
\pgfpathlineto{\pgfqpoint{3.959423in}{2.458092in}}%
\pgfpathlineto{\pgfqpoint{3.959684in}{2.471067in}}%
\pgfpathlineto{\pgfqpoint{3.960206in}{2.388335in}}%
\pgfpathlineto{\pgfqpoint{3.960988in}{2.408736in}}%
\pgfpathlineto{\pgfqpoint{3.961248in}{2.500250in}}%
\pgfpathlineto{\pgfqpoint{3.961509in}{2.331592in}}%
\pgfpathlineto{\pgfqpoint{3.961769in}{2.446534in}}%
\pgfpathlineto{\pgfqpoint{3.962547in}{2.196894in}}%
\pgfpathlineto{\pgfqpoint{3.962807in}{2.473087in}}%
\pgfpathlineto{\pgfqpoint{3.963325in}{2.245460in}}%
\pgfpathlineto{\pgfqpoint{3.963842in}{2.299406in}}%
\pgfpathlineto{\pgfqpoint{3.964101in}{2.495895in}}%
\pgfpathlineto{\pgfqpoint{3.964359in}{2.158749in}}%
\pgfpathlineto{\pgfqpoint{3.964876in}{2.418049in}}%
\pgfpathlineto{\pgfqpoint{3.965391in}{2.460774in}}%
\pgfpathlineto{\pgfqpoint{3.965649in}{2.255898in}}%
\pgfpathlineto{\pgfqpoint{3.966421in}{2.122613in}}%
\pgfpathlineto{\pgfqpoint{3.966934in}{2.490791in}}%
\pgfpathlineto{\pgfqpoint{3.967191in}{2.237549in}}%
\pgfpathlineto{\pgfqpoint{3.967960in}{2.319896in}}%
\pgfpathlineto{\pgfqpoint{3.969239in}{2.485476in}}%
\pgfpathlineto{\pgfqpoint{3.970513in}{2.256308in}}%
\pgfpathlineto{\pgfqpoint{3.971530in}{2.512224in}}%
\pgfpathlineto{\pgfqpoint{3.971784in}{2.415219in}}%
\pgfpathlineto{\pgfqpoint{3.972038in}{2.293656in}}%
\pgfpathlineto{\pgfqpoint{3.972292in}{2.422947in}}%
\pgfpathlineto{\pgfqpoint{3.972545in}{2.415066in}}%
\pgfpathlineto{\pgfqpoint{3.972798in}{2.465570in}}%
\pgfpathlineto{\pgfqpoint{3.973051in}{2.242250in}}%
\pgfpathlineto{\pgfqpoint{3.973304in}{2.514801in}}%
\pgfpathlineto{\pgfqpoint{3.973810in}{2.446456in}}%
\pgfpathlineto{\pgfqpoint{3.974567in}{2.450396in}}%
\pgfpathlineto{\pgfqpoint{3.975574in}{2.243869in}}%
\pgfpathlineto{\pgfqpoint{3.976579in}{2.498491in}}%
\pgfpathlineto{\pgfqpoint{3.976328in}{2.074884in}}%
\pgfpathlineto{\pgfqpoint{3.976830in}{2.458416in}}%
\pgfpathlineto{\pgfqpoint{3.977332in}{2.251670in}}%
\pgfpathlineto{\pgfqpoint{3.977832in}{2.369827in}}%
\pgfpathlineto{\pgfqpoint{3.978082in}{2.451453in}}%
\pgfpathlineto{\pgfqpoint{3.978832in}{2.394620in}}%
\pgfpathlineto{\pgfqpoint{3.979580in}{2.281890in}}%
\pgfpathlineto{\pgfqpoint{3.979829in}{2.322220in}}%
\pgfpathlineto{\pgfqpoint{3.980078in}{2.472912in}}%
\pgfpathlineto{\pgfqpoint{3.980327in}{2.307980in}}%
\pgfpathlineto{\pgfqpoint{3.981073in}{2.422375in}}%
\pgfpathlineto{\pgfqpoint{3.981321in}{2.509203in}}%
\pgfpathlineto{\pgfqpoint{3.981817in}{2.346337in}}%
\pgfpathlineto{\pgfqpoint{3.982065in}{2.458169in}}%
\pgfpathlineto{\pgfqpoint{3.982312in}{2.152050in}}%
\pgfpathlineto{\pgfqpoint{3.983054in}{2.304586in}}%
\pgfpathlineto{\pgfqpoint{3.984042in}{2.525108in}}%
\pgfpathlineto{\pgfqpoint{3.984535in}{2.306846in}}%
\pgfpathlineto{\pgfqpoint{3.985273in}{2.394555in}}%
\pgfpathlineto{\pgfqpoint{3.986255in}{2.167477in}}%
\pgfpathlineto{\pgfqpoint{3.986501in}{2.312143in}}%
\pgfpathlineto{\pgfqpoint{3.986746in}{2.466529in}}%
\pgfpathlineto{\pgfqpoint{3.987480in}{2.257631in}}%
\pgfpathlineto{\pgfqpoint{3.987725in}{2.441624in}}%
\pgfpathlineto{\pgfqpoint{3.987969in}{2.352988in}}%
\pgfpathlineto{\pgfqpoint{3.988457in}{2.496549in}}%
\pgfpathlineto{\pgfqpoint{3.988945in}{2.390127in}}%
\pgfpathlineto{\pgfqpoint{3.989189in}{2.453127in}}%
\pgfpathlineto{\pgfqpoint{3.989432in}{2.342535in}}%
\pgfpathlineto{\pgfqpoint{3.989919in}{2.378876in}}%
\pgfpathlineto{\pgfqpoint{3.990648in}{2.421738in}}%
\pgfpathlineto{\pgfqpoint{3.990891in}{2.188824in}}%
\pgfpathlineto{\pgfqpoint{3.991860in}{2.494296in}}%
\pgfpathlineto{\pgfqpoint{3.992102in}{2.426051in}}%
\pgfpathlineto{\pgfqpoint{3.992827in}{2.334942in}}%
\pgfpathlineto{\pgfqpoint{3.992586in}{2.452817in}}%
\pgfpathlineto{\pgfqpoint{3.993310in}{2.389801in}}%
\pgfpathlineto{\pgfqpoint{3.993551in}{2.524011in}}%
\pgfpathlineto{\pgfqpoint{3.994033in}{2.363919in}}%
\pgfpathlineto{\pgfqpoint{3.994515in}{2.468819in}}%
\pgfpathlineto{\pgfqpoint{3.995236in}{2.297132in}}%
\pgfpathlineto{\pgfqpoint{3.995716in}{2.348861in}}%
\pgfpathlineto{\pgfqpoint{3.996195in}{2.498859in}}%
\pgfpathlineto{\pgfqpoint{3.996674in}{2.440168in}}%
\pgfpathlineto{\pgfqpoint{3.997153in}{2.293408in}}%
\pgfpathlineto{\pgfqpoint{3.997630in}{2.384930in}}%
\pgfpathlineto{\pgfqpoint{3.998346in}{2.453598in}}%
\pgfpathlineto{\pgfqpoint{3.998585in}{2.134707in}}%
\pgfpathlineto{\pgfqpoint{3.999537in}{2.306818in}}%
\pgfpathlineto{\pgfqpoint{3.999774in}{2.453465in}}%
\pgfpathlineto{\pgfqpoint{4.000487in}{2.446677in}}%
\pgfpathlineto{\pgfqpoint{4.000961in}{2.469827in}}%
\pgfpathlineto{\pgfqpoint{4.001671in}{1.846547in}}%
\pgfpathlineto{\pgfqpoint{4.001907in}{2.420240in}}%
\pgfpathlineto{\pgfqpoint{4.002852in}{2.210118in}}%
\pgfpathlineto{\pgfqpoint{4.004265in}{2.428819in}}%
\pgfpathlineto{\pgfqpoint{4.004500in}{2.418901in}}%
\pgfpathlineto{\pgfqpoint{4.004735in}{2.201132in}}%
\pgfpathlineto{\pgfqpoint{4.005439in}{2.504059in}}%
\pgfpathlineto{\pgfqpoint{4.005673in}{2.267425in}}%
\pgfpathlineto{\pgfqpoint{4.006610in}{2.481690in}}%
\pgfpathlineto{\pgfqpoint{4.006843in}{2.471202in}}%
\pgfpathlineto{\pgfqpoint{4.007077in}{2.183647in}}%
\pgfpathlineto{\pgfqpoint{4.007777in}{2.472244in}}%
\pgfpathlineto{\pgfqpoint{4.008010in}{2.362512in}}%
\pgfpathlineto{\pgfqpoint{4.008243in}{2.386046in}}%
\pgfpathlineto{\pgfqpoint{4.008476in}{2.255672in}}%
\pgfpathlineto{\pgfqpoint{4.008709in}{2.521866in}}%
\pgfpathlineto{\pgfqpoint{4.009406in}{2.267378in}}%
\pgfpathlineto{\pgfqpoint{4.010566in}{2.469406in}}%
\pgfpathlineto{\pgfqpoint{4.010797in}{2.375313in}}%
\pgfpathlineto{\pgfqpoint{4.011029in}{2.394337in}}%
\pgfpathlineto{\pgfqpoint{4.011722in}{2.450015in}}%
\pgfpathlineto{\pgfqpoint{4.012184in}{2.226511in}}%
\pgfpathlineto{\pgfqpoint{4.012876in}{2.441364in}}%
\pgfpathlineto{\pgfqpoint{4.013336in}{2.339922in}}%
\pgfpathlineto{\pgfqpoint{4.013566in}{2.221853in}}%
\pgfpathlineto{\pgfqpoint{4.014026in}{2.425598in}}%
\pgfpathlineto{\pgfqpoint{4.014256in}{2.292380in}}%
\pgfpathlineto{\pgfqpoint{4.014485in}{2.436307in}}%
\pgfpathlineto{\pgfqpoint{4.015403in}{2.345378in}}%
\pgfpathlineto{\pgfqpoint{4.015860in}{1.955412in}}%
\pgfpathlineto{\pgfqpoint{4.016089in}{2.387347in}}%
\pgfpathlineto{\pgfqpoint{4.016546in}{2.273510in}}%
\pgfpathlineto{\pgfqpoint{4.017686in}{2.537277in}}%
\pgfpathlineto{\pgfqpoint{4.017003in}{2.215565in}}%
\pgfpathlineto{\pgfqpoint{4.017914in}{2.463205in}}%
\pgfpathlineto{\pgfqpoint{4.018824in}{2.530276in}}%
\pgfpathlineto{\pgfqpoint{4.019278in}{2.191292in}}%
\pgfpathlineto{\pgfqpoint{4.020185in}{2.513339in}}%
\pgfpathlineto{\pgfqpoint{4.020411in}{2.424421in}}%
\pgfpathlineto{\pgfqpoint{4.020637in}{2.464680in}}%
\pgfpathlineto{\pgfqpoint{4.021090in}{2.411955in}}%
\pgfpathlineto{\pgfqpoint{4.021315in}{2.440279in}}%
\pgfpathlineto{\pgfqpoint{4.022218in}{2.352971in}}%
\pgfpathlineto{\pgfqpoint{4.022443in}{2.413181in}}%
\pgfpathlineto{\pgfqpoint{4.022668in}{2.457103in}}%
\pgfpathlineto{\pgfqpoint{4.022893in}{2.400013in}}%
\pgfpathlineto{\pgfqpoint{4.023568in}{2.509279in}}%
\pgfpathlineto{\pgfqpoint{4.024241in}{2.275103in}}%
\pgfpathlineto{\pgfqpoint{4.024914in}{2.260512in}}%
\pgfpathlineto{\pgfqpoint{4.025361in}{2.453627in}}%
\pgfpathlineto{\pgfqpoint{4.026032in}{2.194484in}}%
\pgfpathlineto{\pgfqpoint{4.026478in}{2.388944in}}%
\pgfpathlineto{\pgfqpoint{4.026924in}{2.430413in}}%
\pgfpathlineto{\pgfqpoint{4.027147in}{2.357646in}}%
\pgfpathlineto{\pgfqpoint{4.027592in}{2.126965in}}%
\pgfpathlineto{\pgfqpoint{4.028482in}{2.494505in}}%
\pgfpathlineto{\pgfqpoint{4.029591in}{2.295982in}}%
\pgfpathlineto{\pgfqpoint{4.029147in}{2.551720in}}%
\pgfpathlineto{\pgfqpoint{4.029812in}{2.328546in}}%
\pgfpathlineto{\pgfqpoint{4.030033in}{2.325277in}}%
\pgfpathlineto{\pgfqpoint{4.030476in}{2.429113in}}%
\pgfpathlineto{\pgfqpoint{4.031138in}{2.393887in}}%
\pgfpathlineto{\pgfqpoint{4.031359in}{2.146588in}}%
\pgfpathlineto{\pgfqpoint{4.032020in}{2.482224in}}%
\pgfpathlineto{\pgfqpoint{4.032240in}{2.392404in}}%
\pgfpathlineto{\pgfqpoint{4.032460in}{2.391235in}}%
\pgfpathlineto{\pgfqpoint{4.032680in}{2.419242in}}%
\pgfpathlineto{\pgfqpoint{4.032900in}{2.299965in}}%
\pgfpathlineto{\pgfqpoint{4.033559in}{2.465355in}}%
\pgfpathlineto{\pgfqpoint{4.033778in}{2.424268in}}%
\pgfpathlineto{\pgfqpoint{4.033997in}{2.471740in}}%
\pgfpathlineto{\pgfqpoint{4.034216in}{2.411344in}}%
\pgfpathlineto{\pgfqpoint{4.034435in}{2.465521in}}%
\pgfpathlineto{\pgfqpoint{4.035529in}{2.470909in}}%
\pgfpathlineto{\pgfqpoint{4.035747in}{2.181590in}}%
\pgfpathlineto{\pgfqpoint{4.036184in}{2.523208in}}%
\pgfpathlineto{\pgfqpoint{4.036837in}{2.296378in}}%
\pgfpathlineto{\pgfqpoint{4.037055in}{2.289955in}}%
\pgfpathlineto{\pgfqpoint{4.037273in}{2.500936in}}%
\pgfpathlineto{\pgfqpoint{4.038142in}{2.324670in}}%
\pgfpathlineto{\pgfqpoint{4.038792in}{2.256846in}}%
\pgfpathlineto{\pgfqpoint{4.039442in}{2.468544in}}%
\pgfpathlineto{\pgfqpoint{4.039658in}{2.254032in}}%
\pgfpathlineto{\pgfqpoint{4.040523in}{2.361315in}}%
\pgfpathlineto{\pgfqpoint{4.041170in}{2.261891in}}%
\pgfpathlineto{\pgfqpoint{4.041385in}{2.472735in}}%
\pgfpathlineto{\pgfqpoint{4.042031in}{2.494935in}}%
\pgfpathlineto{\pgfqpoint{4.042676in}{2.253502in}}%
\pgfpathlineto{\pgfqpoint{4.043534in}{2.501703in}}%
\pgfpathlineto{\pgfqpoint{4.043962in}{2.489435in}}%
\pgfpathlineto{\pgfqpoint{4.044390in}{2.399490in}}%
\pgfpathlineto{\pgfqpoint{4.045032in}{2.238227in}}%
\pgfpathlineto{\pgfqpoint{4.045245in}{2.366466in}}%
\pgfpathlineto{\pgfqpoint{4.045458in}{2.508689in}}%
\pgfpathlineto{\pgfqpoint{4.045672in}{2.313106in}}%
\pgfpathlineto{\pgfqpoint{4.046098in}{2.372964in}}%
\pgfpathlineto{\pgfqpoint{4.046311in}{2.332726in}}%
\pgfpathlineto{\pgfqpoint{4.046524in}{2.456140in}}%
\pgfpathlineto{\pgfqpoint{4.047162in}{2.376235in}}%
\pgfpathlineto{\pgfqpoint{4.047587in}{2.267427in}}%
\pgfpathlineto{\pgfqpoint{4.048435in}{2.502709in}}%
\pgfpathlineto{\pgfqpoint{4.048647in}{2.422420in}}%
\pgfpathlineto{\pgfqpoint{4.049493in}{2.277715in}}%
\pgfpathlineto{\pgfqpoint{4.049282in}{2.493972in}}%
\pgfpathlineto{\pgfqpoint{4.049915in}{2.331730in}}%
\pgfpathlineto{\pgfqpoint{4.050548in}{2.459739in}}%
\pgfpathlineto{\pgfqpoint{4.050970in}{2.408630in}}%
\pgfpathlineto{\pgfqpoint{4.051180in}{2.382576in}}%
\pgfpathlineto{\pgfqpoint{4.051391in}{2.418990in}}%
\pgfpathlineto{\pgfqpoint{4.051601in}{2.515755in}}%
\pgfpathlineto{\pgfqpoint{4.051811in}{2.339565in}}%
\pgfpathlineto{\pgfqpoint{4.052021in}{2.354581in}}%
\pgfpathlineto{\pgfqpoint{4.052231in}{1.823994in}}%
\pgfpathlineto{\pgfqpoint{4.052651in}{2.488296in}}%
\pgfpathlineto{\pgfqpoint{4.053071in}{2.301616in}}%
\pgfpathlineto{\pgfqpoint{4.053699in}{2.512295in}}%
\pgfpathlineto{\pgfqpoint{4.054326in}{2.450481in}}%
\pgfpathlineto{\pgfqpoint{4.054535in}{2.394027in}}%
\pgfpathlineto{\pgfqpoint{4.055161in}{2.462132in}}%
\pgfpathlineto{\pgfqpoint{4.055578in}{2.401537in}}%
\pgfpathlineto{\pgfqpoint{4.055786in}{2.518383in}}%
\pgfpathlineto{\pgfqpoint{4.056202in}{2.472856in}}%
\pgfpathlineto{\pgfqpoint{4.056618in}{2.476195in}}%
\pgfpathlineto{\pgfqpoint{4.057241in}{2.086224in}}%
\pgfpathlineto{\pgfqpoint{4.058277in}{2.492082in}}%
\pgfpathlineto{\pgfqpoint{4.058484in}{2.426127in}}%
\pgfpathlineto{\pgfqpoint{4.059105in}{2.479287in}}%
\pgfpathlineto{\pgfqpoint{4.058898in}{2.395688in}}%
\pgfpathlineto{\pgfqpoint{4.059311in}{2.464844in}}%
\pgfpathlineto{\pgfqpoint{4.059518in}{2.348624in}}%
\pgfpathlineto{\pgfqpoint{4.060343in}{2.481490in}}%
\pgfpathlineto{\pgfqpoint{4.060548in}{2.514654in}}%
\pgfpathlineto{\pgfqpoint{4.060960in}{2.428696in}}%
\pgfpathlineto{\pgfqpoint{4.061782in}{2.256502in}}%
\pgfpathlineto{\pgfqpoint{4.061987in}{2.439342in}}%
\pgfpathlineto{\pgfqpoint{4.062193in}{2.466734in}}%
\pgfpathlineto{\pgfqpoint{4.062398in}{2.357339in}}%
\pgfpathlineto{\pgfqpoint{4.062603in}{2.338806in}}%
\pgfpathlineto{\pgfqpoint{4.063217in}{2.480774in}}%
\pgfpathlineto{\pgfqpoint{4.063421in}{2.428361in}}%
\pgfpathlineto{\pgfqpoint{4.064239in}{2.562866in}}%
\pgfpathlineto{\pgfqpoint{4.064647in}{1.989596in}}%
\pgfpathlineto{\pgfqpoint{4.065258in}{2.434502in}}%
\pgfpathlineto{\pgfqpoint{4.065868in}{2.402145in}}%
\pgfpathlineto{\pgfqpoint{4.066478in}{2.512472in}}%
\pgfpathlineto{\pgfqpoint{4.066275in}{2.375778in}}%
\pgfpathlineto{\pgfqpoint{4.067087in}{2.474322in}}%
\pgfpathlineto{\pgfqpoint{4.067897in}{2.287014in}}%
\pgfpathlineto{\pgfqpoint{4.067492in}{2.498683in}}%
\pgfpathlineto{\pgfqpoint{4.068706in}{2.346332in}}%
\pgfpathlineto{\pgfqpoint{4.069513in}{2.455907in}}%
\pgfpathlineto{\pgfqpoint{4.069109in}{2.343545in}}%
\pgfpathlineto{\pgfqpoint{4.069916in}{2.408279in}}%
\pgfpathlineto{\pgfqpoint{4.070721in}{2.464420in}}%
\pgfpathlineto{\pgfqpoint{4.070922in}{2.187864in}}%
\pgfpathlineto{\pgfqpoint{4.071524in}{2.456265in}}%
\pgfpathlineto{\pgfqpoint{4.072125in}{2.416668in}}%
\pgfpathlineto{\pgfqpoint{4.072726in}{2.373454in}}%
\pgfpathlineto{\pgfqpoint{4.072926in}{2.453351in}}%
\pgfpathlineto{\pgfqpoint{4.073126in}{2.485852in}}%
\pgfpathlineto{\pgfqpoint{4.073326in}{2.167708in}}%
\pgfpathlineto{\pgfqpoint{4.073526in}{2.524734in}}%
\pgfpathlineto{\pgfqpoint{4.074324in}{2.320442in}}%
\pgfpathlineto{\pgfqpoint{4.074523in}{2.368039in}}%
\pgfpathlineto{\pgfqpoint{4.074921in}{2.206454in}}%
\pgfpathlineto{\pgfqpoint{4.075120in}{2.139121in}}%
\pgfpathlineto{\pgfqpoint{4.075319in}{2.435032in}}%
\pgfpathlineto{\pgfqpoint{4.075518in}{2.466006in}}%
\pgfpathlineto{\pgfqpoint{4.076511in}{2.221541in}}%
\pgfpathlineto{\pgfqpoint{4.076709in}{2.322962in}}%
\pgfpathlineto{\pgfqpoint{4.076907in}{2.331164in}}%
\pgfpathlineto{\pgfqpoint{4.077105in}{2.218065in}}%
\pgfpathlineto{\pgfqpoint{4.077303in}{2.491431in}}%
\pgfpathlineto{\pgfqpoint{4.077897in}{2.232705in}}%
\pgfpathlineto{\pgfqpoint{4.078489in}{2.478977in}}%
\pgfpathlineto{\pgfqpoint{4.078686in}{2.066762in}}%
\pgfpathlineto{\pgfqpoint{4.079081in}{2.403744in}}%
\pgfpathlineto{\pgfqpoint{4.079278in}{2.209012in}}%
\pgfpathlineto{\pgfqpoint{4.079868in}{2.480494in}}%
\pgfpathlineto{\pgfqpoint{4.080065in}{2.409853in}}%
\pgfpathlineto{\pgfqpoint{4.080458in}{2.350083in}}%
\pgfpathlineto{\pgfqpoint{4.080655in}{2.468229in}}%
\pgfpathlineto{\pgfqpoint{4.080851in}{2.228441in}}%
\pgfpathlineto{\pgfqpoint{4.081243in}{2.507172in}}%
\pgfpathlineto{\pgfqpoint{4.081831in}{2.298256in}}%
\pgfpathlineto{\pgfqpoint{4.082418in}{2.461457in}}%
\pgfpathlineto{\pgfqpoint{4.082223in}{2.065233in}}%
\pgfpathlineto{\pgfqpoint{4.083005in}{2.422807in}}%
\pgfpathlineto{\pgfqpoint{4.083200in}{2.312438in}}%
\pgfpathlineto{\pgfqpoint{4.083980in}{2.345420in}}%
\pgfpathlineto{\pgfqpoint{4.084175in}{2.467630in}}%
\pgfpathlineto{\pgfqpoint{4.084369in}{2.263198in}}%
\pgfpathlineto{\pgfqpoint{4.084953in}{2.409514in}}%
\pgfpathlineto{\pgfqpoint{4.085536in}{2.133354in}}%
\pgfpathlineto{\pgfqpoint{4.085342in}{2.466889in}}%
\pgfpathlineto{\pgfqpoint{4.085924in}{2.275464in}}%
\pgfpathlineto{\pgfqpoint{4.086699in}{2.483663in}}%
\pgfpathlineto{\pgfqpoint{4.086312in}{2.236343in}}%
\pgfpathlineto{\pgfqpoint{4.087086in}{2.396898in}}%
\pgfpathlineto{\pgfqpoint{4.088052in}{2.555313in}}%
\pgfpathlineto{\pgfqpoint{4.087666in}{2.247960in}}%
\pgfpathlineto{\pgfqpoint{4.088245in}{2.449187in}}%
\pgfpathlineto{\pgfqpoint{4.089401in}{2.149039in}}%
\pgfpathlineto{\pgfqpoint{4.089593in}{2.422199in}}%
\pgfpathlineto{\pgfqpoint{4.090553in}{2.378865in}}%
\pgfpathlineto{\pgfqpoint{4.091320in}{2.406380in}}%
\pgfpathlineto{\pgfqpoint{4.091512in}{2.240917in}}%
\pgfpathlineto{\pgfqpoint{4.092085in}{2.492818in}}%
\pgfpathlineto{\pgfqpoint{4.092658in}{2.300254in}}%
\pgfpathlineto{\pgfqpoint{4.092849in}{2.273301in}}%
\pgfpathlineto{\pgfqpoint{4.093040in}{2.371570in}}%
\pgfpathlineto{\pgfqpoint{4.093612in}{2.483568in}}%
\pgfpathlineto{\pgfqpoint{4.093421in}{2.275361in}}%
\pgfpathlineto{\pgfqpoint{4.094183in}{2.461269in}}%
\pgfpathlineto{\pgfqpoint{4.095133in}{2.492760in}}%
\pgfpathlineto{\pgfqpoint{4.095323in}{2.196994in}}%
\pgfpathlineto{\pgfqpoint{4.095512in}{2.521616in}}%
\pgfpathlineto{\pgfqpoint{4.096459in}{2.406157in}}%
\pgfpathlineto{\pgfqpoint{4.097027in}{2.482014in}}%
\pgfpathlineto{\pgfqpoint{4.096838in}{2.391124in}}%
\pgfpathlineto{\pgfqpoint{4.097215in}{2.464364in}}%
\pgfpathlineto{\pgfqpoint{4.097404in}{2.330143in}}%
\pgfpathlineto{\pgfqpoint{4.097593in}{2.503710in}}%
\pgfpathlineto{\pgfqpoint{4.098347in}{2.341976in}}%
\pgfpathlineto{\pgfqpoint{4.098724in}{2.426450in}}%
\pgfpathlineto{\pgfqpoint{4.098912in}{2.323442in}}%
\pgfpathlineto{\pgfqpoint{4.099100in}{2.118579in}}%
\pgfpathlineto{\pgfqpoint{4.099288in}{2.444269in}}%
\pgfpathlineto{\pgfqpoint{4.099851in}{2.353049in}}%
\pgfpathlineto{\pgfqpoint{4.100039in}{2.510488in}}%
\pgfpathlineto{\pgfqpoint{4.100226in}{2.255647in}}%
\pgfpathlineto{\pgfqpoint{4.100789in}{2.410061in}}%
\pgfpathlineto{\pgfqpoint{4.100976in}{2.189480in}}%
\pgfpathlineto{\pgfqpoint{4.101911in}{2.234940in}}%
\pgfpathlineto{\pgfqpoint{4.103216in}{2.474121in}}%
\pgfpathlineto{\pgfqpoint{4.103402in}{2.224447in}}%
\pgfpathlineto{\pgfqpoint{4.104332in}{2.339325in}}%
\pgfpathlineto{\pgfqpoint{4.104518in}{2.311140in}}%
\pgfpathlineto{\pgfqpoint{4.104703in}{2.392246in}}%
\pgfpathlineto{\pgfqpoint{4.104889in}{2.521080in}}%
\pgfpathlineto{\pgfqpoint{4.105074in}{2.098077in}}%
\pgfpathlineto{\pgfqpoint{4.105630in}{2.311356in}}%
\pgfpathlineto{\pgfqpoint{4.105815in}{2.290028in}}%
\pgfpathlineto{\pgfqpoint{4.106000in}{2.485753in}}%
\pgfpathlineto{\pgfqpoint{4.106924in}{2.464263in}}%
\pgfpathlineto{\pgfqpoint{4.107293in}{2.477808in}}%
\pgfpathlineto{\pgfqpoint{4.108214in}{2.235483in}}%
\pgfpathlineto{\pgfqpoint{4.108950in}{2.470187in}}%
\pgfpathlineto{\pgfqpoint{4.109317in}{2.230094in}}%
\pgfpathlineto{\pgfqpoint{4.109500in}{2.451120in}}%
\pgfpathlineto{\pgfqpoint{4.110050in}{2.274685in}}%
\pgfpathlineto{\pgfqpoint{4.110234in}{2.542531in}}%
\pgfpathlineto{\pgfqpoint{4.111148in}{2.396183in}}%
\pgfpathlineto{\pgfqpoint{4.111696in}{2.474444in}}%
\pgfpathlineto{\pgfqpoint{4.111879in}{2.350567in}}%
\pgfpathlineto{\pgfqpoint{4.112244in}{2.420973in}}%
\pgfpathlineto{\pgfqpoint{4.112972in}{2.155640in}}%
\pgfpathlineto{\pgfqpoint{4.113699in}{2.469815in}}%
\pgfpathlineto{\pgfqpoint{4.114425in}{2.458151in}}%
\pgfpathlineto{\pgfqpoint{4.115150in}{2.252355in}}%
\pgfpathlineto{\pgfqpoint{4.114969in}{2.514557in}}%
\pgfpathlineto{\pgfqpoint{4.115512in}{2.474319in}}%
\pgfpathlineto{\pgfqpoint{4.116054in}{2.476405in}}%
\pgfpathlineto{\pgfqpoint{4.117497in}{2.084705in}}%
\pgfpathlineto{\pgfqpoint{4.118576in}{2.484446in}}%
\pgfpathlineto{\pgfqpoint{4.118756in}{2.426292in}}%
\pgfpathlineto{\pgfqpoint{4.119115in}{2.491494in}}%
\pgfpathlineto{\pgfqpoint{4.120368in}{2.209756in}}%
\pgfpathlineto{\pgfqpoint{4.121083in}{2.434343in}}%
\pgfpathlineto{\pgfqpoint{4.121618in}{2.426838in}}%
\pgfpathlineto{\pgfqpoint{4.121796in}{2.427576in}}%
\pgfpathlineto{\pgfqpoint{4.122687in}{2.223125in}}%
\pgfpathlineto{\pgfqpoint{4.122331in}{2.451593in}}%
\pgfpathlineto{\pgfqpoint{4.122864in}{2.438017in}}%
\pgfpathlineto{\pgfqpoint{4.123042in}{2.436056in}}%
\pgfpathlineto{\pgfqpoint{4.123575in}{2.495734in}}%
\pgfpathlineto{\pgfqpoint{4.123397in}{2.279495in}}%
\pgfpathlineto{\pgfqpoint{4.124107in}{2.457496in}}%
\pgfpathlineto{\pgfqpoint{4.124284in}{2.185490in}}%
\pgfpathlineto{\pgfqpoint{4.125346in}{2.229544in}}%
\pgfpathlineto{\pgfqpoint{4.125523in}{2.137705in}}%
\pgfpathlineto{\pgfqpoint{4.126053in}{2.348443in}}%
\pgfpathlineto{\pgfqpoint{4.126934in}{2.515994in}}%
\pgfpathlineto{\pgfqpoint{4.126758in}{2.330440in}}%
\pgfpathlineto{\pgfqpoint{4.127286in}{2.420268in}}%
\pgfpathlineto{\pgfqpoint{4.127813in}{2.256451in}}%
\pgfpathlineto{\pgfqpoint{4.128340in}{2.364335in}}%
\pgfpathlineto{\pgfqpoint{4.128866in}{2.451894in}}%
\pgfpathlineto{\pgfqpoint{4.129217in}{2.309468in}}%
\pgfpathlineto{\pgfqpoint{4.129392in}{2.373695in}}%
\pgfpathlineto{\pgfqpoint{4.129917in}{2.068342in}}%
\pgfpathlineto{\pgfqpoint{4.129742in}{2.434064in}}%
\pgfpathlineto{\pgfqpoint{4.130266in}{2.104245in}}%
\pgfpathlineto{\pgfqpoint{4.131313in}{2.490085in}}%
\pgfpathlineto{\pgfqpoint{4.131488in}{2.400516in}}%
\pgfpathlineto{\pgfqpoint{4.132184in}{2.265918in}}%
\pgfpathlineto{\pgfqpoint{4.132531in}{2.423450in}}%
\pgfpathlineto{\pgfqpoint{4.132705in}{2.439343in}}%
\pgfpathlineto{\pgfqpoint{4.132879in}{2.246557in}}%
\pgfpathlineto{\pgfqpoint{4.133399in}{2.522409in}}%
\pgfpathlineto{\pgfqpoint{4.133746in}{2.468292in}}%
\pgfpathlineto{\pgfqpoint{4.134093in}{2.292985in}}%
\pgfpathlineto{\pgfqpoint{4.134785in}{2.371669in}}%
\pgfpathlineto{\pgfqpoint{4.134957in}{2.467248in}}%
\pgfpathlineto{\pgfqpoint{4.135303in}{2.058998in}}%
\pgfpathlineto{\pgfqpoint{4.135648in}{2.418767in}}%
\pgfpathlineto{\pgfqpoint{4.135820in}{2.140976in}}%
\pgfpathlineto{\pgfqpoint{4.136682in}{2.501979in}}%
\pgfpathlineto{\pgfqpoint{4.137370in}{2.284383in}}%
\pgfpathlineto{\pgfqpoint{4.138056in}{2.314914in}}%
\pgfpathlineto{\pgfqpoint{4.138228in}{2.501552in}}%
\pgfpathlineto{\pgfqpoint{4.139084in}{2.281250in}}%
\pgfpathlineto{\pgfqpoint{4.139597in}{2.521898in}}%
\pgfpathlineto{\pgfqpoint{4.139426in}{2.210589in}}%
\pgfpathlineto{\pgfqpoint{4.140621in}{2.478913in}}%
\pgfpathlineto{\pgfqpoint{4.141473in}{2.288884in}}%
\pgfpathlineto{\pgfqpoint{4.141303in}{2.481902in}}%
\pgfpathlineto{\pgfqpoint{4.141813in}{2.410174in}}%
\pgfpathlineto{\pgfqpoint{4.141983in}{2.499078in}}%
\pgfpathlineto{\pgfqpoint{4.142493in}{2.307568in}}%
\pgfpathlineto{\pgfqpoint{4.143002in}{2.471714in}}%
\pgfpathlineto{\pgfqpoint{4.143510in}{2.248449in}}%
\pgfpathlineto{\pgfqpoint{4.143679in}{2.496619in}}%
\pgfpathlineto{\pgfqpoint{4.144356in}{2.305913in}}%
\pgfpathlineto{\pgfqpoint{4.144525in}{2.529210in}}%
\pgfpathlineto{\pgfqpoint{4.145537in}{2.401089in}}%
\pgfpathlineto{\pgfqpoint{4.145706in}{2.358719in}}%
\pgfpathlineto{\pgfqpoint{4.145874in}{2.469249in}}%
\pgfpathlineto{\pgfqpoint{4.146042in}{2.566022in}}%
\pgfpathlineto{\pgfqpoint{4.146379in}{2.273726in}}%
\pgfpathlineto{\pgfqpoint{4.146883in}{2.532873in}}%
\pgfpathlineto{\pgfqpoint{4.147051in}{2.211102in}}%
\pgfpathlineto{\pgfqpoint{4.148058in}{2.396141in}}%
\pgfpathlineto{\pgfqpoint{4.148393in}{2.253223in}}%
\pgfpathlineto{\pgfqpoint{4.148894in}{2.358921in}}%
\pgfpathlineto{\pgfqpoint{4.149229in}{2.242285in}}%
\pgfpathlineto{\pgfqpoint{4.150063in}{2.471580in}}%
\pgfpathlineto{\pgfqpoint{4.150896in}{2.167819in}}%
\pgfpathlineto{\pgfqpoint{4.150563in}{2.523114in}}%
\pgfpathlineto{\pgfqpoint{4.151229in}{2.310779in}}%
\pgfpathlineto{\pgfqpoint{4.151561in}{2.215786in}}%
\pgfpathlineto{\pgfqpoint{4.152225in}{2.497583in}}%
\pgfpathlineto{\pgfqpoint{4.152391in}{2.309098in}}%
\pgfpathlineto{\pgfqpoint{4.152723in}{2.501341in}}%
\pgfpathlineto{\pgfqpoint{4.153220in}{2.433401in}}%
\pgfpathlineto{\pgfqpoint{4.153385in}{2.456281in}}%
\pgfpathlineto{\pgfqpoint{4.153551in}{2.383720in}}%
\pgfpathlineto{\pgfqpoint{4.153881in}{2.432537in}}%
\pgfpathlineto{\pgfqpoint{4.154707in}{2.167809in}}%
\pgfpathlineto{\pgfqpoint{4.154377in}{2.516021in}}%
\pgfpathlineto{\pgfqpoint{4.154871in}{2.271695in}}%
\pgfpathlineto{\pgfqpoint{4.155695in}{2.471824in}}%
\pgfpathlineto{\pgfqpoint{4.156024in}{2.334739in}}%
\pgfpathlineto{\pgfqpoint{4.156188in}{2.300521in}}%
\pgfpathlineto{\pgfqpoint{4.156353in}{2.443680in}}%
\pgfpathlineto{\pgfqpoint{4.156517in}{2.411383in}}%
\pgfpathlineto{\pgfqpoint{4.157173in}{2.541629in}}%
\pgfpathlineto{\pgfqpoint{4.156845in}{2.307010in}}%
\pgfpathlineto{\pgfqpoint{4.157337in}{2.414201in}}%
\pgfpathlineto{\pgfqpoint{4.157829in}{2.196327in}}%
\pgfpathlineto{\pgfqpoint{4.158156in}{2.471972in}}%
\pgfpathlineto{\pgfqpoint{4.158320in}{2.241203in}}%
\pgfpathlineto{\pgfqpoint{4.158483in}{2.425037in}}%
\pgfpathlineto{\pgfqpoint{4.159463in}{2.393926in}}%
\pgfpathlineto{\pgfqpoint{4.159626in}{2.389474in}}%
\pgfpathlineto{\pgfqpoint{4.159789in}{2.447745in}}%
\pgfpathlineto{\pgfqpoint{4.160277in}{2.342188in}}%
\pgfpathlineto{\pgfqpoint{4.160440in}{2.021627in}}%
\pgfpathlineto{\pgfqpoint{4.160928in}{2.479122in}}%
\pgfpathlineto{\pgfqpoint{4.161253in}{2.356153in}}%
\pgfpathlineto{\pgfqpoint{4.161415in}{2.345137in}}%
\pgfpathlineto{\pgfqpoint{4.161578in}{2.475787in}}%
\pgfpathlineto{\pgfqpoint{4.162388in}{2.465999in}}%
\pgfpathlineto{\pgfqpoint{4.163197in}{2.239632in}}%
\pgfpathlineto{\pgfqpoint{4.163359in}{2.441929in}}%
\pgfpathlineto{\pgfqpoint{4.163521in}{2.451453in}}%
\pgfpathlineto{\pgfqpoint{4.163682in}{2.416979in}}%
\pgfpathlineto{\pgfqpoint{4.164489in}{2.244787in}}%
\pgfpathlineto{\pgfqpoint{4.164650in}{2.485662in}}%
\pgfpathlineto{\pgfqpoint{4.164811in}{2.293952in}}%
\pgfpathlineto{\pgfqpoint{4.165776in}{2.482527in}}%
\pgfpathlineto{\pgfqpoint{4.165133in}{2.275736in}}%
\pgfpathlineto{\pgfqpoint{4.165937in}{2.404430in}}%
\pgfpathlineto{\pgfqpoint{4.166258in}{2.440762in}}%
\pgfpathlineto{\pgfqpoint{4.167060in}{2.243556in}}%
\pgfpathlineto{\pgfqpoint{4.168180in}{2.481550in}}%
\pgfpathlineto{\pgfqpoint{4.168818in}{2.505033in}}%
\pgfpathlineto{\pgfqpoint{4.169296in}{2.098534in}}%
\pgfpathlineto{\pgfqpoint{4.169774in}{2.467851in}}%
\pgfpathlineto{\pgfqpoint{4.169615in}{1.953451in}}%
\pgfpathlineto{\pgfqpoint{4.170411in}{2.178216in}}%
\pgfpathlineto{\pgfqpoint{4.170569in}{2.083184in}}%
\pgfpathlineto{\pgfqpoint{4.171046in}{2.424251in}}%
\pgfpathlineto{\pgfqpoint{4.171680in}{2.466071in}}%
\pgfpathlineto{\pgfqpoint{4.171363in}{2.383472in}}%
\pgfpathlineto{\pgfqpoint{4.171839in}{2.392929in}}%
\pgfpathlineto{\pgfqpoint{4.172155in}{2.350062in}}%
\pgfpathlineto{\pgfqpoint{4.172472in}{2.526419in}}%
\pgfpathlineto{\pgfqpoint{4.172788in}{2.224071in}}%
\pgfpathlineto{\pgfqpoint{4.173262in}{2.378969in}}%
\pgfpathlineto{\pgfqpoint{4.173735in}{2.203360in}}%
\pgfpathlineto{\pgfqpoint{4.173893in}{2.472703in}}%
\pgfpathlineto{\pgfqpoint{4.174050in}{2.356800in}}%
\pgfpathlineto{\pgfqpoint{4.174208in}{2.468407in}}%
\pgfpathlineto{\pgfqpoint{4.174365in}{2.270698in}}%
\pgfpathlineto{\pgfqpoint{4.175152in}{2.457459in}}%
\pgfpathlineto{\pgfqpoint{4.175309in}{2.450081in}}%
\pgfpathlineto{\pgfqpoint{4.175466in}{2.452584in}}%
\pgfpathlineto{\pgfqpoint{4.176251in}{2.191853in}}%
\pgfpathlineto{\pgfqpoint{4.176094in}{2.483827in}}%
\pgfpathlineto{\pgfqpoint{4.176408in}{2.425941in}}%
\pgfpathlineto{\pgfqpoint{4.176564in}{2.512351in}}%
\pgfpathlineto{\pgfqpoint{4.176877in}{2.213271in}}%
\pgfpathlineto{\pgfqpoint{4.177347in}{2.444123in}}%
\pgfpathlineto{\pgfqpoint{4.178284in}{2.258357in}}%
\pgfpathlineto{\pgfqpoint{4.178440in}{2.441773in}}%
\pgfpathlineto{\pgfqpoint{4.178907in}{2.349215in}}%
\pgfpathlineto{\pgfqpoint{4.179219in}{2.426494in}}%
\pgfpathlineto{\pgfqpoint{4.179375in}{2.493156in}}%
\pgfpathlineto{\pgfqpoint{4.179530in}{2.115627in}}%
\pgfpathlineto{\pgfqpoint{4.179686in}{2.449893in}}%
\pgfpathlineto{\pgfqpoint{4.179841in}{2.214159in}}%
\pgfpathlineto{\pgfqpoint{4.180463in}{2.501763in}}%
\pgfpathlineto{\pgfqpoint{4.180773in}{2.372479in}}%
\pgfpathlineto{\pgfqpoint{4.181702in}{2.536691in}}%
\pgfpathlineto{\pgfqpoint{4.181083in}{2.347403in}}%
\pgfpathlineto{\pgfqpoint{4.181857in}{2.416668in}}%
\pgfpathlineto{\pgfqpoint{4.182012in}{2.416573in}}%
\pgfpathlineto{\pgfqpoint{4.182784in}{2.187767in}}%
\pgfpathlineto{\pgfqpoint{4.182321in}{2.424573in}}%
\pgfpathlineto{\pgfqpoint{4.183093in}{2.410758in}}%
\pgfpathlineto{\pgfqpoint{4.183401in}{2.429880in}}%
\pgfpathlineto{\pgfqpoint{4.183556in}{2.392434in}}%
\pgfpathlineto{\pgfqpoint{4.183710in}{2.263252in}}%
\pgfpathlineto{\pgfqpoint{4.184171in}{2.458918in}}%
\pgfpathlineto{\pgfqpoint{4.184633in}{2.354912in}}%
\pgfpathlineto{\pgfqpoint{4.185401in}{2.458606in}}%
\pgfpathlineto{\pgfqpoint{4.185247in}{2.289494in}}%
\pgfpathlineto{\pgfqpoint{4.185707in}{2.420670in}}%
\pgfpathlineto{\pgfqpoint{4.186167in}{2.524256in}}%
\pgfpathlineto{\pgfqpoint{4.186932in}{2.271569in}}%
\pgfpathlineto{\pgfqpoint{4.187085in}{2.487998in}}%
\pgfpathlineto{\pgfqpoint{4.188001in}{2.419365in}}%
\pgfpathlineto{\pgfqpoint{4.189371in}{2.257773in}}%
\pgfpathlineto{\pgfqpoint{4.190585in}{2.507134in}}%
\pgfpathlineto{\pgfqpoint{4.191342in}{2.312247in}}%
\pgfpathlineto{\pgfqpoint{4.191645in}{2.469273in}}%
\pgfpathlineto{\pgfqpoint{4.191796in}{2.468777in}}%
\pgfpathlineto{\pgfqpoint{4.191947in}{2.491586in}}%
\pgfpathlineto{\pgfqpoint{4.192098in}{2.394651in}}%
\pgfpathlineto{\pgfqpoint{4.192249in}{2.186748in}}%
\pgfpathlineto{\pgfqpoint{4.193154in}{2.432402in}}%
\pgfpathlineto{\pgfqpoint{4.193304in}{2.351338in}}%
\pgfpathlineto{\pgfqpoint{4.193606in}{2.459270in}}%
\pgfpathlineto{\pgfqpoint{4.194057in}{2.379613in}}%
\pgfpathlineto{\pgfqpoint{4.194207in}{2.520437in}}%
\pgfpathlineto{\pgfqpoint{4.194808in}{2.316344in}}%
\pgfpathlineto{\pgfqpoint{4.195108in}{2.461266in}}%
\pgfpathlineto{\pgfqpoint{4.196006in}{2.152405in}}%
\pgfpathlineto{\pgfqpoint{4.195857in}{2.484072in}}%
\pgfpathlineto{\pgfqpoint{4.196306in}{2.300502in}}%
\pgfpathlineto{\pgfqpoint{4.196455in}{2.489353in}}%
\pgfpathlineto{\pgfqpoint{4.197053in}{2.187449in}}%
\pgfpathlineto{\pgfqpoint{4.197351in}{2.402228in}}%
\pgfpathlineto{\pgfqpoint{4.197798in}{2.492950in}}%
\pgfpathlineto{\pgfqpoint{4.198394in}{2.132706in}}%
\pgfpathlineto{\pgfqpoint{4.199583in}{2.522344in}}%
\pgfpathlineto{\pgfqpoint{4.200472in}{2.234035in}}%
\pgfpathlineto{\pgfqpoint{4.200324in}{2.548964in}}%
\pgfpathlineto{\pgfqpoint{4.200768in}{2.370019in}}%
\pgfpathlineto{\pgfqpoint{4.201064in}{2.470371in}}%
\pgfpathlineto{\pgfqpoint{4.201212in}{2.286229in}}%
\pgfpathlineto{\pgfqpoint{4.201802in}{2.333747in}}%
\pgfpathlineto{\pgfqpoint{4.201950in}{2.078968in}}%
\pgfpathlineto{\pgfqpoint{4.202687in}{2.489090in}}%
\pgfpathlineto{\pgfqpoint{4.202834in}{2.293286in}}%
\pgfpathlineto{\pgfqpoint{4.202982in}{2.539920in}}%
\pgfpathlineto{\pgfqpoint{4.203717in}{2.276185in}}%
\pgfpathlineto{\pgfqpoint{4.204011in}{2.416172in}}%
\pgfpathlineto{\pgfqpoint{4.204744in}{2.247118in}}%
\pgfpathlineto{\pgfqpoint{4.205330in}{2.516591in}}%
\pgfpathlineto{\pgfqpoint{4.205769in}{2.329280in}}%
\pgfpathlineto{\pgfqpoint{4.205915in}{2.167478in}}%
\pgfpathlineto{\pgfqpoint{4.206354in}{2.490428in}}%
\pgfpathlineto{\pgfqpoint{4.206792in}{2.305143in}}%
\pgfpathlineto{\pgfqpoint{4.207811in}{2.459281in}}%
\pgfpathlineto{\pgfqpoint{4.207957in}{2.434818in}}%
\pgfpathlineto{\pgfqpoint{4.208248in}{2.181439in}}%
\pgfpathlineto{\pgfqpoint{4.208829in}{2.459622in}}%
\pgfpathlineto{\pgfqpoint{4.209264in}{2.347605in}}%
\pgfpathlineto{\pgfqpoint{4.210134in}{2.561771in}}%
\pgfpathlineto{\pgfqpoint{4.209844in}{2.324477in}}%
\pgfpathlineto{\pgfqpoint{4.210568in}{2.461947in}}%
\pgfpathlineto{\pgfqpoint{4.210857in}{2.229900in}}%
\pgfpathlineto{\pgfqpoint{4.211578in}{2.334662in}}%
\pgfpathlineto{\pgfqpoint{4.211723in}{2.462349in}}%
\pgfpathlineto{\pgfqpoint{4.211867in}{2.268532in}}%
\pgfpathlineto{\pgfqpoint{4.212587in}{2.393812in}}%
\pgfpathlineto{\pgfqpoint{4.213306in}{2.145275in}}%
\pgfpathlineto{\pgfqpoint{4.212875in}{2.471959in}}%
\pgfpathlineto{\pgfqpoint{4.213737in}{2.311211in}}%
\pgfpathlineto{\pgfqpoint{4.214167in}{2.540078in}}%
\pgfpathlineto{\pgfqpoint{4.214597in}{2.290014in}}%
\pgfpathlineto{\pgfqpoint{4.214883in}{2.419029in}}%
\pgfpathlineto{\pgfqpoint{4.215169in}{2.334936in}}%
\pgfpathlineto{\pgfqpoint{4.215312in}{2.504602in}}%
\pgfpathlineto{\pgfqpoint{4.215455in}{2.260522in}}%
\pgfpathlineto{\pgfqpoint{4.216312in}{2.520749in}}%
\pgfpathlineto{\pgfqpoint{4.216455in}{2.385255in}}%
\pgfpathlineto{\pgfqpoint{4.216597in}{2.228693in}}%
\pgfpathlineto{\pgfqpoint{4.216882in}{2.484521in}}%
\pgfpathlineto{\pgfqpoint{4.217452in}{2.330852in}}%
\pgfpathlineto{\pgfqpoint{4.217736in}{2.376129in}}%
\pgfpathlineto{\pgfqpoint{4.217878in}{2.101734in}}%
\pgfpathlineto{\pgfqpoint{4.218304in}{2.473493in}}%
\pgfpathlineto{\pgfqpoint{4.218730in}{2.354423in}}%
\pgfpathlineto{\pgfqpoint{4.219439in}{2.505519in}}%
\pgfpathlineto{\pgfqpoint{4.219014in}{2.117951in}}%
\pgfpathlineto{\pgfqpoint{4.219580in}{2.314681in}}%
\pgfpathlineto{\pgfqpoint{4.219722in}{2.193976in}}%
\pgfpathlineto{\pgfqpoint{4.220005in}{2.458637in}}%
\pgfpathlineto{\pgfqpoint{4.220570in}{2.406632in}}%
\pgfpathlineto{\pgfqpoint{4.220852in}{2.240017in}}%
\pgfpathlineto{\pgfqpoint{4.221134in}{2.528802in}}%
\pgfpathlineto{\pgfqpoint{4.221557in}{2.386229in}}%
\pgfpathlineto{\pgfqpoint{4.221839in}{2.425796in}}%
\pgfpathlineto{\pgfqpoint{4.222120in}{2.322948in}}%
\pgfpathlineto{\pgfqpoint{4.222261in}{2.294833in}}%
\pgfpathlineto{\pgfqpoint{4.222402in}{2.422343in}}%
\pgfpathlineto{\pgfqpoint{4.222542in}{2.505241in}}%
\pgfpathlineto{\pgfqpoint{4.223104in}{2.255301in}}%
\pgfpathlineto{\pgfqpoint{4.223525in}{2.440946in}}%
\pgfpathlineto{\pgfqpoint{4.223665in}{2.421984in}}%
\pgfpathlineto{\pgfqpoint{4.223805in}{2.436415in}}%
\pgfpathlineto{\pgfqpoint{4.223946in}{2.535820in}}%
\pgfpathlineto{\pgfqpoint{4.224506in}{2.304293in}}%
\pgfpathlineto{\pgfqpoint{4.224785in}{2.355663in}}%
\pgfpathlineto{\pgfqpoint{4.225065in}{2.465153in}}%
\pgfpathlineto{\pgfqpoint{4.225344in}{2.206783in}}%
\pgfpathlineto{\pgfqpoint{4.225484in}{2.337417in}}%
\pgfpathlineto{\pgfqpoint{4.225623in}{2.162050in}}%
\pgfpathlineto{\pgfqpoint{4.226042in}{2.500145in}}%
\pgfpathlineto{\pgfqpoint{4.226460in}{2.324096in}}%
\pgfpathlineto{\pgfqpoint{4.227156in}{2.458041in}}%
\pgfpathlineto{\pgfqpoint{4.227295in}{2.372841in}}%
\pgfpathlineto{\pgfqpoint{4.227434in}{2.560078in}}%
\pgfpathlineto{\pgfqpoint{4.227712in}{2.271951in}}%
\pgfpathlineto{\pgfqpoint{4.228405in}{2.501947in}}%
\pgfpathlineto{\pgfqpoint{4.228544in}{2.143363in}}%
\pgfpathlineto{\pgfqpoint{4.229513in}{2.430120in}}%
\pgfpathlineto{\pgfqpoint{4.230066in}{2.274534in}}%
\pgfpathlineto{\pgfqpoint{4.229928in}{2.469383in}}%
\pgfpathlineto{\pgfqpoint{4.230342in}{2.388971in}}%
\pgfpathlineto{\pgfqpoint{4.230480in}{2.521802in}}%
\pgfpathlineto{\pgfqpoint{4.231031in}{2.320849in}}%
\pgfpathlineto{\pgfqpoint{4.231444in}{2.423207in}}%
\pgfpathlineto{\pgfqpoint{4.231582in}{2.465488in}}%
\pgfpathlineto{\pgfqpoint{4.231995in}{2.347309in}}%
\pgfpathlineto{\pgfqpoint{4.232269in}{2.388712in}}%
\pgfpathlineto{\pgfqpoint{4.232819in}{2.459537in}}%
\pgfpathlineto{\pgfqpoint{4.233504in}{2.202968in}}%
\pgfpathlineto{\pgfqpoint{4.234462in}{2.472779in}}%
\pgfpathlineto{\pgfqpoint{4.234735in}{2.376219in}}%
\pgfpathlineto{\pgfqpoint{4.234872in}{2.201722in}}%
\pgfpathlineto{\pgfqpoint{4.235554in}{2.427927in}}%
\pgfpathlineto{\pgfqpoint{4.235826in}{2.221214in}}%
\pgfpathlineto{\pgfqpoint{4.236099in}{2.510047in}}%
\pgfpathlineto{\pgfqpoint{4.236915in}{2.378125in}}%
\pgfpathlineto{\pgfqpoint{4.237186in}{2.460637in}}%
\pgfpathlineto{\pgfqpoint{4.238000in}{2.221679in}}%
\pgfpathlineto{\pgfqpoint{4.238678in}{2.478007in}}%
\pgfpathlineto{\pgfqpoint{4.239083in}{2.363350in}}%
\pgfpathlineto{\pgfqpoint{4.239219in}{2.269715in}}%
\pgfpathlineto{\pgfqpoint{4.239624in}{2.484329in}}%
\pgfpathlineto{\pgfqpoint{4.239894in}{2.399639in}}%
\pgfpathlineto{\pgfqpoint{4.240029in}{2.492076in}}%
\pgfpathlineto{\pgfqpoint{4.240433in}{2.289485in}}%
\pgfpathlineto{\pgfqpoint{4.240972in}{2.434314in}}%
\pgfpathlineto{\pgfqpoint{4.241241in}{2.468537in}}%
\pgfpathlineto{\pgfqpoint{4.241779in}{2.236925in}}%
\pgfpathlineto{\pgfqpoint{4.242450in}{2.311264in}}%
\pgfpathlineto{\pgfqpoint{4.242986in}{2.448562in}}%
\pgfpathlineto{\pgfqpoint{4.243388in}{1.964388in}}%
\pgfpathlineto{\pgfqpoint{4.244190in}{2.491379in}}%
\pgfpathlineto{\pgfqpoint{4.244590in}{2.451243in}}%
\pgfpathlineto{\pgfqpoint{4.244857in}{2.209167in}}%
\pgfpathlineto{\pgfqpoint{4.245390in}{2.475128in}}%
\pgfpathlineto{\pgfqpoint{4.245656in}{2.297615in}}%
\pgfpathlineto{\pgfqpoint{4.246322in}{2.413037in}}%
\pgfpathlineto{\pgfqpoint{4.246056in}{2.209270in}}%
\pgfpathlineto{\pgfqpoint{4.246587in}{2.378219in}}%
\pgfpathlineto{\pgfqpoint{4.246720in}{2.203990in}}%
\pgfpathlineto{\pgfqpoint{4.246853in}{2.522329in}}%
\pgfpathlineto{\pgfqpoint{4.247649in}{2.422198in}}%
\pgfpathlineto{\pgfqpoint{4.248046in}{2.351661in}}%
\pgfpathlineto{\pgfqpoint{4.248310in}{2.487135in}}%
\pgfpathlineto{\pgfqpoint{4.248707in}{2.414707in}}%
\pgfpathlineto{\pgfqpoint{4.248839in}{2.417083in}}%
\pgfpathlineto{\pgfqpoint{4.249631in}{2.278195in}}%
\pgfpathlineto{\pgfqpoint{4.249499in}{2.478390in}}%
\pgfpathlineto{\pgfqpoint{4.250027in}{2.290081in}}%
\pgfpathlineto{\pgfqpoint{4.250422in}{2.527908in}}%
\pgfpathlineto{\pgfqpoint{4.251211in}{2.491415in}}%
\pgfpathlineto{\pgfqpoint{4.251342in}{2.212809in}}%
\pgfpathlineto{\pgfqpoint{4.252260in}{2.278187in}}%
\pgfpathlineto{\pgfqpoint{4.252523in}{2.495446in}}%
\pgfpathlineto{\pgfqpoint{4.253438in}{2.390888in}}%
\pgfpathlineto{\pgfqpoint{4.253961in}{2.158577in}}%
\pgfpathlineto{\pgfqpoint{4.254222in}{2.468443in}}%
\pgfpathlineto{\pgfqpoint{4.254613in}{2.316693in}}%
\pgfpathlineto{\pgfqpoint{4.254743in}{2.528927in}}%
\pgfpathlineto{\pgfqpoint{4.255394in}{2.283674in}}%
\pgfpathlineto{\pgfqpoint{4.255654in}{2.332065in}}%
\pgfpathlineto{\pgfqpoint{4.255784in}{2.359913in}}%
\pgfpathlineto{\pgfqpoint{4.255914in}{2.097031in}}%
\pgfpathlineto{\pgfqpoint{4.256434in}{2.453919in}}%
\pgfpathlineto{\pgfqpoint{4.256823in}{2.411602in}}%
\pgfpathlineto{\pgfqpoint{4.257471in}{2.008172in}}%
\pgfpathlineto{\pgfqpoint{4.257341in}{2.448556in}}%
\pgfpathlineto{\pgfqpoint{4.257859in}{2.439830in}}%
\pgfpathlineto{\pgfqpoint{4.257988in}{2.466890in}}%
\pgfpathlineto{\pgfqpoint{4.258117in}{2.308547in}}%
\pgfpathlineto{\pgfqpoint{4.258247in}{2.364864in}}%
\pgfpathlineto{\pgfqpoint{4.258376in}{2.272062in}}%
\pgfpathlineto{\pgfqpoint{4.258892in}{2.436878in}}%
\pgfpathlineto{\pgfqpoint{4.259021in}{2.430966in}}%
\pgfpathlineto{\pgfqpoint{4.259150in}{2.521657in}}%
\pgfpathlineto{\pgfqpoint{4.259537in}{2.111935in}}%
\pgfpathlineto{\pgfqpoint{4.260052in}{2.436671in}}%
\pgfpathlineto{\pgfqpoint{4.260181in}{2.443807in}}%
\pgfpathlineto{\pgfqpoint{4.260438in}{2.417268in}}%
\pgfpathlineto{\pgfqpoint{4.260566in}{2.441054in}}%
\pgfpathlineto{\pgfqpoint{4.261337in}{2.456324in}}%
\pgfpathlineto{\pgfqpoint{4.261721in}{2.225609in}}%
\pgfpathlineto{\pgfqpoint{4.262234in}{2.482936in}}%
\pgfpathlineto{\pgfqpoint{4.262106in}{2.204894in}}%
\pgfpathlineto{\pgfqpoint{4.263001in}{2.397338in}}%
\pgfpathlineto{\pgfqpoint{4.263384in}{2.347731in}}%
\pgfpathlineto{\pgfqpoint{4.263512in}{2.533864in}}%
\pgfpathlineto{\pgfqpoint{4.264786in}{2.187454in}}%
\pgfpathlineto{\pgfqpoint{4.265676in}{2.500923in}}%
\pgfpathlineto{\pgfqpoint{4.265930in}{2.434914in}}%
\pgfpathlineto{\pgfqpoint{4.266057in}{2.417156in}}%
\pgfpathlineto{\pgfqpoint{4.266184in}{2.180504in}}%
\pgfpathlineto{\pgfqpoint{4.266437in}{2.490059in}}%
\pgfpathlineto{\pgfqpoint{4.267071in}{2.274068in}}%
\pgfpathlineto{\pgfqpoint{4.267450in}{2.455263in}}%
\pgfpathlineto{\pgfqpoint{4.267956in}{2.202023in}}%
\pgfpathlineto{\pgfqpoint{4.268208in}{2.316484in}}%
\pgfpathlineto{\pgfqpoint{4.268334in}{2.192275in}}%
\pgfpathlineto{\pgfqpoint{4.269091in}{2.495554in}}%
\pgfpathlineto{\pgfqpoint{4.269217in}{2.339130in}}%
\pgfpathlineto{\pgfqpoint{4.269343in}{2.300650in}}%
\pgfpathlineto{\pgfqpoint{4.270097in}{2.408889in}}%
\pgfpathlineto{\pgfqpoint{4.270474in}{2.526104in}}%
\pgfpathlineto{\pgfqpoint{4.270976in}{2.394034in}}%
\pgfpathlineto{\pgfqpoint{4.271101in}{2.286181in}}%
\pgfpathlineto{\pgfqpoint{4.271853in}{2.502216in}}%
\pgfpathlineto{\pgfqpoint{4.271978in}{2.403431in}}%
\pgfpathlineto{\pgfqpoint{4.272103in}{2.407857in}}%
\pgfpathlineto{\pgfqpoint{4.272228in}{2.401414in}}%
\pgfpathlineto{\pgfqpoint{4.273102in}{2.204506in}}%
\pgfpathlineto{\pgfqpoint{4.272603in}{2.502553in}}%
\pgfpathlineto{\pgfqpoint{4.273227in}{2.408864in}}%
\pgfpathlineto{\pgfqpoint{4.273726in}{2.516475in}}%
\pgfpathlineto{\pgfqpoint{4.273850in}{2.280328in}}%
\pgfpathlineto{\pgfqpoint{4.274473in}{2.456592in}}%
\pgfpathlineto{\pgfqpoint{4.274846in}{2.342952in}}%
\pgfpathlineto{\pgfqpoint{4.275218in}{2.489696in}}%
\pgfpathlineto{\pgfqpoint{4.275715in}{2.356947in}}%
\pgfpathlineto{\pgfqpoint{4.275839in}{2.459774in}}%
\pgfpathlineto{\pgfqpoint{4.276706in}{2.310806in}}%
\pgfpathlineto{\pgfqpoint{4.276829in}{2.088323in}}%
\pgfpathlineto{\pgfqpoint{4.277571in}{2.461371in}}%
\pgfpathlineto{\pgfqpoint{4.277818in}{2.282925in}}%
\pgfpathlineto{\pgfqpoint{4.278557in}{2.435049in}}%
\pgfpathlineto{\pgfqpoint{4.278434in}{2.264571in}}%
\pgfpathlineto{\pgfqpoint{4.278927in}{2.410685in}}%
\pgfpathlineto{\pgfqpoint{4.279665in}{2.468060in}}%
\pgfpathlineto{\pgfqpoint{4.280156in}{2.271038in}}%
\pgfpathlineto{\pgfqpoint{4.281014in}{2.486333in}}%
\pgfpathlineto{\pgfqpoint{4.280524in}{2.060725in}}%
\pgfpathlineto{\pgfqpoint{4.281259in}{2.296642in}}%
\pgfpathlineto{\pgfqpoint{4.281381in}{2.201565in}}%
\pgfpathlineto{\pgfqpoint{4.281504in}{2.434308in}}%
\pgfpathlineto{\pgfqpoint{4.282115in}{2.272874in}}%
\pgfpathlineto{\pgfqpoint{4.282725in}{2.513256in}}%
\pgfpathlineto{\pgfqpoint{4.282603in}{2.259043in}}%
\pgfpathlineto{\pgfqpoint{4.283335in}{2.505771in}}%
\pgfpathlineto{\pgfqpoint{4.283700in}{2.144276in}}%
\pgfpathlineto{\pgfqpoint{4.284429in}{2.371135in}}%
\pgfpathlineto{\pgfqpoint{4.285158in}{2.468812in}}%
\pgfpathlineto{\pgfqpoint{4.285036in}{2.204252in}}%
\pgfpathlineto{\pgfqpoint{4.285279in}{2.320994in}}%
\pgfpathlineto{\pgfqpoint{4.285400in}{2.293832in}}%
\pgfpathlineto{\pgfqpoint{4.285521in}{2.473187in}}%
\pgfpathlineto{\pgfqpoint{4.285764in}{2.392818in}}%
\pgfpathlineto{\pgfqpoint{4.285885in}{2.510104in}}%
\pgfpathlineto{\pgfqpoint{4.286006in}{2.114301in}}%
\pgfpathlineto{\pgfqpoint{4.286852in}{2.388786in}}%
\pgfpathlineto{\pgfqpoint{4.287697in}{2.071626in}}%
\pgfpathlineto{\pgfqpoint{4.287576in}{2.497050in}}%
\pgfpathlineto{\pgfqpoint{4.287938in}{2.414835in}}%
\pgfpathlineto{\pgfqpoint{4.288901in}{2.224585in}}%
\pgfpathlineto{\pgfqpoint{4.288660in}{2.438584in}}%
\pgfpathlineto{\pgfqpoint{4.289021in}{2.266450in}}%
\pgfpathlineto{\pgfqpoint{4.289261in}{2.512063in}}%
\pgfpathlineto{\pgfqpoint{4.289381in}{2.152553in}}%
\pgfpathlineto{\pgfqpoint{4.290101in}{2.408054in}}%
\pgfpathlineto{\pgfqpoint{4.290221in}{2.242664in}}%
\pgfpathlineto{\pgfqpoint{4.291179in}{2.317973in}}%
\pgfpathlineto{\pgfqpoint{4.291657in}{2.510721in}}%
\pgfpathlineto{\pgfqpoint{4.292015in}{2.252704in}}%
\pgfpathlineto{\pgfqpoint{4.292254in}{2.445757in}}%
\pgfpathlineto{\pgfqpoint{4.292849in}{2.186586in}}%
\pgfpathlineto{\pgfqpoint{4.292730in}{2.482312in}}%
\pgfpathlineto{\pgfqpoint{4.293326in}{2.285083in}}%
\pgfpathlineto{\pgfqpoint{4.294395in}{2.470349in}}%
\pgfpathlineto{\pgfqpoint{4.293801in}{2.193275in}}%
\pgfpathlineto{\pgfqpoint{4.294514in}{2.418337in}}%
\pgfpathlineto{\pgfqpoint{4.295580in}{2.188254in}}%
\pgfpathlineto{\pgfqpoint{4.294869in}{2.527351in}}%
\pgfpathlineto{\pgfqpoint{4.295935in}{2.255894in}}%
\pgfpathlineto{\pgfqpoint{4.296762in}{2.494623in}}%
\pgfpathlineto{\pgfqpoint{4.297116in}{2.379667in}}%
\pgfpathlineto{\pgfqpoint{4.297234in}{2.139728in}}%
\pgfpathlineto{\pgfqpoint{4.297823in}{2.520088in}}%
\pgfpathlineto{\pgfqpoint{4.298058in}{2.370467in}}%
\pgfpathlineto{\pgfqpoint{4.298176in}{2.511603in}}%
\pgfpathlineto{\pgfqpoint{4.298528in}{2.269373in}}%
\pgfpathlineto{\pgfqpoint{4.298998in}{2.379948in}}%
\pgfpathlineto{\pgfqpoint{4.299819in}{2.042357in}}%
\pgfpathlineto{\pgfqpoint{4.300053in}{2.387382in}}%
\pgfpathlineto{\pgfqpoint{4.300404in}{2.484917in}}%
\pgfpathlineto{\pgfqpoint{4.300755in}{2.367025in}}%
\pgfpathlineto{\pgfqpoint{4.300872in}{2.251832in}}%
\pgfpathlineto{\pgfqpoint{4.300989in}{2.463638in}}%
\pgfpathlineto{\pgfqpoint{4.301923in}{2.299313in}}%
\pgfpathlineto{\pgfqpoint{4.302505in}{2.525575in}}%
\pgfpathlineto{\pgfqpoint{4.302854in}{2.317348in}}%
\pgfpathlineto{\pgfqpoint{4.302971in}{2.252645in}}%
\pgfpathlineto{\pgfqpoint{4.303087in}{2.512231in}}%
\pgfpathlineto{\pgfqpoint{4.303784in}{2.344518in}}%
\pgfpathlineto{\pgfqpoint{4.304480in}{2.483169in}}%
\pgfpathlineto{\pgfqpoint{4.304248in}{2.283412in}}%
\pgfpathlineto{\pgfqpoint{4.304596in}{2.480293in}}%
\pgfpathlineto{\pgfqpoint{4.305406in}{2.127541in}}%
\pgfpathlineto{\pgfqpoint{4.305521in}{2.544462in}}%
\pgfpathlineto{\pgfqpoint{4.305753in}{2.375889in}}%
\pgfpathlineto{\pgfqpoint{4.305868in}{2.342676in}}%
\pgfpathlineto{\pgfqpoint{4.305984in}{2.512826in}}%
\pgfpathlineto{\pgfqpoint{4.306676in}{2.401012in}}%
\pgfpathlineto{\pgfqpoint{4.307022in}{2.472609in}}%
\pgfpathlineto{\pgfqpoint{4.307137in}{2.376127in}}%
\pgfpathlineto{\pgfqpoint{4.307252in}{2.215374in}}%
\pgfpathlineto{\pgfqpoint{4.307942in}{2.482798in}}%
\pgfpathlineto{\pgfqpoint{4.308172in}{2.262015in}}%
\pgfpathlineto{\pgfqpoint{4.308861in}{2.489400in}}%
\pgfpathlineto{\pgfqpoint{4.309319in}{2.422768in}}%
\pgfpathlineto{\pgfqpoint{4.309548in}{2.468168in}}%
\pgfpathlineto{\pgfqpoint{4.310463in}{2.253687in}}%
\pgfpathlineto{\pgfqpoint{4.310692in}{2.491123in}}%
\pgfpathlineto{\pgfqpoint{4.311034in}{2.239529in}}%
\pgfpathlineto{\pgfqpoint{4.311604in}{2.438193in}}%
\pgfpathlineto{\pgfqpoint{4.311718in}{2.314136in}}%
\pgfpathlineto{\pgfqpoint{4.312060in}{2.449584in}}%
\pgfpathlineto{\pgfqpoint{4.312629in}{2.422965in}}%
\pgfpathlineto{\pgfqpoint{4.313424in}{2.485006in}}%
\pgfpathlineto{\pgfqpoint{4.312970in}{2.169152in}}%
\pgfpathlineto{\pgfqpoint{4.313537in}{2.375383in}}%
\pgfpathlineto{\pgfqpoint{4.314217in}{2.216698in}}%
\pgfpathlineto{\pgfqpoint{4.314104in}{2.472168in}}%
\pgfpathlineto{\pgfqpoint{4.314331in}{2.338678in}}%
\pgfpathlineto{\pgfqpoint{4.315235in}{2.495436in}}%
\pgfpathlineto{\pgfqpoint{4.314670in}{2.283886in}}%
\pgfpathlineto{\pgfqpoint{4.315461in}{2.383195in}}%
\pgfpathlineto{\pgfqpoint{4.315800in}{2.361902in}}%
\pgfpathlineto{\pgfqpoint{4.315687in}{2.444762in}}%
\pgfpathlineto{\pgfqpoint{4.316138in}{2.393825in}}%
\pgfpathlineto{\pgfqpoint{4.316702in}{2.521613in}}%
\pgfpathlineto{\pgfqpoint{4.316927in}{2.298804in}}%
\pgfpathlineto{\pgfqpoint{4.317039in}{2.458832in}}%
\pgfpathlineto{\pgfqpoint{4.317152in}{2.292858in}}%
\pgfpathlineto{\pgfqpoint{4.317714in}{2.497765in}}%
\pgfpathlineto{\pgfqpoint{4.318163in}{2.376605in}}%
\pgfpathlineto{\pgfqpoint{4.318947in}{2.309274in}}%
\pgfpathlineto{\pgfqpoint{4.318499in}{2.427148in}}%
\pgfpathlineto{\pgfqpoint{4.319059in}{2.347148in}}%
\pgfpathlineto{\pgfqpoint{4.320178in}{2.520594in}}%
\pgfpathlineto{\pgfqpoint{4.319731in}{2.331836in}}%
\pgfpathlineto{\pgfqpoint{4.320289in}{2.446472in}}%
\pgfpathlineto{\pgfqpoint{4.320513in}{2.476203in}}%
\pgfpathlineto{\pgfqpoint{4.320624in}{2.463678in}}%
\pgfpathlineto{\pgfqpoint{4.321182in}{2.242796in}}%
\pgfpathlineto{\pgfqpoint{4.321738in}{2.318163in}}%
\pgfpathlineto{\pgfqpoint{4.322294in}{2.522886in}}%
\pgfpathlineto{\pgfqpoint{4.322738in}{2.290680in}}%
\pgfpathlineto{\pgfqpoint{4.322960in}{2.520670in}}%
\pgfpathlineto{\pgfqpoint{4.323736in}{2.280435in}}%
\pgfpathlineto{\pgfqpoint{4.324290in}{2.366776in}}%
\pgfpathlineto{\pgfqpoint{4.324732in}{2.493731in}}%
\pgfpathlineto{\pgfqpoint{4.325174in}{2.472720in}}%
\pgfpathlineto{\pgfqpoint{4.325284in}{2.273759in}}%
\pgfpathlineto{\pgfqpoint{4.326166in}{2.486484in}}%
\pgfpathlineto{\pgfqpoint{4.326276in}{2.457720in}}%
\pgfpathlineto{\pgfqpoint{4.327265in}{2.086648in}}%
\pgfpathlineto{\pgfqpoint{4.326936in}{2.516280in}}%
\pgfpathlineto{\pgfqpoint{4.327485in}{2.382142in}}%
\pgfpathlineto{\pgfqpoint{4.328362in}{2.527025in}}%
\pgfpathlineto{\pgfqpoint{4.327924in}{2.244310in}}%
\pgfpathlineto{\pgfqpoint{4.328581in}{2.396339in}}%
\pgfpathlineto{\pgfqpoint{4.328691in}{2.398808in}}%
\pgfpathlineto{\pgfqpoint{4.328800in}{2.484967in}}%
\pgfpathlineto{\pgfqpoint{4.329019in}{2.294344in}}%
\pgfpathlineto{\pgfqpoint{4.329675in}{2.358274in}}%
\pgfpathlineto{\pgfqpoint{4.330547in}{2.144396in}}%
\pgfpathlineto{\pgfqpoint{4.330220in}{2.487836in}}%
\pgfpathlineto{\pgfqpoint{4.330656in}{2.222657in}}%
\pgfpathlineto{\pgfqpoint{4.331744in}{2.503770in}}%
\pgfpathlineto{\pgfqpoint{4.331418in}{2.170404in}}%
\pgfpathlineto{\pgfqpoint{4.331853in}{2.489211in}}%
\pgfpathlineto{\pgfqpoint{4.332179in}{2.253940in}}%
\pgfpathlineto{\pgfqpoint{4.332721in}{2.514732in}}%
\pgfpathlineto{\pgfqpoint{4.332938in}{2.429688in}}%
\pgfpathlineto{\pgfqpoint{4.333046in}{2.509403in}}%
\pgfpathlineto{\pgfqpoint{4.333588in}{2.180194in}}%
\pgfpathlineto{\pgfqpoint{4.333912in}{2.379828in}}%
\pgfpathlineto{\pgfqpoint{4.334237in}{2.255095in}}%
\pgfpathlineto{\pgfqpoint{4.334345in}{2.446482in}}%
\pgfpathlineto{\pgfqpoint{4.334668in}{2.365293in}}%
\pgfpathlineto{\pgfqpoint{4.335531in}{2.497950in}}%
\pgfpathlineto{\pgfqpoint{4.335208in}{2.198485in}}%
\pgfpathlineto{\pgfqpoint{4.335746in}{2.452422in}}%
\pgfpathlineto{\pgfqpoint{4.336284in}{2.219655in}}%
\pgfpathlineto{\pgfqpoint{4.336499in}{2.519029in}}%
\pgfpathlineto{\pgfqpoint{4.337036in}{2.240485in}}%
\pgfpathlineto{\pgfqpoint{4.337358in}{2.504800in}}%
\pgfpathlineto{\pgfqpoint{4.338429in}{2.409150in}}%
\pgfpathlineto{\pgfqpoint{4.338750in}{2.522709in}}%
\pgfpathlineto{\pgfqpoint{4.339498in}{2.249613in}}%
\pgfpathlineto{\pgfqpoint{4.340563in}{2.510382in}}%
\pgfpathlineto{\pgfqpoint{4.340670in}{2.344658in}}%
\pgfpathlineto{\pgfqpoint{4.341414in}{2.473404in}}%
\pgfpathlineto{\pgfqpoint{4.340882in}{2.052761in}}%
\pgfpathlineto{\pgfqpoint{4.341732in}{2.351213in}}%
\pgfpathlineto{\pgfqpoint{4.341839in}{2.262703in}}%
\pgfpathlineto{\pgfqpoint{4.342581in}{2.460405in}}%
\pgfpathlineto{\pgfqpoint{4.342793in}{2.343738in}}%
\pgfpathlineto{\pgfqpoint{4.343004in}{2.511139in}}%
\pgfpathlineto{\pgfqpoint{4.343639in}{2.315274in}}%
\pgfpathlineto{\pgfqpoint{4.343956in}{2.438575in}}%
\pgfpathlineto{\pgfqpoint{4.344799in}{2.270444in}}%
\pgfpathlineto{\pgfqpoint{4.344167in}{2.541934in}}%
\pgfpathlineto{\pgfqpoint{4.345115in}{2.394642in}}%
\pgfpathlineto{\pgfqpoint{4.345852in}{2.514047in}}%
\pgfpathlineto{\pgfqpoint{4.345957in}{2.263065in}}%
\pgfpathlineto{\pgfqpoint{4.346272in}{2.461462in}}%
\pgfpathlineto{\pgfqpoint{4.347531in}{2.262498in}}%
\pgfpathlineto{\pgfqpoint{4.347740in}{2.343415in}}%
\pgfpathlineto{\pgfqpoint{4.348158in}{2.494737in}}%
\pgfpathlineto{\pgfqpoint{4.348263in}{2.280575in}}%
\pgfpathlineto{\pgfqpoint{4.348785in}{2.432250in}}%
\pgfpathlineto{\pgfqpoint{4.349515in}{2.255808in}}%
\pgfpathlineto{\pgfqpoint{4.349202in}{2.484888in}}%
\pgfpathlineto{\pgfqpoint{4.349932in}{2.355014in}}%
\pgfpathlineto{\pgfqpoint{4.350556in}{2.493723in}}%
\pgfpathlineto{\pgfqpoint{4.350244in}{2.234535in}}%
\pgfpathlineto{\pgfqpoint{4.350660in}{2.475418in}}%
\pgfpathlineto{\pgfqpoint{4.351387in}{2.140908in}}%
\pgfpathlineto{\pgfqpoint{4.350972in}{2.503045in}}%
\pgfpathlineto{\pgfqpoint{4.351802in}{2.395286in}}%
\pgfpathlineto{\pgfqpoint{4.352113in}{2.361854in}}%
\pgfpathlineto{\pgfqpoint{4.352216in}{2.497244in}}%
\pgfpathlineto{\pgfqpoint{4.352320in}{2.526578in}}%
\pgfpathlineto{\pgfqpoint{4.352527in}{2.360675in}}%
\pgfpathlineto{\pgfqpoint{4.353457in}{1.977530in}}%
\pgfpathlineto{\pgfqpoint{4.353147in}{2.464862in}}%
\pgfpathlineto{\pgfqpoint{4.353560in}{2.409063in}}%
\pgfpathlineto{\pgfqpoint{4.353870in}{2.486970in}}%
\pgfpathlineto{\pgfqpoint{4.353973in}{2.459818in}}%
\pgfpathlineto{\pgfqpoint{4.354076in}{2.090077in}}%
\pgfpathlineto{\pgfqpoint{4.354591in}{2.519684in}}%
\pgfpathlineto{\pgfqpoint{4.355003in}{2.449574in}}%
\pgfpathlineto{\pgfqpoint{4.356030in}{2.277446in}}%
\pgfpathlineto{\pgfqpoint{4.355311in}{2.499756in}}%
\pgfpathlineto{\pgfqpoint{4.356133in}{2.416390in}}%
\pgfpathlineto{\pgfqpoint{4.356646in}{2.501719in}}%
\pgfpathlineto{\pgfqpoint{4.357056in}{2.343372in}}%
\pgfpathlineto{\pgfqpoint{4.357158in}{2.452892in}}%
\pgfpathlineto{\pgfqpoint{4.357772in}{2.507307in}}%
\pgfpathlineto{\pgfqpoint{4.358282in}{1.981806in}}%
\pgfpathlineto{\pgfqpoint{4.358384in}{2.484456in}}%
\pgfpathlineto{\pgfqpoint{4.359404in}{2.278518in}}%
\pgfpathlineto{\pgfqpoint{4.359607in}{2.464477in}}%
\pgfpathlineto{\pgfqpoint{4.359811in}{1.998459in}}%
\pgfpathlineto{\pgfqpoint{4.360624in}{2.476501in}}%
\pgfpathlineto{\pgfqpoint{4.360928in}{2.078023in}}%
\pgfpathlineto{\pgfqpoint{4.361536in}{2.499571in}}%
\pgfpathlineto{\pgfqpoint{4.362043in}{2.470961in}}%
\pgfpathlineto{\pgfqpoint{4.362750in}{2.196641in}}%
\pgfpathlineto{\pgfqpoint{4.362346in}{2.551197in}}%
\pgfpathlineto{\pgfqpoint{4.363154in}{2.418390in}}%
\pgfpathlineto{\pgfqpoint{4.363255in}{2.415147in}}%
\pgfpathlineto{\pgfqpoint{4.364263in}{2.519814in}}%
\pgfpathlineto{\pgfqpoint{4.364364in}{2.116538in}}%
\pgfpathlineto{\pgfqpoint{4.364866in}{2.460062in}}%
\pgfpathlineto{\pgfqpoint{4.365168in}{2.068923in}}%
\pgfpathlineto{\pgfqpoint{4.365569in}{2.378960in}}%
\pgfpathlineto{\pgfqpoint{4.365670in}{2.268966in}}%
\pgfpathlineto{\pgfqpoint{4.365971in}{2.457138in}}%
\pgfpathlineto{\pgfqpoint{4.366672in}{2.358064in}}%
\pgfpathlineto{\pgfqpoint{4.366772in}{2.348811in}}%
\pgfpathlineto{\pgfqpoint{4.366872in}{2.370012in}}%
\pgfpathlineto{\pgfqpoint{4.366972in}{2.504847in}}%
\pgfpathlineto{\pgfqpoint{4.367671in}{2.320351in}}%
\pgfpathlineto{\pgfqpoint{4.367971in}{2.402706in}}%
\pgfpathlineto{\pgfqpoint{4.368470in}{2.190547in}}%
\pgfpathlineto{\pgfqpoint{4.368569in}{2.507169in}}%
\pgfpathlineto{\pgfqpoint{4.368968in}{2.424597in}}%
\pgfpathlineto{\pgfqpoint{4.369067in}{2.439788in}}%
\pgfpathlineto{\pgfqpoint{4.369167in}{2.380526in}}%
\pgfpathlineto{\pgfqpoint{4.369266in}{1.959730in}}%
\pgfpathlineto{\pgfqpoint{4.369465in}{2.445193in}}%
\pgfpathlineto{\pgfqpoint{4.370260in}{2.374076in}}%
\pgfpathlineto{\pgfqpoint{4.370458in}{2.158960in}}%
\pgfpathlineto{\pgfqpoint{4.370954in}{2.470689in}}%
\pgfpathlineto{\pgfqpoint{4.371350in}{2.372485in}}%
\pgfpathlineto{\pgfqpoint{4.372536in}{2.502484in}}%
\pgfpathlineto{\pgfqpoint{4.371943in}{2.218009in}}%
\pgfpathlineto{\pgfqpoint{4.372635in}{2.446756in}}%
\pgfpathlineto{\pgfqpoint{4.373522in}{2.178893in}}%
\pgfpathlineto{\pgfqpoint{4.373029in}{2.483036in}}%
\pgfpathlineto{\pgfqpoint{4.373719in}{2.447879in}}%
\pgfpathlineto{\pgfqpoint{4.373817in}{2.468017in}}%
\pgfpathlineto{\pgfqpoint{4.373916in}{2.329495in}}%
\pgfpathlineto{\pgfqpoint{4.374014in}{2.414444in}}%
\pgfpathlineto{\pgfqpoint{4.374506in}{2.176587in}}%
\pgfpathlineto{\pgfqpoint{4.374899in}{2.512225in}}%
\pgfpathlineto{\pgfqpoint{4.375095in}{2.436545in}}%
\pgfpathlineto{\pgfqpoint{4.375193in}{2.455708in}}%
\pgfpathlineto{\pgfqpoint{4.375585in}{2.345700in}}%
\pgfpathlineto{\pgfqpoint{4.376075in}{2.239822in}}%
\pgfpathlineto{\pgfqpoint{4.375977in}{2.515316in}}%
\pgfpathlineto{\pgfqpoint{4.376662in}{2.298513in}}%
\pgfpathlineto{\pgfqpoint{4.376955in}{2.475874in}}%
\pgfpathlineto{\pgfqpoint{4.377248in}{2.259455in}}%
\pgfpathlineto{\pgfqpoint{4.377736in}{2.382167in}}%
\pgfpathlineto{\pgfqpoint{4.377833in}{2.093302in}}%
\pgfpathlineto{\pgfqpoint{4.378028in}{2.514205in}}%
\pgfpathlineto{\pgfqpoint{4.378807in}{2.331429in}}%
\pgfpathlineto{\pgfqpoint{4.379779in}{2.467081in}}%
\pgfpathlineto{\pgfqpoint{4.379487in}{2.266202in}}%
\pgfpathlineto{\pgfqpoint{4.379876in}{2.334686in}}%
\pgfpathlineto{\pgfqpoint{4.380748in}{2.428236in}}%
\pgfpathlineto{\pgfqpoint{4.380457in}{2.233513in}}%
\pgfpathlineto{\pgfqpoint{4.380942in}{2.420488in}}%
\pgfpathlineto{\pgfqpoint{4.381812in}{2.504711in}}%
\pgfpathlineto{\pgfqpoint{4.382005in}{2.249807in}}%
\pgfpathlineto{\pgfqpoint{4.383162in}{2.516893in}}%
\pgfpathlineto{\pgfqpoint{4.384123in}{2.268036in}}%
\pgfpathlineto{\pgfqpoint{4.384507in}{2.315702in}}%
\pgfpathlineto{\pgfqpoint{4.384987in}{2.469205in}}%
\pgfpathlineto{\pgfqpoint{4.385657in}{2.466410in}}%
\pgfpathlineto{\pgfqpoint{4.386422in}{2.226237in}}%
\pgfpathlineto{\pgfqpoint{4.386804in}{2.324960in}}%
\pgfpathlineto{\pgfqpoint{4.386900in}{2.507059in}}%
\pgfpathlineto{\pgfqpoint{4.387567in}{2.259353in}}%
\pgfpathlineto{\pgfqpoint{4.387948in}{2.405630in}}%
\pgfpathlineto{\pgfqpoint{4.388614in}{2.099755in}}%
\pgfpathlineto{\pgfqpoint{4.388519in}{2.487926in}}%
\pgfpathlineto{\pgfqpoint{4.388899in}{2.356277in}}%
\pgfpathlineto{\pgfqpoint{4.389374in}{2.481178in}}%
\pgfpathlineto{\pgfqpoint{4.389848in}{2.257373in}}%
\pgfpathlineto{\pgfqpoint{4.390037in}{2.446774in}}%
\pgfpathlineto{\pgfqpoint{4.390794in}{2.184985in}}%
\pgfpathlineto{\pgfqpoint{4.390321in}{2.502631in}}%
\pgfpathlineto{\pgfqpoint{4.390983in}{2.320732in}}%
\pgfpathlineto{\pgfqpoint{4.391361in}{2.550339in}}%
\pgfpathlineto{\pgfqpoint{4.391927in}{2.291706in}}%
\pgfpathlineto{\pgfqpoint{4.392116in}{2.381049in}}%
\pgfpathlineto{\pgfqpoint{4.392399in}{2.268452in}}%
\pgfpathlineto{\pgfqpoint{4.392493in}{2.405874in}}%
\pgfpathlineto{\pgfqpoint{4.393152in}{2.361176in}}%
\pgfpathlineto{\pgfqpoint{4.393715in}{2.531805in}}%
\pgfpathlineto{\pgfqpoint{4.394185in}{2.437451in}}%
\pgfpathlineto{\pgfqpoint{4.394372in}{2.130080in}}%
\pgfpathlineto{\pgfqpoint{4.394653in}{2.437566in}}%
\pgfpathlineto{\pgfqpoint{4.395215in}{2.260772in}}%
\pgfpathlineto{\pgfqpoint{4.396243in}{2.539827in}}%
\pgfpathlineto{\pgfqpoint{4.396337in}{2.462889in}}%
\pgfpathlineto{\pgfqpoint{4.396803in}{2.208400in}}%
\pgfpathlineto{\pgfqpoint{4.396989in}{2.509293in}}%
\pgfpathlineto{\pgfqpoint{4.397455in}{2.259772in}}%
\pgfpathlineto{\pgfqpoint{4.398756in}{2.514490in}}%
\pgfpathlineto{\pgfqpoint{4.400146in}{2.222659in}}%
\pgfpathlineto{\pgfqpoint{4.400331in}{2.483651in}}%
\pgfpathlineto{\pgfqpoint{4.401162in}{2.224698in}}%
\pgfpathlineto{\pgfqpoint{4.401254in}{2.166144in}}%
\pgfpathlineto{\pgfqpoint{4.401347in}{2.508535in}}%
\pgfpathlineto{\pgfqpoint{4.401808in}{2.484482in}}%
\pgfpathlineto{\pgfqpoint{4.401900in}{2.524201in}}%
\pgfpathlineto{\pgfqpoint{4.402360in}{2.275117in}}%
\pgfpathlineto{\pgfqpoint{4.402452in}{2.440563in}}%
\pgfpathlineto{\pgfqpoint{4.403095in}{2.486554in}}%
\pgfpathlineto{\pgfqpoint{4.403554in}{2.237923in}}%
\pgfpathlineto{\pgfqpoint{4.404013in}{2.500315in}}%
\pgfpathlineto{\pgfqpoint{4.404196in}{2.194972in}}%
\pgfpathlineto{\pgfqpoint{4.404654in}{2.451151in}}%
\pgfpathlineto{\pgfqpoint{4.404837in}{2.064199in}}%
\pgfpathlineto{\pgfqpoint{4.405568in}{2.517483in}}%
\pgfpathlineto{\pgfqpoint{4.405751in}{2.426463in}}%
\pgfpathlineto{\pgfqpoint{4.405842in}{2.466819in}}%
\pgfpathlineto{\pgfqpoint{4.406207in}{2.194553in}}%
\pgfpathlineto{\pgfqpoint{4.406572in}{2.443856in}}%
\pgfpathlineto{\pgfqpoint{4.406663in}{2.195572in}}%
\pgfpathlineto{\pgfqpoint{4.407391in}{2.532497in}}%
\pgfpathlineto{\pgfqpoint{4.407663in}{2.321587in}}%
\pgfpathlineto{\pgfqpoint{4.408299in}{2.139868in}}%
\pgfpathlineto{\pgfqpoint{4.408843in}{2.514309in}}%
\pgfpathlineto{\pgfqpoint{4.409567in}{2.523248in}}%
\pgfpathlineto{\pgfqpoint{4.410200in}{2.183336in}}%
\pgfpathlineto{\pgfqpoint{4.411103in}{2.479746in}}%
\pgfpathlineto{\pgfqpoint{4.411373in}{2.394528in}}%
\pgfpathlineto{\pgfqpoint{4.411463in}{2.382144in}}%
\pgfpathlineto{\pgfqpoint{4.411553in}{2.444049in}}%
\pgfpathlineto{\pgfqpoint{4.412003in}{2.545576in}}%
\pgfpathlineto{\pgfqpoint{4.412363in}{2.255534in}}%
\pgfpathlineto{\pgfqpoint{4.412542in}{2.426588in}}%
\pgfpathlineto{\pgfqpoint{4.412812in}{2.041196in}}%
\pgfpathlineto{\pgfqpoint{4.413350in}{2.458977in}}%
\pgfpathlineto{\pgfqpoint{4.413708in}{2.245143in}}%
\pgfpathlineto{\pgfqpoint{4.413888in}{2.501769in}}%
\pgfpathlineto{\pgfqpoint{4.414603in}{2.208905in}}%
\pgfpathlineto{\pgfqpoint{4.414871in}{2.414674in}}%
\pgfpathlineto{\pgfqpoint{4.415407in}{2.247399in}}%
\pgfpathlineto{\pgfqpoint{4.415050in}{2.531761in}}%
\pgfpathlineto{\pgfqpoint{4.415853in}{2.306838in}}%
\pgfpathlineto{\pgfqpoint{4.415942in}{2.535257in}}%
\pgfpathlineto{\pgfqpoint{4.416921in}{2.403553in}}%
\pgfpathlineto{\pgfqpoint{4.417010in}{2.235172in}}%
\pgfpathlineto{\pgfqpoint{4.417898in}{2.445631in}}%
\pgfpathlineto{\pgfqpoint{4.417987in}{2.416207in}}%
\pgfpathlineto{\pgfqpoint{4.418253in}{2.251429in}}%
\pgfpathlineto{\pgfqpoint{4.418873in}{2.496240in}}%
\pgfpathlineto{\pgfqpoint{4.419138in}{2.367175in}}%
\pgfpathlineto{\pgfqpoint{4.419227in}{2.381883in}}%
\pgfpathlineto{\pgfqpoint{4.419315in}{2.281930in}}%
\pgfpathlineto{\pgfqpoint{4.419845in}{2.323978in}}%
\pgfpathlineto{\pgfqpoint{4.419933in}{2.252982in}}%
\pgfpathlineto{\pgfqpoint{4.420286in}{2.464017in}}%
\pgfpathlineto{\pgfqpoint{4.420815in}{2.299807in}}%
\pgfpathlineto{\pgfqpoint{4.421256in}{2.519719in}}%
\pgfpathlineto{\pgfqpoint{4.421695in}{2.114804in}}%
\pgfpathlineto{\pgfqpoint{4.421959in}{2.444720in}}%
\pgfpathlineto{\pgfqpoint{4.422661in}{2.182697in}}%
\pgfpathlineto{\pgfqpoint{4.422749in}{2.500177in}}%
\pgfpathlineto{\pgfqpoint{4.423012in}{2.423187in}}%
\pgfpathlineto{\pgfqpoint{4.423100in}{2.468939in}}%
\pgfpathlineto{\pgfqpoint{4.423275in}{2.258594in}}%
\pgfpathlineto{\pgfqpoint{4.424063in}{2.465467in}}%
\pgfpathlineto{\pgfqpoint{4.425023in}{2.244416in}}%
\pgfpathlineto{\pgfqpoint{4.424237in}{2.531862in}}%
\pgfpathlineto{\pgfqpoint{4.425198in}{2.400697in}}%
\pgfpathlineto{\pgfqpoint{4.425285in}{2.394263in}}%
\pgfpathlineto{\pgfqpoint{4.425372in}{2.459061in}}%
\pgfpathlineto{\pgfqpoint{4.425633in}{2.165928in}}%
\pgfpathlineto{\pgfqpoint{4.426416in}{2.336915in}}%
\pgfpathlineto{\pgfqpoint{4.426851in}{2.090498in}}%
\pgfpathlineto{\pgfqpoint{4.427458in}{2.507354in}}%
\pgfpathlineto{\pgfqpoint{4.428584in}{2.171850in}}%
\pgfpathlineto{\pgfqpoint{4.429190in}{2.552024in}}%
\pgfpathlineto{\pgfqpoint{4.429794in}{2.403715in}}%
\pgfpathlineto{\pgfqpoint{4.430569in}{2.216145in}}%
\pgfpathlineto{\pgfqpoint{4.430052in}{2.521013in}}%
\pgfpathlineto{\pgfqpoint{4.430827in}{2.477272in}}%
\pgfpathlineto{\pgfqpoint{4.431086in}{2.430133in}}%
\pgfpathlineto{\pgfqpoint{4.431172in}{2.525743in}}%
\pgfpathlineto{\pgfqpoint{4.431859in}{2.200798in}}%
\pgfpathlineto{\pgfqpoint{4.431945in}{2.526864in}}%
\pgfpathlineto{\pgfqpoint{4.432288in}{2.392238in}}%
\pgfpathlineto{\pgfqpoint{4.433315in}{2.248060in}}%
\pgfpathlineto{\pgfqpoint{4.432973in}{2.450792in}}%
\pgfpathlineto{\pgfqpoint{4.433572in}{2.288916in}}%
\pgfpathlineto{\pgfqpoint{4.434085in}{2.523504in}}%
\pgfpathlineto{\pgfqpoint{4.434426in}{2.255203in}}%
\pgfpathlineto{\pgfqpoint{4.434767in}{2.421315in}}%
\pgfpathlineto{\pgfqpoint{4.435449in}{2.200707in}}%
\pgfpathlineto{\pgfqpoint{4.435108in}{2.506494in}}%
\pgfpathlineto{\pgfqpoint{4.435874in}{2.220209in}}%
\pgfpathlineto{\pgfqpoint{4.435959in}{2.474665in}}%
\pgfpathlineto{\pgfqpoint{4.436638in}{2.134771in}}%
\pgfpathlineto{\pgfqpoint{4.437063in}{2.462618in}}%
\pgfpathlineto{\pgfqpoint{4.437994in}{1.739905in}}%
\pgfpathlineto{\pgfqpoint{4.438163in}{2.408598in}}%
\pgfpathlineto{\pgfqpoint{4.438755in}{2.475833in}}%
\pgfpathlineto{\pgfqpoint{4.439008in}{2.250028in}}%
\pgfpathlineto{\pgfqpoint{4.439683in}{2.513543in}}%
\pgfpathlineto{\pgfqpoint{4.439177in}{2.110255in}}%
\pgfpathlineto{\pgfqpoint{4.440104in}{2.300247in}}%
\pgfpathlineto{\pgfqpoint{4.440356in}{2.422483in}}%
\pgfpathlineto{\pgfqpoint{4.440693in}{2.099032in}}%
\pgfpathlineto{\pgfqpoint{4.441113in}{2.410817in}}%
\pgfpathlineto{\pgfqpoint{4.441197in}{1.989419in}}%
\pgfpathlineto{\pgfqpoint{4.441365in}{2.486538in}}%
\pgfpathlineto{\pgfqpoint{4.442203in}{2.270809in}}%
\pgfpathlineto{\pgfqpoint{4.442371in}{2.502035in}}%
\pgfpathlineto{\pgfqpoint{4.442706in}{2.253411in}}%
\pgfpathlineto{\pgfqpoint{4.443374in}{2.380811in}}%
\pgfpathlineto{\pgfqpoint{4.443625in}{2.298436in}}%
\pgfpathlineto{\pgfqpoint{4.443542in}{2.491931in}}%
\pgfpathlineto{\pgfqpoint{4.444042in}{2.393948in}}%
\pgfpathlineto{\pgfqpoint{4.444542in}{2.522116in}}%
\pgfpathlineto{\pgfqpoint{4.444626in}{2.284371in}}%
\pgfpathlineto{\pgfqpoint{4.445125in}{2.390740in}}%
\pgfpathlineto{\pgfqpoint{4.446039in}{2.220596in}}%
\pgfpathlineto{\pgfqpoint{4.445375in}{2.522457in}}%
\pgfpathlineto{\pgfqpoint{4.446205in}{2.304637in}}%
\pgfpathlineto{\pgfqpoint{4.447117in}{2.497661in}}%
\pgfpathlineto{\pgfqpoint{4.446537in}{2.241444in}}%
\pgfpathlineto{\pgfqpoint{4.447283in}{2.341440in}}%
\pgfpathlineto{\pgfqpoint{4.447696in}{2.438629in}}%
\pgfpathlineto{\pgfqpoint{4.448275in}{2.215469in}}%
\pgfpathlineto{\pgfqpoint{4.448770in}{2.479316in}}%
\pgfpathlineto{\pgfqpoint{4.449100in}{2.074752in}}%
\pgfpathlineto{\pgfqpoint{4.449429in}{2.413081in}}%
\pgfpathlineto{\pgfqpoint{4.450581in}{2.232947in}}%
\pgfpathlineto{\pgfqpoint{4.449758in}{2.505363in}}%
\pgfpathlineto{\pgfqpoint{4.450663in}{2.391246in}}%
\pgfpathlineto{\pgfqpoint{4.450991in}{2.493657in}}%
\pgfpathlineto{\pgfqpoint{4.451073in}{2.261361in}}%
\pgfpathlineto{\pgfqpoint{4.451401in}{2.359946in}}%
\pgfpathlineto{\pgfqpoint{4.452056in}{2.179797in}}%
\pgfpathlineto{\pgfqpoint{4.452220in}{2.446157in}}%
\pgfpathlineto{\pgfqpoint{4.452302in}{2.239647in}}%
\pgfpathlineto{\pgfqpoint{4.453282in}{2.500486in}}%
\pgfpathlineto{\pgfqpoint{4.453364in}{2.451181in}}%
\pgfpathlineto{\pgfqpoint{4.454342in}{2.096667in}}%
\pgfpathlineto{\pgfqpoint{4.454179in}{2.483919in}}%
\pgfpathlineto{\pgfqpoint{4.454423in}{2.387068in}}%
\pgfpathlineto{\pgfqpoint{4.454667in}{2.399349in}}%
\pgfpathlineto{\pgfqpoint{4.454749in}{2.303330in}}%
\pgfpathlineto{\pgfqpoint{4.455318in}{2.504401in}}%
\pgfpathlineto{\pgfqpoint{4.455723in}{2.344796in}}%
\pgfpathlineto{\pgfqpoint{4.456696in}{2.540016in}}%
\pgfpathlineto{\pgfqpoint{4.456210in}{2.175899in}}%
\pgfpathlineto{\pgfqpoint{4.456939in}{2.492858in}}%
\pgfpathlineto{\pgfqpoint{4.457101in}{2.211509in}}%
\pgfpathlineto{\pgfqpoint{4.457181in}{2.518062in}}%
\pgfpathlineto{\pgfqpoint{4.458231in}{2.223635in}}%
\pgfpathlineto{\pgfqpoint{4.459279in}{2.526950in}}%
\pgfpathlineto{\pgfqpoint{4.458715in}{2.039913in}}%
\pgfpathlineto{\pgfqpoint{4.459359in}{2.434485in}}%
\pgfpathlineto{\pgfqpoint{4.460243in}{2.174582in}}%
\pgfpathlineto{\pgfqpoint{4.459761in}{2.469120in}}%
\pgfpathlineto{\pgfqpoint{4.460484in}{2.348111in}}%
\pgfpathlineto{\pgfqpoint{4.460805in}{2.519787in}}%
\pgfpathlineto{\pgfqpoint{4.461285in}{2.241101in}}%
\pgfpathlineto{\pgfqpoint{4.461686in}{2.506924in}}%
\pgfpathlineto{\pgfqpoint{4.461926in}{2.246628in}}%
\pgfpathlineto{\pgfqpoint{4.462884in}{2.276833in}}%
\pgfpathlineto{\pgfqpoint{4.463123in}{2.456684in}}%
\pgfpathlineto{\pgfqpoint{4.463840in}{2.185323in}}%
\pgfpathlineto{\pgfqpoint{4.464000in}{2.376853in}}%
\pgfpathlineto{\pgfqpoint{4.464238in}{2.350140in}}%
\pgfpathlineto{\pgfqpoint{4.464318in}{2.426435in}}%
\pgfpathlineto{\pgfqpoint{4.464477in}{2.149060in}}%
\pgfpathlineto{\pgfqpoint{4.465271in}{2.548986in}}%
\pgfpathlineto{\pgfqpoint{4.466380in}{2.200300in}}%
\pgfpathlineto{\pgfqpoint{4.466697in}{2.517669in}}%
\pgfpathlineto{\pgfqpoint{4.467250in}{2.138888in}}%
\pgfpathlineto{\pgfqpoint{4.467486in}{2.486319in}}%
\pgfpathlineto{\pgfqpoint{4.467960in}{2.226581in}}%
\pgfpathlineto{\pgfqpoint{4.467802in}{2.487522in}}%
\pgfpathlineto{\pgfqpoint{4.468669in}{2.387033in}}%
\pgfpathlineto{\pgfqpoint{4.468905in}{2.473792in}}%
\pgfpathlineto{\pgfqpoint{4.469141in}{2.197751in}}%
\pgfpathlineto{\pgfqpoint{4.469298in}{2.361648in}}%
\pgfpathlineto{\pgfqpoint{4.469376in}{2.003824in}}%
\pgfpathlineto{\pgfqpoint{4.470004in}{2.513790in}}%
\pgfpathlineto{\pgfqpoint{4.470318in}{2.233559in}}%
\pgfpathlineto{\pgfqpoint{4.471414in}{2.494405in}}%
\pgfpathlineto{\pgfqpoint{4.472118in}{2.168148in}}%
\pgfpathlineto{\pgfqpoint{4.472586in}{2.356714in}}%
\pgfpathlineto{\pgfqpoint{4.473443in}{2.502438in}}%
\pgfpathlineto{\pgfqpoint{4.472820in}{2.155441in}}%
\pgfpathlineto{\pgfqpoint{4.473520in}{2.430888in}}%
\pgfpathlineto{\pgfqpoint{4.473598in}{2.135890in}}%
\pgfpathlineto{\pgfqpoint{4.473987in}{2.456165in}}%
\pgfpathlineto{\pgfqpoint{4.474608in}{2.412016in}}%
\pgfpathlineto{\pgfqpoint{4.474996in}{2.473087in}}%
\pgfpathlineto{\pgfqpoint{4.474841in}{2.393952in}}%
\pgfpathlineto{\pgfqpoint{4.475229in}{2.415335in}}%
\pgfpathlineto{\pgfqpoint{4.475306in}{2.120839in}}%
\pgfpathlineto{\pgfqpoint{4.475384in}{2.470231in}}%
\pgfpathlineto{\pgfqpoint{4.476235in}{2.392312in}}%
\pgfpathlineto{\pgfqpoint{4.476312in}{2.557999in}}%
\pgfpathlineto{\pgfqpoint{4.476467in}{2.309776in}}%
\pgfpathlineto{\pgfqpoint{4.477316in}{2.368754in}}%
\pgfpathlineto{\pgfqpoint{4.478317in}{2.128441in}}%
\pgfpathlineto{\pgfqpoint{4.477624in}{2.470233in}}%
\pgfpathlineto{\pgfqpoint{4.478394in}{2.292647in}}%
\pgfpathlineto{\pgfqpoint{4.479470in}{2.512190in}}%
\pgfpathlineto{\pgfqpoint{4.479546in}{2.450635in}}%
\pgfpathlineto{\pgfqpoint{4.479700in}{2.500902in}}%
\pgfpathlineto{\pgfqpoint{4.480236in}{2.242067in}}%
\pgfpathlineto{\pgfqpoint{4.480313in}{2.399697in}}%
\pgfpathlineto{\pgfqpoint{4.481078in}{2.032040in}}%
\pgfpathlineto{\pgfqpoint{4.481154in}{2.560981in}}%
\pgfpathlineto{\pgfqpoint{4.481383in}{2.495088in}}%
\pgfpathlineto{\pgfqpoint{4.481765in}{2.106313in}}%
\pgfpathlineto{\pgfqpoint{4.482528in}{2.418867in}}%
\pgfpathlineto{\pgfqpoint{4.482832in}{2.517302in}}%
\pgfpathlineto{\pgfqpoint{4.483213in}{2.284302in}}%
\pgfpathlineto{\pgfqpoint{4.483517in}{2.443040in}}%
\pgfpathlineto{\pgfqpoint{4.483896in}{2.503008in}}%
\pgfpathlineto{\pgfqpoint{4.484731in}{2.227334in}}%
\pgfpathlineto{\pgfqpoint{4.485488in}{2.525959in}}%
\pgfpathlineto{\pgfqpoint{4.485715in}{2.223086in}}%
\pgfpathlineto{\pgfqpoint{4.485866in}{2.510716in}}%
\pgfpathlineto{\pgfqpoint{4.486847in}{2.297675in}}%
\pgfpathlineto{\pgfqpoint{4.487074in}{2.402673in}}%
\pgfpathlineto{\pgfqpoint{4.487977in}{2.509253in}}%
\pgfpathlineto{\pgfqpoint{4.487525in}{2.184576in}}%
\pgfpathlineto{\pgfqpoint{4.488127in}{2.423200in}}%
\pgfpathlineto{\pgfqpoint{4.489028in}{2.258090in}}%
\pgfpathlineto{\pgfqpoint{4.488878in}{2.492386in}}%
\pgfpathlineto{\pgfqpoint{4.489253in}{2.333697in}}%
\pgfpathlineto{\pgfqpoint{4.489778in}{2.467260in}}%
\pgfpathlineto{\pgfqpoint{4.489628in}{2.241441in}}%
\pgfpathlineto{\pgfqpoint{4.490077in}{2.418788in}}%
\pgfpathlineto{\pgfqpoint{4.490152in}{2.003924in}}%
\pgfpathlineto{\pgfqpoint{4.490825in}{2.491834in}}%
\pgfpathlineto{\pgfqpoint{4.491198in}{2.276309in}}%
\pgfpathlineto{\pgfqpoint{4.492242in}{2.527661in}}%
\pgfpathlineto{\pgfqpoint{4.491497in}{2.175392in}}%
\pgfpathlineto{\pgfqpoint{4.492316in}{2.443474in}}%
\pgfpathlineto{\pgfqpoint{4.492763in}{2.489293in}}%
\pgfpathlineto{\pgfqpoint{4.492911in}{2.323972in}}%
\pgfpathlineto{\pgfqpoint{4.493060in}{2.359250in}}%
\pgfpathlineto{\pgfqpoint{4.493134in}{2.359239in}}%
\pgfpathlineto{\pgfqpoint{4.493283in}{2.253859in}}%
\pgfpathlineto{\pgfqpoint{4.493506in}{2.545654in}}%
\pgfpathlineto{\pgfqpoint{4.494099in}{2.448274in}}%
\pgfpathlineto{\pgfqpoint{4.494173in}{2.490291in}}%
\pgfpathlineto{\pgfqpoint{4.494692in}{2.255116in}}%
\pgfpathlineto{\pgfqpoint{4.495062in}{2.487107in}}%
\pgfpathlineto{\pgfqpoint{4.496170in}{2.293856in}}%
\pgfpathlineto{\pgfqpoint{4.496907in}{2.507237in}}%
\pgfpathlineto{\pgfqpoint{4.496833in}{1.803856in}}%
\pgfpathlineto{\pgfqpoint{4.497348in}{2.423666in}}%
\pgfpathlineto{\pgfqpoint{4.498083in}{2.137377in}}%
\pgfpathlineto{\pgfqpoint{4.497716in}{2.540162in}}%
\pgfpathlineto{\pgfqpoint{4.498377in}{2.427893in}}%
\pgfpathlineto{\pgfqpoint{4.498597in}{2.544791in}}%
\pgfpathlineto{\pgfqpoint{4.498523in}{2.325157in}}%
\pgfpathlineto{\pgfqpoint{4.499476in}{2.458658in}}%
\pgfpathlineto{\pgfqpoint{4.500061in}{2.230426in}}%
\pgfpathlineto{\pgfqpoint{4.499842in}{2.505913in}}%
\pgfpathlineto{\pgfqpoint{4.500572in}{2.323479in}}%
\pgfpathlineto{\pgfqpoint{4.501229in}{2.095361in}}%
\pgfpathlineto{\pgfqpoint{4.501739in}{2.481832in}}%
\pgfpathlineto{\pgfqpoint{4.501812in}{2.480696in}}%
\pgfpathlineto{\pgfqpoint{4.501884in}{2.212912in}}%
\pgfpathlineto{\pgfqpoint{4.502829in}{2.556350in}}%
\pgfpathlineto{\pgfqpoint{4.502902in}{2.370768in}}%
\pgfpathlineto{\pgfqpoint{4.503410in}{2.483906in}}%
\pgfpathlineto{\pgfqpoint{4.503772in}{2.213932in}}%
\pgfpathlineto{\pgfqpoint{4.504062in}{2.437202in}}%
\pgfpathlineto{\pgfqpoint{4.504135in}{2.433434in}}%
\pgfpathlineto{\pgfqpoint{4.504786in}{2.182427in}}%
\pgfpathlineto{\pgfqpoint{4.504858in}{2.532047in}}%
\pgfpathlineto{\pgfqpoint{4.505219in}{2.318936in}}%
\pgfpathlineto{\pgfqpoint{4.505580in}{2.533302in}}%
\pgfpathlineto{\pgfqpoint{4.506085in}{2.277346in}}%
\pgfpathlineto{\pgfqpoint{4.506301in}{2.452865in}}%
\pgfpathlineto{\pgfqpoint{4.507164in}{2.181591in}}%
\pgfpathlineto{\pgfqpoint{4.506517in}{2.481868in}}%
\pgfpathlineto{\pgfqpoint{4.507380in}{2.466826in}}%
\pgfpathlineto{\pgfqpoint{4.508528in}{2.117915in}}%
\pgfpathlineto{\pgfqpoint{4.508098in}{2.527363in}}%
\pgfpathlineto{\pgfqpoint{4.508671in}{2.385686in}}%
\pgfpathlineto{\pgfqpoint{4.509172in}{2.529397in}}%
\pgfpathlineto{\pgfqpoint{4.509530in}{2.328486in}}%
\pgfpathlineto{\pgfqpoint{4.509815in}{2.496493in}}%
\pgfpathlineto{\pgfqpoint{4.510030in}{2.210346in}}%
\pgfpathlineto{\pgfqpoint{4.510957in}{2.360510in}}%
\pgfpathlineto{\pgfqpoint{4.511313in}{2.510832in}}%
\pgfpathlineto{\pgfqpoint{4.511526in}{2.241988in}}%
\pgfpathlineto{\pgfqpoint{4.512024in}{2.410379in}}%
\pgfpathlineto{\pgfqpoint{4.512805in}{2.280879in}}%
\pgfpathlineto{\pgfqpoint{4.512450in}{2.536666in}}%
\pgfpathlineto{\pgfqpoint{4.513088in}{2.371890in}}%
\pgfpathlineto{\pgfqpoint{4.513443in}{2.530165in}}%
\pgfpathlineto{\pgfqpoint{4.513867in}{2.081973in}}%
\pgfpathlineto{\pgfqpoint{4.514150in}{2.394749in}}%
\pgfpathlineto{\pgfqpoint{4.514857in}{2.489741in}}%
\pgfpathlineto{\pgfqpoint{4.515210in}{2.173209in}}%
\pgfpathlineto{\pgfqpoint{4.515280in}{2.522021in}}%
\pgfpathlineto{\pgfqpoint{4.516336in}{2.356832in}}%
\pgfpathlineto{\pgfqpoint{4.516477in}{2.433356in}}%
\pgfpathlineto{\pgfqpoint{4.516547in}{2.403096in}}%
\pgfpathlineto{\pgfqpoint{4.517531in}{2.153139in}}%
\pgfpathlineto{\pgfqpoint{4.517110in}{2.473615in}}%
\pgfpathlineto{\pgfqpoint{4.517601in}{2.433108in}}%
\pgfpathlineto{\pgfqpoint{4.517741in}{2.407483in}}%
\pgfpathlineto{\pgfqpoint{4.518791in}{2.130161in}}%
\pgfpathlineto{\pgfqpoint{4.517951in}{2.505185in}}%
\pgfpathlineto{\pgfqpoint{4.518861in}{2.364746in}}%
\pgfpathlineto{\pgfqpoint{4.518931in}{2.510952in}}%
\pgfpathlineto{\pgfqpoint{4.519001in}{2.093046in}}%
\pgfpathlineto{\pgfqpoint{4.519909in}{2.474465in}}%
\pgfpathlineto{\pgfqpoint{4.520745in}{2.109909in}}%
\pgfpathlineto{\pgfqpoint{4.520675in}{2.478093in}}%
\pgfpathlineto{\pgfqpoint{4.521023in}{2.391574in}}%
\pgfpathlineto{\pgfqpoint{4.521857in}{2.265556in}}%
\pgfpathlineto{\pgfqpoint{4.521371in}{2.504797in}}%
\pgfpathlineto{\pgfqpoint{4.522204in}{2.341674in}}%
\pgfpathlineto{\pgfqpoint{4.522759in}{2.506717in}}%
\pgfpathlineto{\pgfqpoint{4.522620in}{2.050108in}}%
\pgfpathlineto{\pgfqpoint{4.523174in}{2.410249in}}%
\pgfpathlineto{\pgfqpoint{4.523244in}{2.167666in}}%
\pgfpathlineto{\pgfqpoint{4.524211in}{2.515979in}}%
\pgfpathlineto{\pgfqpoint{4.524280in}{2.286997in}}%
\pgfpathlineto{\pgfqpoint{4.525177in}{2.490185in}}%
\pgfpathlineto{\pgfqpoint{4.525315in}{2.424589in}}%
\pgfpathlineto{\pgfqpoint{4.525865in}{2.215153in}}%
\pgfpathlineto{\pgfqpoint{4.525659in}{2.452168in}}%
\pgfpathlineto{\pgfqpoint{4.526415in}{2.298181in}}%
\pgfpathlineto{\pgfqpoint{4.527307in}{2.503417in}}%
\pgfpathlineto{\pgfqpoint{4.527513in}{2.494358in}}%
\pgfpathlineto{\pgfqpoint{4.527855in}{2.227102in}}%
\pgfpathlineto{\pgfqpoint{4.528608in}{2.412870in}}%
\pgfpathlineto{\pgfqpoint{4.529086in}{2.533548in}}%
\pgfpathlineto{\pgfqpoint{4.529427in}{2.155569in}}%
\pgfpathlineto{\pgfqpoint{4.529563in}{2.330875in}}%
\pgfpathlineto{\pgfqpoint{4.529631in}{2.236142in}}%
\pgfpathlineto{\pgfqpoint{4.530108in}{2.567617in}}%
\pgfpathlineto{\pgfqpoint{4.530653in}{2.318744in}}%
\pgfpathlineto{\pgfqpoint{4.530789in}{2.247755in}}%
\pgfpathlineto{\pgfqpoint{4.531400in}{2.466988in}}%
\pgfpathlineto{\pgfqpoint{4.532214in}{2.133252in}}%
\pgfpathlineto{\pgfqpoint{4.531604in}{2.499323in}}%
\pgfpathlineto{\pgfqpoint{4.532485in}{2.395347in}}%
\pgfpathlineto{\pgfqpoint{4.532553in}{2.397376in}}%
\pgfpathlineto{\pgfqpoint{4.532620in}{2.504768in}}%
\pgfpathlineto{\pgfqpoint{4.533161in}{2.246260in}}%
\pgfpathlineto{\pgfqpoint{4.533567in}{2.350768in}}%
\pgfpathlineto{\pgfqpoint{4.533702in}{2.193604in}}%
\pgfpathlineto{\pgfqpoint{4.534039in}{2.481760in}}%
\pgfpathlineto{\pgfqpoint{4.534107in}{2.210866in}}%
\pgfpathlineto{\pgfqpoint{4.534377in}{2.546951in}}%
\pgfpathlineto{\pgfqpoint{4.534781in}{2.075478in}}%
\pgfpathlineto{\pgfqpoint{4.535185in}{2.461430in}}%
\pgfpathlineto{\pgfqpoint{4.535588in}{2.481881in}}%
\pgfpathlineto{\pgfqpoint{4.536327in}{2.146039in}}%
\pgfpathlineto{\pgfqpoint{4.536931in}{2.466866in}}%
\pgfpathlineto{\pgfqpoint{4.537466in}{2.409372in}}%
\pgfpathlineto{\pgfqpoint{4.538469in}{2.083898in}}%
\pgfpathlineto{\pgfqpoint{4.538202in}{2.528198in}}%
\pgfpathlineto{\pgfqpoint{4.538536in}{2.108979in}}%
\pgfpathlineto{\pgfqpoint{4.539203in}{2.493400in}}%
\pgfpathlineto{\pgfqpoint{4.539669in}{2.419257in}}%
\pgfpathlineto{\pgfqpoint{4.540068in}{1.906987in}}%
\pgfpathlineto{\pgfqpoint{4.540401in}{2.519564in}}%
\pgfpathlineto{\pgfqpoint{4.540667in}{2.287567in}}%
\pgfpathlineto{\pgfqpoint{4.541595in}{2.513691in}}%
\pgfpathlineto{\pgfqpoint{4.540799in}{2.200409in}}%
\pgfpathlineto{\pgfqpoint{4.541794in}{2.429902in}}%
\pgfpathlineto{\pgfqpoint{4.542125in}{2.466015in}}%
\pgfpathlineto{\pgfqpoint{4.542192in}{2.261822in}}%
\pgfpathlineto{\pgfqpoint{4.542258in}{2.224060in}}%
\pgfpathlineto{\pgfqpoint{4.542655in}{2.463058in}}%
\pgfpathlineto{\pgfqpoint{4.542985in}{2.385525in}}%
\pgfpathlineto{\pgfqpoint{4.543249in}{2.299779in}}%
\pgfpathlineto{\pgfqpoint{4.544107in}{2.519870in}}%
\pgfpathlineto{\pgfqpoint{4.544370in}{2.088017in}}%
\pgfpathlineto{\pgfqpoint{4.545225in}{2.469211in}}%
\pgfpathlineto{\pgfqpoint{4.546078in}{2.138698in}}%
\pgfpathlineto{\pgfqpoint{4.545947in}{2.480334in}}%
\pgfpathlineto{\pgfqpoint{4.546341in}{2.254175in}}%
\pgfpathlineto{\pgfqpoint{4.547192in}{2.478116in}}%
\pgfpathlineto{\pgfqpoint{4.547061in}{2.158760in}}%
\pgfpathlineto{\pgfqpoint{4.547453in}{2.345717in}}%
\pgfpathlineto{\pgfqpoint{4.547649in}{2.136058in}}%
\pgfpathlineto{\pgfqpoint{4.548107in}{2.473960in}}%
\pgfpathlineto{\pgfqpoint{4.548368in}{2.310892in}}%
\pgfpathlineto{\pgfqpoint{4.549019in}{2.501799in}}%
\pgfpathlineto{\pgfqpoint{4.548498in}{2.014476in}}%
\pgfpathlineto{\pgfqpoint{4.549475in}{2.322210in}}%
\pgfpathlineto{\pgfqpoint{4.549865in}{2.524372in}}%
\pgfpathlineto{\pgfqpoint{4.549995in}{2.295204in}}%
\pgfpathlineto{\pgfqpoint{4.550579in}{2.358546in}}%
\pgfpathlineto{\pgfqpoint{4.550644in}{2.203607in}}%
\pgfpathlineto{\pgfqpoint{4.551293in}{2.483795in}}%
\pgfpathlineto{\pgfqpoint{4.551616in}{2.305691in}}%
\pgfpathlineto{\pgfqpoint{4.552780in}{2.569299in}}%
\pgfpathlineto{\pgfqpoint{4.551811in}{2.104432in}}%
\pgfpathlineto{\pgfqpoint{4.552844in}{2.474453in}}%
\pgfpathlineto{\pgfqpoint{4.553038in}{2.081579in}}%
\pgfpathlineto{\pgfqpoint{4.553876in}{2.492099in}}%
\pgfpathlineto{\pgfqpoint{4.554005in}{2.413318in}}%
\pgfpathlineto{\pgfqpoint{4.554069in}{2.417306in}}%
\pgfpathlineto{\pgfqpoint{4.554198in}{2.225655in}}%
\pgfpathlineto{\pgfqpoint{4.554841in}{2.479684in}}%
\pgfpathlineto{\pgfqpoint{4.555226in}{2.325552in}}%
\pgfpathlineto{\pgfqpoint{4.555803in}{2.534035in}}%
\pgfpathlineto{\pgfqpoint{4.555611in}{1.937942in}}%
\pgfpathlineto{\pgfqpoint{4.556379in}{2.489346in}}%
\pgfpathlineto{\pgfqpoint{4.556763in}{2.223394in}}%
\pgfpathlineto{\pgfqpoint{4.556891in}{2.499668in}}%
\pgfpathlineto{\pgfqpoint{4.557466in}{2.369932in}}%
\pgfpathlineto{\pgfqpoint{4.557849in}{2.314070in}}%
\pgfpathlineto{\pgfqpoint{4.558614in}{2.522519in}}%
\pgfpathlineto{\pgfqpoint{4.559059in}{2.218004in}}%
\pgfpathlineto{\pgfqpoint{4.559822in}{2.298701in}}%
\pgfpathlineto{\pgfqpoint{4.560646in}{2.468526in}}%
\pgfpathlineto{\pgfqpoint{4.560520in}{2.126082in}}%
\pgfpathlineto{\pgfqpoint{4.560963in}{2.414672in}}%
\pgfpathlineto{\pgfqpoint{4.561785in}{2.559510in}}%
\pgfpathlineto{\pgfqpoint{4.561469in}{2.168126in}}%
\pgfpathlineto{\pgfqpoint{4.562038in}{2.434865in}}%
\pgfpathlineto{\pgfqpoint{4.562984in}{2.109767in}}%
\pgfpathlineto{\pgfqpoint{4.562543in}{2.503775in}}%
\pgfpathlineto{\pgfqpoint{4.563174in}{2.248238in}}%
\pgfpathlineto{\pgfqpoint{4.564117in}{2.482030in}}%
\pgfpathlineto{\pgfqpoint{4.564180in}{2.077877in}}%
\pgfpathlineto{\pgfqpoint{4.564243in}{2.420266in}}%
\pgfpathlineto{\pgfqpoint{4.564431in}{2.006241in}}%
\pgfpathlineto{\pgfqpoint{4.564683in}{2.525559in}}%
\pgfpathlineto{\pgfqpoint{4.565310in}{2.282468in}}%
\pgfpathlineto{\pgfqpoint{4.566061in}{2.497421in}}%
\pgfpathlineto{\pgfqpoint{4.566561in}{2.477892in}}%
\pgfpathlineto{\pgfqpoint{4.567373in}{2.274507in}}%
\pgfpathlineto{\pgfqpoint{4.566811in}{2.506417in}}%
\pgfpathlineto{\pgfqpoint{4.567747in}{2.396303in}}%
\pgfpathlineto{\pgfqpoint{4.568494in}{2.538086in}}%
\pgfpathlineto{\pgfqpoint{4.568681in}{2.208965in}}%
\pgfpathlineto{\pgfqpoint{4.568743in}{2.111485in}}%
\pgfpathlineto{\pgfqpoint{4.568992in}{2.524212in}}%
\pgfpathlineto{\pgfqpoint{4.569612in}{2.359865in}}%
\pgfpathlineto{\pgfqpoint{4.570356in}{2.179186in}}%
\pgfpathlineto{\pgfqpoint{4.570790in}{2.505444in}}%
\pgfpathlineto{\pgfqpoint{4.570975in}{2.158464in}}%
\pgfpathlineto{\pgfqpoint{4.571408in}{2.531593in}}%
\pgfpathlineto{\pgfqpoint{4.571963in}{2.169567in}}%
\pgfpathlineto{\pgfqpoint{4.573134in}{2.496290in}}%
\pgfpathlineto{\pgfqpoint{4.573626in}{2.099643in}}%
\pgfpathlineto{\pgfqpoint{4.573503in}{2.524427in}}%
\pgfpathlineto{\pgfqpoint{4.574240in}{2.275885in}}%
\pgfpathlineto{\pgfqpoint{4.574302in}{2.540114in}}%
\pgfpathlineto{\pgfqpoint{4.574976in}{1.961139in}}%
\pgfpathlineto{\pgfqpoint{4.575343in}{2.319813in}}%
\pgfpathlineto{\pgfqpoint{4.575527in}{2.040384in}}%
\pgfpathlineto{\pgfqpoint{4.576444in}{2.461364in}}%
\pgfpathlineto{\pgfqpoint{4.576871in}{2.130612in}}%
\pgfpathlineto{\pgfqpoint{4.577358in}{2.540618in}}%
\pgfpathlineto{\pgfqpoint{4.577541in}{2.453872in}}%
\pgfpathlineto{\pgfqpoint{4.578454in}{2.138908in}}%
\pgfpathlineto{\pgfqpoint{4.578028in}{2.526333in}}%
\pgfpathlineto{\pgfqpoint{4.578757in}{2.230222in}}%
\pgfpathlineto{\pgfqpoint{4.578879in}{2.547060in}}%
\pgfpathlineto{\pgfqpoint{4.579909in}{2.496544in}}%
\pgfpathlineto{\pgfqpoint{4.579970in}{2.512786in}}%
\pgfpathlineto{\pgfqpoint{4.580333in}{2.339300in}}%
\pgfpathlineto{\pgfqpoint{4.580394in}{2.423642in}}%
\pgfpathlineto{\pgfqpoint{4.580696in}{2.215119in}}%
\pgfpathlineto{\pgfqpoint{4.580877in}{2.542936in}}%
\pgfpathlineto{\pgfqpoint{4.581541in}{2.341510in}}%
\pgfpathlineto{\pgfqpoint{4.582023in}{2.495620in}}%
\pgfpathlineto{\pgfqpoint{4.581722in}{2.190462in}}%
\pgfpathlineto{\pgfqpoint{4.582565in}{2.377509in}}%
\pgfpathlineto{\pgfqpoint{4.583467in}{2.237102in}}%
\pgfpathlineto{\pgfqpoint{4.583347in}{2.461192in}}%
\pgfpathlineto{\pgfqpoint{4.583587in}{2.263286in}}%
\pgfpathlineto{\pgfqpoint{4.583887in}{2.117669in}}%
\pgfpathlineto{\pgfqpoint{4.584726in}{2.545417in}}%
\pgfpathlineto{\pgfqpoint{4.585504in}{2.301638in}}%
\pgfpathlineto{\pgfqpoint{4.585862in}{2.392345in}}%
\pgfpathlineto{\pgfqpoint{4.586280in}{2.521011in}}%
\pgfpathlineto{\pgfqpoint{4.586638in}{2.275627in}}%
\pgfpathlineto{\pgfqpoint{4.586817in}{2.466905in}}%
\pgfpathlineto{\pgfqpoint{4.587769in}{2.169824in}}%
\pgfpathlineto{\pgfqpoint{4.587114in}{2.542638in}}%
\pgfpathlineto{\pgfqpoint{4.587888in}{2.367683in}}%
\pgfpathlineto{\pgfqpoint{4.588006in}{2.530314in}}%
\pgfpathlineto{\pgfqpoint{4.588778in}{2.198587in}}%
\pgfpathlineto{\pgfqpoint{4.588956in}{2.415520in}}%
\pgfpathlineto{\pgfqpoint{4.589726in}{2.497110in}}%
\pgfpathlineto{\pgfqpoint{4.589962in}{2.217288in}}%
\pgfpathlineto{\pgfqpoint{4.590908in}{2.523300in}}%
\pgfpathlineto{\pgfqpoint{4.591085in}{2.390856in}}%
\pgfpathlineto{\pgfqpoint{4.591320in}{2.519289in}}%
\pgfpathlineto{\pgfqpoint{4.591379in}{2.101768in}}%
\pgfpathlineto{\pgfqpoint{4.591909in}{2.288598in}}%
\pgfpathlineto{\pgfqpoint{4.592792in}{1.982382in}}%
\pgfpathlineto{\pgfqpoint{4.592380in}{2.516484in}}%
\pgfpathlineto{\pgfqpoint{4.592968in}{2.424703in}}%
\pgfpathlineto{\pgfqpoint{4.593203in}{2.047146in}}%
\pgfpathlineto{\pgfqpoint{4.593848in}{2.490661in}}%
\pgfpathlineto{\pgfqpoint{4.594082in}{2.269616in}}%
\pgfpathlineto{\pgfqpoint{4.594375in}{2.445311in}}%
\pgfpathlineto{\pgfqpoint{4.595252in}{2.395305in}}%
\pgfpathlineto{\pgfqpoint{4.595368in}{2.528627in}}%
\pgfpathlineto{\pgfqpoint{4.595544in}{2.174107in}}%
\pgfpathlineto{\pgfqpoint{4.595894in}{2.457597in}}%
\pgfpathlineto{\pgfqpoint{4.595952in}{2.049570in}}%
\pgfpathlineto{\pgfqpoint{4.596418in}{2.524998in}}%
\pgfpathlineto{\pgfqpoint{4.597000in}{2.312407in}}%
\pgfpathlineto{\pgfqpoint{4.597465in}{2.181284in}}%
\pgfpathlineto{\pgfqpoint{4.598046in}{2.473223in}}%
\pgfpathlineto{\pgfqpoint{4.598857in}{2.221330in}}%
\pgfpathlineto{\pgfqpoint{4.598626in}{2.514886in}}%
\pgfpathlineto{\pgfqpoint{4.599205in}{2.235106in}}%
\pgfpathlineto{\pgfqpoint{4.600187in}{2.488356in}}%
\pgfpathlineto{\pgfqpoint{4.599320in}{2.076367in}}%
\pgfpathlineto{\pgfqpoint{4.600360in}{2.447302in}}%
\pgfpathlineto{\pgfqpoint{4.600418in}{2.441526in}}%
\pgfpathlineto{\pgfqpoint{4.601283in}{2.181554in}}%
\pgfpathlineto{\pgfqpoint{4.601167in}{2.537964in}}%
\pgfpathlineto{\pgfqpoint{4.601513in}{2.353808in}}%
\pgfpathlineto{\pgfqpoint{4.602203in}{2.553744in}}%
\pgfpathlineto{\pgfqpoint{4.601973in}{2.167941in}}%
\pgfpathlineto{\pgfqpoint{4.602433in}{2.304036in}}%
\pgfpathlineto{\pgfqpoint{4.602490in}{2.144678in}}%
\pgfpathlineto{\pgfqpoint{4.602834in}{2.488522in}}%
\pgfpathlineto{\pgfqpoint{4.603522in}{2.231258in}}%
\pgfpathlineto{\pgfqpoint{4.604266in}{2.512800in}}%
\pgfpathlineto{\pgfqpoint{4.603751in}{2.034890in}}%
\pgfpathlineto{\pgfqpoint{4.604666in}{2.388215in}}%
\pgfpathlineto{\pgfqpoint{4.604723in}{2.385714in}}%
\pgfpathlineto{\pgfqpoint{4.604838in}{2.504922in}}%
\pgfpathlineto{\pgfqpoint{4.605465in}{2.263069in}}%
\pgfpathlineto{\pgfqpoint{4.605579in}{2.435979in}}%
\pgfpathlineto{\pgfqpoint{4.605636in}{1.961199in}}%
\pgfpathlineto{\pgfqpoint{4.606490in}{2.520149in}}%
\pgfpathlineto{\pgfqpoint{4.606661in}{2.327011in}}%
\pgfpathlineto{\pgfqpoint{4.606888in}{2.491512in}}%
\pgfpathlineto{\pgfqpoint{4.607456in}{2.300391in}}%
\pgfpathlineto{\pgfqpoint{4.607796in}{2.429374in}}%
\pgfpathlineto{\pgfqpoint{4.608816in}{2.234136in}}%
\pgfpathlineto{\pgfqpoint{4.608080in}{2.529035in}}%
\pgfpathlineto{\pgfqpoint{4.608872in}{2.456298in}}%
\pgfpathlineto{\pgfqpoint{4.609776in}{2.165734in}}%
\pgfpathlineto{\pgfqpoint{4.609268in}{2.514980in}}%
\pgfpathlineto{\pgfqpoint{4.610058in}{2.380531in}}%
\pgfpathlineto{\pgfqpoint{4.610115in}{2.513429in}}%
\pgfpathlineto{\pgfqpoint{4.610622in}{2.064678in}}%
\pgfpathlineto{\pgfqpoint{4.611185in}{2.508407in}}%
\pgfpathlineto{\pgfqpoint{4.611522in}{2.116376in}}%
\pgfpathlineto{\pgfqpoint{4.611579in}{2.541543in}}%
\pgfpathlineto{\pgfqpoint{4.612309in}{2.230120in}}%
\pgfpathlineto{\pgfqpoint{4.612701in}{2.486689in}}%
\pgfpathlineto{\pgfqpoint{4.613205in}{2.216498in}}%
\pgfpathlineto{\pgfqpoint{4.613429in}{2.366116in}}%
\pgfpathlineto{\pgfqpoint{4.614379in}{2.563264in}}%
\pgfpathlineto{\pgfqpoint{4.613709in}{2.280991in}}%
\pgfpathlineto{\pgfqpoint{4.614547in}{2.465089in}}%
\pgfpathlineto{\pgfqpoint{4.614658in}{2.098849in}}%
\pgfpathlineto{\pgfqpoint{4.614714in}{2.486161in}}%
\pgfpathlineto{\pgfqpoint{4.615661in}{2.324595in}}%
\pgfpathlineto{\pgfqpoint{4.615995in}{2.526182in}}%
\pgfpathlineto{\pgfqpoint{4.616162in}{2.185415in}}%
\pgfpathlineto{\pgfqpoint{4.616273in}{2.273834in}}%
\pgfpathlineto{\pgfqpoint{4.616329in}{2.124840in}}%
\pgfpathlineto{\pgfqpoint{4.616440in}{2.555770in}}%
\pgfpathlineto{\pgfqpoint{4.617273in}{2.453182in}}%
\pgfpathlineto{\pgfqpoint{4.617328in}{2.485261in}}%
\pgfpathlineto{\pgfqpoint{4.617771in}{2.298772in}}%
\pgfpathlineto{\pgfqpoint{4.617937in}{2.337532in}}%
\pgfpathlineto{\pgfqpoint{4.618933in}{2.184215in}}%
\pgfpathlineto{\pgfqpoint{4.618048in}{2.537094in}}%
\pgfpathlineto{\pgfqpoint{4.619043in}{2.339105in}}%
\pgfpathlineto{\pgfqpoint{4.619540in}{2.160097in}}%
\pgfpathlineto{\pgfqpoint{4.619374in}{2.478968in}}%
\pgfpathlineto{\pgfqpoint{4.619816in}{2.412936in}}%
\pgfpathlineto{\pgfqpoint{4.620036in}{2.539216in}}%
\pgfpathlineto{\pgfqpoint{4.620146in}{2.131780in}}%
\pgfpathlineto{\pgfqpoint{4.620862in}{2.346539in}}%
\pgfpathlineto{\pgfqpoint{4.621246in}{2.211320in}}%
\pgfpathlineto{\pgfqpoint{4.621191in}{2.479125in}}%
\pgfpathlineto{\pgfqpoint{4.621795in}{2.388114in}}%
\pgfpathlineto{\pgfqpoint{4.622617in}{2.536005in}}%
\pgfpathlineto{\pgfqpoint{4.622508in}{2.268317in}}%
\pgfpathlineto{\pgfqpoint{4.622836in}{2.395110in}}%
\pgfpathlineto{\pgfqpoint{4.623165in}{1.932548in}}%
\pgfpathlineto{\pgfqpoint{4.623766in}{2.464230in}}%
\pgfpathlineto{\pgfqpoint{4.623930in}{2.258272in}}%
\pgfpathlineto{\pgfqpoint{4.624421in}{2.495061in}}%
\pgfpathlineto{\pgfqpoint{4.624857in}{2.185627in}}%
\pgfpathlineto{\pgfqpoint{4.625020in}{2.279565in}}%
\pgfpathlineto{\pgfqpoint{4.625129in}{2.502942in}}%
\pgfpathlineto{\pgfqpoint{4.625292in}{1.996798in}}%
\pgfpathlineto{\pgfqpoint{4.626270in}{2.437387in}}%
\pgfpathlineto{\pgfqpoint{4.626542in}{2.241477in}}%
\pgfpathlineto{\pgfqpoint{4.627301in}{2.488862in}}%
\pgfpathlineto{\pgfqpoint{4.627355in}{2.426833in}}%
\pgfpathlineto{\pgfqpoint{4.627625in}{2.494063in}}%
\pgfpathlineto{\pgfqpoint{4.627842in}{2.161486in}}%
\pgfpathlineto{\pgfqpoint{4.628382in}{2.408520in}}%
\pgfpathlineto{\pgfqpoint{4.628976in}{1.872418in}}%
\pgfpathlineto{\pgfqpoint{4.629407in}{2.506017in}}%
\pgfpathlineto{\pgfqpoint{4.629461in}{2.164801in}}%
\pgfpathlineto{\pgfqpoint{4.630215in}{2.482042in}}%
\pgfpathlineto{\pgfqpoint{4.630591in}{2.450266in}}%
\pgfpathlineto{\pgfqpoint{4.631236in}{2.126757in}}%
\pgfpathlineto{\pgfqpoint{4.630860in}{2.470738in}}%
\pgfpathlineto{\pgfqpoint{4.631718in}{2.410708in}}%
\pgfpathlineto{\pgfqpoint{4.632628in}{2.480769in}}%
\pgfpathlineto{\pgfqpoint{4.632468in}{2.219302in}}%
\pgfpathlineto{\pgfqpoint{4.632682in}{2.438747in}}%
\pgfpathlineto{\pgfqpoint{4.632789in}{2.528246in}}%
\pgfpathlineto{\pgfqpoint{4.633750in}{2.169079in}}%
\pgfpathlineto{\pgfqpoint{4.633910in}{2.493987in}}%
\pgfpathlineto{\pgfqpoint{4.634549in}{2.052631in}}%
\pgfpathlineto{\pgfqpoint{4.634869in}{2.323217in}}%
\pgfpathlineto{\pgfqpoint{4.634922in}{2.323205in}}%
\pgfpathlineto{\pgfqpoint{4.635241in}{2.513945in}}%
\pgfpathlineto{\pgfqpoint{4.635613in}{2.148420in}}%
\pgfpathlineto{\pgfqpoint{4.636090in}{2.426536in}}%
\pgfpathlineto{\pgfqpoint{4.636621in}{2.496659in}}%
\pgfpathlineto{\pgfqpoint{4.636409in}{2.008831in}}%
\pgfpathlineto{\pgfqpoint{4.636832in}{2.455135in}}%
\pgfpathlineto{\pgfqpoint{4.636885in}{2.230892in}}%
\pgfpathlineto{\pgfqpoint{4.636991in}{2.510773in}}%
\pgfpathlineto{\pgfqpoint{4.637943in}{2.274470in}}%
\pgfpathlineto{\pgfqpoint{4.638260in}{2.547887in}}%
\pgfpathlineto{\pgfqpoint{4.638893in}{2.256648in}}%
\pgfpathlineto{\pgfqpoint{4.639209in}{2.451877in}}%
\pgfpathlineto{\pgfqpoint{4.639419in}{2.016601in}}%
\pgfpathlineto{\pgfqpoint{4.639367in}{2.504798in}}%
\pgfpathlineto{\pgfqpoint{4.640365in}{2.225879in}}%
\pgfpathlineto{\pgfqpoint{4.640680in}{2.009542in}}%
\pgfpathlineto{\pgfqpoint{4.640470in}{2.472952in}}%
\pgfpathlineto{\pgfqpoint{4.640890in}{2.417099in}}%
\pgfpathlineto{\pgfqpoint{4.641676in}{2.516784in}}%
\pgfpathlineto{\pgfqpoint{4.641624in}{2.158573in}}%
\pgfpathlineto{\pgfqpoint{4.641990in}{2.425096in}}%
\pgfpathlineto{\pgfqpoint{4.642043in}{2.220639in}}%
\pgfpathlineto{\pgfqpoint{4.643035in}{2.493324in}}%
\pgfpathlineto{\pgfqpoint{4.643087in}{2.221420in}}%
\pgfpathlineto{\pgfqpoint{4.644078in}{2.490439in}}%
\pgfpathlineto{\pgfqpoint{4.644026in}{2.153645in}}%
\pgfpathlineto{\pgfqpoint{4.644234in}{2.421884in}}%
\pgfpathlineto{\pgfqpoint{4.644702in}{2.493095in}}%
\pgfpathlineto{\pgfqpoint{4.644962in}{2.252344in}}%
\pgfpathlineto{\pgfqpoint{4.645014in}{2.402383in}}%
\pgfpathlineto{\pgfqpoint{4.645066in}{2.206124in}}%
\pgfpathlineto{\pgfqpoint{4.645896in}{2.491969in}}%
\pgfpathlineto{\pgfqpoint{4.646103in}{2.466691in}}%
\pgfpathlineto{\pgfqpoint{4.646362in}{2.515867in}}%
\pgfpathlineto{\pgfqpoint{4.647190in}{2.231128in}}%
\pgfpathlineto{\pgfqpoint{4.647912in}{2.453557in}}%
\pgfpathlineto{\pgfqpoint{4.648222in}{2.353015in}}%
\pgfpathlineto{\pgfqpoint{4.648273in}{2.102067in}}%
\pgfpathlineto{\pgfqpoint{4.648686in}{2.497949in}}%
\pgfpathlineto{\pgfqpoint{4.649303in}{2.454499in}}%
\pgfpathlineto{\pgfqpoint{4.649663in}{2.516256in}}%
\pgfpathlineto{\pgfqpoint{4.650433in}{2.182781in}}%
\pgfpathlineto{\pgfqpoint{4.651252in}{2.571965in}}%
\pgfpathlineto{\pgfqpoint{4.651559in}{2.347158in}}%
\pgfpathlineto{\pgfqpoint{4.652479in}{2.146367in}}%
\pgfpathlineto{\pgfqpoint{4.651866in}{2.495677in}}%
\pgfpathlineto{\pgfqpoint{4.652632in}{2.356903in}}%
\pgfpathlineto{\pgfqpoint{4.652989in}{2.253633in}}%
\pgfpathlineto{\pgfqpoint{4.652734in}{2.451798in}}%
\pgfpathlineto{\pgfqpoint{4.653346in}{2.442225in}}%
\pgfpathlineto{\pgfqpoint{4.654058in}{2.493882in}}%
\pgfpathlineto{\pgfqpoint{4.653702in}{2.242502in}}%
\pgfpathlineto{\pgfqpoint{4.654312in}{2.419698in}}%
\pgfpathlineto{\pgfqpoint{4.655023in}{2.228059in}}%
\pgfpathlineto{\pgfqpoint{4.655277in}{2.473282in}}%
\pgfpathlineto{\pgfqpoint{4.655378in}{2.432599in}}%
\pgfpathlineto{\pgfqpoint{4.655429in}{2.503079in}}%
\pgfpathlineto{\pgfqpoint{4.655530in}{1.899154in}}%
\pgfpathlineto{\pgfqpoint{4.656441in}{2.452649in}}%
\pgfpathlineto{\pgfqpoint{4.656896in}{2.501941in}}%
\pgfpathlineto{\pgfqpoint{4.657552in}{2.141296in}}%
\pgfpathlineto{\pgfqpoint{4.657855in}{2.514986in}}%
\pgfpathlineto{\pgfqpoint{4.658660in}{2.361538in}}%
\pgfpathlineto{\pgfqpoint{4.659263in}{2.234694in}}%
\pgfpathlineto{\pgfqpoint{4.658962in}{2.541084in}}%
\pgfpathlineto{\pgfqpoint{4.659715in}{2.382959in}}%
\pgfpathlineto{\pgfqpoint{4.660367in}{2.542666in}}%
\pgfpathlineto{\pgfqpoint{4.659866in}{2.282131in}}%
\pgfpathlineto{\pgfqpoint{4.660617in}{2.394949in}}%
\pgfpathlineto{\pgfqpoint{4.660868in}{2.138944in}}%
\pgfpathlineto{\pgfqpoint{4.661318in}{2.516390in}}%
\pgfpathlineto{\pgfqpoint{4.661767in}{2.249642in}}%
\pgfpathlineto{\pgfqpoint{4.662117in}{2.494441in}}%
\pgfpathlineto{\pgfqpoint{4.662167in}{2.135938in}}%
\pgfpathlineto{\pgfqpoint{4.662864in}{2.449805in}}%
\pgfpathlineto{\pgfqpoint{4.663760in}{2.212371in}}%
\pgfpathlineto{\pgfqpoint{4.663064in}{2.516671in}}%
\pgfpathlineto{\pgfqpoint{4.663959in}{2.322384in}}%
\pgfpathlineto{\pgfqpoint{4.664008in}{2.321078in}}%
\pgfpathlineto{\pgfqpoint{4.664108in}{2.507392in}}%
\pgfpathlineto{\pgfqpoint{4.664207in}{1.920826in}}%
\pgfpathlineto{\pgfqpoint{4.665149in}{2.443785in}}%
\pgfpathlineto{\pgfqpoint{4.665199in}{2.501736in}}%
\pgfpathlineto{\pgfqpoint{4.665694in}{2.274277in}}%
\pgfpathlineto{\pgfqpoint{4.666040in}{2.475962in}}%
\pgfpathlineto{\pgfqpoint{4.666089in}{2.231462in}}%
\pgfpathlineto{\pgfqpoint{4.666435in}{2.499388in}}%
\pgfpathlineto{\pgfqpoint{4.667126in}{2.440434in}}%
\pgfpathlineto{\pgfqpoint{4.668062in}{2.128460in}}%
\pgfpathlineto{\pgfqpoint{4.667865in}{2.503401in}}%
\pgfpathlineto{\pgfqpoint{4.668406in}{2.219461in}}%
\pgfpathlineto{\pgfqpoint{4.669241in}{2.507337in}}%
\pgfpathlineto{\pgfqpoint{4.669044in}{2.141083in}}%
\pgfpathlineto{\pgfqpoint{4.669535in}{2.453750in}}%
\pgfpathlineto{\pgfqpoint{4.670074in}{2.241918in}}%
\pgfpathlineto{\pgfqpoint{4.669682in}{2.513488in}}%
\pgfpathlineto{\pgfqpoint{4.670661in}{2.400478in}}%
\pgfpathlineto{\pgfqpoint{4.671052in}{2.482951in}}%
\pgfpathlineto{\pgfqpoint{4.671296in}{2.187376in}}%
\pgfpathlineto{\pgfqpoint{4.671638in}{2.430136in}}%
\pgfpathlineto{\pgfqpoint{4.671687in}{2.135894in}}%
\pgfpathlineto{\pgfqpoint{4.672515in}{2.509579in}}%
\pgfpathlineto{\pgfqpoint{4.672710in}{2.440021in}}%
\pgfpathlineto{\pgfqpoint{4.673147in}{2.480764in}}%
\pgfpathlineto{\pgfqpoint{4.673050in}{2.237142in}}%
\pgfpathlineto{\pgfqpoint{4.673585in}{2.352248in}}%
\pgfpathlineto{\pgfqpoint{4.674409in}{2.146729in}}%
\pgfpathlineto{\pgfqpoint{4.673876in}{2.492172in}}%
\pgfpathlineto{\pgfqpoint{4.674652in}{2.388185in}}%
\pgfpathlineto{\pgfqpoint{4.674700in}{2.560131in}}%
\pgfpathlineto{\pgfqpoint{4.674942in}{2.090543in}}%
\pgfpathlineto{\pgfqpoint{4.675668in}{2.364491in}}%
\pgfpathlineto{\pgfqpoint{4.675716in}{2.059330in}}%
\pgfpathlineto{\pgfqpoint{4.675764in}{2.525001in}}%
\pgfpathlineto{\pgfqpoint{4.676729in}{2.313066in}}%
\pgfpathlineto{\pgfqpoint{4.676778in}{2.547182in}}%
\pgfpathlineto{\pgfqpoint{4.677740in}{2.166593in}}%
\pgfpathlineto{\pgfqpoint{4.677836in}{2.389132in}}%
\pgfpathlineto{\pgfqpoint{4.678317in}{2.527359in}}%
\pgfpathlineto{\pgfqpoint{4.678077in}{2.140808in}}%
\pgfpathlineto{\pgfqpoint{4.678941in}{2.417706in}}%
\pgfpathlineto{\pgfqpoint{4.678989in}{2.043201in}}%
\pgfpathlineto{\pgfqpoint{4.679324in}{2.520596in}}%
\pgfpathlineto{\pgfqpoint{4.680042in}{2.441964in}}%
\pgfpathlineto{\pgfqpoint{4.680090in}{2.443925in}}%
\pgfpathlineto{\pgfqpoint{4.680472in}{2.195675in}}%
\pgfpathlineto{\pgfqpoint{4.680664in}{2.499747in}}%
\pgfpathlineto{\pgfqpoint{4.681236in}{2.263013in}}%
\pgfpathlineto{\pgfqpoint{4.682046in}{2.216708in}}%
\pgfpathlineto{\pgfqpoint{4.681475in}{2.520966in}}%
\pgfpathlineto{\pgfqpoint{4.682237in}{2.267297in}}%
\pgfpathlineto{\pgfqpoint{4.682807in}{2.488861in}}%
\pgfpathlineto{\pgfqpoint{4.683329in}{2.331122in}}%
\pgfpathlineto{\pgfqpoint{4.683472in}{2.234443in}}%
\pgfpathlineto{\pgfqpoint{4.683567in}{2.458022in}}%
\pgfpathlineto{\pgfqpoint{4.683709in}{2.452502in}}%
\pgfpathlineto{\pgfqpoint{4.683756in}{1.814251in}}%
\pgfpathlineto{\pgfqpoint{4.683804in}{2.492901in}}%
\pgfpathlineto{\pgfqpoint{4.684798in}{2.288852in}}%
\pgfpathlineto{\pgfqpoint{4.685271in}{2.530529in}}%
\pgfpathlineto{\pgfqpoint{4.684940in}{2.170488in}}%
\pgfpathlineto{\pgfqpoint{4.685932in}{2.481797in}}%
\pgfpathlineto{\pgfqpoint{4.686591in}{2.168962in}}%
\pgfpathlineto{\pgfqpoint{4.686780in}{2.536313in}}%
\pgfpathlineto{\pgfqpoint{4.687062in}{2.312974in}}%
\pgfpathlineto{\pgfqpoint{4.687438in}{2.514350in}}%
\pgfpathlineto{\pgfqpoint{4.687532in}{1.781761in}}%
\pgfpathlineto{\pgfqpoint{4.688190in}{2.485047in}}%
\pgfpathlineto{\pgfqpoint{4.688705in}{2.264933in}}%
\pgfpathlineto{\pgfqpoint{4.688940in}{2.524024in}}%
\pgfpathlineto{\pgfqpoint{4.689314in}{2.452909in}}%
\pgfpathlineto{\pgfqpoint{4.689922in}{2.119137in}}%
\pgfpathlineto{\pgfqpoint{4.689595in}{2.511548in}}%
\pgfpathlineto{\pgfqpoint{4.690389in}{2.315532in}}%
\pgfpathlineto{\pgfqpoint{4.690622in}{2.524914in}}%
\pgfpathlineto{\pgfqpoint{4.691368in}{2.222370in}}%
\pgfpathlineto{\pgfqpoint{4.691508in}{2.436981in}}%
\pgfpathlineto{\pgfqpoint{4.691554in}{2.461817in}}%
\pgfpathlineto{\pgfqpoint{4.691647in}{2.246321in}}%
\pgfpathlineto{\pgfqpoint{4.691787in}{2.391448in}}%
\pgfpathlineto{\pgfqpoint{4.691833in}{1.869730in}}%
\pgfpathlineto{\pgfqpoint{4.692623in}{2.498216in}}%
\pgfpathlineto{\pgfqpoint{4.692902in}{2.346442in}}%
\pgfpathlineto{\pgfqpoint{4.693690in}{2.525332in}}%
\pgfpathlineto{\pgfqpoint{4.693273in}{2.163781in}}%
\pgfpathlineto{\pgfqpoint{4.694014in}{2.373055in}}%
\pgfpathlineto{\pgfqpoint{4.694060in}{2.503834in}}%
\pgfpathlineto{\pgfqpoint{4.694708in}{2.161709in}}%
\pgfpathlineto{\pgfqpoint{4.694985in}{2.265955in}}%
\pgfpathlineto{\pgfqpoint{4.695031in}{2.078385in}}%
\pgfpathlineto{\pgfqpoint{4.695769in}{2.527336in}}%
\pgfpathlineto{\pgfqpoint{4.696045in}{2.315971in}}%
\pgfpathlineto{\pgfqpoint{4.696184in}{2.540000in}}%
\pgfpathlineto{\pgfqpoint{4.696690in}{2.109464in}}%
\pgfpathlineto{\pgfqpoint{4.697241in}{2.433390in}}%
\pgfpathlineto{\pgfqpoint{4.697654in}{2.192311in}}%
\pgfpathlineto{\pgfqpoint{4.697700in}{2.486214in}}%
\pgfpathlineto{\pgfqpoint{4.698388in}{2.320187in}}%
\pgfpathlineto{\pgfqpoint{4.698571in}{2.483922in}}%
\pgfpathlineto{\pgfqpoint{4.698663in}{2.260156in}}%
\pgfpathlineto{\pgfqpoint{4.698846in}{2.329471in}}%
\pgfpathlineto{\pgfqpoint{4.699120in}{2.183587in}}%
\pgfpathlineto{\pgfqpoint{4.699211in}{2.523062in}}%
\pgfpathlineto{\pgfqpoint{4.699897in}{2.346132in}}%
\pgfpathlineto{\pgfqpoint{4.700398in}{2.522588in}}%
\pgfpathlineto{\pgfqpoint{4.700216in}{2.263025in}}%
\pgfpathlineto{\pgfqpoint{4.701036in}{2.424249in}}%
\pgfpathlineto{\pgfqpoint{4.701945in}{2.029863in}}%
\pgfpathlineto{\pgfqpoint{4.701673in}{2.520070in}}%
\pgfpathlineto{\pgfqpoint{4.702127in}{2.300066in}}%
\pgfpathlineto{\pgfqpoint{4.702445in}{2.469565in}}%
\pgfpathlineto{\pgfqpoint{4.703170in}{2.203465in}}%
\pgfpathlineto{\pgfqpoint{4.703261in}{2.443290in}}%
\pgfpathlineto{\pgfqpoint{4.703487in}{2.199747in}}%
\pgfpathlineto{\pgfqpoint{4.703713in}{2.488807in}}%
\pgfpathlineto{\pgfqpoint{4.704075in}{2.459412in}}%
\pgfpathlineto{\pgfqpoint{4.704346in}{2.535874in}}%
\pgfpathlineto{\pgfqpoint{4.704662in}{2.188291in}}%
\pgfpathlineto{\pgfqpoint{4.705023in}{2.500472in}}%
\pgfpathlineto{\pgfqpoint{4.705924in}{2.510993in}}%
\pgfpathlineto{\pgfqpoint{4.706104in}{2.184353in}}%
\pgfpathlineto{\pgfqpoint{4.706598in}{2.537047in}}%
\pgfpathlineto{\pgfqpoint{4.707182in}{2.435798in}}%
\pgfpathlineto{\pgfqpoint{4.707227in}{2.184904in}}%
\pgfpathlineto{\pgfqpoint{4.707541in}{2.528333in}}%
\pgfpathlineto{\pgfqpoint{4.708302in}{2.221087in}}%
\pgfpathlineto{\pgfqpoint{4.709062in}{2.518608in}}%
\pgfpathlineto{\pgfqpoint{4.708704in}{2.111659in}}%
\pgfpathlineto{\pgfqpoint{4.709374in}{2.421269in}}%
\pgfpathlineto{\pgfqpoint{4.710222in}{1.983493in}}%
\pgfpathlineto{\pgfqpoint{4.710444in}{2.497929in}}%
\pgfpathlineto{\pgfqpoint{4.710489in}{2.355909in}}%
\pgfpathlineto{\pgfqpoint{4.710578in}{2.378790in}}%
\pgfpathlineto{\pgfqpoint{4.710667in}{2.524065in}}%
\pgfpathlineto{\pgfqpoint{4.710978in}{2.244864in}}%
\pgfpathlineto{\pgfqpoint{4.711733in}{2.475713in}}%
\pgfpathlineto{\pgfqpoint{4.711822in}{2.142675in}}%
\pgfpathlineto{\pgfqpoint{4.712753in}{2.532166in}}%
\pgfpathlineto{\pgfqpoint{4.712842in}{2.348986in}}%
\pgfpathlineto{\pgfqpoint{4.713372in}{2.492261in}}%
\pgfpathlineto{\pgfqpoint{4.712974in}{2.033409in}}%
\pgfpathlineto{\pgfqpoint{4.713947in}{2.424008in}}%
\pgfpathlineto{\pgfqpoint{4.714168in}{2.493613in}}%
\pgfpathlineto{\pgfqpoint{4.714520in}{2.274926in}}%
\pgfpathlineto{\pgfqpoint{4.715357in}{2.514020in}}%
\pgfpathlineto{\pgfqpoint{4.715577in}{2.016308in}}%
\pgfpathlineto{\pgfqpoint{4.716544in}{2.524906in}}%
\pgfpathlineto{\pgfqpoint{4.716676in}{2.329018in}}%
\pgfpathlineto{\pgfqpoint{4.716719in}{2.015778in}}%
\pgfpathlineto{\pgfqpoint{4.717640in}{2.533880in}}%
\pgfpathlineto{\pgfqpoint{4.717771in}{2.438209in}}%
\pgfpathlineto{\pgfqpoint{4.718383in}{2.197461in}}%
\pgfpathlineto{\pgfqpoint{4.718165in}{2.502002in}}%
\pgfpathlineto{\pgfqpoint{4.718863in}{2.441965in}}%
\pgfpathlineto{\pgfqpoint{4.719082in}{2.337173in}}%
\pgfpathlineto{\pgfqpoint{4.718994in}{2.454926in}}%
\pgfpathlineto{\pgfqpoint{4.719343in}{2.341225in}}%
\pgfpathlineto{\pgfqpoint{4.719387in}{2.115906in}}%
\pgfpathlineto{\pgfqpoint{4.719823in}{2.551219in}}%
\pgfpathlineto{\pgfqpoint{4.720432in}{2.410811in}}%
\pgfpathlineto{\pgfqpoint{4.721040in}{2.522374in}}%
\pgfpathlineto{\pgfqpoint{4.720780in}{2.128196in}}%
\pgfpathlineto{\pgfqpoint{4.721474in}{2.464026in}}%
\pgfpathlineto{\pgfqpoint{4.721648in}{2.117716in}}%
\pgfpathlineto{\pgfqpoint{4.722081in}{2.541463in}}%
\pgfpathlineto{\pgfqpoint{4.722600in}{2.340083in}}%
\pgfpathlineto{\pgfqpoint{4.723206in}{2.498793in}}%
\pgfpathlineto{\pgfqpoint{4.722903in}{1.972497in}}%
\pgfpathlineto{\pgfqpoint{4.723724in}{2.452252in}}%
\pgfpathlineto{\pgfqpoint{4.723939in}{2.530723in}}%
\pgfpathlineto{\pgfqpoint{4.724801in}{2.237773in}}%
\pgfpathlineto{\pgfqpoint{4.725059in}{2.498377in}}%
\pgfpathlineto{\pgfqpoint{4.725403in}{2.200864in}}%
\pgfpathlineto{\pgfqpoint{4.725919in}{2.475935in}}%
\pgfpathlineto{\pgfqpoint{4.726219in}{2.265638in}}%
\pgfpathlineto{\pgfqpoint{4.726648in}{2.488826in}}%
\pgfpathlineto{\pgfqpoint{4.727076in}{2.389089in}}%
\pgfpathlineto{\pgfqpoint{4.727675in}{2.531757in}}%
\pgfpathlineto{\pgfqpoint{4.727803in}{2.156607in}}%
\pgfpathlineto{\pgfqpoint{4.728102in}{2.450600in}}%
\pgfpathlineto{\pgfqpoint{4.728657in}{2.070437in}}%
\pgfpathlineto{\pgfqpoint{4.728486in}{2.527527in}}%
\pgfpathlineto{\pgfqpoint{4.729211in}{2.404435in}}%
\pgfpathlineto{\pgfqpoint{4.729807in}{2.508061in}}%
\pgfpathlineto{\pgfqpoint{4.729552in}{2.143079in}}%
\pgfpathlineto{\pgfqpoint{4.730275in}{2.485354in}}%
\pgfpathlineto{\pgfqpoint{4.731335in}{2.084524in}}%
\pgfpathlineto{\pgfqpoint{4.730699in}{2.495481in}}%
\pgfpathlineto{\pgfqpoint{4.731378in}{2.387339in}}%
\pgfpathlineto{\pgfqpoint{4.731928in}{2.128645in}}%
\pgfpathlineto{\pgfqpoint{4.731674in}{2.468654in}}%
\pgfpathlineto{\pgfqpoint{4.732394in}{2.393157in}}%
\pgfpathlineto{\pgfqpoint{4.732901in}{2.515604in}}%
\pgfpathlineto{\pgfqpoint{4.732647in}{2.225776in}}%
\pgfpathlineto{\pgfqpoint{4.733365in}{2.402946in}}%
\pgfpathlineto{\pgfqpoint{4.733871in}{2.142991in}}%
\pgfpathlineto{\pgfqpoint{4.733576in}{2.525390in}}%
\pgfpathlineto{\pgfqpoint{4.734460in}{2.290592in}}%
\pgfpathlineto{\pgfqpoint{4.735175in}{2.471398in}}%
\pgfpathlineto{\pgfqpoint{4.735385in}{2.144685in}}%
\pgfpathlineto{\pgfqpoint{4.735553in}{2.276795in}}%
\pgfpathlineto{\pgfqpoint{4.736265in}{2.495035in}}%
\pgfpathlineto{\pgfqpoint{4.735930in}{2.124535in}}%
\pgfpathlineto{\pgfqpoint{4.736391in}{2.445931in}}%
\pgfpathlineto{\pgfqpoint{4.736433in}{1.981230in}}%
\pgfpathlineto{\pgfqpoint{4.736768in}{2.553840in}}%
\pgfpathlineto{\pgfqpoint{4.737478in}{2.455008in}}%
\pgfpathlineto{\pgfqpoint{4.737979in}{2.117215in}}%
\pgfpathlineto{\pgfqpoint{4.738563in}{2.128243in}}%
\pgfpathlineto{\pgfqpoint{4.738605in}{2.525919in}}%
\pgfpathlineto{\pgfqpoint{4.738896in}{2.023828in}}%
\pgfpathlineto{\pgfqpoint{4.739687in}{2.332453in}}%
\pgfpathlineto{\pgfqpoint{4.740517in}{2.503450in}}%
\pgfpathlineto{\pgfqpoint{4.740185in}{2.189288in}}%
\pgfpathlineto{\pgfqpoint{4.740807in}{2.424251in}}%
\pgfpathlineto{\pgfqpoint{4.740849in}{2.162251in}}%
\pgfpathlineto{\pgfqpoint{4.741883in}{2.467106in}}%
\pgfpathlineto{\pgfqpoint{4.742627in}{2.234510in}}%
\pgfpathlineto{\pgfqpoint{4.742544in}{2.511622in}}%
\pgfpathlineto{\pgfqpoint{4.742998in}{2.409878in}}%
\pgfpathlineto{\pgfqpoint{4.743081in}{2.524441in}}%
\pgfpathlineto{\pgfqpoint{4.743328in}{2.208904in}}%
\pgfpathlineto{\pgfqpoint{4.743987in}{2.347194in}}%
\pgfpathlineto{\pgfqpoint{4.744603in}{2.204879in}}%
\pgfpathlineto{\pgfqpoint{4.744192in}{2.492198in}}%
\pgfpathlineto{\pgfqpoint{4.745055in}{2.395043in}}%
\pgfpathlineto{\pgfqpoint{4.745096in}{2.464949in}}%
\pgfpathlineto{\pgfqpoint{4.745301in}{2.250770in}}%
\pgfpathlineto{\pgfqpoint{4.746120in}{2.443868in}}%
\pgfpathlineto{\pgfqpoint{4.746366in}{2.528776in}}%
\pgfpathlineto{\pgfqpoint{4.747265in}{1.921931in}}%
\pgfpathlineto{\pgfqpoint{4.748040in}{2.497564in}}%
\pgfpathlineto{\pgfqpoint{4.748406in}{2.417639in}}%
\pgfpathlineto{\pgfqpoint{4.749057in}{2.154241in}}%
\pgfpathlineto{\pgfqpoint{4.748813in}{2.470029in}}%
\pgfpathlineto{\pgfqpoint{4.749463in}{2.286570in}}%
\pgfpathlineto{\pgfqpoint{4.749747in}{2.499336in}}%
\pgfpathlineto{\pgfqpoint{4.750315in}{2.252734in}}%
\pgfpathlineto{\pgfqpoint{4.750599in}{2.466292in}}%
\pgfpathlineto{\pgfqpoint{4.751165in}{2.474541in}}%
\pgfpathlineto{\pgfqpoint{4.751812in}{2.129496in}}%
\pgfpathlineto{\pgfqpoint{4.752377in}{2.523981in}}%
\pgfpathlineto{\pgfqpoint{4.752498in}{2.107000in}}%
\pgfpathlineto{\pgfqpoint{4.752941in}{2.375709in}}%
\pgfpathlineto{\pgfqpoint{4.753907in}{2.256269in}}%
\pgfpathlineto{\pgfqpoint{4.754027in}{2.478622in}}%
\pgfpathlineto{\pgfqpoint{4.754790in}{2.482248in}}%
\pgfpathlineto{\pgfqpoint{4.755111in}{2.172848in}}%
\pgfpathlineto{\pgfqpoint{4.755711in}{2.493017in}}%
\pgfpathlineto{\pgfqpoint{4.755311in}{2.029461in}}%
\pgfpathlineto{\pgfqpoint{4.756231in}{2.478333in}}%
\pgfpathlineto{\pgfqpoint{4.757189in}{2.159975in}}%
\pgfpathlineto{\pgfqpoint{4.756631in}{2.530468in}}%
\pgfpathlineto{\pgfqpoint{4.757309in}{2.454132in}}%
\pgfpathlineto{\pgfqpoint{4.757867in}{2.553424in}}%
\pgfpathlineto{\pgfqpoint{4.757747in}{2.179078in}}%
\pgfpathlineto{\pgfqpoint{4.757907in}{2.360572in}}%
\pgfpathlineto{\pgfqpoint{4.758742in}{2.154788in}}%
\pgfpathlineto{\pgfqpoint{4.758901in}{2.532666in}}%
\pgfpathlineto{\pgfqpoint{4.758980in}{2.417310in}}%
\pgfpathlineto{\pgfqpoint{4.759377in}{2.527901in}}%
\pgfpathlineto{\pgfqpoint{4.759179in}{2.157694in}}%
\pgfpathlineto{\pgfqpoint{4.759774in}{2.403585in}}%
\pgfpathlineto{\pgfqpoint{4.760803in}{2.127189in}}%
\pgfpathlineto{\pgfqpoint{4.760328in}{2.551340in}}%
\pgfpathlineto{\pgfqpoint{4.760882in}{2.307937in}}%
\pgfpathlineto{\pgfqpoint{4.761869in}{2.512963in}}%
\pgfpathlineto{\pgfqpoint{4.761790in}{2.172614in}}%
\pgfpathlineto{\pgfqpoint{4.761987in}{2.312238in}}%
\pgfpathlineto{\pgfqpoint{4.762972in}{2.500819in}}%
\pgfpathlineto{\pgfqpoint{4.762105in}{2.122636in}}%
\pgfpathlineto{\pgfqpoint{4.763090in}{2.440667in}}%
\pgfpathlineto{\pgfqpoint{4.764033in}{2.137012in}}%
\pgfpathlineto{\pgfqpoint{4.763876in}{2.506991in}}%
\pgfpathlineto{\pgfqpoint{4.764189in}{2.409920in}}%
\pgfpathlineto{\pgfqpoint{4.764660in}{2.530357in}}%
\pgfpathlineto{\pgfqpoint{4.765169in}{2.215172in}}%
\pgfpathlineto{\pgfqpoint{4.765247in}{2.410600in}}%
\pgfpathlineto{\pgfqpoint{4.766263in}{2.088741in}}%
\pgfpathlineto{\pgfqpoint{4.766224in}{2.503849in}}%
\pgfpathlineto{\pgfqpoint{4.766419in}{2.170176in}}%
\pgfpathlineto{\pgfqpoint{4.767043in}{2.512315in}}%
\pgfpathlineto{\pgfqpoint{4.767316in}{2.169092in}}%
\pgfpathlineto{\pgfqpoint{4.767510in}{2.485064in}}%
\pgfpathlineto{\pgfqpoint{4.768405in}{1.949963in}}%
\pgfpathlineto{\pgfqpoint{4.768599in}{2.427521in}}%
\pgfpathlineto{\pgfqpoint{4.769297in}{2.498317in}}%
\pgfpathlineto{\pgfqpoint{4.769258in}{2.001982in}}%
\pgfpathlineto{\pgfqpoint{4.769529in}{2.381534in}}%
\pgfpathlineto{\pgfqpoint{4.770071in}{2.133889in}}%
\pgfpathlineto{\pgfqpoint{4.769839in}{2.547646in}}%
\pgfpathlineto{\pgfqpoint{4.770613in}{2.526407in}}%
\pgfpathlineto{\pgfqpoint{4.771076in}{2.187212in}}%
\pgfpathlineto{\pgfqpoint{4.771731in}{2.297515in}}%
\pgfpathlineto{\pgfqpoint{4.772655in}{2.544270in}}%
\pgfpathlineto{\pgfqpoint{4.772463in}{2.098954in}}%
\pgfpathlineto{\pgfqpoint{4.772809in}{2.475989in}}%
\pgfpathlineto{\pgfqpoint{4.773846in}{2.181241in}}%
\pgfpathlineto{\pgfqpoint{4.772886in}{2.570773in}}%
\pgfpathlineto{\pgfqpoint{4.773922in}{2.390226in}}%
\pgfpathlineto{\pgfqpoint{4.773999in}{2.481826in}}%
\pgfpathlineto{\pgfqpoint{4.774803in}{2.285533in}}%
\pgfpathlineto{\pgfqpoint{4.775262in}{2.035550in}}%
\pgfpathlineto{\pgfqpoint{4.775415in}{2.545767in}}%
\pgfpathlineto{\pgfqpoint{4.775606in}{2.303443in}}%
\pgfpathlineto{\pgfqpoint{4.775911in}{2.526311in}}%
\pgfpathlineto{\pgfqpoint{4.776597in}{2.172655in}}%
\pgfpathlineto{\pgfqpoint{4.776711in}{2.333947in}}%
\pgfpathlineto{\pgfqpoint{4.777092in}{2.545628in}}%
\pgfpathlineto{\pgfqpoint{4.777282in}{2.222434in}}%
\pgfpathlineto{\pgfqpoint{4.777776in}{2.371106in}}%
\pgfpathlineto{\pgfqpoint{4.777890in}{2.147532in}}%
\pgfpathlineto{\pgfqpoint{4.778498in}{2.529229in}}%
\pgfpathlineto{\pgfqpoint{4.778801in}{2.427865in}}%
\pgfpathlineto{\pgfqpoint{4.779218in}{2.464977in}}%
\pgfpathlineto{\pgfqpoint{4.779407in}{2.118623in}}%
\pgfpathlineto{\pgfqpoint{4.779634in}{2.341750in}}%
\pgfpathlineto{\pgfqpoint{4.780238in}{2.189872in}}%
\pgfpathlineto{\pgfqpoint{4.780125in}{2.508309in}}%
\pgfpathlineto{\pgfqpoint{4.780691in}{2.433171in}}%
\pgfpathlineto{\pgfqpoint{4.780918in}{2.505171in}}%
\pgfpathlineto{\pgfqpoint{4.781295in}{2.199986in}}%
\pgfpathlineto{\pgfqpoint{4.781596in}{2.422324in}}%
\pgfpathlineto{\pgfqpoint{4.782574in}{2.093085in}}%
\pgfpathlineto{\pgfqpoint{4.782461in}{2.491877in}}%
\pgfpathlineto{\pgfqpoint{4.782724in}{2.279867in}}%
\pgfpathlineto{\pgfqpoint{4.782836in}{2.516479in}}%
\pgfpathlineto{\pgfqpoint{4.782874in}{2.234230in}}%
\pgfpathlineto{\pgfqpoint{4.783736in}{2.463201in}}%
\pgfpathlineto{\pgfqpoint{4.783774in}{1.970117in}}%
\pgfpathlineto{\pgfqpoint{4.784148in}{2.509198in}}%
\pgfpathlineto{\pgfqpoint{4.784821in}{2.144671in}}%
\pgfpathlineto{\pgfqpoint{4.785866in}{2.495588in}}%
\pgfpathlineto{\pgfqpoint{4.785941in}{2.398706in}}%
\pgfpathlineto{\pgfqpoint{4.786723in}{2.081662in}}%
\pgfpathlineto{\pgfqpoint{4.786090in}{2.539349in}}%
\pgfpathlineto{\pgfqpoint{4.787020in}{2.351246in}}%
\pgfpathlineto{\pgfqpoint{4.787317in}{2.516692in}}%
\pgfpathlineto{\pgfqpoint{4.787651in}{2.216218in}}%
\pgfpathlineto{\pgfqpoint{4.787985in}{2.378305in}}%
\pgfpathlineto{\pgfqpoint{4.788800in}{2.118932in}}%
\pgfpathlineto{\pgfqpoint{4.788171in}{2.513812in}}%
\pgfpathlineto{\pgfqpoint{4.789059in}{2.233047in}}%
\pgfpathlineto{\pgfqpoint{4.790020in}{2.502383in}}%
\pgfpathlineto{\pgfqpoint{4.789872in}{2.111436in}}%
\pgfpathlineto{\pgfqpoint{4.790168in}{2.493879in}}%
\pgfpathlineto{\pgfqpoint{4.790684in}{2.148597in}}%
\pgfpathlineto{\pgfqpoint{4.791273in}{2.504739in}}%
\pgfpathlineto{\pgfqpoint{4.791457in}{1.876994in}}%
\pgfpathlineto{\pgfqpoint{4.792155in}{2.516391in}}%
\pgfpathlineto{\pgfqpoint{4.792412in}{2.286538in}}%
\pgfpathlineto{\pgfqpoint{4.793292in}{2.513505in}}%
\pgfpathlineto{\pgfqpoint{4.793146in}{2.229250in}}%
\pgfpathlineto{\pgfqpoint{4.793549in}{2.476659in}}%
\pgfpathlineto{\pgfqpoint{4.793622in}{2.491261in}}%
\pgfpathlineto{\pgfqpoint{4.794609in}{2.142507in}}%
\pgfpathlineto{\pgfqpoint{4.794937in}{2.504626in}}%
\pgfpathlineto{\pgfqpoint{4.795739in}{2.360747in}}%
\pgfpathlineto{\pgfqpoint{4.796321in}{2.520107in}}%
\pgfpathlineto{\pgfqpoint{4.796503in}{2.181847in}}%
\pgfpathlineto{\pgfqpoint{4.796866in}{2.464072in}}%
\pgfpathlineto{\pgfqpoint{4.797084in}{2.121886in}}%
\pgfpathlineto{\pgfqpoint{4.797556in}{2.530569in}}%
\pgfpathlineto{\pgfqpoint{4.797991in}{2.363314in}}%
\pgfpathlineto{\pgfqpoint{4.798353in}{2.552946in}}%
\pgfpathlineto{\pgfqpoint{4.798714in}{2.165241in}}%
\pgfpathlineto{\pgfqpoint{4.799112in}{2.507966in}}%
\pgfpathlineto{\pgfqpoint{4.800086in}{2.195254in}}%
\pgfpathlineto{\pgfqpoint{4.799581in}{2.524375in}}%
\pgfpathlineto{\pgfqpoint{4.800194in}{2.389070in}}%
\pgfpathlineto{\pgfqpoint{4.800950in}{2.515193in}}%
\pgfpathlineto{\pgfqpoint{4.800266in}{2.097890in}}%
\pgfpathlineto{\pgfqpoint{4.801274in}{2.378618in}}%
\pgfpathlineto{\pgfqpoint{4.801813in}{2.132729in}}%
\pgfpathlineto{\pgfqpoint{4.801849in}{2.523138in}}%
\pgfpathlineto{\pgfqpoint{4.802279in}{2.345574in}}%
\pgfpathlineto{\pgfqpoint{4.802996in}{2.487528in}}%
\pgfpathlineto{\pgfqpoint{4.802458in}{2.247642in}}%
\pgfpathlineto{\pgfqpoint{4.803389in}{2.399046in}}%
\pgfpathlineto{\pgfqpoint{4.803711in}{2.069447in}}%
\pgfpathlineto{\pgfqpoint{4.804425in}{2.475765in}}%
\pgfpathlineto{\pgfqpoint{4.804496in}{2.274037in}}%
\pgfpathlineto{\pgfqpoint{4.804675in}{2.512770in}}%
\pgfpathlineto{\pgfqpoint{4.804639in}{2.101636in}}%
\pgfpathlineto{\pgfqpoint{4.805601in}{2.401482in}}%
\pgfpathlineto{\pgfqpoint{4.806027in}{2.081935in}}%
\pgfpathlineto{\pgfqpoint{4.805779in}{2.509147in}}%
\pgfpathlineto{\pgfqpoint{4.806489in}{2.355389in}}%
\pgfpathlineto{\pgfqpoint{4.806667in}{2.532263in}}%
\pgfpathlineto{\pgfqpoint{4.806844in}{2.203929in}}%
\pgfpathlineto{\pgfqpoint{4.807588in}{2.352149in}}%
\pgfpathlineto{\pgfqpoint{4.808049in}{2.520060in}}%
\pgfpathlineto{\pgfqpoint{4.808437in}{2.147559in}}%
\pgfpathlineto{\pgfqpoint{4.808649in}{2.469339in}}%
\pgfpathlineto{\pgfqpoint{4.809461in}{2.173335in}}%
\pgfpathlineto{\pgfqpoint{4.808791in}{2.503011in}}%
\pgfpathlineto{\pgfqpoint{4.809743in}{2.498153in}}%
\pgfpathlineto{\pgfqpoint{4.810693in}{2.222845in}}%
\pgfpathlineto{\pgfqpoint{4.810728in}{2.503080in}}%
\pgfpathlineto{\pgfqpoint{4.810834in}{2.447199in}}%
\pgfpathlineto{\pgfqpoint{4.810869in}{2.556749in}}%
\pgfpathlineto{\pgfqpoint{4.811185in}{2.072699in}}%
\pgfpathlineto{\pgfqpoint{4.811887in}{2.280045in}}%
\pgfpathlineto{\pgfqpoint{4.811957in}{2.447885in}}%
\pgfpathlineto{\pgfqpoint{4.812237in}{2.164713in}}%
\pgfpathlineto{\pgfqpoint{4.812272in}{2.523621in}}%
\pgfpathlineto{\pgfqpoint{4.813356in}{2.425145in}}%
\pgfpathlineto{\pgfqpoint{4.813845in}{2.484283in}}%
\pgfpathlineto{\pgfqpoint{4.813461in}{2.276387in}}%
\pgfpathlineto{\pgfqpoint{4.813985in}{2.419741in}}%
\pgfpathlineto{\pgfqpoint{4.814925in}{2.127700in}}%
\pgfpathlineto{\pgfqpoint{4.814368in}{2.510731in}}%
\pgfpathlineto{\pgfqpoint{4.815065in}{2.164211in}}%
\pgfpathlineto{\pgfqpoint{4.815134in}{2.517319in}}%
\pgfpathlineto{\pgfqpoint{4.815308in}{2.162208in}}%
\pgfpathlineto{\pgfqpoint{4.816177in}{2.389607in}}%
\pgfpathlineto{\pgfqpoint{4.817009in}{2.501677in}}%
\pgfpathlineto{\pgfqpoint{4.816627in}{2.155075in}}%
\pgfpathlineto{\pgfqpoint{4.817216in}{2.493092in}}%
\pgfpathlineto{\pgfqpoint{4.818046in}{2.102417in}}%
\pgfpathlineto{\pgfqpoint{4.817943in}{2.563003in}}%
\pgfpathlineto{\pgfqpoint{4.818357in}{2.244559in}}%
\pgfpathlineto{\pgfqpoint{4.818495in}{2.519712in}}%
\pgfpathlineto{\pgfqpoint{4.818840in}{2.093889in}}%
\pgfpathlineto{\pgfqpoint{4.819460in}{2.438837in}}%
\pgfpathlineto{\pgfqpoint{4.820217in}{2.205933in}}%
\pgfpathlineto{\pgfqpoint{4.819667in}{2.556334in}}%
\pgfpathlineto{\pgfqpoint{4.820561in}{2.416154in}}%
\pgfpathlineto{\pgfqpoint{4.821179in}{2.534030in}}%
\pgfpathlineto{\pgfqpoint{4.821110in}{2.244279in}}%
\pgfpathlineto{\pgfqpoint{4.821624in}{2.423680in}}%
\pgfpathlineto{\pgfqpoint{4.822617in}{1.992378in}}%
\pgfpathlineto{\pgfqpoint{4.822411in}{2.519594in}}%
\pgfpathlineto{\pgfqpoint{4.822719in}{2.397221in}}%
\pgfpathlineto{\pgfqpoint{4.823402in}{2.552314in}}%
\pgfpathlineto{\pgfqpoint{4.823368in}{2.066571in}}%
\pgfpathlineto{\pgfqpoint{4.823709in}{2.495827in}}%
\pgfpathlineto{\pgfqpoint{4.824118in}{2.137508in}}%
\pgfpathlineto{\pgfqpoint{4.824696in}{2.524499in}}%
\pgfpathlineto{\pgfqpoint{4.824832in}{2.350770in}}%
\pgfpathlineto{\pgfqpoint{4.825546in}{1.910190in}}%
\pgfpathlineto{\pgfqpoint{4.825206in}{2.541286in}}%
\pgfpathlineto{\pgfqpoint{4.825885in}{2.401524in}}%
\pgfpathlineto{\pgfqpoint{4.825919in}{2.400407in}}%
\pgfpathlineto{\pgfqpoint{4.825953in}{2.431126in}}%
\pgfpathlineto{\pgfqpoint{4.826461in}{2.179109in}}%
\pgfpathlineto{\pgfqpoint{4.826732in}{2.533332in}}%
\pgfpathlineto{\pgfqpoint{4.827071in}{2.378250in}}%
\pgfpathlineto{\pgfqpoint{4.827882in}{2.118693in}}%
\pgfpathlineto{\pgfqpoint{4.827915in}{2.522959in}}%
\pgfpathlineto{\pgfqpoint{4.828118in}{2.347062in}}%
\pgfpathlineto{\pgfqpoint{4.828185in}{2.504997in}}%
\pgfpathlineto{\pgfqpoint{4.828960in}{2.153046in}}%
\pgfpathlineto{\pgfqpoint{4.829162in}{2.441421in}}%
\pgfpathlineto{\pgfqpoint{4.829196in}{2.055822in}}%
\pgfpathlineto{\pgfqpoint{4.829902in}{2.531124in}}%
\pgfpathlineto{\pgfqpoint{4.830238in}{2.391992in}}%
\pgfpathlineto{\pgfqpoint{4.830406in}{2.550919in}}%
\pgfpathlineto{\pgfqpoint{4.830305in}{2.039932in}}%
\pgfpathlineto{\pgfqpoint{4.831311in}{2.459631in}}%
\pgfpathlineto{\pgfqpoint{4.832214in}{2.088731in}}%
\pgfpathlineto{\pgfqpoint{4.831579in}{2.532952in}}%
\pgfpathlineto{\pgfqpoint{4.832381in}{2.092220in}}%
\pgfpathlineto{\pgfqpoint{4.832815in}{2.544911in}}%
\pgfpathlineto{\pgfqpoint{4.833515in}{2.493147in}}%
\pgfpathlineto{\pgfqpoint{4.833548in}{2.493033in}}%
\pgfpathlineto{\pgfqpoint{4.833648in}{1.984660in}}%
\pgfpathlineto{\pgfqpoint{4.833748in}{2.509362in}}%
\pgfpathlineto{\pgfqpoint{4.834646in}{2.222542in}}%
\pgfpathlineto{\pgfqpoint{4.834779in}{2.522070in}}%
\pgfpathlineto{\pgfqpoint{4.835211in}{2.205235in}}%
\pgfpathlineto{\pgfqpoint{4.835774in}{2.397695in}}%
\pgfpathlineto{\pgfqpoint{4.836304in}{2.511984in}}%
\pgfpathlineto{\pgfqpoint{4.836271in}{2.332735in}}%
\pgfpathlineto{\pgfqpoint{4.836337in}{2.466636in}}%
\pgfpathlineto{\pgfqpoint{4.836767in}{1.959481in}}%
\pgfpathlineto{\pgfqpoint{4.836933in}{2.516532in}}%
\pgfpathlineto{\pgfqpoint{4.837461in}{2.113779in}}%
\pgfpathlineto{\pgfqpoint{4.837527in}{2.554834in}}%
\pgfpathlineto{\pgfqpoint{4.837692in}{2.050083in}}%
\pgfpathlineto{\pgfqpoint{4.838582in}{2.449820in}}%
\pgfpathlineto{\pgfqpoint{4.839568in}{2.036015in}}%
\pgfpathlineto{\pgfqpoint{4.838976in}{2.529855in}}%
\pgfpathlineto{\pgfqpoint{4.839699in}{2.318796in}}%
\pgfpathlineto{\pgfqpoint{4.840519in}{2.501354in}}%
\pgfpathlineto{\pgfqpoint{4.840224in}{2.168518in}}%
\pgfpathlineto{\pgfqpoint{4.840749in}{2.441932in}}%
\pgfpathlineto{\pgfqpoint{4.841305in}{2.139372in}}%
\pgfpathlineto{\pgfqpoint{4.841174in}{2.532128in}}%
\pgfpathlineto{\pgfqpoint{4.841861in}{2.385965in}}%
\pgfpathlineto{\pgfqpoint{4.842644in}{2.537027in}}%
\pgfpathlineto{\pgfqpoint{4.842252in}{2.160964in}}%
\pgfpathlineto{\pgfqpoint{4.842970in}{2.439402in}}%
\pgfpathlineto{\pgfqpoint{4.843002in}{2.440691in}}%
\pgfpathlineto{\pgfqpoint{4.843621in}{2.031836in}}%
\pgfpathlineto{\pgfqpoint{4.844043in}{2.496498in}}%
\pgfpathlineto{\pgfqpoint{4.844108in}{2.217081in}}%
\pgfpathlineto{\pgfqpoint{4.844303in}{2.511585in}}%
\pgfpathlineto{\pgfqpoint{4.844823in}{2.064654in}}%
\pgfpathlineto{\pgfqpoint{4.845244in}{2.415807in}}%
\pgfpathlineto{\pgfqpoint{4.845276in}{2.418950in}}%
\pgfpathlineto{\pgfqpoint{4.845341in}{2.202628in}}%
\pgfpathlineto{\pgfqpoint{4.846247in}{2.547402in}}%
\pgfpathlineto{\pgfqpoint{4.846377in}{2.432399in}}%
\pgfpathlineto{\pgfqpoint{4.847474in}{2.239223in}}%
\pgfpathlineto{\pgfqpoint{4.846570in}{2.574199in}}%
\pgfpathlineto{\pgfqpoint{4.847538in}{2.346875in}}%
\pgfpathlineto{\pgfqpoint{4.848504in}{2.513069in}}%
\pgfpathlineto{\pgfqpoint{4.848440in}{2.137261in}}%
\pgfpathlineto{\pgfqpoint{4.848665in}{2.455490in}}%
\pgfpathlineto{\pgfqpoint{4.849692in}{2.147284in}}%
\pgfpathlineto{\pgfqpoint{4.848954in}{2.538332in}}%
\pgfpathlineto{\pgfqpoint{4.849756in}{2.453353in}}%
\pgfpathlineto{\pgfqpoint{4.850493in}{2.081305in}}%
\pgfpathlineto{\pgfqpoint{4.850173in}{2.497559in}}%
\pgfpathlineto{\pgfqpoint{4.850781in}{2.398032in}}%
\pgfpathlineto{\pgfqpoint{4.850813in}{2.546582in}}%
\pgfpathlineto{\pgfqpoint{4.851548in}{2.217126in}}%
\pgfpathlineto{\pgfqpoint{4.851867in}{2.430943in}}%
\pgfpathlineto{\pgfqpoint{4.852855in}{2.182202in}}%
\pgfpathlineto{\pgfqpoint{4.852409in}{2.511239in}}%
\pgfpathlineto{\pgfqpoint{4.852951in}{2.190982in}}%
\pgfpathlineto{\pgfqpoint{4.853968in}{2.510634in}}%
\pgfpathlineto{\pgfqpoint{4.853650in}{2.126028in}}%
\pgfpathlineto{\pgfqpoint{4.854063in}{2.408564in}}%
\pgfpathlineto{\pgfqpoint{4.854507in}{2.075925in}}%
\pgfpathlineto{\pgfqpoint{4.854887in}{2.568106in}}%
\pgfpathlineto{\pgfqpoint{4.855172in}{2.418086in}}%
\pgfpathlineto{\pgfqpoint{4.855267in}{2.154076in}}%
\pgfpathlineto{\pgfqpoint{4.855584in}{2.513056in}}%
\pgfpathlineto{\pgfqpoint{4.856279in}{2.315222in}}%
\pgfpathlineto{\pgfqpoint{4.857382in}{2.578200in}}%
\pgfpathlineto{\pgfqpoint{4.857288in}{2.175564in}}%
\pgfpathlineto{\pgfqpoint{4.857414in}{2.512096in}}%
\pgfpathlineto{\pgfqpoint{4.857791in}{2.256836in}}%
\pgfpathlineto{\pgfqpoint{4.858137in}{2.548287in}}%
\pgfpathlineto{\pgfqpoint{4.858546in}{2.310142in}}%
\pgfpathlineto{\pgfqpoint{4.858922in}{2.535345in}}%
\pgfpathlineto{\pgfqpoint{4.859267in}{2.111060in}}%
\pgfpathlineto{\pgfqpoint{4.859675in}{2.413201in}}%
\pgfpathlineto{\pgfqpoint{4.860207in}{2.106007in}}%
\pgfpathlineto{\pgfqpoint{4.860144in}{2.509735in}}%
\pgfpathlineto{\pgfqpoint{4.860770in}{2.337122in}}%
\pgfpathlineto{\pgfqpoint{4.861363in}{2.477968in}}%
\pgfpathlineto{\pgfqpoint{4.861332in}{2.231426in}}%
\pgfpathlineto{\pgfqpoint{4.861862in}{2.366572in}}%
\pgfpathlineto{\pgfqpoint{4.862733in}{2.195168in}}%
\pgfpathlineto{\pgfqpoint{4.862360in}{2.520300in}}%
\pgfpathlineto{\pgfqpoint{4.862982in}{2.259827in}}%
\pgfpathlineto{\pgfqpoint{4.863975in}{2.509161in}}%
\pgfpathlineto{\pgfqpoint{4.863324in}{2.201747in}}%
\pgfpathlineto{\pgfqpoint{4.864099in}{2.383227in}}%
\pgfpathlineto{\pgfqpoint{4.864719in}{2.547373in}}%
\pgfpathlineto{\pgfqpoint{4.864161in}{2.259561in}}%
\pgfpathlineto{\pgfqpoint{4.865028in}{2.375052in}}%
\pgfpathlineto{\pgfqpoint{4.866017in}{2.231277in}}%
\pgfpathlineto{\pgfqpoint{4.865121in}{2.518045in}}%
\pgfpathlineto{\pgfqpoint{4.866140in}{2.266757in}}%
\pgfpathlineto{\pgfqpoint{4.866356in}{2.519452in}}%
\pgfpathlineto{\pgfqpoint{4.866602in}{2.106856in}}%
\pgfpathlineto{\pgfqpoint{4.867280in}{2.475658in}}%
\pgfpathlineto{\pgfqpoint{4.868263in}{2.101469in}}%
\pgfpathlineto{\pgfqpoint{4.867833in}{2.516949in}}%
\pgfpathlineto{\pgfqpoint{4.868447in}{2.244297in}}%
\pgfpathlineto{\pgfqpoint{4.868815in}{1.783448in}}%
\pgfpathlineto{\pgfqpoint{4.869550in}{2.527789in}}%
\pgfpathlineto{\pgfqpoint{4.869917in}{2.259102in}}%
\pgfpathlineto{\pgfqpoint{4.870437in}{2.573268in}}%
\pgfpathlineto{\pgfqpoint{4.870681in}{2.338625in}}%
\pgfpathlineto{\pgfqpoint{4.870864in}{2.529575in}}%
\pgfpathlineto{\pgfqpoint{4.871321in}{2.174366in}}%
\pgfpathlineto{\pgfqpoint{4.871382in}{2.316391in}}%
\pgfpathlineto{\pgfqpoint{4.871413in}{1.777068in}}%
\pgfpathlineto{\pgfqpoint{4.872417in}{2.504921in}}%
\pgfpathlineto{\pgfqpoint{4.872478in}{2.450924in}}%
\pgfpathlineto{\pgfqpoint{4.872782in}{2.165852in}}%
\pgfpathlineto{\pgfqpoint{4.873115in}{2.513934in}}%
\pgfpathlineto{\pgfqpoint{4.873601in}{2.322486in}}%
\pgfpathlineto{\pgfqpoint{4.874630in}{2.543472in}}%
\pgfpathlineto{\pgfqpoint{4.874600in}{2.143763in}}%
\pgfpathlineto{\pgfqpoint{4.874690in}{2.473713in}}%
\pgfpathlineto{\pgfqpoint{4.875053in}{1.978016in}}%
\pgfpathlineto{\pgfqpoint{4.875687in}{2.515544in}}%
\pgfpathlineto{\pgfqpoint{4.875808in}{2.430041in}}%
\pgfpathlineto{\pgfqpoint{4.876350in}{2.019949in}}%
\pgfpathlineto{\pgfqpoint{4.876139in}{2.557388in}}%
\pgfpathlineto{\pgfqpoint{4.876892in}{2.403839in}}%
\pgfpathlineto{\pgfqpoint{4.877703in}{2.180734in}}%
\pgfpathlineto{\pgfqpoint{4.878003in}{2.525678in}}%
\pgfpathlineto{\pgfqpoint{4.878842in}{1.971794in}}%
\pgfpathlineto{\pgfqpoint{4.878303in}{2.526439in}}%
\pgfpathlineto{\pgfqpoint{4.879111in}{2.243245in}}%
\pgfpathlineto{\pgfqpoint{4.880157in}{2.514589in}}%
\pgfpathlineto{\pgfqpoint{4.879590in}{2.206160in}}%
\pgfpathlineto{\pgfqpoint{4.880217in}{2.378527in}}%
\pgfpathlineto{\pgfqpoint{4.880933in}{2.194968in}}%
\pgfpathlineto{\pgfqpoint{4.880336in}{2.495760in}}%
\pgfpathlineto{\pgfqpoint{4.881171in}{2.380791in}}%
\pgfpathlineto{\pgfqpoint{4.881349in}{2.517985in}}%
\pgfpathlineto{\pgfqpoint{4.881558in}{2.095719in}}%
\pgfpathlineto{\pgfqpoint{4.882182in}{2.306296in}}%
\pgfpathlineto{\pgfqpoint{4.882212in}{2.089928in}}%
\pgfpathlineto{\pgfqpoint{4.883191in}{2.555606in}}%
\pgfpathlineto{\pgfqpoint{4.883309in}{2.178889in}}%
\pgfpathlineto{\pgfqpoint{4.883487in}{1.991799in}}%
\pgfpathlineto{\pgfqpoint{4.883872in}{2.523844in}}%
\pgfpathlineto{\pgfqpoint{4.884226in}{2.405211in}}%
\pgfpathlineto{\pgfqpoint{4.884758in}{2.181220in}}%
\pgfpathlineto{\pgfqpoint{4.885348in}{2.509996in}}%
\pgfpathlineto{\pgfqpoint{4.886202in}{2.130894in}}%
\pgfpathlineto{\pgfqpoint{4.886085in}{2.519859in}}%
\pgfpathlineto{\pgfqpoint{4.886496in}{2.443432in}}%
\pgfpathlineto{\pgfqpoint{4.886526in}{2.519027in}}%
\pgfpathlineto{\pgfqpoint{4.886790in}{2.216599in}}%
\pgfpathlineto{\pgfqpoint{4.887612in}{2.471477in}}%
\pgfpathlineto{\pgfqpoint{4.887642in}{2.469387in}}%
\pgfpathlineto{\pgfqpoint{4.887788in}{2.514321in}}%
\pgfpathlineto{\pgfqpoint{4.888725in}{2.186130in}}%
\pgfpathlineto{\pgfqpoint{4.889718in}{2.512385in}}%
\pgfpathlineto{\pgfqpoint{4.888959in}{2.163214in}}%
\pgfpathlineto{\pgfqpoint{4.889864in}{2.503108in}}%
\pgfpathlineto{\pgfqpoint{4.890214in}{2.132819in}}%
\pgfpathlineto{\pgfqpoint{4.890506in}{2.512589in}}%
\pgfpathlineto{\pgfqpoint{4.891146in}{2.451789in}}%
\pgfpathlineto{\pgfqpoint{4.891466in}{2.544181in}}%
\pgfpathlineto{\pgfqpoint{4.891204in}{2.187828in}}%
\pgfpathlineto{\pgfqpoint{4.892133in}{2.478825in}}%
\pgfpathlineto{\pgfqpoint{4.893177in}{2.139256in}}%
\pgfpathlineto{\pgfqpoint{4.892945in}{2.534340in}}%
\pgfpathlineto{\pgfqpoint{4.893235in}{2.476355in}}%
\pgfpathlineto{\pgfqpoint{4.893726in}{2.210812in}}%
\pgfpathlineto{\pgfqpoint{4.894160in}{2.540082in}}%
\pgfpathlineto{\pgfqpoint{4.894362in}{2.264462in}}%
\pgfpathlineto{\pgfqpoint{4.894419in}{2.495345in}}%
\pgfpathlineto{\pgfqpoint{4.895371in}{1.885160in}}%
\pgfpathlineto{\pgfqpoint{4.895486in}{2.476650in}}%
\pgfpathlineto{\pgfqpoint{4.895774in}{2.031988in}}%
\pgfpathlineto{\pgfqpoint{4.896061in}{2.558635in}}%
\pgfpathlineto{\pgfqpoint{4.896578in}{2.187079in}}%
\pgfpathlineto{\pgfqpoint{4.896779in}{2.558221in}}%
\pgfpathlineto{\pgfqpoint{4.897668in}{2.120175in}}%
\pgfpathlineto{\pgfqpoint{4.897697in}{2.374532in}}%
\pgfpathlineto{\pgfqpoint{4.898040in}{2.093703in}}%
\pgfpathlineto{\pgfqpoint{4.898641in}{2.526872in}}%
\pgfpathlineto{\pgfqpoint{4.898783in}{2.372142in}}%
\pgfpathlineto{\pgfqpoint{4.899753in}{2.528182in}}%
\pgfpathlineto{\pgfqpoint{4.898869in}{2.091712in}}%
\pgfpathlineto{\pgfqpoint{4.899896in}{2.516485in}}%
\pgfpathlineto{\pgfqpoint{4.900920in}{2.082529in}}%
\pgfpathlineto{\pgfqpoint{4.901034in}{2.374215in}}%
\pgfpathlineto{\pgfqpoint{4.901914in}{2.504885in}}%
\pgfpathlineto{\pgfqpoint{4.901857in}{2.204948in}}%
\pgfpathlineto{\pgfqpoint{4.902027in}{2.496657in}}%
\pgfpathlineto{\pgfqpoint{4.902056in}{1.961382in}}%
\pgfpathlineto{\pgfqpoint{4.902565in}{2.526316in}}%
\pgfpathlineto{\pgfqpoint{4.903131in}{2.247659in}}%
\pgfpathlineto{\pgfqpoint{4.903216in}{2.498875in}}%
\pgfpathlineto{\pgfqpoint{4.904007in}{2.050620in}}%
\pgfpathlineto{\pgfqpoint{4.904261in}{2.375576in}}%
\pgfpathlineto{\pgfqpoint{4.904796in}{2.569460in}}%
\pgfpathlineto{\pgfqpoint{4.904993in}{2.034554in}}%
\pgfpathlineto{\pgfqpoint{4.905359in}{2.485884in}}%
\pgfpathlineto{\pgfqpoint{4.905387in}{2.074305in}}%
\pgfpathlineto{\pgfqpoint{4.906230in}{2.518719in}}%
\pgfpathlineto{\pgfqpoint{4.906454in}{2.293360in}}%
\pgfpathlineto{\pgfqpoint{4.907491in}{2.529394in}}%
\pgfpathlineto{\pgfqpoint{4.906707in}{1.986575in}}%
\pgfpathlineto{\pgfqpoint{4.907575in}{2.400793in}}%
\pgfpathlineto{\pgfqpoint{4.908609in}{2.162608in}}%
\pgfpathlineto{\pgfqpoint{4.907799in}{2.528202in}}%
\pgfpathlineto{\pgfqpoint{4.908665in}{2.354808in}}%
\pgfpathlineto{\pgfqpoint{4.909139in}{2.199070in}}%
\pgfpathlineto{\pgfqpoint{4.909807in}{2.536246in}}%
\pgfpathlineto{\pgfqpoint{4.910419in}{2.213872in}}%
\pgfpathlineto{\pgfqpoint{4.910974in}{2.387259in}}%
\pgfpathlineto{\pgfqpoint{4.911252in}{2.520563in}}%
\pgfpathlineto{\pgfqpoint{4.911529in}{2.142673in}}%
\pgfpathlineto{\pgfqpoint{4.912111in}{2.496787in}}%
\pgfpathlineto{\pgfqpoint{4.912277in}{2.168306in}}%
\pgfpathlineto{\pgfqpoint{4.912940in}{2.522580in}}%
\pgfpathlineto{\pgfqpoint{4.913217in}{2.480720in}}%
\pgfpathlineto{\pgfqpoint{4.914209in}{2.134109in}}%
\pgfpathlineto{\pgfqpoint{4.913879in}{2.503047in}}%
\pgfpathlineto{\pgfqpoint{4.914347in}{2.243060in}}%
\pgfpathlineto{\pgfqpoint{4.915420in}{2.070960in}}%
\pgfpathlineto{\pgfqpoint{4.915502in}{2.521350in}}%
\pgfpathlineto{\pgfqpoint{4.916571in}{2.176778in}}%
\pgfpathlineto{\pgfqpoint{4.916380in}{2.528961in}}%
\pgfpathlineto{\pgfqpoint{4.916626in}{2.453878in}}%
\pgfpathlineto{\pgfqpoint{4.917556in}{2.290592in}}%
\pgfpathlineto{\pgfqpoint{4.917502in}{2.513067in}}%
\pgfpathlineto{\pgfqpoint{4.917693in}{2.347770in}}%
\pgfpathlineto{\pgfqpoint{4.918212in}{2.504005in}}%
\pgfpathlineto{\pgfqpoint{4.918184in}{1.962474in}}%
\pgfpathlineto{\pgfqpoint{4.918812in}{2.456724in}}%
\pgfpathlineto{\pgfqpoint{4.919138in}{2.094527in}}%
\pgfpathlineto{\pgfqpoint{4.919084in}{2.512327in}}%
\pgfpathlineto{\pgfqpoint{4.919900in}{2.332005in}}%
\pgfpathlineto{\pgfqpoint{4.920280in}{2.498791in}}%
\pgfpathlineto{\pgfqpoint{4.920633in}{2.071767in}}%
\pgfpathlineto{\pgfqpoint{4.920986in}{2.219042in}}%
\pgfpathlineto{\pgfqpoint{4.921636in}{2.495126in}}%
\pgfpathlineto{\pgfqpoint{4.921338in}{1.950438in}}%
\pgfpathlineto{\pgfqpoint{4.922177in}{2.407655in}}%
\pgfpathlineto{\pgfqpoint{4.922447in}{2.072723in}}%
\pgfpathlineto{\pgfqpoint{4.922339in}{2.505173in}}%
\pgfpathlineto{\pgfqpoint{4.923284in}{2.239642in}}%
\pgfpathlineto{\pgfqpoint{4.923364in}{2.127965in}}%
\pgfpathlineto{\pgfqpoint{4.924388in}{2.554694in}}%
\pgfpathlineto{\pgfqpoint{4.924871in}{2.084262in}}%
\pgfpathlineto{\pgfqpoint{4.925489in}{2.266306in}}%
\pgfpathlineto{\pgfqpoint{4.926426in}{2.555873in}}%
\pgfpathlineto{\pgfqpoint{4.926132in}{2.059471in}}%
\pgfpathlineto{\pgfqpoint{4.926560in}{2.276934in}}%
\pgfpathlineto{\pgfqpoint{4.926587in}{2.020725in}}%
\pgfpathlineto{\pgfqpoint{4.926908in}{2.544134in}}%
\pgfpathlineto{\pgfqpoint{4.927656in}{2.402455in}}%
\pgfpathlineto{\pgfqpoint{4.928722in}{2.058300in}}%
\pgfpathlineto{\pgfqpoint{4.928695in}{2.559144in}}%
\pgfpathlineto{\pgfqpoint{4.928775in}{2.365384in}}%
\pgfpathlineto{\pgfqpoint{4.928855in}{2.512939in}}%
\pgfpathlineto{\pgfqpoint{4.929174in}{2.119240in}}%
\pgfpathlineto{\pgfqpoint{4.929758in}{2.392331in}}%
\pgfpathlineto{\pgfqpoint{4.930713in}{2.111459in}}%
\pgfpathlineto{\pgfqpoint{4.930369in}{2.489947in}}%
\pgfpathlineto{\pgfqpoint{4.930846in}{2.449917in}}%
\pgfpathlineto{\pgfqpoint{4.930952in}{2.523841in}}%
\pgfpathlineto{\pgfqpoint{4.931031in}{2.061964in}}%
\pgfpathlineto{\pgfqpoint{4.931666in}{2.323251in}}%
\pgfpathlineto{\pgfqpoint{4.932115in}{2.177216in}}%
\pgfpathlineto{\pgfqpoint{4.932643in}{2.524772in}}%
\pgfpathlineto{\pgfqpoint{4.932748in}{2.432513in}}%
\pgfpathlineto{\pgfqpoint{4.933328in}{2.081380in}}%
\pgfpathlineto{\pgfqpoint{4.933223in}{2.538213in}}%
\pgfpathlineto{\pgfqpoint{4.933854in}{2.300921in}}%
\pgfpathlineto{\pgfqpoint{4.934327in}{2.501219in}}%
\pgfpathlineto{\pgfqpoint{4.934905in}{2.201318in}}%
\pgfpathlineto{\pgfqpoint{4.934984in}{2.432695in}}%
\pgfpathlineto{\pgfqpoint{4.935455in}{2.525425in}}%
\pgfpathlineto{\pgfqpoint{4.935167in}{2.060403in}}%
\pgfpathlineto{\pgfqpoint{4.935665in}{2.354122in}}%
\pgfpathlineto{\pgfqpoint{4.936319in}{2.059437in}}%
\pgfpathlineto{\pgfqpoint{4.936502in}{2.540629in}}%
\pgfpathlineto{\pgfqpoint{4.936763in}{2.208667in}}%
\pgfpathlineto{\pgfqpoint{4.937755in}{2.530181in}}%
\pgfpathlineto{\pgfqpoint{4.937129in}{2.202172in}}%
\pgfpathlineto{\pgfqpoint{4.937937in}{2.527192in}}%
\pgfpathlineto{\pgfqpoint{4.938562in}{2.135282in}}%
\pgfpathlineto{\pgfqpoint{4.938796in}{2.528939in}}%
\pgfpathlineto{\pgfqpoint{4.939081in}{2.317008in}}%
\pgfpathlineto{\pgfqpoint{4.940145in}{2.530374in}}%
\pgfpathlineto{\pgfqpoint{4.939782in}{2.169974in}}%
\pgfpathlineto{\pgfqpoint{4.940197in}{2.513696in}}%
\pgfpathlineto{\pgfqpoint{4.940249in}{2.162552in}}%
\pgfpathlineto{\pgfqpoint{4.941309in}{2.456858in}}%
\pgfpathlineto{\pgfqpoint{4.941903in}{2.025348in}}%
\pgfpathlineto{\pgfqpoint{4.941981in}{2.551324in}}%
\pgfpathlineto{\pgfqpoint{4.942419in}{2.260399in}}%
\pgfpathlineto{\pgfqpoint{4.942522in}{2.501097in}}%
\pgfpathlineto{\pgfqpoint{4.943140in}{2.199655in}}%
\pgfpathlineto{\pgfqpoint{4.943526in}{2.350807in}}%
\pgfpathlineto{\pgfqpoint{4.943963in}{2.194862in}}%
\pgfpathlineto{\pgfqpoint{4.943988in}{2.523483in}}%
\pgfpathlineto{\pgfqpoint{4.944604in}{2.419754in}}%
\pgfpathlineto{\pgfqpoint{4.945321in}{1.944593in}}%
\pgfpathlineto{\pgfqpoint{4.945731in}{2.502875in}}%
\pgfpathlineto{\pgfqpoint{4.946548in}{2.510587in}}%
\pgfpathlineto{\pgfqpoint{4.946854in}{2.065031in}}%
\pgfpathlineto{\pgfqpoint{4.947950in}{2.535106in}}%
\pgfpathlineto{\pgfqpoint{4.947721in}{2.035680in}}%
\pgfpathlineto{\pgfqpoint{4.947975in}{2.313109in}}%
\pgfpathlineto{\pgfqpoint{4.948433in}{2.528908in}}%
\pgfpathlineto{\pgfqpoint{4.948357in}{2.228554in}}%
\pgfpathlineto{\pgfqpoint{4.949093in}{2.377878in}}%
\pgfpathlineto{\pgfqpoint{4.949397in}{2.506200in}}%
\pgfpathlineto{\pgfqpoint{4.949372in}{2.183457in}}%
\pgfpathlineto{\pgfqpoint{4.950157in}{2.447551in}}%
\pgfpathlineto{\pgfqpoint{4.950890in}{2.090319in}}%
\pgfpathlineto{\pgfqpoint{4.950612in}{2.539441in}}%
\pgfpathlineto{\pgfqpoint{4.951244in}{2.471699in}}%
\pgfpathlineto{\pgfqpoint{4.952026in}{2.529447in}}%
\pgfpathlineto{\pgfqpoint{4.952000in}{2.178585in}}%
\pgfpathlineto{\pgfqpoint{4.952202in}{2.268693in}}%
\pgfpathlineto{\pgfqpoint{4.952630in}{2.173572in}}%
\pgfpathlineto{\pgfqpoint{4.952378in}{2.494379in}}%
\pgfpathlineto{\pgfqpoint{4.952831in}{2.426337in}}%
\pgfpathlineto{\pgfqpoint{4.953108in}{2.520235in}}%
\pgfpathlineto{\pgfqpoint{4.952982in}{2.120416in}}%
\pgfpathlineto{\pgfqpoint{4.953836in}{2.316972in}}%
\pgfpathlineto{\pgfqpoint{4.953861in}{2.137003in}}%
\pgfpathlineto{\pgfqpoint{4.954036in}{2.505260in}}%
\pgfpathlineto{\pgfqpoint{4.954913in}{2.475173in}}%
\pgfpathlineto{\pgfqpoint{4.955188in}{2.200778in}}%
\pgfpathlineto{\pgfqpoint{4.955488in}{2.531954in}}%
\pgfpathlineto{\pgfqpoint{4.955988in}{2.226675in}}%
\pgfpathlineto{\pgfqpoint{4.956013in}{2.545877in}}%
\pgfpathlineto{\pgfqpoint{4.956213in}{2.168254in}}%
\pgfpathlineto{\pgfqpoint{4.957110in}{2.492247in}}%
\pgfpathlineto{\pgfqpoint{4.957955in}{2.072797in}}%
\pgfpathlineto{\pgfqpoint{4.957781in}{2.537483in}}%
\pgfpathlineto{\pgfqpoint{4.958253in}{2.126747in}}%
\pgfpathlineto{\pgfqpoint{4.959369in}{2.532923in}}%
\pgfpathlineto{\pgfqpoint{4.959790in}{2.117449in}}%
\pgfpathlineto{\pgfqpoint{4.959567in}{2.560627in}}%
\pgfpathlineto{\pgfqpoint{4.960507in}{2.313356in}}%
\pgfpathlineto{\pgfqpoint{4.961370in}{2.493343in}}%
\pgfpathlineto{\pgfqpoint{4.961346in}{2.031544in}}%
\pgfpathlineto{\pgfqpoint{4.961641in}{2.412894in}}%
\pgfpathlineto{\pgfqpoint{4.961666in}{2.065186in}}%
\pgfpathlineto{\pgfqpoint{4.962404in}{2.535428in}}%
\pgfpathlineto{\pgfqpoint{4.962748in}{2.299884in}}%
\pgfpathlineto{\pgfqpoint{4.963288in}{2.513455in}}%
\pgfpathlineto{\pgfqpoint{4.963117in}{2.171867in}}%
\pgfpathlineto{\pgfqpoint{4.963828in}{2.444445in}}%
\pgfpathlineto{\pgfqpoint{4.964342in}{2.184942in}}%
\pgfpathlineto{\pgfqpoint{4.964562in}{2.530226in}}%
\pgfpathlineto{\pgfqpoint{4.964929in}{2.470245in}}%
\pgfpathlineto{\pgfqpoint{4.965100in}{1.984415in}}%
\pgfpathlineto{\pgfqpoint{4.965954in}{2.494308in}}%
\pgfpathlineto{\pgfqpoint{4.966101in}{2.349915in}}%
\pgfpathlineto{\pgfqpoint{4.966198in}{2.420096in}}%
\pgfpathlineto{\pgfqpoint{4.966271in}{2.336335in}}%
\pgfpathlineto{\pgfqpoint{4.966295in}{2.534137in}}%
\pgfpathlineto{\pgfqpoint{4.966928in}{2.207808in}}%
\pgfpathlineto{\pgfqpoint{4.967390in}{2.452170in}}%
\pgfpathlineto{\pgfqpoint{4.967415in}{2.537484in}}%
\pgfpathlineto{\pgfqpoint{4.967585in}{2.145818in}}%
\pgfpathlineto{\pgfqpoint{4.968458in}{2.454756in}}%
\pgfpathlineto{\pgfqpoint{4.969402in}{2.181201in}}%
\pgfpathlineto{\pgfqpoint{4.968918in}{2.509991in}}%
\pgfpathlineto{\pgfqpoint{4.969572in}{2.372240in}}%
\pgfpathlineto{\pgfqpoint{4.970586in}{2.549541in}}%
\pgfpathlineto{\pgfqpoint{4.970538in}{2.042167in}}%
\pgfpathlineto{\pgfqpoint{4.970682in}{2.459997in}}%
\pgfpathlineto{\pgfqpoint{4.970996in}{2.518526in}}%
\pgfpathlineto{\pgfqpoint{4.970730in}{2.173694in}}%
\pgfpathlineto{\pgfqpoint{4.971068in}{2.332256in}}%
\pgfpathlineto{\pgfqpoint{4.971092in}{2.136497in}}%
\pgfpathlineto{\pgfqpoint{4.971742in}{2.530478in}}%
\pgfpathlineto{\pgfqpoint{4.972150in}{2.413540in}}%
\pgfpathlineto{\pgfqpoint{4.973015in}{2.523202in}}%
\pgfpathlineto{\pgfqpoint{4.972823in}{2.025647in}}%
\pgfpathlineto{\pgfqpoint{4.973111in}{2.388469in}}%
\pgfpathlineto{\pgfqpoint{4.973901in}{2.015622in}}%
\pgfpathlineto{\pgfqpoint{4.973254in}{2.537802in}}%
\pgfpathlineto{\pgfqpoint{4.974212in}{2.432878in}}%
\pgfpathlineto{\pgfqpoint{4.975048in}{2.091461in}}%
\pgfpathlineto{\pgfqpoint{4.975215in}{2.502248in}}%
\pgfpathlineto{\pgfqpoint{4.975310in}{2.228281in}}%
\pgfpathlineto{\pgfqpoint{4.975644in}{2.537323in}}%
\pgfpathlineto{\pgfqpoint{4.975597in}{2.201884in}}%
\pgfpathlineto{\pgfqpoint{4.976430in}{2.356100in}}%
\pgfpathlineto{\pgfqpoint{4.977024in}{2.541005in}}%
\pgfpathlineto{\pgfqpoint{4.976477in}{2.245508in}}%
\pgfpathlineto{\pgfqpoint{4.977570in}{2.452396in}}%
\pgfpathlineto{\pgfqpoint{4.978139in}{2.111371in}}%
\pgfpathlineto{\pgfqpoint{4.978163in}{2.527615in}}%
\pgfpathlineto{\pgfqpoint{4.978684in}{2.317596in}}%
\pgfpathlineto{\pgfqpoint{4.979228in}{2.520204in}}%
\pgfpathlineto{\pgfqpoint{4.979582in}{2.219031in}}%
\pgfpathlineto{\pgfqpoint{4.979794in}{2.437227in}}%
\pgfpathlineto{\pgfqpoint{4.980407in}{2.039661in}}%
\pgfpathlineto{\pgfqpoint{4.979959in}{2.478898in}}%
\pgfpathlineto{\pgfqpoint{4.980902in}{2.351767in}}%
\pgfpathlineto{\pgfqpoint{4.981184in}{2.516469in}}%
\pgfpathlineto{\pgfqpoint{4.981655in}{2.152012in}}%
\pgfpathlineto{\pgfqpoint{4.982007in}{2.460947in}}%
\pgfpathlineto{\pgfqpoint{4.982804in}{2.150120in}}%
\pgfpathlineto{\pgfqpoint{4.982664in}{2.504393in}}%
\pgfpathlineto{\pgfqpoint{4.983132in}{2.334901in}}%
\pgfpathlineto{\pgfqpoint{4.983179in}{2.492698in}}%
\pgfpathlineto{\pgfqpoint{4.983670in}{2.029984in}}%
\pgfpathlineto{\pgfqpoint{4.984208in}{2.264161in}}%
\pgfpathlineto{\pgfqpoint{4.984955in}{2.242190in}}%
\pgfpathlineto{\pgfqpoint{4.984558in}{2.554241in}}%
\pgfpathlineto{\pgfqpoint{4.984978in}{2.392565in}}%
\pgfpathlineto{\pgfqpoint{4.985025in}{2.530955in}}%
\pgfpathlineto{\pgfqpoint{4.985723in}{1.873483in}}%
\pgfpathlineto{\pgfqpoint{4.986072in}{2.404548in}}%
\pgfpathlineto{\pgfqpoint{4.986583in}{2.009309in}}%
\pgfpathlineto{\pgfqpoint{4.986909in}{2.510969in}}%
\pgfpathlineto{\pgfqpoint{4.987071in}{2.334973in}}%
\pgfpathlineto{\pgfqpoint{4.987164in}{2.540199in}}%
\pgfpathlineto{\pgfqpoint{4.987789in}{2.080224in}}%
\pgfpathlineto{\pgfqpoint{4.988183in}{2.407625in}}%
\pgfpathlineto{\pgfqpoint{4.988461in}{2.495582in}}%
\pgfpathlineto{\pgfqpoint{4.988576in}{2.184403in}}%
\pgfpathlineto{\pgfqpoint{4.989223in}{2.377523in}}%
\pgfpathlineto{\pgfqpoint{4.989361in}{2.085303in}}%
\pgfpathlineto{\pgfqpoint{4.989684in}{2.525435in}}%
\pgfpathlineto{\pgfqpoint{4.990306in}{2.350394in}}%
\pgfpathlineto{\pgfqpoint{4.990513in}{2.552713in}}%
\pgfpathlineto{\pgfqpoint{4.991019in}{2.076096in}}%
\pgfpathlineto{\pgfqpoint{4.991410in}{2.414696in}}%
\pgfpathlineto{\pgfqpoint{4.992511in}{2.006699in}}%
\pgfpathlineto{\pgfqpoint{4.992121in}{2.538832in}}%
\pgfpathlineto{\pgfqpoint{4.992534in}{2.379842in}}%
\pgfpathlineto{\pgfqpoint{4.992831in}{2.509067in}}%
\pgfpathlineto{\pgfqpoint{4.992671in}{2.253174in}}%
\pgfpathlineto{\pgfqpoint{4.993494in}{2.454401in}}%
\pgfpathlineto{\pgfqpoint{4.993517in}{1.874654in}}%
\pgfpathlineto{\pgfqpoint{4.994316in}{2.560521in}}%
\pgfpathlineto{\pgfqpoint{4.994612in}{2.110945in}}%
\pgfpathlineto{\pgfqpoint{4.995045in}{2.508790in}}%
\pgfpathlineto{\pgfqpoint{4.995728in}{2.466444in}}%
\pgfpathlineto{\pgfqpoint{4.996250in}{2.135008in}}%
\pgfpathlineto{\pgfqpoint{4.996432in}{2.516486in}}%
\pgfpathlineto{\pgfqpoint{4.996953in}{2.323874in}}%
\pgfpathlineto{\pgfqpoint{4.997565in}{2.509905in}}%
\pgfpathlineto{\pgfqpoint{4.997203in}{2.158193in}}%
\pgfpathlineto{\pgfqpoint{4.998085in}{2.480642in}}%
\pgfpathlineto{\pgfqpoint{4.998108in}{2.140404in}}%
\pgfpathlineto{\pgfqpoint{4.998808in}{2.520180in}}%
\pgfpathlineto{\pgfqpoint{4.999191in}{2.421312in}}%
\pgfpathlineto{\pgfqpoint{4.999304in}{2.164267in}}%
\pgfpathlineto{\pgfqpoint{4.999281in}{2.465146in}}%
\pgfpathlineto{\pgfqpoint{5.000317in}{2.248295in}}%
\pgfpathlineto{\pgfqpoint{5.001013in}{2.544010in}}%
\pgfpathlineto{\pgfqpoint{5.000901in}{2.185265in}}%
\pgfpathlineto{\pgfqpoint{5.001440in}{2.366998in}}%
\pgfpathlineto{\pgfqpoint{5.001754in}{2.155746in}}%
\pgfpathlineto{\pgfqpoint{5.001843in}{2.538533in}}%
\pgfpathlineto{\pgfqpoint{5.002492in}{2.374879in}}%
\pgfpathlineto{\pgfqpoint{5.002984in}{2.047751in}}%
\pgfpathlineto{\pgfqpoint{5.003542in}{2.550130in}}%
\pgfpathlineto{\pgfqpoint{5.003565in}{2.218132in}}%
\pgfpathlineto{\pgfqpoint{5.004657in}{2.464773in}}%
\pgfpathlineto{\pgfqpoint{5.005191in}{2.114870in}}%
\pgfpathlineto{\pgfqpoint{5.005502in}{2.542437in}}%
\pgfpathlineto{\pgfqpoint{5.005768in}{2.393200in}}%
\pgfpathlineto{\pgfqpoint{5.005990in}{2.074497in}}%
\pgfpathlineto{\pgfqpoint{5.006212in}{2.580126in}}%
\pgfpathlineto{\pgfqpoint{5.006854in}{2.234685in}}%
\pgfpathlineto{\pgfqpoint{5.006965in}{2.507267in}}%
\pgfpathlineto{\pgfqpoint{5.007297in}{2.085328in}}%
\pgfpathlineto{\pgfqpoint{5.007982in}{2.424442in}}%
\pgfpathlineto{\pgfqpoint{5.008004in}{2.122585in}}%
\pgfpathlineto{\pgfqpoint{5.008313in}{2.509121in}}%
\pgfpathlineto{\pgfqpoint{5.009085in}{2.384252in}}%
\pgfpathlineto{\pgfqpoint{5.009239in}{2.504318in}}%
\pgfpathlineto{\pgfqpoint{5.009701in}{2.072630in}}%
\pgfpathlineto{\pgfqpoint{5.010163in}{2.387107in}}%
\pgfpathlineto{\pgfqpoint{5.011216in}{2.235233in}}%
\pgfpathlineto{\pgfqpoint{5.010777in}{2.540000in}}%
\pgfpathlineto{\pgfqpoint{5.011238in}{2.455733in}}%
\pgfpathlineto{\pgfqpoint{5.011260in}{2.566450in}}%
\pgfpathlineto{\pgfqpoint{5.012223in}{2.148229in}}%
\pgfpathlineto{\pgfqpoint{5.012332in}{2.419051in}}%
\pgfpathlineto{\pgfqpoint{5.012725in}{2.186747in}}%
\pgfpathlineto{\pgfqpoint{5.012529in}{2.514732in}}%
\pgfpathlineto{\pgfqpoint{5.013424in}{2.405105in}}%
\pgfpathlineto{\pgfqpoint{5.014251in}{2.537089in}}%
\pgfpathlineto{\pgfqpoint{5.014317in}{2.121206in}}%
\pgfpathlineto{\pgfqpoint{5.014469in}{2.476892in}}%
\pgfpathlineto{\pgfqpoint{5.015164in}{2.274522in}}%
\pgfpathlineto{\pgfqpoint{5.014947in}{2.517564in}}%
\pgfpathlineto{\pgfqpoint{5.015576in}{2.323028in}}%
\pgfpathlineto{\pgfqpoint{5.016573in}{2.510782in}}%
\pgfpathlineto{\pgfqpoint{5.016616in}{2.256419in}}%
\pgfpathlineto{\pgfqpoint{5.016703in}{2.459435in}}%
\pgfpathlineto{\pgfqpoint{5.016768in}{2.509337in}}%
\pgfpathlineto{\pgfqpoint{5.016833in}{2.391829in}}%
\pgfpathlineto{\pgfqpoint{5.016876in}{2.071218in}}%
\pgfpathlineto{\pgfqpoint{5.017826in}{2.532989in}}%
\pgfpathlineto{\pgfqpoint{5.017934in}{2.460871in}}%
\pgfpathlineto{\pgfqpoint{5.018753in}{2.169374in}}%
\pgfpathlineto{\pgfqpoint{5.018516in}{2.540169in}}%
\pgfpathlineto{\pgfqpoint{5.019033in}{2.337805in}}%
\pgfpathlineto{\pgfqpoint{5.019054in}{2.539781in}}%
\pgfpathlineto{\pgfqpoint{5.019420in}{2.058385in}}%
\pgfpathlineto{\pgfqpoint{5.020150in}{2.420814in}}%
\pgfpathlineto{\pgfqpoint{5.020901in}{2.212972in}}%
\pgfpathlineto{\pgfqpoint{5.020815in}{2.504450in}}%
\pgfpathlineto{\pgfqpoint{5.021265in}{2.306782in}}%
\pgfpathlineto{\pgfqpoint{5.021757in}{2.556490in}}%
\pgfpathlineto{\pgfqpoint{5.022034in}{2.003096in}}%
\pgfpathlineto{\pgfqpoint{5.022376in}{2.364217in}}%
\pgfpathlineto{\pgfqpoint{5.023144in}{1.969441in}}%
\pgfpathlineto{\pgfqpoint{5.022952in}{2.518885in}}%
\pgfpathlineto{\pgfqpoint{5.023485in}{2.362495in}}%
\pgfpathlineto{\pgfqpoint{5.024399in}{2.565904in}}%
\pgfpathlineto{\pgfqpoint{5.024335in}{2.152742in}}%
\pgfpathlineto{\pgfqpoint{5.024612in}{2.490940in}}%
\pgfpathlineto{\pgfqpoint{5.024633in}{2.526000in}}%
\pgfpathlineto{\pgfqpoint{5.024866in}{2.196827in}}%
\pgfpathlineto{\pgfqpoint{5.025587in}{2.347874in}}%
\pgfpathlineto{\pgfqpoint{5.025735in}{2.138777in}}%
\pgfpathlineto{\pgfqpoint{5.025693in}{2.568178in}}%
\pgfpathlineto{\pgfqpoint{5.026687in}{2.364019in}}%
\pgfpathlineto{\pgfqpoint{5.026772in}{2.511929in}}%
\pgfpathlineto{\pgfqpoint{5.027025in}{2.083685in}}%
\pgfpathlineto{\pgfqpoint{5.027806in}{2.382053in}}%
\pgfpathlineto{\pgfqpoint{5.028480in}{2.196176in}}%
\pgfpathlineto{\pgfqpoint{5.028732in}{2.554779in}}%
\pgfpathlineto{\pgfqpoint{5.028879in}{2.342199in}}%
\pgfpathlineto{\pgfqpoint{5.029824in}{2.524786in}}%
\pgfpathlineto{\pgfqpoint{5.029677in}{2.102997in}}%
\pgfpathlineto{\pgfqpoint{5.029950in}{2.321986in}}%
\pgfpathlineto{\pgfqpoint{5.030264in}{1.764172in}}%
\pgfpathlineto{\pgfqpoint{5.030285in}{2.526972in}}%
\pgfpathlineto{\pgfqpoint{5.031039in}{2.381160in}}%
\pgfpathlineto{\pgfqpoint{5.031603in}{2.536799in}}%
\pgfpathlineto{\pgfqpoint{5.031958in}{1.924808in}}%
\pgfpathlineto{\pgfqpoint{5.032104in}{2.299454in}}%
\pgfpathlineto{\pgfqpoint{5.032125in}{2.134216in}}%
\pgfpathlineto{\pgfqpoint{5.032917in}{2.518704in}}%
\pgfpathlineto{\pgfqpoint{5.033188in}{2.424327in}}%
\pgfpathlineto{\pgfqpoint{5.033292in}{2.014548in}}%
\pgfpathlineto{\pgfqpoint{5.033604in}{2.537141in}}%
\pgfpathlineto{\pgfqpoint{5.034289in}{2.377426in}}%
\pgfpathlineto{\pgfqpoint{5.034704in}{2.522232in}}%
\pgfpathlineto{\pgfqpoint{5.034600in}{2.099326in}}%
\pgfpathlineto{\pgfqpoint{5.035243in}{2.299858in}}%
\pgfpathlineto{\pgfqpoint{5.035864in}{1.967994in}}%
\pgfpathlineto{\pgfqpoint{5.036174in}{2.531639in}}%
\pgfpathlineto{\pgfqpoint{5.036360in}{2.263809in}}%
\pgfpathlineto{\pgfqpoint{5.036752in}{2.534194in}}%
\pgfpathlineto{\pgfqpoint{5.037061in}{1.947227in}}%
\pgfpathlineto{\pgfqpoint{5.037494in}{2.467616in}}%
\pgfpathlineto{\pgfqpoint{5.037803in}{2.201060in}}%
\pgfpathlineto{\pgfqpoint{5.037535in}{2.525436in}}%
\pgfpathlineto{\pgfqpoint{5.038605in}{2.251660in}}%
\pgfpathlineto{\pgfqpoint{5.038975in}{2.532083in}}%
\pgfpathlineto{\pgfqpoint{5.039570in}{2.091914in}}%
\pgfpathlineto{\pgfqpoint{5.039734in}{2.457663in}}%
\pgfpathlineto{\pgfqpoint{5.040041in}{2.190027in}}%
\pgfpathlineto{\pgfqpoint{5.040655in}{2.526133in}}%
\pgfpathlineto{\pgfqpoint{5.040839in}{2.216739in}}%
\pgfpathlineto{\pgfqpoint{5.041615in}{2.541694in}}%
\pgfpathlineto{\pgfqpoint{5.041513in}{2.125258in}}%
\pgfpathlineto{\pgfqpoint{5.041961in}{2.367060in}}%
\pgfpathlineto{\pgfqpoint{5.042532in}{2.521752in}}%
\pgfpathlineto{\pgfqpoint{5.042369in}{2.158300in}}%
\pgfpathlineto{\pgfqpoint{5.042999in}{2.346932in}}%
\pgfpathlineto{\pgfqpoint{5.043406in}{2.031620in}}%
\pgfpathlineto{\pgfqpoint{5.043182in}{2.525562in}}%
\pgfpathlineto{\pgfqpoint{5.044096in}{2.389373in}}%
\pgfpathlineto{\pgfqpoint{5.044745in}{2.202463in}}%
\pgfpathlineto{\pgfqpoint{5.045170in}{2.500466in}}%
\pgfpathlineto{\pgfqpoint{5.046200in}{1.999753in}}%
\pgfpathlineto{\pgfqpoint{5.045897in}{2.533679in}}%
\pgfpathlineto{\pgfqpoint{5.046281in}{2.222714in}}%
\pgfpathlineto{\pgfqpoint{5.046362in}{2.528611in}}%
\pgfpathlineto{\pgfqpoint{5.046624in}{2.128422in}}%
\pgfpathlineto{\pgfqpoint{5.047409in}{2.466705in}}%
\pgfpathlineto{\pgfqpoint{5.048454in}{2.158192in}}%
\pgfpathlineto{\pgfqpoint{5.047631in}{2.486452in}}%
\pgfpathlineto{\pgfqpoint{5.048575in}{2.276415in}}%
\pgfpathlineto{\pgfqpoint{5.048675in}{2.558120in}}%
\pgfpathlineto{\pgfqpoint{5.048976in}{2.142450in}}%
\pgfpathlineto{\pgfqpoint{5.049697in}{2.414641in}}%
\pgfpathlineto{\pgfqpoint{5.050397in}{2.083226in}}%
\pgfpathlineto{\pgfqpoint{5.050717in}{2.538687in}}%
\pgfpathlineto{\pgfqpoint{5.050797in}{2.350419in}}%
\pgfpathlineto{\pgfqpoint{5.051136in}{2.525132in}}%
\pgfpathlineto{\pgfqpoint{5.051216in}{2.195864in}}%
\pgfpathlineto{\pgfqpoint{5.051813in}{2.465788in}}%
\pgfpathlineto{\pgfqpoint{5.051973in}{1.945335in}}%
\pgfpathlineto{\pgfqpoint{5.052669in}{2.511724in}}%
\pgfpathlineto{\pgfqpoint{5.052907in}{2.386305in}}%
\pgfpathlineto{\pgfqpoint{5.053225in}{2.506067in}}%
\pgfpathlineto{\pgfqpoint{5.053344in}{2.062902in}}%
\pgfpathlineto{\pgfqpoint{5.053999in}{2.382796in}}%
\pgfpathlineto{\pgfqpoint{5.054018in}{2.385102in}}%
\pgfpathlineto{\pgfqpoint{5.054948in}{2.539569in}}%
\pgfpathlineto{\pgfqpoint{5.054553in}{2.194087in}}%
\pgfpathlineto{\pgfqpoint{5.055126in}{2.475982in}}%
\pgfpathlineto{\pgfqpoint{5.055225in}{2.126333in}}%
\pgfpathlineto{\pgfqpoint{5.056015in}{2.530201in}}%
\pgfpathlineto{\pgfqpoint{5.056232in}{2.242933in}}%
\pgfpathlineto{\pgfqpoint{5.056330in}{2.532167in}}%
\pgfpathlineto{\pgfqpoint{5.057216in}{2.122441in}}%
\pgfpathlineto{\pgfqpoint{5.057373in}{2.437064in}}%
\pgfpathlineto{\pgfqpoint{5.057962in}{2.517665in}}%
\pgfpathlineto{\pgfqpoint{5.058492in}{2.072266in}}%
\pgfpathlineto{\pgfqpoint{5.058727in}{2.542501in}}%
\pgfpathlineto{\pgfqpoint{5.059628in}{2.384713in}}%
\pgfpathlineto{\pgfqpoint{5.060331in}{1.970517in}}%
\pgfpathlineto{\pgfqpoint{5.060272in}{2.576413in}}%
\pgfpathlineto{\pgfqpoint{5.060741in}{2.208596in}}%
\pgfpathlineto{\pgfqpoint{5.061754in}{2.525786in}}%
\pgfpathlineto{\pgfqpoint{5.061286in}{2.132254in}}%
\pgfpathlineto{\pgfqpoint{5.061870in}{2.422691in}}%
\pgfpathlineto{\pgfqpoint{5.062783in}{2.150626in}}%
\pgfpathlineto{\pgfqpoint{5.062220in}{2.522117in}}%
\pgfpathlineto{\pgfqpoint{5.062958in}{2.326198in}}%
\pgfpathlineto{\pgfqpoint{5.063171in}{2.520730in}}%
\pgfpathlineto{\pgfqpoint{5.063210in}{2.173321in}}%
\pgfpathlineto{\pgfqpoint{5.064062in}{2.366413in}}%
\pgfpathlineto{\pgfqpoint{5.064333in}{2.068568in}}%
\pgfpathlineto{\pgfqpoint{5.064159in}{2.545686in}}%
\pgfpathlineto{\pgfqpoint{5.065125in}{2.342580in}}%
\pgfpathlineto{\pgfqpoint{5.065164in}{2.530651in}}%
\pgfpathlineto{\pgfqpoint{5.065607in}{2.047710in}}%
\pgfpathlineto{\pgfqpoint{5.066205in}{2.348968in}}%
\pgfpathlineto{\pgfqpoint{5.067166in}{2.004781in}}%
\pgfpathlineto{\pgfqpoint{5.067070in}{2.518660in}}%
\pgfpathlineto{\pgfqpoint{5.067301in}{2.476666in}}%
\pgfpathlineto{\pgfqpoint{5.068145in}{2.196951in}}%
\pgfpathlineto{\pgfqpoint{5.067857in}{2.535432in}}%
\pgfpathlineto{\pgfqpoint{5.068394in}{2.430888in}}%
\pgfpathlineto{\pgfqpoint{5.068930in}{2.534944in}}%
\pgfpathlineto{\pgfqpoint{5.068834in}{2.215443in}}%
\pgfpathlineto{\pgfqpoint{5.069427in}{2.292401in}}%
\pgfpathlineto{\pgfqpoint{5.069885in}{1.998157in}}%
\pgfpathlineto{\pgfqpoint{5.070209in}{2.524009in}}%
\pgfpathlineto{\pgfqpoint{5.070533in}{2.244192in}}%
\pgfpathlineto{\pgfqpoint{5.070933in}{2.552183in}}%
\pgfpathlineto{\pgfqpoint{5.071428in}{2.168486in}}%
\pgfpathlineto{\pgfqpoint{5.071656in}{2.434646in}}%
\pgfpathlineto{\pgfqpoint{5.071960in}{2.175245in}}%
\pgfpathlineto{\pgfqpoint{5.072283in}{2.544709in}}%
\pgfpathlineto{\pgfqpoint{5.072757in}{2.283736in}}%
\pgfpathlineto{\pgfqpoint{5.073817in}{2.525278in}}%
\pgfpathlineto{\pgfqpoint{5.072852in}{2.134352in}}%
\pgfpathlineto{\pgfqpoint{5.073874in}{2.415689in}}%
\pgfpathlineto{\pgfqpoint{5.074365in}{2.534350in}}%
\pgfpathlineto{\pgfqpoint{5.074932in}{2.055807in}}%
\pgfpathlineto{\pgfqpoint{5.075949in}{2.514679in}}%
\pgfpathlineto{\pgfqpoint{5.075045in}{1.981247in}}%
\pgfpathlineto{\pgfqpoint{5.076043in}{2.367941in}}%
\pgfpathlineto{\pgfqpoint{5.076419in}{2.525593in}}%
\pgfpathlineto{\pgfqpoint{5.077076in}{2.093765in}}%
\pgfpathlineto{\pgfqpoint{5.077170in}{2.503373in}}%
\pgfpathlineto{\pgfqpoint{5.077395in}{1.956614in}}%
\pgfpathlineto{\pgfqpoint{5.077357in}{2.539604in}}%
\pgfpathlineto{\pgfqpoint{5.078313in}{2.320992in}}%
\pgfpathlineto{\pgfqpoint{5.078818in}{2.519808in}}%
\pgfpathlineto{\pgfqpoint{5.078425in}{2.123798in}}%
\pgfpathlineto{\pgfqpoint{5.079434in}{2.487812in}}%
\pgfpathlineto{\pgfqpoint{5.079620in}{2.025470in}}%
\pgfpathlineto{\pgfqpoint{5.079993in}{2.548003in}}%
\pgfpathlineto{\pgfqpoint{5.080552in}{2.365208in}}%
\pgfpathlineto{\pgfqpoint{5.081519in}{2.478799in}}%
\pgfpathlineto{\pgfqpoint{5.080589in}{2.155641in}}%
\pgfpathlineto{\pgfqpoint{5.081649in}{2.353070in}}%
\pgfpathlineto{\pgfqpoint{5.082594in}{2.195400in}}%
\pgfpathlineto{\pgfqpoint{5.081797in}{2.520662in}}%
\pgfpathlineto{\pgfqpoint{5.082761in}{2.260587in}}%
\pgfpathlineto{\pgfqpoint{5.083815in}{2.539157in}}%
\pgfpathlineto{\pgfqpoint{5.083390in}{2.166404in}}%
\pgfpathlineto{\pgfqpoint{5.083871in}{2.248594in}}%
\pgfpathlineto{\pgfqpoint{5.084645in}{2.541712in}}%
\pgfpathlineto{\pgfqpoint{5.083907in}{2.051188in}}%
\pgfpathlineto{\pgfqpoint{5.085014in}{2.470100in}}%
\pgfpathlineto{\pgfqpoint{5.085143in}{2.559963in}}%
\pgfpathlineto{\pgfqpoint{5.086117in}{2.016728in}}%
\pgfpathlineto{\pgfqpoint{5.086980in}{2.539089in}}%
\pgfpathlineto{\pgfqpoint{5.087237in}{2.525300in}}%
\pgfpathlineto{\pgfqpoint{5.088261in}{2.237540in}}%
\pgfpathlineto{\pgfqpoint{5.088353in}{2.423201in}}%
\pgfpathlineto{\pgfqpoint{5.089411in}{2.137579in}}%
\pgfpathlineto{\pgfqpoint{5.089265in}{2.515765in}}%
\pgfpathlineto{\pgfqpoint{5.089484in}{2.299615in}}%
\pgfpathlineto{\pgfqpoint{5.090412in}{2.535338in}}%
\pgfpathlineto{\pgfqpoint{5.089939in}{2.154173in}}%
\pgfpathlineto{\pgfqpoint{5.090594in}{2.403437in}}%
\pgfpathlineto{\pgfqpoint{5.091284in}{2.567877in}}%
\pgfpathlineto{\pgfqpoint{5.091665in}{1.925408in}}%
\pgfpathlineto{\pgfqpoint{5.091683in}{2.507597in}}%
\pgfpathlineto{\pgfqpoint{5.092788in}{2.305051in}}%
\pgfpathlineto{\pgfqpoint{5.092824in}{2.520656in}}%
\pgfpathlineto{\pgfqpoint{5.093763in}{2.030727in}}%
\pgfpathlineto{\pgfqpoint{5.093907in}{2.412247in}}%
\pgfpathlineto{\pgfqpoint{5.094646in}{2.537938in}}%
\pgfpathlineto{\pgfqpoint{5.093944in}{2.170141in}}%
\pgfpathlineto{\pgfqpoint{5.095006in}{2.396753in}}%
\pgfpathlineto{\pgfqpoint{5.096084in}{2.165864in}}%
\pgfpathlineto{\pgfqpoint{5.095114in}{2.527315in}}%
\pgfpathlineto{\pgfqpoint{5.096102in}{2.246542in}}%
\pgfpathlineto{\pgfqpoint{5.096532in}{2.545331in}}%
\pgfpathlineto{\pgfqpoint{5.096962in}{2.167125in}}%
\pgfpathlineto{\pgfqpoint{5.097213in}{2.344831in}}%
\pgfpathlineto{\pgfqpoint{5.098214in}{2.488206in}}%
\pgfpathlineto{\pgfqpoint{5.098160in}{2.056672in}}%
\pgfpathlineto{\pgfqpoint{5.098321in}{2.376907in}}%
\pgfpathlineto{\pgfqpoint{5.098981in}{2.118626in}}%
\pgfpathlineto{\pgfqpoint{5.098535in}{2.494885in}}%
\pgfpathlineto{\pgfqpoint{5.099408in}{2.341843in}}%
\pgfpathlineto{\pgfqpoint{5.099640in}{2.527752in}}%
\pgfpathlineto{\pgfqpoint{5.099782in}{1.723479in}}%
\pgfpathlineto{\pgfqpoint{5.100511in}{2.399522in}}%
\pgfpathlineto{\pgfqpoint{5.100954in}{2.023395in}}%
\pgfpathlineto{\pgfqpoint{5.100777in}{2.521921in}}%
\pgfpathlineto{\pgfqpoint{5.101628in}{2.378260in}}%
\pgfpathlineto{\pgfqpoint{5.101911in}{2.527965in}}%
\pgfpathlineto{\pgfqpoint{5.102247in}{1.989565in}}%
\pgfpathlineto{\pgfqpoint{5.102636in}{2.330231in}}%
\pgfpathlineto{\pgfqpoint{5.103131in}{2.540519in}}%
\pgfpathlineto{\pgfqpoint{5.103765in}{2.036146in}}%
\pgfpathlineto{\pgfqpoint{5.104593in}{2.511753in}}%
\pgfpathlineto{\pgfqpoint{5.104892in}{2.429143in}}%
\pgfpathlineto{\pgfqpoint{5.105892in}{2.102290in}}%
\pgfpathlineto{\pgfqpoint{5.105524in}{2.501082in}}%
\pgfpathlineto{\pgfqpoint{5.106015in}{2.327903in}}%
\pgfpathlineto{\pgfqpoint{5.107030in}{2.514468in}}%
\pgfpathlineto{\pgfqpoint{5.106628in}{2.074371in}}%
\pgfpathlineto{\pgfqpoint{5.107100in}{2.392012in}}%
\pgfpathlineto{\pgfqpoint{5.107869in}{2.109100in}}%
\pgfpathlineto{\pgfqpoint{5.107624in}{2.577966in}}%
\pgfpathlineto{\pgfqpoint{5.108218in}{2.338243in}}%
\pgfpathlineto{\pgfqpoint{5.109315in}{2.519357in}}%
\pgfpathlineto{\pgfqpoint{5.108740in}{2.077837in}}%
\pgfpathlineto{\pgfqpoint{5.109332in}{2.510553in}}%
\pgfpathlineto{\pgfqpoint{5.109715in}{2.159830in}}%
\pgfpathlineto{\pgfqpoint{5.109854in}{2.522935in}}%
\pgfpathlineto{\pgfqpoint{5.110478in}{2.247359in}}%
\pgfpathlineto{\pgfqpoint{5.111501in}{2.523403in}}%
\pgfpathlineto{\pgfqpoint{5.111016in}{2.152632in}}%
\pgfpathlineto{\pgfqpoint{5.111622in}{2.481954in}}%
\pgfpathlineto{\pgfqpoint{5.111985in}{2.024712in}}%
\pgfpathlineto{\pgfqpoint{5.112348in}{2.534309in}}%
\pgfpathlineto{\pgfqpoint{5.112727in}{2.461846in}}%
\pgfpathlineto{\pgfqpoint{5.113434in}{2.062289in}}%
\pgfpathlineto{\pgfqpoint{5.113675in}{2.558299in}}%
\pgfpathlineto{\pgfqpoint{5.113882in}{2.253325in}}%
\pgfpathlineto{\pgfqpoint{5.114157in}{2.513312in}}%
\pgfpathlineto{\pgfqpoint{5.114878in}{2.003186in}}%
\pgfpathlineto{\pgfqpoint{5.115016in}{2.480467in}}%
\pgfpathlineto{\pgfqpoint{5.115770in}{2.546273in}}%
\pgfpathlineto{\pgfqpoint{5.115153in}{2.150137in}}%
\pgfpathlineto{\pgfqpoint{5.116044in}{2.432338in}}%
\pgfpathlineto{\pgfqpoint{5.117121in}{1.913570in}}%
\pgfpathlineto{\pgfqpoint{5.116198in}{2.549054in}}%
\pgfpathlineto{\pgfqpoint{5.117172in}{2.286792in}}%
\pgfpathlineto{\pgfqpoint{5.117940in}{2.526327in}}%
\pgfpathlineto{\pgfqpoint{5.117206in}{1.938361in}}%
\pgfpathlineto{\pgfqpoint{5.118281in}{2.397904in}}%
\pgfpathlineto{\pgfqpoint{5.118774in}{2.044981in}}%
\pgfpathlineto{\pgfqpoint{5.118927in}{2.525593in}}%
\pgfpathlineto{\pgfqpoint{5.119318in}{2.219708in}}%
\pgfpathlineto{\pgfqpoint{5.120031in}{2.504705in}}%
\pgfpathlineto{\pgfqpoint{5.119861in}{2.060749in}}%
\pgfpathlineto{\pgfqpoint{5.120438in}{2.410401in}}%
\pgfpathlineto{\pgfqpoint{5.121250in}{2.529956in}}%
\pgfpathlineto{\pgfqpoint{5.120539in}{1.972159in}}%
\pgfpathlineto{\pgfqpoint{5.121385in}{2.456872in}}%
\pgfpathlineto{\pgfqpoint{5.121994in}{2.535510in}}%
\pgfpathlineto{\pgfqpoint{5.122483in}{2.044680in}}%
\pgfpathlineto{\pgfqpoint{5.122651in}{2.525975in}}%
\pgfpathlineto{\pgfqpoint{5.123577in}{2.041378in}}%
\pgfpathlineto{\pgfqpoint{5.123594in}{2.381543in}}%
\pgfpathlineto{\pgfqpoint{5.124367in}{2.550529in}}%
\pgfpathlineto{\pgfqpoint{5.123863in}{2.137616in}}%
\pgfpathlineto{\pgfqpoint{5.124535in}{2.337951in}}%
\pgfpathlineto{\pgfqpoint{5.125272in}{2.060031in}}%
\pgfpathlineto{\pgfqpoint{5.125373in}{2.546754in}}%
\pgfpathlineto{\pgfqpoint{5.125641in}{2.257618in}}%
\pgfpathlineto{\pgfqpoint{5.125691in}{2.532774in}}%
\pgfpathlineto{\pgfqpoint{5.126293in}{1.841301in}}%
\pgfpathlineto{\pgfqpoint{5.126744in}{2.280739in}}%
\pgfpathlineto{\pgfqpoint{5.126977in}{2.076527in}}%
\pgfpathlineto{\pgfqpoint{5.127094in}{2.517400in}}%
\pgfpathlineto{\pgfqpoint{5.127794in}{2.347874in}}%
\pgfpathlineto{\pgfqpoint{5.128077in}{2.545029in}}%
\pgfpathlineto{\pgfqpoint{5.127827in}{2.079443in}}%
\pgfpathlineto{\pgfqpoint{5.128892in}{2.326970in}}%
\pgfpathlineto{\pgfqpoint{5.128908in}{2.161555in}}%
\pgfpathlineto{\pgfqpoint{5.129887in}{2.534548in}}%
\pgfpathlineto{\pgfqpoint{5.129986in}{2.415001in}}%
\pgfpathlineto{\pgfqpoint{5.131012in}{2.130465in}}%
\pgfpathlineto{\pgfqpoint{5.130284in}{2.543367in}}%
\pgfpathlineto{\pgfqpoint{5.131062in}{2.341439in}}%
\pgfpathlineto{\pgfqpoint{5.131508in}{2.541450in}}%
\pgfpathlineto{\pgfqpoint{5.132118in}{2.157936in}}%
\pgfpathlineto{\pgfqpoint{5.132184in}{2.474579in}}%
\pgfpathlineto{\pgfqpoint{5.132382in}{2.111903in}}%
\pgfpathlineto{\pgfqpoint{5.132316in}{2.516313in}}%
\pgfpathlineto{\pgfqpoint{5.133303in}{2.400244in}}%
\pgfpathlineto{\pgfqpoint{5.133632in}{2.553366in}}%
\pgfpathlineto{\pgfqpoint{5.133549in}{2.029151in}}%
\pgfpathlineto{\pgfqpoint{5.134304in}{2.445974in}}%
\pgfpathlineto{\pgfqpoint{5.134960in}{1.977954in}}%
\pgfpathlineto{\pgfqpoint{5.135287in}{2.528517in}}%
\pgfpathlineto{\pgfqpoint{5.135402in}{2.333941in}}%
\pgfpathlineto{\pgfqpoint{5.135500in}{2.525476in}}%
\pgfpathlineto{\pgfqpoint{5.136202in}{2.102935in}}%
\pgfpathlineto{\pgfqpoint{5.136529in}{2.416494in}}%
\pgfpathlineto{\pgfqpoint{5.137620in}{2.083180in}}%
\pgfpathlineto{\pgfqpoint{5.136969in}{2.550875in}}%
\pgfpathlineto{\pgfqpoint{5.137636in}{2.410528in}}%
\pgfpathlineto{\pgfqpoint{5.138189in}{2.169173in}}%
\pgfpathlineto{\pgfqpoint{5.138124in}{2.509816in}}%
\pgfpathlineto{\pgfqpoint{5.138741in}{2.403272in}}%
\pgfpathlineto{\pgfqpoint{5.138887in}{2.548532in}}%
\pgfpathlineto{\pgfqpoint{5.138920in}{2.081594in}}%
\pgfpathlineto{\pgfqpoint{5.139714in}{2.449009in}}%
\pgfpathlineto{\pgfqpoint{5.139730in}{2.005790in}}%
\pgfpathlineto{\pgfqpoint{5.139940in}{2.520475in}}%
\pgfpathlineto{\pgfqpoint{5.140813in}{2.451651in}}%
\pgfpathlineto{\pgfqpoint{5.141555in}{2.217867in}}%
\pgfpathlineto{\pgfqpoint{5.141491in}{2.518280in}}%
\pgfpathlineto{\pgfqpoint{5.141910in}{2.244274in}}%
\pgfpathlineto{\pgfqpoint{5.142971in}{2.541909in}}%
\pgfpathlineto{\pgfqpoint{5.142602in}{2.203535in}}%
\pgfpathlineto{\pgfqpoint{5.143020in}{2.294421in}}%
\pgfpathlineto{\pgfqpoint{5.143277in}{2.533538in}}%
\pgfpathlineto{\pgfqpoint{5.143309in}{2.077762in}}%
\pgfpathlineto{\pgfqpoint{5.144127in}{2.379279in}}%
\pgfpathlineto{\pgfqpoint{5.144847in}{2.066175in}}%
\pgfpathlineto{\pgfqpoint{5.144799in}{2.518969in}}%
\pgfpathlineto{\pgfqpoint{5.145231in}{2.237936in}}%
\pgfpathlineto{\pgfqpoint{5.146220in}{2.525658in}}%
\pgfpathlineto{\pgfqpoint{5.145774in}{2.132514in}}%
\pgfpathlineto{\pgfqpoint{5.146364in}{2.412969in}}%
\pgfpathlineto{\pgfqpoint{5.146794in}{1.900509in}}%
\pgfpathlineto{\pgfqpoint{5.146635in}{2.525019in}}%
\pgfpathlineto{\pgfqpoint{5.147462in}{2.474454in}}%
\pgfpathlineto{\pgfqpoint{5.147494in}{1.959776in}}%
\pgfpathlineto{\pgfqpoint{5.148224in}{2.539349in}}%
\pgfpathlineto{\pgfqpoint{5.148605in}{2.359282in}}%
\pgfpathlineto{\pgfqpoint{5.149587in}{2.547065in}}%
\pgfpathlineto{\pgfqpoint{5.148637in}{1.844258in}}%
\pgfpathlineto{\pgfqpoint{5.149682in}{2.436636in}}%
\pgfpathlineto{\pgfqpoint{5.150045in}{2.012124in}}%
\pgfpathlineto{\pgfqpoint{5.150251in}{2.533200in}}%
\pgfpathlineto{\pgfqpoint{5.150787in}{2.440936in}}%
\pgfpathlineto{\pgfqpoint{5.150803in}{2.538124in}}%
\pgfpathlineto{\pgfqpoint{5.151008in}{2.035057in}}%
\pgfpathlineto{\pgfqpoint{5.151890in}{2.403174in}}%
\pgfpathlineto{\pgfqpoint{5.152958in}{2.082562in}}%
\pgfpathlineto{\pgfqpoint{5.152205in}{2.555862in}}%
\pgfpathlineto{\pgfqpoint{5.152990in}{2.260487in}}%
\pgfpathlineto{\pgfqpoint{5.153648in}{2.518877in}}%
\pgfpathlineto{\pgfqpoint{5.153962in}{2.184071in}}%
\pgfpathlineto{\pgfqpoint{5.154102in}{2.438980in}}%
\pgfpathlineto{\pgfqpoint{5.155134in}{2.162283in}}%
\pgfpathlineto{\pgfqpoint{5.154462in}{2.575124in}}%
\pgfpathlineto{\pgfqpoint{5.155196in}{2.387295in}}%
\pgfpathlineto{\pgfqpoint{5.155898in}{2.528663in}}%
\pgfpathlineto{\pgfqpoint{5.155820in}{2.022484in}}%
\pgfpathlineto{\pgfqpoint{5.156303in}{2.475217in}}%
\pgfpathlineto{\pgfqpoint{5.156879in}{2.505675in}}%
\pgfpathlineto{\pgfqpoint{5.157407in}{2.212703in}}%
\pgfpathlineto{\pgfqpoint{5.157981in}{2.562028in}}%
\pgfpathlineto{\pgfqpoint{5.157826in}{2.094650in}}%
\pgfpathlineto{\pgfqpoint{5.158524in}{2.387098in}}%
\pgfpathlineto{\pgfqpoint{5.158803in}{2.537941in}}%
\pgfpathlineto{\pgfqpoint{5.158942in}{2.178061in}}%
\pgfpathlineto{\pgfqpoint{5.159591in}{2.382871in}}%
\pgfpathlineto{\pgfqpoint{5.159885in}{2.108831in}}%
\pgfpathlineto{\pgfqpoint{5.159900in}{2.573818in}}%
\pgfpathlineto{\pgfqpoint{5.160687in}{2.340936in}}%
\pgfpathlineto{\pgfqpoint{5.161195in}{2.538821in}}%
\pgfpathlineto{\pgfqpoint{5.161241in}{2.131524in}}%
\pgfpathlineto{\pgfqpoint{5.161795in}{2.394381in}}%
\pgfpathlineto{\pgfqpoint{5.162455in}{2.536944in}}%
\pgfpathlineto{\pgfqpoint{5.162317in}{2.275976in}}%
\pgfpathlineto{\pgfqpoint{5.162639in}{2.413789in}}%
\pgfpathlineto{\pgfqpoint{5.163727in}{2.535627in}}%
\pgfpathlineto{\pgfqpoint{5.163757in}{2.090823in}}%
\pgfpathlineto{\pgfqpoint{5.163880in}{2.528253in}}%
\pgfpathlineto{\pgfqpoint{5.164720in}{1.916491in}}%
\pgfpathlineto{\pgfqpoint{5.164873in}{2.457875in}}%
\pgfpathlineto{\pgfqpoint{5.165254in}{2.026636in}}%
\pgfpathlineto{\pgfqpoint{5.164918in}{2.559590in}}%
\pgfpathlineto{\pgfqpoint{5.165970in}{2.288809in}}%
\pgfpathlineto{\pgfqpoint{5.166700in}{2.545966in}}%
\pgfpathlineto{\pgfqpoint{5.166152in}{2.119168in}}%
\pgfpathlineto{\pgfqpoint{5.167079in}{2.409149in}}%
\pgfpathlineto{\pgfqpoint{5.167413in}{2.127214in}}%
\pgfpathlineto{\pgfqpoint{5.167549in}{2.510254in}}%
\pgfpathlineto{\pgfqpoint{5.168110in}{2.278082in}}%
\pgfpathlineto{\pgfqpoint{5.168277in}{2.557151in}}%
\pgfpathlineto{\pgfqpoint{5.168655in}{2.131321in}}%
\pgfpathlineto{\pgfqpoint{5.169229in}{2.389531in}}%
\pgfpathlineto{\pgfqpoint{5.169576in}{1.867602in}}%
\pgfpathlineto{\pgfqpoint{5.169274in}{2.516579in}}%
\pgfpathlineto{\pgfqpoint{5.170330in}{2.236733in}}%
\pgfpathlineto{\pgfqpoint{5.171338in}{2.562803in}}%
\pgfpathlineto{\pgfqpoint{5.171067in}{2.091428in}}%
\pgfpathlineto{\pgfqpoint{5.171458in}{2.444669in}}%
\pgfpathlineto{\pgfqpoint{5.171503in}{2.077472in}}%
\pgfpathlineto{\pgfqpoint{5.172224in}{2.530831in}}%
\pgfpathlineto{\pgfqpoint{5.172583in}{2.262410in}}%
\pgfpathlineto{\pgfqpoint{5.172658in}{2.533296in}}%
\pgfpathlineto{\pgfqpoint{5.172838in}{1.953872in}}%
\pgfpathlineto{\pgfqpoint{5.173706in}{2.478866in}}%
\pgfpathlineto{\pgfqpoint{5.174393in}{2.022448in}}%
\pgfpathlineto{\pgfqpoint{5.174019in}{2.541350in}}%
\pgfpathlineto{\pgfqpoint{5.174825in}{2.375994in}}%
\pgfpathlineto{\pgfqpoint{5.175852in}{2.545210in}}%
\pgfpathlineto{\pgfqpoint{5.175361in}{2.136186in}}%
\pgfpathlineto{\pgfqpoint{5.175882in}{2.472773in}}%
\pgfpathlineto{\pgfqpoint{5.175941in}{2.191785in}}%
\pgfpathlineto{\pgfqpoint{5.176536in}{2.516507in}}%
\pgfpathlineto{\pgfqpoint{5.176995in}{2.374935in}}%
\pgfpathlineto{\pgfqpoint{5.177470in}{2.540068in}}%
\pgfpathlineto{\pgfqpoint{5.177766in}{1.917440in}}%
\pgfpathlineto{\pgfqpoint{5.178106in}{2.446091in}}%
\pgfpathlineto{\pgfqpoint{5.178195in}{2.000512in}}%
\pgfpathlineto{\pgfqpoint{5.178224in}{2.529432in}}%
\pgfpathlineto{\pgfqpoint{5.179214in}{2.155078in}}%
\pgfpathlineto{\pgfqpoint{5.180245in}{2.521515in}}%
\pgfpathlineto{\pgfqpoint{5.179774in}{2.115778in}}%
\pgfpathlineto{\pgfqpoint{5.180333in}{2.440354in}}%
\pgfpathlineto{\pgfqpoint{5.180451in}{2.010255in}}%
\pgfpathlineto{\pgfqpoint{5.180892in}{2.535607in}}%
\pgfpathlineto{\pgfqpoint{5.181479in}{2.136056in}}%
\pgfpathlineto{\pgfqpoint{5.182534in}{2.534953in}}%
\pgfpathlineto{\pgfqpoint{5.182227in}{1.896132in}}%
\pgfpathlineto{\pgfqpoint{5.182593in}{2.533167in}}%
\pgfpathlineto{\pgfqpoint{5.183207in}{1.997622in}}%
\pgfpathlineto{\pgfqpoint{5.182900in}{2.547475in}}%
\pgfpathlineto{\pgfqpoint{5.183733in}{2.361560in}}%
\pgfpathlineto{\pgfqpoint{5.184025in}{2.536409in}}%
\pgfpathlineto{\pgfqpoint{5.184462in}{2.157702in}}%
\pgfpathlineto{\pgfqpoint{5.184841in}{2.397370in}}%
\pgfpathlineto{\pgfqpoint{5.185757in}{2.101956in}}%
\pgfpathlineto{\pgfqpoint{5.184899in}{2.512935in}}%
\pgfpathlineto{\pgfqpoint{5.185902in}{2.373185in}}%
\pgfpathlineto{\pgfqpoint{5.186018in}{2.531027in}}%
\pgfpathlineto{\pgfqpoint{5.186569in}{2.062241in}}%
\pgfpathlineto{\pgfqpoint{5.186975in}{2.437331in}}%
\pgfpathlineto{\pgfqpoint{5.187496in}{2.106119in}}%
\pgfpathlineto{\pgfqpoint{5.187221in}{2.556890in}}%
\pgfpathlineto{\pgfqpoint{5.188074in}{2.226515in}}%
\pgfpathlineto{\pgfqpoint{5.189128in}{2.523526in}}%
\pgfpathlineto{\pgfqpoint{5.188767in}{2.166671in}}%
\pgfpathlineto{\pgfqpoint{5.189200in}{2.466177in}}%
\pgfpathlineto{\pgfqpoint{5.190178in}{2.161255in}}%
\pgfpathlineto{\pgfqpoint{5.189919in}{2.532325in}}%
\pgfpathlineto{\pgfqpoint{5.190308in}{2.420716in}}%
\pgfpathlineto{\pgfqpoint{5.190423in}{2.544534in}}%
\pgfpathlineto{\pgfqpoint{5.190997in}{2.215675in}}%
\pgfpathlineto{\pgfqpoint{5.191413in}{2.417462in}}%
\pgfpathlineto{\pgfqpoint{5.192000in}{2.087141in}}%
\pgfpathlineto{\pgfqpoint{5.191971in}{2.533536in}}%
\pgfpathlineto{\pgfqpoint{5.192529in}{2.321439in}}%
\pgfpathlineto{\pgfqpoint{5.192743in}{2.550452in}}%
\pgfpathlineto{\pgfqpoint{5.192701in}{2.113578in}}%
\pgfpathlineto{\pgfqpoint{5.193628in}{2.381572in}}%
\pgfpathlineto{\pgfqpoint{5.193757in}{1.960932in}}%
\pgfpathlineto{\pgfqpoint{5.194511in}{2.546687in}}%
\pgfpathlineto{\pgfqpoint{5.194725in}{2.244403in}}%
\pgfpathlineto{\pgfqpoint{5.195080in}{2.537699in}}%
\pgfpathlineto{\pgfqpoint{5.195222in}{2.208168in}}%
\pgfpathlineto{\pgfqpoint{5.195833in}{2.422243in}}%
\pgfpathlineto{\pgfqpoint{5.196853in}{2.106237in}}%
\pgfpathlineto{\pgfqpoint{5.195946in}{2.546791in}}%
\pgfpathlineto{\pgfqpoint{5.196938in}{2.299445in}}%
\pgfpathlineto{\pgfqpoint{5.197983in}{2.548711in}}%
\pgfpathlineto{\pgfqpoint{5.196980in}{2.074244in}}%
\pgfpathlineto{\pgfqpoint{5.198040in}{2.254837in}}%
\pgfpathlineto{\pgfqpoint{5.198054in}{2.112731in}}%
\pgfpathlineto{\pgfqpoint{5.198970in}{2.573329in}}%
\pgfpathlineto{\pgfqpoint{5.199139in}{2.262935in}}%
\pgfpathlineto{\pgfqpoint{5.200039in}{2.548435in}}%
\pgfpathlineto{\pgfqpoint{5.199884in}{1.986533in}}%
\pgfpathlineto{\pgfqpoint{5.200249in}{2.409410in}}%
\pgfpathlineto{\pgfqpoint{5.200446in}{2.058161in}}%
\pgfpathlineto{\pgfqpoint{5.200404in}{2.524251in}}%
\pgfpathlineto{\pgfqpoint{5.201301in}{2.375707in}}%
\pgfpathlineto{\pgfqpoint{5.202140in}{2.530539in}}%
\pgfpathlineto{\pgfqpoint{5.201399in}{2.027945in}}%
\pgfpathlineto{\pgfqpoint{5.202406in}{2.476864in}}%
\pgfpathlineto{\pgfqpoint{5.202824in}{2.116328in}}%
\pgfpathlineto{\pgfqpoint{5.202699in}{2.484479in}}%
\pgfpathlineto{\pgfqpoint{5.203521in}{2.343310in}}%
\pgfpathlineto{\pgfqpoint{5.204273in}{2.506205in}}%
\pgfpathlineto{\pgfqpoint{5.204301in}{1.973519in}}%
\pgfpathlineto{\pgfqpoint{5.204634in}{2.461446in}}%
\pgfpathlineto{\pgfqpoint{5.204648in}{2.016777in}}%
\pgfpathlineto{\pgfqpoint{5.204898in}{2.521412in}}%
\pgfpathlineto{\pgfqpoint{5.205744in}{2.351424in}}%
\pgfpathlineto{\pgfqpoint{5.205800in}{1.998777in}}%
\pgfpathlineto{\pgfqpoint{5.205952in}{2.515145in}}%
\pgfpathlineto{\pgfqpoint{5.206616in}{2.384710in}}%
\pgfpathlineto{\pgfqpoint{5.207500in}{2.516170in}}%
\pgfpathlineto{\pgfqpoint{5.207128in}{2.109243in}}%
\pgfpathlineto{\pgfqpoint{5.207721in}{2.354453in}}%
\pgfpathlineto{\pgfqpoint{5.207845in}{2.133609in}}%
\pgfpathlineto{\pgfqpoint{5.208052in}{2.534756in}}%
\pgfpathlineto{\pgfqpoint{5.208809in}{2.468446in}}%
\pgfpathlineto{\pgfqpoint{5.209016in}{2.549354in}}%
\pgfpathlineto{\pgfqpoint{5.209153in}{2.051572in}}%
\pgfpathlineto{\pgfqpoint{5.209689in}{2.326203in}}%
\pgfpathlineto{\pgfqpoint{5.209703in}{2.031565in}}%
\pgfpathlineto{\pgfqpoint{5.209867in}{2.559300in}}%
\pgfpathlineto{\pgfqpoint{5.210786in}{2.513975in}}%
\pgfpathlineto{\pgfqpoint{5.211018in}{2.153805in}}%
\pgfpathlineto{\pgfqpoint{5.211620in}{2.531252in}}%
\pgfpathlineto{\pgfqpoint{5.211893in}{2.321459in}}%
\pgfpathlineto{\pgfqpoint{5.212562in}{2.542235in}}%
\pgfpathlineto{\pgfqpoint{5.212139in}{1.987800in}}%
\pgfpathlineto{\pgfqpoint{5.213012in}{2.414271in}}%
\pgfpathlineto{\pgfqpoint{5.213910in}{2.530890in}}%
\pgfpathlineto{\pgfqpoint{5.213257in}{1.966752in}}%
\pgfpathlineto{\pgfqpoint{5.214114in}{2.486708in}}%
\pgfpathlineto{\pgfqpoint{5.214901in}{2.039394in}}%
\pgfpathlineto{\pgfqpoint{5.214358in}{2.513723in}}%
\pgfpathlineto{\pgfqpoint{5.215226in}{2.234602in}}%
\pgfpathlineto{\pgfqpoint{5.215335in}{2.585534in}}%
\pgfpathlineto{\pgfqpoint{5.215389in}{2.119564in}}%
\pgfpathlineto{\pgfqpoint{5.216336in}{2.429913in}}%
\pgfpathlineto{\pgfqpoint{5.216539in}{1.993894in}}%
\pgfpathlineto{\pgfqpoint{5.217025in}{2.551067in}}%
\pgfpathlineto{\pgfqpoint{5.217443in}{2.293350in}}%
\pgfpathlineto{\pgfqpoint{5.217699in}{2.541077in}}%
\pgfpathlineto{\pgfqpoint{5.218237in}{2.068369in}}%
\pgfpathlineto{\pgfqpoint{5.218547in}{2.321943in}}%
\pgfpathlineto{\pgfqpoint{5.219473in}{2.009123in}}%
\pgfpathlineto{\pgfqpoint{5.219366in}{2.532982in}}%
\pgfpathlineto{\pgfqpoint{5.219661in}{2.135781in}}%
\pgfpathlineto{\pgfqpoint{5.219728in}{2.542588in}}%
\pgfpathlineto{\pgfqpoint{5.220786in}{2.407958in}}%
\pgfpathlineto{\pgfqpoint{5.221721in}{1.930533in}}%
\pgfpathlineto{\pgfqpoint{5.221681in}{2.525544in}}%
\pgfpathlineto{\pgfqpoint{5.221881in}{2.411668in}}%
\pgfpathlineto{\pgfqpoint{5.222947in}{2.531408in}}%
\pgfpathlineto{\pgfqpoint{5.221935in}{2.152837in}}%
\pgfpathlineto{\pgfqpoint{5.222974in}{2.436071in}}%
\pgfpathlineto{\pgfqpoint{5.223918in}{2.052100in}}%
\pgfpathlineto{\pgfqpoint{5.223931in}{2.518004in}}%
\pgfpathlineto{\pgfqpoint{5.224090in}{2.284351in}}%
\pgfpathlineto{\pgfqpoint{5.224183in}{2.526822in}}%
\pgfpathlineto{\pgfqpoint{5.224753in}{2.082377in}}%
\pgfpathlineto{\pgfqpoint{5.225203in}{2.396333in}}%
\pgfpathlineto{\pgfqpoint{5.226248in}{2.011645in}}%
\pgfpathlineto{\pgfqpoint{5.225256in}{2.534298in}}%
\pgfpathlineto{\pgfqpoint{5.226327in}{2.271479in}}%
\pgfpathlineto{\pgfqpoint{5.227210in}{2.541885in}}%
\pgfpathlineto{\pgfqpoint{5.226775in}{2.239670in}}%
\pgfpathlineto{\pgfqpoint{5.227434in}{2.256241in}}%
\pgfpathlineto{\pgfqpoint{5.227553in}{2.551700in}}%
\pgfpathlineto{\pgfqpoint{5.228329in}{2.162500in}}%
\pgfpathlineto{\pgfqpoint{5.228526in}{2.249139in}}%
\pgfpathlineto{\pgfqpoint{5.228539in}{2.098804in}}%
\pgfpathlineto{\pgfqpoint{5.229339in}{2.530605in}}%
\pgfpathlineto{\pgfqpoint{5.229614in}{2.416499in}}%
\pgfpathlineto{\pgfqpoint{5.229732in}{2.160495in}}%
\pgfpathlineto{\pgfqpoint{5.230138in}{2.546175in}}%
\pgfpathlineto{\pgfqpoint{5.230713in}{2.311747in}}%
\pgfpathlineto{\pgfqpoint{5.231744in}{2.498797in}}%
\pgfpathlineto{\pgfqpoint{5.231392in}{1.984899in}}%
\pgfpathlineto{\pgfqpoint{5.231835in}{2.450597in}}%
\pgfpathlineto{\pgfqpoint{5.232525in}{2.043798in}}%
\pgfpathlineto{\pgfqpoint{5.232356in}{2.538834in}}%
\pgfpathlineto{\pgfqpoint{5.232954in}{2.280845in}}%
\pgfpathlineto{\pgfqpoint{5.233565in}{2.536121in}}%
\pgfpathlineto{\pgfqpoint{5.233669in}{2.143632in}}%
\pgfpathlineto{\pgfqpoint{5.234071in}{2.521179in}}%
\pgfpathlineto{\pgfqpoint{5.234990in}{2.048617in}}%
\pgfpathlineto{\pgfqpoint{5.234433in}{2.523810in}}%
\pgfpathlineto{\pgfqpoint{5.235210in}{2.263889in}}%
\pgfpathlineto{\pgfqpoint{5.235907in}{2.541352in}}%
\pgfpathlineto{\pgfqpoint{5.235752in}{2.159306in}}%
\pgfpathlineto{\pgfqpoint{5.236320in}{2.276859in}}%
\pgfpathlineto{\pgfqpoint{5.236346in}{2.420322in}}%
\pgfpathlineto{\pgfqpoint{5.236384in}{2.193827in}}%
\pgfpathlineto{\pgfqpoint{5.236397in}{2.383367in}}%
\pgfpathlineto{\pgfqpoint{5.236410in}{2.519293in}}%
\pgfpathlineto{\pgfqpoint{5.236848in}{1.836174in}}%
\pgfpathlineto{\pgfqpoint{5.237505in}{2.409765in}}%
\pgfpathlineto{\pgfqpoint{5.237723in}{2.510641in}}%
\pgfpathlineto{\pgfqpoint{5.238096in}{2.035358in}}%
\pgfpathlineto{\pgfqpoint{5.238455in}{2.423124in}}%
\pgfpathlineto{\pgfqpoint{5.238468in}{2.034652in}}%
\pgfpathlineto{\pgfqpoint{5.238929in}{2.554354in}}%
\pgfpathlineto{\pgfqpoint{5.239570in}{2.181615in}}%
\pgfpathlineto{\pgfqpoint{5.240566in}{2.574858in}}%
\pgfpathlineto{\pgfqpoint{5.239838in}{2.028320in}}%
\pgfpathlineto{\pgfqpoint{5.240707in}{2.410295in}}%
\pgfpathlineto{\pgfqpoint{5.240796in}{2.523807in}}%
\pgfpathlineto{\pgfqpoint{5.241000in}{2.054354in}}%
\pgfpathlineto{\pgfqpoint{5.241217in}{2.456793in}}%
\pgfpathlineto{\pgfqpoint{5.241230in}{2.042733in}}%
\pgfpathlineto{\pgfqpoint{5.241930in}{2.556363in}}%
\pgfpathlineto{\pgfqpoint{5.242324in}{2.418824in}}%
\pgfpathlineto{\pgfqpoint{5.242807in}{2.124528in}}%
\pgfpathlineto{\pgfqpoint{5.243150in}{2.527208in}}%
\pgfpathlineto{\pgfqpoint{5.243429in}{2.234532in}}%
\pgfpathlineto{\pgfqpoint{5.244075in}{2.053721in}}%
\pgfpathlineto{\pgfqpoint{5.244556in}{2.517585in}}%
\pgfpathlineto{\pgfqpoint{5.244821in}{2.192485in}}%
\pgfpathlineto{\pgfqpoint{5.244606in}{2.542838in}}%
\pgfpathlineto{\pgfqpoint{5.245667in}{2.291457in}}%
\pgfpathlineto{\pgfqpoint{5.246108in}{2.527260in}}%
\pgfpathlineto{\pgfqpoint{5.246549in}{2.091071in}}%
\pgfpathlineto{\pgfqpoint{5.246776in}{2.444405in}}%
\pgfpathlineto{\pgfqpoint{5.247530in}{2.022636in}}%
\pgfpathlineto{\pgfqpoint{5.247706in}{2.521659in}}%
\pgfpathlineto{\pgfqpoint{5.247881in}{2.417567in}}%
\pgfpathlineto{\pgfqpoint{5.248608in}{2.542240in}}%
\pgfpathlineto{\pgfqpoint{5.248834in}{2.166276in}}%
\pgfpathlineto{\pgfqpoint{5.248909in}{2.360211in}}%
\pgfpathlineto{\pgfqpoint{5.249934in}{2.122849in}}%
\pgfpathlineto{\pgfqpoint{5.249172in}{2.574643in}}%
\pgfpathlineto{\pgfqpoint{5.250009in}{2.412088in}}%
\pgfpathlineto{\pgfqpoint{5.251044in}{2.529901in}}%
\pgfpathlineto{\pgfqpoint{5.250745in}{2.151973in}}%
\pgfpathlineto{\pgfqpoint{5.251081in}{2.235586in}}%
\pgfpathlineto{\pgfqpoint{5.251430in}{2.041568in}}%
\pgfpathlineto{\pgfqpoint{5.251567in}{2.559705in}}%
\pgfpathlineto{\pgfqpoint{5.252089in}{2.297388in}}%
\pgfpathlineto{\pgfqpoint{5.252263in}{2.561682in}}%
\pgfpathlineto{\pgfqpoint{5.252734in}{2.123224in}}%
\pgfpathlineto{\pgfqpoint{5.253193in}{2.384885in}}%
\pgfpathlineto{\pgfqpoint{5.254121in}{2.529502in}}%
\pgfpathlineto{\pgfqpoint{5.254307in}{1.849894in}}%
\pgfpathlineto{\pgfqpoint{5.254714in}{2.534615in}}%
\pgfpathlineto{\pgfqpoint{5.255430in}{2.468581in}}%
\pgfpathlineto{\pgfqpoint{5.256230in}{2.096014in}}%
\pgfpathlineto{\pgfqpoint{5.256390in}{2.536305in}}%
\pgfpathlineto{\pgfqpoint{5.256538in}{2.426682in}}%
\pgfpathlineto{\pgfqpoint{5.256771in}{2.541639in}}%
\pgfpathlineto{\pgfqpoint{5.256624in}{1.993162in}}%
\pgfpathlineto{\pgfqpoint{5.257630in}{2.475405in}}%
\pgfpathlineto{\pgfqpoint{5.258304in}{1.676231in}}%
\pgfpathlineto{\pgfqpoint{5.257802in}{2.552791in}}%
\pgfpathlineto{\pgfqpoint{5.258732in}{2.322140in}}%
\pgfpathlineto{\pgfqpoint{5.258928in}{2.548305in}}%
\pgfpathlineto{\pgfqpoint{5.258940in}{2.100238in}}%
\pgfpathlineto{\pgfqpoint{5.259844in}{2.393708in}}%
\pgfpathlineto{\pgfqpoint{5.260332in}{2.545007in}}%
\pgfpathlineto{\pgfqpoint{5.260319in}{2.190363in}}%
\pgfpathlineto{\pgfqpoint{5.260940in}{2.516405in}}%
\pgfpathlineto{\pgfqpoint{5.261208in}{1.899803in}}%
\pgfpathlineto{\pgfqpoint{5.261670in}{2.541502in}}%
\pgfpathlineto{\pgfqpoint{5.262070in}{2.231600in}}%
\pgfpathlineto{\pgfqpoint{5.262519in}{2.560883in}}%
\pgfpathlineto{\pgfqpoint{5.262592in}{2.093451in}}%
\pgfpathlineto{\pgfqpoint{5.263185in}{2.518732in}}%
\pgfpathlineto{\pgfqpoint{5.264092in}{2.104415in}}%
\pgfpathlineto{\pgfqpoint{5.264116in}{2.551638in}}%
\pgfpathlineto{\pgfqpoint{5.264297in}{2.411386in}}%
\pgfpathlineto{\pgfqpoint{5.264502in}{2.512280in}}%
\pgfpathlineto{\pgfqpoint{5.264345in}{2.113888in}}%
\pgfpathlineto{\pgfqpoint{5.265382in}{2.407929in}}%
\pgfpathlineto{\pgfqpoint{5.266200in}{1.938306in}}%
\pgfpathlineto{\pgfqpoint{5.265888in}{2.515418in}}%
\pgfpathlineto{\pgfqpoint{5.266512in}{2.158704in}}%
\pgfpathlineto{\pgfqpoint{5.267328in}{2.539014in}}%
\pgfpathlineto{\pgfqpoint{5.267172in}{2.104014in}}%
\pgfpathlineto{\pgfqpoint{5.267628in}{2.374627in}}%
\pgfpathlineto{\pgfqpoint{5.267652in}{2.518322in}}%
\pgfpathlineto{\pgfqpoint{5.268525in}{2.085383in}}%
\pgfpathlineto{\pgfqpoint{5.268740in}{2.399540in}}%
\pgfpathlineto{\pgfqpoint{5.269158in}{2.530002in}}%
\pgfpathlineto{\pgfqpoint{5.269218in}{1.965622in}}%
\pgfpathlineto{\pgfqpoint{5.269754in}{2.404447in}}%
\pgfpathlineto{\pgfqpoint{5.269766in}{1.963572in}}%
\pgfpathlineto{\pgfqpoint{5.270147in}{2.501710in}}%
\pgfpathlineto{\pgfqpoint{5.270861in}{2.425844in}}%
\pgfpathlineto{\pgfqpoint{5.271894in}{2.532848in}}%
\pgfpathlineto{\pgfqpoint{5.271277in}{2.064928in}}%
\pgfpathlineto{\pgfqpoint{5.271917in}{2.501993in}}%
\pgfpathlineto{\pgfqpoint{5.272214in}{1.787184in}}%
\pgfpathlineto{\pgfqpoint{5.272960in}{2.533059in}}%
\pgfpathlineto{\pgfqpoint{5.273019in}{2.398415in}}%
\pgfpathlineto{\pgfqpoint{5.273326in}{2.558285in}}%
\pgfpathlineto{\pgfqpoint{5.273385in}{2.137207in}}%
\pgfpathlineto{\pgfqpoint{5.274093in}{2.397389in}}%
\pgfpathlineto{\pgfqpoint{5.274259in}{2.046487in}}%
\pgfpathlineto{\pgfqpoint{5.274954in}{2.516001in}}%
\pgfpathlineto{\pgfqpoint{5.275201in}{2.315506in}}%
\pgfpathlineto{\pgfqpoint{5.276246in}{2.528026in}}%
\pgfpathlineto{\pgfqpoint{5.275847in}{2.070881in}}%
\pgfpathlineto{\pgfqpoint{5.276317in}{2.466286in}}%
\pgfpathlineto{\pgfqpoint{5.277395in}{2.194482in}}%
\pgfpathlineto{\pgfqpoint{5.276505in}{2.532093in}}%
\pgfpathlineto{\pgfqpoint{5.277407in}{2.329834in}}%
\pgfpathlineto{\pgfqpoint{5.277851in}{2.553990in}}%
\pgfpathlineto{\pgfqpoint{5.278400in}{2.159621in}}%
\pgfpathlineto{\pgfqpoint{5.278517in}{2.422850in}}%
\pgfpathlineto{\pgfqpoint{5.278540in}{1.986408in}}%
\pgfpathlineto{\pgfqpoint{5.279077in}{2.586186in}}%
\pgfpathlineto{\pgfqpoint{5.279625in}{2.391577in}}%
\pgfpathlineto{\pgfqpoint{5.279939in}{2.530109in}}%
\pgfpathlineto{\pgfqpoint{5.279776in}{2.062636in}}%
\pgfpathlineto{\pgfqpoint{5.280694in}{2.332918in}}%
\pgfpathlineto{\pgfqpoint{5.280973in}{2.055111in}}%
\pgfpathlineto{\pgfqpoint{5.280938in}{2.542152in}}%
\pgfpathlineto{\pgfqpoint{5.281796in}{2.445545in}}%
\pgfpathlineto{\pgfqpoint{5.282491in}{2.105884in}}%
\pgfpathlineto{\pgfqpoint{5.282410in}{2.516607in}}%
\pgfpathlineto{\pgfqpoint{5.282918in}{2.343964in}}%
\pgfpathlineto{\pgfqpoint{5.283207in}{2.535667in}}%
\pgfpathlineto{\pgfqpoint{5.282976in}{2.074574in}}%
\pgfpathlineto{\pgfqpoint{5.284003in}{2.168111in}}%
\pgfpathlineto{\pgfqpoint{5.284383in}{1.971718in}}%
\pgfpathlineto{\pgfqpoint{5.284153in}{2.561274in}}%
\pgfpathlineto{\pgfqpoint{5.285073in}{2.366987in}}%
\pgfpathlineto{\pgfqpoint{5.285785in}{2.544688in}}%
\pgfpathlineto{\pgfqpoint{5.285579in}{1.993792in}}%
\pgfpathlineto{\pgfqpoint{5.286141in}{2.378301in}}%
\pgfpathlineto{\pgfqpoint{5.286187in}{2.027698in}}%
\pgfpathlineto{\pgfqpoint{5.286782in}{2.533640in}}%
\pgfpathlineto{\pgfqpoint{5.287240in}{2.353340in}}%
\pgfpathlineto{\pgfqpoint{5.288279in}{2.534462in}}%
\pgfpathlineto{\pgfqpoint{5.287560in}{2.107531in}}%
\pgfpathlineto{\pgfqpoint{5.288359in}{2.486967in}}%
\pgfpathlineto{\pgfqpoint{5.288450in}{2.063714in}}%
\pgfpathlineto{\pgfqpoint{5.288895in}{2.541974in}}%
\pgfpathlineto{\pgfqpoint{5.289464in}{2.364114in}}%
\pgfpathlineto{\pgfqpoint{5.290260in}{2.532786in}}%
\pgfpathlineto{\pgfqpoint{5.290441in}{1.979228in}}%
\pgfpathlineto{\pgfqpoint{5.290589in}{2.480249in}}%
\pgfpathlineto{\pgfqpoint{5.290736in}{1.953151in}}%
\pgfpathlineto{\pgfqpoint{5.291631in}{2.517620in}}%
\pgfpathlineto{\pgfqpoint{5.291699in}{2.335752in}}%
\pgfpathlineto{\pgfqpoint{5.292186in}{2.520116in}}%
\pgfpathlineto{\pgfqpoint{5.292174in}{2.134483in}}%
\pgfpathlineto{\pgfqpoint{5.292818in}{2.465383in}}%
\pgfpathlineto{\pgfqpoint{5.293765in}{2.108572in}}%
\pgfpathlineto{\pgfqpoint{5.293337in}{2.543114in}}%
\pgfpathlineto{\pgfqpoint{5.293923in}{2.480144in}}%
\pgfpathlineto{\pgfqpoint{5.294643in}{2.564015in}}%
\pgfpathlineto{\pgfqpoint{5.294440in}{1.836497in}}%
\pgfpathlineto{\pgfqpoint{5.294980in}{2.434325in}}%
\pgfpathlineto{\pgfqpoint{5.295742in}{2.523529in}}%
\pgfpathlineto{\pgfqpoint{5.296101in}{2.087198in}}%
\pgfpathlineto{\pgfqpoint{5.296560in}{2.550800in}}%
\pgfpathlineto{\pgfqpoint{5.297208in}{2.459203in}}%
\pgfpathlineto{\pgfqpoint{5.297219in}{2.128152in}}%
\pgfpathlineto{\pgfqpoint{5.298112in}{2.525433in}}%
\pgfpathlineto{\pgfqpoint{5.298313in}{2.460500in}}%
\pgfpathlineto{\pgfqpoint{5.298558in}{2.522509in}}%
\pgfpathlineto{\pgfqpoint{5.299281in}{2.207502in}}%
\pgfpathlineto{\pgfqpoint{5.299370in}{2.359477in}}%
\pgfpathlineto{\pgfqpoint{5.300247in}{1.983834in}}%
\pgfpathlineto{\pgfqpoint{5.299570in}{2.567875in}}%
\pgfpathlineto{\pgfqpoint{5.300457in}{2.267816in}}%
\pgfpathlineto{\pgfqpoint{5.300845in}{2.553646in}}%
\pgfpathlineto{\pgfqpoint{5.300734in}{2.021947in}}%
\pgfpathlineto{\pgfqpoint{5.301576in}{2.408750in}}%
\pgfpathlineto{\pgfqpoint{5.302393in}{1.947331in}}%
\pgfpathlineto{\pgfqpoint{5.302062in}{2.520940in}}%
\pgfpathlineto{\pgfqpoint{5.302680in}{2.367415in}}%
\pgfpathlineto{\pgfqpoint{5.302966in}{2.517689in}}%
\pgfpathlineto{\pgfqpoint{5.303440in}{2.017714in}}%
\pgfpathlineto{\pgfqpoint{5.303726in}{2.225129in}}%
\pgfpathlineto{\pgfqpoint{5.304144in}{2.131545in}}%
\pgfpathlineto{\pgfqpoint{5.304177in}{2.562476in}}%
\pgfpathlineto{\pgfqpoint{5.304473in}{2.377402in}}%
\pgfpathlineto{\pgfqpoint{5.304824in}{2.569319in}}%
\pgfpathlineto{\pgfqpoint{5.305099in}{2.100820in}}%
\pgfpathlineto{\pgfqpoint{5.305592in}{2.492354in}}%
\pgfpathlineto{\pgfqpoint{5.306358in}{2.120239in}}%
\pgfpathlineto{\pgfqpoint{5.305909in}{2.518874in}}%
\pgfpathlineto{\pgfqpoint{5.306828in}{2.419839in}}%
\pgfpathlineto{\pgfqpoint{5.306838in}{2.419917in}}%
\pgfpathlineto{\pgfqpoint{5.307046in}{1.944030in}}%
\pgfpathlineto{\pgfqpoint{5.306915in}{2.521247in}}%
\pgfpathlineto{\pgfqpoint{5.307962in}{2.136002in}}%
\pgfpathlineto{\pgfqpoint{5.308038in}{2.535535in}}%
\pgfpathlineto{\pgfqpoint{5.309071in}{2.360067in}}%
\pgfpathlineto{\pgfqpoint{5.309234in}{1.996426in}}%
\pgfpathlineto{\pgfqpoint{5.310069in}{2.556133in}}%
\pgfpathlineto{\pgfqpoint{5.310167in}{2.436510in}}%
\pgfpathlineto{\pgfqpoint{5.311033in}{1.995735in}}%
\pgfpathlineto{\pgfqpoint{5.311130in}{2.521979in}}%
\pgfpathlineto{\pgfqpoint{5.311238in}{2.282393in}}%
\pgfpathlineto{\pgfqpoint{5.312220in}{2.540157in}}%
\pgfpathlineto{\pgfqpoint{5.311443in}{2.140313in}}%
\pgfpathlineto{\pgfqpoint{5.312350in}{2.477531in}}%
\pgfpathlineto{\pgfqpoint{5.312985in}{1.978957in}}%
\pgfpathlineto{\pgfqpoint{5.312619in}{2.522790in}}%
\pgfpathlineto{\pgfqpoint{5.313458in}{2.390561in}}%
\pgfpathlineto{\pgfqpoint{5.314425in}{2.529413in}}%
\pgfpathlineto{\pgfqpoint{5.313738in}{2.146672in}}%
\pgfpathlineto{\pgfqpoint{5.314553in}{2.472063in}}%
\pgfpathlineto{\pgfqpoint{5.315539in}{2.125537in}}%
\pgfpathlineto{\pgfqpoint{5.315325in}{2.511092in}}%
\pgfpathlineto{\pgfqpoint{5.315656in}{2.255732in}}%
\pgfpathlineto{\pgfqpoint{5.316052in}{2.522761in}}%
\pgfpathlineto{\pgfqpoint{5.316244in}{2.123478in}}%
\pgfpathlineto{\pgfqpoint{5.316778in}{2.471247in}}%
\pgfpathlineto{\pgfqpoint{5.316927in}{2.088411in}}%
\pgfpathlineto{\pgfqpoint{5.317098in}{2.502056in}}%
\pgfpathlineto{\pgfqpoint{5.317108in}{1.987598in}}%
\pgfpathlineto{\pgfqpoint{5.317481in}{2.542257in}}%
\pgfpathlineto{\pgfqpoint{5.318205in}{2.435509in}}%
\pgfpathlineto{\pgfqpoint{5.318800in}{2.046116in}}%
\pgfpathlineto{\pgfqpoint{5.318492in}{2.524567in}}%
\pgfpathlineto{\pgfqpoint{5.319277in}{2.410962in}}%
\pgfpathlineto{\pgfqpoint{5.319288in}{2.545695in}}%
\pgfpathlineto{\pgfqpoint{5.319924in}{1.960494in}}%
\pgfpathlineto{\pgfqpoint{5.320379in}{2.335198in}}%
\pgfpathlineto{\pgfqpoint{5.321372in}{2.591801in}}%
\pgfpathlineto{\pgfqpoint{5.320854in}{2.056890in}}%
\pgfpathlineto{\pgfqpoint{5.321498in}{2.385670in}}%
\pgfpathlineto{\pgfqpoint{5.322352in}{1.997712in}}%
\pgfpathlineto{\pgfqpoint{5.322141in}{2.538544in}}%
\pgfpathlineto{\pgfqpoint{5.322604in}{2.222842in}}%
\pgfpathlineto{\pgfqpoint{5.323655in}{2.541794in}}%
\pgfpathlineto{\pgfqpoint{5.323519in}{2.104316in}}%
\pgfpathlineto{\pgfqpoint{5.323718in}{2.521715in}}%
\pgfpathlineto{\pgfqpoint{5.324054in}{2.003104in}}%
\pgfpathlineto{\pgfqpoint{5.324567in}{2.524621in}}%
\pgfpathlineto{\pgfqpoint{5.324829in}{2.404967in}}%
\pgfpathlineto{\pgfqpoint{5.325248in}{2.549499in}}%
\pgfpathlineto{\pgfqpoint{5.325436in}{2.067248in}}%
\pgfpathlineto{\pgfqpoint{5.325927in}{2.482536in}}%
\pgfpathlineto{\pgfqpoint{5.326542in}{2.024365in}}%
\pgfpathlineto{\pgfqpoint{5.326344in}{2.490950in}}%
\pgfpathlineto{\pgfqpoint{5.327053in}{2.293626in}}%
\pgfpathlineto{\pgfqpoint{5.327323in}{2.550847in}}%
\pgfpathlineto{\pgfqpoint{5.327136in}{1.958998in}}%
\pgfpathlineto{\pgfqpoint{5.328165in}{2.440439in}}%
\pgfpathlineto{\pgfqpoint{5.328310in}{2.100926in}}%
\pgfpathlineto{\pgfqpoint{5.328705in}{2.554107in}}%
\pgfpathlineto{\pgfqpoint{5.329264in}{2.186905in}}%
\pgfpathlineto{\pgfqpoint{5.330206in}{2.529037in}}%
\pgfpathlineto{\pgfqpoint{5.329544in}{2.098464in}}%
\pgfpathlineto{\pgfqpoint{5.330382in}{2.519462in}}%
\pgfpathlineto{\pgfqpoint{5.331187in}{2.048400in}}%
\pgfpathlineto{\pgfqpoint{5.330867in}{2.540162in}}%
\pgfpathlineto{\pgfqpoint{5.331496in}{2.463625in}}%
\pgfpathlineto{\pgfqpoint{5.332391in}{2.123882in}}%
\pgfpathlineto{\pgfqpoint{5.332412in}{2.551955in}}%
\pgfpathlineto{\pgfqpoint{5.332617in}{2.138632in}}%
\pgfpathlineto{\pgfqpoint{5.333274in}{2.544353in}}%
\pgfpathlineto{\pgfqpoint{5.332669in}{2.131570in}}%
\pgfpathlineto{\pgfqpoint{5.333736in}{2.480527in}}%
\pgfpathlineto{\pgfqpoint{5.334135in}{2.173529in}}%
\pgfpathlineto{\pgfqpoint{5.334504in}{2.541868in}}%
\pgfpathlineto{\pgfqpoint{5.334852in}{2.214126in}}%
\pgfpathlineto{\pgfqpoint{5.335770in}{2.549427in}}%
\pgfpathlineto{\pgfqpoint{5.335852in}{1.738530in}}%
\pgfpathlineto{\pgfqpoint{5.335985in}{2.482343in}}%
\pgfpathlineto{\pgfqpoint{5.337013in}{1.737715in}}%
\pgfpathlineto{\pgfqpoint{5.336158in}{2.583851in}}%
\pgfpathlineto{\pgfqpoint{5.337105in}{2.111488in}}%
\pgfpathlineto{\pgfqpoint{5.338181in}{2.547944in}}%
\pgfpathlineto{\pgfqpoint{5.338191in}{1.994457in}}%
\pgfpathlineto{\pgfqpoint{5.338211in}{2.386830in}}%
\pgfpathlineto{\pgfqpoint{5.338698in}{2.087632in}}%
\pgfpathlineto{\pgfqpoint{5.338728in}{2.539522in}}%
\pgfpathlineto{\pgfqpoint{5.339315in}{2.271440in}}%
\pgfpathlineto{\pgfqpoint{5.339487in}{2.550746in}}%
\pgfpathlineto{\pgfqpoint{5.339548in}{2.081754in}}%
\pgfpathlineto{\pgfqpoint{5.340416in}{2.462450in}}%
\pgfpathlineto{\pgfqpoint{5.340618in}{2.082771in}}%
\pgfpathlineto{\pgfqpoint{5.341374in}{2.534905in}}%
\pgfpathlineto{\pgfqpoint{5.341525in}{2.505670in}}%
\pgfpathlineto{\pgfqpoint{5.342399in}{2.111335in}}%
\pgfpathlineto{\pgfqpoint{5.342168in}{2.543747in}}%
\pgfpathlineto{\pgfqpoint{5.342650in}{2.322249in}}%
\pgfpathlineto{\pgfqpoint{5.343082in}{2.540569in}}%
\pgfpathlineto{\pgfqpoint{5.342791in}{1.998457in}}%
\pgfpathlineto{\pgfqpoint{5.343733in}{2.413390in}}%
\pgfpathlineto{\pgfqpoint{5.344283in}{1.962962in}}%
\pgfpathlineto{\pgfqpoint{5.344423in}{2.528980in}}%
\pgfpathlineto{\pgfqpoint{5.344842in}{2.385106in}}%
\pgfpathlineto{\pgfqpoint{5.345501in}{2.532464in}}%
\pgfpathlineto{\pgfqpoint{5.344942in}{1.916356in}}%
\pgfpathlineto{\pgfqpoint{5.345939in}{2.463094in}}%
\pgfpathlineto{\pgfqpoint{5.346029in}{2.089658in}}%
\pgfpathlineto{\pgfqpoint{5.346238in}{2.526804in}}%
\pgfpathlineto{\pgfqpoint{5.347043in}{2.462206in}}%
\pgfpathlineto{\pgfqpoint{5.347827in}{2.106487in}}%
\pgfpathlineto{\pgfqpoint{5.348085in}{2.541667in}}%
\pgfpathlineto{\pgfqpoint{5.348144in}{2.403806in}}%
\pgfpathlineto{\pgfqpoint{5.349144in}{2.548740in}}%
\pgfpathlineto{\pgfqpoint{5.348362in}{1.914386in}}%
\pgfpathlineto{\pgfqpoint{5.349262in}{2.510073in}}%
\pgfpathlineto{\pgfqpoint{5.350338in}{2.022001in}}%
\pgfpathlineto{\pgfqpoint{5.349687in}{2.529064in}}%
\pgfpathlineto{\pgfqpoint{5.350377in}{2.402892in}}%
\pgfpathlineto{\pgfqpoint{5.350417in}{2.545967in}}%
\pgfpathlineto{\pgfqpoint{5.350397in}{2.015779in}}%
\pgfpathlineto{\pgfqpoint{5.351470in}{2.449450in}}%
\pgfpathlineto{\pgfqpoint{5.352393in}{2.107220in}}%
\pgfpathlineto{\pgfqpoint{5.351676in}{2.511410in}}%
\pgfpathlineto{\pgfqpoint{5.352569in}{2.375117in}}%
\pgfpathlineto{\pgfqpoint{5.353451in}{2.528890in}}%
\pgfpathlineto{\pgfqpoint{5.353637in}{1.995889in}}%
\pgfpathlineto{\pgfqpoint{5.353676in}{2.343572in}}%
\pgfpathlineto{\pgfqpoint{5.354076in}{2.531201in}}%
\pgfpathlineto{\pgfqpoint{5.354223in}{2.152940in}}%
\pgfpathlineto{\pgfqpoint{5.354779in}{2.375830in}}%
\pgfpathlineto{\pgfqpoint{5.354818in}{2.155865in}}%
\pgfpathlineto{\pgfqpoint{5.355617in}{2.529225in}}%
\pgfpathlineto{\pgfqpoint{5.355880in}{2.380023in}}%
\pgfpathlineto{\pgfqpoint{5.356589in}{2.529945in}}%
\pgfpathlineto{\pgfqpoint{5.355928in}{1.953254in}}%
\pgfpathlineto{\pgfqpoint{5.356958in}{2.359219in}}%
\pgfpathlineto{\pgfqpoint{5.357705in}{2.125475in}}%
\pgfpathlineto{\pgfqpoint{5.357685in}{2.524604in}}%
\pgfpathlineto{\pgfqpoint{5.358053in}{2.471756in}}%
\pgfpathlineto{\pgfqpoint{5.358972in}{2.542670in}}%
\pgfpathlineto{\pgfqpoint{5.359088in}{2.092254in}}%
\pgfpathlineto{\pgfqpoint{5.359136in}{2.513087in}}%
\pgfpathlineto{\pgfqpoint{5.360119in}{2.041561in}}%
\pgfpathlineto{\pgfqpoint{5.359281in}{2.532191in}}%
\pgfpathlineto{\pgfqpoint{5.360244in}{2.376074in}}%
\pgfpathlineto{\pgfqpoint{5.360870in}{2.563390in}}%
\pgfpathlineto{\pgfqpoint{5.360495in}{2.021783in}}%
\pgfpathlineto{\pgfqpoint{5.361360in}{2.446814in}}%
\pgfpathlineto{\pgfqpoint{5.362089in}{1.891171in}}%
\pgfpathlineto{\pgfqpoint{5.361485in}{2.548232in}}%
\pgfpathlineto{\pgfqpoint{5.362492in}{2.349466in}}%
\pgfpathlineto{\pgfqpoint{5.362597in}{2.545797in}}%
\pgfpathlineto{\pgfqpoint{5.363305in}{2.091368in}}%
\pgfpathlineto{\pgfqpoint{5.363601in}{2.476599in}}%
\pgfpathlineto{\pgfqpoint{5.363802in}{2.146711in}}%
\pgfpathlineto{\pgfqpoint{5.364250in}{2.534280in}}%
\pgfpathlineto{\pgfqpoint{5.364717in}{2.393657in}}%
\pgfpathlineto{\pgfqpoint{5.365792in}{2.538100in}}%
\pgfpathlineto{\pgfqpoint{5.365688in}{2.070615in}}%
\pgfpathlineto{\pgfqpoint{5.365811in}{2.344893in}}%
\pgfpathlineto{\pgfqpoint{5.366467in}{1.977350in}}%
\pgfpathlineto{\pgfqpoint{5.366818in}{2.541731in}}%
\pgfpathlineto{\pgfqpoint{5.366893in}{2.368316in}}%
\pgfpathlineto{\pgfqpoint{5.367689in}{2.530426in}}%
\pgfpathlineto{\pgfqpoint{5.367424in}{1.995394in}}%
\pgfpathlineto{\pgfqpoint{5.368020in}{2.480445in}}%
\pgfpathlineto{\pgfqpoint{5.369030in}{2.553456in}}%
\pgfpathlineto{\pgfqpoint{5.368398in}{2.046153in}}%
\pgfpathlineto{\pgfqpoint{5.369040in}{2.444344in}}%
\pgfpathlineto{\pgfqpoint{5.369332in}{2.037601in}}%
\pgfpathlineto{\pgfqpoint{5.369426in}{2.531465in}}%
\pgfpathlineto{\pgfqpoint{5.370142in}{2.411619in}}%
\pgfpathlineto{\pgfqpoint{5.370687in}{2.544771in}}%
\pgfpathlineto{\pgfqpoint{5.370743in}{1.893832in}}%
\pgfpathlineto{\pgfqpoint{5.371203in}{2.351635in}}%
\pgfpathlineto{\pgfqpoint{5.371353in}{2.062022in}}%
\pgfpathlineto{\pgfqpoint{5.371625in}{2.521168in}}%
\pgfpathlineto{\pgfqpoint{5.372299in}{2.371118in}}%
\pgfpathlineto{\pgfqpoint{5.372449in}{2.549841in}}%
\pgfpathlineto{\pgfqpoint{5.373169in}{2.128623in}}%
\pgfpathlineto{\pgfqpoint{5.373412in}{2.411220in}}%
\pgfpathlineto{\pgfqpoint{5.373953in}{2.128816in}}%
\pgfpathlineto{\pgfqpoint{5.373701in}{2.557185in}}%
\pgfpathlineto{\pgfqpoint{5.374530in}{2.308863in}}%
\pgfpathlineto{\pgfqpoint{5.375469in}{2.573294in}}%
\pgfpathlineto{\pgfqpoint{5.375377in}{1.994596in}}%
\pgfpathlineto{\pgfqpoint{5.375627in}{2.239282in}}%
\pgfpathlineto{\pgfqpoint{5.375637in}{2.094580in}}%
\pgfpathlineto{\pgfqpoint{5.376138in}{2.547481in}}%
\pgfpathlineto{\pgfqpoint{5.376731in}{2.328207in}}%
\pgfpathlineto{\pgfqpoint{5.376796in}{2.110078in}}%
\pgfpathlineto{\pgfqpoint{5.377083in}{2.532868in}}%
\pgfpathlineto{\pgfqpoint{5.377832in}{2.397521in}}%
\pgfpathlineto{\pgfqpoint{5.377878in}{2.048789in}}%
\pgfpathlineto{\pgfqpoint{5.378717in}{2.545688in}}%
\pgfpathlineto{\pgfqpoint{5.378911in}{2.427396in}}%
\pgfpathlineto{\pgfqpoint{5.379150in}{2.517231in}}%
\pgfpathlineto{\pgfqpoint{5.379813in}{2.126708in}}%
\pgfpathlineto{\pgfqpoint{5.380024in}{2.461505in}}%
\pgfpathlineto{\pgfqpoint{5.380585in}{1.764791in}}%
\pgfpathlineto{\pgfqpoint{5.380061in}{2.545123in}}%
\pgfpathlineto{\pgfqpoint{5.381135in}{2.428773in}}%
\pgfpathlineto{\pgfqpoint{5.381758in}{2.115612in}}%
\pgfpathlineto{\pgfqpoint{5.381849in}{2.524351in}}%
\pgfpathlineto{\pgfqpoint{5.382215in}{2.193554in}}%
\pgfpathlineto{\pgfqpoint{5.382224in}{2.524349in}}%
\pgfpathlineto{\pgfqpoint{5.382855in}{1.942696in}}%
\pgfpathlineto{\pgfqpoint{5.383320in}{2.429743in}}%
\pgfpathlineto{\pgfqpoint{5.383365in}{2.101845in}}%
\pgfpathlineto{\pgfqpoint{5.383657in}{2.560791in}}%
\pgfpathlineto{\pgfqpoint{5.384431in}{2.335234in}}%
\pgfpathlineto{\pgfqpoint{5.385040in}{2.526332in}}%
\pgfpathlineto{\pgfqpoint{5.384685in}{2.038001in}}%
\pgfpathlineto{\pgfqpoint{5.385548in}{2.415989in}}%
\pgfpathlineto{\pgfqpoint{5.386572in}{2.543522in}}%
\pgfpathlineto{\pgfqpoint{5.386635in}{2.053339in}}%
\pgfpathlineto{\pgfqpoint{5.386825in}{2.551601in}}%
\pgfpathlineto{\pgfqpoint{5.387123in}{2.011581in}}%
\pgfpathlineto{\pgfqpoint{5.387747in}{2.408473in}}%
\pgfpathlineto{\pgfqpoint{5.387873in}{2.164715in}}%
\pgfpathlineto{\pgfqpoint{5.388837in}{2.538236in}}%
\pgfpathlineto{\pgfqpoint{5.389907in}{2.045315in}}%
\pgfpathlineto{\pgfqpoint{5.389170in}{2.548499in}}%
\pgfpathlineto{\pgfqpoint{5.389943in}{2.293029in}}%
\pgfpathlineto{\pgfqpoint{5.390024in}{2.528966in}}%
\pgfpathlineto{\pgfqpoint{5.390176in}{2.030940in}}%
\pgfpathlineto{\pgfqpoint{5.391054in}{2.403358in}}%
\pgfpathlineto{\pgfqpoint{5.392029in}{1.927424in}}%
\pgfpathlineto{\pgfqpoint{5.391466in}{2.537541in}}%
\pgfpathlineto{\pgfqpoint{5.392172in}{2.358598in}}%
\pgfpathlineto{\pgfqpoint{5.392645in}{2.542586in}}%
\pgfpathlineto{\pgfqpoint{5.393145in}{2.147356in}}%
\pgfpathlineto{\pgfqpoint{5.393260in}{2.384809in}}%
\pgfpathlineto{\pgfqpoint{5.393456in}{2.087867in}}%
\pgfpathlineto{\pgfqpoint{5.393937in}{2.524216in}}%
\pgfpathlineto{\pgfqpoint{5.394363in}{2.370956in}}%
\pgfpathlineto{\pgfqpoint{5.395198in}{2.532944in}}%
\pgfpathlineto{\pgfqpoint{5.394941in}{2.012128in}}%
\pgfpathlineto{\pgfqpoint{5.395464in}{2.291177in}}%
\pgfpathlineto{\pgfqpoint{5.396216in}{2.539029in}}%
\pgfpathlineto{\pgfqpoint{5.395685in}{2.060242in}}%
\pgfpathlineto{\pgfqpoint{5.396579in}{2.399149in}}%
\pgfpathlineto{\pgfqpoint{5.397162in}{1.821290in}}%
\pgfpathlineto{\pgfqpoint{5.397356in}{2.552588in}}%
\pgfpathlineto{\pgfqpoint{5.397682in}{2.360233in}}%
\pgfpathlineto{\pgfqpoint{5.397797in}{2.541969in}}%
\pgfpathlineto{\pgfqpoint{5.398334in}{2.161145in}}%
\pgfpathlineto{\pgfqpoint{5.398791in}{2.478265in}}%
\pgfpathlineto{\pgfqpoint{5.399055in}{2.024729in}}%
\pgfpathlineto{\pgfqpoint{5.398835in}{2.554973in}}%
\pgfpathlineto{\pgfqpoint{5.399898in}{2.259754in}}%
\pgfpathlineto{\pgfqpoint{5.400380in}{2.554764in}}%
\pgfpathlineto{\pgfqpoint{5.400966in}{2.068247in}}%
\pgfpathlineto{\pgfqpoint{5.401010in}{2.444644in}}%
\pgfpathlineto{\pgfqpoint{5.401866in}{2.061817in}}%
\pgfpathlineto{\pgfqpoint{5.401683in}{2.508045in}}%
\pgfpathlineto{\pgfqpoint{5.402119in}{2.381422in}}%
\pgfpathlineto{\pgfqpoint{5.402320in}{2.545875in}}%
\pgfpathlineto{\pgfqpoint{5.403034in}{1.840927in}}%
\pgfpathlineto{\pgfqpoint{5.403173in}{2.452995in}}%
\pgfpathlineto{\pgfqpoint{5.404078in}{2.530642in}}%
\pgfpathlineto{\pgfqpoint{5.404286in}{2.052504in}}%
\pgfpathlineto{\pgfqpoint{5.404338in}{2.542501in}}%
\pgfpathlineto{\pgfqpoint{5.405395in}{2.435140in}}%
\pgfpathlineto{\pgfqpoint{5.406035in}{2.095399in}}%
\pgfpathlineto{\pgfqpoint{5.405499in}{2.532138in}}%
\pgfpathlineto{\pgfqpoint{5.406511in}{2.308560in}}%
\pgfpathlineto{\pgfqpoint{5.406942in}{2.537260in}}%
\pgfpathlineto{\pgfqpoint{5.406580in}{2.125880in}}%
\pgfpathlineto{\pgfqpoint{5.407631in}{2.454079in}}%
\pgfpathlineto{\pgfqpoint{5.407993in}{1.913418in}}%
\pgfpathlineto{\pgfqpoint{5.408328in}{2.547222in}}%
\pgfpathlineto{\pgfqpoint{5.408732in}{2.189392in}}%
\pgfpathlineto{\pgfqpoint{5.409196in}{2.529768in}}%
\pgfpathlineto{\pgfqpoint{5.409804in}{1.969387in}}%
\pgfpathlineto{\pgfqpoint{5.409847in}{2.414050in}}%
\pgfpathlineto{\pgfqpoint{5.410344in}{2.102246in}}%
\pgfpathlineto{\pgfqpoint{5.410027in}{2.535181in}}%
\pgfpathlineto{\pgfqpoint{5.410951in}{2.350004in}}%
\pgfpathlineto{\pgfqpoint{5.411352in}{2.557009in}}%
\pgfpathlineto{\pgfqpoint{5.411583in}{1.770013in}}%
\pgfpathlineto{\pgfqpoint{5.412052in}{2.524905in}}%
\pgfpathlineto{\pgfqpoint{5.412809in}{2.089800in}}%
\pgfpathlineto{\pgfqpoint{5.413158in}{2.347822in}}%
\pgfpathlineto{\pgfqpoint{5.413905in}{2.559566in}}%
\pgfpathlineto{\pgfqpoint{5.413931in}{2.079868in}}%
\pgfpathlineto{\pgfqpoint{5.414270in}{2.386298in}}%
\pgfpathlineto{\pgfqpoint{5.414346in}{1.962644in}}%
\pgfpathlineto{\pgfqpoint{5.414922in}{2.556353in}}%
\pgfpathlineto{\pgfqpoint{5.415379in}{2.285042in}}%
\pgfpathlineto{\pgfqpoint{5.415692in}{2.561622in}}%
\pgfpathlineto{\pgfqpoint{5.416426in}{2.048504in}}%
\pgfpathlineto{\pgfqpoint{5.416494in}{2.472017in}}%
\pgfpathlineto{\pgfqpoint{5.417555in}{1.758815in}}%
\pgfpathlineto{\pgfqpoint{5.417597in}{2.538635in}}%
\pgfpathlineto{\pgfqpoint{5.418454in}{2.070012in}}%
\pgfpathlineto{\pgfqpoint{5.418320in}{2.544103in}}%
\pgfpathlineto{\pgfqpoint{5.418714in}{2.320080in}}%
\pgfpathlineto{\pgfqpoint{5.419251in}{2.524583in}}%
\pgfpathlineto{\pgfqpoint{5.419577in}{2.139094in}}%
\pgfpathlineto{\pgfqpoint{5.419828in}{2.415644in}}%
\pgfpathlineto{\pgfqpoint{5.420096in}{2.023006in}}%
\pgfpathlineto{\pgfqpoint{5.420806in}{2.545826in}}%
\pgfpathlineto{\pgfqpoint{5.420940in}{2.396052in}}%
\pgfpathlineto{\pgfqpoint{5.421565in}{1.813325in}}%
\pgfpathlineto{\pgfqpoint{5.421915in}{2.512667in}}%
\pgfpathlineto{\pgfqpoint{5.422015in}{2.426815in}}%
\pgfpathlineto{\pgfqpoint{5.422655in}{2.547566in}}%
\pgfpathlineto{\pgfqpoint{5.422938in}{2.002197in}}%
\pgfpathlineto{\pgfqpoint{5.423120in}{2.512061in}}%
\pgfpathlineto{\pgfqpoint{5.424058in}{1.942099in}}%
\pgfpathlineto{\pgfqpoint{5.424091in}{2.543001in}}%
\pgfpathlineto{\pgfqpoint{5.424232in}{2.413969in}}%
\pgfpathlineto{\pgfqpoint{5.424546in}{1.996824in}}%
\pgfpathlineto{\pgfqpoint{5.425067in}{2.535789in}}%
\pgfpathlineto{\pgfqpoint{5.425340in}{2.187047in}}%
\pgfpathlineto{\pgfqpoint{5.425521in}{2.542796in}}%
\pgfpathlineto{\pgfqpoint{5.426231in}{2.068172in}}%
\pgfpathlineto{\pgfqpoint{5.426461in}{2.504793in}}%
\pgfpathlineto{\pgfqpoint{5.426791in}{1.972446in}}%
\pgfpathlineto{\pgfqpoint{5.427251in}{2.541608in}}%
\pgfpathlineto{\pgfqpoint{5.427572in}{2.267980in}}%
\pgfpathlineto{\pgfqpoint{5.428344in}{2.539546in}}%
\pgfpathlineto{\pgfqpoint{5.428467in}{2.039667in}}%
\pgfpathlineto{\pgfqpoint{5.428688in}{2.375959in}}%
\pgfpathlineto{\pgfqpoint{5.429138in}{2.069375in}}%
\pgfpathlineto{\pgfqpoint{5.428966in}{2.558635in}}%
\pgfpathlineto{\pgfqpoint{5.429784in}{2.422397in}}%
\pgfpathlineto{\pgfqpoint{5.430291in}{2.559391in}}%
\pgfpathlineto{\pgfqpoint{5.430013in}{2.063678in}}%
\pgfpathlineto{\pgfqpoint{5.430821in}{2.430218in}}%
\pgfpathlineto{\pgfqpoint{5.430976in}{1.984189in}}%
\pgfpathlineto{\pgfqpoint{5.431334in}{2.540130in}}%
\pgfpathlineto{\pgfqpoint{5.431928in}{2.323595in}}%
\pgfpathlineto{\pgfqpoint{5.432733in}{2.546658in}}%
\pgfpathlineto{\pgfqpoint{5.432172in}{1.933741in}}%
\pgfpathlineto{\pgfqpoint{5.433033in}{2.472774in}}%
\pgfpathlineto{\pgfqpoint{5.433633in}{2.560621in}}%
\pgfpathlineto{\pgfqpoint{5.434143in}{2.053156in}}%
\pgfpathlineto{\pgfqpoint{5.434652in}{2.544682in}}%
\pgfpathlineto{\pgfqpoint{5.434426in}{2.038971in}}%
\pgfpathlineto{\pgfqpoint{5.435258in}{2.249768in}}%
\pgfpathlineto{\pgfqpoint{5.435266in}{1.938404in}}%
\pgfpathlineto{\pgfqpoint{5.436096in}{2.566303in}}%
\pgfpathlineto{\pgfqpoint{5.436361in}{2.413711in}}%
\pgfpathlineto{\pgfqpoint{5.437374in}{1.985735in}}%
\pgfpathlineto{\pgfqpoint{5.437326in}{2.534471in}}%
\pgfpathlineto{\pgfqpoint{5.437471in}{2.362628in}}%
\pgfpathlineto{\pgfqpoint{5.438545in}{2.533240in}}%
\pgfpathlineto{\pgfqpoint{5.438401in}{2.089306in}}%
\pgfpathlineto{\pgfqpoint{5.438569in}{2.457863in}}%
\pgfpathlineto{\pgfqpoint{5.439632in}{2.076645in}}%
\pgfpathlineto{\pgfqpoint{5.439544in}{2.550183in}}%
\pgfpathlineto{\pgfqpoint{5.439688in}{2.275042in}}%
\pgfpathlineto{\pgfqpoint{5.440215in}{1.892228in}}%
\pgfpathlineto{\pgfqpoint{5.439919in}{2.524364in}}%
\pgfpathlineto{\pgfqpoint{5.440780in}{2.417837in}}%
\pgfpathlineto{\pgfqpoint{5.441234in}{2.589407in}}%
\pgfpathlineto{\pgfqpoint{5.441528in}{2.005326in}}%
\pgfpathlineto{\pgfqpoint{5.441886in}{2.474121in}}%
\pgfpathlineto{\pgfqpoint{5.442513in}{1.941552in}}%
\pgfpathlineto{\pgfqpoint{5.442013in}{2.550962in}}%
\pgfpathlineto{\pgfqpoint{5.443004in}{2.385819in}}%
\pgfpathlineto{\pgfqpoint{5.443361in}{2.537811in}}%
\pgfpathlineto{\pgfqpoint{5.444057in}{2.070950in}}%
\pgfpathlineto{\pgfqpoint{5.444112in}{2.443255in}}%
\pgfpathlineto{\pgfqpoint{5.444246in}{1.988518in}}%
\pgfpathlineto{\pgfqpoint{5.444562in}{2.590759in}}%
\pgfpathlineto{\pgfqpoint{5.445217in}{2.437014in}}%
\pgfpathlineto{\pgfqpoint{5.445666in}{2.550996in}}%
\pgfpathlineto{\pgfqpoint{5.446177in}{2.080299in}}%
\pgfpathlineto{\pgfqpoint{5.446311in}{2.422075in}}%
\pgfpathlineto{\pgfqpoint{5.446429in}{1.964403in}}%
\pgfpathlineto{\pgfqpoint{5.447167in}{2.532141in}}%
\pgfpathlineto{\pgfqpoint{5.447426in}{2.400362in}}%
\pgfpathlineto{\pgfqpoint{5.448029in}{2.515671in}}%
\pgfpathlineto{\pgfqpoint{5.448201in}{1.787981in}}%
\pgfpathlineto{\pgfqpoint{5.448522in}{2.297373in}}%
\pgfpathlineto{\pgfqpoint{5.449210in}{2.039150in}}%
\pgfpathlineto{\pgfqpoint{5.448725in}{2.545930in}}%
\pgfpathlineto{\pgfqpoint{5.449608in}{2.134933in}}%
\pgfpathlineto{\pgfqpoint{5.449880in}{2.539003in}}%
\pgfpathlineto{\pgfqpoint{5.449990in}{2.045037in}}%
\pgfpathlineto{\pgfqpoint{5.450721in}{2.450172in}}%
\pgfpathlineto{\pgfqpoint{5.451405in}{1.903579in}}%
\pgfpathlineto{\pgfqpoint{5.451778in}{2.528238in}}%
\pgfpathlineto{\pgfqpoint{5.451817in}{2.117028in}}%
\pgfpathlineto{\pgfqpoint{5.451964in}{2.560116in}}%
\pgfpathlineto{\pgfqpoint{5.452685in}{1.895611in}}%
\pgfpathlineto{\pgfqpoint{5.452933in}{2.456142in}}%
\pgfpathlineto{\pgfqpoint{5.453420in}{2.122729in}}%
\pgfpathlineto{\pgfqpoint{5.453783in}{2.555333in}}%
\pgfpathlineto{\pgfqpoint{5.454038in}{2.448526in}}%
\pgfpathlineto{\pgfqpoint{5.454123in}{2.529022in}}%
\pgfpathlineto{\pgfqpoint{5.454609in}{2.032990in}}%
\pgfpathlineto{\pgfqpoint{5.454948in}{2.320911in}}%
\pgfpathlineto{\pgfqpoint{5.455725in}{1.928558in}}%
\pgfpathlineto{\pgfqpoint{5.455487in}{2.562115in}}%
\pgfpathlineto{\pgfqpoint{5.456056in}{2.374394in}}%
\pgfpathlineto{\pgfqpoint{5.457030in}{2.543851in}}%
\pgfpathlineto{\pgfqpoint{5.456194in}{2.101426in}}%
\pgfpathlineto{\pgfqpoint{5.457160in}{2.394284in}}%
\pgfpathlineto{\pgfqpoint{5.458125in}{2.059904in}}%
\pgfpathlineto{\pgfqpoint{5.457490in}{2.552554in}}%
\pgfpathlineto{\pgfqpoint{5.458255in}{2.438378in}}%
\pgfpathlineto{\pgfqpoint{5.458507in}{2.520505in}}%
\pgfpathlineto{\pgfqpoint{5.458728in}{2.039627in}}%
\pgfpathlineto{\pgfqpoint{5.459339in}{2.444779in}}%
\pgfpathlineto{\pgfqpoint{5.459384in}{2.010685in}}%
\pgfpathlineto{\pgfqpoint{5.459369in}{2.559682in}}%
\pgfpathlineto{\pgfqpoint{5.460450in}{2.415570in}}%
\pgfpathlineto{\pgfqpoint{5.461475in}{1.787678in}}%
\pgfpathlineto{\pgfqpoint{5.460853in}{2.516355in}}%
\pgfpathlineto{\pgfqpoint{5.461551in}{2.331765in}}%
\pgfpathlineto{\pgfqpoint{5.462082in}{2.540656in}}%
\pgfpathlineto{\pgfqpoint{5.462180in}{1.962970in}}%
\pgfpathlineto{\pgfqpoint{5.462664in}{2.396071in}}%
\pgfpathlineto{\pgfqpoint{5.463609in}{2.022649in}}%
\pgfpathlineto{\pgfqpoint{5.463571in}{2.566515in}}%
\pgfpathlineto{\pgfqpoint{5.463775in}{2.304389in}}%
\pgfpathlineto{\pgfqpoint{5.464581in}{2.590046in}}%
\pgfpathlineto{\pgfqpoint{5.464039in}{2.004610in}}%
\pgfpathlineto{\pgfqpoint{5.464875in}{2.474669in}}%
\pgfpathlineto{\pgfqpoint{5.465303in}{1.885874in}}%
\pgfpathlineto{\pgfqpoint{5.465747in}{2.526317in}}%
\pgfpathlineto{\pgfqpoint{5.465987in}{2.424173in}}%
\pgfpathlineto{\pgfqpoint{5.466459in}{2.077956in}}%
\pgfpathlineto{\pgfqpoint{5.466017in}{2.560518in}}%
\pgfpathlineto{\pgfqpoint{5.467096in}{2.438090in}}%
\pgfpathlineto{\pgfqpoint{5.467597in}{1.971145in}}%
\pgfpathlineto{\pgfqpoint{5.467508in}{2.559006in}}%
\pgfpathlineto{\pgfqpoint{5.468076in}{2.452620in}}%
\pgfpathlineto{\pgfqpoint{5.468419in}{2.583863in}}%
\pgfpathlineto{\pgfqpoint{5.468210in}{2.088077in}}%
\pgfpathlineto{\pgfqpoint{5.469179in}{2.401332in}}%
\pgfpathlineto{\pgfqpoint{5.469455in}{2.538278in}}%
\pgfpathlineto{\pgfqpoint{5.469284in}{2.136736in}}%
\pgfpathlineto{\pgfqpoint{5.469492in}{2.418833in}}%
\pgfpathlineto{\pgfqpoint{5.469500in}{1.971625in}}%
\pgfpathlineto{\pgfqpoint{5.470436in}{2.526585in}}%
\pgfpathlineto{\pgfqpoint{5.470607in}{2.132195in}}%
\pgfpathlineto{\pgfqpoint{5.471126in}{2.570880in}}%
\pgfpathlineto{\pgfqpoint{5.471252in}{2.077974in}}%
\pgfpathlineto{\pgfqpoint{5.471719in}{2.321437in}}%
\pgfpathlineto{\pgfqpoint{5.472799in}{2.547278in}}%
\pgfpathlineto{\pgfqpoint{5.472304in}{1.902736in}}%
\pgfpathlineto{\pgfqpoint{5.472836in}{2.429766in}}%
\pgfpathlineto{\pgfqpoint{5.473640in}{2.035061in}}%
\pgfpathlineto{\pgfqpoint{5.473374in}{2.564163in}}%
\pgfpathlineto{\pgfqpoint{5.473949in}{2.331056in}}%
\pgfpathlineto{\pgfqpoint{5.474325in}{2.576431in}}%
\pgfpathlineto{\pgfqpoint{5.474861in}{2.089195in}}%
\pgfpathlineto{\pgfqpoint{5.475030in}{2.483794in}}%
\pgfpathlineto{\pgfqpoint{5.475464in}{1.718304in}}%
\pgfpathlineto{\pgfqpoint{5.475060in}{2.536315in}}%
\pgfpathlineto{\pgfqpoint{5.476138in}{2.396935in}}%
\pgfpathlineto{\pgfqpoint{5.476321in}{1.917302in}}%
\pgfpathlineto{\pgfqpoint{5.476270in}{2.550567in}}%
\pgfpathlineto{\pgfqpoint{5.477243in}{2.339333in}}%
\pgfpathlineto{\pgfqpoint{5.478025in}{2.552710in}}%
\pgfpathlineto{\pgfqpoint{5.477389in}{2.087795in}}%
\pgfpathlineto{\pgfqpoint{5.478353in}{2.401668in}}%
\pgfpathlineto{\pgfqpoint{5.478935in}{2.099542in}}%
\pgfpathlineto{\pgfqpoint{5.478637in}{2.513169in}}%
\pgfpathlineto{\pgfqpoint{5.479466in}{2.329239in}}%
\pgfpathlineto{\pgfqpoint{5.480135in}{2.552495in}}%
\pgfpathlineto{\pgfqpoint{5.480461in}{2.033688in}}%
\pgfpathlineto{\pgfqpoint{5.480577in}{2.442017in}}%
\pgfpathlineto{\pgfqpoint{5.480969in}{1.906966in}}%
\pgfpathlineto{\pgfqpoint{5.481208in}{2.514963in}}%
\pgfpathlineto{\pgfqpoint{5.481700in}{2.242651in}}%
\pgfpathlineto{\pgfqpoint{5.482365in}{2.559737in}}%
\pgfpathlineto{\pgfqpoint{5.482271in}{2.006401in}}%
\pgfpathlineto{\pgfqpoint{5.482812in}{2.471370in}}%
\pgfpathlineto{\pgfqpoint{5.483554in}{2.023119in}}%
\pgfpathlineto{\pgfqpoint{5.483612in}{2.515045in}}%
\pgfpathlineto{\pgfqpoint{5.483921in}{2.450178in}}%
\pgfpathlineto{\pgfqpoint{5.484281in}{1.940744in}}%
\pgfpathlineto{\pgfqpoint{5.484554in}{2.538697in}}%
\pgfpathlineto{\pgfqpoint{5.485049in}{2.307087in}}%
\pgfpathlineto{\pgfqpoint{5.486010in}{2.553116in}}%
\pgfpathlineto{\pgfqpoint{5.485666in}{2.112426in}}%
\pgfpathlineto{\pgfqpoint{5.486160in}{2.385678in}}%
\pgfpathlineto{\pgfqpoint{5.486460in}{2.031717in}}%
\pgfpathlineto{\pgfqpoint{5.486260in}{2.558108in}}%
\pgfpathlineto{\pgfqpoint{5.487268in}{2.316363in}}%
\pgfpathlineto{\pgfqpoint{5.487318in}{2.536790in}}%
\pgfpathlineto{\pgfqpoint{5.487482in}{2.007782in}}%
\pgfpathlineto{\pgfqpoint{5.488380in}{2.422293in}}%
\pgfpathlineto{\pgfqpoint{5.488878in}{1.967722in}}%
\pgfpathlineto{\pgfqpoint{5.489020in}{2.540805in}}%
\pgfpathlineto{\pgfqpoint{5.489496in}{2.395806in}}%
\pgfpathlineto{\pgfqpoint{5.489766in}{2.538091in}}%
\pgfpathlineto{\pgfqpoint{5.489744in}{1.861258in}}%
\pgfpathlineto{\pgfqpoint{5.490602in}{2.473412in}}%
\pgfpathlineto{\pgfqpoint{5.491119in}{2.082058in}}%
\pgfpathlineto{\pgfqpoint{5.490623in}{2.543833in}}%
\pgfpathlineto{\pgfqpoint{5.491712in}{2.450660in}}%
\pgfpathlineto{\pgfqpoint{5.491903in}{2.552308in}}%
\pgfpathlineto{\pgfqpoint{5.492390in}{2.022485in}}%
\pgfpathlineto{\pgfqpoint{5.492813in}{2.440296in}}%
\pgfpathlineto{\pgfqpoint{5.493390in}{1.982912in}}%
\pgfpathlineto{\pgfqpoint{5.493798in}{2.523464in}}%
\pgfpathlineto{\pgfqpoint{5.493917in}{2.256063in}}%
\pgfpathlineto{\pgfqpoint{5.494304in}{2.562984in}}%
\pgfpathlineto{\pgfqpoint{5.494830in}{1.827163in}}%
\pgfpathlineto{\pgfqpoint{5.495033in}{2.486311in}}%
\pgfpathlineto{\pgfqpoint{5.495439in}{1.955423in}}%
\pgfpathlineto{\pgfqpoint{5.495418in}{2.522156in}}%
\pgfpathlineto{\pgfqpoint{5.496139in}{2.380400in}}%
\pgfpathlineto{\pgfqpoint{5.496956in}{2.540640in}}%
\pgfpathlineto{\pgfqpoint{5.496419in}{1.893595in}}%
\pgfpathlineto{\pgfqpoint{5.497249in}{2.498990in}}%
\pgfpathlineto{\pgfqpoint{5.497716in}{2.058851in}}%
\pgfpathlineto{\pgfqpoint{5.498335in}{2.542970in}}%
\pgfpathlineto{\pgfqpoint{5.498370in}{2.138735in}}%
\pgfpathlineto{\pgfqpoint{5.498377in}{2.116568in}}%
\pgfpathlineto{\pgfqpoint{5.498662in}{2.557869in}}%
\pgfpathlineto{\pgfqpoint{5.499363in}{2.426037in}}%
\pgfpathlineto{\pgfqpoint{5.499661in}{2.543604in}}%
\pgfpathlineto{\pgfqpoint{5.499419in}{2.053091in}}%
\pgfpathlineto{\pgfqpoint{5.500478in}{2.502704in}}%
\pgfpathlineto{\pgfqpoint{5.501101in}{2.538441in}}%
\pgfpathlineto{\pgfqpoint{5.500963in}{2.084067in}}%
\pgfpathlineto{\pgfqpoint{5.501550in}{2.394867in}}%
\pgfpathlineto{\pgfqpoint{5.502439in}{2.100182in}}%
\pgfpathlineto{\pgfqpoint{5.501798in}{2.548078in}}%
\pgfpathlineto{\pgfqpoint{5.502659in}{2.395112in}}%
\pgfpathlineto{\pgfqpoint{5.503216in}{2.553499in}}%
\pgfpathlineto{\pgfqpoint{5.502728in}{2.015994in}}%
\pgfpathlineto{\pgfqpoint{5.503766in}{2.365061in}}%
\pgfpathlineto{\pgfqpoint{5.504514in}{2.504468in}}%
\pgfpathlineto{\pgfqpoint{5.504322in}{1.998800in}}%
\pgfpathlineto{\pgfqpoint{5.504877in}{2.367176in}}%
\pgfpathlineto{\pgfqpoint{5.505780in}{2.107157in}}%
\pgfpathlineto{\pgfqpoint{5.505096in}{2.550108in}}%
\pgfpathlineto{\pgfqpoint{5.505991in}{2.257759in}}%
\pgfpathlineto{\pgfqpoint{5.506142in}{2.560126in}}%
\pgfpathlineto{\pgfqpoint{5.506708in}{2.048008in}}%
\pgfpathlineto{\pgfqpoint{5.507103in}{2.376563in}}%
\pgfpathlineto{\pgfqpoint{5.508124in}{1.905571in}}%
\pgfpathlineto{\pgfqpoint{5.507124in}{2.525885in}}%
\pgfpathlineto{\pgfqpoint{5.508219in}{2.321456in}}%
\pgfpathlineto{\pgfqpoint{5.508891in}{2.537000in}}%
\pgfpathlineto{\pgfqpoint{5.508592in}{1.924758in}}%
\pgfpathlineto{\pgfqpoint{5.509331in}{2.386003in}}%
\pgfpathlineto{\pgfqpoint{5.509629in}{2.027947in}}%
\pgfpathlineto{\pgfqpoint{5.510049in}{2.566220in}}%
\pgfpathlineto{\pgfqpoint{5.510441in}{2.380572in}}%
\pgfpathlineto{\pgfqpoint{5.511009in}{2.548659in}}%
\pgfpathlineto{\pgfqpoint{5.510914in}{1.922082in}}%
\pgfpathlineto{\pgfqpoint{5.511548in}{2.464061in}}%
\pgfpathlineto{\pgfqpoint{5.511757in}{2.064119in}}%
\pgfpathlineto{\pgfqpoint{5.512134in}{2.537829in}}%
\pgfpathlineto{\pgfqpoint{5.512666in}{2.298251in}}%
\pgfpathlineto{\pgfqpoint{5.512713in}{2.076247in}}%
\pgfpathlineto{\pgfqpoint{5.513344in}{2.543806in}}%
\pgfpathlineto{\pgfqpoint{5.513753in}{2.279450in}}%
\pgfpathlineto{\pgfqpoint{5.513894in}{2.533979in}}%
\pgfpathlineto{\pgfqpoint{5.514276in}{2.093278in}}%
\pgfpathlineto{\pgfqpoint{5.514872in}{2.526319in}}%
\pgfpathlineto{\pgfqpoint{5.515760in}{2.051936in}}%
\pgfpathlineto{\pgfqpoint{5.515980in}{2.402156in}}%
\pgfpathlineto{\pgfqpoint{5.516600in}{2.539452in}}%
\pgfpathlineto{\pgfqpoint{5.516667in}{2.070019in}}%
\pgfpathlineto{\pgfqpoint{5.517080in}{2.417486in}}%
\pgfpathlineto{\pgfqpoint{5.517911in}{1.904422in}}%
\pgfpathlineto{\pgfqpoint{5.517645in}{2.544862in}}%
\pgfpathlineto{\pgfqpoint{5.518189in}{2.414027in}}%
\pgfpathlineto{\pgfqpoint{5.518932in}{2.556969in}}%
\pgfpathlineto{\pgfqpoint{5.518892in}{1.989517in}}%
\pgfpathlineto{\pgfqpoint{5.519290in}{2.381776in}}%
\pgfpathlineto{\pgfqpoint{5.519852in}{1.774545in}}%
\pgfpathlineto{\pgfqpoint{5.520380in}{2.547181in}}%
\pgfpathlineto{\pgfqpoint{5.520407in}{2.075491in}}%
\pgfpathlineto{\pgfqpoint{5.520915in}{2.536058in}}%
\pgfpathlineto{\pgfqpoint{5.520512in}{1.996154in}}%
\pgfpathlineto{\pgfqpoint{5.521521in}{2.410169in}}%
\pgfpathlineto{\pgfqpoint{5.521870in}{2.551863in}}%
\pgfpathlineto{\pgfqpoint{5.521995in}{2.084404in}}%
\pgfpathlineto{\pgfqpoint{5.522599in}{2.494187in}}%
\pgfpathlineto{\pgfqpoint{5.523249in}{2.026243in}}%
\pgfpathlineto{\pgfqpoint{5.522796in}{2.542619in}}%
\pgfpathlineto{\pgfqpoint{5.523708in}{2.426836in}}%
\pgfpathlineto{\pgfqpoint{5.523819in}{1.899502in}}%
\pgfpathlineto{\pgfqpoint{5.524682in}{2.547073in}}%
\pgfpathlineto{\pgfqpoint{5.524820in}{2.363925in}}%
\pgfpathlineto{\pgfqpoint{5.525844in}{2.543193in}}%
\pgfpathlineto{\pgfqpoint{5.525564in}{2.033628in}}%
\pgfpathlineto{\pgfqpoint{5.525929in}{2.407426in}}%
\pgfpathlineto{\pgfqpoint{5.527016in}{2.079779in}}%
\pgfpathlineto{\pgfqpoint{5.526892in}{2.524496in}}%
\pgfpathlineto{\pgfqpoint{5.527035in}{2.387024in}}%
\pgfpathlineto{\pgfqpoint{5.528028in}{2.522324in}}%
\pgfpathlineto{\pgfqpoint{5.527957in}{1.910987in}}%
\pgfpathlineto{\pgfqpoint{5.528145in}{2.398098in}}%
\pgfpathlineto{\pgfqpoint{5.528255in}{2.011359in}}%
\pgfpathlineto{\pgfqpoint{5.529239in}{2.562478in}}%
\pgfpathlineto{\pgfqpoint{5.529252in}{2.411330in}}%
\pgfpathlineto{\pgfqpoint{5.530279in}{2.535271in}}%
\pgfpathlineto{\pgfqpoint{5.530234in}{1.987860in}}%
\pgfpathlineto{\pgfqpoint{5.530330in}{2.430436in}}%
\pgfpathlineto{\pgfqpoint{5.530337in}{1.771891in}}%
\pgfpathlineto{\pgfqpoint{5.531393in}{2.548934in}}%
\pgfpathlineto{\pgfqpoint{5.531438in}{2.444594in}}%
\pgfpathlineto{\pgfqpoint{5.531618in}{2.558139in}}%
\pgfpathlineto{\pgfqpoint{5.531515in}{1.989231in}}%
\pgfpathlineto{\pgfqpoint{5.532517in}{2.412781in}}%
\pgfpathlineto{\pgfqpoint{5.533421in}{2.166118in}}%
\pgfpathlineto{\pgfqpoint{5.533280in}{2.571111in}}%
\pgfpathlineto{\pgfqpoint{5.533626in}{2.408710in}}%
\pgfpathlineto{\pgfqpoint{5.534182in}{2.590534in}}%
\pgfpathlineto{\pgfqpoint{5.534521in}{2.017309in}}%
\pgfpathlineto{\pgfqpoint{5.534725in}{2.468059in}}%
\pgfpathlineto{\pgfqpoint{5.534967in}{1.908719in}}%
\pgfpathlineto{\pgfqpoint{5.534795in}{2.528657in}}%
\pgfpathlineto{\pgfqpoint{5.535834in}{2.393616in}}%
\pgfpathlineto{\pgfqpoint{5.536292in}{2.550841in}}%
\pgfpathlineto{\pgfqpoint{5.536152in}{1.989047in}}%
\pgfpathlineto{\pgfqpoint{5.536940in}{2.460708in}}%
\pgfpathlineto{\pgfqpoint{5.537321in}{2.057711in}}%
\pgfpathlineto{\pgfqpoint{5.537961in}{2.533435in}}%
\pgfpathlineto{\pgfqpoint{5.538056in}{2.179799in}}%
\pgfpathlineto{\pgfqpoint{5.538961in}{2.537839in}}%
\pgfpathlineto{\pgfqpoint{5.538202in}{2.043629in}}%
\pgfpathlineto{\pgfqpoint{5.539169in}{2.365426in}}%
\pgfpathlineto{\pgfqpoint{5.539964in}{1.872691in}}%
\pgfpathlineto{\pgfqpoint{5.539725in}{2.546205in}}%
\pgfpathlineto{\pgfqpoint{5.540286in}{2.265646in}}%
\pgfpathlineto{\pgfqpoint{5.540657in}{2.543125in}}%
\pgfpathlineto{\pgfqpoint{5.540399in}{2.041600in}}%
\pgfpathlineto{\pgfqpoint{5.541399in}{2.338770in}}%
\pgfpathlineto{\pgfqpoint{5.542234in}{2.517089in}}%
\pgfpathlineto{\pgfqpoint{5.542391in}{1.738994in}}%
\pgfpathlineto{\pgfqpoint{5.542510in}{2.453686in}}%
\pgfpathlineto{\pgfqpoint{5.542685in}{1.947110in}}%
\pgfpathlineto{\pgfqpoint{5.542691in}{2.561006in}}%
\pgfpathlineto{\pgfqpoint{5.543624in}{2.139945in}}%
\pgfpathlineto{\pgfqpoint{5.543724in}{2.562401in}}%
\pgfpathlineto{\pgfqpoint{5.544136in}{2.070697in}}%
\pgfpathlineto{\pgfqpoint{5.544741in}{2.468871in}}%
\pgfpathlineto{\pgfqpoint{5.545818in}{2.028405in}}%
\pgfpathlineto{\pgfqpoint{5.545239in}{2.542176in}}%
\pgfpathlineto{\pgfqpoint{5.545855in}{2.354406in}}%
\pgfpathlineto{\pgfqpoint{5.546892in}{2.529338in}}%
\pgfpathlineto{\pgfqpoint{5.545874in}{1.922991in}}%
\pgfpathlineto{\pgfqpoint{5.546967in}{2.419402in}}%
\pgfpathlineto{\pgfqpoint{5.547493in}{2.084982in}}%
\pgfpathlineto{\pgfqpoint{5.548050in}{2.552220in}}%
\pgfpathlineto{\pgfqpoint{5.548075in}{2.432984in}}%
\pgfpathlineto{\pgfqpoint{5.548081in}{2.546756in}}%
\pgfpathlineto{\pgfqpoint{5.549020in}{1.965042in}}%
\pgfpathlineto{\pgfqpoint{5.549175in}{2.290025in}}%
\pgfpathlineto{\pgfqpoint{5.549988in}{1.930555in}}%
\pgfpathlineto{\pgfqpoint{5.550179in}{2.559999in}}%
\pgfpathlineto{\pgfqpoint{5.550277in}{2.184181in}}%
\pgfpathlineto{\pgfqpoint{5.551193in}{2.529740in}}%
\pgfpathlineto{\pgfqpoint{5.550696in}{2.058873in}}%
\pgfpathlineto{\pgfqpoint{5.551396in}{2.458550in}}%
\pgfpathlineto{\pgfqpoint{5.551696in}{2.014973in}}%
\pgfpathlineto{\pgfqpoint{5.552174in}{2.570087in}}%
\pgfpathlineto{\pgfqpoint{5.552505in}{2.478577in}}%
\pgfpathlineto{\pgfqpoint{5.553593in}{2.114424in}}%
\pgfpathlineto{\pgfqpoint{5.553245in}{2.535356in}}%
\pgfpathlineto{\pgfqpoint{5.553617in}{2.440418in}}%
\pgfpathlineto{\pgfqpoint{5.554252in}{2.540830in}}%
\pgfpathlineto{\pgfqpoint{5.553636in}{2.067050in}}%
\pgfpathlineto{\pgfqpoint{5.554672in}{2.330450in}}%
\pgfpathlineto{\pgfqpoint{5.554812in}{2.030346in}}%
\pgfpathlineto{\pgfqpoint{5.555342in}{2.537296in}}%
\pgfpathlineto{\pgfqpoint{5.555779in}{2.432774in}}%
\pgfpathlineto{\pgfqpoint{5.556131in}{2.530315in}}%
\pgfpathlineto{\pgfqpoint{5.556077in}{2.079089in}}%
\pgfpathlineto{\pgfqpoint{5.556774in}{2.469755in}}%
\pgfpathlineto{\pgfqpoint{5.557567in}{2.094355in}}%
\pgfpathlineto{\pgfqpoint{5.557627in}{2.581682in}}%
\pgfpathlineto{\pgfqpoint{5.557881in}{2.561419in}}%
\pgfpathlineto{\pgfqpoint{5.558546in}{1.733649in}}%
\pgfpathlineto{\pgfqpoint{5.558998in}{2.431790in}}%
\pgfpathlineto{\pgfqpoint{5.560069in}{2.015785in}}%
\pgfpathlineto{\pgfqpoint{5.559648in}{2.517702in}}%
\pgfpathlineto{\pgfqpoint{5.560117in}{2.360990in}}%
\pgfpathlineto{\pgfqpoint{5.560766in}{2.542998in}}%
\pgfpathlineto{\pgfqpoint{5.560502in}{2.041354in}}%
\pgfpathlineto{\pgfqpoint{5.561228in}{2.376256in}}%
\pgfpathlineto{\pgfqpoint{5.561462in}{2.544946in}}%
\pgfpathlineto{\pgfqpoint{5.561312in}{2.071912in}}%
\pgfpathlineto{\pgfqpoint{5.562300in}{2.470432in}}%
\pgfpathlineto{\pgfqpoint{5.562521in}{1.971914in}}%
\pgfpathlineto{\pgfqpoint{5.562742in}{2.546180in}}%
\pgfpathlineto{\pgfqpoint{5.563411in}{2.462903in}}%
\pgfpathlineto{\pgfqpoint{5.563471in}{1.871709in}}%
\pgfpathlineto{\pgfqpoint{5.563954in}{2.524441in}}%
\pgfpathlineto{\pgfqpoint{5.564519in}{2.398397in}}%
\pgfpathlineto{\pgfqpoint{5.565132in}{2.539161in}}%
\pgfpathlineto{\pgfqpoint{5.565054in}{1.935743in}}%
\pgfpathlineto{\pgfqpoint{5.565624in}{2.441971in}}%
\pgfpathlineto{\pgfqpoint{5.566259in}{2.064128in}}%
\pgfpathlineto{\pgfqpoint{5.566608in}{2.561522in}}%
\pgfpathlineto{\pgfqpoint{5.566733in}{2.470738in}}%
\pgfpathlineto{\pgfqpoint{5.566952in}{1.927102in}}%
\pgfpathlineto{\pgfqpoint{5.566928in}{2.533569in}}%
\pgfpathlineto{\pgfqpoint{5.567803in}{2.320280in}}%
\pgfpathlineto{\pgfqpoint{5.568558in}{2.538389in}}%
\pgfpathlineto{\pgfqpoint{5.568534in}{1.964849in}}%
\pgfpathlineto{\pgfqpoint{5.568911in}{2.333051in}}%
\pgfpathlineto{\pgfqpoint{5.570005in}{2.534043in}}%
\pgfpathlineto{\pgfqpoint{5.569946in}{2.074832in}}%
\pgfpathlineto{\pgfqpoint{5.570011in}{2.444828in}}%
\pgfpathlineto{\pgfqpoint{5.570034in}{1.609141in}}%
\pgfpathlineto{\pgfqpoint{5.570598in}{2.589380in}}%
\pgfpathlineto{\pgfqpoint{5.571125in}{2.291740in}}%
\pgfpathlineto{\pgfqpoint{5.572190in}{2.536426in}}%
\pgfpathlineto{\pgfqpoint{5.571944in}{2.085481in}}%
\pgfpathlineto{\pgfqpoint{5.572237in}{2.486388in}}%
\pgfpathlineto{\pgfqpoint{5.572289in}{1.978945in}}%
\pgfpathlineto{\pgfqpoint{5.572896in}{2.538689in}}%
\pgfpathlineto{\pgfqpoint{5.573357in}{2.310204in}}%
\pgfpathlineto{\pgfqpoint{5.574125in}{2.518057in}}%
\pgfpathlineto{\pgfqpoint{5.573840in}{1.956329in}}%
\pgfpathlineto{\pgfqpoint{5.574468in}{2.389838in}}%
\pgfpathlineto{\pgfqpoint{5.575078in}{2.529449in}}%
\pgfpathlineto{\pgfqpoint{5.574614in}{2.163739in}}%
\pgfpathlineto{\pgfqpoint{5.575507in}{2.248041in}}%
\pgfpathlineto{\pgfqpoint{5.575948in}{2.133142in}}%
\pgfpathlineto{\pgfqpoint{5.576365in}{2.554534in}}%
\pgfpathlineto{\pgfqpoint{5.576608in}{2.392268in}}%
\pgfpathlineto{\pgfqpoint{5.577278in}{2.558956in}}%
\pgfpathlineto{\pgfqpoint{5.577503in}{2.053289in}}%
\pgfpathlineto{\pgfqpoint{5.577693in}{2.410448in}}%
\pgfpathlineto{\pgfqpoint{5.578218in}{2.054678in}}%
\pgfpathlineto{\pgfqpoint{5.578523in}{2.521377in}}%
\pgfpathlineto{\pgfqpoint{5.578805in}{2.330100in}}%
\pgfpathlineto{\pgfqpoint{5.578914in}{2.550152in}}%
\pgfpathlineto{\pgfqpoint{5.579633in}{1.938798in}}%
\pgfpathlineto{\pgfqpoint{5.579891in}{2.333446in}}%
\pgfpathlineto{\pgfqpoint{5.579897in}{2.045203in}}%
\pgfpathlineto{\pgfqpoint{5.579977in}{2.574488in}}%
\pgfpathlineto{\pgfqpoint{5.580997in}{2.446649in}}%
\pgfpathlineto{\pgfqpoint{5.581117in}{2.525478in}}%
\pgfpathlineto{\pgfqpoint{5.581403in}{2.019383in}}%
\pgfpathlineto{\pgfqpoint{5.581923in}{2.346108in}}%
\pgfpathlineto{\pgfqpoint{5.581929in}{1.944433in}}%
\pgfpathlineto{\pgfqpoint{5.581997in}{2.551326in}}%
\pgfpathlineto{\pgfqpoint{5.583029in}{2.388543in}}%
\pgfpathlineto{\pgfqpoint{5.583991in}{2.564998in}}%
\pgfpathlineto{\pgfqpoint{5.583149in}{2.010603in}}%
\pgfpathlineto{\pgfqpoint{5.584133in}{2.513736in}}%
\pgfpathlineto{\pgfqpoint{5.584797in}{1.883867in}}%
\pgfpathlineto{\pgfqpoint{5.584689in}{2.549986in}}%
\pgfpathlineto{\pgfqpoint{5.585239in}{2.484153in}}%
\pgfpathlineto{\pgfqpoint{5.585347in}{2.550746in}}%
\pgfpathlineto{\pgfqpoint{5.585664in}{2.077206in}}%
\pgfpathlineto{\pgfqpoint{5.585879in}{2.407366in}}%
\pgfpathlineto{\pgfqpoint{5.585885in}{2.107539in}}%
\pgfpathlineto{\pgfqpoint{5.585930in}{2.597485in}}%
\pgfpathlineto{\pgfqpoint{5.586986in}{2.426444in}}%
\pgfpathlineto{\pgfqpoint{5.587161in}{2.546707in}}%
\pgfpathlineto{\pgfqpoint{5.587150in}{2.169478in}}%
\pgfpathlineto{\pgfqpoint{5.587285in}{2.282452in}}%
\pgfpathlineto{\pgfqpoint{5.587291in}{2.126576in}}%
\pgfpathlineto{\pgfqpoint{5.588384in}{2.562553in}}%
\pgfpathlineto{\pgfqpoint{5.588395in}{2.304626in}}%
\pgfpathlineto{\pgfqpoint{5.589215in}{2.563102in}}%
\pgfpathlineto{\pgfqpoint{5.588423in}{1.986186in}}%
\pgfpathlineto{\pgfqpoint{5.589512in}{2.459281in}}%
\pgfpathlineto{\pgfqpoint{5.590263in}{2.017068in}}%
\pgfpathlineto{\pgfqpoint{5.589866in}{2.578402in}}%
\pgfpathlineto{\pgfqpoint{5.590622in}{2.440401in}}%
\pgfpathlineto{\pgfqpoint{5.590627in}{2.546814in}}%
\pgfpathlineto{\pgfqpoint{5.591080in}{1.897495in}}%
\pgfpathlineto{\pgfqpoint{5.591728in}{2.381876in}}%
\pgfpathlineto{\pgfqpoint{5.592057in}{2.549562in}}%
\pgfpathlineto{\pgfqpoint{5.592241in}{2.117959in}}%
\pgfpathlineto{\pgfqpoint{5.592826in}{2.397810in}}%
\pgfpathlineto{\pgfqpoint{5.593315in}{1.933896in}}%
\pgfpathlineto{\pgfqpoint{5.593293in}{2.542453in}}%
\pgfpathlineto{\pgfqpoint{5.593949in}{2.208811in}}%
\pgfpathlineto{\pgfqpoint{5.595057in}{2.545836in}}%
\pgfpathlineto{\pgfqpoint{5.594891in}{1.672470in}}%
\pgfpathlineto{\pgfqpoint{5.595063in}{2.376330in}}%
\pgfpathlineto{\pgfqpoint{5.596025in}{2.531783in}}%
\pgfpathlineto{\pgfqpoint{5.595395in}{2.090611in}}%
\pgfpathlineto{\pgfqpoint{5.596174in}{2.521486in}}%
\pgfpathlineto{\pgfqpoint{5.596196in}{2.034331in}}%
\pgfpathlineto{\pgfqpoint{5.596224in}{2.530186in}}%
\pgfpathlineto{\pgfqpoint{5.597288in}{2.312989in}}%
\pgfpathlineto{\pgfqpoint{5.597679in}{2.531420in}}%
\pgfpathlineto{\pgfqpoint{5.598130in}{2.061500in}}%
\pgfpathlineto{\pgfqpoint{5.598399in}{2.336423in}}%
\pgfpathlineto{\pgfqpoint{5.598476in}{2.551553in}}%
\pgfpathlineto{\pgfqpoint{5.598432in}{2.047450in}}%
\pgfpathlineto{\pgfqpoint{5.599507in}{2.453981in}}%
\pgfpathlineto{\pgfqpoint{5.600214in}{2.009735in}}%
\pgfpathlineto{\pgfqpoint{5.600126in}{2.552736in}}%
\pgfpathlineto{\pgfqpoint{5.600618in}{2.329519in}}%
\pgfpathlineto{\pgfqpoint{5.601001in}{2.151060in}}%
\pgfpathlineto{\pgfqpoint{5.601535in}{2.539894in}}%
\pgfpathlineto{\pgfqpoint{5.601699in}{2.169145in}}%
\pgfpathlineto{\pgfqpoint{5.602690in}{2.537494in}}%
\pgfpathlineto{\pgfqpoint{5.602303in}{2.079679in}}%
\pgfpathlineto{\pgfqpoint{5.602809in}{2.273412in}}%
\pgfpathlineto{\pgfqpoint{5.603016in}{2.538693in}}%
\pgfpathlineto{\pgfqpoint{5.603651in}{1.895376in}}%
\pgfpathlineto{\pgfqpoint{5.603922in}{2.381537in}}%
\pgfpathlineto{\pgfqpoint{5.603928in}{2.023100in}}%
\pgfpathlineto{\pgfqpoint{5.605027in}{2.554280in}}%
\pgfpathlineto{\pgfqpoint{5.605032in}{2.425220in}}%
\pgfpathlineto{\pgfqpoint{5.606118in}{1.976623in}}%
\pgfpathlineto{\pgfqpoint{5.605238in}{2.536597in}}%
\pgfpathlineto{\pgfqpoint{5.606145in}{2.278348in}}%
\pgfpathlineto{\pgfqpoint{5.606845in}{2.548859in}}%
\pgfpathlineto{\pgfqpoint{5.606840in}{2.003007in}}%
\pgfpathlineto{\pgfqpoint{5.607260in}{2.449611in}}%
\pgfpathlineto{\pgfqpoint{5.607276in}{2.018936in}}%
\pgfpathlineto{\pgfqpoint{5.608292in}{2.541433in}}%
\pgfpathlineto{\pgfqpoint{5.608372in}{2.319465in}}%
\pgfpathlineto{\pgfqpoint{5.608560in}{2.554579in}}%
\pgfpathlineto{\pgfqpoint{5.608887in}{2.036534in}}%
\pgfpathlineto{\pgfqpoint{5.609487in}{2.456033in}}%
\pgfpathlineto{\pgfqpoint{5.609519in}{1.901177in}}%
\pgfpathlineto{\pgfqpoint{5.610385in}{2.565530in}}%
\pgfpathlineto{\pgfqpoint{5.610609in}{2.425452in}}%
\pgfpathlineto{\pgfqpoint{5.611441in}{2.577128in}}%
\pgfpathlineto{\pgfqpoint{5.611382in}{2.077230in}}%
\pgfpathlineto{\pgfqpoint{5.611611in}{2.440746in}}%
\pgfpathlineto{\pgfqpoint{5.612446in}{2.049316in}}%
\pgfpathlineto{\pgfqpoint{5.612521in}{2.544024in}}%
\pgfpathlineto{\pgfqpoint{5.612723in}{2.331171in}}%
\pgfpathlineto{\pgfqpoint{5.613094in}{2.565777in}}%
\pgfpathlineto{\pgfqpoint{5.613439in}{1.991203in}}%
\pgfpathlineto{\pgfqpoint{5.613826in}{2.367625in}}%
\pgfpathlineto{\pgfqpoint{5.614852in}{1.690528in}}%
\pgfpathlineto{\pgfqpoint{5.614487in}{2.547961in}}%
\pgfpathlineto{\pgfqpoint{5.614931in}{2.312133in}}%
\pgfpathlineto{\pgfqpoint{5.615137in}{2.540095in}}%
\pgfpathlineto{\pgfqpoint{5.614989in}{1.974473in}}%
\pgfpathlineto{\pgfqpoint{5.616039in}{2.374294in}}%
\pgfpathlineto{\pgfqpoint{5.617071in}{2.035705in}}%
\pgfpathlineto{\pgfqpoint{5.616787in}{2.545516in}}%
\pgfpathlineto{\pgfqpoint{5.617139in}{2.383503in}}%
\pgfpathlineto{\pgfqpoint{5.617811in}{2.577189in}}%
\pgfpathlineto{\pgfqpoint{5.617733in}{2.057883in}}%
\pgfpathlineto{\pgfqpoint{5.618252in}{2.479975in}}%
\pgfpathlineto{\pgfqpoint{5.619032in}{2.081116in}}%
\pgfpathlineto{\pgfqpoint{5.619121in}{2.546415in}}%
\pgfpathlineto{\pgfqpoint{5.619362in}{2.460962in}}%
\pgfpathlineto{\pgfqpoint{5.619628in}{2.533551in}}%
\pgfpathlineto{\pgfqpoint{5.619487in}{1.957820in}}%
\pgfpathlineto{\pgfqpoint{5.620453in}{2.389492in}}%
\pgfpathlineto{\pgfqpoint{5.621401in}{1.677566in}}%
\pgfpathlineto{\pgfqpoint{5.621131in}{2.537032in}}%
\pgfpathlineto{\pgfqpoint{5.621568in}{1.992376in}}%
\pgfpathlineto{\pgfqpoint{5.622373in}{2.562747in}}%
\pgfpathlineto{\pgfqpoint{5.621874in}{1.983186in}}%
\pgfpathlineto{\pgfqpoint{5.622679in}{2.340511in}}%
\pgfpathlineto{\pgfqpoint{5.623229in}{2.087809in}}%
\pgfpathlineto{\pgfqpoint{5.623633in}{2.536602in}}%
\pgfpathlineto{\pgfqpoint{5.623783in}{2.436254in}}%
\pgfpathlineto{\pgfqpoint{5.623979in}{2.524880in}}%
\pgfpathlineto{\pgfqpoint{5.624233in}{1.941980in}}%
\pgfpathlineto{\pgfqpoint{5.624811in}{2.314024in}}%
\pgfpathlineto{\pgfqpoint{5.624971in}{2.077692in}}%
\pgfpathlineto{\pgfqpoint{5.625389in}{2.522227in}}%
\pgfpathlineto{\pgfqpoint{5.625914in}{2.413282in}}%
\pgfpathlineto{\pgfqpoint{5.626820in}{2.177499in}}%
\pgfpathlineto{\pgfqpoint{5.626244in}{2.538406in}}%
\pgfpathlineto{\pgfqpoint{5.627030in}{2.296705in}}%
\pgfpathlineto{\pgfqpoint{5.628071in}{2.548445in}}%
\pgfpathlineto{\pgfqpoint{5.627076in}{1.984571in}}%
\pgfpathlineto{\pgfqpoint{5.628138in}{2.454234in}}%
\pgfpathlineto{\pgfqpoint{5.628752in}{2.072348in}}%
\pgfpathlineto{\pgfqpoint{5.628506in}{2.567810in}}%
\pgfpathlineto{\pgfqpoint{5.629248in}{2.478374in}}%
\pgfpathlineto{\pgfqpoint{5.630248in}{1.992593in}}%
\pgfpathlineto{\pgfqpoint{5.629891in}{2.528269in}}%
\pgfpathlineto{\pgfqpoint{5.630365in}{2.129670in}}%
\pgfpathlineto{\pgfqpoint{5.631433in}{2.574705in}}%
\pgfpathlineto{\pgfqpoint{5.630828in}{2.020043in}}%
\pgfpathlineto{\pgfqpoint{5.631479in}{2.465502in}}%
\pgfpathlineto{\pgfqpoint{5.631936in}{2.041274in}}%
\pgfpathlineto{\pgfqpoint{5.631758in}{2.553580in}}%
\pgfpathlineto{\pgfqpoint{5.632590in}{2.323432in}}%
\pgfpathlineto{\pgfqpoint{5.633476in}{2.526966in}}%
\pgfpathlineto{\pgfqpoint{5.633203in}{1.944107in}}%
\pgfpathlineto{\pgfqpoint{5.633699in}{2.300606in}}%
\pgfpathlineto{\pgfqpoint{5.634305in}{1.867916in}}%
\pgfpathlineto{\pgfqpoint{5.633749in}{2.559650in}}%
\pgfpathlineto{\pgfqpoint{5.634784in}{2.346621in}}%
\pgfpathlineto{\pgfqpoint{5.635399in}{2.540326in}}%
\pgfpathlineto{\pgfqpoint{5.635479in}{1.948465in}}%
\pgfpathlineto{\pgfqpoint{5.635892in}{2.315669in}}%
\pgfpathlineto{\pgfqpoint{5.636871in}{2.106681in}}%
\pgfpathlineto{\pgfqpoint{5.636430in}{2.550582in}}%
\pgfpathlineto{\pgfqpoint{5.636992in}{2.399310in}}%
\pgfpathlineto{\pgfqpoint{5.637638in}{2.556485in}}%
\pgfpathlineto{\pgfqpoint{5.637859in}{2.117591in}}%
\pgfpathlineto{\pgfqpoint{5.638074in}{2.442414in}}%
\pgfpathlineto{\pgfqpoint{5.638764in}{2.034115in}}%
\pgfpathlineto{\pgfqpoint{5.639163in}{2.536485in}}%
\pgfpathlineto{\pgfqpoint{5.639188in}{2.258908in}}%
\pgfpathlineto{\pgfqpoint{5.640155in}{2.565011in}}%
\pgfpathlineto{\pgfqpoint{5.639946in}{1.957644in}}%
\pgfpathlineto{\pgfqpoint{5.640309in}{2.434042in}}%
\pgfpathlineto{\pgfqpoint{5.641402in}{2.003369in}}%
\pgfpathlineto{\pgfqpoint{5.640319in}{2.528989in}}%
\pgfpathlineto{\pgfqpoint{5.641417in}{2.425906in}}%
\pgfpathlineto{\pgfqpoint{5.641646in}{1.973607in}}%
\pgfpathlineto{\pgfqpoint{5.641899in}{2.549669in}}%
\pgfpathlineto{\pgfqpoint{5.642523in}{2.131608in}}%
\pgfpathlineto{\pgfqpoint{5.642696in}{2.538689in}}%
\pgfpathlineto{\pgfqpoint{5.643087in}{1.999340in}}%
\pgfpathlineto{\pgfqpoint{5.643635in}{2.471716in}}%
\pgfpathlineto{\pgfqpoint{5.644282in}{1.827215in}}%
\pgfpathlineto{\pgfqpoint{5.644730in}{2.564895in}}%
\pgfpathlineto{\pgfqpoint{5.644745in}{2.347141in}}%
\pgfpathlineto{\pgfqpoint{5.644883in}{2.533155in}}%
\pgfpathlineto{\pgfqpoint{5.645222in}{2.120200in}}%
\pgfpathlineto{\pgfqpoint{5.645861in}{2.482289in}}%
\pgfpathlineto{\pgfqpoint{5.646489in}{2.024946in}}%
\pgfpathlineto{\pgfqpoint{5.646509in}{2.554256in}}%
\pgfpathlineto{\pgfqpoint{5.646970in}{2.254540in}}%
\pgfpathlineto{\pgfqpoint{5.647900in}{2.531606in}}%
\pgfpathlineto{\pgfqpoint{5.647239in}{2.019836in}}%
\pgfpathlineto{\pgfqpoint{5.648080in}{2.287919in}}%
\pgfpathlineto{\pgfqpoint{5.648100in}{1.946291in}}%
\pgfpathlineto{\pgfqpoint{5.648188in}{2.554079in}}%
\pgfpathlineto{\pgfqpoint{5.649178in}{2.229820in}}%
\pgfpathlineto{\pgfqpoint{5.649943in}{2.567494in}}%
\pgfpathlineto{\pgfqpoint{5.650210in}{2.058430in}}%
\pgfpathlineto{\pgfqpoint{5.650288in}{2.283953in}}%
\pgfpathlineto{\pgfqpoint{5.651400in}{2.554074in}}%
\pgfpathlineto{\pgfqpoint{5.651123in}{2.021400in}}%
\pgfpathlineto{\pgfqpoint{5.651405in}{2.486400in}}%
\pgfpathlineto{\pgfqpoint{5.651768in}{1.895274in}}%
\pgfpathlineto{\pgfqpoint{5.651516in}{2.574651in}}%
\pgfpathlineto{\pgfqpoint{5.652513in}{2.472854in}}%
\pgfpathlineto{\pgfqpoint{5.653277in}{2.071846in}}%
\pgfpathlineto{\pgfqpoint{5.652886in}{2.590873in}}%
\pgfpathlineto{\pgfqpoint{5.653629in}{2.358774in}}%
\pgfpathlineto{\pgfqpoint{5.653740in}{2.541201in}}%
\pgfpathlineto{\pgfqpoint{5.654299in}{1.956338in}}%
\pgfpathlineto{\pgfqpoint{5.654742in}{2.478710in}}%
\pgfpathlineto{\pgfqpoint{5.655362in}{2.024090in}}%
\pgfpathlineto{\pgfqpoint{5.655611in}{2.553380in}}%
\pgfpathlineto{\pgfqpoint{5.655851in}{2.409475in}}%
\pgfpathlineto{\pgfqpoint{5.655885in}{2.592524in}}%
\pgfpathlineto{\pgfqpoint{5.656728in}{2.074091in}}%
\pgfpathlineto{\pgfqpoint{5.656944in}{2.435293in}}%
\pgfpathlineto{\pgfqpoint{5.657599in}{1.954805in}}%
\pgfpathlineto{\pgfqpoint{5.656972in}{2.517102in}}%
\pgfpathlineto{\pgfqpoint{5.658057in}{2.309331in}}%
\pgfpathlineto{\pgfqpoint{5.658896in}{1.883091in}}%
\pgfpathlineto{\pgfqpoint{5.658515in}{2.551698in}}%
\pgfpathlineto{\pgfqpoint{5.659163in}{2.278609in}}%
\pgfpathlineto{\pgfqpoint{5.659648in}{2.549745in}}%
\pgfpathlineto{\pgfqpoint{5.659215in}{2.021857in}}%
\pgfpathlineto{\pgfqpoint{5.660271in}{2.494252in}}%
\pgfpathlineto{\pgfqpoint{5.661366in}{1.885644in}}%
\pgfpathlineto{\pgfqpoint{5.660968in}{2.596311in}}%
\pgfpathlineto{\pgfqpoint{5.661385in}{2.360866in}}%
\pgfpathlineto{\pgfqpoint{5.662071in}{1.977905in}}%
\pgfpathlineto{\pgfqpoint{5.661702in}{2.555946in}}%
\pgfpathlineto{\pgfqpoint{5.662487in}{2.257717in}}%
\pgfpathlineto{\pgfqpoint{5.662704in}{2.543746in}}%
\pgfpathlineto{\pgfqpoint{5.662780in}{2.015842in}}%
\pgfpathlineto{\pgfqpoint{5.663595in}{2.240212in}}%
\pgfpathlineto{\pgfqpoint{5.664414in}{2.537609in}}%
\pgfpathlineto{\pgfqpoint{5.664047in}{1.992469in}}%
\pgfpathlineto{\pgfqpoint{5.664710in}{2.449686in}}%
\pgfpathlineto{\pgfqpoint{5.665471in}{2.065505in}}%
\pgfpathlineto{\pgfqpoint{5.665560in}{2.543277in}}%
\pgfpathlineto{\pgfqpoint{5.665822in}{2.425531in}}%
\pgfpathlineto{\pgfqpoint{5.666324in}{2.523360in}}%
\pgfpathlineto{\pgfqpoint{5.665851in}{1.929460in}}%
\pgfpathlineto{\pgfqpoint{5.666918in}{2.399595in}}%
\pgfpathlineto{\pgfqpoint{5.667375in}{1.979076in}}%
\pgfpathlineto{\pgfqpoint{5.667786in}{2.544501in}}%
\pgfpathlineto{\pgfqpoint{5.668024in}{2.463330in}}%
\pgfpathlineto{\pgfqpoint{5.668709in}{1.933062in}}%
\pgfpathlineto{\pgfqpoint{5.668876in}{2.534286in}}%
\pgfpathlineto{\pgfqpoint{5.669141in}{2.257036in}}%
\pgfpathlineto{\pgfqpoint{5.670135in}{2.553871in}}%
\pgfpathlineto{\pgfqpoint{5.669898in}{1.992673in}}%
\pgfpathlineto{\pgfqpoint{5.670251in}{2.404644in}}%
\pgfpathlineto{\pgfqpoint{5.670372in}{2.057371in}}%
\pgfpathlineto{\pgfqpoint{5.670437in}{2.548688in}}%
\pgfpathlineto{\pgfqpoint{5.671358in}{2.334191in}}%
\pgfpathlineto{\pgfqpoint{5.671585in}{2.573691in}}%
\pgfpathlineto{\pgfqpoint{5.672397in}{1.923324in}}%
\pgfpathlineto{\pgfqpoint{5.672476in}{2.449741in}}%
\pgfpathlineto{\pgfqpoint{5.673494in}{2.552911in}}%
\pgfpathlineto{\pgfqpoint{5.673591in}{2.062572in}}%
\pgfpathlineto{\pgfqpoint{5.673963in}{2.534113in}}%
\pgfpathlineto{\pgfqpoint{5.674336in}{1.906628in}}%
\pgfpathlineto{\pgfqpoint{5.674703in}{2.464142in}}%
\pgfpathlineto{\pgfqpoint{5.675597in}{2.112213in}}%
\pgfpathlineto{\pgfqpoint{5.675345in}{2.562154in}}%
\pgfpathlineto{\pgfqpoint{5.675816in}{2.371739in}}%
\pgfpathlineto{\pgfqpoint{5.676717in}{2.577710in}}%
\pgfpathlineto{\pgfqpoint{5.676640in}{1.973915in}}%
\pgfpathlineto{\pgfqpoint{5.676909in}{2.383247in}}%
\pgfpathlineto{\pgfqpoint{5.677748in}{1.956360in}}%
\pgfpathlineto{\pgfqpoint{5.678008in}{2.549156in}}%
\pgfpathlineto{\pgfqpoint{5.678021in}{2.278817in}}%
\pgfpathlineto{\pgfqpoint{5.678745in}{2.557222in}}%
\pgfpathlineto{\pgfqpoint{5.678822in}{2.003959in}}%
\pgfpathlineto{\pgfqpoint{5.679131in}{2.209788in}}%
\pgfpathlineto{\pgfqpoint{5.680224in}{2.563585in}}%
\pgfpathlineto{\pgfqpoint{5.679281in}{2.043024in}}%
\pgfpathlineto{\pgfqpoint{5.680247in}{2.421523in}}%
\pgfpathlineto{\pgfqpoint{5.680939in}{2.559860in}}%
\pgfpathlineto{\pgfqpoint{5.681093in}{2.070587in}}%
\pgfpathlineto{\pgfqpoint{5.681323in}{2.427258in}}%
\pgfpathlineto{\pgfqpoint{5.682100in}{1.821249in}}%
\pgfpathlineto{\pgfqpoint{5.681418in}{2.541815in}}%
\pgfpathlineto{\pgfqpoint{5.682433in}{2.095686in}}%
\pgfpathlineto{\pgfqpoint{5.683401in}{1.980263in}}%
\pgfpathlineto{\pgfqpoint{5.683549in}{2.537143in}}%
\pgfpathlineto{\pgfqpoint{5.684210in}{2.034988in}}%
\pgfpathlineto{\pgfqpoint{5.684663in}{2.364360in}}%
\pgfpathlineto{\pgfqpoint{5.685146in}{2.530078in}}%
\pgfpathlineto{\pgfqpoint{5.685142in}{2.039804in}}%
\pgfpathlineto{\pgfqpoint{5.685768in}{2.494617in}}%
\pgfpathlineto{\pgfqpoint{5.686398in}{2.079200in}}%
\pgfpathlineto{\pgfqpoint{5.686224in}{2.548342in}}%
\pgfpathlineto{\pgfqpoint{5.686880in}{2.443010in}}%
\pgfpathlineto{\pgfqpoint{5.687842in}{1.947026in}}%
\pgfpathlineto{\pgfqpoint{5.687638in}{2.535745in}}%
\pgfpathlineto{\pgfqpoint{5.687994in}{2.174366in}}%
\pgfpathlineto{\pgfqpoint{5.689051in}{2.535472in}}%
\pgfpathlineto{\pgfqpoint{5.689055in}{2.110470in}}%
\pgfpathlineto{\pgfqpoint{5.689108in}{2.451043in}}%
\pgfpathlineto{\pgfqpoint{5.689654in}{2.535914in}}%
\pgfpathlineto{\pgfqpoint{5.689437in}{1.963995in}}%
\pgfpathlineto{\pgfqpoint{5.690066in}{2.288560in}}%
\pgfpathlineto{\pgfqpoint{5.690070in}{2.014308in}}%
\pgfpathlineto{\pgfqpoint{5.690158in}{2.576321in}}%
\pgfpathlineto{\pgfqpoint{5.691171in}{2.406912in}}%
\pgfpathlineto{\pgfqpoint{5.691506in}{2.115279in}}%
\pgfpathlineto{\pgfqpoint{5.692277in}{2.548676in}}%
\pgfpathlineto{\pgfqpoint{5.693144in}{1.891141in}}%
\pgfpathlineto{\pgfqpoint{5.693390in}{2.415280in}}%
\pgfpathlineto{\pgfqpoint{5.693535in}{2.549803in}}%
\pgfpathlineto{\pgfqpoint{5.693622in}{2.003543in}}%
\pgfpathlineto{\pgfqpoint{5.694499in}{2.400054in}}%
\pgfpathlineto{\pgfqpoint{5.694749in}{1.988284in}}%
\pgfpathlineto{\pgfqpoint{5.695038in}{2.538605in}}%
\pgfpathlineto{\pgfqpoint{5.695466in}{2.413930in}}%
\pgfpathlineto{\pgfqpoint{5.695842in}{2.532634in}}%
\pgfpathlineto{\pgfqpoint{5.696344in}{1.981730in}}%
\pgfpathlineto{\pgfqpoint{5.696566in}{2.295232in}}%
\pgfpathlineto{\pgfqpoint{5.696945in}{1.713511in}}%
\pgfpathlineto{\pgfqpoint{5.697080in}{2.545573in}}%
\pgfpathlineto{\pgfqpoint{5.697667in}{2.489683in}}%
\pgfpathlineto{\pgfqpoint{5.697698in}{2.551577in}}%
\pgfpathlineto{\pgfqpoint{5.697954in}{2.085434in}}%
\pgfpathlineto{\pgfqpoint{5.698761in}{2.449533in}}%
\pgfpathlineto{\pgfqpoint{5.699151in}{1.918203in}}%
\pgfpathlineto{\pgfqpoint{5.698887in}{2.558552in}}%
\pgfpathlineto{\pgfqpoint{5.699870in}{2.441937in}}%
\pgfpathlineto{\pgfqpoint{5.700458in}{2.581598in}}%
\pgfpathlineto{\pgfqpoint{5.700324in}{2.117334in}}%
\pgfpathlineto{\pgfqpoint{5.700812in}{2.386975in}}%
\pgfpathlineto{\pgfqpoint{5.701390in}{1.917575in}}%
\pgfpathlineto{\pgfqpoint{5.701833in}{2.525920in}}%
\pgfpathlineto{\pgfqpoint{5.701919in}{2.346044in}}%
\pgfpathlineto{\pgfqpoint{5.702792in}{2.549077in}}%
\pgfpathlineto{\pgfqpoint{5.702590in}{2.052389in}}%
\pgfpathlineto{\pgfqpoint{5.703024in}{2.516561in}}%
\pgfpathlineto{\pgfqpoint{5.703693in}{2.039825in}}%
\pgfpathlineto{\pgfqpoint{5.703393in}{2.556516in}}%
\pgfpathlineto{\pgfqpoint{5.704134in}{2.467274in}}%
\pgfpathlineto{\pgfqpoint{5.704631in}{2.564196in}}%
\pgfpathlineto{\pgfqpoint{5.704271in}{1.839909in}}%
\pgfpathlineto{\pgfqpoint{5.705208in}{2.396825in}}%
\pgfpathlineto{\pgfqpoint{5.705225in}{2.047124in}}%
\pgfpathlineto{\pgfqpoint{5.705716in}{2.547899in}}%
\pgfpathlineto{\pgfqpoint{5.706317in}{2.481940in}}%
\pgfpathlineto{\pgfqpoint{5.706857in}{1.999663in}}%
\pgfpathlineto{\pgfqpoint{5.707117in}{2.576429in}}%
\pgfpathlineto{\pgfqpoint{5.707431in}{2.307734in}}%
\pgfpathlineto{\pgfqpoint{5.708110in}{2.551846in}}%
\pgfpathlineto{\pgfqpoint{5.707461in}{1.994397in}}%
\pgfpathlineto{\pgfqpoint{5.708539in}{2.257970in}}%
\pgfpathlineto{\pgfqpoint{5.708543in}{1.971276in}}%
\pgfpathlineto{\pgfqpoint{5.708572in}{2.562763in}}%
\pgfpathlineto{\pgfqpoint{5.709647in}{2.395483in}}%
\pgfpathlineto{\pgfqpoint{5.710070in}{1.970890in}}%
\pgfpathlineto{\pgfqpoint{5.709656in}{2.587584in}}%
\pgfpathlineto{\pgfqpoint{5.710753in}{2.315746in}}%
\pgfpathlineto{\pgfqpoint{5.711170in}{2.547116in}}%
\pgfpathlineto{\pgfqpoint{5.711650in}{2.030086in}}%
\pgfpathlineto{\pgfqpoint{5.711865in}{2.420904in}}%
\pgfpathlineto{\pgfqpoint{5.712973in}{2.536698in}}%
\pgfpathlineto{\pgfqpoint{5.712264in}{2.066725in}}%
\pgfpathlineto{\pgfqpoint{5.712977in}{2.433801in}}%
\pgfpathlineto{\pgfqpoint{5.713614in}{2.022337in}}%
\pgfpathlineto{\pgfqpoint{5.713685in}{2.552541in}}%
\pgfpathlineto{\pgfqpoint{5.714087in}{2.435838in}}%
\pgfpathlineto{\pgfqpoint{5.714827in}{2.579015in}}%
\pgfpathlineto{\pgfqpoint{5.714284in}{2.022307in}}%
\pgfpathlineto{\pgfqpoint{5.715190in}{2.359225in}}%
\pgfpathlineto{\pgfqpoint{5.716073in}{1.875818in}}%
\pgfpathlineto{\pgfqpoint{5.716173in}{2.556961in}}%
\pgfpathlineto{\pgfqpoint{5.716302in}{2.230857in}}%
\pgfpathlineto{\pgfqpoint{5.716930in}{2.539442in}}%
\pgfpathlineto{\pgfqpoint{5.716726in}{1.939598in}}%
\pgfpathlineto{\pgfqpoint{5.717412in}{2.338613in}}%
\pgfpathlineto{\pgfqpoint{5.717818in}{2.135055in}}%
\pgfpathlineto{\pgfqpoint{5.717706in}{2.564190in}}%
\pgfpathlineto{\pgfqpoint{5.718435in}{2.323353in}}%
\pgfpathlineto{\pgfqpoint{5.719403in}{2.549802in}}%
\pgfpathlineto{\pgfqpoint{5.719262in}{2.023342in}}%
\pgfpathlineto{\pgfqpoint{5.719543in}{2.380102in}}%
\pgfpathlineto{\pgfqpoint{5.719944in}{1.564940in}}%
\pgfpathlineto{\pgfqpoint{5.720401in}{2.554994in}}%
\pgfpathlineto{\pgfqpoint{5.720652in}{2.393941in}}%
\pgfpathlineto{\pgfqpoint{5.721525in}{2.554439in}}%
\pgfpathlineto{\pgfqpoint{5.721451in}{2.010543in}}%
\pgfpathlineto{\pgfqpoint{5.721755in}{2.458000in}}%
\pgfpathlineto{\pgfqpoint{5.721837in}{2.012628in}}%
\pgfpathlineto{\pgfqpoint{5.722842in}{2.541160in}}%
\pgfpathlineto{\pgfqpoint{5.722862in}{2.375900in}}%
\pgfpathlineto{\pgfqpoint{5.723169in}{2.536889in}}%
\pgfpathlineto{\pgfqpoint{5.723554in}{1.883278in}}%
\pgfpathlineto{\pgfqpoint{5.723967in}{2.279219in}}%
\pgfpathlineto{\pgfqpoint{5.723979in}{2.357787in}}%
\pgfpathlineto{\pgfqpoint{5.724232in}{2.554158in}}%
\pgfpathlineto{\pgfqpoint{5.724008in}{1.855523in}}%
\pgfpathlineto{\pgfqpoint{5.725089in}{2.382181in}}%
\pgfpathlineto{\pgfqpoint{5.725952in}{1.926888in}}%
\pgfpathlineto{\pgfqpoint{5.725948in}{2.547681in}}%
\pgfpathlineto{\pgfqpoint{5.726200in}{2.316554in}}%
\pgfpathlineto{\pgfqpoint{5.727106in}{1.913074in}}%
\pgfpathlineto{\pgfqpoint{5.726757in}{2.527648in}}%
\pgfpathlineto{\pgfqpoint{5.727260in}{2.493267in}}%
\pgfpathlineto{\pgfqpoint{5.727677in}{2.564544in}}%
\pgfpathlineto{\pgfqpoint{5.727621in}{1.943791in}}%
\pgfpathlineto{\pgfqpoint{5.728349in}{2.425793in}}%
\pgfpathlineto{\pgfqpoint{5.729444in}{1.943205in}}%
\pgfpathlineto{\pgfqpoint{5.728685in}{2.541796in}}%
\pgfpathlineto{\pgfqpoint{5.729456in}{2.304352in}}%
\pgfpathlineto{\pgfqpoint{5.730531in}{2.547018in}}%
\pgfpathlineto{\pgfqpoint{5.730052in}{1.933469in}}%
\pgfpathlineto{\pgfqpoint{5.730568in}{2.458414in}}%
\pgfpathlineto{\pgfqpoint{5.730785in}{2.098677in}}%
\pgfpathlineto{\pgfqpoint{5.731428in}{2.546149in}}%
\pgfpathlineto{\pgfqpoint{5.731672in}{2.324008in}}%
\pgfpathlineto{\pgfqpoint{5.732774in}{2.562293in}}%
\pgfpathlineto{\pgfqpoint{5.732174in}{1.874659in}}%
\pgfpathlineto{\pgfqpoint{5.732786in}{2.464802in}}%
\pgfpathlineto{\pgfqpoint{5.733798in}{2.562505in}}%
\pgfpathlineto{\pgfqpoint{5.733606in}{2.057090in}}%
\pgfpathlineto{\pgfqpoint{5.733874in}{2.546090in}}%
\pgfpathlineto{\pgfqpoint{5.733898in}{2.058530in}}%
\pgfpathlineto{\pgfqpoint{5.734305in}{2.565953in}}%
\pgfpathlineto{\pgfqpoint{5.734986in}{2.437450in}}%
\pgfpathlineto{\pgfqpoint{5.735992in}{2.104139in}}%
\pgfpathlineto{\pgfqpoint{5.735304in}{2.547610in}}%
\pgfpathlineto{\pgfqpoint{5.736095in}{2.447581in}}%
\pgfpathlineto{\pgfqpoint{5.736766in}{1.888841in}}%
\pgfpathlineto{\pgfqpoint{5.736770in}{2.555720in}}%
\pgfpathlineto{\pgfqpoint{5.737206in}{2.429448in}}%
\pgfpathlineto{\pgfqpoint{5.737803in}{2.555263in}}%
\pgfpathlineto{\pgfqpoint{5.738195in}{1.923955in}}%
\pgfpathlineto{\pgfqpoint{5.738301in}{2.371801in}}%
\pgfpathlineto{\pgfqpoint{5.739091in}{1.994300in}}%
\pgfpathlineto{\pgfqpoint{5.738428in}{2.547530in}}%
\pgfpathlineto{\pgfqpoint{5.739414in}{2.220751in}}%
\pgfpathlineto{\pgfqpoint{5.739611in}{2.529252in}}%
\pgfpathlineto{\pgfqpoint{5.739481in}{2.075313in}}%
\pgfpathlineto{\pgfqpoint{5.740524in}{2.331556in}}%
\pgfpathlineto{\pgfqpoint{5.741034in}{2.067220in}}%
\pgfpathlineto{\pgfqpoint{5.741446in}{2.548380in}}%
\pgfpathlineto{\pgfqpoint{5.741626in}{2.166160in}}%
\pgfpathlineto{\pgfqpoint{5.742691in}{2.540478in}}%
\pgfpathlineto{\pgfqpoint{5.742535in}{2.055898in}}%
\pgfpathlineto{\pgfqpoint{5.742738in}{2.424190in}}%
\pgfpathlineto{\pgfqpoint{5.743816in}{2.038520in}}%
\pgfpathlineto{\pgfqpoint{5.743035in}{2.540947in}}%
\pgfpathlineto{\pgfqpoint{5.743847in}{2.419443in}}%
\pgfpathlineto{\pgfqpoint{5.743956in}{2.556867in}}%
\pgfpathlineto{\pgfqpoint{5.744482in}{2.033185in}}%
\pgfpathlineto{\pgfqpoint{5.744933in}{2.458663in}}%
\pgfpathlineto{\pgfqpoint{5.745081in}{1.849756in}}%
\pgfpathlineto{\pgfqpoint{5.745928in}{2.551984in}}%
\pgfpathlineto{\pgfqpoint{5.746040in}{2.381552in}}%
\pgfpathlineto{\pgfqpoint{5.746405in}{2.569366in}}%
\pgfpathlineto{\pgfqpoint{5.746079in}{1.961308in}}%
\pgfpathlineto{\pgfqpoint{5.747148in}{2.436773in}}%
\pgfpathlineto{\pgfqpoint{5.747415in}{1.978649in}}%
\pgfpathlineto{\pgfqpoint{5.748014in}{2.561831in}}%
\pgfpathlineto{\pgfqpoint{5.748257in}{2.315273in}}%
\pgfpathlineto{\pgfqpoint{5.748319in}{2.547322in}}%
\pgfpathlineto{\pgfqpoint{5.748554in}{1.925724in}}%
\pgfpathlineto{\pgfqpoint{5.749371in}{2.469087in}}%
\pgfpathlineto{\pgfqpoint{5.749832in}{1.999399in}}%
\pgfpathlineto{\pgfqpoint{5.749544in}{2.555053in}}%
\pgfpathlineto{\pgfqpoint{5.750486in}{2.352029in}}%
\pgfpathlineto{\pgfqpoint{5.750954in}{2.545322in}}%
\pgfpathlineto{\pgfqpoint{5.750593in}{2.012322in}}%
\pgfpathlineto{\pgfqpoint{5.751597in}{2.543098in}}%
\pgfpathlineto{\pgfqpoint{5.751777in}{2.041172in}}%
\pgfpathlineto{\pgfqpoint{5.752053in}{2.545668in}}%
\pgfpathlineto{\pgfqpoint{5.752714in}{2.165711in}}%
\pgfpathlineto{\pgfqpoint{5.753663in}{2.555004in}}%
\pgfpathlineto{\pgfqpoint{5.753694in}{1.723558in}}%
\pgfpathlineto{\pgfqpoint{5.753827in}{2.310429in}}%
\pgfpathlineto{\pgfqpoint{5.754600in}{2.563921in}}%
\pgfpathlineto{\pgfqpoint{5.754858in}{1.994436in}}%
\pgfpathlineto{\pgfqpoint{5.754934in}{2.199585in}}%
\pgfpathlineto{\pgfqpoint{5.755238in}{2.559101in}}%
\pgfpathlineto{\pgfqpoint{5.755879in}{2.052168in}}%
\pgfpathlineto{\pgfqpoint{5.756049in}{2.394862in}}%
\pgfpathlineto{\pgfqpoint{5.756174in}{1.947408in}}%
\pgfpathlineto{\pgfqpoint{5.756511in}{2.572175in}}%
\pgfpathlineto{\pgfqpoint{5.757158in}{2.473002in}}%
\pgfpathlineto{\pgfqpoint{5.758154in}{1.877989in}}%
\pgfpathlineto{\pgfqpoint{5.757996in}{2.560868in}}%
\pgfpathlineto{\pgfqpoint{5.758283in}{2.271358in}}%
\pgfpathlineto{\pgfqpoint{5.758373in}{2.539688in}}%
\pgfpathlineto{\pgfqpoint{5.759092in}{1.841137in}}%
\pgfpathlineto{\pgfqpoint{5.759393in}{2.394096in}}%
\pgfpathlineto{\pgfqpoint{5.760290in}{1.993116in}}%
\pgfpathlineto{\pgfqpoint{5.760343in}{2.563094in}}%
\pgfpathlineto{\pgfqpoint{5.760485in}{2.455505in}}%
\pgfpathlineto{\pgfqpoint{5.761032in}{2.535155in}}%
\pgfpathlineto{\pgfqpoint{5.760961in}{2.057362in}}%
\pgfpathlineto{\pgfqpoint{5.761560in}{2.525254in}}%
\pgfpathlineto{\pgfqpoint{5.761564in}{1.972931in}}%
\pgfpathlineto{\pgfqpoint{5.762139in}{2.536757in}}%
\pgfpathlineto{\pgfqpoint{5.762669in}{2.462027in}}%
\pgfpathlineto{\pgfqpoint{5.763154in}{2.125342in}}%
\pgfpathlineto{\pgfqpoint{5.763396in}{2.527672in}}%
\pgfpathlineto{\pgfqpoint{5.763783in}{2.327480in}}%
\pgfpathlineto{\pgfqpoint{5.764478in}{2.534775in}}%
\pgfpathlineto{\pgfqpoint{5.764225in}{1.865044in}}%
\pgfpathlineto{\pgfqpoint{5.764879in}{2.394449in}}%
\pgfpathlineto{\pgfqpoint{5.765220in}{2.018100in}}%
\pgfpathlineto{\pgfqpoint{5.765657in}{2.555607in}}%
\pgfpathlineto{\pgfqpoint{5.765987in}{2.336453in}}%
\pgfpathlineto{\pgfqpoint{5.766752in}{2.546635in}}%
\pgfpathlineto{\pgfqpoint{5.766102in}{1.958161in}}%
\pgfpathlineto{\pgfqpoint{5.767092in}{2.397273in}}%
\pgfpathlineto{\pgfqpoint{5.767096in}{1.982938in}}%
\pgfpathlineto{\pgfqpoint{5.767454in}{2.569887in}}%
\pgfpathlineto{\pgfqpoint{5.768202in}{2.400327in}}%
\pgfpathlineto{\pgfqpoint{5.768320in}{1.772926in}}%
\pgfpathlineto{\pgfqpoint{5.768959in}{2.545414in}}%
\pgfpathlineto{\pgfqpoint{5.769312in}{2.401216in}}%
\pgfpathlineto{\pgfqpoint{5.770207in}{2.607237in}}%
\pgfpathlineto{\pgfqpoint{5.770218in}{1.777243in}}%
\pgfpathlineto{\pgfqpoint{5.770420in}{2.463914in}}%
\pgfpathlineto{\pgfqpoint{5.770570in}{2.009960in}}%
\pgfpathlineto{\pgfqpoint{5.770544in}{2.587269in}}%
\pgfpathlineto{\pgfqpoint{5.771528in}{2.168810in}}%
\pgfpathlineto{\pgfqpoint{5.771645in}{2.544146in}}%
\pgfpathlineto{\pgfqpoint{5.771747in}{2.005034in}}%
\pgfpathlineto{\pgfqpoint{5.772641in}{2.418489in}}%
\pgfpathlineto{\pgfqpoint{5.772757in}{2.551291in}}%
\pgfpathlineto{\pgfqpoint{5.773376in}{2.072289in}}%
\pgfpathlineto{\pgfqpoint{5.773743in}{2.392202in}}%
\pgfpathlineto{\pgfqpoint{5.774317in}{1.961271in}}%
\pgfpathlineto{\pgfqpoint{5.773761in}{2.550863in}}%
\pgfpathlineto{\pgfqpoint{5.774850in}{2.405078in}}%
\pgfpathlineto{\pgfqpoint{5.775379in}{2.532999in}}%
\pgfpathlineto{\pgfqpoint{5.775813in}{1.932265in}}%
\pgfpathlineto{\pgfqpoint{5.775961in}{2.512876in}}%
\pgfpathlineto{\pgfqpoint{5.776463in}{1.930812in}}%
\pgfpathlineto{\pgfqpoint{5.776344in}{2.563148in}}%
\pgfpathlineto{\pgfqpoint{5.777073in}{2.279473in}}%
\pgfpathlineto{\pgfqpoint{5.777761in}{2.541410in}}%
\pgfpathlineto{\pgfqpoint{5.777426in}{1.980089in}}%
\pgfpathlineto{\pgfqpoint{5.778186in}{2.463840in}}%
\pgfpathlineto{\pgfqpoint{5.778340in}{1.997337in}}%
\pgfpathlineto{\pgfqpoint{5.778419in}{2.541404in}}%
\pgfpathlineto{\pgfqpoint{5.779299in}{2.228005in}}%
\pgfpathlineto{\pgfqpoint{5.779313in}{2.561763in}}%
\pgfpathlineto{\pgfqpoint{5.779847in}{1.961683in}}%
\pgfpathlineto{\pgfqpoint{5.780413in}{2.462823in}}%
\pgfpathlineto{\pgfqpoint{5.780953in}{2.047987in}}%
\pgfpathlineto{\pgfqpoint{5.780874in}{2.563830in}}%
\pgfpathlineto{\pgfqpoint{5.781531in}{2.218208in}}%
\pgfpathlineto{\pgfqpoint{5.781909in}{2.566842in}}%
\pgfpathlineto{\pgfqpoint{5.781838in}{1.970319in}}%
\pgfpathlineto{\pgfqpoint{5.782643in}{2.462656in}}%
\pgfpathlineto{\pgfqpoint{5.783045in}{1.936519in}}%
\pgfpathlineto{\pgfqpoint{5.783286in}{2.574354in}}%
\pgfpathlineto{\pgfqpoint{5.783748in}{2.368858in}}%
\pgfpathlineto{\pgfqpoint{5.784617in}{2.550248in}}%
\pgfpathlineto{\pgfqpoint{5.783975in}{1.912391in}}%
\pgfpathlineto{\pgfqpoint{5.784858in}{2.348789in}}%
\pgfpathlineto{\pgfqpoint{5.785731in}{2.549418in}}%
\pgfpathlineto{\pgfqpoint{5.785565in}{1.866582in}}%
\pgfpathlineto{\pgfqpoint{5.785968in}{2.389532in}}%
\pgfpathlineto{\pgfqpoint{5.786871in}{1.986940in}}%
\pgfpathlineto{\pgfqpoint{5.786176in}{2.568414in}}%
\pgfpathlineto{\pgfqpoint{5.787064in}{2.414526in}}%
\pgfpathlineto{\pgfqpoint{5.787283in}{2.580132in}}%
\pgfpathlineto{\pgfqpoint{5.787497in}{1.863221in}}%
\pgfpathlineto{\pgfqpoint{5.788169in}{2.474400in}}%
\pgfpathlineto{\pgfqpoint{5.788618in}{1.920296in}}%
\pgfpathlineto{\pgfqpoint{5.789063in}{2.548308in}}%
\pgfpathlineto{\pgfqpoint{5.789281in}{2.331477in}}%
\pgfpathlineto{\pgfqpoint{5.789932in}{2.573779in}}%
\pgfpathlineto{\pgfqpoint{5.790030in}{1.944148in}}%
\pgfpathlineto{\pgfqpoint{5.790390in}{2.381398in}}%
\pgfpathlineto{\pgfqpoint{5.791084in}{2.566609in}}%
\pgfpathlineto{\pgfqpoint{5.790994in}{1.976934in}}%
\pgfpathlineto{\pgfqpoint{5.791419in}{2.264373in}}%
\pgfpathlineto{\pgfqpoint{5.791503in}{1.585817in}}%
\pgfpathlineto{\pgfqpoint{5.791983in}{2.576293in}}%
\pgfpathlineto{\pgfqpoint{5.792495in}{2.258307in}}%
\pgfpathlineto{\pgfqpoint{5.793519in}{2.558456in}}%
\pgfpathlineto{\pgfqpoint{5.793391in}{2.021508in}}%
\pgfpathlineto{\pgfqpoint{5.793606in}{2.423404in}}%
\pgfpathlineto{\pgfqpoint{5.793883in}{2.027295in}}%
\pgfpathlineto{\pgfqpoint{5.794354in}{2.578684in}}%
\pgfpathlineto{\pgfqpoint{5.794718in}{2.355680in}}%
\pgfpathlineto{\pgfqpoint{5.795785in}{2.061115in}}%
\pgfpathlineto{\pgfqpoint{5.795184in}{2.560833in}}%
\pgfpathlineto{\pgfqpoint{5.795823in}{2.412627in}}%
\pgfpathlineto{\pgfqpoint{5.796726in}{2.539974in}}%
\pgfpathlineto{\pgfqpoint{5.796020in}{1.960832in}}%
\pgfpathlineto{\pgfqpoint{5.796898in}{2.191464in}}%
\pgfpathlineto{\pgfqpoint{5.797104in}{1.868000in}}%
\pgfpathlineto{\pgfqpoint{5.797486in}{2.557746in}}%
\pgfpathlineto{\pgfqpoint{5.798001in}{2.410245in}}%
\pgfpathlineto{\pgfqpoint{5.798776in}{2.537506in}}%
\pgfpathlineto{\pgfqpoint{5.798639in}{2.084809in}}%
\pgfpathlineto{\pgfqpoint{5.799040in}{2.480157in}}%
\pgfpathlineto{\pgfqpoint{5.799043in}{1.947154in}}%
\pgfpathlineto{\pgfqpoint{5.799686in}{2.579741in}}%
\pgfpathlineto{\pgfqpoint{5.800151in}{2.444534in}}%
\pgfpathlineto{\pgfqpoint{5.801096in}{2.005275in}}%
\pgfpathlineto{\pgfqpoint{5.801225in}{2.568352in}}%
\pgfpathlineto{\pgfqpoint{5.801262in}{2.338248in}}%
\pgfpathlineto{\pgfqpoint{5.801514in}{2.537140in}}%
\pgfpathlineto{\pgfqpoint{5.801324in}{1.907311in}}%
\pgfpathlineto{\pgfqpoint{5.802371in}{2.335679in}}%
\pgfpathlineto{\pgfqpoint{5.802969in}{1.963844in}}%
\pgfpathlineto{\pgfqpoint{5.803132in}{2.567018in}}%
\pgfpathlineto{\pgfqpoint{5.803467in}{2.347145in}}%
\pgfpathlineto{\pgfqpoint{5.803900in}{2.553164in}}%
\pgfpathlineto{\pgfqpoint{5.804438in}{2.095270in}}%
\pgfpathlineto{\pgfqpoint{5.804573in}{2.416940in}}%
\pgfpathlineto{\pgfqpoint{5.805390in}{2.009880in}}%
\pgfpathlineto{\pgfqpoint{5.804584in}{2.557177in}}%
\pgfpathlineto{\pgfqpoint{5.805684in}{2.431030in}}%
\pgfpathlineto{\pgfqpoint{5.806448in}{1.870321in}}%
\pgfpathlineto{\pgfqpoint{5.806586in}{2.550704in}}%
\pgfpathlineto{\pgfqpoint{5.806795in}{2.356662in}}%
\pgfpathlineto{\pgfqpoint{5.807235in}{1.911937in}}%
\pgfpathlineto{\pgfqpoint{5.807090in}{2.548766in}}%
\pgfpathlineto{\pgfqpoint{5.807902in}{2.337531in}}%
\pgfpathlineto{\pgfqpoint{5.808693in}{2.549294in}}%
\pgfpathlineto{\pgfqpoint{5.808382in}{1.958441in}}%
\pgfpathlineto{\pgfqpoint{5.809007in}{2.430288in}}%
\pgfpathlineto{\pgfqpoint{5.809021in}{1.887673in}}%
\pgfpathlineto{\pgfqpoint{5.809849in}{2.538574in}}%
\pgfpathlineto{\pgfqpoint{5.810116in}{2.520104in}}%
\pgfpathlineto{\pgfqpoint{5.810263in}{1.971495in}}%
\pgfpathlineto{\pgfqpoint{5.810823in}{2.529210in}}%
\pgfpathlineto{\pgfqpoint{5.811232in}{2.321867in}}%
\pgfpathlineto{\pgfqpoint{5.811561in}{2.559524in}}%
\pgfpathlineto{\pgfqpoint{5.811568in}{1.982840in}}%
\pgfpathlineto{\pgfqpoint{5.812345in}{2.402535in}}%
\pgfpathlineto{\pgfqpoint{5.812905in}{1.974022in}}%
\pgfpathlineto{\pgfqpoint{5.813137in}{2.538523in}}%
\pgfpathlineto{\pgfqpoint{5.813445in}{2.357954in}}%
\pgfpathlineto{\pgfqpoint{5.814513in}{2.560601in}}%
\pgfpathlineto{\pgfqpoint{5.813905in}{1.962600in}}%
\pgfpathlineto{\pgfqpoint{5.814556in}{2.409656in}}%
\pgfpathlineto{\pgfqpoint{5.815525in}{1.887842in}}%
\pgfpathlineto{\pgfqpoint{5.815120in}{2.550146in}}%
\pgfpathlineto{\pgfqpoint{5.815663in}{2.439716in}}%
\pgfpathlineto{\pgfqpoint{5.816134in}{2.565187in}}%
\pgfpathlineto{\pgfqpoint{5.816038in}{1.866545in}}%
\pgfpathlineto{\pgfqpoint{5.816774in}{2.474691in}}%
\pgfpathlineto{\pgfqpoint{5.817624in}{1.871887in}}%
\pgfpathlineto{\pgfqpoint{5.817421in}{2.537661in}}%
\pgfpathlineto{\pgfqpoint{5.817889in}{2.370804in}}%
\pgfpathlineto{\pgfqpoint{5.818416in}{1.812667in}}%
\pgfpathlineto{\pgfqpoint{5.818949in}{2.543100in}}%
\pgfpathlineto{\pgfqpoint{5.818988in}{2.286107in}}%
\pgfpathlineto{\pgfqpoint{5.819240in}{2.559175in}}%
\pgfpathlineto{\pgfqpoint{5.819853in}{2.000213in}}%
\pgfpathlineto{\pgfqpoint{5.820101in}{2.391652in}}%
\pgfpathlineto{\pgfqpoint{5.820683in}{1.949721in}}%
\pgfpathlineto{\pgfqpoint{5.821109in}{2.577210in}}%
\pgfpathlineto{\pgfqpoint{5.821207in}{2.446397in}}%
\pgfpathlineto{\pgfqpoint{5.821701in}{2.536642in}}%
\pgfpathlineto{\pgfqpoint{5.821726in}{2.060847in}}%
\pgfpathlineto{\pgfqpoint{5.822271in}{2.360147in}}%
\pgfpathlineto{\pgfqpoint{5.823317in}{1.784974in}}%
\pgfpathlineto{\pgfqpoint{5.822825in}{2.549614in}}%
\pgfpathlineto{\pgfqpoint{5.823382in}{2.364446in}}%
\pgfpathlineto{\pgfqpoint{5.823672in}{2.529562in}}%
\pgfpathlineto{\pgfqpoint{5.824399in}{1.998550in}}%
\pgfpathlineto{\pgfqpoint{5.824489in}{2.365516in}}%
\pgfpathlineto{\pgfqpoint{5.825481in}{1.897635in}}%
\pgfpathlineto{\pgfqpoint{5.824760in}{2.543480in}}%
\pgfpathlineto{\pgfqpoint{5.825597in}{2.339794in}}%
\pgfpathlineto{\pgfqpoint{5.826368in}{2.571072in}}%
\pgfpathlineto{\pgfqpoint{5.825693in}{1.958288in}}%
\pgfpathlineto{\pgfqpoint{5.826708in}{2.529643in}}%
\pgfpathlineto{\pgfqpoint{5.827743in}{2.000699in}}%
\pgfpathlineto{\pgfqpoint{5.826762in}{2.584683in}}%
\pgfpathlineto{\pgfqpoint{5.827819in}{2.405978in}}%
\pgfpathlineto{\pgfqpoint{5.828040in}{2.546403in}}%
\pgfpathlineto{\pgfqpoint{5.828781in}{1.815007in}}%
\pgfpathlineto{\pgfqpoint{5.828928in}{2.441930in}}%
\pgfpathlineto{\pgfqpoint{5.829655in}{2.046801in}}%
\pgfpathlineto{\pgfqpoint{5.829349in}{2.592695in}}%
\pgfpathlineto{\pgfqpoint{5.830034in}{2.361410in}}%
\pgfpathlineto{\pgfqpoint{5.830495in}{2.544922in}}%
\pgfpathlineto{\pgfqpoint{5.831013in}{2.051063in}}%
\pgfpathlineto{\pgfqpoint{5.831146in}{2.463558in}}%
\pgfpathlineto{\pgfqpoint{5.832025in}{2.081856in}}%
\pgfpathlineto{\pgfqpoint{5.831977in}{2.552408in}}%
\pgfpathlineto{\pgfqpoint{5.832259in}{2.431322in}}%
\pgfpathlineto{\pgfqpoint{5.833148in}{2.548430in}}%
\pgfpathlineto{\pgfqpoint{5.832639in}{1.971096in}}%
\pgfpathlineto{\pgfqpoint{5.833366in}{2.360768in}}%
\pgfpathlineto{\pgfqpoint{5.834450in}{1.922562in}}%
\pgfpathlineto{\pgfqpoint{5.834334in}{2.562797in}}%
\pgfpathlineto{\pgfqpoint{5.834460in}{2.376337in}}%
\pgfpathlineto{\pgfqpoint{5.835492in}{2.565340in}}%
\pgfpathlineto{\pgfqpoint{5.835356in}{2.012667in}}%
\pgfpathlineto{\pgfqpoint{5.835567in}{2.360223in}}%
\pgfpathlineto{\pgfqpoint{5.835680in}{1.842708in}}%
\pgfpathlineto{\pgfqpoint{5.836640in}{2.550302in}}%
\pgfpathlineto{\pgfqpoint{5.836668in}{2.319366in}}%
\pgfpathlineto{\pgfqpoint{5.837169in}{2.536913in}}%
\pgfpathlineto{\pgfqpoint{5.837003in}{2.014958in}}%
\pgfpathlineto{\pgfqpoint{5.837779in}{2.409351in}}%
\pgfpathlineto{\pgfqpoint{5.838821in}{2.064836in}}%
\pgfpathlineto{\pgfqpoint{5.838880in}{2.551894in}}%
\pgfpathlineto{\pgfqpoint{5.838890in}{2.228311in}}%
\pgfpathlineto{\pgfqpoint{5.839684in}{2.561215in}}%
\pgfpathlineto{\pgfqpoint{5.838949in}{1.952369in}}%
\pgfpathlineto{\pgfqpoint{5.840001in}{2.421135in}}%
\pgfpathlineto{\pgfqpoint{5.840275in}{2.541699in}}%
\pgfpathlineto{\pgfqpoint{5.840756in}{2.073051in}}%
\pgfpathlineto{\pgfqpoint{5.841026in}{2.255285in}}%
\pgfpathlineto{\pgfqpoint{5.841032in}{2.015247in}}%
\pgfpathlineto{\pgfqpoint{5.841806in}{2.551158in}}%
\pgfpathlineto{\pgfqpoint{5.842134in}{2.089586in}}%
\pgfpathlineto{\pgfqpoint{5.843157in}{2.539020in}}%
\pgfpathlineto{\pgfqpoint{5.842144in}{2.024920in}}%
\pgfpathlineto{\pgfqpoint{5.843246in}{2.498275in}}%
\pgfpathlineto{\pgfqpoint{5.843484in}{2.066716in}}%
\pgfpathlineto{\pgfqpoint{5.843296in}{2.543906in}}%
\pgfpathlineto{\pgfqpoint{5.844356in}{2.268181in}}%
\pgfpathlineto{\pgfqpoint{5.845235in}{2.540685in}}%
\pgfpathlineto{\pgfqpoint{5.845244in}{2.010732in}}%
\pgfpathlineto{\pgfqpoint{5.845468in}{2.339151in}}%
\pgfpathlineto{\pgfqpoint{5.845474in}{2.323885in}}%
\pgfpathlineto{\pgfqpoint{5.845477in}{2.371269in}}%
\pgfpathlineto{\pgfqpoint{5.845483in}{2.365092in}}%
\pgfpathlineto{\pgfqpoint{5.845649in}{2.566326in}}%
\pgfpathlineto{\pgfqpoint{5.845738in}{1.709436in}}%
\pgfpathlineto{\pgfqpoint{5.846590in}{2.478315in}}%
\pgfpathlineto{\pgfqpoint{5.846899in}{1.948225in}}%
\pgfpathlineto{\pgfqpoint{5.846993in}{2.574760in}}%
\pgfpathlineto{\pgfqpoint{5.847699in}{2.281060in}}%
\pgfpathlineto{\pgfqpoint{5.848163in}{2.558888in}}%
\pgfpathlineto{\pgfqpoint{5.848717in}{2.059952in}}%
\pgfpathlineto{\pgfqpoint{5.848815in}{2.501749in}}%
\pgfpathlineto{\pgfqpoint{5.849639in}{2.005904in}}%
\pgfpathlineto{\pgfqpoint{5.849651in}{2.549474in}}%
\pgfpathlineto{\pgfqpoint{5.849928in}{2.288631in}}%
\pgfpathlineto{\pgfqpoint{5.850662in}{2.542629in}}%
\pgfpathlineto{\pgfqpoint{5.850098in}{1.933906in}}%
\pgfpathlineto{\pgfqpoint{5.851041in}{2.437973in}}%
\pgfpathlineto{\pgfqpoint{5.851840in}{2.088615in}}%
\pgfpathlineto{\pgfqpoint{5.851601in}{2.558903in}}%
\pgfpathlineto{\pgfqpoint{5.852151in}{2.385224in}}%
\pgfpathlineto{\pgfqpoint{5.852601in}{2.540504in}}%
\pgfpathlineto{\pgfqpoint{5.852314in}{1.864844in}}%
\pgfpathlineto{\pgfqpoint{5.853252in}{2.453564in}}%
\pgfpathlineto{\pgfqpoint{5.854107in}{1.945311in}}%
\pgfpathlineto{\pgfqpoint{5.854179in}{2.527259in}}%
\pgfpathlineto{\pgfqpoint{5.854362in}{2.428122in}}%
\pgfpathlineto{\pgfqpoint{5.854834in}{1.941722in}}%
\pgfpathlineto{\pgfqpoint{5.854906in}{2.539421in}}%
\pgfpathlineto{\pgfqpoint{5.855461in}{2.439211in}}%
\pgfpathlineto{\pgfqpoint{5.855952in}{2.549572in}}%
\pgfpathlineto{\pgfqpoint{5.856006in}{1.910632in}}%
\pgfpathlineto{\pgfqpoint{5.856565in}{2.446217in}}%
\pgfpathlineto{\pgfqpoint{5.857643in}{2.054451in}}%
\pgfpathlineto{\pgfqpoint{5.856966in}{2.549665in}}%
\pgfpathlineto{\pgfqpoint{5.857676in}{2.382639in}}%
\pgfpathlineto{\pgfqpoint{5.858338in}{2.532769in}}%
\pgfpathlineto{\pgfqpoint{5.858558in}{1.964392in}}%
\pgfpathlineto{\pgfqpoint{5.858781in}{2.383049in}}%
\pgfpathlineto{\pgfqpoint{5.858941in}{1.896160in}}%
\pgfpathlineto{\pgfqpoint{5.859046in}{2.563369in}}%
\pgfpathlineto{\pgfqpoint{5.859892in}{2.291259in}}%
\pgfpathlineto{\pgfqpoint{5.860680in}{2.572202in}}%
\pgfpathlineto{\pgfqpoint{5.860298in}{1.978158in}}%
\pgfpathlineto{\pgfqpoint{5.861003in}{2.388296in}}%
\pgfpathlineto{\pgfqpoint{5.862060in}{1.999351in}}%
\pgfpathlineto{\pgfqpoint{5.861668in}{2.571856in}}%
\pgfpathlineto{\pgfqpoint{5.862114in}{2.214652in}}%
\pgfpathlineto{\pgfqpoint{5.862765in}{2.538106in}}%
\pgfpathlineto{\pgfqpoint{5.862173in}{1.967789in}}%
\pgfpathlineto{\pgfqpoint{5.863225in}{2.246974in}}%
\pgfpathlineto{\pgfqpoint{5.864127in}{2.548047in}}%
\pgfpathlineto{\pgfqpoint{5.864033in}{1.966678in}}%
\pgfpathlineto{\pgfqpoint{5.864338in}{2.488208in}}%
\pgfpathlineto{\pgfqpoint{5.864708in}{2.006689in}}%
\pgfpathlineto{\pgfqpoint{5.864805in}{2.555646in}}%
\pgfpathlineto{\pgfqpoint{5.865452in}{2.330897in}}%
\pgfpathlineto{\pgfqpoint{5.865669in}{2.577191in}}%
\pgfpathlineto{\pgfqpoint{5.866078in}{1.991420in}}%
\pgfpathlineto{\pgfqpoint{5.866569in}{2.492814in}}%
\pgfpathlineto{\pgfqpoint{5.866899in}{1.845341in}}%
\pgfpathlineto{\pgfqpoint{5.867220in}{2.573945in}}%
\pgfpathlineto{\pgfqpoint{5.867677in}{2.411047in}}%
\pgfpathlineto{\pgfqpoint{5.867948in}{2.548107in}}%
\pgfpathlineto{\pgfqpoint{5.868050in}{1.826500in}}%
\pgfpathlineto{\pgfqpoint{5.868786in}{2.465421in}}%
\pgfpathlineto{\pgfqpoint{5.869308in}{1.927056in}}%
\pgfpathlineto{\pgfqpoint{5.868966in}{2.556928in}}%
\pgfpathlineto{\pgfqpoint{5.869897in}{2.425238in}}%
\pgfpathlineto{\pgfqpoint{5.870325in}{2.553118in}}%
\pgfpathlineto{\pgfqpoint{5.870664in}{1.933271in}}%
\pgfpathlineto{\pgfqpoint{5.870999in}{2.327821in}}%
\pgfpathlineto{\pgfqpoint{5.871282in}{2.061681in}}%
\pgfpathlineto{\pgfqpoint{5.871510in}{2.561940in}}%
\pgfpathlineto{\pgfqpoint{5.872081in}{2.259616in}}%
\pgfpathlineto{\pgfqpoint{5.872375in}{2.568826in}}%
\pgfpathlineto{\pgfqpoint{5.873152in}{1.902295in}}%
\pgfpathlineto{\pgfqpoint{5.873189in}{2.499510in}}%
\pgfpathlineto{\pgfqpoint{5.873962in}{1.918051in}}%
\pgfpathlineto{\pgfqpoint{5.873896in}{2.598631in}}%
\pgfpathlineto{\pgfqpoint{5.874300in}{2.408086in}}%
\pgfpathlineto{\pgfqpoint{5.874303in}{2.408331in}}%
\pgfpathlineto{\pgfqpoint{5.874306in}{2.355021in}}%
\pgfpathlineto{\pgfqpoint{5.875017in}{2.046872in}}%
\pgfpathlineto{\pgfqpoint{5.875094in}{2.549534in}}%
\pgfpathlineto{\pgfqpoint{5.875414in}{2.395668in}}%
\pgfpathlineto{\pgfqpoint{5.875977in}{2.549852in}}%
\pgfpathlineto{\pgfqpoint{5.876143in}{2.024008in}}%
\pgfpathlineto{\pgfqpoint{5.876520in}{2.481138in}}%
\pgfpathlineto{\pgfqpoint{5.876979in}{2.007884in}}%
\pgfpathlineto{\pgfqpoint{5.877167in}{2.561223in}}%
\pgfpathlineto{\pgfqpoint{5.877630in}{2.376325in}}%
\pgfpathlineto{\pgfqpoint{5.878304in}{1.949991in}}%
\pgfpathlineto{\pgfqpoint{5.878188in}{2.545856in}}%
\pgfpathlineto{\pgfqpoint{5.878736in}{2.431163in}}%
\pgfpathlineto{\pgfqpoint{5.878869in}{2.549803in}}%
\pgfpathlineto{\pgfqpoint{5.878988in}{1.654735in}}%
\pgfpathlineto{\pgfqpoint{5.879835in}{2.408957in}}%
\pgfpathlineto{\pgfqpoint{5.879985in}{1.919280in}}%
\pgfpathlineto{\pgfqpoint{5.879996in}{2.546134in}}%
\pgfpathlineto{\pgfqpoint{5.880946in}{2.362134in}}%
\pgfpathlineto{\pgfqpoint{5.881628in}{2.552841in}}%
\pgfpathlineto{\pgfqpoint{5.881510in}{1.759916in}}%
\pgfpathlineto{\pgfqpoint{5.882057in}{2.466443in}}%
\pgfpathlineto{\pgfqpoint{5.883159in}{2.019727in}}%
\pgfpathlineto{\pgfqpoint{5.882223in}{2.533946in}}%
\pgfpathlineto{\pgfqpoint{5.883167in}{2.388118in}}%
\pgfpathlineto{\pgfqpoint{5.883599in}{2.543560in}}%
\pgfpathlineto{\pgfqpoint{5.883835in}{1.921876in}}%
\pgfpathlineto{\pgfqpoint{5.884261in}{2.205240in}}%
\pgfpathlineto{\pgfqpoint{5.884952in}{1.932782in}}%
\pgfpathlineto{\pgfqpoint{5.885014in}{2.548241in}}%
\pgfpathlineto{\pgfqpoint{5.885354in}{2.323808in}}%
\pgfpathlineto{\pgfqpoint{5.886089in}{2.553351in}}%
\pgfpathlineto{\pgfqpoint{5.886376in}{2.002652in}}%
\pgfpathlineto{\pgfqpoint{5.886465in}{2.386311in}}%
\pgfpathlineto{\pgfqpoint{5.887155in}{1.949785in}}%
\pgfpathlineto{\pgfqpoint{5.887055in}{2.566112in}}%
\pgfpathlineto{\pgfqpoint{5.887581in}{2.183458in}}%
\pgfpathlineto{\pgfqpoint{5.887778in}{2.537618in}}%
\pgfpathlineto{\pgfqpoint{5.888638in}{1.980995in}}%
\pgfpathlineto{\pgfqpoint{5.888691in}{2.354801in}}%
\pgfpathlineto{\pgfqpoint{5.888998in}{2.567720in}}%
\pgfpathlineto{\pgfqpoint{5.888746in}{1.973042in}}%
\pgfpathlineto{\pgfqpoint{5.889756in}{2.435442in}}%
\pgfpathlineto{\pgfqpoint{5.890773in}{1.861017in}}%
\pgfpathlineto{\pgfqpoint{5.890786in}{2.561842in}}%
\pgfpathlineto{\pgfqpoint{5.890866in}{2.120027in}}%
\pgfpathlineto{\pgfqpoint{5.891918in}{2.582559in}}%
\pgfpathlineto{\pgfqpoint{5.891442in}{1.877555in}}%
\pgfpathlineto{\pgfqpoint{5.891979in}{2.444792in}}%
\pgfpathlineto{\pgfqpoint{5.892463in}{2.565071in}}%
\pgfpathlineto{\pgfqpoint{5.892251in}{2.095060in}}%
\pgfpathlineto{\pgfqpoint{5.893072in}{2.391595in}}%
\pgfpathlineto{\pgfqpoint{5.893130in}{1.878017in}}%
\pgfpathlineto{\pgfqpoint{5.893779in}{2.535747in}}%
\pgfpathlineto{\pgfqpoint{5.894182in}{2.263927in}}%
\pgfpathlineto{\pgfqpoint{5.895247in}{2.579034in}}%
\pgfpathlineto{\pgfqpoint{5.894685in}{2.089289in}}%
\pgfpathlineto{\pgfqpoint{5.895291in}{2.352052in}}%
\pgfpathlineto{\pgfqpoint{5.896131in}{2.038412in}}%
\pgfpathlineto{\pgfqpoint{5.896340in}{2.545215in}}%
\pgfpathlineto{\pgfqpoint{5.896397in}{2.436285in}}%
\pgfpathlineto{\pgfqpoint{5.896610in}{2.563538in}}%
\pgfpathlineto{\pgfqpoint{5.896931in}{1.953102in}}%
\pgfpathlineto{\pgfqpoint{5.897498in}{2.445076in}}%
\pgfpathlineto{\pgfqpoint{5.898512in}{1.968124in}}%
\pgfpathlineto{\pgfqpoint{5.897957in}{2.564259in}}%
\pgfpathlineto{\pgfqpoint{5.898610in}{2.366951in}}%
\pgfpathlineto{\pgfqpoint{5.899664in}{2.557885in}}%
\pgfpathlineto{\pgfqpoint{5.899294in}{2.084297in}}%
\pgfpathlineto{\pgfqpoint{5.899716in}{2.394937in}}%
\pgfpathlineto{\pgfqpoint{5.899845in}{1.912145in}}%
\pgfpathlineto{\pgfqpoint{5.900805in}{2.573449in}}%
\pgfpathlineto{\pgfqpoint{5.900824in}{2.397081in}}%
\pgfpathlineto{\pgfqpoint{5.901580in}{2.563507in}}%
\pgfpathlineto{\pgfqpoint{5.901615in}{2.015872in}}%
\pgfpathlineto{\pgfqpoint{5.901908in}{2.399414in}}%
\pgfpathlineto{\pgfqpoint{5.902711in}{1.800900in}}%
\pgfpathlineto{\pgfqpoint{5.902276in}{2.575498in}}%
\pgfpathlineto{\pgfqpoint{5.903016in}{2.372992in}}%
\pgfpathlineto{\pgfqpoint{5.903091in}{2.573793in}}%
\pgfpathlineto{\pgfqpoint{5.903351in}{1.976054in}}%
\pgfpathlineto{\pgfqpoint{5.904124in}{2.390563in}}%
\pgfpathlineto{\pgfqpoint{5.904600in}{2.044662in}}%
\pgfpathlineto{\pgfqpoint{5.904135in}{2.563867in}}%
\pgfpathlineto{\pgfqpoint{5.905232in}{2.479927in}}%
\pgfpathlineto{\pgfqpoint{5.906233in}{1.924404in}}%
\pgfpathlineto{\pgfqpoint{5.905515in}{2.557979in}}%
\pgfpathlineto{\pgfqpoint{5.906337in}{2.303327in}}%
\pgfpathlineto{\pgfqpoint{5.906770in}{2.587578in}}%
\pgfpathlineto{\pgfqpoint{5.907256in}{1.976855in}}%
\pgfpathlineto{\pgfqpoint{5.907447in}{2.401565in}}%
\pgfpathlineto{\pgfqpoint{5.908260in}{2.012231in}}%
\pgfpathlineto{\pgfqpoint{5.907502in}{2.558306in}}%
\pgfpathlineto{\pgfqpoint{5.908556in}{2.401224in}}%
\pgfpathlineto{\pgfqpoint{5.908784in}{1.941475in}}%
\pgfpathlineto{\pgfqpoint{5.908773in}{2.570072in}}%
\pgfpathlineto{\pgfqpoint{5.909650in}{2.346704in}}%
\pgfpathlineto{\pgfqpoint{5.910040in}{2.564156in}}%
\pgfpathlineto{\pgfqpoint{5.909911in}{1.947114in}}%
\pgfpathlineto{\pgfqpoint{5.910762in}{2.426505in}}%
\pgfpathlineto{\pgfqpoint{5.911096in}{1.924935in}}%
\pgfpathlineto{\pgfqpoint{5.911863in}{2.552142in}}%
\pgfpathlineto{\pgfqpoint{5.911873in}{2.319181in}}%
\pgfpathlineto{\pgfqpoint{5.912568in}{2.559216in}}%
\pgfpathlineto{\pgfqpoint{5.912534in}{1.891238in}}%
\pgfpathlineto{\pgfqpoint{5.912984in}{2.439529in}}%
\pgfpathlineto{\pgfqpoint{5.913939in}{1.863360in}}%
\pgfpathlineto{\pgfqpoint{5.913607in}{2.568614in}}%
\pgfpathlineto{\pgfqpoint{5.914095in}{2.416481in}}%
\pgfpathlineto{\pgfqpoint{5.915118in}{2.536238in}}%
\pgfpathlineto{\pgfqpoint{5.914218in}{1.650639in}}%
\pgfpathlineto{\pgfqpoint{5.915209in}{2.512851in}}%
\pgfpathlineto{\pgfqpoint{5.915921in}{1.866782in}}%
\pgfpathlineto{\pgfqpoint{5.916200in}{2.567391in}}%
\pgfpathlineto{\pgfqpoint{5.916319in}{2.314911in}}%
\pgfpathlineto{\pgfqpoint{5.916654in}{2.561630in}}%
\pgfpathlineto{\pgfqpoint{5.916822in}{1.986375in}}%
\pgfpathlineto{\pgfqpoint{5.917429in}{2.418949in}}%
\pgfpathlineto{\pgfqpoint{5.918086in}{2.006914in}}%
\pgfpathlineto{\pgfqpoint{5.917872in}{2.575848in}}%
\pgfpathlineto{\pgfqpoint{5.918541in}{2.337527in}}%
\pgfpathlineto{\pgfqpoint{5.919584in}{2.541865in}}%
\pgfpathlineto{\pgfqpoint{5.919385in}{1.999635in}}%
\pgfpathlineto{\pgfqpoint{5.919648in}{2.388050in}}%
\pgfpathlineto{\pgfqpoint{5.920696in}{1.976657in}}%
\pgfpathlineto{\pgfqpoint{5.920678in}{2.556402in}}%
\pgfpathlineto{\pgfqpoint{5.920760in}{2.364233in}}%
\pgfpathlineto{\pgfqpoint{5.921494in}{2.052955in}}%
\pgfpathlineto{\pgfqpoint{5.921063in}{2.556787in}}%
\pgfpathlineto{\pgfqpoint{5.921868in}{2.078746in}}%
\pgfpathlineto{\pgfqpoint{5.922634in}{2.538490in}}%
\pgfpathlineto{\pgfqpoint{5.922795in}{1.925425in}}%
\pgfpathlineto{\pgfqpoint{5.922982in}{2.429016in}}%
\pgfpathlineto{\pgfqpoint{5.923100in}{2.023393in}}%
\pgfpathlineto{\pgfqpoint{5.923835in}{2.557498in}}%
\pgfpathlineto{\pgfqpoint{5.924098in}{2.253077in}}%
\pgfpathlineto{\pgfqpoint{5.924722in}{2.557149in}}%
\pgfpathlineto{\pgfqpoint{5.924350in}{1.835345in}}%
\pgfpathlineto{\pgfqpoint{5.925210in}{2.430857in}}%
\pgfpathlineto{\pgfqpoint{5.925292in}{1.996186in}}%
\pgfpathlineto{\pgfqpoint{5.925566in}{2.537856in}}%
\pgfpathlineto{\pgfqpoint{5.926320in}{2.407861in}}%
\pgfpathlineto{\pgfqpoint{5.926777in}{1.924614in}}%
\pgfpathlineto{\pgfqpoint{5.927154in}{2.568687in}}%
\pgfpathlineto{\pgfqpoint{5.927422in}{2.146658in}}%
\pgfpathlineto{\pgfqpoint{5.928188in}{2.537786in}}%
\pgfpathlineto{\pgfqpoint{5.928059in}{1.869576in}}%
\pgfpathlineto{\pgfqpoint{5.928534in}{2.364720in}}%
\pgfpathlineto{\pgfqpoint{5.928617in}{1.966219in}}%
\pgfpathlineto{\pgfqpoint{5.928834in}{2.567386in}}%
\pgfpathlineto{\pgfqpoint{5.929635in}{2.354759in}}%
\pgfpathlineto{\pgfqpoint{5.929884in}{2.550857in}}%
\pgfpathlineto{\pgfqpoint{5.930528in}{2.072402in}}%
\pgfpathlineto{\pgfqpoint{5.930741in}{2.373172in}}%
\pgfpathlineto{\pgfqpoint{5.931736in}{1.891633in}}%
\pgfpathlineto{\pgfqpoint{5.931200in}{2.585575in}}%
\pgfpathlineto{\pgfqpoint{5.931849in}{2.398302in}}%
\pgfpathlineto{\pgfqpoint{5.932892in}{2.575974in}}%
\pgfpathlineto{\pgfqpoint{5.931912in}{1.988635in}}%
\pgfpathlineto{\pgfqpoint{5.932944in}{2.291726in}}%
\pgfpathlineto{\pgfqpoint{5.933136in}{2.055828in}}%
\pgfpathlineto{\pgfqpoint{5.933937in}{2.548345in}}%
\pgfpathlineto{\pgfqpoint{5.934049in}{2.517114in}}%
\pgfpathlineto{\pgfqpoint{5.934805in}{2.559369in}}%
\pgfpathlineto{\pgfqpoint{5.934970in}{1.827256in}}%
\pgfpathlineto{\pgfqpoint{5.935148in}{2.453408in}}%
\pgfpathlineto{\pgfqpoint{5.935769in}{1.988131in}}%
\pgfpathlineto{\pgfqpoint{5.935486in}{2.584759in}}%
\pgfpathlineto{\pgfqpoint{5.936260in}{2.200449in}}%
\pgfpathlineto{\pgfqpoint{5.936532in}{2.548674in}}%
\pgfpathlineto{\pgfqpoint{5.937304in}{2.076385in}}%
\pgfpathlineto{\pgfqpoint{5.937373in}{2.402632in}}%
\pgfpathlineto{\pgfqpoint{5.938104in}{2.550927in}}%
\pgfpathlineto{\pgfqpoint{5.937761in}{2.001749in}}%
\pgfpathlineto{\pgfqpoint{5.938134in}{2.396079in}}%
\pgfpathlineto{\pgfqpoint{5.939028in}{2.062371in}}%
\pgfpathlineto{\pgfqpoint{5.939078in}{2.586841in}}%
\pgfpathlineto{\pgfqpoint{5.939245in}{2.377727in}}%
\pgfpathlineto{\pgfqpoint{5.940321in}{1.940486in}}%
\pgfpathlineto{\pgfqpoint{5.939606in}{2.592882in}}%
\pgfpathlineto{\pgfqpoint{5.940353in}{2.422997in}}%
\pgfpathlineto{\pgfqpoint{5.940797in}{2.533007in}}%
\pgfpathlineto{\pgfqpoint{5.940456in}{1.883477in}}%
\pgfpathlineto{\pgfqpoint{5.941458in}{2.480905in}}%
\pgfpathlineto{\pgfqpoint{5.942549in}{1.971138in}}%
\pgfpathlineto{\pgfqpoint{5.941620in}{2.541182in}}%
\pgfpathlineto{\pgfqpoint{5.942571in}{2.420942in}}%
\pgfpathlineto{\pgfqpoint{5.943032in}{1.963289in}}%
\pgfpathlineto{\pgfqpoint{5.943317in}{2.547667in}}%
\pgfpathlineto{\pgfqpoint{5.943670in}{2.331839in}}%
\pgfpathlineto{\pgfqpoint{5.943782in}{2.574550in}}%
\pgfpathlineto{\pgfqpoint{5.943916in}{2.012342in}}%
\pgfpathlineto{\pgfqpoint{5.944781in}{2.467924in}}%
\pgfpathlineto{\pgfqpoint{5.945480in}{1.756938in}}%
\pgfpathlineto{\pgfqpoint{5.945131in}{2.548779in}}%
\pgfpathlineto{\pgfqpoint{5.945892in}{2.470095in}}%
\pgfpathlineto{\pgfqpoint{5.945955in}{2.549906in}}%
\pgfpathlineto{\pgfqpoint{5.946650in}{1.994872in}}%
\pgfpathlineto{\pgfqpoint{5.946993in}{2.422275in}}%
\pgfpathlineto{\pgfqpoint{5.947232in}{1.874804in}}%
\pgfpathlineto{\pgfqpoint{5.947857in}{2.545484in}}%
\pgfpathlineto{\pgfqpoint{5.948103in}{2.393052in}}%
\pgfpathlineto{\pgfqpoint{5.948230in}{2.547706in}}%
\pgfpathlineto{\pgfqpoint{5.948738in}{1.973198in}}%
\pgfpathlineto{\pgfqpoint{5.949209in}{2.448368in}}%
\pgfpathlineto{\pgfqpoint{5.949452in}{1.987450in}}%
\pgfpathlineto{\pgfqpoint{5.950079in}{2.535426in}}%
\pgfpathlineto{\pgfqpoint{5.950318in}{2.455035in}}%
\pgfpathlineto{\pgfqpoint{5.950874in}{2.547451in}}%
\pgfpathlineto{\pgfqpoint{5.950349in}{1.800325in}}%
\pgfpathlineto{\pgfqpoint{5.951412in}{2.465095in}}%
\pgfpathlineto{\pgfqpoint{5.951678in}{1.901225in}}%
\pgfpathlineto{\pgfqpoint{5.952456in}{2.586338in}}%
\pgfpathlineto{\pgfqpoint{5.952523in}{2.385114in}}%
\pgfpathlineto{\pgfqpoint{5.952625in}{2.085618in}}%
\pgfpathlineto{\pgfqpoint{5.953397in}{2.546866in}}%
\pgfpathlineto{\pgfqpoint{5.953632in}{2.360471in}}%
\pgfpathlineto{\pgfqpoint{5.953770in}{2.568940in}}%
\pgfpathlineto{\pgfqpoint{5.954357in}{1.966916in}}%
\pgfpathlineto{\pgfqpoint{5.954723in}{2.367323in}}%
\pgfpathlineto{\pgfqpoint{5.955194in}{1.985015in}}%
\pgfpathlineto{\pgfqpoint{5.955111in}{2.594410in}}%
\pgfpathlineto{\pgfqpoint{5.955831in}{2.178808in}}%
\pgfpathlineto{\pgfqpoint{5.956082in}{2.540328in}}%
\pgfpathlineto{\pgfqpoint{5.956659in}{1.878598in}}%
\pgfpathlineto{\pgfqpoint{5.956944in}{2.507536in}}%
\pgfpathlineto{\pgfqpoint{5.956965in}{1.884765in}}%
\pgfpathlineto{\pgfqpoint{5.957159in}{2.551099in}}%
\pgfpathlineto{\pgfqpoint{5.958057in}{2.445303in}}%
\pgfpathlineto{\pgfqpoint{5.958431in}{2.538792in}}%
\pgfpathlineto{\pgfqpoint{5.958887in}{1.994469in}}%
\pgfpathlineto{\pgfqpoint{5.959159in}{2.354918in}}%
\pgfpathlineto{\pgfqpoint{5.959495in}{1.942882in}}%
\pgfpathlineto{\pgfqpoint{5.959842in}{2.575834in}}%
\pgfpathlineto{\pgfqpoint{5.960266in}{2.409627in}}%
\pgfpathlineto{\pgfqpoint{5.960904in}{2.559884in}}%
\pgfpathlineto{\pgfqpoint{5.961175in}{2.011190in}}%
\pgfpathlineto{\pgfqpoint{5.961362in}{2.419578in}}%
\pgfpathlineto{\pgfqpoint{5.961703in}{1.971189in}}%
\pgfpathlineto{\pgfqpoint{5.961960in}{2.559340in}}%
\pgfpathlineto{\pgfqpoint{5.962473in}{2.093176in}}%
\pgfpathlineto{\pgfqpoint{5.962840in}{2.558479in}}%
\pgfpathlineto{\pgfqpoint{5.963259in}{1.893008in}}%
\pgfpathlineto{\pgfqpoint{5.963582in}{2.249502in}}%
\pgfpathlineto{\pgfqpoint{5.964146in}{1.981230in}}%
\pgfpathlineto{\pgfqpoint{5.964276in}{2.558072in}}%
\pgfpathlineto{\pgfqpoint{5.964689in}{2.371999in}}%
\pgfpathlineto{\pgfqpoint{5.965124in}{2.557294in}}%
\pgfpathlineto{\pgfqpoint{5.965448in}{1.800401in}}%
\pgfpathlineto{\pgfqpoint{5.965797in}{2.416814in}}%
\pgfpathlineto{\pgfqpoint{5.966552in}{1.851647in}}%
\pgfpathlineto{\pgfqpoint{5.965945in}{2.555760in}}%
\pgfpathlineto{\pgfqpoint{5.966905in}{2.410221in}}%
\pgfpathlineto{\pgfqpoint{5.967654in}{2.572902in}}%
\pgfpathlineto{\pgfqpoint{5.967594in}{2.091896in}}%
\pgfpathlineto{\pgfqpoint{5.968013in}{2.374285in}}%
\pgfpathlineto{\pgfqpoint{5.968849in}{1.926891in}}%
\pgfpathlineto{\pgfqpoint{5.969048in}{2.569693in}}%
\pgfpathlineto{\pgfqpoint{5.969110in}{2.447776in}}%
\pgfpathlineto{\pgfqpoint{5.969729in}{2.534089in}}%
\pgfpathlineto{\pgfqpoint{5.969914in}{2.058786in}}%
\pgfpathlineto{\pgfqpoint{5.970202in}{2.342543in}}%
\pgfpathlineto{\pgfqpoint{5.970605in}{2.063132in}}%
\pgfpathlineto{\pgfqpoint{5.970600in}{2.564520in}}%
\pgfpathlineto{\pgfqpoint{5.971313in}{2.297519in}}%
\pgfpathlineto{\pgfqpoint{5.971319in}{2.561838in}}%
\pgfpathlineto{\pgfqpoint{5.971317in}{1.871548in}}%
\pgfpathlineto{\pgfqpoint{5.972427in}{2.487924in}}%
\pgfpathlineto{\pgfqpoint{5.973363in}{1.957634in}}%
\pgfpathlineto{\pgfqpoint{5.972545in}{2.578411in}}%
\pgfpathlineto{\pgfqpoint{5.973540in}{2.384234in}}%
\pgfpathlineto{\pgfqpoint{5.973545in}{2.487602in}}%
\pgfpathlineto{\pgfqpoint{5.973549in}{2.331500in}}%
\pgfpathlineto{\pgfqpoint{5.974519in}{1.998505in}}%
\pgfpathlineto{\pgfqpoint{5.973710in}{2.560311in}}%
\pgfpathlineto{\pgfqpoint{5.974660in}{2.175027in}}%
\pgfpathlineto{\pgfqpoint{5.974809in}{2.549050in}}%
\pgfpathlineto{\pgfqpoint{5.975546in}{2.024692in}}%
\pgfpathlineto{\pgfqpoint{5.975772in}{2.485907in}}%
\pgfpathlineto{\pgfqpoint{5.976264in}{1.894436in}}%
\pgfpathlineto{\pgfqpoint{5.976232in}{2.549889in}}%
\pgfpathlineto{\pgfqpoint{5.976883in}{2.188316in}}%
\pgfpathlineto{\pgfqpoint{5.977075in}{2.591395in}}%
\pgfpathlineto{\pgfqpoint{5.977897in}{2.019701in}}%
\pgfpathlineto{\pgfqpoint{5.977996in}{2.395134in}}%
\pgfpathlineto{\pgfqpoint{5.978961in}{2.563113in}}%
\pgfpathlineto{\pgfqpoint{5.978367in}{1.952840in}}%
\pgfpathlineto{\pgfqpoint{5.979097in}{2.386227in}}%
\pgfpathlineto{\pgfqpoint{5.980171in}{2.024481in}}%
\pgfpathlineto{\pgfqpoint{5.979413in}{2.560048in}}%
\pgfpathlineto{\pgfqpoint{5.980207in}{2.283040in}}%
\pgfpathlineto{\pgfqpoint{5.981282in}{2.542747in}}%
\pgfpathlineto{\pgfqpoint{5.981280in}{1.950329in}}%
\pgfpathlineto{\pgfqpoint{5.981316in}{2.458225in}}%
\pgfpathlineto{\pgfqpoint{5.981318in}{1.946828in}}%
\pgfpathlineto{\pgfqpoint{5.982175in}{2.549300in}}%
\pgfpathlineto{\pgfqpoint{5.982426in}{2.327370in}}%
\pgfpathlineto{\pgfqpoint{5.982451in}{2.566586in}}%
\pgfpathlineto{\pgfqpoint{5.982471in}{1.754451in}}%
\pgfpathlineto{\pgfqpoint{5.983538in}{2.402400in}}%
\pgfpathlineto{\pgfqpoint{5.983729in}{2.548947in}}%
\pgfpathlineto{\pgfqpoint{5.984287in}{1.928036in}}%
\pgfpathlineto{\pgfqpoint{5.984647in}{2.461701in}}%
\pgfpathlineto{\pgfqpoint{5.984963in}{1.985207in}}%
\pgfpathlineto{\pgfqpoint{5.984903in}{2.552249in}}%
\pgfpathlineto{\pgfqpoint{5.985758in}{2.381504in}}%
\pgfpathlineto{\pgfqpoint{5.986024in}{2.545573in}}%
\pgfpathlineto{\pgfqpoint{5.986603in}{1.811891in}}%
\pgfpathlineto{\pgfqpoint{5.986870in}{2.521422in}}%
\pgfpathlineto{\pgfqpoint{5.986977in}{2.013354in}}%
\pgfpathlineto{\pgfqpoint{5.986988in}{2.561210in}}%
\pgfpathlineto{\pgfqpoint{5.987981in}{2.355849in}}%
\pgfpathlineto{\pgfqpoint{5.988726in}{2.567751in}}%
\pgfpathlineto{\pgfqpoint{5.988108in}{2.010440in}}%
\pgfpathlineto{\pgfqpoint{5.989094in}{2.456793in}}%
\pgfpathlineto{\pgfqpoint{5.990057in}{2.054767in}}%
\pgfpathlineto{\pgfqpoint{5.989273in}{2.548005in}}%
\pgfpathlineto{\pgfqpoint{5.990206in}{2.272000in}}%
\pgfpathlineto{\pgfqpoint{5.991189in}{2.538036in}}%
\pgfpathlineto{\pgfqpoint{5.991217in}{2.002868in}}%
\pgfpathlineto{\pgfqpoint{5.991321in}{2.390707in}}%
\pgfpathlineto{\pgfqpoint{5.991528in}{2.552680in}}%
\pgfpathlineto{\pgfqpoint{5.991789in}{1.847473in}}%
\pgfpathlineto{\pgfqpoint{5.992386in}{2.424834in}}%
\pgfpathlineto{\pgfqpoint{5.992410in}{1.623758in}}%
\pgfpathlineto{\pgfqpoint{5.992820in}{2.558749in}}%
\pgfpathlineto{\pgfqpoint{5.993496in}{2.426659in}}%
\pgfpathlineto{\pgfqpoint{5.994251in}{1.823679in}}%
\pgfpathlineto{\pgfqpoint{5.994333in}{2.543700in}}%
\pgfpathlineto{\pgfqpoint{5.994610in}{2.269974in}}%
\pgfpathlineto{\pgfqpoint{5.994727in}{2.533531in}}%
\pgfpathlineto{\pgfqpoint{5.995477in}{1.523834in}}%
\pgfpathlineto{\pgfqpoint{5.995723in}{2.473701in}}%
\pgfpathlineto{\pgfqpoint{5.996747in}{2.013119in}}%
\pgfpathlineto{\pgfqpoint{5.996590in}{2.564705in}}%
\pgfpathlineto{\pgfqpoint{5.996828in}{2.449257in}}%
\pgfpathlineto{\pgfqpoint{5.997146in}{2.543311in}}%
\pgfpathlineto{\pgfqpoint{5.996852in}{2.029800in}}%
\pgfpathlineto{\pgfqpoint{5.997937in}{2.424977in}}%
\pgfpathlineto{\pgfqpoint{5.997961in}{1.828893in}}%
\pgfpathlineto{\pgfqpoint{5.998843in}{2.564695in}}%
\pgfpathlineto{\pgfqpoint{5.999042in}{2.345645in}}%
\pgfpathlineto{\pgfqpoint{5.999439in}{2.565528in}}%
\pgfpathlineto{\pgfqpoint{6.000139in}{1.988366in}}%
\pgfpathlineto{\pgfqpoint{6.000151in}{2.430953in}}%
\pgfpathlineto{\pgfqpoint{6.000505in}{1.892883in}}%
\pgfpathlineto{\pgfqpoint{6.000848in}{2.605033in}}%
\pgfpathlineto{\pgfqpoint{6.001260in}{2.426933in}}%
\pgfpathlineto{\pgfqpoint{6.002199in}{2.562235in}}%
\pgfpathlineto{\pgfqpoint{6.001590in}{1.939150in}}%
\pgfpathlineto{\pgfqpoint{6.002360in}{2.477533in}}%
\pgfpathlineto{\pgfqpoint{6.002612in}{1.974977in}}%
\pgfpathlineto{\pgfqpoint{6.002549in}{2.561266in}}%
\pgfpathlineto{\pgfqpoint{6.003472in}{2.402249in}}%
\pgfpathlineto{\pgfqpoint{6.003658in}{1.921646in}}%
\pgfpathlineto{\pgfqpoint{6.004135in}{2.542599in}}%
\pgfpathlineto{\pgfqpoint{6.004583in}{2.363888in}}%
\pgfpathlineto{\pgfqpoint{6.005676in}{2.568372in}}%
\pgfpathlineto{\pgfqpoint{6.005249in}{1.960673in}}%
\pgfpathlineto{\pgfqpoint{6.005691in}{2.422199in}}%
\pgfpathlineto{\pgfqpoint{6.006008in}{1.806359in}}%
\pgfpathlineto{\pgfqpoint{6.006668in}{2.550233in}}%
\pgfpathlineto{\pgfqpoint{6.006800in}{2.338475in}}%
\pgfpathlineto{\pgfqpoint{6.007024in}{2.558705in}}%
\pgfpathlineto{\pgfqpoint{6.007412in}{2.006587in}}%
\pgfpathlineto{\pgfqpoint{6.007911in}{2.491510in}}%
\pgfpathlineto{\pgfqpoint{6.008078in}{1.908499in}}%
\pgfpathlineto{\pgfqpoint{6.008636in}{2.568666in}}%
\pgfpathlineto{\pgfqpoint{6.009023in}{2.234533in}}%
\pgfpathlineto{\pgfqpoint{6.009902in}{2.564991in}}%
\pgfpathlineto{\pgfqpoint{6.009429in}{1.958690in}}%
\pgfpathlineto{\pgfqpoint{6.010134in}{2.393966in}}%
\pgfpathlineto{\pgfqpoint{6.010845in}{2.569095in}}%
\pgfpathlineto{\pgfqpoint{6.010735in}{1.886422in}}%
\pgfpathlineto{\pgfqpoint{6.011192in}{2.418692in}}%
\pgfpathlineto{\pgfqpoint{6.012055in}{1.988316in}}%
\pgfpathlineto{\pgfqpoint{6.011785in}{2.555224in}}%
\pgfpathlineto{\pgfqpoint{6.012301in}{2.369059in}}%
\pgfpathlineto{\pgfqpoint{6.012306in}{2.566170in}}%
\pgfpathlineto{\pgfqpoint{6.012898in}{1.999820in}}%
\pgfpathlineto{\pgfqpoint{6.013408in}{2.399991in}}%
\pgfpathlineto{\pgfqpoint{6.013458in}{1.821447in}}%
\pgfpathlineto{\pgfqpoint{6.013563in}{2.569564in}}%
\pgfpathlineto{\pgfqpoint{6.014518in}{2.460246in}}%
\pgfpathlineto{\pgfqpoint{6.015329in}{2.021963in}}%
\pgfpathlineto{\pgfqpoint{6.015255in}{2.554818in}}%
\pgfpathlineto{\pgfqpoint{6.015629in}{2.294905in}}%
\pgfpathlineto{\pgfqpoint{6.016120in}{2.542798in}}%
\pgfpathlineto{\pgfqpoint{6.016110in}{1.998890in}}%
\pgfpathlineto{\pgfqpoint{6.016742in}{2.451923in}}%
\pgfpathlineto{\pgfqpoint{6.017301in}{2.035885in}}%
\pgfpathlineto{\pgfqpoint{6.017651in}{2.558068in}}%
\pgfpathlineto{\pgfqpoint{6.017853in}{2.242061in}}%
\pgfpathlineto{\pgfqpoint{6.018233in}{2.564993in}}%
\pgfpathlineto{\pgfqpoint{6.017874in}{1.828738in}}%
\pgfpathlineto{\pgfqpoint{6.018966in}{2.357243in}}%
\pgfpathlineto{\pgfqpoint{6.020045in}{2.556949in}}%
\pgfpathlineto{\pgfqpoint{6.019145in}{2.015945in}}%
\pgfpathlineto{\pgfqpoint{6.020067in}{2.435034in}}%
\pgfpathlineto{\pgfqpoint{6.020070in}{1.926979in}}%
\pgfpathlineto{\pgfqpoint{6.020801in}{2.580234in}}%
\pgfpathlineto{\pgfqpoint{6.021178in}{2.310620in}}%
\pgfpathlineto{\pgfqpoint{6.021811in}{2.585331in}}%
\pgfpathlineto{\pgfqpoint{6.021651in}{1.983003in}}%
\pgfpathlineto{\pgfqpoint{6.022291in}{2.451608in}}%
\pgfpathlineto{\pgfqpoint{6.022764in}{1.757658in}}%
\pgfpathlineto{\pgfqpoint{6.022754in}{2.562196in}}%
\pgfpathlineto{\pgfqpoint{6.023400in}{2.169014in}}%
\pgfpathlineto{\pgfqpoint{6.023884in}{2.562843in}}%
\pgfpathlineto{\pgfqpoint{6.023482in}{1.910345in}}%
\pgfpathlineto{\pgfqpoint{6.024512in}{2.481229in}}%
\pgfpathlineto{\pgfqpoint{6.024625in}{1.956606in}}%
\pgfpathlineto{\pgfqpoint{6.025099in}{2.574853in}}%
\pgfpathlineto{\pgfqpoint{6.025626in}{2.417743in}}%
\pgfpathlineto{\pgfqpoint{6.025840in}{2.575207in}}%
\pgfpathlineto{\pgfqpoint{6.025963in}{1.646128in}}%
\pgfpathlineto{\pgfqpoint{6.026648in}{2.415419in}}%
\pgfpathlineto{\pgfqpoint{6.027700in}{1.905362in}}%
\pgfpathlineto{\pgfqpoint{6.027358in}{2.546870in}}%
\pgfpathlineto{\pgfqpoint{6.027758in}{2.400090in}}%
\pgfpathlineto{\pgfqpoint{6.028392in}{1.928395in}}%
\pgfpathlineto{\pgfqpoint{6.028737in}{2.562308in}}%
\pgfpathlineto{\pgfqpoint{6.028869in}{2.189722in}}%
\pgfpathlineto{\pgfqpoint{6.029963in}{2.550431in}}%
\pgfpathlineto{\pgfqpoint{6.028905in}{1.883183in}}%
\pgfpathlineto{\pgfqpoint{6.029981in}{2.377359in}}%
\pgfpathlineto{\pgfqpoint{6.030183in}{2.556017in}}%
\pgfpathlineto{\pgfqpoint{6.030455in}{1.936613in}}%
\pgfpathlineto{\pgfqpoint{6.031092in}{2.515836in}}%
\pgfpathlineto{\pgfqpoint{6.031246in}{1.783201in}}%
\pgfpathlineto{\pgfqpoint{6.031763in}{2.553055in}}%
\pgfpathlineto{\pgfqpoint{6.032204in}{2.336479in}}%
\pgfpathlineto{\pgfqpoint{6.032267in}{2.576952in}}%
\pgfpathlineto{\pgfqpoint{6.033152in}{2.000119in}}%
\pgfpathlineto{\pgfqpoint{6.033313in}{2.323735in}}%
\pgfpathlineto{\pgfqpoint{6.033743in}{1.981249in}}%
\pgfpathlineto{\pgfqpoint{6.034260in}{2.573881in}}%
\pgfpathlineto{\pgfqpoint{6.034422in}{2.328625in}}%
\pgfpathlineto{\pgfqpoint{6.035517in}{2.574688in}}%
\pgfpathlineto{\pgfqpoint{6.034944in}{2.010578in}}%
\pgfpathlineto{\pgfqpoint{6.035537in}{2.571219in}}%
\pgfpathlineto{\pgfqpoint{6.036362in}{1.999135in}}%
\pgfpathlineto{\pgfqpoint{6.036649in}{2.404685in}}%
\pgfpathlineto{\pgfqpoint{6.036831in}{2.561238in}}%
\pgfpathlineto{\pgfqpoint{6.037679in}{1.790419in}}%
\pgfpathlineto{\pgfqpoint{6.037749in}{2.409080in}}%
\pgfpathlineto{\pgfqpoint{6.037833in}{1.842466in}}%
\pgfpathlineto{\pgfqpoint{6.038597in}{2.571091in}}%
\pgfpathlineto{\pgfqpoint{6.038861in}{2.293595in}}%
\pgfpathlineto{\pgfqpoint{6.039065in}{2.563414in}}%
\pgfpathlineto{\pgfqpoint{6.039163in}{1.995042in}}%
\pgfpathlineto{\pgfqpoint{6.039973in}{2.417447in}}%
\pgfpathlineto{\pgfqpoint{6.040279in}{1.998671in}}%
\pgfpathlineto{\pgfqpoint{6.040967in}{2.587995in}}%
\pgfpathlineto{\pgfqpoint{6.041083in}{2.246838in}}%
\pgfpathlineto{\pgfqpoint{6.041721in}{2.564446in}}%
\pgfpathlineto{\pgfqpoint{6.041770in}{1.971401in}}%
\pgfpathlineto{\pgfqpoint{6.042194in}{2.277177in}}%
\pgfpathlineto{\pgfqpoint{6.042858in}{2.565273in}}%
\pgfpathlineto{\pgfqpoint{6.042247in}{1.789815in}}%
\pgfpathlineto{\pgfqpoint{6.043307in}{2.400606in}}%
\pgfpathlineto{\pgfqpoint{6.043590in}{2.555189in}}%
\pgfpathlineto{\pgfqpoint{6.043684in}{1.920397in}}%
\pgfpathlineto{\pgfqpoint{6.044405in}{2.362830in}}%
\pgfpathlineto{\pgfqpoint{6.045272in}{1.910281in}}%
\pgfpathlineto{\pgfqpoint{6.045410in}{2.546374in}}%
\pgfpathlineto{\pgfqpoint{6.045513in}{2.310935in}}%
\pgfpathlineto{\pgfqpoint{6.045881in}{2.567841in}}%
\pgfpathlineto{\pgfqpoint{6.046321in}{1.894208in}}%
\pgfpathlineto{\pgfqpoint{6.046625in}{2.354436in}}%
\pgfpathlineto{\pgfqpoint{6.046802in}{1.974965in}}%
\pgfpathlineto{\pgfqpoint{6.047413in}{2.566196in}}%
\pgfpathlineto{\pgfqpoint{6.047733in}{2.419724in}}%
\pgfpathlineto{\pgfqpoint{6.047962in}{2.554979in}}%
\pgfpathlineto{\pgfqpoint{6.048141in}{1.846729in}}%
\pgfpathlineto{\pgfqpoint{6.048844in}{2.469054in}}%
\pgfpathlineto{\pgfqpoint{6.049602in}{1.994624in}}%
\pgfpathlineto{\pgfqpoint{6.049769in}{2.578972in}}%
\pgfpathlineto{\pgfqpoint{6.049954in}{2.283596in}}%
\pgfpathlineto{\pgfqpoint{6.050117in}{2.555594in}}%
\pgfpathlineto{\pgfqpoint{6.050231in}{2.026679in}}%
\pgfpathlineto{\pgfqpoint{6.051066in}{2.433810in}}%
\pgfpathlineto{\pgfqpoint{6.051831in}{1.904894in}}%
\pgfpathlineto{\pgfqpoint{6.051147in}{2.558176in}}%
\pgfpathlineto{\pgfqpoint{6.052178in}{2.357639in}}%
\pgfpathlineto{\pgfqpoint{6.052402in}{2.544898in}}%
\pgfpathlineto{\pgfqpoint{6.053010in}{1.938650in}}%
\pgfpathlineto{\pgfqpoint{6.053287in}{2.464697in}}%
\pgfpathlineto{\pgfqpoint{6.053604in}{1.741362in}}%
\pgfpathlineto{\pgfqpoint{6.054033in}{2.551134in}}%
\pgfpathlineto{\pgfqpoint{6.054397in}{2.361408in}}%
\pgfpathlineto{\pgfqpoint{6.054440in}{2.573185in}}%
\pgfpathlineto{\pgfqpoint{6.054601in}{1.986133in}}%
\pgfpathlineto{\pgfqpoint{6.055509in}{2.501833in}}%
\pgfpathlineto{\pgfqpoint{6.056445in}{1.787184in}}%
\pgfpathlineto{\pgfqpoint{6.055909in}{2.558738in}}%
\pgfpathlineto{\pgfqpoint{6.056621in}{2.203023in}}%
\pgfpathlineto{\pgfqpoint{6.057076in}{2.562164in}}%
\pgfpathlineto{\pgfqpoint{6.057298in}{1.760768in}}%
\pgfpathlineto{\pgfqpoint{6.057733in}{2.521434in}}%
\pgfpathlineto{\pgfqpoint{6.058358in}{1.786658in}}%
\pgfpathlineto{\pgfqpoint{6.058068in}{2.571995in}}%
\pgfpathlineto{\pgfqpoint{6.058847in}{2.213248in}}%
\pgfpathlineto{\pgfqpoint{6.059186in}{2.582302in}}%
\pgfpathlineto{\pgfqpoint{6.059735in}{1.906042in}}%
\pgfpathlineto{\pgfqpoint{6.059959in}{2.485197in}}%
\pgfpathlineto{\pgfqpoint{6.060012in}{1.808535in}}%
\pgfpathlineto{\pgfqpoint{6.060769in}{2.537214in}}%
\pgfpathlineto{\pgfqpoint{6.061079in}{2.325053in}}%
\pgfpathlineto{\pgfqpoint{6.061585in}{2.548159in}}%
\pgfpathlineto{\pgfqpoint{6.061188in}{1.915448in}}%
\pgfpathlineto{\pgfqpoint{6.062189in}{2.373296in}}%
\pgfpathlineto{\pgfqpoint{6.063089in}{2.560371in}}%
\pgfpathlineto{\pgfqpoint{6.062336in}{2.029998in}}%
\pgfpathlineto{\pgfqpoint{6.063254in}{2.377792in}}%
\pgfpathlineto{\pgfqpoint{6.063271in}{1.994730in}}%
\pgfpathlineto{\pgfqpoint{6.063774in}{2.559896in}}%
\pgfpathlineto{\pgfqpoint{6.064364in}{2.399935in}}%
\pgfpathlineto{\pgfqpoint{6.064565in}{2.577049in}}%
\pgfpathlineto{\pgfqpoint{6.064853in}{1.991543in}}%
\pgfpathlineto{\pgfqpoint{6.065473in}{2.351905in}}%
\pgfpathlineto{\pgfqpoint{6.065752in}{1.928007in}}%
\pgfpathlineto{\pgfqpoint{6.065732in}{2.566130in}}%
\pgfpathlineto{\pgfqpoint{6.066584in}{2.302216in}}%
\pgfpathlineto{\pgfqpoint{6.066863in}{2.557446in}}%
\pgfpathlineto{\pgfqpoint{6.067522in}{1.735443in}}%
\pgfpathlineto{\pgfqpoint{6.067694in}{2.523428in}}%
\pgfpathlineto{\pgfqpoint{6.068412in}{1.973273in}}%
\pgfpathlineto{\pgfqpoint{6.067733in}{2.555310in}}%
\pgfpathlineto{\pgfqpoint{6.068807in}{2.349912in}}%
\pgfpathlineto{\pgfqpoint{6.069238in}{2.571818in}}%
\pgfpathlineto{\pgfqpoint{6.069508in}{1.722109in}}%
\pgfpathlineto{\pgfqpoint{6.069917in}{2.247812in}}%
\pgfpathlineto{\pgfqpoint{6.070320in}{2.555891in}}%
\pgfpathlineto{\pgfqpoint{6.070288in}{1.766941in}}%
\pgfpathlineto{\pgfqpoint{6.071028in}{2.310511in}}%
\pgfpathlineto{\pgfqpoint{6.072071in}{1.882080in}}%
\pgfpathlineto{\pgfqpoint{6.071372in}{2.565382in}}%
\pgfpathlineto{\pgfqpoint{6.072135in}{2.376623in}}%
\pgfpathlineto{\pgfqpoint{6.072403in}{2.549809in}}%
\pgfpathlineto{\pgfqpoint{6.072982in}{1.942672in}}%
\pgfpathlineto{\pgfqpoint{6.073237in}{2.438464in}}%
\pgfpathlineto{\pgfqpoint{6.073369in}{1.923820in}}%
\pgfpathlineto{\pgfqpoint{6.074049in}{2.584934in}}%
\pgfpathlineto{\pgfqpoint{6.074346in}{2.438304in}}%
\pgfpathlineto{\pgfqpoint{6.075290in}{2.579103in}}%
\pgfpathlineto{\pgfqpoint{6.074543in}{1.809945in}}%
\pgfpathlineto{\pgfqpoint{6.075456in}{2.457188in}}%
\pgfpathlineto{\pgfqpoint{6.076016in}{1.876997in}}%
\pgfpathlineto{\pgfqpoint{6.075550in}{2.551464in}}%
\pgfpathlineto{\pgfqpoint{6.076568in}{2.411414in}}%
\pgfpathlineto{\pgfqpoint{6.077039in}{2.582110in}}%
\pgfpathlineto{\pgfqpoint{6.077141in}{1.934635in}}%
\pgfpathlineto{\pgfqpoint{6.077647in}{2.317572in}}%
\pgfpathlineto{\pgfqpoint{6.078343in}{1.849652in}}%
\pgfpathlineto{\pgfqpoint{6.078371in}{2.578324in}}%
\pgfpathlineto{\pgfqpoint{6.078757in}{2.393068in}}%
\pgfpathlineto{\pgfqpoint{6.079568in}{1.920350in}}%
\pgfpathlineto{\pgfqpoint{6.079808in}{2.568590in}}%
\pgfpathlineto{\pgfqpoint{6.079868in}{2.320697in}}%
\pgfpathlineto{\pgfqpoint{6.080733in}{2.554132in}}%
\pgfpathlineto{\pgfqpoint{6.080066in}{1.987636in}}%
\pgfpathlineto{\pgfqpoint{6.080979in}{2.409932in}}%
\pgfpathlineto{\pgfqpoint{6.081664in}{2.549830in}}%
\pgfpathlineto{\pgfqpoint{6.081192in}{1.989208in}}%
\pgfpathlineto{\pgfqpoint{6.082081in}{2.530425in}}%
\pgfpathlineto{\pgfqpoint{6.082160in}{1.918059in}}%
\pgfpathlineto{\pgfqpoint{6.082737in}{2.558502in}}%
\pgfpathlineto{\pgfqpoint{6.083193in}{2.234161in}}%
\pgfpathlineto{\pgfqpoint{6.083738in}{2.577125in}}%
\pgfpathlineto{\pgfqpoint{6.084275in}{2.018380in}}%
\pgfpathlineto{\pgfqpoint{6.084305in}{2.455821in}}%
\pgfpathlineto{\pgfqpoint{6.084858in}{1.844285in}}%
\pgfpathlineto{\pgfqpoint{6.085119in}{2.554914in}}%
\pgfpathlineto{\pgfqpoint{6.085417in}{2.376606in}}%
\pgfpathlineto{\pgfqpoint{6.085976in}{2.568519in}}%
\pgfpathlineto{\pgfqpoint{6.085870in}{1.834288in}}%
\pgfpathlineto{\pgfqpoint{6.086484in}{2.481482in}}%
\pgfpathlineto{\pgfqpoint{6.087533in}{1.896130in}}%
\pgfpathlineto{\pgfqpoint{6.087500in}{2.557289in}}%
\pgfpathlineto{\pgfqpoint{6.087596in}{2.376343in}}%
\pgfpathlineto{\pgfqpoint{6.088322in}{2.578688in}}%
\pgfpathlineto{\pgfqpoint{6.087712in}{1.948104in}}%
\pgfpathlineto{\pgfqpoint{6.088703in}{2.484031in}}%
\pgfpathlineto{\pgfqpoint{6.088791in}{1.861093in}}%
\pgfpathlineto{\pgfqpoint{6.088779in}{2.561167in}}%
\pgfpathlineto{\pgfqpoint{6.089813in}{2.376941in}}%
\pgfpathlineto{\pgfqpoint{6.090084in}{2.561539in}}%
\pgfpathlineto{\pgfqpoint{6.090844in}{1.896474in}}%
\pgfpathlineto{\pgfqpoint{6.090915in}{2.334582in}}%
\pgfpathlineto{\pgfqpoint{6.090994in}{1.879198in}}%
\pgfpathlineto{\pgfqpoint{6.091350in}{2.598216in}}%
\pgfpathlineto{\pgfqpoint{6.092023in}{2.372371in}}%
\pgfpathlineto{\pgfqpoint{6.092767in}{2.558418in}}%
\pgfpathlineto{\pgfqpoint{6.092565in}{1.957356in}}%
\pgfpathlineto{\pgfqpoint{6.093124in}{2.343765in}}%
\pgfpathlineto{\pgfqpoint{6.093854in}{1.843281in}}%
\pgfpathlineto{\pgfqpoint{6.093987in}{2.577112in}}%
\pgfpathlineto{\pgfqpoint{6.094233in}{2.417153in}}%
\pgfpathlineto{\pgfqpoint{6.095281in}{2.556734in}}%
\pgfpathlineto{\pgfqpoint{6.094839in}{1.972972in}}%
\pgfpathlineto{\pgfqpoint{6.095342in}{2.434943in}}%
\pgfpathlineto{\pgfqpoint{6.095507in}{1.809732in}}%
\pgfpathlineto{\pgfqpoint{6.095444in}{2.550015in}}%
\pgfpathlineto{\pgfqpoint{6.096453in}{2.380287in}}%
\pgfpathlineto{\pgfqpoint{6.096597in}{1.854868in}}%
\pgfpathlineto{\pgfqpoint{6.097150in}{2.563009in}}%
\pgfpathlineto{\pgfqpoint{6.097487in}{2.405171in}}%
\pgfpathlineto{\pgfqpoint{6.097489in}{2.591524in}}%
\pgfpathlineto{\pgfqpoint{6.097711in}{1.968217in}}%
\pgfpathlineto{\pgfqpoint{6.098594in}{2.488537in}}%
\pgfpathlineto{\pgfqpoint{6.099174in}{2.040509in}}%
\pgfpathlineto{\pgfqpoint{6.099295in}{2.570075in}}%
\pgfpathlineto{\pgfqpoint{6.099705in}{2.399527in}}%
\pgfpathlineto{\pgfqpoint{6.100754in}{1.928170in}}%
\pgfpathlineto{\pgfqpoint{6.099767in}{2.557943in}}%
\pgfpathlineto{\pgfqpoint{6.100816in}{2.306334in}}%
\pgfpathlineto{\pgfqpoint{6.101496in}{2.567184in}}%
\pgfpathlineto{\pgfqpoint{6.101862in}{1.785433in}}%
\pgfpathlineto{\pgfqpoint{6.101927in}{2.350875in}}%
\pgfpathlineto{\pgfqpoint{6.102986in}{1.958733in}}%
\pgfpathlineto{\pgfqpoint{6.102680in}{2.555063in}}%
\pgfpathlineto{\pgfqpoint{6.103031in}{2.337602in}}%
\pgfpathlineto{\pgfqpoint{6.103852in}{2.557036in}}%
\pgfpathlineto{\pgfqpoint{6.104080in}{1.974017in}}%
\pgfpathlineto{\pgfqpoint{6.104144in}{2.485458in}}%
\pgfpathlineto{\pgfqpoint{6.104454in}{1.915351in}}%
\pgfpathlineto{\pgfqpoint{6.104692in}{2.542986in}}%
\pgfpathlineto{\pgfqpoint{6.105256in}{2.225853in}}%
\pgfpathlineto{\pgfqpoint{6.105345in}{2.555869in}}%
\pgfpathlineto{\pgfqpoint{6.105393in}{1.875384in}}%
\pgfpathlineto{\pgfqpoint{6.106370in}{2.489274in}}%
\pgfpathlineto{\pgfqpoint{6.106784in}{1.995250in}}%
\pgfpathlineto{\pgfqpoint{6.106665in}{2.563559in}}%
\pgfpathlineto{\pgfqpoint{6.107480in}{2.460796in}}%
\pgfpathlineto{\pgfqpoint{6.108105in}{2.557562in}}%
\pgfpathlineto{\pgfqpoint{6.108307in}{1.888393in}}%
\pgfpathlineto{\pgfqpoint{6.108581in}{2.470724in}}%
\pgfpathlineto{\pgfqpoint{6.108777in}{1.907657in}}%
\pgfpathlineto{\pgfqpoint{6.109127in}{2.560179in}}%
\pgfpathlineto{\pgfqpoint{6.109690in}{2.374918in}}%
\pgfpathlineto{\pgfqpoint{6.110339in}{2.552210in}}%
\pgfpathlineto{\pgfqpoint{6.109755in}{1.656718in}}%
\pgfpathlineto{\pgfqpoint{6.110799in}{2.371209in}}%
\pgfpathlineto{\pgfqpoint{6.110812in}{1.984701in}}%
\pgfpathlineto{\pgfqpoint{6.111275in}{2.558085in}}%
\pgfpathlineto{\pgfqpoint{6.111909in}{2.445776in}}%
\pgfpathlineto{\pgfqpoint{6.112418in}{1.691208in}}%
\pgfpathlineto{\pgfqpoint{6.112107in}{2.574089in}}%
\pgfpathlineto{\pgfqpoint{6.113016in}{2.401616in}}%
\pgfpathlineto{\pgfqpoint{6.113829in}{2.571047in}}%
\pgfpathlineto{\pgfqpoint{6.113067in}{1.950766in}}%
\pgfpathlineto{\pgfqpoint{6.114128in}{2.421667in}}%
\pgfpathlineto{\pgfqpoint{6.114907in}{1.870844in}}%
\pgfpathlineto{\pgfqpoint{6.115068in}{2.568824in}}%
\pgfpathlineto{\pgfqpoint{6.115236in}{2.034767in}}%
\pgfpathlineto{\pgfqpoint{6.115593in}{2.561144in}}%
\pgfpathlineto{\pgfqpoint{6.115367in}{1.894403in}}%
\pgfpathlineto{\pgfqpoint{6.116347in}{2.355155in}}%
\pgfpathlineto{\pgfqpoint{6.116963in}{2.576378in}}%
\pgfpathlineto{\pgfqpoint{6.116948in}{1.878311in}}%
\pgfpathlineto{\pgfqpoint{6.117451in}{2.301826in}}%
\pgfpathlineto{\pgfqpoint{6.118483in}{1.679971in}}%
\pgfpathlineto{\pgfqpoint{6.117918in}{2.570022in}}%
\pgfpathlineto{\pgfqpoint{6.118562in}{2.358004in}}%
\pgfpathlineto{\pgfqpoint{6.119437in}{2.550118in}}%
\pgfpathlineto{\pgfqpoint{6.118741in}{1.561506in}}%
\pgfpathlineto{\pgfqpoint{6.119646in}{2.493352in}}%
\pgfpathlineto{\pgfqpoint{6.120099in}{1.716413in}}%
\pgfpathlineto{\pgfqpoint{6.120287in}{2.570334in}}%
\pgfpathlineto{\pgfqpoint{6.120757in}{2.153723in}}%
\pgfpathlineto{\pgfqpoint{6.121176in}{2.548736in}}%
\pgfpathlineto{\pgfqpoint{6.121323in}{2.036197in}}%
\pgfpathlineto{\pgfqpoint{6.121869in}{2.382171in}}%
\pgfpathlineto{\pgfqpoint{6.122708in}{1.915349in}}%
\pgfpathlineto{\pgfqpoint{6.122464in}{2.560217in}}%
\pgfpathlineto{\pgfqpoint{6.122980in}{2.299959in}}%
\pgfpathlineto{\pgfqpoint{6.123074in}{2.547992in}}%
\pgfpathlineto{\pgfqpoint{6.123734in}{1.815782in}}%
\pgfpathlineto{\pgfqpoint{6.124094in}{2.492973in}}%
\pgfpathlineto{\pgfqpoint{6.125103in}{1.659120in}}%
\pgfpathlineto{\pgfqpoint{6.124463in}{2.551314in}}%
\pgfpathlineto{\pgfqpoint{6.125205in}{2.398447in}}%
\pgfpathlineto{\pgfqpoint{6.125876in}{2.573541in}}%
\pgfpathlineto{\pgfqpoint{6.125620in}{1.717855in}}%
\pgfpathlineto{\pgfqpoint{6.126313in}{2.412534in}}%
\pgfpathlineto{\pgfqpoint{6.126444in}{1.870284in}}%
\pgfpathlineto{\pgfqpoint{6.126740in}{2.553530in}}%
\pgfpathlineto{\pgfqpoint{6.127424in}{2.265040in}}%
\pgfpathlineto{\pgfqpoint{6.128101in}{2.550668in}}%
\pgfpathlineto{\pgfqpoint{6.128106in}{1.750152in}}%
\pgfpathlineto{\pgfqpoint{6.128535in}{2.240206in}}%
\pgfpathlineto{\pgfqpoint{6.128552in}{2.550776in}}%
\pgfpathlineto{\pgfqpoint{6.129159in}{1.661582in}}%
\pgfpathlineto{\pgfqpoint{6.129647in}{2.495815in}}%
\pgfpathlineto{\pgfqpoint{6.130389in}{1.866227in}}%
\pgfpathlineto{\pgfqpoint{6.129763in}{2.563585in}}%
\pgfpathlineto{\pgfqpoint{6.130757in}{2.417194in}}%
\pgfpathlineto{\pgfqpoint{6.130844in}{2.566751in}}%
\pgfpathlineto{\pgfqpoint{6.131707in}{1.888172in}}%
\pgfpathlineto{\pgfqpoint{6.131845in}{2.355529in}}%
\pgfpathlineto{\pgfqpoint{6.132621in}{1.805104in}}%
\pgfpathlineto{\pgfqpoint{6.132463in}{2.554831in}}%
\pgfpathlineto{\pgfqpoint{6.132955in}{2.357352in}}%
\pgfpathlineto{\pgfqpoint{6.133808in}{2.560141in}}%
\pgfpathlineto{\pgfqpoint{6.133800in}{1.862646in}}%
\pgfpathlineto{\pgfqpoint{6.134066in}{2.439703in}}%
\pgfpathlineto{\pgfqpoint{6.134204in}{1.928461in}}%
\pgfpathlineto{\pgfqpoint{6.134877in}{2.559992in}}%
\pgfpathlineto{\pgfqpoint{6.135175in}{2.458113in}}%
\pgfpathlineto{\pgfqpoint{6.136034in}{2.557024in}}%
\pgfpathlineto{\pgfqpoint{6.135698in}{1.931969in}}%
\pgfpathlineto{\pgfqpoint{6.136269in}{2.407062in}}%
\pgfpathlineto{\pgfqpoint{6.136299in}{1.831074in}}%
\pgfpathlineto{\pgfqpoint{6.137009in}{2.576972in}}%
\pgfpathlineto{\pgfqpoint{6.137378in}{2.492133in}}%
\pgfpathlineto{\pgfqpoint{6.137380in}{2.492150in}}%
\pgfpathlineto{\pgfqpoint{6.137381in}{2.489329in}}%
\pgfpathlineto{\pgfqpoint{6.138172in}{1.732873in}}%
\pgfpathlineto{\pgfqpoint{6.137520in}{2.568115in}}%
\pgfpathlineto{\pgfqpoint{6.138492in}{2.391163in}}%
\pgfpathlineto{\pgfqpoint{6.139165in}{2.541963in}}%
\pgfpathlineto{\pgfqpoint{6.138827in}{1.717987in}}%
\pgfpathlineto{\pgfqpoint{6.139330in}{2.303782in}}%
\pgfpathlineto{\pgfqpoint{6.139643in}{1.981689in}}%
\pgfpathlineto{\pgfqpoint{6.139434in}{2.564670in}}%
\pgfpathlineto{\pgfqpoint{6.140440in}{2.256141in}}%
\pgfpathlineto{\pgfqpoint{6.140688in}{2.554541in}}%
\pgfpathlineto{\pgfqpoint{6.140446in}{1.999762in}}%
\pgfpathlineto{\pgfqpoint{6.141552in}{2.507926in}}%
\pgfpathlineto{\pgfqpoint{6.141991in}{1.972290in}}%
\pgfpathlineto{\pgfqpoint{6.141721in}{2.561889in}}%
\pgfpathlineto{\pgfqpoint{6.142662in}{2.439739in}}%
\pgfpathlineto{\pgfqpoint{6.143055in}{2.561925in}}%
\pgfpathlineto{\pgfqpoint{6.143317in}{1.842834in}}%
\pgfpathlineto{\pgfqpoint{6.143771in}{2.462508in}}%
\pgfpathlineto{\pgfqpoint{6.144561in}{1.877273in}}%
\pgfpathlineto{\pgfqpoint{6.144389in}{2.588643in}}%
\pgfpathlineto{\pgfqpoint{6.144882in}{2.383515in}}%
\pgfpathlineto{\pgfqpoint{6.145392in}{2.579349in}}%
\pgfpathlineto{\pgfqpoint{6.145386in}{1.904248in}}%
\pgfpathlineto{\pgfqpoint{6.145993in}{2.462487in}}%
\pgfpathlineto{\pgfqpoint{6.146016in}{1.805822in}}%
\pgfpathlineto{\pgfqpoint{6.146436in}{2.559553in}}%
\pgfpathlineto{\pgfqpoint{6.147104in}{2.416269in}}%
\pgfpathlineto{\pgfqpoint{6.147228in}{2.568957in}}%
\pgfpathlineto{\pgfqpoint{6.147386in}{1.986857in}}%
\pgfpathlineto{\pgfqpoint{6.148215in}{2.451107in}}%
\pgfpathlineto{\pgfqpoint{6.149289in}{1.841022in}}%
\pgfpathlineto{\pgfqpoint{6.148592in}{2.556132in}}%
\pgfpathlineto{\pgfqpoint{6.149326in}{2.426453in}}%
\pgfpathlineto{\pgfqpoint{6.149914in}{1.927587in}}%
\pgfpathlineto{\pgfqpoint{6.149469in}{2.579019in}}%
\pgfpathlineto{\pgfqpoint{6.150438in}{2.396011in}}%
\pgfpathlineto{\pgfqpoint{6.150937in}{2.554868in}}%
\pgfpathlineto{\pgfqpoint{6.151325in}{1.857153in}}%
\pgfpathlineto{\pgfqpoint{6.151534in}{2.458242in}}%
\pgfpathlineto{\pgfqpoint{6.151536in}{1.915932in}}%
\pgfpathlineto{\pgfqpoint{6.151679in}{2.576604in}}%
\pgfpathlineto{\pgfqpoint{6.152644in}{2.389454in}}%
\pgfpathlineto{\pgfqpoint{6.153390in}{2.574594in}}%
\pgfpathlineto{\pgfqpoint{6.153450in}{1.879027in}}%
\pgfpathlineto{\pgfqpoint{6.153754in}{2.419864in}}%
\pgfpathlineto{\pgfqpoint{6.154507in}{1.775616in}}%
\pgfpathlineto{\pgfqpoint{6.154764in}{2.561242in}}%
\pgfpathlineto{\pgfqpoint{6.154848in}{2.275228in}}%
\pgfpathlineto{\pgfqpoint{6.155398in}{2.553780in}}%
\pgfpathlineto{\pgfqpoint{6.155407in}{2.000714in}}%
\pgfpathlineto{\pgfqpoint{6.155960in}{2.442754in}}%
\pgfpathlineto{\pgfqpoint{6.155985in}{1.765561in}}%
\pgfpathlineto{\pgfqpoint{6.156753in}{2.569769in}}%
\pgfpathlineto{\pgfqpoint{6.157073in}{2.346984in}}%
\pgfpathlineto{\pgfqpoint{6.157419in}{2.572569in}}%
\pgfpathlineto{\pgfqpoint{6.157498in}{1.655788in}}%
\pgfpathlineto{\pgfqpoint{6.158185in}{2.412771in}}%
\pgfpathlineto{\pgfqpoint{6.158870in}{1.911919in}}%
\pgfpathlineto{\pgfqpoint{6.158248in}{2.557872in}}%
\pgfpathlineto{\pgfqpoint{6.159297in}{2.273430in}}%
\pgfpathlineto{\pgfqpoint{6.159553in}{2.554004in}}%
\pgfpathlineto{\pgfqpoint{6.159951in}{1.843988in}}%
\pgfpathlineto{\pgfqpoint{6.160407in}{2.313278in}}%
\pgfpathlineto{\pgfqpoint{6.160731in}{1.862634in}}%
\pgfpathlineto{\pgfqpoint{6.160690in}{2.566506in}}%
\pgfpathlineto{\pgfqpoint{6.161506in}{2.240642in}}%
\pgfpathlineto{\pgfqpoint{6.161578in}{2.561210in}}%
\pgfpathlineto{\pgfqpoint{6.162520in}{1.907522in}}%
\pgfpathlineto{\pgfqpoint{6.162616in}{2.489852in}}%
\pgfpathlineto{\pgfqpoint{6.163029in}{1.863194in}}%
\pgfpathlineto{\pgfqpoint{6.162859in}{2.575517in}}%
\pgfpathlineto{\pgfqpoint{6.163726in}{2.459543in}}%
\pgfpathlineto{\pgfqpoint{6.164135in}{1.865264in}}%
\pgfpathlineto{\pgfqpoint{6.164617in}{2.549904in}}%
\pgfpathlineto{\pgfqpoint{6.164840in}{2.376191in}}%
\pgfpathlineto{\pgfqpoint{6.165812in}{2.561806in}}%
\pgfpathlineto{\pgfqpoint{6.164963in}{1.923764in}}%
\pgfpathlineto{\pgfqpoint{6.165950in}{2.379201in}}%
\pgfpathlineto{\pgfqpoint{6.166267in}{2.576182in}}%
\pgfpathlineto{\pgfqpoint{6.165963in}{1.824703in}}%
\pgfpathlineto{\pgfqpoint{6.167013in}{2.432525in}}%
\pgfpathlineto{\pgfqpoint{6.167895in}{1.719715in}}%
\pgfpathlineto{\pgfqpoint{6.167125in}{2.552923in}}%
\pgfpathlineto{\pgfqpoint{6.168124in}{2.430738in}}%
\pgfpathlineto{\pgfqpoint{6.169141in}{2.572947in}}%
\pgfpathlineto{\pgfqpoint{6.168889in}{1.924299in}}%
\pgfpathlineto{\pgfqpoint{6.169180in}{2.288628in}}%
\pgfpathlineto{\pgfqpoint{6.170046in}{2.046494in}}%
\pgfpathlineto{\pgfqpoint{6.169238in}{2.551113in}}%
\pgfpathlineto{\pgfqpoint{6.170290in}{2.206408in}}%
\pgfpathlineto{\pgfqpoint{6.170652in}{2.555721in}}%
\pgfpathlineto{\pgfqpoint{6.170960in}{2.061513in}}%
\pgfpathlineto{\pgfqpoint{6.171401in}{2.223010in}}%
\pgfpathlineto{\pgfqpoint{6.172475in}{1.861594in}}%
\pgfpathlineto{\pgfqpoint{6.172267in}{2.569046in}}%
\pgfpathlineto{\pgfqpoint{6.172508in}{2.333671in}}%
\pgfpathlineto{\pgfqpoint{6.173294in}{2.562558in}}%
\pgfpathlineto{\pgfqpoint{6.173158in}{2.028151in}}%
\pgfpathlineto{\pgfqpoint{6.173617in}{2.341481in}}%
\pgfpathlineto{\pgfqpoint{6.174573in}{1.873242in}}%
\pgfpathlineto{\pgfqpoint{6.174241in}{2.551334in}}%
\pgfpathlineto{\pgfqpoint{6.174725in}{2.454990in}}%
\pgfpathlineto{\pgfqpoint{6.175753in}{1.853740in}}%
\pgfpathlineto{\pgfqpoint{6.175557in}{2.534263in}}%
\pgfpathlineto{\pgfqpoint{6.175835in}{2.258847in}}%
\pgfpathlineto{\pgfqpoint{6.175938in}{2.574135in}}%
\pgfpathlineto{\pgfqpoint{6.176512in}{1.939304in}}%
\pgfpathlineto{\pgfqpoint{6.176945in}{2.337106in}}%
\pgfpathlineto{\pgfqpoint{6.177223in}{1.970887in}}%
\pgfpathlineto{\pgfqpoint{6.177744in}{2.550297in}}%
\pgfpathlineto{\pgfqpoint{6.178054in}{2.451007in}}%
\pgfpathlineto{\pgfqpoint{6.178208in}{2.571461in}}%
\pgfpathlineto{\pgfqpoint{6.178863in}{1.877367in}}%
\pgfpathlineto{\pgfqpoint{6.179151in}{2.309578in}}%
\pgfpathlineto{\pgfqpoint{6.179979in}{1.630331in}}%
\pgfpathlineto{\pgfqpoint{6.179973in}{2.551158in}}%
\pgfpathlineto{\pgfqpoint{6.180259in}{2.344704in}}%
\pgfpathlineto{\pgfqpoint{6.180970in}{2.565200in}}%
\pgfpathlineto{\pgfqpoint{6.181199in}{1.836361in}}%
\pgfpathlineto{\pgfqpoint{6.181369in}{2.426064in}}%
\pgfpathlineto{\pgfqpoint{6.182314in}{1.941195in}}%
\pgfpathlineto{\pgfqpoint{6.181826in}{2.562805in}}%
\pgfpathlineto{\pgfqpoint{6.182480in}{2.335895in}}%
\pgfpathlineto{\pgfqpoint{6.183530in}{2.559630in}}%
\pgfpathlineto{\pgfqpoint{6.182781in}{1.896629in}}%
\pgfpathlineto{\pgfqpoint{6.183591in}{2.386787in}}%
\pgfpathlineto{\pgfqpoint{6.183783in}{1.990826in}}%
\pgfpathlineto{\pgfqpoint{6.183597in}{2.559081in}}%
\pgfpathlineto{\pgfqpoint{6.184615in}{2.334295in}}%
\pgfpathlineto{\pgfqpoint{6.185420in}{2.574598in}}%
\pgfpathlineto{\pgfqpoint{6.184918in}{1.932470in}}%
\pgfpathlineto{\pgfqpoint{6.185726in}{2.325547in}}%
\pgfpathlineto{\pgfqpoint{6.185979in}{2.579869in}}%
\pgfpathlineto{\pgfqpoint{6.186327in}{1.800232in}}%
\pgfpathlineto{\pgfqpoint{6.186838in}{2.449593in}}%
\pgfpathlineto{\pgfqpoint{6.187722in}{2.031320in}}%
\pgfpathlineto{\pgfqpoint{6.187085in}{2.586951in}}%
\pgfpathlineto{\pgfqpoint{6.187948in}{2.236850in}}%
\pgfpathlineto{\pgfqpoint{6.188843in}{2.551344in}}%
\pgfpathlineto{\pgfqpoint{6.189052in}{1.813208in}}%
\pgfpathlineto{\pgfqpoint{6.189059in}{2.516327in}}%
\pgfpathlineto{\pgfqpoint{6.190140in}{1.868730in}}%
\pgfpathlineto{\pgfqpoint{6.189977in}{2.577501in}}%
\pgfpathlineto{\pgfqpoint{6.190170in}{2.367265in}}%
\pgfpathlineto{\pgfqpoint{6.190491in}{2.569154in}}%
\pgfpathlineto{\pgfqpoint{6.190938in}{1.855351in}}%
\pgfpathlineto{\pgfqpoint{6.191283in}{2.508688in}}%
\pgfpathlineto{\pgfqpoint{6.191474in}{1.749084in}}%
\pgfpathlineto{\pgfqpoint{6.191759in}{2.567277in}}%
\pgfpathlineto{\pgfqpoint{6.192394in}{2.294386in}}%
\pgfpathlineto{\pgfqpoint{6.193201in}{2.552397in}}%
\pgfpathlineto{\pgfqpoint{6.192431in}{1.930734in}}%
\pgfpathlineto{\pgfqpoint{6.193506in}{2.463101in}}%
\pgfpathlineto{\pgfqpoint{6.193573in}{1.954581in}}%
\pgfpathlineto{\pgfqpoint{6.194040in}{2.589298in}}%
\pgfpathlineto{\pgfqpoint{6.194613in}{2.437366in}}%
\pgfpathlineto{\pgfqpoint{6.195266in}{2.551967in}}%
\pgfpathlineto{\pgfqpoint{6.195623in}{1.864856in}}%
\pgfpathlineto{\pgfqpoint{6.195719in}{2.364504in}}%
\pgfpathlineto{\pgfqpoint{6.196756in}{1.988722in}}%
\pgfpathlineto{\pgfqpoint{6.196084in}{2.581209in}}%
\pgfpathlineto{\pgfqpoint{6.196830in}{2.511110in}}%
\pgfpathlineto{\pgfqpoint{6.197543in}{1.788946in}}%
\pgfpathlineto{\pgfqpoint{6.197561in}{2.564938in}}%
\pgfpathlineto{\pgfqpoint{6.197941in}{2.331343in}}%
\pgfpathlineto{\pgfqpoint{6.198243in}{2.553691in}}%
\pgfpathlineto{\pgfqpoint{6.198056in}{1.791336in}}%
\pgfpathlineto{\pgfqpoint{6.199051in}{2.474561in}}%
\pgfpathlineto{\pgfqpoint{6.199800in}{1.889466in}}%
\pgfpathlineto{\pgfqpoint{6.199295in}{2.580486in}}%
\pgfpathlineto{\pgfqpoint{6.200163in}{2.212061in}}%
\pgfpathlineto{\pgfqpoint{6.201104in}{2.553990in}}%
\pgfpathlineto{\pgfqpoint{6.200525in}{1.866216in}}%
\pgfpathlineto{\pgfqpoint{6.201276in}{2.363736in}}%
\pgfpathlineto{\pgfqpoint{6.201847in}{1.734061in}}%
\pgfpathlineto{\pgfqpoint{6.201979in}{2.575567in}}%
\pgfpathlineto{\pgfqpoint{6.202383in}{2.296196in}}%
\pgfpathlineto{\pgfqpoint{6.203210in}{2.572576in}}%
\pgfpathlineto{\pgfqpoint{6.202725in}{1.925457in}}%
\pgfpathlineto{\pgfqpoint{6.203495in}{2.373303in}}%
\pgfpathlineto{\pgfqpoint{6.204571in}{1.814576in}}%
\pgfpathlineto{\pgfqpoint{6.203697in}{2.577973in}}%
\pgfpathlineto{\pgfqpoint{6.204606in}{2.122027in}}%
\pgfpathlineto{\pgfqpoint{6.204940in}{2.570081in}}%
\pgfpathlineto{\pgfqpoint{6.205181in}{1.898888in}}%
\pgfpathlineto{\pgfqpoint{6.205717in}{2.154570in}}%
\pgfpathlineto{\pgfqpoint{6.206129in}{2.554948in}}%
\pgfpathlineto{\pgfqpoint{6.206799in}{1.985545in}}%
\pgfpathlineto{\pgfqpoint{6.206829in}{2.359203in}}%
\pgfpathlineto{\pgfqpoint{6.207107in}{1.987986in}}%
\pgfpathlineto{\pgfqpoint{6.207646in}{2.582532in}}%
\pgfpathlineto{\pgfqpoint{6.207938in}{2.275947in}}%
\pgfpathlineto{\pgfqpoint{6.208565in}{2.574142in}}%
\pgfpathlineto{\pgfqpoint{6.208479in}{1.497886in}}%
\pgfpathlineto{\pgfqpoint{6.209050in}{2.432481in}}%
\pgfpathlineto{\pgfqpoint{6.209123in}{1.832623in}}%
\pgfpathlineto{\pgfqpoint{6.209550in}{2.585073in}}%
\pgfpathlineto{\pgfqpoint{6.210162in}{2.229479in}}%
\pgfpathlineto{\pgfqpoint{6.211204in}{2.569753in}}%
\pgfpathlineto{\pgfqpoint{6.210319in}{1.940234in}}%
\pgfpathlineto{\pgfqpoint{6.211273in}{2.169218in}}%
\pgfpathlineto{\pgfqpoint{6.211498in}{2.556547in}}%
\pgfpathlineto{\pgfqpoint{6.211957in}{1.987802in}}%
\pgfpathlineto{\pgfqpoint{6.212386in}{2.510043in}}%
\pgfpathlineto{\pgfqpoint{6.212779in}{1.888386in}}%
\pgfpathlineto{\pgfqpoint{6.212794in}{2.574479in}}%
\pgfpathlineto{\pgfqpoint{6.213497in}{2.225915in}}%
\pgfpathlineto{\pgfqpoint{6.213970in}{2.599153in}}%
\pgfpathlineto{\pgfqpoint{6.214192in}{1.998312in}}%
\pgfpathlineto{\pgfqpoint{6.214610in}{2.378018in}}%
\pgfpathlineto{\pgfqpoint{6.214862in}{1.834506in}}%
\pgfpathlineto{\pgfqpoint{6.215135in}{2.555737in}}%
\pgfpathlineto{\pgfqpoint{6.215721in}{2.372299in}}%
\pgfpathlineto{\pgfqpoint{6.216667in}{2.565598in}}%
\pgfpathlineto{\pgfqpoint{6.215737in}{1.850702in}}%
\pgfpathlineto{\pgfqpoint{6.216831in}{2.445635in}}%
\pgfpathlineto{\pgfqpoint{6.217190in}{1.897203in}}%
\pgfpathlineto{\pgfqpoint{6.217235in}{2.572254in}}%
\pgfpathlineto{\pgfqpoint{6.217942in}{2.391033in}}%
\pgfpathlineto{\pgfqpoint{6.218521in}{2.571289in}}%
\pgfpathlineto{\pgfqpoint{6.218856in}{1.952344in}}%
\pgfpathlineto{\pgfqpoint{6.219050in}{2.420079in}}%
\pgfpathlineto{\pgfqpoint{6.219332in}{1.876570in}}%
\pgfpathlineto{\pgfqpoint{6.220118in}{2.567566in}}%
\pgfpathlineto{\pgfqpoint{6.220162in}{2.229871in}}%
\pgfpathlineto{\pgfqpoint{6.220876in}{2.574600in}}%
\pgfpathlineto{\pgfqpoint{6.221215in}{1.920582in}}%
\pgfpathlineto{\pgfqpoint{6.221272in}{2.249661in}}%
\pgfpathlineto{\pgfqpoint{6.221854in}{1.811498in}}%
\pgfpathlineto{\pgfqpoint{6.221274in}{2.566815in}}%
\pgfpathlineto{\pgfqpoint{6.222358in}{2.373155in}}%
\pgfpathlineto{\pgfqpoint{6.223348in}{2.572638in}}%
\pgfpathlineto{\pgfqpoint{6.223010in}{1.897766in}}%
\pgfpathlineto{\pgfqpoint{6.223470in}{2.479162in}}%
\pgfpathlineto{\pgfqpoint{6.223631in}{2.549522in}}%
\pgfpathlineto{\pgfqpoint{6.223544in}{1.943545in}}%
\pgfpathlineto{\pgfqpoint{6.224056in}{2.401352in}}%
\pgfpathlineto{\pgfqpoint{6.224058in}{1.918195in}}%
\pgfpathlineto{\pgfqpoint{6.224433in}{2.570549in}}%
\pgfpathlineto{\pgfqpoint{6.225167in}{2.504113in}}%
\pgfpathlineto{\pgfqpoint{6.225254in}{1.975345in}}%
\pgfpathlineto{\pgfqpoint{6.225709in}{2.561055in}}%
\pgfpathlineto{\pgfqpoint{6.226280in}{2.288927in}}%
\pgfpathlineto{\pgfqpoint{6.226768in}{2.549825in}}%
\pgfpathlineto{\pgfqpoint{6.227212in}{1.910190in}}%
\pgfpathlineto{\pgfqpoint{6.227391in}{2.434182in}}%
\pgfpathlineto{\pgfqpoint{6.227641in}{1.808162in}}%
\pgfpathlineto{\pgfqpoint{6.227891in}{2.564552in}}%
\pgfpathlineto{\pgfqpoint{6.228502in}{2.275317in}}%
\pgfpathlineto{\pgfqpoint{6.228851in}{2.558292in}}%
\pgfpathlineto{\pgfqpoint{6.229387in}{1.835068in}}%
\pgfpathlineto{\pgfqpoint{6.229613in}{2.270382in}}%
\pgfpathlineto{\pgfqpoint{6.230698in}{2.590782in}}%
\pgfpathlineto{\pgfqpoint{6.230339in}{1.984292in}}%
\pgfpathlineto{\pgfqpoint{6.230726in}{2.492986in}}%
\pgfpathlineto{\pgfqpoint{6.231511in}{1.961255in}}%
\pgfpathlineto{\pgfqpoint{6.231496in}{2.545245in}}%
\pgfpathlineto{\pgfqpoint{6.231839in}{2.020550in}}%
\pgfpathlineto{\pgfqpoint{6.232071in}{2.561583in}}%
\pgfpathlineto{\pgfqpoint{6.232240in}{1.974952in}}%
\pgfpathlineto{\pgfqpoint{6.232951in}{2.390697in}}%
\pgfpathlineto{\pgfqpoint{6.233173in}{2.570081in}}%
\pgfpathlineto{\pgfqpoint{6.233191in}{1.821198in}}%
\pgfpathlineto{\pgfqpoint{6.234062in}{2.434781in}}%
\pgfpathlineto{\pgfqpoint{6.234554in}{1.798204in}}%
\pgfpathlineto{\pgfqpoint{6.234352in}{2.554927in}}%
\pgfpathlineto{\pgfqpoint{6.235173in}{2.405878in}}%
\pgfpathlineto{\pgfqpoint{6.235750in}{2.582261in}}%
\pgfpathlineto{\pgfqpoint{6.236233in}{1.948263in}}%
\pgfpathlineto{\pgfqpoint{6.236283in}{2.361219in}}%
\pgfpathlineto{\pgfqpoint{6.236992in}{2.015966in}}%
\pgfpathlineto{\pgfqpoint{6.236957in}{2.559719in}}%
\pgfpathlineto{\pgfqpoint{6.237375in}{2.254152in}}%
\pgfpathlineto{\pgfqpoint{6.237392in}{2.574172in}}%
\pgfpathlineto{\pgfqpoint{6.237608in}{1.913907in}}%
\pgfpathlineto{\pgfqpoint{6.238486in}{2.379133in}}%
\pgfpathlineto{\pgfqpoint{6.238633in}{2.560688in}}%
\pgfpathlineto{\pgfqpoint{6.238804in}{1.940090in}}%
\pgfpathlineto{\pgfqpoint{6.239511in}{2.461803in}}%
\pgfpathlineto{\pgfqpoint{6.240553in}{1.964568in}}%
\pgfpathlineto{\pgfqpoint{6.239707in}{2.576050in}}%
\pgfpathlineto{\pgfqpoint{6.240621in}{2.505157in}}%
\pgfpathlineto{\pgfqpoint{6.241235in}{1.921790in}}%
\pgfpathlineto{\pgfqpoint{6.241007in}{2.569464in}}%
\pgfpathlineto{\pgfqpoint{6.241732in}{2.384459in}}%
\pgfpathlineto{\pgfqpoint{6.242389in}{2.547128in}}%
\pgfpathlineto{\pgfqpoint{6.242708in}{1.943389in}}%
\pgfpathlineto{\pgfqpoint{6.242842in}{2.421692in}}%
\pgfpathlineto{\pgfqpoint{6.243425in}{1.907300in}}%
\pgfpathlineto{\pgfqpoint{6.243568in}{2.579673in}}%
\pgfpathlineto{\pgfqpoint{6.243953in}{2.276524in}}%
\pgfpathlineto{\pgfqpoint{6.244309in}{2.583519in}}%
\pgfpathlineto{\pgfqpoint{6.244017in}{1.830177in}}%
\pgfpathlineto{\pgfqpoint{6.245065in}{2.511288in}}%
\pgfpathlineto{\pgfqpoint{6.245570in}{1.939304in}}%
\pgfpathlineto{\pgfqpoint{6.246022in}{2.574583in}}%
\pgfpathlineto{\pgfqpoint{6.246177in}{2.355820in}}%
\pgfpathlineto{\pgfqpoint{6.247128in}{2.549662in}}%
\pgfpathlineto{\pgfqpoint{6.246464in}{1.978130in}}%
\pgfpathlineto{\pgfqpoint{6.247283in}{2.403964in}}%
\pgfpathlineto{\pgfqpoint{6.248231in}{1.885414in}}%
\pgfpathlineto{\pgfqpoint{6.247601in}{2.555420in}}%
\pgfpathlineto{\pgfqpoint{6.248394in}{2.331526in}}%
\pgfpathlineto{\pgfqpoint{6.249167in}{2.568954in}}%
\pgfpathlineto{\pgfqpoint{6.249225in}{1.930182in}}%
\pgfpathlineto{\pgfqpoint{6.249506in}{2.454883in}}%
\pgfpathlineto{\pgfqpoint{6.250044in}{1.896939in}}%
\pgfpathlineto{\pgfqpoint{6.250085in}{2.578026in}}%
\pgfpathlineto{\pgfqpoint{6.250617in}{2.445253in}}%
\pgfpathlineto{\pgfqpoint{6.250625in}{2.525452in}}%
\pgfpathlineto{\pgfqpoint{6.250627in}{2.274432in}}%
\pgfpathlineto{\pgfqpoint{6.250632in}{2.385944in}}%
\pgfpathlineto{\pgfqpoint{6.251038in}{1.921234in}}%
\pgfpathlineto{\pgfqpoint{6.250848in}{2.561039in}}%
\pgfpathlineto{\pgfqpoint{6.251742in}{2.259020in}}%
\pgfpathlineto{\pgfqpoint{6.251790in}{2.548856in}}%
\pgfpathlineto{\pgfqpoint{6.252356in}{1.937968in}}%
\pgfpathlineto{\pgfqpoint{6.252854in}{2.405471in}}%
\pgfpathlineto{\pgfqpoint{6.253198in}{2.560553in}}%
\pgfpathlineto{\pgfqpoint{6.253078in}{1.844101in}}%
\pgfpathlineto{\pgfqpoint{6.253955in}{2.434702in}}%
\pgfpathlineto{\pgfqpoint{6.254432in}{1.946267in}}%
\pgfpathlineto{\pgfqpoint{6.254061in}{2.569352in}}%
\pgfpathlineto{\pgfqpoint{6.255065in}{2.504086in}}%
\pgfpathlineto{\pgfqpoint{6.255758in}{1.976670in}}%
\pgfpathlineto{\pgfqpoint{6.255473in}{2.569388in}}%
\pgfpathlineto{\pgfqpoint{6.256175in}{2.518663in}}%
\pgfpathlineto{\pgfqpoint{6.256823in}{1.838017in}}%
\pgfpathlineto{\pgfqpoint{6.256639in}{2.550562in}}%
\pgfpathlineto{\pgfqpoint{6.257290in}{2.392821in}}%
\pgfpathlineto{\pgfqpoint{6.258343in}{2.581049in}}%
\pgfpathlineto{\pgfqpoint{6.257597in}{1.888121in}}%
\pgfpathlineto{\pgfqpoint{6.258401in}{2.464003in}}%
\pgfpathlineto{\pgfqpoint{6.259292in}{1.680282in}}%
\pgfpathlineto{\pgfqpoint{6.258654in}{2.573859in}}%
\pgfpathlineto{\pgfqpoint{6.259512in}{2.375075in}}%
\pgfpathlineto{\pgfqpoint{6.259692in}{2.551926in}}%
\pgfpathlineto{\pgfqpoint{6.260525in}{1.882754in}}%
\pgfpathlineto{\pgfqpoint{6.260621in}{2.493499in}}%
\pgfpathlineto{\pgfqpoint{6.261450in}{1.908289in}}%
\pgfpathlineto{\pgfqpoint{6.261383in}{2.537791in}}%
\pgfpathlineto{\pgfqpoint{6.261732in}{2.402216in}}%
\pgfpathlineto{\pgfqpoint{6.262003in}{2.554442in}}%
\pgfpathlineto{\pgfqpoint{6.262788in}{1.874089in}}%
\pgfpathlineto{\pgfqpoint{6.262835in}{2.385596in}}%
\pgfpathlineto{\pgfqpoint{6.263942in}{1.918045in}}%
\pgfpathlineto{\pgfqpoint{6.263013in}{2.558455in}}%
\pgfpathlineto{\pgfqpoint{6.263944in}{2.316087in}}%
\pgfpathlineto{\pgfqpoint{6.264608in}{2.575330in}}%
\pgfpathlineto{\pgfqpoint{6.264083in}{1.949696in}}%
\pgfpathlineto{\pgfqpoint{6.265054in}{2.529341in}}%
\pgfpathlineto{\pgfqpoint{6.265331in}{1.597629in}}%
\pgfpathlineto{\pgfqpoint{6.265849in}{2.572800in}}%
\pgfpathlineto{\pgfqpoint{6.266165in}{2.391800in}}%
\pgfpathlineto{\pgfqpoint{6.266786in}{2.584684in}}%
\pgfpathlineto{\pgfqpoint{6.267077in}{1.819097in}}%
\pgfpathlineto{\pgfqpoint{6.267277in}{2.465645in}}%
\pgfpathlineto{\pgfqpoint{6.267543in}{1.485172in}}%
\pgfpathlineto{\pgfqpoint{6.267437in}{2.563790in}}%
\pgfpathlineto{\pgfqpoint{6.268389in}{2.346361in}}%
\pgfpathlineto{\pgfqpoint{6.268560in}{2.572075in}}%
\pgfpathlineto{\pgfqpoint{6.268764in}{1.852030in}}%
\pgfpathlineto{\pgfqpoint{6.269500in}{2.482979in}}%
\pgfpathlineto{\pgfqpoint{6.269586in}{1.928689in}}%
\pgfpathlineto{\pgfqpoint{6.269836in}{2.561175in}}%
\pgfpathlineto{\pgfqpoint{6.270610in}{2.480949in}}%
\pgfpathlineto{\pgfqpoint{6.271520in}{2.570130in}}%
\pgfpathlineto{\pgfqpoint{6.271388in}{1.898915in}}%
\pgfpathlineto{\pgfqpoint{6.271719in}{2.485580in}}%
\pgfpathlineto{\pgfqpoint{6.272373in}{1.908379in}}%
\pgfpathlineto{\pgfqpoint{6.272752in}{2.573234in}}%
\pgfpathlineto{\pgfqpoint{6.272830in}{2.452138in}}%
\pgfpathlineto{\pgfqpoint{6.273626in}{1.900890in}}%
\pgfpathlineto{\pgfqpoint{6.273722in}{2.585517in}}%
\pgfpathlineto{\pgfqpoint{6.273940in}{2.363766in}}%
\pgfpathlineto{\pgfqpoint{6.274233in}{2.566180in}}%
\pgfpathlineto{\pgfqpoint{6.274496in}{1.966710in}}%
\pgfpathlineto{\pgfqpoint{6.275050in}{2.437527in}}%
\pgfpathlineto{\pgfqpoint{6.275835in}{1.727850in}}%
\pgfpathlineto{\pgfqpoint{6.275383in}{2.567930in}}%
\pgfpathlineto{\pgfqpoint{6.276161in}{2.176347in}}%
\pgfpathlineto{\pgfqpoint{6.277024in}{2.564113in}}%
\pgfpathlineto{\pgfqpoint{6.276241in}{1.815938in}}%
\pgfpathlineto{\pgfqpoint{6.277272in}{2.423704in}}%
\pgfpathlineto{\pgfqpoint{6.278286in}{1.698360in}}%
\pgfpathlineto{\pgfqpoint{6.278029in}{2.569644in}}%
\pgfpathlineto{\pgfqpoint{6.278384in}{2.370893in}}%
\pgfpathlineto{\pgfqpoint{6.279132in}{2.556738in}}%
\pgfpathlineto{\pgfqpoint{6.278446in}{1.862286in}}%
\pgfpathlineto{\pgfqpoint{6.279494in}{2.471728in}}%
\pgfpathlineto{\pgfqpoint{6.280400in}{1.939699in}}%
\pgfpathlineto{\pgfqpoint{6.280245in}{2.575285in}}%
\pgfpathlineto{\pgfqpoint{6.280605in}{2.332317in}}%
\pgfpathlineto{\pgfqpoint{6.281166in}{2.579046in}}%
\pgfpathlineto{\pgfqpoint{6.281512in}{1.712648in}}%
\pgfpathlineto{\pgfqpoint{6.281715in}{2.406960in}}%
\pgfpathlineto{\pgfqpoint{6.281793in}{1.856212in}}%
\pgfpathlineto{\pgfqpoint{6.282429in}{2.566189in}}%
\pgfpathlineto{\pgfqpoint{6.282826in}{2.363738in}}%
\pgfpathlineto{\pgfqpoint{6.283224in}{2.561555in}}%
\pgfpathlineto{\pgfqpoint{6.283778in}{1.736215in}}%
\pgfpathlineto{\pgfqpoint{6.283932in}{2.489969in}}%
\pgfpathlineto{\pgfqpoint{6.284141in}{1.737020in}}%
\pgfpathlineto{\pgfqpoint{6.283998in}{2.566138in}}%
\pgfpathlineto{\pgfqpoint{6.285044in}{2.400935in}}%
\pgfpathlineto{\pgfqpoint{6.285970in}{1.818969in}}%
\pgfpathlineto{\pgfqpoint{6.285445in}{2.572025in}}%
\pgfpathlineto{\pgfqpoint{6.286149in}{2.150726in}}%
\pgfpathlineto{\pgfqpoint{6.287085in}{2.582912in}}%
\pgfpathlineto{\pgfqpoint{6.286392in}{1.719584in}}%
\pgfpathlineto{\pgfqpoint{6.287260in}{2.432139in}}%
\pgfpathlineto{\pgfqpoint{6.287709in}{1.908277in}}%
\pgfpathlineto{\pgfqpoint{6.287511in}{2.558546in}}%
\pgfpathlineto{\pgfqpoint{6.288370in}{2.392688in}}%
\pgfpathlineto{\pgfqpoint{6.289057in}{2.551570in}}%
\pgfpathlineto{\pgfqpoint{6.288691in}{1.962571in}}%
\pgfpathlineto{\pgfqpoint{6.289478in}{2.380408in}}%
\pgfpathlineto{\pgfqpoint{6.290568in}{1.969730in}}%
\pgfpathlineto{\pgfqpoint{6.290385in}{2.556457in}}%
\pgfpathlineto{\pgfqpoint{6.290589in}{2.323683in}}%
\pgfpathlineto{\pgfqpoint{6.291189in}{2.560970in}}%
\pgfpathlineto{\pgfqpoint{6.291356in}{1.859299in}}%
\pgfpathlineto{\pgfqpoint{6.291701in}{2.380923in}}%
\pgfpathlineto{\pgfqpoint{6.291963in}{2.561278in}}%
\pgfpathlineto{\pgfqpoint{6.292519in}{1.835556in}}%
\pgfpathlineto{\pgfqpoint{6.292811in}{2.440242in}}%
\pgfpathlineto{\pgfqpoint{6.293791in}{1.840044in}}%
\pgfpathlineto{\pgfqpoint{6.293696in}{2.563321in}}%
\pgfpathlineto{\pgfqpoint{6.293922in}{2.451855in}}%
\pgfpathlineto{\pgfqpoint{6.294357in}{1.773264in}}%
\pgfpathlineto{\pgfqpoint{6.294802in}{2.556497in}}%
\pgfpathlineto{\pgfqpoint{6.295031in}{2.374996in}}%
\pgfpathlineto{\pgfqpoint{6.295054in}{2.583597in}}%
\pgfpathlineto{\pgfqpoint{6.296045in}{1.898725in}}%
\pgfpathlineto{\pgfqpoint{6.296142in}{2.428452in}}%
\pgfpathlineto{\pgfqpoint{6.297087in}{1.835776in}}%
\pgfpathlineto{\pgfqpoint{6.296189in}{2.575690in}}%
\pgfpathlineto{\pgfqpoint{6.297253in}{2.395913in}}%
\pgfpathlineto{\pgfqpoint{6.297722in}{2.570429in}}%
\pgfpathlineto{\pgfqpoint{6.298273in}{1.840551in}}%
\pgfpathlineto{\pgfqpoint{6.298364in}{2.436495in}}%
\pgfpathlineto{\pgfqpoint{6.298482in}{2.556642in}}%
\pgfpathlineto{\pgfqpoint{6.298439in}{1.888417in}}%
\pgfpathlineto{\pgfqpoint{6.299430in}{2.308912in}}%
\pgfpathlineto{\pgfqpoint{6.299943in}{1.744816in}}%
\pgfpathlineto{\pgfqpoint{6.300243in}{2.569223in}}%
\pgfpathlineto{\pgfqpoint{6.300541in}{2.373210in}}%
\pgfpathlineto{\pgfqpoint{6.300991in}{1.881476in}}%
\pgfpathlineto{\pgfqpoint{6.301283in}{2.572019in}}%
\pgfpathlineto{\pgfqpoint{6.301632in}{2.418902in}}%
\pgfpathlineto{\pgfqpoint{6.302421in}{2.562750in}}%
\pgfpathlineto{\pgfqpoint{6.301862in}{1.950073in}}%
\pgfpathlineto{\pgfqpoint{6.302742in}{2.339739in}}%
\pgfpathlineto{\pgfqpoint{6.303331in}{1.942038in}}%
\pgfpathlineto{\pgfqpoint{6.303473in}{2.562708in}}%
\pgfpathlineto{\pgfqpoint{6.303851in}{2.233535in}}%
\pgfpathlineto{\pgfqpoint{6.304945in}{2.557654in}}%
\pgfpathlineto{\pgfqpoint{6.304907in}{1.759100in}}%
\pgfpathlineto{\pgfqpoint{6.304963in}{2.310230in}}%
\pgfpathlineto{\pgfqpoint{6.306069in}{2.559694in}}%
\pgfpathlineto{\pgfqpoint{6.304997in}{1.977611in}}%
\pgfpathlineto{\pgfqpoint{6.306074in}{2.440216in}}%
\pgfpathlineto{\pgfqpoint{6.306146in}{1.789611in}}%
\pgfpathlineto{\pgfqpoint{6.307080in}{2.564971in}}%
\pgfpathlineto{\pgfqpoint{6.307184in}{2.447281in}}%
\pgfpathlineto{\pgfqpoint{6.307904in}{2.555465in}}%
\pgfpathlineto{\pgfqpoint{6.307529in}{1.886881in}}%
\pgfpathlineto{\pgfqpoint{6.308291in}{2.278626in}}%
\pgfpathlineto{\pgfqpoint{6.309365in}{1.722132in}}%
\pgfpathlineto{\pgfqpoint{6.308706in}{2.560791in}}%
\pgfpathlineto{\pgfqpoint{6.309401in}{2.282483in}}%
\pgfpathlineto{\pgfqpoint{6.309787in}{2.558875in}}%
\pgfpathlineto{\pgfqpoint{6.309852in}{1.959777in}}%
\pgfpathlineto{\pgfqpoint{6.310511in}{2.411397in}}%
\pgfpathlineto{\pgfqpoint{6.310599in}{1.803885in}}%
\pgfpathlineto{\pgfqpoint{6.310766in}{2.572039in}}%
\pgfpathlineto{\pgfqpoint{6.311623in}{2.196165in}}%
\pgfpathlineto{\pgfqpoint{6.312476in}{1.959855in}}%
\pgfpathlineto{\pgfqpoint{6.312519in}{2.568066in}}%
\pgfpathlineto{\pgfqpoint{6.312728in}{2.248529in}}%
\pgfpathlineto{\pgfqpoint{6.313734in}{2.566858in}}%
\pgfpathlineto{\pgfqpoint{6.313077in}{1.933229in}}%
\pgfpathlineto{\pgfqpoint{6.313839in}{2.303064in}}%
\pgfpathlineto{\pgfqpoint{6.314250in}{2.560052in}}%
\pgfpathlineto{\pgfqpoint{6.313917in}{1.732040in}}%
\pgfpathlineto{\pgfqpoint{6.314950in}{2.468539in}}%
\pgfpathlineto{\pgfqpoint{6.315269in}{1.489069in}}%
\pgfpathlineto{\pgfqpoint{6.314970in}{2.576107in}}%
\pgfpathlineto{\pgfqpoint{6.316062in}{2.293491in}}%
\pgfpathlineto{\pgfqpoint{6.316591in}{2.563396in}}%
\pgfpathlineto{\pgfqpoint{6.317009in}{1.976763in}}%
\pgfpathlineto{\pgfqpoint{6.317172in}{2.441395in}}%
\pgfpathlineto{\pgfqpoint{6.317746in}{1.715598in}}%
\pgfpathlineto{\pgfqpoint{6.318029in}{2.556352in}}%
\pgfpathlineto{\pgfqpoint{6.318282in}{2.426509in}}%
\pgfpathlineto{\pgfqpoint{6.319146in}{2.573481in}}%
\pgfpathlineto{\pgfqpoint{6.319158in}{1.899003in}}%
\pgfpathlineto{\pgfqpoint{6.319394in}{2.524600in}}%
\pgfpathlineto{\pgfqpoint{6.320392in}{1.783043in}}%
\pgfpathlineto{\pgfqpoint{6.320202in}{2.568113in}}%
\pgfpathlineto{\pgfqpoint{6.320504in}{2.367738in}}%
\pgfpathlineto{\pgfqpoint{6.321058in}{2.585504in}}%
\pgfpathlineto{\pgfqpoint{6.320601in}{2.005650in}}%
\pgfpathlineto{\pgfqpoint{6.321615in}{2.412185in}}%
\pgfpathlineto{\pgfqpoint{6.321916in}{1.888954in}}%
\pgfpathlineto{\pgfqpoint{6.322273in}{2.575969in}}%
\pgfpathlineto{\pgfqpoint{6.322727in}{2.180283in}}%
\pgfpathlineto{\pgfqpoint{6.323699in}{2.561467in}}%
\pgfpathlineto{\pgfqpoint{6.322743in}{1.845277in}}%
\pgfpathlineto{\pgfqpoint{6.323838in}{2.478734in}}%
\pgfpathlineto{\pgfqpoint{6.323989in}{2.594243in}}%
\pgfpathlineto{\pgfqpoint{6.324612in}{1.742486in}}%
\pgfpathlineto{\pgfqpoint{6.324941in}{2.337205in}}%
\pgfpathlineto{\pgfqpoint{6.325866in}{1.838930in}}%
\pgfpathlineto{\pgfqpoint{6.325685in}{2.569173in}}%
\pgfpathlineto{\pgfqpoint{6.326051in}{2.466692in}}%
\pgfpathlineto{\pgfqpoint{6.327024in}{1.922292in}}%
\pgfpathlineto{\pgfqpoint{6.326179in}{2.588134in}}%
\pgfpathlineto{\pgfqpoint{6.327162in}{2.275360in}}%
\pgfpathlineto{\pgfqpoint{6.327705in}{2.576010in}}%
\pgfpathlineto{\pgfqpoint{6.328131in}{1.854286in}}%
\pgfpathlineto{\pgfqpoint{6.328273in}{2.399985in}}%
\pgfpathlineto{\pgfqpoint{6.328473in}{1.883535in}}%
\pgfpathlineto{\pgfqpoint{6.328442in}{2.557240in}}%
\pgfpathlineto{\pgfqpoint{6.329384in}{2.219651in}}%
\pgfpathlineto{\pgfqpoint{6.330170in}{2.561552in}}%
\pgfpathlineto{\pgfqpoint{6.329592in}{1.937369in}}%
\pgfpathlineto{\pgfqpoint{6.330496in}{2.438852in}}%
\pgfpathlineto{\pgfqpoint{6.330959in}{1.902654in}}%
\pgfpathlineto{\pgfqpoint{6.330628in}{2.555956in}}%
\pgfpathlineto{\pgfqpoint{6.331608in}{2.373732in}}%
\pgfpathlineto{\pgfqpoint{6.332307in}{1.983394in}}%
\pgfpathlineto{\pgfqpoint{6.331973in}{2.565403in}}%
\pgfpathlineto{\pgfqpoint{6.332703in}{2.441364in}}%
\pgfpathlineto{\pgfqpoint{6.333692in}{2.571739in}}%
\pgfpathlineto{\pgfqpoint{6.333608in}{1.983880in}}%
\pgfpathlineto{\pgfqpoint{6.333812in}{2.383535in}}%
\pgfpathlineto{\pgfqpoint{6.334909in}{1.884987in}}%
\pgfpathlineto{\pgfqpoint{6.334070in}{2.587748in}}%
\pgfpathlineto{\pgfqpoint{6.334922in}{2.435851in}}%
\pgfpathlineto{\pgfqpoint{6.334979in}{2.580739in}}%
\pgfpathlineto{\pgfqpoint{6.335710in}{1.806374in}}%
\pgfpathlineto{\pgfqpoint{6.336022in}{2.355345in}}%
\pgfpathlineto{\pgfqpoint{6.336215in}{1.962082in}}%
\pgfpathlineto{\pgfqpoint{6.336353in}{2.566698in}}%
\pgfpathlineto{\pgfqpoint{6.337132in}{2.326208in}}%
\pgfpathlineto{\pgfqpoint{6.337791in}{2.573274in}}%
\pgfpathlineto{\pgfqpoint{6.337914in}{1.807048in}}%
\pgfpathlineto{\pgfqpoint{6.338243in}{2.508518in}}%
\pgfpathlineto{\pgfqpoint{6.338621in}{1.918591in}}%
\pgfpathlineto{\pgfqpoint{6.339230in}{2.557465in}}%
\pgfpathlineto{\pgfqpoint{6.339354in}{2.373886in}}%
\pgfpathlineto{\pgfqpoint{6.340227in}{1.787193in}}%
\pgfpathlineto{\pgfqpoint{6.339435in}{2.568587in}}%
\pgfpathlineto{\pgfqpoint{6.340464in}{2.461595in}}%
\pgfpathlineto{\pgfqpoint{6.341230in}{1.979610in}}%
\pgfpathlineto{\pgfqpoint{6.341481in}{2.565572in}}%
\pgfpathlineto{\pgfqpoint{6.341560in}{2.362372in}}%
\pgfpathlineto{\pgfqpoint{6.341561in}{2.587835in}}%
\pgfpathlineto{\pgfqpoint{6.342219in}{1.885683in}}%
\pgfpathlineto{\pgfqpoint{6.342671in}{2.419671in}}%
\pgfpathlineto{\pgfqpoint{6.343048in}{1.901257in}}%
\pgfpathlineto{\pgfqpoint{6.342987in}{2.575593in}}%
\pgfpathlineto{\pgfqpoint{6.343782in}{2.366311in}}%
\pgfpathlineto{\pgfqpoint{6.344542in}{1.908028in}}%
\pgfpathlineto{\pgfqpoint{6.344632in}{2.556869in}}%
\pgfpathlineto{\pgfqpoint{6.344888in}{2.328898in}}%
\pgfpathlineto{\pgfqpoint{6.345358in}{2.566020in}}%
\pgfpathlineto{\pgfqpoint{6.345002in}{1.883684in}}%
\pgfpathlineto{\pgfqpoint{6.345998in}{2.442006in}}%
\pgfpathlineto{\pgfqpoint{6.346796in}{1.719640in}}%
\pgfpathlineto{\pgfqpoint{6.346398in}{2.575338in}}%
\pgfpathlineto{\pgfqpoint{6.347109in}{2.407490in}}%
\pgfpathlineto{\pgfqpoint{6.347554in}{1.903726in}}%
\pgfpathlineto{\pgfqpoint{6.347904in}{2.561727in}}%
\pgfpathlineto{\pgfqpoint{6.348206in}{2.348576in}}%
\pgfpathlineto{\pgfqpoint{6.348221in}{2.556954in}}%
\pgfpathlineto{\pgfqpoint{6.348556in}{1.815521in}}%
\pgfpathlineto{\pgfqpoint{6.349315in}{2.165741in}}%
\pgfpathlineto{\pgfqpoint{6.349641in}{2.569113in}}%
\pgfpathlineto{\pgfqpoint{6.349619in}{1.920482in}}%
\pgfpathlineto{\pgfqpoint{6.350433in}{2.499594in}}%
\pgfpathlineto{\pgfqpoint{6.351006in}{1.799633in}}%
\pgfpathlineto{\pgfqpoint{6.350458in}{2.568753in}}%
\pgfpathlineto{\pgfqpoint{6.351546in}{2.201126in}}%
\pgfpathlineto{\pgfqpoint{6.352538in}{2.562020in}}%
\pgfpathlineto{\pgfqpoint{6.352226in}{1.899720in}}%
\pgfpathlineto{\pgfqpoint{6.352659in}{2.476521in}}%
\pgfpathlineto{\pgfqpoint{6.353109in}{1.819877in}}%
\pgfpathlineto{\pgfqpoint{6.353293in}{2.601070in}}%
\pgfpathlineto{\pgfqpoint{6.353772in}{2.305184in}}%
\pgfpathlineto{\pgfqpoint{6.354752in}{2.566256in}}%
\pgfpathlineto{\pgfqpoint{6.354113in}{1.772286in}}%
\pgfpathlineto{\pgfqpoint{6.354884in}{2.432960in}}%
\pgfpathlineto{\pgfqpoint{6.355510in}{1.873154in}}%
\pgfpathlineto{\pgfqpoint{6.355146in}{2.594241in}}%
\pgfpathlineto{\pgfqpoint{6.355994in}{2.378451in}}%
\pgfpathlineto{\pgfqpoint{6.356179in}{2.569236in}}%
\pgfpathlineto{\pgfqpoint{6.356815in}{1.846417in}}%
\pgfpathlineto{\pgfqpoint{6.357104in}{2.463960in}}%
\pgfpathlineto{\pgfqpoint{6.358201in}{1.773485in}}%
\pgfpathlineto{\pgfqpoint{6.357801in}{2.560697in}}%
\pgfpathlineto{\pgfqpoint{6.358215in}{2.398706in}}%
\pgfpathlineto{\pgfqpoint{6.358390in}{2.586856in}}%
\pgfpathlineto{\pgfqpoint{6.358474in}{1.933939in}}%
\pgfpathlineto{\pgfqpoint{6.359325in}{2.468400in}}%
\pgfpathlineto{\pgfqpoint{6.360103in}{1.872957in}}%
\pgfpathlineto{\pgfqpoint{6.359860in}{2.563957in}}%
\pgfpathlineto{\pgfqpoint{6.360436in}{2.390457in}}%
\pgfpathlineto{\pgfqpoint{6.360743in}{1.967773in}}%
\pgfpathlineto{\pgfqpoint{6.361537in}{2.582370in}}%
\pgfpathlineto{\pgfqpoint{6.361546in}{2.339666in}}%
\pgfpathlineto{\pgfqpoint{6.361576in}{2.608215in}}%
\pgfpathlineto{\pgfqpoint{6.362578in}{1.934698in}}%
\pgfpathlineto{\pgfqpoint{6.362656in}{2.411855in}}%
\pgfpathlineto{\pgfqpoint{6.363762in}{1.821220in}}%
\pgfpathlineto{\pgfqpoint{6.362939in}{2.572458in}}%
\pgfpathlineto{\pgfqpoint{6.363768in}{2.257587in}}%
\pgfpathlineto{\pgfqpoint{6.363947in}{2.575855in}}%
\pgfpathlineto{\pgfqpoint{6.364133in}{1.829894in}}%
\pgfpathlineto{\pgfqpoint{6.364882in}{2.440186in}}%
\pgfpathlineto{\pgfqpoint{6.365194in}{1.790404in}}%
\pgfpathlineto{\pgfqpoint{6.365579in}{2.586014in}}%
\pgfpathlineto{\pgfqpoint{6.365993in}{2.395244in}}%
\pgfpathlineto{\pgfqpoint{6.366750in}{2.570482in}}%
\pgfpathlineto{\pgfqpoint{6.367089in}{1.823018in}}%
\pgfpathlineto{\pgfqpoint{6.367102in}{2.277567in}}%
\pgfpathlineto{\pgfqpoint{6.367404in}{1.932300in}}%
\pgfpathlineto{\pgfqpoint{6.367729in}{2.593938in}}%
\pgfpathlineto{\pgfqpoint{6.368211in}{2.338499in}}%
\pgfpathlineto{\pgfqpoint{6.368755in}{2.560347in}}%
\pgfpathlineto{\pgfqpoint{6.369067in}{1.731977in}}%
\pgfpathlineto{\pgfqpoint{6.369321in}{2.345597in}}%
\pgfpathlineto{\pgfqpoint{6.370195in}{1.646637in}}%
\pgfpathlineto{\pgfqpoint{6.369331in}{2.570336in}}%
\pgfpathlineto{\pgfqpoint{6.370416in}{2.308237in}}%
\pgfpathlineto{\pgfqpoint{6.371033in}{2.564374in}}%
\pgfpathlineto{\pgfqpoint{6.371451in}{1.929317in}}%
\pgfpathlineto{\pgfqpoint{6.371527in}{2.378904in}}%
\pgfpathlineto{\pgfqpoint{6.372208in}{2.588543in}}%
\pgfpathlineto{\pgfqpoint{6.371935in}{1.968579in}}%
\pgfpathlineto{\pgfqpoint{6.372634in}{2.561198in}}%
\pgfpathlineto{\pgfqpoint{6.373265in}{1.965940in}}%
\pgfpathlineto{\pgfqpoint{6.373745in}{2.217174in}}%
\pgfpathlineto{\pgfqpoint{6.374055in}{2.575284in}}%
\pgfpathlineto{\pgfqpoint{6.374776in}{1.898642in}}%
\pgfpathlineto{\pgfqpoint{6.374856in}{2.434824in}}%
\pgfpathlineto{\pgfqpoint{6.375880in}{1.814930in}}%
\pgfpathlineto{\pgfqpoint{6.375248in}{2.566970in}}%
\pgfpathlineto{\pgfqpoint{6.375937in}{2.391709in}}%
\pgfpathlineto{\pgfqpoint{6.376074in}{2.573063in}}%
\pgfpathlineto{\pgfqpoint{6.376709in}{1.838538in}}%
\pgfpathlineto{\pgfqpoint{6.377048in}{2.473061in}}%
\pgfpathlineto{\pgfqpoint{6.377307in}{1.824389in}}%
\pgfpathlineto{\pgfqpoint{6.377720in}{2.569709in}}%
\pgfpathlineto{\pgfqpoint{6.378159in}{2.371776in}}%
\pgfpathlineto{\pgfqpoint{6.378816in}{2.580213in}}%
\pgfpathlineto{\pgfqpoint{6.379108in}{1.899372in}}%
\pgfpathlineto{\pgfqpoint{6.379270in}{2.453770in}}%
\pgfpathlineto{\pgfqpoint{6.379318in}{1.926029in}}%
\pgfpathlineto{\pgfqpoint{6.379774in}{2.563592in}}%
\pgfpathlineto{\pgfqpoint{6.380382in}{2.204363in}}%
\pgfpathlineto{\pgfqpoint{6.381361in}{2.564886in}}%
\pgfpathlineto{\pgfqpoint{6.380719in}{1.953647in}}%
\pgfpathlineto{\pgfqpoint{6.381493in}{2.344001in}}%
\pgfpathlineto{\pgfqpoint{6.381970in}{1.906509in}}%
\pgfpathlineto{\pgfqpoint{6.382115in}{2.560606in}}%
\pgfpathlineto{\pgfqpoint{6.382603in}{2.209074in}}%
\pgfpathlineto{\pgfqpoint{6.383022in}{2.568424in}}%
\pgfpathlineto{\pgfqpoint{6.383701in}{1.956565in}}%
\pgfpathlineto{\pgfqpoint{6.383715in}{2.446181in}}%
\pgfpathlineto{\pgfqpoint{6.383903in}{1.895121in}}%
\pgfpathlineto{\pgfqpoint{6.384433in}{2.576421in}}%
\pgfpathlineto{\pgfqpoint{6.384825in}{2.381014in}}%
\pgfpathlineto{\pgfqpoint{6.385272in}{2.589087in}}%
\pgfpathlineto{\pgfqpoint{6.385217in}{1.895000in}}%
\pgfpathlineto{\pgfqpoint{6.385935in}{2.278600in}}%
\pgfpathlineto{\pgfqpoint{6.386777in}{1.873462in}}%
\pgfpathlineto{\pgfqpoint{6.386994in}{2.560175in}}%
\pgfpathlineto{\pgfqpoint{6.387044in}{2.210615in}}%
\pgfpathlineto{\pgfqpoint{6.388056in}{2.567473in}}%
\pgfpathlineto{\pgfqpoint{6.388139in}{1.696359in}}%
\pgfpathlineto{\pgfqpoint{6.388155in}{2.480848in}}%
\pgfpathlineto{\pgfqpoint{6.389244in}{1.872407in}}%
\pgfpathlineto{\pgfqpoint{6.389060in}{2.566277in}}%
\pgfpathlineto{\pgfqpoint{6.389267in}{2.268375in}}%
\pgfpathlineto{\pgfqpoint{6.389336in}{2.569850in}}%
\pgfpathlineto{\pgfqpoint{6.389655in}{1.868975in}}%
\pgfpathlineto{\pgfqpoint{6.390378in}{2.290393in}}%
\pgfpathlineto{\pgfqpoint{6.390507in}{2.556217in}}%
\pgfpathlineto{\pgfqpoint{6.390670in}{1.885235in}}%
\pgfpathlineto{\pgfqpoint{6.391491in}{2.475555in}}%
\pgfpathlineto{\pgfqpoint{6.392049in}{1.827436in}}%
\pgfpathlineto{\pgfqpoint{6.392474in}{2.553538in}}%
\pgfpathlineto{\pgfqpoint{6.392604in}{2.399388in}}%
\pgfpathlineto{\pgfqpoint{6.392607in}{2.571496in}}%
\pgfpathlineto{\pgfqpoint{6.393589in}{1.871731in}}%
\pgfpathlineto{\pgfqpoint{6.393707in}{2.326707in}}%
\pgfpathlineto{\pgfqpoint{6.393921in}{1.803830in}}%
\pgfpathlineto{\pgfqpoint{6.393749in}{2.575227in}}%
\pgfpathlineto{\pgfqpoint{6.394818in}{2.383292in}}%
\pgfpathlineto{\pgfqpoint{6.395686in}{1.890762in}}%
\pgfpathlineto{\pgfqpoint{6.394828in}{2.565102in}}%
\pgfpathlineto{\pgfqpoint{6.395926in}{2.369639in}}%
\pgfpathlineto{\pgfqpoint{6.396407in}{2.559090in}}%
\pgfpathlineto{\pgfqpoint{6.396625in}{1.881454in}}%
\pgfpathlineto{\pgfqpoint{6.397035in}{2.332810in}}%
\pgfpathlineto{\pgfqpoint{6.397337in}{1.881592in}}%
\pgfpathlineto{\pgfqpoint{6.397511in}{2.591281in}}%
\pgfpathlineto{\pgfqpoint{6.398145in}{2.437883in}}%
\pgfpathlineto{\pgfqpoint{6.398590in}{1.911381in}}%
\pgfpathlineto{\pgfqpoint{6.398817in}{2.561390in}}%
\pgfpathlineto{\pgfqpoint{6.399255in}{2.343309in}}%
\pgfpathlineto{\pgfqpoint{6.400021in}{2.557957in}}%
\pgfpathlineto{\pgfqpoint{6.399330in}{1.837718in}}%
\pgfpathlineto{\pgfqpoint{6.400366in}{2.397160in}}%
\pgfpathlineto{\pgfqpoint{6.401473in}{1.866148in}}%
\pgfpathlineto{\pgfqpoint{6.400962in}{2.559852in}}%
\pgfpathlineto{\pgfqpoint{6.401477in}{2.276621in}}%
\pgfpathlineto{\pgfqpoint{6.402175in}{2.597914in}}%
\pgfpathlineto{\pgfqpoint{6.401929in}{1.902849in}}%
\pgfpathlineto{\pgfqpoint{6.402587in}{2.370981in}}%
\pgfpathlineto{\pgfqpoint{6.403175in}{1.946680in}}%
\pgfpathlineto{\pgfqpoint{6.402955in}{2.581506in}}%
\pgfpathlineto{\pgfqpoint{6.403697in}{2.440655in}}%
\pgfpathlineto{\pgfqpoint{6.404467in}{2.557869in}}%
\pgfpathlineto{\pgfqpoint{6.404586in}{1.748200in}}%
\pgfpathlineto{\pgfqpoint{6.404804in}{2.533248in}}%
\pgfpathlineto{\pgfqpoint{6.405013in}{1.896723in}}%
\pgfpathlineto{\pgfqpoint{6.405068in}{2.558178in}}%
\pgfpathlineto{\pgfqpoint{6.405916in}{2.230525in}}%
\pgfpathlineto{\pgfqpoint{6.406685in}{2.578808in}}%
\pgfpathlineto{\pgfqpoint{6.406150in}{1.802453in}}%
\pgfpathlineto{\pgfqpoint{6.407028in}{2.399113in}}%
\pgfpathlineto{\pgfqpoint{6.407748in}{1.890590in}}%
\pgfpathlineto{\pgfqpoint{6.407491in}{2.577996in}}%
\pgfpathlineto{\pgfqpoint{6.408139in}{2.433670in}}%
\pgfpathlineto{\pgfqpoint{6.408367in}{2.571775in}}%
\pgfpathlineto{\pgfqpoint{6.408393in}{1.973509in}}%
\pgfpathlineto{\pgfqpoint{6.409244in}{2.253540in}}%
\pgfpathlineto{\pgfqpoint{6.410069in}{1.842970in}}%
\pgfpathlineto{\pgfqpoint{6.409670in}{2.590108in}}%
\pgfpathlineto{\pgfqpoint{6.410355in}{2.434768in}}%
\pgfpathlineto{\pgfqpoint{6.410679in}{1.834604in}}%
\pgfpathlineto{\pgfqpoint{6.410683in}{2.566684in}}%
\pgfpathlineto{\pgfqpoint{6.411466in}{2.286175in}}%
\pgfpathlineto{\pgfqpoint{6.412359in}{2.584081in}}%
\pgfpathlineto{\pgfqpoint{6.411687in}{1.760098in}}%
\pgfpathlineto{\pgfqpoint{6.412577in}{2.456198in}}%
\pgfpathlineto{\pgfqpoint{6.412800in}{1.959520in}}%
\pgfpathlineto{\pgfqpoint{6.413481in}{2.585349in}}%
\pgfpathlineto{\pgfqpoint{6.413688in}{2.200958in}}%
\pgfpathlineto{\pgfqpoint{6.414666in}{1.880174in}}%
\pgfpathlineto{\pgfqpoint{6.414049in}{2.598175in}}%
\pgfpathlineto{\pgfqpoint{6.414795in}{2.319496in}}%
\pgfpathlineto{\pgfqpoint{6.415164in}{2.567945in}}%
\pgfpathlineto{\pgfqpoint{6.415555in}{1.828544in}}%
\pgfpathlineto{\pgfqpoint{6.415906in}{2.459410in}}%
\pgfpathlineto{\pgfqpoint{6.416369in}{1.894499in}}%
\pgfpathlineto{\pgfqpoint{6.416352in}{2.566353in}}%
\pgfpathlineto{\pgfqpoint{6.417018in}{2.264381in}}%
\pgfpathlineto{\pgfqpoint{6.417023in}{2.556857in}}%
\pgfpathlineto{\pgfqpoint{6.417173in}{1.868193in}}%
\pgfpathlineto{\pgfqpoint{6.418131in}{2.471352in}}%
\pgfpathlineto{\pgfqpoint{6.419158in}{1.953072in}}%
\pgfpathlineto{\pgfqpoint{6.418709in}{2.561775in}}%
\pgfpathlineto{\pgfqpoint{6.419243in}{2.285480in}}%
\pgfpathlineto{\pgfqpoint{6.419328in}{2.570897in}}%
\pgfpathlineto{\pgfqpoint{6.419424in}{1.842125in}}%
\pgfpathlineto{\pgfqpoint{6.420354in}{2.406220in}}%
\pgfpathlineto{\pgfqpoint{6.421173in}{1.863215in}}%
\pgfpathlineto{\pgfqpoint{6.421202in}{2.579565in}}%
\pgfpathlineto{\pgfqpoint{6.421461in}{2.432268in}}%
\pgfpathlineto{\pgfqpoint{6.422096in}{2.591084in}}%
\pgfpathlineto{\pgfqpoint{6.422528in}{1.793881in}}%
\pgfpathlineto{\pgfqpoint{6.422570in}{2.452281in}}%
\pgfpathlineto{\pgfqpoint{6.422787in}{1.800681in}}%
\pgfpathlineto{\pgfqpoint{6.423049in}{2.589327in}}%
\pgfpathlineto{\pgfqpoint{6.423681in}{2.205145in}}%
\pgfpathlineto{\pgfqpoint{6.424652in}{2.573635in}}%
\pgfpathlineto{\pgfqpoint{6.424467in}{1.885260in}}%
\pgfpathlineto{\pgfqpoint{6.424792in}{2.276950in}}%
\pgfpathlineto{\pgfqpoint{6.425073in}{2.592757in}}%
\pgfpathlineto{\pgfqpoint{6.424804in}{2.007037in}}%
\pgfpathlineto{\pgfqpoint{6.425902in}{2.457528in}}%
\pgfpathlineto{\pgfqpoint{6.426940in}{1.942947in}}%
\pgfpathlineto{\pgfqpoint{6.426878in}{2.563674in}}%
\pgfpathlineto{\pgfqpoint{6.427013in}{2.527822in}}%
\pgfpathlineto{\pgfqpoint{6.427494in}{1.902701in}}%
\pgfpathlineto{\pgfqpoint{6.427980in}{2.588668in}}%
\pgfpathlineto{\pgfqpoint{6.428125in}{2.423282in}}%
\pgfpathlineto{\pgfqpoint{6.428519in}{1.953971in}}%
\pgfpathlineto{\pgfqpoint{6.428463in}{2.565268in}}%
\pgfpathlineto{\pgfqpoint{6.429227in}{2.497454in}}%
\pgfpathlineto{\pgfqpoint{6.430016in}{2.564558in}}%
\pgfpathlineto{\pgfqpoint{6.429993in}{1.996164in}}%
\pgfpathlineto{\pgfqpoint{6.430335in}{2.426174in}}%
\pgfpathlineto{\pgfqpoint{6.431211in}{1.940807in}}%
\pgfpathlineto{\pgfqpoint{6.431041in}{2.573384in}}%
\pgfpathlineto{\pgfqpoint{6.431446in}{2.400493in}}%
\pgfpathlineto{\pgfqpoint{6.431926in}{2.555803in}}%
\pgfpathlineto{\pgfqpoint{6.432539in}{1.876713in}}%
\pgfpathlineto{\pgfqpoint{6.432555in}{2.500906in}}%
\pgfpathlineto{\pgfqpoint{6.433194in}{1.904595in}}%
\pgfpathlineto{\pgfqpoint{6.433234in}{2.569577in}}%
\pgfpathlineto{\pgfqpoint{6.433667in}{2.478947in}}%
\pgfpathlineto{\pgfqpoint{6.434239in}{1.934924in}}%
\pgfpathlineto{\pgfqpoint{6.434396in}{2.561732in}}%
\pgfpathlineto{\pgfqpoint{6.434778in}{2.455826in}}%
\pgfpathlineto{\pgfqpoint{6.435877in}{1.672689in}}%
\pgfpathlineto{\pgfqpoint{6.435814in}{2.566503in}}%
\pgfpathlineto{\pgfqpoint{6.435889in}{2.232396in}}%
\pgfpathlineto{\pgfqpoint{6.436163in}{2.573791in}}%
\pgfpathlineto{\pgfqpoint{6.436305in}{1.884062in}}%
\pgfpathlineto{\pgfqpoint{6.437000in}{2.434460in}}%
\pgfpathlineto{\pgfqpoint{6.437154in}{1.903559in}}%
\pgfpathlineto{\pgfqpoint{6.438092in}{2.589745in}}%
\pgfpathlineto{\pgfqpoint{6.438112in}{2.215245in}}%
\pgfpathlineto{\pgfqpoint{6.439144in}{2.570501in}}%
\pgfpathlineto{\pgfqpoint{6.439104in}{1.741329in}}%
\pgfpathlineto{\pgfqpoint{6.439223in}{2.468899in}}%
\pgfpathlineto{\pgfqpoint{6.440247in}{1.742353in}}%
\pgfpathlineto{\pgfqpoint{6.439692in}{2.575875in}}%
\pgfpathlineto{\pgfqpoint{6.440334in}{2.190854in}}%
\pgfpathlineto{\pgfqpoint{6.440864in}{2.572841in}}%
\pgfpathlineto{\pgfqpoint{6.441414in}{1.800894in}}%
\pgfpathlineto{\pgfqpoint{6.441445in}{2.258006in}}%
\pgfpathlineto{\pgfqpoint{6.442312in}{2.574466in}}%
\pgfpathlineto{\pgfqpoint{6.441705in}{1.657918in}}%
\pgfpathlineto{\pgfqpoint{6.442558in}{2.449804in}}%
\pgfpathlineto{\pgfqpoint{6.443085in}{1.756419in}}%
\pgfpathlineto{\pgfqpoint{6.443520in}{2.564274in}}%
\pgfpathlineto{\pgfqpoint{6.443670in}{2.415543in}}%
\pgfpathlineto{\pgfqpoint{6.444077in}{2.557590in}}%
\pgfpathlineto{\pgfqpoint{6.444139in}{1.987695in}}%
\pgfpathlineto{\pgfqpoint{6.444781in}{2.450664in}}%
\pgfpathlineto{\pgfqpoint{6.445504in}{1.710375in}}%
\pgfpathlineto{\pgfqpoint{6.444900in}{2.576968in}}%
\pgfpathlineto{\pgfqpoint{6.445893in}{2.287386in}}%
\pgfpathlineto{\pgfqpoint{6.446568in}{2.567278in}}%
\pgfpathlineto{\pgfqpoint{6.446567in}{1.900383in}}%
\pgfpathlineto{\pgfqpoint{6.447004in}{2.482901in}}%
\pgfpathlineto{\pgfqpoint{6.447951in}{1.732461in}}%
\pgfpathlineto{\pgfqpoint{6.447960in}{2.573888in}}%
\pgfpathlineto{\pgfqpoint{6.448116in}{2.343547in}}%
\pgfpathlineto{\pgfqpoint{6.448986in}{2.568839in}}%
\pgfpathlineto{\pgfqpoint{6.449168in}{1.832087in}}%
\pgfpathlineto{\pgfqpoint{6.449226in}{2.437275in}}%
\pgfpathlineto{\pgfqpoint{6.449547in}{1.802187in}}%
\pgfpathlineto{\pgfqpoint{6.449468in}{2.569144in}}%
\pgfpathlineto{\pgfqpoint{6.450336in}{2.519930in}}%
\pgfpathlineto{\pgfqpoint{6.451239in}{1.893175in}}%
\pgfpathlineto{\pgfqpoint{6.450566in}{2.565045in}}%
\pgfpathlineto{\pgfqpoint{6.451448in}{2.308819in}}%
\pgfpathlineto{\pgfqpoint{6.452422in}{2.581068in}}%
\pgfpathlineto{\pgfqpoint{6.452445in}{1.901710in}}%
\pgfpathlineto{\pgfqpoint{6.452559in}{2.359208in}}%
\pgfpathlineto{\pgfqpoint{6.452764in}{1.724284in}}%
\pgfpathlineto{\pgfqpoint{6.453520in}{2.560554in}}%
\pgfpathlineto{\pgfqpoint{6.453670in}{2.308824in}}%
\pgfpathlineto{\pgfqpoint{6.453791in}{2.586892in}}%
\pgfpathlineto{\pgfqpoint{6.454179in}{1.897330in}}%
\pgfpathlineto{\pgfqpoint{6.454781in}{2.340543in}}%
\pgfpathlineto{\pgfqpoint{6.455094in}{2.560939in}}%
\pgfpathlineto{\pgfqpoint{6.455768in}{1.664353in}}%
\pgfpathlineto{\pgfqpoint{6.455891in}{2.428994in}}%
\pgfpathlineto{\pgfqpoint{6.456888in}{1.920554in}}%
\pgfpathlineto{\pgfqpoint{6.456832in}{2.568649in}}%
\pgfpathlineto{\pgfqpoint{6.457002in}{2.324727in}}%
\pgfpathlineto{\pgfqpoint{6.457796in}{2.588447in}}%
\pgfpathlineto{\pgfqpoint{6.457534in}{1.893069in}}%
\pgfpathlineto{\pgfqpoint{6.458112in}{2.294497in}}%
\pgfpathlineto{\pgfqpoint{6.458137in}{1.926001in}}%
\pgfpathlineto{\pgfqpoint{6.459110in}{2.571651in}}%
\pgfpathlineto{\pgfqpoint{6.459216in}{2.399780in}}%
\pgfpathlineto{\pgfqpoint{6.459306in}{2.563591in}}%
\pgfpathlineto{\pgfqpoint{6.459267in}{1.769872in}}%
\pgfpathlineto{\pgfqpoint{6.460325in}{2.357518in}}%
\pgfpathlineto{\pgfqpoint{6.460797in}{1.760678in}}%
\pgfpathlineto{\pgfqpoint{6.460546in}{2.575727in}}%
\pgfpathlineto{\pgfqpoint{6.461436in}{2.451230in}}%
\pgfpathlineto{\pgfqpoint{6.461737in}{1.881835in}}%
\pgfpathlineto{\pgfqpoint{6.461865in}{2.582399in}}%
\pgfpathlineto{\pgfqpoint{6.462546in}{2.206140in}}%
\pgfpathlineto{\pgfqpoint{6.463542in}{2.553270in}}%
\pgfpathlineto{\pgfqpoint{6.463033in}{1.894767in}}%
\pgfpathlineto{\pgfqpoint{6.463657in}{2.375518in}}%
\pgfpathlineto{\pgfqpoint{6.463859in}{1.875229in}}%
\pgfpathlineto{\pgfqpoint{6.464732in}{2.579851in}}%
\pgfpathlineto{\pgfqpoint{6.464768in}{2.336536in}}%
\pgfpathlineto{\pgfqpoint{6.465092in}{1.852314in}}%
\pgfpathlineto{\pgfqpoint{6.465426in}{2.575008in}}%
\pgfpathlineto{\pgfqpoint{6.465879in}{2.353366in}}%
\pgfpathlineto{\pgfqpoint{6.465977in}{2.580964in}}%
\pgfpathlineto{\pgfqpoint{6.466295in}{1.847930in}}%
\pgfpathlineto{\pgfqpoint{6.466989in}{2.220415in}}%
\pgfpathlineto{\pgfqpoint{6.467372in}{2.583868in}}%
\pgfpathlineto{\pgfqpoint{6.467100in}{1.751863in}}%
\pgfpathlineto{\pgfqpoint{6.468103in}{2.529365in}}%
\pgfpathlineto{\pgfqpoint{6.468126in}{1.784031in}}%
\pgfpathlineto{\pgfqpoint{6.468346in}{2.585916in}}%
\pgfpathlineto{\pgfqpoint{6.469214in}{2.351532in}}%
\pgfpathlineto{\pgfqpoint{6.469892in}{2.556289in}}%
\pgfpathlineto{\pgfqpoint{6.470252in}{1.946218in}}%
\pgfpathlineto{\pgfqpoint{6.470325in}{2.315669in}}%
\pgfpathlineto{\pgfqpoint{6.470925in}{2.563930in}}%
\pgfpathlineto{\pgfqpoint{6.470893in}{1.722159in}}%
\pgfpathlineto{\pgfqpoint{6.471437in}{2.430645in}}%
\pgfpathlineto{\pgfqpoint{6.471628in}{2.576741in}}%
\pgfpathlineto{\pgfqpoint{6.471760in}{1.859337in}}%
\pgfpathlineto{\pgfqpoint{6.472498in}{2.480153in}}%
\pgfpathlineto{\pgfqpoint{6.473350in}{1.706926in}}%
\pgfpathlineto{\pgfqpoint{6.473532in}{2.566254in}}%
\pgfpathlineto{\pgfqpoint{6.473609in}{2.292633in}}%
\pgfpathlineto{\pgfqpoint{6.473745in}{2.573967in}}%
\pgfpathlineto{\pgfqpoint{6.474518in}{1.863419in}}%
\pgfpathlineto{\pgfqpoint{6.474720in}{2.422063in}}%
\pgfpathlineto{\pgfqpoint{6.474814in}{1.901835in}}%
\pgfpathlineto{\pgfqpoint{6.475723in}{2.559628in}}%
\pgfpathlineto{\pgfqpoint{6.475812in}{2.048794in}}%
\pgfpathlineto{\pgfqpoint{6.475847in}{2.580349in}}%
\pgfpathlineto{\pgfqpoint{6.475887in}{1.805765in}}%
\pgfpathlineto{\pgfqpoint{6.476923in}{2.155615in}}%
\pgfpathlineto{\pgfqpoint{6.477411in}{2.564568in}}%
\pgfpathlineto{\pgfqpoint{6.477551in}{1.837572in}}%
\pgfpathlineto{\pgfqpoint{6.478035in}{2.486273in}}%
\pgfpathlineto{\pgfqpoint{6.478136in}{1.835971in}}%
\pgfpathlineto{\pgfqpoint{6.478837in}{2.562207in}}%
\pgfpathlineto{\pgfqpoint{6.479145in}{2.458561in}}%
\pgfpathlineto{\pgfqpoint{6.479260in}{2.602950in}}%
\pgfpathlineto{\pgfqpoint{6.479684in}{1.918988in}}%
\pgfpathlineto{\pgfqpoint{6.480248in}{2.395014in}}%
\pgfpathlineto{\pgfqpoint{6.481214in}{1.908484in}}%
\pgfpathlineto{\pgfqpoint{6.481306in}{2.570318in}}%
\pgfpathlineto{\pgfqpoint{6.481357in}{2.374580in}}%
\pgfpathlineto{\pgfqpoint{6.482366in}{2.572531in}}%
\pgfpathlineto{\pgfqpoint{6.481588in}{1.819478in}}%
\pgfpathlineto{\pgfqpoint{6.482468in}{2.458979in}}%
\pgfpathlineto{\pgfqpoint{6.482764in}{1.846447in}}%
\pgfpathlineto{\pgfqpoint{6.482737in}{2.569273in}}%
\pgfpathlineto{\pgfqpoint{6.483579in}{2.424777in}}%
\pgfpathlineto{\pgfqpoint{6.484319in}{2.591204in}}%
\pgfpathlineto{\pgfqpoint{6.484230in}{1.801756in}}%
\pgfpathlineto{\pgfqpoint{6.484680in}{2.448662in}}%
\pgfpathlineto{\pgfqpoint{6.485029in}{1.805103in}}%
\pgfpathlineto{\pgfqpoint{6.485143in}{2.581634in}}%
\pgfpathlineto{\pgfqpoint{6.485790in}{2.409438in}}%
\pgfpathlineto{\pgfqpoint{6.485968in}{2.569881in}}%
\pgfpathlineto{\pgfqpoint{6.486155in}{1.946434in}}%
\pgfpathlineto{\pgfqpoint{6.486901in}{2.427611in}}%
\pgfpathlineto{\pgfqpoint{6.487219in}{1.839588in}}%
\pgfpathlineto{\pgfqpoint{6.487427in}{2.574217in}}%
\pgfpathlineto{\pgfqpoint{6.488013in}{2.357143in}}%
\pgfpathlineto{\pgfqpoint{6.488084in}{2.563706in}}%
\pgfpathlineto{\pgfqpoint{6.488201in}{1.801665in}}%
\pgfpathlineto{\pgfqpoint{6.489124in}{2.413214in}}%
\pgfpathlineto{\pgfqpoint{6.489728in}{1.842424in}}%
\pgfpathlineto{\pgfqpoint{6.489688in}{2.562678in}}%
\pgfpathlineto{\pgfqpoint{6.490235in}{2.358846in}}%
\pgfpathlineto{\pgfqpoint{6.491252in}{2.574117in}}%
\pgfpathlineto{\pgfqpoint{6.490929in}{1.850675in}}%
\pgfpathlineto{\pgfqpoint{6.491345in}{2.298405in}}%
\pgfpathlineto{\pgfqpoint{6.492034in}{1.882720in}}%
\pgfpathlineto{\pgfqpoint{6.492426in}{2.558884in}}%
\pgfpathlineto{\pgfqpoint{6.492451in}{2.429701in}}%
\pgfpathlineto{\pgfqpoint{6.493312in}{2.569991in}}%
\pgfpathlineto{\pgfqpoint{6.492602in}{1.878951in}}%
\pgfpathlineto{\pgfqpoint{6.493561in}{2.420438in}}%
\pgfpathlineto{\pgfqpoint{6.494050in}{1.891534in}}%
\pgfpathlineto{\pgfqpoint{6.493970in}{2.578407in}}%
\pgfpathlineto{\pgfqpoint{6.494672in}{2.322205in}}%
\pgfpathlineto{\pgfqpoint{6.495102in}{2.556668in}}%
\pgfpathlineto{\pgfqpoint{6.495628in}{1.699856in}}%
\pgfpathlineto{\pgfqpoint{6.495782in}{2.528590in}}%
\pgfpathlineto{\pgfqpoint{6.495846in}{1.649547in}}%
\pgfpathlineto{\pgfqpoint{6.496292in}{2.587158in}}%
\pgfpathlineto{\pgfqpoint{6.496893in}{2.376021in}}%
\pgfpathlineto{\pgfqpoint{6.497231in}{2.564361in}}%
\pgfpathlineto{\pgfqpoint{6.497462in}{1.778887in}}%
\pgfpathlineto{\pgfqpoint{6.498003in}{2.396232in}}%
\pgfpathlineto{\pgfqpoint{6.498199in}{1.856599in}}%
\pgfpathlineto{\pgfqpoint{6.498973in}{2.580009in}}%
\pgfpathlineto{\pgfqpoint{6.499113in}{2.474132in}}%
\pgfpathlineto{\pgfqpoint{6.500188in}{1.602827in}}%
\pgfpathlineto{\pgfqpoint{6.499544in}{2.582684in}}%
\pgfpathlineto{\pgfqpoint{6.500224in}{2.255330in}}%
\pgfpathlineto{\pgfqpoint{6.500830in}{2.562839in}}%
\pgfpathlineto{\pgfqpoint{6.501136in}{1.837990in}}%
\pgfpathlineto{\pgfqpoint{6.501336in}{2.476229in}}%
\pgfpathlineto{\pgfqpoint{6.501499in}{1.858800in}}%
\pgfpathlineto{\pgfqpoint{6.501842in}{2.549021in}}%
\pgfpathlineto{\pgfqpoint{6.502447in}{2.443842in}}%
\pgfpathlineto{\pgfqpoint{6.503012in}{1.907182in}}%
\pgfpathlineto{\pgfqpoint{6.502751in}{2.582769in}}%
\pgfpathlineto{\pgfqpoint{6.503558in}{2.440696in}}%
\pgfpathlineto{\pgfqpoint{6.504000in}{1.859527in}}%
\pgfpathlineto{\pgfqpoint{6.503922in}{2.564794in}}%
\pgfpathlineto{\pgfqpoint{6.504666in}{2.251972in}}%
\pgfpathlineto{\pgfqpoint{6.505557in}{2.579294in}}%
\pgfpathlineto{\pgfqpoint{6.505491in}{1.952749in}}%
\pgfpathlineto{\pgfqpoint{6.505778in}{2.465740in}}%
\pgfpathlineto{\pgfqpoint{6.505792in}{1.952145in}}%
\pgfpathlineto{\pgfqpoint{6.506011in}{2.575631in}}%
\pgfpathlineto{\pgfqpoint{6.506889in}{2.407159in}}%
\pgfpathlineto{\pgfqpoint{6.507191in}{2.566374in}}%
\pgfpathlineto{\pgfqpoint{6.507318in}{1.908507in}}%
\pgfpathlineto{\pgfqpoint{6.507995in}{2.384048in}}%
\pgfpathlineto{\pgfqpoint{6.508050in}{1.870609in}}%
\pgfpathlineto{\pgfqpoint{6.508486in}{2.580244in}}%
\pgfpathlineto{\pgfqpoint{6.509106in}{2.480016in}}%
\pgfpathlineto{\pgfqpoint{6.510177in}{1.954096in}}%
\pgfpathlineto{\pgfqpoint{6.509196in}{2.576882in}}%
\pgfpathlineto{\pgfqpoint{6.510217in}{2.262034in}}%
\pgfpathlineto{\pgfqpoint{6.510321in}{2.567649in}}%
\pgfpathlineto{\pgfqpoint{6.510590in}{1.787241in}}%
\pgfpathlineto{\pgfqpoint{6.511327in}{2.417992in}}%
\pgfpathlineto{\pgfqpoint{6.511977in}{1.950923in}}%
\pgfpathlineto{\pgfqpoint{6.511581in}{2.566444in}}%
\pgfpathlineto{\pgfqpoint{6.512438in}{2.293201in}}%
\pgfpathlineto{\pgfqpoint{6.512594in}{1.893327in}}%
\pgfpathlineto{\pgfqpoint{6.513103in}{2.561279in}}%
\pgfpathlineto{\pgfqpoint{6.513546in}{2.352892in}}%
\pgfpathlineto{\pgfqpoint{6.514066in}{2.574812in}}%
\pgfpathlineto{\pgfqpoint{6.514491in}{1.664443in}}%
\pgfpathlineto{\pgfqpoint{6.514658in}{2.456419in}}%
\pgfpathlineto{\pgfqpoint{6.515425in}{1.953325in}}%
\pgfpathlineto{\pgfqpoint{6.515045in}{2.583930in}}%
\pgfpathlineto{\pgfqpoint{6.515768in}{2.405976in}}%
\pgfpathlineto{\pgfqpoint{6.516293in}{2.575438in}}%
\pgfpathlineto{\pgfqpoint{6.515820in}{1.846102in}}%
\pgfpathlineto{\pgfqpoint{6.516874in}{2.318113in}}%
\pgfpathlineto{\pgfqpoint{6.517145in}{1.854969in}}%
\pgfpathlineto{\pgfqpoint{6.517555in}{2.560401in}}%
\pgfpathlineto{\pgfqpoint{6.517985in}{2.385622in}}%
\pgfpathlineto{\pgfqpoint{6.518386in}{1.942519in}}%
\pgfpathlineto{\pgfqpoint{6.518677in}{2.571515in}}%
\pgfpathlineto{\pgfqpoint{6.519085in}{2.438940in}}%
\pgfpathlineto{\pgfqpoint{6.519876in}{2.563995in}}%
\pgfpathlineto{\pgfqpoint{6.519990in}{1.801966in}}%
\pgfpathlineto{\pgfqpoint{6.520194in}{2.466692in}}%
\pgfpathlineto{\pgfqpoint{6.520867in}{1.878362in}}%
\pgfpathlineto{\pgfqpoint{6.520302in}{2.607978in}}%
\pgfpathlineto{\pgfqpoint{6.521305in}{2.292539in}}%
\pgfpathlineto{\pgfqpoint{6.522346in}{2.572537in}}%
\pgfpathlineto{\pgfqpoint{6.522219in}{1.724371in}}%
\pgfpathlineto{\pgfqpoint{6.522416in}{2.464935in}}%
\pgfpathlineto{\pgfqpoint{6.522907in}{1.913735in}}%
\pgfpathlineto{\pgfqpoint{6.523433in}{2.572252in}}%
\pgfpathlineto{\pgfqpoint{6.523527in}{2.415703in}}%
\pgfpathlineto{\pgfqpoint{6.524013in}{2.558686in}}%
\pgfpathlineto{\pgfqpoint{6.524256in}{1.803023in}}%
\pgfpathlineto{\pgfqpoint{6.524638in}{2.384727in}}%
\pgfpathlineto{\pgfqpoint{6.524893in}{1.868662in}}%
\pgfpathlineto{\pgfqpoint{6.525126in}{2.571551in}}%
\pgfpathlineto{\pgfqpoint{6.525748in}{2.295474in}}%
\pgfpathlineto{\pgfqpoint{6.526341in}{2.566703in}}%
\pgfpathlineto{\pgfqpoint{6.526576in}{1.831160in}}%
\pgfpathlineto{\pgfqpoint{6.526859in}{2.346090in}}%
\pgfpathlineto{\pgfqpoint{6.526942in}{2.577844in}}%
\pgfpathlineto{\pgfqpoint{6.527247in}{1.729706in}}%
\pgfpathlineto{\pgfqpoint{6.527970in}{2.395455in}}%
\pgfpathlineto{\pgfqpoint{6.528279in}{1.923193in}}%
\pgfpathlineto{\pgfqpoint{6.528750in}{2.574500in}}%
\pgfpathlineto{\pgfqpoint{6.529081in}{2.343008in}}%
\pgfpathlineto{\pgfqpoint{6.530121in}{2.571705in}}%
\pgfpathlineto{\pgfqpoint{6.529993in}{1.884546in}}%
\pgfpathlineto{\pgfqpoint{6.530191in}{2.385666in}}%
\pgfpathlineto{\pgfqpoint{6.530669in}{1.794292in}}%
\pgfpathlineto{\pgfqpoint{6.530242in}{2.568331in}}%
\pgfpathlineto{\pgfqpoint{6.531302in}{2.375339in}}%
\pgfpathlineto{\pgfqpoint{6.531423in}{1.721637in}}%
\pgfpathlineto{\pgfqpoint{6.532120in}{2.572338in}}%
\pgfpathlineto{\pgfqpoint{6.532411in}{2.369958in}}%
\pgfpathlineto{\pgfqpoint{6.532419in}{2.584365in}}%
\pgfpathlineto{\pgfqpoint{6.532888in}{1.784153in}}%
\pgfpathlineto{\pgfqpoint{6.533521in}{2.384647in}}%
\pgfpathlineto{\pgfqpoint{6.534418in}{1.946662in}}%
\pgfpathlineto{\pgfqpoint{6.533813in}{2.581952in}}%
\pgfpathlineto{\pgfqpoint{6.534631in}{2.394441in}}%
\pgfpathlineto{\pgfqpoint{6.534659in}{2.568888in}}%
\pgfpathlineto{\pgfqpoint{6.535095in}{1.925608in}}%
\pgfpathlineto{\pgfqpoint{6.535743in}{2.468190in}}%
\pgfpathlineto{\pgfqpoint{6.536666in}{2.564691in}}%
\pgfpathlineto{\pgfqpoint{6.536142in}{1.557151in}}%
\pgfpathlineto{\pgfqpoint{6.536830in}{2.244701in}}%
\pgfpathlineto{\pgfqpoint{6.537621in}{1.676612in}}%
\pgfpathlineto{\pgfqpoint{6.537937in}{2.576981in}}%
\pgfpathlineto{\pgfqpoint{6.537941in}{2.293841in}}%
\pgfpathlineto{\pgfqpoint{6.538876in}{2.565618in}}%
\pgfpathlineto{\pgfqpoint{6.538455in}{1.916594in}}%
\pgfpathlineto{\pgfqpoint{6.539052in}{2.437926in}}%
\pgfpathlineto{\pgfqpoint{6.539093in}{1.843862in}}%
\pgfpathlineto{\pgfqpoint{6.539218in}{2.579352in}}%
\pgfpathlineto{\pgfqpoint{6.540164in}{2.369300in}}%
\pgfpathlineto{\pgfqpoint{6.541185in}{2.573313in}}%
\pgfpathlineto{\pgfqpoint{6.540407in}{1.857877in}}%
\pgfpathlineto{\pgfqpoint{6.541275in}{2.428181in}}%
\pgfpathlineto{\pgfqpoint{6.541394in}{1.908727in}}%
\pgfpathlineto{\pgfqpoint{6.541522in}{2.574085in}}%
\pgfpathlineto{\pgfqpoint{6.542386in}{2.363279in}}%
\pgfpathlineto{\pgfqpoint{6.542906in}{2.578440in}}%
\pgfpathlineto{\pgfqpoint{6.542897in}{1.877147in}}%
\pgfpathlineto{\pgfqpoint{6.543456in}{2.323114in}}%
\pgfpathlineto{\pgfqpoint{6.543958in}{1.801507in}}%
\pgfpathlineto{\pgfqpoint{6.543842in}{2.564265in}}%
\pgfpathlineto{\pgfqpoint{6.544567in}{2.194535in}}%
\pgfpathlineto{\pgfqpoint{6.545116in}{2.575311in}}%
\pgfpathlineto{\pgfqpoint{6.545125in}{1.905847in}}%
\pgfpathlineto{\pgfqpoint{6.545678in}{2.459846in}}%
\pgfpathlineto{\pgfqpoint{6.546512in}{1.940424in}}%
\pgfpathlineto{\pgfqpoint{6.545844in}{2.574121in}}%
\pgfpathlineto{\pgfqpoint{6.546789in}{2.323355in}}%
\pgfpathlineto{\pgfqpoint{6.546807in}{2.601573in}}%
\pgfpathlineto{\pgfqpoint{6.547682in}{1.689102in}}%
\pgfpathlineto{\pgfqpoint{6.547900in}{2.410645in}}%
\pgfpathlineto{\pgfqpoint{6.548406in}{1.828931in}}%
\pgfpathlineto{\pgfqpoint{6.548787in}{2.578917in}}%
\pgfpathlineto{\pgfqpoint{6.549012in}{1.989150in}}%
\pgfpathlineto{\pgfqpoint{6.550095in}{2.558016in}}%
\pgfpathlineto{\pgfqpoint{6.549842in}{1.774297in}}%
\pgfpathlineto{\pgfqpoint{6.550123in}{2.385820in}}%
\pgfpathlineto{\pgfqpoint{6.551089in}{2.583792in}}%
\pgfpathlineto{\pgfqpoint{6.550564in}{1.693161in}}%
\pgfpathlineto{\pgfqpoint{6.551234in}{2.420324in}}%
\pgfpathlineto{\pgfqpoint{6.551392in}{1.762751in}}%
\pgfpathlineto{\pgfqpoint{6.552098in}{2.559581in}}%
\pgfpathlineto{\pgfqpoint{6.552345in}{2.352766in}}%
\pgfpathlineto{\pgfqpoint{6.552999in}{2.583829in}}%
\pgfpathlineto{\pgfqpoint{6.553440in}{1.790811in}}%
\pgfpathlineto{\pgfqpoint{6.553456in}{2.486871in}}%
\pgfpathlineto{\pgfqpoint{6.553709in}{1.874864in}}%
\pgfpathlineto{\pgfqpoint{6.554537in}{2.563245in}}%
\pgfpathlineto{\pgfqpoint{6.554568in}{2.337489in}}%
\pgfpathlineto{\pgfqpoint{6.554605in}{2.583412in}}%
\pgfpathlineto{\pgfqpoint{6.555657in}{1.809007in}}%
\pgfpathlineto{\pgfqpoint{6.555681in}{2.439003in}}%
\pgfpathlineto{\pgfqpoint{6.556788in}{1.920574in}}%
\pgfpathlineto{\pgfqpoint{6.556148in}{2.561652in}}%
\pgfpathlineto{\pgfqpoint{6.556792in}{2.174020in}}%
\pgfpathlineto{\pgfqpoint{6.557099in}{2.580366in}}%
\pgfpathlineto{\pgfqpoint{6.557221in}{1.588188in}}%
\pgfpathlineto{\pgfqpoint{6.557903in}{2.487534in}}%
\pgfpathlineto{\pgfqpoint{6.558122in}{1.918871in}}%
\pgfpathlineto{\pgfqpoint{6.557944in}{2.570155in}}%
\pgfpathlineto{\pgfqpoint{6.559014in}{2.367834in}}%
\pgfpathlineto{\pgfqpoint{6.559939in}{2.562426in}}%
\pgfpathlineto{\pgfqpoint{6.559483in}{1.804367in}}%
\pgfpathlineto{\pgfqpoint{6.560125in}{2.409228in}}%
\pgfpathlineto{\pgfqpoint{6.560729in}{1.898515in}}%
\pgfpathlineto{\pgfqpoint{6.560882in}{2.565928in}}%
\pgfpathlineto{\pgfqpoint{6.561237in}{2.264747in}}%
\pgfpathlineto{\pgfqpoint{6.561727in}{2.567997in}}%
\pgfpathlineto{\pgfqpoint{6.561814in}{1.751886in}}%
\pgfpathlineto{\pgfqpoint{6.562348in}{2.392666in}}%
\pgfpathlineto{\pgfqpoint{6.562805in}{2.597593in}}%
\pgfpathlineto{\pgfqpoint{6.562422in}{1.817371in}}%
\pgfpathlineto{\pgfqpoint{6.562965in}{2.047712in}}%
\pgfusepath{stroke}%
\end{pgfscope}%
\begin{pgfscope}%
\pgfpathrectangle{\pgfqpoint{0.535225in}{0.370679in}}{\pgfqpoint{6.314775in}{3.181174in}}%
\pgfusepath{clip}%
\pgfsetrectcap%
\pgfsetroundjoin%
\pgfsetlinewidth{3.011250pt}%
\definecolor{currentstroke}{rgb}{0.000000,0.000000,1.000000}%
\pgfsetstrokecolor{currentstroke}%
\pgfsetdash{}{0pt}%
\pgfpathmoveto{\pgfqpoint{0.525225in}{1.483099in}}%
\pgfpathlineto{\pgfqpoint{0.822260in}{1.484464in}}%
\pgfpathlineto{\pgfqpoint{1.116406in}{1.490987in}}%
\pgfpathlineto{\pgfqpoint{1.288470in}{1.333636in}}%
\pgfpathlineto{\pgfqpoint{1.410551in}{1.572483in}}%
\pgfpathlineto{\pgfqpoint{1.505245in}{1.680930in}}%
\pgfpathlineto{\pgfqpoint{1.582616in}{1.602111in}}%
\pgfpathlineto{\pgfqpoint{1.704697in}{1.649171in}}%
\pgfpathlineto{\pgfqpoint{1.754680in}{1.622967in}}%
\pgfpathlineto{\pgfqpoint{1.799391in}{1.670674in}}%
\pgfpathlineto{\pgfqpoint{1.839837in}{1.598192in}}%
\pgfpathlineto{\pgfqpoint{1.876761in}{1.502222in}}%
\pgfpathlineto{\pgfqpoint{1.910729in}{1.546246in}}%
\pgfpathlineto{\pgfqpoint{1.942177in}{1.555372in}}%
\pgfpathlineto{\pgfqpoint{1.971455in}{1.657766in}}%
\pgfpathlineto{\pgfqpoint{1.998843in}{1.339830in}}%
\pgfpathlineto{\pgfqpoint{2.024570in}{1.444975in}}%
\pgfpathlineto{\pgfqpoint{2.048826in}{1.553684in}}%
\pgfpathlineto{\pgfqpoint{2.071770in}{1.772077in}}%
\pgfpathlineto{\pgfqpoint{2.093537in}{1.571634in}}%
\pgfpathlineto{\pgfqpoint{2.114241in}{1.702734in}}%
\pgfpathlineto{\pgfqpoint{2.133983in}{1.749844in}}%
\pgfpathlineto{\pgfqpoint{2.152847in}{1.680659in}}%
\pgfpathlineto{\pgfqpoint{2.170907in}{1.688016in}}%
\pgfpathlineto{\pgfqpoint{2.188231in}{1.670120in}}%
\pgfpathlineto{\pgfqpoint{2.204874in}{1.668250in}}%
\pgfpathlineto{\pgfqpoint{2.220890in}{1.768929in}}%
\pgfpathlineto{\pgfqpoint{2.236323in}{1.758852in}}%
\pgfpathlineto{\pgfqpoint{2.251214in}{1.666266in}}%
\pgfpathlineto{\pgfqpoint{2.265601in}{1.780659in}}%
\pgfpathlineto{\pgfqpoint{2.279516in}{1.719119in}}%
\pgfpathlineto{\pgfqpoint{2.292989in}{1.621842in}}%
\pgfpathlineto{\pgfqpoint{2.306047in}{1.671069in}}%
\pgfpathlineto{\pgfqpoint{2.318716in}{1.549407in}}%
\pgfpathlineto{\pgfqpoint{2.331017in}{1.816995in}}%
\pgfpathlineto{\pgfqpoint{2.342971in}{1.719715in}}%
\pgfpathlineto{\pgfqpoint{2.354599in}{1.701725in}}%
\pgfpathlineto{\pgfqpoint{2.365916in}{1.629218in}}%
\pgfpathlineto{\pgfqpoint{2.376939in}{1.703916in}}%
\pgfpathlineto{\pgfqpoint{2.387683in}{1.671318in}}%
\pgfpathlineto{\pgfqpoint{2.398161in}{1.736736in}}%
\pgfpathlineto{\pgfqpoint{2.408387in}{1.718737in}}%
\pgfpathlineto{\pgfqpoint{2.418373in}{1.752407in}}%
\pgfpathlineto{\pgfqpoint{2.428129in}{1.740865in}}%
\pgfpathlineto{\pgfqpoint{2.437665in}{1.691878in}}%
\pgfpathlineto{\pgfqpoint{2.446992in}{1.764091in}}%
\pgfpathlineto{\pgfqpoint{2.456119in}{1.594979in}}%
\pgfpathlineto{\pgfqpoint{2.465053in}{1.839348in}}%
\pgfpathlineto{\pgfqpoint{2.473803in}{1.777239in}}%
\pgfpathlineto{\pgfqpoint{2.482376in}{1.824377in}}%
\pgfpathlineto{\pgfqpoint{2.490780in}{1.754338in}}%
\pgfpathlineto{\pgfqpoint{2.499020in}{1.672173in}}%
\pgfpathlineto{\pgfqpoint{2.507103in}{1.687446in}}%
\pgfpathlineto{\pgfqpoint{2.515036in}{1.749733in}}%
\pgfpathlineto{\pgfqpoint{2.522822in}{1.755258in}}%
\pgfpathlineto{\pgfqpoint{2.530469in}{1.710292in}}%
\pgfpathlineto{\pgfqpoint{2.537980in}{1.748362in}}%
\pgfpathlineto{\pgfqpoint{2.545360in}{1.807857in}}%
\pgfpathlineto{\pgfqpoint{2.552614in}{1.744279in}}%
\pgfpathlineto{\pgfqpoint{2.559747in}{1.776433in}}%
\pgfpathlineto{\pgfqpoint{2.566761in}{1.756608in}}%
\pgfpathlineto{\pgfqpoint{2.573662in}{1.667095in}}%
\pgfpathlineto{\pgfqpoint{2.580451in}{1.768958in}}%
\pgfpathlineto{\pgfqpoint{2.587135in}{1.763856in}}%
\pgfpathlineto{\pgfqpoint{2.593714in}{1.845540in}}%
\pgfpathlineto{\pgfqpoint{2.600193in}{1.806771in}}%
\pgfpathlineto{\pgfqpoint{2.606574in}{1.820332in}}%
\pgfpathlineto{\pgfqpoint{2.612861in}{1.719257in}}%
\pgfpathlineto{\pgfqpoint{2.619057in}{1.748466in}}%
\pgfpathlineto{\pgfqpoint{2.625163in}{1.744702in}}%
\pgfpathlineto{\pgfqpoint{2.631182in}{1.817770in}}%
\pgfpathlineto{\pgfqpoint{2.637117in}{1.705086in}}%
\pgfpathlineto{\pgfqpoint{2.642971in}{1.669388in}}%
\pgfpathlineto{\pgfqpoint{2.648744in}{1.772991in}}%
\pgfpathlineto{\pgfqpoint{2.654441in}{1.621377in}}%
\pgfpathlineto{\pgfqpoint{2.660061in}{1.808171in}}%
\pgfpathlineto{\pgfqpoint{2.665609in}{1.385552in}}%
\pgfpathlineto{\pgfqpoint{2.671084in}{1.783456in}}%
\pgfpathlineto{\pgfqpoint{2.676490in}{1.787987in}}%
\pgfpathlineto{\pgfqpoint{2.681828in}{1.731607in}}%
\pgfpathlineto{\pgfqpoint{2.687100in}{1.750731in}}%
\pgfpathlineto{\pgfqpoint{2.692307in}{1.696972in}}%
\pgfpathlineto{\pgfqpoint{2.697451in}{1.857698in}}%
\pgfpathlineto{\pgfqpoint{2.702533in}{1.818323in}}%
\pgfpathlineto{\pgfqpoint{2.707555in}{1.719267in}}%
\pgfpathlineto{\pgfqpoint{2.712518in}{1.581758in}}%
\pgfpathlineto{\pgfqpoint{2.717424in}{1.850500in}}%
\pgfpathlineto{\pgfqpoint{2.722274in}{1.785282in}}%
\pgfpathlineto{\pgfqpoint{2.727069in}{1.752543in}}%
\pgfpathlineto{\pgfqpoint{2.731811in}{1.765334in}}%
\pgfpathlineto{\pgfqpoint{2.736500in}{1.800749in}}%
\pgfpathlineto{\pgfqpoint{2.741138in}{1.929698in}}%
\pgfpathlineto{\pgfqpoint{2.745726in}{1.879250in}}%
\pgfpathlineto{\pgfqpoint{2.750264in}{1.856787in}}%
\pgfpathlineto{\pgfqpoint{2.754755in}{1.870108in}}%
\pgfpathlineto{\pgfqpoint{2.759199in}{1.683744in}}%
\pgfpathlineto{\pgfqpoint{2.763596in}{1.716495in}}%
\pgfpathlineto{\pgfqpoint{2.776522in}{1.884263in}}%
\pgfpathlineto{\pgfqpoint{2.780745in}{1.644071in}}%
\pgfpathlineto{\pgfqpoint{2.784926in}{1.779003in}}%
\pgfpathlineto{\pgfqpoint{2.789066in}{1.789596in}}%
\pgfpathlineto{\pgfqpoint{2.793166in}{1.778945in}}%
\pgfpathlineto{\pgfqpoint{2.797227in}{1.752174in}}%
\pgfpathlineto{\pgfqpoint{2.801249in}{1.853561in}}%
\pgfpathlineto{\pgfqpoint{2.805234in}{1.781329in}}%
\pgfpathlineto{\pgfqpoint{2.809181in}{1.889944in}}%
\pgfpathlineto{\pgfqpoint{2.813093in}{1.890710in}}%
\pgfpathlineto{\pgfqpoint{2.820809in}{1.687612in}}%
\pgfpathlineto{\pgfqpoint{2.824615in}{1.720967in}}%
\pgfpathlineto{\pgfqpoint{2.828387in}{1.691571in}}%
\pgfpathlineto{\pgfqpoint{2.832126in}{1.919934in}}%
\pgfpathlineto{\pgfqpoint{2.835832in}{1.818292in}}%
\pgfpathlineto{\pgfqpoint{2.839506in}{1.789956in}}%
\pgfpathlineto{\pgfqpoint{2.843149in}{1.865748in}}%
\pgfpathlineto{\pgfqpoint{2.846760in}{1.814735in}}%
\pgfpathlineto{\pgfqpoint{2.850341in}{1.854576in}}%
\pgfpathlineto{\pgfqpoint{2.853893in}{1.808599in}}%
\pgfpathlineto{\pgfqpoint{2.857414in}{1.674833in}}%
\pgfpathlineto{\pgfqpoint{2.860907in}{1.878756in}}%
\pgfpathlineto{\pgfqpoint{2.864371in}{1.893901in}}%
\pgfpathlineto{\pgfqpoint{2.867807in}{1.625547in}}%
\pgfpathlineto{\pgfqpoint{2.871216in}{1.825077in}}%
\pgfpathlineto{\pgfqpoint{2.874597in}{1.620240in}}%
\pgfpathlineto{\pgfqpoint{2.877952in}{1.875643in}}%
\pgfpathlineto{\pgfqpoint{2.881280in}{1.855960in}}%
\pgfpathlineto{\pgfqpoint{2.884583in}{1.851857in}}%
\pgfpathlineto{\pgfqpoint{2.887860in}{1.760104in}}%
\pgfpathlineto{\pgfqpoint{2.891111in}{1.852428in}}%
\pgfpathlineto{\pgfqpoint{2.897541in}{1.885584in}}%
\pgfpathlineto{\pgfqpoint{2.900720in}{1.713846in}}%
\pgfpathlineto{\pgfqpoint{2.903875in}{1.845863in}}%
\pgfpathlineto{\pgfqpoint{2.907007in}{1.871832in}}%
\pgfpathlineto{\pgfqpoint{2.910116in}{1.855828in}}%
\pgfpathlineto{\pgfqpoint{2.913202in}{1.875391in}}%
\pgfpathlineto{\pgfqpoint{2.916266in}{1.903913in}}%
\pgfpathlineto{\pgfqpoint{2.922329in}{1.685776in}}%
\pgfpathlineto{\pgfqpoint{2.925328in}{1.473188in}}%
\pgfpathlineto{\pgfqpoint{2.928306in}{1.848197in}}%
\pgfpathlineto{\pgfqpoint{2.931263in}{1.913211in}}%
\pgfpathlineto{\pgfqpoint{2.934200in}{1.722973in}}%
\pgfpathlineto{\pgfqpoint{2.937116in}{1.847805in}}%
\pgfpathlineto{\pgfqpoint{2.940013in}{1.540422in}}%
\pgfpathlineto{\pgfqpoint{2.942890in}{1.884450in}}%
\pgfpathlineto{\pgfqpoint{2.945748in}{1.916662in}}%
\pgfpathlineto{\pgfqpoint{2.948586in}{1.787371in}}%
\pgfpathlineto{\pgfqpoint{2.951406in}{1.710282in}}%
\pgfpathlineto{\pgfqpoint{2.954207in}{1.671293in}}%
\pgfpathlineto{\pgfqpoint{2.956990in}{1.906519in}}%
\pgfpathlineto{\pgfqpoint{2.959754in}{1.925965in}}%
\pgfpathlineto{\pgfqpoint{2.962501in}{1.792878in}}%
\pgfpathlineto{\pgfqpoint{2.965230in}{1.839140in}}%
\pgfpathlineto{\pgfqpoint{2.967942in}{1.827960in}}%
\pgfpathlineto{\pgfqpoint{2.970636in}{1.790838in}}%
\pgfpathlineto{\pgfqpoint{2.973313in}{1.928819in}}%
\pgfpathlineto{\pgfqpoint{2.975974in}{1.902872in}}%
\pgfpathlineto{\pgfqpoint{2.978618in}{1.570953in}}%
\pgfpathlineto{\pgfqpoint{2.981246in}{1.859404in}}%
\pgfpathlineto{\pgfqpoint{2.983857in}{1.764006in}}%
\pgfpathlineto{\pgfqpoint{2.986453in}{1.897679in}}%
\pgfpathlineto{\pgfqpoint{2.989032in}{1.839489in}}%
\pgfpathlineto{\pgfqpoint{2.991597in}{1.869612in}}%
\pgfpathlineto{\pgfqpoint{2.994145in}{2.007044in}}%
\pgfpathlineto{\pgfqpoint{2.999197in}{1.600256in}}%
\pgfpathlineto{\pgfqpoint{3.001701in}{1.864225in}}%
\pgfpathlineto{\pgfqpoint{3.004190in}{1.858102in}}%
\pgfpathlineto{\pgfqpoint{3.006664in}{1.827812in}}%
\pgfpathlineto{\pgfqpoint{3.009124in}{1.890711in}}%
\pgfpathlineto{\pgfqpoint{3.011570in}{1.832683in}}%
\pgfpathlineto{\pgfqpoint{3.014002in}{1.926107in}}%
\pgfpathlineto{\pgfqpoint{3.016420in}{1.817945in}}%
\pgfpathlineto{\pgfqpoint{3.018824in}{1.556864in}}%
\pgfpathlineto{\pgfqpoint{3.021215in}{1.806328in}}%
\pgfpathlineto{\pgfqpoint{3.023593in}{1.879502in}}%
\pgfpathlineto{\pgfqpoint{3.025957in}{1.662277in}}%
\pgfpathlineto{\pgfqpoint{3.028308in}{1.670994in}}%
\pgfpathlineto{\pgfqpoint{3.030646in}{1.618786in}}%
\pgfpathlineto{\pgfqpoint{3.032971in}{1.761082in}}%
\pgfpathlineto{\pgfqpoint{3.035284in}{1.768826in}}%
\pgfpathlineto{\pgfqpoint{3.037584in}{1.818590in}}%
\pgfpathlineto{\pgfqpoint{3.042147in}{1.661874in}}%
\pgfpathlineto{\pgfqpoint{3.044410in}{1.884514in}}%
\pgfpathlineto{\pgfqpoint{3.046661in}{1.796327in}}%
\pgfpathlineto{\pgfqpoint{3.048901in}{1.786537in}}%
\pgfpathlineto{\pgfqpoint{3.051128in}{1.755640in}}%
\pgfpathlineto{\pgfqpoint{3.053344in}{1.798689in}}%
\pgfpathlineto{\pgfqpoint{3.055549in}{1.750012in}}%
\pgfpathlineto{\pgfqpoint{3.057742in}{1.726785in}}%
\pgfpathlineto{\pgfqpoint{3.059924in}{1.902299in}}%
\pgfpathlineto{\pgfqpoint{3.062095in}{1.961383in}}%
\pgfpathlineto{\pgfqpoint{3.064254in}{1.807662in}}%
\pgfpathlineto{\pgfqpoint{3.066403in}{1.839233in}}%
\pgfpathlineto{\pgfqpoint{3.068541in}{1.918025in}}%
\pgfpathlineto{\pgfqpoint{3.070668in}{1.859366in}}%
\pgfpathlineto{\pgfqpoint{3.072784in}{1.918017in}}%
\pgfpathlineto{\pgfqpoint{3.074890in}{1.856464in}}%
\pgfpathlineto{\pgfqpoint{3.076986in}{1.923676in}}%
\pgfpathlineto{\pgfqpoint{3.081146in}{1.674577in}}%
\pgfpathlineto{\pgfqpoint{3.083211in}{1.925677in}}%
\pgfpathlineto{\pgfqpoint{3.085267in}{1.833985in}}%
\pgfpathlineto{\pgfqpoint{3.087312in}{1.811049in}}%
\pgfpathlineto{\pgfqpoint{3.089347in}{1.648166in}}%
\pgfpathlineto{\pgfqpoint{3.091373in}{1.941063in}}%
\pgfpathlineto{\pgfqpoint{3.093389in}{1.895835in}}%
\pgfpathlineto{\pgfqpoint{3.095395in}{1.433249in}}%
\pgfpathlineto{\pgfqpoint{3.099380in}{1.903957in}}%
\pgfpathlineto{\pgfqpoint{3.101358in}{1.924957in}}%
\pgfpathlineto{\pgfqpoint{3.105287in}{1.863058in}}%
\pgfpathlineto{\pgfqpoint{3.107238in}{1.789137in}}%
\pgfpathlineto{\pgfqpoint{3.109181in}{1.765567in}}%
\pgfpathlineto{\pgfqpoint{3.113038in}{1.903481in}}%
\pgfpathlineto{\pgfqpoint{3.114954in}{1.717278in}}%
\pgfpathlineto{\pgfqpoint{3.116862in}{1.872255in}}%
\pgfpathlineto{\pgfqpoint{3.118760in}{1.927349in}}%
\pgfpathlineto{\pgfqpoint{3.120651in}{1.809353in}}%
\pgfpathlineto{\pgfqpoint{3.122532in}{1.971318in}}%
\pgfpathlineto{\pgfqpoint{3.124406in}{1.832202in}}%
\pgfpathlineto{\pgfqpoint{3.126271in}{1.974298in}}%
\pgfpathlineto{\pgfqpoint{3.128128in}{1.698950in}}%
\pgfpathlineto{\pgfqpoint{3.129978in}{1.901692in}}%
\pgfpathlineto{\pgfqpoint{3.131819in}{1.943928in}}%
\pgfpathlineto{\pgfqpoint{3.133652in}{1.837462in}}%
\pgfpathlineto{\pgfqpoint{3.135477in}{1.830642in}}%
\pgfpathlineto{\pgfqpoint{3.137294in}{1.864402in}}%
\pgfpathlineto{\pgfqpoint{3.140906in}{1.780049in}}%
\pgfpathlineto{\pgfqpoint{3.144487in}{1.968830in}}%
\pgfpathlineto{\pgfqpoint{3.146266in}{1.826573in}}%
\pgfpathlineto{\pgfqpoint{3.149803in}{1.896814in}}%
\pgfpathlineto{\pgfqpoint{3.151560in}{1.908413in}}%
\pgfpathlineto{\pgfqpoint{3.155053in}{1.874531in}}%
\pgfpathlineto{\pgfqpoint{3.156788in}{1.904761in}}%
\pgfpathlineto{\pgfqpoint{3.158517in}{1.912263in}}%
\pgfpathlineto{\pgfqpoint{3.160238in}{1.804754in}}%
\pgfpathlineto{\pgfqpoint{3.161953in}{1.848576in}}%
\pgfpathlineto{\pgfqpoint{3.163661in}{1.933323in}}%
\pgfpathlineto{\pgfqpoint{3.165362in}{1.606808in}}%
\pgfpathlineto{\pgfqpoint{3.167056in}{1.569336in}}%
\pgfpathlineto{\pgfqpoint{3.168743in}{1.677370in}}%
\pgfpathlineto{\pgfqpoint{3.170424in}{1.906877in}}%
\pgfpathlineto{\pgfqpoint{3.172098in}{1.783802in}}%
\pgfpathlineto{\pgfqpoint{3.173765in}{1.851521in}}%
\pgfpathlineto{\pgfqpoint{3.175426in}{1.831417in}}%
\pgfpathlineto{\pgfqpoint{3.177080in}{1.896727in}}%
\pgfpathlineto{\pgfqpoint{3.178728in}{1.733483in}}%
\pgfpathlineto{\pgfqpoint{3.180370in}{1.976054in}}%
\pgfpathlineto{\pgfqpoint{3.182005in}{1.650744in}}%
\pgfpathlineto{\pgfqpoint{3.185257in}{1.858787in}}%
\pgfpathlineto{\pgfqpoint{3.186874in}{1.833435in}}%
\pgfpathlineto{\pgfqpoint{3.188484in}{1.883186in}}%
\pgfpathlineto{\pgfqpoint{3.190089in}{1.847813in}}%
\pgfpathlineto{\pgfqpoint{3.191687in}{1.931308in}}%
\pgfpathlineto{\pgfqpoint{3.193279in}{1.924487in}}%
\pgfpathlineto{\pgfqpoint{3.194866in}{1.945798in}}%
\pgfpathlineto{\pgfqpoint{3.196446in}{1.783368in}}%
\pgfpathlineto{\pgfqpoint{3.198021in}{1.996186in}}%
\pgfpathlineto{\pgfqpoint{3.201153in}{1.868972in}}%
\pgfpathlineto{\pgfqpoint{3.202710in}{1.884164in}}%
\pgfpathlineto{\pgfqpoint{3.204262in}{1.778900in}}%
\pgfpathlineto{\pgfqpoint{3.205808in}{1.936041in}}%
\pgfpathlineto{\pgfqpoint{3.207348in}{1.720009in}}%
\pgfpathlineto{\pgfqpoint{3.208883in}{1.921321in}}%
\pgfpathlineto{\pgfqpoint{3.210412in}{1.643413in}}%
\pgfpathlineto{\pgfqpoint{3.211936in}{1.926776in}}%
\pgfpathlineto{\pgfqpoint{3.213454in}{1.822378in}}%
\pgfpathlineto{\pgfqpoint{3.214967in}{1.932539in}}%
\pgfpathlineto{\pgfqpoint{3.216474in}{1.884615in}}%
\pgfpathlineto{\pgfqpoint{3.217977in}{1.932238in}}%
\pgfpathlineto{\pgfqpoint{3.219473in}{1.940170in}}%
\pgfpathlineto{\pgfqpoint{3.220965in}{1.820589in}}%
\pgfpathlineto{\pgfqpoint{3.223933in}{1.931526in}}%
\pgfpathlineto{\pgfqpoint{3.225409in}{1.991819in}}%
\pgfpathlineto{\pgfqpoint{3.228346in}{1.618416in}}%
\pgfpathlineto{\pgfqpoint{3.229806in}{1.895534in}}%
\pgfpathlineto{\pgfqpoint{3.231262in}{1.819676in}}%
\pgfpathlineto{\pgfqpoint{3.234159in}{1.962786in}}%
\pgfpathlineto{\pgfqpoint{3.235600in}{1.907728in}}%
\pgfpathlineto{\pgfqpoint{3.237036in}{1.786090in}}%
\pgfpathlineto{\pgfqpoint{3.238467in}{1.818837in}}%
\pgfpathlineto{\pgfqpoint{3.239894in}{1.923362in}}%
\pgfpathlineto{\pgfqpoint{3.241315in}{1.923578in}}%
\pgfpathlineto{\pgfqpoint{3.242732in}{1.768164in}}%
\pgfpathlineto{\pgfqpoint{3.244144in}{1.881051in}}%
\pgfpathlineto{\pgfqpoint{3.245552in}{1.719997in}}%
\pgfpathlineto{\pgfqpoint{3.248353in}{1.988805in}}%
\pgfpathlineto{\pgfqpoint{3.252520in}{1.655102in}}%
\pgfpathlineto{\pgfqpoint{3.253900in}{1.902064in}}%
\pgfpathlineto{\pgfqpoint{3.255276in}{1.773099in}}%
\pgfpathlineto{\pgfqpoint{3.256647in}{1.920569in}}%
\pgfpathlineto{\pgfqpoint{3.258014in}{1.828358in}}%
\pgfpathlineto{\pgfqpoint{3.259376in}{1.985279in}}%
\pgfpathlineto{\pgfqpoint{3.260734in}{1.963318in}}%
\pgfpathlineto{\pgfqpoint{3.262087in}{1.918706in}}%
\pgfpathlineto{\pgfqpoint{3.263437in}{1.912665in}}%
\pgfpathlineto{\pgfqpoint{3.264782in}{1.887007in}}%
\pgfpathlineto{\pgfqpoint{3.266123in}{1.833467in}}%
\pgfpathlineto{\pgfqpoint{3.267459in}{1.877709in}}%
\pgfpathlineto{\pgfqpoint{3.268792in}{1.792360in}}%
\pgfpathlineto{\pgfqpoint{3.270120in}{1.787880in}}%
\pgfpathlineto{\pgfqpoint{3.272764in}{1.883220in}}%
\pgfpathlineto{\pgfqpoint{3.275391in}{1.565969in}}%
\pgfpathlineto{\pgfqpoint{3.276699in}{1.877120in}}%
\pgfpathlineto{\pgfqpoint{3.278003in}{1.836020in}}%
\pgfpathlineto{\pgfqpoint{3.280598in}{1.884038in}}%
\pgfpathlineto{\pgfqpoint{3.281890in}{1.885624in}}%
\pgfpathlineto{\pgfqpoint{3.283178in}{1.933149in}}%
\pgfpathlineto{\pgfqpoint{3.285742in}{1.886686in}}%
\pgfpathlineto{\pgfqpoint{3.287019in}{1.939848in}}%
\pgfpathlineto{\pgfqpoint{3.289560in}{1.727108in}}%
\pgfpathlineto{\pgfqpoint{3.290825in}{1.941506in}}%
\pgfpathlineto{\pgfqpoint{3.292086in}{1.891204in}}%
\pgfpathlineto{\pgfqpoint{3.293343in}{1.914019in}}%
\pgfpathlineto{\pgfqpoint{3.294597in}{1.872874in}}%
\pgfpathlineto{\pgfqpoint{3.295847in}{1.916993in}}%
\pgfpathlineto{\pgfqpoint{3.297093in}{1.883033in}}%
\pgfpathlineto{\pgfqpoint{3.298336in}{1.932806in}}%
\pgfpathlineto{\pgfqpoint{3.299575in}{1.822102in}}%
\pgfpathlineto{\pgfqpoint{3.300810in}{1.854478in}}%
\pgfpathlineto{\pgfqpoint{3.302042in}{1.855555in}}%
\pgfpathlineto{\pgfqpoint{3.304495in}{1.788359in}}%
\pgfpathlineto{\pgfqpoint{3.306934in}{1.967281in}}%
\pgfpathlineto{\pgfqpoint{3.308148in}{1.827762in}}%
\pgfpathlineto{\pgfqpoint{3.309359in}{1.927523in}}%
\pgfpathlineto{\pgfqpoint{3.310566in}{1.730220in}}%
\pgfpathlineto{\pgfqpoint{3.311770in}{1.770806in}}%
\pgfpathlineto{\pgfqpoint{3.312970in}{1.975460in}}%
\pgfpathlineto{\pgfqpoint{3.314167in}{1.809232in}}%
\pgfpathlineto{\pgfqpoint{3.315361in}{1.832371in}}%
\pgfpathlineto{\pgfqpoint{3.316551in}{1.771808in}}%
\pgfpathlineto{\pgfqpoint{3.317738in}{1.997661in}}%
\pgfpathlineto{\pgfqpoint{3.318922in}{1.948835in}}%
\pgfpathlineto{\pgfqpoint{3.320103in}{1.986061in}}%
\pgfpathlineto{\pgfqpoint{3.321280in}{1.903075in}}%
\pgfpathlineto{\pgfqpoint{3.322454in}{1.962146in}}%
\pgfpathlineto{\pgfqpoint{3.324792in}{1.844067in}}%
\pgfpathlineto{\pgfqpoint{3.325956in}{1.811576in}}%
\pgfpathlineto{\pgfqpoint{3.328275in}{1.946679in}}%
\pgfpathlineto{\pgfqpoint{3.330581in}{1.888558in}}%
\pgfpathlineto{\pgfqpoint{3.331730in}{1.914892in}}%
\pgfpathlineto{\pgfqpoint{3.332875in}{1.275222in}}%
\pgfpathlineto{\pgfqpoint{3.335157in}{1.902592in}}%
\pgfpathlineto{\pgfqpoint{3.336293in}{1.951651in}}%
\pgfpathlineto{\pgfqpoint{3.337426in}{1.655476in}}%
\pgfpathlineto{\pgfqpoint{3.339683in}{1.969908in}}%
\pgfpathlineto{\pgfqpoint{3.340807in}{1.870641in}}%
\pgfpathlineto{\pgfqpoint{3.343047in}{1.995798in}}%
\pgfpathlineto{\pgfqpoint{3.345274in}{1.769099in}}%
\pgfpathlineto{\pgfqpoint{3.346384in}{1.787029in}}%
\pgfpathlineto{\pgfqpoint{3.348594in}{1.946042in}}%
\pgfpathlineto{\pgfqpoint{3.349695in}{1.888176in}}%
\pgfpathlineto{\pgfqpoint{3.350793in}{1.900620in}}%
\pgfpathlineto{\pgfqpoint{3.351888in}{2.001745in}}%
\pgfpathlineto{\pgfqpoint{3.354070in}{1.692688in}}%
\pgfpathlineto{\pgfqpoint{3.356240in}{2.042399in}}%
\pgfpathlineto{\pgfqpoint{3.357321in}{1.996332in}}%
\pgfpathlineto{\pgfqpoint{3.358400in}{1.998439in}}%
\pgfpathlineto{\pgfqpoint{3.360549in}{1.776448in}}%
\pgfpathlineto{\pgfqpoint{3.361619in}{1.885729in}}%
\pgfpathlineto{\pgfqpoint{3.362686in}{1.861996in}}%
\pgfpathlineto{\pgfqpoint{3.363751in}{1.973864in}}%
\pgfpathlineto{\pgfqpoint{3.364814in}{1.718493in}}%
\pgfpathlineto{\pgfqpoint{3.366930in}{1.925735in}}%
\pgfpathlineto{\pgfqpoint{3.367984in}{1.825436in}}%
\pgfpathlineto{\pgfqpoint{3.369036in}{1.979005in}}%
\pgfpathlineto{\pgfqpoint{3.370085in}{1.838463in}}%
\pgfpathlineto{\pgfqpoint{3.371132in}{1.958398in}}%
\pgfpathlineto{\pgfqpoint{3.372176in}{1.648682in}}%
\pgfpathlineto{\pgfqpoint{3.373217in}{1.927261in}}%
\pgfpathlineto{\pgfqpoint{3.374256in}{1.791115in}}%
\pgfpathlineto{\pgfqpoint{3.376326in}{1.919219in}}%
\pgfpathlineto{\pgfqpoint{3.377357in}{1.990449in}}%
\pgfpathlineto{\pgfqpoint{3.379412in}{1.773201in}}%
\pgfpathlineto{\pgfqpoint{3.381457in}{1.838655in}}%
\pgfpathlineto{\pgfqpoint{3.382476in}{1.908821in}}%
\pgfpathlineto{\pgfqpoint{3.383493in}{1.859843in}}%
\pgfpathlineto{\pgfqpoint{3.384507in}{1.984337in}}%
\pgfpathlineto{\pgfqpoint{3.385518in}{1.955104in}}%
\pgfpathlineto{\pgfqpoint{3.386527in}{1.752370in}}%
\pgfpathlineto{\pgfqpoint{3.387534in}{2.005239in}}%
\pgfpathlineto{\pgfqpoint{3.388539in}{1.881576in}}%
\pgfpathlineto{\pgfqpoint{3.389541in}{1.928842in}}%
\pgfpathlineto{\pgfqpoint{3.390540in}{1.708974in}}%
\pgfpathlineto{\pgfqpoint{3.391538in}{1.964810in}}%
\pgfpathlineto{\pgfqpoint{3.393525in}{1.892403in}}%
\pgfpathlineto{\pgfqpoint{3.394516in}{1.990503in}}%
\pgfpathlineto{\pgfqpoint{3.396490in}{1.904656in}}%
\pgfpathlineto{\pgfqpoint{3.397473in}{1.684412in}}%
\pgfpathlineto{\pgfqpoint{3.398454in}{1.934941in}}%
\pgfpathlineto{\pgfqpoint{3.399433in}{1.793226in}}%
\pgfpathlineto{\pgfqpoint{3.400410in}{1.945950in}}%
\pgfpathlineto{\pgfqpoint{3.402356in}{1.832451in}}%
\pgfpathlineto{\pgfqpoint{3.403326in}{1.908573in}}%
\pgfpathlineto{\pgfqpoint{3.404294in}{1.662747in}}%
\pgfpathlineto{\pgfqpoint{3.405260in}{1.731167in}}%
\pgfpathlineto{\pgfqpoint{3.406223in}{1.669963in}}%
\pgfpathlineto{\pgfqpoint{3.407184in}{1.922077in}}%
\pgfpathlineto{\pgfqpoint{3.408143in}{1.620849in}}%
\pgfpathlineto{\pgfqpoint{3.409100in}{1.907790in}}%
\pgfpathlineto{\pgfqpoint{3.410055in}{1.900840in}}%
\pgfpathlineto{\pgfqpoint{3.411007in}{1.900483in}}%
\pgfpathlineto{\pgfqpoint{3.413852in}{2.012684in}}%
\pgfpathlineto{\pgfqpoint{3.415738in}{1.736469in}}%
\pgfpathlineto{\pgfqpoint{3.416678in}{1.870859in}}%
\pgfpathlineto{\pgfqpoint{3.417616in}{1.732121in}}%
\pgfpathlineto{\pgfqpoint{3.418552in}{1.992136in}}%
\pgfpathlineto{\pgfqpoint{3.419485in}{1.947216in}}%
\pgfpathlineto{\pgfqpoint{3.420417in}{1.826660in}}%
\pgfpathlineto{\pgfqpoint{3.421347in}{2.009843in}}%
\pgfpathlineto{\pgfqpoint{3.423200in}{1.871739in}}%
\pgfpathlineto{\pgfqpoint{3.424123in}{1.928936in}}%
\pgfpathlineto{\pgfqpoint{3.425045in}{1.840754in}}%
\pgfpathlineto{\pgfqpoint{3.426882in}{1.913370in}}%
\pgfpathlineto{\pgfqpoint{3.427797in}{1.909465in}}%
\pgfpathlineto{\pgfqpoint{3.428711in}{1.896816in}}%
\pgfpathlineto{\pgfqpoint{3.429623in}{1.904716in}}%
\pgfpathlineto{\pgfqpoint{3.430532in}{1.959728in}}%
\pgfpathlineto{\pgfqpoint{3.432346in}{1.828879in}}%
\pgfpathlineto{\pgfqpoint{3.434152in}{1.971008in}}%
\pgfpathlineto{\pgfqpoint{3.435052in}{1.875267in}}%
\pgfpathlineto{\pgfqpoint{3.436846in}{2.063922in}}%
\pgfpathlineto{\pgfqpoint{3.438633in}{1.892604in}}%
\pgfpathlineto{\pgfqpoint{3.439523in}{1.911972in}}%
\pgfpathlineto{\pgfqpoint{3.440412in}{2.014775in}}%
\pgfpathlineto{\pgfqpoint{3.441299in}{1.785078in}}%
\pgfpathlineto{\pgfqpoint{3.442184in}{1.871085in}}%
\pgfpathlineto{\pgfqpoint{3.443067in}{1.750316in}}%
\pgfpathlineto{\pgfqpoint{3.443949in}{1.970101in}}%
\pgfpathlineto{\pgfqpoint{3.444828in}{1.835436in}}%
\pgfpathlineto{\pgfqpoint{3.446582in}{1.900303in}}%
\pgfpathlineto{\pgfqpoint{3.449198in}{1.751024in}}%
\pgfpathlineto{\pgfqpoint{3.450934in}{1.949182in}}%
\pgfpathlineto{\pgfqpoint{3.451799in}{1.796665in}}%
\pgfpathlineto{\pgfqpoint{3.452663in}{1.956578in}}%
\pgfpathlineto{\pgfqpoint{3.453524in}{1.924179in}}%
\pgfpathlineto{\pgfqpoint{3.454384in}{1.728054in}}%
\pgfpathlineto{\pgfqpoint{3.455242in}{1.917618in}}%
\pgfpathlineto{\pgfqpoint{3.457807in}{1.717424in}}%
\pgfpathlineto{\pgfqpoint{3.458658in}{1.934541in}}%
\pgfpathlineto{\pgfqpoint{3.459507in}{1.891858in}}%
\pgfpathlineto{\pgfqpoint{3.460355in}{1.706027in}}%
\pgfpathlineto{\pgfqpoint{3.461201in}{1.978548in}}%
\pgfpathlineto{\pgfqpoint{3.462046in}{1.891174in}}%
\pgfpathlineto{\pgfqpoint{3.462889in}{1.919812in}}%
\pgfpathlineto{\pgfqpoint{3.463730in}{1.886047in}}%
\pgfpathlineto{\pgfqpoint{3.464569in}{1.921050in}}%
\pgfpathlineto{\pgfqpoint{3.465407in}{1.830221in}}%
\pgfpathlineto{\pgfqpoint{3.466243in}{1.897700in}}%
\pgfpathlineto{\pgfqpoint{3.467078in}{1.826552in}}%
\pgfpathlineto{\pgfqpoint{3.467911in}{1.872788in}}%
\pgfpathlineto{\pgfqpoint{3.468742in}{1.796228in}}%
\pgfpathlineto{\pgfqpoint{3.470400in}{2.010456in}}%
\pgfpathlineto{\pgfqpoint{3.472051in}{1.943132in}}%
\pgfpathlineto{\pgfqpoint{3.472874in}{1.746315in}}%
\pgfpathlineto{\pgfqpoint{3.473696in}{1.808112in}}%
\pgfpathlineto{\pgfqpoint{3.475334in}{2.020753in}}%
\pgfpathlineto{\pgfqpoint{3.476151in}{2.005358in}}%
\pgfpathlineto{\pgfqpoint{3.476966in}{1.922572in}}%
\pgfpathlineto{\pgfqpoint{3.477780in}{1.940878in}}%
\pgfpathlineto{\pgfqpoint{3.478592in}{2.029536in}}%
\pgfpathlineto{\pgfqpoint{3.479403in}{1.873432in}}%
\pgfpathlineto{\pgfqpoint{3.480212in}{1.975553in}}%
\pgfpathlineto{\pgfqpoint{3.481826in}{1.886462in}}%
\pgfpathlineto{\pgfqpoint{3.482630in}{1.918932in}}%
\pgfpathlineto{\pgfqpoint{3.483433in}{1.913940in}}%
\pgfpathlineto{\pgfqpoint{3.484235in}{1.903702in}}%
\pgfpathlineto{\pgfqpoint{3.485034in}{2.042815in}}%
\pgfpathlineto{\pgfqpoint{3.485833in}{1.854383in}}%
\pgfpathlineto{\pgfqpoint{3.486630in}{1.891183in}}%
\pgfpathlineto{\pgfqpoint{3.487425in}{1.966123in}}%
\pgfpathlineto{\pgfqpoint{3.488219in}{1.948446in}}%
\pgfpathlineto{\pgfqpoint{3.489803in}{1.819562in}}%
\pgfpathlineto{\pgfqpoint{3.490592in}{1.975983in}}%
\pgfpathlineto{\pgfqpoint{3.491380in}{1.869638in}}%
\pgfpathlineto{\pgfqpoint{3.492167in}{1.950797in}}%
\pgfpathlineto{\pgfqpoint{3.495299in}{1.838766in}}%
\pgfpathlineto{\pgfqpoint{3.496856in}{2.030122in}}%
\pgfpathlineto{\pgfqpoint{3.497632in}{1.669256in}}%
\pgfpathlineto{\pgfqpoint{3.498407in}{1.873990in}}%
\pgfpathlineto{\pgfqpoint{3.499181in}{1.872710in}}%
\pgfpathlineto{\pgfqpoint{3.499953in}{1.974516in}}%
\pgfpathlineto{\pgfqpoint{3.500724in}{1.944697in}}%
\pgfpathlineto{\pgfqpoint{3.504558in}{1.708961in}}%
\pgfpathlineto{\pgfqpoint{3.505320in}{1.955001in}}%
\pgfpathlineto{\pgfqpoint{3.506082in}{1.469743in}}%
\pgfpathlineto{\pgfqpoint{3.506841in}{1.915059in}}%
\pgfpathlineto{\pgfqpoint{3.507600in}{1.860707in}}%
\pgfpathlineto{\pgfqpoint{3.509113in}{1.736426in}}%
\pgfpathlineto{\pgfqpoint{3.510620in}{1.993064in}}%
\pgfpathlineto{\pgfqpoint{3.511372in}{1.942291in}}%
\pgfpathlineto{\pgfqpoint{3.512122in}{1.787350in}}%
\pgfpathlineto{\pgfqpoint{3.513619in}{1.969418in}}%
\pgfpathlineto{\pgfqpoint{3.514366in}{1.930734in}}%
\pgfpathlineto{\pgfqpoint{3.515111in}{2.055527in}}%
\pgfpathlineto{\pgfqpoint{3.515855in}{1.954775in}}%
\pgfpathlineto{\pgfqpoint{3.516597in}{1.967590in}}%
\pgfpathlineto{\pgfqpoint{3.517338in}{1.934824in}}%
\pgfpathlineto{\pgfqpoint{3.518817in}{2.035787in}}%
\pgfpathlineto{\pgfqpoint{3.519554in}{1.837077in}}%
\pgfpathlineto{\pgfqpoint{3.520291in}{1.975219in}}%
\pgfpathlineto{\pgfqpoint{3.521025in}{1.859586in}}%
\pgfpathlineto{\pgfqpoint{3.521759in}{1.872476in}}%
\pgfpathlineto{\pgfqpoint{3.522491in}{1.982362in}}%
\pgfpathlineto{\pgfqpoint{3.523222in}{1.851429in}}%
\pgfpathlineto{\pgfqpoint{3.523952in}{1.976022in}}%
\pgfpathlineto{\pgfqpoint{3.524681in}{1.844559in}}%
\pgfpathlineto{\pgfqpoint{3.525408in}{1.863302in}}%
\pgfpathlineto{\pgfqpoint{3.526134in}{1.891700in}}%
\pgfpathlineto{\pgfqpoint{3.527582in}{2.014922in}}%
\pgfpathlineto{\pgfqpoint{3.528305in}{1.994754in}}%
\pgfpathlineto{\pgfqpoint{3.529026in}{2.098140in}}%
\pgfpathlineto{\pgfqpoint{3.529746in}{1.701300in}}%
\pgfpathlineto{\pgfqpoint{3.531182in}{2.015413in}}%
\pgfpathlineto{\pgfqpoint{3.531898in}{1.804517in}}%
\pgfpathlineto{\pgfqpoint{3.532613in}{2.005422in}}%
\pgfpathlineto{\pgfqpoint{3.533327in}{1.814197in}}%
\pgfpathlineto{\pgfqpoint{3.534039in}{1.937201in}}%
\pgfpathlineto{\pgfqpoint{3.535461in}{1.840669in}}%
\pgfpathlineto{\pgfqpoint{3.536170in}{2.016423in}}%
\pgfpathlineto{\pgfqpoint{3.536878in}{1.948912in}}%
\pgfpathlineto{\pgfqpoint{3.537585in}{1.872958in}}%
\pgfpathlineto{\pgfqpoint{3.538290in}{1.951869in}}%
\pgfpathlineto{\pgfqpoint{3.539698in}{1.865155in}}%
\pgfpathlineto{\pgfqpoint{3.540400in}{2.004766in}}%
\pgfpathlineto{\pgfqpoint{3.541800in}{1.871747in}}%
\pgfpathlineto{\pgfqpoint{3.542499in}{1.883806in}}%
\pgfpathlineto{\pgfqpoint{3.543892in}{1.761934in}}%
\pgfpathlineto{\pgfqpoint{3.545281in}{1.986426in}}%
\pgfpathlineto{\pgfqpoint{3.545974in}{1.840166in}}%
\pgfpathlineto{\pgfqpoint{3.547356in}{1.952423in}}%
\pgfpathlineto{\pgfqpoint{3.548046in}{1.857639in}}%
\pgfpathlineto{\pgfqpoint{3.548734in}{1.924413in}}%
\pgfpathlineto{\pgfqpoint{3.549421in}{1.779093in}}%
\pgfpathlineto{\pgfqpoint{3.550108in}{2.002282in}}%
\pgfpathlineto{\pgfqpoint{3.550793in}{1.926746in}}%
\pgfpathlineto{\pgfqpoint{3.551476in}{1.671675in}}%
\pgfpathlineto{\pgfqpoint{3.552159in}{1.795573in}}%
\pgfpathlineto{\pgfqpoint{3.552841in}{1.805781in}}%
\pgfpathlineto{\pgfqpoint{3.554201in}{1.699119in}}%
\pgfpathlineto{\pgfqpoint{3.555557in}{2.028843in}}%
\pgfpathlineto{\pgfqpoint{3.557583in}{1.645681in}}%
\pgfpathlineto{\pgfqpoint{3.558928in}{1.908138in}}%
\pgfpathlineto{\pgfqpoint{3.560937in}{1.940379in}}%
\pgfpathlineto{\pgfqpoint{3.561605in}{2.039775in}}%
\pgfpathlineto{\pgfqpoint{3.562272in}{1.835933in}}%
\pgfpathlineto{\pgfqpoint{3.562937in}{1.950793in}}%
\pgfpathlineto{\pgfqpoint{3.563602in}{1.952645in}}%
\pgfpathlineto{\pgfqpoint{3.564266in}{1.651394in}}%
\pgfpathlineto{\pgfqpoint{3.565590in}{2.056902in}}%
\pgfpathlineto{\pgfqpoint{3.567568in}{1.934813in}}%
\pgfpathlineto{\pgfqpoint{3.568225in}{1.734291in}}%
\pgfpathlineto{\pgfqpoint{3.568882in}{1.857511in}}%
\pgfpathlineto{\pgfqpoint{3.569537in}{1.769255in}}%
\pgfpathlineto{\pgfqpoint{3.571497in}{1.955705in}}%
\pgfpathlineto{\pgfqpoint{3.572799in}{1.870533in}}%
\pgfpathlineto{\pgfqpoint{3.573448in}{2.027095in}}%
\pgfpathlineto{\pgfqpoint{3.574097in}{1.980686in}}%
\pgfpathlineto{\pgfqpoint{3.575391in}{1.893088in}}%
\pgfpathlineto{\pgfqpoint{3.576036in}{2.025474in}}%
\pgfpathlineto{\pgfqpoint{3.576680in}{2.024873in}}%
\pgfpathlineto{\pgfqpoint{3.577966in}{1.862617in}}%
\pgfpathlineto{\pgfqpoint{3.578608in}{1.889188in}}%
\pgfpathlineto{\pgfqpoint{3.579248in}{2.005532in}}%
\pgfpathlineto{\pgfqpoint{3.579888in}{1.993159in}}%
\pgfpathlineto{\pgfqpoint{3.581164in}{1.754367in}}%
\pgfpathlineto{\pgfqpoint{3.582437in}{1.989960in}}%
\pgfpathlineto{\pgfqpoint{3.583072in}{1.969920in}}%
\pgfpathlineto{\pgfqpoint{3.583705in}{1.817009in}}%
\pgfpathlineto{\pgfqpoint{3.584338in}{1.830419in}}%
\pgfpathlineto{\pgfqpoint{3.586231in}{1.994869in}}%
\pgfpathlineto{\pgfqpoint{3.586861in}{1.997953in}}%
\pgfpathlineto{\pgfqpoint{3.588116in}{1.864197in}}%
\pgfpathlineto{\pgfqpoint{3.588742in}{1.945074in}}%
\pgfpathlineto{\pgfqpoint{3.590616in}{1.764027in}}%
\pgfpathlineto{\pgfqpoint{3.592481in}{1.996617in}}%
\pgfpathlineto{\pgfqpoint{3.593101in}{1.854128in}}%
\pgfpathlineto{\pgfqpoint{3.593720in}{1.975080in}}%
\pgfpathlineto{\pgfqpoint{3.594338in}{1.941228in}}%
\pgfpathlineto{\pgfqpoint{3.594956in}{1.939734in}}%
\pgfpathlineto{\pgfqpoint{3.595572in}{1.943564in}}%
\pgfpathlineto{\pgfqpoint{3.596188in}{1.871494in}}%
\pgfpathlineto{\pgfqpoint{3.596802in}{2.029950in}}%
\pgfpathlineto{\pgfqpoint{3.597416in}{1.969164in}}%
\pgfpathlineto{\pgfqpoint{3.598641in}{1.866169in}}%
\pgfpathlineto{\pgfqpoint{3.599252in}{2.018924in}}%
\pgfpathlineto{\pgfqpoint{3.599862in}{1.963417in}}%
\pgfpathlineto{\pgfqpoint{3.600471in}{1.999214in}}%
\pgfpathlineto{\pgfqpoint{3.601079in}{1.917955in}}%
\pgfpathlineto{\pgfqpoint{3.601687in}{2.011234in}}%
\pgfpathlineto{\pgfqpoint{3.602294in}{1.957204in}}%
\pgfpathlineto{\pgfqpoint{3.602899in}{1.780001in}}%
\pgfpathlineto{\pgfqpoint{3.603504in}{2.022060in}}%
\pgfpathlineto{\pgfqpoint{3.604108in}{1.899948in}}%
\pgfpathlineto{\pgfqpoint{3.604712in}{2.002058in}}%
\pgfpathlineto{\pgfqpoint{3.605314in}{1.946506in}}%
\pgfpathlineto{\pgfqpoint{3.605915in}{1.928805in}}%
\pgfpathlineto{\pgfqpoint{3.607116in}{1.722433in}}%
\pgfpathlineto{\pgfqpoint{3.608313in}{1.951220in}}%
\pgfpathlineto{\pgfqpoint{3.610697in}{1.908752in}}%
\pgfpathlineto{\pgfqpoint{3.611291in}{1.992158in}}%
\pgfpathlineto{\pgfqpoint{3.611884in}{1.985497in}}%
\pgfpathlineto{\pgfqpoint{3.612476in}{1.863853in}}%
\pgfpathlineto{\pgfqpoint{3.613068in}{1.869377in}}%
\pgfpathlineto{\pgfqpoint{3.614248in}{2.028016in}}%
\pgfpathlineto{\pgfqpoint{3.614837in}{1.864269in}}%
\pgfpathlineto{\pgfqpoint{3.615425in}{1.970356in}}%
\pgfpathlineto{\pgfqpoint{3.616013in}{2.086068in}}%
\pgfpathlineto{\pgfqpoint{3.616599in}{2.074150in}}%
\pgfpathlineto{\pgfqpoint{3.617185in}{2.033593in}}%
\pgfpathlineto{\pgfqpoint{3.617770in}{1.624723in}}%
\pgfpathlineto{\pgfqpoint{3.618354in}{1.949899in}}%
\pgfpathlineto{\pgfqpoint{3.618937in}{1.977737in}}%
\pgfpathlineto{\pgfqpoint{3.619520in}{1.808252in}}%
\pgfpathlineto{\pgfqpoint{3.620102in}{1.879948in}}%
\pgfpathlineto{\pgfqpoint{3.622421in}{2.017983in}}%
\pgfpathlineto{\pgfqpoint{3.622998in}{1.873922in}}%
\pgfpathlineto{\pgfqpoint{3.623575in}{1.926377in}}%
\pgfpathlineto{\pgfqpoint{3.624151in}{1.863484in}}%
\pgfpathlineto{\pgfqpoint{3.625302in}{2.002458in}}%
\pgfpathlineto{\pgfqpoint{3.626448in}{1.824193in}}%
\pgfpathlineto{\pgfqpoint{3.627592in}{1.964170in}}%
\pgfpathlineto{\pgfqpoint{3.629302in}{1.747336in}}%
\pgfpathlineto{\pgfqpoint{3.629871in}{1.916677in}}%
\pgfpathlineto{\pgfqpoint{3.630438in}{1.722421in}}%
\pgfpathlineto{\pgfqpoint{3.631005in}{1.893735in}}%
\pgfpathlineto{\pgfqpoint{3.631572in}{1.908570in}}%
\pgfpathlineto{\pgfqpoint{3.632137in}{1.868551in}}%
\pgfpathlineto{\pgfqpoint{3.633266in}{2.017024in}}%
\pgfpathlineto{\pgfqpoint{3.633829in}{1.795738in}}%
\pgfpathlineto{\pgfqpoint{3.634391in}{2.044893in}}%
\pgfpathlineto{\pgfqpoint{3.634953in}{1.859895in}}%
\pgfpathlineto{\pgfqpoint{3.635514in}{1.999230in}}%
\pgfpathlineto{\pgfqpoint{3.636074in}{1.952046in}}%
\pgfpathlineto{\pgfqpoint{3.636634in}{1.742457in}}%
\pgfpathlineto{\pgfqpoint{3.637192in}{1.928715in}}%
\pgfpathlineto{\pgfqpoint{3.637750in}{1.823692in}}%
\pgfpathlineto{\pgfqpoint{3.638308in}{1.927141in}}%
\pgfpathlineto{\pgfqpoint{3.638864in}{1.932127in}}%
\pgfpathlineto{\pgfqpoint{3.639420in}{1.970146in}}%
\pgfpathlineto{\pgfqpoint{3.639975in}{1.967540in}}%
\pgfpathlineto{\pgfqpoint{3.641083in}{1.849815in}}%
\pgfpathlineto{\pgfqpoint{3.642740in}{2.025343in}}%
\pgfpathlineto{\pgfqpoint{3.643290in}{1.931742in}}%
\pgfpathlineto{\pgfqpoint{3.643840in}{1.462008in}}%
\pgfpathlineto{\pgfqpoint{3.644938in}{1.947884in}}%
\pgfpathlineto{\pgfqpoint{3.645486in}{1.904270in}}%
\pgfpathlineto{\pgfqpoint{3.646034in}{1.978451in}}%
\pgfpathlineto{\pgfqpoint{3.647126in}{1.662218in}}%
\pgfpathlineto{\pgfqpoint{3.648759in}{2.013436in}}%
\pgfpathlineto{\pgfqpoint{3.649302in}{1.903499in}}%
\pgfpathlineto{\pgfqpoint{3.649844in}{2.036882in}}%
\pgfpathlineto{\pgfqpoint{3.650386in}{1.907496in}}%
\pgfpathlineto{\pgfqpoint{3.652007in}{2.001579in}}%
\pgfpathlineto{\pgfqpoint{3.653084in}{1.698560in}}%
\pgfpathlineto{\pgfqpoint{3.653621in}{2.030032in}}%
\pgfpathlineto{\pgfqpoint{3.654158in}{1.938604in}}%
\pgfpathlineto{\pgfqpoint{3.654694in}{1.964595in}}%
\pgfpathlineto{\pgfqpoint{3.655230in}{1.737708in}}%
\pgfpathlineto{\pgfqpoint{3.655765in}{2.053659in}}%
\pgfpathlineto{\pgfqpoint{3.656299in}{1.965611in}}%
\pgfpathlineto{\pgfqpoint{3.656832in}{1.908659in}}%
\pgfpathlineto{\pgfqpoint{3.657365in}{1.936841in}}%
\pgfpathlineto{\pgfqpoint{3.657897in}{2.016438in}}%
\pgfpathlineto{\pgfqpoint{3.658429in}{1.979745in}}%
\pgfpathlineto{\pgfqpoint{3.658959in}{1.887265in}}%
\pgfpathlineto{\pgfqpoint{3.659489in}{2.032380in}}%
\pgfpathlineto{\pgfqpoint{3.660019in}{1.771039in}}%
\pgfpathlineto{\pgfqpoint{3.660548in}{1.900540in}}%
\pgfpathlineto{\pgfqpoint{3.661076in}{2.063282in}}%
\pgfpathlineto{\pgfqpoint{3.661603in}{1.840862in}}%
\pgfpathlineto{\pgfqpoint{3.662130in}{1.963504in}}%
\pgfpathlineto{\pgfqpoint{3.662656in}{1.924247in}}%
\pgfpathlineto{\pgfqpoint{3.663182in}{2.017400in}}%
\pgfpathlineto{\pgfqpoint{3.663707in}{1.814076in}}%
\pgfpathlineto{\pgfqpoint{3.664231in}{1.981181in}}%
\pgfpathlineto{\pgfqpoint{3.664755in}{1.855681in}}%
\pgfpathlineto{\pgfqpoint{3.665278in}{1.994576in}}%
\pgfpathlineto{\pgfqpoint{3.666842in}{1.871941in}}%
\pgfpathlineto{\pgfqpoint{3.667363in}{1.875392in}}%
\pgfpathlineto{\pgfqpoint{3.668920in}{2.005176in}}%
\pgfpathlineto{\pgfqpoint{3.669955in}{1.851415in}}%
\pgfpathlineto{\pgfqpoint{3.670472in}{2.084962in}}%
\pgfpathlineto{\pgfqpoint{3.670988in}{1.710982in}}%
\pgfpathlineto{\pgfqpoint{3.671503in}{1.806406in}}%
\pgfpathlineto{\pgfqpoint{3.672018in}{1.977800in}}%
\pgfpathlineto{\pgfqpoint{3.672532in}{1.879320in}}%
\pgfpathlineto{\pgfqpoint{3.674582in}{2.044455in}}%
\pgfpathlineto{\pgfqpoint{3.675603in}{1.949544in}}%
\pgfpathlineto{\pgfqpoint{3.676113in}{2.004294in}}%
\pgfpathlineto{\pgfqpoint{3.676622in}{1.932152in}}%
\pgfpathlineto{\pgfqpoint{3.677131in}{2.027489in}}%
\pgfpathlineto{\pgfqpoint{3.677638in}{1.879935in}}%
\pgfpathlineto{\pgfqpoint{3.678146in}{2.035290in}}%
\pgfpathlineto{\pgfqpoint{3.680673in}{1.789352in}}%
\pgfpathlineto{\pgfqpoint{3.681177in}{1.917843in}}%
\pgfpathlineto{\pgfqpoint{3.681680in}{1.884813in}}%
\pgfpathlineto{\pgfqpoint{3.682183in}{1.839513in}}%
\pgfpathlineto{\pgfqpoint{3.682684in}{1.886156in}}%
\pgfpathlineto{\pgfqpoint{3.683186in}{1.958693in}}%
\pgfpathlineto{\pgfqpoint{3.683686in}{1.925619in}}%
\pgfpathlineto{\pgfqpoint{3.684686in}{1.851862in}}%
\pgfpathlineto{\pgfqpoint{3.685185in}{2.053339in}}%
\pgfpathlineto{\pgfqpoint{3.685683in}{1.793643in}}%
\pgfpathlineto{\pgfqpoint{3.686181in}{1.926805in}}%
\pgfpathlineto{\pgfqpoint{3.687175in}{2.018280in}}%
\pgfpathlineto{\pgfqpoint{3.687671in}{1.983838in}}%
\pgfpathlineto{\pgfqpoint{3.689156in}{1.767433in}}%
\pgfpathlineto{\pgfqpoint{3.689650in}{1.782414in}}%
\pgfpathlineto{\pgfqpoint{3.690635in}{2.037213in}}%
\pgfpathlineto{\pgfqpoint{3.691619in}{1.777050in}}%
\pgfpathlineto{\pgfqpoint{3.692600in}{2.049808in}}%
\pgfpathlineto{\pgfqpoint{3.693579in}{1.981896in}}%
\pgfpathlineto{\pgfqpoint{3.694067in}{2.005314in}}%
\pgfpathlineto{\pgfqpoint{3.694556in}{1.976146in}}%
\pgfpathlineto{\pgfqpoint{3.695530in}{1.990038in}}%
\pgfpathlineto{\pgfqpoint{3.696502in}{1.936519in}}%
\pgfpathlineto{\pgfqpoint{3.696987in}{2.003155in}}%
\pgfpathlineto{\pgfqpoint{3.697472in}{1.884584in}}%
\pgfpathlineto{\pgfqpoint{3.697956in}{2.050478in}}%
\pgfpathlineto{\pgfqpoint{3.698923in}{2.022163in}}%
\pgfpathlineto{\pgfqpoint{3.700369in}{1.806716in}}%
\pgfpathlineto{\pgfqpoint{3.701810in}{1.937316in}}%
\pgfpathlineto{\pgfqpoint{3.702768in}{1.954135in}}%
\pgfpathlineto{\pgfqpoint{3.703723in}{1.738072in}}%
\pgfpathlineto{\pgfqpoint{3.704201in}{1.748401in}}%
\pgfpathlineto{\pgfqpoint{3.704677in}{2.022278in}}%
\pgfpathlineto{\pgfqpoint{3.705629in}{2.004486in}}%
\pgfpathlineto{\pgfqpoint{3.706578in}{1.781488in}}%
\pgfpathlineto{\pgfqpoint{3.707525in}{2.039214in}}%
\pgfpathlineto{\pgfqpoint{3.707998in}{1.843063in}}%
\pgfpathlineto{\pgfqpoint{3.708470in}{1.917057in}}%
\pgfpathlineto{\pgfqpoint{3.708942in}{1.939330in}}%
\pgfpathlineto{\pgfqpoint{3.709413in}{1.641150in}}%
\pgfpathlineto{\pgfqpoint{3.709884in}{1.966930in}}%
\pgfpathlineto{\pgfqpoint{3.710824in}{1.739892in}}%
\pgfpathlineto{\pgfqpoint{3.711293in}{1.884891in}}%
\pgfpathlineto{\pgfqpoint{3.711762in}{2.002695in}}%
\pgfpathlineto{\pgfqpoint{3.712230in}{1.838667in}}%
\pgfpathlineto{\pgfqpoint{3.712698in}{2.019493in}}%
\pgfpathlineto{\pgfqpoint{3.713165in}{1.872117in}}%
\pgfpathlineto{\pgfqpoint{3.713631in}{1.845916in}}%
\pgfpathlineto{\pgfqpoint{3.714097in}{1.996300in}}%
\pgfpathlineto{\pgfqpoint{3.715028in}{1.962036in}}%
\pgfpathlineto{\pgfqpoint{3.715492in}{2.025879in}}%
\pgfpathlineto{\pgfqpoint{3.715956in}{1.857652in}}%
\pgfpathlineto{\pgfqpoint{3.716883in}{1.879900in}}%
\pgfpathlineto{\pgfqpoint{3.717808in}{1.974560in}}%
\pgfpathlineto{\pgfqpoint{3.718269in}{1.887242in}}%
\pgfpathlineto{\pgfqpoint{3.718730in}{1.942478in}}%
\pgfpathlineto{\pgfqpoint{3.719191in}{2.043940in}}%
\pgfpathlineto{\pgfqpoint{3.719651in}{1.898002in}}%
\pgfpathlineto{\pgfqpoint{3.720569in}{1.941505in}}%
\pgfpathlineto{\pgfqpoint{3.721028in}{1.975017in}}%
\pgfpathlineto{\pgfqpoint{3.721486in}{1.881965in}}%
\pgfpathlineto{\pgfqpoint{3.721943in}{2.028066in}}%
\pgfpathlineto{\pgfqpoint{3.722400in}{1.937050in}}%
\pgfpathlineto{\pgfqpoint{3.723313in}{2.042530in}}%
\pgfpathlineto{\pgfqpoint{3.724224in}{1.987159in}}%
\pgfpathlineto{\pgfqpoint{3.725132in}{1.831073in}}%
\pgfpathlineto{\pgfqpoint{3.725586in}{1.880196in}}%
\pgfpathlineto{\pgfqpoint{3.726039in}{1.889801in}}%
\pgfpathlineto{\pgfqpoint{3.726492in}{1.813885in}}%
\pgfpathlineto{\pgfqpoint{3.727847in}{2.044496in}}%
\pgfpathlineto{\pgfqpoint{3.729197in}{1.938276in}}%
\pgfpathlineto{\pgfqpoint{3.729647in}{1.981570in}}%
\pgfpathlineto{\pgfqpoint{3.731439in}{1.740897in}}%
\pgfpathlineto{\pgfqpoint{3.732333in}{2.009245in}}%
\pgfpathlineto{\pgfqpoint{3.733224in}{2.045910in}}%
\pgfpathlineto{\pgfqpoint{3.733669in}{1.831446in}}%
\pgfpathlineto{\pgfqpoint{3.735002in}{2.069860in}}%
\pgfpathlineto{\pgfqpoint{3.736772in}{1.836147in}}%
\pgfpathlineto{\pgfqpoint{3.737213in}{1.874044in}}%
\pgfpathlineto{\pgfqpoint{3.737654in}{2.053833in}}%
\pgfpathlineto{\pgfqpoint{3.738094in}{2.008413in}}%
\pgfpathlineto{\pgfqpoint{3.738534in}{1.898781in}}%
\pgfpathlineto{\pgfqpoint{3.739413in}{1.926677in}}%
\pgfpathlineto{\pgfqpoint{3.739851in}{2.053785in}}%
\pgfpathlineto{\pgfqpoint{3.740290in}{2.041615in}}%
\pgfpathlineto{\pgfqpoint{3.740727in}{1.906055in}}%
\pgfpathlineto{\pgfqpoint{3.741165in}{2.097764in}}%
\pgfpathlineto{\pgfqpoint{3.742474in}{1.748521in}}%
\pgfpathlineto{\pgfqpoint{3.743779in}{1.974368in}}%
\pgfpathlineto{\pgfqpoint{3.744213in}{1.664819in}}%
\pgfpathlineto{\pgfqpoint{3.744647in}{2.000146in}}%
\pgfpathlineto{\pgfqpoint{3.745080in}{1.946489in}}%
\pgfpathlineto{\pgfqpoint{3.745945in}{1.954000in}}%
\pgfpathlineto{\pgfqpoint{3.746377in}{1.972846in}}%
\pgfpathlineto{\pgfqpoint{3.746808in}{2.070047in}}%
\pgfpathlineto{\pgfqpoint{3.747239in}{1.994973in}}%
\pgfpathlineto{\pgfqpoint{3.747670in}{1.786054in}}%
\pgfpathlineto{\pgfqpoint{3.748100in}{1.987926in}}%
\pgfpathlineto{\pgfqpoint{3.748959in}{1.961529in}}%
\pgfpathlineto{\pgfqpoint{3.749388in}{1.810629in}}%
\pgfpathlineto{\pgfqpoint{3.749817in}{1.987703in}}%
\pgfpathlineto{\pgfqpoint{3.750245in}{2.021004in}}%
\pgfpathlineto{\pgfqpoint{3.751099in}{2.078200in}}%
\pgfpathlineto{\pgfqpoint{3.751526in}{1.893064in}}%
\pgfpathlineto{\pgfqpoint{3.751952in}{1.984105in}}%
\pgfpathlineto{\pgfqpoint{3.752378in}{1.975644in}}%
\pgfpathlineto{\pgfqpoint{3.752804in}{1.753582in}}%
\pgfpathlineto{\pgfqpoint{3.753229in}{1.931999in}}%
\pgfpathlineto{\pgfqpoint{3.754077in}{1.953697in}}%
\pgfpathlineto{\pgfqpoint{3.754501in}{2.061241in}}%
\pgfpathlineto{\pgfqpoint{3.754924in}{1.929665in}}%
\pgfpathlineto{\pgfqpoint{3.755347in}{1.880505in}}%
\pgfpathlineto{\pgfqpoint{3.756192in}{2.072276in}}%
\pgfpathlineto{\pgfqpoint{3.756613in}{1.960425in}}%
\pgfpathlineto{\pgfqpoint{3.757035in}{1.797078in}}%
\pgfpathlineto{\pgfqpoint{3.757455in}{2.001630in}}%
\pgfpathlineto{\pgfqpoint{3.758296in}{2.042486in}}%
\pgfpathlineto{\pgfqpoint{3.758715in}{1.933553in}}%
\pgfpathlineto{\pgfqpoint{3.759134in}{2.001141in}}%
\pgfpathlineto{\pgfqpoint{3.759553in}{1.894544in}}%
\pgfpathlineto{\pgfqpoint{3.759971in}{1.935600in}}%
\pgfpathlineto{\pgfqpoint{3.760389in}{1.789852in}}%
\pgfpathlineto{\pgfqpoint{3.760807in}{1.864325in}}%
\pgfpathlineto{\pgfqpoint{3.761640in}{1.973659in}}%
\pgfpathlineto{\pgfqpoint{3.762057in}{1.884392in}}%
\pgfpathlineto{\pgfqpoint{3.762888in}{1.904160in}}%
\pgfpathlineto{\pgfqpoint{3.763303in}{1.872460in}}%
\pgfpathlineto{\pgfqpoint{3.763718in}{1.961617in}}%
\pgfpathlineto{\pgfqpoint{3.764132in}{1.950070in}}%
\pgfpathlineto{\pgfqpoint{3.764546in}{1.843415in}}%
\pgfpathlineto{\pgfqpoint{3.764959in}{2.063718in}}%
\pgfpathlineto{\pgfqpoint{3.765372in}{1.925436in}}%
\pgfpathlineto{\pgfqpoint{3.765785in}{1.562124in}}%
\pgfpathlineto{\pgfqpoint{3.766197in}{2.031743in}}%
\pgfpathlineto{\pgfqpoint{3.767431in}{1.936555in}}%
\pgfpathlineto{\pgfqpoint{3.767842in}{1.973634in}}%
\pgfpathlineto{\pgfqpoint{3.768252in}{1.743026in}}%
\pgfpathlineto{\pgfqpoint{3.768662in}{1.937141in}}%
\pgfpathlineto{\pgfqpoint{3.769071in}{2.044413in}}%
\pgfpathlineto{\pgfqpoint{3.769480in}{1.756939in}}%
\pgfpathlineto{\pgfqpoint{3.769889in}{1.765914in}}%
\pgfpathlineto{\pgfqpoint{3.770297in}{2.016926in}}%
\pgfpathlineto{\pgfqpoint{3.771112in}{1.960574in}}%
\pgfpathlineto{\pgfqpoint{3.771926in}{1.949440in}}%
\pgfpathlineto{\pgfqpoint{3.772738in}{2.013013in}}%
\pgfpathlineto{\pgfqpoint{3.773144in}{1.844517in}}%
\pgfpathlineto{\pgfqpoint{3.773549in}{1.879259in}}%
\pgfpathlineto{\pgfqpoint{3.773953in}{2.086402in}}%
\pgfpathlineto{\pgfqpoint{3.774762in}{2.031873in}}%
\pgfpathlineto{\pgfqpoint{3.775165in}{1.672585in}}%
\pgfpathlineto{\pgfqpoint{3.775971in}{1.692599in}}%
\pgfpathlineto{\pgfqpoint{3.776776in}{2.006368in}}%
\pgfpathlineto{\pgfqpoint{3.777178in}{1.963110in}}%
\pgfpathlineto{\pgfqpoint{3.777579in}{1.880234in}}%
\pgfpathlineto{\pgfqpoint{3.778380in}{1.888200in}}%
\pgfpathlineto{\pgfqpoint{3.778780in}{1.896935in}}%
\pgfpathlineto{\pgfqpoint{3.779180in}{1.989614in}}%
\pgfpathlineto{\pgfqpoint{3.779580in}{1.976795in}}%
\pgfpathlineto{\pgfqpoint{3.779979in}{1.729528in}}%
\pgfpathlineto{\pgfqpoint{3.780377in}{1.919753in}}%
\pgfpathlineto{\pgfqpoint{3.781173in}{1.873069in}}%
\pgfpathlineto{\pgfqpoint{3.781571in}{2.042517in}}%
\pgfpathlineto{\pgfqpoint{3.782365in}{2.054441in}}%
\pgfpathlineto{\pgfqpoint{3.783157in}{1.891715in}}%
\pgfpathlineto{\pgfqpoint{3.783553in}{2.043656in}}%
\pgfpathlineto{\pgfqpoint{3.783948in}{1.887655in}}%
\pgfpathlineto{\pgfqpoint{3.784343in}{1.932953in}}%
\pgfpathlineto{\pgfqpoint{3.784738in}{1.871749in}}%
\pgfpathlineto{\pgfqpoint{3.785132in}{2.052336in}}%
\pgfpathlineto{\pgfqpoint{3.785526in}{2.003848in}}%
\pgfpathlineto{\pgfqpoint{3.785919in}{1.834706in}}%
\pgfpathlineto{\pgfqpoint{3.786313in}{1.995301in}}%
\pgfpathlineto{\pgfqpoint{3.786705in}{1.976186in}}%
\pgfpathlineto{\pgfqpoint{3.787098in}{1.771895in}}%
\pgfpathlineto{\pgfqpoint{3.787490in}{2.094254in}}%
\pgfpathlineto{\pgfqpoint{3.787881in}{1.969839in}}%
\pgfpathlineto{\pgfqpoint{3.788664in}{2.041029in}}%
\pgfpathlineto{\pgfqpoint{3.789054in}{2.036267in}}%
\pgfpathlineto{\pgfqpoint{3.789834in}{1.878323in}}%
\pgfpathlineto{\pgfqpoint{3.790224in}{2.040001in}}%
\pgfpathlineto{\pgfqpoint{3.791002in}{1.962841in}}%
\pgfpathlineto{\pgfqpoint{3.791390in}{1.901993in}}%
\pgfpathlineto{\pgfqpoint{3.791778in}{1.934177in}}%
\pgfpathlineto{\pgfqpoint{3.793327in}{2.029614in}}%
\pgfpathlineto{\pgfqpoint{3.793713in}{1.838755in}}%
\pgfpathlineto{\pgfqpoint{3.794099in}{2.009179in}}%
\pgfpathlineto{\pgfqpoint{3.794485in}{2.073236in}}%
\pgfpathlineto{\pgfqpoint{3.796024in}{1.732211in}}%
\pgfpathlineto{\pgfqpoint{3.796408in}{2.024776in}}%
\pgfpathlineto{\pgfqpoint{3.797174in}{1.885960in}}%
\pgfpathlineto{\pgfqpoint{3.797557in}{1.571545in}}%
\pgfpathlineto{\pgfqpoint{3.797940in}{1.909897in}}%
\pgfpathlineto{\pgfqpoint{3.798322in}{2.061869in}}%
\pgfpathlineto{\pgfqpoint{3.798704in}{1.903826in}}%
\pgfpathlineto{\pgfqpoint{3.799085in}{1.825812in}}%
\pgfpathlineto{\pgfqpoint{3.799466in}{2.079104in}}%
\pgfpathlineto{\pgfqpoint{3.800227in}{1.991103in}}%
\pgfpathlineto{\pgfqpoint{3.800607in}{2.042514in}}%
\pgfpathlineto{\pgfqpoint{3.800987in}{1.881459in}}%
\pgfpathlineto{\pgfqpoint{3.801746in}{1.901053in}}%
\pgfpathlineto{\pgfqpoint{3.802881in}{2.003753in}}%
\pgfpathlineto{\pgfqpoint{3.803258in}{1.828385in}}%
\pgfpathlineto{\pgfqpoint{3.803636in}{1.967213in}}%
\pgfpathlineto{\pgfqpoint{3.804013in}{1.999789in}}%
\pgfpathlineto{\pgfqpoint{3.804390in}{1.939533in}}%
\pgfpathlineto{\pgfqpoint{3.805142in}{1.949895in}}%
\pgfpathlineto{\pgfqpoint{3.805518in}{1.921863in}}%
\pgfpathlineto{\pgfqpoint{3.805893in}{1.720699in}}%
\pgfpathlineto{\pgfqpoint{3.806643in}{1.867762in}}%
\pgfpathlineto{\pgfqpoint{3.807765in}{2.028078in}}%
\pgfpathlineto{\pgfqpoint{3.808138in}{1.732382in}}%
\pgfpathlineto{\pgfqpoint{3.808884in}{2.021303in}}%
\pgfpathlineto{\pgfqpoint{3.809629in}{2.022997in}}%
\pgfpathlineto{\pgfqpoint{3.810372in}{1.717540in}}%
\pgfpathlineto{\pgfqpoint{3.811484in}{1.994767in}}%
\pgfpathlineto{\pgfqpoint{3.811854in}{1.840174in}}%
\pgfpathlineto{\pgfqpoint{3.812594in}{1.934567in}}%
\pgfpathlineto{\pgfqpoint{3.813332in}{1.992695in}}%
\pgfpathlineto{\pgfqpoint{3.814436in}{1.742983in}}%
\pgfpathlineto{\pgfqpoint{3.815171in}{2.037088in}}%
\pgfpathlineto{\pgfqpoint{3.815538in}{1.873765in}}%
\pgfpathlineto{\pgfqpoint{3.817003in}{1.991129in}}%
\pgfpathlineto{\pgfqpoint{3.817368in}{1.976489in}}%
\pgfpathlineto{\pgfqpoint{3.817733in}{1.979308in}}%
\pgfpathlineto{\pgfqpoint{3.818462in}{2.053552in}}%
\pgfpathlineto{\pgfqpoint{3.819190in}{2.024214in}}%
\pgfpathlineto{\pgfqpoint{3.819554in}{2.044123in}}%
\pgfpathlineto{\pgfqpoint{3.819917in}{1.905091in}}%
\pgfpathlineto{\pgfqpoint{3.820280in}{2.011313in}}%
\pgfpathlineto{\pgfqpoint{3.820642in}{2.074583in}}%
\pgfpathlineto{\pgfqpoint{3.821004in}{2.037769in}}%
\pgfpathlineto{\pgfqpoint{3.821366in}{2.002914in}}%
\pgfpathlineto{\pgfqpoint{3.821728in}{2.086419in}}%
\pgfpathlineto{\pgfqpoint{3.822089in}{1.932701in}}%
\pgfpathlineto{\pgfqpoint{3.822450in}{1.898844in}}%
\pgfpathlineto{\pgfqpoint{3.822811in}{1.918607in}}%
\pgfpathlineto{\pgfqpoint{3.823171in}{2.000524in}}%
\pgfpathlineto{\pgfqpoint{3.823891in}{1.963092in}}%
\pgfpathlineto{\pgfqpoint{3.824251in}{1.766255in}}%
\pgfpathlineto{\pgfqpoint{3.824610in}{2.052832in}}%
\pgfpathlineto{\pgfqpoint{3.824969in}{1.895041in}}%
\pgfpathlineto{\pgfqpoint{3.825327in}{1.980093in}}%
\pgfpathlineto{\pgfqpoint{3.825686in}{1.630174in}}%
\pgfpathlineto{\pgfqpoint{3.826401in}{1.922902in}}%
\pgfpathlineto{\pgfqpoint{3.826759in}{1.906400in}}%
\pgfpathlineto{\pgfqpoint{3.827472in}{2.080104in}}%
\pgfpathlineto{\pgfqpoint{3.827829in}{1.973169in}}%
\pgfpathlineto{\pgfqpoint{3.828896in}{2.073370in}}%
\pgfpathlineto{\pgfqpoint{3.828541in}{1.957846in}}%
\pgfpathlineto{\pgfqpoint{3.829252in}{2.049317in}}%
\pgfpathlineto{\pgfqpoint{3.829607in}{1.871205in}}%
\pgfpathlineto{\pgfqpoint{3.830316in}{1.971366in}}%
\pgfpathlineto{\pgfqpoint{3.830670in}{1.967365in}}%
\pgfpathlineto{\pgfqpoint{3.831024in}{2.112502in}}%
\pgfpathlineto{\pgfqpoint{3.831377in}{1.784308in}}%
\pgfpathlineto{\pgfqpoint{3.832083in}{2.045703in}}%
\pgfpathlineto{\pgfqpoint{3.832788in}{1.815099in}}%
\pgfpathlineto{\pgfqpoint{3.833140in}{2.112810in}}%
\pgfpathlineto{\pgfqpoint{3.833843in}{1.954462in}}%
\pgfpathlineto{\pgfqpoint{3.834545in}{2.034249in}}%
\pgfpathlineto{\pgfqpoint{3.834896in}{1.543659in}}%
\pgfpathlineto{\pgfqpoint{3.836295in}{2.032783in}}%
\pgfpathlineto{\pgfqpoint{3.836993in}{2.059268in}}%
\pgfpathlineto{\pgfqpoint{3.837690in}{1.707643in}}%
\pgfpathlineto{\pgfqpoint{3.838733in}{2.086402in}}%
\pgfpathlineto{\pgfqpoint{3.839080in}{2.073541in}}%
\pgfpathlineto{\pgfqpoint{3.839427in}{1.960403in}}%
\pgfpathlineto{\pgfqpoint{3.840120in}{1.963076in}}%
\pgfpathlineto{\pgfqpoint{3.840466in}{2.008172in}}%
\pgfpathlineto{\pgfqpoint{3.840812in}{1.737851in}}%
\pgfpathlineto{\pgfqpoint{3.841157in}{1.978108in}}%
\pgfpathlineto{\pgfqpoint{3.841502in}{2.075221in}}%
\pgfpathlineto{\pgfqpoint{3.841847in}{1.869728in}}%
\pgfpathlineto{\pgfqpoint{3.842192in}{1.989086in}}%
\pgfpathlineto{\pgfqpoint{3.842536in}{1.856961in}}%
\pgfpathlineto{\pgfqpoint{3.842880in}{2.067512in}}%
\pgfpathlineto{\pgfqpoint{3.843224in}{1.961953in}}%
\pgfpathlineto{\pgfqpoint{3.843910in}{1.998610in}}%
\pgfpathlineto{\pgfqpoint{3.844253in}{1.899863in}}%
\pgfpathlineto{\pgfqpoint{3.844596in}{1.904109in}}%
\pgfpathlineto{\pgfqpoint{3.844938in}{2.042085in}}%
\pgfpathlineto{\pgfqpoint{3.845964in}{2.041066in}}%
\pgfpathlineto{\pgfqpoint{3.846305in}{1.995108in}}%
\pgfpathlineto{\pgfqpoint{3.846987in}{2.020440in}}%
\pgfpathlineto{\pgfqpoint{3.847327in}{2.043139in}}%
\pgfpathlineto{\pgfqpoint{3.848347in}{1.821047in}}%
\pgfpathlineto{\pgfqpoint{3.848686in}{2.045572in}}%
\pgfpathlineto{\pgfqpoint{3.849364in}{2.009443in}}%
\pgfpathlineto{\pgfqpoint{3.850379in}{1.957386in}}%
\pgfpathlineto{\pgfqpoint{3.850717in}{2.096795in}}%
\pgfpathlineto{\pgfqpoint{3.851391in}{1.982448in}}%
\pgfpathlineto{\pgfqpoint{3.851728in}{1.995034in}}%
\pgfpathlineto{\pgfqpoint{3.852065in}{1.913141in}}%
\pgfpathlineto{\pgfqpoint{3.853073in}{1.927004in}}%
\pgfpathlineto{\pgfqpoint{3.853409in}{1.967570in}}%
\pgfpathlineto{\pgfqpoint{3.853744in}{1.853620in}}%
\pgfpathlineto{\pgfqpoint{3.854414in}{1.970860in}}%
\pgfpathlineto{\pgfqpoint{3.854749in}{1.989044in}}%
\pgfpathlineto{\pgfqpoint{3.855751in}{1.854402in}}%
\pgfpathlineto{\pgfqpoint{3.856084in}{1.944524in}}%
\pgfpathlineto{\pgfqpoint{3.856750in}{1.917931in}}%
\pgfpathlineto{\pgfqpoint{3.857083in}{2.046986in}}%
\pgfpathlineto{\pgfqpoint{3.858080in}{1.839859in}}%
\pgfpathlineto{\pgfqpoint{3.858411in}{1.871961in}}%
\pgfpathlineto{\pgfqpoint{3.859735in}{2.094866in}}%
\pgfpathlineto{\pgfqpoint{3.859405in}{1.852352in}}%
\pgfpathlineto{\pgfqpoint{3.860066in}{1.990593in}}%
\pgfpathlineto{\pgfqpoint{3.861385in}{1.916500in}}%
\pgfpathlineto{\pgfqpoint{3.862043in}{2.038798in}}%
\pgfpathlineto{\pgfqpoint{3.862371in}{1.805872in}}%
\pgfpathlineto{\pgfqpoint{3.863355in}{1.857587in}}%
\pgfpathlineto{\pgfqpoint{3.864664in}{2.108696in}}%
\pgfpathlineto{\pgfqpoint{3.864991in}{1.863449in}}%
\pgfpathlineto{\pgfqpoint{3.865643in}{2.004622in}}%
\pgfpathlineto{\pgfqpoint{3.866620in}{2.024335in}}%
\pgfpathlineto{\pgfqpoint{3.866945in}{2.037247in}}%
\pgfpathlineto{\pgfqpoint{3.867594in}{1.935132in}}%
\pgfpathlineto{\pgfqpoint{3.867918in}{1.945957in}}%
\pgfpathlineto{\pgfqpoint{3.868243in}{2.046440in}}%
\pgfpathlineto{\pgfqpoint{3.868890in}{1.920260in}}%
\pgfpathlineto{\pgfqpoint{3.869213in}{1.829183in}}%
\pgfpathlineto{\pgfqpoint{3.869536in}{1.837609in}}%
\pgfpathlineto{\pgfqpoint{3.870182in}{2.093315in}}%
\pgfpathlineto{\pgfqpoint{3.870504in}{2.058485in}}%
\pgfpathlineto{\pgfqpoint{3.871470in}{1.858016in}}%
\pgfpathlineto{\pgfqpoint{3.871148in}{2.075679in}}%
\pgfpathlineto{\pgfqpoint{3.871791in}{1.912657in}}%
\pgfpathlineto{\pgfqpoint{3.872112in}{1.993902in}}%
\pgfpathlineto{\pgfqpoint{3.872433in}{1.989109in}}%
\pgfpathlineto{\pgfqpoint{3.872754in}{1.761205in}}%
\pgfpathlineto{\pgfqpoint{3.873394in}{1.974407in}}%
\pgfpathlineto{\pgfqpoint{3.873714in}{1.984900in}}%
\pgfpathlineto{\pgfqpoint{3.874034in}{1.976318in}}%
\pgfpathlineto{\pgfqpoint{3.874353in}{1.788000in}}%
\pgfpathlineto{\pgfqpoint{3.874672in}{2.091113in}}%
\pgfpathlineto{\pgfqpoint{3.874991in}{1.940278in}}%
\pgfpathlineto{\pgfqpoint{3.875310in}{1.927620in}}%
\pgfpathlineto{\pgfqpoint{3.875629in}{2.066766in}}%
\pgfpathlineto{\pgfqpoint{3.876583in}{2.038698in}}%
\pgfpathlineto{\pgfqpoint{3.876900in}{2.072106in}}%
\pgfpathlineto{\pgfqpoint{3.877217in}{2.008465in}}%
\pgfpathlineto{\pgfqpoint{3.877534in}{2.020558in}}%
\pgfpathlineto{\pgfqpoint{3.877851in}{1.996097in}}%
\pgfpathlineto{\pgfqpoint{3.878168in}{1.937182in}}%
\pgfpathlineto{\pgfqpoint{3.878484in}{2.105891in}}%
\pgfpathlineto{\pgfqpoint{3.878800in}{1.885064in}}%
\pgfpathlineto{\pgfqpoint{3.879116in}{2.068586in}}%
\pgfpathlineto{\pgfqpoint{3.880377in}{1.794640in}}%
\pgfpathlineto{\pgfqpoint{3.881321in}{2.014998in}}%
\pgfpathlineto{\pgfqpoint{3.881635in}{2.011593in}}%
\pgfpathlineto{\pgfqpoint{3.881948in}{1.795948in}}%
\pgfpathlineto{\pgfqpoint{3.882888in}{1.908757in}}%
\pgfpathlineto{\pgfqpoint{3.883201in}{1.895940in}}%
\pgfpathlineto{\pgfqpoint{3.883826in}{1.790214in}}%
\pgfpathlineto{\pgfqpoint{3.884138in}{1.992103in}}%
\pgfpathlineto{\pgfqpoint{3.885073in}{1.721649in}}%
\pgfpathlineto{\pgfqpoint{3.885695in}{1.995542in}}%
\pgfpathlineto{\pgfqpoint{3.886006in}{1.962904in}}%
\pgfpathlineto{\pgfqpoint{3.886317in}{1.780844in}}%
\pgfpathlineto{\pgfqpoint{3.886627in}{1.897468in}}%
\pgfpathlineto{\pgfqpoint{3.886937in}{2.103566in}}%
\pgfpathlineto{\pgfqpoint{3.887866in}{1.990642in}}%
\pgfpathlineto{\pgfqpoint{3.888175in}{1.862350in}}%
\pgfpathlineto{\pgfqpoint{3.888793in}{1.893325in}}%
\pgfpathlineto{\pgfqpoint{3.890026in}{2.027064in}}%
\pgfpathlineto{\pgfqpoint{3.890333in}{2.001006in}}%
\pgfpathlineto{\pgfqpoint{3.890641in}{2.108939in}}%
\pgfpathlineto{\pgfqpoint{3.890948in}{1.985754in}}%
\pgfpathlineto{\pgfqpoint{3.891255in}{2.044144in}}%
\pgfpathlineto{\pgfqpoint{3.891868in}{1.771007in}}%
\pgfpathlineto{\pgfqpoint{3.892174in}{2.058182in}}%
\pgfpathlineto{\pgfqpoint{3.892480in}{2.111869in}}%
\pgfpathlineto{\pgfqpoint{3.892786in}{2.021214in}}%
\pgfpathlineto{\pgfqpoint{3.893092in}{1.983271in}}%
\pgfpathlineto{\pgfqpoint{3.893397in}{1.803629in}}%
\pgfpathlineto{\pgfqpoint{3.894007in}{1.991618in}}%
\pgfpathlineto{\pgfqpoint{3.894312in}{1.852440in}}%
\pgfpathlineto{\pgfqpoint{3.894921in}{1.991647in}}%
\pgfpathlineto{\pgfqpoint{3.895529in}{1.903975in}}%
\pgfpathlineto{\pgfqpoint{3.895833in}{1.820265in}}%
\pgfpathlineto{\pgfqpoint{3.896136in}{2.054238in}}%
\pgfpathlineto{\pgfqpoint{3.897045in}{1.998053in}}%
\pgfpathlineto{\pgfqpoint{3.897348in}{2.070158in}}%
\pgfpathlineto{\pgfqpoint{3.897650in}{1.907989in}}%
\pgfpathlineto{\pgfqpoint{3.897952in}{2.035545in}}%
\pgfpathlineto{\pgfqpoint{3.898254in}{1.774242in}}%
\pgfpathlineto{\pgfqpoint{3.899159in}{1.819465in}}%
\pgfpathlineto{\pgfqpoint{3.900061in}{1.973239in}}%
\pgfpathlineto{\pgfqpoint{3.900362in}{1.936780in}}%
\pgfpathlineto{\pgfqpoint{3.900962in}{1.880485in}}%
\pgfpathlineto{\pgfqpoint{3.901561in}{2.136572in}}%
\pgfpathlineto{\pgfqpoint{3.901861in}{1.903732in}}%
\pgfpathlineto{\pgfqpoint{3.902160in}{1.829333in}}%
\pgfpathlineto{\pgfqpoint{3.902459in}{2.068264in}}%
\pgfpathlineto{\pgfqpoint{3.903354in}{1.917630in}}%
\pgfpathlineto{\pgfqpoint{3.903652in}{1.892349in}}%
\pgfpathlineto{\pgfqpoint{3.903950in}{1.943210in}}%
\pgfpathlineto{\pgfqpoint{3.904546in}{1.936714in}}%
\pgfpathlineto{\pgfqpoint{3.904843in}{2.052378in}}%
\pgfpathlineto{\pgfqpoint{3.906030in}{1.834748in}}%
\pgfpathlineto{\pgfqpoint{3.906326in}{2.051750in}}%
\pgfpathlineto{\pgfqpoint{3.907214in}{1.965821in}}%
\pgfpathlineto{\pgfqpoint{3.907509in}{1.934815in}}%
\pgfpathlineto{\pgfqpoint{3.907804in}{1.711654in}}%
\pgfpathlineto{\pgfqpoint{3.908099in}{2.063460in}}%
\pgfpathlineto{\pgfqpoint{3.908394in}{1.915793in}}%
\pgfpathlineto{\pgfqpoint{3.908689in}{2.113100in}}%
\pgfpathlineto{\pgfqpoint{3.909571in}{2.035414in}}%
\pgfpathlineto{\pgfqpoint{3.910159in}{1.903964in}}%
\pgfpathlineto{\pgfqpoint{3.910452in}{2.071151in}}%
\pgfpathlineto{\pgfqpoint{3.911331in}{1.954872in}}%
\pgfpathlineto{\pgfqpoint{3.911623in}{2.011317in}}%
\pgfpathlineto{\pgfqpoint{3.911916in}{1.958604in}}%
\pgfpathlineto{\pgfqpoint{3.912208in}{1.845726in}}%
\pgfpathlineto{\pgfqpoint{3.912500in}{2.019372in}}%
\pgfpathlineto{\pgfqpoint{3.912792in}{1.936270in}}%
\pgfpathlineto{\pgfqpoint{3.913375in}{2.073212in}}%
\pgfpathlineto{\pgfqpoint{3.913957in}{1.961081in}}%
\pgfpathlineto{\pgfqpoint{3.914247in}{1.711668in}}%
\pgfpathlineto{\pgfqpoint{3.914538in}{2.012393in}}%
\pgfpathlineto{\pgfqpoint{3.914828in}{1.944805in}}%
\pgfpathlineto{\pgfqpoint{3.915988in}{2.074563in}}%
\pgfpathlineto{\pgfqpoint{3.916277in}{1.951307in}}%
\pgfpathlineto{\pgfqpoint{3.917144in}{2.040681in}}%
\pgfpathlineto{\pgfqpoint{3.917433in}{1.953390in}}%
\pgfpathlineto{\pgfqpoint{3.918297in}{2.001516in}}%
\pgfpathlineto{\pgfqpoint{3.918585in}{2.063389in}}%
\pgfpathlineto{\pgfqpoint{3.918873in}{1.918800in}}%
\pgfpathlineto{\pgfqpoint{3.919160in}{1.998796in}}%
\pgfpathlineto{\pgfqpoint{3.919447in}{1.859573in}}%
\pgfpathlineto{\pgfqpoint{3.920021in}{2.102532in}}%
\pgfpathlineto{\pgfqpoint{3.920308in}{1.868951in}}%
\pgfpathlineto{\pgfqpoint{3.920594in}{2.037689in}}%
\pgfpathlineto{\pgfqpoint{3.921167in}{1.841525in}}%
\pgfpathlineto{\pgfqpoint{3.921452in}{1.939600in}}%
\pgfpathlineto{\pgfqpoint{3.922024in}{1.975160in}}%
\pgfpathlineto{\pgfqpoint{3.922309in}{1.967888in}}%
\pgfpathlineto{\pgfqpoint{3.922594in}{1.767091in}}%
\pgfpathlineto{\pgfqpoint{3.923164in}{1.999974in}}%
\pgfpathlineto{\pgfqpoint{3.923448in}{2.082199in}}%
\pgfpathlineto{\pgfqpoint{3.923732in}{1.919917in}}%
\pgfpathlineto{\pgfqpoint{3.924017in}{2.032724in}}%
\pgfpathlineto{\pgfqpoint{3.924300in}{1.746161in}}%
\pgfpathlineto{\pgfqpoint{3.925151in}{1.994183in}}%
\pgfpathlineto{\pgfqpoint{3.925434in}{1.985485in}}%
\pgfpathlineto{\pgfqpoint{3.926000in}{2.036984in}}%
\pgfpathlineto{\pgfqpoint{3.926283in}{1.850254in}}%
\pgfpathlineto{\pgfqpoint{3.926847in}{2.059889in}}%
\pgfpathlineto{\pgfqpoint{3.927411in}{1.969852in}}%
\pgfpathlineto{\pgfqpoint{3.927693in}{2.013462in}}%
\pgfpathlineto{\pgfqpoint{3.927975in}{1.849531in}}%
\pgfpathlineto{\pgfqpoint{3.928818in}{2.005362in}}%
\pgfpathlineto{\pgfqpoint{3.929379in}{2.034094in}}%
\pgfpathlineto{\pgfqpoint{3.930220in}{1.689599in}}%
\pgfpathlineto{\pgfqpoint{3.931059in}{1.818038in}}%
\pgfpathlineto{\pgfqpoint{3.931617in}{2.002133in}}%
\pgfpathlineto{\pgfqpoint{3.932175in}{1.945009in}}%
\pgfpathlineto{\pgfqpoint{3.932732in}{1.891583in}}%
\pgfpathlineto{\pgfqpoint{3.933566in}{1.840564in}}%
\pgfpathlineto{\pgfqpoint{3.933843in}{2.080436in}}%
\pgfpathlineto{\pgfqpoint{3.934398in}{1.882782in}}%
\pgfpathlineto{\pgfqpoint{3.934952in}{2.020567in}}%
\pgfpathlineto{\pgfqpoint{3.935229in}{2.028256in}}%
\pgfpathlineto{\pgfqpoint{3.935505in}{2.014218in}}%
\pgfpathlineto{\pgfqpoint{3.935782in}{2.000351in}}%
\pgfpathlineto{\pgfqpoint{3.936610in}{1.610685in}}%
\pgfpathlineto{\pgfqpoint{3.936334in}{2.067531in}}%
\pgfpathlineto{\pgfqpoint{3.936885in}{1.858273in}}%
\pgfpathlineto{\pgfqpoint{3.937436in}{2.110153in}}%
\pgfpathlineto{\pgfqpoint{3.938261in}{2.080084in}}%
\pgfpathlineto{\pgfqpoint{3.939906in}{1.800088in}}%
\pgfpathlineto{\pgfqpoint{3.940999in}{2.082794in}}%
\pgfpathlineto{\pgfqpoint{3.941272in}{1.768301in}}%
\pgfpathlineto{\pgfqpoint{3.942089in}{1.881898in}}%
\pgfpathlineto{\pgfqpoint{3.942361in}{2.078353in}}%
\pgfpathlineto{\pgfqpoint{3.943176in}{1.922985in}}%
\pgfpathlineto{\pgfqpoint{3.943448in}{1.773351in}}%
\pgfpathlineto{\pgfqpoint{3.943990in}{2.046196in}}%
\pgfpathlineto{\pgfqpoint{3.944261in}{2.116110in}}%
\pgfpathlineto{\pgfqpoint{3.944532in}{1.846887in}}%
\pgfpathlineto{\pgfqpoint{3.945613in}{1.873325in}}%
\pgfpathlineto{\pgfqpoint{3.945883in}{2.081653in}}%
\pgfpathlineto{\pgfqpoint{3.946691in}{2.060444in}}%
\pgfpathlineto{\pgfqpoint{3.946961in}{2.051018in}}%
\pgfpathlineto{\pgfqpoint{3.947230in}{2.089834in}}%
\pgfpathlineto{\pgfqpoint{3.947498in}{1.915237in}}%
\pgfpathlineto{\pgfqpoint{3.948304in}{2.080251in}}%
\pgfpathlineto{\pgfqpoint{3.949376in}{2.080648in}}%
\pgfpathlineto{\pgfqpoint{3.949910in}{1.907802in}}%
\pgfpathlineto{\pgfqpoint{3.950178in}{1.918988in}}%
\pgfpathlineto{\pgfqpoint{3.950978in}{2.074447in}}%
\pgfpathlineto{\pgfqpoint{3.951244in}{1.976767in}}%
\pgfpathlineto{\pgfqpoint{3.951511in}{1.775857in}}%
\pgfpathlineto{\pgfqpoint{3.952043in}{1.950387in}}%
\pgfpathlineto{\pgfqpoint{3.952574in}{1.895046in}}%
\pgfpathlineto{\pgfqpoint{3.952840in}{2.055609in}}%
\pgfpathlineto{\pgfqpoint{3.953635in}{1.840138in}}%
\pgfpathlineto{\pgfqpoint{3.953900in}{1.914482in}}%
\pgfpathlineto{\pgfqpoint{3.954958in}{1.908757in}}%
\pgfpathlineto{\pgfqpoint{3.955222in}{2.088905in}}%
\pgfpathlineto{\pgfqpoint{3.956276in}{1.857418in}}%
\pgfpathlineto{\pgfqpoint{3.957065in}{2.091898in}}%
\pgfpathlineto{\pgfqpoint{3.957328in}{1.940777in}}%
\pgfpathlineto{\pgfqpoint{3.958115in}{1.991077in}}%
\pgfpathlineto{\pgfqpoint{3.958639in}{1.887802in}}%
\pgfpathlineto{\pgfqpoint{3.959684in}{2.132545in}}%
\pgfpathlineto{\pgfqpoint{3.959946in}{1.898424in}}%
\pgfpathlineto{\pgfqpoint{3.960988in}{1.957483in}}%
\pgfpathlineto{\pgfqpoint{3.961248in}{1.944470in}}%
\pgfpathlineto{\pgfqpoint{3.961769in}{2.052696in}}%
\pgfpathlineto{\pgfqpoint{3.962028in}{1.826346in}}%
\pgfpathlineto{\pgfqpoint{3.962807in}{2.048159in}}%
\pgfpathlineto{\pgfqpoint{3.963066in}{1.986504in}}%
\pgfpathlineto{\pgfqpoint{3.963584in}{2.081015in}}%
\pgfpathlineto{\pgfqpoint{3.963842in}{2.067231in}}%
\pgfpathlineto{\pgfqpoint{3.964101in}{1.944352in}}%
\pgfpathlineto{\pgfqpoint{3.964359in}{2.069576in}}%
\pgfpathlineto{\pgfqpoint{3.964876in}{1.969854in}}%
\pgfpathlineto{\pgfqpoint{3.965649in}{1.954710in}}%
\pgfpathlineto{\pgfqpoint{3.965906in}{2.034712in}}%
\pgfpathlineto{\pgfqpoint{3.966678in}{1.893622in}}%
\pgfpathlineto{\pgfqpoint{3.966934in}{2.010800in}}%
\pgfpathlineto{\pgfqpoint{3.967191in}{2.055329in}}%
\pgfpathlineto{\pgfqpoint{3.967447in}{1.852417in}}%
\pgfpathlineto{\pgfqpoint{3.968216in}{2.008794in}}%
\pgfpathlineto{\pgfqpoint{3.968728in}{2.028772in}}%
\pgfpathlineto{\pgfqpoint{3.969239in}{1.887737in}}%
\pgfpathlineto{\pgfqpoint{3.970004in}{2.083209in}}%
\pgfpathlineto{\pgfqpoint{3.970259in}{1.876375in}}%
\pgfpathlineto{\pgfqpoint{3.970513in}{2.069637in}}%
\pgfpathlineto{\pgfqpoint{3.971276in}{2.146393in}}%
\pgfpathlineto{\pgfqpoint{3.971784in}{1.731785in}}%
\pgfpathlineto{\pgfqpoint{3.972038in}{2.102920in}}%
\pgfpathlineto{\pgfqpoint{3.973051in}{2.050701in}}%
\pgfpathlineto{\pgfqpoint{3.973557in}{1.844278in}}%
\pgfpathlineto{\pgfqpoint{3.973810in}{2.119557in}}%
\pgfpathlineto{\pgfqpoint{3.974062in}{1.877664in}}%
\pgfpathlineto{\pgfqpoint{3.974315in}{2.031612in}}%
\pgfpathlineto{\pgfqpoint{3.975323in}{2.015237in}}%
\pgfpathlineto{\pgfqpoint{3.975574in}{2.090455in}}%
\pgfpathlineto{\pgfqpoint{3.975826in}{2.011713in}}%
\pgfpathlineto{\pgfqpoint{3.976328in}{2.026890in}}%
\pgfpathlineto{\pgfqpoint{3.976579in}{1.840475in}}%
\pgfpathlineto{\pgfqpoint{3.977081in}{2.060488in}}%
\pgfpathlineto{\pgfqpoint{3.977332in}{1.948860in}}%
\pgfpathlineto{\pgfqpoint{3.978332in}{1.930082in}}%
\pgfpathlineto{\pgfqpoint{3.978582in}{2.107820in}}%
\pgfpathlineto{\pgfqpoint{3.979580in}{1.843695in}}%
\pgfpathlineto{\pgfqpoint{3.979829in}{2.019952in}}%
\pgfpathlineto{\pgfqpoint{3.980078in}{1.979674in}}%
\pgfpathlineto{\pgfqpoint{3.980576in}{2.067209in}}%
\pgfpathlineto{\pgfqpoint{3.980824in}{2.092244in}}%
\pgfpathlineto{\pgfqpoint{3.981569in}{1.773756in}}%
\pgfpathlineto{\pgfqpoint{3.982065in}{1.904580in}}%
\pgfpathlineto{\pgfqpoint{3.982312in}{2.110418in}}%
\pgfpathlineto{\pgfqpoint{3.983054in}{1.874022in}}%
\pgfpathlineto{\pgfqpoint{3.983302in}{2.071938in}}%
\pgfpathlineto{\pgfqpoint{3.984042in}{1.850287in}}%
\pgfpathlineto{\pgfqpoint{3.984781in}{1.961648in}}%
\pgfpathlineto{\pgfqpoint{3.985273in}{2.062216in}}%
\pgfpathlineto{\pgfqpoint{3.985519in}{1.847724in}}%
\pgfpathlineto{\pgfqpoint{3.985764in}{1.970762in}}%
\pgfpathlineto{\pgfqpoint{3.986010in}{1.945225in}}%
\pgfpathlineto{\pgfqpoint{3.986991in}{1.919339in}}%
\pgfpathlineto{\pgfqpoint{3.987235in}{2.126723in}}%
\pgfpathlineto{\pgfqpoint{3.988213in}{1.957134in}}%
\pgfpathlineto{\pgfqpoint{3.988457in}{2.050055in}}%
\pgfpathlineto{\pgfqpoint{3.988945in}{2.077723in}}%
\pgfpathlineto{\pgfqpoint{3.989919in}{1.842821in}}%
\pgfpathlineto{\pgfqpoint{3.991133in}{2.068172in}}%
\pgfpathlineto{\pgfqpoint{3.991376in}{2.112282in}}%
\pgfpathlineto{\pgfqpoint{3.991618in}{1.967815in}}%
\pgfpathlineto{\pgfqpoint{3.992102in}{2.049505in}}%
\pgfpathlineto{\pgfqpoint{3.992344in}{1.969795in}}%
\pgfpathlineto{\pgfqpoint{3.993069in}{2.091504in}}%
\pgfpathlineto{\pgfqpoint{3.994274in}{1.873995in}}%
\pgfpathlineto{\pgfqpoint{3.994995in}{2.072937in}}%
\pgfpathlineto{\pgfqpoint{3.995476in}{1.987588in}}%
\pgfpathlineto{\pgfqpoint{3.995716in}{2.069465in}}%
\pgfpathlineto{\pgfqpoint{3.996195in}{1.943694in}}%
\pgfpathlineto{\pgfqpoint{3.996435in}{2.060143in}}%
\pgfpathlineto{\pgfqpoint{3.996913in}{1.784035in}}%
\pgfpathlineto{\pgfqpoint{3.997392in}{2.045453in}}%
\pgfpathlineto{\pgfqpoint{3.997630in}{2.132513in}}%
\pgfpathlineto{\pgfqpoint{3.998108in}{1.979596in}}%
\pgfpathlineto{\pgfqpoint{3.998346in}{2.001814in}}%
\pgfpathlineto{\pgfqpoint{3.999299in}{1.997686in}}%
\pgfpathlineto{\pgfqpoint{3.999774in}{2.101334in}}%
\pgfpathlineto{\pgfqpoint{4.000487in}{1.791928in}}%
\pgfpathlineto{\pgfqpoint{4.000961in}{2.000776in}}%
\pgfpathlineto{\pgfqpoint{4.001198in}{2.008965in}}%
\pgfpathlineto{\pgfqpoint{4.002144in}{2.090842in}}%
\pgfpathlineto{\pgfqpoint{4.002380in}{2.056701in}}%
\pgfpathlineto{\pgfqpoint{4.003088in}{1.921822in}}%
\pgfpathlineto{\pgfqpoint{4.003324in}{1.972357in}}%
\pgfpathlineto{\pgfqpoint{4.004265in}{2.105112in}}%
\pgfpathlineto{\pgfqpoint{4.004500in}{2.083652in}}%
\pgfpathlineto{\pgfqpoint{4.004970in}{1.713261in}}%
\pgfpathlineto{\pgfqpoint{4.005439in}{2.130928in}}%
\pgfpathlineto{\pgfqpoint{4.006843in}{1.737050in}}%
\pgfpathlineto{\pgfqpoint{4.008010in}{2.072169in}}%
\pgfpathlineto{\pgfqpoint{4.008941in}{2.096858in}}%
\pgfpathlineto{\pgfqpoint{4.009174in}{1.714769in}}%
\pgfpathlineto{\pgfqpoint{4.010334in}{2.079262in}}%
\pgfpathlineto{\pgfqpoint{4.010797in}{1.919584in}}%
\pgfpathlineto{\pgfqpoint{4.011491in}{2.030632in}}%
\pgfpathlineto{\pgfqpoint{4.011722in}{2.088802in}}%
\pgfpathlineto{\pgfqpoint{4.012415in}{1.791463in}}%
\pgfpathlineto{\pgfqpoint{4.012876in}{1.986120in}}%
\pgfpathlineto{\pgfqpoint{4.013566in}{1.889707in}}%
\pgfpathlineto{\pgfqpoint{4.014026in}{2.062638in}}%
\pgfpathlineto{\pgfqpoint{4.014256in}{1.885592in}}%
\pgfpathlineto{\pgfqpoint{4.014944in}{2.070962in}}%
\pgfpathlineto{\pgfqpoint{4.015173in}{2.081539in}}%
\pgfpathlineto{\pgfqpoint{4.015403in}{2.064566in}}%
\pgfpathlineto{\pgfqpoint{4.016318in}{1.850175in}}%
\pgfpathlineto{\pgfqpoint{4.017686in}{2.113759in}}%
\pgfpathlineto{\pgfqpoint{4.018369in}{1.946880in}}%
\pgfpathlineto{\pgfqpoint{4.018824in}{2.002903in}}%
\pgfpathlineto{\pgfqpoint{4.019051in}{2.077793in}}%
\pgfpathlineto{\pgfqpoint{4.019278in}{1.838961in}}%
\pgfpathlineto{\pgfqpoint{4.019732in}{2.037816in}}%
\pgfpathlineto{\pgfqpoint{4.020864in}{1.701858in}}%
\pgfpathlineto{\pgfqpoint{4.021090in}{1.940816in}}%
\pgfpathlineto{\pgfqpoint{4.021992in}{2.085296in}}%
\pgfpathlineto{\pgfqpoint{4.022443in}{2.104532in}}%
\pgfpathlineto{\pgfqpoint{4.023118in}{1.812544in}}%
\pgfpathlineto{\pgfqpoint{4.023793in}{2.097523in}}%
\pgfpathlineto{\pgfqpoint{4.024017in}{1.977624in}}%
\pgfpathlineto{\pgfqpoint{4.024241in}{1.777223in}}%
\pgfpathlineto{\pgfqpoint{4.024466in}{1.985463in}}%
\pgfpathlineto{\pgfqpoint{4.025138in}{1.887644in}}%
\pgfpathlineto{\pgfqpoint{4.026032in}{2.142292in}}%
\pgfpathlineto{\pgfqpoint{4.026255in}{1.736887in}}%
\pgfpathlineto{\pgfqpoint{4.027147in}{1.999043in}}%
\pgfpathlineto{\pgfqpoint{4.027592in}{2.079523in}}%
\pgfpathlineto{\pgfqpoint{4.028260in}{1.817016in}}%
\pgfpathlineto{\pgfqpoint{4.028482in}{2.074786in}}%
\pgfpathlineto{\pgfqpoint{4.029147in}{1.720758in}}%
\pgfpathlineto{\pgfqpoint{4.029369in}{2.008709in}}%
\pgfpathlineto{\pgfqpoint{4.029591in}{2.035340in}}%
\pgfpathlineto{\pgfqpoint{4.029812in}{2.005931in}}%
\pgfpathlineto{\pgfqpoint{4.030033in}{1.890112in}}%
\pgfpathlineto{\pgfqpoint{4.030697in}{2.090375in}}%
\pgfpathlineto{\pgfqpoint{4.031579in}{2.090678in}}%
\pgfpathlineto{\pgfqpoint{4.031800in}{1.916329in}}%
\pgfpathlineto{\pgfqpoint{4.032020in}{2.105155in}}%
\pgfpathlineto{\pgfqpoint{4.032900in}{2.074386in}}%
\pgfpathlineto{\pgfqpoint{4.033339in}{1.957362in}}%
\pgfpathlineto{\pgfqpoint{4.033778in}{2.100442in}}%
\pgfpathlineto{\pgfqpoint{4.033997in}{2.118272in}}%
\pgfpathlineto{\pgfqpoint{4.034435in}{1.826723in}}%
\pgfpathlineto{\pgfqpoint{4.035092in}{1.945210in}}%
\pgfpathlineto{\pgfqpoint{4.035747in}{2.134700in}}%
\pgfpathlineto{\pgfqpoint{4.035965in}{1.958031in}}%
\pgfpathlineto{\pgfqpoint{4.036184in}{1.834233in}}%
\pgfpathlineto{\pgfqpoint{4.036619in}{2.074037in}}%
\pgfpathlineto{\pgfqpoint{4.036837in}{2.030268in}}%
\pgfpathlineto{\pgfqpoint{4.037055in}{2.063611in}}%
\pgfpathlineto{\pgfqpoint{4.037273in}{2.026902in}}%
\pgfpathlineto{\pgfqpoint{4.037707in}{2.039286in}}%
\pgfpathlineto{\pgfqpoint{4.037925in}{1.967155in}}%
\pgfpathlineto{\pgfqpoint{4.038576in}{2.052751in}}%
\pgfpathlineto{\pgfqpoint{4.038792in}{2.035904in}}%
\pgfpathlineto{\pgfqpoint{4.039226in}{2.089578in}}%
\pgfpathlineto{\pgfqpoint{4.039442in}{1.925292in}}%
\pgfpathlineto{\pgfqpoint{4.039875in}{2.094297in}}%
\pgfpathlineto{\pgfqpoint{4.040307in}{1.958478in}}%
\pgfpathlineto{\pgfqpoint{4.040954in}{2.116149in}}%
\pgfpathlineto{\pgfqpoint{4.041170in}{1.839553in}}%
\pgfpathlineto{\pgfqpoint{4.042031in}{2.062005in}}%
\pgfpathlineto{\pgfqpoint{4.042246in}{1.962167in}}%
\pgfpathlineto{\pgfqpoint{4.042890in}{2.112958in}}%
\pgfpathlineto{\pgfqpoint{4.043105in}{2.120596in}}%
\pgfpathlineto{\pgfqpoint{4.045032in}{1.839112in}}%
\pgfpathlineto{\pgfqpoint{4.045885in}{2.052406in}}%
\pgfpathlineto{\pgfqpoint{4.046098in}{1.988226in}}%
\pgfpathlineto{\pgfqpoint{4.046524in}{2.082223in}}%
\pgfpathlineto{\pgfqpoint{4.047162in}{1.875663in}}%
\pgfpathlineto{\pgfqpoint{4.047799in}{2.142926in}}%
\pgfpathlineto{\pgfqpoint{4.048223in}{1.981493in}}%
\pgfpathlineto{\pgfqpoint{4.048435in}{1.965266in}}%
\pgfpathlineto{\pgfqpoint{4.048647in}{1.829788in}}%
\pgfpathlineto{\pgfqpoint{4.049282in}{2.072476in}}%
\pgfpathlineto{\pgfqpoint{4.049915in}{2.075899in}}%
\pgfpathlineto{\pgfqpoint{4.050970in}{1.746977in}}%
\pgfpathlineto{\pgfqpoint{4.051391in}{2.068357in}}%
\pgfpathlineto{\pgfqpoint{4.052231in}{2.067019in}}%
\pgfpathlineto{\pgfqpoint{4.053699in}{1.821792in}}%
\pgfpathlineto{\pgfqpoint{4.054744in}{2.087829in}}%
\pgfpathlineto{\pgfqpoint{4.054952in}{1.732384in}}%
\pgfpathlineto{\pgfqpoint{4.055578in}{2.113816in}}%
\pgfpathlineto{\pgfqpoint{4.055786in}{2.078587in}}%
\pgfpathlineto{\pgfqpoint{4.056202in}{2.067509in}}%
\pgfpathlineto{\pgfqpoint{4.056410in}{1.948779in}}%
\pgfpathlineto{\pgfqpoint{4.056826in}{2.117886in}}%
\pgfpathlineto{\pgfqpoint{4.057034in}{2.099456in}}%
\pgfpathlineto{\pgfqpoint{4.057241in}{2.138930in}}%
\pgfpathlineto{\pgfqpoint{4.057863in}{1.892870in}}%
\pgfpathlineto{\pgfqpoint{4.058277in}{1.936770in}}%
\pgfpathlineto{\pgfqpoint{4.058898in}{2.074313in}}%
\pgfpathlineto{\pgfqpoint{4.058691in}{1.883449in}}%
\pgfpathlineto{\pgfqpoint{4.059311in}{1.961406in}}%
\pgfpathlineto{\pgfqpoint{4.059518in}{1.881253in}}%
\pgfpathlineto{\pgfqpoint{4.059724in}{2.068191in}}%
\pgfpathlineto{\pgfqpoint{4.060136in}{2.053930in}}%
\pgfpathlineto{\pgfqpoint{4.060343in}{2.086666in}}%
\pgfpathlineto{\pgfqpoint{4.060548in}{1.920626in}}%
\pgfpathlineto{\pgfqpoint{4.061166in}{2.101925in}}%
\pgfpathlineto{\pgfqpoint{4.061371in}{1.989813in}}%
\pgfpathlineto{\pgfqpoint{4.061782in}{2.145153in}}%
\pgfpathlineto{\pgfqpoint{4.061987in}{1.947103in}}%
\pgfpathlineto{\pgfqpoint{4.062193in}{1.887320in}}%
\pgfpathlineto{\pgfqpoint{4.062398in}{2.060780in}}%
\pgfpathlineto{\pgfqpoint{4.063012in}{1.930709in}}%
\pgfpathlineto{\pgfqpoint{4.063830in}{2.086052in}}%
\pgfpathlineto{\pgfqpoint{4.064034in}{1.922225in}}%
\pgfpathlineto{\pgfqpoint{4.064239in}{2.022853in}}%
\pgfpathlineto{\pgfqpoint{4.064851in}{2.067899in}}%
\pgfpathlineto{\pgfqpoint{4.065054in}{1.969729in}}%
\pgfpathlineto{\pgfqpoint{4.066072in}{2.074455in}}%
\pgfpathlineto{\pgfqpoint{4.065868in}{1.878971in}}%
\pgfpathlineto{\pgfqpoint{4.066275in}{2.071333in}}%
\pgfpathlineto{\pgfqpoint{4.066478in}{2.086295in}}%
\pgfpathlineto{\pgfqpoint{4.066681in}{1.780289in}}%
\pgfpathlineto{\pgfqpoint{4.067695in}{1.899121in}}%
\pgfpathlineto{\pgfqpoint{4.067897in}{1.763549in}}%
\pgfpathlineto{\pgfqpoint{4.068099in}{2.150597in}}%
\pgfpathlineto{\pgfqpoint{4.068504in}{1.962679in}}%
\pgfpathlineto{\pgfqpoint{4.069311in}{2.079001in}}%
\pgfpathlineto{\pgfqpoint{4.069513in}{1.835330in}}%
\pgfpathlineto{\pgfqpoint{4.069916in}{2.163891in}}%
\pgfpathlineto{\pgfqpoint{4.070318in}{2.113962in}}%
\pgfpathlineto{\pgfqpoint{4.070520in}{2.145787in}}%
\pgfpathlineto{\pgfqpoint{4.070721in}{2.008122in}}%
\pgfpathlineto{\pgfqpoint{4.070922in}{1.945344in}}%
\pgfpathlineto{\pgfqpoint{4.071323in}{2.108337in}}%
\pgfpathlineto{\pgfqpoint{4.071524in}{2.085376in}}%
\pgfpathlineto{\pgfqpoint{4.071925in}{2.020385in}}%
\pgfpathlineto{\pgfqpoint{4.072326in}{2.096466in}}%
\pgfpathlineto{\pgfqpoint{4.072926in}{1.868722in}}%
\pgfpathlineto{\pgfqpoint{4.073526in}{2.120873in}}%
\pgfpathlineto{\pgfqpoint{4.073925in}{1.997790in}}%
\pgfpathlineto{\pgfqpoint{4.074523in}{2.093418in}}%
\pgfpathlineto{\pgfqpoint{4.075120in}{1.872076in}}%
\pgfpathlineto{\pgfqpoint{4.075518in}{2.130293in}}%
\pgfpathlineto{\pgfqpoint{4.076312in}{2.085849in}}%
\pgfpathlineto{\pgfqpoint{4.077105in}{1.835510in}}%
\pgfpathlineto{\pgfqpoint{4.076709in}{2.137409in}}%
\pgfpathlineto{\pgfqpoint{4.077303in}{1.948134in}}%
\pgfpathlineto{\pgfqpoint{4.077897in}{1.840795in}}%
\pgfpathlineto{\pgfqpoint{4.078292in}{2.113025in}}%
\pgfpathlineto{\pgfqpoint{4.078489in}{1.817819in}}%
\pgfpathlineto{\pgfqpoint{4.079278in}{2.133271in}}%
\pgfpathlineto{\pgfqpoint{4.079672in}{2.086098in}}%
\pgfpathlineto{\pgfqpoint{4.079868in}{1.519722in}}%
\pgfpathlineto{\pgfqpoint{4.080655in}{1.983379in}}%
\pgfpathlineto{\pgfqpoint{4.081243in}{2.078753in}}%
\pgfpathlineto{\pgfqpoint{4.081439in}{1.915405in}}%
\pgfpathlineto{\pgfqpoint{4.081635in}{2.146229in}}%
\pgfpathlineto{\pgfqpoint{4.082418in}{1.961044in}}%
\pgfpathlineto{\pgfqpoint{4.083200in}{2.108433in}}%
\pgfpathlineto{\pgfqpoint{4.083005in}{1.916080in}}%
\pgfpathlineto{\pgfqpoint{4.083980in}{2.065447in}}%
\pgfpathlineto{\pgfqpoint{4.085147in}{1.844833in}}%
\pgfpathlineto{\pgfqpoint{4.086118in}{2.102357in}}%
\pgfpathlineto{\pgfqpoint{4.085730in}{1.809523in}}%
\pgfpathlineto{\pgfqpoint{4.086312in}{1.927950in}}%
\pgfpathlineto{\pgfqpoint{4.086893in}{2.126917in}}%
\pgfpathlineto{\pgfqpoint{4.087086in}{2.077655in}}%
\pgfpathlineto{\pgfqpoint{4.087473in}{1.822084in}}%
\pgfpathlineto{\pgfqpoint{4.088245in}{1.982255in}}%
\pgfpathlineto{\pgfqpoint{4.089016in}{1.981726in}}%
\pgfpathlineto{\pgfqpoint{4.089208in}{2.060898in}}%
\pgfpathlineto{\pgfqpoint{4.089401in}{1.818110in}}%
\pgfpathlineto{\pgfqpoint{4.090170in}{2.109706in}}%
\pgfpathlineto{\pgfqpoint{4.090361in}{1.888417in}}%
\pgfpathlineto{\pgfqpoint{4.091512in}{2.100216in}}%
\pgfpathlineto{\pgfqpoint{4.092849in}{1.872845in}}%
\pgfpathlineto{\pgfqpoint{4.093421in}{1.817446in}}%
\pgfpathlineto{\pgfqpoint{4.093993in}{2.161045in}}%
\pgfpathlineto{\pgfqpoint{4.094183in}{1.744720in}}%
\pgfpathlineto{\pgfqpoint{4.095133in}{1.940044in}}%
\pgfpathlineto{\pgfqpoint{4.095512in}{2.137040in}}%
\pgfpathlineto{\pgfqpoint{4.095891in}{1.926453in}}%
\pgfpathlineto{\pgfqpoint{4.096270in}{2.014033in}}%
\pgfpathlineto{\pgfqpoint{4.096459in}{2.022574in}}%
\pgfpathlineto{\pgfqpoint{4.097215in}{2.142556in}}%
\pgfpathlineto{\pgfqpoint{4.097970in}{1.805326in}}%
\pgfpathlineto{\pgfqpoint{4.098347in}{1.986123in}}%
\pgfpathlineto{\pgfqpoint{4.098535in}{2.108991in}}%
\pgfpathlineto{\pgfqpoint{4.098912in}{1.797640in}}%
\pgfpathlineto{\pgfqpoint{4.099476in}{1.998409in}}%
\pgfpathlineto{\pgfqpoint{4.099663in}{1.914747in}}%
\pgfpathlineto{\pgfqpoint{4.099851in}{2.087914in}}%
\pgfpathlineto{\pgfqpoint{4.100414in}{1.989169in}}%
\pgfpathlineto{\pgfqpoint{4.101350in}{2.086930in}}%
\pgfpathlineto{\pgfqpoint{4.101537in}{2.007078in}}%
\pgfpathlineto{\pgfqpoint{4.102471in}{2.065691in}}%
\pgfpathlineto{\pgfqpoint{4.102844in}{1.833385in}}%
\pgfpathlineto{\pgfqpoint{4.103588in}{2.121748in}}%
\pgfpathlineto{\pgfqpoint{4.103960in}{2.056148in}}%
\pgfpathlineto{\pgfqpoint{4.104332in}{2.086273in}}%
\pgfpathlineto{\pgfqpoint{4.104889in}{1.935512in}}%
\pgfpathlineto{\pgfqpoint{4.105260in}{2.115697in}}%
\pgfpathlineto{\pgfqpoint{4.105445in}{1.521024in}}%
\pgfpathlineto{\pgfqpoint{4.106370in}{2.075338in}}%
\pgfpathlineto{\pgfqpoint{4.106555in}{1.886342in}}%
\pgfpathlineto{\pgfqpoint{4.106739in}{2.156908in}}%
\pgfpathlineto{\pgfqpoint{4.107293in}{2.068543in}}%
\pgfpathlineto{\pgfqpoint{4.107477in}{2.114630in}}%
\pgfpathlineto{\pgfqpoint{4.107662in}{2.021151in}}%
\pgfpathlineto{\pgfqpoint{4.108030in}{2.056021in}}%
\pgfpathlineto{\pgfqpoint{4.108766in}{1.854606in}}%
\pgfpathlineto{\pgfqpoint{4.108582in}{2.063567in}}%
\pgfpathlineto{\pgfqpoint{4.109133in}{1.868888in}}%
\pgfpathlineto{\pgfqpoint{4.110417in}{2.022793in}}%
\pgfpathlineto{\pgfqpoint{4.110966in}{1.958898in}}%
\pgfpathlineto{\pgfqpoint{4.111148in}{2.092912in}}%
\pgfpathlineto{\pgfqpoint{4.111331in}{2.116311in}}%
\pgfpathlineto{\pgfqpoint{4.111696in}{1.967501in}}%
\pgfpathlineto{\pgfqpoint{4.112061in}{2.171992in}}%
\pgfpathlineto{\pgfqpoint{4.112608in}{1.974031in}}%
\pgfpathlineto{\pgfqpoint{4.113154in}{2.079671in}}%
\pgfpathlineto{\pgfqpoint{4.113518in}{1.986525in}}%
\pgfpathlineto{\pgfqpoint{4.113699in}{1.795699in}}%
\pgfpathlineto{\pgfqpoint{4.113881in}{2.054040in}}%
\pgfpathlineto{\pgfqpoint{4.114425in}{2.043709in}}%
\pgfpathlineto{\pgfqpoint{4.114607in}{2.044268in}}%
\pgfpathlineto{\pgfqpoint{4.114788in}{1.897106in}}%
\pgfpathlineto{\pgfqpoint{4.115693in}{1.959391in}}%
\pgfpathlineto{\pgfqpoint{4.116235in}{1.835340in}}%
\pgfpathlineto{\pgfqpoint{4.116596in}{2.149961in}}%
\pgfpathlineto{\pgfqpoint{4.116776in}{1.826942in}}%
\pgfpathlineto{\pgfqpoint{4.117677in}{2.094492in}}%
\pgfpathlineto{\pgfqpoint{4.118576in}{2.202801in}}%
\pgfpathlineto{\pgfqpoint{4.118756in}{1.924565in}}%
\pgfpathlineto{\pgfqpoint{4.119115in}{2.104962in}}%
\pgfpathlineto{\pgfqpoint{4.119831in}{1.950522in}}%
\pgfpathlineto{\pgfqpoint{4.120189in}{2.088078in}}%
\pgfpathlineto{\pgfqpoint{4.120368in}{1.868339in}}%
\pgfpathlineto{\pgfqpoint{4.121083in}{2.031201in}}%
\pgfpathlineto{\pgfqpoint{4.121440in}{1.747153in}}%
\pgfpathlineto{\pgfqpoint{4.121975in}{2.092751in}}%
\pgfpathlineto{\pgfqpoint{4.122509in}{1.894205in}}%
\pgfpathlineto{\pgfqpoint{4.123220in}{2.119785in}}%
\pgfpathlineto{\pgfqpoint{4.123397in}{1.858110in}}%
\pgfpathlineto{\pgfqpoint{4.123930in}{2.108267in}}%
\pgfpathlineto{\pgfqpoint{4.124107in}{1.722890in}}%
\pgfpathlineto{\pgfqpoint{4.124992in}{2.002293in}}%
\pgfpathlineto{\pgfqpoint{4.125523in}{1.845089in}}%
\pgfpathlineto{\pgfqpoint{4.126229in}{2.156994in}}%
\pgfpathlineto{\pgfqpoint{4.127462in}{1.896346in}}%
\pgfpathlineto{\pgfqpoint{4.127638in}{1.759651in}}%
\pgfpathlineto{\pgfqpoint{4.128340in}{2.013208in}}%
\pgfpathlineto{\pgfqpoint{4.128516in}{1.891552in}}%
\pgfpathlineto{\pgfqpoint{4.129392in}{2.113054in}}%
\pgfpathlineto{\pgfqpoint{4.129742in}{2.027001in}}%
\pgfpathlineto{\pgfqpoint{4.130266in}{2.065131in}}%
\pgfpathlineto{\pgfqpoint{4.130616in}{1.846689in}}%
\pgfpathlineto{\pgfqpoint{4.131139in}{2.074211in}}%
\pgfpathlineto{\pgfqpoint{4.131662in}{1.970332in}}%
\pgfpathlineto{\pgfqpoint{4.131836in}{1.940092in}}%
\pgfpathlineto{\pgfqpoint{4.132010in}{2.031877in}}%
\pgfpathlineto{\pgfqpoint{4.132184in}{2.030677in}}%
\pgfpathlineto{\pgfqpoint{4.132879in}{1.935904in}}%
\pgfpathlineto{\pgfqpoint{4.133226in}{2.087635in}}%
\pgfpathlineto{\pgfqpoint{4.134785in}{1.890320in}}%
\pgfpathlineto{\pgfqpoint{4.135303in}{1.760874in}}%
\pgfpathlineto{\pgfqpoint{4.135820in}{2.098042in}}%
\pgfpathlineto{\pgfqpoint{4.136510in}{1.877763in}}%
\pgfpathlineto{\pgfqpoint{4.136682in}{2.043717in}}%
\pgfpathlineto{\pgfqpoint{4.136854in}{2.162870in}}%
\pgfpathlineto{\pgfqpoint{4.137026in}{1.973035in}}%
\pgfpathlineto{\pgfqpoint{4.137541in}{1.973713in}}%
\pgfpathlineto{\pgfqpoint{4.137713in}{1.724340in}}%
\pgfpathlineto{\pgfqpoint{4.137885in}{2.077968in}}%
\pgfpathlineto{\pgfqpoint{4.138570in}{1.919508in}}%
\pgfpathlineto{\pgfqpoint{4.139426in}{2.080463in}}%
\pgfpathlineto{\pgfqpoint{4.139597in}{1.996585in}}%
\pgfpathlineto{\pgfqpoint{4.139939in}{1.780965in}}%
\pgfpathlineto{\pgfqpoint{4.140280in}{2.108814in}}%
\pgfpathlineto{\pgfqpoint{4.140621in}{1.983657in}}%
\pgfpathlineto{\pgfqpoint{4.140792in}{2.153600in}}%
\pgfpathlineto{\pgfqpoint{4.140962in}{1.886701in}}%
\pgfpathlineto{\pgfqpoint{4.141643in}{2.113850in}}%
\pgfpathlineto{\pgfqpoint{4.142153in}{1.825741in}}%
\pgfpathlineto{\pgfqpoint{4.142832in}{2.038310in}}%
\pgfpathlineto{\pgfqpoint{4.143679in}{1.802249in}}%
\pgfpathlineto{\pgfqpoint{4.143341in}{2.103891in}}%
\pgfpathlineto{\pgfqpoint{4.143848in}{1.970983in}}%
\pgfpathlineto{\pgfqpoint{4.144018in}{2.133718in}}%
\pgfpathlineto{\pgfqpoint{4.144525in}{2.018329in}}%
\pgfpathlineto{\pgfqpoint{4.144694in}{1.711408in}}%
\pgfpathlineto{\pgfqpoint{4.145031in}{2.122434in}}%
\pgfpathlineto{\pgfqpoint{4.145537in}{2.101528in}}%
\pgfpathlineto{\pgfqpoint{4.146042in}{1.874837in}}%
\pgfpathlineto{\pgfqpoint{4.146715in}{2.017071in}}%
\pgfpathlineto{\pgfqpoint{4.147051in}{2.156775in}}%
\pgfpathlineto{\pgfqpoint{4.147219in}{1.933211in}}%
\pgfpathlineto{\pgfqpoint{4.147890in}{2.063768in}}%
\pgfpathlineto{\pgfqpoint{4.148058in}{2.084629in}}%
\pgfpathlineto{\pgfqpoint{4.148225in}{1.899747in}}%
\pgfpathlineto{\pgfqpoint{4.148727in}{2.202440in}}%
\pgfpathlineto{\pgfqpoint{4.149229in}{1.900848in}}%
\pgfpathlineto{\pgfqpoint{4.149396in}{2.113326in}}%
\pgfpathlineto{\pgfqpoint{4.150397in}{1.989126in}}%
\pgfpathlineto{\pgfqpoint{4.150563in}{1.993303in}}%
\pgfpathlineto{\pgfqpoint{4.150896in}{2.078839in}}%
\pgfpathlineto{\pgfqpoint{4.151229in}{2.031289in}}%
\pgfpathlineto{\pgfqpoint{4.151395in}{1.917452in}}%
\pgfpathlineto{\pgfqpoint{4.151561in}{2.061470in}}%
\pgfpathlineto{\pgfqpoint{4.152225in}{1.999452in}}%
\pgfpathlineto{\pgfqpoint{4.152557in}{2.047360in}}%
\pgfpathlineto{\pgfqpoint{4.152723in}{1.639421in}}%
\pgfpathlineto{\pgfqpoint{4.152888in}{2.109777in}}%
\pgfpathlineto{\pgfqpoint{4.153716in}{1.894658in}}%
\pgfpathlineto{\pgfqpoint{4.155036in}{2.107725in}}%
\pgfpathlineto{\pgfqpoint{4.156188in}{1.759659in}}%
\pgfpathlineto{\pgfqpoint{4.157337in}{2.167620in}}%
\pgfpathlineto{\pgfqpoint{4.157501in}{2.067643in}}%
\pgfpathlineto{\pgfqpoint{4.157665in}{2.135653in}}%
\pgfpathlineto{\pgfqpoint{4.157829in}{1.740343in}}%
\pgfpathlineto{\pgfqpoint{4.158647in}{2.034553in}}%
\pgfpathlineto{\pgfqpoint{4.159136in}{1.985618in}}%
\pgfpathlineto{\pgfqpoint{4.159789in}{2.142152in}}%
\pgfpathlineto{\pgfqpoint{4.161091in}{1.825558in}}%
\pgfpathlineto{\pgfqpoint{4.161902in}{2.131369in}}%
\pgfpathlineto{\pgfqpoint{4.162226in}{1.848116in}}%
\pgfpathlineto{\pgfqpoint{4.164005in}{2.167820in}}%
\pgfpathlineto{\pgfqpoint{4.164489in}{1.798257in}}%
\pgfpathlineto{\pgfqpoint{4.165294in}{1.989077in}}%
\pgfpathlineto{\pgfqpoint{4.165615in}{2.196793in}}%
\pgfpathlineto{\pgfqpoint{4.166418in}{2.084745in}}%
\pgfpathlineto{\pgfqpoint{4.166579in}{1.851492in}}%
\pgfpathlineto{\pgfqpoint{4.166899in}{2.161375in}}%
\pgfpathlineto{\pgfqpoint{4.167540in}{1.986987in}}%
\pgfpathlineto{\pgfqpoint{4.168020in}{1.725528in}}%
\pgfpathlineto{\pgfqpoint{4.168339in}{1.981533in}}%
\pgfpathlineto{\pgfqpoint{4.168818in}{1.935125in}}%
\pgfpathlineto{\pgfqpoint{4.169296in}{2.078552in}}%
\pgfpathlineto{\pgfqpoint{4.169456in}{1.772317in}}%
\pgfpathlineto{\pgfqpoint{4.169774in}{2.104399in}}%
\pgfpathlineto{\pgfqpoint{4.170252in}{1.978106in}}%
\pgfpathlineto{\pgfqpoint{4.170887in}{2.157626in}}%
\pgfpathlineto{\pgfqpoint{4.171046in}{1.904421in}}%
\pgfpathlineto{\pgfqpoint{4.171363in}{2.068489in}}%
\pgfpathlineto{\pgfqpoint{4.171522in}{1.851434in}}%
\pgfpathlineto{\pgfqpoint{4.171997in}{2.079321in}}%
\pgfpathlineto{\pgfqpoint{4.172313in}{2.067319in}}%
\pgfpathlineto{\pgfqpoint{4.172472in}{2.132813in}}%
\pgfpathlineto{\pgfqpoint{4.172630in}{1.983210in}}%
\pgfpathlineto{\pgfqpoint{4.172946in}{2.089552in}}%
\pgfpathlineto{\pgfqpoint{4.173104in}{1.928824in}}%
\pgfpathlineto{\pgfqpoint{4.173893in}{2.120179in}}%
\pgfpathlineto{\pgfqpoint{4.174050in}{2.079365in}}%
\pgfpathlineto{\pgfqpoint{4.174208in}{2.083503in}}%
\pgfpathlineto{\pgfqpoint{4.174365in}{2.142968in}}%
\pgfpathlineto{\pgfqpoint{4.174523in}{1.863035in}}%
\pgfpathlineto{\pgfqpoint{4.174838in}{2.047879in}}%
\pgfpathlineto{\pgfqpoint{4.174995in}{1.845263in}}%
\pgfpathlineto{\pgfqpoint{4.175623in}{2.120624in}}%
\pgfpathlineto{\pgfqpoint{4.175937in}{1.963881in}}%
\pgfpathlineto{\pgfqpoint{4.176094in}{1.902680in}}%
\pgfpathlineto{\pgfqpoint{4.176408in}{2.062396in}}%
\pgfpathlineto{\pgfqpoint{4.176877in}{1.927638in}}%
\pgfpathlineto{\pgfqpoint{4.177659in}{2.130852in}}%
\pgfpathlineto{\pgfqpoint{4.177816in}{1.995906in}}%
\pgfpathlineto{\pgfqpoint{4.177972in}{1.901829in}}%
\pgfpathlineto{\pgfqpoint{4.178440in}{2.037356in}}%
\pgfpathlineto{\pgfqpoint{4.178907in}{2.005258in}}%
\pgfpathlineto{\pgfqpoint{4.179063in}{2.013662in}}%
\pgfpathlineto{\pgfqpoint{4.179375in}{2.142454in}}%
\pgfpathlineto{\pgfqpoint{4.179997in}{1.918483in}}%
\pgfpathlineto{\pgfqpoint{4.180152in}{2.016080in}}%
\pgfpathlineto{\pgfqpoint{4.180463in}{1.977197in}}%
\pgfpathlineto{\pgfqpoint{4.180773in}{2.041841in}}%
\pgfpathlineto{\pgfqpoint{4.181393in}{1.991096in}}%
\pgfpathlineto{\pgfqpoint{4.181548in}{1.999832in}}%
\pgfpathlineto{\pgfqpoint{4.182166in}{2.158473in}}%
\pgfpathlineto{\pgfqpoint{4.182630in}{2.051174in}}%
\pgfpathlineto{\pgfqpoint{4.183093in}{1.857151in}}%
\pgfpathlineto{\pgfqpoint{4.183556in}{2.112629in}}%
\pgfpathlineto{\pgfqpoint{4.183710in}{2.051583in}}%
\pgfpathlineto{\pgfqpoint{4.184171in}{1.864253in}}%
\pgfpathlineto{\pgfqpoint{4.184786in}{2.005906in}}%
\pgfpathlineto{\pgfqpoint{4.185247in}{2.103235in}}%
\pgfpathlineto{\pgfqpoint{4.185554in}{1.997055in}}%
\pgfpathlineto{\pgfqpoint{4.185707in}{2.075793in}}%
\pgfpathlineto{\pgfqpoint{4.186626in}{1.826332in}}%
\pgfpathlineto{\pgfqpoint{4.186167in}{2.145316in}}%
\pgfpathlineto{\pgfqpoint{4.186779in}{1.999796in}}%
\pgfpathlineto{\pgfqpoint{4.187696in}{1.994482in}}%
\pgfpathlineto{\pgfqpoint{4.187848in}{2.103056in}}%
\pgfpathlineto{\pgfqpoint{4.188001in}{2.013031in}}%
\pgfpathlineto{\pgfqpoint{4.188153in}{2.122698in}}%
\pgfpathlineto{\pgfqpoint{4.188915in}{2.020645in}}%
\pgfpathlineto{\pgfqpoint{4.189067in}{2.043045in}}%
\pgfpathlineto{\pgfqpoint{4.189219in}{1.858142in}}%
\pgfpathlineto{\pgfqpoint{4.189675in}{2.139700in}}%
\pgfpathlineto{\pgfqpoint{4.190282in}{1.892004in}}%
\pgfpathlineto{\pgfqpoint{4.190585in}{1.876320in}}%
\pgfpathlineto{\pgfqpoint{4.191342in}{2.128847in}}%
\pgfpathlineto{\pgfqpoint{4.191947in}{2.134010in}}%
\pgfpathlineto{\pgfqpoint{4.192551in}{1.851468in}}%
\pgfpathlineto{\pgfqpoint{4.193003in}{2.157381in}}%
\pgfpathlineto{\pgfqpoint{4.193756in}{2.092160in}}%
\pgfpathlineto{\pgfqpoint{4.194658in}{1.842730in}}%
\pgfpathlineto{\pgfqpoint{4.194507in}{2.136630in}}%
\pgfpathlineto{\pgfqpoint{4.194958in}{1.925744in}}%
\pgfpathlineto{\pgfqpoint{4.195108in}{2.095574in}}%
\pgfpathlineto{\pgfqpoint{4.195407in}{1.799699in}}%
\pgfpathlineto{\pgfqpoint{4.196156in}{2.082866in}}%
\pgfpathlineto{\pgfqpoint{4.196306in}{1.968133in}}%
\pgfpathlineto{\pgfqpoint{4.196903in}{2.120658in}}%
\pgfpathlineto{\pgfqpoint{4.197202in}{1.987673in}}%
\pgfpathlineto{\pgfqpoint{4.197351in}{2.146922in}}%
\pgfpathlineto{\pgfqpoint{4.198096in}{1.926408in}}%
\pgfpathlineto{\pgfqpoint{4.198245in}{2.031825in}}%
\pgfpathlineto{\pgfqpoint{4.198691in}{2.004558in}}%
\pgfpathlineto{\pgfqpoint{4.198543in}{2.088665in}}%
\pgfpathlineto{\pgfqpoint{4.198840in}{2.083270in}}%
\pgfpathlineto{\pgfqpoint{4.198989in}{2.086439in}}%
\pgfpathlineto{\pgfqpoint{4.199137in}{2.163923in}}%
\pgfpathlineto{\pgfqpoint{4.199286in}{2.010240in}}%
\pgfpathlineto{\pgfqpoint{4.199583in}{2.103397in}}%
\pgfpathlineto{\pgfqpoint{4.199731in}{1.786269in}}%
\pgfpathlineto{\pgfqpoint{4.200620in}{2.079503in}}%
\pgfpathlineto{\pgfqpoint{4.200768in}{2.174391in}}%
\pgfpathlineto{\pgfqpoint{4.201064in}{1.880251in}}%
\pgfpathlineto{\pgfqpoint{4.201802in}{2.151618in}}%
\pgfpathlineto{\pgfqpoint{4.202098in}{2.181078in}}%
\pgfpathlineto{\pgfqpoint{4.203129in}{1.881192in}}%
\pgfpathlineto{\pgfqpoint{4.203276in}{2.163560in}}%
\pgfpathlineto{\pgfqpoint{4.204158in}{1.848156in}}%
\pgfpathlineto{\pgfqpoint{4.204304in}{2.129014in}}%
\pgfpathlineto{\pgfqpoint{4.205477in}{1.887379in}}%
\pgfpathlineto{\pgfqpoint{4.205623in}{2.156985in}}%
\pgfpathlineto{\pgfqpoint{4.206354in}{1.778878in}}%
\pgfpathlineto{\pgfqpoint{4.206500in}{1.910485in}}%
\pgfpathlineto{\pgfqpoint{4.207811in}{2.187716in}}%
\pgfpathlineto{\pgfqpoint{4.206937in}{1.788819in}}%
\pgfpathlineto{\pgfqpoint{4.207957in}{2.101616in}}%
\pgfpathlineto{\pgfqpoint{4.208684in}{2.130287in}}%
\pgfpathlineto{\pgfqpoint{4.208974in}{1.793624in}}%
\pgfpathlineto{\pgfqpoint{4.209554in}{2.125438in}}%
\pgfpathlineto{\pgfqpoint{4.209989in}{1.857958in}}%
\pgfpathlineto{\pgfqpoint{4.210134in}{1.872201in}}%
\pgfpathlineto{\pgfqpoint{4.210278in}{2.085817in}}%
\pgfpathlineto{\pgfqpoint{4.211290in}{2.016853in}}%
\pgfpathlineto{\pgfqpoint{4.211434in}{2.142377in}}%
\pgfpathlineto{\pgfqpoint{4.211867in}{1.820498in}}%
\pgfpathlineto{\pgfqpoint{4.212299in}{2.035142in}}%
\pgfpathlineto{\pgfqpoint{4.212443in}{1.975407in}}%
\pgfpathlineto{\pgfqpoint{4.213018in}{2.091409in}}%
\pgfpathlineto{\pgfqpoint{4.213162in}{2.084422in}}%
\pgfpathlineto{\pgfqpoint{4.213449in}{2.155615in}}%
\pgfpathlineto{\pgfqpoint{4.214310in}{1.867599in}}%
\pgfpathlineto{\pgfqpoint{4.213880in}{2.190175in}}%
\pgfpathlineto{\pgfqpoint{4.214740in}{1.981617in}}%
\pgfpathlineto{\pgfqpoint{4.214883in}{2.171669in}}%
\pgfpathlineto{\pgfqpoint{4.215741in}{1.837058in}}%
\pgfpathlineto{\pgfqpoint{4.215884in}{2.100390in}}%
\pgfpathlineto{\pgfqpoint{4.216027in}{2.164114in}}%
\pgfpathlineto{\pgfqpoint{4.216455in}{1.859910in}}%
\pgfpathlineto{\pgfqpoint{4.216740in}{1.994824in}}%
\pgfpathlineto{\pgfqpoint{4.217736in}{2.155709in}}%
\pgfpathlineto{\pgfqpoint{4.217167in}{1.945923in}}%
\pgfpathlineto{\pgfqpoint{4.218020in}{2.098352in}}%
\pgfpathlineto{\pgfqpoint{4.218588in}{1.874690in}}%
\pgfpathlineto{\pgfqpoint{4.218730in}{2.159155in}}%
\pgfpathlineto{\pgfqpoint{4.219014in}{1.981130in}}%
\pgfpathlineto{\pgfqpoint{4.219580in}{2.167328in}}%
\pgfpathlineto{\pgfqpoint{4.219722in}{1.862890in}}%
\pgfpathlineto{\pgfqpoint{4.220146in}{2.071800in}}%
\pgfpathlineto{\pgfqpoint{4.220711in}{1.500166in}}%
\pgfpathlineto{\pgfqpoint{4.220993in}{2.103502in}}%
\pgfpathlineto{\pgfqpoint{4.221275in}{1.918973in}}%
\pgfpathlineto{\pgfqpoint{4.221839in}{2.185957in}}%
\pgfpathlineto{\pgfqpoint{4.222402in}{2.116786in}}%
\pgfpathlineto{\pgfqpoint{4.222823in}{1.908010in}}%
\pgfpathlineto{\pgfqpoint{4.222683in}{2.142739in}}%
\pgfpathlineto{\pgfqpoint{4.223525in}{2.040807in}}%
\pgfpathlineto{\pgfqpoint{4.224086in}{1.944801in}}%
\pgfpathlineto{\pgfqpoint{4.224226in}{2.074475in}}%
\pgfpathlineto{\pgfqpoint{4.224366in}{2.012756in}}%
\pgfpathlineto{\pgfqpoint{4.224646in}{2.105419in}}%
\pgfpathlineto{\pgfqpoint{4.225205in}{2.003820in}}%
\pgfpathlineto{\pgfqpoint{4.225344in}{1.862495in}}%
\pgfpathlineto{\pgfqpoint{4.226181in}{2.063324in}}%
\pgfpathlineto{\pgfqpoint{4.226321in}{1.942975in}}%
\pgfpathlineto{\pgfqpoint{4.227156in}{2.185179in}}%
\pgfpathlineto{\pgfqpoint{4.227434in}{1.999078in}}%
\pgfpathlineto{\pgfqpoint{4.227573in}{1.782984in}}%
\pgfpathlineto{\pgfqpoint{4.227712in}{2.135338in}}%
\pgfpathlineto{\pgfqpoint{4.228544in}{1.860023in}}%
\pgfpathlineto{\pgfqpoint{4.229098in}{2.097763in}}%
\pgfpathlineto{\pgfqpoint{4.229789in}{2.097215in}}%
\pgfpathlineto{\pgfqpoint{4.230756in}{1.919624in}}%
\pgfpathlineto{\pgfqpoint{4.231995in}{2.117532in}}%
\pgfpathlineto{\pgfqpoint{4.232132in}{2.117540in}}%
\pgfpathlineto{\pgfqpoint{4.232269in}{2.147112in}}%
\pgfpathlineto{\pgfqpoint{4.232407in}{1.944788in}}%
\pgfpathlineto{\pgfqpoint{4.232819in}{2.053587in}}%
\pgfpathlineto{\pgfqpoint{4.232956in}{1.989674in}}%
\pgfpathlineto{\pgfqpoint{4.233504in}{2.174961in}}%
\pgfpathlineto{\pgfqpoint{4.233778in}{2.022082in}}%
\pgfpathlineto{\pgfqpoint{4.234052in}{2.160119in}}%
\pgfpathlineto{\pgfqpoint{4.234462in}{2.151725in}}%
\pgfpathlineto{\pgfqpoint{4.234598in}{1.860603in}}%
\pgfpathlineto{\pgfqpoint{4.235554in}{2.107066in}}%
\pgfpathlineto{\pgfqpoint{4.236507in}{1.989469in}}%
\pgfpathlineto{\pgfqpoint{4.235826in}{2.152993in}}%
\pgfpathlineto{\pgfqpoint{4.236643in}{2.049155in}}%
\pgfpathlineto{\pgfqpoint{4.237458in}{2.160345in}}%
\pgfpathlineto{\pgfqpoint{4.237322in}{1.911309in}}%
\pgfpathlineto{\pgfqpoint{4.237729in}{2.038932in}}%
\pgfpathlineto{\pgfqpoint{4.238271in}{1.864644in}}%
\pgfpathlineto{\pgfqpoint{4.238136in}{2.126079in}}%
\pgfpathlineto{\pgfqpoint{4.238542in}{2.075758in}}%
\pgfpathlineto{\pgfqpoint{4.238678in}{2.094131in}}%
\pgfpathlineto{\pgfqpoint{4.238813in}{2.038823in}}%
\pgfpathlineto{\pgfqpoint{4.238948in}{2.071415in}}%
\pgfpathlineto{\pgfqpoint{4.239624in}{1.880909in}}%
\pgfpathlineto{\pgfqpoint{4.239894in}{2.089158in}}%
\pgfpathlineto{\pgfqpoint{4.240163in}{2.163811in}}%
\pgfpathlineto{\pgfqpoint{4.240568in}{1.966547in}}%
\pgfpathlineto{\pgfqpoint{4.240972in}{2.113330in}}%
\pgfpathlineto{\pgfqpoint{4.241241in}{2.121300in}}%
\pgfpathlineto{\pgfqpoint{4.242181in}{1.937581in}}%
\pgfpathlineto{\pgfqpoint{4.242852in}{2.107133in}}%
\pgfpathlineto{\pgfqpoint{4.243120in}{2.054029in}}%
\pgfpathlineto{\pgfqpoint{4.243521in}{1.688660in}}%
\pgfpathlineto{\pgfqpoint{4.243388in}{2.147235in}}%
\pgfpathlineto{\pgfqpoint{4.244056in}{1.931181in}}%
\pgfpathlineto{\pgfqpoint{4.244190in}{2.175486in}}%
\pgfpathlineto{\pgfqpoint{4.244857in}{1.885756in}}%
\pgfpathlineto{\pgfqpoint{4.245257in}{2.117259in}}%
\pgfpathlineto{\pgfqpoint{4.245656in}{1.942992in}}%
\pgfpathlineto{\pgfqpoint{4.246322in}{2.008553in}}%
\pgfpathlineto{\pgfqpoint{4.246587in}{2.174375in}}%
\pgfpathlineto{\pgfqpoint{4.247383in}{2.024906in}}%
\pgfpathlineto{\pgfqpoint{4.247516in}{2.033858in}}%
\pgfpathlineto{\pgfqpoint{4.247649in}{1.803772in}}%
\pgfpathlineto{\pgfqpoint{4.248310in}{2.145028in}}%
\pgfpathlineto{\pgfqpoint{4.248575in}{2.007354in}}%
\pgfpathlineto{\pgfqpoint{4.249235in}{1.932561in}}%
\pgfpathlineto{\pgfqpoint{4.249103in}{2.136535in}}%
\pgfpathlineto{\pgfqpoint{4.249499in}{2.009166in}}%
\pgfpathlineto{\pgfqpoint{4.250158in}{1.971555in}}%
\pgfpathlineto{\pgfqpoint{4.250553in}{2.174725in}}%
\pgfpathlineto{\pgfqpoint{4.251736in}{1.836502in}}%
\pgfpathlineto{\pgfqpoint{4.252260in}{2.142142in}}%
\pgfpathlineto{\pgfqpoint{4.252915in}{2.121137in}}%
\pgfpathlineto{\pgfqpoint{4.253438in}{1.966112in}}%
\pgfpathlineto{\pgfqpoint{4.253961in}{2.177393in}}%
\pgfpathlineto{\pgfqpoint{4.254352in}{1.912852in}}%
\pgfpathlineto{\pgfqpoint{4.255264in}{2.008766in}}%
\pgfpathlineto{\pgfqpoint{4.255524in}{2.169086in}}%
\pgfpathlineto{\pgfqpoint{4.255654in}{2.001579in}}%
\pgfpathlineto{\pgfqpoint{4.256434in}{2.070338in}}%
\pgfpathlineto{\pgfqpoint{4.256563in}{1.865520in}}%
\pgfpathlineto{\pgfqpoint{4.257082in}{2.134270in}}%
\pgfpathlineto{\pgfqpoint{4.257600in}{1.910844in}}%
\pgfpathlineto{\pgfqpoint{4.258634in}{2.128692in}}%
\pgfpathlineto{\pgfqpoint{4.257859in}{1.858889in}}%
\pgfpathlineto{\pgfqpoint{4.258763in}{1.959793in}}%
\pgfpathlineto{\pgfqpoint{4.258892in}{1.792660in}}%
\pgfpathlineto{\pgfqpoint{4.259408in}{2.114985in}}%
\pgfpathlineto{\pgfqpoint{4.259795in}{2.016445in}}%
\pgfpathlineto{\pgfqpoint{4.260438in}{2.127820in}}%
\pgfpathlineto{\pgfqpoint{4.260695in}{1.900370in}}%
\pgfpathlineto{\pgfqpoint{4.260823in}{2.094137in}}%
\pgfpathlineto{\pgfqpoint{4.261208in}{1.944007in}}%
\pgfpathlineto{\pgfqpoint{4.261080in}{2.147787in}}%
\pgfpathlineto{\pgfqpoint{4.261978in}{2.048868in}}%
\pgfpathlineto{\pgfqpoint{4.262745in}{2.137667in}}%
\pgfpathlineto{\pgfqpoint{4.262362in}{1.906864in}}%
\pgfpathlineto{\pgfqpoint{4.263001in}{2.046199in}}%
\pgfpathlineto{\pgfqpoint{4.264022in}{1.914186in}}%
\pgfpathlineto{\pgfqpoint{4.263640in}{2.076141in}}%
\pgfpathlineto{\pgfqpoint{4.264150in}{1.986108in}}%
\pgfpathlineto{\pgfqpoint{4.264659in}{2.148865in}}%
\pgfpathlineto{\pgfqpoint{4.265168in}{1.916251in}}%
\pgfpathlineto{\pgfqpoint{4.265295in}{2.124360in}}%
\pgfpathlineto{\pgfqpoint{4.265549in}{1.896368in}}%
\pgfpathlineto{\pgfqpoint{4.266437in}{1.962262in}}%
\pgfpathlineto{\pgfqpoint{4.266817in}{1.848780in}}%
\pgfpathlineto{\pgfqpoint{4.266691in}{2.060501in}}%
\pgfpathlineto{\pgfqpoint{4.266944in}{1.895758in}}%
\pgfpathlineto{\pgfqpoint{4.268082in}{2.195162in}}%
\pgfpathlineto{\pgfqpoint{4.268208in}{1.892257in}}%
\pgfpathlineto{\pgfqpoint{4.269217in}{1.979250in}}%
\pgfpathlineto{\pgfqpoint{4.269972in}{1.705945in}}%
\pgfpathlineto{\pgfqpoint{4.269846in}{2.145356in}}%
\pgfpathlineto{\pgfqpoint{4.270223in}{2.025013in}}%
\pgfpathlineto{\pgfqpoint{4.270725in}{1.925919in}}%
\pgfpathlineto{\pgfqpoint{4.270600in}{2.031858in}}%
\pgfpathlineto{\pgfqpoint{4.270851in}{2.030697in}}%
\pgfpathlineto{\pgfqpoint{4.271101in}{2.159676in}}%
\pgfpathlineto{\pgfqpoint{4.271352in}{1.845602in}}%
\pgfpathlineto{\pgfqpoint{4.271853in}{2.008745in}}%
\pgfpathlineto{\pgfqpoint{4.271978in}{1.994930in}}%
\pgfpathlineto{\pgfqpoint{4.272103in}{2.062954in}}%
\pgfpathlineto{\pgfqpoint{4.272728in}{1.876055in}}%
\pgfpathlineto{\pgfqpoint{4.272978in}{2.175343in}}%
\pgfpathlineto{\pgfqpoint{4.273102in}{1.901000in}}%
\pgfpathlineto{\pgfqpoint{4.274100in}{2.089500in}}%
\pgfpathlineto{\pgfqpoint{4.274348in}{2.066751in}}%
\pgfpathlineto{\pgfqpoint{4.274597in}{2.151825in}}%
\pgfpathlineto{\pgfqpoint{4.275591in}{1.863470in}}%
\pgfpathlineto{\pgfqpoint{4.275715in}{2.134464in}}%
\pgfpathlineto{\pgfqpoint{4.275839in}{2.141654in}}%
\pgfpathlineto{\pgfqpoint{4.276953in}{1.725419in}}%
\pgfpathlineto{\pgfqpoint{4.276458in}{2.163540in}}%
\pgfpathlineto{\pgfqpoint{4.277077in}{1.954507in}}%
\pgfpathlineto{\pgfqpoint{4.277324in}{2.132613in}}%
\pgfpathlineto{\pgfqpoint{4.277447in}{2.117670in}}%
\pgfpathlineto{\pgfqpoint{4.277571in}{1.793121in}}%
\pgfpathlineto{\pgfqpoint{4.278188in}{2.170860in}}%
\pgfpathlineto{\pgfqpoint{4.278557in}{2.094667in}}%
\pgfpathlineto{\pgfqpoint{4.278804in}{2.025737in}}%
\pgfpathlineto{\pgfqpoint{4.279419in}{2.152425in}}%
\pgfpathlineto{\pgfqpoint{4.280278in}{1.872480in}}%
\pgfpathlineto{\pgfqpoint{4.280524in}{2.018594in}}%
\pgfpathlineto{\pgfqpoint{4.280769in}{2.166436in}}%
\pgfpathlineto{\pgfqpoint{4.281136in}{1.829886in}}%
\pgfpathlineto{\pgfqpoint{4.281259in}{2.172586in}}%
\pgfpathlineto{\pgfqpoint{4.282237in}{2.127390in}}%
\pgfpathlineto{\pgfqpoint{4.283213in}{1.991278in}}%
\pgfpathlineto{\pgfqpoint{4.282481in}{2.141847in}}%
\pgfpathlineto{\pgfqpoint{4.283335in}{2.025246in}}%
\pgfpathlineto{\pgfqpoint{4.283821in}{2.177830in}}%
\pgfpathlineto{\pgfqpoint{4.283578in}{1.796287in}}%
\pgfpathlineto{\pgfqpoint{4.284186in}{2.051917in}}%
\pgfpathlineto{\pgfqpoint{4.285158in}{1.779202in}}%
\pgfpathlineto{\pgfqpoint{4.284429in}{2.070098in}}%
\pgfpathlineto{\pgfqpoint{4.285279in}{1.996445in}}%
\pgfpathlineto{\pgfqpoint{4.285521in}{2.073881in}}%
\pgfpathlineto{\pgfqpoint{4.285764in}{1.866818in}}%
\pgfpathlineto{\pgfqpoint{4.286248in}{2.025902in}}%
\pgfpathlineto{\pgfqpoint{4.286731in}{2.077974in}}%
\pgfpathlineto{\pgfqpoint{4.287094in}{1.901130in}}%
\pgfpathlineto{\pgfqpoint{4.287214in}{2.136072in}}%
\pgfpathlineto{\pgfqpoint{4.287335in}{1.900611in}}%
\pgfpathlineto{\pgfqpoint{4.288299in}{2.105239in}}%
\pgfpathlineto{\pgfqpoint{4.288420in}{2.113268in}}%
\pgfpathlineto{\pgfqpoint{4.288660in}{1.807381in}}%
\pgfpathlineto{\pgfqpoint{4.289261in}{2.178868in}}%
\pgfpathlineto{\pgfqpoint{4.289501in}{2.102974in}}%
\pgfpathlineto{\pgfqpoint{4.289741in}{2.075105in}}%
\pgfpathlineto{\pgfqpoint{4.290700in}{1.573164in}}%
\pgfpathlineto{\pgfqpoint{4.290221in}{2.199353in}}%
\pgfpathlineto{\pgfqpoint{4.290820in}{2.006722in}}%
\pgfpathlineto{\pgfqpoint{4.290940in}{2.118319in}}%
\pgfpathlineto{\pgfqpoint{4.291179in}{1.966240in}}%
\pgfpathlineto{\pgfqpoint{4.291657in}{2.041707in}}%
\pgfpathlineto{\pgfqpoint{4.291776in}{1.774639in}}%
\pgfpathlineto{\pgfqpoint{4.292134in}{2.184207in}}%
\pgfpathlineto{\pgfqpoint{4.292849in}{1.848497in}}%
\pgfpathlineto{\pgfqpoint{4.293445in}{1.814925in}}%
\pgfpathlineto{\pgfqpoint{4.293920in}{2.152437in}}%
\pgfpathlineto{\pgfqpoint{4.294988in}{1.842417in}}%
\pgfpathlineto{\pgfqpoint{4.294632in}{2.153436in}}%
\pgfpathlineto{\pgfqpoint{4.295106in}{1.936794in}}%
\pgfpathlineto{\pgfqpoint{4.295935in}{2.179034in}}%
\pgfpathlineto{\pgfqpoint{4.296053in}{2.106441in}}%
\pgfpathlineto{\pgfqpoint{4.296171in}{1.784438in}}%
\pgfpathlineto{\pgfqpoint{4.296762in}{2.118811in}}%
\pgfpathlineto{\pgfqpoint{4.297234in}{1.818190in}}%
\pgfpathlineto{\pgfqpoint{4.297351in}{1.814361in}}%
\pgfpathlineto{\pgfqpoint{4.298058in}{1.758062in}}%
\pgfpathlineto{\pgfqpoint{4.298646in}{2.128418in}}%
\pgfpathlineto{\pgfqpoint{4.298998in}{1.923331in}}%
\pgfpathlineto{\pgfqpoint{4.299233in}{2.142269in}}%
\pgfpathlineto{\pgfqpoint{4.299702in}{2.004668in}}%
\pgfpathlineto{\pgfqpoint{4.300404in}{2.163442in}}%
\pgfpathlineto{\pgfqpoint{4.300521in}{1.953180in}}%
\pgfpathlineto{\pgfqpoint{4.300755in}{1.992979in}}%
\pgfpathlineto{\pgfqpoint{4.300872in}{1.981101in}}%
\pgfpathlineto{\pgfqpoint{4.300989in}{2.012272in}}%
\pgfpathlineto{\pgfqpoint{4.301106in}{2.152487in}}%
\pgfpathlineto{\pgfqpoint{4.301456in}{1.896064in}}%
\pgfpathlineto{\pgfqpoint{4.302039in}{2.126913in}}%
\pgfpathlineto{\pgfqpoint{4.302156in}{1.966085in}}%
\pgfpathlineto{\pgfqpoint{4.302738in}{2.178180in}}%
\pgfpathlineto{\pgfqpoint{4.303087in}{2.143756in}}%
\pgfpathlineto{\pgfqpoint{4.303900in}{1.711497in}}%
\pgfpathlineto{\pgfqpoint{4.304132in}{2.167716in}}%
\pgfpathlineto{\pgfqpoint{4.304364in}{2.022283in}}%
\pgfpathlineto{\pgfqpoint{4.304596in}{2.161542in}}%
\pgfpathlineto{\pgfqpoint{4.304712in}{1.805559in}}%
\pgfpathlineto{\pgfqpoint{4.305521in}{2.150502in}}%
\pgfpathlineto{\pgfqpoint{4.306445in}{1.852717in}}%
\pgfpathlineto{\pgfqpoint{4.305868in}{2.158666in}}%
\pgfpathlineto{\pgfqpoint{4.306676in}{2.011876in}}%
\pgfpathlineto{\pgfqpoint{4.307252in}{2.183691in}}%
\pgfpathlineto{\pgfqpoint{4.306906in}{2.001293in}}%
\pgfpathlineto{\pgfqpoint{4.307712in}{2.111820in}}%
\pgfpathlineto{\pgfqpoint{4.308172in}{1.950081in}}%
\pgfpathlineto{\pgfqpoint{4.307942in}{2.177393in}}%
\pgfpathlineto{\pgfqpoint{4.308746in}{2.129625in}}%
\pgfpathlineto{\pgfqpoint{4.309090in}{1.765834in}}%
\pgfpathlineto{\pgfqpoint{4.309434in}{2.227337in}}%
\pgfpathlineto{\pgfqpoint{4.309892in}{2.074929in}}%
\pgfpathlineto{\pgfqpoint{4.310463in}{2.197791in}}%
\pgfpathlineto{\pgfqpoint{4.310349in}{1.927317in}}%
\pgfpathlineto{\pgfqpoint{4.311034in}{2.164985in}}%
\pgfpathlineto{\pgfqpoint{4.312174in}{2.011154in}}%
\pgfpathlineto{\pgfqpoint{4.312742in}{1.915884in}}%
\pgfpathlineto{\pgfqpoint{4.313310in}{2.158081in}}%
\pgfpathlineto{\pgfqpoint{4.313424in}{2.157221in}}%
\pgfpathlineto{\pgfqpoint{4.313651in}{1.837267in}}%
\pgfpathlineto{\pgfqpoint{4.314557in}{2.034964in}}%
\pgfpathlineto{\pgfqpoint{4.314896in}{2.184685in}}%
\pgfpathlineto{\pgfqpoint{4.315009in}{1.932777in}}%
\pgfpathlineto{\pgfqpoint{4.315574in}{2.056763in}}%
\pgfpathlineto{\pgfqpoint{4.316251in}{1.683648in}}%
\pgfpathlineto{\pgfqpoint{4.316138in}{2.106324in}}%
\pgfpathlineto{\pgfqpoint{4.316702in}{2.020860in}}%
\pgfpathlineto{\pgfqpoint{4.317039in}{1.864127in}}%
\pgfpathlineto{\pgfqpoint{4.317826in}{2.209616in}}%
\pgfpathlineto{\pgfqpoint{4.318835in}{1.874239in}}%
\pgfpathlineto{\pgfqpoint{4.319059in}{2.059041in}}%
\pgfpathlineto{\pgfqpoint{4.319395in}{1.856751in}}%
\pgfpathlineto{\pgfqpoint{4.319954in}{2.118989in}}%
\pgfpathlineto{\pgfqpoint{4.320066in}{2.114886in}}%
\pgfpathlineto{\pgfqpoint{4.320178in}{2.126589in}}%
\pgfpathlineto{\pgfqpoint{4.321182in}{2.127127in}}%
\pgfpathlineto{\pgfqpoint{4.321404in}{1.776363in}}%
\pgfpathlineto{\pgfqpoint{4.321516in}{2.128579in}}%
\pgfpathlineto{\pgfqpoint{4.322516in}{2.107876in}}%
\pgfpathlineto{\pgfqpoint{4.322960in}{1.939723in}}%
\pgfpathlineto{\pgfqpoint{4.323071in}{2.175073in}}%
\pgfpathlineto{\pgfqpoint{4.323626in}{1.978613in}}%
\pgfpathlineto{\pgfqpoint{4.323958in}{2.238364in}}%
\pgfpathlineto{\pgfqpoint{4.324400in}{2.102058in}}%
\pgfpathlineto{\pgfqpoint{4.324621in}{1.835185in}}%
\pgfpathlineto{\pgfqpoint{4.324732in}{2.182838in}}%
\pgfpathlineto{\pgfqpoint{4.325504in}{1.985554in}}%
\pgfpathlineto{\pgfqpoint{4.326386in}{2.155107in}}%
\pgfpathlineto{\pgfqpoint{4.326606in}{2.079719in}}%
\pgfpathlineto{\pgfqpoint{4.327485in}{2.155805in}}%
\pgfpathlineto{\pgfqpoint{4.327704in}{1.865269in}}%
\pgfpathlineto{\pgfqpoint{4.328143in}{2.137198in}}%
\pgfpathlineto{\pgfqpoint{4.328909in}{2.013392in}}%
\pgfpathlineto{\pgfqpoint{4.329784in}{1.971035in}}%
\pgfpathlineto{\pgfqpoint{4.330111in}{2.151390in}}%
\pgfpathlineto{\pgfqpoint{4.330220in}{2.173109in}}%
\pgfpathlineto{\pgfqpoint{4.330329in}{1.905977in}}%
\pgfpathlineto{\pgfqpoint{4.331309in}{2.153934in}}%
\pgfpathlineto{\pgfqpoint{4.331744in}{1.919763in}}%
\pgfpathlineto{\pgfqpoint{4.331636in}{2.187046in}}%
\pgfpathlineto{\pgfqpoint{4.332938in}{1.931217in}}%
\pgfpathlineto{\pgfqpoint{4.333912in}{2.166418in}}%
\pgfpathlineto{\pgfqpoint{4.334020in}{2.110370in}}%
\pgfpathlineto{\pgfqpoint{4.334128in}{1.926144in}}%
\pgfpathlineto{\pgfqpoint{4.334453in}{2.210036in}}%
\pgfpathlineto{\pgfqpoint{4.335208in}{1.970314in}}%
\pgfpathlineto{\pgfqpoint{4.335962in}{2.111553in}}%
\pgfpathlineto{\pgfqpoint{4.335531in}{1.773875in}}%
\pgfpathlineto{\pgfqpoint{4.336177in}{2.041204in}}%
\pgfpathlineto{\pgfqpoint{4.336821in}{2.185521in}}%
\pgfpathlineto{\pgfqpoint{4.337251in}{1.752379in}}%
\pgfpathlineto{\pgfqpoint{4.337787in}{2.196190in}}%
\pgfpathlineto{\pgfqpoint{4.338429in}{1.970865in}}%
\pgfpathlineto{\pgfqpoint{4.339284in}{2.171174in}}%
\pgfpathlineto{\pgfqpoint{4.339498in}{2.125914in}}%
\pgfpathlineto{\pgfqpoint{4.339817in}{1.878315in}}%
\pgfpathlineto{\pgfqpoint{4.340563in}{2.066219in}}%
\pgfpathlineto{\pgfqpoint{4.341201in}{1.986679in}}%
\pgfpathlineto{\pgfqpoint{4.340989in}{2.155288in}}%
\pgfpathlineto{\pgfqpoint{4.341414in}{2.119292in}}%
\pgfpathlineto{\pgfqpoint{4.341520in}{2.134180in}}%
\pgfpathlineto{\pgfqpoint{4.341626in}{2.069101in}}%
\pgfpathlineto{\pgfqpoint{4.341732in}{2.075195in}}%
\pgfpathlineto{\pgfqpoint{4.342157in}{1.836304in}}%
\pgfpathlineto{\pgfqpoint{4.342051in}{2.110582in}}%
\pgfpathlineto{\pgfqpoint{4.342793in}{2.089979in}}%
\pgfpathlineto{\pgfqpoint{4.342898in}{2.203467in}}%
\pgfpathlineto{\pgfqpoint{4.343110in}{1.985217in}}%
\pgfpathlineto{\pgfqpoint{4.343744in}{2.152086in}}%
\pgfpathlineto{\pgfqpoint{4.344272in}{1.969631in}}%
\pgfpathlineto{\pgfqpoint{4.344905in}{2.013188in}}%
\pgfpathlineto{\pgfqpoint{4.345115in}{2.117371in}}%
\pgfpathlineto{\pgfqpoint{4.345221in}{1.989527in}}%
\pgfpathlineto{\pgfqpoint{4.345326in}{2.058941in}}%
\pgfpathlineto{\pgfqpoint{4.345431in}{1.879347in}}%
\pgfpathlineto{\pgfqpoint{4.346062in}{2.145627in}}%
\pgfpathlineto{\pgfqpoint{4.346377in}{2.111598in}}%
\pgfpathlineto{\pgfqpoint{4.346482in}{2.107899in}}%
\pgfpathlineto{\pgfqpoint{4.346797in}{1.956378in}}%
\pgfpathlineto{\pgfqpoint{4.346902in}{2.116062in}}%
\pgfpathlineto{\pgfqpoint{4.347531in}{2.057005in}}%
\pgfpathlineto{\pgfqpoint{4.347635in}{2.169084in}}%
\pgfpathlineto{\pgfqpoint{4.347949in}{1.967850in}}%
\pgfpathlineto{\pgfqpoint{4.348576in}{2.092877in}}%
\pgfpathlineto{\pgfqpoint{4.348681in}{2.091734in}}%
\pgfpathlineto{\pgfqpoint{4.348785in}{2.143693in}}%
\pgfpathlineto{\pgfqpoint{4.348889in}{1.916568in}}%
\pgfpathlineto{\pgfqpoint{4.349411in}{2.107491in}}%
\pgfpathlineto{\pgfqpoint{4.349515in}{1.951969in}}%
\pgfpathlineto{\pgfqpoint{4.349828in}{2.163629in}}%
\pgfpathlineto{\pgfqpoint{4.350452in}{1.982064in}}%
\pgfpathlineto{\pgfqpoint{4.350556in}{2.108970in}}%
\pgfpathlineto{\pgfqpoint{4.351076in}{1.911870in}}%
\pgfpathlineto{\pgfqpoint{4.351491in}{2.069814in}}%
\pgfpathlineto{\pgfqpoint{4.351594in}{1.851488in}}%
\pgfpathlineto{\pgfqpoint{4.352009in}{2.149483in}}%
\pgfpathlineto{\pgfqpoint{4.352527in}{2.006338in}}%
\pgfpathlineto{\pgfqpoint{4.352630in}{2.173110in}}%
\pgfpathlineto{\pgfqpoint{4.352734in}{1.936853in}}%
\pgfpathlineto{\pgfqpoint{4.353663in}{2.105051in}}%
\pgfpathlineto{\pgfqpoint{4.353767in}{1.834749in}}%
\pgfpathlineto{\pgfqpoint{4.354076in}{2.223334in}}%
\pgfpathlineto{\pgfqpoint{4.354797in}{1.955361in}}%
\pgfpathlineto{\pgfqpoint{4.354900in}{2.163482in}}%
\pgfpathlineto{\pgfqpoint{4.355106in}{1.656658in}}%
\pgfpathlineto{\pgfqpoint{4.355825in}{2.039077in}}%
\pgfpathlineto{\pgfqpoint{4.355928in}{1.879661in}}%
\pgfpathlineto{\pgfqpoint{4.356236in}{2.158999in}}%
\pgfpathlineto{\pgfqpoint{4.356851in}{1.917412in}}%
\pgfpathlineto{\pgfqpoint{4.357772in}{2.135588in}}%
\pgfpathlineto{\pgfqpoint{4.357976in}{2.105873in}}%
\pgfpathlineto{\pgfqpoint{4.358588in}{1.878942in}}%
\pgfpathlineto{\pgfqpoint{4.359098in}{2.057572in}}%
\pgfpathlineto{\pgfqpoint{4.359302in}{2.025616in}}%
\pgfpathlineto{\pgfqpoint{4.359404in}{2.184607in}}%
\pgfpathlineto{\pgfqpoint{4.360217in}{2.001890in}}%
\pgfpathlineto{\pgfqpoint{4.360421in}{2.158856in}}%
\pgfpathlineto{\pgfqpoint{4.360725in}{2.167508in}}%
\pgfpathlineto{\pgfqpoint{4.361840in}{1.846442in}}%
\pgfpathlineto{\pgfqpoint{4.362548in}{2.186176in}}%
\pgfpathlineto{\pgfqpoint{4.362952in}{2.062510in}}%
\pgfpathlineto{\pgfqpoint{4.363659in}{1.880285in}}%
\pgfpathlineto{\pgfqpoint{4.363154in}{2.186630in}}%
\pgfpathlineto{\pgfqpoint{4.364062in}{1.939140in}}%
\pgfpathlineto{\pgfqpoint{4.364364in}{2.152580in}}%
\pgfpathlineto{\pgfqpoint{4.364263in}{1.898801in}}%
\pgfpathlineto{\pgfqpoint{4.365168in}{2.093400in}}%
\pgfpathlineto{\pgfqpoint{4.365369in}{1.865795in}}%
\pgfpathlineto{\pgfqpoint{4.365971in}{2.220795in}}%
\pgfpathlineto{\pgfqpoint{4.366271in}{2.098793in}}%
\pgfpathlineto{\pgfqpoint{4.366772in}{2.196654in}}%
\pgfpathlineto{\pgfqpoint{4.366672in}{1.902222in}}%
\pgfpathlineto{\pgfqpoint{4.367072in}{2.026262in}}%
\pgfpathlineto{\pgfqpoint{4.367172in}{1.674344in}}%
\pgfpathlineto{\pgfqpoint{4.367472in}{2.154746in}}%
\pgfpathlineto{\pgfqpoint{4.368071in}{2.019417in}}%
\pgfpathlineto{\pgfqpoint{4.368768in}{2.184972in}}%
\pgfpathlineto{\pgfqpoint{4.368270in}{1.956497in}}%
\pgfpathlineto{\pgfqpoint{4.369067in}{1.962826in}}%
\pgfpathlineto{\pgfqpoint{4.369167in}{1.819914in}}%
\pgfpathlineto{\pgfqpoint{4.369763in}{2.177837in}}%
\pgfpathlineto{\pgfqpoint{4.370160in}{1.968940in}}%
\pgfpathlineto{\pgfqpoint{4.370954in}{2.196262in}}%
\pgfpathlineto{\pgfqpoint{4.370855in}{1.781025in}}%
\pgfpathlineto{\pgfqpoint{4.371449in}{2.195621in}}%
\pgfpathlineto{\pgfqpoint{4.372339in}{1.708297in}}%
\pgfpathlineto{\pgfqpoint{4.372635in}{2.025557in}}%
\pgfpathlineto{\pgfqpoint{4.372733in}{2.018588in}}%
\pgfpathlineto{\pgfqpoint{4.372832in}{2.021321in}}%
\pgfpathlineto{\pgfqpoint{4.373817in}{2.163053in}}%
\pgfpathlineto{\pgfqpoint{4.373621in}{1.856712in}}%
\pgfpathlineto{\pgfqpoint{4.373916in}{2.079557in}}%
\pgfpathlineto{\pgfqpoint{4.374014in}{1.903427in}}%
\pgfpathlineto{\pgfqpoint{4.374506in}{2.183062in}}%
\pgfpathlineto{\pgfqpoint{4.374997in}{1.998967in}}%
\pgfpathlineto{\pgfqpoint{4.375291in}{2.180862in}}%
\pgfpathlineto{\pgfqpoint{4.375683in}{1.858914in}}%
\pgfpathlineto{\pgfqpoint{4.375977in}{2.120579in}}%
\pgfpathlineto{\pgfqpoint{4.376955in}{1.828967in}}%
\pgfpathlineto{\pgfqpoint{4.376271in}{2.186943in}}%
\pgfpathlineto{\pgfqpoint{4.377150in}{1.939748in}}%
\pgfpathlineto{\pgfqpoint{4.378126in}{2.185465in}}%
\pgfpathlineto{\pgfqpoint{4.378223in}{2.131111in}}%
\pgfpathlineto{\pgfqpoint{4.378515in}{2.175351in}}%
\pgfpathlineto{\pgfqpoint{4.379390in}{1.853821in}}%
\pgfpathlineto{\pgfqpoint{4.379487in}{2.191717in}}%
\pgfpathlineto{\pgfqpoint{4.380360in}{1.816669in}}%
\pgfpathlineto{\pgfqpoint{4.380554in}{2.067515in}}%
\pgfpathlineto{\pgfqpoint{4.380651in}{2.061341in}}%
\pgfpathlineto{\pgfqpoint{4.380748in}{2.219933in}}%
\pgfpathlineto{\pgfqpoint{4.380845in}{1.986227in}}%
\pgfpathlineto{\pgfqpoint{4.381812in}{2.174047in}}%
\pgfpathlineto{\pgfqpoint{4.382584in}{1.775349in}}%
\pgfpathlineto{\pgfqpoint{4.382969in}{1.949827in}}%
\pgfpathlineto{\pgfqpoint{4.383065in}{2.158514in}}%
\pgfpathlineto{\pgfqpoint{4.383739in}{1.896817in}}%
\pgfpathlineto{\pgfqpoint{4.384027in}{2.029443in}}%
\pgfpathlineto{\pgfqpoint{4.384795in}{1.853968in}}%
\pgfpathlineto{\pgfqpoint{4.384987in}{2.159746in}}%
\pgfpathlineto{\pgfqpoint{4.385178in}{2.133317in}}%
\pgfpathlineto{\pgfqpoint{4.385849in}{2.158710in}}%
\pgfpathlineto{\pgfqpoint{4.386231in}{1.843362in}}%
\pgfpathlineto{\pgfqpoint{4.386518in}{2.145265in}}%
\pgfpathlineto{\pgfqpoint{4.386613in}{1.792998in}}%
\pgfpathlineto{\pgfqpoint{4.387377in}{1.996856in}}%
\pgfpathlineto{\pgfqpoint{4.388043in}{2.179202in}}%
\pgfpathlineto{\pgfqpoint{4.387853in}{1.981529in}}%
\pgfpathlineto{\pgfqpoint{4.388519in}{2.038338in}}%
\pgfpathlineto{\pgfqpoint{4.388614in}{2.038897in}}%
\pgfpathlineto{\pgfqpoint{4.388899in}{2.150338in}}%
\pgfpathlineto{\pgfqpoint{4.389468in}{1.965424in}}%
\pgfpathlineto{\pgfqpoint{4.390226in}{2.148379in}}%
\pgfpathlineto{\pgfqpoint{4.390416in}{1.894600in}}%
\pgfpathlineto{\pgfqpoint{4.390510in}{1.949642in}}%
\pgfpathlineto{\pgfqpoint{4.390605in}{1.966363in}}%
\pgfpathlineto{\pgfqpoint{4.390700in}{2.185214in}}%
\pgfpathlineto{\pgfqpoint{4.390794in}{1.730726in}}%
\pgfpathlineto{\pgfqpoint{4.391739in}{2.143944in}}%
\pgfpathlineto{\pgfqpoint{4.391833in}{1.898815in}}%
\pgfpathlineto{\pgfqpoint{4.392210in}{2.161257in}}%
\pgfpathlineto{\pgfqpoint{4.392775in}{2.074987in}}%
\pgfpathlineto{\pgfqpoint{4.393527in}{2.207821in}}%
\pgfpathlineto{\pgfqpoint{4.393715in}{1.983863in}}%
\pgfpathlineto{\pgfqpoint{4.394278in}{1.805638in}}%
\pgfpathlineto{\pgfqpoint{4.393997in}{2.151490in}}%
\pgfpathlineto{\pgfqpoint{4.394747in}{2.058943in}}%
\pgfpathlineto{\pgfqpoint{4.395028in}{1.815776in}}%
\pgfpathlineto{\pgfqpoint{4.395496in}{2.159537in}}%
\pgfpathlineto{\pgfqpoint{4.395589in}{2.157285in}}%
\pgfpathlineto{\pgfqpoint{4.396150in}{1.933470in}}%
\pgfpathlineto{\pgfqpoint{4.396430in}{2.208027in}}%
\pgfpathlineto{\pgfqpoint{4.396896in}{1.966554in}}%
\pgfpathlineto{\pgfqpoint{4.397641in}{2.183410in}}%
\pgfpathlineto{\pgfqpoint{4.397269in}{1.890262in}}%
\pgfpathlineto{\pgfqpoint{4.398106in}{2.149607in}}%
\pgfpathlineto{\pgfqpoint{4.398478in}{1.663265in}}%
\pgfpathlineto{\pgfqpoint{4.399035in}{2.177635in}}%
\pgfpathlineto{\pgfqpoint{4.399313in}{1.917682in}}%
\pgfpathlineto{\pgfqpoint{4.400053in}{2.209031in}}%
\pgfpathlineto{\pgfqpoint{4.399776in}{1.858538in}}%
\pgfpathlineto{\pgfqpoint{4.400516in}{2.118068in}}%
\pgfpathlineto{\pgfqpoint{4.400608in}{2.114893in}}%
\pgfpathlineto{\pgfqpoint{4.400701in}{2.138528in}}%
\pgfpathlineto{\pgfqpoint{4.401531in}{1.949262in}}%
\pgfpathlineto{\pgfqpoint{4.401439in}{2.184227in}}%
\pgfpathlineto{\pgfqpoint{4.401808in}{2.068853in}}%
\pgfpathlineto{\pgfqpoint{4.402268in}{2.204150in}}%
\pgfpathlineto{\pgfqpoint{4.402544in}{1.933321in}}%
\pgfpathlineto{\pgfqpoint{4.402728in}{2.122556in}}%
\pgfpathlineto{\pgfqpoint{4.402820in}{1.921321in}}%
\pgfpathlineto{\pgfqpoint{4.403371in}{2.142472in}}%
\pgfpathlineto{\pgfqpoint{4.403830in}{1.960000in}}%
\pgfpathlineto{\pgfqpoint{4.404288in}{2.226521in}}%
\pgfpathlineto{\pgfqpoint{4.404746in}{1.825954in}}%
\pgfpathlineto{\pgfqpoint{4.404837in}{2.162984in}}%
\pgfpathlineto{\pgfqpoint{4.405751in}{2.213210in}}%
\pgfpathlineto{\pgfqpoint{4.405933in}{1.775056in}}%
\pgfpathlineto{\pgfqpoint{4.406298in}{2.190925in}}%
\pgfpathlineto{\pgfqpoint{4.407118in}{2.070563in}}%
\pgfpathlineto{\pgfqpoint{4.407209in}{2.078613in}}%
\pgfpathlineto{\pgfqpoint{4.407300in}{1.772829in}}%
\pgfpathlineto{\pgfqpoint{4.407936in}{2.209065in}}%
\pgfpathlineto{\pgfqpoint{4.408299in}{1.920026in}}%
\pgfpathlineto{\pgfqpoint{4.408480in}{2.197986in}}%
\pgfpathlineto{\pgfqpoint{4.408752in}{1.916021in}}%
\pgfpathlineto{\pgfqpoint{4.409658in}{2.136043in}}%
\pgfpathlineto{\pgfqpoint{4.410381in}{1.799486in}}%
\pgfpathlineto{\pgfqpoint{4.410471in}{2.148719in}}%
\pgfpathlineto{\pgfqpoint{4.410742in}{2.110146in}}%
\pgfpathlineto{\pgfqpoint{4.411553in}{1.934714in}}%
\pgfpathlineto{\pgfqpoint{4.411463in}{2.169663in}}%
\pgfpathlineto{\pgfqpoint{4.411823in}{2.091081in}}%
\pgfpathlineto{\pgfqpoint{4.412722in}{2.167700in}}%
\pgfpathlineto{\pgfqpoint{4.412003in}{1.933594in}}%
\pgfpathlineto{\pgfqpoint{4.412812in}{2.034887in}}%
\pgfpathlineto{\pgfqpoint{4.412991in}{1.986198in}}%
\pgfpathlineto{\pgfqpoint{4.413171in}{2.097281in}}%
\pgfpathlineto{\pgfqpoint{4.413350in}{2.039887in}}%
\pgfpathlineto{\pgfqpoint{4.413440in}{2.155381in}}%
\pgfpathlineto{\pgfqpoint{4.413529in}{1.722216in}}%
\pgfpathlineto{\pgfqpoint{4.413798in}{2.236712in}}%
\pgfpathlineto{\pgfqpoint{4.414514in}{2.063728in}}%
\pgfpathlineto{\pgfqpoint{4.415139in}{2.173399in}}%
\pgfpathlineto{\pgfqpoint{4.415050in}{1.823016in}}%
\pgfpathlineto{\pgfqpoint{4.415496in}{2.088426in}}%
\pgfpathlineto{\pgfqpoint{4.415764in}{1.980747in}}%
\pgfpathlineto{\pgfqpoint{4.415675in}{2.209980in}}%
\pgfpathlineto{\pgfqpoint{4.416388in}{2.121252in}}%
\pgfpathlineto{\pgfqpoint{4.416477in}{2.178970in}}%
\pgfpathlineto{\pgfqpoint{4.416566in}{1.852800in}}%
\pgfpathlineto{\pgfqpoint{4.417366in}{2.145012in}}%
\pgfpathlineto{\pgfqpoint{4.417809in}{1.857656in}}%
\pgfpathlineto{\pgfqpoint{4.418076in}{2.145347in}}%
\pgfpathlineto{\pgfqpoint{4.418430in}{2.123982in}}%
\pgfpathlineto{\pgfqpoint{4.418873in}{1.987223in}}%
\pgfpathlineto{\pgfqpoint{4.419315in}{2.136548in}}%
\pgfpathlineto{\pgfqpoint{4.419403in}{2.149463in}}%
\pgfpathlineto{\pgfqpoint{4.419669in}{2.120048in}}%
\pgfpathlineto{\pgfqpoint{4.420110in}{1.965871in}}%
\pgfpathlineto{\pgfqpoint{4.420286in}{2.159835in}}%
\pgfpathlineto{\pgfqpoint{4.420815in}{2.067174in}}%
\pgfpathlineto{\pgfqpoint{4.421080in}{2.145011in}}%
\pgfpathlineto{\pgfqpoint{4.422047in}{1.776835in}}%
\pgfpathlineto{\pgfqpoint{4.423012in}{2.114603in}}%
\pgfpathlineto{\pgfqpoint{4.422837in}{1.750869in}}%
\pgfpathlineto{\pgfqpoint{4.423187in}{2.057226in}}%
\pgfpathlineto{\pgfqpoint{4.423450in}{1.840844in}}%
\pgfpathlineto{\pgfqpoint{4.424237in}{2.162531in}}%
\pgfpathlineto{\pgfqpoint{4.424587in}{2.025109in}}%
\pgfpathlineto{\pgfqpoint{4.424499in}{2.164032in}}%
\pgfpathlineto{\pgfqpoint{4.425372in}{2.046774in}}%
\pgfpathlineto{\pgfqpoint{4.425633in}{1.951445in}}%
\pgfpathlineto{\pgfqpoint{4.426416in}{2.182792in}}%
\pgfpathlineto{\pgfqpoint{4.426851in}{1.877719in}}%
\pgfpathlineto{\pgfqpoint{4.427545in}{2.070234in}}%
\pgfpathlineto{\pgfqpoint{4.427805in}{1.838594in}}%
\pgfpathlineto{\pgfqpoint{4.428152in}{2.143406in}}%
\pgfpathlineto{\pgfqpoint{4.428238in}{2.162794in}}%
\pgfpathlineto{\pgfqpoint{4.428411in}{1.964679in}}%
\pgfpathlineto{\pgfqpoint{4.428671in}{2.016853in}}%
\pgfpathlineto{\pgfqpoint{4.428757in}{1.933054in}}%
\pgfpathlineto{\pgfqpoint{4.429449in}{2.175872in}}%
\pgfpathlineto{\pgfqpoint{4.429621in}{2.095934in}}%
\pgfpathlineto{\pgfqpoint{4.430052in}{2.215528in}}%
\pgfpathlineto{\pgfqpoint{4.430139in}{1.922836in}}%
\pgfpathlineto{\pgfqpoint{4.430397in}{2.001801in}}%
\pgfpathlineto{\pgfqpoint{4.431000in}{1.820474in}}%
\pgfpathlineto{\pgfqpoint{4.430914in}{2.131297in}}%
\pgfpathlineto{\pgfqpoint{4.431515in}{1.906384in}}%
\pgfpathlineto{\pgfqpoint{4.432202in}{2.211218in}}%
\pgfpathlineto{\pgfqpoint{4.433401in}{2.112458in}}%
\pgfpathlineto{\pgfqpoint{4.433828in}{1.932148in}}%
\pgfpathlineto{\pgfqpoint{4.434170in}{2.164094in}}%
\pgfpathlineto{\pgfqpoint{4.434511in}{2.112004in}}%
\pgfpathlineto{\pgfqpoint{4.435278in}{1.839424in}}%
\pgfpathlineto{\pgfqpoint{4.435534in}{2.043684in}}%
\pgfpathlineto{\pgfqpoint{4.435789in}{2.211682in}}%
\pgfpathlineto{\pgfqpoint{4.436299in}{1.869215in}}%
\pgfpathlineto{\pgfqpoint{4.436469in}{2.119154in}}%
\pgfpathlineto{\pgfqpoint{4.437317in}{1.891804in}}%
\pgfpathlineto{\pgfqpoint{4.437063in}{2.119599in}}%
\pgfpathlineto{\pgfqpoint{4.437571in}{2.060682in}}%
\pgfpathlineto{\pgfqpoint{4.437825in}{2.127039in}}%
\pgfpathlineto{\pgfqpoint{4.438332in}{1.894401in}}%
\pgfpathlineto{\pgfqpoint{4.438502in}{2.164397in}}%
\pgfpathlineto{\pgfqpoint{4.439346in}{2.149777in}}%
\pgfpathlineto{\pgfqpoint{4.439430in}{1.834322in}}%
\pgfpathlineto{\pgfqpoint{4.439936in}{2.188600in}}%
\pgfpathlineto{\pgfqpoint{4.440441in}{2.027428in}}%
\pgfpathlineto{\pgfqpoint{4.440525in}{1.988883in}}%
\pgfpathlineto{\pgfqpoint{4.440861in}{2.189448in}}%
\pgfpathlineto{\pgfqpoint{4.441281in}{2.096430in}}%
\pgfpathlineto{\pgfqpoint{4.441365in}{2.114463in}}%
\pgfpathlineto{\pgfqpoint{4.441449in}{2.021184in}}%
\pgfpathlineto{\pgfqpoint{4.441533in}{1.611708in}}%
\pgfpathlineto{\pgfqpoint{4.441952in}{2.139201in}}%
\pgfpathlineto{\pgfqpoint{4.442455in}{2.112864in}}%
\pgfpathlineto{\pgfqpoint{4.442957in}{2.202729in}}%
\pgfpathlineto{\pgfqpoint{4.442789in}{1.971983in}}%
\pgfpathlineto{\pgfqpoint{4.443374in}{2.156514in}}%
\pgfpathlineto{\pgfqpoint{4.443792in}{1.897170in}}%
\pgfpathlineto{\pgfqpoint{4.444459in}{2.100010in}}%
\pgfpathlineto{\pgfqpoint{4.444792in}{2.171677in}}%
\pgfpathlineto{\pgfqpoint{4.444626in}{1.881098in}}%
\pgfpathlineto{\pgfqpoint{4.445291in}{2.014038in}}%
\pgfpathlineto{\pgfqpoint{4.445375in}{2.003026in}}%
\pgfpathlineto{\pgfqpoint{4.446454in}{2.216790in}}%
\pgfpathlineto{\pgfqpoint{4.446288in}{1.868958in}}%
\pgfpathlineto{\pgfqpoint{4.446537in}{2.107517in}}%
\pgfpathlineto{\pgfqpoint{4.447365in}{1.682872in}}%
\pgfpathlineto{\pgfqpoint{4.446786in}{2.177131in}}%
\pgfpathlineto{\pgfqpoint{4.447614in}{1.969960in}}%
\pgfpathlineto{\pgfqpoint{4.448605in}{2.197880in}}%
\pgfpathlineto{\pgfqpoint{4.448687in}{1.987695in}}%
\pgfpathlineto{\pgfqpoint{4.448770in}{1.986162in}}%
\pgfpathlineto{\pgfqpoint{4.449100in}{2.168631in}}%
\pgfpathlineto{\pgfqpoint{4.449264in}{1.853410in}}%
\pgfpathlineto{\pgfqpoint{4.449841in}{2.112893in}}%
\pgfpathlineto{\pgfqpoint{4.450663in}{1.986686in}}%
\pgfpathlineto{\pgfqpoint{4.450581in}{2.169517in}}%
\pgfpathlineto{\pgfqpoint{4.450827in}{2.167004in}}%
\pgfpathlineto{\pgfqpoint{4.450909in}{2.167219in}}%
\pgfpathlineto{\pgfqpoint{4.451237in}{1.824339in}}%
\pgfpathlineto{\pgfqpoint{4.452056in}{2.015687in}}%
\pgfpathlineto{\pgfqpoint{4.452465in}{2.202826in}}%
\pgfpathlineto{\pgfqpoint{4.452874in}{1.938968in}}%
\pgfpathlineto{\pgfqpoint{4.453201in}{2.070273in}}%
\pgfpathlineto{\pgfqpoint{4.453282in}{1.891733in}}%
\pgfpathlineto{\pgfqpoint{4.454016in}{2.180021in}}%
\pgfpathlineto{\pgfqpoint{4.454179in}{2.144464in}}%
\pgfpathlineto{\pgfqpoint{4.454260in}{2.188543in}}%
\pgfpathlineto{\pgfqpoint{4.454342in}{1.949855in}}%
\pgfpathlineto{\pgfqpoint{4.454911in}{2.039982in}}%
\pgfpathlineto{\pgfqpoint{4.455074in}{1.892561in}}%
\pgfpathlineto{\pgfqpoint{4.455155in}{2.186941in}}%
\pgfpathlineto{\pgfqpoint{4.455805in}{2.107847in}}%
\pgfpathlineto{\pgfqpoint{4.455886in}{2.216070in}}%
\pgfpathlineto{\pgfqpoint{4.456210in}{1.795858in}}%
\pgfpathlineto{\pgfqpoint{4.456777in}{2.122587in}}%
\pgfpathlineto{\pgfqpoint{4.457101in}{1.882444in}}%
\pgfpathlineto{\pgfqpoint{4.457586in}{2.194485in}}%
\pgfpathlineto{\pgfqpoint{4.457909in}{2.002227in}}%
\pgfpathlineto{\pgfqpoint{4.458876in}{2.202959in}}%
\pgfpathlineto{\pgfqpoint{4.458070in}{1.887524in}}%
\pgfpathlineto{\pgfqpoint{4.458957in}{2.122899in}}%
\pgfpathlineto{\pgfqpoint{4.459118in}{2.014752in}}%
\pgfpathlineto{\pgfqpoint{4.459198in}{2.167908in}}%
\pgfpathlineto{\pgfqpoint{4.460083in}{2.035762in}}%
\pgfpathlineto{\pgfqpoint{4.460404in}{2.235112in}}%
\pgfpathlineto{\pgfqpoint{4.460484in}{1.938650in}}%
\pgfpathlineto{\pgfqpoint{4.461125in}{2.099005in}}%
\pgfpathlineto{\pgfqpoint{4.462245in}{1.693820in}}%
\pgfpathlineto{\pgfqpoint{4.461766in}{2.181716in}}%
\pgfpathlineto{\pgfqpoint{4.462325in}{2.017562in}}%
\pgfpathlineto{\pgfqpoint{4.462804in}{2.226988in}}%
\pgfpathlineto{\pgfqpoint{4.462565in}{1.917813in}}%
\pgfpathlineto{\pgfqpoint{4.463442in}{2.108554in}}%
\pgfpathlineto{\pgfqpoint{4.463522in}{1.968489in}}%
\pgfpathlineto{\pgfqpoint{4.463681in}{2.206727in}}%
\pgfpathlineto{\pgfqpoint{4.464556in}{2.034489in}}%
\pgfpathlineto{\pgfqpoint{4.465271in}{2.194699in}}%
\pgfpathlineto{\pgfqpoint{4.465509in}{1.972102in}}%
\pgfpathlineto{\pgfqpoint{4.465667in}{2.096177in}}%
\pgfpathlineto{\pgfqpoint{4.465984in}{2.147302in}}%
\pgfpathlineto{\pgfqpoint{4.465826in}{1.918388in}}%
\pgfpathlineto{\pgfqpoint{4.466222in}{2.035160in}}%
\pgfpathlineto{\pgfqpoint{4.466459in}{1.791257in}}%
\pgfpathlineto{\pgfqpoint{4.466934in}{2.113988in}}%
\pgfpathlineto{\pgfqpoint{4.467250in}{2.033389in}}%
\pgfpathlineto{\pgfqpoint{4.467960in}{2.174167in}}%
\pgfpathlineto{\pgfqpoint{4.468039in}{1.889012in}}%
\pgfpathlineto{\pgfqpoint{4.468432in}{2.128176in}}%
\pgfpathlineto{\pgfqpoint{4.468511in}{1.923920in}}%
\pgfpathlineto{\pgfqpoint{4.468747in}{2.175793in}}%
\pgfpathlineto{\pgfqpoint{4.469533in}{2.048467in}}%
\pgfpathlineto{\pgfqpoint{4.469691in}{2.146732in}}%
\pgfpathlineto{\pgfqpoint{4.470083in}{1.875783in}}%
\pgfpathlineto{\pgfqpoint{4.470710in}{2.089930in}}%
\pgfpathlineto{\pgfqpoint{4.471023in}{1.908427in}}%
\pgfpathlineto{\pgfqpoint{4.471101in}{2.183112in}}%
\pgfpathlineto{\pgfqpoint{4.471805in}{1.969640in}}%
\pgfpathlineto{\pgfqpoint{4.472196in}{2.185959in}}%
\pgfpathlineto{\pgfqpoint{4.472039in}{1.914433in}}%
\pgfpathlineto{\pgfqpoint{4.472897in}{2.013830in}}%
\pgfpathlineto{\pgfqpoint{4.473209in}{1.939863in}}%
\pgfpathlineto{\pgfqpoint{4.473365in}{2.150496in}}%
\pgfpathlineto{\pgfqpoint{4.473909in}{2.056543in}}%
\pgfpathlineto{\pgfqpoint{4.474142in}{2.196342in}}%
\pgfpathlineto{\pgfqpoint{4.474220in}{1.924036in}}%
\pgfpathlineto{\pgfqpoint{4.474996in}{2.125957in}}%
\pgfpathlineto{\pgfqpoint{4.475926in}{1.861297in}}%
\pgfpathlineto{\pgfqpoint{4.475461in}{2.178708in}}%
\pgfpathlineto{\pgfqpoint{4.476080in}{2.086740in}}%
\pgfpathlineto{\pgfqpoint{4.477084in}{2.197097in}}%
\pgfpathlineto{\pgfqpoint{4.476544in}{2.014583in}}%
\pgfpathlineto{\pgfqpoint{4.477162in}{2.110471in}}%
\pgfpathlineto{\pgfqpoint{4.477239in}{1.914689in}}%
\pgfpathlineto{\pgfqpoint{4.477393in}{2.204308in}}%
\pgfpathlineto{\pgfqpoint{4.478240in}{2.092463in}}%
\pgfpathlineto{\pgfqpoint{4.478394in}{2.045892in}}%
\pgfpathlineto{\pgfqpoint{4.478932in}{1.913094in}}%
\pgfpathlineto{\pgfqpoint{4.479009in}{2.163242in}}%
\pgfpathlineto{\pgfqpoint{4.479470in}{2.034703in}}%
\pgfpathlineto{\pgfqpoint{4.479546in}{2.168661in}}%
\pgfpathlineto{\pgfqpoint{4.480389in}{1.973226in}}%
\pgfpathlineto{\pgfqpoint{4.480542in}{2.116655in}}%
\pgfpathlineto{\pgfqpoint{4.481001in}{1.866665in}}%
\pgfpathlineto{\pgfqpoint{4.481383in}{2.193646in}}%
\pgfpathlineto{\pgfqpoint{4.481765in}{1.956292in}}%
\pgfpathlineto{\pgfqpoint{4.481841in}{2.232384in}}%
\pgfpathlineto{\pgfqpoint{4.481994in}{1.884867in}}%
\pgfpathlineto{\pgfqpoint{4.482832in}{1.899350in}}%
\pgfpathlineto{\pgfqpoint{4.483289in}{2.165079in}}%
\pgfpathlineto{\pgfqpoint{4.483441in}{1.703153in}}%
\pgfpathlineto{\pgfqpoint{4.483972in}{1.988502in}}%
\pgfpathlineto{\pgfqpoint{4.484276in}{1.820818in}}%
\pgfpathlineto{\pgfqpoint{4.484428in}{2.183026in}}%
\pgfpathlineto{\pgfqpoint{4.484882in}{2.039041in}}%
\pgfpathlineto{\pgfqpoint{4.485034in}{1.963212in}}%
\pgfpathlineto{\pgfqpoint{4.486017in}{2.200554in}}%
\pgfpathlineto{\pgfqpoint{4.486847in}{1.670124in}}%
\pgfpathlineto{\pgfqpoint{4.487300in}{2.013001in}}%
\pgfpathlineto{\pgfqpoint{4.487450in}{1.965645in}}%
\pgfpathlineto{\pgfqpoint{4.488503in}{2.198582in}}%
\pgfpathlineto{\pgfqpoint{4.489403in}{1.885796in}}%
\pgfpathlineto{\pgfqpoint{4.489628in}{2.077236in}}%
\pgfpathlineto{\pgfqpoint{4.489703in}{2.228256in}}%
\pgfpathlineto{\pgfqpoint{4.490451in}{1.957428in}}%
\pgfpathlineto{\pgfqpoint{4.490676in}{2.092675in}}%
\pgfpathlineto{\pgfqpoint{4.491049in}{1.956681in}}%
\pgfpathlineto{\pgfqpoint{4.491348in}{2.222638in}}%
\pgfpathlineto{\pgfqpoint{4.491646in}{1.964698in}}%
\pgfpathlineto{\pgfqpoint{4.491720in}{2.183180in}}%
\pgfpathlineto{\pgfqpoint{4.492763in}{2.022526in}}%
\pgfpathlineto{\pgfqpoint{4.493654in}{2.224108in}}%
\pgfpathlineto{\pgfqpoint{4.493134in}{1.928193in}}%
\pgfpathlineto{\pgfqpoint{4.493802in}{2.151971in}}%
\pgfpathlineto{\pgfqpoint{4.494618in}{1.712250in}}%
\pgfpathlineto{\pgfqpoint{4.493951in}{2.166895in}}%
\pgfpathlineto{\pgfqpoint{4.494914in}{2.049527in}}%
\pgfpathlineto{\pgfqpoint{4.494988in}{2.220922in}}%
\pgfpathlineto{\pgfqpoint{4.495431in}{1.926902in}}%
\pgfpathlineto{\pgfqpoint{4.496022in}{2.112827in}}%
\pgfpathlineto{\pgfqpoint{4.496538in}{1.833647in}}%
\pgfpathlineto{\pgfqpoint{4.496907in}{2.181211in}}%
\pgfpathlineto{\pgfqpoint{4.497201in}{2.040044in}}%
\pgfpathlineto{\pgfqpoint{4.498010in}{2.265916in}}%
\pgfpathlineto{\pgfqpoint{4.497716in}{1.968694in}}%
\pgfpathlineto{\pgfqpoint{4.498230in}{2.183283in}}%
\pgfpathlineto{\pgfqpoint{4.498303in}{1.738573in}}%
\pgfpathlineto{\pgfqpoint{4.499183in}{2.186968in}}%
\pgfpathlineto{\pgfqpoint{4.499330in}{2.082295in}}%
\pgfpathlineto{\pgfqpoint{4.499695in}{2.184438in}}%
\pgfpathlineto{\pgfqpoint{4.499476in}{2.060020in}}%
\pgfpathlineto{\pgfqpoint{4.499842in}{2.124351in}}%
\pgfpathlineto{\pgfqpoint{4.499915in}{1.913519in}}%
\pgfpathlineto{\pgfqpoint{4.500499in}{2.174190in}}%
\pgfpathlineto{\pgfqpoint{4.500937in}{2.122322in}}%
\pgfpathlineto{\pgfqpoint{4.501666in}{2.244780in}}%
\pgfpathlineto{\pgfqpoint{4.501229in}{2.014138in}}%
\pgfpathlineto{\pgfqpoint{4.502103in}{2.172358in}}%
\pgfpathlineto{\pgfqpoint{4.502539in}{1.855876in}}%
\pgfpathlineto{\pgfqpoint{4.502321in}{2.183231in}}%
\pgfpathlineto{\pgfqpoint{4.503192in}{2.097480in}}%
\pgfpathlineto{\pgfqpoint{4.503555in}{2.163425in}}%
\pgfpathlineto{\pgfqpoint{4.503483in}{2.030189in}}%
\pgfpathlineto{\pgfqpoint{4.503772in}{2.091911in}}%
\pgfpathlineto{\pgfqpoint{4.503845in}{1.844584in}}%
\pgfpathlineto{\pgfqpoint{4.504062in}{2.234917in}}%
\pgfpathlineto{\pgfqpoint{4.504858in}{2.012096in}}%
\pgfpathlineto{\pgfqpoint{4.505868in}{2.169029in}}%
\pgfpathlineto{\pgfqpoint{4.505652in}{1.963218in}}%
\pgfpathlineto{\pgfqpoint{4.506013in}{2.099302in}}%
\pgfpathlineto{\pgfqpoint{4.506085in}{1.915123in}}%
\pgfpathlineto{\pgfqpoint{4.506805in}{2.191541in}}%
\pgfpathlineto{\pgfqpoint{4.507092in}{2.008525in}}%
\pgfpathlineto{\pgfqpoint{4.507523in}{2.158660in}}%
\pgfpathlineto{\pgfqpoint{4.507236in}{1.881638in}}%
\pgfpathlineto{\pgfqpoint{4.508313in}{2.145949in}}%
\pgfpathlineto{\pgfqpoint{4.509315in}{1.962915in}}%
\pgfpathlineto{\pgfqpoint{4.508814in}{2.252497in}}%
\pgfpathlineto{\pgfqpoint{4.509387in}{2.206076in}}%
\pgfpathlineto{\pgfqpoint{4.510672in}{1.953319in}}%
\pgfpathlineto{\pgfqpoint{4.511028in}{2.219369in}}%
\pgfpathlineto{\pgfqpoint{4.511099in}{1.766243in}}%
\pgfpathlineto{\pgfqpoint{4.511668in}{2.105667in}}%
\pgfpathlineto{\pgfqpoint{4.511740in}{1.751987in}}%
\pgfpathlineto{\pgfqpoint{4.511882in}{2.205087in}}%
\pgfpathlineto{\pgfqpoint{4.512734in}{2.147081in}}%
\pgfpathlineto{\pgfqpoint{4.513726in}{1.904389in}}%
\pgfpathlineto{\pgfqpoint{4.513655in}{2.235588in}}%
\pgfpathlineto{\pgfqpoint{4.513797in}{2.072920in}}%
\pgfpathlineto{\pgfqpoint{4.514292in}{2.198495in}}%
\pgfpathlineto{\pgfqpoint{4.514150in}{1.858525in}}%
\pgfpathlineto{\pgfqpoint{4.514857in}{2.080210in}}%
\pgfpathlineto{\pgfqpoint{4.515492in}{1.908341in}}%
\pgfpathlineto{\pgfqpoint{4.515773in}{2.188341in}}%
\pgfpathlineto{\pgfqpoint{4.515914in}{2.084545in}}%
\pgfpathlineto{\pgfqpoint{4.516829in}{2.223976in}}%
\pgfpathlineto{\pgfqpoint{4.516477in}{1.873864in}}%
\pgfpathlineto{\pgfqpoint{4.517039in}{2.159293in}}%
\pgfpathlineto{\pgfqpoint{4.517811in}{1.897750in}}%
\pgfpathlineto{\pgfqpoint{4.518021in}{2.168884in}}%
\pgfpathlineto{\pgfqpoint{4.518091in}{2.142684in}}%
\pgfpathlineto{\pgfqpoint{4.518161in}{2.144013in}}%
\pgfpathlineto{\pgfqpoint{4.519141in}{1.933172in}}%
\pgfpathlineto{\pgfqpoint{4.518721in}{2.211855in}}%
\pgfpathlineto{\pgfqpoint{4.519211in}{2.135351in}}%
\pgfpathlineto{\pgfqpoint{4.519281in}{2.137460in}}%
\pgfpathlineto{\pgfqpoint{4.520188in}{1.898490in}}%
\pgfpathlineto{\pgfqpoint{4.519909in}{2.191022in}}%
\pgfpathlineto{\pgfqpoint{4.520397in}{2.013911in}}%
\pgfpathlineto{\pgfqpoint{4.520466in}{2.182943in}}%
\pgfpathlineto{\pgfqpoint{4.521371in}{1.938430in}}%
\pgfpathlineto{\pgfqpoint{4.521510in}{2.052928in}}%
\pgfpathlineto{\pgfqpoint{4.521996in}{2.184962in}}%
\pgfpathlineto{\pgfqpoint{4.522482in}{1.965768in}}%
\pgfpathlineto{\pgfqpoint{4.522620in}{2.064569in}}%
\pgfpathlineto{\pgfqpoint{4.523174in}{1.764660in}}%
\pgfpathlineto{\pgfqpoint{4.522897in}{2.189440in}}%
\pgfpathlineto{\pgfqpoint{4.523659in}{1.968706in}}%
\pgfpathlineto{\pgfqpoint{4.524625in}{2.236464in}}%
\pgfpathlineto{\pgfqpoint{4.524142in}{1.945286in}}%
\pgfpathlineto{\pgfqpoint{4.524763in}{2.123377in}}%
\pgfpathlineto{\pgfqpoint{4.525315in}{1.812171in}}%
\pgfpathlineto{\pgfqpoint{4.525590in}{2.190875in}}%
\pgfpathlineto{\pgfqpoint{4.525865in}{1.960004in}}%
\pgfpathlineto{\pgfqpoint{4.526896in}{1.883912in}}%
\pgfpathlineto{\pgfqpoint{4.526964in}{2.236141in}}%
\pgfpathlineto{\pgfqpoint{4.527855in}{1.924615in}}%
\pgfpathlineto{\pgfqpoint{4.528061in}{1.958525in}}%
\pgfpathlineto{\pgfqpoint{4.528881in}{2.169674in}}%
\pgfpathlineto{\pgfqpoint{4.529017in}{1.860319in}}%
\pgfpathlineto{\pgfqpoint{4.529222in}{2.107294in}}%
\pgfpathlineto{\pgfqpoint{4.529495in}{1.950912in}}%
\pgfpathlineto{\pgfqpoint{4.530176in}{2.218743in}}%
\pgfpathlineto{\pgfqpoint{4.530313in}{2.086389in}}%
\pgfpathlineto{\pgfqpoint{4.531196in}{2.204670in}}%
\pgfpathlineto{\pgfqpoint{4.531264in}{1.999569in}}%
\pgfpathlineto{\pgfqpoint{4.531332in}{2.103225in}}%
\pgfpathlineto{\pgfqpoint{4.532214in}{1.907797in}}%
\pgfpathlineto{\pgfqpoint{4.531536in}{2.139372in}}%
\pgfpathlineto{\pgfqpoint{4.532553in}{1.953529in}}%
\pgfpathlineto{\pgfqpoint{4.533634in}{2.229908in}}%
\pgfpathlineto{\pgfqpoint{4.533702in}{2.173667in}}%
\pgfpathlineto{\pgfqpoint{4.534039in}{1.920017in}}%
\pgfpathlineto{\pgfqpoint{4.534646in}{2.162547in}}%
\pgfpathlineto{\pgfqpoint{4.535521in}{2.240511in}}%
\pgfpathlineto{\pgfqpoint{4.534848in}{1.896899in}}%
\pgfpathlineto{\pgfqpoint{4.535588in}{2.036613in}}%
\pgfpathlineto{\pgfqpoint{4.536260in}{1.857080in}}%
\pgfpathlineto{\pgfqpoint{4.535790in}{2.175030in}}%
\pgfpathlineto{\pgfqpoint{4.536327in}{2.155707in}}%
\pgfpathlineto{\pgfqpoint{4.536394in}{2.180680in}}%
\pgfpathlineto{\pgfqpoint{4.536797in}{1.992992in}}%
\pgfpathlineto{\pgfqpoint{4.537266in}{2.124708in}}%
\pgfpathlineto{\pgfqpoint{4.537466in}{1.869867in}}%
\pgfpathlineto{\pgfqpoint{4.537533in}{2.246391in}}%
\pgfpathlineto{\pgfqpoint{4.538402in}{1.962296in}}%
\pgfpathlineto{\pgfqpoint{4.539203in}{2.185648in}}%
\pgfpathlineto{\pgfqpoint{4.539403in}{1.755470in}}%
\pgfpathlineto{\pgfqpoint{4.539536in}{2.047288in}}%
\pgfpathlineto{\pgfqpoint{4.540467in}{2.169152in}}%
\pgfpathlineto{\pgfqpoint{4.540068in}{1.935626in}}%
\pgfpathlineto{\pgfqpoint{4.540600in}{2.135037in}}%
\pgfpathlineto{\pgfqpoint{4.541529in}{1.954938in}}%
\pgfpathlineto{\pgfqpoint{4.540866in}{2.181677in}}%
\pgfpathlineto{\pgfqpoint{4.541662in}{2.091835in}}%
\pgfpathlineto{\pgfqpoint{4.541927in}{1.775067in}}%
\pgfpathlineto{\pgfqpoint{4.542721in}{2.206825in}}%
\pgfpathlineto{\pgfqpoint{4.543711in}{2.236863in}}%
\pgfpathlineto{\pgfqpoint{4.543843in}{1.932284in}}%
\pgfpathlineto{\pgfqpoint{4.545028in}{2.212959in}}%
\pgfpathlineto{\pgfqpoint{4.544567in}{1.818951in}}%
\pgfpathlineto{\pgfqpoint{4.545094in}{2.138339in}}%
\pgfpathlineto{\pgfqpoint{4.545488in}{1.900272in}}%
\pgfpathlineto{\pgfqpoint{4.546144in}{2.166921in}}%
\pgfpathlineto{\pgfqpoint{4.546275in}{2.052949in}}%
\pgfpathlineto{\pgfqpoint{4.546537in}{1.877012in}}%
\pgfpathlineto{\pgfqpoint{4.547388in}{2.211050in}}%
\pgfpathlineto{\pgfqpoint{4.547911in}{1.738081in}}%
\pgfpathlineto{\pgfqpoint{4.548498in}{2.040785in}}%
\pgfpathlineto{\pgfqpoint{4.548694in}{2.244646in}}%
\pgfpathlineto{\pgfqpoint{4.549410in}{1.930681in}}%
\pgfpathlineto{\pgfqpoint{4.549605in}{2.110576in}}%
\pgfpathlineto{\pgfqpoint{4.550320in}{1.772760in}}%
\pgfpathlineto{\pgfqpoint{4.550190in}{2.158842in}}%
\pgfpathlineto{\pgfqpoint{4.550709in}{2.095565in}}%
\pgfpathlineto{\pgfqpoint{4.550774in}{2.206631in}}%
\pgfpathlineto{\pgfqpoint{4.551293in}{1.724667in}}%
\pgfpathlineto{\pgfqpoint{4.551811in}{2.155379in}}%
\pgfpathlineto{\pgfqpoint{4.551940in}{2.014224in}}%
\pgfpathlineto{\pgfqpoint{4.552715in}{2.225902in}}%
\pgfpathlineto{\pgfqpoint{4.553360in}{1.918961in}}%
\pgfpathlineto{\pgfqpoint{4.553876in}{2.027327in}}%
\pgfpathlineto{\pgfqpoint{4.554712in}{1.862524in}}%
\pgfpathlineto{\pgfqpoint{4.554776in}{2.203937in}}%
\pgfpathlineto{\pgfqpoint{4.554969in}{2.014165in}}%
\pgfpathlineto{\pgfqpoint{4.555611in}{2.211109in}}%
\pgfpathlineto{\pgfqpoint{4.555418in}{1.928639in}}%
\pgfpathlineto{\pgfqpoint{4.556187in}{2.145992in}}%
\pgfpathlineto{\pgfqpoint{4.556315in}{2.229352in}}%
\pgfpathlineto{\pgfqpoint{4.556507in}{2.085596in}}%
\pgfpathlineto{\pgfqpoint{4.556891in}{2.117366in}}%
\pgfpathlineto{\pgfqpoint{4.557658in}{1.827075in}}%
\pgfpathlineto{\pgfqpoint{4.557211in}{2.291683in}}%
\pgfpathlineto{\pgfqpoint{4.558040in}{2.033724in}}%
\pgfpathlineto{\pgfqpoint{4.558996in}{2.225496in}}%
\pgfpathlineto{\pgfqpoint{4.558232in}{1.759922in}}%
\pgfpathlineto{\pgfqpoint{4.559250in}{2.198936in}}%
\pgfpathlineto{\pgfqpoint{4.560329in}{1.938126in}}%
\pgfpathlineto{\pgfqpoint{4.559822in}{2.202255in}}%
\pgfpathlineto{\pgfqpoint{4.560456in}{2.023126in}}%
\pgfpathlineto{\pgfqpoint{4.560836in}{2.225576in}}%
\pgfpathlineto{\pgfqpoint{4.560646in}{1.991029in}}%
\pgfpathlineto{\pgfqpoint{4.560900in}{2.078271in}}%
\pgfpathlineto{\pgfqpoint{4.560963in}{1.875118in}}%
\pgfpathlineto{\pgfqpoint{4.561216in}{2.214779in}}%
\pgfpathlineto{\pgfqpoint{4.561975in}{2.151701in}}%
\pgfpathlineto{\pgfqpoint{4.562543in}{1.751911in}}%
\pgfpathlineto{\pgfqpoint{4.562354in}{2.167682in}}%
\pgfpathlineto{\pgfqpoint{4.562858in}{2.131915in}}%
\pgfpathlineto{\pgfqpoint{4.563236in}{2.228847in}}%
\pgfpathlineto{\pgfqpoint{4.562984in}{1.927796in}}%
\pgfpathlineto{\pgfqpoint{4.563929in}{2.198696in}}%
\pgfpathlineto{\pgfqpoint{4.564683in}{1.827301in}}%
\pgfpathlineto{\pgfqpoint{4.564431in}{2.209299in}}%
\pgfpathlineto{\pgfqpoint{4.564996in}{2.060906in}}%
\pgfpathlineto{\pgfqpoint{4.565435in}{2.244671in}}%
\pgfpathlineto{\pgfqpoint{4.565122in}{1.820735in}}%
\pgfpathlineto{\pgfqpoint{4.566124in}{2.216271in}}%
\pgfpathlineto{\pgfqpoint{4.567248in}{1.910298in}}%
\pgfpathlineto{\pgfqpoint{4.567311in}{2.088287in}}%
\pgfpathlineto{\pgfqpoint{4.567622in}{1.859351in}}%
\pgfpathlineto{\pgfqpoint{4.567560in}{2.198087in}}%
\pgfpathlineto{\pgfqpoint{4.567809in}{2.052132in}}%
\pgfpathlineto{\pgfqpoint{4.568556in}{2.209153in}}%
\pgfpathlineto{\pgfqpoint{4.568307in}{1.890264in}}%
\pgfpathlineto{\pgfqpoint{4.568929in}{2.109037in}}%
\pgfpathlineto{\pgfqpoint{4.569054in}{2.102913in}}%
\pgfpathlineto{\pgfqpoint{4.569426in}{1.846509in}}%
\pgfpathlineto{\pgfqpoint{4.569736in}{2.194970in}}%
\pgfpathlineto{\pgfqpoint{4.570170in}{2.060319in}}%
\pgfpathlineto{\pgfqpoint{4.570232in}{1.942203in}}%
\pgfpathlineto{\pgfqpoint{4.570728in}{2.222113in}}%
\pgfpathlineto{\pgfqpoint{4.571284in}{1.975002in}}%
\pgfpathlineto{\pgfqpoint{4.571840in}{2.218288in}}%
\pgfpathlineto{\pgfqpoint{4.572395in}{2.036764in}}%
\pgfpathlineto{\pgfqpoint{4.572765in}{1.912802in}}%
\pgfpathlineto{\pgfqpoint{4.572888in}{2.163755in}}%
\pgfpathlineto{\pgfqpoint{4.573319in}{2.078746in}}%
\pgfpathlineto{\pgfqpoint{4.573626in}{2.236266in}}%
\pgfpathlineto{\pgfqpoint{4.573442in}{1.981761in}}%
\pgfpathlineto{\pgfqpoint{4.574424in}{2.182001in}}%
\pgfpathlineto{\pgfqpoint{4.574486in}{1.752218in}}%
\pgfpathlineto{\pgfqpoint{4.574731in}{2.239006in}}%
\pgfpathlineto{\pgfqpoint{4.575527in}{2.087219in}}%
\pgfpathlineto{\pgfqpoint{4.576505in}{1.886237in}}%
\pgfpathlineto{\pgfqpoint{4.575772in}{2.265026in}}%
\pgfpathlineto{\pgfqpoint{4.576627in}{1.984152in}}%
\pgfpathlineto{\pgfqpoint{4.577358in}{2.218549in}}%
\pgfpathlineto{\pgfqpoint{4.577176in}{1.794209in}}%
\pgfpathlineto{\pgfqpoint{4.577785in}{2.112924in}}%
\pgfpathlineto{\pgfqpoint{4.577967in}{2.170667in}}%
\pgfpathlineto{\pgfqpoint{4.578089in}{1.896161in}}%
\pgfpathlineto{\pgfqpoint{4.578271in}{2.066558in}}%
\pgfpathlineto{\pgfqpoint{4.578332in}{1.880622in}}%
\pgfpathlineto{\pgfqpoint{4.579303in}{2.238254in}}%
\pgfpathlineto{\pgfqpoint{4.579425in}{1.888922in}}%
\pgfpathlineto{\pgfqpoint{4.580273in}{2.248732in}}%
\pgfpathlineto{\pgfqpoint{4.580575in}{2.176153in}}%
\pgfpathlineto{\pgfqpoint{4.581300in}{1.740392in}}%
\pgfpathlineto{\pgfqpoint{4.580877in}{2.202196in}}%
\pgfpathlineto{\pgfqpoint{4.581722in}{2.002698in}}%
\pgfpathlineto{\pgfqpoint{4.581782in}{2.244026in}}%
\pgfpathlineto{\pgfqpoint{4.582084in}{1.988091in}}%
\pgfpathlineto{\pgfqpoint{4.582806in}{2.107425in}}%
\pgfpathlineto{\pgfqpoint{4.583587in}{1.915080in}}%
\pgfpathlineto{\pgfqpoint{4.583407in}{2.214341in}}%
\pgfpathlineto{\pgfqpoint{4.583767in}{2.112086in}}%
\pgfpathlineto{\pgfqpoint{4.583827in}{2.209551in}}%
\pgfpathlineto{\pgfqpoint{4.584067in}{1.991633in}}%
\pgfpathlineto{\pgfqpoint{4.584846in}{2.096038in}}%
\pgfpathlineto{\pgfqpoint{4.585145in}{2.223037in}}%
\pgfpathlineto{\pgfqpoint{4.585982in}{1.945572in}}%
\pgfpathlineto{\pgfqpoint{4.586220in}{2.225992in}}%
\pgfpathlineto{\pgfqpoint{4.587055in}{2.173031in}}%
\pgfpathlineto{\pgfqpoint{4.587471in}{1.864558in}}%
\pgfpathlineto{\pgfqpoint{4.587709in}{2.291196in}}%
\pgfpathlineto{\pgfqpoint{4.588185in}{1.891381in}}%
\pgfpathlineto{\pgfqpoint{4.588719in}{2.230872in}}%
\pgfpathlineto{\pgfqpoint{4.588659in}{1.717479in}}%
\pgfpathlineto{\pgfqpoint{4.589489in}{2.127328in}}%
\pgfpathlineto{\pgfqpoint{4.590435in}{1.848595in}}%
\pgfpathlineto{\pgfqpoint{4.590081in}{2.204971in}}%
\pgfpathlineto{\pgfqpoint{4.590612in}{1.963538in}}%
\pgfpathlineto{\pgfqpoint{4.591379in}{2.216876in}}%
\pgfpathlineto{\pgfqpoint{4.591144in}{1.904819in}}%
\pgfpathlineto{\pgfqpoint{4.591851in}{2.205780in}}%
\pgfpathlineto{\pgfqpoint{4.592792in}{1.966492in}}%
\pgfpathlineto{\pgfqpoint{4.592204in}{2.238242in}}%
\pgfpathlineto{\pgfqpoint{4.593144in}{2.062712in}}%
\pgfpathlineto{\pgfqpoint{4.593730in}{2.230619in}}%
\pgfpathlineto{\pgfqpoint{4.593848in}{1.682730in}}%
\pgfpathlineto{\pgfqpoint{4.594258in}{2.151642in}}%
\pgfpathlineto{\pgfqpoint{4.594609in}{1.947132in}}%
\pgfpathlineto{\pgfqpoint{4.595018in}{2.276972in}}%
\pgfpathlineto{\pgfqpoint{4.595368in}{1.981716in}}%
\pgfpathlineto{\pgfqpoint{4.596418in}{2.217060in}}%
\pgfpathlineto{\pgfqpoint{4.596476in}{2.122657in}}%
\pgfpathlineto{\pgfqpoint{4.597349in}{2.012392in}}%
\pgfpathlineto{\pgfqpoint{4.597000in}{2.179918in}}%
\pgfpathlineto{\pgfqpoint{4.597639in}{2.012791in}}%
\pgfpathlineto{\pgfqpoint{4.598568in}{2.188382in}}%
\pgfpathlineto{\pgfqpoint{4.598336in}{1.874337in}}%
\pgfpathlineto{\pgfqpoint{4.598799in}{2.148740in}}%
\pgfpathlineto{\pgfqpoint{4.599552in}{2.240292in}}%
\pgfpathlineto{\pgfqpoint{4.599956in}{1.857301in}}%
\pgfpathlineto{\pgfqpoint{4.600995in}{2.243075in}}%
\pgfpathlineto{\pgfqpoint{4.601052in}{2.091921in}}%
\pgfpathlineto{\pgfqpoint{4.601915in}{2.004962in}}%
\pgfpathlineto{\pgfqpoint{4.601283in}{2.242834in}}%
\pgfpathlineto{\pgfqpoint{4.601973in}{2.097052in}}%
\pgfpathlineto{\pgfqpoint{4.602605in}{2.204981in}}%
\pgfpathlineto{\pgfqpoint{4.602318in}{1.811382in}}%
\pgfpathlineto{\pgfqpoint{4.603006in}{2.119033in}}%
\pgfpathlineto{\pgfqpoint{4.603121in}{1.880301in}}%
\pgfpathlineto{\pgfqpoint{4.603637in}{2.210589in}}%
\pgfpathlineto{\pgfqpoint{4.604152in}{2.042623in}}%
\pgfpathlineto{\pgfqpoint{4.604209in}{2.209801in}}%
\pgfpathlineto{\pgfqpoint{4.604323in}{1.909470in}}%
\pgfpathlineto{\pgfqpoint{4.605237in}{2.038876in}}%
\pgfpathlineto{\pgfqpoint{4.605351in}{2.213386in}}%
\pgfpathlineto{\pgfqpoint{4.605408in}{1.900469in}}%
\pgfpathlineto{\pgfqpoint{4.605921in}{2.091288in}}%
\pgfpathlineto{\pgfqpoint{4.605978in}{1.728221in}}%
\pgfpathlineto{\pgfqpoint{4.606945in}{2.233349in}}%
\pgfpathlineto{\pgfqpoint{4.607002in}{2.189857in}}%
\pgfpathlineto{\pgfqpoint{4.607796in}{2.025489in}}%
\pgfpathlineto{\pgfqpoint{4.607513in}{2.264512in}}%
\pgfpathlineto{\pgfqpoint{4.608136in}{2.089231in}}%
\pgfpathlineto{\pgfqpoint{4.608646in}{2.191273in}}%
\pgfpathlineto{\pgfqpoint{4.608589in}{1.933378in}}%
\pgfpathlineto{\pgfqpoint{4.609155in}{2.058023in}}%
\pgfpathlineto{\pgfqpoint{4.609720in}{1.805533in}}%
\pgfpathlineto{\pgfqpoint{4.609268in}{2.192767in}}%
\pgfpathlineto{\pgfqpoint{4.610171in}{1.898409in}}%
\pgfpathlineto{\pgfqpoint{4.611241in}{2.244270in}}%
\pgfpathlineto{\pgfqpoint{4.611297in}{1.987811in}}%
\pgfpathlineto{\pgfqpoint{4.612252in}{1.906826in}}%
\pgfpathlineto{\pgfqpoint{4.612421in}{2.277110in}}%
\pgfpathlineto{\pgfqpoint{4.613205in}{1.776394in}}%
\pgfpathlineto{\pgfqpoint{4.613541in}{2.097200in}}%
\pgfpathlineto{\pgfqpoint{4.613709in}{2.193153in}}%
\pgfpathlineto{\pgfqpoint{4.613932in}{1.691994in}}%
\pgfpathlineto{\pgfqpoint{4.614658in}{2.141278in}}%
\pgfpathlineto{\pgfqpoint{4.614714in}{1.998337in}}%
\pgfpathlineto{\pgfqpoint{4.614770in}{2.231589in}}%
\pgfpathlineto{\pgfqpoint{4.615717in}{2.156263in}}%
\pgfpathlineto{\pgfqpoint{4.616051in}{2.199257in}}%
\pgfpathlineto{\pgfqpoint{4.616162in}{2.046009in}}%
\pgfpathlineto{\pgfqpoint{4.616218in}{2.136235in}}%
\pgfpathlineto{\pgfqpoint{4.616273in}{1.886403in}}%
\pgfpathlineto{\pgfqpoint{4.617106in}{2.229031in}}%
\pgfpathlineto{\pgfqpoint{4.617328in}{1.972262in}}%
\pgfpathlineto{\pgfqpoint{4.617771in}{2.270402in}}%
\pgfpathlineto{\pgfqpoint{4.618214in}{1.956036in}}%
\pgfpathlineto{\pgfqpoint{4.618435in}{2.017913in}}%
\pgfpathlineto{\pgfqpoint{4.619098in}{1.787297in}}%
\pgfpathlineto{\pgfqpoint{4.618822in}{2.234814in}}%
\pgfpathlineto{\pgfqpoint{4.619485in}{1.979507in}}%
\pgfpathlineto{\pgfqpoint{4.619981in}{2.224033in}}%
\pgfpathlineto{\pgfqpoint{4.620091in}{1.970615in}}%
\pgfpathlineto{\pgfqpoint{4.620587in}{2.016597in}}%
\pgfpathlineto{\pgfqpoint{4.621027in}{2.221876in}}%
\pgfpathlineto{\pgfqpoint{4.620972in}{1.894998in}}%
\pgfpathlineto{\pgfqpoint{4.621740in}{2.202130in}}%
\pgfpathlineto{\pgfqpoint{4.621905in}{1.884595in}}%
\pgfpathlineto{\pgfqpoint{4.622617in}{2.230151in}}%
\pgfpathlineto{\pgfqpoint{4.622836in}{2.189280in}}%
\pgfpathlineto{\pgfqpoint{4.623547in}{1.855934in}}%
\pgfpathlineto{\pgfqpoint{4.623165in}{2.211069in}}%
\pgfpathlineto{\pgfqpoint{4.623984in}{2.094785in}}%
\pgfpathlineto{\pgfqpoint{4.624039in}{2.236170in}}%
\pgfpathlineto{\pgfqpoint{4.625020in}{1.940999in}}%
\pgfpathlineto{\pgfqpoint{4.625727in}{2.219935in}}%
\pgfpathlineto{\pgfqpoint{4.625564in}{1.848479in}}%
\pgfpathlineto{\pgfqpoint{4.626650in}{2.180244in}}%
\pgfpathlineto{\pgfqpoint{4.626921in}{1.967861in}}%
\pgfpathlineto{\pgfqpoint{4.627625in}{2.244121in}}%
\pgfpathlineto{\pgfqpoint{4.627734in}{2.056872in}}%
\pgfpathlineto{\pgfqpoint{4.627950in}{2.233902in}}%
\pgfpathlineto{\pgfqpoint{4.628544in}{1.887595in}}%
\pgfpathlineto{\pgfqpoint{4.628868in}{2.182098in}}%
\pgfpathlineto{\pgfqpoint{4.629246in}{1.916379in}}%
\pgfpathlineto{\pgfqpoint{4.629784in}{2.254964in}}%
\pgfpathlineto{\pgfqpoint{4.630000in}{2.144551in}}%
\pgfpathlineto{\pgfqpoint{4.630054in}{2.221864in}}%
\pgfpathlineto{\pgfqpoint{4.630484in}{1.841238in}}%
\pgfpathlineto{\pgfqpoint{4.630914in}{2.151296in}}%
\pgfpathlineto{\pgfqpoint{4.630967in}{1.806889in}}%
\pgfpathlineto{\pgfqpoint{4.631128in}{2.201442in}}%
\pgfpathlineto{\pgfqpoint{4.631986in}{2.175528in}}%
\pgfpathlineto{\pgfqpoint{4.632949in}{2.240758in}}%
\pgfpathlineto{\pgfqpoint{4.632093in}{1.822310in}}%
\pgfpathlineto{\pgfqpoint{4.633003in}{2.185320in}}%
\pgfpathlineto{\pgfqpoint{4.633056in}{1.773762in}}%
\pgfpathlineto{\pgfqpoint{4.633910in}{2.211446in}}%
\pgfpathlineto{\pgfqpoint{4.634123in}{1.960812in}}%
\pgfpathlineto{\pgfqpoint{4.634176in}{1.889323in}}%
\pgfpathlineto{\pgfqpoint{4.634336in}{2.223456in}}%
\pgfpathlineto{\pgfqpoint{4.635028in}{2.101373in}}%
\pgfpathlineto{\pgfqpoint{4.635506in}{2.249179in}}%
\pgfpathlineto{\pgfqpoint{4.635294in}{1.953392in}}%
\pgfpathlineto{\pgfqpoint{4.636090in}{2.036898in}}%
\pgfpathlineto{\pgfqpoint{4.636196in}{2.200926in}}%
\pgfpathlineto{\pgfqpoint{4.636991in}{1.897548in}}%
\pgfpathlineto{\pgfqpoint{4.637150in}{2.227394in}}%
\pgfpathlineto{\pgfqpoint{4.638101in}{2.026679in}}%
\pgfpathlineto{\pgfqpoint{4.638998in}{2.251037in}}%
\pgfpathlineto{\pgfqpoint{4.638260in}{1.872378in}}%
\pgfpathlineto{\pgfqpoint{4.639261in}{2.171362in}}%
\pgfpathlineto{\pgfqpoint{4.639682in}{1.819999in}}%
\pgfpathlineto{\pgfqpoint{4.639630in}{2.232074in}}%
\pgfpathlineto{\pgfqpoint{4.640260in}{2.115333in}}%
\pgfpathlineto{\pgfqpoint{4.640313in}{2.235470in}}%
\pgfpathlineto{\pgfqpoint{4.640733in}{1.876669in}}%
\pgfpathlineto{\pgfqpoint{4.641362in}{2.143217in}}%
\pgfpathlineto{\pgfqpoint{4.642356in}{2.195432in}}%
\pgfpathlineto{\pgfqpoint{4.642565in}{1.902117in}}%
\pgfpathlineto{\pgfqpoint{4.642774in}{2.221733in}}%
\pgfpathlineto{\pgfqpoint{4.643713in}{2.093411in}}%
\pgfpathlineto{\pgfqpoint{4.644234in}{2.242440in}}%
\pgfpathlineto{\pgfqpoint{4.644078in}{1.929965in}}%
\pgfpathlineto{\pgfqpoint{4.644546in}{2.165210in}}%
\pgfpathlineto{\pgfqpoint{4.644858in}{1.856056in}}%
\pgfpathlineto{\pgfqpoint{4.644650in}{2.247228in}}%
\pgfpathlineto{\pgfqpoint{4.645636in}{2.131105in}}%
\pgfpathlineto{\pgfqpoint{4.646051in}{2.233012in}}%
\pgfpathlineto{\pgfqpoint{4.646414in}{1.983223in}}%
\pgfpathlineto{\pgfqpoint{4.646724in}{2.193083in}}%
\pgfpathlineto{\pgfqpoint{4.647138in}{1.950386in}}%
\pgfpathlineto{\pgfqpoint{4.647500in}{2.249309in}}%
\pgfpathlineto{\pgfqpoint{4.647758in}{1.992336in}}%
\pgfpathlineto{\pgfqpoint{4.647809in}{2.250553in}}%
\pgfpathlineto{\pgfqpoint{4.648737in}{1.944758in}}%
\pgfpathlineto{\pgfqpoint{4.648891in}{2.243575in}}%
\pgfpathlineto{\pgfqpoint{4.649252in}{1.936445in}}%
\pgfpathlineto{\pgfqpoint{4.650022in}{2.207340in}}%
\pgfpathlineto{\pgfqpoint{4.650996in}{1.875570in}}%
\pgfpathlineto{\pgfqpoint{4.650945in}{2.208567in}}%
\pgfpathlineto{\pgfqpoint{4.651150in}{2.089312in}}%
\pgfpathlineto{\pgfqpoint{4.651355in}{2.261038in}}%
\pgfpathlineto{\pgfqpoint{4.651457in}{1.892243in}}%
\pgfpathlineto{\pgfqpoint{4.652275in}{2.242596in}}%
\pgfpathlineto{\pgfqpoint{4.653397in}{1.652481in}}%
\pgfpathlineto{\pgfqpoint{4.653448in}{2.058814in}}%
\pgfpathlineto{\pgfqpoint{4.654109in}{2.244953in}}%
\pgfpathlineto{\pgfqpoint{4.654363in}{2.003700in}}%
\pgfpathlineto{\pgfqpoint{4.654566in}{2.086960in}}%
\pgfpathlineto{\pgfqpoint{4.654668in}{2.245994in}}%
\pgfpathlineto{\pgfqpoint{4.655277in}{1.861489in}}%
\pgfpathlineto{\pgfqpoint{4.655429in}{2.149226in}}%
\pgfpathlineto{\pgfqpoint{4.656087in}{1.893996in}}%
\pgfpathlineto{\pgfqpoint{4.656441in}{2.249005in}}%
\pgfpathlineto{\pgfqpoint{4.656492in}{2.233941in}}%
\pgfpathlineto{\pgfqpoint{4.656593in}{1.973253in}}%
\pgfpathlineto{\pgfqpoint{4.657703in}{2.005357in}}%
\pgfpathlineto{\pgfqpoint{4.657804in}{2.215658in}}%
\pgfpathlineto{\pgfqpoint{4.658560in}{1.846209in}}%
\pgfpathlineto{\pgfqpoint{4.658811in}{2.047880in}}%
\pgfpathlineto{\pgfqpoint{4.659062in}{1.879443in}}%
\pgfpathlineto{\pgfqpoint{4.659665in}{2.240968in}}%
\pgfpathlineto{\pgfqpoint{4.659916in}{2.060615in}}%
\pgfpathlineto{\pgfqpoint{4.660367in}{2.233565in}}%
\pgfpathlineto{\pgfqpoint{4.660166in}{1.929843in}}%
\pgfpathlineto{\pgfqpoint{4.661018in}{2.191166in}}%
\pgfpathlineto{\pgfqpoint{4.661068in}{1.983104in}}%
\pgfpathlineto{\pgfqpoint{4.661867in}{2.264451in}}%
\pgfpathlineto{\pgfqpoint{4.662117in}{2.215229in}}%
\pgfpathlineto{\pgfqpoint{4.662765in}{1.805209in}}%
\pgfpathlineto{\pgfqpoint{4.662416in}{2.223685in}}%
\pgfpathlineto{\pgfqpoint{4.663213in}{2.017998in}}%
\pgfpathlineto{\pgfqpoint{4.664058in}{2.247968in}}%
\pgfpathlineto{\pgfqpoint{4.664008in}{1.631467in}}%
\pgfpathlineto{\pgfqpoint{4.664356in}{2.098580in}}%
\pgfpathlineto{\pgfqpoint{4.664653in}{2.239275in}}%
\pgfpathlineto{\pgfqpoint{4.665050in}{1.988200in}}%
\pgfpathlineto{\pgfqpoint{4.665397in}{2.127942in}}%
\pgfpathlineto{\pgfqpoint{4.665941in}{1.907347in}}%
\pgfpathlineto{\pgfqpoint{4.665842in}{2.258396in}}%
\pgfpathlineto{\pgfqpoint{4.666484in}{2.144258in}}%
\pgfpathlineto{\pgfqpoint{4.667027in}{2.252347in}}%
\pgfpathlineto{\pgfqpoint{4.666929in}{1.913829in}}%
\pgfpathlineto{\pgfqpoint{4.667569in}{2.230798in}}%
\pgfpathlineto{\pgfqpoint{4.668357in}{1.926183in}}%
\pgfpathlineto{\pgfqpoint{4.667816in}{2.264762in}}%
\pgfpathlineto{\pgfqpoint{4.668652in}{2.031462in}}%
\pgfpathlineto{\pgfqpoint{4.668946in}{2.205902in}}%
\pgfpathlineto{\pgfqpoint{4.669143in}{1.658835in}}%
\pgfpathlineto{\pgfqpoint{4.669780in}{2.109435in}}%
\pgfpathlineto{\pgfqpoint{4.669927in}{2.124560in}}%
\pgfpathlineto{\pgfqpoint{4.669878in}{2.066136in}}%
\pgfpathlineto{\pgfqpoint{4.669976in}{2.071920in}}%
\pgfpathlineto{\pgfqpoint{4.670025in}{1.954985in}}%
\pgfpathlineto{\pgfqpoint{4.670123in}{2.266738in}}%
\pgfpathlineto{\pgfqpoint{4.671003in}{2.051740in}}%
\pgfpathlineto{\pgfqpoint{4.671882in}{2.240129in}}%
\pgfpathlineto{\pgfqpoint{4.671345in}{1.738410in}}%
\pgfpathlineto{\pgfqpoint{4.672028in}{2.156702in}}%
\pgfpathlineto{\pgfqpoint{4.672320in}{1.778516in}}%
\pgfpathlineto{\pgfqpoint{4.672564in}{2.234593in}}%
\pgfpathlineto{\pgfqpoint{4.673099in}{2.207951in}}%
\pgfpathlineto{\pgfqpoint{4.673682in}{1.986946in}}%
\pgfpathlineto{\pgfqpoint{4.673827in}{2.257396in}}%
\pgfpathlineto{\pgfqpoint{4.674167in}{2.200913in}}%
\pgfpathlineto{\pgfqpoint{4.674215in}{2.265201in}}%
\pgfpathlineto{\pgfqpoint{4.674603in}{1.970148in}}%
\pgfpathlineto{\pgfqpoint{4.675184in}{2.174849in}}%
\pgfpathlineto{\pgfqpoint{4.675909in}{1.790467in}}%
\pgfpathlineto{\pgfqpoint{4.675474in}{2.214098in}}%
\pgfpathlineto{\pgfqpoint{4.676295in}{2.122045in}}%
\pgfpathlineto{\pgfqpoint{4.676536in}{2.000878in}}%
\pgfpathlineto{\pgfqpoint{4.676729in}{2.177535in}}%
\pgfpathlineto{\pgfqpoint{4.676826in}{2.020539in}}%
\pgfpathlineto{\pgfqpoint{4.676874in}{2.257050in}}%
\pgfpathlineto{\pgfqpoint{4.676922in}{1.881820in}}%
\pgfpathlineto{\pgfqpoint{4.677933in}{2.037775in}}%
\pgfpathlineto{\pgfqpoint{4.678653in}{2.242264in}}%
\pgfpathlineto{\pgfqpoint{4.678029in}{1.807081in}}%
\pgfpathlineto{\pgfqpoint{4.678941in}{2.176995in}}%
\pgfpathlineto{\pgfqpoint{4.678989in}{1.923291in}}%
\pgfpathlineto{\pgfqpoint{4.679755in}{2.250208in}}%
\pgfpathlineto{\pgfqpoint{4.680042in}{2.092268in}}%
\pgfpathlineto{\pgfqpoint{4.680329in}{2.279660in}}%
\pgfpathlineto{\pgfqpoint{4.680568in}{1.974361in}}%
\pgfpathlineto{\pgfqpoint{4.681093in}{2.064797in}}%
\pgfpathlineto{\pgfqpoint{4.681141in}{1.953010in}}%
\pgfpathlineto{\pgfqpoint{4.681284in}{2.249065in}}%
\pgfpathlineto{\pgfqpoint{4.682237in}{1.957413in}}%
\pgfpathlineto{\pgfqpoint{4.683187in}{2.220501in}}%
\pgfpathlineto{\pgfqpoint{4.682569in}{1.934286in}}%
\pgfpathlineto{\pgfqpoint{4.683329in}{2.213123in}}%
\pgfpathlineto{\pgfqpoint{4.683519in}{1.894279in}}%
\pgfpathlineto{\pgfqpoint{4.684467in}{2.054415in}}%
\pgfpathlineto{\pgfqpoint{4.684514in}{2.256046in}}%
\pgfpathlineto{\pgfqpoint{4.685318in}{1.955914in}}%
\pgfpathlineto{\pgfqpoint{4.685554in}{2.080455in}}%
\pgfpathlineto{\pgfqpoint{4.685601in}{1.885797in}}%
\pgfpathlineto{\pgfqpoint{4.686497in}{2.301324in}}%
\pgfpathlineto{\pgfqpoint{4.686591in}{2.090818in}}%
\pgfpathlineto{\pgfqpoint{4.687297in}{2.258079in}}%
\pgfpathlineto{\pgfqpoint{4.687673in}{2.012049in}}%
\pgfpathlineto{\pgfqpoint{4.688330in}{2.249017in}}%
\pgfpathlineto{\pgfqpoint{4.688237in}{1.911776in}}%
\pgfpathlineto{\pgfqpoint{4.688799in}{2.188935in}}%
\pgfpathlineto{\pgfqpoint{4.689361in}{1.743365in}}%
\pgfpathlineto{\pgfqpoint{4.689642in}{2.199844in}}%
\pgfpathlineto{\pgfqpoint{4.689922in}{2.088403in}}%
\pgfpathlineto{\pgfqpoint{4.690109in}{2.269355in}}%
\pgfpathlineto{\pgfqpoint{4.690062in}{1.908277in}}%
\pgfpathlineto{\pgfqpoint{4.690995in}{2.067871in}}%
\pgfpathlineto{\pgfqpoint{4.691321in}{2.225389in}}%
\pgfpathlineto{\pgfqpoint{4.691554in}{2.006146in}}%
\pgfpathlineto{\pgfqpoint{4.691647in}{2.027460in}}%
\pgfpathlineto{\pgfqpoint{4.691694in}{1.849756in}}%
\pgfpathlineto{\pgfqpoint{4.691926in}{2.234276in}}%
\pgfpathlineto{\pgfqpoint{4.692763in}{1.917299in}}%
\pgfpathlineto{\pgfqpoint{4.693273in}{2.239217in}}%
\pgfpathlineto{\pgfqpoint{4.693829in}{1.779939in}}%
\pgfpathlineto{\pgfqpoint{4.693875in}{2.097703in}}%
\pgfpathlineto{\pgfqpoint{4.693922in}{2.097132in}}%
\pgfpathlineto{\pgfqpoint{4.694661in}{2.222887in}}%
\pgfpathlineto{\pgfqpoint{4.694569in}{2.032473in}}%
\pgfpathlineto{\pgfqpoint{4.695031in}{2.147081in}}%
\pgfpathlineto{\pgfqpoint{4.695262in}{2.014609in}}%
\pgfpathlineto{\pgfqpoint{4.695861in}{2.242229in}}%
\pgfpathlineto{\pgfqpoint{4.696091in}{2.161162in}}%
\pgfpathlineto{\pgfqpoint{4.696644in}{2.248565in}}%
\pgfpathlineto{\pgfqpoint{4.696276in}{1.899853in}}%
\pgfpathlineto{\pgfqpoint{4.696782in}{2.116062in}}%
\pgfpathlineto{\pgfqpoint{4.696828in}{1.984213in}}%
\pgfpathlineto{\pgfqpoint{4.697241in}{2.229675in}}%
\pgfpathlineto{\pgfqpoint{4.697884in}{2.087589in}}%
\pgfpathlineto{\pgfqpoint{4.698296in}{2.231337in}}%
\pgfpathlineto{\pgfqpoint{4.698113in}{1.929792in}}%
\pgfpathlineto{\pgfqpoint{4.699074in}{2.215629in}}%
\pgfpathlineto{\pgfqpoint{4.699577in}{1.881113in}}%
\pgfpathlineto{\pgfqpoint{4.699166in}{2.254687in}}%
\pgfpathlineto{\pgfqpoint{4.700170in}{2.185016in}}%
\pgfpathlineto{\pgfqpoint{4.700216in}{2.201293in}}%
\pgfpathlineto{\pgfqpoint{4.700672in}{1.974263in}}%
\pgfpathlineto{\pgfqpoint{4.700763in}{2.160457in}}%
\pgfpathlineto{\pgfqpoint{4.701809in}{1.829403in}}%
\pgfpathlineto{\pgfqpoint{4.701446in}{2.242125in}}%
\pgfpathlineto{\pgfqpoint{4.701855in}{2.136109in}}%
\pgfpathlineto{\pgfqpoint{4.702082in}{1.947717in}}%
\pgfpathlineto{\pgfqpoint{4.702173in}{2.227732in}}%
\pgfpathlineto{\pgfqpoint{4.702944in}{2.148222in}}%
\pgfpathlineto{\pgfqpoint{4.703306in}{2.244373in}}%
\pgfpathlineto{\pgfqpoint{4.703215in}{1.825689in}}%
\pgfpathlineto{\pgfqpoint{4.703894in}{2.148499in}}%
\pgfpathlineto{\pgfqpoint{4.704391in}{1.919607in}}%
\pgfpathlineto{\pgfqpoint{4.704030in}{2.247092in}}%
\pgfpathlineto{\pgfqpoint{4.705023in}{2.020520in}}%
\pgfpathlineto{\pgfqpoint{4.706014in}{2.210812in}}%
\pgfpathlineto{\pgfqpoint{4.705789in}{1.705701in}}%
\pgfpathlineto{\pgfqpoint{4.706149in}{2.159151in}}%
\pgfpathlineto{\pgfqpoint{4.706778in}{1.849791in}}%
\pgfpathlineto{\pgfqpoint{4.706373in}{2.242638in}}%
\pgfpathlineto{\pgfqpoint{4.707182in}{2.073329in}}%
\pgfpathlineto{\pgfqpoint{4.707227in}{2.242113in}}%
\pgfpathlineto{\pgfqpoint{4.707854in}{1.759983in}}%
\pgfpathlineto{\pgfqpoint{4.708302in}{2.156674in}}%
\pgfpathlineto{\pgfqpoint{4.708838in}{2.011834in}}%
\pgfpathlineto{\pgfqpoint{4.708883in}{2.280120in}}%
\pgfpathlineto{\pgfqpoint{4.709419in}{2.122317in}}%
\pgfpathlineto{\pgfqpoint{4.710043in}{2.228743in}}%
\pgfpathlineto{\pgfqpoint{4.710222in}{1.935996in}}%
\pgfpathlineto{\pgfqpoint{4.710444in}{2.043218in}}%
\pgfpathlineto{\pgfqpoint{4.711112in}{2.214280in}}%
\pgfpathlineto{\pgfqpoint{4.711467in}{1.928886in}}%
\pgfpathlineto{\pgfqpoint{4.711511in}{2.224916in}}%
\pgfpathlineto{\pgfqpoint{4.712576in}{2.103409in}}%
\pgfpathlineto{\pgfqpoint{4.712753in}{2.187442in}}%
\pgfpathlineto{\pgfqpoint{4.712974in}{2.005374in}}%
\pgfpathlineto{\pgfqpoint{4.713107in}{2.143809in}}%
\pgfpathlineto{\pgfqpoint{4.714035in}{1.832413in}}%
\pgfpathlineto{\pgfqpoint{4.713240in}{2.220655in}}%
\pgfpathlineto{\pgfqpoint{4.714212in}{2.060915in}}%
\pgfpathlineto{\pgfqpoint{4.714785in}{2.235154in}}%
\pgfpathlineto{\pgfqpoint{4.715049in}{1.923926in}}%
\pgfpathlineto{\pgfqpoint{4.715313in}{2.060600in}}%
\pgfpathlineto{\pgfqpoint{4.715357in}{1.869704in}}%
\pgfpathlineto{\pgfqpoint{4.716324in}{2.268301in}}%
\pgfpathlineto{\pgfqpoint{4.716412in}{2.102510in}}%
\pgfpathlineto{\pgfqpoint{4.717245in}{1.754116in}}%
\pgfpathlineto{\pgfqpoint{4.716676in}{2.228537in}}%
\pgfpathlineto{\pgfqpoint{4.717465in}{2.130646in}}%
\pgfpathlineto{\pgfqpoint{4.717990in}{2.249924in}}%
\pgfpathlineto{\pgfqpoint{4.718033in}{1.976610in}}%
\pgfpathlineto{\pgfqpoint{4.718558in}{2.207427in}}%
\pgfpathlineto{\pgfqpoint{4.719648in}{1.900269in}}%
\pgfpathlineto{\pgfqpoint{4.718994in}{2.277915in}}%
\pgfpathlineto{\pgfqpoint{4.719692in}{1.984189in}}%
\pgfpathlineto{\pgfqpoint{4.719735in}{2.199846in}}%
\pgfpathlineto{\pgfqpoint{4.719953in}{1.937448in}}%
\pgfpathlineto{\pgfqpoint{4.720780in}{2.094513in}}%
\pgfpathlineto{\pgfqpoint{4.721431in}{1.877705in}}%
\pgfpathlineto{\pgfqpoint{4.721257in}{2.238398in}}%
\pgfpathlineto{\pgfqpoint{4.721691in}{2.095785in}}%
\pgfpathlineto{\pgfqpoint{4.721908in}{2.246378in}}%
\pgfpathlineto{\pgfqpoint{4.722384in}{1.909803in}}%
\pgfpathlineto{\pgfqpoint{4.722817in}{2.203552in}}%
\pgfpathlineto{\pgfqpoint{4.723767in}{1.773895in}}%
\pgfpathlineto{\pgfqpoint{4.723206in}{2.249018in}}%
\pgfpathlineto{\pgfqpoint{4.723896in}{2.029371in}}%
\pgfpathlineto{\pgfqpoint{4.724155in}{2.240435in}}%
\pgfpathlineto{\pgfqpoint{4.723983in}{1.962449in}}%
\pgfpathlineto{\pgfqpoint{4.725016in}{2.141481in}}%
\pgfpathlineto{\pgfqpoint{4.725403in}{2.242176in}}%
\pgfpathlineto{\pgfqpoint{4.725102in}{2.017346in}}%
\pgfpathlineto{\pgfqpoint{4.725919in}{2.047240in}}%
\pgfpathlineto{\pgfqpoint{4.726948in}{1.957400in}}%
\pgfpathlineto{\pgfqpoint{4.726562in}{2.252831in}}%
\pgfpathlineto{\pgfqpoint{4.726990in}{2.050588in}}%
\pgfpathlineto{\pgfqpoint{4.728102in}{2.219731in}}%
\pgfpathlineto{\pgfqpoint{4.727376in}{1.978016in}}%
\pgfpathlineto{\pgfqpoint{4.728145in}{2.147960in}}%
\pgfpathlineto{\pgfqpoint{4.728401in}{1.905330in}}%
\pgfpathlineto{\pgfqpoint{4.728529in}{2.228865in}}%
\pgfpathlineto{\pgfqpoint{4.729254in}{2.131422in}}%
\pgfpathlineto{\pgfqpoint{4.729339in}{2.138180in}}%
\pgfpathlineto{\pgfqpoint{4.729381in}{2.109063in}}%
\pgfpathlineto{\pgfqpoint{4.729594in}{1.678402in}}%
\pgfpathlineto{\pgfqpoint{4.730275in}{2.250164in}}%
\pgfpathlineto{\pgfqpoint{4.730529in}{2.004079in}}%
\pgfpathlineto{\pgfqpoint{4.730784in}{2.268474in}}%
\pgfpathlineto{\pgfqpoint{4.730996in}{1.916083in}}%
\pgfpathlineto{\pgfqpoint{4.731674in}{2.222726in}}%
\pgfpathlineto{\pgfqpoint{4.732182in}{1.775965in}}%
\pgfpathlineto{\pgfqpoint{4.732647in}{2.229678in}}%
\pgfpathlineto{\pgfqpoint{4.732816in}{2.115556in}}%
\pgfpathlineto{\pgfqpoint{4.733534in}{2.243144in}}%
\pgfpathlineto{\pgfqpoint{4.733491in}{1.970060in}}%
\pgfpathlineto{\pgfqpoint{4.733913in}{2.128372in}}%
\pgfpathlineto{\pgfqpoint{4.734755in}{1.928287in}}%
\pgfpathlineto{\pgfqpoint{4.734502in}{2.238925in}}%
\pgfpathlineto{\pgfqpoint{4.735007in}{2.001238in}}%
\pgfpathlineto{\pgfqpoint{4.735846in}{2.247487in}}%
\pgfpathlineto{\pgfqpoint{4.735175in}{1.932188in}}%
\pgfpathlineto{\pgfqpoint{4.736140in}{2.225034in}}%
\pgfpathlineto{\pgfqpoint{4.736265in}{1.880776in}}%
\pgfpathlineto{\pgfqpoint{4.736558in}{2.250811in}}%
\pgfpathlineto{\pgfqpoint{4.737228in}{2.180186in}}%
\pgfpathlineto{\pgfqpoint{4.737437in}{2.242226in}}%
\pgfpathlineto{\pgfqpoint{4.737520in}{2.024460in}}%
\pgfpathlineto{\pgfqpoint{4.737771in}{2.035236in}}%
\pgfpathlineto{\pgfqpoint{4.737813in}{1.861982in}}%
\pgfpathlineto{\pgfqpoint{4.738521in}{2.242250in}}%
\pgfpathlineto{\pgfqpoint{4.738813in}{2.198322in}}%
\pgfpathlineto{\pgfqpoint{4.739021in}{1.989493in}}%
\pgfpathlineto{\pgfqpoint{4.739437in}{2.261107in}}%
\pgfpathlineto{\pgfqpoint{4.739604in}{2.238653in}}%
\pgfpathlineto{\pgfqpoint{4.739645in}{2.300815in}}%
\pgfpathlineto{\pgfqpoint{4.740475in}{1.909657in}}%
\pgfpathlineto{\pgfqpoint{4.740641in}{2.278008in}}%
\pgfpathlineto{\pgfqpoint{4.740973in}{1.787424in}}%
\pgfpathlineto{\pgfqpoint{4.741842in}{1.920297in}}%
\pgfpathlineto{\pgfqpoint{4.742668in}{2.247335in}}%
\pgfpathlineto{\pgfqpoint{4.742957in}{2.232117in}}%
\pgfpathlineto{\pgfqpoint{4.743740in}{1.854833in}}%
\pgfpathlineto{\pgfqpoint{4.744069in}{2.167762in}}%
\pgfpathlineto{\pgfqpoint{4.744562in}{1.847164in}}%
\pgfpathlineto{\pgfqpoint{4.744274in}{2.270337in}}%
\pgfpathlineto{\pgfqpoint{4.745219in}{2.112159in}}%
\pgfpathlineto{\pgfqpoint{4.746284in}{1.746861in}}%
\pgfpathlineto{\pgfqpoint{4.745629in}{2.226189in}}%
\pgfpathlineto{\pgfqpoint{4.746325in}{2.118114in}}%
\pgfpathlineto{\pgfqpoint{4.746652in}{2.236193in}}%
\pgfpathlineto{\pgfqpoint{4.747306in}{1.930745in}}%
\pgfpathlineto{\pgfqpoint{4.747428in}{2.203487in}}%
\pgfpathlineto{\pgfqpoint{4.748080in}{1.975202in}}%
\pgfpathlineto{\pgfqpoint{4.748447in}{2.288405in}}%
\pgfpathlineto{\pgfqpoint{4.748528in}{2.102801in}}%
\pgfpathlineto{\pgfqpoint{4.748569in}{2.287616in}}%
\pgfpathlineto{\pgfqpoint{4.749463in}{1.952943in}}%
\pgfpathlineto{\pgfqpoint{4.749585in}{2.008387in}}%
\pgfpathlineto{\pgfqpoint{4.749626in}{1.843457in}}%
\pgfpathlineto{\pgfqpoint{4.750437in}{2.272151in}}%
\pgfpathlineto{\pgfqpoint{4.750639in}{2.011008in}}%
\pgfpathlineto{\pgfqpoint{4.751368in}{1.923802in}}%
\pgfpathlineto{\pgfqpoint{4.751812in}{2.262236in}}%
\pgfpathlineto{\pgfqpoint{4.752861in}{1.909951in}}%
\pgfpathlineto{\pgfqpoint{4.752941in}{2.102632in}}%
\pgfpathlineto{\pgfqpoint{4.752982in}{2.267904in}}%
\pgfpathlineto{\pgfqpoint{4.753626in}{1.892143in}}%
\pgfpathlineto{\pgfqpoint{4.753987in}{2.248905in}}%
\pgfpathlineto{\pgfqpoint{4.754910in}{1.927052in}}%
\pgfpathlineto{\pgfqpoint{4.754950in}{2.288459in}}%
\pgfpathlineto{\pgfqpoint{4.755111in}{2.129709in}}%
\pgfpathlineto{\pgfqpoint{4.755191in}{1.818664in}}%
\pgfpathlineto{\pgfqpoint{4.756031in}{2.220061in}}%
\pgfpathlineto{\pgfqpoint{4.756111in}{1.993379in}}%
\pgfpathlineto{\pgfqpoint{4.756311in}{2.248645in}}%
\pgfpathlineto{\pgfqpoint{4.756950in}{1.889168in}}%
\pgfpathlineto{\pgfqpoint{4.757229in}{2.135786in}}%
\pgfpathlineto{\pgfqpoint{4.757787in}{2.253952in}}%
\pgfpathlineto{\pgfqpoint{4.757349in}{1.908962in}}%
\pgfpathlineto{\pgfqpoint{4.757986in}{2.058923in}}%
\pgfpathlineto{\pgfqpoint{4.758026in}{1.888979in}}%
\pgfpathlineto{\pgfqpoint{4.758543in}{2.277846in}}%
\pgfpathlineto{\pgfqpoint{4.759059in}{2.180173in}}%
\pgfpathlineto{\pgfqpoint{4.759734in}{1.872468in}}%
\pgfpathlineto{\pgfqpoint{4.759298in}{2.217986in}}%
\pgfpathlineto{\pgfqpoint{4.760051in}{2.061831in}}%
\pgfpathlineto{\pgfqpoint{4.760724in}{2.260467in}}%
\pgfpathlineto{\pgfqpoint{4.760763in}{1.927943in}}%
\pgfpathlineto{\pgfqpoint{4.761119in}{1.938753in}}%
\pgfpathlineto{\pgfqpoint{4.762263in}{2.244585in}}%
\pgfpathlineto{\pgfqpoint{4.761790in}{1.906255in}}%
\pgfpathlineto{\pgfqpoint{4.762381in}{2.151653in}}%
\pgfpathlineto{\pgfqpoint{4.763050in}{1.911957in}}%
\pgfpathlineto{\pgfqpoint{4.763326in}{2.260935in}}%
\pgfpathlineto{\pgfqpoint{4.763444in}{2.188614in}}%
\pgfpathlineto{\pgfqpoint{4.764386in}{2.305084in}}%
\pgfpathlineto{\pgfqpoint{4.763679in}{1.782244in}}%
\pgfpathlineto{\pgfqpoint{4.764425in}{2.084736in}}%
\pgfpathlineto{\pgfqpoint{4.764464in}{1.849568in}}%
\pgfpathlineto{\pgfqpoint{4.764660in}{2.277144in}}%
\pgfpathlineto{\pgfqpoint{4.765482in}{2.080089in}}%
\pgfpathlineto{\pgfqpoint{4.766458in}{2.242224in}}%
\pgfpathlineto{\pgfqpoint{4.766068in}{1.936688in}}%
\pgfpathlineto{\pgfqpoint{4.766575in}{2.175497in}}%
\pgfpathlineto{\pgfqpoint{4.767433in}{2.240727in}}%
\pgfpathlineto{\pgfqpoint{4.767744in}{1.757307in}}%
\pgfpathlineto{\pgfqpoint{4.768482in}{2.254547in}}%
\pgfpathlineto{\pgfqpoint{4.768870in}{2.212030in}}%
\pgfpathlineto{\pgfqpoint{4.768909in}{1.910779in}}%
\pgfpathlineto{\pgfqpoint{4.769917in}{2.235493in}}%
\pgfpathlineto{\pgfqpoint{4.769955in}{2.180175in}}%
\pgfpathlineto{\pgfqpoint{4.770187in}{2.276547in}}%
\pgfpathlineto{\pgfqpoint{4.770265in}{2.093344in}}%
\pgfpathlineto{\pgfqpoint{4.771230in}{1.908016in}}%
\pgfpathlineto{\pgfqpoint{4.770497in}{2.213079in}}%
\pgfpathlineto{\pgfqpoint{4.771346in}{2.081634in}}%
\pgfpathlineto{\pgfqpoint{4.772001in}{2.227798in}}%
\pgfpathlineto{\pgfqpoint{4.771847in}{2.023409in}}%
\pgfpathlineto{\pgfqpoint{4.772078in}{2.223086in}}%
\pgfpathlineto{\pgfqpoint{4.772117in}{1.958809in}}%
\pgfpathlineto{\pgfqpoint{4.773001in}{2.281887in}}%
\pgfpathlineto{\pgfqpoint{4.773193in}{2.144665in}}%
\pgfpathlineto{\pgfqpoint{4.773769in}{2.213455in}}%
\pgfpathlineto{\pgfqpoint{4.773500in}{2.025919in}}%
\pgfpathlineto{\pgfqpoint{4.773961in}{2.198185in}}%
\pgfpathlineto{\pgfqpoint{4.774803in}{1.912748in}}%
\pgfpathlineto{\pgfqpoint{4.774535in}{2.271526in}}%
\pgfpathlineto{\pgfqpoint{4.775071in}{2.097470in}}%
\pgfpathlineto{\pgfqpoint{4.776102in}{2.264938in}}%
\pgfpathlineto{\pgfqpoint{4.775911in}{1.772138in}}%
\pgfpathlineto{\pgfqpoint{4.776178in}{2.223170in}}%
\pgfpathlineto{\pgfqpoint{4.776292in}{1.800541in}}%
\pgfpathlineto{\pgfqpoint{4.776445in}{2.240518in}}%
\pgfpathlineto{\pgfqpoint{4.777320in}{1.909354in}}%
\pgfpathlineto{\pgfqpoint{4.778004in}{2.284833in}}%
\pgfpathlineto{\pgfqpoint{4.778422in}{2.065886in}}%
\pgfpathlineto{\pgfqpoint{4.779331in}{1.860543in}}%
\pgfpathlineto{\pgfqpoint{4.778801in}{2.254471in}}%
\pgfpathlineto{\pgfqpoint{4.779407in}{2.004705in}}%
\pgfpathlineto{\pgfqpoint{4.780125in}{2.300845in}}%
\pgfpathlineto{\pgfqpoint{4.780012in}{1.791330in}}%
\pgfpathlineto{\pgfqpoint{4.780503in}{2.028709in}}%
\pgfpathlineto{\pgfqpoint{4.780767in}{1.947460in}}%
\pgfpathlineto{\pgfqpoint{4.780805in}{2.283805in}}%
\pgfpathlineto{\pgfqpoint{4.781483in}{2.157452in}}%
\pgfpathlineto{\pgfqpoint{4.782160in}{2.281626in}}%
\pgfpathlineto{\pgfqpoint{4.781784in}{1.759722in}}%
\pgfpathlineto{\pgfqpoint{4.782536in}{2.219707in}}%
\pgfpathlineto{\pgfqpoint{4.782611in}{1.902544in}}%
\pgfpathlineto{\pgfqpoint{4.783437in}{2.226332in}}%
\pgfpathlineto{\pgfqpoint{4.783662in}{2.019768in}}%
\pgfpathlineto{\pgfqpoint{4.784672in}{2.256847in}}%
\pgfpathlineto{\pgfqpoint{4.783736in}{1.824824in}}%
\pgfpathlineto{\pgfqpoint{4.784784in}{2.104493in}}%
\pgfpathlineto{\pgfqpoint{4.785381in}{2.275571in}}%
\pgfpathlineto{\pgfqpoint{4.785680in}{1.947266in}}%
\pgfpathlineto{\pgfqpoint{4.785903in}{2.143651in}}%
\pgfpathlineto{\pgfqpoint{4.786760in}{1.844822in}}%
\pgfpathlineto{\pgfqpoint{4.786574in}{2.246339in}}%
\pgfpathlineto{\pgfqpoint{4.786983in}{2.075120in}}%
\pgfpathlineto{\pgfqpoint{4.787985in}{2.281777in}}%
\pgfpathlineto{\pgfqpoint{4.787243in}{1.827946in}}%
\pgfpathlineto{\pgfqpoint{4.788097in}{2.241466in}}%
\pgfpathlineto{\pgfqpoint{4.789207in}{1.865028in}}%
\pgfpathlineto{\pgfqpoint{4.788319in}{2.278025in}}%
\pgfpathlineto{\pgfqpoint{4.789244in}{2.124274in}}%
\pgfpathlineto{\pgfqpoint{4.789836in}{2.261747in}}%
\pgfpathlineto{\pgfqpoint{4.790205in}{1.896347in}}%
\pgfpathlineto{\pgfqpoint{4.790278in}{2.053163in}}%
\pgfpathlineto{\pgfqpoint{4.790315in}{1.686264in}}%
\pgfpathlineto{\pgfqpoint{4.790463in}{2.302724in}}%
\pgfpathlineto{\pgfqpoint{4.791347in}{2.168637in}}%
\pgfpathlineto{\pgfqpoint{4.792119in}{2.267267in}}%
\pgfpathlineto{\pgfqpoint{4.791751in}{1.833257in}}%
\pgfpathlineto{\pgfqpoint{4.792376in}{2.251026in}}%
\pgfpathlineto{\pgfqpoint{4.792412in}{1.861925in}}%
\pgfpathlineto{\pgfqpoint{4.793475in}{2.217257in}}%
\pgfpathlineto{\pgfqpoint{4.793585in}{1.935721in}}%
\pgfpathlineto{\pgfqpoint{4.794061in}{2.273270in}}%
\pgfpathlineto{\pgfqpoint{4.794572in}{1.975769in}}%
\pgfpathlineto{\pgfqpoint{4.795229in}{2.202265in}}%
\pgfpathlineto{\pgfqpoint{4.795593in}{1.907801in}}%
\pgfpathlineto{\pgfqpoint{4.795666in}{2.128692in}}%
\pgfpathlineto{\pgfqpoint{4.795703in}{1.751902in}}%
\pgfpathlineto{\pgfqpoint{4.796430in}{2.238772in}}%
\pgfpathlineto{\pgfqpoint{4.796757in}{2.051369in}}%
\pgfpathlineto{\pgfqpoint{4.796939in}{2.241350in}}%
\pgfpathlineto{\pgfqpoint{4.796830in}{1.932238in}}%
\pgfpathlineto{\pgfqpoint{4.797882in}{2.111802in}}%
\pgfpathlineto{\pgfqpoint{4.797954in}{2.262637in}}%
\pgfpathlineto{\pgfqpoint{4.798353in}{2.013531in}}%
\pgfpathlineto{\pgfqpoint{4.798389in}{1.852323in}}%
\pgfpathlineto{\pgfqpoint{4.798931in}{2.256582in}}%
\pgfpathlineto{\pgfqpoint{4.799437in}{1.924318in}}%
\pgfpathlineto{\pgfqpoint{4.800158in}{2.293183in}}%
\pgfpathlineto{\pgfqpoint{4.800050in}{1.821744in}}%
\pgfpathlineto{\pgfqpoint{4.800555in}{2.206208in}}%
\pgfpathlineto{\pgfqpoint{4.800878in}{1.945713in}}%
\pgfpathlineto{\pgfqpoint{4.800770in}{2.240323in}}%
\pgfpathlineto{\pgfqpoint{4.801633in}{2.215368in}}%
\pgfpathlineto{\pgfqpoint{4.801669in}{2.306056in}}%
\pgfpathlineto{\pgfqpoint{4.801920in}{1.968723in}}%
\pgfpathlineto{\pgfqpoint{4.802673in}{2.123337in}}%
\pgfpathlineto{\pgfqpoint{4.803461in}{1.926895in}}%
\pgfpathlineto{\pgfqpoint{4.802960in}{2.231711in}}%
\pgfpathlineto{\pgfqpoint{4.803747in}{2.176782in}}%
\pgfpathlineto{\pgfqpoint{4.804532in}{1.938088in}}%
\pgfpathlineto{\pgfqpoint{4.804140in}{2.251249in}}%
\pgfpathlineto{\pgfqpoint{4.804853in}{2.177662in}}%
\pgfpathlineto{\pgfqpoint{4.804889in}{2.272121in}}%
\pgfpathlineto{\pgfqpoint{4.805031in}{1.893195in}}%
\pgfpathlineto{\pgfqpoint{4.805956in}{2.192230in}}%
\pgfpathlineto{\pgfqpoint{4.806134in}{2.033818in}}%
\pgfpathlineto{\pgfqpoint{4.806844in}{2.288531in}}%
\pgfpathlineto{\pgfqpoint{4.807092in}{2.038591in}}%
\pgfpathlineto{\pgfqpoint{4.807836in}{2.264562in}}%
\pgfpathlineto{\pgfqpoint{4.807907in}{1.900496in}}%
\pgfpathlineto{\pgfqpoint{4.808190in}{2.198644in}}%
\pgfpathlineto{\pgfqpoint{4.808225in}{1.766446in}}%
\pgfpathlineto{\pgfqpoint{4.809073in}{2.293013in}}%
\pgfpathlineto{\pgfqpoint{4.809285in}{2.077422in}}%
\pgfpathlineto{\pgfqpoint{4.809813in}{2.259978in}}%
\pgfpathlineto{\pgfqpoint{4.809673in}{1.825122in}}%
\pgfpathlineto{\pgfqpoint{4.810412in}{2.174162in}}%
\pgfpathlineto{\pgfqpoint{4.811045in}{2.253804in}}%
\pgfpathlineto{\pgfqpoint{4.811501in}{1.920001in}}%
\pgfpathlineto{\pgfqpoint{4.812202in}{2.270408in}}%
\pgfpathlineto{\pgfqpoint{4.812622in}{2.202360in}}%
\pgfpathlineto{\pgfqpoint{4.812727in}{1.802970in}}%
\pgfpathlineto{\pgfqpoint{4.813461in}{2.248095in}}%
\pgfpathlineto{\pgfqpoint{4.813741in}{2.018300in}}%
\pgfpathlineto{\pgfqpoint{4.813915in}{2.291492in}}%
\pgfpathlineto{\pgfqpoint{4.813880in}{1.890548in}}%
\pgfpathlineto{\pgfqpoint{4.814821in}{2.126534in}}%
\pgfpathlineto{\pgfqpoint{4.815482in}{1.838854in}}%
\pgfpathlineto{\pgfqpoint{4.815273in}{2.260239in}}%
\pgfpathlineto{\pgfqpoint{4.815864in}{2.095142in}}%
\pgfpathlineto{\pgfqpoint{4.815899in}{2.275827in}}%
\pgfpathlineto{\pgfqpoint{4.816523in}{1.966169in}}%
\pgfpathlineto{\pgfqpoint{4.816939in}{2.204439in}}%
\pgfpathlineto{\pgfqpoint{4.817389in}{1.879728in}}%
\pgfpathlineto{\pgfqpoint{4.817286in}{2.252422in}}%
\pgfpathlineto{\pgfqpoint{4.818046in}{2.181833in}}%
\pgfpathlineto{\pgfqpoint{4.818150in}{2.255100in}}%
\pgfpathlineto{\pgfqpoint{4.818184in}{2.118246in}}%
\pgfpathlineto{\pgfqpoint{4.818909in}{1.546128in}}%
\pgfpathlineto{\pgfqpoint{4.818357in}{2.269658in}}%
\pgfpathlineto{\pgfqpoint{4.819219in}{2.202242in}}%
\pgfpathlineto{\pgfqpoint{4.819701in}{2.277469in}}%
\pgfpathlineto{\pgfqpoint{4.819426in}{1.864411in}}%
\pgfpathlineto{\pgfqpoint{4.820114in}{2.095488in}}%
\pgfpathlineto{\pgfqpoint{4.820836in}{1.882410in}}%
\pgfpathlineto{\pgfqpoint{4.820389in}{2.277636in}}%
\pgfpathlineto{\pgfqpoint{4.821213in}{2.092527in}}%
\pgfpathlineto{\pgfqpoint{4.821898in}{2.018804in}}%
\pgfpathlineto{\pgfqpoint{4.821796in}{2.251882in}}%
\pgfpathlineto{\pgfqpoint{4.822206in}{2.089547in}}%
\pgfpathlineto{\pgfqpoint{4.822309in}{2.276607in}}%
\pgfpathlineto{\pgfqpoint{4.822787in}{1.908920in}}%
\pgfpathlineto{\pgfqpoint{4.823300in}{2.249424in}}%
\pgfpathlineto{\pgfqpoint{4.823709in}{1.952297in}}%
\pgfpathlineto{\pgfqpoint{4.823368in}{2.275985in}}%
\pgfpathlineto{\pgfqpoint{4.824424in}{2.111336in}}%
\pgfpathlineto{\pgfqpoint{4.825274in}{2.266659in}}%
\pgfpathlineto{\pgfqpoint{4.824798in}{1.788493in}}%
\pgfpathlineto{\pgfqpoint{4.825512in}{2.053006in}}%
\pgfpathlineto{\pgfqpoint{4.825885in}{2.291016in}}%
\pgfpathlineto{\pgfqpoint{4.826190in}{1.742519in}}%
\pgfpathlineto{\pgfqpoint{4.826563in}{2.063693in}}%
\pgfpathlineto{\pgfqpoint{4.827240in}{1.851725in}}%
\pgfpathlineto{\pgfqpoint{4.827442in}{2.266974in}}%
\pgfpathlineto{\pgfqpoint{4.827645in}{1.952804in}}%
\pgfpathlineto{\pgfqpoint{4.827713in}{2.263708in}}%
\pgfpathlineto{\pgfqpoint{4.828489in}{1.824481in}}%
\pgfpathlineto{\pgfqpoint{4.828792in}{2.203950in}}%
\pgfpathlineto{\pgfqpoint{4.829129in}{1.632530in}}%
\pgfpathlineto{\pgfqpoint{4.829061in}{2.267888in}}%
\pgfpathlineto{\pgfqpoint{4.829902in}{2.056752in}}%
\pgfpathlineto{\pgfqpoint{4.830204in}{2.260781in}}%
\pgfpathlineto{\pgfqpoint{4.829969in}{1.898756in}}%
\pgfpathlineto{\pgfqpoint{4.831009in}{2.188077in}}%
\pgfpathlineto{\pgfqpoint{4.831646in}{1.946160in}}%
\pgfpathlineto{\pgfqpoint{4.832013in}{2.278219in}}%
\pgfpathlineto{\pgfqpoint{4.832114in}{2.176203in}}%
\pgfpathlineto{\pgfqpoint{4.832682in}{2.293736in}}%
\pgfpathlineto{\pgfqpoint{4.832214in}{1.964183in}}%
\pgfpathlineto{\pgfqpoint{4.833149in}{2.125752in}}%
\pgfpathlineto{\pgfqpoint{4.833748in}{1.775575in}}%
\pgfpathlineto{\pgfqpoint{4.833382in}{2.251646in}}%
\pgfpathlineto{\pgfqpoint{4.834114in}{2.193556in}}%
\pgfpathlineto{\pgfqpoint{4.834214in}{2.239690in}}%
\pgfpathlineto{\pgfqpoint{4.834580in}{1.759348in}}%
\pgfpathlineto{\pgfqpoint{4.835111in}{2.109298in}}%
\pgfpathlineto{\pgfqpoint{4.835476in}{1.926617in}}%
\pgfpathlineto{\pgfqpoint{4.835310in}{2.294462in}}%
\pgfpathlineto{\pgfqpoint{4.836205in}{2.178194in}}%
\pgfpathlineto{\pgfqpoint{4.836966in}{1.962278in}}%
\pgfpathlineto{\pgfqpoint{4.836866in}{2.242665in}}%
\pgfpathlineto{\pgfqpoint{4.837263in}{2.118646in}}%
\pgfpathlineto{\pgfqpoint{4.837362in}{2.274913in}}%
\pgfpathlineto{\pgfqpoint{4.838252in}{1.916189in}}%
\pgfpathlineto{\pgfqpoint{4.838285in}{2.194397in}}%
\pgfpathlineto{\pgfqpoint{4.838318in}{1.876518in}}%
\pgfpathlineto{\pgfqpoint{4.838647in}{2.318191in}}%
\pgfpathlineto{\pgfqpoint{4.839371in}{2.187107in}}%
\pgfpathlineto{\pgfqpoint{4.840323in}{2.285732in}}%
\pgfpathlineto{\pgfqpoint{4.839896in}{1.608678in}}%
\pgfpathlineto{\pgfqpoint{4.840421in}{2.222202in}}%
\pgfpathlineto{\pgfqpoint{4.840945in}{1.846005in}}%
\pgfpathlineto{\pgfqpoint{4.841141in}{2.253378in}}%
\pgfpathlineto{\pgfqpoint{4.841534in}{2.193626in}}%
\pgfpathlineto{\pgfqpoint{4.841566in}{2.278655in}}%
\pgfpathlineto{\pgfqpoint{4.842154in}{2.003711in}}%
\pgfpathlineto{\pgfqpoint{4.842611in}{2.179546in}}%
\pgfpathlineto{\pgfqpoint{4.842709in}{2.101311in}}%
\pgfpathlineto{\pgfqpoint{4.842774in}{2.157863in}}%
\pgfpathlineto{\pgfqpoint{4.842904in}{1.746889in}}%
\pgfpathlineto{\pgfqpoint{4.843393in}{2.263629in}}%
\pgfpathlineto{\pgfqpoint{4.843848in}{2.260344in}}%
\pgfpathlineto{\pgfqpoint{4.843913in}{2.123019in}}%
\pgfpathlineto{\pgfqpoint{4.843978in}{2.205678in}}%
\pgfpathlineto{\pgfqpoint{4.844628in}{1.915470in}}%
\pgfpathlineto{\pgfqpoint{4.844758in}{2.284957in}}%
\pgfpathlineto{\pgfqpoint{4.845082in}{2.200930in}}%
\pgfpathlineto{\pgfqpoint{4.845147in}{2.207806in}}%
\pgfpathlineto{\pgfqpoint{4.846150in}{1.890511in}}%
\pgfpathlineto{\pgfqpoint{4.846215in}{2.312044in}}%
\pgfpathlineto{\pgfqpoint{4.846247in}{2.222402in}}%
\pgfpathlineto{\pgfqpoint{4.846990in}{2.272554in}}%
\pgfpathlineto{\pgfqpoint{4.846635in}{2.030319in}}%
\pgfpathlineto{\pgfqpoint{4.847119in}{2.080698in}}%
\pgfpathlineto{\pgfqpoint{4.848150in}{1.832688in}}%
\pgfpathlineto{\pgfqpoint{4.847603in}{2.262942in}}%
\pgfpathlineto{\pgfqpoint{4.848183in}{1.928739in}}%
\pgfpathlineto{\pgfqpoint{4.849083in}{2.275698in}}%
\pgfpathlineto{\pgfqpoint{4.848729in}{1.859508in}}%
\pgfpathlineto{\pgfqpoint{4.849307in}{2.098808in}}%
\pgfpathlineto{\pgfqpoint{4.849339in}{2.271722in}}%
\pgfpathlineto{\pgfqpoint{4.849853in}{1.960679in}}%
\pgfpathlineto{\pgfqpoint{4.850397in}{2.181255in}}%
\pgfpathlineto{\pgfqpoint{4.850717in}{1.914409in}}%
\pgfpathlineto{\pgfqpoint{4.851037in}{2.318115in}}%
\pgfpathlineto{\pgfqpoint{4.851516in}{1.948620in}}%
\pgfpathlineto{\pgfqpoint{4.852282in}{2.305698in}}%
\pgfpathlineto{\pgfqpoint{4.852473in}{1.896370in}}%
\pgfpathlineto{\pgfqpoint{4.852632in}{2.156029in}}%
\pgfpathlineto{\pgfqpoint{4.852791in}{2.282531in}}%
\pgfpathlineto{\pgfqpoint{4.853491in}{1.926189in}}%
\pgfpathlineto{\pgfqpoint{4.853682in}{2.172169in}}%
\pgfpathlineto{\pgfqpoint{4.853936in}{2.017670in}}%
\pgfpathlineto{\pgfqpoint{4.854285in}{2.283501in}}%
\pgfpathlineto{\pgfqpoint{4.854760in}{2.157956in}}%
\pgfpathlineto{\pgfqpoint{4.855109in}{2.265018in}}%
\pgfpathlineto{\pgfqpoint{4.855330in}{1.950400in}}%
\pgfpathlineto{\pgfqpoint{4.855773in}{2.210040in}}%
\pgfpathlineto{\pgfqpoint{4.856594in}{1.836500in}}%
\pgfpathlineto{\pgfqpoint{4.856752in}{2.272726in}}%
\pgfpathlineto{\pgfqpoint{4.856878in}{2.113280in}}%
\pgfpathlineto{\pgfqpoint{4.857886in}{2.307546in}}%
\pgfpathlineto{\pgfqpoint{4.857193in}{1.797320in}}%
\pgfpathlineto{\pgfqpoint{4.857980in}{2.100359in}}%
\pgfpathlineto{\pgfqpoint{4.858483in}{2.251664in}}%
\pgfpathlineto{\pgfqpoint{4.858546in}{1.815928in}}%
\pgfpathlineto{\pgfqpoint{4.859079in}{2.137117in}}%
\pgfpathlineto{\pgfqpoint{4.860176in}{1.816359in}}%
\pgfpathlineto{\pgfqpoint{4.859173in}{2.267016in}}%
\pgfpathlineto{\pgfqpoint{4.860207in}{1.975714in}}%
\pgfpathlineto{\pgfqpoint{4.860676in}{2.281213in}}%
\pgfpathlineto{\pgfqpoint{4.860738in}{1.907276in}}%
\pgfpathlineto{\pgfqpoint{4.861332in}{2.210546in}}%
\pgfpathlineto{\pgfqpoint{4.861425in}{2.049827in}}%
\pgfpathlineto{\pgfqpoint{4.861581in}{2.277659in}}%
\pgfpathlineto{\pgfqpoint{4.862453in}{2.156234in}}%
\pgfpathlineto{\pgfqpoint{4.862515in}{1.985264in}}%
\pgfpathlineto{\pgfqpoint{4.862640in}{2.307212in}}%
\pgfpathlineto{\pgfqpoint{4.862889in}{2.116649in}}%
\pgfpathlineto{\pgfqpoint{4.862920in}{1.697347in}}%
\pgfpathlineto{\pgfqpoint{4.863603in}{2.270385in}}%
\pgfpathlineto{\pgfqpoint{4.863975in}{2.185222in}}%
\pgfpathlineto{\pgfqpoint{4.864099in}{2.284260in}}%
\pgfpathlineto{\pgfqpoint{4.864409in}{1.985979in}}%
\pgfpathlineto{\pgfqpoint{4.864781in}{2.222487in}}%
\pgfpathlineto{\pgfqpoint{4.865245in}{1.847302in}}%
\pgfpathlineto{\pgfqpoint{4.865337in}{2.305763in}}%
\pgfpathlineto{\pgfqpoint{4.865893in}{2.017082in}}%
\pgfpathlineto{\pgfqpoint{4.866880in}{2.261882in}}%
\pgfpathlineto{\pgfqpoint{4.866633in}{1.878119in}}%
\pgfpathlineto{\pgfqpoint{4.866972in}{2.217526in}}%
\pgfpathlineto{\pgfqpoint{4.867003in}{1.673391in}}%
\pgfpathlineto{\pgfqpoint{4.867403in}{2.294064in}}%
\pgfpathlineto{\pgfqpoint{4.868079in}{2.068452in}}%
\pgfpathlineto{\pgfqpoint{4.868846in}{2.267535in}}%
\pgfpathlineto{\pgfqpoint{4.868968in}{1.880464in}}%
\pgfpathlineto{\pgfqpoint{4.869214in}{2.218483in}}%
\pgfpathlineto{\pgfqpoint{4.869336in}{1.954388in}}%
\pgfpathlineto{\pgfqpoint{4.869428in}{2.280609in}}%
\pgfpathlineto{\pgfqpoint{4.870315in}{2.055865in}}%
\pgfpathlineto{\pgfqpoint{4.870742in}{2.260080in}}%
\pgfpathlineto{\pgfqpoint{4.871382in}{1.932660in}}%
\pgfpathlineto{\pgfqpoint{4.871839in}{2.275449in}}%
\pgfpathlineto{\pgfqpoint{4.872235in}{1.836294in}}%
\pgfpathlineto{\pgfqpoint{4.872751in}{2.081661in}}%
\pgfpathlineto{\pgfqpoint{4.873207in}{1.822512in}}%
\pgfpathlineto{\pgfqpoint{4.872933in}{2.274664in}}%
\pgfpathlineto{\pgfqpoint{4.873813in}{1.888696in}}%
\pgfpathlineto{\pgfqpoint{4.874630in}{2.302095in}}%
\pgfpathlineto{\pgfqpoint{4.874116in}{1.862214in}}%
\pgfpathlineto{\pgfqpoint{4.874932in}{2.182711in}}%
\pgfpathlineto{\pgfqpoint{4.875113in}{2.302496in}}%
\pgfpathlineto{\pgfqpoint{4.876109in}{1.985549in}}%
\pgfpathlineto{\pgfqpoint{4.876561in}{1.890689in}}%
\pgfpathlineto{\pgfqpoint{4.877252in}{2.256843in}}%
\pgfpathlineto{\pgfqpoint{4.877853in}{1.889810in}}%
\pgfpathlineto{\pgfqpoint{4.878213in}{2.282463in}}%
\pgfpathlineto{\pgfqpoint{4.878423in}{2.065238in}}%
\pgfpathlineto{\pgfqpoint{4.879081in}{2.277988in}}%
\pgfpathlineto{\pgfqpoint{4.879321in}{1.974137in}}%
\pgfpathlineto{\pgfqpoint{4.879530in}{2.044312in}}%
\pgfpathlineto{\pgfqpoint{4.879620in}{2.298234in}}%
\pgfpathlineto{\pgfqpoint{4.880366in}{1.913433in}}%
\pgfpathlineto{\pgfqpoint{4.880635in}{2.200972in}}%
\pgfpathlineto{\pgfqpoint{4.881052in}{1.903622in}}%
\pgfpathlineto{\pgfqpoint{4.881677in}{2.295312in}}%
\pgfpathlineto{\pgfqpoint{4.881766in}{2.010980in}}%
\pgfpathlineto{\pgfqpoint{4.882360in}{2.288367in}}%
\pgfpathlineto{\pgfqpoint{4.881944in}{1.883679in}}%
\pgfpathlineto{\pgfqpoint{4.882953in}{2.278695in}}%
\pgfpathlineto{\pgfqpoint{4.883605in}{1.758474in}}%
\pgfpathlineto{\pgfqpoint{4.883901in}{2.299634in}}%
\pgfpathlineto{\pgfqpoint{4.884108in}{1.968852in}}%
\pgfpathlineto{\pgfqpoint{4.884640in}{2.294126in}}%
\pgfpathlineto{\pgfqpoint{4.884670in}{1.832252in}}%
\pgfpathlineto{\pgfqpoint{4.885230in}{2.085482in}}%
\pgfpathlineto{\pgfqpoint{4.885348in}{1.912615in}}%
\pgfpathlineto{\pgfqpoint{4.885937in}{2.296806in}}%
\pgfpathlineto{\pgfqpoint{4.886320in}{1.938685in}}%
\pgfpathlineto{\pgfqpoint{4.886702in}{2.277974in}}%
\pgfpathlineto{\pgfqpoint{4.886555in}{1.913754in}}%
\pgfpathlineto{\pgfqpoint{4.887436in}{2.186988in}}%
\pgfpathlineto{\pgfqpoint{4.887524in}{2.269290in}}%
\pgfpathlineto{\pgfqpoint{4.888111in}{1.809421in}}%
\pgfpathlineto{\pgfqpoint{4.888403in}{2.097849in}}%
\pgfpathlineto{\pgfqpoint{4.888433in}{1.834017in}}%
\pgfpathlineto{\pgfqpoint{4.888813in}{2.272595in}}%
\pgfpathlineto{\pgfqpoint{4.889485in}{2.146799in}}%
\pgfpathlineto{\pgfqpoint{4.889602in}{2.278810in}}%
\pgfpathlineto{\pgfqpoint{4.890389in}{2.072424in}}%
\pgfpathlineto{\pgfqpoint{4.890593in}{2.188429in}}%
\pgfpathlineto{\pgfqpoint{4.891669in}{1.851864in}}%
\pgfpathlineto{\pgfqpoint{4.891437in}{2.295013in}}%
\pgfpathlineto{\pgfqpoint{4.891698in}{2.064578in}}%
\pgfpathlineto{\pgfqpoint{4.891988in}{2.320324in}}%
\pgfpathlineto{\pgfqpoint{4.892742in}{1.967550in}}%
\pgfpathlineto{\pgfqpoint{4.892800in}{2.244628in}}%
\pgfpathlineto{\pgfqpoint{4.893408in}{1.677540in}}%
\pgfpathlineto{\pgfqpoint{4.893437in}{2.274589in}}%
\pgfpathlineto{\pgfqpoint{4.893900in}{2.206255in}}%
\pgfpathlineto{\pgfqpoint{4.894189in}{2.299258in}}%
\pgfpathlineto{\pgfqpoint{4.894564in}{1.947879in}}%
\pgfpathlineto{\pgfqpoint{4.894967in}{2.223403in}}%
\pgfpathlineto{\pgfqpoint{4.895543in}{1.947798in}}%
\pgfpathlineto{\pgfqpoint{4.895572in}{2.297066in}}%
\pgfpathlineto{\pgfqpoint{4.896090in}{2.100868in}}%
\pgfpathlineto{\pgfqpoint{4.896607in}{2.266925in}}%
\pgfpathlineto{\pgfqpoint{4.896176in}{1.947552in}}%
\pgfpathlineto{\pgfqpoint{4.896693in}{2.190924in}}%
\pgfpathlineto{\pgfqpoint{4.897238in}{1.806029in}}%
\pgfpathlineto{\pgfqpoint{4.896980in}{2.262987in}}%
\pgfpathlineto{\pgfqpoint{4.897783in}{2.075171in}}%
\pgfpathlineto{\pgfqpoint{4.898783in}{2.290738in}}%
\pgfpathlineto{\pgfqpoint{4.897983in}{1.684856in}}%
\pgfpathlineto{\pgfqpoint{4.898926in}{2.261654in}}%
\pgfpathlineto{\pgfqpoint{4.899924in}{1.745271in}}%
\pgfpathlineto{\pgfqpoint{4.899953in}{2.268470in}}%
\pgfpathlineto{\pgfqpoint{4.900067in}{2.102515in}}%
\pgfpathlineto{\pgfqpoint{4.900579in}{2.295376in}}%
\pgfpathlineto{\pgfqpoint{4.900124in}{1.937996in}}%
\pgfpathlineto{\pgfqpoint{4.901176in}{2.180600in}}%
\pgfpathlineto{\pgfqpoint{4.901970in}{1.814179in}}%
\pgfpathlineto{\pgfqpoint{4.901829in}{2.288545in}}%
\pgfpathlineto{\pgfqpoint{4.902254in}{2.125669in}}%
\pgfpathlineto{\pgfqpoint{4.903216in}{2.314141in}}%
\pgfpathlineto{\pgfqpoint{4.902424in}{2.004925in}}%
\pgfpathlineto{\pgfqpoint{4.903329in}{2.181038in}}%
\pgfpathlineto{\pgfqpoint{4.903979in}{1.928476in}}%
\pgfpathlineto{\pgfqpoint{4.903950in}{2.289005in}}%
\pgfpathlineto{\pgfqpoint{4.904430in}{1.947613in}}%
\pgfpathlineto{\pgfqpoint{4.904993in}{2.244887in}}%
\pgfpathlineto{\pgfqpoint{4.904824in}{1.901442in}}%
\pgfpathlineto{\pgfqpoint{4.905556in}{2.194592in}}%
\pgfpathlineto{\pgfqpoint{4.906510in}{1.913794in}}%
\pgfpathlineto{\pgfqpoint{4.905949in}{2.277942in}}%
\pgfpathlineto{\pgfqpoint{4.906651in}{2.237475in}}%
\pgfpathlineto{\pgfqpoint{4.906735in}{2.247170in}}%
\pgfpathlineto{\pgfqpoint{4.906819in}{2.006081in}}%
\pgfpathlineto{\pgfqpoint{4.906959in}{2.116232in}}%
\pgfpathlineto{\pgfqpoint{4.907855in}{1.828454in}}%
\pgfpathlineto{\pgfqpoint{4.907603in}{2.300477in}}%
\pgfpathlineto{\pgfqpoint{4.908050in}{2.069159in}}%
\pgfpathlineto{\pgfqpoint{4.908385in}{2.289560in}}%
\pgfpathlineto{\pgfqpoint{4.908218in}{1.782096in}}%
\pgfpathlineto{\pgfqpoint{4.909167in}{2.147180in}}%
\pgfpathlineto{\pgfqpoint{4.909946in}{2.227711in}}%
\pgfpathlineto{\pgfqpoint{4.909807in}{1.890688in}}%
\pgfpathlineto{\pgfqpoint{4.910058in}{2.173500in}}%
\pgfpathlineto{\pgfqpoint{4.910363in}{1.890556in}}%
\pgfpathlineto{\pgfqpoint{4.910947in}{2.328165in}}%
\pgfpathlineto{\pgfqpoint{4.911141in}{2.166322in}}%
\pgfpathlineto{\pgfqpoint{4.912111in}{2.294426in}}%
\pgfpathlineto{\pgfqpoint{4.911335in}{1.830078in}}%
\pgfpathlineto{\pgfqpoint{4.912277in}{2.293838in}}%
\pgfpathlineto{\pgfqpoint{4.912664in}{1.957963in}}%
\pgfpathlineto{\pgfqpoint{4.913244in}{2.310718in}}%
\pgfpathlineto{\pgfqpoint{4.913410in}{2.126216in}}%
\pgfpathlineto{\pgfqpoint{4.914154in}{1.959345in}}%
\pgfpathlineto{\pgfqpoint{4.914127in}{2.292162in}}%
\pgfpathlineto{\pgfqpoint{4.914402in}{2.110797in}}%
\pgfpathlineto{\pgfqpoint{4.914842in}{2.282001in}}%
\pgfpathlineto{\pgfqpoint{4.914815in}{1.841309in}}%
\pgfpathlineto{\pgfqpoint{4.915474in}{2.160099in}}%
\pgfpathlineto{\pgfqpoint{4.916023in}{1.802429in}}%
\pgfpathlineto{\pgfqpoint{4.915886in}{2.268386in}}%
\pgfpathlineto{\pgfqpoint{4.916599in}{2.085786in}}%
\pgfpathlineto{\pgfqpoint{4.916818in}{2.311273in}}%
\pgfpathlineto{\pgfqpoint{4.916681in}{1.874127in}}%
\pgfpathlineto{\pgfqpoint{4.917720in}{2.151536in}}%
\pgfpathlineto{\pgfqpoint{4.918485in}{1.992459in}}%
\pgfpathlineto{\pgfqpoint{4.918730in}{2.276684in}}%
\pgfpathlineto{\pgfqpoint{4.918839in}{2.003881in}}%
\pgfpathlineto{\pgfqpoint{4.919873in}{2.363493in}}%
\pgfpathlineto{\pgfqpoint{4.919601in}{1.971667in}}%
\pgfpathlineto{\pgfqpoint{4.919982in}{2.210813in}}%
\pgfpathlineto{\pgfqpoint{4.920145in}{1.775823in}}%
\pgfpathlineto{\pgfqpoint{4.920498in}{2.257180in}}%
\pgfpathlineto{\pgfqpoint{4.921094in}{2.114853in}}%
\pgfpathlineto{\pgfqpoint{4.921257in}{2.278420in}}%
\pgfpathlineto{\pgfqpoint{4.921744in}{1.888825in}}%
\pgfpathlineto{\pgfqpoint{4.922231in}{2.226063in}}%
\pgfpathlineto{\pgfqpoint{4.922582in}{1.928273in}}%
\pgfpathlineto{\pgfqpoint{4.922339in}{2.307165in}}%
\pgfpathlineto{\pgfqpoint{4.923338in}{2.220323in}}%
\pgfpathlineto{\pgfqpoint{4.923607in}{1.921268in}}%
\pgfpathlineto{\pgfqpoint{4.923391in}{2.303525in}}%
\pgfpathlineto{\pgfqpoint{4.924441in}{1.950684in}}%
\pgfpathlineto{\pgfqpoint{4.924549in}{2.297460in}}%
\pgfpathlineto{\pgfqpoint{4.924844in}{1.910474in}}%
\pgfpathlineto{\pgfqpoint{4.925569in}{2.284700in}}%
\pgfpathlineto{\pgfqpoint{4.926052in}{1.910567in}}%
\pgfpathlineto{\pgfqpoint{4.926801in}{2.077071in}}%
\pgfpathlineto{\pgfqpoint{4.926988in}{2.251861in}}%
\pgfpathlineto{\pgfqpoint{4.926961in}{1.910832in}}%
\pgfpathlineto{\pgfqpoint{4.927922in}{2.159398in}}%
\pgfpathlineto{\pgfqpoint{4.928642in}{1.960137in}}%
\pgfpathlineto{\pgfqpoint{4.928322in}{2.251382in}}%
\pgfpathlineto{\pgfqpoint{4.928908in}{2.134749in}}%
\pgfpathlineto{\pgfqpoint{4.929387in}{2.291510in}}%
\pgfpathlineto{\pgfqpoint{4.929679in}{1.897644in}}%
\pgfpathlineto{\pgfqpoint{4.930024in}{2.285756in}}%
\pgfpathlineto{\pgfqpoint{4.930554in}{1.868490in}}%
\pgfpathlineto{\pgfqpoint{4.930846in}{2.292608in}}%
\pgfpathlineto{\pgfqpoint{4.931137in}{2.223105in}}%
\pgfpathlineto{\pgfqpoint{4.931745in}{2.284316in}}%
\pgfpathlineto{\pgfqpoint{4.932194in}{1.815679in}}%
\pgfpathlineto{\pgfqpoint{4.932221in}{2.339716in}}%
\pgfpathlineto{\pgfqpoint{4.932854in}{1.634501in}}%
\pgfpathlineto{\pgfqpoint{4.933302in}{2.112043in}}%
\pgfpathlineto{\pgfqpoint{4.933591in}{2.329935in}}%
\pgfpathlineto{\pgfqpoint{4.933986in}{2.037785in}}%
\pgfpathlineto{\pgfqpoint{4.934432in}{2.178213in}}%
\pgfpathlineto{\pgfqpoint{4.935088in}{1.828285in}}%
\pgfpathlineto{\pgfqpoint{4.935167in}{2.289700in}}%
\pgfpathlineto{\pgfqpoint{4.935534in}{1.992294in}}%
\pgfpathlineto{\pgfqpoint{4.936005in}{2.273376in}}%
\pgfpathlineto{\pgfqpoint{4.936371in}{1.842258in}}%
\pgfpathlineto{\pgfqpoint{4.936659in}{2.256279in}}%
\pgfpathlineto{\pgfqpoint{4.937207in}{1.789550in}}%
\pgfpathlineto{\pgfqpoint{4.936816in}{2.271417in}}%
\pgfpathlineto{\pgfqpoint{4.937807in}{1.906116in}}%
\pgfpathlineto{\pgfqpoint{4.938354in}{2.275738in}}%
\pgfpathlineto{\pgfqpoint{4.937963in}{1.874817in}}%
\pgfpathlineto{\pgfqpoint{4.938900in}{2.186170in}}%
\pgfpathlineto{\pgfqpoint{4.939834in}{1.895966in}}%
\pgfpathlineto{\pgfqpoint{4.939938in}{2.310393in}}%
\pgfpathlineto{\pgfqpoint{4.939990in}{2.100748in}}%
\pgfpathlineto{\pgfqpoint{4.940223in}{2.291417in}}%
\pgfpathlineto{\pgfqpoint{4.940482in}{1.948416in}}%
\pgfpathlineto{\pgfqpoint{4.941103in}{2.237806in}}%
\pgfpathlineto{\pgfqpoint{4.941129in}{1.721804in}}%
\pgfpathlineto{\pgfqpoint{4.941929in}{2.267723in}}%
\pgfpathlineto{\pgfqpoint{4.942213in}{2.207846in}}%
\pgfpathlineto{\pgfqpoint{4.943011in}{1.891399in}}%
\pgfpathlineto{\pgfqpoint{4.942728in}{2.317131in}}%
\pgfpathlineto{\pgfqpoint{4.943294in}{2.160857in}}%
\pgfpathlineto{\pgfqpoint{4.943449in}{2.304108in}}%
\pgfpathlineto{\pgfqpoint{4.943834in}{1.930122in}}%
\pgfpathlineto{\pgfqpoint{4.944373in}{2.100458in}}%
\pgfpathlineto{\pgfqpoint{4.944655in}{2.304780in}}%
\pgfpathlineto{\pgfqpoint{4.945142in}{1.900285in}}%
\pgfpathlineto{\pgfqpoint{4.945398in}{2.275971in}}%
\pgfpathlineto{\pgfqpoint{4.945705in}{1.806876in}}%
\pgfpathlineto{\pgfqpoint{4.946267in}{2.285696in}}%
\pgfpathlineto{\pgfqpoint{4.946497in}{2.208216in}}%
\pgfpathlineto{\pgfqpoint{4.947364in}{1.796989in}}%
\pgfpathlineto{\pgfqpoint{4.946982in}{2.303471in}}%
\pgfpathlineto{\pgfqpoint{4.947619in}{2.151093in}}%
\pgfpathlineto{\pgfqpoint{4.948001in}{2.284547in}}%
\pgfpathlineto{\pgfqpoint{4.947822in}{1.946460in}}%
\pgfpathlineto{\pgfqpoint{4.948585in}{2.170313in}}%
\pgfpathlineto{\pgfqpoint{4.949270in}{1.842877in}}%
\pgfpathlineto{\pgfqpoint{4.948865in}{2.281579in}}%
\pgfpathlineto{\pgfqpoint{4.949701in}{1.928825in}}%
\pgfpathlineto{\pgfqpoint{4.949727in}{2.293340in}}%
\pgfpathlineto{\pgfqpoint{4.949853in}{1.710807in}}%
\pgfpathlineto{\pgfqpoint{4.950815in}{2.150367in}}%
\pgfpathlineto{\pgfqpoint{4.951320in}{1.916592in}}%
\pgfpathlineto{\pgfqpoint{4.951673in}{2.309851in}}%
\pgfpathlineto{\pgfqpoint{4.951925in}{2.043774in}}%
\pgfpathlineto{\pgfqpoint{4.952932in}{2.299835in}}%
\pgfpathlineto{\pgfqpoint{4.952026in}{1.874038in}}%
\pgfpathlineto{\pgfqpoint{4.953007in}{2.141001in}}%
\pgfpathlineto{\pgfqpoint{4.953936in}{1.937405in}}%
\pgfpathlineto{\pgfqpoint{4.953183in}{2.288331in}}%
\pgfpathlineto{\pgfqpoint{4.954112in}{2.027415in}}%
\pgfpathlineto{\pgfqpoint{4.954813in}{2.317912in}}%
\pgfpathlineto{\pgfqpoint{4.954863in}{1.912741in}}%
\pgfpathlineto{\pgfqpoint{4.955213in}{2.176619in}}%
\pgfpathlineto{\pgfqpoint{4.956237in}{1.914232in}}%
\pgfpathlineto{\pgfqpoint{4.956038in}{2.319751in}}%
\pgfpathlineto{\pgfqpoint{4.956287in}{2.211872in}}%
\pgfpathlineto{\pgfqpoint{4.957010in}{1.712271in}}%
\pgfpathlineto{\pgfqpoint{4.957359in}{2.305794in}}%
\pgfpathlineto{\pgfqpoint{4.958005in}{1.875763in}}%
\pgfpathlineto{\pgfqpoint{4.958502in}{1.983901in}}%
\pgfpathlineto{\pgfqpoint{4.959146in}{2.271728in}}%
\pgfpathlineto{\pgfqpoint{4.958626in}{1.977142in}}%
\pgfpathlineto{\pgfqpoint{4.959617in}{2.146802in}}%
\pgfpathlineto{\pgfqpoint{4.960556in}{1.974955in}}%
\pgfpathlineto{\pgfqpoint{4.959765in}{2.313134in}}%
\pgfpathlineto{\pgfqpoint{4.960680in}{2.203657in}}%
\pgfpathlineto{\pgfqpoint{4.960828in}{2.288716in}}%
\pgfpathlineto{\pgfqpoint{4.960877in}{1.853626in}}%
\pgfpathlineto{\pgfqpoint{4.961764in}{2.271982in}}%
\pgfpathlineto{\pgfqpoint{4.962306in}{1.844913in}}%
\pgfpathlineto{\pgfqpoint{4.962035in}{2.307810in}}%
\pgfpathlineto{\pgfqpoint{4.962871in}{2.158949in}}%
\pgfpathlineto{\pgfqpoint{4.963926in}{2.308005in}}%
\pgfpathlineto{\pgfqpoint{4.962994in}{1.991250in}}%
\pgfpathlineto{\pgfqpoint{4.963950in}{2.211917in}}%
\pgfpathlineto{\pgfqpoint{4.964953in}{1.826210in}}%
\pgfpathlineto{\pgfqpoint{4.964880in}{2.284008in}}%
\pgfpathlineto{\pgfqpoint{4.965051in}{2.026744in}}%
\pgfpathlineto{\pgfqpoint{4.965759in}{2.266329in}}%
\pgfpathlineto{\pgfqpoint{4.965173in}{1.963799in}}%
\pgfpathlineto{\pgfqpoint{4.966149in}{2.235196in}}%
\pgfpathlineto{\pgfqpoint{4.966612in}{1.872000in}}%
\pgfpathlineto{\pgfqpoint{4.967196in}{2.282778in}}%
\pgfpathlineto{\pgfqpoint{4.967245in}{2.222903in}}%
\pgfpathlineto{\pgfqpoint{4.967366in}{2.266456in}}%
\pgfpathlineto{\pgfqpoint{4.967512in}{2.125470in}}%
\pgfpathlineto{\pgfqpoint{4.968482in}{1.856682in}}%
\pgfpathlineto{\pgfqpoint{4.967633in}{2.279479in}}%
\pgfpathlineto{\pgfqpoint{4.968579in}{1.935751in}}%
\pgfpathlineto{\pgfqpoint{4.969281in}{2.314056in}}%
\pgfpathlineto{\pgfqpoint{4.968700in}{1.904162in}}%
\pgfpathlineto{\pgfqpoint{4.969692in}{2.193217in}}%
\pgfpathlineto{\pgfqpoint{4.970176in}{2.307117in}}%
\pgfpathlineto{\pgfqpoint{4.969862in}{2.044223in}}%
\pgfpathlineto{\pgfqpoint{4.970513in}{2.217295in}}%
\pgfpathlineto{\pgfqpoint{4.971236in}{1.878439in}}%
\pgfpathlineto{\pgfqpoint{4.970899in}{2.350606in}}%
\pgfpathlineto{\pgfqpoint{4.971622in}{2.235932in}}%
\pgfpathlineto{\pgfqpoint{4.972535in}{1.923676in}}%
\pgfpathlineto{\pgfqpoint{4.971742in}{2.276289in}}%
\pgfpathlineto{\pgfqpoint{4.972775in}{2.112646in}}%
\pgfpathlineto{\pgfqpoint{4.973039in}{2.282200in}}%
\pgfpathlineto{\pgfqpoint{4.973709in}{1.864805in}}%
\pgfpathlineto{\pgfqpoint{4.973877in}{2.239882in}}%
\pgfpathlineto{\pgfqpoint{4.973949in}{1.797926in}}%
\pgfpathlineto{\pgfqpoint{4.974236in}{2.287155in}}%
\pgfpathlineto{\pgfqpoint{4.974976in}{2.109649in}}%
\pgfpathlineto{\pgfqpoint{4.975859in}{2.311527in}}%
\pgfpathlineto{\pgfqpoint{4.975191in}{1.901927in}}%
\pgfpathlineto{\pgfqpoint{4.976073in}{2.095699in}}%
\pgfpathlineto{\pgfqpoint{4.976668in}{1.809491in}}%
\pgfpathlineto{\pgfqpoint{4.976573in}{2.289454in}}%
\pgfpathlineto{\pgfqpoint{4.976715in}{1.982993in}}%
\pgfpathlineto{\pgfqpoint{4.977428in}{2.325712in}}%
\pgfpathlineto{\pgfqpoint{4.977357in}{1.896150in}}%
\pgfpathlineto{\pgfqpoint{4.977831in}{2.217630in}}%
\pgfpathlineto{\pgfqpoint{4.978210in}{2.312040in}}%
\pgfpathlineto{\pgfqpoint{4.978778in}{1.933430in}}%
\pgfpathlineto{\pgfqpoint{4.978920in}{2.198797in}}%
\pgfpathlineto{\pgfqpoint{4.979417in}{1.797404in}}%
\pgfpathlineto{\pgfqpoint{4.979794in}{2.326989in}}%
\pgfpathlineto{\pgfqpoint{4.980054in}{2.043324in}}%
\pgfpathlineto{\pgfqpoint{4.980643in}{1.960892in}}%
\pgfpathlineto{\pgfqpoint{4.980172in}{2.351791in}}%
\pgfpathlineto{\pgfqpoint{4.981137in}{2.009550in}}%
\pgfpathlineto{\pgfqpoint{4.981349in}{2.285749in}}%
\pgfpathlineto{\pgfqpoint{4.981960in}{1.795810in}}%
\pgfpathlineto{\pgfqpoint{4.982242in}{2.031446in}}%
\pgfpathlineto{\pgfqpoint{4.982945in}{2.288156in}}%
\pgfpathlineto{\pgfqpoint{4.983039in}{1.869035in}}%
\pgfpathlineto{\pgfqpoint{4.983343in}{2.030212in}}%
\pgfpathlineto{\pgfqpoint{4.983951in}{1.915836in}}%
\pgfpathlineto{\pgfqpoint{4.983530in}{2.308524in}}%
\pgfpathlineto{\pgfqpoint{4.984185in}{2.155019in}}%
\pgfpathlineto{\pgfqpoint{4.984861in}{2.294822in}}%
\pgfpathlineto{\pgfqpoint{4.984768in}{1.937943in}}%
\pgfpathlineto{\pgfqpoint{4.985188in}{2.238941in}}%
\pgfpathlineto{\pgfqpoint{4.986095in}{1.926837in}}%
\pgfpathlineto{\pgfqpoint{4.985374in}{2.289456in}}%
\pgfpathlineto{\pgfqpoint{4.986305in}{2.153307in}}%
\pgfpathlineto{\pgfqpoint{4.986630in}{2.279196in}}%
\pgfpathlineto{\pgfqpoint{4.986398in}{1.955703in}}%
\pgfpathlineto{\pgfqpoint{4.987419in}{2.205401in}}%
\pgfpathlineto{\pgfqpoint{4.987488in}{2.001506in}}%
\pgfpathlineto{\pgfqpoint{4.987743in}{2.310687in}}%
\pgfpathlineto{\pgfqpoint{4.988553in}{2.160293in}}%
\pgfpathlineto{\pgfqpoint{4.989477in}{2.276134in}}%
\pgfpathlineto{\pgfqpoint{4.989361in}{1.914712in}}%
\pgfpathlineto{\pgfqpoint{4.989500in}{2.137533in}}%
\pgfpathlineto{\pgfqpoint{4.989800in}{1.705888in}}%
\pgfpathlineto{\pgfqpoint{4.989638in}{2.314255in}}%
\pgfpathlineto{\pgfqpoint{4.990605in}{2.181213in}}%
\pgfpathlineto{\pgfqpoint{4.990720in}{1.963748in}}%
\pgfpathlineto{\pgfqpoint{4.991456in}{2.319295in}}%
\pgfpathlineto{\pgfqpoint{4.991708in}{2.075096in}}%
\pgfpathlineto{\pgfqpoint{4.992511in}{2.297265in}}%
\pgfpathlineto{\pgfqpoint{4.992694in}{1.648068in}}%
\pgfpathlineto{\pgfqpoint{4.992831in}{2.218267in}}%
\pgfpathlineto{\pgfqpoint{4.992877in}{2.249480in}}%
\pgfpathlineto{\pgfqpoint{4.993151in}{1.714552in}}%
\pgfpathlineto{\pgfqpoint{4.993654in}{2.333875in}}%
\pgfpathlineto{\pgfqpoint{4.993974in}{2.167710in}}%
\pgfpathlineto{\pgfqpoint{4.994476in}{2.316569in}}%
\pgfpathlineto{\pgfqpoint{4.994385in}{1.793636in}}%
\pgfpathlineto{\pgfqpoint{4.995068in}{2.222676in}}%
\pgfpathlineto{\pgfqpoint{4.995978in}{1.797337in}}%
\pgfpathlineto{\pgfqpoint{4.995887in}{2.366860in}}%
\pgfpathlineto{\pgfqpoint{4.996159in}{2.087067in}}%
\pgfpathlineto{\pgfqpoint{4.996182in}{2.312598in}}%
\pgfpathlineto{\pgfqpoint{4.997248in}{1.945404in}}%
\pgfpathlineto{\pgfqpoint{4.997904in}{1.650659in}}%
\pgfpathlineto{\pgfqpoint{4.997429in}{2.334921in}}%
\pgfpathlineto{\pgfqpoint{4.998085in}{2.190346in}}%
\pgfpathlineto{\pgfqpoint{4.998921in}{2.311707in}}%
\pgfpathlineto{\pgfqpoint{4.998717in}{1.948101in}}%
\pgfpathlineto{\pgfqpoint{4.999191in}{2.210207in}}%
\pgfpathlineto{\pgfqpoint{4.999552in}{2.298985in}}%
\pgfpathlineto{\pgfqpoint{4.999574in}{2.006489in}}%
\pgfpathlineto{\pgfqpoint{4.999597in}{1.833274in}}%
\pgfpathlineto{\pgfqpoint{4.999822in}{2.323294in}}%
\pgfpathlineto{\pgfqpoint{5.000654in}{2.163253in}}%
\pgfpathlineto{\pgfqpoint{5.001103in}{1.934118in}}%
\pgfpathlineto{\pgfqpoint{5.001754in}{2.292446in}}%
\pgfpathlineto{\pgfqpoint{5.002850in}{1.810690in}}%
\pgfpathlineto{\pgfqpoint{5.002873in}{2.097247in}}%
\pgfpathlineto{\pgfqpoint{5.002984in}{2.283942in}}%
\pgfpathlineto{\pgfqpoint{5.003788in}{1.861024in}}%
\pgfpathlineto{\pgfqpoint{5.003810in}{2.155170in}}%
\pgfpathlineto{\pgfqpoint{5.003832in}{1.574850in}}%
\pgfpathlineto{\pgfqpoint{5.004635in}{2.290612in}}%
\pgfpathlineto{\pgfqpoint{5.004902in}{2.238826in}}%
\pgfpathlineto{\pgfqpoint{5.005324in}{2.031018in}}%
\pgfpathlineto{\pgfqpoint{5.004946in}{2.294677in}}%
\pgfpathlineto{\pgfqpoint{5.006012in}{2.210180in}}%
\pgfpathlineto{\pgfqpoint{5.006101in}{1.716540in}}%
\pgfpathlineto{\pgfqpoint{5.006145in}{2.278051in}}%
\pgfpathlineto{\pgfqpoint{5.007076in}{2.179545in}}%
\pgfpathlineto{\pgfqpoint{5.007253in}{2.304375in}}%
\pgfpathlineto{\pgfqpoint{5.007430in}{1.818546in}}%
\pgfpathlineto{\pgfqpoint{5.008181in}{2.225136in}}%
\pgfpathlineto{\pgfqpoint{5.008578in}{1.650885in}}%
\pgfpathlineto{\pgfqpoint{5.008534in}{2.288640in}}%
\pgfpathlineto{\pgfqpoint{5.009349in}{2.094423in}}%
\pgfpathlineto{\pgfqpoint{5.009855in}{2.301904in}}%
\pgfpathlineto{\pgfqpoint{5.009393in}{1.891107in}}%
\pgfpathlineto{\pgfqpoint{5.010492in}{2.219292in}}%
\pgfpathlineto{\pgfqpoint{5.010668in}{1.972663in}}%
\pgfpathlineto{\pgfqpoint{5.010756in}{2.296251in}}%
\pgfpathlineto{\pgfqpoint{5.011610in}{2.035005in}}%
\pgfpathlineto{\pgfqpoint{5.012026in}{2.281532in}}%
\pgfpathlineto{\pgfqpoint{5.012048in}{1.854479in}}%
\pgfpathlineto{\pgfqpoint{5.012310in}{2.165502in}}%
\pgfpathlineto{\pgfqpoint{5.012332in}{1.735374in}}%
\pgfpathlineto{\pgfqpoint{5.012507in}{2.301517in}}%
\pgfpathlineto{\pgfqpoint{5.013424in}{2.099064in}}%
\pgfpathlineto{\pgfqpoint{5.013707in}{1.891106in}}%
\pgfpathlineto{\pgfqpoint{5.014121in}{2.306411in}}%
\pgfpathlineto{\pgfqpoint{5.014404in}{2.156802in}}%
\pgfpathlineto{\pgfqpoint{5.014838in}{2.299380in}}%
\pgfpathlineto{\pgfqpoint{5.015316in}{1.735893in}}%
\pgfpathlineto{\pgfqpoint{5.015511in}{2.222441in}}%
\pgfpathlineto{\pgfqpoint{5.016097in}{1.915692in}}%
\pgfpathlineto{\pgfqpoint{5.016357in}{2.286913in}}%
\pgfpathlineto{\pgfqpoint{5.016616in}{2.157953in}}%
\pgfpathlineto{\pgfqpoint{5.016638in}{2.284580in}}%
\pgfpathlineto{\pgfqpoint{5.016876in}{1.998140in}}%
\pgfpathlineto{\pgfqpoint{5.017718in}{2.204071in}}%
\pgfpathlineto{\pgfqpoint{5.018193in}{1.888041in}}%
\pgfpathlineto{\pgfqpoint{5.018064in}{2.328067in}}%
\pgfpathlineto{\pgfqpoint{5.018818in}{2.148023in}}%
\pgfpathlineto{\pgfqpoint{5.019334in}{2.311114in}}%
\pgfpathlineto{\pgfqpoint{5.019291in}{2.001979in}}%
\pgfpathlineto{\pgfqpoint{5.019936in}{2.199236in}}%
\pgfpathlineto{\pgfqpoint{5.020686in}{1.869143in}}%
\pgfpathlineto{\pgfqpoint{5.020000in}{2.312745in}}%
\pgfpathlineto{\pgfqpoint{5.020965in}{2.238768in}}%
\pgfpathlineto{\pgfqpoint{5.021093in}{2.323425in}}%
\pgfpathlineto{\pgfqpoint{5.021051in}{1.813811in}}%
\pgfpathlineto{\pgfqpoint{5.021821in}{2.176738in}}%
\pgfpathlineto{\pgfqpoint{5.022760in}{1.822002in}}%
\pgfpathlineto{\pgfqpoint{5.022163in}{2.287598in}}%
\pgfpathlineto{\pgfqpoint{5.022931in}{2.137780in}}%
\pgfpathlineto{\pgfqpoint{5.023272in}{2.320982in}}%
\pgfpathlineto{\pgfqpoint{5.023889in}{1.797946in}}%
\pgfpathlineto{\pgfqpoint{5.024059in}{2.218413in}}%
\pgfpathlineto{\pgfqpoint{5.025057in}{1.920461in}}%
\pgfpathlineto{\pgfqpoint{5.024335in}{2.313969in}}%
\pgfpathlineto{\pgfqpoint{5.025163in}{2.085033in}}%
\pgfpathlineto{\pgfqpoint{5.025608in}{2.353336in}}%
\pgfpathlineto{\pgfqpoint{5.026053in}{1.818964in}}%
\pgfpathlineto{\pgfqpoint{5.026265in}{2.077854in}}%
\pgfpathlineto{\pgfqpoint{5.026286in}{1.966895in}}%
\pgfpathlineto{\pgfqpoint{5.026624in}{2.293978in}}%
\pgfpathlineto{\pgfqpoint{5.027321in}{2.167729in}}%
\pgfpathlineto{\pgfqpoint{5.027974in}{2.326095in}}%
\pgfpathlineto{\pgfqpoint{5.027468in}{1.910229in}}%
\pgfpathlineto{\pgfqpoint{5.028417in}{2.286772in}}%
\pgfpathlineto{\pgfqpoint{5.028963in}{1.979987in}}%
\pgfpathlineto{\pgfqpoint{5.028837in}{2.309951in}}%
\pgfpathlineto{\pgfqpoint{5.029551in}{2.180102in}}%
\pgfpathlineto{\pgfqpoint{5.029656in}{2.031543in}}%
\pgfpathlineto{\pgfqpoint{5.029971in}{2.334124in}}%
\pgfpathlineto{\pgfqpoint{5.030641in}{2.128226in}}%
\pgfpathlineto{\pgfqpoint{5.031060in}{2.333586in}}%
\pgfpathlineto{\pgfqpoint{5.031457in}{1.933010in}}%
\pgfpathlineto{\pgfqpoint{5.031729in}{2.214630in}}%
\pgfpathlineto{\pgfqpoint{5.032688in}{1.798301in}}%
\pgfpathlineto{\pgfqpoint{5.032125in}{2.292430in}}%
\pgfpathlineto{\pgfqpoint{5.032834in}{2.145435in}}%
\pgfpathlineto{\pgfqpoint{5.033500in}{1.852282in}}%
\pgfpathlineto{\pgfqpoint{5.032980in}{2.330528in}}%
\pgfpathlineto{\pgfqpoint{5.033812in}{1.981461in}}%
\pgfpathlineto{\pgfqpoint{5.034393in}{2.317283in}}%
\pgfpathlineto{\pgfqpoint{5.034725in}{1.953540in}}%
\pgfpathlineto{\pgfqpoint{5.034932in}{2.129142in}}%
\pgfpathlineto{\pgfqpoint{5.035243in}{1.910994in}}%
\pgfpathlineto{\pgfqpoint{5.035657in}{2.314466in}}%
\pgfpathlineto{\pgfqpoint{5.035740in}{2.268982in}}%
\pgfpathlineto{\pgfqpoint{5.035760in}{2.317752in}}%
\pgfpathlineto{\pgfqpoint{5.036029in}{1.910497in}}%
\pgfpathlineto{\pgfqpoint{5.036793in}{2.153676in}}%
\pgfpathlineto{\pgfqpoint{5.036896in}{1.782271in}}%
\pgfpathlineto{\pgfqpoint{5.036917in}{2.285069in}}%
\pgfpathlineto{\pgfqpoint{5.037865in}{2.185983in}}%
\pgfpathlineto{\pgfqpoint{5.037906in}{2.299344in}}%
\pgfpathlineto{\pgfqpoint{5.037927in}{1.889047in}}%
\pgfpathlineto{\pgfqpoint{5.038913in}{2.289850in}}%
\pgfpathlineto{\pgfqpoint{5.039693in}{1.633114in}}%
\pgfpathlineto{\pgfqpoint{5.039508in}{2.312095in}}%
\pgfpathlineto{\pgfqpoint{5.040020in}{2.083372in}}%
\pgfpathlineto{\pgfqpoint{5.040655in}{2.344860in}}%
\pgfpathlineto{\pgfqpoint{5.040982in}{1.929906in}}%
\pgfpathlineto{\pgfqpoint{5.041104in}{2.255303in}}%
\pgfpathlineto{\pgfqpoint{5.042185in}{1.871619in}}%
\pgfpathlineto{\pgfqpoint{5.042063in}{2.307906in}}%
\pgfpathlineto{\pgfqpoint{5.042206in}{2.178069in}}%
\pgfpathlineto{\pgfqpoint{5.043162in}{1.865828in}}%
\pgfpathlineto{\pgfqpoint{5.042572in}{2.306448in}}%
\pgfpathlineto{\pgfqpoint{5.043203in}{2.146507in}}%
\pgfpathlineto{\pgfqpoint{5.043487in}{2.310981in}}%
\pgfpathlineto{\pgfqpoint{5.043934in}{1.886466in}}%
\pgfpathlineto{\pgfqpoint{5.044319in}{2.215600in}}%
\pgfpathlineto{\pgfqpoint{5.044623in}{1.904963in}}%
\pgfpathlineto{\pgfqpoint{5.044967in}{2.313808in}}%
\pgfpathlineto{\pgfqpoint{5.045433in}{2.086017in}}%
\pgfpathlineto{\pgfqpoint{5.046483in}{2.287822in}}%
\pgfpathlineto{\pgfqpoint{5.045635in}{2.005241in}}%
\pgfpathlineto{\pgfqpoint{5.046503in}{2.210802in}}%
\pgfpathlineto{\pgfqpoint{5.047268in}{2.293719in}}%
\pgfpathlineto{\pgfqpoint{5.047610in}{1.781223in}}%
\pgfpathlineto{\pgfqpoint{5.048274in}{2.308732in}}%
\pgfpathlineto{\pgfqpoint{5.047932in}{1.767404in}}%
\pgfpathlineto{\pgfqpoint{5.048715in}{2.154944in}}%
\pgfpathlineto{\pgfqpoint{5.049717in}{1.923059in}}%
\pgfpathlineto{\pgfqpoint{5.049737in}{2.330853in}}%
\pgfpathlineto{\pgfqpoint{5.049797in}{2.250822in}}%
\pgfpathlineto{\pgfqpoint{5.050177in}{2.302824in}}%
\pgfpathlineto{\pgfqpoint{5.050257in}{1.970542in}}%
\pgfpathlineto{\pgfqpoint{5.050757in}{2.139795in}}%
\pgfpathlineto{\pgfqpoint{5.051156in}{1.978422in}}%
\pgfpathlineto{\pgfqpoint{5.051016in}{2.293374in}}%
\pgfpathlineto{\pgfqpoint{5.051813in}{2.145272in}}%
\pgfpathlineto{\pgfqpoint{5.051833in}{2.315990in}}%
\pgfpathlineto{\pgfqpoint{5.051893in}{1.870855in}}%
\pgfpathlineto{\pgfqpoint{5.052927in}{2.236792in}}%
\pgfpathlineto{\pgfqpoint{5.053205in}{2.296746in}}%
\pgfpathlineto{\pgfqpoint{5.053285in}{2.062463in}}%
\pgfpathlineto{\pgfqpoint{5.053860in}{2.265016in}}%
\pgfpathlineto{\pgfqpoint{5.054632in}{1.901414in}}%
\pgfpathlineto{\pgfqpoint{5.054909in}{2.313349in}}%
\pgfpathlineto{\pgfqpoint{5.054988in}{1.957469in}}%
\pgfpathlineto{\pgfqpoint{5.055936in}{2.363934in}}%
\pgfpathlineto{\pgfqpoint{5.055265in}{1.797358in}}%
\pgfpathlineto{\pgfqpoint{5.056113in}{2.219396in}}%
\pgfpathlineto{\pgfqpoint{5.056133in}{1.805245in}}%
\pgfpathlineto{\pgfqpoint{5.056586in}{2.306468in}}%
\pgfpathlineto{\pgfqpoint{5.057236in}{1.940324in}}%
\pgfpathlineto{\pgfqpoint{5.057452in}{2.339928in}}%
\pgfpathlineto{\pgfqpoint{5.057373in}{1.936075in}}%
\pgfpathlineto{\pgfqpoint{5.058355in}{2.194421in}}%
\pgfpathlineto{\pgfqpoint{5.058492in}{1.977473in}}%
\pgfpathlineto{\pgfqpoint{5.059295in}{2.325064in}}%
\pgfpathlineto{\pgfqpoint{5.059452in}{2.099624in}}%
\pgfpathlineto{\pgfqpoint{5.059706in}{2.326016in}}%
\pgfpathlineto{\pgfqpoint{5.060019in}{1.983392in}}%
\pgfpathlineto{\pgfqpoint{5.060585in}{2.249135in}}%
\pgfpathlineto{\pgfqpoint{5.060721in}{1.966957in}}%
\pgfpathlineto{\pgfqpoint{5.060955in}{2.291352in}}%
\pgfpathlineto{\pgfqpoint{5.061734in}{2.024910in}}%
\pgfpathlineto{\pgfqpoint{5.061831in}{2.317587in}}%
\pgfpathlineto{\pgfqpoint{5.062609in}{1.804684in}}%
\pgfpathlineto{\pgfqpoint{5.062842in}{2.158579in}}%
\pgfpathlineto{\pgfqpoint{5.063753in}{1.760086in}}%
\pgfpathlineto{\pgfqpoint{5.063404in}{2.319301in}}%
\pgfpathlineto{\pgfqpoint{5.063966in}{1.872858in}}%
\pgfpathlineto{\pgfqpoint{5.064043in}{2.340131in}}%
\pgfpathlineto{\pgfqpoint{5.065087in}{2.273470in}}%
\pgfpathlineto{\pgfqpoint{5.065241in}{1.920223in}}%
\pgfpathlineto{\pgfqpoint{5.065665in}{2.309399in}}%
\pgfpathlineto{\pgfqpoint{5.066224in}{2.063635in}}%
\pgfpathlineto{\pgfqpoint{5.066397in}{2.341120in}}%
\pgfpathlineto{\pgfqpoint{5.067166in}{1.791167in}}%
\pgfpathlineto{\pgfqpoint{5.067339in}{2.218829in}}%
\pgfpathlineto{\pgfqpoint{5.067358in}{1.712827in}}%
\pgfpathlineto{\pgfqpoint{5.068010in}{2.320898in}}%
\pgfpathlineto{\pgfqpoint{5.068451in}{2.161908in}}%
\pgfpathlineto{\pgfqpoint{5.068681in}{1.942643in}}%
\pgfpathlineto{\pgfqpoint{5.068853in}{2.330260in}}%
\pgfpathlineto{\pgfqpoint{5.069446in}{2.241699in}}%
\pgfpathlineto{\pgfqpoint{5.069923in}{2.312170in}}%
\pgfpathlineto{\pgfqpoint{5.070228in}{1.817269in}}%
\pgfpathlineto{\pgfqpoint{5.070533in}{2.210097in}}%
\pgfpathlineto{\pgfqpoint{5.071086in}{1.873626in}}%
\pgfpathlineto{\pgfqpoint{5.071105in}{2.285096in}}%
\pgfpathlineto{\pgfqpoint{5.071618in}{2.132043in}}%
\pgfpathlineto{\pgfqpoint{5.071713in}{2.333080in}}%
\pgfpathlineto{\pgfqpoint{5.072188in}{1.743808in}}%
\pgfpathlineto{\pgfqpoint{5.072738in}{2.275673in}}%
\pgfpathlineto{\pgfqpoint{5.073231in}{1.899077in}}%
\pgfpathlineto{\pgfqpoint{5.073363in}{2.311245in}}%
\pgfpathlineto{\pgfqpoint{5.073855in}{2.098381in}}%
\pgfpathlineto{\pgfqpoint{5.074913in}{2.368412in}}%
\pgfpathlineto{\pgfqpoint{5.074799in}{1.962333in}}%
\pgfpathlineto{\pgfqpoint{5.074969in}{2.237099in}}%
\pgfpathlineto{\pgfqpoint{5.075798in}{1.904569in}}%
\pgfpathlineto{\pgfqpoint{5.075271in}{2.325116in}}%
\pgfpathlineto{\pgfqpoint{5.076062in}{2.156756in}}%
\pgfpathlineto{\pgfqpoint{5.076682in}{2.330241in}}%
\pgfpathlineto{\pgfqpoint{5.076870in}{1.960002in}}%
\pgfpathlineto{\pgfqpoint{5.077151in}{2.217516in}}%
\pgfpathlineto{\pgfqpoint{5.077489in}{2.005043in}}%
\pgfpathlineto{\pgfqpoint{5.078238in}{2.313871in}}%
\pgfpathlineto{\pgfqpoint{5.078257in}{2.057061in}}%
\pgfpathlineto{\pgfqpoint{5.078313in}{2.343489in}}%
\pgfpathlineto{\pgfqpoint{5.078481in}{1.836923in}}%
\pgfpathlineto{\pgfqpoint{5.079378in}{2.157797in}}%
\pgfpathlineto{\pgfqpoint{5.079695in}{2.320894in}}%
\pgfpathlineto{\pgfqpoint{5.080142in}{1.880962in}}%
\pgfpathlineto{\pgfqpoint{5.080440in}{2.216990in}}%
\pgfpathlineto{\pgfqpoint{5.080459in}{1.868599in}}%
\pgfpathlineto{\pgfqpoint{5.081054in}{2.325755in}}%
\pgfpathlineto{\pgfqpoint{5.081537in}{2.187846in}}%
\pgfpathlineto{\pgfqpoint{5.081779in}{2.317439in}}%
\pgfpathlineto{\pgfqpoint{5.081649in}{1.806976in}}%
\pgfpathlineto{\pgfqpoint{5.082650in}{2.197942in}}%
\pgfpathlineto{\pgfqpoint{5.082835in}{1.846573in}}%
\pgfpathlineto{\pgfqpoint{5.083427in}{2.319576in}}%
\pgfpathlineto{\pgfqpoint{5.083760in}{1.977599in}}%
\pgfpathlineto{\pgfqpoint{5.084682in}{2.317469in}}%
\pgfpathlineto{\pgfqpoint{5.083797in}{1.799921in}}%
\pgfpathlineto{\pgfqpoint{5.084885in}{2.279559in}}%
\pgfpathlineto{\pgfqpoint{5.085842in}{1.969464in}}%
\pgfpathlineto{\pgfqpoint{5.085787in}{2.297605in}}%
\pgfpathlineto{\pgfqpoint{5.085989in}{2.180320in}}%
\pgfpathlineto{\pgfqpoint{5.086650in}{2.332725in}}%
\pgfpathlineto{\pgfqpoint{5.086687in}{1.899299in}}%
\pgfpathlineto{\pgfqpoint{5.087072in}{2.174772in}}%
\pgfpathlineto{\pgfqpoint{5.087200in}{1.891581in}}%
\pgfpathlineto{\pgfqpoint{5.087456in}{2.293134in}}%
\pgfpathlineto{\pgfqpoint{5.087877in}{2.220895in}}%
\pgfpathlineto{\pgfqpoint{5.087896in}{2.301758in}}%
\pgfpathlineto{\pgfqpoint{5.088389in}{2.013533in}}%
\pgfpathlineto{\pgfqpoint{5.088973in}{2.286312in}}%
\pgfpathlineto{\pgfqpoint{5.090049in}{1.829162in}}%
\pgfpathlineto{\pgfqpoint{5.089867in}{2.305461in}}%
\pgfpathlineto{\pgfqpoint{5.090103in}{2.189713in}}%
\pgfpathlineto{\pgfqpoint{5.090612in}{2.329309in}}%
\pgfpathlineto{\pgfqpoint{5.090721in}{2.025528in}}%
\pgfpathlineto{\pgfqpoint{5.090958in}{2.128120in}}%
\pgfpathlineto{\pgfqpoint{5.091212in}{1.922833in}}%
\pgfpathlineto{\pgfqpoint{5.091067in}{2.292592in}}%
\pgfpathlineto{\pgfqpoint{5.092028in}{2.269543in}}%
\pgfpathlineto{\pgfqpoint{5.092082in}{2.335522in}}%
\pgfpathlineto{\pgfqpoint{5.092661in}{1.839788in}}%
\pgfpathlineto{\pgfqpoint{5.092878in}{2.316179in}}%
\pgfpathlineto{\pgfqpoint{5.092896in}{1.975900in}}%
\pgfpathlineto{\pgfqpoint{5.093474in}{2.329016in}}%
\pgfpathlineto{\pgfqpoint{5.093998in}{2.086083in}}%
\pgfpathlineto{\pgfqpoint{5.094808in}{2.340774in}}%
\pgfpathlineto{\pgfqpoint{5.094646in}{1.898273in}}%
\pgfpathlineto{\pgfqpoint{5.095114in}{2.241031in}}%
\pgfpathlineto{\pgfqpoint{5.095707in}{1.946542in}}%
\pgfpathlineto{\pgfqpoint{5.095959in}{2.322044in}}%
\pgfpathlineto{\pgfqpoint{5.096102in}{2.059314in}}%
\pgfpathlineto{\pgfqpoint{5.096801in}{2.336453in}}%
\pgfpathlineto{\pgfqpoint{5.097106in}{1.941254in}}%
\pgfpathlineto{\pgfqpoint{5.097213in}{2.219487in}}%
\pgfpathlineto{\pgfqpoint{5.098285in}{1.735339in}}%
\pgfpathlineto{\pgfqpoint{5.097499in}{2.317794in}}%
\pgfpathlineto{\pgfqpoint{5.098321in}{2.049328in}}%
\pgfpathlineto{\pgfqpoint{5.098678in}{2.330688in}}%
\pgfpathlineto{\pgfqpoint{5.099088in}{1.947302in}}%
\pgfpathlineto{\pgfqpoint{5.099426in}{2.234632in}}%
\pgfpathlineto{\pgfqpoint{5.100138in}{1.832660in}}%
\pgfpathlineto{\pgfqpoint{5.099515in}{2.328775in}}%
\pgfpathlineto{\pgfqpoint{5.100529in}{2.232591in}}%
\pgfpathlineto{\pgfqpoint{5.101415in}{2.316428in}}%
\pgfpathlineto{\pgfqpoint{5.101114in}{1.903437in}}%
\pgfpathlineto{\pgfqpoint{5.101575in}{2.270115in}}%
\pgfpathlineto{\pgfqpoint{5.101858in}{1.863176in}}%
\pgfpathlineto{\pgfqpoint{5.102000in}{2.343001in}}%
\pgfpathlineto{\pgfqpoint{5.102672in}{2.069544in}}%
\pgfpathlineto{\pgfqpoint{5.103272in}{2.307062in}}%
\pgfpathlineto{\pgfqpoint{5.102707in}{1.872418in}}%
\pgfpathlineto{\pgfqpoint{5.103783in}{2.112892in}}%
\pgfpathlineto{\pgfqpoint{5.104575in}{2.322653in}}%
\pgfpathlineto{\pgfqpoint{5.104047in}{1.786003in}}%
\pgfpathlineto{\pgfqpoint{5.104909in}{2.228215in}}%
\pgfpathlineto{\pgfqpoint{5.105278in}{1.957967in}}%
\pgfpathlineto{\pgfqpoint{5.105594in}{2.311785in}}%
\pgfpathlineto{\pgfqpoint{5.106050in}{2.000287in}}%
\pgfpathlineto{\pgfqpoint{5.106103in}{2.329776in}}%
\pgfpathlineto{\pgfqpoint{5.106138in}{1.938051in}}%
\pgfpathlineto{\pgfqpoint{5.107170in}{2.279270in}}%
\pgfpathlineto{\pgfqpoint{5.107921in}{1.818935in}}%
\pgfpathlineto{\pgfqpoint{5.108061in}{2.301045in}}%
\pgfpathlineto{\pgfqpoint{5.108287in}{2.165983in}}%
\pgfpathlineto{\pgfqpoint{5.108984in}{1.709320in}}%
\pgfpathlineto{\pgfqpoint{5.108897in}{2.352334in}}%
\pgfpathlineto{\pgfqpoint{5.109367in}{2.196699in}}%
\pgfpathlineto{\pgfqpoint{5.110426in}{2.341110in}}%
\pgfpathlineto{\pgfqpoint{5.110322in}{1.848737in}}%
\pgfpathlineto{\pgfqpoint{5.110461in}{2.218293in}}%
\pgfpathlineto{\pgfqpoint{5.110513in}{1.925661in}}%
\pgfpathlineto{\pgfqpoint{5.111033in}{2.327156in}}%
\pgfpathlineto{\pgfqpoint{5.111587in}{2.165138in}}%
\pgfpathlineto{\pgfqpoint{5.112054in}{2.315302in}}%
\pgfpathlineto{\pgfqpoint{5.112123in}{1.923275in}}%
\pgfpathlineto{\pgfqpoint{5.112693in}{2.278919in}}%
\pgfpathlineto{\pgfqpoint{5.113055in}{1.808093in}}%
\pgfpathlineto{\pgfqpoint{5.113331in}{2.384006in}}%
\pgfpathlineto{\pgfqpoint{5.113813in}{2.079830in}}%
\pgfpathlineto{\pgfqpoint{5.114466in}{2.329915in}}%
\pgfpathlineto{\pgfqpoint{5.114844in}{1.924215in}}%
\pgfpathlineto{\pgfqpoint{5.114930in}{2.275773in}}%
\pgfpathlineto{\pgfqpoint{5.115907in}{1.862429in}}%
\pgfpathlineto{\pgfqpoint{5.115496in}{2.336253in}}%
\pgfpathlineto{\pgfqpoint{5.116061in}{2.079487in}}%
\pgfpathlineto{\pgfqpoint{5.116266in}{2.324468in}}%
\pgfpathlineto{\pgfqpoint{5.116865in}{1.919921in}}%
\pgfpathlineto{\pgfqpoint{5.117189in}{2.238908in}}%
\pgfpathlineto{\pgfqpoint{5.117531in}{1.905165in}}%
\pgfpathlineto{\pgfqpoint{5.117718in}{2.320320in}}%
\pgfpathlineto{\pgfqpoint{5.118298in}{2.230002in}}%
\pgfpathlineto{\pgfqpoint{5.118995in}{2.349719in}}%
\pgfpathlineto{\pgfqpoint{5.118570in}{1.935918in}}%
\pgfpathlineto{\pgfqpoint{5.119386in}{2.187867in}}%
\pgfpathlineto{\pgfqpoint{5.119437in}{2.319085in}}%
\pgfpathlineto{\pgfqpoint{5.120489in}{1.888183in}}%
\pgfpathlineto{\pgfqpoint{5.121402in}{2.340577in}}%
\pgfpathlineto{\pgfqpoint{5.121605in}{2.225843in}}%
\pgfpathlineto{\pgfqpoint{5.122078in}{1.806767in}}%
\pgfpathlineto{\pgfqpoint{5.121774in}{2.314375in}}%
\pgfpathlineto{\pgfqpoint{5.122702in}{2.187355in}}%
\pgfpathlineto{\pgfqpoint{5.123056in}{2.362286in}}%
\pgfpathlineto{\pgfqpoint{5.123544in}{1.771220in}}%
\pgfpathlineto{\pgfqpoint{5.123796in}{2.216037in}}%
\pgfpathlineto{\pgfqpoint{5.123930in}{1.714773in}}%
\pgfpathlineto{\pgfqpoint{5.124350in}{2.330259in}}%
\pgfpathlineto{\pgfqpoint{5.124837in}{2.224898in}}%
\pgfpathlineto{\pgfqpoint{5.124853in}{2.376489in}}%
\pgfpathlineto{\pgfqpoint{5.125339in}{1.897458in}}%
\pgfpathlineto{\pgfqpoint{5.125942in}{2.201963in}}%
\pgfpathlineto{\pgfqpoint{5.126744in}{2.024751in}}%
\pgfpathlineto{\pgfqpoint{5.126894in}{2.333541in}}%
\pgfpathlineto{\pgfqpoint{5.127028in}{2.182791in}}%
\pgfpathlineto{\pgfqpoint{5.127694in}{2.386656in}}%
\pgfpathlineto{\pgfqpoint{5.127228in}{1.896073in}}%
\pgfpathlineto{\pgfqpoint{5.128144in}{2.301626in}}%
\pgfpathlineto{\pgfqpoint{5.128210in}{1.776208in}}%
\pgfpathlineto{\pgfqpoint{5.129141in}{2.347759in}}%
\pgfpathlineto{\pgfqpoint{5.129257in}{2.261747in}}%
\pgfpathlineto{\pgfqpoint{5.129788in}{1.936662in}}%
\pgfpathlineto{\pgfqpoint{5.129688in}{2.332535in}}%
\pgfpathlineto{\pgfqpoint{5.130367in}{2.126010in}}%
\pgfpathlineto{\pgfqpoint{5.130946in}{2.322019in}}%
\pgfpathlineto{\pgfqpoint{5.130748in}{1.764999in}}%
\pgfpathlineto{\pgfqpoint{5.131475in}{2.178751in}}%
\pgfpathlineto{\pgfqpoint{5.132118in}{2.344707in}}%
\pgfpathlineto{\pgfqpoint{5.132250in}{1.981643in}}%
\pgfpathlineto{\pgfqpoint{5.132596in}{2.261878in}}%
\pgfpathlineto{\pgfqpoint{5.132974in}{1.742192in}}%
\pgfpathlineto{\pgfqpoint{5.133287in}{2.359078in}}%
\pgfpathlineto{\pgfqpoint{5.133697in}{2.235729in}}%
\pgfpathlineto{\pgfqpoint{5.133812in}{2.332351in}}%
\pgfpathlineto{\pgfqpoint{5.133861in}{1.905547in}}%
\pgfpathlineto{\pgfqpoint{5.134796in}{2.301953in}}%
\pgfpathlineto{\pgfqpoint{5.135859in}{1.641644in}}%
\pgfpathlineto{\pgfqpoint{5.135532in}{2.353038in}}%
\pgfpathlineto{\pgfqpoint{5.135941in}{1.959596in}}%
\pgfpathlineto{\pgfqpoint{5.136610in}{2.319553in}}%
\pgfpathlineto{\pgfqpoint{5.137067in}{2.247049in}}%
\pgfpathlineto{\pgfqpoint{5.137734in}{1.884888in}}%
\pgfpathlineto{\pgfqpoint{5.137555in}{2.360877in}}%
\pgfpathlineto{\pgfqpoint{5.138173in}{2.233320in}}%
\pgfpathlineto{\pgfqpoint{5.138189in}{2.233083in}}%
\pgfpathlineto{\pgfqpoint{5.138352in}{1.950031in}}%
\pgfpathlineto{\pgfqpoint{5.139195in}{2.348194in}}%
\pgfpathlineto{\pgfqpoint{5.139293in}{2.291803in}}%
\pgfpathlineto{\pgfqpoint{5.139422in}{1.905492in}}%
\pgfpathlineto{\pgfqpoint{5.139649in}{2.329850in}}%
\pgfpathlineto{\pgfqpoint{5.140442in}{2.051200in}}%
\pgfpathlineto{\pgfqpoint{5.140571in}{2.366581in}}%
\pgfpathlineto{\pgfqpoint{5.140845in}{1.832496in}}%
\pgfpathlineto{\pgfqpoint{5.141555in}{2.225653in}}%
\pgfpathlineto{\pgfqpoint{5.141797in}{2.333321in}}%
\pgfpathlineto{\pgfqpoint{5.141749in}{2.029949in}}%
\pgfpathlineto{\pgfqpoint{5.142586in}{2.278052in}}%
\pgfpathlineto{\pgfqpoint{5.143341in}{1.855020in}}%
\pgfpathlineto{\pgfqpoint{5.143068in}{2.369196in}}%
\pgfpathlineto{\pgfqpoint{5.143694in}{2.270509in}}%
\pgfpathlineto{\pgfqpoint{5.143886in}{1.919720in}}%
\pgfpathlineto{\pgfqpoint{5.143790in}{2.345989in}}%
\pgfpathlineto{\pgfqpoint{5.144831in}{2.111270in}}%
\pgfpathlineto{\pgfqpoint{5.145582in}{2.340082in}}%
\pgfpathlineto{\pgfqpoint{5.144879in}{1.901266in}}%
\pgfpathlineto{\pgfqpoint{5.145933in}{2.253337in}}%
\pgfpathlineto{\pgfqpoint{5.146077in}{1.976099in}}%
\pgfpathlineto{\pgfqpoint{5.146459in}{2.381309in}}%
\pgfpathlineto{\pgfqpoint{5.147033in}{2.164741in}}%
\pgfpathlineto{\pgfqpoint{5.147303in}{2.354351in}}%
\pgfpathlineto{\pgfqpoint{5.148066in}{1.889955in}}%
\pgfpathlineto{\pgfqpoint{5.148129in}{2.097968in}}%
\pgfpathlineto{\pgfqpoint{5.148970in}{2.303114in}}%
\pgfpathlineto{\pgfqpoint{5.148320in}{2.023190in}}%
\pgfpathlineto{\pgfqpoint{5.149223in}{2.198217in}}%
\pgfpathlineto{\pgfqpoint{5.150314in}{1.859821in}}%
\pgfpathlineto{\pgfqpoint{5.150219in}{2.384956in}}%
\pgfpathlineto{\pgfqpoint{5.150330in}{2.221854in}}%
\pgfpathlineto{\pgfqpoint{5.150661in}{2.324277in}}%
\pgfpathlineto{\pgfqpoint{5.150519in}{1.581233in}}%
\pgfpathlineto{\pgfqpoint{5.151434in}{2.213738in}}%
\pgfpathlineto{\pgfqpoint{5.152487in}{1.891349in}}%
\pgfpathlineto{\pgfqpoint{5.152094in}{2.332628in}}%
\pgfpathlineto{\pgfqpoint{5.152566in}{2.110760in}}%
\pgfpathlineto{\pgfqpoint{5.152723in}{2.325138in}}%
\pgfpathlineto{\pgfqpoint{5.153586in}{1.702301in}}%
\pgfpathlineto{\pgfqpoint{5.153695in}{2.311289in}}%
\pgfpathlineto{\pgfqpoint{5.154603in}{1.999190in}}%
\pgfpathlineto{\pgfqpoint{5.153915in}{2.336582in}}%
\pgfpathlineto{\pgfqpoint{5.154806in}{2.252508in}}%
\pgfpathlineto{\pgfqpoint{5.154993in}{2.043862in}}%
\pgfpathlineto{\pgfqpoint{5.155462in}{2.378961in}}%
\pgfpathlineto{\pgfqpoint{5.155914in}{2.231669in}}%
\pgfpathlineto{\pgfqpoint{5.156895in}{1.957677in}}%
\pgfpathlineto{\pgfqpoint{5.156972in}{2.332811in}}%
\pgfpathlineto{\pgfqpoint{5.157019in}{2.247988in}}%
\pgfpathlineto{\pgfqpoint{5.157190in}{1.731762in}}%
\pgfpathlineto{\pgfqpoint{5.158012in}{2.307684in}}%
\pgfpathlineto{\pgfqpoint{5.158043in}{2.258660in}}%
\pgfpathlineto{\pgfqpoint{5.158090in}{1.794979in}}%
\pgfpathlineto{\pgfqpoint{5.159143in}{2.355356in}}%
\pgfpathlineto{\pgfqpoint{5.160085in}{1.708241in}}%
\pgfpathlineto{\pgfqpoint{5.160286in}{1.942545in}}%
\pgfpathlineto{\pgfqpoint{5.160948in}{2.346727in}}%
\pgfpathlineto{\pgfqpoint{5.161195in}{1.788904in}}%
\pgfpathlineto{\pgfqpoint{5.161410in}{2.263705in}}%
\pgfpathlineto{\pgfqpoint{5.162163in}{2.310944in}}%
\pgfpathlineto{\pgfqpoint{5.162455in}{1.814124in}}%
\pgfpathlineto{\pgfqpoint{5.163068in}{2.337741in}}%
\pgfpathlineto{\pgfqpoint{5.163574in}{2.191046in}}%
\pgfpathlineto{\pgfqpoint{5.163589in}{2.194936in}}%
\pgfpathlineto{\pgfqpoint{5.163604in}{1.831221in}}%
\pgfpathlineto{\pgfqpoint{5.164430in}{2.362204in}}%
\pgfpathlineto{\pgfqpoint{5.164689in}{2.322730in}}%
\pgfpathlineto{\pgfqpoint{5.165680in}{1.897184in}}%
\pgfpathlineto{\pgfqpoint{5.165467in}{2.343594in}}%
\pgfpathlineto{\pgfqpoint{5.165817in}{2.257733in}}%
\pgfpathlineto{\pgfqpoint{5.166061in}{2.299705in}}%
\pgfpathlineto{\pgfqpoint{5.166015in}{1.703786in}}%
\pgfpathlineto{\pgfqpoint{5.166760in}{2.276323in}}%
\pgfpathlineto{\pgfqpoint{5.167746in}{1.922534in}}%
\pgfpathlineto{\pgfqpoint{5.167443in}{2.370245in}}%
\pgfpathlineto{\pgfqpoint{5.167868in}{2.217777in}}%
\pgfpathlineto{\pgfqpoint{5.167883in}{2.357667in}}%
\pgfpathlineto{\pgfqpoint{5.168443in}{1.870692in}}%
\pgfpathlineto{\pgfqpoint{5.168972in}{2.265311in}}%
\pgfpathlineto{\pgfqpoint{5.169154in}{1.826382in}}%
\pgfpathlineto{\pgfqpoint{5.169576in}{2.342781in}}%
\pgfpathlineto{\pgfqpoint{5.170104in}{2.157324in}}%
\pgfpathlineto{\pgfqpoint{5.170706in}{2.366711in}}%
\pgfpathlineto{\pgfqpoint{5.171022in}{1.951967in}}%
\pgfpathlineto{\pgfqpoint{5.171188in}{2.164610in}}%
\pgfpathlineto{\pgfqpoint{5.172134in}{1.506625in}}%
\pgfpathlineto{\pgfqpoint{5.171999in}{2.355082in}}%
\pgfpathlineto{\pgfqpoint{5.172254in}{2.045128in}}%
\pgfpathlineto{\pgfqpoint{5.173227in}{2.343416in}}%
\pgfpathlineto{\pgfqpoint{5.173018in}{1.885852in}}%
\pgfpathlineto{\pgfqpoint{5.173362in}{2.155397in}}%
\pgfpathlineto{\pgfqpoint{5.174079in}{2.360351in}}%
\pgfpathlineto{\pgfqpoint{5.173541in}{1.900404in}}%
\pgfpathlineto{\pgfqpoint{5.174467in}{2.188779in}}%
\pgfpathlineto{\pgfqpoint{5.175108in}{1.929876in}}%
\pgfpathlineto{\pgfqpoint{5.175436in}{2.344303in}}%
\pgfpathlineto{\pgfqpoint{5.175510in}{1.964983in}}%
\pgfpathlineto{\pgfqpoint{5.176135in}{2.352684in}}%
\pgfpathlineto{\pgfqpoint{5.176001in}{1.937612in}}%
\pgfpathlineto{\pgfqpoint{5.176625in}{2.216923in}}%
\pgfpathlineto{\pgfqpoint{5.176803in}{2.364211in}}%
\pgfpathlineto{\pgfqpoint{5.177440in}{1.832583in}}%
\pgfpathlineto{\pgfqpoint{5.177736in}{2.224559in}}%
\pgfpathlineto{\pgfqpoint{5.177943in}{2.337289in}}%
\pgfpathlineto{\pgfqpoint{5.177899in}{2.155188in}}%
\pgfpathlineto{\pgfqpoint{5.178062in}{2.244784in}}%
\pgfpathlineto{\pgfqpoint{5.178697in}{1.814772in}}%
\pgfpathlineto{\pgfqpoint{5.178756in}{2.335209in}}%
\pgfpathlineto{\pgfqpoint{5.179184in}{2.065772in}}%
\pgfpathlineto{\pgfqpoint{5.179715in}{2.369137in}}%
\pgfpathlineto{\pgfqpoint{5.179730in}{1.907905in}}%
\pgfpathlineto{\pgfqpoint{5.180289in}{2.254089in}}%
\pgfpathlineto{\pgfqpoint{5.180701in}{1.865456in}}%
\pgfpathlineto{\pgfqpoint{5.180966in}{2.336873in}}%
\pgfpathlineto{\pgfqpoint{5.181421in}{2.071688in}}%
\pgfpathlineto{\pgfqpoint{5.181714in}{2.336524in}}%
\pgfpathlineto{\pgfqpoint{5.182242in}{1.964423in}}%
\pgfpathlineto{\pgfqpoint{5.182256in}{2.106000in}}%
\pgfpathlineto{\pgfqpoint{5.182271in}{1.863790in}}%
\pgfpathlineto{\pgfqpoint{5.183163in}{2.332151in}}%
\pgfpathlineto{\pgfqpoint{5.183353in}{2.233570in}}%
\pgfpathlineto{\pgfqpoint{5.183368in}{2.233818in}}%
\pgfpathlineto{\pgfqpoint{5.183441in}{1.823671in}}%
\pgfpathlineto{\pgfqpoint{5.184170in}{2.356849in}}%
\pgfpathlineto{\pgfqpoint{5.184462in}{2.223089in}}%
\pgfpathlineto{\pgfqpoint{5.184695in}{2.345402in}}%
\pgfpathlineto{\pgfqpoint{5.185379in}{1.909352in}}%
\pgfpathlineto{\pgfqpoint{5.185539in}{2.198365in}}%
\pgfpathlineto{\pgfqpoint{5.186105in}{1.920552in}}%
\pgfpathlineto{\pgfqpoint{5.185873in}{2.356201in}}%
\pgfpathlineto{\pgfqpoint{5.186627in}{2.171632in}}%
\pgfpathlineto{\pgfqpoint{5.187612in}{2.345317in}}%
\pgfpathlineto{\pgfqpoint{5.187496in}{1.879603in}}%
\pgfpathlineto{\pgfqpoint{5.187727in}{2.103880in}}%
\pgfpathlineto{\pgfqpoint{5.188378in}{2.365464in}}%
\pgfpathlineto{\pgfqpoint{5.188219in}{1.913234in}}%
\pgfpathlineto{\pgfqpoint{5.188839in}{2.268373in}}%
\pgfpathlineto{\pgfqpoint{5.189315in}{1.928157in}}%
\pgfpathlineto{\pgfqpoint{5.189776in}{2.336894in}}%
\pgfpathlineto{\pgfqpoint{5.189948in}{2.172445in}}%
\pgfpathlineto{\pgfqpoint{5.190939in}{2.332368in}}%
\pgfpathlineto{\pgfqpoint{5.190034in}{1.834667in}}%
\pgfpathlineto{\pgfqpoint{5.191054in}{2.213173in}}%
\pgfpathlineto{\pgfqpoint{5.191140in}{1.871287in}}%
\pgfpathlineto{\pgfqpoint{5.191799in}{2.328693in}}%
\pgfpathlineto{\pgfqpoint{5.192057in}{2.149327in}}%
\pgfpathlineto{\pgfqpoint{5.192143in}{2.314139in}}%
\pgfpathlineto{\pgfqpoint{5.193129in}{2.040767in}}%
\pgfpathlineto{\pgfqpoint{5.193172in}{2.264820in}}%
\pgfpathlineto{\pgfqpoint{5.193657in}{2.350865in}}%
\pgfpathlineto{\pgfqpoint{5.193586in}{1.753886in}}%
\pgfpathlineto{\pgfqpoint{5.194184in}{2.220593in}}%
\pgfpathlineto{\pgfqpoint{5.194696in}{1.821705in}}%
\pgfpathlineto{\pgfqpoint{5.194824in}{2.342914in}}%
\pgfpathlineto{\pgfqpoint{5.195293in}{2.225126in}}%
\pgfpathlineto{\pgfqpoint{5.196187in}{2.349710in}}%
\pgfpathlineto{\pgfqpoint{5.196414in}{1.874734in}}%
\pgfpathlineto{\pgfqpoint{5.196428in}{2.328014in}}%
\pgfpathlineto{\pgfqpoint{5.197531in}{2.266896in}}%
\pgfpathlineto{\pgfqpoint{5.197560in}{1.622834in}}%
\pgfpathlineto{\pgfqpoint{5.197715in}{2.331097in}}%
\pgfpathlineto{\pgfqpoint{5.198646in}{2.136225in}}%
\pgfpathlineto{\pgfqpoint{5.199730in}{2.340657in}}%
\pgfpathlineto{\pgfqpoint{5.199068in}{1.901893in}}%
\pgfpathlineto{\pgfqpoint{5.199744in}{2.123743in}}%
\pgfpathlineto{\pgfqpoint{5.200782in}{2.021058in}}%
\pgfpathlineto{\pgfqpoint{5.200305in}{2.320786in}}%
\pgfpathlineto{\pgfqpoint{5.200810in}{2.143492in}}%
\pgfpathlineto{\pgfqpoint{5.201399in}{2.367008in}}%
\pgfpathlineto{\pgfqpoint{5.201483in}{1.667859in}}%
\pgfpathlineto{\pgfqpoint{5.201902in}{2.154752in}}%
\pgfpathlineto{\pgfqpoint{5.202210in}{1.925157in}}%
\pgfpathlineto{\pgfqpoint{5.202782in}{2.331788in}}%
\pgfpathlineto{\pgfqpoint{5.202978in}{2.262030in}}%
\pgfpathlineto{\pgfqpoint{5.203424in}{2.361105in}}%
\pgfpathlineto{\pgfqpoint{5.203257in}{1.885777in}}%
\pgfpathlineto{\pgfqpoint{5.204064in}{2.221827in}}%
\pgfpathlineto{\pgfqpoint{5.204912in}{1.932679in}}%
\pgfpathlineto{\pgfqpoint{5.204342in}{2.334392in}}%
\pgfpathlineto{\pgfqpoint{5.205079in}{2.252447in}}%
\pgfpathlineto{\pgfqpoint{5.205092in}{2.341027in}}%
\pgfpathlineto{\pgfqpoint{5.205481in}{1.941472in}}%
\pgfpathlineto{\pgfqpoint{5.206160in}{2.265525in}}%
\pgfpathlineto{\pgfqpoint{5.206658in}{1.904624in}}%
\pgfpathlineto{\pgfqpoint{5.206685in}{2.340968in}}%
\pgfpathlineto{\pgfqpoint{5.207266in}{2.232171in}}%
\pgfpathlineto{\pgfqpoint{5.207473in}{2.293598in}}%
\pgfpathlineto{\pgfqpoint{5.207459in}{2.029269in}}%
\pgfpathlineto{\pgfqpoint{5.207556in}{2.241171in}}%
\pgfpathlineto{\pgfqpoint{5.207569in}{1.899113in}}%
\pgfpathlineto{\pgfqpoint{5.207707in}{2.363682in}}%
\pgfpathlineto{\pgfqpoint{5.208658in}{2.166615in}}%
\pgfpathlineto{\pgfqpoint{5.208947in}{2.349421in}}%
\pgfpathlineto{\pgfqpoint{5.209043in}{1.886308in}}%
\pgfpathlineto{\pgfqpoint{5.209799in}{2.250635in}}%
\pgfpathlineto{\pgfqpoint{5.210567in}{1.936632in}}%
\pgfpathlineto{\pgfqpoint{5.210840in}{2.351577in}}%
\pgfpathlineto{\pgfqpoint{5.210895in}{2.145296in}}%
\pgfpathlineto{\pgfqpoint{5.211183in}{2.338729in}}%
\pgfpathlineto{\pgfqpoint{5.211634in}{1.918265in}}%
\pgfpathlineto{\pgfqpoint{5.211989in}{2.318714in}}%
\pgfpathlineto{\pgfqpoint{5.213094in}{1.724930in}}%
\pgfpathlineto{\pgfqpoint{5.212944in}{2.364043in}}%
\pgfpathlineto{\pgfqpoint{5.213107in}{2.061811in}}%
\pgfpathlineto{\pgfqpoint{5.213379in}{2.338393in}}%
\pgfpathlineto{\pgfqpoint{5.213516in}{1.816520in}}%
\pgfpathlineto{\pgfqpoint{5.214236in}{2.273288in}}%
\pgfpathlineto{\pgfqpoint{5.214358in}{1.929453in}}%
\pgfpathlineto{\pgfqpoint{5.215267in}{2.378486in}}%
\pgfpathlineto{\pgfqpoint{5.215348in}{2.264120in}}%
\pgfpathlineto{\pgfqpoint{5.215809in}{2.349519in}}%
\pgfpathlineto{\pgfqpoint{5.215389in}{1.976947in}}%
\pgfpathlineto{\pgfqpoint{5.216336in}{2.321912in}}%
\pgfpathlineto{\pgfqpoint{5.217025in}{1.959163in}}%
\pgfpathlineto{\pgfqpoint{5.216552in}{2.334510in}}%
\pgfpathlineto{\pgfqpoint{5.217443in}{2.140648in}}%
\pgfpathlineto{\pgfqpoint{5.217537in}{2.332656in}}%
\pgfpathlineto{\pgfqpoint{5.218009in}{1.995525in}}%
\pgfpathlineto{\pgfqpoint{5.218547in}{2.234004in}}%
\pgfpathlineto{\pgfqpoint{5.218923in}{1.662177in}}%
\pgfpathlineto{\pgfqpoint{5.218883in}{2.349278in}}%
\pgfpathlineto{\pgfqpoint{5.219621in}{2.107155in}}%
\pgfpathlineto{\pgfqpoint{5.220345in}{2.321089in}}%
\pgfpathlineto{\pgfqpoint{5.219742in}{1.933291in}}%
\pgfpathlineto{\pgfqpoint{5.220733in}{2.161294in}}%
\pgfpathlineto{\pgfqpoint{5.220813in}{2.355794in}}%
\pgfpathlineto{\pgfqpoint{5.221241in}{1.855164in}}%
\pgfpathlineto{\pgfqpoint{5.221841in}{2.264134in}}%
\pgfpathlineto{\pgfqpoint{5.222521in}{1.815930in}}%
\pgfpathlineto{\pgfqpoint{5.222028in}{2.342954in}}%
\pgfpathlineto{\pgfqpoint{5.222947in}{2.284483in}}%
\pgfpathlineto{\pgfqpoint{5.223532in}{1.860875in}}%
\pgfpathlineto{\pgfqpoint{5.223745in}{2.315559in}}%
\pgfpathlineto{\pgfqpoint{5.224090in}{2.160133in}}%
\pgfpathlineto{\pgfqpoint{5.225005in}{2.340307in}}%
\pgfpathlineto{\pgfqpoint{5.224607in}{1.933143in}}%
\pgfpathlineto{\pgfqpoint{5.225190in}{2.312550in}}%
\pgfpathlineto{\pgfqpoint{5.225640in}{1.929162in}}%
\pgfpathlineto{\pgfqpoint{5.225957in}{2.344372in}}%
\pgfpathlineto{\pgfqpoint{5.226300in}{2.308395in}}%
\pgfpathlineto{\pgfqpoint{5.226723in}{1.891000in}}%
\pgfpathlineto{\pgfqpoint{5.226855in}{2.373335in}}%
\pgfpathlineto{\pgfqpoint{5.227434in}{2.072800in}}%
\pgfpathlineto{\pgfqpoint{5.227908in}{2.361427in}}%
\pgfpathlineto{\pgfqpoint{5.228289in}{1.889280in}}%
\pgfpathlineto{\pgfqpoint{5.228552in}{2.248835in}}%
\pgfpathlineto{\pgfqpoint{5.229640in}{1.969146in}}%
\pgfpathlineto{\pgfqpoint{5.229011in}{2.329844in}}%
\pgfpathlineto{\pgfqpoint{5.229667in}{2.115853in}}%
\pgfpathlineto{\pgfqpoint{5.230648in}{2.356566in}}%
\pgfpathlineto{\pgfqpoint{5.230046in}{1.750817in}}%
\pgfpathlineto{\pgfqpoint{5.230752in}{2.280971in}}%
\pgfpathlineto{\pgfqpoint{5.231418in}{1.814163in}}%
\pgfpathlineto{\pgfqpoint{5.231692in}{2.349118in}}%
\pgfpathlineto{\pgfqpoint{5.231861in}{2.169023in}}%
\pgfpathlineto{\pgfqpoint{5.232343in}{2.375479in}}%
\pgfpathlineto{\pgfqpoint{5.232070in}{1.968422in}}%
\pgfpathlineto{\pgfqpoint{5.232967in}{2.290088in}}%
\pgfpathlineto{\pgfqpoint{5.233993in}{1.734631in}}%
\pgfpathlineto{\pgfqpoint{5.234045in}{2.369160in}}%
\pgfpathlineto{\pgfqpoint{5.234084in}{2.161610in}}%
\pgfpathlineto{\pgfqpoint{5.234977in}{2.376724in}}%
\pgfpathlineto{\pgfqpoint{5.234382in}{1.991495in}}%
\pgfpathlineto{\pgfqpoint{5.235184in}{2.109421in}}%
\pgfpathlineto{\pgfqpoint{5.235197in}{2.107977in}}%
\pgfpathlineto{\pgfqpoint{5.236139in}{2.374836in}}%
\pgfpathlineto{\pgfqpoint{5.235559in}{1.920940in}}%
\pgfpathlineto{\pgfqpoint{5.236307in}{2.195236in}}%
\pgfpathlineto{\pgfqpoint{5.236732in}{1.973720in}}%
\pgfpathlineto{\pgfqpoint{5.237144in}{2.374521in}}%
\pgfpathlineto{\pgfqpoint{5.237222in}{2.275282in}}%
\pgfpathlineto{\pgfqpoint{5.237517in}{2.338450in}}%
\pgfpathlineto{\pgfqpoint{5.237826in}{1.836480in}}%
\pgfpathlineto{\pgfqpoint{5.238288in}{2.214264in}}%
\pgfpathlineto{\pgfqpoint{5.238827in}{2.017579in}}%
\pgfpathlineto{\pgfqpoint{5.238699in}{2.370503in}}%
\pgfpathlineto{\pgfqpoint{5.239390in}{2.165843in}}%
\pgfpathlineto{\pgfqpoint{5.240234in}{2.367866in}}%
\pgfpathlineto{\pgfqpoint{5.240285in}{1.895718in}}%
\pgfpathlineto{\pgfqpoint{5.240503in}{2.276528in}}%
\pgfpathlineto{\pgfqpoint{5.241497in}{1.843771in}}%
\pgfpathlineto{\pgfqpoint{5.240834in}{2.355357in}}%
\pgfpathlineto{\pgfqpoint{5.241612in}{2.262804in}}%
\pgfpathlineto{\pgfqpoint{5.241625in}{2.262749in}}%
\pgfpathlineto{\pgfqpoint{5.241968in}{1.967940in}}%
\pgfpathlineto{\pgfqpoint{5.242261in}{2.365222in}}%
\pgfpathlineto{\pgfqpoint{5.242731in}{2.175169in}}%
\pgfpathlineto{\pgfqpoint{5.243353in}{2.350244in}}%
\pgfpathlineto{\pgfqpoint{5.243239in}{1.845996in}}%
\pgfpathlineto{\pgfqpoint{5.243771in}{2.238991in}}%
\pgfpathlineto{\pgfqpoint{5.244859in}{1.856924in}}%
\pgfpathlineto{\pgfqpoint{5.244227in}{2.370103in}}%
\pgfpathlineto{\pgfqpoint{5.244872in}{2.204742in}}%
\pgfpathlineto{\pgfqpoint{5.245629in}{2.333553in}}%
\pgfpathlineto{\pgfqpoint{5.245188in}{1.874350in}}%
\pgfpathlineto{\pgfqpoint{5.245982in}{2.209788in}}%
\pgfpathlineto{\pgfqpoint{5.246108in}{1.830679in}}%
\pgfpathlineto{\pgfqpoint{5.246121in}{2.361394in}}%
\pgfpathlineto{\pgfqpoint{5.247077in}{2.186596in}}%
\pgfpathlineto{\pgfqpoint{5.247906in}{2.348833in}}%
\pgfpathlineto{\pgfqpoint{5.247718in}{1.936342in}}%
\pgfpathlineto{\pgfqpoint{5.248182in}{2.268407in}}%
\pgfpathlineto{\pgfqpoint{5.248483in}{1.951680in}}%
\pgfpathlineto{\pgfqpoint{5.248220in}{2.363851in}}%
\pgfpathlineto{\pgfqpoint{5.249284in}{2.048280in}}%
\pgfpathlineto{\pgfqpoint{5.250258in}{2.360715in}}%
\pgfpathlineto{\pgfqpoint{5.249422in}{1.930864in}}%
\pgfpathlineto{\pgfqpoint{5.250396in}{2.170019in}}%
\pgfpathlineto{\pgfqpoint{5.251106in}{2.341533in}}%
\pgfpathlineto{\pgfqpoint{5.250558in}{1.867458in}}%
\pgfpathlineto{\pgfqpoint{5.251492in}{2.099633in}}%
\pgfpathlineto{\pgfqpoint{5.251591in}{2.352670in}}%
\pgfpathlineto{\pgfqpoint{5.251678in}{2.012306in}}%
\pgfpathlineto{\pgfqpoint{5.252585in}{2.037188in}}%
\pgfpathlineto{\pgfqpoint{5.252598in}{1.997798in}}%
\pgfpathlineto{\pgfqpoint{5.253168in}{2.373652in}}%
\pgfpathlineto{\pgfqpoint{5.253639in}{2.208212in}}%
\pgfpathlineto{\pgfqpoint{5.254059in}{2.371006in}}%
\pgfpathlineto{\pgfqpoint{5.254480in}{1.836416in}}%
\pgfpathlineto{\pgfqpoint{5.254751in}{2.240494in}}%
\pgfpathlineto{\pgfqpoint{5.255553in}{1.963166in}}%
\pgfpathlineto{\pgfqpoint{5.255072in}{2.389958in}}%
\pgfpathlineto{\pgfqpoint{5.255861in}{2.185484in}}%
\pgfpathlineto{\pgfqpoint{5.256906in}{2.377841in}}%
\pgfpathlineto{\pgfqpoint{5.255922in}{1.917483in}}%
\pgfpathlineto{\pgfqpoint{5.256931in}{2.277955in}}%
\pgfpathlineto{\pgfqpoint{5.258010in}{1.828408in}}%
\pgfpathlineto{\pgfqpoint{5.257115in}{2.381267in}}%
\pgfpathlineto{\pgfqpoint{5.258035in}{2.303333in}}%
\pgfpathlineto{\pgfqpoint{5.258867in}{1.929425in}}%
\pgfpathlineto{\pgfqpoint{5.259001in}{2.363644in}}%
\pgfpathlineto{\pgfqpoint{5.259148in}{2.216050in}}%
\pgfpathlineto{\pgfqpoint{5.259490in}{2.338573in}}%
\pgfpathlineto{\pgfqpoint{5.259588in}{1.966188in}}%
\pgfpathlineto{\pgfqpoint{5.260258in}{2.304878in}}%
\pgfpathlineto{\pgfqpoint{5.260879in}{1.921802in}}%
\pgfpathlineto{\pgfqpoint{5.260819in}{2.334687in}}%
\pgfpathlineto{\pgfqpoint{5.261378in}{2.204943in}}%
\pgfpathlineto{\pgfqpoint{5.262180in}{2.364381in}}%
\pgfpathlineto{\pgfqpoint{5.261973in}{1.581690in}}%
\pgfpathlineto{\pgfqpoint{5.262264in}{2.170742in}}%
\pgfpathlineto{\pgfqpoint{5.262277in}{1.782848in}}%
\pgfpathlineto{\pgfqpoint{5.262749in}{2.332655in}}%
\pgfpathlineto{\pgfqpoint{5.263367in}{2.221888in}}%
\pgfpathlineto{\pgfqpoint{5.264092in}{2.406935in}}%
\pgfpathlineto{\pgfqpoint{5.264104in}{1.759738in}}%
\pgfpathlineto{\pgfqpoint{5.264418in}{2.261859in}}%
\pgfpathlineto{\pgfqpoint{5.264430in}{1.668821in}}%
\pgfpathlineto{\pgfqpoint{5.264864in}{2.368482in}}%
\pgfpathlineto{\pgfqpoint{5.265527in}{2.263232in}}%
\pgfpathlineto{\pgfqpoint{5.265791in}{1.883384in}}%
\pgfpathlineto{\pgfqpoint{5.266068in}{2.385027in}}%
\pgfpathlineto{\pgfqpoint{5.266633in}{2.285611in}}%
\pgfpathlineto{\pgfqpoint{5.266801in}{1.895407in}}%
\pgfpathlineto{\pgfqpoint{5.267184in}{2.342454in}}%
\pgfpathlineto{\pgfqpoint{5.267843in}{2.097645in}}%
\pgfpathlineto{\pgfqpoint{5.268573in}{1.878207in}}%
\pgfpathlineto{\pgfqpoint{5.268967in}{2.352222in}}%
\pgfpathlineto{\pgfqpoint{5.269742in}{1.790284in}}%
\pgfpathlineto{\pgfqpoint{5.269086in}{2.411670in}}%
\pgfpathlineto{\pgfqpoint{5.270088in}{2.128680in}}%
\pgfpathlineto{\pgfqpoint{5.271015in}{2.344225in}}%
\pgfpathlineto{\pgfqpoint{5.270469in}{1.918820in}}%
\pgfpathlineto{\pgfqpoint{5.271194in}{2.341347in}}%
\pgfpathlineto{\pgfqpoint{5.272226in}{1.881828in}}%
\pgfpathlineto{\pgfqpoint{5.271241in}{2.357177in}}%
\pgfpathlineto{\pgfqpoint{5.272309in}{2.163008in}}%
\pgfpathlineto{\pgfqpoint{5.272474in}{2.006053in}}%
\pgfpathlineto{\pgfqpoint{5.273326in}{2.381803in}}%
\pgfpathlineto{\pgfqpoint{5.274270in}{1.604574in}}%
\pgfpathlineto{\pgfqpoint{5.273421in}{2.388007in}}%
\pgfpathlineto{\pgfqpoint{5.274435in}{2.172370in}}%
\pgfpathlineto{\pgfqpoint{5.274518in}{2.330966in}}%
\pgfpathlineto{\pgfqpoint{5.275060in}{1.975747in}}%
\pgfpathlineto{\pgfqpoint{5.275224in}{2.261264in}}%
\pgfpathlineto{\pgfqpoint{5.275248in}{1.747519in}}%
\pgfpathlineto{\pgfqpoint{5.275542in}{2.355492in}}%
\pgfpathlineto{\pgfqpoint{5.276329in}{2.190264in}}%
\pgfpathlineto{\pgfqpoint{5.276352in}{2.308965in}}%
\pgfpathlineto{\pgfqpoint{5.276399in}{2.173005in}}%
\pgfpathlineto{\pgfqpoint{5.276411in}{2.201960in}}%
\pgfpathlineto{\pgfqpoint{5.277208in}{1.913897in}}%
\pgfpathlineto{\pgfqpoint{5.276587in}{2.397768in}}%
\pgfpathlineto{\pgfqpoint{5.277512in}{2.263998in}}%
\pgfpathlineto{\pgfqpoint{5.277571in}{1.766519in}}%
\pgfpathlineto{\pgfqpoint{5.277828in}{2.368427in}}%
\pgfpathlineto{\pgfqpoint{5.278611in}{2.254186in}}%
\pgfpathlineto{\pgfqpoint{5.279240in}{2.378713in}}%
\pgfpathlineto{\pgfqpoint{5.278681in}{1.990544in}}%
\pgfpathlineto{\pgfqpoint{5.279625in}{2.220182in}}%
\pgfpathlineto{\pgfqpoint{5.280206in}{1.879219in}}%
\pgfpathlineto{\pgfqpoint{5.280009in}{2.377147in}}%
\pgfpathlineto{\pgfqpoint{5.280729in}{2.165332in}}%
\pgfpathlineto{\pgfqpoint{5.281761in}{2.391727in}}%
\pgfpathlineto{\pgfqpoint{5.281530in}{1.914090in}}%
\pgfpathlineto{\pgfqpoint{5.281843in}{2.317470in}}%
\pgfpathlineto{\pgfqpoint{5.282213in}{1.938492in}}%
\pgfpathlineto{\pgfqpoint{5.282710in}{2.340386in}}%
\pgfpathlineto{\pgfqpoint{5.282953in}{2.239452in}}%
\pgfpathlineto{\pgfqpoint{5.283945in}{2.012995in}}%
\pgfpathlineto{\pgfqpoint{5.283103in}{2.395308in}}%
\pgfpathlineto{\pgfqpoint{5.284060in}{2.100188in}}%
\pgfpathlineto{\pgfqpoint{5.284878in}{2.370123in}}%
\pgfpathlineto{\pgfqpoint{5.285154in}{2.039505in}}%
\pgfpathlineto{\pgfqpoint{5.285188in}{2.368923in}}%
\pgfpathlineto{\pgfqpoint{5.285464in}{1.912259in}}%
\pgfpathlineto{\pgfqpoint{5.285303in}{2.368977in}}%
\pgfpathlineto{\pgfqpoint{5.286313in}{2.027731in}}%
\pgfpathlineto{\pgfqpoint{5.286462in}{2.343061in}}%
\pgfpathlineto{\pgfqpoint{5.287171in}{2.009681in}}%
\pgfpathlineto{\pgfqpoint{5.287434in}{2.183361in}}%
\pgfpathlineto{\pgfqpoint{5.287743in}{2.381929in}}%
\pgfpathlineto{\pgfqpoint{5.288450in}{1.685453in}}%
\pgfpathlineto{\pgfqpoint{5.288542in}{2.180964in}}%
\pgfpathlineto{\pgfqpoint{5.288792in}{2.384846in}}%
\pgfpathlineto{\pgfqpoint{5.288906in}{1.941203in}}%
\pgfpathlineto{\pgfqpoint{5.289601in}{2.221133in}}%
\pgfpathlineto{\pgfqpoint{5.290441in}{1.800335in}}%
\pgfpathlineto{\pgfqpoint{5.290192in}{2.363247in}}%
\pgfpathlineto{\pgfqpoint{5.290714in}{2.180902in}}%
\pgfpathlineto{\pgfqpoint{5.291224in}{1.902934in}}%
\pgfpathlineto{\pgfqpoint{5.291473in}{2.374614in}}%
\pgfpathlineto{\pgfqpoint{5.291745in}{2.324258in}}%
\pgfpathlineto{\pgfqpoint{5.291869in}{2.379029in}}%
\pgfpathlineto{\pgfqpoint{5.292479in}{1.970642in}}%
\pgfpathlineto{\pgfqpoint{5.292750in}{2.160718in}}%
\pgfpathlineto{\pgfqpoint{5.292829in}{1.807613in}}%
\pgfpathlineto{\pgfqpoint{5.293450in}{2.396361in}}%
\pgfpathlineto{\pgfqpoint{5.293833in}{2.118605in}}%
\pgfpathlineto{\pgfqpoint{5.294137in}{2.365545in}}%
\pgfpathlineto{\pgfqpoint{5.294294in}{1.772539in}}%
\pgfpathlineto{\pgfqpoint{5.294935in}{2.141824in}}%
\pgfpathlineto{\pgfqpoint{5.294946in}{1.862565in}}%
\pgfpathlineto{\pgfqpoint{5.295384in}{2.389826in}}%
\pgfpathlineto{\pgfqpoint{5.296034in}{2.189636in}}%
\pgfpathlineto{\pgfqpoint{5.296728in}{2.387791in}}%
\pgfpathlineto{\pgfqpoint{5.296962in}{1.988563in}}%
\pgfpathlineto{\pgfqpoint{5.297041in}{2.248256in}}%
\pgfpathlineto{\pgfqpoint{5.297577in}{1.728125in}}%
\pgfpathlineto{\pgfqpoint{5.297733in}{2.369318in}}%
\pgfpathlineto{\pgfqpoint{5.298145in}{2.265721in}}%
\pgfpathlineto{\pgfqpoint{5.298658in}{2.367335in}}%
\pgfpathlineto{\pgfqpoint{5.299081in}{1.997991in}}%
\pgfpathlineto{\pgfqpoint{5.299236in}{2.229550in}}%
\pgfpathlineto{\pgfqpoint{5.299547in}{1.913770in}}%
\pgfpathlineto{\pgfqpoint{5.299836in}{2.356661in}}%
\pgfpathlineto{\pgfqpoint{5.300335in}{2.179108in}}%
\pgfpathlineto{\pgfqpoint{5.301244in}{2.375793in}}%
\pgfpathlineto{\pgfqpoint{5.300845in}{1.950634in}}%
\pgfpathlineto{\pgfqpoint{5.301443in}{2.220940in}}%
\pgfpathlineto{\pgfqpoint{5.301730in}{1.948281in}}%
\pgfpathlineto{\pgfqpoint{5.302415in}{2.387611in}}%
\pgfpathlineto{\pgfqpoint{5.302525in}{2.312115in}}%
\pgfpathlineto{\pgfqpoint{5.303054in}{2.330473in}}%
\pgfpathlineto{\pgfqpoint{5.302547in}{2.022686in}}%
\pgfpathlineto{\pgfqpoint{5.303109in}{2.304506in}}%
\pgfpathlineto{\pgfqpoint{5.303957in}{1.775586in}}%
\pgfpathlineto{\pgfqpoint{5.303594in}{2.362306in}}%
\pgfpathlineto{\pgfqpoint{5.304221in}{2.231708in}}%
\pgfpathlineto{\pgfqpoint{5.304737in}{2.328323in}}%
\pgfpathlineto{\pgfqpoint{5.304989in}{1.914680in}}%
\pgfpathlineto{\pgfqpoint{5.305296in}{2.287148in}}%
\pgfpathlineto{\pgfqpoint{5.305329in}{1.952190in}}%
\pgfpathlineto{\pgfqpoint{5.305471in}{2.394593in}}%
\pgfpathlineto{\pgfqpoint{5.306401in}{2.309858in}}%
\pgfpathlineto{\pgfqpoint{5.307133in}{1.935155in}}%
\pgfpathlineto{\pgfqpoint{5.306959in}{2.358955in}}%
\pgfpathlineto{\pgfqpoint{5.307504in}{2.284661in}}%
\pgfpathlineto{\pgfqpoint{5.308495in}{2.354744in}}%
\pgfpathlineto{\pgfqpoint{5.308299in}{1.757153in}}%
\pgfpathlineto{\pgfqpoint{5.308528in}{2.251873in}}%
\pgfpathlineto{\pgfqpoint{5.309201in}{1.782345in}}%
\pgfpathlineto{\pgfqpoint{5.308615in}{2.364754in}}%
\pgfpathlineto{\pgfqpoint{5.309635in}{2.296421in}}%
\pgfpathlineto{\pgfqpoint{5.310156in}{2.351540in}}%
\pgfpathlineto{\pgfqpoint{5.310037in}{1.885889in}}%
\pgfpathlineto{\pgfqpoint{5.310459in}{2.177220in}}%
\pgfpathlineto{\pgfqpoint{5.311346in}{1.857695in}}%
\pgfpathlineto{\pgfqpoint{5.310924in}{2.393689in}}%
\pgfpathlineto{\pgfqpoint{5.311562in}{2.241658in}}%
\pgfpathlineto{\pgfqpoint{5.312037in}{1.888138in}}%
\pgfpathlineto{\pgfqpoint{5.311832in}{2.387352in}}%
\pgfpathlineto{\pgfqpoint{5.312684in}{1.894136in}}%
\pgfpathlineto{\pgfqpoint{5.313630in}{2.412730in}}%
\pgfpathlineto{\pgfqpoint{5.312974in}{1.814204in}}%
\pgfpathlineto{\pgfqpoint{5.313802in}{2.268981in}}%
\pgfpathlineto{\pgfqpoint{5.314199in}{1.844310in}}%
\pgfpathlineto{\pgfqpoint{5.314167in}{2.380520in}}%
\pgfpathlineto{\pgfqpoint{5.314907in}{2.236297in}}%
\pgfpathlineto{\pgfqpoint{5.315421in}{2.389663in}}%
\pgfpathlineto{\pgfqpoint{5.315207in}{1.881714in}}%
\pgfpathlineto{\pgfqpoint{5.315988in}{2.248660in}}%
\pgfpathlineto{\pgfqpoint{5.317076in}{1.659354in}}%
\pgfpathlineto{\pgfqpoint{5.316394in}{2.387151in}}%
\pgfpathlineto{\pgfqpoint{5.317098in}{2.234770in}}%
\pgfpathlineto{\pgfqpoint{5.317790in}{1.983730in}}%
\pgfpathlineto{\pgfqpoint{5.317449in}{2.407324in}}%
\pgfpathlineto{\pgfqpoint{5.318194in}{2.249301in}}%
\pgfpathlineto{\pgfqpoint{5.319245in}{2.367419in}}%
\pgfpathlineto{\pgfqpoint{5.318534in}{1.900583in}}%
\pgfpathlineto{\pgfqpoint{5.319298in}{2.290850in}}%
\pgfpathlineto{\pgfqpoint{5.320072in}{1.937449in}}%
\pgfpathlineto{\pgfqpoint{5.320326in}{2.361199in}}%
\pgfpathlineto{\pgfqpoint{5.320410in}{2.119338in}}%
\pgfpathlineto{\pgfqpoint{5.320833in}{2.375263in}}%
\pgfpathlineto{\pgfqpoint{5.321308in}{1.892976in}}%
\pgfpathlineto{\pgfqpoint{5.321519in}{2.308641in}}%
\pgfpathlineto{\pgfqpoint{5.322604in}{1.872643in}}%
\pgfpathlineto{\pgfqpoint{5.322510in}{2.355851in}}%
\pgfpathlineto{\pgfqpoint{5.322636in}{2.060716in}}%
\pgfpathlineto{\pgfqpoint{5.322941in}{2.388893in}}%
\pgfpathlineto{\pgfqpoint{5.323603in}{1.835720in}}%
\pgfpathlineto{\pgfqpoint{5.323750in}{2.326634in}}%
\pgfpathlineto{\pgfqpoint{5.324860in}{1.927491in}}%
\pgfpathlineto{\pgfqpoint{5.324410in}{2.349997in}}%
\pgfpathlineto{\pgfqpoint{5.324871in}{2.179180in}}%
\pgfpathlineto{\pgfqpoint{5.325248in}{2.357026in}}%
\pgfpathlineto{\pgfqpoint{5.325728in}{1.788653in}}%
\pgfpathlineto{\pgfqpoint{5.325989in}{2.283427in}}%
\pgfpathlineto{\pgfqpoint{5.326573in}{1.508579in}}%
\pgfpathlineto{\pgfqpoint{5.326448in}{2.350723in}}%
\pgfpathlineto{\pgfqpoint{5.327105in}{2.199003in}}%
\pgfpathlineto{\pgfqpoint{5.328051in}{1.980859in}}%
\pgfpathlineto{\pgfqpoint{5.327219in}{2.372559in}}%
\pgfpathlineto{\pgfqpoint{5.328082in}{2.239343in}}%
\pgfpathlineto{\pgfqpoint{5.328954in}{2.386167in}}%
\pgfpathlineto{\pgfqpoint{5.328746in}{1.816411in}}%
\pgfpathlineto{\pgfqpoint{5.329182in}{2.330523in}}%
\pgfpathlineto{\pgfqpoint{5.329202in}{1.930915in}}%
\pgfpathlineto{\pgfqpoint{5.329368in}{2.375908in}}%
\pgfpathlineto{\pgfqpoint{5.330289in}{2.249308in}}%
\pgfpathlineto{\pgfqpoint{5.330320in}{2.336377in}}%
\pgfpathlineto{\pgfqpoint{5.330382in}{2.170961in}}%
\pgfpathlineto{\pgfqpoint{5.330516in}{1.771484in}}%
\pgfpathlineto{\pgfqpoint{5.330898in}{2.390073in}}%
\pgfpathlineto{\pgfqpoint{5.331475in}{2.317607in}}%
\pgfpathlineto{\pgfqpoint{5.331939in}{2.384536in}}%
\pgfpathlineto{\pgfqpoint{5.332124in}{1.900582in}}%
\pgfpathlineto{\pgfqpoint{5.332566in}{2.328985in}}%
\pgfpathlineto{\pgfqpoint{5.332761in}{1.812112in}}%
\pgfpathlineto{\pgfqpoint{5.333223in}{2.366592in}}%
\pgfpathlineto{\pgfqpoint{5.333695in}{2.127994in}}%
\pgfpathlineto{\pgfqpoint{5.334279in}{2.362206in}}%
\pgfpathlineto{\pgfqpoint{5.334258in}{1.881144in}}%
\pgfpathlineto{\pgfqpoint{5.334739in}{2.197757in}}%
\pgfpathlineto{\pgfqpoint{5.334749in}{1.666036in}}%
\pgfpathlineto{\pgfqpoint{5.335403in}{2.366655in}}%
\pgfpathlineto{\pgfqpoint{5.335842in}{2.155583in}}%
\pgfpathlineto{\pgfqpoint{5.336413in}{2.360775in}}%
\pgfpathlineto{\pgfqpoint{5.336698in}{1.823596in}}%
\pgfpathlineto{\pgfqpoint{5.336932in}{2.148108in}}%
\pgfpathlineto{\pgfqpoint{5.337978in}{1.971604in}}%
\pgfpathlineto{\pgfqpoint{5.337349in}{2.356406in}}%
\pgfpathlineto{\pgfqpoint{5.338009in}{2.298571in}}%
\pgfpathlineto{\pgfqpoint{5.338171in}{2.383513in}}%
\pgfpathlineto{\pgfqpoint{5.338293in}{1.924232in}}%
\pgfpathlineto{\pgfqpoint{5.338981in}{2.160578in}}%
\pgfpathlineto{\pgfqpoint{5.339356in}{1.814684in}}%
\pgfpathlineto{\pgfqpoint{5.339143in}{2.377470in}}%
\pgfpathlineto{\pgfqpoint{5.340053in}{2.262049in}}%
\pgfpathlineto{\pgfqpoint{5.340588in}{2.357128in}}%
\pgfpathlineto{\pgfqpoint{5.340144in}{1.933016in}}%
\pgfpathlineto{\pgfqpoint{5.341062in}{2.269538in}}%
\pgfpathlineto{\pgfqpoint{5.341112in}{1.979928in}}%
\pgfpathlineto{\pgfqpoint{5.342018in}{2.396219in}}%
\pgfpathlineto{\pgfqpoint{5.342168in}{2.193210in}}%
\pgfpathlineto{\pgfqpoint{5.342971in}{2.396414in}}%
\pgfpathlineto{\pgfqpoint{5.343021in}{1.901055in}}%
\pgfpathlineto{\pgfqpoint{5.343272in}{2.206767in}}%
\pgfpathlineto{\pgfqpoint{5.343943in}{1.759255in}}%
\pgfpathlineto{\pgfqpoint{5.343843in}{2.401661in}}%
\pgfpathlineto{\pgfqpoint{5.344383in}{2.116703in}}%
\pgfpathlineto{\pgfqpoint{5.345152in}{2.373537in}}%
\pgfpathlineto{\pgfqpoint{5.345162in}{1.852862in}}%
\pgfpathlineto{\pgfqpoint{5.345481in}{2.192975in}}%
\pgfpathlineto{\pgfqpoint{5.345591in}{1.915617in}}%
\pgfpathlineto{\pgfqpoint{5.346327in}{2.389841in}}%
\pgfpathlineto{\pgfqpoint{5.346586in}{2.216807in}}%
\pgfpathlineto{\pgfqpoint{5.346596in}{2.379567in}}%
\pgfpathlineto{\pgfqpoint{5.347291in}{1.848379in}}%
\pgfpathlineto{\pgfqpoint{5.347698in}{2.294802in}}%
\pgfpathlineto{\pgfqpoint{5.348362in}{1.916957in}}%
\pgfpathlineto{\pgfqpoint{5.348006in}{2.371218in}}%
\pgfpathlineto{\pgfqpoint{5.348808in}{2.292410in}}%
\pgfpathlineto{\pgfqpoint{5.349242in}{2.385587in}}%
\pgfpathlineto{\pgfqpoint{5.349578in}{1.837523in}}%
\pgfpathlineto{\pgfqpoint{5.349914in}{2.259297in}}%
\pgfpathlineto{\pgfqpoint{5.350180in}{1.870923in}}%
\pgfpathlineto{\pgfqpoint{5.350072in}{2.358047in}}%
\pgfpathlineto{\pgfqpoint{5.351027in}{1.886873in}}%
\pgfpathlineto{\pgfqpoint{5.351794in}{2.370460in}}%
\pgfpathlineto{\pgfqpoint{5.351499in}{1.814774in}}%
\pgfpathlineto{\pgfqpoint{5.352147in}{2.297661in}}%
\pgfpathlineto{\pgfqpoint{5.353089in}{2.379405in}}%
\pgfpathlineto{\pgfqpoint{5.352736in}{1.961606in}}%
\pgfpathlineto{\pgfqpoint{5.353216in}{2.263174in}}%
\pgfpathlineto{\pgfqpoint{5.354096in}{1.878515in}}%
\pgfpathlineto{\pgfqpoint{5.353666in}{2.374546in}}%
\pgfpathlineto{\pgfqpoint{5.354321in}{2.278440in}}%
\pgfpathlineto{\pgfqpoint{5.354857in}{2.366073in}}%
\pgfpathlineto{\pgfqpoint{5.354604in}{1.939804in}}%
\pgfpathlineto{\pgfqpoint{5.355208in}{2.084262in}}%
\pgfpathlineto{\pgfqpoint{5.356259in}{1.865733in}}%
\pgfpathlineto{\pgfqpoint{5.355237in}{2.396445in}}%
\pgfpathlineto{\pgfqpoint{5.356308in}{2.225069in}}%
\pgfpathlineto{\pgfqpoint{5.356385in}{2.376974in}}%
\pgfpathlineto{\pgfqpoint{5.357201in}{1.757215in}}%
\pgfpathlineto{\pgfqpoint{5.357395in}{2.347849in}}%
\pgfpathlineto{\pgfqpoint{5.357521in}{1.952270in}}%
\pgfpathlineto{\pgfqpoint{5.357879in}{2.382062in}}%
\pgfpathlineto{\pgfqpoint{5.358508in}{2.076414in}}%
\pgfpathlineto{\pgfqpoint{5.358885in}{2.394253in}}%
\pgfpathlineto{\pgfqpoint{5.358749in}{1.874328in}}%
\pgfpathlineto{\pgfqpoint{5.359618in}{2.218637in}}%
\pgfpathlineto{\pgfqpoint{5.360350in}{2.370541in}}%
\pgfpathlineto{\pgfqpoint{5.360456in}{1.936427in}}%
\pgfpathlineto{\pgfqpoint{5.360658in}{2.348274in}}%
\pgfpathlineto{\pgfqpoint{5.360774in}{1.854543in}}%
\pgfpathlineto{\pgfqpoint{5.361052in}{2.374451in}}%
\pgfpathlineto{\pgfqpoint{5.361773in}{2.208583in}}%
\pgfpathlineto{\pgfqpoint{5.362396in}{1.897000in}}%
\pgfpathlineto{\pgfqpoint{5.362424in}{2.374117in}}%
\pgfpathlineto{\pgfqpoint{5.362827in}{2.228692in}}%
\pgfpathlineto{\pgfqpoint{5.363037in}{2.379846in}}%
\pgfpathlineto{\pgfqpoint{5.363314in}{1.964363in}}%
\pgfpathlineto{\pgfqpoint{5.363945in}{2.306848in}}%
\pgfpathlineto{\pgfqpoint{5.364059in}{1.835262in}}%
\pgfpathlineto{\pgfqpoint{5.365031in}{2.375756in}}%
\pgfpathlineto{\pgfqpoint{5.365051in}{2.292282in}}%
\pgfpathlineto{\pgfqpoint{5.365678in}{2.391988in}}%
\pgfpathlineto{\pgfqpoint{5.365640in}{1.906893in}}%
\pgfpathlineto{\pgfqpoint{5.366144in}{2.330928in}}%
\pgfpathlineto{\pgfqpoint{5.366448in}{1.742442in}}%
\pgfpathlineto{\pgfqpoint{5.367064in}{2.402827in}}%
\pgfpathlineto{\pgfqpoint{5.367253in}{2.231377in}}%
\pgfpathlineto{\pgfqpoint{5.367906in}{2.364689in}}%
\pgfpathlineto{\pgfqpoint{5.367944in}{1.938641in}}%
\pgfpathlineto{\pgfqpoint{5.368379in}{2.355947in}}%
\pgfpathlineto{\pgfqpoint{5.368879in}{1.898246in}}%
\pgfpathlineto{\pgfqpoint{5.369200in}{2.371789in}}%
\pgfpathlineto{\pgfqpoint{5.369492in}{2.183009in}}%
\pgfpathlineto{\pgfqpoint{5.370245in}{2.374755in}}%
\pgfpathlineto{\pgfqpoint{5.370179in}{1.998639in}}%
\pgfpathlineto{\pgfqpoint{5.370414in}{2.300930in}}%
\pgfpathlineto{\pgfqpoint{5.370424in}{1.933291in}}%
\pgfpathlineto{\pgfqpoint{5.370433in}{2.371461in}}%
\pgfpathlineto{\pgfqpoint{5.371522in}{2.261929in}}%
\pgfpathlineto{\pgfqpoint{5.371962in}{1.979574in}}%
\pgfpathlineto{\pgfqpoint{5.371588in}{2.407600in}}%
\pgfpathlineto{\pgfqpoint{5.372627in}{2.225104in}}%
\pgfpathlineto{\pgfqpoint{5.373430in}{2.403257in}}%
\pgfpathlineto{\pgfqpoint{5.373570in}{1.710019in}}%
\pgfpathlineto{\pgfqpoint{5.373738in}{2.362391in}}%
\pgfpathlineto{\pgfqpoint{5.374707in}{1.739075in}}%
\pgfpathlineto{\pgfqpoint{5.373906in}{2.380020in}}%
\pgfpathlineto{\pgfqpoint{5.374847in}{2.227906in}}%
\pgfpathlineto{\pgfqpoint{5.375135in}{2.392107in}}%
\pgfpathlineto{\pgfqpoint{5.375581in}{1.798580in}}%
\pgfpathlineto{\pgfqpoint{5.375934in}{2.225336in}}%
\pgfpathlineto{\pgfqpoint{5.375952in}{1.827048in}}%
\pgfpathlineto{\pgfqpoint{5.376509in}{2.400884in}}%
\pgfpathlineto{\pgfqpoint{5.377046in}{2.145236in}}%
\pgfpathlineto{\pgfqpoint{5.377480in}{2.397316in}}%
\pgfpathlineto{\pgfqpoint{5.377776in}{1.868424in}}%
\pgfpathlineto{\pgfqpoint{5.378155in}{2.167312in}}%
\pgfpathlineto{\pgfqpoint{5.379021in}{2.395796in}}%
\pgfpathlineto{\pgfqpoint{5.378588in}{1.785055in}}%
\pgfpathlineto{\pgfqpoint{5.379261in}{2.296868in}}%
\pgfpathlineto{\pgfqpoint{5.379537in}{1.819500in}}%
\pgfpathlineto{\pgfqpoint{5.379969in}{2.433557in}}%
\pgfpathlineto{\pgfqpoint{5.380364in}{2.214776in}}%
\pgfpathlineto{\pgfqpoint{5.381025in}{2.369640in}}%
\pgfpathlineto{\pgfqpoint{5.381373in}{1.771969in}}%
\pgfpathlineto{\pgfqpoint{5.381474in}{2.241458in}}%
\pgfpathlineto{\pgfqpoint{5.382544in}{1.889182in}}%
\pgfpathlineto{\pgfqpoint{5.382370in}{2.387476in}}%
\pgfpathlineto{\pgfqpoint{5.382581in}{2.233439in}}%
\pgfpathlineto{\pgfqpoint{5.382864in}{2.365292in}}%
\pgfpathlineto{\pgfqpoint{5.382891in}{1.756182in}}%
\pgfpathlineto{\pgfqpoint{5.383675in}{2.183377in}}%
\pgfpathlineto{\pgfqpoint{5.384240in}{2.002853in}}%
\pgfpathlineto{\pgfqpoint{5.383794in}{2.389231in}}%
\pgfpathlineto{\pgfqpoint{5.384649in}{2.230060in}}%
\pgfpathlineto{\pgfqpoint{5.385403in}{2.425802in}}%
\pgfpathlineto{\pgfqpoint{5.385593in}{1.801310in}}%
\pgfpathlineto{\pgfqpoint{5.385748in}{2.227928in}}%
\pgfpathlineto{\pgfqpoint{5.385757in}{1.989319in}}%
\pgfpathlineto{\pgfqpoint{5.386427in}{2.370113in}}%
\pgfpathlineto{\pgfqpoint{5.386852in}{2.078275in}}%
\pgfpathlineto{\pgfqpoint{5.387313in}{2.386556in}}%
\pgfpathlineto{\pgfqpoint{5.387394in}{1.854936in}}%
\pgfpathlineto{\pgfqpoint{5.387972in}{2.323408in}}%
\pgfpathlineto{\pgfqpoint{5.389044in}{1.925519in}}%
\pgfpathlineto{\pgfqpoint{5.388909in}{2.421407in}}%
\pgfpathlineto{\pgfqpoint{5.389080in}{2.123127in}}%
\pgfpathlineto{\pgfqpoint{5.389557in}{2.414236in}}%
\pgfpathlineto{\pgfqpoint{5.389305in}{1.807856in}}%
\pgfpathlineto{\pgfqpoint{5.390185in}{2.381124in}}%
\pgfpathlineto{\pgfqpoint{5.391189in}{1.917223in}}%
\pgfpathlineto{\pgfqpoint{5.390454in}{2.387780in}}%
\pgfpathlineto{\pgfqpoint{5.391296in}{2.316129in}}%
\pgfpathlineto{\pgfqpoint{5.391681in}{1.958321in}}%
\pgfpathlineto{\pgfqpoint{5.391815in}{2.380530in}}%
\pgfpathlineto{\pgfqpoint{5.392404in}{2.198971in}}%
\pgfpathlineto{\pgfqpoint{5.392815in}{2.380671in}}%
\pgfpathlineto{\pgfqpoint{5.392824in}{1.827744in}}%
\pgfpathlineto{\pgfqpoint{5.393519in}{2.299483in}}%
\pgfpathlineto{\pgfqpoint{5.394346in}{1.927137in}}%
\pgfpathlineto{\pgfqpoint{5.394043in}{2.398242in}}%
\pgfpathlineto{\pgfqpoint{5.394665in}{2.211128in}}%
\pgfpathlineto{\pgfqpoint{5.394701in}{2.368030in}}%
\pgfpathlineto{\pgfqpoint{5.395490in}{1.941954in}}%
\pgfpathlineto{\pgfqpoint{5.395765in}{2.266847in}}%
\pgfpathlineto{\pgfqpoint{5.396101in}{1.896722in}}%
\pgfpathlineto{\pgfqpoint{5.396340in}{2.412931in}}%
\pgfpathlineto{\pgfqpoint{5.396879in}{2.102366in}}%
\pgfpathlineto{\pgfqpoint{5.397682in}{2.375104in}}%
\pgfpathlineto{\pgfqpoint{5.397761in}{1.960385in}}%
\pgfpathlineto{\pgfqpoint{5.397990in}{2.236865in}}%
\pgfpathlineto{\pgfqpoint{5.398659in}{1.640592in}}%
\pgfpathlineto{\pgfqpoint{5.398299in}{2.385500in}}%
\pgfpathlineto{\pgfqpoint{5.399037in}{2.154680in}}%
\pgfpathlineto{\pgfqpoint{5.399775in}{2.392359in}}%
\pgfpathlineto{\pgfqpoint{5.399134in}{1.754281in}}%
\pgfpathlineto{\pgfqpoint{5.400143in}{2.273861in}}%
\pgfpathlineto{\pgfqpoint{5.400756in}{1.875351in}}%
\pgfpathlineto{\pgfqpoint{5.400537in}{2.401970in}}%
\pgfpathlineto{\pgfqpoint{5.401246in}{2.118179in}}%
\pgfpathlineto{\pgfqpoint{5.402241in}{2.387052in}}%
\pgfpathlineto{\pgfqpoint{5.401499in}{2.014994in}}%
\pgfpathlineto{\pgfqpoint{5.402363in}{2.328699in}}%
\pgfpathlineto{\pgfqpoint{5.403104in}{1.866406in}}%
\pgfpathlineto{\pgfqpoint{5.403156in}{2.369873in}}%
\pgfpathlineto{\pgfqpoint{5.403478in}{2.109413in}}%
\pgfpathlineto{\pgfqpoint{5.403869in}{2.388071in}}%
\pgfpathlineto{\pgfqpoint{5.404251in}{1.697165in}}%
\pgfpathlineto{\pgfqpoint{5.404590in}{2.123543in}}%
\pgfpathlineto{\pgfqpoint{5.405196in}{2.397737in}}%
\pgfpathlineto{\pgfqpoint{5.404754in}{1.819600in}}%
\pgfpathlineto{\pgfqpoint{5.405707in}{2.254439in}}%
\pgfpathlineto{\pgfqpoint{5.405767in}{2.368015in}}%
\pgfpathlineto{\pgfqpoint{5.406787in}{1.988783in}}%
\pgfpathlineto{\pgfqpoint{5.406795in}{1.966654in}}%
\pgfpathlineto{\pgfqpoint{5.406864in}{2.419304in}}%
\pgfpathlineto{\pgfqpoint{5.407778in}{2.181003in}}%
\pgfpathlineto{\pgfqpoint{5.408706in}{2.425520in}}%
\pgfpathlineto{\pgfqpoint{5.408715in}{1.925557in}}%
\pgfpathlineto{\pgfqpoint{5.408895in}{2.323575in}}%
\pgfpathlineto{\pgfqpoint{5.409376in}{1.819063in}}%
\pgfpathlineto{\pgfqpoint{5.409993in}{2.388533in}}%
\pgfpathlineto{\pgfqpoint{5.410019in}{2.008837in}}%
\pgfpathlineto{\pgfqpoint{5.410233in}{2.367925in}}%
\pgfpathlineto{\pgfqpoint{5.410207in}{1.964368in}}%
\pgfpathlineto{\pgfqpoint{5.411139in}{2.211780in}}%
\pgfpathlineto{\pgfqpoint{5.411147in}{2.210322in}}%
\pgfpathlineto{\pgfqpoint{5.411156in}{2.257799in}}%
\pgfpathlineto{\pgfqpoint{5.411164in}{1.944338in}}%
\pgfpathlineto{\pgfqpoint{5.411327in}{2.399982in}}%
\pgfpathlineto{\pgfqpoint{5.412265in}{2.108296in}}%
\pgfpathlineto{\pgfqpoint{5.412324in}{2.390920in}}%
\pgfpathlineto{\pgfqpoint{5.412375in}{1.966104in}}%
\pgfpathlineto{\pgfqpoint{5.413413in}{2.192413in}}%
\pgfpathlineto{\pgfqpoint{5.413965in}{1.996763in}}%
\pgfpathlineto{\pgfqpoint{5.413931in}{2.373101in}}%
\pgfpathlineto{\pgfqpoint{5.414516in}{2.285726in}}%
\pgfpathlineto{\pgfqpoint{5.415430in}{2.397003in}}%
\pgfpathlineto{\pgfqpoint{5.415015in}{1.789124in}}%
\pgfpathlineto{\pgfqpoint{5.415573in}{2.294602in}}%
\pgfpathlineto{\pgfqpoint{5.416257in}{2.000751in}}%
\pgfpathlineto{\pgfqpoint{5.415607in}{2.434536in}}%
\pgfpathlineto{\pgfqpoint{5.416679in}{2.188318in}}%
\pgfpathlineto{\pgfqpoint{5.416789in}{2.377613in}}%
\pgfpathlineto{\pgfqpoint{5.417151in}{1.892236in}}%
\pgfpathlineto{\pgfqpoint{5.417790in}{2.289844in}}%
\pgfpathlineto{\pgfqpoint{5.418395in}{1.931058in}}%
\pgfpathlineto{\pgfqpoint{5.418697in}{2.382525in}}%
\pgfpathlineto{\pgfqpoint{5.418899in}{2.168291in}}%
\pgfpathlineto{\pgfqpoint{5.419033in}{2.368237in}}%
\pgfpathlineto{\pgfqpoint{5.419083in}{1.880950in}}%
\pgfpathlineto{\pgfqpoint{5.420012in}{2.268276in}}%
\pgfpathlineto{\pgfqpoint{5.420965in}{1.921905in}}%
\pgfpathlineto{\pgfqpoint{5.421023in}{2.380315in}}%
\pgfpathlineto{\pgfqpoint{5.421115in}{2.226422in}}%
\pgfpathlineto{\pgfqpoint{5.421407in}{2.407709in}}%
\pgfpathlineto{\pgfqpoint{5.422115in}{1.946537in}}%
\pgfpathlineto{\pgfqpoint{5.422223in}{2.251867in}}%
\pgfpathlineto{\pgfqpoint{5.422763in}{2.415645in}}%
\pgfpathlineto{\pgfqpoint{5.423071in}{1.858962in}}%
\pgfpathlineto{\pgfqpoint{5.423320in}{2.320080in}}%
\pgfpathlineto{\pgfqpoint{5.423850in}{1.996914in}}%
\pgfpathlineto{\pgfqpoint{5.423710in}{2.419867in}}%
\pgfpathlineto{\pgfqpoint{5.424430in}{2.329260in}}%
\pgfpathlineto{\pgfqpoint{5.425497in}{1.753080in}}%
\pgfpathlineto{\pgfqpoint{5.424935in}{2.388153in}}%
\pgfpathlineto{\pgfqpoint{5.425546in}{2.182821in}}%
\pgfpathlineto{\pgfqpoint{5.426074in}{2.398290in}}%
\pgfpathlineto{\pgfqpoint{5.426008in}{1.855952in}}%
\pgfpathlineto{\pgfqpoint{5.426643in}{2.234190in}}%
\pgfpathlineto{\pgfqpoint{5.427013in}{1.762920in}}%
\pgfpathlineto{\pgfqpoint{5.427588in}{2.402634in}}%
\pgfpathlineto{\pgfqpoint{5.427753in}{2.233886in}}%
\pgfpathlineto{\pgfqpoint{5.427851in}{2.396404in}}%
\pgfpathlineto{\pgfqpoint{5.427958in}{2.024241in}}%
\pgfpathlineto{\pgfqpoint{5.428770in}{2.370483in}}%
\pgfpathlineto{\pgfqpoint{5.428999in}{1.933493in}}%
\pgfpathlineto{\pgfqpoint{5.429776in}{2.396135in}}%
\pgfpathlineto{\pgfqpoint{5.429874in}{2.151641in}}%
\pgfpathlineto{\pgfqpoint{5.430609in}{2.396617in}}%
\pgfpathlineto{\pgfqpoint{5.430927in}{1.823078in}}%
\pgfpathlineto{\pgfqpoint{5.430984in}{2.303095in}}%
\pgfpathlineto{\pgfqpoint{5.431758in}{1.874887in}}%
\pgfpathlineto{\pgfqpoint{5.431937in}{2.380362in}}%
\pgfpathlineto{\pgfqpoint{5.432099in}{2.291977in}}%
\pgfpathlineto{\pgfqpoint{5.432846in}{1.788319in}}%
\pgfpathlineto{\pgfqpoint{5.432481in}{2.418726in}}%
\pgfpathlineto{\pgfqpoint{5.433179in}{2.153929in}}%
\pgfpathlineto{\pgfqpoint{5.433187in}{2.431657in}}%
\pgfpathlineto{\pgfqpoint{5.434062in}{1.948351in}}%
\pgfpathlineto{\pgfqpoint{5.434288in}{2.378489in}}%
\pgfpathlineto{\pgfqpoint{5.434490in}{1.867710in}}%
\pgfpathlineto{\pgfqpoint{5.435064in}{2.379023in}}%
\pgfpathlineto{\pgfqpoint{5.435403in}{2.128021in}}%
\pgfpathlineto{\pgfqpoint{5.435661in}{2.398988in}}%
\pgfpathlineto{\pgfqpoint{5.435653in}{1.672016in}}%
\pgfpathlineto{\pgfqpoint{5.436514in}{2.291823in}}%
\pgfpathlineto{\pgfqpoint{5.436546in}{1.791970in}}%
\pgfpathlineto{\pgfqpoint{5.437198in}{2.425096in}}%
\pgfpathlineto{\pgfqpoint{5.437615in}{2.051840in}}%
\pgfpathlineto{\pgfqpoint{5.438497in}{2.414086in}}%
\pgfpathlineto{\pgfqpoint{5.438192in}{1.883834in}}%
\pgfpathlineto{\pgfqpoint{5.438729in}{2.307511in}}%
\pgfpathlineto{\pgfqpoint{5.439257in}{1.851768in}}%
\pgfpathlineto{\pgfqpoint{5.439728in}{2.372611in}}%
\pgfpathlineto{\pgfqpoint{5.439864in}{2.145956in}}%
\pgfpathlineto{\pgfqpoint{5.439872in}{2.389875in}}%
\pgfpathlineto{\pgfqpoint{5.440398in}{1.972049in}}%
\pgfpathlineto{\pgfqpoint{5.440979in}{2.334975in}}%
\pgfpathlineto{\pgfqpoint{5.441178in}{1.901557in}}%
\pgfpathlineto{\pgfqpoint{5.441019in}{2.395585in}}%
\pgfpathlineto{\pgfqpoint{5.442092in}{2.284846in}}%
\pgfpathlineto{\pgfqpoint{5.443131in}{1.935359in}}%
\pgfpathlineto{\pgfqpoint{5.442434in}{2.395909in}}%
\pgfpathlineto{\pgfqpoint{5.443202in}{1.999717in}}%
\pgfpathlineto{\pgfqpoint{5.443614in}{2.400782in}}%
\pgfpathlineto{\pgfqpoint{5.443440in}{1.980844in}}%
\pgfpathlineto{\pgfqpoint{5.444310in}{2.211345in}}%
\pgfpathlineto{\pgfqpoint{5.445209in}{1.809550in}}%
\pgfpathlineto{\pgfqpoint{5.444499in}{2.410642in}}%
\pgfpathlineto{\pgfqpoint{5.445406in}{2.264129in}}%
\pgfpathlineto{\pgfqpoint{5.446012in}{2.403594in}}%
\pgfpathlineto{\pgfqpoint{5.446138in}{1.772718in}}%
\pgfpathlineto{\pgfqpoint{5.446515in}{2.286602in}}%
\pgfpathlineto{\pgfqpoint{5.446979in}{1.888051in}}%
\pgfpathlineto{\pgfqpoint{5.447481in}{2.403760in}}%
\pgfpathlineto{\pgfqpoint{5.447637in}{1.932022in}}%
\pgfpathlineto{\pgfqpoint{5.448264in}{2.415863in}}%
\pgfpathlineto{\pgfqpoint{5.447849in}{1.849089in}}%
\pgfpathlineto{\pgfqpoint{5.448749in}{2.294165in}}%
\pgfpathlineto{\pgfqpoint{5.449155in}{1.877719in}}%
\pgfpathlineto{\pgfqpoint{5.449264in}{2.420072in}}%
\pgfpathlineto{\pgfqpoint{5.449857in}{2.296813in}}%
\pgfpathlineto{\pgfqpoint{5.450939in}{1.884588in}}%
\pgfpathlineto{\pgfqpoint{5.450651in}{2.387698in}}%
\pgfpathlineto{\pgfqpoint{5.450978in}{2.136119in}}%
\pgfpathlineto{\pgfqpoint{5.451910in}{2.403395in}}%
\pgfpathlineto{\pgfqpoint{5.451281in}{1.858547in}}%
\pgfpathlineto{\pgfqpoint{5.452096in}{2.352827in}}%
\pgfpathlineto{\pgfqpoint{5.452484in}{1.990315in}}%
\pgfpathlineto{\pgfqpoint{5.452546in}{2.386615in}}%
\pgfpathlineto{\pgfqpoint{5.453211in}{2.198991in}}%
\pgfpathlineto{\pgfqpoint{5.453930in}{1.765514in}}%
\pgfpathlineto{\pgfqpoint{5.454007in}{2.391777in}}%
\pgfpathlineto{\pgfqpoint{5.454169in}{2.040331in}}%
\pgfpathlineto{\pgfqpoint{5.454871in}{2.415800in}}%
\pgfpathlineto{\pgfqpoint{5.454809in}{1.842577in}}%
\pgfpathlineto{\pgfqpoint{5.455279in}{2.404482in}}%
\pgfpathlineto{\pgfqpoint{5.455748in}{1.905771in}}%
\pgfpathlineto{\pgfqpoint{5.456393in}{2.258381in}}%
\pgfpathlineto{\pgfqpoint{5.457321in}{2.383584in}}%
\pgfpathlineto{\pgfqpoint{5.457367in}{1.965323in}}%
\pgfpathlineto{\pgfqpoint{5.457490in}{2.290659in}}%
\pgfpathlineto{\pgfqpoint{5.457972in}{1.897877in}}%
\pgfpathlineto{\pgfqpoint{5.457689in}{2.389712in}}%
\pgfpathlineto{\pgfqpoint{5.458598in}{2.238815in}}%
\pgfpathlineto{\pgfqpoint{5.458713in}{2.394682in}}%
\pgfpathlineto{\pgfqpoint{5.459300in}{1.905557in}}%
\pgfpathlineto{\pgfqpoint{5.459712in}{2.322980in}}%
\pgfpathlineto{\pgfqpoint{5.460184in}{1.900310in}}%
\pgfpathlineto{\pgfqpoint{5.459765in}{2.387763in}}%
\pgfpathlineto{\pgfqpoint{5.460876in}{2.174294in}}%
\pgfpathlineto{\pgfqpoint{5.461134in}{2.401347in}}%
\pgfpathlineto{\pgfqpoint{5.461308in}{1.619876in}}%
\pgfpathlineto{\pgfqpoint{5.461983in}{2.296542in}}%
\pgfpathlineto{\pgfqpoint{5.462104in}{1.999232in}}%
\pgfpathlineto{\pgfqpoint{5.462642in}{2.429167in}}%
\pgfpathlineto{\pgfqpoint{5.463095in}{2.254723in}}%
\pgfpathlineto{\pgfqpoint{5.463292in}{1.848237in}}%
\pgfpathlineto{\pgfqpoint{5.463405in}{2.400085in}}%
\pgfpathlineto{\pgfqpoint{5.464122in}{2.238373in}}%
\pgfpathlineto{\pgfqpoint{5.464551in}{2.417471in}}%
\pgfpathlineto{\pgfqpoint{5.464672in}{1.680183in}}%
\pgfpathlineto{\pgfqpoint{5.465236in}{2.333973in}}%
\pgfpathlineto{\pgfqpoint{5.466205in}{1.921566in}}%
\pgfpathlineto{\pgfqpoint{5.465732in}{2.394376in}}%
\pgfpathlineto{\pgfqpoint{5.466347in}{2.313690in}}%
\pgfpathlineto{\pgfqpoint{5.467126in}{2.414838in}}%
\pgfpathlineto{\pgfqpoint{5.466557in}{1.824642in}}%
\pgfpathlineto{\pgfqpoint{5.467433in}{2.275789in}}%
\pgfpathlineto{\pgfqpoint{5.468397in}{1.938371in}}%
\pgfpathlineto{\pgfqpoint{5.467515in}{2.401040in}}%
\pgfpathlineto{\pgfqpoint{5.468538in}{2.079512in}}%
\pgfpathlineto{\pgfqpoint{5.469440in}{2.409291in}}%
\pgfpathlineto{\pgfqpoint{5.469179in}{1.929833in}}%
\pgfpathlineto{\pgfqpoint{5.469648in}{2.226426in}}%
\pgfpathlineto{\pgfqpoint{5.470199in}{1.994601in}}%
\pgfpathlineto{\pgfqpoint{5.469857in}{2.410900in}}%
\pgfpathlineto{\pgfqpoint{5.470741in}{2.236201in}}%
\pgfpathlineto{\pgfqpoint{5.471445in}{2.399965in}}%
\pgfpathlineto{\pgfqpoint{5.471460in}{1.845409in}}%
\pgfpathlineto{\pgfqpoint{5.471852in}{2.282352in}}%
\pgfpathlineto{\pgfqpoint{5.472111in}{1.937631in}}%
\pgfpathlineto{\pgfqpoint{5.472222in}{2.406472in}}%
\pgfpathlineto{\pgfqpoint{5.472968in}{2.093018in}}%
\pgfpathlineto{\pgfqpoint{5.473500in}{2.414130in}}%
\pgfpathlineto{\pgfqpoint{5.474052in}{1.901068in}}%
\pgfpathlineto{\pgfqpoint{5.474074in}{2.223947in}}%
\pgfpathlineto{\pgfqpoint{5.474861in}{2.007978in}}%
\pgfpathlineto{\pgfqpoint{5.474972in}{2.405409in}}%
\pgfpathlineto{\pgfqpoint{5.475177in}{2.240389in}}%
\pgfpathlineto{\pgfqpoint{5.476226in}{2.377189in}}%
\pgfpathlineto{\pgfqpoint{5.476124in}{1.893871in}}%
\pgfpathlineto{\pgfqpoint{5.476270in}{2.330713in}}%
\pgfpathlineto{\pgfqpoint{5.477192in}{2.021335in}}%
\pgfpathlineto{\pgfqpoint{5.477134in}{2.417011in}}%
\pgfpathlineto{\pgfqpoint{5.477382in}{2.268842in}}%
\pgfpathlineto{\pgfqpoint{5.477638in}{2.394182in}}%
\pgfpathlineto{\pgfqpoint{5.477901in}{1.901831in}}%
\pgfpathlineto{\pgfqpoint{5.478382in}{2.224563in}}%
\pgfpathlineto{\pgfqpoint{5.479430in}{1.786962in}}%
\pgfpathlineto{\pgfqpoint{5.479452in}{2.411828in}}%
\pgfpathlineto{\pgfqpoint{5.479488in}{2.237115in}}%
\pgfpathlineto{\pgfqpoint{5.479692in}{2.426039in}}%
\pgfpathlineto{\pgfqpoint{5.480055in}{1.883779in}}%
\pgfpathlineto{\pgfqpoint{5.480592in}{2.247014in}}%
\pgfpathlineto{\pgfqpoint{5.481258in}{1.996107in}}%
\pgfpathlineto{\pgfqpoint{5.481135in}{2.402491in}}%
\pgfpathlineto{\pgfqpoint{5.481685in}{2.248167in}}%
\pgfpathlineto{\pgfqpoint{5.481931in}{2.398764in}}%
\pgfpathlineto{\pgfqpoint{5.482574in}{1.912522in}}%
\pgfpathlineto{\pgfqpoint{5.482776in}{2.343480in}}%
\pgfpathlineto{\pgfqpoint{5.483223in}{1.932949in}}%
\pgfpathlineto{\pgfqpoint{5.483741in}{2.398931in}}%
\pgfpathlineto{\pgfqpoint{5.483885in}{2.187875in}}%
\pgfpathlineto{\pgfqpoint{5.484000in}{2.417600in}}%
\pgfpathlineto{\pgfqpoint{5.484252in}{1.879002in}}%
\pgfpathlineto{\pgfqpoint{5.484992in}{2.200622in}}%
\pgfpathlineto{\pgfqpoint{5.485178in}{2.392600in}}%
\pgfpathlineto{\pgfqpoint{5.485931in}{1.961485in}}%
\pgfpathlineto{\pgfqpoint{5.486038in}{2.346075in}}%
\pgfpathlineto{\pgfqpoint{5.486217in}{1.842201in}}%
\pgfpathlineto{\pgfqpoint{5.486324in}{2.413703in}}%
\pgfpathlineto{\pgfqpoint{5.487146in}{2.279143in}}%
\pgfpathlineto{\pgfqpoint{5.487368in}{2.379257in}}%
\pgfpathlineto{\pgfqpoint{5.487182in}{1.820269in}}%
\pgfpathlineto{\pgfqpoint{5.488244in}{2.279265in}}%
\pgfpathlineto{\pgfqpoint{5.488365in}{1.964900in}}%
\pgfpathlineto{\pgfqpoint{5.488814in}{2.395376in}}%
\pgfpathlineto{\pgfqpoint{5.489354in}{2.286261in}}%
\pgfpathlineto{\pgfqpoint{5.490120in}{2.416277in}}%
\pgfpathlineto{\pgfqpoint{5.490042in}{1.868324in}}%
\pgfpathlineto{\pgfqpoint{5.490432in}{2.209589in}}%
\pgfpathlineto{\pgfqpoint{5.490567in}{1.996083in}}%
\pgfpathlineto{\pgfqpoint{5.490843in}{2.422778in}}%
\pgfpathlineto{\pgfqpoint{5.491536in}{2.265184in}}%
\pgfpathlineto{\pgfqpoint{5.492510in}{2.404403in}}%
\pgfpathlineto{\pgfqpoint{5.492418in}{1.748250in}}%
\pgfpathlineto{\pgfqpoint{5.492637in}{2.231242in}}%
\pgfpathlineto{\pgfqpoint{5.493017in}{1.717254in}}%
\pgfpathlineto{\pgfqpoint{5.493657in}{2.420419in}}%
\pgfpathlineto{\pgfqpoint{5.493749in}{2.198286in}}%
\pgfpathlineto{\pgfqpoint{5.494149in}{2.419544in}}%
\pgfpathlineto{\pgfqpoint{5.494002in}{1.844929in}}%
\pgfpathlineto{\pgfqpoint{5.494844in}{2.149759in}}%
\pgfpathlineto{\pgfqpoint{5.494956in}{1.939516in}}%
\pgfpathlineto{\pgfqpoint{5.495383in}{2.417436in}}%
\pgfpathlineto{\pgfqpoint{5.495943in}{2.224034in}}%
\pgfpathlineto{\pgfqpoint{5.496349in}{2.395958in}}%
\pgfpathlineto{\pgfqpoint{5.496237in}{1.891467in}}%
\pgfpathlineto{\pgfqpoint{5.497047in}{2.295802in}}%
\pgfpathlineto{\pgfqpoint{5.498140in}{1.970945in}}%
\pgfpathlineto{\pgfqpoint{5.497179in}{2.415220in}}%
\pgfpathlineto{\pgfqpoint{5.498147in}{2.314515in}}%
\pgfpathlineto{\pgfqpoint{5.499183in}{2.439118in}}%
\pgfpathlineto{\pgfqpoint{5.498453in}{1.982902in}}%
\pgfpathlineto{\pgfqpoint{5.499197in}{2.306024in}}%
\pgfpathlineto{\pgfqpoint{5.500070in}{1.745618in}}%
\pgfpathlineto{\pgfqpoint{5.499731in}{2.412813in}}%
\pgfpathlineto{\pgfqpoint{5.500305in}{2.240587in}}%
\pgfpathlineto{\pgfqpoint{5.500727in}{1.807569in}}%
\pgfpathlineto{\pgfqpoint{5.500416in}{2.433586in}}%
\pgfpathlineto{\pgfqpoint{5.501405in}{2.123942in}}%
\pgfpathlineto{\pgfqpoint{5.502005in}{2.428362in}}%
\pgfpathlineto{\pgfqpoint{5.501667in}{1.789909in}}%
\pgfpathlineto{\pgfqpoint{5.502522in}{2.329785in}}%
\pgfpathlineto{\pgfqpoint{5.503072in}{1.972950in}}%
\pgfpathlineto{\pgfqpoint{5.502590in}{2.406383in}}%
\pgfpathlineto{\pgfqpoint{5.503636in}{2.172069in}}%
\pgfpathlineto{\pgfqpoint{5.504541in}{2.406628in}}%
\pgfpathlineto{\pgfqpoint{5.504205in}{1.914790in}}%
\pgfpathlineto{\pgfqpoint{5.504726in}{2.323263in}}%
\pgfpathlineto{\pgfqpoint{5.505807in}{1.759956in}}%
\pgfpathlineto{\pgfqpoint{5.505581in}{2.406659in}}%
\pgfpathlineto{\pgfqpoint{5.505834in}{2.160665in}}%
\pgfpathlineto{\pgfqpoint{5.506326in}{2.426785in}}%
\pgfpathlineto{\pgfqpoint{5.506633in}{1.929115in}}%
\pgfpathlineto{\pgfqpoint{5.506946in}{2.272256in}}%
\pgfpathlineto{\pgfqpoint{5.507008in}{1.834930in}}%
\pgfpathlineto{\pgfqpoint{5.507396in}{2.404027in}}%
\pgfpathlineto{\pgfqpoint{5.508056in}{2.249465in}}%
\pgfpathlineto{\pgfqpoint{5.508762in}{2.022886in}}%
\pgfpathlineto{\pgfqpoint{5.508925in}{2.414369in}}%
\pgfpathlineto{\pgfqpoint{5.509155in}{2.306403in}}%
\pgfpathlineto{\pgfqpoint{5.509697in}{2.417502in}}%
\pgfpathlineto{\pgfqpoint{5.509257in}{1.778678in}}%
\pgfpathlineto{\pgfqpoint{5.510245in}{2.318759in}}%
\pgfpathlineto{\pgfqpoint{5.510299in}{1.869284in}}%
\pgfpathlineto{\pgfqpoint{5.511218in}{2.408045in}}%
\pgfpathlineto{\pgfqpoint{5.511359in}{2.223529in}}%
\pgfpathlineto{\pgfqpoint{5.511979in}{2.435060in}}%
\pgfpathlineto{\pgfqpoint{5.512444in}{1.855179in}}%
\pgfpathlineto{\pgfqpoint{5.512464in}{2.243730in}}%
\pgfpathlineto{\pgfqpoint{5.513337in}{1.878518in}}%
\pgfpathlineto{\pgfqpoint{5.512760in}{2.422798in}}%
\pgfpathlineto{\pgfqpoint{5.513579in}{2.072496in}}%
\pgfpathlineto{\pgfqpoint{5.514370in}{2.414617in}}%
\pgfpathlineto{\pgfqpoint{5.513706in}{1.953956in}}%
\pgfpathlineto{\pgfqpoint{5.514691in}{2.340277in}}%
\pgfpathlineto{\pgfqpoint{5.515433in}{1.925028in}}%
\pgfpathlineto{\pgfqpoint{5.515346in}{2.413848in}}%
\pgfpathlineto{\pgfqpoint{5.515820in}{2.205419in}}%
\pgfpathlineto{\pgfqpoint{5.516361in}{2.392615in}}%
\pgfpathlineto{\pgfqpoint{5.515947in}{2.025954in}}%
\pgfpathlineto{\pgfqpoint{5.516933in}{2.273360in}}%
\pgfpathlineto{\pgfqpoint{5.517186in}{1.787719in}}%
\pgfpathlineto{\pgfqpoint{5.517665in}{2.434337in}}%
\pgfpathlineto{\pgfqpoint{5.518043in}{2.168726in}}%
\pgfpathlineto{\pgfqpoint{5.518720in}{2.394940in}}%
\pgfpathlineto{\pgfqpoint{5.518905in}{1.737054in}}%
\pgfpathlineto{\pgfqpoint{5.519150in}{2.238374in}}%
\pgfpathlineto{\pgfqpoint{5.519448in}{1.934611in}}%
\pgfpathlineto{\pgfqpoint{5.520109in}{2.417735in}}%
\pgfpathlineto{\pgfqpoint{5.520255in}{2.239767in}}%
\pgfpathlineto{\pgfqpoint{5.520433in}{2.417304in}}%
\pgfpathlineto{\pgfqpoint{5.520558in}{1.923587in}}%
\pgfpathlineto{\pgfqpoint{5.521363in}{2.291019in}}%
\pgfpathlineto{\pgfqpoint{5.521442in}{1.951271in}}%
\pgfpathlineto{\pgfqpoint{5.522080in}{2.430483in}}%
\pgfpathlineto{\pgfqpoint{5.522448in}{2.233296in}}%
\pgfpathlineto{\pgfqpoint{5.523065in}{2.426670in}}%
\pgfpathlineto{\pgfqpoint{5.523498in}{1.854098in}}%
\pgfpathlineto{\pgfqpoint{5.523557in}{2.272605in}}%
\pgfpathlineto{\pgfqpoint{5.524009in}{2.425360in}}%
\pgfpathlineto{\pgfqpoint{5.523681in}{1.869355in}}%
\pgfpathlineto{\pgfqpoint{5.524473in}{2.327364in}}%
\pgfpathlineto{\pgfqpoint{5.524532in}{1.471637in}}%
\pgfpathlineto{\pgfqpoint{5.525225in}{2.408040in}}%
\pgfpathlineto{\pgfqpoint{5.525583in}{2.315170in}}%
\pgfpathlineto{\pgfqpoint{5.525851in}{1.737264in}}%
\pgfpathlineto{\pgfqpoint{5.526554in}{2.420980in}}%
\pgfpathlineto{\pgfqpoint{5.526691in}{2.133996in}}%
\pgfpathlineto{\pgfqpoint{5.527457in}{2.421828in}}%
\pgfpathlineto{\pgfqpoint{5.526970in}{1.953529in}}%
\pgfpathlineto{\pgfqpoint{5.527801in}{2.217586in}}%
\pgfpathlineto{\pgfqpoint{5.527808in}{2.217618in}}%
\pgfpathlineto{\pgfqpoint{5.528676in}{2.410324in}}%
\pgfpathlineto{\pgfqpoint{5.528624in}{1.834425in}}%
\pgfpathlineto{\pgfqpoint{5.528909in}{2.291078in}}%
\pgfpathlineto{\pgfqpoint{5.528916in}{1.738965in}}%
\pgfpathlineto{\pgfqpoint{5.529149in}{2.415505in}}%
\pgfpathlineto{\pgfqpoint{5.530014in}{2.249865in}}%
\pgfpathlineto{\pgfqpoint{5.530466in}{2.424294in}}%
\pgfpathlineto{\pgfqpoint{5.530176in}{1.891586in}}%
\pgfpathlineto{\pgfqpoint{5.531123in}{2.319026in}}%
\pgfpathlineto{\pgfqpoint{5.531264in}{1.834582in}}%
\pgfpathlineto{\pgfqpoint{5.532062in}{2.424715in}}%
\pgfpathlineto{\pgfqpoint{5.532222in}{2.133996in}}%
\pgfpathlineto{\pgfqpoint{5.533043in}{2.417148in}}%
\pgfpathlineto{\pgfqpoint{5.532504in}{1.895305in}}%
\pgfpathlineto{\pgfqpoint{5.533331in}{2.167300in}}%
\pgfpathlineto{\pgfqpoint{5.534201in}{2.411928in}}%
\pgfpathlineto{\pgfqpoint{5.533344in}{2.021920in}}%
\pgfpathlineto{\pgfqpoint{5.534444in}{2.253317in}}%
\pgfpathlineto{\pgfqpoint{5.535375in}{1.716627in}}%
\pgfpathlineto{\pgfqpoint{5.534610in}{2.410957in}}%
\pgfpathlineto{\pgfqpoint{5.535547in}{2.293038in}}%
\pgfpathlineto{\pgfqpoint{5.536305in}{2.412026in}}%
\pgfpathlineto{\pgfqpoint{5.536534in}{1.994940in}}%
\pgfpathlineto{\pgfqpoint{5.536597in}{2.157741in}}%
\pgfpathlineto{\pgfqpoint{5.536604in}{1.918813in}}%
\pgfpathlineto{\pgfqpoint{5.536883in}{2.436050in}}%
\pgfpathlineto{\pgfqpoint{5.537701in}{2.027210in}}%
\pgfpathlineto{\pgfqpoint{5.538740in}{2.432070in}}%
\pgfpathlineto{\pgfqpoint{5.538271in}{1.878868in}}%
\pgfpathlineto{\pgfqpoint{5.538815in}{2.318052in}}%
\pgfpathlineto{\pgfqpoint{5.539687in}{1.849962in}}%
\pgfpathlineto{\pgfqpoint{5.539100in}{2.447664in}}%
\pgfpathlineto{\pgfqpoint{5.539926in}{2.310000in}}%
\pgfpathlineto{\pgfqpoint{5.539952in}{1.911062in}}%
\pgfpathlineto{\pgfqpoint{5.540116in}{2.433755in}}%
\pgfpathlineto{\pgfqpoint{5.541028in}{2.196067in}}%
\pgfpathlineto{\pgfqpoint{5.541299in}{2.415025in}}%
\pgfpathlineto{\pgfqpoint{5.541311in}{1.802575in}}%
\pgfpathlineto{\pgfqpoint{5.542127in}{2.311114in}}%
\pgfpathlineto{\pgfqpoint{5.542134in}{1.801027in}}%
\pgfpathlineto{\pgfqpoint{5.542178in}{2.439677in}}%
\pgfpathlineto{\pgfqpoint{5.543236in}{2.095009in}}%
\pgfpathlineto{\pgfqpoint{5.543811in}{2.430764in}}%
\pgfpathlineto{\pgfqpoint{5.543667in}{2.001679in}}%
\pgfpathlineto{\pgfqpoint{5.544348in}{2.300977in}}%
\pgfpathlineto{\pgfqpoint{5.544623in}{1.891205in}}%
\pgfpathlineto{\pgfqpoint{5.545407in}{2.416825in}}%
\pgfpathlineto{\pgfqpoint{5.545457in}{2.197710in}}%
\pgfpathlineto{\pgfqpoint{5.546265in}{2.392973in}}%
\pgfpathlineto{\pgfqpoint{5.546545in}{1.869040in}}%
\pgfpathlineto{\pgfqpoint{5.546570in}{2.254013in}}%
\pgfpathlineto{\pgfqpoint{5.546836in}{1.860071in}}%
\pgfpathlineto{\pgfqpoint{5.546861in}{2.428332in}}%
\pgfpathlineto{\pgfqpoint{5.547661in}{2.213388in}}%
\pgfpathlineto{\pgfqpoint{5.548292in}{2.420312in}}%
\pgfpathlineto{\pgfqpoint{5.548069in}{1.875034in}}%
\pgfpathlineto{\pgfqpoint{5.548774in}{2.293035in}}%
\pgfpathlineto{\pgfqpoint{5.549483in}{2.417365in}}%
\pgfpathlineto{\pgfqpoint{5.548804in}{1.626768in}}%
\pgfpathlineto{\pgfqpoint{5.549871in}{2.282111in}}%
\pgfpathlineto{\pgfqpoint{5.550536in}{1.907757in}}%
\pgfpathlineto{\pgfqpoint{5.550234in}{2.439264in}}%
\pgfpathlineto{\pgfqpoint{5.550978in}{2.213806in}}%
\pgfpathlineto{\pgfqpoint{5.551396in}{2.422607in}}%
\pgfpathlineto{\pgfqpoint{5.551978in}{1.841496in}}%
\pgfpathlineto{\pgfqpoint{5.552095in}{2.366101in}}%
\pgfpathlineto{\pgfqpoint{5.552786in}{1.818119in}}%
\pgfpathlineto{\pgfqpoint{5.552909in}{2.412812in}}%
\pgfpathlineto{\pgfqpoint{5.553208in}{2.308100in}}%
\pgfpathlineto{\pgfqpoint{5.553385in}{2.427802in}}%
\pgfpathlineto{\pgfqpoint{5.554154in}{1.839674in}}%
\pgfpathlineto{\pgfqpoint{5.554264in}{2.391815in}}%
\pgfpathlineto{\pgfqpoint{5.555062in}{1.940280in}}%
\pgfpathlineto{\pgfqpoint{5.555086in}{2.414231in}}%
\pgfpathlineto{\pgfqpoint{5.555372in}{2.209084in}}%
\pgfpathlineto{\pgfqpoint{5.556313in}{2.418801in}}%
\pgfpathlineto{\pgfqpoint{5.556295in}{1.861924in}}%
\pgfpathlineto{\pgfqpoint{5.556465in}{2.182160in}}%
\pgfpathlineto{\pgfqpoint{5.557367in}{1.854206in}}%
\pgfpathlineto{\pgfqpoint{5.557083in}{2.432073in}}%
\pgfpathlineto{\pgfqpoint{5.557567in}{2.286055in}}%
\pgfpathlineto{\pgfqpoint{5.558455in}{2.436511in}}%
\pgfpathlineto{\pgfqpoint{5.558521in}{1.946853in}}%
\pgfpathlineto{\pgfqpoint{5.558672in}{2.347893in}}%
\pgfpathlineto{\pgfqpoint{5.559317in}{1.916952in}}%
\pgfpathlineto{\pgfqpoint{5.559516in}{2.440916in}}%
\pgfpathlineto{\pgfqpoint{5.559787in}{2.230582in}}%
\pgfpathlineto{\pgfqpoint{5.559823in}{2.390865in}}%
\pgfpathlineto{\pgfqpoint{5.560226in}{1.888642in}}%
\pgfpathlineto{\pgfqpoint{5.560898in}{2.263246in}}%
\pgfpathlineto{\pgfqpoint{5.561923in}{1.799318in}}%
\pgfpathlineto{\pgfqpoint{5.561696in}{2.413087in}}%
\pgfpathlineto{\pgfqpoint{5.562007in}{2.192994in}}%
\pgfpathlineto{\pgfqpoint{5.562396in}{2.437938in}}%
\pgfpathlineto{\pgfqpoint{5.562037in}{1.966765in}}%
\pgfpathlineto{\pgfqpoint{5.563113in}{2.253265in}}%
\pgfpathlineto{\pgfqpoint{5.563375in}{1.707497in}}%
\pgfpathlineto{\pgfqpoint{5.564097in}{2.437778in}}%
\pgfpathlineto{\pgfqpoint{5.564222in}{2.207618in}}%
\pgfpathlineto{\pgfqpoint{5.565262in}{2.450631in}}%
\pgfpathlineto{\pgfqpoint{5.565286in}{1.806419in}}%
\pgfpathlineto{\pgfqpoint{5.565316in}{2.136232in}}%
\pgfpathlineto{\pgfqpoint{5.565322in}{1.783189in}}%
\pgfpathlineto{\pgfqpoint{5.565583in}{2.416258in}}%
\pgfpathlineto{\pgfqpoint{5.566419in}{2.335552in}}%
\pgfpathlineto{\pgfqpoint{5.567324in}{1.952946in}}%
\pgfpathlineto{\pgfqpoint{5.566484in}{2.412883in}}%
\pgfpathlineto{\pgfqpoint{5.567501in}{2.327313in}}%
\pgfpathlineto{\pgfqpoint{5.568092in}{2.415055in}}%
\pgfpathlineto{\pgfqpoint{5.568280in}{1.811912in}}%
\pgfpathlineto{\pgfqpoint{5.568605in}{2.404278in}}%
\pgfpathlineto{\pgfqpoint{5.568705in}{1.975665in}}%
\pgfpathlineto{\pgfqpoint{5.569023in}{2.434702in}}%
\pgfpathlineto{\pgfqpoint{5.569711in}{2.308603in}}%
\pgfpathlineto{\pgfqpoint{5.569764in}{2.408969in}}%
\pgfpathlineto{\pgfqpoint{5.570539in}{1.875563in}}%
\pgfpathlineto{\pgfqpoint{5.570815in}{2.341425in}}%
\pgfpathlineto{\pgfqpoint{5.570926in}{1.688950in}}%
\pgfpathlineto{\pgfqpoint{5.571746in}{2.400985in}}%
\pgfpathlineto{\pgfqpoint{5.571927in}{2.236014in}}%
\pgfpathlineto{\pgfqpoint{5.571933in}{2.445569in}}%
\pgfpathlineto{\pgfqpoint{5.572896in}{1.855375in}}%
\pgfpathlineto{\pgfqpoint{5.573036in}{2.298836in}}%
\pgfpathlineto{\pgfqpoint{5.573258in}{1.747096in}}%
\pgfpathlineto{\pgfqpoint{5.574050in}{2.437506in}}%
\pgfpathlineto{\pgfqpoint{5.574149in}{2.250981in}}%
\pgfpathlineto{\pgfqpoint{5.574393in}{2.433977in}}%
\pgfpathlineto{\pgfqpoint{5.575090in}{1.883309in}}%
\pgfpathlineto{\pgfqpoint{5.575252in}{2.301840in}}%
\pgfpathlineto{\pgfqpoint{5.575299in}{1.967939in}}%
\pgfpathlineto{\pgfqpoint{5.576035in}{2.418402in}}%
\pgfpathlineto{\pgfqpoint{5.576359in}{2.188921in}}%
\pgfpathlineto{\pgfqpoint{5.577237in}{2.450365in}}%
\pgfpathlineto{\pgfqpoint{5.576972in}{1.978851in}}%
\pgfpathlineto{\pgfqpoint{5.577457in}{2.156751in}}%
\pgfpathlineto{\pgfqpoint{5.578258in}{1.811928in}}%
\pgfpathlineto{\pgfqpoint{5.577935in}{2.422434in}}%
\pgfpathlineto{\pgfqpoint{5.578558in}{2.295343in}}%
\pgfpathlineto{\pgfqpoint{5.578851in}{2.414480in}}%
\pgfpathlineto{\pgfqpoint{5.578592in}{1.899993in}}%
\pgfpathlineto{\pgfqpoint{5.579667in}{2.280230in}}%
\pgfpathlineto{\pgfqpoint{5.579782in}{1.918483in}}%
\pgfpathlineto{\pgfqpoint{5.580539in}{2.413496in}}%
\pgfpathlineto{\pgfqpoint{5.580768in}{2.205823in}}%
\pgfpathlineto{\pgfqpoint{5.581266in}{2.437812in}}%
\pgfpathlineto{\pgfqpoint{5.581649in}{1.760179in}}%
\pgfpathlineto{\pgfqpoint{5.581877in}{2.285252in}}%
\pgfpathlineto{\pgfqpoint{5.582134in}{1.905863in}}%
\pgfpathlineto{\pgfqpoint{5.582562in}{2.432079in}}%
\pgfpathlineto{\pgfqpoint{5.582967in}{2.346178in}}%
\pgfpathlineto{\pgfqpoint{5.583126in}{2.432014in}}%
\pgfpathlineto{\pgfqpoint{5.583388in}{2.047251in}}%
\pgfpathlineto{\pgfqpoint{5.584053in}{2.370073in}}%
\pgfpathlineto{\pgfqpoint{5.584927in}{1.810167in}}%
\pgfpathlineto{\pgfqpoint{5.585063in}{2.436217in}}%
\pgfpathlineto{\pgfqpoint{5.585166in}{2.280313in}}%
\pgfpathlineto{\pgfqpoint{5.586139in}{1.918616in}}%
\pgfpathlineto{\pgfqpoint{5.585936in}{2.423590in}}%
\pgfpathlineto{\pgfqpoint{5.586275in}{2.154135in}}%
\pgfpathlineto{\pgfqpoint{5.587330in}{2.439297in}}%
\pgfpathlineto{\pgfqpoint{5.586512in}{1.905010in}}%
\pgfpathlineto{\pgfqpoint{5.587387in}{2.324827in}}%
\pgfpathlineto{\pgfqpoint{5.587697in}{1.870850in}}%
\pgfpathlineto{\pgfqpoint{5.588378in}{2.422727in}}%
\pgfpathlineto{\pgfqpoint{5.588479in}{2.363359in}}%
\pgfpathlineto{\pgfqpoint{5.589204in}{2.418693in}}%
\pgfpathlineto{\pgfqpoint{5.589024in}{1.917955in}}%
\pgfpathlineto{\pgfqpoint{5.589400in}{2.256774in}}%
\pgfpathlineto{\pgfqpoint{5.589686in}{1.930044in}}%
\pgfpathlineto{\pgfqpoint{5.589490in}{2.437911in}}%
\pgfpathlineto{\pgfqpoint{5.590510in}{2.076313in}}%
\pgfpathlineto{\pgfqpoint{5.590868in}{2.457207in}}%
\pgfpathlineto{\pgfqpoint{5.590694in}{1.863292in}}%
\pgfpathlineto{\pgfqpoint{5.591622in}{2.205441in}}%
\pgfpathlineto{\pgfqpoint{5.592330in}{2.429864in}}%
\pgfpathlineto{\pgfqpoint{5.592542in}{1.736968in}}%
\pgfpathlineto{\pgfqpoint{5.592737in}{2.354432in}}%
\pgfpathlineto{\pgfqpoint{5.593493in}{1.895924in}}%
\pgfpathlineto{\pgfqpoint{5.593288in}{2.453507in}}%
\pgfpathlineto{\pgfqpoint{5.593843in}{2.198868in}}%
\pgfpathlineto{\pgfqpoint{5.594015in}{2.451904in}}%
\pgfpathlineto{\pgfqpoint{5.594426in}{1.784631in}}%
\pgfpathlineto{\pgfqpoint{5.594947in}{2.135547in}}%
\pgfpathlineto{\pgfqpoint{5.595046in}{1.818967in}}%
\pgfpathlineto{\pgfqpoint{5.595694in}{2.442208in}}%
\pgfpathlineto{\pgfqpoint{5.596053in}{2.073675in}}%
\pgfpathlineto{\pgfqpoint{5.596748in}{2.436328in}}%
\pgfpathlineto{\pgfqpoint{5.596952in}{1.920652in}}%
\pgfpathlineto{\pgfqpoint{5.597167in}{2.377025in}}%
\pgfpathlineto{\pgfqpoint{5.597442in}{1.997499in}}%
\pgfpathlineto{\pgfqpoint{5.597690in}{2.446347in}}%
\pgfpathlineto{\pgfqpoint{5.598278in}{2.311145in}}%
\pgfpathlineto{\pgfqpoint{5.599096in}{2.421065in}}%
\pgfpathlineto{\pgfqpoint{5.598652in}{2.008140in}}%
\pgfpathlineto{\pgfqpoint{5.599381in}{2.198050in}}%
\pgfpathlineto{\pgfqpoint{5.600323in}{1.582518in}}%
\pgfpathlineto{\pgfqpoint{5.600405in}{2.437605in}}%
\pgfpathlineto{\pgfqpoint{5.600465in}{2.195894in}}%
\pgfpathlineto{\pgfqpoint{5.601153in}{2.443004in}}%
\pgfpathlineto{\pgfqpoint{5.600886in}{2.005850in}}%
\pgfpathlineto{\pgfqpoint{5.601579in}{2.312100in}}%
\pgfpathlineto{\pgfqpoint{5.602570in}{1.964664in}}%
\pgfpathlineto{\pgfqpoint{5.602499in}{2.411887in}}%
\pgfpathlineto{\pgfqpoint{5.602690in}{2.239354in}}%
\pgfpathlineto{\pgfqpoint{5.603776in}{2.450707in}}%
\pgfpathlineto{\pgfqpoint{5.603260in}{1.881493in}}%
\pgfpathlineto{\pgfqpoint{5.603803in}{2.388164in}}%
\pgfpathlineto{\pgfqpoint{5.604843in}{1.909775in}}%
\pgfpathlineto{\pgfqpoint{5.604020in}{2.410760in}}%
\pgfpathlineto{\pgfqpoint{5.604913in}{2.243908in}}%
\pgfpathlineto{\pgfqpoint{5.604994in}{2.419681in}}%
\pgfpathlineto{\pgfqpoint{5.605394in}{1.930733in}}%
\pgfpathlineto{\pgfqpoint{5.606021in}{2.263978in}}%
\pgfpathlineto{\pgfqpoint{5.606156in}{1.833415in}}%
\pgfpathlineto{\pgfqpoint{5.606592in}{2.428084in}}%
\pgfpathlineto{\pgfqpoint{5.607131in}{2.245343in}}%
\pgfpathlineto{\pgfqpoint{5.607400in}{2.415280in}}%
\pgfpathlineto{\pgfqpoint{5.607238in}{1.896582in}}%
\pgfpathlineto{\pgfqpoint{5.608238in}{2.368574in}}%
\pgfpathlineto{\pgfqpoint{5.608715in}{2.018434in}}%
\pgfpathlineto{\pgfqpoint{5.608517in}{2.428451in}}%
\pgfpathlineto{\pgfqpoint{5.609347in}{2.029867in}}%
\pgfpathlineto{\pgfqpoint{5.609454in}{2.478819in}}%
\pgfpathlineto{\pgfqpoint{5.609428in}{1.817299in}}%
\pgfpathlineto{\pgfqpoint{5.610459in}{2.394559in}}%
\pgfpathlineto{\pgfqpoint{5.610518in}{1.884693in}}%
\pgfpathlineto{\pgfqpoint{5.610806in}{2.443313in}}%
\pgfpathlineto{\pgfqpoint{5.611569in}{2.257649in}}%
\pgfpathlineto{\pgfqpoint{5.612420in}{2.455787in}}%
\pgfpathlineto{\pgfqpoint{5.611760in}{1.983491in}}%
\pgfpathlineto{\pgfqpoint{5.612680in}{2.334255in}}%
\pgfpathlineto{\pgfqpoint{5.613025in}{1.728910in}}%
\pgfpathlineto{\pgfqpoint{5.612728in}{2.444025in}}%
\pgfpathlineto{\pgfqpoint{5.613794in}{2.298979in}}%
\pgfpathlineto{\pgfqpoint{5.614413in}{2.420375in}}%
\pgfpathlineto{\pgfqpoint{5.614339in}{1.913041in}}%
\pgfpathlineto{\pgfqpoint{5.614905in}{2.356713in}}%
\pgfpathlineto{\pgfqpoint{5.615026in}{1.938876in}}%
\pgfpathlineto{\pgfqpoint{5.615127in}{2.461219in}}%
\pgfpathlineto{\pgfqpoint{5.616029in}{2.261554in}}%
\pgfpathlineto{\pgfqpoint{5.616587in}{2.434420in}}%
\pgfpathlineto{\pgfqpoint{5.616708in}{1.837131in}}%
\pgfpathlineto{\pgfqpoint{5.617139in}{2.249861in}}%
\pgfpathlineto{\pgfqpoint{5.617775in}{2.447170in}}%
\pgfpathlineto{\pgfqpoint{5.617880in}{1.879768in}}%
\pgfpathlineto{\pgfqpoint{5.618252in}{2.391588in}}%
\pgfpathlineto{\pgfqpoint{5.618661in}{1.856513in}}%
\pgfpathlineto{\pgfqpoint{5.618294in}{2.438148in}}%
\pgfpathlineto{\pgfqpoint{5.619367in}{2.256689in}}%
\pgfpathlineto{\pgfqpoint{5.619623in}{2.436281in}}%
\pgfpathlineto{\pgfqpoint{5.619540in}{1.843335in}}%
\pgfpathlineto{\pgfqpoint{5.620479in}{2.274063in}}%
\pgfpathlineto{\pgfqpoint{5.620787in}{2.417908in}}%
\pgfpathlineto{\pgfqpoint{5.621042in}{1.967975in}}%
\pgfpathlineto{\pgfqpoint{5.621344in}{2.389736in}}%
\pgfpathlineto{\pgfqpoint{5.622150in}{1.860270in}}%
\pgfpathlineto{\pgfqpoint{5.622197in}{2.430736in}}%
\pgfpathlineto{\pgfqpoint{5.622456in}{2.262945in}}%
\pgfpathlineto{\pgfqpoint{5.622602in}{2.442031in}}%
\pgfpathlineto{\pgfqpoint{5.623265in}{1.978410in}}%
\pgfpathlineto{\pgfqpoint{5.623571in}{2.332023in}}%
\pgfpathlineto{\pgfqpoint{5.624377in}{1.880544in}}%
\pgfpathlineto{\pgfqpoint{5.623643in}{2.432144in}}%
\pgfpathlineto{\pgfqpoint{5.624682in}{1.961151in}}%
\pgfpathlineto{\pgfqpoint{5.625224in}{2.447064in}}%
\pgfpathlineto{\pgfqpoint{5.624754in}{1.845217in}}%
\pgfpathlineto{\pgfqpoint{5.625796in}{2.388203in}}%
\pgfpathlineto{\pgfqpoint{5.625853in}{1.842310in}}%
\pgfpathlineto{\pgfqpoint{5.626233in}{2.469624in}}%
\pgfpathlineto{\pgfqpoint{5.626907in}{2.208992in}}%
\pgfpathlineto{\pgfqpoint{5.626912in}{2.443582in}}%
\pgfpathlineto{\pgfqpoint{5.627379in}{1.838505in}}%
\pgfpathlineto{\pgfqpoint{5.628020in}{2.302146in}}%
\pgfpathlineto{\pgfqpoint{5.628491in}{2.436896in}}%
\pgfpathlineto{\pgfqpoint{5.628767in}{1.977581in}}%
\pgfpathlineto{\pgfqpoint{5.629125in}{2.279308in}}%
\pgfpathlineto{\pgfqpoint{5.629580in}{1.938994in}}%
\pgfpathlineto{\pgfqpoint{5.629942in}{2.436603in}}%
\pgfpathlineto{\pgfqpoint{5.630232in}{2.095738in}}%
\pgfpathlineto{\pgfqpoint{5.630899in}{2.431944in}}%
\pgfpathlineto{\pgfqpoint{5.630462in}{1.967547in}}%
\pgfpathlineto{\pgfqpoint{5.631347in}{2.276177in}}%
\pgfpathlineto{\pgfqpoint{5.632164in}{1.935835in}}%
\pgfpathlineto{\pgfqpoint{5.631611in}{2.412120in}}%
\pgfpathlineto{\pgfqpoint{5.632459in}{2.241717in}}%
\pgfpathlineto{\pgfqpoint{5.633537in}{2.412977in}}%
\pgfpathlineto{\pgfqpoint{5.632823in}{1.828215in}}%
\pgfpathlineto{\pgfqpoint{5.633572in}{2.344767in}}%
\pgfpathlineto{\pgfqpoint{5.634482in}{2.009658in}}%
\pgfpathlineto{\pgfqpoint{5.633916in}{2.437455in}}%
\pgfpathlineto{\pgfqpoint{5.634688in}{2.135396in}}%
\pgfpathlineto{\pgfqpoint{5.634693in}{2.437810in}}%
\pgfpathlineto{\pgfqpoint{5.634915in}{1.907738in}}%
\pgfpathlineto{\pgfqpoint{5.635801in}{2.275347in}}%
\pgfpathlineto{\pgfqpoint{5.636138in}{1.888243in}}%
\pgfpathlineto{\pgfqpoint{5.636520in}{2.458413in}}%
\pgfpathlineto{\pgfqpoint{5.636906in}{2.298921in}}%
\pgfpathlineto{\pgfqpoint{5.637142in}{2.443290in}}%
\pgfpathlineto{\pgfqpoint{5.637994in}{2.025675in}}%
\pgfpathlineto{\pgfqpoint{5.638009in}{2.270369in}}%
\pgfpathlineto{\pgfqpoint{5.638369in}{1.813183in}}%
\pgfpathlineto{\pgfqpoint{5.639113in}{2.459909in}}%
\pgfpathlineto{\pgfqpoint{5.639118in}{2.360138in}}%
\pgfpathlineto{\pgfqpoint{5.639567in}{1.878565in}}%
\pgfpathlineto{\pgfqpoint{5.639592in}{2.423091in}}%
\pgfpathlineto{\pgfqpoint{5.640239in}{2.125750in}}%
\pgfpathlineto{\pgfqpoint{5.640752in}{2.475477in}}%
\pgfpathlineto{\pgfqpoint{5.641244in}{1.898048in}}%
\pgfpathlineto{\pgfqpoint{5.641348in}{2.184480in}}%
\pgfpathlineto{\pgfqpoint{5.641804in}{1.730336in}}%
\pgfpathlineto{\pgfqpoint{5.642087in}{2.475465in}}%
\pgfpathlineto{\pgfqpoint{5.642434in}{2.209925in}}%
\pgfpathlineto{\pgfqpoint{5.642938in}{2.454279in}}%
\pgfpathlineto{\pgfqpoint{5.642572in}{1.943404in}}%
\pgfpathlineto{\pgfqpoint{5.643546in}{2.290558in}}%
\pgfpathlineto{\pgfqpoint{5.644459in}{2.457060in}}%
\pgfpathlineto{\pgfqpoint{5.643803in}{1.885596in}}%
\pgfpathlineto{\pgfqpoint{5.644641in}{2.284939in}}%
\pgfpathlineto{\pgfqpoint{5.645522in}{1.880339in}}%
\pgfpathlineto{\pgfqpoint{5.645581in}{2.441741in}}%
\pgfpathlineto{\pgfqpoint{5.645748in}{2.327643in}}%
\pgfpathlineto{\pgfqpoint{5.646725in}{1.908401in}}%
\pgfpathlineto{\pgfqpoint{5.646225in}{2.452788in}}%
\pgfpathlineto{\pgfqpoint{5.646852in}{2.315617in}}%
\pgfpathlineto{\pgfqpoint{5.647846in}{2.437910in}}%
\pgfpathlineto{\pgfqpoint{5.647127in}{1.702314in}}%
\pgfpathlineto{\pgfqpoint{5.647953in}{2.255264in}}%
\pgfpathlineto{\pgfqpoint{5.648822in}{1.922906in}}%
\pgfpathlineto{\pgfqpoint{5.648378in}{2.473098in}}%
\pgfpathlineto{\pgfqpoint{5.649052in}{1.999827in}}%
\pgfpathlineto{\pgfqpoint{5.649549in}{2.451903in}}%
\pgfpathlineto{\pgfqpoint{5.649909in}{1.862491in}}%
\pgfpathlineto{\pgfqpoint{5.650162in}{2.305064in}}%
\pgfpathlineto{\pgfqpoint{5.650468in}{1.872503in}}%
\pgfpathlineto{\pgfqpoint{5.651157in}{2.456175in}}%
\pgfpathlineto{\pgfqpoint{5.651279in}{2.157971in}}%
\pgfpathlineto{\pgfqpoint{5.651327in}{2.468139in}}%
\pgfpathlineto{\pgfqpoint{5.651904in}{1.735659in}}%
\pgfpathlineto{\pgfqpoint{5.652388in}{2.239282in}}%
\pgfpathlineto{\pgfqpoint{5.652987in}{1.849538in}}%
\pgfpathlineto{\pgfqpoint{5.652591in}{2.453744in}}%
\pgfpathlineto{\pgfqpoint{5.653479in}{2.333656in}}%
\pgfpathlineto{\pgfqpoint{5.653489in}{2.444954in}}%
\pgfpathlineto{\pgfqpoint{5.654294in}{1.915507in}}%
\pgfpathlineto{\pgfqpoint{5.654573in}{2.366366in}}%
\pgfpathlineto{\pgfqpoint{5.655630in}{1.722670in}}%
\pgfpathlineto{\pgfqpoint{5.654717in}{2.429565in}}%
\pgfpathlineto{\pgfqpoint{5.655683in}{2.322770in}}%
\pgfpathlineto{\pgfqpoint{5.655866in}{2.439405in}}%
\pgfpathlineto{\pgfqpoint{5.656527in}{1.918573in}}%
\pgfpathlineto{\pgfqpoint{5.656790in}{2.330103in}}%
\pgfpathlineto{\pgfqpoint{5.657317in}{1.926692in}}%
\pgfpathlineto{\pgfqpoint{5.657202in}{2.486874in}}%
\pgfpathlineto{\pgfqpoint{5.657904in}{2.234581in}}%
\pgfpathlineto{\pgfqpoint{5.658839in}{2.430290in}}%
\pgfpathlineto{\pgfqpoint{5.658649in}{1.914597in}}%
\pgfpathlineto{\pgfqpoint{5.659015in}{2.290210in}}%
\pgfpathlineto{\pgfqpoint{5.659363in}{1.952405in}}%
\pgfpathlineto{\pgfqpoint{5.659115in}{2.418247in}}%
\pgfpathlineto{\pgfqpoint{5.660095in}{2.312052in}}%
\pgfpathlineto{\pgfqpoint{5.660237in}{2.454062in}}%
\pgfpathlineto{\pgfqpoint{5.660356in}{1.931488in}}%
\pgfpathlineto{\pgfqpoint{5.661205in}{2.321284in}}%
\pgfpathlineto{\pgfqpoint{5.661754in}{1.923321in}}%
\pgfpathlineto{\pgfqpoint{5.662184in}{2.447236in}}%
\pgfpathlineto{\pgfqpoint{5.662312in}{2.384800in}}%
\pgfpathlineto{\pgfqpoint{5.662827in}{2.479872in}}%
\pgfpathlineto{\pgfqpoint{5.662534in}{1.644322in}}%
\pgfpathlineto{\pgfqpoint{5.663275in}{2.356808in}}%
\pgfpathlineto{\pgfqpoint{5.664278in}{1.724456in}}%
\pgfpathlineto{\pgfqpoint{5.664174in}{2.462372in}}%
\pgfpathlineto{\pgfqpoint{5.664386in}{2.265151in}}%
\pgfpathlineto{\pgfqpoint{5.664536in}{2.428883in}}%
\pgfpathlineto{\pgfqpoint{5.664494in}{1.813298in}}%
\pgfpathlineto{\pgfqpoint{5.665480in}{2.229833in}}%
\pgfpathlineto{\pgfqpoint{5.666314in}{1.817993in}}%
\pgfpathlineto{\pgfqpoint{5.665846in}{2.440135in}}%
\pgfpathlineto{\pgfqpoint{5.666586in}{2.295988in}}%
\pgfpathlineto{\pgfqpoint{5.667487in}{2.451223in}}%
\pgfpathlineto{\pgfqpoint{5.667375in}{1.913585in}}%
\pgfpathlineto{\pgfqpoint{5.667669in}{2.205682in}}%
\pgfpathlineto{\pgfqpoint{5.668522in}{1.897348in}}%
\pgfpathlineto{\pgfqpoint{5.668052in}{2.438604in}}%
\pgfpathlineto{\pgfqpoint{5.668778in}{2.133551in}}%
\pgfpathlineto{\pgfqpoint{5.669183in}{2.441339in}}%
\pgfpathlineto{\pgfqpoint{5.669448in}{1.891821in}}%
\pgfpathlineto{\pgfqpoint{5.669898in}{2.320336in}}%
\pgfpathlineto{\pgfqpoint{5.670293in}{1.629192in}}%
\pgfpathlineto{\pgfqpoint{5.670946in}{2.450132in}}%
\pgfpathlineto{\pgfqpoint{5.671016in}{1.804257in}}%
\pgfpathlineto{\pgfqpoint{5.671672in}{2.446696in}}%
\pgfpathlineto{\pgfqpoint{5.672130in}{2.394111in}}%
\pgfpathlineto{\pgfqpoint{5.673020in}{1.851370in}}%
\pgfpathlineto{\pgfqpoint{5.672711in}{2.462722in}}%
\pgfpathlineto{\pgfqpoint{5.673246in}{2.293329in}}%
\pgfpathlineto{\pgfqpoint{5.674124in}{1.895888in}}%
\pgfpathlineto{\pgfqpoint{5.673954in}{2.475915in}}%
\pgfpathlineto{\pgfqpoint{5.674358in}{2.216214in}}%
\pgfpathlineto{\pgfqpoint{5.675070in}{2.456228in}}%
\pgfpathlineto{\pgfqpoint{5.674799in}{1.797401in}}%
\pgfpathlineto{\pgfqpoint{5.675459in}{2.174894in}}%
\pgfpathlineto{\pgfqpoint{5.675464in}{1.971000in}}%
\pgfpathlineto{\pgfqpoint{5.675995in}{2.433004in}}%
\pgfpathlineto{\pgfqpoint{5.676566in}{2.347702in}}%
\pgfpathlineto{\pgfqpoint{5.677233in}{1.621551in}}%
\pgfpathlineto{\pgfqpoint{5.677616in}{2.441487in}}%
\pgfpathlineto{\pgfqpoint{5.677684in}{2.131694in}}%
\pgfpathlineto{\pgfqpoint{5.677958in}{2.442288in}}%
\pgfpathlineto{\pgfqpoint{5.678099in}{2.004784in}}%
\pgfpathlineto{\pgfqpoint{5.678795in}{2.365111in}}%
\pgfpathlineto{\pgfqpoint{5.679599in}{1.903821in}}%
\pgfpathlineto{\pgfqpoint{5.679676in}{2.467376in}}%
\pgfpathlineto{\pgfqpoint{5.679907in}{2.312896in}}%
\pgfpathlineto{\pgfqpoint{5.680383in}{2.426871in}}%
\pgfpathlineto{\pgfqpoint{5.680460in}{1.825052in}}%
\pgfpathlineto{\pgfqpoint{5.680912in}{2.383882in}}%
\pgfpathlineto{\pgfqpoint{5.681658in}{1.803202in}}%
\pgfpathlineto{\pgfqpoint{5.681910in}{2.434878in}}%
\pgfpathlineto{\pgfqpoint{5.682023in}{2.117201in}}%
\pgfpathlineto{\pgfqpoint{5.682370in}{2.452246in}}%
\pgfpathlineto{\pgfqpoint{5.682068in}{1.961588in}}%
\pgfpathlineto{\pgfqpoint{5.683136in}{2.293674in}}%
\pgfpathlineto{\pgfqpoint{5.683361in}{1.909746in}}%
\pgfpathlineto{\pgfqpoint{5.683163in}{2.473062in}}%
\pgfpathlineto{\pgfqpoint{5.684246in}{2.206352in}}%
\pgfpathlineto{\pgfqpoint{5.684380in}{2.469297in}}%
\pgfpathlineto{\pgfqpoint{5.685343in}{1.922556in}}%
\pgfpathlineto{\pgfqpoint{5.685357in}{2.337086in}}%
\pgfpathlineto{\pgfqpoint{5.686224in}{1.892385in}}%
\pgfpathlineto{\pgfqpoint{5.686407in}{2.470173in}}%
\pgfpathlineto{\pgfqpoint{5.686470in}{2.252899in}}%
\pgfpathlineto{\pgfqpoint{5.687192in}{2.465637in}}%
\pgfpathlineto{\pgfqpoint{5.687241in}{1.907997in}}%
\pgfpathlineto{\pgfqpoint{5.687522in}{2.399173in}}%
\pgfpathlineto{\pgfqpoint{5.687980in}{1.833214in}}%
\pgfpathlineto{\pgfqpoint{5.688354in}{2.457146in}}%
\pgfpathlineto{\pgfqpoint{5.688634in}{2.276442in}}%
\pgfpathlineto{\pgfqpoint{5.689224in}{2.459249in}}%
\pgfpathlineto{\pgfqpoint{5.689516in}{1.921285in}}%
\pgfpathlineto{\pgfqpoint{5.689716in}{2.385255in}}%
\pgfpathlineto{\pgfqpoint{5.689818in}{1.826909in}}%
\pgfpathlineto{\pgfqpoint{5.690017in}{2.448393in}}%
\pgfpathlineto{\pgfqpoint{5.690826in}{2.234306in}}%
\pgfpathlineto{\pgfqpoint{5.691876in}{2.457316in}}%
\pgfpathlineto{\pgfqpoint{5.691757in}{1.856137in}}%
\pgfpathlineto{\pgfqpoint{5.691934in}{2.228903in}}%
\pgfpathlineto{\pgfqpoint{5.692462in}{2.451366in}}%
\pgfpathlineto{\pgfqpoint{5.691969in}{1.819224in}}%
\pgfpathlineto{\pgfqpoint{5.693056in}{2.331783in}}%
\pgfpathlineto{\pgfqpoint{5.694131in}{1.902420in}}%
\pgfpathlineto{\pgfqpoint{5.693978in}{2.438812in}}%
\pgfpathlineto{\pgfqpoint{5.694166in}{2.348759in}}%
\pgfpathlineto{\pgfqpoint{5.694639in}{2.444056in}}%
\pgfpathlineto{\pgfqpoint{5.694175in}{1.853619in}}%
\pgfpathlineto{\pgfqpoint{5.695243in}{2.169374in}}%
\pgfpathlineto{\pgfqpoint{5.696296in}{1.731022in}}%
\pgfpathlineto{\pgfqpoint{5.695278in}{2.464023in}}%
\pgfpathlineto{\pgfqpoint{5.696348in}{2.183736in}}%
\pgfpathlineto{\pgfqpoint{5.697263in}{2.448050in}}%
\pgfpathlineto{\pgfqpoint{5.697389in}{1.961662in}}%
\pgfpathlineto{\pgfqpoint{5.697445in}{2.346123in}}%
\pgfpathlineto{\pgfqpoint{5.698219in}{1.873492in}}%
\pgfpathlineto{\pgfqpoint{5.697519in}{2.462819in}}%
\pgfpathlineto{\pgfqpoint{5.698553in}{2.136243in}}%
\pgfpathlineto{\pgfqpoint{5.698852in}{2.458117in}}%
\pgfpathlineto{\pgfqpoint{5.699645in}{1.752194in}}%
\pgfpathlineto{\pgfqpoint{5.699667in}{2.334004in}}%
\pgfpathlineto{\pgfqpoint{5.700743in}{1.887093in}}%
\pgfpathlineto{\pgfqpoint{5.700021in}{2.439259in}}%
\pgfpathlineto{\pgfqpoint{5.700777in}{2.301357in}}%
\pgfpathlineto{\pgfqpoint{5.701583in}{2.454731in}}%
\pgfpathlineto{\pgfqpoint{5.701553in}{1.656722in}}%
\pgfpathlineto{\pgfqpoint{5.701876in}{2.331221in}}%
\pgfpathlineto{\pgfqpoint{5.702745in}{1.904183in}}%
\pgfpathlineto{\pgfqpoint{5.702736in}{2.435434in}}%
\pgfpathlineto{\pgfqpoint{5.702990in}{2.178595in}}%
\pgfpathlineto{\pgfqpoint{5.703633in}{2.465053in}}%
\pgfpathlineto{\pgfqpoint{5.703367in}{1.874384in}}%
\pgfpathlineto{\pgfqpoint{5.704100in}{2.344734in}}%
\pgfpathlineto{\pgfqpoint{5.704738in}{1.884914in}}%
\pgfpathlineto{\pgfqpoint{5.704802in}{2.441615in}}%
\pgfpathlineto{\pgfqpoint{5.705208in}{2.157609in}}%
\pgfpathlineto{\pgfqpoint{5.705532in}{2.459977in}}%
\pgfpathlineto{\pgfqpoint{5.705954in}{1.964940in}}%
\pgfpathlineto{\pgfqpoint{5.706317in}{2.235952in}}%
\pgfpathlineto{\pgfqpoint{5.707134in}{1.908587in}}%
\pgfpathlineto{\pgfqpoint{5.706951in}{2.488276in}}%
\pgfpathlineto{\pgfqpoint{5.707380in}{1.963718in}}%
\pgfpathlineto{\pgfqpoint{5.708399in}{2.483639in}}%
\pgfpathlineto{\pgfqpoint{5.707699in}{1.882512in}}%
\pgfpathlineto{\pgfqpoint{5.708492in}{2.229506in}}%
\pgfpathlineto{\pgfqpoint{5.709512in}{2.475233in}}%
\pgfpathlineto{\pgfqpoint{5.708657in}{1.944534in}}%
\pgfpathlineto{\pgfqpoint{5.709592in}{2.419713in}}%
\pgfpathlineto{\pgfqpoint{5.709863in}{1.736104in}}%
\pgfpathlineto{\pgfqpoint{5.709723in}{2.457808in}}%
\pgfpathlineto{\pgfqpoint{5.710703in}{1.933975in}}%
\pgfpathlineto{\pgfqpoint{5.711132in}{2.443537in}}%
\pgfpathlineto{\pgfqpoint{5.711764in}{1.931082in}}%
\pgfpathlineto{\pgfqpoint{5.711814in}{2.286490in}}%
\pgfpathlineto{\pgfqpoint{5.712113in}{1.853016in}}%
\pgfpathlineto{\pgfqpoint{5.712482in}{2.448297in}}%
\pgfpathlineto{\pgfqpoint{5.712923in}{2.289387in}}%
\pgfpathlineto{\pgfqpoint{5.713836in}{1.914828in}}%
\pgfpathlineto{\pgfqpoint{5.713149in}{2.439969in}}%
\pgfpathlineto{\pgfqpoint{5.714028in}{2.382538in}}%
\pgfpathlineto{\pgfqpoint{5.714342in}{2.440136in}}%
\pgfpathlineto{\pgfqpoint{5.714618in}{1.915605in}}%
\pgfpathlineto{\pgfqpoint{5.715048in}{2.299204in}}%
\pgfpathlineto{\pgfqpoint{5.715052in}{1.919643in}}%
\pgfpathlineto{\pgfqpoint{5.715198in}{2.483378in}}%
\pgfpathlineto{\pgfqpoint{5.716156in}{2.336216in}}%
\pgfpathlineto{\pgfqpoint{5.716714in}{2.445331in}}%
\pgfpathlineto{\pgfqpoint{5.716755in}{1.747549in}}%
\pgfpathlineto{\pgfqpoint{5.717200in}{2.272473in}}%
\pgfpathlineto{\pgfqpoint{5.718000in}{1.837504in}}%
\pgfpathlineto{\pgfqpoint{5.717926in}{2.470738in}}%
\pgfpathlineto{\pgfqpoint{5.718307in}{2.408386in}}%
\pgfpathlineto{\pgfqpoint{5.719403in}{2.009510in}}%
\pgfpathlineto{\pgfqpoint{5.718857in}{2.449093in}}%
\pgfpathlineto{\pgfqpoint{5.719436in}{2.278732in}}%
\pgfpathlineto{\pgfqpoint{5.720084in}{2.472653in}}%
\pgfpathlineto{\pgfqpoint{5.720191in}{1.853468in}}%
\pgfpathlineto{\pgfqpoint{5.720550in}{2.352795in}}%
\pgfpathlineto{\pgfqpoint{5.721541in}{1.816101in}}%
\pgfpathlineto{\pgfqpoint{5.721467in}{2.475582in}}%
\pgfpathlineto{\pgfqpoint{5.721656in}{2.218422in}}%
\pgfpathlineto{\pgfqpoint{5.721968in}{2.455701in}}%
\pgfpathlineto{\pgfqpoint{5.722329in}{1.797382in}}%
\pgfpathlineto{\pgfqpoint{5.722768in}{2.329074in}}%
\pgfpathlineto{\pgfqpoint{5.722977in}{2.457834in}}%
\pgfpathlineto{\pgfqpoint{5.723186in}{1.881146in}}%
\pgfpathlineto{\pgfqpoint{5.723869in}{2.265505in}}%
\pgfpathlineto{\pgfqpoint{5.724069in}{1.968494in}}%
\pgfpathlineto{\pgfqpoint{5.724302in}{2.445065in}}%
\pgfpathlineto{\pgfqpoint{5.724975in}{2.331784in}}%
\pgfpathlineto{\pgfqpoint{5.725077in}{2.446982in}}%
\pgfpathlineto{\pgfqpoint{5.725944in}{1.977311in}}%
\pgfpathlineto{\pgfqpoint{5.726038in}{2.314365in}}%
\pgfpathlineto{\pgfqpoint{5.726659in}{1.750555in}}%
\pgfpathlineto{\pgfqpoint{5.726635in}{2.466460in}}%
\pgfpathlineto{\pgfqpoint{5.727146in}{2.245650in}}%
\pgfpathlineto{\pgfqpoint{5.727608in}{2.512606in}}%
\pgfpathlineto{\pgfqpoint{5.727888in}{1.737852in}}%
\pgfpathlineto{\pgfqpoint{5.728260in}{2.373374in}}%
\pgfpathlineto{\pgfqpoint{5.729097in}{1.983240in}}%
\pgfpathlineto{\pgfqpoint{5.728365in}{2.465833in}}%
\pgfpathlineto{\pgfqpoint{5.729367in}{2.314321in}}%
\pgfpathlineto{\pgfqpoint{5.729867in}{2.454302in}}%
\pgfpathlineto{\pgfqpoint{5.730423in}{1.988415in}}%
\pgfpathlineto{\pgfqpoint{5.730475in}{2.259107in}}%
\pgfpathlineto{\pgfqpoint{5.731054in}{2.433374in}}%
\pgfpathlineto{\pgfqpoint{5.730636in}{1.792197in}}%
\pgfpathlineto{\pgfqpoint{5.731592in}{2.405040in}}%
\pgfpathlineto{\pgfqpoint{5.732506in}{1.923232in}}%
\pgfpathlineto{\pgfqpoint{5.731745in}{2.453873in}}%
\pgfpathlineto{\pgfqpoint{5.732702in}{2.360860in}}%
\pgfpathlineto{\pgfqpoint{5.733654in}{1.851681in}}%
\pgfpathlineto{\pgfqpoint{5.732822in}{2.472820in}}%
\pgfpathlineto{\pgfqpoint{5.733806in}{2.327379in}}%
\pgfpathlineto{\pgfqpoint{5.734540in}{2.452433in}}%
\pgfpathlineto{\pgfqpoint{5.733918in}{1.855499in}}%
\pgfpathlineto{\pgfqpoint{5.734922in}{2.428310in}}%
\pgfpathlineto{\pgfqpoint{5.735444in}{1.891757in}}%
\pgfpathlineto{\pgfqpoint{5.735754in}{2.442720in}}%
\pgfpathlineto{\pgfqpoint{5.736036in}{2.113539in}}%
\pgfpathlineto{\pgfqpoint{5.736337in}{2.450739in}}%
\pgfpathlineto{\pgfqpoint{5.736833in}{1.933199in}}%
\pgfpathlineto{\pgfqpoint{5.737146in}{2.367392in}}%
\pgfpathlineto{\pgfqpoint{5.737348in}{1.735520in}}%
\pgfpathlineto{\pgfqpoint{5.737685in}{2.464204in}}%
\pgfpathlineto{\pgfqpoint{5.738258in}{2.312375in}}%
\pgfpathlineto{\pgfqpoint{5.739024in}{2.451963in}}%
\pgfpathlineto{\pgfqpoint{5.738404in}{1.956374in}}%
\pgfpathlineto{\pgfqpoint{5.739213in}{2.389516in}}%
\pgfpathlineto{\pgfqpoint{5.739757in}{1.895206in}}%
\pgfpathlineto{\pgfqpoint{5.739355in}{2.439571in}}%
\pgfpathlineto{\pgfqpoint{5.740327in}{2.053237in}}%
\pgfpathlineto{\pgfqpoint{5.740755in}{2.458851in}}%
\pgfpathlineto{\pgfqpoint{5.740677in}{1.930946in}}%
\pgfpathlineto{\pgfqpoint{5.741442in}{2.328860in}}%
\pgfpathlineto{\pgfqpoint{5.741568in}{1.871120in}}%
\pgfpathlineto{\pgfqpoint{5.741779in}{2.478277in}}%
\pgfpathlineto{\pgfqpoint{5.742554in}{2.304803in}}%
\pgfpathlineto{\pgfqpoint{5.743355in}{2.451108in}}%
\pgfpathlineto{\pgfqpoint{5.742894in}{1.542162in}}%
\pgfpathlineto{\pgfqpoint{5.743617in}{2.365311in}}%
\pgfpathlineto{\pgfqpoint{5.743706in}{1.851366in}}%
\pgfpathlineto{\pgfqpoint{5.743738in}{2.463868in}}%
\pgfpathlineto{\pgfqpoint{5.744727in}{2.391181in}}%
\pgfpathlineto{\pgfqpoint{5.745419in}{1.854506in}}%
\pgfpathlineto{\pgfqpoint{5.744933in}{2.504381in}}%
\pgfpathlineto{\pgfqpoint{5.745835in}{2.373275in}}%
\pgfpathlineto{\pgfqpoint{5.746552in}{2.468171in}}%
\pgfpathlineto{\pgfqpoint{5.746757in}{1.951113in}}%
\pgfpathlineto{\pgfqpoint{5.746862in}{2.322957in}}%
\pgfpathlineto{\pgfqpoint{5.746866in}{1.742223in}}%
\pgfpathlineto{\pgfqpoint{5.747948in}{2.473937in}}%
\pgfpathlineto{\pgfqpoint{5.747971in}{2.283710in}}%
\pgfpathlineto{\pgfqpoint{5.748974in}{2.463907in}}%
\pgfpathlineto{\pgfqpoint{5.748423in}{1.835785in}}%
\pgfpathlineto{\pgfqpoint{5.749078in}{2.322990in}}%
\pgfpathlineto{\pgfqpoint{5.749440in}{1.900850in}}%
\pgfpathlineto{\pgfqpoint{5.749602in}{2.449113in}}%
\pgfpathlineto{\pgfqpoint{5.750186in}{2.214304in}}%
\pgfpathlineto{\pgfqpoint{5.750581in}{2.462454in}}%
\pgfpathlineto{\pgfqpoint{5.750961in}{1.775406in}}%
\pgfpathlineto{\pgfqpoint{5.751295in}{2.322264in}}%
\pgfpathlineto{\pgfqpoint{5.752397in}{1.884838in}}%
\pgfpathlineto{\pgfqpoint{5.751344in}{2.476430in}}%
\pgfpathlineto{\pgfqpoint{5.752400in}{2.319671in}}%
\pgfpathlineto{\pgfqpoint{5.752591in}{2.480671in}}%
\pgfpathlineto{\pgfqpoint{5.752962in}{1.901841in}}%
\pgfpathlineto{\pgfqpoint{5.753503in}{2.328367in}}%
\pgfpathlineto{\pgfqpoint{5.754075in}{1.863279in}}%
\pgfpathlineto{\pgfqpoint{5.754094in}{2.461915in}}%
\pgfpathlineto{\pgfqpoint{5.754611in}{2.385553in}}%
\pgfpathlineto{\pgfqpoint{5.754797in}{2.481352in}}%
\pgfpathlineto{\pgfqpoint{5.755507in}{1.885325in}}%
\pgfpathlineto{\pgfqpoint{5.755689in}{2.291806in}}%
\pgfpathlineto{\pgfqpoint{5.756519in}{1.894701in}}%
\pgfpathlineto{\pgfqpoint{5.756379in}{2.468329in}}%
\pgfpathlineto{\pgfqpoint{5.756795in}{2.266800in}}%
\pgfpathlineto{\pgfqpoint{5.756814in}{2.452716in}}%
\pgfpathlineto{\pgfqpoint{5.757800in}{1.974082in}}%
\pgfpathlineto{\pgfqpoint{5.757909in}{2.350745in}}%
\pgfpathlineto{\pgfqpoint{5.758147in}{1.710913in}}%
\pgfpathlineto{\pgfqpoint{5.758279in}{2.468294in}}%
\pgfpathlineto{\pgfqpoint{5.759024in}{2.173697in}}%
\pgfpathlineto{\pgfqpoint{5.759772in}{2.466430in}}%
\pgfpathlineto{\pgfqpoint{5.759160in}{1.948479in}}%
\pgfpathlineto{\pgfqpoint{5.760140in}{2.389411in}}%
\pgfpathlineto{\pgfqpoint{5.760238in}{1.740042in}}%
\pgfpathlineto{\pgfqpoint{5.760882in}{2.470553in}}%
\pgfpathlineto{\pgfqpoint{5.761257in}{2.244874in}}%
\pgfpathlineto{\pgfqpoint{5.762356in}{2.463922in}}%
\pgfpathlineto{\pgfqpoint{5.761283in}{1.670342in}}%
\pgfpathlineto{\pgfqpoint{5.762367in}{2.282391in}}%
\pgfpathlineto{\pgfqpoint{5.763042in}{2.013532in}}%
\pgfpathlineto{\pgfqpoint{5.762956in}{2.465601in}}%
\pgfpathlineto{\pgfqpoint{5.763478in}{2.164311in}}%
\pgfpathlineto{\pgfqpoint{5.763783in}{2.450936in}}%
\pgfpathlineto{\pgfqpoint{5.763906in}{1.952848in}}%
\pgfpathlineto{\pgfqpoint{5.764589in}{2.347332in}}%
\pgfpathlineto{\pgfqpoint{5.765665in}{1.868621in}}%
\pgfpathlineto{\pgfqpoint{5.765450in}{2.487069in}}%
\pgfpathlineto{\pgfqpoint{5.765702in}{2.236773in}}%
\pgfpathlineto{\pgfqpoint{5.766623in}{1.908803in}}%
\pgfpathlineto{\pgfqpoint{5.766375in}{2.476663in}}%
\pgfpathlineto{\pgfqpoint{5.766808in}{2.329418in}}%
\pgfpathlineto{\pgfqpoint{5.767454in}{2.469199in}}%
\pgfpathlineto{\pgfqpoint{5.767328in}{1.882816in}}%
\pgfpathlineto{\pgfqpoint{5.767914in}{2.353226in}}%
\pgfpathlineto{\pgfqpoint{5.768279in}{1.836370in}}%
\pgfpathlineto{\pgfqpoint{5.768937in}{2.456218in}}%
\pgfpathlineto{\pgfqpoint{5.769022in}{2.326213in}}%
\pgfpathlineto{\pgfqpoint{5.769910in}{2.482844in}}%
\pgfpathlineto{\pgfqpoint{5.769687in}{1.917510in}}%
\pgfpathlineto{\pgfqpoint{5.770130in}{2.315921in}}%
\pgfpathlineto{\pgfqpoint{5.770420in}{1.822079in}}%
\pgfpathlineto{\pgfqpoint{5.770357in}{2.471129in}}%
\pgfpathlineto{\pgfqpoint{5.771236in}{2.172732in}}%
\pgfpathlineto{\pgfqpoint{5.771484in}{2.475586in}}%
\pgfpathlineto{\pgfqpoint{5.771440in}{1.819227in}}%
\pgfpathlineto{\pgfqpoint{5.772349in}{2.380371in}}%
\pgfpathlineto{\pgfqpoint{5.772775in}{1.955895in}}%
\pgfpathlineto{\pgfqpoint{5.773179in}{2.466685in}}%
\pgfpathlineto{\pgfqpoint{5.773474in}{2.265201in}}%
\pgfpathlineto{\pgfqpoint{5.773816in}{2.467893in}}%
\pgfpathlineto{\pgfqpoint{5.773590in}{1.929336in}}%
\pgfpathlineto{\pgfqpoint{5.774571in}{2.332973in}}%
\pgfpathlineto{\pgfqpoint{5.775618in}{1.783529in}}%
\pgfpathlineto{\pgfqpoint{5.775089in}{2.455513in}}%
\pgfpathlineto{\pgfqpoint{5.775683in}{2.264509in}}%
\pgfpathlineto{\pgfqpoint{5.776304in}{1.775562in}}%
\pgfpathlineto{\pgfqpoint{5.775856in}{2.466932in}}%
\pgfpathlineto{\pgfqpoint{5.776788in}{2.379757in}}%
\pgfpathlineto{\pgfqpoint{5.777131in}{2.465399in}}%
\pgfpathlineto{\pgfqpoint{5.777329in}{1.921499in}}%
\pgfpathlineto{\pgfqpoint{5.777768in}{2.356454in}}%
\pgfpathlineto{\pgfqpoint{5.778743in}{1.830868in}}%
\pgfpathlineto{\pgfqpoint{5.778445in}{2.476880in}}%
\pgfpathlineto{\pgfqpoint{5.778879in}{2.338418in}}%
\pgfpathlineto{\pgfqpoint{5.779199in}{1.905229in}}%
\pgfpathlineto{\pgfqpoint{5.779321in}{2.514047in}}%
\pgfpathlineto{\pgfqpoint{5.779991in}{2.192517in}}%
\pgfpathlineto{\pgfqpoint{5.780838in}{2.484207in}}%
\pgfpathlineto{\pgfqpoint{5.780946in}{1.954853in}}%
\pgfpathlineto{\pgfqpoint{5.781103in}{2.358830in}}%
\pgfpathlineto{\pgfqpoint{5.781363in}{1.910920in}}%
\pgfpathlineto{\pgfqpoint{5.781396in}{2.471414in}}%
\pgfpathlineto{\pgfqpoint{5.782209in}{2.243819in}}%
\pgfpathlineto{\pgfqpoint{5.782362in}{2.478616in}}%
\pgfpathlineto{\pgfqpoint{5.782233in}{1.669809in}}%
\pgfpathlineto{\pgfqpoint{5.783315in}{2.251147in}}%
\pgfpathlineto{\pgfqpoint{5.783777in}{1.970138in}}%
\pgfpathlineto{\pgfqpoint{5.783784in}{2.489057in}}%
\pgfpathlineto{\pgfqpoint{5.784422in}{2.170704in}}%
\pgfpathlineto{\pgfqpoint{5.785095in}{2.448679in}}%
\pgfpathlineto{\pgfqpoint{5.784603in}{1.869506in}}%
\pgfpathlineto{\pgfqpoint{5.785533in}{2.287070in}}%
\pgfpathlineto{\pgfqpoint{5.785816in}{2.468097in}}%
\pgfpathlineto{\pgfqpoint{5.786582in}{1.858923in}}%
\pgfpathlineto{\pgfqpoint{5.786645in}{2.347266in}}%
\pgfpathlineto{\pgfqpoint{5.786832in}{1.884046in}}%
\pgfpathlineto{\pgfqpoint{5.786800in}{2.484011in}}%
\pgfpathlineto{\pgfqpoint{5.787754in}{2.141631in}}%
\pgfpathlineto{\pgfqpoint{5.788211in}{2.481579in}}%
\pgfpathlineto{\pgfqpoint{5.787810in}{1.766907in}}%
\pgfpathlineto{\pgfqpoint{5.788863in}{2.396646in}}%
\pgfpathlineto{\pgfqpoint{5.788927in}{1.837790in}}%
\pgfpathlineto{\pgfqpoint{5.789536in}{2.466651in}}%
\pgfpathlineto{\pgfqpoint{5.789977in}{2.143139in}}%
\pgfpathlineto{\pgfqpoint{5.790596in}{2.469792in}}%
\pgfpathlineto{\pgfqpoint{5.790641in}{1.958011in}}%
\pgfpathlineto{\pgfqpoint{5.791088in}{2.245084in}}%
\pgfpathlineto{\pgfqpoint{5.791858in}{2.449154in}}%
\pgfpathlineto{\pgfqpoint{5.791628in}{2.004829in}}%
\pgfpathlineto{\pgfqpoint{5.792147in}{2.346093in}}%
\pgfpathlineto{\pgfqpoint{5.792199in}{1.924863in}}%
\pgfpathlineto{\pgfqpoint{5.792258in}{2.481445in}}%
\pgfpathlineto{\pgfqpoint{5.793259in}{2.233674in}}%
\pgfpathlineto{\pgfqpoint{5.793894in}{1.784121in}}%
\pgfpathlineto{\pgfqpoint{5.793661in}{2.478715in}}%
\pgfpathlineto{\pgfqpoint{5.794306in}{2.310044in}}%
\pgfpathlineto{\pgfqpoint{5.794368in}{2.491846in}}%
\pgfpathlineto{\pgfqpoint{5.795091in}{1.951069in}}%
\pgfpathlineto{\pgfqpoint{5.795412in}{2.349756in}}%
\pgfpathlineto{\pgfqpoint{5.796192in}{1.848533in}}%
\pgfpathlineto{\pgfqpoint{5.795688in}{2.461525in}}%
\pgfpathlineto{\pgfqpoint{5.796523in}{2.248726in}}%
\pgfpathlineto{\pgfqpoint{5.797111in}{2.471569in}}%
\pgfpathlineto{\pgfqpoint{5.796667in}{1.878223in}}%
\pgfpathlineto{\pgfqpoint{5.797630in}{2.262794in}}%
\pgfpathlineto{\pgfqpoint{5.797874in}{1.960135in}}%
\pgfpathlineto{\pgfqpoint{5.798299in}{2.476963in}}%
\pgfpathlineto{\pgfqpoint{5.798738in}{2.310246in}}%
\pgfpathlineto{\pgfqpoint{5.799666in}{2.448776in}}%
\pgfpathlineto{\pgfqpoint{5.799022in}{1.933715in}}%
\pgfpathlineto{\pgfqpoint{5.799847in}{2.299603in}}%
\pgfpathlineto{\pgfqpoint{5.800359in}{1.810859in}}%
\pgfpathlineto{\pgfqpoint{5.800161in}{2.481303in}}%
\pgfpathlineto{\pgfqpoint{5.800956in}{2.319026in}}%
\pgfpathlineto{\pgfqpoint{5.801610in}{2.462303in}}%
\pgfpathlineto{\pgfqpoint{5.801276in}{1.620364in}}%
\pgfpathlineto{\pgfqpoint{5.802062in}{2.336485in}}%
\pgfpathlineto{\pgfqpoint{5.802748in}{1.902084in}}%
\pgfpathlineto{\pgfqpoint{5.802918in}{2.486912in}}%
\pgfpathlineto{\pgfqpoint{5.803172in}{2.259482in}}%
\pgfpathlineto{\pgfqpoint{5.803735in}{1.884273in}}%
\pgfpathlineto{\pgfqpoint{5.804286in}{2.481861in}}%
\pgfpathlineto{\pgfqpoint{5.805013in}{1.917915in}}%
\pgfpathlineto{\pgfqpoint{5.804573in}{2.490217in}}%
\pgfpathlineto{\pgfqpoint{5.805397in}{2.382858in}}%
\pgfpathlineto{\pgfqpoint{5.806394in}{1.973422in}}%
\pgfpathlineto{\pgfqpoint{5.805424in}{2.464335in}}%
\pgfpathlineto{\pgfqpoint{5.806512in}{2.120863in}}%
\pgfpathlineto{\pgfqpoint{5.807188in}{2.467694in}}%
\pgfpathlineto{\pgfqpoint{5.807399in}{1.874041in}}%
\pgfpathlineto{\pgfqpoint{5.807624in}{2.276764in}}%
\pgfpathlineto{\pgfqpoint{5.808599in}{2.470997in}}%
\pgfpathlineto{\pgfqpoint{5.808060in}{1.871656in}}%
\pgfpathlineto{\pgfqpoint{5.808733in}{2.297194in}}%
\pgfpathlineto{\pgfqpoint{5.809425in}{2.010093in}}%
\pgfpathlineto{\pgfqpoint{5.808867in}{2.446406in}}%
\pgfpathlineto{\pgfqpoint{5.809809in}{2.349652in}}%
\pgfpathlineto{\pgfqpoint{5.809813in}{2.497497in}}%
\pgfpathlineto{\pgfqpoint{5.810613in}{1.826592in}}%
\pgfpathlineto{\pgfqpoint{5.810916in}{2.301629in}}%
\pgfpathlineto{\pgfqpoint{5.811468in}{1.892334in}}%
\pgfpathlineto{\pgfqpoint{5.811375in}{2.488748in}}%
\pgfpathlineto{\pgfqpoint{5.812023in}{2.231578in}}%
\pgfpathlineto{\pgfqpoint{5.812988in}{2.493196in}}%
\pgfpathlineto{\pgfqpoint{5.812362in}{1.991883in}}%
\pgfpathlineto{\pgfqpoint{5.813137in}{2.449091in}}%
\pgfpathlineto{\pgfqpoint{5.814116in}{1.824182in}}%
\pgfpathlineto{\pgfqpoint{5.813700in}{2.478812in}}%
\pgfpathlineto{\pgfqpoint{5.814249in}{2.329238in}}%
\pgfpathlineto{\pgfqpoint{5.814688in}{1.907399in}}%
\pgfpathlineto{\pgfqpoint{5.814391in}{2.462375in}}%
\pgfpathlineto{\pgfqpoint{5.815360in}{2.174588in}}%
\pgfpathlineto{\pgfqpoint{5.816390in}{2.464345in}}%
\pgfpathlineto{\pgfqpoint{5.815568in}{1.922436in}}%
\pgfpathlineto{\pgfqpoint{5.816472in}{2.307877in}}%
\pgfpathlineto{\pgfqpoint{5.816948in}{2.443443in}}%
\pgfpathlineto{\pgfqpoint{5.816912in}{1.920522in}}%
\pgfpathlineto{\pgfqpoint{5.817545in}{2.186063in}}%
\pgfpathlineto{\pgfqpoint{5.818191in}{1.776995in}}%
\pgfpathlineto{\pgfqpoint{5.818426in}{2.470199in}}%
\pgfpathlineto{\pgfqpoint{5.818645in}{2.345760in}}%
\pgfpathlineto{\pgfqpoint{5.818769in}{2.079636in}}%
\pgfpathlineto{\pgfqpoint{5.818766in}{2.475761in}}%
\pgfpathlineto{\pgfqpoint{5.818779in}{2.319442in}}%
\pgfpathlineto{\pgfqpoint{5.819563in}{1.920175in}}%
\pgfpathlineto{\pgfqpoint{5.819409in}{2.487852in}}%
\pgfpathlineto{\pgfqpoint{5.819889in}{2.294236in}}%
\pgfpathlineto{\pgfqpoint{5.820853in}{1.776786in}}%
\pgfpathlineto{\pgfqpoint{5.820914in}{2.483480in}}%
\pgfpathlineto{\pgfqpoint{5.820989in}{2.340077in}}%
\pgfpathlineto{\pgfqpoint{5.822031in}{2.474079in}}%
\pgfpathlineto{\pgfqpoint{5.822038in}{2.007673in}}%
\pgfpathlineto{\pgfqpoint{5.822087in}{2.422815in}}%
\pgfpathlineto{\pgfqpoint{5.822278in}{1.762464in}}%
\pgfpathlineto{\pgfqpoint{5.822258in}{2.472969in}}%
\pgfpathlineto{\pgfqpoint{5.823197in}{2.226934in}}%
\pgfpathlineto{\pgfqpoint{5.823747in}{1.899885in}}%
\pgfpathlineto{\pgfqpoint{5.823462in}{2.483899in}}%
\pgfpathlineto{\pgfqpoint{5.824154in}{2.372180in}}%
\pgfpathlineto{\pgfqpoint{5.824882in}{2.472591in}}%
\pgfpathlineto{\pgfqpoint{5.824866in}{1.815500in}}%
\pgfpathlineto{\pgfqpoint{5.825259in}{2.330587in}}%
\pgfpathlineto{\pgfqpoint{5.825703in}{1.967163in}}%
\pgfpathlineto{\pgfqpoint{5.826246in}{2.494697in}}%
\pgfpathlineto{\pgfqpoint{5.826352in}{2.303585in}}%
\pgfpathlineto{\pgfqpoint{5.826859in}{2.497961in}}%
\pgfpathlineto{\pgfqpoint{5.826929in}{1.973978in}}%
\pgfpathlineto{\pgfqpoint{5.827464in}{2.385847in}}%
\pgfpathlineto{\pgfqpoint{5.827663in}{1.937837in}}%
\pgfpathlineto{\pgfqpoint{5.828270in}{2.467972in}}%
\pgfpathlineto{\pgfqpoint{5.828571in}{2.363040in}}%
\pgfpathlineto{\pgfqpoint{5.829639in}{2.486412in}}%
\pgfpathlineto{\pgfqpoint{5.828702in}{1.918625in}}%
\pgfpathlineto{\pgfqpoint{5.829652in}{2.352660in}}%
\pgfpathlineto{\pgfqpoint{5.829655in}{1.927477in}}%
\pgfpathlineto{\pgfqpoint{5.830714in}{2.454255in}}%
\pgfpathlineto{\pgfqpoint{5.830762in}{2.282058in}}%
\pgfpathlineto{\pgfqpoint{5.831464in}{2.466572in}}%
\pgfpathlineto{\pgfqpoint{5.830991in}{1.921618in}}%
\pgfpathlineto{\pgfqpoint{5.831793in}{2.190971in}}%
\pgfpathlineto{\pgfqpoint{5.832354in}{1.929334in}}%
\pgfpathlineto{\pgfqpoint{5.832006in}{2.469152in}}%
\pgfpathlineto{\pgfqpoint{5.832901in}{2.371518in}}%
\pgfpathlineto{\pgfqpoint{5.833372in}{1.926738in}}%
\pgfpathlineto{\pgfqpoint{5.833009in}{2.499878in}}%
\pgfpathlineto{\pgfqpoint{5.834006in}{2.269848in}}%
\pgfpathlineto{\pgfqpoint{5.834989in}{2.472632in}}%
\pgfpathlineto{\pgfqpoint{5.834932in}{1.987105in}}%
\pgfpathlineto{\pgfqpoint{5.835114in}{2.176891in}}%
\pgfpathlineto{\pgfqpoint{5.835404in}{1.759605in}}%
\pgfpathlineto{\pgfqpoint{5.835636in}{2.486960in}}%
\pgfpathlineto{\pgfqpoint{5.836217in}{2.283047in}}%
\pgfpathlineto{\pgfqpoint{5.836261in}{2.488997in}}%
\pgfpathlineto{\pgfqpoint{5.836612in}{1.964214in}}%
\pgfpathlineto{\pgfqpoint{5.837329in}{2.401954in}}%
\pgfpathlineto{\pgfqpoint{5.837723in}{1.922543in}}%
\pgfpathlineto{\pgfqpoint{5.837657in}{2.482463in}}%
\pgfpathlineto{\pgfqpoint{5.838441in}{2.339811in}}%
\pgfpathlineto{\pgfqpoint{5.839469in}{2.474653in}}%
\pgfpathlineto{\pgfqpoint{5.839145in}{1.836168in}}%
\pgfpathlineto{\pgfqpoint{5.839528in}{2.346379in}}%
\pgfpathlineto{\pgfqpoint{5.840334in}{1.881819in}}%
\pgfpathlineto{\pgfqpoint{5.839889in}{2.488570in}}%
\pgfpathlineto{\pgfqpoint{5.840638in}{2.383255in}}%
\pgfpathlineto{\pgfqpoint{5.841010in}{2.499027in}}%
\pgfpathlineto{\pgfqpoint{5.841044in}{1.777883in}}%
\pgfpathlineto{\pgfqpoint{5.841726in}{2.390244in}}%
\pgfpathlineto{\pgfqpoint{5.841871in}{1.762087in}}%
\pgfpathlineto{\pgfqpoint{5.842218in}{2.469734in}}%
\pgfpathlineto{\pgfqpoint{5.842836in}{2.342347in}}%
\pgfpathlineto{\pgfqpoint{5.843191in}{2.474058in}}%
\pgfpathlineto{\pgfqpoint{5.842975in}{1.564803in}}%
\pgfpathlineto{\pgfqpoint{5.843931in}{2.366148in}}%
\pgfpathlineto{\pgfqpoint{5.844555in}{1.846684in}}%
\pgfpathlineto{\pgfqpoint{5.844980in}{2.458530in}}%
\pgfpathlineto{\pgfqpoint{5.845041in}{2.359253in}}%
\pgfpathlineto{\pgfqpoint{5.845394in}{1.840951in}}%
\pgfpathlineto{\pgfqpoint{5.845274in}{2.484576in}}%
\pgfpathlineto{\pgfqpoint{5.846161in}{2.203945in}}%
\pgfpathlineto{\pgfqpoint{5.846369in}{2.490112in}}%
\pgfpathlineto{\pgfqpoint{5.846691in}{1.934902in}}%
\pgfpathlineto{\pgfqpoint{5.847275in}{2.360670in}}%
\pgfpathlineto{\pgfqpoint{5.848132in}{1.999640in}}%
\pgfpathlineto{\pgfqpoint{5.848154in}{2.466497in}}%
\pgfpathlineto{\pgfqpoint{5.848376in}{2.296892in}}%
\pgfpathlineto{\pgfqpoint{5.849004in}{2.482762in}}%
\pgfpathlineto{\pgfqpoint{5.848711in}{2.021862in}}%
\pgfpathlineto{\pgfqpoint{5.849487in}{2.346619in}}%
\pgfpathlineto{\pgfqpoint{5.849788in}{1.793590in}}%
\pgfpathlineto{\pgfqpoint{5.850486in}{2.475527in}}%
\pgfpathlineto{\pgfqpoint{5.850592in}{2.356024in}}%
\pgfpathlineto{\pgfqpoint{5.851446in}{2.501881in}}%
\pgfpathlineto{\pgfqpoint{5.851259in}{1.895016in}}%
\pgfpathlineto{\pgfqpoint{5.851679in}{2.287255in}}%
\pgfpathlineto{\pgfqpoint{5.851924in}{1.495821in}}%
\pgfpathlineto{\pgfqpoint{5.852269in}{2.477710in}}%
\pgfpathlineto{\pgfqpoint{5.852788in}{2.406510in}}%
\pgfpathlineto{\pgfqpoint{5.853716in}{1.952509in}}%
\pgfpathlineto{\pgfqpoint{5.853171in}{2.473297in}}%
\pgfpathlineto{\pgfqpoint{5.853908in}{2.045014in}}%
\pgfpathlineto{\pgfqpoint{5.854218in}{2.505394in}}%
\pgfpathlineto{\pgfqpoint{5.854690in}{1.937404in}}%
\pgfpathlineto{\pgfqpoint{5.855020in}{2.430377in}}%
\pgfpathlineto{\pgfqpoint{5.855943in}{1.854418in}}%
\pgfpathlineto{\pgfqpoint{5.855584in}{2.480300in}}%
\pgfpathlineto{\pgfqpoint{5.856129in}{2.394171in}}%
\pgfpathlineto{\pgfqpoint{5.857121in}{1.876680in}}%
\pgfpathlineto{\pgfqpoint{5.856281in}{2.513782in}}%
\pgfpathlineto{\pgfqpoint{5.857238in}{2.352551in}}%
\pgfpathlineto{\pgfqpoint{5.857971in}{2.483936in}}%
\pgfpathlineto{\pgfqpoint{5.858296in}{1.725389in}}%
\pgfpathlineto{\pgfqpoint{5.858344in}{2.390042in}}%
\pgfpathlineto{\pgfqpoint{5.858647in}{1.898823in}}%
\pgfpathlineto{\pgfqpoint{5.859265in}{2.492170in}}%
\pgfpathlineto{\pgfqpoint{5.859455in}{2.257613in}}%
\pgfpathlineto{\pgfqpoint{5.860360in}{2.491580in}}%
\pgfpathlineto{\pgfqpoint{5.860188in}{1.990582in}}%
\pgfpathlineto{\pgfqpoint{5.860559in}{2.265014in}}%
\pgfpathlineto{\pgfqpoint{5.861369in}{1.882582in}}%
\pgfpathlineto{\pgfqpoint{5.861204in}{2.504107in}}%
\pgfpathlineto{\pgfqpoint{5.861665in}{2.284818in}}%
\pgfpathlineto{\pgfqpoint{5.862125in}{2.466670in}}%
\pgfpathlineto{\pgfqpoint{5.861839in}{1.924268in}}%
\pgfpathlineto{\pgfqpoint{5.862774in}{2.209016in}}%
\pgfpathlineto{\pgfqpoint{5.863713in}{2.461049in}}%
\pgfpathlineto{\pgfqpoint{5.863277in}{1.887258in}}%
\pgfpathlineto{\pgfqpoint{5.863889in}{2.312607in}}%
\pgfpathlineto{\pgfqpoint{5.864039in}{2.486978in}}%
\pgfpathlineto{\pgfqpoint{5.864203in}{1.946142in}}%
\pgfpathlineto{\pgfqpoint{5.865001in}{2.355222in}}%
\pgfpathlineto{\pgfqpoint{5.866035in}{1.991281in}}%
\pgfpathlineto{\pgfqpoint{5.865608in}{2.487374in}}%
\pgfpathlineto{\pgfqpoint{5.866111in}{2.370739in}}%
\pgfpathlineto{\pgfqpoint{5.867019in}{1.887343in}}%
\pgfpathlineto{\pgfqpoint{5.866721in}{2.521179in}}%
\pgfpathlineto{\pgfqpoint{5.867226in}{2.247981in}}%
\pgfpathlineto{\pgfqpoint{5.867546in}{2.508711in}}%
\pgfpathlineto{\pgfqpoint{5.867363in}{1.937966in}}%
\pgfpathlineto{\pgfqpoint{5.868332in}{2.371751in}}%
\pgfpathlineto{\pgfqpoint{5.869276in}{1.779267in}}%
\pgfpathlineto{\pgfqpoint{5.869343in}{2.488539in}}%
\pgfpathlineto{\pgfqpoint{5.869441in}{2.311078in}}%
\pgfpathlineto{\pgfqpoint{5.870001in}{1.959872in}}%
\pgfpathlineto{\pgfqpoint{5.869856in}{2.520194in}}%
\pgfpathlineto{\pgfqpoint{5.870519in}{2.375980in}}%
\pgfpathlineto{\pgfqpoint{5.871412in}{2.479954in}}%
\pgfpathlineto{\pgfqpoint{5.870739in}{1.897956in}}%
\pgfpathlineto{\pgfqpoint{5.871626in}{2.445613in}}%
\pgfpathlineto{\pgfqpoint{5.872608in}{1.867399in}}%
\pgfpathlineto{\pgfqpoint{5.872047in}{2.470817in}}%
\pgfpathlineto{\pgfqpoint{5.872738in}{2.289747in}}%
\pgfpathlineto{\pgfqpoint{5.873376in}{2.501080in}}%
\pgfpathlineto{\pgfqpoint{5.873445in}{1.956610in}}%
\pgfpathlineto{\pgfqpoint{5.873850in}{2.384266in}}%
\pgfpathlineto{\pgfqpoint{5.874873in}{1.880415in}}%
\pgfpathlineto{\pgfqpoint{5.874197in}{2.488990in}}%
\pgfpathlineto{\pgfqpoint{5.874965in}{2.260106in}}%
\pgfpathlineto{\pgfqpoint{5.875800in}{2.506436in}}%
\pgfpathlineto{\pgfqpoint{5.875566in}{1.823387in}}%
\pgfpathlineto{\pgfqpoint{5.876072in}{2.270305in}}%
\pgfpathlineto{\pgfqpoint{5.876639in}{1.871098in}}%
\pgfpathlineto{\pgfqpoint{5.876197in}{2.489457in}}%
\pgfpathlineto{\pgfqpoint{5.877178in}{2.231253in}}%
\pgfpathlineto{\pgfqpoint{5.878023in}{2.476398in}}%
\pgfpathlineto{\pgfqpoint{5.877386in}{1.998798in}}%
\pgfpathlineto{\pgfqpoint{5.878287in}{2.281761in}}%
\pgfpathlineto{\pgfqpoint{5.879232in}{1.983701in}}%
\pgfpathlineto{\pgfqpoint{5.879053in}{2.499486in}}%
\pgfpathlineto{\pgfqpoint{5.879393in}{2.294664in}}%
\pgfpathlineto{\pgfqpoint{5.879796in}{2.487844in}}%
\pgfpathlineto{\pgfqpoint{5.879691in}{1.889349in}}%
\pgfpathlineto{\pgfqpoint{5.880505in}{2.331785in}}%
\pgfpathlineto{\pgfqpoint{5.880858in}{1.861317in}}%
\pgfpathlineto{\pgfqpoint{5.881158in}{2.489560in}}%
\pgfpathlineto{\pgfqpoint{5.881614in}{2.143894in}}%
\pgfpathlineto{\pgfqpoint{5.882099in}{2.479652in}}%
\pgfpathlineto{\pgfqpoint{5.882597in}{1.711938in}}%
\pgfpathlineto{\pgfqpoint{5.882726in}{2.409047in}}%
\pgfpathlineto{\pgfqpoint{5.883602in}{1.647871in}}%
\pgfpathlineto{\pgfqpoint{5.883686in}{2.511082in}}%
\pgfpathlineto{\pgfqpoint{5.883838in}{2.355444in}}%
\pgfpathlineto{\pgfqpoint{5.884406in}{1.929207in}}%
\pgfpathlineto{\pgfqpoint{5.884107in}{2.506495in}}%
\pgfpathlineto{\pgfqpoint{5.884946in}{2.332320in}}%
\pgfpathlineto{\pgfqpoint{5.885033in}{2.485449in}}%
\pgfpathlineto{\pgfqpoint{5.886013in}{1.947167in}}%
\pgfpathlineto{\pgfqpoint{5.886052in}{2.262348in}}%
\pgfpathlineto{\pgfqpoint{5.886729in}{1.673233in}}%
\pgfpathlineto{\pgfqpoint{5.886273in}{2.491486in}}%
\pgfpathlineto{\pgfqpoint{5.887158in}{2.214886in}}%
\pgfpathlineto{\pgfqpoint{5.888089in}{2.498506in}}%
\pgfpathlineto{\pgfqpoint{5.888136in}{1.983181in}}%
\pgfpathlineto{\pgfqpoint{5.888269in}{2.433245in}}%
\pgfpathlineto{\pgfqpoint{5.888427in}{1.901300in}}%
\pgfpathlineto{\pgfqpoint{5.888502in}{2.480218in}}%
\pgfpathlineto{\pgfqpoint{5.889380in}{2.340007in}}%
\pgfpathlineto{\pgfqpoint{5.889577in}{2.489534in}}%
\pgfpathlineto{\pgfqpoint{5.890218in}{1.909237in}}%
\pgfpathlineto{\pgfqpoint{5.890477in}{2.231917in}}%
\pgfpathlineto{\pgfqpoint{5.890535in}{1.925236in}}%
\pgfpathlineto{\pgfqpoint{5.891426in}{2.494156in}}%
\pgfpathlineto{\pgfqpoint{5.891585in}{2.066117in}}%
\pgfpathlineto{\pgfqpoint{5.891596in}{2.516263in}}%
\pgfpathlineto{\pgfqpoint{5.892147in}{1.584326in}}%
\pgfpathlineto{\pgfqpoint{5.892699in}{2.395794in}}%
\pgfpathlineto{\pgfqpoint{5.893656in}{1.720999in}}%
\pgfpathlineto{\pgfqpoint{5.893766in}{2.496343in}}%
\pgfpathlineto{\pgfqpoint{5.893818in}{2.297190in}}%
\pgfpathlineto{\pgfqpoint{5.894037in}{2.501229in}}%
\pgfpathlineto{\pgfqpoint{5.894786in}{1.855077in}}%
\pgfpathlineto{\pgfqpoint{5.894911in}{2.371600in}}%
\pgfpathlineto{\pgfqpoint{5.895654in}{1.940416in}}%
\pgfpathlineto{\pgfqpoint{5.895053in}{2.503201in}}%
\pgfpathlineto{\pgfqpoint{5.896022in}{2.390739in}}%
\pgfpathlineto{\pgfqpoint{5.896027in}{2.439365in}}%
\pgfpathlineto{\pgfqpoint{5.896063in}{2.204800in}}%
\pgfpathlineto{\pgfqpoint{5.896068in}{2.223332in}}%
\pgfpathlineto{\pgfqpoint{5.896680in}{1.952999in}}%
\pgfpathlineto{\pgfqpoint{5.896743in}{2.480849in}}%
\pgfpathlineto{\pgfqpoint{5.897172in}{2.367206in}}%
\pgfpathlineto{\pgfqpoint{5.898016in}{2.482281in}}%
\pgfpathlineto{\pgfqpoint{5.897856in}{1.786744in}}%
\pgfpathlineto{\pgfqpoint{5.898166in}{2.305265in}}%
\pgfpathlineto{\pgfqpoint{5.898168in}{1.592740in}}%
\pgfpathlineto{\pgfqpoint{5.898258in}{2.478765in}}%
\pgfpathlineto{\pgfqpoint{5.899275in}{1.898242in}}%
\pgfpathlineto{\pgfqpoint{5.899972in}{2.511882in}}%
\pgfpathlineto{\pgfqpoint{5.900388in}{2.202911in}}%
\pgfpathlineto{\pgfqpoint{5.900859in}{2.496290in}}%
\pgfpathlineto{\pgfqpoint{5.900474in}{1.866604in}}%
\pgfpathlineto{\pgfqpoint{5.901505in}{2.429596in}}%
\pgfpathlineto{\pgfqpoint{5.901653in}{1.928352in}}%
\pgfpathlineto{\pgfqpoint{5.902284in}{2.490305in}}%
\pgfpathlineto{\pgfqpoint{5.902617in}{2.228581in}}%
\pgfpathlineto{\pgfqpoint{5.903490in}{2.487871in}}%
\pgfpathlineto{\pgfqpoint{5.903268in}{1.974037in}}%
\pgfpathlineto{\pgfqpoint{5.903729in}{2.351148in}}%
\pgfpathlineto{\pgfqpoint{5.904781in}{2.488380in}}%
\pgfpathlineto{\pgfqpoint{5.904229in}{1.973098in}}%
\pgfpathlineto{\pgfqpoint{5.904821in}{2.308511in}}%
\pgfpathlineto{\pgfqpoint{5.905730in}{1.854236in}}%
\pgfpathlineto{\pgfqpoint{5.905776in}{2.500020in}}%
\pgfpathlineto{\pgfqpoint{5.905933in}{2.228430in}}%
\pgfpathlineto{\pgfqpoint{5.906961in}{2.503205in}}%
\pgfpathlineto{\pgfqpoint{5.906754in}{1.910272in}}%
\pgfpathlineto{\pgfqpoint{5.907041in}{2.405517in}}%
\pgfpathlineto{\pgfqpoint{5.907426in}{1.967667in}}%
\pgfpathlineto{\pgfqpoint{5.908125in}{2.506624in}}%
\pgfpathlineto{\pgfqpoint{5.908152in}{2.336868in}}%
\pgfpathlineto{\pgfqpoint{5.908704in}{1.882364in}}%
\pgfpathlineto{\pgfqpoint{5.908900in}{2.487286in}}%
\pgfpathlineto{\pgfqpoint{5.909259in}{2.399577in}}%
\pgfpathlineto{\pgfqpoint{5.909346in}{1.570776in}}%
\pgfpathlineto{\pgfqpoint{5.910114in}{2.501459in}}%
\pgfpathlineto{\pgfqpoint{5.910370in}{2.285537in}}%
\pgfpathlineto{\pgfqpoint{5.911345in}{2.484695in}}%
\pgfpathlineto{\pgfqpoint{5.910906in}{1.894003in}}%
\pgfpathlineto{\pgfqpoint{5.911479in}{2.343248in}}%
\pgfpathlineto{\pgfqpoint{5.912201in}{1.805157in}}%
\pgfpathlineto{\pgfqpoint{5.912151in}{2.502988in}}%
\pgfpathlineto{\pgfqpoint{5.912592in}{2.111605in}}%
\pgfpathlineto{\pgfqpoint{5.913445in}{2.490819in}}%
\pgfpathlineto{\pgfqpoint{5.912670in}{1.968845in}}%
\pgfpathlineto{\pgfqpoint{5.913704in}{2.274447in}}%
\pgfpathlineto{\pgfqpoint{5.914526in}{2.492795in}}%
\pgfpathlineto{\pgfqpoint{5.914195in}{1.873758in}}%
\pgfpathlineto{\pgfqpoint{5.914794in}{2.246204in}}%
\pgfpathlineto{\pgfqpoint{5.915781in}{2.009135in}}%
\pgfpathlineto{\pgfqpoint{5.914935in}{2.486582in}}%
\pgfpathlineto{\pgfqpoint{5.915906in}{2.192690in}}%
\pgfpathlineto{\pgfqpoint{5.916371in}{2.495416in}}%
\pgfpathlineto{\pgfqpoint{5.916633in}{1.853861in}}%
\pgfpathlineto{\pgfqpoint{5.917017in}{2.285975in}}%
\pgfpathlineto{\pgfqpoint{5.917315in}{1.851511in}}%
\pgfpathlineto{\pgfqpoint{5.917095in}{2.495570in}}%
\pgfpathlineto{\pgfqpoint{5.918123in}{2.394605in}}%
\pgfpathlineto{\pgfqpoint{5.918908in}{2.507685in}}%
\pgfpathlineto{\pgfqpoint{5.918843in}{1.835427in}}%
\pgfpathlineto{\pgfqpoint{5.919140in}{2.273216in}}%
\pgfpathlineto{\pgfqpoint{5.919883in}{1.972029in}}%
\pgfpathlineto{\pgfqpoint{5.920029in}{2.506539in}}%
\pgfpathlineto{\pgfqpoint{5.920248in}{2.257131in}}%
\pgfpathlineto{\pgfqpoint{5.920749in}{2.496455in}}%
\pgfpathlineto{\pgfqpoint{5.920405in}{1.812514in}}%
\pgfpathlineto{\pgfqpoint{5.921358in}{2.386186in}}%
\pgfpathlineto{\pgfqpoint{5.921994in}{1.924625in}}%
\pgfpathlineto{\pgfqpoint{5.921884in}{2.521760in}}%
\pgfpathlineto{\pgfqpoint{5.922468in}{2.058148in}}%
\pgfpathlineto{\pgfqpoint{5.923199in}{2.503173in}}%
\pgfpathlineto{\pgfqpoint{5.923580in}{2.434660in}}%
\pgfpathlineto{\pgfqpoint{5.924248in}{1.978879in}}%
\pgfpathlineto{\pgfqpoint{5.923817in}{2.514975in}}%
\pgfpathlineto{\pgfqpoint{5.924691in}{2.245390in}}%
\pgfpathlineto{\pgfqpoint{5.925249in}{2.494295in}}%
\pgfpathlineto{\pgfqpoint{5.925040in}{1.860659in}}%
\pgfpathlineto{\pgfqpoint{5.925800in}{2.396321in}}%
\pgfpathlineto{\pgfqpoint{5.926143in}{1.826087in}}%
\pgfpathlineto{\pgfqpoint{5.926805in}{2.520261in}}%
\pgfpathlineto{\pgfqpoint{5.926911in}{2.238118in}}%
\pgfpathlineto{\pgfqpoint{5.927814in}{1.803757in}}%
\pgfpathlineto{\pgfqpoint{5.926949in}{2.489968in}}%
\pgfpathlineto{\pgfqpoint{5.928014in}{2.285198in}}%
\pgfpathlineto{\pgfqpoint{5.928100in}{2.516444in}}%
\pgfpathlineto{\pgfqpoint{5.928506in}{1.673058in}}%
\pgfpathlineto{\pgfqpoint{5.929124in}{2.283971in}}%
\pgfpathlineto{\pgfqpoint{5.929655in}{1.784197in}}%
\pgfpathlineto{\pgfqpoint{5.929159in}{2.493492in}}%
\pgfpathlineto{\pgfqpoint{5.930231in}{2.316733in}}%
\pgfpathlineto{\pgfqpoint{5.930826in}{2.495407in}}%
\pgfpathlineto{\pgfqpoint{5.930708in}{1.916400in}}%
\pgfpathlineto{\pgfqpoint{5.931341in}{2.469744in}}%
\pgfpathlineto{\pgfqpoint{5.931701in}{1.853770in}}%
\pgfpathlineto{\pgfqpoint{5.931506in}{2.498959in}}%
\pgfpathlineto{\pgfqpoint{5.932455in}{2.273056in}}%
\pgfpathlineto{\pgfqpoint{5.933493in}{1.929483in}}%
\pgfpathlineto{\pgfqpoint{5.932767in}{2.508444in}}%
\pgfpathlineto{\pgfqpoint{5.933553in}{2.346799in}}%
\pgfpathlineto{\pgfqpoint{5.934052in}{2.486996in}}%
\pgfpathlineto{\pgfqpoint{5.934273in}{1.865140in}}%
\pgfpathlineto{\pgfqpoint{5.934666in}{2.453859in}}%
\pgfpathlineto{\pgfqpoint{5.935029in}{2.496782in}}%
\pgfpathlineto{\pgfqpoint{5.935779in}{1.759119in}}%
\pgfpathlineto{\pgfqpoint{5.936867in}{2.500220in}}%
\pgfpathlineto{\pgfqpoint{5.936891in}{2.360462in}}%
\pgfpathlineto{\pgfqpoint{5.936901in}{1.940617in}}%
\pgfpathlineto{\pgfqpoint{5.937719in}{2.484265in}}%
\pgfpathlineto{\pgfqpoint{5.938003in}{2.267683in}}%
\pgfpathlineto{\pgfqpoint{5.939110in}{2.496276in}}%
\pgfpathlineto{\pgfqpoint{5.938733in}{1.950843in}}%
\pgfpathlineto{\pgfqpoint{5.939115in}{2.403557in}}%
\pgfpathlineto{\pgfqpoint{5.939552in}{1.826083in}}%
\pgfpathlineto{\pgfqpoint{5.940125in}{2.495289in}}%
\pgfpathlineto{\pgfqpoint{5.940228in}{2.225329in}}%
\pgfpathlineto{\pgfqpoint{5.940944in}{2.528185in}}%
\pgfpathlineto{\pgfqpoint{5.940429in}{1.941375in}}%
\pgfpathlineto{\pgfqpoint{5.941336in}{2.372621in}}%
\pgfpathlineto{\pgfqpoint{5.942263in}{1.836129in}}%
\pgfpathlineto{\pgfqpoint{5.942040in}{2.515662in}}%
\pgfpathlineto{\pgfqpoint{5.942446in}{2.201364in}}%
\pgfpathlineto{\pgfqpoint{5.943448in}{2.503540in}}%
\pgfpathlineto{\pgfqpoint{5.942495in}{1.854069in}}%
\pgfpathlineto{\pgfqpoint{5.943563in}{2.489231in}}%
\pgfpathlineto{\pgfqpoint{5.943658in}{1.849021in}}%
\pgfpathlineto{\pgfqpoint{5.943892in}{2.525162in}}%
\pgfpathlineto{\pgfqpoint{5.944677in}{2.323449in}}%
\pgfpathlineto{\pgfqpoint{5.945094in}{2.489871in}}%
\pgfpathlineto{\pgfqpoint{5.945010in}{1.851803in}}%
\pgfpathlineto{\pgfqpoint{5.945662in}{2.381711in}}%
\pgfpathlineto{\pgfqpoint{5.945681in}{1.841178in}}%
\pgfpathlineto{\pgfqpoint{5.946037in}{2.503308in}}%
\pgfpathlineto{\pgfqpoint{5.946773in}{2.367235in}}%
\pgfpathlineto{\pgfqpoint{5.947367in}{1.876346in}}%
\pgfpathlineto{\pgfqpoint{5.947104in}{2.485335in}}%
\pgfpathlineto{\pgfqpoint{5.947883in}{2.338539in}}%
\pgfpathlineto{\pgfqpoint{5.948131in}{1.809181in}}%
\pgfpathlineto{\pgfqpoint{5.947992in}{2.500517in}}%
\pgfpathlineto{\pgfqpoint{5.948988in}{2.257012in}}%
\pgfpathlineto{\pgfqpoint{5.949656in}{2.510308in}}%
\pgfpathlineto{\pgfqpoint{5.949882in}{1.873653in}}%
\pgfpathlineto{\pgfqpoint{5.950100in}{2.334454in}}%
\pgfpathlineto{\pgfqpoint{5.951065in}{2.523943in}}%
\pgfpathlineto{\pgfqpoint{5.950963in}{1.955321in}}%
\pgfpathlineto{\pgfqpoint{5.951207in}{2.314571in}}%
\pgfpathlineto{\pgfqpoint{5.951974in}{1.960415in}}%
\pgfpathlineto{\pgfqpoint{5.951723in}{2.505318in}}%
\pgfpathlineto{\pgfqpoint{5.952310in}{2.307070in}}%
\pgfpathlineto{\pgfqpoint{5.952537in}{2.501189in}}%
\pgfpathlineto{\pgfqpoint{5.953013in}{1.919293in}}%
\pgfpathlineto{\pgfqpoint{5.953423in}{2.406544in}}%
\pgfpathlineto{\pgfqpoint{5.953775in}{1.912774in}}%
\pgfpathlineto{\pgfqpoint{5.954158in}{2.532415in}}%
\pgfpathlineto{\pgfqpoint{5.954535in}{2.329196in}}%
\pgfpathlineto{\pgfqpoint{5.954806in}{1.885237in}}%
\pgfpathlineto{\pgfqpoint{5.955495in}{2.495227in}}%
\pgfpathlineto{\pgfqpoint{5.955609in}{2.342754in}}%
\pgfpathlineto{\pgfqpoint{5.956297in}{2.498852in}}%
\pgfpathlineto{\pgfqpoint{5.955748in}{1.822895in}}%
\pgfpathlineto{\pgfqpoint{5.956718in}{2.336651in}}%
\pgfpathlineto{\pgfqpoint{5.957013in}{1.878184in}}%
\pgfpathlineto{\pgfqpoint{5.957753in}{2.499386in}}%
\pgfpathlineto{\pgfqpoint{5.957831in}{2.215443in}}%
\pgfpathlineto{\pgfqpoint{5.957944in}{2.499748in}}%
\pgfpathlineto{\pgfqpoint{5.958551in}{1.904300in}}%
\pgfpathlineto{\pgfqpoint{5.958943in}{2.333767in}}%
\pgfpathlineto{\pgfqpoint{5.959790in}{1.896413in}}%
\pgfpathlineto{\pgfqpoint{5.959628in}{2.515840in}}%
\pgfpathlineto{\pgfqpoint{5.960055in}{2.235209in}}%
\pgfpathlineto{\pgfqpoint{5.960570in}{2.482140in}}%
\pgfpathlineto{\pgfqpoint{5.960186in}{1.949287in}}%
\pgfpathlineto{\pgfqpoint{5.961166in}{2.293750in}}%
\pgfpathlineto{\pgfqpoint{5.961946in}{2.498394in}}%
\pgfpathlineto{\pgfqpoint{5.961717in}{1.922104in}}%
\pgfpathlineto{\pgfqpoint{5.962277in}{2.406023in}}%
\pgfpathlineto{\pgfqpoint{5.962340in}{1.732666in}}%
\pgfpathlineto{\pgfqpoint{5.962961in}{2.507693in}}%
\pgfpathlineto{\pgfqpoint{5.963387in}{2.320875in}}%
\pgfpathlineto{\pgfqpoint{5.964197in}{2.510202in}}%
\pgfpathlineto{\pgfqpoint{5.963784in}{1.896009in}}%
\pgfpathlineto{\pgfqpoint{5.964490in}{2.396200in}}%
\pgfpathlineto{\pgfqpoint{5.964911in}{1.876604in}}%
\pgfpathlineto{\pgfqpoint{5.965381in}{2.514223in}}%
\pgfpathlineto{\pgfqpoint{5.965601in}{2.313360in}}%
\pgfpathlineto{\pgfqpoint{5.965966in}{1.640946in}}%
\pgfpathlineto{\pgfqpoint{5.966368in}{2.506590in}}%
\pgfpathlineto{\pgfqpoint{5.966677in}{2.229256in}}%
\pgfpathlineto{\pgfqpoint{5.967156in}{2.539699in}}%
\pgfpathlineto{\pgfqpoint{5.967552in}{1.776505in}}%
\pgfpathlineto{\pgfqpoint{5.967787in}{2.250867in}}%
\pgfpathlineto{\pgfqpoint{5.968013in}{1.666123in}}%
\pgfpathlineto{\pgfqpoint{5.968015in}{2.490913in}}%
\pgfpathlineto{\pgfqpoint{5.968881in}{2.397922in}}%
\pgfpathlineto{\pgfqpoint{5.969298in}{2.502857in}}%
\pgfpathlineto{\pgfqpoint{5.968970in}{1.850956in}}%
\pgfpathlineto{\pgfqpoint{5.969452in}{2.431413in}}%
\pgfpathlineto{\pgfqpoint{5.969497in}{1.722961in}}%
\pgfpathlineto{\pgfqpoint{5.970445in}{2.518558in}}%
\pgfpathlineto{\pgfqpoint{5.970561in}{1.964483in}}%
\pgfpathlineto{\pgfqpoint{5.970564in}{2.504976in}}%
\pgfpathlineto{\pgfqpoint{5.970708in}{1.792822in}}%
\pgfpathlineto{\pgfqpoint{5.971673in}{2.440729in}}%
\pgfpathlineto{\pgfqpoint{5.972308in}{1.903121in}}%
\pgfpathlineto{\pgfqpoint{5.971894in}{2.494006in}}%
\pgfpathlineto{\pgfqpoint{5.972784in}{2.329703in}}%
\pgfpathlineto{\pgfqpoint{5.973359in}{2.507568in}}%
\pgfpathlineto{\pgfqpoint{5.973790in}{1.810609in}}%
\pgfpathlineto{\pgfqpoint{5.973894in}{2.408614in}}%
\pgfpathlineto{\pgfqpoint{5.974374in}{1.992056in}}%
\pgfpathlineto{\pgfqpoint{5.974352in}{2.505317in}}%
\pgfpathlineto{\pgfqpoint{5.975004in}{2.178633in}}%
\pgfpathlineto{\pgfqpoint{5.975995in}{2.501575in}}%
\pgfpathlineto{\pgfqpoint{5.976029in}{1.951647in}}%
\pgfpathlineto{\pgfqpoint{5.976115in}{2.415487in}}%
\pgfpathlineto{\pgfqpoint{5.976320in}{2.501198in}}%
\pgfpathlineto{\pgfqpoint{5.976309in}{1.888663in}}%
\pgfpathlineto{\pgfqpoint{5.977212in}{2.388769in}}%
\pgfpathlineto{\pgfqpoint{5.977832in}{1.985173in}}%
\pgfpathlineto{\pgfqpoint{5.977336in}{2.525582in}}%
\pgfpathlineto{\pgfqpoint{5.978322in}{2.278717in}}%
\pgfpathlineto{\pgfqpoint{5.978400in}{2.507697in}}%
\pgfpathlineto{\pgfqpoint{5.978378in}{1.629024in}}%
\pgfpathlineto{\pgfqpoint{5.979433in}{2.395143in}}%
\pgfpathlineto{\pgfqpoint{5.979751in}{1.880664in}}%
\pgfpathlineto{\pgfqpoint{5.980459in}{2.484126in}}%
\pgfpathlineto{\pgfqpoint{5.980540in}{2.392939in}}%
\pgfpathlineto{\pgfqpoint{5.981222in}{2.499021in}}%
\pgfpathlineto{\pgfqpoint{5.981309in}{1.765314in}}%
\pgfpathlineto{\pgfqpoint{5.981623in}{2.393307in}}%
\pgfpathlineto{\pgfqpoint{5.981872in}{1.980357in}}%
\pgfpathlineto{\pgfqpoint{5.982508in}{2.506797in}}%
\pgfpathlineto{\pgfqpoint{5.982733in}{2.222360in}}%
\pgfpathlineto{\pgfqpoint{5.982979in}{2.511942in}}%
\pgfpathlineto{\pgfqpoint{5.983079in}{1.863379in}}%
\pgfpathlineto{\pgfqpoint{5.983846in}{2.468064in}}%
\pgfpathlineto{\pgfqpoint{5.983868in}{1.711999in}}%
\pgfpathlineto{\pgfqpoint{5.984713in}{2.515253in}}%
\pgfpathlineto{\pgfqpoint{5.984956in}{2.362082in}}%
\pgfpathlineto{\pgfqpoint{5.985409in}{2.505118in}}%
\pgfpathlineto{\pgfqpoint{5.985687in}{1.851921in}}%
\pgfpathlineto{\pgfqpoint{5.986057in}{2.187406in}}%
\pgfpathlineto{\pgfqpoint{5.986757in}{1.868413in}}%
\pgfpathlineto{\pgfqpoint{5.986872in}{2.520414in}}%
\pgfpathlineto{\pgfqpoint{5.987164in}{2.086909in}}%
\pgfpathlineto{\pgfqpoint{5.987184in}{2.496247in}}%
\pgfpathlineto{\pgfqpoint{5.987395in}{1.587895in}}%
\pgfpathlineto{\pgfqpoint{5.988277in}{2.406035in}}%
\pgfpathlineto{\pgfqpoint{5.988468in}{1.860570in}}%
\pgfpathlineto{\pgfqpoint{5.989102in}{2.505624in}}%
\pgfpathlineto{\pgfqpoint{5.989387in}{2.143379in}}%
\pgfpathlineto{\pgfqpoint{5.989649in}{2.541177in}}%
\pgfpathlineto{\pgfqpoint{5.989891in}{1.901308in}}%
\pgfpathlineto{\pgfqpoint{5.990498in}{2.329730in}}%
\pgfpathlineto{\pgfqpoint{5.990518in}{1.952255in}}%
\pgfpathlineto{\pgfqpoint{5.990899in}{2.508150in}}%
\pgfpathlineto{\pgfqpoint{5.991606in}{2.338837in}}%
\pgfpathlineto{\pgfqpoint{5.991978in}{2.518846in}}%
\pgfpathlineto{\pgfqpoint{5.992601in}{1.908772in}}%
\pgfpathlineto{\pgfqpoint{5.992714in}{2.447416in}}%
\pgfpathlineto{\pgfqpoint{5.993765in}{1.777676in}}%
\pgfpathlineto{\pgfqpoint{5.992963in}{2.496571in}}%
\pgfpathlineto{\pgfqpoint{5.993825in}{2.419927in}}%
\pgfpathlineto{\pgfqpoint{5.994467in}{2.522695in}}%
\pgfpathlineto{\pgfqpoint{5.994111in}{1.965687in}}%
\pgfpathlineto{\pgfqpoint{5.994929in}{2.363074in}}%
\pgfpathlineto{\pgfqpoint{5.995128in}{1.916283in}}%
\pgfpathlineto{\pgfqpoint{5.995305in}{2.526273in}}%
\pgfpathlineto{\pgfqpoint{5.996041in}{2.098684in}}%
\pgfpathlineto{\pgfqpoint{5.996186in}{2.513557in}}%
\pgfpathlineto{\pgfqpoint{5.996957in}{1.889575in}}%
\pgfpathlineto{\pgfqpoint{5.997153in}{2.276257in}}%
\pgfpathlineto{\pgfqpoint{5.997208in}{2.487263in}}%
\pgfpathlineto{\pgfqpoint{5.998066in}{1.992459in}}%
\pgfpathlineto{\pgfqpoint{5.998265in}{2.393185in}}%
\pgfpathlineto{\pgfqpoint{5.999345in}{2.488666in}}%
\pgfpathlineto{\pgfqpoint{5.999108in}{1.725154in}}%
\pgfpathlineto{\pgfqpoint{5.999377in}{2.437718in}}%
\pgfpathlineto{\pgfqpoint{5.999846in}{1.929974in}}%
\pgfpathlineto{\pgfqpoint{5.999584in}{2.487825in}}%
\pgfpathlineto{\pgfqpoint{6.000488in}{2.318449in}}%
\pgfpathlineto{\pgfqpoint{6.001065in}{2.501775in}}%
\pgfpathlineto{\pgfqpoint{6.000586in}{1.678583in}}%
\pgfpathlineto{\pgfqpoint{6.001592in}{2.352563in}}%
\pgfpathlineto{\pgfqpoint{6.002642in}{1.877508in}}%
\pgfpathlineto{\pgfqpoint{6.001760in}{2.513269in}}%
\pgfpathlineto{\pgfqpoint{6.002701in}{2.282810in}}%
\pgfpathlineto{\pgfqpoint{6.003034in}{2.503371in}}%
\pgfpathlineto{\pgfqpoint{6.003444in}{1.692217in}}%
\pgfpathlineto{\pgfqpoint{6.003812in}{2.417839in}}%
\pgfpathlineto{\pgfqpoint{6.004192in}{1.872144in}}%
\pgfpathlineto{\pgfqpoint{6.004825in}{2.518354in}}%
\pgfpathlineto{\pgfqpoint{6.004926in}{2.287978in}}%
\pgfpathlineto{\pgfqpoint{6.005356in}{2.516291in}}%
\pgfpathlineto{\pgfqpoint{6.005236in}{1.631557in}}%
\pgfpathlineto{\pgfqpoint{6.006038in}{2.402504in}}%
\pgfpathlineto{\pgfqpoint{6.006972in}{1.966014in}}%
\pgfpathlineto{\pgfqpoint{6.006050in}{2.525351in}}%
\pgfpathlineto{\pgfqpoint{6.007154in}{2.201941in}}%
\pgfpathlineto{\pgfqpoint{6.008001in}{2.515142in}}%
\pgfpathlineto{\pgfqpoint{6.007458in}{1.923349in}}%
\pgfpathlineto{\pgfqpoint{6.008268in}{2.384689in}}%
\pgfpathlineto{\pgfqpoint{6.008440in}{1.905547in}}%
\pgfpathlineto{\pgfqpoint{6.008833in}{2.523367in}}%
\pgfpathlineto{\pgfqpoint{6.009377in}{2.322832in}}%
\pgfpathlineto{\pgfqpoint{6.010383in}{2.510537in}}%
\pgfpathlineto{\pgfqpoint{6.009582in}{1.926737in}}%
\pgfpathlineto{\pgfqpoint{6.010489in}{2.436242in}}%
\pgfpathlineto{\pgfqpoint{6.011152in}{1.724831in}}%
\pgfpathlineto{\pgfqpoint{6.011217in}{2.507823in}}%
\pgfpathlineto{\pgfqpoint{6.011603in}{1.997024in}}%
\pgfpathlineto{\pgfqpoint{6.012200in}{2.493029in}}%
\pgfpathlineto{\pgfqpoint{6.011663in}{1.721775in}}%
\pgfpathlineto{\pgfqpoint{6.012716in}{2.400918in}}%
\pgfpathlineto{\pgfqpoint{6.012947in}{1.952544in}}%
\pgfpathlineto{\pgfqpoint{6.013059in}{2.530117in}}%
\pgfpathlineto{\pgfqpoint{6.013825in}{2.324633in}}%
\pgfpathlineto{\pgfqpoint{6.014015in}{2.540780in}}%
\pgfpathlineto{\pgfqpoint{6.014718in}{1.961426in}}%
\pgfpathlineto{\pgfqpoint{6.014932in}{2.377881in}}%
\pgfpathlineto{\pgfqpoint{6.015646in}{1.924130in}}%
\pgfpathlineto{\pgfqpoint{6.015477in}{2.516151in}}%
\pgfpathlineto{\pgfqpoint{6.016042in}{2.408667in}}%
\pgfpathlineto{\pgfqpoint{6.016856in}{2.496162in}}%
\pgfpathlineto{\pgfqpoint{6.016721in}{1.869790in}}%
\pgfpathlineto{\pgfqpoint{6.017149in}{2.339937in}}%
\pgfpathlineto{\pgfqpoint{6.017792in}{1.895387in}}%
\pgfpathlineto{\pgfqpoint{6.017968in}{2.504261in}}%
\pgfpathlineto{\pgfqpoint{6.018262in}{2.159494in}}%
\pgfpathlineto{\pgfqpoint{6.019072in}{2.507025in}}%
\pgfpathlineto{\pgfqpoint{6.019306in}{1.971071in}}%
\pgfpathlineto{\pgfqpoint{6.019373in}{2.421518in}}%
\pgfpathlineto{\pgfqpoint{6.020153in}{1.800888in}}%
\pgfpathlineto{\pgfqpoint{6.019408in}{2.508049in}}%
\pgfpathlineto{\pgfqpoint{6.020486in}{2.296487in}}%
\pgfpathlineto{\pgfqpoint{6.020722in}{2.512709in}}%
\pgfpathlineto{\pgfqpoint{6.021511in}{1.920662in}}%
\pgfpathlineto{\pgfqpoint{6.021598in}{2.464199in}}%
\pgfpathlineto{\pgfqpoint{6.021738in}{1.979953in}}%
\pgfpathlineto{\pgfqpoint{6.022305in}{2.501608in}}%
\pgfpathlineto{\pgfqpoint{6.022711in}{2.267265in}}%
\pgfpathlineto{\pgfqpoint{6.023093in}{1.704483in}}%
\pgfpathlineto{\pgfqpoint{6.023535in}{2.519441in}}%
\pgfpathlineto{\pgfqpoint{6.023817in}{2.347989in}}%
\pgfpathlineto{\pgfqpoint{6.024130in}{2.505581in}}%
\pgfpathlineto{\pgfqpoint{6.024625in}{1.959504in}}%
\pgfpathlineto{\pgfqpoint{6.024926in}{2.410061in}}%
\pgfpathlineto{\pgfqpoint{6.025441in}{2.008759in}}%
\pgfpathlineto{\pgfqpoint{6.025642in}{2.526978in}}%
\pgfpathlineto{\pgfqpoint{6.026037in}{2.130792in}}%
\pgfpathlineto{\pgfqpoint{6.026257in}{2.500172in}}%
\pgfpathlineto{\pgfqpoint{6.026392in}{1.906047in}}%
\pgfpathlineto{\pgfqpoint{6.027148in}{2.192896in}}%
\pgfpathlineto{\pgfqpoint{6.027824in}{2.509654in}}%
\pgfpathlineto{\pgfqpoint{6.027918in}{1.790248in}}%
\pgfpathlineto{\pgfqpoint{6.028263in}{2.456992in}}%
\pgfpathlineto{\pgfqpoint{6.029050in}{1.829067in}}%
\pgfpathlineto{\pgfqpoint{6.029181in}{2.505121in}}%
\pgfpathlineto{\pgfqpoint{6.029374in}{2.144306in}}%
\pgfpathlineto{\pgfqpoint{6.029422in}{2.551873in}}%
\pgfpathlineto{\pgfqpoint{6.030392in}{1.958748in}}%
\pgfpathlineto{\pgfqpoint{6.030489in}{2.428852in}}%
\pgfpathlineto{\pgfqpoint{6.030580in}{2.489947in}}%
\pgfpathlineto{\pgfqpoint{6.030632in}{2.099806in}}%
\pgfpathlineto{\pgfqpoint{6.030654in}{2.390093in}}%
\pgfpathlineto{\pgfqpoint{6.030656in}{1.733257in}}%
\pgfpathlineto{\pgfqpoint{6.031747in}{2.525443in}}%
\pgfpathlineto{\pgfqpoint{6.031763in}{2.383239in}}%
\pgfpathlineto{\pgfqpoint{6.031894in}{1.826960in}}%
\pgfpathlineto{\pgfqpoint{6.032826in}{2.539685in}}%
\pgfpathlineto{\pgfqpoint{6.032872in}{2.159851in}}%
\pgfpathlineto{\pgfqpoint{6.033264in}{2.504561in}}%
\pgfpathlineto{\pgfqpoint{6.033404in}{1.913465in}}%
\pgfpathlineto{\pgfqpoint{6.033983in}{2.445660in}}%
\pgfpathlineto{\pgfqpoint{6.034079in}{1.978722in}}%
\pgfpathlineto{\pgfqpoint{6.034384in}{2.538699in}}%
\pgfpathlineto{\pgfqpoint{6.035095in}{2.276471in}}%
\pgfpathlineto{\pgfqpoint{6.035901in}{2.515670in}}%
\pgfpathlineto{\pgfqpoint{6.035174in}{1.865675in}}%
\pgfpathlineto{\pgfqpoint{6.036207in}{2.410456in}}%
\pgfpathlineto{\pgfqpoint{6.036706in}{1.946351in}}%
\pgfpathlineto{\pgfqpoint{6.036315in}{2.516716in}}%
\pgfpathlineto{\pgfqpoint{6.037318in}{2.310531in}}%
\pgfpathlineto{\pgfqpoint{6.038318in}{2.514876in}}%
\pgfpathlineto{\pgfqpoint{6.038143in}{1.524469in}}%
\pgfpathlineto{\pgfqpoint{6.038429in}{2.439238in}}%
\pgfpathlineto{\pgfqpoint{6.038707in}{1.720925in}}%
\pgfpathlineto{\pgfqpoint{6.038589in}{2.509531in}}%
\pgfpathlineto{\pgfqpoint{6.039540in}{2.421173in}}%
\pgfpathlineto{\pgfqpoint{6.039676in}{2.509031in}}%
\pgfpathlineto{\pgfqpoint{6.040525in}{1.954942in}}%
\pgfpathlineto{\pgfqpoint{6.040647in}{2.390911in}}%
\pgfpathlineto{\pgfqpoint{6.040802in}{1.874942in}}%
\pgfpathlineto{\pgfqpoint{6.041039in}{2.525681in}}%
\pgfpathlineto{\pgfqpoint{6.041758in}{2.369022in}}%
\pgfpathlineto{\pgfqpoint{6.042140in}{1.960914in}}%
\pgfpathlineto{\pgfqpoint{6.041880in}{2.535623in}}%
\pgfpathlineto{\pgfqpoint{6.042868in}{2.296881in}}%
\pgfpathlineto{\pgfqpoint{6.043919in}{2.537228in}}%
\pgfpathlineto{\pgfqpoint{6.043226in}{1.958922in}}%
\pgfpathlineto{\pgfqpoint{6.043980in}{2.449157in}}%
\pgfpathlineto{\pgfqpoint{6.044689in}{1.788745in}}%
\pgfpathlineto{\pgfqpoint{6.044917in}{2.509288in}}%
\pgfpathlineto{\pgfqpoint{6.045090in}{2.433841in}}%
\pgfpathlineto{\pgfqpoint{6.045684in}{2.520273in}}%
\pgfpathlineto{\pgfqpoint{6.045335in}{1.331554in}}%
\pgfpathlineto{\pgfqpoint{6.046193in}{2.358697in}}%
\pgfpathlineto{\pgfqpoint{6.047169in}{1.949043in}}%
\pgfpathlineto{\pgfqpoint{6.046917in}{2.521488in}}%
\pgfpathlineto{\pgfqpoint{6.047300in}{2.338275in}}%
\pgfpathlineto{\pgfqpoint{6.048388in}{2.544396in}}%
\pgfpathlineto{\pgfqpoint{6.047495in}{1.951319in}}%
\pgfpathlineto{\pgfqpoint{6.048411in}{2.422012in}}%
\pgfpathlineto{\pgfqpoint{6.048934in}{1.765687in}}%
\pgfpathlineto{\pgfqpoint{6.049406in}{2.520181in}}%
\pgfpathlineto{\pgfqpoint{6.049528in}{2.250648in}}%
\pgfpathlineto{\pgfqpoint{6.049968in}{2.504000in}}%
\pgfpathlineto{\pgfqpoint{6.049702in}{1.893408in}}%
\pgfpathlineto{\pgfqpoint{6.050638in}{2.316985in}}%
\pgfpathlineto{\pgfqpoint{6.051683in}{2.520639in}}%
\pgfpathlineto{\pgfqpoint{6.051587in}{1.769700in}}%
\pgfpathlineto{\pgfqpoint{6.051714in}{2.321282in}}%
\pgfpathlineto{\pgfqpoint{6.051715in}{1.908835in}}%
\pgfpathlineto{\pgfqpoint{6.051768in}{2.537501in}}%
\pgfpathlineto{\pgfqpoint{6.052824in}{2.310384in}}%
\pgfpathlineto{\pgfqpoint{6.053163in}{2.512459in}}%
\pgfpathlineto{\pgfqpoint{6.053597in}{1.597698in}}%
\pgfpathlineto{\pgfqpoint{6.053933in}{2.400212in}}%
\pgfpathlineto{\pgfqpoint{6.054113in}{1.795587in}}%
\pgfpathlineto{\pgfqpoint{6.054815in}{2.517743in}}%
\pgfpathlineto{\pgfqpoint{6.055043in}{2.410715in}}%
\pgfpathlineto{\pgfqpoint{6.055509in}{1.644436in}}%
\pgfpathlineto{\pgfqpoint{6.056014in}{2.516054in}}%
\pgfpathlineto{\pgfqpoint{6.056145in}{2.331336in}}%
\pgfpathlineto{\pgfqpoint{6.056345in}{2.527767in}}%
\pgfpathlineto{\pgfqpoint{6.056343in}{1.771001in}}%
\pgfpathlineto{\pgfqpoint{6.057253in}{2.242984in}}%
\pgfpathlineto{\pgfqpoint{6.057657in}{1.725261in}}%
\pgfpathlineto{\pgfqpoint{6.058245in}{2.537128in}}%
\pgfpathlineto{\pgfqpoint{6.058358in}{2.191276in}}%
\pgfpathlineto{\pgfqpoint{6.058887in}{2.554325in}}%
\pgfpathlineto{\pgfqpoint{6.058889in}{1.772175in}}%
\pgfpathlineto{\pgfqpoint{6.059466in}{2.221828in}}%
\pgfpathlineto{\pgfqpoint{6.059961in}{1.890468in}}%
\pgfpathlineto{\pgfqpoint{6.060495in}{2.520434in}}%
\pgfpathlineto{\pgfqpoint{6.060575in}{2.396268in}}%
\pgfpathlineto{\pgfqpoint{6.061448in}{1.922408in}}%
\pgfpathlineto{\pgfqpoint{6.061236in}{2.527545in}}%
\pgfpathlineto{\pgfqpoint{6.061686in}{2.257511in}}%
\pgfpathlineto{\pgfqpoint{6.062301in}{2.519046in}}%
\pgfpathlineto{\pgfqpoint{6.061874in}{1.805836in}}%
\pgfpathlineto{\pgfqpoint{6.062796in}{2.297609in}}%
\pgfpathlineto{\pgfqpoint{6.062942in}{2.009646in}}%
\pgfpathlineto{\pgfqpoint{6.062813in}{2.503370in}}%
\pgfpathlineto{\pgfqpoint{6.062982in}{2.370301in}}%
\pgfpathlineto{\pgfqpoint{6.062984in}{2.534124in}}%
\pgfpathlineto{\pgfqpoint{6.063267in}{1.830605in}}%
\pgfpathlineto{\pgfqpoint{6.064091in}{2.295448in}}%
\pgfpathlineto{\pgfqpoint{6.064120in}{1.784055in}}%
\pgfpathlineto{\pgfqpoint{6.064935in}{2.527997in}}%
\pgfpathlineto{\pgfqpoint{6.065193in}{2.330313in}}%
\pgfpathlineto{\pgfqpoint{6.066214in}{2.528862in}}%
\pgfpathlineto{\pgfqpoint{6.065767in}{1.956160in}}%
\pgfpathlineto{\pgfqpoint{6.066305in}{2.451916in}}%
\pgfpathlineto{\pgfqpoint{6.066814in}{1.877771in}}%
\pgfpathlineto{\pgfqpoint{6.067283in}{2.555261in}}%
\pgfpathlineto{\pgfqpoint{6.067416in}{2.303926in}}%
\pgfpathlineto{\pgfqpoint{6.067872in}{2.526234in}}%
\pgfpathlineto{\pgfqpoint{6.068265in}{1.968533in}}%
\pgfpathlineto{\pgfqpoint{6.068530in}{2.439075in}}%
\pgfpathlineto{\pgfqpoint{6.069400in}{1.783192in}}%
\pgfpathlineto{\pgfqpoint{6.069280in}{2.513033in}}%
\pgfpathlineto{\pgfqpoint{6.069642in}{2.363239in}}%
\pgfpathlineto{\pgfqpoint{6.070481in}{2.534851in}}%
\pgfpathlineto{\pgfqpoint{6.070678in}{1.849629in}}%
\pgfpathlineto{\pgfqpoint{6.070741in}{2.480313in}}%
\pgfpathlineto{\pgfqpoint{6.071451in}{1.926083in}}%
\pgfpathlineto{\pgfqpoint{6.071761in}{2.515840in}}%
\pgfpathlineto{\pgfqpoint{6.071851in}{2.256732in}}%
\pgfpathlineto{\pgfqpoint{6.072040in}{2.519682in}}%
\pgfpathlineto{\pgfqpoint{6.072723in}{1.670843in}}%
\pgfpathlineto{\pgfqpoint{6.072962in}{2.375220in}}%
\pgfpathlineto{\pgfqpoint{6.073441in}{1.988049in}}%
\pgfpathlineto{\pgfqpoint{6.073631in}{2.514887in}}%
\pgfpathlineto{\pgfqpoint{6.074070in}{2.072914in}}%
\pgfpathlineto{\pgfqpoint{6.074085in}{2.524469in}}%
\pgfpathlineto{\pgfqpoint{6.074571in}{1.846853in}}%
\pgfpathlineto{\pgfqpoint{6.075183in}{2.466810in}}%
\pgfpathlineto{\pgfqpoint{6.075370in}{1.918737in}}%
\pgfpathlineto{\pgfqpoint{6.076078in}{2.530004in}}%
\pgfpathlineto{\pgfqpoint{6.076296in}{2.305317in}}%
\pgfpathlineto{\pgfqpoint{6.076507in}{2.516171in}}%
\pgfpathlineto{\pgfqpoint{6.076429in}{1.925237in}}%
\pgfpathlineto{\pgfqpoint{6.077406in}{2.418248in}}%
\pgfpathlineto{\pgfqpoint{6.077500in}{1.580057in}}%
\pgfpathlineto{\pgfqpoint{6.078396in}{2.532212in}}%
\pgfpathlineto{\pgfqpoint{6.078518in}{2.232744in}}%
\pgfpathlineto{\pgfqpoint{6.079256in}{2.520244in}}%
\pgfpathlineto{\pgfqpoint{6.078663in}{1.891933in}}%
\pgfpathlineto{\pgfqpoint{6.079629in}{2.282967in}}%
\pgfpathlineto{\pgfqpoint{6.079674in}{1.941988in}}%
\pgfpathlineto{\pgfqpoint{6.079679in}{2.517900in}}%
\pgfpathlineto{\pgfqpoint{6.080740in}{2.298904in}}%
\pgfpathlineto{\pgfqpoint{6.081277in}{2.514475in}}%
\pgfpathlineto{\pgfqpoint{6.081062in}{1.888576in}}%
\pgfpathlineto{\pgfqpoint{6.081850in}{2.488608in}}%
\pgfpathlineto{\pgfqpoint{6.082827in}{1.864509in}}%
\pgfpathlineto{\pgfqpoint{6.082037in}{2.517497in}}%
\pgfpathlineto{\pgfqpoint{6.082962in}{2.367812in}}%
\pgfpathlineto{\pgfqpoint{6.083234in}{2.520923in}}%
\pgfpathlineto{\pgfqpoint{6.083437in}{1.754335in}}%
\pgfpathlineto{\pgfqpoint{6.084025in}{2.430850in}}%
\pgfpathlineto{\pgfqpoint{6.084436in}{1.788799in}}%
\pgfpathlineto{\pgfqpoint{6.085064in}{2.539913in}}%
\pgfpathlineto{\pgfqpoint{6.085136in}{2.435687in}}%
\pgfpathlineto{\pgfqpoint{6.085282in}{2.545347in}}%
\pgfpathlineto{\pgfqpoint{6.085582in}{2.027225in}}%
\pgfpathlineto{\pgfqpoint{6.086155in}{2.397575in}}%
\pgfpathlineto{\pgfqpoint{6.086347in}{1.900560in}}%
\pgfpathlineto{\pgfqpoint{6.086326in}{2.518452in}}%
\pgfpathlineto{\pgfqpoint{6.087266in}{2.447483in}}%
\pgfpathlineto{\pgfqpoint{6.087939in}{1.684938in}}%
\pgfpathlineto{\pgfqpoint{6.087417in}{2.539070in}}%
\pgfpathlineto{\pgfqpoint{6.088381in}{2.355527in}}%
\pgfpathlineto{\pgfqpoint{6.089447in}{2.526178in}}%
\pgfpathlineto{\pgfqpoint{6.088760in}{1.939505in}}%
\pgfpathlineto{\pgfqpoint{6.089491in}{2.395019in}}%
\pgfpathlineto{\pgfqpoint{6.089544in}{1.992049in}}%
\pgfpathlineto{\pgfqpoint{6.089722in}{2.533970in}}%
\pgfpathlineto{\pgfqpoint{6.090601in}{2.339102in}}%
\pgfpathlineto{\pgfqpoint{6.091355in}{2.520837in}}%
\pgfpathlineto{\pgfqpoint{6.091563in}{1.848997in}}%
\pgfpathlineto{\pgfqpoint{6.091697in}{2.362138in}}%
\pgfpathlineto{\pgfqpoint{6.091882in}{1.869456in}}%
\pgfpathlineto{\pgfqpoint{6.091894in}{2.524882in}}%
\pgfpathlineto{\pgfqpoint{6.092807in}{2.318920in}}%
\pgfpathlineto{\pgfqpoint{6.093637in}{2.531647in}}%
\pgfpathlineto{\pgfqpoint{6.092850in}{1.826562in}}%
\pgfpathlineto{\pgfqpoint{6.093916in}{2.209673in}}%
\pgfpathlineto{\pgfqpoint{6.093929in}{2.171567in}}%
\pgfpathlineto{\pgfqpoint{6.093987in}{2.501423in}}%
\pgfpathlineto{\pgfqpoint{6.094027in}{2.290071in}}%
\pgfpathlineto{\pgfqpoint{6.094028in}{2.526360in}}%
\pgfpathlineto{\pgfqpoint{6.095091in}{1.886417in}}%
\pgfpathlineto{\pgfqpoint{6.095137in}{2.335913in}}%
\pgfpathlineto{\pgfqpoint{6.095963in}{1.971735in}}%
\pgfpathlineto{\pgfqpoint{6.095352in}{2.545699in}}%
\pgfpathlineto{\pgfqpoint{6.096247in}{2.366775in}}%
\pgfpathlineto{\pgfqpoint{6.097357in}{2.536628in}}%
\pgfpathlineto{\pgfqpoint{6.096948in}{1.720362in}}%
\pgfpathlineto{\pgfqpoint{6.097359in}{2.387555in}}%
\pgfpathlineto{\pgfqpoint{6.098312in}{2.529969in}}%
\pgfpathlineto{\pgfqpoint{6.098283in}{1.820289in}}%
\pgfpathlineto{\pgfqpoint{6.098408in}{2.169764in}}%
\pgfpathlineto{\pgfqpoint{6.098871in}{1.821868in}}%
\pgfpathlineto{\pgfqpoint{6.098657in}{2.540754in}}%
\pgfpathlineto{\pgfqpoint{6.099516in}{2.172763in}}%
\pgfpathlineto{\pgfqpoint{6.100177in}{2.525919in}}%
\pgfpathlineto{\pgfqpoint{6.099550in}{1.874616in}}%
\pgfpathlineto{\pgfqpoint{6.100629in}{2.436765in}}%
\pgfpathlineto{\pgfqpoint{6.101312in}{1.793405in}}%
\pgfpathlineto{\pgfqpoint{6.101379in}{2.529380in}}%
\pgfpathlineto{\pgfqpoint{6.101741in}{2.245295in}}%
\pgfpathlineto{\pgfqpoint{6.102501in}{2.519913in}}%
\pgfpathlineto{\pgfqpoint{6.101797in}{1.778824in}}%
\pgfpathlineto{\pgfqpoint{6.102854in}{2.471298in}}%
\pgfpathlineto{\pgfqpoint{6.103136in}{1.875119in}}%
\pgfpathlineto{\pgfqpoint{6.103758in}{2.533217in}}%
\pgfpathlineto{\pgfqpoint{6.103967in}{2.351596in}}%
\pgfpathlineto{\pgfqpoint{6.104517in}{1.817230in}}%
\pgfpathlineto{\pgfqpoint{6.104167in}{2.539250in}}%
\pgfpathlineto{\pgfqpoint{6.105077in}{2.247364in}}%
\pgfpathlineto{\pgfqpoint{6.105172in}{2.553035in}}%
\pgfpathlineto{\pgfqpoint{6.105531in}{1.893879in}}%
\pgfpathlineto{\pgfqpoint{6.106187in}{2.422062in}}%
\pgfpathlineto{\pgfqpoint{6.107185in}{1.593425in}}%
\pgfpathlineto{\pgfqpoint{6.106338in}{2.541799in}}%
\pgfpathlineto{\pgfqpoint{6.107300in}{2.356247in}}%
\pgfpathlineto{\pgfqpoint{6.107558in}{2.519076in}}%
\pgfpathlineto{\pgfqpoint{6.108341in}{1.942812in}}%
\pgfpathlineto{\pgfqpoint{6.108391in}{2.248539in}}%
\pgfpathlineto{\pgfqpoint{6.108406in}{1.923910in}}%
\pgfpathlineto{\pgfqpoint{6.108515in}{2.528250in}}%
\pgfpathlineto{\pgfqpoint{6.109499in}{2.519439in}}%
\pgfpathlineto{\pgfqpoint{6.109592in}{1.678309in}}%
\pgfpathlineto{\pgfqpoint{6.110131in}{2.523343in}}%
\pgfpathlineto{\pgfqpoint{6.110613in}{2.183614in}}%
\pgfpathlineto{\pgfqpoint{6.111298in}{2.548045in}}%
\pgfpathlineto{\pgfqpoint{6.111093in}{1.952585in}}%
\pgfpathlineto{\pgfqpoint{6.111726in}{2.456888in}}%
\pgfpathlineto{\pgfqpoint{6.112174in}{1.989703in}}%
\pgfpathlineto{\pgfqpoint{6.112492in}{2.530521in}}%
\pgfpathlineto{\pgfqpoint{6.112842in}{2.258221in}}%
\pgfpathlineto{\pgfqpoint{6.113753in}{2.532295in}}%
\pgfpathlineto{\pgfqpoint{6.113487in}{1.973308in}}%
\pgfpathlineto{\pgfqpoint{6.113953in}{2.432490in}}%
\pgfpathlineto{\pgfqpoint{6.114207in}{1.921376in}}%
\pgfpathlineto{\pgfqpoint{6.114360in}{2.544709in}}%
\pgfpathlineto{\pgfqpoint{6.115070in}{2.153396in}}%
\pgfpathlineto{\pgfqpoint{6.116004in}{2.557149in}}%
\pgfpathlineto{\pgfqpoint{6.115346in}{1.976707in}}%
\pgfpathlineto{\pgfqpoint{6.116184in}{2.390075in}}%
\pgfpathlineto{\pgfqpoint{6.116300in}{1.852593in}}%
\pgfpathlineto{\pgfqpoint{6.116620in}{2.524210in}}%
\pgfpathlineto{\pgfqpoint{6.117233in}{2.245004in}}%
\pgfpathlineto{\pgfqpoint{6.118223in}{2.533977in}}%
\pgfpathlineto{\pgfqpoint{6.117285in}{1.901590in}}%
\pgfpathlineto{\pgfqpoint{6.118344in}{2.402279in}}%
\pgfpathlineto{\pgfqpoint{6.118960in}{2.544729in}}%
\pgfpathlineto{\pgfqpoint{6.118440in}{1.872213in}}%
\pgfpathlineto{\pgfqpoint{6.119448in}{2.258787in}}%
\pgfpathlineto{\pgfqpoint{6.120406in}{1.967786in}}%
\pgfpathlineto{\pgfqpoint{6.120303in}{2.523998in}}%
\pgfpathlineto{\pgfqpoint{6.120559in}{2.299935in}}%
\pgfpathlineto{\pgfqpoint{6.121287in}{2.524049in}}%
\pgfpathlineto{\pgfqpoint{6.120913in}{1.820923in}}%
\pgfpathlineto{\pgfqpoint{6.121668in}{2.344350in}}%
\pgfpathlineto{\pgfqpoint{6.121695in}{1.948593in}}%
\pgfpathlineto{\pgfqpoint{6.122021in}{2.530647in}}%
\pgfpathlineto{\pgfqpoint{6.122777in}{2.408222in}}%
\pgfpathlineto{\pgfqpoint{6.123371in}{2.547551in}}%
\pgfpathlineto{\pgfqpoint{6.123689in}{1.931763in}}%
\pgfpathlineto{\pgfqpoint{6.123877in}{2.363657in}}%
\pgfpathlineto{\pgfqpoint{6.124634in}{1.983195in}}%
\pgfpathlineto{\pgfqpoint{6.124020in}{2.541048in}}%
\pgfpathlineto{\pgfqpoint{6.124989in}{2.166602in}}%
\pgfpathlineto{\pgfqpoint{6.125609in}{2.538206in}}%
\pgfpathlineto{\pgfqpoint{6.125094in}{1.857206in}}%
\pgfpathlineto{\pgfqpoint{6.126101in}{2.363581in}}%
\pgfpathlineto{\pgfqpoint{6.126235in}{1.983749in}}%
\pgfpathlineto{\pgfqpoint{6.127140in}{2.531747in}}%
\pgfpathlineto{\pgfqpoint{6.127211in}{2.311876in}}%
\pgfpathlineto{\pgfqpoint{6.127270in}{2.528206in}}%
\pgfpathlineto{\pgfqpoint{6.127219in}{1.893051in}}%
\pgfpathlineto{\pgfqpoint{6.128324in}{2.441864in}}%
\pgfpathlineto{\pgfqpoint{6.128997in}{1.929660in}}%
\pgfpathlineto{\pgfqpoint{6.128870in}{2.523362in}}%
\pgfpathlineto{\pgfqpoint{6.129435in}{2.383630in}}%
\pgfpathlineto{\pgfqpoint{6.129969in}{1.709525in}}%
\pgfpathlineto{\pgfqpoint{6.130248in}{2.534034in}}%
\pgfpathlineto{\pgfqpoint{6.130546in}{2.357762in}}%
\pgfpathlineto{\pgfqpoint{6.131349in}{2.547848in}}%
\pgfpathlineto{\pgfqpoint{6.130825in}{1.960820in}}%
\pgfpathlineto{\pgfqpoint{6.131657in}{2.420979in}}%
\pgfpathlineto{\pgfqpoint{6.132067in}{1.903203in}}%
\pgfpathlineto{\pgfqpoint{6.131907in}{2.530356in}}%
\pgfpathlineto{\pgfqpoint{6.132769in}{2.402484in}}%
\pgfpathlineto{\pgfqpoint{6.133279in}{2.545516in}}%
\pgfpathlineto{\pgfqpoint{6.133599in}{1.811964in}}%
\pgfpathlineto{\pgfqpoint{6.133875in}{2.405647in}}%
\pgfpathlineto{\pgfqpoint{6.134251in}{1.863900in}}%
\pgfpathlineto{\pgfqpoint{6.134200in}{2.521364in}}%
\pgfpathlineto{\pgfqpoint{6.134985in}{2.365374in}}%
\pgfpathlineto{\pgfqpoint{6.135364in}{2.556209in}}%
\pgfpathlineto{\pgfqpoint{6.135418in}{1.904825in}}%
\pgfpathlineto{\pgfqpoint{6.136096in}{2.422368in}}%
\pgfpathlineto{\pgfqpoint{6.136579in}{1.890771in}}%
\pgfpathlineto{\pgfqpoint{6.136705in}{2.543470in}}%
\pgfpathlineto{\pgfqpoint{6.137209in}{2.017555in}}%
\pgfpathlineto{\pgfqpoint{6.138318in}{2.570069in}}%
\pgfpathlineto{\pgfqpoint{6.137454in}{1.856197in}}%
\pgfpathlineto{\pgfqpoint{6.138320in}{2.325812in}}%
\pgfpathlineto{\pgfqpoint{6.138527in}{1.984264in}}%
\pgfpathlineto{\pgfqpoint{6.138707in}{2.550863in}}%
\pgfpathlineto{\pgfqpoint{6.139429in}{2.201717in}}%
\pgfpathlineto{\pgfqpoint{6.139974in}{2.528979in}}%
\pgfpathlineto{\pgfqpoint{6.140003in}{1.945295in}}%
\pgfpathlineto{\pgfqpoint{6.140541in}{2.443482in}}%
\pgfpathlineto{\pgfqpoint{6.140813in}{2.006545in}}%
\pgfpathlineto{\pgfqpoint{6.141208in}{2.527829in}}%
\pgfpathlineto{\pgfqpoint{6.141651in}{2.228821in}}%
\pgfpathlineto{\pgfqpoint{6.141869in}{2.527049in}}%
\pgfpathlineto{\pgfqpoint{6.142170in}{1.863992in}}%
\pgfpathlineto{\pgfqpoint{6.142763in}{2.462566in}}%
\pgfpathlineto{\pgfqpoint{6.143473in}{1.935910in}}%
\pgfpathlineto{\pgfqpoint{6.143720in}{2.571107in}}%
\pgfpathlineto{\pgfqpoint{6.143875in}{2.345558in}}%
\pgfpathlineto{\pgfqpoint{6.144172in}{1.831927in}}%
\pgfpathlineto{\pgfqpoint{6.144582in}{2.542971in}}%
\pgfpathlineto{\pgfqpoint{6.144979in}{2.413554in}}%
\pgfpathlineto{\pgfqpoint{6.144981in}{2.570460in}}%
\pgfpathlineto{\pgfqpoint{6.145346in}{1.627670in}}%
\pgfpathlineto{\pgfqpoint{6.146088in}{2.301172in}}%
\pgfpathlineto{\pgfqpoint{6.146392in}{2.548513in}}%
\pgfpathlineto{\pgfqpoint{6.146464in}{1.963667in}}%
\pgfpathlineto{\pgfqpoint{6.147196in}{2.367511in}}%
\pgfpathlineto{\pgfqpoint{6.147881in}{1.732806in}}%
\pgfpathlineto{\pgfqpoint{6.147877in}{2.533774in}}%
\pgfpathlineto{\pgfqpoint{6.148307in}{2.339559in}}%
\pgfpathlineto{\pgfqpoint{6.149289in}{2.546652in}}%
\pgfpathlineto{\pgfqpoint{6.148899in}{1.867356in}}%
\pgfpathlineto{\pgfqpoint{6.149418in}{2.398172in}}%
\pgfpathlineto{\pgfqpoint{6.149620in}{1.947408in}}%
\pgfpathlineto{\pgfqpoint{6.149739in}{2.535240in}}%
\pgfpathlineto{\pgfqpoint{6.150527in}{2.415775in}}%
\pgfpathlineto{\pgfqpoint{6.151431in}{2.537924in}}%
\pgfpathlineto{\pgfqpoint{6.150843in}{1.861203in}}%
\pgfpathlineto{\pgfqpoint{6.151637in}{2.530308in}}%
\pgfpathlineto{\pgfqpoint{6.152718in}{1.899955in}}%
\pgfpathlineto{\pgfqpoint{6.151785in}{2.536798in}}%
\pgfpathlineto{\pgfqpoint{6.152748in}{2.418951in}}%
\pgfpathlineto{\pgfqpoint{6.153717in}{1.947293in}}%
\pgfpathlineto{\pgfqpoint{6.152802in}{2.549482in}}%
\pgfpathlineto{\pgfqpoint{6.153859in}{2.413898in}}%
\pgfpathlineto{\pgfqpoint{6.154537in}{2.546545in}}%
\pgfpathlineto{\pgfqpoint{6.154673in}{1.755580in}}%
\pgfpathlineto{\pgfqpoint{6.154969in}{2.372337in}}%
\pgfpathlineto{\pgfqpoint{6.155757in}{1.895582in}}%
\pgfpathlineto{\pgfqpoint{6.155002in}{2.539058in}}%
\pgfpathlineto{\pgfqpoint{6.156081in}{2.260974in}}%
\pgfpathlineto{\pgfqpoint{6.156979in}{2.572877in}}%
\pgfpathlineto{\pgfqpoint{6.157099in}{1.879008in}}%
\pgfpathlineto{\pgfqpoint{6.157192in}{2.352995in}}%
\pgfpathlineto{\pgfqpoint{6.157581in}{1.843558in}}%
\pgfpathlineto{\pgfqpoint{6.157598in}{2.556642in}}%
\pgfpathlineto{\pgfqpoint{6.158299in}{2.267505in}}%
\pgfpathlineto{\pgfqpoint{6.158842in}{2.548828in}}%
\pgfpathlineto{\pgfqpoint{6.158957in}{1.864168in}}%
\pgfpathlineto{\pgfqpoint{6.159411in}{2.353192in}}%
\pgfpathlineto{\pgfqpoint{6.159761in}{2.548872in}}%
\pgfpathlineto{\pgfqpoint{6.160100in}{1.942830in}}%
\pgfpathlineto{\pgfqpoint{6.160521in}{2.322113in}}%
\pgfpathlineto{\pgfqpoint{6.161233in}{2.539110in}}%
\pgfpathlineto{\pgfqpoint{6.161068in}{1.772946in}}%
\pgfpathlineto{\pgfqpoint{6.161632in}{2.469297in}}%
\pgfpathlineto{\pgfqpoint{6.162225in}{1.704514in}}%
\pgfpathlineto{\pgfqpoint{6.161678in}{2.554385in}}%
\pgfpathlineto{\pgfqpoint{6.162744in}{2.353975in}}%
\pgfpathlineto{\pgfqpoint{6.163029in}{2.544260in}}%
\pgfpathlineto{\pgfqpoint{6.163213in}{1.874299in}}%
\pgfpathlineto{\pgfqpoint{6.163853in}{2.450889in}}%
\pgfpathlineto{\pgfqpoint{6.164685in}{1.954481in}}%
\pgfpathlineto{\pgfqpoint{6.164061in}{2.560492in}}%
\pgfpathlineto{\pgfqpoint{6.164964in}{2.361099in}}%
\pgfpathlineto{\pgfqpoint{6.165602in}{1.964350in}}%
\pgfpathlineto{\pgfqpoint{6.165888in}{2.556434in}}%
\pgfpathlineto{\pgfqpoint{6.166074in}{2.395975in}}%
\pgfpathlineto{\pgfqpoint{6.166078in}{2.502589in}}%
\pgfpathlineto{\pgfqpoint{6.166086in}{2.266975in}}%
\pgfpathlineto{\pgfqpoint{6.166099in}{2.313324in}}%
\pgfpathlineto{\pgfqpoint{6.166218in}{1.943807in}}%
\pgfpathlineto{\pgfqpoint{6.166590in}{2.534785in}}%
\pgfpathlineto{\pgfqpoint{6.167209in}{2.267111in}}%
\pgfpathlineto{\pgfqpoint{6.167424in}{2.553610in}}%
\pgfpathlineto{\pgfqpoint{6.168211in}{1.926834in}}%
\pgfpathlineto{\pgfqpoint{6.168320in}{2.427300in}}%
\pgfpathlineto{\pgfqpoint{6.169019in}{1.855701in}}%
\pgfpathlineto{\pgfqpoint{6.169270in}{2.553891in}}%
\pgfpathlineto{\pgfqpoint{6.169434in}{2.231618in}}%
\pgfpathlineto{\pgfqpoint{6.169983in}{2.538102in}}%
\pgfpathlineto{\pgfqpoint{6.170062in}{1.943608in}}%
\pgfpathlineto{\pgfqpoint{6.170547in}{2.415305in}}%
\pgfpathlineto{\pgfqpoint{6.171331in}{1.896889in}}%
\pgfpathlineto{\pgfqpoint{6.170677in}{2.569691in}}%
\pgfpathlineto{\pgfqpoint{6.171660in}{2.371505in}}%
\pgfpathlineto{\pgfqpoint{6.172475in}{1.874528in}}%
\pgfpathlineto{\pgfqpoint{6.171955in}{2.540606in}}%
\pgfpathlineto{\pgfqpoint{6.172673in}{2.094426in}}%
\pgfpathlineto{\pgfqpoint{6.172774in}{2.549537in}}%
\pgfpathlineto{\pgfqpoint{6.172914in}{1.785008in}}%
\pgfpathlineto{\pgfqpoint{6.173784in}{2.215774in}}%
\pgfpathlineto{\pgfqpoint{6.174801in}{2.543058in}}%
\pgfpathlineto{\pgfqpoint{6.174192in}{1.792769in}}%
\pgfpathlineto{\pgfqpoint{6.174895in}{2.430399in}}%
\pgfpathlineto{\pgfqpoint{6.175780in}{1.572233in}}%
\pgfpathlineto{\pgfqpoint{6.175819in}{2.536378in}}%
\pgfpathlineto{\pgfqpoint{6.176007in}{2.356515in}}%
\pgfpathlineto{\pgfqpoint{6.176035in}{2.576925in}}%
\pgfpathlineto{\pgfqpoint{6.176748in}{1.907899in}}%
\pgfpathlineto{\pgfqpoint{6.177118in}{2.355416in}}%
\pgfpathlineto{\pgfqpoint{6.178037in}{2.548470in}}%
\pgfpathlineto{\pgfqpoint{6.177588in}{1.828709in}}%
\pgfpathlineto{\pgfqpoint{6.178229in}{2.466455in}}%
\pgfpathlineto{\pgfqpoint{6.178526in}{1.921347in}}%
\pgfpathlineto{\pgfqpoint{6.179056in}{2.542322in}}%
\pgfpathlineto{\pgfqpoint{6.179341in}{2.233272in}}%
\pgfpathlineto{\pgfqpoint{6.179763in}{2.537463in}}%
\pgfpathlineto{\pgfqpoint{6.179583in}{1.922495in}}%
\pgfpathlineto{\pgfqpoint{6.180452in}{2.402039in}}%
\pgfpathlineto{\pgfqpoint{6.180723in}{1.824713in}}%
\pgfpathlineto{\pgfqpoint{6.180742in}{2.555432in}}%
\pgfpathlineto{\pgfqpoint{6.181564in}{2.320482in}}%
\pgfpathlineto{\pgfqpoint{6.182659in}{2.542545in}}%
\pgfpathlineto{\pgfqpoint{6.181703in}{1.846144in}}%
\pgfpathlineto{\pgfqpoint{6.182674in}{2.360379in}}%
\pgfpathlineto{\pgfqpoint{6.183132in}{1.830881in}}%
\pgfpathlineto{\pgfqpoint{6.183269in}{2.558105in}}%
\pgfpathlineto{\pgfqpoint{6.183785in}{2.339822in}}%
\pgfpathlineto{\pgfqpoint{6.184306in}{2.559367in}}%
\pgfpathlineto{\pgfqpoint{6.184216in}{1.738430in}}%
\pgfpathlineto{\pgfqpoint{6.184896in}{2.408161in}}%
\pgfpathlineto{\pgfqpoint{6.185209in}{2.550883in}}%
\pgfpathlineto{\pgfqpoint{6.185266in}{1.901090in}}%
\pgfpathlineto{\pgfqpoint{6.185975in}{2.452134in}}%
\pgfpathlineto{\pgfqpoint{6.186563in}{1.893622in}}%
\pgfpathlineto{\pgfqpoint{6.186409in}{2.530918in}}%
\pgfpathlineto{\pgfqpoint{6.187085in}{2.394824in}}%
\pgfpathlineto{\pgfqpoint{6.188096in}{2.552483in}}%
\pgfpathlineto{\pgfqpoint{6.187769in}{1.850875in}}%
\pgfpathlineto{\pgfqpoint{6.188197in}{2.469744in}}%
\pgfpathlineto{\pgfqpoint{6.188930in}{1.884344in}}%
\pgfpathlineto{\pgfqpoint{6.188911in}{2.559251in}}%
\pgfpathlineto{\pgfqpoint{6.189310in}{2.293521in}}%
\pgfpathlineto{\pgfqpoint{6.190039in}{2.533181in}}%
\pgfpathlineto{\pgfqpoint{6.189722in}{1.873388in}}%
\pgfpathlineto{\pgfqpoint{6.190419in}{2.486787in}}%
\pgfpathlineto{\pgfqpoint{6.191178in}{1.483327in}}%
\pgfpathlineto{\pgfqpoint{6.190675in}{2.553348in}}%
\pgfpathlineto{\pgfqpoint{6.191531in}{2.287807in}}%
\pgfpathlineto{\pgfqpoint{6.192277in}{2.546098in}}%
\pgfpathlineto{\pgfqpoint{6.191940in}{1.682263in}}%
\pgfpathlineto{\pgfqpoint{6.192642in}{2.269001in}}%
\pgfpathlineto{\pgfqpoint{6.193710in}{1.776875in}}%
\pgfpathlineto{\pgfqpoint{6.193475in}{2.544019in}}%
\pgfpathlineto{\pgfqpoint{6.193752in}{2.283149in}}%
\pgfpathlineto{\pgfqpoint{6.194752in}{1.906914in}}%
\pgfpathlineto{\pgfqpoint{6.194186in}{2.561524in}}%
\pgfpathlineto{\pgfqpoint{6.194858in}{2.330586in}}%
\pgfpathlineto{\pgfqpoint{6.194958in}{2.556425in}}%
\pgfpathlineto{\pgfqpoint{6.195694in}{1.932533in}}%
\pgfpathlineto{\pgfqpoint{6.195969in}{2.469947in}}%
\pgfpathlineto{\pgfqpoint{6.196579in}{1.986422in}}%
\pgfpathlineto{\pgfqpoint{6.196719in}{2.529382in}}%
\pgfpathlineto{\pgfqpoint{6.197082in}{2.319262in}}%
\pgfpathlineto{\pgfqpoint{6.197770in}{2.553547in}}%
\pgfpathlineto{\pgfqpoint{6.197458in}{1.978245in}}%
\pgfpathlineto{\pgfqpoint{6.198193in}{2.359779in}}%
\pgfpathlineto{\pgfqpoint{6.198468in}{1.975966in}}%
\pgfpathlineto{\pgfqpoint{6.198608in}{2.566233in}}%
\pgfpathlineto{\pgfqpoint{6.199299in}{2.075909in}}%
\pgfpathlineto{\pgfqpoint{6.200339in}{2.556650in}}%
\pgfpathlineto{\pgfqpoint{6.200064in}{1.863703in}}%
\pgfpathlineto{\pgfqpoint{6.200411in}{2.487032in}}%
\pgfpathlineto{\pgfqpoint{6.201514in}{1.833468in}}%
\pgfpathlineto{\pgfqpoint{6.200569in}{2.537337in}}%
\pgfpathlineto{\pgfqpoint{6.201522in}{2.386484in}}%
\pgfpathlineto{\pgfqpoint{6.201547in}{2.575908in}}%
\pgfpathlineto{\pgfqpoint{6.202160in}{1.875601in}}%
\pgfpathlineto{\pgfqpoint{6.202633in}{2.484525in}}%
\pgfpathlineto{\pgfqpoint{6.203125in}{1.841336in}}%
\pgfpathlineto{\pgfqpoint{6.203229in}{2.554416in}}%
\pgfpathlineto{\pgfqpoint{6.203746in}{2.215039in}}%
\pgfpathlineto{\pgfqpoint{6.203984in}{2.566070in}}%
\pgfpathlineto{\pgfqpoint{6.204707in}{1.946935in}}%
\pgfpathlineto{\pgfqpoint{6.204856in}{2.351117in}}%
\pgfpathlineto{\pgfqpoint{6.205486in}{1.901313in}}%
\pgfpathlineto{\pgfqpoint{6.205456in}{2.555260in}}%
\pgfpathlineto{\pgfqpoint{6.205965in}{2.154643in}}%
\pgfpathlineto{\pgfqpoint{6.206377in}{2.554423in}}%
\pgfpathlineto{\pgfqpoint{6.206652in}{1.710067in}}%
\pgfpathlineto{\pgfqpoint{6.207077in}{2.295268in}}%
\pgfpathlineto{\pgfqpoint{6.207460in}{2.542419in}}%
\pgfpathlineto{\pgfqpoint{6.207404in}{1.988639in}}%
\pgfpathlineto{\pgfqpoint{6.208186in}{2.433507in}}%
\pgfpathlineto{\pgfqpoint{6.208365in}{1.974945in}}%
\pgfpathlineto{\pgfqpoint{6.209136in}{2.546972in}}%
\pgfpathlineto{\pgfqpoint{6.209297in}{2.450034in}}%
\pgfpathlineto{\pgfqpoint{6.209684in}{1.950201in}}%
\pgfpathlineto{\pgfqpoint{6.209304in}{2.542611in}}%
\pgfpathlineto{\pgfqpoint{6.210409in}{2.398847in}}%
\pgfpathlineto{\pgfqpoint{6.211117in}{2.553432in}}%
\pgfpathlineto{\pgfqpoint{6.211296in}{1.867219in}}%
\pgfpathlineto{\pgfqpoint{6.211519in}{2.359314in}}%
\pgfpathlineto{\pgfqpoint{6.212320in}{2.549380in}}%
\pgfpathlineto{\pgfqpoint{6.212085in}{1.684495in}}%
\pgfpathlineto{\pgfqpoint{6.212620in}{2.375556in}}%
\pgfpathlineto{\pgfqpoint{6.213049in}{1.931734in}}%
\pgfpathlineto{\pgfqpoint{6.212890in}{2.561027in}}%
\pgfpathlineto{\pgfqpoint{6.213731in}{2.387675in}}%
\pgfpathlineto{\pgfqpoint{6.214492in}{2.553970in}}%
\pgfpathlineto{\pgfqpoint{6.213871in}{1.803136in}}%
\pgfpathlineto{\pgfqpoint{6.214841in}{2.451037in}}%
\pgfpathlineto{\pgfqpoint{6.214994in}{1.882422in}}%
\pgfpathlineto{\pgfqpoint{6.215659in}{2.567377in}}%
\pgfpathlineto{\pgfqpoint{6.215951in}{2.332787in}}%
\pgfpathlineto{\pgfqpoint{6.216135in}{2.560504in}}%
\pgfpathlineto{\pgfqpoint{6.217043in}{1.848653in}}%
\pgfpathlineto{\pgfqpoint{6.217063in}{2.339335in}}%
\pgfpathlineto{\pgfqpoint{6.218139in}{1.906210in}}%
\pgfpathlineto{\pgfqpoint{6.217802in}{2.554444in}}%
\pgfpathlineto{\pgfqpoint{6.218167in}{2.464445in}}%
\pgfpathlineto{\pgfqpoint{6.218739in}{2.561549in}}%
\pgfpathlineto{\pgfqpoint{6.219048in}{1.817996in}}%
\pgfpathlineto{\pgfqpoint{6.219268in}{2.275590in}}%
\pgfpathlineto{\pgfqpoint{6.219392in}{1.896262in}}%
\pgfpathlineto{\pgfqpoint{6.219829in}{2.558758in}}%
\pgfpathlineto{\pgfqpoint{6.220338in}{2.482904in}}%
\pgfpathlineto{\pgfqpoint{6.220734in}{2.550861in}}%
\pgfpathlineto{\pgfqpoint{6.220887in}{1.729131in}}%
\pgfpathlineto{\pgfqpoint{6.221447in}{2.515861in}}%
\pgfpathlineto{\pgfqpoint{6.222075in}{1.978895in}}%
\pgfpathlineto{\pgfqpoint{6.221840in}{2.566459in}}%
\pgfpathlineto{\pgfqpoint{6.222557in}{2.260258in}}%
\pgfpathlineto{\pgfqpoint{6.222559in}{2.564699in}}%
\pgfpathlineto{\pgfqpoint{6.222757in}{1.754281in}}%
\pgfpathlineto{\pgfqpoint{6.223669in}{2.414788in}}%
\pgfpathlineto{\pgfqpoint{6.224095in}{1.986110in}}%
\pgfpathlineto{\pgfqpoint{6.224628in}{2.551709in}}%
\pgfpathlineto{\pgfqpoint{6.224775in}{2.389667in}}%
\pgfpathlineto{\pgfqpoint{6.225076in}{2.561488in}}%
\pgfpathlineto{\pgfqpoint{6.224986in}{1.924426in}}%
\pgfpathlineto{\pgfqpoint{6.225885in}{2.519228in}}%
\pgfpathlineto{\pgfqpoint{6.226648in}{1.861594in}}%
\pgfpathlineto{\pgfqpoint{6.226853in}{2.554274in}}%
\pgfpathlineto{\pgfqpoint{6.226995in}{2.440690in}}%
\pgfpathlineto{\pgfqpoint{6.227005in}{2.553837in}}%
\pgfpathlineto{\pgfqpoint{6.227871in}{1.799380in}}%
\pgfpathlineto{\pgfqpoint{6.228096in}{2.310520in}}%
\pgfpathlineto{\pgfqpoint{6.228970in}{1.946610in}}%
\pgfpathlineto{\pgfqpoint{6.228312in}{2.568013in}}%
\pgfpathlineto{\pgfqpoint{6.229205in}{2.270351in}}%
\pgfpathlineto{\pgfqpoint{6.229502in}{2.585931in}}%
\pgfpathlineto{\pgfqpoint{6.230201in}{2.021579in}}%
\pgfpathlineto{\pgfqpoint{6.230316in}{2.370911in}}%
\pgfpathlineto{\pgfqpoint{6.230342in}{1.804134in}}%
\pgfpathlineto{\pgfqpoint{6.230828in}{2.555572in}}%
\pgfpathlineto{\pgfqpoint{6.231422in}{2.328653in}}%
\pgfpathlineto{\pgfqpoint{6.232085in}{2.545060in}}%
\pgfpathlineto{\pgfqpoint{6.231549in}{1.808572in}}%
\pgfpathlineto{\pgfqpoint{6.232532in}{2.303990in}}%
\pgfpathlineto{\pgfqpoint{6.232918in}{2.555713in}}%
\pgfpathlineto{\pgfqpoint{6.233205in}{1.861767in}}%
\pgfpathlineto{\pgfqpoint{6.233643in}{2.362827in}}%
\pgfpathlineto{\pgfqpoint{6.234463in}{1.896854in}}%
\pgfpathlineto{\pgfqpoint{6.234543in}{2.575524in}}%
\pgfpathlineto{\pgfqpoint{6.234750in}{2.347229in}}%
\pgfpathlineto{\pgfqpoint{6.235432in}{2.570014in}}%
\pgfpathlineto{\pgfqpoint{6.235498in}{1.962242in}}%
\pgfpathlineto{\pgfqpoint{6.235859in}{2.485710in}}%
\pgfpathlineto{\pgfqpoint{6.236047in}{1.852539in}}%
\pgfpathlineto{\pgfqpoint{6.236548in}{2.550497in}}%
\pgfpathlineto{\pgfqpoint{6.236970in}{2.316447in}}%
\pgfpathlineto{\pgfqpoint{6.237867in}{1.751386in}}%
\pgfpathlineto{\pgfqpoint{6.238042in}{2.555330in}}%
\pgfpathlineto{\pgfqpoint{6.238081in}{2.294380in}}%
\pgfpathlineto{\pgfqpoint{6.238636in}{1.940626in}}%
\pgfpathlineto{\pgfqpoint{6.238999in}{2.562267in}}%
\pgfpathlineto{\pgfqpoint{6.239187in}{2.456821in}}%
\pgfpathlineto{\pgfqpoint{6.239782in}{2.574227in}}%
\pgfpathlineto{\pgfqpoint{6.239960in}{1.789597in}}%
\pgfpathlineto{\pgfqpoint{6.240125in}{2.477934in}}%
\pgfpathlineto{\pgfqpoint{6.240126in}{1.887609in}}%
\pgfpathlineto{\pgfqpoint{6.240853in}{2.553904in}}%
\pgfpathlineto{\pgfqpoint{6.241236in}{2.396201in}}%
\pgfpathlineto{\pgfqpoint{6.241303in}{1.870539in}}%
\pgfpathlineto{\pgfqpoint{6.241259in}{2.571480in}}%
\pgfpathlineto{\pgfqpoint{6.242348in}{2.152598in}}%
\pgfpathlineto{\pgfqpoint{6.242839in}{2.559755in}}%
\pgfpathlineto{\pgfqpoint{6.242658in}{1.983418in}}%
\pgfpathlineto{\pgfqpoint{6.243460in}{2.490692in}}%
\pgfpathlineto{\pgfqpoint{6.244154in}{1.857430in}}%
\pgfpathlineto{\pgfqpoint{6.243717in}{2.565674in}}%
\pgfpathlineto{\pgfqpoint{6.244572in}{2.288989in}}%
\pgfpathlineto{\pgfqpoint{6.245338in}{2.553889in}}%
\pgfpathlineto{\pgfqpoint{6.244980in}{1.991060in}}%
\pgfpathlineto{\pgfqpoint{6.245684in}{2.399998in}}%
\pgfpathlineto{\pgfqpoint{6.245938in}{1.800697in}}%
\pgfpathlineto{\pgfqpoint{6.246400in}{2.567028in}}%
\pgfpathlineto{\pgfqpoint{6.246796in}{2.137940in}}%
\pgfpathlineto{\pgfqpoint{6.247239in}{2.570704in}}%
\pgfpathlineto{\pgfqpoint{6.246976in}{1.938870in}}%
\pgfpathlineto{\pgfqpoint{6.247906in}{2.330514in}}%
\pgfpathlineto{\pgfqpoint{6.248326in}{1.903071in}}%
\pgfpathlineto{\pgfqpoint{6.248945in}{2.556714in}}%
\pgfpathlineto{\pgfqpoint{6.249018in}{2.315002in}}%
\pgfpathlineto{\pgfqpoint{6.249568in}{2.560420in}}%
\pgfpathlineto{\pgfqpoint{6.249615in}{1.791270in}}%
\pgfpathlineto{\pgfqpoint{6.250128in}{2.471446in}}%
\pgfpathlineto{\pgfqpoint{6.250925in}{1.908350in}}%
\pgfpathlineto{\pgfqpoint{6.250840in}{2.593197in}}%
\pgfpathlineto{\pgfqpoint{6.251240in}{2.245718in}}%
\pgfpathlineto{\pgfqpoint{6.251463in}{2.555619in}}%
\pgfpathlineto{\pgfqpoint{6.251319in}{1.860724in}}%
\pgfpathlineto{\pgfqpoint{6.252353in}{2.366845in}}%
\pgfpathlineto{\pgfqpoint{6.252729in}{2.549628in}}%
\pgfpathlineto{\pgfqpoint{6.252531in}{1.873579in}}%
\pgfpathlineto{\pgfqpoint{6.253457in}{2.462644in}}%
\pgfpathlineto{\pgfqpoint{6.253561in}{1.895936in}}%
\pgfpathlineto{\pgfqpoint{6.253495in}{2.561480in}}%
\pgfpathlineto{\pgfqpoint{6.254568in}{2.163923in}}%
\pgfpathlineto{\pgfqpoint{6.254892in}{2.580570in}}%
\pgfpathlineto{\pgfqpoint{6.254994in}{1.797584in}}%
\pgfpathlineto{\pgfqpoint{6.255680in}{2.438725in}}%
\pgfpathlineto{\pgfqpoint{6.255805in}{1.979320in}}%
\pgfpathlineto{\pgfqpoint{6.255827in}{2.586829in}}%
\pgfpathlineto{\pgfqpoint{6.256792in}{2.317130in}}%
\pgfpathlineto{\pgfqpoint{6.257092in}{2.577316in}}%
\pgfpathlineto{\pgfqpoint{6.257564in}{1.625436in}}%
\pgfpathlineto{\pgfqpoint{6.257875in}{2.374918in}}%
\pgfpathlineto{\pgfqpoint{6.257937in}{1.975357in}}%
\pgfpathlineto{\pgfqpoint{6.258811in}{2.557871in}}%
\pgfpathlineto{\pgfqpoint{6.258986in}{2.191256in}}%
\pgfpathlineto{\pgfqpoint{6.259305in}{2.565619in}}%
\pgfpathlineto{\pgfqpoint{6.259041in}{1.867469in}}%
\pgfpathlineto{\pgfqpoint{6.260099in}{2.498872in}}%
\pgfpathlineto{\pgfqpoint{6.260748in}{1.973909in}}%
\pgfpathlineto{\pgfqpoint{6.260463in}{2.559143in}}%
\pgfpathlineto{\pgfqpoint{6.261211in}{2.356990in}}%
\pgfpathlineto{\pgfqpoint{6.261412in}{2.567916in}}%
\pgfpathlineto{\pgfqpoint{6.261911in}{1.987843in}}%
\pgfpathlineto{\pgfqpoint{6.262320in}{2.468161in}}%
\pgfpathlineto{\pgfqpoint{6.263263in}{1.750729in}}%
\pgfpathlineto{\pgfqpoint{6.262603in}{2.574493in}}%
\pgfpathlineto{\pgfqpoint{6.263431in}{2.396054in}}%
\pgfpathlineto{\pgfqpoint{6.264075in}{2.554325in}}%
\pgfpathlineto{\pgfqpoint{6.263672in}{1.839862in}}%
\pgfpathlineto{\pgfqpoint{6.264541in}{2.364213in}}%
\pgfpathlineto{\pgfqpoint{6.265141in}{1.868123in}}%
\pgfpathlineto{\pgfqpoint{6.265407in}{2.556904in}}%
\pgfpathlineto{\pgfqpoint{6.265651in}{2.349806in}}%
\pgfpathlineto{\pgfqpoint{6.266340in}{2.564581in}}%
\pgfpathlineto{\pgfqpoint{6.265751in}{1.898665in}}%
\pgfpathlineto{\pgfqpoint{6.266761in}{2.463897in}}%
\pgfpathlineto{\pgfqpoint{6.266990in}{1.949631in}}%
\pgfpathlineto{\pgfqpoint{6.267556in}{2.560215in}}%
\pgfpathlineto{\pgfqpoint{6.267872in}{2.267890in}}%
\pgfpathlineto{\pgfqpoint{6.268581in}{2.572553in}}%
\pgfpathlineto{\pgfqpoint{6.267901in}{1.901410in}}%
\pgfpathlineto{\pgfqpoint{6.268982in}{2.357538in}}%
\pgfpathlineto{\pgfqpoint{6.269491in}{1.847071in}}%
\pgfpathlineto{\pgfqpoint{6.269050in}{2.559606in}}%
\pgfpathlineto{\pgfqpoint{6.270092in}{2.399746in}}%
\pgfpathlineto{\pgfqpoint{6.270239in}{1.937650in}}%
\pgfpathlineto{\pgfqpoint{6.270706in}{2.568968in}}%
\pgfpathlineto{\pgfqpoint{6.271202in}{2.342715in}}%
\pgfpathlineto{\pgfqpoint{6.271710in}{2.567434in}}%
\pgfpathlineto{\pgfqpoint{6.271974in}{1.875986in}}%
\pgfpathlineto{\pgfqpoint{6.272314in}{2.478454in}}%
\pgfpathlineto{\pgfqpoint{6.272879in}{1.785726in}}%
\pgfpathlineto{\pgfqpoint{6.272831in}{2.564261in}}%
\pgfpathlineto{\pgfqpoint{6.273427in}{2.389649in}}%
\pgfpathlineto{\pgfqpoint{6.274468in}{2.555836in}}%
\pgfpathlineto{\pgfqpoint{6.273444in}{1.932499in}}%
\pgfpathlineto{\pgfqpoint{6.274537in}{2.426322in}}%
\pgfpathlineto{\pgfqpoint{6.274905in}{1.981300in}}%
\pgfpathlineto{\pgfqpoint{6.274894in}{2.554065in}}%
\pgfpathlineto{\pgfqpoint{6.275647in}{2.442658in}}%
\pgfpathlineto{\pgfqpoint{6.276224in}{1.878101in}}%
\pgfpathlineto{\pgfqpoint{6.276221in}{2.571138in}}%
\pgfpathlineto{\pgfqpoint{6.276758in}{2.361275in}}%
\pgfpathlineto{\pgfqpoint{6.277662in}{2.564865in}}%
\pgfpathlineto{\pgfqpoint{6.277129in}{2.004429in}}%
\pgfpathlineto{\pgfqpoint{6.277870in}{2.449106in}}%
\pgfpathlineto{\pgfqpoint{6.278693in}{1.967783in}}%
\pgfpathlineto{\pgfqpoint{6.278486in}{2.576370in}}%
\pgfpathlineto{\pgfqpoint{6.278982in}{2.181296in}}%
\pgfpathlineto{\pgfqpoint{6.279720in}{2.553434in}}%
\pgfpathlineto{\pgfqpoint{6.279549in}{1.888964in}}%
\pgfpathlineto{\pgfqpoint{6.280094in}{2.459862in}}%
\pgfpathlineto{\pgfqpoint{6.280631in}{1.849520in}}%
\pgfpathlineto{\pgfqpoint{6.280839in}{2.571702in}}%
\pgfpathlineto{\pgfqpoint{6.281206in}{2.406841in}}%
\pgfpathlineto{\pgfqpoint{6.281647in}{2.554260in}}%
\pgfpathlineto{\pgfqpoint{6.282036in}{1.980185in}}%
\pgfpathlineto{\pgfqpoint{6.282316in}{2.419054in}}%
\pgfpathlineto{\pgfqpoint{6.282521in}{1.884193in}}%
\pgfpathlineto{\pgfqpoint{6.282743in}{2.568457in}}%
\pgfpathlineto{\pgfqpoint{6.283428in}{2.351169in}}%
\pgfpathlineto{\pgfqpoint{6.284510in}{2.573011in}}%
\pgfpathlineto{\pgfqpoint{6.283840in}{1.871588in}}%
\pgfpathlineto{\pgfqpoint{6.284538in}{2.389581in}}%
\pgfpathlineto{\pgfqpoint{6.285177in}{1.888621in}}%
\pgfpathlineto{\pgfqpoint{6.285292in}{2.575298in}}%
\pgfpathlineto{\pgfqpoint{6.285647in}{2.290592in}}%
\pgfpathlineto{\pgfqpoint{6.285864in}{2.575815in}}%
\pgfpathlineto{\pgfqpoint{6.286434in}{1.692851in}}%
\pgfpathlineto{\pgfqpoint{6.286758in}{2.319751in}}%
\pgfpathlineto{\pgfqpoint{6.287561in}{2.558394in}}%
\pgfpathlineto{\pgfqpoint{6.287290in}{1.826111in}}%
\pgfpathlineto{\pgfqpoint{6.287860in}{2.227738in}}%
\pgfpathlineto{\pgfqpoint{6.287921in}{1.867918in}}%
\pgfpathlineto{\pgfqpoint{6.288478in}{2.557028in}}%
\pgfpathlineto{\pgfqpoint{6.288969in}{2.277448in}}%
\pgfpathlineto{\pgfqpoint{6.289520in}{2.585630in}}%
\pgfpathlineto{\pgfqpoint{6.289500in}{1.960085in}}%
\pgfpathlineto{\pgfqpoint{6.290082in}{2.481251in}}%
\pgfpathlineto{\pgfqpoint{6.290561in}{1.975017in}}%
\pgfpathlineto{\pgfqpoint{6.291103in}{2.564889in}}%
\pgfpathlineto{\pgfqpoint{6.291190in}{2.445015in}}%
\pgfpathlineto{\pgfqpoint{6.291271in}{2.585106in}}%
\pgfpathlineto{\pgfqpoint{6.292289in}{1.977689in}}%
\pgfpathlineto{\pgfqpoint{6.292300in}{2.395779in}}%
\pgfpathlineto{\pgfqpoint{6.292910in}{1.944136in}}%
\pgfpathlineto{\pgfqpoint{6.293010in}{2.555337in}}%
\pgfpathlineto{\pgfqpoint{6.293411in}{2.218049in}}%
\pgfpathlineto{\pgfqpoint{6.293757in}{2.573416in}}%
\pgfpathlineto{\pgfqpoint{6.293416in}{1.862949in}}%
\pgfpathlineto{\pgfqpoint{6.294522in}{2.391281in}}%
\pgfpathlineto{\pgfqpoint{6.294804in}{2.600667in}}%
\pgfpathlineto{\pgfqpoint{6.295247in}{1.865279in}}%
\pgfpathlineto{\pgfqpoint{6.295630in}{2.529970in}}%
\pgfpathlineto{\pgfqpoint{6.296670in}{1.783010in}}%
\pgfpathlineto{\pgfqpoint{6.296139in}{2.570380in}}%
\pgfpathlineto{\pgfqpoint{6.296741in}{2.426604in}}%
\pgfpathlineto{\pgfqpoint{6.297308in}{2.564416in}}%
\pgfpathlineto{\pgfqpoint{6.297536in}{1.774014in}}%
\pgfpathlineto{\pgfqpoint{6.297801in}{2.236163in}}%
\pgfpathlineto{\pgfqpoint{6.298479in}{1.849453in}}%
\pgfpathlineto{\pgfqpoint{6.298777in}{2.566553in}}%
\pgfpathlineto{\pgfqpoint{6.298911in}{2.502164in}}%
\pgfpathlineto{\pgfqpoint{6.299586in}{2.021543in}}%
\pgfpathlineto{\pgfqpoint{6.298999in}{2.558732in}}%
\pgfpathlineto{\pgfqpoint{6.300023in}{2.299737in}}%
\pgfpathlineto{\pgfqpoint{6.300679in}{2.606063in}}%
\pgfpathlineto{\pgfqpoint{6.301057in}{1.960175in}}%
\pgfpathlineto{\pgfqpoint{6.301134in}{2.441100in}}%
\pgfpathlineto{\pgfqpoint{6.301234in}{1.960177in}}%
\pgfpathlineto{\pgfqpoint{6.301217in}{2.587948in}}%
\pgfpathlineto{\pgfqpoint{6.302245in}{2.419403in}}%
\pgfpathlineto{\pgfqpoint{6.302410in}{2.590447in}}%
\pgfpathlineto{\pgfqpoint{6.303159in}{1.890478in}}%
\pgfpathlineto{\pgfqpoint{6.303354in}{2.439450in}}%
\pgfpathlineto{\pgfqpoint{6.304391in}{1.887192in}}%
\pgfpathlineto{\pgfqpoint{6.304163in}{2.563139in}}%
\pgfpathlineto{\pgfqpoint{6.304466in}{2.194966in}}%
\pgfpathlineto{\pgfqpoint{6.305452in}{2.558829in}}%
\pgfpathlineto{\pgfqpoint{6.304472in}{1.834755in}}%
\pgfpathlineto{\pgfqpoint{6.305577in}{2.532088in}}%
\pgfpathlineto{\pgfqpoint{6.306452in}{1.975039in}}%
\pgfpathlineto{\pgfqpoint{6.305841in}{2.570614in}}%
\pgfpathlineto{\pgfqpoint{6.306690in}{2.374787in}}%
\pgfpathlineto{\pgfqpoint{6.307081in}{2.571800in}}%
\pgfpathlineto{\pgfqpoint{6.307662in}{1.759662in}}%
\pgfpathlineto{\pgfqpoint{6.307801in}{2.542256in}}%
\pgfpathlineto{\pgfqpoint{6.308038in}{1.973863in}}%
\pgfpathlineto{\pgfqpoint{6.308554in}{2.589038in}}%
\pgfpathlineto{\pgfqpoint{6.308911in}{2.470039in}}%
\pgfpathlineto{\pgfqpoint{6.309726in}{2.575298in}}%
\pgfpathlineto{\pgfqpoint{6.309057in}{1.856292in}}%
\pgfpathlineto{\pgfqpoint{6.310008in}{2.518898in}}%
\pgfpathlineto{\pgfqpoint{6.310640in}{1.767139in}}%
\pgfpathlineto{\pgfqpoint{6.310451in}{2.577661in}}%
\pgfpathlineto{\pgfqpoint{6.311119in}{2.392396in}}%
\pgfpathlineto{\pgfqpoint{6.311535in}{2.569662in}}%
\pgfpathlineto{\pgfqpoint{6.311810in}{1.828732in}}%
\pgfpathlineto{\pgfqpoint{6.312231in}{2.434318in}}%
\pgfpathlineto{\pgfqpoint{6.312430in}{1.924742in}}%
\pgfpathlineto{\pgfqpoint{6.312916in}{2.577172in}}%
\pgfpathlineto{\pgfqpoint{6.313342in}{2.411223in}}%
\pgfpathlineto{\pgfqpoint{6.314398in}{2.565410in}}%
\pgfpathlineto{\pgfqpoint{6.313848in}{1.989497in}}%
\pgfpathlineto{\pgfqpoint{6.314450in}{2.425119in}}%
\pgfpathlineto{\pgfqpoint{6.315523in}{1.932532in}}%
\pgfpathlineto{\pgfqpoint{6.314934in}{2.584133in}}%
\pgfpathlineto{\pgfqpoint{6.315560in}{2.331209in}}%
\pgfpathlineto{\pgfqpoint{6.315842in}{2.565767in}}%
\pgfpathlineto{\pgfqpoint{6.316079in}{1.923816in}}%
\pgfpathlineto{\pgfqpoint{6.316672in}{2.419231in}}%
\pgfpathlineto{\pgfqpoint{6.317484in}{1.945080in}}%
\pgfpathlineto{\pgfqpoint{6.316956in}{2.580527in}}%
\pgfpathlineto{\pgfqpoint{6.317783in}{2.345874in}}%
\pgfpathlineto{\pgfqpoint{6.318680in}{2.587650in}}%
\pgfpathlineto{\pgfqpoint{6.318402in}{1.806127in}}%
\pgfpathlineto{\pgfqpoint{6.318894in}{2.498043in}}%
\pgfpathlineto{\pgfqpoint{6.319084in}{1.954655in}}%
\pgfpathlineto{\pgfqpoint{6.319935in}{2.579802in}}%
\pgfpathlineto{\pgfqpoint{6.320005in}{2.429010in}}%
\pgfpathlineto{\pgfqpoint{6.320962in}{1.768578in}}%
\pgfpathlineto{\pgfqpoint{6.320128in}{2.578882in}}%
\pgfpathlineto{\pgfqpoint{6.321116in}{2.424793in}}%
\pgfpathlineto{\pgfqpoint{6.321512in}{2.582316in}}%
\pgfpathlineto{\pgfqpoint{6.321554in}{1.852662in}}%
\pgfpathlineto{\pgfqpoint{6.322222in}{2.278601in}}%
\pgfpathlineto{\pgfqpoint{6.322809in}{1.763531in}}%
\pgfpathlineto{\pgfqpoint{6.322374in}{2.561061in}}%
\pgfpathlineto{\pgfqpoint{6.323327in}{2.378557in}}%
\pgfpathlineto{\pgfqpoint{6.323598in}{2.584739in}}%
\pgfpathlineto{\pgfqpoint{6.323971in}{1.984074in}}%
\pgfpathlineto{\pgfqpoint{6.324438in}{2.411230in}}%
\pgfpathlineto{\pgfqpoint{6.325313in}{1.955205in}}%
\pgfpathlineto{\pgfqpoint{6.325280in}{2.590984in}}%
\pgfpathlineto{\pgfqpoint{6.325545in}{2.370850in}}%
\pgfpathlineto{\pgfqpoint{6.326328in}{2.580383in}}%
\pgfpathlineto{\pgfqpoint{6.326028in}{1.960735in}}%
\pgfpathlineto{\pgfqpoint{6.326655in}{2.413195in}}%
\pgfpathlineto{\pgfqpoint{6.327756in}{1.783256in}}%
\pgfpathlineto{\pgfqpoint{6.327519in}{2.571675in}}%
\pgfpathlineto{\pgfqpoint{6.327765in}{2.451092in}}%
\pgfpathlineto{\pgfqpoint{6.328454in}{2.585339in}}%
\pgfpathlineto{\pgfqpoint{6.327969in}{1.854929in}}%
\pgfpathlineto{\pgfqpoint{6.328870in}{2.379714in}}%
\pgfpathlineto{\pgfqpoint{6.329441in}{1.955379in}}%
\pgfpathlineto{\pgfqpoint{6.329612in}{2.567243in}}%
\pgfpathlineto{\pgfqpoint{6.329980in}{2.435839in}}%
\pgfpathlineto{\pgfqpoint{6.330467in}{2.588487in}}%
\pgfpathlineto{\pgfqpoint{6.330231in}{1.815907in}}%
\pgfpathlineto{\pgfqpoint{6.331084in}{2.405447in}}%
\pgfpathlineto{\pgfqpoint{6.331687in}{1.769033in}}%
\pgfpathlineto{\pgfqpoint{6.331213in}{2.569212in}}%
\pgfpathlineto{\pgfqpoint{6.332195in}{2.347121in}}%
\pgfpathlineto{\pgfqpoint{6.332927in}{1.881075in}}%
\pgfpathlineto{\pgfqpoint{6.332475in}{2.597519in}}%
\pgfpathlineto{\pgfqpoint{6.333302in}{2.389541in}}%
\pgfpathlineto{\pgfqpoint{6.334097in}{2.577481in}}%
\pgfpathlineto{\pgfqpoint{6.333372in}{1.921551in}}%
\pgfpathlineto{\pgfqpoint{6.334412in}{2.527525in}}%
\pgfpathlineto{\pgfqpoint{6.335251in}{1.915224in}}%
\pgfpathlineto{\pgfqpoint{6.335270in}{2.595357in}}%
\pgfpathlineto{\pgfqpoint{6.335524in}{2.312832in}}%
\pgfpathlineto{\pgfqpoint{6.336333in}{1.772515in}}%
\pgfpathlineto{\pgfqpoint{6.336056in}{2.600159in}}%
\pgfpathlineto{\pgfqpoint{6.336533in}{2.436903in}}%
\pgfpathlineto{\pgfqpoint{6.336610in}{2.594438in}}%
\pgfpathlineto{\pgfqpoint{6.337565in}{1.858962in}}%
\pgfpathlineto{\pgfqpoint{6.337644in}{2.463919in}}%
\pgfpathlineto{\pgfqpoint{6.338371in}{1.831662in}}%
\pgfpathlineto{\pgfqpoint{6.337866in}{2.583714in}}%
\pgfpathlineto{\pgfqpoint{6.338755in}{2.493730in}}%
\pgfpathlineto{\pgfqpoint{6.339620in}{1.804056in}}%
\pgfpathlineto{\pgfqpoint{6.339467in}{2.587343in}}%
\pgfpathlineto{\pgfqpoint{6.339867in}{2.138030in}}%
\pgfpathlineto{\pgfqpoint{6.340515in}{2.567345in}}%
\pgfpathlineto{\pgfqpoint{6.340801in}{1.872869in}}%
\pgfpathlineto{\pgfqpoint{6.340978in}{2.365635in}}%
\pgfpathlineto{\pgfqpoint{6.341736in}{2.593801in}}%
\pgfpathlineto{\pgfqpoint{6.341362in}{1.908353in}}%
\pgfpathlineto{\pgfqpoint{6.342089in}{2.475868in}}%
\pgfpathlineto{\pgfqpoint{6.342790in}{1.920925in}}%
\pgfpathlineto{\pgfqpoint{6.342817in}{2.568458in}}%
\pgfpathlineto{\pgfqpoint{6.343201in}{2.289406in}}%
\pgfpathlineto{\pgfqpoint{6.343739in}{2.577555in}}%
\pgfpathlineto{\pgfqpoint{6.343947in}{1.683387in}}%
\pgfpathlineto{\pgfqpoint{6.344313in}{2.507225in}}%
\pgfpathlineto{\pgfqpoint{6.344473in}{1.874596in}}%
\pgfpathlineto{\pgfqpoint{6.345229in}{2.571593in}}%
\pgfpathlineto{\pgfqpoint{6.345424in}{2.364940in}}%
\pgfpathlineto{\pgfqpoint{6.346048in}{2.560155in}}%
\pgfpathlineto{\pgfqpoint{6.346320in}{1.778896in}}%
\pgfpathlineto{\pgfqpoint{6.346535in}{2.447349in}}%
\pgfpathlineto{\pgfqpoint{6.346676in}{1.794573in}}%
\pgfpathlineto{\pgfqpoint{6.346935in}{2.590245in}}%
\pgfpathlineto{\pgfqpoint{6.347646in}{2.510599in}}%
\pgfpathlineto{\pgfqpoint{6.348269in}{1.623109in}}%
\pgfpathlineto{\pgfqpoint{6.348451in}{2.577589in}}%
\pgfpathlineto{\pgfqpoint{6.348757in}{2.483546in}}%
\pgfpathlineto{\pgfqpoint{6.348806in}{1.988947in}}%
\pgfpathlineto{\pgfqpoint{6.349248in}{2.580257in}}%
\pgfpathlineto{\pgfqpoint{6.349867in}{2.309414in}}%
\pgfpathlineto{\pgfqpoint{6.350430in}{2.572654in}}%
\pgfpathlineto{\pgfqpoint{6.350707in}{1.790351in}}%
\pgfpathlineto{\pgfqpoint{6.350978in}{2.415733in}}%
\pgfpathlineto{\pgfqpoint{6.351688in}{1.949671in}}%
\pgfpathlineto{\pgfqpoint{6.352010in}{2.607069in}}%
\pgfpathlineto{\pgfqpoint{6.352080in}{2.531322in}}%
\pgfpathlineto{\pgfqpoint{6.352735in}{2.581348in}}%
\pgfpathlineto{\pgfqpoint{6.352875in}{1.905140in}}%
\pgfpathlineto{\pgfqpoint{6.353155in}{2.375497in}}%
\pgfpathlineto{\pgfqpoint{6.353790in}{1.795180in}}%
\pgfpathlineto{\pgfqpoint{6.353480in}{2.598957in}}%
\pgfpathlineto{\pgfqpoint{6.354265in}{2.353111in}}%
\pgfpathlineto{\pgfqpoint{6.354988in}{2.584577in}}%
\pgfpathlineto{\pgfqpoint{6.354965in}{1.961349in}}%
\pgfpathlineto{\pgfqpoint{6.355375in}{2.360202in}}%
\pgfpathlineto{\pgfqpoint{6.355612in}{1.986808in}}%
\pgfpathlineto{\pgfqpoint{6.355587in}{2.581195in}}%
\pgfpathlineto{\pgfqpoint{6.356485in}{2.400288in}}%
\pgfpathlineto{\pgfqpoint{6.357023in}{2.579853in}}%
\pgfpathlineto{\pgfqpoint{6.357559in}{1.816973in}}%
\pgfpathlineto{\pgfqpoint{6.357596in}{2.564258in}}%
\pgfpathlineto{\pgfqpoint{6.357646in}{1.935729in}}%
\pgfpathlineto{\pgfqpoint{6.357831in}{2.573608in}}%
\pgfpathlineto{\pgfqpoint{6.358707in}{2.334851in}}%
\pgfpathlineto{\pgfqpoint{6.358804in}{2.574472in}}%
\pgfpathlineto{\pgfqpoint{6.358713in}{1.959863in}}%
\pgfpathlineto{\pgfqpoint{6.359818in}{2.415011in}}%
\pgfpathlineto{\pgfqpoint{6.360802in}{1.894254in}}%
\pgfpathlineto{\pgfqpoint{6.360681in}{2.576942in}}%
\pgfpathlineto{\pgfqpoint{6.360929in}{2.297170in}}%
\pgfpathlineto{\pgfqpoint{6.361876in}{2.587453in}}%
\pgfpathlineto{\pgfqpoint{6.361423in}{1.878602in}}%
\pgfpathlineto{\pgfqpoint{6.362040in}{2.339339in}}%
\pgfpathlineto{\pgfqpoint{6.363108in}{1.908537in}}%
\pgfpathlineto{\pgfqpoint{6.362648in}{2.578099in}}%
\pgfpathlineto{\pgfqpoint{6.363151in}{2.306333in}}%
\pgfpathlineto{\pgfqpoint{6.363613in}{2.587196in}}%
\pgfpathlineto{\pgfqpoint{6.363403in}{1.936632in}}%
\pgfpathlineto{\pgfqpoint{6.364261in}{2.453493in}}%
\pgfpathlineto{\pgfqpoint{6.365139in}{1.882865in}}%
\pgfpathlineto{\pgfqpoint{6.364925in}{2.582604in}}%
\pgfpathlineto{\pgfqpoint{6.365372in}{2.381049in}}%
\pgfpathlineto{\pgfqpoint{6.366184in}{2.568354in}}%
\pgfpathlineto{\pgfqpoint{6.365408in}{1.923850in}}%
\pgfpathlineto{\pgfqpoint{6.366464in}{2.520237in}}%
\pgfpathlineto{\pgfqpoint{6.367480in}{1.835152in}}%
\pgfpathlineto{\pgfqpoint{6.367192in}{2.614513in}}%
\pgfpathlineto{\pgfqpoint{6.367575in}{2.220276in}}%
\pgfpathlineto{\pgfqpoint{6.368512in}{2.588165in}}%
\pgfpathlineto{\pgfqpoint{6.368660in}{1.883240in}}%
\pgfpathlineto{\pgfqpoint{6.368687in}{2.325231in}}%
\pgfpathlineto{\pgfqpoint{6.369412in}{1.858654in}}%
\pgfpathlineto{\pgfqpoint{6.369235in}{2.572905in}}%
\pgfpathlineto{\pgfqpoint{6.369794in}{2.278126in}}%
\pgfpathlineto{\pgfqpoint{6.370803in}{2.601879in}}%
\pgfpathlineto{\pgfqpoint{6.370111in}{1.908373in}}%
\pgfpathlineto{\pgfqpoint{6.370905in}{2.423350in}}%
\pgfpathlineto{\pgfqpoint{6.370989in}{1.885325in}}%
\pgfpathlineto{\pgfqpoint{6.371530in}{2.641507in}}%
\pgfpathlineto{\pgfqpoint{6.372016in}{2.385052in}}%
\pgfpathlineto{\pgfqpoint{6.372506in}{2.604152in}}%
\pgfpathlineto{\pgfqpoint{6.372546in}{1.857739in}}%
\pgfpathlineto{\pgfqpoint{6.373128in}{2.526479in}}%
\pgfpathlineto{\pgfqpoint{6.373335in}{1.944213in}}%
\pgfpathlineto{\pgfqpoint{6.374056in}{2.595973in}}%
\pgfpathlineto{\pgfqpoint{6.374240in}{2.419852in}}%
\pgfpathlineto{\pgfqpoint{6.374413in}{1.626075in}}%
\pgfpathlineto{\pgfqpoint{6.375174in}{2.580505in}}%
\pgfpathlineto{\pgfqpoint{6.375348in}{2.236865in}}%
\pgfpathlineto{\pgfqpoint{6.376101in}{2.617944in}}%
\pgfpathlineto{\pgfqpoint{6.376191in}{1.882841in}}%
\pgfpathlineto{\pgfqpoint{6.376459in}{2.341515in}}%
\pgfpathlineto{\pgfqpoint{6.376608in}{2.591413in}}%
\pgfpathlineto{\pgfqpoint{6.377441in}{1.840534in}}%
\pgfpathlineto{\pgfqpoint{6.377569in}{2.450234in}}%
\pgfpathlineto{\pgfqpoint{6.377905in}{1.996639in}}%
\pgfpathlineto{\pgfqpoint{6.378097in}{2.581676in}}%
\pgfpathlineto{\pgfqpoint{6.378679in}{2.361592in}}%
\pgfpathlineto{\pgfqpoint{6.379178in}{2.581196in}}%
\pgfpathlineto{\pgfqpoint{6.378873in}{1.935347in}}%
\pgfpathlineto{\pgfqpoint{6.379790in}{2.406693in}}%
\pgfpathlineto{\pgfqpoint{6.380637in}{1.916481in}}%
\pgfpathlineto{\pgfqpoint{6.379865in}{2.582682in}}%
\pgfpathlineto{\pgfqpoint{6.380902in}{2.371288in}}%
\pgfpathlineto{\pgfqpoint{6.381314in}{2.586012in}}%
\pgfpathlineto{\pgfqpoint{6.381006in}{1.917717in}}%
\pgfpathlineto{\pgfqpoint{6.382014in}{2.491085in}}%
\pgfpathlineto{\pgfqpoint{6.382041in}{1.795201in}}%
\pgfpathlineto{\pgfqpoint{6.382973in}{2.562738in}}%
\pgfpathlineto{\pgfqpoint{6.383125in}{2.399322in}}%
\pgfpathlineto{\pgfqpoint{6.383478in}{2.592739in}}%
\pgfpathlineto{\pgfqpoint{6.383427in}{1.798711in}}%
\pgfpathlineto{\pgfqpoint{6.384234in}{2.500691in}}%
\pgfpathlineto{\pgfqpoint{6.384707in}{1.653716in}}%
\pgfpathlineto{\pgfqpoint{6.385311in}{2.579740in}}%
\pgfpathlineto{\pgfqpoint{6.385346in}{2.362249in}}%
\pgfpathlineto{\pgfqpoint{6.386239in}{2.591209in}}%
\pgfpathlineto{\pgfqpoint{6.385462in}{1.883911in}}%
\pgfpathlineto{\pgfqpoint{6.386456in}{2.368202in}}%
\pgfpathlineto{\pgfqpoint{6.387079in}{1.945967in}}%
\pgfpathlineto{\pgfqpoint{6.387511in}{2.597239in}}%
\pgfpathlineto{\pgfqpoint{6.387561in}{2.340861in}}%
\pgfpathlineto{\pgfqpoint{6.387913in}{2.586180in}}%
\pgfpathlineto{\pgfqpoint{6.387698in}{1.736975in}}%
\pgfpathlineto{\pgfqpoint{6.388672in}{2.318068in}}%
\pgfpathlineto{\pgfqpoint{6.389166in}{2.593356in}}%
\pgfpathlineto{\pgfqpoint{6.388803in}{1.756028in}}%
\pgfpathlineto{\pgfqpoint{6.389784in}{2.379675in}}%
\pgfpathlineto{\pgfqpoint{6.390169in}{1.873304in}}%
\pgfpathlineto{\pgfqpoint{6.390449in}{2.585288in}}%
\pgfpathlineto{\pgfqpoint{6.390895in}{2.317029in}}%
\pgfpathlineto{\pgfqpoint{6.391595in}{2.581479in}}%
\pgfpathlineto{\pgfqpoint{6.391371in}{1.889571in}}%
\pgfpathlineto{\pgfqpoint{6.392007in}{2.471254in}}%
\pgfpathlineto{\pgfqpoint{6.392400in}{1.811857in}}%
\pgfpathlineto{\pgfqpoint{6.392944in}{2.593850in}}%
\pgfpathlineto{\pgfqpoint{6.393118in}{2.293747in}}%
\pgfpathlineto{\pgfqpoint{6.394006in}{2.587649in}}%
\pgfpathlineto{\pgfqpoint{6.393402in}{1.783282in}}%
\pgfpathlineto{\pgfqpoint{6.394228in}{2.428025in}}%
\pgfpathlineto{\pgfqpoint{6.395079in}{1.639608in}}%
\pgfpathlineto{\pgfqpoint{6.394720in}{2.600737in}}%
\pgfpathlineto{\pgfqpoint{6.395339in}{2.404622in}}%
\pgfpathlineto{\pgfqpoint{6.395877in}{2.575592in}}%
\pgfpathlineto{\pgfqpoint{6.396403in}{1.830120in}}%
\pgfpathlineto{\pgfqpoint{6.396450in}{2.451892in}}%
\pgfpathlineto{\pgfqpoint{6.396554in}{1.949847in}}%
\pgfpathlineto{\pgfqpoint{6.397548in}{2.602108in}}%
\pgfpathlineto{\pgfqpoint{6.397561in}{2.428317in}}%
\pgfpathlineto{\pgfqpoint{6.397657in}{2.596521in}}%
\pgfpathlineto{\pgfqpoint{6.398450in}{1.935009in}}%
\pgfpathlineto{\pgfqpoint{6.398668in}{2.461999in}}%
\pgfpathlineto{\pgfqpoint{6.399379in}{1.910840in}}%
\pgfpathlineto{\pgfqpoint{6.399013in}{2.575769in}}%
\pgfpathlineto{\pgfqpoint{6.399778in}{2.402012in}}%
\pgfpathlineto{\pgfqpoint{6.400854in}{2.598759in}}%
\pgfpathlineto{\pgfqpoint{6.400397in}{1.759151in}}%
\pgfpathlineto{\pgfqpoint{6.400889in}{2.479988in}}%
\pgfpathlineto{\pgfqpoint{6.401886in}{1.920646in}}%
\pgfpathlineto{\pgfqpoint{6.401522in}{2.574775in}}%
\pgfpathlineto{\pgfqpoint{6.402001in}{2.303254in}}%
\pgfpathlineto{\pgfqpoint{6.402223in}{2.580986in}}%
\pgfpathlineto{\pgfqpoint{6.402625in}{1.818169in}}%
\pgfpathlineto{\pgfqpoint{6.403114in}{2.521698in}}%
\pgfpathlineto{\pgfqpoint{6.403354in}{1.894603in}}%
\pgfpathlineto{\pgfqpoint{6.403612in}{2.576963in}}%
\pgfpathlineto{\pgfqpoint{6.404226in}{2.332741in}}%
\pgfpathlineto{\pgfqpoint{6.404406in}{2.601751in}}%
\pgfpathlineto{\pgfqpoint{6.404885in}{1.720791in}}%
\pgfpathlineto{\pgfqpoint{6.405337in}{2.185252in}}%
\pgfpathlineto{\pgfqpoint{6.405823in}{2.591576in}}%
\pgfpathlineto{\pgfqpoint{6.405470in}{1.878305in}}%
\pgfpathlineto{\pgfqpoint{6.406449in}{2.356862in}}%
\pgfpathlineto{\pgfqpoint{6.407342in}{1.645076in}}%
\pgfpathlineto{\pgfqpoint{6.406663in}{2.588339in}}%
\pgfpathlineto{\pgfqpoint{6.407559in}{2.405926in}}%
\pgfpathlineto{\pgfqpoint{6.408445in}{2.584758in}}%
\pgfpathlineto{\pgfqpoint{6.408198in}{1.958497in}}%
\pgfpathlineto{\pgfqpoint{6.408670in}{2.492592in}}%
\pgfpathlineto{\pgfqpoint{6.409606in}{1.967221in}}%
\pgfpathlineto{\pgfqpoint{6.409575in}{2.589098in}}%
\pgfpathlineto{\pgfqpoint{6.409782in}{2.386746in}}%
\pgfpathlineto{\pgfqpoint{6.410047in}{2.609792in}}%
\pgfpathlineto{\pgfqpoint{6.410666in}{1.894402in}}%
\pgfpathlineto{\pgfqpoint{6.410884in}{2.337072in}}%
\pgfpathlineto{\pgfqpoint{6.411262in}{1.814444in}}%
\pgfpathlineto{\pgfqpoint{6.411678in}{2.587745in}}%
\pgfpathlineto{\pgfqpoint{6.411994in}{2.437580in}}%
\pgfpathlineto{\pgfqpoint{6.412201in}{2.585141in}}%
\pgfpathlineto{\pgfqpoint{6.412454in}{1.954457in}}%
\pgfpathlineto{\pgfqpoint{6.412955in}{2.460780in}}%
\pgfpathlineto{\pgfqpoint{6.412980in}{1.914336in}}%
\pgfpathlineto{\pgfqpoint{6.413190in}{2.594852in}}%
\pgfpathlineto{\pgfqpoint{6.414067in}{2.275876in}}%
\pgfpathlineto{\pgfqpoint{6.415066in}{2.591804in}}%
\pgfpathlineto{\pgfqpoint{6.414313in}{1.824052in}}%
\pgfpathlineto{\pgfqpoint{6.415177in}{2.417948in}}%
\pgfpathlineto{\pgfqpoint{6.415294in}{1.972008in}}%
\pgfpathlineto{\pgfqpoint{6.415409in}{2.630119in}}%
\pgfpathlineto{\pgfqpoint{6.416288in}{2.421174in}}%
\pgfpathlineto{\pgfqpoint{6.416749in}{2.626278in}}%
\pgfpathlineto{\pgfqpoint{6.416633in}{1.829970in}}%
\pgfpathlineto{\pgfqpoint{6.417394in}{2.412363in}}%
\pgfpathlineto{\pgfqpoint{6.417569in}{2.009158in}}%
\pgfpathlineto{\pgfqpoint{6.417802in}{2.592732in}}%
\pgfpathlineto{\pgfqpoint{6.418505in}{2.133388in}}%
\pgfpathlineto{\pgfqpoint{6.418666in}{2.612042in}}%
\pgfpathlineto{\pgfqpoint{6.419566in}{1.888122in}}%
\pgfpathlineto{\pgfqpoint{6.419618in}{2.476949in}}%
\pgfpathlineto{\pgfqpoint{6.419683in}{2.594735in}}%
\pgfpathlineto{\pgfqpoint{6.419884in}{1.783573in}}%
\pgfpathlineto{\pgfqpoint{6.420728in}{2.352991in}}%
\pgfpathlineto{\pgfqpoint{6.420981in}{2.597129in}}%
\pgfpathlineto{\pgfqpoint{6.421739in}{1.906743in}}%
\pgfpathlineto{\pgfqpoint{6.421840in}{2.447098in}}%
\pgfpathlineto{\pgfqpoint{6.422032in}{1.903036in}}%
\pgfpathlineto{\pgfqpoint{6.422319in}{2.589852in}}%
\pgfpathlineto{\pgfqpoint{6.422950in}{2.336656in}}%
\pgfpathlineto{\pgfqpoint{6.423997in}{2.589980in}}%
\pgfpathlineto{\pgfqpoint{6.424007in}{1.836976in}}%
\pgfpathlineto{\pgfqpoint{6.424062in}{2.467766in}}%
\pgfpathlineto{\pgfqpoint{6.424090in}{1.852466in}}%
\pgfpathlineto{\pgfqpoint{6.424670in}{2.617035in}}%
\pgfpathlineto{\pgfqpoint{6.425172in}{2.231037in}}%
\pgfpathlineto{\pgfqpoint{6.426041in}{2.602801in}}%
\pgfpathlineto{\pgfqpoint{6.426180in}{1.976616in}}%
\pgfpathlineto{\pgfqpoint{6.426284in}{2.461888in}}%
\pgfpathlineto{\pgfqpoint{6.426309in}{1.962019in}}%
\pgfpathlineto{\pgfqpoint{6.426947in}{2.587535in}}%
\pgfpathlineto{\pgfqpoint{6.427395in}{2.296617in}}%
\pgfpathlineto{\pgfqpoint{6.428404in}{2.596409in}}%
\pgfpathlineto{\pgfqpoint{6.427906in}{1.983559in}}%
\pgfpathlineto{\pgfqpoint{6.428505in}{2.399771in}}%
\pgfpathlineto{\pgfqpoint{6.428807in}{1.699743in}}%
\pgfpathlineto{\pgfqpoint{6.429398in}{2.598350in}}%
\pgfpathlineto{\pgfqpoint{6.429616in}{2.291802in}}%
\pgfpathlineto{\pgfqpoint{6.429645in}{2.606219in}}%
\pgfpathlineto{\pgfqpoint{6.429772in}{1.877673in}}%
\pgfpathlineto{\pgfqpoint{6.430727in}{2.397529in}}%
\pgfpathlineto{\pgfqpoint{6.431054in}{2.589760in}}%
\pgfpathlineto{\pgfqpoint{6.431212in}{1.934645in}}%
\pgfpathlineto{\pgfqpoint{6.431832in}{2.468528in}}%
\pgfpathlineto{\pgfqpoint{6.432341in}{1.862771in}}%
\pgfpathlineto{\pgfqpoint{6.432672in}{2.595265in}}%
\pgfpathlineto{\pgfqpoint{6.432943in}{2.062462in}}%
\pgfpathlineto{\pgfqpoint{6.433531in}{2.595929in}}%
\pgfpathlineto{\pgfqpoint{6.433078in}{1.909103in}}%
\pgfpathlineto{\pgfqpoint{6.434054in}{2.240750in}}%
\pgfpathlineto{\pgfqpoint{6.434177in}{2.616771in}}%
\pgfpathlineto{\pgfqpoint{6.434378in}{1.931276in}}%
\pgfpathlineto{\pgfqpoint{6.435167in}{2.482568in}}%
\pgfpathlineto{\pgfqpoint{6.435592in}{1.957930in}}%
\pgfpathlineto{\pgfqpoint{6.435270in}{2.590566in}}%
\pgfpathlineto{\pgfqpoint{6.436277in}{2.468516in}}%
\pgfpathlineto{\pgfqpoint{6.436478in}{2.610348in}}%
\pgfpathlineto{\pgfqpoint{6.436553in}{1.752303in}}%
\pgfpathlineto{\pgfqpoint{6.437387in}{2.499308in}}%
\pgfpathlineto{\pgfqpoint{6.437767in}{1.724627in}}%
\pgfpathlineto{\pgfqpoint{6.438318in}{2.596553in}}%
\pgfpathlineto{\pgfqpoint{6.438498in}{2.332060in}}%
\pgfpathlineto{\pgfqpoint{6.439235in}{2.595887in}}%
\pgfpathlineto{\pgfqpoint{6.439014in}{1.943425in}}%
\pgfpathlineto{\pgfqpoint{6.439609in}{2.326116in}}%
\pgfpathlineto{\pgfqpoint{6.440410in}{2.610453in}}%
\pgfpathlineto{\pgfqpoint{6.440501in}{1.868313in}}%
\pgfpathlineto{\pgfqpoint{6.440720in}{2.439290in}}%
\pgfpathlineto{\pgfqpoint{6.441070in}{1.873093in}}%
\pgfpathlineto{\pgfqpoint{6.441327in}{2.600433in}}%
\pgfpathlineto{\pgfqpoint{6.441832in}{2.330952in}}%
\pgfpathlineto{\pgfqpoint{6.442870in}{2.586890in}}%
\pgfpathlineto{\pgfqpoint{6.442247in}{1.848003in}}%
\pgfpathlineto{\pgfqpoint{6.442943in}{2.462970in}}%
\pgfpathlineto{\pgfqpoint{6.443421in}{1.849120in}}%
\pgfpathlineto{\pgfqpoint{6.443713in}{2.606043in}}%
\pgfpathlineto{\pgfqpoint{6.444055in}{2.339396in}}%
\pgfpathlineto{\pgfqpoint{6.444291in}{2.610719in}}%
\pgfpathlineto{\pgfqpoint{6.444934in}{1.657250in}}%
\pgfpathlineto{\pgfqpoint{6.445166in}{2.390057in}}%
\pgfpathlineto{\pgfqpoint{6.445896in}{1.906044in}}%
\pgfpathlineto{\pgfqpoint{6.445330in}{2.630684in}}%
\pgfpathlineto{\pgfqpoint{6.446276in}{2.305680in}}%
\pgfpathlineto{\pgfqpoint{6.446742in}{2.603007in}}%
\pgfpathlineto{\pgfqpoint{6.447189in}{1.868535in}}%
\pgfpathlineto{\pgfqpoint{6.447388in}{2.444922in}}%
\pgfpathlineto{\pgfqpoint{6.447884in}{1.921073in}}%
\pgfpathlineto{\pgfqpoint{6.447787in}{2.597009in}}%
\pgfpathlineto{\pgfqpoint{6.448498in}{2.401916in}}%
\pgfpathlineto{\pgfqpoint{6.449500in}{2.611814in}}%
\pgfpathlineto{\pgfqpoint{6.449460in}{1.828858in}}%
\pgfpathlineto{\pgfqpoint{6.449611in}{2.571385in}}%
\pgfpathlineto{\pgfqpoint{6.450176in}{1.886596in}}%
\pgfpathlineto{\pgfqpoint{6.449809in}{2.595459in}}%
\pgfpathlineto{\pgfqpoint{6.450722in}{2.360629in}}%
\pgfpathlineto{\pgfqpoint{6.451284in}{2.595226in}}%
\pgfpathlineto{\pgfqpoint{6.451579in}{1.975134in}}%
\pgfpathlineto{\pgfqpoint{6.451834in}{2.461117in}}%
\pgfpathlineto{\pgfqpoint{6.452234in}{1.727081in}}%
\pgfpathlineto{\pgfqpoint{6.451887in}{2.608050in}}%
\pgfpathlineto{\pgfqpoint{6.452945in}{2.388216in}}%
\pgfpathlineto{\pgfqpoint{6.453614in}{1.771759in}}%
\pgfpathlineto{\pgfqpoint{6.453612in}{2.578756in}}%
\pgfpathlineto{\pgfqpoint{6.454054in}{2.443851in}}%
\pgfpathlineto{\pgfqpoint{6.454339in}{2.591737in}}%
\pgfpathlineto{\pgfqpoint{6.454648in}{1.893192in}}%
\pgfpathlineto{\pgfqpoint{6.455164in}{2.449706in}}%
\pgfpathlineto{\pgfqpoint{6.455900in}{1.938225in}}%
\pgfpathlineto{\pgfqpoint{6.455299in}{2.592997in}}%
\pgfpathlineto{\pgfqpoint{6.456276in}{2.277702in}}%
\pgfpathlineto{\pgfqpoint{6.457386in}{2.622760in}}%
\pgfpathlineto{\pgfqpoint{6.456461in}{1.899564in}}%
\pgfpathlineto{\pgfqpoint{6.457387in}{2.284976in}}%
\pgfpathlineto{\pgfqpoint{6.458195in}{2.606508in}}%
\pgfpathlineto{\pgfqpoint{6.457498in}{1.801743in}}%
\pgfpathlineto{\pgfqpoint{6.458500in}{2.438314in}}%
\pgfpathlineto{\pgfqpoint{6.458836in}{1.868027in}}%
\pgfpathlineto{\pgfqpoint{6.459454in}{2.586844in}}%
\pgfpathlineto{\pgfqpoint{6.459610in}{2.470995in}}%
\pgfpathlineto{\pgfqpoint{6.459982in}{2.618988in}}%
\pgfpathlineto{\pgfqpoint{6.460245in}{1.960414in}}%
\pgfpathlineto{\pgfqpoint{6.460719in}{2.445040in}}%
\pgfpathlineto{\pgfqpoint{6.460848in}{1.588035in}}%
\pgfpathlineto{\pgfqpoint{6.461054in}{2.606186in}}%
\pgfpathlineto{\pgfqpoint{6.461830in}{2.275193in}}%
\pgfpathlineto{\pgfqpoint{6.462470in}{2.603638in}}%
\pgfpathlineto{\pgfqpoint{6.462510in}{2.004586in}}%
\pgfpathlineto{\pgfqpoint{6.462941in}{2.528237in}}%
\pgfpathlineto{\pgfqpoint{6.463070in}{1.933162in}}%
\pgfpathlineto{\pgfqpoint{6.463027in}{2.596913in}}%
\pgfpathlineto{\pgfqpoint{6.464052in}{2.434605in}}%
\pgfpathlineto{\pgfqpoint{6.464424in}{2.614777in}}%
\pgfpathlineto{\pgfqpoint{6.464110in}{1.988253in}}%
\pgfpathlineto{\pgfqpoint{6.465163in}{2.572343in}}%
\pgfpathlineto{\pgfqpoint{6.465708in}{1.920664in}}%
\pgfpathlineto{\pgfqpoint{6.466072in}{2.616277in}}%
\pgfpathlineto{\pgfqpoint{6.466275in}{2.223723in}}%
\pgfpathlineto{\pgfqpoint{6.466810in}{2.596625in}}%
\pgfpathlineto{\pgfqpoint{6.467352in}{1.922461in}}%
\pgfpathlineto{\pgfqpoint{6.467386in}{2.480157in}}%
\pgfpathlineto{\pgfqpoint{6.467814in}{1.898155in}}%
\pgfpathlineto{\pgfqpoint{6.467727in}{2.601084in}}%
\pgfpathlineto{\pgfqpoint{6.468497in}{2.392177in}}%
\pgfpathlineto{\pgfqpoint{6.469254in}{2.623303in}}%
\pgfpathlineto{\pgfqpoint{6.468674in}{1.874028in}}%
\pgfpathlineto{\pgfqpoint{6.469607in}{2.422895in}}%
\pgfpathlineto{\pgfqpoint{6.470627in}{1.985327in}}%
\pgfpathlineto{\pgfqpoint{6.470489in}{2.603814in}}%
\pgfpathlineto{\pgfqpoint{6.470717in}{2.403644in}}%
\pgfpathlineto{\pgfqpoint{6.471629in}{2.599008in}}%
\pgfpathlineto{\pgfqpoint{6.471713in}{1.928504in}}%
\pgfpathlineto{\pgfqpoint{6.471828in}{2.399008in}}%
\pgfpathlineto{\pgfqpoint{6.471909in}{1.919018in}}%
\pgfpathlineto{\pgfqpoint{6.472600in}{2.595231in}}%
\pgfpathlineto{\pgfqpoint{6.472939in}{2.355494in}}%
\pgfpathlineto{\pgfqpoint{6.473144in}{2.604336in}}%
\pgfpathlineto{\pgfqpoint{6.472967in}{1.539820in}}%
\pgfpathlineto{\pgfqpoint{6.474049in}{2.536231in}}%
\pgfpathlineto{\pgfqpoint{6.475099in}{1.970225in}}%
\pgfpathlineto{\pgfqpoint{6.474373in}{2.628054in}}%
\pgfpathlineto{\pgfqpoint{6.475161in}{2.334800in}}%
\pgfpathlineto{\pgfqpoint{6.475778in}{2.594387in}}%
\pgfpathlineto{\pgfqpoint{6.476136in}{1.820643in}}%
\pgfpathlineto{\pgfqpoint{6.476272in}{2.512309in}}%
\pgfpathlineto{\pgfqpoint{6.476685in}{1.688376in}}%
\pgfpathlineto{\pgfqpoint{6.477375in}{2.605944in}}%
\pgfpathlineto{\pgfqpoint{6.477383in}{2.493950in}}%
\pgfpathlineto{\pgfqpoint{6.477845in}{1.895670in}}%
\pgfpathlineto{\pgfqpoint{6.477854in}{2.602063in}}%
\pgfpathlineto{\pgfqpoint{6.478494in}{2.384773in}}%
\pgfpathlineto{\pgfqpoint{6.479249in}{2.606636in}}%
\pgfpathlineto{\pgfqpoint{6.479272in}{1.740463in}}%
\pgfpathlineto{\pgfqpoint{6.479606in}{2.556249in}}%
\pgfpathlineto{\pgfqpoint{6.480134in}{1.864024in}}%
\pgfpathlineto{\pgfqpoint{6.480072in}{2.602118in}}%
\pgfpathlineto{\pgfqpoint{6.480716in}{2.321535in}}%
\pgfpathlineto{\pgfqpoint{6.480830in}{2.604167in}}%
\pgfpathlineto{\pgfqpoint{6.481665in}{1.982766in}}%
\pgfpathlineto{\pgfqpoint{6.481828in}{2.438855in}}%
\pgfpathlineto{\pgfqpoint{6.482921in}{1.962089in}}%
\pgfpathlineto{\pgfqpoint{6.482034in}{2.627589in}}%
\pgfpathlineto{\pgfqpoint{6.482938in}{2.504273in}}%
\pgfpathlineto{\pgfqpoint{6.483546in}{2.607520in}}%
\pgfpathlineto{\pgfqpoint{6.484019in}{1.880214in}}%
\pgfpathlineto{\pgfqpoint{6.484048in}{2.493884in}}%
\pgfpathlineto{\pgfqpoint{6.484839in}{1.788107in}}%
\pgfpathlineto{\pgfqpoint{6.484863in}{2.626710in}}%
\pgfpathlineto{\pgfqpoint{6.485159in}{2.388869in}}%
\pgfpathlineto{\pgfqpoint{6.485963in}{2.611176in}}%
\pgfpathlineto{\pgfqpoint{6.485822in}{1.988588in}}%
\pgfpathlineto{\pgfqpoint{6.486270in}{2.553606in}}%
\pgfpathlineto{\pgfqpoint{6.486772in}{1.912117in}}%
\pgfpathlineto{\pgfqpoint{6.486319in}{2.599546in}}%
\pgfpathlineto{\pgfqpoint{6.487382in}{2.454321in}}%
\pgfpathlineto{\pgfqpoint{6.488133in}{2.600749in}}%
\pgfpathlineto{\pgfqpoint{6.488167in}{1.872185in}}%
\pgfpathlineto{\pgfqpoint{6.488491in}{2.489744in}}%
\pgfpathlineto{\pgfqpoint{6.489022in}{1.951546in}}%
\pgfpathlineto{\pgfqpoint{6.488867in}{2.630195in}}%
\pgfpathlineto{\pgfqpoint{6.489602in}{2.407932in}}%
\pgfpathlineto{\pgfqpoint{6.490125in}{2.604222in}}%
\pgfpathlineto{\pgfqpoint{6.489925in}{2.042067in}}%
\pgfpathlineto{\pgfqpoint{6.490705in}{2.298156in}}%
\pgfpathlineto{\pgfqpoint{6.491118in}{1.939214in}}%
\pgfpathlineto{\pgfqpoint{6.491360in}{2.608573in}}%
\pgfpathlineto{\pgfqpoint{6.491812in}{2.338653in}}%
\pgfpathlineto{\pgfqpoint{6.492312in}{2.605592in}}%
\pgfpathlineto{\pgfqpoint{6.492163in}{1.915251in}}%
\pgfpathlineto{\pgfqpoint{6.492923in}{2.505442in}}%
\pgfpathlineto{\pgfqpoint{6.493327in}{1.846668in}}%
\pgfpathlineto{\pgfqpoint{6.493793in}{2.640303in}}%
\pgfpathlineto{\pgfqpoint{6.494035in}{2.395732in}}%
\pgfpathlineto{\pgfqpoint{6.494375in}{2.606938in}}%
\pgfpathlineto{\pgfqpoint{6.494131in}{1.982850in}}%
\pgfpathlineto{\pgfqpoint{6.495143in}{2.419062in}}%
\pgfpathlineto{\pgfqpoint{6.496231in}{1.866103in}}%
\pgfpathlineto{\pgfqpoint{6.495693in}{2.606674in}}%
\pgfpathlineto{\pgfqpoint{6.496254in}{2.490554in}}%
\pgfpathlineto{\pgfqpoint{6.496366in}{1.911933in}}%
\pgfpathlineto{\pgfqpoint{6.497181in}{2.599699in}}%
\pgfpathlineto{\pgfqpoint{6.497365in}{2.323013in}}%
\pgfpathlineto{\pgfqpoint{6.497451in}{2.607506in}}%
\pgfpathlineto{\pgfqpoint{6.498314in}{1.852566in}}%
\pgfpathlineto{\pgfqpoint{6.498476in}{2.508437in}}%
\pgfpathlineto{\pgfqpoint{6.499151in}{2.019218in}}%
\pgfpathlineto{\pgfqpoint{6.499095in}{2.613179in}}%
\pgfpathlineto{\pgfqpoint{6.499588in}{2.434487in}}%
\pgfpathlineto{\pgfqpoint{6.500060in}{2.610178in}}%
\pgfpathlineto{\pgfqpoint{6.499905in}{1.850161in}}%
\pgfpathlineto{\pgfqpoint{6.500698in}{2.474476in}}%
\pgfpathlineto{\pgfqpoint{6.501721in}{1.852471in}}%
\pgfpathlineto{\pgfqpoint{6.500988in}{2.632563in}}%
\pgfpathlineto{\pgfqpoint{6.501809in}{2.326324in}}%
\pgfpathlineto{\pgfqpoint{6.502840in}{2.611276in}}%
\pgfpathlineto{\pgfqpoint{6.502883in}{1.831367in}}%
\pgfpathlineto{\pgfqpoint{6.502920in}{2.409085in}}%
\pgfpathlineto{\pgfqpoint{6.503885in}{1.885107in}}%
\pgfpathlineto{\pgfqpoint{6.503642in}{2.602474in}}%
\pgfpathlineto{\pgfqpoint{6.504029in}{2.150025in}}%
\pgfpathlineto{\pgfqpoint{6.504393in}{2.603397in}}%
\pgfpathlineto{\pgfqpoint{6.504447in}{1.888520in}}%
\pgfpathlineto{\pgfqpoint{6.505141in}{2.456607in}}%
\pgfpathlineto{\pgfqpoint{6.505699in}{1.903341in}}%
\pgfpathlineto{\pgfqpoint{6.505373in}{2.632143in}}%
\pgfpathlineto{\pgfqpoint{6.506251in}{2.372180in}}%
\pgfpathlineto{\pgfqpoint{6.506828in}{2.603447in}}%
\pgfpathlineto{\pgfqpoint{6.506930in}{1.911893in}}%
\pgfpathlineto{\pgfqpoint{6.507361in}{2.328104in}}%
\pgfpathlineto{\pgfqpoint{6.507963in}{1.925642in}}%
\pgfpathlineto{\pgfqpoint{6.507993in}{2.619446in}}%
\pgfpathlineto{\pgfqpoint{6.508471in}{2.180591in}}%
\pgfpathlineto{\pgfqpoint{6.509487in}{2.601115in}}%
\pgfpathlineto{\pgfqpoint{6.508526in}{1.952107in}}%
\pgfpathlineto{\pgfqpoint{6.509582in}{2.480498in}}%
\pgfpathlineto{\pgfqpoint{6.510014in}{1.735086in}}%
\pgfpathlineto{\pgfqpoint{6.509598in}{2.622419in}}%
\pgfpathlineto{\pgfqpoint{6.510694in}{2.322738in}}%
\pgfpathlineto{\pgfqpoint{6.511361in}{2.602299in}}%
\pgfpathlineto{\pgfqpoint{6.511751in}{1.813432in}}%
\pgfpathlineto{\pgfqpoint{6.511805in}{2.523798in}}%
\pgfpathlineto{\pgfqpoint{6.511901in}{1.973278in}}%
\pgfpathlineto{\pgfqpoint{6.512072in}{2.602652in}}%
\pgfpathlineto{\pgfqpoint{6.512916in}{2.407752in}}%
\pgfpathlineto{\pgfqpoint{6.513028in}{2.629437in}}%
\pgfpathlineto{\pgfqpoint{6.513125in}{1.828647in}}%
\pgfpathlineto{\pgfqpoint{6.514026in}{2.227986in}}%
\pgfpathlineto{\pgfqpoint{6.515027in}{2.608614in}}%
\pgfpathlineto{\pgfqpoint{6.514692in}{1.935446in}}%
\pgfpathlineto{\pgfqpoint{6.515137in}{2.499022in}}%
\pgfpathlineto{\pgfqpoint{6.515661in}{1.977546in}}%
\pgfpathlineto{\pgfqpoint{6.515700in}{2.601184in}}%
\pgfpathlineto{\pgfqpoint{6.516248in}{2.543010in}}%
\pgfpathlineto{\pgfqpoint{6.517208in}{1.830390in}}%
\pgfpathlineto{\pgfqpoint{6.517048in}{2.611148in}}%
\pgfpathlineto{\pgfqpoint{6.517361in}{2.307528in}}%
\pgfpathlineto{\pgfqpoint{6.517886in}{2.600362in}}%
\pgfpathlineto{\pgfqpoint{6.517875in}{1.823857in}}%
\pgfpathlineto{\pgfqpoint{6.518472in}{2.529676in}}%
\pgfpathlineto{\pgfqpoint{6.518690in}{1.901501in}}%
\pgfpathlineto{\pgfqpoint{6.518739in}{2.613155in}}%
\pgfpathlineto{\pgfqpoint{6.519583in}{2.252385in}}%
\pgfpathlineto{\pgfqpoint{6.520495in}{2.614850in}}%
\pgfpathlineto{\pgfqpoint{6.520378in}{1.856030in}}%
\pgfpathlineto{\pgfqpoint{6.520695in}{2.491152in}}%
\pgfpathlineto{\pgfqpoint{6.520880in}{2.612689in}}%
\pgfpathlineto{\pgfqpoint{6.521709in}{1.972822in}}%
\pgfpathlineto{\pgfqpoint{6.521805in}{2.431398in}}%
\pgfpathlineto{\pgfqpoint{6.521813in}{1.680510in}}%
\pgfpathlineto{\pgfqpoint{6.521908in}{2.614053in}}%
\pgfpathlineto{\pgfqpoint{6.522915in}{2.266353in}}%
\pgfpathlineto{\pgfqpoint{6.523397in}{2.636098in}}%
\pgfpathlineto{\pgfqpoint{6.523937in}{1.923400in}}%
\pgfpathlineto{\pgfqpoint{6.524026in}{2.407458in}}%
\pgfpathlineto{\pgfqpoint{6.524980in}{1.959768in}}%
\pgfpathlineto{\pgfqpoint{6.524506in}{2.611179in}}%
\pgfpathlineto{\pgfqpoint{6.525137in}{2.554289in}}%
\pgfpathlineto{\pgfqpoint{6.525378in}{1.882420in}}%
\pgfpathlineto{\pgfqpoint{6.525692in}{2.620039in}}%
\pgfpathlineto{\pgfqpoint{6.526249in}{2.294901in}}%
\pgfpathlineto{\pgfqpoint{6.526467in}{2.619364in}}%
\pgfpathlineto{\pgfqpoint{6.526799in}{1.825927in}}%
\pgfpathlineto{\pgfqpoint{6.527360in}{2.578312in}}%
\pgfpathlineto{\pgfqpoint{6.527437in}{1.962393in}}%
\pgfpathlineto{\pgfqpoint{6.527970in}{2.634597in}}%
\pgfpathlineto{\pgfqpoint{6.528472in}{2.397280in}}%
\pgfpathlineto{\pgfqpoint{6.529579in}{2.615459in}}%
\pgfpathlineto{\pgfqpoint{6.528651in}{2.007897in}}%
\pgfpathlineto{\pgfqpoint{6.529582in}{2.459702in}}%
\pgfpathlineto{\pgfqpoint{6.529664in}{1.900494in}}%
\pgfpathlineto{\pgfqpoint{6.529738in}{2.615675in}}%
\pgfpathlineto{\pgfqpoint{6.530693in}{2.297372in}}%
\pgfpathlineto{\pgfqpoint{6.531030in}{2.625546in}}%
\pgfpathlineto{\pgfqpoint{6.531179in}{1.711119in}}%
\pgfpathlineto{\pgfqpoint{6.531804in}{2.490285in}}%
\pgfpathlineto{\pgfqpoint{6.531991in}{1.727304in}}%
\pgfpathlineto{\pgfqpoint{6.532161in}{2.611562in}}%
\pgfpathlineto{\pgfqpoint{6.532914in}{2.387608in}}%
\pgfpathlineto{\pgfqpoint{6.533755in}{2.629421in}}%
\pgfpathlineto{\pgfqpoint{6.533450in}{2.031713in}}%
\pgfpathlineto{\pgfqpoint{6.534025in}{2.527409in}}%
\pgfpathlineto{\pgfqpoint{6.534388in}{1.922848in}}%
\pgfpathlineto{\pgfqpoint{6.534230in}{2.619332in}}%
\pgfpathlineto{\pgfqpoint{6.535136in}{2.442923in}}%
\pgfpathlineto{\pgfqpoint{6.535269in}{2.608950in}}%
\pgfpathlineto{\pgfqpoint{6.535355in}{1.864384in}}%
\pgfpathlineto{\pgfqpoint{6.536248in}{2.569670in}}%
\pgfpathlineto{\pgfqpoint{6.536348in}{1.954576in}}%
\pgfpathlineto{\pgfqpoint{6.536843in}{2.625740in}}%
\pgfpathlineto{\pgfqpoint{6.537359in}{2.445279in}}%
\pgfpathlineto{\pgfqpoint{6.537465in}{2.624595in}}%
\pgfpathlineto{\pgfqpoint{6.538346in}{1.799798in}}%
\pgfpathlineto{\pgfqpoint{6.538467in}{2.333520in}}%
\pgfpathlineto{\pgfqpoint{6.539206in}{1.810451in}}%
\pgfpathlineto{\pgfqpoint{6.539037in}{2.600109in}}%
\pgfpathlineto{\pgfqpoint{6.539577in}{2.027797in}}%
\pgfpathlineto{\pgfqpoint{6.540482in}{2.618957in}}%
\pgfpathlineto{\pgfqpoint{6.540096in}{1.949743in}}%
\pgfpathlineto{\pgfqpoint{6.540688in}{2.362468in}}%
\pgfpathlineto{\pgfqpoint{6.541045in}{2.623266in}}%
\pgfpathlineto{\pgfqpoint{6.541026in}{1.926666in}}%
\pgfpathlineto{\pgfqpoint{6.541800in}{2.498594in}}%
\pgfpathlineto{\pgfqpoint{6.541950in}{1.885899in}}%
\pgfpathlineto{\pgfqpoint{6.542621in}{2.611277in}}%
\pgfpathlineto{\pgfqpoint{6.542911in}{2.335625in}}%
\pgfpathlineto{\pgfqpoint{6.543700in}{2.607087in}}%
\pgfpathlineto{\pgfqpoint{6.543985in}{1.892422in}}%
\pgfpathlineto{\pgfqpoint{6.544022in}{2.358071in}}%
\pgfpathlineto{\pgfqpoint{6.544709in}{1.590033in}}%
\pgfpathlineto{\pgfqpoint{6.544249in}{2.617733in}}%
\pgfpathlineto{\pgfqpoint{6.545131in}{2.389785in}}%
\pgfpathlineto{\pgfqpoint{6.545244in}{2.622441in}}%
\pgfpathlineto{\pgfqpoint{6.545443in}{1.862928in}}%
\pgfpathlineto{\pgfqpoint{6.546242in}{2.453150in}}%
\pgfpathlineto{\pgfqpoint{6.546737in}{1.663572in}}%
\pgfpathlineto{\pgfqpoint{6.547032in}{2.619714in}}%
\pgfpathlineto{\pgfqpoint{6.547353in}{2.452380in}}%
\pgfpathlineto{\pgfqpoint{6.548285in}{2.621935in}}%
\pgfpathlineto{\pgfqpoint{6.547837in}{1.908053in}}%
\pgfpathlineto{\pgfqpoint{6.548462in}{2.420260in}}%
\pgfpathlineto{\pgfqpoint{6.549260in}{1.994258in}}%
\pgfpathlineto{\pgfqpoint{6.548957in}{2.621310in}}%
\pgfpathlineto{\pgfqpoint{6.549573in}{2.345484in}}%
\pgfpathlineto{\pgfqpoint{6.550632in}{2.637536in}}%
\pgfpathlineto{\pgfqpoint{6.550641in}{1.971071in}}%
\pgfpathlineto{\pgfqpoint{6.550684in}{2.493448in}}%
\pgfpathlineto{\pgfqpoint{6.551665in}{1.898178in}}%
\pgfpathlineto{\pgfqpoint{6.551769in}{2.613614in}}%
\pgfpathlineto{\pgfqpoint{6.551794in}{2.300113in}}%
\pgfpathlineto{\pgfqpoint{6.552574in}{2.619084in}}%
\pgfpathlineto{\pgfqpoint{6.551993in}{1.887413in}}%
\pgfpathlineto{\pgfqpoint{6.552906in}{2.580197in}}%
\pgfpathlineto{\pgfqpoint{6.553191in}{1.846496in}}%
\pgfpathlineto{\pgfqpoint{6.553839in}{2.617048in}}%
\pgfpathlineto{\pgfqpoint{6.554017in}{2.218234in}}%
\pgfpathlineto{\pgfqpoint{6.554098in}{2.625701in}}%
\pgfpathlineto{\pgfqpoint{6.554466in}{1.911673in}}%
\pgfpathlineto{\pgfqpoint{6.555128in}{2.379692in}}%
\pgfpathlineto{\pgfqpoint{6.555546in}{1.860727in}}%
\pgfpathlineto{\pgfqpoint{6.555211in}{2.616818in}}%
\pgfpathlineto{\pgfqpoint{6.556215in}{2.450783in}}%
\pgfpathlineto{\pgfqpoint{6.557091in}{2.635291in}}%
\pgfpathlineto{\pgfqpoint{6.556486in}{1.744792in}}%
\pgfpathlineto{\pgfqpoint{6.557326in}{2.453801in}}%
\pgfpathlineto{\pgfqpoint{6.557962in}{2.624251in}}%
\pgfpathlineto{\pgfqpoint{6.558420in}{1.893382in}}%
\pgfpathlineto{\pgfqpoint{6.558434in}{2.483893in}}%
\pgfpathlineto{\pgfqpoint{6.559257in}{1.964233in}}%
\pgfpathlineto{\pgfqpoint{6.558548in}{2.621424in}}%
\pgfpathlineto{\pgfqpoint{6.559545in}{2.317123in}}%
\pgfpathlineto{\pgfqpoint{6.560622in}{2.638003in}}%
\pgfpathlineto{\pgfqpoint{6.560007in}{1.970241in}}%
\pgfpathlineto{\pgfqpoint{6.560656in}{2.469783in}}%
\pgfpathlineto{\pgfqpoint{6.561684in}{1.736744in}}%
\pgfpathlineto{\pgfqpoint{6.561159in}{2.626658in}}%
\pgfpathlineto{\pgfqpoint{6.561766in}{2.326955in}}%
\pgfpathlineto{\pgfqpoint{6.562536in}{2.623999in}}%
\pgfpathlineto{\pgfqpoint{6.562857in}{1.828988in}}%
\pgfpathlineto{\pgfqpoint{6.562878in}{2.470186in}}%
\pgfpathlineto{\pgfqpoint{6.562955in}{2.600754in}}%
\pgfpathlineto{\pgfqpoint{6.562929in}{2.146152in}}%
\pgfpathlineto{\pgfqpoint{6.562965in}{2.347072in}}%
\pgfusepath{stroke}%
\end{pgfscope}%
\begin{pgfscope}%
\pgfpathrectangle{\pgfqpoint{0.535225in}{0.370679in}}{\pgfqpoint{6.314775in}{3.181174in}}%
\pgfusepath{clip}%
\pgfsetrectcap%
\pgfsetroundjoin%
\pgfsetlinewidth{3.011250pt}%
\definecolor{currentstroke}{rgb}{0.933333,0.509804,0.933333}%
\pgfsetstrokecolor{currentstroke}%
\pgfsetdash{}{0pt}%
\pgfpathmoveto{\pgfqpoint{0.860821in}{0.360679in}}%
\pgfpathlineto{\pgfqpoint{1.116406in}{0.552057in}}%
\pgfpathlineto{\pgfqpoint{1.288470in}{0.553131in}}%
\pgfpathlineto{\pgfqpoint{1.410551in}{0.743708in}}%
\pgfpathlineto{\pgfqpoint{1.505245in}{0.710135in}}%
\pgfpathlineto{\pgfqpoint{1.582616in}{0.834992in}}%
\pgfpathlineto{\pgfqpoint{1.648031in}{0.659069in}}%
\pgfpathlineto{\pgfqpoint{1.704697in}{0.679014in}}%
\pgfpathlineto{\pgfqpoint{1.754680in}{0.784133in}}%
\pgfpathlineto{\pgfqpoint{1.799391in}{0.555350in}}%
\pgfpathlineto{\pgfqpoint{1.839837in}{0.761731in}}%
\pgfpathlineto{\pgfqpoint{1.876761in}{0.653176in}}%
\pgfpathlineto{\pgfqpoint{1.910729in}{1.005114in}}%
\pgfpathlineto{\pgfqpoint{1.942177in}{0.969955in}}%
\pgfpathlineto{\pgfqpoint{1.971455in}{0.921099in}}%
\pgfpathlineto{\pgfqpoint{1.998843in}{0.940818in}}%
\pgfpathlineto{\pgfqpoint{2.024570in}{0.851469in}}%
\pgfpathlineto{\pgfqpoint{2.048826in}{0.882759in}}%
\pgfpathlineto{\pgfqpoint{2.071770in}{0.841037in}}%
\pgfpathlineto{\pgfqpoint{2.093537in}{0.913053in}}%
\pgfpathlineto{\pgfqpoint{2.114241in}{0.902140in}}%
\pgfpathlineto{\pgfqpoint{2.133983in}{0.969551in}}%
\pgfpathlineto{\pgfqpoint{2.152847in}{0.743112in}}%
\pgfpathlineto{\pgfqpoint{2.170907in}{0.929972in}}%
\pgfpathlineto{\pgfqpoint{2.188231in}{1.052173in}}%
\pgfpathlineto{\pgfqpoint{2.204874in}{0.998307in}}%
\pgfpathlineto{\pgfqpoint{2.220890in}{1.000601in}}%
\pgfpathlineto{\pgfqpoint{2.236323in}{1.116988in}}%
\pgfpathlineto{\pgfqpoint{2.251214in}{1.005328in}}%
\pgfpathlineto{\pgfqpoint{2.265601in}{0.978684in}}%
\pgfpathlineto{\pgfqpoint{2.279516in}{1.103203in}}%
\pgfpathlineto{\pgfqpoint{2.292989in}{1.087612in}}%
\pgfpathlineto{\pgfqpoint{2.306047in}{0.858500in}}%
\pgfpathlineto{\pgfqpoint{2.318716in}{1.023665in}}%
\pgfpathlineto{\pgfqpoint{2.331017in}{0.936947in}}%
\pgfpathlineto{\pgfqpoint{2.342971in}{1.059279in}}%
\pgfpathlineto{\pgfqpoint{2.354599in}{1.080641in}}%
\pgfpathlineto{\pgfqpoint{2.365916in}{1.033702in}}%
\pgfpathlineto{\pgfqpoint{2.376939in}{0.933886in}}%
\pgfpathlineto{\pgfqpoint{2.387683in}{1.043890in}}%
\pgfpathlineto{\pgfqpoint{2.398161in}{0.990387in}}%
\pgfpathlineto{\pgfqpoint{2.408387in}{1.095304in}}%
\pgfpathlineto{\pgfqpoint{2.418373in}{0.853598in}}%
\pgfpathlineto{\pgfqpoint{2.428129in}{1.182022in}}%
\pgfpathlineto{\pgfqpoint{2.437665in}{1.080246in}}%
\pgfpathlineto{\pgfqpoint{2.446992in}{1.120297in}}%
\pgfpathlineto{\pgfqpoint{2.456119in}{1.037896in}}%
\pgfpathlineto{\pgfqpoint{2.465053in}{1.190356in}}%
\pgfpathlineto{\pgfqpoint{2.473803in}{1.018075in}}%
\pgfpathlineto{\pgfqpoint{2.482376in}{1.085801in}}%
\pgfpathlineto{\pgfqpoint{2.490780in}{0.997790in}}%
\pgfpathlineto{\pgfqpoint{2.507103in}{0.889888in}}%
\pgfpathlineto{\pgfqpoint{2.515036in}{1.042804in}}%
\pgfpathlineto{\pgfqpoint{2.522822in}{1.040019in}}%
\pgfpathlineto{\pgfqpoint{2.530469in}{0.966502in}}%
\pgfpathlineto{\pgfqpoint{2.537980in}{1.083657in}}%
\pgfpathlineto{\pgfqpoint{2.545360in}{1.140655in}}%
\pgfpathlineto{\pgfqpoint{2.552614in}{1.099860in}}%
\pgfpathlineto{\pgfqpoint{2.559747in}{1.106051in}}%
\pgfpathlineto{\pgfqpoint{2.566761in}{1.140686in}}%
\pgfpathlineto{\pgfqpoint{2.573662in}{1.009025in}}%
\pgfpathlineto{\pgfqpoint{2.580451in}{1.112230in}}%
\pgfpathlineto{\pgfqpoint{2.587135in}{1.023664in}}%
\pgfpathlineto{\pgfqpoint{2.593714in}{1.190227in}}%
\pgfpathlineto{\pgfqpoint{2.600193in}{1.134074in}}%
\pgfpathlineto{\pgfqpoint{2.606574in}{1.257890in}}%
\pgfpathlineto{\pgfqpoint{2.612861in}{1.000074in}}%
\pgfpathlineto{\pgfqpoint{2.619057in}{0.980810in}}%
\pgfpathlineto{\pgfqpoint{2.625163in}{1.099822in}}%
\pgfpathlineto{\pgfqpoint{2.631182in}{1.239894in}}%
\pgfpathlineto{\pgfqpoint{2.637117in}{1.055761in}}%
\pgfpathlineto{\pgfqpoint{2.642971in}{0.963709in}}%
\pgfpathlineto{\pgfqpoint{2.648744in}{0.898322in}}%
\pgfpathlineto{\pgfqpoint{2.654441in}{1.089298in}}%
\pgfpathlineto{\pgfqpoint{2.660061in}{1.156746in}}%
\pgfpathlineto{\pgfqpoint{2.665609in}{1.118288in}}%
\pgfpathlineto{\pgfqpoint{2.671084in}{1.210429in}}%
\pgfpathlineto{\pgfqpoint{2.676490in}{1.160276in}}%
\pgfpathlineto{\pgfqpoint{2.681828in}{1.250807in}}%
\pgfpathlineto{\pgfqpoint{2.687100in}{1.160262in}}%
\pgfpathlineto{\pgfqpoint{2.692307in}{1.265356in}}%
\pgfpathlineto{\pgfqpoint{2.697451in}{1.199770in}}%
\pgfpathlineto{\pgfqpoint{2.702533in}{1.114809in}}%
\pgfpathlineto{\pgfqpoint{2.707555in}{0.908517in}}%
\pgfpathlineto{\pgfqpoint{2.712518in}{1.206126in}}%
\pgfpathlineto{\pgfqpoint{2.717424in}{1.037891in}}%
\pgfpathlineto{\pgfqpoint{2.722274in}{0.947189in}}%
\pgfpathlineto{\pgfqpoint{2.727069in}{1.157428in}}%
\pgfpathlineto{\pgfqpoint{2.731811in}{1.150714in}}%
\pgfpathlineto{\pgfqpoint{2.736500in}{1.262555in}}%
\pgfpathlineto{\pgfqpoint{2.741138in}{1.152389in}}%
\pgfpathlineto{\pgfqpoint{2.745726in}{1.242900in}}%
\pgfpathlineto{\pgfqpoint{2.750264in}{1.187921in}}%
\pgfpathlineto{\pgfqpoint{2.754755in}{1.203440in}}%
\pgfpathlineto{\pgfqpoint{2.759199in}{1.026983in}}%
\pgfpathlineto{\pgfqpoint{2.763596in}{1.224744in}}%
\pgfpathlineto{\pgfqpoint{2.767949in}{1.307713in}}%
\pgfpathlineto{\pgfqpoint{2.772257in}{1.199623in}}%
\pgfpathlineto{\pgfqpoint{2.776522in}{0.974056in}}%
\pgfpathlineto{\pgfqpoint{2.780745in}{1.155876in}}%
\pgfpathlineto{\pgfqpoint{2.784926in}{1.248264in}}%
\pgfpathlineto{\pgfqpoint{2.789066in}{1.068849in}}%
\pgfpathlineto{\pgfqpoint{2.793166in}{1.202789in}}%
\pgfpathlineto{\pgfqpoint{2.797227in}{1.192357in}}%
\pgfpathlineto{\pgfqpoint{2.801249in}{1.187862in}}%
\pgfpathlineto{\pgfqpoint{2.805234in}{1.201548in}}%
\pgfpathlineto{\pgfqpoint{2.809181in}{1.224540in}}%
\pgfpathlineto{\pgfqpoint{2.813093in}{1.017725in}}%
\pgfpathlineto{\pgfqpoint{2.816968in}{1.273787in}}%
\pgfpathlineto{\pgfqpoint{2.820809in}{1.020478in}}%
\pgfpathlineto{\pgfqpoint{2.824615in}{1.276948in}}%
\pgfpathlineto{\pgfqpoint{2.828387in}{1.095868in}}%
\pgfpathlineto{\pgfqpoint{2.832126in}{1.124120in}}%
\pgfpathlineto{\pgfqpoint{2.835832in}{1.021832in}}%
\pgfpathlineto{\pgfqpoint{2.839506in}{1.206725in}}%
\pgfpathlineto{\pgfqpoint{2.843149in}{1.202155in}}%
\pgfpathlineto{\pgfqpoint{2.846760in}{1.269835in}}%
\pgfpathlineto{\pgfqpoint{2.850341in}{0.971166in}}%
\pgfpathlineto{\pgfqpoint{2.853893in}{1.207512in}}%
\pgfpathlineto{\pgfqpoint{2.857414in}{1.259233in}}%
\pgfpathlineto{\pgfqpoint{2.860907in}{1.233387in}}%
\pgfpathlineto{\pgfqpoint{2.864371in}{1.280761in}}%
\pgfpathlineto{\pgfqpoint{2.867807in}{1.196820in}}%
\pgfpathlineto{\pgfqpoint{2.871216in}{1.165595in}}%
\pgfpathlineto{\pgfqpoint{2.874597in}{1.225499in}}%
\pgfpathlineto{\pgfqpoint{2.877952in}{1.258745in}}%
\pgfpathlineto{\pgfqpoint{2.881280in}{1.258321in}}%
\pgfpathlineto{\pgfqpoint{2.884583in}{1.249366in}}%
\pgfpathlineto{\pgfqpoint{2.887860in}{1.227321in}}%
\pgfpathlineto{\pgfqpoint{2.891111in}{0.993627in}}%
\pgfpathlineto{\pgfqpoint{2.894339in}{1.262336in}}%
\pgfpathlineto{\pgfqpoint{2.897541in}{1.271642in}}%
\pgfpathlineto{\pgfqpoint{2.900720in}{1.092553in}}%
\pgfpathlineto{\pgfqpoint{2.903875in}{1.131344in}}%
\pgfpathlineto{\pgfqpoint{2.907007in}{1.325016in}}%
\pgfpathlineto{\pgfqpoint{2.913202in}{1.009825in}}%
\pgfpathlineto{\pgfqpoint{2.916266in}{1.293605in}}%
\pgfpathlineto{\pgfqpoint{2.919308in}{1.009131in}}%
\pgfpathlineto{\pgfqpoint{2.922329in}{1.191350in}}%
\pgfpathlineto{\pgfqpoint{2.925328in}{1.213593in}}%
\pgfpathlineto{\pgfqpoint{2.928306in}{1.366791in}}%
\pgfpathlineto{\pgfqpoint{2.931263in}{1.300680in}}%
\pgfpathlineto{\pgfqpoint{2.934200in}{1.130774in}}%
\pgfpathlineto{\pgfqpoint{2.937116in}{1.169956in}}%
\pgfpathlineto{\pgfqpoint{2.940013in}{1.249887in}}%
\pgfpathlineto{\pgfqpoint{2.942890in}{1.280837in}}%
\pgfpathlineto{\pgfqpoint{2.945748in}{1.228414in}}%
\pgfpathlineto{\pgfqpoint{2.948586in}{1.298769in}}%
\pgfpathlineto{\pgfqpoint{2.951406in}{1.280381in}}%
\pgfpathlineto{\pgfqpoint{2.954207in}{1.197491in}}%
\pgfpathlineto{\pgfqpoint{2.956990in}{1.222412in}}%
\pgfpathlineto{\pgfqpoint{2.959754in}{1.309761in}}%
\pgfpathlineto{\pgfqpoint{2.962501in}{1.332550in}}%
\pgfpathlineto{\pgfqpoint{2.965230in}{1.145477in}}%
\pgfpathlineto{\pgfqpoint{2.967942in}{1.157952in}}%
\pgfpathlineto{\pgfqpoint{2.973313in}{1.366870in}}%
\pgfpathlineto{\pgfqpoint{2.975974in}{1.280202in}}%
\pgfpathlineto{\pgfqpoint{2.978618in}{1.263924in}}%
\pgfpathlineto{\pgfqpoint{2.981246in}{1.165575in}}%
\pgfpathlineto{\pgfqpoint{2.983857in}{1.114103in}}%
\pgfpathlineto{\pgfqpoint{2.986453in}{0.930416in}}%
\pgfpathlineto{\pgfqpoint{2.989032in}{1.210682in}}%
\pgfpathlineto{\pgfqpoint{2.991597in}{1.173518in}}%
\pgfpathlineto{\pgfqpoint{2.994145in}{1.270725in}}%
\pgfpathlineto{\pgfqpoint{2.996679in}{1.307186in}}%
\pgfpathlineto{\pgfqpoint{2.999197in}{1.118547in}}%
\pgfpathlineto{\pgfqpoint{3.001701in}{1.182173in}}%
\pgfpathlineto{\pgfqpoint{3.004190in}{1.371475in}}%
\pgfpathlineto{\pgfqpoint{3.006664in}{1.252004in}}%
\pgfpathlineto{\pgfqpoint{3.009124in}{1.198444in}}%
\pgfpathlineto{\pgfqpoint{3.011570in}{1.172546in}}%
\pgfpathlineto{\pgfqpoint{3.014002in}{1.120731in}}%
\pgfpathlineto{\pgfqpoint{3.016420in}{1.236961in}}%
\pgfpathlineto{\pgfqpoint{3.021215in}{0.897439in}}%
\pgfpathlineto{\pgfqpoint{3.023593in}{1.224765in}}%
\pgfpathlineto{\pgfqpoint{3.025957in}{1.304097in}}%
\pgfpathlineto{\pgfqpoint{3.028308in}{1.250237in}}%
\pgfpathlineto{\pgfqpoint{3.030646in}{1.278537in}}%
\pgfpathlineto{\pgfqpoint{3.032971in}{1.265557in}}%
\pgfpathlineto{\pgfqpoint{3.037584in}{1.076451in}}%
\pgfpathlineto{\pgfqpoint{3.039872in}{1.233434in}}%
\pgfpathlineto{\pgfqpoint{3.044410in}{1.326035in}}%
\pgfpathlineto{\pgfqpoint{3.046661in}{1.144010in}}%
\pgfpathlineto{\pgfqpoint{3.048901in}{1.177073in}}%
\pgfpathlineto{\pgfqpoint{3.051128in}{1.177418in}}%
\pgfpathlineto{\pgfqpoint{3.053344in}{1.234644in}}%
\pgfpathlineto{\pgfqpoint{3.055549in}{1.357159in}}%
\pgfpathlineto{\pgfqpoint{3.057742in}{1.282276in}}%
\pgfpathlineto{\pgfqpoint{3.059924in}{1.320535in}}%
\pgfpathlineto{\pgfqpoint{3.062095in}{1.285514in}}%
\pgfpathlineto{\pgfqpoint{3.064254in}{1.278607in}}%
\pgfpathlineto{\pgfqpoint{3.066403in}{1.400993in}}%
\pgfpathlineto{\pgfqpoint{3.068541in}{1.067616in}}%
\pgfpathlineto{\pgfqpoint{3.070668in}{1.222306in}}%
\pgfpathlineto{\pgfqpoint{3.072784in}{0.913917in}}%
\pgfpathlineto{\pgfqpoint{3.074890in}{1.246710in}}%
\pgfpathlineto{\pgfqpoint{3.076986in}{1.327532in}}%
\pgfpathlineto{\pgfqpoint{3.079071in}{1.234637in}}%
\pgfpathlineto{\pgfqpoint{3.081146in}{1.209146in}}%
\pgfpathlineto{\pgfqpoint{3.083211in}{1.296919in}}%
\pgfpathlineto{\pgfqpoint{3.085267in}{1.267856in}}%
\pgfpathlineto{\pgfqpoint{3.087312in}{1.255847in}}%
\pgfpathlineto{\pgfqpoint{3.089347in}{1.200350in}}%
\pgfpathlineto{\pgfqpoint{3.091373in}{1.311183in}}%
\pgfpathlineto{\pgfqpoint{3.093389in}{1.310592in}}%
\pgfpathlineto{\pgfqpoint{3.095395in}{1.375697in}}%
\pgfpathlineto{\pgfqpoint{3.097392in}{1.230507in}}%
\pgfpathlineto{\pgfqpoint{3.099380in}{1.367467in}}%
\pgfpathlineto{\pgfqpoint{3.101358in}{1.105361in}}%
\pgfpathlineto{\pgfqpoint{3.103327in}{1.275301in}}%
\pgfpathlineto{\pgfqpoint{3.105287in}{1.169675in}}%
\pgfpathlineto{\pgfqpoint{3.107238in}{1.362157in}}%
\pgfpathlineto{\pgfqpoint{3.109181in}{1.267163in}}%
\pgfpathlineto{\pgfqpoint{3.111114in}{1.357893in}}%
\pgfpathlineto{\pgfqpoint{3.113038in}{1.279437in}}%
\pgfpathlineto{\pgfqpoint{3.114954in}{1.352736in}}%
\pgfpathlineto{\pgfqpoint{3.116862in}{1.337220in}}%
\pgfpathlineto{\pgfqpoint{3.118760in}{1.246077in}}%
\pgfpathlineto{\pgfqpoint{3.120651in}{1.259911in}}%
\pgfpathlineto{\pgfqpoint{3.122532in}{1.169470in}}%
\pgfpathlineto{\pgfqpoint{3.124406in}{1.319337in}}%
\pgfpathlineto{\pgfqpoint{3.126271in}{1.363931in}}%
\pgfpathlineto{\pgfqpoint{3.128128in}{1.299613in}}%
\pgfpathlineto{\pgfqpoint{3.131819in}{1.300682in}}%
\pgfpathlineto{\pgfqpoint{3.133652in}{1.408024in}}%
\pgfpathlineto{\pgfqpoint{3.135477in}{1.323473in}}%
\pgfpathlineto{\pgfqpoint{3.137294in}{1.070232in}}%
\pgfpathlineto{\pgfqpoint{3.140906in}{1.387689in}}%
\pgfpathlineto{\pgfqpoint{3.142700in}{1.235893in}}%
\pgfpathlineto{\pgfqpoint{3.144487in}{1.309367in}}%
\pgfpathlineto{\pgfqpoint{3.148038in}{1.150981in}}%
\pgfpathlineto{\pgfqpoint{3.149803in}{1.375290in}}%
\pgfpathlineto{\pgfqpoint{3.151560in}{1.247159in}}%
\pgfpathlineto{\pgfqpoint{3.153310in}{1.382021in}}%
\pgfpathlineto{\pgfqpoint{3.156788in}{1.257665in}}%
\pgfpathlineto{\pgfqpoint{3.158517in}{1.385379in}}%
\pgfpathlineto{\pgfqpoint{3.160238in}{1.338471in}}%
\pgfpathlineto{\pgfqpoint{3.161953in}{1.357211in}}%
\pgfpathlineto{\pgfqpoint{3.163661in}{1.293063in}}%
\pgfpathlineto{\pgfqpoint{3.165362in}{1.132094in}}%
\pgfpathlineto{\pgfqpoint{3.168743in}{1.337165in}}%
\pgfpathlineto{\pgfqpoint{3.170424in}{1.263206in}}%
\pgfpathlineto{\pgfqpoint{3.172098in}{1.266871in}}%
\pgfpathlineto{\pgfqpoint{3.173765in}{1.330769in}}%
\pgfpathlineto{\pgfqpoint{3.175426in}{1.348019in}}%
\pgfpathlineto{\pgfqpoint{3.177080in}{1.212907in}}%
\pgfpathlineto{\pgfqpoint{3.178728in}{1.441111in}}%
\pgfpathlineto{\pgfqpoint{3.180370in}{1.249343in}}%
\pgfpathlineto{\pgfqpoint{3.182005in}{1.220928in}}%
\pgfpathlineto{\pgfqpoint{3.185257in}{1.405554in}}%
\pgfpathlineto{\pgfqpoint{3.186874in}{1.330022in}}%
\pgfpathlineto{\pgfqpoint{3.188484in}{1.421658in}}%
\pgfpathlineto{\pgfqpoint{3.190089in}{1.412949in}}%
\pgfpathlineto{\pgfqpoint{3.191687in}{1.228808in}}%
\pgfpathlineto{\pgfqpoint{3.193279in}{1.435130in}}%
\pgfpathlineto{\pgfqpoint{3.194866in}{1.309035in}}%
\pgfpathlineto{\pgfqpoint{3.196446in}{1.333425in}}%
\pgfpathlineto{\pgfqpoint{3.198021in}{1.326258in}}%
\pgfpathlineto{\pgfqpoint{3.199590in}{1.285185in}}%
\pgfpathlineto{\pgfqpoint{3.201153in}{1.387415in}}%
\pgfpathlineto{\pgfqpoint{3.202710in}{1.274273in}}%
\pgfpathlineto{\pgfqpoint{3.204262in}{1.309416in}}%
\pgfpathlineto{\pgfqpoint{3.205808in}{1.429256in}}%
\pgfpathlineto{\pgfqpoint{3.207348in}{1.202846in}}%
\pgfpathlineto{\pgfqpoint{3.208883in}{1.281106in}}%
\pgfpathlineto{\pgfqpoint{3.210412in}{1.067739in}}%
\pgfpathlineto{\pgfqpoint{3.211936in}{1.293204in}}%
\pgfpathlineto{\pgfqpoint{3.213454in}{1.194782in}}%
\pgfpathlineto{\pgfqpoint{3.214967in}{1.249611in}}%
\pgfpathlineto{\pgfqpoint{3.216474in}{1.412437in}}%
\pgfpathlineto{\pgfqpoint{3.217977in}{1.311360in}}%
\pgfpathlineto{\pgfqpoint{3.220965in}{1.462021in}}%
\pgfpathlineto{\pgfqpoint{3.223933in}{1.250564in}}%
\pgfpathlineto{\pgfqpoint{3.226880in}{1.431434in}}%
\pgfpathlineto{\pgfqpoint{3.228346in}{1.402104in}}%
\pgfpathlineto{\pgfqpoint{3.229806in}{1.225400in}}%
\pgfpathlineto{\pgfqpoint{3.232713in}{1.453449in}}%
\pgfpathlineto{\pgfqpoint{3.234159in}{1.421278in}}%
\pgfpathlineto{\pgfqpoint{3.235600in}{1.416231in}}%
\pgfpathlineto{\pgfqpoint{3.237036in}{1.332617in}}%
\pgfpathlineto{\pgfqpoint{3.238467in}{1.375358in}}%
\pgfpathlineto{\pgfqpoint{3.239894in}{1.315929in}}%
\pgfpathlineto{\pgfqpoint{3.241315in}{1.418775in}}%
\pgfpathlineto{\pgfqpoint{3.242732in}{1.280221in}}%
\pgfpathlineto{\pgfqpoint{3.244144in}{1.329786in}}%
\pgfpathlineto{\pgfqpoint{3.245552in}{1.440210in}}%
\pgfpathlineto{\pgfqpoint{3.246955in}{1.400545in}}%
\pgfpathlineto{\pgfqpoint{3.249746in}{1.257034in}}%
\pgfpathlineto{\pgfqpoint{3.251136in}{1.312032in}}%
\pgfpathlineto{\pgfqpoint{3.252520in}{1.129194in}}%
\pgfpathlineto{\pgfqpoint{3.253900in}{1.385939in}}%
\pgfpathlineto{\pgfqpoint{3.255276in}{1.335917in}}%
\pgfpathlineto{\pgfqpoint{3.258014in}{1.159067in}}%
\pgfpathlineto{\pgfqpoint{3.259376in}{1.250102in}}%
\pgfpathlineto{\pgfqpoint{3.260734in}{1.182586in}}%
\pgfpathlineto{\pgfqpoint{3.262087in}{1.394042in}}%
\pgfpathlineto{\pgfqpoint{3.263437in}{1.387611in}}%
\pgfpathlineto{\pgfqpoint{3.264782in}{1.333710in}}%
\pgfpathlineto{\pgfqpoint{3.266123in}{1.348030in}}%
\pgfpathlineto{\pgfqpoint{3.267459in}{1.319938in}}%
\pgfpathlineto{\pgfqpoint{3.268792in}{1.321614in}}%
\pgfpathlineto{\pgfqpoint{3.270120in}{1.362570in}}%
\pgfpathlineto{\pgfqpoint{3.274080in}{1.263931in}}%
\pgfpathlineto{\pgfqpoint{3.276699in}{1.378797in}}%
\pgfpathlineto{\pgfqpoint{3.278003in}{1.406235in}}%
\pgfpathlineto{\pgfqpoint{3.279303in}{1.283735in}}%
\pgfpathlineto{\pgfqpoint{3.280598in}{1.341828in}}%
\pgfpathlineto{\pgfqpoint{3.281890in}{1.483484in}}%
\pgfpathlineto{\pgfqpoint{3.283178in}{1.283773in}}%
\pgfpathlineto{\pgfqpoint{3.284462in}{1.277896in}}%
\pgfpathlineto{\pgfqpoint{3.285742in}{1.313303in}}%
\pgfpathlineto{\pgfqpoint{3.287019in}{1.421963in}}%
\pgfpathlineto{\pgfqpoint{3.288291in}{1.426893in}}%
\pgfpathlineto{\pgfqpoint{3.289560in}{1.282207in}}%
\pgfpathlineto{\pgfqpoint{3.290825in}{1.303927in}}%
\pgfpathlineto{\pgfqpoint{3.292086in}{1.354190in}}%
\pgfpathlineto{\pgfqpoint{3.294597in}{1.294153in}}%
\pgfpathlineto{\pgfqpoint{3.295847in}{1.407447in}}%
\pgfpathlineto{\pgfqpoint{3.297093in}{1.418988in}}%
\pgfpathlineto{\pgfqpoint{3.298336in}{1.221036in}}%
\pgfpathlineto{\pgfqpoint{3.299575in}{1.369990in}}%
\pgfpathlineto{\pgfqpoint{3.300810in}{1.275170in}}%
\pgfpathlineto{\pgfqpoint{3.302042in}{1.457966in}}%
\pgfpathlineto{\pgfqpoint{3.303270in}{1.369125in}}%
\pgfpathlineto{\pgfqpoint{3.304495in}{1.448344in}}%
\pgfpathlineto{\pgfqpoint{3.305716in}{1.274601in}}%
\pgfpathlineto{\pgfqpoint{3.308148in}{1.330601in}}%
\pgfpathlineto{\pgfqpoint{3.309359in}{1.485463in}}%
\pgfpathlineto{\pgfqpoint{3.310566in}{1.489804in}}%
\pgfpathlineto{\pgfqpoint{3.311770in}{1.143422in}}%
\pgfpathlineto{\pgfqpoint{3.314167in}{1.483851in}}%
\pgfpathlineto{\pgfqpoint{3.315361in}{1.342624in}}%
\pgfpathlineto{\pgfqpoint{3.316551in}{1.457436in}}%
\pgfpathlineto{\pgfqpoint{3.317738in}{1.440031in}}%
\pgfpathlineto{\pgfqpoint{3.318922in}{1.142367in}}%
\pgfpathlineto{\pgfqpoint{3.321280in}{1.460680in}}%
\pgfpathlineto{\pgfqpoint{3.322454in}{1.349005in}}%
\pgfpathlineto{\pgfqpoint{3.324792in}{1.478754in}}%
\pgfpathlineto{\pgfqpoint{3.327117in}{1.439188in}}%
\pgfpathlineto{\pgfqpoint{3.328275in}{1.135422in}}%
\pgfpathlineto{\pgfqpoint{3.331730in}{1.441181in}}%
\pgfpathlineto{\pgfqpoint{3.335157in}{1.273218in}}%
\pgfpathlineto{\pgfqpoint{3.336293in}{1.441822in}}%
\pgfpathlineto{\pgfqpoint{3.338556in}{1.321892in}}%
\pgfpathlineto{\pgfqpoint{3.339683in}{0.986915in}}%
\pgfpathlineto{\pgfqpoint{3.340807in}{1.358839in}}%
\pgfpathlineto{\pgfqpoint{3.343047in}{1.284840in}}%
\pgfpathlineto{\pgfqpoint{3.344162in}{1.475427in}}%
\pgfpathlineto{\pgfqpoint{3.345274in}{1.263526in}}%
\pgfpathlineto{\pgfqpoint{3.346384in}{1.457593in}}%
\pgfpathlineto{\pgfqpoint{3.348594in}{1.415282in}}%
\pgfpathlineto{\pgfqpoint{3.349695in}{1.192834in}}%
\pgfpathlineto{\pgfqpoint{3.350793in}{1.442634in}}%
\pgfpathlineto{\pgfqpoint{3.351888in}{1.260467in}}%
\pgfpathlineto{\pgfqpoint{3.352980in}{1.512946in}}%
\pgfpathlineto{\pgfqpoint{3.355156in}{1.160455in}}%
\pgfpathlineto{\pgfqpoint{3.357321in}{1.387458in}}%
\pgfpathlineto{\pgfqpoint{3.358400in}{1.475587in}}%
\pgfpathlineto{\pgfqpoint{3.359476in}{1.267773in}}%
\pgfpathlineto{\pgfqpoint{3.360549in}{1.430371in}}%
\pgfpathlineto{\pgfqpoint{3.361619in}{1.069602in}}%
\pgfpathlineto{\pgfqpoint{3.363751in}{1.363953in}}%
\pgfpathlineto{\pgfqpoint{3.364814in}{1.353138in}}%
\pgfpathlineto{\pgfqpoint{3.365873in}{1.152764in}}%
\pgfpathlineto{\pgfqpoint{3.369036in}{1.412163in}}%
\pgfpathlineto{\pgfqpoint{3.370085in}{1.369653in}}%
\pgfpathlineto{\pgfqpoint{3.371132in}{1.376599in}}%
\pgfpathlineto{\pgfqpoint{3.373217in}{1.432223in}}%
\pgfpathlineto{\pgfqpoint{3.374256in}{1.456325in}}%
\pgfpathlineto{\pgfqpoint{3.375292in}{1.439827in}}%
\pgfpathlineto{\pgfqpoint{3.376326in}{1.314367in}}%
\pgfpathlineto{\pgfqpoint{3.377357in}{1.337232in}}%
\pgfpathlineto{\pgfqpoint{3.378386in}{1.465082in}}%
\pgfpathlineto{\pgfqpoint{3.379412in}{1.428717in}}%
\pgfpathlineto{\pgfqpoint{3.380436in}{1.154643in}}%
\pgfpathlineto{\pgfqpoint{3.381457in}{1.430710in}}%
\pgfpathlineto{\pgfqpoint{3.382476in}{1.408693in}}%
\pgfpathlineto{\pgfqpoint{3.383493in}{1.433652in}}%
\pgfpathlineto{\pgfqpoint{3.384507in}{1.431262in}}%
\pgfpathlineto{\pgfqpoint{3.385518in}{1.336405in}}%
\pgfpathlineto{\pgfqpoint{3.386527in}{1.391937in}}%
\pgfpathlineto{\pgfqpoint{3.387534in}{1.021228in}}%
\pgfpathlineto{\pgfqpoint{3.388539in}{1.381951in}}%
\pgfpathlineto{\pgfqpoint{3.389541in}{1.233382in}}%
\pgfpathlineto{\pgfqpoint{3.390540in}{1.402698in}}%
\pgfpathlineto{\pgfqpoint{3.391538in}{1.372432in}}%
\pgfpathlineto{\pgfqpoint{3.392533in}{1.300060in}}%
\pgfpathlineto{\pgfqpoint{3.394516in}{1.397467in}}%
\pgfpathlineto{\pgfqpoint{3.395504in}{1.442448in}}%
\pgfpathlineto{\pgfqpoint{3.397473in}{1.313817in}}%
\pgfpathlineto{\pgfqpoint{3.399433in}{1.456394in}}%
\pgfpathlineto{\pgfqpoint{3.400410in}{1.417792in}}%
\pgfpathlineto{\pgfqpoint{3.401384in}{1.456279in}}%
\pgfpathlineto{\pgfqpoint{3.402356in}{1.268966in}}%
\pgfpathlineto{\pgfqpoint{3.404294in}{1.498005in}}%
\pgfpathlineto{\pgfqpoint{3.405260in}{1.448087in}}%
\pgfpathlineto{\pgfqpoint{3.406223in}{1.483215in}}%
\pgfpathlineto{\pgfqpoint{3.407184in}{1.304718in}}%
\pgfpathlineto{\pgfqpoint{3.408143in}{1.317296in}}%
\pgfpathlineto{\pgfqpoint{3.409100in}{1.300017in}}%
\pgfpathlineto{\pgfqpoint{3.410055in}{1.419145in}}%
\pgfpathlineto{\pgfqpoint{3.411007in}{1.403955in}}%
\pgfpathlineto{\pgfqpoint{3.411958in}{1.308093in}}%
\pgfpathlineto{\pgfqpoint{3.412906in}{1.444699in}}%
\pgfpathlineto{\pgfqpoint{3.413852in}{1.165081in}}%
\pgfpathlineto{\pgfqpoint{3.414796in}{1.426063in}}%
\pgfpathlineto{\pgfqpoint{3.415738in}{1.362853in}}%
\pgfpathlineto{\pgfqpoint{3.416678in}{1.124224in}}%
\pgfpathlineto{\pgfqpoint{3.417616in}{1.519618in}}%
\pgfpathlineto{\pgfqpoint{3.418552in}{1.240865in}}%
\pgfpathlineto{\pgfqpoint{3.419485in}{1.288902in}}%
\pgfpathlineto{\pgfqpoint{3.420417in}{1.280799in}}%
\pgfpathlineto{\pgfqpoint{3.421347in}{1.294062in}}%
\pgfpathlineto{\pgfqpoint{3.422274in}{1.390816in}}%
\pgfpathlineto{\pgfqpoint{3.424123in}{1.314718in}}%
\pgfpathlineto{\pgfqpoint{3.425045in}{1.426343in}}%
\pgfpathlineto{\pgfqpoint{3.425964in}{1.359618in}}%
\pgfpathlineto{\pgfqpoint{3.426882in}{1.373857in}}%
\pgfpathlineto{\pgfqpoint{3.427797in}{1.175012in}}%
\pgfpathlineto{\pgfqpoint{3.428711in}{1.507803in}}%
\pgfpathlineto{\pgfqpoint{3.429623in}{1.451380in}}%
\pgfpathlineto{\pgfqpoint{3.431440in}{1.378177in}}%
\pgfpathlineto{\pgfqpoint{3.432346in}{1.471332in}}%
\pgfpathlineto{\pgfqpoint{3.433250in}{1.242167in}}%
\pgfpathlineto{\pgfqpoint{3.434152in}{1.483919in}}%
\pgfpathlineto{\pgfqpoint{3.435950in}{1.304948in}}%
\pgfpathlineto{\pgfqpoint{3.437740in}{1.406383in}}%
\pgfpathlineto{\pgfqpoint{3.438633in}{1.305777in}}%
\pgfpathlineto{\pgfqpoint{3.439523in}{1.422579in}}%
\pgfpathlineto{\pgfqpoint{3.440412in}{1.204297in}}%
\pgfpathlineto{\pgfqpoint{3.441299in}{1.217236in}}%
\pgfpathlineto{\pgfqpoint{3.442184in}{1.272066in}}%
\pgfpathlineto{\pgfqpoint{3.443949in}{1.462937in}}%
\pgfpathlineto{\pgfqpoint{3.444828in}{1.445409in}}%
\pgfpathlineto{\pgfqpoint{3.445706in}{1.516657in}}%
\pgfpathlineto{\pgfqpoint{3.447456in}{1.395061in}}%
\pgfpathlineto{\pgfqpoint{3.448328in}{1.518387in}}%
\pgfpathlineto{\pgfqpoint{3.449198in}{1.358172in}}%
\pgfpathlineto{\pgfqpoint{3.450067in}{1.419887in}}%
\pgfpathlineto{\pgfqpoint{3.450934in}{1.185925in}}%
\pgfpathlineto{\pgfqpoint{3.452663in}{1.425970in}}%
\pgfpathlineto{\pgfqpoint{3.453524in}{1.395753in}}%
\pgfpathlineto{\pgfqpoint{3.454384in}{1.271222in}}%
\pgfpathlineto{\pgfqpoint{3.455242in}{1.371896in}}%
\pgfpathlineto{\pgfqpoint{3.456099in}{1.364547in}}%
\pgfpathlineto{\pgfqpoint{3.456954in}{1.328275in}}%
\pgfpathlineto{\pgfqpoint{3.457807in}{1.418042in}}%
\pgfpathlineto{\pgfqpoint{3.458658in}{1.321757in}}%
\pgfpathlineto{\pgfqpoint{3.460355in}{1.505881in}}%
\pgfpathlineto{\pgfqpoint{3.461201in}{1.427334in}}%
\pgfpathlineto{\pgfqpoint{3.462046in}{1.479904in}}%
\pgfpathlineto{\pgfqpoint{3.463730in}{1.331097in}}%
\pgfpathlineto{\pgfqpoint{3.466243in}{1.285040in}}%
\pgfpathlineto{\pgfqpoint{3.467078in}{1.400620in}}%
\pgfpathlineto{\pgfqpoint{3.467911in}{1.397802in}}%
\pgfpathlineto{\pgfqpoint{3.468742in}{1.468028in}}%
\pgfpathlineto{\pgfqpoint{3.469572in}{1.397096in}}%
\pgfpathlineto{\pgfqpoint{3.470400in}{1.496584in}}%
\pgfpathlineto{\pgfqpoint{3.471226in}{1.411603in}}%
\pgfpathlineto{\pgfqpoint{3.472051in}{1.454215in}}%
\pgfpathlineto{\pgfqpoint{3.472874in}{1.255709in}}%
\pgfpathlineto{\pgfqpoint{3.473696in}{1.308647in}}%
\pgfpathlineto{\pgfqpoint{3.475334in}{1.531295in}}%
\pgfpathlineto{\pgfqpoint{3.476151in}{1.330933in}}%
\pgfpathlineto{\pgfqpoint{3.476966in}{1.383091in}}%
\pgfpathlineto{\pgfqpoint{3.478592in}{1.428873in}}%
\pgfpathlineto{\pgfqpoint{3.480212in}{1.273738in}}%
\pgfpathlineto{\pgfqpoint{3.481020in}{1.399046in}}%
\pgfpathlineto{\pgfqpoint{3.481826in}{1.332136in}}%
\pgfpathlineto{\pgfqpoint{3.483433in}{1.425681in}}%
\pgfpathlineto{\pgfqpoint{3.484235in}{1.398086in}}%
\pgfpathlineto{\pgfqpoint{3.485034in}{1.519000in}}%
\pgfpathlineto{\pgfqpoint{3.486630in}{1.363182in}}%
\pgfpathlineto{\pgfqpoint{3.487425in}{1.281022in}}%
\pgfpathlineto{\pgfqpoint{3.489012in}{1.514128in}}%
\pgfpathlineto{\pgfqpoint{3.489803in}{1.472890in}}%
\pgfpathlineto{\pgfqpoint{3.490592in}{1.040976in}}%
\pgfpathlineto{\pgfqpoint{3.492167in}{1.530795in}}%
\pgfpathlineto{\pgfqpoint{3.492952in}{1.292737in}}%
\pgfpathlineto{\pgfqpoint{3.493736in}{1.475997in}}%
\pgfpathlineto{\pgfqpoint{3.494518in}{1.419915in}}%
\pgfpathlineto{\pgfqpoint{3.495299in}{1.455422in}}%
\pgfpathlineto{\pgfqpoint{3.496078in}{1.387331in}}%
\pgfpathlineto{\pgfqpoint{3.496856in}{1.409836in}}%
\pgfpathlineto{\pgfqpoint{3.497632in}{1.393990in}}%
\pgfpathlineto{\pgfqpoint{3.498407in}{1.436627in}}%
\pgfpathlineto{\pgfqpoint{3.499181in}{1.316292in}}%
\pgfpathlineto{\pgfqpoint{3.499953in}{1.550570in}}%
\pgfpathlineto{\pgfqpoint{3.501494in}{1.399973in}}%
\pgfpathlineto{\pgfqpoint{3.502262in}{1.445249in}}%
\pgfpathlineto{\pgfqpoint{3.503029in}{1.332944in}}%
\pgfpathlineto{\pgfqpoint{3.503794in}{1.482791in}}%
\pgfpathlineto{\pgfqpoint{3.504558in}{1.337171in}}%
\pgfpathlineto{\pgfqpoint{3.505320in}{1.491601in}}%
\pgfpathlineto{\pgfqpoint{3.507600in}{1.257094in}}%
\pgfpathlineto{\pgfqpoint{3.509867in}{1.465469in}}%
\pgfpathlineto{\pgfqpoint{3.512871in}{1.391954in}}%
\pgfpathlineto{\pgfqpoint{3.513619in}{1.462695in}}%
\pgfpathlineto{\pgfqpoint{3.514366in}{1.389062in}}%
\pgfpathlineto{\pgfqpoint{3.515855in}{1.496523in}}%
\pgfpathlineto{\pgfqpoint{3.516597in}{1.471036in}}%
\pgfpathlineto{\pgfqpoint{3.517338in}{1.231937in}}%
\pgfpathlineto{\pgfqpoint{3.518078in}{1.342312in}}%
\pgfpathlineto{\pgfqpoint{3.518817in}{1.482091in}}%
\pgfpathlineto{\pgfqpoint{3.519554in}{1.472248in}}%
\pgfpathlineto{\pgfqpoint{3.520291in}{1.405029in}}%
\pgfpathlineto{\pgfqpoint{3.521025in}{1.489850in}}%
\pgfpathlineto{\pgfqpoint{3.523222in}{1.310576in}}%
\pgfpathlineto{\pgfqpoint{3.523952in}{1.389767in}}%
\pgfpathlineto{\pgfqpoint{3.524681in}{1.383132in}}%
\pgfpathlineto{\pgfqpoint{3.525408in}{1.160522in}}%
\pgfpathlineto{\pgfqpoint{3.526134in}{1.257329in}}%
\pgfpathlineto{\pgfqpoint{3.526859in}{1.485026in}}%
\pgfpathlineto{\pgfqpoint{3.527582in}{1.219561in}}%
\pgfpathlineto{\pgfqpoint{3.528305in}{1.347609in}}%
\pgfpathlineto{\pgfqpoint{3.529026in}{1.491200in}}%
\pgfpathlineto{\pgfqpoint{3.529746in}{1.422373in}}%
\pgfpathlineto{\pgfqpoint{3.530464in}{1.223356in}}%
\pgfpathlineto{\pgfqpoint{3.531182in}{1.301368in}}%
\pgfpathlineto{\pgfqpoint{3.532613in}{1.469624in}}%
\pgfpathlineto{\pgfqpoint{3.533327in}{1.465740in}}%
\pgfpathlineto{\pgfqpoint{3.534039in}{1.256256in}}%
\pgfpathlineto{\pgfqpoint{3.534751in}{1.489846in}}%
\pgfpathlineto{\pgfqpoint{3.535461in}{1.276537in}}%
\pgfpathlineto{\pgfqpoint{3.536170in}{1.295964in}}%
\pgfpathlineto{\pgfqpoint{3.538290in}{1.486122in}}%
\pgfpathlineto{\pgfqpoint{3.538994in}{1.529816in}}%
\pgfpathlineto{\pgfqpoint{3.541100in}{1.405145in}}%
\pgfpathlineto{\pgfqpoint{3.541800in}{1.482159in}}%
\pgfpathlineto{\pgfqpoint{3.542499in}{1.347031in}}%
\pgfpathlineto{\pgfqpoint{3.543196in}{1.472568in}}%
\pgfpathlineto{\pgfqpoint{3.543892in}{1.431751in}}%
\pgfpathlineto{\pgfqpoint{3.545281in}{1.465328in}}%
\pgfpathlineto{\pgfqpoint{3.545974in}{1.515224in}}%
\pgfpathlineto{\pgfqpoint{3.548734in}{1.384913in}}%
\pgfpathlineto{\pgfqpoint{3.549421in}{1.399323in}}%
\pgfpathlineto{\pgfqpoint{3.550108in}{1.542856in}}%
\pgfpathlineto{\pgfqpoint{3.550793in}{1.341477in}}%
\pgfpathlineto{\pgfqpoint{3.551476in}{1.347184in}}%
\pgfpathlineto{\pgfqpoint{3.552159in}{1.362267in}}%
\pgfpathlineto{\pgfqpoint{3.552841in}{1.493434in}}%
\pgfpathlineto{\pgfqpoint{3.553522in}{1.430572in}}%
\pgfpathlineto{\pgfqpoint{3.555557in}{1.490861in}}%
\pgfpathlineto{\pgfqpoint{3.556233in}{1.408046in}}%
\pgfpathlineto{\pgfqpoint{3.556908in}{1.503393in}}%
\pgfpathlineto{\pgfqpoint{3.557583in}{1.371306in}}%
\pgfpathlineto{\pgfqpoint{3.558256in}{1.439654in}}%
\pgfpathlineto{\pgfqpoint{3.558928in}{1.549422in}}%
\pgfpathlineto{\pgfqpoint{3.559599in}{1.544186in}}%
\pgfpathlineto{\pgfqpoint{3.561605in}{1.291626in}}%
\pgfpathlineto{\pgfqpoint{3.562272in}{1.482960in}}%
\pgfpathlineto{\pgfqpoint{3.562937in}{1.455824in}}%
\pgfpathlineto{\pgfqpoint{3.563602in}{1.326317in}}%
\pgfpathlineto{\pgfqpoint{3.564266in}{1.380579in}}%
\pgfpathlineto{\pgfqpoint{3.564928in}{1.481965in}}%
\pgfpathlineto{\pgfqpoint{3.565590in}{1.447365in}}%
\pgfpathlineto{\pgfqpoint{3.566910in}{1.568263in}}%
\pgfpathlineto{\pgfqpoint{3.567568in}{1.562855in}}%
\pgfpathlineto{\pgfqpoint{3.569537in}{1.366238in}}%
\pgfpathlineto{\pgfqpoint{3.570192in}{1.571361in}}%
\pgfpathlineto{\pgfqpoint{3.570845in}{1.474893in}}%
\pgfpathlineto{\pgfqpoint{3.571497in}{1.506931in}}%
\pgfpathlineto{\pgfqpoint{3.572149in}{1.330472in}}%
\pgfpathlineto{\pgfqpoint{3.572799in}{1.346260in}}%
\pgfpathlineto{\pgfqpoint{3.573448in}{1.509210in}}%
\pgfpathlineto{\pgfqpoint{3.574097in}{1.490338in}}%
\pgfpathlineto{\pgfqpoint{3.574744in}{1.483117in}}%
\pgfpathlineto{\pgfqpoint{3.575391in}{1.387201in}}%
\pgfpathlineto{\pgfqpoint{3.576680in}{1.513376in}}%
\pgfpathlineto{\pgfqpoint{3.577324in}{1.485500in}}%
\pgfpathlineto{\pgfqpoint{3.578608in}{1.529794in}}%
\pgfpathlineto{\pgfqpoint{3.579248in}{1.497759in}}%
\pgfpathlineto{\pgfqpoint{3.579888in}{1.334755in}}%
\pgfpathlineto{\pgfqpoint{3.580527in}{1.336273in}}%
\pgfpathlineto{\pgfqpoint{3.581801in}{1.532732in}}%
\pgfpathlineto{\pgfqpoint{3.583072in}{1.389355in}}%
\pgfpathlineto{\pgfqpoint{3.584338in}{1.509846in}}%
\pgfpathlineto{\pgfqpoint{3.586231in}{1.293069in}}%
\pgfpathlineto{\pgfqpoint{3.587489in}{1.486055in}}%
\pgfpathlineto{\pgfqpoint{3.588116in}{1.406120in}}%
\pgfpathlineto{\pgfqpoint{3.588742in}{1.464132in}}%
\pgfpathlineto{\pgfqpoint{3.590616in}{1.532649in}}%
\pgfpathlineto{\pgfqpoint{3.591239in}{1.408423in}}%
\pgfpathlineto{\pgfqpoint{3.591860in}{1.454842in}}%
\pgfpathlineto{\pgfqpoint{3.592481in}{1.509219in}}%
\pgfpathlineto{\pgfqpoint{3.594338in}{1.408412in}}%
\pgfpathlineto{\pgfqpoint{3.594956in}{1.425131in}}%
\pgfpathlineto{\pgfqpoint{3.595572in}{1.472083in}}%
\pgfpathlineto{\pgfqpoint{3.596188in}{1.459007in}}%
\pgfpathlineto{\pgfqpoint{3.596802in}{1.474956in}}%
\pgfpathlineto{\pgfqpoint{3.597416in}{1.369264in}}%
\pgfpathlineto{\pgfqpoint{3.598029in}{1.372829in}}%
\pgfpathlineto{\pgfqpoint{3.599252in}{1.516883in}}%
\pgfpathlineto{\pgfqpoint{3.600471in}{1.442950in}}%
\pgfpathlineto{\pgfqpoint{3.601079in}{1.557763in}}%
\pgfpathlineto{\pgfqpoint{3.601687in}{1.332619in}}%
\pgfpathlineto{\pgfqpoint{3.602294in}{1.570842in}}%
\pgfpathlineto{\pgfqpoint{3.602899in}{1.237782in}}%
\pgfpathlineto{\pgfqpoint{3.603504in}{1.373138in}}%
\pgfpathlineto{\pgfqpoint{3.604108in}{1.519062in}}%
\pgfpathlineto{\pgfqpoint{3.604712in}{1.453766in}}%
\pgfpathlineto{\pgfqpoint{3.605314in}{1.157683in}}%
\pgfpathlineto{\pgfqpoint{3.605915in}{1.494604in}}%
\pgfpathlineto{\pgfqpoint{3.606516in}{1.406177in}}%
\pgfpathlineto{\pgfqpoint{3.608313in}{1.527562in}}%
\pgfpathlineto{\pgfqpoint{3.609507in}{1.363826in}}%
\pgfpathlineto{\pgfqpoint{3.610102in}{1.526967in}}%
\pgfpathlineto{\pgfqpoint{3.610697in}{1.520167in}}%
\pgfpathlineto{\pgfqpoint{3.611291in}{1.407408in}}%
\pgfpathlineto{\pgfqpoint{3.611884in}{1.407537in}}%
\pgfpathlineto{\pgfqpoint{3.612476in}{1.426322in}}%
\pgfpathlineto{\pgfqpoint{3.613068in}{1.389479in}}%
\pgfpathlineto{\pgfqpoint{3.613658in}{1.488234in}}%
\pgfpathlineto{\pgfqpoint{3.614248in}{1.392874in}}%
\pgfpathlineto{\pgfqpoint{3.614837in}{1.407711in}}%
\pgfpathlineto{\pgfqpoint{3.615425in}{1.461138in}}%
\pgfpathlineto{\pgfqpoint{3.616013in}{1.366567in}}%
\pgfpathlineto{\pgfqpoint{3.616599in}{1.482783in}}%
\pgfpathlineto{\pgfqpoint{3.617185in}{1.451881in}}%
\pgfpathlineto{\pgfqpoint{3.618354in}{1.345811in}}%
\pgfpathlineto{\pgfqpoint{3.618937in}{1.571722in}}%
\pgfpathlineto{\pgfqpoint{3.619520in}{1.305964in}}%
\pgfpathlineto{\pgfqpoint{3.620102in}{1.376489in}}%
\pgfpathlineto{\pgfqpoint{3.620683in}{1.592050in}}%
\pgfpathlineto{\pgfqpoint{3.621263in}{1.513195in}}%
\pgfpathlineto{\pgfqpoint{3.621842in}{1.472813in}}%
\pgfpathlineto{\pgfqpoint{3.622421in}{1.547293in}}%
\pgfpathlineto{\pgfqpoint{3.624151in}{1.305882in}}%
\pgfpathlineto{\pgfqpoint{3.624727in}{1.530129in}}%
\pgfpathlineto{\pgfqpoint{3.625302in}{1.488116in}}%
\pgfpathlineto{\pgfqpoint{3.625875in}{1.468023in}}%
\pgfpathlineto{\pgfqpoint{3.627592in}{1.558147in}}%
\pgfpathlineto{\pgfqpoint{3.628733in}{1.481859in}}%
\pgfpathlineto{\pgfqpoint{3.629302in}{1.193814in}}%
\pgfpathlineto{\pgfqpoint{3.629871in}{1.215067in}}%
\pgfpathlineto{\pgfqpoint{3.631005in}{1.458356in}}%
\pgfpathlineto{\pgfqpoint{3.631572in}{1.346424in}}%
\pgfpathlineto{\pgfqpoint{3.632137in}{1.359818in}}%
\pgfpathlineto{\pgfqpoint{3.632702in}{1.522569in}}%
\pgfpathlineto{\pgfqpoint{3.633266in}{1.367711in}}%
\pgfpathlineto{\pgfqpoint{3.633829in}{1.312662in}}%
\pgfpathlineto{\pgfqpoint{3.635514in}{1.550078in}}%
\pgfpathlineto{\pgfqpoint{3.636074in}{1.536377in}}%
\pgfpathlineto{\pgfqpoint{3.636634in}{1.623478in}}%
\pgfpathlineto{\pgfqpoint{3.637192in}{1.438793in}}%
\pgfpathlineto{\pgfqpoint{3.637750in}{1.474505in}}%
\pgfpathlineto{\pgfqpoint{3.638308in}{1.502875in}}%
\pgfpathlineto{\pgfqpoint{3.638864in}{1.480178in}}%
\pgfpathlineto{\pgfqpoint{3.641083in}{1.214497in}}%
\pgfpathlineto{\pgfqpoint{3.642740in}{1.614178in}}%
\pgfpathlineto{\pgfqpoint{3.643840in}{1.416512in}}%
\pgfpathlineto{\pgfqpoint{3.644390in}{1.454253in}}%
\pgfpathlineto{\pgfqpoint{3.645486in}{1.552620in}}%
\pgfpathlineto{\pgfqpoint{3.646034in}{1.428915in}}%
\pgfpathlineto{\pgfqpoint{3.646580in}{1.516913in}}%
\pgfpathlineto{\pgfqpoint{3.647671in}{1.357781in}}%
\pgfpathlineto{\pgfqpoint{3.648215in}{1.382693in}}%
\pgfpathlineto{\pgfqpoint{3.648759in}{1.459139in}}%
\pgfpathlineto{\pgfqpoint{3.649844in}{1.280751in}}%
\pgfpathlineto{\pgfqpoint{3.650927in}{1.523976in}}%
\pgfpathlineto{\pgfqpoint{3.651467in}{1.419058in}}%
\pgfpathlineto{\pgfqpoint{3.652007in}{1.362155in}}%
\pgfpathlineto{\pgfqpoint{3.652546in}{1.373521in}}%
\pgfpathlineto{\pgfqpoint{3.653084in}{1.559071in}}%
\pgfpathlineto{\pgfqpoint{3.653621in}{1.497995in}}%
\pgfpathlineto{\pgfqpoint{3.654158in}{1.283486in}}%
\pgfpathlineto{\pgfqpoint{3.654694in}{1.501177in}}%
\pgfpathlineto{\pgfqpoint{3.655765in}{1.364878in}}%
\pgfpathlineto{\pgfqpoint{3.656299in}{1.400383in}}%
\pgfpathlineto{\pgfqpoint{3.657365in}{1.564099in}}%
\pgfpathlineto{\pgfqpoint{3.657897in}{1.484029in}}%
\pgfpathlineto{\pgfqpoint{3.658429in}{1.559170in}}%
\pgfpathlineto{\pgfqpoint{3.659489in}{1.450140in}}%
\pgfpathlineto{\pgfqpoint{3.660019in}{1.521078in}}%
\pgfpathlineto{\pgfqpoint{3.660548in}{1.582515in}}%
\pgfpathlineto{\pgfqpoint{3.661076in}{1.347177in}}%
\pgfpathlineto{\pgfqpoint{3.661603in}{1.484354in}}%
\pgfpathlineto{\pgfqpoint{3.662130in}{1.527037in}}%
\pgfpathlineto{\pgfqpoint{3.662656in}{1.376377in}}%
\pgfpathlineto{\pgfqpoint{3.663182in}{1.547480in}}%
\pgfpathlineto{\pgfqpoint{3.663707in}{1.478518in}}%
\pgfpathlineto{\pgfqpoint{3.664755in}{1.420920in}}%
\pgfpathlineto{\pgfqpoint{3.665278in}{1.526288in}}%
\pgfpathlineto{\pgfqpoint{3.665800in}{1.401383in}}%
\pgfpathlineto{\pgfqpoint{3.666321in}{1.610678in}}%
\pgfpathlineto{\pgfqpoint{3.666842in}{1.366434in}}%
\pgfpathlineto{\pgfqpoint{3.667363in}{1.368601in}}%
\pgfpathlineto{\pgfqpoint{3.667883in}{1.523472in}}%
\pgfpathlineto{\pgfqpoint{3.668402in}{1.403438in}}%
\pgfpathlineto{\pgfqpoint{3.668920in}{1.356356in}}%
\pgfpathlineto{\pgfqpoint{3.669438in}{1.427307in}}%
\pgfpathlineto{\pgfqpoint{3.670472in}{1.218931in}}%
\pgfpathlineto{\pgfqpoint{3.670988in}{1.287005in}}%
\pgfpathlineto{\pgfqpoint{3.671503in}{1.535550in}}%
\pgfpathlineto{\pgfqpoint{3.672532in}{1.503053in}}%
\pgfpathlineto{\pgfqpoint{3.673045in}{1.568983in}}%
\pgfpathlineto{\pgfqpoint{3.674582in}{1.412782in}}%
\pgfpathlineto{\pgfqpoint{3.675093in}{1.538342in}}%
\pgfpathlineto{\pgfqpoint{3.675603in}{1.459093in}}%
\pgfpathlineto{\pgfqpoint{3.676113in}{1.069748in}}%
\pgfpathlineto{\pgfqpoint{3.676622in}{1.493801in}}%
\pgfpathlineto{\pgfqpoint{3.677131in}{1.364204in}}%
\pgfpathlineto{\pgfqpoint{3.678652in}{1.566726in}}%
\pgfpathlineto{\pgfqpoint{3.679664in}{1.477268in}}%
\pgfpathlineto{\pgfqpoint{3.680169in}{1.519816in}}%
\pgfpathlineto{\pgfqpoint{3.680673in}{1.531053in}}%
\pgfpathlineto{\pgfqpoint{3.681680in}{1.502785in}}%
\pgfpathlineto{\pgfqpoint{3.682183in}{1.614427in}}%
\pgfpathlineto{\pgfqpoint{3.683686in}{1.321187in}}%
\pgfpathlineto{\pgfqpoint{3.684187in}{1.550394in}}%
\pgfpathlineto{\pgfqpoint{3.684686in}{1.434732in}}%
\pgfpathlineto{\pgfqpoint{3.685185in}{1.495210in}}%
\pgfpathlineto{\pgfqpoint{3.685683in}{1.396804in}}%
\pgfpathlineto{\pgfqpoint{3.686678in}{1.545173in}}%
\pgfpathlineto{\pgfqpoint{3.687671in}{1.403002in}}%
\pgfpathlineto{\pgfqpoint{3.688167in}{1.601147in}}%
\pgfpathlineto{\pgfqpoint{3.688661in}{1.430153in}}%
\pgfpathlineto{\pgfqpoint{3.689156in}{1.426065in}}%
\pgfpathlineto{\pgfqpoint{3.690143in}{1.585838in}}%
\pgfpathlineto{\pgfqpoint{3.691127in}{1.411549in}}%
\pgfpathlineto{\pgfqpoint{3.691619in}{1.549301in}}%
\pgfpathlineto{\pgfqpoint{3.692110in}{1.535373in}}%
\pgfpathlineto{\pgfqpoint{3.692600in}{1.456835in}}%
\pgfpathlineto{\pgfqpoint{3.693090in}{1.522340in}}%
\pgfpathlineto{\pgfqpoint{3.693579in}{1.572330in}}%
\pgfpathlineto{\pgfqpoint{3.694067in}{1.545937in}}%
\pgfpathlineto{\pgfqpoint{3.694556in}{1.340330in}}%
\pgfpathlineto{\pgfqpoint{3.695043in}{1.493596in}}%
\pgfpathlineto{\pgfqpoint{3.695530in}{1.550558in}}%
\pgfpathlineto{\pgfqpoint{3.696016in}{1.107452in}}%
\pgfpathlineto{\pgfqpoint{3.696502in}{1.503759in}}%
\pgfpathlineto{\pgfqpoint{3.696987in}{1.530484in}}%
\pgfpathlineto{\pgfqpoint{3.697472in}{1.414579in}}%
\pgfpathlineto{\pgfqpoint{3.697956in}{1.540788in}}%
\pgfpathlineto{\pgfqpoint{3.698440in}{1.583337in}}%
\pgfpathlineto{\pgfqpoint{3.698923in}{1.349094in}}%
\pgfpathlineto{\pgfqpoint{3.699887in}{1.375332in}}%
\pgfpathlineto{\pgfqpoint{3.700369in}{1.549466in}}%
\pgfpathlineto{\pgfqpoint{3.700850in}{1.396534in}}%
\pgfpathlineto{\pgfqpoint{3.701810in}{1.448682in}}%
\pgfpathlineto{\pgfqpoint{3.702289in}{1.343780in}}%
\pgfpathlineto{\pgfqpoint{3.702768in}{1.428711in}}%
\pgfpathlineto{\pgfqpoint{3.704201in}{1.565705in}}%
\pgfpathlineto{\pgfqpoint{3.705629in}{1.430678in}}%
\pgfpathlineto{\pgfqpoint{3.706103in}{1.568888in}}%
\pgfpathlineto{\pgfqpoint{3.706578in}{1.489706in}}%
\pgfpathlineto{\pgfqpoint{3.707052in}{1.452453in}}%
\pgfpathlineto{\pgfqpoint{3.707525in}{1.568723in}}%
\pgfpathlineto{\pgfqpoint{3.707998in}{1.487396in}}%
\pgfpathlineto{\pgfqpoint{3.708470in}{1.474557in}}%
\pgfpathlineto{\pgfqpoint{3.708942in}{1.391066in}}%
\pgfpathlineto{\pgfqpoint{3.709413in}{1.474018in}}%
\pgfpathlineto{\pgfqpoint{3.710354in}{1.594368in}}%
\pgfpathlineto{\pgfqpoint{3.711762in}{1.460639in}}%
\pgfpathlineto{\pgfqpoint{3.712230in}{1.490724in}}%
\pgfpathlineto{\pgfqpoint{3.712698in}{1.339643in}}%
\pgfpathlineto{\pgfqpoint{3.713165in}{1.577615in}}%
\pgfpathlineto{\pgfqpoint{3.713631in}{1.502116in}}%
\pgfpathlineto{\pgfqpoint{3.715028in}{1.282541in}}%
\pgfpathlineto{\pgfqpoint{3.715956in}{1.533075in}}%
\pgfpathlineto{\pgfqpoint{3.716420in}{1.532857in}}%
\pgfpathlineto{\pgfqpoint{3.717346in}{1.371146in}}%
\pgfpathlineto{\pgfqpoint{3.717808in}{1.428302in}}%
\pgfpathlineto{\pgfqpoint{3.718269in}{1.488218in}}%
\pgfpathlineto{\pgfqpoint{3.718730in}{1.341532in}}%
\pgfpathlineto{\pgfqpoint{3.719651in}{1.600785in}}%
\pgfpathlineto{\pgfqpoint{3.720110in}{1.509862in}}%
\pgfpathlineto{\pgfqpoint{3.720569in}{1.491547in}}%
\pgfpathlineto{\pgfqpoint{3.721028in}{1.379760in}}%
\pgfpathlineto{\pgfqpoint{3.721486in}{1.404782in}}%
\pgfpathlineto{\pgfqpoint{3.722857in}{1.619988in}}%
\pgfpathlineto{\pgfqpoint{3.724224in}{1.347695in}}%
\pgfpathlineto{\pgfqpoint{3.724678in}{1.465102in}}%
\pgfpathlineto{\pgfqpoint{3.725132in}{1.294142in}}%
\pgfpathlineto{\pgfqpoint{3.725586in}{1.495788in}}%
\pgfpathlineto{\pgfqpoint{3.726944in}{1.450648in}}%
\pgfpathlineto{\pgfqpoint{3.727847in}{1.633625in}}%
\pgfpathlineto{\pgfqpoint{3.728297in}{1.390942in}}%
\pgfpathlineto{\pgfqpoint{3.728748in}{1.439017in}}%
\pgfpathlineto{\pgfqpoint{3.729647in}{1.596511in}}%
\pgfpathlineto{\pgfqpoint{3.730096in}{1.409768in}}%
\pgfpathlineto{\pgfqpoint{3.730544in}{1.561384in}}%
\pgfpathlineto{\pgfqpoint{3.730992in}{1.555157in}}%
\pgfpathlineto{\pgfqpoint{3.731439in}{1.564687in}}%
\pgfpathlineto{\pgfqpoint{3.732333in}{1.431227in}}%
\pgfpathlineto{\pgfqpoint{3.732779in}{1.608217in}}%
\pgfpathlineto{\pgfqpoint{3.733669in}{1.553517in}}%
\pgfpathlineto{\pgfqpoint{3.734114in}{1.369719in}}%
\pgfpathlineto{\pgfqpoint{3.734558in}{1.453930in}}%
\pgfpathlineto{\pgfqpoint{3.735002in}{1.560948in}}%
\pgfpathlineto{\pgfqpoint{3.735445in}{1.420584in}}%
\pgfpathlineto{\pgfqpoint{3.736772in}{1.528723in}}%
\pgfpathlineto{\pgfqpoint{3.737213in}{1.476120in}}%
\pgfpathlineto{\pgfqpoint{3.738094in}{1.493285in}}%
\pgfpathlineto{\pgfqpoint{3.738534in}{1.534419in}}%
\pgfpathlineto{\pgfqpoint{3.738974in}{1.422050in}}%
\pgfpathlineto{\pgfqpoint{3.739851in}{1.449884in}}%
\pgfpathlineto{\pgfqpoint{3.740727in}{1.611588in}}%
\pgfpathlineto{\pgfqpoint{3.742474in}{1.343212in}}%
\pgfpathlineto{\pgfqpoint{3.743779in}{1.577615in}}%
\pgfpathlineto{\pgfqpoint{3.744647in}{1.554221in}}%
\pgfpathlineto{\pgfqpoint{3.745945in}{1.324109in}}%
\pgfpathlineto{\pgfqpoint{3.745513in}{1.556457in}}%
\pgfpathlineto{\pgfqpoint{3.746808in}{1.334621in}}%
\pgfpathlineto{\pgfqpoint{3.748100in}{1.530446in}}%
\pgfpathlineto{\pgfqpoint{3.748959in}{1.394443in}}%
\pgfpathlineto{\pgfqpoint{3.750245in}{1.505579in}}%
\pgfpathlineto{\pgfqpoint{3.750672in}{1.430457in}}%
\pgfpathlineto{\pgfqpoint{3.751099in}{1.444139in}}%
\pgfpathlineto{\pgfqpoint{3.751952in}{1.619406in}}%
\pgfpathlineto{\pgfqpoint{3.752378in}{1.170869in}}%
\pgfpathlineto{\pgfqpoint{3.752804in}{1.533721in}}%
\pgfpathlineto{\pgfqpoint{3.753229in}{1.635717in}}%
\pgfpathlineto{\pgfqpoint{3.753653in}{1.517325in}}%
\pgfpathlineto{\pgfqpoint{3.754077in}{1.404403in}}%
\pgfpathlineto{\pgfqpoint{3.754501in}{1.442906in}}%
\pgfpathlineto{\pgfqpoint{3.754924in}{1.584640in}}%
\pgfpathlineto{\pgfqpoint{3.755770in}{1.570096in}}%
\pgfpathlineto{\pgfqpoint{3.756192in}{1.560810in}}%
\pgfpathlineto{\pgfqpoint{3.756613in}{1.587740in}}%
\pgfpathlineto{\pgfqpoint{3.758715in}{1.306042in}}%
\pgfpathlineto{\pgfqpoint{3.760389in}{1.551852in}}%
\pgfpathlineto{\pgfqpoint{3.760807in}{1.535670in}}%
\pgfpathlineto{\pgfqpoint{3.761224in}{1.573587in}}%
\pgfpathlineto{\pgfqpoint{3.761640in}{1.444191in}}%
\pgfpathlineto{\pgfqpoint{3.762057in}{1.559472in}}%
\pgfpathlineto{\pgfqpoint{3.762472in}{1.558461in}}%
\pgfpathlineto{\pgfqpoint{3.762888in}{1.517421in}}%
\pgfpathlineto{\pgfqpoint{3.763718in}{1.521722in}}%
\pgfpathlineto{\pgfqpoint{3.764959in}{1.597518in}}%
\pgfpathlineto{\pgfqpoint{3.765372in}{1.353557in}}%
\pgfpathlineto{\pgfqpoint{3.765785in}{1.543167in}}%
\pgfpathlineto{\pgfqpoint{3.766609in}{1.561638in}}%
\pgfpathlineto{\pgfqpoint{3.767020in}{1.384831in}}%
\pgfpathlineto{\pgfqpoint{3.767431in}{1.517030in}}%
\pgfpathlineto{\pgfqpoint{3.767842in}{1.519813in}}%
\pgfpathlineto{\pgfqpoint{3.768662in}{1.592515in}}%
\pgfpathlineto{\pgfqpoint{3.769889in}{1.468306in}}%
\pgfpathlineto{\pgfqpoint{3.770297in}{1.481648in}}%
\pgfpathlineto{\pgfqpoint{3.771519in}{1.527315in}}%
\pgfpathlineto{\pgfqpoint{3.771926in}{1.451120in}}%
\pgfpathlineto{\pgfqpoint{3.772332in}{1.593464in}}%
\pgfpathlineto{\pgfqpoint{3.773144in}{1.584593in}}%
\pgfpathlineto{\pgfqpoint{3.773549in}{1.365682in}}%
\pgfpathlineto{\pgfqpoint{3.773953in}{1.447171in}}%
\pgfpathlineto{\pgfqpoint{3.774358in}{1.588286in}}%
\pgfpathlineto{\pgfqpoint{3.774762in}{1.424933in}}%
\pgfpathlineto{\pgfqpoint{3.775165in}{1.405968in}}%
\pgfpathlineto{\pgfqpoint{3.775569in}{1.326170in}}%
\pgfpathlineto{\pgfqpoint{3.775971in}{1.575874in}}%
\pgfpathlineto{\pgfqpoint{3.776776in}{1.512043in}}%
\pgfpathlineto{\pgfqpoint{3.777178in}{1.548096in}}%
\pgfpathlineto{\pgfqpoint{3.777579in}{1.307153in}}%
\pgfpathlineto{\pgfqpoint{3.777980in}{1.420857in}}%
\pgfpathlineto{\pgfqpoint{3.779580in}{1.657747in}}%
\pgfpathlineto{\pgfqpoint{3.779979in}{1.415314in}}%
\pgfpathlineto{\pgfqpoint{3.780776in}{1.572627in}}%
\pgfpathlineto{\pgfqpoint{3.781173in}{1.553242in}}%
\pgfpathlineto{\pgfqpoint{3.781571in}{1.396752in}}%
\pgfpathlineto{\pgfqpoint{3.781968in}{1.544772in}}%
\pgfpathlineto{\pgfqpoint{3.782365in}{1.545505in}}%
\pgfpathlineto{\pgfqpoint{3.784343in}{1.320708in}}%
\pgfpathlineto{\pgfqpoint{3.784738in}{1.566829in}}%
\pgfpathlineto{\pgfqpoint{3.785526in}{1.553541in}}%
\pgfpathlineto{\pgfqpoint{3.786313in}{1.424538in}}%
\pgfpathlineto{\pgfqpoint{3.787098in}{1.432102in}}%
\pgfpathlineto{\pgfqpoint{3.787490in}{1.529632in}}%
\pgfpathlineto{\pgfqpoint{3.787881in}{1.456233in}}%
\pgfpathlineto{\pgfqpoint{3.788273in}{1.432404in}}%
\pgfpathlineto{\pgfqpoint{3.788664in}{1.275699in}}%
\pgfpathlineto{\pgfqpoint{3.789054in}{1.544817in}}%
\pgfpathlineto{\pgfqpoint{3.789834in}{1.518046in}}%
\pgfpathlineto{\pgfqpoint{3.790613in}{1.605062in}}%
\pgfpathlineto{\pgfqpoint{3.792166in}{1.377681in}}%
\pgfpathlineto{\pgfqpoint{3.792553in}{1.384982in}}%
\pgfpathlineto{\pgfqpoint{3.794099in}{1.596194in}}%
\pgfpathlineto{\pgfqpoint{3.794485in}{1.593705in}}%
\pgfpathlineto{\pgfqpoint{3.794870in}{1.333390in}}%
\pgfpathlineto{\pgfqpoint{3.795640in}{1.558753in}}%
\pgfpathlineto{\pgfqpoint{3.796024in}{1.569780in}}%
\pgfpathlineto{\pgfqpoint{3.796408in}{1.663708in}}%
\pgfpathlineto{\pgfqpoint{3.796791in}{1.652418in}}%
\pgfpathlineto{\pgfqpoint{3.797940in}{1.483715in}}%
\pgfpathlineto{\pgfqpoint{3.798704in}{1.604505in}}%
\pgfpathlineto{\pgfqpoint{3.799085in}{1.572621in}}%
\pgfpathlineto{\pgfqpoint{3.799466in}{1.363506in}}%
\pgfpathlineto{\pgfqpoint{3.799847in}{1.591762in}}%
\pgfpathlineto{\pgfqpoint{3.801746in}{1.402813in}}%
\pgfpathlineto{\pgfqpoint{3.802124in}{1.288464in}}%
\pgfpathlineto{\pgfqpoint{3.802881in}{1.581582in}}%
\pgfpathlineto{\pgfqpoint{3.803636in}{1.558292in}}%
\pgfpathlineto{\pgfqpoint{3.804013in}{1.591656in}}%
\pgfpathlineto{\pgfqpoint{3.804390in}{1.542657in}}%
\pgfpathlineto{\pgfqpoint{3.804766in}{1.408601in}}%
\pgfpathlineto{\pgfqpoint{3.805518in}{1.466655in}}%
\pgfpathlineto{\pgfqpoint{3.805893in}{1.560527in}}%
\pgfpathlineto{\pgfqpoint{3.806643in}{1.497172in}}%
\pgfpathlineto{\pgfqpoint{3.807017in}{1.553744in}}%
\pgfpathlineto{\pgfqpoint{3.807391in}{1.243838in}}%
\pgfpathlineto{\pgfqpoint{3.807765in}{1.653968in}}%
\pgfpathlineto{\pgfqpoint{3.808138in}{1.460064in}}%
\pgfpathlineto{\pgfqpoint{3.808511in}{1.616478in}}%
\pgfpathlineto{\pgfqpoint{3.809257in}{1.587277in}}%
\pgfpathlineto{\pgfqpoint{3.810372in}{1.461163in}}%
\pgfpathlineto{\pgfqpoint{3.811854in}{1.595201in}}%
\pgfpathlineto{\pgfqpoint{3.812963in}{1.461550in}}%
\pgfpathlineto{\pgfqpoint{3.813332in}{1.481165in}}%
\pgfpathlineto{\pgfqpoint{3.813700in}{1.441764in}}%
\pgfpathlineto{\pgfqpoint{3.814804in}{1.606404in}}%
\pgfpathlineto{\pgfqpoint{3.815538in}{1.640442in}}%
\pgfpathlineto{\pgfqpoint{3.815905in}{1.515941in}}%
\pgfpathlineto{\pgfqpoint{3.816271in}{1.598611in}}%
\pgfpathlineto{\pgfqpoint{3.816637in}{1.569050in}}%
\pgfpathlineto{\pgfqpoint{3.817003in}{1.356873in}}%
\pgfpathlineto{\pgfqpoint{3.817733in}{1.361064in}}%
\pgfpathlineto{\pgfqpoint{3.818462in}{1.356170in}}%
\pgfpathlineto{\pgfqpoint{3.818826in}{1.609130in}}%
\pgfpathlineto{\pgfqpoint{3.819554in}{1.619553in}}%
\pgfpathlineto{\pgfqpoint{3.819917in}{1.488254in}}%
\pgfpathlineto{\pgfqpoint{3.821366in}{1.613016in}}%
\pgfpathlineto{\pgfqpoint{3.823171in}{1.284920in}}%
\pgfpathlineto{\pgfqpoint{3.823531in}{1.471268in}}%
\pgfpathlineto{\pgfqpoint{3.824251in}{1.464093in}}%
\pgfpathlineto{\pgfqpoint{3.824610in}{1.453066in}}%
\pgfpathlineto{\pgfqpoint{3.824969in}{1.467167in}}%
\pgfpathlineto{\pgfqpoint{3.826044in}{1.526554in}}%
\pgfpathlineto{\pgfqpoint{3.826401in}{1.439370in}}%
\pgfpathlineto{\pgfqpoint{3.826759in}{1.568676in}}%
\pgfpathlineto{\pgfqpoint{3.827116in}{1.559805in}}%
\pgfpathlineto{\pgfqpoint{3.827472in}{1.348042in}}%
\pgfpathlineto{\pgfqpoint{3.827829in}{1.608711in}}%
\pgfpathlineto{\pgfqpoint{3.828185in}{1.528600in}}%
\pgfpathlineto{\pgfqpoint{3.828896in}{1.596964in}}%
\pgfpathlineto{\pgfqpoint{3.829607in}{1.599167in}}%
\pgfpathlineto{\pgfqpoint{3.830670in}{1.115056in}}%
\pgfpathlineto{\pgfqpoint{3.831377in}{1.579814in}}%
\pgfpathlineto{\pgfqpoint{3.831730in}{1.450933in}}%
\pgfpathlineto{\pgfqpoint{3.832083in}{1.445460in}}%
\pgfpathlineto{\pgfqpoint{3.832436in}{1.623732in}}%
\pgfpathlineto{\pgfqpoint{3.833492in}{1.599076in}}%
\pgfpathlineto{\pgfqpoint{3.834896in}{1.499181in}}%
\pgfpathlineto{\pgfqpoint{3.835246in}{1.553298in}}%
\pgfpathlineto{\pgfqpoint{3.835596in}{1.622024in}}%
\pgfpathlineto{\pgfqpoint{3.835946in}{1.378245in}}%
\pgfpathlineto{\pgfqpoint{3.836644in}{1.484763in}}%
\pgfpathlineto{\pgfqpoint{3.837690in}{1.603918in}}%
\pgfpathlineto{\pgfqpoint{3.838038in}{1.426570in}}%
\pgfpathlineto{\pgfqpoint{3.838386in}{1.494353in}}%
\pgfpathlineto{\pgfqpoint{3.838733in}{1.643677in}}%
\pgfpathlineto{\pgfqpoint{3.839427in}{1.534147in}}%
\pgfpathlineto{\pgfqpoint{3.840120in}{1.487390in}}%
\pgfpathlineto{\pgfqpoint{3.840466in}{1.514860in}}%
\pgfpathlineto{\pgfqpoint{3.841502in}{1.670209in}}%
\pgfpathlineto{\pgfqpoint{3.841847in}{1.498238in}}%
\pgfpathlineto{\pgfqpoint{3.842536in}{1.602894in}}%
\pgfpathlineto{\pgfqpoint{3.843224in}{1.641891in}}%
\pgfpathlineto{\pgfqpoint{3.843910in}{1.439795in}}%
\pgfpathlineto{\pgfqpoint{3.844938in}{1.636894in}}%
\pgfpathlineto{\pgfqpoint{3.844596in}{1.372491in}}%
\pgfpathlineto{\pgfqpoint{3.845280in}{1.551475in}}%
\pgfpathlineto{\pgfqpoint{3.845622in}{1.602203in}}%
\pgfpathlineto{\pgfqpoint{3.845964in}{1.497995in}}%
\pgfpathlineto{\pgfqpoint{3.846305in}{1.502464in}}%
\pgfpathlineto{\pgfqpoint{3.846646in}{1.478608in}}%
\pgfpathlineto{\pgfqpoint{3.847327in}{1.600849in}}%
\pgfpathlineto{\pgfqpoint{3.848007in}{1.622306in}}%
\pgfpathlineto{\pgfqpoint{3.848686in}{1.444265in}}%
\pgfpathlineto{\pgfqpoint{3.849025in}{1.612339in}}%
\pgfpathlineto{\pgfqpoint{3.849703in}{1.422978in}}%
\pgfpathlineto{\pgfqpoint{3.850379in}{1.592440in}}%
\pgfpathlineto{\pgfqpoint{3.850717in}{1.507836in}}%
\pgfpathlineto{\pgfqpoint{3.851054in}{1.470523in}}%
\pgfpathlineto{\pgfqpoint{3.852065in}{1.466816in}}%
\pgfpathlineto{\pgfqpoint{3.852401in}{1.648650in}}%
\pgfpathlineto{\pgfqpoint{3.852737in}{1.396386in}}%
\pgfpathlineto{\pgfqpoint{3.853409in}{1.572153in}}%
\pgfpathlineto{\pgfqpoint{3.854079in}{1.355079in}}%
\pgfpathlineto{\pgfqpoint{3.854414in}{1.399389in}}%
\pgfpathlineto{\pgfqpoint{3.855751in}{1.658528in}}%
\pgfpathlineto{\pgfqpoint{3.857416in}{1.439059in}}%
\pgfpathlineto{\pgfqpoint{3.857748in}{1.612072in}}%
\pgfpathlineto{\pgfqpoint{3.858411in}{1.586003in}}%
\pgfpathlineto{\pgfqpoint{3.859405in}{1.350696in}}%
\pgfpathlineto{\pgfqpoint{3.859735in}{1.634209in}}%
\pgfpathlineto{\pgfqpoint{3.860726in}{1.573959in}}%
\pgfpathlineto{\pgfqpoint{3.861055in}{1.564618in}}%
\pgfpathlineto{\pgfqpoint{3.861385in}{1.299762in}}%
\pgfpathlineto{\pgfqpoint{3.861714in}{1.633219in}}%
\pgfpathlineto{\pgfqpoint{3.862043in}{1.633209in}}%
\pgfpathlineto{\pgfqpoint{3.863028in}{1.299665in}}%
\pgfpathlineto{\pgfqpoint{3.862700in}{1.640376in}}%
\pgfpathlineto{\pgfqpoint{3.863355in}{1.344763in}}%
\pgfpathlineto{\pgfqpoint{3.864337in}{1.615581in}}%
\pgfpathlineto{\pgfqpoint{3.864664in}{1.404688in}}%
\pgfpathlineto{\pgfqpoint{3.865317in}{1.542978in}}%
\pgfpathlineto{\pgfqpoint{3.865643in}{1.586654in}}%
\pgfpathlineto{\pgfqpoint{3.865969in}{1.491894in}}%
\pgfpathlineto{\pgfqpoint{3.866294in}{1.465304in}}%
\pgfpathlineto{\pgfqpoint{3.866945in}{1.659807in}}%
\pgfpathlineto{\pgfqpoint{3.867270in}{1.395329in}}%
\pgfpathlineto{\pgfqpoint{3.867918in}{1.574932in}}%
\pgfpathlineto{\pgfqpoint{3.868243in}{1.366355in}}%
\pgfpathlineto{\pgfqpoint{3.868566in}{1.365037in}}%
\pgfpathlineto{\pgfqpoint{3.868890in}{1.621993in}}%
\pgfpathlineto{\pgfqpoint{3.869859in}{1.597044in}}%
\pgfpathlineto{\pgfqpoint{3.870182in}{1.438382in}}%
\pgfpathlineto{\pgfqpoint{3.870826in}{1.624584in}}%
\pgfpathlineto{\pgfqpoint{3.871470in}{1.389174in}}%
\pgfpathlineto{\pgfqpoint{3.871791in}{1.461199in}}%
\pgfpathlineto{\pgfqpoint{3.872754in}{1.615925in}}%
\pgfpathlineto{\pgfqpoint{3.873074in}{1.401423in}}%
\pgfpathlineto{\pgfqpoint{3.874034in}{1.454797in}}%
\pgfpathlineto{\pgfqpoint{3.874353in}{1.617081in}}%
\pgfpathlineto{\pgfqpoint{3.874672in}{1.437009in}}%
\pgfpathlineto{\pgfqpoint{3.874991in}{1.512583in}}%
\pgfpathlineto{\pgfqpoint{3.875310in}{1.505344in}}%
\pgfpathlineto{\pgfqpoint{3.875629in}{1.672628in}}%
\pgfpathlineto{\pgfqpoint{3.876265in}{1.588316in}}%
\pgfpathlineto{\pgfqpoint{3.876583in}{1.389489in}}%
\pgfpathlineto{\pgfqpoint{3.876900in}{1.600370in}}%
\pgfpathlineto{\pgfqpoint{3.877217in}{1.560459in}}%
\pgfpathlineto{\pgfqpoint{3.877534in}{1.567746in}}%
\pgfpathlineto{\pgfqpoint{3.878484in}{1.431750in}}%
\pgfpathlineto{\pgfqpoint{3.878800in}{1.522528in}}%
\pgfpathlineto{\pgfqpoint{3.879116in}{1.526787in}}%
\pgfpathlineto{\pgfqpoint{3.879432in}{1.496113in}}%
\pgfpathlineto{\pgfqpoint{3.880377in}{1.459751in}}%
\pgfpathlineto{\pgfqpoint{3.880692in}{1.652633in}}%
\pgfpathlineto{\pgfqpoint{3.881948in}{1.589252in}}%
\pgfpathlineto{\pgfqpoint{3.882262in}{1.318253in}}%
\pgfpathlineto{\pgfqpoint{3.882575in}{1.633511in}}%
\pgfpathlineto{\pgfqpoint{3.882888in}{1.443729in}}%
\pgfpathlineto{\pgfqpoint{3.883514in}{1.613820in}}%
\pgfpathlineto{\pgfqpoint{3.884138in}{1.574053in}}%
\pgfpathlineto{\pgfqpoint{3.884762in}{1.582644in}}%
\pgfpathlineto{\pgfqpoint{3.885073in}{1.330680in}}%
\pgfpathlineto{\pgfqpoint{3.885384in}{1.559985in}}%
\pgfpathlineto{\pgfqpoint{3.886006in}{1.522320in}}%
\pgfpathlineto{\pgfqpoint{3.886317in}{1.179508in}}%
\pgfpathlineto{\pgfqpoint{3.886627in}{1.571687in}}%
\pgfpathlineto{\pgfqpoint{3.886937in}{1.557346in}}%
\pgfpathlineto{\pgfqpoint{3.887247in}{1.656789in}}%
\pgfpathlineto{\pgfqpoint{3.887866in}{1.522717in}}%
\pgfpathlineto{\pgfqpoint{3.888484in}{1.436703in}}%
\pgfpathlineto{\pgfqpoint{3.889410in}{1.657729in}}%
\pgfpathlineto{\pgfqpoint{3.889718in}{1.492740in}}%
\pgfpathlineto{\pgfqpoint{3.890333in}{1.591182in}}%
\pgfpathlineto{\pgfqpoint{3.890641in}{1.603961in}}%
\pgfpathlineto{\pgfqpoint{3.890948in}{1.574914in}}%
\pgfpathlineto{\pgfqpoint{3.891255in}{1.512427in}}%
\pgfpathlineto{\pgfqpoint{3.892174in}{1.546048in}}%
\pgfpathlineto{\pgfqpoint{3.892480in}{1.593615in}}%
\pgfpathlineto{\pgfqpoint{3.893397in}{1.430416in}}%
\pgfpathlineto{\pgfqpoint{3.893703in}{1.455424in}}%
\pgfpathlineto{\pgfqpoint{3.894617in}{1.627518in}}%
\pgfpathlineto{\pgfqpoint{3.894921in}{1.551001in}}%
\pgfpathlineto{\pgfqpoint{3.895225in}{1.355356in}}%
\pgfpathlineto{\pgfqpoint{3.895833in}{1.632849in}}%
\pgfpathlineto{\pgfqpoint{3.896136in}{1.375566in}}%
\pgfpathlineto{\pgfqpoint{3.897348in}{1.627653in}}%
\pgfpathlineto{\pgfqpoint{3.898556in}{1.366168in}}%
\pgfpathlineto{\pgfqpoint{3.899159in}{1.635251in}}%
\pgfpathlineto{\pgfqpoint{3.899761in}{1.577027in}}%
\pgfpathlineto{\pgfqpoint{3.900362in}{1.601145in}}%
\pgfpathlineto{\pgfqpoint{3.900962in}{1.307374in}}%
\pgfpathlineto{\pgfqpoint{3.902160in}{1.650013in}}%
\pgfpathlineto{\pgfqpoint{3.903354in}{1.422037in}}%
\pgfpathlineto{\pgfqpoint{3.903950in}{1.648103in}}%
\pgfpathlineto{\pgfqpoint{3.904546in}{1.572197in}}%
\pgfpathlineto{\pgfqpoint{3.904843in}{1.556682in}}%
\pgfpathlineto{\pgfqpoint{3.905140in}{1.679995in}}%
\pgfpathlineto{\pgfqpoint{3.905733in}{1.541594in}}%
\pgfpathlineto{\pgfqpoint{3.906918in}{1.360101in}}%
\pgfpathlineto{\pgfqpoint{3.907804in}{1.530442in}}%
\pgfpathlineto{\pgfqpoint{3.908099in}{1.512162in}}%
\pgfpathlineto{\pgfqpoint{3.908394in}{1.496133in}}%
\pgfpathlineto{\pgfqpoint{3.908983in}{1.624217in}}%
\pgfpathlineto{\pgfqpoint{3.910159in}{1.338846in}}%
\pgfpathlineto{\pgfqpoint{3.911623in}{1.593302in}}%
\pgfpathlineto{\pgfqpoint{3.911916in}{1.594156in}}%
\pgfpathlineto{\pgfqpoint{3.912208in}{1.484257in}}%
\pgfpathlineto{\pgfqpoint{3.912792in}{1.552865in}}%
\pgfpathlineto{\pgfqpoint{3.913083in}{1.644750in}}%
\pgfpathlineto{\pgfqpoint{3.913375in}{1.556687in}}%
\pgfpathlineto{\pgfqpoint{3.913666in}{1.358029in}}%
\pgfpathlineto{\pgfqpoint{3.914247in}{1.676192in}}%
\pgfpathlineto{\pgfqpoint{3.914538in}{1.499796in}}%
\pgfpathlineto{\pgfqpoint{3.914828in}{1.515840in}}%
\pgfpathlineto{\pgfqpoint{3.915118in}{1.682513in}}%
\pgfpathlineto{\pgfqpoint{3.915408in}{1.497314in}}%
\pgfpathlineto{\pgfqpoint{3.915698in}{1.585189in}}%
\pgfpathlineto{\pgfqpoint{3.916855in}{1.449354in}}%
\pgfpathlineto{\pgfqpoint{3.917144in}{1.632019in}}%
\pgfpathlineto{\pgfqpoint{3.918009in}{1.569330in}}%
\pgfpathlineto{\pgfqpoint{3.918297in}{1.555431in}}%
\pgfpathlineto{\pgfqpoint{3.919160in}{1.686442in}}%
\pgfpathlineto{\pgfqpoint{3.918873in}{1.518411in}}%
\pgfpathlineto{\pgfqpoint{3.919447in}{1.639015in}}%
\pgfpathlineto{\pgfqpoint{3.919734in}{1.391144in}}%
\pgfpathlineto{\pgfqpoint{3.920594in}{1.597628in}}%
\pgfpathlineto{\pgfqpoint{3.921167in}{1.494642in}}%
\pgfpathlineto{\pgfqpoint{3.921452in}{1.622200in}}%
\pgfpathlineto{\pgfqpoint{3.922024in}{1.507624in}}%
\pgfpathlineto{\pgfqpoint{3.922879in}{1.659521in}}%
\pgfpathlineto{\pgfqpoint{3.923164in}{1.551870in}}%
\pgfpathlineto{\pgfqpoint{3.924017in}{1.599734in}}%
\pgfpathlineto{\pgfqpoint{3.924300in}{1.599427in}}%
\pgfpathlineto{\pgfqpoint{3.924584in}{1.499902in}}%
\pgfpathlineto{\pgfqpoint{3.924868in}{1.610211in}}%
\pgfpathlineto{\pgfqpoint{3.925151in}{1.582904in}}%
\pgfpathlineto{\pgfqpoint{3.925434in}{1.645773in}}%
\pgfpathlineto{\pgfqpoint{3.925717in}{1.612778in}}%
\pgfpathlineto{\pgfqpoint{3.926000in}{1.524594in}}%
\pgfpathlineto{\pgfqpoint{3.926565in}{1.654784in}}%
\pgfpathlineto{\pgfqpoint{3.928256in}{1.395777in}}%
\pgfpathlineto{\pgfqpoint{3.928537in}{1.455219in}}%
\pgfpathlineto{\pgfqpoint{3.928818in}{1.580185in}}%
\pgfpathlineto{\pgfqpoint{3.929379in}{1.450813in}}%
\pgfpathlineto{\pgfqpoint{3.929660in}{1.524007in}}%
\pgfpathlineto{\pgfqpoint{3.929940in}{1.310986in}}%
\pgfpathlineto{\pgfqpoint{3.930220in}{1.684344in}}%
\pgfpathlineto{\pgfqpoint{3.930500in}{1.416806in}}%
\pgfpathlineto{\pgfqpoint{3.931896in}{1.632477in}}%
\pgfpathlineto{\pgfqpoint{3.932175in}{1.716979in}}%
\pgfpathlineto{\pgfqpoint{3.932732in}{1.600008in}}%
\pgfpathlineto{\pgfqpoint{3.933010in}{1.652192in}}%
\pgfpathlineto{\pgfqpoint{3.934121in}{1.352596in}}%
\pgfpathlineto{\pgfqpoint{3.934398in}{1.675728in}}%
\pgfpathlineto{\pgfqpoint{3.935229in}{1.496505in}}%
\pgfpathlineto{\pgfqpoint{3.935505in}{1.545250in}}%
\pgfpathlineto{\pgfqpoint{3.935782in}{1.249818in}}%
\pgfpathlineto{\pgfqpoint{3.936334in}{1.628150in}}%
\pgfpathlineto{\pgfqpoint{3.936610in}{1.714220in}}%
\pgfpathlineto{\pgfqpoint{3.936885in}{1.609154in}}%
\pgfpathlineto{\pgfqpoint{3.937161in}{1.515961in}}%
\pgfpathlineto{\pgfqpoint{3.937436in}{1.671565in}}%
\pgfpathlineto{\pgfqpoint{3.937711in}{1.638599in}}%
\pgfpathlineto{\pgfqpoint{3.937986in}{1.626787in}}%
\pgfpathlineto{\pgfqpoint{3.938261in}{1.456295in}}%
\pgfpathlineto{\pgfqpoint{3.938536in}{1.675973in}}%
\pgfpathlineto{\pgfqpoint{3.939084in}{1.541701in}}%
\pgfpathlineto{\pgfqpoint{3.939358in}{1.593264in}}%
\pgfpathlineto{\pgfqpoint{3.939906in}{1.574116in}}%
\pgfpathlineto{\pgfqpoint{3.940179in}{1.511191in}}%
\pgfpathlineto{\pgfqpoint{3.940453in}{1.665363in}}%
\pgfpathlineto{\pgfqpoint{3.940726in}{1.647711in}}%
\pgfpathlineto{\pgfqpoint{3.941817in}{1.479145in}}%
\pgfpathlineto{\pgfqpoint{3.942361in}{1.648036in}}%
\pgfpathlineto{\pgfqpoint{3.942633in}{1.355528in}}%
\pgfpathlineto{\pgfqpoint{3.943176in}{1.605882in}}%
\pgfpathlineto{\pgfqpoint{3.943448in}{1.651437in}}%
\pgfpathlineto{\pgfqpoint{3.943719in}{1.471119in}}%
\pgfpathlineto{\pgfqpoint{3.944532in}{1.592691in}}%
\pgfpathlineto{\pgfqpoint{3.945073in}{1.617350in}}%
\pgfpathlineto{\pgfqpoint{3.945343in}{1.538517in}}%
\pgfpathlineto{\pgfqpoint{3.945883in}{1.547446in}}%
\pgfpathlineto{\pgfqpoint{3.946153in}{1.675816in}}%
\pgfpathlineto{\pgfqpoint{3.946422in}{1.566776in}}%
\pgfpathlineto{\pgfqpoint{3.946961in}{1.677816in}}%
\pgfpathlineto{\pgfqpoint{3.947230in}{1.190805in}}%
\pgfpathlineto{\pgfqpoint{3.947767in}{1.708790in}}%
\pgfpathlineto{\pgfqpoint{3.948304in}{1.547216in}}%
\pgfpathlineto{\pgfqpoint{3.948572in}{1.555537in}}%
\pgfpathlineto{\pgfqpoint{3.949108in}{1.667348in}}%
\pgfpathlineto{\pgfqpoint{3.949643in}{1.548244in}}%
\pgfpathlineto{\pgfqpoint{3.949910in}{1.529775in}}%
\pgfpathlineto{\pgfqpoint{3.950445in}{1.500779in}}%
\pgfpathlineto{\pgfqpoint{3.950978in}{1.667486in}}%
\pgfpathlineto{\pgfqpoint{3.952309in}{1.425968in}}%
\pgfpathlineto{\pgfqpoint{3.953105in}{1.624116in}}%
\pgfpathlineto{\pgfqpoint{3.953370in}{1.599599in}}%
\pgfpathlineto{\pgfqpoint{3.954165in}{1.383164in}}%
\pgfpathlineto{\pgfqpoint{3.954429in}{1.547600in}}%
\pgfpathlineto{\pgfqpoint{3.954693in}{1.679923in}}%
\pgfpathlineto{\pgfqpoint{3.954958in}{1.494713in}}%
\pgfpathlineto{\pgfqpoint{3.955485in}{1.568035in}}%
\pgfpathlineto{\pgfqpoint{3.955749in}{1.464103in}}%
\pgfpathlineto{\pgfqpoint{3.956013in}{1.594769in}}%
\pgfpathlineto{\pgfqpoint{3.956539in}{1.552948in}}%
\pgfpathlineto{\pgfqpoint{3.957065in}{1.549695in}}%
\pgfpathlineto{\pgfqpoint{3.957328in}{1.594878in}}%
\pgfpathlineto{\pgfqpoint{3.958115in}{1.486759in}}%
\pgfpathlineto{\pgfqpoint{3.958377in}{1.702268in}}%
\pgfpathlineto{\pgfqpoint{3.959423in}{1.663493in}}%
\pgfpathlineto{\pgfqpoint{3.959684in}{1.665595in}}%
\pgfpathlineto{\pgfqpoint{3.960728in}{1.513876in}}%
\pgfpathlineto{\pgfqpoint{3.960988in}{1.527803in}}%
\pgfpathlineto{\pgfqpoint{3.961248in}{1.682448in}}%
\pgfpathlineto{\pgfqpoint{3.961509in}{1.479654in}}%
\pgfpathlineto{\pgfqpoint{3.962028in}{1.513469in}}%
\pgfpathlineto{\pgfqpoint{3.962288in}{1.434860in}}%
\pgfpathlineto{\pgfqpoint{3.962547in}{1.641111in}}%
\pgfpathlineto{\pgfqpoint{3.962807in}{1.537542in}}%
\pgfpathlineto{\pgfqpoint{3.963325in}{1.655303in}}%
\pgfpathlineto{\pgfqpoint{3.963584in}{1.567647in}}%
\pgfpathlineto{\pgfqpoint{3.963842in}{1.320996in}}%
\pgfpathlineto{\pgfqpoint{3.964101in}{1.676133in}}%
\pgfpathlineto{\pgfqpoint{3.964617in}{1.534988in}}%
\pgfpathlineto{\pgfqpoint{3.964876in}{1.506470in}}%
\pgfpathlineto{\pgfqpoint{3.965133in}{1.585274in}}%
\pgfpathlineto{\pgfqpoint{3.965391in}{1.578412in}}%
\pgfpathlineto{\pgfqpoint{3.965649in}{1.632549in}}%
\pgfpathlineto{\pgfqpoint{3.965906in}{1.437097in}}%
\pgfpathlineto{\pgfqpoint{3.966678in}{1.567769in}}%
\pgfpathlineto{\pgfqpoint{3.967447in}{1.518299in}}%
\pgfpathlineto{\pgfqpoint{3.967191in}{1.591277in}}%
\pgfpathlineto{\pgfqpoint{3.967704in}{1.560072in}}%
\pgfpathlineto{\pgfqpoint{3.967960in}{1.565991in}}%
\pgfpathlineto{\pgfqpoint{3.968216in}{1.328360in}}%
\pgfpathlineto{\pgfqpoint{3.968728in}{1.699729in}}%
\pgfpathlineto{\pgfqpoint{3.968983in}{1.624625in}}%
\pgfpathlineto{\pgfqpoint{3.969239in}{1.501753in}}%
\pgfpathlineto{\pgfqpoint{3.969494in}{1.656154in}}%
\pgfpathlineto{\pgfqpoint{3.969749in}{1.566951in}}%
\pgfpathlineto{\pgfqpoint{3.970004in}{1.672359in}}%
\pgfpathlineto{\pgfqpoint{3.970513in}{1.653317in}}%
\pgfpathlineto{\pgfqpoint{3.971784in}{1.297955in}}%
\pgfpathlineto{\pgfqpoint{3.972798in}{1.647762in}}%
\pgfpathlineto{\pgfqpoint{3.973051in}{1.568070in}}%
\pgfpathlineto{\pgfqpoint{3.973304in}{1.385584in}}%
\pgfpathlineto{\pgfqpoint{3.973557in}{1.708827in}}%
\pgfpathlineto{\pgfqpoint{3.973810in}{1.415894in}}%
\pgfpathlineto{\pgfqpoint{3.974062in}{1.734315in}}%
\pgfpathlineto{\pgfqpoint{3.974819in}{1.566821in}}%
\pgfpathlineto{\pgfqpoint{3.976077in}{1.172465in}}%
\pgfpathlineto{\pgfqpoint{3.976579in}{1.671395in}}%
\pgfpathlineto{\pgfqpoint{3.977332in}{1.535964in}}%
\pgfpathlineto{\pgfqpoint{3.977582in}{1.548658in}}%
\pgfpathlineto{\pgfqpoint{3.978082in}{1.458861in}}%
\pgfpathlineto{\pgfqpoint{3.978332in}{1.550128in}}%
\pgfpathlineto{\pgfqpoint{3.979081in}{1.711520in}}%
\pgfpathlineto{\pgfqpoint{3.979580in}{1.627372in}}%
\pgfpathlineto{\pgfqpoint{3.979829in}{1.294087in}}%
\pgfpathlineto{\pgfqpoint{3.980327in}{1.674805in}}%
\pgfpathlineto{\pgfqpoint{3.980576in}{1.628142in}}%
\pgfpathlineto{\pgfqpoint{3.980824in}{1.588778in}}%
\pgfpathlineto{\pgfqpoint{3.981073in}{1.677061in}}%
\pgfpathlineto{\pgfqpoint{3.981321in}{1.634337in}}%
\pgfpathlineto{\pgfqpoint{3.981569in}{1.693029in}}%
\pgfpathlineto{\pgfqpoint{3.981817in}{1.634134in}}%
\pgfpathlineto{\pgfqpoint{3.982065in}{1.661012in}}%
\pgfpathlineto{\pgfqpoint{3.983302in}{1.502698in}}%
\pgfpathlineto{\pgfqpoint{3.984042in}{1.395005in}}%
\pgfpathlineto{\pgfqpoint{3.984781in}{1.717987in}}%
\pgfpathlineto{\pgfqpoint{3.986010in}{1.419188in}}%
\pgfpathlineto{\pgfqpoint{3.986255in}{1.706289in}}%
\pgfpathlineto{\pgfqpoint{3.986501in}{1.303344in}}%
\pgfpathlineto{\pgfqpoint{3.987235in}{1.592247in}}%
\pgfpathlineto{\pgfqpoint{3.987969in}{1.521702in}}%
\pgfpathlineto{\pgfqpoint{3.988457in}{1.691578in}}%
\pgfpathlineto{\pgfqpoint{3.988945in}{1.566862in}}%
\pgfpathlineto{\pgfqpoint{3.989189in}{1.349132in}}%
\pgfpathlineto{\pgfqpoint{3.989676in}{1.625072in}}%
\pgfpathlineto{\pgfqpoint{3.989919in}{1.722088in}}%
\pgfpathlineto{\pgfqpoint{3.990162in}{1.464040in}}%
\pgfpathlineto{\pgfqpoint{3.990891in}{1.682476in}}%
\pgfpathlineto{\pgfqpoint{3.991376in}{1.383693in}}%
\pgfpathlineto{\pgfqpoint{3.991860in}{1.686210in}}%
\pgfpathlineto{\pgfqpoint{3.993310in}{1.518834in}}%
\pgfpathlineto{\pgfqpoint{3.993792in}{1.647410in}}%
\pgfpathlineto{\pgfqpoint{3.994033in}{1.355531in}}%
\pgfpathlineto{\pgfqpoint{3.994274in}{1.651863in}}%
\pgfpathlineto{\pgfqpoint{3.994755in}{1.575237in}}%
\pgfpathlineto{\pgfqpoint{3.994995in}{1.579235in}}%
\pgfpathlineto{\pgfqpoint{3.995236in}{1.440670in}}%
\pgfpathlineto{\pgfqpoint{3.995955in}{1.640927in}}%
\pgfpathlineto{\pgfqpoint{3.996435in}{1.544490in}}%
\pgfpathlineto{\pgfqpoint{3.996913in}{1.649165in}}%
\pgfpathlineto{\pgfqpoint{3.997630in}{1.570054in}}%
\pgfpathlineto{\pgfqpoint{3.997869in}{1.590109in}}%
\pgfpathlineto{\pgfqpoint{3.998346in}{1.436454in}}%
\pgfpathlineto{\pgfqpoint{3.998585in}{1.508376in}}%
\pgfpathlineto{\pgfqpoint{3.999299in}{1.660427in}}%
\pgfpathlineto{\pgfqpoint{3.999537in}{1.614511in}}%
\pgfpathlineto{\pgfqpoint{3.999774in}{1.439306in}}%
\pgfpathlineto{\pgfqpoint{4.000249in}{1.615245in}}%
\pgfpathlineto{\pgfqpoint{4.000487in}{1.569256in}}%
\pgfpathlineto{\pgfqpoint{4.000961in}{1.412831in}}%
\pgfpathlineto{\pgfqpoint{4.001434in}{1.677859in}}%
\pgfpathlineto{\pgfqpoint{4.002380in}{1.458328in}}%
\pgfpathlineto{\pgfqpoint{4.002616in}{1.506633in}}%
\pgfpathlineto{\pgfqpoint{4.003088in}{1.650752in}}%
\pgfpathlineto{\pgfqpoint{4.003324in}{1.490208in}}%
\pgfpathlineto{\pgfqpoint{4.003559in}{1.465662in}}%
\pgfpathlineto{\pgfqpoint{4.003794in}{1.482774in}}%
\pgfpathlineto{\pgfqpoint{4.004735in}{1.651553in}}%
\pgfpathlineto{\pgfqpoint{4.005204in}{1.641155in}}%
\pgfpathlineto{\pgfqpoint{4.005908in}{1.475244in}}%
\pgfpathlineto{\pgfqpoint{4.006142in}{1.570184in}}%
\pgfpathlineto{\pgfqpoint{4.006376in}{1.702960in}}%
\pgfpathlineto{\pgfqpoint{4.006843in}{1.530190in}}%
\pgfpathlineto{\pgfqpoint{4.007077in}{1.617565in}}%
\pgfpathlineto{\pgfqpoint{4.007544in}{1.652179in}}%
\pgfpathlineto{\pgfqpoint{4.008476in}{1.389658in}}%
\pgfpathlineto{\pgfqpoint{4.009174in}{1.670427in}}%
\pgfpathlineto{\pgfqpoint{4.008941in}{1.286077in}}%
\pgfpathlineto{\pgfqpoint{4.009638in}{1.494835in}}%
\pgfpathlineto{\pgfqpoint{4.009870in}{1.503785in}}%
\pgfpathlineto{\pgfqpoint{4.010334in}{1.638841in}}%
\pgfpathlineto{\pgfqpoint{4.010566in}{1.403737in}}%
\pgfpathlineto{\pgfqpoint{4.010797in}{1.368742in}}%
\pgfpathlineto{\pgfqpoint{4.011029in}{1.699498in}}%
\pgfpathlineto{\pgfqpoint{4.011953in}{1.444760in}}%
\pgfpathlineto{\pgfqpoint{4.013566in}{1.737351in}}%
\pgfpathlineto{\pgfqpoint{4.014026in}{1.505456in}}%
\pgfpathlineto{\pgfqpoint{4.014485in}{1.579116in}}%
\pgfpathlineto{\pgfqpoint{4.014715in}{1.783674in}}%
\pgfpathlineto{\pgfqpoint{4.015173in}{1.572377in}}%
\pgfpathlineto{\pgfqpoint{4.015403in}{1.346174in}}%
\pgfpathlineto{\pgfqpoint{4.015631in}{1.651844in}}%
\pgfpathlineto{\pgfqpoint{4.016089in}{1.623754in}}%
\pgfpathlineto{\pgfqpoint{4.017003in}{1.488338in}}%
\pgfpathlineto{\pgfqpoint{4.017231in}{1.592413in}}%
\pgfpathlineto{\pgfqpoint{4.017686in}{1.686583in}}%
\pgfpathlineto{\pgfqpoint{4.017914in}{1.609554in}}%
\pgfpathlineto{\pgfqpoint{4.018824in}{1.486259in}}%
\pgfpathlineto{\pgfqpoint{4.019051in}{1.513918in}}%
\pgfpathlineto{\pgfqpoint{4.019278in}{1.669534in}}%
\pgfpathlineto{\pgfqpoint{4.019505in}{1.389846in}}%
\pgfpathlineto{\pgfqpoint{4.020185in}{1.567655in}}%
\pgfpathlineto{\pgfqpoint{4.020411in}{1.513995in}}%
\pgfpathlineto{\pgfqpoint{4.020637in}{1.648721in}}%
\pgfpathlineto{\pgfqpoint{4.020864in}{1.709410in}}%
\pgfpathlineto{\pgfqpoint{4.021090in}{1.514410in}}%
\pgfpathlineto{\pgfqpoint{4.021315in}{1.522923in}}%
\pgfpathlineto{\pgfqpoint{4.021541in}{1.633167in}}%
\pgfpathlineto{\pgfqpoint{4.021767in}{1.489188in}}%
\pgfpathlineto{\pgfqpoint{4.022443in}{1.536437in}}%
\pgfpathlineto{\pgfqpoint{4.022668in}{1.431687in}}%
\pgfpathlineto{\pgfqpoint{4.022893in}{1.721526in}}%
\pgfpathlineto{\pgfqpoint{4.023118in}{1.571802in}}%
\pgfpathlineto{\pgfqpoint{4.023793in}{1.702729in}}%
\pgfpathlineto{\pgfqpoint{4.023568in}{1.547689in}}%
\pgfpathlineto{\pgfqpoint{4.024241in}{1.653925in}}%
\pgfpathlineto{\pgfqpoint{4.024914in}{1.674831in}}%
\pgfpathlineto{\pgfqpoint{4.025361in}{1.574463in}}%
\pgfpathlineto{\pgfqpoint{4.025585in}{1.597979in}}%
\pgfpathlineto{\pgfqpoint{4.025809in}{1.827866in}}%
\pgfpathlineto{\pgfqpoint{4.026478in}{1.618923in}}%
\pgfpathlineto{\pgfqpoint{4.027147in}{1.470829in}}%
\pgfpathlineto{\pgfqpoint{4.026924in}{1.709821in}}%
\pgfpathlineto{\pgfqpoint{4.027592in}{1.481452in}}%
\pgfpathlineto{\pgfqpoint{4.029591in}{1.776369in}}%
\pgfpathlineto{\pgfqpoint{4.030697in}{1.347938in}}%
\pgfpathlineto{\pgfqpoint{4.030917in}{1.592302in}}%
\pgfpathlineto{\pgfqpoint{4.031138in}{1.708543in}}%
\pgfpathlineto{\pgfqpoint{4.031359in}{1.480211in}}%
\pgfpathlineto{\pgfqpoint{4.031800in}{1.558600in}}%
\pgfpathlineto{\pgfqpoint{4.032020in}{1.542450in}}%
\pgfpathlineto{\pgfqpoint{4.032240in}{1.552712in}}%
\pgfpathlineto{\pgfqpoint{4.032900in}{1.667550in}}%
\pgfpathlineto{\pgfqpoint{4.033120in}{1.628238in}}%
\pgfpathlineto{\pgfqpoint{4.033997in}{1.401776in}}%
\pgfpathlineto{\pgfqpoint{4.033778in}{1.703818in}}%
\pgfpathlineto{\pgfqpoint{4.034216in}{1.402402in}}%
\pgfpathlineto{\pgfqpoint{4.034435in}{1.686974in}}%
\pgfpathlineto{\pgfqpoint{4.035310in}{1.630117in}}%
\pgfpathlineto{\pgfqpoint{4.035965in}{1.744430in}}%
\pgfpathlineto{\pgfqpoint{4.036402in}{1.417610in}}%
\pgfpathlineto{\pgfqpoint{4.037055in}{1.724617in}}%
\pgfpathlineto{\pgfqpoint{4.037273in}{1.508707in}}%
\pgfpathlineto{\pgfqpoint{4.037490in}{1.348790in}}%
\pgfpathlineto{\pgfqpoint{4.037925in}{1.685928in}}%
\pgfpathlineto{\pgfqpoint{4.038142in}{1.629356in}}%
\pgfpathlineto{\pgfqpoint{4.038359in}{1.633156in}}%
\pgfpathlineto{\pgfqpoint{4.038792in}{1.546847in}}%
\pgfpathlineto{\pgfqpoint{4.039226in}{1.689203in}}%
\pgfpathlineto{\pgfqpoint{4.039442in}{1.690884in}}%
\pgfpathlineto{\pgfqpoint{4.040523in}{1.495916in}}%
\pgfpathlineto{\pgfqpoint{4.040738in}{1.603366in}}%
\pgfpathlineto{\pgfqpoint{4.041816in}{1.722198in}}%
\pgfpathlineto{\pgfqpoint{4.041385in}{1.433711in}}%
\pgfpathlineto{\pgfqpoint{4.042031in}{1.681484in}}%
\pgfpathlineto{\pgfqpoint{4.042461in}{1.792388in}}%
\pgfpathlineto{\pgfqpoint{4.043320in}{1.417049in}}%
\pgfpathlineto{\pgfqpoint{4.044176in}{1.730512in}}%
\pgfpathlineto{\pgfqpoint{4.044390in}{1.201844in}}%
\pgfpathlineto{\pgfqpoint{4.045245in}{1.753273in}}%
\pgfpathlineto{\pgfqpoint{4.046524in}{0.935782in}}%
\pgfpathlineto{\pgfqpoint{4.047162in}{1.707116in}}%
\pgfpathlineto{\pgfqpoint{4.047799in}{1.625375in}}%
\pgfpathlineto{\pgfqpoint{4.048223in}{1.487193in}}%
\pgfpathlineto{\pgfqpoint{4.049282in}{1.535239in}}%
\pgfpathlineto{\pgfqpoint{4.050337in}{1.692912in}}%
\pgfpathlineto{\pgfqpoint{4.050548in}{1.644997in}}%
\pgfpathlineto{\pgfqpoint{4.051391in}{1.676844in}}%
\pgfpathlineto{\pgfqpoint{4.051601in}{1.599602in}}%
\pgfpathlineto{\pgfqpoint{4.051811in}{1.691757in}}%
\pgfpathlineto{\pgfqpoint{4.052441in}{1.573795in}}%
\pgfpathlineto{\pgfqpoint{4.052861in}{1.675297in}}%
\pgfpathlineto{\pgfqpoint{4.053489in}{1.535733in}}%
\pgfpathlineto{\pgfqpoint{4.053908in}{1.603093in}}%
\pgfpathlineto{\pgfqpoint{4.054744in}{1.769724in}}%
\pgfpathlineto{\pgfqpoint{4.055161in}{1.741104in}}%
\pgfpathlineto{\pgfqpoint{4.056618in}{1.352448in}}%
\pgfpathlineto{\pgfqpoint{4.057034in}{1.719038in}}%
\pgfpathlineto{\pgfqpoint{4.057656in}{1.632148in}}%
\pgfpathlineto{\pgfqpoint{4.058484in}{1.429236in}}%
\pgfpathlineto{\pgfqpoint{4.059105in}{1.753257in}}%
\pgfpathlineto{\pgfqpoint{4.059724in}{1.659264in}}%
\pgfpathlineto{\pgfqpoint{4.059930in}{1.487434in}}%
\pgfpathlineto{\pgfqpoint{4.060343in}{1.716623in}}%
\pgfpathlineto{\pgfqpoint{4.060754in}{1.510001in}}%
\pgfpathlineto{\pgfqpoint{4.061782in}{1.706857in}}%
\pgfpathlineto{\pgfqpoint{4.061987in}{1.634055in}}%
\pgfpathlineto{\pgfqpoint{4.062193in}{1.553229in}}%
\pgfpathlineto{\pgfqpoint{4.062603in}{1.713943in}}%
\pgfpathlineto{\pgfqpoint{4.063012in}{1.609454in}}%
\pgfpathlineto{\pgfqpoint{4.063830in}{1.671504in}}%
\pgfpathlineto{\pgfqpoint{4.064239in}{1.640977in}}%
\pgfpathlineto{\pgfqpoint{4.064647in}{1.684538in}}%
\pgfpathlineto{\pgfqpoint{4.065665in}{1.564478in}}%
\pgfpathlineto{\pgfqpoint{4.065868in}{1.628656in}}%
\pgfpathlineto{\pgfqpoint{4.066275in}{1.557862in}}%
\pgfpathlineto{\pgfqpoint{4.066681in}{1.598875in}}%
\pgfpathlineto{\pgfqpoint{4.067289in}{1.390740in}}%
\pgfpathlineto{\pgfqpoint{4.067087in}{1.699679in}}%
\pgfpathlineto{\pgfqpoint{4.067695in}{1.590455in}}%
\pgfpathlineto{\pgfqpoint{4.068301in}{1.685047in}}%
\pgfpathlineto{\pgfqpoint{4.068908in}{1.306607in}}%
\pgfpathlineto{\pgfqpoint{4.069109in}{1.687171in}}%
\pgfpathlineto{\pgfqpoint{4.070117in}{1.516111in}}%
\pgfpathlineto{\pgfqpoint{4.070721in}{1.703435in}}%
\pgfpathlineto{\pgfqpoint{4.070922in}{1.627392in}}%
\pgfpathlineto{\pgfqpoint{4.071123in}{1.424257in}}%
\pgfpathlineto{\pgfqpoint{4.071725in}{1.652968in}}%
\pgfpathlineto{\pgfqpoint{4.072125in}{1.454306in}}%
\pgfpathlineto{\pgfqpoint{4.073126in}{1.713317in}}%
\pgfpathlineto{\pgfqpoint{4.073326in}{1.394833in}}%
\pgfpathlineto{\pgfqpoint{4.074124in}{1.531715in}}%
\pgfpathlineto{\pgfqpoint{4.074921in}{1.689499in}}%
\pgfpathlineto{\pgfqpoint{4.074722in}{1.487804in}}%
\pgfpathlineto{\pgfqpoint{4.075120in}{1.621659in}}%
\pgfpathlineto{\pgfqpoint{4.075915in}{1.459071in}}%
\pgfpathlineto{\pgfqpoint{4.075518in}{1.729520in}}%
\pgfpathlineto{\pgfqpoint{4.076114in}{1.579022in}}%
\pgfpathlineto{\pgfqpoint{4.076709in}{1.714723in}}%
\pgfpathlineto{\pgfqpoint{4.077105in}{1.574317in}}%
\pgfpathlineto{\pgfqpoint{4.077303in}{1.409814in}}%
\pgfpathlineto{\pgfqpoint{4.077699in}{1.707575in}}%
\pgfpathlineto{\pgfqpoint{4.078094in}{1.648795in}}%
\pgfpathlineto{\pgfqpoint{4.078489in}{1.665613in}}%
\pgfpathlineto{\pgfqpoint{4.078686in}{1.648207in}}%
\pgfpathlineto{\pgfqpoint{4.079081in}{1.666115in}}%
\pgfpathlineto{\pgfqpoint{4.079672in}{1.445726in}}%
\pgfpathlineto{\pgfqpoint{4.080655in}{1.720267in}}%
\pgfpathlineto{\pgfqpoint{4.080458in}{1.381845in}}%
\pgfpathlineto{\pgfqpoint{4.080851in}{1.694118in}}%
\pgfpathlineto{\pgfqpoint{4.081047in}{1.684805in}}%
\pgfpathlineto{\pgfqpoint{4.081243in}{1.692686in}}%
\pgfpathlineto{\pgfqpoint{4.081439in}{1.762216in}}%
\pgfpathlineto{\pgfqpoint{4.081831in}{1.556512in}}%
\pgfpathlineto{\pgfqpoint{4.082027in}{1.716786in}}%
\pgfpathlineto{\pgfqpoint{4.082418in}{1.305903in}}%
\pgfpathlineto{\pgfqpoint{4.083005in}{1.512951in}}%
\pgfpathlineto{\pgfqpoint{4.083200in}{1.684629in}}%
\pgfpathlineto{\pgfqpoint{4.083980in}{1.623577in}}%
\pgfpathlineto{\pgfqpoint{4.084369in}{1.689654in}}%
\pgfpathlineto{\pgfqpoint{4.084953in}{1.493242in}}%
\pgfpathlineto{\pgfqpoint{4.085730in}{1.652232in}}%
\pgfpathlineto{\pgfqpoint{4.085342in}{1.490978in}}%
\pgfpathlineto{\pgfqpoint{4.085924in}{1.566097in}}%
\pgfpathlineto{\pgfqpoint{4.086118in}{0.924288in}}%
\pgfpathlineto{\pgfqpoint{4.086699in}{1.719449in}}%
\pgfpathlineto{\pgfqpoint{4.086893in}{1.641910in}}%
\pgfpathlineto{\pgfqpoint{4.087086in}{1.759624in}}%
\pgfpathlineto{\pgfqpoint{4.087666in}{1.594896in}}%
\pgfpathlineto{\pgfqpoint{4.088052in}{1.684250in}}%
\pgfpathlineto{\pgfqpoint{4.088631in}{1.514084in}}%
\pgfpathlineto{\pgfqpoint{4.088823in}{1.689958in}}%
\pgfpathlineto{\pgfqpoint{4.089208in}{1.813751in}}%
\pgfpathlineto{\pgfqpoint{4.089401in}{1.570296in}}%
\pgfpathlineto{\pgfqpoint{4.089785in}{1.695297in}}%
\pgfpathlineto{\pgfqpoint{4.090553in}{1.699290in}}%
\pgfpathlineto{\pgfqpoint{4.091320in}{1.526554in}}%
\pgfpathlineto{\pgfqpoint{4.092276in}{1.510816in}}%
\pgfpathlineto{\pgfqpoint{4.092658in}{1.736709in}}%
\pgfpathlineto{\pgfqpoint{4.092849in}{1.742647in}}%
\pgfpathlineto{\pgfqpoint{4.093421in}{1.543178in}}%
\pgfpathlineto{\pgfqpoint{4.093993in}{1.704359in}}%
\pgfpathlineto{\pgfqpoint{4.094183in}{1.701458in}}%
\pgfpathlineto{\pgfqpoint{4.094943in}{1.521114in}}%
\pgfpathlineto{\pgfqpoint{4.095323in}{1.675993in}}%
\pgfpathlineto{\pgfqpoint{4.095512in}{1.675321in}}%
\pgfpathlineto{\pgfqpoint{4.095891in}{1.695551in}}%
\pgfpathlineto{\pgfqpoint{4.096270in}{1.473465in}}%
\pgfpathlineto{\pgfqpoint{4.096459in}{1.706685in}}%
\pgfpathlineto{\pgfqpoint{4.097027in}{1.410264in}}%
\pgfpathlineto{\pgfqpoint{4.097404in}{1.612192in}}%
\pgfpathlineto{\pgfqpoint{4.097593in}{1.408417in}}%
\pgfpathlineto{\pgfqpoint{4.097970in}{1.733198in}}%
\pgfpathlineto{\pgfqpoint{4.098347in}{1.507421in}}%
\pgfpathlineto{\pgfqpoint{4.099100in}{1.683621in}}%
\pgfpathlineto{\pgfqpoint{4.099476in}{1.577466in}}%
\pgfpathlineto{\pgfqpoint{4.099851in}{1.634630in}}%
\pgfpathlineto{\pgfqpoint{4.100039in}{1.492575in}}%
\pgfpathlineto{\pgfqpoint{4.100601in}{1.637638in}}%
\pgfpathlineto{\pgfqpoint{4.100976in}{1.542085in}}%
\pgfpathlineto{\pgfqpoint{4.101537in}{1.722666in}}%
\pgfpathlineto{\pgfqpoint{4.101911in}{1.520464in}}%
\pgfpathlineto{\pgfqpoint{4.102098in}{1.505029in}}%
\pgfpathlineto{\pgfqpoint{4.103402in}{1.719157in}}%
\pgfpathlineto{\pgfqpoint{4.103588in}{1.507946in}}%
\pgfpathlineto{\pgfqpoint{4.104518in}{1.639060in}}%
\pgfpathlineto{\pgfqpoint{4.105074in}{1.538282in}}%
\pgfpathlineto{\pgfqpoint{4.105260in}{1.635364in}}%
\pgfpathlineto{\pgfqpoint{4.105445in}{1.699804in}}%
\pgfpathlineto{\pgfqpoint{4.105815in}{1.520056in}}%
\pgfpathlineto{\pgfqpoint{4.106000in}{1.696659in}}%
\pgfpathlineto{\pgfqpoint{4.106185in}{1.473576in}}%
\pgfpathlineto{\pgfqpoint{4.107109in}{1.642912in}}%
\pgfpathlineto{\pgfqpoint{4.107293in}{1.686804in}}%
\pgfpathlineto{\pgfqpoint{4.107477in}{1.616941in}}%
\pgfpathlineto{\pgfqpoint{4.107662in}{1.653934in}}%
\pgfpathlineto{\pgfqpoint{4.108030in}{1.523239in}}%
\pgfpathlineto{\pgfqpoint{4.108214in}{1.687827in}}%
\pgfpathlineto{\pgfqpoint{4.108582in}{1.677859in}}%
\pgfpathlineto{\pgfqpoint{4.108766in}{1.727056in}}%
\pgfpathlineto{\pgfqpoint{4.108950in}{1.687225in}}%
\pgfpathlineto{\pgfqpoint{4.110234in}{1.288497in}}%
\pgfpathlineto{\pgfqpoint{4.110783in}{1.790097in}}%
\pgfpathlineto{\pgfqpoint{4.111514in}{1.723786in}}%
\pgfpathlineto{\pgfqpoint{4.112608in}{1.373092in}}%
\pgfpathlineto{\pgfqpoint{4.112790in}{1.586119in}}%
\pgfpathlineto{\pgfqpoint{4.112972in}{1.475691in}}%
\pgfpathlineto{\pgfqpoint{4.113336in}{1.702525in}}%
\pgfpathlineto{\pgfqpoint{4.113881in}{1.590644in}}%
\pgfpathlineto{\pgfqpoint{4.114425in}{1.562042in}}%
\pgfpathlineto{\pgfqpoint{4.114607in}{1.621238in}}%
\pgfpathlineto{\pgfqpoint{4.115512in}{1.753233in}}%
\pgfpathlineto{\pgfqpoint{4.114969in}{1.433167in}}%
\pgfpathlineto{\pgfqpoint{4.115693in}{1.645666in}}%
\pgfpathlineto{\pgfqpoint{4.116596in}{1.748092in}}%
\pgfpathlineto{\pgfqpoint{4.116776in}{1.641451in}}%
\pgfpathlineto{\pgfqpoint{4.116957in}{1.728231in}}%
\pgfpathlineto{\pgfqpoint{4.118037in}{1.374014in}}%
\pgfpathlineto{\pgfqpoint{4.118396in}{1.478795in}}%
\pgfpathlineto{\pgfqpoint{4.119294in}{1.724888in}}%
\pgfpathlineto{\pgfqpoint{4.119473in}{1.538680in}}%
\pgfpathlineto{\pgfqpoint{4.119652in}{1.623014in}}%
\pgfpathlineto{\pgfqpoint{4.119831in}{1.380590in}}%
\pgfpathlineto{\pgfqpoint{4.120010in}{1.617240in}}%
\pgfpathlineto{\pgfqpoint{4.120189in}{1.290549in}}%
\pgfpathlineto{\pgfqpoint{4.120904in}{1.709015in}}%
\pgfpathlineto{\pgfqpoint{4.121083in}{1.627781in}}%
\pgfpathlineto{\pgfqpoint{4.121440in}{1.756400in}}%
\pgfpathlineto{\pgfqpoint{4.121975in}{1.574604in}}%
\pgfpathlineto{\pgfqpoint{4.122331in}{1.707307in}}%
\pgfpathlineto{\pgfqpoint{4.123220in}{1.522499in}}%
\pgfpathlineto{\pgfqpoint{4.123042in}{1.719715in}}%
\pgfpathlineto{\pgfqpoint{4.123397in}{1.673174in}}%
\pgfpathlineto{\pgfqpoint{4.123575in}{1.671038in}}%
\pgfpathlineto{\pgfqpoint{4.123752in}{1.572778in}}%
\pgfpathlineto{\pgfqpoint{4.124107in}{1.681556in}}%
\pgfpathlineto{\pgfqpoint{4.124639in}{1.667986in}}%
\pgfpathlineto{\pgfqpoint{4.124816in}{1.652986in}}%
\pgfpathlineto{\pgfqpoint{4.124992in}{1.666829in}}%
\pgfpathlineto{\pgfqpoint{4.125169in}{1.404247in}}%
\pgfpathlineto{\pgfqpoint{4.125523in}{1.764184in}}%
\pgfpathlineto{\pgfqpoint{4.126053in}{1.639341in}}%
\pgfpathlineto{\pgfqpoint{4.126758in}{1.492374in}}%
\pgfpathlineto{\pgfqpoint{4.127110in}{1.707723in}}%
\pgfpathlineto{\pgfqpoint{4.127989in}{1.558204in}}%
\pgfpathlineto{\pgfqpoint{4.127638in}{1.727305in}}%
\pgfpathlineto{\pgfqpoint{4.128165in}{1.663484in}}%
\pgfpathlineto{\pgfqpoint{4.128866in}{1.761319in}}%
\pgfpathlineto{\pgfqpoint{4.129042in}{1.646314in}}%
\pgfpathlineto{\pgfqpoint{4.129217in}{1.494189in}}%
\pgfpathlineto{\pgfqpoint{4.129742in}{1.730429in}}%
\pgfpathlineto{\pgfqpoint{4.130092in}{1.513040in}}%
\pgfpathlineto{\pgfqpoint{4.130616in}{1.652304in}}%
\pgfpathlineto{\pgfqpoint{4.131313in}{1.549422in}}%
\pgfpathlineto{\pgfqpoint{4.131488in}{1.494208in}}%
\pgfpathlineto{\pgfqpoint{4.131662in}{1.654681in}}%
\pgfpathlineto{\pgfqpoint{4.132010in}{1.535277in}}%
\pgfpathlineto{\pgfqpoint{4.132531in}{1.757108in}}%
\pgfpathlineto{\pgfqpoint{4.132705in}{1.440820in}}%
\pgfpathlineto{\pgfqpoint{4.133226in}{1.761940in}}%
\pgfpathlineto{\pgfqpoint{4.133573in}{1.581663in}}%
\pgfpathlineto{\pgfqpoint{4.134093in}{1.758790in}}%
\pgfpathlineto{\pgfqpoint{4.134266in}{1.696045in}}%
\pgfpathlineto{\pgfqpoint{4.134612in}{1.387819in}}%
\pgfpathlineto{\pgfqpoint{4.135130in}{1.789619in}}%
\pgfpathlineto{\pgfqpoint{4.135303in}{1.568344in}}%
\pgfpathlineto{\pgfqpoint{4.135648in}{1.685932in}}%
\pgfpathlineto{\pgfqpoint{4.135820in}{1.575778in}}%
\pgfpathlineto{\pgfqpoint{4.135993in}{1.292217in}}%
\pgfpathlineto{\pgfqpoint{4.136854in}{1.674588in}}%
\pgfpathlineto{\pgfqpoint{4.137885in}{1.563401in}}%
\pgfpathlineto{\pgfqpoint{4.137198in}{1.684160in}}%
\pgfpathlineto{\pgfqpoint{4.138056in}{1.640332in}}%
\pgfpathlineto{\pgfqpoint{4.138228in}{1.756608in}}%
\pgfpathlineto{\pgfqpoint{4.138742in}{1.566638in}}%
\pgfpathlineto{\pgfqpoint{4.138913in}{1.710575in}}%
\pgfpathlineto{\pgfqpoint{4.139768in}{1.370911in}}%
\pgfpathlineto{\pgfqpoint{4.139597in}{1.719627in}}%
\pgfpathlineto{\pgfqpoint{4.140110in}{1.517858in}}%
\pgfpathlineto{\pgfqpoint{4.140621in}{1.733253in}}%
\pgfpathlineto{\pgfqpoint{4.141303in}{1.656412in}}%
\pgfpathlineto{\pgfqpoint{4.141473in}{1.296533in}}%
\pgfpathlineto{\pgfqpoint{4.142153in}{1.754716in}}%
\pgfpathlineto{\pgfqpoint{4.142323in}{1.589319in}}%
\pgfpathlineto{\pgfqpoint{4.142493in}{1.691909in}}%
\pgfpathlineto{\pgfqpoint{4.142832in}{1.536620in}}%
\pgfpathlineto{\pgfqpoint{4.143341in}{1.573790in}}%
\pgfpathlineto{\pgfqpoint{4.143848in}{1.689127in}}%
\pgfpathlineto{\pgfqpoint{4.144018in}{1.500746in}}%
\pgfpathlineto{\pgfqpoint{4.144356in}{1.638927in}}%
\pgfpathlineto{\pgfqpoint{4.144525in}{1.554695in}}%
\pgfpathlineto{\pgfqpoint{4.145200in}{1.681340in}}%
\pgfpathlineto{\pgfqpoint{4.145537in}{1.573759in}}%
\pgfpathlineto{\pgfqpoint{4.145874in}{1.735165in}}%
\pgfpathlineto{\pgfqpoint{4.146042in}{1.561204in}}%
\pgfpathlineto{\pgfqpoint{4.146715in}{1.627498in}}%
\pgfpathlineto{\pgfqpoint{4.146883in}{1.488837in}}%
\pgfpathlineto{\pgfqpoint{4.147387in}{1.690909in}}%
\pgfpathlineto{\pgfqpoint{4.147722in}{1.657920in}}%
\pgfpathlineto{\pgfqpoint{4.147890in}{1.676563in}}%
\pgfpathlineto{\pgfqpoint{4.148058in}{1.547866in}}%
\pgfpathlineto{\pgfqpoint{4.148560in}{1.761618in}}%
\pgfpathlineto{\pgfqpoint{4.148894in}{1.698866in}}%
\pgfpathlineto{\pgfqpoint{4.149229in}{1.795081in}}%
\pgfpathlineto{\pgfqpoint{4.149396in}{1.655251in}}%
\pgfpathlineto{\pgfqpoint{4.149730in}{1.723205in}}%
\pgfpathlineto{\pgfqpoint{4.150730in}{1.533157in}}%
\pgfpathlineto{\pgfqpoint{4.150896in}{1.720716in}}%
\pgfpathlineto{\pgfqpoint{4.151561in}{1.495751in}}%
\pgfpathlineto{\pgfqpoint{4.151893in}{1.677519in}}%
\pgfpathlineto{\pgfqpoint{4.152059in}{1.716836in}}%
\pgfpathlineto{\pgfqpoint{4.152391in}{1.567127in}}%
\pgfpathlineto{\pgfqpoint{4.152557in}{1.713734in}}%
\pgfpathlineto{\pgfqpoint{4.152723in}{1.524813in}}%
\pgfpathlineto{\pgfqpoint{4.153385in}{1.750780in}}%
\pgfpathlineto{\pgfqpoint{4.153716in}{1.619207in}}%
\pgfpathlineto{\pgfqpoint{4.153881in}{1.615055in}}%
\pgfpathlineto{\pgfqpoint{4.154212in}{1.708002in}}%
\pgfpathlineto{\pgfqpoint{4.154377in}{1.561978in}}%
\pgfpathlineto{\pgfqpoint{4.154707in}{1.701796in}}%
\pgfpathlineto{\pgfqpoint{4.154871in}{1.235022in}}%
\pgfpathlineto{\pgfqpoint{4.155695in}{1.603734in}}%
\pgfpathlineto{\pgfqpoint{4.156681in}{1.710498in}}%
\pgfpathlineto{\pgfqpoint{4.156845in}{1.644569in}}%
\pgfpathlineto{\pgfqpoint{4.157173in}{1.650935in}}%
\pgfpathlineto{\pgfqpoint{4.157501in}{1.621341in}}%
\pgfpathlineto{\pgfqpoint{4.158320in}{1.725820in}}%
\pgfpathlineto{\pgfqpoint{4.157829in}{1.617380in}}%
\pgfpathlineto{\pgfqpoint{4.158483in}{1.686290in}}%
\pgfpathlineto{\pgfqpoint{4.159300in}{1.760494in}}%
\pgfpathlineto{\pgfqpoint{4.159463in}{1.495561in}}%
\pgfpathlineto{\pgfqpoint{4.160440in}{1.772927in}}%
\pgfpathlineto{\pgfqpoint{4.160603in}{1.704369in}}%
\pgfpathlineto{\pgfqpoint{4.161253in}{1.771207in}}%
\pgfpathlineto{\pgfqpoint{4.161578in}{1.404158in}}%
\pgfpathlineto{\pgfqpoint{4.162550in}{1.788990in}}%
\pgfpathlineto{\pgfqpoint{4.162712in}{1.740487in}}%
\pgfpathlineto{\pgfqpoint{4.162874in}{1.770123in}}%
\pgfpathlineto{\pgfqpoint{4.163682in}{1.795207in}}%
\pgfpathlineto{\pgfqpoint{4.164166in}{1.310417in}}%
\pgfpathlineto{\pgfqpoint{4.165133in}{1.725117in}}%
\pgfpathlineto{\pgfqpoint{4.165455in}{1.692426in}}%
\pgfpathlineto{\pgfqpoint{4.165776in}{1.670555in}}%
\pgfpathlineto{\pgfqpoint{4.166097in}{1.530882in}}%
\pgfpathlineto{\pgfqpoint{4.166418in}{1.694018in}}%
\pgfpathlineto{\pgfqpoint{4.166739in}{1.684510in}}%
\pgfpathlineto{\pgfqpoint{4.167220in}{1.747471in}}%
\pgfpathlineto{\pgfqpoint{4.167060in}{1.664721in}}%
\pgfpathlineto{\pgfqpoint{4.167380in}{1.729583in}}%
\pgfpathlineto{\pgfqpoint{4.168020in}{1.819149in}}%
\pgfpathlineto{\pgfqpoint{4.168818in}{1.452326in}}%
\pgfpathlineto{\pgfqpoint{4.169933in}{1.729152in}}%
\pgfpathlineto{\pgfqpoint{4.170569in}{1.554296in}}%
\pgfpathlineto{\pgfqpoint{4.171204in}{1.709333in}}%
\pgfpathlineto{\pgfqpoint{4.171363in}{1.722119in}}%
\pgfpathlineto{\pgfqpoint{4.171839in}{1.594446in}}%
\pgfpathlineto{\pgfqpoint{4.172313in}{1.729264in}}%
\pgfpathlineto{\pgfqpoint{4.172472in}{1.701147in}}%
\pgfpathlineto{\pgfqpoint{4.172788in}{1.559963in}}%
\pgfpathlineto{\pgfqpoint{4.173104in}{1.734420in}}%
\pgfpathlineto{\pgfqpoint{4.173262in}{1.707481in}}%
\pgfpathlineto{\pgfqpoint{4.173420in}{1.766319in}}%
\pgfpathlineto{\pgfqpoint{4.173577in}{1.572647in}}%
\pgfpathlineto{\pgfqpoint{4.173893in}{1.734982in}}%
\pgfpathlineto{\pgfqpoint{4.174208in}{1.387935in}}%
\pgfpathlineto{\pgfqpoint{4.174995in}{1.658515in}}%
\pgfpathlineto{\pgfqpoint{4.175309in}{1.688748in}}%
\pgfpathlineto{\pgfqpoint{4.175466in}{1.635339in}}%
\pgfpathlineto{\pgfqpoint{4.176564in}{1.254732in}}%
\pgfpathlineto{\pgfqpoint{4.175937in}{1.707725in}}%
\pgfpathlineto{\pgfqpoint{4.176721in}{1.519653in}}%
\pgfpathlineto{\pgfqpoint{4.177034in}{1.707185in}}%
\pgfpathlineto{\pgfqpoint{4.177659in}{1.496273in}}%
\pgfpathlineto{\pgfqpoint{4.177972in}{1.696497in}}%
\pgfpathlineto{\pgfqpoint{4.178128in}{1.526702in}}%
\pgfpathlineto{\pgfqpoint{4.178907in}{1.701338in}}%
\pgfpathlineto{\pgfqpoint{4.179375in}{1.671759in}}%
\pgfpathlineto{\pgfqpoint{4.179530in}{1.743049in}}%
\pgfpathlineto{\pgfqpoint{4.179997in}{1.396780in}}%
\pgfpathlineto{\pgfqpoint{4.180152in}{1.758398in}}%
\pgfpathlineto{\pgfqpoint{4.180618in}{1.595544in}}%
\pgfpathlineto{\pgfqpoint{4.181393in}{1.754212in}}%
\pgfpathlineto{\pgfqpoint{4.181548in}{1.526186in}}%
\pgfpathlineto{\pgfqpoint{4.181702in}{1.613268in}}%
\pgfpathlineto{\pgfqpoint{4.182784in}{1.714676in}}%
\pgfpathlineto{\pgfqpoint{4.183093in}{1.570617in}}%
\pgfpathlineto{\pgfqpoint{4.183401in}{1.747456in}}%
\pgfpathlineto{\pgfqpoint{4.183864in}{1.693510in}}%
\pgfpathlineto{\pgfqpoint{4.184018in}{1.698261in}}%
\pgfpathlineto{\pgfqpoint{4.184171in}{1.673618in}}%
\pgfpathlineto{\pgfqpoint{4.184479in}{1.528222in}}%
\pgfpathlineto{\pgfqpoint{4.184633in}{1.732536in}}%
\pgfpathlineto{\pgfqpoint{4.184940in}{1.602142in}}%
\pgfpathlineto{\pgfqpoint{4.185094in}{1.747441in}}%
\pgfpathlineto{\pgfqpoint{4.185554in}{1.444834in}}%
\pgfpathlineto{\pgfqpoint{4.186167in}{1.724732in}}%
\pgfpathlineto{\pgfqpoint{4.186320in}{1.800212in}}%
\pgfpathlineto{\pgfqpoint{4.186473in}{1.598886in}}%
\pgfpathlineto{\pgfqpoint{4.187085in}{1.648132in}}%
\pgfpathlineto{\pgfqpoint{4.187390in}{1.542028in}}%
\pgfpathlineto{\pgfqpoint{4.187543in}{1.696715in}}%
\pgfpathlineto{\pgfqpoint{4.187696in}{1.704609in}}%
\pgfpathlineto{\pgfqpoint{4.188306in}{1.443753in}}%
\pgfpathlineto{\pgfqpoint{4.188458in}{1.775024in}}%
\pgfpathlineto{\pgfqpoint{4.188915in}{1.587954in}}%
\pgfpathlineto{\pgfqpoint{4.189219in}{1.805361in}}%
\pgfpathlineto{\pgfqpoint{4.189523in}{1.480946in}}%
\pgfpathlineto{\pgfqpoint{4.189978in}{1.774218in}}%
\pgfpathlineto{\pgfqpoint{4.190282in}{1.422534in}}%
\pgfpathlineto{\pgfqpoint{4.191040in}{1.600392in}}%
\pgfpathlineto{\pgfqpoint{4.191342in}{1.744348in}}%
\pgfpathlineto{\pgfqpoint{4.191947in}{1.612042in}}%
\pgfpathlineto{\pgfqpoint{4.192249in}{1.530294in}}%
\pgfpathlineto{\pgfqpoint{4.192400in}{1.786959in}}%
\pgfpathlineto{\pgfqpoint{4.193003in}{1.571467in}}%
\pgfpathlineto{\pgfqpoint{4.193304in}{1.712214in}}%
\pgfpathlineto{\pgfqpoint{4.193606in}{1.457815in}}%
\pgfpathlineto{\pgfqpoint{4.193756in}{1.610199in}}%
\pgfpathlineto{\pgfqpoint{4.193906in}{1.426153in}}%
\pgfpathlineto{\pgfqpoint{4.194658in}{1.740558in}}%
\pgfpathlineto{\pgfqpoint{4.194808in}{1.569847in}}%
\pgfpathlineto{\pgfqpoint{4.195707in}{1.744441in}}%
\pgfpathlineto{\pgfqpoint{4.196006in}{1.664545in}}%
\pgfpathlineto{\pgfqpoint{4.196156in}{1.735359in}}%
\pgfpathlineto{\pgfqpoint{4.197053in}{1.640825in}}%
\pgfpathlineto{\pgfqpoint{4.197202in}{1.630007in}}%
\pgfpathlineto{\pgfqpoint{4.197649in}{1.600932in}}%
\pgfpathlineto{\pgfqpoint{4.198394in}{1.776467in}}%
\pgfpathlineto{\pgfqpoint{4.198989in}{1.498548in}}%
\pgfpathlineto{\pgfqpoint{4.199583in}{1.702501in}}%
\pgfpathlineto{\pgfqpoint{4.199731in}{1.746045in}}%
\pgfpathlineto{\pgfqpoint{4.199879in}{1.407247in}}%
\pgfpathlineto{\pgfqpoint{4.200027in}{1.754500in}}%
\pgfpathlineto{\pgfqpoint{4.200768in}{1.513245in}}%
\pgfpathlineto{\pgfqpoint{4.201655in}{1.766492in}}%
\pgfpathlineto{\pgfqpoint{4.201802in}{1.690601in}}%
\pgfpathlineto{\pgfqpoint{4.201950in}{1.506702in}}%
\pgfpathlineto{\pgfqpoint{4.202540in}{1.742289in}}%
\pgfpathlineto{\pgfqpoint{4.202834in}{1.705019in}}%
\pgfpathlineto{\pgfqpoint{4.203423in}{1.763801in}}%
\pgfpathlineto{\pgfqpoint{4.204011in}{1.324755in}}%
\pgfpathlineto{\pgfqpoint{4.204744in}{1.766621in}}%
\pgfpathlineto{\pgfqpoint{4.205184in}{1.611562in}}%
\pgfpathlineto{\pgfqpoint{4.205477in}{1.597384in}}%
\pgfpathlineto{\pgfqpoint{4.205915in}{1.697901in}}%
\pgfpathlineto{\pgfqpoint{4.206061in}{1.386418in}}%
\pgfpathlineto{\pgfqpoint{4.206937in}{1.750656in}}%
\pgfpathlineto{\pgfqpoint{4.207520in}{1.494097in}}%
\pgfpathlineto{\pgfqpoint{4.207957in}{1.697553in}}%
\pgfpathlineto{\pgfqpoint{4.208102in}{1.786346in}}%
\pgfpathlineto{\pgfqpoint{4.208684in}{1.642631in}}%
\pgfpathlineto{\pgfqpoint{4.208974in}{1.678196in}}%
\pgfpathlineto{\pgfqpoint{4.209989in}{1.772940in}}%
\pgfpathlineto{\pgfqpoint{4.209554in}{1.619945in}}%
\pgfpathlineto{\pgfqpoint{4.210134in}{1.701217in}}%
\pgfpathlineto{\pgfqpoint{4.210857in}{1.515305in}}%
\pgfpathlineto{\pgfqpoint{4.211145in}{1.712250in}}%
\pgfpathlineto{\pgfqpoint{4.211290in}{1.769073in}}%
\pgfpathlineto{\pgfqpoint{4.211578in}{1.688233in}}%
\pgfpathlineto{\pgfqpoint{4.212299in}{1.733210in}}%
\pgfpathlineto{\pgfqpoint{4.213018in}{1.570172in}}%
\pgfpathlineto{\pgfqpoint{4.212731in}{1.771692in}}%
\pgfpathlineto{\pgfqpoint{4.213162in}{1.612205in}}%
\pgfpathlineto{\pgfqpoint{4.213449in}{1.591640in}}%
\pgfpathlineto{\pgfqpoint{4.214310in}{1.799666in}}%
\pgfpathlineto{\pgfqpoint{4.214454in}{1.639818in}}%
\pgfpathlineto{\pgfqpoint{4.215455in}{1.662472in}}%
\pgfpathlineto{\pgfqpoint{4.215598in}{1.739781in}}%
\pgfpathlineto{\pgfqpoint{4.216169in}{1.591970in}}%
\pgfpathlineto{\pgfqpoint{4.216312in}{1.674061in}}%
\pgfpathlineto{\pgfqpoint{4.217025in}{1.541976in}}%
\pgfpathlineto{\pgfqpoint{4.216597in}{1.746578in}}%
\pgfpathlineto{\pgfqpoint{4.217309in}{1.703858in}}%
\pgfpathlineto{\pgfqpoint{4.217878in}{1.645320in}}%
\pgfpathlineto{\pgfqpoint{4.218020in}{1.759601in}}%
\pgfpathlineto{\pgfqpoint{4.218162in}{1.387990in}}%
\pgfpathlineto{\pgfqpoint{4.218730in}{1.773647in}}%
\pgfpathlineto{\pgfqpoint{4.219155in}{1.648094in}}%
\pgfpathlineto{\pgfqpoint{4.219863in}{1.485084in}}%
\pgfpathlineto{\pgfqpoint{4.219722in}{1.732790in}}%
\pgfpathlineto{\pgfqpoint{4.220287in}{1.545430in}}%
\pgfpathlineto{\pgfqpoint{4.220993in}{1.468208in}}%
\pgfpathlineto{\pgfqpoint{4.221557in}{1.783806in}}%
\pgfpathlineto{\pgfqpoint{4.222542in}{1.528245in}}%
\pgfpathlineto{\pgfqpoint{4.222402in}{1.798307in}}%
\pgfpathlineto{\pgfqpoint{4.222964in}{1.649895in}}%
\pgfpathlineto{\pgfqpoint{4.223665in}{1.770998in}}%
\pgfpathlineto{\pgfqpoint{4.223805in}{1.622371in}}%
\pgfpathlineto{\pgfqpoint{4.224086in}{1.702041in}}%
\pgfpathlineto{\pgfqpoint{4.224785in}{1.394984in}}%
\pgfpathlineto{\pgfqpoint{4.224506in}{1.736134in}}%
\pgfpathlineto{\pgfqpoint{4.225065in}{1.648808in}}%
\pgfpathlineto{\pgfqpoint{4.225205in}{1.759987in}}%
\pgfpathlineto{\pgfqpoint{4.225344in}{1.563283in}}%
\pgfpathlineto{\pgfqpoint{4.226181in}{1.707731in}}%
\pgfpathlineto{\pgfqpoint{4.226460in}{1.677626in}}%
\pgfpathlineto{\pgfqpoint{4.227017in}{1.717991in}}%
\pgfpathlineto{\pgfqpoint{4.227434in}{1.486712in}}%
\pgfpathlineto{\pgfqpoint{4.228405in}{1.802385in}}%
\pgfpathlineto{\pgfqpoint{4.228544in}{1.593187in}}%
\pgfpathlineto{\pgfqpoint{4.228821in}{1.715918in}}%
\pgfpathlineto{\pgfqpoint{4.229098in}{1.510919in}}%
\pgfpathlineto{\pgfqpoint{4.229513in}{1.696442in}}%
\pgfpathlineto{\pgfqpoint{4.229651in}{1.373307in}}%
\pgfpathlineto{\pgfqpoint{4.230618in}{1.706575in}}%
\pgfpathlineto{\pgfqpoint{4.230893in}{1.528175in}}%
\pgfpathlineto{\pgfqpoint{4.231307in}{1.708442in}}%
\pgfpathlineto{\pgfqpoint{4.231720in}{1.638420in}}%
\pgfpathlineto{\pgfqpoint{4.231995in}{1.749468in}}%
\pgfpathlineto{\pgfqpoint{4.232544in}{1.667578in}}%
\pgfpathlineto{\pgfqpoint{4.232681in}{1.484349in}}%
\pgfpathlineto{\pgfqpoint{4.233093in}{1.704822in}}%
\pgfpathlineto{\pgfqpoint{4.233641in}{1.632685in}}%
\pgfpathlineto{\pgfqpoint{4.233915in}{1.556482in}}%
\pgfpathlineto{\pgfqpoint{4.234188in}{1.739442in}}%
\pgfpathlineto{\pgfqpoint{4.234325in}{1.669838in}}%
\pgfpathlineto{\pgfqpoint{4.234462in}{1.765115in}}%
\pgfpathlineto{\pgfqpoint{4.235008in}{1.469088in}}%
\pgfpathlineto{\pgfqpoint{4.235417in}{1.736465in}}%
\pgfpathlineto{\pgfqpoint{4.235690in}{1.778804in}}%
\pgfpathlineto{\pgfqpoint{4.236779in}{1.476207in}}%
\pgfpathlineto{\pgfqpoint{4.237729in}{1.796834in}}%
\pgfpathlineto{\pgfqpoint{4.237865in}{1.650909in}}%
\pgfpathlineto{\pgfqpoint{4.238000in}{1.496158in}}%
\pgfpathlineto{\pgfqpoint{4.238542in}{1.732397in}}%
\pgfpathlineto{\pgfqpoint{4.238813in}{1.558124in}}%
\pgfpathlineto{\pgfqpoint{4.239759in}{1.789300in}}%
\pgfpathlineto{\pgfqpoint{4.239354in}{1.529951in}}%
\pgfpathlineto{\pgfqpoint{4.240029in}{1.701524in}}%
\pgfpathlineto{\pgfqpoint{4.240163in}{1.770202in}}%
\pgfpathlineto{\pgfqpoint{4.240298in}{1.637770in}}%
\pgfpathlineto{\pgfqpoint{4.240433in}{1.671166in}}%
\pgfpathlineto{\pgfqpoint{4.240568in}{1.226258in}}%
\pgfpathlineto{\pgfqpoint{4.240703in}{1.768536in}}%
\pgfpathlineto{\pgfqpoint{4.241644in}{1.336734in}}%
\pgfpathlineto{\pgfqpoint{4.242181in}{1.786775in}}%
\pgfpathlineto{\pgfqpoint{4.242718in}{1.397766in}}%
\pgfpathlineto{\pgfqpoint{4.242852in}{1.382604in}}%
\pgfpathlineto{\pgfqpoint{4.243789in}{1.753878in}}%
\pgfpathlineto{\pgfqpoint{4.244056in}{1.688172in}}%
\pgfpathlineto{\pgfqpoint{4.244323in}{1.771676in}}%
\pgfpathlineto{\pgfqpoint{4.244457in}{1.556489in}}%
\pgfpathlineto{\pgfqpoint{4.244724in}{1.686274in}}%
\pgfpathlineto{\pgfqpoint{4.245124in}{1.506878in}}%
\pgfpathlineto{\pgfqpoint{4.245523in}{1.721030in}}%
\pgfpathlineto{\pgfqpoint{4.245790in}{1.692455in}}%
\pgfpathlineto{\pgfqpoint{4.246720in}{1.775050in}}%
\pgfpathlineto{\pgfqpoint{4.246587in}{1.536808in}}%
\pgfpathlineto{\pgfqpoint{4.246853in}{1.719698in}}%
\pgfpathlineto{\pgfqpoint{4.247383in}{1.493032in}}%
\pgfpathlineto{\pgfqpoint{4.247251in}{1.748203in}}%
\pgfpathlineto{\pgfqpoint{4.247913in}{1.696434in}}%
\pgfpathlineto{\pgfqpoint{4.248046in}{1.751551in}}%
\pgfpathlineto{\pgfqpoint{4.248443in}{1.573171in}}%
\pgfpathlineto{\pgfqpoint{4.248575in}{1.605040in}}%
\pgfpathlineto{\pgfqpoint{4.248707in}{1.381093in}}%
\pgfpathlineto{\pgfqpoint{4.249235in}{1.766903in}}%
\pgfpathlineto{\pgfqpoint{4.249631in}{1.513627in}}%
\pgfpathlineto{\pgfqpoint{4.249895in}{1.801773in}}%
\pgfpathlineto{\pgfqpoint{4.250816in}{1.733544in}}%
\pgfpathlineto{\pgfqpoint{4.251079in}{1.752020in}}%
\pgfpathlineto{\pgfqpoint{4.251473in}{1.357546in}}%
\pgfpathlineto{\pgfqpoint{4.251998in}{1.809649in}}%
\pgfpathlineto{\pgfqpoint{4.252260in}{1.703929in}}%
\pgfpathlineto{\pgfqpoint{4.252392in}{1.807899in}}%
\pgfpathlineto{\pgfqpoint{4.253177in}{1.549883in}}%
\pgfpathlineto{\pgfqpoint{4.253308in}{1.559466in}}%
\pgfpathlineto{\pgfqpoint{4.253569in}{1.813060in}}%
\pgfpathlineto{\pgfqpoint{4.253830in}{1.517682in}}%
\pgfpathlineto{\pgfqpoint{4.254483in}{1.657021in}}%
\pgfpathlineto{\pgfqpoint{4.255264in}{1.812375in}}%
\pgfpathlineto{\pgfqpoint{4.255004in}{1.607491in}}%
\pgfpathlineto{\pgfqpoint{4.255524in}{1.735882in}}%
\pgfpathlineto{\pgfqpoint{4.255654in}{1.319170in}}%
\pgfpathlineto{\pgfqpoint{4.256174in}{1.775164in}}%
\pgfpathlineto{\pgfqpoint{4.256563in}{1.652646in}}%
\pgfpathlineto{\pgfqpoint{4.256693in}{1.744256in}}%
\pgfpathlineto{\pgfqpoint{4.257082in}{1.561369in}}%
\pgfpathlineto{\pgfqpoint{4.257600in}{1.737659in}}%
\pgfpathlineto{\pgfqpoint{4.258376in}{1.533634in}}%
\pgfpathlineto{\pgfqpoint{4.258117in}{1.753816in}}%
\pgfpathlineto{\pgfqpoint{4.258634in}{1.745270in}}%
\pgfpathlineto{\pgfqpoint{4.258892in}{1.502005in}}%
\pgfpathlineto{\pgfqpoint{4.259150in}{1.746994in}}%
\pgfpathlineto{\pgfqpoint{4.259795in}{1.558605in}}%
\pgfpathlineto{\pgfqpoint{4.260309in}{1.798314in}}%
\pgfpathlineto{\pgfqpoint{4.260952in}{1.627967in}}%
\pgfpathlineto{\pgfqpoint{4.261721in}{1.817435in}}%
\pgfpathlineto{\pgfqpoint{4.261337in}{1.621519in}}%
\pgfpathlineto{\pgfqpoint{4.261978in}{1.676008in}}%
\pgfpathlineto{\pgfqpoint{4.262745in}{1.569570in}}%
\pgfpathlineto{\pgfqpoint{4.262234in}{1.735063in}}%
\pgfpathlineto{\pgfqpoint{4.262873in}{1.684521in}}%
\pgfpathlineto{\pgfqpoint{4.263257in}{1.784525in}}%
\pgfpathlineto{\pgfqpoint{4.263384in}{1.577873in}}%
\pgfpathlineto{\pgfqpoint{4.263895in}{1.725690in}}%
\pgfpathlineto{\pgfqpoint{4.264659in}{1.521396in}}%
\pgfpathlineto{\pgfqpoint{4.264786in}{1.764100in}}%
\pgfpathlineto{\pgfqpoint{4.264914in}{1.561555in}}%
\pgfpathlineto{\pgfqpoint{4.266057in}{1.791166in}}%
\pgfpathlineto{\pgfqpoint{4.266564in}{1.257859in}}%
\pgfpathlineto{\pgfqpoint{4.267197in}{1.612779in}}%
\pgfpathlineto{\pgfqpoint{4.267577in}{1.773475in}}%
\pgfpathlineto{\pgfqpoint{4.267450in}{1.489685in}}%
\pgfpathlineto{\pgfqpoint{4.268208in}{1.705004in}}%
\pgfpathlineto{\pgfqpoint{4.268839in}{1.799400in}}%
\pgfpathlineto{\pgfqpoint{4.269343in}{1.484733in}}%
\pgfpathlineto{\pgfqpoint{4.270474in}{1.758686in}}%
\pgfpathlineto{\pgfqpoint{4.270600in}{1.589931in}}%
\pgfpathlineto{\pgfqpoint{4.271603in}{1.607584in}}%
\pgfpathlineto{\pgfqpoint{4.272478in}{1.512031in}}%
\pgfpathlineto{\pgfqpoint{4.272728in}{1.738531in}}%
\pgfpathlineto{\pgfqpoint{4.273477in}{1.803264in}}%
\pgfpathlineto{\pgfqpoint{4.273850in}{1.564191in}}%
\pgfpathlineto{\pgfqpoint{4.273975in}{1.785287in}}%
\pgfpathlineto{\pgfqpoint{4.274970in}{1.709747in}}%
\pgfpathlineto{\pgfqpoint{4.275094in}{1.763968in}}%
\pgfpathlineto{\pgfqpoint{4.275467in}{1.567790in}}%
\pgfpathlineto{\pgfqpoint{4.275963in}{1.695406in}}%
\pgfpathlineto{\pgfqpoint{4.276458in}{1.463719in}}%
\pgfpathlineto{\pgfqpoint{4.276829in}{1.744124in}}%
\pgfpathlineto{\pgfqpoint{4.276953in}{1.742104in}}%
\pgfpathlineto{\pgfqpoint{4.277077in}{1.752824in}}%
\pgfpathlineto{\pgfqpoint{4.277200in}{1.520959in}}%
\pgfpathlineto{\pgfqpoint{4.277941in}{1.809534in}}%
\pgfpathlineto{\pgfqpoint{4.278188in}{1.659801in}}%
\pgfpathlineto{\pgfqpoint{4.278927in}{1.775346in}}%
\pgfpathlineto{\pgfqpoint{4.278681in}{1.344404in}}%
\pgfpathlineto{\pgfqpoint{4.279296in}{1.687651in}}%
\pgfpathlineto{\pgfqpoint{4.279910in}{1.747136in}}%
\pgfpathlineto{\pgfqpoint{4.279665in}{1.642694in}}%
\pgfpathlineto{\pgfqpoint{4.280033in}{1.666807in}}%
\pgfpathlineto{\pgfqpoint{4.280524in}{1.561188in}}%
\pgfpathlineto{\pgfqpoint{4.280278in}{1.746822in}}%
\pgfpathlineto{\pgfqpoint{4.281014in}{1.624094in}}%
\pgfpathlineto{\pgfqpoint{4.281626in}{1.744467in}}%
\pgfpathlineto{\pgfqpoint{4.281748in}{1.529500in}}%
\pgfpathlineto{\pgfqpoint{4.282115in}{1.708228in}}%
\pgfpathlineto{\pgfqpoint{4.282359in}{1.501014in}}%
\pgfpathlineto{\pgfqpoint{4.283335in}{1.588372in}}%
\pgfpathlineto{\pgfqpoint{4.284186in}{1.776187in}}%
\pgfpathlineto{\pgfqpoint{4.283943in}{1.546921in}}%
\pgfpathlineto{\pgfqpoint{4.284429in}{1.651858in}}%
\pgfpathlineto{\pgfqpoint{4.284915in}{1.810547in}}%
\pgfpathlineto{\pgfqpoint{4.285521in}{1.082604in}}%
\pgfpathlineto{\pgfqpoint{4.286248in}{1.764564in}}%
\pgfpathlineto{\pgfqpoint{4.286731in}{1.681289in}}%
\pgfpathlineto{\pgfqpoint{4.287214in}{1.600123in}}%
\pgfpathlineto{\pgfqpoint{4.287335in}{1.648967in}}%
\pgfpathlineto{\pgfqpoint{4.288299in}{1.812140in}}%
\pgfpathlineto{\pgfqpoint{4.287938in}{1.337355in}}%
\pgfpathlineto{\pgfqpoint{4.288420in}{1.716418in}}%
\pgfpathlineto{\pgfqpoint{4.288901in}{1.568785in}}%
\pgfpathlineto{\pgfqpoint{4.289261in}{1.747925in}}%
\pgfpathlineto{\pgfqpoint{4.289381in}{1.707053in}}%
\pgfpathlineto{\pgfqpoint{4.289621in}{1.576002in}}%
\pgfpathlineto{\pgfqpoint{4.290461in}{1.775063in}}%
\pgfpathlineto{\pgfqpoint{4.291059in}{1.546447in}}%
\pgfpathlineto{\pgfqpoint{4.291298in}{1.776264in}}%
\pgfpathlineto{\pgfqpoint{4.291657in}{1.711743in}}%
\pgfpathlineto{\pgfqpoint{4.291776in}{1.723352in}}%
\pgfpathlineto{\pgfqpoint{4.291896in}{1.639500in}}%
\pgfpathlineto{\pgfqpoint{4.292254in}{1.500235in}}%
\pgfpathlineto{\pgfqpoint{4.292373in}{1.767040in}}%
\pgfpathlineto{\pgfqpoint{4.292492in}{1.192478in}}%
\pgfpathlineto{\pgfqpoint{4.292730in}{1.830284in}}%
\pgfpathlineto{\pgfqpoint{4.293445in}{1.623873in}}%
\pgfpathlineto{\pgfqpoint{4.293801in}{1.768499in}}%
\pgfpathlineto{\pgfqpoint{4.294395in}{1.577949in}}%
\pgfpathlineto{\pgfqpoint{4.294514in}{1.566412in}}%
\pgfpathlineto{\pgfqpoint{4.294632in}{1.864741in}}%
\pgfpathlineto{\pgfqpoint{4.295343in}{1.450839in}}%
\pgfpathlineto{\pgfqpoint{4.295580in}{1.729945in}}%
\pgfpathlineto{\pgfqpoint{4.295698in}{1.468400in}}%
\pgfpathlineto{\pgfqpoint{4.296289in}{1.756098in}}%
\pgfpathlineto{\pgfqpoint{4.296644in}{1.644620in}}%
\pgfpathlineto{\pgfqpoint{4.296762in}{1.644643in}}%
\pgfpathlineto{\pgfqpoint{4.296998in}{1.546237in}}%
\pgfpathlineto{\pgfqpoint{4.297823in}{1.822567in}}%
\pgfpathlineto{\pgfqpoint{4.298998in}{1.419675in}}%
\pgfpathlineto{\pgfqpoint{4.299585in}{1.804732in}}%
\pgfpathlineto{\pgfqpoint{4.300170in}{1.726263in}}%
\pgfpathlineto{\pgfqpoint{4.300872in}{1.485643in}}%
\pgfpathlineto{\pgfqpoint{4.300755in}{1.799016in}}%
\pgfpathlineto{\pgfqpoint{4.301223in}{1.580739in}}%
\pgfpathlineto{\pgfqpoint{4.301339in}{1.773166in}}%
\pgfpathlineto{\pgfqpoint{4.302156in}{1.503887in}}%
\pgfpathlineto{\pgfqpoint{4.302389in}{1.731383in}}%
\pgfpathlineto{\pgfqpoint{4.302738in}{1.823971in}}%
\pgfpathlineto{\pgfqpoint{4.302971in}{1.634005in}}%
\pgfpathlineto{\pgfqpoint{4.303319in}{1.695863in}}%
\pgfpathlineto{\pgfqpoint{4.303784in}{1.631811in}}%
\pgfpathlineto{\pgfqpoint{4.304132in}{1.765644in}}%
\pgfpathlineto{\pgfqpoint{4.304248in}{1.754066in}}%
\pgfpathlineto{\pgfqpoint{4.305059in}{1.348729in}}%
\pgfpathlineto{\pgfqpoint{4.304480in}{1.768503in}}%
\pgfpathlineto{\pgfqpoint{4.305290in}{1.677567in}}%
\pgfpathlineto{\pgfqpoint{4.305521in}{1.794521in}}%
\pgfpathlineto{\pgfqpoint{4.306099in}{1.602174in}}%
\pgfpathlineto{\pgfqpoint{4.306215in}{1.783115in}}%
\pgfpathlineto{\pgfqpoint{4.306791in}{1.577817in}}%
\pgfpathlineto{\pgfqpoint{4.306445in}{1.828057in}}%
\pgfpathlineto{\pgfqpoint{4.307252in}{1.790495in}}%
\pgfpathlineto{\pgfqpoint{4.307942in}{1.798607in}}%
\pgfpathlineto{\pgfqpoint{4.308172in}{1.526124in}}%
\pgfpathlineto{\pgfqpoint{4.308287in}{1.821047in}}%
\pgfpathlineto{\pgfqpoint{4.309319in}{1.780841in}}%
\pgfpathlineto{\pgfqpoint{4.310235in}{1.441243in}}%
\pgfpathlineto{\pgfqpoint{4.310349in}{1.813918in}}%
\pgfpathlineto{\pgfqpoint{4.310463in}{1.692632in}}%
\pgfpathlineto{\pgfqpoint{4.311034in}{1.827972in}}%
\pgfpathlineto{\pgfqpoint{4.310692in}{1.649365in}}%
\pgfpathlineto{\pgfqpoint{4.311262in}{1.765335in}}%
\pgfpathlineto{\pgfqpoint{4.311376in}{1.605816in}}%
\pgfpathlineto{\pgfqpoint{4.311946in}{1.794174in}}%
\pgfpathlineto{\pgfqpoint{4.312288in}{1.724934in}}%
\pgfpathlineto{\pgfqpoint{4.312401in}{1.745777in}}%
\pgfpathlineto{\pgfqpoint{4.312515in}{1.526234in}}%
\pgfpathlineto{\pgfqpoint{4.313424in}{1.773497in}}%
\pgfpathlineto{\pgfqpoint{4.313537in}{1.570390in}}%
\pgfpathlineto{\pgfqpoint{4.314217in}{1.855544in}}%
\pgfpathlineto{\pgfqpoint{4.314670in}{1.729357in}}%
\pgfpathlineto{\pgfqpoint{4.315009in}{1.442543in}}%
\pgfpathlineto{\pgfqpoint{4.315348in}{1.802450in}}%
\pgfpathlineto{\pgfqpoint{4.315800in}{1.689769in}}%
\pgfpathlineto{\pgfqpoint{4.316814in}{1.795415in}}%
\pgfpathlineto{\pgfqpoint{4.316251in}{1.528033in}}%
\pgfpathlineto{\pgfqpoint{4.316927in}{1.741682in}}%
\pgfpathlineto{\pgfqpoint{4.317039in}{1.746577in}}%
\pgfpathlineto{\pgfqpoint{4.317377in}{1.579179in}}%
\pgfpathlineto{\pgfqpoint{4.317938in}{1.772802in}}%
\pgfpathlineto{\pgfqpoint{4.318051in}{1.771924in}}%
\pgfpathlineto{\pgfqpoint{4.318499in}{1.600951in}}%
\pgfpathlineto{\pgfqpoint{4.318275in}{1.854363in}}%
\pgfpathlineto{\pgfqpoint{4.319171in}{1.751682in}}%
\pgfpathlineto{\pgfqpoint{4.319954in}{1.829161in}}%
\pgfpathlineto{\pgfqpoint{4.319507in}{1.568994in}}%
\pgfpathlineto{\pgfqpoint{4.320178in}{1.712621in}}%
\pgfpathlineto{\pgfqpoint{4.320736in}{1.502613in}}%
\pgfpathlineto{\pgfqpoint{4.320624in}{1.801371in}}%
\pgfpathlineto{\pgfqpoint{4.321293in}{1.644164in}}%
\pgfpathlineto{\pgfqpoint{4.321738in}{1.808874in}}%
\pgfpathlineto{\pgfqpoint{4.321849in}{1.537692in}}%
\pgfpathlineto{\pgfqpoint{4.321961in}{1.560434in}}%
\pgfpathlineto{\pgfqpoint{4.322072in}{1.452325in}}%
\pgfpathlineto{\pgfqpoint{4.322627in}{1.801366in}}%
\pgfpathlineto{\pgfqpoint{4.322960in}{1.481896in}}%
\pgfpathlineto{\pgfqpoint{4.323404in}{1.849273in}}%
\pgfpathlineto{\pgfqpoint{4.324290in}{1.736358in}}%
\pgfpathlineto{\pgfqpoint{4.324400in}{1.735887in}}%
\pgfpathlineto{\pgfqpoint{4.324732in}{1.593659in}}%
\pgfpathlineto{\pgfqpoint{4.325063in}{1.781741in}}%
\pgfpathlineto{\pgfqpoint{4.325504in}{1.398592in}}%
\pgfpathlineto{\pgfqpoint{4.325394in}{1.818016in}}%
\pgfpathlineto{\pgfqpoint{4.326166in}{1.760736in}}%
\pgfpathlineto{\pgfqpoint{4.326496in}{1.537217in}}%
\pgfpathlineto{\pgfqpoint{4.326936in}{1.844998in}}%
\pgfpathlineto{\pgfqpoint{4.327265in}{1.583241in}}%
\pgfpathlineto{\pgfqpoint{4.327485in}{1.760038in}}%
\pgfpathlineto{\pgfqpoint{4.327595in}{1.283685in}}%
\pgfpathlineto{\pgfqpoint{4.328253in}{1.671012in}}%
\pgfpathlineto{\pgfqpoint{4.328472in}{1.834182in}}%
\pgfpathlineto{\pgfqpoint{4.329128in}{1.540174in}}%
\pgfpathlineto{\pgfqpoint{4.329238in}{1.829822in}}%
\pgfpathlineto{\pgfqpoint{4.330220in}{1.578382in}}%
\pgfpathlineto{\pgfqpoint{4.330983in}{1.821519in}}%
\pgfpathlineto{\pgfqpoint{4.331201in}{1.543881in}}%
\pgfpathlineto{\pgfqpoint{4.331418in}{1.701360in}}%
\pgfpathlineto{\pgfqpoint{4.331636in}{1.785288in}}%
\pgfpathlineto{\pgfqpoint{4.332179in}{1.637026in}}%
\pgfpathlineto{\pgfqpoint{4.332287in}{1.703512in}}%
\pgfpathlineto{\pgfqpoint{4.332396in}{1.475263in}}%
\pgfpathlineto{\pgfqpoint{4.332613in}{1.815556in}}%
\pgfpathlineto{\pgfqpoint{4.333371in}{1.722639in}}%
\pgfpathlineto{\pgfqpoint{4.333696in}{1.766958in}}%
\pgfpathlineto{\pgfqpoint{4.333804in}{1.497325in}}%
\pgfpathlineto{\pgfqpoint{4.334020in}{1.565717in}}%
\pgfpathlineto{\pgfqpoint{4.334128in}{1.557699in}}%
\pgfpathlineto{\pgfqpoint{4.334453in}{1.826673in}}%
\pgfpathlineto{\pgfqpoint{4.334884in}{1.528625in}}%
\pgfpathlineto{\pgfqpoint{4.335208in}{1.789292in}}%
\pgfpathlineto{\pgfqpoint{4.335316in}{1.418201in}}%
\pgfpathlineto{\pgfqpoint{4.335854in}{1.830608in}}%
\pgfpathlineto{\pgfqpoint{4.336284in}{1.664309in}}%
\pgfpathlineto{\pgfqpoint{4.336607in}{1.474361in}}%
\pgfpathlineto{\pgfqpoint{4.337358in}{1.845249in}}%
\pgfpathlineto{\pgfqpoint{4.338001in}{1.474738in}}%
\pgfpathlineto{\pgfqpoint{4.338429in}{1.638155in}}%
\pgfpathlineto{\pgfqpoint{4.338857in}{1.815041in}}%
\pgfpathlineto{\pgfqpoint{4.339604in}{1.705545in}}%
\pgfpathlineto{\pgfqpoint{4.339711in}{1.702363in}}%
\pgfpathlineto{\pgfqpoint{4.340563in}{1.541785in}}%
\pgfpathlineto{\pgfqpoint{4.340137in}{1.758821in}}%
\pgfpathlineto{\pgfqpoint{4.340670in}{1.645318in}}%
\pgfpathlineto{\pgfqpoint{4.341626in}{1.797323in}}%
\pgfpathlineto{\pgfqpoint{4.341095in}{1.585020in}}%
\pgfpathlineto{\pgfqpoint{4.341732in}{1.625998in}}%
\pgfpathlineto{\pgfqpoint{4.342263in}{1.838895in}}%
\pgfpathlineto{\pgfqpoint{4.342475in}{1.616069in}}%
\pgfpathlineto{\pgfqpoint{4.343322in}{1.753814in}}%
\pgfpathlineto{\pgfqpoint{4.343744in}{1.659863in}}%
\pgfpathlineto{\pgfqpoint{4.343639in}{1.783968in}}%
\pgfpathlineto{\pgfqpoint{4.343850in}{1.688751in}}%
\pgfpathlineto{\pgfqpoint{4.344589in}{1.565209in}}%
\pgfpathlineto{\pgfqpoint{4.344061in}{1.776479in}}%
\pgfpathlineto{\pgfqpoint{4.344799in}{1.701898in}}%
\pgfpathlineto{\pgfqpoint{4.345221in}{1.775462in}}%
\pgfpathlineto{\pgfqpoint{4.345326in}{1.591994in}}%
\pgfpathlineto{\pgfqpoint{4.345642in}{1.632682in}}%
\pgfpathlineto{\pgfqpoint{4.345747in}{1.614694in}}%
\pgfpathlineto{\pgfqpoint{4.346482in}{1.842012in}}%
\pgfpathlineto{\pgfqpoint{4.346902in}{1.708477in}}%
\pgfpathlineto{\pgfqpoint{4.347740in}{1.589329in}}%
\pgfpathlineto{\pgfqpoint{4.347111in}{1.785216in}}%
\pgfpathlineto{\pgfqpoint{4.348054in}{1.655714in}}%
\pgfpathlineto{\pgfqpoint{4.348576in}{1.821257in}}%
\pgfpathlineto{\pgfqpoint{4.348681in}{1.508987in}}%
\pgfpathlineto{\pgfqpoint{4.349724in}{1.709285in}}%
\pgfpathlineto{\pgfqpoint{4.350036in}{1.553726in}}%
\pgfpathlineto{\pgfqpoint{4.350348in}{1.787152in}}%
\pgfpathlineto{\pgfqpoint{4.350764in}{1.713367in}}%
\pgfpathlineto{\pgfqpoint{4.351283in}{1.804091in}}%
\pgfpathlineto{\pgfqpoint{4.350972in}{1.635113in}}%
\pgfpathlineto{\pgfqpoint{4.351698in}{1.691269in}}%
\pgfpathlineto{\pgfqpoint{4.352423in}{1.527163in}}%
\pgfpathlineto{\pgfqpoint{4.352113in}{1.728185in}}%
\pgfpathlineto{\pgfqpoint{4.352630in}{1.575614in}}%
\pgfpathlineto{\pgfqpoint{4.353147in}{1.806628in}}%
\pgfpathlineto{\pgfqpoint{4.353663in}{1.532972in}}%
\pgfpathlineto{\pgfqpoint{4.353767in}{0.792334in}}%
\pgfpathlineto{\pgfqpoint{4.354076in}{1.858492in}}%
\pgfpathlineto{\pgfqpoint{4.354694in}{1.276231in}}%
\pgfpathlineto{\pgfqpoint{4.355106in}{1.811231in}}%
\pgfpathlineto{\pgfqpoint{4.355722in}{1.754494in}}%
\pgfpathlineto{\pgfqpoint{4.355825in}{1.229471in}}%
\pgfpathlineto{\pgfqpoint{4.356441in}{1.871121in}}%
\pgfpathlineto{\pgfqpoint{4.356851in}{1.563621in}}%
\pgfpathlineto{\pgfqpoint{4.357056in}{1.502789in}}%
\pgfpathlineto{\pgfqpoint{4.357976in}{1.780429in}}%
\pgfpathlineto{\pgfqpoint{4.358996in}{1.470534in}}%
\pgfpathlineto{\pgfqpoint{4.359098in}{1.848479in}}%
\pgfpathlineto{\pgfqpoint{4.360116in}{1.758850in}}%
\pgfpathlineto{\pgfqpoint{4.360217in}{1.760094in}}%
\pgfpathlineto{\pgfqpoint{4.360827in}{1.499042in}}%
\pgfpathlineto{\pgfqpoint{4.361232in}{1.770536in}}%
\pgfpathlineto{\pgfqpoint{4.361334in}{1.821429in}}%
\pgfpathlineto{\pgfqpoint{4.361739in}{1.521772in}}%
\pgfpathlineto{\pgfqpoint{4.362144in}{1.712654in}}%
\pgfpathlineto{\pgfqpoint{4.362851in}{1.612905in}}%
\pgfpathlineto{\pgfqpoint{4.362649in}{1.808407in}}%
\pgfpathlineto{\pgfqpoint{4.363255in}{1.676288in}}%
\pgfpathlineto{\pgfqpoint{4.363356in}{1.822982in}}%
\pgfpathlineto{\pgfqpoint{4.364263in}{1.654303in}}%
\pgfpathlineto{\pgfqpoint{4.364364in}{1.806735in}}%
\pgfpathlineto{\pgfqpoint{4.364665in}{1.536962in}}%
\pgfpathlineto{\pgfqpoint{4.365168in}{1.818742in}}%
\pgfpathlineto{\pgfqpoint{4.365469in}{1.728051in}}%
\pgfpathlineto{\pgfqpoint{4.366472in}{1.819892in}}%
\pgfpathlineto{\pgfqpoint{4.366171in}{1.679591in}}%
\pgfpathlineto{\pgfqpoint{4.366572in}{1.784975in}}%
\pgfpathlineto{\pgfqpoint{4.367272in}{1.602417in}}%
\pgfpathlineto{\pgfqpoint{4.367172in}{1.785744in}}%
\pgfpathlineto{\pgfqpoint{4.367671in}{1.710518in}}%
\pgfpathlineto{\pgfqpoint{4.367771in}{1.717406in}}%
\pgfpathlineto{\pgfqpoint{4.368170in}{1.842970in}}%
\pgfpathlineto{\pgfqpoint{4.367971in}{1.562081in}}%
\pgfpathlineto{\pgfqpoint{4.368868in}{1.776119in}}%
\pgfpathlineto{\pgfqpoint{4.369167in}{1.281666in}}%
\pgfpathlineto{\pgfqpoint{4.369465in}{1.821143in}}%
\pgfpathlineto{\pgfqpoint{4.369862in}{1.789671in}}%
\pgfpathlineto{\pgfqpoint{4.369962in}{1.801659in}}%
\pgfpathlineto{\pgfqpoint{4.370458in}{1.510220in}}%
\pgfpathlineto{\pgfqpoint{4.370260in}{1.870111in}}%
\pgfpathlineto{\pgfqpoint{4.371251in}{1.570759in}}%
\pgfpathlineto{\pgfqpoint{4.372141in}{1.551791in}}%
\pgfpathlineto{\pgfqpoint{4.372339in}{1.864067in}}%
\pgfpathlineto{\pgfqpoint{4.372437in}{1.435956in}}%
\pgfpathlineto{\pgfqpoint{4.373424in}{1.632809in}}%
\pgfpathlineto{\pgfqpoint{4.373621in}{1.789858in}}%
\pgfpathlineto{\pgfqpoint{4.374211in}{1.139177in}}%
\pgfpathlineto{\pgfqpoint{4.374408in}{1.783380in}}%
\pgfpathlineto{\pgfqpoint{4.375389in}{1.778092in}}%
\pgfpathlineto{\pgfqpoint{4.376173in}{1.557538in}}%
\pgfpathlineto{\pgfqpoint{4.375585in}{1.805034in}}%
\pgfpathlineto{\pgfqpoint{4.376369in}{1.789833in}}%
\pgfpathlineto{\pgfqpoint{4.376466in}{1.840373in}}%
\pgfpathlineto{\pgfqpoint{4.377248in}{1.685665in}}%
\pgfpathlineto{\pgfqpoint{4.377443in}{1.772426in}}%
\pgfpathlineto{\pgfqpoint{4.377541in}{1.484838in}}%
\pgfpathlineto{\pgfqpoint{4.378515in}{1.852522in}}%
\pgfpathlineto{\pgfqpoint{4.379099in}{1.486495in}}%
\pgfpathlineto{\pgfqpoint{4.379779in}{1.624688in}}%
\pgfpathlineto{\pgfqpoint{4.380070in}{1.525351in}}%
\pgfpathlineto{\pgfqpoint{4.380167in}{1.726461in}}%
\pgfpathlineto{\pgfqpoint{4.380554in}{1.883784in}}%
\pgfpathlineto{\pgfqpoint{4.380748in}{1.641634in}}%
\pgfpathlineto{\pgfqpoint{4.381232in}{1.787217in}}%
\pgfpathlineto{\pgfqpoint{4.381908in}{1.860946in}}%
\pgfpathlineto{\pgfqpoint{4.382101in}{1.660804in}}%
\pgfpathlineto{\pgfqpoint{4.382198in}{1.864598in}}%
\pgfpathlineto{\pgfqpoint{4.382391in}{1.482977in}}%
\pgfpathlineto{\pgfqpoint{4.383162in}{1.716032in}}%
\pgfpathlineto{\pgfqpoint{4.383258in}{1.706950in}}%
\pgfpathlineto{\pgfqpoint{4.383354in}{1.743283in}}%
\pgfpathlineto{\pgfqpoint{4.383546in}{1.830048in}}%
\pgfpathlineto{\pgfqpoint{4.384123in}{1.681242in}}%
\pgfpathlineto{\pgfqpoint{4.384219in}{1.691987in}}%
\pgfpathlineto{\pgfqpoint{4.384315in}{1.549057in}}%
\pgfpathlineto{\pgfqpoint{4.384603in}{1.794930in}}%
\pgfpathlineto{\pgfqpoint{4.385274in}{1.564584in}}%
\pgfpathlineto{\pgfqpoint{4.385370in}{1.831969in}}%
\pgfpathlineto{\pgfqpoint{4.386136in}{1.515045in}}%
\pgfpathlineto{\pgfqpoint{4.386422in}{1.682943in}}%
\pgfpathlineto{\pgfqpoint{4.386518in}{1.457209in}}%
\pgfpathlineto{\pgfqpoint{4.387090in}{1.874924in}}%
\pgfpathlineto{\pgfqpoint{4.387472in}{1.748362in}}%
\pgfpathlineto{\pgfqpoint{4.387567in}{1.787043in}}%
\pgfpathlineto{\pgfqpoint{4.387662in}{1.606365in}}%
\pgfpathlineto{\pgfqpoint{4.387853in}{1.737204in}}%
\pgfpathlineto{\pgfqpoint{4.387948in}{1.466528in}}%
\pgfpathlineto{\pgfqpoint{4.388804in}{1.816121in}}%
\pgfpathlineto{\pgfqpoint{4.388899in}{1.745872in}}%
\pgfpathlineto{\pgfqpoint{4.389942in}{1.479022in}}%
\pgfpathlineto{\pgfqpoint{4.389563in}{1.853765in}}%
\pgfpathlineto{\pgfqpoint{4.390037in}{1.677867in}}%
\pgfpathlineto{\pgfqpoint{4.390416in}{1.865152in}}%
\pgfpathlineto{\pgfqpoint{4.390510in}{1.523561in}}%
\pgfpathlineto{\pgfqpoint{4.391172in}{1.716519in}}%
\pgfpathlineto{\pgfqpoint{4.391361in}{1.845131in}}%
\pgfpathlineto{\pgfqpoint{4.391456in}{1.762000in}}%
\pgfpathlineto{\pgfqpoint{4.392210in}{1.508347in}}%
\pgfpathlineto{\pgfqpoint{4.392116in}{1.787720in}}%
\pgfpathlineto{\pgfqpoint{4.392493in}{1.654032in}}%
\pgfpathlineto{\pgfqpoint{4.393057in}{1.578493in}}%
\pgfpathlineto{\pgfqpoint{4.393621in}{1.789737in}}%
\pgfpathlineto{\pgfqpoint{4.394185in}{1.623818in}}%
\pgfpathlineto{\pgfqpoint{4.394091in}{1.851716in}}%
\pgfpathlineto{\pgfqpoint{4.394747in}{1.712672in}}%
\pgfpathlineto{\pgfqpoint{4.394841in}{1.763293in}}%
\pgfpathlineto{\pgfqpoint{4.394934in}{1.466039in}}%
\pgfpathlineto{\pgfqpoint{4.395402in}{1.713027in}}%
\pgfpathlineto{\pgfqpoint{4.395496in}{1.261752in}}%
\pgfpathlineto{\pgfqpoint{4.395683in}{1.866827in}}%
\pgfpathlineto{\pgfqpoint{4.396523in}{1.656578in}}%
\pgfpathlineto{\pgfqpoint{4.396710in}{1.815653in}}%
\pgfpathlineto{\pgfqpoint{4.396803in}{1.746102in}}%
\pgfpathlineto{\pgfqpoint{4.397455in}{1.442021in}}%
\pgfpathlineto{\pgfqpoint{4.397548in}{1.781647in}}%
\pgfpathlineto{\pgfqpoint{4.397920in}{1.625197in}}%
\pgfpathlineto{\pgfqpoint{4.398199in}{1.835793in}}%
\pgfpathlineto{\pgfqpoint{4.398849in}{1.496255in}}%
\pgfpathlineto{\pgfqpoint{4.398942in}{1.360879in}}%
\pgfpathlineto{\pgfqpoint{4.399035in}{1.784096in}}%
\pgfpathlineto{\pgfqpoint{4.399776in}{1.665681in}}%
\pgfpathlineto{\pgfqpoint{4.400701in}{1.818346in}}%
\pgfpathlineto{\pgfqpoint{4.400423in}{1.554712in}}%
\pgfpathlineto{\pgfqpoint{4.400885in}{1.706760in}}%
\pgfpathlineto{\pgfqpoint{4.401254in}{1.445385in}}%
\pgfpathlineto{\pgfqpoint{4.401531in}{1.894917in}}%
\pgfpathlineto{\pgfqpoint{4.401808in}{1.469879in}}%
\pgfpathlineto{\pgfqpoint{4.401992in}{1.854182in}}%
\pgfpathlineto{\pgfqpoint{4.402452in}{1.446857in}}%
\pgfpathlineto{\pgfqpoint{4.403004in}{1.813930in}}%
\pgfpathlineto{\pgfqpoint{4.403463in}{1.562298in}}%
\pgfpathlineto{\pgfqpoint{4.403830in}{1.860428in}}%
\pgfpathlineto{\pgfqpoint{4.404288in}{1.610660in}}%
\pgfpathlineto{\pgfqpoint{4.404837in}{1.811924in}}%
\pgfpathlineto{\pgfqpoint{4.404654in}{1.531016in}}%
\pgfpathlineto{\pgfqpoint{4.405294in}{1.794077in}}%
\pgfpathlineto{\pgfqpoint{4.406389in}{1.489270in}}%
\pgfpathlineto{\pgfqpoint{4.407209in}{1.839350in}}%
\pgfpathlineto{\pgfqpoint{4.407573in}{1.742984in}}%
\pgfpathlineto{\pgfqpoint{4.407845in}{1.474160in}}%
\pgfpathlineto{\pgfqpoint{4.407754in}{1.749451in}}%
\pgfpathlineto{\pgfqpoint{4.408662in}{1.718979in}}%
\pgfpathlineto{\pgfqpoint{4.409296in}{1.547399in}}%
\pgfpathlineto{\pgfqpoint{4.408843in}{1.829765in}}%
\pgfpathlineto{\pgfqpoint{4.409658in}{1.776913in}}%
\pgfpathlineto{\pgfqpoint{4.409929in}{1.677952in}}%
\pgfpathlineto{\pgfqpoint{4.410200in}{1.866407in}}%
\pgfpathlineto{\pgfqpoint{4.410742in}{1.692311in}}%
\pgfpathlineto{\pgfqpoint{4.411733in}{1.637863in}}%
\pgfpathlineto{\pgfqpoint{4.411913in}{1.872080in}}%
\pgfpathlineto{\pgfqpoint{4.412452in}{1.279187in}}%
\pgfpathlineto{\pgfqpoint{4.413081in}{1.791419in}}%
\pgfpathlineto{\pgfqpoint{4.413171in}{1.863209in}}%
\pgfpathlineto{\pgfqpoint{4.413619in}{1.611425in}}%
\pgfpathlineto{\pgfqpoint{4.414067in}{1.795525in}}%
\pgfpathlineto{\pgfqpoint{4.414156in}{1.496529in}}%
\pgfpathlineto{\pgfqpoint{4.414603in}{1.880912in}}%
\pgfpathlineto{\pgfqpoint{4.415139in}{1.640092in}}%
\pgfpathlineto{\pgfqpoint{4.415764in}{1.887528in}}%
\pgfpathlineto{\pgfqpoint{4.415407in}{1.602594in}}%
\pgfpathlineto{\pgfqpoint{4.416031in}{1.721706in}}%
\pgfpathlineto{\pgfqpoint{4.416120in}{1.446929in}}%
\pgfpathlineto{\pgfqpoint{4.417099in}{1.788209in}}%
\pgfpathlineto{\pgfqpoint{4.418164in}{1.621148in}}%
\pgfpathlineto{\pgfqpoint{4.417809in}{1.829615in}}%
\pgfpathlineto{\pgfqpoint{4.418253in}{1.776342in}}%
\pgfpathlineto{\pgfqpoint{4.418341in}{1.799653in}}%
\pgfpathlineto{\pgfqpoint{4.418430in}{1.665191in}}%
\pgfpathlineto{\pgfqpoint{4.418519in}{1.716603in}}%
\pgfpathlineto{\pgfqpoint{4.418607in}{1.509806in}}%
\pgfpathlineto{\pgfqpoint{4.419227in}{1.800979in}}%
\pgfpathlineto{\pgfqpoint{4.419580in}{1.714571in}}%
\pgfpathlineto{\pgfqpoint{4.419757in}{1.615286in}}%
\pgfpathlineto{\pgfqpoint{4.420727in}{1.801619in}}%
\pgfpathlineto{\pgfqpoint{4.421256in}{1.566882in}}%
\pgfpathlineto{\pgfqpoint{4.421783in}{1.809047in}}%
\pgfpathlineto{\pgfqpoint{4.421871in}{1.688178in}}%
\pgfpathlineto{\pgfqpoint{4.422574in}{1.855485in}}%
\pgfpathlineto{\pgfqpoint{4.422135in}{1.400206in}}%
\pgfpathlineto{\pgfqpoint{4.423012in}{1.836904in}}%
\pgfpathlineto{\pgfqpoint{4.423888in}{1.660115in}}%
\pgfpathlineto{\pgfqpoint{4.423275in}{1.872299in}}%
\pgfpathlineto{\pgfqpoint{4.424237in}{1.750820in}}%
\pgfpathlineto{\pgfqpoint{4.424412in}{1.680878in}}%
\pgfpathlineto{\pgfqpoint{4.424936in}{1.850545in}}%
\pgfpathlineto{\pgfqpoint{4.425110in}{1.700596in}}%
\pgfpathlineto{\pgfqpoint{4.425198in}{1.860700in}}%
\pgfpathlineto{\pgfqpoint{4.425285in}{1.681460in}}%
\pgfpathlineto{\pgfqpoint{4.426156in}{1.756413in}}%
\pgfpathlineto{\pgfqpoint{4.426938in}{1.850793in}}%
\pgfpathlineto{\pgfqpoint{4.427285in}{1.364948in}}%
\pgfpathlineto{\pgfqpoint{4.428325in}{1.887226in}}%
\pgfpathlineto{\pgfqpoint{4.428411in}{1.780701in}}%
\pgfpathlineto{\pgfqpoint{4.429449in}{1.396026in}}%
\pgfpathlineto{\pgfqpoint{4.428844in}{1.826607in}}%
\pgfpathlineto{\pgfqpoint{4.429535in}{1.732981in}}%
\pgfpathlineto{\pgfqpoint{4.430139in}{1.560465in}}%
\pgfpathlineto{\pgfqpoint{4.430311in}{1.780332in}}%
\pgfpathlineto{\pgfqpoint{4.430483in}{1.707291in}}%
\pgfpathlineto{\pgfqpoint{4.431086in}{1.852896in}}%
\pgfpathlineto{\pgfqpoint{4.431000in}{1.626189in}}%
\pgfpathlineto{\pgfqpoint{4.431601in}{1.717824in}}%
\pgfpathlineto{\pgfqpoint{4.431945in}{1.849009in}}%
\pgfpathlineto{\pgfqpoint{4.432631in}{1.613384in}}%
\pgfpathlineto{\pgfqpoint{4.432973in}{1.838465in}}%
\pgfpathlineto{\pgfqpoint{4.433144in}{1.504716in}}%
\pgfpathlineto{\pgfqpoint{4.433743in}{1.736714in}}%
\pgfpathlineto{\pgfqpoint{4.433828in}{1.626364in}}%
\pgfpathlineto{\pgfqpoint{4.434170in}{1.801307in}}%
\pgfpathlineto{\pgfqpoint{4.434767in}{1.649818in}}%
\pgfpathlineto{\pgfqpoint{4.435193in}{1.889031in}}%
\pgfpathlineto{\pgfqpoint{4.435704in}{1.389745in}}%
\pgfpathlineto{\pgfqpoint{4.435959in}{1.868538in}}%
\pgfpathlineto{\pgfqpoint{4.437147in}{1.701125in}}%
\pgfpathlineto{\pgfqpoint{4.437740in}{1.858109in}}%
\pgfpathlineto{\pgfqpoint{4.438163in}{1.563115in}}%
\pgfpathlineto{\pgfqpoint{4.438248in}{1.548668in}}%
\pgfpathlineto{\pgfqpoint{4.438332in}{1.593448in}}%
\pgfpathlineto{\pgfqpoint{4.438670in}{1.909214in}}%
\pgfpathlineto{\pgfqpoint{4.439093in}{1.327672in}}%
\pgfpathlineto{\pgfqpoint{4.439430in}{1.856661in}}%
\pgfpathlineto{\pgfqpoint{4.440609in}{1.460124in}}%
\pgfpathlineto{\pgfqpoint{4.441533in}{1.872729in}}%
\pgfpathlineto{\pgfqpoint{4.441700in}{1.843004in}}%
\pgfpathlineto{\pgfqpoint{4.442455in}{1.617765in}}%
\pgfpathlineto{\pgfqpoint{4.442873in}{1.714520in}}%
\pgfpathlineto{\pgfqpoint{4.443291in}{1.809204in}}%
\pgfpathlineto{\pgfqpoint{4.443625in}{1.634839in}}%
\pgfpathlineto{\pgfqpoint{4.443875in}{1.763930in}}%
\pgfpathlineto{\pgfqpoint{4.444126in}{1.910948in}}%
\pgfpathlineto{\pgfqpoint{4.444875in}{1.432698in}}%
\pgfpathlineto{\pgfqpoint{4.445375in}{1.828014in}}%
\pgfpathlineto{\pgfqpoint{4.446039in}{1.704887in}}%
\pgfpathlineto{\pgfqpoint{4.446454in}{1.850523in}}%
\pgfpathlineto{\pgfqpoint{4.446786in}{1.595202in}}%
\pgfpathlineto{\pgfqpoint{4.447117in}{1.736673in}}%
\pgfpathlineto{\pgfqpoint{4.447696in}{1.439303in}}%
\pgfpathlineto{\pgfqpoint{4.448027in}{1.823711in}}%
\pgfpathlineto{\pgfqpoint{4.448110in}{1.779276in}}%
\pgfpathlineto{\pgfqpoint{4.448522in}{1.821759in}}%
\pgfpathlineto{\pgfqpoint{4.448357in}{1.500608in}}%
\pgfpathlineto{\pgfqpoint{4.448687in}{1.776017in}}%
\pgfpathlineto{\pgfqpoint{4.449182in}{1.398356in}}%
\pgfpathlineto{\pgfqpoint{4.449264in}{1.858383in}}%
\pgfpathlineto{\pgfqpoint{4.449758in}{1.716791in}}%
\pgfpathlineto{\pgfqpoint{4.449923in}{1.529850in}}%
\pgfpathlineto{\pgfqpoint{4.450005in}{1.839813in}}%
\pgfpathlineto{\pgfqpoint{4.450663in}{1.759218in}}%
\pgfpathlineto{\pgfqpoint{4.450909in}{1.787895in}}%
\pgfpathlineto{\pgfqpoint{4.450991in}{1.588719in}}%
\pgfpathlineto{\pgfqpoint{4.451811in}{1.904409in}}%
\pgfpathlineto{\pgfqpoint{4.451974in}{1.630821in}}%
\pgfpathlineto{\pgfqpoint{4.452302in}{1.573931in}}%
\pgfpathlineto{\pgfqpoint{4.453201in}{1.873506in}}%
\pgfpathlineto{\pgfqpoint{4.454016in}{1.546809in}}%
\pgfpathlineto{\pgfqpoint{4.454342in}{1.657530in}}%
\pgfpathlineto{\pgfqpoint{4.454586in}{1.858436in}}%
\pgfpathlineto{\pgfqpoint{4.454993in}{1.613699in}}%
\pgfpathlineto{\pgfqpoint{4.455480in}{1.858080in}}%
\pgfpathlineto{\pgfqpoint{4.456615in}{1.463163in}}%
\pgfpathlineto{\pgfqpoint{4.456696in}{1.558450in}}%
\pgfpathlineto{\pgfqpoint{4.457424in}{1.855485in}}%
\pgfpathlineto{\pgfqpoint{4.457020in}{1.500380in}}%
\pgfpathlineto{\pgfqpoint{4.457828in}{1.667780in}}%
\pgfpathlineto{\pgfqpoint{4.457909in}{1.582333in}}%
\pgfpathlineto{\pgfqpoint{4.458070in}{1.888022in}}%
\pgfpathlineto{\pgfqpoint{4.458715in}{1.857631in}}%
\pgfpathlineto{\pgfqpoint{4.458796in}{1.838790in}}%
\pgfpathlineto{\pgfqpoint{4.458876in}{1.515787in}}%
\pgfpathlineto{\pgfqpoint{4.459279in}{1.858831in}}%
\pgfpathlineto{\pgfqpoint{4.459842in}{1.657401in}}%
\pgfpathlineto{\pgfqpoint{4.460323in}{1.868188in}}%
\pgfpathlineto{\pgfqpoint{4.460002in}{1.643885in}}%
\pgfpathlineto{\pgfqpoint{4.460965in}{1.769995in}}%
\pgfpathlineto{\pgfqpoint{4.461285in}{1.835118in}}%
\pgfpathlineto{\pgfqpoint{4.461846in}{1.616342in}}%
\pgfpathlineto{\pgfqpoint{4.462165in}{1.915208in}}%
\pgfpathlineto{\pgfqpoint{4.462086in}{1.512117in}}%
\pgfpathlineto{\pgfqpoint{4.462964in}{1.755351in}}%
\pgfpathlineto{\pgfqpoint{4.463761in}{1.598665in}}%
\pgfpathlineto{\pgfqpoint{4.463123in}{1.847927in}}%
\pgfpathlineto{\pgfqpoint{4.464000in}{1.634187in}}%
\pgfpathlineto{\pgfqpoint{4.464795in}{1.900155in}}%
\pgfpathlineto{\pgfqpoint{4.464556in}{1.546707in}}%
\pgfpathlineto{\pgfqpoint{4.465192in}{1.879548in}}%
\pgfpathlineto{\pgfqpoint{4.465509in}{1.627205in}}%
\pgfpathlineto{\pgfqpoint{4.466301in}{1.760933in}}%
\pgfpathlineto{\pgfqpoint{4.466459in}{1.743197in}}%
\pgfpathlineto{\pgfqpoint{4.466776in}{1.677081in}}%
\pgfpathlineto{\pgfqpoint{4.467723in}{1.871217in}}%
\pgfpathlineto{\pgfqpoint{4.468669in}{1.423060in}}%
\pgfpathlineto{\pgfqpoint{4.468826in}{1.835824in}}%
\pgfpathlineto{\pgfqpoint{4.469848in}{1.496956in}}%
\pgfpathlineto{\pgfqpoint{4.469926in}{1.804884in}}%
\pgfpathlineto{\pgfqpoint{4.470632in}{1.568917in}}%
\pgfpathlineto{\pgfqpoint{4.470867in}{1.855512in}}%
\pgfpathlineto{\pgfqpoint{4.470945in}{1.842303in}}%
\pgfpathlineto{\pgfqpoint{4.471023in}{1.882037in}}%
\pgfpathlineto{\pgfqpoint{4.471493in}{1.671164in}}%
\pgfpathlineto{\pgfqpoint{4.471961in}{1.814265in}}%
\pgfpathlineto{\pgfqpoint{4.473053in}{1.562388in}}%
\pgfpathlineto{\pgfqpoint{4.472586in}{1.877384in}}%
\pgfpathlineto{\pgfqpoint{4.473287in}{1.671561in}}%
\pgfpathlineto{\pgfqpoint{4.473987in}{1.873167in}}%
\pgfpathlineto{\pgfqpoint{4.473676in}{1.537557in}}%
\pgfpathlineto{\pgfqpoint{4.474375in}{1.815659in}}%
\pgfpathlineto{\pgfqpoint{4.475151in}{1.527049in}}%
\pgfpathlineto{\pgfqpoint{4.475229in}{1.869229in}}%
\pgfpathlineto{\pgfqpoint{4.475461in}{1.697006in}}%
\pgfpathlineto{\pgfqpoint{4.475616in}{1.673192in}}%
\pgfpathlineto{\pgfqpoint{4.476158in}{1.861059in}}%
\pgfpathlineto{\pgfqpoint{4.476235in}{1.454797in}}%
\pgfpathlineto{\pgfqpoint{4.477316in}{1.546650in}}%
\pgfpathlineto{\pgfqpoint{4.478240in}{1.840071in}}%
\pgfpathlineto{\pgfqpoint{4.478471in}{1.712372in}}%
\pgfpathlineto{\pgfqpoint{4.479009in}{1.874256in}}%
\pgfpathlineto{\pgfqpoint{4.479163in}{1.634382in}}%
\pgfpathlineto{\pgfqpoint{4.479700in}{1.842836in}}%
\pgfpathlineto{\pgfqpoint{4.480006in}{1.553101in}}%
\pgfpathlineto{\pgfqpoint{4.480695in}{1.856887in}}%
\pgfpathlineto{\pgfqpoint{4.480848in}{1.561481in}}%
\pgfpathlineto{\pgfqpoint{4.481231in}{1.920764in}}%
\pgfpathlineto{\pgfqpoint{4.481994in}{1.813286in}}%
\pgfpathlineto{\pgfqpoint{4.482223in}{1.566920in}}%
\pgfpathlineto{\pgfqpoint{4.482832in}{1.846135in}}%
\pgfpathlineto{\pgfqpoint{4.483060in}{1.625304in}}%
\pgfpathlineto{\pgfqpoint{4.483137in}{1.885183in}}%
\pgfpathlineto{\pgfqpoint{4.483896in}{1.391919in}}%
\pgfpathlineto{\pgfqpoint{4.484124in}{1.730118in}}%
\pgfpathlineto{\pgfqpoint{4.484352in}{1.502684in}}%
\pgfpathlineto{\pgfqpoint{4.485034in}{1.843776in}}%
\pgfpathlineto{\pgfqpoint{4.485110in}{1.898341in}}%
\pgfpathlineto{\pgfqpoint{4.485488in}{1.493205in}}%
\pgfpathlineto{\pgfqpoint{4.485639in}{1.845257in}}%
\pgfpathlineto{\pgfqpoint{4.485715in}{1.517199in}}%
\pgfpathlineto{\pgfqpoint{4.486697in}{1.679980in}}%
\pgfpathlineto{\pgfqpoint{4.486847in}{1.875301in}}%
\pgfpathlineto{\pgfqpoint{4.487224in}{1.584211in}}%
\pgfpathlineto{\pgfqpoint{4.487826in}{1.792691in}}%
\pgfpathlineto{\pgfqpoint{4.488278in}{1.696690in}}%
\pgfpathlineto{\pgfqpoint{4.488578in}{1.863671in}}%
\pgfpathlineto{\pgfqpoint{4.488728in}{1.761190in}}%
\pgfpathlineto{\pgfqpoint{4.488803in}{1.898719in}}%
\pgfpathlineto{\pgfqpoint{4.489703in}{1.648724in}}%
\pgfpathlineto{\pgfqpoint{4.489778in}{1.764184in}}%
\pgfpathlineto{\pgfqpoint{4.490451in}{1.484463in}}%
\pgfpathlineto{\pgfqpoint{4.490227in}{1.838252in}}%
\pgfpathlineto{\pgfqpoint{4.490825in}{1.726278in}}%
\pgfpathlineto{\pgfqpoint{4.490900in}{1.860019in}}%
\pgfpathlineto{\pgfqpoint{4.490974in}{1.506653in}}%
\pgfpathlineto{\pgfqpoint{4.491944in}{1.817984in}}%
\pgfpathlineto{\pgfqpoint{4.492167in}{1.754841in}}%
\pgfpathlineto{\pgfqpoint{4.492316in}{1.827341in}}%
\pgfpathlineto{\pgfqpoint{4.492465in}{1.653784in}}%
\pgfpathlineto{\pgfqpoint{4.493357in}{1.839044in}}%
\pgfpathlineto{\pgfqpoint{4.493431in}{1.875488in}}%
\pgfpathlineto{\pgfqpoint{4.493506in}{1.609635in}}%
\pgfpathlineto{\pgfqpoint{4.494321in}{1.832313in}}%
\pgfpathlineto{\pgfqpoint{4.494840in}{1.494340in}}%
\pgfpathlineto{\pgfqpoint{4.495209in}{1.870657in}}%
\pgfpathlineto{\pgfqpoint{4.495505in}{1.526171in}}%
\pgfpathlineto{\pgfqpoint{4.496022in}{1.853015in}}%
\pgfpathlineto{\pgfqpoint{4.496612in}{1.852844in}}%
\pgfpathlineto{\pgfqpoint{4.497422in}{1.540240in}}%
\pgfpathlineto{\pgfqpoint{4.497275in}{1.870394in}}%
\pgfpathlineto{\pgfqpoint{4.497716in}{1.748634in}}%
\pgfpathlineto{\pgfqpoint{4.498597in}{1.892691in}}%
\pgfpathlineto{\pgfqpoint{4.498230in}{1.412930in}}%
\pgfpathlineto{\pgfqpoint{4.498670in}{1.756578in}}%
\pgfpathlineto{\pgfqpoint{4.498743in}{1.570510in}}%
\pgfpathlineto{\pgfqpoint{4.499476in}{1.796543in}}%
\pgfpathlineto{\pgfqpoint{4.499695in}{1.657932in}}%
\pgfpathlineto{\pgfqpoint{4.499769in}{1.862678in}}%
\pgfpathlineto{\pgfqpoint{4.500791in}{1.843821in}}%
\pgfpathlineto{\pgfqpoint{4.500864in}{1.538074in}}%
\pgfpathlineto{\pgfqpoint{4.501083in}{1.853102in}}%
\pgfpathlineto{\pgfqpoint{4.501884in}{1.813730in}}%
\pgfpathlineto{\pgfqpoint{4.502175in}{1.649409in}}%
\pgfpathlineto{\pgfqpoint{4.502321in}{1.882740in}}%
\pgfpathlineto{\pgfqpoint{4.502975in}{1.782874in}}%
\pgfpathlineto{\pgfqpoint{4.503410in}{1.868946in}}%
\pgfpathlineto{\pgfqpoint{4.503192in}{1.617678in}}%
\pgfpathlineto{\pgfqpoint{4.503772in}{1.766569in}}%
\pgfpathlineto{\pgfqpoint{4.504786in}{1.590747in}}%
\pgfpathlineto{\pgfqpoint{4.503990in}{1.858037in}}%
\pgfpathlineto{\pgfqpoint{4.504858in}{1.736898in}}%
\pgfpathlineto{\pgfqpoint{4.505291in}{1.836836in}}%
\pgfpathlineto{\pgfqpoint{4.505002in}{1.571756in}}%
\pgfpathlineto{\pgfqpoint{4.505868in}{1.676791in}}%
\pgfpathlineto{\pgfqpoint{4.506013in}{1.722727in}}%
\pgfpathlineto{\pgfqpoint{4.506085in}{1.801822in}}%
\pgfpathlineto{\pgfqpoint{4.506877in}{1.536630in}}%
\pgfpathlineto{\pgfqpoint{4.507020in}{1.797052in}}%
\pgfpathlineto{\pgfqpoint{4.507380in}{1.879819in}}%
\pgfpathlineto{\pgfqpoint{4.508241in}{1.533356in}}%
\pgfpathlineto{\pgfqpoint{4.508313in}{1.538815in}}%
\pgfpathlineto{\pgfqpoint{4.509458in}{1.914683in}}%
\pgfpathlineto{\pgfqpoint{4.510030in}{1.530622in}}%
\pgfpathlineto{\pgfqpoint{4.510600in}{1.779506in}}%
\pgfpathlineto{\pgfqpoint{4.510814in}{1.754452in}}%
\pgfpathlineto{\pgfqpoint{4.510885in}{1.844203in}}%
\pgfpathlineto{\pgfqpoint{4.511740in}{1.626587in}}%
\pgfpathlineto{\pgfqpoint{4.511882in}{1.820731in}}%
\pgfpathlineto{\pgfqpoint{4.512024in}{1.543507in}}%
\pgfpathlineto{\pgfqpoint{4.512095in}{1.894401in}}%
\pgfpathlineto{\pgfqpoint{4.513018in}{1.734313in}}%
\pgfpathlineto{\pgfqpoint{4.513655in}{1.872263in}}%
\pgfpathlineto{\pgfqpoint{4.513443in}{1.540771in}}%
\pgfpathlineto{\pgfqpoint{4.514080in}{1.826387in}}%
\pgfpathlineto{\pgfqpoint{4.514927in}{1.641240in}}%
\pgfpathlineto{\pgfqpoint{4.514292in}{1.849768in}}%
\pgfpathlineto{\pgfqpoint{4.515210in}{1.778339in}}%
\pgfpathlineto{\pgfqpoint{4.515562in}{1.516465in}}%
\pgfpathlineto{\pgfqpoint{4.515985in}{1.891773in}}%
\pgfpathlineto{\pgfqpoint{4.516266in}{1.763110in}}%
\pgfpathlineto{\pgfqpoint{4.516547in}{1.616505in}}%
\pgfpathlineto{\pgfqpoint{4.517039in}{1.884820in}}%
\pgfpathlineto{\pgfqpoint{4.517741in}{1.606994in}}%
\pgfpathlineto{\pgfqpoint{4.518161in}{1.755698in}}%
\pgfpathlineto{\pgfqpoint{4.518301in}{1.806021in}}%
\pgfpathlineto{\pgfqpoint{4.518371in}{1.802343in}}%
\pgfpathlineto{\pgfqpoint{4.518441in}{1.540873in}}%
\pgfpathlineto{\pgfqpoint{4.518581in}{1.876194in}}%
\pgfpathlineto{\pgfqpoint{4.519420in}{1.694579in}}%
\pgfpathlineto{\pgfqpoint{4.519839in}{1.856865in}}%
\pgfpathlineto{\pgfqpoint{4.520397in}{1.473146in}}%
\pgfpathlineto{\pgfqpoint{4.520536in}{1.730078in}}%
\pgfpathlineto{\pgfqpoint{4.521023in}{1.342860in}}%
\pgfpathlineto{\pgfqpoint{4.521510in}{1.879237in}}%
\pgfpathlineto{\pgfqpoint{4.521579in}{1.708851in}}%
\pgfpathlineto{\pgfqpoint{4.521857in}{1.865564in}}%
\pgfpathlineto{\pgfqpoint{4.521996in}{1.643736in}}%
\pgfpathlineto{\pgfqpoint{4.522620in}{1.714825in}}%
\pgfpathlineto{\pgfqpoint{4.523520in}{1.510255in}}%
\pgfpathlineto{\pgfqpoint{4.522759in}{1.881239in}}%
\pgfpathlineto{\pgfqpoint{4.523659in}{1.770406in}}%
\pgfpathlineto{\pgfqpoint{4.524211in}{1.846390in}}%
\pgfpathlineto{\pgfqpoint{4.524556in}{1.501519in}}%
\pgfpathlineto{\pgfqpoint{4.525246in}{1.887860in}}%
\pgfpathlineto{\pgfqpoint{4.525659in}{1.767418in}}%
\pgfpathlineto{\pgfqpoint{4.525728in}{1.365785in}}%
\pgfpathlineto{\pgfqpoint{4.526003in}{1.863124in}}%
\pgfpathlineto{\pgfqpoint{4.526758in}{1.715789in}}%
\pgfpathlineto{\pgfqpoint{4.527581in}{1.446977in}}%
\pgfpathlineto{\pgfqpoint{4.527239in}{1.907581in}}%
\pgfpathlineto{\pgfqpoint{4.527787in}{1.778368in}}%
\pgfpathlineto{\pgfqpoint{4.527924in}{1.896791in}}%
\pgfpathlineto{\pgfqpoint{4.528744in}{1.681236in}}%
\pgfpathlineto{\pgfqpoint{4.528881in}{1.883672in}}%
\pgfpathlineto{\pgfqpoint{4.529154in}{1.574582in}}%
\pgfpathlineto{\pgfqpoint{4.530040in}{1.609542in}}%
\pgfpathlineto{\pgfqpoint{4.530449in}{1.910036in}}%
\pgfpathlineto{\pgfqpoint{4.530313in}{1.599348in}}%
\pgfpathlineto{\pgfqpoint{4.531264in}{1.844179in}}%
\pgfpathlineto{\pgfqpoint{4.531332in}{1.592555in}}%
\pgfpathlineto{\pgfqpoint{4.531400in}{1.926355in}}%
\pgfpathlineto{\pgfqpoint{4.532349in}{1.812279in}}%
\pgfpathlineto{\pgfqpoint{4.532553in}{1.590155in}}%
\pgfpathlineto{\pgfqpoint{4.533026in}{1.836174in}}%
\pgfpathlineto{\pgfqpoint{4.533364in}{1.825491in}}%
\pgfpathlineto{\pgfqpoint{4.533972in}{1.904336in}}%
\pgfpathlineto{\pgfqpoint{4.534107in}{1.690150in}}%
\pgfpathlineto{\pgfqpoint{4.534444in}{1.860550in}}%
\pgfpathlineto{\pgfqpoint{4.534983in}{1.462079in}}%
\pgfpathlineto{\pgfqpoint{4.535252in}{1.884556in}}%
\pgfpathlineto{\pgfqpoint{4.535521in}{1.785695in}}%
\pgfpathlineto{\pgfqpoint{4.536193in}{1.852458in}}%
\pgfpathlineto{\pgfqpoint{4.536126in}{1.563732in}}%
\pgfpathlineto{\pgfqpoint{4.536260in}{1.721271in}}%
\pgfpathlineto{\pgfqpoint{4.536327in}{1.554330in}}%
\pgfpathlineto{\pgfqpoint{4.536730in}{1.896682in}}%
\pgfpathlineto{\pgfqpoint{4.537333in}{1.637561in}}%
\pgfpathlineto{\pgfqpoint{4.537399in}{1.920437in}}%
\pgfpathlineto{\pgfqpoint{4.538469in}{1.778742in}}%
\pgfpathlineto{\pgfqpoint{4.538536in}{1.858177in}}%
\pgfpathlineto{\pgfqpoint{4.538736in}{1.549728in}}%
\pgfpathlineto{\pgfqpoint{4.539536in}{1.838836in}}%
\pgfpathlineto{\pgfqpoint{4.540401in}{1.618506in}}%
\pgfpathlineto{\pgfqpoint{4.540268in}{1.910575in}}%
\pgfpathlineto{\pgfqpoint{4.540733in}{1.626422in}}%
\pgfpathlineto{\pgfqpoint{4.541927in}{1.910421in}}%
\pgfpathlineto{\pgfqpoint{4.541264in}{1.478689in}}%
\pgfpathlineto{\pgfqpoint{4.542125in}{1.865872in}}%
\pgfpathlineto{\pgfqpoint{4.543249in}{1.524684in}}%
\pgfpathlineto{\pgfqpoint{4.542390in}{1.876275in}}%
\pgfpathlineto{\pgfqpoint{4.543315in}{1.639029in}}%
\pgfpathlineto{\pgfqpoint{4.544041in}{1.559355in}}%
\pgfpathlineto{\pgfqpoint{4.544436in}{1.861643in}}%
\pgfpathlineto{\pgfqpoint{4.544962in}{1.538131in}}%
\pgfpathlineto{\pgfqpoint{4.545225in}{1.874393in}}%
\pgfpathlineto{\pgfqpoint{4.545554in}{1.669351in}}%
\pgfpathlineto{\pgfqpoint{4.546210in}{1.551702in}}%
\pgfpathlineto{\pgfqpoint{4.546734in}{1.868984in}}%
\pgfpathlineto{\pgfqpoint{4.547323in}{1.543055in}}%
\pgfpathlineto{\pgfqpoint{4.547453in}{1.871721in}}%
\pgfpathlineto{\pgfqpoint{4.547911in}{1.630128in}}%
\pgfpathlineto{\pgfqpoint{4.548172in}{1.920802in}}%
\pgfpathlineto{\pgfqpoint{4.549019in}{1.656042in}}%
\pgfpathlineto{\pgfqpoint{4.549865in}{1.894324in}}%
\pgfpathlineto{\pgfqpoint{4.549735in}{1.617521in}}%
\pgfpathlineto{\pgfqpoint{4.549995in}{1.747183in}}%
\pgfpathlineto{\pgfqpoint{4.550060in}{1.335361in}}%
\pgfpathlineto{\pgfqpoint{4.550969in}{1.878526in}}%
\pgfpathlineto{\pgfqpoint{4.551098in}{1.709456in}}%
\pgfpathlineto{\pgfqpoint{4.551487in}{1.881586in}}%
\pgfpathlineto{\pgfqpoint{4.551681in}{1.644933in}}%
\pgfpathlineto{\pgfqpoint{4.552263in}{1.856259in}}%
\pgfpathlineto{\pgfqpoint{4.553167in}{1.569844in}}%
\pgfpathlineto{\pgfqpoint{4.552651in}{1.900102in}}%
\pgfpathlineto{\pgfqpoint{4.553360in}{1.663868in}}%
\pgfpathlineto{\pgfqpoint{4.553940in}{1.922854in}}%
\pgfpathlineto{\pgfqpoint{4.553747in}{1.474387in}}%
\pgfpathlineto{\pgfqpoint{4.554519in}{1.859616in}}%
\pgfpathlineto{\pgfqpoint{4.554648in}{1.622936in}}%
\pgfpathlineto{\pgfqpoint{4.555290in}{1.879275in}}%
\pgfpathlineto{\pgfqpoint{4.555675in}{1.833295in}}%
\pgfpathlineto{\pgfqpoint{4.555931in}{1.854736in}}%
\pgfpathlineto{\pgfqpoint{4.555867in}{1.606457in}}%
\pgfpathlineto{\pgfqpoint{4.556251in}{1.817425in}}%
\pgfpathlineto{\pgfqpoint{4.556444in}{1.418952in}}%
\pgfpathlineto{\pgfqpoint{4.557211in}{1.842491in}}%
\pgfpathlineto{\pgfqpoint{4.557339in}{1.665726in}}%
\pgfpathlineto{\pgfqpoint{4.557402in}{1.929644in}}%
\pgfpathlineto{\pgfqpoint{4.558168in}{1.590766in}}%
\pgfpathlineto{\pgfqpoint{4.558423in}{1.744643in}}%
\pgfpathlineto{\pgfqpoint{4.558486in}{1.745440in}}%
\pgfpathlineto{\pgfqpoint{4.558805in}{1.470343in}}%
\pgfpathlineto{\pgfqpoint{4.558614in}{1.892881in}}%
\pgfpathlineto{\pgfqpoint{4.559441in}{1.716525in}}%
\pgfpathlineto{\pgfqpoint{4.559758in}{1.953857in}}%
\pgfpathlineto{\pgfqpoint{4.559822in}{1.539954in}}%
\pgfpathlineto{\pgfqpoint{4.560520in}{1.742695in}}%
\pgfpathlineto{\pgfqpoint{4.560900in}{1.888665in}}%
\pgfpathlineto{\pgfqpoint{4.560773in}{1.696613in}}%
\pgfpathlineto{\pgfqpoint{4.560963in}{1.827507in}}%
\pgfpathlineto{\pgfqpoint{4.561026in}{1.313642in}}%
\pgfpathlineto{\pgfqpoint{4.561975in}{1.895558in}}%
\pgfpathlineto{\pgfqpoint{4.562038in}{1.563685in}}%
\pgfpathlineto{\pgfqpoint{4.562164in}{1.906997in}}%
\pgfpathlineto{\pgfqpoint{4.563111in}{1.668816in}}%
\pgfpathlineto{\pgfqpoint{4.563488in}{1.490110in}}%
\pgfpathlineto{\pgfqpoint{4.564054in}{1.881069in}}%
\pgfpathlineto{\pgfqpoint{4.564180in}{1.517554in}}%
\pgfpathlineto{\pgfqpoint{4.564243in}{1.896008in}}%
\pgfpathlineto{\pgfqpoint{4.565310in}{1.720872in}}%
\pgfpathlineto{\pgfqpoint{4.565936in}{1.904344in}}%
\pgfpathlineto{\pgfqpoint{4.565560in}{1.554120in}}%
\pgfpathlineto{\pgfqpoint{4.566436in}{1.801485in}}%
\pgfpathlineto{\pgfqpoint{4.567248in}{1.627969in}}%
\pgfpathlineto{\pgfqpoint{4.566749in}{1.877939in}}%
\pgfpathlineto{\pgfqpoint{4.567560in}{1.641255in}}%
\pgfpathlineto{\pgfqpoint{4.568494in}{1.846049in}}%
\pgfpathlineto{\pgfqpoint{4.568245in}{1.582816in}}%
\pgfpathlineto{\pgfqpoint{4.568619in}{1.824607in}}%
\pgfpathlineto{\pgfqpoint{4.568681in}{1.472060in}}%
\pgfpathlineto{\pgfqpoint{4.568805in}{1.916820in}}%
\pgfpathlineto{\pgfqpoint{4.569736in}{1.593169in}}%
\pgfpathlineto{\pgfqpoint{4.570480in}{1.908720in}}%
\pgfpathlineto{\pgfqpoint{4.570913in}{1.764207in}}%
\pgfpathlineto{\pgfqpoint{4.570975in}{1.588643in}}%
\pgfpathlineto{\pgfqpoint{4.571717in}{1.926430in}}%
\pgfpathlineto{\pgfqpoint{4.572025in}{1.696445in}}%
\pgfpathlineto{\pgfqpoint{4.572949in}{1.899478in}}%
\pgfpathlineto{\pgfqpoint{4.572457in}{1.587926in}}%
\pgfpathlineto{\pgfqpoint{4.573073in}{1.756029in}}%
\pgfpathlineto{\pgfqpoint{4.573196in}{1.596840in}}%
\pgfpathlineto{\pgfqpoint{4.573933in}{1.856466in}}%
\pgfpathlineto{\pgfqpoint{4.574117in}{1.834571in}}%
\pgfpathlineto{\pgfqpoint{4.574179in}{1.832949in}}%
\pgfpathlineto{\pgfqpoint{4.574240in}{1.845210in}}%
\pgfpathlineto{\pgfqpoint{4.574424in}{1.611470in}}%
\pgfpathlineto{\pgfqpoint{4.574792in}{1.888733in}}%
\pgfpathlineto{\pgfqpoint{4.575282in}{1.693741in}}%
\pgfpathlineto{\pgfqpoint{4.575343in}{1.922572in}}%
\pgfpathlineto{\pgfqpoint{4.575588in}{1.570546in}}%
\pgfpathlineto{\pgfqpoint{4.576383in}{1.845114in}}%
\pgfpathlineto{\pgfqpoint{4.576444in}{1.837708in}}%
\pgfpathlineto{\pgfqpoint{4.577358in}{1.543437in}}%
\pgfpathlineto{\pgfqpoint{4.576932in}{1.907454in}}%
\pgfpathlineto{\pgfqpoint{4.577480in}{1.674394in}}%
\pgfpathlineto{\pgfqpoint{4.577906in}{1.923502in}}%
\pgfpathlineto{\pgfqpoint{4.577846in}{1.641477in}}%
\pgfpathlineto{\pgfqpoint{4.578575in}{1.741091in}}%
\pgfpathlineto{\pgfqpoint{4.579182in}{1.668794in}}%
\pgfpathlineto{\pgfqpoint{4.579667in}{1.859839in}}%
\pgfpathlineto{\pgfqpoint{4.580212in}{1.590953in}}%
\pgfpathlineto{\pgfqpoint{4.580152in}{1.887717in}}%
\pgfpathlineto{\pgfqpoint{4.580756in}{1.871741in}}%
\pgfpathlineto{\pgfqpoint{4.581179in}{1.599301in}}%
\pgfpathlineto{\pgfqpoint{4.581360in}{1.904930in}}%
\pgfpathlineto{\pgfqpoint{4.581903in}{1.774978in}}%
\pgfpathlineto{\pgfqpoint{4.582204in}{1.880290in}}%
\pgfpathlineto{\pgfqpoint{4.582023in}{1.582382in}}%
\pgfpathlineto{\pgfqpoint{4.582926in}{1.854095in}}%
\pgfpathlineto{\pgfqpoint{4.582986in}{1.588516in}}%
\pgfpathlineto{\pgfqpoint{4.583947in}{1.926377in}}%
\pgfpathlineto{\pgfqpoint{4.584007in}{1.891828in}}%
\pgfpathlineto{\pgfqpoint{4.585025in}{1.448275in}}%
\pgfpathlineto{\pgfqpoint{4.584786in}{1.918282in}}%
\pgfpathlineto{\pgfqpoint{4.585265in}{1.621229in}}%
\pgfpathlineto{\pgfqpoint{4.585325in}{1.552297in}}%
\pgfpathlineto{\pgfqpoint{4.585922in}{1.879897in}}%
\pgfpathlineto{\pgfqpoint{4.586220in}{1.754499in}}%
\pgfpathlineto{\pgfqpoint{4.587412in}{1.914635in}}%
\pgfpathlineto{\pgfqpoint{4.587055in}{1.580964in}}%
\pgfpathlineto{\pgfqpoint{4.587471in}{1.900236in}}%
\pgfpathlineto{\pgfqpoint{4.588303in}{1.609549in}}%
\pgfpathlineto{\pgfqpoint{4.588600in}{1.654757in}}%
\pgfpathlineto{\pgfqpoint{4.589430in}{1.944396in}}%
\pgfpathlineto{\pgfqpoint{4.589311in}{1.596271in}}%
\pgfpathlineto{\pgfqpoint{4.589844in}{1.889261in}}%
\pgfpathlineto{\pgfqpoint{4.590671in}{1.571067in}}%
\pgfpathlineto{\pgfqpoint{4.590435in}{1.897951in}}%
\pgfpathlineto{\pgfqpoint{4.590908in}{1.884991in}}%
\pgfpathlineto{\pgfqpoint{4.591085in}{1.922541in}}%
\pgfpathlineto{\pgfqpoint{4.591026in}{1.570729in}}%
\pgfpathlineto{\pgfqpoint{4.591497in}{1.828680in}}%
\pgfpathlineto{\pgfqpoint{4.592439in}{1.605598in}}%
\pgfpathlineto{\pgfqpoint{4.591909in}{1.915722in}}%
\pgfpathlineto{\pgfqpoint{4.592615in}{1.721715in}}%
\pgfpathlineto{\pgfqpoint{4.593203in}{1.971566in}}%
\pgfpathlineto{\pgfqpoint{4.592909in}{1.541144in}}%
\pgfpathlineto{\pgfqpoint{4.593730in}{1.789034in}}%
\pgfpathlineto{\pgfqpoint{4.594375in}{1.484166in}}%
\pgfpathlineto{\pgfqpoint{4.593965in}{1.819521in}}%
\pgfpathlineto{\pgfqpoint{4.594726in}{1.807202in}}%
\pgfpathlineto{\pgfqpoint{4.595135in}{1.573433in}}%
\pgfpathlineto{\pgfqpoint{4.595544in}{1.895325in}}%
\pgfpathlineto{\pgfqpoint{4.595602in}{1.602233in}}%
\pgfpathlineto{\pgfqpoint{4.596651in}{1.791937in}}%
\pgfpathlineto{\pgfqpoint{4.596884in}{1.567141in}}%
\pgfpathlineto{\pgfqpoint{4.597349in}{1.930028in}}%
\pgfpathlineto{\pgfqpoint{4.597756in}{1.799597in}}%
\pgfpathlineto{\pgfqpoint{4.598452in}{1.453884in}}%
\pgfpathlineto{\pgfqpoint{4.598799in}{1.932003in}}%
\pgfpathlineto{\pgfqpoint{4.599898in}{1.611915in}}%
\pgfpathlineto{\pgfqpoint{4.600303in}{1.885701in}}%
\pgfpathlineto{\pgfqpoint{4.600187in}{1.574748in}}%
\pgfpathlineto{\pgfqpoint{4.600995in}{1.706417in}}%
\pgfpathlineto{\pgfqpoint{4.601052in}{1.632221in}}%
\pgfpathlineto{\pgfqpoint{4.601800in}{1.911599in}}%
\pgfpathlineto{\pgfqpoint{4.602030in}{1.773957in}}%
\pgfpathlineto{\pgfqpoint{4.602490in}{1.935125in}}%
\pgfpathlineto{\pgfqpoint{4.602777in}{1.670588in}}%
\pgfpathlineto{\pgfqpoint{4.603006in}{1.854361in}}%
\pgfpathlineto{\pgfqpoint{4.603694in}{1.682425in}}%
\pgfpathlineto{\pgfqpoint{4.603236in}{1.891011in}}%
\pgfpathlineto{\pgfqpoint{4.604095in}{1.880735in}}%
\pgfpathlineto{\pgfqpoint{4.605123in}{1.688807in}}%
\pgfpathlineto{\pgfqpoint{4.604838in}{1.938808in}}%
\pgfpathlineto{\pgfqpoint{4.605237in}{1.828569in}}%
\pgfpathlineto{\pgfqpoint{4.606035in}{1.488929in}}%
\pgfpathlineto{\pgfqpoint{4.605351in}{1.901355in}}%
\pgfpathlineto{\pgfqpoint{4.606320in}{1.607270in}}%
\pgfpathlineto{\pgfqpoint{4.606604in}{1.915270in}}%
\pgfpathlineto{\pgfqpoint{4.607456in}{1.715801in}}%
\pgfpathlineto{\pgfqpoint{4.607966in}{1.928181in}}%
\pgfpathlineto{\pgfqpoint{4.607910in}{1.657082in}}%
\pgfpathlineto{\pgfqpoint{4.608533in}{1.864052in}}%
\pgfpathlineto{\pgfqpoint{4.608646in}{1.551632in}}%
\pgfpathlineto{\pgfqpoint{4.608872in}{1.872241in}}%
\pgfpathlineto{\pgfqpoint{4.609663in}{1.685696in}}%
\pgfpathlineto{\pgfqpoint{4.610453in}{1.935837in}}%
\pgfpathlineto{\pgfqpoint{4.610002in}{1.468923in}}%
\pgfpathlineto{\pgfqpoint{4.610735in}{1.883993in}}%
\pgfpathlineto{\pgfqpoint{4.611129in}{1.397155in}}%
\pgfpathlineto{\pgfqpoint{4.611579in}{1.904333in}}%
\pgfpathlineto{\pgfqpoint{4.611803in}{1.706758in}}%
\pgfpathlineto{\pgfqpoint{4.612084in}{1.920828in}}%
\pgfpathlineto{\pgfqpoint{4.611916in}{1.672375in}}%
\pgfpathlineto{\pgfqpoint{4.612925in}{1.842743in}}%
\pgfpathlineto{\pgfqpoint{4.613149in}{1.946505in}}%
\pgfpathlineto{\pgfqpoint{4.613317in}{1.671129in}}%
\pgfpathlineto{\pgfqpoint{4.613932in}{1.836721in}}%
\pgfpathlineto{\pgfqpoint{4.613988in}{1.581072in}}%
\pgfpathlineto{\pgfqpoint{4.614658in}{1.937779in}}%
\pgfpathlineto{\pgfqpoint{4.615049in}{1.740132in}}%
\pgfpathlineto{\pgfqpoint{4.615995in}{1.911905in}}%
\pgfpathlineto{\pgfqpoint{4.615661in}{1.635735in}}%
\pgfpathlineto{\pgfqpoint{4.616162in}{1.867358in}}%
\pgfpathlineto{\pgfqpoint{4.617217in}{1.570509in}}%
\pgfpathlineto{\pgfqpoint{4.616329in}{1.878195in}}%
\pgfpathlineto{\pgfqpoint{4.617273in}{1.864898in}}%
\pgfpathlineto{\pgfqpoint{4.618159in}{1.602929in}}%
\pgfpathlineto{\pgfqpoint{4.617383in}{1.894263in}}%
\pgfpathlineto{\pgfqpoint{4.618380in}{1.817421in}}%
\pgfpathlineto{\pgfqpoint{4.619154in}{1.963296in}}%
\pgfpathlineto{\pgfqpoint{4.618988in}{1.548309in}}%
\pgfpathlineto{\pgfqpoint{4.619430in}{1.827192in}}%
\pgfpathlineto{\pgfqpoint{4.620091in}{1.926153in}}%
\pgfpathlineto{\pgfqpoint{4.620587in}{1.445596in}}%
\pgfpathlineto{\pgfqpoint{4.621356in}{1.908379in}}%
\pgfpathlineto{\pgfqpoint{4.621740in}{1.896113in}}%
\pgfpathlineto{\pgfqpoint{4.622124in}{1.584709in}}%
\pgfpathlineto{\pgfqpoint{4.622563in}{1.932131in}}%
\pgfpathlineto{\pgfqpoint{4.622836in}{1.815877in}}%
\pgfpathlineto{\pgfqpoint{4.623001in}{1.955506in}}%
\pgfpathlineto{\pgfqpoint{4.623329in}{1.669875in}}%
\pgfpathlineto{\pgfqpoint{4.623711in}{1.870345in}}%
\pgfpathlineto{\pgfqpoint{4.624148in}{1.525779in}}%
\pgfpathlineto{\pgfqpoint{4.624257in}{1.909960in}}%
\pgfpathlineto{\pgfqpoint{4.624802in}{1.653021in}}%
\pgfpathlineto{\pgfqpoint{4.625238in}{1.889558in}}%
\pgfpathlineto{\pgfqpoint{4.624911in}{1.628980in}}%
\pgfpathlineto{\pgfqpoint{4.625890in}{1.764418in}}%
\pgfpathlineto{\pgfqpoint{4.625944in}{1.611995in}}%
\pgfpathlineto{\pgfqpoint{4.626053in}{1.931065in}}%
\pgfpathlineto{\pgfqpoint{4.626921in}{1.879145in}}%
\pgfpathlineto{\pgfqpoint{4.627950in}{1.648854in}}%
\pgfpathlineto{\pgfqpoint{4.627409in}{1.939160in}}%
\pgfpathlineto{\pgfqpoint{4.628004in}{1.862549in}}%
\pgfpathlineto{\pgfqpoint{4.628382in}{1.921989in}}%
\pgfpathlineto{\pgfqpoint{4.628274in}{1.778774in}}%
\pgfpathlineto{\pgfqpoint{4.628436in}{1.877006in}}%
\pgfpathlineto{\pgfqpoint{4.628490in}{1.642794in}}%
\pgfpathlineto{\pgfqpoint{4.629461in}{1.899808in}}%
\pgfpathlineto{\pgfqpoint{4.629515in}{1.881786in}}%
\pgfpathlineto{\pgfqpoint{4.629569in}{1.925148in}}%
\pgfpathlineto{\pgfqpoint{4.630215in}{1.586258in}}%
\pgfpathlineto{\pgfqpoint{4.630430in}{1.789100in}}%
\pgfpathlineto{\pgfqpoint{4.630752in}{1.609598in}}%
\pgfpathlineto{\pgfqpoint{4.630806in}{1.917151in}}%
\pgfpathlineto{\pgfqpoint{4.631450in}{1.833856in}}%
\pgfpathlineto{\pgfqpoint{4.632093in}{1.925847in}}%
\pgfpathlineto{\pgfqpoint{4.631772in}{1.668380in}}%
\pgfpathlineto{\pgfqpoint{4.632521in}{1.910335in}}%
\pgfpathlineto{\pgfqpoint{4.633109in}{1.402537in}}%
\pgfpathlineto{\pgfqpoint{4.632896in}{1.964072in}}%
\pgfpathlineto{\pgfqpoint{4.633590in}{1.886382in}}%
\pgfpathlineto{\pgfqpoint{4.633857in}{1.948505in}}%
\pgfpathlineto{\pgfqpoint{4.634230in}{1.600526in}}%
\pgfpathlineto{\pgfqpoint{4.634496in}{1.792260in}}%
\pgfpathlineto{\pgfqpoint{4.635453in}{1.611037in}}%
\pgfpathlineto{\pgfqpoint{4.635400in}{1.894715in}}%
\pgfpathlineto{\pgfqpoint{4.635560in}{1.802271in}}%
\pgfpathlineto{\pgfqpoint{4.635719in}{1.947816in}}%
\pgfpathlineto{\pgfqpoint{4.635772in}{1.648289in}}%
\pgfpathlineto{\pgfqpoint{4.636673in}{1.857778in}}%
\pgfpathlineto{\pgfqpoint{4.637150in}{1.520375in}}%
\pgfpathlineto{\pgfqpoint{4.637362in}{1.914074in}}%
\pgfpathlineto{\pgfqpoint{4.637679in}{1.881057in}}%
\pgfpathlineto{\pgfqpoint{4.637837in}{1.911342in}}%
\pgfpathlineto{\pgfqpoint{4.638101in}{1.686060in}}%
\pgfpathlineto{\pgfqpoint{4.638154in}{1.476473in}}%
\pgfpathlineto{\pgfqpoint{4.638207in}{1.883166in}}%
\pgfpathlineto{\pgfqpoint{4.639103in}{1.828613in}}%
\pgfpathlineto{\pgfqpoint{4.639156in}{1.929922in}}%
\pgfpathlineto{\pgfqpoint{4.639577in}{1.704829in}}%
\pgfpathlineto{\pgfqpoint{4.640103in}{1.832626in}}%
\pgfpathlineto{\pgfqpoint{4.640418in}{1.567319in}}%
\pgfpathlineto{\pgfqpoint{4.641100in}{1.921848in}}%
\pgfpathlineto{\pgfqpoint{4.641257in}{1.731168in}}%
\pgfpathlineto{\pgfqpoint{4.641310in}{1.716953in}}%
\pgfpathlineto{\pgfqpoint{4.641572in}{1.838579in}}%
\pgfpathlineto{\pgfqpoint{4.641781in}{1.786996in}}%
\pgfpathlineto{\pgfqpoint{4.642565in}{1.951374in}}%
\pgfpathlineto{\pgfqpoint{4.642356in}{1.571415in}}%
\pgfpathlineto{\pgfqpoint{4.642826in}{1.856228in}}%
\pgfpathlineto{\pgfqpoint{4.643557in}{1.689223in}}%
\pgfpathlineto{\pgfqpoint{4.643817in}{1.931433in}}%
\pgfpathlineto{\pgfqpoint{4.643921in}{1.812751in}}%
\pgfpathlineto{\pgfqpoint{4.643974in}{1.818370in}}%
\pgfpathlineto{\pgfqpoint{4.644078in}{1.789960in}}%
\pgfpathlineto{\pgfqpoint{4.644754in}{1.527741in}}%
\pgfpathlineto{\pgfqpoint{4.645014in}{1.914958in}}%
\pgfpathlineto{\pgfqpoint{4.645117in}{1.820455in}}%
\pgfpathlineto{\pgfqpoint{4.645429in}{1.936521in}}%
\pgfpathlineto{\pgfqpoint{4.645377in}{1.748891in}}%
\pgfpathlineto{\pgfqpoint{4.645896in}{1.885537in}}%
\pgfpathlineto{\pgfqpoint{4.645948in}{1.455475in}}%
\pgfpathlineto{\pgfqpoint{4.645999in}{1.925654in}}%
\pgfpathlineto{\pgfqpoint{4.646983in}{1.845492in}}%
\pgfpathlineto{\pgfqpoint{4.647396in}{1.375609in}}%
\pgfpathlineto{\pgfqpoint{4.647706in}{1.930605in}}%
\pgfpathlineto{\pgfqpoint{4.648119in}{1.777574in}}%
\pgfpathlineto{\pgfqpoint{4.648428in}{1.638348in}}%
\pgfpathlineto{\pgfqpoint{4.648325in}{1.872652in}}%
\pgfpathlineto{\pgfqpoint{4.648583in}{1.700023in}}%
\pgfpathlineto{\pgfqpoint{4.648634in}{1.970764in}}%
\pgfpathlineto{\pgfqpoint{4.648840in}{1.550773in}}%
\pgfpathlineto{\pgfqpoint{4.649663in}{1.935324in}}%
\pgfpathlineto{\pgfqpoint{4.649714in}{1.649394in}}%
\pgfpathlineto{\pgfqpoint{4.650792in}{1.803195in}}%
\pgfpathlineto{\pgfqpoint{4.651662in}{1.942610in}}%
\pgfpathlineto{\pgfqpoint{4.651559in}{1.482281in}}%
\pgfpathlineto{\pgfqpoint{4.651866in}{1.783901in}}%
\pgfpathlineto{\pgfqpoint{4.653040in}{1.915047in}}%
\pgfpathlineto{\pgfqpoint{4.652479in}{1.642244in}}%
\pgfpathlineto{\pgfqpoint{4.653091in}{1.891064in}}%
\pgfpathlineto{\pgfqpoint{4.653753in}{1.940973in}}%
\pgfpathlineto{\pgfqpoint{4.654160in}{1.605991in}}%
\pgfpathlineto{\pgfqpoint{4.655226in}{1.942593in}}%
\pgfpathlineto{\pgfqpoint{4.655175in}{1.537589in}}%
\pgfpathlineto{\pgfqpoint{4.655277in}{1.887773in}}%
\pgfpathlineto{\pgfqpoint{4.656239in}{1.733913in}}%
\pgfpathlineto{\pgfqpoint{4.656138in}{1.930540in}}%
\pgfpathlineto{\pgfqpoint{4.656391in}{1.842168in}}%
\pgfpathlineto{\pgfqpoint{4.656896in}{1.923452in}}%
\pgfpathlineto{\pgfqpoint{4.657451in}{1.620995in}}%
\pgfpathlineto{\pgfqpoint{4.658107in}{1.950692in}}%
\pgfpathlineto{\pgfqpoint{4.658207in}{1.343331in}}%
\pgfpathlineto{\pgfqpoint{4.658560in}{1.672081in}}%
\pgfpathlineto{\pgfqpoint{4.659464in}{1.932852in}}%
\pgfpathlineto{\pgfqpoint{4.659514in}{1.222278in}}%
\pgfpathlineto{\pgfqpoint{4.659665in}{1.792341in}}%
\pgfpathlineto{\pgfqpoint{4.659765in}{1.948773in}}%
\pgfpathlineto{\pgfqpoint{4.660066in}{1.735245in}}%
\pgfpathlineto{\pgfqpoint{4.660166in}{1.797150in}}%
\pgfpathlineto{\pgfqpoint{4.660417in}{1.571394in}}%
\pgfpathlineto{\pgfqpoint{4.660968in}{1.951367in}}%
\pgfpathlineto{\pgfqpoint{4.661268in}{1.813781in}}%
\pgfpathlineto{\pgfqpoint{4.661368in}{1.911372in}}%
\pgfpathlineto{\pgfqpoint{4.661468in}{1.723482in}}%
\pgfpathlineto{\pgfqpoint{4.661917in}{1.980331in}}%
\pgfpathlineto{\pgfqpoint{4.661867in}{1.437203in}}%
\pgfpathlineto{\pgfqpoint{4.662565in}{1.889185in}}%
\pgfpathlineto{\pgfqpoint{4.662715in}{1.637055in}}%
\pgfpathlineto{\pgfqpoint{4.663064in}{1.977250in}}%
\pgfpathlineto{\pgfqpoint{4.663660in}{1.809553in}}%
\pgfpathlineto{\pgfqpoint{4.664058in}{1.945329in}}%
\pgfpathlineto{\pgfqpoint{4.664356in}{1.388273in}}%
\pgfpathlineto{\pgfqpoint{4.664753in}{1.900300in}}%
\pgfpathlineto{\pgfqpoint{4.665149in}{1.553806in}}%
\pgfpathlineto{\pgfqpoint{4.664852in}{1.946832in}}%
\pgfpathlineto{\pgfqpoint{4.665842in}{1.865702in}}%
\pgfpathlineto{\pgfqpoint{4.666040in}{1.922377in}}%
\pgfpathlineto{\pgfqpoint{4.666336in}{1.607618in}}%
\pgfpathlineto{\pgfqpoint{4.666682in}{1.860747in}}%
\pgfpathlineto{\pgfqpoint{4.666731in}{1.483392in}}%
\pgfpathlineto{\pgfqpoint{4.667471in}{1.947992in}}%
\pgfpathlineto{\pgfqpoint{4.667816in}{1.623031in}}%
\pgfpathlineto{\pgfqpoint{4.667865in}{1.939024in}}%
\pgfpathlineto{\pgfqpoint{4.668209in}{1.567971in}}%
\pgfpathlineto{\pgfqpoint{4.668946in}{1.804226in}}%
\pgfpathlineto{\pgfqpoint{4.668995in}{1.586812in}}%
\pgfpathlineto{\pgfqpoint{4.669780in}{1.931388in}}%
\pgfpathlineto{\pgfqpoint{4.670025in}{1.820822in}}%
\pgfpathlineto{\pgfqpoint{4.670857in}{1.692431in}}%
\pgfpathlineto{\pgfqpoint{4.670368in}{1.956892in}}%
\pgfpathlineto{\pgfqpoint{4.671003in}{1.930770in}}%
\pgfpathlineto{\pgfqpoint{4.671052in}{1.936209in}}%
\pgfpathlineto{\pgfqpoint{4.671491in}{1.500110in}}%
\pgfpathlineto{\pgfqpoint{4.672174in}{1.659675in}}%
\pgfpathlineto{\pgfqpoint{4.673002in}{1.891753in}}%
\pgfpathlineto{\pgfqpoint{4.672515in}{1.595144in}}%
\pgfpathlineto{\pgfqpoint{4.673293in}{1.866043in}}%
\pgfpathlineto{\pgfqpoint{4.673439in}{1.756462in}}%
\pgfpathlineto{\pgfqpoint{4.674264in}{1.968977in}}%
\pgfpathlineto{\pgfqpoint{4.674312in}{1.899111in}}%
\pgfpathlineto{\pgfqpoint{4.674361in}{1.960966in}}%
\pgfpathlineto{\pgfqpoint{4.674991in}{1.568039in}}%
\pgfpathlineto{\pgfqpoint{4.675329in}{1.887798in}}%
\pgfpathlineto{\pgfqpoint{4.675378in}{1.679541in}}%
\pgfpathlineto{\pgfqpoint{4.676247in}{1.917858in}}%
\pgfpathlineto{\pgfqpoint{4.676440in}{1.837685in}}%
\pgfpathlineto{\pgfqpoint{4.677067in}{1.957069in}}%
\pgfpathlineto{\pgfqpoint{4.676585in}{1.688730in}}%
\pgfpathlineto{\pgfqpoint{4.677355in}{1.785505in}}%
\pgfpathlineto{\pgfqpoint{4.678317in}{1.552340in}}%
\pgfpathlineto{\pgfqpoint{4.678125in}{1.939175in}}%
\pgfpathlineto{\pgfqpoint{4.678365in}{1.802564in}}%
\pgfpathlineto{\pgfqpoint{4.678845in}{1.917779in}}%
\pgfpathlineto{\pgfqpoint{4.678461in}{1.614057in}}%
\pgfpathlineto{\pgfqpoint{4.679372in}{1.792801in}}%
\pgfpathlineto{\pgfqpoint{4.679612in}{1.395948in}}%
\pgfpathlineto{\pgfqpoint{4.680138in}{1.925470in}}%
\pgfpathlineto{\pgfqpoint{4.680425in}{1.866028in}}%
\pgfpathlineto{\pgfqpoint{4.681189in}{1.523371in}}%
\pgfpathlineto{\pgfqpoint{4.680902in}{1.934492in}}%
\pgfpathlineto{\pgfqpoint{4.681522in}{1.833834in}}%
\pgfpathlineto{\pgfqpoint{4.682474in}{1.949769in}}%
\pgfpathlineto{\pgfqpoint{4.682094in}{1.660637in}}%
\pgfpathlineto{\pgfqpoint{4.682617in}{1.912370in}}%
\pgfpathlineto{\pgfqpoint{4.682997in}{1.946012in}}%
\pgfpathlineto{\pgfqpoint{4.683804in}{1.642328in}}%
\pgfpathlineto{\pgfqpoint{4.684798in}{1.933839in}}%
\pgfpathlineto{\pgfqpoint{4.684041in}{1.639561in}}%
\pgfpathlineto{\pgfqpoint{4.684987in}{1.819829in}}%
\pgfpathlineto{\pgfqpoint{4.685082in}{1.759677in}}%
\pgfpathlineto{\pgfqpoint{4.685129in}{1.837570in}}%
\pgfpathlineto{\pgfqpoint{4.685176in}{1.957368in}}%
\pgfpathlineto{\pgfqpoint{4.685837in}{1.554084in}}%
\pgfpathlineto{\pgfqpoint{4.686120in}{1.869970in}}%
\pgfpathlineto{\pgfqpoint{4.687062in}{1.594267in}}%
\pgfpathlineto{\pgfqpoint{4.686403in}{1.934340in}}%
\pgfpathlineto{\pgfqpoint{4.687203in}{1.893303in}}%
\pgfpathlineto{\pgfqpoint{4.687344in}{1.509497in}}%
\pgfpathlineto{\pgfqpoint{4.688002in}{1.970472in}}%
\pgfpathlineto{\pgfqpoint{4.688330in}{1.806090in}}%
\pgfpathlineto{\pgfqpoint{4.689314in}{1.950963in}}%
\pgfpathlineto{\pgfqpoint{4.689174in}{1.525727in}}%
\pgfpathlineto{\pgfqpoint{4.689361in}{1.771312in}}%
\pgfpathlineto{\pgfqpoint{4.689408in}{1.583280in}}%
\pgfpathlineto{\pgfqpoint{4.689875in}{1.925348in}}%
\pgfpathlineto{\pgfqpoint{4.690389in}{1.778080in}}%
\pgfpathlineto{\pgfqpoint{4.690436in}{1.977712in}}%
\pgfpathlineto{\pgfqpoint{4.691275in}{1.668663in}}%
\pgfpathlineto{\pgfqpoint{4.691508in}{1.835403in}}%
\pgfpathlineto{\pgfqpoint{4.692531in}{1.967438in}}%
\pgfpathlineto{\pgfqpoint{4.691973in}{1.609886in}}%
\pgfpathlineto{\pgfqpoint{4.692623in}{1.871124in}}%
\pgfpathlineto{\pgfqpoint{4.693690in}{1.529040in}}%
\pgfpathlineto{\pgfqpoint{4.693597in}{1.942343in}}%
\pgfpathlineto{\pgfqpoint{4.693736in}{1.785056in}}%
\pgfpathlineto{\pgfqpoint{4.694292in}{1.930147in}}%
\pgfpathlineto{\pgfqpoint{4.694199in}{1.590227in}}%
\pgfpathlineto{\pgfqpoint{4.694800in}{1.798403in}}%
\pgfpathlineto{\pgfqpoint{4.694892in}{1.972535in}}%
\pgfpathlineto{\pgfqpoint{4.695769in}{1.708545in}}%
\pgfpathlineto{\pgfqpoint{4.696828in}{1.956079in}}%
\pgfpathlineto{\pgfqpoint{4.696368in}{1.510567in}}%
\pgfpathlineto{\pgfqpoint{4.696874in}{1.862681in}}%
\pgfpathlineto{\pgfqpoint{4.697838in}{1.651142in}}%
\pgfpathlineto{\pgfqpoint{4.697471in}{1.946142in}}%
\pgfpathlineto{\pgfqpoint{4.697930in}{1.894864in}}%
\pgfpathlineto{\pgfqpoint{4.698113in}{1.979826in}}%
\pgfpathlineto{\pgfqpoint{4.698021in}{1.641908in}}%
\pgfpathlineto{\pgfqpoint{4.698846in}{1.939140in}}%
\pgfpathlineto{\pgfqpoint{4.699211in}{1.672106in}}%
\pgfpathlineto{\pgfqpoint{4.699486in}{1.964870in}}%
\pgfpathlineto{\pgfqpoint{4.699942in}{1.880797in}}%
\pgfpathlineto{\pgfqpoint{4.700034in}{1.982532in}}%
\pgfpathlineto{\pgfqpoint{4.700490in}{1.507991in}}%
\pgfpathlineto{\pgfqpoint{4.700991in}{1.890711in}}%
\pgfpathlineto{\pgfqpoint{4.701036in}{1.522543in}}%
\pgfpathlineto{\pgfqpoint{4.701855in}{1.954318in}}%
\pgfpathlineto{\pgfqpoint{4.702082in}{1.897646in}}%
\pgfpathlineto{\pgfqpoint{4.703034in}{1.460919in}}%
\pgfpathlineto{\pgfqpoint{4.702672in}{1.932662in}}%
\pgfpathlineto{\pgfqpoint{4.703170in}{1.777538in}}%
\pgfpathlineto{\pgfqpoint{4.703939in}{1.943235in}}%
\pgfpathlineto{\pgfqpoint{4.703442in}{1.694419in}}%
\pgfpathlineto{\pgfqpoint{4.704346in}{1.942055in}}%
\pgfpathlineto{\pgfqpoint{4.705383in}{1.555028in}}%
\pgfpathlineto{\pgfqpoint{4.705519in}{1.693045in}}%
\pgfpathlineto{\pgfqpoint{4.706598in}{1.957651in}}%
\pgfpathlineto{\pgfqpoint{4.706418in}{1.618883in}}%
\pgfpathlineto{\pgfqpoint{4.706643in}{1.875763in}}%
\pgfpathlineto{\pgfqpoint{4.707227in}{1.512608in}}%
\pgfpathlineto{\pgfqpoint{4.707541in}{1.976134in}}%
\pgfpathlineto{\pgfqpoint{4.707765in}{1.779324in}}%
\pgfpathlineto{\pgfqpoint{4.707809in}{1.774200in}}%
\pgfpathlineto{\pgfqpoint{4.707854in}{1.948721in}}%
\pgfpathlineto{\pgfqpoint{4.708257in}{1.743892in}}%
\pgfpathlineto{\pgfqpoint{4.708928in}{1.895134in}}%
\pgfpathlineto{\pgfqpoint{4.709553in}{1.636423in}}%
\pgfpathlineto{\pgfqpoint{4.709731in}{1.941272in}}%
\pgfpathlineto{\pgfqpoint{4.709999in}{1.849632in}}%
\pgfpathlineto{\pgfqpoint{4.710132in}{1.959065in}}%
\pgfpathlineto{\pgfqpoint{4.710311in}{1.611109in}}%
\pgfpathlineto{\pgfqpoint{4.711067in}{1.810501in}}%
\pgfpathlineto{\pgfqpoint{4.711511in}{1.591561in}}%
\pgfpathlineto{\pgfqpoint{4.711556in}{1.978726in}}%
\pgfpathlineto{\pgfqpoint{4.711645in}{1.606532in}}%
\pgfpathlineto{\pgfqpoint{4.711689in}{2.001075in}}%
\pgfpathlineto{\pgfqpoint{4.711955in}{1.526056in}}%
\pgfpathlineto{\pgfqpoint{4.712753in}{1.909532in}}%
\pgfpathlineto{\pgfqpoint{4.713107in}{1.571759in}}%
\pgfpathlineto{\pgfqpoint{4.713593in}{1.935029in}}%
\pgfpathlineto{\pgfqpoint{4.713903in}{1.634721in}}%
\pgfpathlineto{\pgfqpoint{4.714476in}{1.943495in}}%
\pgfpathlineto{\pgfqpoint{4.714565in}{1.434626in}}%
\pgfpathlineto{\pgfqpoint{4.715049in}{1.865432in}}%
\pgfpathlineto{\pgfqpoint{4.715225in}{1.973726in}}%
\pgfpathlineto{\pgfqpoint{4.715709in}{1.741458in}}%
\pgfpathlineto{\pgfqpoint{4.716061in}{1.884147in}}%
\pgfpathlineto{\pgfqpoint{4.717114in}{1.553124in}}%
\pgfpathlineto{\pgfqpoint{4.716237in}{1.980090in}}%
\pgfpathlineto{\pgfqpoint{4.717158in}{1.868355in}}%
\pgfpathlineto{\pgfqpoint{4.717421in}{1.648712in}}%
\pgfpathlineto{\pgfqpoint{4.718296in}{1.963080in}}%
\pgfpathlineto{\pgfqpoint{4.719038in}{1.692749in}}%
\pgfpathlineto{\pgfqpoint{4.718645in}{1.963242in}}%
\pgfpathlineto{\pgfqpoint{4.719474in}{1.731281in}}%
\pgfpathlineto{\pgfqpoint{4.720258in}{1.960205in}}%
\pgfpathlineto{\pgfqpoint{4.719953in}{1.728385in}}%
\pgfpathlineto{\pgfqpoint{4.720562in}{1.816662in}}%
\pgfpathlineto{\pgfqpoint{4.721431in}{1.522365in}}%
\pgfpathlineto{\pgfqpoint{4.720736in}{1.979915in}}%
\pgfpathlineto{\pgfqpoint{4.721648in}{1.768414in}}%
\pgfpathlineto{\pgfqpoint{4.721821in}{1.540601in}}%
\pgfpathlineto{\pgfqpoint{4.722471in}{1.911597in}}%
\pgfpathlineto{\pgfqpoint{4.722514in}{1.356527in}}%
\pgfpathlineto{\pgfqpoint{4.722990in}{2.016696in}}%
\pgfpathlineto{\pgfqpoint{4.723551in}{1.748837in}}%
\pgfpathlineto{\pgfqpoint{4.724284in}{1.951205in}}%
\pgfpathlineto{\pgfqpoint{4.723896in}{1.628974in}}%
\pgfpathlineto{\pgfqpoint{4.724672in}{1.875826in}}%
\pgfpathlineto{\pgfqpoint{4.725145in}{1.445755in}}%
\pgfpathlineto{\pgfqpoint{4.725618in}{1.968074in}}%
\pgfpathlineto{\pgfqpoint{4.725790in}{1.764308in}}%
\pgfpathlineto{\pgfqpoint{4.725919in}{1.961309in}}%
\pgfpathlineto{\pgfqpoint{4.726133in}{1.497306in}}%
\pgfpathlineto{\pgfqpoint{4.726948in}{1.907154in}}%
\pgfpathlineto{\pgfqpoint{4.727760in}{1.982512in}}%
\pgfpathlineto{\pgfqpoint{4.727119in}{1.684854in}}%
\pgfpathlineto{\pgfqpoint{4.727931in}{1.857768in}}%
\pgfpathlineto{\pgfqpoint{4.728444in}{1.728872in}}%
\pgfpathlineto{\pgfqpoint{4.728017in}{1.910309in}}%
\pgfpathlineto{\pgfqpoint{4.728998in}{1.778326in}}%
\pgfpathlineto{\pgfqpoint{4.729041in}{1.958406in}}%
\pgfpathlineto{\pgfqpoint{4.729679in}{1.197189in}}%
\pgfpathlineto{\pgfqpoint{4.730105in}{1.884783in}}%
\pgfpathlineto{\pgfqpoint{4.731039in}{1.731484in}}%
\pgfpathlineto{\pgfqpoint{4.730275in}{1.990349in}}%
\pgfpathlineto{\pgfqpoint{4.731124in}{1.817088in}}%
\pgfpathlineto{\pgfqpoint{4.731632in}{1.994235in}}%
\pgfpathlineto{\pgfqpoint{4.732098in}{1.526619in}}%
\pgfpathlineto{\pgfqpoint{4.732225in}{1.796286in}}%
\pgfpathlineto{\pgfqpoint{4.732394in}{1.970213in}}%
\pgfpathlineto{\pgfqpoint{4.732436in}{1.688596in}}%
\pgfpathlineto{\pgfqpoint{4.733323in}{1.840815in}}%
\pgfpathlineto{\pgfqpoint{4.733660in}{1.670756in}}%
\pgfpathlineto{\pgfqpoint{4.733702in}{1.974626in}}%
\pgfpathlineto{\pgfqpoint{4.734376in}{1.894075in}}%
\pgfpathlineto{\pgfqpoint{4.735133in}{1.942719in}}%
\pgfpathlineto{\pgfqpoint{4.734670in}{1.522233in}}%
\pgfpathlineto{\pgfqpoint{4.735427in}{1.928493in}}%
\pgfpathlineto{\pgfqpoint{4.735553in}{1.602819in}}%
\pgfpathlineto{\pgfqpoint{4.735804in}{1.987408in}}%
\pgfpathlineto{\pgfqpoint{4.736642in}{1.752940in}}%
\pgfpathlineto{\pgfqpoint{4.736935in}{1.618469in}}%
\pgfpathlineto{\pgfqpoint{4.737729in}{1.966463in}}%
\pgfpathlineto{\pgfqpoint{4.738313in}{1.547414in}}%
\pgfpathlineto{\pgfqpoint{4.737854in}{1.997222in}}%
\pgfpathlineto{\pgfqpoint{4.738813in}{1.858061in}}%
\pgfpathlineto{\pgfqpoint{4.739520in}{1.940144in}}%
\pgfpathlineto{\pgfqpoint{4.739645in}{1.681080in}}%
\pgfpathlineto{\pgfqpoint{4.739811in}{1.849510in}}%
\pgfpathlineto{\pgfqpoint{4.739977in}{1.959124in}}%
\pgfpathlineto{\pgfqpoint{4.740890in}{1.382272in}}%
\pgfpathlineto{\pgfqpoint{4.742007in}{1.973458in}}%
\pgfpathlineto{\pgfqpoint{4.742998in}{1.749071in}}%
\pgfpathlineto{\pgfqpoint{4.743122in}{1.833857in}}%
\pgfpathlineto{\pgfqpoint{4.743328in}{1.702720in}}%
\pgfpathlineto{\pgfqpoint{4.743904in}{1.934223in}}%
\pgfpathlineto{\pgfqpoint{4.743987in}{1.601513in}}%
\pgfpathlineto{\pgfqpoint{4.744398in}{1.968268in}}%
\pgfpathlineto{\pgfqpoint{4.745014in}{1.715156in}}%
\pgfpathlineto{\pgfqpoint{4.745424in}{1.599473in}}%
\pgfpathlineto{\pgfqpoint{4.746202in}{1.988135in}}%
\pgfpathlineto{\pgfqpoint{4.746243in}{1.677801in}}%
\pgfpathlineto{\pgfqpoint{4.747306in}{1.762782in}}%
\pgfpathlineto{\pgfqpoint{4.748203in}{1.540240in}}%
\pgfpathlineto{\pgfqpoint{4.748447in}{1.965754in}}%
\pgfpathlineto{\pgfqpoint{4.749138in}{1.597702in}}%
\pgfpathlineto{\pgfqpoint{4.749545in}{1.808574in}}%
\pgfpathlineto{\pgfqpoint{4.750518in}{1.930115in}}%
\pgfpathlineto{\pgfqpoint{4.750315in}{1.694408in}}%
\pgfpathlineto{\pgfqpoint{4.750680in}{1.911157in}}%
\pgfpathlineto{\pgfqpoint{4.750720in}{1.981925in}}%
\pgfpathlineto{\pgfqpoint{4.750761in}{1.555358in}}%
\pgfpathlineto{\pgfqpoint{4.751731in}{1.855505in}}%
\pgfpathlineto{\pgfqpoint{4.752095in}{1.625741in}}%
\pgfpathlineto{\pgfqpoint{4.752135in}{1.936697in}}%
\pgfpathlineto{\pgfqpoint{4.752740in}{1.834754in}}%
\pgfpathlineto{\pgfqpoint{4.753424in}{1.993232in}}%
\pgfpathlineto{\pgfqpoint{4.753143in}{1.728156in}}%
\pgfpathlineto{\pgfqpoint{4.753867in}{1.870323in}}%
\pgfpathlineto{\pgfqpoint{4.753907in}{1.581785in}}%
\pgfpathlineto{\pgfqpoint{4.754068in}{1.946730in}}%
\pgfpathlineto{\pgfqpoint{4.754991in}{1.733852in}}%
\pgfpathlineto{\pgfqpoint{4.756151in}{1.987224in}}%
\pgfpathlineto{\pgfqpoint{4.755231in}{1.633312in}}%
\pgfpathlineto{\pgfqpoint{4.756231in}{1.957515in}}%
\pgfpathlineto{\pgfqpoint{4.756990in}{1.583536in}}%
\pgfpathlineto{\pgfqpoint{4.757110in}{1.967305in}}%
\pgfpathlineto{\pgfqpoint{4.757349in}{1.903681in}}%
\pgfpathlineto{\pgfqpoint{4.758305in}{1.577015in}}%
\pgfpathlineto{\pgfqpoint{4.758225in}{1.975557in}}%
\pgfpathlineto{\pgfqpoint{4.758424in}{1.847833in}}%
\pgfpathlineto{\pgfqpoint{4.759456in}{1.987319in}}%
\pgfpathlineto{\pgfqpoint{4.758940in}{1.646934in}}%
\pgfpathlineto{\pgfqpoint{4.759536in}{1.880117in}}%
\pgfpathlineto{\pgfqpoint{4.760051in}{1.937526in}}%
\pgfpathlineto{\pgfqpoint{4.760130in}{1.699700in}}%
\pgfpathlineto{\pgfqpoint{4.760407in}{1.773216in}}%
\pgfpathlineto{\pgfqpoint{4.760763in}{1.625118in}}%
\pgfpathlineto{\pgfqpoint{4.761000in}{1.947455in}}%
\pgfpathlineto{\pgfqpoint{4.761435in}{1.859890in}}%
\pgfpathlineto{\pgfqpoint{4.762303in}{1.629459in}}%
\pgfpathlineto{\pgfqpoint{4.761514in}{1.956599in}}%
\pgfpathlineto{\pgfqpoint{4.762421in}{1.816388in}}%
\pgfpathlineto{\pgfqpoint{4.762460in}{1.977783in}}%
\pgfpathlineto{\pgfqpoint{4.763286in}{1.438415in}}%
\pgfpathlineto{\pgfqpoint{4.763561in}{1.937736in}}%
\pgfpathlineto{\pgfqpoint{4.763993in}{1.443358in}}%
\pgfpathlineto{\pgfqpoint{4.763836in}{1.977555in}}%
\pgfpathlineto{\pgfqpoint{4.764660in}{1.823092in}}%
\pgfpathlineto{\pgfqpoint{4.764973in}{2.002065in}}%
\pgfpathlineto{\pgfqpoint{4.765325in}{1.680619in}}%
\pgfpathlineto{\pgfqpoint{4.765677in}{1.741322in}}%
\pgfpathlineto{\pgfqpoint{4.765716in}{1.740856in}}%
\pgfpathlineto{\pgfqpoint{4.766029in}{1.956719in}}%
\pgfpathlineto{\pgfqpoint{4.766653in}{1.649832in}}%
\pgfpathlineto{\pgfqpoint{4.766887in}{1.858279in}}%
\pgfpathlineto{\pgfqpoint{4.767588in}{1.575663in}}%
\pgfpathlineto{\pgfqpoint{4.767510in}{1.953123in}}%
\pgfpathlineto{\pgfqpoint{4.767977in}{1.898185in}}%
\pgfpathlineto{\pgfqpoint{4.768133in}{1.961966in}}%
\pgfpathlineto{\pgfqpoint{4.768366in}{1.744616in}}%
\pgfpathlineto{\pgfqpoint{4.768909in}{1.461295in}}%
\pgfpathlineto{\pgfqpoint{4.768638in}{1.956500in}}%
\pgfpathlineto{\pgfqpoint{4.769452in}{1.691198in}}%
\pgfpathlineto{\pgfqpoint{4.770071in}{1.630767in}}%
\pgfpathlineto{\pgfqpoint{4.770613in}{1.988263in}}%
\pgfpathlineto{\pgfqpoint{4.770728in}{1.569341in}}%
\pgfpathlineto{\pgfqpoint{4.771731in}{1.938130in}}%
\pgfpathlineto{\pgfqpoint{4.771886in}{1.857586in}}%
\pgfpathlineto{\pgfqpoint{4.771924in}{2.015679in}}%
\pgfpathlineto{\pgfqpoint{4.771963in}{1.979836in}}%
\pgfpathlineto{\pgfqpoint{4.772732in}{1.604110in}}%
\pgfpathlineto{\pgfqpoint{4.773078in}{1.932147in}}%
\pgfpathlineto{\pgfqpoint{4.773999in}{1.666474in}}%
\pgfpathlineto{\pgfqpoint{4.773155in}{1.952833in}}%
\pgfpathlineto{\pgfqpoint{4.774344in}{1.728129in}}%
\pgfpathlineto{\pgfqpoint{4.775415in}{2.005823in}}%
\pgfpathlineto{\pgfqpoint{4.775147in}{1.697490in}}%
\pgfpathlineto{\pgfqpoint{4.775453in}{1.934806in}}%
\pgfpathlineto{\pgfqpoint{4.776407in}{1.675050in}}%
\pgfpathlineto{\pgfqpoint{4.776445in}{1.949783in}}%
\pgfpathlineto{\pgfqpoint{4.776597in}{1.824197in}}%
\pgfpathlineto{\pgfqpoint{4.777092in}{1.746210in}}%
\pgfpathlineto{\pgfqpoint{4.776978in}{1.956495in}}%
\pgfpathlineto{\pgfqpoint{4.777548in}{1.949886in}}%
\pgfpathlineto{\pgfqpoint{4.777624in}{1.920982in}}%
\pgfpathlineto{\pgfqpoint{4.778270in}{1.645548in}}%
\pgfpathlineto{\pgfqpoint{4.777852in}{2.002078in}}%
\pgfpathlineto{\pgfqpoint{4.778725in}{1.792140in}}%
\pgfpathlineto{\pgfqpoint{4.779028in}{2.005506in}}%
\pgfpathlineto{\pgfqpoint{4.779558in}{1.530624in}}%
\pgfpathlineto{\pgfqpoint{4.779823in}{1.822130in}}%
\pgfpathlineto{\pgfqpoint{4.780012in}{1.948326in}}%
\pgfpathlineto{\pgfqpoint{4.779936in}{1.731317in}}%
\pgfpathlineto{\pgfqpoint{4.780050in}{1.755967in}}%
\pgfpathlineto{\pgfqpoint{4.780691in}{2.015915in}}%
\pgfpathlineto{\pgfqpoint{4.780578in}{1.656106in}}%
\pgfpathlineto{\pgfqpoint{4.781144in}{1.784447in}}%
\pgfpathlineto{\pgfqpoint{4.781596in}{1.669148in}}%
\pgfpathlineto{\pgfqpoint{4.782273in}{1.956563in}}%
\pgfpathlineto{\pgfqpoint{4.783174in}{1.986887in}}%
\pgfpathlineto{\pgfqpoint{4.783437in}{1.700714in}}%
\pgfpathlineto{\pgfqpoint{4.784373in}{2.021402in}}%
\pgfpathlineto{\pgfqpoint{4.784261in}{1.690665in}}%
\pgfpathlineto{\pgfqpoint{4.784560in}{1.879004in}}%
\pgfpathlineto{\pgfqpoint{4.784709in}{1.656966in}}%
\pgfpathlineto{\pgfqpoint{4.784747in}{1.844702in}}%
\pgfpathlineto{\pgfqpoint{4.785045in}{1.436825in}}%
\pgfpathlineto{\pgfqpoint{4.785157in}{1.972313in}}%
\pgfpathlineto{\pgfqpoint{4.785829in}{1.798847in}}%
\pgfpathlineto{\pgfqpoint{4.786797in}{1.948311in}}%
\pgfpathlineto{\pgfqpoint{4.786574in}{1.537101in}}%
\pgfpathlineto{\pgfqpoint{4.786871in}{1.799085in}}%
\pgfpathlineto{\pgfqpoint{4.787429in}{1.971554in}}%
\pgfpathlineto{\pgfqpoint{4.787985in}{1.620711in}}%
\pgfpathlineto{\pgfqpoint{4.788467in}{1.982521in}}%
\pgfpathlineto{\pgfqpoint{4.789096in}{1.933228in}}%
\pgfpathlineto{\pgfqpoint{4.789429in}{1.697871in}}%
\pgfpathlineto{\pgfqpoint{4.790057in}{1.974808in}}%
\pgfpathlineto{\pgfqpoint{4.790205in}{1.914327in}}%
\pgfpathlineto{\pgfqpoint{4.790647in}{1.570637in}}%
\pgfpathlineto{\pgfqpoint{4.790979in}{1.986809in}}%
\pgfpathlineto{\pgfqpoint{4.791347in}{1.826087in}}%
\pgfpathlineto{\pgfqpoint{4.792119in}{1.990753in}}%
\pgfpathlineto{\pgfqpoint{4.791825in}{1.657414in}}%
\pgfpathlineto{\pgfqpoint{4.792449in}{1.890703in}}%
\pgfpathlineto{\pgfqpoint{4.792596in}{1.629291in}}%
\pgfpathlineto{\pgfqpoint{4.792779in}{2.025246in}}%
\pgfpathlineto{\pgfqpoint{4.793549in}{1.689782in}}%
\pgfpathlineto{\pgfqpoint{4.794061in}{1.989719in}}%
\pgfpathlineto{\pgfqpoint{4.793988in}{1.647873in}}%
\pgfpathlineto{\pgfqpoint{4.794645in}{1.881521in}}%
\pgfpathlineto{\pgfqpoint{4.794828in}{1.595917in}}%
\pgfpathlineto{\pgfqpoint{4.795192in}{2.020241in}}%
\pgfpathlineto{\pgfqpoint{4.795703in}{1.948776in}}%
\pgfpathlineto{\pgfqpoint{4.795994in}{1.408744in}}%
\pgfpathlineto{\pgfqpoint{4.796503in}{1.971926in}}%
\pgfpathlineto{\pgfqpoint{4.796757in}{1.747652in}}%
\pgfpathlineto{\pgfqpoint{4.797737in}{2.009763in}}%
\pgfpathlineto{\pgfqpoint{4.797447in}{1.604187in}}%
\pgfpathlineto{\pgfqpoint{4.797882in}{1.854456in}}%
\pgfpathlineto{\pgfqpoint{4.797918in}{1.852174in}}%
\pgfpathlineto{\pgfqpoint{4.797991in}{1.903707in}}%
\pgfpathlineto{\pgfqpoint{4.798787in}{1.477513in}}%
\pgfpathlineto{\pgfqpoint{4.798280in}{1.960373in}}%
\pgfpathlineto{\pgfqpoint{4.799112in}{1.774482in}}%
\pgfpathlineto{\pgfqpoint{4.799365in}{1.947278in}}%
\pgfpathlineto{\pgfqpoint{4.799437in}{1.656770in}}%
\pgfpathlineto{\pgfqpoint{4.800158in}{1.865182in}}%
\pgfpathlineto{\pgfqpoint{4.800591in}{1.614545in}}%
\pgfpathlineto{\pgfqpoint{4.800842in}{1.984548in}}%
\pgfpathlineto{\pgfqpoint{4.801238in}{1.917785in}}%
\pgfpathlineto{\pgfqpoint{4.801597in}{1.703897in}}%
\pgfpathlineto{\pgfqpoint{4.801885in}{1.995887in}}%
\pgfpathlineto{\pgfqpoint{4.802315in}{1.822518in}}%
\pgfpathlineto{\pgfqpoint{4.802351in}{1.983061in}}%
\pgfpathlineto{\pgfqpoint{4.802602in}{1.477497in}}%
\pgfpathlineto{\pgfqpoint{4.803389in}{1.795175in}}%
\pgfpathlineto{\pgfqpoint{4.803604in}{1.978003in}}%
\pgfpathlineto{\pgfqpoint{4.803497in}{1.591598in}}%
\pgfpathlineto{\pgfqpoint{4.804425in}{1.951829in}}%
\pgfpathlineto{\pgfqpoint{4.804675in}{1.547508in}}%
\pgfpathlineto{\pgfqpoint{4.804568in}{1.972065in}}%
\pgfpathlineto{\pgfqpoint{4.805530in}{1.784963in}}%
\pgfpathlineto{\pgfqpoint{4.806454in}{2.030227in}}%
\pgfpathlineto{\pgfqpoint{4.806489in}{1.553440in}}%
\pgfpathlineto{\pgfqpoint{4.806631in}{1.826790in}}%
\pgfpathlineto{\pgfqpoint{4.806738in}{1.942609in}}%
\pgfpathlineto{\pgfqpoint{4.806773in}{1.903034in}}%
\pgfpathlineto{\pgfqpoint{4.807092in}{1.574314in}}%
\pgfpathlineto{\pgfqpoint{4.807340in}{1.974336in}}%
\pgfpathlineto{\pgfqpoint{4.807872in}{1.885127in}}%
\pgfpathlineto{\pgfqpoint{4.808013in}{1.683000in}}%
\pgfpathlineto{\pgfqpoint{4.808084in}{1.974236in}}%
\pgfpathlineto{\pgfqpoint{4.808932in}{1.883663in}}%
\pgfpathlineto{\pgfqpoint{4.809884in}{2.022040in}}%
\pgfpathlineto{\pgfqpoint{4.809038in}{1.632787in}}%
\pgfpathlineto{\pgfqpoint{4.809990in}{1.819346in}}%
\pgfpathlineto{\pgfqpoint{4.810060in}{1.976532in}}%
\pgfpathlineto{\pgfqpoint{4.810306in}{1.662931in}}%
\pgfpathlineto{\pgfqpoint{4.810693in}{1.845892in}}%
\pgfpathlineto{\pgfqpoint{4.810728in}{1.684097in}}%
\pgfpathlineto{\pgfqpoint{4.811431in}{1.970274in}}%
\pgfpathlineto{\pgfqpoint{4.811746in}{1.962666in}}%
\pgfpathlineto{\pgfqpoint{4.811782in}{1.964138in}}%
\pgfpathlineto{\pgfqpoint{4.812132in}{1.616656in}}%
\pgfpathlineto{\pgfqpoint{4.812727in}{2.003775in}}%
\pgfpathlineto{\pgfqpoint{4.812937in}{1.796484in}}%
\pgfpathlineto{\pgfqpoint{4.813112in}{1.950309in}}%
\pgfpathlineto{\pgfqpoint{4.813671in}{1.652491in}}%
\pgfpathlineto{\pgfqpoint{4.814577in}{1.985185in}}%
\pgfpathlineto{\pgfqpoint{4.813985in}{1.487701in}}%
\pgfpathlineto{\pgfqpoint{4.814786in}{1.846147in}}%
\pgfpathlineto{\pgfqpoint{4.814821in}{1.982469in}}%
\pgfpathlineto{\pgfqpoint{4.815204in}{1.614977in}}%
\pgfpathlineto{\pgfqpoint{4.815899in}{1.885964in}}%
\pgfpathlineto{\pgfqpoint{4.816697in}{1.687270in}}%
\pgfpathlineto{\pgfqpoint{4.816385in}{1.993059in}}%
\pgfpathlineto{\pgfqpoint{4.817009in}{1.824973in}}%
\pgfpathlineto{\pgfqpoint{4.817839in}{1.678045in}}%
\pgfpathlineto{\pgfqpoint{4.817701in}{1.993201in}}%
\pgfpathlineto{\pgfqpoint{4.818081in}{1.879288in}}%
\pgfpathlineto{\pgfqpoint{4.818909in}{1.670548in}}%
\pgfpathlineto{\pgfqpoint{4.818840in}{1.948344in}}%
\pgfpathlineto{\pgfqpoint{4.819116in}{1.906177in}}%
\pgfpathlineto{\pgfqpoint{4.819426in}{1.970734in}}%
\pgfpathlineto{\pgfqpoint{4.819185in}{1.818255in}}%
\pgfpathlineto{\pgfqpoint{4.819460in}{1.863568in}}%
\pgfpathlineto{\pgfqpoint{4.819598in}{1.505073in}}%
\pgfpathlineto{\pgfqpoint{4.819977in}{2.023096in}}%
\pgfpathlineto{\pgfqpoint{4.820561in}{1.893833in}}%
\pgfpathlineto{\pgfqpoint{4.820630in}{1.593197in}}%
\pgfpathlineto{\pgfqpoint{4.821487in}{1.990708in}}%
\pgfpathlineto{\pgfqpoint{4.821556in}{1.821204in}}%
\pgfpathlineto{\pgfqpoint{4.821659in}{1.999432in}}%
\pgfpathlineto{\pgfqpoint{4.822548in}{1.643175in}}%
\pgfpathlineto{\pgfqpoint{4.822651in}{1.924781in}}%
\pgfpathlineto{\pgfqpoint{4.823095in}{1.667572in}}%
\pgfpathlineto{\pgfqpoint{4.823675in}{1.992325in}}%
\pgfpathlineto{\pgfqpoint{4.824356in}{2.020801in}}%
\pgfpathlineto{\pgfqpoint{4.824288in}{1.705842in}}%
\pgfpathlineto{\pgfqpoint{4.824424in}{1.883521in}}%
\pgfpathlineto{\pgfqpoint{4.825410in}{1.459120in}}%
\pgfpathlineto{\pgfqpoint{4.825036in}{2.017296in}}%
\pgfpathlineto{\pgfqpoint{4.825512in}{1.702842in}}%
\pgfpathlineto{\pgfqpoint{4.826122in}{2.023703in}}%
\pgfpathlineto{\pgfqpoint{4.825716in}{1.699731in}}%
\pgfpathlineto{\pgfqpoint{4.826631in}{2.018945in}}%
\pgfpathlineto{\pgfqpoint{4.827442in}{1.273892in}}%
\pgfpathlineto{\pgfqpoint{4.827746in}{1.670114in}}%
\pgfpathlineto{\pgfqpoint{4.828286in}{1.967452in}}%
\pgfpathlineto{\pgfqpoint{4.828219in}{1.539316in}}%
\pgfpathlineto{\pgfqpoint{4.828859in}{1.894004in}}%
\pgfpathlineto{\pgfqpoint{4.829364in}{2.027190in}}%
\pgfpathlineto{\pgfqpoint{4.829061in}{1.764979in}}%
\pgfpathlineto{\pgfqpoint{4.829835in}{1.927423in}}%
\pgfpathlineto{\pgfqpoint{4.829869in}{1.399963in}}%
\pgfpathlineto{\pgfqpoint{4.830305in}{1.994999in}}%
\pgfpathlineto{\pgfqpoint{4.830942in}{1.877942in}}%
\pgfpathlineto{\pgfqpoint{4.830976in}{1.955235in}}%
\pgfpathlineto{\pgfqpoint{4.831612in}{1.496157in}}%
\pgfpathlineto{\pgfqpoint{4.832047in}{1.896668in}}%
\pgfpathlineto{\pgfqpoint{4.832281in}{1.990217in}}%
\pgfpathlineto{\pgfqpoint{4.832114in}{1.764226in}}%
\pgfpathlineto{\pgfqpoint{4.832314in}{1.977624in}}%
\pgfpathlineto{\pgfqpoint{4.832682in}{1.709441in}}%
\pgfpathlineto{\pgfqpoint{4.832615in}{2.009431in}}%
\pgfpathlineto{\pgfqpoint{4.833449in}{1.794054in}}%
\pgfpathlineto{\pgfqpoint{4.834114in}{1.539303in}}%
\pgfpathlineto{\pgfqpoint{4.834580in}{2.004268in}}%
\pgfpathlineto{\pgfqpoint{4.834912in}{1.647234in}}%
\pgfpathlineto{\pgfqpoint{4.835708in}{1.778406in}}%
\pgfpathlineto{\pgfqpoint{4.836139in}{2.000363in}}%
\pgfpathlineto{\pgfqpoint{4.835774in}{1.686820in}}%
\pgfpathlineto{\pgfqpoint{4.836833in}{1.848242in}}%
\pgfpathlineto{\pgfqpoint{4.837857in}{1.585317in}}%
\pgfpathlineto{\pgfqpoint{4.837098in}{1.998158in}}%
\pgfpathlineto{\pgfqpoint{4.837890in}{1.912263in}}%
\pgfpathlineto{\pgfqpoint{4.838285in}{1.546148in}}%
\pgfpathlineto{\pgfqpoint{4.838549in}{2.001760in}}%
\pgfpathlineto{\pgfqpoint{4.839009in}{1.805035in}}%
\pgfpathlineto{\pgfqpoint{4.839141in}{1.991741in}}%
\pgfpathlineto{\pgfqpoint{4.839305in}{1.634097in}}%
\pgfpathlineto{\pgfqpoint{4.840093in}{1.958819in}}%
\pgfpathlineto{\pgfqpoint{4.840454in}{1.602428in}}%
\pgfpathlineto{\pgfqpoint{4.840716in}{1.986847in}}%
\pgfpathlineto{\pgfqpoint{4.841207in}{1.926360in}}%
\pgfpathlineto{\pgfqpoint{4.841239in}{1.992066in}}%
\pgfpathlineto{\pgfqpoint{4.842122in}{1.672444in}}%
\pgfpathlineto{\pgfqpoint{4.842252in}{1.953735in}}%
\pgfpathlineto{\pgfqpoint{4.842285in}{1.475346in}}%
\pgfpathlineto{\pgfqpoint{4.843230in}{1.996695in}}%
\pgfpathlineto{\pgfqpoint{4.843360in}{1.714861in}}%
\pgfpathlineto{\pgfqpoint{4.844173in}{2.013209in}}%
\pgfpathlineto{\pgfqpoint{4.844368in}{1.627650in}}%
\pgfpathlineto{\pgfqpoint{4.844466in}{1.943189in}}%
\pgfpathlineto{\pgfqpoint{4.845179in}{1.330242in}}%
\pgfpathlineto{\pgfqpoint{4.844985in}{2.001992in}}%
\pgfpathlineto{\pgfqpoint{4.845568in}{1.909373in}}%
\pgfpathlineto{\pgfqpoint{4.845924in}{1.653276in}}%
\pgfpathlineto{\pgfqpoint{4.846280in}{1.984152in}}%
\pgfpathlineto{\pgfqpoint{4.846667in}{1.782698in}}%
\pgfpathlineto{\pgfqpoint{4.847635in}{1.977335in}}%
\pgfpathlineto{\pgfqpoint{4.846926in}{1.748574in}}%
\pgfpathlineto{\pgfqpoint{4.847764in}{1.847551in}}%
\pgfpathlineto{\pgfqpoint{4.848118in}{1.574811in}}%
\pgfpathlineto{\pgfqpoint{4.848504in}{2.014486in}}%
\pgfpathlineto{\pgfqpoint{4.848826in}{1.828511in}}%
\pgfpathlineto{\pgfqpoint{4.849692in}{2.016072in}}%
\pgfpathlineto{\pgfqpoint{4.849468in}{1.561223in}}%
\pgfpathlineto{\pgfqpoint{4.849981in}{1.935600in}}%
\pgfpathlineto{\pgfqpoint{4.850301in}{1.726124in}}%
\pgfpathlineto{\pgfqpoint{4.850077in}{2.036502in}}%
\pgfpathlineto{\pgfqpoint{4.851069in}{1.805016in}}%
\pgfpathlineto{\pgfqpoint{4.851612in}{1.975856in}}%
\pgfpathlineto{\pgfqpoint{4.851740in}{1.723416in}}%
\pgfpathlineto{\pgfqpoint{4.852186in}{1.953779in}}%
\pgfpathlineto{\pgfqpoint{4.853078in}{1.656345in}}%
\pgfpathlineto{\pgfqpoint{4.852855in}{2.061622in}}%
\pgfpathlineto{\pgfqpoint{4.853269in}{1.898826in}}%
\pgfpathlineto{\pgfqpoint{4.853650in}{1.597066in}}%
\pgfpathlineto{\pgfqpoint{4.854348in}{2.033234in}}%
\pgfpathlineto{\pgfqpoint{4.854570in}{1.713925in}}%
\pgfpathlineto{\pgfqpoint{4.855457in}{1.985059in}}%
\pgfpathlineto{\pgfqpoint{4.855868in}{1.726667in}}%
\pgfpathlineto{\pgfqpoint{4.855773in}{2.038758in}}%
\pgfpathlineto{\pgfqpoint{4.856563in}{1.817792in}}%
\pgfpathlineto{\pgfqpoint{4.856752in}{2.017226in}}%
\pgfpathlineto{\pgfqpoint{4.857004in}{1.644844in}}%
\pgfpathlineto{\pgfqpoint{4.857666in}{1.932684in}}%
\pgfpathlineto{\pgfqpoint{4.858703in}{1.572589in}}%
\pgfpathlineto{\pgfqpoint{4.858074in}{2.015212in}}%
\pgfpathlineto{\pgfqpoint{4.858797in}{1.843977in}}%
\pgfpathlineto{\pgfqpoint{4.858954in}{2.091304in}}%
\pgfpathlineto{\pgfqpoint{4.859048in}{1.625333in}}%
\pgfpathlineto{\pgfqpoint{4.859894in}{1.969919in}}%
\pgfpathlineto{\pgfqpoint{4.860301in}{1.682507in}}%
\pgfpathlineto{\pgfqpoint{4.860613in}{1.990851in}}%
\pgfpathlineto{\pgfqpoint{4.860988in}{1.914546in}}%
\pgfpathlineto{\pgfqpoint{4.861550in}{1.530484in}}%
\pgfpathlineto{\pgfqpoint{4.861456in}{2.015674in}}%
\pgfpathlineto{\pgfqpoint{4.862142in}{1.882097in}}%
\pgfpathlineto{\pgfqpoint{4.862298in}{1.973283in}}%
\pgfpathlineto{\pgfqpoint{4.862235in}{1.701916in}}%
\pgfpathlineto{\pgfqpoint{4.862453in}{1.890783in}}%
\pgfpathlineto{\pgfqpoint{4.863137in}{1.673602in}}%
\pgfpathlineto{\pgfqpoint{4.862609in}{1.990053in}}%
\pgfpathlineto{\pgfqpoint{4.863541in}{1.898252in}}%
\pgfpathlineto{\pgfqpoint{4.864409in}{1.717932in}}%
\pgfpathlineto{\pgfqpoint{4.863913in}{2.023010in}}%
\pgfpathlineto{\pgfqpoint{4.864688in}{1.830259in}}%
\pgfpathlineto{\pgfqpoint{4.865245in}{2.034584in}}%
\pgfpathlineto{\pgfqpoint{4.865368in}{1.507256in}}%
\pgfpathlineto{\pgfqpoint{4.865801in}{1.983643in}}%
\pgfpathlineto{\pgfqpoint{4.866602in}{1.600124in}}%
\pgfpathlineto{\pgfqpoint{4.866880in}{1.986443in}}%
\pgfpathlineto{\pgfqpoint{4.866941in}{1.864753in}}%
\pgfpathlineto{\pgfqpoint{4.867710in}{2.032954in}}%
\pgfpathlineto{\pgfqpoint{4.867741in}{1.663490in}}%
\pgfpathlineto{\pgfqpoint{4.868018in}{1.928703in}}%
\pgfpathlineto{\pgfqpoint{4.868999in}{1.654881in}}%
\pgfpathlineto{\pgfqpoint{4.868601in}{2.027823in}}%
\pgfpathlineto{\pgfqpoint{4.869122in}{1.826809in}}%
\pgfpathlineto{\pgfqpoint{4.869581in}{1.994164in}}%
\pgfpathlineto{\pgfqpoint{4.869978in}{1.753559in}}%
\pgfpathlineto{\pgfqpoint{4.870192in}{1.850154in}}%
\pgfpathlineto{\pgfqpoint{4.870223in}{1.619932in}}%
\pgfpathlineto{\pgfqpoint{4.871017in}{2.036610in}}%
\pgfpathlineto{\pgfqpoint{4.871260in}{1.962766in}}%
\pgfpathlineto{\pgfqpoint{4.871291in}{1.968712in}}%
\pgfpathlineto{\pgfqpoint{4.871321in}{1.930541in}}%
\pgfpathlineto{\pgfqpoint{4.871352in}{1.942263in}}%
\pgfpathlineto{\pgfqpoint{4.871382in}{1.657572in}}%
\pgfpathlineto{\pgfqpoint{4.871413in}{2.013297in}}%
\pgfpathlineto{\pgfqpoint{4.872447in}{1.717984in}}%
\pgfpathlineto{\pgfqpoint{4.873479in}{2.026571in}}%
\pgfpathlineto{\pgfqpoint{4.872782in}{1.546677in}}%
\pgfpathlineto{\pgfqpoint{4.873570in}{1.821492in}}%
\pgfpathlineto{\pgfqpoint{4.874206in}{1.748809in}}%
\pgfpathlineto{\pgfqpoint{4.873631in}{2.016803in}}%
\pgfpathlineto{\pgfqpoint{4.874267in}{1.933972in}}%
\pgfpathlineto{\pgfqpoint{4.874358in}{2.034307in}}%
\pgfpathlineto{\pgfqpoint{4.874509in}{1.688069in}}%
\pgfpathlineto{\pgfqpoint{4.875264in}{1.792896in}}%
\pgfpathlineto{\pgfqpoint{4.875657in}{1.581839in}}%
\pgfpathlineto{\pgfqpoint{4.875415in}{1.999837in}}%
\pgfpathlineto{\pgfqpoint{4.876290in}{1.905765in}}%
\pgfpathlineto{\pgfqpoint{4.876651in}{2.013852in}}%
\pgfpathlineto{\pgfqpoint{4.876440in}{1.707098in}}%
\pgfpathlineto{\pgfqpoint{4.876801in}{1.892428in}}%
\pgfpathlineto{\pgfqpoint{4.877373in}{1.689338in}}%
\pgfpathlineto{\pgfqpoint{4.877192in}{2.063056in}}%
\pgfpathlineto{\pgfqpoint{4.877913in}{1.828679in}}%
\pgfpathlineto{\pgfqpoint{4.878003in}{2.019945in}}%
\pgfpathlineto{\pgfqpoint{4.878093in}{1.586671in}}%
\pgfpathlineto{\pgfqpoint{4.878962in}{1.901135in}}%
\pgfpathlineto{\pgfqpoint{4.878992in}{1.590626in}}%
\pgfpathlineto{\pgfqpoint{4.879022in}{2.043448in}}%
\pgfpathlineto{\pgfqpoint{4.880038in}{1.888522in}}%
\pgfpathlineto{\pgfqpoint{4.880068in}{2.018301in}}%
\pgfpathlineto{\pgfqpoint{4.880157in}{1.337333in}}%
\pgfpathlineto{\pgfqpoint{4.881111in}{1.953053in}}%
\pgfpathlineto{\pgfqpoint{4.881914in}{2.019061in}}%
\pgfpathlineto{\pgfqpoint{4.882271in}{1.456479in}}%
\pgfpathlineto{\pgfqpoint{4.883339in}{2.032088in}}%
\pgfpathlineto{\pgfqpoint{4.882983in}{1.415283in}}%
\pgfpathlineto{\pgfqpoint{4.883398in}{1.940620in}}%
\pgfpathlineto{\pgfqpoint{4.883635in}{1.641708in}}%
\pgfpathlineto{\pgfqpoint{4.883546in}{2.043282in}}%
\pgfpathlineto{\pgfqpoint{4.884492in}{1.978829in}}%
\pgfpathlineto{\pgfqpoint{4.884965in}{1.742922in}}%
\pgfpathlineto{\pgfqpoint{4.885407in}{2.038627in}}%
\pgfpathlineto{\pgfqpoint{4.885584in}{1.858285in}}%
\pgfpathlineto{\pgfqpoint{4.886291in}{2.004702in}}%
\pgfpathlineto{\pgfqpoint{4.885702in}{1.651517in}}%
\pgfpathlineto{\pgfqpoint{4.886673in}{1.929304in}}%
\pgfpathlineto{\pgfqpoint{4.887466in}{1.715518in}}%
\pgfpathlineto{\pgfqpoint{4.887319in}{2.016001in}}%
\pgfpathlineto{\pgfqpoint{4.887788in}{1.890813in}}%
\pgfpathlineto{\pgfqpoint{4.887876in}{1.734664in}}%
\pgfpathlineto{\pgfqpoint{4.887964in}{2.010378in}}%
\pgfpathlineto{\pgfqpoint{4.888579in}{1.901926in}}%
\pgfpathlineto{\pgfqpoint{4.888930in}{2.045999in}}%
\pgfpathlineto{\pgfqpoint{4.888696in}{1.544449in}}%
\pgfpathlineto{\pgfqpoint{4.889660in}{1.878295in}}%
\pgfpathlineto{\pgfqpoint{4.889864in}{1.728805in}}%
\pgfpathlineto{\pgfqpoint{4.889835in}{2.020929in}}%
\pgfpathlineto{\pgfqpoint{4.890564in}{1.857308in}}%
\pgfpathlineto{\pgfqpoint{4.891553in}{1.603310in}}%
\pgfpathlineto{\pgfqpoint{4.891698in}{2.044532in}}%
\pgfpathlineto{\pgfqpoint{4.892540in}{1.629584in}}%
\pgfpathlineto{\pgfqpoint{4.892829in}{1.791224in}}%
\pgfpathlineto{\pgfqpoint{4.893177in}{1.712482in}}%
\pgfpathlineto{\pgfqpoint{4.893206in}{2.003155in}}%
\pgfpathlineto{\pgfqpoint{4.893726in}{1.857465in}}%
\pgfpathlineto{\pgfqpoint{4.894708in}{2.029911in}}%
\pgfpathlineto{\pgfqpoint{4.894362in}{1.507366in}}%
\pgfpathlineto{\pgfqpoint{4.894766in}{1.979825in}}%
\pgfpathlineto{\pgfqpoint{4.894794in}{1.428334in}}%
\pgfpathlineto{\pgfqpoint{4.895428in}{2.017488in}}%
\pgfpathlineto{\pgfqpoint{4.895889in}{1.705988in}}%
\pgfpathlineto{\pgfqpoint{4.896693in}{2.034801in}}%
\pgfpathlineto{\pgfqpoint{4.896406in}{1.501880in}}%
\pgfpathlineto{\pgfqpoint{4.897037in}{1.926992in}}%
\pgfpathlineto{\pgfqpoint{4.897954in}{1.632485in}}%
\pgfpathlineto{\pgfqpoint{4.897467in}{2.039717in}}%
\pgfpathlineto{\pgfqpoint{4.898126in}{1.961598in}}%
\pgfpathlineto{\pgfqpoint{4.898441in}{1.720526in}}%
\pgfpathlineto{\pgfqpoint{4.898955in}{2.051520in}}%
\pgfpathlineto{\pgfqpoint{4.899183in}{1.942695in}}%
\pgfpathlineto{\pgfqpoint{4.899383in}{2.038967in}}%
\pgfpathlineto{\pgfqpoint{4.899525in}{1.709939in}}%
\pgfpathlineto{\pgfqpoint{4.900181in}{1.963760in}}%
\pgfpathlineto{\pgfqpoint{4.900351in}{1.496125in}}%
\pgfpathlineto{\pgfqpoint{4.900238in}{2.036029in}}%
\pgfpathlineto{\pgfqpoint{4.901318in}{1.665560in}}%
\pgfpathlineto{\pgfqpoint{4.901545in}{2.002848in}}%
\pgfpathlineto{\pgfqpoint{4.901800in}{1.575672in}}%
\pgfpathlineto{\pgfqpoint{4.902424in}{1.930206in}}%
\pgfpathlineto{\pgfqpoint{4.902594in}{1.659270in}}%
\pgfpathlineto{\pgfqpoint{4.902707in}{2.028344in}}%
\pgfpathlineto{\pgfqpoint{4.903470in}{2.002137in}}%
\pgfpathlineto{\pgfqpoint{4.903499in}{2.003571in}}%
\pgfpathlineto{\pgfqpoint{4.903894in}{1.641408in}}%
\pgfpathlineto{\pgfqpoint{4.903753in}{2.060034in}}%
\pgfpathlineto{\pgfqpoint{4.904627in}{1.747153in}}%
\pgfpathlineto{\pgfqpoint{4.905274in}{2.025948in}}%
\pgfpathlineto{\pgfqpoint{4.905359in}{1.667638in}}%
\pgfpathlineto{\pgfqpoint{4.905781in}{1.965839in}}%
\pgfpathlineto{\pgfqpoint{4.906118in}{1.677559in}}%
\pgfpathlineto{\pgfqpoint{4.906623in}{2.044608in}}%
\pgfpathlineto{\pgfqpoint{4.906903in}{1.799802in}}%
\pgfpathlineto{\pgfqpoint{4.907491in}{2.037878in}}%
\pgfpathlineto{\pgfqpoint{4.907099in}{1.716816in}}%
\pgfpathlineto{\pgfqpoint{4.908022in}{1.876870in}}%
\pgfpathlineto{\pgfqpoint{4.908050in}{1.874641in}}%
\pgfpathlineto{\pgfqpoint{4.908078in}{1.886153in}}%
\pgfpathlineto{\pgfqpoint{4.908525in}{2.080118in}}%
\pgfpathlineto{\pgfqpoint{4.908497in}{1.498856in}}%
\pgfpathlineto{\pgfqpoint{4.909167in}{1.957919in}}%
\pgfpathlineto{\pgfqpoint{4.909752in}{1.439957in}}%
\pgfpathlineto{\pgfqpoint{4.909306in}{2.075245in}}%
\pgfpathlineto{\pgfqpoint{4.910280in}{1.818810in}}%
\pgfpathlineto{\pgfqpoint{4.910363in}{2.041862in}}%
\pgfpathlineto{\pgfqpoint{4.910558in}{1.524529in}}%
\pgfpathlineto{\pgfqpoint{4.911335in}{1.874069in}}%
\pgfpathlineto{\pgfqpoint{4.912111in}{1.332207in}}%
\pgfpathlineto{\pgfqpoint{4.911695in}{2.042929in}}%
\pgfpathlineto{\pgfqpoint{4.912415in}{1.837492in}}%
\pgfpathlineto{\pgfqpoint{4.912443in}{2.003153in}}%
\pgfpathlineto{\pgfqpoint{4.913327in}{1.715149in}}%
\pgfpathlineto{\pgfqpoint{4.913520in}{1.953499in}}%
\pgfpathlineto{\pgfqpoint{4.913713in}{2.031608in}}%
\pgfpathlineto{\pgfqpoint{4.914622in}{1.595987in}}%
\pgfpathlineto{\pgfqpoint{4.915172in}{2.037948in}}%
\pgfpathlineto{\pgfqpoint{4.915722in}{1.937429in}}%
\pgfpathlineto{\pgfqpoint{4.916270in}{1.625673in}}%
\pgfpathlineto{\pgfqpoint{4.916023in}{2.047358in}}%
\pgfpathlineto{\pgfqpoint{4.916845in}{1.815922in}}%
\pgfpathlineto{\pgfqpoint{4.917228in}{2.045418in}}%
\pgfpathlineto{\pgfqpoint{4.917119in}{1.714344in}}%
\pgfpathlineto{\pgfqpoint{4.917939in}{1.981585in}}%
\pgfpathlineto{\pgfqpoint{4.918103in}{1.689963in}}%
\pgfpathlineto{\pgfqpoint{4.918975in}{2.060879in}}%
\pgfpathlineto{\pgfqpoint{4.919030in}{1.955026in}}%
\pgfpathlineto{\pgfqpoint{4.919220in}{2.011291in}}%
\pgfpathlineto{\pgfqpoint{4.919138in}{1.483021in}}%
\pgfpathlineto{\pgfqpoint{4.920009in}{1.829312in}}%
\pgfpathlineto{\pgfqpoint{4.920036in}{1.546972in}}%
\pgfpathlineto{\pgfqpoint{4.920498in}{2.021809in}}%
\pgfpathlineto{\pgfqpoint{4.921094in}{1.714061in}}%
\pgfpathlineto{\pgfqpoint{4.921500in}{2.011833in}}%
\pgfpathlineto{\pgfqpoint{4.922204in}{1.870575in}}%
\pgfpathlineto{\pgfqpoint{4.922609in}{1.486958in}}%
\pgfpathlineto{\pgfqpoint{4.923122in}{2.075801in}}%
\pgfpathlineto{\pgfqpoint{4.923284in}{1.756371in}}%
\pgfpathlineto{\pgfqpoint{4.924092in}{2.028922in}}%
\pgfpathlineto{\pgfqpoint{4.923634in}{1.628558in}}%
\pgfpathlineto{\pgfqpoint{4.924388in}{1.891262in}}%
\pgfpathlineto{\pgfqpoint{4.924979in}{1.659711in}}%
\pgfpathlineto{\pgfqpoint{4.925113in}{2.051386in}}%
\pgfpathlineto{\pgfqpoint{4.925489in}{1.721709in}}%
\pgfpathlineto{\pgfqpoint{4.925971in}{2.031603in}}%
\pgfpathlineto{\pgfqpoint{4.925596in}{1.626674in}}%
\pgfpathlineto{\pgfqpoint{4.926587in}{1.939190in}}%
\pgfpathlineto{\pgfqpoint{4.927602in}{1.581853in}}%
\pgfpathlineto{\pgfqpoint{4.926961in}{2.016340in}}%
\pgfpathlineto{\pgfqpoint{4.927709in}{1.889835in}}%
\pgfpathlineto{\pgfqpoint{4.928242in}{2.060772in}}%
\pgfpathlineto{\pgfqpoint{4.928535in}{1.569204in}}%
\pgfpathlineto{\pgfqpoint{4.928802in}{1.978760in}}%
\pgfpathlineto{\pgfqpoint{4.929440in}{1.684643in}}%
\pgfpathlineto{\pgfqpoint{4.929068in}{2.073995in}}%
\pgfpathlineto{\pgfqpoint{4.929944in}{1.864844in}}%
\pgfpathlineto{\pgfqpoint{4.930501in}{2.040948in}}%
\pgfpathlineto{\pgfqpoint{4.930713in}{1.721426in}}%
\pgfpathlineto{\pgfqpoint{4.931084in}{2.029548in}}%
\pgfpathlineto{\pgfqpoint{4.931534in}{1.421933in}}%
\pgfpathlineto{\pgfqpoint{4.931719in}{2.032338in}}%
\pgfpathlineto{\pgfqpoint{4.932194in}{1.935056in}}%
\pgfpathlineto{\pgfqpoint{4.933091in}{1.523386in}}%
\pgfpathlineto{\pgfqpoint{4.932273in}{2.024386in}}%
\pgfpathlineto{\pgfqpoint{4.933275in}{1.856592in}}%
\pgfpathlineto{\pgfqpoint{4.933302in}{2.056651in}}%
\pgfpathlineto{\pgfqpoint{4.933933in}{1.548968in}}%
\pgfpathlineto{\pgfqpoint{4.934380in}{1.938351in}}%
\pgfpathlineto{\pgfqpoint{4.934879in}{1.632966in}}%
\pgfpathlineto{\pgfqpoint{4.934826in}{2.056334in}}%
\pgfpathlineto{\pgfqpoint{4.935534in}{1.772066in}}%
\pgfpathlineto{\pgfqpoint{4.936293in}{2.047209in}}%
\pgfpathlineto{\pgfqpoint{4.936424in}{1.706819in}}%
\pgfpathlineto{\pgfqpoint{4.936659in}{2.012359in}}%
\pgfpathlineto{\pgfqpoint{4.936685in}{2.010272in}}%
\pgfpathlineto{\pgfqpoint{4.937390in}{1.433180in}}%
\pgfpathlineto{\pgfqpoint{4.936920in}{2.060419in}}%
\pgfpathlineto{\pgfqpoint{4.937807in}{1.868634in}}%
\pgfpathlineto{\pgfqpoint{4.938432in}{2.039445in}}%
\pgfpathlineto{\pgfqpoint{4.938275in}{1.587097in}}%
\pgfpathlineto{\pgfqpoint{4.938900in}{1.965741in}}%
\pgfpathlineto{\pgfqpoint{4.939237in}{1.513726in}}%
\pgfpathlineto{\pgfqpoint{4.939834in}{2.044642in}}%
\pgfpathlineto{\pgfqpoint{4.939990in}{1.941194in}}%
\pgfpathlineto{\pgfqpoint{4.940326in}{2.060510in}}%
\pgfpathlineto{\pgfqpoint{4.940119in}{1.684272in}}%
\pgfpathlineto{\pgfqpoint{4.940352in}{2.009110in}}%
\pgfpathlineto{\pgfqpoint{4.941077in}{1.332059in}}%
\pgfpathlineto{\pgfqpoint{4.941413in}{2.024513in}}%
\pgfpathlineto{\pgfqpoint{4.941465in}{1.876729in}}%
\pgfpathlineto{\pgfqpoint{4.942316in}{1.549583in}}%
\pgfpathlineto{\pgfqpoint{4.942290in}{2.021144in}}%
\pgfpathlineto{\pgfqpoint{4.942574in}{1.865883in}}%
\pgfpathlineto{\pgfqpoint{4.942909in}{2.020995in}}%
\pgfpathlineto{\pgfqpoint{4.942960in}{1.803836in}}%
\pgfpathlineto{\pgfqpoint{4.943166in}{1.932075in}}%
\pgfpathlineto{\pgfqpoint{4.943577in}{1.734018in}}%
\pgfpathlineto{\pgfqpoint{4.943886in}{1.998799in}}%
\pgfpathlineto{\pgfqpoint{4.944296in}{1.771027in}}%
\pgfpathlineto{\pgfqpoint{4.944937in}{2.051736in}}%
\pgfpathlineto{\pgfqpoint{4.945373in}{1.688289in}}%
\pgfpathlineto{\pgfqpoint{4.945424in}{2.002853in}}%
\pgfpathlineto{\pgfqpoint{4.946293in}{1.707454in}}%
\pgfpathlineto{\pgfqpoint{4.946523in}{2.044128in}}%
\pgfpathlineto{\pgfqpoint{4.946956in}{1.671796in}}%
\pgfpathlineto{\pgfqpoint{4.947593in}{2.044431in}}%
\pgfpathlineto{\pgfqpoint{4.947644in}{1.858586in}}%
\pgfpathlineto{\pgfqpoint{4.948484in}{2.050711in}}%
\pgfpathlineto{\pgfqpoint{4.948382in}{1.722020in}}%
\pgfpathlineto{\pgfqpoint{4.948763in}{1.988379in}}%
\pgfpathlineto{\pgfqpoint{4.948865in}{1.559541in}}%
\pgfpathlineto{\pgfqpoint{4.949701in}{2.048803in}}%
\pgfpathlineto{\pgfqpoint{4.949853in}{2.010786in}}%
\pgfpathlineto{\pgfqpoint{4.950587in}{2.041673in}}%
\pgfpathlineto{\pgfqpoint{4.950081in}{1.731820in}}%
\pgfpathlineto{\pgfqpoint{4.950815in}{1.966806in}}%
\pgfpathlineto{\pgfqpoint{4.951874in}{2.028877in}}%
\pgfpathlineto{\pgfqpoint{4.951950in}{1.558291in}}%
\pgfpathlineto{\pgfqpoint{4.952403in}{2.048250in}}%
\pgfpathlineto{\pgfqpoint{4.953083in}{1.935491in}}%
\pgfpathlineto{\pgfqpoint{4.953786in}{1.560448in}}%
\pgfpathlineto{\pgfqpoint{4.953158in}{2.058509in}}%
\pgfpathlineto{\pgfqpoint{4.954137in}{1.992126in}}%
\pgfpathlineto{\pgfqpoint{4.954863in}{2.039452in}}%
\pgfpathlineto{\pgfqpoint{4.954963in}{1.739538in}}%
\pgfpathlineto{\pgfqpoint{4.955138in}{1.929503in}}%
\pgfpathlineto{\pgfqpoint{4.955863in}{1.573994in}}%
\pgfpathlineto{\pgfqpoint{4.955613in}{2.017342in}}%
\pgfpathlineto{\pgfqpoint{4.956237in}{1.952712in}}%
\pgfpathlineto{\pgfqpoint{4.956960in}{1.743432in}}%
\pgfpathlineto{\pgfqpoint{4.957110in}{2.064631in}}%
\pgfpathlineto{\pgfqpoint{4.957284in}{1.920867in}}%
\pgfpathlineto{\pgfqpoint{4.957508in}{2.075207in}}%
\pgfpathlineto{\pgfqpoint{4.957955in}{1.542255in}}%
\pgfpathlineto{\pgfqpoint{4.958353in}{1.795306in}}%
\pgfpathlineto{\pgfqpoint{4.958750in}{2.089633in}}%
\pgfpathlineto{\pgfqpoint{4.958675in}{1.690766in}}%
\pgfpathlineto{\pgfqpoint{4.959468in}{1.896537in}}%
\pgfpathlineto{\pgfqpoint{4.960012in}{1.577383in}}%
\pgfpathlineto{\pgfqpoint{4.959790in}{2.043767in}}%
\pgfpathlineto{\pgfqpoint{4.960581in}{1.871652in}}%
\pgfpathlineto{\pgfqpoint{4.961222in}{2.036731in}}%
\pgfpathlineto{\pgfqpoint{4.960926in}{1.736885in}}%
\pgfpathlineto{\pgfqpoint{4.961641in}{1.905281in}}%
\pgfpathlineto{\pgfqpoint{4.962281in}{1.702525in}}%
\pgfpathlineto{\pgfqpoint{4.961814in}{2.057211in}}%
\pgfpathlineto{\pgfqpoint{4.962748in}{1.875034in}}%
\pgfpathlineto{\pgfqpoint{4.963485in}{1.739543in}}%
\pgfpathlineto{\pgfqpoint{4.963264in}{2.059226in}}%
\pgfpathlineto{\pgfqpoint{4.963779in}{1.945748in}}%
\pgfpathlineto{\pgfqpoint{4.964146in}{2.037376in}}%
\pgfpathlineto{\pgfqpoint{4.964342in}{1.659805in}}%
\pgfpathlineto{\pgfqpoint{4.964807in}{1.840303in}}%
\pgfpathlineto{\pgfqpoint{4.965540in}{1.469138in}}%
\pgfpathlineto{\pgfqpoint{4.965027in}{2.059069in}}%
\pgfpathlineto{\pgfqpoint{4.965881in}{1.833755in}}%
\pgfpathlineto{\pgfqpoint{4.966369in}{2.021953in}}%
\pgfpathlineto{\pgfqpoint{4.966539in}{1.557145in}}%
\pgfpathlineto{\pgfqpoint{4.966953in}{1.979286in}}%
\pgfpathlineto{\pgfqpoint{4.967755in}{1.429321in}}%
\pgfpathlineto{\pgfqpoint{4.967730in}{2.080265in}}%
\pgfpathlineto{\pgfqpoint{4.968070in}{1.763369in}}%
\pgfpathlineto{\pgfqpoint{4.968410in}{2.047451in}}%
\pgfpathlineto{\pgfqpoint{4.968628in}{1.601212in}}%
\pgfpathlineto{\pgfqpoint{4.969185in}{1.926445in}}%
\pgfpathlineto{\pgfqpoint{4.969644in}{2.041210in}}%
\pgfpathlineto{\pgfqpoint{4.970320in}{1.683203in}}%
\pgfpathlineto{\pgfqpoint{4.970513in}{2.057417in}}%
\pgfpathlineto{\pgfqpoint{4.971429in}{1.664643in}}%
\pgfpathlineto{\pgfqpoint{4.971453in}{1.996055in}}%
\pgfpathlineto{\pgfqpoint{4.972367in}{1.617746in}}%
\pgfpathlineto{\pgfqpoint{4.972175in}{2.060260in}}%
\pgfpathlineto{\pgfqpoint{4.972583in}{1.832293in}}%
\pgfpathlineto{\pgfqpoint{4.973446in}{2.029186in}}%
\pgfpathlineto{\pgfqpoint{4.972679in}{1.710281in}}%
\pgfpathlineto{\pgfqpoint{4.973662in}{2.026606in}}%
\pgfpathlineto{\pgfqpoint{4.973686in}{1.545862in}}%
\pgfpathlineto{\pgfqpoint{4.974212in}{2.043173in}}%
\pgfpathlineto{\pgfqpoint{4.974762in}{1.994187in}}%
\pgfpathlineto{\pgfqpoint{4.974905in}{1.927131in}}%
\pgfpathlineto{\pgfqpoint{4.974976in}{2.016577in}}%
\pgfpathlineto{\pgfqpoint{4.975000in}{1.995278in}}%
\pgfpathlineto{\pgfqpoint{4.975215in}{1.750007in}}%
\pgfpathlineto{\pgfqpoint{4.975954in}{2.053263in}}%
\pgfpathlineto{\pgfqpoint{4.976097in}{1.968534in}}%
\pgfpathlineto{\pgfqpoint{4.976977in}{2.064174in}}%
\pgfpathlineto{\pgfqpoint{4.976905in}{1.656480in}}%
\pgfpathlineto{\pgfqpoint{4.977048in}{2.009980in}}%
\pgfpathlineto{\pgfqpoint{4.977072in}{1.557235in}}%
\pgfpathlineto{\pgfqpoint{4.977950in}{2.073454in}}%
\pgfpathlineto{\pgfqpoint{4.978139in}{1.904787in}}%
\pgfpathlineto{\pgfqpoint{4.978423in}{2.042863in}}%
\pgfpathlineto{\pgfqpoint{4.979062in}{1.525308in}}%
\pgfpathlineto{\pgfqpoint{4.979157in}{1.898929in}}%
\pgfpathlineto{\pgfqpoint{4.979535in}{1.630544in}}%
\pgfpathlineto{\pgfqpoint{4.980195in}{2.037154in}}%
\pgfpathlineto{\pgfqpoint{4.980242in}{1.868403in}}%
\pgfpathlineto{\pgfqpoint{4.981278in}{2.085124in}}%
\pgfpathlineto{\pgfqpoint{4.980949in}{1.682551in}}%
\pgfpathlineto{\pgfqpoint{4.981349in}{1.892765in}}%
\pgfpathlineto{\pgfqpoint{4.982359in}{2.090403in}}%
\pgfpathlineto{\pgfqpoint{4.981983in}{1.715775in}}%
\pgfpathlineto{\pgfqpoint{4.982500in}{1.981921in}}%
\pgfpathlineto{\pgfqpoint{4.983600in}{1.687809in}}%
\pgfpathlineto{\pgfqpoint{4.982617in}{2.041046in}}%
\pgfpathlineto{\pgfqpoint{4.983624in}{1.837556in}}%
\pgfpathlineto{\pgfqpoint{4.984558in}{2.032593in}}%
\pgfpathlineto{\pgfqpoint{4.984371in}{1.545301in}}%
\pgfpathlineto{\pgfqpoint{4.984745in}{1.919927in}}%
\pgfpathlineto{\pgfqpoint{4.985304in}{2.070635in}}%
\pgfpathlineto{\pgfqpoint{4.984955in}{1.655667in}}%
\pgfpathlineto{\pgfqpoint{4.985816in}{1.935389in}}%
\pgfpathlineto{\pgfqpoint{4.985863in}{1.642392in}}%
\pgfpathlineto{\pgfqpoint{4.986444in}{2.067631in}}%
\pgfpathlineto{\pgfqpoint{4.986909in}{1.987385in}}%
\pgfpathlineto{\pgfqpoint{4.987465in}{2.051270in}}%
\pgfpathlineto{\pgfqpoint{4.987071in}{1.500127in}}%
\pgfpathlineto{\pgfqpoint{4.987650in}{2.044079in}}%
\pgfpathlineto{\pgfqpoint{4.988114in}{1.567637in}}%
\pgfpathlineto{\pgfqpoint{4.988761in}{1.981004in}}%
\pgfpathlineto{\pgfqpoint{4.989315in}{1.748332in}}%
\pgfpathlineto{\pgfqpoint{4.989684in}{2.038664in}}%
\pgfpathlineto{\pgfqpoint{4.989869in}{1.943773in}}%
\pgfpathlineto{\pgfqpoint{4.990513in}{2.044267in}}%
\pgfpathlineto{\pgfqpoint{4.989915in}{1.666958in}}%
\pgfpathlineto{\pgfqpoint{4.990766in}{1.761109in}}%
\pgfpathlineto{\pgfqpoint{4.991157in}{1.631660in}}%
\pgfpathlineto{\pgfqpoint{4.991364in}{2.071761in}}%
\pgfpathlineto{\pgfqpoint{4.991823in}{1.912564in}}%
\pgfpathlineto{\pgfqpoint{4.992350in}{2.079844in}}%
\pgfpathlineto{\pgfqpoint{4.992007in}{1.680438in}}%
\pgfpathlineto{\pgfqpoint{4.992946in}{1.958037in}}%
\pgfpathlineto{\pgfqpoint{4.993677in}{1.660347in}}%
\pgfpathlineto{\pgfqpoint{4.992991in}{2.070633in}}%
\pgfpathlineto{\pgfqpoint{4.994042in}{2.004853in}}%
\pgfpathlineto{\pgfqpoint{4.994909in}{1.640463in}}%
\pgfpathlineto{\pgfqpoint{4.994954in}{2.059142in}}%
\pgfpathlineto{\pgfqpoint{4.995159in}{1.701886in}}%
\pgfpathlineto{\pgfqpoint{4.995341in}{2.076978in}}%
\pgfpathlineto{\pgfqpoint{4.995614in}{1.602364in}}%
\pgfpathlineto{\pgfqpoint{4.996273in}{2.017704in}}%
\pgfpathlineto{\pgfqpoint{4.996817in}{1.596639in}}%
\pgfpathlineto{\pgfqpoint{4.996795in}{2.101233in}}%
\pgfpathlineto{\pgfqpoint{4.997520in}{1.868045in}}%
\pgfpathlineto{\pgfqpoint{4.997678in}{2.067835in}}%
\pgfpathlineto{\pgfqpoint{4.998469in}{1.571722in}}%
\pgfpathlineto{\pgfqpoint{4.998627in}{1.906416in}}%
\pgfpathlineto{\pgfqpoint{4.998672in}{2.059076in}}%
\pgfpathlineto{\pgfqpoint{4.999124in}{1.602366in}}%
\pgfpathlineto{\pgfqpoint{4.999732in}{2.003568in}}%
\pgfpathlineto{\pgfqpoint{5.000070in}{1.549453in}}%
\pgfpathlineto{\pgfqpoint{4.999980in}{2.092033in}}%
\pgfpathlineto{\pgfqpoint{5.000834in}{1.820349in}}%
\pgfpathlineto{\pgfqpoint{5.001305in}{2.108374in}}%
\pgfpathlineto{\pgfqpoint{5.001170in}{1.657987in}}%
\pgfpathlineto{\pgfqpoint{5.001978in}{2.065747in}}%
\pgfpathlineto{\pgfqpoint{5.002358in}{1.552050in}}%
\pgfpathlineto{\pgfqpoint{5.002470in}{2.073211in}}%
\pgfpathlineto{\pgfqpoint{5.003118in}{1.932408in}}%
\pgfpathlineto{\pgfqpoint{5.003386in}{2.040332in}}%
\pgfpathlineto{\pgfqpoint{5.003676in}{1.644704in}}%
\pgfpathlineto{\pgfqpoint{5.004033in}{1.853962in}}%
\pgfpathlineto{\pgfqpoint{5.004523in}{1.568834in}}%
\pgfpathlineto{\pgfqpoint{5.005013in}{2.063562in}}%
\pgfpathlineto{\pgfqpoint{5.005102in}{1.949716in}}%
\pgfpathlineto{\pgfqpoint{5.005591in}{2.094642in}}%
\pgfpathlineto{\pgfqpoint{5.005391in}{1.716046in}}%
\pgfpathlineto{\pgfqpoint{5.006168in}{1.985712in}}%
\pgfpathlineto{\pgfqpoint{5.006190in}{1.516861in}}%
\pgfpathlineto{\pgfqpoint{5.006500in}{2.056714in}}%
\pgfpathlineto{\pgfqpoint{5.007275in}{2.034180in}}%
\pgfpathlineto{\pgfqpoint{5.007872in}{1.576504in}}%
\pgfpathlineto{\pgfqpoint{5.007607in}{2.076250in}}%
\pgfpathlineto{\pgfqpoint{5.008379in}{2.015231in}}%
\pgfpathlineto{\pgfqpoint{5.008534in}{1.701297in}}%
\pgfpathlineto{\pgfqpoint{5.008909in}{2.045471in}}%
\pgfpathlineto{\pgfqpoint{5.009613in}{1.865332in}}%
\pgfpathlineto{\pgfqpoint{5.009679in}{2.072720in}}%
\pgfpathlineto{\pgfqpoint{5.009789in}{1.770918in}}%
\pgfpathlineto{\pgfqpoint{5.010712in}{2.025444in}}%
\pgfpathlineto{\pgfqpoint{5.010931in}{1.708941in}}%
\pgfpathlineto{\pgfqpoint{5.011194in}{2.057736in}}%
\pgfpathlineto{\pgfqpoint{5.011807in}{1.854198in}}%
\pgfpathlineto{\pgfqpoint{5.011829in}{2.097850in}}%
\pgfpathlineto{\pgfqpoint{5.012769in}{1.623633in}}%
\pgfpathlineto{\pgfqpoint{5.012922in}{1.935643in}}%
\pgfpathlineto{\pgfqpoint{5.013424in}{2.055576in}}%
\pgfpathlineto{\pgfqpoint{5.013162in}{1.284620in}}%
\pgfpathlineto{\pgfqpoint{5.014012in}{1.909025in}}%
\pgfpathlineto{\pgfqpoint{5.014512in}{2.058861in}}%
\pgfpathlineto{\pgfqpoint{5.014621in}{1.723442in}}%
\pgfpathlineto{\pgfqpoint{5.015034in}{1.852066in}}%
\pgfpathlineto{\pgfqpoint{5.015273in}{1.737769in}}%
\pgfpathlineto{\pgfqpoint{5.015099in}{2.034103in}}%
\pgfpathlineto{\pgfqpoint{5.015750in}{1.938055in}}%
\pgfpathlineto{\pgfqpoint{5.015945in}{2.052057in}}%
\pgfpathlineto{\pgfqpoint{5.016443in}{1.709920in}}%
\pgfpathlineto{\pgfqpoint{5.016833in}{1.826522in}}%
\pgfpathlineto{\pgfqpoint{5.017632in}{2.075192in}}%
\pgfpathlineto{\pgfqpoint{5.017611in}{1.552676in}}%
\pgfpathlineto{\pgfqpoint{5.018021in}{2.040936in}}%
\pgfpathlineto{\pgfqpoint{5.018775in}{1.640231in}}%
\pgfpathlineto{\pgfqpoint{5.019119in}{2.093915in}}%
\pgfpathlineto{\pgfqpoint{5.019141in}{1.881184in}}%
\pgfpathlineto{\pgfqpoint{5.019291in}{2.104611in}}%
\pgfpathlineto{\pgfqpoint{5.019807in}{1.631776in}}%
\pgfpathlineto{\pgfqpoint{5.020236in}{1.939792in}}%
\pgfpathlineto{\pgfqpoint{5.020343in}{1.508106in}}%
\pgfpathlineto{\pgfqpoint{5.020751in}{2.064019in}}%
\pgfpathlineto{\pgfqpoint{5.021307in}{1.987715in}}%
\pgfpathlineto{\pgfqpoint{5.021329in}{2.084443in}}%
\pgfpathlineto{\pgfqpoint{5.021692in}{1.776177in}}%
\pgfpathlineto{\pgfqpoint{5.022397in}{1.987685in}}%
\pgfpathlineto{\pgfqpoint{5.023101in}{1.594989in}}%
\pgfpathlineto{\pgfqpoint{5.022589in}{2.085974in}}%
\pgfpathlineto{\pgfqpoint{5.023506in}{1.902621in}}%
\pgfpathlineto{\pgfqpoint{5.023910in}{2.084395in}}%
\pgfpathlineto{\pgfqpoint{5.023612in}{1.643559in}}%
\pgfpathlineto{\pgfqpoint{5.024612in}{1.937900in}}%
\pgfpathlineto{\pgfqpoint{5.025248in}{1.676003in}}%
\pgfpathlineto{\pgfqpoint{5.024951in}{2.067100in}}%
\pgfpathlineto{\pgfqpoint{5.025587in}{1.803139in}}%
\pgfpathlineto{\pgfqpoint{5.026370in}{2.079721in}}%
\pgfpathlineto{\pgfqpoint{5.025926in}{1.468046in}}%
\pgfpathlineto{\pgfqpoint{5.026687in}{1.836952in}}%
\pgfpathlineto{\pgfqpoint{5.027258in}{2.076131in}}%
\pgfpathlineto{\pgfqpoint{5.027574in}{1.671341in}}%
\pgfpathlineto{\pgfqpoint{5.027827in}{2.005397in}}%
\pgfpathlineto{\pgfqpoint{5.028900in}{1.759035in}}%
\pgfpathlineto{\pgfqpoint{5.028627in}{2.073063in}}%
\pgfpathlineto{\pgfqpoint{5.028942in}{1.975256in}}%
\pgfpathlineto{\pgfqpoint{5.029782in}{1.647808in}}%
\pgfpathlineto{\pgfqpoint{5.029384in}{2.103029in}}%
\pgfpathlineto{\pgfqpoint{5.030034in}{1.945249in}}%
\pgfpathlineto{\pgfqpoint{5.030662in}{1.557005in}}%
\pgfpathlineto{\pgfqpoint{5.031164in}{2.076459in}}%
\pgfpathlineto{\pgfqpoint{5.031937in}{1.685817in}}%
\pgfpathlineto{\pgfqpoint{5.032271in}{1.810043in}}%
\pgfpathlineto{\pgfqpoint{5.032542in}{2.071023in}}%
\pgfpathlineto{\pgfqpoint{5.032667in}{1.667903in}}%
\pgfpathlineto{\pgfqpoint{5.033396in}{1.986544in}}%
\pgfpathlineto{\pgfqpoint{5.034061in}{1.613670in}}%
\pgfpathlineto{\pgfqpoint{5.034248in}{2.079887in}}%
\pgfpathlineto{\pgfqpoint{5.034517in}{1.807002in}}%
\pgfpathlineto{\pgfqpoint{5.034766in}{2.109961in}}%
\pgfpathlineto{\pgfqpoint{5.034932in}{1.362099in}}%
\pgfpathlineto{\pgfqpoint{5.035616in}{1.886633in}}%
\pgfpathlineto{\pgfqpoint{5.036649in}{1.633013in}}%
\pgfpathlineto{\pgfqpoint{5.036504in}{2.054537in}}%
\pgfpathlineto{\pgfqpoint{5.036731in}{1.866986in}}%
\pgfpathlineto{\pgfqpoint{5.037412in}{1.645849in}}%
\pgfpathlineto{\pgfqpoint{5.036896in}{2.072069in}}%
\pgfpathlineto{\pgfqpoint{5.037700in}{1.796168in}}%
\pgfpathlineto{\pgfqpoint{5.038112in}{2.070000in}}%
\pgfpathlineto{\pgfqpoint{5.038564in}{1.671582in}}%
\pgfpathlineto{\pgfqpoint{5.038810in}{1.966591in}}%
\pgfpathlineto{\pgfqpoint{5.039344in}{1.555836in}}%
\pgfpathlineto{\pgfqpoint{5.039734in}{2.077215in}}%
\pgfpathlineto{\pgfqpoint{5.039897in}{2.010489in}}%
\pgfpathlineto{\pgfqpoint{5.040532in}{1.449272in}}%
\pgfpathlineto{\pgfqpoint{5.040266in}{2.061182in}}%
\pgfpathlineto{\pgfqpoint{5.040982in}{2.026763in}}%
\pgfpathlineto{\pgfqpoint{5.041002in}{2.027755in}}%
\pgfpathlineto{\pgfqpoint{5.041023in}{1.985480in}}%
\pgfpathlineto{\pgfqpoint{5.041941in}{1.631582in}}%
\pgfpathlineto{\pgfqpoint{5.041063in}{2.095844in}}%
\pgfpathlineto{\pgfqpoint{5.042145in}{1.897270in}}%
\pgfpathlineto{\pgfqpoint{5.043040in}{2.085159in}}%
\pgfpathlineto{\pgfqpoint{5.042654in}{1.609698in}}%
\pgfpathlineto{\pgfqpoint{5.043264in}{1.944220in}}%
\pgfpathlineto{\pgfqpoint{5.043771in}{2.085872in}}%
\pgfpathlineto{\pgfqpoint{5.043426in}{1.552137in}}%
\pgfpathlineto{\pgfqpoint{5.044339in}{1.991505in}}%
\pgfpathlineto{\pgfqpoint{5.045311in}{1.688009in}}%
\pgfpathlineto{\pgfqpoint{5.044805in}{2.072721in}}%
\pgfpathlineto{\pgfqpoint{5.045453in}{1.900738in}}%
\pgfpathlineto{\pgfqpoint{5.045796in}{2.104432in}}%
\pgfpathlineto{\pgfqpoint{5.046018in}{1.716232in}}%
\pgfpathlineto{\pgfqpoint{5.046563in}{2.015212in}}%
\pgfpathlineto{\pgfqpoint{5.047288in}{1.675218in}}%
\pgfpathlineto{\pgfqpoint{5.047268in}{2.071435in}}%
\pgfpathlineto{\pgfqpoint{5.047651in}{1.923647in}}%
\pgfpathlineto{\pgfqpoint{5.048635in}{2.087323in}}%
\pgfpathlineto{\pgfqpoint{5.048213in}{1.707440in}}%
\pgfpathlineto{\pgfqpoint{5.048755in}{2.074245in}}%
\pgfpathlineto{\pgfqpoint{5.049076in}{1.625170in}}%
\pgfpathlineto{\pgfqpoint{5.049397in}{2.090652in}}%
\pgfpathlineto{\pgfqpoint{5.049897in}{1.843277in}}%
\pgfpathlineto{\pgfqpoint{5.050617in}{2.063596in}}%
\pgfpathlineto{\pgfqpoint{5.049937in}{1.718746in}}%
\pgfpathlineto{\pgfqpoint{5.050996in}{1.958722in}}%
\pgfpathlineto{\pgfqpoint{5.051893in}{1.572922in}}%
\pgfpathlineto{\pgfqpoint{5.052032in}{2.096374in}}%
\pgfpathlineto{\pgfqpoint{5.052072in}{1.847024in}}%
\pgfpathlineto{\pgfqpoint{5.052251in}{2.098685in}}%
\pgfpathlineto{\pgfqpoint{5.052967in}{1.722745in}}%
\pgfpathlineto{\pgfqpoint{5.053185in}{1.948004in}}%
\pgfpathlineto{\pgfqpoint{5.054018in}{1.699973in}}%
\pgfpathlineto{\pgfqpoint{5.053324in}{2.107294in}}%
\pgfpathlineto{\pgfqpoint{5.054276in}{1.855007in}}%
\pgfpathlineto{\pgfqpoint{5.055344in}{2.088408in}}%
\pgfpathlineto{\pgfqpoint{5.054573in}{1.714704in}}%
\pgfpathlineto{\pgfqpoint{5.055383in}{2.015558in}}%
\pgfpathlineto{\pgfqpoint{5.056488in}{1.597644in}}%
\pgfpathlineto{\pgfqpoint{5.055620in}{2.098004in}}%
\pgfpathlineto{\pgfqpoint{5.056507in}{1.920246in}}%
\pgfpathlineto{\pgfqpoint{5.056980in}{2.055457in}}%
\pgfpathlineto{\pgfqpoint{5.056881in}{1.723372in}}%
\pgfpathlineto{\pgfqpoint{5.057570in}{1.917224in}}%
\pgfpathlineto{\pgfqpoint{5.057825in}{1.529698in}}%
\pgfpathlineto{\pgfqpoint{5.058237in}{2.058167in}}%
\pgfpathlineto{\pgfqpoint{5.058649in}{2.025329in}}%
\pgfpathlineto{\pgfqpoint{5.058864in}{1.858608in}}%
\pgfpathlineto{\pgfqpoint{5.058688in}{2.070963in}}%
\pgfpathlineto{\pgfqpoint{5.059041in}{2.007169in}}%
\pgfpathlineto{\pgfqpoint{5.059921in}{1.466732in}}%
\pgfpathlineto{\pgfqpoint{5.059432in}{2.103155in}}%
\pgfpathlineto{\pgfqpoint{5.060136in}{2.010033in}}%
\pgfpathlineto{\pgfqpoint{5.060155in}{2.105796in}}%
\pgfpathlineto{\pgfqpoint{5.061150in}{1.717754in}}%
\pgfpathlineto{\pgfqpoint{5.061228in}{1.923461in}}%
\pgfpathlineto{\pgfqpoint{5.061792in}{2.076167in}}%
\pgfpathlineto{\pgfqpoint{5.061267in}{1.765702in}}%
\pgfpathlineto{\pgfqpoint{5.062065in}{1.911381in}}%
\pgfpathlineto{\pgfqpoint{5.062084in}{1.617882in}}%
\pgfpathlineto{\pgfqpoint{5.062104in}{2.091020in}}%
\pgfpathlineto{\pgfqpoint{5.063171in}{1.977729in}}%
\pgfpathlineto{\pgfqpoint{5.063675in}{2.103808in}}%
\pgfpathlineto{\pgfqpoint{5.063695in}{1.757074in}}%
\pgfpathlineto{\pgfqpoint{5.064236in}{2.008945in}}%
\pgfpathlineto{\pgfqpoint{5.064797in}{1.734787in}}%
\pgfpathlineto{\pgfqpoint{5.064526in}{2.121632in}}%
\pgfpathlineto{\pgfqpoint{5.065337in}{1.989892in}}%
\pgfpathlineto{\pgfqpoint{5.065550in}{2.088462in}}%
\pgfpathlineto{\pgfqpoint{5.065896in}{1.772512in}}%
\pgfpathlineto{\pgfqpoint{5.066416in}{1.936586in}}%
\pgfpathlineto{\pgfqpoint{5.067435in}{1.658525in}}%
\pgfpathlineto{\pgfqpoint{5.066993in}{2.121322in}}%
\pgfpathlineto{\pgfqpoint{5.067473in}{1.953297in}}%
\pgfpathlineto{\pgfqpoint{5.067761in}{2.106292in}}%
\pgfpathlineto{\pgfqpoint{5.067684in}{1.662153in}}%
\pgfpathlineto{\pgfqpoint{5.068604in}{2.040994in}}%
\pgfpathlineto{\pgfqpoint{5.069216in}{1.639613in}}%
\pgfpathlineto{\pgfqpoint{5.069293in}{2.118446in}}%
\pgfpathlineto{\pgfqpoint{5.069713in}{2.001359in}}%
\pgfpathlineto{\pgfqpoint{5.070247in}{2.114277in}}%
\pgfpathlineto{\pgfqpoint{5.070571in}{1.699480in}}%
\pgfpathlineto{\pgfqpoint{5.070781in}{1.955180in}}%
\pgfpathlineto{\pgfqpoint{5.071048in}{1.596877in}}%
\pgfpathlineto{\pgfqpoint{5.071371in}{2.084636in}}%
\pgfpathlineto{\pgfqpoint{5.071865in}{1.952314in}}%
\pgfpathlineto{\pgfqpoint{5.072055in}{2.088267in}}%
\pgfpathlineto{\pgfqpoint{5.072416in}{1.756637in}}%
\pgfpathlineto{\pgfqpoint{5.072966in}{1.898249in}}%
\pgfpathlineto{\pgfqpoint{5.073250in}{2.092839in}}%
\pgfpathlineto{\pgfqpoint{5.073136in}{1.740729in}}%
\pgfpathlineto{\pgfqpoint{5.074082in}{1.929303in}}%
\pgfpathlineto{\pgfqpoint{5.074837in}{2.116940in}}%
\pgfpathlineto{\pgfqpoint{5.074630in}{1.702119in}}%
\pgfpathlineto{\pgfqpoint{5.075158in}{1.855624in}}%
\pgfpathlineto{\pgfqpoint{5.075591in}{2.100815in}}%
\pgfpathlineto{\pgfqpoint{5.075478in}{1.742648in}}%
\pgfpathlineto{\pgfqpoint{5.076099in}{2.008069in}}%
\pgfpathlineto{\pgfqpoint{5.077114in}{1.707222in}}%
\pgfpathlineto{\pgfqpoint{5.076982in}{2.076361in}}%
\pgfpathlineto{\pgfqpoint{5.077207in}{2.015630in}}%
\pgfpathlineto{\pgfqpoint{5.077882in}{1.723477in}}%
\pgfpathlineto{\pgfqpoint{5.078051in}{2.086651in}}%
\pgfpathlineto{\pgfqpoint{5.078331in}{1.807161in}}%
\pgfpathlineto{\pgfqpoint{5.079453in}{2.110134in}}%
\pgfpathlineto{\pgfqpoint{5.078911in}{1.614076in}}%
\pgfpathlineto{\pgfqpoint{5.079471in}{2.095713in}}%
\pgfpathlineto{\pgfqpoint{5.080422in}{2.136867in}}%
\pgfpathlineto{\pgfqpoint{5.080608in}{1.513965in}}%
\pgfpathlineto{\pgfqpoint{5.081314in}{2.087777in}}%
\pgfpathlineto{\pgfqpoint{5.081723in}{1.858889in}}%
\pgfpathlineto{\pgfqpoint{5.082335in}{2.080593in}}%
\pgfpathlineto{\pgfqpoint{5.082224in}{1.672201in}}%
\pgfpathlineto{\pgfqpoint{5.082872in}{2.054293in}}%
\pgfpathlineto{\pgfqpoint{5.083131in}{1.651213in}}%
\pgfpathlineto{\pgfqpoint{5.083593in}{2.067182in}}%
\pgfpathlineto{\pgfqpoint{5.084000in}{1.953197in}}%
\pgfpathlineto{\pgfqpoint{5.084553in}{2.091437in}}%
\pgfpathlineto{\pgfqpoint{5.084664in}{1.604904in}}%
\pgfpathlineto{\pgfqpoint{5.085124in}{1.997758in}}%
\pgfpathlineto{\pgfqpoint{5.085658in}{1.654331in}}%
\pgfpathlineto{\pgfqpoint{5.085750in}{2.084438in}}%
\pgfpathlineto{\pgfqpoint{5.086228in}{1.986722in}}%
\pgfpathlineto{\pgfqpoint{5.086943in}{2.109697in}}%
\pgfpathlineto{\pgfqpoint{5.086411in}{1.757541in}}%
\pgfpathlineto{\pgfqpoint{5.087328in}{1.944808in}}%
\pgfpathlineto{\pgfqpoint{5.088042in}{1.650654in}}%
\pgfpathlineto{\pgfqpoint{5.087401in}{2.102202in}}%
\pgfpathlineto{\pgfqpoint{5.088444in}{1.829337in}}%
\pgfpathlineto{\pgfqpoint{5.089411in}{2.075352in}}%
\pgfpathlineto{\pgfqpoint{5.088535in}{1.647497in}}%
\pgfpathlineto{\pgfqpoint{5.089539in}{1.804986in}}%
\pgfpathlineto{\pgfqpoint{5.089557in}{1.705784in}}%
\pgfpathlineto{\pgfqpoint{5.090431in}{2.119905in}}%
\pgfpathlineto{\pgfqpoint{5.090631in}{1.905395in}}%
\pgfpathlineto{\pgfqpoint{5.090830in}{2.127388in}}%
\pgfpathlineto{\pgfqpoint{5.090740in}{1.608989in}}%
\pgfpathlineto{\pgfqpoint{5.091520in}{1.904162in}}%
\pgfpathlineto{\pgfqpoint{5.092209in}{1.629258in}}%
\pgfpathlineto{\pgfqpoint{5.091756in}{2.124161in}}%
\pgfpathlineto{\pgfqpoint{5.092625in}{1.835960in}}%
\pgfpathlineto{\pgfqpoint{5.092932in}{2.122646in}}%
\pgfpathlineto{\pgfqpoint{5.092806in}{1.589478in}}%
\pgfpathlineto{\pgfqpoint{5.093763in}{1.962502in}}%
\pgfpathlineto{\pgfqpoint{5.093962in}{1.696295in}}%
\pgfpathlineto{\pgfqpoint{5.094214in}{2.076631in}}%
\pgfpathlineto{\pgfqpoint{5.094862in}{1.949809in}}%
\pgfpathlineto{\pgfqpoint{5.095851in}{2.117045in}}%
\pgfpathlineto{\pgfqpoint{5.095725in}{1.512924in}}%
\pgfpathlineto{\pgfqpoint{5.095976in}{1.975327in}}%
\pgfpathlineto{\pgfqpoint{5.096174in}{2.104172in}}%
\pgfpathlineto{\pgfqpoint{5.096084in}{1.723076in}}%
\pgfpathlineto{\pgfqpoint{5.096837in}{1.951891in}}%
\pgfpathlineto{\pgfqpoint{5.096855in}{1.704672in}}%
\pgfpathlineto{\pgfqpoint{5.097016in}{2.104017in}}%
\pgfpathlineto{\pgfqpoint{5.097946in}{1.958133in}}%
\pgfpathlineto{\pgfqpoint{5.098303in}{1.656873in}}%
\pgfpathlineto{\pgfqpoint{5.098392in}{2.114296in}}%
\pgfpathlineto{\pgfqpoint{5.099034in}{1.963847in}}%
\pgfpathlineto{\pgfqpoint{5.099052in}{2.109945in}}%
\pgfpathlineto{\pgfqpoint{5.099658in}{1.728954in}}%
\pgfpathlineto{\pgfqpoint{5.100138in}{2.011901in}}%
\pgfpathlineto{\pgfqpoint{5.100493in}{2.122852in}}%
\pgfpathlineto{\pgfqpoint{5.100617in}{1.643586in}}%
\pgfpathlineto{\pgfqpoint{5.100848in}{1.987545in}}%
\pgfpathlineto{\pgfqpoint{5.101787in}{1.628884in}}%
\pgfpathlineto{\pgfqpoint{5.101575in}{2.078327in}}%
\pgfpathlineto{\pgfqpoint{5.101947in}{1.817261in}}%
\pgfpathlineto{\pgfqpoint{5.102901in}{2.138457in}}%
\pgfpathlineto{\pgfqpoint{5.102406in}{1.790954in}}%
\pgfpathlineto{\pgfqpoint{5.103060in}{2.080105in}}%
\pgfpathlineto{\pgfqpoint{5.104082in}{1.412194in}}%
\pgfpathlineto{\pgfqpoint{5.103836in}{2.136543in}}%
\pgfpathlineto{\pgfqpoint{5.104171in}{1.918852in}}%
\pgfpathlineto{\pgfqpoint{5.104575in}{2.118204in}}%
\pgfpathlineto{\pgfqpoint{5.104435in}{1.640161in}}%
\pgfpathlineto{\pgfqpoint{5.105278in}{2.041544in}}%
\pgfpathlineto{\pgfqpoint{5.106348in}{1.719813in}}%
\pgfpathlineto{\pgfqpoint{5.105629in}{2.102609in}}%
\pgfpathlineto{\pgfqpoint{5.106418in}{1.858522in}}%
\pgfpathlineto{\pgfqpoint{5.106558in}{2.119005in}}%
\pgfpathlineto{\pgfqpoint{5.106610in}{1.522461in}}%
\pgfpathlineto{\pgfqpoint{5.107537in}{2.038445in}}%
\pgfpathlineto{\pgfqpoint{5.107572in}{1.648874in}}%
\pgfpathlineto{\pgfqpoint{5.107607in}{2.116943in}}%
\pgfpathlineto{\pgfqpoint{5.108653in}{1.956078in}}%
\pgfpathlineto{\pgfqpoint{5.109558in}{2.097136in}}%
\pgfpathlineto{\pgfqpoint{5.108810in}{1.672286in}}%
\pgfpathlineto{\pgfqpoint{5.109610in}{2.024000in}}%
\pgfpathlineto{\pgfqpoint{5.110565in}{1.549342in}}%
\pgfpathlineto{\pgfqpoint{5.109923in}{2.100769in}}%
\pgfpathlineto{\pgfqpoint{5.110704in}{1.894117in}}%
\pgfpathlineto{\pgfqpoint{5.110739in}{2.127107in}}%
\pgfpathlineto{\pgfqpoint{5.110998in}{1.659570in}}%
\pgfpathlineto{\pgfqpoint{5.111812in}{2.007167in}}%
\pgfpathlineto{\pgfqpoint{5.112227in}{1.377524in}}%
\pgfpathlineto{\pgfqpoint{5.111985in}{2.111768in}}%
\pgfpathlineto{\pgfqpoint{5.112917in}{1.916010in}}%
\pgfpathlineto{\pgfqpoint{5.113778in}{2.116639in}}%
\pgfpathlineto{\pgfqpoint{5.113710in}{1.610798in}}%
\pgfpathlineto{\pgfqpoint{5.114036in}{1.994133in}}%
\pgfpathlineto{\pgfqpoint{5.114466in}{1.597409in}}%
\pgfpathlineto{\pgfqpoint{5.114294in}{2.131513in}}%
\pgfpathlineto{\pgfqpoint{5.115153in}{1.799288in}}%
\pgfpathlineto{\pgfqpoint{5.115941in}{2.108514in}}%
\pgfpathlineto{\pgfqpoint{5.116232in}{1.638048in}}%
\pgfpathlineto{\pgfqpoint{5.116266in}{1.905934in}}%
\pgfpathlineto{\pgfqpoint{5.116967in}{2.088129in}}%
\pgfpathlineto{\pgfqpoint{5.116745in}{1.753684in}}%
\pgfpathlineto{\pgfqpoint{5.117343in}{2.007366in}}%
\pgfpathlineto{\pgfqpoint{5.117718in}{1.521222in}}%
\pgfpathlineto{\pgfqpoint{5.117497in}{2.126972in}}%
\pgfpathlineto{\pgfqpoint{5.118451in}{1.907635in}}%
\pgfpathlineto{\pgfqpoint{5.118757in}{2.103420in}}%
\pgfpathlineto{\pgfqpoint{5.119199in}{1.645482in}}%
\pgfpathlineto{\pgfqpoint{5.119556in}{1.980964in}}%
\pgfpathlineto{\pgfqpoint{5.119997in}{1.791776in}}%
\pgfpathlineto{\pgfqpoint{5.120370in}{2.126356in}}%
\pgfpathlineto{\pgfqpoint{5.120641in}{1.906451in}}%
\pgfpathlineto{\pgfqpoint{5.121368in}{2.115403in}}%
\pgfpathlineto{\pgfqpoint{5.120810in}{1.612201in}}%
\pgfpathlineto{\pgfqpoint{5.121757in}{2.008851in}}%
\pgfpathlineto{\pgfqpoint{5.122870in}{2.125273in}}%
\pgfpathlineto{\pgfqpoint{5.122112in}{1.709312in}}%
\pgfpathlineto{\pgfqpoint{5.122887in}{2.092767in}}%
\pgfpathlineto{\pgfqpoint{5.122988in}{1.378397in}}%
\pgfpathlineto{\pgfqpoint{5.123140in}{2.140653in}}%
\pgfpathlineto{\pgfqpoint{5.123997in}{2.004188in}}%
\pgfpathlineto{\pgfqpoint{5.124266in}{2.102760in}}%
\pgfpathlineto{\pgfqpoint{5.124518in}{1.761483in}}%
\pgfpathlineto{\pgfqpoint{5.125055in}{1.958104in}}%
\pgfpathlineto{\pgfqpoint{5.125440in}{1.470540in}}%
\pgfpathlineto{\pgfqpoint{5.125457in}{2.144033in}}%
\pgfpathlineto{\pgfqpoint{5.126176in}{1.843420in}}%
\pgfpathlineto{\pgfqpoint{5.126226in}{1.935507in}}%
\pgfpathlineto{\pgfqpoint{5.126761in}{2.123532in}}%
\pgfpathlineto{\pgfqpoint{5.126260in}{1.744074in}}%
\pgfpathlineto{\pgfqpoint{5.127311in}{2.009035in}}%
\pgfpathlineto{\pgfqpoint{5.127961in}{1.797783in}}%
\pgfpathlineto{\pgfqpoint{5.127578in}{2.146274in}}%
\pgfpathlineto{\pgfqpoint{5.128410in}{1.917175in}}%
\pgfpathlineto{\pgfqpoint{5.128975in}{2.129140in}}%
\pgfpathlineto{\pgfqpoint{5.129041in}{1.663675in}}%
\pgfpathlineto{\pgfqpoint{5.129489in}{2.024018in}}%
\pgfpathlineto{\pgfqpoint{5.129506in}{1.622064in}}%
\pgfpathlineto{\pgfqpoint{5.129870in}{2.103199in}}%
\pgfpathlineto{\pgfqpoint{5.130599in}{1.937779in}}%
\pgfpathlineto{\pgfqpoint{5.131029in}{2.141908in}}%
\pgfpathlineto{\pgfqpoint{5.131161in}{1.591753in}}%
\pgfpathlineto{\pgfqpoint{5.131656in}{1.900789in}}%
\pgfpathlineto{\pgfqpoint{5.132068in}{1.771646in}}%
\pgfpathlineto{\pgfqpoint{5.131805in}{2.147663in}}%
\pgfpathlineto{\pgfqpoint{5.132447in}{2.003271in}}%
\pgfpathlineto{\pgfqpoint{5.133270in}{2.146769in}}%
\pgfpathlineto{\pgfqpoint{5.133484in}{1.746857in}}%
\pgfpathlineto{\pgfqpoint{5.133549in}{2.034064in}}%
\pgfpathlineto{\pgfqpoint{5.133993in}{1.763729in}}%
\pgfpathlineto{\pgfqpoint{5.133829in}{2.116494in}}%
\pgfpathlineto{\pgfqpoint{5.134632in}{1.909767in}}%
\pgfpathlineto{\pgfqpoint{5.135516in}{2.124822in}}%
\pgfpathlineto{\pgfqpoint{5.134993in}{1.665577in}}%
\pgfpathlineto{\pgfqpoint{5.135745in}{1.985487in}}%
\pgfpathlineto{\pgfqpoint{5.136545in}{2.115002in}}%
\pgfpathlineto{\pgfqpoint{5.135794in}{1.678341in}}%
\pgfpathlineto{\pgfqpoint{5.136838in}{2.078633in}}%
\pgfpathlineto{\pgfqpoint{5.137229in}{1.575042in}}%
\pgfpathlineto{\pgfqpoint{5.137408in}{2.098707in}}%
\pgfpathlineto{\pgfqpoint{5.137945in}{2.094603in}}%
\pgfpathlineto{\pgfqpoint{5.138498in}{1.694843in}}%
\pgfpathlineto{\pgfqpoint{5.139130in}{1.950743in}}%
\pgfpathlineto{\pgfqpoint{5.139956in}{2.129041in}}%
\pgfpathlineto{\pgfqpoint{5.139422in}{1.727609in}}%
\pgfpathlineto{\pgfqpoint{5.140199in}{1.995575in}}%
\pgfpathlineto{\pgfqpoint{5.140312in}{1.773881in}}%
\pgfpathlineto{\pgfqpoint{5.141217in}{2.145716in}}%
\pgfpathlineto{\pgfqpoint{5.141313in}{1.928032in}}%
\pgfpathlineto{\pgfqpoint{5.142055in}{2.162867in}}%
\pgfpathlineto{\pgfqpoint{5.141442in}{1.558720in}}%
\pgfpathlineto{\pgfqpoint{5.142328in}{1.992231in}}%
\pgfpathlineto{\pgfqpoint{5.142875in}{1.537101in}}%
\pgfpathlineto{\pgfqpoint{5.142779in}{2.094465in}}%
\pgfpathlineto{\pgfqpoint{5.143437in}{2.025200in}}%
\pgfpathlineto{\pgfqpoint{5.144175in}{2.131453in}}%
\pgfpathlineto{\pgfqpoint{5.143998in}{1.420502in}}%
\pgfpathlineto{\pgfqpoint{5.144543in}{2.009685in}}%
\pgfpathlineto{\pgfqpoint{5.144831in}{1.789467in}}%
\pgfpathlineto{\pgfqpoint{5.144895in}{2.116072in}}%
\pgfpathlineto{\pgfqpoint{5.145662in}{1.835402in}}%
\pgfpathlineto{\pgfqpoint{5.146077in}{2.139771in}}%
\pgfpathlineto{\pgfqpoint{5.145853in}{1.654197in}}%
\pgfpathlineto{\pgfqpoint{5.146794in}{1.979096in}}%
\pgfpathlineto{\pgfqpoint{5.146905in}{1.631742in}}%
\pgfpathlineto{\pgfqpoint{5.147128in}{2.149094in}}%
\pgfpathlineto{\pgfqpoint{5.147891in}{2.013325in}}%
\pgfpathlineto{\pgfqpoint{5.148843in}{2.142392in}}%
\pgfpathlineto{\pgfqpoint{5.148272in}{1.706226in}}%
\pgfpathlineto{\pgfqpoint{5.149001in}{2.014962in}}%
\pgfpathlineto{\pgfqpoint{5.149555in}{1.660189in}}%
\pgfpathlineto{\pgfqpoint{5.149539in}{2.114155in}}%
\pgfpathlineto{\pgfqpoint{5.150093in}{1.926603in}}%
\pgfpathlineto{\pgfqpoint{5.150582in}{2.132576in}}%
\pgfpathlineto{\pgfqpoint{5.150361in}{1.791754in}}%
\pgfpathlineto{\pgfqpoint{5.151181in}{1.850123in}}%
\pgfpathlineto{\pgfqpoint{5.152205in}{2.104678in}}%
\pgfpathlineto{\pgfqpoint{5.151811in}{1.740553in}}%
\pgfpathlineto{\pgfqpoint{5.152487in}{1.980582in}}%
\pgfpathlineto{\pgfqpoint{5.153507in}{1.724838in}}%
\pgfpathlineto{\pgfqpoint{5.152754in}{2.166102in}}%
\pgfpathlineto{\pgfqpoint{5.153601in}{1.878277in}}%
\pgfpathlineto{\pgfqpoint{5.154665in}{2.128820in}}%
\pgfpathlineto{\pgfqpoint{5.153883in}{1.594638in}}%
\pgfpathlineto{\pgfqpoint{5.154728in}{2.051924in}}%
\pgfpathlineto{\pgfqpoint{5.155493in}{1.735212in}}%
\pgfpathlineto{\pgfqpoint{5.154869in}{2.120936in}}%
\pgfpathlineto{\pgfqpoint{5.155820in}{2.084638in}}%
\pgfpathlineto{\pgfqpoint{5.155836in}{2.088362in}}%
\pgfpathlineto{\pgfqpoint{5.155914in}{1.922152in}}%
\pgfpathlineto{\pgfqpoint{5.156039in}{2.021125in}}%
\pgfpathlineto{\pgfqpoint{5.156552in}{1.641273in}}%
\pgfpathlineto{\pgfqpoint{5.156848in}{2.113725in}}%
\pgfpathlineto{\pgfqpoint{5.157143in}{2.071106in}}%
\pgfpathlineto{\pgfqpoint{5.157438in}{1.767731in}}%
\pgfpathlineto{\pgfqpoint{5.158090in}{2.140344in}}%
\pgfpathlineto{\pgfqpoint{5.158260in}{1.791249in}}%
\pgfpathlineto{\pgfqpoint{5.158353in}{2.135975in}}%
\pgfpathlineto{\pgfqpoint{5.158601in}{1.754762in}}%
\pgfpathlineto{\pgfqpoint{5.159375in}{2.068780in}}%
\pgfpathlineto{\pgfqpoint{5.160008in}{1.640167in}}%
\pgfpathlineto{\pgfqpoint{5.160224in}{2.120425in}}%
\pgfpathlineto{\pgfqpoint{5.160502in}{1.923424in}}%
\pgfpathlineto{\pgfqpoint{5.160871in}{2.119342in}}%
\pgfpathlineto{\pgfqpoint{5.161133in}{1.737321in}}%
\pgfpathlineto{\pgfqpoint{5.161626in}{2.036814in}}%
\pgfpathlineto{\pgfqpoint{5.161841in}{2.092561in}}%
\pgfpathlineto{\pgfqpoint{5.161672in}{1.775055in}}%
\pgfpathlineto{\pgfqpoint{5.162271in}{1.961196in}}%
\pgfpathlineto{\pgfqpoint{5.163359in}{1.667007in}}%
\pgfpathlineto{\pgfqpoint{5.163053in}{2.145505in}}%
\pgfpathlineto{\pgfqpoint{5.163375in}{1.993987in}}%
\pgfpathlineto{\pgfqpoint{5.163482in}{2.156438in}}%
\pgfpathlineto{\pgfqpoint{5.164170in}{1.742102in}}%
\pgfpathlineto{\pgfqpoint{5.164216in}{1.895865in}}%
\pgfpathlineto{\pgfqpoint{5.164231in}{1.661916in}}%
\pgfpathlineto{\pgfqpoint{5.164811in}{2.122027in}}%
\pgfpathlineto{\pgfqpoint{5.165315in}{1.888027in}}%
\pgfpathlineto{\pgfqpoint{5.166091in}{2.089109in}}%
\pgfpathlineto{\pgfqpoint{5.165757in}{1.744183in}}%
\pgfpathlineto{\pgfqpoint{5.166137in}{2.013613in}}%
\pgfpathlineto{\pgfqpoint{5.166532in}{1.714342in}}%
\pgfpathlineto{\pgfqpoint{5.166745in}{2.131851in}}%
\pgfpathlineto{\pgfqpoint{5.167246in}{1.997617in}}%
\pgfpathlineto{\pgfqpoint{5.167746in}{1.681176in}}%
\pgfpathlineto{\pgfqpoint{5.168201in}{2.107903in}}%
\pgfpathlineto{\pgfqpoint{5.168367in}{1.928535in}}%
\pgfpathlineto{\pgfqpoint{5.168428in}{2.120199in}}%
\pgfpathlineto{\pgfqpoint{5.168549in}{1.784505in}}%
\pgfpathlineto{\pgfqpoint{5.169471in}{1.914660in}}%
\pgfpathlineto{\pgfqpoint{5.170029in}{2.167812in}}%
\pgfpathlineto{\pgfqpoint{5.169682in}{1.678236in}}%
\pgfpathlineto{\pgfqpoint{5.170616in}{2.046959in}}%
\pgfpathlineto{\pgfqpoint{5.171233in}{1.707672in}}%
\pgfpathlineto{\pgfqpoint{5.171293in}{2.124810in}}%
\pgfpathlineto{\pgfqpoint{5.171729in}{1.902434in}}%
\pgfpathlineto{\pgfqpoint{5.172029in}{2.114865in}}%
\pgfpathlineto{\pgfqpoint{5.172808in}{1.701251in}}%
\pgfpathlineto{\pgfqpoint{5.172838in}{1.940356in}}%
\pgfpathlineto{\pgfqpoint{5.173212in}{2.127831in}}%
\pgfpathlineto{\pgfqpoint{5.173945in}{1.480374in}}%
\pgfpathlineto{\pgfqpoint{5.174258in}{2.159433in}}%
\pgfpathlineto{\pgfqpoint{5.175063in}{1.967786in}}%
\pgfpathlineto{\pgfqpoint{5.175926in}{2.154068in}}%
\pgfpathlineto{\pgfqpoint{5.175659in}{1.539957in}}%
\pgfpathlineto{\pgfqpoint{5.176135in}{1.872360in}}%
\pgfpathlineto{\pgfqpoint{5.176921in}{1.677835in}}%
\pgfpathlineto{\pgfqpoint{5.177158in}{2.150129in}}%
\pgfpathlineto{\pgfqpoint{5.177233in}{2.013203in}}%
\pgfpathlineto{\pgfqpoint{5.178254in}{2.155826in}}%
\pgfpathlineto{\pgfqpoint{5.178357in}{1.750548in}}%
\pgfpathlineto{\pgfqpoint{5.178402in}{2.161174in}}%
\pgfpathlineto{\pgfqpoint{5.178815in}{1.579219in}}%
\pgfpathlineto{\pgfqpoint{5.179465in}{1.764731in}}%
\pgfpathlineto{\pgfqpoint{5.179686in}{2.108054in}}%
\pgfpathlineto{\pgfqpoint{5.179907in}{1.663147in}}%
\pgfpathlineto{\pgfqpoint{5.180613in}{2.027707in}}%
\pgfpathlineto{\pgfqpoint{5.181567in}{1.684504in}}%
\pgfpathlineto{\pgfqpoint{5.180936in}{2.147159in}}%
\pgfpathlineto{\pgfqpoint{5.181729in}{1.906237in}}%
\pgfpathlineto{\pgfqpoint{5.181846in}{2.109710in}}%
\pgfpathlineto{\pgfqpoint{5.182432in}{1.736402in}}%
\pgfpathlineto{\pgfqpoint{5.182842in}{1.957121in}}%
\pgfpathlineto{\pgfqpoint{5.183777in}{1.767099in}}%
\pgfpathlineto{\pgfqpoint{5.183295in}{2.132148in}}%
\pgfpathlineto{\pgfqpoint{5.183879in}{1.930517in}}%
\pgfpathlineto{\pgfqpoint{5.184579in}{2.155360in}}%
\pgfpathlineto{\pgfqpoint{5.184418in}{1.629942in}}%
\pgfpathlineto{\pgfqpoint{5.184986in}{1.994415in}}%
\pgfpathlineto{\pgfqpoint{5.185713in}{2.157632in}}%
\pgfpathlineto{\pgfqpoint{5.185466in}{1.780019in}}%
\pgfpathlineto{\pgfqpoint{5.185844in}{2.039427in}}%
\pgfpathlineto{\pgfqpoint{5.186584in}{1.745678in}}%
\pgfpathlineto{\pgfqpoint{5.186932in}{2.137068in}}%
\pgfpathlineto{\pgfqpoint{5.186961in}{1.877280in}}%
\pgfpathlineto{\pgfqpoint{5.187858in}{2.146480in}}%
\pgfpathlineto{\pgfqpoint{5.187756in}{1.716666in}}%
\pgfpathlineto{\pgfqpoint{5.188060in}{1.906096in}}%
\pgfpathlineto{\pgfqpoint{5.188897in}{1.611313in}}%
\pgfpathlineto{\pgfqpoint{5.188493in}{2.135265in}}%
\pgfpathlineto{\pgfqpoint{5.189156in}{1.865352in}}%
\pgfpathlineto{\pgfqpoint{5.189718in}{2.130844in}}%
\pgfpathlineto{\pgfqpoint{5.190078in}{1.668428in}}%
\pgfpathlineto{\pgfqpoint{5.190279in}{1.993387in}}%
\pgfpathlineto{\pgfqpoint{5.191068in}{1.638470in}}%
\pgfpathlineto{\pgfqpoint{5.190624in}{2.141633in}}%
\pgfpathlineto{\pgfqpoint{5.191384in}{1.830218in}}%
\pgfpathlineto{\pgfqpoint{5.191570in}{2.154075in}}%
\pgfpathlineto{\pgfqpoint{5.192129in}{1.652226in}}%
\pgfpathlineto{\pgfqpoint{5.192501in}{2.059071in}}%
\pgfpathlineto{\pgfqpoint{5.193229in}{1.696187in}}%
\pgfpathlineto{\pgfqpoint{5.193414in}{2.152755in}}%
\pgfpathlineto{\pgfqpoint{5.193614in}{2.013605in}}%
\pgfpathlineto{\pgfqpoint{5.193714in}{1.700823in}}%
\pgfpathlineto{\pgfqpoint{5.194027in}{2.125122in}}%
\pgfpathlineto{\pgfqpoint{5.194711in}{2.044682in}}%
\pgfpathlineto{\pgfqpoint{5.194967in}{2.140280in}}%
\pgfpathlineto{\pgfqpoint{5.195535in}{1.710377in}}%
\pgfpathlineto{\pgfqpoint{5.195804in}{2.051125in}}%
\pgfpathlineto{\pgfqpoint{5.196159in}{1.747617in}}%
\pgfpathlineto{\pgfqpoint{5.196640in}{2.136006in}}%
\pgfpathlineto{\pgfqpoint{5.196923in}{2.015768in}}%
\pgfpathlineto{\pgfqpoint{5.197220in}{2.142226in}}%
\pgfpathlineto{\pgfqpoint{5.197150in}{1.741369in}}%
\pgfpathlineto{\pgfqpoint{5.198026in}{1.987221in}}%
\pgfpathlineto{\pgfqpoint{5.199054in}{2.128722in}}%
\pgfpathlineto{\pgfqpoint{5.198491in}{1.666698in}}%
\pgfpathlineto{\pgfqpoint{5.199139in}{2.087690in}}%
\pgfpathlineto{\pgfqpoint{5.199491in}{1.674251in}}%
\pgfpathlineto{\pgfqpoint{5.199575in}{2.115961in}}%
\pgfpathlineto{\pgfqpoint{5.200235in}{2.030848in}}%
\pgfpathlineto{\pgfqpoint{5.200249in}{2.158064in}}%
\pgfpathlineto{\pgfqpoint{5.200838in}{1.719420in}}%
\pgfpathlineto{\pgfqpoint{5.201329in}{1.955934in}}%
\pgfpathlineto{\pgfqpoint{5.201930in}{1.709184in}}%
\pgfpathlineto{\pgfqpoint{5.201455in}{2.125432in}}%
\pgfpathlineto{\pgfqpoint{5.202433in}{1.919940in}}%
\pgfpathlineto{\pgfqpoint{5.203452in}{2.130742in}}%
\pgfpathlineto{\pgfqpoint{5.202741in}{1.672906in}}%
\pgfpathlineto{\pgfqpoint{5.203535in}{2.114239in}}%
\pgfpathlineto{\pgfqpoint{5.203772in}{1.605471in}}%
\pgfpathlineto{\pgfqpoint{5.204176in}{2.154479in}}%
\pgfpathlineto{\pgfqpoint{5.204648in}{2.017948in}}%
\pgfpathlineto{\pgfqpoint{5.205079in}{2.158516in}}%
\pgfpathlineto{\pgfqpoint{5.205509in}{1.710829in}}%
\pgfpathlineto{\pgfqpoint{5.205730in}{2.000431in}}%
\pgfpathlineto{\pgfqpoint{5.206229in}{1.838864in}}%
\pgfpathlineto{\pgfqpoint{5.206768in}{2.145217in}}%
\pgfpathlineto{\pgfqpoint{5.206824in}{2.088391in}}%
\pgfpathlineto{\pgfqpoint{5.206989in}{1.749267in}}%
\pgfpathlineto{\pgfqpoint{5.207487in}{2.129879in}}%
\pgfpathlineto{\pgfqpoint{5.207956in}{1.941889in}}%
\pgfpathlineto{\pgfqpoint{5.208369in}{2.123469in}}%
\pgfpathlineto{\pgfqpoint{5.208548in}{1.571201in}}%
\pgfpathlineto{\pgfqpoint{5.209057in}{2.063812in}}%
\pgfpathlineto{\pgfqpoint{5.209098in}{1.465303in}}%
\pgfpathlineto{\pgfqpoint{5.209826in}{2.195364in}}%
\pgfpathlineto{\pgfqpoint{5.210169in}{1.979645in}}%
\pgfpathlineto{\pgfqpoint{5.211087in}{2.134622in}}%
\pgfpathlineto{\pgfqpoint{5.211183in}{1.640714in}}%
\pgfpathlineto{\pgfqpoint{5.211265in}{2.066164in}}%
\pgfpathlineto{\pgfqpoint{5.212057in}{1.671966in}}%
\pgfpathlineto{\pgfqpoint{5.212044in}{2.139809in}}%
\pgfpathlineto{\pgfqpoint{5.212385in}{1.887383in}}%
\pgfpathlineto{\pgfqpoint{5.212426in}{2.132513in}}%
\pgfpathlineto{\pgfqpoint{5.213094in}{1.755650in}}%
\pgfpathlineto{\pgfqpoint{5.213488in}{2.055557in}}%
\pgfpathlineto{\pgfqpoint{5.214168in}{1.586675in}}%
\pgfpathlineto{\pgfqpoint{5.214141in}{2.153753in}}%
\pgfpathlineto{\pgfqpoint{5.214589in}{2.067057in}}%
\pgfpathlineto{\pgfqpoint{5.215159in}{2.133723in}}%
\pgfpathlineto{\pgfqpoint{5.215186in}{1.785246in}}%
\pgfpathlineto{\pgfqpoint{5.215646in}{2.004279in}}%
\pgfpathlineto{\pgfqpoint{5.215998in}{1.715933in}}%
\pgfpathlineto{\pgfqpoint{5.216431in}{2.171088in}}%
\pgfpathlineto{\pgfqpoint{5.216755in}{1.953142in}}%
\pgfpathlineto{\pgfqpoint{5.217308in}{2.151514in}}%
\pgfpathlineto{\pgfqpoint{5.217402in}{1.718429in}}%
\pgfpathlineto{\pgfqpoint{5.217861in}{2.080603in}}%
\pgfpathlineto{\pgfqpoint{5.218560in}{1.753333in}}%
\pgfpathlineto{\pgfqpoint{5.218533in}{2.152172in}}%
\pgfpathlineto{\pgfqpoint{5.219004in}{1.933033in}}%
\pgfpathlineto{\pgfqpoint{5.219541in}{2.155789in}}%
\pgfpathlineto{\pgfqpoint{5.219527in}{1.635565in}}%
\pgfpathlineto{\pgfqpoint{5.220117in}{1.964789in}}%
\pgfpathlineto{\pgfqpoint{5.220719in}{2.165206in}}%
\pgfpathlineto{\pgfqpoint{5.220184in}{1.752312in}}%
\pgfpathlineto{\pgfqpoint{5.220773in}{2.042253in}}%
\pgfpathlineto{\pgfqpoint{5.221548in}{1.611082in}}%
\pgfpathlineto{\pgfqpoint{5.221414in}{2.156928in}}%
\pgfpathlineto{\pgfqpoint{5.221881in}{1.962892in}}%
\pgfpathlineto{\pgfqpoint{5.222575in}{1.758611in}}%
\pgfpathlineto{\pgfqpoint{5.222215in}{2.166249in}}%
\pgfpathlineto{\pgfqpoint{5.222961in}{2.001849in}}%
\pgfpathlineto{\pgfqpoint{5.223067in}{2.145775in}}%
\pgfpathlineto{\pgfqpoint{5.223333in}{1.860863in}}%
\pgfpathlineto{\pgfqpoint{5.224050in}{1.990304in}}%
\pgfpathlineto{\pgfqpoint{5.224581in}{1.729654in}}%
\pgfpathlineto{\pgfqpoint{5.224766in}{2.153892in}}%
\pgfpathlineto{\pgfqpoint{5.225111in}{1.976374in}}%
\pgfpathlineto{\pgfqpoint{5.225746in}{2.178562in}}%
\pgfpathlineto{\pgfqpoint{5.225613in}{1.686214in}}%
\pgfpathlineto{\pgfqpoint{5.226221in}{2.032638in}}%
\pgfpathlineto{\pgfqpoint{5.226986in}{1.428590in}}%
\pgfpathlineto{\pgfqpoint{5.227171in}{2.172689in}}%
\pgfpathlineto{\pgfqpoint{5.227342in}{1.887549in}}%
\pgfpathlineto{\pgfqpoint{5.227684in}{2.173148in}}%
\pgfpathlineto{\pgfqpoint{5.227619in}{1.756140in}}%
\pgfpathlineto{\pgfqpoint{5.228473in}{2.087572in}}%
\pgfpathlineto{\pgfqpoint{5.229024in}{1.732232in}}%
\pgfpathlineto{\pgfqpoint{5.228933in}{2.150107in}}%
\pgfpathlineto{\pgfqpoint{5.229601in}{1.989525in}}%
\pgfpathlineto{\pgfqpoint{5.230033in}{2.169710in}}%
\pgfpathlineto{\pgfqpoint{5.229902in}{1.697485in}}%
\pgfpathlineto{\pgfqpoint{5.230713in}{2.036265in}}%
\pgfpathlineto{\pgfqpoint{5.231731in}{1.748762in}}%
\pgfpathlineto{\pgfqpoint{5.231653in}{2.154542in}}%
\pgfpathlineto{\pgfqpoint{5.231757in}{1.940833in}}%
\pgfpathlineto{\pgfqpoint{5.232304in}{2.188629in}}%
\pgfpathlineto{\pgfqpoint{5.232811in}{1.588485in}}%
\pgfpathlineto{\pgfqpoint{5.232863in}{2.076763in}}%
\pgfpathlineto{\pgfqpoint{5.232876in}{1.610209in}}%
\pgfpathlineto{\pgfqpoint{5.233552in}{2.131367in}}%
\pgfpathlineto{\pgfqpoint{5.233980in}{1.813416in}}%
\pgfpathlineto{\pgfqpoint{5.234641in}{2.158188in}}%
\pgfpathlineto{\pgfqpoint{5.235029in}{1.634278in}}%
\pgfpathlineto{\pgfqpoint{5.235119in}{2.078374in}}%
\pgfpathlineto{\pgfqpoint{5.235714in}{1.692266in}}%
\pgfpathlineto{\pgfqpoint{5.235985in}{2.188968in}}%
\pgfpathlineto{\pgfqpoint{5.236230in}{1.882703in}}%
\pgfpathlineto{\pgfqpoint{5.236565in}{2.143366in}}%
\pgfpathlineto{\pgfqpoint{5.236951in}{1.675782in}}%
\pgfpathlineto{\pgfqpoint{5.237337in}{2.081446in}}%
\pgfpathlineto{\pgfqpoint{5.237659in}{1.599640in}}%
\pgfpathlineto{\pgfqpoint{5.237505in}{2.141392in}}%
\pgfpathlineto{\pgfqpoint{5.238442in}{1.878373in}}%
\pgfpathlineto{\pgfqpoint{5.239326in}{2.157481in}}%
\pgfpathlineto{\pgfqpoint{5.239288in}{1.724850in}}%
\pgfpathlineto{\pgfqpoint{5.239557in}{2.123378in}}%
\pgfpathlineto{\pgfqpoint{5.239697in}{1.748684in}}%
\pgfpathlineto{\pgfqpoint{5.240311in}{2.183708in}}%
\pgfpathlineto{\pgfqpoint{5.240669in}{2.046424in}}%
\pgfpathlineto{\pgfqpoint{5.241153in}{1.680395in}}%
\pgfpathlineto{\pgfqpoint{5.240949in}{2.133949in}}%
\pgfpathlineto{\pgfqpoint{5.241790in}{1.971126in}}%
\pgfpathlineto{\pgfqpoint{5.241803in}{2.182087in}}%
\pgfpathlineto{\pgfqpoint{5.242782in}{1.584279in}}%
\pgfpathlineto{\pgfqpoint{5.242896in}{2.117958in}}%
\pgfpathlineto{\pgfqpoint{5.243568in}{1.776454in}}%
\pgfpathlineto{\pgfqpoint{5.243936in}{2.131459in}}%
\pgfpathlineto{\pgfqpoint{5.244012in}{2.015381in}}%
\pgfpathlineto{\pgfqpoint{5.244771in}{2.170439in}}%
\pgfpathlineto{\pgfqpoint{5.244113in}{1.562565in}}%
\pgfpathlineto{\pgfqpoint{5.245112in}{2.133710in}}%
\pgfpathlineto{\pgfqpoint{5.245705in}{1.630244in}}%
\pgfpathlineto{\pgfqpoint{5.245957in}{2.145132in}}%
\pgfpathlineto{\pgfqpoint{5.246234in}{1.876942in}}%
\pgfpathlineto{\pgfqpoint{5.247065in}{2.160472in}}%
\pgfpathlineto{\pgfqpoint{5.246801in}{1.734919in}}%
\pgfpathlineto{\pgfqpoint{5.247354in}{2.101831in}}%
\pgfpathlineto{\pgfqpoint{5.248245in}{1.764851in}}%
\pgfpathlineto{\pgfqpoint{5.247605in}{2.193379in}}%
\pgfpathlineto{\pgfqpoint{5.248471in}{2.052227in}}%
\pgfpathlineto{\pgfqpoint{5.249272in}{1.639824in}}%
\pgfpathlineto{\pgfqpoint{5.248821in}{2.153999in}}%
\pgfpathlineto{\pgfqpoint{5.249584in}{1.985519in}}%
\pgfpathlineto{\pgfqpoint{5.250408in}{2.169008in}}%
\pgfpathlineto{\pgfqpoint{5.250021in}{1.685835in}}%
\pgfpathlineto{\pgfqpoint{5.250682in}{1.969769in}}%
\pgfpathlineto{\pgfqpoint{5.251554in}{1.753207in}}%
\pgfpathlineto{\pgfqpoint{5.251579in}{2.165062in}}%
\pgfpathlineto{\pgfqpoint{5.251778in}{1.916394in}}%
\pgfpathlineto{\pgfqpoint{5.252300in}{2.207483in}}%
\pgfpathlineto{\pgfqpoint{5.252809in}{1.699528in}}%
\pgfpathlineto{\pgfqpoint{5.252895in}{2.009744in}}%
\pgfpathlineto{\pgfqpoint{5.253032in}{2.106096in}}%
\pgfpathlineto{\pgfqpoint{5.252970in}{1.872570in}}%
\pgfpathlineto{\pgfqpoint{5.253044in}{2.049456in}}%
\pgfpathlineto{\pgfqpoint{5.253701in}{1.370950in}}%
\pgfpathlineto{\pgfqpoint{5.253218in}{2.190293in}}%
\pgfpathlineto{\pgfqpoint{5.254146in}{2.067829in}}%
\pgfpathlineto{\pgfqpoint{5.254739in}{2.150500in}}%
\pgfpathlineto{\pgfqpoint{5.254578in}{1.777658in}}%
\pgfpathlineto{\pgfqpoint{5.255035in}{1.943729in}}%
\pgfpathlineto{\pgfqpoint{5.256095in}{1.825657in}}%
\pgfpathlineto{\pgfqpoint{5.255454in}{2.147982in}}%
\pgfpathlineto{\pgfqpoint{5.256132in}{1.952904in}}%
\pgfpathlineto{\pgfqpoint{5.257140in}{2.181757in}}%
\pgfpathlineto{\pgfqpoint{5.257029in}{1.804117in}}%
\pgfpathlineto{\pgfqpoint{5.257250in}{2.111709in}}%
\pgfpathlineto{\pgfqpoint{5.258157in}{1.627351in}}%
\pgfpathlineto{\pgfqpoint{5.257336in}{2.184064in}}%
\pgfpathlineto{\pgfqpoint{5.258353in}{1.970440in}}%
\pgfpathlineto{\pgfqpoint{5.258365in}{2.163691in}}%
\pgfpathlineto{\pgfqpoint{5.259038in}{1.721309in}}%
\pgfpathlineto{\pgfqpoint{5.259453in}{1.998444in}}%
\pgfpathlineto{\pgfqpoint{5.259917in}{1.703391in}}%
\pgfpathlineto{\pgfqpoint{5.260173in}{2.162363in}}%
\pgfpathlineto{\pgfqpoint{5.260514in}{1.994598in}}%
\pgfpathlineto{\pgfqpoint{5.261439in}{2.146555in}}%
\pgfpathlineto{\pgfqpoint{5.260940in}{1.692240in}}%
\pgfpathlineto{\pgfqpoint{5.261609in}{1.987521in}}%
\pgfpathlineto{\pgfqpoint{5.262422in}{1.773982in}}%
\pgfpathlineto{\pgfqpoint{5.262446in}{2.149562in}}%
\pgfpathlineto{\pgfqpoint{5.262713in}{2.035359in}}%
\pgfpathlineto{\pgfqpoint{5.263173in}{2.161343in}}%
\pgfpathlineto{\pgfqpoint{5.263197in}{1.718966in}}%
\pgfpathlineto{\pgfqpoint{5.263306in}{2.070422in}}%
\pgfpathlineto{\pgfqpoint{5.263318in}{1.586659in}}%
\pgfpathlineto{\pgfqpoint{5.263802in}{2.221882in}}%
\pgfpathlineto{\pgfqpoint{5.264418in}{2.044753in}}%
\pgfpathlineto{\pgfqpoint{5.264719in}{1.747530in}}%
\pgfpathlineto{\pgfqpoint{5.264671in}{2.165462in}}%
\pgfpathlineto{\pgfqpoint{5.265563in}{1.978416in}}%
\pgfpathlineto{\pgfqpoint{5.265851in}{2.177037in}}%
\pgfpathlineto{\pgfqpoint{5.265587in}{1.673940in}}%
\pgfpathlineto{\pgfqpoint{5.266657in}{1.975659in}}%
\pgfpathlineto{\pgfqpoint{5.266957in}{1.556841in}}%
\pgfpathlineto{\pgfqpoint{5.267268in}{2.188685in}}%
\pgfpathlineto{\pgfqpoint{5.267747in}{2.015018in}}%
\pgfpathlineto{\pgfqpoint{5.268238in}{2.159064in}}%
\pgfpathlineto{\pgfqpoint{5.268453in}{1.789462in}}%
\pgfpathlineto{\pgfqpoint{5.268824in}{2.030419in}}%
\pgfpathlineto{\pgfqpoint{5.269361in}{1.694904in}}%
\pgfpathlineto{\pgfqpoint{5.269492in}{2.167017in}}%
\pgfpathlineto{\pgfqpoint{5.269933in}{2.038687in}}%
\pgfpathlineto{\pgfqpoint{5.270659in}{2.157473in}}%
\pgfpathlineto{\pgfqpoint{5.270290in}{1.748558in}}%
\pgfpathlineto{\pgfqpoint{5.270920in}{1.948535in}}%
\pgfpathlineto{\pgfqpoint{5.271740in}{1.643522in}}%
\pgfpathlineto{\pgfqpoint{5.271692in}{2.156753in}}%
\pgfpathlineto{\pgfqpoint{5.272024in}{2.103333in}}%
\pgfpathlineto{\pgfqpoint{5.272178in}{1.771032in}}%
\pgfpathlineto{\pgfqpoint{5.272439in}{2.166131in}}%
\pgfpathlineto{\pgfqpoint{5.273161in}{2.011410in}}%
\pgfpathlineto{\pgfqpoint{5.273822in}{2.162521in}}%
\pgfpathlineto{\pgfqpoint{5.273916in}{1.740904in}}%
\pgfpathlineto{\pgfqpoint{5.274235in}{2.131606in}}%
\pgfpathlineto{\pgfqpoint{5.274270in}{1.775214in}}%
\pgfpathlineto{\pgfqpoint{5.275307in}{2.167671in}}%
\pgfpathlineto{\pgfqpoint{5.275342in}{2.061710in}}%
\pgfpathlineto{\pgfqpoint{5.276200in}{1.666654in}}%
\pgfpathlineto{\pgfqpoint{5.276282in}{2.209675in}}%
\pgfpathlineto{\pgfqpoint{5.276481in}{1.673809in}}%
\pgfpathlineto{\pgfqpoint{5.276927in}{2.173460in}}%
\pgfpathlineto{\pgfqpoint{5.277430in}{1.655813in}}%
\pgfpathlineto{\pgfqpoint{5.277606in}{2.128925in}}%
\pgfpathlineto{\pgfqpoint{5.278365in}{1.781541in}}%
\pgfpathlineto{\pgfqpoint{5.278505in}{2.207873in}}%
\pgfpathlineto{\pgfqpoint{5.278727in}{1.945016in}}%
\pgfpathlineto{\pgfqpoint{5.279543in}{2.155229in}}%
\pgfpathlineto{\pgfqpoint{5.279590in}{1.716372in}}%
\pgfpathlineto{\pgfqpoint{5.279834in}{1.963278in}}%
\pgfpathlineto{\pgfqpoint{5.280032in}{1.547584in}}%
\pgfpathlineto{\pgfqpoint{5.280020in}{2.193817in}}%
\pgfpathlineto{\pgfqpoint{5.280601in}{1.960529in}}%
\pgfpathlineto{\pgfqpoint{5.281135in}{2.208654in}}%
\pgfpathlineto{\pgfqpoint{5.281147in}{1.639833in}}%
\pgfpathlineto{\pgfqpoint{5.281704in}{2.015580in}}%
\pgfpathlineto{\pgfqpoint{5.282144in}{1.668164in}}%
\pgfpathlineto{\pgfqpoint{5.282132in}{2.169921in}}%
\pgfpathlineto{\pgfqpoint{5.282791in}{2.120086in}}%
\pgfpathlineto{\pgfqpoint{5.283472in}{2.169878in}}%
\pgfpathlineto{\pgfqpoint{5.283253in}{1.727941in}}%
\pgfpathlineto{\pgfqpoint{5.283830in}{2.037372in}}%
\pgfpathlineto{\pgfqpoint{5.284095in}{1.720268in}}%
\pgfpathlineto{\pgfqpoint{5.284924in}{2.197177in}}%
\pgfpathlineto{\pgfqpoint{5.284935in}{2.018161in}}%
\pgfpathlineto{\pgfqpoint{5.285429in}{2.140286in}}%
\pgfpathlineto{\pgfqpoint{5.285521in}{1.766402in}}%
\pgfpathlineto{\pgfqpoint{5.286049in}{2.105137in}}%
\pgfpathlineto{\pgfqpoint{5.286553in}{1.537167in}}%
\pgfpathlineto{\pgfqpoint{5.286336in}{2.161902in}}%
\pgfpathlineto{\pgfqpoint{5.287149in}{2.046795in}}%
\pgfpathlineto{\pgfqpoint{5.287514in}{2.190338in}}%
\pgfpathlineto{\pgfqpoint{5.288131in}{1.828551in}}%
\pgfpathlineto{\pgfqpoint{5.288234in}{1.959452in}}%
\pgfpathlineto{\pgfqpoint{5.288895in}{1.667589in}}%
\pgfpathlineto{\pgfqpoint{5.288678in}{2.211364in}}%
\pgfpathlineto{\pgfqpoint{5.289157in}{2.110116in}}%
\pgfpathlineto{\pgfqpoint{5.289953in}{2.182008in}}%
\pgfpathlineto{\pgfqpoint{5.289998in}{1.848910in}}%
\pgfpathlineto{\pgfqpoint{5.290248in}{2.123492in}}%
\pgfpathlineto{\pgfqpoint{5.291201in}{1.765345in}}%
\pgfpathlineto{\pgfqpoint{5.291156in}{2.205692in}}%
\pgfpathlineto{\pgfqpoint{5.291360in}{2.104637in}}%
\pgfpathlineto{\pgfqpoint{5.292027in}{1.748806in}}%
\pgfpathlineto{\pgfqpoint{5.292265in}{2.186124in}}%
\pgfpathlineto{\pgfqpoint{5.292468in}{1.960555in}}%
\pgfpathlineto{\pgfqpoint{5.292897in}{2.207555in}}%
\pgfpathlineto{\pgfqpoint{5.292807in}{1.194153in}}%
\pgfpathlineto{\pgfqpoint{5.293585in}{2.039255in}}%
\pgfpathlineto{\pgfqpoint{5.293945in}{1.714353in}}%
\pgfpathlineto{\pgfqpoint{5.294553in}{2.170810in}}%
\pgfpathlineto{\pgfqpoint{5.294687in}{2.032527in}}%
\pgfpathlineto{\pgfqpoint{5.295305in}{2.210393in}}%
\pgfpathlineto{\pgfqpoint{5.295249in}{1.737142in}}%
\pgfpathlineto{\pgfqpoint{5.295765in}{2.138330in}}%
\pgfpathlineto{\pgfqpoint{5.296135in}{1.746225in}}%
\pgfpathlineto{\pgfqpoint{5.296493in}{2.166858in}}%
\pgfpathlineto{\pgfqpoint{5.296884in}{1.987604in}}%
\pgfpathlineto{\pgfqpoint{5.297900in}{2.163660in}}%
\pgfpathlineto{\pgfqpoint{5.297219in}{1.644609in}}%
\pgfpathlineto{\pgfqpoint{5.297989in}{2.011497in}}%
\pgfpathlineto{\pgfqpoint{5.298524in}{1.511012in}}%
\pgfpathlineto{\pgfqpoint{5.298513in}{2.182002in}}%
\pgfpathlineto{\pgfqpoint{5.299070in}{1.839939in}}%
\pgfpathlineto{\pgfqpoint{5.299570in}{2.181869in}}%
\pgfpathlineto{\pgfqpoint{5.299836in}{1.732279in}}%
\pgfpathlineto{\pgfqpoint{5.300180in}{2.126352in}}%
\pgfpathlineto{\pgfqpoint{5.301122in}{1.628064in}}%
\pgfpathlineto{\pgfqpoint{5.300989in}{2.167891in}}%
\pgfpathlineto{\pgfqpoint{5.301288in}{1.959031in}}%
\pgfpathlineto{\pgfqpoint{5.302183in}{2.186821in}}%
\pgfpathlineto{\pgfqpoint{5.302205in}{1.756135in}}%
\pgfpathlineto{\pgfqpoint{5.302382in}{2.042973in}}%
\pgfpathlineto{\pgfqpoint{5.303065in}{1.659674in}}%
\pgfpathlineto{\pgfqpoint{5.302536in}{2.198525in}}%
\pgfpathlineto{\pgfqpoint{5.303495in}{1.838647in}}%
\pgfpathlineto{\pgfqpoint{5.303627in}{2.188265in}}%
\pgfpathlineto{\pgfqpoint{5.303924in}{1.637250in}}%
\pgfpathlineto{\pgfqpoint{5.304605in}{1.897332in}}%
\pgfpathlineto{\pgfqpoint{5.305449in}{2.173216in}}%
\pgfpathlineto{\pgfqpoint{5.305164in}{1.759252in}}%
\pgfpathlineto{\pgfqpoint{5.305723in}{2.043222in}}%
\pgfpathlineto{\pgfqpoint{5.305865in}{1.654880in}}%
\pgfpathlineto{\pgfqpoint{5.305975in}{2.207285in}}%
\pgfpathlineto{\pgfqpoint{5.306828in}{2.064369in}}%
\pgfpathlineto{\pgfqpoint{5.307875in}{2.198773in}}%
\pgfpathlineto{\pgfqpoint{5.307384in}{1.702777in}}%
\pgfpathlineto{\pgfqpoint{5.307918in}{2.088360in}}%
\pgfpathlineto{\pgfqpoint{5.308299in}{1.767821in}}%
\pgfpathlineto{\pgfqpoint{5.308245in}{2.227607in}}%
\pgfpathlineto{\pgfqpoint{5.309028in}{1.929654in}}%
\pgfpathlineto{\pgfqpoint{5.309657in}{2.209741in}}%
\pgfpathlineto{\pgfqpoint{5.309570in}{1.688366in}}%
\pgfpathlineto{\pgfqpoint{5.310123in}{2.089808in}}%
\pgfpathlineto{\pgfqpoint{5.310134in}{1.680767in}}%
\pgfpathlineto{\pgfqpoint{5.310275in}{2.183874in}}%
\pgfpathlineto{\pgfqpoint{5.311227in}{1.926493in}}%
\pgfpathlineto{\pgfqpoint{5.312285in}{2.186579in}}%
\pgfpathlineto{\pgfqpoint{5.311638in}{1.633966in}}%
\pgfpathlineto{\pgfqpoint{5.312317in}{1.970856in}}%
\pgfpathlineto{\pgfqpoint{5.313254in}{1.407494in}}%
\pgfpathlineto{\pgfqpoint{5.313189in}{2.185425in}}%
\pgfpathlineto{\pgfqpoint{5.313426in}{1.953595in}}%
\pgfpathlineto{\pgfqpoint{5.313770in}{2.176612in}}%
\pgfpathlineto{\pgfqpoint{5.313931in}{1.665480in}}%
\pgfpathlineto{\pgfqpoint{5.314543in}{2.125857in}}%
\pgfpathlineto{\pgfqpoint{5.314982in}{1.752518in}}%
\pgfpathlineto{\pgfqpoint{5.315250in}{2.188756in}}%
\pgfpathlineto{\pgfqpoint{5.315667in}{1.920134in}}%
\pgfpathlineto{\pgfqpoint{5.316105in}{2.156461in}}%
\pgfpathlineto{\pgfqpoint{5.316020in}{1.765790in}}%
\pgfpathlineto{\pgfqpoint{5.316778in}{2.015879in}}%
\pgfpathlineto{\pgfqpoint{5.316991in}{2.165997in}}%
\pgfpathlineto{\pgfqpoint{5.317481in}{1.852458in}}%
\pgfpathlineto{\pgfqpoint{5.318173in}{2.219796in}}%
\pgfpathlineto{\pgfqpoint{5.317684in}{1.671019in}}%
\pgfpathlineto{\pgfqpoint{5.318587in}{2.026967in}}%
\pgfpathlineto{\pgfqpoint{5.319595in}{1.528791in}}%
\pgfpathlineto{\pgfqpoint{5.318725in}{2.225964in}}%
\pgfpathlineto{\pgfqpoint{5.319691in}{2.076096in}}%
\pgfpathlineto{\pgfqpoint{5.319818in}{1.515994in}}%
\pgfpathlineto{\pgfqpoint{5.320283in}{2.203212in}}%
\pgfpathlineto{\pgfqpoint{5.320759in}{1.997758in}}%
\pgfpathlineto{\pgfqpoint{5.320770in}{2.188689in}}%
\pgfpathlineto{\pgfqpoint{5.321256in}{1.810407in}}%
\pgfpathlineto{\pgfqpoint{5.321867in}{2.116957in}}%
\pgfpathlineto{\pgfqpoint{5.322689in}{1.563010in}}%
\pgfpathlineto{\pgfqpoint{5.322036in}{2.191467in}}%
\pgfpathlineto{\pgfqpoint{5.322994in}{1.888208in}}%
\pgfpathlineto{\pgfqpoint{5.324075in}{2.196601in}}%
\pgfpathlineto{\pgfqpoint{5.323078in}{1.731738in}}%
\pgfpathlineto{\pgfqpoint{5.324106in}{2.036219in}}%
\pgfpathlineto{\pgfqpoint{5.324557in}{1.717508in}}%
\pgfpathlineto{\pgfqpoint{5.324672in}{2.170484in}}%
\pgfpathlineto{\pgfqpoint{5.325185in}{2.120431in}}%
\pgfpathlineto{\pgfqpoint{5.325352in}{2.229610in}}%
\pgfpathlineto{\pgfqpoint{5.325321in}{1.747046in}}%
\pgfpathlineto{\pgfqpoint{5.326041in}{2.104341in}}%
\pgfpathlineto{\pgfqpoint{5.327094in}{1.676953in}}%
\pgfpathlineto{\pgfqpoint{5.326229in}{2.186729in}}%
\pgfpathlineto{\pgfqpoint{5.327146in}{2.081536in}}%
\pgfpathlineto{\pgfqpoint{5.327677in}{1.843387in}}%
\pgfpathlineto{\pgfqpoint{5.327375in}{2.189943in}}%
\pgfpathlineto{\pgfqpoint{5.327843in}{2.051291in}}%
\pgfpathlineto{\pgfqpoint{5.328404in}{2.232509in}}%
\pgfpathlineto{\pgfqpoint{5.328902in}{1.589127in}}%
\pgfpathlineto{\pgfqpoint{5.328943in}{2.047991in}}%
\pgfpathlineto{\pgfqpoint{5.329213in}{1.678642in}}%
\pgfpathlineto{\pgfqpoint{5.329140in}{2.180991in}}%
\pgfpathlineto{\pgfqpoint{5.330040in}{2.120968in}}%
\pgfpathlineto{\pgfqpoint{5.330991in}{1.739703in}}%
\pgfpathlineto{\pgfqpoint{5.330113in}{2.166436in}}%
\pgfpathlineto{\pgfqpoint{5.331156in}{2.040866in}}%
\pgfpathlineto{\pgfqpoint{5.332175in}{2.173390in}}%
\pgfpathlineto{\pgfqpoint{5.331269in}{1.713800in}}%
\pgfpathlineto{\pgfqpoint{5.332227in}{2.014533in}}%
\pgfpathlineto{\pgfqpoint{5.332237in}{1.661918in}}%
\pgfpathlineto{\pgfqpoint{5.333038in}{2.197872in}}%
\pgfpathlineto{\pgfqpoint{5.333336in}{1.975419in}}%
\pgfpathlineto{\pgfqpoint{5.334228in}{2.190960in}}%
\pgfpathlineto{\pgfqpoint{5.334330in}{1.753549in}}%
\pgfpathlineto{\pgfqpoint{5.334422in}{2.012649in}}%
\pgfpathlineto{\pgfqpoint{5.335005in}{1.642858in}}%
\pgfpathlineto{\pgfqpoint{5.334862in}{2.181631in}}%
\pgfpathlineto{\pgfqpoint{5.335526in}{2.114618in}}%
\pgfpathlineto{\pgfqpoint{5.336209in}{2.205881in}}%
\pgfpathlineto{\pgfqpoint{5.335740in}{1.788887in}}%
\pgfpathlineto{\pgfqpoint{5.336555in}{2.169590in}}%
\pgfpathlineto{\pgfqpoint{5.336718in}{1.701010in}}%
\pgfpathlineto{\pgfqpoint{5.337369in}{2.189320in}}%
\pgfpathlineto{\pgfqpoint{5.337663in}{2.050972in}}%
\pgfpathlineto{\pgfqpoint{5.338678in}{2.226892in}}%
\pgfpathlineto{\pgfqpoint{5.337897in}{1.659286in}}%
\pgfpathlineto{\pgfqpoint{5.338769in}{2.047023in}}%
\pgfpathlineto{\pgfqpoint{5.339689in}{1.762874in}}%
\pgfpathlineto{\pgfqpoint{5.338880in}{2.226645in}}%
\pgfpathlineto{\pgfqpoint{5.339861in}{2.038338in}}%
\pgfpathlineto{\pgfqpoint{5.340346in}{2.215292in}}%
\pgfpathlineto{\pgfqpoint{5.340628in}{1.774430in}}%
\pgfpathlineto{\pgfqpoint{5.340961in}{2.112940in}}%
\pgfpathlineto{\pgfqpoint{5.341766in}{1.686744in}}%
\pgfpathlineto{\pgfqpoint{5.341625in}{2.208032in}}%
\pgfpathlineto{\pgfqpoint{5.342068in}{1.948152in}}%
\pgfpathlineto{\pgfqpoint{5.342419in}{2.223557in}}%
\pgfpathlineto{\pgfqpoint{5.342781in}{1.713852in}}%
\pgfpathlineto{\pgfqpoint{5.343172in}{1.976065in}}%
\pgfpathlineto{\pgfqpoint{5.343402in}{1.702881in}}%
\pgfpathlineto{\pgfqpoint{5.343372in}{2.210158in}}%
\pgfpathlineto{\pgfqpoint{5.344193in}{1.903223in}}%
\pgfpathlineto{\pgfqpoint{5.344203in}{2.222714in}}%
\pgfpathlineto{\pgfqpoint{5.345102in}{1.632287in}}%
\pgfpathlineto{\pgfqpoint{5.345301in}{2.036259in}}%
\pgfpathlineto{\pgfqpoint{5.345461in}{2.182249in}}%
\pgfpathlineto{\pgfqpoint{5.345381in}{1.819062in}}%
\pgfpathlineto{\pgfqpoint{5.346357in}{2.097202in}}%
\pgfpathlineto{\pgfqpoint{5.346536in}{1.494574in}}%
\pgfpathlineto{\pgfqpoint{5.346845in}{2.187943in}}%
\pgfpathlineto{\pgfqpoint{5.347460in}{1.947009in}}%
\pgfpathlineto{\pgfqpoint{5.348491in}{1.839464in}}%
\pgfpathlineto{\pgfqpoint{5.348580in}{2.181977in}}%
\pgfpathlineto{\pgfqpoint{5.348738in}{1.727656in}}%
\pgfpathlineto{\pgfqpoint{5.348946in}{2.204409in}}%
\pgfpathlineto{\pgfqpoint{5.349687in}{2.078586in}}%
\pgfpathlineto{\pgfqpoint{5.350239in}{1.737047in}}%
\pgfpathlineto{\pgfqpoint{5.350259in}{2.174394in}}%
\pgfpathlineto{\pgfqpoint{5.350761in}{2.070882in}}%
\pgfpathlineto{\pgfqpoint{5.350889in}{2.206297in}}%
\pgfpathlineto{\pgfqpoint{5.351450in}{1.685607in}}%
\pgfpathlineto{\pgfqpoint{5.351873in}{2.088140in}}%
\pgfpathlineto{\pgfqpoint{5.352579in}{1.688201in}}%
\pgfpathlineto{\pgfqpoint{5.351912in}{2.213024in}}%
\pgfpathlineto{\pgfqpoint{5.352971in}{2.156891in}}%
\pgfpathlineto{\pgfqpoint{5.353871in}{1.813874in}}%
\pgfpathlineto{\pgfqpoint{5.353216in}{2.211823in}}%
\pgfpathlineto{\pgfqpoint{5.354076in}{2.014069in}}%
\pgfpathlineto{\pgfqpoint{5.354223in}{2.196457in}}%
\pgfpathlineto{\pgfqpoint{5.354769in}{1.719589in}}%
\pgfpathlineto{\pgfqpoint{5.355169in}{1.963108in}}%
\pgfpathlineto{\pgfqpoint{5.355276in}{1.706582in}}%
\pgfpathlineto{\pgfqpoint{5.355890in}{2.201998in}}%
\pgfpathlineto{\pgfqpoint{5.356259in}{2.113261in}}%
\pgfpathlineto{\pgfqpoint{5.356580in}{2.194012in}}%
\pgfpathlineto{\pgfqpoint{5.356317in}{1.648512in}}%
\pgfpathlineto{\pgfqpoint{5.357336in}{2.125795in}}%
\pgfpathlineto{\pgfqpoint{5.357763in}{1.757350in}}%
\pgfpathlineto{\pgfqpoint{5.357647in}{2.210302in}}%
\pgfpathlineto{\pgfqpoint{5.358450in}{2.018709in}}%
\pgfpathlineto{\pgfqpoint{5.359174in}{2.230365in}}%
\pgfpathlineto{\pgfqpoint{5.359512in}{1.675732in}}%
\pgfpathlineto{\pgfqpoint{5.359560in}{2.059577in}}%
\pgfpathlineto{\pgfqpoint{5.360485in}{2.205000in}}%
\pgfpathlineto{\pgfqpoint{5.359580in}{1.748879in}}%
\pgfpathlineto{\pgfqpoint{5.360678in}{2.111647in}}%
\pgfpathlineto{\pgfqpoint{5.361514in}{1.775241in}}%
\pgfpathlineto{\pgfqpoint{5.361446in}{2.195607in}}%
\pgfpathlineto{\pgfqpoint{5.361821in}{1.912326in}}%
\pgfpathlineto{\pgfqpoint{5.362319in}{2.201953in}}%
\pgfpathlineto{\pgfqpoint{5.362118in}{1.716118in}}%
\pgfpathlineto{\pgfqpoint{5.362941in}{2.096221in}}%
\pgfpathlineto{\pgfqpoint{5.363343in}{2.191126in}}%
\pgfpathlineto{\pgfqpoint{5.363649in}{1.787495in}}%
\pgfpathlineto{\pgfqpoint{5.363754in}{2.147909in}}%
\pgfpathlineto{\pgfqpoint{5.364593in}{1.674010in}}%
\pgfpathlineto{\pgfqpoint{5.364374in}{2.221119in}}%
\pgfpathlineto{\pgfqpoint{5.364860in}{2.030889in}}%
\pgfpathlineto{\pgfqpoint{5.365830in}{2.248549in}}%
\pgfpathlineto{\pgfqpoint{5.365859in}{1.719312in}}%
\pgfpathlineto{\pgfqpoint{5.365963in}{2.073934in}}%
\pgfpathlineto{\pgfqpoint{5.366647in}{1.584327in}}%
\pgfpathlineto{\pgfqpoint{5.366685in}{2.230211in}}%
\pgfpathlineto{\pgfqpoint{5.367073in}{2.057535in}}%
\pgfpathlineto{\pgfqpoint{5.368010in}{2.185425in}}%
\pgfpathlineto{\pgfqpoint{5.367376in}{1.575094in}}%
\pgfpathlineto{\pgfqpoint{5.368190in}{2.177274in}}%
\pgfpathlineto{\pgfqpoint{5.368983in}{1.695791in}}%
\pgfpathlineto{\pgfqpoint{5.368775in}{2.230618in}}%
\pgfpathlineto{\pgfqpoint{5.369304in}{2.083702in}}%
\pgfpathlineto{\pgfqpoint{5.369520in}{2.202591in}}%
\pgfpathlineto{\pgfqpoint{5.369982in}{1.687271in}}%
\pgfpathlineto{\pgfqpoint{5.370386in}{2.107878in}}%
\pgfpathlineto{\pgfqpoint{5.370743in}{1.740823in}}%
\pgfpathlineto{\pgfqpoint{5.370593in}{2.213417in}}%
\pgfpathlineto{\pgfqpoint{5.371494in}{2.145803in}}%
\pgfpathlineto{\pgfqpoint{5.372328in}{1.788732in}}%
\pgfpathlineto{\pgfqpoint{5.371887in}{2.202063in}}%
\pgfpathlineto{\pgfqpoint{5.372608in}{1.993294in}}%
\pgfpathlineto{\pgfqpoint{5.373150in}{2.199402in}}%
\pgfpathlineto{\pgfqpoint{5.373029in}{1.766031in}}%
\pgfpathlineto{\pgfqpoint{5.373701in}{2.171905in}}%
\pgfpathlineto{\pgfqpoint{5.374176in}{1.757609in}}%
\pgfpathlineto{\pgfqpoint{5.374642in}{2.209465in}}%
\pgfpathlineto{\pgfqpoint{5.374809in}{2.067880in}}%
\pgfpathlineto{\pgfqpoint{5.375702in}{2.202261in}}%
\pgfpathlineto{\pgfqpoint{5.375404in}{1.525914in}}%
\pgfpathlineto{\pgfqpoint{5.375915in}{2.179761in}}%
\pgfpathlineto{\pgfqpoint{5.376768in}{1.673172in}}%
\pgfpathlineto{\pgfqpoint{5.375934in}{2.257227in}}%
\pgfpathlineto{\pgfqpoint{5.377027in}{2.011923in}}%
\pgfpathlineto{\pgfqpoint{5.377314in}{2.238719in}}%
\pgfpathlineto{\pgfqpoint{5.377101in}{1.747377in}}%
\pgfpathlineto{\pgfqpoint{5.378155in}{2.179615in}}%
\pgfpathlineto{\pgfqpoint{5.378349in}{1.708335in}}%
\pgfpathlineto{\pgfqpoint{5.378616in}{2.244730in}}%
\pgfpathlineto{\pgfqpoint{5.379279in}{2.100600in}}%
\pgfpathlineto{\pgfqpoint{5.379601in}{1.760531in}}%
\pgfpathlineto{\pgfqpoint{5.379923in}{2.175355in}}%
\pgfpathlineto{\pgfqpoint{5.380373in}{1.975942in}}%
\pgfpathlineto{\pgfqpoint{5.380860in}{2.189131in}}%
\pgfpathlineto{\pgfqpoint{5.381419in}{1.623405in}}%
\pgfpathlineto{\pgfqpoint{5.381483in}{2.033729in}}%
\pgfpathlineto{\pgfqpoint{5.382005in}{2.218828in}}%
\pgfpathlineto{\pgfqpoint{5.381520in}{1.693319in}}%
\pgfpathlineto{\pgfqpoint{5.382581in}{2.013577in}}%
\pgfpathlineto{\pgfqpoint{5.382699in}{1.809460in}}%
\pgfpathlineto{\pgfqpoint{5.382791in}{2.184985in}}%
\pgfpathlineto{\pgfqpoint{5.383602in}{2.009121in}}%
\pgfpathlineto{\pgfqpoint{5.384304in}{2.192620in}}%
\pgfpathlineto{\pgfqpoint{5.384158in}{1.702372in}}%
\pgfpathlineto{\pgfqpoint{5.384713in}{2.161298in}}%
\pgfpathlineto{\pgfqpoint{5.385321in}{1.651130in}}%
\pgfpathlineto{\pgfqpoint{5.385421in}{2.187282in}}%
\pgfpathlineto{\pgfqpoint{5.385820in}{2.001299in}}%
\pgfpathlineto{\pgfqpoint{5.386807in}{2.240230in}}%
\pgfpathlineto{\pgfqpoint{5.386843in}{1.797179in}}%
\pgfpathlineto{\pgfqpoint{5.386925in}{2.093039in}}%
\pgfpathlineto{\pgfqpoint{5.387611in}{1.644548in}}%
\pgfpathlineto{\pgfqpoint{5.387214in}{2.196421in}}%
\pgfpathlineto{\pgfqpoint{5.388026in}{2.079439in}}%
\pgfpathlineto{\pgfqpoint{5.388468in}{2.201563in}}%
\pgfpathlineto{\pgfqpoint{5.389080in}{1.803152in}}%
\pgfpathlineto{\pgfqpoint{5.389134in}{2.148016in}}%
\pgfpathlineto{\pgfqpoint{5.390239in}{1.726476in}}%
\pgfpathlineto{\pgfqpoint{5.389673in}{2.221781in}}%
\pgfpathlineto{\pgfqpoint{5.390248in}{2.042540in}}%
\pgfpathlineto{\pgfqpoint{5.390920in}{2.259940in}}%
\pgfpathlineto{\pgfqpoint{5.390938in}{1.786093in}}%
\pgfpathlineto{\pgfqpoint{5.391368in}{2.190434in}}%
\pgfpathlineto{\pgfqpoint{5.392083in}{1.607482in}}%
\pgfpathlineto{\pgfqpoint{5.392351in}{2.218192in}}%
\pgfpathlineto{\pgfqpoint{5.392476in}{1.943422in}}%
\pgfpathlineto{\pgfqpoint{5.392859in}{2.233242in}}%
\pgfpathlineto{\pgfqpoint{5.392886in}{1.623819in}}%
\pgfpathlineto{\pgfqpoint{5.393581in}{2.019766in}}%
\pgfpathlineto{\pgfqpoint{5.394346in}{1.777835in}}%
\pgfpathlineto{\pgfqpoint{5.394452in}{2.215369in}}%
\pgfpathlineto{\pgfqpoint{5.394674in}{2.033379in}}%
\pgfpathlineto{\pgfqpoint{5.395685in}{2.218134in}}%
\pgfpathlineto{\pgfqpoint{5.395694in}{1.883169in}}%
\pgfpathlineto{\pgfqpoint{5.395791in}{2.141462in}}%
\pgfpathlineto{\pgfqpoint{5.396535in}{1.596312in}}%
\pgfpathlineto{\pgfqpoint{5.395818in}{2.221068in}}%
\pgfpathlineto{\pgfqpoint{5.396906in}{2.027079in}}%
\pgfpathlineto{\pgfqpoint{5.397797in}{2.213871in}}%
\pgfpathlineto{\pgfqpoint{5.397744in}{1.645996in}}%
\pgfpathlineto{\pgfqpoint{5.398017in}{2.046391in}}%
\pgfpathlineto{\pgfqpoint{5.398510in}{1.799203in}}%
\pgfpathlineto{\pgfqpoint{5.398985in}{2.215337in}}%
\pgfpathlineto{\pgfqpoint{5.399117in}{2.101975in}}%
\pgfpathlineto{\pgfqpoint{5.399801in}{2.212580in}}%
\pgfpathlineto{\pgfqpoint{5.399204in}{1.578851in}}%
\pgfpathlineto{\pgfqpoint{5.400178in}{2.082723in}}%
\pgfpathlineto{\pgfqpoint{5.400327in}{1.767261in}}%
\pgfpathlineto{\pgfqpoint{5.400914in}{2.227126in}}%
\pgfpathlineto{\pgfqpoint{5.401281in}{2.031655in}}%
\pgfpathlineto{\pgfqpoint{5.401298in}{2.222873in}}%
\pgfpathlineto{\pgfqpoint{5.401753in}{1.690548in}}%
\pgfpathlineto{\pgfqpoint{5.402390in}{2.047307in}}%
\pgfpathlineto{\pgfqpoint{5.402912in}{1.516045in}}%
\pgfpathlineto{\pgfqpoint{5.402938in}{2.216610in}}%
\pgfpathlineto{\pgfqpoint{5.403495in}{2.005767in}}%
\pgfpathlineto{\pgfqpoint{5.404182in}{2.219287in}}%
\pgfpathlineto{\pgfqpoint{5.404156in}{1.752432in}}%
\pgfpathlineto{\pgfqpoint{5.404616in}{2.094477in}}%
\pgfpathlineto{\pgfqpoint{5.405326in}{1.760371in}}%
\pgfpathlineto{\pgfqpoint{5.404728in}{2.212910in}}%
\pgfpathlineto{\pgfqpoint{5.405724in}{2.017092in}}%
\pgfpathlineto{\pgfqpoint{5.406226in}{2.233243in}}%
\pgfpathlineto{\pgfqpoint{5.406476in}{1.699114in}}%
\pgfpathlineto{\pgfqpoint{5.406839in}{2.143503in}}%
\pgfpathlineto{\pgfqpoint{5.407623in}{1.614880in}}%
\pgfpathlineto{\pgfqpoint{5.407649in}{2.210882in}}%
\pgfpathlineto{\pgfqpoint{5.407959in}{1.910775in}}%
\pgfpathlineto{\pgfqpoint{5.408870in}{2.250900in}}%
\pgfpathlineto{\pgfqpoint{5.408913in}{1.613692in}}%
\pgfpathlineto{\pgfqpoint{5.409076in}{2.170876in}}%
\pgfpathlineto{\pgfqpoint{5.409290in}{1.794138in}}%
\pgfpathlineto{\pgfqpoint{5.409101in}{2.229287in}}%
\pgfpathlineto{\pgfqpoint{5.410190in}{2.026940in}}%
\pgfpathlineto{\pgfqpoint{5.410618in}{2.202461in}}%
\pgfpathlineto{\pgfqpoint{5.410207in}{1.789535in}}%
\pgfpathlineto{\pgfqpoint{5.411310in}{2.195925in}}%
\pgfpathlineto{\pgfqpoint{5.412316in}{1.728333in}}%
\pgfpathlineto{\pgfqpoint{5.411702in}{2.258254in}}%
\pgfpathlineto{\pgfqpoint{5.412409in}{2.033891in}}%
\pgfpathlineto{\pgfqpoint{5.412826in}{2.236393in}}%
\pgfpathlineto{\pgfqpoint{5.413056in}{1.785255in}}%
\pgfpathlineto{\pgfqpoint{5.413523in}{2.138512in}}%
\pgfpathlineto{\pgfqpoint{5.414380in}{1.616311in}}%
\pgfpathlineto{\pgfqpoint{5.414482in}{2.232880in}}%
\pgfpathlineto{\pgfqpoint{5.414643in}{1.954709in}}%
\pgfpathlineto{\pgfqpoint{5.415235in}{2.247651in}}%
\pgfpathlineto{\pgfqpoint{5.415590in}{1.817420in}}%
\pgfpathlineto{\pgfqpoint{5.415751in}{1.902915in}}%
\pgfpathlineto{\pgfqpoint{5.415802in}{1.687270in}}%
\pgfpathlineto{\pgfqpoint{5.415785in}{2.236204in}}%
\pgfpathlineto{\pgfqpoint{5.416789in}{2.137977in}}%
\pgfpathlineto{\pgfqpoint{5.417588in}{2.208313in}}%
\pgfpathlineto{\pgfqpoint{5.417412in}{1.640564in}}%
\pgfpathlineto{\pgfqpoint{5.417900in}{2.158683in}}%
\pgfpathlineto{\pgfqpoint{5.418353in}{1.723986in}}%
\pgfpathlineto{\pgfqpoint{5.417950in}{2.242427in}}%
\pgfpathlineto{\pgfqpoint{5.419016in}{1.828598in}}%
\pgfpathlineto{\pgfqpoint{5.419678in}{2.237296in}}%
\pgfpathlineto{\pgfqpoint{5.420088in}{1.591996in}}%
\pgfpathlineto{\pgfqpoint{5.420129in}{2.111773in}}%
\pgfpathlineto{\pgfqpoint{5.420196in}{2.212889in}}%
\pgfpathlineto{\pgfqpoint{5.420363in}{1.686044in}}%
\pgfpathlineto{\pgfqpoint{5.420898in}{2.146344in}}%
\pgfpathlineto{\pgfqpoint{5.421023in}{1.602672in}}%
\pgfpathlineto{\pgfqpoint{5.421890in}{2.218040in}}%
\pgfpathlineto{\pgfqpoint{5.422007in}{2.128610in}}%
\pgfpathlineto{\pgfqpoint{5.422788in}{2.245611in}}%
\pgfpathlineto{\pgfqpoint{5.422323in}{1.562049in}}%
\pgfpathlineto{\pgfqpoint{5.422946in}{2.077774in}}%
\pgfpathlineto{\pgfqpoint{5.423660in}{1.818435in}}%
\pgfpathlineto{\pgfqpoint{5.423170in}{2.247607in}}%
\pgfpathlineto{\pgfqpoint{5.424058in}{1.979326in}}%
\pgfpathlineto{\pgfqpoint{5.425009in}{1.437130in}}%
\pgfpathlineto{\pgfqpoint{5.425166in}{2.230310in}}%
\pgfpathlineto{\pgfqpoint{5.426115in}{1.787865in}}%
\pgfpathlineto{\pgfqpoint{5.425711in}{2.236085in}}%
\pgfpathlineto{\pgfqpoint{5.426288in}{1.817007in}}%
\pgfpathlineto{\pgfqpoint{5.427235in}{2.240302in}}%
\pgfpathlineto{\pgfqpoint{5.426651in}{1.758995in}}%
\pgfpathlineto{\pgfqpoint{5.427399in}{2.088654in}}%
\pgfpathlineto{\pgfqpoint{5.428089in}{1.778679in}}%
\pgfpathlineto{\pgfqpoint{5.428097in}{2.247687in}}%
\pgfpathlineto{\pgfqpoint{5.428516in}{1.835628in}}%
\pgfpathlineto{\pgfqpoint{5.429130in}{2.240575in}}%
\pgfpathlineto{\pgfqpoint{5.429466in}{1.753314in}}%
\pgfpathlineto{\pgfqpoint{5.429629in}{2.098815in}}%
\pgfpathlineto{\pgfqpoint{5.430307in}{2.250895in}}%
\pgfpathlineto{\pgfqpoint{5.429940in}{1.818047in}}%
\pgfpathlineto{\pgfqpoint{5.430707in}{2.049678in}}%
\pgfpathlineto{\pgfqpoint{5.431326in}{1.856713in}}%
\pgfpathlineto{\pgfqpoint{5.431212in}{2.255856in}}%
\pgfpathlineto{\pgfqpoint{5.431815in}{1.948152in}}%
\pgfpathlineto{\pgfqpoint{5.431904in}{2.227274in}}%
\pgfpathlineto{\pgfqpoint{5.432457in}{1.682056in}}%
\pgfpathlineto{\pgfqpoint{5.432928in}{2.116472in}}%
\pgfpathlineto{\pgfqpoint{5.433892in}{1.727347in}}%
\pgfpathlineto{\pgfqpoint{5.433252in}{2.238384in}}%
\pgfpathlineto{\pgfqpoint{5.434037in}{2.117823in}}%
\pgfpathlineto{\pgfqpoint{5.434555in}{2.233390in}}%
\pgfpathlineto{\pgfqpoint{5.434183in}{1.770787in}}%
\pgfpathlineto{\pgfqpoint{5.435112in}{2.106779in}}%
\pgfpathlineto{\pgfqpoint{5.436023in}{1.752075in}}%
\pgfpathlineto{\pgfqpoint{5.435886in}{2.233035in}}%
\pgfpathlineto{\pgfqpoint{5.436217in}{2.106398in}}%
\pgfpathlineto{\pgfqpoint{5.437358in}{1.568044in}}%
\pgfpathlineto{\pgfqpoint{5.436796in}{2.243152in}}%
\pgfpathlineto{\pgfqpoint{5.437366in}{2.024813in}}%
\pgfpathlineto{\pgfqpoint{5.437623in}{2.233763in}}%
\pgfpathlineto{\pgfqpoint{5.437422in}{1.808158in}}%
\pgfpathlineto{\pgfqpoint{5.438481in}{2.171921in}}%
\pgfpathlineto{\pgfqpoint{5.439568in}{1.757520in}}%
\pgfpathlineto{\pgfqpoint{5.438657in}{2.218779in}}%
\pgfpathlineto{\pgfqpoint{5.439600in}{2.055381in}}%
\pgfpathlineto{\pgfqpoint{5.439656in}{2.241016in}}%
\pgfpathlineto{\pgfqpoint{5.439744in}{1.576416in}}%
\pgfpathlineto{\pgfqpoint{5.440693in}{2.161169in}}%
\pgfpathlineto{\pgfqpoint{5.441385in}{1.665080in}}%
\pgfpathlineto{\pgfqpoint{5.440844in}{2.216586in}}%
\pgfpathlineto{\pgfqpoint{5.441799in}{2.110435in}}%
\pgfpathlineto{\pgfqpoint{5.442227in}{1.723701in}}%
\pgfpathlineto{\pgfqpoint{5.442426in}{2.242046in}}%
\pgfpathlineto{\pgfqpoint{5.442886in}{2.113168in}}%
\pgfpathlineto{\pgfqpoint{5.443661in}{2.249036in}}%
\pgfpathlineto{\pgfqpoint{5.443867in}{1.736409in}}%
\pgfpathlineto{\pgfqpoint{5.443962in}{2.127148in}}%
\pgfpathlineto{\pgfqpoint{5.444009in}{1.637949in}}%
\pgfpathlineto{\pgfqpoint{5.444286in}{2.224663in}}%
\pgfpathlineto{\pgfqpoint{5.445067in}{2.042228in}}%
\pgfpathlineto{\pgfqpoint{5.445524in}{2.245127in}}%
\pgfpathlineto{\pgfqpoint{5.446107in}{1.571875in}}%
\pgfpathlineto{\pgfqpoint{5.446162in}{2.119351in}}%
\pgfpathlineto{\pgfqpoint{5.446500in}{1.705633in}}%
\pgfpathlineto{\pgfqpoint{5.446421in}{2.235036in}}%
\pgfpathlineto{\pgfqpoint{5.447269in}{2.163177in}}%
\pgfpathlineto{\pgfqpoint{5.447849in}{2.233433in}}%
\pgfpathlineto{\pgfqpoint{5.447653in}{1.698098in}}%
\pgfpathlineto{\pgfqpoint{5.448068in}{2.025072in}}%
\pgfpathlineto{\pgfqpoint{5.448624in}{1.683265in}}%
\pgfpathlineto{\pgfqpoint{5.448741in}{2.264775in}}%
\pgfpathlineto{\pgfqpoint{5.449170in}{1.948661in}}%
\pgfpathlineto{\pgfqpoint{5.449522in}{2.239586in}}%
\pgfpathlineto{\pgfqpoint{5.449514in}{1.760385in}}%
\pgfpathlineto{\pgfqpoint{5.450278in}{1.924068in}}%
\pgfpathlineto{\pgfqpoint{5.451118in}{2.230260in}}%
\pgfpathlineto{\pgfqpoint{5.450418in}{1.692067in}}%
\pgfpathlineto{\pgfqpoint{5.451398in}{2.208756in}}%
\pgfpathlineto{\pgfqpoint{5.452011in}{1.625306in}}%
\pgfpathlineto{\pgfqpoint{5.452205in}{2.261601in}}%
\pgfpathlineto{\pgfqpoint{5.452507in}{2.017185in}}%
\pgfpathlineto{\pgfqpoint{5.452747in}{2.243653in}}%
\pgfpathlineto{\pgfqpoint{5.452553in}{1.736198in}}%
\pgfpathlineto{\pgfqpoint{5.453621in}{2.116444in}}%
\pgfpathlineto{\pgfqpoint{5.454208in}{1.799927in}}%
\pgfpathlineto{\pgfqpoint{5.454455in}{2.257042in}}%
\pgfpathlineto{\pgfqpoint{5.454694in}{2.060908in}}%
\pgfpathlineto{\pgfqpoint{5.454802in}{2.233445in}}%
\pgfpathlineto{\pgfqpoint{5.454879in}{1.784183in}}%
\pgfpathlineto{\pgfqpoint{5.455802in}{2.125772in}}%
\pgfpathlineto{\pgfqpoint{5.456900in}{1.679696in}}%
\pgfpathlineto{\pgfqpoint{5.456846in}{2.263749in}}%
\pgfpathlineto{\pgfqpoint{5.456907in}{2.147264in}}%
\pgfpathlineto{\pgfqpoint{5.457934in}{1.591745in}}%
\pgfpathlineto{\pgfqpoint{5.457643in}{2.255515in}}%
\pgfpathlineto{\pgfqpoint{5.458010in}{2.070672in}}%
\pgfpathlineto{\pgfqpoint{5.458889in}{2.249055in}}%
\pgfpathlineto{\pgfqpoint{5.458614in}{1.729000in}}%
\pgfpathlineto{\pgfqpoint{5.459117in}{2.054419in}}%
\pgfpathlineto{\pgfqpoint{5.459956in}{2.241771in}}%
\pgfpathlineto{\pgfqpoint{5.459659in}{1.781281in}}%
\pgfpathlineto{\pgfqpoint{5.460207in}{2.155943in}}%
\pgfpathlineto{\pgfqpoint{5.460686in}{1.678124in}}%
\pgfpathlineto{\pgfqpoint{5.460640in}{2.224591in}}%
\pgfpathlineto{\pgfqpoint{5.461316in}{2.069859in}}%
\pgfpathlineto{\pgfqpoint{5.461407in}{1.563554in}}%
\pgfpathlineto{\pgfqpoint{5.461369in}{2.249056in}}%
\pgfpathlineto{\pgfqpoint{5.462430in}{1.973421in}}%
\pgfpathlineto{\pgfqpoint{5.463518in}{2.250117in}}%
\pgfpathlineto{\pgfqpoint{5.462521in}{1.715239in}}%
\pgfpathlineto{\pgfqpoint{5.463533in}{1.985388in}}%
\pgfpathlineto{\pgfqpoint{5.463586in}{1.657474in}}%
\pgfpathlineto{\pgfqpoint{5.464446in}{2.255781in}}%
\pgfpathlineto{\pgfqpoint{5.464634in}{1.921347in}}%
\pgfpathlineto{\pgfqpoint{5.465431in}{2.256455in}}%
\pgfpathlineto{\pgfqpoint{5.465048in}{1.777147in}}%
\pgfpathlineto{\pgfqpoint{5.465747in}{2.129961in}}%
\pgfpathlineto{\pgfqpoint{5.466272in}{1.772199in}}%
\pgfpathlineto{\pgfqpoint{5.466572in}{2.253465in}}%
\pgfpathlineto{\pgfqpoint{5.466849in}{2.206788in}}%
\pgfpathlineto{\pgfqpoint{5.467395in}{2.260052in}}%
\pgfpathlineto{\pgfqpoint{5.467276in}{1.601293in}}%
\pgfpathlineto{\pgfqpoint{5.467874in}{2.075615in}}%
\pgfpathlineto{\pgfqpoint{5.468008in}{1.749173in}}%
\pgfpathlineto{\pgfqpoint{5.468135in}{2.231058in}}%
\pgfpathlineto{\pgfqpoint{5.468978in}{2.078792in}}%
\pgfpathlineto{\pgfqpoint{5.469574in}{2.241232in}}%
\pgfpathlineto{\pgfqpoint{5.469648in}{1.853673in}}%
\pgfpathlineto{\pgfqpoint{5.470087in}{2.155954in}}%
\pgfpathlineto{\pgfqpoint{5.470807in}{1.852718in}}%
\pgfpathlineto{\pgfqpoint{5.470889in}{2.256550in}}%
\pgfpathlineto{\pgfqpoint{5.471223in}{1.980412in}}%
\pgfpathlineto{\pgfqpoint{5.471801in}{2.245285in}}%
\pgfpathlineto{\pgfqpoint{5.471963in}{1.688896in}}%
\pgfpathlineto{\pgfqpoint{5.472341in}{2.145630in}}%
\pgfpathlineto{\pgfqpoint{5.473234in}{1.678073in}}%
\pgfpathlineto{\pgfqpoint{5.473441in}{2.217505in}}%
\pgfpathlineto{\pgfqpoint{5.473448in}{2.039891in}}%
\pgfpathlineto{\pgfqpoint{5.473603in}{2.223431in}}%
\pgfpathlineto{\pgfqpoint{5.473507in}{1.806154in}}%
\pgfpathlineto{\pgfqpoint{5.474560in}{2.110197in}}%
\pgfpathlineto{\pgfqpoint{5.475030in}{1.610966in}}%
\pgfpathlineto{\pgfqpoint{5.474906in}{2.221247in}}%
\pgfpathlineto{\pgfqpoint{5.475537in}{2.014079in}}%
\pgfpathlineto{\pgfqpoint{5.476300in}{2.246669in}}%
\pgfpathlineto{\pgfqpoint{5.476592in}{1.619609in}}%
\pgfpathlineto{\pgfqpoint{5.476651in}{2.082929in}}%
\pgfpathlineto{\pgfqpoint{5.477375in}{2.245552in}}%
\pgfpathlineto{\pgfqpoint{5.477273in}{1.782207in}}%
\pgfpathlineto{\pgfqpoint{5.477740in}{2.118593in}}%
\pgfpathlineto{\pgfqpoint{5.477820in}{1.647603in}}%
\pgfpathlineto{\pgfqpoint{5.478746in}{2.226579in}}%
\pgfpathlineto{\pgfqpoint{5.478848in}{2.117844in}}%
\pgfpathlineto{\pgfqpoint{5.478892in}{1.818895in}}%
\pgfpathlineto{\pgfqpoint{5.479801in}{2.292042in}}%
\pgfpathlineto{\pgfqpoint{5.479953in}{1.961722in}}%
\pgfpathlineto{\pgfqpoint{5.480961in}{2.271984in}}%
\pgfpathlineto{\pgfqpoint{5.480113in}{1.863960in}}%
\pgfpathlineto{\pgfqpoint{5.481063in}{2.137379in}}%
\pgfpathlineto{\pgfqpoint{5.481345in}{1.666129in}}%
\pgfpathlineto{\pgfqpoint{5.481765in}{2.224961in}}%
\pgfpathlineto{\pgfqpoint{5.482170in}{2.027284in}}%
\pgfpathlineto{\pgfqpoint{5.483028in}{2.235014in}}%
\pgfpathlineto{\pgfqpoint{5.482235in}{1.679370in}}%
\pgfpathlineto{\pgfqpoint{5.483281in}{2.143334in}}%
\pgfpathlineto{\pgfqpoint{5.483432in}{1.811330in}}%
\pgfpathlineto{\pgfqpoint{5.483972in}{2.246731in}}%
\pgfpathlineto{\pgfqpoint{5.484389in}{2.071192in}}%
\pgfpathlineto{\pgfqpoint{5.485006in}{2.255078in}}%
\pgfpathlineto{\pgfqpoint{5.484769in}{1.816100in}}%
\pgfpathlineto{\pgfqpoint{5.485465in}{2.027611in}}%
\pgfpathlineto{\pgfqpoint{5.485809in}{1.737874in}}%
\pgfpathlineto{\pgfqpoint{5.485737in}{2.265912in}}%
\pgfpathlineto{\pgfqpoint{5.486568in}{2.055666in}}%
\pgfpathlineto{\pgfqpoint{5.487061in}{2.256843in}}%
\pgfpathlineto{\pgfqpoint{5.487153in}{1.775118in}}%
\pgfpathlineto{\pgfqpoint{5.487674in}{2.107134in}}%
\pgfpathlineto{\pgfqpoint{5.488316in}{1.743697in}}%
\pgfpathlineto{\pgfqpoint{5.488551in}{2.235817in}}%
\pgfpathlineto{\pgfqpoint{5.488785in}{1.955622in}}%
\pgfpathlineto{\pgfqpoint{5.489404in}{2.247336in}}%
\pgfpathlineto{\pgfqpoint{5.489318in}{1.650452in}}%
\pgfpathlineto{\pgfqpoint{5.489900in}{2.053329in}}%
\pgfpathlineto{\pgfqpoint{5.490354in}{2.277707in}}%
\pgfpathlineto{\pgfqpoint{5.490099in}{1.717626in}}%
\pgfpathlineto{\pgfqpoint{5.491020in}{2.135763in}}%
\pgfpathlineto{\pgfqpoint{5.491437in}{1.738393in}}%
\pgfpathlineto{\pgfqpoint{5.491409in}{2.245821in}}%
\pgfpathlineto{\pgfqpoint{5.492129in}{2.040311in}}%
\pgfpathlineto{\pgfqpoint{5.493059in}{2.260205in}}%
\pgfpathlineto{\pgfqpoint{5.492693in}{1.698199in}}%
\pgfpathlineto{\pgfqpoint{5.493235in}{2.041423in}}%
\pgfpathlineto{\pgfqpoint{5.493355in}{1.792080in}}%
\pgfpathlineto{\pgfqpoint{5.493397in}{2.270852in}}%
\pgfpathlineto{\pgfqpoint{5.494283in}{2.040932in}}%
\pgfpathlineto{\pgfqpoint{5.495166in}{2.266170in}}%
\pgfpathlineto{\pgfqpoint{5.494977in}{1.785667in}}%
\pgfpathlineto{\pgfqpoint{5.495390in}{2.138438in}}%
\pgfpathlineto{\pgfqpoint{5.495733in}{1.678756in}}%
\pgfpathlineto{\pgfqpoint{5.496076in}{2.237744in}}%
\pgfpathlineto{\pgfqpoint{5.496509in}{1.826557in}}%
\pgfpathlineto{\pgfqpoint{5.496719in}{2.248502in}}%
\pgfpathlineto{\pgfqpoint{5.496865in}{1.728267in}}%
\pgfpathlineto{\pgfqpoint{5.497618in}{2.197651in}}%
\pgfpathlineto{\pgfqpoint{5.498502in}{1.794847in}}%
\pgfpathlineto{\pgfqpoint{5.498467in}{2.263756in}}%
\pgfpathlineto{\pgfqpoint{5.498731in}{1.912931in}}%
\pgfpathlineto{\pgfqpoint{5.499814in}{2.283061in}}%
\pgfpathlineto{\pgfqpoint{5.499426in}{1.575546in}}%
\pgfpathlineto{\pgfqpoint{5.499855in}{2.114502in}}%
\pgfpathlineto{\pgfqpoint{5.500430in}{1.686548in}}%
\pgfpathlineto{\pgfqpoint{5.499925in}{2.287463in}}%
\pgfpathlineto{\pgfqpoint{5.500921in}{2.083689in}}%
\pgfpathlineto{\pgfqpoint{5.501819in}{2.244947in}}%
\pgfpathlineto{\pgfqpoint{5.501356in}{1.833358in}}%
\pgfpathlineto{\pgfqpoint{5.502032in}{2.142290in}}%
\pgfpathlineto{\pgfqpoint{5.502845in}{1.613611in}}%
\pgfpathlineto{\pgfqpoint{5.502790in}{2.262547in}}%
\pgfpathlineto{\pgfqpoint{5.503134in}{2.045630in}}%
\pgfpathlineto{\pgfqpoint{5.503581in}{2.250782in}}%
\pgfpathlineto{\pgfqpoint{5.503670in}{1.761916in}}%
\pgfpathlineto{\pgfqpoint{5.504239in}{2.182028in}}%
\pgfpathlineto{\pgfqpoint{5.504877in}{1.692821in}}%
\pgfpathlineto{\pgfqpoint{5.505246in}{2.260295in}}%
\pgfpathlineto{\pgfqpoint{5.505342in}{2.174869in}}%
\pgfpathlineto{\pgfqpoint{5.505349in}{2.245664in}}%
\pgfpathlineto{\pgfqpoint{5.506264in}{1.641550in}}%
\pgfpathlineto{\pgfqpoint{5.506428in}{2.082596in}}%
\pgfpathlineto{\pgfqpoint{5.507096in}{1.757194in}}%
\pgfpathlineto{\pgfqpoint{5.507307in}{2.275539in}}%
\pgfpathlineto{\pgfqpoint{5.507532in}{1.981138in}}%
\pgfpathlineto{\pgfqpoint{5.508558in}{2.272842in}}%
\pgfpathlineto{\pgfqpoint{5.507954in}{1.693418in}}%
\pgfpathlineto{\pgfqpoint{5.508647in}{2.088592in}}%
\pgfpathlineto{\pgfqpoint{5.508701in}{2.269649in}}%
\pgfpathlineto{\pgfqpoint{5.508660in}{1.808377in}}%
\pgfpathlineto{\pgfqpoint{5.509731in}{2.155739in}}%
\pgfpathlineto{\pgfqpoint{5.510293in}{1.648277in}}%
\pgfpathlineto{\pgfqpoint{5.510232in}{2.262140in}}%
\pgfpathlineto{\pgfqpoint{5.510840in}{2.133922in}}%
\pgfpathlineto{\pgfqpoint{5.511191in}{1.771379in}}%
\pgfpathlineto{\pgfqpoint{5.511797in}{2.248596in}}%
\pgfpathlineto{\pgfqpoint{5.511952in}{1.986280in}}%
\pgfpathlineto{\pgfqpoint{5.513029in}{2.252201in}}%
\pgfpathlineto{\pgfqpoint{5.512605in}{1.691284in}}%
\pgfpathlineto{\pgfqpoint{5.513069in}{2.222152in}}%
\pgfpathlineto{\pgfqpoint{5.514162in}{1.768781in}}%
\pgfpathlineto{\pgfqpoint{5.513760in}{2.260594in}}%
\pgfpathlineto{\pgfqpoint{5.514176in}{2.011898in}}%
\pgfpathlineto{\pgfqpoint{5.514343in}{2.268021in}}%
\pgfpathlineto{\pgfqpoint{5.515106in}{1.835297in}}%
\pgfpathlineto{\pgfqpoint{5.515293in}{2.169617in}}%
\pgfpathlineto{\pgfqpoint{5.516341in}{2.276902in}}%
\pgfpathlineto{\pgfqpoint{5.515754in}{1.724683in}}%
\pgfpathlineto{\pgfqpoint{5.516354in}{2.161979in}}%
\pgfpathlineto{\pgfqpoint{5.516434in}{1.733960in}}%
\pgfpathlineto{\pgfqpoint{5.516887in}{2.258765in}}%
\pgfpathlineto{\pgfqpoint{5.517459in}{1.998756in}}%
\pgfpathlineto{\pgfqpoint{5.518037in}{2.294036in}}%
\pgfpathlineto{\pgfqpoint{5.518169in}{1.869903in}}%
\pgfpathlineto{\pgfqpoint{5.518574in}{2.213005in}}%
\pgfpathlineto{\pgfqpoint{5.519634in}{1.649814in}}%
\pgfpathlineto{\pgfqpoint{5.519078in}{2.257498in}}%
\pgfpathlineto{\pgfqpoint{5.519686in}{2.070598in}}%
\pgfpathlineto{\pgfqpoint{5.520070in}{2.272410in}}%
\pgfpathlineto{\pgfqpoint{5.520116in}{1.692444in}}%
\pgfpathlineto{\pgfqpoint{5.520789in}{2.105703in}}%
\pgfpathlineto{\pgfqpoint{5.520928in}{1.597149in}}%
\pgfpathlineto{\pgfqpoint{5.521659in}{2.260173in}}%
\pgfpathlineto{\pgfqpoint{5.521896in}{2.036031in}}%
\pgfpathlineto{\pgfqpoint{5.522658in}{2.290578in}}%
\pgfpathlineto{\pgfqpoint{5.522481in}{1.675715in}}%
\pgfpathlineto{\pgfqpoint{5.523013in}{2.116930in}}%
\pgfpathlineto{\pgfqpoint{5.523393in}{1.851006in}}%
\pgfpathlineto{\pgfqpoint{5.523426in}{2.265652in}}%
\pgfpathlineto{\pgfqpoint{5.524120in}{2.107073in}}%
\pgfpathlineto{\pgfqpoint{5.524709in}{2.281618in}}%
\pgfpathlineto{\pgfqpoint{5.524441in}{1.731759in}}%
\pgfpathlineto{\pgfqpoint{5.525231in}{2.220798in}}%
\pgfpathlineto{\pgfqpoint{5.525668in}{1.857626in}}%
\pgfpathlineto{\pgfqpoint{5.525935in}{2.290554in}}%
\pgfpathlineto{\pgfqpoint{5.526352in}{2.056411in}}%
\pgfpathlineto{\pgfqpoint{5.527425in}{2.247574in}}%
\pgfpathlineto{\pgfqpoint{5.527405in}{1.734112in}}%
\pgfpathlineto{\pgfqpoint{5.527457in}{2.178431in}}%
\pgfpathlineto{\pgfqpoint{5.528132in}{1.616060in}}%
\pgfpathlineto{\pgfqpoint{5.527613in}{2.291474in}}%
\pgfpathlineto{\pgfqpoint{5.528573in}{1.946860in}}%
\pgfpathlineto{\pgfqpoint{5.529013in}{2.285840in}}%
\pgfpathlineto{\pgfqpoint{5.528909in}{1.677246in}}%
\pgfpathlineto{\pgfqpoint{5.529691in}{2.255591in}}%
\pgfpathlineto{\pgfqpoint{5.530434in}{1.848442in}}%
\pgfpathlineto{\pgfqpoint{5.530221in}{2.281766in}}%
\pgfpathlineto{\pgfqpoint{5.530801in}{2.166646in}}%
\pgfpathlineto{\pgfqpoint{5.530839in}{2.263036in}}%
\pgfpathlineto{\pgfqpoint{5.530878in}{1.711487in}}%
\pgfpathlineto{\pgfqpoint{5.531901in}{2.090994in}}%
\pgfpathlineto{\pgfqpoint{5.532614in}{2.285427in}}%
\pgfpathlineto{\pgfqpoint{5.532370in}{1.851639in}}%
\pgfpathlineto{\pgfqpoint{5.532979in}{2.119443in}}%
\pgfpathlineto{\pgfqpoint{5.533498in}{1.739392in}}%
\pgfpathlineto{\pgfqpoint{5.533421in}{2.281885in}}%
\pgfpathlineto{\pgfqpoint{5.534086in}{2.108075in}}%
\pgfpathlineto{\pgfqpoint{5.534252in}{1.806984in}}%
\pgfpathlineto{\pgfqpoint{5.534916in}{2.315004in}}%
\pgfpathlineto{\pgfqpoint{5.535191in}{1.954949in}}%
\pgfpathlineto{\pgfqpoint{5.535815in}{2.267306in}}%
\pgfpathlineto{\pgfqpoint{5.535789in}{1.842041in}}%
\pgfpathlineto{\pgfqpoint{5.536298in}{2.123894in}}%
\pgfpathlineto{\pgfqpoint{5.537042in}{1.784583in}}%
\pgfpathlineto{\pgfqpoint{5.537105in}{2.271263in}}%
\pgfpathlineto{\pgfqpoint{5.537410in}{2.028476in}}%
\pgfpathlineto{\pgfqpoint{5.538189in}{2.249670in}}%
\pgfpathlineto{\pgfqpoint{5.537619in}{1.831916in}}%
\pgfpathlineto{\pgfqpoint{5.538525in}{2.243017in}}%
\pgfpathlineto{\pgfqpoint{5.538961in}{1.763812in}}%
\pgfpathlineto{\pgfqpoint{5.539232in}{2.290146in}}%
\pgfpathlineto{\pgfqpoint{5.539636in}{2.047726in}}%
\pgfpathlineto{\pgfqpoint{5.539699in}{2.284590in}}%
\pgfpathlineto{\pgfqpoint{5.540556in}{1.855937in}}%
\pgfpathlineto{\pgfqpoint{5.540764in}{2.176878in}}%
\pgfpathlineto{\pgfqpoint{5.541336in}{1.769401in}}%
\pgfpathlineto{\pgfqpoint{5.541204in}{2.280431in}}%
\pgfpathlineto{\pgfqpoint{5.541876in}{2.112969in}}%
\pgfpathlineto{\pgfqpoint{5.541983in}{2.253231in}}%
\pgfpathlineto{\pgfqpoint{5.542253in}{1.723282in}}%
\pgfpathlineto{\pgfqpoint{5.542986in}{2.120043in}}%
\pgfpathlineto{\pgfqpoint{5.543499in}{1.738655in}}%
\pgfpathlineto{\pgfqpoint{5.543955in}{2.278035in}}%
\pgfpathlineto{\pgfqpoint{5.544092in}{2.026080in}}%
\pgfpathlineto{\pgfqpoint{5.544548in}{2.283645in}}%
\pgfpathlineto{\pgfqpoint{5.544404in}{1.847114in}}%
\pgfpathlineto{\pgfqpoint{5.545208in}{2.153907in}}%
\pgfpathlineto{\pgfqpoint{5.546079in}{1.856169in}}%
\pgfpathlineto{\pgfqpoint{5.545886in}{2.294116in}}%
\pgfpathlineto{\pgfqpoint{5.546315in}{2.049517in}}%
\pgfpathlineto{\pgfqpoint{5.547283in}{2.286961in}}%
\pgfpathlineto{\pgfqpoint{5.547227in}{1.811875in}}%
\pgfpathlineto{\pgfqpoint{5.547432in}{2.200495in}}%
\pgfpathlineto{\pgfqpoint{5.547506in}{1.813850in}}%
\pgfpathlineto{\pgfqpoint{5.548007in}{2.262947in}}%
\pgfpathlineto{\pgfqpoint{5.548551in}{1.990097in}}%
\pgfpathlineto{\pgfqpoint{5.549495in}{2.285065in}}%
\pgfpathlineto{\pgfqpoint{5.549212in}{1.878436in}}%
\pgfpathlineto{\pgfqpoint{5.549680in}{2.146644in}}%
\pgfpathlineto{\pgfqpoint{5.550437in}{1.686548in}}%
\pgfpathlineto{\pgfqpoint{5.549927in}{2.280053in}}%
\pgfpathlineto{\pgfqpoint{5.550788in}{1.909413in}}%
\pgfpathlineto{\pgfqpoint{5.551518in}{2.292178in}}%
\pgfpathlineto{\pgfqpoint{5.551856in}{1.574357in}}%
\pgfpathlineto{\pgfqpoint{5.551899in}{2.165318in}}%
\pgfpathlineto{\pgfqpoint{5.552652in}{1.712306in}}%
\pgfpathlineto{\pgfqpoint{5.552187in}{2.257599in}}%
\pgfpathlineto{\pgfqpoint{5.553013in}{1.985718in}}%
\pgfpathlineto{\pgfqpoint{5.553581in}{2.271292in}}%
\pgfpathlineto{\pgfqpoint{5.553831in}{1.767510in}}%
\pgfpathlineto{\pgfqpoint{5.554124in}{2.194045in}}%
\pgfpathlineto{\pgfqpoint{5.554167in}{1.878520in}}%
\pgfpathlineto{\pgfqpoint{5.554550in}{2.271970in}}%
\pgfpathlineto{\pgfqpoint{5.555238in}{2.082419in}}%
\pgfpathlineto{\pgfqpoint{5.556143in}{2.267120in}}%
\pgfpathlineto{\pgfqpoint{5.555378in}{1.732504in}}%
\pgfpathlineto{\pgfqpoint{5.556301in}{2.145001in}}%
\pgfpathlineto{\pgfqpoint{5.557361in}{1.696709in}}%
\pgfpathlineto{\pgfqpoint{5.557143in}{2.285361in}}%
\pgfpathlineto{\pgfqpoint{5.557410in}{2.170890in}}%
\pgfpathlineto{\pgfqpoint{5.557905in}{1.820962in}}%
\pgfpathlineto{\pgfqpoint{5.557500in}{2.290133in}}%
\pgfpathlineto{\pgfqpoint{5.558521in}{2.105000in}}%
\pgfpathlineto{\pgfqpoint{5.559221in}{2.256943in}}%
\pgfpathlineto{\pgfqpoint{5.558666in}{1.775286in}}%
\pgfpathlineto{\pgfqpoint{5.559630in}{2.133076in}}%
\pgfpathlineto{\pgfqpoint{5.559865in}{1.793869in}}%
\pgfpathlineto{\pgfqpoint{5.560352in}{2.291760in}}%
\pgfpathlineto{\pgfqpoint{5.560694in}{2.195365in}}%
\pgfpathlineto{\pgfqpoint{5.561624in}{2.273640in}}%
\pgfpathlineto{\pgfqpoint{5.561660in}{1.717483in}}%
\pgfpathlineto{\pgfqpoint{5.561785in}{2.142529in}}%
\pgfpathlineto{\pgfqpoint{5.562103in}{1.790646in}}%
\pgfpathlineto{\pgfqpoint{5.561821in}{2.292457in}}%
\pgfpathlineto{\pgfqpoint{5.562886in}{2.016755in}}%
\pgfpathlineto{\pgfqpoint{5.563828in}{2.262432in}}%
\pgfpathlineto{\pgfqpoint{5.563465in}{1.730145in}}%
\pgfpathlineto{\pgfqpoint{5.563995in}{2.213364in}}%
\pgfpathlineto{\pgfqpoint{5.564222in}{1.766102in}}%
\pgfpathlineto{\pgfqpoint{5.564067in}{2.264688in}}%
\pgfpathlineto{\pgfqpoint{5.565108in}{2.166270in}}%
\pgfpathlineto{\pgfqpoint{5.565233in}{2.304146in}}%
\pgfpathlineto{\pgfqpoint{5.565155in}{1.792235in}}%
\pgfpathlineto{\pgfqpoint{5.566217in}{2.175556in}}%
\pgfpathlineto{\pgfqpoint{5.566827in}{1.674990in}}%
\pgfpathlineto{\pgfqpoint{5.566419in}{2.287517in}}%
\pgfpathlineto{\pgfqpoint{5.567348in}{1.978629in}}%
\pgfpathlineto{\pgfqpoint{5.567968in}{2.309553in}}%
\pgfpathlineto{\pgfqpoint{5.567991in}{1.614936in}}%
\pgfpathlineto{\pgfqpoint{5.568463in}{2.195204in}}%
\pgfpathlineto{\pgfqpoint{5.568976in}{1.847497in}}%
\pgfpathlineto{\pgfqpoint{5.569123in}{2.300261in}}%
\pgfpathlineto{\pgfqpoint{5.569558in}{2.055843in}}%
\pgfpathlineto{\pgfqpoint{5.569729in}{2.286606in}}%
\pgfpathlineto{\pgfqpoint{5.570504in}{1.705382in}}%
\pgfpathlineto{\pgfqpoint{5.570668in}{2.121503in}}%
\pgfpathlineto{\pgfqpoint{5.570768in}{1.586832in}}%
\pgfpathlineto{\pgfqpoint{5.570897in}{2.302236in}}%
\pgfpathlineto{\pgfqpoint{5.571769in}{2.107775in}}%
\pgfpathlineto{\pgfqpoint{5.572780in}{2.283982in}}%
\pgfpathlineto{\pgfqpoint{5.572155in}{1.842263in}}%
\pgfpathlineto{\pgfqpoint{5.572885in}{2.209347in}}%
\pgfpathlineto{\pgfqpoint{5.573852in}{1.770608in}}%
\pgfpathlineto{\pgfqpoint{5.573893in}{2.300416in}}%
\pgfpathlineto{\pgfqpoint{5.574009in}{2.006365in}}%
\pgfpathlineto{\pgfqpoint{5.574887in}{2.281673in}}%
\pgfpathlineto{\pgfqpoint{5.574311in}{1.746040in}}%
\pgfpathlineto{\pgfqpoint{5.575107in}{2.127429in}}%
\pgfpathlineto{\pgfqpoint{5.575217in}{1.643989in}}%
\pgfpathlineto{\pgfqpoint{5.575264in}{2.267795in}}%
\pgfpathlineto{\pgfqpoint{5.576214in}{2.167519in}}%
\pgfpathlineto{\pgfqpoint{5.576683in}{1.876774in}}%
\pgfpathlineto{\pgfqpoint{5.576636in}{2.277098in}}%
\pgfpathlineto{\pgfqpoint{5.577324in}{2.043910in}}%
\pgfpathlineto{\pgfqpoint{5.577814in}{2.268262in}}%
\pgfpathlineto{\pgfqpoint{5.578339in}{1.681502in}}%
\pgfpathlineto{\pgfqpoint{5.578442in}{2.201179in}}%
\pgfpathlineto{\pgfqpoint{5.579477in}{1.810949in}}%
\pgfpathlineto{\pgfqpoint{5.578465in}{2.314013in}}%
\pgfpathlineto{\pgfqpoint{5.579569in}{1.971994in}}%
\pgfpathlineto{\pgfqpoint{5.579816in}{2.310828in}}%
\pgfpathlineto{\pgfqpoint{5.579914in}{1.571492in}}%
\pgfpathlineto{\pgfqpoint{5.580676in}{2.161368in}}%
\pgfpathlineto{\pgfqpoint{5.581740in}{1.702019in}}%
\pgfpathlineto{\pgfqpoint{5.581277in}{2.275605in}}%
\pgfpathlineto{\pgfqpoint{5.581786in}{1.957436in}}%
\pgfpathlineto{\pgfqpoint{5.582750in}{2.291076in}}%
\pgfpathlineto{\pgfqpoint{5.581974in}{1.882573in}}%
\pgfpathlineto{\pgfqpoint{5.582898in}{2.230567in}}%
\pgfpathlineto{\pgfqpoint{5.583513in}{2.298551in}}%
\pgfpathlineto{\pgfqpoint{5.582921in}{1.743926in}}%
\pgfpathlineto{\pgfqpoint{5.583968in}{2.246993in}}%
\pgfpathlineto{\pgfqpoint{5.584559in}{1.646750in}}%
\pgfpathlineto{\pgfqpoint{5.584400in}{2.299343in}}%
\pgfpathlineto{\pgfqpoint{5.585075in}{2.089043in}}%
\pgfpathlineto{\pgfqpoint{5.585902in}{2.301975in}}%
\pgfpathlineto{\pgfqpoint{5.585477in}{1.530920in}}%
\pgfpathlineto{\pgfqpoint{5.586173in}{2.219504in}}%
\pgfpathlineto{\pgfqpoint{5.586478in}{1.793358in}}%
\pgfpathlineto{\pgfqpoint{5.586732in}{2.277235in}}%
\pgfpathlineto{\pgfqpoint{5.587285in}{2.090154in}}%
\pgfpathlineto{\pgfqpoint{5.587302in}{1.803252in}}%
\pgfpathlineto{\pgfqpoint{5.588057in}{2.295221in}}%
\pgfpathlineto{\pgfqpoint{5.588372in}{2.128850in}}%
\pgfpathlineto{\pgfqpoint{5.589417in}{2.281752in}}%
\pgfpathlineto{\pgfqpoint{5.589355in}{1.729576in}}%
\pgfpathlineto{\pgfqpoint{5.589484in}{2.146404in}}%
\pgfpathlineto{\pgfqpoint{5.589776in}{1.667426in}}%
\pgfpathlineto{\pgfqpoint{5.589625in}{2.279625in}}%
\pgfpathlineto{\pgfqpoint{5.590538in}{2.173456in}}%
\pgfpathlineto{\pgfqpoint{5.590929in}{2.315334in}}%
\pgfpathlineto{\pgfqpoint{5.590840in}{1.894640in}}%
\pgfpathlineto{\pgfqpoint{5.591644in}{2.180901in}}%
\pgfpathlineto{\pgfqpoint{5.591834in}{1.867075in}}%
\pgfpathlineto{\pgfqpoint{5.592698in}{2.324782in}}%
\pgfpathlineto{\pgfqpoint{5.592754in}{1.979467in}}%
\pgfpathlineto{\pgfqpoint{5.593749in}{2.301205in}}%
\pgfpathlineto{\pgfqpoint{5.593421in}{1.689780in}}%
\pgfpathlineto{\pgfqpoint{5.593865in}{2.186601in}}%
\pgfpathlineto{\pgfqpoint{5.594032in}{1.814045in}}%
\pgfpathlineto{\pgfqpoint{5.593988in}{2.276756in}}%
\pgfpathlineto{\pgfqpoint{5.594980in}{2.042560in}}%
\pgfpathlineto{\pgfqpoint{5.595920in}{2.302192in}}%
\pgfpathlineto{\pgfqpoint{5.595456in}{1.793991in}}%
\pgfpathlineto{\pgfqpoint{5.596091in}{2.167405in}}%
\pgfpathlineto{\pgfqpoint{5.596169in}{1.634063in}}%
\pgfpathlineto{\pgfqpoint{5.596323in}{2.276579in}}%
\pgfpathlineto{\pgfqpoint{5.597184in}{2.104005in}}%
\pgfpathlineto{\pgfqpoint{5.597503in}{2.272959in}}%
\pgfpathlineto{\pgfqpoint{5.597707in}{1.815238in}}%
\pgfpathlineto{\pgfqpoint{5.598295in}{2.191597in}}%
\pgfpathlineto{\pgfqpoint{5.598306in}{1.767409in}}%
\pgfpathlineto{\pgfqpoint{5.598361in}{2.273242in}}%
\pgfpathlineto{\pgfqpoint{5.599409in}{1.995669in}}%
\pgfpathlineto{\pgfqpoint{5.600121in}{2.306791in}}%
\pgfpathlineto{\pgfqpoint{5.599748in}{1.860733in}}%
\pgfpathlineto{\pgfqpoint{5.600525in}{2.139812in}}%
\pgfpathlineto{\pgfqpoint{5.601175in}{2.300635in}}%
\pgfpathlineto{\pgfqpoint{5.600820in}{1.799649in}}%
\pgfpathlineto{\pgfqpoint{5.601633in}{2.245778in}}%
\pgfpathlineto{\pgfqpoint{5.602374in}{1.846932in}}%
\pgfpathlineto{\pgfqpoint{5.602673in}{2.262325in}}%
\pgfpathlineto{\pgfqpoint{5.602744in}{2.128843in}}%
\pgfpathlineto{\pgfqpoint{5.603543in}{2.296003in}}%
\pgfpathlineto{\pgfqpoint{5.603233in}{1.862540in}}%
\pgfpathlineto{\pgfqpoint{5.603868in}{2.228661in}}%
\pgfpathlineto{\pgfqpoint{5.604074in}{1.627984in}}%
\pgfpathlineto{\pgfqpoint{5.604816in}{2.297473in}}%
\pgfpathlineto{\pgfqpoint{5.604978in}{2.122787in}}%
\pgfpathlineto{\pgfqpoint{5.606037in}{2.293434in}}%
\pgfpathlineto{\pgfqpoint{5.605303in}{1.733589in}}%
\pgfpathlineto{\pgfqpoint{5.606091in}{2.222225in}}%
\pgfpathlineto{\pgfqpoint{5.606485in}{1.791575in}}%
\pgfpathlineto{\pgfqpoint{5.606420in}{2.288645in}}%
\pgfpathlineto{\pgfqpoint{5.607201in}{2.096658in}}%
\pgfpathlineto{\pgfqpoint{5.608265in}{2.288400in}}%
\pgfpathlineto{\pgfqpoint{5.608055in}{1.806437in}}%
\pgfpathlineto{\pgfqpoint{5.608308in}{2.246725in}}%
\pgfpathlineto{\pgfqpoint{5.608635in}{1.747763in}}%
\pgfpathlineto{\pgfqpoint{5.608849in}{2.305387in}}%
\pgfpathlineto{\pgfqpoint{5.609417in}{2.221468in}}%
\pgfpathlineto{\pgfqpoint{5.610128in}{2.283366in}}%
\pgfpathlineto{\pgfqpoint{5.610011in}{1.830176in}}%
\pgfpathlineto{\pgfqpoint{5.610475in}{2.184054in}}%
\pgfpathlineto{\pgfqpoint{5.610481in}{1.889406in}}%
\pgfpathlineto{\pgfqpoint{5.610556in}{2.327866in}}%
\pgfpathlineto{\pgfqpoint{5.611584in}{2.279688in}}%
\pgfpathlineto{\pgfqpoint{5.611766in}{1.686111in}}%
\pgfpathlineto{\pgfqpoint{5.612154in}{2.301730in}}%
\pgfpathlineto{\pgfqpoint{5.612696in}{2.116180in}}%
\pgfpathlineto{\pgfqpoint{5.613163in}{2.291714in}}%
\pgfpathlineto{\pgfqpoint{5.613624in}{1.809541in}}%
\pgfpathlineto{\pgfqpoint{5.613805in}{2.114242in}}%
\pgfpathlineto{\pgfqpoint{5.614239in}{1.815126in}}%
\pgfpathlineto{\pgfqpoint{5.614122in}{2.295281in}}%
\pgfpathlineto{\pgfqpoint{5.614598in}{2.162054in}}%
\pgfpathlineto{\pgfqpoint{5.615095in}{2.301336in}}%
\pgfpathlineto{\pgfqpoint{5.614762in}{1.857558in}}%
\pgfpathlineto{\pgfqpoint{5.615697in}{2.143194in}}%
\pgfpathlineto{\pgfqpoint{5.615839in}{1.615949in}}%
\pgfpathlineto{\pgfqpoint{5.616376in}{2.300352in}}%
\pgfpathlineto{\pgfqpoint{5.616808in}{1.835042in}}%
\pgfpathlineto{\pgfqpoint{5.617885in}{2.303970in}}%
\pgfpathlineto{\pgfqpoint{5.617922in}{2.138836in}}%
\pgfpathlineto{\pgfqpoint{5.618525in}{2.329115in}}%
\pgfpathlineto{\pgfqpoint{5.618252in}{1.747527in}}%
\pgfpathlineto{\pgfqpoint{5.618886in}{2.212051in}}%
\pgfpathlineto{\pgfqpoint{5.618891in}{1.766921in}}%
\pgfpathlineto{\pgfqpoint{5.619477in}{2.318007in}}%
\pgfpathlineto{\pgfqpoint{5.619994in}{2.235642in}}%
\pgfpathlineto{\pgfqpoint{5.620365in}{1.752164in}}%
\pgfpathlineto{\pgfqpoint{5.620995in}{2.284524in}}%
\pgfpathlineto{\pgfqpoint{5.621110in}{2.054585in}}%
\pgfpathlineto{\pgfqpoint{5.621776in}{2.288769in}}%
\pgfpathlineto{\pgfqpoint{5.622165in}{1.799449in}}%
\pgfpathlineto{\pgfqpoint{5.622223in}{2.239570in}}%
\pgfpathlineto{\pgfqpoint{5.622887in}{1.782178in}}%
\pgfpathlineto{\pgfqpoint{5.623203in}{2.317892in}}%
\pgfpathlineto{\pgfqpoint{5.623332in}{2.164070in}}%
\pgfpathlineto{\pgfqpoint{5.624284in}{2.321903in}}%
\pgfpathlineto{\pgfqpoint{5.624088in}{1.913599in}}%
\pgfpathlineto{\pgfqpoint{5.624362in}{2.273329in}}%
\pgfpathlineto{\pgfqpoint{5.624383in}{1.837887in}}%
\pgfpathlineto{\pgfqpoint{5.625059in}{2.297685in}}%
\pgfpathlineto{\pgfqpoint{5.625471in}{1.949370in}}%
\pgfpathlineto{\pgfqpoint{5.625657in}{2.289480in}}%
\pgfpathlineto{\pgfqpoint{5.626439in}{1.657358in}}%
\pgfpathlineto{\pgfqpoint{5.626583in}{2.146846in}}%
\pgfpathlineto{\pgfqpoint{5.626958in}{2.317460in}}%
\pgfpathlineto{\pgfqpoint{5.627189in}{1.875176in}}%
\pgfpathlineto{\pgfqpoint{5.627533in}{2.227031in}}%
\pgfpathlineto{\pgfqpoint{5.628501in}{1.692387in}}%
\pgfpathlineto{\pgfqpoint{5.627815in}{2.303110in}}%
\pgfpathlineto{\pgfqpoint{5.628644in}{2.095438in}}%
\pgfpathlineto{\pgfqpoint{5.628818in}{2.287562in}}%
\pgfpathlineto{\pgfqpoint{5.628752in}{1.651589in}}%
\pgfpathlineto{\pgfqpoint{5.629748in}{2.253192in}}%
\pgfpathlineto{\pgfqpoint{5.629779in}{1.672672in}}%
\pgfpathlineto{\pgfqpoint{5.630482in}{2.312469in}}%
\pgfpathlineto{\pgfqpoint{5.630859in}{2.143572in}}%
\pgfpathlineto{\pgfqpoint{5.631347in}{2.302488in}}%
\pgfpathlineto{\pgfqpoint{5.631067in}{1.719579in}}%
\pgfpathlineto{\pgfqpoint{5.631956in}{2.199524in}}%
\pgfpathlineto{\pgfqpoint{5.632007in}{1.597425in}}%
\pgfpathlineto{\pgfqpoint{5.632352in}{2.294323in}}%
\pgfpathlineto{\pgfqpoint{5.633066in}{2.211684in}}%
\pgfpathlineto{\pgfqpoint{5.633461in}{1.715594in}}%
\pgfpathlineto{\pgfqpoint{5.633279in}{2.299489in}}%
\pgfpathlineto{\pgfqpoint{5.634164in}{2.209666in}}%
\pgfpathlineto{\pgfqpoint{5.634703in}{2.326121in}}%
\pgfpathlineto{\pgfqpoint{5.635051in}{1.746912in}}%
\pgfpathlineto{\pgfqpoint{5.635228in}{2.140225in}}%
\pgfpathlineto{\pgfqpoint{5.635338in}{1.785432in}}%
\pgfpathlineto{\pgfqpoint{5.635681in}{2.303725in}}%
\pgfpathlineto{\pgfqpoint{5.636334in}{2.174842in}}%
\pgfpathlineto{\pgfqpoint{5.636786in}{2.303052in}}%
\pgfpathlineto{\pgfqpoint{5.636515in}{1.740042in}}%
\pgfpathlineto{\pgfqpoint{5.637448in}{2.218458in}}%
\pgfpathlineto{\pgfqpoint{5.637864in}{1.761099in}}%
\pgfpathlineto{\pgfqpoint{5.637503in}{2.304008in}}%
\pgfpathlineto{\pgfqpoint{5.638554in}{2.066931in}}%
\pgfpathlineto{\pgfqpoint{5.639103in}{2.306228in}}%
\pgfpathlineto{\pgfqpoint{5.639288in}{1.805280in}}%
\pgfpathlineto{\pgfqpoint{5.639667in}{2.147765in}}%
\pgfpathlineto{\pgfqpoint{5.639891in}{1.765571in}}%
\pgfpathlineto{\pgfqpoint{5.639751in}{2.319286in}}%
\pgfpathlineto{\pgfqpoint{5.640742in}{2.179966in}}%
\pgfpathlineto{\pgfqpoint{5.640777in}{2.324572in}}%
\pgfpathlineto{\pgfqpoint{5.641804in}{1.879478in}}%
\pgfpathlineto{\pgfqpoint{5.641839in}{2.219195in}}%
\pgfpathlineto{\pgfqpoint{5.642013in}{1.814141in}}%
\pgfpathlineto{\pgfqpoint{5.642577in}{2.321329in}}%
\pgfpathlineto{\pgfqpoint{5.642948in}{2.099719in}}%
\pgfpathlineto{\pgfqpoint{5.643709in}{2.309645in}}%
\pgfpathlineto{\pgfqpoint{5.643976in}{1.812546in}}%
\pgfpathlineto{\pgfqpoint{5.644045in}{2.267872in}}%
\pgfpathlineto{\pgfqpoint{5.644237in}{1.843312in}}%
\pgfpathlineto{\pgfqpoint{5.644868in}{2.324770in}}%
\pgfpathlineto{\pgfqpoint{5.645153in}{1.986181in}}%
\pgfpathlineto{\pgfqpoint{5.645768in}{2.293101in}}%
\pgfpathlineto{\pgfqpoint{5.646131in}{1.812617in}}%
\pgfpathlineto{\pgfqpoint{5.646269in}{2.232571in}}%
\pgfpathlineto{\pgfqpoint{5.646833in}{2.316126in}}%
\pgfpathlineto{\pgfqpoint{5.646813in}{1.747413in}}%
\pgfpathlineto{\pgfqpoint{5.647357in}{2.224748in}}%
\pgfpathlineto{\pgfqpoint{5.648281in}{1.781736in}}%
\pgfpathlineto{\pgfqpoint{5.648027in}{2.344046in}}%
\pgfpathlineto{\pgfqpoint{5.648466in}{2.015486in}}%
\pgfpathlineto{\pgfqpoint{5.649027in}{2.315120in}}%
\pgfpathlineto{\pgfqpoint{5.648857in}{1.733730in}}%
\pgfpathlineto{\pgfqpoint{5.649592in}{2.268316in}}%
\pgfpathlineto{\pgfqpoint{5.649680in}{1.839526in}}%
\pgfpathlineto{\pgfqpoint{5.650186in}{2.300429in}}%
\pgfpathlineto{\pgfqpoint{5.650721in}{2.153469in}}%
\pgfpathlineto{\pgfqpoint{5.651720in}{2.320286in}}%
\pgfpathlineto{\pgfqpoint{5.651574in}{1.662060in}}%
\pgfpathlineto{\pgfqpoint{5.651831in}{2.216110in}}%
\pgfpathlineto{\pgfqpoint{5.652189in}{1.853596in}}%
\pgfpathlineto{\pgfqpoint{5.652803in}{2.322944in}}%
\pgfpathlineto{\pgfqpoint{5.652939in}{2.158938in}}%
\pgfpathlineto{\pgfqpoint{5.653648in}{2.309088in}}%
\pgfpathlineto{\pgfqpoint{5.653947in}{1.801529in}}%
\pgfpathlineto{\pgfqpoint{5.654053in}{2.268039in}}%
\pgfpathlineto{\pgfqpoint{5.655160in}{1.817437in}}%
\pgfpathlineto{\pgfqpoint{5.654458in}{2.319720in}}%
\pgfpathlineto{\pgfqpoint{5.655169in}{2.065070in}}%
\pgfpathlineto{\pgfqpoint{5.655184in}{2.295721in}}%
\pgfpathlineto{\pgfqpoint{5.655318in}{1.791246in}}%
\pgfpathlineto{\pgfqpoint{5.656278in}{2.208244in}}%
\pgfpathlineto{\pgfqpoint{5.657255in}{1.705642in}}%
\pgfpathlineto{\pgfqpoint{5.656680in}{2.335691in}}%
\pgfpathlineto{\pgfqpoint{5.657388in}{2.148589in}}%
\pgfpathlineto{\pgfqpoint{5.657775in}{2.306352in}}%
\pgfpathlineto{\pgfqpoint{5.658000in}{1.839629in}}%
\pgfpathlineto{\pgfqpoint{5.658491in}{2.143188in}}%
\pgfpathlineto{\pgfqpoint{5.658525in}{1.715878in}}%
\pgfpathlineto{\pgfqpoint{5.659163in}{2.329589in}}%
\pgfpathlineto{\pgfqpoint{5.659596in}{2.216080in}}%
\pgfpathlineto{\pgfqpoint{5.660380in}{1.822801in}}%
\pgfpathlineto{\pgfqpoint{5.660223in}{2.310094in}}%
\pgfpathlineto{\pgfqpoint{5.660712in}{2.118818in}}%
\pgfpathlineto{\pgfqpoint{5.661143in}{2.316409in}}%
\pgfpathlineto{\pgfqpoint{5.661428in}{1.814329in}}%
\pgfpathlineto{\pgfqpoint{5.661797in}{2.253089in}}%
\pgfpathlineto{\pgfqpoint{5.662638in}{1.757693in}}%
\pgfpathlineto{\pgfqpoint{5.662652in}{2.307484in}}%
\pgfpathlineto{\pgfqpoint{5.662907in}{2.101260in}}%
\pgfpathlineto{\pgfqpoint{5.663468in}{2.312815in}}%
\pgfpathlineto{\pgfqpoint{5.663091in}{1.720634in}}%
\pgfpathlineto{\pgfqpoint{5.664019in}{2.286787in}}%
\pgfpathlineto{\pgfqpoint{5.664692in}{1.724190in}}%
\pgfpathlineto{\pgfqpoint{5.664532in}{2.333491in}}%
\pgfpathlineto{\pgfqpoint{5.665133in}{2.082741in}}%
\pgfpathlineto{\pgfqpoint{5.665480in}{2.299079in}}%
\pgfpathlineto{\pgfqpoint{5.665635in}{1.840119in}}%
\pgfpathlineto{\pgfqpoint{5.666244in}{2.109484in}}%
\pgfpathlineto{\pgfqpoint{5.666679in}{1.773383in}}%
\pgfpathlineto{\pgfqpoint{5.666328in}{2.320564in}}%
\pgfpathlineto{\pgfqpoint{5.667347in}{2.224168in}}%
\pgfpathlineto{\pgfqpoint{5.667534in}{2.303684in}}%
\pgfpathlineto{\pgfqpoint{5.667557in}{1.788346in}}%
\pgfpathlineto{\pgfqpoint{5.668439in}{2.154686in}}%
\pgfpathlineto{\pgfqpoint{5.668625in}{1.772467in}}%
\pgfpathlineto{\pgfqpoint{5.669002in}{2.320632in}}%
\pgfpathlineto{\pgfqpoint{5.669546in}{2.208345in}}%
\pgfpathlineto{\pgfqpoint{5.670390in}{1.644089in}}%
\pgfpathlineto{\pgfqpoint{5.670042in}{2.305614in}}%
\pgfpathlineto{\pgfqpoint{5.670654in}{2.062130in}}%
\pgfpathlineto{\pgfqpoint{5.670775in}{2.315989in}}%
\pgfpathlineto{\pgfqpoint{5.670835in}{1.696115in}}%
\pgfpathlineto{\pgfqpoint{5.671765in}{2.270483in}}%
\pgfpathlineto{\pgfqpoint{5.672310in}{1.836782in}}%
\pgfpathlineto{\pgfqpoint{5.672171in}{2.317795in}}%
\pgfpathlineto{\pgfqpoint{5.672882in}{2.123471in}}%
\pgfpathlineto{\pgfqpoint{5.672992in}{2.325886in}}%
\pgfpathlineto{\pgfqpoint{5.673871in}{1.816038in}}%
\pgfpathlineto{\pgfqpoint{5.673991in}{2.183371in}}%
\pgfpathlineto{\pgfqpoint{5.674248in}{1.815066in}}%
\pgfpathlineto{\pgfqpoint{5.674987in}{2.337512in}}%
\pgfpathlineto{\pgfqpoint{5.675084in}{2.229425in}}%
\pgfpathlineto{\pgfqpoint{5.676132in}{2.327988in}}%
\pgfpathlineto{\pgfqpoint{5.675798in}{1.560700in}}%
\pgfpathlineto{\pgfqpoint{5.676173in}{2.109394in}}%
\pgfpathlineto{\pgfqpoint{5.676182in}{1.526631in}}%
\pgfpathlineto{\pgfqpoint{5.677128in}{2.330355in}}%
\pgfpathlineto{\pgfqpoint{5.677279in}{2.176446in}}%
\pgfpathlineto{\pgfqpoint{5.677994in}{2.330141in}}%
\pgfpathlineto{\pgfqpoint{5.677411in}{1.737554in}}%
\pgfpathlineto{\pgfqpoint{5.678376in}{2.168497in}}%
\pgfpathlineto{\pgfqpoint{5.678908in}{1.800587in}}%
\pgfpathlineto{\pgfqpoint{5.679054in}{2.316515in}}%
\pgfpathlineto{\pgfqpoint{5.679490in}{1.988959in}}%
\pgfpathlineto{\pgfqpoint{5.680242in}{2.330275in}}%
\pgfpathlineto{\pgfqpoint{5.680297in}{1.684706in}}%
\pgfpathlineto{\pgfqpoint{5.680595in}{2.166065in}}%
\pgfpathlineto{\pgfqpoint{5.681229in}{1.851600in}}%
\pgfpathlineto{\pgfqpoint{5.681649in}{2.326923in}}%
\pgfpathlineto{\pgfqpoint{5.681703in}{2.026816in}}%
\pgfpathlineto{\pgfqpoint{5.682672in}{2.329888in}}%
\pgfpathlineto{\pgfqpoint{5.681779in}{1.814127in}}%
\pgfpathlineto{\pgfqpoint{5.682816in}{2.170982in}}%
\pgfpathlineto{\pgfqpoint{5.683842in}{1.607966in}}%
\pgfpathlineto{\pgfqpoint{5.682830in}{2.317132in}}%
\pgfpathlineto{\pgfqpoint{5.683922in}{2.262107in}}%
\pgfpathlineto{\pgfqpoint{5.684627in}{1.698989in}}%
\pgfpathlineto{\pgfqpoint{5.684344in}{2.341538in}}%
\pgfpathlineto{\pgfqpoint{5.685039in}{2.132355in}}%
\pgfpathlineto{\pgfqpoint{5.685084in}{2.304582in}}%
\pgfpathlineto{\pgfqpoint{5.686037in}{1.794140in}}%
\pgfpathlineto{\pgfqpoint{5.686153in}{2.256165in}}%
\pgfpathlineto{\pgfqpoint{5.686403in}{1.759862in}}%
\pgfpathlineto{\pgfqpoint{5.686845in}{2.323013in}}%
\pgfpathlineto{\pgfqpoint{5.687264in}{2.247503in}}%
\pgfpathlineto{\pgfqpoint{5.687722in}{1.806518in}}%
\pgfpathlineto{\pgfqpoint{5.687825in}{2.323362in}}%
\pgfpathlineto{\pgfqpoint{5.688380in}{2.031735in}}%
\pgfpathlineto{\pgfqpoint{5.688554in}{2.326930in}}%
\pgfpathlineto{\pgfqpoint{5.688816in}{1.752127in}}%
\pgfpathlineto{\pgfqpoint{5.689494in}{2.218978in}}%
\pgfpathlineto{\pgfqpoint{5.689937in}{1.876366in}}%
\pgfpathlineto{\pgfqpoint{5.690105in}{2.343756in}}%
\pgfpathlineto{\pgfqpoint{5.690610in}{2.070365in}}%
\pgfpathlineto{\pgfqpoint{5.690976in}{2.316750in}}%
\pgfpathlineto{\pgfqpoint{5.691166in}{1.833896in}}%
\pgfpathlineto{\pgfqpoint{5.691722in}{2.205876in}}%
\pgfpathlineto{\pgfqpoint{5.692462in}{1.499327in}}%
\pgfpathlineto{\pgfqpoint{5.692202in}{2.339139in}}%
\pgfpathlineto{\pgfqpoint{5.692832in}{2.133433in}}%
\pgfpathlineto{\pgfqpoint{5.693644in}{2.321833in}}%
\pgfpathlineto{\pgfqpoint{5.693284in}{1.819004in}}%
\pgfpathlineto{\pgfqpoint{5.693947in}{2.240145in}}%
\pgfpathlineto{\pgfqpoint{5.694863in}{2.305818in}}%
\pgfpathlineto{\pgfqpoint{5.694429in}{1.805577in}}%
\pgfpathlineto{\pgfqpoint{5.694985in}{2.265966in}}%
\pgfpathlineto{\pgfqpoint{5.695453in}{1.801122in}}%
\pgfpathlineto{\pgfqpoint{5.695523in}{2.331890in}}%
\pgfpathlineto{\pgfqpoint{5.696095in}{2.180529in}}%
\pgfpathlineto{\pgfqpoint{5.697041in}{2.322316in}}%
\pgfpathlineto{\pgfqpoint{5.696605in}{1.587307in}}%
\pgfpathlineto{\pgfqpoint{5.697206in}{2.259447in}}%
\pgfpathlineto{\pgfqpoint{5.697589in}{1.901571in}}%
\pgfpathlineto{\pgfqpoint{5.697971in}{2.314520in}}%
\pgfpathlineto{\pgfqpoint{5.698314in}{2.136064in}}%
\pgfpathlineto{\pgfqpoint{5.698492in}{2.366148in}}%
\pgfpathlineto{\pgfqpoint{5.698354in}{1.393478in}}%
\pgfpathlineto{\pgfqpoint{5.699424in}{2.300107in}}%
\pgfpathlineto{\pgfqpoint{5.699844in}{1.548158in}}%
\pgfpathlineto{\pgfqpoint{5.700211in}{2.374202in}}%
\pgfpathlineto{\pgfqpoint{5.700535in}{2.169793in}}%
\pgfpathlineto{\pgfqpoint{5.700738in}{2.294252in}}%
\pgfpathlineto{\pgfqpoint{5.701605in}{1.810961in}}%
\pgfpathlineto{\pgfqpoint{5.701644in}{2.231274in}}%
\pgfpathlineto{\pgfqpoint{5.702496in}{1.734653in}}%
\pgfpathlineto{\pgfqpoint{5.702315in}{2.352553in}}%
\pgfpathlineto{\pgfqpoint{5.702753in}{2.188178in}}%
\pgfpathlineto{\pgfqpoint{5.703680in}{2.341905in}}%
\pgfpathlineto{\pgfqpoint{5.703723in}{1.776450in}}%
\pgfpathlineto{\pgfqpoint{5.703860in}{2.270872in}}%
\pgfpathlineto{\pgfqpoint{5.703942in}{1.812302in}}%
\pgfpathlineto{\pgfqpoint{5.704502in}{2.332644in}}%
\pgfpathlineto{\pgfqpoint{5.704973in}{2.068600in}}%
\pgfpathlineto{\pgfqpoint{5.705314in}{2.350464in}}%
\pgfpathlineto{\pgfqpoint{5.705494in}{1.851267in}}%
\pgfpathlineto{\pgfqpoint{5.706091in}{2.196597in}}%
\pgfpathlineto{\pgfqpoint{5.706713in}{1.758720in}}%
\pgfpathlineto{\pgfqpoint{5.706342in}{2.351309in}}%
\pgfpathlineto{\pgfqpoint{5.707206in}{1.982424in}}%
\pgfpathlineto{\pgfqpoint{5.707737in}{2.342855in}}%
\pgfpathlineto{\pgfqpoint{5.707809in}{1.719413in}}%
\pgfpathlineto{\pgfqpoint{5.708318in}{2.228921in}}%
\pgfpathlineto{\pgfqpoint{5.708483in}{2.337571in}}%
\pgfpathlineto{\pgfqpoint{5.708369in}{1.514960in}}%
\pgfpathlineto{\pgfqpoint{5.709394in}{2.204643in}}%
\pgfpathlineto{\pgfqpoint{5.710289in}{1.661483in}}%
\pgfpathlineto{\pgfqpoint{5.710348in}{2.324577in}}%
\pgfpathlineto{\pgfqpoint{5.710500in}{2.099107in}}%
\pgfpathlineto{\pgfqpoint{5.710871in}{2.344672in}}%
\pgfpathlineto{\pgfqpoint{5.710977in}{1.827392in}}%
\pgfpathlineto{\pgfqpoint{5.711612in}{2.184968in}}%
\pgfpathlineto{\pgfqpoint{5.711789in}{1.753395in}}%
\pgfpathlineto{\pgfqpoint{5.712604in}{2.344971in}}%
\pgfpathlineto{\pgfqpoint{5.712726in}{2.059512in}}%
\pgfpathlineto{\pgfqpoint{5.712818in}{2.331585in}}%
\pgfpathlineto{\pgfqpoint{5.712935in}{1.769305in}}%
\pgfpathlineto{\pgfqpoint{5.713832in}{2.284195in}}%
\pgfpathlineto{\pgfqpoint{5.714739in}{2.334609in}}%
\pgfpathlineto{\pgfqpoint{5.714948in}{1.781539in}}%
\pgfpathlineto{\pgfqpoint{5.716040in}{2.335471in}}%
\pgfpathlineto{\pgfqpoint{5.716061in}{2.159634in}}%
\pgfpathlineto{\pgfqpoint{5.716917in}{1.539134in}}%
\pgfpathlineto{\pgfqpoint{5.717021in}{2.331665in}}%
\pgfpathlineto{\pgfqpoint{5.717167in}{2.092057in}}%
\pgfpathlineto{\pgfqpoint{5.717606in}{2.356368in}}%
\pgfpathlineto{\pgfqpoint{5.717839in}{1.816919in}}%
\pgfpathlineto{\pgfqpoint{5.718274in}{2.128858in}}%
\pgfpathlineto{\pgfqpoint{5.719196in}{1.821665in}}%
\pgfpathlineto{\pgfqpoint{5.718390in}{2.351861in}}%
\pgfpathlineto{\pgfqpoint{5.719378in}{2.205528in}}%
\pgfpathlineto{\pgfqpoint{5.719861in}{2.342419in}}%
\pgfpathlineto{\pgfqpoint{5.719935in}{1.862318in}}%
\pgfpathlineto{\pgfqpoint{5.720311in}{2.126843in}}%
\pgfpathlineto{\pgfqpoint{5.721105in}{1.787123in}}%
\pgfpathlineto{\pgfqpoint{5.720883in}{2.338175in}}%
\pgfpathlineto{\pgfqpoint{5.721418in}{2.118572in}}%
\pgfpathlineto{\pgfqpoint{5.721475in}{2.355232in}}%
\pgfpathlineto{\pgfqpoint{5.721701in}{1.716824in}}%
\pgfpathlineto{\pgfqpoint{5.722530in}{2.189194in}}%
\pgfpathlineto{\pgfqpoint{5.723006in}{1.703552in}}%
\pgfpathlineto{\pgfqpoint{5.723116in}{2.310063in}}%
\pgfpathlineto{\pgfqpoint{5.723636in}{2.142177in}}%
\pgfpathlineto{\pgfqpoint{5.724147in}{2.321930in}}%
\pgfpathlineto{\pgfqpoint{5.724632in}{1.788124in}}%
\pgfpathlineto{\pgfqpoint{5.724743in}{2.104528in}}%
\pgfpathlineto{\pgfqpoint{5.725163in}{1.858793in}}%
\pgfpathlineto{\pgfqpoint{5.724828in}{2.335443in}}%
\pgfpathlineto{\pgfqpoint{5.725838in}{2.170345in}}%
\pgfpathlineto{\pgfqpoint{5.725875in}{2.360012in}}%
\pgfpathlineto{\pgfqpoint{5.726590in}{1.832944in}}%
\pgfpathlineto{\pgfqpoint{5.726943in}{2.147638in}}%
\pgfpathlineto{\pgfqpoint{5.727920in}{1.775265in}}%
\pgfpathlineto{\pgfqpoint{5.727754in}{2.340547in}}%
\pgfpathlineto{\pgfqpoint{5.728054in}{2.158000in}}%
\pgfpathlineto{\pgfqpoint{5.728523in}{1.683014in}}%
\pgfpathlineto{\pgfqpoint{5.728082in}{2.354444in}}%
\pgfpathlineto{\pgfqpoint{5.729117in}{2.159093in}}%
\pgfpathlineto{\pgfqpoint{5.729177in}{2.358323in}}%
\pgfpathlineto{\pgfqpoint{5.729561in}{1.761063in}}%
\pgfpathlineto{\pgfqpoint{5.730229in}{2.226670in}}%
\pgfpathlineto{\pgfqpoint{5.730282in}{2.354328in}}%
\pgfpathlineto{\pgfqpoint{5.730933in}{1.319511in}}%
\pgfpathlineto{\pgfqpoint{5.731307in}{2.070768in}}%
\pgfpathlineto{\pgfqpoint{5.731311in}{1.762149in}}%
\pgfpathlineto{\pgfqpoint{5.731684in}{2.364705in}}%
\pgfpathlineto{\pgfqpoint{5.732414in}{2.081011in}}%
\pgfpathlineto{\pgfqpoint{5.733234in}{2.339067in}}%
\pgfpathlineto{\pgfqpoint{5.732422in}{1.743962in}}%
\pgfpathlineto{\pgfqpoint{5.733526in}{2.253723in}}%
\pgfpathlineto{\pgfqpoint{5.734145in}{1.670281in}}%
\pgfpathlineto{\pgfqpoint{5.733957in}{2.349424in}}%
\pgfpathlineto{\pgfqpoint{5.734635in}{2.325712in}}%
\pgfpathlineto{\pgfqpoint{5.734926in}{1.633892in}}%
\pgfpathlineto{\pgfqpoint{5.735328in}{2.333967in}}%
\pgfpathlineto{\pgfqpoint{5.735750in}{2.114544in}}%
\pgfpathlineto{\pgfqpoint{5.735952in}{2.348741in}}%
\pgfpathlineto{\pgfqpoint{5.736333in}{1.696885in}}%
\pgfpathlineto{\pgfqpoint{5.736861in}{2.211981in}}%
\pgfpathlineto{\pgfqpoint{5.737807in}{1.778218in}}%
\pgfpathlineto{\pgfqpoint{5.737372in}{2.326309in}}%
\pgfpathlineto{\pgfqpoint{5.737969in}{2.224099in}}%
\pgfpathlineto{\pgfqpoint{5.738041in}{1.729332in}}%
\pgfpathlineto{\pgfqpoint{5.738495in}{2.347951in}}%
\pgfpathlineto{\pgfqpoint{5.739083in}{2.188109in}}%
\pgfpathlineto{\pgfqpoint{5.739666in}{2.362860in}}%
\pgfpathlineto{\pgfqpoint{5.739993in}{1.875149in}}%
\pgfpathlineto{\pgfqpoint{5.740189in}{2.292573in}}%
\pgfpathlineto{\pgfqpoint{5.740720in}{1.766402in}}%
\pgfpathlineto{\pgfqpoint{5.741203in}{2.350883in}}%
\pgfpathlineto{\pgfqpoint{5.741305in}{2.037869in}}%
\pgfpathlineto{\pgfqpoint{5.741787in}{2.343028in}}%
\pgfpathlineto{\pgfqpoint{5.741579in}{1.789012in}}%
\pgfpathlineto{\pgfqpoint{5.742417in}{2.130054in}}%
\pgfpathlineto{\pgfqpoint{5.742453in}{2.356244in}}%
\pgfpathlineto{\pgfqpoint{5.742636in}{1.729231in}}%
\pgfpathlineto{\pgfqpoint{5.743480in}{2.184120in}}%
\pgfpathlineto{\pgfqpoint{5.743543in}{2.348237in}}%
\pgfpathlineto{\pgfqpoint{5.744587in}{1.803467in}}%
\pgfpathlineto{\pgfqpoint{5.744688in}{2.364902in}}%
\pgfpathlineto{\pgfqpoint{5.745699in}{2.241845in}}%
\pgfpathlineto{\pgfqpoint{5.746095in}{1.824073in}}%
\pgfpathlineto{\pgfqpoint{5.745967in}{2.330209in}}%
\pgfpathlineto{\pgfqpoint{5.746800in}{2.057288in}}%
\pgfpathlineto{\pgfqpoint{5.747388in}{2.383573in}}%
\pgfpathlineto{\pgfqpoint{5.747744in}{1.462316in}}%
\pgfpathlineto{\pgfqpoint{5.747914in}{2.220901in}}%
\pgfpathlineto{\pgfqpoint{5.748022in}{2.360250in}}%
\pgfpathlineto{\pgfqpoint{5.748334in}{1.626189in}}%
\pgfpathlineto{\pgfqpoint{5.749005in}{2.236376in}}%
\pgfpathlineto{\pgfqpoint{5.749128in}{1.657293in}}%
\pgfpathlineto{\pgfqpoint{5.749763in}{2.351694in}}%
\pgfpathlineto{\pgfqpoint{5.750113in}{2.168768in}}%
\pgfpathlineto{\pgfqpoint{5.750428in}{2.352675in}}%
\pgfpathlineto{\pgfqpoint{5.750616in}{1.897087in}}%
\pgfpathlineto{\pgfqpoint{5.751226in}{2.255768in}}%
\pgfpathlineto{\pgfqpoint{5.751352in}{1.865817in}}%
\pgfpathlineto{\pgfqpoint{5.751800in}{2.351578in}}%
\pgfpathlineto{\pgfqpoint{5.752343in}{1.970942in}}%
\pgfpathlineto{\pgfqpoint{5.752431in}{2.353683in}}%
\pgfpathlineto{\pgfqpoint{5.752981in}{1.387629in}}%
\pgfpathlineto{\pgfqpoint{5.753454in}{2.193126in}}%
\pgfpathlineto{\pgfqpoint{5.754413in}{1.796021in}}%
\pgfpathlineto{\pgfqpoint{5.753865in}{2.348739in}}%
\pgfpathlineto{\pgfqpoint{5.754562in}{2.154704in}}%
\pgfpathlineto{\pgfqpoint{5.754683in}{2.388584in}}%
\pgfpathlineto{\pgfqpoint{5.755503in}{1.775785in}}%
\pgfpathlineto{\pgfqpoint{5.755670in}{2.241261in}}%
\pgfpathlineto{\pgfqpoint{5.756549in}{1.667483in}}%
\pgfpathlineto{\pgfqpoint{5.755708in}{2.362183in}}%
\pgfpathlineto{\pgfqpoint{5.756784in}{2.049023in}}%
\pgfpathlineto{\pgfqpoint{5.757018in}{2.370847in}}%
\pgfpathlineto{\pgfqpoint{5.756882in}{1.777073in}}%
\pgfpathlineto{\pgfqpoint{5.757902in}{2.250568in}}%
\pgfpathlineto{\pgfqpoint{5.758671in}{1.775855in}}%
\pgfpathlineto{\pgfqpoint{5.758735in}{2.360251in}}%
\pgfpathlineto{\pgfqpoint{5.759013in}{2.240650in}}%
\pgfpathlineto{\pgfqpoint{5.759058in}{1.609327in}}%
\pgfpathlineto{\pgfqpoint{5.759851in}{2.364281in}}%
\pgfpathlineto{\pgfqpoint{5.760125in}{2.196262in}}%
\pgfpathlineto{\pgfqpoint{5.760489in}{1.704705in}}%
\pgfpathlineto{\pgfqpoint{5.761062in}{2.335995in}}%
\pgfpathlineto{\pgfqpoint{5.761197in}{2.186884in}}%
\pgfpathlineto{\pgfqpoint{5.761788in}{2.351813in}}%
\pgfpathlineto{\pgfqpoint{5.761795in}{1.813689in}}%
\pgfpathlineto{\pgfqpoint{5.762303in}{2.220263in}}%
\pgfpathlineto{\pgfqpoint{5.763031in}{1.841143in}}%
\pgfpathlineto{\pgfqpoint{5.763310in}{2.354382in}}%
\pgfpathlineto{\pgfqpoint{5.763411in}{2.168552in}}%
\pgfpathlineto{\pgfqpoint{5.763935in}{2.358682in}}%
\pgfpathlineto{\pgfqpoint{5.763798in}{1.820842in}}%
\pgfpathlineto{\pgfqpoint{5.764519in}{2.260194in}}%
\pgfpathlineto{\pgfqpoint{5.764868in}{1.875442in}}%
\pgfpathlineto{\pgfqpoint{5.764764in}{2.376120in}}%
\pgfpathlineto{\pgfqpoint{5.765632in}{2.116340in}}%
\pgfpathlineto{\pgfqpoint{5.766357in}{2.356010in}}%
\pgfpathlineto{\pgfqpoint{5.766098in}{1.821167in}}%
\pgfpathlineto{\pgfqpoint{5.766741in}{2.184103in}}%
\pgfpathlineto{\pgfqpoint{5.766871in}{1.780220in}}%
\pgfpathlineto{\pgfqpoint{5.767376in}{2.348724in}}%
\pgfpathlineto{\pgfqpoint{5.767841in}{2.192757in}}%
\pgfpathlineto{\pgfqpoint{5.768776in}{2.356624in}}%
\pgfpathlineto{\pgfqpoint{5.768202in}{1.726508in}}%
\pgfpathlineto{\pgfqpoint{5.768945in}{2.304120in}}%
\pgfpathlineto{\pgfqpoint{5.768948in}{1.835582in}}%
\pgfpathlineto{\pgfqpoint{5.770013in}{2.362904in}}%
\pgfpathlineto{\pgfqpoint{5.770057in}{2.031450in}}%
\pgfpathlineto{\pgfqpoint{5.771133in}{2.361062in}}%
\pgfpathlineto{\pgfqpoint{5.770376in}{1.886178in}}%
\pgfpathlineto{\pgfqpoint{5.771170in}{2.197571in}}%
\pgfpathlineto{\pgfqpoint{5.771791in}{2.351027in}}%
\pgfpathlineto{\pgfqpoint{5.771374in}{1.772755in}}%
\pgfpathlineto{\pgfqpoint{5.772276in}{2.228396in}}%
\pgfpathlineto{\pgfqpoint{5.772455in}{1.784096in}}%
\pgfpathlineto{\pgfqpoint{5.772338in}{2.384015in}}%
\pgfpathlineto{\pgfqpoint{5.773387in}{2.018871in}}%
\pgfpathlineto{\pgfqpoint{5.773561in}{2.374281in}}%
\pgfpathlineto{\pgfqpoint{5.774400in}{1.768177in}}%
\pgfpathlineto{\pgfqpoint{5.774498in}{2.270566in}}%
\pgfpathlineto{\pgfqpoint{5.775165in}{1.765893in}}%
\pgfpathlineto{\pgfqpoint{5.774759in}{2.375836in}}%
\pgfpathlineto{\pgfqpoint{5.775610in}{2.211401in}}%
\pgfpathlineto{\pgfqpoint{5.776142in}{2.350912in}}%
\pgfpathlineto{\pgfqpoint{5.776059in}{1.919626in}}%
\pgfpathlineto{\pgfqpoint{5.776232in}{2.350220in}}%
\pgfpathlineto{\pgfqpoint{5.776236in}{1.561338in}}%
\pgfpathlineto{\pgfqpoint{5.776301in}{2.356258in}}%
\pgfpathlineto{\pgfqpoint{5.777343in}{2.261568in}}%
\pgfpathlineto{\pgfqpoint{5.778398in}{1.810224in}}%
\pgfpathlineto{\pgfqpoint{5.778268in}{2.357809in}}%
\pgfpathlineto{\pgfqpoint{5.778455in}{2.144145in}}%
\pgfpathlineto{\pgfqpoint{5.779051in}{2.352720in}}%
\pgfpathlineto{\pgfqpoint{5.779188in}{1.759204in}}%
\pgfpathlineto{\pgfqpoint{5.779568in}{2.225371in}}%
\pgfpathlineto{\pgfqpoint{5.780359in}{1.843949in}}%
\pgfpathlineto{\pgfqpoint{5.780259in}{2.364552in}}%
\pgfpathlineto{\pgfqpoint{5.780678in}{2.292194in}}%
\pgfpathlineto{\pgfqpoint{5.781331in}{1.823274in}}%
\pgfpathlineto{\pgfqpoint{5.781314in}{2.372617in}}%
\pgfpathlineto{\pgfqpoint{5.781788in}{1.931898in}}%
\pgfpathlineto{\pgfqpoint{5.781792in}{2.366231in}}%
\pgfpathlineto{\pgfqpoint{5.782476in}{1.529066in}}%
\pgfpathlineto{\pgfqpoint{5.782899in}{2.049219in}}%
\pgfpathlineto{\pgfqpoint{5.783851in}{2.351329in}}%
\pgfpathlineto{\pgfqpoint{5.783258in}{1.856656in}}%
\pgfpathlineto{\pgfqpoint{5.784004in}{2.256327in}}%
\pgfpathlineto{\pgfqpoint{5.784861in}{1.777694in}}%
\pgfpathlineto{\pgfqpoint{5.784698in}{2.363743in}}%
\pgfpathlineto{\pgfqpoint{5.785112in}{2.283505in}}%
\pgfpathlineto{\pgfqpoint{5.786017in}{1.736190in}}%
\pgfpathlineto{\pgfqpoint{5.785554in}{2.367012in}}%
\pgfpathlineto{\pgfqpoint{5.786229in}{2.134407in}}%
\pgfpathlineto{\pgfqpoint{5.786878in}{2.353438in}}%
\pgfpathlineto{\pgfqpoint{5.787223in}{1.844450in}}%
\pgfpathlineto{\pgfqpoint{5.787339in}{2.313229in}}%
\pgfpathlineto{\pgfqpoint{5.787585in}{1.833720in}}%
\pgfpathlineto{\pgfqpoint{5.787441in}{2.375850in}}%
\pgfpathlineto{\pgfqpoint{5.788457in}{2.111921in}}%
\pgfpathlineto{\pgfqpoint{5.788579in}{2.368690in}}%
\pgfpathlineto{\pgfqpoint{5.788502in}{1.739222in}}%
\pgfpathlineto{\pgfqpoint{5.789571in}{2.283845in}}%
\pgfpathlineto{\pgfqpoint{5.790453in}{1.698385in}}%
\pgfpathlineto{\pgfqpoint{5.789666in}{2.382200in}}%
\pgfpathlineto{\pgfqpoint{5.790683in}{2.047348in}}%
\pgfpathlineto{\pgfqpoint{5.791691in}{2.365562in}}%
\pgfpathlineto{\pgfqpoint{5.791618in}{1.645168in}}%
\pgfpathlineto{\pgfqpoint{5.791792in}{2.118201in}}%
\pgfpathlineto{\pgfqpoint{5.792043in}{1.663732in}}%
\pgfpathlineto{\pgfqpoint{5.792366in}{2.364600in}}%
\pgfpathlineto{\pgfqpoint{5.792901in}{2.122003in}}%
\pgfpathlineto{\pgfqpoint{5.793054in}{1.847081in}}%
\pgfpathlineto{\pgfqpoint{5.793942in}{2.373307in}}%
\pgfpathlineto{\pgfqpoint{5.793987in}{2.252152in}}%
\pgfpathlineto{\pgfqpoint{5.794382in}{2.356367in}}%
\pgfpathlineto{\pgfqpoint{5.794001in}{1.942048in}}%
\pgfpathlineto{\pgfqpoint{5.795053in}{2.205647in}}%
\pgfpathlineto{\pgfqpoint{5.795226in}{1.754800in}}%
\pgfpathlineto{\pgfqpoint{5.796106in}{2.372750in}}%
\pgfpathlineto{\pgfqpoint{5.796161in}{2.290050in}}%
\pgfpathlineto{\pgfqpoint{5.796536in}{1.730264in}}%
\pgfpathlineto{\pgfqpoint{5.797180in}{2.377838in}}%
\pgfpathlineto{\pgfqpoint{5.797273in}{2.153948in}}%
\pgfpathlineto{\pgfqpoint{5.797943in}{2.344181in}}%
\pgfpathlineto{\pgfqpoint{5.798066in}{1.717109in}}%
\pgfpathlineto{\pgfqpoint{5.798382in}{2.194793in}}%
\pgfpathlineto{\pgfqpoint{5.799300in}{2.363229in}}%
\pgfpathlineto{\pgfqpoint{5.799378in}{1.882249in}}%
\pgfpathlineto{\pgfqpoint{5.799495in}{2.269720in}}%
\pgfpathlineto{\pgfqpoint{5.799980in}{1.913111in}}%
\pgfpathlineto{\pgfqpoint{5.800106in}{2.360701in}}%
\pgfpathlineto{\pgfqpoint{5.800605in}{2.141088in}}%
\pgfpathlineto{\pgfqpoint{5.800942in}{2.353719in}}%
\pgfpathlineto{\pgfqpoint{5.801470in}{1.711042in}}%
\pgfpathlineto{\pgfqpoint{5.801715in}{2.286091in}}%
\pgfpathlineto{\pgfqpoint{5.802358in}{1.646261in}}%
\pgfpathlineto{\pgfqpoint{5.802116in}{2.373049in}}%
\pgfpathlineto{\pgfqpoint{5.802826in}{2.122686in}}%
\pgfpathlineto{\pgfqpoint{5.803277in}{2.355021in}}%
\pgfpathlineto{\pgfqpoint{5.802942in}{1.701018in}}%
\pgfpathlineto{\pgfqpoint{5.803934in}{2.182407in}}%
\pgfpathlineto{\pgfqpoint{5.804330in}{1.832926in}}%
\pgfpathlineto{\pgfqpoint{5.804813in}{2.370161in}}%
\pgfpathlineto{\pgfqpoint{5.805043in}{2.201281in}}%
\pgfpathlineto{\pgfqpoint{5.805616in}{1.698827in}}%
\pgfpathlineto{\pgfqpoint{5.805960in}{2.397779in}}%
\pgfpathlineto{\pgfqpoint{5.806128in}{2.182325in}}%
\pgfpathlineto{\pgfqpoint{5.806263in}{2.361055in}}%
\pgfpathlineto{\pgfqpoint{5.807147in}{1.669041in}}%
\pgfpathlineto{\pgfqpoint{5.807238in}{2.224287in}}%
\pgfpathlineto{\pgfqpoint{5.807409in}{1.677414in}}%
\pgfpathlineto{\pgfqpoint{5.807684in}{2.371172in}}%
\pgfpathlineto{\pgfqpoint{5.808191in}{2.258927in}}%
\pgfpathlineto{\pgfqpoint{5.809175in}{2.358150in}}%
\pgfpathlineto{\pgfqpoint{5.809054in}{1.785448in}}%
\pgfpathlineto{\pgfqpoint{5.809302in}{2.291703in}}%
\pgfpathlineto{\pgfqpoint{5.809913in}{1.861020in}}%
\pgfpathlineto{\pgfqpoint{5.810060in}{2.361067in}}%
\pgfpathlineto{\pgfqpoint{5.810416in}{2.201373in}}%
\pgfpathlineto{\pgfqpoint{5.810799in}{2.379127in}}%
\pgfpathlineto{\pgfqpoint{5.810743in}{1.625874in}}%
\pgfpathlineto{\pgfqpoint{5.811525in}{2.224617in}}%
\pgfpathlineto{\pgfqpoint{5.812312in}{1.786338in}}%
\pgfpathlineto{\pgfqpoint{5.812252in}{2.366995in}}%
\pgfpathlineto{\pgfqpoint{5.812634in}{2.151531in}}%
\pgfpathlineto{\pgfqpoint{5.813723in}{2.369643in}}%
\pgfpathlineto{\pgfqpoint{5.813551in}{1.670839in}}%
\pgfpathlineto{\pgfqpoint{5.813743in}{2.115053in}}%
\pgfpathlineto{\pgfqpoint{5.813809in}{2.397933in}}%
\pgfpathlineto{\pgfqpoint{5.813984in}{1.652844in}}%
\pgfpathlineto{\pgfqpoint{5.814859in}{2.251900in}}%
\pgfpathlineto{\pgfqpoint{5.814935in}{1.675496in}}%
\pgfpathlineto{\pgfqpoint{5.815406in}{2.367503in}}%
\pgfpathlineto{\pgfqpoint{5.815966in}{1.950186in}}%
\pgfpathlineto{\pgfqpoint{5.816269in}{2.366994in}}%
\pgfpathlineto{\pgfqpoint{5.816351in}{1.782847in}}%
\pgfpathlineto{\pgfqpoint{5.817080in}{2.267933in}}%
\pgfpathlineto{\pgfqpoint{5.817404in}{1.858994in}}%
\pgfpathlineto{\pgfqpoint{5.817172in}{2.390775in}}%
\pgfpathlineto{\pgfqpoint{5.818187in}{2.087443in}}%
\pgfpathlineto{\pgfqpoint{5.818750in}{2.365142in}}%
\pgfpathlineto{\pgfqpoint{5.819096in}{1.790763in}}%
\pgfpathlineto{\pgfqpoint{5.819299in}{2.262136in}}%
\pgfpathlineto{\pgfqpoint{5.819439in}{1.815816in}}%
\pgfpathlineto{\pgfqpoint{5.820006in}{2.383472in}}%
\pgfpathlineto{\pgfqpoint{5.820407in}{2.213065in}}%
\pgfpathlineto{\pgfqpoint{5.820576in}{2.390409in}}%
\pgfpathlineto{\pgfqpoint{5.820664in}{1.846092in}}%
\pgfpathlineto{\pgfqpoint{5.821515in}{2.225782in}}%
\pgfpathlineto{\pgfqpoint{5.821613in}{1.818091in}}%
\pgfpathlineto{\pgfqpoint{5.822563in}{2.381414in}}%
\pgfpathlineto{\pgfqpoint{5.822624in}{2.266730in}}%
\pgfpathlineto{\pgfqpoint{5.822628in}{2.370224in}}%
\pgfpathlineto{\pgfqpoint{5.822722in}{1.561955in}}%
\pgfpathlineto{\pgfqpoint{5.823734in}{2.263205in}}%
\pgfpathlineto{\pgfqpoint{5.824824in}{1.938636in}}%
\pgfpathlineto{\pgfqpoint{5.824257in}{2.385307in}}%
\pgfpathlineto{\pgfqpoint{5.824840in}{2.283240in}}%
\pgfpathlineto{\pgfqpoint{5.825211in}{2.368311in}}%
\pgfpathlineto{\pgfqpoint{5.825275in}{1.621554in}}%
\pgfpathlineto{\pgfqpoint{5.825931in}{2.246176in}}%
\pgfpathlineto{\pgfqpoint{5.826859in}{1.751035in}}%
\pgfpathlineto{\pgfqpoint{5.826060in}{2.379537in}}%
\pgfpathlineto{\pgfqpoint{5.827041in}{2.168915in}}%
\pgfpathlineto{\pgfqpoint{5.827048in}{2.373076in}}%
\pgfpathlineto{\pgfqpoint{5.827339in}{1.643006in}}%
\pgfpathlineto{\pgfqpoint{5.828162in}{2.267956in}}%
\pgfpathlineto{\pgfqpoint{5.828753in}{1.805039in}}%
\pgfpathlineto{\pgfqpoint{5.828702in}{2.367785in}}%
\pgfpathlineto{\pgfqpoint{5.829132in}{2.303367in}}%
\pgfpathlineto{\pgfqpoint{5.830024in}{2.365187in}}%
\pgfpathlineto{\pgfqpoint{5.829426in}{1.877341in}}%
\pgfpathlineto{\pgfqpoint{5.830225in}{2.342571in}}%
\pgfpathlineto{\pgfqpoint{5.831181in}{1.847175in}}%
\pgfpathlineto{\pgfqpoint{5.830571in}{2.416637in}}%
\pgfpathlineto{\pgfqpoint{5.831337in}{2.117742in}}%
\pgfpathlineto{\pgfqpoint{5.832091in}{2.384407in}}%
\pgfpathlineto{\pgfqpoint{5.832145in}{1.710045in}}%
\pgfpathlineto{\pgfqpoint{5.832446in}{2.206493in}}%
\pgfpathlineto{\pgfqpoint{5.833539in}{1.557825in}}%
\pgfpathlineto{\pgfqpoint{5.832743in}{2.401605in}}%
\pgfpathlineto{\pgfqpoint{5.833555in}{2.078009in}}%
\pgfpathlineto{\pgfqpoint{5.833628in}{2.388318in}}%
\pgfpathlineto{\pgfqpoint{5.833792in}{1.769235in}}%
\pgfpathlineto{\pgfqpoint{5.834668in}{2.186152in}}%
\pgfpathlineto{\pgfqpoint{5.835388in}{2.379380in}}%
\pgfpathlineto{\pgfqpoint{5.835749in}{1.899185in}}%
\pgfpathlineto{\pgfqpoint{5.835774in}{2.219150in}}%
\pgfpathlineto{\pgfqpoint{5.835872in}{1.648845in}}%
\pgfpathlineto{\pgfqpoint{5.836565in}{2.387048in}}%
\pgfpathlineto{\pgfqpoint{5.836884in}{2.180546in}}%
\pgfpathlineto{\pgfqpoint{5.837885in}{1.919126in}}%
\pgfpathlineto{\pgfqpoint{5.837669in}{2.397721in}}%
\pgfpathlineto{\pgfqpoint{5.837991in}{2.187039in}}%
\pgfpathlineto{\pgfqpoint{5.838753in}{2.395091in}}%
\pgfpathlineto{\pgfqpoint{5.838946in}{1.562092in}}%
\pgfpathlineto{\pgfqpoint{5.839099in}{2.159841in}}%
\pgfpathlineto{\pgfqpoint{5.839569in}{1.856761in}}%
\pgfpathlineto{\pgfqpoint{5.839167in}{2.386662in}}%
\pgfpathlineto{\pgfqpoint{5.840209in}{2.152672in}}%
\pgfpathlineto{\pgfqpoint{5.841280in}{1.839719in}}%
\pgfpathlineto{\pgfqpoint{5.840976in}{2.421481in}}%
\pgfpathlineto{\pgfqpoint{5.841317in}{2.210425in}}%
\pgfpathlineto{\pgfqpoint{5.842372in}{1.737366in}}%
\pgfpathlineto{\pgfqpoint{5.841844in}{2.374507in}}%
\pgfpathlineto{\pgfqpoint{5.842385in}{2.188512in}}%
\pgfpathlineto{\pgfqpoint{5.843067in}{2.404325in}}%
\pgfpathlineto{\pgfqpoint{5.842416in}{1.786279in}}%
\pgfpathlineto{\pgfqpoint{5.843496in}{2.202194in}}%
\pgfpathlineto{\pgfqpoint{5.844263in}{2.390205in}}%
\pgfpathlineto{\pgfqpoint{5.843697in}{1.814972in}}%
\pgfpathlineto{\pgfqpoint{5.844583in}{2.219592in}}%
\pgfpathlineto{\pgfqpoint{5.844940in}{1.877889in}}%
\pgfpathlineto{\pgfqpoint{5.845007in}{2.378539in}}%
\pgfpathlineto{\pgfqpoint{5.845692in}{2.273429in}}%
\pgfpathlineto{\pgfqpoint{5.845793in}{1.944975in}}%
\pgfpathlineto{\pgfqpoint{5.846733in}{2.389520in}}%
\pgfpathlineto{\pgfqpoint{5.846804in}{2.245803in}}%
\pgfpathlineto{\pgfqpoint{5.847464in}{2.382480in}}%
\pgfpathlineto{\pgfqpoint{5.847103in}{1.844910in}}%
\pgfpathlineto{\pgfqpoint{5.847910in}{2.282294in}}%
\pgfpathlineto{\pgfqpoint{5.848806in}{1.856771in}}%
\pgfpathlineto{\pgfqpoint{5.848501in}{2.395513in}}%
\pgfpathlineto{\pgfqpoint{5.849019in}{2.049060in}}%
\pgfpathlineto{\pgfqpoint{5.849201in}{2.379667in}}%
\pgfpathlineto{\pgfqpoint{5.850098in}{1.906898in}}%
\pgfpathlineto{\pgfqpoint{5.850131in}{2.376789in}}%
\pgfpathlineto{\pgfqpoint{5.850259in}{1.823165in}}%
\pgfpathlineto{\pgfqpoint{5.850983in}{2.391660in}}%
\pgfpathlineto{\pgfqpoint{5.851244in}{2.202220in}}%
\pgfpathlineto{\pgfqpoint{5.851710in}{2.381570in}}%
\pgfpathlineto{\pgfqpoint{5.852269in}{1.855275in}}%
\pgfpathlineto{\pgfqpoint{5.852347in}{2.180635in}}%
\pgfpathlineto{\pgfqpoint{5.853282in}{1.750339in}}%
\pgfpathlineto{\pgfqpoint{5.853144in}{2.383474in}}%
\pgfpathlineto{\pgfqpoint{5.853454in}{2.216966in}}%
\pgfpathlineto{\pgfqpoint{5.854275in}{2.416963in}}%
\pgfpathlineto{\pgfqpoint{5.853866in}{1.817471in}}%
\pgfpathlineto{\pgfqpoint{5.854546in}{2.210280in}}%
\pgfpathlineto{\pgfqpoint{5.854675in}{1.597017in}}%
\pgfpathlineto{\pgfqpoint{5.854564in}{2.385869in}}%
\pgfpathlineto{\pgfqpoint{5.855656in}{2.292526in}}%
\pgfpathlineto{\pgfqpoint{5.856610in}{2.366242in}}%
\pgfpathlineto{\pgfqpoint{5.856497in}{1.721719in}}%
\pgfpathlineto{\pgfqpoint{5.856745in}{2.328351in}}%
\pgfpathlineto{\pgfqpoint{5.857754in}{1.845977in}}%
\pgfpathlineto{\pgfqpoint{5.857494in}{2.416610in}}%
\pgfpathlineto{\pgfqpoint{5.857855in}{2.224637in}}%
\pgfpathlineto{\pgfqpoint{5.858751in}{2.400390in}}%
\pgfpathlineto{\pgfqpoint{5.858010in}{1.684211in}}%
\pgfpathlineto{\pgfqpoint{5.858965in}{2.255660in}}%
\pgfpathlineto{\pgfqpoint{5.859066in}{1.908813in}}%
\pgfpathlineto{\pgfqpoint{5.859239in}{2.386476in}}%
\pgfpathlineto{\pgfqpoint{5.859966in}{2.256964in}}%
\pgfpathlineto{\pgfqpoint{5.860890in}{2.396747in}}%
\pgfpathlineto{\pgfqpoint{5.860333in}{1.850845in}}%
\pgfpathlineto{\pgfqpoint{5.861077in}{2.330035in}}%
\pgfpathlineto{\pgfqpoint{5.862040in}{1.832549in}}%
\pgfpathlineto{\pgfqpoint{5.861919in}{2.359998in}}%
\pgfpathlineto{\pgfqpoint{5.862190in}{2.244223in}}%
\pgfpathlineto{\pgfqpoint{5.862323in}{2.393931in}}%
\pgfpathlineto{\pgfqpoint{5.862771in}{1.886738in}}%
\pgfpathlineto{\pgfqpoint{5.863169in}{2.262562in}}%
\pgfpathlineto{\pgfqpoint{5.864262in}{1.782164in}}%
\pgfpathlineto{\pgfqpoint{5.864054in}{2.386285in}}%
\pgfpathlineto{\pgfqpoint{5.864277in}{2.268883in}}%
\pgfpathlineto{\pgfqpoint{5.864526in}{2.403585in}}%
\pgfpathlineto{\pgfqpoint{5.864823in}{1.755809in}}%
\pgfpathlineto{\pgfqpoint{5.865379in}{2.188700in}}%
\pgfpathlineto{\pgfqpoint{5.866406in}{1.786532in}}%
\pgfpathlineto{\pgfqpoint{5.866111in}{2.390098in}}%
\pgfpathlineto{\pgfqpoint{5.866467in}{2.014617in}}%
\pgfpathlineto{\pgfqpoint{5.866733in}{2.415148in}}%
\pgfpathlineto{\pgfqpoint{5.866753in}{1.519219in}}%
\pgfpathlineto{\pgfqpoint{5.867578in}{2.283007in}}%
\pgfpathlineto{\pgfqpoint{5.868219in}{1.701033in}}%
\pgfpathlineto{\pgfqpoint{5.867747in}{2.385232in}}%
\pgfpathlineto{\pgfqpoint{5.868687in}{2.115358in}}%
\pgfpathlineto{\pgfqpoint{5.868971in}{2.392102in}}%
\pgfpathlineto{\pgfqpoint{5.868966in}{1.739983in}}%
\pgfpathlineto{\pgfqpoint{5.869798in}{2.234609in}}%
\pgfpathlineto{\pgfqpoint{5.870690in}{1.662748in}}%
\pgfpathlineto{\pgfqpoint{5.869955in}{2.395582in}}%
\pgfpathlineto{\pgfqpoint{5.870909in}{2.171275in}}%
\pgfpathlineto{\pgfqpoint{5.871706in}{2.386128in}}%
\pgfpathlineto{\pgfqpoint{5.871204in}{1.858205in}}%
\pgfpathlineto{\pgfqpoint{5.872018in}{2.077240in}}%
\pgfpathlineto{\pgfqpoint{5.872260in}{2.395448in}}%
\pgfpathlineto{\pgfqpoint{5.873020in}{1.713374in}}%
\pgfpathlineto{\pgfqpoint{5.873135in}{2.294825in}}%
\pgfpathlineto{\pgfqpoint{5.873965in}{1.835565in}}%
\pgfpathlineto{\pgfqpoint{5.873505in}{2.401809in}}%
\pgfpathlineto{\pgfqpoint{5.874246in}{2.276097in}}%
\pgfpathlineto{\pgfqpoint{5.875208in}{1.434732in}}%
\pgfpathlineto{\pgfqpoint{5.874662in}{2.398626in}}%
\pgfpathlineto{\pgfqpoint{5.875297in}{2.126455in}}%
\pgfpathlineto{\pgfqpoint{5.875351in}{2.401769in}}%
\pgfpathlineto{\pgfqpoint{5.875557in}{1.842005in}}%
\pgfpathlineto{\pgfqpoint{5.876405in}{2.152630in}}%
\pgfpathlineto{\pgfqpoint{5.877440in}{1.823197in}}%
\pgfpathlineto{\pgfqpoint{5.876973in}{2.384165in}}%
\pgfpathlineto{\pgfqpoint{5.877505in}{2.182940in}}%
\pgfpathlineto{\pgfqpoint{5.878599in}{2.407114in}}%
\pgfpathlineto{\pgfqpoint{5.878307in}{1.805663in}}%
\pgfpathlineto{\pgfqpoint{5.878616in}{2.228395in}}%
\pgfpathlineto{\pgfqpoint{5.878818in}{1.638784in}}%
\pgfpathlineto{\pgfqpoint{5.879209in}{2.395940in}}%
\pgfpathlineto{\pgfqpoint{5.879668in}{2.163994in}}%
\pgfpathlineto{\pgfqpoint{5.880013in}{2.379525in}}%
\pgfpathlineto{\pgfqpoint{5.880101in}{1.676274in}}%
\pgfpathlineto{\pgfqpoint{5.880779in}{2.246973in}}%
\pgfpathlineto{\pgfqpoint{5.881620in}{2.425270in}}%
\pgfpathlineto{\pgfqpoint{5.881521in}{1.885304in}}%
\pgfpathlineto{\pgfqpoint{5.881888in}{2.260458in}}%
\pgfpathlineto{\pgfqpoint{5.882782in}{1.559692in}}%
\pgfpathlineto{\pgfqpoint{5.882737in}{2.406224in}}%
\pgfpathlineto{\pgfqpoint{5.882996in}{2.269788in}}%
\pgfpathlineto{\pgfqpoint{5.883779in}{2.383940in}}%
\pgfpathlineto{\pgfqpoint{5.883456in}{1.771070in}}%
\pgfpathlineto{\pgfqpoint{5.884096in}{2.223480in}}%
\pgfpathlineto{\pgfqpoint{5.884236in}{1.862184in}}%
\pgfpathlineto{\pgfqpoint{5.884493in}{2.411798in}}%
\pgfpathlineto{\pgfqpoint{5.885204in}{2.247796in}}%
\pgfpathlineto{\pgfqpoint{5.885893in}{2.403680in}}%
\pgfpathlineto{\pgfqpoint{5.885924in}{1.723211in}}%
\pgfpathlineto{\pgfqpoint{5.886306in}{2.273400in}}%
\pgfpathlineto{\pgfqpoint{5.887247in}{1.838492in}}%
\pgfpathlineto{\pgfqpoint{5.887122in}{2.418978in}}%
\pgfpathlineto{\pgfqpoint{5.887417in}{2.267308in}}%
\pgfpathlineto{\pgfqpoint{5.888364in}{1.853886in}}%
\pgfpathlineto{\pgfqpoint{5.888022in}{2.413760in}}%
\pgfpathlineto{\pgfqpoint{5.888527in}{2.190913in}}%
\pgfpathlineto{\pgfqpoint{5.889220in}{2.407443in}}%
\pgfpathlineto{\pgfqpoint{5.888910in}{1.811252in}}%
\pgfpathlineto{\pgfqpoint{5.889638in}{2.212136in}}%
\pgfpathlineto{\pgfqpoint{5.890488in}{2.423201in}}%
\pgfpathlineto{\pgfqpoint{5.889917in}{1.858266in}}%
\pgfpathlineto{\pgfqpoint{5.890748in}{2.214238in}}%
\pgfpathlineto{\pgfqpoint{5.890910in}{1.758130in}}%
\pgfpathlineto{\pgfqpoint{5.891646in}{2.420677in}}%
\pgfpathlineto{\pgfqpoint{5.891852in}{2.236703in}}%
\pgfpathlineto{\pgfqpoint{5.892457in}{2.422100in}}%
\pgfpathlineto{\pgfqpoint{5.892460in}{1.801211in}}%
\pgfpathlineto{\pgfqpoint{5.892962in}{2.327724in}}%
\pgfpathlineto{\pgfqpoint{5.893352in}{1.933111in}}%
\pgfpathlineto{\pgfqpoint{5.893557in}{2.407340in}}%
\pgfpathlineto{\pgfqpoint{5.894075in}{2.127401in}}%
\pgfpathlineto{\pgfqpoint{5.895114in}{2.453938in}}%
\pgfpathlineto{\pgfqpoint{5.894696in}{1.778874in}}%
\pgfpathlineto{\pgfqpoint{5.895187in}{2.263248in}}%
\pgfpathlineto{\pgfqpoint{5.895877in}{1.864647in}}%
\pgfpathlineto{\pgfqpoint{5.895446in}{2.421074in}}%
\pgfpathlineto{\pgfqpoint{5.896297in}{2.172546in}}%
\pgfpathlineto{\pgfqpoint{5.897243in}{2.424706in}}%
\pgfpathlineto{\pgfqpoint{5.897325in}{1.882695in}}%
\pgfpathlineto{\pgfqpoint{5.897414in}{2.320448in}}%
\pgfpathlineto{\pgfqpoint{5.898377in}{1.780128in}}%
\pgfpathlineto{\pgfqpoint{5.898369in}{2.442663in}}%
\pgfpathlineto{\pgfqpoint{5.898523in}{2.225848in}}%
\pgfpathlineto{\pgfqpoint{5.898997in}{1.685355in}}%
\pgfpathlineto{\pgfqpoint{5.899224in}{2.426374in}}%
\pgfpathlineto{\pgfqpoint{5.899616in}{2.326909in}}%
\pgfpathlineto{\pgfqpoint{5.899975in}{2.412647in}}%
\pgfpathlineto{\pgfqpoint{5.900611in}{1.933015in}}%
\pgfpathlineto{\pgfqpoint{5.900719in}{2.305089in}}%
\pgfpathlineto{\pgfqpoint{5.900808in}{1.695987in}}%
\pgfpathlineto{\pgfqpoint{5.900787in}{2.400200in}}%
\pgfpathlineto{\pgfqpoint{5.901830in}{2.106083in}}%
\pgfpathlineto{\pgfqpoint{5.902636in}{2.409441in}}%
\pgfpathlineto{\pgfqpoint{5.902231in}{1.826257in}}%
\pgfpathlineto{\pgfqpoint{5.902941in}{2.283718in}}%
\pgfpathlineto{\pgfqpoint{5.903408in}{1.895808in}}%
\pgfpathlineto{\pgfqpoint{5.903977in}{2.413994in}}%
\pgfpathlineto{\pgfqpoint{5.904055in}{2.051607in}}%
\pgfpathlineto{\pgfqpoint{5.904891in}{2.402628in}}%
\pgfpathlineto{\pgfqpoint{5.904816in}{1.899131in}}%
\pgfpathlineto{\pgfqpoint{5.905165in}{2.284896in}}%
\pgfpathlineto{\pgfqpoint{5.905416in}{1.888854in}}%
\pgfpathlineto{\pgfqpoint{5.906135in}{2.419361in}}%
\pgfpathlineto{\pgfqpoint{5.906276in}{2.198680in}}%
\pgfpathlineto{\pgfqpoint{5.906834in}{2.438302in}}%
\pgfpathlineto{\pgfqpoint{5.906542in}{1.696431in}}%
\pgfpathlineto{\pgfqpoint{5.907386in}{2.349592in}}%
\pgfpathlineto{\pgfqpoint{5.908017in}{1.631127in}}%
\pgfpathlineto{\pgfqpoint{5.907638in}{2.432611in}}%
\pgfpathlineto{\pgfqpoint{5.908498in}{2.214804in}}%
\pgfpathlineto{\pgfqpoint{5.909515in}{2.398873in}}%
\pgfpathlineto{\pgfqpoint{5.909473in}{1.855055in}}%
\pgfpathlineto{\pgfqpoint{5.909605in}{2.256159in}}%
\pgfpathlineto{\pgfqpoint{5.910254in}{1.926057in}}%
\pgfpathlineto{\pgfqpoint{5.909922in}{2.422052in}}%
\pgfpathlineto{\pgfqpoint{5.910714in}{2.303618in}}%
\pgfpathlineto{\pgfqpoint{5.911093in}{1.765525in}}%
\pgfpathlineto{\pgfqpoint{5.911695in}{2.401363in}}%
\pgfpathlineto{\pgfqpoint{5.911823in}{2.217761in}}%
\pgfpathlineto{\pgfqpoint{5.912317in}{2.398888in}}%
\pgfpathlineto{\pgfqpoint{5.912592in}{1.834034in}}%
\pgfpathlineto{\pgfqpoint{5.912935in}{2.236417in}}%
\pgfpathlineto{\pgfqpoint{5.913142in}{2.410971in}}%
\pgfpathlineto{\pgfqpoint{5.913118in}{1.883079in}}%
\pgfpathlineto{\pgfqpoint{5.914046in}{2.272891in}}%
\pgfpathlineto{\pgfqpoint{5.915008in}{1.769687in}}%
\pgfpathlineto{\pgfqpoint{5.914482in}{2.441219in}}%
\pgfpathlineto{\pgfqpoint{5.915159in}{2.146018in}}%
\pgfpathlineto{\pgfqpoint{5.915453in}{2.417761in}}%
\pgfpathlineto{\pgfqpoint{5.915846in}{1.826950in}}%
\pgfpathlineto{\pgfqpoint{5.916270in}{2.290627in}}%
\pgfpathlineto{\pgfqpoint{5.916394in}{1.793032in}}%
\pgfpathlineto{\pgfqpoint{5.917027in}{2.434279in}}%
\pgfpathlineto{\pgfqpoint{5.917382in}{2.215602in}}%
\pgfpathlineto{\pgfqpoint{5.918285in}{2.374227in}}%
\pgfpathlineto{\pgfqpoint{5.917724in}{1.865357in}}%
\pgfpathlineto{\pgfqpoint{5.918492in}{2.259715in}}%
\pgfpathlineto{\pgfqpoint{5.918846in}{1.836441in}}%
\pgfpathlineto{\pgfqpoint{5.918673in}{2.411437in}}%
\pgfpathlineto{\pgfqpoint{5.919599in}{2.302422in}}%
\pgfpathlineto{\pgfqpoint{5.919834in}{2.413772in}}%
\pgfpathlineto{\pgfqpoint{5.919751in}{1.916702in}}%
\pgfpathlineto{\pgfqpoint{5.920701in}{2.307603in}}%
\pgfpathlineto{\pgfqpoint{5.921520in}{1.720681in}}%
\pgfpathlineto{\pgfqpoint{5.921104in}{2.405965in}}%
\pgfpathlineto{\pgfqpoint{5.921812in}{2.047023in}}%
\pgfpathlineto{\pgfqpoint{5.922491in}{2.414138in}}%
\pgfpathlineto{\pgfqpoint{5.922289in}{1.740693in}}%
\pgfpathlineto{\pgfqpoint{5.922926in}{2.311380in}}%
\pgfpathlineto{\pgfqpoint{5.923700in}{2.437848in}}%
\pgfpathlineto{\pgfqpoint{5.923383in}{1.780561in}}%
\pgfpathlineto{\pgfqpoint{5.924021in}{2.235358in}}%
\pgfpathlineto{\pgfqpoint{5.924875in}{1.677240in}}%
\pgfpathlineto{\pgfqpoint{5.924556in}{2.432811in}}%
\pgfpathlineto{\pgfqpoint{5.925132in}{2.117010in}}%
\pgfpathlineto{\pgfqpoint{5.925770in}{2.401330in}}%
\pgfpathlineto{\pgfqpoint{5.926229in}{1.881177in}}%
\pgfpathlineto{\pgfqpoint{5.926244in}{2.232947in}}%
\pgfpathlineto{\pgfqpoint{5.926875in}{1.786072in}}%
\pgfpathlineto{\pgfqpoint{5.926599in}{2.410969in}}%
\pgfpathlineto{\pgfqpoint{5.927351in}{2.297199in}}%
\pgfpathlineto{\pgfqpoint{5.927766in}{2.405533in}}%
\pgfpathlineto{\pgfqpoint{5.927584in}{1.636102in}}%
\pgfpathlineto{\pgfqpoint{5.928453in}{2.109606in}}%
\pgfpathlineto{\pgfqpoint{5.928635in}{1.741450in}}%
\pgfpathlineto{\pgfqpoint{5.928859in}{2.423795in}}%
\pgfpathlineto{\pgfqpoint{5.929562in}{2.260727in}}%
\pgfpathlineto{\pgfqpoint{5.929618in}{2.418245in}}%
\pgfpathlineto{\pgfqpoint{5.930090in}{1.871274in}}%
\pgfpathlineto{\pgfqpoint{5.930668in}{2.254485in}}%
\pgfpathlineto{\pgfqpoint{5.931549in}{1.812268in}}%
\pgfpathlineto{\pgfqpoint{5.930716in}{2.395716in}}%
\pgfpathlineto{\pgfqpoint{5.931779in}{2.095907in}}%
\pgfpathlineto{\pgfqpoint{5.932382in}{2.393462in}}%
\pgfpathlineto{\pgfqpoint{5.932059in}{1.951147in}}%
\pgfpathlineto{\pgfqpoint{5.932889in}{2.259722in}}%
\pgfpathlineto{\pgfqpoint{5.932982in}{2.429481in}}%
\pgfpathlineto{\pgfqpoint{5.933775in}{1.854342in}}%
\pgfpathlineto{\pgfqpoint{5.933967in}{2.162404in}}%
\pgfpathlineto{\pgfqpoint{5.934989in}{1.864165in}}%
\pgfpathlineto{\pgfqpoint{5.934213in}{2.438957in}}%
\pgfpathlineto{\pgfqpoint{5.935074in}{2.145793in}}%
\pgfpathlineto{\pgfqpoint{5.935640in}{2.435147in}}%
\pgfpathlineto{\pgfqpoint{5.936131in}{1.842477in}}%
\pgfpathlineto{\pgfqpoint{5.936183in}{2.271343in}}%
\pgfpathlineto{\pgfqpoint{5.936312in}{1.903258in}}%
\pgfpathlineto{\pgfqpoint{5.936203in}{2.425100in}}%
\pgfpathlineto{\pgfqpoint{5.937294in}{2.105392in}}%
\pgfpathlineto{\pgfqpoint{5.937322in}{2.396915in}}%
\pgfpathlineto{\pgfqpoint{5.938233in}{1.766078in}}%
\pgfpathlineto{\pgfqpoint{5.938405in}{2.291239in}}%
\pgfpathlineto{\pgfqpoint{5.939124in}{1.622461in}}%
\pgfpathlineto{\pgfqpoint{5.939144in}{2.414248in}}%
\pgfpathlineto{\pgfqpoint{5.939515in}{2.151932in}}%
\pgfpathlineto{\pgfqpoint{5.940279in}{2.419951in}}%
\pgfpathlineto{\pgfqpoint{5.940564in}{1.733582in}}%
\pgfpathlineto{\pgfqpoint{5.940625in}{2.116658in}}%
\pgfpathlineto{\pgfqpoint{5.940878in}{2.407527in}}%
\pgfpathlineto{\pgfqpoint{5.940682in}{1.778308in}}%
\pgfpathlineto{\pgfqpoint{5.941740in}{2.214488in}}%
\pgfpathlineto{\pgfqpoint{5.942216in}{1.804049in}}%
\pgfpathlineto{\pgfqpoint{5.942339in}{2.412615in}}%
\pgfpathlineto{\pgfqpoint{5.942837in}{2.328445in}}%
\pgfpathlineto{\pgfqpoint{5.942932in}{2.444539in}}%
\pgfpathlineto{\pgfqpoint{5.942900in}{1.903476in}}%
\pgfpathlineto{\pgfqpoint{5.943930in}{2.118488in}}%
\pgfpathlineto{\pgfqpoint{5.944558in}{1.457159in}}%
\pgfpathlineto{\pgfqpoint{5.944334in}{2.413716in}}%
\pgfpathlineto{\pgfqpoint{5.945036in}{2.178699in}}%
\pgfpathlineto{\pgfqpoint{5.945621in}{2.448465in}}%
\pgfpathlineto{\pgfqpoint{5.945293in}{1.899619in}}%
\pgfpathlineto{\pgfqpoint{5.946149in}{2.347453in}}%
\pgfpathlineto{\pgfqpoint{5.947155in}{1.903407in}}%
\pgfpathlineto{\pgfqpoint{5.946275in}{2.428904in}}%
\pgfpathlineto{\pgfqpoint{5.947266in}{2.155285in}}%
\pgfpathlineto{\pgfqpoint{5.947589in}{2.417843in}}%
\pgfpathlineto{\pgfqpoint{5.948288in}{1.896294in}}%
\pgfpathlineto{\pgfqpoint{5.948380in}{2.317214in}}%
\pgfpathlineto{\pgfqpoint{5.948931in}{1.654598in}}%
\pgfpathlineto{\pgfqpoint{5.948998in}{2.462148in}}%
\pgfpathlineto{\pgfqpoint{5.949488in}{2.336947in}}%
\pgfpathlineto{\pgfqpoint{5.949699in}{2.430710in}}%
\pgfpathlineto{\pgfqpoint{5.950210in}{1.687678in}}%
\pgfpathlineto{\pgfqpoint{5.950512in}{2.183693in}}%
\pgfpathlineto{\pgfqpoint{5.950647in}{1.765429in}}%
\pgfpathlineto{\pgfqpoint{5.951051in}{2.419006in}}%
\pgfpathlineto{\pgfqpoint{5.951623in}{2.132645in}}%
\pgfpathlineto{\pgfqpoint{5.952303in}{2.433902in}}%
\pgfpathlineto{\pgfqpoint{5.952131in}{1.809296in}}%
\pgfpathlineto{\pgfqpoint{5.952732in}{2.328808in}}%
\pgfpathlineto{\pgfqpoint{5.953154in}{1.863211in}}%
\pgfpathlineto{\pgfqpoint{5.953085in}{2.445258in}}%
\pgfpathlineto{\pgfqpoint{5.953844in}{2.170967in}}%
\pgfpathlineto{\pgfqpoint{5.954523in}{1.903928in}}%
\pgfpathlineto{\pgfqpoint{5.953908in}{2.424545in}}%
\pgfpathlineto{\pgfqpoint{5.954611in}{2.080797in}}%
\pgfpathlineto{\pgfqpoint{5.954972in}{2.431428in}}%
\pgfpathlineto{\pgfqpoint{5.955050in}{1.888938in}}%
\pgfpathlineto{\pgfqpoint{5.955720in}{2.430919in}}%
\pgfpathlineto{\pgfqpoint{5.956661in}{1.867552in}}%
\pgfpathlineto{\pgfqpoint{5.956132in}{2.430932in}}%
\pgfpathlineto{\pgfqpoint{5.956831in}{2.145830in}}%
\pgfpathlineto{\pgfqpoint{5.957576in}{2.457227in}}%
\pgfpathlineto{\pgfqpoint{5.957833in}{1.824143in}}%
\pgfpathlineto{\pgfqpoint{5.957944in}{2.249100in}}%
\pgfpathlineto{\pgfqpoint{5.958355in}{1.757268in}}%
\pgfpathlineto{\pgfqpoint{5.958910in}{2.451408in}}%
\pgfpathlineto{\pgfqpoint{5.959030in}{2.346041in}}%
\pgfpathlineto{\pgfqpoint{5.960127in}{2.462682in}}%
\pgfpathlineto{\pgfqpoint{5.959914in}{1.641829in}}%
\pgfpathlineto{\pgfqpoint{5.960137in}{2.276149in}}%
\pgfpathlineto{\pgfqpoint{5.960832in}{1.932167in}}%
\pgfpathlineto{\pgfqpoint{5.961213in}{2.426548in}}%
\pgfpathlineto{\pgfqpoint{5.961248in}{2.156849in}}%
\pgfpathlineto{\pgfqpoint{5.961416in}{2.443238in}}%
\pgfpathlineto{\pgfqpoint{5.961887in}{1.820469in}}%
\pgfpathlineto{\pgfqpoint{5.962365in}{2.289257in}}%
\pgfpathlineto{\pgfqpoint{5.962675in}{1.855224in}}%
\pgfpathlineto{\pgfqpoint{5.963257in}{2.428521in}}%
\pgfpathlineto{\pgfqpoint{5.963471in}{2.168799in}}%
\pgfpathlineto{\pgfqpoint{5.963577in}{2.426511in}}%
\pgfpathlineto{\pgfqpoint{5.964211in}{1.904181in}}%
\pgfpathlineto{\pgfqpoint{5.964582in}{2.327455in}}%
\pgfpathlineto{\pgfqpoint{5.964886in}{1.890200in}}%
\pgfpathlineto{\pgfqpoint{5.965342in}{2.433307in}}%
\pgfpathlineto{\pgfqpoint{5.965696in}{2.213877in}}%
\pgfpathlineto{\pgfqpoint{5.965726in}{2.429792in}}%
\pgfpathlineto{\pgfqpoint{5.965931in}{1.536080in}}%
\pgfpathlineto{\pgfqpoint{5.966806in}{2.306215in}}%
\pgfpathlineto{\pgfqpoint{5.967327in}{1.732132in}}%
\pgfpathlineto{\pgfqpoint{5.967718in}{2.433955in}}%
\pgfpathlineto{\pgfqpoint{5.967918in}{2.197899in}}%
\pgfpathlineto{\pgfqpoint{5.968442in}{2.430327in}}%
\pgfpathlineto{\pgfqpoint{5.968768in}{1.705360in}}%
\pgfpathlineto{\pgfqpoint{5.969030in}{2.216335in}}%
\pgfpathlineto{\pgfqpoint{5.969245in}{2.438050in}}%
\pgfpathlineto{\pgfqpoint{5.969179in}{1.849668in}}%
\pgfpathlineto{\pgfqpoint{5.970145in}{2.326046in}}%
\pgfpathlineto{\pgfqpoint{5.971171in}{1.810041in}}%
\pgfpathlineto{\pgfqpoint{5.970582in}{2.432026in}}%
\pgfpathlineto{\pgfqpoint{5.971256in}{2.290495in}}%
\pgfpathlineto{\pgfqpoint{5.971258in}{2.443926in}}%
\pgfpathlineto{\pgfqpoint{5.971290in}{1.672271in}}%
\pgfpathlineto{\pgfqpoint{5.972365in}{2.318701in}}%
\pgfpathlineto{\pgfqpoint{5.972386in}{1.433757in}}%
\pgfpathlineto{\pgfqpoint{5.972468in}{2.413927in}}%
\pgfpathlineto{\pgfqpoint{5.973479in}{2.193700in}}%
\pgfpathlineto{\pgfqpoint{5.973533in}{2.426418in}}%
\pgfpathlineto{\pgfqpoint{5.974182in}{1.915391in}}%
\pgfpathlineto{\pgfqpoint{5.974594in}{2.338464in}}%
\pgfpathlineto{\pgfqpoint{5.975666in}{1.905069in}}%
\pgfpathlineto{\pgfqpoint{5.975022in}{2.447439in}}%
\pgfpathlineto{\pgfqpoint{5.975709in}{2.176929in}}%
\pgfpathlineto{\pgfqpoint{5.976074in}{2.454525in}}%
\pgfpathlineto{\pgfqpoint{5.976381in}{1.812866in}}%
\pgfpathlineto{\pgfqpoint{5.976820in}{2.289959in}}%
\pgfpathlineto{\pgfqpoint{5.977124in}{2.421093in}}%
\pgfpathlineto{\pgfqpoint{5.977504in}{1.770905in}}%
\pgfpathlineto{\pgfqpoint{5.977927in}{2.247590in}}%
\pgfpathlineto{\pgfqpoint{5.978535in}{1.726339in}}%
\pgfpathlineto{\pgfqpoint{5.978111in}{2.451863in}}%
\pgfpathlineto{\pgfqpoint{5.979035in}{2.256891in}}%
\pgfpathlineto{\pgfqpoint{5.979091in}{2.434730in}}%
\pgfpathlineto{\pgfqpoint{5.979590in}{1.787705in}}%
\pgfpathlineto{\pgfqpoint{5.980147in}{2.372689in}}%
\pgfpathlineto{\pgfqpoint{5.980189in}{1.869271in}}%
\pgfpathlineto{\pgfqpoint{5.980446in}{2.452773in}}%
\pgfpathlineto{\pgfqpoint{5.981260in}{2.150895in}}%
\pgfpathlineto{\pgfqpoint{5.982219in}{2.429909in}}%
\pgfpathlineto{\pgfqpoint{5.981910in}{1.898860in}}%
\pgfpathlineto{\pgfqpoint{5.982371in}{2.182351in}}%
\pgfpathlineto{\pgfqpoint{5.982981in}{2.423610in}}%
\pgfpathlineto{\pgfqpoint{5.983063in}{1.481075in}}%
\pgfpathlineto{\pgfqpoint{5.983480in}{2.334333in}}%
\pgfpathlineto{\pgfqpoint{5.984032in}{1.829210in}}%
\pgfpathlineto{\pgfqpoint{5.983950in}{2.447989in}}%
\pgfpathlineto{\pgfqpoint{5.984592in}{2.283900in}}%
\pgfpathlineto{\pgfqpoint{5.985210in}{2.458306in}}%
\pgfpathlineto{\pgfqpoint{5.985378in}{1.796325in}}%
\pgfpathlineto{\pgfqpoint{5.985696in}{2.108281in}}%
\pgfpathlineto{\pgfqpoint{5.986573in}{1.579622in}}%
\pgfpathlineto{\pgfqpoint{5.986474in}{2.428354in}}%
\pgfpathlineto{\pgfqpoint{5.986788in}{2.050821in}}%
\pgfpathlineto{\pgfqpoint{5.987213in}{2.427116in}}%
\pgfpathlineto{\pgfqpoint{5.987382in}{1.710518in}}%
\pgfpathlineto{\pgfqpoint{5.987898in}{2.363186in}}%
\pgfpathlineto{\pgfqpoint{5.988641in}{1.827605in}}%
\pgfpathlineto{\pgfqpoint{5.988520in}{2.471667in}}%
\pgfpathlineto{\pgfqpoint{5.989008in}{2.257827in}}%
\pgfpathlineto{\pgfqpoint{5.990073in}{2.450893in}}%
\pgfpathlineto{\pgfqpoint{5.989266in}{1.892409in}}%
\pgfpathlineto{\pgfqpoint{5.990094in}{2.215430in}}%
\pgfpathlineto{\pgfqpoint{5.990895in}{1.762385in}}%
\pgfpathlineto{\pgfqpoint{5.990886in}{2.432064in}}%
\pgfpathlineto{\pgfqpoint{5.991204in}{2.370981in}}%
\pgfpathlineto{\pgfqpoint{5.991811in}{1.886450in}}%
\pgfpathlineto{\pgfqpoint{5.991869in}{2.443741in}}%
\pgfpathlineto{\pgfqpoint{5.992317in}{2.295311in}}%
\pgfpathlineto{\pgfqpoint{5.992441in}{1.899244in}}%
\pgfpathlineto{\pgfqpoint{5.993340in}{2.433850in}}%
\pgfpathlineto{\pgfqpoint{5.993418in}{2.003339in}}%
\pgfpathlineto{\pgfqpoint{5.993752in}{2.435877in}}%
\pgfpathlineto{\pgfqpoint{5.993654in}{1.798023in}}%
\pgfpathlineto{\pgfqpoint{5.994530in}{2.146726in}}%
\pgfpathlineto{\pgfqpoint{5.995440in}{1.789312in}}%
\pgfpathlineto{\pgfqpoint{5.994729in}{2.475983in}}%
\pgfpathlineto{\pgfqpoint{5.995634in}{2.259557in}}%
\pgfpathlineto{\pgfqpoint{5.996149in}{2.447684in}}%
\pgfpathlineto{\pgfqpoint{5.996368in}{1.820403in}}%
\pgfpathlineto{\pgfqpoint{5.996747in}{2.314267in}}%
\pgfpathlineto{\pgfqpoint{5.996837in}{1.783089in}}%
\pgfpathlineto{\pgfqpoint{5.997129in}{2.471097in}}%
\pgfpathlineto{\pgfqpoint{5.997856in}{2.040263in}}%
\pgfpathlineto{\pgfqpoint{5.998147in}{2.439507in}}%
\pgfpathlineto{\pgfqpoint{5.998034in}{1.885022in}}%
\pgfpathlineto{\pgfqpoint{5.998969in}{2.318044in}}%
\pgfpathlineto{\pgfqpoint{5.999146in}{2.444037in}}%
\pgfpathlineto{\pgfqpoint{6.000015in}{1.934237in}}%
\pgfpathlineto{\pgfqpoint{6.000064in}{2.184922in}}%
\pgfpathlineto{\pgfqpoint{6.000113in}{1.886306in}}%
\pgfpathlineto{\pgfqpoint{6.000166in}{2.450202in}}%
\pgfpathlineto{\pgfqpoint{6.001171in}{2.322324in}}%
\pgfpathlineto{\pgfqpoint{6.001537in}{2.442877in}}%
\pgfpathlineto{\pgfqpoint{6.002129in}{1.879810in}}%
\pgfpathlineto{\pgfqpoint{6.002271in}{2.228000in}}%
\pgfpathlineto{\pgfqpoint{6.003015in}{1.895182in}}%
\pgfpathlineto{\pgfqpoint{6.003286in}{2.444446in}}%
\pgfpathlineto{\pgfqpoint{6.003379in}{2.301119in}}%
\pgfpathlineto{\pgfqpoint{6.003594in}{2.449296in}}%
\pgfpathlineto{\pgfqpoint{6.004424in}{1.865640in}}%
\pgfpathlineto{\pgfqpoint{6.004490in}{2.388718in}}%
\pgfpathlineto{\pgfqpoint{6.004853in}{1.796757in}}%
\pgfpathlineto{\pgfqpoint{6.005495in}{2.430740in}}%
\pgfpathlineto{\pgfqpoint{6.005602in}{2.296843in}}%
\pgfpathlineto{\pgfqpoint{6.006349in}{1.819907in}}%
\pgfpathlineto{\pgfqpoint{6.005808in}{2.453216in}}%
\pgfpathlineto{\pgfqpoint{6.006718in}{2.151447in}}%
\pgfpathlineto{\pgfqpoint{6.007634in}{2.438445in}}%
\pgfpathlineto{\pgfqpoint{6.006848in}{1.813642in}}%
\pgfpathlineto{\pgfqpoint{6.007829in}{2.363512in}}%
\pgfpathlineto{\pgfqpoint{6.008143in}{1.822248in}}%
\pgfpathlineto{\pgfqpoint{6.008519in}{2.477047in}}%
\pgfpathlineto{\pgfqpoint{6.008941in}{2.204805in}}%
\pgfpathlineto{\pgfqpoint{6.009861in}{2.439878in}}%
\pgfpathlineto{\pgfqpoint{6.009189in}{1.873145in}}%
\pgfpathlineto{\pgfqpoint{6.010048in}{2.275562in}}%
\pgfpathlineto{\pgfqpoint{6.010749in}{1.881653in}}%
\pgfpathlineto{\pgfqpoint{6.010928in}{2.443990in}}%
\pgfpathlineto{\pgfqpoint{6.011157in}{2.140567in}}%
\pgfpathlineto{\pgfqpoint{6.011810in}{2.451941in}}%
\pgfpathlineto{\pgfqpoint{6.012179in}{1.852945in}}%
\pgfpathlineto{\pgfqpoint{6.012266in}{2.106605in}}%
\pgfpathlineto{\pgfqpoint{6.013152in}{1.792193in}}%
\pgfpathlineto{\pgfqpoint{6.013214in}{2.438598in}}%
\pgfpathlineto{\pgfqpoint{6.013373in}{2.038571in}}%
\pgfpathlineto{\pgfqpoint{6.013912in}{2.445114in}}%
\pgfpathlineto{\pgfqpoint{6.014217in}{1.819989in}}%
\pgfpathlineto{\pgfqpoint{6.014485in}{2.327035in}}%
\pgfpathlineto{\pgfqpoint{6.014844in}{2.451030in}}%
\pgfpathlineto{\pgfqpoint{6.015084in}{1.852417in}}%
\pgfpathlineto{\pgfqpoint{6.015584in}{2.232901in}}%
\pgfpathlineto{\pgfqpoint{6.015681in}{1.802994in}}%
\pgfpathlineto{\pgfqpoint{6.016110in}{2.447227in}}%
\pgfpathlineto{\pgfqpoint{6.016692in}{2.367731in}}%
\pgfpathlineto{\pgfqpoint{6.016709in}{2.452222in}}%
\pgfpathlineto{\pgfqpoint{6.017520in}{1.952300in}}%
\pgfpathlineto{\pgfqpoint{6.017765in}{2.299175in}}%
\pgfpathlineto{\pgfqpoint{6.017888in}{1.865194in}}%
\pgfpathlineto{\pgfqpoint{6.018229in}{2.443260in}}%
\pgfpathlineto{\pgfqpoint{6.018874in}{2.307476in}}%
\pgfpathlineto{\pgfqpoint{6.018968in}{2.454060in}}%
\pgfpathlineto{\pgfqpoint{6.019500in}{1.901730in}}%
\pgfpathlineto{\pgfqpoint{6.019968in}{2.336513in}}%
\pgfpathlineto{\pgfqpoint{6.020253in}{1.796366in}}%
\pgfpathlineto{\pgfqpoint{6.020811in}{2.457410in}}%
\pgfpathlineto{\pgfqpoint{6.021079in}{2.266099in}}%
\pgfpathlineto{\pgfqpoint{6.021424in}{2.447422in}}%
\pgfpathlineto{\pgfqpoint{6.021223in}{1.862870in}}%
\pgfpathlineto{\pgfqpoint{6.022185in}{2.290855in}}%
\pgfpathlineto{\pgfqpoint{6.023244in}{1.742797in}}%
\pgfpathlineto{\pgfqpoint{6.022764in}{2.447507in}}%
\pgfpathlineto{\pgfqpoint{6.023297in}{2.235222in}}%
\pgfpathlineto{\pgfqpoint{6.024204in}{2.457801in}}%
\pgfpathlineto{\pgfqpoint{6.024146in}{1.826969in}}%
\pgfpathlineto{\pgfqpoint{6.024403in}{2.007563in}}%
\pgfpathlineto{\pgfqpoint{6.024409in}{1.806554in}}%
\pgfpathlineto{\pgfqpoint{6.024407in}{2.444932in}}%
\pgfpathlineto{\pgfqpoint{6.025501in}{2.243022in}}%
\pgfpathlineto{\pgfqpoint{6.026470in}{2.450498in}}%
\pgfpathlineto{\pgfqpoint{6.026043in}{1.818771in}}%
\pgfpathlineto{\pgfqpoint{6.026608in}{2.447362in}}%
\pgfpathlineto{\pgfqpoint{6.027218in}{1.846355in}}%
\pgfpathlineto{\pgfqpoint{6.027070in}{2.454256in}}%
\pgfpathlineto{\pgfqpoint{6.027720in}{2.303201in}}%
\pgfpathlineto{\pgfqpoint{6.028377in}{1.918788in}}%
\pgfpathlineto{\pgfqpoint{6.028279in}{2.440676in}}%
\pgfpathlineto{\pgfqpoint{6.028833in}{2.166290in}}%
\pgfpathlineto{\pgfqpoint{6.028873in}{2.451899in}}%
\pgfpathlineto{\pgfqpoint{6.028937in}{1.816965in}}%
\pgfpathlineto{\pgfqpoint{6.029943in}{2.431882in}}%
\pgfpathlineto{\pgfqpoint{6.031023in}{1.902470in}}%
\pgfpathlineto{\pgfqpoint{6.030600in}{2.438002in}}%
\pgfpathlineto{\pgfqpoint{6.031054in}{2.322396in}}%
\pgfpathlineto{\pgfqpoint{6.032066in}{2.447272in}}%
\pgfpathlineto{\pgfqpoint{6.031613in}{1.948583in}}%
\pgfpathlineto{\pgfqpoint{6.032161in}{2.354016in}}%
\pgfpathlineto{\pgfqpoint{6.033045in}{1.781054in}}%
\pgfpathlineto{\pgfqpoint{6.032964in}{2.469841in}}%
\pgfpathlineto{\pgfqpoint{6.033270in}{2.096831in}}%
\pgfpathlineto{\pgfqpoint{6.034286in}{2.452295in}}%
\pgfpathlineto{\pgfqpoint{6.033971in}{1.890461in}}%
\pgfpathlineto{\pgfqpoint{6.034380in}{2.292282in}}%
\pgfpathlineto{\pgfqpoint{6.035298in}{1.804832in}}%
\pgfpathlineto{\pgfqpoint{6.034844in}{2.468784in}}%
\pgfpathlineto{\pgfqpoint{6.035492in}{2.206413in}}%
\pgfpathlineto{\pgfqpoint{6.035754in}{2.454576in}}%
\pgfpathlineto{\pgfqpoint{6.035598in}{1.906804in}}%
\pgfpathlineto{\pgfqpoint{6.036602in}{2.337883in}}%
\pgfpathlineto{\pgfqpoint{6.036622in}{1.658634in}}%
\pgfpathlineto{\pgfqpoint{6.037226in}{2.458183in}}%
\pgfpathlineto{\pgfqpoint{6.037716in}{2.138855in}}%
\pgfpathlineto{\pgfqpoint{6.038423in}{2.457412in}}%
\pgfpathlineto{\pgfqpoint{6.038051in}{1.807105in}}%
\pgfpathlineto{\pgfqpoint{6.038828in}{2.301246in}}%
\pgfpathlineto{\pgfqpoint{6.039050in}{2.457730in}}%
\pgfpathlineto{\pgfqpoint{6.039596in}{1.809829in}}%
\pgfpathlineto{\pgfqpoint{6.039926in}{2.275733in}}%
\pgfpathlineto{\pgfqpoint{6.040314in}{1.940852in}}%
\pgfpathlineto{\pgfqpoint{6.040029in}{2.458712in}}%
\pgfpathlineto{\pgfqpoint{6.041037in}{2.166797in}}%
\pgfpathlineto{\pgfqpoint{6.042023in}{2.455402in}}%
\pgfpathlineto{\pgfqpoint{6.041335in}{1.818714in}}%
\pgfpathlineto{\pgfqpoint{6.042150in}{2.242533in}}%
\pgfpathlineto{\pgfqpoint{6.042831in}{1.906667in}}%
\pgfpathlineto{\pgfqpoint{6.043064in}{2.461833in}}%
\pgfpathlineto{\pgfqpoint{6.043251in}{2.304822in}}%
\pgfpathlineto{\pgfqpoint{6.043636in}{2.459775in}}%
\pgfpathlineto{\pgfqpoint{6.043896in}{1.844936in}}%
\pgfpathlineto{\pgfqpoint{6.044359in}{2.390908in}}%
\pgfpathlineto{\pgfqpoint{6.045352in}{1.883144in}}%
\pgfpathlineto{\pgfqpoint{6.044685in}{2.479898in}}%
\pgfpathlineto{\pgfqpoint{6.045469in}{2.247006in}}%
\pgfpathlineto{\pgfqpoint{6.046529in}{2.468265in}}%
\pgfpathlineto{\pgfqpoint{6.046391in}{1.738954in}}%
\pgfpathlineto{\pgfqpoint{6.046579in}{2.296716in}}%
\pgfpathlineto{\pgfqpoint{6.047602in}{1.740747in}}%
\pgfpathlineto{\pgfqpoint{6.046623in}{2.477545in}}%
\pgfpathlineto{\pgfqpoint{6.047687in}{2.339691in}}%
\pgfpathlineto{\pgfqpoint{6.048000in}{2.486462in}}%
\pgfpathlineto{\pgfqpoint{6.048491in}{1.877825in}}%
\pgfpathlineto{\pgfqpoint{6.048795in}{2.359707in}}%
\pgfpathlineto{\pgfqpoint{6.048987in}{1.912333in}}%
\pgfpathlineto{\pgfqpoint{6.049475in}{2.440057in}}%
\pgfpathlineto{\pgfqpoint{6.049907in}{2.191213in}}%
\pgfpathlineto{\pgfqpoint{6.050606in}{2.471990in}}%
\pgfpathlineto{\pgfqpoint{6.050553in}{1.835507in}}%
\pgfpathlineto{\pgfqpoint{6.051020in}{2.424132in}}%
\pgfpathlineto{\pgfqpoint{6.051838in}{1.719868in}}%
\pgfpathlineto{\pgfqpoint{6.051897in}{2.443574in}}%
\pgfpathlineto{\pgfqpoint{6.052136in}{2.269391in}}%
\pgfpathlineto{\pgfqpoint{6.052212in}{2.495060in}}%
\pgfpathlineto{\pgfqpoint{6.052155in}{1.853360in}}%
\pgfpathlineto{\pgfqpoint{6.053253in}{2.430493in}}%
\pgfpathlineto{\pgfqpoint{6.054057in}{1.833737in}}%
\pgfpathlineto{\pgfqpoint{6.053894in}{2.495501in}}%
\pgfpathlineto{\pgfqpoint{6.054367in}{2.354235in}}%
\pgfpathlineto{\pgfqpoint{6.054830in}{1.801434in}}%
\pgfpathlineto{\pgfqpoint{6.054815in}{2.462321in}}%
\pgfpathlineto{\pgfqpoint{6.055476in}{2.308755in}}%
\pgfpathlineto{\pgfqpoint{6.056076in}{2.465536in}}%
\pgfpathlineto{\pgfqpoint{6.056550in}{1.901202in}}%
\pgfpathlineto{\pgfqpoint{6.056585in}{2.324710in}}%
\pgfpathlineto{\pgfqpoint{6.056705in}{1.918219in}}%
\pgfpathlineto{\pgfqpoint{6.056923in}{2.456889in}}%
\pgfpathlineto{\pgfqpoint{6.057694in}{2.305922in}}%
\pgfpathlineto{\pgfqpoint{6.057731in}{2.465524in}}%
\pgfpathlineto{\pgfqpoint{6.057899in}{1.874482in}}%
\pgfpathlineto{\pgfqpoint{6.058802in}{2.250726in}}%
\pgfpathlineto{\pgfqpoint{6.059164in}{1.738849in}}%
\pgfpathlineto{\pgfqpoint{6.059208in}{2.455800in}}%
\pgfpathlineto{\pgfqpoint{6.059912in}{2.128077in}}%
\pgfpathlineto{\pgfqpoint{6.060804in}{2.472479in}}%
\pgfpathlineto{\pgfqpoint{6.060924in}{1.673162in}}%
\pgfpathlineto{\pgfqpoint{6.061024in}{2.287080in}}%
\pgfpathlineto{\pgfqpoint{6.061129in}{1.939075in}}%
\pgfpathlineto{\pgfqpoint{6.061662in}{2.469148in}}%
\pgfpathlineto{\pgfqpoint{6.062125in}{2.291748in}}%
\pgfpathlineto{\pgfqpoint{6.062281in}{2.462761in}}%
\pgfpathlineto{\pgfqpoint{6.062379in}{1.874039in}}%
\pgfpathlineto{\pgfqpoint{6.063236in}{2.372476in}}%
\pgfpathlineto{\pgfqpoint{6.064287in}{1.629039in}}%
\pgfpathlineto{\pgfqpoint{6.064010in}{2.478424in}}%
\pgfpathlineto{\pgfqpoint{6.064347in}{2.254622in}}%
\pgfpathlineto{\pgfqpoint{6.064553in}{2.455060in}}%
\pgfpathlineto{\pgfqpoint{6.065045in}{1.893606in}}%
\pgfpathlineto{\pgfqpoint{6.065458in}{2.345566in}}%
\pgfpathlineto{\pgfqpoint{6.065712in}{1.712534in}}%
\pgfpathlineto{\pgfqpoint{6.066504in}{2.469279in}}%
\pgfpathlineto{\pgfqpoint{6.066569in}{2.305391in}}%
\pgfpathlineto{\pgfqpoint{6.067407in}{2.452215in}}%
\pgfpathlineto{\pgfqpoint{6.067192in}{1.838777in}}%
\pgfpathlineto{\pgfqpoint{6.067680in}{2.360598in}}%
\pgfpathlineto{\pgfqpoint{6.068190in}{1.931221in}}%
\pgfpathlineto{\pgfqpoint{6.067720in}{2.466582in}}%
\pgfpathlineto{\pgfqpoint{6.068793in}{2.281609in}}%
\pgfpathlineto{\pgfqpoint{6.069119in}{2.482687in}}%
\pgfpathlineto{\pgfqpoint{6.069170in}{1.892402in}}%
\pgfpathlineto{\pgfqpoint{6.069904in}{2.368249in}}%
\pgfpathlineto{\pgfqpoint{6.070329in}{1.729336in}}%
\pgfpathlineto{\pgfqpoint{6.070772in}{2.477143in}}%
\pgfpathlineto{\pgfqpoint{6.071013in}{2.184633in}}%
\pgfpathlineto{\pgfqpoint{6.071239in}{2.490444in}}%
\pgfpathlineto{\pgfqpoint{6.072031in}{1.827055in}}%
\pgfpathlineto{\pgfqpoint{6.072125in}{2.334279in}}%
\pgfpathlineto{\pgfqpoint{6.072908in}{1.891186in}}%
\pgfpathlineto{\pgfqpoint{6.072398in}{2.480843in}}%
\pgfpathlineto{\pgfqpoint{6.073059in}{2.365716in}}%
\pgfpathlineto{\pgfqpoint{6.073061in}{2.466915in}}%
\pgfpathlineto{\pgfqpoint{6.073235in}{1.921515in}}%
\pgfpathlineto{\pgfqpoint{6.074170in}{2.384483in}}%
\pgfpathlineto{\pgfqpoint{6.074448in}{1.891626in}}%
\pgfpathlineto{\pgfqpoint{6.074661in}{2.458797in}}%
\pgfpathlineto{\pgfqpoint{6.075279in}{2.122872in}}%
\pgfpathlineto{\pgfqpoint{6.076226in}{2.487359in}}%
\pgfpathlineto{\pgfqpoint{6.075857in}{1.756591in}}%
\pgfpathlineto{\pgfqpoint{6.076390in}{2.264296in}}%
\pgfpathlineto{\pgfqpoint{6.076703in}{2.481258in}}%
\pgfpathlineto{\pgfqpoint{6.076684in}{1.573835in}}%
\pgfpathlineto{\pgfqpoint{6.077496in}{2.313827in}}%
\pgfpathlineto{\pgfqpoint{6.078371in}{1.880789in}}%
\pgfpathlineto{\pgfqpoint{6.078383in}{2.479458in}}%
\pgfpathlineto{\pgfqpoint{6.078607in}{2.271174in}}%
\pgfpathlineto{\pgfqpoint{6.078745in}{2.483167in}}%
\pgfpathlineto{\pgfqpoint{6.079327in}{1.947488in}}%
\pgfpathlineto{\pgfqpoint{6.079716in}{2.301688in}}%
\pgfpathlineto{\pgfqpoint{6.080299in}{1.861570in}}%
\pgfpathlineto{\pgfqpoint{6.079834in}{2.481853in}}%
\pgfpathlineto{\pgfqpoint{6.080826in}{2.239058in}}%
\pgfpathlineto{\pgfqpoint{6.081124in}{2.465306in}}%
\pgfpathlineto{\pgfqpoint{6.081708in}{1.852923in}}%
\pgfpathlineto{\pgfqpoint{6.081938in}{2.397915in}}%
\pgfpathlineto{\pgfqpoint{6.083046in}{1.734134in}}%
\pgfpathlineto{\pgfqpoint{6.082808in}{2.485172in}}%
\pgfpathlineto{\pgfqpoint{6.083051in}{2.156697in}}%
\pgfpathlineto{\pgfqpoint{6.084058in}{2.493184in}}%
\pgfpathlineto{\pgfqpoint{6.084118in}{1.899273in}}%
\pgfpathlineto{\pgfqpoint{6.084165in}{2.284955in}}%
\pgfpathlineto{\pgfqpoint{6.084390in}{1.754370in}}%
\pgfpathlineto{\pgfqpoint{6.085213in}{2.456451in}}%
\pgfpathlineto{\pgfqpoint{6.085274in}{2.267771in}}%
\pgfpathlineto{\pgfqpoint{6.085558in}{2.469418in}}%
\pgfpathlineto{\pgfqpoint{6.085364in}{1.765892in}}%
\pgfpathlineto{\pgfqpoint{6.086383in}{2.307621in}}%
\pgfpathlineto{\pgfqpoint{6.087186in}{1.827314in}}%
\pgfpathlineto{\pgfqpoint{6.087243in}{2.489618in}}%
\pgfpathlineto{\pgfqpoint{6.087495in}{2.249893in}}%
\pgfpathlineto{\pgfqpoint{6.088204in}{2.471931in}}%
\pgfpathlineto{\pgfqpoint{6.088318in}{1.743780in}}%
\pgfpathlineto{\pgfqpoint{6.088604in}{2.299219in}}%
\pgfpathlineto{\pgfqpoint{6.089579in}{1.795169in}}%
\pgfpathlineto{\pgfqpoint{6.089306in}{2.512666in}}%
\pgfpathlineto{\pgfqpoint{6.089715in}{2.250356in}}%
\pgfpathlineto{\pgfqpoint{6.089948in}{1.641508in}}%
\pgfpathlineto{\pgfqpoint{6.090501in}{2.467839in}}%
\pgfpathlineto{\pgfqpoint{6.090818in}{2.306460in}}%
\pgfpathlineto{\pgfqpoint{6.091121in}{2.467926in}}%
\pgfpathlineto{\pgfqpoint{6.091415in}{1.812644in}}%
\pgfpathlineto{\pgfqpoint{6.091930in}{2.356835in}}%
\pgfpathlineto{\pgfqpoint{6.092435in}{1.816258in}}%
\pgfpathlineto{\pgfqpoint{6.092877in}{2.471513in}}%
\pgfpathlineto{\pgfqpoint{6.093040in}{2.379940in}}%
\pgfpathlineto{\pgfqpoint{6.093599in}{2.483068in}}%
\pgfpathlineto{\pgfqpoint{6.093072in}{1.866516in}}%
\pgfpathlineto{\pgfqpoint{6.093897in}{2.440124in}}%
\pgfpathlineto{\pgfqpoint{6.094843in}{1.908771in}}%
\pgfpathlineto{\pgfqpoint{6.094940in}{2.496975in}}%
\pgfpathlineto{\pgfqpoint{6.095008in}{2.341513in}}%
\pgfpathlineto{\pgfqpoint{6.096011in}{1.930958in}}%
\pgfpathlineto{\pgfqpoint{6.095243in}{2.489547in}}%
\pgfpathlineto{\pgfqpoint{6.096118in}{2.219315in}}%
\pgfpathlineto{\pgfqpoint{6.096332in}{2.473589in}}%
\pgfpathlineto{\pgfqpoint{6.096157in}{1.873803in}}%
\pgfpathlineto{\pgfqpoint{6.097228in}{2.332022in}}%
\pgfpathlineto{\pgfqpoint{6.097926in}{1.862253in}}%
\pgfpathlineto{\pgfqpoint{6.097393in}{2.472893in}}%
\pgfpathlineto{\pgfqpoint{6.098339in}{2.345214in}}%
\pgfpathlineto{\pgfqpoint{6.099189in}{2.483916in}}%
\pgfpathlineto{\pgfqpoint{6.099309in}{1.793796in}}%
\pgfpathlineto{\pgfqpoint{6.099433in}{2.153879in}}%
\pgfpathlineto{\pgfqpoint{6.100082in}{1.931945in}}%
\pgfpathlineto{\pgfqpoint{6.099600in}{2.490516in}}%
\pgfpathlineto{\pgfqpoint{6.100542in}{2.334495in}}%
\pgfpathlineto{\pgfqpoint{6.100895in}{1.945198in}}%
\pgfpathlineto{\pgfqpoint{6.100950in}{2.469043in}}%
\pgfpathlineto{\pgfqpoint{6.101647in}{2.258919in}}%
\pgfpathlineto{\pgfqpoint{6.102120in}{2.481116in}}%
\pgfpathlineto{\pgfqpoint{6.101886in}{1.718959in}}%
\pgfpathlineto{\pgfqpoint{6.102758in}{2.261675in}}%
\pgfpathlineto{\pgfqpoint{6.102847in}{2.489411in}}%
\pgfpathlineto{\pgfqpoint{6.103103in}{1.789553in}}%
\pgfpathlineto{\pgfqpoint{6.103868in}{2.335579in}}%
\pgfpathlineto{\pgfqpoint{6.104734in}{1.738471in}}%
\pgfpathlineto{\pgfqpoint{6.104642in}{2.479767in}}%
\pgfpathlineto{\pgfqpoint{6.104978in}{2.325967in}}%
\pgfpathlineto{\pgfqpoint{6.105740in}{2.478915in}}%
\pgfpathlineto{\pgfqpoint{6.105930in}{1.899961in}}%
\pgfpathlineto{\pgfqpoint{6.106086in}{2.259977in}}%
\pgfpathlineto{\pgfqpoint{6.107179in}{2.528938in}}%
\pgfpathlineto{\pgfqpoint{6.106205in}{1.877697in}}%
\pgfpathlineto{\pgfqpoint{6.107184in}{2.380577in}}%
\pgfpathlineto{\pgfqpoint{6.107856in}{1.827271in}}%
\pgfpathlineto{\pgfqpoint{6.107502in}{2.472553in}}%
\pgfpathlineto{\pgfqpoint{6.108294in}{2.378701in}}%
\pgfpathlineto{\pgfqpoint{6.108799in}{2.479313in}}%
\pgfpathlineto{\pgfqpoint{6.108984in}{1.869194in}}%
\pgfpathlineto{\pgfqpoint{6.109399in}{2.349810in}}%
\pgfpathlineto{\pgfqpoint{6.110122in}{1.687530in}}%
\pgfpathlineto{\pgfqpoint{6.109834in}{2.486150in}}%
\pgfpathlineto{\pgfqpoint{6.110511in}{2.208422in}}%
\pgfpathlineto{\pgfqpoint{6.110539in}{2.473936in}}%
\pgfpathlineto{\pgfqpoint{6.110679in}{1.831682in}}%
\pgfpathlineto{\pgfqpoint{6.111624in}{2.452707in}}%
\pgfpathlineto{\pgfqpoint{6.112291in}{1.862974in}}%
\pgfpathlineto{\pgfqpoint{6.112441in}{2.468345in}}%
\pgfpathlineto{\pgfqpoint{6.112735in}{2.244057in}}%
\pgfpathlineto{\pgfqpoint{6.113501in}{2.478775in}}%
\pgfpathlineto{\pgfqpoint{6.113343in}{1.683750in}}%
\pgfpathlineto{\pgfqpoint{6.113844in}{2.306970in}}%
\pgfpathlineto{\pgfqpoint{6.114870in}{1.785021in}}%
\pgfpathlineto{\pgfqpoint{6.114798in}{2.481500in}}%
\pgfpathlineto{\pgfqpoint{6.114953in}{2.301652in}}%
\pgfpathlineto{\pgfqpoint{6.115127in}{2.496079in}}%
\pgfpathlineto{\pgfqpoint{6.115237in}{1.794748in}}%
\pgfpathlineto{\pgfqpoint{6.116064in}{2.365806in}}%
\pgfpathlineto{\pgfqpoint{6.116067in}{2.324909in}}%
\pgfpathlineto{\pgfqpoint{6.116072in}{2.428344in}}%
\pgfpathlineto{\pgfqpoint{6.116187in}{1.929758in}}%
\pgfpathlineto{\pgfqpoint{6.116918in}{2.490257in}}%
\pgfpathlineto{\pgfqpoint{6.117185in}{2.015756in}}%
\pgfpathlineto{\pgfqpoint{6.117650in}{2.492127in}}%
\pgfpathlineto{\pgfqpoint{6.117558in}{1.872319in}}%
\pgfpathlineto{\pgfqpoint{6.118296in}{2.308128in}}%
\pgfpathlineto{\pgfqpoint{6.118578in}{2.500479in}}%
\pgfpathlineto{\pgfqpoint{6.119326in}{1.745052in}}%
\pgfpathlineto{\pgfqpoint{6.119384in}{2.304146in}}%
\pgfpathlineto{\pgfqpoint{6.120212in}{1.821532in}}%
\pgfpathlineto{\pgfqpoint{6.120493in}{2.473021in}}%
\pgfpathlineto{\pgfqpoint{6.120494in}{2.287402in}}%
\pgfpathlineto{\pgfqpoint{6.121314in}{2.479649in}}%
\pgfpathlineto{\pgfqpoint{6.121274in}{1.705350in}}%
\pgfpathlineto{\pgfqpoint{6.121602in}{2.252754in}}%
\pgfpathlineto{\pgfqpoint{6.122547in}{2.530986in}}%
\pgfpathlineto{\pgfqpoint{6.121847in}{1.835186in}}%
\pgfpathlineto{\pgfqpoint{6.122696in}{2.379295in}}%
\pgfpathlineto{\pgfqpoint{6.123511in}{1.823825in}}%
\pgfpathlineto{\pgfqpoint{6.123326in}{2.472587in}}%
\pgfpathlineto{\pgfqpoint{6.123807in}{2.277831in}}%
\pgfpathlineto{\pgfqpoint{6.123987in}{1.852117in}}%
\pgfpathlineto{\pgfqpoint{6.124839in}{2.500514in}}%
\pgfpathlineto{\pgfqpoint{6.124917in}{2.287887in}}%
\pgfpathlineto{\pgfqpoint{6.125617in}{2.496984in}}%
\pgfpathlineto{\pgfqpoint{6.125271in}{1.444944in}}%
\pgfpathlineto{\pgfqpoint{6.126028in}{2.310010in}}%
\pgfpathlineto{\pgfqpoint{6.126288in}{1.852136in}}%
\pgfpathlineto{\pgfqpoint{6.126887in}{2.488312in}}%
\pgfpathlineto{\pgfqpoint{6.127129in}{2.248306in}}%
\pgfpathlineto{\pgfqpoint{6.127440in}{2.496533in}}%
\pgfpathlineto{\pgfqpoint{6.127888in}{1.974040in}}%
\pgfpathlineto{\pgfqpoint{6.128242in}{2.414267in}}%
\pgfpathlineto{\pgfqpoint{6.129238in}{2.484097in}}%
\pgfpathlineto{\pgfqpoint{6.128376in}{1.717119in}}%
\pgfpathlineto{\pgfqpoint{6.129343in}{2.305513in}}%
\pgfpathlineto{\pgfqpoint{6.129576in}{1.851949in}}%
\pgfpathlineto{\pgfqpoint{6.129593in}{2.480306in}}%
\pgfpathlineto{\pgfqpoint{6.130455in}{2.220623in}}%
\pgfpathlineto{\pgfqpoint{6.131402in}{2.502976in}}%
\pgfpathlineto{\pgfqpoint{6.130864in}{1.703035in}}%
\pgfpathlineto{\pgfqpoint{6.131567in}{2.366166in}}%
\pgfpathlineto{\pgfqpoint{6.131632in}{1.818894in}}%
\pgfpathlineto{\pgfqpoint{6.132499in}{2.503339in}}%
\pgfpathlineto{\pgfqpoint{6.132679in}{2.225126in}}%
\pgfpathlineto{\pgfqpoint{6.133610in}{2.485196in}}%
\pgfpathlineto{\pgfqpoint{6.133036in}{1.866442in}}%
\pgfpathlineto{\pgfqpoint{6.133789in}{2.298557in}}%
\pgfpathlineto{\pgfqpoint{6.134408in}{1.893887in}}%
\pgfpathlineto{\pgfqpoint{6.134543in}{2.501048in}}%
\pgfpathlineto{\pgfqpoint{6.134897in}{2.378043in}}%
\pgfpathlineto{\pgfqpoint{6.135969in}{1.779247in}}%
\pgfpathlineto{\pgfqpoint{6.135745in}{2.490787in}}%
\pgfpathlineto{\pgfqpoint{6.136006in}{2.258753in}}%
\pgfpathlineto{\pgfqpoint{6.136345in}{2.514099in}}%
\pgfpathlineto{\pgfqpoint{6.136226in}{1.776720in}}%
\pgfpathlineto{\pgfqpoint{6.137118in}{2.401413in}}%
\pgfpathlineto{\pgfqpoint{6.138147in}{1.695161in}}%
\pgfpathlineto{\pgfqpoint{6.138036in}{2.499185in}}%
\pgfpathlineto{\pgfqpoint{6.138227in}{2.273889in}}%
\pgfpathlineto{\pgfqpoint{6.138497in}{2.489979in}}%
\pgfpathlineto{\pgfqpoint{6.139254in}{1.763554in}}%
\pgfpathlineto{\pgfqpoint{6.139336in}{2.400902in}}%
\pgfpathlineto{\pgfqpoint{6.139457in}{1.735135in}}%
\pgfpathlineto{\pgfqpoint{6.139621in}{2.489153in}}%
\pgfpathlineto{\pgfqpoint{6.140446in}{2.436418in}}%
\pgfpathlineto{\pgfqpoint{6.141546in}{1.782345in}}%
\pgfpathlineto{\pgfqpoint{6.140842in}{2.504968in}}%
\pgfpathlineto{\pgfqpoint{6.141556in}{2.392158in}}%
\pgfpathlineto{\pgfqpoint{6.141558in}{2.491255in}}%
\pgfpathlineto{\pgfqpoint{6.141959in}{1.769073in}}%
\pgfpathlineto{\pgfqpoint{6.142665in}{2.460737in}}%
\pgfpathlineto{\pgfqpoint{6.142786in}{1.838381in}}%
\pgfpathlineto{\pgfqpoint{6.143283in}{2.504983in}}%
\pgfpathlineto{\pgfqpoint{6.143777in}{2.156320in}}%
\pgfpathlineto{\pgfqpoint{6.144841in}{2.484499in}}%
\pgfpathlineto{\pgfqpoint{6.144028in}{1.885189in}}%
\pgfpathlineto{\pgfqpoint{6.144888in}{2.380460in}}%
\pgfpathlineto{\pgfqpoint{6.145085in}{1.811377in}}%
\pgfpathlineto{\pgfqpoint{6.145137in}{2.504542in}}%
\pgfpathlineto{\pgfqpoint{6.146001in}{2.223094in}}%
\pgfpathlineto{\pgfqpoint{6.146149in}{2.507741in}}%
\pgfpathlineto{\pgfqpoint{6.146713in}{1.947156in}}%
\pgfpathlineto{\pgfqpoint{6.147113in}{2.396446in}}%
\pgfpathlineto{\pgfqpoint{6.147669in}{1.920643in}}%
\pgfpathlineto{\pgfqpoint{6.147965in}{2.483815in}}%
\pgfpathlineto{\pgfqpoint{6.148224in}{2.292082in}}%
\pgfpathlineto{\pgfqpoint{6.148654in}{2.482107in}}%
\pgfpathlineto{\pgfqpoint{6.148740in}{1.897643in}}%
\pgfpathlineto{\pgfqpoint{6.149319in}{2.273542in}}%
\pgfpathlineto{\pgfqpoint{6.149410in}{1.740681in}}%
\pgfpathlineto{\pgfqpoint{6.150138in}{2.524790in}}%
\pgfpathlineto{\pgfqpoint{6.150427in}{2.342355in}}%
\pgfpathlineto{\pgfqpoint{6.151245in}{2.489000in}}%
\pgfpathlineto{\pgfqpoint{6.151264in}{1.907190in}}%
\pgfpathlineto{\pgfqpoint{6.151537in}{2.410484in}}%
\pgfpathlineto{\pgfqpoint{6.152258in}{1.912356in}}%
\pgfpathlineto{\pgfqpoint{6.152247in}{2.485065in}}%
\pgfpathlineto{\pgfqpoint{6.152647in}{2.440503in}}%
\pgfpathlineto{\pgfqpoint{6.152842in}{2.509702in}}%
\pgfpathlineto{\pgfqpoint{6.152949in}{1.746377in}}%
\pgfpathlineto{\pgfqpoint{6.153754in}{2.392927in}}%
\pgfpathlineto{\pgfqpoint{6.154735in}{1.871960in}}%
\pgfpathlineto{\pgfqpoint{6.154599in}{2.512775in}}%
\pgfpathlineto{\pgfqpoint{6.154866in}{2.266282in}}%
\pgfpathlineto{\pgfqpoint{6.155060in}{2.493663in}}%
\pgfpathlineto{\pgfqpoint{6.155819in}{1.824056in}}%
\pgfpathlineto{\pgfqpoint{6.155979in}{2.442861in}}%
\pgfpathlineto{\pgfqpoint{6.156367in}{1.728763in}}%
\pgfpathlineto{\pgfqpoint{6.156296in}{2.488710in}}%
\pgfpathlineto{\pgfqpoint{6.157089in}{2.404191in}}%
\pgfpathlineto{\pgfqpoint{6.157120in}{1.848265in}}%
\pgfpathlineto{\pgfqpoint{6.157850in}{2.502175in}}%
\pgfpathlineto{\pgfqpoint{6.158202in}{2.273010in}}%
\pgfpathlineto{\pgfqpoint{6.159015in}{2.484741in}}%
\pgfpathlineto{\pgfqpoint{6.159223in}{1.853772in}}%
\pgfpathlineto{\pgfqpoint{6.159314in}{2.380745in}}%
\pgfpathlineto{\pgfqpoint{6.159705in}{2.497501in}}%
\pgfpathlineto{\pgfqpoint{6.159377in}{1.750904in}}%
\pgfpathlineto{\pgfqpoint{6.160382in}{2.382009in}}%
\pgfpathlineto{\pgfqpoint{6.160440in}{1.802840in}}%
\pgfpathlineto{\pgfqpoint{6.160625in}{2.521796in}}%
\pgfpathlineto{\pgfqpoint{6.161492in}{2.193856in}}%
\pgfpathlineto{\pgfqpoint{6.162357in}{2.510843in}}%
\pgfpathlineto{\pgfqpoint{6.161880in}{1.764011in}}%
\pgfpathlineto{\pgfqpoint{6.162604in}{2.384034in}}%
\pgfpathlineto{\pgfqpoint{6.162978in}{2.498409in}}%
\pgfpathlineto{\pgfqpoint{6.163624in}{1.812593in}}%
\pgfpathlineto{\pgfqpoint{6.163713in}{2.372580in}}%
\pgfpathlineto{\pgfqpoint{6.164542in}{1.901373in}}%
\pgfpathlineto{\pgfqpoint{6.164262in}{2.500196in}}%
\pgfpathlineto{\pgfqpoint{6.164824in}{2.267193in}}%
\pgfpathlineto{\pgfqpoint{6.165535in}{2.507885in}}%
\pgfpathlineto{\pgfqpoint{6.165202in}{1.928179in}}%
\pgfpathlineto{\pgfqpoint{6.165934in}{2.319621in}}%
\pgfpathlineto{\pgfqpoint{6.167042in}{1.842344in}}%
\pgfpathlineto{\pgfqpoint{6.165943in}{2.509364in}}%
\pgfpathlineto{\pgfqpoint{6.167045in}{2.184783in}}%
\pgfpathlineto{\pgfqpoint{6.167523in}{2.512622in}}%
\pgfpathlineto{\pgfqpoint{6.167623in}{1.910418in}}%
\pgfpathlineto{\pgfqpoint{6.168158in}{2.401416in}}%
\pgfpathlineto{\pgfqpoint{6.168597in}{1.839916in}}%
\pgfpathlineto{\pgfqpoint{6.168380in}{2.503348in}}%
\pgfpathlineto{\pgfqpoint{6.169268in}{2.431821in}}%
\pgfpathlineto{\pgfqpoint{6.169973in}{2.518503in}}%
\pgfpathlineto{\pgfqpoint{6.169809in}{1.877907in}}%
\pgfpathlineto{\pgfqpoint{6.170350in}{2.437034in}}%
\pgfpathlineto{\pgfqpoint{6.171080in}{1.758111in}}%
\pgfpathlineto{\pgfqpoint{6.171090in}{2.504467in}}%
\pgfpathlineto{\pgfqpoint{6.171460in}{2.337998in}}%
\pgfpathlineto{\pgfqpoint{6.171951in}{1.759257in}}%
\pgfpathlineto{\pgfqpoint{6.172460in}{2.522365in}}%
\pgfpathlineto{\pgfqpoint{6.172571in}{2.293902in}}%
\pgfpathlineto{\pgfqpoint{6.173047in}{2.499493in}}%
\pgfpathlineto{\pgfqpoint{6.173009in}{1.739213in}}%
\pgfpathlineto{\pgfqpoint{6.173681in}{2.462418in}}%
\pgfpathlineto{\pgfqpoint{6.174056in}{1.860070in}}%
\pgfpathlineto{\pgfqpoint{6.174501in}{2.530768in}}%
\pgfpathlineto{\pgfqpoint{6.174792in}{2.337660in}}%
\pgfpathlineto{\pgfqpoint{6.175008in}{1.997264in}}%
\pgfpathlineto{\pgfqpoint{6.175214in}{2.504281in}}%
\pgfpathlineto{\pgfqpoint{6.175900in}{2.304190in}}%
\pgfpathlineto{\pgfqpoint{6.176439in}{2.510624in}}%
\pgfpathlineto{\pgfqpoint{6.175959in}{1.874662in}}%
\pgfpathlineto{\pgfqpoint{6.177010in}{2.293574in}}%
\pgfpathlineto{\pgfqpoint{6.177859in}{1.731274in}}%
\pgfpathlineto{\pgfqpoint{6.177236in}{2.515036in}}%
\pgfpathlineto{\pgfqpoint{6.178118in}{2.275910in}}%
\pgfpathlineto{\pgfqpoint{6.178936in}{2.521613in}}%
\pgfpathlineto{\pgfqpoint{6.178140in}{1.788823in}}%
\pgfpathlineto{\pgfqpoint{6.179232in}{2.384772in}}%
\pgfpathlineto{\pgfqpoint{6.179271in}{1.765876in}}%
\pgfpathlineto{\pgfqpoint{6.179908in}{2.520982in}}%
\pgfpathlineto{\pgfqpoint{6.180343in}{2.364423in}}%
\pgfpathlineto{\pgfqpoint{6.180360in}{1.851336in}}%
\pgfpathlineto{\pgfqpoint{6.180850in}{2.528219in}}%
\pgfpathlineto{\pgfqpoint{6.181453in}{2.229272in}}%
\pgfpathlineto{\pgfqpoint{6.181626in}{2.502301in}}%
\pgfpathlineto{\pgfqpoint{6.182298in}{1.703939in}}%
\pgfpathlineto{\pgfqpoint{6.182563in}{2.272948in}}%
\pgfpathlineto{\pgfqpoint{6.183026in}{2.486649in}}%
\pgfpathlineto{\pgfqpoint{6.182721in}{1.883281in}}%
\pgfpathlineto{\pgfqpoint{6.183671in}{2.417405in}}%
\pgfpathlineto{\pgfqpoint{6.184780in}{1.897135in}}%
\pgfpathlineto{\pgfqpoint{6.184061in}{2.500759in}}%
\pgfpathlineto{\pgfqpoint{6.184782in}{2.367135in}}%
\pgfpathlineto{\pgfqpoint{6.185852in}{1.891898in}}%
\pgfpathlineto{\pgfqpoint{6.185617in}{2.496150in}}%
\pgfpathlineto{\pgfqpoint{6.185883in}{2.361769in}}%
\pgfpathlineto{\pgfqpoint{6.186718in}{2.524836in}}%
\pgfpathlineto{\pgfqpoint{6.186302in}{1.902249in}}%
\pgfpathlineto{\pgfqpoint{6.186990in}{2.356164in}}%
\pgfpathlineto{\pgfqpoint{6.187147in}{1.894491in}}%
\pgfpathlineto{\pgfqpoint{6.187813in}{2.514717in}}%
\pgfpathlineto{\pgfqpoint{6.188100in}{2.461652in}}%
\pgfpathlineto{\pgfqpoint{6.188546in}{1.813403in}}%
\pgfpathlineto{\pgfqpoint{6.188260in}{2.505969in}}%
\pgfpathlineto{\pgfqpoint{6.189210in}{2.395944in}}%
\pgfpathlineto{\pgfqpoint{6.190248in}{2.505866in}}%
\pgfpathlineto{\pgfqpoint{6.189323in}{1.689062in}}%
\pgfpathlineto{\pgfqpoint{6.190319in}{2.364229in}}%
\pgfpathlineto{\pgfqpoint{6.191253in}{1.797578in}}%
\pgfpathlineto{\pgfqpoint{6.190400in}{2.502252in}}%
\pgfpathlineto{\pgfqpoint{6.191427in}{2.313176in}}%
\pgfpathlineto{\pgfqpoint{6.191856in}{2.517002in}}%
\pgfpathlineto{\pgfqpoint{6.191607in}{1.971989in}}%
\pgfpathlineto{\pgfqpoint{6.192536in}{2.477261in}}%
\pgfpathlineto{\pgfqpoint{6.193029in}{1.865483in}}%
\pgfpathlineto{\pgfqpoint{6.193400in}{2.508869in}}%
\pgfpathlineto{\pgfqpoint{6.193647in}{2.304599in}}%
\pgfpathlineto{\pgfqpoint{6.193936in}{2.520113in}}%
\pgfpathlineto{\pgfqpoint{6.194165in}{1.813907in}}%
\pgfpathlineto{\pgfqpoint{6.194757in}{2.302559in}}%
\pgfpathlineto{\pgfqpoint{6.195188in}{1.849526in}}%
\pgfpathlineto{\pgfqpoint{6.195406in}{2.509956in}}%
\pgfpathlineto{\pgfqpoint{6.195867in}{1.978380in}}%
\pgfpathlineto{\pgfqpoint{6.196116in}{2.537270in}}%
\pgfpathlineto{\pgfqpoint{6.195889in}{1.896634in}}%
\pgfpathlineto{\pgfqpoint{6.196979in}{2.451983in}}%
\pgfpathlineto{\pgfqpoint{6.197346in}{1.877809in}}%
\pgfpathlineto{\pgfqpoint{6.197378in}{2.533483in}}%
\pgfpathlineto{\pgfqpoint{6.198090in}{2.395596in}}%
\pgfpathlineto{\pgfqpoint{6.199127in}{1.922341in}}%
\pgfpathlineto{\pgfqpoint{6.198629in}{2.531771in}}%
\pgfpathlineto{\pgfqpoint{6.199203in}{2.230189in}}%
\pgfpathlineto{\pgfqpoint{6.199626in}{2.519552in}}%
\pgfpathlineto{\pgfqpoint{6.200219in}{1.922075in}}%
\pgfpathlineto{\pgfqpoint{6.200315in}{2.445986in}}%
\pgfpathlineto{\pgfqpoint{6.201021in}{1.876692in}}%
\pgfpathlineto{\pgfqpoint{6.201367in}{2.545232in}}%
\pgfpathlineto{\pgfqpoint{6.201428in}{2.306053in}}%
\pgfpathlineto{\pgfqpoint{6.201703in}{2.521944in}}%
\pgfpathlineto{\pgfqpoint{6.202211in}{1.932641in}}%
\pgfpathlineto{\pgfqpoint{6.202530in}{2.393751in}}%
\pgfpathlineto{\pgfqpoint{6.203198in}{1.912502in}}%
\pgfpathlineto{\pgfqpoint{6.203335in}{2.510028in}}%
\pgfpathlineto{\pgfqpoint{6.203640in}{2.238039in}}%
\pgfpathlineto{\pgfqpoint{6.203780in}{2.521925in}}%
\pgfpathlineto{\pgfqpoint{6.204407in}{1.966753in}}%
\pgfpathlineto{\pgfqpoint{6.204751in}{2.374793in}}%
\pgfpathlineto{\pgfqpoint{6.204830in}{1.648999in}}%
\pgfpathlineto{\pgfqpoint{6.204858in}{2.529948in}}%
\pgfpathlineto{\pgfqpoint{6.205863in}{2.246642in}}%
\pgfpathlineto{\pgfqpoint{6.206352in}{2.517019in}}%
\pgfpathlineto{\pgfqpoint{6.206711in}{1.827999in}}%
\pgfpathlineto{\pgfqpoint{6.206974in}{2.400972in}}%
\pgfpathlineto{\pgfqpoint{6.207703in}{1.755083in}}%
\pgfpathlineto{\pgfqpoint{6.207933in}{2.523440in}}%
\pgfpathlineto{\pgfqpoint{6.208086in}{2.281954in}}%
\pgfpathlineto{\pgfqpoint{6.208426in}{2.514905in}}%
\pgfpathlineto{\pgfqpoint{6.208462in}{1.895818in}}%
\pgfpathlineto{\pgfqpoint{6.209197in}{2.387920in}}%
\pgfpathlineto{\pgfqpoint{6.209794in}{1.912645in}}%
\pgfpathlineto{\pgfqpoint{6.209439in}{2.519955in}}%
\pgfpathlineto{\pgfqpoint{6.210309in}{2.131183in}}%
\pgfpathlineto{\pgfqpoint{6.210704in}{2.514046in}}%
\pgfpathlineto{\pgfqpoint{6.210350in}{1.883123in}}%
\pgfpathlineto{\pgfqpoint{6.211421in}{2.316579in}}%
\pgfpathlineto{\pgfqpoint{6.211638in}{1.889876in}}%
\pgfpathlineto{\pgfqpoint{6.211557in}{2.528479in}}%
\pgfpathlineto{\pgfqpoint{6.212529in}{2.425623in}}%
\pgfpathlineto{\pgfqpoint{6.212927in}{1.920233in}}%
\pgfpathlineto{\pgfqpoint{6.213016in}{2.525006in}}%
\pgfpathlineto{\pgfqpoint{6.213641in}{2.372986in}}%
\pgfpathlineto{\pgfqpoint{6.214602in}{2.535171in}}%
\pgfpathlineto{\pgfqpoint{6.214293in}{1.789043in}}%
\pgfpathlineto{\pgfqpoint{6.214747in}{2.271242in}}%
\pgfpathlineto{\pgfqpoint{6.215547in}{1.780461in}}%
\pgfpathlineto{\pgfqpoint{6.215012in}{2.522383in}}%
\pgfpathlineto{\pgfqpoint{6.215857in}{2.423697in}}%
\pgfpathlineto{\pgfqpoint{6.215973in}{1.797788in}}%
\pgfpathlineto{\pgfqpoint{6.216589in}{2.512095in}}%
\pgfpathlineto{\pgfqpoint{6.216967in}{2.356474in}}%
\pgfpathlineto{\pgfqpoint{6.217779in}{2.538111in}}%
\pgfpathlineto{\pgfqpoint{6.217516in}{1.878673in}}%
\pgfpathlineto{\pgfqpoint{6.218078in}{2.386312in}}%
\pgfpathlineto{\pgfqpoint{6.218396in}{1.883798in}}%
\pgfpathlineto{\pgfqpoint{6.218484in}{2.525545in}}%
\pgfpathlineto{\pgfqpoint{6.219188in}{2.337785in}}%
\pgfpathlineto{\pgfqpoint{6.220080in}{2.544869in}}%
\pgfpathlineto{\pgfqpoint{6.220278in}{1.875427in}}%
\pgfpathlineto{\pgfqpoint{6.220299in}{2.403980in}}%
\pgfpathlineto{\pgfqpoint{6.220500in}{1.769498in}}%
\pgfpathlineto{\pgfqpoint{6.220905in}{2.523117in}}%
\pgfpathlineto{\pgfqpoint{6.221409in}{2.395927in}}%
\pgfpathlineto{\pgfqpoint{6.222262in}{2.524337in}}%
\pgfpathlineto{\pgfqpoint{6.221894in}{2.005304in}}%
\pgfpathlineto{\pgfqpoint{6.222517in}{2.340922in}}%
\pgfpathlineto{\pgfqpoint{6.222683in}{1.869665in}}%
\pgfpathlineto{\pgfqpoint{6.223383in}{2.518396in}}%
\pgfpathlineto{\pgfqpoint{6.223629in}{2.278600in}}%
\pgfpathlineto{\pgfqpoint{6.223858in}{1.895092in}}%
\pgfpathlineto{\pgfqpoint{6.224558in}{2.527118in}}%
\pgfpathlineto{\pgfqpoint{6.224734in}{2.217771in}}%
\pgfpathlineto{\pgfqpoint{6.225624in}{2.522064in}}%
\pgfpathlineto{\pgfqpoint{6.225284in}{1.772349in}}%
\pgfpathlineto{\pgfqpoint{6.225845in}{2.441840in}}%
\pgfpathlineto{\pgfqpoint{6.226713in}{2.019318in}}%
\pgfpathlineto{\pgfqpoint{6.226625in}{2.531363in}}%
\pgfpathlineto{\pgfqpoint{6.226956in}{2.355230in}}%
\pgfpathlineto{\pgfqpoint{6.227929in}{2.538021in}}%
\pgfpathlineto{\pgfqpoint{6.227913in}{1.883597in}}%
\pgfpathlineto{\pgfqpoint{6.228064in}{2.461068in}}%
\pgfpathlineto{\pgfqpoint{6.228637in}{1.606795in}}%
\pgfpathlineto{\pgfqpoint{6.228832in}{2.537080in}}%
\pgfpathlineto{\pgfqpoint{6.229174in}{2.336162in}}%
\pgfpathlineto{\pgfqpoint{6.229175in}{2.538879in}}%
\pgfpathlineto{\pgfqpoint{6.229207in}{1.824419in}}%
\pgfpathlineto{\pgfqpoint{6.230284in}{2.288645in}}%
\pgfpathlineto{\pgfqpoint{6.230743in}{1.635059in}}%
\pgfpathlineto{\pgfqpoint{6.231151in}{2.555721in}}%
\pgfpathlineto{\pgfqpoint{6.231394in}{2.345889in}}%
\pgfpathlineto{\pgfqpoint{6.232498in}{2.534581in}}%
\pgfpathlineto{\pgfqpoint{6.232211in}{1.967027in}}%
\pgfpathlineto{\pgfqpoint{6.232507in}{2.398714in}}%
\pgfpathlineto{\pgfqpoint{6.232716in}{1.822932in}}%
\pgfpathlineto{\pgfqpoint{6.232914in}{2.576095in}}%
\pgfpathlineto{\pgfqpoint{6.233622in}{2.309176in}}%
\pgfpathlineto{\pgfqpoint{6.233760in}{2.520915in}}%
\pgfpathlineto{\pgfqpoint{6.233981in}{1.888264in}}%
\pgfpathlineto{\pgfqpoint{6.234731in}{2.456281in}}%
\pgfpathlineto{\pgfqpoint{6.235687in}{1.829281in}}%
\pgfpathlineto{\pgfqpoint{6.235118in}{2.563075in}}%
\pgfpathlineto{\pgfqpoint{6.235843in}{2.242060in}}%
\pgfpathlineto{\pgfqpoint{6.236285in}{2.530898in}}%
\pgfpathlineto{\pgfqpoint{6.236594in}{1.832322in}}%
\pgfpathlineto{\pgfqpoint{6.236923in}{2.282822in}}%
\pgfpathlineto{\pgfqpoint{6.237175in}{1.894100in}}%
\pgfpathlineto{\pgfqpoint{6.237701in}{2.533837in}}%
\pgfpathlineto{\pgfqpoint{6.238032in}{2.284259in}}%
\pgfpathlineto{\pgfqpoint{6.238616in}{2.522868in}}%
\pgfpathlineto{\pgfqpoint{6.238526in}{1.754629in}}%
\pgfpathlineto{\pgfqpoint{6.239143in}{2.445685in}}%
\pgfpathlineto{\pgfqpoint{6.239988in}{1.879314in}}%
\pgfpathlineto{\pgfqpoint{6.240119in}{2.543612in}}%
\pgfpathlineto{\pgfqpoint{6.240255in}{2.261727in}}%
\pgfpathlineto{\pgfqpoint{6.240757in}{2.551605in}}%
\pgfpathlineto{\pgfqpoint{6.241093in}{1.832560in}}%
\pgfpathlineto{\pgfqpoint{6.241364in}{2.334378in}}%
\pgfpathlineto{\pgfqpoint{6.242116in}{1.873173in}}%
\pgfpathlineto{\pgfqpoint{6.242346in}{2.540963in}}%
\pgfpathlineto{\pgfqpoint{6.242476in}{2.206963in}}%
\pgfpathlineto{\pgfqpoint{6.243041in}{2.530709in}}%
\pgfpathlineto{\pgfqpoint{6.243444in}{1.936497in}}%
\pgfpathlineto{\pgfqpoint{6.243588in}{2.432910in}}%
\pgfpathlineto{\pgfqpoint{6.244037in}{1.905836in}}%
\pgfpathlineto{\pgfqpoint{6.244607in}{2.534715in}}%
\pgfpathlineto{\pgfqpoint{6.244700in}{2.157969in}}%
\pgfpathlineto{\pgfqpoint{6.244720in}{2.546804in}}%
\pgfpathlineto{\pgfqpoint{6.244814in}{1.750693in}}%
\pgfpathlineto{\pgfqpoint{6.245811in}{2.385614in}}%
\pgfpathlineto{\pgfqpoint{6.246361in}{1.831850in}}%
\pgfpathlineto{\pgfqpoint{6.246848in}{2.547472in}}%
\pgfpathlineto{\pgfqpoint{6.246923in}{2.232843in}}%
\pgfpathlineto{\pgfqpoint{6.247228in}{2.545294in}}%
\pgfpathlineto{\pgfqpoint{6.247443in}{1.898774in}}%
\pgfpathlineto{\pgfqpoint{6.248034in}{2.333856in}}%
\pgfpathlineto{\pgfqpoint{6.248756in}{1.966963in}}%
\pgfpathlineto{\pgfqpoint{6.248695in}{2.546654in}}%
\pgfpathlineto{\pgfqpoint{6.249143in}{2.249501in}}%
\pgfpathlineto{\pgfqpoint{6.249271in}{2.531690in}}%
\pgfpathlineto{\pgfqpoint{6.249897in}{1.378395in}}%
\pgfpathlineto{\pgfqpoint{6.250256in}{2.450739in}}%
\pgfpathlineto{\pgfqpoint{6.250916in}{1.817712in}}%
\pgfpathlineto{\pgfqpoint{6.250300in}{2.529235in}}%
\pgfpathlineto{\pgfqpoint{6.251367in}{2.432945in}}%
\pgfpathlineto{\pgfqpoint{6.251735in}{1.912566in}}%
\pgfpathlineto{\pgfqpoint{6.252433in}{2.531137in}}%
\pgfpathlineto{\pgfqpoint{6.252478in}{2.288723in}}%
\pgfpathlineto{\pgfqpoint{6.252760in}{2.544657in}}%
\pgfpathlineto{\pgfqpoint{6.252518in}{1.947271in}}%
\pgfpathlineto{\pgfqpoint{6.253588in}{2.334631in}}%
\pgfpathlineto{\pgfqpoint{6.254156in}{1.800369in}}%
\pgfpathlineto{\pgfqpoint{6.254162in}{2.525413in}}%
\pgfpathlineto{\pgfqpoint{6.254697in}{2.264404in}}%
\pgfpathlineto{\pgfqpoint{6.254843in}{2.552858in}}%
\pgfpathlineto{\pgfqpoint{6.255245in}{1.707804in}}%
\pgfpathlineto{\pgfqpoint{6.255809in}{2.370432in}}%
\pgfpathlineto{\pgfqpoint{6.256206in}{2.532229in}}%
\pgfpathlineto{\pgfqpoint{6.256385in}{1.866774in}}%
\pgfpathlineto{\pgfqpoint{6.256917in}{2.370544in}}%
\pgfpathlineto{\pgfqpoint{6.257307in}{1.804161in}}%
\pgfpathlineto{\pgfqpoint{6.257420in}{2.548959in}}%
\pgfpathlineto{\pgfqpoint{6.258027in}{2.331865in}}%
\pgfpathlineto{\pgfqpoint{6.258986in}{2.566658in}}%
\pgfpathlineto{\pgfqpoint{6.258322in}{1.843203in}}%
\pgfpathlineto{\pgfqpoint{6.259137in}{2.352824in}}%
\pgfpathlineto{\pgfqpoint{6.259182in}{1.885680in}}%
\pgfpathlineto{\pgfqpoint{6.259764in}{2.542423in}}%
\pgfpathlineto{\pgfqpoint{6.260247in}{2.345655in}}%
\pgfpathlineto{\pgfqpoint{6.260847in}{2.539536in}}%
\pgfpathlineto{\pgfqpoint{6.261255in}{1.936458in}}%
\pgfpathlineto{\pgfqpoint{6.261356in}{2.417330in}}%
\pgfpathlineto{\pgfqpoint{6.261545in}{1.821000in}}%
\pgfpathlineto{\pgfqpoint{6.261978in}{2.571583in}}%
\pgfpathlineto{\pgfqpoint{6.262467in}{2.360326in}}%
\pgfpathlineto{\pgfqpoint{6.263030in}{1.793660in}}%
\pgfpathlineto{\pgfqpoint{6.262854in}{2.544035in}}%
\pgfpathlineto{\pgfqpoint{6.263576in}{2.308423in}}%
\pgfpathlineto{\pgfqpoint{6.263894in}{2.563311in}}%
\pgfpathlineto{\pgfqpoint{6.264556in}{1.929588in}}%
\pgfpathlineto{\pgfqpoint{6.264687in}{2.394151in}}%
\pgfpathlineto{\pgfqpoint{6.264961in}{1.825227in}}%
\pgfpathlineto{\pgfqpoint{6.265069in}{2.526307in}}%
\pgfpathlineto{\pgfqpoint{6.265798in}{2.300482in}}%
\pgfpathlineto{\pgfqpoint{6.266332in}{2.543764in}}%
\pgfpathlineto{\pgfqpoint{6.266318in}{1.667423in}}%
\pgfpathlineto{\pgfqpoint{6.266909in}{2.353000in}}%
\pgfpathlineto{\pgfqpoint{6.267465in}{1.661890in}}%
\pgfpathlineto{\pgfqpoint{6.267011in}{2.568073in}}%
\pgfpathlineto{\pgfqpoint{6.268017in}{2.372097in}}%
\pgfpathlineto{\pgfqpoint{6.268636in}{2.548705in}}%
\pgfpathlineto{\pgfqpoint{6.268888in}{1.919549in}}%
\pgfpathlineto{\pgfqpoint{6.269126in}{2.322884in}}%
\pgfpathlineto{\pgfqpoint{6.270025in}{1.879757in}}%
\pgfpathlineto{\pgfqpoint{6.269278in}{2.578997in}}%
\pgfpathlineto{\pgfqpoint{6.270237in}{2.279137in}}%
\pgfpathlineto{\pgfqpoint{6.270833in}{2.528890in}}%
\pgfpathlineto{\pgfqpoint{6.271261in}{1.964133in}}%
\pgfpathlineto{\pgfqpoint{6.271349in}{2.400880in}}%
\pgfpathlineto{\pgfqpoint{6.271661in}{1.794545in}}%
\pgfpathlineto{\pgfqpoint{6.271477in}{2.558779in}}%
\pgfpathlineto{\pgfqpoint{6.272460in}{2.390431in}}%
\pgfpathlineto{\pgfqpoint{6.273165in}{2.547922in}}%
\pgfpathlineto{\pgfqpoint{6.273264in}{1.813104in}}%
\pgfpathlineto{\pgfqpoint{6.273570in}{2.466940in}}%
\pgfpathlineto{\pgfqpoint{6.274372in}{1.760117in}}%
\pgfpathlineto{\pgfqpoint{6.273726in}{2.560112in}}%
\pgfpathlineto{\pgfqpoint{6.274683in}{2.132351in}}%
\pgfpathlineto{\pgfqpoint{6.274987in}{2.543669in}}%
\pgfpathlineto{\pgfqpoint{6.275710in}{1.770077in}}%
\pgfpathlineto{\pgfqpoint{6.275794in}{2.446814in}}%
\pgfpathlineto{\pgfqpoint{6.276742in}{1.877301in}}%
\pgfpathlineto{\pgfqpoint{6.276643in}{2.536459in}}%
\pgfpathlineto{\pgfqpoint{6.276906in}{2.296289in}}%
\pgfpathlineto{\pgfqpoint{6.277793in}{2.555378in}}%
\pgfpathlineto{\pgfqpoint{6.276967in}{1.844083in}}%
\pgfpathlineto{\pgfqpoint{6.278017in}{2.384686in}}%
\pgfpathlineto{\pgfqpoint{6.278704in}{1.926139in}}%
\pgfpathlineto{\pgfqpoint{6.279050in}{2.558692in}}%
\pgfpathlineto{\pgfqpoint{6.279128in}{2.310242in}}%
\pgfpathlineto{\pgfqpoint{6.279455in}{2.567153in}}%
\pgfpathlineto{\pgfqpoint{6.279140in}{1.445669in}}%
\pgfpathlineto{\pgfqpoint{6.280238in}{2.425962in}}%
\pgfpathlineto{\pgfqpoint{6.280460in}{1.850449in}}%
\pgfpathlineto{\pgfqpoint{6.281253in}{2.551031in}}%
\pgfpathlineto{\pgfqpoint{6.281349in}{2.325190in}}%
\pgfpathlineto{\pgfqpoint{6.281360in}{1.962828in}}%
\pgfpathlineto{\pgfqpoint{6.282189in}{2.562675in}}%
\pgfpathlineto{\pgfqpoint{6.282459in}{2.382104in}}%
\pgfpathlineto{\pgfqpoint{6.282528in}{2.550776in}}%
\pgfpathlineto{\pgfqpoint{6.283385in}{1.861112in}}%
\pgfpathlineto{\pgfqpoint{6.283558in}{2.351131in}}%
\pgfpathlineto{\pgfqpoint{6.284660in}{1.693122in}}%
\pgfpathlineto{\pgfqpoint{6.284489in}{2.539622in}}%
\pgfpathlineto{\pgfqpoint{6.284668in}{2.402072in}}%
\pgfpathlineto{\pgfqpoint{6.285477in}{1.810468in}}%
\pgfpathlineto{\pgfqpoint{6.285028in}{2.560282in}}%
\pgfpathlineto{\pgfqpoint{6.285778in}{2.259424in}}%
\pgfpathlineto{\pgfqpoint{6.286547in}{2.545520in}}%
\pgfpathlineto{\pgfqpoint{6.285795in}{2.005822in}}%
\pgfpathlineto{\pgfqpoint{6.286888in}{2.377351in}}%
\pgfpathlineto{\pgfqpoint{6.287148in}{1.873526in}}%
\pgfpathlineto{\pgfqpoint{6.287121in}{2.536267in}}%
\pgfpathlineto{\pgfqpoint{6.287999in}{2.348205in}}%
\pgfpathlineto{\pgfqpoint{6.288364in}{1.923061in}}%
\pgfpathlineto{\pgfqpoint{6.288626in}{2.541785in}}%
\pgfpathlineto{\pgfqpoint{6.289109in}{2.360505in}}%
\pgfpathlineto{\pgfqpoint{6.289360in}{2.551999in}}%
\pgfpathlineto{\pgfqpoint{6.289275in}{1.931275in}}%
\pgfpathlineto{\pgfqpoint{6.290219in}{2.366601in}}%
\pgfpathlineto{\pgfqpoint{6.291139in}{1.856688in}}%
\pgfpathlineto{\pgfqpoint{6.291019in}{2.557827in}}%
\pgfpathlineto{\pgfqpoint{6.291326in}{2.432166in}}%
\pgfpathlineto{\pgfqpoint{6.292045in}{2.556812in}}%
\pgfpathlineto{\pgfqpoint{6.291727in}{1.781924in}}%
\pgfpathlineto{\pgfqpoint{6.292435in}{2.465767in}}%
\pgfpathlineto{\pgfqpoint{6.293288in}{1.905248in}}%
\pgfpathlineto{\pgfqpoint{6.293058in}{2.546122in}}%
\pgfpathlineto{\pgfqpoint{6.293547in}{2.325704in}}%
\pgfpathlineto{\pgfqpoint{6.294012in}{1.908441in}}%
\pgfpathlineto{\pgfqpoint{6.293690in}{2.531251in}}%
\pgfpathlineto{\pgfqpoint{6.294658in}{2.305804in}}%
\pgfpathlineto{\pgfqpoint{6.294805in}{2.543637in}}%
\pgfpathlineto{\pgfqpoint{6.295751in}{1.857859in}}%
\pgfpathlineto{\pgfqpoint{6.295769in}{2.418786in}}%
\pgfpathlineto{\pgfqpoint{6.296305in}{1.723162in}}%
\pgfpathlineto{\pgfqpoint{6.296170in}{2.559499in}}%
\pgfpathlineto{\pgfqpoint{6.296879in}{2.268458in}}%
\pgfpathlineto{\pgfqpoint{6.297473in}{2.552395in}}%
\pgfpathlineto{\pgfqpoint{6.297028in}{1.721579in}}%
\pgfpathlineto{\pgfqpoint{6.297990in}{2.219879in}}%
\pgfpathlineto{\pgfqpoint{6.298074in}{1.690762in}}%
\pgfpathlineto{\pgfqpoint{6.298491in}{2.546178in}}%
\pgfpathlineto{\pgfqpoint{6.299081in}{2.336068in}}%
\pgfpathlineto{\pgfqpoint{6.299498in}{2.558842in}}%
\pgfpathlineto{\pgfqpoint{6.299262in}{1.865752in}}%
\pgfpathlineto{\pgfqpoint{6.300191in}{2.442582in}}%
\pgfpathlineto{\pgfqpoint{6.300410in}{1.901103in}}%
\pgfpathlineto{\pgfqpoint{6.301185in}{2.545654in}}%
\pgfpathlineto{\pgfqpoint{6.301302in}{2.371407in}}%
\pgfpathlineto{\pgfqpoint{6.301684in}{2.550434in}}%
\pgfpathlineto{\pgfqpoint{6.302115in}{1.649736in}}%
\pgfpathlineto{\pgfqpoint{6.302410in}{2.469845in}}%
\pgfpathlineto{\pgfqpoint{6.302988in}{1.774959in}}%
\pgfpathlineto{\pgfqpoint{6.303068in}{2.555646in}}%
\pgfpathlineto{\pgfqpoint{6.303521in}{2.344357in}}%
\pgfpathlineto{\pgfqpoint{6.303816in}{2.597078in}}%
\pgfpathlineto{\pgfqpoint{6.304545in}{1.936685in}}%
\pgfpathlineto{\pgfqpoint{6.304630in}{2.455577in}}%
\pgfpathlineto{\pgfqpoint{6.304889in}{1.896682in}}%
\pgfpathlineto{\pgfqpoint{6.304744in}{2.550357in}}%
\pgfpathlineto{\pgfqpoint{6.305740in}{2.449812in}}%
\pgfpathlineto{\pgfqpoint{6.306365in}{2.558854in}}%
\pgfpathlineto{\pgfqpoint{6.306135in}{1.832892in}}%
\pgfpathlineto{\pgfqpoint{6.306844in}{2.322210in}}%
\pgfpathlineto{\pgfqpoint{6.307776in}{1.847931in}}%
\pgfpathlineto{\pgfqpoint{6.306900in}{2.552896in}}%
\pgfpathlineto{\pgfqpoint{6.307952in}{2.340666in}}%
\pgfpathlineto{\pgfqpoint{6.308525in}{2.568123in}}%
\pgfpathlineto{\pgfqpoint{6.308418in}{1.841650in}}%
\pgfpathlineto{\pgfqpoint{6.309063in}{2.440943in}}%
\pgfpathlineto{\pgfqpoint{6.309126in}{1.943619in}}%
\pgfpathlineto{\pgfqpoint{6.309782in}{2.575650in}}%
\pgfpathlineto{\pgfqpoint{6.310175in}{2.392550in}}%
\pgfpathlineto{\pgfqpoint{6.310299in}{2.578929in}}%
\pgfpathlineto{\pgfqpoint{6.310963in}{1.907198in}}%
\pgfpathlineto{\pgfqpoint{6.311286in}{2.463464in}}%
\pgfpathlineto{\pgfqpoint{6.311815in}{1.820484in}}%
\pgfpathlineto{\pgfqpoint{6.311611in}{2.559248in}}%
\pgfpathlineto{\pgfqpoint{6.312396in}{2.437489in}}%
\pgfpathlineto{\pgfqpoint{6.313250in}{2.568480in}}%
\pgfpathlineto{\pgfqpoint{6.312705in}{1.816647in}}%
\pgfpathlineto{\pgfqpoint{6.313504in}{2.406497in}}%
\pgfpathlineto{\pgfqpoint{6.314013in}{1.862109in}}%
\pgfpathlineto{\pgfqpoint{6.314086in}{2.575218in}}%
\pgfpathlineto{\pgfqpoint{6.314615in}{2.315457in}}%
\pgfpathlineto{\pgfqpoint{6.315717in}{2.547789in}}%
\pgfpathlineto{\pgfqpoint{6.315516in}{1.768979in}}%
\pgfpathlineto{\pgfqpoint{6.315726in}{2.388396in}}%
\pgfpathlineto{\pgfqpoint{6.315937in}{1.975680in}}%
\pgfpathlineto{\pgfqpoint{6.316516in}{2.565293in}}%
\pgfpathlineto{\pgfqpoint{6.316835in}{2.514693in}}%
\pgfpathlineto{\pgfqpoint{6.317656in}{1.965734in}}%
\pgfpathlineto{\pgfqpoint{6.317470in}{2.567078in}}%
\pgfpathlineto{\pgfqpoint{6.317950in}{2.392411in}}%
\pgfpathlineto{\pgfqpoint{6.318488in}{2.579357in}}%
\pgfpathlineto{\pgfqpoint{6.318367in}{2.056780in}}%
\pgfpathlineto{\pgfqpoint{6.318842in}{2.359997in}}%
\pgfpathlineto{\pgfqpoint{6.319545in}{1.677866in}}%
\pgfpathlineto{\pgfqpoint{6.319540in}{2.551429in}}%
\pgfpathlineto{\pgfqpoint{6.319952in}{2.464926in}}%
\pgfpathlineto{\pgfqpoint{6.320190in}{2.009883in}}%
\pgfpathlineto{\pgfqpoint{6.320502in}{2.563430in}}%
\pgfpathlineto{\pgfqpoint{6.321065in}{2.444520in}}%
\pgfpathlineto{\pgfqpoint{6.321348in}{2.570977in}}%
\pgfpathlineto{\pgfqpoint{6.321763in}{1.767419in}}%
\pgfpathlineto{\pgfqpoint{6.322175in}{2.369174in}}%
\pgfpathlineto{\pgfqpoint{6.322292in}{2.566540in}}%
\pgfpathlineto{\pgfqpoint{6.323022in}{1.959432in}}%
\pgfpathlineto{\pgfqpoint{6.323286in}{2.409518in}}%
\pgfpathlineto{\pgfqpoint{6.324074in}{1.955400in}}%
\pgfpathlineto{\pgfqpoint{6.324075in}{2.561594in}}%
\pgfpathlineto{\pgfqpoint{6.324397in}{2.386415in}}%
\pgfpathlineto{\pgfqpoint{6.324416in}{1.811812in}}%
\pgfpathlineto{\pgfqpoint{6.325285in}{2.573046in}}%
\pgfpathlineto{\pgfqpoint{6.325485in}{2.425550in}}%
\pgfpathlineto{\pgfqpoint{6.326390in}{2.574198in}}%
\pgfpathlineto{\pgfqpoint{6.325754in}{1.868830in}}%
\pgfpathlineto{\pgfqpoint{6.326594in}{2.450765in}}%
\pgfpathlineto{\pgfqpoint{6.326872in}{1.857247in}}%
\pgfpathlineto{\pgfqpoint{6.327436in}{2.557284in}}%
\pgfpathlineto{\pgfqpoint{6.327704in}{2.287603in}}%
\pgfpathlineto{\pgfqpoint{6.328452in}{2.584207in}}%
\pgfpathlineto{\pgfqpoint{6.328517in}{1.848302in}}%
\pgfpathlineto{\pgfqpoint{6.328816in}{2.447412in}}%
\pgfpathlineto{\pgfqpoint{6.329635in}{1.872423in}}%
\pgfpathlineto{\pgfqpoint{6.329765in}{2.576518in}}%
\pgfpathlineto{\pgfqpoint{6.329927in}{2.431927in}}%
\pgfpathlineto{\pgfqpoint{6.330877in}{1.908256in}}%
\pgfpathlineto{\pgfqpoint{6.330194in}{2.550175in}}%
\pgfpathlineto{\pgfqpoint{6.331037in}{2.400672in}}%
\pgfpathlineto{\pgfqpoint{6.331779in}{2.561362in}}%
\pgfpathlineto{\pgfqpoint{6.331584in}{1.755083in}}%
\pgfpathlineto{\pgfqpoint{6.332146in}{2.514357in}}%
\pgfpathlineto{\pgfqpoint{6.332214in}{1.927976in}}%
\pgfpathlineto{\pgfqpoint{6.332411in}{2.573290in}}%
\pgfpathlineto{\pgfqpoint{6.333257in}{2.358284in}}%
\pgfpathlineto{\pgfqpoint{6.333600in}{2.567933in}}%
\pgfpathlineto{\pgfqpoint{6.333421in}{1.937373in}}%
\pgfpathlineto{\pgfqpoint{6.334368in}{2.393152in}}%
\pgfpathlineto{\pgfqpoint{6.335198in}{1.733292in}}%
\pgfpathlineto{\pgfqpoint{6.334888in}{2.559265in}}%
\pgfpathlineto{\pgfqpoint{6.335464in}{2.351539in}}%
\pgfpathlineto{\pgfqpoint{6.335961in}{2.581162in}}%
\pgfpathlineto{\pgfqpoint{6.335625in}{1.905045in}}%
\pgfpathlineto{\pgfqpoint{6.336574in}{2.330639in}}%
\pgfpathlineto{\pgfqpoint{6.337599in}{1.942958in}}%
\pgfpathlineto{\pgfqpoint{6.337167in}{2.564970in}}%
\pgfpathlineto{\pgfqpoint{6.337681in}{2.322208in}}%
\pgfpathlineto{\pgfqpoint{6.338233in}{2.557549in}}%
\pgfpathlineto{\pgfqpoint{6.337899in}{1.682725in}}%
\pgfpathlineto{\pgfqpoint{6.338792in}{2.438277in}}%
\pgfpathlineto{\pgfqpoint{6.339386in}{1.910628in}}%
\pgfpathlineto{\pgfqpoint{6.339264in}{2.581940in}}%
\pgfpathlineto{\pgfqpoint{6.339902in}{2.104509in}}%
\pgfpathlineto{\pgfqpoint{6.340719in}{2.565385in}}%
\pgfpathlineto{\pgfqpoint{6.340766in}{1.893877in}}%
\pgfpathlineto{\pgfqpoint{6.341013in}{2.437933in}}%
\pgfpathlineto{\pgfqpoint{6.341667in}{1.938061in}}%
\pgfpathlineto{\pgfqpoint{6.341671in}{2.584954in}}%
\pgfpathlineto{\pgfqpoint{6.342124in}{2.415828in}}%
\pgfpathlineto{\pgfqpoint{6.342535in}{2.572890in}}%
\pgfpathlineto{\pgfqpoint{6.342802in}{2.001086in}}%
\pgfpathlineto{\pgfqpoint{6.343067in}{2.448139in}}%
\pgfpathlineto{\pgfqpoint{6.344049in}{1.847952in}}%
\pgfpathlineto{\pgfqpoint{6.343294in}{2.589152in}}%
\pgfpathlineto{\pgfqpoint{6.344178in}{2.479859in}}%
\pgfpathlineto{\pgfqpoint{6.344974in}{2.580788in}}%
\pgfpathlineto{\pgfqpoint{6.344993in}{1.837685in}}%
\pgfpathlineto{\pgfqpoint{6.345272in}{2.420197in}}%
\pgfpathlineto{\pgfqpoint{6.345883in}{1.868303in}}%
\pgfpathlineto{\pgfqpoint{6.346285in}{2.574367in}}%
\pgfpathlineto{\pgfqpoint{6.346383in}{2.450245in}}%
\pgfpathlineto{\pgfqpoint{6.346698in}{2.580972in}}%
\pgfpathlineto{\pgfqpoint{6.346703in}{1.726716in}}%
\pgfpathlineto{\pgfqpoint{6.347456in}{2.372932in}}%
\pgfpathlineto{\pgfqpoint{6.348101in}{1.889763in}}%
\pgfpathlineto{\pgfqpoint{6.347729in}{2.572972in}}%
\pgfpathlineto{\pgfqpoint{6.348567in}{2.296622in}}%
\pgfpathlineto{\pgfqpoint{6.349400in}{2.583252in}}%
\pgfpathlineto{\pgfqpoint{6.349005in}{1.923773in}}%
\pgfpathlineto{\pgfqpoint{6.349677in}{2.283474in}}%
\pgfpathlineto{\pgfqpoint{6.350153in}{1.698211in}}%
\pgfpathlineto{\pgfqpoint{6.350077in}{2.571468in}}%
\pgfpathlineto{\pgfqpoint{6.350787in}{2.382048in}}%
\pgfpathlineto{\pgfqpoint{6.351474in}{2.572416in}}%
\pgfpathlineto{\pgfqpoint{6.351286in}{1.838783in}}%
\pgfpathlineto{\pgfqpoint{6.351899in}{2.496713in}}%
\pgfpathlineto{\pgfqpoint{6.351996in}{1.893425in}}%
\pgfpathlineto{\pgfqpoint{6.351989in}{2.567551in}}%
\pgfpathlineto{\pgfqpoint{6.353011in}{2.249110in}}%
\pgfpathlineto{\pgfqpoint{6.353255in}{2.573181in}}%
\pgfpathlineto{\pgfqpoint{6.353336in}{1.845965in}}%
\pgfpathlineto{\pgfqpoint{6.354122in}{2.372284in}}%
\pgfpathlineto{\pgfqpoint{6.354736in}{2.569642in}}%
\pgfpathlineto{\pgfqpoint{6.354364in}{1.990143in}}%
\pgfpathlineto{\pgfqpoint{6.355233in}{2.472170in}}%
\pgfpathlineto{\pgfqpoint{6.355779in}{1.891234in}}%
\pgfpathlineto{\pgfqpoint{6.355661in}{2.590512in}}%
\pgfpathlineto{\pgfqpoint{6.356344in}{2.128560in}}%
\pgfpathlineto{\pgfqpoint{6.356925in}{2.582992in}}%
\pgfpathlineto{\pgfqpoint{6.357366in}{1.892192in}}%
\pgfpathlineto{\pgfqpoint{6.357456in}{2.340060in}}%
\pgfpathlineto{\pgfqpoint{6.357477in}{2.583230in}}%
\pgfpathlineto{\pgfqpoint{6.358186in}{1.805415in}}%
\pgfpathlineto{\pgfqpoint{6.358567in}{2.464192in}}%
\pgfpathlineto{\pgfqpoint{6.359668in}{1.893277in}}%
\pgfpathlineto{\pgfqpoint{6.359352in}{2.583523in}}%
\pgfpathlineto{\pgfqpoint{6.359676in}{2.435979in}}%
\pgfpathlineto{\pgfqpoint{6.360648in}{2.572469in}}%
\pgfpathlineto{\pgfqpoint{6.360223in}{1.818463in}}%
\pgfpathlineto{\pgfqpoint{6.360786in}{2.521396in}}%
\pgfpathlineto{\pgfqpoint{6.361329in}{1.921134in}}%
\pgfpathlineto{\pgfqpoint{6.361782in}{2.587246in}}%
\pgfpathlineto{\pgfqpoint{6.361898in}{2.366628in}}%
\pgfpathlineto{\pgfqpoint{6.362185in}{2.580692in}}%
\pgfpathlineto{\pgfqpoint{6.362861in}{1.947778in}}%
\pgfpathlineto{\pgfqpoint{6.363009in}{2.484935in}}%
\pgfpathlineto{\pgfqpoint{6.363530in}{1.941644in}}%
\pgfpathlineto{\pgfqpoint{6.363173in}{2.578201in}}%
\pgfpathlineto{\pgfqpoint{6.364122in}{2.344600in}}%
\pgfpathlineto{\pgfqpoint{6.364502in}{2.571485in}}%
\pgfpathlineto{\pgfqpoint{6.364242in}{1.939577in}}%
\pgfpathlineto{\pgfqpoint{6.365232in}{2.448498in}}%
\pgfpathlineto{\pgfqpoint{6.365398in}{1.915082in}}%
\pgfpathlineto{\pgfqpoint{6.365989in}{2.582214in}}%
\pgfpathlineto{\pgfqpoint{6.366343in}{2.359457in}}%
\pgfpathlineto{\pgfqpoint{6.367157in}{2.577613in}}%
\pgfpathlineto{\pgfqpoint{6.366635in}{1.895316in}}%
\pgfpathlineto{\pgfqpoint{6.367455in}{2.451504in}}%
\pgfpathlineto{\pgfqpoint{6.368562in}{1.781630in}}%
\pgfpathlineto{\pgfqpoint{6.368108in}{2.593371in}}%
\pgfpathlineto{\pgfqpoint{6.368565in}{2.430860in}}%
\pgfpathlineto{\pgfqpoint{6.368868in}{2.580317in}}%
\pgfpathlineto{\pgfqpoint{6.368841in}{1.967524in}}%
\pgfpathlineto{\pgfqpoint{6.369676in}{2.415834in}}%
\pgfpathlineto{\pgfqpoint{6.370712in}{2.589210in}}%
\pgfpathlineto{\pgfqpoint{6.370165in}{1.908745in}}%
\pgfpathlineto{\pgfqpoint{6.370786in}{2.485391in}}%
\pgfpathlineto{\pgfqpoint{6.371214in}{2.007906in}}%
\pgfpathlineto{\pgfqpoint{6.371701in}{2.576508in}}%
\pgfpathlineto{\pgfqpoint{6.371897in}{2.266510in}}%
\pgfpathlineto{\pgfqpoint{6.372560in}{2.570964in}}%
\pgfpathlineto{\pgfqpoint{6.372408in}{1.995278in}}%
\pgfpathlineto{\pgfqpoint{6.373007in}{2.433814in}}%
\pgfpathlineto{\pgfqpoint{6.373466in}{1.574933in}}%
\pgfpathlineto{\pgfqpoint{6.373223in}{2.587352in}}%
\pgfpathlineto{\pgfqpoint{6.374117in}{2.442946in}}%
\pgfpathlineto{\pgfqpoint{6.374998in}{2.578858in}}%
\pgfpathlineto{\pgfqpoint{6.375176in}{1.885106in}}%
\pgfpathlineto{\pgfqpoint{6.375227in}{2.508616in}}%
\pgfpathlineto{\pgfqpoint{6.375683in}{2.026414in}}%
\pgfpathlineto{\pgfqpoint{6.375310in}{2.601383in}}%
\pgfpathlineto{\pgfqpoint{6.376338in}{2.440469in}}%
\pgfpathlineto{\pgfqpoint{6.376893in}{1.990778in}}%
\pgfpathlineto{\pgfqpoint{6.377250in}{2.591616in}}%
\pgfpathlineto{\pgfqpoint{6.377449in}{2.350328in}}%
\pgfpathlineto{\pgfqpoint{6.378025in}{2.580866in}}%
\pgfpathlineto{\pgfqpoint{6.377803in}{1.795761in}}%
\pgfpathlineto{\pgfqpoint{6.378559in}{2.325192in}}%
\pgfpathlineto{\pgfqpoint{6.379412in}{2.571636in}}%
\pgfpathlineto{\pgfqpoint{6.379311in}{1.834065in}}%
\pgfpathlineto{\pgfqpoint{6.379609in}{2.383434in}}%
\pgfpathlineto{\pgfqpoint{6.379998in}{1.900827in}}%
\pgfpathlineto{\pgfqpoint{6.380113in}{2.586506in}}%
\pgfpathlineto{\pgfqpoint{6.380719in}{2.455611in}}%
\pgfpathlineto{\pgfqpoint{6.381212in}{1.799615in}}%
\pgfpathlineto{\pgfqpoint{6.381077in}{2.588879in}}%
\pgfpathlineto{\pgfqpoint{6.381831in}{2.428240in}}%
\pgfpathlineto{\pgfqpoint{6.382357in}{1.942163in}}%
\pgfpathlineto{\pgfqpoint{6.382692in}{2.569307in}}%
\pgfpathlineto{\pgfqpoint{6.382942in}{2.388097in}}%
\pgfpathlineto{\pgfqpoint{6.384004in}{2.593122in}}%
\pgfpathlineto{\pgfqpoint{6.383483in}{1.928931in}}%
\pgfpathlineto{\pgfqpoint{6.384052in}{2.422356in}}%
\pgfpathlineto{\pgfqpoint{6.384545in}{1.833267in}}%
\pgfpathlineto{\pgfqpoint{6.384261in}{2.623238in}}%
\pgfpathlineto{\pgfqpoint{6.385163in}{2.310009in}}%
\pgfpathlineto{\pgfqpoint{6.385540in}{2.576865in}}%
\pgfpathlineto{\pgfqpoint{6.386093in}{1.994347in}}%
\pgfpathlineto{\pgfqpoint{6.386273in}{2.431061in}}%
\pgfpathlineto{\pgfqpoint{6.386620in}{1.799361in}}%
\pgfpathlineto{\pgfqpoint{6.386502in}{2.576229in}}%
\pgfpathlineto{\pgfqpoint{6.387383in}{2.280612in}}%
\pgfpathlineto{\pgfqpoint{6.387892in}{2.588028in}}%
\pgfpathlineto{\pgfqpoint{6.388238in}{1.849968in}}%
\pgfpathlineto{\pgfqpoint{6.388495in}{2.449913in}}%
\pgfpathlineto{\pgfqpoint{6.389137in}{1.835289in}}%
\pgfpathlineto{\pgfqpoint{6.389247in}{2.594955in}}%
\pgfpathlineto{\pgfqpoint{6.389606in}{2.274003in}}%
\pgfpathlineto{\pgfqpoint{6.390286in}{2.575355in}}%
\pgfpathlineto{\pgfqpoint{6.390483in}{1.778652in}}%
\pgfpathlineto{\pgfqpoint{6.390717in}{2.541963in}}%
\pgfpathlineto{\pgfqpoint{6.391681in}{1.786719in}}%
\pgfpathlineto{\pgfqpoint{6.391495in}{2.582418in}}%
\pgfpathlineto{\pgfqpoint{6.391829in}{2.375906in}}%
\pgfpathlineto{\pgfqpoint{6.391971in}{2.015970in}}%
\pgfpathlineto{\pgfqpoint{6.392159in}{2.590103in}}%
\pgfpathlineto{\pgfqpoint{6.392939in}{2.346736in}}%
\pgfpathlineto{\pgfqpoint{6.393647in}{2.590502in}}%
\pgfpathlineto{\pgfqpoint{6.393603in}{1.864303in}}%
\pgfpathlineto{\pgfqpoint{6.394049in}{2.446173in}}%
\pgfpathlineto{\pgfqpoint{6.394436in}{1.868499in}}%
\pgfpathlineto{\pgfqpoint{6.394429in}{2.585699in}}%
\pgfpathlineto{\pgfqpoint{6.395160in}{2.367532in}}%
\pgfpathlineto{\pgfqpoint{6.395869in}{2.585753in}}%
\pgfpathlineto{\pgfqpoint{6.395200in}{1.861718in}}%
\pgfpathlineto{\pgfqpoint{6.396271in}{2.410149in}}%
\pgfpathlineto{\pgfqpoint{6.396372in}{1.877754in}}%
\pgfpathlineto{\pgfqpoint{6.396988in}{2.584195in}}%
\pgfpathlineto{\pgfqpoint{6.397381in}{2.392366in}}%
\pgfpathlineto{\pgfqpoint{6.397387in}{2.597214in}}%
\pgfpathlineto{\pgfqpoint{6.397647in}{1.884870in}}%
\pgfpathlineto{\pgfqpoint{6.398489in}{2.162380in}}%
\pgfpathlineto{\pgfqpoint{6.398495in}{2.589915in}}%
\pgfpathlineto{\pgfqpoint{6.398514in}{1.968167in}}%
\pgfpathlineto{\pgfqpoint{6.399602in}{2.429614in}}%
\pgfpathlineto{\pgfqpoint{6.400329in}{1.908953in}}%
\pgfpathlineto{\pgfqpoint{6.400098in}{2.592039in}}%
\pgfpathlineto{\pgfqpoint{6.400712in}{2.347599in}}%
\pgfpathlineto{\pgfqpoint{6.401072in}{2.587388in}}%
\pgfpathlineto{\pgfqpoint{6.401001in}{1.981571in}}%
\pgfpathlineto{\pgfqpoint{6.401822in}{2.438209in}}%
\pgfpathlineto{\pgfqpoint{6.402048in}{1.925622in}}%
\pgfpathlineto{\pgfqpoint{6.402516in}{2.609427in}}%
\pgfpathlineto{\pgfqpoint{6.402933in}{2.400880in}}%
\pgfpathlineto{\pgfqpoint{6.403653in}{2.605242in}}%
\pgfpathlineto{\pgfqpoint{6.403585in}{1.801815in}}%
\pgfpathlineto{\pgfqpoint{6.404045in}{2.496393in}}%
\pgfpathlineto{\pgfqpoint{6.404179in}{1.535260in}}%
\pgfpathlineto{\pgfqpoint{6.404937in}{2.586382in}}%
\pgfpathlineto{\pgfqpoint{6.405157in}{2.314807in}}%
\pgfpathlineto{\pgfqpoint{6.405334in}{2.607797in}}%
\pgfpathlineto{\pgfqpoint{6.405239in}{1.815151in}}%
\pgfpathlineto{\pgfqpoint{6.406267in}{2.297721in}}%
\pgfpathlineto{\pgfqpoint{6.406796in}{1.817370in}}%
\pgfpathlineto{\pgfqpoint{6.407142in}{2.608119in}}%
\pgfpathlineto{\pgfqpoint{6.407377in}{2.463186in}}%
\pgfpathlineto{\pgfqpoint{6.407635in}{1.908215in}}%
\pgfpathlineto{\pgfqpoint{6.407754in}{2.592295in}}%
\pgfpathlineto{\pgfqpoint{6.408487in}{2.097210in}}%
\pgfpathlineto{\pgfqpoint{6.408918in}{2.602258in}}%
\pgfpathlineto{\pgfqpoint{6.408575in}{1.857611in}}%
\pgfpathlineto{\pgfqpoint{6.409598in}{2.336182in}}%
\pgfpathlineto{\pgfqpoint{6.409696in}{1.893214in}}%
\pgfpathlineto{\pgfqpoint{6.410624in}{2.587355in}}%
\pgfpathlineto{\pgfqpoint{6.410707in}{2.091940in}}%
\pgfpathlineto{\pgfqpoint{6.411465in}{2.604399in}}%
\pgfpathlineto{\pgfqpoint{6.411294in}{1.855030in}}%
\pgfpathlineto{\pgfqpoint{6.411819in}{2.471888in}}%
\pgfpathlineto{\pgfqpoint{6.412491in}{1.915561in}}%
\pgfpathlineto{\pgfqpoint{6.411962in}{2.604680in}}%
\pgfpathlineto{\pgfqpoint{6.412930in}{2.364499in}}%
\pgfpathlineto{\pgfqpoint{6.412991in}{2.595515in}}%
\pgfpathlineto{\pgfqpoint{6.413740in}{1.826776in}}%
\pgfpathlineto{\pgfqpoint{6.414041in}{2.437485in}}%
\pgfpathlineto{\pgfqpoint{6.414333in}{1.850240in}}%
\pgfpathlineto{\pgfqpoint{6.415041in}{2.587694in}}%
\pgfpathlineto{\pgfqpoint{6.415152in}{2.499466in}}%
\pgfpathlineto{\pgfqpoint{6.415614in}{1.883912in}}%
\pgfpathlineto{\pgfqpoint{6.415757in}{2.590918in}}%
\pgfpathlineto{\pgfqpoint{6.416264in}{2.431354in}}%
\pgfpathlineto{\pgfqpoint{6.416550in}{2.590718in}}%
\pgfpathlineto{\pgfqpoint{6.417214in}{1.806384in}}%
\pgfpathlineto{\pgfqpoint{6.417372in}{2.551166in}}%
\pgfpathlineto{\pgfqpoint{6.418341in}{1.783926in}}%
\pgfpathlineto{\pgfqpoint{6.418130in}{2.585544in}}%
\pgfpathlineto{\pgfqpoint{6.418484in}{2.132529in}}%
\pgfpathlineto{\pgfqpoint{6.419454in}{2.587659in}}%
\pgfpathlineto{\pgfqpoint{6.418544in}{1.998396in}}%
\pgfpathlineto{\pgfqpoint{6.419595in}{2.500523in}}%
\pgfpathlineto{\pgfqpoint{6.419920in}{1.823598in}}%
\pgfpathlineto{\pgfqpoint{6.420485in}{2.607064in}}%
\pgfpathlineto{\pgfqpoint{6.420707in}{2.349107in}}%
\pgfpathlineto{\pgfqpoint{6.421159in}{2.588961in}}%
\pgfpathlineto{\pgfqpoint{6.421443in}{1.982666in}}%
\pgfpathlineto{\pgfqpoint{6.421816in}{2.469949in}}%
\pgfpathlineto{\pgfqpoint{6.422553in}{1.899739in}}%
\pgfpathlineto{\pgfqpoint{6.421906in}{2.605995in}}%
\pgfpathlineto{\pgfqpoint{6.422927in}{2.436518in}}%
\pgfpathlineto{\pgfqpoint{6.423142in}{1.666170in}}%
\pgfpathlineto{\pgfqpoint{6.423534in}{2.607800in}}%
\pgfpathlineto{\pgfqpoint{6.424037in}{2.345024in}}%
\pgfpathlineto{\pgfqpoint{6.424343in}{2.606651in}}%
\pgfpathlineto{\pgfqpoint{6.424685in}{1.879346in}}%
\pgfpathlineto{\pgfqpoint{6.425147in}{2.502276in}}%
\pgfpathlineto{\pgfqpoint{6.425973in}{1.889433in}}%
\pgfpathlineto{\pgfqpoint{6.425353in}{2.599195in}}%
\pgfpathlineto{\pgfqpoint{6.426258in}{2.432527in}}%
\pgfpathlineto{\pgfqpoint{6.426882in}{1.993021in}}%
\pgfpathlineto{\pgfqpoint{6.427344in}{2.599879in}}%
\pgfpathlineto{\pgfqpoint{6.427368in}{2.492007in}}%
\pgfpathlineto{\pgfqpoint{6.427789in}{1.863008in}}%
\pgfpathlineto{\pgfqpoint{6.428358in}{2.595491in}}%
\pgfpathlineto{\pgfqpoint{6.428480in}{2.270105in}}%
\pgfpathlineto{\pgfqpoint{6.428927in}{2.612518in}}%
\pgfpathlineto{\pgfqpoint{6.429206in}{1.992228in}}%
\pgfpathlineto{\pgfqpoint{6.429592in}{2.500028in}}%
\pgfpathlineto{\pgfqpoint{6.429702in}{1.937129in}}%
\pgfpathlineto{\pgfqpoint{6.430000in}{2.603758in}}%
\pgfpathlineto{\pgfqpoint{6.430703in}{2.250466in}}%
\pgfpathlineto{\pgfqpoint{6.430821in}{2.591152in}}%
\pgfpathlineto{\pgfqpoint{6.431187in}{1.976868in}}%
\pgfpathlineto{\pgfqpoint{6.431815in}{2.506163in}}%
\pgfpathlineto{\pgfqpoint{6.431832in}{2.605614in}}%
\pgfpathlineto{\pgfqpoint{6.432742in}{1.960424in}}%
\pgfpathlineto{\pgfqpoint{6.432906in}{2.421302in}}%
\pgfpathlineto{\pgfqpoint{6.433221in}{1.948554in}}%
\pgfpathlineto{\pgfqpoint{6.433910in}{2.611229in}}%
\pgfpathlineto{\pgfqpoint{6.434017in}{2.333271in}}%
\pgfpathlineto{\pgfqpoint{6.434031in}{2.604325in}}%
\pgfpathlineto{\pgfqpoint{6.434790in}{1.918188in}}%
\pgfpathlineto{\pgfqpoint{6.435128in}{2.491242in}}%
\pgfpathlineto{\pgfqpoint{6.435652in}{1.941247in}}%
\pgfpathlineto{\pgfqpoint{6.436041in}{2.597317in}}%
\pgfpathlineto{\pgfqpoint{6.436240in}{2.394102in}}%
\pgfpathlineto{\pgfqpoint{6.436613in}{2.609096in}}%
\pgfpathlineto{\pgfqpoint{6.437327in}{1.959684in}}%
\pgfpathlineto{\pgfqpoint{6.437350in}{2.482836in}}%
\pgfpathlineto{\pgfqpoint{6.438386in}{1.966885in}}%
\pgfpathlineto{\pgfqpoint{6.438356in}{2.609623in}}%
\pgfpathlineto{\pgfqpoint{6.438461in}{2.474993in}}%
\pgfpathlineto{\pgfqpoint{6.438944in}{1.850749in}}%
\pgfpathlineto{\pgfqpoint{6.439341in}{2.616492in}}%
\pgfpathlineto{\pgfqpoint{6.439572in}{2.263799in}}%
\pgfpathlineto{\pgfqpoint{6.440642in}{2.591985in}}%
\pgfpathlineto{\pgfqpoint{6.439908in}{1.954491in}}%
\pgfpathlineto{\pgfqpoint{6.440683in}{2.487364in}}%
\pgfpathlineto{\pgfqpoint{6.441091in}{1.896227in}}%
\pgfpathlineto{\pgfqpoint{6.441058in}{2.594694in}}%
\pgfpathlineto{\pgfqpoint{6.441794in}{2.469437in}}%
\pgfpathlineto{\pgfqpoint{6.442736in}{1.798779in}}%
\pgfpathlineto{\pgfqpoint{6.442548in}{2.621887in}}%
\pgfpathlineto{\pgfqpoint{6.442906in}{2.372445in}}%
\pgfpathlineto{\pgfqpoint{6.443774in}{2.609715in}}%
\pgfpathlineto{\pgfqpoint{6.443734in}{1.808756in}}%
\pgfpathlineto{\pgfqpoint{6.444018in}{2.465800in}}%
\pgfpathlineto{\pgfqpoint{6.444520in}{1.962646in}}%
\pgfpathlineto{\pgfqpoint{6.444580in}{2.598661in}}%
\pgfpathlineto{\pgfqpoint{6.445130in}{2.240791in}}%
\pgfpathlineto{\pgfqpoint{6.445999in}{2.612704in}}%
\pgfpathlineto{\pgfqpoint{6.446120in}{1.961789in}}%
\pgfpathlineto{\pgfqpoint{6.446242in}{2.480869in}}%
\pgfpathlineto{\pgfqpoint{6.447006in}{2.626384in}}%
\pgfpathlineto{\pgfqpoint{6.446520in}{1.986399in}}%
\pgfpathlineto{\pgfqpoint{6.447300in}{2.244230in}}%
\pgfpathlineto{\pgfqpoint{6.447912in}{1.957235in}}%
\pgfpathlineto{\pgfqpoint{6.447399in}{2.604113in}}%
\pgfpathlineto{\pgfqpoint{6.448408in}{2.199284in}}%
\pgfpathlineto{\pgfqpoint{6.448683in}{2.611476in}}%
\pgfpathlineto{\pgfqpoint{6.449034in}{1.937036in}}%
\pgfpathlineto{\pgfqpoint{6.449520in}{2.551510in}}%
\pgfpathlineto{\pgfqpoint{6.449695in}{1.962066in}}%
\pgfpathlineto{\pgfqpoint{6.449840in}{2.607109in}}%
\pgfpathlineto{\pgfqpoint{6.450631in}{2.353215in}}%
\pgfpathlineto{\pgfqpoint{6.451032in}{2.624127in}}%
\pgfpathlineto{\pgfqpoint{6.451081in}{1.905540in}}%
\pgfpathlineto{\pgfqpoint{6.451743in}{2.421964in}}%
\pgfpathlineto{\pgfqpoint{6.451825in}{1.940516in}}%
\pgfpathlineto{\pgfqpoint{6.451959in}{2.624700in}}%
\pgfpathlineto{\pgfqpoint{6.452853in}{2.376539in}}%
\pgfpathlineto{\pgfqpoint{6.453418in}{2.630217in}}%
\pgfpathlineto{\pgfqpoint{6.453293in}{1.975244in}}%
\pgfpathlineto{\pgfqpoint{6.453963in}{2.356744in}}%
\pgfpathlineto{\pgfqpoint{6.454854in}{2.075017in}}%
\pgfpathlineto{\pgfqpoint{6.454627in}{2.615548in}}%
\pgfpathlineto{\pgfqpoint{6.455074in}{2.320504in}}%
\pgfpathlineto{\pgfqpoint{6.455396in}{2.600570in}}%
\pgfpathlineto{\pgfqpoint{6.455156in}{1.907666in}}%
\pgfpathlineto{\pgfqpoint{6.456186in}{2.458144in}}%
\pgfpathlineto{\pgfqpoint{6.457269in}{1.820369in}}%
\pgfpathlineto{\pgfqpoint{6.456792in}{2.623105in}}%
\pgfpathlineto{\pgfqpoint{6.457298in}{2.253183in}}%
\pgfpathlineto{\pgfqpoint{6.458327in}{2.621687in}}%
\pgfpathlineto{\pgfqpoint{6.458130in}{1.803712in}}%
\pgfpathlineto{\pgfqpoint{6.458409in}{2.556307in}}%
\pgfpathlineto{\pgfqpoint{6.458584in}{1.950462in}}%
\pgfpathlineto{\pgfqpoint{6.459023in}{2.607885in}}%
\pgfpathlineto{\pgfqpoint{6.459524in}{2.441566in}}%
\pgfpathlineto{\pgfqpoint{6.459907in}{1.910167in}}%
\pgfpathlineto{\pgfqpoint{6.459996in}{2.603449in}}%
\pgfpathlineto{\pgfqpoint{6.460610in}{2.519861in}}%
\pgfpathlineto{\pgfqpoint{6.461244in}{2.616381in}}%
\pgfpathlineto{\pgfqpoint{6.461001in}{1.984650in}}%
\pgfpathlineto{\pgfqpoint{6.461714in}{2.469758in}}%
\pgfpathlineto{\pgfqpoint{6.462218in}{1.682371in}}%
\pgfpathlineto{\pgfqpoint{6.462466in}{2.632867in}}%
\pgfpathlineto{\pgfqpoint{6.462824in}{2.470735in}}%
\pgfpathlineto{\pgfqpoint{6.463584in}{2.620428in}}%
\pgfpathlineto{\pgfqpoint{6.463415in}{1.640120in}}%
\pgfpathlineto{\pgfqpoint{6.463935in}{2.498222in}}%
\pgfpathlineto{\pgfqpoint{6.464327in}{1.982595in}}%
\pgfpathlineto{\pgfqpoint{6.464769in}{2.600390in}}%
\pgfpathlineto{\pgfqpoint{6.465046in}{2.472391in}}%
\pgfpathlineto{\pgfqpoint{6.465811in}{2.608207in}}%
\pgfpathlineto{\pgfqpoint{6.465404in}{1.937186in}}%
\pgfpathlineto{\pgfqpoint{6.466156in}{2.420643in}}%
\pgfpathlineto{\pgfqpoint{6.466693in}{1.821727in}}%
\pgfpathlineto{\pgfqpoint{6.466579in}{2.607817in}}%
\pgfpathlineto{\pgfqpoint{6.467267in}{2.281098in}}%
\pgfpathlineto{\pgfqpoint{6.467941in}{2.623077in}}%
\pgfpathlineto{\pgfqpoint{6.467502in}{1.899218in}}%
\pgfpathlineto{\pgfqpoint{6.468378in}{2.430719in}}%
\pgfpathlineto{\pgfqpoint{6.468947in}{2.627593in}}%
\pgfpathlineto{\pgfqpoint{6.469170in}{1.895633in}}%
\pgfpathlineto{\pgfqpoint{6.469489in}{2.525321in}}%
\pgfpathlineto{\pgfqpoint{6.469781in}{1.764224in}}%
\pgfpathlineto{\pgfqpoint{6.469564in}{2.624930in}}%
\pgfpathlineto{\pgfqpoint{6.470601in}{2.453224in}}%
\pgfpathlineto{\pgfqpoint{6.471278in}{2.630594in}}%
\pgfpathlineto{\pgfqpoint{6.471115in}{1.946034in}}%
\pgfpathlineto{\pgfqpoint{6.471711in}{2.425329in}}%
\pgfpathlineto{\pgfqpoint{6.471818in}{1.951020in}}%
\pgfpathlineto{\pgfqpoint{6.471765in}{2.618207in}}%
\pgfpathlineto{\pgfqpoint{6.472822in}{2.387250in}}%
\pgfpathlineto{\pgfqpoint{6.473132in}{2.623030in}}%
\pgfpathlineto{\pgfqpoint{6.473056in}{1.930347in}}%
\pgfpathlineto{\pgfqpoint{6.473933in}{2.491248in}}%
\pgfpathlineto{\pgfqpoint{6.473997in}{1.821706in}}%
\pgfpathlineto{\pgfqpoint{6.474683in}{2.626420in}}%
\pgfpathlineto{\pgfqpoint{6.475043in}{2.377645in}}%
\pgfpathlineto{\pgfqpoint{6.475510in}{2.610535in}}%
\pgfpathlineto{\pgfqpoint{6.476046in}{1.893934in}}%
\pgfpathlineto{\pgfqpoint{6.476155in}{2.488661in}}%
\pgfpathlineto{\pgfqpoint{6.477121in}{1.832483in}}%
\pgfpathlineto{\pgfqpoint{6.476970in}{2.605096in}}%
\pgfpathlineto{\pgfqpoint{6.477266in}{2.331505in}}%
\pgfpathlineto{\pgfqpoint{6.478165in}{2.619450in}}%
\pgfpathlineto{\pgfqpoint{6.477935in}{1.979346in}}%
\pgfpathlineto{\pgfqpoint{6.478376in}{2.465815in}}%
\pgfpathlineto{\pgfqpoint{6.478428in}{1.876004in}}%
\pgfpathlineto{\pgfqpoint{6.478405in}{2.633574in}}%
\pgfpathlineto{\pgfqpoint{6.479487in}{2.418512in}}%
\pgfpathlineto{\pgfqpoint{6.480591in}{2.619562in}}%
\pgfpathlineto{\pgfqpoint{6.480116in}{1.923802in}}%
\pgfpathlineto{\pgfqpoint{6.480596in}{2.538826in}}%
\pgfpathlineto{\pgfqpoint{6.481015in}{1.934298in}}%
\pgfpathlineto{\pgfqpoint{6.480902in}{2.620796in}}%
\pgfpathlineto{\pgfqpoint{6.481707in}{2.433923in}}%
\pgfpathlineto{\pgfqpoint{6.482670in}{2.614252in}}%
\pgfpathlineto{\pgfqpoint{6.481712in}{1.882186in}}%
\pgfpathlineto{\pgfqpoint{6.482818in}{2.525467in}}%
\pgfpathlineto{\pgfqpoint{6.483574in}{1.892646in}}%
\pgfpathlineto{\pgfqpoint{6.483357in}{2.608123in}}%
\pgfpathlineto{\pgfqpoint{6.483930in}{2.255825in}}%
\pgfpathlineto{\pgfqpoint{6.484122in}{2.620821in}}%
\pgfpathlineto{\pgfqpoint{6.484088in}{1.900319in}}%
\pgfpathlineto{\pgfqpoint{6.485042in}{2.494462in}}%
\pgfpathlineto{\pgfqpoint{6.485176in}{1.869512in}}%
\pgfpathlineto{\pgfqpoint{6.485575in}{2.626992in}}%
\pgfpathlineto{\pgfqpoint{6.486151in}{2.454326in}}%
\pgfpathlineto{\pgfqpoint{6.486473in}{2.616131in}}%
\pgfpathlineto{\pgfqpoint{6.486635in}{1.988948in}}%
\pgfpathlineto{\pgfqpoint{6.487261in}{2.545380in}}%
\pgfpathlineto{\pgfqpoint{6.487472in}{1.999348in}}%
\pgfpathlineto{\pgfqpoint{6.487726in}{2.618918in}}%
\pgfpathlineto{\pgfqpoint{6.488372in}{2.457362in}}%
\pgfpathlineto{\pgfqpoint{6.489249in}{2.615451in}}%
\pgfpathlineto{\pgfqpoint{6.489115in}{1.962097in}}%
\pgfpathlineto{\pgfqpoint{6.489483in}{2.442703in}}%
\pgfpathlineto{\pgfqpoint{6.490464in}{1.819205in}}%
\pgfpathlineto{\pgfqpoint{6.490270in}{2.637554in}}%
\pgfpathlineto{\pgfqpoint{6.490594in}{2.378609in}}%
\pgfpathlineto{\pgfqpoint{6.490814in}{2.626739in}}%
\pgfpathlineto{\pgfqpoint{6.490956in}{1.959082in}}%
\pgfpathlineto{\pgfqpoint{6.491706in}{2.551339in}}%
\pgfpathlineto{\pgfqpoint{6.492572in}{1.718059in}}%
\pgfpathlineto{\pgfqpoint{6.492056in}{2.620864in}}%
\pgfpathlineto{\pgfqpoint{6.492818in}{2.347318in}}%
\pgfpathlineto{\pgfqpoint{6.493655in}{2.625990in}}%
\pgfpathlineto{\pgfqpoint{6.493528in}{1.867019in}}%
\pgfpathlineto{\pgfqpoint{6.493930in}{2.578875in}}%
\pgfpathlineto{\pgfqpoint{6.493973in}{1.645430in}}%
\pgfpathlineto{\pgfqpoint{6.494336in}{2.623870in}}%
\pgfpathlineto{\pgfqpoint{6.495041in}{2.408466in}}%
\pgfpathlineto{\pgfqpoint{6.496057in}{1.917079in}}%
\pgfpathlineto{\pgfqpoint{6.495892in}{2.625738in}}%
\pgfpathlineto{\pgfqpoint{6.496151in}{2.421100in}}%
\pgfpathlineto{\pgfqpoint{6.497005in}{2.617495in}}%
\pgfpathlineto{\pgfqpoint{6.496276in}{1.841423in}}%
\pgfpathlineto{\pgfqpoint{6.497262in}{2.416830in}}%
\pgfpathlineto{\pgfqpoint{6.498065in}{1.955832in}}%
\pgfpathlineto{\pgfqpoint{6.498039in}{2.626942in}}%
\pgfpathlineto{\pgfqpoint{6.498372in}{2.571810in}}%
\pgfpathlineto{\pgfqpoint{6.499201in}{2.039510in}}%
\pgfpathlineto{\pgfqpoint{6.499055in}{2.621971in}}%
\pgfpathlineto{\pgfqpoint{6.499484in}{2.424890in}}%
\pgfpathlineto{\pgfqpoint{6.500563in}{2.026485in}}%
\pgfpathlineto{\pgfqpoint{6.500142in}{2.627461in}}%
\pgfpathlineto{\pgfqpoint{6.500569in}{2.516972in}}%
\pgfpathlineto{\pgfqpoint{6.500569in}{2.619984in}}%
\pgfpathlineto{\pgfqpoint{6.501361in}{1.804035in}}%
\pgfpathlineto{\pgfqpoint{6.501679in}{2.431993in}}%
\pgfpathlineto{\pgfqpoint{6.502394in}{1.932607in}}%
\pgfpathlineto{\pgfqpoint{6.501804in}{2.656545in}}%
\pgfpathlineto{\pgfqpoint{6.502790in}{2.343876in}}%
\pgfpathlineto{\pgfqpoint{6.503543in}{2.624877in}}%
\pgfpathlineto{\pgfqpoint{6.503888in}{1.963726in}}%
\pgfpathlineto{\pgfqpoint{6.503902in}{2.579143in}}%
\pgfpathlineto{\pgfqpoint{6.504211in}{1.895292in}}%
\pgfpathlineto{\pgfqpoint{6.504614in}{2.626013in}}%
\pgfpathlineto{\pgfqpoint{6.505014in}{2.292487in}}%
\pgfpathlineto{\pgfqpoint{6.505910in}{2.623326in}}%
\pgfpathlineto{\pgfqpoint{6.505740in}{1.804105in}}%
\pgfpathlineto{\pgfqpoint{6.506125in}{2.441443in}}%
\pgfpathlineto{\pgfqpoint{6.506549in}{1.975831in}}%
\pgfpathlineto{\pgfqpoint{6.506943in}{2.629492in}}%
\pgfpathlineto{\pgfqpoint{6.507234in}{2.488447in}}%
\pgfpathlineto{\pgfqpoint{6.507752in}{2.630303in}}%
\pgfpathlineto{\pgfqpoint{6.507238in}{1.883542in}}%
\pgfpathlineto{\pgfqpoint{6.508345in}{2.533890in}}%
\pgfpathlineto{\pgfqpoint{6.509308in}{1.979926in}}%
\pgfpathlineto{\pgfqpoint{6.508533in}{2.636188in}}%
\pgfpathlineto{\pgfqpoint{6.509456in}{2.489401in}}%
\pgfpathlineto{\pgfqpoint{6.509996in}{1.980822in}}%
\pgfpathlineto{\pgfqpoint{6.509818in}{2.624856in}}%
\pgfpathlineto{\pgfqpoint{6.510401in}{2.503055in}}%
\pgfpathlineto{\pgfqpoint{6.511412in}{2.633813in}}%
\pgfpathlineto{\pgfqpoint{6.510464in}{1.771774in}}%
\pgfpathlineto{\pgfqpoint{6.511511in}{2.501362in}}%
\pgfpathlineto{\pgfqpoint{6.512067in}{1.979980in}}%
\pgfpathlineto{\pgfqpoint{6.511922in}{2.634684in}}%
\pgfpathlineto{\pgfqpoint{6.512622in}{2.354995in}}%
\pgfpathlineto{\pgfqpoint{6.513589in}{2.660701in}}%
\pgfpathlineto{\pgfqpoint{6.513181in}{2.003043in}}%
\pgfpathlineto{\pgfqpoint{6.513733in}{2.523364in}}%
\pgfpathlineto{\pgfqpoint{6.513922in}{1.664268in}}%
\pgfpathlineto{\pgfqpoint{6.514656in}{2.633800in}}%
\pgfpathlineto{\pgfqpoint{6.514845in}{2.284541in}}%
\pgfpathlineto{\pgfqpoint{6.515403in}{2.622887in}}%
\pgfpathlineto{\pgfqpoint{6.515372in}{1.941375in}}%
\pgfpathlineto{\pgfqpoint{6.515955in}{2.369587in}}%
\pgfpathlineto{\pgfqpoint{6.516332in}{1.794738in}}%
\pgfpathlineto{\pgfqpoint{6.516620in}{2.632175in}}%
\pgfpathlineto{\pgfqpoint{6.517065in}{2.436223in}}%
\pgfpathlineto{\pgfqpoint{6.517364in}{2.636870in}}%
\pgfpathlineto{\pgfqpoint{6.517295in}{1.959330in}}%
\pgfpathlineto{\pgfqpoint{6.518176in}{2.390694in}}%
\pgfpathlineto{\pgfqpoint{6.518822in}{1.894343in}}%
\pgfpathlineto{\pgfqpoint{6.519251in}{2.629869in}}%
\pgfpathlineto{\pgfqpoint{6.519285in}{2.179386in}}%
\pgfpathlineto{\pgfqpoint{6.519485in}{2.644542in}}%
\pgfpathlineto{\pgfqpoint{6.519662in}{1.908556in}}%
\pgfpathlineto{\pgfqpoint{6.520396in}{2.499374in}}%
\pgfpathlineto{\pgfqpoint{6.520672in}{1.885361in}}%
\pgfpathlineto{\pgfqpoint{6.520869in}{2.662776in}}%
\pgfpathlineto{\pgfqpoint{6.521508in}{2.446658in}}%
\pgfpathlineto{\pgfqpoint{6.521738in}{1.828422in}}%
\pgfpathlineto{\pgfqpoint{6.521724in}{2.643516in}}%
\pgfpathlineto{\pgfqpoint{6.522617in}{2.333326in}}%
\pgfpathlineto{\pgfqpoint{6.523539in}{2.632328in}}%
\pgfpathlineto{\pgfqpoint{6.523582in}{1.950499in}}%
\pgfpathlineto{\pgfqpoint{6.523728in}{2.521667in}}%
\pgfpathlineto{\pgfqpoint{6.523884in}{1.904543in}}%
\pgfpathlineto{\pgfqpoint{6.523914in}{2.630799in}}%
\pgfpathlineto{\pgfqpoint{6.524839in}{2.164978in}}%
\pgfpathlineto{\pgfqpoint{6.525381in}{2.654035in}}%
\pgfpathlineto{\pgfqpoint{6.525611in}{1.777130in}}%
\pgfpathlineto{\pgfqpoint{6.525950in}{2.544277in}}%
\pgfpathlineto{\pgfqpoint{6.527027in}{1.849392in}}%
\pgfpathlineto{\pgfqpoint{6.526372in}{2.636799in}}%
\pgfpathlineto{\pgfqpoint{6.527063in}{2.420097in}}%
\pgfpathlineto{\pgfqpoint{6.527393in}{2.657990in}}%
\pgfpathlineto{\pgfqpoint{6.527456in}{1.923711in}}%
\pgfpathlineto{\pgfqpoint{6.528174in}{2.482323in}}%
\pgfpathlineto{\pgfqpoint{6.528901in}{1.971705in}}%
\pgfpathlineto{\pgfqpoint{6.528727in}{2.625273in}}%
\pgfpathlineto{\pgfqpoint{6.529285in}{2.375294in}}%
\pgfpathlineto{\pgfqpoint{6.530133in}{2.627166in}}%
\pgfpathlineto{\pgfqpoint{6.529287in}{1.705813in}}%
\pgfpathlineto{\pgfqpoint{6.530396in}{2.382219in}}%
\pgfpathlineto{\pgfqpoint{6.531481in}{2.640174in}}%
\pgfpathlineto{\pgfqpoint{6.531476in}{2.000612in}}%
\pgfpathlineto{\pgfqpoint{6.531507in}{2.539057in}}%
\pgfpathlineto{\pgfqpoint{6.532442in}{1.829801in}}%
\pgfpathlineto{\pgfqpoint{6.531886in}{2.633815in}}%
\pgfpathlineto{\pgfqpoint{6.532617in}{2.526721in}}%
\pgfpathlineto{\pgfqpoint{6.532742in}{2.033462in}}%
\pgfpathlineto{\pgfqpoint{6.533689in}{2.629264in}}%
\pgfpathlineto{\pgfqpoint{6.533727in}{2.483174in}}%
\pgfpathlineto{\pgfqpoint{6.534267in}{2.646458in}}%
\pgfpathlineto{\pgfqpoint{6.534071in}{1.806407in}}%
\pgfpathlineto{\pgfqpoint{6.534838in}{2.500271in}}%
\pgfpathlineto{\pgfqpoint{6.535540in}{1.905964in}}%
\pgfpathlineto{\pgfqpoint{6.535700in}{2.628009in}}%
\pgfpathlineto{\pgfqpoint{6.535948in}{2.575940in}}%
\pgfpathlineto{\pgfqpoint{6.535978in}{1.823978in}}%
\pgfpathlineto{\pgfqpoint{6.536529in}{2.639385in}}%
\pgfpathlineto{\pgfqpoint{6.537060in}{2.367243in}}%
\pgfpathlineto{\pgfqpoint{6.537837in}{2.628118in}}%
\pgfpathlineto{\pgfqpoint{6.537291in}{1.949241in}}%
\pgfpathlineto{\pgfqpoint{6.538171in}{2.503088in}}%
\pgfpathlineto{\pgfqpoint{6.539248in}{2.001788in}}%
\pgfpathlineto{\pgfqpoint{6.539152in}{2.642705in}}%
\pgfpathlineto{\pgfqpoint{6.539282in}{2.548682in}}%
\pgfpathlineto{\pgfqpoint{6.539894in}{2.025959in}}%
\pgfpathlineto{\pgfqpoint{6.540389in}{2.660980in}}%
\pgfpathlineto{\pgfqpoint{6.540391in}{2.541881in}}%
\pgfpathlineto{\pgfqpoint{6.541481in}{2.659859in}}%
\pgfpathlineto{\pgfqpoint{6.541475in}{2.025005in}}%
\pgfpathlineto{\pgfqpoint{6.541500in}{2.544836in}}%
\pgfpathlineto{\pgfqpoint{6.542062in}{1.854285in}}%
\pgfpathlineto{\pgfqpoint{6.542368in}{2.656053in}}%
\pgfpathlineto{\pgfqpoint{6.542612in}{2.409473in}}%
\pgfpathlineto{\pgfqpoint{6.543671in}{2.649874in}}%
\pgfpathlineto{\pgfqpoint{6.543009in}{1.963602in}}%
\pgfpathlineto{\pgfqpoint{6.543723in}{2.407724in}}%
\pgfpathlineto{\pgfqpoint{6.544626in}{1.950230in}}%
\pgfpathlineto{\pgfqpoint{6.544062in}{2.656599in}}%
\pgfpathlineto{\pgfqpoint{6.544834in}{2.109085in}}%
\pgfpathlineto{\pgfqpoint{6.545015in}{2.654580in}}%
\pgfpathlineto{\pgfqpoint{6.545325in}{2.006933in}}%
\pgfpathlineto{\pgfqpoint{6.545945in}{2.345109in}}%
\pgfpathlineto{\pgfqpoint{6.546728in}{1.950410in}}%
\pgfpathlineto{\pgfqpoint{6.546551in}{2.669423in}}%
\pgfpathlineto{\pgfqpoint{6.546961in}{2.530265in}}%
\pgfpathlineto{\pgfqpoint{6.547502in}{2.649337in}}%
\pgfpathlineto{\pgfqpoint{6.547522in}{1.773356in}}%
\pgfpathlineto{\pgfqpoint{6.548071in}{2.401369in}}%
\pgfpathlineto{\pgfqpoint{6.548074in}{1.829435in}}%
\pgfpathlineto{\pgfqpoint{6.548155in}{2.656737in}}%
\pgfpathlineto{\pgfqpoint{6.549182in}{2.400412in}}%
\pgfpathlineto{\pgfqpoint{6.550138in}{2.654444in}}%
\pgfpathlineto{\pgfqpoint{6.550155in}{1.794319in}}%
\pgfpathlineto{\pgfqpoint{6.550293in}{2.461814in}}%
\pgfpathlineto{\pgfqpoint{6.550828in}{1.970076in}}%
\pgfpathlineto{\pgfqpoint{6.551188in}{2.657659in}}%
\pgfpathlineto{\pgfqpoint{6.551404in}{2.407260in}}%
\pgfpathlineto{\pgfqpoint{6.551665in}{2.647973in}}%
\pgfpathlineto{\pgfqpoint{6.551708in}{2.026563in}}%
\pgfpathlineto{\pgfqpoint{6.552514in}{2.449456in}}%
\pgfpathlineto{\pgfqpoint{6.552571in}{1.885571in}}%
\pgfpathlineto{\pgfqpoint{6.553358in}{2.642088in}}%
\pgfpathlineto{\pgfqpoint{6.553625in}{2.410675in}}%
\pgfpathlineto{\pgfqpoint{6.554513in}{1.975768in}}%
\pgfpathlineto{\pgfqpoint{6.554456in}{2.650943in}}%
\pgfpathlineto{\pgfqpoint{6.554734in}{2.515161in}}%
\pgfpathlineto{\pgfqpoint{6.555674in}{1.873725in}}%
\pgfpathlineto{\pgfqpoint{6.555094in}{2.642565in}}%
\pgfpathlineto{\pgfqpoint{6.555837in}{2.206115in}}%
\pgfpathlineto{\pgfqpoint{6.555843in}{2.649945in}}%
\pgfpathlineto{\pgfqpoint{6.556808in}{1.956684in}}%
\pgfpathlineto{\pgfqpoint{6.556948in}{2.499423in}}%
\pgfpathlineto{\pgfqpoint{6.557038in}{2.023987in}}%
\pgfpathlineto{\pgfqpoint{6.557362in}{2.660555in}}%
\pgfpathlineto{\pgfqpoint{6.558059in}{2.549507in}}%
\pgfpathlineto{\pgfqpoint{6.558773in}{2.017328in}}%
\pgfpathlineto{\pgfqpoint{6.558119in}{2.632896in}}%
\pgfpathlineto{\pgfqpoint{6.559169in}{2.240978in}}%
\pgfpathlineto{\pgfqpoint{6.559777in}{2.652001in}}%
\pgfpathlineto{\pgfqpoint{6.559369in}{1.985629in}}%
\pgfpathlineto{\pgfqpoint{6.560281in}{2.559383in}}%
\pgfpathlineto{\pgfqpoint{6.560698in}{2.635794in}}%
\pgfpathlineto{\pgfqpoint{6.560545in}{1.928328in}}%
\pgfpathlineto{\pgfqpoint{6.561379in}{2.468560in}}%
\pgfpathlineto{\pgfqpoint{6.561929in}{1.926515in}}%
\pgfpathlineto{\pgfqpoint{6.562142in}{2.640804in}}%
\pgfpathlineto{\pgfqpoint{6.562490in}{2.387756in}}%
\pgfpathlineto{\pgfqpoint{6.562965in}{2.661193in}}%
\pgfpathlineto{\pgfqpoint{6.562772in}{2.052216in}}%
\pgfpathlineto{\pgfqpoint{6.562965in}{2.661193in}}%
\pgfusepath{stroke}%
\end{pgfscope}%
\begin{pgfscope}%
\pgfsetrectcap%
\pgfsetmiterjoin%
\pgfsetlinewidth{0.803000pt}%
\definecolor{currentstroke}{rgb}{0.000000,0.000000,0.000000}%
\pgfsetstrokecolor{currentstroke}%
\pgfsetdash{}{0pt}%
\pgfpathmoveto{\pgfqpoint{0.535225in}{0.370679in}}%
\pgfpathlineto{\pgfqpoint{0.535225in}{3.551852in}}%
\pgfusepath{stroke}%
\end{pgfscope}%
\begin{pgfscope}%
\pgfsetrectcap%
\pgfsetmiterjoin%
\pgfsetlinewidth{0.803000pt}%
\definecolor{currentstroke}{rgb}{0.000000,0.000000,0.000000}%
\pgfsetstrokecolor{currentstroke}%
\pgfsetdash{}{0pt}%
\pgfpathmoveto{\pgfqpoint{6.850000in}{0.370679in}}%
\pgfpathlineto{\pgfqpoint{6.850000in}{3.551852in}}%
\pgfusepath{stroke}%
\end{pgfscope}%
\begin{pgfscope}%
\pgfsetrectcap%
\pgfsetmiterjoin%
\pgfsetlinewidth{0.803000pt}%
\definecolor{currentstroke}{rgb}{0.000000,0.000000,0.000000}%
\pgfsetstrokecolor{currentstroke}%
\pgfsetdash{}{0pt}%
\pgfpathmoveto{\pgfqpoint{0.535225in}{0.370679in}}%
\pgfpathlineto{\pgfqpoint{6.850000in}{0.370679in}}%
\pgfusepath{stroke}%
\end{pgfscope}%
\begin{pgfscope}%
\pgfsetrectcap%
\pgfsetmiterjoin%
\pgfsetlinewidth{0.803000pt}%
\definecolor{currentstroke}{rgb}{0.000000,0.000000,0.000000}%
\pgfsetstrokecolor{currentstroke}%
\pgfsetdash{}{0pt}%
\pgfpathmoveto{\pgfqpoint{0.535225in}{3.551852in}}%
\pgfpathlineto{\pgfqpoint{6.850000in}{3.551852in}}%
\pgfusepath{stroke}%
\end{pgfscope}%
\begin{pgfscope}%
\pgfsetbuttcap%
\pgfsetmiterjoin%
\definecolor{currentfill}{rgb}{1.000000,1.000000,1.000000}%
\pgfsetfillcolor{currentfill}%
\pgfsetfillopacity{0.800000}%
\pgfsetlinewidth{1.003750pt}%
\definecolor{currentstroke}{rgb}{0.800000,0.800000,0.800000}%
\pgfsetstrokecolor{currentstroke}%
\pgfsetstrokeopacity{0.800000}%
\pgfsetdash{}{0pt}%
\pgfpathmoveto{\pgfqpoint{0.632447in}{0.440123in}}%
\pgfpathlineto{\pgfqpoint{1.636692in}{0.440123in}}%
\pgfpathquadraticcurveto{\pgfqpoint{1.664470in}{0.440123in}}{\pgfqpoint{1.664470in}{0.467901in}}%
\pgfpathlineto{\pgfqpoint{1.664470in}{1.422376in}}%
\pgfpathquadraticcurveto{\pgfqpoint{1.664470in}{1.450154in}}{\pgfqpoint{1.636692in}{1.450154in}}%
\pgfpathlineto{\pgfqpoint{0.632447in}{1.450154in}}%
\pgfpathquadraticcurveto{\pgfqpoint{0.604669in}{1.450154in}}{\pgfqpoint{0.604669in}{1.422376in}}%
\pgfpathlineto{\pgfqpoint{0.604669in}{0.467901in}}%
\pgfpathquadraticcurveto{\pgfqpoint{0.604669in}{0.440123in}}{\pgfqpoint{0.632447in}{0.440123in}}%
\pgfpathlineto{\pgfqpoint{0.632447in}{0.440123in}}%
\pgfpathclose%
\pgfusepath{stroke,fill}%
\end{pgfscope}%
\begin{pgfscope}%
\pgfsetrectcap%
\pgfsetroundjoin%
\pgfsetlinewidth{3.011250pt}%
\definecolor{currentstroke}{rgb}{0.647059,0.164706,0.164706}%
\pgfsetstrokecolor{currentstroke}%
\pgfsetdash{}{0pt}%
\pgfpathmoveto{\pgfqpoint{0.660225in}{1.345987in}}%
\pgfpathlineto{\pgfqpoint{0.799114in}{1.345987in}}%
\pgfpathlineto{\pgfqpoint{0.938003in}{1.345987in}}%
\pgfusepath{stroke}%
\end{pgfscope}%
\begin{pgfscope}%
\definecolor{textcolor}{rgb}{0.000000,0.000000,0.000000}%
\pgfsetstrokecolor{textcolor}%
\pgfsetfillcolor{textcolor}%
\pgftext[x=1.049114in,y=1.297376in,left,base]{\color{textcolor}\rmfamily\fontsize{10.000000}{12.000000}\selectfont brownian}%
\end{pgfscope}%
\begin{pgfscope}%
\pgfsetrectcap%
\pgfsetroundjoin%
\pgfsetlinewidth{3.011250pt}%
\definecolor{currentstroke}{rgb}{1.000000,0.411765,0.705882}%
\pgfsetstrokecolor{currentstroke}%
\pgfsetdash{}{0pt}%
\pgfpathmoveto{\pgfqpoint{0.660225in}{1.152314in}}%
\pgfpathlineto{\pgfqpoint{0.799114in}{1.152314in}}%
\pgfpathlineto{\pgfqpoint{0.938003in}{1.152314in}}%
\pgfusepath{stroke}%
\end{pgfscope}%
\begin{pgfscope}%
\definecolor{textcolor}{rgb}{0.000000,0.000000,0.000000}%
\pgfsetstrokecolor{textcolor}%
\pgfsetfillcolor{textcolor}%
\pgftext[x=1.049114in,y=1.103703in,left,base]{\color{textcolor}\rmfamily\fontsize{10.000000}{12.000000}\selectfont pink}%
\end{pgfscope}%
\begin{pgfscope}%
\pgfsetrectcap%
\pgfsetroundjoin%
\pgfsetlinewidth{3.011250pt}%
\definecolor{currentstroke}{rgb}{1.000000,1.000000,0.000000}%
\pgfsetstrokecolor{currentstroke}%
\pgfsetdash{}{0pt}%
\pgfpathmoveto{\pgfqpoint{0.660225in}{0.958642in}}%
\pgfpathlineto{\pgfqpoint{0.799114in}{0.958642in}}%
\pgfpathlineto{\pgfqpoint{0.938003in}{0.958642in}}%
\pgfusepath{stroke}%
\end{pgfscope}%
\begin{pgfscope}%
\definecolor{textcolor}{rgb}{0.000000,0.000000,0.000000}%
\pgfsetstrokecolor{textcolor}%
\pgfsetfillcolor{textcolor}%
\pgftext[x=1.049114in,y=0.910031in,left,base]{\color{textcolor}\rmfamily\fontsize{10.000000}{12.000000}\selectfont white}%
\end{pgfscope}%
\begin{pgfscope}%
\pgfsetrectcap%
\pgfsetroundjoin%
\pgfsetlinewidth{3.011250pt}%
\definecolor{currentstroke}{rgb}{0.000000,0.000000,1.000000}%
\pgfsetstrokecolor{currentstroke}%
\pgfsetdash{}{0pt}%
\pgfpathmoveto{\pgfqpoint{0.660225in}{0.764969in}}%
\pgfpathlineto{\pgfqpoint{0.799114in}{0.764969in}}%
\pgfpathlineto{\pgfqpoint{0.938003in}{0.764969in}}%
\pgfusepath{stroke}%
\end{pgfscope}%
\begin{pgfscope}%
\definecolor{textcolor}{rgb}{0.000000,0.000000,0.000000}%
\pgfsetstrokecolor{textcolor}%
\pgfsetfillcolor{textcolor}%
\pgftext[x=1.049114in,y=0.716358in,left,base]{\color{textcolor}\rmfamily\fontsize{10.000000}{12.000000}\selectfont blue}%
\end{pgfscope}%
\begin{pgfscope}%
\pgfsetrectcap%
\pgfsetroundjoin%
\pgfsetlinewidth{3.011250pt}%
\definecolor{currentstroke}{rgb}{0.933333,0.509804,0.933333}%
\pgfsetstrokecolor{currentstroke}%
\pgfsetdash{}{0pt}%
\pgfpathmoveto{\pgfqpoint{0.660225in}{0.571296in}}%
\pgfpathlineto{\pgfqpoint{0.799114in}{0.571296in}}%
\pgfpathlineto{\pgfqpoint{0.938003in}{0.571296in}}%
\pgfusepath{stroke}%
\end{pgfscope}%
\begin{pgfscope}%
\definecolor{textcolor}{rgb}{0.000000,0.000000,0.000000}%
\pgfsetstrokecolor{textcolor}%
\pgfsetfillcolor{textcolor}%
\pgftext[x=1.049114in,y=0.522685in,left,base]{\color{textcolor}\rmfamily\fontsize{10.000000}{12.000000}\selectfont violet}%
\end{pgfscope}%
\begin{pgfscope}%
\definecolor{textcolor}{rgb}{0.000000,0.000000,0.000000}%
\pgfsetstrokecolor{textcolor}%
\pgfsetfillcolor{textcolor}%
\pgftext[x=3.500000in,y=3.920000in,,top]{\color{textcolor}\rmfamily\fontsize{12.000000}{14.400000}\selectfont Colored Noise}%
\end{pgfscope}%
\end{pgfpicture}%
\makeatother%
\endgroup%
}
    \end{center}
    \caption{\emph{Rumori}.}
\end{figure}

\end{itemize}



\subsection{Altre considerazioni}

Per finire, aggiungiamo alcune considerazioni finali sul rapporto tra il segnale visto nel dominio del tempo e il suo spettro.


\subsection{DC offset}

Fin qui abbiamo sempre assunto che il segnale considerato nel dominio del tempo sia sempre esattamente centrato rispetto all'asse dello 0 (che può essere considerato la pressione d'ambiente, l'assenza di voltaggio, la posizione di riposo della singola particella d'aria o della membrana del timpano e del microfono). È possibile però che il segnale che consideriamo sia invece traslato rispetto allo 0: questo può accadere come risultato di un particolare tipo di azione meccanica (per esempio, l'arco del violino tende a ``tirare'' la corda in una sola direzione, e quindi produrrà un segnale decentrato), o di specifiche tecniche di sintesi e trattamento (come, ancora una volta, la sintesi per modulazione di frequenza), o per problemi elettrici nella catena elettroacustica (come l'immissione nel percorso del segnale di una componente di corrente continua, da cui il nome di \emph{DC offset} che viene dato alla traslazione verticale della forma d'onda rispetto alla sua posizione normale). Questo tipo di traslazione può essere visto come la somma di un valore costante al segnale; ma un valore costante può essere ottenuto da una funzione sinusoidale con frequenza nulla e termine di fase e ampiezza non nulla: $k \cdot \sin(2\pi \cdot 0t + \phi ) = k \cdot \sin \phi $ per qualsiasi $t$. Ha quindi senso considerare nella descrizione spettrale di un segnale anche una componente di frequenza 0 e fase di $\frac{\pi}{2}$, la cui ampiezza $k$ corrisponderà alla traslazione sull'asse verticale della forma d'onda, dal momento che $\sin \frac{\pi}{2} = 1$.


\subsection{Altre relazioni tra rappresentazione temporale e frequenziale}

Esistono poi alcune caratteristiche della rappresentazione nel dominio del tempo che possono fornire indizi interessanti sullo spettro del segnale: naturalmente, la periodicità o aperiodicità può, almeno in casi semplici, essere stimata guardando la forma d'onda. Abbiamo inoltre già parlato della relazione tra assenza di armoniche pari e antisimmetricità. Un'altra osservazione utile è che la presenza di armoniche acute produce, in generale, una forma d'onda più ``spigolosa'' --- e che, al converso, ``spigoli'' nel segnale produrranno spettri ricchi di armoniche acute, fino al caso limite di una discontinuità, un  ``salto verticale'', nella forma d'onda, che teoricamente produce un delta di Dirac contenente ogni possibile frequenza a pari intensità (il famigerato ``click'').


\subsection{Analisi della scena uditiva}

In tutto questo, non abbiamo accennato, se non di sfuggita, al caso comunissimo in cui più corpi emettano vibrazioni acustiche simultaneamente, nella stessa scena uditiva --- che siano le corde di un pianoforte o corpi senza nessun tipo di relazione materiale o causale. Da un punto di vista puramente astratto, matematico, potremmo dire che non c'è nulla da aggiungere: due note diverse suonate simultaneamente produrranno un segnale sonoro dato dalla somma algebrica delle loro forme d'onda nel tempo, e tutte le considerazioni fatte fin qui restano immutate. Quindi, tipicamente, le strutture spettrali armoniche di ciascuna delle due note si sommeranno, tipicamente producendo uno spettro che all'occhio apparirà come uno spettro inarmonico a componenti discrete.

L'esperienza però ci dice che la percezione che abbiamo di questo fenomeno è profondamente diversa: siamo perfettamente in grado di capire che ci sono più sorgenti sonore in gioco, ciascuna delle quali caratterizzata da uno spettro armonico; un ascoltatore allenato è capace di individuarne con sicurezza le relazioni di frequenza (``una terza maggiore''), ascriverle a specifiche categorie timbriche (``un violino e un flauto'') e, se ha la fortuna di avere l'orecchio assoluto, discernere la fondamentale prodotta da ciascuna delle due (``un mi e un sol diesis in chiave di violino''). 

Ancora più evidente è la nostra capacità di selezionare, in una scena piena di suoni che consideriamo di disturbo, una  sorgente sonora che ci interessa: dal momento che non viviamo in ambienti perfettamente silenziosi, è ciò che facciamo ogni volta che parliamo con qualcuno. Addirittura, c'è un fenomeno psicoacustico chiamato ``effetto cocktail party'' che mostra come siamo capaci di selezionare e condurre una conversazione che ci interessa in una stanza piena di altre conversazioni che potenzialmente si svolgono con un tono di voce più alto, come spesso accade in una festa. Il cervello umano è un riduttore di rumore straordinario.

Queste capacità riposano su meccanismi cognitivi estremamente complessi e dei quali abbiamo una comprensione molto limitata: questi meccanismi ci permettono di individuare e separare porzioni di spettro in base alle loro caratteristiche, in maniera raffinatissima, e sono fondamentali nel funzionamento cognitivo dell'udito e nella nostra capacità di utilizzare l'informazione acustica. Qui si aprirebbe il discorso estremamente ampio dell'\emph{analisi della scena uditiva}, trattato in maniera approfondita nel testo classico di Albert S. Bregman intitolato appunto \emph{Auditory Scene Analysis}.





%
%\section{Rappresentazione nel dominio della frequenza}
%
%\subsection{Risuonatori} 
%
%Un risuonatore è un corpo che risponde in maniera diversa a fenomeni vibratori a frequenze diverse, restando inerte in risposta ad alcune frequenze e mettendosi a sua volta in vibrazione in risposta ad altre. Un risuonatore può essere più o meno selettivo, cioè può reagire in maniera più o meno esclusiva a frequenze precisamente individuate. Risuonatori sono una corda di chitarra, un flauto, una stanza, una cassa armonica. Risuonatori sono anche certe cellule nel nostro orecchio, le cellule acustiche contenute nell'organo di Corti, che si mettono in vibrazione quando un suono a frequenze specifiche le colpisce, inviando impulsi elettrici al cervello.%
%\footnote{Dal punto di vista fisiologico e meccanico questa è una grossa semplificazione; dal punto di vista funzionale, rispetto all'operazione di analisi del segnale acustico che l'organo di Corti compie, mi pare che non tradisca la sostanza della cosa.}
%
%Supponiamo che, in una data scena sonora, due emittenti stiano vibrando a due frequenze diverse. Se misuriamo l'andamento della pressione sonora prodotta, vedremo che questo evolverà nel tempo secondo la somma algebrica dei comportamenti vibratori di ciascuno dei due emittenti. Un'analisi superficiale di tale andamento combinato difficilmente potrà permettere di ricostruire l'andamento individuale, in termini di frequenza e ampiezza, di ciascuno dei due.
%
%D'altra parte, risuonatori accordati rispettivamente a ciascuna delle due frequenze vibratorie dei due emittenti verranno eccitati in maniera proporzionale all'intensità di ciascuna delle due vibrazioni; risuonatori accordati ad altre frequenze non verranno eccitati affatto, o lo saranno in maniera soltanto marginale. Questo principio, che si applica anche alle cellule del nostro orecchio, ci permette di individuare con precisione quali frequenze sono presenti in una data scena sonora: solo le cellule accordate alle frequenze effettivamente presenti nella scena sonora si metteranno in vibrazione e invieranno di conseguenza impulsi al cervello.
%
%
%\subsection{Teorema e trasformata di Fourier}
%
%Esiste uno strumento matematico che modellizza questo comportamento in maniera molto interessante, e un teorema fondamentale che ne enuncia un fondamento teorico che incontreremo continuamente: si tratta rispettivamente della \emph{trasformata di Fourier} e del \emph{teorema di Fourier}. Entrambi richiedono una matematica più avanzata di quella che useremo qui, e per questa ragione li tratteremo in maniera decisamente romanzata, ma potete trovare una loro enunciazione molto bella e accessibile qui: \url{https://jackschaedler.github.io/circles-sines-signals/index.html}
%
%Il teorema di Fourier dice che ogni funzione del tempo, che esibisca o no periodicità, può essere rappresentata come una somma di funzioni sinusoidali con frequenze, ampiezze e fasi differenti. Una conseguenza di questo teorema è che ogni fenomeno oscillatorio che sia regolare nel tempo, e che quindi esibisca una regolarità a un dato periodo $\Delta t$ ovvero a una frequenza $f$ che chiameremo \emph{fondamentale}, può essere scomposto in una \emph{serie} di sinusoidi con ampiezze e fasi differenti, e con frequenze tutte multiple intere della fondamentale $f$.
%
%Esiste un'operazione matematica che permette di passare da un segnale descritto come funzione del tempo a una sua rappresentazione in termini di somma di sinusoidi: la \emph{trasformata di Fourier} (\emph{Fourier transform}, abbreviata in \emph{FT}).
%
%La trasformata di Fourier è invertibile: partendo dalle informazioni di frequenza, ampiezza e fase di ciascuna sinusoide e sommandole insieme, possiamo ottenere un segnale esattamente identico al segnale originale. Questa scomposizione è svolta analiticamente: partendo da un segnale espresso come un'equazione che lo descrive rispetto a un dominio temporale infinito otteniamo un'altra equazione che lo descrive in termini di una somma infinita di sinusoidi.%, che sarà un integrale nel caso di un segnale non periodico e una sommatoria se il segnale è periodico.
%
%Diciamo allora che la FT prende un segnale rappresentato nel \emph{dominio del tempo} (cioè in termini del suo andamento considerato dal punto di vista temporale: vibrazioni, oscillazioni, variazioni di pressione nell'aria o di tensione nel cavo elettrico) e restituisce una sua rappresentazione nel \emph{dominio della frequenza} (cioè in termini del suo andamento considerato dal punto di vista frequenziale: per ciascuna possibile frequenza reale, l'ampiezza e la fase a cui questa è presente nel segnale).
%
%L'informazione frequenziale è evidentemente contenuta nella rappresentazione nel dominio del tempo, tanto è vero che può esserne estratta: ma non è contenuta in maniera esplicita. In casi semplici, è possibile cercare in maniera empirica regolarità in porzioni sempre più ampie del segnale per provare a individuarne componenti di frequenze sempre più gravi; un occhio allenato può cogliere alcune informazioni sul contenuto frequenziale di un segnale non elementare; ma in molti casi si tratta di un'informazione che, ancorché presente, è ben nascosta nella complessità di comportamento del segnale.
%
%Se un segnale viene sottoposto alla trasformata di Fourier, otteniamo una rappresentazione nel dominio frequenziale a cui si possono applicare, in maniera speculare, molte delle considerazioni fatte per il dominio del tempo. In particolare, una rappresentazione frequenziale pura non contiene informazioni esplicite sul comportamento del segnale nel tempo. Questo vuol dire che tutto ciò che noi individuiamo e descriviamo come movimento e variazione del suono nel tempo (successioni di note, di timbri, evoluzioni, ritmi) viene codificato come informazione temporale. Semplificando all'estremo, la trasformata di Fourier di un ritmo a 120 BPM, corrispondente a una periodicità di 0.5 secondi, mostrerà con evidenza una componente alla corrispondente frequenza di 2 Hz; fenomeni temporali che non presentano caratteristiche evidenti di ripetitività si rifletteranno in fenomeni frequenziali complessi e irregolari, difficili se non impossibili da interpretare a occhio. 
%
%La specularità delle due rappresentazioni, temporale e frequenziale, è completa e sorprendente, e produce a volte risultati decisamente controintuitivi. Possiamo a buon diritto considerare la trasformata di Fourier una scatola magica che scambia tra loro le nozioni di tempo e frequenza in qualsiasi frase noi scriviamo o diciamo. L'invertibilità della trasformata di Fourier, che tra l'altro si realizza applicando nuovamente la stessa operazione con un cambio di segno, può essere considerata collegata a questa stessa idea. E un'altra maniera di considerare la trasformata di Fourier è come un sistema di infiniti risuonatori infinitamente selettivi, ciascuno dei quali viene eccitato da una frequenza che può anche essere estremamente bassa, corrispondente a un periodo di un secondo, un minuto, un mese, un anno, un secolo.
%
%
%\subsection{Trasformata a breve termine e trasformata discreta di Fourier}
%
%Abbiamo detto che la nostra percezione del suono non avviene in termini di variazioni micro-temporali, ma in termini di altezza e timbro che sono fenomeni che hanno a che fare, almeno in qualche misura, con la frequenza. Tuttavia, sarebbe inesatto e incompleto dire che la nostra percezione avviene nel dominio della frequenza: le nozioni temporali sono ampiamente presenti nella nostra percezione. Ciò che avviene normalmente è che tendiamo a interpretare i fenomeni di breve periodicità in termini frequenziali e quelli di lunga periodicità in termini temporali. 
%
%
%
%D'altra parte, abbiamo detto che la 
%
%
%L'informazione frequenziale non è, in senso proprio, contenuta nella rappresentazione nel dominio del tempo: il modo banale per rilevarla è considerare porzioni del segnale ampie nel tempo 
%
%
%
%
%
%
%Quando consideriamo fenomeni acustici che misuriamo nel mondo sensibile, non ne conosciamo un'equazione che li descriva in maniera esatta --- né, in generale, in maniera ragionevole anche se inesatta. Per fare ciò utilizziamo una variante della trasformata di Fourier che ci permettono di lavorare con un segnale che misuriamo su una porzione finita di tempo: si tratta della \emph{trasformata discreta di Fourier} (\emph{discrete Fourier transform}, abbreviata in \emph{DFT}). Anche la DFT, che è a sua volta un caso particolare della \emph{trasformata a breve termine di Fourier} (\emph{short-term Fourier transform} o \emph{STFT}) è invertibile, e ci permette tra l'altro di avere una rappresentazione del fenomeno sonoro contingente più prossima alla nostra percezione.
%
%
%
%
%
%
%
%Se, in una data scena acustica, due emittenti vibrano simultaneamente a due frequenze diverse, e abbiamo due risuonatori accordati rispettivamente alle due frequenze, entrambi verranno eccitati, in maniera proporzionale all'intensità di ciascuno dei due emittenti. 
%
%
%Supponiamo che, in una data scena sonora, siano presenti due emittenti che vibrano a frequenze diverse. 
%
%
%
%Come abbiamo detto, il suono è in qualche modo definito da rapide oscillazioni di un corpo che producono variazioni di pressione nell'aria che lo circonda. Queste rapide oscillazioni possono susseguirsi in maniera più o meno regolare nel tempo; la nostra percezione, probabilmente per ottime ragioni evolutive che hanno a che fare con distinguere un leone da una zanzara, tende a focalizzarsi sui fenomeni più regolari. 
%
%
%





