
\chapter{Pressione, ampiezza, intensità}

\section{Pressione sonora}

Il suono è costituito da (o associato a) onde di pressione longitudinali che si propagano in un mezzo elastico (tipicamente l'aria).

L'onda longitudinale è tale perché lo spostamento di energia avviene lungo lo stesso asse dello spostamento di materia (come per esempio nel caso di una molla che viene colpita a un'estremità). Il caso opposto, che non riguarda il suono, è l'onda trasversale, in cui lo spostamento di energia avviene perpendicolarmente rispetto all'asse dello spostamento di materia: l'esempio classico è l'onda del mare, in cui l'energia si sposta orizzontalmente (dal mare aperto verso la riva) mentre la materia si sposta verticalmente (la singola molecola d'acqua si alza e si abbassa a causa del moto ondoso).

Il suono sarà in generale prodotto da un corpo emittente, che per semplicità immaginiamo puntiforme. Nel nostro modello, consideriamo anche la presenza di un recettore, pure puntiforme. Il corpo emittente oscilla in maniera più o meno regolare attorno a un suo punto di riposo. Così facendo, crea nell'aria un'alternanza di compressioni nella direzione del recettore e compressioni nella direzione opposta a quella del recettore (che possiamo considerare decompressioni nella direzione del recettore). Questi fronti di compressioni e decompressioni si propagano in forma di superfici sferiche. 

La pressione si misura in pascal (\unit{Pa}) e i suoi multipli e sottomultipli, come 
\begin{itemize}
\item il millipascal (\unit{mPa}: $\frac{1}{1000}$\unit{Pa}), 
\item il micropascal (\unit{\micro\pascal}: $\frac{1}{1000000}$\unit{Pa}), 
\item il kilopascal (\unit{\kilo\pascal}: \qty{1000}{Pa}),
\item il \unit{bar} (\unit{bar}: \qty{100000}{Pa}).
\end{itemize}

La pressione atmosferica standard al livello del mare è poco più di \qty{100000}{Pa}. Il fenomeno sonoro provoca piccole variazioni nella pressione atmosferica: le variazioni tipiche dei suoni udibili avvengono in un ambito compreso all'incirca tra \qty{20}{\micro\pascal} e \qty{20}{\pascal}. 

Il recettore del fenomeno sonoro (ad esempio la membrana del timpano, o una membrana microfonica) è immerso nell'atmosfera, che esercita eguale pressione da entrambi i lati: di conseguenza, possiamo considerare in generale nulla la pressione atmosferica complessiva a cui la membrana è sottoposta.%
\footnote{Questo non è vero ad esempio nel caso di sbalzi d'altitudine o di immersioni subacquee: in questi casi infatti bisogna compensare le variazioni di pressione a cui il timpano è sottoposto.}
Il fenomeno sonoro provoca invece l'applicazione di una pressione direzionata sul recettore, che ne sarà compresso verso l'interno o verso l'esterno: ha senso quindi dire che il recettore è sottoposto a una pressione sonora dell'ambito descritto sopra (tra \qty{20}{\micro\pascal} e \qty{20}{\pascal}). Per convenzione, parleremo di pressione positiva quando la pressione esterna è superiore alla pressione interna (e quindi la membrana del timpano sarà spinta verso l'interno del cranio); di pressione negativa nel caso opposto (in cui la membrana del timpano è ``risucchiata'' verso l'esterno del cranio).

L'onda sonora impiega tempo a propagarsi: al livello del mare, con aria secca e temperatura di \qty{20}{\celsius}, la velocità di propagazione è di circa \qty{343}{m / s}. Questo vuol dire che, approssimativamente, il suono in queste condizioni percorre \qty{30}{cm} in \qty{1}{ms}; \qty{1}{m} in \qty{3}{ms}; e \qty{1}{km} in \qty{3}{s}.



\section{Ampiezza e rappresentazione nel dominio del tempo}

Ci interessa individuare alcune grandezze che variano in maniera idealmente proporzionale%
\footnote{Nel mondo reale, questa proporzionalità è sempre approssimativa. Rispetto alle definizioni della fisica classica, e in un contesto ideale in cui non esistono fenomeni parassiti che disturbano l'emissione, la trasmissione e la ricezione del suono (come ad esempio rumori d'ambiente, correnti d'aria, superfici riflettenti, l'elasticità non perfetta dei materiali e così via), è spesso utile considerarla esatta per semplicità.}
tra loro, per lo stesso fenomeno sonoro:

\begin{itemize}
\item lo spostamento dell'emittente $s_e$
\item lo spostamento del recettore $s_r$ (se il mezzo elastico non induce distorsioni: in generale possiamo considerare che le distorsioni indotte dall'aria siano estremamente ridotte)
\item le variazioni di pressione atmosferica $\Delta p$ (in assenza di altri fenomeni perturbativi)
\item la tensione elettrica nella catena elettroacustica $V$ (assumendo che trasduttori, conduttori e amplificatori si comportino in maniera ideale)
\end{itemize}

Possiamo allora scrivere che
\begin{equation}
s_e \propto s_r \propto \Delta p \propto V
\end{equation}
dove il simbolo $\propto$ indica una relazione di proporzionalità diretta.

Poiché la percezione e il trattamento del fenomeno sonoro avvengono in generale in termini astratti rispetto alla materialità delle grandezze fisiche coinvolte, spesso utilizziamo il concetto di \emph{ampiezza} del segnale sonoro. L'ampiezza è una misura adimensionale proporzionale alle grandezze elencate sopra; le rappresenta tutte, e non coincide con nessuna in particolare. Il suo ambito è arbitrario e convenzionale; in condizioni normali, si adotta talvolta un ambito compreso tra -1 e 1, ma ci sono moltissime eccezioni. Il concetto di ampiezza è cruciale nella rappresentazione digitale del suono. Quindi
\begin{equation}
A \propto s_e \propto s_r \propto \Delta p \propto V
\end{equation}

Se si traccia l'andamento di una qualsiasi di queste grandezze rispetto al tempo si ottiene un grafico che rappresenta il fenomeno sonoro \emph{nel dominio del tempo}. Il fatto che si tratti di grandezze tutte proporzionali tra loro fa sì che, cambiando l'unità di misura sull'asse verticale, il grafico mantenga la stessa forma. Possiamo quindi parlare di \emph{forma d'onda} (in inglese \emph{waveform}). Questo tipo di grafico è talvolta chiamato \emph{audiogramma}.

INSERIRE QUI ESEMPI DI FORME D'ONDA

\begin{figure}
    \begin{center}
       \scalebox{0.6} {%% Creator: Matplotlib, PGF backend
%%
%% To include the figure in your LaTeX document, write
%%   \input{<filename>.pgf}
%%
%% Make sure the required packages are loaded in your preamble
%%   \usepackage{pgf}
%%
%% Also ensure that all the required font packages are loaded; for instance,
%% the lmodern package is sometimes necessary when using math font.
%%   \usepackage{lmodern}
%%
%% Figures using additional raster images can only be included by \input if
%% they are in the same directory as the main LaTeX file. For loading figures
%% from other directories you can use the `import` package
%%   \usepackage{import}
%%
%% and then include the figures with
%%   \import{<path to file>}{<filename>.pgf}
%%
%% Matplotlib used the following preamble
%%   
%%   \makeatletter\@ifpackageloaded{underscore}{}{\usepackage[strings]{underscore}}\makeatother
%%
\begingroup%
\makeatletter%
\begin{pgfpicture}%
\pgfpathrectangle{\pgfpointorigin}{\pgfqpoint{6.400000in}{4.800000in}}%
\pgfusepath{use as bounding box, clip}%
\begin{pgfscope}%
\pgfsetbuttcap%
\pgfsetmiterjoin%
\definecolor{currentfill}{rgb}{1.000000,1.000000,1.000000}%
\pgfsetfillcolor{currentfill}%
\pgfsetlinewidth{0.000000pt}%
\definecolor{currentstroke}{rgb}{1.000000,1.000000,1.000000}%
\pgfsetstrokecolor{currentstroke}%
\pgfsetdash{}{0pt}%
\pgfpathmoveto{\pgfqpoint{0.000000in}{0.000000in}}%
\pgfpathlineto{\pgfqpoint{6.400000in}{0.000000in}}%
\pgfpathlineto{\pgfqpoint{6.400000in}{4.800000in}}%
\pgfpathlineto{\pgfqpoint{0.000000in}{4.800000in}}%
\pgfpathlineto{\pgfqpoint{0.000000in}{0.000000in}}%
\pgfpathclose%
\pgfusepath{fill}%
\end{pgfscope}%
\begin{pgfscope}%
\pgfsetbuttcap%
\pgfsetmiterjoin%
\definecolor{currentfill}{rgb}{1.000000,1.000000,1.000000}%
\pgfsetfillcolor{currentfill}%
\pgfsetlinewidth{0.000000pt}%
\definecolor{currentstroke}{rgb}{0.000000,0.000000,0.000000}%
\pgfsetstrokecolor{currentstroke}%
\pgfsetstrokeopacity{0.000000}%
\pgfsetdash{}{0pt}%
\pgfpathmoveto{\pgfqpoint{0.800000in}{0.528000in}}%
\pgfpathlineto{\pgfqpoint{5.760000in}{0.528000in}}%
\pgfpathlineto{\pgfqpoint{5.760000in}{4.224000in}}%
\pgfpathlineto{\pgfqpoint{0.800000in}{4.224000in}}%
\pgfpathlineto{\pgfqpoint{0.800000in}{0.528000in}}%
\pgfpathclose%
\pgfusepath{fill}%
\end{pgfscope}%
\begin{pgfscope}%
\pgfsetbuttcap%
\pgfsetroundjoin%
\definecolor{currentfill}{rgb}{0.000000,0.000000,0.000000}%
\pgfsetfillcolor{currentfill}%
\pgfsetlinewidth{0.803000pt}%
\definecolor{currentstroke}{rgb}{0.000000,0.000000,0.000000}%
\pgfsetstrokecolor{currentstroke}%
\pgfsetdash{}{0pt}%
\pgfsys@defobject{currentmarker}{\pgfqpoint{0.000000in}{-0.048611in}}{\pgfqpoint{0.000000in}{0.000000in}}{%
\pgfpathmoveto{\pgfqpoint{0.000000in}{0.000000in}}%
\pgfpathlineto{\pgfqpoint{0.000000in}{-0.048611in}}%
\pgfusepath{stroke,fill}%
}%
\begin{pgfscope}%
\pgfsys@transformshift{1.025455in}{0.528000in}%
\pgfsys@useobject{currentmarker}{}%
\end{pgfscope}%
\end{pgfscope}%
\begin{pgfscope}%
\definecolor{textcolor}{rgb}{0.000000,0.000000,0.000000}%
\pgfsetstrokecolor{textcolor}%
\pgfsetfillcolor{textcolor}%
\pgftext[x=1.025455in,y=0.430778in,,top]{\color{textcolor}\rmfamily\fontsize{10.000000}{12.000000}\selectfont \(\displaystyle {0}\)}%
\end{pgfscope}%
\begin{pgfscope}%
\pgfsetbuttcap%
\pgfsetroundjoin%
\definecolor{currentfill}{rgb}{0.000000,0.000000,0.000000}%
\pgfsetfillcolor{currentfill}%
\pgfsetlinewidth{0.803000pt}%
\definecolor{currentstroke}{rgb}{0.000000,0.000000,0.000000}%
\pgfsetstrokecolor{currentstroke}%
\pgfsetdash{}{0pt}%
\pgfsys@defobject{currentmarker}{\pgfqpoint{0.000000in}{-0.048611in}}{\pgfqpoint{0.000000in}{0.000000in}}{%
\pgfpathmoveto{\pgfqpoint{0.000000in}{0.000000in}}%
\pgfpathlineto{\pgfqpoint{0.000000in}{-0.048611in}}%
\pgfusepath{stroke,fill}%
}%
\begin{pgfscope}%
\pgfsys@transformshift{1.640837in}{0.528000in}%
\pgfsys@useobject{currentmarker}{}%
\end{pgfscope}%
\end{pgfscope}%
\begin{pgfscope}%
\definecolor{textcolor}{rgb}{0.000000,0.000000,0.000000}%
\pgfsetstrokecolor{textcolor}%
\pgfsetfillcolor{textcolor}%
\pgftext[x=1.640837in,y=0.430778in,,top]{\color{textcolor}\rmfamily\fontsize{10.000000}{12.000000}\selectfont \(\displaystyle {50000}\)}%
\end{pgfscope}%
\begin{pgfscope}%
\pgfsetbuttcap%
\pgfsetroundjoin%
\definecolor{currentfill}{rgb}{0.000000,0.000000,0.000000}%
\pgfsetfillcolor{currentfill}%
\pgfsetlinewidth{0.803000pt}%
\definecolor{currentstroke}{rgb}{0.000000,0.000000,0.000000}%
\pgfsetstrokecolor{currentstroke}%
\pgfsetdash{}{0pt}%
\pgfsys@defobject{currentmarker}{\pgfqpoint{0.000000in}{-0.048611in}}{\pgfqpoint{0.000000in}{0.000000in}}{%
\pgfpathmoveto{\pgfqpoint{0.000000in}{0.000000in}}%
\pgfpathlineto{\pgfqpoint{0.000000in}{-0.048611in}}%
\pgfusepath{stroke,fill}%
}%
\begin{pgfscope}%
\pgfsys@transformshift{2.256219in}{0.528000in}%
\pgfsys@useobject{currentmarker}{}%
\end{pgfscope}%
\end{pgfscope}%
\begin{pgfscope}%
\definecolor{textcolor}{rgb}{0.000000,0.000000,0.000000}%
\pgfsetstrokecolor{textcolor}%
\pgfsetfillcolor{textcolor}%
\pgftext[x=2.256219in,y=0.430778in,,top]{\color{textcolor}\rmfamily\fontsize{10.000000}{12.000000}\selectfont \(\displaystyle {100000}\)}%
\end{pgfscope}%
\begin{pgfscope}%
\pgfsetbuttcap%
\pgfsetroundjoin%
\definecolor{currentfill}{rgb}{0.000000,0.000000,0.000000}%
\pgfsetfillcolor{currentfill}%
\pgfsetlinewidth{0.803000pt}%
\definecolor{currentstroke}{rgb}{0.000000,0.000000,0.000000}%
\pgfsetstrokecolor{currentstroke}%
\pgfsetdash{}{0pt}%
\pgfsys@defobject{currentmarker}{\pgfqpoint{0.000000in}{-0.048611in}}{\pgfqpoint{0.000000in}{0.000000in}}{%
\pgfpathmoveto{\pgfqpoint{0.000000in}{0.000000in}}%
\pgfpathlineto{\pgfqpoint{0.000000in}{-0.048611in}}%
\pgfusepath{stroke,fill}%
}%
\begin{pgfscope}%
\pgfsys@transformshift{2.871602in}{0.528000in}%
\pgfsys@useobject{currentmarker}{}%
\end{pgfscope}%
\end{pgfscope}%
\begin{pgfscope}%
\definecolor{textcolor}{rgb}{0.000000,0.000000,0.000000}%
\pgfsetstrokecolor{textcolor}%
\pgfsetfillcolor{textcolor}%
\pgftext[x=2.871602in,y=0.430778in,,top]{\color{textcolor}\rmfamily\fontsize{10.000000}{12.000000}\selectfont \(\displaystyle {150000}\)}%
\end{pgfscope}%
\begin{pgfscope}%
\pgfsetbuttcap%
\pgfsetroundjoin%
\definecolor{currentfill}{rgb}{0.000000,0.000000,0.000000}%
\pgfsetfillcolor{currentfill}%
\pgfsetlinewidth{0.803000pt}%
\definecolor{currentstroke}{rgb}{0.000000,0.000000,0.000000}%
\pgfsetstrokecolor{currentstroke}%
\pgfsetdash{}{0pt}%
\pgfsys@defobject{currentmarker}{\pgfqpoint{0.000000in}{-0.048611in}}{\pgfqpoint{0.000000in}{0.000000in}}{%
\pgfpathmoveto{\pgfqpoint{0.000000in}{0.000000in}}%
\pgfpathlineto{\pgfqpoint{0.000000in}{-0.048611in}}%
\pgfusepath{stroke,fill}%
}%
\begin{pgfscope}%
\pgfsys@transformshift{3.486984in}{0.528000in}%
\pgfsys@useobject{currentmarker}{}%
\end{pgfscope}%
\end{pgfscope}%
\begin{pgfscope}%
\definecolor{textcolor}{rgb}{0.000000,0.000000,0.000000}%
\pgfsetstrokecolor{textcolor}%
\pgfsetfillcolor{textcolor}%
\pgftext[x=3.486984in,y=0.430778in,,top]{\color{textcolor}\rmfamily\fontsize{10.000000}{12.000000}\selectfont \(\displaystyle {200000}\)}%
\end{pgfscope}%
\begin{pgfscope}%
\pgfsetbuttcap%
\pgfsetroundjoin%
\definecolor{currentfill}{rgb}{0.000000,0.000000,0.000000}%
\pgfsetfillcolor{currentfill}%
\pgfsetlinewidth{0.803000pt}%
\definecolor{currentstroke}{rgb}{0.000000,0.000000,0.000000}%
\pgfsetstrokecolor{currentstroke}%
\pgfsetdash{}{0pt}%
\pgfsys@defobject{currentmarker}{\pgfqpoint{0.000000in}{-0.048611in}}{\pgfqpoint{0.000000in}{0.000000in}}{%
\pgfpathmoveto{\pgfqpoint{0.000000in}{0.000000in}}%
\pgfpathlineto{\pgfqpoint{0.000000in}{-0.048611in}}%
\pgfusepath{stroke,fill}%
}%
\begin{pgfscope}%
\pgfsys@transformshift{4.102366in}{0.528000in}%
\pgfsys@useobject{currentmarker}{}%
\end{pgfscope}%
\end{pgfscope}%
\begin{pgfscope}%
\definecolor{textcolor}{rgb}{0.000000,0.000000,0.000000}%
\pgfsetstrokecolor{textcolor}%
\pgfsetfillcolor{textcolor}%
\pgftext[x=4.102366in,y=0.430778in,,top]{\color{textcolor}\rmfamily\fontsize{10.000000}{12.000000}\selectfont \(\displaystyle {250000}\)}%
\end{pgfscope}%
\begin{pgfscope}%
\pgfsetbuttcap%
\pgfsetroundjoin%
\definecolor{currentfill}{rgb}{0.000000,0.000000,0.000000}%
\pgfsetfillcolor{currentfill}%
\pgfsetlinewidth{0.803000pt}%
\definecolor{currentstroke}{rgb}{0.000000,0.000000,0.000000}%
\pgfsetstrokecolor{currentstroke}%
\pgfsetdash{}{0pt}%
\pgfsys@defobject{currentmarker}{\pgfqpoint{0.000000in}{-0.048611in}}{\pgfqpoint{0.000000in}{0.000000in}}{%
\pgfpathmoveto{\pgfqpoint{0.000000in}{0.000000in}}%
\pgfpathlineto{\pgfqpoint{0.000000in}{-0.048611in}}%
\pgfusepath{stroke,fill}%
}%
\begin{pgfscope}%
\pgfsys@transformshift{4.717748in}{0.528000in}%
\pgfsys@useobject{currentmarker}{}%
\end{pgfscope}%
\end{pgfscope}%
\begin{pgfscope}%
\definecolor{textcolor}{rgb}{0.000000,0.000000,0.000000}%
\pgfsetstrokecolor{textcolor}%
\pgfsetfillcolor{textcolor}%
\pgftext[x=4.717748in,y=0.430778in,,top]{\color{textcolor}\rmfamily\fontsize{10.000000}{12.000000}\selectfont \(\displaystyle {300000}\)}%
\end{pgfscope}%
\begin{pgfscope}%
\pgfsetbuttcap%
\pgfsetroundjoin%
\definecolor{currentfill}{rgb}{0.000000,0.000000,0.000000}%
\pgfsetfillcolor{currentfill}%
\pgfsetlinewidth{0.803000pt}%
\definecolor{currentstroke}{rgb}{0.000000,0.000000,0.000000}%
\pgfsetstrokecolor{currentstroke}%
\pgfsetdash{}{0pt}%
\pgfsys@defobject{currentmarker}{\pgfqpoint{0.000000in}{-0.048611in}}{\pgfqpoint{0.000000in}{0.000000in}}{%
\pgfpathmoveto{\pgfqpoint{0.000000in}{0.000000in}}%
\pgfpathlineto{\pgfqpoint{0.000000in}{-0.048611in}}%
\pgfusepath{stroke,fill}%
}%
\begin{pgfscope}%
\pgfsys@transformshift{5.333131in}{0.528000in}%
\pgfsys@useobject{currentmarker}{}%
\end{pgfscope}%
\end{pgfscope}%
\begin{pgfscope}%
\definecolor{textcolor}{rgb}{0.000000,0.000000,0.000000}%
\pgfsetstrokecolor{textcolor}%
\pgfsetfillcolor{textcolor}%
\pgftext[x=5.333131in,y=0.430778in,,top]{\color{textcolor}\rmfamily\fontsize{10.000000}{12.000000}\selectfont \(\displaystyle {350000}\)}%
\end{pgfscope}%
\begin{pgfscope}%
\pgfsetbuttcap%
\pgfsetroundjoin%
\definecolor{currentfill}{rgb}{0.000000,0.000000,0.000000}%
\pgfsetfillcolor{currentfill}%
\pgfsetlinewidth{0.803000pt}%
\definecolor{currentstroke}{rgb}{0.000000,0.000000,0.000000}%
\pgfsetstrokecolor{currentstroke}%
\pgfsetdash{}{0pt}%
\pgfsys@defobject{currentmarker}{\pgfqpoint{-0.048611in}{0.000000in}}{\pgfqpoint{-0.000000in}{0.000000in}}{%
\pgfpathmoveto{\pgfqpoint{-0.000000in}{0.000000in}}%
\pgfpathlineto{\pgfqpoint{-0.048611in}{0.000000in}}%
\pgfusepath{stroke,fill}%
}%
\begin{pgfscope}%
\pgfsys@transformshift{0.800000in}{0.546200in}%
\pgfsys@useobject{currentmarker}{}%
\end{pgfscope}%
\end{pgfscope}%
\begin{pgfscope}%
\definecolor{textcolor}{rgb}{0.000000,0.000000,0.000000}%
\pgfsetstrokecolor{textcolor}%
\pgfsetfillcolor{textcolor}%
\pgftext[x=0.417283in, y=0.497975in, left, base]{\color{textcolor}\rmfamily\fontsize{10.000000}{12.000000}\selectfont \(\displaystyle {\ensuremath{-}2.0}\)}%
\end{pgfscope}%
\begin{pgfscope}%
\pgfsetbuttcap%
\pgfsetroundjoin%
\definecolor{currentfill}{rgb}{0.000000,0.000000,0.000000}%
\pgfsetfillcolor{currentfill}%
\pgfsetlinewidth{0.803000pt}%
\definecolor{currentstroke}{rgb}{0.000000,0.000000,0.000000}%
\pgfsetstrokecolor{currentstroke}%
\pgfsetdash{}{0pt}%
\pgfsys@defobject{currentmarker}{\pgfqpoint{-0.048611in}{0.000000in}}{\pgfqpoint{-0.000000in}{0.000000in}}{%
\pgfpathmoveto{\pgfqpoint{-0.000000in}{0.000000in}}%
\pgfpathlineto{\pgfqpoint{-0.048611in}{0.000000in}}%
\pgfusepath{stroke,fill}%
}%
\begin{pgfscope}%
\pgfsys@transformshift{0.800000in}{1.003650in}%
\pgfsys@useobject{currentmarker}{}%
\end{pgfscope}%
\end{pgfscope}%
\begin{pgfscope}%
\definecolor{textcolor}{rgb}{0.000000,0.000000,0.000000}%
\pgfsetstrokecolor{textcolor}%
\pgfsetfillcolor{textcolor}%
\pgftext[x=0.417283in, y=0.955425in, left, base]{\color{textcolor}\rmfamily\fontsize{10.000000}{12.000000}\selectfont \(\displaystyle {\ensuremath{-}1.5}\)}%
\end{pgfscope}%
\begin{pgfscope}%
\pgfsetbuttcap%
\pgfsetroundjoin%
\definecolor{currentfill}{rgb}{0.000000,0.000000,0.000000}%
\pgfsetfillcolor{currentfill}%
\pgfsetlinewidth{0.803000pt}%
\definecolor{currentstroke}{rgb}{0.000000,0.000000,0.000000}%
\pgfsetstrokecolor{currentstroke}%
\pgfsetdash{}{0pt}%
\pgfsys@defobject{currentmarker}{\pgfqpoint{-0.048611in}{0.000000in}}{\pgfqpoint{-0.000000in}{0.000000in}}{%
\pgfpathmoveto{\pgfqpoint{-0.000000in}{0.000000in}}%
\pgfpathlineto{\pgfqpoint{-0.048611in}{0.000000in}}%
\pgfusepath{stroke,fill}%
}%
\begin{pgfscope}%
\pgfsys@transformshift{0.800000in}{1.461100in}%
\pgfsys@useobject{currentmarker}{}%
\end{pgfscope}%
\end{pgfscope}%
\begin{pgfscope}%
\definecolor{textcolor}{rgb}{0.000000,0.000000,0.000000}%
\pgfsetstrokecolor{textcolor}%
\pgfsetfillcolor{textcolor}%
\pgftext[x=0.417283in, y=1.412875in, left, base]{\color{textcolor}\rmfamily\fontsize{10.000000}{12.000000}\selectfont \(\displaystyle {\ensuremath{-}1.0}\)}%
\end{pgfscope}%
\begin{pgfscope}%
\pgfsetbuttcap%
\pgfsetroundjoin%
\definecolor{currentfill}{rgb}{0.000000,0.000000,0.000000}%
\pgfsetfillcolor{currentfill}%
\pgfsetlinewidth{0.803000pt}%
\definecolor{currentstroke}{rgb}{0.000000,0.000000,0.000000}%
\pgfsetstrokecolor{currentstroke}%
\pgfsetdash{}{0pt}%
\pgfsys@defobject{currentmarker}{\pgfqpoint{-0.048611in}{0.000000in}}{\pgfqpoint{-0.000000in}{0.000000in}}{%
\pgfpathmoveto{\pgfqpoint{-0.000000in}{0.000000in}}%
\pgfpathlineto{\pgfqpoint{-0.048611in}{0.000000in}}%
\pgfusepath{stroke,fill}%
}%
\begin{pgfscope}%
\pgfsys@transformshift{0.800000in}{1.918550in}%
\pgfsys@useobject{currentmarker}{}%
\end{pgfscope}%
\end{pgfscope}%
\begin{pgfscope}%
\definecolor{textcolor}{rgb}{0.000000,0.000000,0.000000}%
\pgfsetstrokecolor{textcolor}%
\pgfsetfillcolor{textcolor}%
\pgftext[x=0.417283in, y=1.870325in, left, base]{\color{textcolor}\rmfamily\fontsize{10.000000}{12.000000}\selectfont \(\displaystyle {\ensuremath{-}0.5}\)}%
\end{pgfscope}%
\begin{pgfscope}%
\pgfsetbuttcap%
\pgfsetroundjoin%
\definecolor{currentfill}{rgb}{0.000000,0.000000,0.000000}%
\pgfsetfillcolor{currentfill}%
\pgfsetlinewidth{0.803000pt}%
\definecolor{currentstroke}{rgb}{0.000000,0.000000,0.000000}%
\pgfsetstrokecolor{currentstroke}%
\pgfsetdash{}{0pt}%
\pgfsys@defobject{currentmarker}{\pgfqpoint{-0.048611in}{0.000000in}}{\pgfqpoint{-0.000000in}{0.000000in}}{%
\pgfpathmoveto{\pgfqpoint{-0.000000in}{0.000000in}}%
\pgfpathlineto{\pgfqpoint{-0.048611in}{0.000000in}}%
\pgfusepath{stroke,fill}%
}%
\begin{pgfscope}%
\pgfsys@transformshift{0.800000in}{2.376000in}%
\pgfsys@useobject{currentmarker}{}%
\end{pgfscope}%
\end{pgfscope}%
\begin{pgfscope}%
\definecolor{textcolor}{rgb}{0.000000,0.000000,0.000000}%
\pgfsetstrokecolor{textcolor}%
\pgfsetfillcolor{textcolor}%
\pgftext[x=0.525308in, y=2.327775in, left, base]{\color{textcolor}\rmfamily\fontsize{10.000000}{12.000000}\selectfont \(\displaystyle {0.0}\)}%
\end{pgfscope}%
\begin{pgfscope}%
\pgfsetbuttcap%
\pgfsetroundjoin%
\definecolor{currentfill}{rgb}{0.000000,0.000000,0.000000}%
\pgfsetfillcolor{currentfill}%
\pgfsetlinewidth{0.803000pt}%
\definecolor{currentstroke}{rgb}{0.000000,0.000000,0.000000}%
\pgfsetstrokecolor{currentstroke}%
\pgfsetdash{}{0pt}%
\pgfsys@defobject{currentmarker}{\pgfqpoint{-0.048611in}{0.000000in}}{\pgfqpoint{-0.000000in}{0.000000in}}{%
\pgfpathmoveto{\pgfqpoint{-0.000000in}{0.000000in}}%
\pgfpathlineto{\pgfqpoint{-0.048611in}{0.000000in}}%
\pgfusepath{stroke,fill}%
}%
\begin{pgfscope}%
\pgfsys@transformshift{0.800000in}{2.833450in}%
\pgfsys@useobject{currentmarker}{}%
\end{pgfscope}%
\end{pgfscope}%
\begin{pgfscope}%
\definecolor{textcolor}{rgb}{0.000000,0.000000,0.000000}%
\pgfsetstrokecolor{textcolor}%
\pgfsetfillcolor{textcolor}%
\pgftext[x=0.525308in, y=2.785225in, left, base]{\color{textcolor}\rmfamily\fontsize{10.000000}{12.000000}\selectfont \(\displaystyle {0.5}\)}%
\end{pgfscope}%
\begin{pgfscope}%
\pgfsetbuttcap%
\pgfsetroundjoin%
\definecolor{currentfill}{rgb}{0.000000,0.000000,0.000000}%
\pgfsetfillcolor{currentfill}%
\pgfsetlinewidth{0.803000pt}%
\definecolor{currentstroke}{rgb}{0.000000,0.000000,0.000000}%
\pgfsetstrokecolor{currentstroke}%
\pgfsetdash{}{0pt}%
\pgfsys@defobject{currentmarker}{\pgfqpoint{-0.048611in}{0.000000in}}{\pgfqpoint{-0.000000in}{0.000000in}}{%
\pgfpathmoveto{\pgfqpoint{-0.000000in}{0.000000in}}%
\pgfpathlineto{\pgfqpoint{-0.048611in}{0.000000in}}%
\pgfusepath{stroke,fill}%
}%
\begin{pgfscope}%
\pgfsys@transformshift{0.800000in}{3.290900in}%
\pgfsys@useobject{currentmarker}{}%
\end{pgfscope}%
\end{pgfscope}%
\begin{pgfscope}%
\definecolor{textcolor}{rgb}{0.000000,0.000000,0.000000}%
\pgfsetstrokecolor{textcolor}%
\pgfsetfillcolor{textcolor}%
\pgftext[x=0.525308in, y=3.242675in, left, base]{\color{textcolor}\rmfamily\fontsize{10.000000}{12.000000}\selectfont \(\displaystyle {1.0}\)}%
\end{pgfscope}%
\begin{pgfscope}%
\pgfsetbuttcap%
\pgfsetroundjoin%
\definecolor{currentfill}{rgb}{0.000000,0.000000,0.000000}%
\pgfsetfillcolor{currentfill}%
\pgfsetlinewidth{0.803000pt}%
\definecolor{currentstroke}{rgb}{0.000000,0.000000,0.000000}%
\pgfsetstrokecolor{currentstroke}%
\pgfsetdash{}{0pt}%
\pgfsys@defobject{currentmarker}{\pgfqpoint{-0.048611in}{0.000000in}}{\pgfqpoint{-0.000000in}{0.000000in}}{%
\pgfpathmoveto{\pgfqpoint{-0.000000in}{0.000000in}}%
\pgfpathlineto{\pgfqpoint{-0.048611in}{0.000000in}}%
\pgfusepath{stroke,fill}%
}%
\begin{pgfscope}%
\pgfsys@transformshift{0.800000in}{3.748350in}%
\pgfsys@useobject{currentmarker}{}%
\end{pgfscope}%
\end{pgfscope}%
\begin{pgfscope}%
\definecolor{textcolor}{rgb}{0.000000,0.000000,0.000000}%
\pgfsetstrokecolor{textcolor}%
\pgfsetfillcolor{textcolor}%
\pgftext[x=0.525308in, y=3.700125in, left, base]{\color{textcolor}\rmfamily\fontsize{10.000000}{12.000000}\selectfont \(\displaystyle {1.5}\)}%
\end{pgfscope}%
\begin{pgfscope}%
\pgfsetbuttcap%
\pgfsetroundjoin%
\definecolor{currentfill}{rgb}{0.000000,0.000000,0.000000}%
\pgfsetfillcolor{currentfill}%
\pgfsetlinewidth{0.803000pt}%
\definecolor{currentstroke}{rgb}{0.000000,0.000000,0.000000}%
\pgfsetstrokecolor{currentstroke}%
\pgfsetdash{}{0pt}%
\pgfsys@defobject{currentmarker}{\pgfqpoint{-0.048611in}{0.000000in}}{\pgfqpoint{-0.000000in}{0.000000in}}{%
\pgfpathmoveto{\pgfqpoint{-0.000000in}{0.000000in}}%
\pgfpathlineto{\pgfqpoint{-0.048611in}{0.000000in}}%
\pgfusepath{stroke,fill}%
}%
\begin{pgfscope}%
\pgfsys@transformshift{0.800000in}{4.205800in}%
\pgfsys@useobject{currentmarker}{}%
\end{pgfscope}%
\end{pgfscope}%
\begin{pgfscope}%
\definecolor{textcolor}{rgb}{0.000000,0.000000,0.000000}%
\pgfsetstrokecolor{textcolor}%
\pgfsetfillcolor{textcolor}%
\pgftext[x=0.525308in, y=4.157575in, left, base]{\color{textcolor}\rmfamily\fontsize{10.000000}{12.000000}\selectfont \(\displaystyle {2.0}\)}%
\end{pgfscope}%
\begin{pgfscope}%
\pgfpathrectangle{\pgfqpoint{0.800000in}{0.528000in}}{\pgfqpoint{4.960000in}{3.696000in}}%
\pgfusepath{clip}%
\pgfsetrectcap%
\pgfsetroundjoin%
\pgfsetlinewidth{1.505625pt}%
\definecolor{currentstroke}{rgb}{0.121569,0.466667,0.705882}%
\pgfsetstrokecolor{currentstroke}%
\pgfsetdash{}{0pt}%
\pgfpathmoveto{\pgfqpoint{1.025455in}{2.376000in}}%
\pgfpathlineto{\pgfqpoint{1.026131in}{4.056000in}}%
\pgfpathlineto{\pgfqpoint{1.026858in}{3.648003in}}%
\pgfpathlineto{\pgfqpoint{1.027473in}{3.865471in}}%
\pgfpathlineto{\pgfqpoint{1.028199in}{3.684518in}}%
\pgfpathlineto{\pgfqpoint{1.028581in}{3.767955in}}%
\pgfpathlineto{\pgfqpoint{1.028815in}{3.799651in}}%
\pgfpathlineto{\pgfqpoint{1.029368in}{3.691596in}}%
\pgfpathlineto{\pgfqpoint{1.029528in}{3.678771in}}%
\pgfpathlineto{\pgfqpoint{1.030082in}{3.752304in}}%
\pgfpathlineto{\pgfqpoint{1.030156in}{3.754790in}}%
\pgfpathlineto{\pgfqpoint{1.030501in}{3.710940in}}%
\pgfpathlineto{\pgfqpoint{1.030882in}{3.662054in}}%
\pgfpathlineto{\pgfqpoint{1.031498in}{3.717126in}}%
\pgfpathlineto{\pgfqpoint{1.031596in}{3.713919in}}%
\pgfpathlineto{\pgfqpoint{1.033602in}{3.616730in}}%
\pgfpathlineto{\pgfqpoint{1.033873in}{3.632020in}}%
\pgfpathlineto{\pgfqpoint{1.034193in}{3.650206in}}%
\pgfpathlineto{\pgfqpoint{1.034685in}{3.606901in}}%
\pgfpathlineto{\pgfqpoint{1.035535in}{3.618729in}}%
\pgfpathlineto{\pgfqpoint{1.036125in}{3.572035in}}%
\pgfpathlineto{\pgfqpoint{1.037664in}{3.538706in}}%
\pgfpathlineto{\pgfqpoint{1.036876in}{3.587947in}}%
\pgfpathlineto{\pgfqpoint{1.037713in}{3.539054in}}%
\pgfpathlineto{\pgfqpoint{1.038218in}{3.557642in}}%
\pgfpathlineto{\pgfqpoint{1.038599in}{3.536311in}}%
\pgfpathlineto{\pgfqpoint{1.040248in}{3.486752in}}%
\pgfpathlineto{\pgfqpoint{1.041738in}{3.456663in}}%
\pgfpathlineto{\pgfqpoint{1.040901in}{3.497967in}}%
\pgfpathlineto{\pgfqpoint{1.042008in}{3.463133in}}%
\pgfpathlineto{\pgfqpoint{1.042242in}{3.468446in}}%
\pgfpathlineto{\pgfqpoint{1.042673in}{3.448332in}}%
\pgfpathlineto{\pgfqpoint{1.044396in}{3.401340in}}%
\pgfpathlineto{\pgfqpoint{1.044458in}{3.401013in}}%
\pgfpathlineto{\pgfqpoint{1.044814in}{3.408511in}}%
\pgfpathlineto{\pgfqpoint{1.044938in}{3.409832in}}%
\pgfpathlineto{\pgfqpoint{1.045294in}{3.396478in}}%
\pgfpathlineto{\pgfqpoint{1.047165in}{3.344924in}}%
\pgfpathlineto{\pgfqpoint{1.047178in}{3.344937in}}%
\pgfpathlineto{\pgfqpoint{1.047448in}{3.349468in}}%
\pgfpathlineto{\pgfqpoint{1.047621in}{3.351601in}}%
\pgfpathlineto{\pgfqpoint{1.048002in}{3.338418in}}%
\pgfpathlineto{\pgfqpoint{1.049885in}{3.288554in}}%
\pgfpathlineto{\pgfqpoint{1.049910in}{3.288622in}}%
\pgfpathlineto{\pgfqpoint{1.050316in}{3.293629in}}%
\pgfpathlineto{\pgfqpoint{1.050599in}{3.286467in}}%
\pgfpathlineto{\pgfqpoint{1.052593in}{3.231978in}}%
\pgfpathlineto{\pgfqpoint{1.052778in}{3.233670in}}%
\pgfpathlineto{\pgfqpoint{1.052999in}{3.235847in}}%
\pgfpathlineto{\pgfqpoint{1.053356in}{3.225994in}}%
\pgfpathlineto{\pgfqpoint{1.055313in}{3.175261in}}%
\pgfpathlineto{\pgfqpoint{1.055424in}{3.175928in}}%
\pgfpathlineto{\pgfqpoint{1.055682in}{3.178198in}}%
\pgfpathlineto{\pgfqpoint{1.056014in}{3.170487in}}%
\pgfpathlineto{\pgfqpoint{1.058021in}{3.118432in}}%
\pgfpathlineto{\pgfqpoint{1.058218in}{3.119755in}}%
\pgfpathlineto{\pgfqpoint{1.058365in}{3.120648in}}%
\pgfpathlineto{\pgfqpoint{1.058685in}{3.114229in}}%
\pgfpathlineto{\pgfqpoint{1.060741in}{3.061525in}}%
\pgfpathlineto{\pgfqpoint{1.061024in}{3.063131in}}%
\pgfpathlineto{\pgfqpoint{1.061061in}{3.063178in}}%
\pgfpathlineto{\pgfqpoint{1.061245in}{3.061168in}}%
\pgfpathlineto{\pgfqpoint{1.073270in}{2.804655in}}%
\pgfpathlineto{\pgfqpoint{1.073750in}{2.789510in}}%
\pgfpathlineto{\pgfqpoint{1.075657in}{2.747672in}}%
\pgfpathlineto{\pgfqpoint{1.075793in}{2.747858in}}%
\pgfpathlineto{\pgfqpoint{1.075854in}{2.747882in}}%
\pgfpathlineto{\pgfqpoint{1.076039in}{2.746457in}}%
\pgfpathlineto{\pgfqpoint{1.079140in}{2.675570in}}%
\pgfpathlineto{\pgfqpoint{1.081085in}{2.633353in}}%
\pgfpathlineto{\pgfqpoint{1.081220in}{2.633421in}}%
\pgfpathlineto{\pgfqpoint{1.081417in}{2.632193in}}%
\pgfpathlineto{\pgfqpoint{1.083202in}{2.589308in}}%
\pgfpathlineto{\pgfqpoint{1.085282in}{2.547580in}}%
\pgfpathlineto{\pgfqpoint{1.085516in}{2.545498in}}%
\pgfpathlineto{\pgfqpoint{1.094230in}{2.352103in}}%
\pgfpathlineto{\pgfqpoint{1.097504in}{2.289953in}}%
\pgfpathlineto{\pgfqpoint{1.097873in}{2.282143in}}%
\pgfpathlineto{\pgfqpoint{1.100322in}{2.231661in}}%
\pgfpathlineto{\pgfqpoint{1.102143in}{2.187256in}}%
\pgfpathlineto{\pgfqpoint{1.104211in}{2.147184in}}%
\pgfpathlineto{\pgfqpoint{1.104482in}{2.143767in}}%
\pgfpathlineto{\pgfqpoint{1.110771in}{2.004110in}}%
\pgfpathlineto{\pgfqpoint{1.110931in}{2.004326in}}%
\pgfpathlineto{\pgfqpoint{1.111054in}{2.003948in}}%
\pgfpathlineto{\pgfqpoint{1.111116in}{2.003310in}}%
\pgfpathlineto{\pgfqpoint{1.112814in}{1.964121in}}%
\pgfpathlineto{\pgfqpoint{1.114796in}{1.918214in}}%
\pgfpathlineto{\pgfqpoint{1.114956in}{1.918566in}}%
\pgfpathlineto{\pgfqpoint{1.115017in}{1.918635in}}%
\pgfpathlineto{\pgfqpoint{1.115227in}{1.916757in}}%
\pgfpathlineto{\pgfqpoint{1.127965in}{1.635895in}}%
\pgfpathlineto{\pgfqpoint{1.129577in}{1.602588in}}%
\pgfpathlineto{\pgfqpoint{1.129971in}{1.605076in}}%
\pgfpathlineto{\pgfqpoint{1.130254in}{1.598281in}}%
\pgfpathlineto{\pgfqpoint{1.132260in}{1.544991in}}%
\pgfpathlineto{\pgfqpoint{1.132519in}{1.547470in}}%
\pgfpathlineto{\pgfqpoint{1.132654in}{1.548365in}}%
\pgfpathlineto{\pgfqpoint{1.132974in}{1.540573in}}%
\pgfpathlineto{\pgfqpoint{1.134943in}{1.487281in}}%
\pgfpathlineto{\pgfqpoint{1.135153in}{1.489426in}}%
\pgfpathlineto{\pgfqpoint{1.135362in}{1.491714in}}%
\pgfpathlineto{\pgfqpoint{1.135731in}{1.480842in}}%
\pgfpathlineto{\pgfqpoint{1.137639in}{1.429411in}}%
\pgfpathlineto{\pgfqpoint{1.137737in}{1.430165in}}%
\pgfpathlineto{\pgfqpoint{1.138082in}{1.435232in}}%
\pgfpathlineto{\pgfqpoint{1.138426in}{1.424307in}}%
\pgfpathlineto{\pgfqpoint{1.140322in}{1.371322in}}%
\pgfpathlineto{\pgfqpoint{1.140420in}{1.372145in}}%
\pgfpathlineto{\pgfqpoint{1.140790in}{1.378990in}}%
\pgfpathlineto{\pgfqpoint{1.141146in}{1.366350in}}%
\pgfpathlineto{\pgfqpoint{1.143005in}{1.312920in}}%
\pgfpathlineto{\pgfqpoint{1.143066in}{1.313264in}}%
\pgfpathlineto{\pgfqpoint{1.143509in}{1.323097in}}%
\pgfpathlineto{\pgfqpoint{1.143854in}{1.309162in}}%
\pgfpathlineto{\pgfqpoint{1.145700in}{1.254033in}}%
\pgfpathlineto{\pgfqpoint{1.145713in}{1.254066in}}%
\pgfpathlineto{\pgfqpoint{1.145983in}{1.261861in}}%
\pgfpathlineto{\pgfqpoint{1.146217in}{1.267744in}}%
\pgfpathlineto{\pgfqpoint{1.146648in}{1.244769in}}%
\pgfpathlineto{\pgfqpoint{1.148346in}{1.194582in}}%
\pgfpathlineto{\pgfqpoint{1.148383in}{1.194358in}}%
\pgfpathlineto{\pgfqpoint{1.148642in}{1.202867in}}%
\pgfpathlineto{\pgfqpoint{1.148937in}{1.213294in}}%
\pgfpathlineto{\pgfqpoint{1.149380in}{1.184083in}}%
\pgfpathlineto{\pgfqpoint{1.150291in}{1.186620in}}%
\pgfpathlineto{\pgfqpoint{1.150993in}{1.134542in}}%
\pgfpathlineto{\pgfqpoint{1.151066in}{1.133271in}}%
\pgfpathlineto{\pgfqpoint{1.151460in}{1.153842in}}%
\pgfpathlineto{\pgfqpoint{1.151645in}{1.160517in}}%
\pgfpathlineto{\pgfqpoint{1.152088in}{1.124714in}}%
\pgfpathlineto{\pgfqpoint{1.152408in}{1.101794in}}%
\pgfpathlineto{\pgfqpoint{1.152999in}{1.135270in}}%
\pgfpathlineto{\pgfqpoint{1.153159in}{1.129103in}}%
\pgfpathlineto{\pgfqpoint{1.155103in}{1.034874in}}%
\pgfpathlineto{\pgfqpoint{1.155276in}{1.044844in}}%
\pgfpathlineto{\pgfqpoint{1.155719in}{1.089946in}}%
\pgfpathlineto{\pgfqpoint{1.156186in}{1.024161in}}%
\pgfpathlineto{\pgfqpoint{1.156445in}{0.997210in}}%
\pgfpathlineto{\pgfqpoint{1.157023in}{1.071844in}}%
\pgfpathlineto{\pgfqpoint{1.157073in}{1.073229in}}%
\pgfpathlineto{\pgfqpoint{1.157343in}{1.037763in}}%
\pgfpathlineto{\pgfqpoint{1.157786in}{0.952349in}}%
\pgfpathlineto{\pgfqpoint{1.158340in}{1.061873in}}%
\pgfpathlineto{\pgfqpoint{1.158426in}{1.068181in}}%
\pgfpathlineto{\pgfqpoint{1.158783in}{0.981804in}}%
\pgfpathlineto{\pgfqpoint{1.159128in}{0.886529in}}%
\pgfpathlineto{\pgfqpoint{1.159645in}{1.078072in}}%
\pgfpathlineto{\pgfqpoint{1.159793in}{1.107213in}}%
\pgfpathlineto{\pgfqpoint{1.160186in}{0.876497in}}%
\pgfpathlineto{\pgfqpoint{1.160469in}{0.696000in}}%
\pgfpathlineto{\pgfqpoint{1.160851in}{1.249397in}}%
\pgfpathlineto{\pgfqpoint{1.161823in}{4.056000in}}%
\pgfpathlineto{\pgfqpoint{1.162771in}{3.730654in}}%
\pgfpathlineto{\pgfqpoint{1.163165in}{3.865471in}}%
\pgfpathlineto{\pgfqpoint{1.163792in}{3.687416in}}%
\pgfpathlineto{\pgfqpoint{1.163866in}{3.683819in}}%
\pgfpathlineto{\pgfqpoint{1.164211in}{3.751759in}}%
\pgfpathlineto{\pgfqpoint{1.164506in}{3.799651in}}%
\pgfpathlineto{\pgfqpoint{1.165085in}{3.687935in}}%
\pgfpathlineto{\pgfqpoint{1.165220in}{3.678771in}}%
\pgfpathlineto{\pgfqpoint{1.165725in}{3.748065in}}%
\pgfpathlineto{\pgfqpoint{1.165848in}{3.754790in}}%
\pgfpathlineto{\pgfqpoint{1.166279in}{3.693807in}}%
\pgfpathlineto{\pgfqpoint{1.166574in}{3.662054in}}%
\pgfpathlineto{\pgfqpoint{1.167189in}{3.717126in}}%
\pgfpathlineto{\pgfqpoint{1.167276in}{3.714684in}}%
\pgfpathlineto{\pgfqpoint{1.169294in}{3.616730in}}%
\pgfpathlineto{\pgfqpoint{1.169626in}{3.637392in}}%
\pgfpathlineto{\pgfqpoint{1.169885in}{3.650206in}}%
\pgfpathlineto{\pgfqpoint{1.170377in}{3.606901in}}%
\pgfpathlineto{\pgfqpoint{1.171226in}{3.618729in}}%
\pgfpathlineto{\pgfqpoint{1.171817in}{3.572035in}}%
\pgfpathlineto{\pgfqpoint{1.173356in}{3.538706in}}%
\pgfpathlineto{\pgfqpoint{1.172568in}{3.587947in}}%
\pgfpathlineto{\pgfqpoint{1.173405in}{3.539054in}}%
\pgfpathlineto{\pgfqpoint{1.173909in}{3.557642in}}%
\pgfpathlineto{\pgfqpoint{1.174291in}{3.536311in}}%
\pgfpathlineto{\pgfqpoint{1.175940in}{3.486752in}}%
\pgfpathlineto{\pgfqpoint{1.177429in}{3.456663in}}%
\pgfpathlineto{\pgfqpoint{1.176592in}{3.497967in}}%
\pgfpathlineto{\pgfqpoint{1.177700in}{3.463133in}}%
\pgfpathlineto{\pgfqpoint{1.177934in}{3.468446in}}%
\pgfpathlineto{\pgfqpoint{1.178365in}{3.448332in}}%
\pgfpathlineto{\pgfqpoint{1.180088in}{3.401340in}}%
\pgfpathlineto{\pgfqpoint{1.180149in}{3.401013in}}%
\pgfpathlineto{\pgfqpoint{1.180506in}{3.408511in}}%
\pgfpathlineto{\pgfqpoint{1.180629in}{3.409832in}}%
\pgfpathlineto{\pgfqpoint{1.180986in}{3.396478in}}%
\pgfpathlineto{\pgfqpoint{1.182857in}{3.344924in}}%
\pgfpathlineto{\pgfqpoint{1.182869in}{3.344937in}}%
\pgfpathlineto{\pgfqpoint{1.183140in}{3.349468in}}%
\pgfpathlineto{\pgfqpoint{1.183312in}{3.351601in}}%
\pgfpathlineto{\pgfqpoint{1.183694in}{3.338418in}}%
\pgfpathlineto{\pgfqpoint{1.185577in}{3.288554in}}%
\pgfpathlineto{\pgfqpoint{1.185602in}{3.288622in}}%
\pgfpathlineto{\pgfqpoint{1.186008in}{3.293629in}}%
\pgfpathlineto{\pgfqpoint{1.186291in}{3.286467in}}%
\pgfpathlineto{\pgfqpoint{1.188285in}{3.231978in}}%
\pgfpathlineto{\pgfqpoint{1.188469in}{3.233670in}}%
\pgfpathlineto{\pgfqpoint{1.188691in}{3.235847in}}%
\pgfpathlineto{\pgfqpoint{1.189048in}{3.225994in}}%
\pgfpathlineto{\pgfqpoint{1.191005in}{3.175261in}}%
\pgfpathlineto{\pgfqpoint{1.191115in}{3.175928in}}%
\pgfpathlineto{\pgfqpoint{1.191374in}{3.178198in}}%
\pgfpathlineto{\pgfqpoint{1.191706in}{3.170487in}}%
\pgfpathlineto{\pgfqpoint{1.193712in}{3.118432in}}%
\pgfpathlineto{\pgfqpoint{1.193909in}{3.119755in}}%
\pgfpathlineto{\pgfqpoint{1.194057in}{3.120648in}}%
\pgfpathlineto{\pgfqpoint{1.194377in}{3.114229in}}%
\pgfpathlineto{\pgfqpoint{1.196432in}{3.061525in}}%
\pgfpathlineto{\pgfqpoint{1.196715in}{3.063131in}}%
\pgfpathlineto{\pgfqpoint{1.196752in}{3.063178in}}%
\pgfpathlineto{\pgfqpoint{1.196937in}{3.061168in}}%
\pgfpathlineto{\pgfqpoint{1.208962in}{2.804655in}}%
\pgfpathlineto{\pgfqpoint{1.209442in}{2.789510in}}%
\pgfpathlineto{\pgfqpoint{1.211349in}{2.747672in}}%
\pgfpathlineto{\pgfqpoint{1.211485in}{2.747858in}}%
\pgfpathlineto{\pgfqpoint{1.211546in}{2.747882in}}%
\pgfpathlineto{\pgfqpoint{1.211731in}{2.746457in}}%
\pgfpathlineto{\pgfqpoint{1.214832in}{2.675570in}}%
\pgfpathlineto{\pgfqpoint{1.216777in}{2.633353in}}%
\pgfpathlineto{\pgfqpoint{1.216912in}{2.633421in}}%
\pgfpathlineto{\pgfqpoint{1.217109in}{2.632193in}}%
\pgfpathlineto{\pgfqpoint{1.218894in}{2.589308in}}%
\pgfpathlineto{\pgfqpoint{1.220974in}{2.547580in}}%
\pgfpathlineto{\pgfqpoint{1.221208in}{2.545498in}}%
\pgfpathlineto{\pgfqpoint{1.229921in}{2.352103in}}%
\pgfpathlineto{\pgfqpoint{1.233195in}{2.289953in}}%
\pgfpathlineto{\pgfqpoint{1.233565in}{2.282143in}}%
\pgfpathlineto{\pgfqpoint{1.236014in}{2.231661in}}%
\pgfpathlineto{\pgfqpoint{1.237835in}{2.187256in}}%
\pgfpathlineto{\pgfqpoint{1.239903in}{2.147184in}}%
\pgfpathlineto{\pgfqpoint{1.240174in}{2.143767in}}%
\pgfpathlineto{\pgfqpoint{1.246463in}{2.004110in}}%
\pgfpathlineto{\pgfqpoint{1.246623in}{2.004326in}}%
\pgfpathlineto{\pgfqpoint{1.246746in}{2.003948in}}%
\pgfpathlineto{\pgfqpoint{1.246808in}{2.003310in}}%
\pgfpathlineto{\pgfqpoint{1.248506in}{1.964121in}}%
\pgfpathlineto{\pgfqpoint{1.250488in}{1.918214in}}%
\pgfpathlineto{\pgfqpoint{1.250648in}{1.918566in}}%
\pgfpathlineto{\pgfqpoint{1.250709in}{1.918635in}}%
\pgfpathlineto{\pgfqpoint{1.250918in}{1.916757in}}%
\pgfpathlineto{\pgfqpoint{1.263657in}{1.635895in}}%
\pgfpathlineto{\pgfqpoint{1.265269in}{1.602588in}}%
\pgfpathlineto{\pgfqpoint{1.265663in}{1.605076in}}%
\pgfpathlineto{\pgfqpoint{1.265946in}{1.598281in}}%
\pgfpathlineto{\pgfqpoint{1.267952in}{1.544991in}}%
\pgfpathlineto{\pgfqpoint{1.268211in}{1.547470in}}%
\pgfpathlineto{\pgfqpoint{1.268346in}{1.548365in}}%
\pgfpathlineto{\pgfqpoint{1.268666in}{1.540573in}}%
\pgfpathlineto{\pgfqpoint{1.270635in}{1.487281in}}%
\pgfpathlineto{\pgfqpoint{1.270844in}{1.489426in}}%
\pgfpathlineto{\pgfqpoint{1.271054in}{1.491714in}}%
\pgfpathlineto{\pgfqpoint{1.271423in}{1.480842in}}%
\pgfpathlineto{\pgfqpoint{1.273331in}{1.429411in}}%
\pgfpathlineto{\pgfqpoint{1.273429in}{1.430165in}}%
\pgfpathlineto{\pgfqpoint{1.273774in}{1.435232in}}%
\pgfpathlineto{\pgfqpoint{1.274118in}{1.424307in}}%
\pgfpathlineto{\pgfqpoint{1.276014in}{1.371322in}}%
\pgfpathlineto{\pgfqpoint{1.276112in}{1.372145in}}%
\pgfpathlineto{\pgfqpoint{1.276481in}{1.378990in}}%
\pgfpathlineto{\pgfqpoint{1.276838in}{1.366350in}}%
\pgfpathlineto{\pgfqpoint{1.278697in}{1.312920in}}%
\pgfpathlineto{\pgfqpoint{1.278758in}{1.313264in}}%
\pgfpathlineto{\pgfqpoint{1.279201in}{1.323097in}}%
\pgfpathlineto{\pgfqpoint{1.279546in}{1.309162in}}%
\pgfpathlineto{\pgfqpoint{1.281392in}{1.254033in}}%
\pgfpathlineto{\pgfqpoint{1.281404in}{1.254066in}}%
\pgfpathlineto{\pgfqpoint{1.281675in}{1.261861in}}%
\pgfpathlineto{\pgfqpoint{1.281909in}{1.267744in}}%
\pgfpathlineto{\pgfqpoint{1.282340in}{1.244769in}}%
\pgfpathlineto{\pgfqpoint{1.284038in}{1.194582in}}%
\pgfpathlineto{\pgfqpoint{1.284075in}{1.194358in}}%
\pgfpathlineto{\pgfqpoint{1.284334in}{1.202867in}}%
\pgfpathlineto{\pgfqpoint{1.284629in}{1.213294in}}%
\pgfpathlineto{\pgfqpoint{1.285072in}{1.184083in}}%
\pgfpathlineto{\pgfqpoint{1.285983in}{1.186620in}}%
\pgfpathlineto{\pgfqpoint{1.286684in}{1.134542in}}%
\pgfpathlineto{\pgfqpoint{1.286758in}{1.133271in}}%
\pgfpathlineto{\pgfqpoint{1.287152in}{1.153842in}}%
\pgfpathlineto{\pgfqpoint{1.287337in}{1.160517in}}%
\pgfpathlineto{\pgfqpoint{1.287780in}{1.124714in}}%
\pgfpathlineto{\pgfqpoint{1.288100in}{1.101794in}}%
\pgfpathlineto{\pgfqpoint{1.288690in}{1.135270in}}%
\pgfpathlineto{\pgfqpoint{1.288850in}{1.129103in}}%
\pgfpathlineto{\pgfqpoint{1.290795in}{1.034874in}}%
\pgfpathlineto{\pgfqpoint{1.290967in}{1.044844in}}%
\pgfpathlineto{\pgfqpoint{1.291410in}{1.089946in}}%
\pgfpathlineto{\pgfqpoint{1.291878in}{1.024161in}}%
\pgfpathlineto{\pgfqpoint{1.292137in}{0.997210in}}%
\pgfpathlineto{\pgfqpoint{1.292715in}{1.071844in}}%
\pgfpathlineto{\pgfqpoint{1.292764in}{1.073229in}}%
\pgfpathlineto{\pgfqpoint{1.293035in}{1.037763in}}%
\pgfpathlineto{\pgfqpoint{1.293478in}{0.952349in}}%
\pgfpathlineto{\pgfqpoint{1.294032in}{1.061873in}}%
\pgfpathlineto{\pgfqpoint{1.294118in}{1.068181in}}%
\pgfpathlineto{\pgfqpoint{1.294475in}{0.981804in}}%
\pgfpathlineto{\pgfqpoint{1.294820in}{0.886529in}}%
\pgfpathlineto{\pgfqpoint{1.295337in}{1.078072in}}%
\pgfpathlineto{\pgfqpoint{1.295484in}{1.107213in}}%
\pgfpathlineto{\pgfqpoint{1.295878in}{0.876497in}}%
\pgfpathlineto{\pgfqpoint{1.296161in}{0.696000in}}%
\pgfpathlineto{\pgfqpoint{1.296543in}{1.249397in}}%
\pgfpathlineto{\pgfqpoint{1.296826in}{2.323869in}}%
\pgfpathlineto{\pgfqpoint{1.296826in}{2.323869in}}%
\pgfusepath{stroke}%
\end{pgfscope}%
\begin{pgfscope}%
\pgfpathrectangle{\pgfqpoint{0.800000in}{0.528000in}}{\pgfqpoint{4.960000in}{3.696000in}}%
\pgfusepath{clip}%
\pgfsetrectcap%
\pgfsetroundjoin%
\pgfsetlinewidth{1.505625pt}%
\definecolor{currentstroke}{rgb}{1.000000,0.498039,0.054902}%
\pgfsetstrokecolor{currentstroke}%
\pgfsetdash{}{0pt}%
\pgfpathmoveto{\pgfqpoint{1.025455in}{2.376000in}}%
\pgfpathlineto{\pgfqpoint{1.030218in}{3.150546in}}%
\pgfpathlineto{\pgfqpoint{1.032236in}{3.225575in}}%
\pgfpathlineto{\pgfqpoint{1.032285in}{3.225541in}}%
\pgfpathlineto{\pgfqpoint{1.032667in}{3.222793in}}%
\pgfpathlineto{\pgfqpoint{1.033762in}{3.194594in}}%
\pgfpathlineto{\pgfqpoint{1.039030in}{3.019751in}}%
\pgfpathlineto{\pgfqpoint{1.039571in}{3.022184in}}%
\pgfpathlineto{\pgfqpoint{1.040790in}{3.042708in}}%
\pgfpathlineto{\pgfqpoint{1.045811in}{3.150059in}}%
\pgfpathlineto{\pgfqpoint{1.046341in}{3.148370in}}%
\pgfpathlineto{\pgfqpoint{1.047522in}{3.133761in}}%
\pgfpathlineto{\pgfqpoint{1.052593in}{3.047010in}}%
\pgfpathlineto{\pgfqpoint{1.053356in}{3.049960in}}%
\pgfpathlineto{\pgfqpoint{1.054796in}{3.069511in}}%
\pgfpathlineto{\pgfqpoint{1.059387in}{3.139869in}}%
\pgfpathlineto{\pgfqpoint{1.059682in}{3.139415in}}%
\pgfpathlineto{\pgfqpoint{1.060556in}{3.133221in}}%
\pgfpathlineto{\pgfqpoint{1.062513in}{3.099353in}}%
\pgfpathlineto{\pgfqpoint{1.066082in}{3.047043in}}%
\pgfpathlineto{\pgfqpoint{1.066291in}{3.047096in}}%
\pgfpathlineto{\pgfqpoint{1.066599in}{3.047996in}}%
\pgfpathlineto{\pgfqpoint{1.067633in}{3.057912in}}%
\pgfpathlineto{\pgfqpoint{1.070057in}{3.108952in}}%
\pgfpathlineto{\pgfqpoint{1.072925in}{3.150056in}}%
\pgfpathlineto{\pgfqpoint{1.073356in}{3.149038in}}%
\pgfpathlineto{\pgfqpoint{1.074341in}{3.138307in}}%
\pgfpathlineto{\pgfqpoint{1.076482in}{3.084355in}}%
\pgfpathlineto{\pgfqpoint{1.079731in}{3.019751in}}%
\pgfpathlineto{\pgfqpoint{1.080100in}{3.020904in}}%
\pgfpathlineto{\pgfqpoint{1.081011in}{3.033802in}}%
\pgfpathlineto{\pgfqpoint{1.082870in}{3.099250in}}%
\pgfpathlineto{\pgfqpoint{1.086513in}{3.225575in}}%
\pgfpathlineto{\pgfqpoint{1.086747in}{3.224711in}}%
\pgfpathlineto{\pgfqpoint{1.087448in}{3.210653in}}%
\pgfpathlineto{\pgfqpoint{1.088790in}{3.129514in}}%
\pgfpathlineto{\pgfqpoint{1.090993in}{2.835127in}}%
\pgfpathlineto{\pgfqpoint{1.099584in}{1.530600in}}%
\pgfpathlineto{\pgfqpoint{1.100088in}{1.526425in}}%
\pgfpathlineto{\pgfqpoint{1.100654in}{1.531278in}}%
\pgfpathlineto{\pgfqpoint{1.101971in}{1.571636in}}%
\pgfpathlineto{\pgfqpoint{1.106870in}{1.732249in}}%
\pgfpathlineto{\pgfqpoint{1.106931in}{1.732218in}}%
\pgfpathlineto{\pgfqpoint{1.107387in}{1.730078in}}%
\pgfpathlineto{\pgfqpoint{1.108568in}{1.710876in}}%
\pgfpathlineto{\pgfqpoint{1.113663in}{1.601941in}}%
\pgfpathlineto{\pgfqpoint{1.114279in}{1.604257in}}%
\pgfpathlineto{\pgfqpoint{1.115571in}{1.621984in}}%
\pgfpathlineto{\pgfqpoint{1.120445in}{1.704990in}}%
\pgfpathlineto{\pgfqpoint{1.120937in}{1.703721in}}%
\pgfpathlineto{\pgfqpoint{1.122057in}{1.692289in}}%
\pgfpathlineto{\pgfqpoint{1.127226in}{1.612131in}}%
\pgfpathlineto{\pgfqpoint{1.128223in}{1.616951in}}%
\pgfpathlineto{\pgfqpoint{1.129934in}{1.643797in}}%
\pgfpathlineto{\pgfqpoint{1.134008in}{1.704990in}}%
\pgfpathlineto{\pgfqpoint{1.134057in}{1.704977in}}%
\pgfpathlineto{\pgfqpoint{1.134550in}{1.703477in}}%
\pgfpathlineto{\pgfqpoint{1.135706in}{1.690601in}}%
\pgfpathlineto{\pgfqpoint{1.140802in}{1.601941in}}%
\pgfpathlineto{\pgfqpoint{1.141848in}{1.608724in}}%
\pgfpathlineto{\pgfqpoint{1.143546in}{1.644768in}}%
\pgfpathlineto{\pgfqpoint{1.147583in}{1.732249in}}%
\pgfpathlineto{\pgfqpoint{1.147780in}{1.731903in}}%
\pgfpathlineto{\pgfqpoint{1.148469in}{1.725439in}}%
\pgfpathlineto{\pgfqpoint{1.149922in}{1.686007in}}%
\pgfpathlineto{\pgfqpoint{1.154365in}{1.526425in}}%
\pgfpathlineto{\pgfqpoint{1.155042in}{1.534212in}}%
\pgfpathlineto{\pgfqpoint{1.156199in}{1.587787in}}%
\pgfpathlineto{\pgfqpoint{1.158119in}{1.800135in}}%
\pgfpathlineto{\pgfqpoint{1.161860in}{2.526302in}}%
\pgfpathlineto{\pgfqpoint{1.166008in}{3.158016in}}%
\pgfpathlineto{\pgfqpoint{1.167928in}{3.225575in}}%
\pgfpathlineto{\pgfqpoint{1.168026in}{3.225432in}}%
\pgfpathlineto{\pgfqpoint{1.168531in}{3.220201in}}%
\pgfpathlineto{\pgfqpoint{1.169897in}{3.177004in}}%
\pgfpathlineto{\pgfqpoint{1.174697in}{3.019753in}}%
\pgfpathlineto{\pgfqpoint{1.175017in}{3.020498in}}%
\pgfpathlineto{\pgfqpoint{1.175066in}{3.020760in}}%
\pgfpathlineto{\pgfqpoint{1.176014in}{3.032705in}}%
\pgfpathlineto{\pgfqpoint{1.178549in}{3.101656in}}%
\pgfpathlineto{\pgfqpoint{1.181331in}{3.149885in}}%
\pgfpathlineto{\pgfqpoint{1.181540in}{3.150050in}}%
\pgfpathlineto{\pgfqpoint{1.181971in}{3.148735in}}%
\pgfpathlineto{\pgfqpoint{1.183066in}{3.136290in}}%
\pgfpathlineto{\pgfqpoint{1.188285in}{3.047010in}}%
\pgfpathlineto{\pgfqpoint{1.189257in}{3.051760in}}%
\pgfpathlineto{\pgfqpoint{1.190968in}{3.078882in}}%
\pgfpathlineto{\pgfqpoint{1.195054in}{3.139868in}}%
\pgfpathlineto{\pgfqpoint{1.195534in}{3.138814in}}%
\pgfpathlineto{\pgfqpoint{1.196605in}{3.128773in}}%
\pgfpathlineto{\pgfqpoint{1.199374in}{3.074804in}}%
\pgfpathlineto{\pgfqpoint{1.201860in}{3.047010in}}%
\pgfpathlineto{\pgfqpoint{1.201897in}{3.047020in}}%
\pgfpathlineto{\pgfqpoint{1.202389in}{3.048489in}}%
\pgfpathlineto{\pgfqpoint{1.203534in}{3.061101in}}%
\pgfpathlineto{\pgfqpoint{1.206918in}{3.133652in}}%
\pgfpathlineto{\pgfqpoint{1.208642in}{3.150059in}}%
\pgfpathlineto{\pgfqpoint{1.208925in}{3.149560in}}%
\pgfpathlineto{\pgfqpoint{1.209749in}{3.142552in}}%
\pgfpathlineto{\pgfqpoint{1.211509in}{3.103942in}}%
\pgfpathlineto{\pgfqpoint{1.215423in}{3.019751in}}%
\pgfpathlineto{\pgfqpoint{1.215571in}{3.019934in}}%
\pgfpathlineto{\pgfqpoint{1.216174in}{3.024561in}}%
\pgfpathlineto{\pgfqpoint{1.217491in}{3.056036in}}%
\pgfpathlineto{\pgfqpoint{1.222205in}{3.225575in}}%
\pgfpathlineto{\pgfqpoint{1.223128in}{3.211058in}}%
\pgfpathlineto{\pgfqpoint{1.224469in}{3.130603in}}%
\pgfpathlineto{\pgfqpoint{1.226660in}{2.839402in}}%
\pgfpathlineto{\pgfqpoint{1.235312in}{1.529997in}}%
\pgfpathlineto{\pgfqpoint{1.235780in}{1.526425in}}%
\pgfpathlineto{\pgfqpoint{1.236334in}{1.531077in}}%
\pgfpathlineto{\pgfqpoint{1.237638in}{1.570617in}}%
\pgfpathlineto{\pgfqpoint{1.242561in}{1.732249in}}%
\pgfpathlineto{\pgfqpoint{1.242648in}{1.732188in}}%
\pgfpathlineto{\pgfqpoint{1.243152in}{1.729426in}}%
\pgfpathlineto{\pgfqpoint{1.244432in}{1.706764in}}%
\pgfpathlineto{\pgfqpoint{1.249355in}{1.601941in}}%
\pgfpathlineto{\pgfqpoint{1.249749in}{1.602914in}}%
\pgfpathlineto{\pgfqpoint{1.250758in}{1.613264in}}%
\pgfpathlineto{\pgfqpoint{1.253441in}{1.671262in}}%
\pgfpathlineto{\pgfqpoint{1.256075in}{1.704974in}}%
\pgfpathlineto{\pgfqpoint{1.256321in}{1.704803in}}%
\pgfpathlineto{\pgfqpoint{1.256568in}{1.704013in}}%
\pgfpathlineto{\pgfqpoint{1.257614in}{1.694233in}}%
\pgfpathlineto{\pgfqpoint{1.260371in}{1.640472in}}%
\pgfpathlineto{\pgfqpoint{1.262918in}{1.612131in}}%
\pgfpathlineto{\pgfqpoint{1.263398in}{1.613271in}}%
\pgfpathlineto{\pgfqpoint{1.264494in}{1.623831in}}%
\pgfpathlineto{\pgfqpoint{1.267472in}{1.682033in}}%
\pgfpathlineto{\pgfqpoint{1.269700in}{1.704990in}}%
\pgfpathlineto{\pgfqpoint{1.269811in}{1.704927in}}%
\pgfpathlineto{\pgfqpoint{1.270389in}{1.702544in}}%
\pgfpathlineto{\pgfqpoint{1.271706in}{1.685253in}}%
\pgfpathlineto{\pgfqpoint{1.276494in}{1.601941in}}%
\pgfpathlineto{\pgfqpoint{1.277158in}{1.604714in}}%
\pgfpathlineto{\pgfqpoint{1.278438in}{1.624575in}}%
\pgfpathlineto{\pgfqpoint{1.283275in}{1.732249in}}%
\pgfpathlineto{\pgfqpoint{1.284112in}{1.726174in}}%
\pgfpathlineto{\pgfqpoint{1.285515in}{1.689662in}}%
\pgfpathlineto{\pgfqpoint{1.290057in}{1.526425in}}%
\pgfpathlineto{\pgfqpoint{1.290807in}{1.536050in}}%
\pgfpathlineto{\pgfqpoint{1.292026in}{1.597667in}}%
\pgfpathlineto{\pgfqpoint{1.294032in}{1.834139in}}%
\pgfpathlineto{\pgfqpoint{1.296826in}{2.373393in}}%
\pgfpathlineto{\pgfqpoint{1.296826in}{2.373393in}}%
\pgfusepath{stroke}%
\end{pgfscope}%
\begin{pgfscope}%
\pgfpathrectangle{\pgfqpoint{0.800000in}{0.528000in}}{\pgfqpoint{4.960000in}{3.696000in}}%
\pgfusepath{clip}%
\pgfsetrectcap%
\pgfsetroundjoin%
\pgfsetlinewidth{1.505625pt}%
\definecolor{currentstroke}{rgb}{0.172549,0.627451,0.172549}%
\pgfsetstrokecolor{currentstroke}%
\pgfsetdash{}{0pt}%
\pgfpathmoveto{\pgfqpoint{0.000000in}{0.000000in}}%
\pgfusepath{stroke}%
\end{pgfscope}%
\begin{pgfscope}%
\pgfpathrectangle{\pgfqpoint{0.800000in}{0.528000in}}{\pgfqpoint{4.960000in}{3.696000in}}%
\pgfusepath{clip}%
\pgfsetrectcap%
\pgfsetroundjoin%
\pgfsetlinewidth{1.505625pt}%
\definecolor{currentstroke}{rgb}{0.839216,0.152941,0.156863}%
\pgfsetstrokecolor{currentstroke}%
\pgfsetdash{}{0pt}%
\pgfpathmoveto{\pgfqpoint{1.025455in}{2.376000in}}%
\pgfpathlineto{\pgfqpoint{1.026131in}{4.056000in}}%
\pgfpathlineto{\pgfqpoint{1.026858in}{3.648003in}}%
\pgfpathlineto{\pgfqpoint{1.027473in}{3.865471in}}%
\pgfpathlineto{\pgfqpoint{1.028199in}{3.684518in}}%
\pgfpathlineto{\pgfqpoint{1.028581in}{3.767955in}}%
\pgfpathlineto{\pgfqpoint{1.028815in}{3.799651in}}%
\pgfpathlineto{\pgfqpoint{1.029368in}{3.691596in}}%
\pgfpathlineto{\pgfqpoint{1.029528in}{3.678771in}}%
\pgfpathlineto{\pgfqpoint{1.030082in}{3.752304in}}%
\pgfpathlineto{\pgfqpoint{1.030156in}{3.754790in}}%
\pgfpathlineto{\pgfqpoint{1.030501in}{3.710940in}}%
\pgfpathlineto{\pgfqpoint{1.030882in}{3.662054in}}%
\pgfpathlineto{\pgfqpoint{1.031498in}{3.717126in}}%
\pgfpathlineto{\pgfqpoint{1.031596in}{3.713919in}}%
\pgfpathlineto{\pgfqpoint{1.033602in}{3.616730in}}%
\pgfpathlineto{\pgfqpoint{1.033873in}{3.632020in}}%
\pgfpathlineto{\pgfqpoint{1.034193in}{3.650206in}}%
\pgfpathlineto{\pgfqpoint{1.034685in}{3.606901in}}%
\pgfpathlineto{\pgfqpoint{1.035535in}{3.618729in}}%
\pgfpathlineto{\pgfqpoint{1.036125in}{3.572035in}}%
\pgfpathlineto{\pgfqpoint{1.037664in}{3.538706in}}%
\pgfpathlineto{\pgfqpoint{1.036876in}{3.587947in}}%
\pgfpathlineto{\pgfqpoint{1.037713in}{3.539054in}}%
\pgfpathlineto{\pgfqpoint{1.038218in}{3.557642in}}%
\pgfpathlineto{\pgfqpoint{1.038599in}{3.536311in}}%
\pgfpathlineto{\pgfqpoint{1.040248in}{3.486752in}}%
\pgfpathlineto{\pgfqpoint{1.041738in}{3.456663in}}%
\pgfpathlineto{\pgfqpoint{1.040901in}{3.497967in}}%
\pgfpathlineto{\pgfqpoint{1.042008in}{3.463133in}}%
\pgfpathlineto{\pgfqpoint{1.042242in}{3.468446in}}%
\pgfpathlineto{\pgfqpoint{1.042673in}{3.448332in}}%
\pgfpathlineto{\pgfqpoint{1.044396in}{3.401340in}}%
\pgfpathlineto{\pgfqpoint{1.044458in}{3.401013in}}%
\pgfpathlineto{\pgfqpoint{1.044814in}{3.408511in}}%
\pgfpathlineto{\pgfqpoint{1.044938in}{3.409832in}}%
\pgfpathlineto{\pgfqpoint{1.045294in}{3.396478in}}%
\pgfpathlineto{\pgfqpoint{1.047165in}{3.344924in}}%
\pgfpathlineto{\pgfqpoint{1.047178in}{3.344937in}}%
\pgfpathlineto{\pgfqpoint{1.047448in}{3.349468in}}%
\pgfpathlineto{\pgfqpoint{1.047621in}{3.351601in}}%
\pgfpathlineto{\pgfqpoint{1.048002in}{3.338418in}}%
\pgfpathlineto{\pgfqpoint{1.049885in}{3.288554in}}%
\pgfpathlineto{\pgfqpoint{1.049910in}{3.288622in}}%
\pgfpathlineto{\pgfqpoint{1.050316in}{3.293629in}}%
\pgfpathlineto{\pgfqpoint{1.050599in}{3.286467in}}%
\pgfpathlineto{\pgfqpoint{1.052593in}{3.231978in}}%
\pgfpathlineto{\pgfqpoint{1.052778in}{3.233670in}}%
\pgfpathlineto{\pgfqpoint{1.052999in}{3.235847in}}%
\pgfpathlineto{\pgfqpoint{1.053356in}{3.225994in}}%
\pgfpathlineto{\pgfqpoint{1.055313in}{3.175261in}}%
\pgfpathlineto{\pgfqpoint{1.055424in}{3.175928in}}%
\pgfpathlineto{\pgfqpoint{1.055682in}{3.178198in}}%
\pgfpathlineto{\pgfqpoint{1.056014in}{3.170487in}}%
\pgfpathlineto{\pgfqpoint{1.058021in}{3.118432in}}%
\pgfpathlineto{\pgfqpoint{1.058218in}{3.119755in}}%
\pgfpathlineto{\pgfqpoint{1.058365in}{3.120648in}}%
\pgfpathlineto{\pgfqpoint{1.058685in}{3.114229in}}%
\pgfpathlineto{\pgfqpoint{1.060741in}{3.061525in}}%
\pgfpathlineto{\pgfqpoint{1.061024in}{3.063131in}}%
\pgfpathlineto{\pgfqpoint{1.061061in}{3.063178in}}%
\pgfpathlineto{\pgfqpoint{1.061245in}{3.061168in}}%
\pgfpathlineto{\pgfqpoint{1.073270in}{2.804655in}}%
\pgfpathlineto{\pgfqpoint{1.073750in}{2.789510in}}%
\pgfpathlineto{\pgfqpoint{1.075657in}{2.747672in}}%
\pgfpathlineto{\pgfqpoint{1.075793in}{2.747858in}}%
\pgfpathlineto{\pgfqpoint{1.075854in}{2.747882in}}%
\pgfpathlineto{\pgfqpoint{1.076039in}{2.746457in}}%
\pgfpathlineto{\pgfqpoint{1.079140in}{2.675570in}}%
\pgfpathlineto{\pgfqpoint{1.081085in}{2.633353in}}%
\pgfpathlineto{\pgfqpoint{1.081220in}{2.633421in}}%
\pgfpathlineto{\pgfqpoint{1.081417in}{2.632193in}}%
\pgfpathlineto{\pgfqpoint{1.083202in}{2.589308in}}%
\pgfpathlineto{\pgfqpoint{1.085282in}{2.547580in}}%
\pgfpathlineto{\pgfqpoint{1.085516in}{2.545498in}}%
\pgfpathlineto{\pgfqpoint{1.094230in}{2.352103in}}%
\pgfpathlineto{\pgfqpoint{1.097504in}{2.289953in}}%
\pgfpathlineto{\pgfqpoint{1.097873in}{2.282143in}}%
\pgfpathlineto{\pgfqpoint{1.100322in}{2.231661in}}%
\pgfpathlineto{\pgfqpoint{1.102143in}{2.187256in}}%
\pgfpathlineto{\pgfqpoint{1.104211in}{2.147184in}}%
\pgfpathlineto{\pgfqpoint{1.104482in}{2.143767in}}%
\pgfpathlineto{\pgfqpoint{1.110771in}{2.004110in}}%
\pgfpathlineto{\pgfqpoint{1.110931in}{2.004326in}}%
\pgfpathlineto{\pgfqpoint{1.111054in}{2.003948in}}%
\pgfpathlineto{\pgfqpoint{1.111116in}{2.003310in}}%
\pgfpathlineto{\pgfqpoint{1.112814in}{1.964121in}}%
\pgfpathlineto{\pgfqpoint{1.114796in}{1.918214in}}%
\pgfpathlineto{\pgfqpoint{1.114956in}{1.918566in}}%
\pgfpathlineto{\pgfqpoint{1.115017in}{1.918635in}}%
\pgfpathlineto{\pgfqpoint{1.115227in}{1.916757in}}%
\pgfpathlineto{\pgfqpoint{1.127965in}{1.635895in}}%
\pgfpathlineto{\pgfqpoint{1.129577in}{1.602588in}}%
\pgfpathlineto{\pgfqpoint{1.129971in}{1.605076in}}%
\pgfpathlineto{\pgfqpoint{1.130254in}{1.598281in}}%
\pgfpathlineto{\pgfqpoint{1.132260in}{1.544991in}}%
\pgfpathlineto{\pgfqpoint{1.132519in}{1.547470in}}%
\pgfpathlineto{\pgfqpoint{1.132654in}{1.548365in}}%
\pgfpathlineto{\pgfqpoint{1.132974in}{1.540573in}}%
\pgfpathlineto{\pgfqpoint{1.134943in}{1.487281in}}%
\pgfpathlineto{\pgfqpoint{1.135153in}{1.489426in}}%
\pgfpathlineto{\pgfqpoint{1.135362in}{1.491714in}}%
\pgfpathlineto{\pgfqpoint{1.135731in}{1.480842in}}%
\pgfpathlineto{\pgfqpoint{1.137639in}{1.429411in}}%
\pgfpathlineto{\pgfqpoint{1.137737in}{1.430165in}}%
\pgfpathlineto{\pgfqpoint{1.138082in}{1.435232in}}%
\pgfpathlineto{\pgfqpoint{1.138426in}{1.424307in}}%
\pgfpathlineto{\pgfqpoint{1.140322in}{1.371322in}}%
\pgfpathlineto{\pgfqpoint{1.140420in}{1.372145in}}%
\pgfpathlineto{\pgfqpoint{1.140790in}{1.378990in}}%
\pgfpathlineto{\pgfqpoint{1.141146in}{1.366350in}}%
\pgfpathlineto{\pgfqpoint{1.143005in}{1.312920in}}%
\pgfpathlineto{\pgfqpoint{1.143066in}{1.313264in}}%
\pgfpathlineto{\pgfqpoint{1.143509in}{1.323097in}}%
\pgfpathlineto{\pgfqpoint{1.143854in}{1.309162in}}%
\pgfpathlineto{\pgfqpoint{1.145700in}{1.254033in}}%
\pgfpathlineto{\pgfqpoint{1.145713in}{1.254066in}}%
\pgfpathlineto{\pgfqpoint{1.145983in}{1.261861in}}%
\pgfpathlineto{\pgfqpoint{1.146217in}{1.267744in}}%
\pgfpathlineto{\pgfqpoint{1.146648in}{1.244769in}}%
\pgfpathlineto{\pgfqpoint{1.148346in}{1.194582in}}%
\pgfpathlineto{\pgfqpoint{1.148383in}{1.194358in}}%
\pgfpathlineto{\pgfqpoint{1.148642in}{1.202867in}}%
\pgfpathlineto{\pgfqpoint{1.148937in}{1.213294in}}%
\pgfpathlineto{\pgfqpoint{1.149380in}{1.184083in}}%
\pgfpathlineto{\pgfqpoint{1.150291in}{1.186620in}}%
\pgfpathlineto{\pgfqpoint{1.150993in}{1.134542in}}%
\pgfpathlineto{\pgfqpoint{1.151066in}{1.133271in}}%
\pgfpathlineto{\pgfqpoint{1.151460in}{1.153842in}}%
\pgfpathlineto{\pgfqpoint{1.151645in}{1.160517in}}%
\pgfpathlineto{\pgfqpoint{1.152088in}{1.124714in}}%
\pgfpathlineto{\pgfqpoint{1.152408in}{1.101794in}}%
\pgfpathlineto{\pgfqpoint{1.152999in}{1.135270in}}%
\pgfpathlineto{\pgfqpoint{1.153159in}{1.129103in}}%
\pgfpathlineto{\pgfqpoint{1.155103in}{1.034874in}}%
\pgfpathlineto{\pgfqpoint{1.155276in}{1.044844in}}%
\pgfpathlineto{\pgfqpoint{1.155719in}{1.089946in}}%
\pgfpathlineto{\pgfqpoint{1.156186in}{1.024161in}}%
\pgfpathlineto{\pgfqpoint{1.156445in}{0.997210in}}%
\pgfpathlineto{\pgfqpoint{1.157023in}{1.071844in}}%
\pgfpathlineto{\pgfqpoint{1.157073in}{1.073229in}}%
\pgfpathlineto{\pgfqpoint{1.157343in}{1.037763in}}%
\pgfpathlineto{\pgfqpoint{1.157786in}{0.952349in}}%
\pgfpathlineto{\pgfqpoint{1.158340in}{1.061873in}}%
\pgfpathlineto{\pgfqpoint{1.158426in}{1.068181in}}%
\pgfpathlineto{\pgfqpoint{1.158783in}{0.981804in}}%
\pgfpathlineto{\pgfqpoint{1.159128in}{0.886529in}}%
\pgfpathlineto{\pgfqpoint{1.159645in}{1.078072in}}%
\pgfpathlineto{\pgfqpoint{1.159793in}{1.107213in}}%
\pgfpathlineto{\pgfqpoint{1.160186in}{0.876497in}}%
\pgfpathlineto{\pgfqpoint{1.160469in}{0.696000in}}%
\pgfpathlineto{\pgfqpoint{1.160851in}{1.249397in}}%
\pgfpathlineto{\pgfqpoint{1.161823in}{4.056000in}}%
\pgfpathlineto{\pgfqpoint{1.162771in}{3.730654in}}%
\pgfpathlineto{\pgfqpoint{1.163165in}{3.865471in}}%
\pgfpathlineto{\pgfqpoint{1.163792in}{3.687416in}}%
\pgfpathlineto{\pgfqpoint{1.163866in}{3.683819in}}%
\pgfpathlineto{\pgfqpoint{1.164211in}{3.751759in}}%
\pgfpathlineto{\pgfqpoint{1.164506in}{3.799651in}}%
\pgfpathlineto{\pgfqpoint{1.165085in}{3.687935in}}%
\pgfpathlineto{\pgfqpoint{1.165220in}{3.678771in}}%
\pgfpathlineto{\pgfqpoint{1.165725in}{3.748065in}}%
\pgfpathlineto{\pgfqpoint{1.165848in}{3.754790in}}%
\pgfpathlineto{\pgfqpoint{1.166279in}{3.693807in}}%
\pgfpathlineto{\pgfqpoint{1.166574in}{3.662054in}}%
\pgfpathlineto{\pgfqpoint{1.167189in}{3.717126in}}%
\pgfpathlineto{\pgfqpoint{1.167276in}{3.714684in}}%
\pgfpathlineto{\pgfqpoint{1.169294in}{3.616730in}}%
\pgfpathlineto{\pgfqpoint{1.169626in}{3.637392in}}%
\pgfpathlineto{\pgfqpoint{1.169885in}{3.650206in}}%
\pgfpathlineto{\pgfqpoint{1.170377in}{3.606901in}}%
\pgfpathlineto{\pgfqpoint{1.171226in}{3.618729in}}%
\pgfpathlineto{\pgfqpoint{1.171817in}{3.572035in}}%
\pgfpathlineto{\pgfqpoint{1.173356in}{3.538706in}}%
\pgfpathlineto{\pgfqpoint{1.172568in}{3.587947in}}%
\pgfpathlineto{\pgfqpoint{1.173405in}{3.539054in}}%
\pgfpathlineto{\pgfqpoint{1.173909in}{3.557642in}}%
\pgfpathlineto{\pgfqpoint{1.174291in}{3.536311in}}%
\pgfpathlineto{\pgfqpoint{1.175940in}{3.486752in}}%
\pgfpathlineto{\pgfqpoint{1.177429in}{3.456663in}}%
\pgfpathlineto{\pgfqpoint{1.176592in}{3.497967in}}%
\pgfpathlineto{\pgfqpoint{1.177700in}{3.463133in}}%
\pgfpathlineto{\pgfqpoint{1.177934in}{3.468446in}}%
\pgfpathlineto{\pgfqpoint{1.178365in}{3.448332in}}%
\pgfpathlineto{\pgfqpoint{1.180088in}{3.401340in}}%
\pgfpathlineto{\pgfqpoint{1.180149in}{3.401013in}}%
\pgfpathlineto{\pgfqpoint{1.180506in}{3.408511in}}%
\pgfpathlineto{\pgfqpoint{1.180629in}{3.409832in}}%
\pgfpathlineto{\pgfqpoint{1.180986in}{3.396478in}}%
\pgfpathlineto{\pgfqpoint{1.182857in}{3.344924in}}%
\pgfpathlineto{\pgfqpoint{1.182869in}{3.344937in}}%
\pgfpathlineto{\pgfqpoint{1.183140in}{3.349468in}}%
\pgfpathlineto{\pgfqpoint{1.183312in}{3.351601in}}%
\pgfpathlineto{\pgfqpoint{1.183694in}{3.338418in}}%
\pgfpathlineto{\pgfqpoint{1.185577in}{3.288554in}}%
\pgfpathlineto{\pgfqpoint{1.185602in}{3.288622in}}%
\pgfpathlineto{\pgfqpoint{1.186008in}{3.293629in}}%
\pgfpathlineto{\pgfqpoint{1.186291in}{3.286467in}}%
\pgfpathlineto{\pgfqpoint{1.188285in}{3.231978in}}%
\pgfpathlineto{\pgfqpoint{1.188469in}{3.233670in}}%
\pgfpathlineto{\pgfqpoint{1.188691in}{3.235847in}}%
\pgfpathlineto{\pgfqpoint{1.189048in}{3.225994in}}%
\pgfpathlineto{\pgfqpoint{1.191005in}{3.175261in}}%
\pgfpathlineto{\pgfqpoint{1.191115in}{3.175928in}}%
\pgfpathlineto{\pgfqpoint{1.191374in}{3.178198in}}%
\pgfpathlineto{\pgfqpoint{1.191706in}{3.170487in}}%
\pgfpathlineto{\pgfqpoint{1.193712in}{3.118432in}}%
\pgfpathlineto{\pgfqpoint{1.193909in}{3.119755in}}%
\pgfpathlineto{\pgfqpoint{1.194057in}{3.120648in}}%
\pgfpathlineto{\pgfqpoint{1.194377in}{3.114229in}}%
\pgfpathlineto{\pgfqpoint{1.196432in}{3.061525in}}%
\pgfpathlineto{\pgfqpoint{1.196715in}{3.063131in}}%
\pgfpathlineto{\pgfqpoint{1.196752in}{3.063178in}}%
\pgfpathlineto{\pgfqpoint{1.196937in}{3.061168in}}%
\pgfpathlineto{\pgfqpoint{1.208962in}{2.804655in}}%
\pgfpathlineto{\pgfqpoint{1.209442in}{2.789510in}}%
\pgfpathlineto{\pgfqpoint{1.211349in}{2.747672in}}%
\pgfpathlineto{\pgfqpoint{1.211485in}{2.747858in}}%
\pgfpathlineto{\pgfqpoint{1.211546in}{2.747882in}}%
\pgfpathlineto{\pgfqpoint{1.211731in}{2.746457in}}%
\pgfpathlineto{\pgfqpoint{1.214832in}{2.675570in}}%
\pgfpathlineto{\pgfqpoint{1.216777in}{2.633353in}}%
\pgfpathlineto{\pgfqpoint{1.216912in}{2.633421in}}%
\pgfpathlineto{\pgfqpoint{1.217109in}{2.632193in}}%
\pgfpathlineto{\pgfqpoint{1.218894in}{2.589308in}}%
\pgfpathlineto{\pgfqpoint{1.220974in}{2.547580in}}%
\pgfpathlineto{\pgfqpoint{1.221208in}{2.545498in}}%
\pgfpathlineto{\pgfqpoint{1.229921in}{2.352103in}}%
\pgfpathlineto{\pgfqpoint{1.233195in}{2.289953in}}%
\pgfpathlineto{\pgfqpoint{1.233565in}{2.282143in}}%
\pgfpathlineto{\pgfqpoint{1.236014in}{2.231661in}}%
\pgfpathlineto{\pgfqpoint{1.237835in}{2.187256in}}%
\pgfpathlineto{\pgfqpoint{1.239903in}{2.147184in}}%
\pgfpathlineto{\pgfqpoint{1.240174in}{2.143767in}}%
\pgfpathlineto{\pgfqpoint{1.246463in}{2.004110in}}%
\pgfpathlineto{\pgfqpoint{1.246623in}{2.004326in}}%
\pgfpathlineto{\pgfqpoint{1.246746in}{2.003948in}}%
\pgfpathlineto{\pgfqpoint{1.246808in}{2.003310in}}%
\pgfpathlineto{\pgfqpoint{1.248506in}{1.964121in}}%
\pgfpathlineto{\pgfqpoint{1.250488in}{1.918214in}}%
\pgfpathlineto{\pgfqpoint{1.250648in}{1.918566in}}%
\pgfpathlineto{\pgfqpoint{1.250709in}{1.918635in}}%
\pgfpathlineto{\pgfqpoint{1.250918in}{1.916757in}}%
\pgfpathlineto{\pgfqpoint{1.263657in}{1.635895in}}%
\pgfpathlineto{\pgfqpoint{1.265269in}{1.602588in}}%
\pgfpathlineto{\pgfqpoint{1.265663in}{1.605076in}}%
\pgfpathlineto{\pgfqpoint{1.265946in}{1.598281in}}%
\pgfpathlineto{\pgfqpoint{1.267952in}{1.544991in}}%
\pgfpathlineto{\pgfqpoint{1.268211in}{1.547470in}}%
\pgfpathlineto{\pgfqpoint{1.268346in}{1.548365in}}%
\pgfpathlineto{\pgfqpoint{1.268666in}{1.540573in}}%
\pgfpathlineto{\pgfqpoint{1.270635in}{1.487281in}}%
\pgfpathlineto{\pgfqpoint{1.270844in}{1.489426in}}%
\pgfpathlineto{\pgfqpoint{1.271054in}{1.491714in}}%
\pgfpathlineto{\pgfqpoint{1.271423in}{1.480842in}}%
\pgfpathlineto{\pgfqpoint{1.273331in}{1.429411in}}%
\pgfpathlineto{\pgfqpoint{1.273429in}{1.430165in}}%
\pgfpathlineto{\pgfqpoint{1.273774in}{1.435232in}}%
\pgfpathlineto{\pgfqpoint{1.274118in}{1.424307in}}%
\pgfpathlineto{\pgfqpoint{1.276014in}{1.371322in}}%
\pgfpathlineto{\pgfqpoint{1.276112in}{1.372145in}}%
\pgfpathlineto{\pgfqpoint{1.276481in}{1.378990in}}%
\pgfpathlineto{\pgfqpoint{1.276838in}{1.366350in}}%
\pgfpathlineto{\pgfqpoint{1.278697in}{1.312920in}}%
\pgfpathlineto{\pgfqpoint{1.278758in}{1.313264in}}%
\pgfpathlineto{\pgfqpoint{1.279201in}{1.323097in}}%
\pgfpathlineto{\pgfqpoint{1.279546in}{1.309162in}}%
\pgfpathlineto{\pgfqpoint{1.281392in}{1.254033in}}%
\pgfpathlineto{\pgfqpoint{1.281404in}{1.254066in}}%
\pgfpathlineto{\pgfqpoint{1.281675in}{1.261861in}}%
\pgfpathlineto{\pgfqpoint{1.281909in}{1.267744in}}%
\pgfpathlineto{\pgfqpoint{1.282340in}{1.244769in}}%
\pgfpathlineto{\pgfqpoint{1.284038in}{1.194582in}}%
\pgfpathlineto{\pgfqpoint{1.284075in}{1.194358in}}%
\pgfpathlineto{\pgfqpoint{1.284334in}{1.202867in}}%
\pgfpathlineto{\pgfqpoint{1.284629in}{1.213294in}}%
\pgfpathlineto{\pgfqpoint{1.285072in}{1.184083in}}%
\pgfpathlineto{\pgfqpoint{1.285983in}{1.186620in}}%
\pgfpathlineto{\pgfqpoint{1.286684in}{1.134542in}}%
\pgfpathlineto{\pgfqpoint{1.286758in}{1.133271in}}%
\pgfpathlineto{\pgfqpoint{1.287152in}{1.153842in}}%
\pgfpathlineto{\pgfqpoint{1.287337in}{1.160517in}}%
\pgfpathlineto{\pgfqpoint{1.287780in}{1.124714in}}%
\pgfpathlineto{\pgfqpoint{1.288100in}{1.101794in}}%
\pgfpathlineto{\pgfqpoint{1.288690in}{1.135270in}}%
\pgfpathlineto{\pgfqpoint{1.288850in}{1.129103in}}%
\pgfpathlineto{\pgfqpoint{1.290795in}{1.034874in}}%
\pgfpathlineto{\pgfqpoint{1.290967in}{1.044844in}}%
\pgfpathlineto{\pgfqpoint{1.291410in}{1.089946in}}%
\pgfpathlineto{\pgfqpoint{1.291878in}{1.024161in}}%
\pgfpathlineto{\pgfqpoint{1.292137in}{0.997210in}}%
\pgfpathlineto{\pgfqpoint{1.292715in}{1.071844in}}%
\pgfpathlineto{\pgfqpoint{1.292764in}{1.073229in}}%
\pgfpathlineto{\pgfqpoint{1.293035in}{1.037763in}}%
\pgfpathlineto{\pgfqpoint{1.293478in}{0.952349in}}%
\pgfpathlineto{\pgfqpoint{1.294032in}{1.061873in}}%
\pgfpathlineto{\pgfqpoint{1.294118in}{1.068181in}}%
\pgfpathlineto{\pgfqpoint{1.294475in}{0.981804in}}%
\pgfpathlineto{\pgfqpoint{1.294820in}{0.886529in}}%
\pgfpathlineto{\pgfqpoint{1.295337in}{1.078072in}}%
\pgfpathlineto{\pgfqpoint{1.295484in}{1.107213in}}%
\pgfpathlineto{\pgfqpoint{1.295878in}{0.876497in}}%
\pgfpathlineto{\pgfqpoint{1.296161in}{0.696000in}}%
\pgfpathlineto{\pgfqpoint{1.296543in}{1.249397in}}%
\pgfpathlineto{\pgfqpoint{1.296826in}{2.323869in}}%
\pgfpathlineto{\pgfqpoint{1.296826in}{2.323869in}}%
\pgfusepath{stroke}%
\end{pgfscope}%
\begin{pgfscope}%
\pgfpathrectangle{\pgfqpoint{0.800000in}{0.528000in}}{\pgfqpoint{4.960000in}{3.696000in}}%
\pgfusepath{clip}%
\pgfsetrectcap%
\pgfsetroundjoin%
\pgfsetlinewidth{1.505625pt}%
\definecolor{currentstroke}{rgb}{0.580392,0.403922,0.741176}%
\pgfsetstrokecolor{currentstroke}%
\pgfsetdash{}{0pt}%
\pgfpathmoveto{\pgfqpoint{1.025455in}{2.376000in}}%
\pgfpathlineto{\pgfqpoint{1.030218in}{3.150546in}}%
\pgfpathlineto{\pgfqpoint{1.032236in}{3.225575in}}%
\pgfpathlineto{\pgfqpoint{1.032285in}{3.225541in}}%
\pgfpathlineto{\pgfqpoint{1.032667in}{3.222793in}}%
\pgfpathlineto{\pgfqpoint{1.033762in}{3.194594in}}%
\pgfpathlineto{\pgfqpoint{1.039030in}{3.019751in}}%
\pgfpathlineto{\pgfqpoint{1.039571in}{3.022184in}}%
\pgfpathlineto{\pgfqpoint{1.040790in}{3.042708in}}%
\pgfpathlineto{\pgfqpoint{1.045811in}{3.150059in}}%
\pgfpathlineto{\pgfqpoint{1.046341in}{3.148370in}}%
\pgfpathlineto{\pgfqpoint{1.047522in}{3.133761in}}%
\pgfpathlineto{\pgfqpoint{1.052593in}{3.047010in}}%
\pgfpathlineto{\pgfqpoint{1.053356in}{3.049960in}}%
\pgfpathlineto{\pgfqpoint{1.054796in}{3.069511in}}%
\pgfpathlineto{\pgfqpoint{1.059387in}{3.139869in}}%
\pgfpathlineto{\pgfqpoint{1.059682in}{3.139415in}}%
\pgfpathlineto{\pgfqpoint{1.060556in}{3.133221in}}%
\pgfpathlineto{\pgfqpoint{1.062513in}{3.099353in}}%
\pgfpathlineto{\pgfqpoint{1.066082in}{3.047043in}}%
\pgfpathlineto{\pgfqpoint{1.066291in}{3.047096in}}%
\pgfpathlineto{\pgfqpoint{1.066599in}{3.047996in}}%
\pgfpathlineto{\pgfqpoint{1.067633in}{3.057912in}}%
\pgfpathlineto{\pgfqpoint{1.070057in}{3.108952in}}%
\pgfpathlineto{\pgfqpoint{1.072925in}{3.150056in}}%
\pgfpathlineto{\pgfqpoint{1.073356in}{3.149038in}}%
\pgfpathlineto{\pgfqpoint{1.074341in}{3.138307in}}%
\pgfpathlineto{\pgfqpoint{1.076482in}{3.084355in}}%
\pgfpathlineto{\pgfqpoint{1.079731in}{3.019751in}}%
\pgfpathlineto{\pgfqpoint{1.080100in}{3.020904in}}%
\pgfpathlineto{\pgfqpoint{1.081011in}{3.033802in}}%
\pgfpathlineto{\pgfqpoint{1.082870in}{3.099250in}}%
\pgfpathlineto{\pgfqpoint{1.086513in}{3.225575in}}%
\pgfpathlineto{\pgfqpoint{1.086747in}{3.224711in}}%
\pgfpathlineto{\pgfqpoint{1.087448in}{3.210653in}}%
\pgfpathlineto{\pgfqpoint{1.088790in}{3.129514in}}%
\pgfpathlineto{\pgfqpoint{1.090993in}{2.835127in}}%
\pgfpathlineto{\pgfqpoint{1.099584in}{1.530600in}}%
\pgfpathlineto{\pgfqpoint{1.100088in}{1.526425in}}%
\pgfpathlineto{\pgfqpoint{1.100654in}{1.531278in}}%
\pgfpathlineto{\pgfqpoint{1.101971in}{1.571636in}}%
\pgfpathlineto{\pgfqpoint{1.106870in}{1.732249in}}%
\pgfpathlineto{\pgfqpoint{1.106931in}{1.732218in}}%
\pgfpathlineto{\pgfqpoint{1.107387in}{1.730078in}}%
\pgfpathlineto{\pgfqpoint{1.108568in}{1.710876in}}%
\pgfpathlineto{\pgfqpoint{1.113663in}{1.601941in}}%
\pgfpathlineto{\pgfqpoint{1.114279in}{1.604257in}}%
\pgfpathlineto{\pgfqpoint{1.115571in}{1.621984in}}%
\pgfpathlineto{\pgfqpoint{1.120445in}{1.704990in}}%
\pgfpathlineto{\pgfqpoint{1.120937in}{1.703721in}}%
\pgfpathlineto{\pgfqpoint{1.122057in}{1.692289in}}%
\pgfpathlineto{\pgfqpoint{1.127226in}{1.612131in}}%
\pgfpathlineto{\pgfqpoint{1.128223in}{1.616951in}}%
\pgfpathlineto{\pgfqpoint{1.129934in}{1.643797in}}%
\pgfpathlineto{\pgfqpoint{1.134008in}{1.704990in}}%
\pgfpathlineto{\pgfqpoint{1.134057in}{1.704977in}}%
\pgfpathlineto{\pgfqpoint{1.134550in}{1.703477in}}%
\pgfpathlineto{\pgfqpoint{1.135706in}{1.690601in}}%
\pgfpathlineto{\pgfqpoint{1.140802in}{1.601941in}}%
\pgfpathlineto{\pgfqpoint{1.141848in}{1.608724in}}%
\pgfpathlineto{\pgfqpoint{1.143546in}{1.644768in}}%
\pgfpathlineto{\pgfqpoint{1.147583in}{1.732249in}}%
\pgfpathlineto{\pgfqpoint{1.147780in}{1.731903in}}%
\pgfpathlineto{\pgfqpoint{1.148469in}{1.725439in}}%
\pgfpathlineto{\pgfqpoint{1.149922in}{1.686007in}}%
\pgfpathlineto{\pgfqpoint{1.154365in}{1.526425in}}%
\pgfpathlineto{\pgfqpoint{1.155042in}{1.534212in}}%
\pgfpathlineto{\pgfqpoint{1.156199in}{1.587787in}}%
\pgfpathlineto{\pgfqpoint{1.158119in}{1.800135in}}%
\pgfpathlineto{\pgfqpoint{1.161860in}{2.526302in}}%
\pgfpathlineto{\pgfqpoint{1.166008in}{3.158016in}}%
\pgfpathlineto{\pgfqpoint{1.167928in}{3.225575in}}%
\pgfpathlineto{\pgfqpoint{1.168026in}{3.225432in}}%
\pgfpathlineto{\pgfqpoint{1.168531in}{3.220201in}}%
\pgfpathlineto{\pgfqpoint{1.169897in}{3.177004in}}%
\pgfpathlineto{\pgfqpoint{1.174697in}{3.019753in}}%
\pgfpathlineto{\pgfqpoint{1.175017in}{3.020498in}}%
\pgfpathlineto{\pgfqpoint{1.175066in}{3.020760in}}%
\pgfpathlineto{\pgfqpoint{1.176014in}{3.032705in}}%
\pgfpathlineto{\pgfqpoint{1.178549in}{3.101656in}}%
\pgfpathlineto{\pgfqpoint{1.181331in}{3.149885in}}%
\pgfpathlineto{\pgfqpoint{1.181540in}{3.150050in}}%
\pgfpathlineto{\pgfqpoint{1.181971in}{3.148735in}}%
\pgfpathlineto{\pgfqpoint{1.183066in}{3.136290in}}%
\pgfpathlineto{\pgfqpoint{1.188285in}{3.047010in}}%
\pgfpathlineto{\pgfqpoint{1.189257in}{3.051760in}}%
\pgfpathlineto{\pgfqpoint{1.190968in}{3.078882in}}%
\pgfpathlineto{\pgfqpoint{1.195054in}{3.139868in}}%
\pgfpathlineto{\pgfqpoint{1.195534in}{3.138814in}}%
\pgfpathlineto{\pgfqpoint{1.196605in}{3.128773in}}%
\pgfpathlineto{\pgfqpoint{1.199374in}{3.074804in}}%
\pgfpathlineto{\pgfqpoint{1.201860in}{3.047010in}}%
\pgfpathlineto{\pgfqpoint{1.201897in}{3.047020in}}%
\pgfpathlineto{\pgfqpoint{1.202389in}{3.048489in}}%
\pgfpathlineto{\pgfqpoint{1.203534in}{3.061101in}}%
\pgfpathlineto{\pgfqpoint{1.206918in}{3.133652in}}%
\pgfpathlineto{\pgfqpoint{1.208642in}{3.150059in}}%
\pgfpathlineto{\pgfqpoint{1.208925in}{3.149560in}}%
\pgfpathlineto{\pgfqpoint{1.209749in}{3.142552in}}%
\pgfpathlineto{\pgfqpoint{1.211509in}{3.103942in}}%
\pgfpathlineto{\pgfqpoint{1.215423in}{3.019751in}}%
\pgfpathlineto{\pgfqpoint{1.215571in}{3.019934in}}%
\pgfpathlineto{\pgfqpoint{1.216174in}{3.024561in}}%
\pgfpathlineto{\pgfqpoint{1.217491in}{3.056036in}}%
\pgfpathlineto{\pgfqpoint{1.222205in}{3.225575in}}%
\pgfpathlineto{\pgfqpoint{1.223128in}{3.211058in}}%
\pgfpathlineto{\pgfqpoint{1.224469in}{3.130603in}}%
\pgfpathlineto{\pgfqpoint{1.226660in}{2.839402in}}%
\pgfpathlineto{\pgfqpoint{1.235312in}{1.529997in}}%
\pgfpathlineto{\pgfqpoint{1.235780in}{1.526425in}}%
\pgfpathlineto{\pgfqpoint{1.236334in}{1.531077in}}%
\pgfpathlineto{\pgfqpoint{1.237638in}{1.570617in}}%
\pgfpathlineto{\pgfqpoint{1.242561in}{1.732249in}}%
\pgfpathlineto{\pgfqpoint{1.242648in}{1.732188in}}%
\pgfpathlineto{\pgfqpoint{1.243152in}{1.729426in}}%
\pgfpathlineto{\pgfqpoint{1.244432in}{1.706764in}}%
\pgfpathlineto{\pgfqpoint{1.249355in}{1.601941in}}%
\pgfpathlineto{\pgfqpoint{1.249749in}{1.602914in}}%
\pgfpathlineto{\pgfqpoint{1.250758in}{1.613264in}}%
\pgfpathlineto{\pgfqpoint{1.253441in}{1.671262in}}%
\pgfpathlineto{\pgfqpoint{1.256075in}{1.704974in}}%
\pgfpathlineto{\pgfqpoint{1.256321in}{1.704803in}}%
\pgfpathlineto{\pgfqpoint{1.256568in}{1.704013in}}%
\pgfpathlineto{\pgfqpoint{1.257614in}{1.694233in}}%
\pgfpathlineto{\pgfqpoint{1.260371in}{1.640472in}}%
\pgfpathlineto{\pgfqpoint{1.262918in}{1.612131in}}%
\pgfpathlineto{\pgfqpoint{1.263398in}{1.613271in}}%
\pgfpathlineto{\pgfqpoint{1.264494in}{1.623831in}}%
\pgfpathlineto{\pgfqpoint{1.267472in}{1.682033in}}%
\pgfpathlineto{\pgfqpoint{1.269700in}{1.704990in}}%
\pgfpathlineto{\pgfqpoint{1.269811in}{1.704927in}}%
\pgfpathlineto{\pgfqpoint{1.270389in}{1.702544in}}%
\pgfpathlineto{\pgfqpoint{1.271706in}{1.685253in}}%
\pgfpathlineto{\pgfqpoint{1.276494in}{1.601941in}}%
\pgfpathlineto{\pgfqpoint{1.277158in}{1.604714in}}%
\pgfpathlineto{\pgfqpoint{1.278438in}{1.624575in}}%
\pgfpathlineto{\pgfqpoint{1.283275in}{1.732249in}}%
\pgfpathlineto{\pgfqpoint{1.284112in}{1.726174in}}%
\pgfpathlineto{\pgfqpoint{1.285515in}{1.689662in}}%
\pgfpathlineto{\pgfqpoint{1.290057in}{1.526425in}}%
\pgfpathlineto{\pgfqpoint{1.290807in}{1.536050in}}%
\pgfpathlineto{\pgfqpoint{1.292026in}{1.597667in}}%
\pgfpathlineto{\pgfqpoint{1.294032in}{1.834139in}}%
\pgfpathlineto{\pgfqpoint{1.296826in}{2.373393in}}%
\pgfpathlineto{\pgfqpoint{1.296826in}{2.373393in}}%
\pgfusepath{stroke}%
\end{pgfscope}%
\begin{pgfscope}%
\pgfpathrectangle{\pgfqpoint{0.800000in}{0.528000in}}{\pgfqpoint{4.960000in}{3.696000in}}%
\pgfusepath{clip}%
\pgfsetrectcap%
\pgfsetroundjoin%
\pgfsetlinewidth{1.505625pt}%
\definecolor{currentstroke}{rgb}{0.549020,0.337255,0.294118}%
\pgfsetstrokecolor{currentstroke}%
\pgfsetdash{}{0pt}%
\pgfpathmoveto{\pgfqpoint{1.025455in}{2.376000in}}%
\pgfpathlineto{\pgfqpoint{1.026131in}{4.056000in}}%
\pgfpathlineto{\pgfqpoint{1.026858in}{3.648003in}}%
\pgfpathlineto{\pgfqpoint{1.027473in}{3.865471in}}%
\pgfpathlineto{\pgfqpoint{1.028199in}{3.684518in}}%
\pgfpathlineto{\pgfqpoint{1.028581in}{3.767955in}}%
\pgfpathlineto{\pgfqpoint{1.028815in}{3.799651in}}%
\pgfpathlineto{\pgfqpoint{1.029368in}{3.691596in}}%
\pgfpathlineto{\pgfqpoint{1.029528in}{3.678771in}}%
\pgfpathlineto{\pgfqpoint{1.030082in}{3.752304in}}%
\pgfpathlineto{\pgfqpoint{1.030156in}{3.754790in}}%
\pgfpathlineto{\pgfqpoint{1.030501in}{3.710940in}}%
\pgfpathlineto{\pgfqpoint{1.030882in}{3.662054in}}%
\pgfpathlineto{\pgfqpoint{1.031498in}{3.717126in}}%
\pgfpathlineto{\pgfqpoint{1.031596in}{3.713919in}}%
\pgfpathlineto{\pgfqpoint{1.033602in}{3.616730in}}%
\pgfpathlineto{\pgfqpoint{1.033873in}{3.632020in}}%
\pgfpathlineto{\pgfqpoint{1.034193in}{3.650206in}}%
\pgfpathlineto{\pgfqpoint{1.034685in}{3.606901in}}%
\pgfpathlineto{\pgfqpoint{1.035535in}{3.618729in}}%
\pgfpathlineto{\pgfqpoint{1.036125in}{3.572035in}}%
\pgfpathlineto{\pgfqpoint{1.037664in}{3.538706in}}%
\pgfpathlineto{\pgfqpoint{1.036876in}{3.587947in}}%
\pgfpathlineto{\pgfqpoint{1.037713in}{3.539054in}}%
\pgfpathlineto{\pgfqpoint{1.038218in}{3.557642in}}%
\pgfpathlineto{\pgfqpoint{1.038599in}{3.536311in}}%
\pgfpathlineto{\pgfqpoint{1.040248in}{3.486752in}}%
\pgfpathlineto{\pgfqpoint{1.041738in}{3.456663in}}%
\pgfpathlineto{\pgfqpoint{1.040901in}{3.497967in}}%
\pgfpathlineto{\pgfqpoint{1.042008in}{3.463133in}}%
\pgfpathlineto{\pgfqpoint{1.042242in}{3.468446in}}%
\pgfpathlineto{\pgfqpoint{1.042673in}{3.448332in}}%
\pgfpathlineto{\pgfqpoint{1.044396in}{3.401340in}}%
\pgfpathlineto{\pgfqpoint{1.044458in}{3.401013in}}%
\pgfpathlineto{\pgfqpoint{1.044814in}{3.408511in}}%
\pgfpathlineto{\pgfqpoint{1.044938in}{3.409832in}}%
\pgfpathlineto{\pgfqpoint{1.045294in}{3.396478in}}%
\pgfpathlineto{\pgfqpoint{1.047165in}{3.344924in}}%
\pgfpathlineto{\pgfqpoint{1.047178in}{3.344937in}}%
\pgfpathlineto{\pgfqpoint{1.047448in}{3.349468in}}%
\pgfpathlineto{\pgfqpoint{1.047621in}{3.351601in}}%
\pgfpathlineto{\pgfqpoint{1.048002in}{3.338418in}}%
\pgfpathlineto{\pgfqpoint{1.049885in}{3.288554in}}%
\pgfpathlineto{\pgfqpoint{1.049910in}{3.288622in}}%
\pgfpathlineto{\pgfqpoint{1.050316in}{3.293629in}}%
\pgfpathlineto{\pgfqpoint{1.050599in}{3.286467in}}%
\pgfpathlineto{\pgfqpoint{1.052593in}{3.231978in}}%
\pgfpathlineto{\pgfqpoint{1.052778in}{3.233670in}}%
\pgfpathlineto{\pgfqpoint{1.052999in}{3.235847in}}%
\pgfpathlineto{\pgfqpoint{1.053356in}{3.225994in}}%
\pgfpathlineto{\pgfqpoint{1.055313in}{3.175261in}}%
\pgfpathlineto{\pgfqpoint{1.055424in}{3.175928in}}%
\pgfpathlineto{\pgfqpoint{1.055682in}{3.178198in}}%
\pgfpathlineto{\pgfqpoint{1.056014in}{3.170487in}}%
\pgfpathlineto{\pgfqpoint{1.058021in}{3.118432in}}%
\pgfpathlineto{\pgfqpoint{1.058218in}{3.119755in}}%
\pgfpathlineto{\pgfqpoint{1.058365in}{3.120648in}}%
\pgfpathlineto{\pgfqpoint{1.058685in}{3.114229in}}%
\pgfpathlineto{\pgfqpoint{1.060741in}{3.061525in}}%
\pgfpathlineto{\pgfqpoint{1.061024in}{3.063131in}}%
\pgfpathlineto{\pgfqpoint{1.061061in}{3.063178in}}%
\pgfpathlineto{\pgfqpoint{1.061245in}{3.061168in}}%
\pgfpathlineto{\pgfqpoint{1.073270in}{2.804655in}}%
\pgfpathlineto{\pgfqpoint{1.073750in}{2.789510in}}%
\pgfpathlineto{\pgfqpoint{1.075657in}{2.747672in}}%
\pgfpathlineto{\pgfqpoint{1.075793in}{2.747858in}}%
\pgfpathlineto{\pgfqpoint{1.075854in}{2.747882in}}%
\pgfpathlineto{\pgfqpoint{1.076039in}{2.746457in}}%
\pgfpathlineto{\pgfqpoint{1.079140in}{2.675570in}}%
\pgfpathlineto{\pgfqpoint{1.081085in}{2.633353in}}%
\pgfpathlineto{\pgfqpoint{1.081220in}{2.633421in}}%
\pgfpathlineto{\pgfqpoint{1.081417in}{2.632193in}}%
\pgfpathlineto{\pgfqpoint{1.083202in}{2.589308in}}%
\pgfpathlineto{\pgfqpoint{1.085282in}{2.547580in}}%
\pgfpathlineto{\pgfqpoint{1.085516in}{2.545498in}}%
\pgfpathlineto{\pgfqpoint{1.094230in}{2.352103in}}%
\pgfpathlineto{\pgfqpoint{1.097504in}{2.289953in}}%
\pgfpathlineto{\pgfqpoint{1.097873in}{2.282143in}}%
\pgfpathlineto{\pgfqpoint{1.100322in}{2.231661in}}%
\pgfpathlineto{\pgfqpoint{1.102143in}{2.187256in}}%
\pgfpathlineto{\pgfqpoint{1.104211in}{2.147184in}}%
\pgfpathlineto{\pgfqpoint{1.104482in}{2.143767in}}%
\pgfpathlineto{\pgfqpoint{1.110771in}{2.004110in}}%
\pgfpathlineto{\pgfqpoint{1.110931in}{2.004326in}}%
\pgfpathlineto{\pgfqpoint{1.111054in}{2.003948in}}%
\pgfpathlineto{\pgfqpoint{1.111116in}{2.003310in}}%
\pgfpathlineto{\pgfqpoint{1.112814in}{1.964121in}}%
\pgfpathlineto{\pgfqpoint{1.114796in}{1.918214in}}%
\pgfpathlineto{\pgfqpoint{1.114956in}{1.918566in}}%
\pgfpathlineto{\pgfqpoint{1.115017in}{1.918635in}}%
\pgfpathlineto{\pgfqpoint{1.115227in}{1.916757in}}%
\pgfpathlineto{\pgfqpoint{1.127965in}{1.635895in}}%
\pgfpathlineto{\pgfqpoint{1.129577in}{1.602588in}}%
\pgfpathlineto{\pgfqpoint{1.129971in}{1.605076in}}%
\pgfpathlineto{\pgfqpoint{1.130254in}{1.598281in}}%
\pgfpathlineto{\pgfqpoint{1.132260in}{1.544991in}}%
\pgfpathlineto{\pgfqpoint{1.132519in}{1.547470in}}%
\pgfpathlineto{\pgfqpoint{1.132654in}{1.548365in}}%
\pgfpathlineto{\pgfqpoint{1.132974in}{1.540573in}}%
\pgfpathlineto{\pgfqpoint{1.134943in}{1.487281in}}%
\pgfpathlineto{\pgfqpoint{1.135153in}{1.489426in}}%
\pgfpathlineto{\pgfqpoint{1.135362in}{1.491714in}}%
\pgfpathlineto{\pgfqpoint{1.135731in}{1.480842in}}%
\pgfpathlineto{\pgfqpoint{1.137639in}{1.429411in}}%
\pgfpathlineto{\pgfqpoint{1.137737in}{1.430165in}}%
\pgfpathlineto{\pgfqpoint{1.138082in}{1.435232in}}%
\pgfpathlineto{\pgfqpoint{1.138426in}{1.424307in}}%
\pgfpathlineto{\pgfqpoint{1.140322in}{1.371322in}}%
\pgfpathlineto{\pgfqpoint{1.140420in}{1.372145in}}%
\pgfpathlineto{\pgfqpoint{1.140790in}{1.378990in}}%
\pgfpathlineto{\pgfqpoint{1.141146in}{1.366350in}}%
\pgfpathlineto{\pgfqpoint{1.143005in}{1.312920in}}%
\pgfpathlineto{\pgfqpoint{1.143066in}{1.313264in}}%
\pgfpathlineto{\pgfqpoint{1.143509in}{1.323097in}}%
\pgfpathlineto{\pgfqpoint{1.143854in}{1.309162in}}%
\pgfpathlineto{\pgfqpoint{1.145700in}{1.254033in}}%
\pgfpathlineto{\pgfqpoint{1.145713in}{1.254066in}}%
\pgfpathlineto{\pgfqpoint{1.145983in}{1.261861in}}%
\pgfpathlineto{\pgfqpoint{1.146217in}{1.267744in}}%
\pgfpathlineto{\pgfqpoint{1.146648in}{1.244769in}}%
\pgfpathlineto{\pgfqpoint{1.148346in}{1.194582in}}%
\pgfpathlineto{\pgfqpoint{1.148383in}{1.194358in}}%
\pgfpathlineto{\pgfqpoint{1.148642in}{1.202867in}}%
\pgfpathlineto{\pgfqpoint{1.148937in}{1.213294in}}%
\pgfpathlineto{\pgfqpoint{1.149380in}{1.184083in}}%
\pgfpathlineto{\pgfqpoint{1.150291in}{1.186620in}}%
\pgfpathlineto{\pgfqpoint{1.150993in}{1.134542in}}%
\pgfpathlineto{\pgfqpoint{1.151066in}{1.133271in}}%
\pgfpathlineto{\pgfqpoint{1.151460in}{1.153842in}}%
\pgfpathlineto{\pgfqpoint{1.151645in}{1.160517in}}%
\pgfpathlineto{\pgfqpoint{1.152088in}{1.124714in}}%
\pgfpathlineto{\pgfqpoint{1.152408in}{1.101794in}}%
\pgfpathlineto{\pgfqpoint{1.152999in}{1.135270in}}%
\pgfpathlineto{\pgfqpoint{1.153159in}{1.129103in}}%
\pgfpathlineto{\pgfqpoint{1.155103in}{1.034874in}}%
\pgfpathlineto{\pgfqpoint{1.155276in}{1.044844in}}%
\pgfpathlineto{\pgfqpoint{1.155719in}{1.089946in}}%
\pgfpathlineto{\pgfqpoint{1.156186in}{1.024161in}}%
\pgfpathlineto{\pgfqpoint{1.156445in}{0.997210in}}%
\pgfpathlineto{\pgfqpoint{1.157023in}{1.071844in}}%
\pgfpathlineto{\pgfqpoint{1.157073in}{1.073229in}}%
\pgfpathlineto{\pgfqpoint{1.157343in}{1.037763in}}%
\pgfpathlineto{\pgfqpoint{1.157786in}{0.952349in}}%
\pgfpathlineto{\pgfqpoint{1.158340in}{1.061873in}}%
\pgfpathlineto{\pgfqpoint{1.158426in}{1.068181in}}%
\pgfpathlineto{\pgfqpoint{1.158783in}{0.981804in}}%
\pgfpathlineto{\pgfqpoint{1.159128in}{0.886529in}}%
\pgfpathlineto{\pgfqpoint{1.159645in}{1.078072in}}%
\pgfpathlineto{\pgfqpoint{1.159793in}{1.107213in}}%
\pgfpathlineto{\pgfqpoint{1.160186in}{0.876497in}}%
\pgfpathlineto{\pgfqpoint{1.160469in}{0.696000in}}%
\pgfpathlineto{\pgfqpoint{1.160851in}{1.249397in}}%
\pgfpathlineto{\pgfqpoint{1.161823in}{4.056000in}}%
\pgfpathlineto{\pgfqpoint{1.162771in}{3.730654in}}%
\pgfpathlineto{\pgfqpoint{1.163165in}{3.865471in}}%
\pgfpathlineto{\pgfqpoint{1.163792in}{3.687416in}}%
\pgfpathlineto{\pgfqpoint{1.163866in}{3.683819in}}%
\pgfpathlineto{\pgfqpoint{1.164211in}{3.751759in}}%
\pgfpathlineto{\pgfqpoint{1.164506in}{3.799651in}}%
\pgfpathlineto{\pgfqpoint{1.165085in}{3.687935in}}%
\pgfpathlineto{\pgfqpoint{1.165220in}{3.678771in}}%
\pgfpathlineto{\pgfqpoint{1.165725in}{3.748065in}}%
\pgfpathlineto{\pgfqpoint{1.165848in}{3.754790in}}%
\pgfpathlineto{\pgfqpoint{1.166279in}{3.693807in}}%
\pgfpathlineto{\pgfqpoint{1.166574in}{3.662054in}}%
\pgfpathlineto{\pgfqpoint{1.167189in}{3.717126in}}%
\pgfpathlineto{\pgfqpoint{1.167276in}{3.714684in}}%
\pgfpathlineto{\pgfqpoint{1.169294in}{3.616730in}}%
\pgfpathlineto{\pgfqpoint{1.169626in}{3.637392in}}%
\pgfpathlineto{\pgfqpoint{1.169885in}{3.650206in}}%
\pgfpathlineto{\pgfqpoint{1.170377in}{3.606901in}}%
\pgfpathlineto{\pgfqpoint{1.171226in}{3.618729in}}%
\pgfpathlineto{\pgfqpoint{1.171817in}{3.572035in}}%
\pgfpathlineto{\pgfqpoint{1.173356in}{3.538706in}}%
\pgfpathlineto{\pgfqpoint{1.172568in}{3.587947in}}%
\pgfpathlineto{\pgfqpoint{1.173405in}{3.539054in}}%
\pgfpathlineto{\pgfqpoint{1.173909in}{3.557642in}}%
\pgfpathlineto{\pgfqpoint{1.174291in}{3.536311in}}%
\pgfpathlineto{\pgfqpoint{1.175940in}{3.486752in}}%
\pgfpathlineto{\pgfqpoint{1.177429in}{3.456663in}}%
\pgfpathlineto{\pgfqpoint{1.176592in}{3.497967in}}%
\pgfpathlineto{\pgfqpoint{1.177700in}{3.463133in}}%
\pgfpathlineto{\pgfqpoint{1.177934in}{3.468446in}}%
\pgfpathlineto{\pgfqpoint{1.178365in}{3.448332in}}%
\pgfpathlineto{\pgfqpoint{1.180088in}{3.401340in}}%
\pgfpathlineto{\pgfqpoint{1.180149in}{3.401013in}}%
\pgfpathlineto{\pgfqpoint{1.180506in}{3.408511in}}%
\pgfpathlineto{\pgfqpoint{1.180629in}{3.409832in}}%
\pgfpathlineto{\pgfqpoint{1.180986in}{3.396478in}}%
\pgfpathlineto{\pgfqpoint{1.182857in}{3.344924in}}%
\pgfpathlineto{\pgfqpoint{1.182869in}{3.344937in}}%
\pgfpathlineto{\pgfqpoint{1.183140in}{3.349468in}}%
\pgfpathlineto{\pgfqpoint{1.183312in}{3.351601in}}%
\pgfpathlineto{\pgfqpoint{1.183694in}{3.338418in}}%
\pgfpathlineto{\pgfqpoint{1.185577in}{3.288554in}}%
\pgfpathlineto{\pgfqpoint{1.185602in}{3.288622in}}%
\pgfpathlineto{\pgfqpoint{1.186008in}{3.293629in}}%
\pgfpathlineto{\pgfqpoint{1.186291in}{3.286467in}}%
\pgfpathlineto{\pgfqpoint{1.188285in}{3.231978in}}%
\pgfpathlineto{\pgfqpoint{1.188469in}{3.233670in}}%
\pgfpathlineto{\pgfqpoint{1.188691in}{3.235847in}}%
\pgfpathlineto{\pgfqpoint{1.189048in}{3.225994in}}%
\pgfpathlineto{\pgfqpoint{1.191005in}{3.175261in}}%
\pgfpathlineto{\pgfqpoint{1.191115in}{3.175928in}}%
\pgfpathlineto{\pgfqpoint{1.191374in}{3.178198in}}%
\pgfpathlineto{\pgfqpoint{1.191706in}{3.170487in}}%
\pgfpathlineto{\pgfqpoint{1.193712in}{3.118432in}}%
\pgfpathlineto{\pgfqpoint{1.193909in}{3.119755in}}%
\pgfpathlineto{\pgfqpoint{1.194057in}{3.120648in}}%
\pgfpathlineto{\pgfqpoint{1.194377in}{3.114229in}}%
\pgfpathlineto{\pgfqpoint{1.196432in}{3.061525in}}%
\pgfpathlineto{\pgfqpoint{1.196715in}{3.063131in}}%
\pgfpathlineto{\pgfqpoint{1.196752in}{3.063178in}}%
\pgfpathlineto{\pgfqpoint{1.196937in}{3.061168in}}%
\pgfpathlineto{\pgfqpoint{1.208962in}{2.804655in}}%
\pgfpathlineto{\pgfqpoint{1.209442in}{2.789510in}}%
\pgfpathlineto{\pgfqpoint{1.211349in}{2.747672in}}%
\pgfpathlineto{\pgfqpoint{1.211485in}{2.747858in}}%
\pgfpathlineto{\pgfqpoint{1.211546in}{2.747882in}}%
\pgfpathlineto{\pgfqpoint{1.211731in}{2.746457in}}%
\pgfpathlineto{\pgfqpoint{1.214832in}{2.675570in}}%
\pgfpathlineto{\pgfqpoint{1.216777in}{2.633353in}}%
\pgfpathlineto{\pgfqpoint{1.216912in}{2.633421in}}%
\pgfpathlineto{\pgfqpoint{1.217109in}{2.632193in}}%
\pgfpathlineto{\pgfqpoint{1.218894in}{2.589308in}}%
\pgfpathlineto{\pgfqpoint{1.220974in}{2.547580in}}%
\pgfpathlineto{\pgfqpoint{1.221208in}{2.545498in}}%
\pgfpathlineto{\pgfqpoint{1.229921in}{2.352103in}}%
\pgfpathlineto{\pgfqpoint{1.233195in}{2.289953in}}%
\pgfpathlineto{\pgfqpoint{1.233565in}{2.282143in}}%
\pgfpathlineto{\pgfqpoint{1.236014in}{2.231661in}}%
\pgfpathlineto{\pgfqpoint{1.237835in}{2.187256in}}%
\pgfpathlineto{\pgfqpoint{1.239903in}{2.147184in}}%
\pgfpathlineto{\pgfqpoint{1.240174in}{2.143767in}}%
\pgfpathlineto{\pgfqpoint{1.246463in}{2.004110in}}%
\pgfpathlineto{\pgfqpoint{1.246623in}{2.004326in}}%
\pgfpathlineto{\pgfqpoint{1.246746in}{2.003948in}}%
\pgfpathlineto{\pgfqpoint{1.246808in}{2.003310in}}%
\pgfpathlineto{\pgfqpoint{1.248506in}{1.964121in}}%
\pgfpathlineto{\pgfqpoint{1.250488in}{1.918214in}}%
\pgfpathlineto{\pgfqpoint{1.250648in}{1.918566in}}%
\pgfpathlineto{\pgfqpoint{1.250709in}{1.918635in}}%
\pgfpathlineto{\pgfqpoint{1.250918in}{1.916757in}}%
\pgfpathlineto{\pgfqpoint{1.263657in}{1.635895in}}%
\pgfpathlineto{\pgfqpoint{1.265269in}{1.602588in}}%
\pgfpathlineto{\pgfqpoint{1.265663in}{1.605076in}}%
\pgfpathlineto{\pgfqpoint{1.265946in}{1.598281in}}%
\pgfpathlineto{\pgfqpoint{1.267952in}{1.544991in}}%
\pgfpathlineto{\pgfqpoint{1.268211in}{1.547470in}}%
\pgfpathlineto{\pgfqpoint{1.268346in}{1.548365in}}%
\pgfpathlineto{\pgfqpoint{1.268666in}{1.540573in}}%
\pgfpathlineto{\pgfqpoint{1.270635in}{1.487281in}}%
\pgfpathlineto{\pgfqpoint{1.270844in}{1.489426in}}%
\pgfpathlineto{\pgfqpoint{1.271054in}{1.491714in}}%
\pgfpathlineto{\pgfqpoint{1.271423in}{1.480842in}}%
\pgfpathlineto{\pgfqpoint{1.273331in}{1.429411in}}%
\pgfpathlineto{\pgfqpoint{1.273429in}{1.430165in}}%
\pgfpathlineto{\pgfqpoint{1.273774in}{1.435232in}}%
\pgfpathlineto{\pgfqpoint{1.274118in}{1.424307in}}%
\pgfpathlineto{\pgfqpoint{1.276014in}{1.371322in}}%
\pgfpathlineto{\pgfqpoint{1.276112in}{1.372145in}}%
\pgfpathlineto{\pgfqpoint{1.276481in}{1.378990in}}%
\pgfpathlineto{\pgfqpoint{1.276838in}{1.366350in}}%
\pgfpathlineto{\pgfqpoint{1.278697in}{1.312920in}}%
\pgfpathlineto{\pgfqpoint{1.278758in}{1.313264in}}%
\pgfpathlineto{\pgfqpoint{1.279201in}{1.323097in}}%
\pgfpathlineto{\pgfqpoint{1.279546in}{1.309162in}}%
\pgfpathlineto{\pgfqpoint{1.281392in}{1.254033in}}%
\pgfpathlineto{\pgfqpoint{1.281404in}{1.254066in}}%
\pgfpathlineto{\pgfqpoint{1.281675in}{1.261861in}}%
\pgfpathlineto{\pgfqpoint{1.281909in}{1.267744in}}%
\pgfpathlineto{\pgfqpoint{1.282340in}{1.244769in}}%
\pgfpathlineto{\pgfqpoint{1.284038in}{1.194582in}}%
\pgfpathlineto{\pgfqpoint{1.284075in}{1.194358in}}%
\pgfpathlineto{\pgfqpoint{1.284334in}{1.202867in}}%
\pgfpathlineto{\pgfqpoint{1.284629in}{1.213294in}}%
\pgfpathlineto{\pgfqpoint{1.285072in}{1.184083in}}%
\pgfpathlineto{\pgfqpoint{1.285983in}{1.186620in}}%
\pgfpathlineto{\pgfqpoint{1.286684in}{1.134542in}}%
\pgfpathlineto{\pgfqpoint{1.286758in}{1.133271in}}%
\pgfpathlineto{\pgfqpoint{1.287152in}{1.153842in}}%
\pgfpathlineto{\pgfqpoint{1.287337in}{1.160517in}}%
\pgfpathlineto{\pgfqpoint{1.287780in}{1.124714in}}%
\pgfpathlineto{\pgfqpoint{1.288100in}{1.101794in}}%
\pgfpathlineto{\pgfqpoint{1.288690in}{1.135270in}}%
\pgfpathlineto{\pgfqpoint{1.288850in}{1.129103in}}%
\pgfpathlineto{\pgfqpoint{1.290795in}{1.034874in}}%
\pgfpathlineto{\pgfqpoint{1.290967in}{1.044844in}}%
\pgfpathlineto{\pgfqpoint{1.291410in}{1.089946in}}%
\pgfpathlineto{\pgfqpoint{1.291878in}{1.024161in}}%
\pgfpathlineto{\pgfqpoint{1.292137in}{0.997210in}}%
\pgfpathlineto{\pgfqpoint{1.292715in}{1.071844in}}%
\pgfpathlineto{\pgfqpoint{1.292764in}{1.073229in}}%
\pgfpathlineto{\pgfqpoint{1.293035in}{1.037763in}}%
\pgfpathlineto{\pgfqpoint{1.293478in}{0.952349in}}%
\pgfpathlineto{\pgfqpoint{1.294032in}{1.061873in}}%
\pgfpathlineto{\pgfqpoint{1.294118in}{1.068181in}}%
\pgfpathlineto{\pgfqpoint{1.294475in}{0.981804in}}%
\pgfpathlineto{\pgfqpoint{1.294820in}{0.886529in}}%
\pgfpathlineto{\pgfqpoint{1.295337in}{1.078072in}}%
\pgfpathlineto{\pgfqpoint{1.295484in}{1.107213in}}%
\pgfpathlineto{\pgfqpoint{1.295878in}{0.876497in}}%
\pgfpathlineto{\pgfqpoint{1.296161in}{0.696000in}}%
\pgfpathlineto{\pgfqpoint{1.296543in}{1.249397in}}%
\pgfpathlineto{\pgfqpoint{1.296826in}{2.323869in}}%
\pgfpathlineto{\pgfqpoint{1.296826in}{2.323869in}}%
\pgfusepath{stroke}%
\end{pgfscope}%
\begin{pgfscope}%
\pgfpathrectangle{\pgfqpoint{0.800000in}{0.528000in}}{\pgfqpoint{4.960000in}{3.696000in}}%
\pgfusepath{clip}%
\pgfsetrectcap%
\pgfsetroundjoin%
\pgfsetlinewidth{1.505625pt}%
\definecolor{currentstroke}{rgb}{0.890196,0.466667,0.760784}%
\pgfsetstrokecolor{currentstroke}%
\pgfsetdash{}{0pt}%
\pgfpathmoveto{\pgfqpoint{1.025455in}{2.376000in}}%
\pgfpathlineto{\pgfqpoint{1.030218in}{3.150546in}}%
\pgfpathlineto{\pgfqpoint{1.032236in}{3.225575in}}%
\pgfpathlineto{\pgfqpoint{1.032285in}{3.225541in}}%
\pgfpathlineto{\pgfqpoint{1.032667in}{3.222793in}}%
\pgfpathlineto{\pgfqpoint{1.033762in}{3.194594in}}%
\pgfpathlineto{\pgfqpoint{1.039030in}{3.019751in}}%
\pgfpathlineto{\pgfqpoint{1.039571in}{3.022184in}}%
\pgfpathlineto{\pgfqpoint{1.040790in}{3.042708in}}%
\pgfpathlineto{\pgfqpoint{1.045811in}{3.150059in}}%
\pgfpathlineto{\pgfqpoint{1.046341in}{3.148370in}}%
\pgfpathlineto{\pgfqpoint{1.047522in}{3.133761in}}%
\pgfpathlineto{\pgfqpoint{1.052593in}{3.047010in}}%
\pgfpathlineto{\pgfqpoint{1.053356in}{3.049960in}}%
\pgfpathlineto{\pgfqpoint{1.054796in}{3.069511in}}%
\pgfpathlineto{\pgfqpoint{1.059387in}{3.139869in}}%
\pgfpathlineto{\pgfqpoint{1.059682in}{3.139415in}}%
\pgfpathlineto{\pgfqpoint{1.060556in}{3.133221in}}%
\pgfpathlineto{\pgfqpoint{1.062513in}{3.099353in}}%
\pgfpathlineto{\pgfqpoint{1.066082in}{3.047043in}}%
\pgfpathlineto{\pgfqpoint{1.066291in}{3.047096in}}%
\pgfpathlineto{\pgfqpoint{1.066599in}{3.047996in}}%
\pgfpathlineto{\pgfqpoint{1.067633in}{3.057912in}}%
\pgfpathlineto{\pgfqpoint{1.070057in}{3.108952in}}%
\pgfpathlineto{\pgfqpoint{1.072925in}{3.150056in}}%
\pgfpathlineto{\pgfqpoint{1.073356in}{3.149038in}}%
\pgfpathlineto{\pgfqpoint{1.074341in}{3.138307in}}%
\pgfpathlineto{\pgfqpoint{1.076482in}{3.084355in}}%
\pgfpathlineto{\pgfqpoint{1.079731in}{3.019751in}}%
\pgfpathlineto{\pgfqpoint{1.080100in}{3.020904in}}%
\pgfpathlineto{\pgfqpoint{1.081011in}{3.033802in}}%
\pgfpathlineto{\pgfqpoint{1.082870in}{3.099250in}}%
\pgfpathlineto{\pgfqpoint{1.086513in}{3.225575in}}%
\pgfpathlineto{\pgfqpoint{1.086747in}{3.224711in}}%
\pgfpathlineto{\pgfqpoint{1.087448in}{3.210653in}}%
\pgfpathlineto{\pgfqpoint{1.088790in}{3.129514in}}%
\pgfpathlineto{\pgfqpoint{1.090993in}{2.835127in}}%
\pgfpathlineto{\pgfqpoint{1.099584in}{1.530600in}}%
\pgfpathlineto{\pgfqpoint{1.100088in}{1.526425in}}%
\pgfpathlineto{\pgfqpoint{1.100654in}{1.531278in}}%
\pgfpathlineto{\pgfqpoint{1.101971in}{1.571636in}}%
\pgfpathlineto{\pgfqpoint{1.106870in}{1.732249in}}%
\pgfpathlineto{\pgfqpoint{1.106931in}{1.732218in}}%
\pgfpathlineto{\pgfqpoint{1.107387in}{1.730078in}}%
\pgfpathlineto{\pgfqpoint{1.108568in}{1.710876in}}%
\pgfpathlineto{\pgfqpoint{1.113663in}{1.601941in}}%
\pgfpathlineto{\pgfqpoint{1.114279in}{1.604257in}}%
\pgfpathlineto{\pgfqpoint{1.115571in}{1.621984in}}%
\pgfpathlineto{\pgfqpoint{1.120445in}{1.704990in}}%
\pgfpathlineto{\pgfqpoint{1.120937in}{1.703721in}}%
\pgfpathlineto{\pgfqpoint{1.122057in}{1.692289in}}%
\pgfpathlineto{\pgfqpoint{1.127226in}{1.612131in}}%
\pgfpathlineto{\pgfqpoint{1.128223in}{1.616951in}}%
\pgfpathlineto{\pgfqpoint{1.129934in}{1.643797in}}%
\pgfpathlineto{\pgfqpoint{1.134008in}{1.704990in}}%
\pgfpathlineto{\pgfqpoint{1.134057in}{1.704977in}}%
\pgfpathlineto{\pgfqpoint{1.134550in}{1.703477in}}%
\pgfpathlineto{\pgfqpoint{1.135706in}{1.690601in}}%
\pgfpathlineto{\pgfqpoint{1.140802in}{1.601941in}}%
\pgfpathlineto{\pgfqpoint{1.141848in}{1.608724in}}%
\pgfpathlineto{\pgfqpoint{1.143546in}{1.644768in}}%
\pgfpathlineto{\pgfqpoint{1.147583in}{1.732249in}}%
\pgfpathlineto{\pgfqpoint{1.147780in}{1.731903in}}%
\pgfpathlineto{\pgfqpoint{1.148469in}{1.725439in}}%
\pgfpathlineto{\pgfqpoint{1.149922in}{1.686007in}}%
\pgfpathlineto{\pgfqpoint{1.154365in}{1.526425in}}%
\pgfpathlineto{\pgfqpoint{1.155042in}{1.534212in}}%
\pgfpathlineto{\pgfqpoint{1.156199in}{1.587787in}}%
\pgfpathlineto{\pgfqpoint{1.158119in}{1.800135in}}%
\pgfpathlineto{\pgfqpoint{1.161860in}{2.526302in}}%
\pgfpathlineto{\pgfqpoint{1.166008in}{3.158016in}}%
\pgfpathlineto{\pgfqpoint{1.167928in}{3.225575in}}%
\pgfpathlineto{\pgfqpoint{1.168026in}{3.225432in}}%
\pgfpathlineto{\pgfqpoint{1.168531in}{3.220201in}}%
\pgfpathlineto{\pgfqpoint{1.169897in}{3.177004in}}%
\pgfpathlineto{\pgfqpoint{1.174697in}{3.019753in}}%
\pgfpathlineto{\pgfqpoint{1.175017in}{3.020498in}}%
\pgfpathlineto{\pgfqpoint{1.175066in}{3.020760in}}%
\pgfpathlineto{\pgfqpoint{1.176014in}{3.032705in}}%
\pgfpathlineto{\pgfqpoint{1.178549in}{3.101656in}}%
\pgfpathlineto{\pgfqpoint{1.181331in}{3.149885in}}%
\pgfpathlineto{\pgfqpoint{1.181540in}{3.150050in}}%
\pgfpathlineto{\pgfqpoint{1.181971in}{3.148735in}}%
\pgfpathlineto{\pgfqpoint{1.183066in}{3.136290in}}%
\pgfpathlineto{\pgfqpoint{1.188285in}{3.047010in}}%
\pgfpathlineto{\pgfqpoint{1.189257in}{3.051760in}}%
\pgfpathlineto{\pgfqpoint{1.190968in}{3.078882in}}%
\pgfpathlineto{\pgfqpoint{1.195054in}{3.139868in}}%
\pgfpathlineto{\pgfqpoint{1.195534in}{3.138814in}}%
\pgfpathlineto{\pgfqpoint{1.196605in}{3.128773in}}%
\pgfpathlineto{\pgfqpoint{1.199374in}{3.074804in}}%
\pgfpathlineto{\pgfqpoint{1.201860in}{3.047010in}}%
\pgfpathlineto{\pgfqpoint{1.201897in}{3.047020in}}%
\pgfpathlineto{\pgfqpoint{1.202389in}{3.048489in}}%
\pgfpathlineto{\pgfqpoint{1.203534in}{3.061101in}}%
\pgfpathlineto{\pgfqpoint{1.206918in}{3.133652in}}%
\pgfpathlineto{\pgfqpoint{1.208642in}{3.150059in}}%
\pgfpathlineto{\pgfqpoint{1.208925in}{3.149560in}}%
\pgfpathlineto{\pgfqpoint{1.209749in}{3.142552in}}%
\pgfpathlineto{\pgfqpoint{1.211509in}{3.103942in}}%
\pgfpathlineto{\pgfqpoint{1.215423in}{3.019751in}}%
\pgfpathlineto{\pgfqpoint{1.215571in}{3.019934in}}%
\pgfpathlineto{\pgfqpoint{1.216174in}{3.024561in}}%
\pgfpathlineto{\pgfqpoint{1.217491in}{3.056036in}}%
\pgfpathlineto{\pgfqpoint{1.222205in}{3.225575in}}%
\pgfpathlineto{\pgfqpoint{1.223128in}{3.211058in}}%
\pgfpathlineto{\pgfqpoint{1.224469in}{3.130603in}}%
\pgfpathlineto{\pgfqpoint{1.226660in}{2.839402in}}%
\pgfpathlineto{\pgfqpoint{1.235312in}{1.529997in}}%
\pgfpathlineto{\pgfqpoint{1.235780in}{1.526425in}}%
\pgfpathlineto{\pgfqpoint{1.236334in}{1.531077in}}%
\pgfpathlineto{\pgfqpoint{1.237638in}{1.570617in}}%
\pgfpathlineto{\pgfqpoint{1.242561in}{1.732249in}}%
\pgfpathlineto{\pgfqpoint{1.242648in}{1.732188in}}%
\pgfpathlineto{\pgfqpoint{1.243152in}{1.729426in}}%
\pgfpathlineto{\pgfqpoint{1.244432in}{1.706764in}}%
\pgfpathlineto{\pgfqpoint{1.249355in}{1.601941in}}%
\pgfpathlineto{\pgfqpoint{1.249749in}{1.602914in}}%
\pgfpathlineto{\pgfqpoint{1.250758in}{1.613264in}}%
\pgfpathlineto{\pgfqpoint{1.253441in}{1.671262in}}%
\pgfpathlineto{\pgfqpoint{1.256075in}{1.704974in}}%
\pgfpathlineto{\pgfqpoint{1.256321in}{1.704803in}}%
\pgfpathlineto{\pgfqpoint{1.256568in}{1.704013in}}%
\pgfpathlineto{\pgfqpoint{1.257614in}{1.694233in}}%
\pgfpathlineto{\pgfqpoint{1.260371in}{1.640472in}}%
\pgfpathlineto{\pgfqpoint{1.262918in}{1.612131in}}%
\pgfpathlineto{\pgfqpoint{1.263398in}{1.613271in}}%
\pgfpathlineto{\pgfqpoint{1.264494in}{1.623831in}}%
\pgfpathlineto{\pgfqpoint{1.267472in}{1.682033in}}%
\pgfpathlineto{\pgfqpoint{1.269700in}{1.704990in}}%
\pgfpathlineto{\pgfqpoint{1.269811in}{1.704927in}}%
\pgfpathlineto{\pgfqpoint{1.270389in}{1.702544in}}%
\pgfpathlineto{\pgfqpoint{1.271706in}{1.685253in}}%
\pgfpathlineto{\pgfqpoint{1.276494in}{1.601941in}}%
\pgfpathlineto{\pgfqpoint{1.277158in}{1.604714in}}%
\pgfpathlineto{\pgfqpoint{1.278438in}{1.624575in}}%
\pgfpathlineto{\pgfqpoint{1.283275in}{1.732249in}}%
\pgfpathlineto{\pgfqpoint{1.284112in}{1.726174in}}%
\pgfpathlineto{\pgfqpoint{1.285515in}{1.689662in}}%
\pgfpathlineto{\pgfqpoint{1.290057in}{1.526425in}}%
\pgfpathlineto{\pgfqpoint{1.290807in}{1.536050in}}%
\pgfpathlineto{\pgfqpoint{1.292026in}{1.597667in}}%
\pgfpathlineto{\pgfqpoint{1.294032in}{1.834139in}}%
\pgfpathlineto{\pgfqpoint{1.296826in}{2.373393in}}%
\pgfpathlineto{\pgfqpoint{1.296826in}{2.373393in}}%
\pgfusepath{stroke}%
\end{pgfscope}%
\begin{pgfscope}%
\pgfpathrectangle{\pgfqpoint{0.800000in}{0.528000in}}{\pgfqpoint{4.960000in}{3.696000in}}%
\pgfusepath{clip}%
\pgfsetrectcap%
\pgfsetroundjoin%
\pgfsetlinewidth{1.505625pt}%
\definecolor{currentstroke}{rgb}{0.498039,0.498039,0.498039}%
\pgfsetstrokecolor{currentstroke}%
\pgfsetdash{}{0pt}%
\pgfpathmoveto{\pgfqpoint{1.025455in}{2.376000in}}%
\pgfpathlineto{\pgfqpoint{1.026131in}{4.056000in}}%
\pgfpathlineto{\pgfqpoint{1.026858in}{3.648003in}}%
\pgfpathlineto{\pgfqpoint{1.027473in}{3.865471in}}%
\pgfpathlineto{\pgfqpoint{1.028199in}{3.684518in}}%
\pgfpathlineto{\pgfqpoint{1.028581in}{3.767955in}}%
\pgfpathlineto{\pgfqpoint{1.028815in}{3.799651in}}%
\pgfpathlineto{\pgfqpoint{1.029368in}{3.691596in}}%
\pgfpathlineto{\pgfqpoint{1.029528in}{3.678771in}}%
\pgfpathlineto{\pgfqpoint{1.030082in}{3.752304in}}%
\pgfpathlineto{\pgfqpoint{1.030156in}{3.754790in}}%
\pgfpathlineto{\pgfqpoint{1.030501in}{3.710940in}}%
\pgfpathlineto{\pgfqpoint{1.030882in}{3.662054in}}%
\pgfpathlineto{\pgfqpoint{1.031498in}{3.717126in}}%
\pgfpathlineto{\pgfqpoint{1.031596in}{3.713919in}}%
\pgfpathlineto{\pgfqpoint{1.033602in}{3.616730in}}%
\pgfpathlineto{\pgfqpoint{1.033873in}{3.632020in}}%
\pgfpathlineto{\pgfqpoint{1.034193in}{3.650206in}}%
\pgfpathlineto{\pgfqpoint{1.034685in}{3.606901in}}%
\pgfpathlineto{\pgfqpoint{1.035535in}{3.618729in}}%
\pgfpathlineto{\pgfqpoint{1.036125in}{3.572035in}}%
\pgfpathlineto{\pgfqpoint{1.037664in}{3.538706in}}%
\pgfpathlineto{\pgfqpoint{1.036876in}{3.587947in}}%
\pgfpathlineto{\pgfqpoint{1.037713in}{3.539054in}}%
\pgfpathlineto{\pgfqpoint{1.038218in}{3.557642in}}%
\pgfpathlineto{\pgfqpoint{1.038599in}{3.536311in}}%
\pgfpathlineto{\pgfqpoint{1.040248in}{3.486752in}}%
\pgfpathlineto{\pgfqpoint{1.041738in}{3.456663in}}%
\pgfpathlineto{\pgfqpoint{1.040901in}{3.497967in}}%
\pgfpathlineto{\pgfqpoint{1.042008in}{3.463133in}}%
\pgfpathlineto{\pgfqpoint{1.042242in}{3.468446in}}%
\pgfpathlineto{\pgfqpoint{1.042673in}{3.448332in}}%
\pgfpathlineto{\pgfqpoint{1.044396in}{3.401340in}}%
\pgfpathlineto{\pgfqpoint{1.044458in}{3.401013in}}%
\pgfpathlineto{\pgfqpoint{1.044814in}{3.408511in}}%
\pgfpathlineto{\pgfqpoint{1.044938in}{3.409832in}}%
\pgfpathlineto{\pgfqpoint{1.045294in}{3.396478in}}%
\pgfpathlineto{\pgfqpoint{1.047165in}{3.344924in}}%
\pgfpathlineto{\pgfqpoint{1.047178in}{3.344937in}}%
\pgfpathlineto{\pgfqpoint{1.047448in}{3.349468in}}%
\pgfpathlineto{\pgfqpoint{1.047621in}{3.351601in}}%
\pgfpathlineto{\pgfqpoint{1.048002in}{3.338418in}}%
\pgfpathlineto{\pgfqpoint{1.049885in}{3.288554in}}%
\pgfpathlineto{\pgfqpoint{1.049910in}{3.288622in}}%
\pgfpathlineto{\pgfqpoint{1.050316in}{3.293629in}}%
\pgfpathlineto{\pgfqpoint{1.050599in}{3.286467in}}%
\pgfpathlineto{\pgfqpoint{1.052593in}{3.231978in}}%
\pgfpathlineto{\pgfqpoint{1.052778in}{3.233670in}}%
\pgfpathlineto{\pgfqpoint{1.052999in}{3.235847in}}%
\pgfpathlineto{\pgfqpoint{1.053356in}{3.225994in}}%
\pgfpathlineto{\pgfqpoint{1.055313in}{3.175261in}}%
\pgfpathlineto{\pgfqpoint{1.055424in}{3.175928in}}%
\pgfpathlineto{\pgfqpoint{1.055682in}{3.178198in}}%
\pgfpathlineto{\pgfqpoint{1.056014in}{3.170487in}}%
\pgfpathlineto{\pgfqpoint{1.058021in}{3.118432in}}%
\pgfpathlineto{\pgfqpoint{1.058218in}{3.119755in}}%
\pgfpathlineto{\pgfqpoint{1.058365in}{3.120648in}}%
\pgfpathlineto{\pgfqpoint{1.058685in}{3.114229in}}%
\pgfpathlineto{\pgfqpoint{1.060741in}{3.061525in}}%
\pgfpathlineto{\pgfqpoint{1.061024in}{3.063131in}}%
\pgfpathlineto{\pgfqpoint{1.061061in}{3.063178in}}%
\pgfpathlineto{\pgfqpoint{1.061245in}{3.061168in}}%
\pgfpathlineto{\pgfqpoint{1.073270in}{2.804655in}}%
\pgfpathlineto{\pgfqpoint{1.073750in}{2.789510in}}%
\pgfpathlineto{\pgfqpoint{1.075657in}{2.747672in}}%
\pgfpathlineto{\pgfqpoint{1.075793in}{2.747858in}}%
\pgfpathlineto{\pgfqpoint{1.075854in}{2.747882in}}%
\pgfpathlineto{\pgfqpoint{1.076039in}{2.746457in}}%
\pgfpathlineto{\pgfqpoint{1.079140in}{2.675570in}}%
\pgfpathlineto{\pgfqpoint{1.081085in}{2.633353in}}%
\pgfpathlineto{\pgfqpoint{1.081220in}{2.633421in}}%
\pgfpathlineto{\pgfqpoint{1.081417in}{2.632193in}}%
\pgfpathlineto{\pgfqpoint{1.083202in}{2.589308in}}%
\pgfpathlineto{\pgfqpoint{1.085282in}{2.547580in}}%
\pgfpathlineto{\pgfqpoint{1.085516in}{2.545498in}}%
\pgfpathlineto{\pgfqpoint{1.094230in}{2.352103in}}%
\pgfpathlineto{\pgfqpoint{1.097504in}{2.289953in}}%
\pgfpathlineto{\pgfqpoint{1.097873in}{2.282143in}}%
\pgfpathlineto{\pgfqpoint{1.100322in}{2.231661in}}%
\pgfpathlineto{\pgfqpoint{1.102143in}{2.187256in}}%
\pgfpathlineto{\pgfqpoint{1.104211in}{2.147184in}}%
\pgfpathlineto{\pgfqpoint{1.104482in}{2.143767in}}%
\pgfpathlineto{\pgfqpoint{1.110771in}{2.004110in}}%
\pgfpathlineto{\pgfqpoint{1.110931in}{2.004326in}}%
\pgfpathlineto{\pgfqpoint{1.111054in}{2.003948in}}%
\pgfpathlineto{\pgfqpoint{1.111116in}{2.003310in}}%
\pgfpathlineto{\pgfqpoint{1.112814in}{1.964121in}}%
\pgfpathlineto{\pgfqpoint{1.114796in}{1.918214in}}%
\pgfpathlineto{\pgfqpoint{1.114956in}{1.918566in}}%
\pgfpathlineto{\pgfqpoint{1.115017in}{1.918635in}}%
\pgfpathlineto{\pgfqpoint{1.115227in}{1.916757in}}%
\pgfpathlineto{\pgfqpoint{1.127965in}{1.635895in}}%
\pgfpathlineto{\pgfqpoint{1.129577in}{1.602588in}}%
\pgfpathlineto{\pgfqpoint{1.129971in}{1.605076in}}%
\pgfpathlineto{\pgfqpoint{1.130254in}{1.598281in}}%
\pgfpathlineto{\pgfqpoint{1.132260in}{1.544991in}}%
\pgfpathlineto{\pgfqpoint{1.132519in}{1.547470in}}%
\pgfpathlineto{\pgfqpoint{1.132654in}{1.548365in}}%
\pgfpathlineto{\pgfqpoint{1.132974in}{1.540573in}}%
\pgfpathlineto{\pgfqpoint{1.134943in}{1.487281in}}%
\pgfpathlineto{\pgfqpoint{1.135153in}{1.489426in}}%
\pgfpathlineto{\pgfqpoint{1.135362in}{1.491714in}}%
\pgfpathlineto{\pgfqpoint{1.135731in}{1.480842in}}%
\pgfpathlineto{\pgfqpoint{1.137639in}{1.429411in}}%
\pgfpathlineto{\pgfqpoint{1.137737in}{1.430165in}}%
\pgfpathlineto{\pgfqpoint{1.138082in}{1.435232in}}%
\pgfpathlineto{\pgfqpoint{1.138426in}{1.424307in}}%
\pgfpathlineto{\pgfqpoint{1.140322in}{1.371322in}}%
\pgfpathlineto{\pgfqpoint{1.140420in}{1.372145in}}%
\pgfpathlineto{\pgfqpoint{1.140790in}{1.378990in}}%
\pgfpathlineto{\pgfqpoint{1.141146in}{1.366350in}}%
\pgfpathlineto{\pgfqpoint{1.143005in}{1.312920in}}%
\pgfpathlineto{\pgfqpoint{1.143066in}{1.313264in}}%
\pgfpathlineto{\pgfqpoint{1.143509in}{1.323097in}}%
\pgfpathlineto{\pgfqpoint{1.143854in}{1.309162in}}%
\pgfpathlineto{\pgfqpoint{1.145700in}{1.254033in}}%
\pgfpathlineto{\pgfqpoint{1.145713in}{1.254066in}}%
\pgfpathlineto{\pgfqpoint{1.145983in}{1.261861in}}%
\pgfpathlineto{\pgfqpoint{1.146217in}{1.267744in}}%
\pgfpathlineto{\pgfqpoint{1.146648in}{1.244769in}}%
\pgfpathlineto{\pgfqpoint{1.148346in}{1.194582in}}%
\pgfpathlineto{\pgfqpoint{1.148383in}{1.194358in}}%
\pgfpathlineto{\pgfqpoint{1.148642in}{1.202867in}}%
\pgfpathlineto{\pgfqpoint{1.148937in}{1.213294in}}%
\pgfpathlineto{\pgfqpoint{1.149380in}{1.184083in}}%
\pgfpathlineto{\pgfqpoint{1.150291in}{1.186620in}}%
\pgfpathlineto{\pgfqpoint{1.150993in}{1.134542in}}%
\pgfpathlineto{\pgfqpoint{1.151066in}{1.133271in}}%
\pgfpathlineto{\pgfqpoint{1.151460in}{1.153842in}}%
\pgfpathlineto{\pgfqpoint{1.151645in}{1.160517in}}%
\pgfpathlineto{\pgfqpoint{1.152088in}{1.124714in}}%
\pgfpathlineto{\pgfqpoint{1.152408in}{1.101794in}}%
\pgfpathlineto{\pgfqpoint{1.152999in}{1.135270in}}%
\pgfpathlineto{\pgfqpoint{1.153159in}{1.129103in}}%
\pgfpathlineto{\pgfqpoint{1.155103in}{1.034874in}}%
\pgfpathlineto{\pgfqpoint{1.155276in}{1.044844in}}%
\pgfpathlineto{\pgfqpoint{1.155719in}{1.089946in}}%
\pgfpathlineto{\pgfqpoint{1.156186in}{1.024161in}}%
\pgfpathlineto{\pgfqpoint{1.156445in}{0.997210in}}%
\pgfpathlineto{\pgfqpoint{1.157023in}{1.071844in}}%
\pgfpathlineto{\pgfqpoint{1.157073in}{1.073229in}}%
\pgfpathlineto{\pgfqpoint{1.157343in}{1.037763in}}%
\pgfpathlineto{\pgfqpoint{1.157786in}{0.952349in}}%
\pgfpathlineto{\pgfqpoint{1.158340in}{1.061873in}}%
\pgfpathlineto{\pgfqpoint{1.158426in}{1.068181in}}%
\pgfpathlineto{\pgfqpoint{1.158783in}{0.981804in}}%
\pgfpathlineto{\pgfqpoint{1.159128in}{0.886529in}}%
\pgfpathlineto{\pgfqpoint{1.159645in}{1.078072in}}%
\pgfpathlineto{\pgfqpoint{1.159793in}{1.107213in}}%
\pgfpathlineto{\pgfqpoint{1.160186in}{0.876497in}}%
\pgfpathlineto{\pgfqpoint{1.160469in}{0.696000in}}%
\pgfpathlineto{\pgfqpoint{1.160851in}{1.249397in}}%
\pgfpathlineto{\pgfqpoint{1.161823in}{4.056000in}}%
\pgfpathlineto{\pgfqpoint{1.162771in}{3.730654in}}%
\pgfpathlineto{\pgfqpoint{1.163165in}{3.865471in}}%
\pgfpathlineto{\pgfqpoint{1.163792in}{3.687416in}}%
\pgfpathlineto{\pgfqpoint{1.163866in}{3.683819in}}%
\pgfpathlineto{\pgfqpoint{1.164211in}{3.751759in}}%
\pgfpathlineto{\pgfqpoint{1.164506in}{3.799651in}}%
\pgfpathlineto{\pgfqpoint{1.165085in}{3.687935in}}%
\pgfpathlineto{\pgfqpoint{1.165220in}{3.678771in}}%
\pgfpathlineto{\pgfqpoint{1.165725in}{3.748065in}}%
\pgfpathlineto{\pgfqpoint{1.165848in}{3.754790in}}%
\pgfpathlineto{\pgfqpoint{1.166279in}{3.693807in}}%
\pgfpathlineto{\pgfqpoint{1.166574in}{3.662054in}}%
\pgfpathlineto{\pgfqpoint{1.167189in}{3.717126in}}%
\pgfpathlineto{\pgfqpoint{1.167276in}{3.714684in}}%
\pgfpathlineto{\pgfqpoint{1.169294in}{3.616730in}}%
\pgfpathlineto{\pgfqpoint{1.169626in}{3.637392in}}%
\pgfpathlineto{\pgfqpoint{1.169885in}{3.650206in}}%
\pgfpathlineto{\pgfqpoint{1.170377in}{3.606901in}}%
\pgfpathlineto{\pgfqpoint{1.171226in}{3.618729in}}%
\pgfpathlineto{\pgfqpoint{1.171817in}{3.572035in}}%
\pgfpathlineto{\pgfqpoint{1.173356in}{3.538706in}}%
\pgfpathlineto{\pgfqpoint{1.172568in}{3.587947in}}%
\pgfpathlineto{\pgfqpoint{1.173405in}{3.539054in}}%
\pgfpathlineto{\pgfqpoint{1.173909in}{3.557642in}}%
\pgfpathlineto{\pgfqpoint{1.174291in}{3.536311in}}%
\pgfpathlineto{\pgfqpoint{1.175940in}{3.486752in}}%
\pgfpathlineto{\pgfqpoint{1.177429in}{3.456663in}}%
\pgfpathlineto{\pgfqpoint{1.176592in}{3.497967in}}%
\pgfpathlineto{\pgfqpoint{1.177700in}{3.463133in}}%
\pgfpathlineto{\pgfqpoint{1.177934in}{3.468446in}}%
\pgfpathlineto{\pgfqpoint{1.178365in}{3.448332in}}%
\pgfpathlineto{\pgfqpoint{1.180088in}{3.401340in}}%
\pgfpathlineto{\pgfqpoint{1.180149in}{3.401013in}}%
\pgfpathlineto{\pgfqpoint{1.180506in}{3.408511in}}%
\pgfpathlineto{\pgfqpoint{1.180629in}{3.409832in}}%
\pgfpathlineto{\pgfqpoint{1.180986in}{3.396478in}}%
\pgfpathlineto{\pgfqpoint{1.182857in}{3.344924in}}%
\pgfpathlineto{\pgfqpoint{1.182869in}{3.344937in}}%
\pgfpathlineto{\pgfqpoint{1.183140in}{3.349468in}}%
\pgfpathlineto{\pgfqpoint{1.183312in}{3.351601in}}%
\pgfpathlineto{\pgfqpoint{1.183694in}{3.338418in}}%
\pgfpathlineto{\pgfqpoint{1.185577in}{3.288554in}}%
\pgfpathlineto{\pgfqpoint{1.185602in}{3.288622in}}%
\pgfpathlineto{\pgfqpoint{1.186008in}{3.293629in}}%
\pgfpathlineto{\pgfqpoint{1.186291in}{3.286467in}}%
\pgfpathlineto{\pgfqpoint{1.188285in}{3.231978in}}%
\pgfpathlineto{\pgfqpoint{1.188469in}{3.233670in}}%
\pgfpathlineto{\pgfqpoint{1.188691in}{3.235847in}}%
\pgfpathlineto{\pgfqpoint{1.189048in}{3.225994in}}%
\pgfpathlineto{\pgfqpoint{1.191005in}{3.175261in}}%
\pgfpathlineto{\pgfqpoint{1.191115in}{3.175928in}}%
\pgfpathlineto{\pgfqpoint{1.191374in}{3.178198in}}%
\pgfpathlineto{\pgfqpoint{1.191706in}{3.170487in}}%
\pgfpathlineto{\pgfqpoint{1.193712in}{3.118432in}}%
\pgfpathlineto{\pgfqpoint{1.193909in}{3.119755in}}%
\pgfpathlineto{\pgfqpoint{1.194057in}{3.120648in}}%
\pgfpathlineto{\pgfqpoint{1.194377in}{3.114229in}}%
\pgfpathlineto{\pgfqpoint{1.196432in}{3.061525in}}%
\pgfpathlineto{\pgfqpoint{1.196715in}{3.063131in}}%
\pgfpathlineto{\pgfqpoint{1.196752in}{3.063178in}}%
\pgfpathlineto{\pgfqpoint{1.196937in}{3.061168in}}%
\pgfpathlineto{\pgfqpoint{1.208962in}{2.804655in}}%
\pgfpathlineto{\pgfqpoint{1.209442in}{2.789510in}}%
\pgfpathlineto{\pgfqpoint{1.211349in}{2.747672in}}%
\pgfpathlineto{\pgfqpoint{1.211485in}{2.747858in}}%
\pgfpathlineto{\pgfqpoint{1.211546in}{2.747882in}}%
\pgfpathlineto{\pgfqpoint{1.211731in}{2.746457in}}%
\pgfpathlineto{\pgfqpoint{1.214832in}{2.675570in}}%
\pgfpathlineto{\pgfqpoint{1.216777in}{2.633353in}}%
\pgfpathlineto{\pgfqpoint{1.216912in}{2.633421in}}%
\pgfpathlineto{\pgfqpoint{1.217109in}{2.632193in}}%
\pgfpathlineto{\pgfqpoint{1.218894in}{2.589308in}}%
\pgfpathlineto{\pgfqpoint{1.220974in}{2.547580in}}%
\pgfpathlineto{\pgfqpoint{1.221208in}{2.545498in}}%
\pgfpathlineto{\pgfqpoint{1.229921in}{2.352103in}}%
\pgfpathlineto{\pgfqpoint{1.233195in}{2.289953in}}%
\pgfpathlineto{\pgfqpoint{1.233565in}{2.282143in}}%
\pgfpathlineto{\pgfqpoint{1.236014in}{2.231661in}}%
\pgfpathlineto{\pgfqpoint{1.237835in}{2.187256in}}%
\pgfpathlineto{\pgfqpoint{1.239903in}{2.147184in}}%
\pgfpathlineto{\pgfqpoint{1.240174in}{2.143767in}}%
\pgfpathlineto{\pgfqpoint{1.246463in}{2.004110in}}%
\pgfpathlineto{\pgfqpoint{1.246623in}{2.004326in}}%
\pgfpathlineto{\pgfqpoint{1.246746in}{2.003948in}}%
\pgfpathlineto{\pgfqpoint{1.246808in}{2.003310in}}%
\pgfpathlineto{\pgfqpoint{1.248506in}{1.964121in}}%
\pgfpathlineto{\pgfqpoint{1.250488in}{1.918214in}}%
\pgfpathlineto{\pgfqpoint{1.250648in}{1.918566in}}%
\pgfpathlineto{\pgfqpoint{1.250709in}{1.918635in}}%
\pgfpathlineto{\pgfqpoint{1.250918in}{1.916757in}}%
\pgfpathlineto{\pgfqpoint{1.263657in}{1.635895in}}%
\pgfpathlineto{\pgfqpoint{1.265269in}{1.602588in}}%
\pgfpathlineto{\pgfqpoint{1.265663in}{1.605076in}}%
\pgfpathlineto{\pgfqpoint{1.265946in}{1.598281in}}%
\pgfpathlineto{\pgfqpoint{1.267952in}{1.544991in}}%
\pgfpathlineto{\pgfqpoint{1.268211in}{1.547470in}}%
\pgfpathlineto{\pgfqpoint{1.268346in}{1.548365in}}%
\pgfpathlineto{\pgfqpoint{1.268666in}{1.540573in}}%
\pgfpathlineto{\pgfqpoint{1.270635in}{1.487281in}}%
\pgfpathlineto{\pgfqpoint{1.270844in}{1.489426in}}%
\pgfpathlineto{\pgfqpoint{1.271054in}{1.491714in}}%
\pgfpathlineto{\pgfqpoint{1.271423in}{1.480842in}}%
\pgfpathlineto{\pgfqpoint{1.273331in}{1.429411in}}%
\pgfpathlineto{\pgfqpoint{1.273429in}{1.430165in}}%
\pgfpathlineto{\pgfqpoint{1.273774in}{1.435232in}}%
\pgfpathlineto{\pgfqpoint{1.274118in}{1.424307in}}%
\pgfpathlineto{\pgfqpoint{1.276014in}{1.371322in}}%
\pgfpathlineto{\pgfqpoint{1.276112in}{1.372145in}}%
\pgfpathlineto{\pgfqpoint{1.276481in}{1.378990in}}%
\pgfpathlineto{\pgfqpoint{1.276838in}{1.366350in}}%
\pgfpathlineto{\pgfqpoint{1.278697in}{1.312920in}}%
\pgfpathlineto{\pgfqpoint{1.278758in}{1.313264in}}%
\pgfpathlineto{\pgfqpoint{1.279201in}{1.323097in}}%
\pgfpathlineto{\pgfqpoint{1.279546in}{1.309162in}}%
\pgfpathlineto{\pgfqpoint{1.281392in}{1.254033in}}%
\pgfpathlineto{\pgfqpoint{1.281404in}{1.254066in}}%
\pgfpathlineto{\pgfqpoint{1.281675in}{1.261861in}}%
\pgfpathlineto{\pgfqpoint{1.281909in}{1.267744in}}%
\pgfpathlineto{\pgfqpoint{1.282340in}{1.244769in}}%
\pgfpathlineto{\pgfqpoint{1.284038in}{1.194582in}}%
\pgfpathlineto{\pgfqpoint{1.284075in}{1.194358in}}%
\pgfpathlineto{\pgfqpoint{1.284334in}{1.202867in}}%
\pgfpathlineto{\pgfqpoint{1.284629in}{1.213294in}}%
\pgfpathlineto{\pgfqpoint{1.285072in}{1.184083in}}%
\pgfpathlineto{\pgfqpoint{1.285983in}{1.186620in}}%
\pgfpathlineto{\pgfqpoint{1.286684in}{1.134542in}}%
\pgfpathlineto{\pgfqpoint{1.286758in}{1.133271in}}%
\pgfpathlineto{\pgfqpoint{1.287152in}{1.153842in}}%
\pgfpathlineto{\pgfqpoint{1.287337in}{1.160517in}}%
\pgfpathlineto{\pgfqpoint{1.287780in}{1.124714in}}%
\pgfpathlineto{\pgfqpoint{1.288100in}{1.101794in}}%
\pgfpathlineto{\pgfqpoint{1.288690in}{1.135270in}}%
\pgfpathlineto{\pgfqpoint{1.288850in}{1.129103in}}%
\pgfpathlineto{\pgfqpoint{1.290795in}{1.034874in}}%
\pgfpathlineto{\pgfqpoint{1.290967in}{1.044844in}}%
\pgfpathlineto{\pgfqpoint{1.291410in}{1.089946in}}%
\pgfpathlineto{\pgfqpoint{1.291878in}{1.024161in}}%
\pgfpathlineto{\pgfqpoint{1.292137in}{0.997210in}}%
\pgfpathlineto{\pgfqpoint{1.292715in}{1.071844in}}%
\pgfpathlineto{\pgfqpoint{1.292764in}{1.073229in}}%
\pgfpathlineto{\pgfqpoint{1.293035in}{1.037763in}}%
\pgfpathlineto{\pgfqpoint{1.293478in}{0.952349in}}%
\pgfpathlineto{\pgfqpoint{1.294032in}{1.061873in}}%
\pgfpathlineto{\pgfqpoint{1.294118in}{1.068181in}}%
\pgfpathlineto{\pgfqpoint{1.294475in}{0.981804in}}%
\pgfpathlineto{\pgfqpoint{1.294820in}{0.886529in}}%
\pgfpathlineto{\pgfqpoint{1.295337in}{1.078072in}}%
\pgfpathlineto{\pgfqpoint{1.295484in}{1.107213in}}%
\pgfpathlineto{\pgfqpoint{1.295878in}{0.876497in}}%
\pgfpathlineto{\pgfqpoint{1.296161in}{0.696000in}}%
\pgfpathlineto{\pgfqpoint{1.296543in}{1.249397in}}%
\pgfpathlineto{\pgfqpoint{1.296826in}{2.323869in}}%
\pgfpathlineto{\pgfqpoint{1.296826in}{2.323869in}}%
\pgfusepath{stroke}%
\end{pgfscope}%
\begin{pgfscope}%
\pgfpathrectangle{\pgfqpoint{0.800000in}{0.528000in}}{\pgfqpoint{4.960000in}{3.696000in}}%
\pgfusepath{clip}%
\pgfsetrectcap%
\pgfsetroundjoin%
\pgfsetlinewidth{1.505625pt}%
\definecolor{currentstroke}{rgb}{0.737255,0.741176,0.133333}%
\pgfsetstrokecolor{currentstroke}%
\pgfsetdash{}{0pt}%
\pgfpathmoveto{\pgfqpoint{1.025455in}{2.376000in}}%
\pgfpathlineto{\pgfqpoint{1.030218in}{3.150546in}}%
\pgfpathlineto{\pgfqpoint{1.032236in}{3.225575in}}%
\pgfpathlineto{\pgfqpoint{1.032285in}{3.225541in}}%
\pgfpathlineto{\pgfqpoint{1.032667in}{3.222793in}}%
\pgfpathlineto{\pgfqpoint{1.033762in}{3.194594in}}%
\pgfpathlineto{\pgfqpoint{1.039030in}{3.019751in}}%
\pgfpathlineto{\pgfqpoint{1.039571in}{3.022184in}}%
\pgfpathlineto{\pgfqpoint{1.040790in}{3.042708in}}%
\pgfpathlineto{\pgfqpoint{1.045811in}{3.150059in}}%
\pgfpathlineto{\pgfqpoint{1.046341in}{3.148370in}}%
\pgfpathlineto{\pgfqpoint{1.047522in}{3.133761in}}%
\pgfpathlineto{\pgfqpoint{1.052593in}{3.047010in}}%
\pgfpathlineto{\pgfqpoint{1.053356in}{3.049960in}}%
\pgfpathlineto{\pgfqpoint{1.054796in}{3.069511in}}%
\pgfpathlineto{\pgfqpoint{1.059387in}{3.139869in}}%
\pgfpathlineto{\pgfqpoint{1.059682in}{3.139415in}}%
\pgfpathlineto{\pgfqpoint{1.060556in}{3.133221in}}%
\pgfpathlineto{\pgfqpoint{1.062513in}{3.099353in}}%
\pgfpathlineto{\pgfqpoint{1.066082in}{3.047043in}}%
\pgfpathlineto{\pgfqpoint{1.066291in}{3.047096in}}%
\pgfpathlineto{\pgfqpoint{1.066599in}{3.047996in}}%
\pgfpathlineto{\pgfqpoint{1.067633in}{3.057912in}}%
\pgfpathlineto{\pgfqpoint{1.070057in}{3.108952in}}%
\pgfpathlineto{\pgfqpoint{1.072925in}{3.150056in}}%
\pgfpathlineto{\pgfqpoint{1.073356in}{3.149038in}}%
\pgfpathlineto{\pgfqpoint{1.074341in}{3.138307in}}%
\pgfpathlineto{\pgfqpoint{1.076482in}{3.084355in}}%
\pgfpathlineto{\pgfqpoint{1.079731in}{3.019751in}}%
\pgfpathlineto{\pgfqpoint{1.080100in}{3.020904in}}%
\pgfpathlineto{\pgfqpoint{1.081011in}{3.033802in}}%
\pgfpathlineto{\pgfqpoint{1.082870in}{3.099250in}}%
\pgfpathlineto{\pgfqpoint{1.086513in}{3.225575in}}%
\pgfpathlineto{\pgfqpoint{1.086747in}{3.224711in}}%
\pgfpathlineto{\pgfqpoint{1.087448in}{3.210653in}}%
\pgfpathlineto{\pgfqpoint{1.088790in}{3.129514in}}%
\pgfpathlineto{\pgfqpoint{1.090993in}{2.835127in}}%
\pgfpathlineto{\pgfqpoint{1.099584in}{1.530600in}}%
\pgfpathlineto{\pgfqpoint{1.100088in}{1.526425in}}%
\pgfpathlineto{\pgfqpoint{1.100654in}{1.531278in}}%
\pgfpathlineto{\pgfqpoint{1.101971in}{1.571636in}}%
\pgfpathlineto{\pgfqpoint{1.106870in}{1.732249in}}%
\pgfpathlineto{\pgfqpoint{1.106931in}{1.732218in}}%
\pgfpathlineto{\pgfqpoint{1.107387in}{1.730078in}}%
\pgfpathlineto{\pgfqpoint{1.108568in}{1.710876in}}%
\pgfpathlineto{\pgfqpoint{1.113663in}{1.601941in}}%
\pgfpathlineto{\pgfqpoint{1.114279in}{1.604257in}}%
\pgfpathlineto{\pgfqpoint{1.115571in}{1.621984in}}%
\pgfpathlineto{\pgfqpoint{1.120445in}{1.704990in}}%
\pgfpathlineto{\pgfqpoint{1.120937in}{1.703721in}}%
\pgfpathlineto{\pgfqpoint{1.122057in}{1.692289in}}%
\pgfpathlineto{\pgfqpoint{1.127226in}{1.612131in}}%
\pgfpathlineto{\pgfqpoint{1.128223in}{1.616951in}}%
\pgfpathlineto{\pgfqpoint{1.129934in}{1.643797in}}%
\pgfpathlineto{\pgfqpoint{1.134008in}{1.704990in}}%
\pgfpathlineto{\pgfqpoint{1.134057in}{1.704977in}}%
\pgfpathlineto{\pgfqpoint{1.134550in}{1.703477in}}%
\pgfpathlineto{\pgfqpoint{1.135706in}{1.690601in}}%
\pgfpathlineto{\pgfqpoint{1.140802in}{1.601941in}}%
\pgfpathlineto{\pgfqpoint{1.141848in}{1.608724in}}%
\pgfpathlineto{\pgfqpoint{1.143546in}{1.644768in}}%
\pgfpathlineto{\pgfqpoint{1.147583in}{1.732249in}}%
\pgfpathlineto{\pgfqpoint{1.147780in}{1.731903in}}%
\pgfpathlineto{\pgfqpoint{1.148469in}{1.725439in}}%
\pgfpathlineto{\pgfqpoint{1.149922in}{1.686007in}}%
\pgfpathlineto{\pgfqpoint{1.154365in}{1.526425in}}%
\pgfpathlineto{\pgfqpoint{1.155042in}{1.534212in}}%
\pgfpathlineto{\pgfqpoint{1.156199in}{1.587787in}}%
\pgfpathlineto{\pgfqpoint{1.158119in}{1.800135in}}%
\pgfpathlineto{\pgfqpoint{1.161860in}{2.526302in}}%
\pgfpathlineto{\pgfqpoint{1.166008in}{3.158016in}}%
\pgfpathlineto{\pgfqpoint{1.167928in}{3.225575in}}%
\pgfpathlineto{\pgfqpoint{1.168026in}{3.225432in}}%
\pgfpathlineto{\pgfqpoint{1.168531in}{3.220201in}}%
\pgfpathlineto{\pgfqpoint{1.169897in}{3.177004in}}%
\pgfpathlineto{\pgfqpoint{1.174697in}{3.019753in}}%
\pgfpathlineto{\pgfqpoint{1.175017in}{3.020498in}}%
\pgfpathlineto{\pgfqpoint{1.175066in}{3.020760in}}%
\pgfpathlineto{\pgfqpoint{1.176014in}{3.032705in}}%
\pgfpathlineto{\pgfqpoint{1.178549in}{3.101656in}}%
\pgfpathlineto{\pgfqpoint{1.181331in}{3.149885in}}%
\pgfpathlineto{\pgfqpoint{1.181540in}{3.150050in}}%
\pgfpathlineto{\pgfqpoint{1.181971in}{3.148735in}}%
\pgfpathlineto{\pgfqpoint{1.183066in}{3.136290in}}%
\pgfpathlineto{\pgfqpoint{1.188285in}{3.047010in}}%
\pgfpathlineto{\pgfqpoint{1.189257in}{3.051760in}}%
\pgfpathlineto{\pgfqpoint{1.190968in}{3.078882in}}%
\pgfpathlineto{\pgfqpoint{1.195054in}{3.139868in}}%
\pgfpathlineto{\pgfqpoint{1.195534in}{3.138814in}}%
\pgfpathlineto{\pgfqpoint{1.196605in}{3.128773in}}%
\pgfpathlineto{\pgfqpoint{1.199374in}{3.074804in}}%
\pgfpathlineto{\pgfqpoint{1.201860in}{3.047010in}}%
\pgfpathlineto{\pgfqpoint{1.201897in}{3.047020in}}%
\pgfpathlineto{\pgfqpoint{1.202389in}{3.048489in}}%
\pgfpathlineto{\pgfqpoint{1.203534in}{3.061101in}}%
\pgfpathlineto{\pgfqpoint{1.206918in}{3.133652in}}%
\pgfpathlineto{\pgfqpoint{1.208642in}{3.150059in}}%
\pgfpathlineto{\pgfqpoint{1.208925in}{3.149560in}}%
\pgfpathlineto{\pgfqpoint{1.209749in}{3.142552in}}%
\pgfpathlineto{\pgfqpoint{1.211509in}{3.103942in}}%
\pgfpathlineto{\pgfqpoint{1.215423in}{3.019751in}}%
\pgfpathlineto{\pgfqpoint{1.215571in}{3.019934in}}%
\pgfpathlineto{\pgfqpoint{1.216174in}{3.024561in}}%
\pgfpathlineto{\pgfqpoint{1.217491in}{3.056036in}}%
\pgfpathlineto{\pgfqpoint{1.222205in}{3.225575in}}%
\pgfpathlineto{\pgfqpoint{1.223128in}{3.211058in}}%
\pgfpathlineto{\pgfqpoint{1.224469in}{3.130603in}}%
\pgfpathlineto{\pgfqpoint{1.226660in}{2.839402in}}%
\pgfpathlineto{\pgfqpoint{1.235312in}{1.529997in}}%
\pgfpathlineto{\pgfqpoint{1.235780in}{1.526425in}}%
\pgfpathlineto{\pgfqpoint{1.236334in}{1.531077in}}%
\pgfpathlineto{\pgfqpoint{1.237638in}{1.570617in}}%
\pgfpathlineto{\pgfqpoint{1.242561in}{1.732249in}}%
\pgfpathlineto{\pgfqpoint{1.242648in}{1.732188in}}%
\pgfpathlineto{\pgfqpoint{1.243152in}{1.729426in}}%
\pgfpathlineto{\pgfqpoint{1.244432in}{1.706764in}}%
\pgfpathlineto{\pgfqpoint{1.249355in}{1.601941in}}%
\pgfpathlineto{\pgfqpoint{1.249749in}{1.602914in}}%
\pgfpathlineto{\pgfqpoint{1.250758in}{1.613264in}}%
\pgfpathlineto{\pgfqpoint{1.253441in}{1.671262in}}%
\pgfpathlineto{\pgfqpoint{1.256075in}{1.704974in}}%
\pgfpathlineto{\pgfqpoint{1.256321in}{1.704803in}}%
\pgfpathlineto{\pgfqpoint{1.256568in}{1.704013in}}%
\pgfpathlineto{\pgfqpoint{1.257614in}{1.694233in}}%
\pgfpathlineto{\pgfqpoint{1.260371in}{1.640472in}}%
\pgfpathlineto{\pgfqpoint{1.262918in}{1.612131in}}%
\pgfpathlineto{\pgfqpoint{1.263398in}{1.613271in}}%
\pgfpathlineto{\pgfqpoint{1.264494in}{1.623831in}}%
\pgfpathlineto{\pgfqpoint{1.267472in}{1.682033in}}%
\pgfpathlineto{\pgfqpoint{1.269700in}{1.704990in}}%
\pgfpathlineto{\pgfqpoint{1.269811in}{1.704927in}}%
\pgfpathlineto{\pgfqpoint{1.270389in}{1.702544in}}%
\pgfpathlineto{\pgfqpoint{1.271706in}{1.685253in}}%
\pgfpathlineto{\pgfqpoint{1.276494in}{1.601941in}}%
\pgfpathlineto{\pgfqpoint{1.277158in}{1.604714in}}%
\pgfpathlineto{\pgfqpoint{1.278438in}{1.624575in}}%
\pgfpathlineto{\pgfqpoint{1.283275in}{1.732249in}}%
\pgfpathlineto{\pgfqpoint{1.284112in}{1.726174in}}%
\pgfpathlineto{\pgfqpoint{1.285515in}{1.689662in}}%
\pgfpathlineto{\pgfqpoint{1.290057in}{1.526425in}}%
\pgfpathlineto{\pgfqpoint{1.290807in}{1.536050in}}%
\pgfpathlineto{\pgfqpoint{1.292026in}{1.597667in}}%
\pgfpathlineto{\pgfqpoint{1.294032in}{1.834139in}}%
\pgfpathlineto{\pgfqpoint{1.296826in}{2.373393in}}%
\pgfpathlineto{\pgfqpoint{1.296826in}{2.373393in}}%
\pgfusepath{stroke}%
\end{pgfscope}%
\begin{pgfscope}%
\pgfpathrectangle{\pgfqpoint{0.800000in}{0.528000in}}{\pgfqpoint{4.960000in}{3.696000in}}%
\pgfusepath{clip}%
\pgfsetrectcap%
\pgfsetroundjoin%
\pgfsetlinewidth{1.505625pt}%
\definecolor{currentstroke}{rgb}{0.090196,0.745098,0.811765}%
\pgfsetstrokecolor{currentstroke}%
\pgfsetdash{}{0pt}%
\pgfpathmoveto{\pgfqpoint{1.025455in}{2.375942in}}%
\pgfpathlineto{\pgfqpoint{1.026525in}{2.375801in}}%
\pgfpathlineto{\pgfqpoint{1.025750in}{2.376079in}}%
\pgfpathlineto{\pgfqpoint{1.026562in}{2.375925in}}%
\pgfpathlineto{\pgfqpoint{1.027658in}{2.376105in}}%
\pgfpathlineto{\pgfqpoint{1.026673in}{2.375856in}}%
\pgfpathlineto{\pgfqpoint{1.027707in}{2.376009in}}%
\pgfpathlineto{\pgfqpoint{1.028888in}{2.375908in}}%
\pgfpathlineto{\pgfqpoint{1.027731in}{2.376075in}}%
\pgfpathlineto{\pgfqpoint{1.028975in}{2.376063in}}%
\pgfpathlineto{\pgfqpoint{1.033479in}{2.375873in}}%
\pgfpathlineto{\pgfqpoint{1.034488in}{2.375828in}}%
\pgfpathlineto{\pgfqpoint{1.033516in}{2.375944in}}%
\pgfpathlineto{\pgfqpoint{1.034611in}{2.375899in}}%
\pgfpathlineto{\pgfqpoint{1.036815in}{2.376020in}}%
\pgfpathlineto{\pgfqpoint{1.034636in}{2.376021in}}%
\pgfpathlineto{\pgfqpoint{1.036827in}{2.376075in}}%
\pgfpathlineto{\pgfqpoint{1.037331in}{2.376166in}}%
\pgfpathlineto{\pgfqpoint{1.037528in}{2.375828in}}%
\pgfpathlineto{\pgfqpoint{1.037922in}{2.375920in}}%
\pgfpathlineto{\pgfqpoint{1.041713in}{2.374675in}}%
\pgfpathlineto{\pgfqpoint{1.042796in}{2.374127in}}%
\pgfpathlineto{\pgfqpoint{1.042267in}{2.375140in}}%
\pgfpathlineto{\pgfqpoint{1.042845in}{2.374304in}}%
\pgfpathlineto{\pgfqpoint{1.044285in}{2.377466in}}%
\pgfpathlineto{\pgfqpoint{1.044458in}{2.376922in}}%
\pgfpathlineto{\pgfqpoint{1.045085in}{2.377348in}}%
\pgfpathlineto{\pgfqpoint{1.045885in}{2.376000in}}%
\pgfpathlineto{\pgfqpoint{1.047510in}{2.376010in}}%
\pgfpathlineto{\pgfqpoint{1.056458in}{2.376001in}}%
\pgfpathlineto{\pgfqpoint{1.060777in}{2.375998in}}%
\pgfpathlineto{\pgfqpoint{1.062230in}{2.375993in}}%
\pgfpathlineto{\pgfqpoint{1.063904in}{2.376003in}}%
\pgfpathlineto{\pgfqpoint{1.065996in}{2.375998in}}%
\pgfpathlineto{\pgfqpoint{1.067534in}{2.376006in}}%
\pgfpathlineto{\pgfqpoint{1.068974in}{2.376000in}}%
\pgfpathlineto{\pgfqpoint{1.085799in}{2.376008in}}%
\pgfpathlineto{\pgfqpoint{1.087424in}{2.376008in}}%
\pgfpathlineto{\pgfqpoint{1.185577in}{2.375984in}}%
\pgfpathlineto{\pgfqpoint{1.186795in}{2.376007in}}%
\pgfpathlineto{\pgfqpoint{1.191152in}{2.376003in}}%
\pgfpathlineto{\pgfqpoint{1.193343in}{2.375999in}}%
\pgfpathlineto{\pgfqpoint{1.210131in}{2.375991in}}%
\pgfpathlineto{\pgfqpoint{1.211374in}{2.376013in}}%
\pgfpathlineto{\pgfqpoint{1.212605in}{2.375998in}}%
\pgfpathlineto{\pgfqpoint{1.215078in}{2.376003in}}%
\pgfpathlineto{\pgfqpoint{1.216814in}{2.376008in}}%
\pgfpathlineto{\pgfqpoint{1.237183in}{2.376000in}}%
\pgfpathlineto{\pgfqpoint{1.238758in}{2.376006in}}%
\pgfpathlineto{\pgfqpoint{1.240211in}{2.375990in}}%
\pgfpathlineto{\pgfqpoint{1.242291in}{2.376004in}}%
\pgfpathlineto{\pgfqpoint{1.244014in}{2.375998in}}%
\pgfpathlineto{\pgfqpoint{1.250291in}{2.376012in}}%
\pgfpathlineto{\pgfqpoint{1.251497in}{2.375989in}}%
\pgfpathlineto{\pgfqpoint{1.253269in}{2.375998in}}%
\pgfpathlineto{\pgfqpoint{1.257146in}{2.375990in}}%
\pgfpathlineto{\pgfqpoint{1.258734in}{2.375999in}}%
\pgfpathlineto{\pgfqpoint{1.261011in}{2.376006in}}%
\pgfpathlineto{\pgfqpoint{1.269429in}{2.376003in}}%
\pgfpathlineto{\pgfqpoint{1.289860in}{2.375995in}}%
\pgfpathlineto{\pgfqpoint{1.291189in}{2.375997in}}%
\pgfpathlineto{\pgfqpoint{1.296518in}{2.375995in}}%
\pgfpathlineto{\pgfqpoint{1.298081in}{2.376011in}}%
\pgfpathlineto{\pgfqpoint{1.299447in}{2.375997in}}%
\pgfpathlineto{\pgfqpoint{1.301158in}{2.376009in}}%
\pgfpathlineto{\pgfqpoint{1.302647in}{2.375987in}}%
\pgfpathlineto{\pgfqpoint{1.303977in}{2.376000in}}%
\pgfpathlineto{\pgfqpoint{1.311127in}{2.376004in}}%
\pgfpathlineto{\pgfqpoint{1.312973in}{2.376002in}}%
\pgfpathlineto{\pgfqpoint{1.316752in}{2.375976in}}%
\pgfpathlineto{\pgfqpoint{1.317466in}{2.377794in}}%
\pgfpathlineto{\pgfqpoint{1.318303in}{2.377274in}}%
\pgfpathlineto{\pgfqpoint{1.319620in}{2.375108in}}%
\pgfpathlineto{\pgfqpoint{1.319632in}{2.375124in}}%
\pgfpathlineto{\pgfqpoint{1.320629in}{2.374631in}}%
\pgfpathlineto{\pgfqpoint{1.319817in}{2.375354in}}%
\pgfpathlineto{\pgfqpoint{1.320764in}{2.375299in}}%
\pgfpathlineto{\pgfqpoint{1.322057in}{2.376955in}}%
\pgfpathlineto{\pgfqpoint{1.320826in}{2.375161in}}%
\pgfpathlineto{\pgfqpoint{1.322217in}{2.376764in}}%
\pgfpathlineto{\pgfqpoint{1.323447in}{2.374116in}}%
\pgfpathlineto{\pgfqpoint{1.323620in}{2.374603in}}%
\pgfpathlineto{\pgfqpoint{1.325047in}{2.376124in}}%
\pgfpathlineto{\pgfqpoint{1.325072in}{2.376118in}}%
\pgfpathlineto{\pgfqpoint{1.325318in}{2.375709in}}%
\pgfpathlineto{\pgfqpoint{1.326069in}{2.376798in}}%
\pgfpathlineto{\pgfqpoint{1.326106in}{2.376708in}}%
\pgfpathlineto{\pgfqpoint{1.326820in}{2.377144in}}%
\pgfpathlineto{\pgfqpoint{1.326450in}{2.375909in}}%
\pgfpathlineto{\pgfqpoint{1.327115in}{2.376172in}}%
\pgfpathlineto{\pgfqpoint{1.327681in}{2.376597in}}%
\pgfpathlineto{\pgfqpoint{1.327312in}{2.376608in}}%
\pgfpathlineto{\pgfqpoint{1.327804in}{2.376963in}}%
\pgfpathlineto{\pgfqpoint{1.328100in}{2.377518in}}%
\pgfpathlineto{\pgfqpoint{1.328813in}{2.376080in}}%
\pgfpathlineto{\pgfqpoint{1.328863in}{2.376163in}}%
\pgfpathlineto{\pgfqpoint{1.329601in}{2.374701in}}%
\pgfpathlineto{\pgfqpoint{1.330069in}{2.375045in}}%
\pgfpathlineto{\pgfqpoint{1.331435in}{2.376350in}}%
\pgfpathlineto{\pgfqpoint{1.330241in}{2.374854in}}%
\pgfpathlineto{\pgfqpoint{1.331496in}{2.376232in}}%
\pgfpathlineto{\pgfqpoint{1.333330in}{2.374759in}}%
\pgfpathlineto{\pgfqpoint{1.331890in}{2.376770in}}%
\pgfpathlineto{\pgfqpoint{1.333453in}{2.375044in}}%
\pgfpathlineto{\pgfqpoint{1.334844in}{2.376825in}}%
\pgfpathlineto{\pgfqpoint{1.333712in}{2.374856in}}%
\pgfpathlineto{\pgfqpoint{1.334856in}{2.376816in}}%
\pgfpathlineto{\pgfqpoint{1.336703in}{2.375238in}}%
\pgfpathlineto{\pgfqpoint{1.334893in}{2.377001in}}%
\pgfpathlineto{\pgfqpoint{1.337133in}{2.376021in}}%
\pgfpathlineto{\pgfqpoint{1.338118in}{2.377250in}}%
\pgfpathlineto{\pgfqpoint{1.338352in}{2.376679in}}%
\pgfpathlineto{\pgfqpoint{1.339275in}{2.376135in}}%
\pgfpathlineto{\pgfqpoint{1.338598in}{2.376715in}}%
\pgfpathlineto{\pgfqpoint{1.339496in}{2.376218in}}%
\pgfpathlineto{\pgfqpoint{1.341626in}{2.376597in}}%
\pgfpathlineto{\pgfqpoint{1.341663in}{2.375621in}}%
\pgfpathlineto{\pgfqpoint{1.342684in}{2.355965in}}%
\pgfpathlineto{\pgfqpoint{1.342610in}{2.380064in}}%
\pgfpathlineto{\pgfqpoint{1.342856in}{2.372783in}}%
\pgfpathlineto{\pgfqpoint{1.343029in}{2.391352in}}%
\pgfpathlineto{\pgfqpoint{1.343287in}{2.360522in}}%
\pgfpathlineto{\pgfqpoint{1.344001in}{2.384240in}}%
\pgfpathlineto{\pgfqpoint{1.344875in}{2.334826in}}%
\pgfpathlineto{\pgfqpoint{1.344592in}{2.406281in}}%
\pgfpathlineto{\pgfqpoint{1.345096in}{2.392681in}}%
\pgfpathlineto{\pgfqpoint{1.345133in}{2.402692in}}%
\pgfpathlineto{\pgfqpoint{1.345281in}{2.350991in}}%
\pgfpathlineto{\pgfqpoint{1.345946in}{2.360245in}}%
\pgfpathlineto{\pgfqpoint{1.346943in}{2.288370in}}%
\pgfpathlineto{\pgfqpoint{1.346635in}{2.417327in}}%
\pgfpathlineto{\pgfqpoint{1.347066in}{2.343219in}}%
\pgfpathlineto{\pgfqpoint{1.347952in}{2.449006in}}%
\pgfpathlineto{\pgfqpoint{1.347558in}{2.292356in}}%
\pgfpathlineto{\pgfqpoint{1.348161in}{2.332392in}}%
\pgfpathlineto{\pgfqpoint{1.348173in}{2.329956in}}%
\pgfpathlineto{\pgfqpoint{1.348801in}{2.416415in}}%
\pgfpathlineto{\pgfqpoint{1.349343in}{2.448424in}}%
\pgfpathlineto{\pgfqpoint{1.349490in}{2.307617in}}%
\pgfpathlineto{\pgfqpoint{1.349884in}{2.416008in}}%
\pgfpathlineto{\pgfqpoint{1.350007in}{2.306149in}}%
\pgfpathlineto{\pgfqpoint{1.350327in}{2.431743in}}%
\pgfpathlineto{\pgfqpoint{1.351004in}{2.375969in}}%
\pgfpathlineto{\pgfqpoint{1.351829in}{2.461139in}}%
\pgfpathlineto{\pgfqpoint{1.351607in}{2.296531in}}%
\pgfpathlineto{\pgfqpoint{1.352099in}{2.382092in}}%
\pgfpathlineto{\pgfqpoint{1.352493in}{2.311576in}}%
\pgfpathlineto{\pgfqpoint{1.353072in}{2.438692in}}%
\pgfpathlineto{\pgfqpoint{1.353219in}{2.346147in}}%
\pgfpathlineto{\pgfqpoint{1.354007in}{2.452218in}}%
\pgfpathlineto{\pgfqpoint{1.353429in}{2.296060in}}%
\pgfpathlineto{\pgfqpoint{1.354327in}{2.365563in}}%
\pgfpathlineto{\pgfqpoint{1.355398in}{2.261686in}}%
\pgfpathlineto{\pgfqpoint{1.355016in}{2.488043in}}%
\pgfpathlineto{\pgfqpoint{1.355435in}{2.344332in}}%
\pgfpathlineto{\pgfqpoint{1.355964in}{2.471789in}}%
\pgfpathlineto{\pgfqpoint{1.356210in}{2.286277in}}%
\pgfpathlineto{\pgfqpoint{1.356543in}{2.338495in}}%
\pgfpathlineto{\pgfqpoint{1.356887in}{2.460608in}}%
\pgfpathlineto{\pgfqpoint{1.357170in}{2.278114in}}%
\pgfpathlineto{\pgfqpoint{1.357687in}{2.418029in}}%
\pgfpathlineto{\pgfqpoint{1.357835in}{2.429133in}}%
\pgfpathlineto{\pgfqpoint{1.358093in}{2.323410in}}%
\pgfpathlineto{\pgfqpoint{1.358438in}{2.349114in}}%
\pgfpathlineto{\pgfqpoint{1.359324in}{2.285571in}}%
\pgfpathlineto{\pgfqpoint{1.358832in}{2.468919in}}%
\pgfpathlineto{\pgfqpoint{1.359509in}{2.370998in}}%
\pgfpathlineto{\pgfqpoint{1.360789in}{2.465432in}}%
\pgfpathlineto{\pgfqpoint{1.360321in}{2.296598in}}%
\pgfpathlineto{\pgfqpoint{1.360826in}{2.436457in}}%
\pgfpathlineto{\pgfqpoint{1.360936in}{2.258505in}}%
\pgfpathlineto{\pgfqpoint{1.361527in}{2.448496in}}%
\pgfpathlineto{\pgfqpoint{1.361946in}{2.398538in}}%
\pgfpathlineto{\pgfqpoint{1.362081in}{2.338384in}}%
\pgfpathlineto{\pgfqpoint{1.362019in}{2.444124in}}%
\pgfpathlineto{\pgfqpoint{1.362155in}{2.363336in}}%
\pgfpathlineto{\pgfqpoint{1.363041in}{2.288228in}}%
\pgfpathlineto{\pgfqpoint{1.362647in}{2.452816in}}%
\pgfpathlineto{\pgfqpoint{1.363238in}{2.410874in}}%
\pgfpathlineto{\pgfqpoint{1.364124in}{2.282744in}}%
\pgfpathlineto{\pgfqpoint{1.364383in}{2.459434in}}%
\pgfpathlineto{\pgfqpoint{1.364506in}{2.499757in}}%
\pgfpathlineto{\pgfqpoint{1.364899in}{2.241875in}}%
\pgfpathlineto{\pgfqpoint{1.365367in}{2.369005in}}%
\pgfpathlineto{\pgfqpoint{1.365601in}{2.316853in}}%
\pgfpathlineto{\pgfqpoint{1.366241in}{2.461755in}}%
\pgfpathlineto{\pgfqpoint{1.366266in}{2.508130in}}%
\pgfpathlineto{\pgfqpoint{1.366635in}{2.278491in}}%
\pgfpathlineto{\pgfqpoint{1.367324in}{2.395781in}}%
\pgfpathlineto{\pgfqpoint{1.368136in}{2.489932in}}%
\pgfpathlineto{\pgfqpoint{1.368481in}{2.288815in}}%
\pgfpathlineto{\pgfqpoint{1.368875in}{2.447935in}}%
\pgfpathlineto{\pgfqpoint{1.368604in}{2.250883in}}%
\pgfpathlineto{\pgfqpoint{1.369613in}{2.308786in}}%
\pgfpathlineto{\pgfqpoint{1.370339in}{2.261504in}}%
\pgfpathlineto{\pgfqpoint{1.370007in}{2.554731in}}%
\pgfpathlineto{\pgfqpoint{1.370598in}{2.345377in}}%
\pgfpathlineto{\pgfqpoint{1.370795in}{2.484450in}}%
\pgfpathlineto{\pgfqpoint{1.371472in}{2.292263in}}%
\pgfpathlineto{\pgfqpoint{1.371693in}{2.358866in}}%
\pgfpathlineto{\pgfqpoint{1.372333in}{2.255213in}}%
\pgfpathlineto{\pgfqpoint{1.371878in}{2.516159in}}%
\pgfpathlineto{\pgfqpoint{1.372789in}{2.408839in}}%
\pgfpathlineto{\pgfqpoint{1.372949in}{2.277810in}}%
\pgfpathlineto{\pgfqpoint{1.373158in}{2.468505in}}%
\pgfpathlineto{\pgfqpoint{1.373749in}{2.456411in}}%
\pgfpathlineto{\pgfqpoint{1.373773in}{2.485722in}}%
\pgfpathlineto{\pgfqpoint{1.374192in}{2.247271in}}%
\pgfpathlineto{\pgfqpoint{1.374770in}{2.388932in}}%
\pgfpathlineto{\pgfqpoint{1.375582in}{2.494974in}}%
\pgfpathlineto{\pgfqpoint{1.375915in}{2.267096in}}%
\pgfpathlineto{\pgfqpoint{1.376284in}{2.464317in}}%
\pgfpathlineto{\pgfqpoint{1.376050in}{2.223982in}}%
\pgfpathlineto{\pgfqpoint{1.377096in}{2.366872in}}%
\pgfpathlineto{\pgfqpoint{1.377909in}{2.279673in}}%
\pgfpathlineto{\pgfqpoint{1.377490in}{2.474301in}}%
\pgfpathlineto{\pgfqpoint{1.378179in}{2.391705in}}%
\pgfpathlineto{\pgfqpoint{1.379324in}{2.537217in}}%
\pgfpathlineto{\pgfqpoint{1.379029in}{2.262656in}}%
\pgfpathlineto{\pgfqpoint{1.379361in}{2.476916in}}%
\pgfpathlineto{\pgfqpoint{1.379779in}{2.245377in}}%
\pgfpathlineto{\pgfqpoint{1.380099in}{2.483654in}}%
\pgfpathlineto{\pgfqpoint{1.380604in}{2.353203in}}%
\pgfpathlineto{\pgfqpoint{1.381170in}{2.500258in}}%
\pgfpathlineto{\pgfqpoint{1.380764in}{2.250542in}}%
\pgfpathlineto{\pgfqpoint{1.381712in}{2.353567in}}%
\pgfpathlineto{\pgfqpoint{1.381946in}{2.465392in}}%
\pgfpathlineto{\pgfqpoint{1.382610in}{2.303531in}}%
\pgfpathlineto{\pgfqpoint{1.383373in}{2.259375in}}%
\pgfpathlineto{\pgfqpoint{1.383066in}{2.482859in}}%
\pgfpathlineto{\pgfqpoint{1.383632in}{2.398366in}}%
\pgfpathlineto{\pgfqpoint{1.384038in}{2.454017in}}%
\pgfpathlineto{\pgfqpoint{1.384481in}{2.270125in}}%
\pgfpathlineto{\pgfqpoint{1.384690in}{2.345595in}}%
\pgfpathlineto{\pgfqpoint{1.385219in}{2.251688in}}%
\pgfpathlineto{\pgfqpoint{1.384875in}{2.516263in}}%
\pgfpathlineto{\pgfqpoint{1.385773in}{2.380212in}}%
\pgfpathlineto{\pgfqpoint{1.386561in}{2.257812in}}%
\pgfpathlineto{\pgfqpoint{1.385909in}{2.428965in}}%
\pgfpathlineto{\pgfqpoint{1.386672in}{2.411685in}}%
\pgfpathlineto{\pgfqpoint{1.386758in}{2.497976in}}%
\pgfpathlineto{\pgfqpoint{1.387078in}{2.234227in}}%
\pgfpathlineto{\pgfqpoint{1.387767in}{2.379750in}}%
\pgfpathlineto{\pgfqpoint{1.388924in}{2.225194in}}%
\pgfpathlineto{\pgfqpoint{1.388592in}{2.503296in}}%
\pgfpathlineto{\pgfqpoint{1.388949in}{2.260262in}}%
\pgfpathlineto{\pgfqpoint{1.389490in}{2.484823in}}%
\pgfpathlineto{\pgfqpoint{1.390118in}{2.361560in}}%
\pgfpathlineto{\pgfqpoint{1.390807in}{2.262452in}}%
\pgfpathlineto{\pgfqpoint{1.391102in}{2.477281in}}%
\pgfpathlineto{\pgfqpoint{1.392025in}{2.231568in}}%
\pgfpathlineto{\pgfqpoint{1.392284in}{2.414043in}}%
\pgfpathlineto{\pgfqpoint{1.392345in}{2.494852in}}%
\pgfpathlineto{\pgfqpoint{1.392764in}{2.257266in}}%
\pgfpathlineto{\pgfqpoint{1.393256in}{2.363215in}}%
\pgfpathlineto{\pgfqpoint{1.393453in}{2.421999in}}%
\pgfpathlineto{\pgfqpoint{1.393662in}{2.328363in}}%
\pgfpathlineto{\pgfqpoint{1.393798in}{2.338400in}}%
\pgfpathlineto{\pgfqpoint{1.394635in}{2.250279in}}%
\pgfpathlineto{\pgfqpoint{1.394167in}{2.520866in}}%
\pgfpathlineto{\pgfqpoint{1.394672in}{2.368414in}}%
\pgfpathlineto{\pgfqpoint{1.394942in}{2.455977in}}%
\pgfpathlineto{\pgfqpoint{1.395127in}{2.294383in}}%
\pgfpathlineto{\pgfqpoint{1.395755in}{2.301895in}}%
\pgfpathlineto{\pgfqpoint{1.396062in}{2.489246in}}%
\pgfpathlineto{\pgfqpoint{1.395878in}{2.274982in}}%
\pgfpathlineto{\pgfqpoint{1.396936in}{2.409158in}}%
\pgfpathlineto{\pgfqpoint{1.397736in}{2.282579in}}%
\pgfpathlineto{\pgfqpoint{1.397281in}{2.520828in}}%
\pgfpathlineto{\pgfqpoint{1.398019in}{2.414086in}}%
\pgfpathlineto{\pgfqpoint{1.398056in}{2.469751in}}%
\pgfpathlineto{\pgfqpoint{1.398352in}{2.281709in}}%
\pgfpathlineto{\pgfqpoint{1.399139in}{2.440997in}}%
\pgfpathlineto{\pgfqpoint{1.399176in}{2.468048in}}%
\pgfpathlineto{\pgfqpoint{1.399324in}{2.290671in}}%
\pgfpathlineto{\pgfqpoint{1.400087in}{2.297225in}}%
\pgfpathlineto{\pgfqpoint{1.400210in}{2.279527in}}%
\pgfpathlineto{\pgfqpoint{1.400370in}{2.421369in}}%
\pgfpathlineto{\pgfqpoint{1.400407in}{2.468803in}}%
\pgfpathlineto{\pgfqpoint{1.401342in}{2.277911in}}%
\pgfpathlineto{\pgfqpoint{1.401429in}{2.291599in}}%
\pgfpathlineto{\pgfqpoint{1.401589in}{2.450984in}}%
\pgfpathlineto{\pgfqpoint{1.402659in}{2.348524in}}%
\pgfpathlineto{\pgfqpoint{1.403189in}{2.272041in}}%
\pgfpathlineto{\pgfqpoint{1.402844in}{2.486261in}}%
\pgfpathlineto{\pgfqpoint{1.403779in}{2.320293in}}%
\pgfpathlineto{\pgfqpoint{1.404542in}{2.292328in}}%
\pgfpathlineto{\pgfqpoint{1.404715in}{2.477250in}}%
\pgfpathlineto{\pgfqpoint{1.404813in}{2.368529in}}%
\pgfpathlineto{\pgfqpoint{1.405035in}{2.263856in}}%
\pgfpathlineto{\pgfqpoint{1.405945in}{2.468314in}}%
\pgfpathlineto{\pgfqpoint{1.406881in}{2.249797in}}%
\pgfpathlineto{\pgfqpoint{1.406721in}{2.476658in}}%
\pgfpathlineto{\pgfqpoint{1.407152in}{2.415193in}}%
\pgfpathlineto{\pgfqpoint{1.407176in}{2.453418in}}%
\pgfpathlineto{\pgfqpoint{1.408149in}{2.284435in}}%
\pgfpathlineto{\pgfqpoint{1.408235in}{2.366257in}}%
\pgfpathlineto{\pgfqpoint{1.408764in}{2.295563in}}%
\pgfpathlineto{\pgfqpoint{1.409059in}{2.494137in}}%
\pgfpathlineto{\pgfqpoint{1.409318in}{2.364030in}}%
\pgfpathlineto{\pgfqpoint{1.410302in}{2.484100in}}%
\pgfpathlineto{\pgfqpoint{1.409995in}{2.258750in}}%
\pgfpathlineto{\pgfqpoint{1.410450in}{2.427727in}}%
\pgfpathlineto{\pgfqpoint{1.410745in}{2.279003in}}%
\pgfpathlineto{\pgfqpoint{1.411422in}{2.458521in}}%
\pgfpathlineto{\pgfqpoint{1.411582in}{2.385859in}}%
\pgfpathlineto{\pgfqpoint{1.412173in}{2.486676in}}%
\pgfpathlineto{\pgfqpoint{1.412481in}{2.298720in}}%
\pgfpathlineto{\pgfqpoint{1.412555in}{2.334936in}}%
\pgfpathlineto{\pgfqpoint{1.412604in}{2.274584in}}%
\pgfpathlineto{\pgfqpoint{1.412899in}{2.441258in}}%
\pgfpathlineto{\pgfqpoint{1.413638in}{2.362111in}}%
\pgfpathlineto{\pgfqpoint{1.414019in}{2.447320in}}%
\pgfpathlineto{\pgfqpoint{1.414339in}{2.286959in}}%
\pgfpathlineto{\pgfqpoint{1.414745in}{2.369225in}}%
\pgfpathlineto{\pgfqpoint{1.414955in}{2.286945in}}%
\pgfpathlineto{\pgfqpoint{1.415238in}{2.460936in}}%
\pgfpathlineto{\pgfqpoint{1.415755in}{2.377536in}}%
\pgfpathlineto{\pgfqpoint{1.415878in}{2.442619in}}%
\pgfpathlineto{\pgfqpoint{1.416210in}{2.265808in}}%
\pgfpathlineto{\pgfqpoint{1.416850in}{2.348065in}}%
\pgfpathlineto{\pgfqpoint{1.417650in}{2.430252in}}%
\pgfpathlineto{\pgfqpoint{1.417453in}{2.297560in}}%
\pgfpathlineto{\pgfqpoint{1.417982in}{2.390261in}}%
\pgfpathlineto{\pgfqpoint{1.418068in}{2.253418in}}%
\pgfpathlineto{\pgfqpoint{1.418364in}{2.454514in}}%
\pgfpathlineto{\pgfqpoint{1.419090in}{2.385014in}}%
\pgfpathlineto{\pgfqpoint{1.419508in}{2.439231in}}%
\pgfpathlineto{\pgfqpoint{1.419324in}{2.290316in}}%
\pgfpathlineto{\pgfqpoint{1.420235in}{2.423359in}}%
\pgfpathlineto{\pgfqpoint{1.421182in}{2.277228in}}%
\pgfpathlineto{\pgfqpoint{1.420813in}{2.457125in}}%
\pgfpathlineto{\pgfqpoint{1.421392in}{2.396869in}}%
\pgfpathlineto{\pgfqpoint{1.421810in}{2.306044in}}%
\pgfpathlineto{\pgfqpoint{1.421490in}{2.457728in}}%
\pgfpathlineto{\pgfqpoint{1.422585in}{2.370289in}}%
\pgfpathlineto{\pgfqpoint{1.422622in}{2.453723in}}%
\pgfpathlineto{\pgfqpoint{1.423028in}{2.318265in}}%
\pgfpathlineto{\pgfqpoint{1.423681in}{2.333166in}}%
\pgfpathlineto{\pgfqpoint{1.423952in}{2.459072in}}%
\pgfpathlineto{\pgfqpoint{1.423804in}{2.312499in}}%
\pgfpathlineto{\pgfqpoint{1.424813in}{2.356384in}}%
\pgfpathlineto{\pgfqpoint{1.425638in}{2.289907in}}%
\pgfpathlineto{\pgfqpoint{1.425158in}{2.434530in}}%
\pgfpathlineto{\pgfqpoint{1.425908in}{2.369551in}}%
\pgfpathlineto{\pgfqpoint{1.426856in}{2.320841in}}%
\pgfpathlineto{\pgfqpoint{1.427053in}{2.470463in}}%
\pgfpathlineto{\pgfqpoint{1.427065in}{2.483320in}}%
\pgfpathlineto{\pgfqpoint{1.427976in}{2.292103in}}%
\pgfpathlineto{\pgfqpoint{1.428075in}{2.379289in}}%
\pgfpathlineto{\pgfqpoint{1.428764in}{2.291029in}}%
\pgfpathlineto{\pgfqpoint{1.428308in}{2.454000in}}%
\pgfpathlineto{\pgfqpoint{1.429219in}{2.323000in}}%
\pgfpathlineto{\pgfqpoint{1.430179in}{2.456664in}}%
\pgfpathlineto{\pgfqpoint{1.430007in}{2.313858in}}%
\pgfpathlineto{\pgfqpoint{1.430376in}{2.379893in}}%
\pgfpathlineto{\pgfqpoint{1.431102in}{2.319726in}}%
\pgfpathlineto{\pgfqpoint{1.430905in}{2.434127in}}%
\pgfpathlineto{\pgfqpoint{1.431398in}{2.421943in}}%
\pgfpathlineto{\pgfqpoint{1.431422in}{2.438041in}}%
\pgfpathlineto{\pgfqpoint{1.431890in}{2.299237in}}%
\pgfpathlineto{\pgfqpoint{1.432431in}{2.359732in}}%
\pgfpathlineto{\pgfqpoint{1.433133in}{2.316543in}}%
\pgfpathlineto{\pgfqpoint{1.433256in}{2.434309in}}%
\pgfpathlineto{\pgfqpoint{1.433564in}{2.324672in}}%
\pgfpathlineto{\pgfqpoint{1.433884in}{2.431740in}}%
\pgfpathlineto{\pgfqpoint{1.434167in}{2.320895in}}%
\pgfpathlineto{\pgfqpoint{1.434696in}{2.371397in}}%
\pgfpathlineto{\pgfqpoint{1.434991in}{2.308345in}}%
\pgfpathlineto{\pgfqpoint{1.435275in}{2.412847in}}%
\pgfpathlineto{\pgfqpoint{1.435656in}{2.383913in}}%
\pgfpathlineto{\pgfqpoint{1.435681in}{2.440892in}}%
\pgfpathlineto{\pgfqpoint{1.436025in}{2.331901in}}%
\pgfpathlineto{\pgfqpoint{1.436751in}{2.374472in}}%
\pgfpathlineto{\pgfqpoint{1.437478in}{2.311748in}}%
\pgfpathlineto{\pgfqpoint{1.437601in}{2.435086in}}%
\pgfpathlineto{\pgfqpoint{1.437773in}{2.400282in}}%
\pgfpathlineto{\pgfqpoint{1.438105in}{2.319739in}}%
\pgfpathlineto{\pgfqpoint{1.438228in}{2.425950in}}%
\pgfpathlineto{\pgfqpoint{1.438795in}{2.402509in}}%
\pgfpathlineto{\pgfqpoint{1.438831in}{2.426718in}}%
\pgfpathlineto{\pgfqpoint{1.439755in}{2.302355in}}%
\pgfpathlineto{\pgfqpoint{1.439878in}{2.371813in}}%
\pgfpathlineto{\pgfqpoint{1.440604in}{2.311883in}}%
\pgfpathlineto{\pgfqpoint{1.440062in}{2.435565in}}%
\pgfpathlineto{\pgfqpoint{1.440887in}{2.392668in}}%
\pgfpathlineto{\pgfqpoint{1.441958in}{2.448124in}}%
\pgfpathlineto{\pgfqpoint{1.441835in}{2.317780in}}%
\pgfpathlineto{\pgfqpoint{1.441995in}{2.395170in}}%
\pgfpathlineto{\pgfqpoint{1.442868in}{2.321620in}}%
\pgfpathlineto{\pgfqpoint{1.442733in}{2.422300in}}%
\pgfpathlineto{\pgfqpoint{1.443115in}{2.359247in}}%
\pgfpathlineto{\pgfqpoint{1.443188in}{2.440980in}}%
\pgfpathlineto{\pgfqpoint{1.443681in}{2.318454in}}%
\pgfpathlineto{\pgfqpoint{1.444235in}{2.377867in}}%
\pgfpathlineto{\pgfqpoint{1.444961in}{2.328575in}}%
\pgfpathlineto{\pgfqpoint{1.445071in}{2.433833in}}%
\pgfpathlineto{\pgfqpoint{1.445379in}{2.353952in}}%
\pgfpathlineto{\pgfqpoint{1.445638in}{2.423429in}}%
\pgfpathlineto{\pgfqpoint{1.445576in}{2.331772in}}%
\pgfpathlineto{\pgfqpoint{1.446511in}{2.390423in}}%
\pgfpathlineto{\pgfqpoint{1.446807in}{2.309040in}}%
\pgfpathlineto{\pgfqpoint{1.446930in}{2.434509in}}%
\pgfpathlineto{\pgfqpoint{1.447619in}{2.390569in}}%
\pgfpathlineto{\pgfqpoint{1.448038in}{2.323319in}}%
\pgfpathlineto{\pgfqpoint{1.447705in}{2.417972in}}%
\pgfpathlineto{\pgfqpoint{1.448727in}{2.383026in}}%
\pgfpathlineto{\pgfqpoint{1.448764in}{2.438861in}}%
\pgfpathlineto{\pgfqpoint{1.449084in}{2.327753in}}%
\pgfpathlineto{\pgfqpoint{1.449835in}{2.377141in}}%
\pgfpathlineto{\pgfqpoint{1.449933in}{2.294118in}}%
\pgfpathlineto{\pgfqpoint{1.450413in}{2.413310in}}%
\pgfpathlineto{\pgfqpoint{1.450967in}{2.349534in}}%
\pgfpathlineto{\pgfqpoint{1.451631in}{2.424719in}}%
\pgfpathlineto{\pgfqpoint{1.451164in}{2.300696in}}%
\pgfpathlineto{\pgfqpoint{1.452148in}{2.410790in}}%
\pgfpathlineto{\pgfqpoint{1.453047in}{2.299469in}}%
\pgfpathlineto{\pgfqpoint{1.452481in}{2.425123in}}%
\pgfpathlineto{\pgfqpoint{1.453256in}{2.399977in}}%
\pgfpathlineto{\pgfqpoint{1.453724in}{2.435475in}}%
\pgfpathlineto{\pgfqpoint{1.453859in}{2.318383in}}%
\pgfpathlineto{\pgfqpoint{1.454253in}{2.372419in}}%
\pgfpathlineto{\pgfqpoint{1.454905in}{2.302892in}}%
\pgfpathlineto{\pgfqpoint{1.454967in}{2.422853in}}%
\pgfpathlineto{\pgfqpoint{1.455348in}{2.394216in}}%
\pgfpathlineto{\pgfqpoint{1.456001in}{2.418316in}}%
\pgfpathlineto{\pgfqpoint{1.456148in}{2.319711in}}%
\pgfpathlineto{\pgfqpoint{1.456444in}{2.388575in}}%
\pgfpathlineto{\pgfqpoint{1.456764in}{2.304986in}}%
\pgfpathlineto{\pgfqpoint{1.456862in}{2.424961in}}%
\pgfpathlineto{\pgfqpoint{1.457564in}{2.368145in}}%
\pgfpathlineto{\pgfqpoint{1.457847in}{2.429401in}}%
\pgfpathlineto{\pgfqpoint{1.457785in}{2.304931in}}%
\pgfpathlineto{\pgfqpoint{1.458610in}{2.335561in}}%
\pgfpathlineto{\pgfqpoint{1.459262in}{2.319666in}}%
\pgfpathlineto{\pgfqpoint{1.458905in}{2.417699in}}%
\pgfpathlineto{\pgfqpoint{1.459484in}{2.381981in}}%
\pgfpathlineto{\pgfqpoint{1.459730in}{2.434629in}}%
\pgfpathlineto{\pgfqpoint{1.459890in}{2.309760in}}%
\pgfpathlineto{\pgfqpoint{1.460481in}{2.339167in}}%
\pgfpathlineto{\pgfqpoint{1.460493in}{2.320276in}}%
\pgfpathlineto{\pgfqpoint{1.460973in}{2.431973in}}%
\pgfpathlineto{\pgfqpoint{1.461539in}{2.377959in}}%
\pgfpathlineto{\pgfqpoint{1.461576in}{2.433008in}}%
\pgfpathlineto{\pgfqpoint{1.461724in}{2.311492in}}%
\pgfpathlineto{\pgfqpoint{1.462671in}{2.418244in}}%
\pgfpathlineto{\pgfqpoint{1.462967in}{2.310898in}}%
\pgfpathlineto{\pgfqpoint{1.462819in}{2.431741in}}%
\pgfpathlineto{\pgfqpoint{1.463828in}{2.374144in}}%
\pgfpathlineto{\pgfqpoint{1.464702in}{2.429031in}}%
\pgfpathlineto{\pgfqpoint{1.464825in}{2.306373in}}%
\pgfpathlineto{\pgfqpoint{1.464850in}{2.282253in}}%
\pgfpathlineto{\pgfqpoint{1.464911in}{2.417859in}}%
\pgfpathlineto{\pgfqpoint{1.465896in}{2.372353in}}%
\pgfpathlineto{\pgfqpoint{1.466278in}{2.426941in}}%
\pgfpathlineto{\pgfqpoint{1.466081in}{2.279030in}}%
\pgfpathlineto{\pgfqpoint{1.467016in}{2.394144in}}%
\pgfpathlineto{\pgfqpoint{1.467951in}{2.285001in}}%
\pgfpathlineto{\pgfqpoint{1.467521in}{2.418727in}}%
\pgfpathlineto{\pgfqpoint{1.468099in}{2.390014in}}%
\pgfpathlineto{\pgfqpoint{1.468136in}{2.419381in}}%
\pgfpathlineto{\pgfqpoint{1.468579in}{2.329459in}}%
\pgfpathlineto{\pgfqpoint{1.469158in}{2.361268in}}%
\pgfpathlineto{\pgfqpoint{1.469810in}{2.290704in}}%
\pgfpathlineto{\pgfqpoint{1.470118in}{2.417307in}}%
\pgfpathlineto{\pgfqpoint{1.470253in}{2.378244in}}%
\pgfpathlineto{\pgfqpoint{1.470265in}{2.378731in}}%
\pgfpathlineto{\pgfqpoint{1.470314in}{2.351030in}}%
\pgfpathlineto{\pgfqpoint{1.471053in}{2.304241in}}%
\pgfpathlineto{\pgfqpoint{1.471176in}{2.416511in}}%
\pgfpathlineto{\pgfqpoint{1.471398in}{2.364945in}}%
\pgfpathlineto{\pgfqpoint{1.471964in}{2.442126in}}%
\pgfpathlineto{\pgfqpoint{1.471668in}{2.308096in}}%
\pgfpathlineto{\pgfqpoint{1.472567in}{2.413955in}}%
\pgfpathlineto{\pgfqpoint{1.472924in}{2.322516in}}%
\pgfpathlineto{\pgfqpoint{1.472751in}{2.417510in}}%
\pgfpathlineto{\pgfqpoint{1.473736in}{2.382943in}}%
\pgfpathlineto{\pgfqpoint{1.473810in}{2.419914in}}%
\pgfpathlineto{\pgfqpoint{1.474142in}{2.337889in}}%
\pgfpathlineto{\pgfqpoint{1.474167in}{2.305400in}}%
\pgfpathlineto{\pgfqpoint{1.474905in}{2.419890in}}%
\pgfpathlineto{\pgfqpoint{1.475225in}{2.389736in}}%
\pgfpathlineto{\pgfqpoint{1.475237in}{2.389789in}}%
\pgfpathlineto{\pgfqpoint{1.475397in}{2.311247in}}%
\pgfpathlineto{\pgfqpoint{1.475681in}{2.431233in}}%
\pgfpathlineto{\pgfqpoint{1.476370in}{2.368459in}}%
\pgfpathlineto{\pgfqpoint{1.477453in}{2.429245in}}%
\pgfpathlineto{\pgfqpoint{1.476628in}{2.312922in}}%
\pgfpathlineto{\pgfqpoint{1.477490in}{2.392259in}}%
\pgfpathlineto{\pgfqpoint{1.477859in}{2.293601in}}%
\pgfpathlineto{\pgfqpoint{1.477551in}{2.418473in}}%
\pgfpathlineto{\pgfqpoint{1.478622in}{2.361537in}}%
\pgfpathlineto{\pgfqpoint{1.479287in}{2.417502in}}%
\pgfpathlineto{\pgfqpoint{1.479127in}{2.318852in}}%
\pgfpathlineto{\pgfqpoint{1.479681in}{2.349685in}}%
\pgfpathlineto{\pgfqpoint{1.479742in}{2.294316in}}%
\pgfpathlineto{\pgfqpoint{1.479816in}{2.419076in}}%
\pgfpathlineto{\pgfqpoint{1.480776in}{2.353044in}}%
\pgfpathlineto{\pgfqpoint{1.481379in}{2.429585in}}%
\pgfpathlineto{\pgfqpoint{1.480985in}{2.278689in}}%
\pgfpathlineto{\pgfqpoint{1.481970in}{2.394201in}}%
\pgfpathlineto{\pgfqpoint{1.482856in}{2.297766in}}%
\pgfpathlineto{\pgfqpoint{1.482154in}{2.425075in}}%
\pgfpathlineto{\pgfqpoint{1.483090in}{2.371974in}}%
\pgfpathlineto{\pgfqpoint{1.483742in}{2.425776in}}%
\pgfpathlineto{\pgfqpoint{1.484074in}{2.338379in}}%
\pgfpathlineto{\pgfqpoint{1.484714in}{2.299300in}}%
\pgfpathlineto{\pgfqpoint{1.484247in}{2.428482in}}%
\pgfpathlineto{\pgfqpoint{1.485170in}{2.357207in}}%
\pgfpathlineto{\pgfqpoint{1.486093in}{2.445850in}}%
\pgfpathlineto{\pgfqpoint{1.485957in}{2.309665in}}%
\pgfpathlineto{\pgfqpoint{1.486277in}{2.362879in}}%
\pgfpathlineto{\pgfqpoint{1.486573in}{2.322208in}}%
\pgfpathlineto{\pgfqpoint{1.486856in}{2.439074in}}%
\pgfpathlineto{\pgfqpoint{1.487324in}{2.401569in}}%
\pgfpathlineto{\pgfqpoint{1.488099in}{2.417360in}}%
\pgfpathlineto{\pgfqpoint{1.487816in}{2.325074in}}%
\pgfpathlineto{\pgfqpoint{1.488259in}{2.369287in}}%
\pgfpathlineto{\pgfqpoint{1.489071in}{2.317100in}}%
\pgfpathlineto{\pgfqpoint{1.489219in}{2.434644in}}%
\pgfpathlineto{\pgfqpoint{1.489317in}{2.393687in}}%
\pgfpathlineto{\pgfqpoint{1.489822in}{2.426670in}}%
\pgfpathlineto{\pgfqpoint{1.489687in}{2.338671in}}%
\pgfpathlineto{\pgfqpoint{1.490277in}{2.344325in}}%
\pgfpathlineto{\pgfqpoint{1.490314in}{2.324514in}}%
\pgfpathlineto{\pgfqpoint{1.491065in}{2.420380in}}%
\pgfpathlineto{\pgfqpoint{1.491274in}{2.376781in}}%
\pgfpathlineto{\pgfqpoint{1.491681in}{2.411614in}}%
\pgfpathlineto{\pgfqpoint{1.491521in}{2.333630in}}%
\pgfpathlineto{\pgfqpoint{1.492382in}{2.389161in}}%
\pgfpathlineto{\pgfqpoint{1.492764in}{2.311878in}}%
\pgfpathlineto{\pgfqpoint{1.492444in}{2.414659in}}%
\pgfpathlineto{\pgfqpoint{1.493490in}{2.391244in}}%
\pgfpathlineto{\pgfqpoint{1.494044in}{2.319278in}}%
\pgfpathlineto{\pgfqpoint{1.493687in}{2.418196in}}%
\pgfpathlineto{\pgfqpoint{1.494671in}{2.332091in}}%
\pgfpathlineto{\pgfqpoint{1.494720in}{2.419500in}}%
\pgfpathlineto{\pgfqpoint{1.495102in}{2.331954in}}%
\pgfpathlineto{\pgfqpoint{1.495816in}{2.386858in}}%
\pgfpathlineto{\pgfqpoint{1.495902in}{2.294592in}}%
\pgfpathlineto{\pgfqpoint{1.496284in}{2.420864in}}%
\pgfpathlineto{\pgfqpoint{1.496973in}{2.361823in}}%
\pgfpathlineto{\pgfqpoint{1.497404in}{2.409912in}}%
\pgfpathlineto{\pgfqpoint{1.497736in}{2.322458in}}%
\pgfpathlineto{\pgfqpoint{1.498080in}{2.376660in}}%
\pgfpathlineto{\pgfqpoint{1.499016in}{2.320495in}}%
\pgfpathlineto{\pgfqpoint{1.498659in}{2.431418in}}%
\pgfpathlineto{\pgfqpoint{1.499139in}{2.401897in}}%
\pgfpathlineto{\pgfqpoint{1.499890in}{2.412363in}}%
\pgfpathlineto{\pgfqpoint{1.499631in}{2.317153in}}%
\pgfpathlineto{\pgfqpoint{1.500173in}{2.375553in}}%
\pgfpathlineto{\pgfqpoint{1.500862in}{2.314160in}}%
\pgfpathlineto{\pgfqpoint{1.501010in}{2.442207in}}%
\pgfpathlineto{\pgfqpoint{1.501268in}{2.386169in}}%
\pgfpathlineto{\pgfqpoint{1.501490in}{2.319770in}}%
\pgfpathlineto{\pgfqpoint{1.501785in}{2.431505in}}%
\pgfpathlineto{\pgfqpoint{1.502216in}{2.397299in}}%
\pgfpathlineto{\pgfqpoint{1.502856in}{2.420710in}}%
\pgfpathlineto{\pgfqpoint{1.502696in}{2.329076in}}%
\pgfpathlineto{\pgfqpoint{1.503287in}{2.361745in}}%
\pgfpathlineto{\pgfqpoint{1.503964in}{2.329475in}}%
\pgfpathlineto{\pgfqpoint{1.504136in}{2.434488in}}%
\pgfpathlineto{\pgfqpoint{1.504382in}{2.364690in}}%
\pgfpathlineto{\pgfqpoint{1.504739in}{2.426901in}}%
\pgfpathlineto{\pgfqpoint{1.504542in}{2.329951in}}%
\pgfpathlineto{\pgfqpoint{1.505527in}{2.390221in}}%
\pgfpathlineto{\pgfqpoint{1.506437in}{2.311586in}}%
\pgfpathlineto{\pgfqpoint{1.505982in}{2.428283in}}%
\pgfpathlineto{\pgfqpoint{1.506610in}{2.415057in}}%
\pgfpathlineto{\pgfqpoint{1.507668in}{2.299858in}}%
\pgfpathlineto{\pgfqpoint{1.507767in}{2.379440in}}%
\pgfpathlineto{\pgfqpoint{1.507853in}{2.430297in}}%
\pgfpathlineto{\pgfqpoint{1.508284in}{2.317597in}}%
\pgfpathlineto{\pgfqpoint{1.508862in}{2.387088in}}%
\pgfpathlineto{\pgfqpoint{1.509539in}{2.316877in}}%
\pgfpathlineto{\pgfqpoint{1.509699in}{2.423515in}}%
\pgfpathlineto{\pgfqpoint{1.509982in}{2.350314in}}%
\pgfpathlineto{\pgfqpoint{1.510782in}{2.305536in}}%
\pgfpathlineto{\pgfqpoint{1.510930in}{2.421593in}}%
\pgfpathlineto{\pgfqpoint{1.511053in}{2.386465in}}%
\pgfpathlineto{\pgfqpoint{1.511693in}{2.418822in}}%
\pgfpathlineto{\pgfqpoint{1.511397in}{2.306964in}}%
\pgfpathlineto{\pgfqpoint{1.512185in}{2.415042in}}%
\pgfpathlineto{\pgfqpoint{1.512628in}{2.314139in}}%
\pgfpathlineto{\pgfqpoint{1.512567in}{2.424727in}}%
\pgfpathlineto{\pgfqpoint{1.513354in}{2.393999in}}%
\pgfpathlineto{\pgfqpoint{1.513551in}{2.430814in}}%
\pgfpathlineto{\pgfqpoint{1.513736in}{2.329683in}}%
\pgfpathlineto{\pgfqpoint{1.514240in}{2.356201in}}%
\pgfpathlineto{\pgfqpoint{1.514363in}{2.347818in}}%
\pgfpathlineto{\pgfqpoint{1.514413in}{2.395271in}}%
\pgfpathlineto{\pgfqpoint{1.514671in}{2.428053in}}%
\pgfpathlineto{\pgfqpoint{1.514499in}{2.319071in}}%
\pgfpathlineto{\pgfqpoint{1.515459in}{2.367384in}}%
\pgfpathlineto{\pgfqpoint{1.515742in}{2.318993in}}%
\pgfpathlineto{\pgfqpoint{1.515902in}{2.444798in}}%
\pgfpathlineto{\pgfqpoint{1.516505in}{2.425381in}}%
\pgfpathlineto{\pgfqpoint{1.516985in}{2.325920in}}%
\pgfpathlineto{\pgfqpoint{1.517268in}{2.430284in}}%
\pgfpathlineto{\pgfqpoint{1.517699in}{2.377319in}}%
\pgfpathlineto{\pgfqpoint{1.517773in}{2.431780in}}%
\pgfpathlineto{\pgfqpoint{1.518203in}{2.320102in}}%
\pgfpathlineto{\pgfqpoint{1.518794in}{2.372058in}}%
\pgfpathlineto{\pgfqpoint{1.519447in}{2.332995in}}%
\pgfpathlineto{\pgfqpoint{1.519631in}{2.445501in}}%
\pgfpathlineto{\pgfqpoint{1.519890in}{2.385975in}}%
\pgfpathlineto{\pgfqpoint{1.520702in}{2.319996in}}%
\pgfpathlineto{\pgfqpoint{1.520259in}{2.431704in}}%
\pgfpathlineto{\pgfqpoint{1.520997in}{2.387224in}}%
\pgfpathlineto{\pgfqpoint{1.521502in}{2.421376in}}%
\pgfpathlineto{\pgfqpoint{1.521330in}{2.311101in}}%
\pgfpathlineto{\pgfqpoint{1.521920in}{2.340888in}}%
\pgfpathlineto{\pgfqpoint{1.522560in}{2.310521in}}%
\pgfpathlineto{\pgfqpoint{1.522745in}{2.434592in}}%
\pgfpathlineto{\pgfqpoint{1.523003in}{2.376816in}}%
\pgfpathlineto{\pgfqpoint{1.523176in}{2.324844in}}%
\pgfpathlineto{\pgfqpoint{1.523730in}{2.429680in}}%
\pgfpathlineto{\pgfqpoint{1.524087in}{2.394543in}}%
\pgfpathlineto{\pgfqpoint{1.524603in}{2.431597in}}%
\pgfpathlineto{\pgfqpoint{1.524431in}{2.313279in}}%
\pgfpathlineto{\pgfqpoint{1.525145in}{2.381643in}}%
\pgfpathlineto{\pgfqpoint{1.526290in}{2.309166in}}%
\pgfpathlineto{\pgfqpoint{1.525834in}{2.420181in}}%
\pgfpathlineto{\pgfqpoint{1.526302in}{2.321159in}}%
\pgfpathlineto{\pgfqpoint{1.527459in}{2.424826in}}%
\pgfpathlineto{\pgfqpoint{1.527533in}{2.316490in}}%
\pgfpathlineto{\pgfqpoint{1.528665in}{2.358450in}}%
\pgfpathlineto{\pgfqpoint{1.528948in}{2.429903in}}%
\pgfpathlineto{\pgfqpoint{1.529391in}{2.323084in}}%
\pgfpathlineto{\pgfqpoint{1.529773in}{2.357358in}}%
\pgfpathlineto{\pgfqpoint{1.530794in}{2.436804in}}%
\pgfpathlineto{\pgfqpoint{1.530634in}{2.332604in}}%
\pgfpathlineto{\pgfqpoint{1.530917in}{2.399848in}}%
\pgfpathlineto{\pgfqpoint{1.531410in}{2.419253in}}%
\pgfpathlineto{\pgfqpoint{1.531250in}{2.324329in}}%
\pgfpathlineto{\pgfqpoint{1.531840in}{2.364269in}}%
\pgfpathlineto{\pgfqpoint{1.531865in}{2.326938in}}%
\pgfpathlineto{\pgfqpoint{1.532665in}{2.418710in}}%
\pgfpathlineto{\pgfqpoint{1.532936in}{2.385984in}}%
\pgfpathlineto{\pgfqpoint{1.533108in}{2.327431in}}%
\pgfpathlineto{\pgfqpoint{1.533280in}{2.418474in}}%
\pgfpathlineto{\pgfqpoint{1.533883in}{2.399056in}}%
\pgfpathlineto{\pgfqpoint{1.534523in}{2.437710in}}%
\pgfpathlineto{\pgfqpoint{1.534351in}{2.335613in}}%
\pgfpathlineto{\pgfqpoint{1.534954in}{2.345817in}}%
\pgfpathlineto{\pgfqpoint{1.535594in}{2.324549in}}%
\pgfpathlineto{\pgfqpoint{1.535766in}{2.422052in}}%
\pgfpathlineto{\pgfqpoint{1.536050in}{2.352504in}}%
\pgfpathlineto{\pgfqpoint{1.536062in}{2.352442in}}%
\pgfpathlineto{\pgfqpoint{1.536394in}{2.413740in}}%
\pgfpathlineto{\pgfqpoint{1.536210in}{2.311084in}}%
\pgfpathlineto{\pgfqpoint{1.537206in}{2.370703in}}%
\pgfpathlineto{\pgfqpoint{1.537637in}{2.424611in}}%
\pgfpathlineto{\pgfqpoint{1.537453in}{2.322689in}}%
\pgfpathlineto{\pgfqpoint{1.538043in}{2.340762in}}%
\pgfpathlineto{\pgfqpoint{1.538068in}{2.319162in}}%
\pgfpathlineto{\pgfqpoint{1.538622in}{2.426094in}}%
\pgfpathlineto{\pgfqpoint{1.539114in}{2.373967in}}%
\pgfpathlineto{\pgfqpoint{1.539483in}{2.426941in}}%
\pgfpathlineto{\pgfqpoint{1.539323in}{2.319894in}}%
\pgfpathlineto{\pgfqpoint{1.540246in}{2.400796in}}%
\pgfpathlineto{\pgfqpoint{1.541182in}{2.316167in}}%
\pgfpathlineto{\pgfqpoint{1.540714in}{2.418613in}}%
\pgfpathlineto{\pgfqpoint{1.541330in}{2.410973in}}%
\pgfpathlineto{\pgfqpoint{1.542351in}{2.419236in}}%
\pgfpathlineto{\pgfqpoint{1.541674in}{2.339385in}}%
\pgfpathlineto{\pgfqpoint{1.542376in}{2.393314in}}%
\pgfpathlineto{\pgfqpoint{1.542425in}{2.330257in}}%
\pgfpathlineto{\pgfqpoint{1.543336in}{2.417491in}}%
\pgfpathlineto{\pgfqpoint{1.543508in}{2.349989in}}%
\pgfpathlineto{\pgfqpoint{1.543533in}{2.330426in}}%
\pgfpathlineto{\pgfqpoint{1.543840in}{2.419285in}}%
\pgfpathlineto{\pgfqpoint{1.544579in}{2.394612in}}%
\pgfpathlineto{\pgfqpoint{1.545526in}{2.343016in}}%
\pgfpathlineto{\pgfqpoint{1.545071in}{2.413249in}}%
\pgfpathlineto{\pgfqpoint{1.545662in}{2.400818in}}%
\pgfpathlineto{\pgfqpoint{1.545686in}{2.429821in}}%
\pgfpathlineto{\pgfqpoint{1.546142in}{2.336093in}}%
\pgfpathlineto{\pgfqpoint{1.546733in}{2.366372in}}%
\pgfpathlineto{\pgfqpoint{1.546770in}{2.339325in}}%
\pgfpathlineto{\pgfqpoint{1.547557in}{2.418654in}}%
\pgfpathlineto{\pgfqpoint{1.547828in}{2.368578in}}%
\pgfpathlineto{\pgfqpoint{1.548813in}{2.422741in}}%
\pgfpathlineto{\pgfqpoint{1.547988in}{2.333317in}}%
\pgfpathlineto{\pgfqpoint{1.548973in}{2.373331in}}%
\pgfpathlineto{\pgfqpoint{1.549231in}{2.345573in}}%
\pgfpathlineto{\pgfqpoint{1.549416in}{2.420001in}}%
\pgfpathlineto{\pgfqpoint{1.550019in}{2.403543in}}%
\pgfpathlineto{\pgfqpoint{1.550659in}{2.415673in}}%
\pgfpathlineto{\pgfqpoint{1.550326in}{2.334772in}}%
\pgfpathlineto{\pgfqpoint{1.551065in}{2.359592in}}%
\pgfpathlineto{\pgfqpoint{1.551102in}{2.317654in}}%
\pgfpathlineto{\pgfqpoint{1.551286in}{2.415273in}}%
\pgfpathlineto{\pgfqpoint{1.552173in}{2.345553in}}%
\pgfpathlineto{\pgfqpoint{1.552530in}{2.411608in}}%
\pgfpathlineto{\pgfqpoint{1.552960in}{2.326754in}}%
\pgfpathlineto{\pgfqpoint{1.553354in}{2.377647in}}%
\pgfpathlineto{\pgfqpoint{1.553453in}{2.329443in}}%
\pgfpathlineto{\pgfqpoint{1.553514in}{2.422629in}}%
\pgfpathlineto{\pgfqpoint{1.554117in}{2.395395in}}%
\pgfpathlineto{\pgfqpoint{1.554376in}{2.422044in}}%
\pgfpathlineto{\pgfqpoint{1.554216in}{2.322960in}}%
\pgfpathlineto{\pgfqpoint{1.555188in}{2.363055in}}%
\pgfpathlineto{\pgfqpoint{1.555606in}{2.413174in}}%
\pgfpathlineto{\pgfqpoint{1.556074in}{2.322083in}}%
\pgfpathlineto{\pgfqpoint{1.556296in}{2.366515in}}%
\pgfpathlineto{\pgfqpoint{1.556566in}{2.339409in}}%
\pgfpathlineto{\pgfqpoint{1.557243in}{2.414539in}}%
\pgfpathlineto{\pgfqpoint{1.557379in}{2.366399in}}%
\pgfpathlineto{\pgfqpoint{1.558228in}{2.424516in}}%
\pgfpathlineto{\pgfqpoint{1.558413in}{2.338303in}}%
\pgfpathlineto{\pgfqpoint{1.558499in}{2.393741in}}%
\pgfpathlineto{\pgfqpoint{1.559188in}{2.338399in}}%
\pgfpathlineto{\pgfqpoint{1.558733in}{2.421950in}}%
\pgfpathlineto{\pgfqpoint{1.559619in}{2.374233in}}%
\pgfpathlineto{\pgfqpoint{1.560259in}{2.342987in}}%
\pgfpathlineto{\pgfqpoint{1.560579in}{2.427280in}}%
\pgfpathlineto{\pgfqpoint{1.560689in}{2.399583in}}%
\pgfpathlineto{\pgfqpoint{1.561342in}{2.413937in}}%
\pgfpathlineto{\pgfqpoint{1.561034in}{2.334010in}}%
\pgfpathlineto{\pgfqpoint{1.561748in}{2.369970in}}%
\pgfpathlineto{\pgfqpoint{1.562080in}{2.339632in}}%
\pgfpathlineto{\pgfqpoint{1.561945in}{2.421813in}}%
\pgfpathlineto{\pgfqpoint{1.562806in}{2.383461in}}%
\pgfpathlineto{\pgfqpoint{1.562831in}{2.386282in}}%
\pgfpathlineto{\pgfqpoint{1.562843in}{2.369802in}}%
\pgfpathlineto{\pgfqpoint{1.562880in}{2.331359in}}%
\pgfpathlineto{\pgfqpoint{1.563705in}{2.423205in}}%
\pgfpathlineto{\pgfqpoint{1.563951in}{2.357630in}}%
\pgfpathlineto{\pgfqpoint{1.564308in}{2.421515in}}%
\pgfpathlineto{\pgfqpoint{1.564160in}{2.344301in}}%
\pgfpathlineto{\pgfqpoint{1.565096in}{2.383116in}}%
\pgfpathlineto{\pgfqpoint{1.565994in}{2.322699in}}%
\pgfpathlineto{\pgfqpoint{1.566179in}{2.413284in}}%
\pgfpathlineto{\pgfqpoint{1.566191in}{2.413538in}}%
\pgfpathlineto{\pgfqpoint{1.566203in}{2.405062in}}%
\pgfpathlineto{\pgfqpoint{1.567237in}{2.338379in}}%
\pgfpathlineto{\pgfqpoint{1.566917in}{2.421503in}}%
\pgfpathlineto{\pgfqpoint{1.567323in}{2.380255in}}%
\pgfpathlineto{\pgfqpoint{1.567422in}{2.417927in}}%
\pgfpathlineto{\pgfqpoint{1.568333in}{2.330293in}}%
\pgfpathlineto{\pgfqpoint{1.568431in}{2.399715in}}%
\pgfpathlineto{\pgfqpoint{1.569120in}{2.327675in}}%
\pgfpathlineto{\pgfqpoint{1.569268in}{2.414308in}}%
\pgfpathlineto{\pgfqpoint{1.569576in}{2.338239in}}%
\pgfpathlineto{\pgfqpoint{1.570019in}{2.417543in}}%
\pgfpathlineto{\pgfqpoint{1.570683in}{2.350447in}}%
\pgfpathlineto{\pgfqpoint{1.570979in}{2.337474in}}%
\pgfpathlineto{\pgfqpoint{1.571262in}{2.420827in}}%
\pgfpathlineto{\pgfqpoint{1.571742in}{2.385399in}}%
\pgfpathlineto{\pgfqpoint{1.572382in}{2.415283in}}%
\pgfpathlineto{\pgfqpoint{1.572246in}{2.331054in}}%
\pgfpathlineto{\pgfqpoint{1.572640in}{2.369029in}}%
\pgfpathlineto{\pgfqpoint{1.573477in}{2.338704in}}%
\pgfpathlineto{\pgfqpoint{1.573120in}{2.413967in}}%
\pgfpathlineto{\pgfqpoint{1.573723in}{2.391863in}}%
\pgfpathlineto{\pgfqpoint{1.574376in}{2.407710in}}%
\pgfpathlineto{\pgfqpoint{1.574105in}{2.339516in}}%
\pgfpathlineto{\pgfqpoint{1.574806in}{2.381693in}}%
\pgfpathlineto{\pgfqpoint{1.575606in}{2.411808in}}%
\pgfpathlineto{\pgfqpoint{1.575360in}{2.341456in}}%
\pgfpathlineto{\pgfqpoint{1.575779in}{2.360342in}}%
\pgfpathlineto{\pgfqpoint{1.576591in}{2.347944in}}%
\pgfpathlineto{\pgfqpoint{1.576234in}{2.412007in}}%
\pgfpathlineto{\pgfqpoint{1.576812in}{2.397768in}}%
\pgfpathlineto{\pgfqpoint{1.576837in}{2.409430in}}%
\pgfpathlineto{\pgfqpoint{1.577219in}{2.350333in}}%
\pgfpathlineto{\pgfqpoint{1.577896in}{2.386691in}}%
\pgfpathlineto{\pgfqpoint{1.578265in}{2.347367in}}%
\pgfpathlineto{\pgfqpoint{1.577957in}{2.402942in}}%
\pgfpathlineto{\pgfqpoint{1.579052in}{2.349002in}}%
\pgfpathlineto{\pgfqpoint{1.579951in}{2.409312in}}%
\pgfpathlineto{\pgfqpoint{1.580099in}{2.338647in}}%
\pgfpathlineto{\pgfqpoint{1.580246in}{2.372455in}}%
\pgfpathlineto{\pgfqpoint{1.580874in}{2.342713in}}%
\pgfpathlineto{\pgfqpoint{1.580566in}{2.401714in}}%
\pgfpathlineto{\pgfqpoint{1.581379in}{2.351783in}}%
\pgfpathlineto{\pgfqpoint{1.582302in}{2.405958in}}%
\pgfpathlineto{\pgfqpoint{1.582166in}{2.341095in}}%
\pgfpathlineto{\pgfqpoint{1.582560in}{2.390880in}}%
\pgfpathlineto{\pgfqpoint{1.583212in}{2.328997in}}%
\pgfpathlineto{\pgfqpoint{1.583065in}{2.404309in}}%
\pgfpathlineto{\pgfqpoint{1.583668in}{2.383905in}}%
\pgfpathlineto{\pgfqpoint{1.584172in}{2.399531in}}%
\pgfpathlineto{\pgfqpoint{1.584000in}{2.333897in}}%
\pgfpathlineto{\pgfqpoint{1.584739in}{2.367406in}}%
\pgfpathlineto{\pgfqpoint{1.585280in}{2.350470in}}%
\pgfpathlineto{\pgfqpoint{1.584911in}{2.410035in}}%
\pgfpathlineto{\pgfqpoint{1.585859in}{2.358010in}}%
\pgfpathlineto{\pgfqpoint{1.585871in}{2.358080in}}%
\pgfpathlineto{\pgfqpoint{1.585883in}{2.353963in}}%
\pgfpathlineto{\pgfqpoint{1.586339in}{2.331249in}}%
\pgfpathlineto{\pgfqpoint{1.586154in}{2.403677in}}%
\pgfpathlineto{\pgfqpoint{1.586966in}{2.365120in}}%
\pgfpathlineto{\pgfqpoint{1.588049in}{2.411322in}}%
\pgfpathlineto{\pgfqpoint{1.587139in}{2.326181in}}%
\pgfpathlineto{\pgfqpoint{1.588086in}{2.380535in}}%
\pgfpathlineto{\pgfqpoint{1.588357in}{2.346958in}}%
\pgfpathlineto{\pgfqpoint{1.588542in}{2.404478in}}%
\pgfpathlineto{\pgfqpoint{1.589169in}{2.394830in}}%
\pgfpathlineto{\pgfqpoint{1.590252in}{2.323232in}}%
\pgfpathlineto{\pgfqpoint{1.589268in}{2.412773in}}%
\pgfpathlineto{\pgfqpoint{1.590302in}{2.374675in}}%
\pgfpathlineto{\pgfqpoint{1.590400in}{2.409103in}}%
\pgfpathlineto{\pgfqpoint{1.590696in}{2.349414in}}%
\pgfpathlineto{\pgfqpoint{1.591422in}{2.382498in}}%
\pgfpathlineto{\pgfqpoint{1.591483in}{2.335844in}}%
\pgfpathlineto{\pgfqpoint{1.591619in}{2.410455in}}%
\pgfpathlineto{\pgfqpoint{1.592554in}{2.352687in}}%
\pgfpathlineto{\pgfqpoint{1.593514in}{2.407677in}}%
\pgfpathlineto{\pgfqpoint{1.593379in}{2.327355in}}%
\pgfpathlineto{\pgfqpoint{1.593723in}{2.380420in}}%
\pgfpathlineto{\pgfqpoint{1.594597in}{2.338414in}}%
\pgfpathlineto{\pgfqpoint{1.594757in}{2.409876in}}%
\pgfpathlineto{\pgfqpoint{1.594794in}{2.393777in}}%
\pgfpathlineto{\pgfqpoint{1.595508in}{2.401388in}}%
\pgfpathlineto{\pgfqpoint{1.595225in}{2.342435in}}%
\pgfpathlineto{\pgfqpoint{1.595717in}{2.360578in}}%
\pgfpathlineto{\pgfqpoint{1.596492in}{2.333472in}}%
\pgfpathlineto{\pgfqpoint{1.596111in}{2.400328in}}%
\pgfpathlineto{\pgfqpoint{1.596714in}{2.392592in}}%
\pgfpathlineto{\pgfqpoint{1.597366in}{2.401648in}}%
\pgfpathlineto{\pgfqpoint{1.597711in}{2.340502in}}%
\pgfpathlineto{\pgfqpoint{1.597723in}{2.342070in}}%
\pgfpathlineto{\pgfqpoint{1.598326in}{2.336723in}}%
\pgfpathlineto{\pgfqpoint{1.597969in}{2.405202in}}%
\pgfpathlineto{\pgfqpoint{1.598572in}{2.387902in}}%
\pgfpathlineto{\pgfqpoint{1.599225in}{2.402363in}}%
\pgfpathlineto{\pgfqpoint{1.599594in}{2.339531in}}%
\pgfpathlineto{\pgfqpoint{1.599643in}{2.358298in}}%
\pgfpathlineto{\pgfqpoint{1.599668in}{2.361354in}}%
\pgfpathlineto{\pgfqpoint{1.600320in}{2.403846in}}%
\pgfpathlineto{\pgfqpoint{1.600185in}{2.336419in}}%
\pgfpathlineto{\pgfqpoint{1.600763in}{2.364543in}}%
\pgfpathlineto{\pgfqpoint{1.601231in}{2.332604in}}%
\pgfpathlineto{\pgfqpoint{1.601083in}{2.410978in}}%
\pgfpathlineto{\pgfqpoint{1.601859in}{2.372360in}}%
\pgfpathlineto{\pgfqpoint{1.602942in}{2.409989in}}%
\pgfpathlineto{\pgfqpoint{1.602019in}{2.333935in}}%
\pgfpathlineto{\pgfqpoint{1.602979in}{2.388945in}}%
\pgfpathlineto{\pgfqpoint{1.603311in}{2.332851in}}%
\pgfpathlineto{\pgfqpoint{1.603434in}{2.405706in}}%
\pgfpathlineto{\pgfqpoint{1.604086in}{2.385301in}}%
\pgfpathlineto{\pgfqpoint{1.605132in}{2.330195in}}%
\pgfpathlineto{\pgfqpoint{1.604197in}{2.415156in}}%
\pgfpathlineto{\pgfqpoint{1.605219in}{2.366152in}}%
\pgfpathlineto{\pgfqpoint{1.606055in}{2.416221in}}%
\pgfpathlineto{\pgfqpoint{1.605822in}{2.346612in}}%
\pgfpathlineto{\pgfqpoint{1.606326in}{2.379912in}}%
\pgfpathlineto{\pgfqpoint{1.606437in}{2.331487in}}%
\pgfpathlineto{\pgfqpoint{1.606572in}{2.409235in}}%
\pgfpathlineto{\pgfqpoint{1.607483in}{2.344196in}}%
\pgfpathlineto{\pgfqpoint{1.608406in}{2.402985in}}%
\pgfpathlineto{\pgfqpoint{1.608271in}{2.313496in}}%
\pgfpathlineto{\pgfqpoint{1.608628in}{2.389745in}}%
\pgfpathlineto{\pgfqpoint{1.609514in}{2.336486in}}%
\pgfpathlineto{\pgfqpoint{1.609182in}{2.413156in}}%
\pgfpathlineto{\pgfqpoint{1.609748in}{2.378666in}}%
\pgfpathlineto{\pgfqpoint{1.610400in}{2.413649in}}%
\pgfpathlineto{\pgfqpoint{1.610142in}{2.343327in}}%
\pgfpathlineto{\pgfqpoint{1.610745in}{2.361333in}}%
\pgfpathlineto{\pgfqpoint{1.611385in}{2.304952in}}%
\pgfpathlineto{\pgfqpoint{1.611545in}{2.403427in}}%
\pgfpathlineto{\pgfqpoint{1.611840in}{2.361800in}}%
\pgfpathlineto{\pgfqpoint{1.612295in}{2.403954in}}%
\pgfpathlineto{\pgfqpoint{1.612628in}{2.327689in}}%
\pgfpathlineto{\pgfqpoint{1.612935in}{2.360758in}}%
\pgfpathlineto{\pgfqpoint{1.613243in}{2.330889in}}%
\pgfpathlineto{\pgfqpoint{1.613526in}{2.411307in}}%
\pgfpathlineto{\pgfqpoint{1.613982in}{2.375923in}}%
\pgfpathlineto{\pgfqpoint{1.614659in}{2.406349in}}%
\pgfpathlineto{\pgfqpoint{1.614511in}{2.310391in}}%
\pgfpathlineto{\pgfqpoint{1.615040in}{2.362014in}}%
\pgfpathlineto{\pgfqpoint{1.615742in}{2.322576in}}%
\pgfpathlineto{\pgfqpoint{1.615889in}{2.406584in}}%
\pgfpathlineto{\pgfqpoint{1.616148in}{2.361947in}}%
\pgfpathlineto{\pgfqpoint{1.617243in}{2.408887in}}%
\pgfpathlineto{\pgfqpoint{1.616357in}{2.322468in}}%
\pgfpathlineto{\pgfqpoint{1.617305in}{2.371372in}}%
\pgfpathlineto{\pgfqpoint{1.617625in}{2.316271in}}%
\pgfpathlineto{\pgfqpoint{1.617772in}{2.403647in}}%
\pgfpathlineto{\pgfqpoint{1.618277in}{2.385797in}}%
\pgfpathlineto{\pgfqpoint{1.619102in}{2.413274in}}%
\pgfpathlineto{\pgfqpoint{1.618855in}{2.328099in}}%
\pgfpathlineto{\pgfqpoint{1.619372in}{2.386751in}}%
\pgfpathlineto{\pgfqpoint{1.619471in}{2.319008in}}%
\pgfpathlineto{\pgfqpoint{1.620357in}{2.409633in}}%
\pgfpathlineto{\pgfqpoint{1.620529in}{2.360548in}}%
\pgfpathlineto{\pgfqpoint{1.620972in}{2.409192in}}%
\pgfpathlineto{\pgfqpoint{1.620738in}{2.331682in}}%
\pgfpathlineto{\pgfqpoint{1.621686in}{2.394084in}}%
\pgfpathlineto{\pgfqpoint{1.621698in}{2.394763in}}%
\pgfpathlineto{\pgfqpoint{1.621945in}{2.350980in}}%
\pgfpathlineto{\pgfqpoint{1.622597in}{2.322862in}}%
\pgfpathlineto{\pgfqpoint{1.622215in}{2.416036in}}%
\pgfpathlineto{\pgfqpoint{1.623015in}{2.383207in}}%
\pgfpathlineto{\pgfqpoint{1.623151in}{2.331304in}}%
\pgfpathlineto{\pgfqpoint{1.623471in}{2.409693in}}%
\pgfpathlineto{\pgfqpoint{1.624062in}{2.395715in}}%
\pgfpathlineto{\pgfqpoint{1.624086in}{2.408249in}}%
\pgfpathlineto{\pgfqpoint{1.624443in}{2.322084in}}%
\pgfpathlineto{\pgfqpoint{1.625071in}{2.345322in}}%
\pgfpathlineto{\pgfqpoint{1.625711in}{2.328906in}}%
\pgfpathlineto{\pgfqpoint{1.625342in}{2.416388in}}%
\pgfpathlineto{\pgfqpoint{1.625797in}{2.388068in}}%
\pgfpathlineto{\pgfqpoint{1.626585in}{2.408566in}}%
\pgfpathlineto{\pgfqpoint{1.626265in}{2.324282in}}%
\pgfpathlineto{\pgfqpoint{1.626868in}{2.376226in}}%
\pgfpathlineto{\pgfqpoint{1.627569in}{2.315525in}}%
\pgfpathlineto{\pgfqpoint{1.627200in}{2.408391in}}%
\pgfpathlineto{\pgfqpoint{1.627975in}{2.370205in}}%
\pgfpathlineto{\pgfqpoint{1.628455in}{2.414985in}}%
\pgfpathlineto{\pgfqpoint{1.628603in}{2.333835in}}%
\pgfpathlineto{\pgfqpoint{1.629108in}{2.382115in}}%
\pgfpathlineto{\pgfqpoint{1.629428in}{2.324313in}}%
\pgfpathlineto{\pgfqpoint{1.629711in}{2.403539in}}%
\pgfpathlineto{\pgfqpoint{1.630215in}{2.381389in}}%
\pgfpathlineto{\pgfqpoint{1.630314in}{2.410703in}}%
\pgfpathlineto{\pgfqpoint{1.630683in}{2.318080in}}%
\pgfpathlineto{\pgfqpoint{1.631188in}{2.374944in}}%
\pgfpathlineto{\pgfqpoint{1.631926in}{2.335860in}}%
\pgfpathlineto{\pgfqpoint{1.631582in}{2.410026in}}%
\pgfpathlineto{\pgfqpoint{1.632295in}{2.378035in}}%
\pgfpathlineto{\pgfqpoint{1.632382in}{2.403999in}}%
\pgfpathlineto{\pgfqpoint{1.632517in}{2.322132in}}%
\pgfpathlineto{\pgfqpoint{1.632529in}{2.312696in}}%
\pgfpathlineto{\pgfqpoint{1.633428in}{2.410031in}}%
\pgfpathlineto{\pgfqpoint{1.633575in}{2.349967in}}%
\pgfpathlineto{\pgfqpoint{1.633797in}{2.327496in}}%
\pgfpathlineto{\pgfqpoint{1.634732in}{2.406708in}}%
\pgfpathlineto{\pgfqpoint{1.635655in}{2.302887in}}%
\pgfpathlineto{\pgfqpoint{1.635951in}{2.386911in}}%
\pgfpathlineto{\pgfqpoint{1.636554in}{2.406406in}}%
\pgfpathlineto{\pgfqpoint{1.636874in}{2.339853in}}%
\pgfpathlineto{\pgfqpoint{1.637514in}{2.324193in}}%
\pgfpathlineto{\pgfqpoint{1.637846in}{2.403860in}}%
\pgfpathlineto{\pgfqpoint{1.637945in}{2.367143in}}%
\pgfpathlineto{\pgfqpoint{1.638769in}{2.301628in}}%
\pgfpathlineto{\pgfqpoint{1.638400in}{2.400523in}}%
\pgfpathlineto{\pgfqpoint{1.638954in}{2.387480in}}%
\pgfpathlineto{\pgfqpoint{1.639446in}{2.406547in}}%
\pgfpathlineto{\pgfqpoint{1.639385in}{2.337082in}}%
\pgfpathlineto{\pgfqpoint{1.639938in}{2.382982in}}%
\pgfpathlineto{\pgfqpoint{1.640628in}{2.319350in}}%
\pgfpathlineto{\pgfqpoint{1.640911in}{2.404838in}}%
\pgfpathlineto{\pgfqpoint{1.641046in}{2.373601in}}%
\pgfpathlineto{\pgfqpoint{1.641514in}{2.407655in}}%
\pgfpathlineto{\pgfqpoint{1.641883in}{2.309714in}}%
\pgfpathlineto{\pgfqpoint{1.642166in}{2.385364in}}%
\pgfpathlineto{\pgfqpoint{1.643114in}{2.331298in}}%
\pgfpathlineto{\pgfqpoint{1.642560in}{2.404313in}}%
\pgfpathlineto{\pgfqpoint{1.643261in}{2.389139in}}%
\pgfpathlineto{\pgfqpoint{1.643372in}{2.407809in}}%
\pgfpathlineto{\pgfqpoint{1.643705in}{2.351400in}}%
\pgfpathlineto{\pgfqpoint{1.643741in}{2.319718in}}%
\pgfpathlineto{\pgfqpoint{1.644628in}{2.410624in}}%
\pgfpathlineto{\pgfqpoint{1.644788in}{2.358005in}}%
\pgfpathlineto{\pgfqpoint{1.645674in}{2.405781in}}%
\pgfpathlineto{\pgfqpoint{1.644997in}{2.320368in}}%
\pgfpathlineto{\pgfqpoint{1.645920in}{2.386859in}}%
\pgfpathlineto{\pgfqpoint{1.646498in}{2.412309in}}%
\pgfpathlineto{\pgfqpoint{1.646855in}{2.319255in}}%
\pgfpathlineto{\pgfqpoint{1.647028in}{2.388438in}}%
\pgfpathlineto{\pgfqpoint{1.647471in}{2.321391in}}%
\pgfpathlineto{\pgfqpoint{1.647754in}{2.412217in}}%
\pgfpathlineto{\pgfqpoint{1.648246in}{2.371359in}}%
\pgfpathlineto{\pgfqpoint{1.648357in}{2.405490in}}%
\pgfpathlineto{\pgfqpoint{1.648714in}{2.328405in}}%
\pgfpathlineto{\pgfqpoint{1.649317in}{2.358556in}}%
\pgfpathlineto{\pgfqpoint{1.649969in}{2.323935in}}%
\pgfpathlineto{\pgfqpoint{1.649625in}{2.413777in}}%
\pgfpathlineto{\pgfqpoint{1.650388in}{2.383848in}}%
\pgfpathlineto{\pgfqpoint{1.650880in}{2.410118in}}%
\pgfpathlineto{\pgfqpoint{1.650585in}{2.315457in}}%
\pgfpathlineto{\pgfqpoint{1.651508in}{2.398007in}}%
\pgfpathlineto{\pgfqpoint{1.651828in}{2.321383in}}%
\pgfpathlineto{\pgfqpoint{1.652628in}{2.392459in}}%
\pgfpathlineto{\pgfqpoint{1.652738in}{2.416338in}}%
\pgfpathlineto{\pgfqpoint{1.653071in}{2.329363in}}%
\pgfpathlineto{\pgfqpoint{1.653563in}{2.383171in}}%
\pgfpathlineto{\pgfqpoint{1.653698in}{2.313924in}}%
\pgfpathlineto{\pgfqpoint{1.654585in}{2.415566in}}%
\pgfpathlineto{\pgfqpoint{1.654695in}{2.363164in}}%
\pgfpathlineto{\pgfqpoint{1.655852in}{2.413671in}}%
\pgfpathlineto{\pgfqpoint{1.654941in}{2.316159in}}%
\pgfpathlineto{\pgfqpoint{1.655901in}{2.389185in}}%
\pgfpathlineto{\pgfqpoint{1.656812in}{2.310944in}}%
\pgfpathlineto{\pgfqpoint{1.656443in}{2.412246in}}%
\pgfpathlineto{\pgfqpoint{1.657034in}{2.368922in}}%
\pgfpathlineto{\pgfqpoint{1.657711in}{2.415234in}}%
\pgfpathlineto{\pgfqpoint{1.658043in}{2.324364in}}%
\pgfpathlineto{\pgfqpoint{1.658055in}{2.318700in}}%
\pgfpathlineto{\pgfqpoint{1.658941in}{2.415513in}}%
\pgfpathlineto{\pgfqpoint{1.658954in}{2.414787in}}%
\pgfpathlineto{\pgfqpoint{1.658966in}{2.415480in}}%
\pgfpathlineto{\pgfqpoint{1.659064in}{2.362689in}}%
\pgfpathlineto{\pgfqpoint{1.659926in}{2.315833in}}%
\pgfpathlineto{\pgfqpoint{1.659569in}{2.414945in}}%
\pgfpathlineto{\pgfqpoint{1.660148in}{2.370775in}}%
\pgfpathlineto{\pgfqpoint{1.660837in}{2.413896in}}%
\pgfpathlineto{\pgfqpoint{1.661157in}{2.331462in}}%
\pgfpathlineto{\pgfqpoint{1.661169in}{2.318770in}}%
\pgfpathlineto{\pgfqpoint{1.662080in}{2.416825in}}%
\pgfpathlineto{\pgfqpoint{1.662228in}{2.361267in}}%
\pgfpathlineto{\pgfqpoint{1.662683in}{2.418517in}}%
\pgfpathlineto{\pgfqpoint{1.663040in}{2.311270in}}%
\pgfpathlineto{\pgfqpoint{1.663372in}{2.386503in}}%
\pgfpathlineto{\pgfqpoint{1.664295in}{2.327423in}}%
\pgfpathlineto{\pgfqpoint{1.663951in}{2.413250in}}%
\pgfpathlineto{\pgfqpoint{1.664418in}{2.388862in}}%
\pgfpathlineto{\pgfqpoint{1.665194in}{2.413195in}}%
\pgfpathlineto{\pgfqpoint{1.664911in}{2.325733in}}%
\pgfpathlineto{\pgfqpoint{1.665464in}{2.348464in}}%
\pgfpathlineto{\pgfqpoint{1.666154in}{2.308047in}}%
\pgfpathlineto{\pgfqpoint{1.665821in}{2.419399in}}%
\pgfpathlineto{\pgfqpoint{1.666314in}{2.402623in}}%
\pgfpathlineto{\pgfqpoint{1.666326in}{2.406701in}}%
\pgfpathlineto{\pgfqpoint{1.666769in}{2.336764in}}%
\pgfpathlineto{\pgfqpoint{1.667311in}{2.366993in}}%
\pgfpathlineto{\pgfqpoint{1.668012in}{2.325705in}}%
\pgfpathlineto{\pgfqpoint{1.667569in}{2.407504in}}%
\pgfpathlineto{\pgfqpoint{1.668271in}{2.400782in}}%
\pgfpathlineto{\pgfqpoint{1.668935in}{2.417392in}}%
\pgfpathlineto{\pgfqpoint{1.668640in}{2.329108in}}%
\pgfpathlineto{\pgfqpoint{1.669243in}{2.332253in}}%
\pgfpathlineto{\pgfqpoint{1.669268in}{2.307847in}}%
\pgfpathlineto{\pgfqpoint{1.669440in}{2.407924in}}%
\pgfpathlineto{\pgfqpoint{1.670314in}{2.369576in}}%
\pgfpathlineto{\pgfqpoint{1.671397in}{2.409038in}}%
\pgfpathlineto{\pgfqpoint{1.671126in}{2.323916in}}%
\pgfpathlineto{\pgfqpoint{1.671458in}{2.397267in}}%
\pgfpathlineto{\pgfqpoint{1.672381in}{2.313733in}}%
\pgfpathlineto{\pgfqpoint{1.672049in}{2.409835in}}%
\pgfpathlineto{\pgfqpoint{1.672541in}{2.398257in}}%
\pgfpathlineto{\pgfqpoint{1.672554in}{2.403724in}}%
\pgfpathlineto{\pgfqpoint{1.672997in}{2.329674in}}%
\pgfpathlineto{\pgfqpoint{1.673551in}{2.362013in}}%
\pgfpathlineto{\pgfqpoint{1.674240in}{2.320819in}}%
\pgfpathlineto{\pgfqpoint{1.674412in}{2.409845in}}%
\pgfpathlineto{\pgfqpoint{1.674646in}{2.378164in}}%
\pgfpathlineto{\pgfqpoint{1.675495in}{2.320697in}}%
\pgfpathlineto{\pgfqpoint{1.675151in}{2.405115in}}%
\pgfpathlineto{\pgfqpoint{1.675631in}{2.381260in}}%
\pgfpathlineto{\pgfqpoint{1.675741in}{2.406176in}}%
\pgfpathlineto{\pgfqpoint{1.676111in}{2.329959in}}%
\pgfpathlineto{\pgfqpoint{1.676677in}{2.346631in}}%
\pgfpathlineto{\pgfqpoint{1.677354in}{2.317498in}}%
\pgfpathlineto{\pgfqpoint{1.676997in}{2.411706in}}%
\pgfpathlineto{\pgfqpoint{1.677514in}{2.397450in}}%
\pgfpathlineto{\pgfqpoint{1.677637in}{2.408323in}}%
\pgfpathlineto{\pgfqpoint{1.677969in}{2.320021in}}%
\pgfpathlineto{\pgfqpoint{1.678547in}{2.354637in}}%
\pgfpathlineto{\pgfqpoint{1.678560in}{2.354602in}}%
\pgfpathlineto{\pgfqpoint{1.678597in}{2.329236in}}%
\pgfpathlineto{\pgfqpoint{1.678855in}{2.411872in}}%
\pgfpathlineto{\pgfqpoint{1.679594in}{2.377233in}}%
\pgfpathlineto{\pgfqpoint{1.680123in}{2.410398in}}%
\pgfpathlineto{\pgfqpoint{1.680467in}{2.326079in}}%
\pgfpathlineto{\pgfqpoint{1.680763in}{2.396246in}}%
\pgfpathlineto{\pgfqpoint{1.681083in}{2.319483in}}%
\pgfpathlineto{\pgfqpoint{1.681341in}{2.403711in}}%
\pgfpathlineto{\pgfqpoint{1.681932in}{2.384671in}}%
\pgfpathlineto{\pgfqpoint{1.681994in}{2.408228in}}%
\pgfpathlineto{\pgfqpoint{1.682941in}{2.328695in}}%
\pgfpathlineto{\pgfqpoint{1.683027in}{2.380654in}}%
\pgfpathlineto{\pgfqpoint{1.683224in}{2.411094in}}%
\pgfpathlineto{\pgfqpoint{1.683557in}{2.342252in}}%
\pgfpathlineto{\pgfqpoint{1.684074in}{2.371618in}}%
\pgfpathlineto{\pgfqpoint{1.684172in}{2.323895in}}%
\pgfpathlineto{\pgfqpoint{1.684467in}{2.405965in}}%
\pgfpathlineto{\pgfqpoint{1.685021in}{2.381537in}}%
\pgfpathlineto{\pgfqpoint{1.685095in}{2.407963in}}%
\pgfpathlineto{\pgfqpoint{1.685403in}{2.340915in}}%
\pgfpathlineto{\pgfqpoint{1.686031in}{2.334108in}}%
\pgfpathlineto{\pgfqpoint{1.686326in}{2.411572in}}%
\pgfpathlineto{\pgfqpoint{1.686449in}{2.367209in}}%
\pgfpathlineto{\pgfqpoint{1.686646in}{2.336998in}}%
\pgfpathlineto{\pgfqpoint{1.686929in}{2.400819in}}%
\pgfpathlineto{\pgfqpoint{1.686954in}{2.409342in}}%
\pgfpathlineto{\pgfqpoint{1.687274in}{2.316639in}}%
\pgfpathlineto{\pgfqpoint{1.687987in}{2.371684in}}%
\pgfpathlineto{\pgfqpoint{1.688504in}{2.335309in}}%
\pgfpathlineto{\pgfqpoint{1.688209in}{2.400688in}}%
\pgfpathlineto{\pgfqpoint{1.689046in}{2.384516in}}%
\pgfpathlineto{\pgfqpoint{1.690043in}{2.407108in}}%
\pgfpathlineto{\pgfqpoint{1.689747in}{2.324280in}}%
\pgfpathlineto{\pgfqpoint{1.690104in}{2.374188in}}%
\pgfpathlineto{\pgfqpoint{1.690363in}{2.317699in}}%
\pgfpathlineto{\pgfqpoint{1.690683in}{2.400616in}}%
\pgfpathlineto{\pgfqpoint{1.691224in}{2.368401in}}%
\pgfpathlineto{\pgfqpoint{1.692147in}{2.401286in}}%
\pgfpathlineto{\pgfqpoint{1.692221in}{2.320661in}}%
\pgfpathlineto{\pgfqpoint{1.692344in}{2.375054in}}%
\pgfpathlineto{\pgfqpoint{1.692541in}{2.404773in}}%
\pgfpathlineto{\pgfqpoint{1.692677in}{2.364947in}}%
\pgfpathlineto{\pgfqpoint{1.692837in}{2.326625in}}%
\pgfpathlineto{\pgfqpoint{1.693157in}{2.406009in}}%
\pgfpathlineto{\pgfqpoint{1.693747in}{2.397074in}}%
\pgfpathlineto{\pgfqpoint{1.694387in}{2.403239in}}%
\pgfpathlineto{\pgfqpoint{1.694080in}{2.333794in}}%
\pgfpathlineto{\pgfqpoint{1.694523in}{2.380181in}}%
\pgfpathlineto{\pgfqpoint{1.694695in}{2.331331in}}%
\pgfpathlineto{\pgfqpoint{1.694991in}{2.403084in}}%
\pgfpathlineto{\pgfqpoint{1.695606in}{2.401515in}}%
\pgfpathlineto{\pgfqpoint{1.695631in}{2.409427in}}%
\pgfpathlineto{\pgfqpoint{1.695938in}{2.329021in}}%
\pgfpathlineto{\pgfqpoint{1.696541in}{2.347153in}}%
\pgfpathlineto{\pgfqpoint{1.696554in}{2.335582in}}%
\pgfpathlineto{\pgfqpoint{1.696861in}{2.409344in}}%
\pgfpathlineto{\pgfqpoint{1.697600in}{2.390213in}}%
\pgfpathlineto{\pgfqpoint{1.697784in}{2.337729in}}%
\pgfpathlineto{\pgfqpoint{1.698104in}{2.410811in}}%
\pgfpathlineto{\pgfqpoint{1.698695in}{2.391440in}}%
\pgfpathlineto{\pgfqpoint{1.698720in}{2.409606in}}%
\pgfpathlineto{\pgfqpoint{1.699027in}{2.328845in}}%
\pgfpathlineto{\pgfqpoint{1.699778in}{2.372824in}}%
\pgfpathlineto{\pgfqpoint{1.700123in}{2.345385in}}%
\pgfpathlineto{\pgfqpoint{1.699963in}{2.417543in}}%
\pgfpathlineto{\pgfqpoint{1.700898in}{2.359109in}}%
\pgfpathlineto{\pgfqpoint{1.701206in}{2.414399in}}%
\pgfpathlineto{\pgfqpoint{1.701366in}{2.333873in}}%
\pgfpathlineto{\pgfqpoint{1.701994in}{2.349934in}}%
\pgfpathlineto{\pgfqpoint{1.703064in}{2.407829in}}%
\pgfpathlineto{\pgfqpoint{1.702117in}{2.339282in}}%
\pgfpathlineto{\pgfqpoint{1.703126in}{2.360958in}}%
\pgfpathlineto{\pgfqpoint{1.703224in}{2.344857in}}%
\pgfpathlineto{\pgfqpoint{1.703557in}{2.398284in}}%
\pgfpathlineto{\pgfqpoint{1.704061in}{2.380023in}}%
\pgfpathlineto{\pgfqpoint{1.704295in}{2.406242in}}%
\pgfpathlineto{\pgfqpoint{1.704467in}{2.332830in}}%
\pgfpathlineto{\pgfqpoint{1.705157in}{2.379089in}}%
\pgfpathlineto{\pgfqpoint{1.706301in}{2.344346in}}%
\pgfpathlineto{\pgfqpoint{1.705415in}{2.402898in}}%
\pgfpathlineto{\pgfqpoint{1.706314in}{2.345317in}}%
\pgfpathlineto{\pgfqpoint{1.707384in}{2.401796in}}%
\pgfpathlineto{\pgfqpoint{1.706929in}{2.345093in}}%
\pgfpathlineto{\pgfqpoint{1.707520in}{2.362177in}}%
\pgfpathlineto{\pgfqpoint{1.707557in}{2.341811in}}%
\pgfpathlineto{\pgfqpoint{1.708000in}{2.401345in}}%
\pgfpathlineto{\pgfqpoint{1.708603in}{2.385371in}}%
\pgfpathlineto{\pgfqpoint{1.709415in}{2.343192in}}%
\pgfpathlineto{\pgfqpoint{1.709735in}{2.402806in}}%
\pgfpathlineto{\pgfqpoint{1.710006in}{2.337451in}}%
\pgfpathlineto{\pgfqpoint{1.710350in}{2.408495in}}%
\pgfpathlineto{\pgfqpoint{1.710867in}{2.394148in}}%
\pgfpathlineto{\pgfqpoint{1.711594in}{2.403800in}}%
\pgfpathlineto{\pgfqpoint{1.711261in}{2.349585in}}%
\pgfpathlineto{\pgfqpoint{1.711852in}{2.352690in}}%
\pgfpathlineto{\pgfqpoint{1.711864in}{2.344543in}}%
\pgfpathlineto{\pgfqpoint{1.712824in}{2.410482in}}%
\pgfpathlineto{\pgfqpoint{1.712910in}{2.373973in}}%
\pgfpathlineto{\pgfqpoint{1.713452in}{2.409335in}}%
\pgfpathlineto{\pgfqpoint{1.713120in}{2.336526in}}%
\pgfpathlineto{\pgfqpoint{1.714055in}{2.409235in}}%
\pgfpathlineto{\pgfqpoint{1.714067in}{2.409344in}}%
\pgfpathlineto{\pgfqpoint{1.714092in}{2.401694in}}%
\pgfpathlineto{\pgfqpoint{1.714978in}{2.341958in}}%
\pgfpathlineto{\pgfqpoint{1.715212in}{2.389985in}}%
\pgfpathlineto{\pgfqpoint{1.716209in}{2.338902in}}%
\pgfpathlineto{\pgfqpoint{1.715914in}{2.413096in}}%
\pgfpathlineto{\pgfqpoint{1.716357in}{2.365629in}}%
\pgfpathlineto{\pgfqpoint{1.717157in}{2.414080in}}%
\pgfpathlineto{\pgfqpoint{1.716812in}{2.345761in}}%
\pgfpathlineto{\pgfqpoint{1.717440in}{2.350155in}}%
\pgfpathlineto{\pgfqpoint{1.717452in}{2.349800in}}%
\pgfpathlineto{\pgfqpoint{1.717489in}{2.373742in}}%
\pgfpathlineto{\pgfqpoint{1.718400in}{2.411746in}}%
\pgfpathlineto{\pgfqpoint{1.718067in}{2.345623in}}%
\pgfpathlineto{\pgfqpoint{1.718560in}{2.350422in}}%
\pgfpathlineto{\pgfqpoint{1.719286in}{2.341155in}}%
\pgfpathlineto{\pgfqpoint{1.719015in}{2.413445in}}%
\pgfpathlineto{\pgfqpoint{1.719581in}{2.395990in}}%
\pgfpathlineto{\pgfqpoint{1.720246in}{2.412590in}}%
\pgfpathlineto{\pgfqpoint{1.719901in}{2.344028in}}%
\pgfpathlineto{\pgfqpoint{1.720627in}{2.372326in}}%
\pgfpathlineto{\pgfqpoint{1.721760in}{2.342860in}}%
\pgfpathlineto{\pgfqpoint{1.721477in}{2.411746in}}%
\pgfpathlineto{\pgfqpoint{1.721772in}{2.346043in}}%
\pgfpathlineto{\pgfqpoint{1.721784in}{2.346016in}}%
\pgfpathlineto{\pgfqpoint{1.722104in}{2.416023in}}%
\pgfpathlineto{\pgfqpoint{1.722375in}{2.342372in}}%
\pgfpathlineto{\pgfqpoint{1.722904in}{2.354264in}}%
\pgfpathlineto{\pgfqpoint{1.723950in}{2.411557in}}%
\pgfpathlineto{\pgfqpoint{1.723150in}{2.342631in}}%
\pgfpathlineto{\pgfqpoint{1.724061in}{2.379180in}}%
\pgfpathlineto{\pgfqpoint{1.724381in}{2.342328in}}%
\pgfpathlineto{\pgfqpoint{1.724529in}{2.403371in}}%
\pgfpathlineto{\pgfqpoint{1.725193in}{2.416556in}}%
\pgfpathlineto{\pgfqpoint{1.724849in}{2.337935in}}%
\pgfpathlineto{\pgfqpoint{1.725575in}{2.359901in}}%
\pgfpathlineto{\pgfqpoint{1.725624in}{2.335762in}}%
\pgfpathlineto{\pgfqpoint{1.725809in}{2.418586in}}%
\pgfpathlineto{\pgfqpoint{1.726658in}{2.376808in}}%
\pgfpathlineto{\pgfqpoint{1.726707in}{2.341565in}}%
\pgfpathlineto{\pgfqpoint{1.727027in}{2.413286in}}%
\pgfpathlineto{\pgfqpoint{1.727667in}{2.423977in}}%
\pgfpathlineto{\pgfqpoint{1.727470in}{2.341184in}}%
\pgfpathlineto{\pgfqpoint{1.727827in}{2.368070in}}%
\pgfpathlineto{\pgfqpoint{1.727950in}{2.336125in}}%
\pgfpathlineto{\pgfqpoint{1.728898in}{2.423010in}}%
\pgfpathlineto{\pgfqpoint{1.728923in}{2.416896in}}%
\pgfpathlineto{\pgfqpoint{1.729821in}{2.335561in}}%
\pgfpathlineto{\pgfqpoint{1.730080in}{2.391272in}}%
\pgfpathlineto{\pgfqpoint{1.730757in}{2.424715in}}%
\pgfpathlineto{\pgfqpoint{1.730572in}{2.336672in}}%
\pgfpathlineto{\pgfqpoint{1.731027in}{2.350915in}}%
\pgfpathlineto{\pgfqpoint{1.731064in}{2.333106in}}%
\pgfpathlineto{\pgfqpoint{1.732000in}{2.425981in}}%
\pgfpathlineto{\pgfqpoint{1.732098in}{2.389141in}}%
\pgfpathlineto{\pgfqpoint{1.732615in}{2.421143in}}%
\pgfpathlineto{\pgfqpoint{1.732910in}{2.336057in}}%
\pgfpathlineto{\pgfqpoint{1.733009in}{2.353384in}}%
\pgfpathlineto{\pgfqpoint{1.733526in}{2.334436in}}%
\pgfpathlineto{\pgfqpoint{1.733858in}{2.426127in}}%
\pgfpathlineto{\pgfqpoint{1.734067in}{2.367587in}}%
\pgfpathlineto{\pgfqpoint{1.735089in}{2.421679in}}%
\pgfpathlineto{\pgfqpoint{1.734769in}{2.331410in}}%
\pgfpathlineto{\pgfqpoint{1.735200in}{2.392126in}}%
\pgfpathlineto{\pgfqpoint{1.735704in}{2.423739in}}%
\pgfpathlineto{\pgfqpoint{1.736012in}{2.338688in}}%
\pgfpathlineto{\pgfqpoint{1.736332in}{2.419113in}}%
\pgfpathlineto{\pgfqpoint{1.736627in}{2.330778in}}%
\pgfpathlineto{\pgfqpoint{1.736947in}{2.426538in}}%
\pgfpathlineto{\pgfqpoint{1.737464in}{2.398647in}}%
\pgfpathlineto{\pgfqpoint{1.737563in}{2.426543in}}%
\pgfpathlineto{\pgfqpoint{1.737858in}{2.335260in}}%
\pgfpathlineto{\pgfqpoint{1.738326in}{2.387781in}}%
\pgfpathlineto{\pgfqpoint{1.738473in}{2.335679in}}%
\pgfpathlineto{\pgfqpoint{1.738806in}{2.424037in}}%
\pgfpathlineto{\pgfqpoint{1.739409in}{2.422475in}}%
\pgfpathlineto{\pgfqpoint{1.739421in}{2.424367in}}%
\pgfpathlineto{\pgfqpoint{1.739717in}{2.329171in}}%
\pgfpathlineto{\pgfqpoint{1.740184in}{2.379272in}}%
\pgfpathlineto{\pgfqpoint{1.740332in}{2.333279in}}%
\pgfpathlineto{\pgfqpoint{1.740652in}{2.427756in}}%
\pgfpathlineto{\pgfqpoint{1.741267in}{2.415173in}}%
\pgfpathlineto{\pgfqpoint{1.741280in}{2.416369in}}%
\pgfpathlineto{\pgfqpoint{1.741538in}{2.344166in}}%
\pgfpathlineto{\pgfqpoint{1.741563in}{2.331989in}}%
\pgfpathlineto{\pgfqpoint{1.742510in}{2.429306in}}%
\pgfpathlineto{\pgfqpoint{1.742596in}{2.368436in}}%
\pgfpathlineto{\pgfqpoint{1.743753in}{2.428338in}}%
\pgfpathlineto{\pgfqpoint{1.742806in}{2.329900in}}%
\pgfpathlineto{\pgfqpoint{1.743766in}{2.422620in}}%
\pgfpathlineto{\pgfqpoint{1.744664in}{2.328974in}}%
\pgfpathlineto{\pgfqpoint{1.744898in}{2.404332in}}%
\pgfpathlineto{\pgfqpoint{1.745895in}{2.331297in}}%
\pgfpathlineto{\pgfqpoint{1.745612in}{2.429684in}}%
\pgfpathlineto{\pgfqpoint{1.746104in}{2.392927in}}%
\pgfpathlineto{\pgfqpoint{1.746843in}{2.426783in}}%
\pgfpathlineto{\pgfqpoint{1.746510in}{2.325542in}}%
\pgfpathlineto{\pgfqpoint{1.747187in}{2.373230in}}%
\pgfpathlineto{\pgfqpoint{1.747753in}{2.328302in}}%
\pgfpathlineto{\pgfqpoint{1.747458in}{2.425249in}}%
\pgfpathlineto{\pgfqpoint{1.748295in}{2.373833in}}%
\pgfpathlineto{\pgfqpoint{1.748701in}{2.431278in}}%
\pgfpathlineto{\pgfqpoint{1.748369in}{2.323177in}}%
\pgfpathlineto{\pgfqpoint{1.749083in}{2.331928in}}%
\pgfpathlineto{\pgfqpoint{1.749612in}{2.321781in}}%
\pgfpathlineto{\pgfqpoint{1.749316in}{2.426526in}}%
\pgfpathlineto{\pgfqpoint{1.749932in}{2.421915in}}%
\pgfpathlineto{\pgfqpoint{1.750560in}{2.431275in}}%
\pgfpathlineto{\pgfqpoint{1.750227in}{2.323409in}}%
\pgfpathlineto{\pgfqpoint{1.750806in}{2.369306in}}%
\pgfpathlineto{\pgfqpoint{1.751458in}{2.316794in}}%
\pgfpathlineto{\pgfqpoint{1.751803in}{2.436392in}}%
\pgfpathlineto{\pgfqpoint{1.751901in}{2.400080in}}%
\pgfpathlineto{\pgfqpoint{1.752701in}{2.324994in}}%
\pgfpathlineto{\pgfqpoint{1.752418in}{2.438792in}}%
\pgfpathlineto{\pgfqpoint{1.752972in}{2.402252in}}%
\pgfpathlineto{\pgfqpoint{1.753649in}{2.435710in}}%
\pgfpathlineto{\pgfqpoint{1.753316in}{2.318702in}}%
\pgfpathlineto{\pgfqpoint{1.753920in}{2.330394in}}%
\pgfpathlineto{\pgfqpoint{1.754547in}{2.313952in}}%
\pgfpathlineto{\pgfqpoint{1.754264in}{2.434926in}}%
\pgfpathlineto{\pgfqpoint{1.754867in}{2.423660in}}%
\pgfpathlineto{\pgfqpoint{1.755507in}{2.441339in}}%
\pgfpathlineto{\pgfqpoint{1.755163in}{2.319349in}}%
\pgfpathlineto{\pgfqpoint{1.755766in}{2.334927in}}%
\pgfpathlineto{\pgfqpoint{1.756406in}{2.314275in}}%
\pgfpathlineto{\pgfqpoint{1.756750in}{2.437708in}}%
\pgfpathlineto{\pgfqpoint{1.756812in}{2.375157in}}%
\pgfpathlineto{\pgfqpoint{1.757366in}{2.438703in}}%
\pgfpathlineto{\pgfqpoint{1.757636in}{2.317514in}}%
\pgfpathlineto{\pgfqpoint{1.758006in}{2.420514in}}%
\pgfpathlineto{\pgfqpoint{1.758252in}{2.313926in}}%
\pgfpathlineto{\pgfqpoint{1.758596in}{2.440861in}}%
\pgfpathlineto{\pgfqpoint{1.759113in}{2.414823in}}%
\pgfpathlineto{\pgfqpoint{1.760110in}{2.310282in}}%
\pgfpathlineto{\pgfqpoint{1.759212in}{2.434528in}}%
\pgfpathlineto{\pgfqpoint{1.760307in}{2.363326in}}%
\pgfpathlineto{\pgfqpoint{1.760455in}{2.439026in}}%
\pgfpathlineto{\pgfqpoint{1.761341in}{2.312354in}}%
\pgfpathlineto{\pgfqpoint{1.761403in}{2.357603in}}%
\pgfpathlineto{\pgfqpoint{1.761969in}{2.315512in}}%
\pgfpathlineto{\pgfqpoint{1.762313in}{2.440051in}}%
\pgfpathlineto{\pgfqpoint{1.762498in}{2.369721in}}%
\pgfpathlineto{\pgfqpoint{1.763544in}{2.443985in}}%
\pgfpathlineto{\pgfqpoint{1.763199in}{2.307279in}}%
\pgfpathlineto{\pgfqpoint{1.763618in}{2.385270in}}%
\pgfpathlineto{\pgfqpoint{1.764159in}{2.441779in}}%
\pgfpathlineto{\pgfqpoint{1.763815in}{2.311748in}}%
\pgfpathlineto{\pgfqpoint{1.764799in}{2.428712in}}%
\pgfpathlineto{\pgfqpoint{1.765058in}{2.312738in}}%
\pgfpathlineto{\pgfqpoint{1.765390in}{2.443327in}}%
\pgfpathlineto{\pgfqpoint{1.765932in}{2.402612in}}%
\pgfpathlineto{\pgfqpoint{1.765993in}{2.443291in}}%
\pgfpathlineto{\pgfqpoint{1.766289in}{2.309951in}}%
\pgfpathlineto{\pgfqpoint{1.766879in}{2.348472in}}%
\pgfpathlineto{\pgfqpoint{1.767643in}{2.313062in}}%
\pgfpathlineto{\pgfqpoint{1.767224in}{2.447681in}}%
\pgfpathlineto{\pgfqpoint{1.767950in}{2.385272in}}%
\pgfpathlineto{\pgfqpoint{1.769083in}{2.452327in}}%
\pgfpathlineto{\pgfqpoint{1.768147in}{2.312545in}}%
\pgfpathlineto{\pgfqpoint{1.769119in}{2.435721in}}%
\pgfpathlineto{\pgfqpoint{1.770006in}{2.316857in}}%
\pgfpathlineto{\pgfqpoint{1.769698in}{2.449750in}}%
\pgfpathlineto{\pgfqpoint{1.770264in}{2.396424in}}%
\pgfpathlineto{\pgfqpoint{1.770326in}{2.454877in}}%
\pgfpathlineto{\pgfqpoint{1.770744in}{2.315978in}}%
\pgfpathlineto{\pgfqpoint{1.771298in}{2.362401in}}%
\pgfpathlineto{\pgfqpoint{1.771987in}{2.314705in}}%
\pgfpathlineto{\pgfqpoint{1.772172in}{2.452268in}}%
\pgfpathlineto{\pgfqpoint{1.772393in}{2.365686in}}%
\pgfpathlineto{\pgfqpoint{1.773415in}{2.456147in}}%
\pgfpathlineto{\pgfqpoint{1.772603in}{2.312156in}}%
\pgfpathlineto{\pgfqpoint{1.773563in}{2.391469in}}%
\pgfpathlineto{\pgfqpoint{1.773846in}{2.312305in}}%
\pgfpathlineto{\pgfqpoint{1.774030in}{2.457160in}}%
\pgfpathlineto{\pgfqpoint{1.774621in}{2.426313in}}%
\pgfpathlineto{\pgfqpoint{1.775273in}{2.460792in}}%
\pgfpathlineto{\pgfqpoint{1.775089in}{2.313814in}}%
\pgfpathlineto{\pgfqpoint{1.775655in}{2.350112in}}%
\pgfpathlineto{\pgfqpoint{1.776332in}{2.304391in}}%
\pgfpathlineto{\pgfqpoint{1.775889in}{2.459209in}}%
\pgfpathlineto{\pgfqpoint{1.776504in}{2.453981in}}%
\pgfpathlineto{\pgfqpoint{1.776516in}{2.456309in}}%
\pgfpathlineto{\pgfqpoint{1.776824in}{2.324117in}}%
\pgfpathlineto{\pgfqpoint{1.776923in}{2.337950in}}%
\pgfpathlineto{\pgfqpoint{1.777575in}{2.303538in}}%
\pgfpathlineto{\pgfqpoint{1.777132in}{2.461507in}}%
\pgfpathlineto{\pgfqpoint{1.778006in}{2.367222in}}%
\pgfpathlineto{\pgfqpoint{1.778116in}{2.369304in}}%
\pgfpathlineto{\pgfqpoint{1.778178in}{2.315056in}}%
\pgfpathlineto{\pgfqpoint{1.778806in}{2.299258in}}%
\pgfpathlineto{\pgfqpoint{1.778990in}{2.464253in}}%
\pgfpathlineto{\pgfqpoint{1.779224in}{2.364873in}}%
\pgfpathlineto{\pgfqpoint{1.780233in}{2.462484in}}%
\pgfpathlineto{\pgfqpoint{1.779433in}{2.297883in}}%
\pgfpathlineto{\pgfqpoint{1.780381in}{2.392975in}}%
\pgfpathlineto{\pgfqpoint{1.780664in}{2.296288in}}%
\pgfpathlineto{\pgfqpoint{1.780849in}{2.465868in}}%
\pgfpathlineto{\pgfqpoint{1.781439in}{2.437589in}}%
\pgfpathlineto{\pgfqpoint{1.782092in}{2.466341in}}%
\pgfpathlineto{\pgfqpoint{1.781907in}{2.291039in}}%
\pgfpathlineto{\pgfqpoint{1.782473in}{2.358127in}}%
\pgfpathlineto{\pgfqpoint{1.782535in}{2.291084in}}%
\pgfpathlineto{\pgfqpoint{1.782707in}{2.462697in}}%
\pgfpathlineto{\pgfqpoint{1.783569in}{2.366731in}}%
\pgfpathlineto{\pgfqpoint{1.783950in}{2.466845in}}%
\pgfpathlineto{\pgfqpoint{1.783766in}{2.287634in}}%
\pgfpathlineto{\pgfqpoint{1.784726in}{2.391982in}}%
\pgfpathlineto{\pgfqpoint{1.785009in}{2.286479in}}%
\pgfpathlineto{\pgfqpoint{1.785193in}{2.463517in}}%
\pgfpathlineto{\pgfqpoint{1.785772in}{2.423908in}}%
\pgfpathlineto{\pgfqpoint{1.785809in}{2.461490in}}%
\pgfpathlineto{\pgfqpoint{1.786252in}{2.293479in}}%
\pgfpathlineto{\pgfqpoint{1.786818in}{2.360486in}}%
\pgfpathlineto{\pgfqpoint{1.786867in}{2.291604in}}%
\pgfpathlineto{\pgfqpoint{1.787052in}{2.463646in}}%
\pgfpathlineto{\pgfqpoint{1.787913in}{2.380432in}}%
\pgfpathlineto{\pgfqpoint{1.788110in}{2.301565in}}%
\pgfpathlineto{\pgfqpoint{1.788282in}{2.459024in}}%
\pgfpathlineto{\pgfqpoint{1.788886in}{2.436126in}}%
\pgfpathlineto{\pgfqpoint{1.788910in}{2.456260in}}%
\pgfpathlineto{\pgfqpoint{1.789341in}{2.311505in}}%
\pgfpathlineto{\pgfqpoint{1.789919in}{2.365623in}}%
\pgfpathlineto{\pgfqpoint{1.789969in}{2.313859in}}%
\pgfpathlineto{\pgfqpoint{1.790153in}{2.449742in}}%
\pgfpathlineto{\pgfqpoint{1.791015in}{2.385294in}}%
\pgfpathlineto{\pgfqpoint{1.791212in}{2.327481in}}%
\pgfpathlineto{\pgfqpoint{1.791384in}{2.430556in}}%
\pgfpathlineto{\pgfqpoint{1.792098in}{2.380158in}}%
\pgfpathlineto{\pgfqpoint{1.792627in}{2.424410in}}%
\pgfpathlineto{\pgfqpoint{1.792922in}{2.341247in}}%
\pgfpathlineto{\pgfqpoint{1.793218in}{2.401469in}}%
\pgfpathlineto{\pgfqpoint{1.793242in}{2.421099in}}%
\pgfpathlineto{\pgfqpoint{1.793919in}{2.331867in}}%
\pgfpathlineto{\pgfqpoint{1.794289in}{2.364144in}}%
\pgfpathlineto{\pgfqpoint{1.794535in}{2.332671in}}%
\pgfpathlineto{\pgfqpoint{1.795089in}{2.424794in}}%
\pgfpathlineto{\pgfqpoint{1.795384in}{2.371278in}}%
\pgfpathlineto{\pgfqpoint{1.795778in}{2.332212in}}%
\pgfpathlineto{\pgfqpoint{1.796332in}{2.413540in}}%
\pgfpathlineto{\pgfqpoint{1.796344in}{2.413903in}}%
\pgfpathlineto{\pgfqpoint{1.796356in}{2.397730in}}%
\pgfpathlineto{\pgfqpoint{1.797021in}{2.322153in}}%
\pgfpathlineto{\pgfqpoint{1.797181in}{2.406387in}}%
\pgfpathlineto{\pgfqpoint{1.797476in}{2.378276in}}%
\pgfpathlineto{\pgfqpoint{1.797636in}{2.313438in}}%
\pgfpathlineto{\pgfqpoint{1.797796in}{2.416667in}}%
\pgfpathlineto{\pgfqpoint{1.798584in}{2.367355in}}%
\pgfpathlineto{\pgfqpoint{1.799655in}{2.420050in}}%
\pgfpathlineto{\pgfqpoint{1.799495in}{2.324509in}}%
\pgfpathlineto{\pgfqpoint{1.799704in}{2.375796in}}%
\pgfpathlineto{\pgfqpoint{1.800381in}{2.421756in}}%
\pgfpathlineto{\pgfqpoint{1.800110in}{2.320565in}}%
\pgfpathlineto{\pgfqpoint{1.800713in}{2.342159in}}%
\pgfpathlineto{\pgfqpoint{1.801353in}{2.319102in}}%
\pgfpathlineto{\pgfqpoint{1.800898in}{2.427011in}}%
\pgfpathlineto{\pgfqpoint{1.801809in}{2.362213in}}%
\pgfpathlineto{\pgfqpoint{1.802584in}{2.324744in}}%
\pgfpathlineto{\pgfqpoint{1.802227in}{2.430112in}}%
\pgfpathlineto{\pgfqpoint{1.802719in}{2.417675in}}%
\pgfpathlineto{\pgfqpoint{1.803470in}{2.437994in}}%
\pgfpathlineto{\pgfqpoint{1.803199in}{2.321726in}}%
\pgfpathlineto{\pgfqpoint{1.803679in}{2.345836in}}%
\pgfpathlineto{\pgfqpoint{1.804442in}{2.317118in}}%
\pgfpathlineto{\pgfqpoint{1.804085in}{2.435579in}}%
\pgfpathlineto{\pgfqpoint{1.804553in}{2.382967in}}%
\pgfpathlineto{\pgfqpoint{1.805329in}{2.438604in}}%
\pgfpathlineto{\pgfqpoint{1.805058in}{2.323212in}}%
\pgfpathlineto{\pgfqpoint{1.805612in}{2.333348in}}%
\pgfpathlineto{\pgfqpoint{1.806301in}{2.317403in}}%
\pgfpathlineto{\pgfqpoint{1.805944in}{2.440453in}}%
\pgfpathlineto{\pgfqpoint{1.806449in}{2.421278in}}%
\pgfpathlineto{\pgfqpoint{1.806559in}{2.444212in}}%
\pgfpathlineto{\pgfqpoint{1.806916in}{2.319895in}}%
\pgfpathlineto{\pgfqpoint{1.807285in}{2.375101in}}%
\pgfpathlineto{\pgfqpoint{1.808110in}{2.318081in}}%
\pgfpathlineto{\pgfqpoint{1.807802in}{2.448959in}}%
\pgfpathlineto{\pgfqpoint{1.808369in}{2.394557in}}%
\pgfpathlineto{\pgfqpoint{1.809033in}{2.453175in}}%
\pgfpathlineto{\pgfqpoint{1.809353in}{2.312200in}}%
\pgfpathlineto{\pgfqpoint{1.809452in}{2.372872in}}%
\pgfpathlineto{\pgfqpoint{1.809661in}{2.455964in}}%
\pgfpathlineto{\pgfqpoint{1.809932in}{2.335857in}}%
\pgfpathlineto{\pgfqpoint{1.809969in}{2.309458in}}%
\pgfpathlineto{\pgfqpoint{1.810276in}{2.458495in}}%
\pgfpathlineto{\pgfqpoint{1.810892in}{2.453715in}}%
\pgfpathlineto{\pgfqpoint{1.810941in}{2.420575in}}%
\pgfpathlineto{\pgfqpoint{1.811827in}{2.308312in}}%
\pgfpathlineto{\pgfqpoint{1.811519in}{2.456416in}}%
\pgfpathlineto{\pgfqpoint{1.812085in}{2.374474in}}%
\pgfpathlineto{\pgfqpoint{1.812135in}{2.461088in}}%
\pgfpathlineto{\pgfqpoint{1.812455in}{2.309000in}}%
\pgfpathlineto{\pgfqpoint{1.813193in}{2.388149in}}%
\pgfpathlineto{\pgfqpoint{1.813710in}{2.314718in}}%
\pgfpathlineto{\pgfqpoint{1.813378in}{2.454356in}}%
\pgfpathlineto{\pgfqpoint{1.813845in}{2.415740in}}%
\pgfpathlineto{\pgfqpoint{1.814609in}{2.455563in}}%
\pgfpathlineto{\pgfqpoint{1.814325in}{2.311010in}}%
\pgfpathlineto{\pgfqpoint{1.814879in}{2.352056in}}%
\pgfpathlineto{\pgfqpoint{1.814941in}{2.310371in}}%
\pgfpathlineto{\pgfqpoint{1.815224in}{2.454431in}}%
\pgfpathlineto{\pgfqpoint{1.815839in}{2.447627in}}%
\pgfpathlineto{\pgfqpoint{1.815852in}{2.448873in}}%
\pgfpathlineto{\pgfqpoint{1.815962in}{2.356752in}}%
\pgfpathlineto{\pgfqpoint{1.816012in}{2.364986in}}%
\pgfpathlineto{\pgfqpoint{1.816799in}{2.310293in}}%
\pgfpathlineto{\pgfqpoint{1.816467in}{2.451020in}}%
\pgfpathlineto{\pgfqpoint{1.817070in}{2.435307in}}%
\pgfpathlineto{\pgfqpoint{1.817082in}{2.448964in}}%
\pgfpathlineto{\pgfqpoint{1.818030in}{2.308232in}}%
\pgfpathlineto{\pgfqpoint{1.818129in}{2.387006in}}%
\pgfpathlineto{\pgfqpoint{1.818153in}{2.383529in}}%
\pgfpathlineto{\pgfqpoint{1.818165in}{2.398091in}}%
\pgfpathlineto{\pgfqpoint{1.818941in}{2.443426in}}%
\pgfpathlineto{\pgfqpoint{1.818658in}{2.306961in}}%
\pgfpathlineto{\pgfqpoint{1.819224in}{2.341274in}}%
\pgfpathlineto{\pgfqpoint{1.819273in}{2.303961in}}%
\pgfpathlineto{\pgfqpoint{1.820061in}{2.446556in}}%
\pgfpathlineto{\pgfqpoint{1.820307in}{2.370527in}}%
\pgfpathlineto{\pgfqpoint{1.820676in}{2.449213in}}%
\pgfpathlineto{\pgfqpoint{1.820504in}{2.300930in}}%
\pgfpathlineto{\pgfqpoint{1.821107in}{2.319563in}}%
\pgfpathlineto{\pgfqpoint{1.821747in}{2.298495in}}%
\pgfpathlineto{\pgfqpoint{1.821919in}{2.449545in}}%
\pgfpathlineto{\pgfqpoint{1.822165in}{2.366938in}}%
\pgfpathlineto{\pgfqpoint{1.822645in}{2.454474in}}%
\pgfpathlineto{\pgfqpoint{1.822362in}{2.302209in}}%
\pgfpathlineto{\pgfqpoint{1.822978in}{2.306818in}}%
\pgfpathlineto{\pgfqpoint{1.822990in}{2.302378in}}%
\pgfpathlineto{\pgfqpoint{1.823261in}{2.453353in}}%
\pgfpathlineto{\pgfqpoint{1.823741in}{2.408157in}}%
\pgfpathlineto{\pgfqpoint{1.824504in}{2.463464in}}%
\pgfpathlineto{\pgfqpoint{1.824221in}{2.305278in}}%
\pgfpathlineto{\pgfqpoint{1.824812in}{2.329373in}}%
\pgfpathlineto{\pgfqpoint{1.824848in}{2.305074in}}%
\pgfpathlineto{\pgfqpoint{1.825747in}{2.470125in}}%
\pgfpathlineto{\pgfqpoint{1.825895in}{2.355971in}}%
\pgfpathlineto{\pgfqpoint{1.826362in}{2.472944in}}%
\pgfpathlineto{\pgfqpoint{1.826695in}{2.299461in}}%
\pgfpathlineto{\pgfqpoint{1.827064in}{2.390240in}}%
\pgfpathlineto{\pgfqpoint{1.827938in}{2.296982in}}%
\pgfpathlineto{\pgfqpoint{1.827605in}{2.468786in}}%
\pgfpathlineto{\pgfqpoint{1.828172in}{2.360218in}}%
\pgfpathlineto{\pgfqpoint{1.828221in}{2.468883in}}%
\pgfpathlineto{\pgfqpoint{1.829033in}{2.293312in}}%
\pgfpathlineto{\pgfqpoint{1.829328in}{2.456616in}}%
\pgfpathlineto{\pgfqpoint{1.829341in}{2.457453in}}%
\pgfpathlineto{\pgfqpoint{1.829365in}{2.412047in}}%
\pgfpathlineto{\pgfqpoint{1.830276in}{2.288040in}}%
\pgfpathlineto{\pgfqpoint{1.829452in}{2.470777in}}%
\pgfpathlineto{\pgfqpoint{1.830473in}{2.405984in}}%
\pgfpathlineto{\pgfqpoint{1.830892in}{2.285621in}}%
\pgfpathlineto{\pgfqpoint{1.830695in}{2.477359in}}%
\pgfpathlineto{\pgfqpoint{1.831175in}{2.451664in}}%
\pgfpathlineto{\pgfqpoint{1.831925in}{2.480572in}}%
\pgfpathlineto{\pgfqpoint{1.831507in}{2.286539in}}%
\pgfpathlineto{\pgfqpoint{1.832085in}{2.358385in}}%
\pgfpathlineto{\pgfqpoint{1.832750in}{2.281771in}}%
\pgfpathlineto{\pgfqpoint{1.832553in}{2.481829in}}%
\pgfpathlineto{\pgfqpoint{1.833144in}{2.446792in}}%
\pgfpathlineto{\pgfqpoint{1.833784in}{2.487325in}}%
\pgfpathlineto{\pgfqpoint{1.834116in}{2.280065in}}%
\pgfpathlineto{\pgfqpoint{1.834227in}{2.389882in}}%
\pgfpathlineto{\pgfqpoint{1.834399in}{2.489991in}}%
\pgfpathlineto{\pgfqpoint{1.834608in}{2.282631in}}%
\pgfpathlineto{\pgfqpoint{1.834719in}{2.293775in}}%
\pgfpathlineto{\pgfqpoint{1.835359in}{2.276445in}}%
\pgfpathlineto{\pgfqpoint{1.835642in}{2.498042in}}%
\pgfpathlineto{\pgfqpoint{1.835778in}{2.347051in}}%
\pgfpathlineto{\pgfqpoint{1.836873in}{2.500410in}}%
\pgfpathlineto{\pgfqpoint{1.836590in}{2.264910in}}%
\pgfpathlineto{\pgfqpoint{1.836959in}{2.402569in}}%
\pgfpathlineto{\pgfqpoint{1.837205in}{2.265037in}}%
\pgfpathlineto{\pgfqpoint{1.837501in}{2.498953in}}%
\pgfpathlineto{\pgfqpoint{1.838067in}{2.384477in}}%
\pgfpathlineto{\pgfqpoint{1.838732in}{2.506340in}}%
\pgfpathlineto{\pgfqpoint{1.839064in}{2.262886in}}%
\pgfpathlineto{\pgfqpoint{1.839175in}{2.392775in}}%
\pgfpathlineto{\pgfqpoint{1.839975in}{2.507616in}}%
\pgfpathlineto{\pgfqpoint{1.839679in}{2.263812in}}%
\pgfpathlineto{\pgfqpoint{1.840147in}{2.305592in}}%
\pgfpathlineto{\pgfqpoint{1.840922in}{2.260140in}}%
\pgfpathlineto{\pgfqpoint{1.840590in}{2.508289in}}%
\pgfpathlineto{\pgfqpoint{1.841168in}{2.413224in}}%
\pgfpathlineto{\pgfqpoint{1.841205in}{2.508135in}}%
\pgfpathlineto{\pgfqpoint{1.842153in}{2.253920in}}%
\pgfpathlineto{\pgfqpoint{1.842264in}{2.395009in}}%
\pgfpathlineto{\pgfqpoint{1.843064in}{2.513172in}}%
\pgfpathlineto{\pgfqpoint{1.842768in}{2.253524in}}%
\pgfpathlineto{\pgfqpoint{1.843224in}{2.338163in}}%
\pgfpathlineto{\pgfqpoint{1.844011in}{2.244517in}}%
\pgfpathlineto{\pgfqpoint{1.843679in}{2.512450in}}%
\pgfpathlineto{\pgfqpoint{1.844270in}{2.444306in}}%
\pgfpathlineto{\pgfqpoint{1.844307in}{2.515290in}}%
\pgfpathlineto{\pgfqpoint{1.845242in}{2.243893in}}%
\pgfpathlineto{\pgfqpoint{1.845365in}{2.400377in}}%
\pgfpathlineto{\pgfqpoint{1.846165in}{2.510636in}}%
\pgfpathlineto{\pgfqpoint{1.845858in}{2.243511in}}%
\pgfpathlineto{\pgfqpoint{1.846411in}{2.315229in}}%
\pgfpathlineto{\pgfqpoint{1.847101in}{2.232168in}}%
\pgfpathlineto{\pgfqpoint{1.846781in}{2.511480in}}%
\pgfpathlineto{\pgfqpoint{1.847359in}{2.433366in}}%
\pgfpathlineto{\pgfqpoint{1.847396in}{2.509501in}}%
\pgfpathlineto{\pgfqpoint{1.848344in}{2.228824in}}%
\pgfpathlineto{\pgfqpoint{1.848455in}{2.388960in}}%
\pgfpathlineto{\pgfqpoint{1.849255in}{2.513811in}}%
\pgfpathlineto{\pgfqpoint{1.848959in}{2.225902in}}%
\pgfpathlineto{\pgfqpoint{1.849439in}{2.285648in}}%
\pgfpathlineto{\pgfqpoint{1.849575in}{2.224884in}}%
\pgfpathlineto{\pgfqpoint{1.849870in}{2.516536in}}%
\pgfpathlineto{\pgfqpoint{1.850448in}{2.427685in}}%
\pgfpathlineto{\pgfqpoint{1.851113in}{2.517643in}}%
\pgfpathlineto{\pgfqpoint{1.851433in}{2.223820in}}%
\pgfpathlineto{\pgfqpoint{1.851544in}{2.383249in}}%
\pgfpathlineto{\pgfqpoint{1.851728in}{2.514242in}}%
\pgfpathlineto{\pgfqpoint{1.852048in}{2.227741in}}%
\pgfpathlineto{\pgfqpoint{1.852528in}{2.289463in}}%
\pgfpathlineto{\pgfqpoint{1.853291in}{2.228778in}}%
\pgfpathlineto{\pgfqpoint{1.853464in}{2.517644in}}%
\pgfpathlineto{\pgfqpoint{1.853562in}{2.494146in}}%
\pgfpathlineto{\pgfqpoint{1.854079in}{2.515331in}}%
\pgfpathlineto{\pgfqpoint{1.853907in}{2.230388in}}%
\pgfpathlineto{\pgfqpoint{1.854473in}{2.292657in}}%
\pgfpathlineto{\pgfqpoint{1.854535in}{2.230170in}}%
\pgfpathlineto{\pgfqpoint{1.854670in}{2.439359in}}%
\pgfpathlineto{\pgfqpoint{1.855322in}{2.516265in}}%
\pgfpathlineto{\pgfqpoint{1.855150in}{2.229183in}}%
\pgfpathlineto{\pgfqpoint{1.855728in}{2.283450in}}%
\pgfpathlineto{\pgfqpoint{1.856393in}{2.229662in}}%
\pgfpathlineto{\pgfqpoint{1.856553in}{2.526120in}}%
\pgfpathlineto{\pgfqpoint{1.856799in}{2.351034in}}%
\pgfpathlineto{\pgfqpoint{1.857168in}{2.526085in}}%
\pgfpathlineto{\pgfqpoint{1.857008in}{2.225589in}}%
\pgfpathlineto{\pgfqpoint{1.857993in}{2.433050in}}%
\pgfpathlineto{\pgfqpoint{1.858239in}{2.230455in}}%
\pgfpathlineto{\pgfqpoint{1.858411in}{2.523162in}}%
\pgfpathlineto{\pgfqpoint{1.859101in}{2.422426in}}%
\pgfpathlineto{\pgfqpoint{1.859642in}{2.523807in}}%
\pgfpathlineto{\pgfqpoint{1.860098in}{2.232106in}}%
\pgfpathlineto{\pgfqpoint{1.860196in}{2.378565in}}%
\pgfpathlineto{\pgfqpoint{1.860258in}{2.522987in}}%
\pgfpathlineto{\pgfqpoint{1.860713in}{2.234819in}}%
\pgfpathlineto{\pgfqpoint{1.861168in}{2.319736in}}%
\pgfpathlineto{\pgfqpoint{1.861944in}{2.233160in}}%
\pgfpathlineto{\pgfqpoint{1.861488in}{2.516224in}}%
\pgfpathlineto{\pgfqpoint{1.862227in}{2.496542in}}%
\pgfpathlineto{\pgfqpoint{1.862239in}{2.496520in}}%
\pgfpathlineto{\pgfqpoint{1.863064in}{2.230250in}}%
\pgfpathlineto{\pgfqpoint{1.862731in}{2.512233in}}%
\pgfpathlineto{\pgfqpoint{1.863335in}{2.496593in}}%
\pgfpathlineto{\pgfqpoint{1.863347in}{2.511488in}}%
\pgfpathlineto{\pgfqpoint{1.863679in}{2.227641in}}%
\pgfpathlineto{\pgfqpoint{1.864258in}{2.320233in}}%
\pgfpathlineto{\pgfqpoint{1.864922in}{2.221714in}}%
\pgfpathlineto{\pgfqpoint{1.864590in}{2.504489in}}%
\pgfpathlineto{\pgfqpoint{1.865316in}{2.488324in}}%
\pgfpathlineto{\pgfqpoint{1.865328in}{2.490542in}}%
\pgfpathlineto{\pgfqpoint{1.865488in}{2.335047in}}%
\pgfpathlineto{\pgfqpoint{1.866153in}{2.215805in}}%
\pgfpathlineto{\pgfqpoint{1.865821in}{2.502639in}}%
\pgfpathlineto{\pgfqpoint{1.866547in}{2.484207in}}%
\pgfpathlineto{\pgfqpoint{1.867064in}{2.499206in}}%
\pgfpathlineto{\pgfqpoint{1.866768in}{2.212633in}}%
\pgfpathlineto{\pgfqpoint{1.867334in}{2.343129in}}%
\pgfpathlineto{\pgfqpoint{1.868011in}{2.207468in}}%
\pgfpathlineto{\pgfqpoint{1.867679in}{2.508025in}}%
\pgfpathlineto{\pgfqpoint{1.868381in}{2.447476in}}%
\pgfpathlineto{\pgfqpoint{1.868910in}{2.509443in}}%
\pgfpathlineto{\pgfqpoint{1.869242in}{2.210708in}}%
\pgfpathlineto{\pgfqpoint{1.869365in}{2.242394in}}%
\pgfpathlineto{\pgfqpoint{1.870153in}{2.511844in}}%
\pgfpathlineto{\pgfqpoint{1.869870in}{2.209287in}}%
\pgfpathlineto{\pgfqpoint{1.870448in}{2.257791in}}%
\pgfpathlineto{\pgfqpoint{1.871101in}{2.204956in}}%
\pgfpathlineto{\pgfqpoint{1.870768in}{2.511853in}}%
\pgfpathlineto{\pgfqpoint{1.871470in}{2.443317in}}%
\pgfpathlineto{\pgfqpoint{1.872011in}{2.514875in}}%
\pgfpathlineto{\pgfqpoint{1.872344in}{2.208054in}}%
\pgfpathlineto{\pgfqpoint{1.872454in}{2.246298in}}%
\pgfpathlineto{\pgfqpoint{1.872479in}{2.273528in}}%
\pgfpathlineto{\pgfqpoint{1.873378in}{2.518747in}}%
\pgfpathlineto{\pgfqpoint{1.872959in}{2.205916in}}%
\pgfpathlineto{\pgfqpoint{1.873562in}{2.211683in}}%
\pgfpathlineto{\pgfqpoint{1.873574in}{2.206957in}}%
\pgfpathlineto{\pgfqpoint{1.873858in}{2.517276in}}%
\pgfpathlineto{\pgfqpoint{1.873956in}{2.468722in}}%
\pgfpathlineto{\pgfqpoint{1.874621in}{2.522581in}}%
\pgfpathlineto{\pgfqpoint{1.874190in}{2.205224in}}%
\pgfpathlineto{\pgfqpoint{1.874928in}{2.242401in}}%
\pgfpathlineto{\pgfqpoint{1.874953in}{2.271907in}}%
\pgfpathlineto{\pgfqpoint{1.875236in}{2.528474in}}%
\pgfpathlineto{\pgfqpoint{1.875421in}{2.208668in}}%
\pgfpathlineto{\pgfqpoint{1.876036in}{2.209722in}}%
\pgfpathlineto{\pgfqpoint{1.876048in}{2.208969in}}%
\pgfpathlineto{\pgfqpoint{1.876085in}{2.258609in}}%
\pgfpathlineto{\pgfqpoint{1.876467in}{2.531829in}}%
\pgfpathlineto{\pgfqpoint{1.876664in}{2.205649in}}%
\pgfpathlineto{\pgfqpoint{1.877218in}{2.299157in}}%
\pgfpathlineto{\pgfqpoint{1.877894in}{2.201259in}}%
\pgfpathlineto{\pgfqpoint{1.877698in}{2.532328in}}%
\pgfpathlineto{\pgfqpoint{1.878178in}{2.508309in}}%
\pgfpathlineto{\pgfqpoint{1.878941in}{2.539503in}}%
\pgfpathlineto{\pgfqpoint{1.878510in}{2.200239in}}%
\pgfpathlineto{\pgfqpoint{1.879088in}{2.269593in}}%
\pgfpathlineto{\pgfqpoint{1.879741in}{2.200615in}}%
\pgfpathlineto{\pgfqpoint{1.879556in}{2.539274in}}%
\pgfpathlineto{\pgfqpoint{1.880122in}{2.431406in}}%
\pgfpathlineto{\pgfqpoint{1.880171in}{2.537224in}}%
\pgfpathlineto{\pgfqpoint{1.880368in}{2.196939in}}%
\pgfpathlineto{\pgfqpoint{1.881230in}{2.433226in}}%
\pgfpathlineto{\pgfqpoint{1.882030in}{2.541406in}}%
\pgfpathlineto{\pgfqpoint{1.881599in}{2.199574in}}%
\pgfpathlineto{\pgfqpoint{1.882190in}{2.245819in}}%
\pgfpathlineto{\pgfqpoint{1.882227in}{2.197029in}}%
\pgfpathlineto{\pgfqpoint{1.883138in}{2.536931in}}%
\pgfpathlineto{\pgfqpoint{1.883211in}{2.430434in}}%
\pgfpathlineto{\pgfqpoint{1.883888in}{2.538150in}}%
\pgfpathlineto{\pgfqpoint{1.884073in}{2.193801in}}%
\pgfpathlineto{\pgfqpoint{1.884319in}{2.419912in}}%
\pgfpathlineto{\pgfqpoint{1.885131in}{2.540890in}}%
\pgfpathlineto{\pgfqpoint{1.884688in}{2.194625in}}%
\pgfpathlineto{\pgfqpoint{1.885291in}{2.217883in}}%
\pgfpathlineto{\pgfqpoint{1.885304in}{2.196852in}}%
\pgfpathlineto{\pgfqpoint{1.885747in}{2.544899in}}%
\pgfpathlineto{\pgfqpoint{1.886214in}{2.506041in}}%
\pgfpathlineto{\pgfqpoint{1.886977in}{2.544224in}}%
\pgfpathlineto{\pgfqpoint{1.886547in}{2.194695in}}%
\pgfpathlineto{\pgfqpoint{1.887150in}{2.214003in}}%
\pgfpathlineto{\pgfqpoint{1.887162in}{2.193976in}}%
\pgfpathlineto{\pgfqpoint{1.887605in}{2.542712in}}%
\pgfpathlineto{\pgfqpoint{1.888073in}{2.504892in}}%
\pgfpathlineto{\pgfqpoint{1.888836in}{2.544191in}}%
\pgfpathlineto{\pgfqpoint{1.888405in}{2.193235in}}%
\pgfpathlineto{\pgfqpoint{1.888984in}{2.270958in}}%
\pgfpathlineto{\pgfqpoint{1.889021in}{2.194144in}}%
\pgfpathlineto{\pgfqpoint{1.889464in}{2.538927in}}%
\pgfpathlineto{\pgfqpoint{1.890017in}{2.430630in}}%
\pgfpathlineto{\pgfqpoint{1.890079in}{2.538497in}}%
\pgfpathlineto{\pgfqpoint{1.890879in}{2.192553in}}%
\pgfpathlineto{\pgfqpoint{1.891125in}{2.422596in}}%
\pgfpathlineto{\pgfqpoint{1.891937in}{2.535290in}}%
\pgfpathlineto{\pgfqpoint{1.891494in}{2.192959in}}%
\pgfpathlineto{\pgfqpoint{1.892110in}{2.199627in}}%
\pgfpathlineto{\pgfqpoint{1.892122in}{2.193998in}}%
\pgfpathlineto{\pgfqpoint{1.892417in}{2.505806in}}%
\pgfpathlineto{\pgfqpoint{1.892516in}{2.499829in}}%
\pgfpathlineto{\pgfqpoint{1.892541in}{2.532847in}}%
\pgfpathlineto{\pgfqpoint{1.893353in}{2.188205in}}%
\pgfpathlineto{\pgfqpoint{1.893476in}{2.249668in}}%
\pgfpathlineto{\pgfqpoint{1.893968in}{2.190596in}}%
\pgfpathlineto{\pgfqpoint{1.893796in}{2.531251in}}%
\pgfpathlineto{\pgfqpoint{1.894350in}{2.443323in}}%
\pgfpathlineto{\pgfqpoint{1.894411in}{2.530313in}}%
\pgfpathlineto{\pgfqpoint{1.895211in}{2.188377in}}%
\pgfpathlineto{\pgfqpoint{1.895457in}{2.437066in}}%
\pgfpathlineto{\pgfqpoint{1.896270in}{2.529239in}}%
\pgfpathlineto{\pgfqpoint{1.895827in}{2.187492in}}%
\pgfpathlineto{\pgfqpoint{1.896381in}{2.312215in}}%
\pgfpathlineto{\pgfqpoint{1.896442in}{2.185686in}}%
\pgfpathlineto{\pgfqpoint{1.896885in}{2.532208in}}%
\pgfpathlineto{\pgfqpoint{1.897341in}{2.468846in}}%
\pgfpathlineto{\pgfqpoint{1.898128in}{2.533816in}}%
\pgfpathlineto{\pgfqpoint{1.897685in}{2.185874in}}%
\pgfpathlineto{\pgfqpoint{1.898362in}{2.282353in}}%
\pgfpathlineto{\pgfqpoint{1.898436in}{2.244830in}}%
\pgfpathlineto{\pgfqpoint{1.898497in}{2.404520in}}%
\pgfpathlineto{\pgfqpoint{1.898744in}{2.539738in}}%
\pgfpathlineto{\pgfqpoint{1.898916in}{2.192544in}}%
\pgfpathlineto{\pgfqpoint{1.899494in}{2.286370in}}%
\pgfpathlineto{\pgfqpoint{1.900159in}{2.191371in}}%
\pgfpathlineto{\pgfqpoint{1.899987in}{2.540633in}}%
\pgfpathlineto{\pgfqpoint{1.900541in}{2.438236in}}%
\pgfpathlineto{\pgfqpoint{1.900602in}{2.540426in}}%
\pgfpathlineto{\pgfqpoint{1.900774in}{2.192756in}}%
\pgfpathlineto{\pgfqpoint{1.901673in}{2.477063in}}%
\pgfpathlineto{\pgfqpoint{1.902461in}{2.544976in}}%
\pgfpathlineto{\pgfqpoint{1.902017in}{2.194631in}}%
\pgfpathlineto{\pgfqpoint{1.902608in}{2.233146in}}%
\pgfpathlineto{\pgfqpoint{1.902633in}{2.195985in}}%
\pgfpathlineto{\pgfqpoint{1.903076in}{2.547027in}}%
\pgfpathlineto{\pgfqpoint{1.903630in}{2.413721in}}%
\pgfpathlineto{\pgfqpoint{1.903704in}{2.545238in}}%
\pgfpathlineto{\pgfqpoint{1.903876in}{2.196921in}}%
\pgfpathlineto{\pgfqpoint{1.904590in}{2.258844in}}%
\pgfpathlineto{\pgfqpoint{1.905230in}{2.216134in}}%
\pgfpathlineto{\pgfqpoint{1.905550in}{2.530749in}}%
\pgfpathlineto{\pgfqpoint{1.905648in}{2.369829in}}%
\pgfpathlineto{\pgfqpoint{1.906473in}{2.218181in}}%
\pgfpathlineto{\pgfqpoint{1.906177in}{2.541837in}}%
\pgfpathlineto{\pgfqpoint{1.906731in}{2.394403in}}%
\pgfpathlineto{\pgfqpoint{1.907408in}{2.550214in}}%
\pgfpathlineto{\pgfqpoint{1.906965in}{2.216757in}}%
\pgfpathlineto{\pgfqpoint{1.907888in}{2.480909in}}%
\pgfpathlineto{\pgfqpoint{1.908651in}{2.561262in}}%
\pgfpathlineto{\pgfqpoint{1.908196in}{2.229842in}}%
\pgfpathlineto{\pgfqpoint{1.908750in}{2.322592in}}%
\pgfpathlineto{\pgfqpoint{1.909464in}{2.224964in}}%
\pgfpathlineto{\pgfqpoint{1.909279in}{2.553374in}}%
\pgfpathlineto{\pgfqpoint{1.909833in}{2.412067in}}%
\pgfpathlineto{\pgfqpoint{1.909894in}{2.512675in}}%
\pgfpathlineto{\pgfqpoint{1.910190in}{2.212748in}}%
\pgfpathlineto{\pgfqpoint{1.910977in}{2.487642in}}%
\pgfpathlineto{\pgfqpoint{1.911310in}{2.217723in}}%
\pgfpathlineto{\pgfqpoint{1.911137in}{2.511037in}}%
\pgfpathlineto{\pgfqpoint{1.912134in}{2.422952in}}%
\pgfpathlineto{\pgfqpoint{1.912368in}{2.546852in}}%
\pgfpathlineto{\pgfqpoint{1.912528in}{2.240222in}}%
\pgfpathlineto{\pgfqpoint{1.913144in}{2.245015in}}%
\pgfpathlineto{\pgfqpoint{1.913599in}{2.546253in}}%
\pgfpathlineto{\pgfqpoint{1.913907in}{2.242938in}}%
\pgfpathlineto{\pgfqpoint{1.914325in}{2.325799in}}%
\pgfpathlineto{\pgfqpoint{1.914350in}{2.248599in}}%
\pgfpathlineto{\pgfqpoint{1.914817in}{2.507653in}}%
\pgfpathlineto{\pgfqpoint{1.915408in}{2.425129in}}%
\pgfpathlineto{\pgfqpoint{1.915925in}{2.490124in}}%
\pgfpathlineto{\pgfqpoint{1.915580in}{2.236004in}}%
\pgfpathlineto{\pgfqpoint{1.916528in}{2.445627in}}%
\pgfpathlineto{\pgfqpoint{1.917340in}{2.482315in}}%
\pgfpathlineto{\pgfqpoint{1.916811in}{2.249225in}}%
\pgfpathlineto{\pgfqpoint{1.917476in}{2.282743in}}%
\pgfpathlineto{\pgfqpoint{1.917500in}{2.239273in}}%
\pgfpathlineto{\pgfqpoint{1.917968in}{2.469826in}}%
\pgfpathlineto{\pgfqpoint{1.918522in}{2.404096in}}%
\pgfpathlineto{\pgfqpoint{1.918571in}{2.470765in}}%
\pgfpathlineto{\pgfqpoint{1.918854in}{2.276338in}}%
\pgfpathlineto{\pgfqpoint{1.919334in}{2.306545in}}%
\pgfpathlineto{\pgfqpoint{1.919347in}{2.295819in}}%
\pgfpathlineto{\pgfqpoint{1.919814in}{2.443448in}}%
\pgfpathlineto{\pgfqpoint{1.920319in}{2.410668in}}%
\pgfpathlineto{\pgfqpoint{1.920417in}{2.449376in}}%
\pgfpathlineto{\pgfqpoint{1.920577in}{2.291305in}}%
\pgfpathlineto{\pgfqpoint{1.921131in}{2.322495in}}%
\pgfpathlineto{\pgfqpoint{1.921193in}{2.279155in}}%
\pgfpathlineto{\pgfqpoint{1.921464in}{2.433578in}}%
\pgfpathlineto{\pgfqpoint{1.922042in}{2.408309in}}%
\pgfpathlineto{\pgfqpoint{1.922694in}{2.456927in}}%
\pgfpathlineto{\pgfqpoint{1.923039in}{2.271887in}}%
\pgfpathlineto{\pgfqpoint{1.923125in}{2.372054in}}%
\pgfpathlineto{\pgfqpoint{1.923310in}{2.456049in}}%
\pgfpathlineto{\pgfqpoint{1.923642in}{2.272117in}}%
\pgfpathlineto{\pgfqpoint{1.924159in}{2.357281in}}%
\pgfpathlineto{\pgfqpoint{1.924245in}{2.271164in}}%
\pgfpathlineto{\pgfqpoint{1.924553in}{2.454389in}}%
\pgfpathlineto{\pgfqpoint{1.925156in}{2.436348in}}%
\pgfpathlineto{\pgfqpoint{1.926190in}{2.250148in}}%
\pgfpathlineto{\pgfqpoint{1.925882in}{2.488969in}}%
\pgfpathlineto{\pgfqpoint{1.926362in}{2.377502in}}%
\pgfpathlineto{\pgfqpoint{1.927076in}{2.465308in}}%
\pgfpathlineto{\pgfqpoint{1.927334in}{2.225379in}}%
\pgfpathlineto{\pgfqpoint{1.927470in}{2.379736in}}%
\pgfpathlineto{\pgfqpoint{1.928479in}{2.490043in}}%
\pgfpathlineto{\pgfqpoint{1.928307in}{2.260647in}}%
\pgfpathlineto{\pgfqpoint{1.928540in}{2.364750in}}%
\pgfpathlineto{\pgfqpoint{1.929205in}{2.246251in}}%
\pgfpathlineto{\pgfqpoint{1.928885in}{2.504246in}}%
\pgfpathlineto{\pgfqpoint{1.929660in}{2.345913in}}%
\pgfpathlineto{\pgfqpoint{1.930411in}{2.175022in}}%
\pgfpathlineto{\pgfqpoint{1.929931in}{2.515744in}}%
\pgfpathlineto{\pgfqpoint{1.930596in}{2.421151in}}%
\pgfpathlineto{\pgfqpoint{1.930768in}{2.591261in}}%
\pgfpathlineto{\pgfqpoint{1.931100in}{2.165925in}}%
\pgfpathlineto{\pgfqpoint{1.931630in}{2.354456in}}%
\pgfpathlineto{\pgfqpoint{1.932294in}{2.160100in}}%
\pgfpathlineto{\pgfqpoint{1.931999in}{2.511542in}}%
\pgfpathlineto{\pgfqpoint{1.932602in}{2.466438in}}%
\pgfpathlineto{\pgfqpoint{1.932713in}{2.604683in}}%
\pgfpathlineto{\pgfqpoint{1.932996in}{2.186677in}}%
\pgfpathlineto{\pgfqpoint{1.933697in}{2.411922in}}%
\pgfpathlineto{\pgfqpoint{1.934153in}{2.287597in}}%
\pgfpathlineto{\pgfqpoint{1.934620in}{2.486006in}}%
\pgfpathlineto{\pgfqpoint{1.934707in}{2.474281in}}%
\pgfpathlineto{\pgfqpoint{1.934719in}{2.475064in}}%
\pgfpathlineto{\pgfqpoint{1.934780in}{2.416463in}}%
\pgfpathlineto{\pgfqpoint{1.936048in}{2.212839in}}%
\pgfpathlineto{\pgfqpoint{1.935568in}{2.520928in}}%
\pgfpathlineto{\pgfqpoint{1.936073in}{2.251639in}}%
\pgfpathlineto{\pgfqpoint{1.936368in}{2.539145in}}%
\pgfpathlineto{\pgfqpoint{1.936737in}{2.222123in}}%
\pgfpathlineto{\pgfqpoint{1.937193in}{2.306784in}}%
\pgfpathlineto{\pgfqpoint{1.937267in}{2.281432in}}%
\pgfpathlineto{\pgfqpoint{1.937476in}{2.478795in}}%
\pgfpathlineto{\pgfqpoint{1.938276in}{2.624043in}}%
\pgfpathlineto{\pgfqpoint{1.937894in}{2.184298in}}%
\pgfpathlineto{\pgfqpoint{1.938510in}{2.336748in}}%
\pgfpathlineto{\pgfqpoint{1.938645in}{2.197110in}}%
\pgfpathlineto{\pgfqpoint{1.938965in}{2.484314in}}%
\pgfpathlineto{\pgfqpoint{1.939593in}{2.360166in}}%
\pgfpathlineto{\pgfqpoint{1.940196in}{2.488713in}}%
\pgfpathlineto{\pgfqpoint{1.940479in}{2.279068in}}%
\pgfpathlineto{\pgfqpoint{1.940651in}{2.330704in}}%
\pgfpathlineto{\pgfqpoint{1.941463in}{2.190196in}}%
\pgfpathlineto{\pgfqpoint{1.941156in}{2.576824in}}%
\pgfpathlineto{\pgfqpoint{1.941734in}{2.379482in}}%
\pgfpathlineto{\pgfqpoint{1.941907in}{2.549664in}}%
\pgfpathlineto{\pgfqpoint{1.942337in}{2.206597in}}%
\pgfpathlineto{\pgfqpoint{1.942793in}{2.291869in}}%
\pgfpathlineto{\pgfqpoint{1.943851in}{2.611839in}}%
\pgfpathlineto{\pgfqpoint{1.943470in}{2.167365in}}%
\pgfpathlineto{\pgfqpoint{1.944036in}{2.381889in}}%
\pgfpathlineto{\pgfqpoint{1.944233in}{2.155529in}}%
\pgfpathlineto{\pgfqpoint{1.944885in}{2.500058in}}%
\pgfpathlineto{\pgfqpoint{1.945168in}{2.312491in}}%
\pgfpathlineto{\pgfqpoint{1.945783in}{2.526317in}}%
\pgfpathlineto{\pgfqpoint{1.945340in}{2.284166in}}%
\pgfpathlineto{\pgfqpoint{1.946251in}{2.288764in}}%
\pgfpathlineto{\pgfqpoint{1.946436in}{2.227464in}}%
\pgfpathlineto{\pgfqpoint{1.946719in}{2.606847in}}%
\pgfpathlineto{\pgfqpoint{1.946731in}{2.627854in}}%
\pgfpathlineto{\pgfqpoint{1.947100in}{2.159458in}}%
\pgfpathlineto{\pgfqpoint{1.947703in}{2.404033in}}%
\pgfpathlineto{\pgfqpoint{1.948343in}{2.197417in}}%
\pgfpathlineto{\pgfqpoint{1.948737in}{2.577569in}}%
\pgfpathlineto{\pgfqpoint{1.949057in}{2.136196in}}%
\pgfpathlineto{\pgfqpoint{1.949439in}{2.596241in}}%
\pgfpathlineto{\pgfqpoint{1.950226in}{2.355655in}}%
\pgfpathlineto{\pgfqpoint{1.950473in}{2.497107in}}%
\pgfpathlineto{\pgfqpoint{1.950743in}{2.238585in}}%
\pgfpathlineto{\pgfqpoint{1.951396in}{2.465171in}}%
\pgfpathlineto{\pgfqpoint{1.952343in}{2.623367in}}%
\pgfpathlineto{\pgfqpoint{1.952700in}{2.149559in}}%
\pgfpathlineto{\pgfqpoint{1.953070in}{2.552134in}}%
\pgfpathlineto{\pgfqpoint{1.953906in}{2.270886in}}%
\pgfpathlineto{\pgfqpoint{1.954670in}{2.177121in}}%
\pgfpathlineto{\pgfqpoint{1.954300in}{2.555491in}}%
\pgfpathlineto{\pgfqpoint{1.954916in}{2.419065in}}%
\pgfpathlineto{\pgfqpoint{1.955039in}{2.599254in}}%
\pgfpathlineto{\pgfqpoint{1.955371in}{2.209865in}}%
\pgfpathlineto{\pgfqpoint{1.956048in}{2.467059in}}%
\pgfpathlineto{\pgfqpoint{1.956097in}{2.479606in}}%
\pgfpathlineto{\pgfqpoint{1.956183in}{2.375985in}}%
\pgfpathlineto{\pgfqpoint{1.956343in}{2.209961in}}%
\pgfpathlineto{\pgfqpoint{1.957106in}{2.466727in}}%
\pgfpathlineto{\pgfqpoint{1.957291in}{2.362964in}}%
\pgfpathlineto{\pgfqpoint{1.957931in}{2.590246in}}%
\pgfpathlineto{\pgfqpoint{1.957599in}{2.191151in}}%
\pgfpathlineto{\pgfqpoint{1.958276in}{2.202072in}}%
\pgfpathlineto{\pgfqpoint{1.958313in}{2.158896in}}%
\pgfpathlineto{\pgfqpoint{1.958694in}{2.494553in}}%
\pgfpathlineto{\pgfqpoint{1.959260in}{2.464239in}}%
\pgfpathlineto{\pgfqpoint{1.959888in}{2.559213in}}%
\pgfpathlineto{\pgfqpoint{1.959568in}{2.215628in}}%
\pgfpathlineto{\pgfqpoint{1.960245in}{2.304118in}}%
\pgfpathlineto{\pgfqpoint{1.960282in}{2.233460in}}%
\pgfpathlineto{\pgfqpoint{1.960602in}{2.516546in}}%
\pgfpathlineto{\pgfqpoint{1.961316in}{2.403598in}}%
\pgfpathlineto{\pgfqpoint{1.961956in}{2.296235in}}%
\pgfpathlineto{\pgfqpoint{1.961710in}{2.466452in}}%
\pgfpathlineto{\pgfqpoint{1.962510in}{2.345276in}}%
\pgfpathlineto{\pgfqpoint{1.963482in}{2.540529in}}%
\pgfpathlineto{\pgfqpoint{1.963186in}{2.216327in}}%
\pgfpathlineto{\pgfqpoint{1.963666in}{2.390288in}}%
\pgfpathlineto{\pgfqpoint{1.963913in}{2.180004in}}%
\pgfpathlineto{\pgfqpoint{1.964159in}{2.479608in}}%
\pgfpathlineto{\pgfqpoint{1.964750in}{2.420789in}}%
\pgfpathlineto{\pgfqpoint{1.965488in}{2.519761in}}%
\pgfpathlineto{\pgfqpoint{1.965143in}{2.246910in}}%
\pgfpathlineto{\pgfqpoint{1.965722in}{2.342720in}}%
\pgfpathlineto{\pgfqpoint{1.965845in}{2.240882in}}%
\pgfpathlineto{\pgfqpoint{1.966386in}{2.469934in}}%
\pgfpathlineto{\pgfqpoint{1.966842in}{2.310239in}}%
\pgfpathlineto{\pgfqpoint{1.966854in}{2.305653in}}%
\pgfpathlineto{\pgfqpoint{1.967248in}{2.467592in}}%
\pgfpathlineto{\pgfqpoint{1.967580in}{2.422985in}}%
\pgfpathlineto{\pgfqpoint{1.968294in}{2.498269in}}%
\pgfpathlineto{\pgfqpoint{1.968639in}{2.267575in}}%
\pgfpathlineto{\pgfqpoint{1.968676in}{2.233651in}}%
\pgfpathlineto{\pgfqpoint{1.969082in}{2.478378in}}%
\pgfpathlineto{\pgfqpoint{1.969660in}{2.355751in}}%
\pgfpathlineto{\pgfqpoint{1.970140in}{2.472160in}}%
\pgfpathlineto{\pgfqpoint{1.970596in}{2.279205in}}%
\pgfpathlineto{\pgfqpoint{1.970743in}{2.334244in}}%
\pgfpathlineto{\pgfqpoint{1.971703in}{2.274135in}}%
\pgfpathlineto{\pgfqpoint{1.971322in}{2.486358in}}%
\pgfpathlineto{\pgfqpoint{1.971826in}{2.375366in}}%
\pgfpathlineto{\pgfqpoint{1.972011in}{2.466596in}}%
\pgfpathlineto{\pgfqpoint{1.972393in}{2.261681in}}%
\pgfpathlineto{\pgfqpoint{1.972946in}{2.411788in}}%
\pgfpathlineto{\pgfqpoint{1.973673in}{2.244032in}}%
\pgfpathlineto{\pgfqpoint{1.973217in}{2.512798in}}%
\pgfpathlineto{\pgfqpoint{1.974042in}{2.466263in}}%
\pgfpathlineto{\pgfqpoint{1.974140in}{2.573024in}}%
\pgfpathlineto{\pgfqpoint{1.974362in}{2.259142in}}%
\pgfpathlineto{\pgfqpoint{1.974916in}{2.391458in}}%
\pgfpathlineto{\pgfqpoint{1.975876in}{2.256474in}}%
\pgfpathlineto{\pgfqpoint{1.975925in}{2.549611in}}%
\pgfpathlineto{\pgfqpoint{1.976306in}{2.233072in}}%
\pgfpathlineto{\pgfqpoint{1.977143in}{2.404846in}}%
\pgfpathlineto{\pgfqpoint{1.977882in}{2.533657in}}%
\pgfpathlineto{\pgfqpoint{1.978091in}{2.217887in}}%
\pgfpathlineto{\pgfqpoint{1.978226in}{2.350459in}}%
\pgfpathlineto{\pgfqpoint{1.979285in}{2.195944in}}%
\pgfpathlineto{\pgfqpoint{1.978300in}{2.560185in}}%
\pgfpathlineto{\pgfqpoint{1.979359in}{2.204554in}}%
\pgfpathlineto{\pgfqpoint{1.979654in}{2.558575in}}%
\pgfpathlineto{\pgfqpoint{1.980023in}{2.159964in}}%
\pgfpathlineto{\pgfqpoint{1.980516in}{2.377700in}}%
\pgfpathlineto{\pgfqpoint{1.981057in}{2.247005in}}%
\pgfpathlineto{\pgfqpoint{1.981586in}{2.511976in}}%
\pgfpathlineto{\pgfqpoint{1.982559in}{2.558908in}}%
\pgfpathlineto{\pgfqpoint{1.982239in}{2.226365in}}%
\pgfpathlineto{\pgfqpoint{1.982620in}{2.414260in}}%
\pgfpathlineto{\pgfqpoint{1.982916in}{2.237949in}}%
\pgfpathlineto{\pgfqpoint{1.983285in}{2.541300in}}%
\pgfpathlineto{\pgfqpoint{1.983691in}{2.450277in}}%
\pgfpathlineto{\pgfqpoint{1.984528in}{2.568444in}}%
\pgfpathlineto{\pgfqpoint{1.984171in}{2.191792in}}%
\pgfpathlineto{\pgfqpoint{1.984663in}{2.388744in}}%
\pgfpathlineto{\pgfqpoint{1.984762in}{2.225909in}}%
\pgfpathlineto{\pgfqpoint{1.985340in}{2.515259in}}%
\pgfpathlineto{\pgfqpoint{1.985746in}{2.464374in}}%
\pgfpathlineto{\pgfqpoint{1.986189in}{2.554465in}}%
\pgfpathlineto{\pgfqpoint{1.985943in}{2.256134in}}%
\pgfpathlineto{\pgfqpoint{1.986374in}{2.298890in}}%
\pgfpathlineto{\pgfqpoint{1.986399in}{2.265815in}}%
\pgfpathlineto{\pgfqpoint{1.986596in}{2.526604in}}%
\pgfpathlineto{\pgfqpoint{1.987408in}{2.432446in}}%
\pgfpathlineto{\pgfqpoint{1.988269in}{2.550428in}}%
\pgfpathlineto{\pgfqpoint{1.987728in}{2.262623in}}%
\pgfpathlineto{\pgfqpoint{1.988479in}{2.276151in}}%
\pgfpathlineto{\pgfqpoint{1.989082in}{2.512310in}}%
\pgfpathlineto{\pgfqpoint{1.989352in}{2.244015in}}%
\pgfpathlineto{\pgfqpoint{1.989685in}{2.316440in}}%
\pgfpathlineto{\pgfqpoint{1.990534in}{2.202754in}}%
\pgfpathlineto{\pgfqpoint{1.990325in}{2.529211in}}%
\pgfpathlineto{\pgfqpoint{1.990719in}{2.405215in}}%
\pgfpathlineto{\pgfqpoint{1.991162in}{2.517763in}}%
\pgfpathlineto{\pgfqpoint{1.991211in}{2.255593in}}%
\pgfpathlineto{\pgfqpoint{1.991802in}{2.320443in}}%
\pgfpathlineto{\pgfqpoint{1.992146in}{2.305917in}}%
\pgfpathlineto{\pgfqpoint{1.991974in}{2.506213in}}%
\pgfpathlineto{\pgfqpoint{1.992356in}{2.425317in}}%
\pgfpathlineto{\pgfqpoint{1.993205in}{2.557644in}}%
\pgfpathlineto{\pgfqpoint{1.992454in}{2.249585in}}%
\pgfpathlineto{\pgfqpoint{1.993439in}{2.351314in}}%
\pgfpathlineto{\pgfqpoint{1.994239in}{2.207265in}}%
\pgfpathlineto{\pgfqpoint{1.994029in}{2.553951in}}%
\pgfpathlineto{\pgfqpoint{1.994522in}{2.329455in}}%
\pgfpathlineto{\pgfqpoint{1.994559in}{2.530992in}}%
\pgfpathlineto{\pgfqpoint{1.995432in}{2.213248in}}%
\pgfpathlineto{\pgfqpoint{1.995642in}{2.398415in}}%
\pgfpathlineto{\pgfqpoint{1.995789in}{2.563722in}}%
\pgfpathlineto{\pgfqpoint{1.995851in}{2.238738in}}%
\pgfpathlineto{\pgfqpoint{1.996749in}{2.425008in}}%
\pgfpathlineto{\pgfqpoint{1.997796in}{2.230699in}}%
\pgfpathlineto{\pgfqpoint{1.997439in}{2.532000in}}%
\pgfpathlineto{\pgfqpoint{1.997832in}{2.429770in}}%
\pgfpathlineto{\pgfqpoint{1.998152in}{2.555416in}}%
\pgfpathlineto{\pgfqpoint{1.998214in}{2.214753in}}%
\pgfpathlineto{\pgfqpoint{1.998928in}{2.368815in}}%
\pgfpathlineto{\pgfqpoint{1.999912in}{2.528571in}}%
\pgfpathlineto{\pgfqpoint{1.999556in}{2.270473in}}%
\pgfpathlineto{\pgfqpoint{1.999962in}{2.300943in}}%
\pgfpathlineto{\pgfqpoint{1.999986in}{2.243661in}}%
\pgfpathlineto{\pgfqpoint{2.000208in}{2.520976in}}%
\pgfpathlineto{\pgfqpoint{2.001020in}{2.456271in}}%
\pgfpathlineto{\pgfqpoint{2.001869in}{2.532531in}}%
\pgfpathlineto{\pgfqpoint{2.001919in}{2.227626in}}%
\pgfpathlineto{\pgfqpoint{2.002066in}{2.290751in}}%
\pgfpathlineto{\pgfqpoint{2.002288in}{2.580552in}}%
\pgfpathlineto{\pgfqpoint{2.002940in}{2.280110in}}%
\pgfpathlineto{\pgfqpoint{2.003223in}{2.435882in}}%
\pgfpathlineto{\pgfqpoint{2.004122in}{2.218855in}}%
\pgfpathlineto{\pgfqpoint{2.004036in}{2.502211in}}%
\pgfpathlineto{\pgfqpoint{2.004306in}{2.467128in}}%
\pgfpathlineto{\pgfqpoint{2.004331in}{2.569063in}}%
\pgfpathlineto{\pgfqpoint{2.004799in}{2.249607in}}%
\pgfpathlineto{\pgfqpoint{2.005389in}{2.309486in}}%
\pgfpathlineto{\pgfqpoint{2.006411in}{2.557366in}}%
\pgfpathlineto{\pgfqpoint{2.006054in}{2.243987in}}%
\pgfpathlineto{\pgfqpoint{2.006534in}{2.441407in}}%
\pgfpathlineto{\pgfqpoint{2.006657in}{2.267539in}}%
\pgfpathlineto{\pgfqpoint{2.007211in}{2.514998in}}%
\pgfpathlineto{\pgfqpoint{2.007629in}{2.473612in}}%
\pgfpathlineto{\pgfqpoint{2.008245in}{2.227892in}}%
\pgfpathlineto{\pgfqpoint{2.008466in}{2.551300in}}%
\pgfpathlineto{\pgfqpoint{2.008836in}{2.353252in}}%
\pgfpathlineto{\pgfqpoint{2.009389in}{2.501747in}}%
\pgfpathlineto{\pgfqpoint{2.009451in}{2.259699in}}%
\pgfpathlineto{\pgfqpoint{2.009955in}{2.385821in}}%
\pgfpathlineto{\pgfqpoint{2.010940in}{2.509085in}}%
\pgfpathlineto{\pgfqpoint{2.010177in}{2.268074in}}%
\pgfpathlineto{\pgfqpoint{2.011051in}{2.404751in}}%
\pgfpathlineto{\pgfqpoint{2.011962in}{2.268077in}}%
\pgfpathlineto{\pgfqpoint{2.011752in}{2.517912in}}%
\pgfpathlineto{\pgfqpoint{2.012134in}{2.444922in}}%
\pgfpathlineto{\pgfqpoint{2.012183in}{2.520395in}}%
\pgfpathlineto{\pgfqpoint{2.013155in}{2.262494in}}%
\pgfpathlineto{\pgfqpoint{2.013192in}{2.308555in}}%
\pgfpathlineto{\pgfqpoint{2.013254in}{2.270539in}}%
\pgfpathlineto{\pgfqpoint{2.013402in}{2.515887in}}%
\pgfpathlineto{\pgfqpoint{2.014091in}{2.376279in}}%
\pgfpathlineto{\pgfqpoint{2.014214in}{2.514581in}}%
\pgfpathlineto{\pgfqpoint{2.014903in}{2.282804in}}%
\pgfpathlineto{\pgfqpoint{2.015186in}{2.366909in}}%
\pgfpathlineto{\pgfqpoint{2.016085in}{2.261838in}}%
\pgfpathlineto{\pgfqpoint{2.015875in}{2.559774in}}%
\pgfpathlineto{\pgfqpoint{2.016257in}{2.416626in}}%
\pgfpathlineto{\pgfqpoint{2.017131in}{2.474952in}}%
\pgfpathlineto{\pgfqpoint{2.017279in}{2.253179in}}%
\pgfpathlineto{\pgfqpoint{2.017328in}{2.363246in}}%
\pgfpathlineto{\pgfqpoint{2.017365in}{2.283142in}}%
\pgfpathlineto{\pgfqpoint{2.017635in}{2.505368in}}%
\pgfpathlineto{\pgfqpoint{2.018325in}{2.452421in}}%
\pgfpathlineto{\pgfqpoint{2.018349in}{2.526657in}}%
\pgfpathlineto{\pgfqpoint{2.019039in}{2.281408in}}%
\pgfpathlineto{\pgfqpoint{2.019408in}{2.364576in}}%
\pgfpathlineto{\pgfqpoint{2.020060in}{2.268836in}}%
\pgfpathlineto{\pgfqpoint{2.020011in}{2.517328in}}%
\pgfpathlineto{\pgfqpoint{2.020380in}{2.432863in}}%
\pgfpathlineto{\pgfqpoint{2.021205in}{2.490574in}}%
\pgfpathlineto{\pgfqpoint{2.020663in}{2.254784in}}%
\pgfpathlineto{\pgfqpoint{2.021402in}{2.277507in}}%
\pgfpathlineto{\pgfqpoint{2.021845in}{2.244074in}}%
\pgfpathlineto{\pgfqpoint{2.022054in}{2.509440in}}%
\pgfpathlineto{\pgfqpoint{2.022337in}{2.337516in}}%
\pgfpathlineto{\pgfqpoint{2.022485in}{2.510163in}}%
\pgfpathlineto{\pgfqpoint{2.023039in}{2.264515in}}%
\pgfpathlineto{\pgfqpoint{2.023445in}{2.354975in}}%
\pgfpathlineto{\pgfqpoint{2.024368in}{2.262096in}}%
\pgfpathlineto{\pgfqpoint{2.024134in}{2.500341in}}%
\pgfpathlineto{\pgfqpoint{2.024503in}{2.440778in}}%
\pgfpathlineto{\pgfqpoint{2.025340in}{2.512047in}}%
\pgfpathlineto{\pgfqpoint{2.024786in}{2.269571in}}%
\pgfpathlineto{\pgfqpoint{2.025525in}{2.313516in}}%
\pgfpathlineto{\pgfqpoint{2.025549in}{2.290493in}}%
\pgfpathlineto{\pgfqpoint{2.026189in}{2.480971in}}%
\pgfpathlineto{\pgfqpoint{2.026583in}{2.390694in}}%
\pgfpathlineto{\pgfqpoint{2.027100in}{2.502551in}}%
\pgfpathlineto{\pgfqpoint{2.027162in}{2.267457in}}%
\pgfpathlineto{\pgfqpoint{2.027703in}{2.419277in}}%
\pgfpathlineto{\pgfqpoint{2.028232in}{2.485087in}}%
\pgfpathlineto{\pgfqpoint{2.028491in}{2.267480in}}%
\pgfpathlineto{\pgfqpoint{2.028688in}{2.330219in}}%
\pgfpathlineto{\pgfqpoint{2.029315in}{2.290886in}}%
\pgfpathlineto{\pgfqpoint{2.029475in}{2.491721in}}%
\pgfpathlineto{\pgfqpoint{2.029758in}{2.338460in}}%
\pgfpathlineto{\pgfqpoint{2.030312in}{2.458459in}}%
\pgfpathlineto{\pgfqpoint{2.030460in}{2.283302in}}%
\pgfpathlineto{\pgfqpoint{2.030854in}{2.320611in}}%
\pgfpathlineto{\pgfqpoint{2.031297in}{2.269150in}}%
\pgfpathlineto{\pgfqpoint{2.031506in}{2.455337in}}%
\pgfpathlineto{\pgfqpoint{2.031912in}{2.424325in}}%
\pgfpathlineto{\pgfqpoint{2.031937in}{2.478196in}}%
\pgfpathlineto{\pgfqpoint{2.032638in}{2.280094in}}%
\pgfpathlineto{\pgfqpoint{2.033008in}{2.394960in}}%
\pgfpathlineto{\pgfqpoint{2.033352in}{2.288297in}}%
\pgfpathlineto{\pgfqpoint{2.033586in}{2.474866in}}%
\pgfpathlineto{\pgfqpoint{2.034103in}{2.401655in}}%
\pgfpathlineto{\pgfqpoint{2.034805in}{2.487301in}}%
\pgfpathlineto{\pgfqpoint{2.034263in}{2.281354in}}%
\pgfpathlineto{\pgfqpoint{2.034965in}{2.373534in}}%
\pgfpathlineto{\pgfqpoint{2.035002in}{2.272510in}}%
\pgfpathlineto{\pgfqpoint{2.035642in}{2.512787in}}%
\pgfpathlineto{\pgfqpoint{2.036048in}{2.424206in}}%
\pgfpathlineto{\pgfqpoint{2.036577in}{2.477991in}}%
\pgfpathlineto{\pgfqpoint{2.037057in}{2.288022in}}%
\pgfpathlineto{\pgfqpoint{2.037131in}{2.365744in}}%
\pgfpathlineto{\pgfqpoint{2.037217in}{2.310470in}}%
\pgfpathlineto{\pgfqpoint{2.037254in}{2.399431in}}%
\pgfpathlineto{\pgfqpoint{2.037722in}{2.476120in}}%
\pgfpathlineto{\pgfqpoint{2.037968in}{2.279855in}}%
\pgfpathlineto{\pgfqpoint{2.038349in}{2.410082in}}%
\pgfpathlineto{\pgfqpoint{2.039137in}{2.288163in}}%
\pgfpathlineto{\pgfqpoint{2.038940in}{2.487946in}}%
\pgfpathlineto{\pgfqpoint{2.039445in}{2.402714in}}%
\pgfpathlineto{\pgfqpoint{2.039777in}{2.461623in}}%
\pgfpathlineto{\pgfqpoint{2.039937in}{2.302845in}}%
\pgfpathlineto{\pgfqpoint{2.040565in}{2.441516in}}%
\pgfpathlineto{\pgfqpoint{2.041402in}{2.474146in}}%
\pgfpathlineto{\pgfqpoint{2.040762in}{2.253345in}}%
\pgfpathlineto{\pgfqpoint{2.041562in}{2.355966in}}%
\pgfpathlineto{\pgfqpoint{2.042091in}{2.285208in}}%
\pgfpathlineto{\pgfqpoint{2.041820in}{2.481800in}}%
\pgfpathlineto{\pgfqpoint{2.042645in}{2.437849in}}%
\pgfpathlineto{\pgfqpoint{2.042817in}{2.288386in}}%
\pgfpathlineto{\pgfqpoint{2.043063in}{2.532698in}}%
\pgfpathlineto{\pgfqpoint{2.043826in}{2.376718in}}%
\pgfpathlineto{\pgfqpoint{2.043900in}{2.460582in}}%
\pgfpathlineto{\pgfqpoint{2.044060in}{2.307954in}}%
\pgfpathlineto{\pgfqpoint{2.044429in}{2.375191in}}%
\pgfpathlineto{\pgfqpoint{2.044466in}{2.262858in}}%
\pgfpathlineto{\pgfqpoint{2.045106in}{2.480804in}}%
\pgfpathlineto{\pgfqpoint{2.045512in}{2.444732in}}%
\pgfpathlineto{\pgfqpoint{2.045537in}{2.490561in}}%
\pgfpathlineto{\pgfqpoint{2.046522in}{2.285060in}}%
\pgfpathlineto{\pgfqpoint{2.046595in}{2.400789in}}%
\pgfpathlineto{\pgfqpoint{2.047248in}{2.289184in}}%
\pgfpathlineto{\pgfqpoint{2.047186in}{2.506814in}}%
\pgfpathlineto{\pgfqpoint{2.047691in}{2.427033in}}%
\pgfpathlineto{\pgfqpoint{2.048109in}{2.462813in}}%
\pgfpathlineto{\pgfqpoint{2.047851in}{2.294672in}}%
\pgfpathlineto{\pgfqpoint{2.048577in}{2.298038in}}%
\pgfpathlineto{\pgfqpoint{2.048589in}{2.265075in}}%
\pgfpathlineto{\pgfqpoint{2.048823in}{2.488000in}}%
\pgfpathlineto{\pgfqpoint{2.049635in}{2.437420in}}%
\pgfpathlineto{\pgfqpoint{2.049660in}{2.488888in}}%
\pgfpathlineto{\pgfqpoint{2.050226in}{2.293908in}}%
\pgfpathlineto{\pgfqpoint{2.050718in}{2.382375in}}%
\pgfpathlineto{\pgfqpoint{2.050952in}{2.279799in}}%
\pgfpathlineto{\pgfqpoint{2.050891in}{2.485394in}}%
\pgfpathlineto{\pgfqpoint{2.051691in}{2.447759in}}%
\pgfpathlineto{\pgfqpoint{2.052528in}{2.505607in}}%
\pgfpathlineto{\pgfqpoint{2.051974in}{2.273640in}}%
\pgfpathlineto{\pgfqpoint{2.052700in}{2.323125in}}%
\pgfpathlineto{\pgfqpoint{2.053155in}{2.289156in}}%
\pgfpathlineto{\pgfqpoint{2.053365in}{2.484685in}}%
\pgfpathlineto{\pgfqpoint{2.053758in}{2.424247in}}%
\pgfpathlineto{\pgfqpoint{2.054288in}{2.501322in}}%
\pgfpathlineto{\pgfqpoint{2.053931in}{2.287684in}}%
\pgfpathlineto{\pgfqpoint{2.054768in}{2.289362in}}%
\pgfpathlineto{\pgfqpoint{2.055826in}{2.477835in}}%
\pgfpathlineto{\pgfqpoint{2.055678in}{2.288982in}}%
\pgfpathlineto{\pgfqpoint{2.055974in}{2.344719in}}%
\pgfpathlineto{\pgfqpoint{2.056848in}{2.287741in}}%
\pgfpathlineto{\pgfqpoint{2.056651in}{2.506464in}}%
\pgfpathlineto{\pgfqpoint{2.057032in}{2.409374in}}%
\pgfpathlineto{\pgfqpoint{2.057057in}{2.461557in}}%
\pgfpathlineto{\pgfqpoint{2.057660in}{2.286424in}}%
\pgfpathlineto{\pgfqpoint{2.058115in}{2.358045in}}%
\pgfpathlineto{\pgfqpoint{2.058485in}{2.286450in}}%
\pgfpathlineto{\pgfqpoint{2.059112in}{2.490650in}}%
\pgfpathlineto{\pgfqpoint{2.059125in}{2.497515in}}%
\pgfpathlineto{\pgfqpoint{2.059728in}{2.314094in}}%
\pgfpathlineto{\pgfqpoint{2.059986in}{2.372885in}}%
\pgfpathlineto{\pgfqpoint{2.060614in}{2.298788in}}%
\pgfpathlineto{\pgfqpoint{2.060774in}{2.496700in}}%
\pgfpathlineto{\pgfqpoint{2.061094in}{2.379130in}}%
\pgfpathlineto{\pgfqpoint{2.061795in}{2.302576in}}%
\pgfpathlineto{\pgfqpoint{2.061992in}{2.465165in}}%
\pgfpathlineto{\pgfqpoint{2.062214in}{2.352197in}}%
\pgfpathlineto{\pgfqpoint{2.062817in}{2.483573in}}%
\pgfpathlineto{\pgfqpoint{2.062608in}{2.297610in}}%
\pgfpathlineto{\pgfqpoint{2.063346in}{2.396809in}}%
\pgfpathlineto{\pgfqpoint{2.063937in}{2.319154in}}%
\pgfpathlineto{\pgfqpoint{2.064060in}{2.463347in}}%
\pgfpathlineto{\pgfqpoint{2.064441in}{2.389349in}}%
\pgfpathlineto{\pgfqpoint{2.064897in}{2.478709in}}%
\pgfpathlineto{\pgfqpoint{2.065143in}{2.312565in}}%
\pgfpathlineto{\pgfqpoint{2.065537in}{2.364234in}}%
\pgfpathlineto{\pgfqpoint{2.065561in}{2.296339in}}%
\pgfpathlineto{\pgfqpoint{2.066115in}{2.472309in}}%
\pgfpathlineto{\pgfqpoint{2.066632in}{2.399726in}}%
\pgfpathlineto{\pgfqpoint{2.066940in}{2.474963in}}%
\pgfpathlineto{\pgfqpoint{2.066743in}{2.297260in}}%
\pgfpathlineto{\pgfqpoint{2.067777in}{2.453651in}}%
\pgfpathlineto{\pgfqpoint{2.067937in}{2.304739in}}%
\pgfpathlineto{\pgfqpoint{2.067875in}{2.462383in}}%
\pgfpathlineto{\pgfqpoint{2.068958in}{2.361552in}}%
\pgfpathlineto{\pgfqpoint{2.069820in}{2.475640in}}%
\pgfpathlineto{\pgfqpoint{2.069266in}{2.307154in}}%
\pgfpathlineto{\pgfqpoint{2.070054in}{2.325750in}}%
\pgfpathlineto{\pgfqpoint{2.070238in}{2.494425in}}%
\pgfpathlineto{\pgfqpoint{2.070435in}{2.307440in}}%
\pgfpathlineto{\pgfqpoint{2.071211in}{2.376211in}}%
\pgfpathlineto{\pgfqpoint{2.071740in}{2.306830in}}%
\pgfpathlineto{\pgfqpoint{2.071998in}{2.468146in}}%
\pgfpathlineto{\pgfqpoint{2.072294in}{2.418784in}}%
\pgfpathlineto{\pgfqpoint{2.072491in}{2.331058in}}%
\pgfpathlineto{\pgfqpoint{2.072884in}{2.308042in}}%
\pgfpathlineto{\pgfqpoint{2.072700in}{2.458819in}}%
\pgfpathlineto{\pgfqpoint{2.073500in}{2.393468in}}%
\pgfpathlineto{\pgfqpoint{2.073943in}{2.491231in}}%
\pgfpathlineto{\pgfqpoint{2.074201in}{2.314774in}}%
\pgfpathlineto{\pgfqpoint{2.074558in}{2.319389in}}%
\pgfpathlineto{\pgfqpoint{2.074571in}{2.311933in}}%
\pgfpathlineto{\pgfqpoint{2.075186in}{2.449606in}}%
\pgfpathlineto{\pgfqpoint{2.075543in}{2.390261in}}%
\pgfpathlineto{\pgfqpoint{2.075998in}{2.460699in}}%
\pgfpathlineto{\pgfqpoint{2.076195in}{2.304826in}}%
\pgfpathlineto{\pgfqpoint{2.076614in}{2.330672in}}%
\pgfpathlineto{\pgfqpoint{2.077032in}{2.315451in}}%
\pgfpathlineto{\pgfqpoint{2.076823in}{2.480584in}}%
\pgfpathlineto{\pgfqpoint{2.077549in}{2.385125in}}%
\pgfpathlineto{\pgfqpoint{2.078066in}{2.474602in}}%
\pgfpathlineto{\pgfqpoint{2.077857in}{2.311272in}}%
\pgfpathlineto{\pgfqpoint{2.078632in}{2.362196in}}%
\pgfpathlineto{\pgfqpoint{2.078694in}{2.327003in}}%
\pgfpathlineto{\pgfqpoint{2.078854in}{2.411051in}}%
\pgfpathlineto{\pgfqpoint{2.079691in}{2.469497in}}%
\pgfpathlineto{\pgfqpoint{2.079149in}{2.299358in}}%
\pgfpathlineto{\pgfqpoint{2.079937in}{2.374882in}}%
\pgfpathlineto{\pgfqpoint{2.080318in}{2.312890in}}%
\pgfpathlineto{\pgfqpoint{2.080528in}{2.468774in}}%
\pgfpathlineto{\pgfqpoint{2.081020in}{2.391466in}}%
\pgfpathlineto{\pgfqpoint{2.081352in}{2.461138in}}%
\pgfpathlineto{\pgfqpoint{2.081512in}{2.322823in}}%
\pgfpathlineto{\pgfqpoint{2.082103in}{2.337048in}}%
\pgfpathlineto{\pgfqpoint{2.082177in}{2.462897in}}%
\pgfpathlineto{\pgfqpoint{2.082841in}{2.314218in}}%
\pgfpathlineto{\pgfqpoint{2.083248in}{2.359494in}}%
\pgfpathlineto{\pgfqpoint{2.083272in}{2.319501in}}%
\pgfpathlineto{\pgfqpoint{2.083814in}{2.489316in}}%
\pgfpathlineto{\pgfqpoint{2.084331in}{2.399158in}}%
\pgfpathlineto{\pgfqpoint{2.084638in}{2.456474in}}%
\pgfpathlineto{\pgfqpoint{2.084811in}{2.323347in}}%
\pgfpathlineto{\pgfqpoint{2.085377in}{2.368547in}}%
\pgfpathlineto{\pgfqpoint{2.085635in}{2.314208in}}%
\pgfpathlineto{\pgfqpoint{2.086288in}{2.456996in}}%
\pgfpathlineto{\pgfqpoint{2.086509in}{2.338414in}}%
\pgfpathlineto{\pgfqpoint{2.087518in}{2.465856in}}%
\pgfpathlineto{\pgfqpoint{2.087174in}{2.322021in}}%
\pgfpathlineto{\pgfqpoint{2.087654in}{2.399354in}}%
\pgfpathlineto{\pgfqpoint{2.087998in}{2.324503in}}%
\pgfpathlineto{\pgfqpoint{2.087937in}{2.481990in}}%
\pgfpathlineto{\pgfqpoint{2.088737in}{2.412395in}}%
\pgfpathlineto{\pgfqpoint{2.088761in}{2.470467in}}%
\pgfpathlineto{\pgfqpoint{2.089758in}{2.323091in}}%
\pgfpathlineto{\pgfqpoint{2.089820in}{2.375285in}}%
\pgfpathlineto{\pgfqpoint{2.090349in}{2.320467in}}%
\pgfpathlineto{\pgfqpoint{2.090398in}{2.453457in}}%
\pgfpathlineto{\pgfqpoint{2.090903in}{2.379848in}}%
\pgfpathlineto{\pgfqpoint{2.091641in}{2.471688in}}%
\pgfpathlineto{\pgfqpoint{2.091814in}{2.334074in}}%
\pgfpathlineto{\pgfqpoint{2.091998in}{2.360412in}}%
\pgfpathlineto{\pgfqpoint{2.092121in}{2.314324in}}%
\pgfpathlineto{\pgfqpoint{2.092060in}{2.465378in}}%
\pgfpathlineto{\pgfqpoint{2.092860in}{2.404447in}}%
\pgfpathlineto{\pgfqpoint{2.093278in}{2.439283in}}%
\pgfpathlineto{\pgfqpoint{2.093906in}{2.325294in}}%
\pgfpathlineto{\pgfqpoint{2.093943in}{2.377622in}}%
\pgfpathlineto{\pgfqpoint{2.094731in}{2.330441in}}%
\pgfpathlineto{\pgfqpoint{2.094521in}{2.453035in}}%
\pgfpathlineto{\pgfqpoint{2.095026in}{2.410374in}}%
\pgfpathlineto{\pgfqpoint{2.095764in}{2.450548in}}%
\pgfpathlineto{\pgfqpoint{2.095826in}{2.327633in}}%
\pgfpathlineto{\pgfqpoint{2.096072in}{2.365493in}}%
\pgfpathlineto{\pgfqpoint{2.096847in}{2.329840in}}%
\pgfpathlineto{\pgfqpoint{2.096171in}{2.447110in}}%
\pgfpathlineto{\pgfqpoint{2.096971in}{2.426402in}}%
\pgfpathlineto{\pgfqpoint{2.097401in}{2.456135in}}%
\pgfpathlineto{\pgfqpoint{2.097672in}{2.335564in}}%
\pgfpathlineto{\pgfqpoint{2.098004in}{2.374758in}}%
\pgfpathlineto{\pgfqpoint{2.098029in}{2.323241in}}%
\pgfpathlineto{\pgfqpoint{2.098226in}{2.462131in}}%
\pgfpathlineto{\pgfqpoint{2.099112in}{2.347274in}}%
\pgfpathlineto{\pgfqpoint{2.099875in}{2.449924in}}%
\pgfpathlineto{\pgfqpoint{2.099223in}{2.330805in}}%
\pgfpathlineto{\pgfqpoint{2.100207in}{2.347239in}}%
\pgfpathlineto{\pgfqpoint{2.100761in}{2.330683in}}%
\pgfpathlineto{\pgfqpoint{2.100281in}{2.439087in}}%
\pgfpathlineto{\pgfqpoint{2.101081in}{2.409217in}}%
\pgfpathlineto{\pgfqpoint{2.101524in}{2.456601in}}%
\pgfpathlineto{\pgfqpoint{2.101574in}{2.310138in}}%
\pgfpathlineto{\pgfqpoint{2.102152in}{2.331063in}}%
\pgfpathlineto{\pgfqpoint{2.102164in}{2.329975in}}%
\pgfpathlineto{\pgfqpoint{2.102324in}{2.408430in}}%
\pgfpathlineto{\pgfqpoint{2.102349in}{2.450052in}}%
\pgfpathlineto{\pgfqpoint{2.103346in}{2.324754in}}%
\pgfpathlineto{\pgfqpoint{2.103407in}{2.367410in}}%
\pgfpathlineto{\pgfqpoint{2.103937in}{2.323396in}}%
\pgfpathlineto{\pgfqpoint{2.103986in}{2.461700in}}%
\pgfpathlineto{\pgfqpoint{2.104491in}{2.392107in}}%
\pgfpathlineto{\pgfqpoint{2.105217in}{2.460437in}}%
\pgfpathlineto{\pgfqpoint{2.105266in}{2.333570in}}%
\pgfpathlineto{\pgfqpoint{2.106300in}{2.319572in}}%
\pgfpathlineto{\pgfqpoint{2.105635in}{2.470242in}}%
\pgfpathlineto{\pgfqpoint{2.106312in}{2.333283in}}%
\pgfpathlineto{\pgfqpoint{2.106866in}{2.461797in}}%
\pgfpathlineto{\pgfqpoint{2.107038in}{2.325241in}}%
\pgfpathlineto{\pgfqpoint{2.107432in}{2.393470in}}%
\pgfpathlineto{\pgfqpoint{2.107481in}{2.320109in}}%
\pgfpathlineto{\pgfqpoint{2.107691in}{2.474555in}}%
\pgfpathlineto{\pgfqpoint{2.108564in}{2.340426in}}%
\pgfpathlineto{\pgfqpoint{2.109746in}{2.462103in}}%
\pgfpathlineto{\pgfqpoint{2.109401in}{2.320895in}}%
\pgfpathlineto{\pgfqpoint{2.109783in}{2.407892in}}%
\pgfpathlineto{\pgfqpoint{2.109992in}{2.323755in}}%
\pgfpathlineto{\pgfqpoint{2.110558in}{2.461975in}}%
\pgfpathlineto{\pgfqpoint{2.110915in}{2.347163in}}%
\pgfpathlineto{\pgfqpoint{2.111038in}{2.318120in}}%
\pgfpathlineto{\pgfqpoint{2.110977in}{2.478338in}}%
\pgfpathlineto{\pgfqpoint{2.111777in}{2.396126in}}%
\pgfpathlineto{\pgfqpoint{2.111801in}{2.472357in}}%
\pgfpathlineto{\pgfqpoint{2.112798in}{2.301692in}}%
\pgfpathlineto{\pgfqpoint{2.112872in}{2.347315in}}%
\pgfpathlineto{\pgfqpoint{2.113450in}{2.468782in}}%
\pgfpathlineto{\pgfqpoint{2.113389in}{2.320815in}}%
\pgfpathlineto{\pgfqpoint{2.113783in}{2.344880in}}%
\pgfpathlineto{\pgfqpoint{2.114214in}{2.329792in}}%
\pgfpathlineto{\pgfqpoint{2.114681in}{2.482618in}}%
\pgfpathlineto{\pgfqpoint{2.114866in}{2.362482in}}%
\pgfpathlineto{\pgfqpoint{2.115100in}{2.481492in}}%
\pgfpathlineto{\pgfqpoint{2.115149in}{2.316585in}}%
\pgfpathlineto{\pgfqpoint{2.115961in}{2.368461in}}%
\pgfpathlineto{\pgfqpoint{2.116921in}{2.320862in}}%
\pgfpathlineto{\pgfqpoint{2.116737in}{2.458818in}}%
\pgfpathlineto{\pgfqpoint{2.117081in}{2.325648in}}%
\pgfpathlineto{\pgfqpoint{2.117561in}{2.480311in}}%
\pgfpathlineto{\pgfqpoint{2.117512in}{2.317686in}}%
\pgfpathlineto{\pgfqpoint{2.118287in}{2.399134in}}%
\pgfpathlineto{\pgfqpoint{2.118854in}{2.318805in}}%
\pgfpathlineto{\pgfqpoint{2.118804in}{2.472846in}}%
\pgfpathlineto{\pgfqpoint{2.119444in}{2.335967in}}%
\pgfpathlineto{\pgfqpoint{2.120429in}{2.476531in}}%
\pgfpathlineto{\pgfqpoint{2.119875in}{2.311596in}}%
\pgfpathlineto{\pgfqpoint{2.120589in}{2.371838in}}%
\pgfpathlineto{\pgfqpoint{2.121204in}{2.308944in}}%
\pgfpathlineto{\pgfqpoint{2.121266in}{2.483467in}}%
\pgfpathlineto{\pgfqpoint{2.121660in}{2.426074in}}%
\pgfpathlineto{\pgfqpoint{2.121672in}{2.465181in}}%
\pgfpathlineto{\pgfqpoint{2.122250in}{2.324260in}}%
\pgfpathlineto{\pgfqpoint{2.122743in}{2.344892in}}%
\pgfpathlineto{\pgfqpoint{2.123727in}{2.478385in}}%
\pgfpathlineto{\pgfqpoint{2.123666in}{2.315304in}}%
\pgfpathlineto{\pgfqpoint{2.123875in}{2.364835in}}%
\pgfpathlineto{\pgfqpoint{2.124601in}{2.310877in}}%
\pgfpathlineto{\pgfqpoint{2.124552in}{2.492646in}}%
\pgfpathlineto{\pgfqpoint{2.124934in}{2.416110in}}%
\pgfpathlineto{\pgfqpoint{2.125377in}{2.477127in}}%
\pgfpathlineto{\pgfqpoint{2.125426in}{2.317527in}}%
\pgfpathlineto{\pgfqpoint{2.126004in}{2.347995in}}%
\pgfpathlineto{\pgfqpoint{2.126964in}{2.313504in}}%
\pgfpathlineto{\pgfqpoint{2.126595in}{2.455265in}}%
\pgfpathlineto{\pgfqpoint{2.126977in}{2.368998in}}%
\pgfpathlineto{\pgfqpoint{2.127014in}{2.484540in}}%
\pgfpathlineto{\pgfqpoint{2.127370in}{2.303628in}}%
\pgfpathlineto{\pgfqpoint{2.128084in}{2.360167in}}%
\pgfpathlineto{\pgfqpoint{2.128601in}{2.319507in}}%
\pgfpathlineto{\pgfqpoint{2.128244in}{2.487519in}}%
\pgfpathlineto{\pgfqpoint{2.129044in}{2.441682in}}%
\pgfpathlineto{\pgfqpoint{2.129069in}{2.470710in}}%
\pgfpathlineto{\pgfqpoint{2.129327in}{2.314772in}}%
\pgfpathlineto{\pgfqpoint{2.130103in}{2.363203in}}%
\pgfpathlineto{\pgfqpoint{2.130152in}{2.320130in}}%
\pgfpathlineto{\pgfqpoint{2.130201in}{2.418048in}}%
\pgfpathlineto{\pgfqpoint{2.130275in}{2.415745in}}%
\pgfpathlineto{\pgfqpoint{2.130706in}{2.482828in}}%
\pgfpathlineto{\pgfqpoint{2.131075in}{2.310954in}}%
\pgfpathlineto{\pgfqpoint{2.131346in}{2.326714in}}%
\pgfpathlineto{\pgfqpoint{2.131543in}{2.463477in}}%
\pgfpathlineto{\pgfqpoint{2.131481in}{2.308276in}}%
\pgfpathlineto{\pgfqpoint{2.132404in}{2.333502in}}%
\pgfpathlineto{\pgfqpoint{2.133118in}{2.309515in}}%
\pgfpathlineto{\pgfqpoint{2.133180in}{2.472243in}}%
\pgfpathlineto{\pgfqpoint{2.133463in}{2.369937in}}%
\pgfpathlineto{\pgfqpoint{2.134004in}{2.482642in}}%
\pgfpathlineto{\pgfqpoint{2.133844in}{2.321252in}}%
\pgfpathlineto{\pgfqpoint{2.134570in}{2.366406in}}%
\pgfpathlineto{\pgfqpoint{2.135493in}{2.310559in}}%
\pgfpathlineto{\pgfqpoint{2.134829in}{2.481431in}}%
\pgfpathlineto{\pgfqpoint{2.135530in}{2.420445in}}%
\pgfpathlineto{\pgfqpoint{2.136466in}{2.477361in}}%
\pgfpathlineto{\pgfqpoint{2.136404in}{2.308741in}}%
\pgfpathlineto{\pgfqpoint{2.136601in}{2.391133in}}%
\pgfpathlineto{\pgfqpoint{2.136823in}{2.306525in}}%
\pgfpathlineto{\pgfqpoint{2.136872in}{2.475147in}}%
\pgfpathlineto{\pgfqpoint{2.137672in}{2.429658in}}%
\pgfpathlineto{\pgfqpoint{2.138115in}{2.487557in}}%
\pgfpathlineto{\pgfqpoint{2.138460in}{2.312407in}}%
\pgfpathlineto{\pgfqpoint{2.138743in}{2.340822in}}%
\pgfpathlineto{\pgfqpoint{2.139186in}{2.309089in}}%
\pgfpathlineto{\pgfqpoint{2.138940in}{2.479952in}}%
\pgfpathlineto{\pgfqpoint{2.139826in}{2.357411in}}%
\pgfpathlineto{\pgfqpoint{2.140158in}{2.492357in}}%
\pgfpathlineto{\pgfqpoint{2.140527in}{2.315973in}}%
\pgfpathlineto{\pgfqpoint{2.140909in}{2.342344in}}%
\pgfpathlineto{\pgfqpoint{2.140933in}{2.301665in}}%
\pgfpathlineto{\pgfqpoint{2.141807in}{2.484260in}}%
\pgfpathlineto{\pgfqpoint{2.142004in}{2.350693in}}%
\pgfpathlineto{\pgfqpoint{2.142620in}{2.483357in}}%
\pgfpathlineto{\pgfqpoint{2.142570in}{2.297946in}}%
\pgfpathlineto{\pgfqpoint{2.143161in}{2.390178in}}%
\pgfpathlineto{\pgfqpoint{2.143297in}{2.299077in}}%
\pgfpathlineto{\pgfqpoint{2.143444in}{2.483800in}}%
\pgfpathlineto{\pgfqpoint{2.144232in}{2.366852in}}%
\pgfpathlineto{\pgfqpoint{2.144269in}{2.497046in}}%
\pgfpathlineto{\pgfqpoint{2.144933in}{2.295844in}}%
\pgfpathlineto{\pgfqpoint{2.145327in}{2.336150in}}%
\pgfpathlineto{\pgfqpoint{2.145857in}{2.299477in}}%
\pgfpathlineto{\pgfqpoint{2.145906in}{2.483525in}}%
\pgfpathlineto{\pgfqpoint{2.146300in}{2.435223in}}%
\pgfpathlineto{\pgfqpoint{2.146730in}{2.479312in}}%
\pgfpathlineto{\pgfqpoint{2.146681in}{2.305668in}}%
\pgfpathlineto{\pgfqpoint{2.147370in}{2.344536in}}%
\pgfpathlineto{\pgfqpoint{2.148220in}{2.296023in}}%
\pgfpathlineto{\pgfqpoint{2.147555in}{2.487368in}}%
\pgfpathlineto{\pgfqpoint{2.148355in}{2.435507in}}%
\pgfpathlineto{\pgfqpoint{2.148380in}{2.480268in}}%
\pgfpathlineto{\pgfqpoint{2.149044in}{2.291174in}}%
\pgfpathlineto{\pgfqpoint{2.149426in}{2.346880in}}%
\pgfpathlineto{\pgfqpoint{2.150275in}{2.306453in}}%
\pgfpathlineto{\pgfqpoint{2.150029in}{2.484904in}}%
\pgfpathlineto{\pgfqpoint{2.150398in}{2.370049in}}%
\pgfpathlineto{\pgfqpoint{2.151247in}{2.486466in}}%
\pgfpathlineto{\pgfqpoint{2.150693in}{2.316760in}}%
\pgfpathlineto{\pgfqpoint{2.151493in}{2.329612in}}%
\pgfpathlineto{\pgfqpoint{2.152023in}{2.297465in}}%
\pgfpathlineto{\pgfqpoint{2.151666in}{2.486008in}}%
\pgfpathlineto{\pgfqpoint{2.152564in}{2.377315in}}%
\pgfpathlineto{\pgfqpoint{2.152897in}{2.498383in}}%
\pgfpathlineto{\pgfqpoint{2.152749in}{2.288774in}}%
\pgfpathlineto{\pgfqpoint{2.153647in}{2.348527in}}%
\pgfpathlineto{\pgfqpoint{2.154386in}{2.291938in}}%
\pgfpathlineto{\pgfqpoint{2.153721in}{2.495791in}}%
\pgfpathlineto{\pgfqpoint{2.154743in}{2.366073in}}%
\pgfpathlineto{\pgfqpoint{2.155210in}{2.304307in}}%
\pgfpathlineto{\pgfqpoint{2.155358in}{2.494390in}}%
\pgfpathlineto{\pgfqpoint{2.155666in}{2.412288in}}%
\pgfpathlineto{\pgfqpoint{2.156589in}{2.496914in}}%
\pgfpathlineto{\pgfqpoint{2.156133in}{2.304655in}}%
\pgfpathlineto{\pgfqpoint{2.156638in}{2.353710in}}%
\pgfpathlineto{\pgfqpoint{2.156860in}{2.290684in}}%
\pgfpathlineto{\pgfqpoint{2.157007in}{2.486089in}}%
\pgfpathlineto{\pgfqpoint{2.157696in}{2.358897in}}%
\pgfpathlineto{\pgfqpoint{2.157832in}{2.495706in}}%
\pgfpathlineto{\pgfqpoint{2.158496in}{2.296102in}}%
\pgfpathlineto{\pgfqpoint{2.158792in}{2.362094in}}%
\pgfpathlineto{\pgfqpoint{2.159727in}{2.300612in}}%
\pgfpathlineto{\pgfqpoint{2.159469in}{2.488982in}}%
\pgfpathlineto{\pgfqpoint{2.159850in}{2.378571in}}%
\pgfpathlineto{\pgfqpoint{2.160700in}{2.494427in}}%
\pgfpathlineto{\pgfqpoint{2.160552in}{2.308005in}}%
\pgfpathlineto{\pgfqpoint{2.160946in}{2.320131in}}%
\pgfpathlineto{\pgfqpoint{2.160958in}{2.299133in}}%
\pgfpathlineto{\pgfqpoint{2.161930in}{2.491320in}}%
\pgfpathlineto{\pgfqpoint{2.162016in}{2.375516in}}%
\pgfpathlineto{\pgfqpoint{2.162349in}{2.492683in}}%
\pgfpathlineto{\pgfqpoint{2.162189in}{2.295535in}}%
\pgfpathlineto{\pgfqpoint{2.163100in}{2.338416in}}%
\pgfpathlineto{\pgfqpoint{2.163838in}{2.290785in}}%
\pgfpathlineto{\pgfqpoint{2.163580in}{2.488801in}}%
\pgfpathlineto{\pgfqpoint{2.164170in}{2.364847in}}%
\pgfpathlineto{\pgfqpoint{2.164810in}{2.490627in}}%
\pgfpathlineto{\pgfqpoint{2.165069in}{2.301337in}}%
\pgfpathlineto{\pgfqpoint{2.165253in}{2.371027in}}%
\pgfpathlineto{\pgfqpoint{2.166300in}{2.295281in}}%
\pgfpathlineto{\pgfqpoint{2.166041in}{2.492312in}}%
\pgfpathlineto{\pgfqpoint{2.166336in}{2.392661in}}%
\pgfpathlineto{\pgfqpoint{2.166460in}{2.493987in}}%
\pgfpathlineto{\pgfqpoint{2.167124in}{2.304358in}}%
\pgfpathlineto{\pgfqpoint{2.167420in}{2.363877in}}%
\pgfpathlineto{\pgfqpoint{2.167949in}{2.292583in}}%
\pgfpathlineto{\pgfqpoint{2.167690in}{2.483128in}}%
\pgfpathlineto{\pgfqpoint{2.168490in}{2.446301in}}%
\pgfpathlineto{\pgfqpoint{2.169340in}{2.483552in}}%
\pgfpathlineto{\pgfqpoint{2.168773in}{2.305870in}}%
\pgfpathlineto{\pgfqpoint{2.169561in}{2.354030in}}%
\pgfpathlineto{\pgfqpoint{2.170410in}{2.301155in}}%
\pgfpathlineto{\pgfqpoint{2.170152in}{2.489859in}}%
\pgfpathlineto{\pgfqpoint{2.170644in}{2.368831in}}%
\pgfpathlineto{\pgfqpoint{2.171395in}{2.485572in}}%
\pgfpathlineto{\pgfqpoint{2.171641in}{2.295297in}}%
\pgfpathlineto{\pgfqpoint{2.171727in}{2.342895in}}%
\pgfpathlineto{\pgfqpoint{2.172466in}{2.307663in}}%
\pgfpathlineto{\pgfqpoint{2.172626in}{2.489827in}}%
\pgfpathlineto{\pgfqpoint{2.172835in}{2.345899in}}%
\pgfpathlineto{\pgfqpoint{2.173290in}{2.297260in}}%
\pgfpathlineto{\pgfqpoint{2.173032in}{2.487023in}}%
\pgfpathlineto{\pgfqpoint{2.173733in}{2.403207in}}%
\pgfpathlineto{\pgfqpoint{2.174681in}{2.490251in}}%
\pgfpathlineto{\pgfqpoint{2.174521in}{2.302309in}}%
\pgfpathlineto{\pgfqpoint{2.174816in}{2.373928in}}%
\pgfpathlineto{\pgfqpoint{2.175764in}{2.296294in}}%
\pgfpathlineto{\pgfqpoint{2.175506in}{2.491193in}}%
\pgfpathlineto{\pgfqpoint{2.175887in}{2.426215in}}%
\pgfpathlineto{\pgfqpoint{2.176736in}{2.500495in}}%
\pgfpathlineto{\pgfqpoint{2.176576in}{2.303684in}}%
\pgfpathlineto{\pgfqpoint{2.176958in}{2.344838in}}%
\pgfpathlineto{\pgfqpoint{2.177401in}{2.295198in}}%
\pgfpathlineto{\pgfqpoint{2.177967in}{2.489852in}}%
\pgfpathlineto{\pgfqpoint{2.178041in}{2.354505in}}%
\pgfpathlineto{\pgfqpoint{2.178792in}{2.493283in}}%
\pgfpathlineto{\pgfqpoint{2.178226in}{2.303880in}}%
\pgfpathlineto{\pgfqpoint{2.179136in}{2.327162in}}%
\pgfpathlineto{\pgfqpoint{2.179875in}{2.294824in}}%
\pgfpathlineto{\pgfqpoint{2.179616in}{2.499688in}}%
\pgfpathlineto{\pgfqpoint{2.179998in}{2.415657in}}%
\pgfpathlineto{\pgfqpoint{2.180847in}{2.494774in}}%
\pgfpathlineto{\pgfqpoint{2.180687in}{2.300797in}}%
\pgfpathlineto{\pgfqpoint{2.181081in}{2.316700in}}%
\pgfpathlineto{\pgfqpoint{2.181106in}{2.288487in}}%
\pgfpathlineto{\pgfqpoint{2.182078in}{2.493247in}}%
\pgfpathlineto{\pgfqpoint{2.182152in}{2.351694in}}%
\pgfpathlineto{\pgfqpoint{2.182484in}{2.486661in}}%
\pgfpathlineto{\pgfqpoint{2.182743in}{2.298863in}}%
\pgfpathlineto{\pgfqpoint{2.183247in}{2.329012in}}%
\pgfpathlineto{\pgfqpoint{2.183567in}{2.295411in}}%
\pgfpathlineto{\pgfqpoint{2.183309in}{2.487145in}}%
\pgfpathlineto{\pgfqpoint{2.183703in}{2.442493in}}%
\pgfpathlineto{\pgfqpoint{2.184539in}{2.494517in}}%
\pgfpathlineto{\pgfqpoint{2.183973in}{2.299375in}}%
\pgfpathlineto{\pgfqpoint{2.184773in}{2.340133in}}%
\pgfpathlineto{\pgfqpoint{2.185216in}{2.288976in}}%
\pgfpathlineto{\pgfqpoint{2.184958in}{2.496177in}}%
\pgfpathlineto{\pgfqpoint{2.185856in}{2.361076in}}%
\pgfpathlineto{\pgfqpoint{2.186189in}{2.492892in}}%
\pgfpathlineto{\pgfqpoint{2.186853in}{2.290536in}}%
\pgfpathlineto{\pgfqpoint{2.186952in}{2.320291in}}%
\pgfpathlineto{\pgfqpoint{2.186964in}{2.319527in}}%
\pgfpathlineto{\pgfqpoint{2.186976in}{2.353125in}}%
\pgfpathlineto{\pgfqpoint{2.187419in}{2.489948in}}%
\pgfpathlineto{\pgfqpoint{2.187259in}{2.295919in}}%
\pgfpathlineto{\pgfqpoint{2.188072in}{2.323777in}}%
\pgfpathlineto{\pgfqpoint{2.188909in}{2.291977in}}%
\pgfpathlineto{\pgfqpoint{2.189069in}{2.498874in}}%
\pgfpathlineto{\pgfqpoint{2.189143in}{2.362019in}}%
\pgfpathlineto{\pgfqpoint{2.189893in}{2.488998in}}%
\pgfpathlineto{\pgfqpoint{2.189327in}{2.293243in}}%
\pgfpathlineto{\pgfqpoint{2.190226in}{2.348552in}}%
\pgfpathlineto{\pgfqpoint{2.190558in}{2.282957in}}%
\pgfpathlineto{\pgfqpoint{2.190299in}{2.496782in}}%
\pgfpathlineto{\pgfqpoint{2.191309in}{2.375113in}}%
\pgfpathlineto{\pgfqpoint{2.191530in}{2.494423in}}%
\pgfpathlineto{\pgfqpoint{2.191383in}{2.286892in}}%
\pgfpathlineto{\pgfqpoint{2.192392in}{2.374702in}}%
\pgfpathlineto{\pgfqpoint{2.193019in}{2.283209in}}%
\pgfpathlineto{\pgfqpoint{2.193179in}{2.496756in}}%
\pgfpathlineto{\pgfqpoint{2.193475in}{2.416861in}}%
\pgfpathlineto{\pgfqpoint{2.194410in}{2.504001in}}%
\pgfpathlineto{\pgfqpoint{2.194263in}{2.283073in}}%
\pgfpathlineto{\pgfqpoint{2.194546in}{2.375430in}}%
\pgfpathlineto{\pgfqpoint{2.194669in}{2.273632in}}%
\pgfpathlineto{\pgfqpoint{2.195235in}{2.499650in}}%
\pgfpathlineto{\pgfqpoint{2.195616in}{2.398217in}}%
\pgfpathlineto{\pgfqpoint{2.195641in}{2.504923in}}%
\pgfpathlineto{\pgfqpoint{2.195899in}{2.275953in}}%
\pgfpathlineto{\pgfqpoint{2.196699in}{2.342643in}}%
\pgfpathlineto{\pgfqpoint{2.197130in}{2.278660in}}%
\pgfpathlineto{\pgfqpoint{2.197696in}{2.508762in}}%
\pgfpathlineto{\pgfqpoint{2.197795in}{2.399884in}}%
\pgfpathlineto{\pgfqpoint{2.198521in}{2.509381in}}%
\pgfpathlineto{\pgfqpoint{2.198779in}{2.275299in}}%
\pgfpathlineto{\pgfqpoint{2.198866in}{2.321108in}}%
\pgfpathlineto{\pgfqpoint{2.199604in}{2.277432in}}%
\pgfpathlineto{\pgfqpoint{2.199346in}{2.506941in}}%
\pgfpathlineto{\pgfqpoint{2.199641in}{2.438563in}}%
\pgfpathlineto{\pgfqpoint{2.199752in}{2.509676in}}%
\pgfpathlineto{\pgfqpoint{2.200010in}{2.265099in}}%
\pgfpathlineto{\pgfqpoint{2.200699in}{2.390765in}}%
\pgfpathlineto{\pgfqpoint{2.201241in}{2.266753in}}%
\pgfpathlineto{\pgfqpoint{2.200982in}{2.509923in}}%
\pgfpathlineto{\pgfqpoint{2.201782in}{2.432904in}}%
\pgfpathlineto{\pgfqpoint{2.201807in}{2.510483in}}%
\pgfpathlineto{\pgfqpoint{2.202472in}{2.261649in}}%
\pgfpathlineto{\pgfqpoint{2.202866in}{2.320258in}}%
\pgfpathlineto{\pgfqpoint{2.203702in}{2.260007in}}%
\pgfpathlineto{\pgfqpoint{2.203862in}{2.515345in}}%
\pgfpathlineto{\pgfqpoint{2.203949in}{2.397539in}}%
\pgfpathlineto{\pgfqpoint{2.204675in}{2.512983in}}%
\pgfpathlineto{\pgfqpoint{2.204121in}{2.256343in}}%
\pgfpathlineto{\pgfqpoint{2.205019in}{2.323789in}}%
\pgfpathlineto{\pgfqpoint{2.205758in}{2.250873in}}%
\pgfpathlineto{\pgfqpoint{2.205499in}{2.518721in}}%
\pgfpathlineto{\pgfqpoint{2.205893in}{2.467457in}}%
\pgfpathlineto{\pgfqpoint{2.206730in}{2.526607in}}%
\pgfpathlineto{\pgfqpoint{2.206582in}{2.251427in}}%
\pgfpathlineto{\pgfqpoint{2.206964in}{2.367774in}}%
\pgfpathlineto{\pgfqpoint{2.207813in}{2.240671in}}%
\pgfpathlineto{\pgfqpoint{2.207961in}{2.529474in}}%
\pgfpathlineto{\pgfqpoint{2.208059in}{2.425899in}}%
\pgfpathlineto{\pgfqpoint{2.208786in}{2.532878in}}%
\pgfpathlineto{\pgfqpoint{2.208638in}{2.247800in}}%
\pgfpathlineto{\pgfqpoint{2.209032in}{2.279173in}}%
\pgfpathlineto{\pgfqpoint{2.209044in}{2.238887in}}%
\pgfpathlineto{\pgfqpoint{2.210016in}{2.531137in}}%
\pgfpathlineto{\pgfqpoint{2.210102in}{2.405716in}}%
\pgfpathlineto{\pgfqpoint{2.210841in}{2.534329in}}%
\pgfpathlineto{\pgfqpoint{2.211099in}{2.239584in}}%
\pgfpathlineto{\pgfqpoint{2.211173in}{2.332158in}}%
\pgfpathlineto{\pgfqpoint{2.211924in}{2.234228in}}%
\pgfpathlineto{\pgfqpoint{2.212072in}{2.544309in}}%
\pgfpathlineto{\pgfqpoint{2.212256in}{2.368520in}}%
\pgfpathlineto{\pgfqpoint{2.213302in}{2.551865in}}%
\pgfpathlineto{\pgfqpoint{2.213155in}{2.229097in}}%
\pgfpathlineto{\pgfqpoint{2.213339in}{2.369572in}}%
\pgfpathlineto{\pgfqpoint{2.213979in}{2.231813in}}%
\pgfpathlineto{\pgfqpoint{2.214127in}{2.555059in}}%
\pgfpathlineto{\pgfqpoint{2.214422in}{2.428147in}}%
\pgfpathlineto{\pgfqpoint{2.214533in}{2.551414in}}%
\pgfpathlineto{\pgfqpoint{2.215210in}{2.227498in}}%
\pgfpathlineto{\pgfqpoint{2.215493in}{2.348111in}}%
\pgfpathlineto{\pgfqpoint{2.216035in}{2.231633in}}%
\pgfpathlineto{\pgfqpoint{2.216182in}{2.552613in}}%
\pgfpathlineto{\pgfqpoint{2.216576in}{2.491305in}}%
\pgfpathlineto{\pgfqpoint{2.216589in}{2.544092in}}%
\pgfpathlineto{\pgfqpoint{2.217265in}{2.228920in}}%
\pgfpathlineto{\pgfqpoint{2.217647in}{2.372853in}}%
\pgfpathlineto{\pgfqpoint{2.218090in}{2.235076in}}%
\pgfpathlineto{\pgfqpoint{2.218238in}{2.536026in}}%
\pgfpathlineto{\pgfqpoint{2.218742in}{2.438142in}}%
\pgfpathlineto{\pgfqpoint{2.219062in}{2.517362in}}%
\pgfpathlineto{\pgfqpoint{2.218915in}{2.237074in}}%
\pgfpathlineto{\pgfqpoint{2.219715in}{2.313921in}}%
\pgfpathlineto{\pgfqpoint{2.220145in}{2.251611in}}%
\pgfpathlineto{\pgfqpoint{2.220293in}{2.503735in}}%
\pgfpathlineto{\pgfqpoint{2.220798in}{2.423036in}}%
\pgfpathlineto{\pgfqpoint{2.221118in}{2.488969in}}%
\pgfpathlineto{\pgfqpoint{2.220970in}{2.259464in}}%
\pgfpathlineto{\pgfqpoint{2.221770in}{2.321045in}}%
\pgfpathlineto{\pgfqpoint{2.222201in}{2.262823in}}%
\pgfpathlineto{\pgfqpoint{2.221942in}{2.484333in}}%
\pgfpathlineto{\pgfqpoint{2.222853in}{2.408276in}}%
\pgfpathlineto{\pgfqpoint{2.223185in}{2.468733in}}%
\pgfpathlineto{\pgfqpoint{2.223025in}{2.264407in}}%
\pgfpathlineto{\pgfqpoint{2.223419in}{2.303696in}}%
\pgfpathlineto{\pgfqpoint{2.223432in}{2.275030in}}%
\pgfpathlineto{\pgfqpoint{2.223592in}{2.472292in}}%
\pgfpathlineto{\pgfqpoint{2.224478in}{2.397130in}}%
\pgfpathlineto{\pgfqpoint{2.224502in}{2.404713in}}%
\pgfpathlineto{\pgfqpoint{2.224564in}{2.322137in}}%
\pgfpathlineto{\pgfqpoint{2.224650in}{2.323752in}}%
\pgfpathlineto{\pgfqpoint{2.225081in}{2.287489in}}%
\pgfpathlineto{\pgfqpoint{2.225241in}{2.463176in}}%
\pgfpathlineto{\pgfqpoint{2.225733in}{2.400306in}}%
\pgfpathlineto{\pgfqpoint{2.226312in}{2.296642in}}%
\pgfpathlineto{\pgfqpoint{2.226065in}{2.458751in}}%
\pgfpathlineto{\pgfqpoint{2.226841in}{2.391381in}}%
\pgfpathlineto{\pgfqpoint{2.226890in}{2.450219in}}%
\pgfpathlineto{\pgfqpoint{2.227136in}{2.301399in}}%
\pgfpathlineto{\pgfqpoint{2.227924in}{2.370321in}}%
\pgfpathlineto{\pgfqpoint{2.228367in}{2.308345in}}%
\pgfpathlineto{\pgfqpoint{2.228121in}{2.444272in}}%
\pgfpathlineto{\pgfqpoint{2.229019in}{2.398831in}}%
\pgfpathlineto{\pgfqpoint{2.229192in}{2.315872in}}%
\pgfpathlineto{\pgfqpoint{2.229770in}{2.438757in}}%
\pgfpathlineto{\pgfqpoint{2.230127in}{2.401779in}}%
\pgfpathlineto{\pgfqpoint{2.230422in}{2.323429in}}%
\pgfpathlineto{\pgfqpoint{2.230176in}{2.441620in}}%
\pgfpathlineto{\pgfqpoint{2.231321in}{2.359658in}}%
\pgfpathlineto{\pgfqpoint{2.231825in}{2.435250in}}%
\pgfpathlineto{\pgfqpoint{2.231653in}{2.328266in}}%
\pgfpathlineto{\pgfqpoint{2.232429in}{2.358893in}}%
\pgfpathlineto{\pgfqpoint{2.232650in}{2.430766in}}%
\pgfpathlineto{\pgfqpoint{2.233229in}{2.326605in}}%
\pgfpathlineto{\pgfqpoint{2.233561in}{2.370002in}}%
\pgfpathlineto{\pgfqpoint{2.233635in}{2.327008in}}%
\pgfpathlineto{\pgfqpoint{2.233893in}{2.427934in}}%
\pgfpathlineto{\pgfqpoint{2.234619in}{2.371749in}}%
\pgfpathlineto{\pgfqpoint{2.234705in}{2.426093in}}%
\pgfpathlineto{\pgfqpoint{2.234865in}{2.325820in}}%
\pgfpathlineto{\pgfqpoint{2.235715in}{2.351652in}}%
\pgfpathlineto{\pgfqpoint{2.235949in}{2.425893in}}%
\pgfpathlineto{\pgfqpoint{2.236109in}{2.330885in}}%
\pgfpathlineto{\pgfqpoint{2.236884in}{2.382491in}}%
\pgfpathlineto{\pgfqpoint{2.237339in}{2.327615in}}%
\pgfpathlineto{\pgfqpoint{2.237179in}{2.418579in}}%
\pgfpathlineto{\pgfqpoint{2.237967in}{2.400146in}}%
\pgfpathlineto{\pgfqpoint{2.238004in}{2.421198in}}%
\pgfpathlineto{\pgfqpoint{2.238570in}{2.329934in}}%
\pgfpathlineto{\pgfqpoint{2.238964in}{2.350013in}}%
\pgfpathlineto{\pgfqpoint{2.238976in}{2.331189in}}%
\pgfpathlineto{\pgfqpoint{2.239235in}{2.418139in}}%
\pgfpathlineto{\pgfqpoint{2.240035in}{2.402560in}}%
\pgfpathlineto{\pgfqpoint{2.240059in}{2.417536in}}%
\pgfpathlineto{\pgfqpoint{2.240625in}{2.335971in}}%
\pgfpathlineto{\pgfqpoint{2.241068in}{2.360340in}}%
\pgfpathlineto{\pgfqpoint{2.241450in}{2.336217in}}%
\pgfpathlineto{\pgfqpoint{2.241290in}{2.416574in}}%
\pgfpathlineto{\pgfqpoint{2.242102in}{2.406545in}}%
\pgfpathlineto{\pgfqpoint{2.242115in}{2.415107in}}%
\pgfpathlineto{\pgfqpoint{2.242681in}{2.337827in}}%
\pgfpathlineto{\pgfqpoint{2.243136in}{2.353384in}}%
\pgfpathlineto{\pgfqpoint{2.243505in}{2.335668in}}%
\pgfpathlineto{\pgfqpoint{2.243764in}{2.416278in}}%
\pgfpathlineto{\pgfqpoint{2.244158in}{2.401165in}}%
\pgfpathlineto{\pgfqpoint{2.244588in}{2.413973in}}%
\pgfpathlineto{\pgfqpoint{2.244736in}{2.338863in}}%
\pgfpathlineto{\pgfqpoint{2.245192in}{2.356474in}}%
\pgfpathlineto{\pgfqpoint{2.245561in}{2.340178in}}%
\pgfpathlineto{\pgfqpoint{2.245413in}{2.415825in}}%
\pgfpathlineto{\pgfqpoint{2.246225in}{2.411146in}}%
\pgfpathlineto{\pgfqpoint{2.246644in}{2.415129in}}%
\pgfpathlineto{\pgfqpoint{2.246447in}{2.343220in}}%
\pgfpathlineto{\pgfqpoint{2.247161in}{2.385534in}}%
\pgfpathlineto{\pgfqpoint{2.247272in}{2.343905in}}%
\pgfpathlineto{\pgfqpoint{2.247468in}{2.415660in}}%
\pgfpathlineto{\pgfqpoint{2.248256in}{2.388572in}}%
\pgfpathlineto{\pgfqpoint{2.248699in}{2.414199in}}%
\pgfpathlineto{\pgfqpoint{2.248515in}{2.344546in}}%
\pgfpathlineto{\pgfqpoint{2.249315in}{2.349743in}}%
\pgfpathlineto{\pgfqpoint{2.249327in}{2.344000in}}%
\pgfpathlineto{\pgfqpoint{2.249524in}{2.415064in}}%
\pgfpathlineto{\pgfqpoint{2.250324in}{2.394566in}}%
\pgfpathlineto{\pgfqpoint{2.250755in}{2.411124in}}%
\pgfpathlineto{\pgfqpoint{2.250570in}{2.347551in}}%
\pgfpathlineto{\pgfqpoint{2.251382in}{2.348778in}}%
\pgfpathlineto{\pgfqpoint{2.251801in}{2.346287in}}%
\pgfpathlineto{\pgfqpoint{2.251579in}{2.409866in}}%
\pgfpathlineto{\pgfqpoint{2.251973in}{2.397517in}}%
\pgfpathlineto{\pgfqpoint{2.252822in}{2.410844in}}%
\pgfpathlineto{\pgfqpoint{2.252625in}{2.347400in}}%
\pgfpathlineto{\pgfqpoint{2.253019in}{2.352914in}}%
\pgfpathlineto{\pgfqpoint{2.253856in}{2.348557in}}%
\pgfpathlineto{\pgfqpoint{2.253647in}{2.409821in}}%
\pgfpathlineto{\pgfqpoint{2.254016in}{2.381435in}}%
\pgfpathlineto{\pgfqpoint{2.254878in}{2.406997in}}%
\pgfpathlineto{\pgfqpoint{2.254681in}{2.344377in}}%
\pgfpathlineto{\pgfqpoint{2.255087in}{2.346551in}}%
\pgfpathlineto{\pgfqpoint{2.255924in}{2.343361in}}%
\pgfpathlineto{\pgfqpoint{2.255702in}{2.402655in}}%
\pgfpathlineto{\pgfqpoint{2.256072in}{2.371346in}}%
\pgfpathlineto{\pgfqpoint{2.256933in}{2.404021in}}%
\pgfpathlineto{\pgfqpoint{2.256748in}{2.345431in}}%
\pgfpathlineto{\pgfqpoint{2.257142in}{2.346239in}}%
\pgfpathlineto{\pgfqpoint{2.257155in}{2.342412in}}%
\pgfpathlineto{\pgfqpoint{2.257758in}{2.404110in}}%
\pgfpathlineto{\pgfqpoint{2.258127in}{2.370706in}}%
\pgfpathlineto{\pgfqpoint{2.259001in}{2.405904in}}%
\pgfpathlineto{\pgfqpoint{2.258804in}{2.343052in}}%
\pgfpathlineto{\pgfqpoint{2.259198in}{2.349673in}}%
\pgfpathlineto{\pgfqpoint{2.260035in}{2.341230in}}%
\pgfpathlineto{\pgfqpoint{2.259407in}{2.399254in}}%
\pgfpathlineto{\pgfqpoint{2.260207in}{2.385567in}}%
\pgfpathlineto{\pgfqpoint{2.261068in}{2.400417in}}%
\pgfpathlineto{\pgfqpoint{2.260859in}{2.344177in}}%
\pgfpathlineto{\pgfqpoint{2.261253in}{2.351750in}}%
\pgfpathlineto{\pgfqpoint{2.261278in}{2.343741in}}%
\pgfpathlineto{\pgfqpoint{2.261893in}{2.400252in}}%
\pgfpathlineto{\pgfqpoint{2.262275in}{2.391817in}}%
\pgfpathlineto{\pgfqpoint{2.263124in}{2.400261in}}%
\pgfpathlineto{\pgfqpoint{2.262927in}{2.347241in}}%
\pgfpathlineto{\pgfqpoint{2.263296in}{2.361015in}}%
\pgfpathlineto{\pgfqpoint{2.263333in}{2.345937in}}%
\pgfpathlineto{\pgfqpoint{2.263948in}{2.395732in}}%
\pgfpathlineto{\pgfqpoint{2.264318in}{2.383235in}}%
\pgfpathlineto{\pgfqpoint{2.265216in}{2.396665in}}%
\pgfpathlineto{\pgfqpoint{2.264982in}{2.348360in}}%
\pgfpathlineto{\pgfqpoint{2.265364in}{2.359569in}}%
\pgfpathlineto{\pgfqpoint{2.266213in}{2.347408in}}%
\pgfpathlineto{\pgfqpoint{2.266053in}{2.397782in}}%
\pgfpathlineto{\pgfqpoint{2.266385in}{2.383098in}}%
\pgfpathlineto{\pgfqpoint{2.267271in}{2.395235in}}%
\pgfpathlineto{\pgfqpoint{2.267050in}{2.346589in}}%
\pgfpathlineto{\pgfqpoint{2.267431in}{2.357817in}}%
\pgfpathlineto{\pgfqpoint{2.267456in}{2.350260in}}%
\pgfpathlineto{\pgfqpoint{2.268108in}{2.393158in}}%
\pgfpathlineto{\pgfqpoint{2.268478in}{2.388037in}}%
\pgfpathlineto{\pgfqpoint{2.269339in}{2.398655in}}%
\pgfpathlineto{\pgfqpoint{2.268687in}{2.349092in}}%
\pgfpathlineto{\pgfqpoint{2.269462in}{2.369959in}}%
\pgfpathlineto{\pgfqpoint{2.269511in}{2.352534in}}%
\pgfpathlineto{\pgfqpoint{2.270164in}{2.396657in}}%
\pgfpathlineto{\pgfqpoint{2.270496in}{2.377888in}}%
\pgfpathlineto{\pgfqpoint{2.271395in}{2.394959in}}%
\pgfpathlineto{\pgfqpoint{2.270742in}{2.353742in}}%
\pgfpathlineto{\pgfqpoint{2.271493in}{2.369437in}}%
\pgfpathlineto{\pgfqpoint{2.272219in}{2.390564in}}%
\pgfpathlineto{\pgfqpoint{2.271567in}{2.356144in}}%
\pgfpathlineto{\pgfqpoint{2.272367in}{2.365890in}}%
\pgfpathlineto{\pgfqpoint{2.272798in}{2.356060in}}%
\pgfpathlineto{\pgfqpoint{2.272638in}{2.390933in}}%
\pgfpathlineto{\pgfqpoint{2.273425in}{2.380951in}}%
\pgfpathlineto{\pgfqpoint{2.273462in}{2.391738in}}%
\pgfpathlineto{\pgfqpoint{2.274028in}{2.358904in}}%
\pgfpathlineto{\pgfqpoint{2.274533in}{2.379480in}}%
\pgfpathlineto{\pgfqpoint{2.274853in}{2.357834in}}%
\pgfpathlineto{\pgfqpoint{2.275518in}{2.389994in}}%
\pgfpathlineto{\pgfqpoint{2.275678in}{2.360645in}}%
\pgfpathlineto{\pgfqpoint{2.276330in}{2.385783in}}%
\pgfpathlineto{\pgfqpoint{2.276884in}{2.369636in}}%
\pgfpathlineto{\pgfqpoint{2.277733in}{2.361782in}}%
\pgfpathlineto{\pgfqpoint{2.277167in}{2.385211in}}%
\pgfpathlineto{\pgfqpoint{2.277967in}{2.380091in}}%
\pgfpathlineto{\pgfqpoint{2.279038in}{2.384655in}}%
\pgfpathlineto{\pgfqpoint{2.278151in}{2.362258in}}%
\pgfpathlineto{\pgfqpoint{2.279062in}{2.379841in}}%
\pgfpathlineto{\pgfqpoint{2.279911in}{2.365735in}}%
\pgfpathlineto{\pgfqpoint{2.279653in}{2.386118in}}%
\pgfpathlineto{\pgfqpoint{2.280194in}{2.367240in}}%
\pgfpathlineto{\pgfqpoint{2.280318in}{2.365278in}}%
\pgfpathlineto{\pgfqpoint{2.280268in}{2.385193in}}%
\pgfpathlineto{\pgfqpoint{2.280662in}{2.380563in}}%
\pgfpathlineto{\pgfqpoint{2.280687in}{2.386209in}}%
\pgfpathlineto{\pgfqpoint{2.281438in}{2.363418in}}%
\pgfpathlineto{\pgfqpoint{2.281745in}{2.376484in}}%
\pgfpathlineto{\pgfqpoint{2.281967in}{2.359031in}}%
\pgfpathlineto{\pgfqpoint{2.282324in}{2.387680in}}%
\pgfpathlineto{\pgfqpoint{2.282841in}{2.380997in}}%
\pgfpathlineto{\pgfqpoint{2.283468in}{2.354809in}}%
\pgfpathlineto{\pgfqpoint{2.283764in}{2.394185in}}%
\pgfpathlineto{\pgfqpoint{2.283776in}{2.402980in}}%
\pgfpathlineto{\pgfqpoint{2.284711in}{2.355561in}}%
\pgfpathlineto{\pgfqpoint{2.284847in}{2.376728in}}%
\pgfpathlineto{\pgfqpoint{2.285758in}{2.410260in}}%
\pgfpathlineto{\pgfqpoint{2.285142in}{2.341320in}}%
\pgfpathlineto{\pgfqpoint{2.285893in}{2.366877in}}%
\pgfpathlineto{\pgfqpoint{2.285942in}{2.340118in}}%
\pgfpathlineto{\pgfqpoint{2.286890in}{2.423929in}}%
\pgfpathlineto{\pgfqpoint{2.286902in}{2.432969in}}%
\pgfpathlineto{\pgfqpoint{2.287185in}{2.308493in}}%
\pgfpathlineto{\pgfqpoint{2.287887in}{2.366223in}}%
\pgfpathlineto{\pgfqpoint{2.288416in}{2.287557in}}%
\pgfpathlineto{\pgfqpoint{2.288785in}{2.461259in}}%
\pgfpathlineto{\pgfqpoint{2.288982in}{2.388311in}}%
\pgfpathlineto{\pgfqpoint{2.289647in}{2.254025in}}%
\pgfpathlineto{\pgfqpoint{2.289290in}{2.471222in}}%
\pgfpathlineto{\pgfqpoint{2.290041in}{2.421361in}}%
\pgfpathlineto{\pgfqpoint{2.290422in}{2.452760in}}%
\pgfpathlineto{\pgfqpoint{2.290102in}{2.298808in}}%
\pgfpathlineto{\pgfqpoint{2.290631in}{2.344657in}}%
\pgfpathlineto{\pgfqpoint{2.291702in}{2.187950in}}%
\pgfpathlineto{\pgfqpoint{2.291333in}{2.484928in}}%
\pgfpathlineto{\pgfqpoint{2.291739in}{2.310141in}}%
\pgfpathlineto{\pgfqpoint{2.292342in}{2.455858in}}%
\pgfpathlineto{\pgfqpoint{2.292699in}{2.276156in}}%
\pgfpathlineto{\pgfqpoint{2.292871in}{2.399783in}}%
\pgfpathlineto{\pgfqpoint{2.293659in}{2.290117in}}%
\pgfpathlineto{\pgfqpoint{2.292994in}{2.456309in}}%
\pgfpathlineto{\pgfqpoint{2.293942in}{2.408136in}}%
\pgfpathlineto{\pgfqpoint{2.294410in}{2.510486in}}%
\pgfpathlineto{\pgfqpoint{2.294693in}{2.221505in}}%
\pgfpathlineto{\pgfqpoint{2.294804in}{2.278308in}}%
\pgfpathlineto{\pgfqpoint{2.294890in}{2.363528in}}%
\pgfpathlineto{\pgfqpoint{2.295887in}{2.466455in}}%
\pgfpathlineto{\pgfqpoint{2.295714in}{2.284144in}}%
\pgfpathlineto{\pgfqpoint{2.295998in}{2.376532in}}%
\pgfpathlineto{\pgfqpoint{2.296736in}{2.184004in}}%
\pgfpathlineto{\pgfqpoint{2.296330in}{2.482310in}}%
\pgfpathlineto{\pgfqpoint{2.297142in}{2.331913in}}%
\pgfpathlineto{\pgfqpoint{2.297314in}{2.512804in}}%
\pgfpathlineto{\pgfqpoint{2.297708in}{2.174585in}}%
\pgfpathlineto{\pgfqpoint{2.298336in}{2.465608in}}%
\pgfpathlineto{\pgfqpoint{2.298631in}{2.233283in}}%
\pgfpathlineto{\pgfqpoint{2.299284in}{2.576535in}}%
\pgfpathlineto{\pgfqpoint{2.299345in}{2.557106in}}%
\pgfpathlineto{\pgfqpoint{2.299370in}{2.582566in}}%
\pgfpathlineto{\pgfqpoint{2.299641in}{2.115967in}}%
\pgfpathlineto{\pgfqpoint{2.300330in}{2.387723in}}%
\pgfpathlineto{\pgfqpoint{2.300625in}{2.180912in}}%
\pgfpathlineto{\pgfqpoint{2.301277in}{2.574567in}}%
\pgfpathlineto{\pgfqpoint{2.301339in}{2.516196in}}%
\pgfpathlineto{\pgfqpoint{2.301376in}{2.585526in}}%
\pgfpathlineto{\pgfqpoint{2.301721in}{2.155256in}}%
\pgfpathlineto{\pgfqpoint{2.302397in}{2.417027in}}%
\pgfpathlineto{\pgfqpoint{2.302619in}{2.089184in}}%
\pgfpathlineto{\pgfqpoint{2.303357in}{2.516857in}}%
\pgfpathlineto{\pgfqpoint{2.303616in}{2.191342in}}%
\pgfpathlineto{\pgfqpoint{2.304231in}{2.654563in}}%
\pgfpathlineto{\pgfqpoint{2.304613in}{2.069778in}}%
\pgfpathlineto{\pgfqpoint{2.304785in}{2.324214in}}%
\pgfpathlineto{\pgfqpoint{2.305499in}{2.136236in}}%
\pgfpathlineto{\pgfqpoint{2.305745in}{2.513133in}}%
\pgfpathlineto{\pgfqpoint{2.305905in}{2.294031in}}%
\pgfpathlineto{\pgfqpoint{2.306016in}{2.223311in}}%
\pgfpathlineto{\pgfqpoint{2.306127in}{2.444679in}}%
\pgfpathlineto{\pgfqpoint{2.306311in}{2.680835in}}%
\pgfpathlineto{\pgfqpoint{2.306545in}{2.083555in}}%
\pgfpathlineto{\pgfqpoint{2.307234in}{2.497793in}}%
\pgfpathlineto{\pgfqpoint{2.307493in}{2.049327in}}%
\pgfpathlineto{\pgfqpoint{2.308231in}{2.562635in}}%
\pgfpathlineto{\pgfqpoint{2.308305in}{2.537029in}}%
\pgfpathlineto{\pgfqpoint{2.309117in}{2.633149in}}%
\pgfpathlineto{\pgfqpoint{2.308502in}{2.173053in}}%
\pgfpathlineto{\pgfqpoint{2.309351in}{2.385029in}}%
\pgfpathlineto{\pgfqpoint{2.309499in}{1.991281in}}%
\pgfpathlineto{\pgfqpoint{2.310139in}{2.525996in}}%
\pgfpathlineto{\pgfqpoint{2.310484in}{2.240665in}}%
\pgfpathlineto{\pgfqpoint{2.311161in}{2.676161in}}%
\pgfpathlineto{\pgfqpoint{2.311444in}{2.065599in}}%
\pgfpathlineto{\pgfqpoint{2.311456in}{2.039816in}}%
\pgfpathlineto{\pgfqpoint{2.312022in}{2.561724in}}%
\pgfpathlineto{\pgfqpoint{2.312465in}{2.153659in}}%
\pgfpathlineto{\pgfqpoint{2.313228in}{2.659207in}}%
\pgfpathlineto{\pgfqpoint{2.313487in}{2.146842in}}%
\pgfpathlineto{\pgfqpoint{2.313597in}{2.311939in}}%
\pgfpathlineto{\pgfqpoint{2.314028in}{2.621968in}}%
\pgfpathlineto{\pgfqpoint{2.314410in}{1.987861in}}%
\pgfpathlineto{\pgfqpoint{2.314742in}{2.406590in}}%
\pgfpathlineto{\pgfqpoint{2.315788in}{2.155188in}}%
\pgfpathlineto{\pgfqpoint{2.315173in}{2.499033in}}%
\pgfpathlineto{\pgfqpoint{2.315874in}{2.346484in}}%
\pgfpathlineto{\pgfqpoint{2.316022in}{2.667586in}}%
\pgfpathlineto{\pgfqpoint{2.316404in}{2.055554in}}%
\pgfpathlineto{\pgfqpoint{2.317019in}{2.424833in}}%
\pgfpathlineto{\pgfqpoint{2.317314in}{2.135451in}}%
\pgfpathlineto{\pgfqpoint{2.317967in}{2.530763in}}%
\pgfpathlineto{\pgfqpoint{2.318102in}{2.675223in}}%
\pgfpathlineto{\pgfqpoint{2.318287in}{2.273531in}}%
\pgfpathlineto{\pgfqpoint{2.319271in}{2.049895in}}%
\pgfpathlineto{\pgfqpoint{2.318988in}{2.572525in}}%
\pgfpathlineto{\pgfqpoint{2.319394in}{2.270621in}}%
\pgfpathlineto{\pgfqpoint{2.320120in}{2.565566in}}%
\pgfpathlineto{\pgfqpoint{2.319788in}{2.193155in}}%
\pgfpathlineto{\pgfqpoint{2.320514in}{2.341465in}}%
\pgfpathlineto{\pgfqpoint{2.320945in}{2.625954in}}%
\pgfpathlineto{\pgfqpoint{2.321314in}{2.052413in}}%
\pgfpathlineto{\pgfqpoint{2.321622in}{2.406220in}}%
\pgfpathlineto{\pgfqpoint{2.322656in}{2.104692in}}%
\pgfpathlineto{\pgfqpoint{2.321905in}{2.485829in}}%
\pgfpathlineto{\pgfqpoint{2.322767in}{2.256583in}}%
\pgfpathlineto{\pgfqpoint{2.322939in}{2.682645in}}%
\pgfpathlineto{\pgfqpoint{2.323259in}{2.082734in}}%
\pgfpathlineto{\pgfqpoint{2.323887in}{2.358302in}}%
\pgfpathlineto{\pgfqpoint{2.324970in}{2.608073in}}%
\pgfpathlineto{\pgfqpoint{2.324724in}{2.131309in}}%
\pgfpathlineto{\pgfqpoint{2.325080in}{2.474325in}}%
\pgfpathlineto{\pgfqpoint{2.325339in}{2.103629in}}%
\pgfpathlineto{\pgfqpoint{2.325819in}{2.531523in}}%
\pgfpathlineto{\pgfqpoint{2.326225in}{2.286620in}}%
\pgfpathlineto{\pgfqpoint{2.327136in}{2.494513in}}%
\pgfpathlineto{\pgfqpoint{2.326570in}{2.215803in}}%
\pgfpathlineto{\pgfqpoint{2.327173in}{2.246238in}}%
\pgfpathlineto{\pgfqpoint{2.327579in}{2.097766in}}%
\pgfpathlineto{\pgfqpoint{2.327924in}{2.662814in}}%
\pgfpathlineto{\pgfqpoint{2.328305in}{2.136784in}}%
\pgfpathlineto{\pgfqpoint{2.329204in}{2.560068in}}%
\pgfpathlineto{\pgfqpoint{2.329450in}{2.290783in}}%
\pgfpathlineto{\pgfqpoint{2.329647in}{2.076948in}}%
\pgfpathlineto{\pgfqpoint{2.329782in}{2.545500in}}%
\pgfpathlineto{\pgfqpoint{2.329856in}{2.655123in}}%
\pgfpathlineto{\pgfqpoint{2.330274in}{2.063363in}}%
\pgfpathlineto{\pgfqpoint{2.330767in}{2.334484in}}%
\pgfpathlineto{\pgfqpoint{2.331493in}{2.106858in}}%
\pgfpathlineto{\pgfqpoint{2.330828in}{2.504376in}}%
\pgfpathlineto{\pgfqpoint{2.331800in}{2.441949in}}%
\pgfpathlineto{\pgfqpoint{2.332674in}{2.605702in}}%
\pgfpathlineto{\pgfqpoint{2.332416in}{2.181219in}}%
\pgfpathlineto{\pgfqpoint{2.332920in}{2.474554in}}%
\pgfpathlineto{\pgfqpoint{2.333228in}{2.145212in}}%
\pgfpathlineto{\pgfqpoint{2.333905in}{2.517098in}}%
\pgfpathlineto{\pgfqpoint{2.333991in}{2.516057in}}%
\pgfpathlineto{\pgfqpoint{2.334767in}{2.685514in}}%
\pgfpathlineto{\pgfqpoint{2.334471in}{2.053690in}}%
\pgfpathlineto{\pgfqpoint{2.335037in}{2.370339in}}%
\pgfpathlineto{\pgfqpoint{2.335185in}{2.158821in}}%
\pgfpathlineto{\pgfqpoint{2.336096in}{2.549632in}}%
\pgfpathlineto{\pgfqpoint{2.336157in}{2.246399in}}%
\pgfpathlineto{\pgfqpoint{2.336785in}{2.518962in}}%
\pgfpathlineto{\pgfqpoint{2.336404in}{2.119017in}}%
\pgfpathlineto{\pgfqpoint{2.337265in}{2.274246in}}%
\pgfpathlineto{\pgfqpoint{2.337425in}{2.173232in}}%
\pgfpathlineto{\pgfqpoint{2.337671in}{2.526490in}}%
\pgfpathlineto{\pgfqpoint{2.338299in}{2.351927in}}%
\pgfpathlineto{\pgfqpoint{2.339013in}{2.595283in}}%
\pgfpathlineto{\pgfqpoint{2.339271in}{2.147674in}}%
\pgfpathlineto{\pgfqpoint{2.339357in}{2.188210in}}%
\pgfpathlineto{\pgfqpoint{2.339382in}{2.111364in}}%
\pgfpathlineto{\pgfqpoint{2.339616in}{2.602975in}}%
\pgfpathlineto{\pgfqpoint{2.340428in}{2.330283in}}%
\pgfpathlineto{\pgfqpoint{2.340871in}{2.591815in}}%
\pgfpathlineto{\pgfqpoint{2.341339in}{2.100064in}}%
\pgfpathlineto{\pgfqpoint{2.341585in}{2.523356in}}%
\pgfpathlineto{\pgfqpoint{2.341610in}{2.540970in}}%
\pgfpathlineto{\pgfqpoint{2.341967in}{2.240395in}}%
\pgfpathlineto{\pgfqpoint{2.342557in}{2.388453in}}%
\pgfpathlineto{\pgfqpoint{2.343210in}{2.156626in}}%
\pgfpathlineto{\pgfqpoint{2.342730in}{2.521081in}}%
\pgfpathlineto{\pgfqpoint{2.343653in}{2.412137in}}%
\pgfpathlineto{\pgfqpoint{2.344613in}{2.566563in}}%
\pgfpathlineto{\pgfqpoint{2.344330in}{2.183821in}}%
\pgfpathlineto{\pgfqpoint{2.344711in}{2.426841in}}%
\pgfpathlineto{\pgfqpoint{2.345056in}{2.191422in}}%
\pgfpathlineto{\pgfqpoint{2.345757in}{2.514815in}}%
\pgfpathlineto{\pgfqpoint{2.345807in}{2.431680in}}%
\pgfpathlineto{\pgfqpoint{2.345856in}{2.579149in}}%
\pgfpathlineto{\pgfqpoint{2.346299in}{2.169716in}}%
\pgfpathlineto{\pgfqpoint{2.346890in}{2.321107in}}%
\pgfpathlineto{\pgfqpoint{2.346914in}{2.222994in}}%
\pgfpathlineto{\pgfqpoint{2.347603in}{2.528140in}}%
\pgfpathlineto{\pgfqpoint{2.347973in}{2.399947in}}%
\pgfpathlineto{\pgfqpoint{2.348834in}{2.569688in}}%
\pgfpathlineto{\pgfqpoint{2.348157in}{2.189164in}}%
\pgfpathlineto{\pgfqpoint{2.349031in}{2.346885in}}%
\pgfpathlineto{\pgfqpoint{2.349117in}{2.239270in}}%
\pgfpathlineto{\pgfqpoint{2.349573in}{2.529207in}}%
\pgfpathlineto{\pgfqpoint{2.350139in}{2.277674in}}%
\pgfpathlineto{\pgfqpoint{2.350459in}{2.522800in}}%
\pgfpathlineto{\pgfqpoint{2.351160in}{2.199700in}}%
\pgfpathlineto{\pgfqpoint{2.351234in}{2.293152in}}%
\pgfpathlineto{\pgfqpoint{2.352194in}{2.190267in}}%
\pgfpathlineto{\pgfqpoint{2.351308in}{2.558095in}}%
\pgfpathlineto{\pgfqpoint{2.352330in}{2.343980in}}%
\pgfpathlineto{\pgfqpoint{2.352551in}{2.609112in}}%
\pgfpathlineto{\pgfqpoint{2.353117in}{2.149604in}}%
\pgfpathlineto{\pgfqpoint{2.353425in}{2.330492in}}%
\pgfpathlineto{\pgfqpoint{2.353782in}{2.590682in}}%
\pgfpathlineto{\pgfqpoint{2.354139in}{2.239006in}}%
\pgfpathlineto{\pgfqpoint{2.354237in}{2.254186in}}%
\pgfpathlineto{\pgfqpoint{2.354963in}{2.162900in}}%
\pgfpathlineto{\pgfqpoint{2.354607in}{2.538019in}}%
\pgfpathlineto{\pgfqpoint{2.355308in}{2.394347in}}%
\pgfpathlineto{\pgfqpoint{2.356367in}{2.566602in}}%
\pgfpathlineto{\pgfqpoint{2.356108in}{2.175083in}}%
\pgfpathlineto{\pgfqpoint{2.356403in}{2.411045in}}%
\pgfpathlineto{\pgfqpoint{2.356834in}{2.155118in}}%
\pgfpathlineto{\pgfqpoint{2.356662in}{2.575096in}}%
\pgfpathlineto{\pgfqpoint{2.357474in}{2.479515in}}%
\pgfpathlineto{\pgfqpoint{2.357597in}{2.577328in}}%
\pgfpathlineto{\pgfqpoint{2.358065in}{2.123606in}}%
\pgfpathlineto{\pgfqpoint{2.358557in}{2.334185in}}%
\pgfpathlineto{\pgfqpoint{2.359296in}{2.215728in}}%
\pgfpathlineto{\pgfqpoint{2.358730in}{2.555940in}}%
\pgfpathlineto{\pgfqpoint{2.359640in}{2.471642in}}%
\pgfpathlineto{\pgfqpoint{2.359911in}{2.171101in}}%
\pgfpathlineto{\pgfqpoint{2.360366in}{2.534705in}}%
\pgfpathlineto{\pgfqpoint{2.360736in}{2.425461in}}%
\pgfpathlineto{\pgfqpoint{2.361597in}{2.574607in}}%
\pgfpathlineto{\pgfqpoint{2.361770in}{2.141593in}}%
\pgfpathlineto{\pgfqpoint{2.361831in}{2.408277in}}%
\pgfpathlineto{\pgfqpoint{2.362890in}{2.216549in}}%
\pgfpathlineto{\pgfqpoint{2.362533in}{2.567759in}}%
\pgfpathlineto{\pgfqpoint{2.362926in}{2.408547in}}%
\pgfpathlineto{\pgfqpoint{2.363665in}{2.566870in}}%
\pgfpathlineto{\pgfqpoint{2.363000in}{2.144591in}}%
\pgfpathlineto{\pgfqpoint{2.363997in}{2.270377in}}%
\pgfpathlineto{\pgfqpoint{2.364846in}{2.205137in}}%
\pgfpathlineto{\pgfqpoint{2.364477in}{2.551553in}}%
\pgfpathlineto{\pgfqpoint{2.365080in}{2.327858in}}%
\pgfpathlineto{\pgfqpoint{2.365708in}{2.553035in}}%
\pgfpathlineto{\pgfqpoint{2.365474in}{2.219997in}}%
\pgfpathlineto{\pgfqpoint{2.365954in}{2.296336in}}%
\pgfpathlineto{\pgfqpoint{2.365979in}{2.181633in}}%
\pgfpathlineto{\pgfqpoint{2.366237in}{2.546289in}}%
\pgfpathlineto{\pgfqpoint{2.367037in}{2.413564in}}%
\pgfpathlineto{\pgfqpoint{2.367370in}{2.556555in}}%
\pgfpathlineto{\pgfqpoint{2.367210in}{2.214638in}}%
\pgfpathlineto{\pgfqpoint{2.367911in}{2.277566in}}%
\pgfpathlineto{\pgfqpoint{2.367936in}{2.148229in}}%
\pgfpathlineto{\pgfqpoint{2.368182in}{2.555786in}}%
\pgfpathlineto{\pgfqpoint{2.368982in}{2.401019in}}%
\pgfpathlineto{\pgfqpoint{2.369413in}{2.582828in}}%
\pgfpathlineto{\pgfqpoint{2.369166in}{2.196922in}}%
\pgfpathlineto{\pgfqpoint{2.370065in}{2.358679in}}%
\pgfpathlineto{\pgfqpoint{2.370914in}{2.162486in}}%
\pgfpathlineto{\pgfqpoint{2.370656in}{2.543702in}}%
\pgfpathlineto{\pgfqpoint{2.371148in}{2.498589in}}%
\pgfpathlineto{\pgfqpoint{2.371173in}{2.578857in}}%
\pgfpathlineto{\pgfqpoint{2.371640in}{2.145066in}}%
\pgfpathlineto{\pgfqpoint{2.372219in}{2.389976in}}%
\pgfpathlineto{\pgfqpoint{2.372871in}{2.111740in}}%
\pgfpathlineto{\pgfqpoint{2.372416in}{2.590294in}}%
\pgfpathlineto{\pgfqpoint{2.373326in}{2.403297in}}%
\pgfpathlineto{\pgfqpoint{2.374348in}{2.551713in}}%
\pgfpathlineto{\pgfqpoint{2.374114in}{2.152028in}}%
\pgfpathlineto{\pgfqpoint{2.374459in}{2.462449in}}%
\pgfpathlineto{\pgfqpoint{2.375345in}{2.215465in}}%
\pgfpathlineto{\pgfqpoint{2.375173in}{2.519916in}}%
\pgfpathlineto{\pgfqpoint{2.375554in}{2.478291in}}%
\pgfpathlineto{\pgfqpoint{2.376120in}{2.551591in}}%
\pgfpathlineto{\pgfqpoint{2.375948in}{2.229937in}}%
\pgfpathlineto{\pgfqpoint{2.376465in}{2.283026in}}%
\pgfpathlineto{\pgfqpoint{2.376477in}{2.282505in}}%
\pgfpathlineto{\pgfqpoint{2.377351in}{2.578878in}}%
\pgfpathlineto{\pgfqpoint{2.376588in}{2.173562in}}%
\pgfpathlineto{\pgfqpoint{2.377585in}{2.304076in}}%
\pgfpathlineto{\pgfqpoint{2.377646in}{2.468597in}}%
\pgfpathlineto{\pgfqpoint{2.377806in}{2.157967in}}%
\pgfpathlineto{\pgfqpoint{2.377819in}{2.127144in}}%
\pgfpathlineto{\pgfqpoint{2.378065in}{2.534861in}}%
\pgfpathlineto{\pgfqpoint{2.378840in}{2.334551in}}%
\pgfpathlineto{\pgfqpoint{2.379296in}{2.558338in}}%
\pgfpathlineto{\pgfqpoint{2.379050in}{2.135449in}}%
\pgfpathlineto{\pgfqpoint{2.379948in}{2.363875in}}%
\pgfpathlineto{\pgfqpoint{2.380293in}{2.169665in}}%
\pgfpathlineto{\pgfqpoint{2.380120in}{2.556131in}}%
\pgfpathlineto{\pgfqpoint{2.381031in}{2.472798in}}%
\pgfpathlineto{\pgfqpoint{2.381363in}{2.534704in}}%
\pgfpathlineto{\pgfqpoint{2.381536in}{2.185358in}}%
\pgfpathlineto{\pgfqpoint{2.381893in}{2.314864in}}%
\pgfpathlineto{\pgfqpoint{2.382766in}{2.127332in}}%
\pgfpathlineto{\pgfqpoint{2.382274in}{2.533511in}}%
\pgfpathlineto{\pgfqpoint{2.382939in}{2.483718in}}%
\pgfpathlineto{\pgfqpoint{2.382963in}{2.519213in}}%
\pgfpathlineto{\pgfqpoint{2.383271in}{2.244079in}}%
\pgfpathlineto{\pgfqpoint{2.383948in}{2.367075in}}%
\pgfpathlineto{\pgfqpoint{2.383997in}{2.139926in}}%
\pgfpathlineto{\pgfqpoint{2.384194in}{2.534504in}}%
\pgfpathlineto{\pgfqpoint{2.385031in}{2.432848in}}%
\pgfpathlineto{\pgfqpoint{2.385068in}{2.524844in}}%
\pgfpathlineto{\pgfqpoint{2.385240in}{2.143439in}}%
\pgfpathlineto{\pgfqpoint{2.386114in}{2.406367in}}%
\pgfpathlineto{\pgfqpoint{2.386471in}{2.169901in}}%
\pgfpathlineto{\pgfqpoint{2.386299in}{2.525586in}}%
\pgfpathlineto{\pgfqpoint{2.387197in}{2.479239in}}%
\pgfpathlineto{\pgfqpoint{2.387222in}{2.508534in}}%
\pgfpathlineto{\pgfqpoint{2.387714in}{2.141453in}}%
\pgfpathlineto{\pgfqpoint{2.388182in}{2.365365in}}%
\pgfpathlineto{\pgfqpoint{2.388945in}{2.171091in}}%
\pgfpathlineto{\pgfqpoint{2.389166in}{2.545820in}}%
\pgfpathlineto{\pgfqpoint{2.389265in}{2.476494in}}%
\pgfpathlineto{\pgfqpoint{2.389277in}{2.476445in}}%
\pgfpathlineto{\pgfqpoint{2.390188in}{2.171851in}}%
\pgfpathlineto{\pgfqpoint{2.390336in}{2.517076in}}%
\pgfpathlineto{\pgfqpoint{2.390360in}{2.492894in}}%
\pgfpathlineto{\pgfqpoint{2.390409in}{2.543510in}}%
\pgfpathlineto{\pgfqpoint{2.390594in}{2.238850in}}%
\pgfpathlineto{\pgfqpoint{2.391382in}{2.335344in}}%
\pgfpathlineto{\pgfqpoint{2.391419in}{2.172455in}}%
\pgfpathlineto{\pgfqpoint{2.392169in}{2.514641in}}%
\pgfpathlineto{\pgfqpoint{2.392465in}{2.441379in}}%
\pgfpathlineto{\pgfqpoint{2.392871in}{2.504331in}}%
\pgfpathlineto{\pgfqpoint{2.392662in}{2.177949in}}%
\pgfpathlineto{\pgfqpoint{2.393536in}{2.344753in}}%
\pgfpathlineto{\pgfqpoint{2.393893in}{2.210441in}}%
\pgfpathlineto{\pgfqpoint{2.394114in}{2.523466in}}%
\pgfpathlineto{\pgfqpoint{2.394606in}{2.439448in}}%
\pgfpathlineto{\pgfqpoint{2.395345in}{2.534175in}}%
\pgfpathlineto{\pgfqpoint{2.395136in}{2.232773in}}%
\pgfpathlineto{\pgfqpoint{2.395677in}{2.331685in}}%
\pgfpathlineto{\pgfqpoint{2.395714in}{2.293334in}}%
\pgfpathlineto{\pgfqpoint{2.395825in}{2.373196in}}%
\pgfpathlineto{\pgfqpoint{2.396576in}{2.529216in}}%
\pgfpathlineto{\pgfqpoint{2.396366in}{2.202981in}}%
\pgfpathlineto{\pgfqpoint{2.396920in}{2.335734in}}%
\pgfpathlineto{\pgfqpoint{2.397609in}{2.215886in}}%
\pgfpathlineto{\pgfqpoint{2.397105in}{2.503737in}}%
\pgfpathlineto{\pgfqpoint{2.397794in}{2.491375in}}%
\pgfpathlineto{\pgfqpoint{2.397806in}{2.516996in}}%
\pgfpathlineto{\pgfqpoint{2.398016in}{2.277342in}}%
\pgfpathlineto{\pgfqpoint{2.398803in}{2.287320in}}%
\pgfpathlineto{\pgfqpoint{2.398840in}{2.242799in}}%
\pgfpathlineto{\pgfqpoint{2.399049in}{2.556832in}}%
\pgfpathlineto{\pgfqpoint{2.399837in}{2.357128in}}%
\pgfpathlineto{\pgfqpoint{2.400280in}{2.550271in}}%
\pgfpathlineto{\pgfqpoint{2.400489in}{2.252686in}}%
\pgfpathlineto{\pgfqpoint{2.400945in}{2.415405in}}%
\pgfpathlineto{\pgfqpoint{2.401314in}{2.224403in}}%
\pgfpathlineto{\pgfqpoint{2.401523in}{2.563481in}}%
\pgfpathlineto{\pgfqpoint{2.402028in}{2.479132in}}%
\pgfpathlineto{\pgfqpoint{2.402766in}{2.550542in}}%
\pgfpathlineto{\pgfqpoint{2.402557in}{2.239045in}}%
\pgfpathlineto{\pgfqpoint{2.403037in}{2.329232in}}%
\pgfpathlineto{\pgfqpoint{2.403123in}{2.297603in}}%
\pgfpathlineto{\pgfqpoint{2.403259in}{2.432154in}}%
\pgfpathlineto{\pgfqpoint{2.403997in}{2.564251in}}%
\pgfpathlineto{\pgfqpoint{2.403788in}{2.254601in}}%
\pgfpathlineto{\pgfqpoint{2.404317in}{2.331125in}}%
\pgfpathlineto{\pgfqpoint{2.405240in}{2.534088in}}%
\pgfpathlineto{\pgfqpoint{2.405437in}{2.261729in}}%
\pgfpathlineto{\pgfqpoint{2.406471in}{2.561672in}}%
\pgfpathlineto{\pgfqpoint{2.406286in}{2.256395in}}%
\pgfpathlineto{\pgfqpoint{2.406631in}{2.342099in}}%
\pgfpathlineto{\pgfqpoint{2.407554in}{2.246716in}}%
\pgfpathlineto{\pgfqpoint{2.406976in}{2.488113in}}%
\pgfpathlineto{\pgfqpoint{2.407677in}{2.475133in}}%
\pgfpathlineto{\pgfqpoint{2.407714in}{2.555867in}}%
\pgfpathlineto{\pgfqpoint{2.408157in}{2.278450in}}%
\pgfpathlineto{\pgfqpoint{2.408723in}{2.310242in}}%
\pgfpathlineto{\pgfqpoint{2.408785in}{2.262585in}}%
\pgfpathlineto{\pgfqpoint{2.408945in}{2.538844in}}%
\pgfpathlineto{\pgfqpoint{2.409437in}{2.433423in}}%
\pgfpathlineto{\pgfqpoint{2.410188in}{2.536203in}}%
\pgfpathlineto{\pgfqpoint{2.410028in}{2.249307in}}%
\pgfpathlineto{\pgfqpoint{2.410508in}{2.310477in}}%
\pgfpathlineto{\pgfqpoint{2.411259in}{2.258758in}}%
\pgfpathlineto{\pgfqpoint{2.411419in}{2.542063in}}%
\pgfpathlineto{\pgfqpoint{2.411616in}{2.298480in}}%
\pgfpathlineto{\pgfqpoint{2.412502in}{2.229993in}}%
\pgfpathlineto{\pgfqpoint{2.411936in}{2.477685in}}%
\pgfpathlineto{\pgfqpoint{2.412526in}{2.356036in}}%
\pgfpathlineto{\pgfqpoint{2.412662in}{2.547882in}}%
\pgfpathlineto{\pgfqpoint{2.413006in}{2.256819in}}%
\pgfpathlineto{\pgfqpoint{2.413622in}{2.286003in}}%
\pgfpathlineto{\pgfqpoint{2.413732in}{2.239342in}}%
\pgfpathlineto{\pgfqpoint{2.413892in}{2.528366in}}%
\pgfpathlineto{\pgfqpoint{2.414286in}{2.445237in}}%
\pgfpathlineto{\pgfqpoint{2.415136in}{2.521417in}}%
\pgfpathlineto{\pgfqpoint{2.414976in}{2.220693in}}%
\pgfpathlineto{\pgfqpoint{2.415345in}{2.321942in}}%
\pgfpathlineto{\pgfqpoint{2.416366in}{2.535235in}}%
\pgfpathlineto{\pgfqpoint{2.416206in}{2.259316in}}%
\pgfpathlineto{\pgfqpoint{2.416514in}{2.348295in}}%
\pgfpathlineto{\pgfqpoint{2.417449in}{2.228412in}}%
\pgfpathlineto{\pgfqpoint{2.416896in}{2.484303in}}%
\pgfpathlineto{\pgfqpoint{2.417486in}{2.406527in}}%
\pgfpathlineto{\pgfqpoint{2.417609in}{2.540066in}}%
\pgfpathlineto{\pgfqpoint{2.417966in}{2.262769in}}%
\pgfpathlineto{\pgfqpoint{2.418569in}{2.305929in}}%
\pgfpathlineto{\pgfqpoint{2.418692in}{2.230362in}}%
\pgfpathlineto{\pgfqpoint{2.418840in}{2.507899in}}%
\pgfpathlineto{\pgfqpoint{2.419628in}{2.400248in}}%
\pgfpathlineto{\pgfqpoint{2.420083in}{2.526157in}}%
\pgfpathlineto{\pgfqpoint{2.419923in}{2.201609in}}%
\pgfpathlineto{\pgfqpoint{2.420723in}{2.413017in}}%
\pgfpathlineto{\pgfqpoint{2.421166in}{2.224773in}}%
\pgfpathlineto{\pgfqpoint{2.421326in}{2.514418in}}%
\pgfpathlineto{\pgfqpoint{2.421819in}{2.413523in}}%
\pgfpathlineto{\pgfqpoint{2.422557in}{2.523391in}}%
\pgfpathlineto{\pgfqpoint{2.422397in}{2.234256in}}%
\pgfpathlineto{\pgfqpoint{2.422889in}{2.317387in}}%
\pgfpathlineto{\pgfqpoint{2.423640in}{2.214287in}}%
\pgfpathlineto{\pgfqpoint{2.423812in}{2.516518in}}%
\pgfpathlineto{\pgfqpoint{2.423985in}{2.345295in}}%
\pgfpathlineto{\pgfqpoint{2.424317in}{2.483414in}}%
\pgfpathlineto{\pgfqpoint{2.424046in}{2.262386in}}%
\pgfpathlineto{\pgfqpoint{2.424748in}{2.308850in}}%
\pgfpathlineto{\pgfqpoint{2.424883in}{2.217667in}}%
\pgfpathlineto{\pgfqpoint{2.425043in}{2.529079in}}%
\pgfpathlineto{\pgfqpoint{2.425819in}{2.410296in}}%
\pgfpathlineto{\pgfqpoint{2.426286in}{2.517330in}}%
\pgfpathlineto{\pgfqpoint{2.426114in}{2.228051in}}%
\pgfpathlineto{\pgfqpoint{2.426914in}{2.393637in}}%
\pgfpathlineto{\pgfqpoint{2.427357in}{2.245418in}}%
\pgfpathlineto{\pgfqpoint{2.427517in}{2.514673in}}%
\pgfpathlineto{\pgfqpoint{2.427997in}{2.381272in}}%
\pgfpathlineto{\pgfqpoint{2.428760in}{2.506741in}}%
\pgfpathlineto{\pgfqpoint{2.428588in}{2.230697in}}%
\pgfpathlineto{\pgfqpoint{2.429080in}{2.317172in}}%
\pgfpathlineto{\pgfqpoint{2.429732in}{2.241497in}}%
\pgfpathlineto{\pgfqpoint{2.429991in}{2.523369in}}%
\pgfpathlineto{\pgfqpoint{2.430163in}{2.343358in}}%
\pgfpathlineto{\pgfqpoint{2.430175in}{2.343581in}}%
\pgfpathlineto{\pgfqpoint{2.430200in}{2.332863in}}%
\pgfpathlineto{\pgfqpoint{2.431062in}{2.248191in}}%
\pgfpathlineto{\pgfqpoint{2.431209in}{2.481646in}}%
\pgfpathlineto{\pgfqpoint{2.431234in}{2.512119in}}%
\pgfpathlineto{\pgfqpoint{2.431689in}{2.266454in}}%
\pgfpathlineto{\pgfqpoint{2.432169in}{2.352439in}}%
\pgfpathlineto{\pgfqpoint{2.432305in}{2.252238in}}%
\pgfpathlineto{\pgfqpoint{2.432465in}{2.491193in}}%
\pgfpathlineto{\pgfqpoint{2.433252in}{2.435249in}}%
\pgfpathlineto{\pgfqpoint{2.433708in}{2.480359in}}%
\pgfpathlineto{\pgfqpoint{2.433535in}{2.262173in}}%
\pgfpathlineto{\pgfqpoint{2.434299in}{2.398293in}}%
\pgfpathlineto{\pgfqpoint{2.435185in}{2.262564in}}%
\pgfpathlineto{\pgfqpoint{2.434926in}{2.485196in}}%
\pgfpathlineto{\pgfqpoint{2.435332in}{2.435184in}}%
\pgfpathlineto{\pgfqpoint{2.436182in}{2.491281in}}%
\pgfpathlineto{\pgfqpoint{2.436022in}{2.276866in}}%
\pgfpathlineto{\pgfqpoint{2.436379in}{2.342417in}}%
\pgfpathlineto{\pgfqpoint{2.436428in}{2.270106in}}%
\pgfpathlineto{\pgfqpoint{2.437388in}{2.457906in}}%
\pgfpathlineto{\pgfqpoint{2.437425in}{2.470049in}}%
\pgfpathlineto{\pgfqpoint{2.437880in}{2.287032in}}%
\pgfpathlineto{\pgfqpoint{2.438262in}{2.401184in}}%
\pgfpathlineto{\pgfqpoint{2.438495in}{2.276740in}}%
\pgfpathlineto{\pgfqpoint{2.439062in}{2.474597in}}%
\pgfpathlineto{\pgfqpoint{2.439369in}{2.379865in}}%
\pgfpathlineto{\pgfqpoint{2.440145in}{2.276704in}}%
\pgfpathlineto{\pgfqpoint{2.439899in}{2.473163in}}%
\pgfpathlineto{\pgfqpoint{2.440292in}{2.448309in}}%
\pgfpathlineto{\pgfqpoint{2.441129in}{2.495651in}}%
\pgfpathlineto{\pgfqpoint{2.440772in}{2.293987in}}%
\pgfpathlineto{\pgfqpoint{2.441326in}{2.334310in}}%
\pgfpathlineto{\pgfqpoint{2.441388in}{2.272077in}}%
\pgfpathlineto{\pgfqpoint{2.441548in}{2.462089in}}%
\pgfpathlineto{\pgfqpoint{2.442323in}{2.421196in}}%
\pgfpathlineto{\pgfqpoint{2.442385in}{2.479569in}}%
\pgfpathlineto{\pgfqpoint{2.442619in}{2.287995in}}%
\pgfpathlineto{\pgfqpoint{2.443394in}{2.329680in}}%
\pgfpathlineto{\pgfqpoint{2.443615in}{2.475642in}}%
\pgfpathlineto{\pgfqpoint{2.443862in}{2.288505in}}%
\pgfpathlineto{\pgfqpoint{2.444477in}{2.324249in}}%
\pgfpathlineto{\pgfqpoint{2.445105in}{2.286600in}}%
\pgfpathlineto{\pgfqpoint{2.445265in}{2.465814in}}%
\pgfpathlineto{\pgfqpoint{2.445548in}{2.370626in}}%
\pgfpathlineto{\pgfqpoint{2.446077in}{2.478999in}}%
\pgfpathlineto{\pgfqpoint{2.446335in}{2.273787in}}%
\pgfpathlineto{\pgfqpoint{2.446680in}{2.393378in}}%
\pgfpathlineto{\pgfqpoint{2.447578in}{2.277348in}}%
\pgfpathlineto{\pgfqpoint{2.447320in}{2.485351in}}%
\pgfpathlineto{\pgfqpoint{2.447812in}{2.341104in}}%
\pgfpathlineto{\pgfqpoint{2.448563in}{2.487004in}}%
\pgfpathlineto{\pgfqpoint{2.448809in}{2.292258in}}%
\pgfpathlineto{\pgfqpoint{2.448908in}{2.341462in}}%
\pgfpathlineto{\pgfqpoint{2.449794in}{2.484570in}}%
\pgfpathlineto{\pgfqpoint{2.450052in}{2.283114in}}%
\pgfpathlineto{\pgfqpoint{2.451037in}{2.481940in}}%
\pgfpathlineto{\pgfqpoint{2.451197in}{2.339910in}}%
\pgfpathlineto{\pgfqpoint{2.451295in}{2.288268in}}%
\pgfpathlineto{\pgfqpoint{2.451455in}{2.468113in}}%
\pgfpathlineto{\pgfqpoint{2.452182in}{2.403064in}}%
\pgfpathlineto{\pgfqpoint{2.452280in}{2.486783in}}%
\pgfpathlineto{\pgfqpoint{2.452538in}{2.288715in}}%
\pgfpathlineto{\pgfqpoint{2.453142in}{2.329063in}}%
\pgfpathlineto{\pgfqpoint{2.453769in}{2.293397in}}%
\pgfpathlineto{\pgfqpoint{2.453511in}{2.487596in}}%
\pgfpathlineto{\pgfqpoint{2.454212in}{2.394553in}}%
\pgfpathlineto{\pgfqpoint{2.454754in}{2.482804in}}%
\pgfpathlineto{\pgfqpoint{2.454532in}{2.305242in}}%
\pgfpathlineto{\pgfqpoint{2.454963in}{2.326847in}}%
\pgfpathlineto{\pgfqpoint{2.455012in}{2.289645in}}%
\pgfpathlineto{\pgfqpoint{2.455160in}{2.433417in}}%
\pgfpathlineto{\pgfqpoint{2.455997in}{2.478847in}}%
\pgfpathlineto{\pgfqpoint{2.455640in}{2.300121in}}%
\pgfpathlineto{\pgfqpoint{2.456218in}{2.334316in}}%
\pgfpathlineto{\pgfqpoint{2.456255in}{2.301246in}}%
\pgfpathlineto{\pgfqpoint{2.457215in}{2.470900in}}%
\pgfpathlineto{\pgfqpoint{2.457228in}{2.490233in}}%
\pgfpathlineto{\pgfqpoint{2.457486in}{2.301581in}}%
\pgfpathlineto{\pgfqpoint{2.458225in}{2.323465in}}%
\pgfpathlineto{\pgfqpoint{2.458729in}{2.303381in}}%
\pgfpathlineto{\pgfqpoint{2.458471in}{2.484112in}}%
\pgfpathlineto{\pgfqpoint{2.458877in}{2.422310in}}%
\pgfpathlineto{\pgfqpoint{2.459714in}{2.481452in}}%
\pgfpathlineto{\pgfqpoint{2.459492in}{2.307400in}}%
\pgfpathlineto{\pgfqpoint{2.459935in}{2.328710in}}%
\pgfpathlineto{\pgfqpoint{2.459972in}{2.298368in}}%
\pgfpathlineto{\pgfqpoint{2.460132in}{2.463258in}}%
\pgfpathlineto{\pgfqpoint{2.460908in}{2.398720in}}%
\pgfpathlineto{\pgfqpoint{2.460945in}{2.470894in}}%
\pgfpathlineto{\pgfqpoint{2.461966in}{2.307240in}}%
\pgfpathlineto{\pgfqpoint{2.461978in}{2.311935in}}%
\pgfpathlineto{\pgfqpoint{2.462188in}{2.483668in}}%
\pgfpathlineto{\pgfqpoint{2.462446in}{2.301459in}}%
\pgfpathlineto{\pgfqpoint{2.463135in}{2.398776in}}%
\pgfpathlineto{\pgfqpoint{2.463689in}{2.306385in}}%
\pgfpathlineto{\pgfqpoint{2.463431in}{2.472084in}}%
\pgfpathlineto{\pgfqpoint{2.464231in}{2.417013in}}%
\pgfpathlineto{\pgfqpoint{2.464255in}{2.472134in}}%
\pgfpathlineto{\pgfqpoint{2.464452in}{2.305350in}}%
\pgfpathlineto{\pgfqpoint{2.465301in}{2.336175in}}%
\pgfpathlineto{\pgfqpoint{2.465905in}{2.463834in}}%
\pgfpathlineto{\pgfqpoint{2.465695in}{2.309193in}}%
\pgfpathlineto{\pgfqpoint{2.466508in}{2.377568in}}%
\pgfpathlineto{\pgfqpoint{2.467406in}{2.306760in}}%
\pgfpathlineto{\pgfqpoint{2.467135in}{2.467868in}}%
\pgfpathlineto{\pgfqpoint{2.467628in}{2.342818in}}%
\pgfpathlineto{\pgfqpoint{2.468378in}{2.462043in}}%
\pgfpathlineto{\pgfqpoint{2.468169in}{2.311068in}}%
\pgfpathlineto{\pgfqpoint{2.468625in}{2.328195in}}%
\pgfpathlineto{\pgfqpoint{2.469412in}{2.306191in}}%
\pgfpathlineto{\pgfqpoint{2.469215in}{2.460462in}}%
\pgfpathlineto{\pgfqpoint{2.469597in}{2.444060in}}%
\pgfpathlineto{\pgfqpoint{2.469621in}{2.461502in}}%
\pgfpathlineto{\pgfqpoint{2.469880in}{2.320554in}}%
\pgfpathlineto{\pgfqpoint{2.470606in}{2.353160in}}%
\pgfpathlineto{\pgfqpoint{2.470655in}{2.308781in}}%
\pgfpathlineto{\pgfqpoint{2.470865in}{2.455528in}}%
\pgfpathlineto{\pgfqpoint{2.471677in}{2.419055in}}%
\pgfpathlineto{\pgfqpoint{2.472108in}{2.455903in}}%
\pgfpathlineto{\pgfqpoint{2.471763in}{2.312925in}}%
\pgfpathlineto{\pgfqpoint{2.472735in}{2.329321in}}%
\pgfpathlineto{\pgfqpoint{2.473338in}{2.457466in}}%
\pgfpathlineto{\pgfqpoint{2.473609in}{2.314737in}}%
\pgfpathlineto{\pgfqpoint{2.473929in}{2.400960in}}%
\pgfpathlineto{\pgfqpoint{2.474237in}{2.315653in}}%
\pgfpathlineto{\pgfqpoint{2.474581in}{2.454677in}}%
\pgfpathlineto{\pgfqpoint{2.475000in}{2.438736in}}%
\pgfpathlineto{\pgfqpoint{2.475825in}{2.459518in}}%
\pgfpathlineto{\pgfqpoint{2.475615in}{2.314551in}}%
\pgfpathlineto{\pgfqpoint{2.476009in}{2.340151in}}%
\pgfpathlineto{\pgfqpoint{2.476723in}{2.319675in}}%
\pgfpathlineto{\pgfqpoint{2.476255in}{2.448011in}}%
\pgfpathlineto{\pgfqpoint{2.477018in}{2.416113in}}%
\pgfpathlineto{\pgfqpoint{2.477068in}{2.450232in}}%
\pgfpathlineto{\pgfqpoint{2.477560in}{2.322228in}}%
\pgfpathlineto{\pgfqpoint{2.477954in}{2.325984in}}%
\pgfpathlineto{\pgfqpoint{2.478311in}{2.450465in}}%
\pgfpathlineto{\pgfqpoint{2.478089in}{2.317787in}}%
\pgfpathlineto{\pgfqpoint{2.479172in}{2.362901in}}%
\pgfpathlineto{\pgfqpoint{2.479332in}{2.321480in}}%
\pgfpathlineto{\pgfqpoint{2.479541in}{2.450948in}}%
\pgfpathlineto{\pgfqpoint{2.480280in}{2.372480in}}%
\pgfpathlineto{\pgfqpoint{2.480575in}{2.312596in}}%
\pgfpathlineto{\pgfqpoint{2.480785in}{2.452044in}}%
\pgfpathlineto{\pgfqpoint{2.481400in}{2.357809in}}%
\pgfpathlineto{\pgfqpoint{2.481806in}{2.317741in}}%
\pgfpathlineto{\pgfqpoint{2.482028in}{2.452229in}}%
\pgfpathlineto{\pgfqpoint{2.482409in}{2.387435in}}%
\pgfpathlineto{\pgfqpoint{2.483271in}{2.454678in}}%
\pgfpathlineto{\pgfqpoint{2.483049in}{2.318409in}}%
\pgfpathlineto{\pgfqpoint{2.483480in}{2.339892in}}%
\pgfpathlineto{\pgfqpoint{2.484501in}{2.442488in}}%
\pgfpathlineto{\pgfqpoint{2.484292in}{2.319659in}}%
\pgfpathlineto{\pgfqpoint{2.484674in}{2.355649in}}%
\pgfpathlineto{\pgfqpoint{2.485535in}{2.309800in}}%
\pgfpathlineto{\pgfqpoint{2.484944in}{2.442096in}}%
\pgfpathlineto{\pgfqpoint{2.485671in}{2.373874in}}%
\pgfpathlineto{\pgfqpoint{2.486188in}{2.450463in}}%
\pgfpathlineto{\pgfqpoint{2.486754in}{2.323518in}}%
\pgfpathlineto{\pgfqpoint{2.486778in}{2.315753in}}%
\pgfpathlineto{\pgfqpoint{2.486988in}{2.448689in}}%
\pgfpathlineto{\pgfqpoint{2.487763in}{2.374851in}}%
\pgfpathlineto{\pgfqpoint{2.488231in}{2.445582in}}%
\pgfpathlineto{\pgfqpoint{2.488009in}{2.311967in}}%
\pgfpathlineto{\pgfqpoint{2.488846in}{2.348773in}}%
\pgfpathlineto{\pgfqpoint{2.489252in}{2.304970in}}%
\pgfpathlineto{\pgfqpoint{2.489068in}{2.445390in}}%
\pgfpathlineto{\pgfqpoint{2.489868in}{2.387478in}}%
\pgfpathlineto{\pgfqpoint{2.489904in}{2.444534in}}%
\pgfpathlineto{\pgfqpoint{2.490495in}{2.305522in}}%
\pgfpathlineto{\pgfqpoint{2.490938in}{2.344410in}}%
\pgfpathlineto{\pgfqpoint{2.490975in}{2.341090in}}%
\pgfpathlineto{\pgfqpoint{2.491123in}{2.416950in}}%
\pgfpathlineto{\pgfqpoint{2.491148in}{2.451640in}}%
\pgfpathlineto{\pgfqpoint{2.491726in}{2.313361in}}%
\pgfpathlineto{\pgfqpoint{2.492194in}{2.345693in}}%
\pgfpathlineto{\pgfqpoint{2.492440in}{2.323488in}}%
\pgfpathlineto{\pgfqpoint{2.492391in}{2.446775in}}%
\pgfpathlineto{\pgfqpoint{2.492748in}{2.414303in}}%
\pgfpathlineto{\pgfqpoint{2.493634in}{2.450049in}}%
\pgfpathlineto{\pgfqpoint{2.492969in}{2.317337in}}%
\pgfpathlineto{\pgfqpoint{2.493806in}{2.352126in}}%
\pgfpathlineto{\pgfqpoint{2.494200in}{2.307470in}}%
\pgfpathlineto{\pgfqpoint{2.494028in}{2.444259in}}%
\pgfpathlineto{\pgfqpoint{2.494840in}{2.404252in}}%
\pgfpathlineto{\pgfqpoint{2.494864in}{2.449915in}}%
\pgfpathlineto{\pgfqpoint{2.495443in}{2.299607in}}%
\pgfpathlineto{\pgfqpoint{2.495923in}{2.354395in}}%
\pgfpathlineto{\pgfqpoint{2.496686in}{2.303838in}}%
\pgfpathlineto{\pgfqpoint{2.496108in}{2.444358in}}%
\pgfpathlineto{\pgfqpoint{2.496895in}{2.416709in}}%
\pgfpathlineto{\pgfqpoint{2.497351in}{2.445248in}}%
\pgfpathlineto{\pgfqpoint{2.497191in}{2.337607in}}%
\pgfpathlineto{\pgfqpoint{2.497388in}{2.340373in}}%
\pgfpathlineto{\pgfqpoint{2.497929in}{2.308377in}}%
\pgfpathlineto{\pgfqpoint{2.497757in}{2.438445in}}%
\pgfpathlineto{\pgfqpoint{2.498471in}{2.369635in}}%
\pgfpathlineto{\pgfqpoint{2.498594in}{2.449819in}}%
\pgfpathlineto{\pgfqpoint{2.499160in}{2.304482in}}%
\pgfpathlineto{\pgfqpoint{2.499566in}{2.345674in}}%
\pgfpathlineto{\pgfqpoint{2.500403in}{2.295777in}}%
\pgfpathlineto{\pgfqpoint{2.499837in}{2.458186in}}%
\pgfpathlineto{\pgfqpoint{2.500551in}{2.375688in}}%
\pgfpathlineto{\pgfqpoint{2.501080in}{2.454404in}}%
\pgfpathlineto{\pgfqpoint{2.501141in}{2.308433in}}%
\pgfpathlineto{\pgfqpoint{2.501634in}{2.320237in}}%
\pgfpathlineto{\pgfqpoint{2.501646in}{2.294147in}}%
\pgfpathlineto{\pgfqpoint{2.502323in}{2.455262in}}%
\pgfpathlineto{\pgfqpoint{2.502692in}{2.401133in}}%
\pgfpathlineto{\pgfqpoint{2.503566in}{2.459215in}}%
\pgfpathlineto{\pgfqpoint{2.502889in}{2.297470in}}%
\pgfpathlineto{\pgfqpoint{2.503751in}{2.349723in}}%
\pgfpathlineto{\pgfqpoint{2.503763in}{2.349253in}}%
\pgfpathlineto{\pgfqpoint{2.503800in}{2.377132in}}%
\pgfpathlineto{\pgfqpoint{2.503861in}{2.367246in}}%
\pgfpathlineto{\pgfqpoint{2.504809in}{2.460784in}}%
\pgfpathlineto{\pgfqpoint{2.504132in}{2.298902in}}%
\pgfpathlineto{\pgfqpoint{2.504957in}{2.359124in}}%
\pgfpathlineto{\pgfqpoint{2.505375in}{2.300370in}}%
\pgfpathlineto{\pgfqpoint{2.505646in}{2.437975in}}%
\pgfpathlineto{\pgfqpoint{2.506015in}{2.378708in}}%
\pgfpathlineto{\pgfqpoint{2.506052in}{2.454254in}}%
\pgfpathlineto{\pgfqpoint{2.506618in}{2.310330in}}%
\pgfpathlineto{\pgfqpoint{2.507111in}{2.341221in}}%
\pgfpathlineto{\pgfqpoint{2.507135in}{2.335625in}}%
\pgfpathlineto{\pgfqpoint{2.507184in}{2.401693in}}%
\pgfpathlineto{\pgfqpoint{2.507258in}{2.377797in}}%
\pgfpathlineto{\pgfqpoint{2.507295in}{2.444556in}}%
\pgfpathlineto{\pgfqpoint{2.507357in}{2.310485in}}%
\pgfpathlineto{\pgfqpoint{2.508354in}{2.336892in}}%
\pgfpathlineto{\pgfqpoint{2.508378in}{2.331545in}}%
\pgfpathlineto{\pgfqpoint{2.508403in}{2.363003in}}%
\pgfpathlineto{\pgfqpoint{2.508526in}{2.442330in}}%
\pgfpathlineto{\pgfqpoint{2.509203in}{2.307807in}}%
\pgfpathlineto{\pgfqpoint{2.509511in}{2.366920in}}%
\pgfpathlineto{\pgfqpoint{2.509831in}{2.309946in}}%
\pgfpathlineto{\pgfqpoint{2.509769in}{2.444278in}}%
\pgfpathlineto{\pgfqpoint{2.510495in}{2.410530in}}%
\pgfpathlineto{\pgfqpoint{2.511012in}{2.450245in}}%
\pgfpathlineto{\pgfqpoint{2.511074in}{2.312764in}}%
\pgfpathlineto{\pgfqpoint{2.511554in}{2.377583in}}%
\pgfpathlineto{\pgfqpoint{2.512317in}{2.308062in}}%
\pgfpathlineto{\pgfqpoint{2.511861in}{2.448319in}}%
\pgfpathlineto{\pgfqpoint{2.512649in}{2.408586in}}%
\pgfpathlineto{\pgfqpoint{2.513498in}{2.449756in}}%
\pgfpathlineto{\pgfqpoint{2.513560in}{2.317504in}}%
\pgfpathlineto{\pgfqpoint{2.513732in}{2.377274in}}%
\pgfpathlineto{\pgfqpoint{2.514175in}{2.310590in}}%
\pgfpathlineto{\pgfqpoint{2.514741in}{2.443894in}}%
\pgfpathlineto{\pgfqpoint{2.514827in}{2.389324in}}%
\pgfpathlineto{\pgfqpoint{2.515578in}{2.443833in}}%
\pgfpathlineto{\pgfqpoint{2.515418in}{2.318951in}}%
\pgfpathlineto{\pgfqpoint{2.515923in}{2.345150in}}%
\pgfpathlineto{\pgfqpoint{2.515984in}{2.448830in}}%
\pgfpathlineto{\pgfqpoint{2.516046in}{2.315269in}}%
\pgfpathlineto{\pgfqpoint{2.517018in}{2.347139in}}%
\pgfpathlineto{\pgfqpoint{2.517289in}{2.320714in}}%
\pgfpathlineto{\pgfqpoint{2.517227in}{2.444714in}}%
\pgfpathlineto{\pgfqpoint{2.518027in}{2.396880in}}%
\pgfpathlineto{\pgfqpoint{2.518458in}{2.449617in}}%
\pgfpathlineto{\pgfqpoint{2.519024in}{2.321796in}}%
\pgfpathlineto{\pgfqpoint{2.519111in}{2.331666in}}%
\pgfpathlineto{\pgfqpoint{2.519135in}{2.317554in}}%
\pgfpathlineto{\pgfqpoint{2.519307in}{2.445949in}}%
\pgfpathlineto{\pgfqpoint{2.520095in}{2.394589in}}%
\pgfpathlineto{\pgfqpoint{2.520538in}{2.442424in}}%
\pgfpathlineto{\pgfqpoint{2.521018in}{2.322768in}}%
\pgfpathlineto{\pgfqpoint{2.521178in}{2.363167in}}%
\pgfpathlineto{\pgfqpoint{2.521621in}{2.323632in}}%
\pgfpathlineto{\pgfqpoint{2.521781in}{2.441842in}}%
\pgfpathlineto{\pgfqpoint{2.522163in}{2.392017in}}%
\pgfpathlineto{\pgfqpoint{2.522187in}{2.435802in}}%
\pgfpathlineto{\pgfqpoint{2.522864in}{2.328676in}}%
\pgfpathlineto{\pgfqpoint{2.523246in}{2.334742in}}%
\pgfpathlineto{\pgfqpoint{2.524267in}{2.440521in}}%
\pgfpathlineto{\pgfqpoint{2.524107in}{2.319715in}}%
\pgfpathlineto{\pgfqpoint{2.524415in}{2.366160in}}%
\pgfpathlineto{\pgfqpoint{2.525351in}{2.324460in}}%
\pgfpathlineto{\pgfqpoint{2.524674in}{2.440278in}}%
\pgfpathlineto{\pgfqpoint{2.525461in}{2.373377in}}%
\pgfpathlineto{\pgfqpoint{2.525511in}{2.441125in}}%
\pgfpathlineto{\pgfqpoint{2.525978in}{2.318921in}}%
\pgfpathlineto{\pgfqpoint{2.526557in}{2.347133in}}%
\pgfpathlineto{\pgfqpoint{2.526594in}{2.321355in}}%
\pgfpathlineto{\pgfqpoint{2.526754in}{2.448214in}}%
\pgfpathlineto{\pgfqpoint{2.527640in}{2.342113in}}%
\pgfpathlineto{\pgfqpoint{2.527997in}{2.440620in}}%
\pgfpathlineto{\pgfqpoint{2.528464in}{2.318648in}}%
\pgfpathlineto{\pgfqpoint{2.528797in}{2.412096in}}%
\pgfpathlineto{\pgfqpoint{2.529067in}{2.310302in}}%
\pgfpathlineto{\pgfqpoint{2.529240in}{2.446504in}}%
\pgfpathlineto{\pgfqpoint{2.530003in}{2.377128in}}%
\pgfpathlineto{\pgfqpoint{2.530470in}{2.441102in}}%
\pgfpathlineto{\pgfqpoint{2.530310in}{2.305813in}}%
\pgfpathlineto{\pgfqpoint{2.531098in}{2.377224in}}%
\pgfpathlineto{\pgfqpoint{2.531554in}{2.307291in}}%
\pgfpathlineto{\pgfqpoint{2.531714in}{2.449610in}}%
\pgfpathlineto{\pgfqpoint{2.532107in}{2.432320in}}%
\pgfpathlineto{\pgfqpoint{2.532957in}{2.450445in}}%
\pgfpathlineto{\pgfqpoint{2.532797in}{2.310448in}}%
\pgfpathlineto{\pgfqpoint{2.533006in}{2.338354in}}%
\pgfpathlineto{\pgfqpoint{2.533412in}{2.316629in}}%
\pgfpathlineto{\pgfqpoint{2.533474in}{2.440977in}}%
\pgfpathlineto{\pgfqpoint{2.533954in}{2.366025in}}%
\pgfpathlineto{\pgfqpoint{2.534200in}{2.456191in}}%
\pgfpathlineto{\pgfqpoint{2.534027in}{2.310287in}}%
\pgfpathlineto{\pgfqpoint{2.535049in}{2.382366in}}%
\pgfpathlineto{\pgfqpoint{2.535270in}{2.305747in}}%
\pgfpathlineto{\pgfqpoint{2.535430in}{2.452435in}}%
\pgfpathlineto{\pgfqpoint{2.536144in}{2.389980in}}%
\pgfpathlineto{\pgfqpoint{2.536674in}{2.449850in}}%
\pgfpathlineto{\pgfqpoint{2.536514in}{2.306457in}}%
\pgfpathlineto{\pgfqpoint{2.537215in}{2.394648in}}%
\pgfpathlineto{\pgfqpoint{2.537757in}{2.309836in}}%
\pgfpathlineto{\pgfqpoint{2.537917in}{2.450054in}}%
\pgfpathlineto{\pgfqpoint{2.538310in}{2.425606in}}%
\pgfpathlineto{\pgfqpoint{2.539160in}{2.452392in}}%
\pgfpathlineto{\pgfqpoint{2.538987in}{2.313951in}}%
\pgfpathlineto{\pgfqpoint{2.539295in}{2.367373in}}%
\pgfpathlineto{\pgfqpoint{2.540230in}{2.312707in}}%
\pgfpathlineto{\pgfqpoint{2.539677in}{2.438504in}}%
\pgfpathlineto{\pgfqpoint{2.540366in}{2.389586in}}%
\pgfpathlineto{\pgfqpoint{2.540403in}{2.453545in}}%
\pgfpathlineto{\pgfqpoint{2.540858in}{2.331835in}}%
\pgfpathlineto{\pgfqpoint{2.541449in}{2.333704in}}%
\pgfpathlineto{\pgfqpoint{2.541474in}{2.315430in}}%
\pgfpathlineto{\pgfqpoint{2.541646in}{2.450060in}}%
\pgfpathlineto{\pgfqpoint{2.542544in}{2.348392in}}%
\pgfpathlineto{\pgfqpoint{2.542889in}{2.450053in}}%
\pgfpathlineto{\pgfqpoint{2.542704in}{2.317924in}}%
\pgfpathlineto{\pgfqpoint{2.543664in}{2.374049in}}%
\pgfpathlineto{\pgfqpoint{2.543947in}{2.319108in}}%
\pgfpathlineto{\pgfqpoint{2.543714in}{2.408012in}}%
\pgfpathlineto{\pgfqpoint{2.544095in}{2.390185in}}%
\pgfpathlineto{\pgfqpoint{2.544132in}{2.450461in}}%
\pgfpathlineto{\pgfqpoint{2.545178in}{2.328047in}}%
\pgfpathlineto{\pgfqpoint{2.545190in}{2.320794in}}%
\pgfpathlineto{\pgfqpoint{2.545375in}{2.449248in}}%
\pgfpathlineto{\pgfqpoint{2.546175in}{2.386603in}}%
\pgfpathlineto{\pgfqpoint{2.546606in}{2.450336in}}%
\pgfpathlineto{\pgfqpoint{2.546433in}{2.322179in}}%
\pgfpathlineto{\pgfqpoint{2.547258in}{2.354436in}}%
\pgfpathlineto{\pgfqpoint{2.547664in}{2.321937in}}%
\pgfpathlineto{\pgfqpoint{2.547849in}{2.446845in}}%
\pgfpathlineto{\pgfqpoint{2.548341in}{2.366970in}}%
\pgfpathlineto{\pgfqpoint{2.549092in}{2.447806in}}%
\pgfpathlineto{\pgfqpoint{2.548907in}{2.324216in}}%
\pgfpathlineto{\pgfqpoint{2.549461in}{2.389026in}}%
\pgfpathlineto{\pgfqpoint{2.550335in}{2.444447in}}%
\pgfpathlineto{\pgfqpoint{2.550150in}{2.327730in}}%
\pgfpathlineto{\pgfqpoint{2.550470in}{2.388466in}}%
\pgfpathlineto{\pgfqpoint{2.551381in}{2.329771in}}%
\pgfpathlineto{\pgfqpoint{2.550864in}{2.426710in}}%
\pgfpathlineto{\pgfqpoint{2.551553in}{2.418530in}}%
\pgfpathlineto{\pgfqpoint{2.551578in}{2.441789in}}%
\pgfpathlineto{\pgfqpoint{2.552477in}{2.329216in}}%
\pgfpathlineto{\pgfqpoint{2.552612in}{2.342490in}}%
\pgfpathlineto{\pgfqpoint{2.552809in}{2.435324in}}%
\pgfpathlineto{\pgfqpoint{2.553720in}{2.324977in}}%
\pgfpathlineto{\pgfqpoint{2.554052in}{2.433198in}}%
\pgfpathlineto{\pgfqpoint{2.554901in}{2.409604in}}%
\pgfpathlineto{\pgfqpoint{2.554963in}{2.324015in}}%
\pgfpathlineto{\pgfqpoint{2.555295in}{2.432918in}}%
\pgfpathlineto{\pgfqpoint{2.556021in}{2.381996in}}%
\pgfpathlineto{\pgfqpoint{2.556538in}{2.437587in}}%
\pgfpathlineto{\pgfqpoint{2.556206in}{2.324319in}}%
\pgfpathlineto{\pgfqpoint{2.557141in}{2.397081in}}%
\pgfpathlineto{\pgfqpoint{2.558052in}{2.323084in}}%
\pgfpathlineto{\pgfqpoint{2.557781in}{2.437325in}}%
\pgfpathlineto{\pgfqpoint{2.558273in}{2.370363in}}%
\pgfpathlineto{\pgfqpoint{2.559024in}{2.435219in}}%
\pgfpathlineto{\pgfqpoint{2.559295in}{2.320474in}}%
\pgfpathlineto{\pgfqpoint{2.559381in}{2.388322in}}%
\pgfpathlineto{\pgfqpoint{2.560267in}{2.429187in}}%
\pgfpathlineto{\pgfqpoint{2.560538in}{2.319669in}}%
\pgfpathlineto{\pgfqpoint{2.561498in}{2.430024in}}%
\pgfpathlineto{\pgfqpoint{2.561658in}{2.342263in}}%
\pgfpathlineto{\pgfqpoint{2.562397in}{2.319346in}}%
\pgfpathlineto{\pgfqpoint{2.562027in}{2.427601in}}%
\pgfpathlineto{\pgfqpoint{2.562643in}{2.401026in}}%
\pgfpathlineto{\pgfqpoint{2.562741in}{2.432164in}}%
\pgfpathlineto{\pgfqpoint{2.563000in}{2.331397in}}%
\pgfpathlineto{\pgfqpoint{2.563012in}{2.313309in}}%
\pgfpathlineto{\pgfqpoint{2.563984in}{2.429053in}}%
\pgfpathlineto{\pgfqpoint{2.564070in}{2.372375in}}%
\pgfpathlineto{\pgfqpoint{2.564821in}{2.420800in}}%
\pgfpathlineto{\pgfqpoint{2.564255in}{2.312588in}}%
\pgfpathlineto{\pgfqpoint{2.565178in}{2.377770in}}%
\pgfpathlineto{\pgfqpoint{2.565658in}{2.428113in}}%
\pgfpathlineto{\pgfqpoint{2.565498in}{2.314719in}}%
\pgfpathlineto{\pgfqpoint{2.566212in}{2.353402in}}%
\pgfpathlineto{\pgfqpoint{2.566729in}{2.312883in}}%
\pgfpathlineto{\pgfqpoint{2.566889in}{2.435367in}}%
\pgfpathlineto{\pgfqpoint{2.567283in}{2.401574in}}%
\pgfpathlineto{\pgfqpoint{2.568132in}{2.438379in}}%
\pgfpathlineto{\pgfqpoint{2.567972in}{2.308312in}}%
\pgfpathlineto{\pgfqpoint{2.568353in}{2.360954in}}%
\pgfpathlineto{\pgfqpoint{2.569215in}{2.307952in}}%
\pgfpathlineto{\pgfqpoint{2.569375in}{2.436252in}}%
\pgfpathlineto{\pgfqpoint{2.569436in}{2.369421in}}%
\pgfpathlineto{\pgfqpoint{2.570187in}{2.434132in}}%
\pgfpathlineto{\pgfqpoint{2.570458in}{2.311140in}}%
\pgfpathlineto{\pgfqpoint{2.570532in}{2.376295in}}%
\pgfpathlineto{\pgfqpoint{2.571073in}{2.313652in}}%
\pgfpathlineto{\pgfqpoint{2.570606in}{2.444791in}}%
\pgfpathlineto{\pgfqpoint{2.571640in}{2.376229in}}%
\pgfpathlineto{\pgfqpoint{2.571689in}{2.309207in}}%
\pgfpathlineto{\pgfqpoint{2.571849in}{2.448200in}}%
\pgfpathlineto{\pgfqpoint{2.572649in}{2.415605in}}%
\pgfpathlineto{\pgfqpoint{2.573092in}{2.448941in}}%
\pgfpathlineto{\pgfqpoint{2.572932in}{2.305020in}}%
\pgfpathlineto{\pgfqpoint{2.573732in}{2.377674in}}%
\pgfpathlineto{\pgfqpoint{2.574175in}{2.303117in}}%
\pgfpathlineto{\pgfqpoint{2.574323in}{2.447662in}}%
\pgfpathlineto{\pgfqpoint{2.574827in}{2.385075in}}%
\pgfpathlineto{\pgfqpoint{2.575566in}{2.453917in}}%
\pgfpathlineto{\pgfqpoint{2.575406in}{2.307557in}}%
\pgfpathlineto{\pgfqpoint{2.575910in}{2.349098in}}%
\pgfpathlineto{\pgfqpoint{2.576649in}{2.309212in}}%
\pgfpathlineto{\pgfqpoint{2.576809in}{2.454834in}}%
\pgfpathlineto{\pgfqpoint{2.576993in}{2.371828in}}%
\pgfpathlineto{\pgfqpoint{2.577215in}{2.438148in}}%
\pgfpathlineto{\pgfqpoint{2.577264in}{2.312812in}}%
\pgfpathlineto{\pgfqpoint{2.577867in}{2.332509in}}%
\pgfpathlineto{\pgfqpoint{2.578507in}{2.309051in}}%
\pgfpathlineto{\pgfqpoint{2.578040in}{2.453943in}}%
\pgfpathlineto{\pgfqpoint{2.578938in}{2.355266in}}%
\pgfpathlineto{\pgfqpoint{2.579283in}{2.457880in}}%
\pgfpathlineto{\pgfqpoint{2.579750in}{2.306446in}}%
\pgfpathlineto{\pgfqpoint{2.580046in}{2.370138in}}%
\pgfpathlineto{\pgfqpoint{2.580366in}{2.305762in}}%
\pgfpathlineto{\pgfqpoint{2.580526in}{2.452901in}}%
\pgfpathlineto{\pgfqpoint{2.581141in}{2.380580in}}%
\pgfpathlineto{\pgfqpoint{2.581756in}{2.452645in}}%
\pgfpathlineto{\pgfqpoint{2.581609in}{2.312898in}}%
\pgfpathlineto{\pgfqpoint{2.582200in}{2.368108in}}%
\pgfpathlineto{\pgfqpoint{2.582224in}{2.310018in}}%
\pgfpathlineto{\pgfqpoint{2.583000in}{2.456292in}}%
\pgfpathlineto{\pgfqpoint{2.583295in}{2.390951in}}%
\pgfpathlineto{\pgfqpoint{2.584083in}{2.306642in}}%
\pgfpathlineto{\pgfqpoint{2.583824in}{2.438351in}}%
\pgfpathlineto{\pgfqpoint{2.584206in}{2.386470in}}%
\pgfpathlineto{\pgfqpoint{2.584243in}{2.455894in}}%
\pgfpathlineto{\pgfqpoint{2.584698in}{2.309680in}}%
\pgfpathlineto{\pgfqpoint{2.585289in}{2.342588in}}%
\pgfpathlineto{\pgfqpoint{2.585941in}{2.306310in}}%
\pgfpathlineto{\pgfqpoint{2.585473in}{2.458148in}}%
\pgfpathlineto{\pgfqpoint{2.586372in}{2.352167in}}%
\pgfpathlineto{\pgfqpoint{2.586716in}{2.456685in}}%
\pgfpathlineto{\pgfqpoint{2.586556in}{2.308648in}}%
\pgfpathlineto{\pgfqpoint{2.587480in}{2.363779in}}%
\pgfpathlineto{\pgfqpoint{2.587799in}{2.310064in}}%
\pgfpathlineto{\pgfqpoint{2.587947in}{2.455429in}}%
\pgfpathlineto{\pgfqpoint{2.588452in}{2.390305in}}%
\pgfpathlineto{\pgfqpoint{2.589190in}{2.456805in}}%
\pgfpathlineto{\pgfqpoint{2.589030in}{2.309485in}}%
\pgfpathlineto{\pgfqpoint{2.589523in}{2.350259in}}%
\pgfpathlineto{\pgfqpoint{2.590273in}{2.308680in}}%
\pgfpathlineto{\pgfqpoint{2.590421in}{2.456776in}}%
\pgfpathlineto{\pgfqpoint{2.590507in}{2.396157in}}%
\pgfpathlineto{\pgfqpoint{2.590532in}{2.443970in}}%
\pgfpathlineto{\pgfqpoint{2.590889in}{2.309132in}}%
\pgfpathlineto{\pgfqpoint{2.591590in}{2.335656in}}%
\pgfpathlineto{\pgfqpoint{2.592132in}{2.314741in}}%
\pgfpathlineto{\pgfqpoint{2.591664in}{2.456393in}}%
\pgfpathlineto{\pgfqpoint{2.592673in}{2.368408in}}%
\pgfpathlineto{\pgfqpoint{2.592747in}{2.315012in}}%
\pgfpathlineto{\pgfqpoint{2.592907in}{2.449295in}}%
\pgfpathlineto{\pgfqpoint{2.593609in}{2.417938in}}%
\pgfpathlineto{\pgfqpoint{2.594138in}{2.451523in}}%
\pgfpathlineto{\pgfqpoint{2.593978in}{2.316061in}}%
\pgfpathlineto{\pgfqpoint{2.594593in}{2.337150in}}%
\pgfpathlineto{\pgfqpoint{2.595221in}{2.316195in}}%
\pgfpathlineto{\pgfqpoint{2.595381in}{2.452160in}}%
\pgfpathlineto{\pgfqpoint{2.595664in}{2.378384in}}%
\pgfpathlineto{\pgfqpoint{2.596612in}{2.445682in}}%
\pgfpathlineto{\pgfqpoint{2.596452in}{2.315649in}}%
\pgfpathlineto{\pgfqpoint{2.596759in}{2.384503in}}%
\pgfpathlineto{\pgfqpoint{2.597695in}{2.316156in}}%
\pgfpathlineto{\pgfqpoint{2.597338in}{2.425640in}}%
\pgfpathlineto{\pgfqpoint{2.597830in}{2.407164in}}%
\pgfpathlineto{\pgfqpoint{2.597855in}{2.443878in}}%
\pgfpathlineto{\pgfqpoint{2.598323in}{2.328149in}}%
\pgfpathlineto{\pgfqpoint{2.598889in}{2.346519in}}%
\pgfpathlineto{\pgfqpoint{2.598926in}{2.314489in}}%
\pgfpathlineto{\pgfqpoint{2.599098in}{2.442093in}}%
\pgfpathlineto{\pgfqpoint{2.599984in}{2.350027in}}%
\pgfpathlineto{\pgfqpoint{2.600329in}{2.442733in}}%
\pgfpathlineto{\pgfqpoint{2.600156in}{2.318464in}}%
\pgfpathlineto{\pgfqpoint{2.601190in}{2.389371in}}%
\pgfpathlineto{\pgfqpoint{2.601399in}{2.314997in}}%
\pgfpathlineto{\pgfqpoint{2.601572in}{2.444764in}}%
\pgfpathlineto{\pgfqpoint{2.602273in}{2.410489in}}%
\pgfpathlineto{\pgfqpoint{2.602815in}{2.440496in}}%
\pgfpathlineto{\pgfqpoint{2.602643in}{2.318346in}}%
\pgfpathlineto{\pgfqpoint{2.603344in}{2.372072in}}%
\pgfpathlineto{\pgfqpoint{2.603873in}{2.313936in}}%
\pgfpathlineto{\pgfqpoint{2.604046in}{2.436522in}}%
\pgfpathlineto{\pgfqpoint{2.604439in}{2.393894in}}%
\pgfpathlineto{\pgfqpoint{2.605289in}{2.437293in}}%
\pgfpathlineto{\pgfqpoint{2.605116in}{2.318572in}}%
\pgfpathlineto{\pgfqpoint{2.605486in}{2.373919in}}%
\pgfpathlineto{\pgfqpoint{2.606347in}{2.315851in}}%
\pgfpathlineto{\pgfqpoint{2.606532in}{2.434965in}}%
\pgfpathlineto{\pgfqpoint{2.607590in}{2.316068in}}%
\pgfpathlineto{\pgfqpoint{2.607726in}{2.370407in}}%
\pgfpathlineto{\pgfqpoint{2.607775in}{2.428888in}}%
\pgfpathlineto{\pgfqpoint{2.608809in}{2.331064in}}%
\pgfpathlineto{\pgfqpoint{2.608833in}{2.318565in}}%
\pgfpathlineto{\pgfqpoint{2.609006in}{2.430886in}}%
\pgfpathlineto{\pgfqpoint{2.609830in}{2.402597in}}%
\pgfpathlineto{\pgfqpoint{2.610249in}{2.429591in}}%
\pgfpathlineto{\pgfqpoint{2.610064in}{2.320346in}}%
\pgfpathlineto{\pgfqpoint{2.610778in}{2.359502in}}%
\pgfpathlineto{\pgfqpoint{2.611307in}{2.321716in}}%
\pgfpathlineto{\pgfqpoint{2.611492in}{2.431382in}}%
\pgfpathlineto{\pgfqpoint{2.611849in}{2.375609in}}%
\pgfpathlineto{\pgfqpoint{2.612722in}{2.428681in}}%
\pgfpathlineto{\pgfqpoint{2.612538in}{2.319630in}}%
\pgfpathlineto{\pgfqpoint{2.612919in}{2.370538in}}%
\pgfpathlineto{\pgfqpoint{2.613781in}{2.319303in}}%
\pgfpathlineto{\pgfqpoint{2.613966in}{2.429972in}}%
\pgfpathlineto{\pgfqpoint{2.614015in}{2.380935in}}%
\pgfpathlineto{\pgfqpoint{2.615012in}{2.317569in}}%
\pgfpathlineto{\pgfqpoint{2.614679in}{2.414038in}}%
\pgfpathlineto{\pgfqpoint{2.615147in}{2.372712in}}%
\pgfpathlineto{\pgfqpoint{2.615209in}{2.430089in}}%
\pgfpathlineto{\pgfqpoint{2.616107in}{2.337770in}}%
\pgfpathlineto{\pgfqpoint{2.616218in}{2.355195in}}%
\pgfpathlineto{\pgfqpoint{2.616255in}{2.321528in}}%
\pgfpathlineto{\pgfqpoint{2.616439in}{2.432987in}}%
\pgfpathlineto{\pgfqpoint{2.617276in}{2.413037in}}%
\pgfpathlineto{\pgfqpoint{2.617486in}{2.319148in}}%
\pgfpathlineto{\pgfqpoint{2.617682in}{2.432965in}}%
\pgfpathlineto{\pgfqpoint{2.618384in}{2.411667in}}%
\pgfpathlineto{\pgfqpoint{2.618926in}{2.436763in}}%
\pgfpathlineto{\pgfqpoint{2.618729in}{2.319819in}}%
\pgfpathlineto{\pgfqpoint{2.619369in}{2.374098in}}%
\pgfpathlineto{\pgfqpoint{2.619959in}{2.322572in}}%
\pgfpathlineto{\pgfqpoint{2.620169in}{2.440655in}}%
\pgfpathlineto{\pgfqpoint{2.620464in}{2.371075in}}%
\pgfpathlineto{\pgfqpoint{2.621399in}{2.440579in}}%
\pgfpathlineto{\pgfqpoint{2.621202in}{2.322606in}}%
\pgfpathlineto{\pgfqpoint{2.621572in}{2.370284in}}%
\pgfpathlineto{\pgfqpoint{2.622446in}{2.319544in}}%
\pgfpathlineto{\pgfqpoint{2.622236in}{2.419095in}}%
\pgfpathlineto{\pgfqpoint{2.622593in}{2.384253in}}%
\pgfpathlineto{\pgfqpoint{2.622642in}{2.438820in}}%
\pgfpathlineto{\pgfqpoint{2.623664in}{2.332178in}}%
\pgfpathlineto{\pgfqpoint{2.623676in}{2.319702in}}%
\pgfpathlineto{\pgfqpoint{2.623886in}{2.440343in}}%
\pgfpathlineto{\pgfqpoint{2.624686in}{2.404184in}}%
\pgfpathlineto{\pgfqpoint{2.625116in}{2.443259in}}%
\pgfpathlineto{\pgfqpoint{2.624919in}{2.320672in}}%
\pgfpathlineto{\pgfqpoint{2.625769in}{2.375920in}}%
\pgfpathlineto{\pgfqpoint{2.626150in}{2.322825in}}%
\pgfpathlineto{\pgfqpoint{2.626359in}{2.445358in}}%
\pgfpathlineto{\pgfqpoint{2.626901in}{2.365866in}}%
\pgfpathlineto{\pgfqpoint{2.627602in}{2.446505in}}%
\pgfpathlineto{\pgfqpoint{2.627393in}{2.317518in}}%
\pgfpathlineto{\pgfqpoint{2.628033in}{2.386420in}}%
\pgfpathlineto{\pgfqpoint{2.628636in}{2.320913in}}%
\pgfpathlineto{\pgfqpoint{2.628833in}{2.446204in}}%
\pgfpathlineto{\pgfqpoint{2.629141in}{2.379586in}}%
\pgfpathlineto{\pgfqpoint{2.630076in}{2.449217in}}%
\pgfpathlineto{\pgfqpoint{2.629867in}{2.318994in}}%
\pgfpathlineto{\pgfqpoint{2.630224in}{2.372747in}}%
\pgfpathlineto{\pgfqpoint{2.631110in}{2.323548in}}%
\pgfpathlineto{\pgfqpoint{2.630901in}{2.427717in}}%
\pgfpathlineto{\pgfqpoint{2.631270in}{2.398412in}}%
\pgfpathlineto{\pgfqpoint{2.631319in}{2.449475in}}%
\pgfpathlineto{\pgfqpoint{2.632205in}{2.331197in}}%
\pgfpathlineto{\pgfqpoint{2.632329in}{2.333295in}}%
\pgfpathlineto{\pgfqpoint{2.632341in}{2.320903in}}%
\pgfpathlineto{\pgfqpoint{2.632550in}{2.450173in}}%
\pgfpathlineto{\pgfqpoint{2.633350in}{2.401686in}}%
\pgfpathlineto{\pgfqpoint{2.633793in}{2.449489in}}%
\pgfpathlineto{\pgfqpoint{2.633584in}{2.322359in}}%
\pgfpathlineto{\pgfqpoint{2.634433in}{2.375711in}}%
\pgfpathlineto{\pgfqpoint{2.634815in}{2.326897in}}%
\pgfpathlineto{\pgfqpoint{2.635036in}{2.451865in}}%
\pgfpathlineto{\pgfqpoint{2.635553in}{2.367992in}}%
\pgfpathlineto{\pgfqpoint{2.636267in}{2.449116in}}%
\pgfpathlineto{\pgfqpoint{2.636058in}{2.328770in}}%
\pgfpathlineto{\pgfqpoint{2.636513in}{2.341308in}}%
\pgfpathlineto{\pgfqpoint{2.636747in}{2.328607in}}%
\pgfpathlineto{\pgfqpoint{2.637510in}{2.450893in}}%
\pgfpathlineto{\pgfqpoint{2.637572in}{2.379078in}}%
\pgfpathlineto{\pgfqpoint{2.638335in}{2.433140in}}%
\pgfpathlineto{\pgfqpoint{2.637990in}{2.328905in}}%
\pgfpathlineto{\pgfqpoint{2.638679in}{2.384614in}}%
\pgfpathlineto{\pgfqpoint{2.638753in}{2.450499in}}%
\pgfpathlineto{\pgfqpoint{2.638999in}{2.331849in}}%
\pgfpathlineto{\pgfqpoint{2.639209in}{2.343596in}}%
\pgfpathlineto{\pgfqpoint{2.640242in}{2.327662in}}%
\pgfpathlineto{\pgfqpoint{2.639984in}{2.448532in}}%
\pgfpathlineto{\pgfqpoint{2.640279in}{2.370169in}}%
\pgfpathlineto{\pgfqpoint{2.641227in}{2.449850in}}%
\pgfpathlineto{\pgfqpoint{2.640464in}{2.325872in}}%
\pgfpathlineto{\pgfqpoint{2.641375in}{2.359282in}}%
\pgfpathlineto{\pgfqpoint{2.641707in}{2.325682in}}%
\pgfpathlineto{\pgfqpoint{2.642052in}{2.433954in}}%
\pgfpathlineto{\pgfqpoint{2.642396in}{2.389697in}}%
\pgfpathlineto{\pgfqpoint{2.642950in}{2.328005in}}%
\pgfpathlineto{\pgfqpoint{2.642470in}{2.452149in}}%
\pgfpathlineto{\pgfqpoint{2.643270in}{2.402729in}}%
\pgfpathlineto{\pgfqpoint{2.643713in}{2.451002in}}%
\pgfpathlineto{\pgfqpoint{2.644193in}{2.327029in}}%
\pgfpathlineto{\pgfqpoint{2.644353in}{2.369308in}}%
\pgfpathlineto{\pgfqpoint{2.644944in}{2.448622in}}%
\pgfpathlineto{\pgfqpoint{2.644599in}{2.332575in}}%
\pgfpathlineto{\pgfqpoint{2.645178in}{2.347629in}}%
\pgfpathlineto{\pgfqpoint{2.645424in}{2.326152in}}%
\pgfpathlineto{\pgfqpoint{2.646175in}{2.438310in}}%
\pgfpathlineto{\pgfqpoint{2.646187in}{2.452360in}}%
\pgfpathlineto{\pgfqpoint{2.646667in}{2.322312in}}%
\pgfpathlineto{\pgfqpoint{2.647196in}{2.339347in}}%
\pgfpathlineto{\pgfqpoint{2.647430in}{2.449434in}}%
\pgfpathlineto{\pgfqpoint{2.647910in}{2.318049in}}%
\pgfpathlineto{\pgfqpoint{2.648292in}{2.359640in}}%
\pgfpathlineto{\pgfqpoint{2.649153in}{2.322902in}}%
\pgfpathlineto{\pgfqpoint{2.648673in}{2.448271in}}%
\pgfpathlineto{\pgfqpoint{2.649375in}{2.393974in}}%
\pgfpathlineto{\pgfqpoint{2.649904in}{2.450853in}}%
\pgfpathlineto{\pgfqpoint{2.649756in}{2.327020in}}%
\pgfpathlineto{\pgfqpoint{2.650359in}{2.360158in}}%
\pgfpathlineto{\pgfqpoint{2.650384in}{2.319330in}}%
\pgfpathlineto{\pgfqpoint{2.651147in}{2.448511in}}%
\pgfpathlineto{\pgfqpoint{2.651442in}{2.384394in}}%
\pgfpathlineto{\pgfqpoint{2.652390in}{2.450108in}}%
\pgfpathlineto{\pgfqpoint{2.651627in}{2.316575in}}%
\pgfpathlineto{\pgfqpoint{2.652525in}{2.363380in}}%
\pgfpathlineto{\pgfqpoint{2.652858in}{2.319836in}}%
\pgfpathlineto{\pgfqpoint{2.653215in}{2.434464in}}%
\pgfpathlineto{\pgfqpoint{2.653584in}{2.385583in}}%
\pgfpathlineto{\pgfqpoint{2.653633in}{2.452333in}}%
\pgfpathlineto{\pgfqpoint{2.654101in}{2.318518in}}%
\pgfpathlineto{\pgfqpoint{2.654667in}{2.354040in}}%
\pgfpathlineto{\pgfqpoint{2.654716in}{2.318495in}}%
\pgfpathlineto{\pgfqpoint{2.654864in}{2.450439in}}%
\pgfpathlineto{\pgfqpoint{2.655664in}{2.394771in}}%
\pgfpathlineto{\pgfqpoint{2.656107in}{2.451722in}}%
\pgfpathlineto{\pgfqpoint{2.655959in}{2.318423in}}%
\pgfpathlineto{\pgfqpoint{2.656759in}{2.378735in}}%
\pgfpathlineto{\pgfqpoint{2.657190in}{2.318788in}}%
\pgfpathlineto{\pgfqpoint{2.657350in}{2.452279in}}%
\pgfpathlineto{\pgfqpoint{2.657842in}{2.361522in}}%
\pgfpathlineto{\pgfqpoint{2.658593in}{2.450691in}}%
\pgfpathlineto{\pgfqpoint{2.658433in}{2.316205in}}%
\pgfpathlineto{\pgfqpoint{2.658938in}{2.346124in}}%
\pgfpathlineto{\pgfqpoint{2.659676in}{2.318928in}}%
\pgfpathlineto{\pgfqpoint{2.659812in}{2.434480in}}%
\pgfpathlineto{\pgfqpoint{2.659836in}{2.450806in}}%
\pgfpathlineto{\pgfqpoint{2.660292in}{2.331400in}}%
\pgfpathlineto{\pgfqpoint{2.660858in}{2.362415in}}%
\pgfpathlineto{\pgfqpoint{2.660907in}{2.322970in}}%
\pgfpathlineto{\pgfqpoint{2.661067in}{2.451427in}}%
\pgfpathlineto{\pgfqpoint{2.661965in}{2.350965in}}%
\pgfpathlineto{\pgfqpoint{2.662310in}{2.450084in}}%
\pgfpathlineto{\pgfqpoint{2.662150in}{2.321121in}}%
\pgfpathlineto{\pgfqpoint{2.663122in}{2.407351in}}%
\pgfpathlineto{\pgfqpoint{2.663553in}{2.448056in}}%
\pgfpathlineto{\pgfqpoint{2.663393in}{2.324091in}}%
\pgfpathlineto{\pgfqpoint{2.664168in}{2.385164in}}%
\pgfpathlineto{\pgfqpoint{2.664624in}{2.328995in}}%
\pgfpathlineto{\pgfqpoint{2.664796in}{2.444207in}}%
\pgfpathlineto{\pgfqpoint{2.665288in}{2.375266in}}%
\pgfpathlineto{\pgfqpoint{2.666039in}{2.443810in}}%
\pgfpathlineto{\pgfqpoint{2.665855in}{2.327571in}}%
\pgfpathlineto{\pgfqpoint{2.666384in}{2.357509in}}%
\pgfpathlineto{\pgfqpoint{2.667270in}{2.444193in}}%
\pgfpathlineto{\pgfqpoint{2.667085in}{2.328483in}}%
\pgfpathlineto{\pgfqpoint{2.667467in}{2.351075in}}%
\pgfpathlineto{\pgfqpoint{2.668328in}{2.330330in}}%
\pgfpathlineto{\pgfqpoint{2.668107in}{2.409468in}}%
\pgfpathlineto{\pgfqpoint{2.668378in}{2.370717in}}%
\pgfpathlineto{\pgfqpoint{2.668513in}{2.445460in}}%
\pgfpathlineto{\pgfqpoint{2.668735in}{2.341627in}}%
\pgfpathlineto{\pgfqpoint{2.669473in}{2.370111in}}%
\pgfpathlineto{\pgfqpoint{2.669559in}{2.330239in}}%
\pgfpathlineto{\pgfqpoint{2.669756in}{2.440966in}}%
\pgfpathlineto{\pgfqpoint{2.670556in}{2.378747in}}%
\pgfpathlineto{\pgfqpoint{2.670999in}{2.440792in}}%
\pgfpathlineto{\pgfqpoint{2.670802in}{2.329923in}}%
\pgfpathlineto{\pgfqpoint{2.671651in}{2.367488in}}%
\pgfpathlineto{\pgfqpoint{2.672045in}{2.333747in}}%
\pgfpathlineto{\pgfqpoint{2.672242in}{2.439644in}}%
\pgfpathlineto{\pgfqpoint{2.672735in}{2.371959in}}%
\pgfpathlineto{\pgfqpoint{2.673473in}{2.436228in}}%
\pgfpathlineto{\pgfqpoint{2.673276in}{2.334156in}}%
\pgfpathlineto{\pgfqpoint{2.673842in}{2.372437in}}%
\pgfpathlineto{\pgfqpoint{2.673879in}{2.391125in}}%
\pgfpathlineto{\pgfqpoint{2.674716in}{2.438173in}}%
\pgfpathlineto{\pgfqpoint{2.674519in}{2.333068in}}%
\pgfpathlineto{\pgfqpoint{2.674888in}{2.343689in}}%
\pgfpathlineto{\pgfqpoint{2.674901in}{2.342564in}}%
\pgfpathlineto{\pgfqpoint{2.675245in}{2.407513in}}%
\pgfpathlineto{\pgfqpoint{2.675516in}{2.380477in}}%
\pgfpathlineto{\pgfqpoint{2.675959in}{2.437643in}}%
\pgfpathlineto{\pgfqpoint{2.675750in}{2.335545in}}%
\pgfpathlineto{\pgfqpoint{2.676611in}{2.367034in}}%
\pgfpathlineto{\pgfqpoint{2.676993in}{2.332565in}}%
\pgfpathlineto{\pgfqpoint{2.677190in}{2.434252in}}%
\pgfpathlineto{\pgfqpoint{2.677695in}{2.377412in}}%
\pgfpathlineto{\pgfqpoint{2.678433in}{2.439717in}}%
\pgfpathlineto{\pgfqpoint{2.678224in}{2.332346in}}%
\pgfpathlineto{\pgfqpoint{2.678815in}{2.388826in}}%
\pgfpathlineto{\pgfqpoint{2.679676in}{2.437605in}}%
\pgfpathlineto{\pgfqpoint{2.679467in}{2.332563in}}%
\pgfpathlineto{\pgfqpoint{2.679848in}{2.345992in}}%
\pgfpathlineto{\pgfqpoint{2.680710in}{2.329045in}}%
\pgfpathlineto{\pgfqpoint{2.680501in}{2.414088in}}%
\pgfpathlineto{\pgfqpoint{2.680833in}{2.383911in}}%
\pgfpathlineto{\pgfqpoint{2.680907in}{2.436366in}}%
\pgfpathlineto{\pgfqpoint{2.681104in}{2.338140in}}%
\pgfpathlineto{\pgfqpoint{2.681891in}{2.361113in}}%
\pgfpathlineto{\pgfqpoint{2.681941in}{2.327303in}}%
\pgfpathlineto{\pgfqpoint{2.682138in}{2.433312in}}%
\pgfpathlineto{\pgfqpoint{2.682950in}{2.398970in}}%
\pgfpathlineto{\pgfqpoint{2.683381in}{2.434759in}}%
\pgfpathlineto{\pgfqpoint{2.683184in}{2.325899in}}%
\pgfpathlineto{\pgfqpoint{2.683984in}{2.368935in}}%
\pgfpathlineto{\pgfqpoint{2.684415in}{2.326284in}}%
\pgfpathlineto{\pgfqpoint{2.684611in}{2.433870in}}%
\pgfpathlineto{\pgfqpoint{2.685005in}{2.397703in}}%
\pgfpathlineto{\pgfqpoint{2.685855in}{2.437697in}}%
\pgfpathlineto{\pgfqpoint{2.685658in}{2.327156in}}%
\pgfpathlineto{\pgfqpoint{2.685916in}{2.363011in}}%
\pgfpathlineto{\pgfqpoint{2.686888in}{2.330256in}}%
\pgfpathlineto{\pgfqpoint{2.686679in}{2.416585in}}%
\pgfpathlineto{\pgfqpoint{2.686975in}{2.381000in}}%
\pgfpathlineto{\pgfqpoint{2.687085in}{2.435455in}}%
\pgfpathlineto{\pgfqpoint{2.687307in}{2.340799in}}%
\pgfpathlineto{\pgfqpoint{2.687959in}{2.379606in}}%
\pgfpathlineto{\pgfqpoint{2.688131in}{2.328424in}}%
\pgfpathlineto{\pgfqpoint{2.688328in}{2.436568in}}%
\pgfpathlineto{\pgfqpoint{2.689079in}{2.365992in}}%
\pgfpathlineto{\pgfqpoint{2.689571in}{2.433282in}}%
\pgfpathlineto{\pgfqpoint{2.689375in}{2.332773in}}%
\pgfpathlineto{\pgfqpoint{2.690175in}{2.367687in}}%
\pgfpathlineto{\pgfqpoint{2.690605in}{2.334005in}}%
\pgfpathlineto{\pgfqpoint{2.690802in}{2.433063in}}%
\pgfpathlineto{\pgfqpoint{2.691196in}{2.402005in}}%
\pgfpathlineto{\pgfqpoint{2.692045in}{2.432270in}}%
\pgfpathlineto{\pgfqpoint{2.691848in}{2.335438in}}%
\pgfpathlineto{\pgfqpoint{2.692181in}{2.370462in}}%
\pgfpathlineto{\pgfqpoint{2.693079in}{2.337692in}}%
\pgfpathlineto{\pgfqpoint{2.692870in}{2.418852in}}%
\pgfpathlineto{\pgfqpoint{2.693227in}{2.378570in}}%
\pgfpathlineto{\pgfqpoint{2.693276in}{2.433125in}}%
\pgfpathlineto{\pgfqpoint{2.693498in}{2.345031in}}%
\pgfpathlineto{\pgfqpoint{2.694298in}{2.348716in}}%
\pgfpathlineto{\pgfqpoint{2.694322in}{2.340729in}}%
\pgfpathlineto{\pgfqpoint{2.694519in}{2.435468in}}%
\pgfpathlineto{\pgfqpoint{2.695233in}{2.387942in}}%
\pgfpathlineto{\pgfqpoint{2.695750in}{2.434826in}}%
\pgfpathlineto{\pgfqpoint{2.696008in}{2.342003in}}%
\pgfpathlineto{\pgfqpoint{2.696205in}{2.383694in}}%
\pgfpathlineto{\pgfqpoint{2.697251in}{2.338818in}}%
\pgfpathlineto{\pgfqpoint{2.696993in}{2.434825in}}%
\pgfpathlineto{\pgfqpoint{2.697325in}{2.366486in}}%
\pgfpathlineto{\pgfqpoint{2.698224in}{2.434661in}}%
\pgfpathlineto{\pgfqpoint{2.697658in}{2.339800in}}%
\pgfpathlineto{\pgfqpoint{2.698384in}{2.358893in}}%
\pgfpathlineto{\pgfqpoint{2.698901in}{2.337210in}}%
\pgfpathlineto{\pgfqpoint{2.698642in}{2.421034in}}%
\pgfpathlineto{\pgfqpoint{2.699430in}{2.394400in}}%
\pgfpathlineto{\pgfqpoint{2.699467in}{2.433279in}}%
\pgfpathlineto{\pgfqpoint{2.699725in}{2.336459in}}%
\pgfpathlineto{\pgfqpoint{2.700513in}{2.343627in}}%
\pgfpathlineto{\pgfqpoint{2.700698in}{2.434428in}}%
\pgfpathlineto{\pgfqpoint{2.700956in}{2.339312in}}%
\pgfpathlineto{\pgfqpoint{2.701658in}{2.377958in}}%
\pgfpathlineto{\pgfqpoint{2.702199in}{2.339154in}}%
\pgfpathlineto{\pgfqpoint{2.701941in}{2.432944in}}%
\pgfpathlineto{\pgfqpoint{2.702741in}{2.403299in}}%
\pgfpathlineto{\pgfqpoint{2.703171in}{2.431568in}}%
\pgfpathlineto{\pgfqpoint{2.703442in}{2.341017in}}%
\pgfpathlineto{\pgfqpoint{2.703799in}{2.358418in}}%
\pgfpathlineto{\pgfqpoint{2.704636in}{2.339834in}}%
\pgfpathlineto{\pgfqpoint{2.704414in}{2.432107in}}%
\pgfpathlineto{\pgfqpoint{2.704833in}{2.422415in}}%
\pgfpathlineto{\pgfqpoint{2.705879in}{2.339823in}}%
\pgfpathlineto{\pgfqpoint{2.705658in}{2.429377in}}%
\pgfpathlineto{\pgfqpoint{2.706027in}{2.379731in}}%
\pgfpathlineto{\pgfqpoint{2.706888in}{2.425384in}}%
\pgfpathlineto{\pgfqpoint{2.707110in}{2.336032in}}%
\pgfpathlineto{\pgfqpoint{2.708131in}{2.425508in}}%
\pgfpathlineto{\pgfqpoint{2.708267in}{2.362967in}}%
\pgfpathlineto{\pgfqpoint{2.708353in}{2.338891in}}%
\pgfpathlineto{\pgfqpoint{2.708550in}{2.419182in}}%
\pgfpathlineto{\pgfqpoint{2.709239in}{2.379086in}}%
\pgfpathlineto{\pgfqpoint{2.709362in}{2.427595in}}%
\pgfpathlineto{\pgfqpoint{2.709584in}{2.339973in}}%
\pgfpathlineto{\pgfqpoint{2.710322in}{2.378196in}}%
\pgfpathlineto{\pgfqpoint{2.710827in}{2.339581in}}%
\pgfpathlineto{\pgfqpoint{2.710605in}{2.426925in}}%
\pgfpathlineto{\pgfqpoint{2.711405in}{2.404256in}}%
\pgfpathlineto{\pgfqpoint{2.711848in}{2.426946in}}%
\pgfpathlineto{\pgfqpoint{2.712094in}{2.339617in}}%
\pgfpathlineto{\pgfqpoint{2.712439in}{2.369029in}}%
\pgfpathlineto{\pgfqpoint{2.713337in}{2.338519in}}%
\pgfpathlineto{\pgfqpoint{2.713079in}{2.427270in}}%
\pgfpathlineto{\pgfqpoint{2.713461in}{2.398637in}}%
\pgfpathlineto{\pgfqpoint{2.714322in}{2.424677in}}%
\pgfpathlineto{\pgfqpoint{2.713744in}{2.340352in}}%
\pgfpathlineto{\pgfqpoint{2.714507in}{2.349954in}}%
\pgfpathlineto{\pgfqpoint{2.714568in}{2.338360in}}%
\pgfpathlineto{\pgfqpoint{2.714728in}{2.419120in}}%
\pgfpathlineto{\pgfqpoint{2.715417in}{2.364648in}}%
\pgfpathlineto{\pgfqpoint{2.715553in}{2.426600in}}%
\pgfpathlineto{\pgfqpoint{2.715811in}{2.338299in}}%
\pgfpathlineto{\pgfqpoint{2.716525in}{2.365471in}}%
\pgfpathlineto{\pgfqpoint{2.717042in}{2.341120in}}%
\pgfpathlineto{\pgfqpoint{2.716796in}{2.425217in}}%
\pgfpathlineto{\pgfqpoint{2.717584in}{2.395619in}}%
\pgfpathlineto{\pgfqpoint{2.718027in}{2.425295in}}%
\pgfpathlineto{\pgfqpoint{2.718285in}{2.338984in}}%
\pgfpathlineto{\pgfqpoint{2.718642in}{2.355849in}}%
\pgfpathlineto{\pgfqpoint{2.718704in}{2.343529in}}%
\pgfpathlineto{\pgfqpoint{2.718851in}{2.414180in}}%
\pgfpathlineto{\pgfqpoint{2.719233in}{2.403279in}}%
\pgfpathlineto{\pgfqpoint{2.719270in}{2.424808in}}%
\pgfpathlineto{\pgfqpoint{2.719528in}{2.341939in}}%
\pgfpathlineto{\pgfqpoint{2.720291in}{2.349563in}}%
\pgfpathlineto{\pgfqpoint{2.720501in}{2.424598in}}%
\pgfpathlineto{\pgfqpoint{2.720759in}{2.340675in}}%
\pgfpathlineto{\pgfqpoint{2.721485in}{2.360484in}}%
\pgfpathlineto{\pgfqpoint{2.722002in}{2.340602in}}%
\pgfpathlineto{\pgfqpoint{2.721744in}{2.423796in}}%
\pgfpathlineto{\pgfqpoint{2.722544in}{2.406801in}}%
\pgfpathlineto{\pgfqpoint{2.722974in}{2.425448in}}%
\pgfpathlineto{\pgfqpoint{2.723233in}{2.342966in}}%
\pgfpathlineto{\pgfqpoint{2.723541in}{2.374913in}}%
\pgfpathlineto{\pgfqpoint{2.724476in}{2.339712in}}%
\pgfpathlineto{\pgfqpoint{2.724217in}{2.423728in}}%
\pgfpathlineto{\pgfqpoint{2.724611in}{2.411148in}}%
\pgfpathlineto{\pgfqpoint{2.725448in}{2.422887in}}%
\pgfpathlineto{\pgfqpoint{2.725301in}{2.342378in}}%
\pgfpathlineto{\pgfqpoint{2.725596in}{2.367212in}}%
\pgfpathlineto{\pgfqpoint{2.725657in}{2.341664in}}%
\pgfpathlineto{\pgfqpoint{2.725867in}{2.417501in}}%
\pgfpathlineto{\pgfqpoint{2.726654in}{2.390269in}}%
\pgfpathlineto{\pgfqpoint{2.726691in}{2.424505in}}%
\pgfpathlineto{\pgfqpoint{2.726950in}{2.340860in}}%
\pgfpathlineto{\pgfqpoint{2.727737in}{2.347531in}}%
\pgfpathlineto{\pgfqpoint{2.727922in}{2.424246in}}%
\pgfpathlineto{\pgfqpoint{2.728144in}{2.341748in}}%
\pgfpathlineto{\pgfqpoint{2.728882in}{2.383049in}}%
\pgfpathlineto{\pgfqpoint{2.729424in}{2.340579in}}%
\pgfpathlineto{\pgfqpoint{2.729165in}{2.423053in}}%
\pgfpathlineto{\pgfqpoint{2.729965in}{2.407055in}}%
\pgfpathlineto{\pgfqpoint{2.730396in}{2.423542in}}%
\pgfpathlineto{\pgfqpoint{2.730617in}{2.343967in}}%
\pgfpathlineto{\pgfqpoint{2.730974in}{2.363427in}}%
\pgfpathlineto{\pgfqpoint{2.731897in}{2.342877in}}%
\pgfpathlineto{\pgfqpoint{2.731627in}{2.421264in}}%
\pgfpathlineto{\pgfqpoint{2.732008in}{2.378502in}}%
\pgfpathlineto{\pgfqpoint{2.732857in}{2.420261in}}%
\pgfpathlineto{\pgfqpoint{2.733079in}{2.341148in}}%
\pgfpathlineto{\pgfqpoint{2.733104in}{2.350095in}}%
\pgfpathlineto{\pgfqpoint{2.734100in}{2.421151in}}%
\pgfpathlineto{\pgfqpoint{2.733128in}{2.343205in}}%
\pgfpathlineto{\pgfqpoint{2.734260in}{2.362798in}}%
\pgfpathlineto{\pgfqpoint{2.734322in}{2.341348in}}%
\pgfpathlineto{\pgfqpoint{2.734507in}{2.417455in}}%
\pgfpathlineto{\pgfqpoint{2.735307in}{2.393006in}}%
\pgfpathlineto{\pgfqpoint{2.735331in}{2.420756in}}%
\pgfpathlineto{\pgfqpoint{2.735553in}{2.339684in}}%
\pgfpathlineto{\pgfqpoint{2.736390in}{2.344132in}}%
\pgfpathlineto{\pgfqpoint{2.736562in}{2.420749in}}%
\pgfpathlineto{\pgfqpoint{2.736796in}{2.337902in}}%
\pgfpathlineto{\pgfqpoint{2.737534in}{2.386740in}}%
\pgfpathlineto{\pgfqpoint{2.738027in}{2.338510in}}%
\pgfpathlineto{\pgfqpoint{2.737805in}{2.420461in}}%
\pgfpathlineto{\pgfqpoint{2.738605in}{2.398274in}}%
\pgfpathlineto{\pgfqpoint{2.739036in}{2.423231in}}%
\pgfpathlineto{\pgfqpoint{2.739257in}{2.339568in}}%
\pgfpathlineto{\pgfqpoint{2.739676in}{2.344697in}}%
\pgfpathlineto{\pgfqpoint{2.740267in}{2.424585in}}%
\pgfpathlineto{\pgfqpoint{2.740500in}{2.338427in}}%
\pgfpathlineto{\pgfqpoint{2.740845in}{2.376164in}}%
\pgfpathlineto{\pgfqpoint{2.741731in}{2.335874in}}%
\pgfpathlineto{\pgfqpoint{2.741510in}{2.429700in}}%
\pgfpathlineto{\pgfqpoint{2.741904in}{2.418254in}}%
\pgfpathlineto{\pgfqpoint{2.742740in}{2.432337in}}%
\pgfpathlineto{\pgfqpoint{2.742556in}{2.338035in}}%
\pgfpathlineto{\pgfqpoint{2.742925in}{2.351891in}}%
\pgfpathlineto{\pgfqpoint{2.742937in}{2.351789in}}%
\pgfpathlineto{\pgfqpoint{2.742974in}{2.333553in}}%
\pgfpathlineto{\pgfqpoint{2.743565in}{2.431856in}}%
\pgfpathlineto{\pgfqpoint{2.743959in}{2.419233in}}%
\pgfpathlineto{\pgfqpoint{2.744390in}{2.439629in}}%
\pgfpathlineto{\pgfqpoint{2.744205in}{2.335216in}}%
\pgfpathlineto{\pgfqpoint{2.745017in}{2.340495in}}%
\pgfpathlineto{\pgfqpoint{2.745473in}{2.336404in}}%
\pgfpathlineto{\pgfqpoint{2.745214in}{2.445697in}}%
\pgfpathlineto{\pgfqpoint{2.745596in}{2.398715in}}%
\pgfpathlineto{\pgfqpoint{2.746445in}{2.452255in}}%
\pgfpathlineto{\pgfqpoint{2.746679in}{2.335502in}}%
\pgfpathlineto{\pgfqpoint{2.747676in}{2.454747in}}%
\pgfpathlineto{\pgfqpoint{2.746704in}{2.333362in}}%
\pgfpathlineto{\pgfqpoint{2.747873in}{2.355866in}}%
\pgfpathlineto{\pgfqpoint{2.747934in}{2.331413in}}%
\pgfpathlineto{\pgfqpoint{2.748094in}{2.454169in}}%
\pgfpathlineto{\pgfqpoint{2.748894in}{2.423938in}}%
\pgfpathlineto{\pgfqpoint{2.749325in}{2.457172in}}%
\pgfpathlineto{\pgfqpoint{2.749177in}{2.328583in}}%
\pgfpathlineto{\pgfqpoint{2.749953in}{2.339572in}}%
\pgfpathlineto{\pgfqpoint{2.750408in}{2.324539in}}%
\pgfpathlineto{\pgfqpoint{2.750150in}{2.459285in}}%
\pgfpathlineto{\pgfqpoint{2.750937in}{2.373697in}}%
\pgfpathlineto{\pgfqpoint{2.751380in}{2.458875in}}%
\pgfpathlineto{\pgfqpoint{2.751639in}{2.325257in}}%
\pgfpathlineto{\pgfqpoint{2.752033in}{2.334972in}}%
\pgfpathlineto{\pgfqpoint{2.752870in}{2.325083in}}%
\pgfpathlineto{\pgfqpoint{2.752611in}{2.459741in}}%
\pgfpathlineto{\pgfqpoint{2.752993in}{2.377302in}}%
\pgfpathlineto{\pgfqpoint{2.753030in}{2.458650in}}%
\pgfpathlineto{\pgfqpoint{2.753694in}{2.326672in}}%
\pgfpathlineto{\pgfqpoint{2.754088in}{2.331105in}}%
\pgfpathlineto{\pgfqpoint{2.754100in}{2.327541in}}%
\pgfpathlineto{\pgfqpoint{2.754260in}{2.461069in}}%
\pgfpathlineto{\pgfqpoint{2.755036in}{2.346003in}}%
\pgfpathlineto{\pgfqpoint{2.755491in}{2.461943in}}%
\pgfpathlineto{\pgfqpoint{2.755343in}{2.328794in}}%
\pgfpathlineto{\pgfqpoint{2.756131in}{2.335878in}}%
\pgfpathlineto{\pgfqpoint{2.756574in}{2.325355in}}%
\pgfpathlineto{\pgfqpoint{2.756316in}{2.460079in}}%
\pgfpathlineto{\pgfqpoint{2.756697in}{2.399956in}}%
\pgfpathlineto{\pgfqpoint{2.756722in}{2.458790in}}%
\pgfpathlineto{\pgfqpoint{2.757399in}{2.325911in}}%
\pgfpathlineto{\pgfqpoint{2.757780in}{2.328428in}}%
\pgfpathlineto{\pgfqpoint{2.757854in}{2.389235in}}%
\pgfpathlineto{\pgfqpoint{2.757805in}{2.323523in}}%
\pgfpathlineto{\pgfqpoint{2.757928in}{2.385156in}}%
\pgfpathlineto{\pgfqpoint{2.758371in}{2.457276in}}%
\pgfpathlineto{\pgfqpoint{2.758630in}{2.322942in}}%
\pgfpathlineto{\pgfqpoint{2.759011in}{2.326908in}}%
\pgfpathlineto{\pgfqpoint{2.759023in}{2.326904in}}%
\pgfpathlineto{\pgfqpoint{2.759036in}{2.324243in}}%
\pgfpathlineto{\pgfqpoint{2.759196in}{2.457395in}}%
\pgfpathlineto{\pgfqpoint{2.759590in}{2.441893in}}%
\pgfpathlineto{\pgfqpoint{2.760427in}{2.459895in}}%
\pgfpathlineto{\pgfqpoint{2.759860in}{2.326482in}}%
\pgfpathlineto{\pgfqpoint{2.760623in}{2.373689in}}%
\pgfpathlineto{\pgfqpoint{2.761596in}{2.326166in}}%
\pgfpathlineto{\pgfqpoint{2.761657in}{2.458336in}}%
\pgfpathlineto{\pgfqpoint{2.761731in}{2.355317in}}%
\pgfpathlineto{\pgfqpoint{2.762076in}{2.453535in}}%
\pgfpathlineto{\pgfqpoint{2.762716in}{2.327606in}}%
\pgfpathlineto{\pgfqpoint{2.762814in}{2.355679in}}%
\pgfpathlineto{\pgfqpoint{2.763245in}{2.322811in}}%
\pgfpathlineto{\pgfqpoint{2.763307in}{2.457544in}}%
\pgfpathlineto{\pgfqpoint{2.763922in}{2.361868in}}%
\pgfpathlineto{\pgfqpoint{2.764070in}{2.318249in}}%
\pgfpathlineto{\pgfqpoint{2.764537in}{2.458720in}}%
\pgfpathlineto{\pgfqpoint{2.765030in}{2.365508in}}%
\pgfpathlineto{\pgfqpoint{2.765768in}{2.462596in}}%
\pgfpathlineto{\pgfqpoint{2.765300in}{2.317061in}}%
\pgfpathlineto{\pgfqpoint{2.766113in}{2.331838in}}%
\pgfpathlineto{\pgfqpoint{2.766531in}{2.317801in}}%
\pgfpathlineto{\pgfqpoint{2.766593in}{2.461932in}}%
\pgfpathlineto{\pgfqpoint{2.767183in}{2.367029in}}%
\pgfpathlineto{\pgfqpoint{2.768242in}{2.464666in}}%
\pgfpathlineto{\pgfqpoint{2.768180in}{2.316635in}}%
\pgfpathlineto{\pgfqpoint{2.768279in}{2.368691in}}%
\pgfpathlineto{\pgfqpoint{2.769005in}{2.315595in}}%
\pgfpathlineto{\pgfqpoint{2.769054in}{2.463066in}}%
\pgfpathlineto{\pgfqpoint{2.769362in}{2.409186in}}%
\pgfpathlineto{\pgfqpoint{2.770236in}{2.312103in}}%
\pgfpathlineto{\pgfqpoint{2.769473in}{2.467666in}}%
\pgfpathlineto{\pgfqpoint{2.770556in}{2.322142in}}%
\pgfpathlineto{\pgfqpoint{2.770703in}{2.469672in}}%
\pgfpathlineto{\pgfqpoint{2.771467in}{2.313083in}}%
\pgfpathlineto{\pgfqpoint{2.771676in}{2.363944in}}%
\pgfpathlineto{\pgfqpoint{2.771885in}{2.310630in}}%
\pgfpathlineto{\pgfqpoint{2.771823in}{2.413236in}}%
\pgfpathlineto{\pgfqpoint{2.771910in}{2.400113in}}%
\pgfpathlineto{\pgfqpoint{2.771934in}{2.472945in}}%
\pgfpathlineto{\pgfqpoint{2.772697in}{2.309560in}}%
\pgfpathlineto{\pgfqpoint{2.772993in}{2.331416in}}%
\pgfpathlineto{\pgfqpoint{2.773116in}{2.307755in}}%
\pgfpathlineto{\pgfqpoint{2.773165in}{2.473303in}}%
\pgfpathlineto{\pgfqpoint{2.773953in}{2.340975in}}%
\pgfpathlineto{\pgfqpoint{2.774396in}{2.476754in}}%
\pgfpathlineto{\pgfqpoint{2.774346in}{2.306628in}}%
\pgfpathlineto{\pgfqpoint{2.775060in}{2.325011in}}%
\pgfpathlineto{\pgfqpoint{2.775577in}{2.302681in}}%
\pgfpathlineto{\pgfqpoint{2.775220in}{2.477244in}}%
\pgfpathlineto{\pgfqpoint{2.776020in}{2.412316in}}%
\pgfpathlineto{\pgfqpoint{2.776870in}{2.479739in}}%
\pgfpathlineto{\pgfqpoint{2.776402in}{2.301381in}}%
\pgfpathlineto{\pgfqpoint{2.777103in}{2.324213in}}%
\pgfpathlineto{\pgfqpoint{2.777226in}{2.303453in}}%
\pgfpathlineto{\pgfqpoint{2.777177in}{2.416862in}}%
\pgfpathlineto{\pgfqpoint{2.777251in}{2.397503in}}%
\pgfpathlineto{\pgfqpoint{2.778100in}{2.486579in}}%
\pgfpathlineto{\pgfqpoint{2.777633in}{2.303183in}}%
\pgfpathlineto{\pgfqpoint{2.778346in}{2.322686in}}%
\pgfpathlineto{\pgfqpoint{2.778457in}{2.303762in}}%
\pgfpathlineto{\pgfqpoint{2.778506in}{2.486091in}}%
\pgfpathlineto{\pgfqpoint{2.778900in}{2.421165in}}%
\pgfpathlineto{\pgfqpoint{2.779750in}{2.488066in}}%
\pgfpathlineto{\pgfqpoint{2.779688in}{2.307816in}}%
\pgfpathlineto{\pgfqpoint{2.779983in}{2.321974in}}%
\pgfpathlineto{\pgfqpoint{2.780008in}{2.315780in}}%
\pgfpathlineto{\pgfqpoint{2.780980in}{2.493234in}}%
\pgfpathlineto{\pgfqpoint{2.780513in}{2.300930in}}%
\pgfpathlineto{\pgfqpoint{2.781140in}{2.351765in}}%
\pgfpathlineto{\pgfqpoint{2.781239in}{2.317115in}}%
\pgfpathlineto{\pgfqpoint{2.781177in}{2.374876in}}%
\pgfpathlineto{\pgfqpoint{2.781263in}{2.369338in}}%
\pgfpathlineto{\pgfqpoint{2.782211in}{2.502030in}}%
\pgfpathlineto{\pgfqpoint{2.782162in}{2.298713in}}%
\pgfpathlineto{\pgfqpoint{2.782359in}{2.357685in}}%
\pgfpathlineto{\pgfqpoint{2.782986in}{2.295624in}}%
\pgfpathlineto{\pgfqpoint{2.783036in}{2.506178in}}%
\pgfpathlineto{\pgfqpoint{2.783417in}{2.379916in}}%
\pgfpathlineto{\pgfqpoint{2.784266in}{2.510400in}}%
\pgfpathlineto{\pgfqpoint{2.784217in}{2.292201in}}%
\pgfpathlineto{\pgfqpoint{2.784513in}{2.310074in}}%
\pgfpathlineto{\pgfqpoint{2.785091in}{2.516506in}}%
\pgfpathlineto{\pgfqpoint{2.785448in}{2.292770in}}%
\pgfpathlineto{\pgfqpoint{2.785706in}{2.359729in}}%
\pgfpathlineto{\pgfqpoint{2.786691in}{2.289715in}}%
\pgfpathlineto{\pgfqpoint{2.786740in}{2.523347in}}%
\pgfpathlineto{\pgfqpoint{2.786814in}{2.347610in}}%
\pgfpathlineto{\pgfqpoint{2.787146in}{2.527288in}}%
\pgfpathlineto{\pgfqpoint{2.787097in}{2.285693in}}%
\pgfpathlineto{\pgfqpoint{2.787897in}{2.345328in}}%
\pgfpathlineto{\pgfqpoint{2.787922in}{2.280782in}}%
\pgfpathlineto{\pgfqpoint{2.788796in}{2.536159in}}%
\pgfpathlineto{\pgfqpoint{2.789005in}{2.335250in}}%
\pgfpathlineto{\pgfqpoint{2.789977in}{2.277114in}}%
\pgfpathlineto{\pgfqpoint{2.789620in}{2.544762in}}%
\pgfpathlineto{\pgfqpoint{2.790002in}{2.437044in}}%
\pgfpathlineto{\pgfqpoint{2.790851in}{2.556369in}}%
\pgfpathlineto{\pgfqpoint{2.790383in}{2.273756in}}%
\pgfpathlineto{\pgfqpoint{2.791085in}{2.300512in}}%
\pgfpathlineto{\pgfqpoint{2.791208in}{2.272288in}}%
\pgfpathlineto{\pgfqpoint{2.791245in}{2.518238in}}%
\pgfpathlineto{\pgfqpoint{2.792082in}{2.566132in}}%
\pgfpathlineto{\pgfqpoint{2.792033in}{2.269165in}}%
\pgfpathlineto{\pgfqpoint{2.792291in}{2.336984in}}%
\pgfpathlineto{\pgfqpoint{2.793263in}{2.263082in}}%
\pgfpathlineto{\pgfqpoint{2.792906in}{2.569136in}}%
\pgfpathlineto{\pgfqpoint{2.793288in}{2.442619in}}%
\pgfpathlineto{\pgfqpoint{2.794137in}{2.581591in}}%
\pgfpathlineto{\pgfqpoint{2.794088in}{2.262630in}}%
\pgfpathlineto{\pgfqpoint{2.794371in}{2.289454in}}%
\pgfpathlineto{\pgfqpoint{2.794900in}{2.254880in}}%
\pgfpathlineto{\pgfqpoint{2.794543in}{2.586685in}}%
\pgfpathlineto{\pgfqpoint{2.794937in}{2.513010in}}%
\pgfpathlineto{\pgfqpoint{2.795774in}{2.591991in}}%
\pgfpathlineto{\pgfqpoint{2.795725in}{2.250415in}}%
\pgfpathlineto{\pgfqpoint{2.796008in}{2.303207in}}%
\pgfpathlineto{\pgfqpoint{2.796956in}{2.243618in}}%
\pgfpathlineto{\pgfqpoint{2.796599in}{2.595365in}}%
\pgfpathlineto{\pgfqpoint{2.796993in}{2.528491in}}%
\pgfpathlineto{\pgfqpoint{2.797423in}{2.597152in}}%
\pgfpathlineto{\pgfqpoint{2.797780in}{2.240403in}}%
\pgfpathlineto{\pgfqpoint{2.798051in}{2.334599in}}%
\pgfpathlineto{\pgfqpoint{2.799011in}{2.238324in}}%
\pgfpathlineto{\pgfqpoint{2.798248in}{2.587839in}}%
\pgfpathlineto{\pgfqpoint{2.799146in}{2.374981in}}%
\pgfpathlineto{\pgfqpoint{2.799479in}{2.572504in}}%
\pgfpathlineto{\pgfqpoint{2.799417in}{2.240879in}}%
\pgfpathlineto{\pgfqpoint{2.800217in}{2.345570in}}%
\pgfpathlineto{\pgfqpoint{2.800242in}{2.238070in}}%
\pgfpathlineto{\pgfqpoint{2.800291in}{2.555713in}}%
\pgfpathlineto{\pgfqpoint{2.801313in}{2.413319in}}%
\pgfpathlineto{\pgfqpoint{2.801473in}{2.247527in}}%
\pgfpathlineto{\pgfqpoint{2.801522in}{2.520177in}}%
\pgfpathlineto{\pgfqpoint{2.802420in}{2.379156in}}%
\pgfpathlineto{\pgfqpoint{2.802753in}{2.478476in}}%
\pgfpathlineto{\pgfqpoint{2.802703in}{2.260617in}}%
\pgfpathlineto{\pgfqpoint{2.803503in}{2.320463in}}%
\pgfpathlineto{\pgfqpoint{2.803528in}{2.276561in}}%
\pgfpathlineto{\pgfqpoint{2.803577in}{2.442534in}}%
\pgfpathlineto{\pgfqpoint{2.804586in}{2.405282in}}%
\pgfpathlineto{\pgfqpoint{2.805313in}{2.451706in}}%
\pgfpathlineto{\pgfqpoint{2.805165in}{2.292680in}}%
\pgfpathlineto{\pgfqpoint{2.805645in}{2.376447in}}%
\pgfpathlineto{\pgfqpoint{2.805989in}{2.300838in}}%
\pgfpathlineto{\pgfqpoint{2.805731in}{2.455388in}}%
\pgfpathlineto{\pgfqpoint{2.806740in}{2.391461in}}%
\pgfpathlineto{\pgfqpoint{2.806765in}{2.428511in}}%
\pgfpathlineto{\pgfqpoint{2.806814in}{2.291234in}}%
\pgfpathlineto{\pgfqpoint{2.807848in}{2.390317in}}%
\pgfpathlineto{\pgfqpoint{2.808414in}{2.435706in}}%
\pgfpathlineto{\pgfqpoint{2.808057in}{2.310363in}}%
\pgfpathlineto{\pgfqpoint{2.808882in}{2.321956in}}%
\pgfpathlineto{\pgfqpoint{2.809645in}{2.428831in}}%
\pgfpathlineto{\pgfqpoint{2.809300in}{2.315549in}}%
\pgfpathlineto{\pgfqpoint{2.810100in}{2.364988in}}%
\pgfpathlineto{\pgfqpoint{2.811073in}{2.302177in}}%
\pgfpathlineto{\pgfqpoint{2.810888in}{2.426461in}}%
\pgfpathlineto{\pgfqpoint{2.811196in}{2.377732in}}%
\pgfpathlineto{\pgfqpoint{2.812131in}{2.436848in}}%
\pgfpathlineto{\pgfqpoint{2.811897in}{2.300687in}}%
\pgfpathlineto{\pgfqpoint{2.812279in}{2.374439in}}%
\pgfpathlineto{\pgfqpoint{2.813128in}{2.278557in}}%
\pgfpathlineto{\pgfqpoint{2.813362in}{2.444383in}}%
\pgfpathlineto{\pgfqpoint{2.814359in}{2.267469in}}%
\pgfpathlineto{\pgfqpoint{2.814568in}{2.397681in}}%
\pgfpathlineto{\pgfqpoint{2.814605in}{2.450909in}}%
\pgfpathlineto{\pgfqpoint{2.815183in}{2.260771in}}%
\pgfpathlineto{\pgfqpoint{2.815663in}{2.391204in}}%
\pgfpathlineto{\pgfqpoint{2.816414in}{2.259349in}}%
\pgfpathlineto{\pgfqpoint{2.816672in}{2.453848in}}%
\pgfpathlineto{\pgfqpoint{2.816759in}{2.385983in}}%
\pgfpathlineto{\pgfqpoint{2.817079in}{2.457385in}}%
\pgfpathlineto{\pgfqpoint{2.817239in}{2.256088in}}%
\pgfpathlineto{\pgfqpoint{2.817903in}{2.447354in}}%
\pgfpathlineto{\pgfqpoint{2.818063in}{2.267498in}}%
\pgfpathlineto{\pgfqpoint{2.818297in}{2.452798in}}%
\pgfpathlineto{\pgfqpoint{2.819097in}{2.424196in}}%
\pgfpathlineto{\pgfqpoint{2.819528in}{2.459040in}}%
\pgfpathlineto{\pgfqpoint{2.819294in}{2.270267in}}%
\pgfpathlineto{\pgfqpoint{2.820106in}{2.270536in}}%
\pgfpathlineto{\pgfqpoint{2.820525in}{2.259110in}}%
\pgfpathlineto{\pgfqpoint{2.820759in}{2.460926in}}%
\pgfpathlineto{\pgfqpoint{2.821066in}{2.352029in}}%
\pgfpathlineto{\pgfqpoint{2.822002in}{2.474131in}}%
\pgfpathlineto{\pgfqpoint{2.821349in}{2.247022in}}%
\pgfpathlineto{\pgfqpoint{2.822149in}{2.326212in}}%
\pgfpathlineto{\pgfqpoint{2.822174in}{2.261998in}}%
\pgfpathlineto{\pgfqpoint{2.822408in}{2.468074in}}%
\pgfpathlineto{\pgfqpoint{2.823220in}{2.433394in}}%
\pgfpathlineto{\pgfqpoint{2.823232in}{2.462733in}}%
\pgfpathlineto{\pgfqpoint{2.823405in}{2.286783in}}%
\pgfpathlineto{\pgfqpoint{2.824303in}{2.371881in}}%
\pgfpathlineto{\pgfqpoint{2.825300in}{2.451724in}}%
\pgfpathlineto{\pgfqpoint{2.825054in}{2.295954in}}%
\pgfpathlineto{\pgfqpoint{2.825423in}{2.391823in}}%
\pgfpathlineto{\pgfqpoint{2.825879in}{2.293839in}}%
\pgfpathlineto{\pgfqpoint{2.826125in}{2.453764in}}%
\pgfpathlineto{\pgfqpoint{2.826506in}{2.399552in}}%
\pgfpathlineto{\pgfqpoint{2.827356in}{2.450537in}}%
\pgfpathlineto{\pgfqpoint{2.826703in}{2.298098in}}%
\pgfpathlineto{\pgfqpoint{2.827589in}{2.345497in}}%
\pgfpathlineto{\pgfqpoint{2.827774in}{2.451827in}}%
\pgfpathlineto{\pgfqpoint{2.827946in}{2.310753in}}%
\pgfpathlineto{\pgfqpoint{2.828722in}{2.387669in}}%
\pgfpathlineto{\pgfqpoint{2.828771in}{2.306616in}}%
\pgfpathlineto{\pgfqpoint{2.829423in}{2.442204in}}%
\pgfpathlineto{\pgfqpoint{2.829805in}{2.404133in}}%
\pgfpathlineto{\pgfqpoint{2.830248in}{2.444051in}}%
\pgfpathlineto{\pgfqpoint{2.830826in}{2.302095in}}%
\pgfpathlineto{\pgfqpoint{2.830851in}{2.327168in}}%
\pgfpathlineto{\pgfqpoint{2.831479in}{2.448679in}}%
\pgfpathlineto{\pgfqpoint{2.831651in}{2.293486in}}%
\pgfpathlineto{\pgfqpoint{2.831971in}{2.338821in}}%
\pgfpathlineto{\pgfqpoint{2.832303in}{2.448374in}}%
\pgfpathlineto{\pgfqpoint{2.832476in}{2.285551in}}%
\pgfpathlineto{\pgfqpoint{2.832869in}{2.319070in}}%
\pgfpathlineto{\pgfqpoint{2.833706in}{2.277406in}}%
\pgfpathlineto{\pgfqpoint{2.833534in}{2.440406in}}%
\pgfpathlineto{\pgfqpoint{2.833928in}{2.425218in}}%
\pgfpathlineto{\pgfqpoint{2.833952in}{2.436813in}}%
\pgfpathlineto{\pgfqpoint{2.834531in}{2.273196in}}%
\pgfpathlineto{\pgfqpoint{2.834912in}{2.368924in}}%
\pgfpathlineto{\pgfqpoint{2.835762in}{2.261033in}}%
\pgfpathlineto{\pgfqpoint{2.835589in}{2.436930in}}%
\pgfpathlineto{\pgfqpoint{2.836008in}{2.435136in}}%
\pgfpathlineto{\pgfqpoint{2.836992in}{2.249651in}}%
\pgfpathlineto{\pgfqpoint{2.836857in}{2.445855in}}%
\pgfpathlineto{\pgfqpoint{2.837152in}{2.405703in}}%
\pgfpathlineto{\pgfqpoint{2.838088in}{2.457737in}}%
\pgfpathlineto{\pgfqpoint{2.837817in}{2.236828in}}%
\pgfpathlineto{\pgfqpoint{2.838199in}{2.362872in}}%
\pgfpathlineto{\pgfqpoint{2.839048in}{2.227787in}}%
\pgfpathlineto{\pgfqpoint{2.838494in}{2.467143in}}%
\pgfpathlineto{\pgfqpoint{2.839294in}{2.457021in}}%
\pgfpathlineto{\pgfqpoint{2.839725in}{2.480022in}}%
\pgfpathlineto{\pgfqpoint{2.839454in}{2.230301in}}%
\pgfpathlineto{\pgfqpoint{2.839860in}{2.242478in}}%
\pgfpathlineto{\pgfqpoint{2.840279in}{2.215690in}}%
\pgfpathlineto{\pgfqpoint{2.840549in}{2.491565in}}%
\pgfpathlineto{\pgfqpoint{2.840845in}{2.422970in}}%
\pgfpathlineto{\pgfqpoint{2.841780in}{2.509178in}}%
\pgfpathlineto{\pgfqpoint{2.841509in}{2.208551in}}%
\pgfpathlineto{\pgfqpoint{2.841891in}{2.397094in}}%
\pgfpathlineto{\pgfqpoint{2.842740in}{2.198061in}}%
\pgfpathlineto{\pgfqpoint{2.842186in}{2.522076in}}%
\pgfpathlineto{\pgfqpoint{2.842986in}{2.483470in}}%
\pgfpathlineto{\pgfqpoint{2.843417in}{2.536003in}}%
\pgfpathlineto{\pgfqpoint{2.843565in}{2.191751in}}%
\pgfpathlineto{\pgfqpoint{2.843959in}{2.243832in}}%
\pgfpathlineto{\pgfqpoint{2.844795in}{2.169856in}}%
\pgfpathlineto{\pgfqpoint{2.844648in}{2.549270in}}%
\pgfpathlineto{\pgfqpoint{2.845017in}{2.424065in}}%
\pgfpathlineto{\pgfqpoint{2.845472in}{2.550517in}}%
\pgfpathlineto{\pgfqpoint{2.846026in}{2.155704in}}%
\pgfpathlineto{\pgfqpoint{2.846112in}{2.370972in}}%
\pgfpathlineto{\pgfqpoint{2.846432in}{2.156009in}}%
\pgfpathlineto{\pgfqpoint{2.847109in}{2.547642in}}%
\pgfpathlineto{\pgfqpoint{2.847195in}{2.425970in}}%
\pgfpathlineto{\pgfqpoint{2.847934in}{2.550072in}}%
\pgfpathlineto{\pgfqpoint{2.847257in}{2.147690in}}%
\pgfpathlineto{\pgfqpoint{2.848315in}{2.481421in}}%
\pgfpathlineto{\pgfqpoint{2.849165in}{2.560229in}}%
\pgfpathlineto{\pgfqpoint{2.848488in}{2.146920in}}%
\pgfpathlineto{\pgfqpoint{2.849288in}{2.321602in}}%
\pgfpathlineto{\pgfqpoint{2.849312in}{2.140204in}}%
\pgfpathlineto{\pgfqpoint{2.849989in}{2.559919in}}%
\pgfpathlineto{\pgfqpoint{2.850371in}{2.497266in}}%
\pgfpathlineto{\pgfqpoint{2.851220in}{2.571884in}}%
\pgfpathlineto{\pgfqpoint{2.850543in}{2.137152in}}%
\pgfpathlineto{\pgfqpoint{2.851343in}{2.308290in}}%
\pgfpathlineto{\pgfqpoint{2.851774in}{2.135681in}}%
\pgfpathlineto{\pgfqpoint{2.851626in}{2.569082in}}%
\pgfpathlineto{\pgfqpoint{2.852426in}{2.504649in}}%
\pgfpathlineto{\pgfqpoint{2.852451in}{2.579890in}}%
\pgfpathlineto{\pgfqpoint{2.852599in}{2.129981in}}%
\pgfpathlineto{\pgfqpoint{2.853399in}{2.283206in}}%
\pgfpathlineto{\pgfqpoint{2.853829in}{2.130068in}}%
\pgfpathlineto{\pgfqpoint{2.853682in}{2.576222in}}%
\pgfpathlineto{\pgfqpoint{2.854482in}{2.503890in}}%
\pgfpathlineto{\pgfqpoint{2.854506in}{2.576438in}}%
\pgfpathlineto{\pgfqpoint{2.855060in}{2.118813in}}%
\pgfpathlineto{\pgfqpoint{2.855454in}{2.247341in}}%
\pgfpathlineto{\pgfqpoint{2.856291in}{2.109824in}}%
\pgfpathlineto{\pgfqpoint{2.855737in}{2.572169in}}%
\pgfpathlineto{\pgfqpoint{2.856525in}{2.502333in}}%
\pgfpathlineto{\pgfqpoint{2.856968in}{2.578976in}}%
\pgfpathlineto{\pgfqpoint{2.856709in}{2.120980in}}%
\pgfpathlineto{\pgfqpoint{2.857103in}{2.149471in}}%
\pgfpathlineto{\pgfqpoint{2.857522in}{2.101267in}}%
\pgfpathlineto{\pgfqpoint{2.857792in}{2.573879in}}%
\pgfpathlineto{\pgfqpoint{2.858137in}{2.419908in}}%
\pgfpathlineto{\pgfqpoint{2.858198in}{2.575847in}}%
\pgfpathlineto{\pgfqpoint{2.858248in}{2.271911in}}%
\pgfpathlineto{\pgfqpoint{2.858322in}{2.305199in}}%
\pgfpathlineto{\pgfqpoint{2.858346in}{2.095202in}}%
\pgfpathlineto{\pgfqpoint{2.859023in}{2.571783in}}%
\pgfpathlineto{\pgfqpoint{2.859405in}{2.509399in}}%
\pgfpathlineto{\pgfqpoint{2.859429in}{2.577024in}}%
\pgfpathlineto{\pgfqpoint{2.859577in}{2.091846in}}%
\pgfpathlineto{\pgfqpoint{2.859971in}{2.218976in}}%
\pgfpathlineto{\pgfqpoint{2.860402in}{2.097914in}}%
\pgfpathlineto{\pgfqpoint{2.860254in}{2.577552in}}%
\pgfpathlineto{\pgfqpoint{2.861042in}{2.510723in}}%
\pgfpathlineto{\pgfqpoint{2.861066in}{2.560177in}}%
\pgfpathlineto{\pgfqpoint{2.861485in}{2.587711in}}%
\pgfpathlineto{\pgfqpoint{2.861632in}{2.095190in}}%
\pgfpathlineto{\pgfqpoint{2.861928in}{2.397241in}}%
\pgfpathlineto{\pgfqpoint{2.862038in}{2.091767in}}%
\pgfpathlineto{\pgfqpoint{2.862715in}{2.592470in}}%
\pgfpathlineto{\pgfqpoint{2.863023in}{2.471826in}}%
\pgfpathlineto{\pgfqpoint{2.863269in}{2.082578in}}%
\pgfpathlineto{\pgfqpoint{2.863946in}{2.600633in}}%
\pgfpathlineto{\pgfqpoint{2.864143in}{2.427762in}}%
\pgfpathlineto{\pgfqpoint{2.865177in}{2.603311in}}%
\pgfpathlineto{\pgfqpoint{2.864500in}{2.083159in}}%
\pgfpathlineto{\pgfqpoint{2.865214in}{2.382302in}}%
\pgfpathlineto{\pgfqpoint{2.865731in}{2.077400in}}%
\pgfpathlineto{\pgfqpoint{2.866002in}{2.605926in}}%
\pgfpathlineto{\pgfqpoint{2.866309in}{2.472395in}}%
\pgfpathlineto{\pgfqpoint{2.866962in}{2.067068in}}%
\pgfpathlineto{\pgfqpoint{2.866408in}{2.602848in}}%
\pgfpathlineto{\pgfqpoint{2.867417in}{2.376243in}}%
\pgfpathlineto{\pgfqpoint{2.867638in}{2.601173in}}%
\pgfpathlineto{\pgfqpoint{2.868192in}{2.062576in}}%
\pgfpathlineto{\pgfqpoint{2.868500in}{2.319845in}}%
\pgfpathlineto{\pgfqpoint{2.869423in}{2.058621in}}%
\pgfpathlineto{\pgfqpoint{2.868869in}{2.599858in}}%
\pgfpathlineto{\pgfqpoint{2.869583in}{2.457093in}}%
\pgfpathlineto{\pgfqpoint{2.869694in}{2.586784in}}%
\pgfpathlineto{\pgfqpoint{2.869817in}{2.195363in}}%
\pgfpathlineto{\pgfqpoint{2.870654in}{2.055147in}}%
\pgfpathlineto{\pgfqpoint{2.870100in}{2.601932in}}%
\pgfpathlineto{\pgfqpoint{2.870888in}{2.540792in}}%
\pgfpathlineto{\pgfqpoint{2.871885in}{2.046961in}}%
\pgfpathlineto{\pgfqpoint{2.871737in}{2.602524in}}%
\pgfpathlineto{\pgfqpoint{2.872032in}{2.420655in}}%
\pgfpathlineto{\pgfqpoint{2.872968in}{2.600154in}}%
\pgfpathlineto{\pgfqpoint{2.872291in}{2.041241in}}%
\pgfpathlineto{\pgfqpoint{2.873091in}{2.246778in}}%
\pgfpathlineto{\pgfqpoint{2.873522in}{2.033937in}}%
\pgfpathlineto{\pgfqpoint{2.873374in}{2.598270in}}%
\pgfpathlineto{\pgfqpoint{2.874174in}{2.526980in}}%
\pgfpathlineto{\pgfqpoint{2.874605in}{2.600575in}}%
\pgfpathlineto{\pgfqpoint{2.874752in}{2.040552in}}%
\pgfpathlineto{\pgfqpoint{2.875134in}{2.291161in}}%
\pgfpathlineto{\pgfqpoint{2.875158in}{2.038124in}}%
\pgfpathlineto{\pgfqpoint{2.875835in}{2.592594in}}%
\pgfpathlineto{\pgfqpoint{2.876217in}{2.542861in}}%
\pgfpathlineto{\pgfqpoint{2.876242in}{2.592347in}}%
\pgfpathlineto{\pgfqpoint{2.876266in}{2.473755in}}%
\pgfpathlineto{\pgfqpoint{2.876389in}{2.041553in}}%
\pgfpathlineto{\pgfqpoint{2.877066in}{2.582988in}}%
\pgfpathlineto{\pgfqpoint{2.877361in}{2.452157in}}%
\pgfpathlineto{\pgfqpoint{2.877472in}{2.593204in}}%
\pgfpathlineto{\pgfqpoint{2.878026in}{2.038415in}}%
\pgfpathlineto{\pgfqpoint{2.878408in}{2.339037in}}%
\pgfpathlineto{\pgfqpoint{2.879257in}{2.039220in}}%
\pgfpathlineto{\pgfqpoint{2.878703in}{2.591146in}}%
\pgfpathlineto{\pgfqpoint{2.879503in}{2.538104in}}%
\pgfpathlineto{\pgfqpoint{2.880340in}{2.585770in}}%
\pgfpathlineto{\pgfqpoint{2.879663in}{2.048322in}}%
\pgfpathlineto{\pgfqpoint{2.880475in}{2.113174in}}%
\pgfpathlineto{\pgfqpoint{2.880488in}{2.051300in}}%
\pgfpathlineto{\pgfqpoint{2.881165in}{2.580444in}}%
\pgfpathlineto{\pgfqpoint{2.881509in}{2.441149in}}%
\pgfpathlineto{\pgfqpoint{2.881571in}{2.584775in}}%
\pgfpathlineto{\pgfqpoint{2.881632in}{2.263765in}}%
\pgfpathlineto{\pgfqpoint{2.881694in}{2.271832in}}%
\pgfpathlineto{\pgfqpoint{2.881718in}{2.065770in}}%
\pgfpathlineto{\pgfqpoint{2.882395in}{2.577099in}}%
\pgfpathlineto{\pgfqpoint{2.882777in}{2.538768in}}%
\pgfpathlineto{\pgfqpoint{2.882801in}{2.584758in}}%
\pgfpathlineto{\pgfqpoint{2.883355in}{2.063867in}}%
\pgfpathlineto{\pgfqpoint{2.883663in}{2.339013in}}%
\pgfpathlineto{\pgfqpoint{2.884586in}{2.065852in}}%
\pgfpathlineto{\pgfqpoint{2.884032in}{2.583238in}}%
\pgfpathlineto{\pgfqpoint{2.884746in}{2.453745in}}%
\pgfpathlineto{\pgfqpoint{2.885263in}{2.579337in}}%
\pgfpathlineto{\pgfqpoint{2.885411in}{2.065906in}}%
\pgfpathlineto{\pgfqpoint{2.885792in}{2.305266in}}%
\pgfpathlineto{\pgfqpoint{2.885817in}{2.061677in}}%
\pgfpathlineto{\pgfqpoint{2.886494in}{2.573338in}}%
\pgfpathlineto{\pgfqpoint{2.886875in}{2.525219in}}%
\pgfpathlineto{\pgfqpoint{2.886900in}{2.567170in}}%
\pgfpathlineto{\pgfqpoint{2.886937in}{2.387826in}}%
\pgfpathlineto{\pgfqpoint{2.887048in}{2.061520in}}%
\pgfpathlineto{\pgfqpoint{2.887725in}{2.566491in}}%
\pgfpathlineto{\pgfqpoint{2.888032in}{2.488689in}}%
\pgfpathlineto{\pgfqpoint{2.888278in}{2.066634in}}%
\pgfpathlineto{\pgfqpoint{2.888131in}{2.565920in}}%
\pgfpathlineto{\pgfqpoint{2.889152in}{2.457219in}}%
\pgfpathlineto{\pgfqpoint{2.889361in}{2.562382in}}%
\pgfpathlineto{\pgfqpoint{2.889509in}{2.073670in}}%
\pgfpathlineto{\pgfqpoint{2.890223in}{2.337660in}}%
\pgfpathlineto{\pgfqpoint{2.890740in}{2.083512in}}%
\pgfpathlineto{\pgfqpoint{2.890592in}{2.565761in}}%
\pgfpathlineto{\pgfqpoint{2.891306in}{2.439495in}}%
\pgfpathlineto{\pgfqpoint{2.891823in}{2.564609in}}%
\pgfpathlineto{\pgfqpoint{2.891971in}{2.092415in}}%
\pgfpathlineto{\pgfqpoint{2.892352in}{2.292393in}}%
\pgfpathlineto{\pgfqpoint{2.892377in}{2.091885in}}%
\pgfpathlineto{\pgfqpoint{2.893054in}{2.566103in}}%
\pgfpathlineto{\pgfqpoint{2.893435in}{2.527151in}}%
\pgfpathlineto{\pgfqpoint{2.893608in}{2.093421in}}%
\pgfpathlineto{\pgfqpoint{2.893460in}{2.567191in}}%
\pgfpathlineto{\pgfqpoint{2.894641in}{2.442562in}}%
\pgfpathlineto{\pgfqpoint{2.894691in}{2.569076in}}%
\pgfpathlineto{\pgfqpoint{2.894838in}{2.091400in}}%
\pgfpathlineto{\pgfqpoint{2.895737in}{2.474060in}}%
\pgfpathlineto{\pgfqpoint{2.896475in}{2.090437in}}%
\pgfpathlineto{\pgfqpoint{2.896328in}{2.565882in}}%
\pgfpathlineto{\pgfqpoint{2.896881in}{2.097491in}}%
\pgfpathlineto{\pgfqpoint{2.897558in}{2.564257in}}%
\pgfpathlineto{\pgfqpoint{2.897706in}{2.082234in}}%
\pgfpathlineto{\pgfqpoint{2.898100in}{2.190914in}}%
\pgfpathlineto{\pgfqpoint{2.898937in}{2.075569in}}%
\pgfpathlineto{\pgfqpoint{2.898789in}{2.558918in}}%
\pgfpathlineto{\pgfqpoint{2.899158in}{2.496414in}}%
\pgfpathlineto{\pgfqpoint{2.899195in}{2.554129in}}%
\pgfpathlineto{\pgfqpoint{2.899343in}{2.082293in}}%
\pgfpathlineto{\pgfqpoint{2.900143in}{2.271898in}}%
\pgfpathlineto{\pgfqpoint{2.900168in}{2.074900in}}%
\pgfpathlineto{\pgfqpoint{2.900426in}{2.549577in}}%
\pgfpathlineto{\pgfqpoint{2.901226in}{2.519722in}}%
\pgfpathlineto{\pgfqpoint{2.901251in}{2.552115in}}%
\pgfpathlineto{\pgfqpoint{2.901288in}{2.357837in}}%
\pgfpathlineto{\pgfqpoint{2.901398in}{2.076144in}}%
\pgfpathlineto{\pgfqpoint{2.901657in}{2.547369in}}%
\pgfpathlineto{\pgfqpoint{2.902383in}{2.485164in}}%
\pgfpathlineto{\pgfqpoint{2.902629in}{2.075173in}}%
\pgfpathlineto{\pgfqpoint{2.902481in}{2.549348in}}%
\pgfpathlineto{\pgfqpoint{2.903503in}{2.453793in}}%
\pgfpathlineto{\pgfqpoint{2.904537in}{2.552755in}}%
\pgfpathlineto{\pgfqpoint{2.903860in}{2.076565in}}%
\pgfpathlineto{\pgfqpoint{2.904574in}{2.352400in}}%
\pgfpathlineto{\pgfqpoint{2.905091in}{2.075676in}}%
\pgfpathlineto{\pgfqpoint{2.904943in}{2.551609in}}%
\pgfpathlineto{\pgfqpoint{2.905657in}{2.434099in}}%
\pgfpathlineto{\pgfqpoint{2.905768in}{2.554340in}}%
\pgfpathlineto{\pgfqpoint{2.906321in}{2.079315in}}%
\pgfpathlineto{\pgfqpoint{2.906715in}{2.212771in}}%
\pgfpathlineto{\pgfqpoint{2.907146in}{2.083246in}}%
\pgfpathlineto{\pgfqpoint{2.906998in}{2.558060in}}%
\pgfpathlineto{\pgfqpoint{2.907786in}{2.502785in}}%
\pgfpathlineto{\pgfqpoint{2.907823in}{2.554187in}}%
\pgfpathlineto{\pgfqpoint{2.907946in}{2.227003in}}%
\pgfpathlineto{\pgfqpoint{2.908377in}{2.083163in}}%
\pgfpathlineto{\pgfqpoint{2.908229in}{2.558957in}}%
\pgfpathlineto{\pgfqpoint{2.909029in}{2.507305in}}%
\pgfpathlineto{\pgfqpoint{2.909460in}{2.557043in}}%
\pgfpathlineto{\pgfqpoint{2.909608in}{2.083910in}}%
\pgfpathlineto{\pgfqpoint{2.910001in}{2.197295in}}%
\pgfpathlineto{\pgfqpoint{2.910838in}{2.084708in}}%
\pgfpathlineto{\pgfqpoint{2.910691in}{2.556144in}}%
\pgfpathlineto{\pgfqpoint{2.911072in}{2.523624in}}%
\pgfpathlineto{\pgfqpoint{2.912069in}{2.087538in}}%
\pgfpathlineto{\pgfqpoint{2.911921in}{2.555537in}}%
\pgfpathlineto{\pgfqpoint{2.912278in}{2.430678in}}%
\pgfpathlineto{\pgfqpoint{2.912328in}{2.549608in}}%
\pgfpathlineto{\pgfqpoint{2.913300in}{2.087120in}}%
\pgfpathlineto{\pgfqpoint{2.913558in}{2.546748in}}%
\pgfpathlineto{\pgfqpoint{2.914494in}{2.404677in}}%
\pgfpathlineto{\pgfqpoint{2.914937in}{2.085388in}}%
\pgfpathlineto{\pgfqpoint{2.914789in}{2.547299in}}%
\pgfpathlineto{\pgfqpoint{2.915589in}{2.512859in}}%
\pgfpathlineto{\pgfqpoint{2.916020in}{2.548370in}}%
\pgfpathlineto{\pgfqpoint{2.916168in}{2.078138in}}%
\pgfpathlineto{\pgfqpoint{2.916549in}{2.345428in}}%
\pgfpathlineto{\pgfqpoint{2.917398in}{2.072910in}}%
\pgfpathlineto{\pgfqpoint{2.917251in}{2.547952in}}%
\pgfpathlineto{\pgfqpoint{2.917644in}{2.514468in}}%
\pgfpathlineto{\pgfqpoint{2.917657in}{2.547720in}}%
\pgfpathlineto{\pgfqpoint{2.917804in}{2.070475in}}%
\pgfpathlineto{\pgfqpoint{2.918604in}{2.259989in}}%
\pgfpathlineto{\pgfqpoint{2.919035in}{2.071956in}}%
\pgfpathlineto{\pgfqpoint{2.918887in}{2.548476in}}%
\pgfpathlineto{\pgfqpoint{2.919687in}{2.505704in}}%
\pgfpathlineto{\pgfqpoint{2.920118in}{2.548215in}}%
\pgfpathlineto{\pgfqpoint{2.920266in}{2.077672in}}%
\pgfpathlineto{\pgfqpoint{2.920647in}{2.321531in}}%
\pgfpathlineto{\pgfqpoint{2.921078in}{2.071613in}}%
\pgfpathlineto{\pgfqpoint{2.921349in}{2.548552in}}%
\pgfpathlineto{\pgfqpoint{2.921743in}{2.503930in}}%
\pgfpathlineto{\pgfqpoint{2.921755in}{2.551692in}}%
\pgfpathlineto{\pgfqpoint{2.922309in}{2.064120in}}%
\pgfpathlineto{\pgfqpoint{2.922801in}{2.421898in}}%
\pgfpathlineto{\pgfqpoint{2.923392in}{2.558197in}}%
\pgfpathlineto{\pgfqpoint{2.923946in}{2.062331in}}%
\pgfpathlineto{\pgfqpoint{2.925029in}{2.556350in}}%
\pgfpathlineto{\pgfqpoint{2.925140in}{2.431986in}}%
\pgfpathlineto{\pgfqpoint{2.925583in}{2.070696in}}%
\pgfpathlineto{\pgfqpoint{2.925435in}{2.549639in}}%
\pgfpathlineto{\pgfqpoint{2.926235in}{2.493023in}}%
\pgfpathlineto{\pgfqpoint{2.926666in}{2.548715in}}%
\pgfpathlineto{\pgfqpoint{2.926814in}{2.074684in}}%
\pgfpathlineto{\pgfqpoint{2.927207in}{2.159019in}}%
\pgfpathlineto{\pgfqpoint{2.927220in}{2.068578in}}%
\pgfpathlineto{\pgfqpoint{2.927478in}{2.538891in}}%
\pgfpathlineto{\pgfqpoint{2.928266in}{2.520129in}}%
\pgfpathlineto{\pgfqpoint{2.928857in}{2.067540in}}%
\pgfpathlineto{\pgfqpoint{2.928709in}{2.552690in}}%
\pgfpathlineto{\pgfqpoint{2.929447in}{2.425761in}}%
\pgfpathlineto{\pgfqpoint{2.930346in}{2.550424in}}%
\pgfpathlineto{\pgfqpoint{2.929669in}{2.086296in}}%
\pgfpathlineto{\pgfqpoint{2.930481in}{2.143253in}}%
\pgfpathlineto{\pgfqpoint{2.930900in}{2.079337in}}%
\pgfpathlineto{\pgfqpoint{2.931134in}{2.552462in}}%
\pgfpathlineto{\pgfqpoint{2.931527in}{2.456711in}}%
\pgfpathlineto{\pgfqpoint{2.932364in}{2.563328in}}%
\pgfpathlineto{\pgfqpoint{2.932537in}{2.070466in}}%
\pgfpathlineto{\pgfqpoint{2.932623in}{2.420831in}}%
\pgfpathlineto{\pgfqpoint{2.932943in}{2.074945in}}%
\pgfpathlineto{\pgfqpoint{2.932771in}{2.570938in}}%
\pgfpathlineto{\pgfqpoint{2.933706in}{2.498721in}}%
\pgfpathlineto{\pgfqpoint{2.934001in}{2.573105in}}%
\pgfpathlineto{\pgfqpoint{2.934174in}{2.061652in}}%
\pgfpathlineto{\pgfqpoint{2.934826in}{2.559464in}}%
\pgfpathlineto{\pgfqpoint{2.935404in}{2.053889in}}%
\pgfpathlineto{\pgfqpoint{2.935232in}{2.568161in}}%
\pgfpathlineto{\pgfqpoint{2.936020in}{2.445981in}}%
\pgfpathlineto{\pgfqpoint{2.936894in}{2.564607in}}%
\pgfpathlineto{\pgfqpoint{2.936635in}{2.059373in}}%
\pgfpathlineto{\pgfqpoint{2.937017in}{2.366497in}}%
\pgfpathlineto{\pgfqpoint{2.937460in}{2.068205in}}%
\pgfpathlineto{\pgfqpoint{2.937300in}{2.565596in}}%
\pgfpathlineto{\pgfqpoint{2.938112in}{2.546875in}}%
\pgfpathlineto{\pgfqpoint{2.938949in}{2.568414in}}%
\pgfpathlineto{\pgfqpoint{2.938691in}{2.084633in}}%
\pgfpathlineto{\pgfqpoint{2.939072in}{2.347247in}}%
\pgfpathlineto{\pgfqpoint{2.939515in}{2.097146in}}%
\pgfpathlineto{\pgfqpoint{2.939355in}{2.560814in}}%
\pgfpathlineto{\pgfqpoint{2.940167in}{2.521936in}}%
\pgfpathlineto{\pgfqpoint{2.940180in}{2.552103in}}%
\pgfpathlineto{\pgfqpoint{2.940340in}{2.117047in}}%
\pgfpathlineto{\pgfqpoint{2.941140in}{2.266696in}}%
\pgfpathlineto{\pgfqpoint{2.941164in}{2.124836in}}%
\pgfpathlineto{\pgfqpoint{2.941410in}{2.535163in}}%
\pgfpathlineto{\pgfqpoint{2.942210in}{2.480398in}}%
\pgfpathlineto{\pgfqpoint{2.942235in}{2.530133in}}%
\pgfpathlineto{\pgfqpoint{2.942814in}{2.145403in}}%
\pgfpathlineto{\pgfqpoint{2.943207in}{2.214718in}}%
\pgfpathlineto{\pgfqpoint{2.943638in}{2.155085in}}%
\pgfpathlineto{\pgfqpoint{2.943466in}{2.529812in}}%
\pgfpathlineto{\pgfqpoint{2.944254in}{2.468183in}}%
\pgfpathlineto{\pgfqpoint{2.944697in}{2.524373in}}%
\pgfpathlineto{\pgfqpoint{2.944463in}{2.167178in}}%
\pgfpathlineto{\pgfqpoint{2.945250in}{2.326432in}}%
\pgfpathlineto{\pgfqpoint{2.945694in}{2.178829in}}%
\pgfpathlineto{\pgfqpoint{2.945927in}{2.520409in}}%
\pgfpathlineto{\pgfqpoint{2.946321in}{2.480917in}}%
\pgfpathlineto{\pgfqpoint{2.947158in}{2.516734in}}%
\pgfpathlineto{\pgfqpoint{2.946518in}{2.198062in}}%
\pgfpathlineto{\pgfqpoint{2.947330in}{2.218843in}}%
\pgfpathlineto{\pgfqpoint{2.947343in}{2.217466in}}%
\pgfpathlineto{\pgfqpoint{2.947367in}{2.329353in}}%
\pgfpathlineto{\pgfqpoint{2.948389in}{2.513769in}}%
\pgfpathlineto{\pgfqpoint{2.947749in}{2.222080in}}%
\pgfpathlineto{\pgfqpoint{2.948487in}{2.365860in}}%
\pgfpathlineto{\pgfqpoint{2.948561in}{2.237744in}}%
\pgfpathlineto{\pgfqpoint{2.948795in}{2.504832in}}%
\pgfpathlineto{\pgfqpoint{2.949189in}{2.447013in}}%
\pgfpathlineto{\pgfqpoint{2.949238in}{2.506484in}}%
\pgfpathlineto{\pgfqpoint{2.949386in}{2.244350in}}%
\pgfpathlineto{\pgfqpoint{2.950260in}{2.345718in}}%
\pgfpathlineto{\pgfqpoint{2.950617in}{2.270117in}}%
\pgfpathlineto{\pgfqpoint{2.950469in}{2.506196in}}%
\pgfpathlineto{\pgfqpoint{2.951257in}{2.474975in}}%
\pgfpathlineto{\pgfqpoint{2.951294in}{2.503078in}}%
\pgfpathlineto{\pgfqpoint{2.951318in}{2.449592in}}%
\pgfpathlineto{\pgfqpoint{2.951429in}{2.288961in}}%
\pgfpathlineto{\pgfqpoint{2.951712in}{2.500216in}}%
\pgfpathlineto{\pgfqpoint{2.952426in}{2.420655in}}%
\pgfpathlineto{\pgfqpoint{2.952524in}{2.497180in}}%
\pgfpathlineto{\pgfqpoint{2.953164in}{2.287680in}}%
\pgfpathlineto{\pgfqpoint{2.953460in}{2.319768in}}%
\pgfpathlineto{\pgfqpoint{2.953989in}{2.288331in}}%
\pgfpathlineto{\pgfqpoint{2.953755in}{2.493831in}}%
\pgfpathlineto{\pgfqpoint{2.954469in}{2.405333in}}%
\pgfpathlineto{\pgfqpoint{2.954580in}{2.488549in}}%
\pgfpathlineto{\pgfqpoint{2.955195in}{2.286471in}}%
\pgfpathlineto{\pgfqpoint{2.955515in}{2.316113in}}%
\pgfpathlineto{\pgfqpoint{2.955835in}{2.481641in}}%
\pgfpathlineto{\pgfqpoint{2.956020in}{2.283852in}}%
\pgfpathlineto{\pgfqpoint{2.956414in}{2.293748in}}%
\pgfpathlineto{\pgfqpoint{2.956844in}{2.288311in}}%
\pgfpathlineto{\pgfqpoint{2.956660in}{2.484033in}}%
\pgfpathlineto{\pgfqpoint{2.957054in}{2.470526in}}%
\pgfpathlineto{\pgfqpoint{2.957484in}{2.482376in}}%
\pgfpathlineto{\pgfqpoint{2.957250in}{2.287930in}}%
\pgfpathlineto{\pgfqpoint{2.957644in}{2.298575in}}%
\pgfpathlineto{\pgfqpoint{2.958063in}{2.279044in}}%
\pgfpathlineto{\pgfqpoint{2.957890in}{2.482560in}}%
\pgfpathlineto{\pgfqpoint{2.958580in}{2.407793in}}%
\pgfpathlineto{\pgfqpoint{2.959121in}{2.481437in}}%
\pgfpathlineto{\pgfqpoint{2.958875in}{2.275612in}}%
\pgfpathlineto{\pgfqpoint{2.959207in}{2.334156in}}%
\pgfpathlineto{\pgfqpoint{2.959294in}{2.282745in}}%
\pgfpathlineto{\pgfqpoint{2.959540in}{2.458088in}}%
\pgfpathlineto{\pgfqpoint{2.960266in}{2.398843in}}%
\pgfpathlineto{\pgfqpoint{2.960364in}{2.475539in}}%
\pgfpathlineto{\pgfqpoint{2.960524in}{2.297259in}}%
\pgfpathlineto{\pgfqpoint{2.961373in}{2.268140in}}%
\pgfpathlineto{\pgfqpoint{2.961226in}{2.461336in}}%
\pgfpathlineto{\pgfqpoint{2.961509in}{2.415136in}}%
\pgfpathlineto{\pgfqpoint{2.962457in}{2.517858in}}%
\pgfpathlineto{\pgfqpoint{2.961792in}{2.227471in}}%
\pgfpathlineto{\pgfqpoint{2.962580in}{2.379371in}}%
\pgfpathlineto{\pgfqpoint{2.962617in}{2.266341in}}%
\pgfpathlineto{\pgfqpoint{2.963146in}{2.456420in}}%
\pgfpathlineto{\pgfqpoint{2.963700in}{2.321985in}}%
\pgfpathlineto{\pgfqpoint{2.964647in}{2.224398in}}%
\pgfpathlineto{\pgfqpoint{2.964180in}{2.470197in}}%
\pgfpathlineto{\pgfqpoint{2.964758in}{2.327625in}}%
\pgfpathlineto{\pgfqpoint{2.965410in}{2.496085in}}%
\pgfpathlineto{\pgfqpoint{2.965066in}{2.238040in}}%
\pgfpathlineto{\pgfqpoint{2.965853in}{2.324399in}}%
\pgfpathlineto{\pgfqpoint{2.965890in}{2.212219in}}%
\pgfpathlineto{\pgfqpoint{2.966063in}{2.479607in}}%
\pgfpathlineto{\pgfqpoint{2.966949in}{2.375664in}}%
\pgfpathlineto{\pgfqpoint{2.967983in}{2.489124in}}%
\pgfpathlineto{\pgfqpoint{2.967121in}{2.275632in}}%
\pgfpathlineto{\pgfqpoint{2.968118in}{2.452617in}}%
\pgfpathlineto{\pgfqpoint{2.969152in}{2.290957in}}%
\pgfpathlineto{\pgfqpoint{2.969226in}{2.451134in}}%
\pgfpathlineto{\pgfqpoint{2.970284in}{2.269284in}}%
\pgfpathlineto{\pgfqpoint{2.969275in}{2.493530in}}%
\pgfpathlineto{\pgfqpoint{2.970407in}{2.395723in}}%
\pgfpathlineto{\pgfqpoint{2.970518in}{2.479102in}}%
\pgfpathlineto{\pgfqpoint{2.970826in}{2.280999in}}%
\pgfpathlineto{\pgfqpoint{2.971478in}{2.342030in}}%
\pgfpathlineto{\pgfqpoint{2.971515in}{2.309297in}}%
\pgfpathlineto{\pgfqpoint{2.972204in}{2.479493in}}%
\pgfpathlineto{\pgfqpoint{2.972573in}{2.352894in}}%
\pgfpathlineto{\pgfqpoint{2.973533in}{2.479788in}}%
\pgfpathlineto{\pgfqpoint{2.973275in}{2.263930in}}%
\pgfpathlineto{\pgfqpoint{2.973669in}{2.358797in}}%
\pgfpathlineto{\pgfqpoint{2.974518in}{2.264140in}}%
\pgfpathlineto{\pgfqpoint{2.974284in}{2.530871in}}%
\pgfpathlineto{\pgfqpoint{2.974752in}{2.350619in}}%
\pgfpathlineto{\pgfqpoint{2.975404in}{2.500003in}}%
\pgfpathlineto{\pgfqpoint{2.975663in}{2.271360in}}%
\pgfpathlineto{\pgfqpoint{2.975847in}{2.350062in}}%
\pgfpathlineto{\pgfqpoint{2.976475in}{2.231307in}}%
\pgfpathlineto{\pgfqpoint{2.976647in}{2.485873in}}%
\pgfpathlineto{\pgfqpoint{2.976930in}{2.388248in}}%
\pgfpathlineto{\pgfqpoint{2.977964in}{2.476427in}}%
\pgfpathlineto{\pgfqpoint{2.977103in}{2.264141in}}%
\pgfpathlineto{\pgfqpoint{2.978026in}{2.386223in}}%
\pgfpathlineto{\pgfqpoint{2.978333in}{2.250171in}}%
\pgfpathlineto{\pgfqpoint{2.978518in}{2.481295in}}%
\pgfpathlineto{\pgfqpoint{2.979084in}{2.480336in}}%
\pgfpathlineto{\pgfqpoint{2.979109in}{2.500304in}}%
\pgfpathlineto{\pgfqpoint{2.979564in}{2.252509in}}%
\pgfpathlineto{\pgfqpoint{2.980057in}{2.378898in}}%
\pgfpathlineto{\pgfqpoint{2.980204in}{2.234710in}}%
\pgfpathlineto{\pgfqpoint{2.980881in}{2.499415in}}%
\pgfpathlineto{\pgfqpoint{2.981201in}{2.283968in}}%
\pgfpathlineto{\pgfqpoint{2.981681in}{2.519985in}}%
\pgfpathlineto{\pgfqpoint{2.982026in}{2.267078in}}%
\pgfpathlineto{\pgfqpoint{2.982518in}{2.401729in}}%
\pgfpathlineto{\pgfqpoint{2.983133in}{2.242495in}}%
\pgfpathlineto{\pgfqpoint{2.983478in}{2.498356in}}%
\pgfpathlineto{\pgfqpoint{2.983576in}{2.474380in}}%
\pgfpathlineto{\pgfqpoint{2.983601in}{2.492657in}}%
\pgfpathlineto{\pgfqpoint{2.983933in}{2.221439in}}%
\pgfpathlineto{\pgfqpoint{2.984635in}{2.412892in}}%
\pgfpathlineto{\pgfqpoint{2.985164in}{2.199165in}}%
\pgfpathlineto{\pgfqpoint{2.985336in}{2.554442in}}%
\pgfpathlineto{\pgfqpoint{2.985829in}{2.372937in}}%
\pgfpathlineto{\pgfqpoint{2.986456in}{2.490640in}}%
\pgfpathlineto{\pgfqpoint{2.986813in}{2.249757in}}%
\pgfpathlineto{\pgfqpoint{2.986900in}{2.333565in}}%
\pgfpathlineto{\pgfqpoint{2.987638in}{2.212985in}}%
\pgfpathlineto{\pgfqpoint{2.987293in}{2.518897in}}%
\pgfpathlineto{\pgfqpoint{2.987970in}{2.398202in}}%
\pgfpathlineto{\pgfqpoint{2.989041in}{2.516811in}}%
\pgfpathlineto{\pgfqpoint{2.988869in}{2.245319in}}%
\pgfpathlineto{\pgfqpoint{2.989090in}{2.417541in}}%
\pgfpathlineto{\pgfqpoint{2.989164in}{2.518959in}}%
\pgfpathlineto{\pgfqpoint{2.989386in}{2.249218in}}%
\pgfpathlineto{\pgfqpoint{2.990186in}{2.437237in}}%
\pgfpathlineto{\pgfqpoint{2.990629in}{2.260540in}}%
\pgfpathlineto{\pgfqpoint{2.990998in}{2.517434in}}%
\pgfpathlineto{\pgfqpoint{2.991355in}{2.291304in}}%
\pgfpathlineto{\pgfqpoint{2.991527in}{2.462807in}}%
\pgfpathlineto{\pgfqpoint{2.992389in}{2.272309in}}%
\pgfpathlineto{\pgfqpoint{2.992487in}{2.332255in}}%
\pgfpathlineto{\pgfqpoint{2.993213in}{2.272961in}}%
\pgfpathlineto{\pgfqpoint{2.992746in}{2.479565in}}%
\pgfpathlineto{\pgfqpoint{2.993521in}{2.434391in}}%
\pgfpathlineto{\pgfqpoint{2.993533in}{2.444787in}}%
\pgfpathlineto{\pgfqpoint{2.994420in}{2.280951in}}%
\pgfpathlineto{\pgfqpoint{2.994444in}{2.254429in}}%
\pgfpathlineto{\pgfqpoint{2.994690in}{2.498004in}}%
\pgfpathlineto{\pgfqpoint{2.995429in}{2.405370in}}%
\pgfpathlineto{\pgfqpoint{2.995638in}{2.465952in}}%
\pgfpathlineto{\pgfqpoint{2.996192in}{2.282042in}}%
\pgfpathlineto{\pgfqpoint{2.996549in}{2.439206in}}%
\pgfpathlineto{\pgfqpoint{2.996807in}{2.266043in}}%
\pgfpathlineto{\pgfqpoint{2.996610in}{2.461869in}}%
\pgfpathlineto{\pgfqpoint{2.997792in}{2.327765in}}%
\pgfpathlineto{\pgfqpoint{2.998358in}{2.496047in}}%
\pgfpathlineto{\pgfqpoint{2.998678in}{2.269162in}}%
\pgfpathlineto{\pgfqpoint{2.998924in}{2.392270in}}%
\pgfpathlineto{\pgfqpoint{2.999416in}{2.297426in}}%
\pgfpathlineto{\pgfqpoint{2.999072in}{2.441913in}}%
\pgfpathlineto{\pgfqpoint{3.000044in}{2.336179in}}%
\pgfpathlineto{\pgfqpoint{3.000229in}{2.484511in}}%
\pgfpathlineto{\pgfqpoint{3.000672in}{2.274059in}}%
\pgfpathlineto{\pgfqpoint{3.001189in}{2.433485in}}%
\pgfpathlineto{\pgfqpoint{3.002370in}{2.246069in}}%
\pgfpathlineto{\pgfqpoint{3.002050in}{2.528281in}}%
\pgfpathlineto{\pgfqpoint{3.002432in}{2.337360in}}%
\pgfpathlineto{\pgfqpoint{3.002961in}{2.471288in}}%
\pgfpathlineto{\pgfqpoint{3.002518in}{2.222550in}}%
\pgfpathlineto{\pgfqpoint{3.003552in}{2.389323in}}%
\pgfpathlineto{\pgfqpoint{3.004376in}{2.237673in}}%
\pgfpathlineto{\pgfqpoint{3.003909in}{2.494540in}}%
\pgfpathlineto{\pgfqpoint{3.004635in}{2.391745in}}%
\pgfpathlineto{\pgfqpoint{3.005755in}{2.534118in}}%
\pgfpathlineto{\pgfqpoint{3.005509in}{2.273185in}}%
\pgfpathlineto{\pgfqpoint{3.005792in}{2.451758in}}%
\pgfpathlineto{\pgfqpoint{3.005829in}{2.467458in}}%
\pgfpathlineto{\pgfqpoint{3.005915in}{2.417769in}}%
\pgfpathlineto{\pgfqpoint{3.006223in}{2.251029in}}%
\pgfpathlineto{\pgfqpoint{3.006690in}{2.473956in}}%
\pgfpathlineto{\pgfqpoint{3.007072in}{2.342269in}}%
\pgfpathlineto{\pgfqpoint{3.007650in}{2.502302in}}%
\pgfpathlineto{\pgfqpoint{3.007970in}{2.219276in}}%
\pgfpathlineto{\pgfqpoint{3.008056in}{2.258332in}}%
\pgfpathlineto{\pgfqpoint{3.008069in}{2.226886in}}%
\pgfpathlineto{\pgfqpoint{3.008549in}{2.485021in}}%
\pgfpathlineto{\pgfqpoint{3.009115in}{2.350368in}}%
\pgfpathlineto{\pgfqpoint{3.009509in}{2.511787in}}%
\pgfpathlineto{\pgfqpoint{3.009730in}{2.244624in}}%
\pgfpathlineto{\pgfqpoint{3.010247in}{2.404360in}}%
\pgfpathlineto{\pgfqpoint{3.011416in}{2.511334in}}%
\pgfpathlineto{\pgfqpoint{3.010543in}{2.287889in}}%
\pgfpathlineto{\pgfqpoint{3.011441in}{2.461792in}}%
\pgfpathlineto{\pgfqpoint{3.011786in}{2.238838in}}%
\pgfpathlineto{\pgfqpoint{3.012598in}{2.296414in}}%
\pgfpathlineto{\pgfqpoint{3.012647in}{2.386623in}}%
\pgfpathlineto{\pgfqpoint{3.013176in}{2.544746in}}%
\pgfpathlineto{\pgfqpoint{3.013546in}{2.255916in}}%
\pgfpathlineto{\pgfqpoint{3.013779in}{2.451708in}}%
\pgfpathlineto{\pgfqpoint{3.014259in}{2.277274in}}%
\pgfpathlineto{\pgfqpoint{3.013964in}{2.453872in}}%
\pgfpathlineto{\pgfqpoint{3.014912in}{2.380124in}}%
\pgfpathlineto{\pgfqpoint{3.015010in}{2.496995in}}%
\pgfpathlineto{\pgfqpoint{3.015392in}{2.244572in}}%
\pgfpathlineto{\pgfqpoint{3.016056in}{2.419597in}}%
\pgfpathlineto{\pgfqpoint{3.017139in}{2.286091in}}%
\pgfpathlineto{\pgfqpoint{3.016906in}{2.556164in}}%
\pgfpathlineto{\pgfqpoint{3.017238in}{2.289749in}}%
\pgfpathlineto{\pgfqpoint{3.017250in}{2.278515in}}%
\pgfpathlineto{\pgfqpoint{3.017681in}{2.448953in}}%
\pgfpathlineto{\pgfqpoint{3.018272in}{2.345921in}}%
\pgfpathlineto{\pgfqpoint{3.018764in}{2.506866in}}%
\pgfpathlineto{\pgfqpoint{3.019121in}{2.254882in}}%
\pgfpathlineto{\pgfqpoint{3.019195in}{2.290216in}}%
\pgfpathlineto{\pgfqpoint{3.019207in}{2.263868in}}%
\pgfpathlineto{\pgfqpoint{3.019687in}{2.443946in}}%
\pgfpathlineto{\pgfqpoint{3.020266in}{2.337446in}}%
\pgfpathlineto{\pgfqpoint{3.020623in}{2.485919in}}%
\pgfpathlineto{\pgfqpoint{3.020955in}{2.262857in}}%
\pgfpathlineto{\pgfqpoint{3.021386in}{2.365583in}}%
\pgfpathlineto{\pgfqpoint{3.022186in}{2.276794in}}%
\pgfpathlineto{\pgfqpoint{3.022370in}{2.467312in}}%
\pgfpathlineto{\pgfqpoint{3.022432in}{2.441464in}}%
\pgfpathlineto{\pgfqpoint{3.022469in}{2.496300in}}%
\pgfpathlineto{\pgfqpoint{3.022924in}{2.265368in}}%
\pgfpathlineto{\pgfqpoint{3.023490in}{2.380318in}}%
\pgfpathlineto{\pgfqpoint{3.023552in}{2.273842in}}%
\pgfpathlineto{\pgfqpoint{3.024302in}{2.478652in}}%
\pgfpathlineto{\pgfqpoint{3.024684in}{2.311814in}}%
\pgfpathlineto{\pgfqpoint{3.024930in}{2.439337in}}%
\pgfpathlineto{\pgfqpoint{3.024782in}{2.266045in}}%
\pgfpathlineto{\pgfqpoint{3.025804in}{2.337950in}}%
\pgfpathlineto{\pgfqpoint{3.025915in}{2.282580in}}%
\pgfpathlineto{\pgfqpoint{3.026161in}{2.465087in}}%
\pgfpathlineto{\pgfqpoint{3.026813in}{2.413520in}}%
\pgfpathlineto{\pgfqpoint{3.027761in}{2.298336in}}%
\pgfpathlineto{\pgfqpoint{3.027096in}{2.434219in}}%
\pgfpathlineto{\pgfqpoint{3.027921in}{2.385427in}}%
\pgfpathlineto{\pgfqpoint{3.028056in}{2.468519in}}%
\pgfpathlineto{\pgfqpoint{3.028450in}{2.304313in}}%
\pgfpathlineto{\pgfqpoint{3.029004in}{2.341131in}}%
\pgfpathlineto{\pgfqpoint{3.029742in}{2.290178in}}%
\pgfpathlineto{\pgfqpoint{3.029299in}{2.444211in}}%
\pgfpathlineto{\pgfqpoint{3.029853in}{2.405026in}}%
\pgfpathlineto{\pgfqpoint{3.029902in}{2.477019in}}%
\pgfpathlineto{\pgfqpoint{3.030358in}{2.276434in}}%
\pgfpathlineto{\pgfqpoint{3.030936in}{2.366968in}}%
\pgfpathlineto{\pgfqpoint{3.031589in}{2.292913in}}%
\pgfpathlineto{\pgfqpoint{3.031773in}{2.501163in}}%
\pgfpathlineto{\pgfqpoint{3.032032in}{2.371934in}}%
\pgfpathlineto{\pgfqpoint{3.032487in}{2.442102in}}%
\pgfpathlineto{\pgfqpoint{3.032216in}{2.282835in}}%
\pgfpathlineto{\pgfqpoint{3.033139in}{2.381505in}}%
\pgfpathlineto{\pgfqpoint{3.033459in}{2.253141in}}%
\pgfpathlineto{\pgfqpoint{3.033607in}{2.464844in}}%
\pgfpathlineto{\pgfqpoint{3.034222in}{2.393486in}}%
\pgfpathlineto{\pgfqpoint{3.034702in}{2.287757in}}%
\pgfpathlineto{\pgfqpoint{3.034850in}{2.432003in}}%
\pgfpathlineto{\pgfqpoint{3.035601in}{2.481487in}}%
\pgfpathlineto{\pgfqpoint{3.035219in}{2.289135in}}%
\pgfpathlineto{\pgfqpoint{3.035749in}{2.339074in}}%
\pgfpathlineto{\pgfqpoint{3.036573in}{2.278332in}}%
\pgfpathlineto{\pgfqpoint{3.036389in}{2.437258in}}%
\pgfpathlineto{\pgfqpoint{3.036684in}{2.399395in}}%
\pgfpathlineto{\pgfqpoint{3.037250in}{2.449773in}}%
\pgfpathlineto{\pgfqpoint{3.037595in}{2.301241in}}%
\pgfpathlineto{\pgfqpoint{3.037767in}{2.372116in}}%
\pgfpathlineto{\pgfqpoint{3.038936in}{2.244729in}}%
\pgfpathlineto{\pgfqpoint{3.038604in}{2.461673in}}%
\pgfpathlineto{\pgfqpoint{3.038949in}{2.264966in}}%
\pgfpathlineto{\pgfqpoint{3.039835in}{2.437067in}}%
\pgfpathlineto{\pgfqpoint{3.040093in}{2.402676in}}%
\pgfpathlineto{\pgfqpoint{3.041053in}{2.508857in}}%
\pgfpathlineto{\pgfqpoint{3.040893in}{2.283560in}}%
\pgfpathlineto{\pgfqpoint{3.041201in}{2.417485in}}%
\pgfpathlineto{\pgfqpoint{3.041521in}{2.279979in}}%
\pgfpathlineto{\pgfqpoint{3.041890in}{2.458043in}}%
\pgfpathlineto{\pgfqpoint{3.042309in}{2.408353in}}%
\pgfpathlineto{\pgfqpoint{3.042776in}{2.301125in}}%
\pgfpathlineto{\pgfqpoint{3.042432in}{2.438667in}}%
\pgfpathlineto{\pgfqpoint{3.043416in}{2.405281in}}%
\pgfpathlineto{\pgfqpoint{3.043453in}{2.464141in}}%
\pgfpathlineto{\pgfqpoint{3.044007in}{2.273695in}}%
\pgfpathlineto{\pgfqpoint{3.044487in}{2.333260in}}%
\pgfpathlineto{\pgfqpoint{3.045238in}{2.280109in}}%
\pgfpathlineto{\pgfqpoint{3.044807in}{2.450741in}}%
\pgfpathlineto{\pgfqpoint{3.045545in}{2.389166in}}%
\pgfpathlineto{\pgfqpoint{3.046542in}{2.476985in}}%
\pgfpathlineto{\pgfqpoint{3.045878in}{2.312719in}}%
\pgfpathlineto{\pgfqpoint{3.046678in}{2.419249in}}%
\pgfpathlineto{\pgfqpoint{3.047724in}{2.264854in}}%
\pgfpathlineto{\pgfqpoint{3.047256in}{2.444121in}}%
\pgfpathlineto{\pgfqpoint{3.047785in}{2.392873in}}%
\pgfpathlineto{\pgfqpoint{3.048007in}{2.479820in}}%
\pgfpathlineto{\pgfqpoint{3.048352in}{2.276805in}}%
\pgfpathlineto{\pgfqpoint{3.048869in}{2.375876in}}%
\pgfpathlineto{\pgfqpoint{3.048967in}{2.298930in}}%
\pgfpathlineto{\pgfqpoint{3.049644in}{2.457814in}}%
\pgfpathlineto{\pgfqpoint{3.050025in}{2.345646in}}%
\pgfpathlineto{\pgfqpoint{3.050370in}{2.491292in}}%
\pgfpathlineto{\pgfqpoint{3.050825in}{2.282492in}}%
\pgfpathlineto{\pgfqpoint{3.051182in}{2.415646in}}%
\pgfpathlineto{\pgfqpoint{3.051441in}{2.272762in}}%
\pgfpathlineto{\pgfqpoint{3.051736in}{2.429077in}}%
\pgfpathlineto{\pgfqpoint{3.052376in}{2.373129in}}%
\pgfpathlineto{\pgfqpoint{3.053472in}{2.471659in}}%
\pgfpathlineto{\pgfqpoint{3.053299in}{2.287757in}}%
\pgfpathlineto{\pgfqpoint{3.053509in}{2.428463in}}%
\pgfpathlineto{\pgfqpoint{3.053693in}{2.301204in}}%
\pgfpathlineto{\pgfqpoint{3.054210in}{2.451036in}}%
\pgfpathlineto{\pgfqpoint{3.054641in}{2.360397in}}%
\pgfpathlineto{\pgfqpoint{3.055342in}{2.460361in}}%
\pgfpathlineto{\pgfqpoint{3.055170in}{2.263856in}}%
\pgfpathlineto{\pgfqpoint{3.055761in}{2.376251in}}%
\pgfpathlineto{\pgfqpoint{3.056795in}{2.289922in}}%
\pgfpathlineto{\pgfqpoint{3.056585in}{2.467281in}}%
\pgfpathlineto{\pgfqpoint{3.056905in}{2.291546in}}%
\pgfpathlineto{\pgfqpoint{3.057829in}{2.454256in}}%
\pgfpathlineto{\pgfqpoint{3.057521in}{2.274632in}}%
\pgfpathlineto{\pgfqpoint{3.058087in}{2.349631in}}%
\pgfpathlineto{\pgfqpoint{3.058875in}{2.283493in}}%
\pgfpathlineto{\pgfqpoint{3.058579in}{2.435434in}}%
\pgfpathlineto{\pgfqpoint{3.059022in}{2.411327in}}%
\pgfpathlineto{\pgfqpoint{3.059552in}{2.432430in}}%
\pgfpathlineto{\pgfqpoint{3.059958in}{2.321664in}}%
\pgfpathlineto{\pgfqpoint{3.059982in}{2.274984in}}%
\pgfpathlineto{\pgfqpoint{3.060302in}{2.481418in}}%
\pgfpathlineto{\pgfqpoint{3.061016in}{2.421145in}}%
\pgfpathlineto{\pgfqpoint{3.061545in}{2.444772in}}%
\pgfpathlineto{\pgfqpoint{3.061349in}{2.280217in}}%
\pgfpathlineto{\pgfqpoint{3.062099in}{2.402895in}}%
\pgfpathlineto{\pgfqpoint{3.062592in}{2.288154in}}%
\pgfpathlineto{\pgfqpoint{3.062173in}{2.443383in}}%
\pgfpathlineto{\pgfqpoint{3.063244in}{2.361061in}}%
\pgfpathlineto{\pgfqpoint{3.064019in}{2.462372in}}%
\pgfpathlineto{\pgfqpoint{3.063699in}{2.297429in}}%
\pgfpathlineto{\pgfqpoint{3.064315in}{2.306187in}}%
\pgfpathlineto{\pgfqpoint{3.064450in}{2.281992in}}%
\pgfpathlineto{\pgfqpoint{3.065262in}{2.461223in}}%
\pgfpathlineto{\pgfqpoint{3.065361in}{2.376755in}}%
\pgfpathlineto{\pgfqpoint{3.066493in}{2.451775in}}%
\pgfpathlineto{\pgfqpoint{3.066308in}{2.298052in}}%
\pgfpathlineto{\pgfqpoint{3.066518in}{2.421888in}}%
\pgfpathlineto{\pgfqpoint{3.067552in}{2.279531in}}%
\pgfpathlineto{\pgfqpoint{3.067121in}{2.446491in}}%
\pgfpathlineto{\pgfqpoint{3.067625in}{2.397079in}}%
\pgfpathlineto{\pgfqpoint{3.067736in}{2.452101in}}%
\pgfpathlineto{\pgfqpoint{3.068179in}{2.296628in}}%
\pgfpathlineto{\pgfqpoint{3.068659in}{2.334706in}}%
\pgfpathlineto{\pgfqpoint{3.069422in}{2.307227in}}%
\pgfpathlineto{\pgfqpoint{3.068905in}{2.444127in}}%
\pgfpathlineto{\pgfqpoint{3.069570in}{2.434334in}}%
\pgfpathlineto{\pgfqpoint{3.069607in}{2.462690in}}%
\pgfpathlineto{\pgfqpoint{3.070038in}{2.291997in}}%
\pgfpathlineto{\pgfqpoint{3.070505in}{2.365838in}}%
\pgfpathlineto{\pgfqpoint{3.070665in}{2.304404in}}%
\pgfpathlineto{\pgfqpoint{3.070850in}{2.439398in}}%
\pgfpathlineto{\pgfqpoint{3.071601in}{2.383216in}}%
\pgfpathlineto{\pgfqpoint{3.072081in}{2.440193in}}%
\pgfpathlineto{\pgfqpoint{3.071896in}{2.302665in}}%
\pgfpathlineto{\pgfqpoint{3.072512in}{2.337034in}}%
\pgfpathlineto{\pgfqpoint{3.073152in}{2.276416in}}%
\pgfpathlineto{\pgfqpoint{3.073324in}{2.460582in}}%
\pgfpathlineto{\pgfqpoint{3.073545in}{2.383730in}}%
\pgfpathlineto{\pgfqpoint{3.073619in}{2.352742in}}%
\pgfpathlineto{\pgfqpoint{3.073767in}{2.310933in}}%
\pgfpathlineto{\pgfqpoint{3.074567in}{2.456151in}}%
\pgfpathlineto{\pgfqpoint{3.074665in}{2.417497in}}%
\pgfpathlineto{\pgfqpoint{3.075195in}{2.448401in}}%
\pgfpathlineto{\pgfqpoint{3.075010in}{2.292970in}}%
\pgfpathlineto{\pgfqpoint{3.075736in}{2.398563in}}%
\pgfpathlineto{\pgfqpoint{3.076253in}{2.299694in}}%
\pgfpathlineto{\pgfqpoint{3.076438in}{2.437325in}}%
\pgfpathlineto{\pgfqpoint{3.076905in}{2.340372in}}%
\pgfpathlineto{\pgfqpoint{3.077053in}{2.447006in}}%
\pgfpathlineto{\pgfqpoint{3.077496in}{2.325224in}}%
\pgfpathlineto{\pgfqpoint{3.078013in}{2.368850in}}%
\pgfpathlineto{\pgfqpoint{3.078124in}{2.287775in}}%
\pgfpathlineto{\pgfqpoint{3.078296in}{2.450150in}}%
\pgfpathlineto{\pgfqpoint{3.079096in}{2.385269in}}%
\pgfpathlineto{\pgfqpoint{3.080142in}{2.446585in}}%
\pgfpathlineto{\pgfqpoint{3.079367in}{2.303736in}}%
\pgfpathlineto{\pgfqpoint{3.080179in}{2.391984in}}%
\pgfpathlineto{\pgfqpoint{3.081188in}{2.312303in}}%
\pgfpathlineto{\pgfqpoint{3.080770in}{2.458163in}}%
\pgfpathlineto{\pgfqpoint{3.081287in}{2.392158in}}%
\pgfpathlineto{\pgfqpoint{3.081385in}{2.449501in}}%
\pgfpathlineto{\pgfqpoint{3.081816in}{2.300457in}}%
\pgfpathlineto{\pgfqpoint{3.082358in}{2.373473in}}%
\pgfpathlineto{\pgfqpoint{3.082444in}{2.313553in}}%
\pgfpathlineto{\pgfqpoint{3.082862in}{2.425049in}}%
\pgfpathlineto{\pgfqpoint{3.083453in}{2.379732in}}%
\pgfpathlineto{\pgfqpoint{3.084487in}{2.447688in}}%
\pgfpathlineto{\pgfqpoint{3.084339in}{2.298613in}}%
\pgfpathlineto{\pgfqpoint{3.084598in}{2.417576in}}%
\pgfpathlineto{\pgfqpoint{3.085668in}{2.291269in}}%
\pgfpathlineto{\pgfqpoint{3.085311in}{2.448404in}}%
\pgfpathlineto{\pgfqpoint{3.085718in}{2.394446in}}%
\pgfpathlineto{\pgfqpoint{3.086124in}{2.440498in}}%
\pgfpathlineto{\pgfqpoint{3.086247in}{2.320048in}}%
\pgfpathlineto{\pgfqpoint{3.086801in}{2.385167in}}%
\pgfpathlineto{\pgfqpoint{3.087527in}{2.293394in}}%
\pgfpathlineto{\pgfqpoint{3.087798in}{2.436234in}}%
\pgfpathlineto{\pgfqpoint{3.087908in}{2.376710in}}%
\pgfpathlineto{\pgfqpoint{3.088770in}{2.292469in}}%
\pgfpathlineto{\pgfqpoint{3.089041in}{2.449905in}}%
\pgfpathlineto{\pgfqpoint{3.089496in}{2.282405in}}%
\pgfpathlineto{\pgfqpoint{3.090161in}{2.384738in}}%
\pgfpathlineto{\pgfqpoint{3.090875in}{2.448073in}}%
\pgfpathlineto{\pgfqpoint{3.090616in}{2.306396in}}%
\pgfpathlineto{\pgfqpoint{3.091231in}{2.318834in}}%
\pgfpathlineto{\pgfqpoint{3.091330in}{2.284292in}}%
\pgfpathlineto{\pgfqpoint{3.091293in}{2.356152in}}%
\pgfpathlineto{\pgfqpoint{3.091404in}{2.341753in}}%
\pgfpathlineto{\pgfqpoint{3.092265in}{2.465179in}}%
\pgfpathlineto{\pgfqpoint{3.091995in}{2.248142in}}%
\pgfpathlineto{\pgfqpoint{3.092499in}{2.348768in}}%
\pgfpathlineto{\pgfqpoint{3.093238in}{2.235472in}}%
\pgfpathlineto{\pgfqpoint{3.092893in}{2.468406in}}%
\pgfpathlineto{\pgfqpoint{3.093582in}{2.397277in}}%
\pgfpathlineto{\pgfqpoint{3.094628in}{2.484319in}}%
\pgfpathlineto{\pgfqpoint{3.094468in}{2.249934in}}%
\pgfpathlineto{\pgfqpoint{3.094739in}{2.451544in}}%
\pgfpathlineto{\pgfqpoint{3.095687in}{2.255026in}}%
\pgfpathlineto{\pgfqpoint{3.095428in}{2.456383in}}%
\pgfpathlineto{\pgfqpoint{3.095921in}{2.415157in}}%
\pgfpathlineto{\pgfqpoint{3.096622in}{2.450768in}}%
\pgfpathlineto{\pgfqpoint{3.096351in}{2.280375in}}%
\pgfpathlineto{\pgfqpoint{3.096942in}{2.325727in}}%
\pgfpathlineto{\pgfqpoint{3.097582in}{2.267775in}}%
\pgfpathlineto{\pgfqpoint{3.097755in}{2.472148in}}%
\pgfpathlineto{\pgfqpoint{3.097828in}{2.445581in}}%
\pgfpathlineto{\pgfqpoint{3.097841in}{2.482690in}}%
\pgfpathlineto{\pgfqpoint{3.098813in}{2.207330in}}%
\pgfpathlineto{\pgfqpoint{3.098887in}{2.334498in}}%
\pgfpathlineto{\pgfqpoint{3.100007in}{2.271122in}}%
\pgfpathlineto{\pgfqpoint{3.098973in}{2.481004in}}%
\pgfpathlineto{\pgfqpoint{3.100056in}{2.272272in}}%
\pgfpathlineto{\pgfqpoint{3.100942in}{2.476501in}}%
\pgfpathlineto{\pgfqpoint{3.100684in}{2.260926in}}%
\pgfpathlineto{\pgfqpoint{3.101188in}{2.289803in}}%
\pgfpathlineto{\pgfqpoint{3.101459in}{2.460187in}}%
\pgfpathlineto{\pgfqpoint{3.101902in}{2.218856in}}%
\pgfpathlineto{\pgfqpoint{3.102468in}{2.360778in}}%
\pgfpathlineto{\pgfqpoint{3.103158in}{2.218084in}}%
\pgfpathlineto{\pgfqpoint{3.103318in}{2.502870in}}%
\pgfpathlineto{\pgfqpoint{3.103576in}{2.351393in}}%
\pgfpathlineto{\pgfqpoint{3.104105in}{2.467865in}}%
\pgfpathlineto{\pgfqpoint{3.104401in}{2.244968in}}%
\pgfpathlineto{\pgfqpoint{3.104708in}{2.407654in}}%
\pgfpathlineto{\pgfqpoint{3.105631in}{2.256824in}}%
\pgfpathlineto{\pgfqpoint{3.105299in}{2.485480in}}%
\pgfpathlineto{\pgfqpoint{3.105767in}{2.393634in}}%
\pgfpathlineto{\pgfqpoint{3.106087in}{2.463732in}}%
\pgfpathlineto{\pgfqpoint{3.106247in}{2.224677in}}%
\pgfpathlineto{\pgfqpoint{3.106838in}{2.322329in}}%
\pgfpathlineto{\pgfqpoint{3.107502in}{2.230934in}}%
\pgfpathlineto{\pgfqpoint{3.107662in}{2.457845in}}%
\pgfpathlineto{\pgfqpoint{3.107921in}{2.364853in}}%
\pgfpathlineto{\pgfqpoint{3.108450in}{2.455496in}}%
\pgfpathlineto{\pgfqpoint{3.108105in}{2.248858in}}%
\pgfpathlineto{\pgfqpoint{3.109041in}{2.414896in}}%
\pgfpathlineto{\pgfqpoint{3.109976in}{2.257626in}}%
\pgfpathlineto{\pgfqpoint{3.109644in}{2.472905in}}%
\pgfpathlineto{\pgfqpoint{3.110222in}{2.400105in}}%
\pgfpathlineto{\pgfqpoint{3.110431in}{2.449180in}}%
\pgfpathlineto{\pgfqpoint{3.110591in}{2.248991in}}%
\pgfpathlineto{\pgfqpoint{3.111281in}{2.361740in}}%
\pgfpathlineto{\pgfqpoint{3.111847in}{2.258844in}}%
\pgfpathlineto{\pgfqpoint{3.111613in}{2.469673in}}%
\pgfpathlineto{\pgfqpoint{3.112376in}{2.368671in}}%
\pgfpathlineto{\pgfqpoint{3.113201in}{2.452195in}}%
\pgfpathlineto{\pgfqpoint{3.112450in}{2.258786in}}%
\pgfpathlineto{\pgfqpoint{3.113496in}{2.423888in}}%
\pgfpathlineto{\pgfqpoint{3.113681in}{2.274427in}}%
\pgfpathlineto{\pgfqpoint{3.113988in}{2.433486in}}%
\pgfpathlineto{\pgfqpoint{3.114690in}{2.397265in}}%
\pgfpathlineto{\pgfqpoint{3.115071in}{2.430989in}}%
\pgfpathlineto{\pgfqpoint{3.114924in}{2.243952in}}%
\pgfpathlineto{\pgfqpoint{3.115773in}{2.395754in}}%
\pgfpathlineto{\pgfqpoint{3.116154in}{2.285351in}}%
\pgfpathlineto{\pgfqpoint{3.115958in}{2.451769in}}%
\pgfpathlineto{\pgfqpoint{3.116905in}{2.361327in}}%
\pgfpathlineto{\pgfqpoint{3.117533in}{2.449946in}}%
\pgfpathlineto{\pgfqpoint{3.117385in}{2.251318in}}%
\pgfpathlineto{\pgfqpoint{3.117988in}{2.296583in}}%
\pgfpathlineto{\pgfqpoint{3.118001in}{2.267232in}}%
\pgfpathlineto{\pgfqpoint{3.118321in}{2.440609in}}%
\pgfpathlineto{\pgfqpoint{3.119047in}{2.419632in}}%
\pgfpathlineto{\pgfqpoint{3.119244in}{2.272023in}}%
\pgfpathlineto{\pgfqpoint{3.119908in}{2.444720in}}%
\pgfpathlineto{\pgfqpoint{3.120191in}{2.380906in}}%
\pgfpathlineto{\pgfqpoint{3.120462in}{2.294458in}}%
\pgfpathlineto{\pgfqpoint{3.120302in}{2.453256in}}%
\pgfpathlineto{\pgfqpoint{3.121250in}{2.368654in}}%
\pgfpathlineto{\pgfqpoint{3.121890in}{2.447048in}}%
\pgfpathlineto{\pgfqpoint{3.121718in}{2.275889in}}%
\pgfpathlineto{\pgfqpoint{3.122333in}{2.311759in}}%
\pgfpathlineto{\pgfqpoint{3.122776in}{2.428143in}}%
\pgfpathlineto{\pgfqpoint{3.122924in}{2.291587in}}%
\pgfpathlineto{\pgfqpoint{3.123527in}{2.344556in}}%
\pgfpathlineto{\pgfqpoint{3.124093in}{2.317251in}}%
\pgfpathlineto{\pgfqpoint{3.123871in}{2.413475in}}%
\pgfpathlineto{\pgfqpoint{3.124598in}{2.385480in}}%
\pgfpathlineto{\pgfqpoint{3.124647in}{2.454251in}}%
\pgfpathlineto{\pgfqpoint{3.124794in}{2.294100in}}%
\pgfpathlineto{\pgfqpoint{3.125681in}{2.333800in}}%
\pgfpathlineto{\pgfqpoint{3.126222in}{2.430650in}}%
\pgfpathlineto{\pgfqpoint{3.125988in}{2.323948in}}%
\pgfpathlineto{\pgfqpoint{3.126825in}{2.399385in}}%
\pgfpathlineto{\pgfqpoint{3.127256in}{2.273712in}}%
\pgfpathlineto{\pgfqpoint{3.127404in}{2.469792in}}%
\pgfpathlineto{\pgfqpoint{3.127957in}{2.338440in}}%
\pgfpathlineto{\pgfqpoint{3.128979in}{2.443439in}}%
\pgfpathlineto{\pgfqpoint{3.128831in}{2.292016in}}%
\pgfpathlineto{\pgfqpoint{3.129065in}{2.351996in}}%
\pgfpathlineto{\pgfqpoint{3.129730in}{2.306184in}}%
\pgfpathlineto{\pgfqpoint{3.129779in}{2.463643in}}%
\pgfpathlineto{\pgfqpoint{3.130136in}{2.373554in}}%
\pgfpathlineto{\pgfqpoint{3.130567in}{2.435019in}}%
\pgfpathlineto{\pgfqpoint{3.130333in}{2.293929in}}%
\pgfpathlineto{\pgfqpoint{3.131244in}{2.391922in}}%
\pgfpathlineto{\pgfqpoint{3.131588in}{2.275041in}}%
\pgfpathlineto{\pgfqpoint{3.131748in}{2.464270in}}%
\pgfpathlineto{\pgfqpoint{3.132339in}{2.407821in}}%
\pgfpathlineto{\pgfqpoint{3.133422in}{2.307415in}}%
\pgfpathlineto{\pgfqpoint{3.132942in}{2.455685in}}%
\pgfpathlineto{\pgfqpoint{3.133471in}{2.369890in}}%
\pgfpathlineto{\pgfqpoint{3.134099in}{2.448182in}}%
\pgfpathlineto{\pgfqpoint{3.134050in}{2.306375in}}%
\pgfpathlineto{\pgfqpoint{3.134567in}{2.358017in}}%
\pgfpathlineto{\pgfqpoint{3.134665in}{2.302580in}}%
\pgfpathlineto{\pgfqpoint{3.134911in}{2.440837in}}%
\pgfpathlineto{\pgfqpoint{3.135317in}{2.381112in}}%
\pgfpathlineto{\pgfqpoint{3.136081in}{2.448638in}}%
\pgfpathlineto{\pgfqpoint{3.135908in}{2.261383in}}%
\pgfpathlineto{\pgfqpoint{3.136413in}{2.363715in}}%
\pgfpathlineto{\pgfqpoint{3.136524in}{2.288480in}}%
\pgfpathlineto{\pgfqpoint{3.136770in}{2.424235in}}%
\pgfpathlineto{\pgfqpoint{3.137508in}{2.358882in}}%
\pgfpathlineto{\pgfqpoint{3.138530in}{2.424663in}}%
\pgfpathlineto{\pgfqpoint{3.137754in}{2.291650in}}%
\pgfpathlineto{\pgfqpoint{3.138641in}{2.386735in}}%
\pgfpathlineto{\pgfqpoint{3.138997in}{2.304114in}}%
\pgfpathlineto{\pgfqpoint{3.139145in}{2.431684in}}%
\pgfpathlineto{\pgfqpoint{3.139736in}{2.373165in}}%
\pgfpathlineto{\pgfqpoint{3.140376in}{2.447180in}}%
\pgfpathlineto{\pgfqpoint{3.140216in}{2.286254in}}%
\pgfpathlineto{\pgfqpoint{3.140807in}{2.339221in}}%
\pgfpathlineto{\pgfqpoint{3.140831in}{2.303806in}}%
\pgfpathlineto{\pgfqpoint{3.141644in}{2.430155in}}%
\pgfpathlineto{\pgfqpoint{3.141890in}{2.386491in}}%
\pgfpathlineto{\pgfqpoint{3.142074in}{2.297240in}}%
\pgfpathlineto{\pgfqpoint{3.142333in}{2.442251in}}%
\pgfpathlineto{\pgfqpoint{3.143022in}{2.368481in}}%
\pgfpathlineto{\pgfqpoint{3.143490in}{2.451349in}}%
\pgfpathlineto{\pgfqpoint{3.143305in}{2.299794in}}%
\pgfpathlineto{\pgfqpoint{3.143908in}{2.341389in}}%
\pgfpathlineto{\pgfqpoint{3.143933in}{2.287002in}}%
\pgfpathlineto{\pgfqpoint{3.144179in}{2.426327in}}%
\pgfpathlineto{\pgfqpoint{3.144991in}{2.390943in}}%
\pgfpathlineto{\pgfqpoint{3.145348in}{2.420179in}}%
\pgfpathlineto{\pgfqpoint{3.145151in}{2.310857in}}%
\pgfpathlineto{\pgfqpoint{3.146062in}{2.366112in}}%
\pgfpathlineto{\pgfqpoint{3.146394in}{2.322405in}}%
\pgfpathlineto{\pgfqpoint{3.146640in}{2.435110in}}%
\pgfpathlineto{\pgfqpoint{3.147157in}{2.375877in}}%
\pgfpathlineto{\pgfqpoint{3.147810in}{2.439626in}}%
\pgfpathlineto{\pgfqpoint{3.147637in}{2.294234in}}%
\pgfpathlineto{\pgfqpoint{3.148228in}{2.344969in}}%
\pgfpathlineto{\pgfqpoint{3.148253in}{2.303926in}}%
\pgfpathlineto{\pgfqpoint{3.148917in}{2.413624in}}%
\pgfpathlineto{\pgfqpoint{3.149324in}{2.372222in}}%
\pgfpathlineto{\pgfqpoint{3.149742in}{2.430102in}}%
\pgfpathlineto{\pgfqpoint{3.149484in}{2.301817in}}%
\pgfpathlineto{\pgfqpoint{3.150431in}{2.370482in}}%
\pgfpathlineto{\pgfqpoint{3.150924in}{2.418518in}}%
\pgfpathlineto{\pgfqpoint{3.150714in}{2.315398in}}%
\pgfpathlineto{\pgfqpoint{3.151600in}{2.397187in}}%
\pgfpathlineto{\pgfqpoint{3.151945in}{2.308390in}}%
\pgfpathlineto{\pgfqpoint{3.152105in}{2.435028in}}%
\pgfpathlineto{\pgfqpoint{3.152733in}{2.379068in}}%
\pgfpathlineto{\pgfqpoint{3.153373in}{2.420867in}}%
\pgfpathlineto{\pgfqpoint{3.153188in}{2.313102in}}%
\pgfpathlineto{\pgfqpoint{3.153791in}{2.335362in}}%
\pgfpathlineto{\pgfqpoint{3.153816in}{2.289267in}}%
\pgfpathlineto{\pgfqpoint{3.154062in}{2.438008in}}%
\pgfpathlineto{\pgfqpoint{3.154887in}{2.355923in}}%
\pgfpathlineto{\pgfqpoint{3.155231in}{2.453689in}}%
\pgfpathlineto{\pgfqpoint{3.155059in}{2.320567in}}%
\pgfpathlineto{\pgfqpoint{3.155650in}{2.341475in}}%
\pgfpathlineto{\pgfqpoint{3.156265in}{2.303020in}}%
\pgfpathlineto{\pgfqpoint{3.156425in}{2.435532in}}%
\pgfpathlineto{\pgfqpoint{3.156745in}{2.364819in}}%
\pgfpathlineto{\pgfqpoint{3.157693in}{2.424115in}}%
\pgfpathlineto{\pgfqpoint{3.157520in}{2.314034in}}%
\pgfpathlineto{\pgfqpoint{3.157816in}{2.358657in}}%
\pgfpathlineto{\pgfqpoint{3.158136in}{2.295002in}}%
\pgfpathlineto{\pgfqpoint{3.158382in}{2.447553in}}%
\pgfpathlineto{\pgfqpoint{3.158874in}{2.408864in}}%
\pgfpathlineto{\pgfqpoint{3.159551in}{2.450073in}}%
\pgfpathlineto{\pgfqpoint{3.159797in}{2.318646in}}%
\pgfpathlineto{\pgfqpoint{3.159957in}{2.389347in}}%
\pgfpathlineto{\pgfqpoint{3.160585in}{2.281770in}}%
\pgfpathlineto{\pgfqpoint{3.160745in}{2.457008in}}%
\pgfpathlineto{\pgfqpoint{3.161077in}{2.339443in}}%
\pgfpathlineto{\pgfqpoint{3.161533in}{2.435952in}}%
\pgfpathlineto{\pgfqpoint{3.161840in}{2.313295in}}%
\pgfpathlineto{\pgfqpoint{3.162197in}{2.350369in}}%
\pgfpathlineto{\pgfqpoint{3.162948in}{2.292949in}}%
\pgfpathlineto{\pgfqpoint{3.162702in}{2.444671in}}%
\pgfpathlineto{\pgfqpoint{3.163096in}{2.401721in}}%
\pgfpathlineto{\pgfqpoint{3.163884in}{2.450716in}}%
\pgfpathlineto{\pgfqpoint{3.164130in}{2.306926in}}%
\pgfpathlineto{\pgfqpoint{3.164179in}{2.340951in}}%
\pgfpathlineto{\pgfqpoint{3.164930in}{2.253178in}}%
\pgfpathlineto{\pgfqpoint{3.164474in}{2.446390in}}%
\pgfpathlineto{\pgfqpoint{3.165065in}{2.440223in}}%
\pgfpathlineto{\pgfqpoint{3.165090in}{2.473576in}}%
\pgfpathlineto{\pgfqpoint{3.166111in}{2.299679in}}%
\pgfpathlineto{\pgfqpoint{3.166124in}{2.297054in}}%
\pgfpathlineto{\pgfqpoint{3.166345in}{2.440208in}}%
\pgfpathlineto{\pgfqpoint{3.166825in}{2.375336in}}%
\pgfpathlineto{\pgfqpoint{3.167563in}{2.466959in}}%
\pgfpathlineto{\pgfqpoint{3.167391in}{2.276815in}}%
\pgfpathlineto{\pgfqpoint{3.167883in}{2.356729in}}%
\pgfpathlineto{\pgfqpoint{3.168007in}{2.255062in}}%
\pgfpathlineto{\pgfqpoint{3.168253in}{2.468959in}}%
\pgfpathlineto{\pgfqpoint{3.168991in}{2.332244in}}%
\pgfpathlineto{\pgfqpoint{3.169434in}{2.493232in}}%
\pgfpathlineto{\pgfqpoint{3.169262in}{2.279690in}}%
\pgfpathlineto{\pgfqpoint{3.170136in}{2.424741in}}%
\pgfpathlineto{\pgfqpoint{3.170468in}{2.254013in}}%
\pgfpathlineto{\pgfqpoint{3.170628in}{2.469232in}}%
\pgfpathlineto{\pgfqpoint{3.171268in}{2.386147in}}%
\pgfpathlineto{\pgfqpoint{3.171416in}{2.460844in}}%
\pgfpathlineto{\pgfqpoint{3.172339in}{2.277658in}}%
\pgfpathlineto{\pgfqpoint{3.172363in}{2.310385in}}%
\pgfpathlineto{\pgfqpoint{3.172597in}{2.469805in}}%
\pgfpathlineto{\pgfqpoint{3.172843in}{2.283486in}}%
\pgfpathlineto{\pgfqpoint{3.173471in}{2.323820in}}%
\pgfpathlineto{\pgfqpoint{3.173779in}{2.463786in}}%
\pgfpathlineto{\pgfqpoint{3.173557in}{2.271242in}}%
\pgfpathlineto{\pgfqpoint{3.174603in}{2.386634in}}%
\pgfpathlineto{\pgfqpoint{3.174813in}{2.262006in}}%
\pgfpathlineto{\pgfqpoint{3.174973in}{2.481612in}}%
\pgfpathlineto{\pgfqpoint{3.175699in}{2.407631in}}%
\pgfpathlineto{\pgfqpoint{3.176007in}{2.299820in}}%
\pgfpathlineto{\pgfqpoint{3.175760in}{2.450472in}}%
\pgfpathlineto{\pgfqpoint{3.176807in}{2.385062in}}%
\pgfpathlineto{\pgfqpoint{3.176942in}{2.463232in}}%
\pgfpathlineto{\pgfqpoint{3.177188in}{2.285697in}}%
\pgfpathlineto{\pgfqpoint{3.177877in}{2.287761in}}%
\pgfpathlineto{\pgfqpoint{3.177890in}{2.266851in}}%
\pgfpathlineto{\pgfqpoint{3.178136in}{2.460355in}}%
\pgfpathlineto{\pgfqpoint{3.178899in}{2.399284in}}%
\pgfpathlineto{\pgfqpoint{3.179317in}{2.477674in}}%
\pgfpathlineto{\pgfqpoint{3.179133in}{2.276483in}}%
\pgfpathlineto{\pgfqpoint{3.180007in}{2.430550in}}%
\pgfpathlineto{\pgfqpoint{3.180979in}{2.286666in}}%
\pgfpathlineto{\pgfqpoint{3.180499in}{2.448750in}}%
\pgfpathlineto{\pgfqpoint{3.181139in}{2.406984in}}%
\pgfpathlineto{\pgfqpoint{3.181287in}{2.434740in}}%
\pgfpathlineto{\pgfqpoint{3.181508in}{2.303178in}}%
\pgfpathlineto{\pgfqpoint{3.182185in}{2.337589in}}%
\pgfpathlineto{\pgfqpoint{3.182210in}{2.286798in}}%
\pgfpathlineto{\pgfqpoint{3.182456in}{2.446409in}}%
\pgfpathlineto{\pgfqpoint{3.183280in}{2.375411in}}%
\pgfpathlineto{\pgfqpoint{3.183453in}{2.279114in}}%
\pgfpathlineto{\pgfqpoint{3.183625in}{2.449783in}}%
\pgfpathlineto{\pgfqpoint{3.184363in}{2.405808in}}%
\pgfpathlineto{\pgfqpoint{3.185299in}{2.288034in}}%
\pgfpathlineto{\pgfqpoint{3.184843in}{2.420142in}}%
\pgfpathlineto{\pgfqpoint{3.185447in}{2.394239in}}%
\pgfpathlineto{\pgfqpoint{3.185570in}{2.430545in}}%
\pgfpathlineto{\pgfqpoint{3.185816in}{2.309464in}}%
\pgfpathlineto{\pgfqpoint{3.186505in}{2.317353in}}%
\pgfpathlineto{\pgfqpoint{3.186542in}{2.295631in}}%
\pgfpathlineto{\pgfqpoint{3.186702in}{2.442924in}}%
\pgfpathlineto{\pgfqpoint{3.187330in}{2.406001in}}%
\pgfpathlineto{\pgfqpoint{3.187379in}{2.387903in}}%
\pgfpathlineto{\pgfqpoint{3.187416in}{2.409891in}}%
\pgfpathlineto{\pgfqpoint{3.187440in}{2.396168in}}%
\pgfpathlineto{\pgfqpoint{3.188388in}{2.293335in}}%
\pgfpathlineto{\pgfqpoint{3.187933in}{2.456222in}}%
\pgfpathlineto{\pgfqpoint{3.188536in}{2.396515in}}%
\pgfpathlineto{\pgfqpoint{3.188708in}{2.434511in}}%
\pgfpathlineto{\pgfqpoint{3.189016in}{2.315391in}}%
\pgfpathlineto{\pgfqpoint{3.189594in}{2.360672in}}%
\pgfpathlineto{\pgfqpoint{3.189631in}{2.302635in}}%
\pgfpathlineto{\pgfqpoint{3.189877in}{2.450307in}}%
\pgfpathlineto{\pgfqpoint{3.190714in}{2.342055in}}%
\pgfpathlineto{\pgfqpoint{3.191046in}{2.462734in}}%
\pgfpathlineto{\pgfqpoint{3.190862in}{2.284570in}}%
\pgfpathlineto{\pgfqpoint{3.191871in}{2.369598in}}%
\pgfpathlineto{\pgfqpoint{3.192093in}{2.308040in}}%
\pgfpathlineto{\pgfqpoint{3.192228in}{2.426973in}}%
\pgfpathlineto{\pgfqpoint{3.192954in}{2.389134in}}%
\pgfpathlineto{\pgfqpoint{3.193496in}{2.436124in}}%
\pgfpathlineto{\pgfqpoint{3.193951in}{2.299185in}}%
\pgfpathlineto{\pgfqpoint{3.194037in}{2.364767in}}%
\pgfpathlineto{\pgfqpoint{3.194579in}{2.327787in}}%
\pgfpathlineto{\pgfqpoint{3.194111in}{2.437991in}}%
\pgfpathlineto{\pgfqpoint{3.194825in}{2.404755in}}%
\pgfpathlineto{\pgfqpoint{3.194948in}{2.407354in}}%
\pgfpathlineto{\pgfqpoint{3.195133in}{2.335688in}}%
\pgfpathlineto{\pgfqpoint{3.195157in}{2.338490in}}%
\pgfpathlineto{\pgfqpoint{3.195810in}{2.291650in}}%
\pgfpathlineto{\pgfqpoint{3.195342in}{2.450471in}}%
\pgfpathlineto{\pgfqpoint{3.196240in}{2.354585in}}%
\pgfpathlineto{\pgfqpoint{3.197225in}{2.429774in}}%
\pgfpathlineto{\pgfqpoint{3.197028in}{2.303829in}}%
\pgfpathlineto{\pgfqpoint{3.197373in}{2.373480in}}%
\pgfpathlineto{\pgfqpoint{3.198283in}{2.297823in}}%
\pgfpathlineto{\pgfqpoint{3.197828in}{2.424672in}}%
\pgfpathlineto{\pgfqpoint{3.198419in}{2.387284in}}%
\pgfpathlineto{\pgfqpoint{3.198456in}{2.454164in}}%
\pgfpathlineto{\pgfqpoint{3.198899in}{2.293719in}}%
\pgfpathlineto{\pgfqpoint{3.199502in}{2.332131in}}%
\pgfpathlineto{\pgfqpoint{3.200142in}{2.307788in}}%
\pgfpathlineto{\pgfqpoint{3.200290in}{2.437288in}}%
\pgfpathlineto{\pgfqpoint{3.200585in}{2.354999in}}%
\pgfpathlineto{\pgfqpoint{3.200905in}{2.432463in}}%
\pgfpathlineto{\pgfqpoint{3.200757in}{2.304164in}}%
\pgfpathlineto{\pgfqpoint{3.201348in}{2.319091in}}%
\pgfpathlineto{\pgfqpoint{3.201373in}{2.291696in}}%
\pgfpathlineto{\pgfqpoint{3.201570in}{2.427972in}}%
\pgfpathlineto{\pgfqpoint{3.202431in}{2.352954in}}%
\pgfpathlineto{\pgfqpoint{3.202763in}{2.459131in}}%
\pgfpathlineto{\pgfqpoint{3.202591in}{2.283270in}}%
\pgfpathlineto{\pgfqpoint{3.203563in}{2.386870in}}%
\pgfpathlineto{\pgfqpoint{3.203834in}{2.305144in}}%
\pgfpathlineto{\pgfqpoint{3.204610in}{2.427791in}}%
\pgfpathlineto{\pgfqpoint{3.204671in}{2.388199in}}%
\pgfpathlineto{\pgfqpoint{3.204696in}{2.410466in}}%
\pgfpathlineto{\pgfqpoint{3.204720in}{2.448483in}}%
\pgfpathlineto{\pgfqpoint{3.205065in}{2.301732in}}%
\pgfpathlineto{\pgfqpoint{3.205791in}{2.371960in}}%
\pgfpathlineto{\pgfqpoint{3.205890in}{2.444856in}}%
\pgfpathlineto{\pgfqpoint{3.206136in}{2.322913in}}%
\pgfpathlineto{\pgfqpoint{3.206874in}{2.368997in}}%
\pgfpathlineto{\pgfqpoint{3.206923in}{2.253353in}}%
\pgfpathlineto{\pgfqpoint{3.207083in}{2.475720in}}%
\pgfpathlineto{\pgfqpoint{3.207969in}{2.378087in}}%
\pgfpathlineto{\pgfqpoint{3.209053in}{2.456990in}}%
\pgfpathlineto{\pgfqpoint{3.208105in}{2.296360in}}%
\pgfpathlineto{\pgfqpoint{3.209102in}{2.390382in}}%
\pgfpathlineto{\pgfqpoint{3.209397in}{2.274105in}}%
\pgfpathlineto{\pgfqpoint{3.209446in}{2.442208in}}%
\pgfpathlineto{\pgfqpoint{3.210185in}{2.400536in}}%
\pgfpathlineto{\pgfqpoint{3.210246in}{2.441657in}}%
\pgfpathlineto{\pgfqpoint{3.210493in}{2.313100in}}%
\pgfpathlineto{\pgfqpoint{3.211169in}{2.350022in}}%
\pgfpathlineto{\pgfqpoint{3.211256in}{2.279019in}}%
\pgfpathlineto{\pgfqpoint{3.211428in}{2.470536in}}%
\pgfpathlineto{\pgfqpoint{3.212216in}{2.411845in}}%
\pgfpathlineto{\pgfqpoint{3.212228in}{2.412601in}}%
\pgfpathlineto{\pgfqpoint{3.212351in}{2.369604in}}%
\pgfpathlineto{\pgfqpoint{3.212474in}{2.246574in}}%
\pgfpathlineto{\pgfqpoint{3.212634in}{2.465900in}}%
\pgfpathlineto{\pgfqpoint{3.213446in}{2.397015in}}%
\pgfpathlineto{\pgfqpoint{3.213656in}{2.290968in}}%
\pgfpathlineto{\pgfqpoint{3.213902in}{2.451622in}}%
\pgfpathlineto{\pgfqpoint{3.214554in}{2.368029in}}%
\pgfpathlineto{\pgfqpoint{3.214603in}{2.461039in}}%
\pgfpathlineto{\pgfqpoint{3.215563in}{2.266941in}}%
\pgfpathlineto{\pgfqpoint{3.215637in}{2.323603in}}%
\pgfpathlineto{\pgfqpoint{3.216043in}{2.309002in}}%
\pgfpathlineto{\pgfqpoint{3.215809in}{2.440949in}}%
\pgfpathlineto{\pgfqpoint{3.216659in}{2.383623in}}%
\pgfpathlineto{\pgfqpoint{3.216819in}{2.266182in}}%
\pgfpathlineto{\pgfqpoint{3.216991in}{2.473130in}}%
\pgfpathlineto{\pgfqpoint{3.217557in}{2.398835in}}%
\pgfpathlineto{\pgfqpoint{3.218185in}{2.467546in}}%
\pgfpathlineto{\pgfqpoint{3.218025in}{2.260540in}}%
\pgfpathlineto{\pgfqpoint{3.218640in}{2.319364in}}%
\pgfpathlineto{\pgfqpoint{3.219206in}{2.291827in}}%
\pgfpathlineto{\pgfqpoint{3.218973in}{2.446686in}}%
\pgfpathlineto{\pgfqpoint{3.219662in}{2.381433in}}%
\pgfpathlineto{\pgfqpoint{3.219908in}{2.263687in}}%
\pgfpathlineto{\pgfqpoint{3.220154in}{2.465181in}}%
\pgfpathlineto{\pgfqpoint{3.220634in}{2.395363in}}%
\pgfpathlineto{\pgfqpoint{3.221323in}{2.450124in}}%
\pgfpathlineto{\pgfqpoint{3.221102in}{2.287626in}}%
\pgfpathlineto{\pgfqpoint{3.221729in}{2.365271in}}%
\pgfpathlineto{\pgfqpoint{3.222369in}{2.247412in}}%
\pgfpathlineto{\pgfqpoint{3.222529in}{2.472023in}}%
\pgfpathlineto{\pgfqpoint{3.222813in}{2.385318in}}%
\pgfpathlineto{\pgfqpoint{3.222849in}{2.362880in}}%
\pgfpathlineto{\pgfqpoint{3.222985in}{2.285011in}}%
\pgfpathlineto{\pgfqpoint{3.223120in}{2.428595in}}%
\pgfpathlineto{\pgfqpoint{3.223969in}{2.348676in}}%
\pgfpathlineto{\pgfqpoint{3.224499in}{2.455249in}}%
\pgfpathlineto{\pgfqpoint{3.224843in}{2.285729in}}%
\pgfpathlineto{\pgfqpoint{3.225102in}{2.428196in}}%
\pgfpathlineto{\pgfqpoint{3.225446in}{2.285252in}}%
\pgfpathlineto{\pgfqpoint{3.225680in}{2.435027in}}%
\pgfpathlineto{\pgfqpoint{3.226222in}{2.406497in}}%
\pgfpathlineto{\pgfqpoint{3.226714in}{2.281588in}}%
\pgfpathlineto{\pgfqpoint{3.226862in}{2.473783in}}%
\pgfpathlineto{\pgfqpoint{3.227354in}{2.365318in}}%
\pgfpathlineto{\pgfqpoint{3.227465in}{2.437233in}}%
\pgfpathlineto{\pgfqpoint{3.227896in}{2.308248in}}%
\pgfpathlineto{\pgfqpoint{3.228462in}{2.368658in}}%
\pgfpathlineto{\pgfqpoint{3.229176in}{2.306120in}}%
\pgfpathlineto{\pgfqpoint{3.228720in}{2.450599in}}%
\pgfpathlineto{\pgfqpoint{3.229569in}{2.367995in}}%
\pgfpathlineto{\pgfqpoint{3.229779in}{2.308067in}}%
\pgfpathlineto{\pgfqpoint{3.230579in}{2.442138in}}%
\pgfpathlineto{\pgfqpoint{3.230652in}{2.382474in}}%
\pgfpathlineto{\pgfqpoint{3.231206in}{2.467506in}}%
\pgfpathlineto{\pgfqpoint{3.231034in}{2.282171in}}%
\pgfpathlineto{\pgfqpoint{3.231760in}{2.387533in}}%
\pgfpathlineto{\pgfqpoint{3.232412in}{2.440019in}}%
\pgfpathlineto{\pgfqpoint{3.232265in}{2.321663in}}%
\pgfpathlineto{\pgfqpoint{3.232806in}{2.362536in}}%
\pgfpathlineto{\pgfqpoint{3.232892in}{2.300576in}}%
\pgfpathlineto{\pgfqpoint{3.233052in}{2.440774in}}%
\pgfpathlineto{\pgfqpoint{3.233902in}{2.364455in}}%
\pgfpathlineto{\pgfqpoint{3.234308in}{2.441233in}}%
\pgfpathlineto{\pgfqpoint{3.234136in}{2.310627in}}%
\pgfpathlineto{\pgfqpoint{3.235034in}{2.396495in}}%
\pgfpathlineto{\pgfqpoint{3.235046in}{2.397172in}}%
\pgfpathlineto{\pgfqpoint{3.235206in}{2.358828in}}%
\pgfpathlineto{\pgfqpoint{3.235354in}{2.306361in}}%
\pgfpathlineto{\pgfqpoint{3.235502in}{2.455255in}}%
\pgfpathlineto{\pgfqpoint{3.236265in}{2.419950in}}%
\pgfpathlineto{\pgfqpoint{3.237200in}{2.300180in}}%
\pgfpathlineto{\pgfqpoint{3.236757in}{2.452684in}}%
\pgfpathlineto{\pgfqpoint{3.237348in}{2.414572in}}%
\pgfpathlineto{\pgfqpoint{3.237385in}{2.435504in}}%
\pgfpathlineto{\pgfqpoint{3.238369in}{2.333583in}}%
\pgfpathlineto{\pgfqpoint{3.238628in}{2.473197in}}%
\pgfpathlineto{\pgfqpoint{3.238456in}{2.296999in}}%
\pgfpathlineto{\pgfqpoint{3.239551in}{2.371525in}}%
\pgfpathlineto{\pgfqpoint{3.240314in}{2.311712in}}%
\pgfpathlineto{\pgfqpoint{3.240474in}{2.445530in}}%
\pgfpathlineto{\pgfqpoint{3.240659in}{2.368851in}}%
\pgfpathlineto{\pgfqpoint{3.241557in}{2.292526in}}%
\pgfpathlineto{\pgfqpoint{3.241114in}{2.447830in}}%
\pgfpathlineto{\pgfqpoint{3.241656in}{2.386714in}}%
\pgfpathlineto{\pgfqpoint{3.241729in}{2.446991in}}%
\pgfpathlineto{\pgfqpoint{3.242172in}{2.314014in}}%
\pgfpathlineto{\pgfqpoint{3.242726in}{2.332161in}}%
\pgfpathlineto{\pgfqpoint{3.242849in}{2.417112in}}%
\pgfpathlineto{\pgfqpoint{3.242788in}{2.303163in}}%
\pgfpathlineto{\pgfqpoint{3.242899in}{2.404035in}}%
\pgfpathlineto{\pgfqpoint{3.242936in}{2.447328in}}%
\pgfpathlineto{\pgfqpoint{3.243416in}{2.301185in}}%
\pgfpathlineto{\pgfqpoint{3.243969in}{2.342719in}}%
\pgfpathlineto{\pgfqpoint{3.244646in}{2.306656in}}%
\pgfpathlineto{\pgfqpoint{3.244794in}{2.442894in}}%
\pgfpathlineto{\pgfqpoint{3.244880in}{2.415811in}}%
\pgfpathlineto{\pgfqpoint{3.245409in}{2.445986in}}%
\pgfpathlineto{\pgfqpoint{3.245274in}{2.330516in}}%
\pgfpathlineto{\pgfqpoint{3.245754in}{2.331961in}}%
\pgfpathlineto{\pgfqpoint{3.246505in}{2.301020in}}%
\pgfpathlineto{\pgfqpoint{3.246037in}{2.473789in}}%
\pgfpathlineto{\pgfqpoint{3.246763in}{2.398564in}}%
\pgfpathlineto{\pgfqpoint{3.247736in}{2.293438in}}%
\pgfpathlineto{\pgfqpoint{3.247280in}{2.428702in}}%
\pgfpathlineto{\pgfqpoint{3.247846in}{2.394249in}}%
\pgfpathlineto{\pgfqpoint{3.247982in}{2.450212in}}%
\pgfpathlineto{\pgfqpoint{3.248339in}{2.316405in}}%
\pgfpathlineto{\pgfqpoint{3.248905in}{2.319251in}}%
\pgfpathlineto{\pgfqpoint{3.249151in}{2.466770in}}%
\pgfpathlineto{\pgfqpoint{3.248979in}{2.294698in}}%
\pgfpathlineto{\pgfqpoint{3.250074in}{2.349263in}}%
\pgfpathlineto{\pgfqpoint{3.250222in}{2.311182in}}%
\pgfpathlineto{\pgfqpoint{3.250345in}{2.454842in}}%
\pgfpathlineto{\pgfqpoint{3.251108in}{2.435887in}}%
\pgfpathlineto{\pgfqpoint{3.252068in}{2.299733in}}%
\pgfpathlineto{\pgfqpoint{3.251612in}{2.453719in}}%
\pgfpathlineto{\pgfqpoint{3.252215in}{2.420739in}}%
\pgfpathlineto{\pgfqpoint{3.252868in}{2.442187in}}%
\pgfpathlineto{\pgfqpoint{3.252609in}{2.330995in}}%
\pgfpathlineto{\pgfqpoint{3.253188in}{2.362686in}}%
\pgfpathlineto{\pgfqpoint{3.253311in}{2.269916in}}%
\pgfpathlineto{\pgfqpoint{3.253471in}{2.480489in}}%
\pgfpathlineto{\pgfqpoint{3.254246in}{2.397709in}}%
\pgfpathlineto{\pgfqpoint{3.254677in}{2.458164in}}%
\pgfpathlineto{\pgfqpoint{3.254505in}{2.293141in}}%
\pgfpathlineto{\pgfqpoint{3.255354in}{2.405397in}}%
\pgfpathlineto{\pgfqpoint{3.255686in}{2.274532in}}%
\pgfpathlineto{\pgfqpoint{3.255846in}{2.458922in}}%
\pgfpathlineto{\pgfqpoint{3.256499in}{2.347177in}}%
\pgfpathlineto{\pgfqpoint{3.256646in}{2.452509in}}%
\pgfpathlineto{\pgfqpoint{3.256880in}{2.297895in}}%
\pgfpathlineto{\pgfqpoint{3.257606in}{2.365664in}}%
\pgfpathlineto{\pgfqpoint{3.257655in}{2.258552in}}%
\pgfpathlineto{\pgfqpoint{3.257815in}{2.471648in}}%
\pgfpathlineto{\pgfqpoint{3.258702in}{2.372949in}}%
\pgfpathlineto{\pgfqpoint{3.259022in}{2.475094in}}%
\pgfpathlineto{\pgfqpoint{3.258862in}{2.264111in}}%
\pgfpathlineto{\pgfqpoint{3.259834in}{2.400371in}}%
\pgfpathlineto{\pgfqpoint{3.260745in}{2.277565in}}%
\pgfpathlineto{\pgfqpoint{3.260289in}{2.457328in}}%
\pgfpathlineto{\pgfqpoint{3.260942in}{2.374364in}}%
\pgfpathlineto{\pgfqpoint{3.260991in}{2.473503in}}%
\pgfpathlineto{\pgfqpoint{3.261237in}{2.287098in}}%
\pgfpathlineto{\pgfqpoint{3.262062in}{2.394949in}}%
\pgfpathlineto{\pgfqpoint{3.263206in}{2.246568in}}%
\pgfpathlineto{\pgfqpoint{3.262172in}{2.460617in}}%
\pgfpathlineto{\pgfqpoint{3.263231in}{2.306106in}}%
\pgfpathlineto{\pgfqpoint{3.263366in}{2.483644in}}%
\pgfpathlineto{\pgfqpoint{3.263612in}{2.295700in}}%
\pgfpathlineto{\pgfqpoint{3.264339in}{2.347070in}}%
\pgfpathlineto{\pgfqpoint{3.265225in}{2.454904in}}%
\pgfpathlineto{\pgfqpoint{3.264400in}{2.282895in}}%
\pgfpathlineto{\pgfqpoint{3.265459in}{2.362734in}}%
\pgfpathlineto{\pgfqpoint{3.266295in}{2.272038in}}%
\pgfpathlineto{\pgfqpoint{3.265742in}{2.456199in}}%
\pgfpathlineto{\pgfqpoint{3.266517in}{2.424632in}}%
\pgfpathlineto{\pgfqpoint{3.266542in}{2.446412in}}%
\pgfpathlineto{\pgfqpoint{3.267477in}{2.312815in}}%
\pgfpathlineto{\pgfqpoint{3.267526in}{2.343382in}}%
\pgfpathlineto{\pgfqpoint{3.267551in}{2.273654in}}%
\pgfpathlineto{\pgfqpoint{3.267723in}{2.486950in}}%
\pgfpathlineto{\pgfqpoint{3.268622in}{2.383693in}}%
\pgfpathlineto{\pgfqpoint{3.268757in}{2.281661in}}%
\pgfpathlineto{\pgfqpoint{3.268917in}{2.464095in}}%
\pgfpathlineto{\pgfqpoint{3.269742in}{2.370174in}}%
\pgfpathlineto{\pgfqpoint{3.270640in}{2.276405in}}%
\pgfpathlineto{\pgfqpoint{3.270086in}{2.445753in}}%
\pgfpathlineto{\pgfqpoint{3.270751in}{2.391230in}}%
\pgfpathlineto{\pgfqpoint{3.270886in}{2.447994in}}%
\pgfpathlineto{\pgfqpoint{3.271120in}{2.312411in}}%
\pgfpathlineto{\pgfqpoint{3.271846in}{2.343514in}}%
\pgfpathlineto{\pgfqpoint{3.272068in}{2.477188in}}%
\pgfpathlineto{\pgfqpoint{3.271895in}{2.287320in}}%
\pgfpathlineto{\pgfqpoint{3.272978in}{2.378245in}}%
\pgfpathlineto{\pgfqpoint{3.273102in}{2.291404in}}%
\pgfpathlineto{\pgfqpoint{3.273262in}{2.469912in}}%
\pgfpathlineto{\pgfqpoint{3.274098in}{2.350803in}}%
\pgfpathlineto{\pgfqpoint{3.274985in}{2.301002in}}%
\pgfpathlineto{\pgfqpoint{3.274431in}{2.442858in}}%
\pgfpathlineto{\pgfqpoint{3.275083in}{2.379288in}}%
\pgfpathlineto{\pgfqpoint{3.275231in}{2.454998in}}%
\pgfpathlineto{\pgfqpoint{3.275588in}{2.305160in}}%
\pgfpathlineto{\pgfqpoint{3.276178in}{2.357149in}}%
\pgfpathlineto{\pgfqpoint{3.276228in}{2.279416in}}%
\pgfpathlineto{\pgfqpoint{3.276400in}{2.468596in}}%
\pgfpathlineto{\pgfqpoint{3.277286in}{2.358008in}}%
\pgfpathlineto{\pgfqpoint{3.277594in}{2.467004in}}%
\pgfpathlineto{\pgfqpoint{3.278074in}{2.303951in}}%
\pgfpathlineto{\pgfqpoint{3.278406in}{2.390056in}}%
\pgfpathlineto{\pgfqpoint{3.278702in}{2.313323in}}%
\pgfpathlineto{\pgfqpoint{3.279477in}{2.452133in}}%
\pgfpathlineto{\pgfqpoint{3.279502in}{2.412785in}}%
\pgfpathlineto{\pgfqpoint{3.280560in}{2.290749in}}%
\pgfpathlineto{\pgfqpoint{3.280092in}{2.446867in}}%
\pgfpathlineto{\pgfqpoint{3.280609in}{2.404985in}}%
\pgfpathlineto{\pgfqpoint{3.280720in}{2.470180in}}%
\pgfpathlineto{\pgfqpoint{3.281175in}{2.302647in}}%
\pgfpathlineto{\pgfqpoint{3.281545in}{2.358009in}}%
\pgfpathlineto{\pgfqpoint{3.282406in}{2.289366in}}%
\pgfpathlineto{\pgfqpoint{3.281951in}{2.439971in}}%
\pgfpathlineto{\pgfqpoint{3.282542in}{2.418782in}}%
\pgfpathlineto{\pgfqpoint{3.282578in}{2.444248in}}%
\pgfpathlineto{\pgfqpoint{3.282800in}{2.328711in}}%
\pgfpathlineto{\pgfqpoint{3.283600in}{2.347407in}}%
\pgfpathlineto{\pgfqpoint{3.283649in}{2.288136in}}%
\pgfpathlineto{\pgfqpoint{3.283822in}{2.462979in}}%
\pgfpathlineto{\pgfqpoint{3.284695in}{2.346841in}}%
\pgfpathlineto{\pgfqpoint{3.285668in}{2.451889in}}%
\pgfpathlineto{\pgfqpoint{3.285520in}{2.314516in}}%
\pgfpathlineto{\pgfqpoint{3.285815in}{2.376191in}}%
\pgfpathlineto{\pgfqpoint{3.286751in}{2.279300in}}%
\pgfpathlineto{\pgfqpoint{3.286283in}{2.451996in}}%
\pgfpathlineto{\pgfqpoint{3.286849in}{2.392816in}}%
\pgfpathlineto{\pgfqpoint{3.286898in}{2.445402in}}%
\pgfpathlineto{\pgfqpoint{3.287157in}{2.328976in}}%
\pgfpathlineto{\pgfqpoint{3.287920in}{2.345856in}}%
\pgfpathlineto{\pgfqpoint{3.288597in}{2.283824in}}%
\pgfpathlineto{\pgfqpoint{3.288129in}{2.485676in}}%
\pgfpathlineto{\pgfqpoint{3.289003in}{2.358610in}}%
\pgfpathlineto{\pgfqpoint{3.290000in}{2.453123in}}%
\pgfpathlineto{\pgfqpoint{3.289840in}{2.311276in}}%
\pgfpathlineto{\pgfqpoint{3.290123in}{2.399285in}}%
\pgfpathlineto{\pgfqpoint{3.291083in}{2.295066in}}%
\pgfpathlineto{\pgfqpoint{3.290615in}{2.447051in}}%
\pgfpathlineto{\pgfqpoint{3.291218in}{2.427779in}}%
\pgfpathlineto{\pgfqpoint{3.291255in}{2.484883in}}%
\pgfpathlineto{\pgfqpoint{3.291711in}{2.300107in}}%
\pgfpathlineto{\pgfqpoint{3.292289in}{2.343305in}}%
\pgfpathlineto{\pgfqpoint{3.292941in}{2.285610in}}%
\pgfpathlineto{\pgfqpoint{3.292461in}{2.459848in}}%
\pgfpathlineto{\pgfqpoint{3.293372in}{2.375351in}}%
\pgfpathlineto{\pgfqpoint{3.293385in}{2.375422in}}%
\pgfpathlineto{\pgfqpoint{3.293397in}{2.370234in}}%
\pgfpathlineto{\pgfqpoint{3.293557in}{2.289399in}}%
\pgfpathlineto{\pgfqpoint{3.293606in}{2.459228in}}%
\pgfpathlineto{\pgfqpoint{3.294529in}{2.333435in}}%
\pgfpathlineto{\pgfqpoint{3.295563in}{2.498166in}}%
\pgfpathlineto{\pgfqpoint{3.295403in}{2.284474in}}%
\pgfpathlineto{\pgfqpoint{3.295674in}{2.404846in}}%
\pgfpathlineto{\pgfqpoint{3.296031in}{2.287649in}}%
\pgfpathlineto{\pgfqpoint{3.296720in}{2.435385in}}%
\pgfpathlineto{\pgfqpoint{3.296769in}{2.420132in}}%
\pgfpathlineto{\pgfqpoint{3.297532in}{2.465141in}}%
\pgfpathlineto{\pgfqpoint{3.297274in}{2.271169in}}%
\pgfpathlineto{\pgfqpoint{3.297803in}{2.353945in}}%
\pgfpathlineto{\pgfqpoint{3.298689in}{2.475272in}}%
\pgfpathlineto{\pgfqpoint{3.298517in}{2.289925in}}%
\pgfpathlineto{\pgfqpoint{3.298874in}{2.347727in}}%
\pgfpathlineto{\pgfqpoint{3.299735in}{2.276381in}}%
\pgfpathlineto{\pgfqpoint{3.299280in}{2.453553in}}%
\pgfpathlineto{\pgfqpoint{3.299846in}{2.394233in}}%
\pgfpathlineto{\pgfqpoint{3.299883in}{2.510812in}}%
\pgfpathlineto{\pgfqpoint{3.300880in}{2.278234in}}%
\pgfpathlineto{\pgfqpoint{3.300941in}{2.346808in}}%
\pgfpathlineto{\pgfqpoint{3.301594in}{2.273213in}}%
\pgfpathlineto{\pgfqpoint{3.301040in}{2.487562in}}%
\pgfpathlineto{\pgfqpoint{3.302061in}{2.304300in}}%
\pgfpathlineto{\pgfqpoint{3.302074in}{2.304420in}}%
\pgfpathlineto{\pgfqpoint{3.303009in}{2.543111in}}%
\pgfpathlineto{\pgfqpoint{3.302837in}{2.212512in}}%
\pgfpathlineto{\pgfqpoint{3.303194in}{2.354978in}}%
\pgfpathlineto{\pgfqpoint{3.304068in}{2.262474in}}%
\pgfpathlineto{\pgfqpoint{3.304191in}{2.501483in}}%
\pgfpathlineto{\pgfqpoint{3.304289in}{2.395248in}}%
\pgfpathlineto{\pgfqpoint{3.304683in}{2.260692in}}%
\pgfpathlineto{\pgfqpoint{3.304855in}{2.481721in}}%
\pgfpathlineto{\pgfqpoint{3.304941in}{2.457625in}}%
\pgfpathlineto{\pgfqpoint{3.305360in}{2.514766in}}%
\pgfpathlineto{\pgfqpoint{3.305200in}{2.256718in}}%
\pgfpathlineto{\pgfqpoint{3.306000in}{2.364578in}}%
\pgfpathlineto{\pgfqpoint{3.306086in}{2.511118in}}%
\pgfpathlineto{\pgfqpoint{3.306357in}{2.301068in}}%
\pgfpathlineto{\pgfqpoint{3.307034in}{2.334256in}}%
\pgfpathlineto{\pgfqpoint{3.307157in}{2.175025in}}%
\pgfpathlineto{\pgfqpoint{3.307317in}{2.565749in}}%
\pgfpathlineto{\pgfqpoint{3.308117in}{2.405207in}}%
\pgfpathlineto{\pgfqpoint{3.309003in}{2.216054in}}%
\pgfpathlineto{\pgfqpoint{3.308548in}{2.496420in}}%
\pgfpathlineto{\pgfqpoint{3.309163in}{2.490459in}}%
\pgfpathlineto{\pgfqpoint{3.309803in}{2.511285in}}%
\pgfpathlineto{\pgfqpoint{3.309631in}{2.262301in}}%
\pgfpathlineto{\pgfqpoint{3.310135in}{2.307662in}}%
\pgfpathlineto{\pgfqpoint{3.310258in}{2.200758in}}%
\pgfpathlineto{\pgfqpoint{3.310406in}{2.524953in}}%
\pgfpathlineto{\pgfqpoint{3.310418in}{2.560156in}}%
\pgfpathlineto{\pgfqpoint{3.310874in}{2.238867in}}%
\pgfpathlineto{\pgfqpoint{3.311440in}{2.311352in}}%
\pgfpathlineto{\pgfqpoint{3.311489in}{2.203140in}}%
\pgfpathlineto{\pgfqpoint{3.311649in}{2.544539in}}%
\pgfpathlineto{\pgfqpoint{3.312535in}{2.352576in}}%
\pgfpathlineto{\pgfqpoint{3.313508in}{2.527339in}}%
\pgfpathlineto{\pgfqpoint{3.313348in}{2.198185in}}%
\pgfpathlineto{\pgfqpoint{3.313643in}{2.368160in}}%
\pgfpathlineto{\pgfqpoint{3.314591in}{2.225962in}}%
\pgfpathlineto{\pgfqpoint{3.314123in}{2.483854in}}%
\pgfpathlineto{\pgfqpoint{3.314701in}{2.386391in}}%
\pgfpathlineto{\pgfqpoint{3.314751in}{2.545642in}}%
\pgfpathlineto{\pgfqpoint{3.315206in}{2.220239in}}%
\pgfpathlineto{\pgfqpoint{3.315784in}{2.304971in}}%
\pgfpathlineto{\pgfqpoint{3.315821in}{2.238028in}}%
\pgfpathlineto{\pgfqpoint{3.315981in}{2.483639in}}%
\pgfpathlineto{\pgfqpoint{3.316868in}{2.349469in}}%
\pgfpathlineto{\pgfqpoint{3.317852in}{2.487533in}}%
\pgfpathlineto{\pgfqpoint{3.317052in}{2.252936in}}%
\pgfpathlineto{\pgfqpoint{3.317988in}{2.382345in}}%
\pgfpathlineto{\pgfqpoint{3.318911in}{2.262118in}}%
\pgfpathlineto{\pgfqpoint{3.318468in}{2.476140in}}%
\pgfpathlineto{\pgfqpoint{3.319034in}{2.410335in}}%
\pgfpathlineto{\pgfqpoint{3.319071in}{2.474710in}}%
\pgfpathlineto{\pgfqpoint{3.319526in}{2.293508in}}%
\pgfpathlineto{\pgfqpoint{3.320104in}{2.341293in}}%
\pgfpathlineto{\pgfqpoint{3.320769in}{2.271958in}}%
\pgfpathlineto{\pgfqpoint{3.320314in}{2.475010in}}%
\pgfpathlineto{\pgfqpoint{3.321188in}{2.354500in}}%
\pgfpathlineto{\pgfqpoint{3.322184in}{2.462865in}}%
\pgfpathlineto{\pgfqpoint{3.321384in}{2.301759in}}%
\pgfpathlineto{\pgfqpoint{3.322308in}{2.383675in}}%
\pgfpathlineto{\pgfqpoint{3.323243in}{2.285729in}}%
\pgfpathlineto{\pgfqpoint{3.322788in}{2.453137in}}%
\pgfpathlineto{\pgfqpoint{3.323366in}{2.399144in}}%
\pgfpathlineto{\pgfqpoint{3.323428in}{2.457908in}}%
\pgfpathlineto{\pgfqpoint{3.323871in}{2.312226in}}%
\pgfpathlineto{\pgfqpoint{3.324437in}{2.346776in}}%
\pgfpathlineto{\pgfqpoint{3.324486in}{2.296087in}}%
\pgfpathlineto{\pgfqpoint{3.324646in}{2.446592in}}%
\pgfpathlineto{\pgfqpoint{3.325532in}{2.365241in}}%
\pgfpathlineto{\pgfqpoint{3.325729in}{2.324656in}}%
\pgfpathlineto{\pgfqpoint{3.325840in}{2.396938in}}%
\pgfpathlineto{\pgfqpoint{3.325889in}{2.450883in}}%
\pgfpathlineto{\pgfqpoint{3.326357in}{2.302722in}}%
\pgfpathlineto{\pgfqpoint{3.326923in}{2.366181in}}%
\pgfpathlineto{\pgfqpoint{3.327588in}{2.301325in}}%
\pgfpathlineto{\pgfqpoint{3.327735in}{2.449480in}}%
\pgfpathlineto{\pgfqpoint{3.327748in}{2.455491in}}%
\pgfpathlineto{\pgfqpoint{3.327969in}{2.317105in}}%
\pgfpathlineto{\pgfqpoint{3.328658in}{2.364569in}}%
\pgfpathlineto{\pgfqpoint{3.329458in}{2.321888in}}%
\pgfpathlineto{\pgfqpoint{3.329003in}{2.451794in}}%
\pgfpathlineto{\pgfqpoint{3.329729in}{2.403531in}}%
\pgfpathlineto{\pgfqpoint{3.330197in}{2.428231in}}%
\pgfpathlineto{\pgfqpoint{3.329926in}{2.333493in}}%
\pgfpathlineto{\pgfqpoint{3.330640in}{2.350254in}}%
\pgfpathlineto{\pgfqpoint{3.330689in}{2.310669in}}%
\pgfpathlineto{\pgfqpoint{3.330861in}{2.436375in}}%
\pgfpathlineto{\pgfqpoint{3.331698in}{2.384142in}}%
\pgfpathlineto{\pgfqpoint{3.332104in}{2.436642in}}%
\pgfpathlineto{\pgfqpoint{3.331920in}{2.316036in}}%
\pgfpathlineto{\pgfqpoint{3.332806in}{2.390660in}}%
\pgfpathlineto{\pgfqpoint{3.333311in}{2.434603in}}%
\pgfpathlineto{\pgfqpoint{3.333052in}{2.344791in}}%
\pgfpathlineto{\pgfqpoint{3.333495in}{2.348629in}}%
\pgfpathlineto{\pgfqpoint{3.334184in}{2.338867in}}%
\pgfpathlineto{\pgfqpoint{3.333963in}{2.414947in}}%
\pgfpathlineto{\pgfqpoint{3.334492in}{2.388961in}}%
\pgfpathlineto{\pgfqpoint{3.335255in}{2.418031in}}%
\pgfpathlineto{\pgfqpoint{3.335403in}{2.327150in}}%
\pgfpathlineto{\pgfqpoint{3.335600in}{2.390570in}}%
\pgfpathlineto{\pgfqpoint{3.336424in}{2.425658in}}%
\pgfpathlineto{\pgfqpoint{3.336154in}{2.344562in}}%
\pgfpathlineto{\pgfqpoint{3.336523in}{2.366586in}}%
\pgfpathlineto{\pgfqpoint{3.336634in}{2.336354in}}%
\pgfpathlineto{\pgfqpoint{3.337101in}{2.408391in}}%
\pgfpathlineto{\pgfqpoint{3.337594in}{2.393526in}}%
\pgfpathlineto{\pgfqpoint{3.338283in}{2.412289in}}%
\pgfpathlineto{\pgfqpoint{3.338504in}{2.335432in}}%
\pgfpathlineto{\pgfqpoint{3.338677in}{2.384970in}}%
\pgfpathlineto{\pgfqpoint{3.339747in}{2.343850in}}%
\pgfpathlineto{\pgfqpoint{3.339526in}{2.417877in}}%
\pgfpathlineto{\pgfqpoint{3.339821in}{2.375158in}}%
\pgfpathlineto{\pgfqpoint{3.340757in}{2.410251in}}%
\pgfpathlineto{\pgfqpoint{3.340474in}{2.353358in}}%
\pgfpathlineto{\pgfqpoint{3.340904in}{2.368104in}}%
\pgfpathlineto{\pgfqpoint{3.341606in}{2.342632in}}%
\pgfpathlineto{\pgfqpoint{3.341409in}{2.404899in}}%
\pgfpathlineto{\pgfqpoint{3.341914in}{2.379100in}}%
\pgfpathlineto{\pgfqpoint{3.342640in}{2.409347in}}%
\pgfpathlineto{\pgfqpoint{3.342824in}{2.346282in}}%
\pgfpathlineto{\pgfqpoint{3.343034in}{2.386870in}}%
\pgfpathlineto{\pgfqpoint{3.343969in}{2.346021in}}%
\pgfpathlineto{\pgfqpoint{3.343821in}{2.411459in}}%
\pgfpathlineto{\pgfqpoint{3.344191in}{2.381649in}}%
\pgfpathlineto{\pgfqpoint{3.345052in}{2.405238in}}%
\pgfpathlineto{\pgfqpoint{3.344744in}{2.343103in}}%
\pgfpathlineto{\pgfqpoint{3.345187in}{2.368927in}}%
\pgfpathlineto{\pgfqpoint{3.345926in}{2.334040in}}%
\pgfpathlineto{\pgfqpoint{3.345717in}{2.409299in}}%
\pgfpathlineto{\pgfqpoint{3.346246in}{2.393422in}}%
\pgfpathlineto{\pgfqpoint{3.346258in}{2.393988in}}%
\pgfpathlineto{\pgfqpoint{3.346431in}{2.366030in}}%
\pgfpathlineto{\pgfqpoint{3.346652in}{2.368060in}}%
\pgfpathlineto{\pgfqpoint{3.346701in}{2.343862in}}%
\pgfpathlineto{\pgfqpoint{3.346947in}{2.413938in}}%
\pgfpathlineto{\pgfqpoint{3.347747in}{2.376745in}}%
\pgfpathlineto{\pgfqpoint{3.348387in}{2.347294in}}%
\pgfpathlineto{\pgfqpoint{3.348178in}{2.414560in}}%
\pgfpathlineto{\pgfqpoint{3.348794in}{2.399951in}}%
\pgfpathlineto{\pgfqpoint{3.349360in}{2.414828in}}%
\pgfpathlineto{\pgfqpoint{3.349040in}{2.343003in}}%
\pgfpathlineto{\pgfqpoint{3.349778in}{2.358091in}}%
\pgfpathlineto{\pgfqpoint{3.349815in}{2.344843in}}%
\pgfpathlineto{\pgfqpoint{3.350012in}{2.414046in}}%
\pgfpathlineto{\pgfqpoint{3.350886in}{2.359647in}}%
\pgfpathlineto{\pgfqpoint{3.351255in}{2.414774in}}%
\pgfpathlineto{\pgfqpoint{3.351501in}{2.341534in}}%
\pgfpathlineto{\pgfqpoint{3.352043in}{2.370157in}}%
\pgfpathlineto{\pgfqpoint{3.352067in}{2.371372in}}%
\pgfpathlineto{\pgfqpoint{3.352264in}{2.340826in}}%
\pgfpathlineto{\pgfqpoint{3.352277in}{2.336407in}}%
\pgfpathlineto{\pgfqpoint{3.353138in}{2.423902in}}%
\pgfpathlineto{\pgfqpoint{3.353347in}{2.341694in}}%
\pgfpathlineto{\pgfqpoint{3.354320in}{2.422346in}}%
\pgfpathlineto{\pgfqpoint{3.354135in}{2.338170in}}%
\pgfpathlineto{\pgfqpoint{3.354492in}{2.357423in}}%
\pgfpathlineto{\pgfqpoint{3.355390in}{2.339090in}}%
\pgfpathlineto{\pgfqpoint{3.354960in}{2.411056in}}%
\pgfpathlineto{\pgfqpoint{3.355501in}{2.398667in}}%
\pgfpathlineto{\pgfqpoint{3.355612in}{2.421457in}}%
\pgfpathlineto{\pgfqpoint{3.355809in}{2.348813in}}%
\pgfpathlineto{\pgfqpoint{3.356510in}{2.356896in}}%
\pgfpathlineto{\pgfqpoint{3.357446in}{2.423296in}}%
\pgfpathlineto{\pgfqpoint{3.356609in}{2.338888in}}%
\pgfpathlineto{\pgfqpoint{3.357643in}{2.364268in}}%
\pgfpathlineto{\pgfqpoint{3.358480in}{2.346543in}}%
\pgfpathlineto{\pgfqpoint{3.358074in}{2.409045in}}%
\pgfpathlineto{\pgfqpoint{3.358590in}{2.387488in}}%
\pgfpathlineto{\pgfqpoint{3.358689in}{2.413505in}}%
\pgfpathlineto{\pgfqpoint{3.359095in}{2.350089in}}%
\pgfpathlineto{\pgfqpoint{3.359661in}{2.359402in}}%
\pgfpathlineto{\pgfqpoint{3.359710in}{2.336141in}}%
\pgfpathlineto{\pgfqpoint{3.360560in}{2.417192in}}%
\pgfpathlineto{\pgfqpoint{3.360757in}{2.362264in}}%
\pgfpathlineto{\pgfqpoint{3.361766in}{2.417590in}}%
\pgfpathlineto{\pgfqpoint{3.361581in}{2.345537in}}%
\pgfpathlineto{\pgfqpoint{3.361914in}{2.374102in}}%
\pgfpathlineto{\pgfqpoint{3.362824in}{2.336647in}}%
\pgfpathlineto{\pgfqpoint{3.362394in}{2.402161in}}%
\pgfpathlineto{\pgfqpoint{3.362972in}{2.399950in}}%
\pgfpathlineto{\pgfqpoint{3.363034in}{2.412686in}}%
\pgfpathlineto{\pgfqpoint{3.363218in}{2.347523in}}%
\pgfpathlineto{\pgfqpoint{3.364043in}{2.343105in}}%
\pgfpathlineto{\pgfqpoint{3.363686in}{2.397614in}}%
\pgfpathlineto{\pgfqpoint{3.364092in}{2.369992in}}%
\pgfpathlineto{\pgfqpoint{3.364252in}{2.412367in}}%
\pgfpathlineto{\pgfqpoint{3.364424in}{2.353899in}}%
\pgfpathlineto{\pgfqpoint{3.365163in}{2.356391in}}%
\pgfpathlineto{\pgfqpoint{3.365914in}{2.348605in}}%
\pgfpathlineto{\pgfqpoint{3.366086in}{2.406670in}}%
\pgfpathlineto{\pgfqpoint{3.366160in}{2.393243in}}%
\pgfpathlineto{\pgfqpoint{3.367144in}{2.348594in}}%
\pgfpathlineto{\pgfqpoint{3.366726in}{2.400822in}}%
\pgfpathlineto{\pgfqpoint{3.367280in}{2.390422in}}%
\pgfpathlineto{\pgfqpoint{3.367366in}{2.412681in}}%
\pgfpathlineto{\pgfqpoint{3.367563in}{2.354828in}}%
\pgfpathlineto{\pgfqpoint{3.368240in}{2.360685in}}%
\pgfpathlineto{\pgfqpoint{3.369015in}{2.352962in}}%
\pgfpathlineto{\pgfqpoint{3.369212in}{2.411403in}}%
\pgfpathlineto{\pgfqpoint{3.369298in}{2.380775in}}%
\pgfpathlineto{\pgfqpoint{3.370492in}{2.407363in}}%
\pgfpathlineto{\pgfqpoint{3.369434in}{2.350442in}}%
\pgfpathlineto{\pgfqpoint{3.370504in}{2.403478in}}%
\pgfpathlineto{\pgfqpoint{3.370664in}{2.350921in}}%
\pgfpathlineto{\pgfqpoint{3.371637in}{2.392147in}}%
\pgfpathlineto{\pgfqpoint{3.371698in}{2.415731in}}%
\pgfpathlineto{\pgfqpoint{3.371895in}{2.347624in}}%
\pgfpathlineto{\pgfqpoint{3.372695in}{2.364431in}}%
\pgfpathlineto{\pgfqpoint{3.373089in}{2.353196in}}%
\pgfpathlineto{\pgfqpoint{3.372929in}{2.405077in}}%
\pgfpathlineto{\pgfqpoint{3.373507in}{2.387766in}}%
\pgfpathlineto{\pgfqpoint{3.373544in}{2.398987in}}%
\pgfpathlineto{\pgfqpoint{3.374357in}{2.348896in}}%
\pgfpathlineto{\pgfqpoint{3.374553in}{2.368085in}}%
\pgfpathlineto{\pgfqpoint{3.374984in}{2.347348in}}%
\pgfpathlineto{\pgfqpoint{3.374800in}{2.411411in}}%
\pgfpathlineto{\pgfqpoint{3.375661in}{2.363163in}}%
\pgfpathlineto{\pgfqpoint{3.375993in}{2.401588in}}%
\pgfpathlineto{\pgfqpoint{3.376240in}{2.345924in}}%
\pgfpathlineto{\pgfqpoint{3.376781in}{2.373740in}}%
\pgfpathlineto{\pgfqpoint{3.377483in}{2.343894in}}%
\pgfpathlineto{\pgfqpoint{3.377249in}{2.401191in}}%
\pgfpathlineto{\pgfqpoint{3.377864in}{2.389555in}}%
\pgfpathlineto{\pgfqpoint{3.377926in}{2.403222in}}%
\pgfpathlineto{\pgfqpoint{3.378098in}{2.341112in}}%
\pgfpathlineto{\pgfqpoint{3.378849in}{2.372656in}}%
\pgfpathlineto{\pgfqpoint{3.379341in}{2.347573in}}%
\pgfpathlineto{\pgfqpoint{3.379107in}{2.405880in}}%
\pgfpathlineto{\pgfqpoint{3.380043in}{2.356457in}}%
\pgfpathlineto{\pgfqpoint{3.380363in}{2.402306in}}%
\pgfpathlineto{\pgfqpoint{3.380572in}{2.347760in}}%
\pgfpathlineto{\pgfqpoint{3.381150in}{2.363313in}}%
\pgfpathlineto{\pgfqpoint{3.381200in}{2.348289in}}%
\pgfpathlineto{\pgfqpoint{3.381446in}{2.395252in}}%
\pgfpathlineto{\pgfqpoint{3.382184in}{2.386616in}}%
\pgfpathlineto{\pgfqpoint{3.382233in}{2.403149in}}%
\pgfpathlineto{\pgfqpoint{3.382430in}{2.350550in}}%
\pgfpathlineto{\pgfqpoint{3.383267in}{2.377447in}}%
\pgfpathlineto{\pgfqpoint{3.383477in}{2.397041in}}%
\pgfpathlineto{\pgfqpoint{3.383698in}{2.352594in}}%
\pgfpathlineto{\pgfqpoint{3.384203in}{2.371520in}}%
\pgfpathlineto{\pgfqpoint{3.384892in}{2.349650in}}%
\pgfpathlineto{\pgfqpoint{3.384695in}{2.396526in}}%
\pgfpathlineto{\pgfqpoint{3.385273in}{2.384151in}}%
\pgfpathlineto{\pgfqpoint{3.385347in}{2.401804in}}%
\pgfpathlineto{\pgfqpoint{3.385520in}{2.352413in}}%
\pgfpathlineto{\pgfqpoint{3.386307in}{2.374006in}}%
\pgfpathlineto{\pgfqpoint{3.386750in}{2.349442in}}%
\pgfpathlineto{\pgfqpoint{3.386529in}{2.400615in}}%
\pgfpathlineto{\pgfqpoint{3.387489in}{2.365610in}}%
\pgfpathlineto{\pgfqpoint{3.388400in}{2.399509in}}%
\pgfpathlineto{\pgfqpoint{3.388006in}{2.357521in}}%
\pgfpathlineto{\pgfqpoint{3.388523in}{2.369905in}}%
\pgfpathlineto{\pgfqpoint{3.388646in}{2.345821in}}%
\pgfpathlineto{\pgfqpoint{3.388880in}{2.401700in}}%
\pgfpathlineto{\pgfqpoint{3.389606in}{2.377902in}}%
\pgfpathlineto{\pgfqpoint{3.389680in}{2.400903in}}%
\pgfpathlineto{\pgfqpoint{3.389926in}{2.348640in}}%
\pgfpathlineto{\pgfqpoint{3.390627in}{2.353171in}}%
\pgfpathlineto{\pgfqpoint{3.391046in}{2.321604in}}%
\pgfpathlineto{\pgfqpoint{3.391353in}{2.416617in}}%
\pgfpathlineto{\pgfqpoint{3.391723in}{2.353519in}}%
\pgfpathlineto{\pgfqpoint{3.392240in}{2.410901in}}%
\pgfpathlineto{\pgfqpoint{3.392067in}{2.337127in}}%
\pgfpathlineto{\pgfqpoint{3.392830in}{2.355167in}}%
\pgfpathlineto{\pgfqpoint{3.392855in}{2.312947in}}%
\pgfpathlineto{\pgfqpoint{3.393876in}{2.434991in}}%
\pgfpathlineto{\pgfqpoint{3.393938in}{2.359485in}}%
\pgfpathlineto{\pgfqpoint{3.394098in}{2.395874in}}%
\pgfpathlineto{\pgfqpoint{3.394566in}{2.430557in}}%
\pgfpathlineto{\pgfqpoint{3.394172in}{2.304048in}}%
\pgfpathlineto{\pgfqpoint{3.394787in}{2.319099in}}%
\pgfpathlineto{\pgfqpoint{3.394836in}{2.316629in}}%
\pgfpathlineto{\pgfqpoint{3.394873in}{2.353782in}}%
\pgfpathlineto{\pgfqpoint{3.395280in}{2.447053in}}%
\pgfpathlineto{\pgfqpoint{3.394996in}{2.291054in}}%
\pgfpathlineto{\pgfqpoint{3.395956in}{2.328010in}}%
\pgfpathlineto{\pgfqpoint{3.396670in}{2.272039in}}%
\pgfpathlineto{\pgfqpoint{3.396486in}{2.436280in}}%
\pgfpathlineto{\pgfqpoint{3.396990in}{2.405407in}}%
\pgfpathlineto{\pgfqpoint{3.397064in}{2.467443in}}%
\pgfpathlineto{\pgfqpoint{3.397175in}{2.303364in}}%
\pgfpathlineto{\pgfqpoint{3.397864in}{2.309746in}}%
\pgfpathlineto{\pgfqpoint{3.397876in}{2.291520in}}%
\pgfpathlineto{\pgfqpoint{3.398640in}{2.434759in}}%
\pgfpathlineto{\pgfqpoint{3.398947in}{2.343773in}}%
\pgfpathlineto{\pgfqpoint{3.399427in}{2.460682in}}%
\pgfpathlineto{\pgfqpoint{3.399760in}{2.205762in}}%
\pgfpathlineto{\pgfqpoint{3.400080in}{2.394506in}}%
\pgfpathlineto{\pgfqpoint{3.400966in}{2.291620in}}%
\pgfpathlineto{\pgfqpoint{3.401027in}{2.455515in}}%
\pgfpathlineto{\pgfqpoint{3.401187in}{2.368689in}}%
\pgfpathlineto{\pgfqpoint{3.402221in}{2.485459in}}%
\pgfpathlineto{\pgfqpoint{3.402160in}{2.243992in}}%
\pgfpathlineto{\pgfqpoint{3.402307in}{2.404784in}}%
\pgfpathlineto{\pgfqpoint{3.403009in}{2.471048in}}%
\pgfpathlineto{\pgfqpoint{3.402369in}{2.275516in}}%
\pgfpathlineto{\pgfqpoint{3.403292in}{2.375769in}}%
\pgfpathlineto{\pgfqpoint{3.403846in}{2.279495in}}%
\pgfpathlineto{\pgfqpoint{3.404203in}{2.464295in}}%
\pgfpathlineto{\pgfqpoint{3.404387in}{2.404669in}}%
\pgfpathlineto{\pgfqpoint{3.404535in}{2.282706in}}%
\pgfpathlineto{\pgfqpoint{3.404596in}{2.485090in}}%
\pgfpathlineto{\pgfqpoint{3.405360in}{2.408501in}}%
\pgfpathlineto{\pgfqpoint{3.405803in}{2.469034in}}%
\pgfpathlineto{\pgfqpoint{3.405507in}{2.293197in}}%
\pgfpathlineto{\pgfqpoint{3.406443in}{2.337707in}}%
\pgfpathlineto{\pgfqpoint{3.406504in}{2.484471in}}%
\pgfpathlineto{\pgfqpoint{3.407144in}{2.300679in}}%
\pgfpathlineto{\pgfqpoint{3.407587in}{2.385628in}}%
\pgfpathlineto{\pgfqpoint{3.408609in}{2.280008in}}%
\pgfpathlineto{\pgfqpoint{3.408166in}{2.437561in}}%
\pgfpathlineto{\pgfqpoint{3.408683in}{2.398146in}}%
\pgfpathlineto{\pgfqpoint{3.409323in}{2.313183in}}%
\pgfpathlineto{\pgfqpoint{3.408953in}{2.463338in}}%
\pgfpathlineto{\pgfqpoint{3.409618in}{2.418158in}}%
\pgfpathlineto{\pgfqpoint{3.410418in}{2.464312in}}%
\pgfpathlineto{\pgfqpoint{3.409778in}{2.298208in}}%
\pgfpathlineto{\pgfqpoint{3.410689in}{2.402032in}}%
\pgfpathlineto{\pgfqpoint{3.410984in}{2.230961in}}%
\pgfpathlineto{\pgfqpoint{3.411144in}{2.451867in}}%
\pgfpathlineto{\pgfqpoint{3.411809in}{2.362083in}}%
\pgfpathlineto{\pgfqpoint{3.412695in}{2.442058in}}%
\pgfpathlineto{\pgfqpoint{3.412153in}{2.299922in}}%
\pgfpathlineto{\pgfqpoint{3.412867in}{2.314721in}}%
\pgfpathlineto{\pgfqpoint{3.413347in}{2.251602in}}%
\pgfpathlineto{\pgfqpoint{3.413520in}{2.462299in}}%
\pgfpathlineto{\pgfqpoint{3.413901in}{2.412941in}}%
\pgfpathlineto{\pgfqpoint{3.414541in}{2.286923in}}%
\pgfpathlineto{\pgfqpoint{3.414332in}{2.448326in}}%
\pgfpathlineto{\pgfqpoint{3.415033in}{2.389812in}}%
\pgfpathlineto{\pgfqpoint{3.415883in}{2.462594in}}%
\pgfpathlineto{\pgfqpoint{3.415255in}{2.279970in}}%
\pgfpathlineto{\pgfqpoint{3.416129in}{2.392594in}}%
\pgfpathlineto{\pgfqpoint{3.416449in}{2.294302in}}%
\pgfpathlineto{\pgfqpoint{3.416596in}{2.436309in}}%
\pgfpathlineto{\pgfqpoint{3.417249in}{2.353963in}}%
\pgfpathlineto{\pgfqpoint{3.418221in}{2.484918in}}%
\pgfpathlineto{\pgfqpoint{3.417606in}{2.296850in}}%
\pgfpathlineto{\pgfqpoint{3.418344in}{2.344965in}}%
\pgfpathlineto{\pgfqpoint{3.419267in}{2.291100in}}%
\pgfpathlineto{\pgfqpoint{3.418923in}{2.445502in}}%
\pgfpathlineto{\pgfqpoint{3.419366in}{2.379095in}}%
\pgfpathlineto{\pgfqpoint{3.419427in}{2.456245in}}%
\pgfpathlineto{\pgfqpoint{3.419563in}{2.313407in}}%
\pgfpathlineto{\pgfqpoint{3.420399in}{2.366500in}}%
\pgfpathlineto{\pgfqpoint{3.420916in}{2.297992in}}%
\pgfpathlineto{\pgfqpoint{3.421286in}{2.493458in}}%
\pgfpathlineto{\pgfqpoint{3.421507in}{2.368565in}}%
\pgfpathlineto{\pgfqpoint{3.421778in}{2.477612in}}%
\pgfpathlineto{\pgfqpoint{3.421618in}{2.296799in}}%
\pgfpathlineto{\pgfqpoint{3.422627in}{2.406629in}}%
\pgfpathlineto{\pgfqpoint{3.423046in}{2.290860in}}%
\pgfpathlineto{\pgfqpoint{3.423624in}{2.470200in}}%
\pgfpathlineto{\pgfqpoint{3.423784in}{2.333749in}}%
\pgfpathlineto{\pgfqpoint{3.424338in}{2.471605in}}%
\pgfpathlineto{\pgfqpoint{3.423932in}{2.287641in}}%
\pgfpathlineto{\pgfqpoint{3.424879in}{2.346280in}}%
\pgfpathlineto{\pgfqpoint{3.424904in}{2.308810in}}%
\pgfpathlineto{\pgfqpoint{3.425187in}{2.483681in}}%
\pgfpathlineto{\pgfqpoint{3.425926in}{2.422597in}}%
\pgfpathlineto{\pgfqpoint{3.425963in}{2.495047in}}%
\pgfpathlineto{\pgfqpoint{3.426123in}{2.304883in}}%
\pgfpathlineto{\pgfqpoint{3.427009in}{2.368417in}}%
\pgfpathlineto{\pgfqpoint{3.427316in}{2.234548in}}%
\pgfpathlineto{\pgfqpoint{3.427538in}{2.486918in}}%
\pgfpathlineto{\pgfqpoint{3.428104in}{2.370903in}}%
\pgfpathlineto{\pgfqpoint{3.428719in}{2.474903in}}%
\pgfpathlineto{\pgfqpoint{3.428867in}{2.279378in}}%
\pgfpathlineto{\pgfqpoint{3.429199in}{2.342507in}}%
\pgfpathlineto{\pgfqpoint{3.429729in}{2.283451in}}%
\pgfpathlineto{\pgfqpoint{3.429864in}{2.514686in}}%
\pgfpathlineto{\pgfqpoint{3.430295in}{2.373570in}}%
\pgfpathlineto{\pgfqpoint{3.430639in}{2.469844in}}%
\pgfpathlineto{\pgfqpoint{3.430418in}{2.263469in}}%
\pgfpathlineto{\pgfqpoint{3.430861in}{2.315718in}}%
\pgfpathlineto{\pgfqpoint{3.430886in}{2.274148in}}%
\pgfpathlineto{\pgfqpoint{3.431390in}{2.473112in}}%
\pgfpathlineto{\pgfqpoint{3.431969in}{2.294250in}}%
\pgfpathlineto{\pgfqpoint{3.432203in}{2.484217in}}%
\pgfpathlineto{\pgfqpoint{3.432067in}{2.281615in}}%
\pgfpathlineto{\pgfqpoint{3.433089in}{2.322984in}}%
\pgfpathlineto{\pgfqpoint{3.433249in}{2.295052in}}%
\pgfpathlineto{\pgfqpoint{3.433753in}{2.496902in}}%
\pgfpathlineto{\pgfqpoint{3.434172in}{2.339258in}}%
\pgfpathlineto{\pgfqpoint{3.435255in}{2.451810in}}%
\pgfpathlineto{\pgfqpoint{3.434307in}{2.296334in}}%
\pgfpathlineto{\pgfqpoint{3.435304in}{2.409876in}}%
\pgfpathlineto{\pgfqpoint{3.435575in}{2.300451in}}%
\pgfpathlineto{\pgfqpoint{3.436079in}{2.436041in}}%
\pgfpathlineto{\pgfqpoint{3.436104in}{2.472293in}}%
\pgfpathlineto{\pgfqpoint{3.436227in}{2.301133in}}%
\pgfpathlineto{\pgfqpoint{3.437138in}{2.335976in}}%
\pgfpathlineto{\pgfqpoint{3.437593in}{2.467392in}}%
\pgfpathlineto{\pgfqpoint{3.437926in}{2.304103in}}%
\pgfpathlineto{\pgfqpoint{3.438516in}{2.414129in}}%
\pgfpathlineto{\pgfqpoint{3.439304in}{2.274295in}}%
\pgfpathlineto{\pgfqpoint{3.439206in}{2.475749in}}%
\pgfpathlineto{\pgfqpoint{3.439624in}{2.392442in}}%
\pgfpathlineto{\pgfqpoint{3.439673in}{2.474923in}}%
\pgfpathlineto{\pgfqpoint{3.440510in}{2.280423in}}%
\pgfpathlineto{\pgfqpoint{3.440732in}{2.393323in}}%
\pgfpathlineto{\pgfqpoint{3.441286in}{2.276457in}}%
\pgfpathlineto{\pgfqpoint{3.441556in}{2.491135in}}%
\pgfpathlineto{\pgfqpoint{3.441889in}{2.336206in}}%
\pgfpathlineto{\pgfqpoint{3.442824in}{2.259265in}}%
\pgfpathlineto{\pgfqpoint{3.442024in}{2.490367in}}%
\pgfpathlineto{\pgfqpoint{3.442910in}{2.357026in}}%
\pgfpathlineto{\pgfqpoint{3.442972in}{2.472656in}}%
\pgfpathlineto{\pgfqpoint{3.443599in}{2.283983in}}%
\pgfpathlineto{\pgfqpoint{3.443981in}{2.326526in}}%
\pgfpathlineto{\pgfqpoint{3.444202in}{2.297603in}}%
\pgfpathlineto{\pgfqpoint{3.444596in}{2.484945in}}%
\pgfpathlineto{\pgfqpoint{3.445064in}{2.363756in}}%
\pgfpathlineto{\pgfqpoint{3.446135in}{2.502507in}}%
\pgfpathlineto{\pgfqpoint{3.445150in}{2.282641in}}%
\pgfpathlineto{\pgfqpoint{3.446258in}{2.428052in}}%
\pgfpathlineto{\pgfqpoint{3.446430in}{2.291683in}}%
\pgfpathlineto{\pgfqpoint{3.446836in}{2.476537in}}%
\pgfpathlineto{\pgfqpoint{3.447390in}{2.368751in}}%
\pgfpathlineto{\pgfqpoint{3.447402in}{2.369921in}}%
\pgfpathlineto{\pgfqpoint{3.447452in}{2.282487in}}%
\pgfpathlineto{\pgfqpoint{3.447464in}{2.267808in}}%
\pgfpathlineto{\pgfqpoint{3.447612in}{2.451242in}}%
\pgfpathlineto{\pgfqpoint{3.448436in}{2.429116in}}%
\pgfpathlineto{\pgfqpoint{3.449236in}{2.483211in}}%
\pgfpathlineto{\pgfqpoint{3.448855in}{2.278051in}}%
\pgfpathlineto{\pgfqpoint{3.449507in}{2.378499in}}%
\pgfpathlineto{\pgfqpoint{3.449790in}{2.250644in}}%
\pgfpathlineto{\pgfqpoint{3.449950in}{2.464391in}}%
\pgfpathlineto{\pgfqpoint{3.450615in}{2.375608in}}%
\pgfpathlineto{\pgfqpoint{3.451255in}{2.293682in}}%
\pgfpathlineto{\pgfqpoint{3.451562in}{2.488459in}}%
\pgfpathlineto{\pgfqpoint{3.451710in}{2.388835in}}%
\pgfpathlineto{\pgfqpoint{3.451722in}{2.389519in}}%
\pgfpathlineto{\pgfqpoint{3.451747in}{2.361678in}}%
\pgfpathlineto{\pgfqpoint{3.452116in}{2.248212in}}%
\pgfpathlineto{\pgfqpoint{3.452276in}{2.494505in}}%
\pgfpathlineto{\pgfqpoint{3.452842in}{2.384205in}}%
\pgfpathlineto{\pgfqpoint{3.452892in}{2.322330in}}%
\pgfpathlineto{\pgfqpoint{3.453322in}{2.301414in}}%
\pgfpathlineto{\pgfqpoint{3.453101in}{2.456288in}}%
\pgfpathlineto{\pgfqpoint{3.453790in}{2.414932in}}%
\pgfpathlineto{\pgfqpoint{3.454639in}{2.522162in}}%
\pgfpathlineto{\pgfqpoint{3.454442in}{2.273792in}}%
\pgfpathlineto{\pgfqpoint{3.454775in}{2.353565in}}%
\pgfpathlineto{\pgfqpoint{3.454922in}{2.297974in}}%
\pgfpathlineto{\pgfqpoint{3.455439in}{2.468753in}}%
\pgfpathlineto{\pgfqpoint{3.455882in}{2.342574in}}%
\pgfpathlineto{\pgfqpoint{3.456953in}{2.500614in}}%
\pgfpathlineto{\pgfqpoint{3.456781in}{2.279487in}}%
\pgfpathlineto{\pgfqpoint{3.457089in}{2.407600in}}%
\pgfpathlineto{\pgfqpoint{3.457261in}{2.289519in}}%
\pgfpathlineto{\pgfqpoint{3.457753in}{2.448310in}}%
\pgfpathlineto{\pgfqpoint{3.458209in}{2.363316in}}%
\pgfpathlineto{\pgfqpoint{3.459292in}{2.484302in}}%
\pgfpathlineto{\pgfqpoint{3.459107in}{2.284250in}}%
\pgfpathlineto{\pgfqpoint{3.459415in}{2.435021in}}%
\pgfpathlineto{\pgfqpoint{3.459562in}{2.285547in}}%
\pgfpathlineto{\pgfqpoint{3.460092in}{2.458137in}}%
\pgfpathlineto{\pgfqpoint{3.460547in}{2.361564in}}%
\pgfpathlineto{\pgfqpoint{3.461618in}{2.493015in}}%
\pgfpathlineto{\pgfqpoint{3.460756in}{2.300824in}}%
\pgfpathlineto{\pgfqpoint{3.461679in}{2.391075in}}%
\pgfpathlineto{\pgfqpoint{3.461925in}{2.307153in}}%
\pgfpathlineto{\pgfqpoint{3.462405in}{2.472688in}}%
\pgfpathlineto{\pgfqpoint{3.462787in}{2.377011in}}%
\pgfpathlineto{\pgfqpoint{3.463944in}{2.502146in}}%
\pgfpathlineto{\pgfqpoint{3.463759in}{2.297331in}}%
\pgfpathlineto{\pgfqpoint{3.463981in}{2.435090in}}%
\pgfpathlineto{\pgfqpoint{3.464239in}{2.305696in}}%
\pgfpathlineto{\pgfqpoint{3.464744in}{2.457376in}}%
\pgfpathlineto{\pgfqpoint{3.465113in}{2.372269in}}%
\pgfpathlineto{\pgfqpoint{3.465519in}{2.451727in}}%
\pgfpathlineto{\pgfqpoint{3.466073in}{2.305638in}}%
\pgfpathlineto{\pgfqpoint{3.466085in}{2.293339in}}%
\pgfpathlineto{\pgfqpoint{3.466258in}{2.481269in}}%
\pgfpathlineto{\pgfqpoint{3.467033in}{2.440770in}}%
\pgfpathlineto{\pgfqpoint{3.467845in}{2.473017in}}%
\pgfpathlineto{\pgfqpoint{3.467279in}{2.312272in}}%
\pgfpathlineto{\pgfqpoint{3.467993in}{2.345059in}}%
\pgfpathlineto{\pgfqpoint{3.468424in}{2.295447in}}%
\pgfpathlineto{\pgfqpoint{3.468584in}{2.476197in}}%
\pgfpathlineto{\pgfqpoint{3.469113in}{2.327895in}}%
\pgfpathlineto{\pgfqpoint{3.469125in}{2.313924in}}%
\pgfpathlineto{\pgfqpoint{3.469384in}{2.455903in}}%
\pgfpathlineto{\pgfqpoint{3.470122in}{2.425566in}}%
\pgfpathlineto{\pgfqpoint{3.470159in}{2.475826in}}%
\pgfpathlineto{\pgfqpoint{3.470332in}{2.328018in}}%
\pgfpathlineto{\pgfqpoint{3.471169in}{2.358361in}}%
\pgfpathlineto{\pgfqpoint{3.471205in}{2.310153in}}%
\pgfpathlineto{\pgfqpoint{3.471759in}{2.452674in}}%
\pgfpathlineto{\pgfqpoint{3.472239in}{2.388495in}}%
\pgfpathlineto{\pgfqpoint{3.472485in}{2.466826in}}%
\pgfpathlineto{\pgfqpoint{3.472842in}{2.315842in}}%
\pgfpathlineto{\pgfqpoint{3.473396in}{2.430784in}}%
\pgfpathlineto{\pgfqpoint{3.474393in}{2.295143in}}%
\pgfpathlineto{\pgfqpoint{3.474085in}{2.452901in}}%
\pgfpathlineto{\pgfqpoint{3.474565in}{2.392491in}}%
\pgfpathlineto{\pgfqpoint{3.474812in}{2.454233in}}%
\pgfpathlineto{\pgfqpoint{3.475169in}{2.292173in}}%
\pgfpathlineto{\pgfqpoint{3.475636in}{2.371962in}}%
\pgfpathlineto{\pgfqpoint{3.476104in}{2.320326in}}%
\pgfpathlineto{\pgfqpoint{3.475735in}{2.448470in}}%
\pgfpathlineto{\pgfqpoint{3.476387in}{2.422801in}}%
\pgfpathlineto{\pgfqpoint{3.476424in}{2.449052in}}%
\pgfpathlineto{\pgfqpoint{3.476719in}{2.315444in}}%
\pgfpathlineto{\pgfqpoint{3.477408in}{2.350725in}}%
\pgfpathlineto{\pgfqpoint{3.477495in}{2.306694in}}%
\pgfpathlineto{\pgfqpoint{3.477827in}{2.433305in}}%
\pgfpathlineto{\pgfqpoint{3.478504in}{2.357983in}}%
\pgfpathlineto{\pgfqpoint{3.478528in}{2.363951in}}%
\pgfpathlineto{\pgfqpoint{3.478541in}{2.354702in}}%
\pgfpathlineto{\pgfqpoint{3.479033in}{2.325532in}}%
\pgfpathlineto{\pgfqpoint{3.478750in}{2.442038in}}%
\pgfpathlineto{\pgfqpoint{3.479612in}{2.402379in}}%
\pgfpathlineto{\pgfqpoint{3.479968in}{2.297842in}}%
\pgfpathlineto{\pgfqpoint{3.480252in}{2.447050in}}%
\pgfpathlineto{\pgfqpoint{3.480781in}{2.346146in}}%
\pgfpathlineto{\pgfqpoint{3.481076in}{2.448371in}}%
\pgfpathlineto{\pgfqpoint{3.481359in}{2.324558in}}%
\pgfpathlineto{\pgfqpoint{3.482012in}{2.422240in}}%
\pgfpathlineto{\pgfqpoint{3.482295in}{2.296697in}}%
\pgfpathlineto{\pgfqpoint{3.482578in}{2.437656in}}%
\pgfpathlineto{\pgfqpoint{3.483255in}{2.374089in}}%
\pgfpathlineto{\pgfqpoint{3.483685in}{2.328693in}}%
\pgfpathlineto{\pgfqpoint{3.483402in}{2.451788in}}%
\pgfpathlineto{\pgfqpoint{3.484079in}{2.380586in}}%
\pgfpathlineto{\pgfqpoint{3.484338in}{2.437589in}}%
\pgfpathlineto{\pgfqpoint{3.484608in}{2.317252in}}%
\pgfpathlineto{\pgfqpoint{3.485175in}{2.354349in}}%
\pgfpathlineto{\pgfqpoint{3.485532in}{2.328605in}}%
\pgfpathlineto{\pgfqpoint{3.485728in}{2.436411in}}%
\pgfpathlineto{\pgfqpoint{3.486258in}{2.357102in}}%
\pgfpathlineto{\pgfqpoint{3.486652in}{2.432602in}}%
\pgfpathlineto{\pgfqpoint{3.486935in}{2.316878in}}%
\pgfpathlineto{\pgfqpoint{3.487365in}{2.381615in}}%
\pgfpathlineto{\pgfqpoint{3.487845in}{2.330831in}}%
\pgfpathlineto{\pgfqpoint{3.488067in}{2.423465in}}%
\pgfpathlineto{\pgfqpoint{3.488485in}{2.362559in}}%
\pgfpathlineto{\pgfqpoint{3.488633in}{2.348540in}}%
\pgfpathlineto{\pgfqpoint{3.488719in}{2.380213in}}%
\pgfpathlineto{\pgfqpoint{3.488965in}{2.447963in}}%
\pgfpathlineto{\pgfqpoint{3.489372in}{2.323813in}}%
\pgfpathlineto{\pgfqpoint{3.489802in}{2.365620in}}%
\pgfpathlineto{\pgfqpoint{3.490172in}{2.334715in}}%
\pgfpathlineto{\pgfqpoint{3.490516in}{2.406028in}}%
\pgfpathlineto{\pgfqpoint{3.490910in}{2.364801in}}%
\pgfpathlineto{\pgfqpoint{3.491685in}{2.324914in}}%
\pgfpathlineto{\pgfqpoint{3.491292in}{2.442483in}}%
\pgfpathlineto{\pgfqpoint{3.491956in}{2.396195in}}%
\pgfpathlineto{\pgfqpoint{3.492682in}{2.414668in}}%
\pgfpathlineto{\pgfqpoint{3.492485in}{2.325797in}}%
\pgfpathlineto{\pgfqpoint{3.493052in}{2.378195in}}%
\pgfpathlineto{\pgfqpoint{3.493261in}{2.335189in}}%
\pgfpathlineto{\pgfqpoint{3.493618in}{2.444404in}}%
\pgfpathlineto{\pgfqpoint{3.494135in}{2.389386in}}%
\pgfpathlineto{\pgfqpoint{3.494996in}{2.418258in}}%
\pgfpathlineto{\pgfqpoint{3.494812in}{2.329612in}}%
\pgfpathlineto{\pgfqpoint{3.495230in}{2.389859in}}%
\pgfpathlineto{\pgfqpoint{3.495587in}{2.324152in}}%
\pgfpathlineto{\pgfqpoint{3.495932in}{2.425936in}}%
\pgfpathlineto{\pgfqpoint{3.496350in}{2.347841in}}%
\pgfpathlineto{\pgfqpoint{3.497322in}{2.425487in}}%
\pgfpathlineto{\pgfqpoint{3.497138in}{2.315468in}}%
\pgfpathlineto{\pgfqpoint{3.497482in}{2.384106in}}%
\pgfpathlineto{\pgfqpoint{3.497913in}{2.323000in}}%
\pgfpathlineto{\pgfqpoint{3.498270in}{2.422388in}}%
\pgfpathlineto{\pgfqpoint{3.498590in}{2.378514in}}%
\pgfpathlineto{\pgfqpoint{3.499070in}{2.411927in}}%
\pgfpathlineto{\pgfqpoint{3.499451in}{2.333025in}}%
\pgfpathlineto{\pgfqpoint{3.499735in}{2.398271in}}%
\pgfpathlineto{\pgfqpoint{3.500239in}{2.325989in}}%
\pgfpathlineto{\pgfqpoint{3.500584in}{2.414239in}}%
\pgfpathlineto{\pgfqpoint{3.500855in}{2.357523in}}%
\pgfpathlineto{\pgfqpoint{3.501950in}{2.417727in}}%
\pgfpathlineto{\pgfqpoint{3.501778in}{2.343700in}}%
\pgfpathlineto{\pgfqpoint{3.501987in}{2.393707in}}%
\pgfpathlineto{\pgfqpoint{3.502602in}{2.325076in}}%
\pgfpathlineto{\pgfqpoint{3.502910in}{2.411354in}}%
\pgfpathlineto{\pgfqpoint{3.503107in}{2.374239in}}%
\pgfpathlineto{\pgfqpoint{3.503993in}{2.342517in}}%
\pgfpathlineto{\pgfqpoint{3.503710in}{2.420422in}}%
\pgfpathlineto{\pgfqpoint{3.504227in}{2.371168in}}%
\pgfpathlineto{\pgfqpoint{3.504276in}{2.423495in}}%
\pgfpathlineto{\pgfqpoint{3.504916in}{2.327462in}}%
\pgfpathlineto{\pgfqpoint{3.505322in}{2.362187in}}%
\pgfpathlineto{\pgfqpoint{3.506307in}{2.334114in}}%
\pgfpathlineto{\pgfqpoint{3.506024in}{2.419041in}}%
\pgfpathlineto{\pgfqpoint{3.506455in}{2.352680in}}%
\pgfpathlineto{\pgfqpoint{3.506590in}{2.418638in}}%
\pgfpathlineto{\pgfqpoint{3.507230in}{2.335821in}}%
\pgfpathlineto{\pgfqpoint{3.507624in}{2.388331in}}%
\pgfpathlineto{\pgfqpoint{3.507784in}{2.338150in}}%
\pgfpathlineto{\pgfqpoint{3.508338in}{2.407005in}}%
\pgfpathlineto{\pgfqpoint{3.508756in}{2.360089in}}%
\pgfpathlineto{\pgfqpoint{3.509630in}{2.332879in}}%
\pgfpathlineto{\pgfqpoint{3.508904in}{2.413574in}}%
\pgfpathlineto{\pgfqpoint{3.509790in}{2.376357in}}%
\pgfpathlineto{\pgfqpoint{3.510282in}{2.411230in}}%
\pgfpathlineto{\pgfqpoint{3.510098in}{2.331028in}}%
\pgfpathlineto{\pgfqpoint{3.510898in}{2.373649in}}%
\pgfpathlineto{\pgfqpoint{3.510910in}{2.373626in}}%
\pgfpathlineto{\pgfqpoint{3.511944in}{2.328124in}}%
\pgfpathlineto{\pgfqpoint{3.511439in}{2.415492in}}%
\pgfpathlineto{\pgfqpoint{3.512005in}{2.379999in}}%
\pgfpathlineto{\pgfqpoint{3.513051in}{2.413545in}}%
\pgfpathlineto{\pgfqpoint{3.512731in}{2.342109in}}%
\pgfpathlineto{\pgfqpoint{3.513113in}{2.377236in}}%
\pgfpathlineto{\pgfqpoint{3.513753in}{2.414820in}}%
\pgfpathlineto{\pgfqpoint{3.513335in}{2.328266in}}%
\pgfpathlineto{\pgfqpoint{3.513925in}{2.350934in}}%
\pgfpathlineto{\pgfqpoint{3.514258in}{2.327183in}}%
\pgfpathlineto{\pgfqpoint{3.514356in}{2.410186in}}%
\pgfpathlineto{\pgfqpoint{3.514885in}{2.375847in}}%
\pgfpathlineto{\pgfqpoint{3.515378in}{2.421119in}}%
\pgfpathlineto{\pgfqpoint{3.515784in}{2.324645in}}%
\pgfpathlineto{\pgfqpoint{3.515993in}{2.381566in}}%
\pgfpathlineto{\pgfqpoint{3.516571in}{2.324355in}}%
\pgfpathlineto{\pgfqpoint{3.516768in}{2.411177in}}%
\pgfpathlineto{\pgfqpoint{3.517113in}{2.372854in}}%
\pgfpathlineto{\pgfqpoint{3.517691in}{2.417761in}}%
\pgfpathlineto{\pgfqpoint{3.518098in}{2.317964in}}%
\pgfpathlineto{\pgfqpoint{3.518221in}{2.375396in}}%
\pgfpathlineto{\pgfqpoint{3.519316in}{2.411798in}}%
\pgfpathlineto{\pgfqpoint{3.518898in}{2.323248in}}%
\pgfpathlineto{\pgfqpoint{3.519328in}{2.391445in}}%
\pgfpathlineto{\pgfqpoint{3.520424in}{2.324939in}}%
\pgfpathlineto{\pgfqpoint{3.520018in}{2.415762in}}%
\pgfpathlineto{\pgfqpoint{3.520448in}{2.359907in}}%
\pgfpathlineto{\pgfqpoint{3.521396in}{2.408685in}}%
\pgfpathlineto{\pgfqpoint{3.521211in}{2.324694in}}%
\pgfpathlineto{\pgfqpoint{3.521568in}{2.383388in}}%
\pgfpathlineto{\pgfqpoint{3.521814in}{2.343778in}}%
\pgfpathlineto{\pgfqpoint{3.522331in}{2.415998in}}%
\pgfpathlineto{\pgfqpoint{3.522676in}{2.380015in}}%
\pgfpathlineto{\pgfqpoint{3.523710in}{2.411681in}}%
\pgfpathlineto{\pgfqpoint{3.523538in}{2.329233in}}%
\pgfpathlineto{\pgfqpoint{3.523771in}{2.380294in}}%
\pgfpathlineto{\pgfqpoint{3.524288in}{2.346211in}}%
\pgfpathlineto{\pgfqpoint{3.524645in}{2.423149in}}%
\pgfpathlineto{\pgfqpoint{3.524867in}{2.386154in}}%
\pgfpathlineto{\pgfqpoint{3.525876in}{2.329361in}}%
\pgfpathlineto{\pgfqpoint{3.525568in}{2.418889in}}%
\pgfpathlineto{\pgfqpoint{3.525999in}{2.375809in}}%
\pgfpathlineto{\pgfqpoint{3.526959in}{2.421899in}}%
\pgfpathlineto{\pgfqpoint{3.526676in}{2.342680in}}%
\pgfpathlineto{\pgfqpoint{3.527119in}{2.388735in}}%
\pgfpathlineto{\pgfqpoint{3.528202in}{2.331157in}}%
\pgfpathlineto{\pgfqpoint{3.527894in}{2.414759in}}%
\pgfpathlineto{\pgfqpoint{3.528239in}{2.367536in}}%
\pgfpathlineto{\pgfqpoint{3.529285in}{2.420909in}}%
\pgfpathlineto{\pgfqpoint{3.529076in}{2.337950in}}%
\pgfpathlineto{\pgfqpoint{3.529371in}{2.381530in}}%
\pgfpathlineto{\pgfqpoint{3.530516in}{2.339637in}}%
\pgfpathlineto{\pgfqpoint{3.530221in}{2.420204in}}%
\pgfpathlineto{\pgfqpoint{3.530528in}{2.341782in}}%
\pgfpathlineto{\pgfqpoint{3.531611in}{2.416130in}}%
\pgfpathlineto{\pgfqpoint{3.531402in}{2.341261in}}%
\pgfpathlineto{\pgfqpoint{3.531673in}{2.393280in}}%
\pgfpathlineto{\pgfqpoint{3.531858in}{2.342933in}}%
\pgfpathlineto{\pgfqpoint{3.532534in}{2.416774in}}%
\pgfpathlineto{\pgfqpoint{3.532842in}{2.346984in}}%
\pgfpathlineto{\pgfqpoint{3.533925in}{2.414830in}}%
\pgfpathlineto{\pgfqpoint{3.533753in}{2.340203in}}%
\pgfpathlineto{\pgfqpoint{3.533999in}{2.382403in}}%
\pgfpathlineto{\pgfqpoint{3.535082in}{2.348500in}}%
\pgfpathlineto{\pgfqpoint{3.534848in}{2.414291in}}%
\pgfpathlineto{\pgfqpoint{3.535156in}{2.352022in}}%
\pgfpathlineto{\pgfqpoint{3.536239in}{2.410363in}}%
\pgfpathlineto{\pgfqpoint{3.536079in}{2.346379in}}%
\pgfpathlineto{\pgfqpoint{3.536313in}{2.384422in}}%
\pgfpathlineto{\pgfqpoint{3.537458in}{2.342871in}}%
\pgfpathlineto{\pgfqpoint{3.537162in}{2.416835in}}%
\pgfpathlineto{\pgfqpoint{3.537470in}{2.345246in}}%
\pgfpathlineto{\pgfqpoint{3.538553in}{2.404625in}}%
\pgfpathlineto{\pgfqpoint{3.538614in}{2.387205in}}%
\pgfpathlineto{\pgfqpoint{3.539501in}{2.401889in}}%
\pgfpathlineto{\pgfqpoint{3.539771in}{2.349670in}}%
\pgfpathlineto{\pgfqpoint{3.540842in}{2.402961in}}%
\pgfpathlineto{\pgfqpoint{3.539857in}{2.341638in}}%
\pgfpathlineto{\pgfqpoint{3.541076in}{2.387862in}}%
\pgfpathlineto{\pgfqpoint{3.542171in}{2.339200in}}%
\pgfpathlineto{\pgfqpoint{3.541765in}{2.400548in}}%
\pgfpathlineto{\pgfqpoint{3.542196in}{2.355738in}}%
\pgfpathlineto{\pgfqpoint{3.543156in}{2.407244in}}%
\pgfpathlineto{\pgfqpoint{3.542959in}{2.346467in}}%
\pgfpathlineto{\pgfqpoint{3.543316in}{2.382532in}}%
\pgfpathlineto{\pgfqpoint{3.543747in}{2.349200in}}%
\pgfpathlineto{\pgfqpoint{3.544079in}{2.400809in}}%
\pgfpathlineto{\pgfqpoint{3.544436in}{2.366569in}}%
\pgfpathlineto{\pgfqpoint{3.545482in}{2.403297in}}%
\pgfpathlineto{\pgfqpoint{3.544497in}{2.342636in}}%
\pgfpathlineto{\pgfqpoint{3.545581in}{2.386889in}}%
\pgfpathlineto{\pgfqpoint{3.546061in}{2.348676in}}%
\pgfpathlineto{\pgfqpoint{3.546405in}{2.407740in}}%
\pgfpathlineto{\pgfqpoint{3.546725in}{2.362612in}}%
\pgfpathlineto{\pgfqpoint{3.547599in}{2.349566in}}%
\pgfpathlineto{\pgfqpoint{3.547328in}{2.402798in}}%
\pgfpathlineto{\pgfqpoint{3.547759in}{2.369739in}}%
\pgfpathlineto{\pgfqpoint{3.548731in}{2.413338in}}%
\pgfpathlineto{\pgfqpoint{3.548387in}{2.351410in}}%
\pgfpathlineto{\pgfqpoint{3.548879in}{2.380479in}}%
\pgfpathlineto{\pgfqpoint{3.549839in}{2.349838in}}%
\pgfpathlineto{\pgfqpoint{3.549654in}{2.406565in}}%
\pgfpathlineto{\pgfqpoint{3.549987in}{2.365731in}}%
\pgfpathlineto{\pgfqpoint{3.551045in}{2.410127in}}%
\pgfpathlineto{\pgfqpoint{3.550762in}{2.349074in}}%
\pgfpathlineto{\pgfqpoint{3.551131in}{2.385910in}}%
\pgfpathlineto{\pgfqpoint{3.551451in}{2.351434in}}%
\pgfpathlineto{\pgfqpoint{3.551944in}{2.393889in}}%
\pgfpathlineto{\pgfqpoint{3.551981in}{2.411448in}}%
\pgfpathlineto{\pgfqpoint{3.552153in}{2.343906in}}%
\pgfpathlineto{\pgfqpoint{3.553002in}{2.366248in}}%
\pgfpathlineto{\pgfqpoint{3.553076in}{2.346714in}}%
\pgfpathlineto{\pgfqpoint{3.553371in}{2.402257in}}%
\pgfpathlineto{\pgfqpoint{3.554097in}{2.366553in}}%
\pgfpathlineto{\pgfqpoint{3.554294in}{2.407215in}}%
\pgfpathlineto{\pgfqpoint{3.554479in}{2.347676in}}%
\pgfpathlineto{\pgfqpoint{3.555254in}{2.384136in}}%
\pgfpathlineto{\pgfqpoint{3.555402in}{2.341639in}}%
\pgfpathlineto{\pgfqpoint{3.555685in}{2.401911in}}%
\pgfpathlineto{\pgfqpoint{3.556350in}{2.379576in}}%
\pgfpathlineto{\pgfqpoint{3.556608in}{2.405796in}}%
\pgfpathlineto{\pgfqpoint{3.556879in}{2.350321in}}%
\pgfpathlineto{\pgfqpoint{3.557457in}{2.378031in}}%
\pgfpathlineto{\pgfqpoint{3.557716in}{2.345495in}}%
\pgfpathlineto{\pgfqpoint{3.557999in}{2.403785in}}%
\pgfpathlineto{\pgfqpoint{3.558590in}{2.361457in}}%
\pgfpathlineto{\pgfqpoint{3.558922in}{2.403721in}}%
\pgfpathlineto{\pgfqpoint{3.559205in}{2.351634in}}%
\pgfpathlineto{\pgfqpoint{3.559784in}{2.369983in}}%
\pgfpathlineto{\pgfqpoint{3.560042in}{2.346599in}}%
\pgfpathlineto{\pgfqpoint{3.559857in}{2.399667in}}%
\pgfpathlineto{\pgfqpoint{3.560879in}{2.374601in}}%
\pgfpathlineto{\pgfqpoint{3.561519in}{2.349868in}}%
\pgfpathlineto{\pgfqpoint{3.561248in}{2.403752in}}%
\pgfpathlineto{\pgfqpoint{3.561913in}{2.375064in}}%
\pgfpathlineto{\pgfqpoint{3.562860in}{2.401593in}}%
\pgfpathlineto{\pgfqpoint{3.562356in}{2.353508in}}%
\pgfpathlineto{\pgfqpoint{3.563020in}{2.374146in}}%
\pgfpathlineto{\pgfqpoint{3.563833in}{2.352618in}}%
\pgfpathlineto{\pgfqpoint{3.563562in}{2.403115in}}%
\pgfpathlineto{\pgfqpoint{3.564116in}{2.379866in}}%
\pgfpathlineto{\pgfqpoint{3.564485in}{2.398672in}}%
\pgfpathlineto{\pgfqpoint{3.564300in}{2.353103in}}%
\pgfpathlineto{\pgfqpoint{3.565125in}{2.367699in}}%
\pgfpathlineto{\pgfqpoint{3.565691in}{2.352332in}}%
\pgfpathlineto{\pgfqpoint{3.565876in}{2.404404in}}%
\pgfpathlineto{\pgfqpoint{3.566097in}{2.391070in}}%
\pgfpathlineto{\pgfqpoint{3.566799in}{2.403326in}}%
\pgfpathlineto{\pgfqpoint{3.566282in}{2.349241in}}%
\pgfpathlineto{\pgfqpoint{3.567057in}{2.362948in}}%
\pgfpathlineto{\pgfqpoint{3.567070in}{2.354541in}}%
\pgfpathlineto{\pgfqpoint{3.567956in}{2.400307in}}%
\pgfpathlineto{\pgfqpoint{3.568153in}{2.365718in}}%
\pgfpathlineto{\pgfqpoint{3.569113in}{2.401945in}}%
\pgfpathlineto{\pgfqpoint{3.568596in}{2.352152in}}%
\pgfpathlineto{\pgfqpoint{3.569260in}{2.375299in}}%
\pgfpathlineto{\pgfqpoint{3.569310in}{2.351644in}}%
\pgfpathlineto{\pgfqpoint{3.569950in}{2.396187in}}%
\pgfpathlineto{\pgfqpoint{3.570257in}{2.382598in}}%
\pgfpathlineto{\pgfqpoint{3.571193in}{2.392993in}}%
\pgfpathlineto{\pgfqpoint{3.570910in}{2.358695in}}%
\pgfpathlineto{\pgfqpoint{3.571353in}{2.378800in}}%
\pgfpathlineto{\pgfqpoint{3.571624in}{2.355576in}}%
\pgfpathlineto{\pgfqpoint{3.572251in}{2.393989in}}%
\pgfpathlineto{\pgfqpoint{3.572473in}{2.359239in}}%
\pgfpathlineto{\pgfqpoint{3.573039in}{2.394452in}}%
\pgfpathlineto{\pgfqpoint{3.572547in}{2.355684in}}%
\pgfpathlineto{\pgfqpoint{3.573593in}{2.373487in}}%
\pgfpathlineto{\pgfqpoint{3.573999in}{2.354257in}}%
\pgfpathlineto{\pgfqpoint{3.574196in}{2.392038in}}%
\pgfpathlineto{\pgfqpoint{3.574651in}{2.386531in}}%
\pgfpathlineto{\pgfqpoint{3.574664in}{2.390635in}}%
\pgfpathlineto{\pgfqpoint{3.574774in}{2.353141in}}%
\pgfpathlineto{\pgfqpoint{3.575685in}{2.360888in}}%
\pgfpathlineto{\pgfqpoint{3.576313in}{2.354052in}}%
\pgfpathlineto{\pgfqpoint{3.576510in}{2.391582in}}%
\pgfpathlineto{\pgfqpoint{3.576780in}{2.363202in}}%
\pgfpathlineto{\pgfqpoint{3.577654in}{2.388990in}}%
\pgfpathlineto{\pgfqpoint{3.577088in}{2.356298in}}%
\pgfpathlineto{\pgfqpoint{3.577913in}{2.378370in}}%
\pgfpathlineto{\pgfqpoint{3.578011in}{2.357205in}}%
\pgfpathlineto{\pgfqpoint{3.578356in}{2.393058in}}%
\pgfpathlineto{\pgfqpoint{3.579008in}{2.378709in}}%
\pgfpathlineto{\pgfqpoint{3.579624in}{2.393670in}}%
\pgfpathlineto{\pgfqpoint{3.579845in}{2.361074in}}%
\pgfpathlineto{\pgfqpoint{3.580103in}{2.378173in}}%
\pgfpathlineto{\pgfqpoint{3.580916in}{2.360362in}}%
\pgfpathlineto{\pgfqpoint{3.580411in}{2.389949in}}%
\pgfpathlineto{\pgfqpoint{3.581236in}{2.363148in}}%
\pgfpathlineto{\pgfqpoint{3.582245in}{2.389335in}}%
\pgfpathlineto{\pgfqpoint{3.582147in}{2.358407in}}%
\pgfpathlineto{\pgfqpoint{3.582393in}{2.385242in}}%
\pgfpathlineto{\pgfqpoint{3.583217in}{2.357460in}}%
\pgfpathlineto{\pgfqpoint{3.582713in}{2.393565in}}%
\pgfpathlineto{\pgfqpoint{3.583537in}{2.363258in}}%
\pgfpathlineto{\pgfqpoint{3.584559in}{2.394795in}}%
\pgfpathlineto{\pgfqpoint{3.583993in}{2.354185in}}%
\pgfpathlineto{\pgfqpoint{3.584657in}{2.379385in}}%
\pgfpathlineto{\pgfqpoint{3.585519in}{2.355752in}}%
\pgfpathlineto{\pgfqpoint{3.585482in}{2.398372in}}%
\pgfpathlineto{\pgfqpoint{3.585765in}{2.375893in}}%
\pgfpathlineto{\pgfqpoint{3.586860in}{2.394152in}}%
\pgfpathlineto{\pgfqpoint{3.586442in}{2.350897in}}%
\pgfpathlineto{\pgfqpoint{3.586873in}{2.390490in}}%
\pgfpathlineto{\pgfqpoint{3.587217in}{2.354483in}}%
\pgfpathlineto{\pgfqpoint{3.587783in}{2.401275in}}%
\pgfpathlineto{\pgfqpoint{3.587993in}{2.369619in}}%
\pgfpathlineto{\pgfqpoint{3.588251in}{2.390480in}}%
\pgfpathlineto{\pgfqpoint{3.588756in}{2.348586in}}%
\pgfpathlineto{\pgfqpoint{3.589100in}{2.376403in}}%
\pgfpathlineto{\pgfqpoint{3.589531in}{2.356757in}}%
\pgfpathlineto{\pgfqpoint{3.590097in}{2.395244in}}%
\pgfpathlineto{\pgfqpoint{3.590196in}{2.378141in}}%
\pgfpathlineto{\pgfqpoint{3.591267in}{2.391970in}}%
\pgfpathlineto{\pgfqpoint{3.591057in}{2.356427in}}%
\pgfpathlineto{\pgfqpoint{3.591291in}{2.378741in}}%
\pgfpathlineto{\pgfqpoint{3.591907in}{2.359228in}}%
\pgfpathlineto{\pgfqpoint{3.592190in}{2.394866in}}%
\pgfpathlineto{\pgfqpoint{3.592387in}{2.373154in}}%
\pgfpathlineto{\pgfqpoint{3.593113in}{2.394174in}}%
\pgfpathlineto{\pgfqpoint{3.592916in}{2.358096in}}%
\pgfpathlineto{\pgfqpoint{3.593494in}{2.375028in}}%
\pgfpathlineto{\pgfqpoint{3.594220in}{2.354457in}}%
\pgfpathlineto{\pgfqpoint{3.594503in}{2.390682in}}%
\pgfpathlineto{\pgfqpoint{3.594614in}{2.364234in}}%
\pgfpathlineto{\pgfqpoint{3.595427in}{2.393374in}}%
\pgfpathlineto{\pgfqpoint{3.594676in}{2.358178in}}%
\pgfpathlineto{\pgfqpoint{3.595734in}{2.370722in}}%
\pgfpathlineto{\pgfqpoint{3.596350in}{2.393759in}}%
\pgfpathlineto{\pgfqpoint{3.596534in}{2.353707in}}%
\pgfpathlineto{\pgfqpoint{3.596854in}{2.374644in}}%
\pgfpathlineto{\pgfqpoint{3.597457in}{2.360100in}}%
\pgfpathlineto{\pgfqpoint{3.597740in}{2.392841in}}%
\pgfpathlineto{\pgfqpoint{3.598011in}{2.361066in}}%
\pgfpathlineto{\pgfqpoint{3.598663in}{2.395272in}}%
\pgfpathlineto{\pgfqpoint{3.598934in}{2.354444in}}%
\pgfpathlineto{\pgfqpoint{3.599180in}{2.374376in}}%
\pgfpathlineto{\pgfqpoint{3.599574in}{2.392888in}}%
\pgfpathlineto{\pgfqpoint{3.599328in}{2.359054in}}%
\pgfpathlineto{\pgfqpoint{3.599759in}{2.365082in}}%
\pgfpathlineto{\pgfqpoint{3.599796in}{2.354580in}}%
\pgfpathlineto{\pgfqpoint{3.600054in}{2.397626in}}%
\pgfpathlineto{\pgfqpoint{3.600830in}{2.381219in}}%
\pgfpathlineto{\pgfqpoint{3.600965in}{2.394288in}}%
\pgfpathlineto{\pgfqpoint{3.601174in}{2.354335in}}%
\pgfpathlineto{\pgfqpoint{3.601937in}{2.386370in}}%
\pgfpathlineto{\pgfqpoint{3.602097in}{2.352418in}}%
\pgfpathlineto{\pgfqpoint{3.602331in}{2.394912in}}%
\pgfpathlineto{\pgfqpoint{3.603070in}{2.370437in}}%
\pgfpathlineto{\pgfqpoint{3.603488in}{2.351906in}}%
\pgfpathlineto{\pgfqpoint{3.603254in}{2.396745in}}%
\pgfpathlineto{\pgfqpoint{3.604066in}{2.376743in}}%
\pgfpathlineto{\pgfqpoint{3.604694in}{2.394212in}}%
\pgfpathlineto{\pgfqpoint{3.604423in}{2.351215in}}%
\pgfpathlineto{\pgfqpoint{3.605186in}{2.381609in}}%
\pgfpathlineto{\pgfqpoint{3.605826in}{2.345229in}}%
\pgfpathlineto{\pgfqpoint{3.605568in}{2.399798in}}%
\pgfpathlineto{\pgfqpoint{3.606306in}{2.368688in}}%
\pgfpathlineto{\pgfqpoint{3.606922in}{2.400981in}}%
\pgfpathlineto{\pgfqpoint{3.606737in}{2.348234in}}%
\pgfpathlineto{\pgfqpoint{3.607439in}{2.386458in}}%
\pgfpathlineto{\pgfqpoint{3.607845in}{2.408141in}}%
\pgfpathlineto{\pgfqpoint{3.608128in}{2.342481in}}%
\pgfpathlineto{\pgfqpoint{3.608485in}{2.375184in}}%
\pgfpathlineto{\pgfqpoint{3.608583in}{2.345531in}}%
\pgfpathlineto{\pgfqpoint{3.608768in}{2.403644in}}%
\pgfpathlineto{\pgfqpoint{3.609605in}{2.369647in}}%
\pgfpathlineto{\pgfqpoint{3.610233in}{2.400813in}}%
\pgfpathlineto{\pgfqpoint{3.610442in}{2.342776in}}%
\pgfpathlineto{\pgfqpoint{3.610725in}{2.382955in}}%
\pgfpathlineto{\pgfqpoint{3.611353in}{2.344963in}}%
\pgfpathlineto{\pgfqpoint{3.611082in}{2.401605in}}%
\pgfpathlineto{\pgfqpoint{3.611845in}{2.371720in}}%
\pgfpathlineto{\pgfqpoint{3.612460in}{2.396898in}}%
\pgfpathlineto{\pgfqpoint{3.612276in}{2.347454in}}%
\pgfpathlineto{\pgfqpoint{3.612953in}{2.382296in}}%
\pgfpathlineto{\pgfqpoint{3.613666in}{2.343771in}}%
\pgfpathlineto{\pgfqpoint{3.613383in}{2.403476in}}%
\pgfpathlineto{\pgfqpoint{3.614134in}{2.355326in}}%
\pgfpathlineto{\pgfqpoint{3.614762in}{2.403419in}}%
\pgfpathlineto{\pgfqpoint{3.614577in}{2.346928in}}%
\pgfpathlineto{\pgfqpoint{3.615266in}{2.380093in}}%
\pgfpathlineto{\pgfqpoint{3.615968in}{2.349177in}}%
\pgfpathlineto{\pgfqpoint{3.615685in}{2.405940in}}%
\pgfpathlineto{\pgfqpoint{3.616116in}{2.384098in}}%
\pgfpathlineto{\pgfqpoint{3.616596in}{2.402678in}}%
\pgfpathlineto{\pgfqpoint{3.616879in}{2.346658in}}%
\pgfpathlineto{\pgfqpoint{3.617211in}{2.380629in}}%
\pgfpathlineto{\pgfqpoint{3.618331in}{2.347525in}}%
\pgfpathlineto{\pgfqpoint{3.617974in}{2.399581in}}%
\pgfpathlineto{\pgfqpoint{3.618393in}{2.368367in}}%
\pgfpathlineto{\pgfqpoint{3.619353in}{2.399004in}}%
\pgfpathlineto{\pgfqpoint{3.619254in}{2.350300in}}%
\pgfpathlineto{\pgfqpoint{3.619537in}{2.384686in}}%
\pgfpathlineto{\pgfqpoint{3.620633in}{2.351163in}}%
\pgfpathlineto{\pgfqpoint{3.619820in}{2.392899in}}%
\pgfpathlineto{\pgfqpoint{3.620682in}{2.364143in}}%
\pgfpathlineto{\pgfqpoint{3.620903in}{2.393147in}}%
\pgfpathlineto{\pgfqpoint{3.621543in}{2.343021in}}%
\pgfpathlineto{\pgfqpoint{3.621851in}{2.373587in}}%
\pgfpathlineto{\pgfqpoint{3.622011in}{2.348845in}}%
\pgfpathlineto{\pgfqpoint{3.622737in}{2.392575in}}%
\pgfpathlineto{\pgfqpoint{3.622983in}{2.365386in}}%
\pgfpathlineto{\pgfqpoint{3.624116in}{2.392965in}}%
\pgfpathlineto{\pgfqpoint{3.623845in}{2.347307in}}%
\pgfpathlineto{\pgfqpoint{3.624165in}{2.378174in}}%
\pgfpathlineto{\pgfqpoint{3.625211in}{2.353029in}}%
\pgfpathlineto{\pgfqpoint{3.624928in}{2.392812in}}%
\pgfpathlineto{\pgfqpoint{3.625285in}{2.368476in}}%
\pgfpathlineto{\pgfqpoint{3.625629in}{2.396055in}}%
\pgfpathlineto{\pgfqpoint{3.626134in}{2.353872in}}%
\pgfpathlineto{\pgfqpoint{3.626417in}{2.390678in}}%
\pgfpathlineto{\pgfqpoint{3.627045in}{2.354811in}}%
\pgfpathlineto{\pgfqpoint{3.626540in}{2.397950in}}%
\pgfpathlineto{\pgfqpoint{3.627574in}{2.357186in}}%
\pgfpathlineto{\pgfqpoint{3.627931in}{2.398123in}}%
\pgfpathlineto{\pgfqpoint{3.628485in}{2.356308in}}%
\pgfpathlineto{\pgfqpoint{3.628731in}{2.379102in}}%
\pgfpathlineto{\pgfqpoint{3.629876in}{2.354576in}}%
\pgfpathlineto{\pgfqpoint{3.629703in}{2.399583in}}%
\pgfpathlineto{\pgfqpoint{3.629900in}{2.362274in}}%
\pgfpathlineto{\pgfqpoint{3.630233in}{2.397566in}}%
\pgfpathlineto{\pgfqpoint{3.630799in}{2.354593in}}%
\pgfpathlineto{\pgfqpoint{3.631033in}{2.379955in}}%
\pgfpathlineto{\pgfqpoint{3.632017in}{2.398905in}}%
\pgfpathlineto{\pgfqpoint{3.632189in}{2.356421in}}%
\pgfpathlineto{\pgfqpoint{3.632940in}{2.394882in}}%
\pgfpathlineto{\pgfqpoint{3.633346in}{2.380330in}}%
\pgfpathlineto{\pgfqpoint{3.634503in}{2.355985in}}%
\pgfpathlineto{\pgfqpoint{3.634331in}{2.400985in}}%
\pgfpathlineto{\pgfqpoint{3.634516in}{2.358763in}}%
\pgfpathlineto{\pgfqpoint{3.635254in}{2.398119in}}%
\pgfpathlineto{\pgfqpoint{3.635426in}{2.354096in}}%
\pgfpathlineto{\pgfqpoint{3.635673in}{2.378194in}}%
\pgfpathlineto{\pgfqpoint{3.636362in}{2.357021in}}%
\pgfpathlineto{\pgfqpoint{3.635722in}{2.399508in}}%
\pgfpathlineto{\pgfqpoint{3.636903in}{2.357147in}}%
\pgfpathlineto{\pgfqpoint{3.636965in}{2.379032in}}%
\pgfpathlineto{\pgfqpoint{3.637580in}{2.397411in}}%
\pgfpathlineto{\pgfqpoint{3.637753in}{2.356628in}}%
\pgfpathlineto{\pgfqpoint{3.638085in}{2.384106in}}%
\pgfpathlineto{\pgfqpoint{3.639168in}{2.355007in}}%
\pgfpathlineto{\pgfqpoint{3.638934in}{2.393428in}}%
\pgfpathlineto{\pgfqpoint{3.639279in}{2.370713in}}%
\pgfpathlineto{\pgfqpoint{3.639771in}{2.393113in}}%
\pgfpathlineto{\pgfqpoint{3.640103in}{2.347262in}}%
\pgfpathlineto{\pgfqpoint{3.640399in}{2.385901in}}%
\pgfpathlineto{\pgfqpoint{3.641026in}{2.349421in}}%
\pgfpathlineto{\pgfqpoint{3.641174in}{2.395324in}}%
\pgfpathlineto{\pgfqpoint{3.641543in}{2.367076in}}%
\pgfpathlineto{\pgfqpoint{3.642602in}{2.402967in}}%
\pgfpathlineto{\pgfqpoint{3.642429in}{2.344165in}}%
\pgfpathlineto{\pgfqpoint{3.642737in}{2.384124in}}%
\pgfpathlineto{\pgfqpoint{3.642885in}{2.349091in}}%
\pgfpathlineto{\pgfqpoint{3.643537in}{2.404827in}}%
\pgfpathlineto{\pgfqpoint{3.643882in}{2.369753in}}%
\pgfpathlineto{\pgfqpoint{3.644460in}{2.406631in}}%
\pgfpathlineto{\pgfqpoint{3.644768in}{2.351714in}}%
\pgfpathlineto{\pgfqpoint{3.645026in}{2.380189in}}%
\pgfpathlineto{\pgfqpoint{3.645703in}{2.356742in}}%
\pgfpathlineto{\pgfqpoint{3.645863in}{2.408125in}}%
\pgfpathlineto{\pgfqpoint{3.646159in}{2.356789in}}%
\pgfpathlineto{\pgfqpoint{3.646171in}{2.356698in}}%
\pgfpathlineto{\pgfqpoint{3.646183in}{2.359981in}}%
\pgfpathlineto{\pgfqpoint{3.646786in}{2.402755in}}%
\pgfpathlineto{\pgfqpoint{3.647094in}{2.354279in}}%
\pgfpathlineto{\pgfqpoint{3.647316in}{2.383408in}}%
\pgfpathlineto{\pgfqpoint{3.647906in}{2.357219in}}%
\pgfpathlineto{\pgfqpoint{3.647722in}{2.400588in}}%
\pgfpathlineto{\pgfqpoint{3.648177in}{2.398736in}}%
\pgfpathlineto{\pgfqpoint{3.648189in}{2.403342in}}%
\pgfpathlineto{\pgfqpoint{3.648780in}{2.357704in}}%
\pgfpathlineto{\pgfqpoint{3.649199in}{2.368525in}}%
\pgfpathlineto{\pgfqpoint{3.650171in}{2.355921in}}%
\pgfpathlineto{\pgfqpoint{3.649605in}{2.398187in}}%
\pgfpathlineto{\pgfqpoint{3.650282in}{2.369915in}}%
\pgfpathlineto{\pgfqpoint{3.650528in}{2.401057in}}%
\pgfpathlineto{\pgfqpoint{3.651106in}{2.354445in}}%
\pgfpathlineto{\pgfqpoint{3.651463in}{2.393906in}}%
\pgfpathlineto{\pgfqpoint{3.652497in}{2.353415in}}%
\pgfpathlineto{\pgfqpoint{3.652608in}{2.377112in}}%
\pgfpathlineto{\pgfqpoint{3.652756in}{2.395184in}}%
\pgfpathlineto{\pgfqpoint{3.653420in}{2.352349in}}%
\pgfpathlineto{\pgfqpoint{3.653789in}{2.384869in}}%
\pgfpathlineto{\pgfqpoint{3.654811in}{2.351946in}}%
\pgfpathlineto{\pgfqpoint{3.654602in}{2.389596in}}%
\pgfpathlineto{\pgfqpoint{3.654909in}{2.375158in}}%
\pgfpathlineto{\pgfqpoint{3.655980in}{2.392797in}}%
\pgfpathlineto{\pgfqpoint{3.655734in}{2.352202in}}%
\pgfpathlineto{\pgfqpoint{3.656017in}{2.382220in}}%
\pgfpathlineto{\pgfqpoint{3.656202in}{2.352642in}}%
\pgfpathlineto{\pgfqpoint{3.656768in}{2.394825in}}%
\pgfpathlineto{\pgfqpoint{3.657199in}{2.363432in}}%
\pgfpathlineto{\pgfqpoint{3.658269in}{2.404399in}}%
\pgfpathlineto{\pgfqpoint{3.658036in}{2.348343in}}%
\pgfpathlineto{\pgfqpoint{3.658319in}{2.383330in}}%
\pgfpathlineto{\pgfqpoint{3.659414in}{2.353726in}}%
\pgfpathlineto{\pgfqpoint{3.659192in}{2.405650in}}%
\pgfpathlineto{\pgfqpoint{3.659488in}{2.361939in}}%
\pgfpathlineto{\pgfqpoint{3.659648in}{2.404416in}}%
\pgfpathlineto{\pgfqpoint{3.660374in}{2.350770in}}%
\pgfpathlineto{\pgfqpoint{3.660620in}{2.377506in}}%
\pgfpathlineto{\pgfqpoint{3.660842in}{2.353123in}}%
\pgfpathlineto{\pgfqpoint{3.661039in}{2.398538in}}%
\pgfpathlineto{\pgfqpoint{3.661765in}{2.364308in}}%
\pgfpathlineto{\pgfqpoint{3.662269in}{2.397663in}}%
\pgfpathlineto{\pgfqpoint{3.662651in}{2.358192in}}%
\pgfpathlineto{\pgfqpoint{3.662897in}{2.377767in}}%
\pgfpathlineto{\pgfqpoint{3.663451in}{2.357500in}}%
\pgfpathlineto{\pgfqpoint{3.663635in}{2.397383in}}%
\pgfpathlineto{\pgfqpoint{3.664029in}{2.358863in}}%
\pgfpathlineto{\pgfqpoint{3.664546in}{2.400196in}}%
\pgfpathlineto{\pgfqpoint{3.664940in}{2.358057in}}%
\pgfpathlineto{\pgfqpoint{3.665174in}{2.373563in}}%
\pgfpathlineto{\pgfqpoint{3.665395in}{2.360550in}}%
\pgfpathlineto{\pgfqpoint{3.665912in}{2.407453in}}%
\pgfpathlineto{\pgfqpoint{3.666257in}{2.375808in}}%
\pgfpathlineto{\pgfqpoint{3.666368in}{2.406342in}}%
\pgfpathlineto{\pgfqpoint{3.666319in}{2.361743in}}%
\pgfpathlineto{\pgfqpoint{3.667352in}{2.372023in}}%
\pgfpathlineto{\pgfqpoint{3.667734in}{2.399994in}}%
\pgfpathlineto{\pgfqpoint{3.668115in}{2.359997in}}%
\pgfpathlineto{\pgfqpoint{3.668448in}{2.371154in}}%
\pgfpathlineto{\pgfqpoint{3.669494in}{2.357759in}}%
\pgfpathlineto{\pgfqpoint{3.669100in}{2.399842in}}%
\pgfpathlineto{\pgfqpoint{3.669519in}{2.373133in}}%
\pgfpathlineto{\pgfqpoint{3.670011in}{2.401283in}}%
\pgfpathlineto{\pgfqpoint{3.670195in}{2.354068in}}%
\pgfpathlineto{\pgfqpoint{3.670626in}{2.369458in}}%
\pgfpathlineto{\pgfqpoint{3.671106in}{2.349882in}}%
\pgfpathlineto{\pgfqpoint{3.671377in}{2.403915in}}%
\pgfpathlineto{\pgfqpoint{3.671722in}{2.375733in}}%
\pgfpathlineto{\pgfqpoint{3.671832in}{2.400573in}}%
\pgfpathlineto{\pgfqpoint{3.672017in}{2.355224in}}%
\pgfpathlineto{\pgfqpoint{3.672829in}{2.377428in}}%
\pgfpathlineto{\pgfqpoint{3.672842in}{2.377269in}}%
\pgfpathlineto{\pgfqpoint{3.672866in}{2.384534in}}%
\pgfpathlineto{\pgfqpoint{3.673211in}{2.399719in}}%
\pgfpathlineto{\pgfqpoint{3.673395in}{2.357849in}}%
\pgfpathlineto{\pgfqpoint{3.673949in}{2.370353in}}%
\pgfpathlineto{\pgfqpoint{3.674122in}{2.395799in}}%
\pgfpathlineto{\pgfqpoint{3.674319in}{2.360595in}}%
\pgfpathlineto{\pgfqpoint{3.675057in}{2.379051in}}%
\pgfpathlineto{\pgfqpoint{3.675771in}{2.360572in}}%
\pgfpathlineto{\pgfqpoint{3.675955in}{2.394236in}}%
\pgfpathlineto{\pgfqpoint{3.676165in}{2.371627in}}%
\pgfpathlineto{\pgfqpoint{3.676879in}{2.394595in}}%
\pgfpathlineto{\pgfqpoint{3.676239in}{2.361336in}}%
\pgfpathlineto{\pgfqpoint{3.677297in}{2.380874in}}%
\pgfpathlineto{\pgfqpoint{3.678097in}{2.361918in}}%
\pgfpathlineto{\pgfqpoint{3.677802in}{2.393943in}}%
\pgfpathlineto{\pgfqpoint{3.678257in}{2.390174in}}%
\pgfpathlineto{\pgfqpoint{3.679192in}{2.394699in}}%
\pgfpathlineto{\pgfqpoint{3.679020in}{2.361886in}}%
\pgfpathlineto{\pgfqpoint{3.679315in}{2.374970in}}%
\pgfpathlineto{\pgfqpoint{3.680300in}{2.361835in}}%
\pgfpathlineto{\pgfqpoint{3.679660in}{2.392697in}}%
\pgfpathlineto{\pgfqpoint{3.680435in}{2.369500in}}%
\pgfpathlineto{\pgfqpoint{3.681445in}{2.389974in}}%
\pgfpathlineto{\pgfqpoint{3.681235in}{2.358404in}}%
\pgfpathlineto{\pgfqpoint{3.681568in}{2.379904in}}%
\pgfpathlineto{\pgfqpoint{3.682085in}{2.357985in}}%
\pgfpathlineto{\pgfqpoint{3.682368in}{2.392448in}}%
\pgfpathlineto{\pgfqpoint{3.682675in}{2.377924in}}%
\pgfpathlineto{\pgfqpoint{3.682909in}{2.391730in}}%
\pgfpathlineto{\pgfqpoint{3.683475in}{2.358987in}}%
\pgfpathlineto{\pgfqpoint{3.683832in}{2.389499in}}%
\pgfpathlineto{\pgfqpoint{3.683845in}{2.390867in}}%
\pgfpathlineto{\pgfqpoint{3.684411in}{2.355369in}}%
\pgfpathlineto{\pgfqpoint{3.684829in}{2.376802in}}%
\pgfpathlineto{\pgfqpoint{3.685814in}{2.349693in}}%
\pgfpathlineto{\pgfqpoint{3.685506in}{2.392180in}}%
\pgfpathlineto{\pgfqpoint{3.685925in}{2.371631in}}%
\pgfpathlineto{\pgfqpoint{3.686897in}{2.394862in}}%
\pgfpathlineto{\pgfqpoint{3.686737in}{2.345812in}}%
\pgfpathlineto{\pgfqpoint{3.687045in}{2.384912in}}%
\pgfpathlineto{\pgfqpoint{3.687205in}{2.346965in}}%
\pgfpathlineto{\pgfqpoint{3.687832in}{2.397257in}}%
\pgfpathlineto{\pgfqpoint{3.688214in}{2.365187in}}%
\pgfpathlineto{\pgfqpoint{3.689235in}{2.401292in}}%
\pgfpathlineto{\pgfqpoint{3.689063in}{2.355677in}}%
\pgfpathlineto{\pgfqpoint{3.689371in}{2.383141in}}%
\pgfpathlineto{\pgfqpoint{3.689543in}{2.352007in}}%
\pgfpathlineto{\pgfqpoint{3.689703in}{2.399541in}}%
\pgfpathlineto{\pgfqpoint{3.690540in}{2.369908in}}%
\pgfpathlineto{\pgfqpoint{3.691094in}{2.398648in}}%
\pgfpathlineto{\pgfqpoint{3.691389in}{2.351589in}}%
\pgfpathlineto{\pgfqpoint{3.691660in}{2.373263in}}%
\pgfpathlineto{\pgfqpoint{3.691857in}{2.353515in}}%
\pgfpathlineto{\pgfqpoint{3.692485in}{2.398192in}}%
\pgfpathlineto{\pgfqpoint{3.692817in}{2.365827in}}%
\pgfpathlineto{\pgfqpoint{3.693888in}{2.399476in}}%
\pgfpathlineto{\pgfqpoint{3.693728in}{2.356762in}}%
\pgfpathlineto{\pgfqpoint{3.693962in}{2.378709in}}%
\pgfpathlineto{\pgfqpoint{3.694195in}{2.357231in}}%
\pgfpathlineto{\pgfqpoint{3.694811in}{2.397387in}}%
\pgfpathlineto{\pgfqpoint{3.695143in}{2.368052in}}%
\pgfpathlineto{\pgfqpoint{3.695746in}{2.396967in}}%
\pgfpathlineto{\pgfqpoint{3.695857in}{2.361814in}}%
\pgfpathlineto{\pgfqpoint{3.696263in}{2.377300in}}%
\pgfpathlineto{\pgfqpoint{3.696275in}{2.377417in}}%
\pgfpathlineto{\pgfqpoint{3.696300in}{2.368148in}}%
\pgfpathlineto{\pgfqpoint{3.697248in}{2.356040in}}%
\pgfpathlineto{\pgfqpoint{3.696657in}{2.393066in}}%
\pgfpathlineto{\pgfqpoint{3.697408in}{2.367976in}}%
\pgfpathlineto{\pgfqpoint{3.697703in}{2.355898in}}%
\pgfpathlineto{\pgfqpoint{3.698048in}{2.389153in}}%
\pgfpathlineto{\pgfqpoint{3.698405in}{2.377981in}}%
\pgfpathlineto{\pgfqpoint{3.698946in}{2.389148in}}%
\pgfpathlineto{\pgfqpoint{3.699094in}{2.357762in}}%
\pgfpathlineto{\pgfqpoint{3.699488in}{2.377942in}}%
\pgfpathlineto{\pgfqpoint{3.699562in}{2.358065in}}%
\pgfpathlineto{\pgfqpoint{3.700337in}{2.388800in}}%
\pgfpathlineto{\pgfqpoint{3.700583in}{2.381430in}}%
\pgfpathlineto{\pgfqpoint{3.701260in}{2.389307in}}%
\pgfpathlineto{\pgfqpoint{3.700952in}{2.359448in}}%
\pgfpathlineto{\pgfqpoint{3.701383in}{2.370723in}}%
\pgfpathlineto{\pgfqpoint{3.701408in}{2.359213in}}%
\pgfpathlineto{\pgfqpoint{3.701728in}{2.390011in}}%
\pgfpathlineto{\pgfqpoint{3.702491in}{2.368494in}}%
\pgfpathlineto{\pgfqpoint{3.702577in}{2.393188in}}%
\pgfpathlineto{\pgfqpoint{3.702860in}{2.362291in}}%
\pgfpathlineto{\pgfqpoint{3.703611in}{2.378604in}}%
\pgfpathlineto{\pgfqpoint{3.704694in}{2.359885in}}%
\pgfpathlineto{\pgfqpoint{3.704411in}{2.395203in}}%
\pgfpathlineto{\pgfqpoint{3.704718in}{2.374463in}}%
\pgfpathlineto{\pgfqpoint{3.704878in}{2.394537in}}%
\pgfpathlineto{\pgfqpoint{3.705026in}{2.360474in}}%
\pgfpathlineto{\pgfqpoint{3.705826in}{2.381008in}}%
\pgfpathlineto{\pgfqpoint{3.706909in}{2.363357in}}%
\pgfpathlineto{\pgfqpoint{3.706257in}{2.393613in}}%
\pgfpathlineto{\pgfqpoint{3.706983in}{2.367231in}}%
\pgfpathlineto{\pgfqpoint{3.707943in}{2.393621in}}%
\pgfpathlineto{\pgfqpoint{3.707980in}{2.363339in}}%
\pgfpathlineto{\pgfqpoint{3.708128in}{2.383371in}}%
\pgfpathlineto{\pgfqpoint{3.708152in}{2.384035in}}%
\pgfpathlineto{\pgfqpoint{3.708189in}{2.373530in}}%
\pgfpathlineto{\pgfqpoint{3.709235in}{2.359956in}}%
\pgfpathlineto{\pgfqpoint{3.708854in}{2.401904in}}%
\pgfpathlineto{\pgfqpoint{3.709272in}{2.375791in}}%
\pgfpathlineto{\pgfqpoint{3.710220in}{2.404563in}}%
\pgfpathlineto{\pgfqpoint{3.710158in}{2.357291in}}%
\pgfpathlineto{\pgfqpoint{3.710380in}{2.381959in}}%
\pgfpathlineto{\pgfqpoint{3.711537in}{2.353594in}}%
\pgfpathlineto{\pgfqpoint{3.711131in}{2.404857in}}%
\pgfpathlineto{\pgfqpoint{3.711549in}{2.355618in}}%
\pgfpathlineto{\pgfqpoint{3.712054in}{2.402703in}}%
\pgfpathlineto{\pgfqpoint{3.712669in}{2.372271in}}%
\pgfpathlineto{\pgfqpoint{3.713358in}{2.360962in}}%
\pgfpathlineto{\pgfqpoint{3.712965in}{2.398977in}}%
\pgfpathlineto{\pgfqpoint{3.713765in}{2.376666in}}%
\pgfpathlineto{\pgfqpoint{3.713777in}{2.376788in}}%
\pgfpathlineto{\pgfqpoint{3.713789in}{2.369992in}}%
\pgfpathlineto{\pgfqpoint{3.714725in}{2.360150in}}%
\pgfpathlineto{\pgfqpoint{3.714786in}{2.398977in}}%
\pgfpathlineto{\pgfqpoint{3.714885in}{2.377903in}}%
\pgfpathlineto{\pgfqpoint{3.714897in}{2.377899in}}%
\pgfpathlineto{\pgfqpoint{3.715241in}{2.401352in}}%
\pgfpathlineto{\pgfqpoint{3.715525in}{2.360055in}}%
\pgfpathlineto{\pgfqpoint{3.715955in}{2.371678in}}%
\pgfpathlineto{\pgfqpoint{3.716435in}{2.360098in}}%
\pgfpathlineto{\pgfqpoint{3.716152in}{2.401241in}}%
\pgfpathlineto{\pgfqpoint{3.717038in}{2.380361in}}%
\pgfpathlineto{\pgfqpoint{3.717063in}{2.400249in}}%
\pgfpathlineto{\pgfqpoint{3.717814in}{2.359196in}}%
\pgfpathlineto{\pgfqpoint{3.718134in}{2.385639in}}%
\pgfpathlineto{\pgfqpoint{3.718269in}{2.362103in}}%
\pgfpathlineto{\pgfqpoint{3.718897in}{2.397218in}}%
\pgfpathlineto{\pgfqpoint{3.719254in}{2.367554in}}%
\pgfpathlineto{\pgfqpoint{3.719352in}{2.396617in}}%
\pgfpathlineto{\pgfqpoint{3.720091in}{2.361575in}}%
\pgfpathlineto{\pgfqpoint{3.720423in}{2.384036in}}%
\pgfpathlineto{\pgfqpoint{3.721469in}{2.356049in}}%
\pgfpathlineto{\pgfqpoint{3.720731in}{2.397211in}}%
\pgfpathlineto{\pgfqpoint{3.721568in}{2.364525in}}%
\pgfpathlineto{\pgfqpoint{3.722577in}{2.398717in}}%
\pgfpathlineto{\pgfqpoint{3.722392in}{2.355172in}}%
\pgfpathlineto{\pgfqpoint{3.722737in}{2.371793in}}%
\pgfpathlineto{\pgfqpoint{3.723328in}{2.357619in}}%
\pgfpathlineto{\pgfqpoint{3.723032in}{2.399651in}}%
\pgfpathlineto{\pgfqpoint{3.723869in}{2.365146in}}%
\pgfpathlineto{\pgfqpoint{3.724423in}{2.401192in}}%
\pgfpathlineto{\pgfqpoint{3.724251in}{2.358033in}}%
\pgfpathlineto{\pgfqpoint{3.724989in}{2.377330in}}%
\pgfpathlineto{\pgfqpoint{3.725346in}{2.401694in}}%
\pgfpathlineto{\pgfqpoint{3.725641in}{2.360812in}}%
\pgfpathlineto{\pgfqpoint{3.726072in}{2.378167in}}%
\pgfpathlineto{\pgfqpoint{3.726921in}{2.359015in}}%
\pgfpathlineto{\pgfqpoint{3.726281in}{2.397605in}}%
\pgfpathlineto{\pgfqpoint{3.727168in}{2.381422in}}%
\pgfpathlineto{\pgfqpoint{3.727204in}{2.396105in}}%
\pgfpathlineto{\pgfqpoint{3.727844in}{2.357410in}}%
\pgfpathlineto{\pgfqpoint{3.728201in}{2.376835in}}%
\pgfpathlineto{\pgfqpoint{3.728312in}{2.357144in}}%
\pgfpathlineto{\pgfqpoint{3.728608in}{2.393212in}}%
\pgfpathlineto{\pgfqpoint{3.729309in}{2.378343in}}%
\pgfpathlineto{\pgfqpoint{3.729321in}{2.378407in}}%
\pgfpathlineto{\pgfqpoint{3.729334in}{2.375559in}}%
\pgfpathlineto{\pgfqpoint{3.730171in}{2.358121in}}%
\pgfpathlineto{\pgfqpoint{3.729998in}{2.392094in}}%
\pgfpathlineto{\pgfqpoint{3.730417in}{2.381298in}}%
\pgfpathlineto{\pgfqpoint{3.730466in}{2.390808in}}%
\pgfpathlineto{\pgfqpoint{3.731032in}{2.357693in}}%
\pgfpathlineto{\pgfqpoint{3.731438in}{2.375523in}}%
\pgfpathlineto{\pgfqpoint{3.731955in}{2.354872in}}%
\pgfpathlineto{\pgfqpoint{3.731857in}{2.388874in}}%
\pgfpathlineto{\pgfqpoint{3.732534in}{2.373497in}}%
\pgfpathlineto{\pgfqpoint{3.732792in}{2.386462in}}%
\pgfpathlineto{\pgfqpoint{3.732952in}{2.356279in}}%
\pgfpathlineto{\pgfqpoint{3.733654in}{2.383254in}}%
\pgfpathlineto{\pgfqpoint{3.733875in}{2.356835in}}%
\pgfpathlineto{\pgfqpoint{3.734454in}{2.386611in}}%
\pgfpathlineto{\pgfqpoint{3.734860in}{2.375606in}}%
\pgfpathlineto{\pgfqpoint{3.735844in}{2.392347in}}%
\pgfpathlineto{\pgfqpoint{3.735697in}{2.358514in}}%
\pgfpathlineto{\pgfqpoint{3.736004in}{2.383258in}}%
\pgfpathlineto{\pgfqpoint{3.737075in}{2.355335in}}%
\pgfpathlineto{\pgfqpoint{3.736312in}{2.392176in}}%
\pgfpathlineto{\pgfqpoint{3.737174in}{2.369863in}}%
\pgfpathlineto{\pgfqpoint{3.737235in}{2.388642in}}%
\pgfpathlineto{\pgfqpoint{3.737543in}{2.356323in}}%
\pgfpathlineto{\pgfqpoint{3.738294in}{2.377569in}}%
\pgfpathlineto{\pgfqpoint{3.739106in}{2.389597in}}%
\pgfpathlineto{\pgfqpoint{3.738466in}{2.361184in}}%
\pgfpathlineto{\pgfqpoint{3.739328in}{2.372809in}}%
\pgfpathlineto{\pgfqpoint{3.740337in}{2.361962in}}%
\pgfpathlineto{\pgfqpoint{3.739561in}{2.389031in}}%
\pgfpathlineto{\pgfqpoint{3.740423in}{2.376724in}}%
\pgfpathlineto{\pgfqpoint{3.741395in}{2.389073in}}%
\pgfpathlineto{\pgfqpoint{3.741198in}{2.360683in}}%
\pgfpathlineto{\pgfqpoint{3.741506in}{2.374467in}}%
\pgfpathlineto{\pgfqpoint{3.742454in}{2.360778in}}%
\pgfpathlineto{\pgfqpoint{3.742355in}{2.388535in}}%
\pgfpathlineto{\pgfqpoint{3.742626in}{2.367641in}}%
\pgfpathlineto{\pgfqpoint{3.743377in}{2.360559in}}%
\pgfpathlineto{\pgfqpoint{3.742811in}{2.388681in}}%
\pgfpathlineto{\pgfqpoint{3.743598in}{2.379267in}}%
\pgfpathlineto{\pgfqpoint{3.743684in}{2.389617in}}%
\pgfpathlineto{\pgfqpoint{3.744448in}{2.358801in}}%
\pgfpathlineto{\pgfqpoint{3.744718in}{2.381575in}}%
\pgfpathlineto{\pgfqpoint{3.745826in}{2.357394in}}%
\pgfpathlineto{\pgfqpoint{3.745543in}{2.389802in}}%
\pgfpathlineto{\pgfqpoint{3.745863in}{2.371436in}}%
\pgfpathlineto{\pgfqpoint{3.746921in}{2.391427in}}%
\pgfpathlineto{\pgfqpoint{3.746294in}{2.358111in}}%
\pgfpathlineto{\pgfqpoint{3.746971in}{2.377409in}}%
\pgfpathlineto{\pgfqpoint{3.747204in}{2.357793in}}%
\pgfpathlineto{\pgfqpoint{3.747389in}{2.390437in}}%
\pgfpathlineto{\pgfqpoint{3.748066in}{2.370529in}}%
\pgfpathlineto{\pgfqpoint{3.748546in}{2.386759in}}%
\pgfpathlineto{\pgfqpoint{3.748127in}{2.359926in}}%
\pgfpathlineto{\pgfqpoint{3.749186in}{2.380255in}}%
\pgfpathlineto{\pgfqpoint{3.750269in}{2.362418in}}%
\pgfpathlineto{\pgfqpoint{3.749617in}{2.387981in}}%
\pgfpathlineto{\pgfqpoint{3.750355in}{2.373192in}}%
\pgfpathlineto{\pgfqpoint{3.750540in}{2.388423in}}%
\pgfpathlineto{\pgfqpoint{3.751327in}{2.358972in}}%
\pgfpathlineto{\pgfqpoint{3.751475in}{2.381609in}}%
\pgfpathlineto{\pgfqpoint{3.752214in}{2.390360in}}%
\pgfpathlineto{\pgfqpoint{3.751795in}{2.359196in}}%
\pgfpathlineto{\pgfqpoint{3.752521in}{2.374413in}}%
\pgfpathlineto{\pgfqpoint{3.753124in}{2.395140in}}%
\pgfpathlineto{\pgfqpoint{3.752706in}{2.362440in}}%
\pgfpathlineto{\pgfqpoint{3.753506in}{2.365686in}}%
\pgfpathlineto{\pgfqpoint{3.754441in}{2.362594in}}%
\pgfpathlineto{\pgfqpoint{3.753580in}{2.397328in}}%
\pgfpathlineto{\pgfqpoint{3.754478in}{2.383690in}}%
\pgfpathlineto{\pgfqpoint{3.754503in}{2.396957in}}%
\pgfpathlineto{\pgfqpoint{3.754897in}{2.361204in}}%
\pgfpathlineto{\pgfqpoint{3.755574in}{2.377454in}}%
\pgfpathlineto{\pgfqpoint{3.755820in}{2.364134in}}%
\pgfpathlineto{\pgfqpoint{3.756337in}{2.395787in}}%
\pgfpathlineto{\pgfqpoint{3.756706in}{2.367959in}}%
\pgfpathlineto{\pgfqpoint{3.757518in}{2.365884in}}%
\pgfpathlineto{\pgfqpoint{3.756792in}{2.395701in}}%
\pgfpathlineto{\pgfqpoint{3.757678in}{2.380088in}}%
\pgfpathlineto{\pgfqpoint{3.758614in}{2.395668in}}%
\pgfpathlineto{\pgfqpoint{3.758441in}{2.363713in}}%
\pgfpathlineto{\pgfqpoint{3.758786in}{2.378020in}}%
\pgfpathlineto{\pgfqpoint{3.759820in}{2.360969in}}%
\pgfpathlineto{\pgfqpoint{3.759537in}{2.399459in}}%
\pgfpathlineto{\pgfqpoint{3.759931in}{2.366961in}}%
\pgfpathlineto{\pgfqpoint{3.759992in}{2.399451in}}%
\pgfpathlineto{\pgfqpoint{3.760287in}{2.360452in}}%
\pgfpathlineto{\pgfqpoint{3.761063in}{2.379903in}}%
\pgfpathlineto{\pgfqpoint{3.762121in}{2.358354in}}%
\pgfpathlineto{\pgfqpoint{3.761826in}{2.396566in}}%
\pgfpathlineto{\pgfqpoint{3.762195in}{2.369101in}}%
\pgfpathlineto{\pgfqpoint{3.762220in}{2.367207in}}%
\pgfpathlineto{\pgfqpoint{3.762244in}{2.371047in}}%
\pgfpathlineto{\pgfqpoint{3.762281in}{2.395289in}}%
\pgfpathlineto{\pgfqpoint{3.763032in}{2.356291in}}%
\pgfpathlineto{\pgfqpoint{3.763364in}{2.377636in}}%
\pgfpathlineto{\pgfqpoint{3.764411in}{2.358262in}}%
\pgfpathlineto{\pgfqpoint{3.763660in}{2.395561in}}%
\pgfpathlineto{\pgfqpoint{3.764497in}{2.364020in}}%
\pgfpathlineto{\pgfqpoint{3.765038in}{2.398659in}}%
\pgfpathlineto{\pgfqpoint{3.764878in}{2.358512in}}%
\pgfpathlineto{\pgfqpoint{3.765654in}{2.379665in}}%
\pgfpathlineto{\pgfqpoint{3.766712in}{2.359141in}}%
\pgfpathlineto{\pgfqpoint{3.766417in}{2.400520in}}%
\pgfpathlineto{\pgfqpoint{3.766798in}{2.363778in}}%
\pgfpathlineto{\pgfqpoint{3.767807in}{2.403336in}}%
\pgfpathlineto{\pgfqpoint{3.767241in}{2.358912in}}%
\pgfpathlineto{\pgfqpoint{3.767955in}{2.377875in}}%
\pgfpathlineto{\pgfqpoint{3.769087in}{2.359184in}}%
\pgfpathlineto{\pgfqpoint{3.768730in}{2.403305in}}%
\pgfpathlineto{\pgfqpoint{3.769100in}{2.363354in}}%
\pgfpathlineto{\pgfqpoint{3.769654in}{2.401328in}}%
\pgfpathlineto{\pgfqpoint{3.769949in}{2.359754in}}%
\pgfpathlineto{\pgfqpoint{3.770232in}{2.381399in}}%
\pgfpathlineto{\pgfqpoint{3.770417in}{2.359522in}}%
\pgfpathlineto{\pgfqpoint{3.771044in}{2.397465in}}%
\pgfpathlineto{\pgfqpoint{3.771414in}{2.368988in}}%
\pgfpathlineto{\pgfqpoint{3.771980in}{2.395676in}}%
\pgfpathlineto{\pgfqpoint{3.771807in}{2.361031in}}%
\pgfpathlineto{\pgfqpoint{3.772521in}{2.369680in}}%
\pgfpathlineto{\pgfqpoint{3.773370in}{2.393984in}}%
\pgfpathlineto{\pgfqpoint{3.773555in}{2.360849in}}%
\pgfpathlineto{\pgfqpoint{3.773629in}{2.374211in}}%
\pgfpathlineto{\pgfqpoint{3.774466in}{2.359303in}}%
\pgfpathlineto{\pgfqpoint{3.773838in}{2.393980in}}%
\pgfpathlineto{\pgfqpoint{3.774712in}{2.384597in}}%
\pgfpathlineto{\pgfqpoint{3.774774in}{2.391819in}}%
\pgfpathlineto{\pgfqpoint{3.774860in}{2.363279in}}%
\pgfpathlineto{\pgfqpoint{3.774909in}{2.368593in}}%
\pgfpathlineto{\pgfqpoint{3.775401in}{2.360550in}}%
\pgfpathlineto{\pgfqpoint{3.775623in}{2.392799in}}%
\pgfpathlineto{\pgfqpoint{3.776004in}{2.373619in}}%
\pgfpathlineto{\pgfqpoint{3.776090in}{2.393004in}}%
\pgfpathlineto{\pgfqpoint{3.776792in}{2.358734in}}%
\pgfpathlineto{\pgfqpoint{3.777112in}{2.380278in}}%
\pgfpathlineto{\pgfqpoint{3.777260in}{2.359594in}}%
\pgfpathlineto{\pgfqpoint{3.777949in}{2.389593in}}%
\pgfpathlineto{\pgfqpoint{3.778232in}{2.373451in}}%
\pgfpathlineto{\pgfqpoint{3.778884in}{2.390620in}}%
\pgfpathlineto{\pgfqpoint{3.779069in}{2.361433in}}%
\pgfpathlineto{\pgfqpoint{3.779364in}{2.385027in}}%
\pgfpathlineto{\pgfqpoint{3.780460in}{2.358080in}}%
\pgfpathlineto{\pgfqpoint{3.780152in}{2.389755in}}%
\pgfpathlineto{\pgfqpoint{3.780509in}{2.369637in}}%
\pgfpathlineto{\pgfqpoint{3.781075in}{2.390905in}}%
\pgfpathlineto{\pgfqpoint{3.781383in}{2.356464in}}%
\pgfpathlineto{\pgfqpoint{3.781641in}{2.375095in}}%
\pgfpathlineto{\pgfqpoint{3.781850in}{2.358521in}}%
\pgfpathlineto{\pgfqpoint{3.782478in}{2.392931in}}%
\pgfpathlineto{\pgfqpoint{3.782786in}{2.361174in}}%
\pgfpathlineto{\pgfqpoint{3.783869in}{2.394499in}}%
\pgfpathlineto{\pgfqpoint{3.783955in}{2.379134in}}%
\pgfpathlineto{\pgfqpoint{3.784915in}{2.361162in}}%
\pgfpathlineto{\pgfqpoint{3.784804in}{2.393108in}}%
\pgfpathlineto{\pgfqpoint{3.785087in}{2.365745in}}%
\pgfpathlineto{\pgfqpoint{3.785838in}{2.358463in}}%
\pgfpathlineto{\pgfqpoint{3.785272in}{2.393294in}}%
\pgfpathlineto{\pgfqpoint{3.786097in}{2.379684in}}%
\pgfpathlineto{\pgfqpoint{3.786306in}{2.358476in}}%
\pgfpathlineto{\pgfqpoint{3.786195in}{2.393379in}}%
\pgfpathlineto{\pgfqpoint{3.787093in}{2.386973in}}%
\pgfpathlineto{\pgfqpoint{3.787118in}{2.390427in}}%
\pgfpathlineto{\pgfqpoint{3.787697in}{2.357978in}}%
\pgfpathlineto{\pgfqpoint{3.788127in}{2.372980in}}%
\pgfpathlineto{\pgfqpoint{3.788620in}{2.356530in}}%
\pgfpathlineto{\pgfqpoint{3.788484in}{2.386915in}}%
\pgfpathlineto{\pgfqpoint{3.789235in}{2.368753in}}%
\pgfpathlineto{\pgfqpoint{3.790244in}{2.387986in}}%
\pgfpathlineto{\pgfqpoint{3.789543in}{2.358065in}}%
\pgfpathlineto{\pgfqpoint{3.790367in}{2.383324in}}%
\pgfpathlineto{\pgfqpoint{3.791450in}{2.355014in}}%
\pgfpathlineto{\pgfqpoint{3.790577in}{2.388673in}}%
\pgfpathlineto{\pgfqpoint{3.791475in}{2.362742in}}%
\pgfpathlineto{\pgfqpoint{3.792423in}{2.390722in}}%
\pgfpathlineto{\pgfqpoint{3.791918in}{2.353585in}}%
\pgfpathlineto{\pgfqpoint{3.792583in}{2.376907in}}%
\pgfpathlineto{\pgfqpoint{3.792841in}{2.355299in}}%
\pgfpathlineto{\pgfqpoint{3.793346in}{2.395959in}}%
\pgfpathlineto{\pgfqpoint{3.793715in}{2.367862in}}%
\pgfpathlineto{\pgfqpoint{3.793937in}{2.398642in}}%
\pgfpathlineto{\pgfqpoint{3.793764in}{2.359244in}}%
\pgfpathlineto{\pgfqpoint{3.794884in}{2.386294in}}%
\pgfpathlineto{\pgfqpoint{3.795610in}{2.357898in}}%
\pgfpathlineto{\pgfqpoint{3.795192in}{2.396147in}}%
\pgfpathlineto{\pgfqpoint{3.796017in}{2.368054in}}%
\pgfpathlineto{\pgfqpoint{3.796238in}{2.394132in}}%
\pgfpathlineto{\pgfqpoint{3.796521in}{2.357633in}}%
\pgfpathlineto{\pgfqpoint{3.797272in}{2.378196in}}%
\pgfpathlineto{\pgfqpoint{3.797444in}{2.361438in}}%
\pgfpathlineto{\pgfqpoint{3.797949in}{2.392785in}}%
\pgfpathlineto{\pgfqpoint{3.798380in}{2.370664in}}%
\pgfpathlineto{\pgfqpoint{3.798404in}{2.394114in}}%
\pgfpathlineto{\pgfqpoint{3.799217in}{2.359964in}}%
\pgfpathlineto{\pgfqpoint{3.799487in}{2.380538in}}%
\pgfpathlineto{\pgfqpoint{3.799684in}{2.359301in}}%
\pgfpathlineto{\pgfqpoint{3.800238in}{2.395778in}}%
\pgfpathlineto{\pgfqpoint{3.800644in}{2.371547in}}%
\pgfpathlineto{\pgfqpoint{3.801617in}{2.401302in}}%
\pgfpathlineto{\pgfqpoint{3.800743in}{2.360843in}}%
\pgfpathlineto{\pgfqpoint{3.801777in}{2.379537in}}%
\pgfpathlineto{\pgfqpoint{3.802010in}{2.362011in}}%
\pgfpathlineto{\pgfqpoint{3.802540in}{2.401219in}}%
\pgfpathlineto{\pgfqpoint{3.802921in}{2.365767in}}%
\pgfpathlineto{\pgfqpoint{3.803500in}{2.362683in}}%
\pgfpathlineto{\pgfqpoint{3.802995in}{2.402854in}}%
\pgfpathlineto{\pgfqpoint{3.803893in}{2.384409in}}%
\pgfpathlineto{\pgfqpoint{3.804373in}{2.405240in}}%
\pgfpathlineto{\pgfqpoint{3.803955in}{2.365202in}}%
\pgfpathlineto{\pgfqpoint{3.804989in}{2.378559in}}%
\pgfpathlineto{\pgfqpoint{3.805937in}{2.363659in}}%
\pgfpathlineto{\pgfqpoint{3.805297in}{2.404089in}}%
\pgfpathlineto{\pgfqpoint{3.806109in}{2.367839in}}%
\pgfpathlineto{\pgfqpoint{3.806121in}{2.367661in}}%
\pgfpathlineto{\pgfqpoint{3.806195in}{2.381337in}}%
\pgfpathlineto{\pgfqpoint{3.806220in}{2.395513in}}%
\pgfpathlineto{\pgfqpoint{3.806404in}{2.363045in}}%
\pgfpathlineto{\pgfqpoint{3.807290in}{2.372191in}}%
\pgfpathlineto{\pgfqpoint{3.807869in}{2.365310in}}%
\pgfpathlineto{\pgfqpoint{3.808053in}{2.396792in}}%
\pgfpathlineto{\pgfqpoint{3.808361in}{2.377812in}}%
\pgfpathlineto{\pgfqpoint{3.809432in}{2.398932in}}%
\pgfpathlineto{\pgfqpoint{3.809149in}{2.364794in}}%
\pgfpathlineto{\pgfqpoint{3.809456in}{2.379207in}}%
\pgfpathlineto{\pgfqpoint{3.810527in}{2.361154in}}%
\pgfpathlineto{\pgfqpoint{3.810355in}{2.399126in}}%
\pgfpathlineto{\pgfqpoint{3.810576in}{2.372439in}}%
\pgfpathlineto{\pgfqpoint{3.810995in}{2.360773in}}%
\pgfpathlineto{\pgfqpoint{3.810823in}{2.398440in}}%
\pgfpathlineto{\pgfqpoint{3.811598in}{2.377601in}}%
\pgfpathlineto{\pgfqpoint{3.811746in}{2.397874in}}%
\pgfpathlineto{\pgfqpoint{3.811918in}{2.361291in}}%
\pgfpathlineto{\pgfqpoint{3.812693in}{2.377616in}}%
\pgfpathlineto{\pgfqpoint{3.813690in}{2.359373in}}%
\pgfpathlineto{\pgfqpoint{3.813136in}{2.395022in}}%
\pgfpathlineto{\pgfqpoint{3.813801in}{2.368250in}}%
\pgfpathlineto{\pgfqpoint{3.814453in}{2.392521in}}%
\pgfpathlineto{\pgfqpoint{3.814613in}{2.356136in}}%
\pgfpathlineto{\pgfqpoint{3.814933in}{2.382103in}}%
\pgfpathlineto{\pgfqpoint{3.815536in}{2.352052in}}%
\pgfpathlineto{\pgfqpoint{3.815844in}{2.395352in}}%
\pgfpathlineto{\pgfqpoint{3.816066in}{2.369450in}}%
\pgfpathlineto{\pgfqpoint{3.816927in}{2.350369in}}%
\pgfpathlineto{\pgfqpoint{3.816632in}{2.398087in}}%
\pgfpathlineto{\pgfqpoint{3.817087in}{2.397023in}}%
\pgfpathlineto{\pgfqpoint{3.818035in}{2.400614in}}%
\pgfpathlineto{\pgfqpoint{3.817395in}{2.350831in}}%
\pgfpathlineto{\pgfqpoint{3.818072in}{2.375297in}}%
\pgfpathlineto{\pgfqpoint{3.818109in}{2.381617in}}%
\pgfpathlineto{\pgfqpoint{3.818958in}{2.403953in}}%
\pgfpathlineto{\pgfqpoint{3.818330in}{2.351660in}}%
\pgfpathlineto{\pgfqpoint{3.819192in}{2.375712in}}%
\pgfpathlineto{\pgfqpoint{3.819733in}{2.354420in}}%
\pgfpathlineto{\pgfqpoint{3.819426in}{2.406276in}}%
\pgfpathlineto{\pgfqpoint{3.820287in}{2.375602in}}%
\pgfpathlineto{\pgfqpoint{3.820349in}{2.406745in}}%
\pgfpathlineto{\pgfqpoint{3.820669in}{2.356416in}}%
\pgfpathlineto{\pgfqpoint{3.821383in}{2.379512in}}%
\pgfpathlineto{\pgfqpoint{3.821961in}{2.357957in}}%
\pgfpathlineto{\pgfqpoint{3.822220in}{2.403677in}}%
\pgfpathlineto{\pgfqpoint{3.822527in}{2.361193in}}%
\pgfpathlineto{\pgfqpoint{3.823278in}{2.357129in}}%
\pgfpathlineto{\pgfqpoint{3.823155in}{2.403212in}}%
\pgfpathlineto{\pgfqpoint{3.823500in}{2.363852in}}%
\pgfpathlineto{\pgfqpoint{3.824558in}{2.403961in}}%
\pgfpathlineto{\pgfqpoint{3.823746in}{2.354387in}}%
\pgfpathlineto{\pgfqpoint{3.824620in}{2.374347in}}%
\pgfpathlineto{\pgfqpoint{3.825136in}{2.353782in}}%
\pgfpathlineto{\pgfqpoint{3.825481in}{2.406274in}}%
\pgfpathlineto{\pgfqpoint{3.825703in}{2.367550in}}%
\pgfpathlineto{\pgfqpoint{3.826404in}{2.404721in}}%
\pgfpathlineto{\pgfqpoint{3.826540in}{2.352211in}}%
\pgfpathlineto{\pgfqpoint{3.826847in}{2.396326in}}%
\pgfpathlineto{\pgfqpoint{3.827352in}{2.402787in}}%
\pgfpathlineto{\pgfqpoint{3.827475in}{2.347942in}}%
\pgfpathlineto{\pgfqpoint{3.827856in}{2.375936in}}%
\pgfpathlineto{\pgfqpoint{3.828866in}{2.342613in}}%
\pgfpathlineto{\pgfqpoint{3.828287in}{2.399408in}}%
\pgfpathlineto{\pgfqpoint{3.828964in}{2.375744in}}%
\pgfpathlineto{\pgfqpoint{3.829210in}{2.396743in}}%
\pgfpathlineto{\pgfqpoint{3.829789in}{2.342530in}}%
\pgfpathlineto{\pgfqpoint{3.830109in}{2.390217in}}%
\pgfpathlineto{\pgfqpoint{3.830133in}{2.394051in}}%
\pgfpathlineto{\pgfqpoint{3.830256in}{2.342768in}}%
\pgfpathlineto{\pgfqpoint{3.831106in}{2.376013in}}%
\pgfpathlineto{\pgfqpoint{3.832115in}{2.343277in}}%
\pgfpathlineto{\pgfqpoint{3.831758in}{2.390438in}}%
\pgfpathlineto{\pgfqpoint{3.832201in}{2.376438in}}%
\pgfpathlineto{\pgfqpoint{3.833149in}{2.391525in}}%
\pgfpathlineto{\pgfqpoint{3.833038in}{2.343319in}}%
\pgfpathlineto{\pgfqpoint{3.833309in}{2.383039in}}%
\pgfpathlineto{\pgfqpoint{3.834072in}{2.396169in}}%
\pgfpathlineto{\pgfqpoint{3.833506in}{2.344984in}}%
\pgfpathlineto{\pgfqpoint{3.834330in}{2.376729in}}%
\pgfpathlineto{\pgfqpoint{3.834429in}{2.346072in}}%
\pgfpathlineto{\pgfqpoint{3.834995in}{2.399367in}}%
\pgfpathlineto{\pgfqpoint{3.835426in}{2.365311in}}%
\pgfpathlineto{\pgfqpoint{3.835918in}{2.399162in}}%
\pgfpathlineto{\pgfqpoint{3.836275in}{2.345626in}}%
\pgfpathlineto{\pgfqpoint{3.836546in}{2.376430in}}%
\pgfpathlineto{\pgfqpoint{3.836743in}{2.346593in}}%
\pgfpathlineto{\pgfqpoint{3.837309in}{2.402922in}}%
\pgfpathlineto{\pgfqpoint{3.837506in}{2.383431in}}%
\pgfpathlineto{\pgfqpoint{3.838232in}{2.405396in}}%
\pgfpathlineto{\pgfqpoint{3.837666in}{2.349829in}}%
\pgfpathlineto{\pgfqpoint{3.838564in}{2.362711in}}%
\pgfpathlineto{\pgfqpoint{3.838589in}{2.352525in}}%
\pgfpathlineto{\pgfqpoint{3.839155in}{2.403894in}}%
\pgfpathlineto{\pgfqpoint{3.839610in}{2.400072in}}%
\pgfpathlineto{\pgfqpoint{3.840078in}{2.402788in}}%
\pgfpathlineto{\pgfqpoint{3.840029in}{2.352033in}}%
\pgfpathlineto{\pgfqpoint{3.840484in}{2.354979in}}%
\pgfpathlineto{\pgfqpoint{3.840496in}{2.351282in}}%
\pgfpathlineto{\pgfqpoint{3.841001in}{2.402906in}}%
\pgfpathlineto{\pgfqpoint{3.841444in}{2.378429in}}%
\pgfpathlineto{\pgfqpoint{3.841924in}{2.403946in}}%
\pgfpathlineto{\pgfqpoint{3.842343in}{2.354906in}}%
\pgfpathlineto{\pgfqpoint{3.842552in}{2.379587in}}%
\pgfpathlineto{\pgfqpoint{3.843266in}{2.356480in}}%
\pgfpathlineto{\pgfqpoint{3.842847in}{2.400798in}}%
\pgfpathlineto{\pgfqpoint{3.843684in}{2.369884in}}%
\pgfpathlineto{\pgfqpoint{3.844570in}{2.354024in}}%
\pgfpathlineto{\pgfqpoint{3.843770in}{2.394667in}}%
\pgfpathlineto{\pgfqpoint{3.844669in}{2.377914in}}%
\pgfpathlineto{\pgfqpoint{3.845604in}{2.396733in}}%
\pgfpathlineto{\pgfqpoint{3.845493in}{2.354172in}}%
\pgfpathlineto{\pgfqpoint{3.845776in}{2.383865in}}%
\pgfpathlineto{\pgfqpoint{3.846527in}{2.397235in}}%
\pgfpathlineto{\pgfqpoint{3.846416in}{2.353577in}}%
\pgfpathlineto{\pgfqpoint{3.846810in}{2.370349in}}%
\pgfpathlineto{\pgfqpoint{3.847487in}{2.352099in}}%
\pgfpathlineto{\pgfqpoint{3.847450in}{2.397226in}}%
\pgfpathlineto{\pgfqpoint{3.847647in}{2.385265in}}%
\pgfpathlineto{\pgfqpoint{3.848373in}{2.396622in}}%
\pgfpathlineto{\pgfqpoint{3.848410in}{2.350934in}}%
\pgfpathlineto{\pgfqpoint{3.848706in}{2.371669in}}%
\pgfpathlineto{\pgfqpoint{3.849813in}{2.351420in}}%
\pgfpathlineto{\pgfqpoint{3.849752in}{2.403117in}}%
\pgfpathlineto{\pgfqpoint{3.849826in}{2.353535in}}%
\pgfpathlineto{\pgfqpoint{3.850675in}{2.402802in}}%
\pgfpathlineto{\pgfqpoint{3.850749in}{2.348632in}}%
\pgfpathlineto{\pgfqpoint{3.850958in}{2.369974in}}%
\pgfpathlineto{\pgfqpoint{3.851672in}{2.348329in}}%
\pgfpathlineto{\pgfqpoint{3.851598in}{2.404884in}}%
\pgfpathlineto{\pgfqpoint{3.852029in}{2.369108in}}%
\pgfpathlineto{\pgfqpoint{3.852521in}{2.409432in}}%
\pgfpathlineto{\pgfqpoint{3.852139in}{2.348855in}}%
\pgfpathlineto{\pgfqpoint{3.853149in}{2.383239in}}%
\pgfpathlineto{\pgfqpoint{3.853998in}{2.348633in}}%
\pgfpathlineto{\pgfqpoint{3.853235in}{2.406923in}}%
\pgfpathlineto{\pgfqpoint{3.854281in}{2.370587in}}%
\pgfpathlineto{\pgfqpoint{3.855081in}{2.402261in}}%
\pgfpathlineto{\pgfqpoint{3.854909in}{2.346923in}}%
\pgfpathlineto{\pgfqpoint{3.855364in}{2.349794in}}%
\pgfpathlineto{\pgfqpoint{3.855376in}{2.349452in}}%
\pgfpathlineto{\pgfqpoint{3.855426in}{2.376055in}}%
\pgfpathlineto{\pgfqpoint{3.856472in}{2.408570in}}%
\pgfpathlineto{\pgfqpoint{3.855856in}{2.350264in}}%
\pgfpathlineto{\pgfqpoint{3.856533in}{2.377255in}}%
\pgfpathlineto{\pgfqpoint{3.857222in}{2.351207in}}%
\pgfpathlineto{\pgfqpoint{3.857382in}{2.408510in}}%
\pgfpathlineto{\pgfqpoint{3.857604in}{2.385487in}}%
\pgfpathlineto{\pgfqpoint{3.857616in}{2.385600in}}%
\pgfpathlineto{\pgfqpoint{3.857690in}{2.350376in}}%
\pgfpathlineto{\pgfqpoint{3.858306in}{2.408293in}}%
\pgfpathlineto{\pgfqpoint{3.858724in}{2.375412in}}%
\pgfpathlineto{\pgfqpoint{3.859696in}{2.413905in}}%
\pgfpathlineto{\pgfqpoint{3.859044in}{2.353135in}}%
\pgfpathlineto{\pgfqpoint{3.859819in}{2.360224in}}%
\pgfpathlineto{\pgfqpoint{3.860164in}{2.414028in}}%
\pgfpathlineto{\pgfqpoint{3.859967in}{2.354670in}}%
\pgfpathlineto{\pgfqpoint{3.860915in}{2.361789in}}%
\pgfpathlineto{\pgfqpoint{3.861826in}{2.352297in}}%
\pgfpathlineto{\pgfqpoint{3.861087in}{2.412591in}}%
\pgfpathlineto{\pgfqpoint{3.861961in}{2.378105in}}%
\pgfpathlineto{\pgfqpoint{3.862946in}{2.410163in}}%
\pgfpathlineto{\pgfqpoint{3.862749in}{2.354173in}}%
\pgfpathlineto{\pgfqpoint{3.863044in}{2.371509in}}%
\pgfpathlineto{\pgfqpoint{3.864066in}{2.352529in}}%
\pgfpathlineto{\pgfqpoint{3.863413in}{2.409421in}}%
\pgfpathlineto{\pgfqpoint{3.864176in}{2.359367in}}%
\pgfpathlineto{\pgfqpoint{3.864336in}{2.409803in}}%
\pgfpathlineto{\pgfqpoint{3.865001in}{2.349899in}}%
\pgfpathlineto{\pgfqpoint{3.865370in}{2.366862in}}%
\pgfpathlineto{\pgfqpoint{3.865924in}{2.342734in}}%
\pgfpathlineto{\pgfqpoint{3.866195in}{2.409404in}}%
\pgfpathlineto{\pgfqpoint{3.866478in}{2.361027in}}%
\pgfpathlineto{\pgfqpoint{3.867586in}{2.411713in}}%
\pgfpathlineto{\pgfqpoint{3.867315in}{2.341619in}}%
\pgfpathlineto{\pgfqpoint{3.867635in}{2.368080in}}%
\pgfpathlineto{\pgfqpoint{3.868238in}{2.338705in}}%
\pgfpathlineto{\pgfqpoint{3.868053in}{2.411465in}}%
\pgfpathlineto{\pgfqpoint{3.868718in}{2.361126in}}%
\pgfpathlineto{\pgfqpoint{3.869444in}{2.415076in}}%
\pgfpathlineto{\pgfqpoint{3.869161in}{2.341100in}}%
\pgfpathlineto{\pgfqpoint{3.869838in}{2.384415in}}%
\pgfpathlineto{\pgfqpoint{3.870084in}{2.342975in}}%
\pgfpathlineto{\pgfqpoint{3.869912in}{2.416185in}}%
\pgfpathlineto{\pgfqpoint{3.870785in}{2.387147in}}%
\pgfpathlineto{\pgfqpoint{3.870835in}{2.416795in}}%
\pgfpathlineto{\pgfqpoint{3.871007in}{2.346783in}}%
\pgfpathlineto{\pgfqpoint{3.871856in}{2.362279in}}%
\pgfpathlineto{\pgfqpoint{3.871881in}{2.347545in}}%
\pgfpathlineto{\pgfqpoint{3.872225in}{2.410236in}}%
\pgfpathlineto{\pgfqpoint{3.872952in}{2.364943in}}%
\pgfpathlineto{\pgfqpoint{3.873616in}{2.405941in}}%
\pgfpathlineto{\pgfqpoint{3.873272in}{2.347637in}}%
\pgfpathlineto{\pgfqpoint{3.874096in}{2.398785in}}%
\pgfpathlineto{\pgfqpoint{3.874195in}{2.350957in}}%
\pgfpathlineto{\pgfqpoint{3.874552in}{2.399827in}}%
\pgfpathlineto{\pgfqpoint{3.875253in}{2.367813in}}%
\pgfpathlineto{\pgfqpoint{3.875499in}{2.400385in}}%
\pgfpathlineto{\pgfqpoint{3.875635in}{2.356914in}}%
\pgfpathlineto{\pgfqpoint{3.876361in}{2.371422in}}%
\pgfpathlineto{\pgfqpoint{3.877025in}{2.347261in}}%
\pgfpathlineto{\pgfqpoint{3.877149in}{2.398016in}}%
\pgfpathlineto{\pgfqpoint{3.877161in}{2.398247in}}%
\pgfpathlineto{\pgfqpoint{3.877222in}{2.378664in}}%
\pgfpathlineto{\pgfqpoint{3.877493in}{2.344557in}}%
\pgfpathlineto{\pgfqpoint{3.878133in}{2.403566in}}%
\pgfpathlineto{\pgfqpoint{3.878342in}{2.364769in}}%
\pgfpathlineto{\pgfqpoint{3.878601in}{2.404486in}}%
\pgfpathlineto{\pgfqpoint{3.878896in}{2.341173in}}%
\pgfpathlineto{\pgfqpoint{3.879315in}{2.358636in}}%
\pgfpathlineto{\pgfqpoint{3.879807in}{2.340279in}}%
\pgfpathlineto{\pgfqpoint{3.879536in}{2.414560in}}%
\pgfpathlineto{\pgfqpoint{3.880385in}{2.377114in}}%
\pgfpathlineto{\pgfqpoint{3.880878in}{2.404057in}}%
\pgfpathlineto{\pgfqpoint{3.880718in}{2.341440in}}%
\pgfpathlineto{\pgfqpoint{3.881481in}{2.374070in}}%
\pgfpathlineto{\pgfqpoint{3.882601in}{2.335490in}}%
\pgfpathlineto{\pgfqpoint{3.881887in}{2.415309in}}%
\pgfpathlineto{\pgfqpoint{3.882699in}{2.346003in}}%
\pgfpathlineto{\pgfqpoint{3.882761in}{2.414636in}}%
\pgfpathlineto{\pgfqpoint{3.883081in}{2.344749in}}%
\pgfpathlineto{\pgfqpoint{3.883832in}{2.383276in}}%
\pgfpathlineto{\pgfqpoint{3.884804in}{2.329014in}}%
\pgfpathlineto{\pgfqpoint{3.884607in}{2.428687in}}%
\pgfpathlineto{\pgfqpoint{3.884927in}{2.357161in}}%
\pgfpathlineto{\pgfqpoint{3.885653in}{2.436998in}}%
\pgfpathlineto{\pgfqpoint{3.885752in}{2.308768in}}%
\pgfpathlineto{\pgfqpoint{3.886047in}{2.406849in}}%
\pgfpathlineto{\pgfqpoint{3.886109in}{2.421056in}}%
\pgfpathlineto{\pgfqpoint{3.886232in}{2.362359in}}%
\pgfpathlineto{\pgfqpoint{3.886281in}{2.291446in}}%
\pgfpathlineto{\pgfqpoint{3.887056in}{2.430157in}}%
\pgfpathlineto{\pgfqpoint{3.887364in}{2.340266in}}%
\pgfpathlineto{\pgfqpoint{3.888287in}{2.441427in}}%
\pgfpathlineto{\pgfqpoint{3.888385in}{2.303448in}}%
\pgfpathlineto{\pgfqpoint{3.888459in}{2.341924in}}%
\pgfpathlineto{\pgfqpoint{3.889444in}{2.282942in}}%
\pgfpathlineto{\pgfqpoint{3.888804in}{2.454980in}}%
\pgfpathlineto{\pgfqpoint{3.889567in}{2.339407in}}%
\pgfpathlineto{\pgfqpoint{3.890207in}{2.449787in}}%
\pgfpathlineto{\pgfqpoint{3.890490in}{2.277069in}}%
\pgfpathlineto{\pgfqpoint{3.890687in}{2.369303in}}%
\pgfpathlineto{\pgfqpoint{3.890736in}{2.450020in}}%
\pgfpathlineto{\pgfqpoint{3.891019in}{2.279859in}}%
\pgfpathlineto{\pgfqpoint{3.891832in}{2.407856in}}%
\pgfpathlineto{\pgfqpoint{3.892078in}{2.286301in}}%
\pgfpathlineto{\pgfqpoint{3.892828in}{2.431421in}}%
\pgfpathlineto{\pgfqpoint{3.893887in}{2.459009in}}%
\pgfpathlineto{\pgfqpoint{3.893653in}{2.276390in}}%
\pgfpathlineto{\pgfqpoint{3.893936in}{2.437028in}}%
\pgfpathlineto{\pgfqpoint{3.894170in}{2.279898in}}%
\pgfpathlineto{\pgfqpoint{3.894416in}{2.455859in}}%
\pgfpathlineto{\pgfqpoint{3.895352in}{2.366650in}}%
\pgfpathlineto{\pgfqpoint{3.896004in}{2.438954in}}%
\pgfpathlineto{\pgfqpoint{3.895758in}{2.294034in}}%
\pgfpathlineto{\pgfqpoint{3.896459in}{2.385279in}}%
\pgfpathlineto{\pgfqpoint{3.897333in}{2.268729in}}%
\pgfpathlineto{\pgfqpoint{3.896521in}{2.458920in}}%
\pgfpathlineto{\pgfqpoint{3.897542in}{2.399094in}}%
\pgfpathlineto{\pgfqpoint{3.898108in}{2.456378in}}%
\pgfpathlineto{\pgfqpoint{3.897862in}{2.278225in}}%
\pgfpathlineto{\pgfqpoint{3.898662in}{2.429191in}}%
\pgfpathlineto{\pgfqpoint{3.898908in}{2.287658in}}%
\pgfpathlineto{\pgfqpoint{3.899684in}{2.434440in}}%
\pgfpathlineto{\pgfqpoint{3.899807in}{2.369709in}}%
\pgfpathlineto{\pgfqpoint{3.900213in}{2.447480in}}%
\pgfpathlineto{\pgfqpoint{3.900496in}{2.292291in}}%
\pgfpathlineto{\pgfqpoint{3.900927in}{2.398656in}}%
\pgfpathlineto{\pgfqpoint{3.901013in}{2.270345in}}%
\pgfpathlineto{\pgfqpoint{3.901259in}{2.458209in}}%
\pgfpathlineto{\pgfqpoint{3.902072in}{2.291368in}}%
\pgfpathlineto{\pgfqpoint{3.902305in}{2.464916in}}%
\pgfpathlineto{\pgfqpoint{3.902588in}{2.282843in}}%
\pgfpathlineto{\pgfqpoint{3.903204in}{2.326167in}}%
\pgfpathlineto{\pgfqpoint{3.903881in}{2.445929in}}%
\pgfpathlineto{\pgfqpoint{3.904164in}{2.289360in}}%
\pgfpathlineto{\pgfqpoint{3.904250in}{2.298349in}}%
\pgfpathlineto{\pgfqpoint{3.904275in}{2.308803in}}%
\pgfpathlineto{\pgfqpoint{3.905210in}{2.269912in}}%
\pgfpathlineto{\pgfqpoint{3.905456in}{2.463957in}}%
\pgfpathlineto{\pgfqpoint{3.905493in}{2.446375in}}%
\pgfpathlineto{\pgfqpoint{3.905739in}{2.267342in}}%
\pgfpathlineto{\pgfqpoint{3.905985in}{2.461695in}}%
\pgfpathlineto{\pgfqpoint{3.906650in}{2.385732in}}%
\pgfpathlineto{\pgfqpoint{3.907561in}{2.460635in}}%
\pgfpathlineto{\pgfqpoint{3.906785in}{2.276958in}}%
\pgfpathlineto{\pgfqpoint{3.907745in}{2.410483in}}%
\pgfpathlineto{\pgfqpoint{3.908890in}{2.258503in}}%
\pgfpathlineto{\pgfqpoint{3.908607in}{2.472120in}}%
\pgfpathlineto{\pgfqpoint{3.908902in}{2.269918in}}%
\pgfpathlineto{\pgfqpoint{3.909124in}{2.476789in}}%
\pgfpathlineto{\pgfqpoint{3.909407in}{2.259810in}}%
\pgfpathlineto{\pgfqpoint{3.910096in}{2.409032in}}%
\pgfpathlineto{\pgfqpoint{3.910453in}{2.275148in}}%
\pgfpathlineto{\pgfqpoint{3.910182in}{2.468232in}}%
\pgfpathlineto{\pgfqpoint{3.911191in}{2.383589in}}%
\pgfpathlineto{\pgfqpoint{3.912275in}{2.477346in}}%
\pgfpathlineto{\pgfqpoint{3.912028in}{2.265110in}}%
\pgfpathlineto{\pgfqpoint{3.912311in}{2.447792in}}%
\pgfpathlineto{\pgfqpoint{3.913075in}{2.248720in}}%
\pgfpathlineto{\pgfqpoint{3.913321in}{2.485965in}}%
\pgfpathlineto{\pgfqpoint{3.913407in}{2.452571in}}%
\pgfpathlineto{\pgfqpoint{3.913431in}{2.427958in}}%
\pgfpathlineto{\pgfqpoint{3.913604in}{2.246346in}}%
\pgfpathlineto{\pgfqpoint{3.913850in}{2.478649in}}%
\pgfpathlineto{\pgfqpoint{3.914539in}{2.410747in}}%
\pgfpathlineto{\pgfqpoint{3.914551in}{2.412297in}}%
\pgfpathlineto{\pgfqpoint{3.914625in}{2.299094in}}%
\pgfpathlineto{\pgfqpoint{3.914650in}{2.261593in}}%
\pgfpathlineto{\pgfqpoint{3.914896in}{2.474176in}}%
\pgfpathlineto{\pgfqpoint{3.915721in}{2.293419in}}%
\pgfpathlineto{\pgfqpoint{3.916471in}{2.481783in}}%
\pgfpathlineto{\pgfqpoint{3.916742in}{2.254252in}}%
\pgfpathlineto{\pgfqpoint{3.916816in}{2.302964in}}%
\pgfpathlineto{\pgfqpoint{3.917271in}{2.245164in}}%
\pgfpathlineto{\pgfqpoint{3.916988in}{2.489910in}}%
\pgfpathlineto{\pgfqpoint{3.917899in}{2.305547in}}%
\pgfpathlineto{\pgfqpoint{3.918035in}{2.486430in}}%
\pgfpathlineto{\pgfqpoint{3.918318in}{2.251074in}}%
\pgfpathlineto{\pgfqpoint{3.919019in}{2.378833in}}%
\pgfpathlineto{\pgfqpoint{3.919893in}{2.251695in}}%
\pgfpathlineto{\pgfqpoint{3.919610in}{2.486314in}}%
\pgfpathlineto{\pgfqpoint{3.920102in}{2.390787in}}%
\pgfpathlineto{\pgfqpoint{3.921185in}{2.505096in}}%
\pgfpathlineto{\pgfqpoint{3.920939in}{2.240601in}}%
\pgfpathlineto{\pgfqpoint{3.921222in}{2.437955in}}%
\pgfpathlineto{\pgfqpoint{3.921271in}{2.474074in}}%
\pgfpathlineto{\pgfqpoint{3.921370in}{2.385326in}}%
\pgfpathlineto{\pgfqpoint{3.921456in}{2.235324in}}%
\pgfpathlineto{\pgfqpoint{3.922231in}{2.509845in}}%
\pgfpathlineto{\pgfqpoint{3.922515in}{2.251155in}}%
\pgfpathlineto{\pgfqpoint{3.922761in}{2.506614in}}%
\pgfpathlineto{\pgfqpoint{3.923548in}{2.232396in}}%
\pgfpathlineto{\pgfqpoint{3.923622in}{2.297665in}}%
\pgfpathlineto{\pgfqpoint{3.924595in}{2.218725in}}%
\pgfpathlineto{\pgfqpoint{3.924324in}{2.509649in}}%
\pgfpathlineto{\pgfqpoint{3.924718in}{2.322724in}}%
\pgfpathlineto{\pgfqpoint{3.925370in}{2.520900in}}%
\pgfpathlineto{\pgfqpoint{3.925641in}{2.209159in}}%
\pgfpathlineto{\pgfqpoint{3.925838in}{2.369102in}}%
\pgfpathlineto{\pgfqpoint{3.925899in}{2.524582in}}%
\pgfpathlineto{\pgfqpoint{3.926170in}{2.213110in}}%
\pgfpathlineto{\pgfqpoint{3.927031in}{2.467555in}}%
\pgfpathlineto{\pgfqpoint{3.928262in}{2.202435in}}%
\pgfpathlineto{\pgfqpoint{3.927991in}{2.516025in}}%
\pgfpathlineto{\pgfqpoint{3.928275in}{2.218514in}}%
\pgfpathlineto{\pgfqpoint{3.929038in}{2.539520in}}%
\pgfpathlineto{\pgfqpoint{3.929308in}{2.188236in}}%
\pgfpathlineto{\pgfqpoint{3.929395in}{2.263147in}}%
\pgfpathlineto{\pgfqpoint{3.929468in}{2.451946in}}%
\pgfpathlineto{\pgfqpoint{3.929530in}{2.442066in}}%
\pgfpathlineto{\pgfqpoint{3.929567in}{2.539626in}}%
\pgfpathlineto{\pgfqpoint{3.929838in}{2.186048in}}%
\pgfpathlineto{\pgfqpoint{3.930638in}{2.473725in}}%
\pgfpathlineto{\pgfqpoint{3.931413in}{2.203310in}}%
\pgfpathlineto{\pgfqpoint{3.931659in}{2.527480in}}%
\pgfpathlineto{\pgfqpoint{3.931733in}{2.492936in}}%
\pgfpathlineto{\pgfqpoint{3.931745in}{2.494517in}}%
\pgfpathlineto{\pgfqpoint{3.931844in}{2.396460in}}%
\pgfpathlineto{\pgfqpoint{3.932976in}{2.173250in}}%
\pgfpathlineto{\pgfqpoint{3.932705in}{2.557137in}}%
\pgfpathlineto{\pgfqpoint{3.933001in}{2.214667in}}%
\pgfpathlineto{\pgfqpoint{3.933234in}{2.558610in}}%
\pgfpathlineto{\pgfqpoint{3.933505in}{2.171784in}}%
\pgfpathlineto{\pgfqpoint{3.934121in}{2.275298in}}%
\pgfpathlineto{\pgfqpoint{3.934281in}{2.541639in}}%
\pgfpathlineto{\pgfqpoint{3.935081in}{2.185260in}}%
\pgfpathlineto{\pgfqpoint{3.935364in}{2.459688in}}%
\pgfpathlineto{\pgfqpoint{3.936127in}{2.164581in}}%
\pgfpathlineto{\pgfqpoint{3.935844in}{2.547505in}}%
\pgfpathlineto{\pgfqpoint{3.936348in}{2.503131in}}%
\pgfpathlineto{\pgfqpoint{3.937419in}{2.583330in}}%
\pgfpathlineto{\pgfqpoint{3.937173in}{2.152143in}}%
\pgfpathlineto{\pgfqpoint{3.937444in}{2.513413in}}%
\pgfpathlineto{\pgfqpoint{3.937690in}{2.159085in}}%
\pgfpathlineto{\pgfqpoint{3.937936in}{2.568347in}}%
\pgfpathlineto{\pgfqpoint{3.938601in}{2.378874in}}%
\pgfpathlineto{\pgfqpoint{3.939511in}{2.569033in}}%
\pgfpathlineto{\pgfqpoint{3.939265in}{2.163050in}}%
\pgfpathlineto{\pgfqpoint{3.939696in}{2.392311in}}%
\pgfpathlineto{\pgfqpoint{3.940828in}{2.135152in}}%
\pgfpathlineto{\pgfqpoint{3.940558in}{2.604974in}}%
\pgfpathlineto{\pgfqpoint{3.940853in}{2.190118in}}%
\pgfpathlineto{\pgfqpoint{3.941087in}{2.605163in}}%
\pgfpathlineto{\pgfqpoint{3.941358in}{2.136157in}}%
\pgfpathlineto{\pgfqpoint{3.941973in}{2.249026in}}%
\pgfpathlineto{\pgfqpoint{3.943179in}{2.595299in}}%
\pgfpathlineto{\pgfqpoint{3.942921in}{2.153896in}}%
\pgfpathlineto{\pgfqpoint{3.943204in}{2.510183in}}%
\pgfpathlineto{\pgfqpoint{3.943967in}{2.133291in}}%
\pgfpathlineto{\pgfqpoint{3.944225in}{2.621805in}}%
\pgfpathlineto{\pgfqpoint{3.944299in}{2.548147in}}%
\pgfpathlineto{\pgfqpoint{3.944496in}{2.124224in}}%
\pgfpathlineto{\pgfqpoint{3.945271in}{2.625667in}}%
\pgfpathlineto{\pgfqpoint{3.945665in}{2.345997in}}%
\pgfpathlineto{\pgfqpoint{3.946318in}{2.619687in}}%
\pgfpathlineto{\pgfqpoint{3.946588in}{2.134174in}}%
\pgfpathlineto{\pgfqpoint{3.946773in}{2.373625in}}%
\pgfpathlineto{\pgfqpoint{3.947634in}{2.116638in}}%
\pgfpathlineto{\pgfqpoint{3.947364in}{2.620853in}}%
\pgfpathlineto{\pgfqpoint{3.947844in}{2.385232in}}%
\pgfpathlineto{\pgfqpoint{3.948410in}{2.645544in}}%
\pgfpathlineto{\pgfqpoint{3.948681in}{2.106403in}}%
\pgfpathlineto{\pgfqpoint{3.948964in}{2.527218in}}%
\pgfpathlineto{\pgfqpoint{3.949198in}{2.109032in}}%
\pgfpathlineto{\pgfqpoint{3.949456in}{2.632050in}}%
\pgfpathlineto{\pgfqpoint{3.950121in}{2.415215in}}%
\pgfpathlineto{\pgfqpoint{3.950133in}{2.415968in}}%
\pgfpathlineto{\pgfqpoint{3.950145in}{2.394647in}}%
\pgfpathlineto{\pgfqpoint{3.951290in}{2.097335in}}%
\pgfpathlineto{\pgfqpoint{3.951019in}{2.623369in}}%
\pgfpathlineto{\pgfqpoint{3.951302in}{2.118881in}}%
\pgfpathlineto{\pgfqpoint{3.952065in}{2.643988in}}%
\pgfpathlineto{\pgfqpoint{3.952336in}{2.090246in}}%
\pgfpathlineto{\pgfqpoint{3.952422in}{2.212143in}}%
\pgfpathlineto{\pgfqpoint{3.952853in}{2.090101in}}%
\pgfpathlineto{\pgfqpoint{3.952582in}{2.643403in}}%
\pgfpathlineto{\pgfqpoint{3.953001in}{2.455985in}}%
\pgfpathlineto{\pgfqpoint{3.953111in}{2.636624in}}%
\pgfpathlineto{\pgfqpoint{3.953382in}{2.092094in}}%
\pgfpathlineto{\pgfqpoint{3.954071in}{2.438898in}}%
\pgfpathlineto{\pgfqpoint{3.954945in}{2.085683in}}%
\pgfpathlineto{\pgfqpoint{3.954674in}{2.635120in}}%
\pgfpathlineto{\pgfqpoint{3.955167in}{2.484234in}}%
\pgfpathlineto{\pgfqpoint{3.956237in}{2.638500in}}%
\pgfpathlineto{\pgfqpoint{3.955991in}{2.075793in}}%
\pgfpathlineto{\pgfqpoint{3.956274in}{2.547851in}}%
\pgfpathlineto{\pgfqpoint{3.956508in}{2.069707in}}%
\pgfpathlineto{\pgfqpoint{3.956754in}{2.635793in}}%
\pgfpathlineto{\pgfqpoint{3.957271in}{2.612247in}}%
\pgfpathlineto{\pgfqpoint{3.957284in}{2.623944in}}%
\pgfpathlineto{\pgfqpoint{3.957554in}{2.067476in}}%
\pgfpathlineto{\pgfqpoint{3.958010in}{2.276139in}}%
\pgfpathlineto{\pgfqpoint{3.959117in}{2.072133in}}%
\pgfpathlineto{\pgfqpoint{3.958317in}{2.612265in}}%
\pgfpathlineto{\pgfqpoint{3.959154in}{2.208036in}}%
\pgfpathlineto{\pgfqpoint{3.959364in}{2.603979in}}%
\pgfpathlineto{\pgfqpoint{3.960164in}{2.060373in}}%
\pgfpathlineto{\pgfqpoint{3.960324in}{2.435088in}}%
\pgfpathlineto{\pgfqpoint{3.961210in}{2.050798in}}%
\pgfpathlineto{\pgfqpoint{3.960410in}{2.603097in}}%
\pgfpathlineto{\pgfqpoint{3.961407in}{2.429897in}}%
\pgfpathlineto{\pgfqpoint{3.962514in}{2.603129in}}%
\pgfpathlineto{\pgfqpoint{3.961727in}{2.055207in}}%
\pgfpathlineto{\pgfqpoint{3.962551in}{2.578824in}}%
\pgfpathlineto{\pgfqpoint{3.963290in}{2.060529in}}%
\pgfpathlineto{\pgfqpoint{3.963561in}{2.616064in}}%
\pgfpathlineto{\pgfqpoint{3.963942in}{2.310732in}}%
\pgfpathlineto{\pgfqpoint{3.965124in}{2.629582in}}%
\pgfpathlineto{\pgfqpoint{3.964853in}{2.060902in}}%
\pgfpathlineto{\pgfqpoint{3.965148in}{2.585172in}}%
\pgfpathlineto{\pgfqpoint{3.965259in}{2.463576in}}%
\pgfpathlineto{\pgfqpoint{3.965899in}{2.060941in}}%
\pgfpathlineto{\pgfqpoint{3.965641in}{2.626861in}}%
\pgfpathlineto{\pgfqpoint{3.966477in}{2.181303in}}%
\pgfpathlineto{\pgfqpoint{3.966490in}{2.181293in}}%
\pgfpathlineto{\pgfqpoint{3.966687in}{2.621198in}}%
\pgfpathlineto{\pgfqpoint{3.966945in}{2.062552in}}%
\pgfpathlineto{\pgfqpoint{3.967462in}{2.064156in}}%
\pgfpathlineto{\pgfqpoint{3.967474in}{2.063886in}}%
\pgfpathlineto{\pgfqpoint{3.967991in}{2.052062in}}%
\pgfpathlineto{\pgfqpoint{3.968779in}{2.634378in}}%
\pgfpathlineto{\pgfqpoint{3.969554in}{2.048148in}}%
\pgfpathlineto{\pgfqpoint{3.969813in}{2.637441in}}%
\pgfpathlineto{\pgfqpoint{3.969936in}{2.517128in}}%
\pgfpathlineto{\pgfqpoint{3.971117in}{2.050712in}}%
\pgfpathlineto{\pgfqpoint{3.970859in}{2.637459in}}%
\pgfpathlineto{\pgfqpoint{3.971228in}{2.226468in}}%
\pgfpathlineto{\pgfqpoint{3.972422in}{2.642770in}}%
\pgfpathlineto{\pgfqpoint{3.972164in}{2.039245in}}%
\pgfpathlineto{\pgfqpoint{3.972434in}{2.630499in}}%
\pgfpathlineto{\pgfqpoint{3.973210in}{2.030174in}}%
\pgfpathlineto{\pgfqpoint{3.973468in}{2.645422in}}%
\pgfpathlineto{\pgfqpoint{3.973542in}{2.624181in}}%
\pgfpathlineto{\pgfqpoint{3.973727in}{2.034332in}}%
\pgfpathlineto{\pgfqpoint{3.974514in}{2.647972in}}%
\pgfpathlineto{\pgfqpoint{3.974982in}{2.425688in}}%
\pgfpathlineto{\pgfqpoint{3.976077in}{2.652860in}}%
\pgfpathlineto{\pgfqpoint{3.975819in}{2.032793in}}%
\pgfpathlineto{\pgfqpoint{3.976114in}{2.557469in}}%
\pgfpathlineto{\pgfqpoint{3.976594in}{2.656452in}}%
\pgfpathlineto{\pgfqpoint{3.976853in}{2.027752in}}%
\pgfpathlineto{\pgfqpoint{3.977222in}{2.609039in}}%
\pgfpathlineto{\pgfqpoint{3.977899in}{2.025797in}}%
\pgfpathlineto{\pgfqpoint{3.978157in}{2.653459in}}%
\pgfpathlineto{\pgfqpoint{3.978490in}{2.109977in}}%
\pgfpathlineto{\pgfqpoint{3.978637in}{2.480541in}}%
\pgfpathlineto{\pgfqpoint{3.978674in}{2.654098in}}%
\pgfpathlineto{\pgfqpoint{3.979462in}{2.038053in}}%
\pgfpathlineto{\pgfqpoint{3.979757in}{2.563111in}}%
\pgfpathlineto{\pgfqpoint{3.980237in}{2.660615in}}%
\pgfpathlineto{\pgfqpoint{3.979979in}{2.035306in}}%
\pgfpathlineto{\pgfqpoint{3.980484in}{2.075331in}}%
\pgfpathlineto{\pgfqpoint{3.981542in}{2.034481in}}%
\pgfpathlineto{\pgfqpoint{3.980754in}{2.661333in}}%
\pgfpathlineto{\pgfqpoint{3.981579in}{2.105292in}}%
\pgfpathlineto{\pgfqpoint{3.981616in}{2.083462in}}%
\pgfpathlineto{\pgfqpoint{3.981665in}{2.233097in}}%
\pgfpathlineto{\pgfqpoint{3.982317in}{2.661875in}}%
\pgfpathlineto{\pgfqpoint{3.982059in}{2.042889in}}%
\pgfpathlineto{\pgfqpoint{3.982847in}{2.654167in}}%
\pgfpathlineto{\pgfqpoint{3.983105in}{2.043704in}}%
\pgfpathlineto{\pgfqpoint{3.983880in}{2.665181in}}%
\pgfpathlineto{\pgfqpoint{3.983954in}{2.623514in}}%
\pgfpathlineto{\pgfqpoint{3.984397in}{2.666453in}}%
\pgfpathlineto{\pgfqpoint{3.984151in}{2.057467in}}%
\pgfpathlineto{\pgfqpoint{3.984668in}{2.059986in}}%
\pgfpathlineto{\pgfqpoint{3.985444in}{2.673697in}}%
\pgfpathlineto{\pgfqpoint{3.985259in}{2.050969in}}%
\pgfpathlineto{\pgfqpoint{3.985764in}{2.072950in}}%
\pgfpathlineto{\pgfqpoint{3.985776in}{2.053001in}}%
\pgfpathlineto{\pgfqpoint{3.985960in}{2.670781in}}%
\pgfpathlineto{\pgfqpoint{3.986539in}{2.587202in}}%
\pgfpathlineto{\pgfqpoint{3.987007in}{2.672571in}}%
\pgfpathlineto{\pgfqpoint{3.987339in}{2.044130in}}%
\pgfpathlineto{\pgfqpoint{3.987659in}{2.610759in}}%
\pgfpathlineto{\pgfqpoint{3.988902in}{2.030050in}}%
\pgfpathlineto{\pgfqpoint{3.988040in}{2.663653in}}%
\pgfpathlineto{\pgfqpoint{3.988927in}{2.080991in}}%
\pgfpathlineto{\pgfqpoint{3.989087in}{2.664450in}}%
\pgfpathlineto{\pgfqpoint{3.989936in}{2.028483in}}%
\pgfpathlineto{\pgfqpoint{3.990157in}{2.582943in}}%
\pgfpathlineto{\pgfqpoint{3.990982in}{2.029520in}}%
\pgfpathlineto{\pgfqpoint{3.990231in}{2.663559in}}%
\pgfpathlineto{\pgfqpoint{3.991142in}{2.589956in}}%
\pgfpathlineto{\pgfqpoint{3.991782in}{2.666188in}}%
\pgfpathlineto{\pgfqpoint{3.991499in}{2.030399in}}%
\pgfpathlineto{\pgfqpoint{3.992250in}{2.574444in}}%
\pgfpathlineto{\pgfqpoint{3.992299in}{2.665153in}}%
\pgfpathlineto{\pgfqpoint{3.993062in}{2.030374in}}%
\pgfpathlineto{\pgfqpoint{3.993382in}{2.625077in}}%
\pgfpathlineto{\pgfqpoint{3.993579in}{2.028050in}}%
\pgfpathlineto{\pgfqpoint{3.994379in}{2.660613in}}%
\pgfpathlineto{\pgfqpoint{3.994662in}{2.155205in}}%
\pgfpathlineto{\pgfqpoint{3.994896in}{2.656284in}}%
\pgfpathlineto{\pgfqpoint{3.995659in}{2.037469in}}%
\pgfpathlineto{\pgfqpoint{3.995868in}{2.559448in}}%
\pgfpathlineto{\pgfqpoint{3.995942in}{2.653643in}}%
\pgfpathlineto{\pgfqpoint{3.996040in}{2.394459in}}%
\pgfpathlineto{\pgfqpoint{3.996176in}{2.034249in}}%
\pgfpathlineto{\pgfqpoint{3.996976in}{2.660283in}}%
\pgfpathlineto{\pgfqpoint{3.997210in}{2.041273in}}%
\pgfpathlineto{\pgfqpoint{3.997739in}{2.028251in}}%
\pgfpathlineto{\pgfqpoint{3.997493in}{2.653734in}}%
\pgfpathlineto{\pgfqpoint{3.997862in}{2.342241in}}%
\pgfpathlineto{\pgfqpoint{3.998010in}{2.650206in}}%
\pgfpathlineto{\pgfqpoint{3.998256in}{2.029986in}}%
\pgfpathlineto{\pgfqpoint{3.999007in}{2.569507in}}%
\pgfpathlineto{\pgfqpoint{3.999056in}{2.650114in}}%
\pgfpathlineto{\pgfqpoint{3.999302in}{2.034435in}}%
\pgfpathlineto{\pgfqpoint{4.000127in}{2.629414in}}%
\pgfpathlineto{\pgfqpoint{4.000336in}{2.036542in}}%
\pgfpathlineto{\pgfqpoint{4.000607in}{2.647043in}}%
\pgfpathlineto{\pgfqpoint{4.001493in}{2.292933in}}%
\pgfpathlineto{\pgfqpoint{4.002170in}{2.649350in}}%
\pgfpathlineto{\pgfqpoint{4.001899in}{2.043536in}}%
\pgfpathlineto{\pgfqpoint{4.002674in}{2.624009in}}%
\pgfpathlineto{\pgfqpoint{4.002699in}{2.642732in}}%
\pgfpathlineto{\pgfqpoint{4.002945in}{2.060718in}}%
\pgfpathlineto{\pgfqpoint{4.003425in}{2.133086in}}%
\pgfpathlineto{\pgfqpoint{4.003462in}{2.064454in}}%
\pgfpathlineto{\pgfqpoint{4.003745in}{2.635429in}}%
\pgfpathlineto{\pgfqpoint{4.004520in}{2.112149in}}%
\pgfpathlineto{\pgfqpoint{4.004767in}{2.626496in}}%
\pgfpathlineto{\pgfqpoint{4.005542in}{2.067898in}}%
\pgfpathlineto{\pgfqpoint{4.005776in}{2.585344in}}%
\pgfpathlineto{\pgfqpoint{4.006330in}{2.632123in}}%
\pgfpathlineto{\pgfqpoint{4.006059in}{2.073878in}}%
\pgfpathlineto{\pgfqpoint{4.006883in}{2.597903in}}%
\pgfpathlineto{\pgfqpoint{4.008139in}{2.066987in}}%
\pgfpathlineto{\pgfqpoint{4.008287in}{2.439009in}}%
\pgfpathlineto{\pgfqpoint{4.008939in}{2.642713in}}%
\pgfpathlineto{\pgfqpoint{4.008656in}{2.088624in}}%
\pgfpathlineto{\pgfqpoint{4.009419in}{2.565065in}}%
\pgfpathlineto{\pgfqpoint{4.009468in}{2.621192in}}%
\pgfpathlineto{\pgfqpoint{4.010219in}{2.103977in}}%
\pgfpathlineto{\pgfqpoint{4.010527in}{2.570198in}}%
\pgfpathlineto{\pgfqpoint{4.010563in}{2.574257in}}%
\pgfpathlineto{\pgfqpoint{4.010600in}{2.476987in}}%
\pgfpathlineto{\pgfqpoint{4.010748in}{2.098885in}}%
\pgfpathlineto{\pgfqpoint{4.011031in}{2.626749in}}%
\pgfpathlineto{\pgfqpoint{4.011782in}{2.129436in}}%
\pgfpathlineto{\pgfqpoint{4.013099in}{2.602303in}}%
\pgfpathlineto{\pgfqpoint{4.012816in}{2.118122in}}%
\pgfpathlineto{\pgfqpoint{4.013222in}{2.367592in}}%
\pgfpathlineto{\pgfqpoint{4.013333in}{2.169539in}}%
\pgfpathlineto{\pgfqpoint{4.013603in}{2.612879in}}%
\pgfpathlineto{\pgfqpoint{4.014366in}{2.199638in}}%
\pgfpathlineto{\pgfqpoint{4.015400in}{2.155987in}}%
\pgfpathlineto{\pgfqpoint{4.015228in}{2.561563in}}%
\pgfpathlineto{\pgfqpoint{4.015437in}{2.247636in}}%
\pgfpathlineto{\pgfqpoint{4.015708in}{2.569491in}}%
\pgfpathlineto{\pgfqpoint{4.016508in}{2.184815in}}%
\pgfpathlineto{\pgfqpoint{4.017813in}{2.551443in}}%
\pgfpathlineto{\pgfqpoint{4.017850in}{2.479023in}}%
\pgfpathlineto{\pgfqpoint{4.017973in}{2.238184in}}%
\pgfpathlineto{\pgfqpoint{4.018317in}{2.522912in}}%
\pgfpathlineto{\pgfqpoint{4.019019in}{2.263770in}}%
\pgfpathlineto{\pgfqpoint{4.019043in}{2.233428in}}%
\pgfpathlineto{\pgfqpoint{4.019893in}{2.491063in}}%
\pgfpathlineto{\pgfqpoint{4.020102in}{2.282861in}}%
\pgfpathlineto{\pgfqpoint{4.020397in}{2.485772in}}%
\pgfpathlineto{\pgfqpoint{4.021185in}{2.282500in}}%
\pgfpathlineto{\pgfqpoint{4.021271in}{2.347381in}}%
\pgfpathlineto{\pgfqpoint{4.021456in}{2.494595in}}%
\pgfpathlineto{\pgfqpoint{4.021628in}{2.282570in}}%
\pgfpathlineto{\pgfqpoint{4.022502in}{2.465506in}}%
\pgfpathlineto{\pgfqpoint{4.022662in}{2.310543in}}%
\pgfpathlineto{\pgfqpoint{4.023733in}{2.329247in}}%
\pgfpathlineto{\pgfqpoint{4.024582in}{2.446597in}}%
\pgfpathlineto{\pgfqpoint{4.024853in}{2.342951in}}%
\pgfpathlineto{\pgfqpoint{4.024939in}{2.323815in}}%
\pgfpathlineto{\pgfqpoint{4.025628in}{2.414115in}}%
\pgfpathlineto{\pgfqpoint{4.025973in}{2.339979in}}%
\pgfpathlineto{\pgfqpoint{4.026010in}{2.336454in}}%
\pgfpathlineto{\pgfqpoint{4.026083in}{2.383700in}}%
\pgfpathlineto{\pgfqpoint{4.026133in}{2.378755in}}%
\pgfpathlineto{\pgfqpoint{4.027056in}{2.319464in}}%
\pgfpathlineto{\pgfqpoint{4.027314in}{2.421452in}}%
\pgfpathlineto{\pgfqpoint{4.028090in}{2.320439in}}%
\pgfpathlineto{\pgfqpoint{4.028656in}{2.366611in}}%
\pgfpathlineto{\pgfqpoint{4.029357in}{2.433022in}}%
\pgfpathlineto{\pgfqpoint{4.029099in}{2.315019in}}%
\pgfpathlineto{\pgfqpoint{4.029751in}{2.349293in}}%
\pgfpathlineto{\pgfqpoint{4.030453in}{2.427592in}}%
\pgfpathlineto{\pgfqpoint{4.030170in}{2.332391in}}%
\pgfpathlineto{\pgfqpoint{4.030970in}{2.394578in}}%
\pgfpathlineto{\pgfqpoint{4.031708in}{2.281304in}}%
\pgfpathlineto{\pgfqpoint{4.031499in}{2.486070in}}%
\pgfpathlineto{\pgfqpoint{4.032040in}{2.429262in}}%
\pgfpathlineto{\pgfqpoint{4.032926in}{2.472228in}}%
\pgfpathlineto{\pgfqpoint{4.032742in}{2.286880in}}%
\pgfpathlineto{\pgfqpoint{4.033111in}{2.415400in}}%
\pgfpathlineto{\pgfqpoint{4.033271in}{2.210351in}}%
\pgfpathlineto{\pgfqpoint{4.033591in}{2.515063in}}%
\pgfpathlineto{\pgfqpoint{4.034293in}{2.337381in}}%
\pgfpathlineto{\pgfqpoint{4.034748in}{2.255587in}}%
\pgfpathlineto{\pgfqpoint{4.034526in}{2.437688in}}%
\pgfpathlineto{\pgfqpoint{4.034969in}{2.397521in}}%
\pgfpathlineto{\pgfqpoint{4.035622in}{2.497211in}}%
\pgfpathlineto{\pgfqpoint{4.035413in}{2.269634in}}%
\pgfpathlineto{\pgfqpoint{4.036053in}{2.369494in}}%
\pgfpathlineto{\pgfqpoint{4.036853in}{2.209692in}}%
\pgfpathlineto{\pgfqpoint{4.036508in}{2.470425in}}%
\pgfpathlineto{\pgfqpoint{4.037099in}{2.458248in}}%
\pgfpathlineto{\pgfqpoint{4.037148in}{2.513456in}}%
\pgfpathlineto{\pgfqpoint{4.037382in}{2.230034in}}%
\pgfpathlineto{\pgfqpoint{4.037936in}{2.356362in}}%
\pgfpathlineto{\pgfqpoint{4.038908in}{2.217220in}}%
\pgfpathlineto{\pgfqpoint{4.038600in}{2.486535in}}%
\pgfpathlineto{\pgfqpoint{4.039019in}{2.340790in}}%
\pgfpathlineto{\pgfqpoint{4.039831in}{2.553811in}}%
\pgfpathlineto{\pgfqpoint{4.039523in}{2.256151in}}%
\pgfpathlineto{\pgfqpoint{4.040114in}{2.297594in}}%
\pgfpathlineto{\pgfqpoint{4.040631in}{2.462747in}}%
\pgfpathlineto{\pgfqpoint{4.041049in}{2.227960in}}%
\pgfpathlineto{\pgfqpoint{4.041579in}{2.180069in}}%
\pgfpathlineto{\pgfqpoint{4.041222in}{2.490681in}}%
\pgfpathlineto{\pgfqpoint{4.041825in}{2.441499in}}%
\pgfpathlineto{\pgfqpoint{4.041911in}{2.536490in}}%
\pgfpathlineto{\pgfqpoint{4.042329in}{2.290151in}}%
\pgfpathlineto{\pgfqpoint{4.042859in}{2.387730in}}%
\pgfpathlineto{\pgfqpoint{4.043696in}{2.229518in}}%
\pgfpathlineto{\pgfqpoint{4.043376in}{2.481953in}}%
\pgfpathlineto{\pgfqpoint{4.043929in}{2.450706in}}%
\pgfpathlineto{\pgfqpoint{4.044016in}{2.514715in}}%
\pgfpathlineto{\pgfqpoint{4.044311in}{2.279123in}}%
\pgfpathlineto{\pgfqpoint{4.044988in}{2.367067in}}%
\pgfpathlineto{\pgfqpoint{4.045209in}{2.286906in}}%
\pgfpathlineto{\pgfqpoint{4.045320in}{2.420747in}}%
\pgfpathlineto{\pgfqpoint{4.046108in}{2.514047in}}%
\pgfpathlineto{\pgfqpoint{4.045763in}{2.217037in}}%
\pgfpathlineto{\pgfqpoint{4.046366in}{2.345838in}}%
\pgfpathlineto{\pgfqpoint{4.046465in}{2.284067in}}%
\pgfpathlineto{\pgfqpoint{4.046637in}{2.463319in}}%
\pgfpathlineto{\pgfqpoint{4.047400in}{2.411181in}}%
\pgfpathlineto{\pgfqpoint{4.048176in}{2.503298in}}%
\pgfpathlineto{\pgfqpoint{4.047843in}{2.247663in}}%
\pgfpathlineto{\pgfqpoint{4.048373in}{2.265111in}}%
\pgfpathlineto{\pgfqpoint{4.048385in}{2.256507in}}%
\pgfpathlineto{\pgfqpoint{4.048693in}{2.506656in}}%
\pgfpathlineto{\pgfqpoint{4.049246in}{2.407825in}}%
\pgfpathlineto{\pgfqpoint{4.050256in}{2.467709in}}%
\pgfpathlineto{\pgfqpoint{4.049911in}{2.263340in}}%
\pgfpathlineto{\pgfqpoint{4.050329in}{2.396130in}}%
\pgfpathlineto{\pgfqpoint{4.050416in}{2.254813in}}%
\pgfpathlineto{\pgfqpoint{4.050785in}{2.511660in}}%
\pgfpathlineto{\pgfqpoint{4.051449in}{2.372926in}}%
\pgfpathlineto{\pgfqpoint{4.052569in}{2.250226in}}%
\pgfpathlineto{\pgfqpoint{4.052336in}{2.451839in}}%
\pgfpathlineto{\pgfqpoint{4.052619in}{2.324564in}}%
\pgfpathlineto{\pgfqpoint{4.052877in}{2.530060in}}%
\pgfpathlineto{\pgfqpoint{4.053209in}{2.254410in}}%
\pgfpathlineto{\pgfqpoint{4.053788in}{2.374745in}}%
\pgfpathlineto{\pgfqpoint{4.054588in}{2.221816in}}%
\pgfpathlineto{\pgfqpoint{4.054440in}{2.462067in}}%
\pgfpathlineto{\pgfqpoint{4.054809in}{2.446069in}}%
\pgfpathlineto{\pgfqpoint{4.054969in}{2.513743in}}%
\pgfpathlineto{\pgfqpoint{4.055179in}{2.272015in}}%
\pgfpathlineto{\pgfqpoint{4.055757in}{2.355761in}}%
\pgfpathlineto{\pgfqpoint{4.056717in}{2.265104in}}%
\pgfpathlineto{\pgfqpoint{4.056212in}{2.419442in}}%
\pgfpathlineto{\pgfqpoint{4.056803in}{2.392351in}}%
\pgfpathlineto{\pgfqpoint{4.057591in}{2.527717in}}%
\pgfpathlineto{\pgfqpoint{4.057443in}{2.287620in}}%
\pgfpathlineto{\pgfqpoint{4.057812in}{2.365911in}}%
\pgfpathlineto{\pgfqpoint{4.057923in}{2.301334in}}%
\pgfpathlineto{\pgfqpoint{4.058083in}{2.420676in}}%
\pgfpathlineto{\pgfqpoint{4.058354in}{2.415724in}}%
\pgfpathlineto{\pgfqpoint{4.059117in}{2.489473in}}%
\pgfpathlineto{\pgfqpoint{4.058834in}{2.243301in}}%
\pgfpathlineto{\pgfqpoint{4.059252in}{2.348499in}}%
\pgfpathlineto{\pgfqpoint{4.059289in}{2.220549in}}%
\pgfpathlineto{\pgfqpoint{4.059683in}{2.481047in}}%
\pgfpathlineto{\pgfqpoint{4.060348in}{2.373098in}}%
\pgfpathlineto{\pgfqpoint{4.060889in}{2.274777in}}%
\pgfpathlineto{\pgfqpoint{4.061136in}{2.462735in}}%
\pgfpathlineto{\pgfqpoint{4.061726in}{2.498172in}}%
\pgfpathlineto{\pgfqpoint{4.061505in}{2.264989in}}%
\pgfpathlineto{\pgfqpoint{4.061874in}{2.379566in}}%
\pgfpathlineto{\pgfqpoint{4.063006in}{2.261440in}}%
\pgfpathlineto{\pgfqpoint{4.062797in}{2.439869in}}%
\pgfpathlineto{\pgfqpoint{4.063031in}{2.298728in}}%
\pgfpathlineto{\pgfqpoint{4.063782in}{2.470830in}}%
\pgfpathlineto{\pgfqpoint{4.063474in}{2.237287in}}%
\pgfpathlineto{\pgfqpoint{4.064151in}{2.359101in}}%
\pgfpathlineto{\pgfqpoint{4.064754in}{2.430974in}}%
\pgfpathlineto{\pgfqpoint{4.065000in}{2.302544in}}%
\pgfpathlineto{\pgfqpoint{4.065271in}{2.384125in}}%
\pgfpathlineto{\pgfqpoint{4.065652in}{2.304775in}}%
\pgfpathlineto{\pgfqpoint{4.065923in}{2.490069in}}%
\pgfpathlineto{\pgfqpoint{4.066391in}{2.369112in}}%
\pgfpathlineto{\pgfqpoint{4.067302in}{2.467745in}}%
\pgfpathlineto{\pgfqpoint{4.067179in}{2.275850in}}%
\pgfpathlineto{\pgfqpoint{4.067511in}{2.406605in}}%
\pgfpathlineto{\pgfqpoint{4.067659in}{2.285218in}}%
\pgfpathlineto{\pgfqpoint{4.068016in}{2.452142in}}%
\pgfpathlineto{\pgfqpoint{4.068680in}{2.382451in}}%
\pgfpathlineto{\pgfqpoint{4.069542in}{2.445463in}}%
\pgfpathlineto{\pgfqpoint{4.069246in}{2.297587in}}%
\pgfpathlineto{\pgfqpoint{4.069751in}{2.355467in}}%
\pgfpathlineto{\pgfqpoint{4.069800in}{2.314167in}}%
\pgfpathlineto{\pgfqpoint{4.070108in}{2.477034in}}%
\pgfpathlineto{\pgfqpoint{4.070859in}{2.349902in}}%
\pgfpathlineto{\pgfqpoint{4.071499in}{2.440498in}}%
\pgfpathlineto{\pgfqpoint{4.071831in}{2.283515in}}%
\pgfpathlineto{\pgfqpoint{4.071929in}{2.291611in}}%
\pgfpathlineto{\pgfqpoint{4.072102in}{2.457506in}}%
\pgfpathlineto{\pgfqpoint{4.073259in}{2.376861in}}%
\pgfpathlineto{\pgfqpoint{4.073320in}{2.304449in}}%
\pgfpathlineto{\pgfqpoint{4.074256in}{2.449175in}}%
\pgfpathlineto{\pgfqpoint{4.074342in}{2.404058in}}%
\pgfpathlineto{\pgfqpoint{4.074859in}{2.442631in}}%
\pgfpathlineto{\pgfqpoint{4.074440in}{2.300432in}}%
\pgfpathlineto{\pgfqpoint{4.075388in}{2.352493in}}%
\pgfpathlineto{\pgfqpoint{4.075991in}{2.285601in}}%
\pgfpathlineto{\pgfqpoint{4.076262in}{2.457193in}}%
\pgfpathlineto{\pgfqpoint{4.076422in}{2.398274in}}%
\pgfpathlineto{\pgfqpoint{4.076459in}{2.441493in}}%
\pgfpathlineto{\pgfqpoint{4.076557in}{2.276679in}}%
\pgfpathlineto{\pgfqpoint{4.077480in}{2.332219in}}%
\pgfpathlineto{\pgfqpoint{4.078034in}{2.312988in}}%
\pgfpathlineto{\pgfqpoint{4.078354in}{2.441053in}}%
\pgfpathlineto{\pgfqpoint{4.078428in}{2.427720in}}%
\pgfpathlineto{\pgfqpoint{4.079055in}{2.483532in}}%
\pgfpathlineto{\pgfqpoint{4.078809in}{2.307006in}}%
\pgfpathlineto{\pgfqpoint{4.079351in}{2.312211in}}%
\pgfpathlineto{\pgfqpoint{4.080225in}{2.270857in}}%
\pgfpathlineto{\pgfqpoint{4.080003in}{2.424053in}}%
\pgfpathlineto{\pgfqpoint{4.080335in}{2.361367in}}%
\pgfpathlineto{\pgfqpoint{4.080446in}{2.505263in}}%
\pgfpathlineto{\pgfqpoint{4.080779in}{2.255760in}}%
\pgfpathlineto{\pgfqpoint{4.081443in}{2.378141in}}%
\pgfpathlineto{\pgfqpoint{4.082219in}{2.285275in}}%
\pgfpathlineto{\pgfqpoint{4.081837in}{2.466324in}}%
\pgfpathlineto{\pgfqpoint{4.082502in}{2.432837in}}%
\pgfpathlineto{\pgfqpoint{4.082612in}{2.501353in}}%
\pgfpathlineto{\pgfqpoint{4.082895in}{2.276960in}}%
\pgfpathlineto{\pgfqpoint{4.083548in}{2.341901in}}%
\pgfpathlineto{\pgfqpoint{4.084360in}{2.235981in}}%
\pgfpathlineto{\pgfqpoint{4.084028in}{2.476795in}}%
\pgfpathlineto{\pgfqpoint{4.084569in}{2.447321in}}%
\pgfpathlineto{\pgfqpoint{4.084680in}{2.527233in}}%
\pgfpathlineto{\pgfqpoint{4.084926in}{2.276687in}}%
\pgfpathlineto{\pgfqpoint{4.085591in}{2.354864in}}%
\pgfpathlineto{\pgfqpoint{4.085763in}{2.281590in}}%
\pgfpathlineto{\pgfqpoint{4.086145in}{2.490594in}}%
\pgfpathlineto{\pgfqpoint{4.086625in}{2.380310in}}%
\pgfpathlineto{\pgfqpoint{4.087363in}{2.505114in}}%
\pgfpathlineto{\pgfqpoint{4.087166in}{2.273510in}}%
\pgfpathlineto{\pgfqpoint{4.087708in}{2.292460in}}%
\pgfpathlineto{\pgfqpoint{4.088779in}{2.524775in}}%
\pgfpathlineto{\pgfqpoint{4.088532in}{2.224175in}}%
\pgfpathlineto{\pgfqpoint{4.088975in}{2.417452in}}%
\pgfpathlineto{\pgfqpoint{4.089099in}{2.280121in}}%
\pgfpathlineto{\pgfqpoint{4.089135in}{2.235092in}}%
\pgfpathlineto{\pgfqpoint{4.089480in}{2.465240in}}%
\pgfpathlineto{\pgfqpoint{4.090083in}{2.387609in}}%
\pgfpathlineto{\pgfqpoint{4.091006in}{2.512430in}}%
\pgfpathlineto{\pgfqpoint{4.090539in}{2.266637in}}%
\pgfpathlineto{\pgfqpoint{4.091154in}{2.335852in}}%
\pgfpathlineto{\pgfqpoint{4.091388in}{2.230207in}}%
\pgfpathlineto{\pgfqpoint{4.091597in}{2.539609in}}%
\pgfpathlineto{\pgfqpoint{4.092249in}{2.340640in}}%
\pgfpathlineto{\pgfqpoint{4.092988in}{2.546656in}}%
\pgfpathlineto{\pgfqpoint{4.092754in}{2.250810in}}%
\pgfpathlineto{\pgfqpoint{4.093308in}{2.275146in}}%
\pgfpathlineto{\pgfqpoint{4.093345in}{2.246574in}}%
\pgfpathlineto{\pgfqpoint{4.093714in}{2.469044in}}%
\pgfpathlineto{\pgfqpoint{4.094292in}{2.378531in}}%
\pgfpathlineto{\pgfqpoint{4.094563in}{2.487603in}}%
\pgfpathlineto{\pgfqpoint{4.094932in}{2.287133in}}%
\pgfpathlineto{\pgfqpoint{4.095375in}{2.370089in}}%
\pgfpathlineto{\pgfqpoint{4.096311in}{2.238084in}}%
\pgfpathlineto{\pgfqpoint{4.095819in}{2.466158in}}%
\pgfpathlineto{\pgfqpoint{4.096471in}{2.377330in}}%
\pgfpathlineto{\pgfqpoint{4.097209in}{2.505880in}}%
\pgfpathlineto{\pgfqpoint{4.096902in}{2.223064in}}%
\pgfpathlineto{\pgfqpoint{4.097554in}{2.360789in}}%
\pgfpathlineto{\pgfqpoint{4.097591in}{2.299115in}}%
\pgfpathlineto{\pgfqpoint{4.098132in}{2.471828in}}%
\pgfpathlineto{\pgfqpoint{4.098612in}{2.423486in}}%
\pgfpathlineto{\pgfqpoint{4.099326in}{2.497421in}}%
\pgfpathlineto{\pgfqpoint{4.099117in}{2.268760in}}%
\pgfpathlineto{\pgfqpoint{4.099658in}{2.367688in}}%
\pgfpathlineto{\pgfqpoint{4.099855in}{2.254354in}}%
\pgfpathlineto{\pgfqpoint{4.100225in}{2.467742in}}%
\pgfpathlineto{\pgfqpoint{4.100742in}{2.431247in}}%
\pgfpathlineto{\pgfqpoint{4.100778in}{2.441742in}}%
\pgfpathlineto{\pgfqpoint{4.100803in}{2.438333in}}%
\pgfpathlineto{\pgfqpoint{4.101418in}{2.482204in}}%
\pgfpathlineto{\pgfqpoint{4.101049in}{2.277501in}}%
\pgfpathlineto{\pgfqpoint{4.101800in}{2.357199in}}%
\pgfpathlineto{\pgfqpoint{4.102662in}{2.255753in}}%
\pgfpathlineto{\pgfqpoint{4.102292in}{2.460198in}}%
\pgfpathlineto{\pgfqpoint{4.102846in}{2.403944in}}%
\pgfpathlineto{\pgfqpoint{4.103769in}{2.481552in}}%
\pgfpathlineto{\pgfqpoint{4.103412in}{2.241150in}}%
\pgfpathlineto{\pgfqpoint{4.103892in}{2.337075in}}%
\pgfpathlineto{\pgfqpoint{4.104015in}{2.272777in}}%
\pgfpathlineto{\pgfqpoint{4.104938in}{2.443317in}}%
\pgfpathlineto{\pgfqpoint{4.105148in}{2.478291in}}%
\pgfpathlineto{\pgfqpoint{4.105308in}{2.293832in}}%
\pgfpathlineto{\pgfqpoint{4.105985in}{2.387507in}}%
\pgfpathlineto{\pgfqpoint{4.106292in}{2.270300in}}%
\pgfpathlineto{\pgfqpoint{4.106538in}{2.472301in}}%
\pgfpathlineto{\pgfqpoint{4.107018in}{2.401526in}}%
\pgfpathlineto{\pgfqpoint{4.107929in}{2.485623in}}%
\pgfpathlineto{\pgfqpoint{4.107560in}{2.234162in}}%
\pgfpathlineto{\pgfqpoint{4.108102in}{2.359253in}}%
\pgfpathlineto{\pgfqpoint{4.109025in}{2.271505in}}%
\pgfpathlineto{\pgfqpoint{4.108705in}{2.478613in}}%
\pgfpathlineto{\pgfqpoint{4.109111in}{2.381246in}}%
\pgfpathlineto{\pgfqpoint{4.109382in}{2.449537in}}%
\pgfpathlineto{\pgfqpoint{4.109529in}{2.313738in}}%
\pgfpathlineto{\pgfqpoint{4.110218in}{2.380910in}}%
\pgfpathlineto{\pgfqpoint{4.110748in}{2.466797in}}%
\pgfpathlineto{\pgfqpoint{4.110366in}{2.259265in}}%
\pgfpathlineto{\pgfqpoint{4.111351in}{2.407310in}}%
\pgfpathlineto{\pgfqpoint{4.111794in}{2.298888in}}%
\pgfpathlineto{\pgfqpoint{4.111474in}{2.479572in}}%
\pgfpathlineto{\pgfqpoint{4.112495in}{2.325339in}}%
\pgfpathlineto{\pgfqpoint{4.112852in}{2.475228in}}%
\pgfpathlineto{\pgfqpoint{4.113185in}{2.249583in}}%
\pgfpathlineto{\pgfqpoint{4.113652in}{2.393538in}}%
\pgfpathlineto{\pgfqpoint{4.114588in}{2.250944in}}%
\pgfpathlineto{\pgfqpoint{4.114305in}{2.462878in}}%
\pgfpathlineto{\pgfqpoint{4.114760in}{2.392520in}}%
\pgfpathlineto{\pgfqpoint{4.115018in}{2.456935in}}%
\pgfpathlineto{\pgfqpoint{4.115252in}{2.278252in}}%
\pgfpathlineto{\pgfqpoint{4.115806in}{2.390653in}}%
\pgfpathlineto{\pgfqpoint{4.115843in}{2.285331in}}%
\pgfpathlineto{\pgfqpoint{4.116397in}{2.452150in}}%
\pgfpathlineto{\pgfqpoint{4.116914in}{2.379116in}}%
\pgfpathlineto{\pgfqpoint{4.117357in}{2.318665in}}%
\pgfpathlineto{\pgfqpoint{4.117074in}{2.486298in}}%
\pgfpathlineto{\pgfqpoint{4.117972in}{2.396558in}}%
\pgfpathlineto{\pgfqpoint{4.118514in}{2.477228in}}%
\pgfpathlineto{\pgfqpoint{4.118797in}{2.246341in}}%
\pgfpathlineto{\pgfqpoint{4.119080in}{2.394927in}}%
\pgfpathlineto{\pgfqpoint{4.119178in}{2.469371in}}%
\pgfpathlineto{\pgfqpoint{4.119375in}{2.302379in}}%
\pgfpathlineto{\pgfqpoint{4.120138in}{2.375999in}}%
\pgfpathlineto{\pgfqpoint{4.120237in}{2.323394in}}%
\pgfpathlineto{\pgfqpoint{4.120606in}{2.448898in}}%
\pgfpathlineto{\pgfqpoint{4.121234in}{2.393442in}}%
\pgfpathlineto{\pgfqpoint{4.122120in}{2.451397in}}%
\pgfpathlineto{\pgfqpoint{4.121751in}{2.299233in}}%
\pgfpathlineto{\pgfqpoint{4.122292in}{2.362578in}}%
\pgfpathlineto{\pgfqpoint{4.122341in}{2.243991in}}%
\pgfpathlineto{\pgfqpoint{4.122711in}{2.494857in}}%
\pgfpathlineto{\pgfqpoint{4.123375in}{2.386358in}}%
\pgfpathlineto{\pgfqpoint{4.124151in}{2.447294in}}%
\pgfpathlineto{\pgfqpoint{4.123732in}{2.328273in}}%
\pgfpathlineto{\pgfqpoint{4.124434in}{2.328910in}}%
\pgfpathlineto{\pgfqpoint{4.125283in}{2.286464in}}%
\pgfpathlineto{\pgfqpoint{4.125061in}{2.447880in}}%
\pgfpathlineto{\pgfqpoint{4.125320in}{2.390438in}}%
\pgfpathlineto{\pgfqpoint{4.126243in}{2.468523in}}%
\pgfpathlineto{\pgfqpoint{4.125874in}{2.291741in}}%
\pgfpathlineto{\pgfqpoint{4.126428in}{2.394174in}}%
\pgfpathlineto{\pgfqpoint{4.126945in}{2.461273in}}%
\pgfpathlineto{\pgfqpoint{4.126563in}{2.295476in}}%
\pgfpathlineto{\pgfqpoint{4.127252in}{2.342831in}}%
\pgfpathlineto{\pgfqpoint{4.128089in}{2.289275in}}%
\pgfpathlineto{\pgfqpoint{4.127695in}{2.445275in}}%
\pgfpathlineto{\pgfqpoint{4.128274in}{2.406899in}}%
\pgfpathlineto{\pgfqpoint{4.129061in}{2.459928in}}%
\pgfpathlineto{\pgfqpoint{4.128668in}{2.294047in}}%
\pgfpathlineto{\pgfqpoint{4.129332in}{2.344500in}}%
\pgfpathlineto{\pgfqpoint{4.129406in}{2.269327in}}%
\pgfpathlineto{\pgfqpoint{4.129775in}{2.502819in}}%
\pgfpathlineto{\pgfqpoint{4.130403in}{2.408297in}}%
\pgfpathlineto{\pgfqpoint{4.131621in}{2.273431in}}%
\pgfpathlineto{\pgfqpoint{4.130477in}{2.460290in}}%
\pgfpathlineto{\pgfqpoint{4.131634in}{2.288206in}}%
\pgfpathlineto{\pgfqpoint{4.131880in}{2.460122in}}%
\pgfpathlineto{\pgfqpoint{4.132212in}{2.266066in}}%
\pgfpathlineto{\pgfqpoint{4.132766in}{2.390608in}}%
\pgfpathlineto{\pgfqpoint{4.133738in}{2.286586in}}%
\pgfpathlineto{\pgfqpoint{4.133332in}{2.456720in}}%
\pgfpathlineto{\pgfqpoint{4.133874in}{2.383022in}}%
\pgfpathlineto{\pgfqpoint{4.133935in}{2.471472in}}%
\pgfpathlineto{\pgfqpoint{4.134329in}{2.291209in}}%
\pgfpathlineto{\pgfqpoint{4.134957in}{2.360941in}}%
\pgfpathlineto{\pgfqpoint{4.135757in}{2.281310in}}%
\pgfpathlineto{\pgfqpoint{4.135375in}{2.456429in}}%
\pgfpathlineto{\pgfqpoint{4.136003in}{2.385758in}}%
\pgfpathlineto{\pgfqpoint{4.136077in}{2.462365in}}%
\pgfpathlineto{\pgfqpoint{4.136508in}{2.276643in}}%
\pgfpathlineto{\pgfqpoint{4.136938in}{2.344515in}}%
\pgfpathlineto{\pgfqpoint{4.137098in}{2.287676in}}%
\pgfpathlineto{\pgfqpoint{4.137480in}{2.473162in}}%
\pgfpathlineto{\pgfqpoint{4.137984in}{2.379980in}}%
\pgfpathlineto{\pgfqpoint{4.138908in}{2.470587in}}%
\pgfpathlineto{\pgfqpoint{4.138551in}{2.305372in}}%
\pgfpathlineto{\pgfqpoint{4.139068in}{2.329900in}}%
\pgfpathlineto{\pgfqpoint{4.139831in}{2.451665in}}%
\pgfpathlineto{\pgfqpoint{4.140040in}{2.270197in}}%
\pgfpathlineto{\pgfqpoint{4.140212in}{2.377162in}}%
\pgfpathlineto{\pgfqpoint{4.140631in}{2.271084in}}%
\pgfpathlineto{\pgfqpoint{4.140323in}{2.440293in}}%
\pgfpathlineto{\pgfqpoint{4.140975in}{2.428509in}}%
\pgfpathlineto{\pgfqpoint{4.141012in}{2.468428in}}%
\pgfpathlineto{\pgfqpoint{4.141443in}{2.295270in}}%
\pgfpathlineto{\pgfqpoint{4.142009in}{2.324617in}}%
\pgfpathlineto{\pgfqpoint{4.142846in}{2.305300in}}%
\pgfpathlineto{\pgfqpoint{4.142452in}{2.438456in}}%
\pgfpathlineto{\pgfqpoint{4.143043in}{2.401318in}}%
\pgfpathlineto{\pgfqpoint{4.143351in}{2.433772in}}%
\pgfpathlineto{\pgfqpoint{4.143560in}{2.303193in}}%
\pgfpathlineto{\pgfqpoint{4.144089in}{2.363389in}}%
\pgfpathlineto{\pgfqpoint{4.144175in}{2.294312in}}%
\pgfpathlineto{\pgfqpoint{4.144544in}{2.433635in}}%
\pgfpathlineto{\pgfqpoint{4.145135in}{2.393450in}}%
\pgfpathlineto{\pgfqpoint{4.145997in}{2.435949in}}%
\pgfpathlineto{\pgfqpoint{4.145652in}{2.308998in}}%
\pgfpathlineto{\pgfqpoint{4.146194in}{2.374147in}}%
\pgfpathlineto{\pgfqpoint{4.146354in}{2.310622in}}%
\pgfpathlineto{\pgfqpoint{4.146588in}{2.425007in}}%
\pgfpathlineto{\pgfqpoint{4.147289in}{2.383488in}}%
\pgfpathlineto{\pgfqpoint{4.147720in}{2.295085in}}%
\pgfpathlineto{\pgfqpoint{4.147523in}{2.429576in}}%
\pgfpathlineto{\pgfqpoint{4.147991in}{2.408035in}}%
\pgfpathlineto{\pgfqpoint{4.148089in}{2.446175in}}%
\pgfpathlineto{\pgfqpoint{4.148311in}{2.314270in}}%
\pgfpathlineto{\pgfqpoint{4.149061in}{2.365568in}}%
\pgfpathlineto{\pgfqpoint{4.149160in}{2.306719in}}%
\pgfpathlineto{\pgfqpoint{4.149529in}{2.436758in}}%
\pgfpathlineto{\pgfqpoint{4.150095in}{2.399915in}}%
\pgfpathlineto{\pgfqpoint{4.150120in}{2.432972in}}%
\pgfpathlineto{\pgfqpoint{4.150563in}{2.318184in}}%
\pgfpathlineto{\pgfqpoint{4.151191in}{2.376982in}}%
\pgfpathlineto{\pgfqpoint{4.152224in}{2.455608in}}%
\pgfpathlineto{\pgfqpoint{4.151252in}{2.286078in}}%
\pgfpathlineto{\pgfqpoint{4.152348in}{2.395372in}}%
\pgfpathlineto{\pgfqpoint{4.153357in}{2.311764in}}%
\pgfpathlineto{\pgfqpoint{4.153172in}{2.423861in}}%
\pgfpathlineto{\pgfqpoint{4.153541in}{2.360476in}}%
\pgfpathlineto{\pgfqpoint{4.153652in}{2.443465in}}%
\pgfpathlineto{\pgfqpoint{4.154071in}{2.287013in}}%
\pgfpathlineto{\pgfqpoint{4.154624in}{2.347202in}}%
\pgfpathlineto{\pgfqpoint{4.155474in}{2.290378in}}%
\pgfpathlineto{\pgfqpoint{4.155166in}{2.444181in}}%
\pgfpathlineto{\pgfqpoint{4.155683in}{2.392637in}}%
\pgfpathlineto{\pgfqpoint{4.155757in}{2.455214in}}%
\pgfpathlineto{\pgfqpoint{4.156175in}{2.294018in}}%
\pgfpathlineto{\pgfqpoint{4.156741in}{2.360290in}}%
\pgfpathlineto{\pgfqpoint{4.157591in}{2.290017in}}%
\pgfpathlineto{\pgfqpoint{4.157258in}{2.430109in}}%
\pgfpathlineto{\pgfqpoint{4.157800in}{2.417638in}}%
\pgfpathlineto{\pgfqpoint{4.157837in}{2.429118in}}%
\pgfpathlineto{\pgfqpoint{4.157849in}{2.435418in}}%
\pgfpathlineto{\pgfqpoint{4.158169in}{2.311786in}}%
\pgfpathlineto{\pgfqpoint{4.158821in}{2.368110in}}%
\pgfpathlineto{\pgfqpoint{4.159695in}{2.282016in}}%
\pgfpathlineto{\pgfqpoint{4.159301in}{2.428735in}}%
\pgfpathlineto{\pgfqpoint{4.159880in}{2.410219in}}%
\pgfpathlineto{\pgfqpoint{4.160052in}{2.426946in}}%
\pgfpathlineto{\pgfqpoint{4.160286in}{2.309608in}}%
\pgfpathlineto{\pgfqpoint{4.160926in}{2.376189in}}%
\pgfpathlineto{\pgfqpoint{4.161787in}{2.305810in}}%
\pgfpathlineto{\pgfqpoint{4.161443in}{2.429723in}}%
\pgfpathlineto{\pgfqpoint{4.161997in}{2.395315in}}%
\pgfpathlineto{\pgfqpoint{4.162034in}{2.438653in}}%
\pgfpathlineto{\pgfqpoint{4.162378in}{2.318699in}}%
\pgfpathlineto{\pgfqpoint{4.163055in}{2.331778in}}%
\pgfpathlineto{\pgfqpoint{4.163437in}{2.427256in}}%
\pgfpathlineto{\pgfqpoint{4.163904in}{2.296196in}}%
\pgfpathlineto{\pgfqpoint{4.164298in}{2.401018in}}%
\pgfpathlineto{\pgfqpoint{4.164495in}{2.311705in}}%
\pgfpathlineto{\pgfqpoint{4.164864in}{2.415449in}}%
\pgfpathlineto{\pgfqpoint{4.165431in}{2.347527in}}%
\pgfpathlineto{\pgfqpoint{4.166378in}{2.422773in}}%
\pgfpathlineto{\pgfqpoint{4.166009in}{2.296828in}}%
\pgfpathlineto{\pgfqpoint{4.166551in}{2.380820in}}%
\pgfpathlineto{\pgfqpoint{4.166587in}{2.304295in}}%
\pgfpathlineto{\pgfqpoint{4.166969in}{2.414704in}}%
\pgfpathlineto{\pgfqpoint{4.167646in}{2.411880in}}%
\pgfpathlineto{\pgfqpoint{4.167671in}{2.423284in}}%
\pgfpathlineto{\pgfqpoint{4.168003in}{2.308514in}}%
\pgfpathlineto{\pgfqpoint{4.168667in}{2.351630in}}%
\pgfpathlineto{\pgfqpoint{4.169529in}{2.306156in}}%
\pgfpathlineto{\pgfqpoint{4.169320in}{2.425787in}}%
\pgfpathlineto{\pgfqpoint{4.169738in}{2.389108in}}%
\pgfpathlineto{\pgfqpoint{4.169911in}{2.419985in}}%
\pgfpathlineto{\pgfqpoint{4.170120in}{2.321972in}}%
\pgfpathlineto{\pgfqpoint{4.170772in}{2.371715in}}%
\pgfpathlineto{\pgfqpoint{4.170920in}{2.317294in}}%
\pgfpathlineto{\pgfqpoint{4.171264in}{2.418976in}}%
\pgfpathlineto{\pgfqpoint{4.171843in}{2.413279in}}%
\pgfpathlineto{\pgfqpoint{4.171855in}{2.420949in}}%
\pgfpathlineto{\pgfqpoint{4.172323in}{2.312826in}}%
\pgfpathlineto{\pgfqpoint{4.172877in}{2.368523in}}%
\pgfpathlineto{\pgfqpoint{4.172914in}{2.328788in}}%
\pgfpathlineto{\pgfqpoint{4.173455in}{2.415587in}}%
\pgfpathlineto{\pgfqpoint{4.173960in}{2.391299in}}%
\pgfpathlineto{\pgfqpoint{4.174883in}{2.423431in}}%
\pgfpathlineto{\pgfqpoint{4.174427in}{2.327419in}}%
\pgfpathlineto{\pgfqpoint{4.175006in}{2.354372in}}%
\pgfpathlineto{\pgfqpoint{4.175117in}{2.307616in}}%
\pgfpathlineto{\pgfqpoint{4.175474in}{2.427113in}}%
\pgfpathlineto{\pgfqpoint{4.176015in}{2.364973in}}%
\pgfpathlineto{\pgfqpoint{4.176987in}{2.427586in}}%
\pgfpathlineto{\pgfqpoint{4.176532in}{2.327588in}}%
\pgfpathlineto{\pgfqpoint{4.177098in}{2.352796in}}%
\pgfpathlineto{\pgfqpoint{4.177923in}{2.316951in}}%
\pgfpathlineto{\pgfqpoint{4.177578in}{2.429198in}}%
\pgfpathlineto{\pgfqpoint{4.178120in}{2.365529in}}%
\pgfpathlineto{\pgfqpoint{4.179006in}{2.414901in}}%
\pgfpathlineto{\pgfqpoint{4.178637in}{2.320399in}}%
\pgfpathlineto{\pgfqpoint{4.179203in}{2.347651in}}%
\pgfpathlineto{\pgfqpoint{4.180015in}{2.328529in}}%
\pgfpathlineto{\pgfqpoint{4.179707in}{2.410211in}}%
\pgfpathlineto{\pgfqpoint{4.180175in}{2.368545in}}%
\pgfpathlineto{\pgfqpoint{4.180520in}{2.428361in}}%
\pgfpathlineto{\pgfqpoint{4.180729in}{2.317987in}}%
\pgfpathlineto{\pgfqpoint{4.181270in}{2.373796in}}%
\pgfpathlineto{\pgfqpoint{4.182120in}{2.322652in}}%
\pgfpathlineto{\pgfqpoint{4.181874in}{2.413930in}}%
\pgfpathlineto{\pgfqpoint{4.182366in}{2.378655in}}%
\pgfpathlineto{\pgfqpoint{4.182575in}{2.413525in}}%
\pgfpathlineto{\pgfqpoint{4.182944in}{2.328571in}}%
\pgfpathlineto{\pgfqpoint{4.183474in}{2.386067in}}%
\pgfpathlineto{\pgfqpoint{4.183523in}{2.314759in}}%
\pgfpathlineto{\pgfqpoint{4.183904in}{2.402313in}}%
\pgfpathlineto{\pgfqpoint{4.184569in}{2.387055in}}%
\pgfpathlineto{\pgfqpoint{4.184667in}{2.424422in}}%
\pgfpathlineto{\pgfqpoint{4.185061in}{2.333319in}}%
\pgfpathlineto{\pgfqpoint{4.185615in}{2.338047in}}%
\pgfpathlineto{\pgfqpoint{4.186083in}{2.423455in}}%
\pgfpathlineto{\pgfqpoint{4.186317in}{2.318996in}}%
\pgfpathlineto{\pgfqpoint{4.186858in}{2.377707in}}%
\pgfpathlineto{\pgfqpoint{4.187720in}{2.335219in}}%
\pgfpathlineto{\pgfqpoint{4.187264in}{2.402568in}}%
\pgfpathlineto{\pgfqpoint{4.187954in}{2.385994in}}%
\pgfpathlineto{\pgfqpoint{4.188187in}{2.428654in}}%
\pgfpathlineto{\pgfqpoint{4.188397in}{2.323084in}}%
\pgfpathlineto{\pgfqpoint{4.188975in}{2.344880in}}%
\pgfpathlineto{\pgfqpoint{4.189270in}{2.324226in}}%
\pgfpathlineto{\pgfqpoint{4.189615in}{2.427266in}}%
\pgfpathlineto{\pgfqpoint{4.190021in}{2.359339in}}%
\pgfpathlineto{\pgfqpoint{4.190981in}{2.409116in}}%
\pgfpathlineto{\pgfqpoint{4.190674in}{2.341378in}}%
\pgfpathlineto{\pgfqpoint{4.191154in}{2.399425in}}%
\pgfpathlineto{\pgfqpoint{4.191917in}{2.327151in}}%
\pgfpathlineto{\pgfqpoint{4.191720in}{2.436661in}}%
\pgfpathlineto{\pgfqpoint{4.192274in}{2.380727in}}%
\pgfpathlineto{\pgfqpoint{4.192409in}{2.413431in}}%
\pgfpathlineto{\pgfqpoint{4.192790in}{2.331096in}}%
\pgfpathlineto{\pgfqpoint{4.193295in}{2.343884in}}%
\pgfpathlineto{\pgfqpoint{4.193997in}{2.327269in}}%
\pgfpathlineto{\pgfqpoint{4.193800in}{2.422160in}}%
\pgfpathlineto{\pgfqpoint{4.194329in}{2.410006in}}%
\pgfpathlineto{\pgfqpoint{4.194723in}{2.328071in}}%
\pgfpathlineto{\pgfqpoint{4.195240in}{2.416695in}}%
\pgfpathlineto{\pgfqpoint{4.195547in}{2.362902in}}%
\pgfpathlineto{\pgfqpoint{4.196495in}{2.346815in}}%
\pgfpathlineto{\pgfqpoint{4.195867in}{2.428035in}}%
\pgfpathlineto{\pgfqpoint{4.196544in}{2.389361in}}%
\pgfpathlineto{\pgfqpoint{4.197283in}{2.428559in}}%
\pgfpathlineto{\pgfqpoint{4.197517in}{2.322938in}}%
\pgfpathlineto{\pgfqpoint{4.197640in}{2.381861in}}%
\pgfpathlineto{\pgfqpoint{4.197677in}{2.327376in}}%
\pgfpathlineto{\pgfqpoint{4.198464in}{2.409801in}}%
\pgfpathlineto{\pgfqpoint{4.198735in}{2.399823in}}%
\pgfpathlineto{\pgfqpoint{4.199597in}{2.321246in}}%
\pgfpathlineto{\pgfqpoint{4.199400in}{2.438346in}}%
\pgfpathlineto{\pgfqpoint{4.199892in}{2.366427in}}%
\pgfpathlineto{\pgfqpoint{4.200827in}{2.423598in}}%
\pgfpathlineto{\pgfqpoint{4.200187in}{2.333142in}}%
\pgfpathlineto{\pgfqpoint{4.200975in}{2.361275in}}%
\pgfpathlineto{\pgfqpoint{4.201701in}{2.330558in}}%
\pgfpathlineto{\pgfqpoint{4.201763in}{2.414048in}}%
\pgfpathlineto{\pgfqpoint{4.202070in}{2.376450in}}%
\pgfpathlineto{\pgfqpoint{4.202403in}{2.317880in}}%
\pgfpathlineto{\pgfqpoint{4.202353in}{2.410762in}}%
\pgfpathlineto{\pgfqpoint{4.202883in}{2.391430in}}%
\pgfpathlineto{\pgfqpoint{4.202932in}{2.435837in}}%
\pgfpathlineto{\pgfqpoint{4.203129in}{2.332623in}}%
\pgfpathlineto{\pgfqpoint{4.203953in}{2.361063in}}%
\pgfpathlineto{\pgfqpoint{4.204113in}{2.405857in}}%
\pgfpathlineto{\pgfqpoint{4.204507in}{2.332946in}}%
\pgfpathlineto{\pgfqpoint{4.205073in}{2.364294in}}%
\pgfpathlineto{\pgfqpoint{4.205197in}{2.337776in}}%
\pgfpathlineto{\pgfqpoint{4.205295in}{2.413158in}}%
\pgfpathlineto{\pgfqpoint{4.206107in}{2.377922in}}%
\pgfpathlineto{\pgfqpoint{4.207067in}{2.418114in}}%
\pgfpathlineto{\pgfqpoint{4.206957in}{2.347946in}}%
\pgfpathlineto{\pgfqpoint{4.207227in}{2.396023in}}%
\pgfpathlineto{\pgfqpoint{4.208298in}{2.344798in}}%
\pgfpathlineto{\pgfqpoint{4.208249in}{2.418091in}}%
\pgfpathlineto{\pgfqpoint{4.208384in}{2.377035in}}%
\pgfpathlineto{\pgfqpoint{4.208495in}{2.424942in}}%
\pgfpathlineto{\pgfqpoint{4.208877in}{2.344199in}}%
\pgfpathlineto{\pgfqpoint{4.209467in}{2.371992in}}%
\pgfpathlineto{\pgfqpoint{4.210083in}{2.336080in}}%
\pgfpathlineto{\pgfqpoint{4.210021in}{2.420012in}}%
\pgfpathlineto{\pgfqpoint{4.210563in}{2.384637in}}%
\pgfpathlineto{\pgfqpoint{4.210612in}{2.421727in}}%
\pgfpathlineto{\pgfqpoint{4.210797in}{2.337457in}}%
\pgfpathlineto{\pgfqpoint{4.211646in}{2.353165in}}%
\pgfpathlineto{\pgfqpoint{4.211670in}{2.339608in}}%
\pgfpathlineto{\pgfqpoint{4.212027in}{2.410529in}}%
\pgfpathlineto{\pgfqpoint{4.212680in}{2.396357in}}%
\pgfpathlineto{\pgfqpoint{4.212963in}{2.411733in}}%
\pgfpathlineto{\pgfqpoint{4.213603in}{2.339129in}}%
\pgfpathlineto{\pgfqpoint{4.213713in}{2.376820in}}%
\pgfpathlineto{\pgfqpoint{4.213763in}{2.338586in}}%
\pgfpathlineto{\pgfqpoint{4.214132in}{2.413028in}}%
\pgfpathlineto{\pgfqpoint{4.214809in}{2.377076in}}%
\pgfpathlineto{\pgfqpoint{4.215929in}{2.414879in}}%
\pgfpathlineto{\pgfqpoint{4.215387in}{2.343155in}}%
\pgfpathlineto{\pgfqpoint{4.215941in}{2.402309in}}%
\pgfpathlineto{\pgfqpoint{4.216569in}{2.339447in}}%
\pgfpathlineto{\pgfqpoint{4.216175in}{2.407536in}}%
\pgfpathlineto{\pgfqpoint{4.217061in}{2.369492in}}%
\pgfpathlineto{\pgfqpoint{4.217443in}{2.345482in}}%
\pgfpathlineto{\pgfqpoint{4.217344in}{2.407372in}}%
\pgfpathlineto{\pgfqpoint{4.217689in}{2.406736in}}%
\pgfpathlineto{\pgfqpoint{4.218280in}{2.423235in}}%
\pgfpathlineto{\pgfqpoint{4.218489in}{2.323062in}}%
\pgfpathlineto{\pgfqpoint{4.218747in}{2.382360in}}%
\pgfpathlineto{\pgfqpoint{4.219523in}{2.332314in}}%
\pgfpathlineto{\pgfqpoint{4.219461in}{2.415789in}}%
\pgfpathlineto{\pgfqpoint{4.219892in}{2.347911in}}%
\pgfpathlineto{\pgfqpoint{4.220643in}{2.432301in}}%
\pgfpathlineto{\pgfqpoint{4.220704in}{2.337925in}}%
\pgfpathlineto{\pgfqpoint{4.221024in}{2.382435in}}%
\pgfpathlineto{\pgfqpoint{4.221344in}{2.421902in}}%
\pgfpathlineto{\pgfqpoint{4.221283in}{2.325612in}}%
\pgfpathlineto{\pgfqpoint{4.222095in}{2.373016in}}%
\pgfpathlineto{\pgfqpoint{4.223067in}{2.326313in}}%
\pgfpathlineto{\pgfqpoint{4.222427in}{2.418299in}}%
\pgfpathlineto{\pgfqpoint{4.223215in}{2.352833in}}%
\pgfpathlineto{\pgfqpoint{4.223855in}{2.411548in}}%
\pgfpathlineto{\pgfqpoint{4.224076in}{2.330882in}}%
\pgfpathlineto{\pgfqpoint{4.224360in}{2.380279in}}%
\pgfpathlineto{\pgfqpoint{4.225160in}{2.344532in}}%
\pgfpathlineto{\pgfqpoint{4.225036in}{2.408734in}}%
\pgfpathlineto{\pgfqpoint{4.225480in}{2.370986in}}%
\pgfpathlineto{\pgfqpoint{4.226218in}{2.421260in}}%
\pgfpathlineto{\pgfqpoint{4.225861in}{2.332635in}}%
\pgfpathlineto{\pgfqpoint{4.226575in}{2.381562in}}%
\pgfpathlineto{\pgfqpoint{4.226870in}{2.332943in}}%
\pgfpathlineto{\pgfqpoint{4.226661in}{2.423289in}}%
\pgfpathlineto{\pgfqpoint{4.227670in}{2.394968in}}%
\pgfpathlineto{\pgfqpoint{4.227953in}{2.334407in}}%
\pgfpathlineto{\pgfqpoint{4.228335in}{2.429174in}}%
\pgfpathlineto{\pgfqpoint{4.228864in}{2.360004in}}%
\pgfpathlineto{\pgfqpoint{4.229036in}{2.428601in}}%
\pgfpathlineto{\pgfqpoint{4.228975in}{2.326225in}}%
\pgfpathlineto{\pgfqpoint{4.229972in}{2.371286in}}%
\pgfpathlineto{\pgfqpoint{4.231055in}{2.333968in}}%
\pgfpathlineto{\pgfqpoint{4.230809in}{2.401370in}}%
\pgfpathlineto{\pgfqpoint{4.231092in}{2.359363in}}%
\pgfpathlineto{\pgfqpoint{4.231843in}{2.424557in}}%
\pgfpathlineto{\pgfqpoint{4.231633in}{2.338087in}}%
\pgfpathlineto{\pgfqpoint{4.232187in}{2.340410in}}%
\pgfpathlineto{\pgfqpoint{4.233246in}{2.407644in}}%
\pgfpathlineto{\pgfqpoint{4.233344in}{2.375888in}}%
\pgfpathlineto{\pgfqpoint{4.233566in}{2.342382in}}%
\pgfpathlineto{\pgfqpoint{4.233923in}{2.416981in}}%
\pgfpathlineto{\pgfqpoint{4.234464in}{2.356652in}}%
\pgfpathlineto{\pgfqpoint{4.234636in}{2.410410in}}%
\pgfpathlineto{\pgfqpoint{4.234563in}{2.337871in}}%
\pgfpathlineto{\pgfqpoint{4.235584in}{2.372195in}}%
\pgfpathlineto{\pgfqpoint{4.235658in}{2.336956in}}%
\pgfpathlineto{\pgfqpoint{4.236027in}{2.429300in}}%
\pgfpathlineto{\pgfqpoint{4.236593in}{2.401324in}}%
\pgfpathlineto{\pgfqpoint{4.236741in}{2.427122in}}%
\pgfpathlineto{\pgfqpoint{4.236680in}{2.346410in}}%
\pgfpathlineto{\pgfqpoint{4.237504in}{2.357118in}}%
\pgfpathlineto{\pgfqpoint{4.238452in}{2.340902in}}%
\pgfpathlineto{\pgfqpoint{4.237590in}{2.398784in}}%
\pgfpathlineto{\pgfqpoint{4.238563in}{2.384632in}}%
\pgfpathlineto{\pgfqpoint{4.239560in}{2.414297in}}%
\pgfpathlineto{\pgfqpoint{4.239449in}{2.349706in}}%
\pgfpathlineto{\pgfqpoint{4.239621in}{2.371134in}}%
\pgfpathlineto{\pgfqpoint{4.239867in}{2.341162in}}%
\pgfpathlineto{\pgfqpoint{4.239695in}{2.418986in}}%
\pgfpathlineto{\pgfqpoint{4.240729in}{2.370150in}}%
\pgfpathlineto{\pgfqpoint{4.241615in}{2.401844in}}%
\pgfpathlineto{\pgfqpoint{4.241418in}{2.344315in}}%
\pgfpathlineto{\pgfqpoint{4.241849in}{2.375879in}}%
\pgfpathlineto{\pgfqpoint{4.242243in}{2.337285in}}%
\pgfpathlineto{\pgfqpoint{4.242058in}{2.405734in}}%
\pgfpathlineto{\pgfqpoint{4.242969in}{2.356389in}}%
\pgfpathlineto{\pgfqpoint{4.243732in}{2.419393in}}%
\pgfpathlineto{\pgfqpoint{4.243646in}{2.341254in}}%
\pgfpathlineto{\pgfqpoint{4.244076in}{2.352916in}}%
\pgfpathlineto{\pgfqpoint{4.244433in}{2.429973in}}%
\pgfpathlineto{\pgfqpoint{4.245049in}{2.351287in}}%
\pgfpathlineto{\pgfqpoint{4.245184in}{2.363100in}}%
\pgfpathlineto{\pgfqpoint{4.245738in}{2.348089in}}%
\pgfpathlineto{\pgfqpoint{4.246206in}{2.400572in}}%
\pgfpathlineto{\pgfqpoint{4.246255in}{2.385325in}}%
\pgfpathlineto{\pgfqpoint{4.246698in}{2.403313in}}%
\pgfpathlineto{\pgfqpoint{4.246439in}{2.340461in}}%
\pgfpathlineto{\pgfqpoint{4.247313in}{2.371727in}}%
\pgfpathlineto{\pgfqpoint{4.248212in}{2.346496in}}%
\pgfpathlineto{\pgfqpoint{4.247387in}{2.415544in}}%
\pgfpathlineto{\pgfqpoint{4.248421in}{2.363355in}}%
\pgfpathlineto{\pgfqpoint{4.249061in}{2.408285in}}%
\pgfpathlineto{\pgfqpoint{4.249246in}{2.347307in}}%
\pgfpathlineto{\pgfqpoint{4.249553in}{2.369287in}}%
\pgfpathlineto{\pgfqpoint{4.250636in}{2.342648in}}%
\pgfpathlineto{\pgfqpoint{4.249763in}{2.408765in}}%
\pgfpathlineto{\pgfqpoint{4.250661in}{2.355540in}}%
\pgfpathlineto{\pgfqpoint{4.251424in}{2.410468in}}%
\pgfpathlineto{\pgfqpoint{4.251338in}{2.341896in}}%
\pgfpathlineto{\pgfqpoint{4.251769in}{2.360024in}}%
\pgfpathlineto{\pgfqpoint{4.252126in}{2.426988in}}%
\pgfpathlineto{\pgfqpoint{4.252372in}{2.347018in}}%
\pgfpathlineto{\pgfqpoint{4.252876in}{2.363716in}}%
\pgfpathlineto{\pgfqpoint{4.253430in}{2.328842in}}%
\pgfpathlineto{\pgfqpoint{4.253209in}{2.406606in}}%
\pgfpathlineto{\pgfqpoint{4.253873in}{2.380885in}}%
\pgfpathlineto{\pgfqpoint{4.253910in}{2.407475in}}%
\pgfpathlineto{\pgfqpoint{4.254132in}{2.338260in}}%
\pgfpathlineto{\pgfqpoint{4.254981in}{2.383792in}}%
\pgfpathlineto{\pgfqpoint{4.255203in}{2.346251in}}%
\pgfpathlineto{\pgfqpoint{4.255092in}{2.410029in}}%
\pgfpathlineto{\pgfqpoint{4.256126in}{2.363193in}}%
\pgfpathlineto{\pgfqpoint{4.256286in}{2.415147in}}%
\pgfpathlineto{\pgfqpoint{4.256396in}{2.348646in}}%
\pgfpathlineto{\pgfqpoint{4.257258in}{2.374530in}}%
\pgfpathlineto{\pgfqpoint{4.257652in}{2.345132in}}%
\pgfpathlineto{\pgfqpoint{4.257467in}{2.411250in}}%
\pgfpathlineto{\pgfqpoint{4.258366in}{2.375586in}}%
\pgfpathlineto{\pgfqpoint{4.259252in}{2.410937in}}%
\pgfpathlineto{\pgfqpoint{4.259030in}{2.346924in}}%
\pgfpathlineto{\pgfqpoint{4.259424in}{2.360878in}}%
\pgfpathlineto{\pgfqpoint{4.260064in}{2.351041in}}%
\pgfpathlineto{\pgfqpoint{4.259830in}{2.418400in}}%
\pgfpathlineto{\pgfqpoint{4.260335in}{2.378191in}}%
\pgfpathlineto{\pgfqpoint{4.260913in}{2.401093in}}%
\pgfpathlineto{\pgfqpoint{4.261123in}{2.330143in}}%
\pgfpathlineto{\pgfqpoint{4.261443in}{2.381240in}}%
\pgfpathlineto{\pgfqpoint{4.261824in}{2.324903in}}%
\pgfpathlineto{\pgfqpoint{4.261615in}{2.417005in}}%
\pgfpathlineto{\pgfqpoint{4.262550in}{2.370851in}}%
\pgfpathlineto{\pgfqpoint{4.262796in}{2.410282in}}%
\pgfpathlineto{\pgfqpoint{4.262907in}{2.340121in}}%
\pgfpathlineto{\pgfqpoint{4.263670in}{2.389019in}}%
\pgfpathlineto{\pgfqpoint{4.264790in}{2.346592in}}%
\pgfpathlineto{\pgfqpoint{4.263990in}{2.418743in}}%
\pgfpathlineto{\pgfqpoint{4.264815in}{2.361378in}}%
\pgfpathlineto{\pgfqpoint{4.265762in}{2.416708in}}%
\pgfpathlineto{\pgfqpoint{4.265332in}{2.345324in}}%
\pgfpathlineto{\pgfqpoint{4.265910in}{2.358624in}}%
\pgfpathlineto{\pgfqpoint{4.265996in}{2.337263in}}%
\pgfpathlineto{\pgfqpoint{4.266255in}{2.404618in}}%
\pgfpathlineto{\pgfqpoint{4.266907in}{2.381988in}}%
\pgfpathlineto{\pgfqpoint{4.267522in}{2.416476in}}%
\pgfpathlineto{\pgfqpoint{4.267756in}{2.355188in}}%
\pgfpathlineto{\pgfqpoint{4.268002in}{2.371445in}}%
\pgfpathlineto{\pgfqpoint{4.268384in}{2.399353in}}%
\pgfpathlineto{\pgfqpoint{4.268815in}{2.336538in}}%
\pgfpathlineto{\pgfqpoint{4.269110in}{2.374370in}}%
\pgfpathlineto{\pgfqpoint{4.269516in}{2.324404in}}%
\pgfpathlineto{\pgfqpoint{4.269307in}{2.416873in}}%
\pgfpathlineto{\pgfqpoint{4.270242in}{2.365060in}}%
\pgfpathlineto{\pgfqpoint{4.270489in}{2.414308in}}%
\pgfpathlineto{\pgfqpoint{4.270587in}{2.346107in}}%
\pgfpathlineto{\pgfqpoint{4.271375in}{2.389419in}}%
\pgfpathlineto{\pgfqpoint{4.272310in}{2.340583in}}%
\pgfpathlineto{\pgfqpoint{4.271670in}{2.413784in}}%
\pgfpathlineto{\pgfqpoint{4.272507in}{2.371672in}}%
\pgfpathlineto{\pgfqpoint{4.273442in}{2.410995in}}%
\pgfpathlineto{\pgfqpoint{4.273012in}{2.345415in}}%
\pgfpathlineto{\pgfqpoint{4.273602in}{2.363507in}}%
\pgfpathlineto{\pgfqpoint{4.273676in}{2.337932in}}%
\pgfpathlineto{\pgfqpoint{4.274513in}{2.412773in}}%
\pgfpathlineto{\pgfqpoint{4.274587in}{2.382322in}}%
\pgfpathlineto{\pgfqpoint{4.275215in}{2.420608in}}%
\pgfpathlineto{\pgfqpoint{4.274809in}{2.352850in}}%
\pgfpathlineto{\pgfqpoint{4.275695in}{2.381193in}}%
\pgfpathlineto{\pgfqpoint{4.276076in}{2.396403in}}%
\pgfpathlineto{\pgfqpoint{4.275769in}{2.362183in}}%
\pgfpathlineto{\pgfqpoint{4.276175in}{2.363997in}}%
\pgfpathlineto{\pgfqpoint{4.277209in}{2.335682in}}%
\pgfpathlineto{\pgfqpoint{4.276987in}{2.413042in}}%
\pgfpathlineto{\pgfqpoint{4.277246in}{2.373625in}}%
\pgfpathlineto{\pgfqpoint{4.278169in}{2.417476in}}%
\pgfpathlineto{\pgfqpoint{4.277898in}{2.331773in}}%
\pgfpathlineto{\pgfqpoint{4.278353in}{2.370306in}}%
\pgfpathlineto{\pgfqpoint{4.278439in}{2.376802in}}%
\pgfpathlineto{\pgfqpoint{4.278575in}{2.347581in}}%
\pgfpathlineto{\pgfqpoint{4.278587in}{2.345882in}}%
\pgfpathlineto{\pgfqpoint{4.278673in}{2.409080in}}%
\pgfpathlineto{\pgfqpoint{4.279326in}{2.379116in}}%
\pgfpathlineto{\pgfqpoint{4.279362in}{2.419801in}}%
\pgfpathlineto{\pgfqpoint{4.280002in}{2.335556in}}%
\pgfpathlineto{\pgfqpoint{4.280433in}{2.392244in}}%
\pgfpathlineto{\pgfqpoint{4.280655in}{2.355836in}}%
\pgfpathlineto{\pgfqpoint{4.280532in}{2.402290in}}%
\pgfpathlineto{\pgfqpoint{4.280679in}{2.361449in}}%
\pgfpathlineto{\pgfqpoint{4.281369in}{2.342325in}}%
\pgfpathlineto{\pgfqpoint{4.281135in}{2.408959in}}%
\pgfpathlineto{\pgfqpoint{4.281787in}{2.359008in}}%
\pgfpathlineto{\pgfqpoint{4.282895in}{2.428673in}}%
\pgfpathlineto{\pgfqpoint{4.282095in}{2.333956in}}%
\pgfpathlineto{\pgfqpoint{4.282932in}{2.375153in}}%
\pgfpathlineto{\pgfqpoint{4.283141in}{2.349502in}}%
\pgfpathlineto{\pgfqpoint{4.283965in}{2.403466in}}%
\pgfpathlineto{\pgfqpoint{4.284039in}{2.366192in}}%
\pgfpathlineto{\pgfqpoint{4.284679in}{2.408578in}}%
\pgfpathlineto{\pgfqpoint{4.284187in}{2.330441in}}%
\pgfpathlineto{\pgfqpoint{4.285172in}{2.393082in}}%
\pgfpathlineto{\pgfqpoint{4.285578in}{2.340975in}}%
\pgfpathlineto{\pgfqpoint{4.285849in}{2.412740in}}%
\pgfpathlineto{\pgfqpoint{4.286304in}{2.366071in}}%
\pgfpathlineto{\pgfqpoint{4.287042in}{2.424093in}}%
\pgfpathlineto{\pgfqpoint{4.286981in}{2.346156in}}%
\pgfpathlineto{\pgfqpoint{4.287412in}{2.368025in}}%
\pgfpathlineto{\pgfqpoint{4.287769in}{2.408419in}}%
\pgfpathlineto{\pgfqpoint{4.287682in}{2.347118in}}%
\pgfpathlineto{\pgfqpoint{4.288519in}{2.370578in}}%
\pgfpathlineto{\pgfqpoint{4.289061in}{2.344548in}}%
\pgfpathlineto{\pgfqpoint{4.288815in}{2.403172in}}%
\pgfpathlineto{\pgfqpoint{4.289639in}{2.356427in}}%
\pgfpathlineto{\pgfqpoint{4.290575in}{2.422589in}}%
\pgfpathlineto{\pgfqpoint{4.289775in}{2.343941in}}%
\pgfpathlineto{\pgfqpoint{4.290796in}{2.364777in}}%
\pgfpathlineto{\pgfqpoint{4.291867in}{2.336566in}}%
\pgfpathlineto{\pgfqpoint{4.291769in}{2.406590in}}%
\pgfpathlineto{\pgfqpoint{4.291892in}{2.362279in}}%
\pgfpathlineto{\pgfqpoint{4.292359in}{2.410405in}}%
\pgfpathlineto{\pgfqpoint{4.292901in}{2.350730in}}%
\pgfpathlineto{\pgfqpoint{4.292999in}{2.369307in}}%
\pgfpathlineto{\pgfqpoint{4.294033in}{2.419099in}}%
\pgfpathlineto{\pgfqpoint{4.293947in}{2.350901in}}%
\pgfpathlineto{\pgfqpoint{4.294156in}{2.388120in}}%
\pgfpathlineto{\pgfqpoint{4.294821in}{2.345959in}}%
\pgfpathlineto{\pgfqpoint{4.294735in}{2.423739in}}%
\pgfpathlineto{\pgfqpoint{4.295276in}{2.374260in}}%
\pgfpathlineto{\pgfqpoint{4.295449in}{2.409746in}}%
\pgfpathlineto{\pgfqpoint{4.296015in}{2.346447in}}%
\pgfpathlineto{\pgfqpoint{4.296396in}{2.389598in}}%
\pgfpathlineto{\pgfqpoint{4.296593in}{2.349267in}}%
\pgfpathlineto{\pgfqpoint{4.296975in}{2.410475in}}%
\pgfpathlineto{\pgfqpoint{4.297504in}{2.376308in}}%
\pgfpathlineto{\pgfqpoint{4.297553in}{2.415898in}}%
\pgfpathlineto{\pgfqpoint{4.297787in}{2.348888in}}%
\pgfpathlineto{\pgfqpoint{4.298612in}{2.375524in}}%
\pgfpathlineto{\pgfqpoint{4.299547in}{2.348721in}}%
\pgfpathlineto{\pgfqpoint{4.299461in}{2.415190in}}%
\pgfpathlineto{\pgfqpoint{4.299744in}{2.367123in}}%
\pgfpathlineto{\pgfqpoint{4.300519in}{2.407224in}}%
\pgfpathlineto{\pgfqpoint{4.300236in}{2.349076in}}%
\pgfpathlineto{\pgfqpoint{4.300864in}{2.378069in}}%
\pgfpathlineto{\pgfqpoint{4.301332in}{2.340547in}}%
\pgfpathlineto{\pgfqpoint{4.301713in}{2.420525in}}%
\pgfpathlineto{\pgfqpoint{4.302009in}{2.354880in}}%
\pgfpathlineto{\pgfqpoint{4.302415in}{2.426957in}}%
\pgfpathlineto{\pgfqpoint{4.302722in}{2.347149in}}%
\pgfpathlineto{\pgfqpoint{4.303165in}{2.378721in}}%
\pgfpathlineto{\pgfqpoint{4.303485in}{2.410048in}}%
\pgfpathlineto{\pgfqpoint{4.304285in}{2.345604in}}%
\pgfpathlineto{\pgfqpoint{4.305233in}{2.407221in}}%
\pgfpathlineto{\pgfqpoint{4.305418in}{2.374935in}}%
\pgfpathlineto{\pgfqpoint{4.305467in}{2.340200in}}%
\pgfpathlineto{\pgfqpoint{4.306427in}{2.414009in}}%
\pgfpathlineto{\pgfqpoint{4.306525in}{2.365967in}}%
\pgfpathlineto{\pgfqpoint{4.307141in}{2.409052in}}%
\pgfpathlineto{\pgfqpoint{4.306907in}{2.344775in}}%
\pgfpathlineto{\pgfqpoint{4.307645in}{2.387070in}}%
\pgfpathlineto{\pgfqpoint{4.308285in}{2.348368in}}%
\pgfpathlineto{\pgfqpoint{4.308199in}{2.405719in}}%
\pgfpathlineto{\pgfqpoint{4.308765in}{2.373394in}}%
\pgfpathlineto{\pgfqpoint{4.309393in}{2.427525in}}%
\pgfpathlineto{\pgfqpoint{4.309012in}{2.344100in}}%
\pgfpathlineto{\pgfqpoint{4.309848in}{2.354239in}}%
\pgfpathlineto{\pgfqpoint{4.309861in}{2.353535in}}%
\pgfpathlineto{\pgfqpoint{4.310070in}{2.388926in}}%
\pgfpathlineto{\pgfqpoint{4.310095in}{2.413526in}}%
\pgfpathlineto{\pgfqpoint{4.311067in}{2.346437in}}%
\pgfpathlineto{\pgfqpoint{4.311178in}{2.397273in}}%
\pgfpathlineto{\pgfqpoint{4.311965in}{2.346856in}}%
\pgfpathlineto{\pgfqpoint{4.312199in}{2.399012in}}%
\pgfpathlineto{\pgfqpoint{4.312298in}{2.378243in}}%
\pgfpathlineto{\pgfqpoint{4.312347in}{2.412229in}}%
\pgfpathlineto{\pgfqpoint{4.313147in}{2.347875in}}%
\pgfpathlineto{\pgfqpoint{4.313405in}{2.377192in}}%
\pgfpathlineto{\pgfqpoint{4.314181in}{2.341142in}}%
\pgfpathlineto{\pgfqpoint{4.314107in}{2.408532in}}%
\pgfpathlineto{\pgfqpoint{4.314267in}{2.396323in}}%
\pgfpathlineto{\pgfqpoint{4.314968in}{2.410289in}}%
\pgfpathlineto{\pgfqpoint{4.314759in}{2.346486in}}%
\pgfpathlineto{\pgfqpoint{4.315239in}{2.362200in}}%
\pgfpathlineto{\pgfqpoint{4.315953in}{2.347513in}}%
\pgfpathlineto{\pgfqpoint{4.315682in}{2.407009in}}%
\pgfpathlineto{\pgfqpoint{4.316322in}{2.373915in}}%
\pgfpathlineto{\pgfqpoint{4.317073in}{2.428588in}}%
\pgfpathlineto{\pgfqpoint{4.316679in}{2.344551in}}%
\pgfpathlineto{\pgfqpoint{4.317442in}{2.387115in}}%
\pgfpathlineto{\pgfqpoint{4.318070in}{2.344875in}}%
\pgfpathlineto{\pgfqpoint{4.317775in}{2.418652in}}%
\pgfpathlineto{\pgfqpoint{4.318464in}{2.397020in}}%
\pgfpathlineto{\pgfqpoint{4.318488in}{2.414801in}}%
\pgfpathlineto{\pgfqpoint{4.319473in}{2.342392in}}%
\pgfpathlineto{\pgfqpoint{4.319559in}{2.382599in}}%
\pgfpathlineto{\pgfqpoint{4.319768in}{2.347499in}}%
\pgfpathlineto{\pgfqpoint{4.320593in}{2.419681in}}%
\pgfpathlineto{\pgfqpoint{4.320679in}{2.376601in}}%
\pgfpathlineto{\pgfqpoint{4.321282in}{2.409469in}}%
\pgfpathlineto{\pgfqpoint{4.321024in}{2.342042in}}%
\pgfpathlineto{\pgfqpoint{4.321799in}{2.398726in}}%
\pgfpathlineto{\pgfqpoint{4.322575in}{2.342098in}}%
\pgfpathlineto{\pgfqpoint{4.321959in}{2.405238in}}%
\pgfpathlineto{\pgfqpoint{4.322932in}{2.364429in}}%
\pgfpathlineto{\pgfqpoint{4.323350in}{2.409312in}}%
\pgfpathlineto{\pgfqpoint{4.323965in}{2.341241in}}%
\pgfpathlineto{\pgfqpoint{4.324088in}{2.399216in}}%
\pgfpathlineto{\pgfqpoint{4.325221in}{2.341769in}}%
\pgfpathlineto{\pgfqpoint{4.324741in}{2.412703in}}%
\pgfpathlineto{\pgfqpoint{4.325233in}{2.350950in}}%
\pgfpathlineto{\pgfqpoint{4.326156in}{2.417288in}}%
\pgfpathlineto{\pgfqpoint{4.325368in}{2.339130in}}%
\pgfpathlineto{\pgfqpoint{4.326365in}{2.377631in}}%
\pgfpathlineto{\pgfqpoint{4.326771in}{2.343587in}}%
\pgfpathlineto{\pgfqpoint{4.326870in}{2.421304in}}%
\pgfpathlineto{\pgfqpoint{4.327498in}{2.363383in}}%
\pgfpathlineto{\pgfqpoint{4.328273in}{2.421210in}}%
\pgfpathlineto{\pgfqpoint{4.328162in}{2.329910in}}%
\pgfpathlineto{\pgfqpoint{4.328642in}{2.389372in}}%
\pgfpathlineto{\pgfqpoint{4.328741in}{2.338834in}}%
\pgfpathlineto{\pgfqpoint{4.328962in}{2.412572in}}%
\pgfpathlineto{\pgfqpoint{4.329775in}{2.372407in}}%
\pgfpathlineto{\pgfqpoint{4.329811in}{2.405539in}}%
\pgfpathlineto{\pgfqpoint{4.330242in}{2.342941in}}%
\pgfpathlineto{\pgfqpoint{4.330882in}{2.385924in}}%
\pgfpathlineto{\pgfqpoint{4.330956in}{2.340731in}}%
\pgfpathlineto{\pgfqpoint{4.331768in}{2.403496in}}%
\pgfpathlineto{\pgfqpoint{4.332015in}{2.364132in}}%
\pgfpathlineto{\pgfqpoint{4.333147in}{2.417235in}}%
\pgfpathlineto{\pgfqpoint{4.333048in}{2.345706in}}%
\pgfpathlineto{\pgfqpoint{4.333159in}{2.408583in}}%
\pgfpathlineto{\pgfqpoint{4.333455in}{2.338161in}}%
\pgfpathlineto{\pgfqpoint{4.333861in}{2.408699in}}%
\pgfpathlineto{\pgfqpoint{4.334304in}{2.356377in}}%
\pgfpathlineto{\pgfqpoint{4.334538in}{2.411422in}}%
\pgfpathlineto{\pgfqpoint{4.334636in}{2.346504in}}%
\pgfpathlineto{\pgfqpoint{4.335448in}{2.382703in}}%
\pgfpathlineto{\pgfqpoint{4.336408in}{2.329119in}}%
\pgfpathlineto{\pgfqpoint{4.335941in}{2.407928in}}%
\pgfpathlineto{\pgfqpoint{4.336581in}{2.366952in}}%
\pgfpathlineto{\pgfqpoint{4.337344in}{2.415278in}}%
\pgfpathlineto{\pgfqpoint{4.337233in}{2.340729in}}%
\pgfpathlineto{\pgfqpoint{4.337701in}{2.392133in}}%
\pgfpathlineto{\pgfqpoint{4.338624in}{2.341558in}}%
\pgfpathlineto{\pgfqpoint{4.338058in}{2.403818in}}%
\pgfpathlineto{\pgfqpoint{4.338833in}{2.372206in}}%
\pgfpathlineto{\pgfqpoint{4.339436in}{2.415073in}}%
\pgfpathlineto{\pgfqpoint{4.339350in}{2.318759in}}%
\pgfpathlineto{\pgfqpoint{4.339916in}{2.351067in}}%
\pgfpathlineto{\pgfqpoint{4.339941in}{2.337813in}}%
\pgfpathlineto{\pgfqpoint{4.340285in}{2.415703in}}%
\pgfpathlineto{\pgfqpoint{4.340975in}{2.393849in}}%
\pgfpathlineto{\pgfqpoint{4.341541in}{2.407033in}}%
\pgfpathlineto{\pgfqpoint{4.341430in}{2.337162in}}%
\pgfpathlineto{\pgfqpoint{4.342008in}{2.344211in}}%
\pgfpathlineto{\pgfqpoint{4.342144in}{2.337995in}}%
\pgfpathlineto{\pgfqpoint{4.342378in}{2.408953in}}%
\pgfpathlineto{\pgfqpoint{4.343079in}{2.411113in}}%
\pgfpathlineto{\pgfqpoint{4.342845in}{2.345184in}}%
\pgfpathlineto{\pgfqpoint{4.343239in}{2.374561in}}%
\pgfpathlineto{\pgfqpoint{4.343522in}{2.329523in}}%
\pgfpathlineto{\pgfqpoint{4.343805in}{2.408232in}}%
\pgfpathlineto{\pgfqpoint{4.344310in}{2.388058in}}%
\pgfpathlineto{\pgfqpoint{4.345048in}{2.410170in}}%
\pgfpathlineto{\pgfqpoint{4.344950in}{2.328480in}}%
\pgfpathlineto{\pgfqpoint{4.345405in}{2.383815in}}%
\pgfpathlineto{\pgfqpoint{4.345442in}{2.388679in}}%
\pgfpathlineto{\pgfqpoint{4.345467in}{2.367489in}}%
\pgfpathlineto{\pgfqpoint{4.346341in}{2.316736in}}%
\pgfpathlineto{\pgfqpoint{4.345713in}{2.405309in}}%
\pgfpathlineto{\pgfqpoint{4.346550in}{2.378812in}}%
\pgfpathlineto{\pgfqpoint{4.346599in}{2.411663in}}%
\pgfpathlineto{\pgfqpoint{4.347042in}{2.327027in}}%
\pgfpathlineto{\pgfqpoint{4.347608in}{2.339454in}}%
\pgfpathlineto{\pgfqpoint{4.347633in}{2.319545in}}%
\pgfpathlineto{\pgfqpoint{4.348064in}{2.419662in}}%
\pgfpathlineto{\pgfqpoint{4.348642in}{2.393594in}}%
\pgfpathlineto{\pgfqpoint{4.349245in}{2.427252in}}%
\pgfpathlineto{\pgfqpoint{4.349134in}{2.329340in}}%
\pgfpathlineto{\pgfqpoint{4.349701in}{2.361646in}}%
\pgfpathlineto{\pgfqpoint{4.350525in}{2.325697in}}%
\pgfpathlineto{\pgfqpoint{4.350094in}{2.426928in}}%
\pgfpathlineto{\pgfqpoint{4.350771in}{2.398738in}}%
\pgfpathlineto{\pgfqpoint{4.351522in}{2.423246in}}%
\pgfpathlineto{\pgfqpoint{4.351830in}{2.309587in}}%
\pgfpathlineto{\pgfqpoint{4.352187in}{2.440643in}}%
\pgfpathlineto{\pgfqpoint{4.352999in}{2.373082in}}%
\pgfpathlineto{\pgfqpoint{4.353639in}{2.423325in}}%
\pgfpathlineto{\pgfqpoint{4.353233in}{2.314220in}}%
\pgfpathlineto{\pgfqpoint{4.353898in}{2.370205in}}%
\pgfpathlineto{\pgfqpoint{4.354045in}{2.298765in}}%
\pgfpathlineto{\pgfqpoint{4.354304in}{2.423803in}}%
\pgfpathlineto{\pgfqpoint{4.354968in}{2.410504in}}%
\pgfpathlineto{\pgfqpoint{4.355448in}{2.303884in}}%
\pgfpathlineto{\pgfqpoint{4.355707in}{2.440658in}}%
\pgfpathlineto{\pgfqpoint{4.356199in}{2.356718in}}%
\pgfpathlineto{\pgfqpoint{4.357159in}{2.431906in}}%
\pgfpathlineto{\pgfqpoint{4.356765in}{2.325114in}}%
\pgfpathlineto{\pgfqpoint{4.357319in}{2.382683in}}%
\pgfpathlineto{\pgfqpoint{4.358267in}{2.298316in}}%
\pgfpathlineto{\pgfqpoint{4.357824in}{2.427910in}}%
\pgfpathlineto{\pgfqpoint{4.358427in}{2.384702in}}%
\pgfpathlineto{\pgfqpoint{4.358858in}{2.287762in}}%
\pgfpathlineto{\pgfqpoint{4.359214in}{2.453540in}}%
\pgfpathlineto{\pgfqpoint{4.359596in}{2.335653in}}%
\pgfpathlineto{\pgfqpoint{4.359793in}{2.438432in}}%
\pgfpathlineto{\pgfqpoint{4.359694in}{2.313840in}}%
\pgfpathlineto{\pgfqpoint{4.360765in}{2.391076in}}%
\pgfpathlineto{\pgfqpoint{4.361787in}{2.289681in}}%
\pgfpathlineto{\pgfqpoint{4.361331in}{2.429563in}}%
\pgfpathlineto{\pgfqpoint{4.361848in}{2.407209in}}%
\pgfpathlineto{\pgfqpoint{4.362734in}{2.444222in}}%
\pgfpathlineto{\pgfqpoint{4.362378in}{2.287448in}}%
\pgfpathlineto{\pgfqpoint{4.362882in}{2.359996in}}%
\pgfpathlineto{\pgfqpoint{4.363202in}{2.302826in}}%
\pgfpathlineto{\pgfqpoint{4.363313in}{2.431791in}}%
\pgfpathlineto{\pgfqpoint{4.363941in}{2.360840in}}%
\pgfpathlineto{\pgfqpoint{4.364851in}{2.438444in}}%
\pgfpathlineto{\pgfqpoint{4.364605in}{2.296246in}}%
\pgfpathlineto{\pgfqpoint{4.365061in}{2.385474in}}%
\pgfpathlineto{\pgfqpoint{4.365196in}{2.297299in}}%
\pgfpathlineto{\pgfqpoint{4.365417in}{2.433772in}}%
\pgfpathlineto{\pgfqpoint{4.366107in}{2.420945in}}%
\pgfpathlineto{\pgfqpoint{4.366254in}{2.452481in}}%
\pgfpathlineto{\pgfqpoint{4.366722in}{2.295384in}}%
\pgfpathlineto{\pgfqpoint{4.367116in}{2.380576in}}%
\pgfpathlineto{\pgfqpoint{4.368125in}{2.284372in}}%
\pgfpathlineto{\pgfqpoint{4.367670in}{2.440833in}}%
\pgfpathlineto{\pgfqpoint{4.368187in}{2.388759in}}%
\pgfpathlineto{\pgfqpoint{4.368371in}{2.448112in}}%
\pgfpathlineto{\pgfqpoint{4.368716in}{2.290798in}}%
\pgfpathlineto{\pgfqpoint{4.369270in}{2.351643in}}%
\pgfpathlineto{\pgfqpoint{4.369282in}{2.352103in}}%
\pgfpathlineto{\pgfqpoint{4.369381in}{2.319362in}}%
\pgfpathlineto{\pgfqpoint{4.370242in}{2.293115in}}%
\pgfpathlineto{\pgfqpoint{4.369774in}{2.438047in}}%
\pgfpathlineto{\pgfqpoint{4.370427in}{2.393663in}}%
\pgfpathlineto{\pgfqpoint{4.371190in}{2.441053in}}%
\pgfpathlineto{\pgfqpoint{4.370821in}{2.297409in}}%
\pgfpathlineto{\pgfqpoint{4.371424in}{2.346669in}}%
\pgfpathlineto{\pgfqpoint{4.372347in}{2.288749in}}%
\pgfpathlineto{\pgfqpoint{4.371891in}{2.455736in}}%
\pgfpathlineto{\pgfqpoint{4.372396in}{2.374007in}}%
\pgfpathlineto{\pgfqpoint{4.373294in}{2.452915in}}%
\pgfpathlineto{\pgfqpoint{4.372925in}{2.281219in}}%
\pgfpathlineto{\pgfqpoint{4.373491in}{2.340313in}}%
\pgfpathlineto{\pgfqpoint{4.373861in}{2.442030in}}%
\pgfpathlineto{\pgfqpoint{4.373750in}{2.286464in}}%
\pgfpathlineto{\pgfqpoint{4.374304in}{2.324778in}}%
\pgfpathlineto{\pgfqpoint{4.375030in}{2.280689in}}%
\pgfpathlineto{\pgfqpoint{4.374722in}{2.448562in}}%
\pgfpathlineto{\pgfqpoint{4.375350in}{2.415072in}}%
\pgfpathlineto{\pgfqpoint{4.375399in}{2.456315in}}%
\pgfpathlineto{\pgfqpoint{4.375744in}{2.278044in}}%
\pgfpathlineto{\pgfqpoint{4.376359in}{2.338066in}}%
\pgfpathlineto{\pgfqpoint{4.376433in}{2.267974in}}%
\pgfpathlineto{\pgfqpoint{4.376814in}{2.455583in}}%
\pgfpathlineto{\pgfqpoint{4.377331in}{2.411833in}}%
\pgfpathlineto{\pgfqpoint{4.378082in}{2.490336in}}%
\pgfpathlineto{\pgfqpoint{4.377848in}{2.245417in}}%
\pgfpathlineto{\pgfqpoint{4.378390in}{2.324973in}}%
\pgfpathlineto{\pgfqpoint{4.379239in}{2.271491in}}%
\pgfpathlineto{\pgfqpoint{4.378907in}{2.459931in}}%
\pgfpathlineto{\pgfqpoint{4.379399in}{2.351227in}}%
\pgfpathlineto{\pgfqpoint{4.379461in}{2.492907in}}%
\pgfpathlineto{\pgfqpoint{4.379928in}{2.295959in}}%
\pgfpathlineto{\pgfqpoint{4.380507in}{2.351356in}}%
\pgfpathlineto{\pgfqpoint{4.381565in}{2.455958in}}%
\pgfpathlineto{\pgfqpoint{4.380630in}{2.308073in}}%
\pgfpathlineto{\pgfqpoint{4.381701in}{2.399533in}}%
\pgfpathlineto{\pgfqpoint{4.382008in}{2.305614in}}%
\pgfpathlineto{\pgfqpoint{4.382279in}{2.436020in}}%
\pgfpathlineto{\pgfqpoint{4.382882in}{2.351808in}}%
\pgfpathlineto{\pgfqpoint{4.383042in}{2.435683in}}%
\pgfpathlineto{\pgfqpoint{4.383387in}{2.311561in}}%
\pgfpathlineto{\pgfqpoint{4.383990in}{2.353998in}}%
\pgfpathlineto{\pgfqpoint{4.384421in}{2.413176in}}%
\pgfpathlineto{\pgfqpoint{4.384076in}{2.306672in}}%
\pgfpathlineto{\pgfqpoint{4.385147in}{2.383392in}}%
\pgfpathlineto{\pgfqpoint{4.385479in}{2.336468in}}%
\pgfpathlineto{\pgfqpoint{4.385639in}{2.397266in}}%
\pgfpathlineto{\pgfqpoint{4.386217in}{2.378309in}}%
\pgfpathlineto{\pgfqpoint{4.386784in}{2.399847in}}%
\pgfpathlineto{\pgfqpoint{4.386624in}{2.343941in}}%
\pgfpathlineto{\pgfqpoint{4.387313in}{2.371038in}}%
\pgfpathlineto{\pgfqpoint{4.387547in}{2.357110in}}%
\pgfpathlineto{\pgfqpoint{4.387473in}{2.392038in}}%
\pgfpathlineto{\pgfqpoint{4.388420in}{2.368213in}}%
\pgfpathlineto{\pgfqpoint{4.389159in}{2.396665in}}%
\pgfpathlineto{\pgfqpoint{4.389233in}{2.354501in}}%
\pgfpathlineto{\pgfqpoint{4.389553in}{2.382136in}}%
\pgfpathlineto{\pgfqpoint{4.390414in}{2.425954in}}%
\pgfpathlineto{\pgfqpoint{4.390205in}{2.341479in}}%
\pgfpathlineto{\pgfqpoint{4.390599in}{2.380051in}}%
\pgfpathlineto{\pgfqpoint{4.391325in}{2.323099in}}%
\pgfpathlineto{\pgfqpoint{4.390980in}{2.430186in}}%
\pgfpathlineto{\pgfqpoint{4.391707in}{2.371025in}}%
\pgfpathlineto{\pgfqpoint{4.391879in}{2.337489in}}%
\pgfpathlineto{\pgfqpoint{4.392445in}{2.406275in}}%
\pgfpathlineto{\pgfqpoint{4.392839in}{2.349086in}}%
\pgfpathlineto{\pgfqpoint{4.393147in}{2.428936in}}%
\pgfpathlineto{\pgfqpoint{4.393454in}{2.327762in}}%
\pgfpathlineto{\pgfqpoint{4.393947in}{2.349831in}}%
\pgfpathlineto{\pgfqpoint{4.394882in}{2.299410in}}%
\pgfpathlineto{\pgfqpoint{4.394537in}{2.461305in}}%
\pgfpathlineto{\pgfqpoint{4.395005in}{2.356792in}}%
\pgfpathlineto{\pgfqpoint{4.395042in}{2.421546in}}%
\pgfpathlineto{\pgfqpoint{4.395424in}{2.331010in}}%
\pgfpathlineto{\pgfqpoint{4.396088in}{2.341716in}}%
\pgfpathlineto{\pgfqpoint{4.396974in}{2.311860in}}%
\pgfpathlineto{\pgfqpoint{4.396359in}{2.426095in}}%
\pgfpathlineto{\pgfqpoint{4.397171in}{2.370382in}}%
\pgfpathlineto{\pgfqpoint{4.398070in}{2.473938in}}%
\pgfpathlineto{\pgfqpoint{4.397824in}{2.332259in}}%
\pgfpathlineto{\pgfqpoint{4.398217in}{2.371734in}}%
\pgfpathlineto{\pgfqpoint{4.399067in}{2.312082in}}%
\pgfpathlineto{\pgfqpoint{4.398820in}{2.442361in}}%
\pgfpathlineto{\pgfqpoint{4.399288in}{2.392810in}}%
\pgfpathlineto{\pgfqpoint{4.400137in}{2.447271in}}%
\pgfpathlineto{\pgfqpoint{4.399793in}{2.321670in}}%
\pgfpathlineto{\pgfqpoint{4.400371in}{2.380312in}}%
\pgfpathlineto{\pgfqpoint{4.401171in}{2.322402in}}%
\pgfpathlineto{\pgfqpoint{4.400580in}{2.435072in}}%
\pgfpathlineto{\pgfqpoint{4.401467in}{2.392163in}}%
\pgfpathlineto{\pgfqpoint{4.402427in}{2.321007in}}%
\pgfpathlineto{\pgfqpoint{4.402057in}{2.419864in}}%
\pgfpathlineto{\pgfqpoint{4.402562in}{2.396818in}}%
\pgfpathlineto{\pgfqpoint{4.403091in}{2.427506in}}%
\pgfpathlineto{\pgfqpoint{4.402956in}{2.331022in}}%
\pgfpathlineto{\pgfqpoint{4.403510in}{2.351802in}}%
\pgfpathlineto{\pgfqpoint{4.403522in}{2.351447in}}%
\pgfpathlineto{\pgfqpoint{4.403584in}{2.379407in}}%
\pgfpathlineto{\pgfqpoint{4.403707in}{2.371480in}}%
\pgfpathlineto{\pgfqpoint{4.404334in}{2.432509in}}%
\pgfpathlineto{\pgfqpoint{4.404076in}{2.325679in}}%
\pgfpathlineto{\pgfqpoint{4.404777in}{2.337800in}}%
\pgfpathlineto{\pgfqpoint{4.404913in}{2.320285in}}%
\pgfpathlineto{\pgfqpoint{4.405590in}{2.446220in}}%
\pgfpathlineto{\pgfqpoint{4.405774in}{2.382260in}}%
\pgfpathlineto{\pgfqpoint{4.406402in}{2.455610in}}%
\pgfpathlineto{\pgfqpoint{4.406057in}{2.309362in}}%
\pgfpathlineto{\pgfqpoint{4.406882in}{2.379104in}}%
\pgfpathlineto{\pgfqpoint{4.407436in}{2.307480in}}%
\pgfpathlineto{\pgfqpoint{4.407645in}{2.445844in}}%
\pgfpathlineto{\pgfqpoint{4.408027in}{2.342860in}}%
\pgfpathlineto{\pgfqpoint{4.408088in}{2.432117in}}%
\pgfpathlineto{\pgfqpoint{4.408679in}{2.330468in}}%
\pgfpathlineto{\pgfqpoint{4.409159in}{2.371566in}}%
\pgfpathlineto{\pgfqpoint{4.409208in}{2.362068in}}%
\pgfpathlineto{\pgfqpoint{4.409319in}{2.438841in}}%
\pgfpathlineto{\pgfqpoint{4.409331in}{2.451007in}}%
\pgfpathlineto{\pgfqpoint{4.409491in}{2.324467in}}%
\pgfpathlineto{\pgfqpoint{4.410303in}{2.327033in}}%
\pgfpathlineto{\pgfqpoint{4.410328in}{2.318869in}}%
\pgfpathlineto{\pgfqpoint{4.410587in}{2.453115in}}%
\pgfpathlineto{\pgfqpoint{4.411288in}{2.392084in}}%
\pgfpathlineto{\pgfqpoint{4.411325in}{2.432571in}}%
\pgfpathlineto{\pgfqpoint{4.411707in}{2.325941in}}%
\pgfpathlineto{\pgfqpoint{4.412359in}{2.352911in}}%
\pgfpathlineto{\pgfqpoint{4.412433in}{2.321461in}}%
\pgfpathlineto{\pgfqpoint{4.412642in}{2.434803in}}%
\pgfpathlineto{\pgfqpoint{4.413282in}{2.373068in}}%
\pgfpathlineto{\pgfqpoint{4.413897in}{2.433431in}}%
\pgfpathlineto{\pgfqpoint{4.413688in}{2.309161in}}%
\pgfpathlineto{\pgfqpoint{4.414390in}{2.385380in}}%
\pgfpathlineto{\pgfqpoint{4.415350in}{2.324119in}}%
\pgfpathlineto{\pgfqpoint{4.415153in}{2.441569in}}%
\pgfpathlineto{\pgfqpoint{4.415510in}{2.361194in}}%
\pgfpathlineto{\pgfqpoint{4.416396in}{2.455502in}}%
\pgfpathlineto{\pgfqpoint{4.416199in}{2.314495in}}%
\pgfpathlineto{\pgfqpoint{4.416593in}{2.336248in}}%
\pgfpathlineto{\pgfqpoint{4.417454in}{2.305014in}}%
\pgfpathlineto{\pgfqpoint{4.417085in}{2.426953in}}%
\pgfpathlineto{\pgfqpoint{4.417503in}{2.378568in}}%
\pgfpathlineto{\pgfqpoint{4.417651in}{2.467803in}}%
\pgfpathlineto{\pgfqpoint{4.417971in}{2.321008in}}%
\pgfpathlineto{\pgfqpoint{4.418537in}{2.365909in}}%
\pgfpathlineto{\pgfqpoint{4.418697in}{2.308965in}}%
\pgfpathlineto{\pgfqpoint{4.418919in}{2.457878in}}%
\pgfpathlineto{\pgfqpoint{4.419596in}{2.427717in}}%
\pgfpathlineto{\pgfqpoint{4.419953in}{2.305012in}}%
\pgfpathlineto{\pgfqpoint{4.420162in}{2.448007in}}%
\pgfpathlineto{\pgfqpoint{4.420790in}{2.368943in}}%
\pgfpathlineto{\pgfqpoint{4.421417in}{2.449755in}}%
\pgfpathlineto{\pgfqpoint{4.421602in}{2.320216in}}%
\pgfpathlineto{\pgfqpoint{4.421910in}{2.393518in}}%
\pgfpathlineto{\pgfqpoint{4.422439in}{2.309491in}}%
\pgfpathlineto{\pgfqpoint{4.422673in}{2.454660in}}%
\pgfpathlineto{\pgfqpoint{4.423067in}{2.355866in}}%
\pgfpathlineto{\pgfqpoint{4.423928in}{2.456028in}}%
\pgfpathlineto{\pgfqpoint{4.423719in}{2.299257in}}%
\pgfpathlineto{\pgfqpoint{4.424174in}{2.352998in}}%
\pgfpathlineto{\pgfqpoint{4.424593in}{2.442844in}}%
\pgfpathlineto{\pgfqpoint{4.424839in}{2.324737in}}%
\pgfpathlineto{\pgfqpoint{4.424962in}{2.297597in}}%
\pgfpathlineto{\pgfqpoint{4.425171in}{2.457701in}}%
\pgfpathlineto{\pgfqpoint{4.425614in}{2.417035in}}%
\pgfpathlineto{\pgfqpoint{4.425860in}{2.435220in}}%
\pgfpathlineto{\pgfqpoint{4.425787in}{2.350803in}}%
\pgfpathlineto{\pgfqpoint{4.426057in}{2.362998in}}%
\pgfpathlineto{\pgfqpoint{4.426217in}{2.308990in}}%
\pgfpathlineto{\pgfqpoint{4.426427in}{2.460999in}}%
\pgfpathlineto{\pgfqpoint{4.427103in}{2.428519in}}%
\pgfpathlineto{\pgfqpoint{4.427460in}{2.299122in}}%
\pgfpathlineto{\pgfqpoint{4.427670in}{2.459782in}}%
\pgfpathlineto{\pgfqpoint{4.428334in}{2.406753in}}%
\pgfpathlineto{\pgfqpoint{4.428925in}{2.470088in}}%
\pgfpathlineto{\pgfqpoint{4.428716in}{2.303039in}}%
\pgfpathlineto{\pgfqpoint{4.429430in}{2.397678in}}%
\pgfpathlineto{\pgfqpoint{4.429971in}{2.294041in}}%
\pgfpathlineto{\pgfqpoint{4.430180in}{2.470765in}}%
\pgfpathlineto{\pgfqpoint{4.430586in}{2.380113in}}%
\pgfpathlineto{\pgfqpoint{4.431423in}{2.448102in}}%
\pgfpathlineto{\pgfqpoint{4.431226in}{2.298249in}}%
\pgfpathlineto{\pgfqpoint{4.431682in}{2.354688in}}%
\pgfpathlineto{\pgfqpoint{4.431743in}{2.332382in}}%
\pgfpathlineto{\pgfqpoint{4.431866in}{2.437670in}}%
\pgfpathlineto{\pgfqpoint{4.432679in}{2.450286in}}%
\pgfpathlineto{\pgfqpoint{4.432470in}{2.308112in}}%
\pgfpathlineto{\pgfqpoint{4.432851in}{2.339038in}}%
\pgfpathlineto{\pgfqpoint{4.433713in}{2.295975in}}%
\pgfpathlineto{\pgfqpoint{4.433134in}{2.443510in}}%
\pgfpathlineto{\pgfqpoint{4.433848in}{2.426853in}}%
\pgfpathlineto{\pgfqpoint{4.433922in}{2.477985in}}%
\pgfpathlineto{\pgfqpoint{4.434254in}{2.306035in}}%
\pgfpathlineto{\pgfqpoint{4.434882in}{2.341871in}}%
\pgfpathlineto{\pgfqpoint{4.434956in}{2.291886in}}%
\pgfpathlineto{\pgfqpoint{4.435177in}{2.478483in}}%
\pgfpathlineto{\pgfqpoint{4.435830in}{2.401407in}}%
\pgfpathlineto{\pgfqpoint{4.436433in}{2.463203in}}%
\pgfpathlineto{\pgfqpoint{4.436211in}{2.288026in}}%
\pgfpathlineto{\pgfqpoint{4.436937in}{2.400338in}}%
\pgfpathlineto{\pgfqpoint{4.437466in}{2.305961in}}%
\pgfpathlineto{\pgfqpoint{4.437676in}{2.465036in}}%
\pgfpathlineto{\pgfqpoint{4.438082in}{2.381172in}}%
\pgfpathlineto{\pgfqpoint{4.438931in}{2.470046in}}%
\pgfpathlineto{\pgfqpoint{4.438710in}{2.301697in}}%
\pgfpathlineto{\pgfqpoint{4.439165in}{2.351007in}}%
\pgfpathlineto{\pgfqpoint{4.440174in}{2.483453in}}%
\pgfpathlineto{\pgfqpoint{4.439965in}{2.296715in}}%
\pgfpathlineto{\pgfqpoint{4.440334in}{2.367204in}}%
\pgfpathlineto{\pgfqpoint{4.441208in}{2.291203in}}%
\pgfpathlineto{\pgfqpoint{4.440876in}{2.438434in}}%
\pgfpathlineto{\pgfqpoint{4.441393in}{2.435464in}}%
\pgfpathlineto{\pgfqpoint{4.441430in}{2.482867in}}%
\pgfpathlineto{\pgfqpoint{4.441762in}{2.305099in}}%
\pgfpathlineto{\pgfqpoint{4.442402in}{2.343698in}}%
\pgfpathlineto{\pgfqpoint{4.442463in}{2.284649in}}%
\pgfpathlineto{\pgfqpoint{4.442685in}{2.472263in}}%
\pgfpathlineto{\pgfqpoint{4.443325in}{2.392659in}}%
\pgfpathlineto{\pgfqpoint{4.443928in}{2.459629in}}%
\pgfpathlineto{\pgfqpoint{4.443719in}{2.294875in}}%
\pgfpathlineto{\pgfqpoint{4.444433in}{2.394032in}}%
\pgfpathlineto{\pgfqpoint{4.444457in}{2.395430in}}%
\pgfpathlineto{\pgfqpoint{4.444519in}{2.346838in}}%
\pgfpathlineto{\pgfqpoint{4.444962in}{2.305753in}}%
\pgfpathlineto{\pgfqpoint{4.445183in}{2.464017in}}%
\pgfpathlineto{\pgfqpoint{4.445577in}{2.369746in}}%
\pgfpathlineto{\pgfqpoint{4.446439in}{2.466830in}}%
\pgfpathlineto{\pgfqpoint{4.446106in}{2.309373in}}%
\pgfpathlineto{\pgfqpoint{4.446673in}{2.348293in}}%
\pgfpathlineto{\pgfqpoint{4.447103in}{2.436708in}}%
\pgfpathlineto{\pgfqpoint{4.446759in}{2.311499in}}%
\pgfpathlineto{\pgfqpoint{4.447423in}{2.321605in}}%
\pgfpathlineto{\pgfqpoint{4.447460in}{2.295834in}}%
\pgfpathlineto{\pgfqpoint{4.447694in}{2.466525in}}%
\pgfpathlineto{\pgfqpoint{4.448396in}{2.418729in}}%
\pgfpathlineto{\pgfqpoint{4.448716in}{2.294576in}}%
\pgfpathlineto{\pgfqpoint{4.448937in}{2.462298in}}%
\pgfpathlineto{\pgfqpoint{4.449565in}{2.373648in}}%
\pgfpathlineto{\pgfqpoint{4.450193in}{2.436457in}}%
\pgfpathlineto{\pgfqpoint{4.449959in}{2.311319in}}%
\pgfpathlineto{\pgfqpoint{4.450685in}{2.396292in}}%
\pgfpathlineto{\pgfqpoint{4.451153in}{2.317265in}}%
\pgfpathlineto{\pgfqpoint{4.451448in}{2.446318in}}%
\pgfpathlineto{\pgfqpoint{4.451829in}{2.377022in}}%
\pgfpathlineto{\pgfqpoint{4.452666in}{2.452013in}}%
\pgfpathlineto{\pgfqpoint{4.452457in}{2.308756in}}%
\pgfpathlineto{\pgfqpoint{4.452925in}{2.366736in}}%
\pgfpathlineto{\pgfqpoint{4.453011in}{2.302761in}}%
\pgfpathlineto{\pgfqpoint{4.453897in}{2.441044in}}%
\pgfpathlineto{\pgfqpoint{4.453922in}{2.458894in}}%
\pgfpathlineto{\pgfqpoint{4.454869in}{2.309870in}}%
\pgfpathlineto{\pgfqpoint{4.454894in}{2.317019in}}%
\pgfpathlineto{\pgfqpoint{4.454943in}{2.299504in}}%
\pgfpathlineto{\pgfqpoint{4.455177in}{2.452456in}}%
\pgfpathlineto{\pgfqpoint{4.455682in}{2.393134in}}%
\pgfpathlineto{\pgfqpoint{4.456408in}{2.437638in}}%
\pgfpathlineto{\pgfqpoint{4.456186in}{2.290340in}}%
\pgfpathlineto{\pgfqpoint{4.456703in}{2.336448in}}%
\pgfpathlineto{\pgfqpoint{4.457466in}{2.298369in}}%
\pgfpathlineto{\pgfqpoint{4.457097in}{2.429954in}}%
\pgfpathlineto{\pgfqpoint{4.457639in}{2.426373in}}%
\pgfpathlineto{\pgfqpoint{4.457676in}{2.450449in}}%
\pgfpathlineto{\pgfqpoint{4.458008in}{2.312495in}}%
\pgfpathlineto{\pgfqpoint{4.458660in}{2.319683in}}%
\pgfpathlineto{\pgfqpoint{4.458685in}{2.303730in}}%
\pgfpathlineto{\pgfqpoint{4.458919in}{2.452480in}}%
\pgfpathlineto{\pgfqpoint{4.459694in}{2.384888in}}%
\pgfpathlineto{\pgfqpoint{4.460211in}{2.433138in}}%
\pgfpathlineto{\pgfqpoint{4.459916in}{2.302560in}}%
\pgfpathlineto{\pgfqpoint{4.460433in}{2.333889in}}%
\pgfpathlineto{\pgfqpoint{4.461208in}{2.299538in}}%
\pgfpathlineto{\pgfqpoint{4.460839in}{2.432806in}}%
\pgfpathlineto{\pgfqpoint{4.461319in}{2.398050in}}%
\pgfpathlineto{\pgfqpoint{4.461405in}{2.448506in}}%
\pgfpathlineto{\pgfqpoint{4.461749in}{2.308632in}}%
\pgfpathlineto{\pgfqpoint{4.462316in}{2.312588in}}%
\pgfpathlineto{\pgfqpoint{4.462414in}{2.305000in}}%
\pgfpathlineto{\pgfqpoint{4.462599in}{2.406744in}}%
\pgfpathlineto{\pgfqpoint{4.462685in}{2.469017in}}%
\pgfpathlineto{\pgfqpoint{4.462956in}{2.323882in}}%
\pgfpathlineto{\pgfqpoint{4.463608in}{2.333020in}}%
\pgfpathlineto{\pgfqpoint{4.463682in}{2.292043in}}%
\pgfpathlineto{\pgfqpoint{4.463805in}{2.376802in}}%
\pgfpathlineto{\pgfqpoint{4.463940in}{2.442187in}}%
\pgfpathlineto{\pgfqpoint{4.464211in}{2.325469in}}%
\pgfpathlineto{\pgfqpoint{4.464876in}{2.351375in}}%
\pgfpathlineto{\pgfqpoint{4.464937in}{2.292529in}}%
\pgfpathlineto{\pgfqpoint{4.465183in}{2.448137in}}%
\pgfpathlineto{\pgfqpoint{4.465823in}{2.422439in}}%
\pgfpathlineto{\pgfqpoint{4.465836in}{2.423924in}}%
\pgfpathlineto{\pgfqpoint{4.466106in}{2.330111in}}%
\pgfpathlineto{\pgfqpoint{4.466168in}{2.339231in}}%
\pgfpathlineto{\pgfqpoint{4.466709in}{2.321902in}}%
\pgfpathlineto{\pgfqpoint{4.466451in}{2.458776in}}%
\pgfpathlineto{\pgfqpoint{4.467239in}{2.371387in}}%
\pgfpathlineto{\pgfqpoint{4.467349in}{2.328222in}}%
\pgfpathlineto{\pgfqpoint{4.467719in}{2.425499in}}%
\pgfpathlineto{\pgfqpoint{4.468260in}{2.393889in}}%
\pgfpathlineto{\pgfqpoint{4.468334in}{2.428770in}}%
\pgfpathlineto{\pgfqpoint{4.468568in}{2.331984in}}%
\pgfpathlineto{\pgfqpoint{4.468605in}{2.301241in}}%
\pgfpathlineto{\pgfqpoint{4.468962in}{2.433093in}}%
\pgfpathlineto{\pgfqpoint{4.469614in}{2.399792in}}%
\pgfpathlineto{\pgfqpoint{4.470685in}{2.313169in}}%
\pgfpathlineto{\pgfqpoint{4.470205in}{2.418964in}}%
\pgfpathlineto{\pgfqpoint{4.470771in}{2.387565in}}%
\pgfpathlineto{\pgfqpoint{4.470857in}{2.424899in}}%
\pgfpathlineto{\pgfqpoint{4.471189in}{2.341353in}}%
\pgfpathlineto{\pgfqpoint{4.471497in}{2.349397in}}%
\pgfpathlineto{\pgfqpoint{4.471522in}{2.326939in}}%
\pgfpathlineto{\pgfqpoint{4.472100in}{2.417443in}}%
\pgfpathlineto{\pgfqpoint{4.472592in}{2.355821in}}%
\pgfpathlineto{\pgfqpoint{4.473479in}{2.421716in}}%
\pgfpathlineto{\pgfqpoint{4.473196in}{2.315622in}}%
\pgfpathlineto{\pgfqpoint{4.473602in}{2.332620in}}%
\pgfpathlineto{\pgfqpoint{4.473626in}{2.312130in}}%
\pgfpathlineto{\pgfqpoint{4.473959in}{2.412436in}}%
\pgfpathlineto{\pgfqpoint{4.474697in}{2.335018in}}%
\pgfpathlineto{\pgfqpoint{4.474820in}{2.427796in}}%
\pgfpathlineto{\pgfqpoint{4.475300in}{2.315955in}}%
\pgfpathlineto{\pgfqpoint{4.475817in}{2.364307in}}%
\pgfpathlineto{\pgfqpoint{4.475891in}{2.421610in}}%
\pgfpathlineto{\pgfqpoint{4.476125in}{2.320124in}}%
\pgfpathlineto{\pgfqpoint{4.476925in}{2.373289in}}%
\pgfpathlineto{\pgfqpoint{4.476949in}{2.379162in}}%
\pgfpathlineto{\pgfqpoint{4.477614in}{2.414534in}}%
\pgfpathlineto{\pgfqpoint{4.477220in}{2.341323in}}%
\pgfpathlineto{\pgfqpoint{4.478045in}{2.382433in}}%
\pgfpathlineto{\pgfqpoint{4.478599in}{2.313954in}}%
\pgfpathlineto{\pgfqpoint{4.478956in}{2.425125in}}%
\pgfpathlineto{\pgfqpoint{4.479165in}{2.365242in}}%
\pgfpathlineto{\pgfqpoint{4.479657in}{2.422167in}}%
\pgfpathlineto{\pgfqpoint{4.479916in}{2.326941in}}%
\pgfpathlineto{\pgfqpoint{4.480272in}{2.365341in}}%
\pgfpathlineto{\pgfqpoint{4.481072in}{2.312332in}}%
\pgfpathlineto{\pgfqpoint{4.480777in}{2.421066in}}%
\pgfpathlineto{\pgfqpoint{4.481368in}{2.375562in}}%
\pgfpathlineto{\pgfqpoint{4.481429in}{2.450140in}}%
\pgfpathlineto{\pgfqpoint{4.482389in}{2.321073in}}%
\pgfpathlineto{\pgfqpoint{4.482562in}{2.430657in}}%
\pgfpathlineto{\pgfqpoint{4.482648in}{2.412461in}}%
\pgfpathlineto{\pgfqpoint{4.482672in}{2.462698in}}%
\pgfpathlineto{\pgfqpoint{4.483546in}{2.323751in}}%
\pgfpathlineto{\pgfqpoint{4.483743in}{2.374802in}}%
\pgfpathlineto{\pgfqpoint{4.483903in}{2.471079in}}%
\pgfpathlineto{\pgfqpoint{4.484076in}{2.310519in}}%
\pgfpathlineto{\pgfqpoint{4.484703in}{2.372459in}}%
\pgfpathlineto{\pgfqpoint{4.485405in}{2.298716in}}%
\pgfpathlineto{\pgfqpoint{4.485146in}{2.450717in}}%
\pgfpathlineto{\pgfqpoint{4.485774in}{2.430984in}}%
\pgfpathlineto{\pgfqpoint{4.486648in}{2.316964in}}%
\pgfpathlineto{\pgfqpoint{4.486266in}{2.440023in}}%
\pgfpathlineto{\pgfqpoint{4.486980in}{2.369179in}}%
\pgfpathlineto{\pgfqpoint{4.487620in}{2.449647in}}%
\pgfpathlineto{\pgfqpoint{4.487756in}{2.320444in}}%
\pgfpathlineto{\pgfqpoint{4.488088in}{2.378610in}}%
\pgfpathlineto{\pgfqpoint{4.488863in}{2.423687in}}%
\pgfpathlineto{\pgfqpoint{4.489023in}{2.311059in}}%
\pgfpathlineto{\pgfqpoint{4.489995in}{2.443459in}}%
\pgfpathlineto{\pgfqpoint{4.489602in}{2.302891in}}%
\pgfpathlineto{\pgfqpoint{4.490254in}{2.333448in}}%
\pgfpathlineto{\pgfqpoint{4.490365in}{2.304417in}}%
\pgfpathlineto{\pgfqpoint{4.490734in}{2.425665in}}%
\pgfpathlineto{\pgfqpoint{4.491103in}{2.369250in}}%
\pgfpathlineto{\pgfqpoint{4.491768in}{2.463492in}}%
\pgfpathlineto{\pgfqpoint{4.492075in}{2.313795in}}%
\pgfpathlineto{\pgfqpoint{4.492162in}{2.327627in}}%
\pgfpathlineto{\pgfqpoint{4.492186in}{2.334024in}}%
\pgfpathlineto{\pgfqpoint{4.492568in}{2.454420in}}%
\pgfpathlineto{\pgfqpoint{4.492765in}{2.301768in}}%
\pgfpathlineto{\pgfqpoint{4.493306in}{2.349419in}}%
\pgfpathlineto{\pgfqpoint{4.494242in}{2.445995in}}%
\pgfpathlineto{\pgfqpoint{4.494069in}{2.292249in}}%
\pgfpathlineto{\pgfqpoint{4.494500in}{2.386898in}}%
\pgfpathlineto{\pgfqpoint{4.495312in}{2.296356in}}%
\pgfpathlineto{\pgfqpoint{4.494931in}{2.450081in}}%
\pgfpathlineto{\pgfqpoint{4.495620in}{2.360035in}}%
\pgfpathlineto{\pgfqpoint{4.496309in}{2.449845in}}%
\pgfpathlineto{\pgfqpoint{4.495866in}{2.299712in}}%
\pgfpathlineto{\pgfqpoint{4.496802in}{2.418326in}}%
\pgfpathlineto{\pgfqpoint{4.497712in}{2.311448in}}%
\pgfpathlineto{\pgfqpoint{4.497540in}{2.422697in}}%
\pgfpathlineto{\pgfqpoint{4.498008in}{2.398635in}}%
\pgfpathlineto{\pgfqpoint{4.498155in}{2.433302in}}%
\pgfpathlineto{\pgfqpoint{4.498352in}{2.314829in}}%
\pgfpathlineto{\pgfqpoint{4.499017in}{2.337856in}}%
\pgfpathlineto{\pgfqpoint{4.499583in}{2.307191in}}%
\pgfpathlineto{\pgfqpoint{4.500014in}{2.429867in}}%
\pgfpathlineto{\pgfqpoint{4.500075in}{2.393929in}}%
\pgfpathlineto{\pgfqpoint{4.500826in}{2.314842in}}%
\pgfpathlineto{\pgfqpoint{4.500445in}{2.443177in}}%
\pgfpathlineto{\pgfqpoint{4.501195in}{2.370803in}}%
\pgfpathlineto{\pgfqpoint{4.501749in}{2.437228in}}%
\pgfpathlineto{\pgfqpoint{4.501515in}{2.314576in}}%
\pgfpathlineto{\pgfqpoint{4.502303in}{2.392830in}}%
\pgfpathlineto{\pgfqpoint{4.503288in}{2.312671in}}%
\pgfpathlineto{\pgfqpoint{4.502919in}{2.426401in}}%
\pgfpathlineto{\pgfqpoint{4.503435in}{2.373666in}}%
\pgfpathlineto{\pgfqpoint{4.504162in}{2.431075in}}%
\pgfpathlineto{\pgfqpoint{4.504014in}{2.323103in}}%
\pgfpathlineto{\pgfqpoint{4.504445in}{2.357388in}}%
\pgfpathlineto{\pgfqpoint{4.504531in}{2.326020in}}%
\pgfpathlineto{\pgfqpoint{4.504974in}{2.423996in}}%
\pgfpathlineto{\pgfqpoint{4.505442in}{2.401416in}}%
\pgfpathlineto{\pgfqpoint{4.506020in}{2.421926in}}%
\pgfpathlineto{\pgfqpoint{4.505835in}{2.334189in}}%
\pgfpathlineto{\pgfqpoint{4.506463in}{2.356433in}}%
\pgfpathlineto{\pgfqpoint{4.506500in}{2.352123in}}%
\pgfpathlineto{\pgfqpoint{4.506512in}{2.361542in}}%
\pgfpathlineto{\pgfqpoint{4.506709in}{2.427165in}}%
\pgfpathlineto{\pgfqpoint{4.506931in}{2.318627in}}%
\pgfpathlineto{\pgfqpoint{4.507595in}{2.360086in}}%
\pgfpathlineto{\pgfqpoint{4.508235in}{2.317167in}}%
\pgfpathlineto{\pgfqpoint{4.508075in}{2.429351in}}%
\pgfpathlineto{\pgfqpoint{4.508654in}{2.380909in}}%
\pgfpathlineto{\pgfqpoint{4.509712in}{2.424569in}}%
\pgfpathlineto{\pgfqpoint{4.509491in}{2.323320in}}%
\pgfpathlineto{\pgfqpoint{4.509762in}{2.377404in}}%
\pgfpathlineto{\pgfqpoint{4.509774in}{2.377485in}}%
\pgfpathlineto{\pgfqpoint{4.510426in}{2.433940in}}%
\pgfpathlineto{\pgfqpoint{4.510722in}{2.320930in}}%
\pgfpathlineto{\pgfqpoint{4.510832in}{2.342663in}}%
\pgfpathlineto{\pgfqpoint{4.511349in}{2.333774in}}%
\pgfpathlineto{\pgfqpoint{4.511152in}{2.412426in}}%
\pgfpathlineto{\pgfqpoint{4.511632in}{2.390112in}}%
\pgfpathlineto{\pgfqpoint{4.512223in}{2.437004in}}%
\pgfpathlineto{\pgfqpoint{4.511965in}{2.328748in}}%
\pgfpathlineto{\pgfqpoint{4.512703in}{2.356093in}}%
\pgfpathlineto{\pgfqpoint{4.513195in}{2.330106in}}%
\pgfpathlineto{\pgfqpoint{4.513023in}{2.459694in}}%
\pgfpathlineto{\pgfqpoint{4.513503in}{2.399200in}}%
\pgfpathlineto{\pgfqpoint{4.514266in}{2.429231in}}%
\pgfpathlineto{\pgfqpoint{4.513983in}{2.331789in}}%
\pgfpathlineto{\pgfqpoint{4.514488in}{2.355615in}}%
\pgfpathlineto{\pgfqpoint{4.514512in}{2.336628in}}%
\pgfpathlineto{\pgfqpoint{4.515374in}{2.429392in}}%
\pgfpathlineto{\pgfqpoint{4.515558in}{2.382054in}}%
\pgfpathlineto{\pgfqpoint{4.515780in}{2.323669in}}%
\pgfpathlineto{\pgfqpoint{4.516137in}{2.432802in}}%
\pgfpathlineto{\pgfqpoint{4.516617in}{2.388333in}}%
\pgfpathlineto{\pgfqpoint{4.516740in}{2.439039in}}%
\pgfpathlineto{\pgfqpoint{4.517085in}{2.334146in}}%
\pgfpathlineto{\pgfqpoint{4.517602in}{2.370934in}}%
\pgfpathlineto{\pgfqpoint{4.517651in}{2.337873in}}%
\pgfpathlineto{\pgfqpoint{4.517983in}{2.425982in}}%
\pgfpathlineto{\pgfqpoint{4.518709in}{2.352409in}}%
\pgfpathlineto{\pgfqpoint{4.519842in}{2.425327in}}%
\pgfpathlineto{\pgfqpoint{4.518906in}{2.339261in}}%
\pgfpathlineto{\pgfqpoint{4.519866in}{2.408262in}}%
\pgfpathlineto{\pgfqpoint{4.520752in}{2.334925in}}%
\pgfpathlineto{\pgfqpoint{4.520063in}{2.420406in}}%
\pgfpathlineto{\pgfqpoint{4.521011in}{2.354551in}}%
\pgfpathlineto{\pgfqpoint{4.521085in}{2.421415in}}%
\pgfpathlineto{\pgfqpoint{4.521811in}{2.337272in}}%
\pgfpathlineto{\pgfqpoint{4.522217in}{2.408932in}}%
\pgfpathlineto{\pgfqpoint{4.523263in}{2.341563in}}%
\pgfpathlineto{\pgfqpoint{4.522943in}{2.427360in}}%
\pgfpathlineto{\pgfqpoint{4.523337in}{2.374641in}}%
\pgfpathlineto{\pgfqpoint{4.523977in}{2.425220in}}%
\pgfpathlineto{\pgfqpoint{4.523878in}{2.334676in}}%
\pgfpathlineto{\pgfqpoint{4.524420in}{2.362344in}}%
\pgfpathlineto{\pgfqpoint{4.524912in}{2.327617in}}%
\pgfpathlineto{\pgfqpoint{4.524789in}{2.417007in}}%
\pgfpathlineto{\pgfqpoint{4.525515in}{2.376177in}}%
\pgfpathlineto{\pgfqpoint{4.526217in}{2.322503in}}%
\pgfpathlineto{\pgfqpoint{4.526045in}{2.418610in}}%
\pgfpathlineto{\pgfqpoint{4.526438in}{2.391526in}}%
\pgfpathlineto{\pgfqpoint{4.526660in}{2.435098in}}%
\pgfpathlineto{\pgfqpoint{4.527005in}{2.335892in}}%
\pgfpathlineto{\pgfqpoint{4.527435in}{2.360526in}}%
\pgfpathlineto{\pgfqpoint{4.528014in}{2.335783in}}%
\pgfpathlineto{\pgfqpoint{4.527891in}{2.429255in}}%
\pgfpathlineto{\pgfqpoint{4.528482in}{2.410855in}}%
\pgfpathlineto{\pgfqpoint{4.528518in}{2.431450in}}%
\pgfpathlineto{\pgfqpoint{4.528826in}{2.339612in}}%
\pgfpathlineto{\pgfqpoint{4.529565in}{2.396359in}}%
\pgfpathlineto{\pgfqpoint{4.529934in}{2.331364in}}%
\pgfpathlineto{\pgfqpoint{4.529762in}{2.434333in}}%
\pgfpathlineto{\pgfqpoint{4.530721in}{2.351228in}}%
\pgfpathlineto{\pgfqpoint{4.531005in}{2.431083in}}%
\pgfpathlineto{\pgfqpoint{4.531177in}{2.326367in}}%
\pgfpathlineto{\pgfqpoint{4.531854in}{2.370037in}}%
\pgfpathlineto{\pgfqpoint{4.532518in}{2.328536in}}%
\pgfpathlineto{\pgfqpoint{4.532863in}{2.431163in}}%
\pgfpathlineto{\pgfqpoint{4.532974in}{2.343144in}}%
\pgfpathlineto{\pgfqpoint{4.533971in}{2.423098in}}%
\pgfpathlineto{\pgfqpoint{4.533577in}{2.316765in}}%
\pgfpathlineto{\pgfqpoint{4.534155in}{2.383555in}}%
\pgfpathlineto{\pgfqpoint{4.534377in}{2.340539in}}%
\pgfpathlineto{\pgfqpoint{4.534672in}{2.408688in}}%
\pgfpathlineto{\pgfqpoint{4.534709in}{2.443401in}}%
\pgfpathlineto{\pgfqpoint{4.534894in}{2.306701in}}%
\pgfpathlineto{\pgfqpoint{4.535768in}{2.410411in}}%
\pgfpathlineto{\pgfqpoint{4.536137in}{2.325215in}}%
\pgfpathlineto{\pgfqpoint{4.536358in}{2.419765in}}%
\pgfpathlineto{\pgfqpoint{4.536900in}{2.366258in}}%
\pgfpathlineto{\pgfqpoint{4.537589in}{2.428947in}}%
\pgfpathlineto{\pgfqpoint{4.537380in}{2.321902in}}%
\pgfpathlineto{\pgfqpoint{4.537897in}{2.361071in}}%
\pgfpathlineto{\pgfqpoint{4.538537in}{2.315744in}}%
\pgfpathlineto{\pgfqpoint{4.538918in}{2.421091in}}%
\pgfpathlineto{\pgfqpoint{4.538980in}{2.374350in}}%
\pgfpathlineto{\pgfqpoint{4.539669in}{2.429234in}}%
\pgfpathlineto{\pgfqpoint{4.539226in}{2.328319in}}%
\pgfpathlineto{\pgfqpoint{4.540088in}{2.387204in}}%
\pgfpathlineto{\pgfqpoint{4.540395in}{2.321405in}}%
\pgfpathlineto{\pgfqpoint{4.540285in}{2.427212in}}%
\pgfpathlineto{\pgfqpoint{4.541232in}{2.350549in}}%
\pgfpathlineto{\pgfqpoint{4.541515in}{2.425750in}}%
\pgfpathlineto{\pgfqpoint{4.541700in}{2.324877in}}%
\pgfpathlineto{\pgfqpoint{4.542328in}{2.345224in}}%
\pgfpathlineto{\pgfqpoint{4.542943in}{2.331199in}}%
\pgfpathlineto{\pgfqpoint{4.542648in}{2.422455in}}%
\pgfpathlineto{\pgfqpoint{4.543238in}{2.381563in}}%
\pgfpathlineto{\pgfqpoint{4.543374in}{2.421828in}}%
\pgfpathlineto{\pgfqpoint{4.543571in}{2.323697in}}%
\pgfpathlineto{\pgfqpoint{4.544297in}{2.348969in}}%
\pgfpathlineto{\pgfqpoint{4.544321in}{2.347963in}}%
\pgfpathlineto{\pgfqpoint{4.544346in}{2.364620in}}%
\pgfpathlineto{\pgfqpoint{4.545232in}{2.432919in}}%
\pgfpathlineto{\pgfqpoint{4.544728in}{2.319204in}}%
\pgfpathlineto{\pgfqpoint{4.545417in}{2.334930in}}%
\pgfpathlineto{\pgfqpoint{4.545848in}{2.424152in}}%
\pgfpathlineto{\pgfqpoint{4.545958in}{2.325792in}}%
\pgfpathlineto{\pgfqpoint{4.546561in}{2.344340in}}%
\pgfpathlineto{\pgfqpoint{4.546660in}{2.332669in}}%
\pgfpathlineto{\pgfqpoint{4.546968in}{2.417580in}}%
\pgfpathlineto{\pgfqpoint{4.547558in}{2.392174in}}%
\pgfpathlineto{\pgfqpoint{4.547595in}{2.433175in}}%
\pgfpathlineto{\pgfqpoint{4.547817in}{2.316789in}}%
\pgfpathlineto{\pgfqpoint{4.548617in}{2.347453in}}%
\pgfpathlineto{\pgfqpoint{4.548826in}{2.418903in}}%
\pgfpathlineto{\pgfqpoint{4.549048in}{2.319671in}}%
\pgfpathlineto{\pgfqpoint{4.549737in}{2.354404in}}%
\pgfpathlineto{\pgfqpoint{4.549761in}{2.354608in}}%
\pgfpathlineto{\pgfqpoint{4.549798in}{2.351541in}}%
\pgfpathlineto{\pgfqpoint{4.550278in}{2.305265in}}%
\pgfpathlineto{\pgfqpoint{4.550069in}{2.424769in}}%
\pgfpathlineto{\pgfqpoint{4.550746in}{2.403402in}}%
\pgfpathlineto{\pgfqpoint{4.551300in}{2.423288in}}%
\pgfpathlineto{\pgfqpoint{4.551054in}{2.327446in}}%
\pgfpathlineto{\pgfqpoint{4.551829in}{2.388442in}}%
\pgfpathlineto{\pgfqpoint{4.552752in}{2.323487in}}%
\pgfpathlineto{\pgfqpoint{4.552432in}{2.427257in}}%
\pgfpathlineto{\pgfqpoint{4.552974in}{2.364072in}}%
\pgfpathlineto{\pgfqpoint{4.553380in}{2.323304in}}%
\pgfpathlineto{\pgfqpoint{4.553134in}{2.430408in}}%
\pgfpathlineto{\pgfqpoint{4.553638in}{2.398703in}}%
\pgfpathlineto{\pgfqpoint{4.554278in}{2.425376in}}%
\pgfpathlineto{\pgfqpoint{4.554611in}{2.316080in}}%
\pgfpathlineto{\pgfqpoint{4.554709in}{2.363912in}}%
\pgfpathlineto{\pgfqpoint{4.555854in}{2.327865in}}%
\pgfpathlineto{\pgfqpoint{4.555521in}{2.419460in}}%
\pgfpathlineto{\pgfqpoint{4.555866in}{2.336211in}}%
\pgfpathlineto{\pgfqpoint{4.556248in}{2.428191in}}%
\pgfpathlineto{\pgfqpoint{4.556469in}{2.296224in}}%
\pgfpathlineto{\pgfqpoint{4.557048in}{2.393626in}}%
\pgfpathlineto{\pgfqpoint{4.557700in}{2.307853in}}%
\pgfpathlineto{\pgfqpoint{4.557478in}{2.417433in}}%
\pgfpathlineto{\pgfqpoint{4.558155in}{2.381263in}}%
\pgfpathlineto{\pgfqpoint{4.558611in}{2.434488in}}%
\pgfpathlineto{\pgfqpoint{4.558931in}{2.311557in}}%
\pgfpathlineto{\pgfqpoint{4.559275in}{2.401550in}}%
\pgfpathlineto{\pgfqpoint{4.559558in}{2.303054in}}%
\pgfpathlineto{\pgfqpoint{4.559952in}{2.430969in}}%
\pgfpathlineto{\pgfqpoint{4.560444in}{2.385770in}}%
\pgfpathlineto{\pgfqpoint{4.560531in}{2.416194in}}%
\pgfpathlineto{\pgfqpoint{4.561417in}{2.320405in}}%
\pgfpathlineto{\pgfqpoint{4.561515in}{2.352358in}}%
\pgfpathlineto{\pgfqpoint{4.561737in}{2.432002in}}%
\pgfpathlineto{\pgfqpoint{4.562032in}{2.310332in}}%
\pgfpathlineto{\pgfqpoint{4.562623in}{2.355252in}}%
\pgfpathlineto{\pgfqpoint{4.562648in}{2.310801in}}%
\pgfpathlineto{\pgfqpoint{4.563029in}{2.425797in}}%
\pgfpathlineto{\pgfqpoint{4.563718in}{2.379112in}}%
\pgfpathlineto{\pgfqpoint{4.564863in}{2.427038in}}%
\pgfpathlineto{\pgfqpoint{4.564506in}{2.314282in}}%
\pgfpathlineto{\pgfqpoint{4.564875in}{2.425359in}}%
\pgfpathlineto{\pgfqpoint{4.565121in}{2.307045in}}%
\pgfpathlineto{\pgfqpoint{4.565441in}{2.432449in}}%
\pgfpathlineto{\pgfqpoint{4.566057in}{2.408980in}}%
\pgfpathlineto{\pgfqpoint{4.566229in}{2.421325in}}%
\pgfpathlineto{\pgfqpoint{4.566352in}{2.318849in}}%
\pgfpathlineto{\pgfqpoint{4.567029in}{2.352417in}}%
\pgfpathlineto{\pgfqpoint{4.567595in}{2.316248in}}%
\pgfpathlineto{\pgfqpoint{4.567977in}{2.430944in}}%
\pgfpathlineto{\pgfqpoint{4.568063in}{2.405730in}}%
\pgfpathlineto{\pgfqpoint{4.568592in}{2.433022in}}%
\pgfpathlineto{\pgfqpoint{4.568211in}{2.304761in}}%
\pgfpathlineto{\pgfqpoint{4.568814in}{2.330491in}}%
\pgfpathlineto{\pgfqpoint{4.568900in}{2.307920in}}%
\pgfpathlineto{\pgfqpoint{4.569208in}{2.421269in}}%
\pgfpathlineto{\pgfqpoint{4.569835in}{2.415019in}}%
\pgfpathlineto{\pgfqpoint{4.570684in}{2.316830in}}%
\pgfpathlineto{\pgfqpoint{4.570401in}{2.420654in}}%
\pgfpathlineto{\pgfqpoint{4.570955in}{2.408633in}}%
\pgfpathlineto{\pgfqpoint{4.570980in}{2.407376in}}%
\pgfpathlineto{\pgfqpoint{4.571041in}{2.417183in}}%
\pgfpathlineto{\pgfqpoint{4.571681in}{2.435228in}}%
\pgfpathlineto{\pgfqpoint{4.571374in}{2.307512in}}%
\pgfpathlineto{\pgfqpoint{4.571964in}{2.333938in}}%
\pgfpathlineto{\pgfqpoint{4.572014in}{2.305856in}}%
\pgfpathlineto{\pgfqpoint{4.572223in}{2.424545in}}%
\pgfpathlineto{\pgfqpoint{4.572974in}{2.386553in}}%
\pgfpathlineto{\pgfqpoint{4.573454in}{2.426427in}}%
\pgfpathlineto{\pgfqpoint{4.573847in}{2.301987in}}%
\pgfpathlineto{\pgfqpoint{4.574106in}{2.416634in}}%
\pgfpathlineto{\pgfqpoint{4.574389in}{2.302242in}}%
\pgfpathlineto{\pgfqpoint{4.574771in}{2.428967in}}%
\pgfpathlineto{\pgfqpoint{4.575275in}{2.396453in}}%
\pgfpathlineto{\pgfqpoint{4.575349in}{2.436188in}}%
\pgfpathlineto{\pgfqpoint{4.575718in}{2.310246in}}%
\pgfpathlineto{\pgfqpoint{4.576223in}{2.345092in}}%
\pgfpathlineto{\pgfqpoint{4.576974in}{2.315005in}}%
\pgfpathlineto{\pgfqpoint{4.576580in}{2.439925in}}%
\pgfpathlineto{\pgfqpoint{4.577306in}{2.383123in}}%
\pgfpathlineto{\pgfqpoint{4.577355in}{2.431065in}}%
\pgfpathlineto{\pgfqpoint{4.577552in}{2.301649in}}%
\pgfpathlineto{\pgfqpoint{4.578451in}{2.420978in}}%
\pgfpathlineto{\pgfqpoint{4.578487in}{2.428455in}}%
\pgfpathlineto{\pgfqpoint{4.578820in}{2.317982in}}%
\pgfpathlineto{\pgfqpoint{4.579287in}{2.400072in}}%
\pgfpathlineto{\pgfqpoint{4.579398in}{2.304418in}}%
\pgfpathlineto{\pgfqpoint{4.580309in}{2.443969in}}%
\pgfpathlineto{\pgfqpoint{4.580395in}{2.383982in}}%
\pgfpathlineto{\pgfqpoint{4.581072in}{2.428244in}}%
\pgfpathlineto{\pgfqpoint{4.580641in}{2.314458in}}%
\pgfpathlineto{\pgfqpoint{4.581171in}{2.339898in}}%
\pgfpathlineto{\pgfqpoint{4.581257in}{2.321506in}}%
\pgfpathlineto{\pgfqpoint{4.581552in}{2.429060in}}%
\pgfpathlineto{\pgfqpoint{4.582241in}{2.381371in}}%
\pgfpathlineto{\pgfqpoint{4.582512in}{2.302947in}}%
\pgfpathlineto{\pgfqpoint{4.582352in}{2.416165in}}%
\pgfpathlineto{\pgfqpoint{4.582967in}{2.405488in}}%
\pgfpathlineto{\pgfqpoint{4.583423in}{2.438123in}}%
\pgfpathlineto{\pgfqpoint{4.583731in}{2.313284in}}%
\pgfpathlineto{\pgfqpoint{4.584051in}{2.402257in}}%
\pgfpathlineto{\pgfqpoint{4.584371in}{2.296702in}}%
\pgfpathlineto{\pgfqpoint{4.584211in}{2.425552in}}%
\pgfpathlineto{\pgfqpoint{4.585171in}{2.376409in}}%
\pgfpathlineto{\pgfqpoint{4.585244in}{2.437677in}}%
\pgfpathlineto{\pgfqpoint{4.586204in}{2.302925in}}%
\pgfpathlineto{\pgfqpoint{4.586217in}{2.301255in}}%
\pgfpathlineto{\pgfqpoint{4.586487in}{2.410516in}}%
\pgfpathlineto{\pgfqpoint{4.586524in}{2.424634in}}%
\pgfpathlineto{\pgfqpoint{4.586746in}{2.314664in}}%
\pgfpathlineto{\pgfqpoint{4.587435in}{2.328937in}}%
\pgfpathlineto{\pgfqpoint{4.588051in}{2.313144in}}%
\pgfpathlineto{\pgfqpoint{4.587718in}{2.433167in}}%
\pgfpathlineto{\pgfqpoint{4.588334in}{2.400705in}}%
\pgfpathlineto{\pgfqpoint{4.588974in}{2.439797in}}%
\pgfpathlineto{\pgfqpoint{4.588691in}{2.314011in}}%
\pgfpathlineto{\pgfqpoint{4.589404in}{2.360661in}}%
\pgfpathlineto{\pgfqpoint{4.590217in}{2.422124in}}%
\pgfpathlineto{\pgfqpoint{4.589909in}{2.302338in}}%
\pgfpathlineto{\pgfqpoint{4.590426in}{2.351698in}}%
\pgfpathlineto{\pgfqpoint{4.591152in}{2.305474in}}%
\pgfpathlineto{\pgfqpoint{4.590832in}{2.425576in}}%
\pgfpathlineto{\pgfqpoint{4.591497in}{2.379016in}}%
\pgfpathlineto{\pgfqpoint{4.592063in}{2.440449in}}%
\pgfpathlineto{\pgfqpoint{4.591755in}{2.304977in}}%
\pgfpathlineto{\pgfqpoint{4.592617in}{2.393846in}}%
\pgfpathlineto{\pgfqpoint{4.592998in}{2.315823in}}%
\pgfpathlineto{\pgfqpoint{4.593294in}{2.422554in}}%
\pgfpathlineto{\pgfqpoint{4.593810in}{2.376267in}}%
\pgfpathlineto{\pgfqpoint{4.593921in}{2.440903in}}%
\pgfpathlineto{\pgfqpoint{4.594229in}{2.305830in}}%
\pgfpathlineto{\pgfqpoint{4.594844in}{2.324228in}}%
\pgfpathlineto{\pgfqpoint{4.594906in}{2.316371in}}%
\pgfpathlineto{\pgfqpoint{4.595152in}{2.421991in}}%
\pgfpathlineto{\pgfqpoint{4.595644in}{2.372707in}}%
\pgfpathlineto{\pgfqpoint{4.596395in}{2.433285in}}%
\pgfpathlineto{\pgfqpoint{4.596137in}{2.326780in}}%
\pgfpathlineto{\pgfqpoint{4.596666in}{2.345848in}}%
\pgfpathlineto{\pgfqpoint{4.597318in}{2.311361in}}%
\pgfpathlineto{\pgfqpoint{4.597626in}{2.439420in}}%
\pgfpathlineto{\pgfqpoint{4.597724in}{2.392638in}}%
\pgfpathlineto{\pgfqpoint{4.598414in}{2.421836in}}%
\pgfpathlineto{\pgfqpoint{4.597934in}{2.320938in}}%
\pgfpathlineto{\pgfqpoint{4.598537in}{2.325743in}}%
\pgfpathlineto{\pgfqpoint{4.598561in}{2.312900in}}%
\pgfpathlineto{\pgfqpoint{4.599447in}{2.432226in}}%
\pgfpathlineto{\pgfqpoint{4.599558in}{2.400635in}}%
\pgfpathlineto{\pgfqpoint{4.600420in}{2.292424in}}%
\pgfpathlineto{\pgfqpoint{4.600223in}{2.431647in}}%
\pgfpathlineto{\pgfqpoint{4.600654in}{2.404732in}}%
\pgfpathlineto{\pgfqpoint{4.600727in}{2.431599in}}%
\pgfpathlineto{\pgfqpoint{4.601047in}{2.323966in}}%
\pgfpathlineto{\pgfqpoint{4.601589in}{2.360061in}}%
\pgfpathlineto{\pgfqpoint{4.601897in}{2.416253in}}%
\pgfpathlineto{\pgfqpoint{4.601663in}{2.307021in}}%
\pgfpathlineto{\pgfqpoint{4.602241in}{2.336501in}}%
\pgfpathlineto{\pgfqpoint{4.602881in}{2.310837in}}%
\pgfpathlineto{\pgfqpoint{4.602574in}{2.450083in}}%
\pgfpathlineto{\pgfqpoint{4.603300in}{2.404438in}}%
\pgfpathlineto{\pgfqpoint{4.603361in}{2.435893in}}%
\pgfpathlineto{\pgfqpoint{4.603509in}{2.309412in}}%
\pgfpathlineto{\pgfqpoint{4.604420in}{2.414436in}}%
\pgfpathlineto{\pgfqpoint{4.605035in}{2.423723in}}%
\pgfpathlineto{\pgfqpoint{4.604752in}{2.321369in}}%
\pgfpathlineto{\pgfqpoint{4.605244in}{2.385079in}}%
\pgfpathlineto{\pgfqpoint{4.605970in}{2.303582in}}%
\pgfpathlineto{\pgfqpoint{4.606278in}{2.431826in}}%
\pgfpathlineto{\pgfqpoint{4.606352in}{2.386196in}}%
\pgfpathlineto{\pgfqpoint{4.607017in}{2.429153in}}%
\pgfpathlineto{\pgfqpoint{4.606610in}{2.306308in}}%
\pgfpathlineto{\pgfqpoint{4.607164in}{2.343614in}}%
\pgfpathlineto{\pgfqpoint{4.607201in}{2.306204in}}%
\pgfpathlineto{\pgfqpoint{4.608124in}{2.425060in}}%
\pgfpathlineto{\pgfqpoint{4.608223in}{2.385688in}}%
\pgfpathlineto{\pgfqpoint{4.608826in}{2.413648in}}%
\pgfpathlineto{\pgfqpoint{4.608444in}{2.300286in}}%
\pgfpathlineto{\pgfqpoint{4.609195in}{2.339256in}}%
\pgfpathlineto{\pgfqpoint{4.609724in}{2.314110in}}%
\pgfpathlineto{\pgfqpoint{4.609367in}{2.439399in}}%
\pgfpathlineto{\pgfqpoint{4.609897in}{2.387170in}}%
\pgfpathlineto{\pgfqpoint{4.610118in}{2.425873in}}%
\pgfpathlineto{\pgfqpoint{4.610303in}{2.306636in}}%
\pgfpathlineto{\pgfqpoint{4.610943in}{2.338611in}}%
\pgfpathlineto{\pgfqpoint{4.611090in}{2.375058in}}%
\pgfpathlineto{\pgfqpoint{4.611226in}{2.426016in}}%
\pgfpathlineto{\pgfqpoint{4.611546in}{2.304509in}}%
\pgfpathlineto{\pgfqpoint{4.612124in}{2.326243in}}%
\pgfpathlineto{\pgfqpoint{4.612161in}{2.312207in}}%
\pgfpathlineto{\pgfqpoint{4.612469in}{2.438505in}}%
\pgfpathlineto{\pgfqpoint{4.613146in}{2.389262in}}%
\pgfpathlineto{\pgfqpoint{4.613810in}{2.422735in}}%
\pgfpathlineto{\pgfqpoint{4.613392in}{2.307234in}}%
\pgfpathlineto{\pgfqpoint{4.613958in}{2.347991in}}%
\pgfpathlineto{\pgfqpoint{4.614647in}{2.303149in}}%
\pgfpathlineto{\pgfqpoint{4.614253in}{2.425828in}}%
\pgfpathlineto{\pgfqpoint{4.614893in}{2.416781in}}%
\pgfpathlineto{\pgfqpoint{4.615029in}{2.426364in}}%
\pgfpathlineto{\pgfqpoint{4.615226in}{2.304596in}}%
\pgfpathlineto{\pgfqpoint{4.615817in}{2.373447in}}%
\pgfpathlineto{\pgfqpoint{4.615866in}{2.306904in}}%
\pgfpathlineto{\pgfqpoint{4.616161in}{2.448593in}}%
\pgfpathlineto{\pgfqpoint{4.616900in}{2.412486in}}%
\pgfpathlineto{\pgfqpoint{4.617367in}{2.439105in}}%
\pgfpathlineto{\pgfqpoint{4.617097in}{2.317001in}}%
\pgfpathlineto{\pgfqpoint{4.618020in}{2.427394in}}%
\pgfpathlineto{\pgfqpoint{4.618327in}{2.286545in}}%
\pgfpathlineto{\pgfqpoint{4.618143in}{2.433966in}}%
\pgfpathlineto{\pgfqpoint{4.619189in}{2.395263in}}%
\pgfpathlineto{\pgfqpoint{4.619263in}{2.451647in}}%
\pgfpathlineto{\pgfqpoint{4.619570in}{2.321404in}}%
\pgfpathlineto{\pgfqpoint{4.620149in}{2.332772in}}%
\pgfpathlineto{\pgfqpoint{4.620186in}{2.308462in}}%
\pgfpathlineto{\pgfqpoint{4.621109in}{2.438914in}}%
\pgfpathlineto{\pgfqpoint{4.621195in}{2.393674in}}%
\pgfpathlineto{\pgfqpoint{4.621257in}{2.421845in}}%
\pgfpathlineto{\pgfqpoint{4.621392in}{2.331216in}}%
\pgfpathlineto{\pgfqpoint{4.621429in}{2.292032in}}%
\pgfpathlineto{\pgfqpoint{4.622364in}{2.433773in}}%
\pgfpathlineto{\pgfqpoint{4.622463in}{2.401063in}}%
\pgfpathlineto{\pgfqpoint{4.622475in}{2.401865in}}%
\pgfpathlineto{\pgfqpoint{4.622647in}{2.342771in}}%
\pgfpathlineto{\pgfqpoint{4.623300in}{2.307792in}}%
\pgfpathlineto{\pgfqpoint{4.623583in}{2.417068in}}%
\pgfpathlineto{\pgfqpoint{4.624210in}{2.444533in}}%
\pgfpathlineto{\pgfqpoint{4.623890in}{2.314923in}}%
\pgfpathlineto{\pgfqpoint{4.624506in}{2.321045in}}%
\pgfpathlineto{\pgfqpoint{4.625121in}{2.305506in}}%
\pgfpathlineto{\pgfqpoint{4.624998in}{2.425369in}}%
\pgfpathlineto{\pgfqpoint{4.625417in}{2.398990in}}%
\pgfpathlineto{\pgfqpoint{4.625527in}{2.424736in}}%
\pgfpathlineto{\pgfqpoint{4.626377in}{2.312790in}}%
\pgfpathlineto{\pgfqpoint{4.626450in}{2.333903in}}%
\pgfpathlineto{\pgfqpoint{4.627373in}{2.428329in}}%
\pgfpathlineto{\pgfqpoint{4.626980in}{2.300536in}}%
\pgfpathlineto{\pgfqpoint{4.627570in}{2.353590in}}%
\pgfpathlineto{\pgfqpoint{4.627607in}{2.311394in}}%
\pgfpathlineto{\pgfqpoint{4.627903in}{2.434564in}}%
\pgfpathlineto{\pgfqpoint{4.628629in}{2.416675in}}%
\pgfpathlineto{\pgfqpoint{4.629133in}{2.436893in}}%
\pgfpathlineto{\pgfqpoint{4.628826in}{2.303365in}}%
\pgfpathlineto{\pgfqpoint{4.629577in}{2.339301in}}%
\pgfpathlineto{\pgfqpoint{4.630093in}{2.302433in}}%
\pgfpathlineto{\pgfqpoint{4.629724in}{2.440431in}}%
\pgfpathlineto{\pgfqpoint{4.630500in}{2.438080in}}%
\pgfpathlineto{\pgfqpoint{4.630512in}{2.438609in}}%
\pgfpathlineto{\pgfqpoint{4.630573in}{2.396759in}}%
\pgfpathlineto{\pgfqpoint{4.630623in}{2.319995in}}%
\pgfpathlineto{\pgfqpoint{4.631607in}{2.433127in}}%
\pgfpathlineto{\pgfqpoint{4.631693in}{2.376552in}}%
\pgfpathlineto{\pgfqpoint{4.631927in}{2.296146in}}%
\pgfpathlineto{\pgfqpoint{4.632850in}{2.448205in}}%
\pgfpathlineto{\pgfqpoint{4.632863in}{2.450093in}}%
\pgfpathlineto{\pgfqpoint{4.633133in}{2.325267in}}%
\pgfpathlineto{\pgfqpoint{4.633786in}{2.286299in}}%
\pgfpathlineto{\pgfqpoint{4.633638in}{2.427198in}}%
\pgfpathlineto{\pgfqpoint{4.634056in}{2.403276in}}%
\pgfpathlineto{\pgfqpoint{4.634093in}{2.436148in}}%
\pgfpathlineto{\pgfqpoint{4.635041in}{2.302584in}}%
\pgfpathlineto{\pgfqpoint{4.635103in}{2.341029in}}%
\pgfpathlineto{\pgfqpoint{4.635952in}{2.431342in}}%
\pgfpathlineto{\pgfqpoint{4.635632in}{2.306467in}}%
\pgfpathlineto{\pgfqpoint{4.636223in}{2.353798in}}%
\pgfpathlineto{\pgfqpoint{4.636272in}{2.316463in}}%
\pgfpathlineto{\pgfqpoint{4.636567in}{2.428511in}}%
\pgfpathlineto{\pgfqpoint{4.637281in}{2.408606in}}%
\pgfpathlineto{\pgfqpoint{4.638106in}{2.301998in}}%
\pgfpathlineto{\pgfqpoint{4.637798in}{2.442318in}}%
\pgfpathlineto{\pgfqpoint{4.638352in}{2.411809in}}%
\pgfpathlineto{\pgfqpoint{4.638413in}{2.439215in}}%
\pgfpathlineto{\pgfqpoint{4.638721in}{2.323917in}}%
\pgfpathlineto{\pgfqpoint{4.639287in}{2.331314in}}%
\pgfpathlineto{\pgfqpoint{4.639336in}{2.313146in}}%
\pgfpathlineto{\pgfqpoint{4.639632in}{2.419994in}}%
\pgfpathlineto{\pgfqpoint{4.640223in}{2.406405in}}%
\pgfpathlineto{\pgfqpoint{4.640272in}{2.436237in}}%
\pgfpathlineto{\pgfqpoint{4.640580in}{2.305514in}}%
\pgfpathlineto{\pgfqpoint{4.641109in}{2.363400in}}%
\pgfpathlineto{\pgfqpoint{4.641810in}{2.314166in}}%
\pgfpathlineto{\pgfqpoint{4.641515in}{2.432616in}}%
\pgfpathlineto{\pgfqpoint{4.642192in}{2.389849in}}%
\pgfpathlineto{\pgfqpoint{4.642253in}{2.420683in}}%
\pgfpathlineto{\pgfqpoint{4.642438in}{2.307261in}}%
\pgfpathlineto{\pgfqpoint{4.643041in}{2.322449in}}%
\pgfpathlineto{\pgfqpoint{4.643669in}{2.310022in}}%
\pgfpathlineto{\pgfqpoint{4.643324in}{2.420673in}}%
\pgfpathlineto{\pgfqpoint{4.643878in}{2.389380in}}%
\pgfpathlineto{\pgfqpoint{4.644100in}{2.430552in}}%
\pgfpathlineto{\pgfqpoint{4.644296in}{2.306288in}}%
\pgfpathlineto{\pgfqpoint{4.644875in}{2.352566in}}%
\pgfpathlineto{\pgfqpoint{4.644912in}{2.318140in}}%
\pgfpathlineto{\pgfqpoint{4.645835in}{2.427018in}}%
\pgfpathlineto{\pgfqpoint{4.645921in}{2.391847in}}%
\pgfpathlineto{\pgfqpoint{4.646450in}{2.442940in}}%
\pgfpathlineto{\pgfqpoint{4.646758in}{2.301047in}}%
\pgfpathlineto{\pgfqpoint{4.647090in}{2.422879in}}%
\pgfpathlineto{\pgfqpoint{4.647373in}{2.315574in}}%
\pgfpathlineto{\pgfqpoint{4.647816in}{2.429610in}}%
\pgfpathlineto{\pgfqpoint{4.648223in}{2.383552in}}%
\pgfpathlineto{\pgfqpoint{4.648887in}{2.418404in}}%
\pgfpathlineto{\pgfqpoint{4.649183in}{2.325270in}}%
\pgfpathlineto{\pgfqpoint{4.649195in}{2.326605in}}%
\pgfpathlineto{\pgfqpoint{4.649232in}{2.312009in}}%
\pgfpathlineto{\pgfqpoint{4.649540in}{2.431396in}}%
\pgfpathlineto{\pgfqpoint{4.649552in}{2.438627in}}%
\pgfpathlineto{\pgfqpoint{4.649847in}{2.305686in}}%
\pgfpathlineto{\pgfqpoint{4.650450in}{2.325642in}}%
\pgfpathlineto{\pgfqpoint{4.650475in}{2.307975in}}%
\pgfpathlineto{\pgfqpoint{4.650906in}{2.426318in}}%
\pgfpathlineto{\pgfqpoint{4.651484in}{2.393276in}}%
\pgfpathlineto{\pgfqpoint{4.652641in}{2.428510in}}%
\pgfpathlineto{\pgfqpoint{4.652321in}{2.308896in}}%
\pgfpathlineto{\pgfqpoint{4.652653in}{2.426861in}}%
\pgfpathlineto{\pgfqpoint{4.653564in}{2.306373in}}%
\pgfpathlineto{\pgfqpoint{4.653244in}{2.435211in}}%
\pgfpathlineto{\pgfqpoint{4.653786in}{2.396040in}}%
\pgfpathlineto{\pgfqpoint{4.654007in}{2.425452in}}%
\pgfpathlineto{\pgfqpoint{4.654167in}{2.313338in}}%
\pgfpathlineto{\pgfqpoint{4.654709in}{2.351417in}}%
\pgfpathlineto{\pgfqpoint{4.655410in}{2.306262in}}%
\pgfpathlineto{\pgfqpoint{4.655743in}{2.431891in}}%
\pgfpathlineto{\pgfqpoint{4.655779in}{2.395134in}}%
\pgfpathlineto{\pgfqpoint{4.656666in}{2.308158in}}%
\pgfpathlineto{\pgfqpoint{4.656346in}{2.427230in}}%
\pgfpathlineto{\pgfqpoint{4.656863in}{2.394585in}}%
\pgfpathlineto{\pgfqpoint{4.656973in}{2.424403in}}%
\pgfpathlineto{\pgfqpoint{4.657884in}{2.307408in}}%
\pgfpathlineto{\pgfqpoint{4.658832in}{2.431421in}}%
\pgfpathlineto{\pgfqpoint{4.659041in}{2.331677in}}%
\pgfpathlineto{\pgfqpoint{4.659755in}{2.307355in}}%
\pgfpathlineto{\pgfqpoint{4.660038in}{2.428787in}}%
\pgfpathlineto{\pgfqpoint{4.660063in}{2.437904in}}%
\pgfpathlineto{\pgfqpoint{4.660358in}{2.314030in}}%
\pgfpathlineto{\pgfqpoint{4.660936in}{2.336335in}}%
\pgfpathlineto{\pgfqpoint{4.660986in}{2.304820in}}%
\pgfpathlineto{\pgfqpoint{4.661281in}{2.426015in}}%
\pgfpathlineto{\pgfqpoint{4.661884in}{2.418888in}}%
\pgfpathlineto{\pgfqpoint{4.661921in}{2.426923in}}%
\pgfpathlineto{\pgfqpoint{4.662216in}{2.305283in}}%
\pgfpathlineto{\pgfqpoint{4.662770in}{2.337023in}}%
\pgfpathlineto{\pgfqpoint{4.663447in}{2.303916in}}%
\pgfpathlineto{\pgfqpoint{4.663127in}{2.434751in}}%
\pgfpathlineto{\pgfqpoint{4.663755in}{2.417771in}}%
\pgfpathlineto{\pgfqpoint{4.663890in}{2.433055in}}%
\pgfpathlineto{\pgfqpoint{4.664075in}{2.306395in}}%
\pgfpathlineto{\pgfqpoint{4.664616in}{2.345762in}}%
\pgfpathlineto{\pgfqpoint{4.664998in}{2.419967in}}%
\pgfpathlineto{\pgfqpoint{4.665244in}{2.313780in}}%
\pgfpathlineto{\pgfqpoint{4.665293in}{2.304057in}}%
\pgfpathlineto{\pgfqpoint{4.665576in}{2.429356in}}%
\pgfpathlineto{\pgfqpoint{4.666093in}{2.359750in}}%
\pgfpathlineto{\pgfqpoint{4.666229in}{2.440818in}}%
\pgfpathlineto{\pgfqpoint{4.666549in}{2.309449in}}%
\pgfpathlineto{\pgfqpoint{4.667152in}{2.323533in}}%
\pgfpathlineto{\pgfqpoint{4.667176in}{2.312180in}}%
\pgfpathlineto{\pgfqpoint{4.667410in}{2.426424in}}%
\pgfpathlineto{\pgfqpoint{4.668161in}{2.402152in}}%
\pgfpathlineto{\pgfqpoint{4.668703in}{2.436473in}}%
\pgfpathlineto{\pgfqpoint{4.668395in}{2.287996in}}%
\pgfpathlineto{\pgfqpoint{4.669306in}{2.435986in}}%
\pgfpathlineto{\pgfqpoint{4.669318in}{2.439761in}}%
\pgfpathlineto{\pgfqpoint{4.669638in}{2.306066in}}%
\pgfpathlineto{\pgfqpoint{4.670167in}{2.363854in}}%
\pgfpathlineto{\pgfqpoint{4.670253in}{2.326172in}}%
\pgfpathlineto{\pgfqpoint{4.670524in}{2.437221in}}%
\pgfpathlineto{\pgfqpoint{4.671176in}{2.421947in}}%
\pgfpathlineto{\pgfqpoint{4.671312in}{2.436768in}}%
\pgfpathlineto{\pgfqpoint{4.671386in}{2.376327in}}%
\pgfpathlineto{\pgfqpoint{4.671484in}{2.287438in}}%
\pgfpathlineto{\pgfqpoint{4.672432in}{2.442523in}}%
\pgfpathlineto{\pgfqpoint{4.672481in}{2.388144in}}%
\pgfpathlineto{\pgfqpoint{4.672518in}{2.409850in}}%
\pgfpathlineto{\pgfqpoint{4.672727in}{2.314195in}}%
\pgfpathlineto{\pgfqpoint{4.673650in}{2.443754in}}%
\pgfpathlineto{\pgfqpoint{4.674573in}{2.296422in}}%
\pgfpathlineto{\pgfqpoint{4.674819in}{2.415510in}}%
\pgfpathlineto{\pgfqpoint{4.674893in}{2.441878in}}%
\pgfpathlineto{\pgfqpoint{4.675189in}{2.314367in}}%
\pgfpathlineto{\pgfqpoint{4.675718in}{2.387845in}}%
\pgfpathlineto{\pgfqpoint{4.676419in}{2.310134in}}%
\pgfpathlineto{\pgfqpoint{4.676112in}{2.422058in}}%
\pgfpathlineto{\pgfqpoint{4.676813in}{2.403428in}}%
\pgfpathlineto{\pgfqpoint{4.677662in}{2.300817in}}%
\pgfpathlineto{\pgfqpoint{4.677367in}{2.426303in}}%
\pgfpathlineto{\pgfqpoint{4.677909in}{2.407926in}}%
\pgfpathlineto{\pgfqpoint{4.678733in}{2.436651in}}%
\pgfpathlineto{\pgfqpoint{4.678290in}{2.298840in}}%
\pgfpathlineto{\pgfqpoint{4.678979in}{2.361680in}}%
\pgfpathlineto{\pgfqpoint{4.679964in}{2.442584in}}%
\pgfpathlineto{\pgfqpoint{4.679533in}{2.324629in}}%
\pgfpathlineto{\pgfqpoint{4.680062in}{2.358238in}}%
\pgfpathlineto{\pgfqpoint{4.680136in}{2.291927in}}%
\pgfpathlineto{\pgfqpoint{4.681084in}{2.437756in}}%
\pgfpathlineto{\pgfqpoint{4.681146in}{2.380074in}}%
\pgfpathlineto{\pgfqpoint{4.682315in}{2.452412in}}%
\pgfpathlineto{\pgfqpoint{4.681995in}{2.308205in}}%
\pgfpathlineto{\pgfqpoint{4.682327in}{2.443229in}}%
\pgfpathlineto{\pgfqpoint{4.682610in}{2.303569in}}%
\pgfpathlineto{\pgfqpoint{4.683447in}{2.399526in}}%
\pgfpathlineto{\pgfqpoint{4.683681in}{2.438416in}}%
\pgfpathlineto{\pgfqpoint{4.683829in}{2.301018in}}%
\pgfpathlineto{\pgfqpoint{4.684456in}{2.332390in}}%
\pgfpathlineto{\pgfqpoint{4.685084in}{2.294327in}}%
\pgfpathlineto{\pgfqpoint{4.684924in}{2.435117in}}%
\pgfpathlineto{\pgfqpoint{4.685330in}{2.398371in}}%
\pgfpathlineto{\pgfqpoint{4.686019in}{2.431947in}}%
\pgfpathlineto{\pgfqpoint{4.685650in}{2.312763in}}%
\pgfpathlineto{\pgfqpoint{4.686266in}{2.349469in}}%
\pgfpathlineto{\pgfqpoint{4.686942in}{2.301313in}}%
\pgfpathlineto{\pgfqpoint{4.686586in}{2.434335in}}%
\pgfpathlineto{\pgfqpoint{4.687336in}{2.385156in}}%
\pgfpathlineto{\pgfqpoint{4.687816in}{2.439900in}}%
\pgfpathlineto{\pgfqpoint{4.688198in}{2.314930in}}%
\pgfpathlineto{\pgfqpoint{4.688456in}{2.409814in}}%
\pgfpathlineto{\pgfqpoint{4.688789in}{2.304117in}}%
\pgfpathlineto{\pgfqpoint{4.688604in}{2.435213in}}%
\pgfpathlineto{\pgfqpoint{4.689613in}{2.379751in}}%
\pgfpathlineto{\pgfqpoint{4.689724in}{2.441134in}}%
\pgfpathlineto{\pgfqpoint{4.690032in}{2.314111in}}%
\pgfpathlineto{\pgfqpoint{4.690696in}{2.355612in}}%
\pgfpathlineto{\pgfqpoint{4.691718in}{2.445577in}}%
\pgfpathlineto{\pgfqpoint{4.691238in}{2.322901in}}%
\pgfpathlineto{\pgfqpoint{4.691804in}{2.365331in}}%
\pgfpathlineto{\pgfqpoint{4.691890in}{2.310008in}}%
\pgfpathlineto{\pgfqpoint{4.692826in}{2.436521in}}%
\pgfpathlineto{\pgfqpoint{4.692899in}{2.377375in}}%
\pgfpathlineto{\pgfqpoint{4.694056in}{2.438140in}}%
\pgfpathlineto{\pgfqpoint{4.693121in}{2.318273in}}%
\pgfpathlineto{\pgfqpoint{4.694069in}{2.428540in}}%
\pgfpathlineto{\pgfqpoint{4.694979in}{2.305503in}}%
\pgfpathlineto{\pgfqpoint{4.694819in}{2.430317in}}%
\pgfpathlineto{\pgfqpoint{4.695201in}{2.399414in}}%
\pgfpathlineto{\pgfqpoint{4.695275in}{2.417317in}}%
\pgfpathlineto{\pgfqpoint{4.695546in}{2.316057in}}%
\pgfpathlineto{\pgfqpoint{4.695730in}{2.334328in}}%
\pgfpathlineto{\pgfqpoint{4.696198in}{2.318754in}}%
\pgfpathlineto{\pgfqpoint{4.695927in}{2.432027in}}%
\pgfpathlineto{\pgfqpoint{4.696641in}{2.406123in}}%
\pgfpathlineto{\pgfqpoint{4.697145in}{2.433885in}}%
\pgfpathlineto{\pgfqpoint{4.697416in}{2.318742in}}%
\pgfpathlineto{\pgfqpoint{4.697749in}{2.410732in}}%
\pgfpathlineto{\pgfqpoint{4.697909in}{2.429725in}}%
\pgfpathlineto{\pgfqpoint{4.698069in}{2.301418in}}%
\pgfpathlineto{\pgfqpoint{4.698585in}{2.381053in}}%
\pgfpathlineto{\pgfqpoint{4.698635in}{2.307046in}}%
\pgfpathlineto{\pgfqpoint{4.699029in}{2.432112in}}%
\pgfpathlineto{\pgfqpoint{4.699693in}{2.376736in}}%
\pgfpathlineto{\pgfqpoint{4.700235in}{2.434918in}}%
\pgfpathlineto{\pgfqpoint{4.700518in}{2.315669in}}%
\pgfpathlineto{\pgfqpoint{4.700850in}{2.417546in}}%
\pgfpathlineto{\pgfqpoint{4.701170in}{2.300773in}}%
\pgfpathlineto{\pgfqpoint{4.701441in}{2.434815in}}%
\pgfpathlineto{\pgfqpoint{4.702007in}{2.386793in}}%
\pgfpathlineto{\pgfqpoint{4.702229in}{2.436506in}}%
\pgfpathlineto{\pgfqpoint{4.702376in}{2.324815in}}%
\pgfpathlineto{\pgfqpoint{4.703078in}{2.363930in}}%
\pgfpathlineto{\pgfqpoint{4.703152in}{2.353019in}}%
\pgfpathlineto{\pgfqpoint{4.703275in}{2.393251in}}%
\pgfpathlineto{\pgfqpoint{4.704087in}{2.436780in}}%
\pgfpathlineto{\pgfqpoint{4.704272in}{2.297254in}}%
\pgfpathlineto{\pgfqpoint{4.704321in}{2.337304in}}%
\pgfpathlineto{\pgfqpoint{4.704530in}{2.454547in}}%
\pgfpathlineto{\pgfqpoint{4.704825in}{2.305226in}}%
\pgfpathlineto{\pgfqpoint{4.705429in}{2.341217in}}%
\pgfpathlineto{\pgfqpoint{4.705502in}{2.323951in}}%
\pgfpathlineto{\pgfqpoint{4.706413in}{2.431729in}}%
\pgfpathlineto{\pgfqpoint{4.706499in}{2.371678in}}%
\pgfpathlineto{\pgfqpoint{4.707632in}{2.454289in}}%
\pgfpathlineto{\pgfqpoint{4.707349in}{2.290662in}}%
\pgfpathlineto{\pgfqpoint{4.707705in}{2.409313in}}%
\pgfpathlineto{\pgfqpoint{4.707915in}{2.309790in}}%
\pgfpathlineto{\pgfqpoint{4.708419in}{2.449675in}}%
\pgfpathlineto{\pgfqpoint{4.708825in}{2.392889in}}%
\pgfpathlineto{\pgfqpoint{4.709502in}{2.431587in}}%
\pgfpathlineto{\pgfqpoint{4.709145in}{2.335347in}}%
\pgfpathlineto{\pgfqpoint{4.709724in}{2.363851in}}%
\pgfpathlineto{\pgfqpoint{4.710450in}{2.285437in}}%
\pgfpathlineto{\pgfqpoint{4.710721in}{2.444860in}}%
\pgfpathlineto{\pgfqpoint{4.710745in}{2.434332in}}%
\pgfpathlineto{\pgfqpoint{4.711004in}{2.317611in}}%
\pgfpathlineto{\pgfqpoint{4.711509in}{2.438409in}}%
\pgfpathlineto{\pgfqpoint{4.712062in}{2.359612in}}%
\pgfpathlineto{\pgfqpoint{4.713182in}{2.436976in}}%
\pgfpathlineto{\pgfqpoint{4.712862in}{2.306380in}}%
\pgfpathlineto{\pgfqpoint{4.713232in}{2.422664in}}%
\pgfpathlineto{\pgfqpoint{4.713527in}{2.293625in}}%
\pgfpathlineto{\pgfqpoint{4.713810in}{2.440794in}}%
\pgfpathlineto{\pgfqpoint{4.714438in}{2.398978in}}%
\pgfpathlineto{\pgfqpoint{4.714598in}{2.435481in}}%
\pgfpathlineto{\pgfqpoint{4.715336in}{2.331339in}}%
\pgfpathlineto{\pgfqpoint{4.715373in}{2.350083in}}%
\pgfpathlineto{\pgfqpoint{4.715964in}{2.304480in}}%
\pgfpathlineto{\pgfqpoint{4.716296in}{2.434965in}}%
\pgfpathlineto{\pgfqpoint{4.716407in}{2.389640in}}%
\pgfpathlineto{\pgfqpoint{4.717072in}{2.431515in}}%
\pgfpathlineto{\pgfqpoint{4.716628in}{2.300653in}}%
\pgfpathlineto{\pgfqpoint{4.717367in}{2.351729in}}%
\pgfpathlineto{\pgfqpoint{4.718462in}{2.325813in}}%
\pgfpathlineto{\pgfqpoint{4.717699in}{2.418817in}}%
\pgfpathlineto{\pgfqpoint{4.718487in}{2.337334in}}%
\pgfpathlineto{\pgfqpoint{4.719422in}{2.433792in}}%
\pgfpathlineto{\pgfqpoint{4.719041in}{2.295477in}}%
\pgfpathlineto{\pgfqpoint{4.719632in}{2.367557in}}%
\pgfpathlineto{\pgfqpoint{4.719668in}{2.317542in}}%
\pgfpathlineto{\pgfqpoint{4.719976in}{2.437382in}}%
\pgfpathlineto{\pgfqpoint{4.720715in}{2.384429in}}%
\pgfpathlineto{\pgfqpoint{4.721822in}{2.433052in}}%
\pgfpathlineto{\pgfqpoint{4.721576in}{2.317986in}}%
\pgfpathlineto{\pgfqpoint{4.721847in}{2.426962in}}%
\pgfpathlineto{\pgfqpoint{4.722142in}{2.299915in}}%
\pgfpathlineto{\pgfqpoint{4.722622in}{2.428562in}}%
\pgfpathlineto{\pgfqpoint{4.723028in}{2.390605in}}%
\pgfpathlineto{\pgfqpoint{4.723090in}{2.431245in}}%
\pgfpathlineto{\pgfqpoint{4.723435in}{2.328470in}}%
\pgfpathlineto{\pgfqpoint{4.724001in}{2.329488in}}%
\pgfpathlineto{\pgfqpoint{4.724616in}{2.315500in}}%
\pgfpathlineto{\pgfqpoint{4.724321in}{2.434801in}}%
\pgfpathlineto{\pgfqpoint{4.724850in}{2.379459in}}%
\pgfpathlineto{\pgfqpoint{4.724936in}{2.450028in}}%
\pgfpathlineto{\pgfqpoint{4.725293in}{2.305491in}}%
\pgfpathlineto{\pgfqpoint{4.725908in}{2.341419in}}%
\pgfpathlineto{\pgfqpoint{4.726807in}{2.422001in}}%
\pgfpathlineto{\pgfqpoint{4.725958in}{2.328331in}}%
\pgfpathlineto{\pgfqpoint{4.727053in}{2.359586in}}%
\pgfpathlineto{\pgfqpoint{4.727693in}{2.306152in}}%
\pgfpathlineto{\pgfqpoint{4.728075in}{2.444978in}}%
\pgfpathlineto{\pgfqpoint{4.728136in}{2.396295in}}%
\pgfpathlineto{\pgfqpoint{4.728628in}{2.435565in}}%
\pgfpathlineto{\pgfqpoint{4.728321in}{2.304308in}}%
\pgfpathlineto{\pgfqpoint{4.728961in}{2.321972in}}%
\pgfpathlineto{\pgfqpoint{4.728985in}{2.301798in}}%
\pgfpathlineto{\pgfqpoint{4.729847in}{2.429549in}}%
\pgfpathlineto{\pgfqpoint{4.730007in}{2.376915in}}%
\pgfpathlineto{\pgfqpoint{4.730228in}{2.311185in}}%
\pgfpathlineto{\pgfqpoint{4.730438in}{2.420307in}}%
\pgfpathlineto{\pgfqpoint{4.730487in}{2.437058in}}%
\pgfpathlineto{\pgfqpoint{4.730832in}{2.293068in}}%
\pgfpathlineto{\pgfqpoint{4.731435in}{2.341156in}}%
\pgfpathlineto{\pgfqpoint{4.731472in}{2.305956in}}%
\pgfpathlineto{\pgfqpoint{4.731742in}{2.433747in}}%
\pgfpathlineto{\pgfqpoint{4.732493in}{2.399200in}}%
\pgfpathlineto{\pgfqpoint{4.733588in}{2.448331in}}%
\pgfpathlineto{\pgfqpoint{4.732690in}{2.317220in}}%
\pgfpathlineto{\pgfqpoint{4.733625in}{2.429594in}}%
\pgfpathlineto{\pgfqpoint{4.733945in}{2.298978in}}%
\pgfpathlineto{\pgfqpoint{4.734376in}{2.444390in}}%
\pgfpathlineto{\pgfqpoint{4.734782in}{2.395279in}}%
\pgfpathlineto{\pgfqpoint{4.734819in}{2.425995in}}%
\pgfpathlineto{\pgfqpoint{4.735755in}{2.315560in}}%
\pgfpathlineto{\pgfqpoint{4.735853in}{2.350884in}}%
\pgfpathlineto{\pgfqpoint{4.735865in}{2.350791in}}%
\pgfpathlineto{\pgfqpoint{4.735878in}{2.352768in}}%
\pgfpathlineto{\pgfqpoint{4.736727in}{2.433353in}}%
\pgfpathlineto{\pgfqpoint{4.736345in}{2.320984in}}%
\pgfpathlineto{\pgfqpoint{4.736948in}{2.353234in}}%
\pgfpathlineto{\pgfqpoint{4.737625in}{2.312028in}}%
\pgfpathlineto{\pgfqpoint{4.737478in}{2.434744in}}%
\pgfpathlineto{\pgfqpoint{4.738031in}{2.399326in}}%
\pgfpathlineto{\pgfqpoint{4.738721in}{2.421846in}}%
\pgfpathlineto{\pgfqpoint{4.738881in}{2.304381in}}%
\pgfpathlineto{\pgfqpoint{4.739139in}{2.406156in}}%
\pgfpathlineto{\pgfqpoint{4.739336in}{2.409169in}}%
\pgfpathlineto{\pgfqpoint{4.739250in}{2.390077in}}%
\pgfpathlineto{\pgfqpoint{4.739361in}{2.395596in}}%
\pgfpathlineto{\pgfqpoint{4.739447in}{2.306640in}}%
\pgfpathlineto{\pgfqpoint{4.740382in}{2.441613in}}%
\pgfpathlineto{\pgfqpoint{4.740468in}{2.397900in}}%
\pgfpathlineto{\pgfqpoint{4.740739in}{2.307146in}}%
\pgfpathlineto{\pgfqpoint{4.741170in}{2.439681in}}%
\pgfpathlineto{\pgfqpoint{4.741576in}{2.399716in}}%
\pgfpathlineto{\pgfqpoint{4.741613in}{2.431576in}}%
\pgfpathlineto{\pgfqpoint{4.742536in}{2.310195in}}%
\pgfpathlineto{\pgfqpoint{4.742647in}{2.352947in}}%
\pgfpathlineto{\pgfqpoint{4.743484in}{2.439599in}}%
\pgfpathlineto{\pgfqpoint{4.743213in}{2.320133in}}%
\pgfpathlineto{\pgfqpoint{4.743742in}{2.368865in}}%
\pgfpathlineto{\pgfqpoint{4.743841in}{2.310648in}}%
\pgfpathlineto{\pgfqpoint{4.744271in}{2.441344in}}%
\pgfpathlineto{\pgfqpoint{4.744850in}{2.369416in}}%
\pgfpathlineto{\pgfqpoint{4.746007in}{2.430690in}}%
\pgfpathlineto{\pgfqpoint{4.745662in}{2.300871in}}%
\pgfpathlineto{\pgfqpoint{4.746031in}{2.421199in}}%
\pgfpathlineto{\pgfqpoint{4.746942in}{2.314421in}}%
\pgfpathlineto{\pgfqpoint{4.746598in}{2.439190in}}%
\pgfpathlineto{\pgfqpoint{4.747164in}{2.383595in}}%
\pgfpathlineto{\pgfqpoint{4.747521in}{2.328068in}}%
\pgfpathlineto{\pgfqpoint{4.747373in}{2.444890in}}%
\pgfpathlineto{\pgfqpoint{4.748333in}{2.372844in}}%
\pgfpathlineto{\pgfqpoint{4.749219in}{2.423641in}}%
\pgfpathlineto{\pgfqpoint{4.748764in}{2.299099in}}%
\pgfpathlineto{\pgfqpoint{4.749391in}{2.312809in}}%
\pgfpathlineto{\pgfqpoint{4.749404in}{2.309373in}}%
\pgfpathlineto{\pgfqpoint{4.749699in}{2.445646in}}%
\pgfpathlineto{\pgfqpoint{4.750216in}{2.388071in}}%
\pgfpathlineto{\pgfqpoint{4.750462in}{2.442879in}}%
\pgfpathlineto{\pgfqpoint{4.750622in}{2.331780in}}%
\pgfpathlineto{\pgfqpoint{4.751188in}{2.334742in}}%
\pgfpathlineto{\pgfqpoint{4.751533in}{2.427114in}}%
\pgfpathlineto{\pgfqpoint{4.751853in}{2.303227in}}%
\pgfpathlineto{\pgfqpoint{4.752395in}{2.361065in}}%
\pgfpathlineto{\pgfqpoint{4.752493in}{2.310668in}}%
\pgfpathlineto{\pgfqpoint{4.752801in}{2.437384in}}%
\pgfpathlineto{\pgfqpoint{4.753478in}{2.365762in}}%
\pgfpathlineto{\pgfqpoint{4.753995in}{2.424605in}}%
\pgfpathlineto{\pgfqpoint{4.753711in}{2.330383in}}%
\pgfpathlineto{\pgfqpoint{4.754635in}{2.423089in}}%
\pgfpathlineto{\pgfqpoint{4.754881in}{2.327615in}}%
\pgfpathlineto{\pgfqpoint{4.754955in}{2.301991in}}%
\pgfpathlineto{\pgfqpoint{4.755878in}{2.433061in}}%
\pgfpathlineto{\pgfqpoint{4.755939in}{2.394617in}}%
\pgfpathlineto{\pgfqpoint{4.757084in}{2.432147in}}%
\pgfpathlineto{\pgfqpoint{4.756198in}{2.331072in}}%
\pgfpathlineto{\pgfqpoint{4.757096in}{2.432120in}}%
\pgfpathlineto{\pgfqpoint{4.758007in}{2.298876in}}%
\pgfpathlineto{\pgfqpoint{4.758228in}{2.395345in}}%
\pgfpathlineto{\pgfqpoint{4.758327in}{2.437238in}}%
\pgfpathlineto{\pgfqpoint{4.758684in}{2.318397in}}%
\pgfpathlineto{\pgfqpoint{4.759238in}{2.337961in}}%
\pgfpathlineto{\pgfqpoint{4.760185in}{2.439985in}}%
\pgfpathlineto{\pgfqpoint{4.759287in}{2.327421in}}%
\pgfpathlineto{\pgfqpoint{4.760419in}{2.363408in}}%
\pgfpathlineto{\pgfqpoint{4.761145in}{2.290394in}}%
\pgfpathlineto{\pgfqpoint{4.760961in}{2.432682in}}%
\pgfpathlineto{\pgfqpoint{4.761453in}{2.427179in}}%
\pgfpathlineto{\pgfqpoint{4.761465in}{2.427748in}}%
\pgfpathlineto{\pgfqpoint{4.761613in}{2.392304in}}%
\pgfpathlineto{\pgfqpoint{4.761908in}{2.323211in}}%
\pgfpathlineto{\pgfqpoint{4.762093in}{2.430490in}}%
\pgfpathlineto{\pgfqpoint{4.762745in}{2.357031in}}%
\pgfpathlineto{\pgfqpoint{4.763311in}{2.440003in}}%
\pgfpathlineto{\pgfqpoint{4.763607in}{2.307577in}}%
\pgfpathlineto{\pgfqpoint{4.763927in}{2.429585in}}%
\pgfpathlineto{\pgfqpoint{4.764247in}{2.291216in}}%
\pgfpathlineto{\pgfqpoint{4.765108in}{2.404399in}}%
\pgfpathlineto{\pgfqpoint{4.765748in}{2.414440in}}%
\pgfpathlineto{\pgfqpoint{4.766068in}{2.314181in}}%
\pgfpathlineto{\pgfqpoint{4.766130in}{2.344109in}}%
\pgfpathlineto{\pgfqpoint{4.766979in}{2.433279in}}%
\pgfpathlineto{\pgfqpoint{4.766647in}{2.310000in}}%
\pgfpathlineto{\pgfqpoint{4.767250in}{2.355995in}}%
\pgfpathlineto{\pgfqpoint{4.767336in}{2.311748in}}%
\pgfpathlineto{\pgfqpoint{4.767582in}{2.433917in}}%
\pgfpathlineto{\pgfqpoint{4.768333in}{2.396999in}}%
\pgfpathlineto{\pgfqpoint{4.769428in}{2.430300in}}%
\pgfpathlineto{\pgfqpoint{4.769194in}{2.316600in}}%
\pgfpathlineto{\pgfqpoint{4.769453in}{2.418157in}}%
\pgfpathlineto{\pgfqpoint{4.769785in}{2.294402in}}%
\pgfpathlineto{\pgfqpoint{4.770142in}{2.429824in}}%
\pgfpathlineto{\pgfqpoint{4.770610in}{2.375529in}}%
\pgfpathlineto{\pgfqpoint{4.770721in}{2.423033in}}%
\pgfpathlineto{\pgfqpoint{4.771644in}{2.310452in}}%
\pgfpathlineto{\pgfqpoint{4.771705in}{2.359897in}}%
\pgfpathlineto{\pgfqpoint{4.772542in}{2.444741in}}%
\pgfpathlineto{\pgfqpoint{4.772259in}{2.314753in}}%
\pgfpathlineto{\pgfqpoint{4.772788in}{2.354981in}}%
\pgfpathlineto{\pgfqpoint{4.772899in}{2.304760in}}%
\pgfpathlineto{\pgfqpoint{4.773342in}{2.442584in}}%
\pgfpathlineto{\pgfqpoint{4.773822in}{2.405039in}}%
\pgfpathlineto{\pgfqpoint{4.774573in}{2.425010in}}%
\pgfpathlineto{\pgfqpoint{4.774241in}{2.325598in}}%
\pgfpathlineto{\pgfqpoint{4.774671in}{2.357948in}}%
\pgfpathlineto{\pgfqpoint{4.774721in}{2.307215in}}%
\pgfpathlineto{\pgfqpoint{4.775705in}{2.441877in}}%
\pgfpathlineto{\pgfqpoint{4.775754in}{2.396574in}}%
\pgfpathlineto{\pgfqpoint{4.776616in}{2.295866in}}%
\pgfpathlineto{\pgfqpoint{4.776431in}{2.429205in}}%
\pgfpathlineto{\pgfqpoint{4.776850in}{2.397949in}}%
\pgfpathlineto{\pgfqpoint{4.777478in}{2.441866in}}%
\pgfpathlineto{\pgfqpoint{4.777207in}{2.324955in}}%
\pgfpathlineto{\pgfqpoint{4.777773in}{2.335085in}}%
\pgfpathlineto{\pgfqpoint{4.778450in}{2.308163in}}%
\pgfpathlineto{\pgfqpoint{4.778278in}{2.437940in}}%
\pgfpathlineto{\pgfqpoint{4.778770in}{2.404049in}}%
\pgfpathlineto{\pgfqpoint{4.778905in}{2.426441in}}%
\pgfpathlineto{\pgfqpoint{4.779102in}{2.318024in}}%
\pgfpathlineto{\pgfqpoint{4.779767in}{2.331422in}}%
\pgfpathlineto{\pgfqpoint{4.779779in}{2.330545in}}%
\pgfpathlineto{\pgfqpoint{4.779927in}{2.394050in}}%
\pgfpathlineto{\pgfqpoint{4.780641in}{2.430987in}}%
\pgfpathlineto{\pgfqpoint{4.780321in}{2.310502in}}%
\pgfpathlineto{\pgfqpoint{4.780997in}{2.354037in}}%
\pgfpathlineto{\pgfqpoint{4.781564in}{2.317944in}}%
\pgfpathlineto{\pgfqpoint{4.781207in}{2.432877in}}%
\pgfpathlineto{\pgfqpoint{4.781994in}{2.427448in}}%
\pgfpathlineto{\pgfqpoint{4.782007in}{2.437708in}}%
\pgfpathlineto{\pgfqpoint{4.782142in}{2.315715in}}%
\pgfpathlineto{\pgfqpoint{4.783077in}{2.407059in}}%
\pgfpathlineto{\pgfqpoint{4.783385in}{2.313773in}}%
\pgfpathlineto{\pgfqpoint{4.783250in}{2.421431in}}%
\pgfpathlineto{\pgfqpoint{4.784259in}{2.390334in}}%
\pgfpathlineto{\pgfqpoint{4.784911in}{2.427086in}}%
\pgfpathlineto{\pgfqpoint{4.785244in}{2.314481in}}%
\pgfpathlineto{\pgfqpoint{4.785342in}{2.375498in}}%
\pgfpathlineto{\pgfqpoint{4.786524in}{2.316676in}}%
\pgfpathlineto{\pgfqpoint{4.786142in}{2.429579in}}%
\pgfpathlineto{\pgfqpoint{4.786536in}{2.324101in}}%
\pgfpathlineto{\pgfqpoint{4.787373in}{2.419537in}}%
\pgfpathlineto{\pgfqpoint{4.787114in}{2.321672in}}%
\pgfpathlineto{\pgfqpoint{4.787668in}{2.364601in}}%
\pgfpathlineto{\pgfqpoint{4.787730in}{2.325964in}}%
\pgfpathlineto{\pgfqpoint{4.788382in}{2.316954in}}%
\pgfpathlineto{\pgfqpoint{4.788025in}{2.423596in}}%
\pgfpathlineto{\pgfqpoint{4.788677in}{2.388925in}}%
\pgfpathlineto{\pgfqpoint{4.789256in}{2.430236in}}%
\pgfpathlineto{\pgfqpoint{4.788985in}{2.324098in}}%
\pgfpathlineto{\pgfqpoint{4.789711in}{2.356707in}}%
\pgfpathlineto{\pgfqpoint{4.790191in}{2.325262in}}%
\pgfpathlineto{\pgfqpoint{4.790487in}{2.430037in}}%
\pgfpathlineto{\pgfqpoint{4.790671in}{2.406588in}}%
\pgfpathlineto{\pgfqpoint{4.791127in}{2.422551in}}%
\pgfpathlineto{\pgfqpoint{4.791484in}{2.328078in}}%
\pgfpathlineto{\pgfqpoint{4.791754in}{2.394558in}}%
\pgfpathlineto{\pgfqpoint{4.792185in}{2.334355in}}%
\pgfpathlineto{\pgfqpoint{4.791927in}{2.435756in}}%
\pgfpathlineto{\pgfqpoint{4.792887in}{2.379031in}}%
\pgfpathlineto{\pgfqpoint{4.793588in}{2.426324in}}%
\pgfpathlineto{\pgfqpoint{4.793293in}{2.319988in}}%
\pgfpathlineto{\pgfqpoint{4.793896in}{2.350514in}}%
\pgfpathlineto{\pgfqpoint{4.793908in}{2.350688in}}%
\pgfpathlineto{\pgfqpoint{4.793933in}{2.339417in}}%
\pgfpathlineto{\pgfqpoint{4.794056in}{2.323342in}}%
\pgfpathlineto{\pgfqpoint{4.794228in}{2.426970in}}%
\pgfpathlineto{\pgfqpoint{4.794979in}{2.394944in}}%
\pgfpathlineto{\pgfqpoint{4.795028in}{2.439682in}}%
\pgfpathlineto{\pgfqpoint{4.795287in}{2.325619in}}%
\pgfpathlineto{\pgfqpoint{4.796087in}{2.396701in}}%
\pgfpathlineto{\pgfqpoint{4.797170in}{2.312134in}}%
\pgfpathlineto{\pgfqpoint{4.796690in}{2.428974in}}%
\pgfpathlineto{\pgfqpoint{4.797281in}{2.384251in}}%
\pgfpathlineto{\pgfqpoint{4.798130in}{2.449780in}}%
\pgfpathlineto{\pgfqpoint{4.797687in}{2.321521in}}%
\pgfpathlineto{\pgfqpoint{4.798364in}{2.354483in}}%
\pgfpathlineto{\pgfqpoint{4.799496in}{2.315860in}}%
\pgfpathlineto{\pgfqpoint{4.798548in}{2.435545in}}%
\pgfpathlineto{\pgfqpoint{4.799508in}{2.316249in}}%
\pgfpathlineto{\pgfqpoint{4.800616in}{2.442312in}}%
\pgfpathlineto{\pgfqpoint{4.800271in}{2.311704in}}%
\pgfpathlineto{\pgfqpoint{4.800714in}{2.338053in}}%
\pgfpathlineto{\pgfqpoint{4.800788in}{2.323237in}}%
\pgfpathlineto{\pgfqpoint{4.801133in}{2.403248in}}%
\pgfpathlineto{\pgfqpoint{4.801182in}{2.395462in}}%
\pgfpathlineto{\pgfqpoint{4.801231in}{2.442147in}}%
\pgfpathlineto{\pgfqpoint{4.801502in}{2.326808in}}%
\pgfpathlineto{\pgfqpoint{4.802290in}{2.401651in}}%
\pgfpathlineto{\pgfqpoint{4.803360in}{2.309729in}}%
\pgfpathlineto{\pgfqpoint{4.802979in}{2.430179in}}%
\pgfpathlineto{\pgfqpoint{4.803447in}{2.371518in}}%
\pgfpathlineto{\pgfqpoint{4.804333in}{2.439887in}}%
\pgfpathlineto{\pgfqpoint{4.803877in}{2.317286in}}%
\pgfpathlineto{\pgfqpoint{4.804493in}{2.346340in}}%
\pgfpathlineto{\pgfqpoint{4.804505in}{2.346337in}}%
\pgfpathlineto{\pgfqpoint{4.804751in}{2.426044in}}%
\pgfpathlineto{\pgfqpoint{4.804616in}{2.329883in}}%
\pgfpathlineto{\pgfqpoint{4.805625in}{2.353991in}}%
\pgfpathlineto{\pgfqpoint{4.806462in}{2.303881in}}%
\pgfpathlineto{\pgfqpoint{4.806634in}{2.426688in}}%
\pgfpathlineto{\pgfqpoint{4.806647in}{2.425934in}}%
\pgfpathlineto{\pgfqpoint{4.806991in}{2.316244in}}%
\pgfpathlineto{\pgfqpoint{4.807434in}{2.438115in}}%
\pgfpathlineto{\pgfqpoint{4.807840in}{2.411583in}}%
\pgfpathlineto{\pgfqpoint{4.807865in}{2.427064in}}%
\pgfpathlineto{\pgfqpoint{4.808850in}{2.303597in}}%
\pgfpathlineto{\pgfqpoint{4.808899in}{2.356811in}}%
\pgfpathlineto{\pgfqpoint{4.809576in}{2.299692in}}%
\pgfpathlineto{\pgfqpoint{4.809157in}{2.441044in}}%
\pgfpathlineto{\pgfqpoint{4.809871in}{2.400252in}}%
\pgfpathlineto{\pgfqpoint{4.810967in}{2.433903in}}%
\pgfpathlineto{\pgfqpoint{4.810093in}{2.321441in}}%
\pgfpathlineto{\pgfqpoint{4.810991in}{2.407671in}}%
\pgfpathlineto{\pgfqpoint{4.811939in}{2.294050in}}%
\pgfpathlineto{\pgfqpoint{4.811754in}{2.427887in}}%
\pgfpathlineto{\pgfqpoint{4.812099in}{2.400126in}}%
\pgfpathlineto{\pgfqpoint{4.812271in}{2.446034in}}%
\pgfpathlineto{\pgfqpoint{4.812677in}{2.305465in}}%
\pgfpathlineto{\pgfqpoint{4.813120in}{2.357464in}}%
\pgfpathlineto{\pgfqpoint{4.813194in}{2.321637in}}%
\pgfpathlineto{\pgfqpoint{4.814093in}{2.427748in}}%
\pgfpathlineto{\pgfqpoint{4.814191in}{2.382883in}}%
\pgfpathlineto{\pgfqpoint{4.815373in}{2.433529in}}%
\pgfpathlineto{\pgfqpoint{4.815040in}{2.290824in}}%
\pgfpathlineto{\pgfqpoint{4.815385in}{2.423946in}}%
\pgfpathlineto{\pgfqpoint{4.815791in}{2.322901in}}%
\pgfpathlineto{\pgfqpoint{4.815988in}{2.441120in}}%
\pgfpathlineto{\pgfqpoint{4.816517in}{2.402597in}}%
\pgfpathlineto{\pgfqpoint{4.817219in}{2.438282in}}%
\pgfpathlineto{\pgfqpoint{4.816911in}{2.326157in}}%
\pgfpathlineto{\pgfqpoint{4.817576in}{2.377021in}}%
\pgfpathlineto{\pgfqpoint{4.818142in}{2.295623in}}%
\pgfpathlineto{\pgfqpoint{4.817847in}{2.442376in}}%
\pgfpathlineto{\pgfqpoint{4.818720in}{2.342008in}}%
\pgfpathlineto{\pgfqpoint{4.818757in}{2.334461in}}%
\pgfpathlineto{\pgfqpoint{4.819077in}{2.428775in}}%
\pgfpathlineto{\pgfqpoint{4.819533in}{2.382626in}}%
\pgfpathlineto{\pgfqpoint{4.820320in}{2.422902in}}%
\pgfpathlineto{\pgfqpoint{4.820567in}{2.321302in}}%
\pgfpathlineto{\pgfqpoint{4.820603in}{2.334465in}}%
\pgfpathlineto{\pgfqpoint{4.820616in}{2.334336in}}%
\pgfpathlineto{\pgfqpoint{4.820628in}{2.335804in}}%
\pgfpathlineto{\pgfqpoint{4.820936in}{2.443431in}}%
\pgfpathlineto{\pgfqpoint{4.821243in}{2.306396in}}%
\pgfpathlineto{\pgfqpoint{4.821773in}{2.373115in}}%
\pgfpathlineto{\pgfqpoint{4.821859in}{2.318760in}}%
\pgfpathlineto{\pgfqpoint{4.822179in}{2.434120in}}%
\pgfpathlineto{\pgfqpoint{4.822880in}{2.361158in}}%
\pgfpathlineto{\pgfqpoint{4.824037in}{2.443104in}}%
\pgfpathlineto{\pgfqpoint{4.823705in}{2.311902in}}%
\pgfpathlineto{\pgfqpoint{4.824173in}{2.400489in}}%
\pgfpathlineto{\pgfqpoint{4.824960in}{2.323909in}}%
\pgfpathlineto{\pgfqpoint{4.824579in}{2.428469in}}%
\pgfpathlineto{\pgfqpoint{4.825268in}{2.398351in}}%
\pgfpathlineto{\pgfqpoint{4.825391in}{2.427861in}}%
\pgfpathlineto{\pgfqpoint{4.825563in}{2.332135in}}%
\pgfpathlineto{\pgfqpoint{4.826376in}{2.395663in}}%
\pgfpathlineto{\pgfqpoint{4.826437in}{2.438868in}}%
\pgfpathlineto{\pgfqpoint{4.826807in}{2.298802in}}%
\pgfpathlineto{\pgfqpoint{4.827373in}{2.360294in}}%
\pgfpathlineto{\pgfqpoint{4.827422in}{2.323343in}}%
\pgfpathlineto{\pgfqpoint{4.827754in}{2.433455in}}%
\pgfpathlineto{\pgfqpoint{4.828431in}{2.373323in}}%
\pgfpathlineto{\pgfqpoint{4.828493in}{2.437590in}}%
\pgfpathlineto{\pgfqpoint{4.828665in}{2.326025in}}%
\pgfpathlineto{\pgfqpoint{4.829600in}{2.418776in}}%
\pgfpathlineto{\pgfqpoint{4.830351in}{2.439503in}}%
\pgfpathlineto{\pgfqpoint{4.829908in}{2.305273in}}%
\pgfpathlineto{\pgfqpoint{4.830474in}{2.352642in}}%
\pgfpathlineto{\pgfqpoint{4.830487in}{2.352690in}}%
\pgfpathlineto{\pgfqpoint{4.830499in}{2.348545in}}%
\pgfpathlineto{\pgfqpoint{4.830523in}{2.326715in}}%
\pgfpathlineto{\pgfqpoint{4.830856in}{2.444555in}}%
\pgfpathlineto{\pgfqpoint{4.831557in}{2.378172in}}%
\pgfpathlineto{\pgfqpoint{4.831594in}{2.440749in}}%
\pgfpathlineto{\pgfqpoint{4.832271in}{2.324708in}}%
\pgfpathlineto{\pgfqpoint{4.832690in}{2.426819in}}%
\pgfpathlineto{\pgfqpoint{4.833453in}{2.443667in}}%
\pgfpathlineto{\pgfqpoint{4.832997in}{2.299954in}}%
\pgfpathlineto{\pgfqpoint{4.833563in}{2.349633in}}%
\pgfpathlineto{\pgfqpoint{4.833625in}{2.334815in}}%
\pgfpathlineto{\pgfqpoint{4.833957in}{2.450657in}}%
\pgfpathlineto{\pgfqpoint{4.834450in}{2.397669in}}%
\pgfpathlineto{\pgfqpoint{4.834696in}{2.430164in}}%
\pgfpathlineto{\pgfqpoint{4.835360in}{2.324769in}}%
\pgfpathlineto{\pgfqpoint{4.835533in}{2.383686in}}%
\pgfpathlineto{\pgfqpoint{4.836099in}{2.297527in}}%
\pgfpathlineto{\pgfqpoint{4.835803in}{2.441847in}}%
\pgfpathlineto{\pgfqpoint{4.836702in}{2.327952in}}%
\pgfpathlineto{\pgfqpoint{4.837059in}{2.444932in}}%
\pgfpathlineto{\pgfqpoint{4.837896in}{2.371512in}}%
\pgfpathlineto{\pgfqpoint{4.838462in}{2.319016in}}%
\pgfpathlineto{\pgfqpoint{4.838917in}{2.453412in}}%
\pgfpathlineto{\pgfqpoint{4.838991in}{2.362357in}}%
\pgfpathlineto{\pgfqpoint{4.839422in}{2.444482in}}%
\pgfpathlineto{\pgfqpoint{4.839200in}{2.309345in}}%
\pgfpathlineto{\pgfqpoint{4.840123in}{2.412518in}}%
\pgfpathlineto{\pgfqpoint{4.840886in}{2.428601in}}%
\pgfpathlineto{\pgfqpoint{4.841046in}{2.316230in}}%
\pgfpathlineto{\pgfqpoint{4.841120in}{2.348098in}}%
\pgfpathlineto{\pgfqpoint{4.841637in}{2.307271in}}%
\pgfpathlineto{\pgfqpoint{4.841268in}{2.440448in}}%
\pgfpathlineto{\pgfqpoint{4.841970in}{2.407580in}}%
\pgfpathlineto{\pgfqpoint{4.842006in}{2.453749in}}%
\pgfpathlineto{\pgfqpoint{4.842893in}{2.308438in}}%
\pgfpathlineto{\pgfqpoint{4.843028in}{2.354435in}}%
\pgfpathlineto{\pgfqpoint{4.843348in}{2.455062in}}%
\pgfpathlineto{\pgfqpoint{4.843508in}{2.323728in}}%
\pgfpathlineto{\pgfqpoint{4.844123in}{2.345216in}}%
\pgfpathlineto{\pgfqpoint{4.844751in}{2.288335in}}%
\pgfpathlineto{\pgfqpoint{4.845108in}{2.449025in}}%
\pgfpathlineto{\pgfqpoint{4.845170in}{2.397229in}}%
\pgfpathlineto{\pgfqpoint{4.845231in}{2.407113in}}%
\pgfpathlineto{\pgfqpoint{4.845366in}{2.294769in}}%
\pgfpathlineto{\pgfqpoint{4.845834in}{2.445397in}}%
\pgfpathlineto{\pgfqpoint{4.846326in}{2.393107in}}%
\pgfpathlineto{\pgfqpoint{4.846450in}{2.473795in}}%
\pgfpathlineto{\pgfqpoint{4.846597in}{2.312077in}}%
\pgfpathlineto{\pgfqpoint{4.847446in}{2.418724in}}%
\pgfpathlineto{\pgfqpoint{4.848308in}{2.466292in}}%
\pgfpathlineto{\pgfqpoint{4.847853in}{2.286722in}}%
\pgfpathlineto{\pgfqpoint{4.848443in}{2.331750in}}%
\pgfpathlineto{\pgfqpoint{4.848468in}{2.308554in}}%
\pgfpathlineto{\pgfqpoint{4.848813in}{2.448421in}}%
\pgfpathlineto{\pgfqpoint{4.849502in}{2.381166in}}%
\pgfpathlineto{\pgfqpoint{4.849551in}{2.466086in}}%
\pgfpathlineto{\pgfqpoint{4.849699in}{2.300309in}}%
\pgfpathlineto{\pgfqpoint{4.850634in}{2.429094in}}%
\pgfpathlineto{\pgfqpoint{4.850930in}{2.281262in}}%
\pgfpathlineto{\pgfqpoint{4.851410in}{2.489784in}}%
\pgfpathlineto{\pgfqpoint{4.851791in}{2.373771in}}%
\pgfpathlineto{\pgfqpoint{4.851926in}{2.469595in}}%
\pgfpathlineto{\pgfqpoint{4.852788in}{2.305770in}}%
\pgfpathlineto{\pgfqpoint{4.852874in}{2.348549in}}%
\pgfpathlineto{\pgfqpoint{4.853317in}{2.309745in}}%
\pgfpathlineto{\pgfqpoint{4.853157in}{2.439116in}}%
\pgfpathlineto{\pgfqpoint{4.853674in}{2.418653in}}%
\pgfpathlineto{\pgfqpoint{4.854511in}{2.475728in}}%
\pgfpathlineto{\pgfqpoint{4.854031in}{2.282775in}}%
\pgfpathlineto{\pgfqpoint{4.854683in}{2.340927in}}%
\pgfpathlineto{\pgfqpoint{4.854806in}{2.324115in}}%
\pgfpathlineto{\pgfqpoint{4.854905in}{2.377549in}}%
\pgfpathlineto{\pgfqpoint{4.855028in}{2.465713in}}%
\pgfpathlineto{\pgfqpoint{4.855890in}{2.307806in}}%
\pgfpathlineto{\pgfqpoint{4.855988in}{2.344234in}}%
\pgfpathlineto{\pgfqpoint{4.856000in}{2.343788in}}%
\pgfpathlineto{\pgfqpoint{4.856062in}{2.377452in}}%
\pgfpathlineto{\pgfqpoint{4.856874in}{2.474231in}}%
\pgfpathlineto{\pgfqpoint{4.856419in}{2.308169in}}%
\pgfpathlineto{\pgfqpoint{4.857096in}{2.330441in}}%
\pgfpathlineto{\pgfqpoint{4.857170in}{2.295618in}}%
\pgfpathlineto{\pgfqpoint{4.857600in}{2.461959in}}%
\pgfpathlineto{\pgfqpoint{4.858093in}{2.402186in}}%
\pgfpathlineto{\pgfqpoint{4.858130in}{2.449812in}}%
\pgfpathlineto{\pgfqpoint{4.859003in}{2.317711in}}%
\pgfpathlineto{\pgfqpoint{4.859200in}{2.416881in}}%
\pgfpathlineto{\pgfqpoint{4.859348in}{2.448503in}}%
\pgfpathlineto{\pgfqpoint{4.859496in}{2.352528in}}%
\pgfpathlineto{\pgfqpoint{4.860333in}{2.299706in}}%
\pgfpathlineto{\pgfqpoint{4.859976in}{2.473297in}}%
\pgfpathlineto{\pgfqpoint{4.860566in}{2.418235in}}%
\pgfpathlineto{\pgfqpoint{4.860702in}{2.455033in}}%
\pgfpathlineto{\pgfqpoint{4.860837in}{2.311682in}}%
\pgfpathlineto{\pgfqpoint{4.861588in}{2.335994in}}%
\pgfpathlineto{\pgfqpoint{4.862117in}{2.315258in}}%
\pgfpathlineto{\pgfqpoint{4.862326in}{2.427846in}}%
\pgfpathlineto{\pgfqpoint{4.862523in}{2.396249in}}%
\pgfpathlineto{\pgfqpoint{4.863594in}{2.460145in}}%
\pgfpathlineto{\pgfqpoint{4.863385in}{2.291722in}}%
\pgfpathlineto{\pgfqpoint{4.863631in}{2.421706in}}%
\pgfpathlineto{\pgfqpoint{4.864579in}{2.311602in}}%
\pgfpathlineto{\pgfqpoint{4.864406in}{2.468384in}}%
\pgfpathlineto{\pgfqpoint{4.864776in}{2.373352in}}%
\pgfpathlineto{\pgfqpoint{4.865551in}{2.427903in}}%
\pgfpathlineto{\pgfqpoint{4.865723in}{2.316334in}}%
\pgfpathlineto{\pgfqpoint{4.865822in}{2.326102in}}%
\pgfpathlineto{\pgfqpoint{4.866449in}{2.265700in}}%
\pgfpathlineto{\pgfqpoint{4.865969in}{2.439587in}}%
\pgfpathlineto{\pgfqpoint{4.866671in}{2.402992in}}%
\pgfpathlineto{\pgfqpoint{4.867533in}{2.479306in}}%
\pgfpathlineto{\pgfqpoint{4.867176in}{2.283341in}}%
\pgfpathlineto{\pgfqpoint{4.867754in}{2.344385in}}%
\pgfpathlineto{\pgfqpoint{4.868271in}{2.425363in}}%
\pgfpathlineto{\pgfqpoint{4.868443in}{2.323772in}}%
\pgfpathlineto{\pgfqpoint{4.868456in}{2.320940in}}%
\pgfpathlineto{\pgfqpoint{4.868800in}{2.414573in}}%
\pgfpathlineto{\pgfqpoint{4.869280in}{2.375896in}}%
\pgfpathlineto{\pgfqpoint{4.870351in}{2.413058in}}%
\pgfpathlineto{\pgfqpoint{4.869576in}{2.325000in}}%
\pgfpathlineto{\pgfqpoint{4.870388in}{2.382869in}}%
\pgfpathlineto{\pgfqpoint{4.871225in}{2.335276in}}%
\pgfpathlineto{\pgfqpoint{4.870880in}{2.413218in}}%
\pgfpathlineto{\pgfqpoint{4.871471in}{2.405697in}}%
\pgfpathlineto{\pgfqpoint{4.872222in}{2.422898in}}%
\pgfpathlineto{\pgfqpoint{4.871816in}{2.342907in}}%
\pgfpathlineto{\pgfqpoint{4.872431in}{2.361484in}}%
\pgfpathlineto{\pgfqpoint{4.872542in}{2.354337in}}%
\pgfpathlineto{\pgfqpoint{4.873145in}{2.337972in}}%
\pgfpathlineto{\pgfqpoint{4.872985in}{2.411991in}}%
\pgfpathlineto{\pgfqpoint{4.873613in}{2.365908in}}%
\pgfpathlineto{\pgfqpoint{4.874511in}{2.420175in}}%
\pgfpathlineto{\pgfqpoint{4.874166in}{2.331494in}}%
\pgfpathlineto{\pgfqpoint{4.874696in}{2.360929in}}%
\pgfpathlineto{\pgfqpoint{4.875693in}{2.298483in}}%
\pgfpathlineto{\pgfqpoint{4.875225in}{2.425403in}}%
\pgfpathlineto{\pgfqpoint{4.875742in}{2.369294in}}%
\pgfpathlineto{\pgfqpoint{4.876456in}{2.439963in}}%
\pgfpathlineto{\pgfqpoint{4.876308in}{2.330559in}}%
\pgfpathlineto{\pgfqpoint{4.876776in}{2.332738in}}%
\pgfpathlineto{\pgfqpoint{4.876800in}{2.310094in}}%
\pgfpathlineto{\pgfqpoint{4.877354in}{2.430021in}}%
\pgfpathlineto{\pgfqpoint{4.877834in}{2.358194in}}%
\pgfpathlineto{\pgfqpoint{4.878573in}{2.439538in}}%
\pgfpathlineto{\pgfqpoint{4.878917in}{2.305393in}}%
\pgfpathlineto{\pgfqpoint{4.879274in}{2.448763in}}%
\pgfpathlineto{\pgfqpoint{4.879889in}{2.273705in}}%
\pgfpathlineto{\pgfqpoint{4.880271in}{2.387889in}}%
\pgfpathlineto{\pgfqpoint{4.881133in}{2.268675in}}%
\pgfpathlineto{\pgfqpoint{4.880665in}{2.478555in}}%
\pgfpathlineto{\pgfqpoint{4.881354in}{2.466285in}}%
\pgfpathlineto{\pgfqpoint{4.881366in}{2.492062in}}%
\pgfpathlineto{\pgfqpoint{4.881834in}{2.231261in}}%
\pgfpathlineto{\pgfqpoint{4.882412in}{2.382640in}}%
\pgfpathlineto{\pgfqpoint{4.882511in}{2.289097in}}%
\pgfpathlineto{\pgfqpoint{4.883372in}{2.448824in}}%
\pgfpathlineto{\pgfqpoint{4.883409in}{2.459455in}}%
\pgfpathlineto{\pgfqpoint{4.883754in}{2.265605in}}%
\pgfpathlineto{\pgfqpoint{4.884406in}{2.418490in}}%
\pgfpathlineto{\pgfqpoint{4.884665in}{2.268890in}}%
\pgfpathlineto{\pgfqpoint{4.885403in}{2.451042in}}%
\pgfpathlineto{\pgfqpoint{4.885514in}{2.393221in}}%
\pgfpathlineto{\pgfqpoint{4.886376in}{2.543935in}}%
\pgfpathlineto{\pgfqpoint{4.885772in}{2.277908in}}%
\pgfpathlineto{\pgfqpoint{4.886585in}{2.356749in}}%
\pgfpathlineto{\pgfqpoint{4.886732in}{2.167032in}}%
\pgfpathlineto{\pgfqpoint{4.887102in}{2.466885in}}%
\pgfpathlineto{\pgfqpoint{4.887717in}{2.316421in}}%
\pgfpathlineto{\pgfqpoint{4.888308in}{2.573402in}}%
\pgfpathlineto{\pgfqpoint{4.888628in}{2.244715in}}%
\pgfpathlineto{\pgfqpoint{4.888665in}{2.185105in}}%
\pgfpathlineto{\pgfqpoint{4.889256in}{2.472481in}}%
\pgfpathlineto{\pgfqpoint{4.889686in}{2.333128in}}%
\pgfpathlineto{\pgfqpoint{4.890449in}{2.456466in}}%
\pgfpathlineto{\pgfqpoint{4.890634in}{2.288372in}}%
\pgfpathlineto{\pgfqpoint{4.890819in}{2.371262in}}%
\pgfpathlineto{\pgfqpoint{4.891569in}{2.198523in}}%
\pgfpathlineto{\pgfqpoint{4.891200in}{2.531931in}}%
\pgfpathlineto{\pgfqpoint{4.891902in}{2.417663in}}%
\pgfpathlineto{\pgfqpoint{4.892197in}{2.472512in}}%
\pgfpathlineto{\pgfqpoint{4.892517in}{2.324884in}}%
\pgfpathlineto{\pgfqpoint{4.893551in}{2.166366in}}%
\pgfpathlineto{\pgfqpoint{4.893194in}{2.575045in}}%
\pgfpathlineto{\pgfqpoint{4.893649in}{2.263252in}}%
\pgfpathlineto{\pgfqpoint{4.894117in}{2.537425in}}%
\pgfpathlineto{\pgfqpoint{4.894523in}{2.220901in}}%
\pgfpathlineto{\pgfqpoint{4.894905in}{2.311971in}}%
\pgfpathlineto{\pgfqpoint{4.895496in}{2.228565in}}%
\pgfpathlineto{\pgfqpoint{4.895311in}{2.516423in}}%
\pgfpathlineto{\pgfqpoint{4.895939in}{2.364041in}}%
\pgfpathlineto{\pgfqpoint{4.896062in}{2.528756in}}%
\pgfpathlineto{\pgfqpoint{4.896468in}{2.202244in}}%
\pgfpathlineto{\pgfqpoint{4.897083in}{2.470003in}}%
\pgfpathlineto{\pgfqpoint{4.897096in}{2.470623in}}%
\pgfpathlineto{\pgfqpoint{4.897182in}{2.432298in}}%
\pgfpathlineto{\pgfqpoint{4.898449in}{2.214277in}}%
\pgfpathlineto{\pgfqpoint{4.898142in}{2.533874in}}%
\pgfpathlineto{\pgfqpoint{4.898462in}{2.214793in}}%
\pgfpathlineto{\pgfqpoint{4.898991in}{2.556700in}}%
\pgfpathlineto{\pgfqpoint{4.899336in}{2.191328in}}%
\pgfpathlineto{\pgfqpoint{4.899705in}{2.402089in}}%
\pgfpathlineto{\pgfqpoint{4.900332in}{2.255743in}}%
\pgfpathlineto{\pgfqpoint{4.900086in}{2.486159in}}%
\pgfpathlineto{\pgfqpoint{4.900825in}{2.376975in}}%
\pgfpathlineto{\pgfqpoint{4.900936in}{2.563054in}}%
\pgfpathlineto{\pgfqpoint{4.901329in}{2.234314in}}%
\pgfpathlineto{\pgfqpoint{4.901969in}{2.451601in}}%
\pgfpathlineto{\pgfqpoint{4.903015in}{2.494143in}}%
\pgfpathlineto{\pgfqpoint{4.902240in}{2.229622in}}%
\pgfpathlineto{\pgfqpoint{4.903102in}{2.474335in}}%
\pgfpathlineto{\pgfqpoint{4.903175in}{2.380317in}}%
\pgfpathlineto{\pgfqpoint{4.904222in}{2.166672in}}%
\pgfpathlineto{\pgfqpoint{4.903865in}{2.549569in}}%
\pgfpathlineto{\pgfqpoint{4.904332in}{2.288697in}}%
\pgfpathlineto{\pgfqpoint{4.905071in}{2.471564in}}%
\pgfpathlineto{\pgfqpoint{4.905305in}{2.279348in}}%
\pgfpathlineto{\pgfqpoint{4.905440in}{2.308677in}}%
\pgfpathlineto{\pgfqpoint{4.905822in}{2.540126in}}%
\pgfpathlineto{\pgfqpoint{4.906154in}{2.207818in}}%
\pgfpathlineto{\pgfqpoint{4.906929in}{2.420364in}}%
\pgfpathlineto{\pgfqpoint{4.907902in}{2.471563in}}%
\pgfpathlineto{\pgfqpoint{4.908148in}{2.269960in}}%
\pgfpathlineto{\pgfqpoint{4.908751in}{2.542734in}}%
\pgfpathlineto{\pgfqpoint{4.909059in}{2.251137in}}%
\pgfpathlineto{\pgfqpoint{4.909108in}{2.201544in}}%
\pgfpathlineto{\pgfqpoint{4.909280in}{2.446569in}}%
\pgfpathlineto{\pgfqpoint{4.910105in}{2.317448in}}%
\pgfpathlineto{\pgfqpoint{4.910695in}{2.548896in}}%
\pgfpathlineto{\pgfqpoint{4.911052in}{2.218890in}}%
\pgfpathlineto{\pgfqpoint{4.911274in}{2.363982in}}%
\pgfpathlineto{\pgfqpoint{4.912025in}{2.262308in}}%
\pgfpathlineto{\pgfqpoint{4.911668in}{2.465918in}}%
\pgfpathlineto{\pgfqpoint{4.912455in}{2.307553in}}%
\pgfpathlineto{\pgfqpoint{4.913637in}{2.521511in}}%
\pgfpathlineto{\pgfqpoint{4.913034in}{2.292348in}}%
\pgfpathlineto{\pgfqpoint{4.913649in}{2.511272in}}%
\pgfpathlineto{\pgfqpoint{4.913994in}{2.191941in}}%
\pgfpathlineto{\pgfqpoint{4.914843in}{2.365281in}}%
\pgfpathlineto{\pgfqpoint{4.915594in}{2.548127in}}%
\pgfpathlineto{\pgfqpoint{4.915385in}{2.279845in}}%
\pgfpathlineto{\pgfqpoint{4.915852in}{2.342661in}}%
\pgfpathlineto{\pgfqpoint{4.915951in}{2.193840in}}%
\pgfpathlineto{\pgfqpoint{4.916554in}{2.472969in}}%
\pgfpathlineto{\pgfqpoint{4.916972in}{2.312695in}}%
\pgfpathlineto{\pgfqpoint{4.917699in}{2.463507in}}%
\pgfpathlineto{\pgfqpoint{4.917895in}{2.296176in}}%
\pgfpathlineto{\pgfqpoint{4.918105in}{2.388376in}}%
\pgfpathlineto{\pgfqpoint{4.918117in}{2.388822in}}%
\pgfpathlineto{\pgfqpoint{4.918154in}{2.355398in}}%
\pgfpathlineto{\pgfqpoint{4.918905in}{2.213510in}}%
\pgfpathlineto{\pgfqpoint{4.918535in}{2.508370in}}%
\pgfpathlineto{\pgfqpoint{4.919249in}{2.383127in}}%
\pgfpathlineto{\pgfqpoint{4.919520in}{2.459490in}}%
\pgfpathlineto{\pgfqpoint{4.920185in}{2.297997in}}%
\pgfpathlineto{\pgfqpoint{4.920259in}{2.332132in}}%
\pgfpathlineto{\pgfqpoint{4.920886in}{2.231653in}}%
\pgfpathlineto{\pgfqpoint{4.920529in}{2.515013in}}%
\pgfpathlineto{\pgfqpoint{4.921305in}{2.394944in}}%
\pgfpathlineto{\pgfqpoint{4.921452in}{2.477613in}}%
\pgfpathlineto{\pgfqpoint{4.921809in}{2.255673in}}%
\pgfpathlineto{\pgfqpoint{4.922252in}{2.331123in}}%
\pgfpathlineto{\pgfqpoint{4.923151in}{2.300685in}}%
\pgfpathlineto{\pgfqpoint{4.922622in}{2.458704in}}%
\pgfpathlineto{\pgfqpoint{4.923298in}{2.368071in}}%
\pgfpathlineto{\pgfqpoint{4.923409in}{2.483477in}}%
\pgfpathlineto{\pgfqpoint{4.923828in}{2.235148in}}%
\pgfpathlineto{\pgfqpoint{4.924418in}{2.411486in}}%
\pgfpathlineto{\pgfqpoint{4.924455in}{2.435372in}}%
\pgfpathlineto{\pgfqpoint{4.925194in}{2.310095in}}%
\pgfpathlineto{\pgfqpoint{4.925785in}{2.262153in}}%
\pgfpathlineto{\pgfqpoint{4.925452in}{2.482875in}}%
\pgfpathlineto{\pgfqpoint{4.926129in}{2.385282in}}%
\pgfpathlineto{\pgfqpoint{4.926351in}{2.486661in}}%
\pgfpathlineto{\pgfqpoint{4.926695in}{2.266377in}}%
\pgfpathlineto{\pgfqpoint{4.927151in}{2.330406in}}%
\pgfpathlineto{\pgfqpoint{4.927163in}{2.329402in}}%
\pgfpathlineto{\pgfqpoint{4.927274in}{2.407590in}}%
\pgfpathlineto{\pgfqpoint{4.927569in}{2.461130in}}%
\pgfpathlineto{\pgfqpoint{4.928074in}{2.300125in}}%
\pgfpathlineto{\pgfqpoint{4.928788in}{2.232464in}}%
\pgfpathlineto{\pgfqpoint{4.928480in}{2.495350in}}%
\pgfpathlineto{\pgfqpoint{4.929083in}{2.369145in}}%
\pgfpathlineto{\pgfqpoint{4.929625in}{2.275456in}}%
\pgfpathlineto{\pgfqpoint{4.929206in}{2.464570in}}%
\pgfpathlineto{\pgfqpoint{4.930215in}{2.356419in}}%
\pgfpathlineto{\pgfqpoint{4.931274in}{2.494226in}}%
\pgfpathlineto{\pgfqpoint{4.930757in}{2.251925in}}%
\pgfpathlineto{\pgfqpoint{4.931360in}{2.421040in}}%
\pgfpathlineto{\pgfqpoint{4.931409in}{2.429075in}}%
\pgfpathlineto{\pgfqpoint{4.931520in}{2.391363in}}%
\pgfpathlineto{\pgfqpoint{4.931618in}{2.217756in}}%
\pgfpathlineto{\pgfqpoint{4.932492in}{2.453510in}}%
\pgfpathlineto{\pgfqpoint{4.932665in}{2.323361in}}%
\pgfpathlineto{\pgfqpoint{4.933391in}{2.513662in}}%
\pgfpathlineto{\pgfqpoint{4.933588in}{2.213725in}}%
\pgfpathlineto{\pgfqpoint{4.933797in}{2.362522in}}%
\pgfpathlineto{\pgfqpoint{4.934560in}{2.249800in}}%
\pgfpathlineto{\pgfqpoint{4.934105in}{2.457093in}}%
\pgfpathlineto{\pgfqpoint{4.934745in}{2.418439in}}%
\pgfpathlineto{\pgfqpoint{4.935348in}{2.518460in}}%
\pgfpathlineto{\pgfqpoint{4.934991in}{2.249761in}}%
\pgfpathlineto{\pgfqpoint{4.935766in}{2.350367in}}%
\pgfpathlineto{\pgfqpoint{4.936542in}{2.222890in}}%
\pgfpathlineto{\pgfqpoint{4.936209in}{2.500855in}}%
\pgfpathlineto{\pgfqpoint{4.936837in}{2.402476in}}%
\pgfpathlineto{\pgfqpoint{4.937945in}{2.276646in}}%
\pgfpathlineto{\pgfqpoint{4.937157in}{2.441150in}}%
\pgfpathlineto{\pgfqpoint{4.938031in}{2.380980in}}%
\pgfpathlineto{\pgfqpoint{4.938314in}{2.505317in}}%
\pgfpathlineto{\pgfqpoint{4.938498in}{2.226658in}}%
\pgfpathlineto{\pgfqpoint{4.939175in}{2.411374in}}%
\pgfpathlineto{\pgfqpoint{4.939471in}{2.237348in}}%
\pgfpathlineto{\pgfqpoint{4.940222in}{2.482434in}}%
\pgfpathlineto{\pgfqpoint{4.940246in}{2.516525in}}%
\pgfpathlineto{\pgfqpoint{4.940480in}{2.304155in}}%
\pgfpathlineto{\pgfqpoint{4.941280in}{2.391897in}}%
\pgfpathlineto{\pgfqpoint{4.941440in}{2.213804in}}%
\pgfpathlineto{\pgfqpoint{4.941649in}{2.436444in}}%
\pgfpathlineto{\pgfqpoint{4.942425in}{2.320245in}}%
\pgfpathlineto{\pgfqpoint{4.943052in}{2.507170in}}%
\pgfpathlineto{\pgfqpoint{4.943360in}{2.272458in}}%
\pgfpathlineto{\pgfqpoint{4.943409in}{2.238857in}}%
\pgfpathlineto{\pgfqpoint{4.944025in}{2.468358in}}%
\pgfpathlineto{\pgfqpoint{4.944418in}{2.303952in}}%
\pgfpathlineto{\pgfqpoint{4.945157in}{2.505871in}}%
\pgfpathlineto{\pgfqpoint{4.944812in}{2.286941in}}%
\pgfpathlineto{\pgfqpoint{4.945538in}{2.335926in}}%
\pgfpathlineto{\pgfqpoint{4.946031in}{2.492571in}}%
\pgfpathlineto{\pgfqpoint{4.946314in}{2.228769in}}%
\pgfpathlineto{\pgfqpoint{4.946831in}{2.408852in}}%
\pgfpathlineto{\pgfqpoint{4.947717in}{2.283729in}}%
\pgfpathlineto{\pgfqpoint{4.946978in}{2.420014in}}%
\pgfpathlineto{\pgfqpoint{4.947914in}{2.419254in}}%
\pgfpathlineto{\pgfqpoint{4.947963in}{2.508842in}}%
\pgfpathlineto{\pgfqpoint{4.948320in}{2.253013in}}%
\pgfpathlineto{\pgfqpoint{4.949009in}{2.397412in}}%
\pgfpathlineto{\pgfqpoint{4.949206in}{2.256098in}}%
\pgfpathlineto{\pgfqpoint{4.950018in}{2.429488in}}%
\pgfpathlineto{\pgfqpoint{4.950055in}{2.497136in}}%
\pgfpathlineto{\pgfqpoint{4.950585in}{2.291508in}}%
\pgfpathlineto{\pgfqpoint{4.951101in}{2.376952in}}%
\pgfpathlineto{\pgfqpoint{4.951212in}{2.233815in}}%
\pgfpathlineto{\pgfqpoint{4.951520in}{2.438711in}}%
\pgfpathlineto{\pgfqpoint{4.952209in}{2.356725in}}%
\pgfpathlineto{\pgfqpoint{4.952898in}{2.490337in}}%
\pgfpathlineto{\pgfqpoint{4.952677in}{2.279135in}}%
\pgfpathlineto{\pgfqpoint{4.953145in}{2.298748in}}%
\pgfpathlineto{\pgfqpoint{4.954092in}{2.266805in}}%
\pgfpathlineto{\pgfqpoint{4.953735in}{2.461907in}}%
\pgfpathlineto{\pgfqpoint{4.954203in}{2.303520in}}%
\pgfpathlineto{\pgfqpoint{4.954954in}{2.489436in}}%
\pgfpathlineto{\pgfqpoint{4.955311in}{2.323958in}}%
\pgfpathlineto{\pgfqpoint{4.955471in}{2.276953in}}%
\pgfpathlineto{\pgfqpoint{4.955692in}{2.455180in}}%
\pgfpathlineto{\pgfqpoint{4.955803in}{2.495581in}}%
\pgfpathlineto{\pgfqpoint{4.956160in}{2.237181in}}%
\pgfpathlineto{\pgfqpoint{4.956603in}{2.372480in}}%
\pgfpathlineto{\pgfqpoint{4.957034in}{2.422068in}}%
\pgfpathlineto{\pgfqpoint{4.957378in}{2.332471in}}%
\pgfpathlineto{\pgfqpoint{4.957551in}{2.240111in}}%
\pgfpathlineto{\pgfqpoint{4.957895in}{2.514834in}}%
\pgfpathlineto{\pgfqpoint{4.958400in}{2.391265in}}%
\pgfpathlineto{\pgfqpoint{4.958597in}{2.442364in}}%
\pgfpathlineto{\pgfqpoint{4.959052in}{2.299464in}}%
\pgfpathlineto{\pgfqpoint{4.959458in}{2.315942in}}%
\pgfpathlineto{\pgfqpoint{4.960345in}{2.265174in}}%
\pgfpathlineto{\pgfqpoint{4.959815in}{2.506530in}}%
\pgfpathlineto{\pgfqpoint{4.960505in}{2.404656in}}%
\pgfpathlineto{\pgfqpoint{4.960689in}{2.482090in}}%
\pgfpathlineto{\pgfqpoint{4.961034in}{2.276468in}}%
\pgfpathlineto{\pgfqpoint{4.961563in}{2.318630in}}%
\pgfpathlineto{\pgfqpoint{4.962437in}{2.248280in}}%
\pgfpathlineto{\pgfqpoint{4.961908in}{2.472143in}}%
\pgfpathlineto{\pgfqpoint{4.962560in}{2.332244in}}%
\pgfpathlineto{\pgfqpoint{4.962781in}{2.487702in}}%
\pgfpathlineto{\pgfqpoint{4.963138in}{2.292169in}}%
\pgfpathlineto{\pgfqpoint{4.963680in}{2.361355in}}%
\pgfpathlineto{\pgfqpoint{4.964701in}{2.487285in}}%
\pgfpathlineto{\pgfqpoint{4.964517in}{2.267280in}}%
\pgfpathlineto{\pgfqpoint{4.964874in}{2.422758in}}%
\pgfpathlineto{\pgfqpoint{4.965883in}{2.289439in}}%
\pgfpathlineto{\pgfqpoint{4.965575in}{2.442219in}}%
\pgfpathlineto{\pgfqpoint{4.966031in}{2.354944in}}%
\pgfpathlineto{\pgfqpoint{4.966781in}{2.478294in}}%
\pgfpathlineto{\pgfqpoint{4.966449in}{2.285174in}}%
\pgfpathlineto{\pgfqpoint{4.967126in}{2.342038in}}%
\pgfpathlineto{\pgfqpoint{4.967311in}{2.264750in}}%
\pgfpathlineto{\pgfqpoint{4.967643in}{2.463016in}}%
\pgfpathlineto{\pgfqpoint{4.968148in}{2.402613in}}%
\pgfpathlineto{\pgfqpoint{4.969034in}{2.465109in}}%
\pgfpathlineto{\pgfqpoint{4.969218in}{2.322912in}}%
\pgfpathlineto{\pgfqpoint{4.969341in}{2.279942in}}%
\pgfpathlineto{\pgfqpoint{4.969563in}{2.476506in}}%
\pgfpathlineto{\pgfqpoint{4.970252in}{2.410939in}}%
\pgfpathlineto{\pgfqpoint{4.970966in}{2.476884in}}%
\pgfpathlineto{\pgfqpoint{4.970744in}{2.307410in}}%
\pgfpathlineto{\pgfqpoint{4.971274in}{2.329936in}}%
\pgfpathlineto{\pgfqpoint{4.971323in}{2.285293in}}%
\pgfpathlineto{\pgfqpoint{4.971828in}{2.444169in}}%
\pgfpathlineto{\pgfqpoint{4.972320in}{2.342074in}}%
\pgfpathlineto{\pgfqpoint{4.972529in}{2.445608in}}%
\pgfpathlineto{\pgfqpoint{4.972837in}{2.309239in}}%
\pgfpathlineto{\pgfqpoint{4.973415in}{2.328318in}}%
\pgfpathlineto{\pgfqpoint{4.973920in}{2.448927in}}%
\pgfpathlineto{\pgfqpoint{4.974252in}{2.288484in}}%
\pgfpathlineto{\pgfqpoint{4.974671in}{2.363607in}}%
\pgfpathlineto{\pgfqpoint{4.975064in}{2.320852in}}%
\pgfpathlineto{\pgfqpoint{4.975126in}{2.411225in}}%
\pgfpathlineto{\pgfqpoint{4.975532in}{2.378686in}}%
\pgfpathlineto{\pgfqpoint{4.975815in}{2.444194in}}%
\pgfpathlineto{\pgfqpoint{4.975655in}{2.291142in}}%
\pgfpathlineto{\pgfqpoint{4.976628in}{2.384280in}}%
\pgfpathlineto{\pgfqpoint{4.977046in}{2.308070in}}%
\pgfpathlineto{\pgfqpoint{4.977391in}{2.431558in}}%
\pgfpathlineto{\pgfqpoint{4.977760in}{2.360066in}}%
\pgfpathlineto{\pgfqpoint{4.977908in}{2.447738in}}%
\pgfpathlineto{\pgfqpoint{4.978437in}{2.314594in}}%
\pgfpathlineto{\pgfqpoint{4.978892in}{2.397985in}}%
\pgfpathlineto{\pgfqpoint{4.979237in}{2.314354in}}%
\pgfpathlineto{\pgfqpoint{4.979298in}{2.437335in}}%
\pgfpathlineto{\pgfqpoint{4.979963in}{2.381647in}}%
\pgfpathlineto{\pgfqpoint{4.979988in}{2.437165in}}%
\pgfpathlineto{\pgfqpoint{4.980628in}{2.286801in}}%
\pgfpathlineto{\pgfqpoint{4.981046in}{2.325421in}}%
\pgfpathlineto{\pgfqpoint{4.981378in}{2.430290in}}%
\pgfpathlineto{\pgfqpoint{4.982018in}{2.314655in}}%
\pgfpathlineto{\pgfqpoint{4.982277in}{2.410456in}}%
\pgfpathlineto{\pgfqpoint{4.982597in}{2.304327in}}%
\pgfpathlineto{\pgfqpoint{4.982769in}{2.457538in}}%
\pgfpathlineto{\pgfqpoint{4.983446in}{2.362776in}}%
\pgfpathlineto{\pgfqpoint{4.984160in}{2.439429in}}%
\pgfpathlineto{\pgfqpoint{4.984000in}{2.318326in}}%
\pgfpathlineto{\pgfqpoint{4.984541in}{2.329699in}}%
\pgfpathlineto{\pgfqpoint{4.985551in}{2.431954in}}%
\pgfpathlineto{\pgfqpoint{4.985391in}{2.306348in}}%
\pgfpathlineto{\pgfqpoint{4.985698in}{2.405827in}}%
\pgfpathlineto{\pgfqpoint{4.986031in}{2.336035in}}%
\pgfpathlineto{\pgfqpoint{4.986252in}{2.413198in}}%
\pgfpathlineto{\pgfqpoint{4.986658in}{2.402967in}}%
\pgfpathlineto{\pgfqpoint{4.987643in}{2.438005in}}%
\pgfpathlineto{\pgfqpoint{4.987434in}{2.314170in}}%
\pgfpathlineto{\pgfqpoint{4.987741in}{2.392425in}}%
\pgfpathlineto{\pgfqpoint{4.988824in}{2.331270in}}%
\pgfpathlineto{\pgfqpoint{4.988332in}{2.425537in}}%
\pgfpathlineto{\pgfqpoint{4.988886in}{2.367377in}}%
\pgfpathlineto{\pgfqpoint{4.989034in}{2.432036in}}%
\pgfpathlineto{\pgfqpoint{4.989391in}{2.315546in}}%
\pgfpathlineto{\pgfqpoint{4.989994in}{2.374746in}}%
\pgfpathlineto{\pgfqpoint{4.990424in}{2.416238in}}%
\pgfpathlineto{\pgfqpoint{4.990203in}{2.314238in}}%
\pgfpathlineto{\pgfqpoint{4.991040in}{2.370291in}}%
\pgfpathlineto{\pgfqpoint{4.992172in}{2.315084in}}%
\pgfpathlineto{\pgfqpoint{4.991114in}{2.436537in}}%
\pgfpathlineto{\pgfqpoint{4.992197in}{2.327770in}}%
\pgfpathlineto{\pgfqpoint{4.992504in}{2.454403in}}%
\pgfpathlineto{\pgfqpoint{4.992332in}{2.315871in}}%
\pgfpathlineto{\pgfqpoint{4.993341in}{2.374739in}}%
\pgfpathlineto{\pgfqpoint{4.993895in}{2.450304in}}%
\pgfpathlineto{\pgfqpoint{4.994252in}{2.335904in}}%
\pgfpathlineto{\pgfqpoint{4.994351in}{2.348042in}}%
\pgfpathlineto{\pgfqpoint{4.995077in}{2.315694in}}%
\pgfpathlineto{\pgfqpoint{4.994584in}{2.435199in}}%
\pgfpathlineto{\pgfqpoint{4.995421in}{2.399966in}}%
\pgfpathlineto{\pgfqpoint{4.995975in}{2.421651in}}%
\pgfpathlineto{\pgfqpoint{4.995766in}{2.323650in}}%
\pgfpathlineto{\pgfqpoint{4.996443in}{2.367952in}}%
\pgfpathlineto{\pgfqpoint{4.997157in}{2.308079in}}%
\pgfpathlineto{\pgfqpoint{4.996677in}{2.456314in}}%
\pgfpathlineto{\pgfqpoint{4.997514in}{2.401176in}}%
\pgfpathlineto{\pgfqpoint{4.997526in}{2.402106in}}%
\pgfpathlineto{\pgfqpoint{4.997723in}{2.338497in}}%
\pgfpathlineto{\pgfqpoint{4.997834in}{2.351805in}}%
\pgfpathlineto{\pgfqpoint{4.998006in}{2.326291in}}%
\pgfpathlineto{\pgfqpoint{4.998757in}{2.451357in}}%
\pgfpathlineto{\pgfqpoint{4.998904in}{2.379275in}}%
\pgfpathlineto{\pgfqpoint{4.998917in}{2.379402in}}%
\pgfpathlineto{\pgfqpoint{4.998941in}{2.368691in}}%
\pgfpathlineto{\pgfqpoint{4.999963in}{2.292749in}}%
\pgfpathlineto{\pgfqpoint{4.999446in}{2.459416in}}%
\pgfpathlineto{\pgfqpoint{5.000024in}{2.375825in}}%
\pgfpathlineto{\pgfqpoint{5.000837in}{2.439648in}}%
\pgfpathlineto{\pgfqpoint{5.001083in}{2.330019in}}%
\pgfpathlineto{\pgfqpoint{5.001107in}{2.344319in}}%
\pgfpathlineto{\pgfqpoint{5.002227in}{2.460770in}}%
\pgfpathlineto{\pgfqpoint{5.001895in}{2.290000in}}%
\pgfpathlineto{\pgfqpoint{5.002387in}{2.411225in}}%
\pgfpathlineto{\pgfqpoint{5.002744in}{2.329912in}}%
\pgfpathlineto{\pgfqpoint{5.003507in}{2.399909in}}%
\pgfpathlineto{\pgfqpoint{5.003618in}{2.463475in}}%
\pgfpathlineto{\pgfqpoint{5.003987in}{2.301955in}}%
\pgfpathlineto{\pgfqpoint{5.004591in}{2.383290in}}%
\pgfpathlineto{\pgfqpoint{5.005021in}{2.425469in}}%
\pgfpathlineto{\pgfqpoint{5.004849in}{2.316443in}}%
\pgfpathlineto{\pgfqpoint{5.005354in}{2.361703in}}%
\pgfpathlineto{\pgfqpoint{5.006080in}{2.317481in}}%
\pgfpathlineto{\pgfqpoint{5.005710in}{2.434118in}}%
\pgfpathlineto{\pgfqpoint{5.006387in}{2.425519in}}%
\pgfpathlineto{\pgfqpoint{5.006412in}{2.457470in}}%
\pgfpathlineto{\pgfqpoint{5.006769in}{2.305243in}}%
\pgfpathlineto{\pgfqpoint{5.007446in}{2.361426in}}%
\pgfpathlineto{\pgfqpoint{5.007470in}{2.338047in}}%
\pgfpathlineto{\pgfqpoint{5.008492in}{2.442770in}}%
\pgfpathlineto{\pgfqpoint{5.008504in}{2.445539in}}%
\pgfpathlineto{\pgfqpoint{5.008861in}{2.299443in}}%
\pgfpathlineto{\pgfqpoint{5.009390in}{2.395918in}}%
\pgfpathlineto{\pgfqpoint{5.009723in}{2.321808in}}%
\pgfpathlineto{\pgfqpoint{5.009895in}{2.403278in}}%
\pgfpathlineto{\pgfqpoint{5.010461in}{2.396915in}}%
\pgfpathlineto{\pgfqpoint{5.011286in}{2.430853in}}%
\pgfpathlineto{\pgfqpoint{5.010941in}{2.324669in}}%
\pgfpathlineto{\pgfqpoint{5.011544in}{2.374128in}}%
\pgfpathlineto{\pgfqpoint{5.011766in}{2.313064in}}%
\pgfpathlineto{\pgfqpoint{5.011987in}{2.420828in}}%
\pgfpathlineto{\pgfqpoint{5.012554in}{2.391992in}}%
\pgfpathlineto{\pgfqpoint{5.013366in}{2.440881in}}%
\pgfpathlineto{\pgfqpoint{5.013034in}{2.344287in}}%
\pgfpathlineto{\pgfqpoint{5.013600in}{2.349349in}}%
\pgfpathlineto{\pgfqpoint{5.013723in}{2.303652in}}%
\pgfpathlineto{\pgfqpoint{5.014067in}{2.425107in}}%
\pgfpathlineto{\pgfqpoint{5.014707in}{2.337939in}}%
\pgfpathlineto{\pgfqpoint{5.015446in}{2.431305in}}%
\pgfpathlineto{\pgfqpoint{5.015803in}{2.334519in}}%
\pgfpathlineto{\pgfqpoint{5.016664in}{2.310175in}}%
\pgfpathlineto{\pgfqpoint{5.016147in}{2.429646in}}%
\pgfpathlineto{\pgfqpoint{5.016726in}{2.366426in}}%
\pgfpathlineto{\pgfqpoint{5.016837in}{2.419676in}}%
\pgfpathlineto{\pgfqpoint{5.017206in}{2.353464in}}%
\pgfpathlineto{\pgfqpoint{5.017834in}{2.365044in}}%
\pgfpathlineto{\pgfqpoint{5.018597in}{2.305167in}}%
\pgfpathlineto{\pgfqpoint{5.018227in}{2.449072in}}%
\pgfpathlineto{\pgfqpoint{5.018892in}{2.384929in}}%
\pgfpathlineto{\pgfqpoint{5.018929in}{2.437310in}}%
\pgfpathlineto{\pgfqpoint{5.019446in}{2.325762in}}%
\pgfpathlineto{\pgfqpoint{5.019975in}{2.339031in}}%
\pgfpathlineto{\pgfqpoint{5.020677in}{2.321177in}}%
\pgfpathlineto{\pgfqpoint{5.020320in}{2.451428in}}%
\pgfpathlineto{\pgfqpoint{5.020972in}{2.374082in}}%
\pgfpathlineto{\pgfqpoint{5.021009in}{2.434362in}}%
\pgfpathlineto{\pgfqpoint{5.021538in}{2.311552in}}%
\pgfpathlineto{\pgfqpoint{5.022067in}{2.343690in}}%
\pgfpathlineto{\pgfqpoint{5.023101in}{2.450723in}}%
\pgfpathlineto{\pgfqpoint{5.022929in}{2.324436in}}%
\pgfpathlineto{\pgfqpoint{5.023310in}{2.387003in}}%
\pgfpathlineto{\pgfqpoint{5.023470in}{2.305857in}}%
\pgfpathlineto{\pgfqpoint{5.023803in}{2.428541in}}%
\pgfpathlineto{\pgfqpoint{5.024455in}{2.350083in}}%
\pgfpathlineto{\pgfqpoint{5.025181in}{2.449592in}}%
\pgfpathlineto{\pgfqpoint{5.025550in}{2.330338in}}%
\pgfpathlineto{\pgfqpoint{5.026572in}{2.430680in}}%
\pgfpathlineto{\pgfqpoint{5.026400in}{2.310865in}}%
\pgfpathlineto{\pgfqpoint{5.026757in}{2.392909in}}%
\pgfpathlineto{\pgfqpoint{5.027741in}{2.328398in}}%
\pgfpathlineto{\pgfqpoint{5.027261in}{2.414645in}}%
\pgfpathlineto{\pgfqpoint{5.027901in}{2.368315in}}%
\pgfpathlineto{\pgfqpoint{5.027963in}{2.443518in}}%
\pgfpathlineto{\pgfqpoint{5.028320in}{2.319112in}}%
\pgfpathlineto{\pgfqpoint{5.028997in}{2.372955in}}%
\pgfpathlineto{\pgfqpoint{5.029710in}{2.330936in}}%
\pgfpathlineto{\pgfqpoint{5.030055in}{2.434957in}}%
\pgfpathlineto{\pgfqpoint{5.030067in}{2.427382in}}%
\pgfpathlineto{\pgfqpoint{5.031212in}{2.317173in}}%
\pgfpathlineto{\pgfqpoint{5.031273in}{2.326949in}}%
\pgfpathlineto{\pgfqpoint{5.032430in}{2.408889in}}%
\pgfpathlineto{\pgfqpoint{5.032455in}{2.399862in}}%
\pgfpathlineto{\pgfqpoint{5.033021in}{2.335910in}}%
\pgfpathlineto{\pgfqpoint{5.033403in}{2.407407in}}%
\pgfpathlineto{\pgfqpoint{5.033637in}{2.347561in}}%
\pgfpathlineto{\pgfqpoint{5.034080in}{2.423691in}}%
\pgfpathlineto{\pgfqpoint{5.033784in}{2.335445in}}%
\pgfpathlineto{\pgfqpoint{5.034793in}{2.409038in}}%
\pgfpathlineto{\pgfqpoint{5.035089in}{2.336766in}}%
\pgfpathlineto{\pgfqpoint{5.034855in}{2.432193in}}%
\pgfpathlineto{\pgfqpoint{5.035987in}{2.359151in}}%
\pgfpathlineto{\pgfqpoint{5.036307in}{2.408729in}}%
\pgfpathlineto{\pgfqpoint{5.036123in}{2.343292in}}%
\pgfpathlineto{\pgfqpoint{5.037144in}{2.369456in}}%
\pgfpathlineto{\pgfqpoint{5.037243in}{2.342727in}}%
\pgfpathlineto{\pgfqpoint{5.037833in}{2.400407in}}%
\pgfpathlineto{\pgfqpoint{5.038215in}{2.371591in}}%
\pgfpathlineto{\pgfqpoint{5.038732in}{2.411411in}}%
\pgfpathlineto{\pgfqpoint{5.038461in}{2.334103in}}%
\pgfpathlineto{\pgfqpoint{5.039310in}{2.362152in}}%
\pgfpathlineto{\pgfqpoint{5.040012in}{2.319816in}}%
\pgfpathlineto{\pgfqpoint{5.039397in}{2.412714in}}%
\pgfpathlineto{\pgfqpoint{5.040381in}{2.384927in}}%
\pgfpathlineto{\pgfqpoint{5.041083in}{2.405782in}}%
\pgfpathlineto{\pgfqpoint{5.040541in}{2.347425in}}%
\pgfpathlineto{\pgfqpoint{5.041427in}{2.351752in}}%
\pgfpathlineto{\pgfqpoint{5.041735in}{2.423656in}}%
\pgfpathlineto{\pgfqpoint{5.042006in}{2.337429in}}%
\pgfpathlineto{\pgfqpoint{5.042646in}{2.390524in}}%
\pgfpathlineto{\pgfqpoint{5.043470in}{2.338590in}}%
\pgfpathlineto{\pgfqpoint{5.043273in}{2.423845in}}%
\pgfpathlineto{\pgfqpoint{5.043766in}{2.370440in}}%
\pgfpathlineto{\pgfqpoint{5.044073in}{2.397016in}}%
\pgfpathlineto{\pgfqpoint{5.044000in}{2.338524in}}%
\pgfpathlineto{\pgfqpoint{5.044873in}{2.375454in}}%
\pgfpathlineto{\pgfqpoint{5.045021in}{2.330878in}}%
\pgfpathlineto{\pgfqpoint{5.045637in}{2.403712in}}%
\pgfpathlineto{\pgfqpoint{5.046006in}{2.364125in}}%
\pgfpathlineto{\pgfqpoint{5.046387in}{2.401162in}}%
\pgfpathlineto{\pgfqpoint{5.046880in}{2.344429in}}%
\pgfpathlineto{\pgfqpoint{5.047101in}{2.365896in}}%
\pgfpathlineto{\pgfqpoint{5.047815in}{2.345983in}}%
\pgfpathlineto{\pgfqpoint{5.047175in}{2.415027in}}%
\pgfpathlineto{\pgfqpoint{5.048184in}{2.374034in}}%
\pgfpathlineto{\pgfqpoint{5.048750in}{2.412663in}}%
\pgfpathlineto{\pgfqpoint{5.048935in}{2.346825in}}%
\pgfpathlineto{\pgfqpoint{5.049292in}{2.372658in}}%
\pgfpathlineto{\pgfqpoint{5.049526in}{2.416767in}}%
\pgfpathlineto{\pgfqpoint{5.049735in}{2.346828in}}%
\pgfpathlineto{\pgfqpoint{5.050387in}{2.372067in}}%
\pgfpathlineto{\pgfqpoint{5.050781in}{2.344826in}}%
\pgfpathlineto{\pgfqpoint{5.051101in}{2.406528in}}%
\pgfpathlineto{\pgfqpoint{5.051409in}{2.383178in}}%
\pgfpathlineto{\pgfqpoint{5.052135in}{2.395262in}}%
\pgfpathlineto{\pgfqpoint{5.051753in}{2.341986in}}%
\pgfpathlineto{\pgfqpoint{5.052492in}{2.377132in}}%
\pgfpathlineto{\pgfqpoint{5.052541in}{2.352093in}}%
\pgfpathlineto{\pgfqpoint{5.053021in}{2.398333in}}%
\pgfpathlineto{\pgfqpoint{5.053587in}{2.378144in}}%
\pgfpathlineto{\pgfqpoint{5.054400in}{2.400577in}}%
\pgfpathlineto{\pgfqpoint{5.053920in}{2.350247in}}%
\pgfpathlineto{\pgfqpoint{5.054646in}{2.356556in}}%
\pgfpathlineto{\pgfqpoint{5.054707in}{2.345843in}}%
\pgfpathlineto{\pgfqpoint{5.054929in}{2.388153in}}%
\pgfpathlineto{\pgfqpoint{5.055040in}{2.413339in}}%
\pgfpathlineto{\pgfqpoint{5.055667in}{2.345704in}}%
\pgfpathlineto{\pgfqpoint{5.056012in}{2.358239in}}%
\pgfpathlineto{\pgfqpoint{5.056270in}{2.347154in}}%
\pgfpathlineto{\pgfqpoint{5.056381in}{2.394119in}}%
\pgfpathlineto{\pgfqpoint{5.057181in}{2.407038in}}%
\pgfpathlineto{\pgfqpoint{5.056566in}{2.347590in}}%
\pgfpathlineto{\pgfqpoint{5.057452in}{2.377809in}}%
\pgfpathlineto{\pgfqpoint{5.058141in}{2.339216in}}%
\pgfpathlineto{\pgfqpoint{5.057969in}{2.402102in}}%
\pgfpathlineto{\pgfqpoint{5.058547in}{2.383000in}}%
\pgfpathlineto{\pgfqpoint{5.059618in}{2.327588in}}%
\pgfpathlineto{\pgfqpoint{5.059212in}{2.407232in}}%
\pgfpathlineto{\pgfqpoint{5.059655in}{2.377665in}}%
\pgfpathlineto{\pgfqpoint{5.060000in}{2.408084in}}%
\pgfpathlineto{\pgfqpoint{5.059716in}{2.335111in}}%
\pgfpathlineto{\pgfqpoint{5.060800in}{2.404639in}}%
\pgfpathlineto{\pgfqpoint{5.061181in}{2.341537in}}%
\pgfpathlineto{\pgfqpoint{5.061341in}{2.418599in}}%
\pgfpathlineto{\pgfqpoint{5.061969in}{2.353532in}}%
\pgfpathlineto{\pgfqpoint{5.062904in}{2.412928in}}%
\pgfpathlineto{\pgfqpoint{5.062584in}{2.343775in}}%
\pgfpathlineto{\pgfqpoint{5.063187in}{2.368009in}}%
\pgfpathlineto{\pgfqpoint{5.063544in}{2.347059in}}%
\pgfpathlineto{\pgfqpoint{5.063938in}{2.398292in}}%
\pgfpathlineto{\pgfqpoint{5.064196in}{2.393143in}}%
\pgfpathlineto{\pgfqpoint{5.064209in}{2.393400in}}%
\pgfpathlineto{\pgfqpoint{5.064295in}{2.372865in}}%
\pgfpathlineto{\pgfqpoint{5.065120in}{2.338615in}}%
\pgfpathlineto{\pgfqpoint{5.064726in}{2.401748in}}%
\pgfpathlineto{\pgfqpoint{5.065390in}{2.382812in}}%
\pgfpathlineto{\pgfqpoint{5.066289in}{2.401661in}}%
\pgfpathlineto{\pgfqpoint{5.065907in}{2.342094in}}%
\pgfpathlineto{\pgfqpoint{5.066461in}{2.371270in}}%
\pgfpathlineto{\pgfqpoint{5.067470in}{2.340376in}}%
\pgfpathlineto{\pgfqpoint{5.067076in}{2.400756in}}%
\pgfpathlineto{\pgfqpoint{5.067581in}{2.364329in}}%
\pgfpathlineto{\pgfqpoint{5.068406in}{2.407589in}}%
\pgfpathlineto{\pgfqpoint{5.068258in}{2.342391in}}%
\pgfpathlineto{\pgfqpoint{5.068676in}{2.369847in}}%
\pgfpathlineto{\pgfqpoint{5.069046in}{2.346915in}}%
\pgfpathlineto{\pgfqpoint{5.069193in}{2.410857in}}%
\pgfpathlineto{\pgfqpoint{5.069759in}{2.392651in}}%
\pgfpathlineto{\pgfqpoint{5.069883in}{2.399066in}}%
\pgfpathlineto{\pgfqpoint{5.070264in}{2.349625in}}%
\pgfpathlineto{\pgfqpoint{5.070683in}{2.371237in}}%
\pgfpathlineto{\pgfqpoint{5.071396in}{2.339882in}}%
\pgfpathlineto{\pgfqpoint{5.071544in}{2.398964in}}%
\pgfpathlineto{\pgfqpoint{5.071766in}{2.389498in}}%
\pgfpathlineto{\pgfqpoint{5.072566in}{2.398669in}}%
\pgfpathlineto{\pgfqpoint{5.071827in}{2.347433in}}%
\pgfpathlineto{\pgfqpoint{5.072849in}{2.373713in}}%
\pgfpathlineto{\pgfqpoint{5.073292in}{2.355498in}}%
\pgfpathlineto{\pgfqpoint{5.073095in}{2.397736in}}%
\pgfpathlineto{\pgfqpoint{5.073759in}{2.379080in}}%
\pgfpathlineto{\pgfqpoint{5.074646in}{2.406983in}}%
\pgfpathlineto{\pgfqpoint{5.074178in}{2.347214in}}%
\pgfpathlineto{\pgfqpoint{5.074830in}{2.356287in}}%
\pgfpathlineto{\pgfqpoint{5.074855in}{2.346900in}}%
\pgfpathlineto{\pgfqpoint{5.075421in}{2.409799in}}%
\pgfpathlineto{\pgfqpoint{5.075913in}{2.372494in}}%
\pgfpathlineto{\pgfqpoint{5.076196in}{2.404459in}}%
\pgfpathlineto{\pgfqpoint{5.076393in}{2.347713in}}%
\pgfpathlineto{\pgfqpoint{5.077033in}{2.384078in}}%
\pgfpathlineto{\pgfqpoint{5.077181in}{2.345948in}}%
\pgfpathlineto{\pgfqpoint{5.077759in}{2.400370in}}%
\pgfpathlineto{\pgfqpoint{5.078166in}{2.370071in}}%
\pgfpathlineto{\pgfqpoint{5.078535in}{2.421440in}}%
\pgfpathlineto{\pgfqpoint{5.078732in}{2.341662in}}%
\pgfpathlineto{\pgfqpoint{5.079249in}{2.361055in}}%
\pgfpathlineto{\pgfqpoint{5.079298in}{2.403912in}}%
\pgfpathlineto{\pgfqpoint{5.079323in}{2.414017in}}%
\pgfpathlineto{\pgfqpoint{5.080283in}{2.339651in}}%
\pgfpathlineto{\pgfqpoint{5.080356in}{2.377989in}}%
\pgfpathlineto{\pgfqpoint{5.081058in}{2.335198in}}%
\pgfpathlineto{\pgfqpoint{5.080886in}{2.410464in}}%
\pgfpathlineto{\pgfqpoint{5.081476in}{2.369293in}}%
\pgfpathlineto{\pgfqpoint{5.082621in}{2.321512in}}%
\pgfpathlineto{\pgfqpoint{5.082424in}{2.407949in}}%
\pgfpathlineto{\pgfqpoint{5.082633in}{2.326538in}}%
\pgfpathlineto{\pgfqpoint{5.083199in}{2.403023in}}%
\pgfpathlineto{\pgfqpoint{5.083778in}{2.387771in}}%
\pgfpathlineto{\pgfqpoint{5.084184in}{2.329043in}}%
\pgfpathlineto{\pgfqpoint{5.084775in}{2.407619in}}%
\pgfpathlineto{\pgfqpoint{5.084996in}{2.358015in}}%
\pgfpathlineto{\pgfqpoint{5.085550in}{2.402146in}}%
\pgfpathlineto{\pgfqpoint{5.085747in}{2.342653in}}%
\pgfpathlineto{\pgfqpoint{5.086116in}{2.374353in}}%
\pgfpathlineto{\pgfqpoint{5.087113in}{2.409010in}}%
\pgfpathlineto{\pgfqpoint{5.086535in}{2.343095in}}%
\pgfpathlineto{\pgfqpoint{5.087261in}{2.382831in}}%
\pgfpathlineto{\pgfqpoint{5.088098in}{2.336136in}}%
\pgfpathlineto{\pgfqpoint{5.087901in}{2.403564in}}%
\pgfpathlineto{\pgfqpoint{5.088381in}{2.363011in}}%
\pgfpathlineto{\pgfqpoint{5.088689in}{2.401325in}}%
\pgfpathlineto{\pgfqpoint{5.088886in}{2.335828in}}%
\pgfpathlineto{\pgfqpoint{5.089526in}{2.376603in}}%
\pgfpathlineto{\pgfqpoint{5.089673in}{2.348591in}}%
\pgfpathlineto{\pgfqpoint{5.090252in}{2.401929in}}%
\pgfpathlineto{\pgfqpoint{5.090584in}{2.380233in}}%
\pgfpathlineto{\pgfqpoint{5.091039in}{2.407681in}}%
\pgfpathlineto{\pgfqpoint{5.091482in}{2.353779in}}%
\pgfpathlineto{\pgfqpoint{5.092282in}{2.346665in}}%
\pgfpathlineto{\pgfqpoint{5.091827in}{2.404343in}}%
\pgfpathlineto{\pgfqpoint{5.092504in}{2.376011in}}%
\pgfpathlineto{\pgfqpoint{5.093193in}{2.396187in}}%
\pgfpathlineto{\pgfqpoint{5.093070in}{2.343141in}}%
\pgfpathlineto{\pgfqpoint{5.093587in}{2.352383in}}%
\pgfpathlineto{\pgfqpoint{5.093599in}{2.346187in}}%
\pgfpathlineto{\pgfqpoint{5.094104in}{2.398287in}}%
\pgfpathlineto{\pgfqpoint{5.094670in}{2.361946in}}%
\pgfpathlineto{\pgfqpoint{5.094892in}{2.395524in}}%
\pgfpathlineto{\pgfqpoint{5.095433in}{2.347497in}}%
\pgfpathlineto{\pgfqpoint{5.095815in}{2.384866in}}%
\pgfpathlineto{\pgfqpoint{5.096221in}{2.346688in}}%
\pgfpathlineto{\pgfqpoint{5.096590in}{2.401395in}}%
\pgfpathlineto{\pgfqpoint{5.096947in}{2.366156in}}%
\pgfpathlineto{\pgfqpoint{5.098042in}{2.406537in}}%
\pgfpathlineto{\pgfqpoint{5.097796in}{2.341561in}}%
\pgfpathlineto{\pgfqpoint{5.098067in}{2.375538in}}%
\pgfpathlineto{\pgfqpoint{5.098584in}{2.343808in}}%
\pgfpathlineto{\pgfqpoint{5.098141in}{2.409279in}}%
\pgfpathlineto{\pgfqpoint{5.099162in}{2.385537in}}%
\pgfpathlineto{\pgfqpoint{5.099606in}{2.399325in}}%
\pgfpathlineto{\pgfqpoint{5.099901in}{2.345980in}}%
\pgfpathlineto{\pgfqpoint{5.100122in}{2.365691in}}%
\pgfpathlineto{\pgfqpoint{5.100689in}{2.344518in}}%
\pgfpathlineto{\pgfqpoint{5.100492in}{2.405774in}}%
\pgfpathlineto{\pgfqpoint{5.101156in}{2.374833in}}%
\pgfpathlineto{\pgfqpoint{5.102055in}{2.402780in}}%
\pgfpathlineto{\pgfqpoint{5.101230in}{2.347915in}}%
\pgfpathlineto{\pgfqpoint{5.102252in}{2.365487in}}%
\pgfpathlineto{\pgfqpoint{5.102842in}{2.406247in}}%
\pgfpathlineto{\pgfqpoint{5.102707in}{2.345347in}}%
\pgfpathlineto{\pgfqpoint{5.103298in}{2.359889in}}%
\pgfpathlineto{\pgfqpoint{5.103482in}{2.344439in}}%
\pgfpathlineto{\pgfqpoint{5.103630in}{2.407322in}}%
\pgfpathlineto{\pgfqpoint{5.104196in}{2.377086in}}%
\pgfpathlineto{\pgfqpoint{5.104418in}{2.400872in}}%
\pgfpathlineto{\pgfqpoint{5.104602in}{2.352703in}}%
\pgfpathlineto{\pgfqpoint{5.105255in}{2.371423in}}%
\pgfpathlineto{\pgfqpoint{5.106178in}{2.352445in}}%
\pgfpathlineto{\pgfqpoint{5.106227in}{2.395598in}}%
\pgfpathlineto{\pgfqpoint{5.106326in}{2.384182in}}%
\pgfpathlineto{\pgfqpoint{5.106633in}{2.345421in}}%
\pgfpathlineto{\pgfqpoint{5.107015in}{2.400452in}}%
\pgfpathlineto{\pgfqpoint{5.107433in}{2.370509in}}%
\pgfpathlineto{\pgfqpoint{5.107802in}{2.403798in}}%
\pgfpathlineto{\pgfqpoint{5.108184in}{2.345754in}}%
\pgfpathlineto{\pgfqpoint{5.108553in}{2.384673in}}%
\pgfpathlineto{\pgfqpoint{5.109107in}{2.406510in}}%
\pgfpathlineto{\pgfqpoint{5.108972in}{2.343220in}}%
\pgfpathlineto{\pgfqpoint{5.109636in}{2.370976in}}%
\pgfpathlineto{\pgfqpoint{5.109747in}{2.348969in}}%
\pgfpathlineto{\pgfqpoint{5.109895in}{2.406479in}}%
\pgfpathlineto{\pgfqpoint{5.110559in}{2.379091in}}%
\pgfpathlineto{\pgfqpoint{5.110670in}{2.402842in}}%
\pgfpathlineto{\pgfqpoint{5.110867in}{2.348737in}}%
\pgfpathlineto{\pgfqpoint{5.111642in}{2.356264in}}%
\pgfpathlineto{\pgfqpoint{5.112233in}{2.404332in}}%
\pgfpathlineto{\pgfqpoint{5.112085in}{2.350855in}}%
\pgfpathlineto{\pgfqpoint{5.112787in}{2.374416in}}%
\pgfpathlineto{\pgfqpoint{5.113021in}{2.406001in}}%
\pgfpathlineto{\pgfqpoint{5.113205in}{2.350903in}}%
\pgfpathlineto{\pgfqpoint{5.113858in}{2.376068in}}%
\pgfpathlineto{\pgfqpoint{5.113969in}{2.345330in}}%
\pgfpathlineto{\pgfqpoint{5.114559in}{2.400340in}}%
\pgfpathlineto{\pgfqpoint{5.114965in}{2.376961in}}%
\pgfpathlineto{\pgfqpoint{5.115335in}{2.394574in}}%
\pgfpathlineto{\pgfqpoint{5.115187in}{2.350136in}}%
\pgfpathlineto{\pgfqpoint{5.115495in}{2.364591in}}%
\pgfpathlineto{\pgfqpoint{5.115962in}{2.347334in}}%
\pgfpathlineto{\pgfqpoint{5.116110in}{2.399511in}}%
\pgfpathlineto{\pgfqpoint{5.116602in}{2.364477in}}%
\pgfpathlineto{\pgfqpoint{5.116885in}{2.398485in}}%
\pgfpathlineto{\pgfqpoint{5.117082in}{2.345434in}}%
\pgfpathlineto{\pgfqpoint{5.117747in}{2.378000in}}%
\pgfpathlineto{\pgfqpoint{5.117858in}{2.346914in}}%
\pgfpathlineto{\pgfqpoint{5.118239in}{2.398891in}}%
\pgfpathlineto{\pgfqpoint{5.118867in}{2.375408in}}%
\pgfpathlineto{\pgfqpoint{5.119409in}{2.356883in}}%
\pgfpathlineto{\pgfqpoint{5.119027in}{2.403728in}}%
\pgfpathlineto{\pgfqpoint{5.119532in}{2.377196in}}%
\pgfpathlineto{\pgfqpoint{5.120381in}{2.403424in}}%
\pgfpathlineto{\pgfqpoint{5.120184in}{2.347359in}}%
\pgfpathlineto{\pgfqpoint{5.120639in}{2.377104in}}%
\pgfpathlineto{\pgfqpoint{5.121095in}{2.421907in}}%
\pgfpathlineto{\pgfqpoint{5.120898in}{2.349768in}}%
\pgfpathlineto{\pgfqpoint{5.121661in}{2.362878in}}%
\pgfpathlineto{\pgfqpoint{5.122522in}{2.329402in}}%
\pgfpathlineto{\pgfqpoint{5.121919in}{2.422579in}}%
\pgfpathlineto{\pgfqpoint{5.122719in}{2.403622in}}%
\pgfpathlineto{\pgfqpoint{5.122732in}{2.408951in}}%
\pgfpathlineto{\pgfqpoint{5.123335in}{2.339231in}}%
\pgfpathlineto{\pgfqpoint{5.123765in}{2.382436in}}%
\pgfpathlineto{\pgfqpoint{5.124849in}{2.337348in}}%
\pgfpathlineto{\pgfqpoint{5.124381in}{2.403607in}}%
\pgfpathlineto{\pgfqpoint{5.124885in}{2.362262in}}%
\pgfpathlineto{\pgfqpoint{5.125193in}{2.406590in}}%
\pgfpathlineto{\pgfqpoint{5.125661in}{2.333803in}}%
\pgfpathlineto{\pgfqpoint{5.126042in}{2.388553in}}%
\pgfpathlineto{\pgfqpoint{5.126473in}{2.337272in}}%
\pgfpathlineto{\pgfqpoint{5.126842in}{2.411917in}}%
\pgfpathlineto{\pgfqpoint{5.127199in}{2.349933in}}%
\pgfpathlineto{\pgfqpoint{5.127655in}{2.415348in}}%
\pgfpathlineto{\pgfqpoint{5.127962in}{2.346118in}}%
\pgfpathlineto{\pgfqpoint{5.128369in}{2.382403in}}%
\pgfpathlineto{\pgfqpoint{5.128787in}{2.344420in}}%
\pgfpathlineto{\pgfqpoint{5.128467in}{2.410970in}}%
\pgfpathlineto{\pgfqpoint{5.129267in}{2.393904in}}%
\pgfpathlineto{\pgfqpoint{5.129710in}{2.402487in}}%
\pgfpathlineto{\pgfqpoint{5.129587in}{2.331218in}}%
\pgfpathlineto{\pgfqpoint{5.130313in}{2.379932in}}%
\pgfpathlineto{\pgfqpoint{5.130387in}{2.331318in}}%
\pgfpathlineto{\pgfqpoint{5.130941in}{2.409123in}}%
\pgfpathlineto{\pgfqpoint{5.131433in}{2.373906in}}%
\pgfpathlineto{\pgfqpoint{5.131753in}{2.412440in}}%
\pgfpathlineto{\pgfqpoint{5.132024in}{2.339232in}}%
\pgfpathlineto{\pgfqpoint{5.132565in}{2.399733in}}%
\pgfpathlineto{\pgfqpoint{5.132578in}{2.399955in}}%
\pgfpathlineto{\pgfqpoint{5.132615in}{2.380963in}}%
\pgfpathlineto{\pgfqpoint{5.132824in}{2.349578in}}%
\pgfpathlineto{\pgfqpoint{5.133402in}{2.404730in}}%
\pgfpathlineto{\pgfqpoint{5.133772in}{2.366338in}}%
\pgfpathlineto{\pgfqpoint{5.134215in}{2.403528in}}%
\pgfpathlineto{\pgfqpoint{5.134818in}{2.340040in}}%
\pgfpathlineto{\pgfqpoint{5.134879in}{2.374138in}}%
\pgfpathlineto{\pgfqpoint{5.135310in}{2.340464in}}%
\pgfpathlineto{\pgfqpoint{5.135458in}{2.403036in}}%
\pgfpathlineto{\pgfqpoint{5.135950in}{2.380934in}}%
\pgfpathlineto{\pgfqpoint{5.135987in}{2.410559in}}%
\pgfpathlineto{\pgfqpoint{5.136947in}{2.338575in}}%
\pgfpathlineto{\pgfqpoint{5.137045in}{2.361096in}}%
\pgfpathlineto{\pgfqpoint{5.137513in}{2.412605in}}%
\pgfpathlineto{\pgfqpoint{5.137772in}{2.346205in}}%
\pgfpathlineto{\pgfqpoint{5.138141in}{2.354664in}}%
\pgfpathlineto{\pgfqpoint{5.138325in}{2.424236in}}%
\pgfpathlineto{\pgfqpoint{5.138596in}{2.339298in}}%
\pgfpathlineto{\pgfqpoint{5.139359in}{2.379106in}}%
\pgfpathlineto{\pgfqpoint{5.139421in}{2.334957in}}%
\pgfpathlineto{\pgfqpoint{5.139975in}{2.415951in}}%
\pgfpathlineto{\pgfqpoint{5.140467in}{2.372715in}}%
\pgfpathlineto{\pgfqpoint{5.140787in}{2.414812in}}%
\pgfpathlineto{\pgfqpoint{5.141390in}{2.331102in}}%
\pgfpathlineto{\pgfqpoint{5.141624in}{2.402692in}}%
\pgfpathlineto{\pgfqpoint{5.141882in}{2.330411in}}%
\pgfpathlineto{\pgfqpoint{5.142436in}{2.417957in}}%
\pgfpathlineto{\pgfqpoint{5.142781in}{2.360276in}}%
\pgfpathlineto{\pgfqpoint{5.142793in}{2.360323in}}%
\pgfpathlineto{\pgfqpoint{5.143248in}{2.411262in}}%
\pgfpathlineto{\pgfqpoint{5.143519in}{2.342064in}}%
\pgfpathlineto{\pgfqpoint{5.143925in}{2.370890in}}%
\pgfpathlineto{\pgfqpoint{5.144073in}{2.413433in}}%
\pgfpathlineto{\pgfqpoint{5.144356in}{2.334364in}}%
\pgfpathlineto{\pgfqpoint{5.145058in}{2.372661in}}%
\pgfpathlineto{\pgfqpoint{5.145095in}{2.385536in}}%
\pgfpathlineto{\pgfqpoint{5.145132in}{2.352476in}}%
\pgfpathlineto{\pgfqpoint{5.145168in}{2.329472in}}%
\pgfpathlineto{\pgfqpoint{5.145722in}{2.414076in}}%
\pgfpathlineto{\pgfqpoint{5.146215in}{2.374984in}}%
\pgfpathlineto{\pgfqpoint{5.146535in}{2.413994in}}%
\pgfpathlineto{\pgfqpoint{5.146818in}{2.330659in}}%
\pgfpathlineto{\pgfqpoint{5.147359in}{2.409489in}}%
\pgfpathlineto{\pgfqpoint{5.147630in}{2.321900in}}%
\pgfpathlineto{\pgfqpoint{5.147778in}{2.411219in}}%
\pgfpathlineto{\pgfqpoint{5.148565in}{2.382826in}}%
\pgfpathlineto{\pgfqpoint{5.148602in}{2.415458in}}%
\pgfpathlineto{\pgfqpoint{5.149279in}{2.332393in}}%
\pgfpathlineto{\pgfqpoint{5.149661in}{2.367058in}}%
\pgfpathlineto{\pgfqpoint{5.150092in}{2.329979in}}%
\pgfpathlineto{\pgfqpoint{5.149833in}{2.416872in}}%
\pgfpathlineto{\pgfqpoint{5.150227in}{2.400136in}}%
\pgfpathlineto{\pgfqpoint{5.150252in}{2.411667in}}%
\pgfpathlineto{\pgfqpoint{5.150928in}{2.334921in}}%
\pgfpathlineto{\pgfqpoint{5.151285in}{2.362525in}}%
\pgfpathlineto{\pgfqpoint{5.151753in}{2.333072in}}%
\pgfpathlineto{\pgfqpoint{5.151888in}{2.411469in}}%
\pgfpathlineto{\pgfqpoint{5.152258in}{2.379270in}}%
\pgfpathlineto{\pgfqpoint{5.152713in}{2.418960in}}%
\pgfpathlineto{\pgfqpoint{5.152565in}{2.332738in}}%
\pgfpathlineto{\pgfqpoint{5.153341in}{2.374842in}}%
\pgfpathlineto{\pgfqpoint{5.153390in}{2.331416in}}%
\pgfpathlineto{\pgfqpoint{5.153538in}{2.411626in}}%
\pgfpathlineto{\pgfqpoint{5.154448in}{2.376228in}}%
\pgfpathlineto{\pgfqpoint{5.154768in}{2.411121in}}%
\pgfpathlineto{\pgfqpoint{5.155027in}{2.333708in}}%
\pgfpathlineto{\pgfqpoint{5.155359in}{2.343736in}}%
\pgfpathlineto{\pgfqpoint{5.155864in}{2.336560in}}%
\pgfpathlineto{\pgfqpoint{5.155593in}{2.420855in}}%
\pgfpathlineto{\pgfqpoint{5.156344in}{2.397194in}}%
\pgfpathlineto{\pgfqpoint{5.156405in}{2.410859in}}%
\pgfpathlineto{\pgfqpoint{5.156676in}{2.326299in}}%
\pgfpathlineto{\pgfqpoint{5.157291in}{2.365250in}}%
\pgfpathlineto{\pgfqpoint{5.157365in}{2.389501in}}%
\pgfpathlineto{\pgfqpoint{5.157464in}{2.345777in}}%
\pgfpathlineto{\pgfqpoint{5.157488in}{2.326655in}}%
\pgfpathlineto{\pgfqpoint{5.157648in}{2.417186in}}%
\pgfpathlineto{\pgfqpoint{5.158547in}{2.369715in}}%
\pgfpathlineto{\pgfqpoint{5.159138in}{2.336460in}}%
\pgfpathlineto{\pgfqpoint{5.159704in}{2.407795in}}%
\pgfpathlineto{\pgfqpoint{5.159950in}{2.333808in}}%
\pgfpathlineto{\pgfqpoint{5.160516in}{2.416370in}}%
\pgfpathlineto{\pgfqpoint{5.160885in}{2.366235in}}%
\pgfpathlineto{\pgfqpoint{5.161328in}{2.406711in}}%
\pgfpathlineto{\pgfqpoint{5.161599in}{2.338337in}}%
\pgfpathlineto{\pgfqpoint{5.161981in}{2.359448in}}%
\pgfpathlineto{\pgfqpoint{5.162571in}{2.411690in}}%
\pgfpathlineto{\pgfqpoint{5.162424in}{2.338589in}}%
\pgfpathlineto{\pgfqpoint{5.163187in}{2.385761in}}%
\pgfpathlineto{\pgfqpoint{5.163248in}{2.331958in}}%
\pgfpathlineto{\pgfqpoint{5.163396in}{2.420677in}}%
\pgfpathlineto{\pgfqpoint{5.164295in}{2.380448in}}%
\pgfpathlineto{\pgfqpoint{5.165033in}{2.415027in}}%
\pgfpathlineto{\pgfqpoint{5.164898in}{2.334759in}}%
\pgfpathlineto{\pgfqpoint{5.165451in}{2.410134in}}%
\pgfpathlineto{\pgfqpoint{5.165710in}{2.338914in}}%
\pgfpathlineto{\pgfqpoint{5.165858in}{2.413657in}}%
\pgfpathlineto{\pgfqpoint{5.166608in}{2.366377in}}%
\pgfpathlineto{\pgfqpoint{5.166621in}{2.365956in}}%
\pgfpathlineto{\pgfqpoint{5.166658in}{2.400371in}}%
\pgfpathlineto{\pgfqpoint{5.166682in}{2.417180in}}%
\pgfpathlineto{\pgfqpoint{5.167691in}{2.335282in}}%
\pgfpathlineto{\pgfqpoint{5.167704in}{2.337147in}}%
\pgfpathlineto{\pgfqpoint{5.168331in}{2.408590in}}%
\pgfpathlineto{\pgfqpoint{5.168947in}{2.383573in}}%
\pgfpathlineto{\pgfqpoint{5.169833in}{2.340591in}}%
\pgfpathlineto{\pgfqpoint{5.169144in}{2.410971in}}%
\pgfpathlineto{\pgfqpoint{5.170055in}{2.371227in}}%
\pgfpathlineto{\pgfqpoint{5.171199in}{2.408105in}}%
\pgfpathlineto{\pgfqpoint{5.170645in}{2.337861in}}%
\pgfpathlineto{\pgfqpoint{5.171211in}{2.408059in}}%
\pgfpathlineto{\pgfqpoint{5.172294in}{2.336524in}}%
\pgfpathlineto{\pgfqpoint{5.172024in}{2.414856in}}%
\pgfpathlineto{\pgfqpoint{5.172393in}{2.358476in}}%
\pgfpathlineto{\pgfqpoint{5.173254in}{2.408380in}}%
\pgfpathlineto{\pgfqpoint{5.173119in}{2.333099in}}%
\pgfpathlineto{\pgfqpoint{5.173501in}{2.355334in}}%
\pgfpathlineto{\pgfqpoint{5.174079in}{2.413082in}}%
\pgfpathlineto{\pgfqpoint{5.173931in}{2.333838in}}%
\pgfpathlineto{\pgfqpoint{5.174658in}{2.377720in}}%
\pgfpathlineto{\pgfqpoint{5.174904in}{2.411225in}}%
\pgfpathlineto{\pgfqpoint{5.174756in}{2.334686in}}%
\pgfpathlineto{\pgfqpoint{5.175531in}{2.371402in}}%
\pgfpathlineto{\pgfqpoint{5.175581in}{2.335744in}}%
\pgfpathlineto{\pgfqpoint{5.176553in}{2.408652in}}%
\pgfpathlineto{\pgfqpoint{5.176627in}{2.376166in}}%
\pgfpathlineto{\pgfqpoint{5.177365in}{2.405440in}}%
\pgfpathlineto{\pgfqpoint{5.177230in}{2.342217in}}%
\pgfpathlineto{\pgfqpoint{5.177759in}{2.389818in}}%
\pgfpathlineto{\pgfqpoint{5.178202in}{2.407781in}}%
\pgfpathlineto{\pgfqpoint{5.178054in}{2.336301in}}%
\pgfpathlineto{\pgfqpoint{5.178818in}{2.375027in}}%
\pgfpathlineto{\pgfqpoint{5.179691in}{2.334066in}}%
\pgfpathlineto{\pgfqpoint{5.179839in}{2.419350in}}%
\pgfpathlineto{\pgfqpoint{5.179913in}{2.388906in}}%
\pgfpathlineto{\pgfqpoint{5.180664in}{2.419253in}}%
\pgfpathlineto{\pgfqpoint{5.180516in}{2.333762in}}%
\pgfpathlineto{\pgfqpoint{5.180996in}{2.382253in}}%
\pgfpathlineto{\pgfqpoint{5.181328in}{2.341653in}}%
\pgfpathlineto{\pgfqpoint{5.181476in}{2.424675in}}%
\pgfpathlineto{\pgfqpoint{5.182202in}{2.360037in}}%
\pgfpathlineto{\pgfqpoint{5.182301in}{2.424081in}}%
\pgfpathlineto{\pgfqpoint{5.182978in}{2.345732in}}%
\pgfpathlineto{\pgfqpoint{5.183285in}{2.357766in}}%
\pgfpathlineto{\pgfqpoint{5.184159in}{2.338076in}}%
\pgfpathlineto{\pgfqpoint{5.183950in}{2.420857in}}%
\pgfpathlineto{\pgfqpoint{5.184319in}{2.377915in}}%
\pgfpathlineto{\pgfqpoint{5.184774in}{2.424175in}}%
\pgfpathlineto{\pgfqpoint{5.184971in}{2.335598in}}%
\pgfpathlineto{\pgfqpoint{5.185402in}{2.380585in}}%
\pgfpathlineto{\pgfqpoint{5.185796in}{2.342019in}}%
\pgfpathlineto{\pgfqpoint{5.186411in}{2.417550in}}%
\pgfpathlineto{\pgfqpoint{5.186510in}{2.373323in}}%
\pgfpathlineto{\pgfqpoint{5.187101in}{2.344454in}}%
\pgfpathlineto{\pgfqpoint{5.186953in}{2.403953in}}%
\pgfpathlineto{\pgfqpoint{5.187211in}{2.397368in}}%
\pgfpathlineto{\pgfqpoint{5.187236in}{2.415635in}}%
\pgfpathlineto{\pgfqpoint{5.188258in}{2.338846in}}%
\pgfpathlineto{\pgfqpoint{5.188270in}{2.338178in}}%
\pgfpathlineto{\pgfqpoint{5.188454in}{2.385851in}}%
\pgfpathlineto{\pgfqpoint{5.188885in}{2.409512in}}%
\pgfpathlineto{\pgfqpoint{5.189082in}{2.332944in}}%
\pgfpathlineto{\pgfqpoint{5.189525in}{2.371380in}}%
\pgfpathlineto{\pgfqpoint{5.189907in}{2.336619in}}%
\pgfpathlineto{\pgfqpoint{5.190534in}{2.421803in}}%
\pgfpathlineto{\pgfqpoint{5.190633in}{2.372943in}}%
\pgfpathlineto{\pgfqpoint{5.190645in}{2.372953in}}%
\pgfpathlineto{\pgfqpoint{5.191211in}{2.338725in}}%
\pgfpathlineto{\pgfqpoint{5.191347in}{2.415148in}}%
\pgfpathlineto{\pgfqpoint{5.191679in}{2.386785in}}%
\pgfpathlineto{\pgfqpoint{5.192184in}{2.409901in}}%
\pgfpathlineto{\pgfqpoint{5.192036in}{2.334445in}}%
\pgfpathlineto{\pgfqpoint{5.192750in}{2.377428in}}%
\pgfpathlineto{\pgfqpoint{5.193205in}{2.337958in}}%
\pgfpathlineto{\pgfqpoint{5.193008in}{2.411001in}}%
\pgfpathlineto{\pgfqpoint{5.193796in}{2.384737in}}%
\pgfpathlineto{\pgfqpoint{5.194645in}{2.412585in}}%
\pgfpathlineto{\pgfqpoint{5.194510in}{2.333219in}}%
\pgfpathlineto{\pgfqpoint{5.194879in}{2.353746in}}%
\pgfpathlineto{\pgfqpoint{5.195064in}{2.397028in}}%
\pgfpathlineto{\pgfqpoint{5.195334in}{2.329924in}}%
\pgfpathlineto{\pgfqpoint{5.196036in}{2.379523in}}%
\pgfpathlineto{\pgfqpoint{5.196147in}{2.342522in}}%
\pgfpathlineto{\pgfqpoint{5.196627in}{2.391899in}}%
\pgfpathlineto{\pgfqpoint{5.197131in}{2.364728in}}%
\pgfpathlineto{\pgfqpoint{5.197168in}{2.419386in}}%
\pgfpathlineto{\pgfqpoint{5.197710in}{2.350734in}}%
\pgfpathlineto{\pgfqpoint{5.198227in}{2.358887in}}%
\pgfpathlineto{\pgfqpoint{5.198682in}{2.337872in}}%
\pgfpathlineto{\pgfqpoint{5.198830in}{2.419606in}}%
\pgfpathlineto{\pgfqpoint{5.199310in}{2.365067in}}%
\pgfpathlineto{\pgfqpoint{5.200454in}{2.419405in}}%
\pgfpathlineto{\pgfqpoint{5.199470in}{2.329250in}}%
\pgfpathlineto{\pgfqpoint{5.200467in}{2.411951in}}%
\pgfpathlineto{\pgfqpoint{5.200590in}{2.338621in}}%
\pgfpathlineto{\pgfqpoint{5.201267in}{2.413655in}}%
\pgfpathlineto{\pgfqpoint{5.201636in}{2.345186in}}%
\pgfpathlineto{\pgfqpoint{5.201648in}{2.345139in}}%
\pgfpathlineto{\pgfqpoint{5.201661in}{2.348440in}}%
\pgfpathlineto{\pgfqpoint{5.202104in}{2.427849in}}%
\pgfpathlineto{\pgfqpoint{5.202436in}{2.337198in}}%
\pgfpathlineto{\pgfqpoint{5.202891in}{2.416411in}}%
\pgfpathlineto{\pgfqpoint{5.203384in}{2.342632in}}%
\pgfpathlineto{\pgfqpoint{5.204085in}{2.356336in}}%
\pgfpathlineto{\pgfqpoint{5.205107in}{2.413451in}}%
\pgfpathlineto{\pgfqpoint{5.204947in}{2.338270in}}%
\pgfpathlineto{\pgfqpoint{5.205254in}{2.396223in}}%
\pgfpathlineto{\pgfqpoint{5.205734in}{2.326790in}}%
\pgfpathlineto{\pgfqpoint{5.205931in}{2.411292in}}%
\pgfpathlineto{\pgfqpoint{5.206387in}{2.365332in}}%
\pgfpathlineto{\pgfqpoint{5.206916in}{2.410711in}}%
\pgfpathlineto{\pgfqpoint{5.206534in}{2.333589in}}%
\pgfpathlineto{\pgfqpoint{5.207285in}{2.350062in}}%
\pgfpathlineto{\pgfqpoint{5.208085in}{2.333075in}}%
\pgfpathlineto{\pgfqpoint{5.207704in}{2.408231in}}%
\pgfpathlineto{\pgfqpoint{5.208344in}{2.373784in}}%
\pgfpathlineto{\pgfqpoint{5.209267in}{2.411445in}}%
\pgfpathlineto{\pgfqpoint{5.208873in}{2.331207in}}%
\pgfpathlineto{\pgfqpoint{5.209439in}{2.378205in}}%
\pgfpathlineto{\pgfqpoint{5.209661in}{2.326152in}}%
\pgfpathlineto{\pgfqpoint{5.210054in}{2.406811in}}%
\pgfpathlineto{\pgfqpoint{5.210559in}{2.356535in}}%
\pgfpathlineto{\pgfqpoint{5.210596in}{2.404149in}}%
\pgfpathlineto{\pgfqpoint{5.211236in}{2.330647in}}%
\pgfpathlineto{\pgfqpoint{5.211679in}{2.377405in}}%
\pgfpathlineto{\pgfqpoint{5.212688in}{2.401017in}}%
\pgfpathlineto{\pgfqpoint{5.212024in}{2.331960in}}%
\pgfpathlineto{\pgfqpoint{5.212774in}{2.373688in}}%
\pgfpathlineto{\pgfqpoint{5.212811in}{2.332745in}}%
\pgfpathlineto{\pgfqpoint{5.213537in}{2.404503in}}%
\pgfpathlineto{\pgfqpoint{5.213870in}{2.395925in}}%
\pgfpathlineto{\pgfqpoint{5.214325in}{2.409066in}}%
\pgfpathlineto{\pgfqpoint{5.214485in}{2.340595in}}%
\pgfpathlineto{\pgfqpoint{5.214793in}{2.375852in}}%
\pgfpathlineto{\pgfqpoint{5.215273in}{2.344149in}}%
\pgfpathlineto{\pgfqpoint{5.215113in}{2.410067in}}%
\pgfpathlineto{\pgfqpoint{5.215876in}{2.391548in}}%
\pgfpathlineto{\pgfqpoint{5.216799in}{2.409255in}}%
\pgfpathlineto{\pgfqpoint{5.216848in}{2.348725in}}%
\pgfpathlineto{\pgfqpoint{5.216934in}{2.384524in}}%
\pgfpathlineto{\pgfqpoint{5.217882in}{2.340093in}}%
\pgfpathlineto{\pgfqpoint{5.217587in}{2.416439in}}%
\pgfpathlineto{\pgfqpoint{5.218079in}{2.353082in}}%
\pgfpathlineto{\pgfqpoint{5.219162in}{2.409541in}}%
\pgfpathlineto{\pgfqpoint{5.218916in}{2.339179in}}%
\pgfpathlineto{\pgfqpoint{5.219322in}{2.371027in}}%
\pgfpathlineto{\pgfqpoint{5.219704in}{2.339097in}}%
\pgfpathlineto{\pgfqpoint{5.219950in}{2.412166in}}%
\pgfpathlineto{\pgfqpoint{5.220504in}{2.345163in}}%
\pgfpathlineto{\pgfqpoint{5.221611in}{2.405838in}}%
\pgfpathlineto{\pgfqpoint{5.221279in}{2.343475in}}%
\pgfpathlineto{\pgfqpoint{5.221660in}{2.389592in}}%
\pgfpathlineto{\pgfqpoint{5.222534in}{2.412415in}}%
\pgfpathlineto{\pgfqpoint{5.222854in}{2.332498in}}%
\pgfpathlineto{\pgfqpoint{5.223322in}{2.414896in}}%
\pgfpathlineto{\pgfqpoint{5.224048in}{2.385262in}}%
\pgfpathlineto{\pgfqpoint{5.224159in}{2.375830in}}%
\pgfpathlineto{\pgfqpoint{5.224110in}{2.414663in}}%
\pgfpathlineto{\pgfqpoint{5.224196in}{2.408021in}}%
\pgfpathlineto{\pgfqpoint{5.224996in}{2.418255in}}%
\pgfpathlineto{\pgfqpoint{5.224614in}{2.334554in}}%
\pgfpathlineto{\pgfqpoint{5.225193in}{2.351699in}}%
\pgfpathlineto{\pgfqpoint{5.226177in}{2.335577in}}%
\pgfpathlineto{\pgfqpoint{5.225784in}{2.420031in}}%
\pgfpathlineto{\pgfqpoint{5.226300in}{2.344085in}}%
\pgfpathlineto{\pgfqpoint{5.227359in}{2.420900in}}%
\pgfpathlineto{\pgfqpoint{5.226977in}{2.325477in}}%
\pgfpathlineto{\pgfqpoint{5.227445in}{2.400856in}}%
\pgfpathlineto{\pgfqpoint{5.227457in}{2.401501in}}%
\pgfpathlineto{\pgfqpoint{5.227556in}{2.354718in}}%
\pgfpathlineto{\pgfqpoint{5.227765in}{2.329730in}}%
\pgfpathlineto{\pgfqpoint{5.228147in}{2.414870in}}%
\pgfpathlineto{\pgfqpoint{5.228664in}{2.351657in}}%
\pgfpathlineto{\pgfqpoint{5.229722in}{2.408776in}}%
\pgfpathlineto{\pgfqpoint{5.229439in}{2.334685in}}%
\pgfpathlineto{\pgfqpoint{5.229796in}{2.392011in}}%
\pgfpathlineto{\pgfqpoint{5.230510in}{2.401612in}}%
\pgfpathlineto{\pgfqpoint{5.230128in}{2.336675in}}%
\pgfpathlineto{\pgfqpoint{5.230867in}{2.383985in}}%
\pgfpathlineto{\pgfqpoint{5.231802in}{2.341603in}}%
\pgfpathlineto{\pgfqpoint{5.231568in}{2.399152in}}%
\pgfpathlineto{\pgfqpoint{5.231962in}{2.390406in}}%
\pgfpathlineto{\pgfqpoint{5.232417in}{2.407292in}}%
\pgfpathlineto{\pgfqpoint{5.232590in}{2.339430in}}%
\pgfpathlineto{\pgfqpoint{5.233033in}{2.373252in}}%
\pgfpathlineto{\pgfqpoint{5.233377in}{2.342827in}}%
\pgfpathlineto{\pgfqpoint{5.233205in}{2.412637in}}%
\pgfpathlineto{\pgfqpoint{5.233956in}{2.387300in}}%
\pgfpathlineto{\pgfqpoint{5.234891in}{2.411029in}}%
\pgfpathlineto{\pgfqpoint{5.234940in}{2.346752in}}%
\pgfpathlineto{\pgfqpoint{5.235039in}{2.372435in}}%
\pgfpathlineto{\pgfqpoint{5.235174in}{2.346914in}}%
\pgfpathlineto{\pgfqpoint{5.235679in}{2.421853in}}%
\pgfpathlineto{\pgfqpoint{5.236171in}{2.360918in}}%
\pgfpathlineto{\pgfqpoint{5.236467in}{2.414673in}}%
\pgfpathlineto{\pgfqpoint{5.236737in}{2.351808in}}%
\pgfpathlineto{\pgfqpoint{5.237279in}{2.369359in}}%
\pgfpathlineto{\pgfqpoint{5.237980in}{2.348106in}}%
\pgfpathlineto{\pgfqpoint{5.238030in}{2.407960in}}%
\pgfpathlineto{\pgfqpoint{5.238350in}{2.387341in}}%
\pgfpathlineto{\pgfqpoint{5.238817in}{2.401078in}}%
\pgfpathlineto{\pgfqpoint{5.238768in}{2.349211in}}%
\pgfpathlineto{\pgfqpoint{5.239433in}{2.374321in}}%
\pgfpathlineto{\pgfqpoint{5.240343in}{2.341748in}}%
\pgfpathlineto{\pgfqpoint{5.239605in}{2.402487in}}%
\pgfpathlineto{\pgfqpoint{5.240380in}{2.393190in}}%
\pgfpathlineto{\pgfqpoint{5.241402in}{2.410494in}}%
\pgfpathlineto{\pgfqpoint{5.241119in}{2.340962in}}%
\pgfpathlineto{\pgfqpoint{5.241439in}{2.374783in}}%
\pgfpathlineto{\pgfqpoint{5.241907in}{2.342538in}}%
\pgfpathlineto{\pgfqpoint{5.242288in}{2.410890in}}%
\pgfpathlineto{\pgfqpoint{5.242534in}{2.377624in}}%
\pgfpathlineto{\pgfqpoint{5.243076in}{2.417777in}}%
\pgfpathlineto{\pgfqpoint{5.242694in}{2.342865in}}%
\pgfpathlineto{\pgfqpoint{5.243556in}{2.360249in}}%
\pgfpathlineto{\pgfqpoint{5.244257in}{2.342489in}}%
\pgfpathlineto{\pgfqpoint{5.243863in}{2.415013in}}%
\pgfpathlineto{\pgfqpoint{5.244614in}{2.383169in}}%
\pgfpathlineto{\pgfqpoint{5.245439in}{2.412907in}}%
\pgfpathlineto{\pgfqpoint{5.245045in}{2.340821in}}%
\pgfpathlineto{\pgfqpoint{5.245710in}{2.376523in}}%
\pgfpathlineto{\pgfqpoint{5.246608in}{2.346982in}}%
\pgfpathlineto{\pgfqpoint{5.246214in}{2.409891in}}%
\pgfpathlineto{\pgfqpoint{5.246645in}{2.382948in}}%
\pgfpathlineto{\pgfqpoint{5.247002in}{2.410814in}}%
\pgfpathlineto{\pgfqpoint{5.247396in}{2.343718in}}%
\pgfpathlineto{\pgfqpoint{5.247703in}{2.362572in}}%
\pgfpathlineto{\pgfqpoint{5.248183in}{2.343386in}}%
\pgfpathlineto{\pgfqpoint{5.247790in}{2.407919in}}%
\pgfpathlineto{\pgfqpoint{5.248787in}{2.380902in}}%
\pgfpathlineto{\pgfqpoint{5.249353in}{2.402740in}}%
\pgfpathlineto{\pgfqpoint{5.249747in}{2.341567in}}%
\pgfpathlineto{\pgfqpoint{5.249907in}{2.388498in}}%
\pgfpathlineto{\pgfqpoint{5.249931in}{2.382780in}}%
\pgfpathlineto{\pgfqpoint{5.250534in}{2.344133in}}%
\pgfpathlineto{\pgfqpoint{5.250140in}{2.401038in}}%
\pgfpathlineto{\pgfqpoint{5.251063in}{2.362962in}}%
\pgfpathlineto{\pgfqpoint{5.251703in}{2.400078in}}%
\pgfpathlineto{\pgfqpoint{5.251642in}{2.352454in}}%
\pgfpathlineto{\pgfqpoint{5.252085in}{2.360201in}}%
\pgfpathlineto{\pgfqpoint{5.252885in}{2.353218in}}%
\pgfpathlineto{\pgfqpoint{5.252934in}{2.413608in}}%
\pgfpathlineto{\pgfqpoint{5.253156in}{2.376702in}}%
\pgfpathlineto{\pgfqpoint{5.253217in}{2.358262in}}%
\pgfpathlineto{\pgfqpoint{5.253254in}{2.386168in}}%
\pgfpathlineto{\pgfqpoint{5.253710in}{2.407627in}}%
\pgfpathlineto{\pgfqpoint{5.253660in}{2.347903in}}%
\pgfpathlineto{\pgfqpoint{5.254350in}{2.380003in}}%
\pgfpathlineto{\pgfqpoint{5.254448in}{2.344725in}}%
\pgfpathlineto{\pgfqpoint{5.254497in}{2.404063in}}%
\pgfpathlineto{\pgfqpoint{5.255260in}{2.392070in}}%
\pgfpathlineto{\pgfqpoint{5.256060in}{2.408733in}}%
\pgfpathlineto{\pgfqpoint{5.256011in}{2.344986in}}%
\pgfpathlineto{\pgfqpoint{5.256331in}{2.355974in}}%
\pgfpathlineto{\pgfqpoint{5.256380in}{2.381672in}}%
\pgfpathlineto{\pgfqpoint{5.256836in}{2.407360in}}%
\pgfpathlineto{\pgfqpoint{5.256786in}{2.342546in}}%
\pgfpathlineto{\pgfqpoint{5.257488in}{2.383571in}}%
\pgfpathlineto{\pgfqpoint{5.257574in}{2.339973in}}%
\pgfpathlineto{\pgfqpoint{5.257623in}{2.409400in}}%
\pgfpathlineto{\pgfqpoint{5.258497in}{2.397743in}}%
\pgfpathlineto{\pgfqpoint{5.258534in}{2.404859in}}%
\pgfpathlineto{\pgfqpoint{5.259125in}{2.348273in}}%
\pgfpathlineto{\pgfqpoint{5.259925in}{2.340393in}}%
\pgfpathlineto{\pgfqpoint{5.259543in}{2.410433in}}%
\pgfpathlineto{\pgfqpoint{5.260183in}{2.372191in}}%
\pgfpathlineto{\pgfqpoint{5.260319in}{2.407826in}}%
\pgfpathlineto{\pgfqpoint{5.260700in}{2.337127in}}%
\pgfpathlineto{\pgfqpoint{5.261217in}{2.353380in}}%
\pgfpathlineto{\pgfqpoint{5.261488in}{2.333855in}}%
\pgfpathlineto{\pgfqpoint{5.261537in}{2.405453in}}%
\pgfpathlineto{\pgfqpoint{5.262288in}{2.351959in}}%
\pgfpathlineto{\pgfqpoint{5.263100in}{2.407543in}}%
\pgfpathlineto{\pgfqpoint{5.263051in}{2.341907in}}%
\pgfpathlineto{\pgfqpoint{5.263396in}{2.351062in}}%
\pgfpathlineto{\pgfqpoint{5.263888in}{2.412803in}}%
\pgfpathlineto{\pgfqpoint{5.263839in}{2.339867in}}%
\pgfpathlineto{\pgfqpoint{5.264528in}{2.370148in}}%
\pgfpathlineto{\pgfqpoint{5.265451in}{2.410906in}}%
\pgfpathlineto{\pgfqpoint{5.265402in}{2.344324in}}%
\pgfpathlineto{\pgfqpoint{5.265623in}{2.382580in}}%
\pgfpathlineto{\pgfqpoint{5.266522in}{2.335267in}}%
\pgfpathlineto{\pgfqpoint{5.266239in}{2.409255in}}%
\pgfpathlineto{\pgfqpoint{5.266743in}{2.369394in}}%
\pgfpathlineto{\pgfqpoint{5.267310in}{2.343518in}}%
\pgfpathlineto{\pgfqpoint{5.267703in}{2.417157in}}%
\pgfpathlineto{\pgfqpoint{5.267802in}{2.409502in}}%
\pgfpathlineto{\pgfqpoint{5.268786in}{2.336807in}}%
\pgfpathlineto{\pgfqpoint{5.268590in}{2.419247in}}%
\pgfpathlineto{\pgfqpoint{5.268934in}{2.394016in}}%
\pgfpathlineto{\pgfqpoint{5.269377in}{2.427089in}}%
\pgfpathlineto{\pgfqpoint{5.269586in}{2.328319in}}%
\pgfpathlineto{\pgfqpoint{5.270005in}{2.377499in}}%
\pgfpathlineto{\pgfqpoint{5.270423in}{2.334799in}}%
\pgfpathlineto{\pgfqpoint{5.270165in}{2.427825in}}%
\pgfpathlineto{\pgfqpoint{5.270817in}{2.397470in}}%
\pgfpathlineto{\pgfqpoint{5.271740in}{2.403401in}}%
\pgfpathlineto{\pgfqpoint{5.271211in}{2.339777in}}%
\pgfpathlineto{\pgfqpoint{5.271814in}{2.358239in}}%
\pgfpathlineto{\pgfqpoint{5.271937in}{2.336829in}}%
\pgfpathlineto{\pgfqpoint{5.272503in}{2.395759in}}%
\pgfpathlineto{\pgfqpoint{5.273303in}{2.406650in}}%
\pgfpathlineto{\pgfqpoint{5.272713in}{2.342508in}}%
\pgfpathlineto{\pgfqpoint{5.273439in}{2.365036in}}%
\pgfpathlineto{\pgfqpoint{5.273500in}{2.345742in}}%
\pgfpathlineto{\pgfqpoint{5.274091in}{2.399348in}}%
\pgfpathlineto{\pgfqpoint{5.274546in}{2.363083in}}%
\pgfpathlineto{\pgfqpoint{5.275642in}{2.393533in}}%
\pgfpathlineto{\pgfqpoint{5.275063in}{2.355352in}}%
\pgfpathlineto{\pgfqpoint{5.275679in}{2.379020in}}%
\pgfpathlineto{\pgfqpoint{5.276602in}{2.328367in}}%
\pgfpathlineto{\pgfqpoint{5.276429in}{2.399466in}}%
\pgfpathlineto{\pgfqpoint{5.276749in}{2.389456in}}%
\pgfpathlineto{\pgfqpoint{5.276786in}{2.394639in}}%
\pgfpathlineto{\pgfqpoint{5.277414in}{2.348784in}}%
\pgfpathlineto{\pgfqpoint{5.277648in}{2.353435in}}%
\pgfpathlineto{\pgfqpoint{5.278189in}{2.331941in}}%
\pgfpathlineto{\pgfqpoint{5.278399in}{2.411073in}}%
\pgfpathlineto{\pgfqpoint{5.278620in}{2.382207in}}%
\pgfpathlineto{\pgfqpoint{5.278916in}{2.401430in}}%
\pgfpathlineto{\pgfqpoint{5.279642in}{2.338833in}}%
\pgfpathlineto{\pgfqpoint{5.280491in}{2.328672in}}%
\pgfpathlineto{\pgfqpoint{5.280220in}{2.428009in}}%
\pgfpathlineto{\pgfqpoint{5.280639in}{2.386331in}}%
\pgfpathlineto{\pgfqpoint{5.281082in}{2.419011in}}%
\pgfpathlineto{\pgfqpoint{5.281316in}{2.321593in}}%
\pgfpathlineto{\pgfqpoint{5.281783in}{2.416571in}}%
\pgfpathlineto{\pgfqpoint{5.282817in}{2.321966in}}%
\pgfpathlineto{\pgfqpoint{5.282977in}{2.392989in}}%
\pgfpathlineto{\pgfqpoint{5.283002in}{2.433745in}}%
\pgfpathlineto{\pgfqpoint{5.283236in}{2.319582in}}%
\pgfpathlineto{\pgfqpoint{5.284060in}{2.349001in}}%
\pgfpathlineto{\pgfqpoint{5.284540in}{2.430980in}}%
\pgfpathlineto{\pgfqpoint{5.284183in}{2.316986in}}%
\pgfpathlineto{\pgfqpoint{5.285168in}{2.355056in}}%
\pgfpathlineto{\pgfqpoint{5.285993in}{2.288570in}}%
\pgfpathlineto{\pgfqpoint{5.285648in}{2.430512in}}%
\pgfpathlineto{\pgfqpoint{5.286276in}{2.336962in}}%
\pgfpathlineto{\pgfqpoint{5.286460in}{2.464270in}}%
\pgfpathlineto{\pgfqpoint{5.286953in}{2.316409in}}%
\pgfpathlineto{\pgfqpoint{5.287420in}{2.376273in}}%
\pgfpathlineto{\pgfqpoint{5.288060in}{2.319733in}}%
\pgfpathlineto{\pgfqpoint{5.288245in}{2.398203in}}%
\pgfpathlineto{\pgfqpoint{5.288393in}{2.383706in}}%
\pgfpathlineto{\pgfqpoint{5.288516in}{2.482511in}}%
\pgfpathlineto{\pgfqpoint{5.288774in}{2.272506in}}%
\pgfpathlineto{\pgfqpoint{5.289476in}{2.335937in}}%
\pgfpathlineto{\pgfqpoint{5.289586in}{2.310075in}}%
\pgfpathlineto{\pgfqpoint{5.289906in}{2.395697in}}%
\pgfpathlineto{\pgfqpoint{5.290066in}{2.458020in}}%
\pgfpathlineto{\pgfqpoint{5.290829in}{2.297665in}}%
\pgfpathlineto{\pgfqpoint{5.291002in}{2.376493in}}%
\pgfpathlineto{\pgfqpoint{5.291297in}{2.466458in}}%
\pgfpathlineto{\pgfqpoint{5.291556in}{2.290394in}}%
\pgfpathlineto{\pgfqpoint{5.292183in}{2.413308in}}%
\pgfpathlineto{\pgfqpoint{5.292909in}{2.333687in}}%
\pgfpathlineto{\pgfqpoint{5.293254in}{2.426341in}}%
\pgfpathlineto{\pgfqpoint{5.293303in}{2.393551in}}%
\pgfpathlineto{\pgfqpoint{5.293599in}{2.286535in}}%
\pgfpathlineto{\pgfqpoint{5.294066in}{2.469797in}}%
\pgfpathlineto{\pgfqpoint{5.294497in}{2.316896in}}%
\pgfpathlineto{\pgfqpoint{5.294768in}{2.453600in}}%
\pgfpathlineto{\pgfqpoint{5.295666in}{2.355749in}}%
\pgfpathlineto{\pgfqpoint{5.296380in}{2.299429in}}%
\pgfpathlineto{\pgfqpoint{5.296725in}{2.423596in}}%
\pgfpathlineto{\pgfqpoint{5.296848in}{2.463414in}}%
\pgfpathlineto{\pgfqpoint{5.297020in}{2.360304in}}%
\pgfpathlineto{\pgfqpoint{5.297229in}{2.302379in}}%
\pgfpathlineto{\pgfqpoint{5.297525in}{2.446768in}}%
\pgfpathlineto{\pgfqpoint{5.298116in}{2.379375in}}%
\pgfpathlineto{\pgfqpoint{5.299186in}{2.292253in}}%
\pgfpathlineto{\pgfqpoint{5.298817in}{2.423339in}}%
\pgfpathlineto{\pgfqpoint{5.299309in}{2.344577in}}%
\pgfpathlineto{\pgfqpoint{5.299629in}{2.459276in}}%
\pgfpathlineto{\pgfqpoint{5.300011in}{2.300531in}}%
\pgfpathlineto{\pgfqpoint{5.300479in}{2.410548in}}%
\pgfpathlineto{\pgfqpoint{5.301254in}{2.335520in}}%
\pgfpathlineto{\pgfqpoint{5.301574in}{2.418278in}}%
\pgfpathlineto{\pgfqpoint{5.302399in}{2.461612in}}%
\pgfpathlineto{\pgfqpoint{5.301919in}{2.300806in}}%
\pgfpathlineto{\pgfqpoint{5.302571in}{2.364881in}}%
\pgfpathlineto{\pgfqpoint{5.302620in}{2.306857in}}%
\pgfpathlineto{\pgfqpoint{5.303076in}{2.443811in}}%
\pgfpathlineto{\pgfqpoint{5.303666in}{2.394673in}}%
\pgfpathlineto{\pgfqpoint{5.304368in}{2.429604in}}%
\pgfpathlineto{\pgfqpoint{5.304122in}{2.341782in}}%
\pgfpathlineto{\pgfqpoint{5.304552in}{2.371445in}}%
\pgfpathlineto{\pgfqpoint{5.304688in}{2.291592in}}%
\pgfpathlineto{\pgfqpoint{5.305192in}{2.438558in}}%
\pgfpathlineto{\pgfqpoint{5.305685in}{2.357399in}}%
\pgfpathlineto{\pgfqpoint{5.305845in}{2.452472in}}%
\pgfpathlineto{\pgfqpoint{5.306276in}{2.327280in}}%
\pgfpathlineto{\pgfqpoint{5.306780in}{2.348604in}}%
\pgfpathlineto{\pgfqpoint{5.307482in}{2.304413in}}%
\pgfpathlineto{\pgfqpoint{5.307149in}{2.434928in}}%
\pgfpathlineto{\pgfqpoint{5.307679in}{2.383531in}}%
\pgfpathlineto{\pgfqpoint{5.307962in}{2.451820in}}%
\pgfpathlineto{\pgfqpoint{5.308343in}{2.286559in}}%
\pgfpathlineto{\pgfqpoint{5.308811in}{2.415468in}}%
\pgfpathlineto{\pgfqpoint{5.309143in}{2.331136in}}%
\pgfpathlineto{\pgfqpoint{5.309512in}{2.424346in}}%
\pgfpathlineto{\pgfqpoint{5.309906in}{2.413402in}}%
\pgfpathlineto{\pgfqpoint{5.310731in}{2.446994in}}%
\pgfpathlineto{\pgfqpoint{5.310263in}{2.294543in}}%
\pgfpathlineto{\pgfqpoint{5.310916in}{2.341597in}}%
\pgfpathlineto{\pgfqpoint{5.310952in}{2.281024in}}%
\pgfpathlineto{\pgfqpoint{5.311445in}{2.459648in}}%
\pgfpathlineto{\pgfqpoint{5.311999in}{2.392688in}}%
\pgfpathlineto{\pgfqpoint{5.313032in}{2.284259in}}%
\pgfpathlineto{\pgfqpoint{5.312700in}{2.438992in}}%
\pgfpathlineto{\pgfqpoint{5.313205in}{2.361844in}}%
\pgfpathlineto{\pgfqpoint{5.314226in}{2.463145in}}%
\pgfpathlineto{\pgfqpoint{5.313894in}{2.287137in}}%
\pgfpathlineto{\pgfqpoint{5.314374in}{2.430863in}}%
\pgfpathlineto{\pgfqpoint{5.314706in}{2.325923in}}%
\pgfpathlineto{\pgfqpoint{5.315063in}{2.438968in}}%
\pgfpathlineto{\pgfqpoint{5.315482in}{2.433700in}}%
\pgfpathlineto{\pgfqpoint{5.316515in}{2.282254in}}%
\pgfpathlineto{\pgfqpoint{5.316294in}{2.446588in}}%
\pgfpathlineto{\pgfqpoint{5.316712in}{2.337925in}}%
\pgfpathlineto{\pgfqpoint{5.316995in}{2.466319in}}%
\pgfpathlineto{\pgfqpoint{5.317205in}{2.327425in}}%
\pgfpathlineto{\pgfqpoint{5.317869in}{2.383489in}}%
\pgfpathlineto{\pgfqpoint{5.318595in}{2.273766in}}%
\pgfpathlineto{\pgfqpoint{5.318263in}{2.439701in}}%
\pgfpathlineto{\pgfqpoint{5.318940in}{2.427116in}}%
\pgfpathlineto{\pgfqpoint{5.319765in}{2.468371in}}%
\pgfpathlineto{\pgfqpoint{5.319445in}{2.273424in}}%
\pgfpathlineto{\pgfqpoint{5.319974in}{2.339576in}}%
\pgfpathlineto{\pgfqpoint{5.320245in}{2.320882in}}%
\pgfpathlineto{\pgfqpoint{5.320626in}{2.436888in}}%
\pgfpathlineto{\pgfqpoint{5.320995in}{2.390454in}}%
\pgfpathlineto{\pgfqpoint{5.321857in}{2.448933in}}%
\pgfpathlineto{\pgfqpoint{5.321365in}{2.280140in}}%
\pgfpathlineto{\pgfqpoint{5.322017in}{2.381021in}}%
\pgfpathlineto{\pgfqpoint{5.322066in}{2.280076in}}%
\pgfpathlineto{\pgfqpoint{5.322546in}{2.458718in}}%
\pgfpathlineto{\pgfqpoint{5.323125in}{2.374320in}}%
\pgfpathlineto{\pgfqpoint{5.324146in}{2.274873in}}%
\pgfpathlineto{\pgfqpoint{5.323814in}{2.434085in}}%
\pgfpathlineto{\pgfqpoint{5.324306in}{2.350399in}}%
\pgfpathlineto{\pgfqpoint{5.325328in}{2.469751in}}%
\pgfpathlineto{\pgfqpoint{5.324848in}{2.277800in}}%
\pgfpathlineto{\pgfqpoint{5.325500in}{2.394533in}}%
\pgfpathlineto{\pgfqpoint{5.325697in}{2.328937in}}%
\pgfpathlineto{\pgfqpoint{5.326165in}{2.435347in}}%
\pgfpathlineto{\pgfqpoint{5.326595in}{2.422657in}}%
\pgfpathlineto{\pgfqpoint{5.327617in}{2.290583in}}%
\pgfpathlineto{\pgfqpoint{5.327395in}{2.448690in}}%
\pgfpathlineto{\pgfqpoint{5.327851in}{2.358586in}}%
\pgfpathlineto{\pgfqpoint{5.327888in}{2.348827in}}%
\pgfpathlineto{\pgfqpoint{5.328011in}{2.415153in}}%
\pgfpathlineto{\pgfqpoint{5.328048in}{2.408653in}}%
\pgfpathlineto{\pgfqpoint{5.328109in}{2.449310in}}%
\pgfpathlineto{\pgfqpoint{5.328319in}{2.313735in}}%
\pgfpathlineto{\pgfqpoint{5.329106in}{2.355430in}}%
\pgfpathlineto{\pgfqpoint{5.330189in}{2.452281in}}%
\pgfpathlineto{\pgfqpoint{5.329697in}{2.287780in}}%
\pgfpathlineto{\pgfqpoint{5.330288in}{2.377933in}}%
\pgfpathlineto{\pgfqpoint{5.330325in}{2.392982in}}%
\pgfpathlineto{\pgfqpoint{5.330362in}{2.351592in}}%
\pgfpathlineto{\pgfqpoint{5.330399in}{2.284407in}}%
\pgfpathlineto{\pgfqpoint{5.330891in}{2.464614in}}%
\pgfpathlineto{\pgfqpoint{5.331457in}{2.377301in}}%
\pgfpathlineto{\pgfqpoint{5.331469in}{2.377416in}}%
\pgfpathlineto{\pgfqpoint{5.331482in}{2.374970in}}%
\pgfpathlineto{\pgfqpoint{5.332479in}{2.298833in}}%
\pgfpathlineto{\pgfqpoint{5.331592in}{2.430445in}}%
\pgfpathlineto{\pgfqpoint{5.332602in}{2.364929in}}%
\pgfpathlineto{\pgfqpoint{5.332639in}{2.353838in}}%
\pgfpathlineto{\pgfqpoint{5.332675in}{2.368163in}}%
\pgfpathlineto{\pgfqpoint{5.333672in}{2.460566in}}%
\pgfpathlineto{\pgfqpoint{5.333180in}{2.310094in}}%
\pgfpathlineto{\pgfqpoint{5.333808in}{2.406646in}}%
\pgfpathlineto{\pgfqpoint{5.334571in}{2.312562in}}%
\pgfpathlineto{\pgfqpoint{5.334374in}{2.425454in}}%
\pgfpathlineto{\pgfqpoint{5.334903in}{2.400508in}}%
\pgfpathlineto{\pgfqpoint{5.335752in}{2.458520in}}%
\pgfpathlineto{\pgfqpoint{5.335260in}{2.288779in}}%
\pgfpathlineto{\pgfqpoint{5.335925in}{2.345149in}}%
\pgfpathlineto{\pgfqpoint{5.336651in}{2.295714in}}%
\pgfpathlineto{\pgfqpoint{5.336442in}{2.468649in}}%
\pgfpathlineto{\pgfqpoint{5.337008in}{2.390183in}}%
\pgfpathlineto{\pgfqpoint{5.337143in}{2.442720in}}%
\pgfpathlineto{\pgfqpoint{5.337340in}{2.318734in}}%
\pgfpathlineto{\pgfqpoint{5.337894in}{2.377323in}}%
\pgfpathlineto{\pgfqpoint{5.338029in}{2.287559in}}%
\pgfpathlineto{\pgfqpoint{5.338522in}{2.444091in}}%
\pgfpathlineto{\pgfqpoint{5.339014in}{2.345999in}}%
\pgfpathlineto{\pgfqpoint{5.339223in}{2.463385in}}%
\pgfpathlineto{\pgfqpoint{5.339420in}{2.312259in}}%
\pgfpathlineto{\pgfqpoint{5.340097in}{2.338640in}}%
\pgfpathlineto{\pgfqpoint{5.340811in}{2.282541in}}%
\pgfpathlineto{\pgfqpoint{5.340602in}{2.432485in}}%
\pgfpathlineto{\pgfqpoint{5.341131in}{2.410932in}}%
\pgfpathlineto{\pgfqpoint{5.341303in}{2.446722in}}%
\pgfpathlineto{\pgfqpoint{5.341377in}{2.365595in}}%
\pgfpathlineto{\pgfqpoint{5.341463in}{2.372150in}}%
\pgfpathlineto{\pgfqpoint{5.342189in}{2.299948in}}%
\pgfpathlineto{\pgfqpoint{5.341992in}{2.470541in}}%
\pgfpathlineto{\pgfqpoint{5.342558in}{2.402549in}}%
\pgfpathlineto{\pgfqpoint{5.342694in}{2.439709in}}%
\pgfpathlineto{\pgfqpoint{5.342878in}{2.314744in}}%
\pgfpathlineto{\pgfqpoint{5.343026in}{2.336610in}}%
\pgfpathlineto{\pgfqpoint{5.343580in}{2.290367in}}%
\pgfpathlineto{\pgfqpoint{5.343371in}{2.451353in}}%
\pgfpathlineto{\pgfqpoint{5.344023in}{2.405831in}}%
\pgfpathlineto{\pgfqpoint{5.344774in}{2.463049in}}%
\pgfpathlineto{\pgfqpoint{5.344429in}{2.297005in}}%
\pgfpathlineto{\pgfqpoint{5.345082in}{2.353395in}}%
\pgfpathlineto{\pgfqpoint{5.345648in}{2.320718in}}%
\pgfpathlineto{\pgfqpoint{5.345463in}{2.436898in}}%
\pgfpathlineto{\pgfqpoint{5.345968in}{2.391871in}}%
\pgfpathlineto{\pgfqpoint{5.346842in}{2.449726in}}%
\pgfpathlineto{\pgfqpoint{5.346349in}{2.284607in}}%
\pgfpathlineto{\pgfqpoint{5.347014in}{2.356246in}}%
\pgfpathlineto{\pgfqpoint{5.347075in}{2.301611in}}%
\pgfpathlineto{\pgfqpoint{5.347543in}{2.469086in}}%
\pgfpathlineto{\pgfqpoint{5.348097in}{2.406831in}}%
\pgfpathlineto{\pgfqpoint{5.349131in}{2.282359in}}%
\pgfpathlineto{\pgfqpoint{5.348922in}{2.454644in}}%
\pgfpathlineto{\pgfqpoint{5.349315in}{2.347240in}}%
\pgfpathlineto{\pgfqpoint{5.350325in}{2.462116in}}%
\pgfpathlineto{\pgfqpoint{5.349980in}{2.304564in}}%
\pgfpathlineto{\pgfqpoint{5.350460in}{2.397280in}}%
\pgfpathlineto{\pgfqpoint{5.350509in}{2.308569in}}%
\pgfpathlineto{\pgfqpoint{5.351014in}{2.438444in}}%
\pgfpathlineto{\pgfqpoint{5.351555in}{2.415301in}}%
\pgfpathlineto{\pgfqpoint{5.351691in}{2.459696in}}%
\pgfpathlineto{\pgfqpoint{5.351900in}{2.301959in}}%
\pgfpathlineto{\pgfqpoint{5.352565in}{2.352731in}}%
\pgfpathlineto{\pgfqpoint{5.353315in}{2.300192in}}%
\pgfpathlineto{\pgfqpoint{5.353106in}{2.443020in}}%
\pgfpathlineto{\pgfqpoint{5.353635in}{2.415685in}}%
\pgfpathlineto{\pgfqpoint{5.354706in}{2.295333in}}%
\pgfpathlineto{\pgfqpoint{5.354460in}{2.443718in}}%
\pgfpathlineto{\pgfqpoint{5.354940in}{2.364819in}}%
\pgfpathlineto{\pgfqpoint{5.355162in}{2.453901in}}%
\pgfpathlineto{\pgfqpoint{5.355445in}{2.309225in}}%
\pgfpathlineto{\pgfqpoint{5.356011in}{2.346006in}}%
\pgfpathlineto{\pgfqpoint{5.356060in}{2.315272in}}%
\pgfpathlineto{\pgfqpoint{5.356589in}{2.418802in}}%
\pgfpathlineto{\pgfqpoint{5.357057in}{2.391320in}}%
\pgfpathlineto{\pgfqpoint{5.357081in}{2.413315in}}%
\pgfpathlineto{\pgfqpoint{5.357463in}{2.323240in}}%
\pgfpathlineto{\pgfqpoint{5.358128in}{2.361006in}}%
\pgfpathlineto{\pgfqpoint{5.358165in}{2.324649in}}%
\pgfpathlineto{\pgfqpoint{5.358558in}{2.420137in}}%
\pgfpathlineto{\pgfqpoint{5.359211in}{2.376444in}}%
\pgfpathlineto{\pgfqpoint{5.359334in}{2.412378in}}%
\pgfpathlineto{\pgfqpoint{5.360171in}{2.336066in}}%
\pgfpathlineto{\pgfqpoint{5.360343in}{2.411797in}}%
\pgfpathlineto{\pgfqpoint{5.360823in}{2.342206in}}%
\pgfpathlineto{\pgfqpoint{5.360651in}{2.418000in}}%
\pgfpathlineto{\pgfqpoint{5.361451in}{2.404300in}}%
\pgfpathlineto{\pgfqpoint{5.361894in}{2.422918in}}%
\pgfpathlineto{\pgfqpoint{5.361721in}{2.339032in}}%
\pgfpathlineto{\pgfqpoint{5.362485in}{2.342954in}}%
\pgfpathlineto{\pgfqpoint{5.362731in}{2.426029in}}%
\pgfpathlineto{\pgfqpoint{5.363691in}{2.369724in}}%
\pgfpathlineto{\pgfqpoint{5.363728in}{2.338038in}}%
\pgfpathlineto{\pgfqpoint{5.364614in}{2.415843in}}%
\pgfpathlineto{\pgfqpoint{5.364761in}{2.401960in}}%
\pgfpathlineto{\pgfqpoint{5.364786in}{2.415179in}}%
\pgfpathlineto{\pgfqpoint{5.365697in}{2.330563in}}%
\pgfpathlineto{\pgfqpoint{5.365771in}{2.335173in}}%
\pgfpathlineto{\pgfqpoint{5.365783in}{2.331130in}}%
\pgfpathlineto{\pgfqpoint{5.366041in}{2.421592in}}%
\pgfpathlineto{\pgfqpoint{5.366755in}{2.369361in}}%
\pgfpathlineto{\pgfqpoint{5.366866in}{2.429098in}}%
\pgfpathlineto{\pgfqpoint{5.367457in}{2.328580in}}%
\pgfpathlineto{\pgfqpoint{5.367838in}{2.361301in}}%
\pgfpathlineto{\pgfqpoint{5.368257in}{2.344907in}}%
\pgfpathlineto{\pgfqpoint{5.368749in}{2.413098in}}%
\pgfpathlineto{\pgfqpoint{5.368872in}{2.401461in}}%
\pgfpathlineto{\pgfqpoint{5.369352in}{2.417632in}}%
\pgfpathlineto{\pgfqpoint{5.369094in}{2.322934in}}%
\pgfpathlineto{\pgfqpoint{5.369795in}{2.387218in}}%
\pgfpathlineto{\pgfqpoint{5.369918in}{2.323579in}}%
\pgfpathlineto{\pgfqpoint{5.370177in}{2.424376in}}%
\pgfpathlineto{\pgfqpoint{5.370915in}{2.370364in}}%
\pgfpathlineto{\pgfqpoint{5.371161in}{2.319109in}}%
\pgfpathlineto{\pgfqpoint{5.371014in}{2.421433in}}%
\pgfpathlineto{\pgfqpoint{5.372011in}{2.365079in}}%
\pgfpathlineto{\pgfqpoint{5.372651in}{2.419917in}}%
\pgfpathlineto{\pgfqpoint{5.372798in}{2.329398in}}%
\pgfpathlineto{\pgfqpoint{5.373106in}{2.358411in}}%
\pgfpathlineto{\pgfqpoint{5.373217in}{2.315400in}}%
\pgfpathlineto{\pgfqpoint{5.373463in}{2.426023in}}%
\pgfpathlineto{\pgfqpoint{5.374189in}{2.366957in}}%
\pgfpathlineto{\pgfqpoint{5.374300in}{2.424261in}}%
\pgfpathlineto{\pgfqpoint{5.374866in}{2.330020in}}%
\pgfpathlineto{\pgfqpoint{5.375248in}{2.359417in}}%
\pgfpathlineto{\pgfqpoint{5.376097in}{2.315367in}}%
\pgfpathlineto{\pgfqpoint{5.375937in}{2.410487in}}%
\pgfpathlineto{\pgfqpoint{5.376331in}{2.401915in}}%
\pgfpathlineto{\pgfqpoint{5.376343in}{2.402251in}}%
\pgfpathlineto{\pgfqpoint{5.376392in}{2.377966in}}%
\pgfpathlineto{\pgfqpoint{5.376921in}{2.324921in}}%
\pgfpathlineto{\pgfqpoint{5.376749in}{2.415555in}}%
\pgfpathlineto{\pgfqpoint{5.377525in}{2.361948in}}%
\pgfpathlineto{\pgfqpoint{5.378398in}{2.443883in}}%
\pgfpathlineto{\pgfqpoint{5.378152in}{2.319813in}}%
\pgfpathlineto{\pgfqpoint{5.378632in}{2.372821in}}%
\pgfpathlineto{\pgfqpoint{5.378977in}{2.331822in}}%
\pgfpathlineto{\pgfqpoint{5.379223in}{2.422532in}}%
\pgfpathlineto{\pgfqpoint{5.379789in}{2.338386in}}%
\pgfpathlineto{\pgfqpoint{5.380626in}{2.318463in}}%
\pgfpathlineto{\pgfqpoint{5.380060in}{2.419319in}}%
\pgfpathlineto{\pgfqpoint{5.380835in}{2.369269in}}%
\pgfpathlineto{\pgfqpoint{5.381192in}{2.417959in}}%
\pgfpathlineto{\pgfqpoint{5.381438in}{2.318871in}}%
\pgfpathlineto{\pgfqpoint{5.381943in}{2.370302in}}%
\pgfpathlineto{\pgfqpoint{5.382263in}{2.322304in}}%
\pgfpathlineto{\pgfqpoint{5.382608in}{2.419907in}}%
\pgfpathlineto{\pgfqpoint{5.383026in}{2.375052in}}%
\pgfpathlineto{\pgfqpoint{5.383063in}{2.353619in}}%
\pgfpathlineto{\pgfqpoint{5.383912in}{2.320457in}}%
\pgfpathlineto{\pgfqpoint{5.383346in}{2.418341in}}%
\pgfpathlineto{\pgfqpoint{5.384134in}{2.393266in}}%
\pgfpathlineto{\pgfqpoint{5.384983in}{2.422630in}}%
\pgfpathlineto{\pgfqpoint{5.384737in}{2.327057in}}%
\pgfpathlineto{\pgfqpoint{5.385217in}{2.378342in}}%
\pgfpathlineto{\pgfqpoint{5.385561in}{2.314356in}}%
\pgfpathlineto{\pgfqpoint{5.385808in}{2.425484in}}%
\pgfpathlineto{\pgfqpoint{5.386324in}{2.372474in}}%
\pgfpathlineto{\pgfqpoint{5.386386in}{2.314076in}}%
\pgfpathlineto{\pgfqpoint{5.386632in}{2.430472in}}%
\pgfpathlineto{\pgfqpoint{5.387408in}{2.377066in}}%
\pgfpathlineto{\pgfqpoint{5.387457in}{2.434253in}}%
\pgfpathlineto{\pgfqpoint{5.388441in}{2.323591in}}%
\pgfpathlineto{\pgfqpoint{5.388515in}{2.380819in}}%
\pgfpathlineto{\pgfqpoint{5.389266in}{2.326780in}}%
\pgfpathlineto{\pgfqpoint{5.389106in}{2.428136in}}%
\pgfpathlineto{\pgfqpoint{5.389660in}{2.338233in}}%
\pgfpathlineto{\pgfqpoint{5.389672in}{2.328854in}}%
\pgfpathlineto{\pgfqpoint{5.389931in}{2.431381in}}%
\pgfpathlineto{\pgfqpoint{5.390706in}{2.373184in}}%
\pgfpathlineto{\pgfqpoint{5.390755in}{2.429779in}}%
\pgfpathlineto{\pgfqpoint{5.391728in}{2.324084in}}%
\pgfpathlineto{\pgfqpoint{5.391801in}{2.368970in}}%
\pgfpathlineto{\pgfqpoint{5.392552in}{2.322029in}}%
\pgfpathlineto{\pgfqpoint{5.392392in}{2.416133in}}%
\pgfpathlineto{\pgfqpoint{5.392909in}{2.366022in}}%
\pgfpathlineto{\pgfqpoint{5.393217in}{2.421435in}}%
\pgfpathlineto{\pgfqpoint{5.393364in}{2.334795in}}%
\pgfpathlineto{\pgfqpoint{5.393377in}{2.326587in}}%
\pgfpathlineto{\pgfqpoint{5.394054in}{2.435410in}}%
\pgfpathlineto{\pgfqpoint{5.394398in}{2.384439in}}%
\pgfpathlineto{\pgfqpoint{5.395432in}{2.336645in}}%
\pgfpathlineto{\pgfqpoint{5.394866in}{2.432008in}}%
\pgfpathlineto{\pgfqpoint{5.395568in}{2.364385in}}%
\pgfpathlineto{\pgfqpoint{5.395691in}{2.427144in}}%
\pgfpathlineto{\pgfqpoint{5.395838in}{2.333966in}}%
\pgfpathlineto{\pgfqpoint{5.396638in}{2.351708in}}%
\pgfpathlineto{\pgfqpoint{5.396663in}{2.322908in}}%
\pgfpathlineto{\pgfqpoint{5.397697in}{2.408010in}}%
\pgfpathlineto{\pgfqpoint{5.397709in}{2.405278in}}%
\pgfpathlineto{\pgfqpoint{5.397906in}{2.321586in}}%
\pgfpathlineto{\pgfqpoint{5.398238in}{2.417601in}}%
\pgfpathlineto{\pgfqpoint{5.398915in}{2.358390in}}%
\pgfpathlineto{\pgfqpoint{5.399801in}{2.433979in}}%
\pgfpathlineto{\pgfqpoint{5.399961in}{2.322087in}}%
\pgfpathlineto{\pgfqpoint{5.400023in}{2.375445in}}%
\pgfpathlineto{\pgfqpoint{5.400774in}{2.319783in}}%
\pgfpathlineto{\pgfqpoint{5.400626in}{2.431128in}}%
\pgfpathlineto{\pgfqpoint{5.401131in}{2.372664in}}%
\pgfpathlineto{\pgfqpoint{5.401451in}{2.428047in}}%
\pgfpathlineto{\pgfqpoint{5.401598in}{2.333943in}}%
\pgfpathlineto{\pgfqpoint{5.402103in}{2.343669in}}%
\pgfpathlineto{\pgfqpoint{5.402841in}{2.323180in}}%
\pgfpathlineto{\pgfqpoint{5.402263in}{2.434190in}}%
\pgfpathlineto{\pgfqpoint{5.403063in}{2.411371in}}%
\pgfpathlineto{\pgfqpoint{5.403912in}{2.440965in}}%
\pgfpathlineto{\pgfqpoint{5.403567in}{2.325150in}}%
\pgfpathlineto{\pgfqpoint{5.404146in}{2.366815in}}%
\pgfpathlineto{\pgfqpoint{5.404884in}{2.308114in}}%
\pgfpathlineto{\pgfqpoint{5.404737in}{2.436804in}}%
\pgfpathlineto{\pgfqpoint{5.405241in}{2.383700in}}%
\pgfpathlineto{\pgfqpoint{5.405254in}{2.383620in}}%
\pgfpathlineto{\pgfqpoint{5.405709in}{2.316366in}}%
\pgfpathlineto{\pgfqpoint{5.405549in}{2.417444in}}%
\pgfpathlineto{\pgfqpoint{5.406349in}{2.389235in}}%
\pgfpathlineto{\pgfqpoint{5.407198in}{2.439701in}}%
\pgfpathlineto{\pgfqpoint{5.407346in}{2.322299in}}%
\pgfpathlineto{\pgfqpoint{5.407432in}{2.381548in}}%
\pgfpathlineto{\pgfqpoint{5.407666in}{2.320850in}}%
\pgfpathlineto{\pgfqpoint{5.408011in}{2.451906in}}%
\pgfpathlineto{\pgfqpoint{5.408564in}{2.339267in}}%
\pgfpathlineto{\pgfqpoint{5.408995in}{2.308635in}}%
\pgfpathlineto{\pgfqpoint{5.408835in}{2.451199in}}%
\pgfpathlineto{\pgfqpoint{5.409537in}{2.364990in}}%
\pgfpathlineto{\pgfqpoint{5.409660in}{2.430766in}}%
\pgfpathlineto{\pgfqpoint{5.409807in}{2.317257in}}%
\pgfpathlineto{\pgfqpoint{5.410607in}{2.356764in}}%
\pgfpathlineto{\pgfqpoint{5.410632in}{2.324161in}}%
\pgfpathlineto{\pgfqpoint{5.411297in}{2.431146in}}%
\pgfpathlineto{\pgfqpoint{5.411691in}{2.396532in}}%
\pgfpathlineto{\pgfqpoint{5.412121in}{2.446990in}}%
\pgfpathlineto{\pgfqpoint{5.411863in}{2.328465in}}%
\pgfpathlineto{\pgfqpoint{5.412577in}{2.348913in}}%
\pgfpathlineto{\pgfqpoint{5.413094in}{2.313514in}}%
\pgfpathlineto{\pgfqpoint{5.412946in}{2.443523in}}%
\pgfpathlineto{\pgfqpoint{5.413635in}{2.361132in}}%
\pgfpathlineto{\pgfqpoint{5.413758in}{2.430153in}}%
\pgfpathlineto{\pgfqpoint{5.413918in}{2.311801in}}%
\pgfpathlineto{\pgfqpoint{5.414718in}{2.338107in}}%
\pgfpathlineto{\pgfqpoint{5.414743in}{2.317179in}}%
\pgfpathlineto{\pgfqpoint{5.415407in}{2.436301in}}%
\pgfpathlineto{\pgfqpoint{5.415789in}{2.394610in}}%
\pgfpathlineto{\pgfqpoint{5.416232in}{2.450814in}}%
\pgfpathlineto{\pgfqpoint{5.415974in}{2.322427in}}%
\pgfpathlineto{\pgfqpoint{5.416687in}{2.336979in}}%
\pgfpathlineto{\pgfqpoint{5.416786in}{2.330009in}}%
\pgfpathlineto{\pgfqpoint{5.417032in}{2.430145in}}%
\pgfpathlineto{\pgfqpoint{5.417044in}{2.443581in}}%
\pgfpathlineto{\pgfqpoint{5.418029in}{2.324035in}}%
\pgfpathlineto{\pgfqpoint{5.418091in}{2.387972in}}%
\pgfpathlineto{\pgfqpoint{5.418841in}{2.321337in}}%
\pgfpathlineto{\pgfqpoint{5.418694in}{2.428565in}}%
\pgfpathlineto{\pgfqpoint{5.419210in}{2.372562in}}%
\pgfpathlineto{\pgfqpoint{5.420084in}{2.328146in}}%
\pgfpathlineto{\pgfqpoint{5.419518in}{2.429561in}}%
\pgfpathlineto{\pgfqpoint{5.420294in}{2.369167in}}%
\pgfpathlineto{\pgfqpoint{5.421155in}{2.445981in}}%
\pgfpathlineto{\pgfqpoint{5.420897in}{2.328507in}}%
\pgfpathlineto{\pgfqpoint{5.421389in}{2.362782in}}%
\pgfpathlineto{\pgfqpoint{5.422127in}{2.319064in}}%
\pgfpathlineto{\pgfqpoint{5.421980in}{2.443789in}}%
\pgfpathlineto{\pgfqpoint{5.422497in}{2.367679in}}%
\pgfpathlineto{\pgfqpoint{5.422952in}{2.318294in}}%
\pgfpathlineto{\pgfqpoint{5.422792in}{2.432886in}}%
\pgfpathlineto{\pgfqpoint{5.423580in}{2.370418in}}%
\pgfpathlineto{\pgfqpoint{5.424441in}{2.431004in}}%
\pgfpathlineto{\pgfqpoint{5.423777in}{2.331169in}}%
\pgfpathlineto{\pgfqpoint{5.424675in}{2.369586in}}%
\pgfpathlineto{\pgfqpoint{5.425414in}{2.328428in}}%
\pgfpathlineto{\pgfqpoint{5.425266in}{2.437351in}}%
\pgfpathlineto{\pgfqpoint{5.425783in}{2.361532in}}%
\pgfpathlineto{\pgfqpoint{5.426238in}{2.321571in}}%
\pgfpathlineto{\pgfqpoint{5.426078in}{2.440879in}}%
\pgfpathlineto{\pgfqpoint{5.426854in}{2.366199in}}%
\pgfpathlineto{\pgfqpoint{5.426903in}{2.439283in}}%
\pgfpathlineto{\pgfqpoint{5.427050in}{2.318432in}}%
\pgfpathlineto{\pgfqpoint{5.427949in}{2.379425in}}%
\pgfpathlineto{\pgfqpoint{5.428700in}{2.327340in}}%
\pgfpathlineto{\pgfqpoint{5.428552in}{2.434848in}}%
\pgfpathlineto{\pgfqpoint{5.429069in}{2.355541in}}%
\pgfpathlineto{\pgfqpoint{5.429524in}{2.319050in}}%
\pgfpathlineto{\pgfqpoint{5.429364in}{2.441852in}}%
\pgfpathlineto{\pgfqpoint{5.430140in}{2.365205in}}%
\pgfpathlineto{\pgfqpoint{5.430189in}{2.445616in}}%
\pgfpathlineto{\pgfqpoint{5.430349in}{2.315075in}}%
\pgfpathlineto{\pgfqpoint{5.431235in}{2.373024in}}%
\pgfpathlineto{\pgfqpoint{5.431986in}{2.315313in}}%
\pgfpathlineto{\pgfqpoint{5.431838in}{2.434534in}}%
\pgfpathlineto{\pgfqpoint{5.432355in}{2.359733in}}%
\pgfpathlineto{\pgfqpoint{5.432810in}{2.315126in}}%
\pgfpathlineto{\pgfqpoint{5.432663in}{2.435505in}}%
\pgfpathlineto{\pgfqpoint{5.433426in}{2.356574in}}%
\pgfpathlineto{\pgfqpoint{5.434312in}{2.444386in}}%
\pgfpathlineto{\pgfqpoint{5.433635in}{2.306414in}}%
\pgfpathlineto{\pgfqpoint{5.434534in}{2.373459in}}%
\pgfpathlineto{\pgfqpoint{5.435284in}{2.309117in}}%
\pgfpathlineto{\pgfqpoint{5.435137in}{2.440215in}}%
\pgfpathlineto{\pgfqpoint{5.435641in}{2.362319in}}%
\pgfpathlineto{\pgfqpoint{5.436786in}{2.445263in}}%
\pgfpathlineto{\pgfqpoint{5.436109in}{2.312636in}}%
\pgfpathlineto{\pgfqpoint{5.436835in}{2.397920in}}%
\pgfpathlineto{\pgfqpoint{5.436946in}{2.306395in}}%
\pgfpathlineto{\pgfqpoint{5.437610in}{2.451995in}}%
\pgfpathlineto{\pgfqpoint{5.437906in}{2.394955in}}%
\pgfpathlineto{\pgfqpoint{5.438435in}{2.444815in}}%
\pgfpathlineto{\pgfqpoint{5.438090in}{2.336421in}}%
\pgfpathlineto{\pgfqpoint{5.438952in}{2.354498in}}%
\pgfpathlineto{\pgfqpoint{5.439272in}{2.424881in}}%
\pgfpathlineto{\pgfqpoint{5.439654in}{2.350976in}}%
\pgfpathlineto{\pgfqpoint{5.440146in}{2.373400in}}%
\pgfpathlineto{\pgfqpoint{5.441217in}{2.330962in}}%
\pgfpathlineto{\pgfqpoint{5.441007in}{2.414592in}}%
\pgfpathlineto{\pgfqpoint{5.441253in}{2.360287in}}%
\pgfpathlineto{\pgfqpoint{5.441783in}{2.419261in}}%
\pgfpathlineto{\pgfqpoint{5.441930in}{2.317761in}}%
\pgfpathlineto{\pgfqpoint{5.442398in}{2.391941in}}%
\pgfpathlineto{\pgfqpoint{5.442570in}{2.413509in}}%
\pgfpathlineto{\pgfqpoint{5.442718in}{2.337818in}}%
\pgfpathlineto{\pgfqpoint{5.442755in}{2.310983in}}%
\pgfpathlineto{\pgfqpoint{5.443186in}{2.405171in}}%
\pgfpathlineto{\pgfqpoint{5.443764in}{2.375934in}}%
\pgfpathlineto{\pgfqpoint{5.444109in}{2.413798in}}%
\pgfpathlineto{\pgfqpoint{5.444318in}{2.326980in}}%
\pgfpathlineto{\pgfqpoint{5.444921in}{2.388976in}}%
\pgfpathlineto{\pgfqpoint{5.445844in}{2.334090in}}%
\pgfpathlineto{\pgfqpoint{5.445660in}{2.414957in}}%
\pgfpathlineto{\pgfqpoint{5.446078in}{2.360221in}}%
\pgfpathlineto{\pgfqpoint{5.446435in}{2.414166in}}%
\pgfpathlineto{\pgfqpoint{5.446632in}{2.329714in}}%
\pgfpathlineto{\pgfqpoint{5.447247in}{2.393721in}}%
\pgfpathlineto{\pgfqpoint{5.448183in}{2.332490in}}%
\pgfpathlineto{\pgfqpoint{5.447838in}{2.407202in}}%
\pgfpathlineto{\pgfqpoint{5.448404in}{2.369078in}}%
\pgfpathlineto{\pgfqpoint{5.448453in}{2.353073in}}%
\pgfpathlineto{\pgfqpoint{5.448613in}{2.397519in}}%
\pgfpathlineto{\pgfqpoint{5.448737in}{2.394418in}}%
\pgfpathlineto{\pgfqpoint{5.449537in}{2.405948in}}%
\pgfpathlineto{\pgfqpoint{5.448970in}{2.339140in}}%
\pgfpathlineto{\pgfqpoint{5.449795in}{2.370192in}}%
\pgfpathlineto{\pgfqpoint{5.450533in}{2.344188in}}%
\pgfpathlineto{\pgfqpoint{5.450324in}{2.412749in}}%
\pgfpathlineto{\pgfqpoint{5.450878in}{2.373694in}}%
\pgfpathlineto{\pgfqpoint{5.451100in}{2.408022in}}%
\pgfpathlineto{\pgfqpoint{5.451309in}{2.339324in}}%
\pgfpathlineto{\pgfqpoint{5.451998in}{2.376977in}}%
\pgfpathlineto{\pgfqpoint{5.452097in}{2.326926in}}%
\pgfpathlineto{\pgfqpoint{5.453007in}{2.405812in}}%
\pgfpathlineto{\pgfqpoint{5.453130in}{2.358329in}}%
\pgfpathlineto{\pgfqpoint{5.453167in}{2.380862in}}%
\pgfpathlineto{\pgfqpoint{5.453807in}{2.422740in}}%
\pgfpathlineto{\pgfqpoint{5.453660in}{2.341035in}}%
\pgfpathlineto{\pgfqpoint{5.454250in}{2.374159in}}%
\pgfpathlineto{\pgfqpoint{5.454940in}{2.346764in}}%
\pgfpathlineto{\pgfqpoint{5.454595in}{2.412923in}}%
\pgfpathlineto{\pgfqpoint{5.455321in}{2.379851in}}%
\pgfpathlineto{\pgfqpoint{5.456441in}{2.413357in}}%
\pgfpathlineto{\pgfqpoint{5.456035in}{2.350301in}}%
\pgfpathlineto{\pgfqpoint{5.456466in}{2.397648in}}%
\pgfpathlineto{\pgfqpoint{5.457512in}{2.338827in}}%
\pgfpathlineto{\pgfqpoint{5.457229in}{2.421872in}}%
\pgfpathlineto{\pgfqpoint{5.457610in}{2.351813in}}%
\pgfpathlineto{\pgfqpoint{5.458780in}{2.439796in}}%
\pgfpathlineto{\pgfqpoint{5.458312in}{2.335008in}}%
\pgfpathlineto{\pgfqpoint{5.458829in}{2.392261in}}%
\pgfpathlineto{\pgfqpoint{5.459173in}{2.335965in}}%
\pgfpathlineto{\pgfqpoint{5.459567in}{2.434287in}}%
\pgfpathlineto{\pgfqpoint{5.459986in}{2.362014in}}%
\pgfpathlineto{\pgfqpoint{5.460355in}{2.427384in}}%
\pgfpathlineto{\pgfqpoint{5.460232in}{2.333324in}}%
\pgfpathlineto{\pgfqpoint{5.461167in}{2.418277in}}%
\pgfpathlineto{\pgfqpoint{5.461807in}{2.322977in}}%
\pgfpathlineto{\pgfqpoint{5.461943in}{2.430883in}}%
\pgfpathlineto{\pgfqpoint{5.462620in}{2.334769in}}%
\pgfpathlineto{\pgfqpoint{5.463518in}{2.433461in}}%
\pgfpathlineto{\pgfqpoint{5.463395in}{2.333952in}}%
\pgfpathlineto{\pgfqpoint{5.463752in}{2.381247in}}%
\pgfpathlineto{\pgfqpoint{5.464183in}{2.341422in}}%
\pgfpathlineto{\pgfqpoint{5.464318in}{2.432291in}}%
\pgfpathlineto{\pgfqpoint{5.464884in}{2.365466in}}%
\pgfpathlineto{\pgfqpoint{5.465770in}{2.338265in}}%
\pgfpathlineto{\pgfqpoint{5.465106in}{2.419452in}}%
\pgfpathlineto{\pgfqpoint{5.465856in}{2.381049in}}%
\pgfpathlineto{\pgfqpoint{5.465893in}{2.408362in}}%
\pgfpathlineto{\pgfqpoint{5.466546in}{2.334519in}}%
\pgfpathlineto{\pgfqpoint{5.466976in}{2.390843in}}%
\pgfpathlineto{\pgfqpoint{5.467346in}{2.328314in}}%
\pgfpathlineto{\pgfqpoint{5.467715in}{2.408719in}}%
\pgfpathlineto{\pgfqpoint{5.468146in}{2.346531in}}%
\pgfpathlineto{\pgfqpoint{5.468478in}{2.407407in}}%
\pgfpathlineto{\pgfqpoint{5.468909in}{2.338498in}}%
\pgfpathlineto{\pgfqpoint{5.469303in}{2.397795in}}%
\pgfpathlineto{\pgfqpoint{5.469709in}{2.341214in}}%
\pgfpathlineto{\pgfqpoint{5.470066in}{2.400528in}}%
\pgfpathlineto{\pgfqpoint{5.470533in}{2.359821in}}%
\pgfpathlineto{\pgfqpoint{5.471063in}{2.394095in}}%
\pgfpathlineto{\pgfqpoint{5.471284in}{2.346145in}}%
\pgfpathlineto{\pgfqpoint{5.471456in}{2.358534in}}%
\pgfpathlineto{\pgfqpoint{5.471469in}{2.344288in}}%
\pgfpathlineto{\pgfqpoint{5.472380in}{2.399098in}}%
\pgfpathlineto{\pgfqpoint{5.472540in}{2.387549in}}%
\pgfpathlineto{\pgfqpoint{5.473044in}{2.337032in}}%
\pgfpathlineto{\pgfqpoint{5.473155in}{2.395487in}}%
\pgfpathlineto{\pgfqpoint{5.473660in}{2.376065in}}%
\pgfpathlineto{\pgfqpoint{5.474386in}{2.398509in}}%
\pgfpathlineto{\pgfqpoint{5.473832in}{2.338491in}}%
\pgfpathlineto{\pgfqpoint{5.474743in}{2.366809in}}%
\pgfpathlineto{\pgfqpoint{5.475420in}{2.344710in}}%
\pgfpathlineto{\pgfqpoint{5.475173in}{2.399252in}}%
\pgfpathlineto{\pgfqpoint{5.475740in}{2.377060in}}%
\pgfpathlineto{\pgfqpoint{5.476010in}{2.404311in}}%
\pgfpathlineto{\pgfqpoint{5.476404in}{2.337485in}}%
\pgfpathlineto{\pgfqpoint{5.476884in}{2.393827in}}%
\pgfpathlineto{\pgfqpoint{5.477955in}{2.334737in}}%
\pgfpathlineto{\pgfqpoint{5.477672in}{2.404970in}}%
\pgfpathlineto{\pgfqpoint{5.478066in}{2.374561in}}%
\pgfpathlineto{\pgfqpoint{5.478336in}{2.409834in}}%
\pgfpathlineto{\pgfqpoint{5.478767in}{2.330631in}}%
\pgfpathlineto{\pgfqpoint{5.479210in}{2.399078in}}%
\pgfpathlineto{\pgfqpoint{5.479223in}{2.399847in}}%
\pgfpathlineto{\pgfqpoint{5.479469in}{2.356329in}}%
\pgfpathlineto{\pgfqpoint{5.480330in}{2.322815in}}%
\pgfpathlineto{\pgfqpoint{5.479998in}{2.408801in}}%
\pgfpathlineto{\pgfqpoint{5.480552in}{2.369918in}}%
\pgfpathlineto{\pgfqpoint{5.481672in}{2.415536in}}%
\pgfpathlineto{\pgfqpoint{5.481106in}{2.324848in}}%
\pgfpathlineto{\pgfqpoint{5.481696in}{2.396552in}}%
\pgfpathlineto{\pgfqpoint{5.482644in}{2.316016in}}%
\pgfpathlineto{\pgfqpoint{5.482447in}{2.416203in}}%
\pgfpathlineto{\pgfqpoint{5.482804in}{2.399024in}}%
\pgfpathlineto{\pgfqpoint{5.483210in}{2.412414in}}%
\pgfpathlineto{\pgfqpoint{5.483419in}{2.330795in}}%
\pgfpathlineto{\pgfqpoint{5.483789in}{2.363247in}}%
\pgfpathlineto{\pgfqpoint{5.484195in}{2.328660in}}%
\pgfpathlineto{\pgfqpoint{5.484503in}{2.406787in}}%
\pgfpathlineto{\pgfqpoint{5.484712in}{2.382173in}}%
\pgfpathlineto{\pgfqpoint{5.485512in}{2.421011in}}%
\pgfpathlineto{\pgfqpoint{5.485647in}{2.325717in}}%
\pgfpathlineto{\pgfqpoint{5.485807in}{2.369283in}}%
\pgfpathlineto{\pgfqpoint{5.486287in}{2.426623in}}%
\pgfpathlineto{\pgfqpoint{5.486410in}{2.333891in}}%
\pgfpathlineto{\pgfqpoint{5.486459in}{2.318704in}}%
\pgfpathlineto{\pgfqpoint{5.486743in}{2.414097in}}%
\pgfpathlineto{\pgfqpoint{5.487395in}{2.393156in}}%
\pgfpathlineto{\pgfqpoint{5.488023in}{2.327757in}}%
\pgfpathlineto{\pgfqpoint{5.487530in}{2.419027in}}%
\pgfpathlineto{\pgfqpoint{5.488256in}{2.401489in}}%
\pgfpathlineto{\pgfqpoint{5.488355in}{2.428075in}}%
\pgfpathlineto{\pgfqpoint{5.488724in}{2.325846in}}%
\pgfpathlineto{\pgfqpoint{5.489290in}{2.361641in}}%
\pgfpathlineto{\pgfqpoint{5.489524in}{2.313478in}}%
\pgfpathlineto{\pgfqpoint{5.489819in}{2.413798in}}%
\pgfpathlineto{\pgfqpoint{5.490398in}{2.350973in}}%
\pgfpathlineto{\pgfqpoint{5.491506in}{2.413184in}}%
\pgfpathlineto{\pgfqpoint{5.491149in}{2.318661in}}%
\pgfpathlineto{\pgfqpoint{5.491567in}{2.401812in}}%
\pgfpathlineto{\pgfqpoint{5.492699in}{2.312364in}}%
\pgfpathlineto{\pgfqpoint{5.492281in}{2.415394in}}%
\pgfpathlineto{\pgfqpoint{5.492736in}{2.347356in}}%
\pgfpathlineto{\pgfqpoint{5.492773in}{2.339159in}}%
\pgfpathlineto{\pgfqpoint{5.492810in}{2.377868in}}%
\pgfpathlineto{\pgfqpoint{5.493819in}{2.412426in}}%
\pgfpathlineto{\pgfqpoint{5.493475in}{2.313011in}}%
\pgfpathlineto{\pgfqpoint{5.493918in}{2.384559in}}%
\pgfpathlineto{\pgfqpoint{5.494263in}{2.311092in}}%
\pgfpathlineto{\pgfqpoint{5.494607in}{2.419423in}}%
\pgfpathlineto{\pgfqpoint{5.495063in}{2.327594in}}%
\pgfpathlineto{\pgfqpoint{5.495432in}{2.415826in}}%
\pgfpathlineto{\pgfqpoint{5.495826in}{2.318811in}}%
\pgfpathlineto{\pgfqpoint{5.496244in}{2.409836in}}%
\pgfpathlineto{\pgfqpoint{5.496613in}{2.319414in}}%
\pgfpathlineto{\pgfqpoint{5.496995in}{2.410335in}}%
\pgfpathlineto{\pgfqpoint{5.497438in}{2.359059in}}%
\pgfpathlineto{\pgfqpoint{5.497795in}{2.411401in}}%
\pgfpathlineto{\pgfqpoint{5.498115in}{2.330988in}}%
\pgfpathlineto{\pgfqpoint{5.498595in}{2.401697in}}%
\pgfpathlineto{\pgfqpoint{5.498903in}{2.327713in}}%
\pgfpathlineto{\pgfqpoint{5.499346in}{2.404379in}}%
\pgfpathlineto{\pgfqpoint{5.499801in}{2.370328in}}%
\pgfpathlineto{\pgfqpoint{5.500158in}{2.416342in}}%
\pgfpathlineto{\pgfqpoint{5.500478in}{2.344878in}}%
\pgfpathlineto{\pgfqpoint{5.500958in}{2.413191in}}%
\pgfpathlineto{\pgfqpoint{5.501253in}{2.347430in}}%
\pgfpathlineto{\pgfqpoint{5.502115in}{2.372954in}}%
\pgfpathlineto{\pgfqpoint{5.502398in}{2.347322in}}%
\pgfpathlineto{\pgfqpoint{5.502521in}{2.410792in}}%
\pgfpathlineto{\pgfqpoint{5.503210in}{2.370070in}}%
\pgfpathlineto{\pgfqpoint{5.503309in}{2.410096in}}%
\pgfpathlineto{\pgfqpoint{5.503973in}{2.352229in}}%
\pgfpathlineto{\pgfqpoint{5.504318in}{2.377412in}}%
\pgfpathlineto{\pgfqpoint{5.504773in}{2.351654in}}%
\pgfpathlineto{\pgfqpoint{5.504896in}{2.404307in}}%
\pgfpathlineto{\pgfqpoint{5.505413in}{2.380978in}}%
\pgfpathlineto{\pgfqpoint{5.506472in}{2.407749in}}%
\pgfpathlineto{\pgfqpoint{5.506275in}{2.352832in}}%
\pgfpathlineto{\pgfqpoint{5.506509in}{2.385855in}}%
\pgfpathlineto{\pgfqpoint{5.507136in}{2.356601in}}%
\pgfpathlineto{\pgfqpoint{5.507259in}{2.408199in}}%
\pgfpathlineto{\pgfqpoint{5.507616in}{2.374933in}}%
\pgfpathlineto{\pgfqpoint{5.508047in}{2.404044in}}%
\pgfpathlineto{\pgfqpoint{5.508121in}{2.352090in}}%
\pgfpathlineto{\pgfqpoint{5.508687in}{2.368562in}}%
\pgfpathlineto{\pgfqpoint{5.509499in}{2.352591in}}%
\pgfpathlineto{\pgfqpoint{5.508847in}{2.401541in}}%
\pgfpathlineto{\pgfqpoint{5.509770in}{2.372265in}}%
\pgfpathlineto{\pgfqpoint{5.510386in}{2.397381in}}%
\pgfpathlineto{\pgfqpoint{5.510730in}{2.352008in}}%
\pgfpathlineto{\pgfqpoint{5.510853in}{2.365851in}}%
\pgfpathlineto{\pgfqpoint{5.511272in}{2.348005in}}%
\pgfpathlineto{\pgfqpoint{5.511173in}{2.399207in}}%
\pgfpathlineto{\pgfqpoint{5.511912in}{2.394781in}}%
\pgfpathlineto{\pgfqpoint{5.512059in}{2.346341in}}%
\pgfpathlineto{\pgfqpoint{5.512736in}{2.403958in}}%
\pgfpathlineto{\pgfqpoint{5.513204in}{2.366162in}}%
\pgfpathlineto{\pgfqpoint{5.513524in}{2.406896in}}%
\pgfpathlineto{\pgfqpoint{5.513622in}{2.348719in}}%
\pgfpathlineto{\pgfqpoint{5.514373in}{2.383979in}}%
\pgfpathlineto{\pgfqpoint{5.515099in}{2.399723in}}%
\pgfpathlineto{\pgfqpoint{5.515506in}{2.357557in}}%
\pgfpathlineto{\pgfqpoint{5.515875in}{2.396983in}}%
\pgfpathlineto{\pgfqpoint{5.516466in}{2.347038in}}%
\pgfpathlineto{\pgfqpoint{5.516724in}{2.384415in}}%
\pgfpathlineto{\pgfqpoint{5.517253in}{2.355923in}}%
\pgfpathlineto{\pgfqpoint{5.516872in}{2.397614in}}%
\pgfpathlineto{\pgfqpoint{5.517869in}{2.369662in}}%
\pgfpathlineto{\pgfqpoint{5.518422in}{2.394005in}}%
\pgfpathlineto{\pgfqpoint{5.518546in}{2.355617in}}%
\pgfpathlineto{\pgfqpoint{5.518989in}{2.383764in}}%
\pgfpathlineto{\pgfqpoint{5.519038in}{2.389784in}}%
\pgfpathlineto{\pgfqpoint{5.519050in}{2.383934in}}%
\pgfpathlineto{\pgfqpoint{5.520096in}{2.358969in}}%
\pgfpathlineto{\pgfqpoint{5.519813in}{2.391566in}}%
\pgfpathlineto{\pgfqpoint{5.520182in}{2.362545in}}%
\pgfpathlineto{\pgfqpoint{5.520872in}{2.357284in}}%
\pgfpathlineto{\pgfqpoint{5.521364in}{2.388988in}}%
\pgfpathlineto{\pgfqpoint{5.522435in}{2.358067in}}%
\pgfpathlineto{\pgfqpoint{5.522595in}{2.382283in}}%
\pgfpathlineto{\pgfqpoint{5.523419in}{2.393488in}}%
\pgfpathlineto{\pgfqpoint{5.523210in}{2.357148in}}%
\pgfpathlineto{\pgfqpoint{5.523715in}{2.388107in}}%
\pgfpathlineto{\pgfqpoint{5.523998in}{2.357960in}}%
\pgfpathlineto{\pgfqpoint{5.524195in}{2.392870in}}%
\pgfpathlineto{\pgfqpoint{5.524872in}{2.366687in}}%
\pgfpathlineto{\pgfqpoint{5.524982in}{2.391800in}}%
\pgfpathlineto{\pgfqpoint{5.525549in}{2.362924in}}%
\pgfpathlineto{\pgfqpoint{5.525992in}{2.374229in}}%
\pgfpathlineto{\pgfqpoint{5.526545in}{2.390265in}}%
\pgfpathlineto{\pgfqpoint{5.526336in}{2.364657in}}%
\pgfpathlineto{\pgfqpoint{5.526878in}{2.371923in}}%
\pgfpathlineto{\pgfqpoint{5.527173in}{2.365879in}}%
\pgfpathlineto{\pgfqpoint{5.527321in}{2.389180in}}%
\pgfpathlineto{\pgfqpoint{5.527973in}{2.370155in}}%
\pgfpathlineto{\pgfqpoint{5.528096in}{2.388291in}}%
\pgfpathlineto{\pgfqpoint{5.528724in}{2.368974in}}%
\pgfpathlineto{\pgfqpoint{5.529093in}{2.373539in}}%
\pgfpathlineto{\pgfqpoint{5.529499in}{2.369683in}}%
\pgfpathlineto{\pgfqpoint{5.529647in}{2.385358in}}%
\pgfpathlineto{\pgfqpoint{5.530152in}{2.374796in}}%
\pgfpathlineto{\pgfqpoint{5.530422in}{2.384444in}}%
\pgfpathlineto{\pgfqpoint{5.530275in}{2.371129in}}%
\pgfpathlineto{\pgfqpoint{5.531382in}{2.378437in}}%
\pgfpathlineto{\pgfqpoint{5.531665in}{2.371403in}}%
\pgfpathlineto{\pgfqpoint{5.531973in}{2.382610in}}%
\pgfpathlineto{\pgfqpoint{5.532527in}{2.374341in}}%
\pgfpathlineto{\pgfqpoint{5.533130in}{2.373406in}}%
\pgfpathlineto{\pgfqpoint{5.532749in}{2.381335in}}%
\pgfpathlineto{\pgfqpoint{5.533475in}{2.377748in}}%
\pgfpathlineto{\pgfqpoint{5.533536in}{2.378991in}}%
\pgfpathlineto{\pgfqpoint{5.533893in}{2.374699in}}%
\pgfpathlineto{\pgfqpoint{5.534484in}{2.375928in}}%
\pgfpathlineto{\pgfqpoint{5.534545in}{2.375992in}}%
\pgfpathlineto{\pgfqpoint{5.534545in}{2.375992in}}%
\pgfusepath{stroke}%
\end{pgfscope}%
\begin{pgfscope}%
\pgfpathrectangle{\pgfqpoint{0.800000in}{0.528000in}}{\pgfqpoint{4.960000in}{3.696000in}}%
\pgfusepath{clip}%
\pgfsetrectcap%
\pgfsetroundjoin%
\pgfsetlinewidth{1.505625pt}%
\definecolor{currentstroke}{rgb}{0.121569,0.466667,0.705882}%
\pgfsetstrokecolor{currentstroke}%
\pgfsetdash{}{0pt}%
\pgfpathmoveto{\pgfqpoint{1.025455in}{2.375942in}}%
\pgfpathlineto{\pgfqpoint{1.026525in}{2.375801in}}%
\pgfpathlineto{\pgfqpoint{1.025750in}{2.376079in}}%
\pgfpathlineto{\pgfqpoint{1.026562in}{2.375925in}}%
\pgfpathlineto{\pgfqpoint{1.027658in}{2.376105in}}%
\pgfpathlineto{\pgfqpoint{1.026673in}{2.375856in}}%
\pgfpathlineto{\pgfqpoint{1.027707in}{2.376009in}}%
\pgfpathlineto{\pgfqpoint{1.028888in}{2.375908in}}%
\pgfpathlineto{\pgfqpoint{1.027731in}{2.376075in}}%
\pgfpathlineto{\pgfqpoint{1.028975in}{2.376063in}}%
\pgfpathlineto{\pgfqpoint{1.033479in}{2.375873in}}%
\pgfpathlineto{\pgfqpoint{1.034488in}{2.375828in}}%
\pgfpathlineto{\pgfqpoint{1.033516in}{2.375944in}}%
\pgfpathlineto{\pgfqpoint{1.034611in}{2.375899in}}%
\pgfpathlineto{\pgfqpoint{1.036815in}{2.376020in}}%
\pgfpathlineto{\pgfqpoint{1.034636in}{2.376021in}}%
\pgfpathlineto{\pgfqpoint{1.036827in}{2.376075in}}%
\pgfpathlineto{\pgfqpoint{1.037331in}{2.376166in}}%
\pgfpathlineto{\pgfqpoint{1.037528in}{2.375828in}}%
\pgfpathlineto{\pgfqpoint{1.037922in}{2.375920in}}%
\pgfpathlineto{\pgfqpoint{1.041713in}{2.374675in}}%
\pgfpathlineto{\pgfqpoint{1.042796in}{2.374127in}}%
\pgfpathlineto{\pgfqpoint{1.042267in}{2.375140in}}%
\pgfpathlineto{\pgfqpoint{1.042845in}{2.374304in}}%
\pgfpathlineto{\pgfqpoint{1.044285in}{2.377466in}}%
\pgfpathlineto{\pgfqpoint{1.044458in}{2.376922in}}%
\pgfpathlineto{\pgfqpoint{1.045085in}{2.377348in}}%
\pgfpathlineto{\pgfqpoint{1.045885in}{2.376000in}}%
\pgfpathlineto{\pgfqpoint{1.047510in}{2.376010in}}%
\pgfpathlineto{\pgfqpoint{1.056458in}{2.376001in}}%
\pgfpathlineto{\pgfqpoint{1.060777in}{2.375998in}}%
\pgfpathlineto{\pgfqpoint{1.062230in}{2.375993in}}%
\pgfpathlineto{\pgfqpoint{1.063904in}{2.376003in}}%
\pgfpathlineto{\pgfqpoint{1.065996in}{2.375998in}}%
\pgfpathlineto{\pgfqpoint{1.067534in}{2.376006in}}%
\pgfpathlineto{\pgfqpoint{1.068974in}{2.376000in}}%
\pgfpathlineto{\pgfqpoint{1.085799in}{2.376008in}}%
\pgfpathlineto{\pgfqpoint{1.087424in}{2.376008in}}%
\pgfpathlineto{\pgfqpoint{1.185577in}{2.375984in}}%
\pgfpathlineto{\pgfqpoint{1.186795in}{2.376007in}}%
\pgfpathlineto{\pgfqpoint{1.191152in}{2.376003in}}%
\pgfpathlineto{\pgfqpoint{1.193343in}{2.375999in}}%
\pgfpathlineto{\pgfqpoint{1.210131in}{2.375991in}}%
\pgfpathlineto{\pgfqpoint{1.211374in}{2.376013in}}%
\pgfpathlineto{\pgfqpoint{1.212605in}{2.375998in}}%
\pgfpathlineto{\pgfqpoint{1.215078in}{2.376003in}}%
\pgfpathlineto{\pgfqpoint{1.216814in}{2.376008in}}%
\pgfpathlineto{\pgfqpoint{1.237183in}{2.376000in}}%
\pgfpathlineto{\pgfqpoint{1.238758in}{2.376006in}}%
\pgfpathlineto{\pgfqpoint{1.240211in}{2.375990in}}%
\pgfpathlineto{\pgfqpoint{1.242291in}{2.376004in}}%
\pgfpathlineto{\pgfqpoint{1.244014in}{2.375998in}}%
\pgfpathlineto{\pgfqpoint{1.250291in}{2.376012in}}%
\pgfpathlineto{\pgfqpoint{1.251497in}{2.375989in}}%
\pgfpathlineto{\pgfqpoint{1.253269in}{2.375998in}}%
\pgfpathlineto{\pgfqpoint{1.257146in}{2.375990in}}%
\pgfpathlineto{\pgfqpoint{1.258734in}{2.375999in}}%
\pgfpathlineto{\pgfqpoint{1.261011in}{2.376006in}}%
\pgfpathlineto{\pgfqpoint{1.269429in}{2.376003in}}%
\pgfpathlineto{\pgfqpoint{1.289860in}{2.375995in}}%
\pgfpathlineto{\pgfqpoint{1.291189in}{2.375997in}}%
\pgfpathlineto{\pgfqpoint{1.296518in}{2.375995in}}%
\pgfpathlineto{\pgfqpoint{1.298081in}{2.376011in}}%
\pgfpathlineto{\pgfqpoint{1.299447in}{2.375997in}}%
\pgfpathlineto{\pgfqpoint{1.301158in}{2.376009in}}%
\pgfpathlineto{\pgfqpoint{1.302647in}{2.375987in}}%
\pgfpathlineto{\pgfqpoint{1.303977in}{2.376000in}}%
\pgfpathlineto{\pgfqpoint{1.311127in}{2.376004in}}%
\pgfpathlineto{\pgfqpoint{1.312973in}{2.376002in}}%
\pgfpathlineto{\pgfqpoint{1.316752in}{2.375976in}}%
\pgfpathlineto{\pgfqpoint{1.317466in}{2.377794in}}%
\pgfpathlineto{\pgfqpoint{1.318303in}{2.377274in}}%
\pgfpathlineto{\pgfqpoint{1.319620in}{2.375108in}}%
\pgfpathlineto{\pgfqpoint{1.319632in}{2.375124in}}%
\pgfpathlineto{\pgfqpoint{1.320629in}{2.374631in}}%
\pgfpathlineto{\pgfqpoint{1.319817in}{2.375354in}}%
\pgfpathlineto{\pgfqpoint{1.320764in}{2.375299in}}%
\pgfpathlineto{\pgfqpoint{1.322057in}{2.376955in}}%
\pgfpathlineto{\pgfqpoint{1.320826in}{2.375161in}}%
\pgfpathlineto{\pgfqpoint{1.322217in}{2.376764in}}%
\pgfpathlineto{\pgfqpoint{1.323447in}{2.374116in}}%
\pgfpathlineto{\pgfqpoint{1.323620in}{2.374603in}}%
\pgfpathlineto{\pgfqpoint{1.325047in}{2.376124in}}%
\pgfpathlineto{\pgfqpoint{1.325072in}{2.376118in}}%
\pgfpathlineto{\pgfqpoint{1.325318in}{2.375709in}}%
\pgfpathlineto{\pgfqpoint{1.326069in}{2.376798in}}%
\pgfpathlineto{\pgfqpoint{1.326106in}{2.376708in}}%
\pgfpathlineto{\pgfqpoint{1.326820in}{2.377144in}}%
\pgfpathlineto{\pgfqpoint{1.326450in}{2.375909in}}%
\pgfpathlineto{\pgfqpoint{1.327115in}{2.376172in}}%
\pgfpathlineto{\pgfqpoint{1.327681in}{2.376597in}}%
\pgfpathlineto{\pgfqpoint{1.327312in}{2.376608in}}%
\pgfpathlineto{\pgfqpoint{1.327804in}{2.376963in}}%
\pgfpathlineto{\pgfqpoint{1.328100in}{2.377518in}}%
\pgfpathlineto{\pgfqpoint{1.328813in}{2.376080in}}%
\pgfpathlineto{\pgfqpoint{1.328863in}{2.376163in}}%
\pgfpathlineto{\pgfqpoint{1.329601in}{2.374701in}}%
\pgfpathlineto{\pgfqpoint{1.330069in}{2.375045in}}%
\pgfpathlineto{\pgfqpoint{1.331435in}{2.376350in}}%
\pgfpathlineto{\pgfqpoint{1.330241in}{2.374854in}}%
\pgfpathlineto{\pgfqpoint{1.331496in}{2.376232in}}%
\pgfpathlineto{\pgfqpoint{1.333330in}{2.374759in}}%
\pgfpathlineto{\pgfqpoint{1.331890in}{2.376770in}}%
\pgfpathlineto{\pgfqpoint{1.333453in}{2.375044in}}%
\pgfpathlineto{\pgfqpoint{1.334844in}{2.376825in}}%
\pgfpathlineto{\pgfqpoint{1.333712in}{2.374856in}}%
\pgfpathlineto{\pgfqpoint{1.334856in}{2.376816in}}%
\pgfpathlineto{\pgfqpoint{1.336703in}{2.375238in}}%
\pgfpathlineto{\pgfqpoint{1.334893in}{2.377001in}}%
\pgfpathlineto{\pgfqpoint{1.337133in}{2.376021in}}%
\pgfpathlineto{\pgfqpoint{1.338118in}{2.377250in}}%
\pgfpathlineto{\pgfqpoint{1.338352in}{2.376679in}}%
\pgfpathlineto{\pgfqpoint{1.339275in}{2.376135in}}%
\pgfpathlineto{\pgfqpoint{1.338598in}{2.376715in}}%
\pgfpathlineto{\pgfqpoint{1.339496in}{2.376218in}}%
\pgfpathlineto{\pgfqpoint{1.341626in}{2.376597in}}%
\pgfpathlineto{\pgfqpoint{1.341663in}{2.375621in}}%
\pgfpathlineto{\pgfqpoint{1.342684in}{2.355965in}}%
\pgfpathlineto{\pgfqpoint{1.342610in}{2.380064in}}%
\pgfpathlineto{\pgfqpoint{1.342856in}{2.372783in}}%
\pgfpathlineto{\pgfqpoint{1.343029in}{2.391352in}}%
\pgfpathlineto{\pgfqpoint{1.343287in}{2.360522in}}%
\pgfpathlineto{\pgfqpoint{1.344001in}{2.384240in}}%
\pgfpathlineto{\pgfqpoint{1.344875in}{2.334826in}}%
\pgfpathlineto{\pgfqpoint{1.344592in}{2.406281in}}%
\pgfpathlineto{\pgfqpoint{1.345096in}{2.392681in}}%
\pgfpathlineto{\pgfqpoint{1.345133in}{2.402692in}}%
\pgfpathlineto{\pgfqpoint{1.345281in}{2.350991in}}%
\pgfpathlineto{\pgfqpoint{1.345946in}{2.360245in}}%
\pgfpathlineto{\pgfqpoint{1.346943in}{2.288370in}}%
\pgfpathlineto{\pgfqpoint{1.346635in}{2.417327in}}%
\pgfpathlineto{\pgfqpoint{1.347066in}{2.343219in}}%
\pgfpathlineto{\pgfqpoint{1.347952in}{2.449006in}}%
\pgfpathlineto{\pgfqpoint{1.347558in}{2.292356in}}%
\pgfpathlineto{\pgfqpoint{1.348161in}{2.332392in}}%
\pgfpathlineto{\pgfqpoint{1.348173in}{2.329956in}}%
\pgfpathlineto{\pgfqpoint{1.348801in}{2.416415in}}%
\pgfpathlineto{\pgfqpoint{1.349343in}{2.448424in}}%
\pgfpathlineto{\pgfqpoint{1.349490in}{2.307617in}}%
\pgfpathlineto{\pgfqpoint{1.349884in}{2.416008in}}%
\pgfpathlineto{\pgfqpoint{1.350007in}{2.306149in}}%
\pgfpathlineto{\pgfqpoint{1.350327in}{2.431743in}}%
\pgfpathlineto{\pgfqpoint{1.351004in}{2.375969in}}%
\pgfpathlineto{\pgfqpoint{1.351829in}{2.461139in}}%
\pgfpathlineto{\pgfqpoint{1.351607in}{2.296531in}}%
\pgfpathlineto{\pgfqpoint{1.352099in}{2.382092in}}%
\pgfpathlineto{\pgfqpoint{1.352493in}{2.311576in}}%
\pgfpathlineto{\pgfqpoint{1.353072in}{2.438692in}}%
\pgfpathlineto{\pgfqpoint{1.353219in}{2.346147in}}%
\pgfpathlineto{\pgfqpoint{1.354007in}{2.452218in}}%
\pgfpathlineto{\pgfqpoint{1.353429in}{2.296060in}}%
\pgfpathlineto{\pgfqpoint{1.354327in}{2.365563in}}%
\pgfpathlineto{\pgfqpoint{1.355398in}{2.261686in}}%
\pgfpathlineto{\pgfqpoint{1.355016in}{2.488043in}}%
\pgfpathlineto{\pgfqpoint{1.355435in}{2.344332in}}%
\pgfpathlineto{\pgfqpoint{1.355964in}{2.471789in}}%
\pgfpathlineto{\pgfqpoint{1.356210in}{2.286277in}}%
\pgfpathlineto{\pgfqpoint{1.356543in}{2.338495in}}%
\pgfpathlineto{\pgfqpoint{1.356887in}{2.460608in}}%
\pgfpathlineto{\pgfqpoint{1.357170in}{2.278114in}}%
\pgfpathlineto{\pgfqpoint{1.357687in}{2.418029in}}%
\pgfpathlineto{\pgfqpoint{1.357835in}{2.429133in}}%
\pgfpathlineto{\pgfqpoint{1.358093in}{2.323410in}}%
\pgfpathlineto{\pgfqpoint{1.358438in}{2.349114in}}%
\pgfpathlineto{\pgfqpoint{1.359324in}{2.285571in}}%
\pgfpathlineto{\pgfqpoint{1.358832in}{2.468919in}}%
\pgfpathlineto{\pgfqpoint{1.359509in}{2.370998in}}%
\pgfpathlineto{\pgfqpoint{1.360789in}{2.465432in}}%
\pgfpathlineto{\pgfqpoint{1.360321in}{2.296598in}}%
\pgfpathlineto{\pgfqpoint{1.360826in}{2.436457in}}%
\pgfpathlineto{\pgfqpoint{1.360936in}{2.258505in}}%
\pgfpathlineto{\pgfqpoint{1.361527in}{2.448496in}}%
\pgfpathlineto{\pgfqpoint{1.361946in}{2.398538in}}%
\pgfpathlineto{\pgfqpoint{1.362081in}{2.338384in}}%
\pgfpathlineto{\pgfqpoint{1.362019in}{2.444124in}}%
\pgfpathlineto{\pgfqpoint{1.362155in}{2.363336in}}%
\pgfpathlineto{\pgfqpoint{1.363041in}{2.288228in}}%
\pgfpathlineto{\pgfqpoint{1.362647in}{2.452816in}}%
\pgfpathlineto{\pgfqpoint{1.363238in}{2.410874in}}%
\pgfpathlineto{\pgfqpoint{1.364124in}{2.282744in}}%
\pgfpathlineto{\pgfqpoint{1.364383in}{2.459434in}}%
\pgfpathlineto{\pgfqpoint{1.364506in}{2.499757in}}%
\pgfpathlineto{\pgfqpoint{1.364899in}{2.241875in}}%
\pgfpathlineto{\pgfqpoint{1.365367in}{2.369005in}}%
\pgfpathlineto{\pgfqpoint{1.365601in}{2.316853in}}%
\pgfpathlineto{\pgfqpoint{1.366241in}{2.461755in}}%
\pgfpathlineto{\pgfqpoint{1.366266in}{2.508130in}}%
\pgfpathlineto{\pgfqpoint{1.366635in}{2.278491in}}%
\pgfpathlineto{\pgfqpoint{1.367324in}{2.395781in}}%
\pgfpathlineto{\pgfqpoint{1.368136in}{2.489932in}}%
\pgfpathlineto{\pgfqpoint{1.368481in}{2.288815in}}%
\pgfpathlineto{\pgfqpoint{1.368875in}{2.447935in}}%
\pgfpathlineto{\pgfqpoint{1.368604in}{2.250883in}}%
\pgfpathlineto{\pgfqpoint{1.369613in}{2.308786in}}%
\pgfpathlineto{\pgfqpoint{1.370339in}{2.261504in}}%
\pgfpathlineto{\pgfqpoint{1.370007in}{2.554731in}}%
\pgfpathlineto{\pgfqpoint{1.370598in}{2.345377in}}%
\pgfpathlineto{\pgfqpoint{1.370795in}{2.484450in}}%
\pgfpathlineto{\pgfqpoint{1.371472in}{2.292263in}}%
\pgfpathlineto{\pgfqpoint{1.371693in}{2.358866in}}%
\pgfpathlineto{\pgfqpoint{1.372333in}{2.255213in}}%
\pgfpathlineto{\pgfqpoint{1.371878in}{2.516159in}}%
\pgfpathlineto{\pgfqpoint{1.372789in}{2.408839in}}%
\pgfpathlineto{\pgfqpoint{1.372949in}{2.277810in}}%
\pgfpathlineto{\pgfqpoint{1.373158in}{2.468505in}}%
\pgfpathlineto{\pgfqpoint{1.373749in}{2.456411in}}%
\pgfpathlineto{\pgfqpoint{1.373773in}{2.485722in}}%
\pgfpathlineto{\pgfqpoint{1.374192in}{2.247271in}}%
\pgfpathlineto{\pgfqpoint{1.374770in}{2.388932in}}%
\pgfpathlineto{\pgfqpoint{1.375582in}{2.494974in}}%
\pgfpathlineto{\pgfqpoint{1.375915in}{2.267096in}}%
\pgfpathlineto{\pgfqpoint{1.376284in}{2.464317in}}%
\pgfpathlineto{\pgfqpoint{1.376050in}{2.223982in}}%
\pgfpathlineto{\pgfqpoint{1.377096in}{2.366872in}}%
\pgfpathlineto{\pgfqpoint{1.377909in}{2.279673in}}%
\pgfpathlineto{\pgfqpoint{1.377490in}{2.474301in}}%
\pgfpathlineto{\pgfqpoint{1.378179in}{2.391705in}}%
\pgfpathlineto{\pgfqpoint{1.379324in}{2.537217in}}%
\pgfpathlineto{\pgfqpoint{1.379029in}{2.262656in}}%
\pgfpathlineto{\pgfqpoint{1.379361in}{2.476916in}}%
\pgfpathlineto{\pgfqpoint{1.379779in}{2.245377in}}%
\pgfpathlineto{\pgfqpoint{1.380099in}{2.483654in}}%
\pgfpathlineto{\pgfqpoint{1.380604in}{2.353203in}}%
\pgfpathlineto{\pgfqpoint{1.381170in}{2.500258in}}%
\pgfpathlineto{\pgfqpoint{1.380764in}{2.250542in}}%
\pgfpathlineto{\pgfqpoint{1.381712in}{2.353567in}}%
\pgfpathlineto{\pgfqpoint{1.381946in}{2.465392in}}%
\pgfpathlineto{\pgfqpoint{1.382610in}{2.303531in}}%
\pgfpathlineto{\pgfqpoint{1.383373in}{2.259375in}}%
\pgfpathlineto{\pgfqpoint{1.383066in}{2.482859in}}%
\pgfpathlineto{\pgfqpoint{1.383632in}{2.398366in}}%
\pgfpathlineto{\pgfqpoint{1.384038in}{2.454017in}}%
\pgfpathlineto{\pgfqpoint{1.384481in}{2.270125in}}%
\pgfpathlineto{\pgfqpoint{1.384690in}{2.345595in}}%
\pgfpathlineto{\pgfqpoint{1.385219in}{2.251688in}}%
\pgfpathlineto{\pgfqpoint{1.384875in}{2.516263in}}%
\pgfpathlineto{\pgfqpoint{1.385773in}{2.380212in}}%
\pgfpathlineto{\pgfqpoint{1.386561in}{2.257812in}}%
\pgfpathlineto{\pgfqpoint{1.385909in}{2.428965in}}%
\pgfpathlineto{\pgfqpoint{1.386672in}{2.411685in}}%
\pgfpathlineto{\pgfqpoint{1.386758in}{2.497976in}}%
\pgfpathlineto{\pgfqpoint{1.387078in}{2.234227in}}%
\pgfpathlineto{\pgfqpoint{1.387767in}{2.379750in}}%
\pgfpathlineto{\pgfqpoint{1.388924in}{2.225194in}}%
\pgfpathlineto{\pgfqpoint{1.388592in}{2.503296in}}%
\pgfpathlineto{\pgfqpoint{1.388949in}{2.260262in}}%
\pgfpathlineto{\pgfqpoint{1.389490in}{2.484823in}}%
\pgfpathlineto{\pgfqpoint{1.390118in}{2.361560in}}%
\pgfpathlineto{\pgfqpoint{1.390807in}{2.262452in}}%
\pgfpathlineto{\pgfqpoint{1.391102in}{2.477281in}}%
\pgfpathlineto{\pgfqpoint{1.392025in}{2.231568in}}%
\pgfpathlineto{\pgfqpoint{1.392284in}{2.414043in}}%
\pgfpathlineto{\pgfqpoint{1.392345in}{2.494852in}}%
\pgfpathlineto{\pgfqpoint{1.392764in}{2.257266in}}%
\pgfpathlineto{\pgfqpoint{1.393256in}{2.363215in}}%
\pgfpathlineto{\pgfqpoint{1.393453in}{2.421999in}}%
\pgfpathlineto{\pgfqpoint{1.393662in}{2.328363in}}%
\pgfpathlineto{\pgfqpoint{1.393798in}{2.338400in}}%
\pgfpathlineto{\pgfqpoint{1.394635in}{2.250279in}}%
\pgfpathlineto{\pgfqpoint{1.394167in}{2.520866in}}%
\pgfpathlineto{\pgfqpoint{1.394672in}{2.368414in}}%
\pgfpathlineto{\pgfqpoint{1.394942in}{2.455977in}}%
\pgfpathlineto{\pgfqpoint{1.395127in}{2.294383in}}%
\pgfpathlineto{\pgfqpoint{1.395755in}{2.301895in}}%
\pgfpathlineto{\pgfqpoint{1.396062in}{2.489246in}}%
\pgfpathlineto{\pgfqpoint{1.395878in}{2.274982in}}%
\pgfpathlineto{\pgfqpoint{1.396936in}{2.409158in}}%
\pgfpathlineto{\pgfqpoint{1.397736in}{2.282579in}}%
\pgfpathlineto{\pgfqpoint{1.397281in}{2.520828in}}%
\pgfpathlineto{\pgfqpoint{1.398019in}{2.414086in}}%
\pgfpathlineto{\pgfqpoint{1.398056in}{2.469751in}}%
\pgfpathlineto{\pgfqpoint{1.398352in}{2.281709in}}%
\pgfpathlineto{\pgfqpoint{1.399139in}{2.440997in}}%
\pgfpathlineto{\pgfqpoint{1.399176in}{2.468048in}}%
\pgfpathlineto{\pgfqpoint{1.399324in}{2.290671in}}%
\pgfpathlineto{\pgfqpoint{1.400087in}{2.297225in}}%
\pgfpathlineto{\pgfqpoint{1.400210in}{2.279527in}}%
\pgfpathlineto{\pgfqpoint{1.400370in}{2.421369in}}%
\pgfpathlineto{\pgfqpoint{1.400407in}{2.468803in}}%
\pgfpathlineto{\pgfqpoint{1.401342in}{2.277911in}}%
\pgfpathlineto{\pgfqpoint{1.401429in}{2.291599in}}%
\pgfpathlineto{\pgfqpoint{1.401589in}{2.450984in}}%
\pgfpathlineto{\pgfqpoint{1.402659in}{2.348524in}}%
\pgfpathlineto{\pgfqpoint{1.403189in}{2.272041in}}%
\pgfpathlineto{\pgfqpoint{1.402844in}{2.486261in}}%
\pgfpathlineto{\pgfqpoint{1.403779in}{2.320293in}}%
\pgfpathlineto{\pgfqpoint{1.404542in}{2.292328in}}%
\pgfpathlineto{\pgfqpoint{1.404715in}{2.477250in}}%
\pgfpathlineto{\pgfqpoint{1.404813in}{2.368529in}}%
\pgfpathlineto{\pgfqpoint{1.405035in}{2.263856in}}%
\pgfpathlineto{\pgfqpoint{1.405945in}{2.468314in}}%
\pgfpathlineto{\pgfqpoint{1.406881in}{2.249797in}}%
\pgfpathlineto{\pgfqpoint{1.406721in}{2.476658in}}%
\pgfpathlineto{\pgfqpoint{1.407152in}{2.415193in}}%
\pgfpathlineto{\pgfqpoint{1.407176in}{2.453418in}}%
\pgfpathlineto{\pgfqpoint{1.408149in}{2.284435in}}%
\pgfpathlineto{\pgfqpoint{1.408235in}{2.366257in}}%
\pgfpathlineto{\pgfqpoint{1.408764in}{2.295563in}}%
\pgfpathlineto{\pgfqpoint{1.409059in}{2.494137in}}%
\pgfpathlineto{\pgfqpoint{1.409318in}{2.364030in}}%
\pgfpathlineto{\pgfqpoint{1.410302in}{2.484100in}}%
\pgfpathlineto{\pgfqpoint{1.409995in}{2.258750in}}%
\pgfpathlineto{\pgfqpoint{1.410450in}{2.427727in}}%
\pgfpathlineto{\pgfqpoint{1.410745in}{2.279003in}}%
\pgfpathlineto{\pgfqpoint{1.411422in}{2.458521in}}%
\pgfpathlineto{\pgfqpoint{1.411582in}{2.385859in}}%
\pgfpathlineto{\pgfqpoint{1.412173in}{2.486676in}}%
\pgfpathlineto{\pgfqpoint{1.412481in}{2.298720in}}%
\pgfpathlineto{\pgfqpoint{1.412555in}{2.334936in}}%
\pgfpathlineto{\pgfqpoint{1.412604in}{2.274584in}}%
\pgfpathlineto{\pgfqpoint{1.412899in}{2.441258in}}%
\pgfpathlineto{\pgfqpoint{1.413638in}{2.362111in}}%
\pgfpathlineto{\pgfqpoint{1.414019in}{2.447320in}}%
\pgfpathlineto{\pgfqpoint{1.414339in}{2.286959in}}%
\pgfpathlineto{\pgfqpoint{1.414745in}{2.369225in}}%
\pgfpathlineto{\pgfqpoint{1.414955in}{2.286945in}}%
\pgfpathlineto{\pgfqpoint{1.415238in}{2.460936in}}%
\pgfpathlineto{\pgfqpoint{1.415755in}{2.377536in}}%
\pgfpathlineto{\pgfqpoint{1.415878in}{2.442619in}}%
\pgfpathlineto{\pgfqpoint{1.416210in}{2.265808in}}%
\pgfpathlineto{\pgfqpoint{1.416850in}{2.348065in}}%
\pgfpathlineto{\pgfqpoint{1.417650in}{2.430252in}}%
\pgfpathlineto{\pgfqpoint{1.417453in}{2.297560in}}%
\pgfpathlineto{\pgfqpoint{1.417982in}{2.390261in}}%
\pgfpathlineto{\pgfqpoint{1.418068in}{2.253418in}}%
\pgfpathlineto{\pgfqpoint{1.418364in}{2.454514in}}%
\pgfpathlineto{\pgfqpoint{1.419090in}{2.385014in}}%
\pgfpathlineto{\pgfqpoint{1.419508in}{2.439231in}}%
\pgfpathlineto{\pgfqpoint{1.419324in}{2.290316in}}%
\pgfpathlineto{\pgfqpoint{1.420235in}{2.423359in}}%
\pgfpathlineto{\pgfqpoint{1.421182in}{2.277228in}}%
\pgfpathlineto{\pgfqpoint{1.420813in}{2.457125in}}%
\pgfpathlineto{\pgfqpoint{1.421392in}{2.396869in}}%
\pgfpathlineto{\pgfqpoint{1.421810in}{2.306044in}}%
\pgfpathlineto{\pgfqpoint{1.421490in}{2.457728in}}%
\pgfpathlineto{\pgfqpoint{1.422585in}{2.370289in}}%
\pgfpathlineto{\pgfqpoint{1.422622in}{2.453723in}}%
\pgfpathlineto{\pgfqpoint{1.423028in}{2.318265in}}%
\pgfpathlineto{\pgfqpoint{1.423681in}{2.333166in}}%
\pgfpathlineto{\pgfqpoint{1.423952in}{2.459072in}}%
\pgfpathlineto{\pgfqpoint{1.423804in}{2.312499in}}%
\pgfpathlineto{\pgfqpoint{1.424813in}{2.356384in}}%
\pgfpathlineto{\pgfqpoint{1.425638in}{2.289907in}}%
\pgfpathlineto{\pgfqpoint{1.425158in}{2.434530in}}%
\pgfpathlineto{\pgfqpoint{1.425908in}{2.369551in}}%
\pgfpathlineto{\pgfqpoint{1.426856in}{2.320841in}}%
\pgfpathlineto{\pgfqpoint{1.427053in}{2.470463in}}%
\pgfpathlineto{\pgfqpoint{1.427065in}{2.483320in}}%
\pgfpathlineto{\pgfqpoint{1.427976in}{2.292103in}}%
\pgfpathlineto{\pgfqpoint{1.428075in}{2.379289in}}%
\pgfpathlineto{\pgfqpoint{1.428764in}{2.291029in}}%
\pgfpathlineto{\pgfqpoint{1.428308in}{2.454000in}}%
\pgfpathlineto{\pgfqpoint{1.429219in}{2.323000in}}%
\pgfpathlineto{\pgfqpoint{1.430179in}{2.456664in}}%
\pgfpathlineto{\pgfqpoint{1.430007in}{2.313858in}}%
\pgfpathlineto{\pgfqpoint{1.430376in}{2.379893in}}%
\pgfpathlineto{\pgfqpoint{1.431102in}{2.319726in}}%
\pgfpathlineto{\pgfqpoint{1.430905in}{2.434127in}}%
\pgfpathlineto{\pgfqpoint{1.431398in}{2.421943in}}%
\pgfpathlineto{\pgfqpoint{1.431422in}{2.438041in}}%
\pgfpathlineto{\pgfqpoint{1.431890in}{2.299237in}}%
\pgfpathlineto{\pgfqpoint{1.432431in}{2.359732in}}%
\pgfpathlineto{\pgfqpoint{1.433133in}{2.316543in}}%
\pgfpathlineto{\pgfqpoint{1.433256in}{2.434309in}}%
\pgfpathlineto{\pgfqpoint{1.433564in}{2.324672in}}%
\pgfpathlineto{\pgfqpoint{1.433884in}{2.431740in}}%
\pgfpathlineto{\pgfqpoint{1.434167in}{2.320895in}}%
\pgfpathlineto{\pgfqpoint{1.434696in}{2.371397in}}%
\pgfpathlineto{\pgfqpoint{1.434991in}{2.308345in}}%
\pgfpathlineto{\pgfqpoint{1.435275in}{2.412847in}}%
\pgfpathlineto{\pgfqpoint{1.435656in}{2.383913in}}%
\pgfpathlineto{\pgfqpoint{1.435681in}{2.440892in}}%
\pgfpathlineto{\pgfqpoint{1.436025in}{2.331901in}}%
\pgfpathlineto{\pgfqpoint{1.436751in}{2.374472in}}%
\pgfpathlineto{\pgfqpoint{1.437478in}{2.311748in}}%
\pgfpathlineto{\pgfqpoint{1.437601in}{2.435086in}}%
\pgfpathlineto{\pgfqpoint{1.437773in}{2.400282in}}%
\pgfpathlineto{\pgfqpoint{1.438105in}{2.319739in}}%
\pgfpathlineto{\pgfqpoint{1.438228in}{2.425950in}}%
\pgfpathlineto{\pgfqpoint{1.438795in}{2.402509in}}%
\pgfpathlineto{\pgfqpoint{1.438831in}{2.426718in}}%
\pgfpathlineto{\pgfqpoint{1.439755in}{2.302355in}}%
\pgfpathlineto{\pgfqpoint{1.439878in}{2.371813in}}%
\pgfpathlineto{\pgfqpoint{1.440604in}{2.311883in}}%
\pgfpathlineto{\pgfqpoint{1.440062in}{2.435565in}}%
\pgfpathlineto{\pgfqpoint{1.440887in}{2.392668in}}%
\pgfpathlineto{\pgfqpoint{1.441958in}{2.448124in}}%
\pgfpathlineto{\pgfqpoint{1.441835in}{2.317780in}}%
\pgfpathlineto{\pgfqpoint{1.441995in}{2.395170in}}%
\pgfpathlineto{\pgfqpoint{1.442868in}{2.321620in}}%
\pgfpathlineto{\pgfqpoint{1.442733in}{2.422300in}}%
\pgfpathlineto{\pgfqpoint{1.443115in}{2.359247in}}%
\pgfpathlineto{\pgfqpoint{1.443188in}{2.440980in}}%
\pgfpathlineto{\pgfqpoint{1.443681in}{2.318454in}}%
\pgfpathlineto{\pgfqpoint{1.444235in}{2.377867in}}%
\pgfpathlineto{\pgfqpoint{1.444961in}{2.328575in}}%
\pgfpathlineto{\pgfqpoint{1.445071in}{2.433833in}}%
\pgfpathlineto{\pgfqpoint{1.445379in}{2.353952in}}%
\pgfpathlineto{\pgfqpoint{1.445638in}{2.423429in}}%
\pgfpathlineto{\pgfqpoint{1.445576in}{2.331772in}}%
\pgfpathlineto{\pgfqpoint{1.446511in}{2.390423in}}%
\pgfpathlineto{\pgfqpoint{1.446807in}{2.309040in}}%
\pgfpathlineto{\pgfqpoint{1.446930in}{2.434509in}}%
\pgfpathlineto{\pgfqpoint{1.447619in}{2.390569in}}%
\pgfpathlineto{\pgfqpoint{1.448038in}{2.323319in}}%
\pgfpathlineto{\pgfqpoint{1.447705in}{2.417972in}}%
\pgfpathlineto{\pgfqpoint{1.448727in}{2.383026in}}%
\pgfpathlineto{\pgfqpoint{1.448764in}{2.438861in}}%
\pgfpathlineto{\pgfqpoint{1.449084in}{2.327753in}}%
\pgfpathlineto{\pgfqpoint{1.449835in}{2.377141in}}%
\pgfpathlineto{\pgfqpoint{1.449933in}{2.294118in}}%
\pgfpathlineto{\pgfqpoint{1.450413in}{2.413310in}}%
\pgfpathlineto{\pgfqpoint{1.450967in}{2.349534in}}%
\pgfpathlineto{\pgfqpoint{1.451631in}{2.424719in}}%
\pgfpathlineto{\pgfqpoint{1.451164in}{2.300696in}}%
\pgfpathlineto{\pgfqpoint{1.452148in}{2.410790in}}%
\pgfpathlineto{\pgfqpoint{1.453047in}{2.299469in}}%
\pgfpathlineto{\pgfqpoint{1.452481in}{2.425123in}}%
\pgfpathlineto{\pgfqpoint{1.453256in}{2.399977in}}%
\pgfpathlineto{\pgfqpoint{1.453724in}{2.435475in}}%
\pgfpathlineto{\pgfqpoint{1.453859in}{2.318383in}}%
\pgfpathlineto{\pgfqpoint{1.454253in}{2.372419in}}%
\pgfpathlineto{\pgfqpoint{1.454905in}{2.302892in}}%
\pgfpathlineto{\pgfqpoint{1.454967in}{2.422853in}}%
\pgfpathlineto{\pgfqpoint{1.455348in}{2.394216in}}%
\pgfpathlineto{\pgfqpoint{1.456001in}{2.418316in}}%
\pgfpathlineto{\pgfqpoint{1.456148in}{2.319711in}}%
\pgfpathlineto{\pgfqpoint{1.456444in}{2.388575in}}%
\pgfpathlineto{\pgfqpoint{1.456764in}{2.304986in}}%
\pgfpathlineto{\pgfqpoint{1.456862in}{2.424961in}}%
\pgfpathlineto{\pgfqpoint{1.457564in}{2.368145in}}%
\pgfpathlineto{\pgfqpoint{1.457847in}{2.429401in}}%
\pgfpathlineto{\pgfqpoint{1.457785in}{2.304931in}}%
\pgfpathlineto{\pgfqpoint{1.458610in}{2.335561in}}%
\pgfpathlineto{\pgfqpoint{1.459262in}{2.319666in}}%
\pgfpathlineto{\pgfqpoint{1.458905in}{2.417699in}}%
\pgfpathlineto{\pgfqpoint{1.459484in}{2.381981in}}%
\pgfpathlineto{\pgfqpoint{1.459730in}{2.434629in}}%
\pgfpathlineto{\pgfqpoint{1.459890in}{2.309760in}}%
\pgfpathlineto{\pgfqpoint{1.460481in}{2.339167in}}%
\pgfpathlineto{\pgfqpoint{1.460493in}{2.320276in}}%
\pgfpathlineto{\pgfqpoint{1.460973in}{2.431973in}}%
\pgfpathlineto{\pgfqpoint{1.461539in}{2.377959in}}%
\pgfpathlineto{\pgfqpoint{1.461576in}{2.433008in}}%
\pgfpathlineto{\pgfqpoint{1.461724in}{2.311492in}}%
\pgfpathlineto{\pgfqpoint{1.462671in}{2.418244in}}%
\pgfpathlineto{\pgfqpoint{1.462967in}{2.310898in}}%
\pgfpathlineto{\pgfqpoint{1.462819in}{2.431741in}}%
\pgfpathlineto{\pgfqpoint{1.463828in}{2.374144in}}%
\pgfpathlineto{\pgfqpoint{1.464702in}{2.429031in}}%
\pgfpathlineto{\pgfqpoint{1.464825in}{2.306373in}}%
\pgfpathlineto{\pgfqpoint{1.464850in}{2.282253in}}%
\pgfpathlineto{\pgfqpoint{1.464911in}{2.417859in}}%
\pgfpathlineto{\pgfqpoint{1.465896in}{2.372353in}}%
\pgfpathlineto{\pgfqpoint{1.466278in}{2.426941in}}%
\pgfpathlineto{\pgfqpoint{1.466081in}{2.279030in}}%
\pgfpathlineto{\pgfqpoint{1.467016in}{2.394144in}}%
\pgfpathlineto{\pgfqpoint{1.467951in}{2.285001in}}%
\pgfpathlineto{\pgfqpoint{1.467521in}{2.418727in}}%
\pgfpathlineto{\pgfqpoint{1.468099in}{2.390014in}}%
\pgfpathlineto{\pgfqpoint{1.468136in}{2.419381in}}%
\pgfpathlineto{\pgfqpoint{1.468579in}{2.329459in}}%
\pgfpathlineto{\pgfqpoint{1.469158in}{2.361268in}}%
\pgfpathlineto{\pgfqpoint{1.469810in}{2.290704in}}%
\pgfpathlineto{\pgfqpoint{1.470118in}{2.417307in}}%
\pgfpathlineto{\pgfqpoint{1.470253in}{2.378244in}}%
\pgfpathlineto{\pgfqpoint{1.470265in}{2.378731in}}%
\pgfpathlineto{\pgfqpoint{1.470314in}{2.351030in}}%
\pgfpathlineto{\pgfqpoint{1.471053in}{2.304241in}}%
\pgfpathlineto{\pgfqpoint{1.471176in}{2.416511in}}%
\pgfpathlineto{\pgfqpoint{1.471398in}{2.364945in}}%
\pgfpathlineto{\pgfqpoint{1.471964in}{2.442126in}}%
\pgfpathlineto{\pgfqpoint{1.471668in}{2.308096in}}%
\pgfpathlineto{\pgfqpoint{1.472567in}{2.413955in}}%
\pgfpathlineto{\pgfqpoint{1.472924in}{2.322516in}}%
\pgfpathlineto{\pgfqpoint{1.472751in}{2.417510in}}%
\pgfpathlineto{\pgfqpoint{1.473736in}{2.382943in}}%
\pgfpathlineto{\pgfqpoint{1.473810in}{2.419914in}}%
\pgfpathlineto{\pgfqpoint{1.474142in}{2.337889in}}%
\pgfpathlineto{\pgfqpoint{1.474167in}{2.305400in}}%
\pgfpathlineto{\pgfqpoint{1.474905in}{2.419890in}}%
\pgfpathlineto{\pgfqpoint{1.475225in}{2.389736in}}%
\pgfpathlineto{\pgfqpoint{1.475237in}{2.389789in}}%
\pgfpathlineto{\pgfqpoint{1.475397in}{2.311247in}}%
\pgfpathlineto{\pgfqpoint{1.475681in}{2.431233in}}%
\pgfpathlineto{\pgfqpoint{1.476370in}{2.368459in}}%
\pgfpathlineto{\pgfqpoint{1.477453in}{2.429245in}}%
\pgfpathlineto{\pgfqpoint{1.476628in}{2.312922in}}%
\pgfpathlineto{\pgfqpoint{1.477490in}{2.392259in}}%
\pgfpathlineto{\pgfqpoint{1.477859in}{2.293601in}}%
\pgfpathlineto{\pgfqpoint{1.477551in}{2.418473in}}%
\pgfpathlineto{\pgfqpoint{1.478622in}{2.361537in}}%
\pgfpathlineto{\pgfqpoint{1.479287in}{2.417502in}}%
\pgfpathlineto{\pgfqpoint{1.479127in}{2.318852in}}%
\pgfpathlineto{\pgfqpoint{1.479681in}{2.349685in}}%
\pgfpathlineto{\pgfqpoint{1.479742in}{2.294316in}}%
\pgfpathlineto{\pgfqpoint{1.479816in}{2.419076in}}%
\pgfpathlineto{\pgfqpoint{1.480776in}{2.353044in}}%
\pgfpathlineto{\pgfqpoint{1.481379in}{2.429585in}}%
\pgfpathlineto{\pgfqpoint{1.480985in}{2.278689in}}%
\pgfpathlineto{\pgfqpoint{1.481970in}{2.394201in}}%
\pgfpathlineto{\pgfqpoint{1.482856in}{2.297766in}}%
\pgfpathlineto{\pgfqpoint{1.482154in}{2.425075in}}%
\pgfpathlineto{\pgfqpoint{1.483090in}{2.371974in}}%
\pgfpathlineto{\pgfqpoint{1.483742in}{2.425776in}}%
\pgfpathlineto{\pgfqpoint{1.484074in}{2.338379in}}%
\pgfpathlineto{\pgfqpoint{1.484714in}{2.299300in}}%
\pgfpathlineto{\pgfqpoint{1.484247in}{2.428482in}}%
\pgfpathlineto{\pgfqpoint{1.485170in}{2.357207in}}%
\pgfpathlineto{\pgfqpoint{1.486093in}{2.445850in}}%
\pgfpathlineto{\pgfqpoint{1.485957in}{2.309665in}}%
\pgfpathlineto{\pgfqpoint{1.486277in}{2.362879in}}%
\pgfpathlineto{\pgfqpoint{1.486573in}{2.322208in}}%
\pgfpathlineto{\pgfqpoint{1.486856in}{2.439074in}}%
\pgfpathlineto{\pgfqpoint{1.487324in}{2.401569in}}%
\pgfpathlineto{\pgfqpoint{1.488099in}{2.417360in}}%
\pgfpathlineto{\pgfqpoint{1.487816in}{2.325074in}}%
\pgfpathlineto{\pgfqpoint{1.488259in}{2.369287in}}%
\pgfpathlineto{\pgfqpoint{1.489071in}{2.317100in}}%
\pgfpathlineto{\pgfqpoint{1.489219in}{2.434644in}}%
\pgfpathlineto{\pgfqpoint{1.489317in}{2.393687in}}%
\pgfpathlineto{\pgfqpoint{1.489822in}{2.426670in}}%
\pgfpathlineto{\pgfqpoint{1.489687in}{2.338671in}}%
\pgfpathlineto{\pgfqpoint{1.490277in}{2.344325in}}%
\pgfpathlineto{\pgfqpoint{1.490314in}{2.324514in}}%
\pgfpathlineto{\pgfqpoint{1.491065in}{2.420380in}}%
\pgfpathlineto{\pgfqpoint{1.491274in}{2.376781in}}%
\pgfpathlineto{\pgfqpoint{1.491681in}{2.411614in}}%
\pgfpathlineto{\pgfqpoint{1.491521in}{2.333630in}}%
\pgfpathlineto{\pgfqpoint{1.492382in}{2.389161in}}%
\pgfpathlineto{\pgfqpoint{1.492764in}{2.311878in}}%
\pgfpathlineto{\pgfqpoint{1.492444in}{2.414659in}}%
\pgfpathlineto{\pgfqpoint{1.493490in}{2.391244in}}%
\pgfpathlineto{\pgfqpoint{1.494044in}{2.319278in}}%
\pgfpathlineto{\pgfqpoint{1.493687in}{2.418196in}}%
\pgfpathlineto{\pgfqpoint{1.494671in}{2.332091in}}%
\pgfpathlineto{\pgfqpoint{1.494720in}{2.419500in}}%
\pgfpathlineto{\pgfqpoint{1.495102in}{2.331954in}}%
\pgfpathlineto{\pgfqpoint{1.495816in}{2.386858in}}%
\pgfpathlineto{\pgfqpoint{1.495902in}{2.294592in}}%
\pgfpathlineto{\pgfqpoint{1.496284in}{2.420864in}}%
\pgfpathlineto{\pgfqpoint{1.496973in}{2.361823in}}%
\pgfpathlineto{\pgfqpoint{1.497404in}{2.409912in}}%
\pgfpathlineto{\pgfqpoint{1.497736in}{2.322458in}}%
\pgfpathlineto{\pgfqpoint{1.498080in}{2.376660in}}%
\pgfpathlineto{\pgfqpoint{1.499016in}{2.320495in}}%
\pgfpathlineto{\pgfqpoint{1.498659in}{2.431418in}}%
\pgfpathlineto{\pgfqpoint{1.499139in}{2.401897in}}%
\pgfpathlineto{\pgfqpoint{1.499890in}{2.412363in}}%
\pgfpathlineto{\pgfqpoint{1.499631in}{2.317153in}}%
\pgfpathlineto{\pgfqpoint{1.500173in}{2.375553in}}%
\pgfpathlineto{\pgfqpoint{1.500862in}{2.314160in}}%
\pgfpathlineto{\pgfqpoint{1.501010in}{2.442207in}}%
\pgfpathlineto{\pgfqpoint{1.501268in}{2.386169in}}%
\pgfpathlineto{\pgfqpoint{1.501490in}{2.319770in}}%
\pgfpathlineto{\pgfqpoint{1.501785in}{2.431505in}}%
\pgfpathlineto{\pgfqpoint{1.502216in}{2.397299in}}%
\pgfpathlineto{\pgfqpoint{1.502856in}{2.420710in}}%
\pgfpathlineto{\pgfqpoint{1.502696in}{2.329076in}}%
\pgfpathlineto{\pgfqpoint{1.503287in}{2.361745in}}%
\pgfpathlineto{\pgfqpoint{1.503964in}{2.329475in}}%
\pgfpathlineto{\pgfqpoint{1.504136in}{2.434488in}}%
\pgfpathlineto{\pgfqpoint{1.504382in}{2.364690in}}%
\pgfpathlineto{\pgfqpoint{1.504739in}{2.426901in}}%
\pgfpathlineto{\pgfqpoint{1.504542in}{2.329951in}}%
\pgfpathlineto{\pgfqpoint{1.505527in}{2.390221in}}%
\pgfpathlineto{\pgfqpoint{1.506437in}{2.311586in}}%
\pgfpathlineto{\pgfqpoint{1.505982in}{2.428283in}}%
\pgfpathlineto{\pgfqpoint{1.506610in}{2.415057in}}%
\pgfpathlineto{\pgfqpoint{1.507668in}{2.299858in}}%
\pgfpathlineto{\pgfqpoint{1.507767in}{2.379440in}}%
\pgfpathlineto{\pgfqpoint{1.507853in}{2.430297in}}%
\pgfpathlineto{\pgfqpoint{1.508284in}{2.317597in}}%
\pgfpathlineto{\pgfqpoint{1.508862in}{2.387088in}}%
\pgfpathlineto{\pgfqpoint{1.509539in}{2.316877in}}%
\pgfpathlineto{\pgfqpoint{1.509699in}{2.423515in}}%
\pgfpathlineto{\pgfqpoint{1.509982in}{2.350314in}}%
\pgfpathlineto{\pgfqpoint{1.510782in}{2.305536in}}%
\pgfpathlineto{\pgfqpoint{1.510930in}{2.421593in}}%
\pgfpathlineto{\pgfqpoint{1.511053in}{2.386465in}}%
\pgfpathlineto{\pgfqpoint{1.511693in}{2.418822in}}%
\pgfpathlineto{\pgfqpoint{1.511397in}{2.306964in}}%
\pgfpathlineto{\pgfqpoint{1.512185in}{2.415042in}}%
\pgfpathlineto{\pgfqpoint{1.512628in}{2.314139in}}%
\pgfpathlineto{\pgfqpoint{1.512567in}{2.424727in}}%
\pgfpathlineto{\pgfqpoint{1.513354in}{2.393999in}}%
\pgfpathlineto{\pgfqpoint{1.513551in}{2.430814in}}%
\pgfpathlineto{\pgfqpoint{1.513736in}{2.329683in}}%
\pgfpathlineto{\pgfqpoint{1.514240in}{2.356201in}}%
\pgfpathlineto{\pgfqpoint{1.514363in}{2.347818in}}%
\pgfpathlineto{\pgfqpoint{1.514413in}{2.395271in}}%
\pgfpathlineto{\pgfqpoint{1.514671in}{2.428053in}}%
\pgfpathlineto{\pgfqpoint{1.514499in}{2.319071in}}%
\pgfpathlineto{\pgfqpoint{1.515459in}{2.367384in}}%
\pgfpathlineto{\pgfqpoint{1.515742in}{2.318993in}}%
\pgfpathlineto{\pgfqpoint{1.515902in}{2.444798in}}%
\pgfpathlineto{\pgfqpoint{1.516505in}{2.425381in}}%
\pgfpathlineto{\pgfqpoint{1.516985in}{2.325920in}}%
\pgfpathlineto{\pgfqpoint{1.517268in}{2.430284in}}%
\pgfpathlineto{\pgfqpoint{1.517699in}{2.377319in}}%
\pgfpathlineto{\pgfqpoint{1.517773in}{2.431780in}}%
\pgfpathlineto{\pgfqpoint{1.518203in}{2.320102in}}%
\pgfpathlineto{\pgfqpoint{1.518794in}{2.372058in}}%
\pgfpathlineto{\pgfqpoint{1.519447in}{2.332995in}}%
\pgfpathlineto{\pgfqpoint{1.519631in}{2.445501in}}%
\pgfpathlineto{\pgfqpoint{1.519890in}{2.385975in}}%
\pgfpathlineto{\pgfqpoint{1.520702in}{2.319996in}}%
\pgfpathlineto{\pgfqpoint{1.520259in}{2.431704in}}%
\pgfpathlineto{\pgfqpoint{1.520997in}{2.387224in}}%
\pgfpathlineto{\pgfqpoint{1.521502in}{2.421376in}}%
\pgfpathlineto{\pgfqpoint{1.521330in}{2.311101in}}%
\pgfpathlineto{\pgfqpoint{1.521920in}{2.340888in}}%
\pgfpathlineto{\pgfqpoint{1.522560in}{2.310521in}}%
\pgfpathlineto{\pgfqpoint{1.522745in}{2.434592in}}%
\pgfpathlineto{\pgfqpoint{1.523003in}{2.376816in}}%
\pgfpathlineto{\pgfqpoint{1.523176in}{2.324844in}}%
\pgfpathlineto{\pgfqpoint{1.523730in}{2.429680in}}%
\pgfpathlineto{\pgfqpoint{1.524087in}{2.394543in}}%
\pgfpathlineto{\pgfqpoint{1.524603in}{2.431597in}}%
\pgfpathlineto{\pgfqpoint{1.524431in}{2.313279in}}%
\pgfpathlineto{\pgfqpoint{1.525145in}{2.381643in}}%
\pgfpathlineto{\pgfqpoint{1.526290in}{2.309166in}}%
\pgfpathlineto{\pgfqpoint{1.525834in}{2.420181in}}%
\pgfpathlineto{\pgfqpoint{1.526302in}{2.321159in}}%
\pgfpathlineto{\pgfqpoint{1.527459in}{2.424826in}}%
\pgfpathlineto{\pgfqpoint{1.527533in}{2.316490in}}%
\pgfpathlineto{\pgfqpoint{1.528665in}{2.358450in}}%
\pgfpathlineto{\pgfqpoint{1.528948in}{2.429903in}}%
\pgfpathlineto{\pgfqpoint{1.529391in}{2.323084in}}%
\pgfpathlineto{\pgfqpoint{1.529773in}{2.357358in}}%
\pgfpathlineto{\pgfqpoint{1.530794in}{2.436804in}}%
\pgfpathlineto{\pgfqpoint{1.530634in}{2.332604in}}%
\pgfpathlineto{\pgfqpoint{1.530917in}{2.399848in}}%
\pgfpathlineto{\pgfqpoint{1.531410in}{2.419253in}}%
\pgfpathlineto{\pgfqpoint{1.531250in}{2.324329in}}%
\pgfpathlineto{\pgfqpoint{1.531840in}{2.364269in}}%
\pgfpathlineto{\pgfqpoint{1.531865in}{2.326938in}}%
\pgfpathlineto{\pgfqpoint{1.532665in}{2.418710in}}%
\pgfpathlineto{\pgfqpoint{1.532936in}{2.385984in}}%
\pgfpathlineto{\pgfqpoint{1.533108in}{2.327431in}}%
\pgfpathlineto{\pgfqpoint{1.533280in}{2.418474in}}%
\pgfpathlineto{\pgfqpoint{1.533883in}{2.399056in}}%
\pgfpathlineto{\pgfqpoint{1.534523in}{2.437710in}}%
\pgfpathlineto{\pgfqpoint{1.534351in}{2.335613in}}%
\pgfpathlineto{\pgfqpoint{1.534954in}{2.345817in}}%
\pgfpathlineto{\pgfqpoint{1.535594in}{2.324549in}}%
\pgfpathlineto{\pgfqpoint{1.535766in}{2.422052in}}%
\pgfpathlineto{\pgfqpoint{1.536050in}{2.352504in}}%
\pgfpathlineto{\pgfqpoint{1.536062in}{2.352442in}}%
\pgfpathlineto{\pgfqpoint{1.536394in}{2.413740in}}%
\pgfpathlineto{\pgfqpoint{1.536210in}{2.311084in}}%
\pgfpathlineto{\pgfqpoint{1.537206in}{2.370703in}}%
\pgfpathlineto{\pgfqpoint{1.537637in}{2.424611in}}%
\pgfpathlineto{\pgfqpoint{1.537453in}{2.322689in}}%
\pgfpathlineto{\pgfqpoint{1.538043in}{2.340762in}}%
\pgfpathlineto{\pgfqpoint{1.538068in}{2.319162in}}%
\pgfpathlineto{\pgfqpoint{1.538622in}{2.426094in}}%
\pgfpathlineto{\pgfqpoint{1.539114in}{2.373967in}}%
\pgfpathlineto{\pgfqpoint{1.539483in}{2.426941in}}%
\pgfpathlineto{\pgfqpoint{1.539323in}{2.319894in}}%
\pgfpathlineto{\pgfqpoint{1.540246in}{2.400796in}}%
\pgfpathlineto{\pgfqpoint{1.541182in}{2.316167in}}%
\pgfpathlineto{\pgfqpoint{1.540714in}{2.418613in}}%
\pgfpathlineto{\pgfqpoint{1.541330in}{2.410973in}}%
\pgfpathlineto{\pgfqpoint{1.542351in}{2.419236in}}%
\pgfpathlineto{\pgfqpoint{1.541674in}{2.339385in}}%
\pgfpathlineto{\pgfqpoint{1.542376in}{2.393314in}}%
\pgfpathlineto{\pgfqpoint{1.542425in}{2.330257in}}%
\pgfpathlineto{\pgfqpoint{1.543336in}{2.417491in}}%
\pgfpathlineto{\pgfqpoint{1.543508in}{2.349989in}}%
\pgfpathlineto{\pgfqpoint{1.543533in}{2.330426in}}%
\pgfpathlineto{\pgfqpoint{1.543840in}{2.419285in}}%
\pgfpathlineto{\pgfqpoint{1.544579in}{2.394612in}}%
\pgfpathlineto{\pgfqpoint{1.545526in}{2.343016in}}%
\pgfpathlineto{\pgfqpoint{1.545071in}{2.413249in}}%
\pgfpathlineto{\pgfqpoint{1.545662in}{2.400818in}}%
\pgfpathlineto{\pgfqpoint{1.545686in}{2.429821in}}%
\pgfpathlineto{\pgfqpoint{1.546142in}{2.336093in}}%
\pgfpathlineto{\pgfqpoint{1.546733in}{2.366372in}}%
\pgfpathlineto{\pgfqpoint{1.546770in}{2.339325in}}%
\pgfpathlineto{\pgfqpoint{1.547557in}{2.418654in}}%
\pgfpathlineto{\pgfqpoint{1.547828in}{2.368578in}}%
\pgfpathlineto{\pgfqpoint{1.548813in}{2.422741in}}%
\pgfpathlineto{\pgfqpoint{1.547988in}{2.333317in}}%
\pgfpathlineto{\pgfqpoint{1.548973in}{2.373331in}}%
\pgfpathlineto{\pgfqpoint{1.549231in}{2.345573in}}%
\pgfpathlineto{\pgfqpoint{1.549416in}{2.420001in}}%
\pgfpathlineto{\pgfqpoint{1.550019in}{2.403543in}}%
\pgfpathlineto{\pgfqpoint{1.550659in}{2.415673in}}%
\pgfpathlineto{\pgfqpoint{1.550326in}{2.334772in}}%
\pgfpathlineto{\pgfqpoint{1.551065in}{2.359592in}}%
\pgfpathlineto{\pgfqpoint{1.551102in}{2.317654in}}%
\pgfpathlineto{\pgfqpoint{1.551286in}{2.415273in}}%
\pgfpathlineto{\pgfqpoint{1.552173in}{2.345553in}}%
\pgfpathlineto{\pgfqpoint{1.552530in}{2.411608in}}%
\pgfpathlineto{\pgfqpoint{1.552960in}{2.326754in}}%
\pgfpathlineto{\pgfqpoint{1.553354in}{2.377647in}}%
\pgfpathlineto{\pgfqpoint{1.553453in}{2.329443in}}%
\pgfpathlineto{\pgfqpoint{1.553514in}{2.422629in}}%
\pgfpathlineto{\pgfqpoint{1.554117in}{2.395395in}}%
\pgfpathlineto{\pgfqpoint{1.554376in}{2.422044in}}%
\pgfpathlineto{\pgfqpoint{1.554216in}{2.322960in}}%
\pgfpathlineto{\pgfqpoint{1.555188in}{2.363055in}}%
\pgfpathlineto{\pgfqpoint{1.555606in}{2.413174in}}%
\pgfpathlineto{\pgfqpoint{1.556074in}{2.322083in}}%
\pgfpathlineto{\pgfqpoint{1.556296in}{2.366515in}}%
\pgfpathlineto{\pgfqpoint{1.556566in}{2.339409in}}%
\pgfpathlineto{\pgfqpoint{1.557243in}{2.414539in}}%
\pgfpathlineto{\pgfqpoint{1.557379in}{2.366399in}}%
\pgfpathlineto{\pgfqpoint{1.558228in}{2.424516in}}%
\pgfpathlineto{\pgfqpoint{1.558413in}{2.338303in}}%
\pgfpathlineto{\pgfqpoint{1.558499in}{2.393741in}}%
\pgfpathlineto{\pgfqpoint{1.559188in}{2.338399in}}%
\pgfpathlineto{\pgfqpoint{1.558733in}{2.421950in}}%
\pgfpathlineto{\pgfqpoint{1.559619in}{2.374233in}}%
\pgfpathlineto{\pgfqpoint{1.560259in}{2.342987in}}%
\pgfpathlineto{\pgfqpoint{1.560579in}{2.427280in}}%
\pgfpathlineto{\pgfqpoint{1.560689in}{2.399583in}}%
\pgfpathlineto{\pgfqpoint{1.561342in}{2.413937in}}%
\pgfpathlineto{\pgfqpoint{1.561034in}{2.334010in}}%
\pgfpathlineto{\pgfqpoint{1.561748in}{2.369970in}}%
\pgfpathlineto{\pgfqpoint{1.562080in}{2.339632in}}%
\pgfpathlineto{\pgfqpoint{1.561945in}{2.421813in}}%
\pgfpathlineto{\pgfqpoint{1.562806in}{2.383461in}}%
\pgfpathlineto{\pgfqpoint{1.562831in}{2.386282in}}%
\pgfpathlineto{\pgfqpoint{1.562843in}{2.369802in}}%
\pgfpathlineto{\pgfqpoint{1.562880in}{2.331359in}}%
\pgfpathlineto{\pgfqpoint{1.563705in}{2.423205in}}%
\pgfpathlineto{\pgfqpoint{1.563951in}{2.357630in}}%
\pgfpathlineto{\pgfqpoint{1.564308in}{2.421515in}}%
\pgfpathlineto{\pgfqpoint{1.564160in}{2.344301in}}%
\pgfpathlineto{\pgfqpoint{1.565096in}{2.383116in}}%
\pgfpathlineto{\pgfqpoint{1.565994in}{2.322699in}}%
\pgfpathlineto{\pgfqpoint{1.566179in}{2.413284in}}%
\pgfpathlineto{\pgfqpoint{1.566191in}{2.413538in}}%
\pgfpathlineto{\pgfqpoint{1.566203in}{2.405062in}}%
\pgfpathlineto{\pgfqpoint{1.567237in}{2.338379in}}%
\pgfpathlineto{\pgfqpoint{1.566917in}{2.421503in}}%
\pgfpathlineto{\pgfqpoint{1.567323in}{2.380255in}}%
\pgfpathlineto{\pgfqpoint{1.567422in}{2.417927in}}%
\pgfpathlineto{\pgfqpoint{1.568333in}{2.330293in}}%
\pgfpathlineto{\pgfqpoint{1.568431in}{2.399715in}}%
\pgfpathlineto{\pgfqpoint{1.569120in}{2.327675in}}%
\pgfpathlineto{\pgfqpoint{1.569268in}{2.414308in}}%
\pgfpathlineto{\pgfqpoint{1.569576in}{2.338239in}}%
\pgfpathlineto{\pgfqpoint{1.570019in}{2.417543in}}%
\pgfpathlineto{\pgfqpoint{1.570683in}{2.350447in}}%
\pgfpathlineto{\pgfqpoint{1.570979in}{2.337474in}}%
\pgfpathlineto{\pgfqpoint{1.571262in}{2.420827in}}%
\pgfpathlineto{\pgfqpoint{1.571742in}{2.385399in}}%
\pgfpathlineto{\pgfqpoint{1.572382in}{2.415283in}}%
\pgfpathlineto{\pgfqpoint{1.572246in}{2.331054in}}%
\pgfpathlineto{\pgfqpoint{1.572640in}{2.369029in}}%
\pgfpathlineto{\pgfqpoint{1.573477in}{2.338704in}}%
\pgfpathlineto{\pgfqpoint{1.573120in}{2.413967in}}%
\pgfpathlineto{\pgfqpoint{1.573723in}{2.391863in}}%
\pgfpathlineto{\pgfqpoint{1.574376in}{2.407710in}}%
\pgfpathlineto{\pgfqpoint{1.574105in}{2.339516in}}%
\pgfpathlineto{\pgfqpoint{1.574806in}{2.381693in}}%
\pgfpathlineto{\pgfqpoint{1.575606in}{2.411808in}}%
\pgfpathlineto{\pgfqpoint{1.575360in}{2.341456in}}%
\pgfpathlineto{\pgfqpoint{1.575779in}{2.360342in}}%
\pgfpathlineto{\pgfqpoint{1.576591in}{2.347944in}}%
\pgfpathlineto{\pgfqpoint{1.576234in}{2.412007in}}%
\pgfpathlineto{\pgfqpoint{1.576812in}{2.397768in}}%
\pgfpathlineto{\pgfqpoint{1.576837in}{2.409430in}}%
\pgfpathlineto{\pgfqpoint{1.577219in}{2.350333in}}%
\pgfpathlineto{\pgfqpoint{1.577896in}{2.386691in}}%
\pgfpathlineto{\pgfqpoint{1.578265in}{2.347367in}}%
\pgfpathlineto{\pgfqpoint{1.577957in}{2.402942in}}%
\pgfpathlineto{\pgfqpoint{1.579052in}{2.349002in}}%
\pgfpathlineto{\pgfqpoint{1.579951in}{2.409312in}}%
\pgfpathlineto{\pgfqpoint{1.580099in}{2.338647in}}%
\pgfpathlineto{\pgfqpoint{1.580246in}{2.372455in}}%
\pgfpathlineto{\pgfqpoint{1.580874in}{2.342713in}}%
\pgfpathlineto{\pgfqpoint{1.580566in}{2.401714in}}%
\pgfpathlineto{\pgfqpoint{1.581379in}{2.351783in}}%
\pgfpathlineto{\pgfqpoint{1.582302in}{2.405958in}}%
\pgfpathlineto{\pgfqpoint{1.582166in}{2.341095in}}%
\pgfpathlineto{\pgfqpoint{1.582560in}{2.390880in}}%
\pgfpathlineto{\pgfqpoint{1.583212in}{2.328997in}}%
\pgfpathlineto{\pgfqpoint{1.583065in}{2.404309in}}%
\pgfpathlineto{\pgfqpoint{1.583668in}{2.383905in}}%
\pgfpathlineto{\pgfqpoint{1.584172in}{2.399531in}}%
\pgfpathlineto{\pgfqpoint{1.584000in}{2.333897in}}%
\pgfpathlineto{\pgfqpoint{1.584739in}{2.367406in}}%
\pgfpathlineto{\pgfqpoint{1.585280in}{2.350470in}}%
\pgfpathlineto{\pgfqpoint{1.584911in}{2.410035in}}%
\pgfpathlineto{\pgfqpoint{1.585859in}{2.358010in}}%
\pgfpathlineto{\pgfqpoint{1.585871in}{2.358080in}}%
\pgfpathlineto{\pgfqpoint{1.585883in}{2.353963in}}%
\pgfpathlineto{\pgfqpoint{1.586339in}{2.331249in}}%
\pgfpathlineto{\pgfqpoint{1.586154in}{2.403677in}}%
\pgfpathlineto{\pgfqpoint{1.586966in}{2.365120in}}%
\pgfpathlineto{\pgfqpoint{1.588049in}{2.411322in}}%
\pgfpathlineto{\pgfqpoint{1.587139in}{2.326181in}}%
\pgfpathlineto{\pgfqpoint{1.588086in}{2.380535in}}%
\pgfpathlineto{\pgfqpoint{1.588357in}{2.346958in}}%
\pgfpathlineto{\pgfqpoint{1.588542in}{2.404478in}}%
\pgfpathlineto{\pgfqpoint{1.589169in}{2.394830in}}%
\pgfpathlineto{\pgfqpoint{1.590252in}{2.323232in}}%
\pgfpathlineto{\pgfqpoint{1.589268in}{2.412773in}}%
\pgfpathlineto{\pgfqpoint{1.590302in}{2.374675in}}%
\pgfpathlineto{\pgfqpoint{1.590400in}{2.409103in}}%
\pgfpathlineto{\pgfqpoint{1.590696in}{2.349414in}}%
\pgfpathlineto{\pgfqpoint{1.591422in}{2.382498in}}%
\pgfpathlineto{\pgfqpoint{1.591483in}{2.335844in}}%
\pgfpathlineto{\pgfqpoint{1.591619in}{2.410455in}}%
\pgfpathlineto{\pgfqpoint{1.592554in}{2.352687in}}%
\pgfpathlineto{\pgfqpoint{1.593514in}{2.407677in}}%
\pgfpathlineto{\pgfqpoint{1.593379in}{2.327355in}}%
\pgfpathlineto{\pgfqpoint{1.593723in}{2.380420in}}%
\pgfpathlineto{\pgfqpoint{1.594597in}{2.338414in}}%
\pgfpathlineto{\pgfqpoint{1.594757in}{2.409876in}}%
\pgfpathlineto{\pgfqpoint{1.594794in}{2.393777in}}%
\pgfpathlineto{\pgfqpoint{1.595508in}{2.401388in}}%
\pgfpathlineto{\pgfqpoint{1.595225in}{2.342435in}}%
\pgfpathlineto{\pgfqpoint{1.595717in}{2.360578in}}%
\pgfpathlineto{\pgfqpoint{1.596492in}{2.333472in}}%
\pgfpathlineto{\pgfqpoint{1.596111in}{2.400328in}}%
\pgfpathlineto{\pgfqpoint{1.596714in}{2.392592in}}%
\pgfpathlineto{\pgfqpoint{1.597366in}{2.401648in}}%
\pgfpathlineto{\pgfqpoint{1.597711in}{2.340502in}}%
\pgfpathlineto{\pgfqpoint{1.597723in}{2.342070in}}%
\pgfpathlineto{\pgfqpoint{1.598326in}{2.336723in}}%
\pgfpathlineto{\pgfqpoint{1.597969in}{2.405202in}}%
\pgfpathlineto{\pgfqpoint{1.598572in}{2.387902in}}%
\pgfpathlineto{\pgfqpoint{1.599225in}{2.402363in}}%
\pgfpathlineto{\pgfqpoint{1.599594in}{2.339531in}}%
\pgfpathlineto{\pgfqpoint{1.599643in}{2.358298in}}%
\pgfpathlineto{\pgfqpoint{1.599668in}{2.361354in}}%
\pgfpathlineto{\pgfqpoint{1.600320in}{2.403846in}}%
\pgfpathlineto{\pgfqpoint{1.600185in}{2.336419in}}%
\pgfpathlineto{\pgfqpoint{1.600763in}{2.364543in}}%
\pgfpathlineto{\pgfqpoint{1.601231in}{2.332604in}}%
\pgfpathlineto{\pgfqpoint{1.601083in}{2.410978in}}%
\pgfpathlineto{\pgfqpoint{1.601859in}{2.372360in}}%
\pgfpathlineto{\pgfqpoint{1.602942in}{2.409989in}}%
\pgfpathlineto{\pgfqpoint{1.602019in}{2.333935in}}%
\pgfpathlineto{\pgfqpoint{1.602979in}{2.388945in}}%
\pgfpathlineto{\pgfqpoint{1.603311in}{2.332851in}}%
\pgfpathlineto{\pgfqpoint{1.603434in}{2.405706in}}%
\pgfpathlineto{\pgfqpoint{1.604086in}{2.385301in}}%
\pgfpathlineto{\pgfqpoint{1.605132in}{2.330195in}}%
\pgfpathlineto{\pgfqpoint{1.604197in}{2.415156in}}%
\pgfpathlineto{\pgfqpoint{1.605219in}{2.366152in}}%
\pgfpathlineto{\pgfqpoint{1.606055in}{2.416221in}}%
\pgfpathlineto{\pgfqpoint{1.605822in}{2.346612in}}%
\pgfpathlineto{\pgfqpoint{1.606326in}{2.379912in}}%
\pgfpathlineto{\pgfqpoint{1.606437in}{2.331487in}}%
\pgfpathlineto{\pgfqpoint{1.606572in}{2.409235in}}%
\pgfpathlineto{\pgfqpoint{1.607483in}{2.344196in}}%
\pgfpathlineto{\pgfqpoint{1.608406in}{2.402985in}}%
\pgfpathlineto{\pgfqpoint{1.608271in}{2.313496in}}%
\pgfpathlineto{\pgfqpoint{1.608628in}{2.389745in}}%
\pgfpathlineto{\pgfqpoint{1.609514in}{2.336486in}}%
\pgfpathlineto{\pgfqpoint{1.609182in}{2.413156in}}%
\pgfpathlineto{\pgfqpoint{1.609748in}{2.378666in}}%
\pgfpathlineto{\pgfqpoint{1.610400in}{2.413649in}}%
\pgfpathlineto{\pgfqpoint{1.610142in}{2.343327in}}%
\pgfpathlineto{\pgfqpoint{1.610745in}{2.361333in}}%
\pgfpathlineto{\pgfqpoint{1.611385in}{2.304952in}}%
\pgfpathlineto{\pgfqpoint{1.611545in}{2.403427in}}%
\pgfpathlineto{\pgfqpoint{1.611840in}{2.361800in}}%
\pgfpathlineto{\pgfqpoint{1.612295in}{2.403954in}}%
\pgfpathlineto{\pgfqpoint{1.612628in}{2.327689in}}%
\pgfpathlineto{\pgfqpoint{1.612935in}{2.360758in}}%
\pgfpathlineto{\pgfqpoint{1.613243in}{2.330889in}}%
\pgfpathlineto{\pgfqpoint{1.613526in}{2.411307in}}%
\pgfpathlineto{\pgfqpoint{1.613982in}{2.375923in}}%
\pgfpathlineto{\pgfqpoint{1.614659in}{2.406349in}}%
\pgfpathlineto{\pgfqpoint{1.614511in}{2.310391in}}%
\pgfpathlineto{\pgfqpoint{1.615040in}{2.362014in}}%
\pgfpathlineto{\pgfqpoint{1.615742in}{2.322576in}}%
\pgfpathlineto{\pgfqpoint{1.615889in}{2.406584in}}%
\pgfpathlineto{\pgfqpoint{1.616148in}{2.361947in}}%
\pgfpathlineto{\pgfqpoint{1.617243in}{2.408887in}}%
\pgfpathlineto{\pgfqpoint{1.616357in}{2.322468in}}%
\pgfpathlineto{\pgfqpoint{1.617305in}{2.371372in}}%
\pgfpathlineto{\pgfqpoint{1.617625in}{2.316271in}}%
\pgfpathlineto{\pgfqpoint{1.617772in}{2.403647in}}%
\pgfpathlineto{\pgfqpoint{1.618277in}{2.385797in}}%
\pgfpathlineto{\pgfqpoint{1.619102in}{2.413274in}}%
\pgfpathlineto{\pgfqpoint{1.618855in}{2.328099in}}%
\pgfpathlineto{\pgfqpoint{1.619372in}{2.386751in}}%
\pgfpathlineto{\pgfqpoint{1.619471in}{2.319008in}}%
\pgfpathlineto{\pgfqpoint{1.620357in}{2.409633in}}%
\pgfpathlineto{\pgfqpoint{1.620529in}{2.360548in}}%
\pgfpathlineto{\pgfqpoint{1.620972in}{2.409192in}}%
\pgfpathlineto{\pgfqpoint{1.620738in}{2.331682in}}%
\pgfpathlineto{\pgfqpoint{1.621686in}{2.394084in}}%
\pgfpathlineto{\pgfqpoint{1.621698in}{2.394763in}}%
\pgfpathlineto{\pgfqpoint{1.621945in}{2.350980in}}%
\pgfpathlineto{\pgfqpoint{1.622597in}{2.322862in}}%
\pgfpathlineto{\pgfqpoint{1.622215in}{2.416036in}}%
\pgfpathlineto{\pgfqpoint{1.623015in}{2.383207in}}%
\pgfpathlineto{\pgfqpoint{1.623151in}{2.331304in}}%
\pgfpathlineto{\pgfqpoint{1.623471in}{2.409693in}}%
\pgfpathlineto{\pgfqpoint{1.624062in}{2.395715in}}%
\pgfpathlineto{\pgfqpoint{1.624086in}{2.408249in}}%
\pgfpathlineto{\pgfqpoint{1.624443in}{2.322084in}}%
\pgfpathlineto{\pgfqpoint{1.625071in}{2.345322in}}%
\pgfpathlineto{\pgfqpoint{1.625711in}{2.328906in}}%
\pgfpathlineto{\pgfqpoint{1.625342in}{2.416388in}}%
\pgfpathlineto{\pgfqpoint{1.625797in}{2.388068in}}%
\pgfpathlineto{\pgfqpoint{1.626585in}{2.408566in}}%
\pgfpathlineto{\pgfqpoint{1.626265in}{2.324282in}}%
\pgfpathlineto{\pgfqpoint{1.626868in}{2.376226in}}%
\pgfpathlineto{\pgfqpoint{1.627569in}{2.315525in}}%
\pgfpathlineto{\pgfqpoint{1.627200in}{2.408391in}}%
\pgfpathlineto{\pgfqpoint{1.627975in}{2.370205in}}%
\pgfpathlineto{\pgfqpoint{1.628455in}{2.414985in}}%
\pgfpathlineto{\pgfqpoint{1.628603in}{2.333835in}}%
\pgfpathlineto{\pgfqpoint{1.629108in}{2.382115in}}%
\pgfpathlineto{\pgfqpoint{1.629428in}{2.324313in}}%
\pgfpathlineto{\pgfqpoint{1.629711in}{2.403539in}}%
\pgfpathlineto{\pgfqpoint{1.630215in}{2.381389in}}%
\pgfpathlineto{\pgfqpoint{1.630314in}{2.410703in}}%
\pgfpathlineto{\pgfqpoint{1.630683in}{2.318080in}}%
\pgfpathlineto{\pgfqpoint{1.631188in}{2.374944in}}%
\pgfpathlineto{\pgfqpoint{1.631926in}{2.335860in}}%
\pgfpathlineto{\pgfqpoint{1.631582in}{2.410026in}}%
\pgfpathlineto{\pgfqpoint{1.632295in}{2.378035in}}%
\pgfpathlineto{\pgfqpoint{1.632382in}{2.403999in}}%
\pgfpathlineto{\pgfqpoint{1.632517in}{2.322132in}}%
\pgfpathlineto{\pgfqpoint{1.632529in}{2.312696in}}%
\pgfpathlineto{\pgfqpoint{1.633428in}{2.410031in}}%
\pgfpathlineto{\pgfqpoint{1.633575in}{2.349967in}}%
\pgfpathlineto{\pgfqpoint{1.633797in}{2.327496in}}%
\pgfpathlineto{\pgfqpoint{1.634732in}{2.406708in}}%
\pgfpathlineto{\pgfqpoint{1.635655in}{2.302887in}}%
\pgfpathlineto{\pgfqpoint{1.635951in}{2.386911in}}%
\pgfpathlineto{\pgfqpoint{1.636554in}{2.406406in}}%
\pgfpathlineto{\pgfqpoint{1.636874in}{2.339853in}}%
\pgfpathlineto{\pgfqpoint{1.637514in}{2.324193in}}%
\pgfpathlineto{\pgfqpoint{1.637846in}{2.403860in}}%
\pgfpathlineto{\pgfqpoint{1.637945in}{2.367143in}}%
\pgfpathlineto{\pgfqpoint{1.638769in}{2.301628in}}%
\pgfpathlineto{\pgfqpoint{1.638400in}{2.400523in}}%
\pgfpathlineto{\pgfqpoint{1.638954in}{2.387480in}}%
\pgfpathlineto{\pgfqpoint{1.639446in}{2.406547in}}%
\pgfpathlineto{\pgfqpoint{1.639385in}{2.337082in}}%
\pgfpathlineto{\pgfqpoint{1.639938in}{2.382982in}}%
\pgfpathlineto{\pgfqpoint{1.640628in}{2.319350in}}%
\pgfpathlineto{\pgfqpoint{1.640911in}{2.404838in}}%
\pgfpathlineto{\pgfqpoint{1.641046in}{2.373601in}}%
\pgfpathlineto{\pgfqpoint{1.641514in}{2.407655in}}%
\pgfpathlineto{\pgfqpoint{1.641883in}{2.309714in}}%
\pgfpathlineto{\pgfqpoint{1.642166in}{2.385364in}}%
\pgfpathlineto{\pgfqpoint{1.643114in}{2.331298in}}%
\pgfpathlineto{\pgfqpoint{1.642560in}{2.404313in}}%
\pgfpathlineto{\pgfqpoint{1.643261in}{2.389139in}}%
\pgfpathlineto{\pgfqpoint{1.643372in}{2.407809in}}%
\pgfpathlineto{\pgfqpoint{1.643705in}{2.351400in}}%
\pgfpathlineto{\pgfqpoint{1.643741in}{2.319718in}}%
\pgfpathlineto{\pgfqpoint{1.644628in}{2.410624in}}%
\pgfpathlineto{\pgfqpoint{1.644788in}{2.358005in}}%
\pgfpathlineto{\pgfqpoint{1.645674in}{2.405781in}}%
\pgfpathlineto{\pgfqpoint{1.644997in}{2.320368in}}%
\pgfpathlineto{\pgfqpoint{1.645920in}{2.386859in}}%
\pgfpathlineto{\pgfqpoint{1.646498in}{2.412309in}}%
\pgfpathlineto{\pgfqpoint{1.646855in}{2.319255in}}%
\pgfpathlineto{\pgfqpoint{1.647028in}{2.388438in}}%
\pgfpathlineto{\pgfqpoint{1.647471in}{2.321391in}}%
\pgfpathlineto{\pgfqpoint{1.647754in}{2.412217in}}%
\pgfpathlineto{\pgfqpoint{1.648246in}{2.371359in}}%
\pgfpathlineto{\pgfqpoint{1.648357in}{2.405490in}}%
\pgfpathlineto{\pgfqpoint{1.648714in}{2.328405in}}%
\pgfpathlineto{\pgfqpoint{1.649317in}{2.358556in}}%
\pgfpathlineto{\pgfqpoint{1.649969in}{2.323935in}}%
\pgfpathlineto{\pgfqpoint{1.649625in}{2.413777in}}%
\pgfpathlineto{\pgfqpoint{1.650388in}{2.383848in}}%
\pgfpathlineto{\pgfqpoint{1.650880in}{2.410118in}}%
\pgfpathlineto{\pgfqpoint{1.650585in}{2.315457in}}%
\pgfpathlineto{\pgfqpoint{1.651508in}{2.398007in}}%
\pgfpathlineto{\pgfqpoint{1.651828in}{2.321383in}}%
\pgfpathlineto{\pgfqpoint{1.652628in}{2.392459in}}%
\pgfpathlineto{\pgfqpoint{1.652738in}{2.416338in}}%
\pgfpathlineto{\pgfqpoint{1.653071in}{2.329363in}}%
\pgfpathlineto{\pgfqpoint{1.653563in}{2.383171in}}%
\pgfpathlineto{\pgfqpoint{1.653698in}{2.313924in}}%
\pgfpathlineto{\pgfqpoint{1.654585in}{2.415566in}}%
\pgfpathlineto{\pgfqpoint{1.654695in}{2.363164in}}%
\pgfpathlineto{\pgfqpoint{1.655852in}{2.413671in}}%
\pgfpathlineto{\pgfqpoint{1.654941in}{2.316159in}}%
\pgfpathlineto{\pgfqpoint{1.655901in}{2.389185in}}%
\pgfpathlineto{\pgfqpoint{1.656812in}{2.310944in}}%
\pgfpathlineto{\pgfqpoint{1.656443in}{2.412246in}}%
\pgfpathlineto{\pgfqpoint{1.657034in}{2.368922in}}%
\pgfpathlineto{\pgfqpoint{1.657711in}{2.415234in}}%
\pgfpathlineto{\pgfqpoint{1.658043in}{2.324364in}}%
\pgfpathlineto{\pgfqpoint{1.658055in}{2.318700in}}%
\pgfpathlineto{\pgfqpoint{1.658941in}{2.415513in}}%
\pgfpathlineto{\pgfqpoint{1.658954in}{2.414787in}}%
\pgfpathlineto{\pgfqpoint{1.658966in}{2.415480in}}%
\pgfpathlineto{\pgfqpoint{1.659064in}{2.362689in}}%
\pgfpathlineto{\pgfqpoint{1.659926in}{2.315833in}}%
\pgfpathlineto{\pgfqpoint{1.659569in}{2.414945in}}%
\pgfpathlineto{\pgfqpoint{1.660148in}{2.370775in}}%
\pgfpathlineto{\pgfqpoint{1.660837in}{2.413896in}}%
\pgfpathlineto{\pgfqpoint{1.661157in}{2.331462in}}%
\pgfpathlineto{\pgfqpoint{1.661169in}{2.318770in}}%
\pgfpathlineto{\pgfqpoint{1.662080in}{2.416825in}}%
\pgfpathlineto{\pgfqpoint{1.662228in}{2.361267in}}%
\pgfpathlineto{\pgfqpoint{1.662683in}{2.418517in}}%
\pgfpathlineto{\pgfqpoint{1.663040in}{2.311270in}}%
\pgfpathlineto{\pgfqpoint{1.663372in}{2.386503in}}%
\pgfpathlineto{\pgfqpoint{1.664295in}{2.327423in}}%
\pgfpathlineto{\pgfqpoint{1.663951in}{2.413250in}}%
\pgfpathlineto{\pgfqpoint{1.664418in}{2.388862in}}%
\pgfpathlineto{\pgfqpoint{1.665194in}{2.413195in}}%
\pgfpathlineto{\pgfqpoint{1.664911in}{2.325733in}}%
\pgfpathlineto{\pgfqpoint{1.665464in}{2.348464in}}%
\pgfpathlineto{\pgfqpoint{1.666154in}{2.308047in}}%
\pgfpathlineto{\pgfqpoint{1.665821in}{2.419399in}}%
\pgfpathlineto{\pgfqpoint{1.666314in}{2.402623in}}%
\pgfpathlineto{\pgfqpoint{1.666326in}{2.406701in}}%
\pgfpathlineto{\pgfqpoint{1.666769in}{2.336764in}}%
\pgfpathlineto{\pgfqpoint{1.667311in}{2.366993in}}%
\pgfpathlineto{\pgfqpoint{1.668012in}{2.325705in}}%
\pgfpathlineto{\pgfqpoint{1.667569in}{2.407504in}}%
\pgfpathlineto{\pgfqpoint{1.668271in}{2.400782in}}%
\pgfpathlineto{\pgfqpoint{1.668935in}{2.417392in}}%
\pgfpathlineto{\pgfqpoint{1.668640in}{2.329108in}}%
\pgfpathlineto{\pgfqpoint{1.669243in}{2.332253in}}%
\pgfpathlineto{\pgfqpoint{1.669268in}{2.307847in}}%
\pgfpathlineto{\pgfqpoint{1.669440in}{2.407924in}}%
\pgfpathlineto{\pgfqpoint{1.670314in}{2.369576in}}%
\pgfpathlineto{\pgfqpoint{1.671397in}{2.409038in}}%
\pgfpathlineto{\pgfqpoint{1.671126in}{2.323916in}}%
\pgfpathlineto{\pgfqpoint{1.671458in}{2.397267in}}%
\pgfpathlineto{\pgfqpoint{1.672381in}{2.313733in}}%
\pgfpathlineto{\pgfqpoint{1.672049in}{2.409835in}}%
\pgfpathlineto{\pgfqpoint{1.672541in}{2.398257in}}%
\pgfpathlineto{\pgfqpoint{1.672554in}{2.403724in}}%
\pgfpathlineto{\pgfqpoint{1.672997in}{2.329674in}}%
\pgfpathlineto{\pgfqpoint{1.673551in}{2.362013in}}%
\pgfpathlineto{\pgfqpoint{1.674240in}{2.320819in}}%
\pgfpathlineto{\pgfqpoint{1.674412in}{2.409845in}}%
\pgfpathlineto{\pgfqpoint{1.674646in}{2.378164in}}%
\pgfpathlineto{\pgfqpoint{1.675495in}{2.320697in}}%
\pgfpathlineto{\pgfqpoint{1.675151in}{2.405115in}}%
\pgfpathlineto{\pgfqpoint{1.675631in}{2.381260in}}%
\pgfpathlineto{\pgfqpoint{1.675741in}{2.406176in}}%
\pgfpathlineto{\pgfqpoint{1.676111in}{2.329959in}}%
\pgfpathlineto{\pgfqpoint{1.676677in}{2.346631in}}%
\pgfpathlineto{\pgfqpoint{1.677354in}{2.317498in}}%
\pgfpathlineto{\pgfqpoint{1.676997in}{2.411706in}}%
\pgfpathlineto{\pgfqpoint{1.677514in}{2.397450in}}%
\pgfpathlineto{\pgfqpoint{1.677637in}{2.408323in}}%
\pgfpathlineto{\pgfqpoint{1.677969in}{2.320021in}}%
\pgfpathlineto{\pgfqpoint{1.678547in}{2.354637in}}%
\pgfpathlineto{\pgfqpoint{1.678560in}{2.354602in}}%
\pgfpathlineto{\pgfqpoint{1.678597in}{2.329236in}}%
\pgfpathlineto{\pgfqpoint{1.678855in}{2.411872in}}%
\pgfpathlineto{\pgfqpoint{1.679594in}{2.377233in}}%
\pgfpathlineto{\pgfqpoint{1.680123in}{2.410398in}}%
\pgfpathlineto{\pgfqpoint{1.680467in}{2.326079in}}%
\pgfpathlineto{\pgfqpoint{1.680763in}{2.396246in}}%
\pgfpathlineto{\pgfqpoint{1.681083in}{2.319483in}}%
\pgfpathlineto{\pgfqpoint{1.681341in}{2.403711in}}%
\pgfpathlineto{\pgfqpoint{1.681932in}{2.384671in}}%
\pgfpathlineto{\pgfqpoint{1.681994in}{2.408228in}}%
\pgfpathlineto{\pgfqpoint{1.682941in}{2.328695in}}%
\pgfpathlineto{\pgfqpoint{1.683027in}{2.380654in}}%
\pgfpathlineto{\pgfqpoint{1.683224in}{2.411094in}}%
\pgfpathlineto{\pgfqpoint{1.683557in}{2.342252in}}%
\pgfpathlineto{\pgfqpoint{1.684074in}{2.371618in}}%
\pgfpathlineto{\pgfqpoint{1.684172in}{2.323895in}}%
\pgfpathlineto{\pgfqpoint{1.684467in}{2.405965in}}%
\pgfpathlineto{\pgfqpoint{1.685021in}{2.381537in}}%
\pgfpathlineto{\pgfqpoint{1.685095in}{2.407963in}}%
\pgfpathlineto{\pgfqpoint{1.685403in}{2.340915in}}%
\pgfpathlineto{\pgfqpoint{1.686031in}{2.334108in}}%
\pgfpathlineto{\pgfqpoint{1.686326in}{2.411572in}}%
\pgfpathlineto{\pgfqpoint{1.686449in}{2.367209in}}%
\pgfpathlineto{\pgfqpoint{1.686646in}{2.336998in}}%
\pgfpathlineto{\pgfqpoint{1.686929in}{2.400819in}}%
\pgfpathlineto{\pgfqpoint{1.686954in}{2.409342in}}%
\pgfpathlineto{\pgfqpoint{1.687274in}{2.316639in}}%
\pgfpathlineto{\pgfqpoint{1.687987in}{2.371684in}}%
\pgfpathlineto{\pgfqpoint{1.688504in}{2.335309in}}%
\pgfpathlineto{\pgfqpoint{1.688209in}{2.400688in}}%
\pgfpathlineto{\pgfqpoint{1.689046in}{2.384516in}}%
\pgfpathlineto{\pgfqpoint{1.690043in}{2.407108in}}%
\pgfpathlineto{\pgfqpoint{1.689747in}{2.324280in}}%
\pgfpathlineto{\pgfqpoint{1.690104in}{2.374188in}}%
\pgfpathlineto{\pgfqpoint{1.690363in}{2.317699in}}%
\pgfpathlineto{\pgfqpoint{1.690683in}{2.400616in}}%
\pgfpathlineto{\pgfqpoint{1.691224in}{2.368401in}}%
\pgfpathlineto{\pgfqpoint{1.692147in}{2.401286in}}%
\pgfpathlineto{\pgfqpoint{1.692221in}{2.320661in}}%
\pgfpathlineto{\pgfqpoint{1.692344in}{2.375054in}}%
\pgfpathlineto{\pgfqpoint{1.692541in}{2.404773in}}%
\pgfpathlineto{\pgfqpoint{1.692677in}{2.364947in}}%
\pgfpathlineto{\pgfqpoint{1.692837in}{2.326625in}}%
\pgfpathlineto{\pgfqpoint{1.693157in}{2.406009in}}%
\pgfpathlineto{\pgfqpoint{1.693747in}{2.397074in}}%
\pgfpathlineto{\pgfqpoint{1.694387in}{2.403239in}}%
\pgfpathlineto{\pgfqpoint{1.694080in}{2.333794in}}%
\pgfpathlineto{\pgfqpoint{1.694523in}{2.380181in}}%
\pgfpathlineto{\pgfqpoint{1.694695in}{2.331331in}}%
\pgfpathlineto{\pgfqpoint{1.694991in}{2.403084in}}%
\pgfpathlineto{\pgfqpoint{1.695606in}{2.401515in}}%
\pgfpathlineto{\pgfqpoint{1.695631in}{2.409427in}}%
\pgfpathlineto{\pgfqpoint{1.695938in}{2.329021in}}%
\pgfpathlineto{\pgfqpoint{1.696541in}{2.347153in}}%
\pgfpathlineto{\pgfqpoint{1.696554in}{2.335582in}}%
\pgfpathlineto{\pgfqpoint{1.696861in}{2.409344in}}%
\pgfpathlineto{\pgfqpoint{1.697600in}{2.390213in}}%
\pgfpathlineto{\pgfqpoint{1.697784in}{2.337729in}}%
\pgfpathlineto{\pgfqpoint{1.698104in}{2.410811in}}%
\pgfpathlineto{\pgfqpoint{1.698695in}{2.391440in}}%
\pgfpathlineto{\pgfqpoint{1.698720in}{2.409606in}}%
\pgfpathlineto{\pgfqpoint{1.699027in}{2.328845in}}%
\pgfpathlineto{\pgfqpoint{1.699778in}{2.372824in}}%
\pgfpathlineto{\pgfqpoint{1.700123in}{2.345385in}}%
\pgfpathlineto{\pgfqpoint{1.699963in}{2.417543in}}%
\pgfpathlineto{\pgfqpoint{1.700898in}{2.359109in}}%
\pgfpathlineto{\pgfqpoint{1.701206in}{2.414399in}}%
\pgfpathlineto{\pgfqpoint{1.701366in}{2.333873in}}%
\pgfpathlineto{\pgfqpoint{1.701994in}{2.349934in}}%
\pgfpathlineto{\pgfqpoint{1.703064in}{2.407829in}}%
\pgfpathlineto{\pgfqpoint{1.702117in}{2.339282in}}%
\pgfpathlineto{\pgfqpoint{1.703126in}{2.360958in}}%
\pgfpathlineto{\pgfqpoint{1.703224in}{2.344857in}}%
\pgfpathlineto{\pgfqpoint{1.703557in}{2.398284in}}%
\pgfpathlineto{\pgfqpoint{1.704061in}{2.380023in}}%
\pgfpathlineto{\pgfqpoint{1.704295in}{2.406242in}}%
\pgfpathlineto{\pgfqpoint{1.704467in}{2.332830in}}%
\pgfpathlineto{\pgfqpoint{1.705157in}{2.379089in}}%
\pgfpathlineto{\pgfqpoint{1.706301in}{2.344346in}}%
\pgfpathlineto{\pgfqpoint{1.705415in}{2.402898in}}%
\pgfpathlineto{\pgfqpoint{1.706314in}{2.345317in}}%
\pgfpathlineto{\pgfqpoint{1.707384in}{2.401796in}}%
\pgfpathlineto{\pgfqpoint{1.706929in}{2.345093in}}%
\pgfpathlineto{\pgfqpoint{1.707520in}{2.362177in}}%
\pgfpathlineto{\pgfqpoint{1.707557in}{2.341811in}}%
\pgfpathlineto{\pgfqpoint{1.708000in}{2.401345in}}%
\pgfpathlineto{\pgfqpoint{1.708603in}{2.385371in}}%
\pgfpathlineto{\pgfqpoint{1.709415in}{2.343192in}}%
\pgfpathlineto{\pgfqpoint{1.709735in}{2.402806in}}%
\pgfpathlineto{\pgfqpoint{1.710006in}{2.337451in}}%
\pgfpathlineto{\pgfqpoint{1.710350in}{2.408495in}}%
\pgfpathlineto{\pgfqpoint{1.710867in}{2.394148in}}%
\pgfpathlineto{\pgfqpoint{1.711594in}{2.403800in}}%
\pgfpathlineto{\pgfqpoint{1.711261in}{2.349585in}}%
\pgfpathlineto{\pgfqpoint{1.711852in}{2.352690in}}%
\pgfpathlineto{\pgfqpoint{1.711864in}{2.344543in}}%
\pgfpathlineto{\pgfqpoint{1.712824in}{2.410482in}}%
\pgfpathlineto{\pgfqpoint{1.712910in}{2.373973in}}%
\pgfpathlineto{\pgfqpoint{1.713452in}{2.409335in}}%
\pgfpathlineto{\pgfqpoint{1.713120in}{2.336526in}}%
\pgfpathlineto{\pgfqpoint{1.714055in}{2.409235in}}%
\pgfpathlineto{\pgfqpoint{1.714067in}{2.409344in}}%
\pgfpathlineto{\pgfqpoint{1.714092in}{2.401694in}}%
\pgfpathlineto{\pgfqpoint{1.714978in}{2.341958in}}%
\pgfpathlineto{\pgfqpoint{1.715212in}{2.389985in}}%
\pgfpathlineto{\pgfqpoint{1.716209in}{2.338902in}}%
\pgfpathlineto{\pgfqpoint{1.715914in}{2.413096in}}%
\pgfpathlineto{\pgfqpoint{1.716357in}{2.365629in}}%
\pgfpathlineto{\pgfqpoint{1.717157in}{2.414080in}}%
\pgfpathlineto{\pgfqpoint{1.716812in}{2.345761in}}%
\pgfpathlineto{\pgfqpoint{1.717440in}{2.350155in}}%
\pgfpathlineto{\pgfqpoint{1.717452in}{2.349800in}}%
\pgfpathlineto{\pgfqpoint{1.717489in}{2.373742in}}%
\pgfpathlineto{\pgfqpoint{1.718400in}{2.411746in}}%
\pgfpathlineto{\pgfqpoint{1.718067in}{2.345623in}}%
\pgfpathlineto{\pgfqpoint{1.718560in}{2.350422in}}%
\pgfpathlineto{\pgfqpoint{1.719286in}{2.341155in}}%
\pgfpathlineto{\pgfqpoint{1.719015in}{2.413445in}}%
\pgfpathlineto{\pgfqpoint{1.719581in}{2.395990in}}%
\pgfpathlineto{\pgfqpoint{1.720246in}{2.412590in}}%
\pgfpathlineto{\pgfqpoint{1.719901in}{2.344028in}}%
\pgfpathlineto{\pgfqpoint{1.720627in}{2.372326in}}%
\pgfpathlineto{\pgfqpoint{1.721760in}{2.342860in}}%
\pgfpathlineto{\pgfqpoint{1.721477in}{2.411746in}}%
\pgfpathlineto{\pgfqpoint{1.721772in}{2.346043in}}%
\pgfpathlineto{\pgfqpoint{1.721784in}{2.346016in}}%
\pgfpathlineto{\pgfqpoint{1.722104in}{2.416023in}}%
\pgfpathlineto{\pgfqpoint{1.722375in}{2.342372in}}%
\pgfpathlineto{\pgfqpoint{1.722904in}{2.354264in}}%
\pgfpathlineto{\pgfqpoint{1.723950in}{2.411557in}}%
\pgfpathlineto{\pgfqpoint{1.723150in}{2.342631in}}%
\pgfpathlineto{\pgfqpoint{1.724061in}{2.379180in}}%
\pgfpathlineto{\pgfqpoint{1.724381in}{2.342328in}}%
\pgfpathlineto{\pgfqpoint{1.724529in}{2.403371in}}%
\pgfpathlineto{\pgfqpoint{1.725193in}{2.416556in}}%
\pgfpathlineto{\pgfqpoint{1.724849in}{2.337935in}}%
\pgfpathlineto{\pgfqpoint{1.725575in}{2.359901in}}%
\pgfpathlineto{\pgfqpoint{1.725624in}{2.335762in}}%
\pgfpathlineto{\pgfqpoint{1.725809in}{2.418586in}}%
\pgfpathlineto{\pgfqpoint{1.726658in}{2.376808in}}%
\pgfpathlineto{\pgfqpoint{1.726707in}{2.341565in}}%
\pgfpathlineto{\pgfqpoint{1.727027in}{2.413286in}}%
\pgfpathlineto{\pgfqpoint{1.727667in}{2.423977in}}%
\pgfpathlineto{\pgfqpoint{1.727470in}{2.341184in}}%
\pgfpathlineto{\pgfqpoint{1.727827in}{2.368070in}}%
\pgfpathlineto{\pgfqpoint{1.727950in}{2.336125in}}%
\pgfpathlineto{\pgfqpoint{1.728898in}{2.423010in}}%
\pgfpathlineto{\pgfqpoint{1.728923in}{2.416896in}}%
\pgfpathlineto{\pgfqpoint{1.729821in}{2.335561in}}%
\pgfpathlineto{\pgfqpoint{1.730080in}{2.391272in}}%
\pgfpathlineto{\pgfqpoint{1.730757in}{2.424715in}}%
\pgfpathlineto{\pgfqpoint{1.730572in}{2.336672in}}%
\pgfpathlineto{\pgfqpoint{1.731027in}{2.350915in}}%
\pgfpathlineto{\pgfqpoint{1.731064in}{2.333106in}}%
\pgfpathlineto{\pgfqpoint{1.732000in}{2.425981in}}%
\pgfpathlineto{\pgfqpoint{1.732098in}{2.389141in}}%
\pgfpathlineto{\pgfqpoint{1.732615in}{2.421143in}}%
\pgfpathlineto{\pgfqpoint{1.732910in}{2.336057in}}%
\pgfpathlineto{\pgfqpoint{1.733009in}{2.353384in}}%
\pgfpathlineto{\pgfqpoint{1.733526in}{2.334436in}}%
\pgfpathlineto{\pgfqpoint{1.733858in}{2.426127in}}%
\pgfpathlineto{\pgfqpoint{1.734067in}{2.367587in}}%
\pgfpathlineto{\pgfqpoint{1.735089in}{2.421679in}}%
\pgfpathlineto{\pgfqpoint{1.734769in}{2.331410in}}%
\pgfpathlineto{\pgfqpoint{1.735200in}{2.392126in}}%
\pgfpathlineto{\pgfqpoint{1.735704in}{2.423739in}}%
\pgfpathlineto{\pgfqpoint{1.736012in}{2.338688in}}%
\pgfpathlineto{\pgfqpoint{1.736332in}{2.419113in}}%
\pgfpathlineto{\pgfqpoint{1.736627in}{2.330778in}}%
\pgfpathlineto{\pgfqpoint{1.736947in}{2.426538in}}%
\pgfpathlineto{\pgfqpoint{1.737464in}{2.398647in}}%
\pgfpathlineto{\pgfqpoint{1.737563in}{2.426543in}}%
\pgfpathlineto{\pgfqpoint{1.737858in}{2.335260in}}%
\pgfpathlineto{\pgfqpoint{1.738326in}{2.387781in}}%
\pgfpathlineto{\pgfqpoint{1.738473in}{2.335679in}}%
\pgfpathlineto{\pgfqpoint{1.738806in}{2.424037in}}%
\pgfpathlineto{\pgfqpoint{1.739409in}{2.422475in}}%
\pgfpathlineto{\pgfqpoint{1.739421in}{2.424367in}}%
\pgfpathlineto{\pgfqpoint{1.739717in}{2.329171in}}%
\pgfpathlineto{\pgfqpoint{1.740184in}{2.379272in}}%
\pgfpathlineto{\pgfqpoint{1.740332in}{2.333279in}}%
\pgfpathlineto{\pgfqpoint{1.740652in}{2.427756in}}%
\pgfpathlineto{\pgfqpoint{1.741267in}{2.415173in}}%
\pgfpathlineto{\pgfqpoint{1.741280in}{2.416369in}}%
\pgfpathlineto{\pgfqpoint{1.741538in}{2.344166in}}%
\pgfpathlineto{\pgfqpoint{1.741563in}{2.331989in}}%
\pgfpathlineto{\pgfqpoint{1.742510in}{2.429306in}}%
\pgfpathlineto{\pgfqpoint{1.742596in}{2.368436in}}%
\pgfpathlineto{\pgfqpoint{1.743753in}{2.428338in}}%
\pgfpathlineto{\pgfqpoint{1.742806in}{2.329900in}}%
\pgfpathlineto{\pgfqpoint{1.743766in}{2.422620in}}%
\pgfpathlineto{\pgfqpoint{1.744664in}{2.328974in}}%
\pgfpathlineto{\pgfqpoint{1.744898in}{2.404332in}}%
\pgfpathlineto{\pgfqpoint{1.745895in}{2.331297in}}%
\pgfpathlineto{\pgfqpoint{1.745612in}{2.429684in}}%
\pgfpathlineto{\pgfqpoint{1.746104in}{2.392927in}}%
\pgfpathlineto{\pgfqpoint{1.746843in}{2.426783in}}%
\pgfpathlineto{\pgfqpoint{1.746510in}{2.325542in}}%
\pgfpathlineto{\pgfqpoint{1.747187in}{2.373230in}}%
\pgfpathlineto{\pgfqpoint{1.747753in}{2.328302in}}%
\pgfpathlineto{\pgfqpoint{1.747458in}{2.425249in}}%
\pgfpathlineto{\pgfqpoint{1.748295in}{2.373833in}}%
\pgfpathlineto{\pgfqpoint{1.748701in}{2.431278in}}%
\pgfpathlineto{\pgfqpoint{1.748369in}{2.323177in}}%
\pgfpathlineto{\pgfqpoint{1.749083in}{2.331928in}}%
\pgfpathlineto{\pgfqpoint{1.749612in}{2.321781in}}%
\pgfpathlineto{\pgfqpoint{1.749316in}{2.426526in}}%
\pgfpathlineto{\pgfqpoint{1.749932in}{2.421915in}}%
\pgfpathlineto{\pgfqpoint{1.750560in}{2.431275in}}%
\pgfpathlineto{\pgfqpoint{1.750227in}{2.323409in}}%
\pgfpathlineto{\pgfqpoint{1.750806in}{2.369306in}}%
\pgfpathlineto{\pgfqpoint{1.751458in}{2.316794in}}%
\pgfpathlineto{\pgfqpoint{1.751803in}{2.436392in}}%
\pgfpathlineto{\pgfqpoint{1.751901in}{2.400080in}}%
\pgfpathlineto{\pgfqpoint{1.752701in}{2.324994in}}%
\pgfpathlineto{\pgfqpoint{1.752418in}{2.438792in}}%
\pgfpathlineto{\pgfqpoint{1.752972in}{2.402252in}}%
\pgfpathlineto{\pgfqpoint{1.753649in}{2.435710in}}%
\pgfpathlineto{\pgfqpoint{1.753316in}{2.318702in}}%
\pgfpathlineto{\pgfqpoint{1.753920in}{2.330394in}}%
\pgfpathlineto{\pgfqpoint{1.754547in}{2.313952in}}%
\pgfpathlineto{\pgfqpoint{1.754264in}{2.434926in}}%
\pgfpathlineto{\pgfqpoint{1.754867in}{2.423660in}}%
\pgfpathlineto{\pgfqpoint{1.755507in}{2.441339in}}%
\pgfpathlineto{\pgfqpoint{1.755163in}{2.319349in}}%
\pgfpathlineto{\pgfqpoint{1.755766in}{2.334927in}}%
\pgfpathlineto{\pgfqpoint{1.756406in}{2.314275in}}%
\pgfpathlineto{\pgfqpoint{1.756750in}{2.437708in}}%
\pgfpathlineto{\pgfqpoint{1.756812in}{2.375157in}}%
\pgfpathlineto{\pgfqpoint{1.757366in}{2.438703in}}%
\pgfpathlineto{\pgfqpoint{1.757636in}{2.317514in}}%
\pgfpathlineto{\pgfqpoint{1.758006in}{2.420514in}}%
\pgfpathlineto{\pgfqpoint{1.758252in}{2.313926in}}%
\pgfpathlineto{\pgfqpoint{1.758596in}{2.440861in}}%
\pgfpathlineto{\pgfqpoint{1.759113in}{2.414823in}}%
\pgfpathlineto{\pgfqpoint{1.760110in}{2.310282in}}%
\pgfpathlineto{\pgfqpoint{1.759212in}{2.434528in}}%
\pgfpathlineto{\pgfqpoint{1.760307in}{2.363326in}}%
\pgfpathlineto{\pgfqpoint{1.760455in}{2.439026in}}%
\pgfpathlineto{\pgfqpoint{1.761341in}{2.312354in}}%
\pgfpathlineto{\pgfqpoint{1.761403in}{2.357603in}}%
\pgfpathlineto{\pgfqpoint{1.761969in}{2.315512in}}%
\pgfpathlineto{\pgfqpoint{1.762313in}{2.440051in}}%
\pgfpathlineto{\pgfqpoint{1.762498in}{2.369721in}}%
\pgfpathlineto{\pgfqpoint{1.763544in}{2.443985in}}%
\pgfpathlineto{\pgfqpoint{1.763199in}{2.307279in}}%
\pgfpathlineto{\pgfqpoint{1.763618in}{2.385270in}}%
\pgfpathlineto{\pgfqpoint{1.764159in}{2.441779in}}%
\pgfpathlineto{\pgfqpoint{1.763815in}{2.311748in}}%
\pgfpathlineto{\pgfqpoint{1.764799in}{2.428712in}}%
\pgfpathlineto{\pgfqpoint{1.765058in}{2.312738in}}%
\pgfpathlineto{\pgfqpoint{1.765390in}{2.443327in}}%
\pgfpathlineto{\pgfqpoint{1.765932in}{2.402612in}}%
\pgfpathlineto{\pgfqpoint{1.765993in}{2.443291in}}%
\pgfpathlineto{\pgfqpoint{1.766289in}{2.309951in}}%
\pgfpathlineto{\pgfqpoint{1.766879in}{2.348472in}}%
\pgfpathlineto{\pgfqpoint{1.767643in}{2.313062in}}%
\pgfpathlineto{\pgfqpoint{1.767224in}{2.447681in}}%
\pgfpathlineto{\pgfqpoint{1.767950in}{2.385272in}}%
\pgfpathlineto{\pgfqpoint{1.769083in}{2.452327in}}%
\pgfpathlineto{\pgfqpoint{1.768147in}{2.312545in}}%
\pgfpathlineto{\pgfqpoint{1.769119in}{2.435721in}}%
\pgfpathlineto{\pgfqpoint{1.770006in}{2.316857in}}%
\pgfpathlineto{\pgfqpoint{1.769698in}{2.449750in}}%
\pgfpathlineto{\pgfqpoint{1.770264in}{2.396424in}}%
\pgfpathlineto{\pgfqpoint{1.770326in}{2.454877in}}%
\pgfpathlineto{\pgfqpoint{1.770744in}{2.315978in}}%
\pgfpathlineto{\pgfqpoint{1.771298in}{2.362401in}}%
\pgfpathlineto{\pgfqpoint{1.771987in}{2.314705in}}%
\pgfpathlineto{\pgfqpoint{1.772172in}{2.452268in}}%
\pgfpathlineto{\pgfqpoint{1.772393in}{2.365686in}}%
\pgfpathlineto{\pgfqpoint{1.773415in}{2.456147in}}%
\pgfpathlineto{\pgfqpoint{1.772603in}{2.312156in}}%
\pgfpathlineto{\pgfqpoint{1.773563in}{2.391469in}}%
\pgfpathlineto{\pgfqpoint{1.773846in}{2.312305in}}%
\pgfpathlineto{\pgfqpoint{1.774030in}{2.457160in}}%
\pgfpathlineto{\pgfqpoint{1.774621in}{2.426313in}}%
\pgfpathlineto{\pgfqpoint{1.775273in}{2.460792in}}%
\pgfpathlineto{\pgfqpoint{1.775089in}{2.313814in}}%
\pgfpathlineto{\pgfqpoint{1.775655in}{2.350112in}}%
\pgfpathlineto{\pgfqpoint{1.776332in}{2.304391in}}%
\pgfpathlineto{\pgfqpoint{1.775889in}{2.459209in}}%
\pgfpathlineto{\pgfqpoint{1.776504in}{2.453981in}}%
\pgfpathlineto{\pgfqpoint{1.776516in}{2.456309in}}%
\pgfpathlineto{\pgfqpoint{1.776824in}{2.324117in}}%
\pgfpathlineto{\pgfqpoint{1.776923in}{2.337950in}}%
\pgfpathlineto{\pgfqpoint{1.777575in}{2.303538in}}%
\pgfpathlineto{\pgfqpoint{1.777132in}{2.461507in}}%
\pgfpathlineto{\pgfqpoint{1.778006in}{2.367222in}}%
\pgfpathlineto{\pgfqpoint{1.778116in}{2.369304in}}%
\pgfpathlineto{\pgfqpoint{1.778178in}{2.315056in}}%
\pgfpathlineto{\pgfqpoint{1.778806in}{2.299258in}}%
\pgfpathlineto{\pgfqpoint{1.778990in}{2.464253in}}%
\pgfpathlineto{\pgfqpoint{1.779224in}{2.364873in}}%
\pgfpathlineto{\pgfqpoint{1.780233in}{2.462484in}}%
\pgfpathlineto{\pgfqpoint{1.779433in}{2.297883in}}%
\pgfpathlineto{\pgfqpoint{1.780381in}{2.392975in}}%
\pgfpathlineto{\pgfqpoint{1.780664in}{2.296288in}}%
\pgfpathlineto{\pgfqpoint{1.780849in}{2.465868in}}%
\pgfpathlineto{\pgfqpoint{1.781439in}{2.437589in}}%
\pgfpathlineto{\pgfqpoint{1.782092in}{2.466341in}}%
\pgfpathlineto{\pgfqpoint{1.781907in}{2.291039in}}%
\pgfpathlineto{\pgfqpoint{1.782473in}{2.358127in}}%
\pgfpathlineto{\pgfqpoint{1.782535in}{2.291084in}}%
\pgfpathlineto{\pgfqpoint{1.782707in}{2.462697in}}%
\pgfpathlineto{\pgfqpoint{1.783569in}{2.366731in}}%
\pgfpathlineto{\pgfqpoint{1.783950in}{2.466845in}}%
\pgfpathlineto{\pgfqpoint{1.783766in}{2.287634in}}%
\pgfpathlineto{\pgfqpoint{1.784726in}{2.391982in}}%
\pgfpathlineto{\pgfqpoint{1.785009in}{2.286479in}}%
\pgfpathlineto{\pgfqpoint{1.785193in}{2.463517in}}%
\pgfpathlineto{\pgfqpoint{1.785772in}{2.423908in}}%
\pgfpathlineto{\pgfqpoint{1.785809in}{2.461490in}}%
\pgfpathlineto{\pgfqpoint{1.786252in}{2.293479in}}%
\pgfpathlineto{\pgfqpoint{1.786818in}{2.360486in}}%
\pgfpathlineto{\pgfqpoint{1.786867in}{2.291604in}}%
\pgfpathlineto{\pgfqpoint{1.787052in}{2.463646in}}%
\pgfpathlineto{\pgfqpoint{1.787913in}{2.380432in}}%
\pgfpathlineto{\pgfqpoint{1.788110in}{2.301565in}}%
\pgfpathlineto{\pgfqpoint{1.788282in}{2.459024in}}%
\pgfpathlineto{\pgfqpoint{1.788886in}{2.436126in}}%
\pgfpathlineto{\pgfqpoint{1.788910in}{2.456260in}}%
\pgfpathlineto{\pgfqpoint{1.789341in}{2.311505in}}%
\pgfpathlineto{\pgfqpoint{1.789919in}{2.365623in}}%
\pgfpathlineto{\pgfqpoint{1.789969in}{2.313859in}}%
\pgfpathlineto{\pgfqpoint{1.790153in}{2.449742in}}%
\pgfpathlineto{\pgfqpoint{1.791015in}{2.385294in}}%
\pgfpathlineto{\pgfqpoint{1.791212in}{2.327481in}}%
\pgfpathlineto{\pgfqpoint{1.791384in}{2.430556in}}%
\pgfpathlineto{\pgfqpoint{1.792098in}{2.380158in}}%
\pgfpathlineto{\pgfqpoint{1.792627in}{2.424410in}}%
\pgfpathlineto{\pgfqpoint{1.792922in}{2.341247in}}%
\pgfpathlineto{\pgfqpoint{1.793218in}{2.401469in}}%
\pgfpathlineto{\pgfqpoint{1.793242in}{2.421099in}}%
\pgfpathlineto{\pgfqpoint{1.793919in}{2.331867in}}%
\pgfpathlineto{\pgfqpoint{1.794289in}{2.364144in}}%
\pgfpathlineto{\pgfqpoint{1.794535in}{2.332671in}}%
\pgfpathlineto{\pgfqpoint{1.795089in}{2.424794in}}%
\pgfpathlineto{\pgfqpoint{1.795384in}{2.371278in}}%
\pgfpathlineto{\pgfqpoint{1.795778in}{2.332212in}}%
\pgfpathlineto{\pgfqpoint{1.796332in}{2.413540in}}%
\pgfpathlineto{\pgfqpoint{1.796344in}{2.413903in}}%
\pgfpathlineto{\pgfqpoint{1.796356in}{2.397730in}}%
\pgfpathlineto{\pgfqpoint{1.797021in}{2.322153in}}%
\pgfpathlineto{\pgfqpoint{1.797181in}{2.406387in}}%
\pgfpathlineto{\pgfqpoint{1.797476in}{2.378276in}}%
\pgfpathlineto{\pgfqpoint{1.797636in}{2.313438in}}%
\pgfpathlineto{\pgfqpoint{1.797796in}{2.416667in}}%
\pgfpathlineto{\pgfqpoint{1.798584in}{2.367355in}}%
\pgfpathlineto{\pgfqpoint{1.799655in}{2.420050in}}%
\pgfpathlineto{\pgfqpoint{1.799495in}{2.324509in}}%
\pgfpathlineto{\pgfqpoint{1.799704in}{2.375796in}}%
\pgfpathlineto{\pgfqpoint{1.800381in}{2.421756in}}%
\pgfpathlineto{\pgfqpoint{1.800110in}{2.320565in}}%
\pgfpathlineto{\pgfqpoint{1.800713in}{2.342159in}}%
\pgfpathlineto{\pgfqpoint{1.801353in}{2.319102in}}%
\pgfpathlineto{\pgfqpoint{1.800898in}{2.427011in}}%
\pgfpathlineto{\pgfqpoint{1.801809in}{2.362213in}}%
\pgfpathlineto{\pgfqpoint{1.802584in}{2.324744in}}%
\pgfpathlineto{\pgfqpoint{1.802227in}{2.430112in}}%
\pgfpathlineto{\pgfqpoint{1.802719in}{2.417675in}}%
\pgfpathlineto{\pgfqpoint{1.803470in}{2.437994in}}%
\pgfpathlineto{\pgfqpoint{1.803199in}{2.321726in}}%
\pgfpathlineto{\pgfqpoint{1.803679in}{2.345836in}}%
\pgfpathlineto{\pgfqpoint{1.804442in}{2.317118in}}%
\pgfpathlineto{\pgfqpoint{1.804085in}{2.435579in}}%
\pgfpathlineto{\pgfqpoint{1.804553in}{2.382967in}}%
\pgfpathlineto{\pgfqpoint{1.805329in}{2.438604in}}%
\pgfpathlineto{\pgfqpoint{1.805058in}{2.323212in}}%
\pgfpathlineto{\pgfqpoint{1.805612in}{2.333348in}}%
\pgfpathlineto{\pgfqpoint{1.806301in}{2.317403in}}%
\pgfpathlineto{\pgfqpoint{1.805944in}{2.440453in}}%
\pgfpathlineto{\pgfqpoint{1.806449in}{2.421278in}}%
\pgfpathlineto{\pgfqpoint{1.806559in}{2.444212in}}%
\pgfpathlineto{\pgfqpoint{1.806916in}{2.319895in}}%
\pgfpathlineto{\pgfqpoint{1.807285in}{2.375101in}}%
\pgfpathlineto{\pgfqpoint{1.808110in}{2.318081in}}%
\pgfpathlineto{\pgfqpoint{1.807802in}{2.448959in}}%
\pgfpathlineto{\pgfqpoint{1.808369in}{2.394557in}}%
\pgfpathlineto{\pgfqpoint{1.809033in}{2.453175in}}%
\pgfpathlineto{\pgfqpoint{1.809353in}{2.312200in}}%
\pgfpathlineto{\pgfqpoint{1.809452in}{2.372872in}}%
\pgfpathlineto{\pgfqpoint{1.809661in}{2.455964in}}%
\pgfpathlineto{\pgfqpoint{1.809932in}{2.335857in}}%
\pgfpathlineto{\pgfqpoint{1.809969in}{2.309458in}}%
\pgfpathlineto{\pgfqpoint{1.810276in}{2.458495in}}%
\pgfpathlineto{\pgfqpoint{1.810892in}{2.453715in}}%
\pgfpathlineto{\pgfqpoint{1.810941in}{2.420575in}}%
\pgfpathlineto{\pgfqpoint{1.811827in}{2.308312in}}%
\pgfpathlineto{\pgfqpoint{1.811519in}{2.456416in}}%
\pgfpathlineto{\pgfqpoint{1.812085in}{2.374474in}}%
\pgfpathlineto{\pgfqpoint{1.812135in}{2.461088in}}%
\pgfpathlineto{\pgfqpoint{1.812455in}{2.309000in}}%
\pgfpathlineto{\pgfqpoint{1.813193in}{2.388149in}}%
\pgfpathlineto{\pgfqpoint{1.813710in}{2.314718in}}%
\pgfpathlineto{\pgfqpoint{1.813378in}{2.454356in}}%
\pgfpathlineto{\pgfqpoint{1.813845in}{2.415740in}}%
\pgfpathlineto{\pgfqpoint{1.814609in}{2.455563in}}%
\pgfpathlineto{\pgfqpoint{1.814325in}{2.311010in}}%
\pgfpathlineto{\pgfqpoint{1.814879in}{2.352056in}}%
\pgfpathlineto{\pgfqpoint{1.814941in}{2.310371in}}%
\pgfpathlineto{\pgfqpoint{1.815224in}{2.454431in}}%
\pgfpathlineto{\pgfqpoint{1.815839in}{2.447627in}}%
\pgfpathlineto{\pgfqpoint{1.815852in}{2.448873in}}%
\pgfpathlineto{\pgfqpoint{1.815962in}{2.356752in}}%
\pgfpathlineto{\pgfqpoint{1.816012in}{2.364986in}}%
\pgfpathlineto{\pgfqpoint{1.816799in}{2.310293in}}%
\pgfpathlineto{\pgfqpoint{1.816467in}{2.451020in}}%
\pgfpathlineto{\pgfqpoint{1.817070in}{2.435307in}}%
\pgfpathlineto{\pgfqpoint{1.817082in}{2.448964in}}%
\pgfpathlineto{\pgfqpoint{1.818030in}{2.308232in}}%
\pgfpathlineto{\pgfqpoint{1.818129in}{2.387006in}}%
\pgfpathlineto{\pgfqpoint{1.818153in}{2.383529in}}%
\pgfpathlineto{\pgfqpoint{1.818165in}{2.398091in}}%
\pgfpathlineto{\pgfqpoint{1.818941in}{2.443426in}}%
\pgfpathlineto{\pgfqpoint{1.818658in}{2.306961in}}%
\pgfpathlineto{\pgfqpoint{1.819224in}{2.341274in}}%
\pgfpathlineto{\pgfqpoint{1.819273in}{2.303961in}}%
\pgfpathlineto{\pgfqpoint{1.820061in}{2.446556in}}%
\pgfpathlineto{\pgfqpoint{1.820307in}{2.370527in}}%
\pgfpathlineto{\pgfqpoint{1.820676in}{2.449213in}}%
\pgfpathlineto{\pgfqpoint{1.820504in}{2.300930in}}%
\pgfpathlineto{\pgfqpoint{1.821107in}{2.319563in}}%
\pgfpathlineto{\pgfqpoint{1.821747in}{2.298495in}}%
\pgfpathlineto{\pgfqpoint{1.821919in}{2.449545in}}%
\pgfpathlineto{\pgfqpoint{1.822165in}{2.366938in}}%
\pgfpathlineto{\pgfqpoint{1.822645in}{2.454474in}}%
\pgfpathlineto{\pgfqpoint{1.822362in}{2.302209in}}%
\pgfpathlineto{\pgfqpoint{1.822978in}{2.306818in}}%
\pgfpathlineto{\pgfqpoint{1.822990in}{2.302378in}}%
\pgfpathlineto{\pgfqpoint{1.823261in}{2.453353in}}%
\pgfpathlineto{\pgfqpoint{1.823741in}{2.408157in}}%
\pgfpathlineto{\pgfqpoint{1.824504in}{2.463464in}}%
\pgfpathlineto{\pgfqpoint{1.824221in}{2.305278in}}%
\pgfpathlineto{\pgfqpoint{1.824812in}{2.329373in}}%
\pgfpathlineto{\pgfqpoint{1.824848in}{2.305074in}}%
\pgfpathlineto{\pgfqpoint{1.825747in}{2.470125in}}%
\pgfpathlineto{\pgfqpoint{1.825895in}{2.355971in}}%
\pgfpathlineto{\pgfqpoint{1.826362in}{2.472944in}}%
\pgfpathlineto{\pgfqpoint{1.826695in}{2.299461in}}%
\pgfpathlineto{\pgfqpoint{1.827064in}{2.390240in}}%
\pgfpathlineto{\pgfqpoint{1.827938in}{2.296982in}}%
\pgfpathlineto{\pgfqpoint{1.827605in}{2.468786in}}%
\pgfpathlineto{\pgfqpoint{1.828172in}{2.360218in}}%
\pgfpathlineto{\pgfqpoint{1.828221in}{2.468883in}}%
\pgfpathlineto{\pgfqpoint{1.829033in}{2.293312in}}%
\pgfpathlineto{\pgfqpoint{1.829328in}{2.456616in}}%
\pgfpathlineto{\pgfqpoint{1.829341in}{2.457453in}}%
\pgfpathlineto{\pgfqpoint{1.829365in}{2.412047in}}%
\pgfpathlineto{\pgfqpoint{1.830276in}{2.288040in}}%
\pgfpathlineto{\pgfqpoint{1.829452in}{2.470777in}}%
\pgfpathlineto{\pgfqpoint{1.830473in}{2.405984in}}%
\pgfpathlineto{\pgfqpoint{1.830892in}{2.285621in}}%
\pgfpathlineto{\pgfqpoint{1.830695in}{2.477359in}}%
\pgfpathlineto{\pgfqpoint{1.831175in}{2.451664in}}%
\pgfpathlineto{\pgfqpoint{1.831925in}{2.480572in}}%
\pgfpathlineto{\pgfqpoint{1.831507in}{2.286539in}}%
\pgfpathlineto{\pgfqpoint{1.832085in}{2.358385in}}%
\pgfpathlineto{\pgfqpoint{1.832750in}{2.281771in}}%
\pgfpathlineto{\pgfqpoint{1.832553in}{2.481829in}}%
\pgfpathlineto{\pgfqpoint{1.833144in}{2.446792in}}%
\pgfpathlineto{\pgfqpoint{1.833784in}{2.487325in}}%
\pgfpathlineto{\pgfqpoint{1.834116in}{2.280065in}}%
\pgfpathlineto{\pgfqpoint{1.834227in}{2.389882in}}%
\pgfpathlineto{\pgfqpoint{1.834399in}{2.489991in}}%
\pgfpathlineto{\pgfqpoint{1.834608in}{2.282631in}}%
\pgfpathlineto{\pgfqpoint{1.834719in}{2.293775in}}%
\pgfpathlineto{\pgfqpoint{1.835359in}{2.276445in}}%
\pgfpathlineto{\pgfqpoint{1.835642in}{2.498042in}}%
\pgfpathlineto{\pgfqpoint{1.835778in}{2.347051in}}%
\pgfpathlineto{\pgfqpoint{1.836873in}{2.500410in}}%
\pgfpathlineto{\pgfqpoint{1.836590in}{2.264910in}}%
\pgfpathlineto{\pgfqpoint{1.836959in}{2.402569in}}%
\pgfpathlineto{\pgfqpoint{1.837205in}{2.265037in}}%
\pgfpathlineto{\pgfqpoint{1.837501in}{2.498953in}}%
\pgfpathlineto{\pgfqpoint{1.838067in}{2.384477in}}%
\pgfpathlineto{\pgfqpoint{1.838732in}{2.506340in}}%
\pgfpathlineto{\pgfqpoint{1.839064in}{2.262886in}}%
\pgfpathlineto{\pgfqpoint{1.839175in}{2.392775in}}%
\pgfpathlineto{\pgfqpoint{1.839975in}{2.507616in}}%
\pgfpathlineto{\pgfqpoint{1.839679in}{2.263812in}}%
\pgfpathlineto{\pgfqpoint{1.840147in}{2.305592in}}%
\pgfpathlineto{\pgfqpoint{1.840922in}{2.260140in}}%
\pgfpathlineto{\pgfqpoint{1.840590in}{2.508289in}}%
\pgfpathlineto{\pgfqpoint{1.841168in}{2.413224in}}%
\pgfpathlineto{\pgfqpoint{1.841205in}{2.508135in}}%
\pgfpathlineto{\pgfqpoint{1.842153in}{2.253920in}}%
\pgfpathlineto{\pgfqpoint{1.842264in}{2.395009in}}%
\pgfpathlineto{\pgfqpoint{1.843064in}{2.513172in}}%
\pgfpathlineto{\pgfqpoint{1.842768in}{2.253524in}}%
\pgfpathlineto{\pgfqpoint{1.843224in}{2.338163in}}%
\pgfpathlineto{\pgfqpoint{1.844011in}{2.244517in}}%
\pgfpathlineto{\pgfqpoint{1.843679in}{2.512450in}}%
\pgfpathlineto{\pgfqpoint{1.844270in}{2.444306in}}%
\pgfpathlineto{\pgfqpoint{1.844307in}{2.515290in}}%
\pgfpathlineto{\pgfqpoint{1.845242in}{2.243893in}}%
\pgfpathlineto{\pgfqpoint{1.845365in}{2.400377in}}%
\pgfpathlineto{\pgfqpoint{1.846165in}{2.510636in}}%
\pgfpathlineto{\pgfqpoint{1.845858in}{2.243511in}}%
\pgfpathlineto{\pgfqpoint{1.846411in}{2.315229in}}%
\pgfpathlineto{\pgfqpoint{1.847101in}{2.232168in}}%
\pgfpathlineto{\pgfqpoint{1.846781in}{2.511480in}}%
\pgfpathlineto{\pgfqpoint{1.847359in}{2.433366in}}%
\pgfpathlineto{\pgfqpoint{1.847396in}{2.509501in}}%
\pgfpathlineto{\pgfqpoint{1.848344in}{2.228824in}}%
\pgfpathlineto{\pgfqpoint{1.848455in}{2.388960in}}%
\pgfpathlineto{\pgfqpoint{1.849255in}{2.513811in}}%
\pgfpathlineto{\pgfqpoint{1.848959in}{2.225902in}}%
\pgfpathlineto{\pgfqpoint{1.849439in}{2.285648in}}%
\pgfpathlineto{\pgfqpoint{1.849575in}{2.224884in}}%
\pgfpathlineto{\pgfqpoint{1.849870in}{2.516536in}}%
\pgfpathlineto{\pgfqpoint{1.850448in}{2.427685in}}%
\pgfpathlineto{\pgfqpoint{1.851113in}{2.517643in}}%
\pgfpathlineto{\pgfqpoint{1.851433in}{2.223820in}}%
\pgfpathlineto{\pgfqpoint{1.851544in}{2.383249in}}%
\pgfpathlineto{\pgfqpoint{1.851728in}{2.514242in}}%
\pgfpathlineto{\pgfqpoint{1.852048in}{2.227741in}}%
\pgfpathlineto{\pgfqpoint{1.852528in}{2.289463in}}%
\pgfpathlineto{\pgfqpoint{1.853291in}{2.228778in}}%
\pgfpathlineto{\pgfqpoint{1.853464in}{2.517644in}}%
\pgfpathlineto{\pgfqpoint{1.853562in}{2.494146in}}%
\pgfpathlineto{\pgfqpoint{1.854079in}{2.515331in}}%
\pgfpathlineto{\pgfqpoint{1.853907in}{2.230388in}}%
\pgfpathlineto{\pgfqpoint{1.854473in}{2.292657in}}%
\pgfpathlineto{\pgfqpoint{1.854535in}{2.230170in}}%
\pgfpathlineto{\pgfqpoint{1.854670in}{2.439359in}}%
\pgfpathlineto{\pgfqpoint{1.855322in}{2.516265in}}%
\pgfpathlineto{\pgfqpoint{1.855150in}{2.229183in}}%
\pgfpathlineto{\pgfqpoint{1.855728in}{2.283450in}}%
\pgfpathlineto{\pgfqpoint{1.856393in}{2.229662in}}%
\pgfpathlineto{\pgfqpoint{1.856553in}{2.526120in}}%
\pgfpathlineto{\pgfqpoint{1.856799in}{2.351034in}}%
\pgfpathlineto{\pgfqpoint{1.857168in}{2.526085in}}%
\pgfpathlineto{\pgfqpoint{1.857008in}{2.225589in}}%
\pgfpathlineto{\pgfqpoint{1.857993in}{2.433050in}}%
\pgfpathlineto{\pgfqpoint{1.858239in}{2.230455in}}%
\pgfpathlineto{\pgfqpoint{1.858411in}{2.523162in}}%
\pgfpathlineto{\pgfqpoint{1.859101in}{2.422426in}}%
\pgfpathlineto{\pgfqpoint{1.859642in}{2.523807in}}%
\pgfpathlineto{\pgfqpoint{1.860098in}{2.232106in}}%
\pgfpathlineto{\pgfqpoint{1.860196in}{2.378565in}}%
\pgfpathlineto{\pgfqpoint{1.860258in}{2.522987in}}%
\pgfpathlineto{\pgfqpoint{1.860713in}{2.234819in}}%
\pgfpathlineto{\pgfqpoint{1.861168in}{2.319736in}}%
\pgfpathlineto{\pgfqpoint{1.861944in}{2.233160in}}%
\pgfpathlineto{\pgfqpoint{1.861488in}{2.516224in}}%
\pgfpathlineto{\pgfqpoint{1.862227in}{2.496542in}}%
\pgfpathlineto{\pgfqpoint{1.862239in}{2.496520in}}%
\pgfpathlineto{\pgfqpoint{1.863064in}{2.230250in}}%
\pgfpathlineto{\pgfqpoint{1.862731in}{2.512233in}}%
\pgfpathlineto{\pgfqpoint{1.863335in}{2.496593in}}%
\pgfpathlineto{\pgfqpoint{1.863347in}{2.511488in}}%
\pgfpathlineto{\pgfqpoint{1.863679in}{2.227641in}}%
\pgfpathlineto{\pgfqpoint{1.864258in}{2.320233in}}%
\pgfpathlineto{\pgfqpoint{1.864922in}{2.221714in}}%
\pgfpathlineto{\pgfqpoint{1.864590in}{2.504489in}}%
\pgfpathlineto{\pgfqpoint{1.865316in}{2.488324in}}%
\pgfpathlineto{\pgfqpoint{1.865328in}{2.490542in}}%
\pgfpathlineto{\pgfqpoint{1.865488in}{2.335047in}}%
\pgfpathlineto{\pgfqpoint{1.866153in}{2.215805in}}%
\pgfpathlineto{\pgfqpoint{1.865821in}{2.502639in}}%
\pgfpathlineto{\pgfqpoint{1.866547in}{2.484207in}}%
\pgfpathlineto{\pgfqpoint{1.867064in}{2.499206in}}%
\pgfpathlineto{\pgfqpoint{1.866768in}{2.212633in}}%
\pgfpathlineto{\pgfqpoint{1.867334in}{2.343129in}}%
\pgfpathlineto{\pgfqpoint{1.868011in}{2.207468in}}%
\pgfpathlineto{\pgfqpoint{1.867679in}{2.508025in}}%
\pgfpathlineto{\pgfqpoint{1.868381in}{2.447476in}}%
\pgfpathlineto{\pgfqpoint{1.868910in}{2.509443in}}%
\pgfpathlineto{\pgfqpoint{1.869242in}{2.210708in}}%
\pgfpathlineto{\pgfqpoint{1.869365in}{2.242394in}}%
\pgfpathlineto{\pgfqpoint{1.870153in}{2.511844in}}%
\pgfpathlineto{\pgfqpoint{1.869870in}{2.209287in}}%
\pgfpathlineto{\pgfqpoint{1.870448in}{2.257791in}}%
\pgfpathlineto{\pgfqpoint{1.871101in}{2.204956in}}%
\pgfpathlineto{\pgfqpoint{1.870768in}{2.511853in}}%
\pgfpathlineto{\pgfqpoint{1.871470in}{2.443317in}}%
\pgfpathlineto{\pgfqpoint{1.872011in}{2.514875in}}%
\pgfpathlineto{\pgfqpoint{1.872344in}{2.208054in}}%
\pgfpathlineto{\pgfqpoint{1.872454in}{2.246298in}}%
\pgfpathlineto{\pgfqpoint{1.872479in}{2.273528in}}%
\pgfpathlineto{\pgfqpoint{1.873378in}{2.518747in}}%
\pgfpathlineto{\pgfqpoint{1.872959in}{2.205916in}}%
\pgfpathlineto{\pgfqpoint{1.873562in}{2.211683in}}%
\pgfpathlineto{\pgfqpoint{1.873574in}{2.206957in}}%
\pgfpathlineto{\pgfqpoint{1.873858in}{2.517276in}}%
\pgfpathlineto{\pgfqpoint{1.873956in}{2.468722in}}%
\pgfpathlineto{\pgfqpoint{1.874621in}{2.522581in}}%
\pgfpathlineto{\pgfqpoint{1.874190in}{2.205224in}}%
\pgfpathlineto{\pgfqpoint{1.874928in}{2.242401in}}%
\pgfpathlineto{\pgfqpoint{1.874953in}{2.271907in}}%
\pgfpathlineto{\pgfqpoint{1.875236in}{2.528474in}}%
\pgfpathlineto{\pgfqpoint{1.875421in}{2.208668in}}%
\pgfpathlineto{\pgfqpoint{1.876036in}{2.209722in}}%
\pgfpathlineto{\pgfqpoint{1.876048in}{2.208969in}}%
\pgfpathlineto{\pgfqpoint{1.876085in}{2.258609in}}%
\pgfpathlineto{\pgfqpoint{1.876467in}{2.531829in}}%
\pgfpathlineto{\pgfqpoint{1.876664in}{2.205649in}}%
\pgfpathlineto{\pgfqpoint{1.877218in}{2.299157in}}%
\pgfpathlineto{\pgfqpoint{1.877894in}{2.201259in}}%
\pgfpathlineto{\pgfqpoint{1.877698in}{2.532328in}}%
\pgfpathlineto{\pgfqpoint{1.878178in}{2.508309in}}%
\pgfpathlineto{\pgfqpoint{1.878941in}{2.539503in}}%
\pgfpathlineto{\pgfqpoint{1.878510in}{2.200239in}}%
\pgfpathlineto{\pgfqpoint{1.879088in}{2.269593in}}%
\pgfpathlineto{\pgfqpoint{1.879741in}{2.200615in}}%
\pgfpathlineto{\pgfqpoint{1.879556in}{2.539274in}}%
\pgfpathlineto{\pgfqpoint{1.880122in}{2.431406in}}%
\pgfpathlineto{\pgfqpoint{1.880171in}{2.537224in}}%
\pgfpathlineto{\pgfqpoint{1.880368in}{2.196939in}}%
\pgfpathlineto{\pgfqpoint{1.881230in}{2.433226in}}%
\pgfpathlineto{\pgfqpoint{1.882030in}{2.541406in}}%
\pgfpathlineto{\pgfqpoint{1.881599in}{2.199574in}}%
\pgfpathlineto{\pgfqpoint{1.882190in}{2.245819in}}%
\pgfpathlineto{\pgfqpoint{1.882227in}{2.197029in}}%
\pgfpathlineto{\pgfqpoint{1.883138in}{2.536931in}}%
\pgfpathlineto{\pgfqpoint{1.883211in}{2.430434in}}%
\pgfpathlineto{\pgfqpoint{1.883888in}{2.538150in}}%
\pgfpathlineto{\pgfqpoint{1.884073in}{2.193801in}}%
\pgfpathlineto{\pgfqpoint{1.884319in}{2.419912in}}%
\pgfpathlineto{\pgfqpoint{1.885131in}{2.540890in}}%
\pgfpathlineto{\pgfqpoint{1.884688in}{2.194625in}}%
\pgfpathlineto{\pgfqpoint{1.885291in}{2.217883in}}%
\pgfpathlineto{\pgfqpoint{1.885304in}{2.196852in}}%
\pgfpathlineto{\pgfqpoint{1.885747in}{2.544899in}}%
\pgfpathlineto{\pgfqpoint{1.886214in}{2.506041in}}%
\pgfpathlineto{\pgfqpoint{1.886977in}{2.544224in}}%
\pgfpathlineto{\pgfqpoint{1.886547in}{2.194695in}}%
\pgfpathlineto{\pgfqpoint{1.887150in}{2.214003in}}%
\pgfpathlineto{\pgfqpoint{1.887162in}{2.193976in}}%
\pgfpathlineto{\pgfqpoint{1.887605in}{2.542712in}}%
\pgfpathlineto{\pgfqpoint{1.888073in}{2.504892in}}%
\pgfpathlineto{\pgfqpoint{1.888836in}{2.544191in}}%
\pgfpathlineto{\pgfqpoint{1.888405in}{2.193235in}}%
\pgfpathlineto{\pgfqpoint{1.888984in}{2.270958in}}%
\pgfpathlineto{\pgfqpoint{1.889021in}{2.194144in}}%
\pgfpathlineto{\pgfqpoint{1.889464in}{2.538927in}}%
\pgfpathlineto{\pgfqpoint{1.890017in}{2.430630in}}%
\pgfpathlineto{\pgfqpoint{1.890079in}{2.538497in}}%
\pgfpathlineto{\pgfqpoint{1.890879in}{2.192553in}}%
\pgfpathlineto{\pgfqpoint{1.891125in}{2.422596in}}%
\pgfpathlineto{\pgfqpoint{1.891937in}{2.535290in}}%
\pgfpathlineto{\pgfqpoint{1.891494in}{2.192959in}}%
\pgfpathlineto{\pgfqpoint{1.892110in}{2.199627in}}%
\pgfpathlineto{\pgfqpoint{1.892122in}{2.193998in}}%
\pgfpathlineto{\pgfqpoint{1.892417in}{2.505806in}}%
\pgfpathlineto{\pgfqpoint{1.892516in}{2.499829in}}%
\pgfpathlineto{\pgfqpoint{1.892541in}{2.532847in}}%
\pgfpathlineto{\pgfqpoint{1.893353in}{2.188205in}}%
\pgfpathlineto{\pgfqpoint{1.893476in}{2.249668in}}%
\pgfpathlineto{\pgfqpoint{1.893968in}{2.190596in}}%
\pgfpathlineto{\pgfqpoint{1.893796in}{2.531251in}}%
\pgfpathlineto{\pgfqpoint{1.894350in}{2.443323in}}%
\pgfpathlineto{\pgfqpoint{1.894411in}{2.530313in}}%
\pgfpathlineto{\pgfqpoint{1.895211in}{2.188377in}}%
\pgfpathlineto{\pgfqpoint{1.895457in}{2.437066in}}%
\pgfpathlineto{\pgfqpoint{1.896270in}{2.529239in}}%
\pgfpathlineto{\pgfqpoint{1.895827in}{2.187492in}}%
\pgfpathlineto{\pgfqpoint{1.896381in}{2.312215in}}%
\pgfpathlineto{\pgfqpoint{1.896442in}{2.185686in}}%
\pgfpathlineto{\pgfqpoint{1.896885in}{2.532208in}}%
\pgfpathlineto{\pgfqpoint{1.897341in}{2.468846in}}%
\pgfpathlineto{\pgfqpoint{1.898128in}{2.533816in}}%
\pgfpathlineto{\pgfqpoint{1.897685in}{2.185874in}}%
\pgfpathlineto{\pgfqpoint{1.898362in}{2.282353in}}%
\pgfpathlineto{\pgfqpoint{1.898436in}{2.244830in}}%
\pgfpathlineto{\pgfqpoint{1.898497in}{2.404520in}}%
\pgfpathlineto{\pgfqpoint{1.898744in}{2.539738in}}%
\pgfpathlineto{\pgfqpoint{1.898916in}{2.192544in}}%
\pgfpathlineto{\pgfqpoint{1.899494in}{2.286370in}}%
\pgfpathlineto{\pgfqpoint{1.900159in}{2.191371in}}%
\pgfpathlineto{\pgfqpoint{1.899987in}{2.540633in}}%
\pgfpathlineto{\pgfqpoint{1.900541in}{2.438236in}}%
\pgfpathlineto{\pgfqpoint{1.900602in}{2.540426in}}%
\pgfpathlineto{\pgfqpoint{1.900774in}{2.192756in}}%
\pgfpathlineto{\pgfqpoint{1.901673in}{2.477063in}}%
\pgfpathlineto{\pgfqpoint{1.902461in}{2.544976in}}%
\pgfpathlineto{\pgfqpoint{1.902017in}{2.194631in}}%
\pgfpathlineto{\pgfqpoint{1.902608in}{2.233146in}}%
\pgfpathlineto{\pgfqpoint{1.902633in}{2.195985in}}%
\pgfpathlineto{\pgfqpoint{1.903076in}{2.547027in}}%
\pgfpathlineto{\pgfqpoint{1.903630in}{2.413721in}}%
\pgfpathlineto{\pgfqpoint{1.903704in}{2.545238in}}%
\pgfpathlineto{\pgfqpoint{1.903876in}{2.196921in}}%
\pgfpathlineto{\pgfqpoint{1.904590in}{2.258844in}}%
\pgfpathlineto{\pgfqpoint{1.905230in}{2.216134in}}%
\pgfpathlineto{\pgfqpoint{1.905550in}{2.530749in}}%
\pgfpathlineto{\pgfqpoint{1.905648in}{2.369829in}}%
\pgfpathlineto{\pgfqpoint{1.906473in}{2.218181in}}%
\pgfpathlineto{\pgfqpoint{1.906177in}{2.541837in}}%
\pgfpathlineto{\pgfqpoint{1.906731in}{2.394403in}}%
\pgfpathlineto{\pgfqpoint{1.907408in}{2.550214in}}%
\pgfpathlineto{\pgfqpoint{1.906965in}{2.216757in}}%
\pgfpathlineto{\pgfqpoint{1.907888in}{2.480909in}}%
\pgfpathlineto{\pgfqpoint{1.908651in}{2.561262in}}%
\pgfpathlineto{\pgfqpoint{1.908196in}{2.229842in}}%
\pgfpathlineto{\pgfqpoint{1.908750in}{2.322592in}}%
\pgfpathlineto{\pgfqpoint{1.909464in}{2.224964in}}%
\pgfpathlineto{\pgfqpoint{1.909279in}{2.553374in}}%
\pgfpathlineto{\pgfqpoint{1.909833in}{2.412067in}}%
\pgfpathlineto{\pgfqpoint{1.909894in}{2.512675in}}%
\pgfpathlineto{\pgfqpoint{1.910190in}{2.212748in}}%
\pgfpathlineto{\pgfqpoint{1.910977in}{2.487642in}}%
\pgfpathlineto{\pgfqpoint{1.911310in}{2.217723in}}%
\pgfpathlineto{\pgfqpoint{1.911137in}{2.511037in}}%
\pgfpathlineto{\pgfqpoint{1.912134in}{2.422952in}}%
\pgfpathlineto{\pgfqpoint{1.912368in}{2.546852in}}%
\pgfpathlineto{\pgfqpoint{1.912528in}{2.240222in}}%
\pgfpathlineto{\pgfqpoint{1.913144in}{2.245015in}}%
\pgfpathlineto{\pgfqpoint{1.913599in}{2.546253in}}%
\pgfpathlineto{\pgfqpoint{1.913907in}{2.242938in}}%
\pgfpathlineto{\pgfqpoint{1.914325in}{2.325799in}}%
\pgfpathlineto{\pgfqpoint{1.914350in}{2.248599in}}%
\pgfpathlineto{\pgfqpoint{1.914817in}{2.507653in}}%
\pgfpathlineto{\pgfqpoint{1.915408in}{2.425129in}}%
\pgfpathlineto{\pgfqpoint{1.915925in}{2.490124in}}%
\pgfpathlineto{\pgfqpoint{1.915580in}{2.236004in}}%
\pgfpathlineto{\pgfqpoint{1.916528in}{2.445627in}}%
\pgfpathlineto{\pgfqpoint{1.917340in}{2.482315in}}%
\pgfpathlineto{\pgfqpoint{1.916811in}{2.249225in}}%
\pgfpathlineto{\pgfqpoint{1.917476in}{2.282743in}}%
\pgfpathlineto{\pgfqpoint{1.917500in}{2.239273in}}%
\pgfpathlineto{\pgfqpoint{1.917968in}{2.469826in}}%
\pgfpathlineto{\pgfqpoint{1.918522in}{2.404096in}}%
\pgfpathlineto{\pgfqpoint{1.918571in}{2.470765in}}%
\pgfpathlineto{\pgfqpoint{1.918854in}{2.276338in}}%
\pgfpathlineto{\pgfqpoint{1.919334in}{2.306545in}}%
\pgfpathlineto{\pgfqpoint{1.919347in}{2.295819in}}%
\pgfpathlineto{\pgfqpoint{1.919814in}{2.443448in}}%
\pgfpathlineto{\pgfqpoint{1.920319in}{2.410668in}}%
\pgfpathlineto{\pgfqpoint{1.920417in}{2.449376in}}%
\pgfpathlineto{\pgfqpoint{1.920577in}{2.291305in}}%
\pgfpathlineto{\pgfqpoint{1.921131in}{2.322495in}}%
\pgfpathlineto{\pgfqpoint{1.921193in}{2.279155in}}%
\pgfpathlineto{\pgfqpoint{1.921464in}{2.433578in}}%
\pgfpathlineto{\pgfqpoint{1.922042in}{2.408309in}}%
\pgfpathlineto{\pgfqpoint{1.922694in}{2.456927in}}%
\pgfpathlineto{\pgfqpoint{1.923039in}{2.271887in}}%
\pgfpathlineto{\pgfqpoint{1.923125in}{2.372054in}}%
\pgfpathlineto{\pgfqpoint{1.923310in}{2.456049in}}%
\pgfpathlineto{\pgfqpoint{1.923642in}{2.272117in}}%
\pgfpathlineto{\pgfqpoint{1.924159in}{2.357281in}}%
\pgfpathlineto{\pgfqpoint{1.924245in}{2.271164in}}%
\pgfpathlineto{\pgfqpoint{1.924553in}{2.454389in}}%
\pgfpathlineto{\pgfqpoint{1.925156in}{2.436348in}}%
\pgfpathlineto{\pgfqpoint{1.926190in}{2.250148in}}%
\pgfpathlineto{\pgfqpoint{1.925882in}{2.488969in}}%
\pgfpathlineto{\pgfqpoint{1.926362in}{2.377502in}}%
\pgfpathlineto{\pgfqpoint{1.927076in}{2.465308in}}%
\pgfpathlineto{\pgfqpoint{1.927334in}{2.225379in}}%
\pgfpathlineto{\pgfqpoint{1.927470in}{2.379736in}}%
\pgfpathlineto{\pgfqpoint{1.928479in}{2.490043in}}%
\pgfpathlineto{\pgfqpoint{1.928307in}{2.260647in}}%
\pgfpathlineto{\pgfqpoint{1.928540in}{2.364750in}}%
\pgfpathlineto{\pgfqpoint{1.929205in}{2.246251in}}%
\pgfpathlineto{\pgfqpoint{1.928885in}{2.504246in}}%
\pgfpathlineto{\pgfqpoint{1.929660in}{2.345913in}}%
\pgfpathlineto{\pgfqpoint{1.930411in}{2.175022in}}%
\pgfpathlineto{\pgfqpoint{1.929931in}{2.515744in}}%
\pgfpathlineto{\pgfqpoint{1.930596in}{2.421151in}}%
\pgfpathlineto{\pgfqpoint{1.930768in}{2.591261in}}%
\pgfpathlineto{\pgfqpoint{1.931100in}{2.165925in}}%
\pgfpathlineto{\pgfqpoint{1.931630in}{2.354456in}}%
\pgfpathlineto{\pgfqpoint{1.932294in}{2.160100in}}%
\pgfpathlineto{\pgfqpoint{1.931999in}{2.511542in}}%
\pgfpathlineto{\pgfqpoint{1.932602in}{2.466438in}}%
\pgfpathlineto{\pgfqpoint{1.932713in}{2.604683in}}%
\pgfpathlineto{\pgfqpoint{1.932996in}{2.186677in}}%
\pgfpathlineto{\pgfqpoint{1.933697in}{2.411922in}}%
\pgfpathlineto{\pgfqpoint{1.934153in}{2.287597in}}%
\pgfpathlineto{\pgfqpoint{1.934620in}{2.486006in}}%
\pgfpathlineto{\pgfqpoint{1.934707in}{2.474281in}}%
\pgfpathlineto{\pgfqpoint{1.934719in}{2.475064in}}%
\pgfpathlineto{\pgfqpoint{1.934780in}{2.416463in}}%
\pgfpathlineto{\pgfqpoint{1.936048in}{2.212839in}}%
\pgfpathlineto{\pgfqpoint{1.935568in}{2.520928in}}%
\pgfpathlineto{\pgfqpoint{1.936073in}{2.251639in}}%
\pgfpathlineto{\pgfqpoint{1.936368in}{2.539145in}}%
\pgfpathlineto{\pgfqpoint{1.936737in}{2.222123in}}%
\pgfpathlineto{\pgfqpoint{1.937193in}{2.306784in}}%
\pgfpathlineto{\pgfqpoint{1.937267in}{2.281432in}}%
\pgfpathlineto{\pgfqpoint{1.937476in}{2.478795in}}%
\pgfpathlineto{\pgfqpoint{1.938276in}{2.624043in}}%
\pgfpathlineto{\pgfqpoint{1.937894in}{2.184298in}}%
\pgfpathlineto{\pgfqpoint{1.938510in}{2.336748in}}%
\pgfpathlineto{\pgfqpoint{1.938645in}{2.197110in}}%
\pgfpathlineto{\pgfqpoint{1.938965in}{2.484314in}}%
\pgfpathlineto{\pgfqpoint{1.939593in}{2.360166in}}%
\pgfpathlineto{\pgfqpoint{1.940196in}{2.488713in}}%
\pgfpathlineto{\pgfqpoint{1.940479in}{2.279068in}}%
\pgfpathlineto{\pgfqpoint{1.940651in}{2.330704in}}%
\pgfpathlineto{\pgfqpoint{1.941463in}{2.190196in}}%
\pgfpathlineto{\pgfqpoint{1.941156in}{2.576824in}}%
\pgfpathlineto{\pgfqpoint{1.941734in}{2.379482in}}%
\pgfpathlineto{\pgfqpoint{1.941907in}{2.549664in}}%
\pgfpathlineto{\pgfqpoint{1.942337in}{2.206597in}}%
\pgfpathlineto{\pgfqpoint{1.942793in}{2.291869in}}%
\pgfpathlineto{\pgfqpoint{1.943851in}{2.611839in}}%
\pgfpathlineto{\pgfqpoint{1.943470in}{2.167365in}}%
\pgfpathlineto{\pgfqpoint{1.944036in}{2.381889in}}%
\pgfpathlineto{\pgfqpoint{1.944233in}{2.155529in}}%
\pgfpathlineto{\pgfqpoint{1.944885in}{2.500058in}}%
\pgfpathlineto{\pgfqpoint{1.945168in}{2.312491in}}%
\pgfpathlineto{\pgfqpoint{1.945783in}{2.526317in}}%
\pgfpathlineto{\pgfqpoint{1.945340in}{2.284166in}}%
\pgfpathlineto{\pgfqpoint{1.946251in}{2.288764in}}%
\pgfpathlineto{\pgfqpoint{1.946436in}{2.227464in}}%
\pgfpathlineto{\pgfqpoint{1.946719in}{2.606847in}}%
\pgfpathlineto{\pgfqpoint{1.946731in}{2.627854in}}%
\pgfpathlineto{\pgfqpoint{1.947100in}{2.159458in}}%
\pgfpathlineto{\pgfqpoint{1.947703in}{2.404033in}}%
\pgfpathlineto{\pgfqpoint{1.948343in}{2.197417in}}%
\pgfpathlineto{\pgfqpoint{1.948737in}{2.577569in}}%
\pgfpathlineto{\pgfqpoint{1.949057in}{2.136196in}}%
\pgfpathlineto{\pgfqpoint{1.949439in}{2.596241in}}%
\pgfpathlineto{\pgfqpoint{1.950226in}{2.355655in}}%
\pgfpathlineto{\pgfqpoint{1.950473in}{2.497107in}}%
\pgfpathlineto{\pgfqpoint{1.950743in}{2.238585in}}%
\pgfpathlineto{\pgfqpoint{1.951396in}{2.465171in}}%
\pgfpathlineto{\pgfqpoint{1.952343in}{2.623367in}}%
\pgfpathlineto{\pgfqpoint{1.952700in}{2.149559in}}%
\pgfpathlineto{\pgfqpoint{1.953070in}{2.552134in}}%
\pgfpathlineto{\pgfqpoint{1.953906in}{2.270886in}}%
\pgfpathlineto{\pgfqpoint{1.954670in}{2.177121in}}%
\pgfpathlineto{\pgfqpoint{1.954300in}{2.555491in}}%
\pgfpathlineto{\pgfqpoint{1.954916in}{2.419065in}}%
\pgfpathlineto{\pgfqpoint{1.955039in}{2.599254in}}%
\pgfpathlineto{\pgfqpoint{1.955371in}{2.209865in}}%
\pgfpathlineto{\pgfqpoint{1.956048in}{2.467059in}}%
\pgfpathlineto{\pgfqpoint{1.956097in}{2.479606in}}%
\pgfpathlineto{\pgfqpoint{1.956183in}{2.375985in}}%
\pgfpathlineto{\pgfqpoint{1.956343in}{2.209961in}}%
\pgfpathlineto{\pgfqpoint{1.957106in}{2.466727in}}%
\pgfpathlineto{\pgfqpoint{1.957291in}{2.362964in}}%
\pgfpathlineto{\pgfqpoint{1.957931in}{2.590246in}}%
\pgfpathlineto{\pgfqpoint{1.957599in}{2.191151in}}%
\pgfpathlineto{\pgfqpoint{1.958276in}{2.202072in}}%
\pgfpathlineto{\pgfqpoint{1.958313in}{2.158896in}}%
\pgfpathlineto{\pgfqpoint{1.958694in}{2.494553in}}%
\pgfpathlineto{\pgfqpoint{1.959260in}{2.464239in}}%
\pgfpathlineto{\pgfqpoint{1.959888in}{2.559213in}}%
\pgfpathlineto{\pgfqpoint{1.959568in}{2.215628in}}%
\pgfpathlineto{\pgfqpoint{1.960245in}{2.304118in}}%
\pgfpathlineto{\pgfqpoint{1.960282in}{2.233460in}}%
\pgfpathlineto{\pgfqpoint{1.960602in}{2.516546in}}%
\pgfpathlineto{\pgfqpoint{1.961316in}{2.403598in}}%
\pgfpathlineto{\pgfqpoint{1.961956in}{2.296235in}}%
\pgfpathlineto{\pgfqpoint{1.961710in}{2.466452in}}%
\pgfpathlineto{\pgfqpoint{1.962510in}{2.345276in}}%
\pgfpathlineto{\pgfqpoint{1.963482in}{2.540529in}}%
\pgfpathlineto{\pgfqpoint{1.963186in}{2.216327in}}%
\pgfpathlineto{\pgfqpoint{1.963666in}{2.390288in}}%
\pgfpathlineto{\pgfqpoint{1.963913in}{2.180004in}}%
\pgfpathlineto{\pgfqpoint{1.964159in}{2.479608in}}%
\pgfpathlineto{\pgfqpoint{1.964750in}{2.420789in}}%
\pgfpathlineto{\pgfqpoint{1.965488in}{2.519761in}}%
\pgfpathlineto{\pgfqpoint{1.965143in}{2.246910in}}%
\pgfpathlineto{\pgfqpoint{1.965722in}{2.342720in}}%
\pgfpathlineto{\pgfqpoint{1.965845in}{2.240882in}}%
\pgfpathlineto{\pgfqpoint{1.966386in}{2.469934in}}%
\pgfpathlineto{\pgfqpoint{1.966842in}{2.310239in}}%
\pgfpathlineto{\pgfqpoint{1.966854in}{2.305653in}}%
\pgfpathlineto{\pgfqpoint{1.967248in}{2.467592in}}%
\pgfpathlineto{\pgfqpoint{1.967580in}{2.422985in}}%
\pgfpathlineto{\pgfqpoint{1.968294in}{2.498269in}}%
\pgfpathlineto{\pgfqpoint{1.968639in}{2.267575in}}%
\pgfpathlineto{\pgfqpoint{1.968676in}{2.233651in}}%
\pgfpathlineto{\pgfqpoint{1.969082in}{2.478378in}}%
\pgfpathlineto{\pgfqpoint{1.969660in}{2.355751in}}%
\pgfpathlineto{\pgfqpoint{1.970140in}{2.472160in}}%
\pgfpathlineto{\pgfqpoint{1.970596in}{2.279205in}}%
\pgfpathlineto{\pgfqpoint{1.970743in}{2.334244in}}%
\pgfpathlineto{\pgfqpoint{1.971703in}{2.274135in}}%
\pgfpathlineto{\pgfqpoint{1.971322in}{2.486358in}}%
\pgfpathlineto{\pgfqpoint{1.971826in}{2.375366in}}%
\pgfpathlineto{\pgfqpoint{1.972011in}{2.466596in}}%
\pgfpathlineto{\pgfqpoint{1.972393in}{2.261681in}}%
\pgfpathlineto{\pgfqpoint{1.972946in}{2.411788in}}%
\pgfpathlineto{\pgfqpoint{1.973673in}{2.244032in}}%
\pgfpathlineto{\pgfqpoint{1.973217in}{2.512798in}}%
\pgfpathlineto{\pgfqpoint{1.974042in}{2.466263in}}%
\pgfpathlineto{\pgfqpoint{1.974140in}{2.573024in}}%
\pgfpathlineto{\pgfqpoint{1.974362in}{2.259142in}}%
\pgfpathlineto{\pgfqpoint{1.974916in}{2.391458in}}%
\pgfpathlineto{\pgfqpoint{1.975876in}{2.256474in}}%
\pgfpathlineto{\pgfqpoint{1.975925in}{2.549611in}}%
\pgfpathlineto{\pgfqpoint{1.976306in}{2.233072in}}%
\pgfpathlineto{\pgfqpoint{1.977143in}{2.404846in}}%
\pgfpathlineto{\pgfqpoint{1.977882in}{2.533657in}}%
\pgfpathlineto{\pgfqpoint{1.978091in}{2.217887in}}%
\pgfpathlineto{\pgfqpoint{1.978226in}{2.350459in}}%
\pgfpathlineto{\pgfqpoint{1.979285in}{2.195944in}}%
\pgfpathlineto{\pgfqpoint{1.978300in}{2.560185in}}%
\pgfpathlineto{\pgfqpoint{1.979359in}{2.204554in}}%
\pgfpathlineto{\pgfqpoint{1.979654in}{2.558575in}}%
\pgfpathlineto{\pgfqpoint{1.980023in}{2.159964in}}%
\pgfpathlineto{\pgfqpoint{1.980516in}{2.377700in}}%
\pgfpathlineto{\pgfqpoint{1.981057in}{2.247005in}}%
\pgfpathlineto{\pgfqpoint{1.981586in}{2.511976in}}%
\pgfpathlineto{\pgfqpoint{1.982559in}{2.558908in}}%
\pgfpathlineto{\pgfqpoint{1.982239in}{2.226365in}}%
\pgfpathlineto{\pgfqpoint{1.982620in}{2.414260in}}%
\pgfpathlineto{\pgfqpoint{1.982916in}{2.237949in}}%
\pgfpathlineto{\pgfqpoint{1.983285in}{2.541300in}}%
\pgfpathlineto{\pgfqpoint{1.983691in}{2.450277in}}%
\pgfpathlineto{\pgfqpoint{1.984528in}{2.568444in}}%
\pgfpathlineto{\pgfqpoint{1.984171in}{2.191792in}}%
\pgfpathlineto{\pgfqpoint{1.984663in}{2.388744in}}%
\pgfpathlineto{\pgfqpoint{1.984762in}{2.225909in}}%
\pgfpathlineto{\pgfqpoint{1.985340in}{2.515259in}}%
\pgfpathlineto{\pgfqpoint{1.985746in}{2.464374in}}%
\pgfpathlineto{\pgfqpoint{1.986189in}{2.554465in}}%
\pgfpathlineto{\pgfqpoint{1.985943in}{2.256134in}}%
\pgfpathlineto{\pgfqpoint{1.986374in}{2.298890in}}%
\pgfpathlineto{\pgfqpoint{1.986399in}{2.265815in}}%
\pgfpathlineto{\pgfqpoint{1.986596in}{2.526604in}}%
\pgfpathlineto{\pgfqpoint{1.987408in}{2.432446in}}%
\pgfpathlineto{\pgfqpoint{1.988269in}{2.550428in}}%
\pgfpathlineto{\pgfqpoint{1.987728in}{2.262623in}}%
\pgfpathlineto{\pgfqpoint{1.988479in}{2.276151in}}%
\pgfpathlineto{\pgfqpoint{1.989082in}{2.512310in}}%
\pgfpathlineto{\pgfqpoint{1.989352in}{2.244015in}}%
\pgfpathlineto{\pgfqpoint{1.989685in}{2.316440in}}%
\pgfpathlineto{\pgfqpoint{1.990534in}{2.202754in}}%
\pgfpathlineto{\pgfqpoint{1.990325in}{2.529211in}}%
\pgfpathlineto{\pgfqpoint{1.990719in}{2.405215in}}%
\pgfpathlineto{\pgfqpoint{1.991162in}{2.517763in}}%
\pgfpathlineto{\pgfqpoint{1.991211in}{2.255593in}}%
\pgfpathlineto{\pgfqpoint{1.991802in}{2.320443in}}%
\pgfpathlineto{\pgfqpoint{1.992146in}{2.305917in}}%
\pgfpathlineto{\pgfqpoint{1.991974in}{2.506213in}}%
\pgfpathlineto{\pgfqpoint{1.992356in}{2.425317in}}%
\pgfpathlineto{\pgfqpoint{1.993205in}{2.557644in}}%
\pgfpathlineto{\pgfqpoint{1.992454in}{2.249585in}}%
\pgfpathlineto{\pgfqpoint{1.993439in}{2.351314in}}%
\pgfpathlineto{\pgfqpoint{1.994239in}{2.207265in}}%
\pgfpathlineto{\pgfqpoint{1.994029in}{2.553951in}}%
\pgfpathlineto{\pgfqpoint{1.994522in}{2.329455in}}%
\pgfpathlineto{\pgfqpoint{1.994559in}{2.530992in}}%
\pgfpathlineto{\pgfqpoint{1.995432in}{2.213248in}}%
\pgfpathlineto{\pgfqpoint{1.995642in}{2.398415in}}%
\pgfpathlineto{\pgfqpoint{1.995789in}{2.563722in}}%
\pgfpathlineto{\pgfqpoint{1.995851in}{2.238738in}}%
\pgfpathlineto{\pgfqpoint{1.996749in}{2.425008in}}%
\pgfpathlineto{\pgfqpoint{1.997796in}{2.230699in}}%
\pgfpathlineto{\pgfqpoint{1.997439in}{2.532000in}}%
\pgfpathlineto{\pgfqpoint{1.997832in}{2.429770in}}%
\pgfpathlineto{\pgfqpoint{1.998152in}{2.555416in}}%
\pgfpathlineto{\pgfqpoint{1.998214in}{2.214753in}}%
\pgfpathlineto{\pgfqpoint{1.998928in}{2.368815in}}%
\pgfpathlineto{\pgfqpoint{1.999912in}{2.528571in}}%
\pgfpathlineto{\pgfqpoint{1.999556in}{2.270473in}}%
\pgfpathlineto{\pgfqpoint{1.999962in}{2.300943in}}%
\pgfpathlineto{\pgfqpoint{1.999986in}{2.243661in}}%
\pgfpathlineto{\pgfqpoint{2.000208in}{2.520976in}}%
\pgfpathlineto{\pgfqpoint{2.001020in}{2.456271in}}%
\pgfpathlineto{\pgfqpoint{2.001869in}{2.532531in}}%
\pgfpathlineto{\pgfqpoint{2.001919in}{2.227626in}}%
\pgfpathlineto{\pgfqpoint{2.002066in}{2.290751in}}%
\pgfpathlineto{\pgfqpoint{2.002288in}{2.580552in}}%
\pgfpathlineto{\pgfqpoint{2.002940in}{2.280110in}}%
\pgfpathlineto{\pgfqpoint{2.003223in}{2.435882in}}%
\pgfpathlineto{\pgfqpoint{2.004122in}{2.218855in}}%
\pgfpathlineto{\pgfqpoint{2.004036in}{2.502211in}}%
\pgfpathlineto{\pgfqpoint{2.004306in}{2.467128in}}%
\pgfpathlineto{\pgfqpoint{2.004331in}{2.569063in}}%
\pgfpathlineto{\pgfqpoint{2.004799in}{2.249607in}}%
\pgfpathlineto{\pgfqpoint{2.005389in}{2.309486in}}%
\pgfpathlineto{\pgfqpoint{2.006411in}{2.557366in}}%
\pgfpathlineto{\pgfqpoint{2.006054in}{2.243987in}}%
\pgfpathlineto{\pgfqpoint{2.006534in}{2.441407in}}%
\pgfpathlineto{\pgfqpoint{2.006657in}{2.267539in}}%
\pgfpathlineto{\pgfqpoint{2.007211in}{2.514998in}}%
\pgfpathlineto{\pgfqpoint{2.007629in}{2.473612in}}%
\pgfpathlineto{\pgfqpoint{2.008245in}{2.227892in}}%
\pgfpathlineto{\pgfqpoint{2.008466in}{2.551300in}}%
\pgfpathlineto{\pgfqpoint{2.008836in}{2.353252in}}%
\pgfpathlineto{\pgfqpoint{2.009389in}{2.501747in}}%
\pgfpathlineto{\pgfqpoint{2.009451in}{2.259699in}}%
\pgfpathlineto{\pgfqpoint{2.009955in}{2.385821in}}%
\pgfpathlineto{\pgfqpoint{2.010940in}{2.509085in}}%
\pgfpathlineto{\pgfqpoint{2.010177in}{2.268074in}}%
\pgfpathlineto{\pgfqpoint{2.011051in}{2.404751in}}%
\pgfpathlineto{\pgfqpoint{2.011962in}{2.268077in}}%
\pgfpathlineto{\pgfqpoint{2.011752in}{2.517912in}}%
\pgfpathlineto{\pgfqpoint{2.012134in}{2.444922in}}%
\pgfpathlineto{\pgfqpoint{2.012183in}{2.520395in}}%
\pgfpathlineto{\pgfqpoint{2.013155in}{2.262494in}}%
\pgfpathlineto{\pgfqpoint{2.013192in}{2.308555in}}%
\pgfpathlineto{\pgfqpoint{2.013254in}{2.270539in}}%
\pgfpathlineto{\pgfqpoint{2.013402in}{2.515887in}}%
\pgfpathlineto{\pgfqpoint{2.014091in}{2.376279in}}%
\pgfpathlineto{\pgfqpoint{2.014214in}{2.514581in}}%
\pgfpathlineto{\pgfqpoint{2.014903in}{2.282804in}}%
\pgfpathlineto{\pgfqpoint{2.015186in}{2.366909in}}%
\pgfpathlineto{\pgfqpoint{2.016085in}{2.261838in}}%
\pgfpathlineto{\pgfqpoint{2.015875in}{2.559774in}}%
\pgfpathlineto{\pgfqpoint{2.016257in}{2.416626in}}%
\pgfpathlineto{\pgfqpoint{2.017131in}{2.474952in}}%
\pgfpathlineto{\pgfqpoint{2.017279in}{2.253179in}}%
\pgfpathlineto{\pgfqpoint{2.017328in}{2.363246in}}%
\pgfpathlineto{\pgfqpoint{2.017365in}{2.283142in}}%
\pgfpathlineto{\pgfqpoint{2.017635in}{2.505368in}}%
\pgfpathlineto{\pgfqpoint{2.018325in}{2.452421in}}%
\pgfpathlineto{\pgfqpoint{2.018349in}{2.526657in}}%
\pgfpathlineto{\pgfqpoint{2.019039in}{2.281408in}}%
\pgfpathlineto{\pgfqpoint{2.019408in}{2.364576in}}%
\pgfpathlineto{\pgfqpoint{2.020060in}{2.268836in}}%
\pgfpathlineto{\pgfqpoint{2.020011in}{2.517328in}}%
\pgfpathlineto{\pgfqpoint{2.020380in}{2.432863in}}%
\pgfpathlineto{\pgfqpoint{2.021205in}{2.490574in}}%
\pgfpathlineto{\pgfqpoint{2.020663in}{2.254784in}}%
\pgfpathlineto{\pgfqpoint{2.021402in}{2.277507in}}%
\pgfpathlineto{\pgfqpoint{2.021845in}{2.244074in}}%
\pgfpathlineto{\pgfqpoint{2.022054in}{2.509440in}}%
\pgfpathlineto{\pgfqpoint{2.022337in}{2.337516in}}%
\pgfpathlineto{\pgfqpoint{2.022485in}{2.510163in}}%
\pgfpathlineto{\pgfqpoint{2.023039in}{2.264515in}}%
\pgfpathlineto{\pgfqpoint{2.023445in}{2.354975in}}%
\pgfpathlineto{\pgfqpoint{2.024368in}{2.262096in}}%
\pgfpathlineto{\pgfqpoint{2.024134in}{2.500341in}}%
\pgfpathlineto{\pgfqpoint{2.024503in}{2.440778in}}%
\pgfpathlineto{\pgfqpoint{2.025340in}{2.512047in}}%
\pgfpathlineto{\pgfqpoint{2.024786in}{2.269571in}}%
\pgfpathlineto{\pgfqpoint{2.025525in}{2.313516in}}%
\pgfpathlineto{\pgfqpoint{2.025549in}{2.290493in}}%
\pgfpathlineto{\pgfqpoint{2.026189in}{2.480971in}}%
\pgfpathlineto{\pgfqpoint{2.026583in}{2.390694in}}%
\pgfpathlineto{\pgfqpoint{2.027100in}{2.502551in}}%
\pgfpathlineto{\pgfqpoint{2.027162in}{2.267457in}}%
\pgfpathlineto{\pgfqpoint{2.027703in}{2.419277in}}%
\pgfpathlineto{\pgfqpoint{2.028232in}{2.485087in}}%
\pgfpathlineto{\pgfqpoint{2.028491in}{2.267480in}}%
\pgfpathlineto{\pgfqpoint{2.028688in}{2.330219in}}%
\pgfpathlineto{\pgfqpoint{2.029315in}{2.290886in}}%
\pgfpathlineto{\pgfqpoint{2.029475in}{2.491721in}}%
\pgfpathlineto{\pgfqpoint{2.029758in}{2.338460in}}%
\pgfpathlineto{\pgfqpoint{2.030312in}{2.458459in}}%
\pgfpathlineto{\pgfqpoint{2.030460in}{2.283302in}}%
\pgfpathlineto{\pgfqpoint{2.030854in}{2.320611in}}%
\pgfpathlineto{\pgfqpoint{2.031297in}{2.269150in}}%
\pgfpathlineto{\pgfqpoint{2.031506in}{2.455337in}}%
\pgfpathlineto{\pgfqpoint{2.031912in}{2.424325in}}%
\pgfpathlineto{\pgfqpoint{2.031937in}{2.478196in}}%
\pgfpathlineto{\pgfqpoint{2.032638in}{2.280094in}}%
\pgfpathlineto{\pgfqpoint{2.033008in}{2.394960in}}%
\pgfpathlineto{\pgfqpoint{2.033352in}{2.288297in}}%
\pgfpathlineto{\pgfqpoint{2.033586in}{2.474866in}}%
\pgfpathlineto{\pgfqpoint{2.034103in}{2.401655in}}%
\pgfpathlineto{\pgfqpoint{2.034805in}{2.487301in}}%
\pgfpathlineto{\pgfqpoint{2.034263in}{2.281354in}}%
\pgfpathlineto{\pgfqpoint{2.034965in}{2.373534in}}%
\pgfpathlineto{\pgfqpoint{2.035002in}{2.272510in}}%
\pgfpathlineto{\pgfqpoint{2.035642in}{2.512787in}}%
\pgfpathlineto{\pgfqpoint{2.036048in}{2.424206in}}%
\pgfpathlineto{\pgfqpoint{2.036577in}{2.477991in}}%
\pgfpathlineto{\pgfqpoint{2.037057in}{2.288022in}}%
\pgfpathlineto{\pgfqpoint{2.037131in}{2.365744in}}%
\pgfpathlineto{\pgfqpoint{2.037217in}{2.310470in}}%
\pgfpathlineto{\pgfqpoint{2.037254in}{2.399431in}}%
\pgfpathlineto{\pgfqpoint{2.037722in}{2.476120in}}%
\pgfpathlineto{\pgfqpoint{2.037968in}{2.279855in}}%
\pgfpathlineto{\pgfqpoint{2.038349in}{2.410082in}}%
\pgfpathlineto{\pgfqpoint{2.039137in}{2.288163in}}%
\pgfpathlineto{\pgfqpoint{2.038940in}{2.487946in}}%
\pgfpathlineto{\pgfqpoint{2.039445in}{2.402714in}}%
\pgfpathlineto{\pgfqpoint{2.039777in}{2.461623in}}%
\pgfpathlineto{\pgfqpoint{2.039937in}{2.302845in}}%
\pgfpathlineto{\pgfqpoint{2.040565in}{2.441516in}}%
\pgfpathlineto{\pgfqpoint{2.041402in}{2.474146in}}%
\pgfpathlineto{\pgfqpoint{2.040762in}{2.253345in}}%
\pgfpathlineto{\pgfqpoint{2.041562in}{2.355966in}}%
\pgfpathlineto{\pgfqpoint{2.042091in}{2.285208in}}%
\pgfpathlineto{\pgfqpoint{2.041820in}{2.481800in}}%
\pgfpathlineto{\pgfqpoint{2.042645in}{2.437849in}}%
\pgfpathlineto{\pgfqpoint{2.042817in}{2.288386in}}%
\pgfpathlineto{\pgfqpoint{2.043063in}{2.532698in}}%
\pgfpathlineto{\pgfqpoint{2.043826in}{2.376718in}}%
\pgfpathlineto{\pgfqpoint{2.043900in}{2.460582in}}%
\pgfpathlineto{\pgfqpoint{2.044060in}{2.307954in}}%
\pgfpathlineto{\pgfqpoint{2.044429in}{2.375191in}}%
\pgfpathlineto{\pgfqpoint{2.044466in}{2.262858in}}%
\pgfpathlineto{\pgfqpoint{2.045106in}{2.480804in}}%
\pgfpathlineto{\pgfqpoint{2.045512in}{2.444732in}}%
\pgfpathlineto{\pgfqpoint{2.045537in}{2.490561in}}%
\pgfpathlineto{\pgfqpoint{2.046522in}{2.285060in}}%
\pgfpathlineto{\pgfqpoint{2.046595in}{2.400789in}}%
\pgfpathlineto{\pgfqpoint{2.047248in}{2.289184in}}%
\pgfpathlineto{\pgfqpoint{2.047186in}{2.506814in}}%
\pgfpathlineto{\pgfqpoint{2.047691in}{2.427033in}}%
\pgfpathlineto{\pgfqpoint{2.048109in}{2.462813in}}%
\pgfpathlineto{\pgfqpoint{2.047851in}{2.294672in}}%
\pgfpathlineto{\pgfqpoint{2.048577in}{2.298038in}}%
\pgfpathlineto{\pgfqpoint{2.048589in}{2.265075in}}%
\pgfpathlineto{\pgfqpoint{2.048823in}{2.488000in}}%
\pgfpathlineto{\pgfqpoint{2.049635in}{2.437420in}}%
\pgfpathlineto{\pgfqpoint{2.049660in}{2.488888in}}%
\pgfpathlineto{\pgfqpoint{2.050226in}{2.293908in}}%
\pgfpathlineto{\pgfqpoint{2.050718in}{2.382375in}}%
\pgfpathlineto{\pgfqpoint{2.050952in}{2.279799in}}%
\pgfpathlineto{\pgfqpoint{2.050891in}{2.485394in}}%
\pgfpathlineto{\pgfqpoint{2.051691in}{2.447759in}}%
\pgfpathlineto{\pgfqpoint{2.052528in}{2.505607in}}%
\pgfpathlineto{\pgfqpoint{2.051974in}{2.273640in}}%
\pgfpathlineto{\pgfqpoint{2.052700in}{2.323125in}}%
\pgfpathlineto{\pgfqpoint{2.053155in}{2.289156in}}%
\pgfpathlineto{\pgfqpoint{2.053365in}{2.484685in}}%
\pgfpathlineto{\pgfqpoint{2.053758in}{2.424247in}}%
\pgfpathlineto{\pgfqpoint{2.054288in}{2.501322in}}%
\pgfpathlineto{\pgfqpoint{2.053931in}{2.287684in}}%
\pgfpathlineto{\pgfqpoint{2.054768in}{2.289362in}}%
\pgfpathlineto{\pgfqpoint{2.055826in}{2.477835in}}%
\pgfpathlineto{\pgfqpoint{2.055678in}{2.288982in}}%
\pgfpathlineto{\pgfqpoint{2.055974in}{2.344719in}}%
\pgfpathlineto{\pgfqpoint{2.056848in}{2.287741in}}%
\pgfpathlineto{\pgfqpoint{2.056651in}{2.506464in}}%
\pgfpathlineto{\pgfqpoint{2.057032in}{2.409374in}}%
\pgfpathlineto{\pgfqpoint{2.057057in}{2.461557in}}%
\pgfpathlineto{\pgfqpoint{2.057660in}{2.286424in}}%
\pgfpathlineto{\pgfqpoint{2.058115in}{2.358045in}}%
\pgfpathlineto{\pgfqpoint{2.058485in}{2.286450in}}%
\pgfpathlineto{\pgfqpoint{2.059112in}{2.490650in}}%
\pgfpathlineto{\pgfqpoint{2.059125in}{2.497515in}}%
\pgfpathlineto{\pgfqpoint{2.059728in}{2.314094in}}%
\pgfpathlineto{\pgfqpoint{2.059986in}{2.372885in}}%
\pgfpathlineto{\pgfqpoint{2.060614in}{2.298788in}}%
\pgfpathlineto{\pgfqpoint{2.060774in}{2.496700in}}%
\pgfpathlineto{\pgfqpoint{2.061094in}{2.379130in}}%
\pgfpathlineto{\pgfqpoint{2.061795in}{2.302576in}}%
\pgfpathlineto{\pgfqpoint{2.061992in}{2.465165in}}%
\pgfpathlineto{\pgfqpoint{2.062214in}{2.352197in}}%
\pgfpathlineto{\pgfqpoint{2.062817in}{2.483573in}}%
\pgfpathlineto{\pgfqpoint{2.062608in}{2.297610in}}%
\pgfpathlineto{\pgfqpoint{2.063346in}{2.396809in}}%
\pgfpathlineto{\pgfqpoint{2.063937in}{2.319154in}}%
\pgfpathlineto{\pgfqpoint{2.064060in}{2.463347in}}%
\pgfpathlineto{\pgfqpoint{2.064441in}{2.389349in}}%
\pgfpathlineto{\pgfqpoint{2.064897in}{2.478709in}}%
\pgfpathlineto{\pgfqpoint{2.065143in}{2.312565in}}%
\pgfpathlineto{\pgfqpoint{2.065537in}{2.364234in}}%
\pgfpathlineto{\pgfqpoint{2.065561in}{2.296339in}}%
\pgfpathlineto{\pgfqpoint{2.066115in}{2.472309in}}%
\pgfpathlineto{\pgfqpoint{2.066632in}{2.399726in}}%
\pgfpathlineto{\pgfqpoint{2.066940in}{2.474963in}}%
\pgfpathlineto{\pgfqpoint{2.066743in}{2.297260in}}%
\pgfpathlineto{\pgfqpoint{2.067777in}{2.453651in}}%
\pgfpathlineto{\pgfqpoint{2.067937in}{2.304739in}}%
\pgfpathlineto{\pgfqpoint{2.067875in}{2.462383in}}%
\pgfpathlineto{\pgfqpoint{2.068958in}{2.361552in}}%
\pgfpathlineto{\pgfqpoint{2.069820in}{2.475640in}}%
\pgfpathlineto{\pgfqpoint{2.069266in}{2.307154in}}%
\pgfpathlineto{\pgfqpoint{2.070054in}{2.325750in}}%
\pgfpathlineto{\pgfqpoint{2.070238in}{2.494425in}}%
\pgfpathlineto{\pgfqpoint{2.070435in}{2.307440in}}%
\pgfpathlineto{\pgfqpoint{2.071211in}{2.376211in}}%
\pgfpathlineto{\pgfqpoint{2.071740in}{2.306830in}}%
\pgfpathlineto{\pgfqpoint{2.071998in}{2.468146in}}%
\pgfpathlineto{\pgfqpoint{2.072294in}{2.418784in}}%
\pgfpathlineto{\pgfqpoint{2.072491in}{2.331058in}}%
\pgfpathlineto{\pgfqpoint{2.072884in}{2.308042in}}%
\pgfpathlineto{\pgfqpoint{2.072700in}{2.458819in}}%
\pgfpathlineto{\pgfqpoint{2.073500in}{2.393468in}}%
\pgfpathlineto{\pgfqpoint{2.073943in}{2.491231in}}%
\pgfpathlineto{\pgfqpoint{2.074201in}{2.314774in}}%
\pgfpathlineto{\pgfqpoint{2.074558in}{2.319389in}}%
\pgfpathlineto{\pgfqpoint{2.074571in}{2.311933in}}%
\pgfpathlineto{\pgfqpoint{2.075186in}{2.449606in}}%
\pgfpathlineto{\pgfqpoint{2.075543in}{2.390261in}}%
\pgfpathlineto{\pgfqpoint{2.075998in}{2.460699in}}%
\pgfpathlineto{\pgfqpoint{2.076195in}{2.304826in}}%
\pgfpathlineto{\pgfqpoint{2.076614in}{2.330672in}}%
\pgfpathlineto{\pgfqpoint{2.077032in}{2.315451in}}%
\pgfpathlineto{\pgfqpoint{2.076823in}{2.480584in}}%
\pgfpathlineto{\pgfqpoint{2.077549in}{2.385125in}}%
\pgfpathlineto{\pgfqpoint{2.078066in}{2.474602in}}%
\pgfpathlineto{\pgfqpoint{2.077857in}{2.311272in}}%
\pgfpathlineto{\pgfqpoint{2.078632in}{2.362196in}}%
\pgfpathlineto{\pgfqpoint{2.078694in}{2.327003in}}%
\pgfpathlineto{\pgfqpoint{2.078854in}{2.411051in}}%
\pgfpathlineto{\pgfqpoint{2.079691in}{2.469497in}}%
\pgfpathlineto{\pgfqpoint{2.079149in}{2.299358in}}%
\pgfpathlineto{\pgfqpoint{2.079937in}{2.374882in}}%
\pgfpathlineto{\pgfqpoint{2.080318in}{2.312890in}}%
\pgfpathlineto{\pgfqpoint{2.080528in}{2.468774in}}%
\pgfpathlineto{\pgfqpoint{2.081020in}{2.391466in}}%
\pgfpathlineto{\pgfqpoint{2.081352in}{2.461138in}}%
\pgfpathlineto{\pgfqpoint{2.081512in}{2.322823in}}%
\pgfpathlineto{\pgfqpoint{2.082103in}{2.337048in}}%
\pgfpathlineto{\pgfqpoint{2.082177in}{2.462897in}}%
\pgfpathlineto{\pgfqpoint{2.082841in}{2.314218in}}%
\pgfpathlineto{\pgfqpoint{2.083248in}{2.359494in}}%
\pgfpathlineto{\pgfqpoint{2.083272in}{2.319501in}}%
\pgfpathlineto{\pgfqpoint{2.083814in}{2.489316in}}%
\pgfpathlineto{\pgfqpoint{2.084331in}{2.399158in}}%
\pgfpathlineto{\pgfqpoint{2.084638in}{2.456474in}}%
\pgfpathlineto{\pgfqpoint{2.084811in}{2.323347in}}%
\pgfpathlineto{\pgfqpoint{2.085377in}{2.368547in}}%
\pgfpathlineto{\pgfqpoint{2.085635in}{2.314208in}}%
\pgfpathlineto{\pgfqpoint{2.086288in}{2.456996in}}%
\pgfpathlineto{\pgfqpoint{2.086509in}{2.338414in}}%
\pgfpathlineto{\pgfqpoint{2.087518in}{2.465856in}}%
\pgfpathlineto{\pgfqpoint{2.087174in}{2.322021in}}%
\pgfpathlineto{\pgfqpoint{2.087654in}{2.399354in}}%
\pgfpathlineto{\pgfqpoint{2.087998in}{2.324503in}}%
\pgfpathlineto{\pgfqpoint{2.087937in}{2.481990in}}%
\pgfpathlineto{\pgfqpoint{2.088737in}{2.412395in}}%
\pgfpathlineto{\pgfqpoint{2.088761in}{2.470467in}}%
\pgfpathlineto{\pgfqpoint{2.089758in}{2.323091in}}%
\pgfpathlineto{\pgfqpoint{2.089820in}{2.375285in}}%
\pgfpathlineto{\pgfqpoint{2.090349in}{2.320467in}}%
\pgfpathlineto{\pgfqpoint{2.090398in}{2.453457in}}%
\pgfpathlineto{\pgfqpoint{2.090903in}{2.379848in}}%
\pgfpathlineto{\pgfqpoint{2.091641in}{2.471688in}}%
\pgfpathlineto{\pgfqpoint{2.091814in}{2.334074in}}%
\pgfpathlineto{\pgfqpoint{2.091998in}{2.360412in}}%
\pgfpathlineto{\pgfqpoint{2.092121in}{2.314324in}}%
\pgfpathlineto{\pgfqpoint{2.092060in}{2.465378in}}%
\pgfpathlineto{\pgfqpoint{2.092860in}{2.404447in}}%
\pgfpathlineto{\pgfqpoint{2.093278in}{2.439283in}}%
\pgfpathlineto{\pgfqpoint{2.093906in}{2.325294in}}%
\pgfpathlineto{\pgfqpoint{2.093943in}{2.377622in}}%
\pgfpathlineto{\pgfqpoint{2.094731in}{2.330441in}}%
\pgfpathlineto{\pgfqpoint{2.094521in}{2.453035in}}%
\pgfpathlineto{\pgfqpoint{2.095026in}{2.410374in}}%
\pgfpathlineto{\pgfqpoint{2.095764in}{2.450548in}}%
\pgfpathlineto{\pgfqpoint{2.095826in}{2.327633in}}%
\pgfpathlineto{\pgfqpoint{2.096072in}{2.365493in}}%
\pgfpathlineto{\pgfqpoint{2.096847in}{2.329840in}}%
\pgfpathlineto{\pgfqpoint{2.096171in}{2.447110in}}%
\pgfpathlineto{\pgfqpoint{2.096971in}{2.426402in}}%
\pgfpathlineto{\pgfqpoint{2.097401in}{2.456135in}}%
\pgfpathlineto{\pgfqpoint{2.097672in}{2.335564in}}%
\pgfpathlineto{\pgfqpoint{2.098004in}{2.374758in}}%
\pgfpathlineto{\pgfqpoint{2.098029in}{2.323241in}}%
\pgfpathlineto{\pgfqpoint{2.098226in}{2.462131in}}%
\pgfpathlineto{\pgfqpoint{2.099112in}{2.347274in}}%
\pgfpathlineto{\pgfqpoint{2.099875in}{2.449924in}}%
\pgfpathlineto{\pgfqpoint{2.099223in}{2.330805in}}%
\pgfpathlineto{\pgfqpoint{2.100207in}{2.347239in}}%
\pgfpathlineto{\pgfqpoint{2.100761in}{2.330683in}}%
\pgfpathlineto{\pgfqpoint{2.100281in}{2.439087in}}%
\pgfpathlineto{\pgfqpoint{2.101081in}{2.409217in}}%
\pgfpathlineto{\pgfqpoint{2.101524in}{2.456601in}}%
\pgfpathlineto{\pgfqpoint{2.101574in}{2.310138in}}%
\pgfpathlineto{\pgfqpoint{2.102152in}{2.331063in}}%
\pgfpathlineto{\pgfqpoint{2.102164in}{2.329975in}}%
\pgfpathlineto{\pgfqpoint{2.102324in}{2.408430in}}%
\pgfpathlineto{\pgfqpoint{2.102349in}{2.450052in}}%
\pgfpathlineto{\pgfqpoint{2.103346in}{2.324754in}}%
\pgfpathlineto{\pgfqpoint{2.103407in}{2.367410in}}%
\pgfpathlineto{\pgfqpoint{2.103937in}{2.323396in}}%
\pgfpathlineto{\pgfqpoint{2.103986in}{2.461700in}}%
\pgfpathlineto{\pgfqpoint{2.104491in}{2.392107in}}%
\pgfpathlineto{\pgfqpoint{2.105217in}{2.460437in}}%
\pgfpathlineto{\pgfqpoint{2.105266in}{2.333570in}}%
\pgfpathlineto{\pgfqpoint{2.106300in}{2.319572in}}%
\pgfpathlineto{\pgfqpoint{2.105635in}{2.470242in}}%
\pgfpathlineto{\pgfqpoint{2.106312in}{2.333283in}}%
\pgfpathlineto{\pgfqpoint{2.106866in}{2.461797in}}%
\pgfpathlineto{\pgfqpoint{2.107038in}{2.325241in}}%
\pgfpathlineto{\pgfqpoint{2.107432in}{2.393470in}}%
\pgfpathlineto{\pgfqpoint{2.107481in}{2.320109in}}%
\pgfpathlineto{\pgfqpoint{2.107691in}{2.474555in}}%
\pgfpathlineto{\pgfqpoint{2.108564in}{2.340426in}}%
\pgfpathlineto{\pgfqpoint{2.109746in}{2.462103in}}%
\pgfpathlineto{\pgfqpoint{2.109401in}{2.320895in}}%
\pgfpathlineto{\pgfqpoint{2.109783in}{2.407892in}}%
\pgfpathlineto{\pgfqpoint{2.109992in}{2.323755in}}%
\pgfpathlineto{\pgfqpoint{2.110558in}{2.461975in}}%
\pgfpathlineto{\pgfqpoint{2.110915in}{2.347163in}}%
\pgfpathlineto{\pgfqpoint{2.111038in}{2.318120in}}%
\pgfpathlineto{\pgfqpoint{2.110977in}{2.478338in}}%
\pgfpathlineto{\pgfqpoint{2.111777in}{2.396126in}}%
\pgfpathlineto{\pgfqpoint{2.111801in}{2.472357in}}%
\pgfpathlineto{\pgfqpoint{2.112798in}{2.301692in}}%
\pgfpathlineto{\pgfqpoint{2.112872in}{2.347315in}}%
\pgfpathlineto{\pgfqpoint{2.113450in}{2.468782in}}%
\pgfpathlineto{\pgfqpoint{2.113389in}{2.320815in}}%
\pgfpathlineto{\pgfqpoint{2.113783in}{2.344880in}}%
\pgfpathlineto{\pgfqpoint{2.114214in}{2.329792in}}%
\pgfpathlineto{\pgfqpoint{2.114681in}{2.482618in}}%
\pgfpathlineto{\pgfqpoint{2.114866in}{2.362482in}}%
\pgfpathlineto{\pgfqpoint{2.115100in}{2.481492in}}%
\pgfpathlineto{\pgfqpoint{2.115149in}{2.316585in}}%
\pgfpathlineto{\pgfqpoint{2.115961in}{2.368461in}}%
\pgfpathlineto{\pgfqpoint{2.116921in}{2.320862in}}%
\pgfpathlineto{\pgfqpoint{2.116737in}{2.458818in}}%
\pgfpathlineto{\pgfqpoint{2.117081in}{2.325648in}}%
\pgfpathlineto{\pgfqpoint{2.117561in}{2.480311in}}%
\pgfpathlineto{\pgfqpoint{2.117512in}{2.317686in}}%
\pgfpathlineto{\pgfqpoint{2.118287in}{2.399134in}}%
\pgfpathlineto{\pgfqpoint{2.118854in}{2.318805in}}%
\pgfpathlineto{\pgfqpoint{2.118804in}{2.472846in}}%
\pgfpathlineto{\pgfqpoint{2.119444in}{2.335967in}}%
\pgfpathlineto{\pgfqpoint{2.120429in}{2.476531in}}%
\pgfpathlineto{\pgfqpoint{2.119875in}{2.311596in}}%
\pgfpathlineto{\pgfqpoint{2.120589in}{2.371838in}}%
\pgfpathlineto{\pgfqpoint{2.121204in}{2.308944in}}%
\pgfpathlineto{\pgfqpoint{2.121266in}{2.483467in}}%
\pgfpathlineto{\pgfqpoint{2.121660in}{2.426074in}}%
\pgfpathlineto{\pgfqpoint{2.121672in}{2.465181in}}%
\pgfpathlineto{\pgfqpoint{2.122250in}{2.324260in}}%
\pgfpathlineto{\pgfqpoint{2.122743in}{2.344892in}}%
\pgfpathlineto{\pgfqpoint{2.123727in}{2.478385in}}%
\pgfpathlineto{\pgfqpoint{2.123666in}{2.315304in}}%
\pgfpathlineto{\pgfqpoint{2.123875in}{2.364835in}}%
\pgfpathlineto{\pgfqpoint{2.124601in}{2.310877in}}%
\pgfpathlineto{\pgfqpoint{2.124552in}{2.492646in}}%
\pgfpathlineto{\pgfqpoint{2.124934in}{2.416110in}}%
\pgfpathlineto{\pgfqpoint{2.125377in}{2.477127in}}%
\pgfpathlineto{\pgfqpoint{2.125426in}{2.317527in}}%
\pgfpathlineto{\pgfqpoint{2.126004in}{2.347995in}}%
\pgfpathlineto{\pgfqpoint{2.126964in}{2.313504in}}%
\pgfpathlineto{\pgfqpoint{2.126595in}{2.455265in}}%
\pgfpathlineto{\pgfqpoint{2.126977in}{2.368998in}}%
\pgfpathlineto{\pgfqpoint{2.127014in}{2.484540in}}%
\pgfpathlineto{\pgfqpoint{2.127370in}{2.303628in}}%
\pgfpathlineto{\pgfqpoint{2.128084in}{2.360167in}}%
\pgfpathlineto{\pgfqpoint{2.128601in}{2.319507in}}%
\pgfpathlineto{\pgfqpoint{2.128244in}{2.487519in}}%
\pgfpathlineto{\pgfqpoint{2.129044in}{2.441682in}}%
\pgfpathlineto{\pgfqpoint{2.129069in}{2.470710in}}%
\pgfpathlineto{\pgfqpoint{2.129327in}{2.314772in}}%
\pgfpathlineto{\pgfqpoint{2.130103in}{2.363203in}}%
\pgfpathlineto{\pgfqpoint{2.130152in}{2.320130in}}%
\pgfpathlineto{\pgfqpoint{2.130201in}{2.418048in}}%
\pgfpathlineto{\pgfqpoint{2.130275in}{2.415745in}}%
\pgfpathlineto{\pgfqpoint{2.130706in}{2.482828in}}%
\pgfpathlineto{\pgfqpoint{2.131075in}{2.310954in}}%
\pgfpathlineto{\pgfqpoint{2.131346in}{2.326714in}}%
\pgfpathlineto{\pgfqpoint{2.131543in}{2.463477in}}%
\pgfpathlineto{\pgfqpoint{2.131481in}{2.308276in}}%
\pgfpathlineto{\pgfqpoint{2.132404in}{2.333502in}}%
\pgfpathlineto{\pgfqpoint{2.133118in}{2.309515in}}%
\pgfpathlineto{\pgfqpoint{2.133180in}{2.472243in}}%
\pgfpathlineto{\pgfqpoint{2.133463in}{2.369937in}}%
\pgfpathlineto{\pgfqpoint{2.134004in}{2.482642in}}%
\pgfpathlineto{\pgfqpoint{2.133844in}{2.321252in}}%
\pgfpathlineto{\pgfqpoint{2.134570in}{2.366406in}}%
\pgfpathlineto{\pgfqpoint{2.135493in}{2.310559in}}%
\pgfpathlineto{\pgfqpoint{2.134829in}{2.481431in}}%
\pgfpathlineto{\pgfqpoint{2.135530in}{2.420445in}}%
\pgfpathlineto{\pgfqpoint{2.136466in}{2.477361in}}%
\pgfpathlineto{\pgfqpoint{2.136404in}{2.308741in}}%
\pgfpathlineto{\pgfqpoint{2.136601in}{2.391133in}}%
\pgfpathlineto{\pgfqpoint{2.136823in}{2.306525in}}%
\pgfpathlineto{\pgfqpoint{2.136872in}{2.475147in}}%
\pgfpathlineto{\pgfqpoint{2.137672in}{2.429658in}}%
\pgfpathlineto{\pgfqpoint{2.138115in}{2.487557in}}%
\pgfpathlineto{\pgfqpoint{2.138460in}{2.312407in}}%
\pgfpathlineto{\pgfqpoint{2.138743in}{2.340822in}}%
\pgfpathlineto{\pgfqpoint{2.139186in}{2.309089in}}%
\pgfpathlineto{\pgfqpoint{2.138940in}{2.479952in}}%
\pgfpathlineto{\pgfqpoint{2.139826in}{2.357411in}}%
\pgfpathlineto{\pgfqpoint{2.140158in}{2.492357in}}%
\pgfpathlineto{\pgfqpoint{2.140527in}{2.315973in}}%
\pgfpathlineto{\pgfqpoint{2.140909in}{2.342344in}}%
\pgfpathlineto{\pgfqpoint{2.140933in}{2.301665in}}%
\pgfpathlineto{\pgfqpoint{2.141807in}{2.484260in}}%
\pgfpathlineto{\pgfqpoint{2.142004in}{2.350693in}}%
\pgfpathlineto{\pgfqpoint{2.142620in}{2.483357in}}%
\pgfpathlineto{\pgfqpoint{2.142570in}{2.297946in}}%
\pgfpathlineto{\pgfqpoint{2.143161in}{2.390178in}}%
\pgfpathlineto{\pgfqpoint{2.143297in}{2.299077in}}%
\pgfpathlineto{\pgfqpoint{2.143444in}{2.483800in}}%
\pgfpathlineto{\pgfqpoint{2.144232in}{2.366852in}}%
\pgfpathlineto{\pgfqpoint{2.144269in}{2.497046in}}%
\pgfpathlineto{\pgfqpoint{2.144933in}{2.295844in}}%
\pgfpathlineto{\pgfqpoint{2.145327in}{2.336150in}}%
\pgfpathlineto{\pgfqpoint{2.145857in}{2.299477in}}%
\pgfpathlineto{\pgfqpoint{2.145906in}{2.483525in}}%
\pgfpathlineto{\pgfqpoint{2.146300in}{2.435223in}}%
\pgfpathlineto{\pgfqpoint{2.146730in}{2.479312in}}%
\pgfpathlineto{\pgfqpoint{2.146681in}{2.305668in}}%
\pgfpathlineto{\pgfqpoint{2.147370in}{2.344536in}}%
\pgfpathlineto{\pgfqpoint{2.148220in}{2.296023in}}%
\pgfpathlineto{\pgfqpoint{2.147555in}{2.487368in}}%
\pgfpathlineto{\pgfqpoint{2.148355in}{2.435507in}}%
\pgfpathlineto{\pgfqpoint{2.148380in}{2.480268in}}%
\pgfpathlineto{\pgfqpoint{2.149044in}{2.291174in}}%
\pgfpathlineto{\pgfqpoint{2.149426in}{2.346880in}}%
\pgfpathlineto{\pgfqpoint{2.150275in}{2.306453in}}%
\pgfpathlineto{\pgfqpoint{2.150029in}{2.484904in}}%
\pgfpathlineto{\pgfqpoint{2.150398in}{2.370049in}}%
\pgfpathlineto{\pgfqpoint{2.151247in}{2.486466in}}%
\pgfpathlineto{\pgfqpoint{2.150693in}{2.316760in}}%
\pgfpathlineto{\pgfqpoint{2.151493in}{2.329612in}}%
\pgfpathlineto{\pgfqpoint{2.152023in}{2.297465in}}%
\pgfpathlineto{\pgfqpoint{2.151666in}{2.486008in}}%
\pgfpathlineto{\pgfqpoint{2.152564in}{2.377315in}}%
\pgfpathlineto{\pgfqpoint{2.152897in}{2.498383in}}%
\pgfpathlineto{\pgfqpoint{2.152749in}{2.288774in}}%
\pgfpathlineto{\pgfqpoint{2.153647in}{2.348527in}}%
\pgfpathlineto{\pgfqpoint{2.154386in}{2.291938in}}%
\pgfpathlineto{\pgfqpoint{2.153721in}{2.495791in}}%
\pgfpathlineto{\pgfqpoint{2.154743in}{2.366073in}}%
\pgfpathlineto{\pgfqpoint{2.155210in}{2.304307in}}%
\pgfpathlineto{\pgfqpoint{2.155358in}{2.494390in}}%
\pgfpathlineto{\pgfqpoint{2.155666in}{2.412288in}}%
\pgfpathlineto{\pgfqpoint{2.156589in}{2.496914in}}%
\pgfpathlineto{\pgfqpoint{2.156133in}{2.304655in}}%
\pgfpathlineto{\pgfqpoint{2.156638in}{2.353710in}}%
\pgfpathlineto{\pgfqpoint{2.156860in}{2.290684in}}%
\pgfpathlineto{\pgfqpoint{2.157007in}{2.486089in}}%
\pgfpathlineto{\pgfqpoint{2.157696in}{2.358897in}}%
\pgfpathlineto{\pgfqpoint{2.157832in}{2.495706in}}%
\pgfpathlineto{\pgfqpoint{2.158496in}{2.296102in}}%
\pgfpathlineto{\pgfqpoint{2.158792in}{2.362094in}}%
\pgfpathlineto{\pgfqpoint{2.159727in}{2.300612in}}%
\pgfpathlineto{\pgfqpoint{2.159469in}{2.488982in}}%
\pgfpathlineto{\pgfqpoint{2.159850in}{2.378571in}}%
\pgfpathlineto{\pgfqpoint{2.160700in}{2.494427in}}%
\pgfpathlineto{\pgfqpoint{2.160552in}{2.308005in}}%
\pgfpathlineto{\pgfqpoint{2.160946in}{2.320131in}}%
\pgfpathlineto{\pgfqpoint{2.160958in}{2.299133in}}%
\pgfpathlineto{\pgfqpoint{2.161930in}{2.491320in}}%
\pgfpathlineto{\pgfqpoint{2.162016in}{2.375516in}}%
\pgfpathlineto{\pgfqpoint{2.162349in}{2.492683in}}%
\pgfpathlineto{\pgfqpoint{2.162189in}{2.295535in}}%
\pgfpathlineto{\pgfqpoint{2.163100in}{2.338416in}}%
\pgfpathlineto{\pgfqpoint{2.163838in}{2.290785in}}%
\pgfpathlineto{\pgfqpoint{2.163580in}{2.488801in}}%
\pgfpathlineto{\pgfqpoint{2.164170in}{2.364847in}}%
\pgfpathlineto{\pgfqpoint{2.164810in}{2.490627in}}%
\pgfpathlineto{\pgfqpoint{2.165069in}{2.301337in}}%
\pgfpathlineto{\pgfqpoint{2.165253in}{2.371027in}}%
\pgfpathlineto{\pgfqpoint{2.166300in}{2.295281in}}%
\pgfpathlineto{\pgfqpoint{2.166041in}{2.492312in}}%
\pgfpathlineto{\pgfqpoint{2.166336in}{2.392661in}}%
\pgfpathlineto{\pgfqpoint{2.166460in}{2.493987in}}%
\pgfpathlineto{\pgfqpoint{2.167124in}{2.304358in}}%
\pgfpathlineto{\pgfqpoint{2.167420in}{2.363877in}}%
\pgfpathlineto{\pgfqpoint{2.167949in}{2.292583in}}%
\pgfpathlineto{\pgfqpoint{2.167690in}{2.483128in}}%
\pgfpathlineto{\pgfqpoint{2.168490in}{2.446301in}}%
\pgfpathlineto{\pgfqpoint{2.169340in}{2.483552in}}%
\pgfpathlineto{\pgfqpoint{2.168773in}{2.305870in}}%
\pgfpathlineto{\pgfqpoint{2.169561in}{2.354030in}}%
\pgfpathlineto{\pgfqpoint{2.170410in}{2.301155in}}%
\pgfpathlineto{\pgfqpoint{2.170152in}{2.489859in}}%
\pgfpathlineto{\pgfqpoint{2.170644in}{2.368831in}}%
\pgfpathlineto{\pgfqpoint{2.171395in}{2.485572in}}%
\pgfpathlineto{\pgfqpoint{2.171641in}{2.295297in}}%
\pgfpathlineto{\pgfqpoint{2.171727in}{2.342895in}}%
\pgfpathlineto{\pgfqpoint{2.172466in}{2.307663in}}%
\pgfpathlineto{\pgfqpoint{2.172626in}{2.489827in}}%
\pgfpathlineto{\pgfqpoint{2.172835in}{2.345899in}}%
\pgfpathlineto{\pgfqpoint{2.173290in}{2.297260in}}%
\pgfpathlineto{\pgfqpoint{2.173032in}{2.487023in}}%
\pgfpathlineto{\pgfqpoint{2.173733in}{2.403207in}}%
\pgfpathlineto{\pgfqpoint{2.174681in}{2.490251in}}%
\pgfpathlineto{\pgfqpoint{2.174521in}{2.302309in}}%
\pgfpathlineto{\pgfqpoint{2.174816in}{2.373928in}}%
\pgfpathlineto{\pgfqpoint{2.175764in}{2.296294in}}%
\pgfpathlineto{\pgfqpoint{2.175506in}{2.491193in}}%
\pgfpathlineto{\pgfqpoint{2.175887in}{2.426215in}}%
\pgfpathlineto{\pgfqpoint{2.176736in}{2.500495in}}%
\pgfpathlineto{\pgfqpoint{2.176576in}{2.303684in}}%
\pgfpathlineto{\pgfqpoint{2.176958in}{2.344838in}}%
\pgfpathlineto{\pgfqpoint{2.177401in}{2.295198in}}%
\pgfpathlineto{\pgfqpoint{2.177967in}{2.489852in}}%
\pgfpathlineto{\pgfqpoint{2.178041in}{2.354505in}}%
\pgfpathlineto{\pgfqpoint{2.178792in}{2.493283in}}%
\pgfpathlineto{\pgfqpoint{2.178226in}{2.303880in}}%
\pgfpathlineto{\pgfqpoint{2.179136in}{2.327162in}}%
\pgfpathlineto{\pgfqpoint{2.179875in}{2.294824in}}%
\pgfpathlineto{\pgfqpoint{2.179616in}{2.499688in}}%
\pgfpathlineto{\pgfqpoint{2.179998in}{2.415657in}}%
\pgfpathlineto{\pgfqpoint{2.180847in}{2.494774in}}%
\pgfpathlineto{\pgfqpoint{2.180687in}{2.300797in}}%
\pgfpathlineto{\pgfqpoint{2.181081in}{2.316700in}}%
\pgfpathlineto{\pgfqpoint{2.181106in}{2.288487in}}%
\pgfpathlineto{\pgfqpoint{2.182078in}{2.493247in}}%
\pgfpathlineto{\pgfqpoint{2.182152in}{2.351694in}}%
\pgfpathlineto{\pgfqpoint{2.182484in}{2.486661in}}%
\pgfpathlineto{\pgfqpoint{2.182743in}{2.298863in}}%
\pgfpathlineto{\pgfqpoint{2.183247in}{2.329012in}}%
\pgfpathlineto{\pgfqpoint{2.183567in}{2.295411in}}%
\pgfpathlineto{\pgfqpoint{2.183309in}{2.487145in}}%
\pgfpathlineto{\pgfqpoint{2.183703in}{2.442493in}}%
\pgfpathlineto{\pgfqpoint{2.184539in}{2.494517in}}%
\pgfpathlineto{\pgfqpoint{2.183973in}{2.299375in}}%
\pgfpathlineto{\pgfqpoint{2.184773in}{2.340133in}}%
\pgfpathlineto{\pgfqpoint{2.185216in}{2.288976in}}%
\pgfpathlineto{\pgfqpoint{2.184958in}{2.496177in}}%
\pgfpathlineto{\pgfqpoint{2.185856in}{2.361076in}}%
\pgfpathlineto{\pgfqpoint{2.186189in}{2.492892in}}%
\pgfpathlineto{\pgfqpoint{2.186853in}{2.290536in}}%
\pgfpathlineto{\pgfqpoint{2.186952in}{2.320291in}}%
\pgfpathlineto{\pgfqpoint{2.186964in}{2.319527in}}%
\pgfpathlineto{\pgfqpoint{2.186976in}{2.353125in}}%
\pgfpathlineto{\pgfqpoint{2.187419in}{2.489948in}}%
\pgfpathlineto{\pgfqpoint{2.187259in}{2.295919in}}%
\pgfpathlineto{\pgfqpoint{2.188072in}{2.323777in}}%
\pgfpathlineto{\pgfqpoint{2.188909in}{2.291977in}}%
\pgfpathlineto{\pgfqpoint{2.189069in}{2.498874in}}%
\pgfpathlineto{\pgfqpoint{2.189143in}{2.362019in}}%
\pgfpathlineto{\pgfqpoint{2.189893in}{2.488998in}}%
\pgfpathlineto{\pgfqpoint{2.189327in}{2.293243in}}%
\pgfpathlineto{\pgfqpoint{2.190226in}{2.348552in}}%
\pgfpathlineto{\pgfqpoint{2.190558in}{2.282957in}}%
\pgfpathlineto{\pgfqpoint{2.190299in}{2.496782in}}%
\pgfpathlineto{\pgfqpoint{2.191309in}{2.375113in}}%
\pgfpathlineto{\pgfqpoint{2.191530in}{2.494423in}}%
\pgfpathlineto{\pgfqpoint{2.191383in}{2.286892in}}%
\pgfpathlineto{\pgfqpoint{2.192392in}{2.374702in}}%
\pgfpathlineto{\pgfqpoint{2.193019in}{2.283209in}}%
\pgfpathlineto{\pgfqpoint{2.193179in}{2.496756in}}%
\pgfpathlineto{\pgfqpoint{2.193475in}{2.416861in}}%
\pgfpathlineto{\pgfqpoint{2.194410in}{2.504001in}}%
\pgfpathlineto{\pgfqpoint{2.194263in}{2.283073in}}%
\pgfpathlineto{\pgfqpoint{2.194546in}{2.375430in}}%
\pgfpathlineto{\pgfqpoint{2.194669in}{2.273632in}}%
\pgfpathlineto{\pgfqpoint{2.195235in}{2.499650in}}%
\pgfpathlineto{\pgfqpoint{2.195616in}{2.398217in}}%
\pgfpathlineto{\pgfqpoint{2.195641in}{2.504923in}}%
\pgfpathlineto{\pgfqpoint{2.195899in}{2.275953in}}%
\pgfpathlineto{\pgfqpoint{2.196699in}{2.342643in}}%
\pgfpathlineto{\pgfqpoint{2.197130in}{2.278660in}}%
\pgfpathlineto{\pgfqpoint{2.197696in}{2.508762in}}%
\pgfpathlineto{\pgfqpoint{2.197795in}{2.399884in}}%
\pgfpathlineto{\pgfqpoint{2.198521in}{2.509381in}}%
\pgfpathlineto{\pgfqpoint{2.198779in}{2.275299in}}%
\pgfpathlineto{\pgfqpoint{2.198866in}{2.321108in}}%
\pgfpathlineto{\pgfqpoint{2.199604in}{2.277432in}}%
\pgfpathlineto{\pgfqpoint{2.199346in}{2.506941in}}%
\pgfpathlineto{\pgfqpoint{2.199641in}{2.438563in}}%
\pgfpathlineto{\pgfqpoint{2.199752in}{2.509676in}}%
\pgfpathlineto{\pgfqpoint{2.200010in}{2.265099in}}%
\pgfpathlineto{\pgfqpoint{2.200699in}{2.390765in}}%
\pgfpathlineto{\pgfqpoint{2.201241in}{2.266753in}}%
\pgfpathlineto{\pgfqpoint{2.200982in}{2.509923in}}%
\pgfpathlineto{\pgfqpoint{2.201782in}{2.432904in}}%
\pgfpathlineto{\pgfqpoint{2.201807in}{2.510483in}}%
\pgfpathlineto{\pgfqpoint{2.202472in}{2.261649in}}%
\pgfpathlineto{\pgfqpoint{2.202866in}{2.320258in}}%
\pgfpathlineto{\pgfqpoint{2.203702in}{2.260007in}}%
\pgfpathlineto{\pgfqpoint{2.203862in}{2.515345in}}%
\pgfpathlineto{\pgfqpoint{2.203949in}{2.397539in}}%
\pgfpathlineto{\pgfqpoint{2.204675in}{2.512983in}}%
\pgfpathlineto{\pgfqpoint{2.204121in}{2.256343in}}%
\pgfpathlineto{\pgfqpoint{2.205019in}{2.323789in}}%
\pgfpathlineto{\pgfqpoint{2.205758in}{2.250873in}}%
\pgfpathlineto{\pgfqpoint{2.205499in}{2.518721in}}%
\pgfpathlineto{\pgfqpoint{2.205893in}{2.467457in}}%
\pgfpathlineto{\pgfqpoint{2.206730in}{2.526607in}}%
\pgfpathlineto{\pgfqpoint{2.206582in}{2.251427in}}%
\pgfpathlineto{\pgfqpoint{2.206964in}{2.367774in}}%
\pgfpathlineto{\pgfqpoint{2.207813in}{2.240671in}}%
\pgfpathlineto{\pgfqpoint{2.207961in}{2.529474in}}%
\pgfpathlineto{\pgfqpoint{2.208059in}{2.425899in}}%
\pgfpathlineto{\pgfqpoint{2.208786in}{2.532878in}}%
\pgfpathlineto{\pgfqpoint{2.208638in}{2.247800in}}%
\pgfpathlineto{\pgfqpoint{2.209032in}{2.279173in}}%
\pgfpathlineto{\pgfqpoint{2.209044in}{2.238887in}}%
\pgfpathlineto{\pgfqpoint{2.210016in}{2.531137in}}%
\pgfpathlineto{\pgfqpoint{2.210102in}{2.405716in}}%
\pgfpathlineto{\pgfqpoint{2.210841in}{2.534329in}}%
\pgfpathlineto{\pgfqpoint{2.211099in}{2.239584in}}%
\pgfpathlineto{\pgfqpoint{2.211173in}{2.332158in}}%
\pgfpathlineto{\pgfqpoint{2.211924in}{2.234228in}}%
\pgfpathlineto{\pgfqpoint{2.212072in}{2.544309in}}%
\pgfpathlineto{\pgfqpoint{2.212256in}{2.368520in}}%
\pgfpathlineto{\pgfqpoint{2.213302in}{2.551865in}}%
\pgfpathlineto{\pgfqpoint{2.213155in}{2.229097in}}%
\pgfpathlineto{\pgfqpoint{2.213339in}{2.369572in}}%
\pgfpathlineto{\pgfqpoint{2.213979in}{2.231813in}}%
\pgfpathlineto{\pgfqpoint{2.214127in}{2.555059in}}%
\pgfpathlineto{\pgfqpoint{2.214422in}{2.428147in}}%
\pgfpathlineto{\pgfqpoint{2.214533in}{2.551414in}}%
\pgfpathlineto{\pgfqpoint{2.215210in}{2.227498in}}%
\pgfpathlineto{\pgfqpoint{2.215493in}{2.348111in}}%
\pgfpathlineto{\pgfqpoint{2.216035in}{2.231633in}}%
\pgfpathlineto{\pgfqpoint{2.216182in}{2.552613in}}%
\pgfpathlineto{\pgfqpoint{2.216576in}{2.491305in}}%
\pgfpathlineto{\pgfqpoint{2.216589in}{2.544092in}}%
\pgfpathlineto{\pgfqpoint{2.217265in}{2.228920in}}%
\pgfpathlineto{\pgfqpoint{2.217647in}{2.372853in}}%
\pgfpathlineto{\pgfqpoint{2.218090in}{2.235076in}}%
\pgfpathlineto{\pgfqpoint{2.218238in}{2.536026in}}%
\pgfpathlineto{\pgfqpoint{2.218742in}{2.438142in}}%
\pgfpathlineto{\pgfqpoint{2.219062in}{2.517362in}}%
\pgfpathlineto{\pgfqpoint{2.218915in}{2.237074in}}%
\pgfpathlineto{\pgfqpoint{2.219715in}{2.313921in}}%
\pgfpathlineto{\pgfqpoint{2.220145in}{2.251611in}}%
\pgfpathlineto{\pgfqpoint{2.220293in}{2.503735in}}%
\pgfpathlineto{\pgfqpoint{2.220798in}{2.423036in}}%
\pgfpathlineto{\pgfqpoint{2.221118in}{2.488969in}}%
\pgfpathlineto{\pgfqpoint{2.220970in}{2.259464in}}%
\pgfpathlineto{\pgfqpoint{2.221770in}{2.321045in}}%
\pgfpathlineto{\pgfqpoint{2.222201in}{2.262823in}}%
\pgfpathlineto{\pgfqpoint{2.221942in}{2.484333in}}%
\pgfpathlineto{\pgfqpoint{2.222853in}{2.408276in}}%
\pgfpathlineto{\pgfqpoint{2.223185in}{2.468733in}}%
\pgfpathlineto{\pgfqpoint{2.223025in}{2.264407in}}%
\pgfpathlineto{\pgfqpoint{2.223419in}{2.303696in}}%
\pgfpathlineto{\pgfqpoint{2.223432in}{2.275030in}}%
\pgfpathlineto{\pgfqpoint{2.223592in}{2.472292in}}%
\pgfpathlineto{\pgfqpoint{2.224478in}{2.397130in}}%
\pgfpathlineto{\pgfqpoint{2.224502in}{2.404713in}}%
\pgfpathlineto{\pgfqpoint{2.224564in}{2.322137in}}%
\pgfpathlineto{\pgfqpoint{2.224650in}{2.323752in}}%
\pgfpathlineto{\pgfqpoint{2.225081in}{2.287489in}}%
\pgfpathlineto{\pgfqpoint{2.225241in}{2.463176in}}%
\pgfpathlineto{\pgfqpoint{2.225733in}{2.400306in}}%
\pgfpathlineto{\pgfqpoint{2.226312in}{2.296642in}}%
\pgfpathlineto{\pgfqpoint{2.226065in}{2.458751in}}%
\pgfpathlineto{\pgfqpoint{2.226841in}{2.391381in}}%
\pgfpathlineto{\pgfqpoint{2.226890in}{2.450219in}}%
\pgfpathlineto{\pgfqpoint{2.227136in}{2.301399in}}%
\pgfpathlineto{\pgfqpoint{2.227924in}{2.370321in}}%
\pgfpathlineto{\pgfqpoint{2.228367in}{2.308345in}}%
\pgfpathlineto{\pgfqpoint{2.228121in}{2.444272in}}%
\pgfpathlineto{\pgfqpoint{2.229019in}{2.398831in}}%
\pgfpathlineto{\pgfqpoint{2.229192in}{2.315872in}}%
\pgfpathlineto{\pgfqpoint{2.229770in}{2.438757in}}%
\pgfpathlineto{\pgfqpoint{2.230127in}{2.401779in}}%
\pgfpathlineto{\pgfqpoint{2.230422in}{2.323429in}}%
\pgfpathlineto{\pgfqpoint{2.230176in}{2.441620in}}%
\pgfpathlineto{\pgfqpoint{2.231321in}{2.359658in}}%
\pgfpathlineto{\pgfqpoint{2.231825in}{2.435250in}}%
\pgfpathlineto{\pgfqpoint{2.231653in}{2.328266in}}%
\pgfpathlineto{\pgfqpoint{2.232429in}{2.358893in}}%
\pgfpathlineto{\pgfqpoint{2.232650in}{2.430766in}}%
\pgfpathlineto{\pgfqpoint{2.233229in}{2.326605in}}%
\pgfpathlineto{\pgfqpoint{2.233561in}{2.370002in}}%
\pgfpathlineto{\pgfqpoint{2.233635in}{2.327008in}}%
\pgfpathlineto{\pgfqpoint{2.233893in}{2.427934in}}%
\pgfpathlineto{\pgfqpoint{2.234619in}{2.371749in}}%
\pgfpathlineto{\pgfqpoint{2.234705in}{2.426093in}}%
\pgfpathlineto{\pgfqpoint{2.234865in}{2.325820in}}%
\pgfpathlineto{\pgfqpoint{2.235715in}{2.351652in}}%
\pgfpathlineto{\pgfqpoint{2.235949in}{2.425893in}}%
\pgfpathlineto{\pgfqpoint{2.236109in}{2.330885in}}%
\pgfpathlineto{\pgfqpoint{2.236884in}{2.382491in}}%
\pgfpathlineto{\pgfqpoint{2.237339in}{2.327615in}}%
\pgfpathlineto{\pgfqpoint{2.237179in}{2.418579in}}%
\pgfpathlineto{\pgfqpoint{2.237967in}{2.400146in}}%
\pgfpathlineto{\pgfqpoint{2.238004in}{2.421198in}}%
\pgfpathlineto{\pgfqpoint{2.238570in}{2.329934in}}%
\pgfpathlineto{\pgfqpoint{2.238964in}{2.350013in}}%
\pgfpathlineto{\pgfqpoint{2.238976in}{2.331189in}}%
\pgfpathlineto{\pgfqpoint{2.239235in}{2.418139in}}%
\pgfpathlineto{\pgfqpoint{2.240035in}{2.402560in}}%
\pgfpathlineto{\pgfqpoint{2.240059in}{2.417536in}}%
\pgfpathlineto{\pgfqpoint{2.240625in}{2.335971in}}%
\pgfpathlineto{\pgfqpoint{2.241068in}{2.360340in}}%
\pgfpathlineto{\pgfqpoint{2.241450in}{2.336217in}}%
\pgfpathlineto{\pgfqpoint{2.241290in}{2.416574in}}%
\pgfpathlineto{\pgfqpoint{2.242102in}{2.406545in}}%
\pgfpathlineto{\pgfqpoint{2.242115in}{2.415107in}}%
\pgfpathlineto{\pgfqpoint{2.242681in}{2.337827in}}%
\pgfpathlineto{\pgfqpoint{2.243136in}{2.353384in}}%
\pgfpathlineto{\pgfqpoint{2.243505in}{2.335668in}}%
\pgfpathlineto{\pgfqpoint{2.243764in}{2.416278in}}%
\pgfpathlineto{\pgfqpoint{2.244158in}{2.401165in}}%
\pgfpathlineto{\pgfqpoint{2.244588in}{2.413973in}}%
\pgfpathlineto{\pgfqpoint{2.244736in}{2.338863in}}%
\pgfpathlineto{\pgfqpoint{2.245192in}{2.356474in}}%
\pgfpathlineto{\pgfqpoint{2.245561in}{2.340178in}}%
\pgfpathlineto{\pgfqpoint{2.245413in}{2.415825in}}%
\pgfpathlineto{\pgfqpoint{2.246225in}{2.411146in}}%
\pgfpathlineto{\pgfqpoint{2.246644in}{2.415129in}}%
\pgfpathlineto{\pgfqpoint{2.246447in}{2.343220in}}%
\pgfpathlineto{\pgfqpoint{2.247161in}{2.385534in}}%
\pgfpathlineto{\pgfqpoint{2.247272in}{2.343905in}}%
\pgfpathlineto{\pgfqpoint{2.247468in}{2.415660in}}%
\pgfpathlineto{\pgfqpoint{2.248256in}{2.388572in}}%
\pgfpathlineto{\pgfqpoint{2.248699in}{2.414199in}}%
\pgfpathlineto{\pgfqpoint{2.248515in}{2.344546in}}%
\pgfpathlineto{\pgfqpoint{2.249315in}{2.349743in}}%
\pgfpathlineto{\pgfqpoint{2.249327in}{2.344000in}}%
\pgfpathlineto{\pgfqpoint{2.249524in}{2.415064in}}%
\pgfpathlineto{\pgfqpoint{2.250324in}{2.394566in}}%
\pgfpathlineto{\pgfqpoint{2.250755in}{2.411124in}}%
\pgfpathlineto{\pgfqpoint{2.250570in}{2.347551in}}%
\pgfpathlineto{\pgfqpoint{2.251382in}{2.348778in}}%
\pgfpathlineto{\pgfqpoint{2.251801in}{2.346287in}}%
\pgfpathlineto{\pgfqpoint{2.251579in}{2.409866in}}%
\pgfpathlineto{\pgfqpoint{2.251973in}{2.397517in}}%
\pgfpathlineto{\pgfqpoint{2.252822in}{2.410844in}}%
\pgfpathlineto{\pgfqpoint{2.252625in}{2.347400in}}%
\pgfpathlineto{\pgfqpoint{2.253019in}{2.352914in}}%
\pgfpathlineto{\pgfqpoint{2.253856in}{2.348557in}}%
\pgfpathlineto{\pgfqpoint{2.253647in}{2.409821in}}%
\pgfpathlineto{\pgfqpoint{2.254016in}{2.381435in}}%
\pgfpathlineto{\pgfqpoint{2.254878in}{2.406997in}}%
\pgfpathlineto{\pgfqpoint{2.254681in}{2.344377in}}%
\pgfpathlineto{\pgfqpoint{2.255087in}{2.346551in}}%
\pgfpathlineto{\pgfqpoint{2.255924in}{2.343361in}}%
\pgfpathlineto{\pgfqpoint{2.255702in}{2.402655in}}%
\pgfpathlineto{\pgfqpoint{2.256072in}{2.371346in}}%
\pgfpathlineto{\pgfqpoint{2.256933in}{2.404021in}}%
\pgfpathlineto{\pgfqpoint{2.256748in}{2.345431in}}%
\pgfpathlineto{\pgfqpoint{2.257142in}{2.346239in}}%
\pgfpathlineto{\pgfqpoint{2.257155in}{2.342412in}}%
\pgfpathlineto{\pgfqpoint{2.257758in}{2.404110in}}%
\pgfpathlineto{\pgfqpoint{2.258127in}{2.370706in}}%
\pgfpathlineto{\pgfqpoint{2.259001in}{2.405904in}}%
\pgfpathlineto{\pgfqpoint{2.258804in}{2.343052in}}%
\pgfpathlineto{\pgfqpoint{2.259198in}{2.349673in}}%
\pgfpathlineto{\pgfqpoint{2.260035in}{2.341230in}}%
\pgfpathlineto{\pgfqpoint{2.259407in}{2.399254in}}%
\pgfpathlineto{\pgfqpoint{2.260207in}{2.385567in}}%
\pgfpathlineto{\pgfqpoint{2.261068in}{2.400417in}}%
\pgfpathlineto{\pgfqpoint{2.260859in}{2.344177in}}%
\pgfpathlineto{\pgfqpoint{2.261253in}{2.351750in}}%
\pgfpathlineto{\pgfqpoint{2.261278in}{2.343741in}}%
\pgfpathlineto{\pgfqpoint{2.261893in}{2.400252in}}%
\pgfpathlineto{\pgfqpoint{2.262275in}{2.391817in}}%
\pgfpathlineto{\pgfqpoint{2.263124in}{2.400261in}}%
\pgfpathlineto{\pgfqpoint{2.262927in}{2.347241in}}%
\pgfpathlineto{\pgfqpoint{2.263296in}{2.361015in}}%
\pgfpathlineto{\pgfqpoint{2.263333in}{2.345937in}}%
\pgfpathlineto{\pgfqpoint{2.263948in}{2.395732in}}%
\pgfpathlineto{\pgfqpoint{2.264318in}{2.383235in}}%
\pgfpathlineto{\pgfqpoint{2.265216in}{2.396665in}}%
\pgfpathlineto{\pgfqpoint{2.264982in}{2.348360in}}%
\pgfpathlineto{\pgfqpoint{2.265364in}{2.359569in}}%
\pgfpathlineto{\pgfqpoint{2.266213in}{2.347408in}}%
\pgfpathlineto{\pgfqpoint{2.266053in}{2.397782in}}%
\pgfpathlineto{\pgfqpoint{2.266385in}{2.383098in}}%
\pgfpathlineto{\pgfqpoint{2.267271in}{2.395235in}}%
\pgfpathlineto{\pgfqpoint{2.267050in}{2.346589in}}%
\pgfpathlineto{\pgfqpoint{2.267431in}{2.357817in}}%
\pgfpathlineto{\pgfqpoint{2.267456in}{2.350260in}}%
\pgfpathlineto{\pgfqpoint{2.268108in}{2.393158in}}%
\pgfpathlineto{\pgfqpoint{2.268478in}{2.388037in}}%
\pgfpathlineto{\pgfqpoint{2.269339in}{2.398655in}}%
\pgfpathlineto{\pgfqpoint{2.268687in}{2.349092in}}%
\pgfpathlineto{\pgfqpoint{2.269462in}{2.369959in}}%
\pgfpathlineto{\pgfqpoint{2.269511in}{2.352534in}}%
\pgfpathlineto{\pgfqpoint{2.270164in}{2.396657in}}%
\pgfpathlineto{\pgfqpoint{2.270496in}{2.377888in}}%
\pgfpathlineto{\pgfqpoint{2.271395in}{2.394959in}}%
\pgfpathlineto{\pgfqpoint{2.270742in}{2.353742in}}%
\pgfpathlineto{\pgfqpoint{2.271493in}{2.369437in}}%
\pgfpathlineto{\pgfqpoint{2.272219in}{2.390564in}}%
\pgfpathlineto{\pgfqpoint{2.271567in}{2.356144in}}%
\pgfpathlineto{\pgfqpoint{2.272367in}{2.365890in}}%
\pgfpathlineto{\pgfqpoint{2.272798in}{2.356060in}}%
\pgfpathlineto{\pgfqpoint{2.272638in}{2.390933in}}%
\pgfpathlineto{\pgfqpoint{2.273425in}{2.380951in}}%
\pgfpathlineto{\pgfqpoint{2.273462in}{2.391738in}}%
\pgfpathlineto{\pgfqpoint{2.274028in}{2.358904in}}%
\pgfpathlineto{\pgfqpoint{2.274533in}{2.379480in}}%
\pgfpathlineto{\pgfqpoint{2.274853in}{2.357834in}}%
\pgfpathlineto{\pgfqpoint{2.275518in}{2.389994in}}%
\pgfpathlineto{\pgfqpoint{2.275678in}{2.360645in}}%
\pgfpathlineto{\pgfqpoint{2.276330in}{2.385783in}}%
\pgfpathlineto{\pgfqpoint{2.276884in}{2.369636in}}%
\pgfpathlineto{\pgfqpoint{2.277733in}{2.361782in}}%
\pgfpathlineto{\pgfqpoint{2.277167in}{2.385211in}}%
\pgfpathlineto{\pgfqpoint{2.277967in}{2.380091in}}%
\pgfpathlineto{\pgfqpoint{2.279038in}{2.384655in}}%
\pgfpathlineto{\pgfqpoint{2.278151in}{2.362258in}}%
\pgfpathlineto{\pgfqpoint{2.279062in}{2.379841in}}%
\pgfpathlineto{\pgfqpoint{2.279911in}{2.365735in}}%
\pgfpathlineto{\pgfqpoint{2.279653in}{2.386118in}}%
\pgfpathlineto{\pgfqpoint{2.280194in}{2.367240in}}%
\pgfpathlineto{\pgfqpoint{2.280318in}{2.365278in}}%
\pgfpathlineto{\pgfqpoint{2.280268in}{2.385193in}}%
\pgfpathlineto{\pgfqpoint{2.280662in}{2.380563in}}%
\pgfpathlineto{\pgfqpoint{2.280687in}{2.386209in}}%
\pgfpathlineto{\pgfqpoint{2.281438in}{2.363418in}}%
\pgfpathlineto{\pgfqpoint{2.281745in}{2.376484in}}%
\pgfpathlineto{\pgfqpoint{2.281967in}{2.359031in}}%
\pgfpathlineto{\pgfqpoint{2.282324in}{2.387680in}}%
\pgfpathlineto{\pgfqpoint{2.282841in}{2.380997in}}%
\pgfpathlineto{\pgfqpoint{2.283468in}{2.354809in}}%
\pgfpathlineto{\pgfqpoint{2.283764in}{2.394185in}}%
\pgfpathlineto{\pgfqpoint{2.283776in}{2.402980in}}%
\pgfpathlineto{\pgfqpoint{2.284711in}{2.355561in}}%
\pgfpathlineto{\pgfqpoint{2.284847in}{2.376728in}}%
\pgfpathlineto{\pgfqpoint{2.285758in}{2.410260in}}%
\pgfpathlineto{\pgfqpoint{2.285142in}{2.341320in}}%
\pgfpathlineto{\pgfqpoint{2.285893in}{2.366877in}}%
\pgfpathlineto{\pgfqpoint{2.285942in}{2.340118in}}%
\pgfpathlineto{\pgfqpoint{2.286890in}{2.423929in}}%
\pgfpathlineto{\pgfqpoint{2.286902in}{2.432969in}}%
\pgfpathlineto{\pgfqpoint{2.287185in}{2.308493in}}%
\pgfpathlineto{\pgfqpoint{2.287887in}{2.366223in}}%
\pgfpathlineto{\pgfqpoint{2.288416in}{2.287557in}}%
\pgfpathlineto{\pgfqpoint{2.288785in}{2.461259in}}%
\pgfpathlineto{\pgfqpoint{2.288982in}{2.388311in}}%
\pgfpathlineto{\pgfqpoint{2.289647in}{2.254025in}}%
\pgfpathlineto{\pgfqpoint{2.289290in}{2.471222in}}%
\pgfpathlineto{\pgfqpoint{2.290041in}{2.421361in}}%
\pgfpathlineto{\pgfqpoint{2.290422in}{2.452760in}}%
\pgfpathlineto{\pgfqpoint{2.290102in}{2.298808in}}%
\pgfpathlineto{\pgfqpoint{2.290631in}{2.344657in}}%
\pgfpathlineto{\pgfqpoint{2.291702in}{2.187950in}}%
\pgfpathlineto{\pgfqpoint{2.291333in}{2.484928in}}%
\pgfpathlineto{\pgfqpoint{2.291739in}{2.310141in}}%
\pgfpathlineto{\pgfqpoint{2.292342in}{2.455858in}}%
\pgfpathlineto{\pgfqpoint{2.292699in}{2.276156in}}%
\pgfpathlineto{\pgfqpoint{2.292871in}{2.399783in}}%
\pgfpathlineto{\pgfqpoint{2.293659in}{2.290117in}}%
\pgfpathlineto{\pgfqpoint{2.292994in}{2.456309in}}%
\pgfpathlineto{\pgfqpoint{2.293942in}{2.408136in}}%
\pgfpathlineto{\pgfqpoint{2.294410in}{2.510486in}}%
\pgfpathlineto{\pgfqpoint{2.294693in}{2.221505in}}%
\pgfpathlineto{\pgfqpoint{2.294804in}{2.278308in}}%
\pgfpathlineto{\pgfqpoint{2.294890in}{2.363528in}}%
\pgfpathlineto{\pgfqpoint{2.295887in}{2.466455in}}%
\pgfpathlineto{\pgfqpoint{2.295714in}{2.284144in}}%
\pgfpathlineto{\pgfqpoint{2.295998in}{2.376532in}}%
\pgfpathlineto{\pgfqpoint{2.296736in}{2.184004in}}%
\pgfpathlineto{\pgfqpoint{2.296330in}{2.482310in}}%
\pgfpathlineto{\pgfqpoint{2.297142in}{2.331913in}}%
\pgfpathlineto{\pgfqpoint{2.297314in}{2.512804in}}%
\pgfpathlineto{\pgfqpoint{2.297708in}{2.174585in}}%
\pgfpathlineto{\pgfqpoint{2.298336in}{2.465608in}}%
\pgfpathlineto{\pgfqpoint{2.298631in}{2.233283in}}%
\pgfpathlineto{\pgfqpoint{2.299284in}{2.576535in}}%
\pgfpathlineto{\pgfqpoint{2.299345in}{2.557106in}}%
\pgfpathlineto{\pgfqpoint{2.299370in}{2.582566in}}%
\pgfpathlineto{\pgfqpoint{2.299641in}{2.115967in}}%
\pgfpathlineto{\pgfqpoint{2.300330in}{2.387723in}}%
\pgfpathlineto{\pgfqpoint{2.300625in}{2.180912in}}%
\pgfpathlineto{\pgfqpoint{2.301277in}{2.574567in}}%
\pgfpathlineto{\pgfqpoint{2.301339in}{2.516196in}}%
\pgfpathlineto{\pgfqpoint{2.301376in}{2.585526in}}%
\pgfpathlineto{\pgfqpoint{2.301721in}{2.155256in}}%
\pgfpathlineto{\pgfqpoint{2.302397in}{2.417027in}}%
\pgfpathlineto{\pgfqpoint{2.302619in}{2.089184in}}%
\pgfpathlineto{\pgfqpoint{2.303357in}{2.516857in}}%
\pgfpathlineto{\pgfqpoint{2.303616in}{2.191342in}}%
\pgfpathlineto{\pgfqpoint{2.304231in}{2.654563in}}%
\pgfpathlineto{\pgfqpoint{2.304613in}{2.069778in}}%
\pgfpathlineto{\pgfqpoint{2.304785in}{2.324214in}}%
\pgfpathlineto{\pgfqpoint{2.305499in}{2.136236in}}%
\pgfpathlineto{\pgfqpoint{2.305745in}{2.513133in}}%
\pgfpathlineto{\pgfqpoint{2.305905in}{2.294031in}}%
\pgfpathlineto{\pgfqpoint{2.306016in}{2.223311in}}%
\pgfpathlineto{\pgfqpoint{2.306127in}{2.444679in}}%
\pgfpathlineto{\pgfqpoint{2.306311in}{2.680835in}}%
\pgfpathlineto{\pgfqpoint{2.306545in}{2.083555in}}%
\pgfpathlineto{\pgfqpoint{2.307234in}{2.497793in}}%
\pgfpathlineto{\pgfqpoint{2.307493in}{2.049327in}}%
\pgfpathlineto{\pgfqpoint{2.308231in}{2.562635in}}%
\pgfpathlineto{\pgfqpoint{2.308305in}{2.537029in}}%
\pgfpathlineto{\pgfqpoint{2.309117in}{2.633149in}}%
\pgfpathlineto{\pgfqpoint{2.308502in}{2.173053in}}%
\pgfpathlineto{\pgfqpoint{2.309351in}{2.385029in}}%
\pgfpathlineto{\pgfqpoint{2.309499in}{1.991281in}}%
\pgfpathlineto{\pgfqpoint{2.310139in}{2.525996in}}%
\pgfpathlineto{\pgfqpoint{2.310484in}{2.240665in}}%
\pgfpathlineto{\pgfqpoint{2.311161in}{2.676161in}}%
\pgfpathlineto{\pgfqpoint{2.311444in}{2.065599in}}%
\pgfpathlineto{\pgfqpoint{2.311456in}{2.039816in}}%
\pgfpathlineto{\pgfqpoint{2.312022in}{2.561724in}}%
\pgfpathlineto{\pgfqpoint{2.312465in}{2.153659in}}%
\pgfpathlineto{\pgfqpoint{2.313228in}{2.659207in}}%
\pgfpathlineto{\pgfqpoint{2.313487in}{2.146842in}}%
\pgfpathlineto{\pgfqpoint{2.313597in}{2.311939in}}%
\pgfpathlineto{\pgfqpoint{2.314028in}{2.621968in}}%
\pgfpathlineto{\pgfqpoint{2.314410in}{1.987861in}}%
\pgfpathlineto{\pgfqpoint{2.314742in}{2.406590in}}%
\pgfpathlineto{\pgfqpoint{2.315788in}{2.155188in}}%
\pgfpathlineto{\pgfqpoint{2.315173in}{2.499033in}}%
\pgfpathlineto{\pgfqpoint{2.315874in}{2.346484in}}%
\pgfpathlineto{\pgfqpoint{2.316022in}{2.667586in}}%
\pgfpathlineto{\pgfqpoint{2.316404in}{2.055554in}}%
\pgfpathlineto{\pgfqpoint{2.317019in}{2.424833in}}%
\pgfpathlineto{\pgfqpoint{2.317314in}{2.135451in}}%
\pgfpathlineto{\pgfqpoint{2.317967in}{2.530763in}}%
\pgfpathlineto{\pgfqpoint{2.318102in}{2.675223in}}%
\pgfpathlineto{\pgfqpoint{2.318287in}{2.273531in}}%
\pgfpathlineto{\pgfqpoint{2.319271in}{2.049895in}}%
\pgfpathlineto{\pgfqpoint{2.318988in}{2.572525in}}%
\pgfpathlineto{\pgfqpoint{2.319394in}{2.270621in}}%
\pgfpathlineto{\pgfqpoint{2.320120in}{2.565566in}}%
\pgfpathlineto{\pgfqpoint{2.319788in}{2.193155in}}%
\pgfpathlineto{\pgfqpoint{2.320514in}{2.341465in}}%
\pgfpathlineto{\pgfqpoint{2.320945in}{2.625954in}}%
\pgfpathlineto{\pgfqpoint{2.321314in}{2.052413in}}%
\pgfpathlineto{\pgfqpoint{2.321622in}{2.406220in}}%
\pgfpathlineto{\pgfqpoint{2.322656in}{2.104692in}}%
\pgfpathlineto{\pgfqpoint{2.321905in}{2.485829in}}%
\pgfpathlineto{\pgfqpoint{2.322767in}{2.256583in}}%
\pgfpathlineto{\pgfqpoint{2.322939in}{2.682645in}}%
\pgfpathlineto{\pgfqpoint{2.323259in}{2.082734in}}%
\pgfpathlineto{\pgfqpoint{2.323887in}{2.358302in}}%
\pgfpathlineto{\pgfqpoint{2.324970in}{2.608073in}}%
\pgfpathlineto{\pgfqpoint{2.324724in}{2.131309in}}%
\pgfpathlineto{\pgfqpoint{2.325080in}{2.474325in}}%
\pgfpathlineto{\pgfqpoint{2.325339in}{2.103629in}}%
\pgfpathlineto{\pgfqpoint{2.325819in}{2.531523in}}%
\pgfpathlineto{\pgfqpoint{2.326225in}{2.286620in}}%
\pgfpathlineto{\pgfqpoint{2.327136in}{2.494513in}}%
\pgfpathlineto{\pgfqpoint{2.326570in}{2.215803in}}%
\pgfpathlineto{\pgfqpoint{2.327173in}{2.246238in}}%
\pgfpathlineto{\pgfqpoint{2.327579in}{2.097766in}}%
\pgfpathlineto{\pgfqpoint{2.327924in}{2.662814in}}%
\pgfpathlineto{\pgfqpoint{2.328305in}{2.136784in}}%
\pgfpathlineto{\pgfqpoint{2.329204in}{2.560068in}}%
\pgfpathlineto{\pgfqpoint{2.329450in}{2.290783in}}%
\pgfpathlineto{\pgfqpoint{2.329647in}{2.076948in}}%
\pgfpathlineto{\pgfqpoint{2.329782in}{2.545500in}}%
\pgfpathlineto{\pgfqpoint{2.329856in}{2.655123in}}%
\pgfpathlineto{\pgfqpoint{2.330274in}{2.063363in}}%
\pgfpathlineto{\pgfqpoint{2.330767in}{2.334484in}}%
\pgfpathlineto{\pgfqpoint{2.331493in}{2.106858in}}%
\pgfpathlineto{\pgfqpoint{2.330828in}{2.504376in}}%
\pgfpathlineto{\pgfqpoint{2.331800in}{2.441949in}}%
\pgfpathlineto{\pgfqpoint{2.332674in}{2.605702in}}%
\pgfpathlineto{\pgfqpoint{2.332416in}{2.181219in}}%
\pgfpathlineto{\pgfqpoint{2.332920in}{2.474554in}}%
\pgfpathlineto{\pgfqpoint{2.333228in}{2.145212in}}%
\pgfpathlineto{\pgfqpoint{2.333905in}{2.517098in}}%
\pgfpathlineto{\pgfqpoint{2.333991in}{2.516057in}}%
\pgfpathlineto{\pgfqpoint{2.334767in}{2.685514in}}%
\pgfpathlineto{\pgfqpoint{2.334471in}{2.053690in}}%
\pgfpathlineto{\pgfqpoint{2.335037in}{2.370339in}}%
\pgfpathlineto{\pgfqpoint{2.335185in}{2.158821in}}%
\pgfpathlineto{\pgfqpoint{2.336096in}{2.549632in}}%
\pgfpathlineto{\pgfqpoint{2.336157in}{2.246399in}}%
\pgfpathlineto{\pgfqpoint{2.336785in}{2.518962in}}%
\pgfpathlineto{\pgfqpoint{2.336404in}{2.119017in}}%
\pgfpathlineto{\pgfqpoint{2.337265in}{2.274246in}}%
\pgfpathlineto{\pgfqpoint{2.337425in}{2.173232in}}%
\pgfpathlineto{\pgfqpoint{2.337671in}{2.526490in}}%
\pgfpathlineto{\pgfqpoint{2.338299in}{2.351927in}}%
\pgfpathlineto{\pgfqpoint{2.339013in}{2.595283in}}%
\pgfpathlineto{\pgfqpoint{2.339271in}{2.147674in}}%
\pgfpathlineto{\pgfqpoint{2.339357in}{2.188210in}}%
\pgfpathlineto{\pgfqpoint{2.339382in}{2.111364in}}%
\pgfpathlineto{\pgfqpoint{2.339616in}{2.602975in}}%
\pgfpathlineto{\pgfqpoint{2.340428in}{2.330283in}}%
\pgfpathlineto{\pgfqpoint{2.340871in}{2.591815in}}%
\pgfpathlineto{\pgfqpoint{2.341339in}{2.100064in}}%
\pgfpathlineto{\pgfqpoint{2.341585in}{2.523356in}}%
\pgfpathlineto{\pgfqpoint{2.341610in}{2.540970in}}%
\pgfpathlineto{\pgfqpoint{2.341967in}{2.240395in}}%
\pgfpathlineto{\pgfqpoint{2.342557in}{2.388453in}}%
\pgfpathlineto{\pgfqpoint{2.343210in}{2.156626in}}%
\pgfpathlineto{\pgfqpoint{2.342730in}{2.521081in}}%
\pgfpathlineto{\pgfqpoint{2.343653in}{2.412137in}}%
\pgfpathlineto{\pgfqpoint{2.344613in}{2.566563in}}%
\pgfpathlineto{\pgfqpoint{2.344330in}{2.183821in}}%
\pgfpathlineto{\pgfqpoint{2.344711in}{2.426841in}}%
\pgfpathlineto{\pgfqpoint{2.345056in}{2.191422in}}%
\pgfpathlineto{\pgfqpoint{2.345757in}{2.514815in}}%
\pgfpathlineto{\pgfqpoint{2.345807in}{2.431680in}}%
\pgfpathlineto{\pgfqpoint{2.345856in}{2.579149in}}%
\pgfpathlineto{\pgfqpoint{2.346299in}{2.169716in}}%
\pgfpathlineto{\pgfqpoint{2.346890in}{2.321107in}}%
\pgfpathlineto{\pgfqpoint{2.346914in}{2.222994in}}%
\pgfpathlineto{\pgfqpoint{2.347603in}{2.528140in}}%
\pgfpathlineto{\pgfqpoint{2.347973in}{2.399947in}}%
\pgfpathlineto{\pgfqpoint{2.348834in}{2.569688in}}%
\pgfpathlineto{\pgfqpoint{2.348157in}{2.189164in}}%
\pgfpathlineto{\pgfqpoint{2.349031in}{2.346885in}}%
\pgfpathlineto{\pgfqpoint{2.349117in}{2.239270in}}%
\pgfpathlineto{\pgfqpoint{2.349573in}{2.529207in}}%
\pgfpathlineto{\pgfqpoint{2.350139in}{2.277674in}}%
\pgfpathlineto{\pgfqpoint{2.350459in}{2.522800in}}%
\pgfpathlineto{\pgfqpoint{2.351160in}{2.199700in}}%
\pgfpathlineto{\pgfqpoint{2.351234in}{2.293152in}}%
\pgfpathlineto{\pgfqpoint{2.352194in}{2.190267in}}%
\pgfpathlineto{\pgfqpoint{2.351308in}{2.558095in}}%
\pgfpathlineto{\pgfqpoint{2.352330in}{2.343980in}}%
\pgfpathlineto{\pgfqpoint{2.352551in}{2.609112in}}%
\pgfpathlineto{\pgfqpoint{2.353117in}{2.149604in}}%
\pgfpathlineto{\pgfqpoint{2.353425in}{2.330492in}}%
\pgfpathlineto{\pgfqpoint{2.353782in}{2.590682in}}%
\pgfpathlineto{\pgfqpoint{2.354139in}{2.239006in}}%
\pgfpathlineto{\pgfqpoint{2.354237in}{2.254186in}}%
\pgfpathlineto{\pgfqpoint{2.354963in}{2.162900in}}%
\pgfpathlineto{\pgfqpoint{2.354607in}{2.538019in}}%
\pgfpathlineto{\pgfqpoint{2.355308in}{2.394347in}}%
\pgfpathlineto{\pgfqpoint{2.356367in}{2.566602in}}%
\pgfpathlineto{\pgfqpoint{2.356108in}{2.175083in}}%
\pgfpathlineto{\pgfqpoint{2.356403in}{2.411045in}}%
\pgfpathlineto{\pgfqpoint{2.356834in}{2.155118in}}%
\pgfpathlineto{\pgfqpoint{2.356662in}{2.575096in}}%
\pgfpathlineto{\pgfqpoint{2.357474in}{2.479515in}}%
\pgfpathlineto{\pgfqpoint{2.357597in}{2.577328in}}%
\pgfpathlineto{\pgfqpoint{2.358065in}{2.123606in}}%
\pgfpathlineto{\pgfqpoint{2.358557in}{2.334185in}}%
\pgfpathlineto{\pgfqpoint{2.359296in}{2.215728in}}%
\pgfpathlineto{\pgfqpoint{2.358730in}{2.555940in}}%
\pgfpathlineto{\pgfqpoint{2.359640in}{2.471642in}}%
\pgfpathlineto{\pgfqpoint{2.359911in}{2.171101in}}%
\pgfpathlineto{\pgfqpoint{2.360366in}{2.534705in}}%
\pgfpathlineto{\pgfqpoint{2.360736in}{2.425461in}}%
\pgfpathlineto{\pgfqpoint{2.361597in}{2.574607in}}%
\pgfpathlineto{\pgfqpoint{2.361770in}{2.141593in}}%
\pgfpathlineto{\pgfqpoint{2.361831in}{2.408277in}}%
\pgfpathlineto{\pgfqpoint{2.362890in}{2.216549in}}%
\pgfpathlineto{\pgfqpoint{2.362533in}{2.567759in}}%
\pgfpathlineto{\pgfqpoint{2.362926in}{2.408547in}}%
\pgfpathlineto{\pgfqpoint{2.363665in}{2.566870in}}%
\pgfpathlineto{\pgfqpoint{2.363000in}{2.144591in}}%
\pgfpathlineto{\pgfqpoint{2.363997in}{2.270377in}}%
\pgfpathlineto{\pgfqpoint{2.364846in}{2.205137in}}%
\pgfpathlineto{\pgfqpoint{2.364477in}{2.551553in}}%
\pgfpathlineto{\pgfqpoint{2.365080in}{2.327858in}}%
\pgfpathlineto{\pgfqpoint{2.365708in}{2.553035in}}%
\pgfpathlineto{\pgfqpoint{2.365474in}{2.219997in}}%
\pgfpathlineto{\pgfqpoint{2.365954in}{2.296336in}}%
\pgfpathlineto{\pgfqpoint{2.365979in}{2.181633in}}%
\pgfpathlineto{\pgfqpoint{2.366237in}{2.546289in}}%
\pgfpathlineto{\pgfqpoint{2.367037in}{2.413564in}}%
\pgfpathlineto{\pgfqpoint{2.367370in}{2.556555in}}%
\pgfpathlineto{\pgfqpoint{2.367210in}{2.214638in}}%
\pgfpathlineto{\pgfqpoint{2.367911in}{2.277566in}}%
\pgfpathlineto{\pgfqpoint{2.367936in}{2.148229in}}%
\pgfpathlineto{\pgfqpoint{2.368182in}{2.555786in}}%
\pgfpathlineto{\pgfqpoint{2.368982in}{2.401019in}}%
\pgfpathlineto{\pgfqpoint{2.369413in}{2.582828in}}%
\pgfpathlineto{\pgfqpoint{2.369166in}{2.196922in}}%
\pgfpathlineto{\pgfqpoint{2.370065in}{2.358679in}}%
\pgfpathlineto{\pgfqpoint{2.370914in}{2.162486in}}%
\pgfpathlineto{\pgfqpoint{2.370656in}{2.543702in}}%
\pgfpathlineto{\pgfqpoint{2.371148in}{2.498589in}}%
\pgfpathlineto{\pgfqpoint{2.371173in}{2.578857in}}%
\pgfpathlineto{\pgfqpoint{2.371640in}{2.145066in}}%
\pgfpathlineto{\pgfqpoint{2.372219in}{2.389976in}}%
\pgfpathlineto{\pgfqpoint{2.372871in}{2.111740in}}%
\pgfpathlineto{\pgfqpoint{2.372416in}{2.590294in}}%
\pgfpathlineto{\pgfqpoint{2.373326in}{2.403297in}}%
\pgfpathlineto{\pgfqpoint{2.374348in}{2.551713in}}%
\pgfpathlineto{\pgfqpoint{2.374114in}{2.152028in}}%
\pgfpathlineto{\pgfqpoint{2.374459in}{2.462449in}}%
\pgfpathlineto{\pgfqpoint{2.375345in}{2.215465in}}%
\pgfpathlineto{\pgfqpoint{2.375173in}{2.519916in}}%
\pgfpathlineto{\pgfqpoint{2.375554in}{2.478291in}}%
\pgfpathlineto{\pgfqpoint{2.376120in}{2.551591in}}%
\pgfpathlineto{\pgfqpoint{2.375948in}{2.229937in}}%
\pgfpathlineto{\pgfqpoint{2.376465in}{2.283026in}}%
\pgfpathlineto{\pgfqpoint{2.376477in}{2.282505in}}%
\pgfpathlineto{\pgfqpoint{2.377351in}{2.578878in}}%
\pgfpathlineto{\pgfqpoint{2.376588in}{2.173562in}}%
\pgfpathlineto{\pgfqpoint{2.377585in}{2.304076in}}%
\pgfpathlineto{\pgfqpoint{2.377646in}{2.468597in}}%
\pgfpathlineto{\pgfqpoint{2.377806in}{2.157967in}}%
\pgfpathlineto{\pgfqpoint{2.377819in}{2.127144in}}%
\pgfpathlineto{\pgfqpoint{2.378065in}{2.534861in}}%
\pgfpathlineto{\pgfqpoint{2.378840in}{2.334551in}}%
\pgfpathlineto{\pgfqpoint{2.379296in}{2.558338in}}%
\pgfpathlineto{\pgfqpoint{2.379050in}{2.135449in}}%
\pgfpathlineto{\pgfqpoint{2.379948in}{2.363875in}}%
\pgfpathlineto{\pgfqpoint{2.380293in}{2.169665in}}%
\pgfpathlineto{\pgfqpoint{2.380120in}{2.556131in}}%
\pgfpathlineto{\pgfqpoint{2.381031in}{2.472798in}}%
\pgfpathlineto{\pgfqpoint{2.381363in}{2.534704in}}%
\pgfpathlineto{\pgfqpoint{2.381536in}{2.185358in}}%
\pgfpathlineto{\pgfqpoint{2.381893in}{2.314864in}}%
\pgfpathlineto{\pgfqpoint{2.382766in}{2.127332in}}%
\pgfpathlineto{\pgfqpoint{2.382274in}{2.533511in}}%
\pgfpathlineto{\pgfqpoint{2.382939in}{2.483718in}}%
\pgfpathlineto{\pgfqpoint{2.382963in}{2.519213in}}%
\pgfpathlineto{\pgfqpoint{2.383271in}{2.244079in}}%
\pgfpathlineto{\pgfqpoint{2.383948in}{2.367075in}}%
\pgfpathlineto{\pgfqpoint{2.383997in}{2.139926in}}%
\pgfpathlineto{\pgfqpoint{2.384194in}{2.534504in}}%
\pgfpathlineto{\pgfqpoint{2.385031in}{2.432848in}}%
\pgfpathlineto{\pgfqpoint{2.385068in}{2.524844in}}%
\pgfpathlineto{\pgfqpoint{2.385240in}{2.143439in}}%
\pgfpathlineto{\pgfqpoint{2.386114in}{2.406367in}}%
\pgfpathlineto{\pgfqpoint{2.386471in}{2.169901in}}%
\pgfpathlineto{\pgfqpoint{2.386299in}{2.525586in}}%
\pgfpathlineto{\pgfqpoint{2.387197in}{2.479239in}}%
\pgfpathlineto{\pgfqpoint{2.387222in}{2.508534in}}%
\pgfpathlineto{\pgfqpoint{2.387714in}{2.141453in}}%
\pgfpathlineto{\pgfqpoint{2.388182in}{2.365365in}}%
\pgfpathlineto{\pgfqpoint{2.388945in}{2.171091in}}%
\pgfpathlineto{\pgfqpoint{2.389166in}{2.545820in}}%
\pgfpathlineto{\pgfqpoint{2.389265in}{2.476494in}}%
\pgfpathlineto{\pgfqpoint{2.389277in}{2.476445in}}%
\pgfpathlineto{\pgfqpoint{2.390188in}{2.171851in}}%
\pgfpathlineto{\pgfqpoint{2.390336in}{2.517076in}}%
\pgfpathlineto{\pgfqpoint{2.390360in}{2.492894in}}%
\pgfpathlineto{\pgfqpoint{2.390409in}{2.543510in}}%
\pgfpathlineto{\pgfqpoint{2.390594in}{2.238850in}}%
\pgfpathlineto{\pgfqpoint{2.391382in}{2.335344in}}%
\pgfpathlineto{\pgfqpoint{2.391419in}{2.172455in}}%
\pgfpathlineto{\pgfqpoint{2.392169in}{2.514641in}}%
\pgfpathlineto{\pgfqpoint{2.392465in}{2.441379in}}%
\pgfpathlineto{\pgfqpoint{2.392871in}{2.504331in}}%
\pgfpathlineto{\pgfqpoint{2.392662in}{2.177949in}}%
\pgfpathlineto{\pgfqpoint{2.393536in}{2.344753in}}%
\pgfpathlineto{\pgfqpoint{2.393893in}{2.210441in}}%
\pgfpathlineto{\pgfqpoint{2.394114in}{2.523466in}}%
\pgfpathlineto{\pgfqpoint{2.394606in}{2.439448in}}%
\pgfpathlineto{\pgfqpoint{2.395345in}{2.534175in}}%
\pgfpathlineto{\pgfqpoint{2.395136in}{2.232773in}}%
\pgfpathlineto{\pgfqpoint{2.395677in}{2.331685in}}%
\pgfpathlineto{\pgfqpoint{2.395714in}{2.293334in}}%
\pgfpathlineto{\pgfqpoint{2.395825in}{2.373196in}}%
\pgfpathlineto{\pgfqpoint{2.396576in}{2.529216in}}%
\pgfpathlineto{\pgfqpoint{2.396366in}{2.202981in}}%
\pgfpathlineto{\pgfqpoint{2.396920in}{2.335734in}}%
\pgfpathlineto{\pgfqpoint{2.397609in}{2.215886in}}%
\pgfpathlineto{\pgfqpoint{2.397105in}{2.503737in}}%
\pgfpathlineto{\pgfqpoint{2.397794in}{2.491375in}}%
\pgfpathlineto{\pgfqpoint{2.397806in}{2.516996in}}%
\pgfpathlineto{\pgfqpoint{2.398016in}{2.277342in}}%
\pgfpathlineto{\pgfqpoint{2.398803in}{2.287320in}}%
\pgfpathlineto{\pgfqpoint{2.398840in}{2.242799in}}%
\pgfpathlineto{\pgfqpoint{2.399049in}{2.556832in}}%
\pgfpathlineto{\pgfqpoint{2.399837in}{2.357128in}}%
\pgfpathlineto{\pgfqpoint{2.400280in}{2.550271in}}%
\pgfpathlineto{\pgfqpoint{2.400489in}{2.252686in}}%
\pgfpathlineto{\pgfqpoint{2.400945in}{2.415405in}}%
\pgfpathlineto{\pgfqpoint{2.401314in}{2.224403in}}%
\pgfpathlineto{\pgfqpoint{2.401523in}{2.563481in}}%
\pgfpathlineto{\pgfqpoint{2.402028in}{2.479132in}}%
\pgfpathlineto{\pgfqpoint{2.402766in}{2.550542in}}%
\pgfpathlineto{\pgfqpoint{2.402557in}{2.239045in}}%
\pgfpathlineto{\pgfqpoint{2.403037in}{2.329232in}}%
\pgfpathlineto{\pgfqpoint{2.403123in}{2.297603in}}%
\pgfpathlineto{\pgfqpoint{2.403259in}{2.432154in}}%
\pgfpathlineto{\pgfqpoint{2.403997in}{2.564251in}}%
\pgfpathlineto{\pgfqpoint{2.403788in}{2.254601in}}%
\pgfpathlineto{\pgfqpoint{2.404317in}{2.331125in}}%
\pgfpathlineto{\pgfqpoint{2.405240in}{2.534088in}}%
\pgfpathlineto{\pgfqpoint{2.405437in}{2.261729in}}%
\pgfpathlineto{\pgfqpoint{2.406471in}{2.561672in}}%
\pgfpathlineto{\pgfqpoint{2.406286in}{2.256395in}}%
\pgfpathlineto{\pgfqpoint{2.406631in}{2.342099in}}%
\pgfpathlineto{\pgfqpoint{2.407554in}{2.246716in}}%
\pgfpathlineto{\pgfqpoint{2.406976in}{2.488113in}}%
\pgfpathlineto{\pgfqpoint{2.407677in}{2.475133in}}%
\pgfpathlineto{\pgfqpoint{2.407714in}{2.555867in}}%
\pgfpathlineto{\pgfqpoint{2.408157in}{2.278450in}}%
\pgfpathlineto{\pgfqpoint{2.408723in}{2.310242in}}%
\pgfpathlineto{\pgfqpoint{2.408785in}{2.262585in}}%
\pgfpathlineto{\pgfqpoint{2.408945in}{2.538844in}}%
\pgfpathlineto{\pgfqpoint{2.409437in}{2.433423in}}%
\pgfpathlineto{\pgfqpoint{2.410188in}{2.536203in}}%
\pgfpathlineto{\pgfqpoint{2.410028in}{2.249307in}}%
\pgfpathlineto{\pgfqpoint{2.410508in}{2.310477in}}%
\pgfpathlineto{\pgfqpoint{2.411259in}{2.258758in}}%
\pgfpathlineto{\pgfqpoint{2.411419in}{2.542063in}}%
\pgfpathlineto{\pgfqpoint{2.411616in}{2.298480in}}%
\pgfpathlineto{\pgfqpoint{2.412502in}{2.229993in}}%
\pgfpathlineto{\pgfqpoint{2.411936in}{2.477685in}}%
\pgfpathlineto{\pgfqpoint{2.412526in}{2.356036in}}%
\pgfpathlineto{\pgfqpoint{2.412662in}{2.547882in}}%
\pgfpathlineto{\pgfqpoint{2.413006in}{2.256819in}}%
\pgfpathlineto{\pgfqpoint{2.413622in}{2.286003in}}%
\pgfpathlineto{\pgfqpoint{2.413732in}{2.239342in}}%
\pgfpathlineto{\pgfqpoint{2.413892in}{2.528366in}}%
\pgfpathlineto{\pgfqpoint{2.414286in}{2.445237in}}%
\pgfpathlineto{\pgfqpoint{2.415136in}{2.521417in}}%
\pgfpathlineto{\pgfqpoint{2.414976in}{2.220693in}}%
\pgfpathlineto{\pgfqpoint{2.415345in}{2.321942in}}%
\pgfpathlineto{\pgfqpoint{2.416366in}{2.535235in}}%
\pgfpathlineto{\pgfqpoint{2.416206in}{2.259316in}}%
\pgfpathlineto{\pgfqpoint{2.416514in}{2.348295in}}%
\pgfpathlineto{\pgfqpoint{2.417449in}{2.228412in}}%
\pgfpathlineto{\pgfqpoint{2.416896in}{2.484303in}}%
\pgfpathlineto{\pgfqpoint{2.417486in}{2.406527in}}%
\pgfpathlineto{\pgfqpoint{2.417609in}{2.540066in}}%
\pgfpathlineto{\pgfqpoint{2.417966in}{2.262769in}}%
\pgfpathlineto{\pgfqpoint{2.418569in}{2.305929in}}%
\pgfpathlineto{\pgfqpoint{2.418692in}{2.230362in}}%
\pgfpathlineto{\pgfqpoint{2.418840in}{2.507899in}}%
\pgfpathlineto{\pgfqpoint{2.419628in}{2.400248in}}%
\pgfpathlineto{\pgfqpoint{2.420083in}{2.526157in}}%
\pgfpathlineto{\pgfqpoint{2.419923in}{2.201609in}}%
\pgfpathlineto{\pgfqpoint{2.420723in}{2.413017in}}%
\pgfpathlineto{\pgfqpoint{2.421166in}{2.224773in}}%
\pgfpathlineto{\pgfqpoint{2.421326in}{2.514418in}}%
\pgfpathlineto{\pgfqpoint{2.421819in}{2.413523in}}%
\pgfpathlineto{\pgfqpoint{2.422557in}{2.523391in}}%
\pgfpathlineto{\pgfqpoint{2.422397in}{2.234256in}}%
\pgfpathlineto{\pgfqpoint{2.422889in}{2.317387in}}%
\pgfpathlineto{\pgfqpoint{2.423640in}{2.214287in}}%
\pgfpathlineto{\pgfqpoint{2.423812in}{2.516518in}}%
\pgfpathlineto{\pgfqpoint{2.423985in}{2.345295in}}%
\pgfpathlineto{\pgfqpoint{2.424317in}{2.483414in}}%
\pgfpathlineto{\pgfqpoint{2.424046in}{2.262386in}}%
\pgfpathlineto{\pgfqpoint{2.424748in}{2.308850in}}%
\pgfpathlineto{\pgfqpoint{2.424883in}{2.217667in}}%
\pgfpathlineto{\pgfqpoint{2.425043in}{2.529079in}}%
\pgfpathlineto{\pgfqpoint{2.425819in}{2.410296in}}%
\pgfpathlineto{\pgfqpoint{2.426286in}{2.517330in}}%
\pgfpathlineto{\pgfqpoint{2.426114in}{2.228051in}}%
\pgfpathlineto{\pgfqpoint{2.426914in}{2.393637in}}%
\pgfpathlineto{\pgfqpoint{2.427357in}{2.245418in}}%
\pgfpathlineto{\pgfqpoint{2.427517in}{2.514673in}}%
\pgfpathlineto{\pgfqpoint{2.427997in}{2.381272in}}%
\pgfpathlineto{\pgfqpoint{2.428760in}{2.506741in}}%
\pgfpathlineto{\pgfqpoint{2.428588in}{2.230697in}}%
\pgfpathlineto{\pgfqpoint{2.429080in}{2.317172in}}%
\pgfpathlineto{\pgfqpoint{2.429732in}{2.241497in}}%
\pgfpathlineto{\pgfqpoint{2.429991in}{2.523369in}}%
\pgfpathlineto{\pgfqpoint{2.430163in}{2.343358in}}%
\pgfpathlineto{\pgfqpoint{2.430175in}{2.343581in}}%
\pgfpathlineto{\pgfqpoint{2.430200in}{2.332863in}}%
\pgfpathlineto{\pgfqpoint{2.431062in}{2.248191in}}%
\pgfpathlineto{\pgfqpoint{2.431209in}{2.481646in}}%
\pgfpathlineto{\pgfqpoint{2.431234in}{2.512119in}}%
\pgfpathlineto{\pgfqpoint{2.431689in}{2.266454in}}%
\pgfpathlineto{\pgfqpoint{2.432169in}{2.352439in}}%
\pgfpathlineto{\pgfqpoint{2.432305in}{2.252238in}}%
\pgfpathlineto{\pgfqpoint{2.432465in}{2.491193in}}%
\pgfpathlineto{\pgfqpoint{2.433252in}{2.435249in}}%
\pgfpathlineto{\pgfqpoint{2.433708in}{2.480359in}}%
\pgfpathlineto{\pgfqpoint{2.433535in}{2.262173in}}%
\pgfpathlineto{\pgfqpoint{2.434299in}{2.398293in}}%
\pgfpathlineto{\pgfqpoint{2.435185in}{2.262564in}}%
\pgfpathlineto{\pgfqpoint{2.434926in}{2.485196in}}%
\pgfpathlineto{\pgfqpoint{2.435332in}{2.435184in}}%
\pgfpathlineto{\pgfqpoint{2.436182in}{2.491281in}}%
\pgfpathlineto{\pgfqpoint{2.436022in}{2.276866in}}%
\pgfpathlineto{\pgfqpoint{2.436379in}{2.342417in}}%
\pgfpathlineto{\pgfqpoint{2.436428in}{2.270106in}}%
\pgfpathlineto{\pgfqpoint{2.437388in}{2.457906in}}%
\pgfpathlineto{\pgfqpoint{2.437425in}{2.470049in}}%
\pgfpathlineto{\pgfqpoint{2.437880in}{2.287032in}}%
\pgfpathlineto{\pgfqpoint{2.438262in}{2.401184in}}%
\pgfpathlineto{\pgfqpoint{2.438495in}{2.276740in}}%
\pgfpathlineto{\pgfqpoint{2.439062in}{2.474597in}}%
\pgfpathlineto{\pgfqpoint{2.439369in}{2.379865in}}%
\pgfpathlineto{\pgfqpoint{2.440145in}{2.276704in}}%
\pgfpathlineto{\pgfqpoint{2.439899in}{2.473163in}}%
\pgfpathlineto{\pgfqpoint{2.440292in}{2.448309in}}%
\pgfpathlineto{\pgfqpoint{2.441129in}{2.495651in}}%
\pgfpathlineto{\pgfqpoint{2.440772in}{2.293987in}}%
\pgfpathlineto{\pgfqpoint{2.441326in}{2.334310in}}%
\pgfpathlineto{\pgfqpoint{2.441388in}{2.272077in}}%
\pgfpathlineto{\pgfqpoint{2.441548in}{2.462089in}}%
\pgfpathlineto{\pgfqpoint{2.442323in}{2.421196in}}%
\pgfpathlineto{\pgfqpoint{2.442385in}{2.479569in}}%
\pgfpathlineto{\pgfqpoint{2.442619in}{2.287995in}}%
\pgfpathlineto{\pgfqpoint{2.443394in}{2.329680in}}%
\pgfpathlineto{\pgfqpoint{2.443615in}{2.475642in}}%
\pgfpathlineto{\pgfqpoint{2.443862in}{2.288505in}}%
\pgfpathlineto{\pgfqpoint{2.444477in}{2.324249in}}%
\pgfpathlineto{\pgfqpoint{2.445105in}{2.286600in}}%
\pgfpathlineto{\pgfqpoint{2.445265in}{2.465814in}}%
\pgfpathlineto{\pgfqpoint{2.445548in}{2.370626in}}%
\pgfpathlineto{\pgfqpoint{2.446077in}{2.478999in}}%
\pgfpathlineto{\pgfqpoint{2.446335in}{2.273787in}}%
\pgfpathlineto{\pgfqpoint{2.446680in}{2.393378in}}%
\pgfpathlineto{\pgfqpoint{2.447578in}{2.277348in}}%
\pgfpathlineto{\pgfqpoint{2.447320in}{2.485351in}}%
\pgfpathlineto{\pgfqpoint{2.447812in}{2.341104in}}%
\pgfpathlineto{\pgfqpoint{2.448563in}{2.487004in}}%
\pgfpathlineto{\pgfqpoint{2.448809in}{2.292258in}}%
\pgfpathlineto{\pgfqpoint{2.448908in}{2.341462in}}%
\pgfpathlineto{\pgfqpoint{2.449794in}{2.484570in}}%
\pgfpathlineto{\pgfqpoint{2.450052in}{2.283114in}}%
\pgfpathlineto{\pgfqpoint{2.451037in}{2.481940in}}%
\pgfpathlineto{\pgfqpoint{2.451197in}{2.339910in}}%
\pgfpathlineto{\pgfqpoint{2.451295in}{2.288268in}}%
\pgfpathlineto{\pgfqpoint{2.451455in}{2.468113in}}%
\pgfpathlineto{\pgfqpoint{2.452182in}{2.403064in}}%
\pgfpathlineto{\pgfqpoint{2.452280in}{2.486783in}}%
\pgfpathlineto{\pgfqpoint{2.452538in}{2.288715in}}%
\pgfpathlineto{\pgfqpoint{2.453142in}{2.329063in}}%
\pgfpathlineto{\pgfqpoint{2.453769in}{2.293397in}}%
\pgfpathlineto{\pgfqpoint{2.453511in}{2.487596in}}%
\pgfpathlineto{\pgfqpoint{2.454212in}{2.394553in}}%
\pgfpathlineto{\pgfqpoint{2.454754in}{2.482804in}}%
\pgfpathlineto{\pgfqpoint{2.454532in}{2.305242in}}%
\pgfpathlineto{\pgfqpoint{2.454963in}{2.326847in}}%
\pgfpathlineto{\pgfqpoint{2.455012in}{2.289645in}}%
\pgfpathlineto{\pgfqpoint{2.455160in}{2.433417in}}%
\pgfpathlineto{\pgfqpoint{2.455997in}{2.478847in}}%
\pgfpathlineto{\pgfqpoint{2.455640in}{2.300121in}}%
\pgfpathlineto{\pgfqpoint{2.456218in}{2.334316in}}%
\pgfpathlineto{\pgfqpoint{2.456255in}{2.301246in}}%
\pgfpathlineto{\pgfqpoint{2.457215in}{2.470900in}}%
\pgfpathlineto{\pgfqpoint{2.457228in}{2.490233in}}%
\pgfpathlineto{\pgfqpoint{2.457486in}{2.301581in}}%
\pgfpathlineto{\pgfqpoint{2.458225in}{2.323465in}}%
\pgfpathlineto{\pgfqpoint{2.458729in}{2.303381in}}%
\pgfpathlineto{\pgfqpoint{2.458471in}{2.484112in}}%
\pgfpathlineto{\pgfqpoint{2.458877in}{2.422310in}}%
\pgfpathlineto{\pgfqpoint{2.459714in}{2.481452in}}%
\pgfpathlineto{\pgfqpoint{2.459492in}{2.307400in}}%
\pgfpathlineto{\pgfqpoint{2.459935in}{2.328710in}}%
\pgfpathlineto{\pgfqpoint{2.459972in}{2.298368in}}%
\pgfpathlineto{\pgfqpoint{2.460132in}{2.463258in}}%
\pgfpathlineto{\pgfqpoint{2.460908in}{2.398720in}}%
\pgfpathlineto{\pgfqpoint{2.460945in}{2.470894in}}%
\pgfpathlineto{\pgfqpoint{2.461966in}{2.307240in}}%
\pgfpathlineto{\pgfqpoint{2.461978in}{2.311935in}}%
\pgfpathlineto{\pgfqpoint{2.462188in}{2.483668in}}%
\pgfpathlineto{\pgfqpoint{2.462446in}{2.301459in}}%
\pgfpathlineto{\pgfqpoint{2.463135in}{2.398776in}}%
\pgfpathlineto{\pgfqpoint{2.463689in}{2.306385in}}%
\pgfpathlineto{\pgfqpoint{2.463431in}{2.472084in}}%
\pgfpathlineto{\pgfqpoint{2.464231in}{2.417013in}}%
\pgfpathlineto{\pgfqpoint{2.464255in}{2.472134in}}%
\pgfpathlineto{\pgfqpoint{2.464452in}{2.305350in}}%
\pgfpathlineto{\pgfqpoint{2.465301in}{2.336175in}}%
\pgfpathlineto{\pgfqpoint{2.465905in}{2.463834in}}%
\pgfpathlineto{\pgfqpoint{2.465695in}{2.309193in}}%
\pgfpathlineto{\pgfqpoint{2.466508in}{2.377568in}}%
\pgfpathlineto{\pgfqpoint{2.467406in}{2.306760in}}%
\pgfpathlineto{\pgfqpoint{2.467135in}{2.467868in}}%
\pgfpathlineto{\pgfqpoint{2.467628in}{2.342818in}}%
\pgfpathlineto{\pgfqpoint{2.468378in}{2.462043in}}%
\pgfpathlineto{\pgfqpoint{2.468169in}{2.311068in}}%
\pgfpathlineto{\pgfqpoint{2.468625in}{2.328195in}}%
\pgfpathlineto{\pgfqpoint{2.469412in}{2.306191in}}%
\pgfpathlineto{\pgfqpoint{2.469215in}{2.460462in}}%
\pgfpathlineto{\pgfqpoint{2.469597in}{2.444060in}}%
\pgfpathlineto{\pgfqpoint{2.469621in}{2.461502in}}%
\pgfpathlineto{\pgfqpoint{2.469880in}{2.320554in}}%
\pgfpathlineto{\pgfqpoint{2.470606in}{2.353160in}}%
\pgfpathlineto{\pgfqpoint{2.470655in}{2.308781in}}%
\pgfpathlineto{\pgfqpoint{2.470865in}{2.455528in}}%
\pgfpathlineto{\pgfqpoint{2.471677in}{2.419055in}}%
\pgfpathlineto{\pgfqpoint{2.472108in}{2.455903in}}%
\pgfpathlineto{\pgfqpoint{2.471763in}{2.312925in}}%
\pgfpathlineto{\pgfqpoint{2.472735in}{2.329321in}}%
\pgfpathlineto{\pgfqpoint{2.473338in}{2.457466in}}%
\pgfpathlineto{\pgfqpoint{2.473609in}{2.314737in}}%
\pgfpathlineto{\pgfqpoint{2.473929in}{2.400960in}}%
\pgfpathlineto{\pgfqpoint{2.474237in}{2.315653in}}%
\pgfpathlineto{\pgfqpoint{2.474581in}{2.454677in}}%
\pgfpathlineto{\pgfqpoint{2.475000in}{2.438736in}}%
\pgfpathlineto{\pgfqpoint{2.475825in}{2.459518in}}%
\pgfpathlineto{\pgfqpoint{2.475615in}{2.314551in}}%
\pgfpathlineto{\pgfqpoint{2.476009in}{2.340151in}}%
\pgfpathlineto{\pgfqpoint{2.476723in}{2.319675in}}%
\pgfpathlineto{\pgfqpoint{2.476255in}{2.448011in}}%
\pgfpathlineto{\pgfqpoint{2.477018in}{2.416113in}}%
\pgfpathlineto{\pgfqpoint{2.477068in}{2.450232in}}%
\pgfpathlineto{\pgfqpoint{2.477560in}{2.322228in}}%
\pgfpathlineto{\pgfqpoint{2.477954in}{2.325984in}}%
\pgfpathlineto{\pgfqpoint{2.478311in}{2.450465in}}%
\pgfpathlineto{\pgfqpoint{2.478089in}{2.317787in}}%
\pgfpathlineto{\pgfqpoint{2.479172in}{2.362901in}}%
\pgfpathlineto{\pgfqpoint{2.479332in}{2.321480in}}%
\pgfpathlineto{\pgfqpoint{2.479541in}{2.450948in}}%
\pgfpathlineto{\pgfqpoint{2.480280in}{2.372480in}}%
\pgfpathlineto{\pgfqpoint{2.480575in}{2.312596in}}%
\pgfpathlineto{\pgfqpoint{2.480785in}{2.452044in}}%
\pgfpathlineto{\pgfqpoint{2.481400in}{2.357809in}}%
\pgfpathlineto{\pgfqpoint{2.481806in}{2.317741in}}%
\pgfpathlineto{\pgfqpoint{2.482028in}{2.452229in}}%
\pgfpathlineto{\pgfqpoint{2.482409in}{2.387435in}}%
\pgfpathlineto{\pgfqpoint{2.483271in}{2.454678in}}%
\pgfpathlineto{\pgfqpoint{2.483049in}{2.318409in}}%
\pgfpathlineto{\pgfqpoint{2.483480in}{2.339892in}}%
\pgfpathlineto{\pgfqpoint{2.484501in}{2.442488in}}%
\pgfpathlineto{\pgfqpoint{2.484292in}{2.319659in}}%
\pgfpathlineto{\pgfqpoint{2.484674in}{2.355649in}}%
\pgfpathlineto{\pgfqpoint{2.485535in}{2.309800in}}%
\pgfpathlineto{\pgfqpoint{2.484944in}{2.442096in}}%
\pgfpathlineto{\pgfqpoint{2.485671in}{2.373874in}}%
\pgfpathlineto{\pgfqpoint{2.486188in}{2.450463in}}%
\pgfpathlineto{\pgfqpoint{2.486754in}{2.323518in}}%
\pgfpathlineto{\pgfqpoint{2.486778in}{2.315753in}}%
\pgfpathlineto{\pgfqpoint{2.486988in}{2.448689in}}%
\pgfpathlineto{\pgfqpoint{2.487763in}{2.374851in}}%
\pgfpathlineto{\pgfqpoint{2.488231in}{2.445582in}}%
\pgfpathlineto{\pgfqpoint{2.488009in}{2.311967in}}%
\pgfpathlineto{\pgfqpoint{2.488846in}{2.348773in}}%
\pgfpathlineto{\pgfqpoint{2.489252in}{2.304970in}}%
\pgfpathlineto{\pgfqpoint{2.489068in}{2.445390in}}%
\pgfpathlineto{\pgfqpoint{2.489868in}{2.387478in}}%
\pgfpathlineto{\pgfqpoint{2.489904in}{2.444534in}}%
\pgfpathlineto{\pgfqpoint{2.490495in}{2.305522in}}%
\pgfpathlineto{\pgfqpoint{2.490938in}{2.344410in}}%
\pgfpathlineto{\pgfqpoint{2.490975in}{2.341090in}}%
\pgfpathlineto{\pgfqpoint{2.491123in}{2.416950in}}%
\pgfpathlineto{\pgfqpoint{2.491148in}{2.451640in}}%
\pgfpathlineto{\pgfqpoint{2.491726in}{2.313361in}}%
\pgfpathlineto{\pgfqpoint{2.492194in}{2.345693in}}%
\pgfpathlineto{\pgfqpoint{2.492440in}{2.323488in}}%
\pgfpathlineto{\pgfqpoint{2.492391in}{2.446775in}}%
\pgfpathlineto{\pgfqpoint{2.492748in}{2.414303in}}%
\pgfpathlineto{\pgfqpoint{2.493634in}{2.450049in}}%
\pgfpathlineto{\pgfqpoint{2.492969in}{2.317337in}}%
\pgfpathlineto{\pgfqpoint{2.493806in}{2.352126in}}%
\pgfpathlineto{\pgfqpoint{2.494200in}{2.307470in}}%
\pgfpathlineto{\pgfqpoint{2.494028in}{2.444259in}}%
\pgfpathlineto{\pgfqpoint{2.494840in}{2.404252in}}%
\pgfpathlineto{\pgfqpoint{2.494864in}{2.449915in}}%
\pgfpathlineto{\pgfqpoint{2.495443in}{2.299607in}}%
\pgfpathlineto{\pgfqpoint{2.495923in}{2.354395in}}%
\pgfpathlineto{\pgfqpoint{2.496686in}{2.303838in}}%
\pgfpathlineto{\pgfqpoint{2.496108in}{2.444358in}}%
\pgfpathlineto{\pgfqpoint{2.496895in}{2.416709in}}%
\pgfpathlineto{\pgfqpoint{2.497351in}{2.445248in}}%
\pgfpathlineto{\pgfqpoint{2.497191in}{2.337607in}}%
\pgfpathlineto{\pgfqpoint{2.497388in}{2.340373in}}%
\pgfpathlineto{\pgfqpoint{2.497929in}{2.308377in}}%
\pgfpathlineto{\pgfqpoint{2.497757in}{2.438445in}}%
\pgfpathlineto{\pgfqpoint{2.498471in}{2.369635in}}%
\pgfpathlineto{\pgfqpoint{2.498594in}{2.449819in}}%
\pgfpathlineto{\pgfqpoint{2.499160in}{2.304482in}}%
\pgfpathlineto{\pgfqpoint{2.499566in}{2.345674in}}%
\pgfpathlineto{\pgfqpoint{2.500403in}{2.295777in}}%
\pgfpathlineto{\pgfqpoint{2.499837in}{2.458186in}}%
\pgfpathlineto{\pgfqpoint{2.500551in}{2.375688in}}%
\pgfpathlineto{\pgfqpoint{2.501080in}{2.454404in}}%
\pgfpathlineto{\pgfqpoint{2.501141in}{2.308433in}}%
\pgfpathlineto{\pgfqpoint{2.501634in}{2.320237in}}%
\pgfpathlineto{\pgfqpoint{2.501646in}{2.294147in}}%
\pgfpathlineto{\pgfqpoint{2.502323in}{2.455262in}}%
\pgfpathlineto{\pgfqpoint{2.502692in}{2.401133in}}%
\pgfpathlineto{\pgfqpoint{2.503566in}{2.459215in}}%
\pgfpathlineto{\pgfqpoint{2.502889in}{2.297470in}}%
\pgfpathlineto{\pgfqpoint{2.503751in}{2.349723in}}%
\pgfpathlineto{\pgfqpoint{2.503763in}{2.349253in}}%
\pgfpathlineto{\pgfqpoint{2.503800in}{2.377132in}}%
\pgfpathlineto{\pgfqpoint{2.503861in}{2.367246in}}%
\pgfpathlineto{\pgfqpoint{2.504809in}{2.460784in}}%
\pgfpathlineto{\pgfqpoint{2.504132in}{2.298902in}}%
\pgfpathlineto{\pgfqpoint{2.504957in}{2.359124in}}%
\pgfpathlineto{\pgfqpoint{2.505375in}{2.300370in}}%
\pgfpathlineto{\pgfqpoint{2.505646in}{2.437975in}}%
\pgfpathlineto{\pgfqpoint{2.506015in}{2.378708in}}%
\pgfpathlineto{\pgfqpoint{2.506052in}{2.454254in}}%
\pgfpathlineto{\pgfqpoint{2.506618in}{2.310330in}}%
\pgfpathlineto{\pgfqpoint{2.507111in}{2.341221in}}%
\pgfpathlineto{\pgfqpoint{2.507135in}{2.335625in}}%
\pgfpathlineto{\pgfqpoint{2.507184in}{2.401693in}}%
\pgfpathlineto{\pgfqpoint{2.507258in}{2.377797in}}%
\pgfpathlineto{\pgfqpoint{2.507295in}{2.444556in}}%
\pgfpathlineto{\pgfqpoint{2.507357in}{2.310485in}}%
\pgfpathlineto{\pgfqpoint{2.508354in}{2.336892in}}%
\pgfpathlineto{\pgfqpoint{2.508378in}{2.331545in}}%
\pgfpathlineto{\pgfqpoint{2.508403in}{2.363003in}}%
\pgfpathlineto{\pgfqpoint{2.508526in}{2.442330in}}%
\pgfpathlineto{\pgfqpoint{2.509203in}{2.307807in}}%
\pgfpathlineto{\pgfqpoint{2.509511in}{2.366920in}}%
\pgfpathlineto{\pgfqpoint{2.509831in}{2.309946in}}%
\pgfpathlineto{\pgfqpoint{2.509769in}{2.444278in}}%
\pgfpathlineto{\pgfqpoint{2.510495in}{2.410530in}}%
\pgfpathlineto{\pgfqpoint{2.511012in}{2.450245in}}%
\pgfpathlineto{\pgfqpoint{2.511074in}{2.312764in}}%
\pgfpathlineto{\pgfqpoint{2.511554in}{2.377583in}}%
\pgfpathlineto{\pgfqpoint{2.512317in}{2.308062in}}%
\pgfpathlineto{\pgfqpoint{2.511861in}{2.448319in}}%
\pgfpathlineto{\pgfqpoint{2.512649in}{2.408586in}}%
\pgfpathlineto{\pgfqpoint{2.513498in}{2.449756in}}%
\pgfpathlineto{\pgfqpoint{2.513560in}{2.317504in}}%
\pgfpathlineto{\pgfqpoint{2.513732in}{2.377274in}}%
\pgfpathlineto{\pgfqpoint{2.514175in}{2.310590in}}%
\pgfpathlineto{\pgfqpoint{2.514741in}{2.443894in}}%
\pgfpathlineto{\pgfqpoint{2.514827in}{2.389324in}}%
\pgfpathlineto{\pgfqpoint{2.515578in}{2.443833in}}%
\pgfpathlineto{\pgfqpoint{2.515418in}{2.318951in}}%
\pgfpathlineto{\pgfqpoint{2.515923in}{2.345150in}}%
\pgfpathlineto{\pgfqpoint{2.515984in}{2.448830in}}%
\pgfpathlineto{\pgfqpoint{2.516046in}{2.315269in}}%
\pgfpathlineto{\pgfqpoint{2.517018in}{2.347139in}}%
\pgfpathlineto{\pgfqpoint{2.517289in}{2.320714in}}%
\pgfpathlineto{\pgfqpoint{2.517227in}{2.444714in}}%
\pgfpathlineto{\pgfqpoint{2.518027in}{2.396880in}}%
\pgfpathlineto{\pgfqpoint{2.518458in}{2.449617in}}%
\pgfpathlineto{\pgfqpoint{2.519024in}{2.321796in}}%
\pgfpathlineto{\pgfqpoint{2.519111in}{2.331666in}}%
\pgfpathlineto{\pgfqpoint{2.519135in}{2.317554in}}%
\pgfpathlineto{\pgfqpoint{2.519307in}{2.445949in}}%
\pgfpathlineto{\pgfqpoint{2.520095in}{2.394589in}}%
\pgfpathlineto{\pgfqpoint{2.520538in}{2.442424in}}%
\pgfpathlineto{\pgfqpoint{2.521018in}{2.322768in}}%
\pgfpathlineto{\pgfqpoint{2.521178in}{2.363167in}}%
\pgfpathlineto{\pgfqpoint{2.521621in}{2.323632in}}%
\pgfpathlineto{\pgfqpoint{2.521781in}{2.441842in}}%
\pgfpathlineto{\pgfqpoint{2.522163in}{2.392017in}}%
\pgfpathlineto{\pgfqpoint{2.522187in}{2.435802in}}%
\pgfpathlineto{\pgfqpoint{2.522864in}{2.328676in}}%
\pgfpathlineto{\pgfqpoint{2.523246in}{2.334742in}}%
\pgfpathlineto{\pgfqpoint{2.524267in}{2.440521in}}%
\pgfpathlineto{\pgfqpoint{2.524107in}{2.319715in}}%
\pgfpathlineto{\pgfqpoint{2.524415in}{2.366160in}}%
\pgfpathlineto{\pgfqpoint{2.525351in}{2.324460in}}%
\pgfpathlineto{\pgfqpoint{2.524674in}{2.440278in}}%
\pgfpathlineto{\pgfqpoint{2.525461in}{2.373377in}}%
\pgfpathlineto{\pgfqpoint{2.525511in}{2.441125in}}%
\pgfpathlineto{\pgfqpoint{2.525978in}{2.318921in}}%
\pgfpathlineto{\pgfqpoint{2.526557in}{2.347133in}}%
\pgfpathlineto{\pgfqpoint{2.526594in}{2.321355in}}%
\pgfpathlineto{\pgfqpoint{2.526754in}{2.448214in}}%
\pgfpathlineto{\pgfqpoint{2.527640in}{2.342113in}}%
\pgfpathlineto{\pgfqpoint{2.527997in}{2.440620in}}%
\pgfpathlineto{\pgfqpoint{2.528464in}{2.318648in}}%
\pgfpathlineto{\pgfqpoint{2.528797in}{2.412096in}}%
\pgfpathlineto{\pgfqpoint{2.529067in}{2.310302in}}%
\pgfpathlineto{\pgfqpoint{2.529240in}{2.446504in}}%
\pgfpathlineto{\pgfqpoint{2.530003in}{2.377128in}}%
\pgfpathlineto{\pgfqpoint{2.530470in}{2.441102in}}%
\pgfpathlineto{\pgfqpoint{2.530310in}{2.305813in}}%
\pgfpathlineto{\pgfqpoint{2.531098in}{2.377224in}}%
\pgfpathlineto{\pgfqpoint{2.531554in}{2.307291in}}%
\pgfpathlineto{\pgfqpoint{2.531714in}{2.449610in}}%
\pgfpathlineto{\pgfqpoint{2.532107in}{2.432320in}}%
\pgfpathlineto{\pgfqpoint{2.532957in}{2.450445in}}%
\pgfpathlineto{\pgfqpoint{2.532797in}{2.310448in}}%
\pgfpathlineto{\pgfqpoint{2.533006in}{2.338354in}}%
\pgfpathlineto{\pgfqpoint{2.533412in}{2.316629in}}%
\pgfpathlineto{\pgfqpoint{2.533474in}{2.440977in}}%
\pgfpathlineto{\pgfqpoint{2.533954in}{2.366025in}}%
\pgfpathlineto{\pgfqpoint{2.534200in}{2.456191in}}%
\pgfpathlineto{\pgfqpoint{2.534027in}{2.310287in}}%
\pgfpathlineto{\pgfqpoint{2.535049in}{2.382366in}}%
\pgfpathlineto{\pgfqpoint{2.535270in}{2.305747in}}%
\pgfpathlineto{\pgfqpoint{2.535430in}{2.452435in}}%
\pgfpathlineto{\pgfqpoint{2.536144in}{2.389980in}}%
\pgfpathlineto{\pgfqpoint{2.536674in}{2.449850in}}%
\pgfpathlineto{\pgfqpoint{2.536514in}{2.306457in}}%
\pgfpathlineto{\pgfqpoint{2.537215in}{2.394648in}}%
\pgfpathlineto{\pgfqpoint{2.537757in}{2.309836in}}%
\pgfpathlineto{\pgfqpoint{2.537917in}{2.450054in}}%
\pgfpathlineto{\pgfqpoint{2.538310in}{2.425606in}}%
\pgfpathlineto{\pgfqpoint{2.539160in}{2.452392in}}%
\pgfpathlineto{\pgfqpoint{2.538987in}{2.313951in}}%
\pgfpathlineto{\pgfqpoint{2.539295in}{2.367373in}}%
\pgfpathlineto{\pgfqpoint{2.540230in}{2.312707in}}%
\pgfpathlineto{\pgfqpoint{2.539677in}{2.438504in}}%
\pgfpathlineto{\pgfqpoint{2.540366in}{2.389586in}}%
\pgfpathlineto{\pgfqpoint{2.540403in}{2.453545in}}%
\pgfpathlineto{\pgfqpoint{2.540858in}{2.331835in}}%
\pgfpathlineto{\pgfqpoint{2.541449in}{2.333704in}}%
\pgfpathlineto{\pgfqpoint{2.541474in}{2.315430in}}%
\pgfpathlineto{\pgfqpoint{2.541646in}{2.450060in}}%
\pgfpathlineto{\pgfqpoint{2.542544in}{2.348392in}}%
\pgfpathlineto{\pgfqpoint{2.542889in}{2.450053in}}%
\pgfpathlineto{\pgfqpoint{2.542704in}{2.317924in}}%
\pgfpathlineto{\pgfqpoint{2.543664in}{2.374049in}}%
\pgfpathlineto{\pgfqpoint{2.543947in}{2.319108in}}%
\pgfpathlineto{\pgfqpoint{2.543714in}{2.408012in}}%
\pgfpathlineto{\pgfqpoint{2.544095in}{2.390185in}}%
\pgfpathlineto{\pgfqpoint{2.544132in}{2.450461in}}%
\pgfpathlineto{\pgfqpoint{2.545178in}{2.328047in}}%
\pgfpathlineto{\pgfqpoint{2.545190in}{2.320794in}}%
\pgfpathlineto{\pgfqpoint{2.545375in}{2.449248in}}%
\pgfpathlineto{\pgfqpoint{2.546175in}{2.386603in}}%
\pgfpathlineto{\pgfqpoint{2.546606in}{2.450336in}}%
\pgfpathlineto{\pgfqpoint{2.546433in}{2.322179in}}%
\pgfpathlineto{\pgfqpoint{2.547258in}{2.354436in}}%
\pgfpathlineto{\pgfqpoint{2.547664in}{2.321937in}}%
\pgfpathlineto{\pgfqpoint{2.547849in}{2.446845in}}%
\pgfpathlineto{\pgfqpoint{2.548341in}{2.366970in}}%
\pgfpathlineto{\pgfqpoint{2.549092in}{2.447806in}}%
\pgfpathlineto{\pgfqpoint{2.548907in}{2.324216in}}%
\pgfpathlineto{\pgfqpoint{2.549461in}{2.389026in}}%
\pgfpathlineto{\pgfqpoint{2.550335in}{2.444447in}}%
\pgfpathlineto{\pgfqpoint{2.550150in}{2.327730in}}%
\pgfpathlineto{\pgfqpoint{2.550470in}{2.388466in}}%
\pgfpathlineto{\pgfqpoint{2.551381in}{2.329771in}}%
\pgfpathlineto{\pgfqpoint{2.550864in}{2.426710in}}%
\pgfpathlineto{\pgfqpoint{2.551553in}{2.418530in}}%
\pgfpathlineto{\pgfqpoint{2.551578in}{2.441789in}}%
\pgfpathlineto{\pgfqpoint{2.552477in}{2.329216in}}%
\pgfpathlineto{\pgfqpoint{2.552612in}{2.342490in}}%
\pgfpathlineto{\pgfqpoint{2.552809in}{2.435324in}}%
\pgfpathlineto{\pgfqpoint{2.553720in}{2.324977in}}%
\pgfpathlineto{\pgfqpoint{2.554052in}{2.433198in}}%
\pgfpathlineto{\pgfqpoint{2.554901in}{2.409604in}}%
\pgfpathlineto{\pgfqpoint{2.554963in}{2.324015in}}%
\pgfpathlineto{\pgfqpoint{2.555295in}{2.432918in}}%
\pgfpathlineto{\pgfqpoint{2.556021in}{2.381996in}}%
\pgfpathlineto{\pgfqpoint{2.556538in}{2.437587in}}%
\pgfpathlineto{\pgfqpoint{2.556206in}{2.324319in}}%
\pgfpathlineto{\pgfqpoint{2.557141in}{2.397081in}}%
\pgfpathlineto{\pgfqpoint{2.558052in}{2.323084in}}%
\pgfpathlineto{\pgfqpoint{2.557781in}{2.437325in}}%
\pgfpathlineto{\pgfqpoint{2.558273in}{2.370363in}}%
\pgfpathlineto{\pgfqpoint{2.559024in}{2.435219in}}%
\pgfpathlineto{\pgfqpoint{2.559295in}{2.320474in}}%
\pgfpathlineto{\pgfqpoint{2.559381in}{2.388322in}}%
\pgfpathlineto{\pgfqpoint{2.560267in}{2.429187in}}%
\pgfpathlineto{\pgfqpoint{2.560538in}{2.319669in}}%
\pgfpathlineto{\pgfqpoint{2.561498in}{2.430024in}}%
\pgfpathlineto{\pgfqpoint{2.561658in}{2.342263in}}%
\pgfpathlineto{\pgfqpoint{2.562397in}{2.319346in}}%
\pgfpathlineto{\pgfqpoint{2.562027in}{2.427601in}}%
\pgfpathlineto{\pgfqpoint{2.562643in}{2.401026in}}%
\pgfpathlineto{\pgfqpoint{2.562741in}{2.432164in}}%
\pgfpathlineto{\pgfqpoint{2.563000in}{2.331397in}}%
\pgfpathlineto{\pgfqpoint{2.563012in}{2.313309in}}%
\pgfpathlineto{\pgfqpoint{2.563984in}{2.429053in}}%
\pgfpathlineto{\pgfqpoint{2.564070in}{2.372375in}}%
\pgfpathlineto{\pgfqpoint{2.564821in}{2.420800in}}%
\pgfpathlineto{\pgfqpoint{2.564255in}{2.312588in}}%
\pgfpathlineto{\pgfqpoint{2.565178in}{2.377770in}}%
\pgfpathlineto{\pgfqpoint{2.565658in}{2.428113in}}%
\pgfpathlineto{\pgfqpoint{2.565498in}{2.314719in}}%
\pgfpathlineto{\pgfqpoint{2.566212in}{2.353402in}}%
\pgfpathlineto{\pgfqpoint{2.566729in}{2.312883in}}%
\pgfpathlineto{\pgfqpoint{2.566889in}{2.435367in}}%
\pgfpathlineto{\pgfqpoint{2.567283in}{2.401574in}}%
\pgfpathlineto{\pgfqpoint{2.568132in}{2.438379in}}%
\pgfpathlineto{\pgfqpoint{2.567972in}{2.308312in}}%
\pgfpathlineto{\pgfqpoint{2.568353in}{2.360954in}}%
\pgfpathlineto{\pgfqpoint{2.569215in}{2.307952in}}%
\pgfpathlineto{\pgfqpoint{2.569375in}{2.436252in}}%
\pgfpathlineto{\pgfqpoint{2.569436in}{2.369421in}}%
\pgfpathlineto{\pgfqpoint{2.570187in}{2.434132in}}%
\pgfpathlineto{\pgfqpoint{2.570458in}{2.311140in}}%
\pgfpathlineto{\pgfqpoint{2.570532in}{2.376295in}}%
\pgfpathlineto{\pgfqpoint{2.571073in}{2.313652in}}%
\pgfpathlineto{\pgfqpoint{2.570606in}{2.444791in}}%
\pgfpathlineto{\pgfqpoint{2.571640in}{2.376229in}}%
\pgfpathlineto{\pgfqpoint{2.571689in}{2.309207in}}%
\pgfpathlineto{\pgfqpoint{2.571849in}{2.448200in}}%
\pgfpathlineto{\pgfqpoint{2.572649in}{2.415605in}}%
\pgfpathlineto{\pgfqpoint{2.573092in}{2.448941in}}%
\pgfpathlineto{\pgfqpoint{2.572932in}{2.305020in}}%
\pgfpathlineto{\pgfqpoint{2.573732in}{2.377674in}}%
\pgfpathlineto{\pgfqpoint{2.574175in}{2.303117in}}%
\pgfpathlineto{\pgfqpoint{2.574323in}{2.447662in}}%
\pgfpathlineto{\pgfqpoint{2.574827in}{2.385075in}}%
\pgfpathlineto{\pgfqpoint{2.575566in}{2.453917in}}%
\pgfpathlineto{\pgfqpoint{2.575406in}{2.307557in}}%
\pgfpathlineto{\pgfqpoint{2.575910in}{2.349098in}}%
\pgfpathlineto{\pgfqpoint{2.576649in}{2.309212in}}%
\pgfpathlineto{\pgfqpoint{2.576809in}{2.454834in}}%
\pgfpathlineto{\pgfqpoint{2.576993in}{2.371828in}}%
\pgfpathlineto{\pgfqpoint{2.577215in}{2.438148in}}%
\pgfpathlineto{\pgfqpoint{2.577264in}{2.312812in}}%
\pgfpathlineto{\pgfqpoint{2.577867in}{2.332509in}}%
\pgfpathlineto{\pgfqpoint{2.578507in}{2.309051in}}%
\pgfpathlineto{\pgfqpoint{2.578040in}{2.453943in}}%
\pgfpathlineto{\pgfqpoint{2.578938in}{2.355266in}}%
\pgfpathlineto{\pgfqpoint{2.579283in}{2.457880in}}%
\pgfpathlineto{\pgfqpoint{2.579750in}{2.306446in}}%
\pgfpathlineto{\pgfqpoint{2.580046in}{2.370138in}}%
\pgfpathlineto{\pgfqpoint{2.580366in}{2.305762in}}%
\pgfpathlineto{\pgfqpoint{2.580526in}{2.452901in}}%
\pgfpathlineto{\pgfqpoint{2.581141in}{2.380580in}}%
\pgfpathlineto{\pgfqpoint{2.581756in}{2.452645in}}%
\pgfpathlineto{\pgfqpoint{2.581609in}{2.312898in}}%
\pgfpathlineto{\pgfqpoint{2.582200in}{2.368108in}}%
\pgfpathlineto{\pgfqpoint{2.582224in}{2.310018in}}%
\pgfpathlineto{\pgfqpoint{2.583000in}{2.456292in}}%
\pgfpathlineto{\pgfqpoint{2.583295in}{2.390951in}}%
\pgfpathlineto{\pgfqpoint{2.584083in}{2.306642in}}%
\pgfpathlineto{\pgfqpoint{2.583824in}{2.438351in}}%
\pgfpathlineto{\pgfqpoint{2.584206in}{2.386470in}}%
\pgfpathlineto{\pgfqpoint{2.584243in}{2.455894in}}%
\pgfpathlineto{\pgfqpoint{2.584698in}{2.309680in}}%
\pgfpathlineto{\pgfqpoint{2.585289in}{2.342588in}}%
\pgfpathlineto{\pgfqpoint{2.585941in}{2.306310in}}%
\pgfpathlineto{\pgfqpoint{2.585473in}{2.458148in}}%
\pgfpathlineto{\pgfqpoint{2.586372in}{2.352167in}}%
\pgfpathlineto{\pgfqpoint{2.586716in}{2.456685in}}%
\pgfpathlineto{\pgfqpoint{2.586556in}{2.308648in}}%
\pgfpathlineto{\pgfqpoint{2.587480in}{2.363779in}}%
\pgfpathlineto{\pgfqpoint{2.587799in}{2.310064in}}%
\pgfpathlineto{\pgfqpoint{2.587947in}{2.455429in}}%
\pgfpathlineto{\pgfqpoint{2.588452in}{2.390305in}}%
\pgfpathlineto{\pgfqpoint{2.589190in}{2.456805in}}%
\pgfpathlineto{\pgfqpoint{2.589030in}{2.309485in}}%
\pgfpathlineto{\pgfqpoint{2.589523in}{2.350259in}}%
\pgfpathlineto{\pgfqpoint{2.590273in}{2.308680in}}%
\pgfpathlineto{\pgfqpoint{2.590421in}{2.456776in}}%
\pgfpathlineto{\pgfqpoint{2.590507in}{2.396157in}}%
\pgfpathlineto{\pgfqpoint{2.590532in}{2.443970in}}%
\pgfpathlineto{\pgfqpoint{2.590889in}{2.309132in}}%
\pgfpathlineto{\pgfqpoint{2.591590in}{2.335656in}}%
\pgfpathlineto{\pgfqpoint{2.592132in}{2.314741in}}%
\pgfpathlineto{\pgfqpoint{2.591664in}{2.456393in}}%
\pgfpathlineto{\pgfqpoint{2.592673in}{2.368408in}}%
\pgfpathlineto{\pgfqpoint{2.592747in}{2.315012in}}%
\pgfpathlineto{\pgfqpoint{2.592907in}{2.449295in}}%
\pgfpathlineto{\pgfqpoint{2.593609in}{2.417938in}}%
\pgfpathlineto{\pgfqpoint{2.594138in}{2.451523in}}%
\pgfpathlineto{\pgfqpoint{2.593978in}{2.316061in}}%
\pgfpathlineto{\pgfqpoint{2.594593in}{2.337150in}}%
\pgfpathlineto{\pgfqpoint{2.595221in}{2.316195in}}%
\pgfpathlineto{\pgfqpoint{2.595381in}{2.452160in}}%
\pgfpathlineto{\pgfqpoint{2.595664in}{2.378384in}}%
\pgfpathlineto{\pgfqpoint{2.596612in}{2.445682in}}%
\pgfpathlineto{\pgfqpoint{2.596452in}{2.315649in}}%
\pgfpathlineto{\pgfqpoint{2.596759in}{2.384503in}}%
\pgfpathlineto{\pgfqpoint{2.597695in}{2.316156in}}%
\pgfpathlineto{\pgfqpoint{2.597338in}{2.425640in}}%
\pgfpathlineto{\pgfqpoint{2.597830in}{2.407164in}}%
\pgfpathlineto{\pgfqpoint{2.597855in}{2.443878in}}%
\pgfpathlineto{\pgfqpoint{2.598323in}{2.328149in}}%
\pgfpathlineto{\pgfqpoint{2.598889in}{2.346519in}}%
\pgfpathlineto{\pgfqpoint{2.598926in}{2.314489in}}%
\pgfpathlineto{\pgfqpoint{2.599098in}{2.442093in}}%
\pgfpathlineto{\pgfqpoint{2.599984in}{2.350027in}}%
\pgfpathlineto{\pgfqpoint{2.600329in}{2.442733in}}%
\pgfpathlineto{\pgfqpoint{2.600156in}{2.318464in}}%
\pgfpathlineto{\pgfqpoint{2.601190in}{2.389371in}}%
\pgfpathlineto{\pgfqpoint{2.601399in}{2.314997in}}%
\pgfpathlineto{\pgfqpoint{2.601572in}{2.444764in}}%
\pgfpathlineto{\pgfqpoint{2.602273in}{2.410489in}}%
\pgfpathlineto{\pgfqpoint{2.602815in}{2.440496in}}%
\pgfpathlineto{\pgfqpoint{2.602643in}{2.318346in}}%
\pgfpathlineto{\pgfqpoint{2.603344in}{2.372072in}}%
\pgfpathlineto{\pgfqpoint{2.603873in}{2.313936in}}%
\pgfpathlineto{\pgfqpoint{2.604046in}{2.436522in}}%
\pgfpathlineto{\pgfqpoint{2.604439in}{2.393894in}}%
\pgfpathlineto{\pgfqpoint{2.605289in}{2.437293in}}%
\pgfpathlineto{\pgfqpoint{2.605116in}{2.318572in}}%
\pgfpathlineto{\pgfqpoint{2.605486in}{2.373919in}}%
\pgfpathlineto{\pgfqpoint{2.606347in}{2.315851in}}%
\pgfpathlineto{\pgfqpoint{2.606532in}{2.434965in}}%
\pgfpathlineto{\pgfqpoint{2.607590in}{2.316068in}}%
\pgfpathlineto{\pgfqpoint{2.607726in}{2.370407in}}%
\pgfpathlineto{\pgfqpoint{2.607775in}{2.428888in}}%
\pgfpathlineto{\pgfqpoint{2.608809in}{2.331064in}}%
\pgfpathlineto{\pgfqpoint{2.608833in}{2.318565in}}%
\pgfpathlineto{\pgfqpoint{2.609006in}{2.430886in}}%
\pgfpathlineto{\pgfqpoint{2.609830in}{2.402597in}}%
\pgfpathlineto{\pgfqpoint{2.610249in}{2.429591in}}%
\pgfpathlineto{\pgfqpoint{2.610064in}{2.320346in}}%
\pgfpathlineto{\pgfqpoint{2.610778in}{2.359502in}}%
\pgfpathlineto{\pgfqpoint{2.611307in}{2.321716in}}%
\pgfpathlineto{\pgfqpoint{2.611492in}{2.431382in}}%
\pgfpathlineto{\pgfqpoint{2.611849in}{2.375609in}}%
\pgfpathlineto{\pgfqpoint{2.612722in}{2.428681in}}%
\pgfpathlineto{\pgfqpoint{2.612538in}{2.319630in}}%
\pgfpathlineto{\pgfqpoint{2.612919in}{2.370538in}}%
\pgfpathlineto{\pgfqpoint{2.613781in}{2.319303in}}%
\pgfpathlineto{\pgfqpoint{2.613966in}{2.429972in}}%
\pgfpathlineto{\pgfqpoint{2.614015in}{2.380935in}}%
\pgfpathlineto{\pgfqpoint{2.615012in}{2.317569in}}%
\pgfpathlineto{\pgfqpoint{2.614679in}{2.414038in}}%
\pgfpathlineto{\pgfqpoint{2.615147in}{2.372712in}}%
\pgfpathlineto{\pgfqpoint{2.615209in}{2.430089in}}%
\pgfpathlineto{\pgfqpoint{2.616107in}{2.337770in}}%
\pgfpathlineto{\pgfqpoint{2.616218in}{2.355195in}}%
\pgfpathlineto{\pgfqpoint{2.616255in}{2.321528in}}%
\pgfpathlineto{\pgfqpoint{2.616439in}{2.432987in}}%
\pgfpathlineto{\pgfqpoint{2.617276in}{2.413037in}}%
\pgfpathlineto{\pgfqpoint{2.617486in}{2.319148in}}%
\pgfpathlineto{\pgfqpoint{2.617682in}{2.432965in}}%
\pgfpathlineto{\pgfqpoint{2.618384in}{2.411667in}}%
\pgfpathlineto{\pgfqpoint{2.618926in}{2.436763in}}%
\pgfpathlineto{\pgfqpoint{2.618729in}{2.319819in}}%
\pgfpathlineto{\pgfqpoint{2.619369in}{2.374098in}}%
\pgfpathlineto{\pgfqpoint{2.619959in}{2.322572in}}%
\pgfpathlineto{\pgfqpoint{2.620169in}{2.440655in}}%
\pgfpathlineto{\pgfqpoint{2.620464in}{2.371075in}}%
\pgfpathlineto{\pgfqpoint{2.621399in}{2.440579in}}%
\pgfpathlineto{\pgfqpoint{2.621202in}{2.322606in}}%
\pgfpathlineto{\pgfqpoint{2.621572in}{2.370284in}}%
\pgfpathlineto{\pgfqpoint{2.622446in}{2.319544in}}%
\pgfpathlineto{\pgfqpoint{2.622236in}{2.419095in}}%
\pgfpathlineto{\pgfqpoint{2.622593in}{2.384253in}}%
\pgfpathlineto{\pgfqpoint{2.622642in}{2.438820in}}%
\pgfpathlineto{\pgfqpoint{2.623664in}{2.332178in}}%
\pgfpathlineto{\pgfqpoint{2.623676in}{2.319702in}}%
\pgfpathlineto{\pgfqpoint{2.623886in}{2.440343in}}%
\pgfpathlineto{\pgfqpoint{2.624686in}{2.404184in}}%
\pgfpathlineto{\pgfqpoint{2.625116in}{2.443259in}}%
\pgfpathlineto{\pgfqpoint{2.624919in}{2.320672in}}%
\pgfpathlineto{\pgfqpoint{2.625769in}{2.375920in}}%
\pgfpathlineto{\pgfqpoint{2.626150in}{2.322825in}}%
\pgfpathlineto{\pgfqpoint{2.626359in}{2.445358in}}%
\pgfpathlineto{\pgfqpoint{2.626901in}{2.365866in}}%
\pgfpathlineto{\pgfqpoint{2.627602in}{2.446505in}}%
\pgfpathlineto{\pgfqpoint{2.627393in}{2.317518in}}%
\pgfpathlineto{\pgfqpoint{2.628033in}{2.386420in}}%
\pgfpathlineto{\pgfqpoint{2.628636in}{2.320913in}}%
\pgfpathlineto{\pgfqpoint{2.628833in}{2.446204in}}%
\pgfpathlineto{\pgfqpoint{2.629141in}{2.379586in}}%
\pgfpathlineto{\pgfqpoint{2.630076in}{2.449217in}}%
\pgfpathlineto{\pgfqpoint{2.629867in}{2.318994in}}%
\pgfpathlineto{\pgfqpoint{2.630224in}{2.372747in}}%
\pgfpathlineto{\pgfqpoint{2.631110in}{2.323548in}}%
\pgfpathlineto{\pgfqpoint{2.630901in}{2.427717in}}%
\pgfpathlineto{\pgfqpoint{2.631270in}{2.398412in}}%
\pgfpathlineto{\pgfqpoint{2.631319in}{2.449475in}}%
\pgfpathlineto{\pgfqpoint{2.632205in}{2.331197in}}%
\pgfpathlineto{\pgfqpoint{2.632329in}{2.333295in}}%
\pgfpathlineto{\pgfqpoint{2.632341in}{2.320903in}}%
\pgfpathlineto{\pgfqpoint{2.632550in}{2.450173in}}%
\pgfpathlineto{\pgfqpoint{2.633350in}{2.401686in}}%
\pgfpathlineto{\pgfqpoint{2.633793in}{2.449489in}}%
\pgfpathlineto{\pgfqpoint{2.633584in}{2.322359in}}%
\pgfpathlineto{\pgfqpoint{2.634433in}{2.375711in}}%
\pgfpathlineto{\pgfqpoint{2.634815in}{2.326897in}}%
\pgfpathlineto{\pgfqpoint{2.635036in}{2.451865in}}%
\pgfpathlineto{\pgfqpoint{2.635553in}{2.367992in}}%
\pgfpathlineto{\pgfqpoint{2.636267in}{2.449116in}}%
\pgfpathlineto{\pgfqpoint{2.636058in}{2.328770in}}%
\pgfpathlineto{\pgfqpoint{2.636513in}{2.341308in}}%
\pgfpathlineto{\pgfqpoint{2.636747in}{2.328607in}}%
\pgfpathlineto{\pgfqpoint{2.637510in}{2.450893in}}%
\pgfpathlineto{\pgfqpoint{2.637572in}{2.379078in}}%
\pgfpathlineto{\pgfqpoint{2.638335in}{2.433140in}}%
\pgfpathlineto{\pgfqpoint{2.637990in}{2.328905in}}%
\pgfpathlineto{\pgfqpoint{2.638679in}{2.384614in}}%
\pgfpathlineto{\pgfqpoint{2.638753in}{2.450499in}}%
\pgfpathlineto{\pgfqpoint{2.638999in}{2.331849in}}%
\pgfpathlineto{\pgfqpoint{2.639209in}{2.343596in}}%
\pgfpathlineto{\pgfqpoint{2.640242in}{2.327662in}}%
\pgfpathlineto{\pgfqpoint{2.639984in}{2.448532in}}%
\pgfpathlineto{\pgfqpoint{2.640279in}{2.370169in}}%
\pgfpathlineto{\pgfqpoint{2.641227in}{2.449850in}}%
\pgfpathlineto{\pgfqpoint{2.640464in}{2.325872in}}%
\pgfpathlineto{\pgfqpoint{2.641375in}{2.359282in}}%
\pgfpathlineto{\pgfqpoint{2.641707in}{2.325682in}}%
\pgfpathlineto{\pgfqpoint{2.642052in}{2.433954in}}%
\pgfpathlineto{\pgfqpoint{2.642396in}{2.389697in}}%
\pgfpathlineto{\pgfqpoint{2.642950in}{2.328005in}}%
\pgfpathlineto{\pgfqpoint{2.642470in}{2.452149in}}%
\pgfpathlineto{\pgfqpoint{2.643270in}{2.402729in}}%
\pgfpathlineto{\pgfqpoint{2.643713in}{2.451002in}}%
\pgfpathlineto{\pgfqpoint{2.644193in}{2.327029in}}%
\pgfpathlineto{\pgfqpoint{2.644353in}{2.369308in}}%
\pgfpathlineto{\pgfqpoint{2.644944in}{2.448622in}}%
\pgfpathlineto{\pgfqpoint{2.644599in}{2.332575in}}%
\pgfpathlineto{\pgfqpoint{2.645178in}{2.347629in}}%
\pgfpathlineto{\pgfqpoint{2.645424in}{2.326152in}}%
\pgfpathlineto{\pgfqpoint{2.646175in}{2.438310in}}%
\pgfpathlineto{\pgfqpoint{2.646187in}{2.452360in}}%
\pgfpathlineto{\pgfqpoint{2.646667in}{2.322312in}}%
\pgfpathlineto{\pgfqpoint{2.647196in}{2.339347in}}%
\pgfpathlineto{\pgfqpoint{2.647430in}{2.449434in}}%
\pgfpathlineto{\pgfqpoint{2.647910in}{2.318049in}}%
\pgfpathlineto{\pgfqpoint{2.648292in}{2.359640in}}%
\pgfpathlineto{\pgfqpoint{2.649153in}{2.322902in}}%
\pgfpathlineto{\pgfqpoint{2.648673in}{2.448271in}}%
\pgfpathlineto{\pgfqpoint{2.649375in}{2.393974in}}%
\pgfpathlineto{\pgfqpoint{2.649904in}{2.450853in}}%
\pgfpathlineto{\pgfqpoint{2.649756in}{2.327020in}}%
\pgfpathlineto{\pgfqpoint{2.650359in}{2.360158in}}%
\pgfpathlineto{\pgfqpoint{2.650384in}{2.319330in}}%
\pgfpathlineto{\pgfqpoint{2.651147in}{2.448511in}}%
\pgfpathlineto{\pgfqpoint{2.651442in}{2.384394in}}%
\pgfpathlineto{\pgfqpoint{2.652390in}{2.450108in}}%
\pgfpathlineto{\pgfqpoint{2.651627in}{2.316575in}}%
\pgfpathlineto{\pgfqpoint{2.652525in}{2.363380in}}%
\pgfpathlineto{\pgfqpoint{2.652858in}{2.319836in}}%
\pgfpathlineto{\pgfqpoint{2.653215in}{2.434464in}}%
\pgfpathlineto{\pgfqpoint{2.653584in}{2.385583in}}%
\pgfpathlineto{\pgfqpoint{2.653633in}{2.452333in}}%
\pgfpathlineto{\pgfqpoint{2.654101in}{2.318518in}}%
\pgfpathlineto{\pgfqpoint{2.654667in}{2.354040in}}%
\pgfpathlineto{\pgfqpoint{2.654716in}{2.318495in}}%
\pgfpathlineto{\pgfqpoint{2.654864in}{2.450439in}}%
\pgfpathlineto{\pgfqpoint{2.655664in}{2.394771in}}%
\pgfpathlineto{\pgfqpoint{2.656107in}{2.451722in}}%
\pgfpathlineto{\pgfqpoint{2.655959in}{2.318423in}}%
\pgfpathlineto{\pgfqpoint{2.656759in}{2.378735in}}%
\pgfpathlineto{\pgfqpoint{2.657190in}{2.318788in}}%
\pgfpathlineto{\pgfqpoint{2.657350in}{2.452279in}}%
\pgfpathlineto{\pgfqpoint{2.657842in}{2.361522in}}%
\pgfpathlineto{\pgfqpoint{2.658593in}{2.450691in}}%
\pgfpathlineto{\pgfqpoint{2.658433in}{2.316205in}}%
\pgfpathlineto{\pgfqpoint{2.658938in}{2.346124in}}%
\pgfpathlineto{\pgfqpoint{2.659676in}{2.318928in}}%
\pgfpathlineto{\pgfqpoint{2.659812in}{2.434480in}}%
\pgfpathlineto{\pgfqpoint{2.659836in}{2.450806in}}%
\pgfpathlineto{\pgfqpoint{2.660292in}{2.331400in}}%
\pgfpathlineto{\pgfqpoint{2.660858in}{2.362415in}}%
\pgfpathlineto{\pgfqpoint{2.660907in}{2.322970in}}%
\pgfpathlineto{\pgfqpoint{2.661067in}{2.451427in}}%
\pgfpathlineto{\pgfqpoint{2.661965in}{2.350965in}}%
\pgfpathlineto{\pgfqpoint{2.662310in}{2.450084in}}%
\pgfpathlineto{\pgfqpoint{2.662150in}{2.321121in}}%
\pgfpathlineto{\pgfqpoint{2.663122in}{2.407351in}}%
\pgfpathlineto{\pgfqpoint{2.663553in}{2.448056in}}%
\pgfpathlineto{\pgfqpoint{2.663393in}{2.324091in}}%
\pgfpathlineto{\pgfqpoint{2.664168in}{2.385164in}}%
\pgfpathlineto{\pgfqpoint{2.664624in}{2.328995in}}%
\pgfpathlineto{\pgfqpoint{2.664796in}{2.444207in}}%
\pgfpathlineto{\pgfqpoint{2.665288in}{2.375266in}}%
\pgfpathlineto{\pgfqpoint{2.666039in}{2.443810in}}%
\pgfpathlineto{\pgfqpoint{2.665855in}{2.327571in}}%
\pgfpathlineto{\pgfqpoint{2.666384in}{2.357509in}}%
\pgfpathlineto{\pgfqpoint{2.667270in}{2.444193in}}%
\pgfpathlineto{\pgfqpoint{2.667085in}{2.328483in}}%
\pgfpathlineto{\pgfqpoint{2.667467in}{2.351075in}}%
\pgfpathlineto{\pgfqpoint{2.668328in}{2.330330in}}%
\pgfpathlineto{\pgfqpoint{2.668107in}{2.409468in}}%
\pgfpathlineto{\pgfqpoint{2.668378in}{2.370717in}}%
\pgfpathlineto{\pgfqpoint{2.668513in}{2.445460in}}%
\pgfpathlineto{\pgfqpoint{2.668735in}{2.341627in}}%
\pgfpathlineto{\pgfqpoint{2.669473in}{2.370111in}}%
\pgfpathlineto{\pgfqpoint{2.669559in}{2.330239in}}%
\pgfpathlineto{\pgfqpoint{2.669756in}{2.440966in}}%
\pgfpathlineto{\pgfqpoint{2.670556in}{2.378747in}}%
\pgfpathlineto{\pgfqpoint{2.670999in}{2.440792in}}%
\pgfpathlineto{\pgfqpoint{2.670802in}{2.329923in}}%
\pgfpathlineto{\pgfqpoint{2.671651in}{2.367488in}}%
\pgfpathlineto{\pgfqpoint{2.672045in}{2.333747in}}%
\pgfpathlineto{\pgfqpoint{2.672242in}{2.439644in}}%
\pgfpathlineto{\pgfqpoint{2.672735in}{2.371959in}}%
\pgfpathlineto{\pgfqpoint{2.673473in}{2.436228in}}%
\pgfpathlineto{\pgfqpoint{2.673276in}{2.334156in}}%
\pgfpathlineto{\pgfqpoint{2.673842in}{2.372437in}}%
\pgfpathlineto{\pgfqpoint{2.673879in}{2.391125in}}%
\pgfpathlineto{\pgfqpoint{2.674716in}{2.438173in}}%
\pgfpathlineto{\pgfqpoint{2.674519in}{2.333068in}}%
\pgfpathlineto{\pgfqpoint{2.674888in}{2.343689in}}%
\pgfpathlineto{\pgfqpoint{2.674901in}{2.342564in}}%
\pgfpathlineto{\pgfqpoint{2.675245in}{2.407513in}}%
\pgfpathlineto{\pgfqpoint{2.675516in}{2.380477in}}%
\pgfpathlineto{\pgfqpoint{2.675959in}{2.437643in}}%
\pgfpathlineto{\pgfqpoint{2.675750in}{2.335545in}}%
\pgfpathlineto{\pgfqpoint{2.676611in}{2.367034in}}%
\pgfpathlineto{\pgfqpoint{2.676993in}{2.332565in}}%
\pgfpathlineto{\pgfqpoint{2.677190in}{2.434252in}}%
\pgfpathlineto{\pgfqpoint{2.677695in}{2.377412in}}%
\pgfpathlineto{\pgfqpoint{2.678433in}{2.439717in}}%
\pgfpathlineto{\pgfqpoint{2.678224in}{2.332346in}}%
\pgfpathlineto{\pgfqpoint{2.678815in}{2.388826in}}%
\pgfpathlineto{\pgfqpoint{2.679676in}{2.437605in}}%
\pgfpathlineto{\pgfqpoint{2.679467in}{2.332563in}}%
\pgfpathlineto{\pgfqpoint{2.679848in}{2.345992in}}%
\pgfpathlineto{\pgfqpoint{2.680710in}{2.329045in}}%
\pgfpathlineto{\pgfqpoint{2.680501in}{2.414088in}}%
\pgfpathlineto{\pgfqpoint{2.680833in}{2.383911in}}%
\pgfpathlineto{\pgfqpoint{2.680907in}{2.436366in}}%
\pgfpathlineto{\pgfqpoint{2.681104in}{2.338140in}}%
\pgfpathlineto{\pgfqpoint{2.681891in}{2.361113in}}%
\pgfpathlineto{\pgfqpoint{2.681941in}{2.327303in}}%
\pgfpathlineto{\pgfqpoint{2.682138in}{2.433312in}}%
\pgfpathlineto{\pgfqpoint{2.682950in}{2.398970in}}%
\pgfpathlineto{\pgfqpoint{2.683381in}{2.434759in}}%
\pgfpathlineto{\pgfqpoint{2.683184in}{2.325899in}}%
\pgfpathlineto{\pgfqpoint{2.683984in}{2.368935in}}%
\pgfpathlineto{\pgfqpoint{2.684415in}{2.326284in}}%
\pgfpathlineto{\pgfqpoint{2.684611in}{2.433870in}}%
\pgfpathlineto{\pgfqpoint{2.685005in}{2.397703in}}%
\pgfpathlineto{\pgfqpoint{2.685855in}{2.437697in}}%
\pgfpathlineto{\pgfqpoint{2.685658in}{2.327156in}}%
\pgfpathlineto{\pgfqpoint{2.685916in}{2.363011in}}%
\pgfpathlineto{\pgfqpoint{2.686888in}{2.330256in}}%
\pgfpathlineto{\pgfqpoint{2.686679in}{2.416585in}}%
\pgfpathlineto{\pgfqpoint{2.686975in}{2.381000in}}%
\pgfpathlineto{\pgfqpoint{2.687085in}{2.435455in}}%
\pgfpathlineto{\pgfqpoint{2.687307in}{2.340799in}}%
\pgfpathlineto{\pgfqpoint{2.687959in}{2.379606in}}%
\pgfpathlineto{\pgfqpoint{2.688131in}{2.328424in}}%
\pgfpathlineto{\pgfqpoint{2.688328in}{2.436568in}}%
\pgfpathlineto{\pgfqpoint{2.689079in}{2.365992in}}%
\pgfpathlineto{\pgfqpoint{2.689571in}{2.433282in}}%
\pgfpathlineto{\pgfqpoint{2.689375in}{2.332773in}}%
\pgfpathlineto{\pgfqpoint{2.690175in}{2.367687in}}%
\pgfpathlineto{\pgfqpoint{2.690605in}{2.334005in}}%
\pgfpathlineto{\pgfqpoint{2.690802in}{2.433063in}}%
\pgfpathlineto{\pgfqpoint{2.691196in}{2.402005in}}%
\pgfpathlineto{\pgfqpoint{2.692045in}{2.432270in}}%
\pgfpathlineto{\pgfqpoint{2.691848in}{2.335438in}}%
\pgfpathlineto{\pgfqpoint{2.692181in}{2.370462in}}%
\pgfpathlineto{\pgfqpoint{2.693079in}{2.337692in}}%
\pgfpathlineto{\pgfqpoint{2.692870in}{2.418852in}}%
\pgfpathlineto{\pgfqpoint{2.693227in}{2.378570in}}%
\pgfpathlineto{\pgfqpoint{2.693276in}{2.433125in}}%
\pgfpathlineto{\pgfqpoint{2.693498in}{2.345031in}}%
\pgfpathlineto{\pgfqpoint{2.694298in}{2.348716in}}%
\pgfpathlineto{\pgfqpoint{2.694322in}{2.340729in}}%
\pgfpathlineto{\pgfqpoint{2.694519in}{2.435468in}}%
\pgfpathlineto{\pgfqpoint{2.695233in}{2.387942in}}%
\pgfpathlineto{\pgfqpoint{2.695750in}{2.434826in}}%
\pgfpathlineto{\pgfqpoint{2.696008in}{2.342003in}}%
\pgfpathlineto{\pgfqpoint{2.696205in}{2.383694in}}%
\pgfpathlineto{\pgfqpoint{2.697251in}{2.338818in}}%
\pgfpathlineto{\pgfqpoint{2.696993in}{2.434825in}}%
\pgfpathlineto{\pgfqpoint{2.697325in}{2.366486in}}%
\pgfpathlineto{\pgfqpoint{2.698224in}{2.434661in}}%
\pgfpathlineto{\pgfqpoint{2.697658in}{2.339800in}}%
\pgfpathlineto{\pgfqpoint{2.698384in}{2.358893in}}%
\pgfpathlineto{\pgfqpoint{2.698901in}{2.337210in}}%
\pgfpathlineto{\pgfqpoint{2.698642in}{2.421034in}}%
\pgfpathlineto{\pgfqpoint{2.699430in}{2.394400in}}%
\pgfpathlineto{\pgfqpoint{2.699467in}{2.433279in}}%
\pgfpathlineto{\pgfqpoint{2.699725in}{2.336459in}}%
\pgfpathlineto{\pgfqpoint{2.700513in}{2.343627in}}%
\pgfpathlineto{\pgfqpoint{2.700698in}{2.434428in}}%
\pgfpathlineto{\pgfqpoint{2.700956in}{2.339312in}}%
\pgfpathlineto{\pgfqpoint{2.701658in}{2.377958in}}%
\pgfpathlineto{\pgfqpoint{2.702199in}{2.339154in}}%
\pgfpathlineto{\pgfqpoint{2.701941in}{2.432944in}}%
\pgfpathlineto{\pgfqpoint{2.702741in}{2.403299in}}%
\pgfpathlineto{\pgfqpoint{2.703171in}{2.431568in}}%
\pgfpathlineto{\pgfqpoint{2.703442in}{2.341017in}}%
\pgfpathlineto{\pgfqpoint{2.703799in}{2.358418in}}%
\pgfpathlineto{\pgfqpoint{2.704636in}{2.339834in}}%
\pgfpathlineto{\pgfqpoint{2.704414in}{2.432107in}}%
\pgfpathlineto{\pgfqpoint{2.704833in}{2.422415in}}%
\pgfpathlineto{\pgfqpoint{2.705879in}{2.339823in}}%
\pgfpathlineto{\pgfqpoint{2.705658in}{2.429377in}}%
\pgfpathlineto{\pgfqpoint{2.706027in}{2.379731in}}%
\pgfpathlineto{\pgfqpoint{2.706888in}{2.425384in}}%
\pgfpathlineto{\pgfqpoint{2.707110in}{2.336032in}}%
\pgfpathlineto{\pgfqpoint{2.708131in}{2.425508in}}%
\pgfpathlineto{\pgfqpoint{2.708267in}{2.362967in}}%
\pgfpathlineto{\pgfqpoint{2.708353in}{2.338891in}}%
\pgfpathlineto{\pgfqpoint{2.708550in}{2.419182in}}%
\pgfpathlineto{\pgfqpoint{2.709239in}{2.379086in}}%
\pgfpathlineto{\pgfqpoint{2.709362in}{2.427595in}}%
\pgfpathlineto{\pgfqpoint{2.709584in}{2.339973in}}%
\pgfpathlineto{\pgfqpoint{2.710322in}{2.378196in}}%
\pgfpathlineto{\pgfqpoint{2.710827in}{2.339581in}}%
\pgfpathlineto{\pgfqpoint{2.710605in}{2.426925in}}%
\pgfpathlineto{\pgfqpoint{2.711405in}{2.404256in}}%
\pgfpathlineto{\pgfqpoint{2.711848in}{2.426946in}}%
\pgfpathlineto{\pgfqpoint{2.712094in}{2.339617in}}%
\pgfpathlineto{\pgfqpoint{2.712439in}{2.369029in}}%
\pgfpathlineto{\pgfqpoint{2.713337in}{2.338519in}}%
\pgfpathlineto{\pgfqpoint{2.713079in}{2.427270in}}%
\pgfpathlineto{\pgfqpoint{2.713461in}{2.398637in}}%
\pgfpathlineto{\pgfqpoint{2.714322in}{2.424677in}}%
\pgfpathlineto{\pgfqpoint{2.713744in}{2.340352in}}%
\pgfpathlineto{\pgfqpoint{2.714507in}{2.349954in}}%
\pgfpathlineto{\pgfqpoint{2.714568in}{2.338360in}}%
\pgfpathlineto{\pgfqpoint{2.714728in}{2.419120in}}%
\pgfpathlineto{\pgfqpoint{2.715417in}{2.364648in}}%
\pgfpathlineto{\pgfqpoint{2.715553in}{2.426600in}}%
\pgfpathlineto{\pgfqpoint{2.715811in}{2.338299in}}%
\pgfpathlineto{\pgfqpoint{2.716525in}{2.365471in}}%
\pgfpathlineto{\pgfqpoint{2.717042in}{2.341120in}}%
\pgfpathlineto{\pgfqpoint{2.716796in}{2.425217in}}%
\pgfpathlineto{\pgfqpoint{2.717584in}{2.395619in}}%
\pgfpathlineto{\pgfqpoint{2.718027in}{2.425295in}}%
\pgfpathlineto{\pgfqpoint{2.718285in}{2.338984in}}%
\pgfpathlineto{\pgfqpoint{2.718642in}{2.355849in}}%
\pgfpathlineto{\pgfqpoint{2.718704in}{2.343529in}}%
\pgfpathlineto{\pgfqpoint{2.718851in}{2.414180in}}%
\pgfpathlineto{\pgfqpoint{2.719233in}{2.403279in}}%
\pgfpathlineto{\pgfqpoint{2.719270in}{2.424808in}}%
\pgfpathlineto{\pgfqpoint{2.719528in}{2.341939in}}%
\pgfpathlineto{\pgfqpoint{2.720291in}{2.349563in}}%
\pgfpathlineto{\pgfqpoint{2.720501in}{2.424598in}}%
\pgfpathlineto{\pgfqpoint{2.720759in}{2.340675in}}%
\pgfpathlineto{\pgfqpoint{2.721485in}{2.360484in}}%
\pgfpathlineto{\pgfqpoint{2.722002in}{2.340602in}}%
\pgfpathlineto{\pgfqpoint{2.721744in}{2.423796in}}%
\pgfpathlineto{\pgfqpoint{2.722544in}{2.406801in}}%
\pgfpathlineto{\pgfqpoint{2.722974in}{2.425448in}}%
\pgfpathlineto{\pgfqpoint{2.723233in}{2.342966in}}%
\pgfpathlineto{\pgfqpoint{2.723541in}{2.374913in}}%
\pgfpathlineto{\pgfqpoint{2.724476in}{2.339712in}}%
\pgfpathlineto{\pgfqpoint{2.724217in}{2.423728in}}%
\pgfpathlineto{\pgfqpoint{2.724611in}{2.411148in}}%
\pgfpathlineto{\pgfqpoint{2.725448in}{2.422887in}}%
\pgfpathlineto{\pgfqpoint{2.725301in}{2.342378in}}%
\pgfpathlineto{\pgfqpoint{2.725596in}{2.367212in}}%
\pgfpathlineto{\pgfqpoint{2.725657in}{2.341664in}}%
\pgfpathlineto{\pgfqpoint{2.725867in}{2.417501in}}%
\pgfpathlineto{\pgfqpoint{2.726654in}{2.390269in}}%
\pgfpathlineto{\pgfqpoint{2.726691in}{2.424505in}}%
\pgfpathlineto{\pgfqpoint{2.726950in}{2.340860in}}%
\pgfpathlineto{\pgfqpoint{2.727737in}{2.347531in}}%
\pgfpathlineto{\pgfqpoint{2.727922in}{2.424246in}}%
\pgfpathlineto{\pgfqpoint{2.728144in}{2.341748in}}%
\pgfpathlineto{\pgfqpoint{2.728882in}{2.383049in}}%
\pgfpathlineto{\pgfqpoint{2.729424in}{2.340579in}}%
\pgfpathlineto{\pgfqpoint{2.729165in}{2.423053in}}%
\pgfpathlineto{\pgfqpoint{2.729965in}{2.407055in}}%
\pgfpathlineto{\pgfqpoint{2.730396in}{2.423542in}}%
\pgfpathlineto{\pgfqpoint{2.730617in}{2.343967in}}%
\pgfpathlineto{\pgfqpoint{2.730974in}{2.363427in}}%
\pgfpathlineto{\pgfqpoint{2.731897in}{2.342877in}}%
\pgfpathlineto{\pgfqpoint{2.731627in}{2.421264in}}%
\pgfpathlineto{\pgfqpoint{2.732008in}{2.378502in}}%
\pgfpathlineto{\pgfqpoint{2.732857in}{2.420261in}}%
\pgfpathlineto{\pgfqpoint{2.733079in}{2.341148in}}%
\pgfpathlineto{\pgfqpoint{2.733104in}{2.350095in}}%
\pgfpathlineto{\pgfqpoint{2.734100in}{2.421151in}}%
\pgfpathlineto{\pgfqpoint{2.733128in}{2.343205in}}%
\pgfpathlineto{\pgfqpoint{2.734260in}{2.362798in}}%
\pgfpathlineto{\pgfqpoint{2.734322in}{2.341348in}}%
\pgfpathlineto{\pgfqpoint{2.734507in}{2.417455in}}%
\pgfpathlineto{\pgfqpoint{2.735307in}{2.393006in}}%
\pgfpathlineto{\pgfqpoint{2.735331in}{2.420756in}}%
\pgfpathlineto{\pgfqpoint{2.735553in}{2.339684in}}%
\pgfpathlineto{\pgfqpoint{2.736390in}{2.344132in}}%
\pgfpathlineto{\pgfqpoint{2.736562in}{2.420749in}}%
\pgfpathlineto{\pgfqpoint{2.736796in}{2.337902in}}%
\pgfpathlineto{\pgfqpoint{2.737534in}{2.386740in}}%
\pgfpathlineto{\pgfqpoint{2.738027in}{2.338510in}}%
\pgfpathlineto{\pgfqpoint{2.737805in}{2.420461in}}%
\pgfpathlineto{\pgfqpoint{2.738605in}{2.398274in}}%
\pgfpathlineto{\pgfqpoint{2.739036in}{2.423231in}}%
\pgfpathlineto{\pgfqpoint{2.739257in}{2.339568in}}%
\pgfpathlineto{\pgfqpoint{2.739676in}{2.344697in}}%
\pgfpathlineto{\pgfqpoint{2.740267in}{2.424585in}}%
\pgfpathlineto{\pgfqpoint{2.740500in}{2.338427in}}%
\pgfpathlineto{\pgfqpoint{2.740845in}{2.376164in}}%
\pgfpathlineto{\pgfqpoint{2.741731in}{2.335874in}}%
\pgfpathlineto{\pgfqpoint{2.741510in}{2.429700in}}%
\pgfpathlineto{\pgfqpoint{2.741904in}{2.418254in}}%
\pgfpathlineto{\pgfqpoint{2.742740in}{2.432337in}}%
\pgfpathlineto{\pgfqpoint{2.742556in}{2.338035in}}%
\pgfpathlineto{\pgfqpoint{2.742925in}{2.351891in}}%
\pgfpathlineto{\pgfqpoint{2.742937in}{2.351789in}}%
\pgfpathlineto{\pgfqpoint{2.742974in}{2.333553in}}%
\pgfpathlineto{\pgfqpoint{2.743565in}{2.431856in}}%
\pgfpathlineto{\pgfqpoint{2.743959in}{2.419233in}}%
\pgfpathlineto{\pgfqpoint{2.744390in}{2.439629in}}%
\pgfpathlineto{\pgfqpoint{2.744205in}{2.335216in}}%
\pgfpathlineto{\pgfqpoint{2.745017in}{2.340495in}}%
\pgfpathlineto{\pgfqpoint{2.745473in}{2.336404in}}%
\pgfpathlineto{\pgfqpoint{2.745214in}{2.445697in}}%
\pgfpathlineto{\pgfqpoint{2.745596in}{2.398715in}}%
\pgfpathlineto{\pgfqpoint{2.746445in}{2.452255in}}%
\pgfpathlineto{\pgfqpoint{2.746679in}{2.335502in}}%
\pgfpathlineto{\pgfqpoint{2.747676in}{2.454747in}}%
\pgfpathlineto{\pgfqpoint{2.746704in}{2.333362in}}%
\pgfpathlineto{\pgfqpoint{2.747873in}{2.355866in}}%
\pgfpathlineto{\pgfqpoint{2.747934in}{2.331413in}}%
\pgfpathlineto{\pgfqpoint{2.748094in}{2.454169in}}%
\pgfpathlineto{\pgfqpoint{2.748894in}{2.423938in}}%
\pgfpathlineto{\pgfqpoint{2.749325in}{2.457172in}}%
\pgfpathlineto{\pgfqpoint{2.749177in}{2.328583in}}%
\pgfpathlineto{\pgfqpoint{2.749953in}{2.339572in}}%
\pgfpathlineto{\pgfqpoint{2.750408in}{2.324539in}}%
\pgfpathlineto{\pgfqpoint{2.750150in}{2.459285in}}%
\pgfpathlineto{\pgfqpoint{2.750937in}{2.373697in}}%
\pgfpathlineto{\pgfqpoint{2.751380in}{2.458875in}}%
\pgfpathlineto{\pgfqpoint{2.751639in}{2.325257in}}%
\pgfpathlineto{\pgfqpoint{2.752033in}{2.334972in}}%
\pgfpathlineto{\pgfqpoint{2.752870in}{2.325083in}}%
\pgfpathlineto{\pgfqpoint{2.752611in}{2.459741in}}%
\pgfpathlineto{\pgfqpoint{2.752993in}{2.377302in}}%
\pgfpathlineto{\pgfqpoint{2.753030in}{2.458650in}}%
\pgfpathlineto{\pgfqpoint{2.753694in}{2.326672in}}%
\pgfpathlineto{\pgfqpoint{2.754088in}{2.331105in}}%
\pgfpathlineto{\pgfqpoint{2.754100in}{2.327541in}}%
\pgfpathlineto{\pgfqpoint{2.754260in}{2.461069in}}%
\pgfpathlineto{\pgfqpoint{2.755036in}{2.346003in}}%
\pgfpathlineto{\pgfqpoint{2.755491in}{2.461943in}}%
\pgfpathlineto{\pgfqpoint{2.755343in}{2.328794in}}%
\pgfpathlineto{\pgfqpoint{2.756131in}{2.335878in}}%
\pgfpathlineto{\pgfqpoint{2.756574in}{2.325355in}}%
\pgfpathlineto{\pgfqpoint{2.756316in}{2.460079in}}%
\pgfpathlineto{\pgfqpoint{2.756697in}{2.399956in}}%
\pgfpathlineto{\pgfqpoint{2.756722in}{2.458790in}}%
\pgfpathlineto{\pgfqpoint{2.757399in}{2.325911in}}%
\pgfpathlineto{\pgfqpoint{2.757780in}{2.328428in}}%
\pgfpathlineto{\pgfqpoint{2.757854in}{2.389235in}}%
\pgfpathlineto{\pgfqpoint{2.757805in}{2.323523in}}%
\pgfpathlineto{\pgfqpoint{2.757928in}{2.385156in}}%
\pgfpathlineto{\pgfqpoint{2.758371in}{2.457276in}}%
\pgfpathlineto{\pgfqpoint{2.758630in}{2.322942in}}%
\pgfpathlineto{\pgfqpoint{2.759011in}{2.326908in}}%
\pgfpathlineto{\pgfqpoint{2.759023in}{2.326904in}}%
\pgfpathlineto{\pgfqpoint{2.759036in}{2.324243in}}%
\pgfpathlineto{\pgfqpoint{2.759196in}{2.457395in}}%
\pgfpathlineto{\pgfqpoint{2.759590in}{2.441893in}}%
\pgfpathlineto{\pgfqpoint{2.760427in}{2.459895in}}%
\pgfpathlineto{\pgfqpoint{2.759860in}{2.326482in}}%
\pgfpathlineto{\pgfqpoint{2.760623in}{2.373689in}}%
\pgfpathlineto{\pgfqpoint{2.761596in}{2.326166in}}%
\pgfpathlineto{\pgfqpoint{2.761657in}{2.458336in}}%
\pgfpathlineto{\pgfqpoint{2.761731in}{2.355317in}}%
\pgfpathlineto{\pgfqpoint{2.762076in}{2.453535in}}%
\pgfpathlineto{\pgfqpoint{2.762716in}{2.327606in}}%
\pgfpathlineto{\pgfqpoint{2.762814in}{2.355679in}}%
\pgfpathlineto{\pgfqpoint{2.763245in}{2.322811in}}%
\pgfpathlineto{\pgfqpoint{2.763307in}{2.457544in}}%
\pgfpathlineto{\pgfqpoint{2.763922in}{2.361868in}}%
\pgfpathlineto{\pgfqpoint{2.764070in}{2.318249in}}%
\pgfpathlineto{\pgfqpoint{2.764537in}{2.458720in}}%
\pgfpathlineto{\pgfqpoint{2.765030in}{2.365508in}}%
\pgfpathlineto{\pgfqpoint{2.765768in}{2.462596in}}%
\pgfpathlineto{\pgfqpoint{2.765300in}{2.317061in}}%
\pgfpathlineto{\pgfqpoint{2.766113in}{2.331838in}}%
\pgfpathlineto{\pgfqpoint{2.766531in}{2.317801in}}%
\pgfpathlineto{\pgfqpoint{2.766593in}{2.461932in}}%
\pgfpathlineto{\pgfqpoint{2.767183in}{2.367029in}}%
\pgfpathlineto{\pgfqpoint{2.768242in}{2.464666in}}%
\pgfpathlineto{\pgfqpoint{2.768180in}{2.316635in}}%
\pgfpathlineto{\pgfqpoint{2.768279in}{2.368691in}}%
\pgfpathlineto{\pgfqpoint{2.769005in}{2.315595in}}%
\pgfpathlineto{\pgfqpoint{2.769054in}{2.463066in}}%
\pgfpathlineto{\pgfqpoint{2.769362in}{2.409186in}}%
\pgfpathlineto{\pgfqpoint{2.770236in}{2.312103in}}%
\pgfpathlineto{\pgfqpoint{2.769473in}{2.467666in}}%
\pgfpathlineto{\pgfqpoint{2.770556in}{2.322142in}}%
\pgfpathlineto{\pgfqpoint{2.770703in}{2.469672in}}%
\pgfpathlineto{\pgfqpoint{2.771467in}{2.313083in}}%
\pgfpathlineto{\pgfqpoint{2.771676in}{2.363944in}}%
\pgfpathlineto{\pgfqpoint{2.771885in}{2.310630in}}%
\pgfpathlineto{\pgfqpoint{2.771823in}{2.413236in}}%
\pgfpathlineto{\pgfqpoint{2.771910in}{2.400113in}}%
\pgfpathlineto{\pgfqpoint{2.771934in}{2.472945in}}%
\pgfpathlineto{\pgfqpoint{2.772697in}{2.309560in}}%
\pgfpathlineto{\pgfqpoint{2.772993in}{2.331416in}}%
\pgfpathlineto{\pgfqpoint{2.773116in}{2.307755in}}%
\pgfpathlineto{\pgfqpoint{2.773165in}{2.473303in}}%
\pgfpathlineto{\pgfqpoint{2.773953in}{2.340975in}}%
\pgfpathlineto{\pgfqpoint{2.774396in}{2.476754in}}%
\pgfpathlineto{\pgfqpoint{2.774346in}{2.306628in}}%
\pgfpathlineto{\pgfqpoint{2.775060in}{2.325011in}}%
\pgfpathlineto{\pgfqpoint{2.775577in}{2.302681in}}%
\pgfpathlineto{\pgfqpoint{2.775220in}{2.477244in}}%
\pgfpathlineto{\pgfqpoint{2.776020in}{2.412316in}}%
\pgfpathlineto{\pgfqpoint{2.776870in}{2.479739in}}%
\pgfpathlineto{\pgfqpoint{2.776402in}{2.301381in}}%
\pgfpathlineto{\pgfqpoint{2.777103in}{2.324213in}}%
\pgfpathlineto{\pgfqpoint{2.777226in}{2.303453in}}%
\pgfpathlineto{\pgfqpoint{2.777177in}{2.416862in}}%
\pgfpathlineto{\pgfqpoint{2.777251in}{2.397503in}}%
\pgfpathlineto{\pgfqpoint{2.778100in}{2.486579in}}%
\pgfpathlineto{\pgfqpoint{2.777633in}{2.303183in}}%
\pgfpathlineto{\pgfqpoint{2.778346in}{2.322686in}}%
\pgfpathlineto{\pgfqpoint{2.778457in}{2.303762in}}%
\pgfpathlineto{\pgfqpoint{2.778506in}{2.486091in}}%
\pgfpathlineto{\pgfqpoint{2.778900in}{2.421165in}}%
\pgfpathlineto{\pgfqpoint{2.779750in}{2.488066in}}%
\pgfpathlineto{\pgfqpoint{2.779688in}{2.307816in}}%
\pgfpathlineto{\pgfqpoint{2.779983in}{2.321974in}}%
\pgfpathlineto{\pgfqpoint{2.780008in}{2.315780in}}%
\pgfpathlineto{\pgfqpoint{2.780980in}{2.493234in}}%
\pgfpathlineto{\pgfqpoint{2.780513in}{2.300930in}}%
\pgfpathlineto{\pgfqpoint{2.781140in}{2.351765in}}%
\pgfpathlineto{\pgfqpoint{2.781239in}{2.317115in}}%
\pgfpathlineto{\pgfqpoint{2.781177in}{2.374876in}}%
\pgfpathlineto{\pgfqpoint{2.781263in}{2.369338in}}%
\pgfpathlineto{\pgfqpoint{2.782211in}{2.502030in}}%
\pgfpathlineto{\pgfqpoint{2.782162in}{2.298713in}}%
\pgfpathlineto{\pgfqpoint{2.782359in}{2.357685in}}%
\pgfpathlineto{\pgfqpoint{2.782986in}{2.295624in}}%
\pgfpathlineto{\pgfqpoint{2.783036in}{2.506178in}}%
\pgfpathlineto{\pgfqpoint{2.783417in}{2.379916in}}%
\pgfpathlineto{\pgfqpoint{2.784266in}{2.510400in}}%
\pgfpathlineto{\pgfqpoint{2.784217in}{2.292201in}}%
\pgfpathlineto{\pgfqpoint{2.784513in}{2.310074in}}%
\pgfpathlineto{\pgfqpoint{2.785091in}{2.516506in}}%
\pgfpathlineto{\pgfqpoint{2.785448in}{2.292770in}}%
\pgfpathlineto{\pgfqpoint{2.785706in}{2.359729in}}%
\pgfpathlineto{\pgfqpoint{2.786691in}{2.289715in}}%
\pgfpathlineto{\pgfqpoint{2.786740in}{2.523347in}}%
\pgfpathlineto{\pgfqpoint{2.786814in}{2.347610in}}%
\pgfpathlineto{\pgfqpoint{2.787146in}{2.527288in}}%
\pgfpathlineto{\pgfqpoint{2.787097in}{2.285693in}}%
\pgfpathlineto{\pgfqpoint{2.787897in}{2.345328in}}%
\pgfpathlineto{\pgfqpoint{2.787922in}{2.280782in}}%
\pgfpathlineto{\pgfqpoint{2.788796in}{2.536159in}}%
\pgfpathlineto{\pgfqpoint{2.789005in}{2.335250in}}%
\pgfpathlineto{\pgfqpoint{2.789977in}{2.277114in}}%
\pgfpathlineto{\pgfqpoint{2.789620in}{2.544762in}}%
\pgfpathlineto{\pgfqpoint{2.790002in}{2.437044in}}%
\pgfpathlineto{\pgfqpoint{2.790851in}{2.556369in}}%
\pgfpathlineto{\pgfqpoint{2.790383in}{2.273756in}}%
\pgfpathlineto{\pgfqpoint{2.791085in}{2.300512in}}%
\pgfpathlineto{\pgfqpoint{2.791208in}{2.272288in}}%
\pgfpathlineto{\pgfqpoint{2.791245in}{2.518238in}}%
\pgfpathlineto{\pgfqpoint{2.792082in}{2.566132in}}%
\pgfpathlineto{\pgfqpoint{2.792033in}{2.269165in}}%
\pgfpathlineto{\pgfqpoint{2.792291in}{2.336984in}}%
\pgfpathlineto{\pgfqpoint{2.793263in}{2.263082in}}%
\pgfpathlineto{\pgfqpoint{2.792906in}{2.569136in}}%
\pgfpathlineto{\pgfqpoint{2.793288in}{2.442619in}}%
\pgfpathlineto{\pgfqpoint{2.794137in}{2.581591in}}%
\pgfpathlineto{\pgfqpoint{2.794088in}{2.262630in}}%
\pgfpathlineto{\pgfqpoint{2.794371in}{2.289454in}}%
\pgfpathlineto{\pgfqpoint{2.794900in}{2.254880in}}%
\pgfpathlineto{\pgfqpoint{2.794543in}{2.586685in}}%
\pgfpathlineto{\pgfqpoint{2.794937in}{2.513010in}}%
\pgfpathlineto{\pgfqpoint{2.795774in}{2.591991in}}%
\pgfpathlineto{\pgfqpoint{2.795725in}{2.250415in}}%
\pgfpathlineto{\pgfqpoint{2.796008in}{2.303207in}}%
\pgfpathlineto{\pgfqpoint{2.796956in}{2.243618in}}%
\pgfpathlineto{\pgfqpoint{2.796599in}{2.595365in}}%
\pgfpathlineto{\pgfqpoint{2.796993in}{2.528491in}}%
\pgfpathlineto{\pgfqpoint{2.797423in}{2.597152in}}%
\pgfpathlineto{\pgfqpoint{2.797780in}{2.240403in}}%
\pgfpathlineto{\pgfqpoint{2.798051in}{2.334599in}}%
\pgfpathlineto{\pgfqpoint{2.799011in}{2.238324in}}%
\pgfpathlineto{\pgfqpoint{2.798248in}{2.587839in}}%
\pgfpathlineto{\pgfqpoint{2.799146in}{2.374981in}}%
\pgfpathlineto{\pgfqpoint{2.799479in}{2.572504in}}%
\pgfpathlineto{\pgfqpoint{2.799417in}{2.240879in}}%
\pgfpathlineto{\pgfqpoint{2.800217in}{2.345570in}}%
\pgfpathlineto{\pgfqpoint{2.800242in}{2.238070in}}%
\pgfpathlineto{\pgfqpoint{2.800291in}{2.555713in}}%
\pgfpathlineto{\pgfqpoint{2.801313in}{2.413319in}}%
\pgfpathlineto{\pgfqpoint{2.801473in}{2.247527in}}%
\pgfpathlineto{\pgfqpoint{2.801522in}{2.520177in}}%
\pgfpathlineto{\pgfqpoint{2.802420in}{2.379156in}}%
\pgfpathlineto{\pgfqpoint{2.802753in}{2.478476in}}%
\pgfpathlineto{\pgfqpoint{2.802703in}{2.260617in}}%
\pgfpathlineto{\pgfqpoint{2.803503in}{2.320463in}}%
\pgfpathlineto{\pgfqpoint{2.803528in}{2.276561in}}%
\pgfpathlineto{\pgfqpoint{2.803577in}{2.442534in}}%
\pgfpathlineto{\pgfqpoint{2.804586in}{2.405282in}}%
\pgfpathlineto{\pgfqpoint{2.805313in}{2.451706in}}%
\pgfpathlineto{\pgfqpoint{2.805165in}{2.292680in}}%
\pgfpathlineto{\pgfqpoint{2.805645in}{2.376447in}}%
\pgfpathlineto{\pgfqpoint{2.805989in}{2.300838in}}%
\pgfpathlineto{\pgfqpoint{2.805731in}{2.455388in}}%
\pgfpathlineto{\pgfqpoint{2.806740in}{2.391461in}}%
\pgfpathlineto{\pgfqpoint{2.806765in}{2.428511in}}%
\pgfpathlineto{\pgfqpoint{2.806814in}{2.291234in}}%
\pgfpathlineto{\pgfqpoint{2.807848in}{2.390317in}}%
\pgfpathlineto{\pgfqpoint{2.808414in}{2.435706in}}%
\pgfpathlineto{\pgfqpoint{2.808057in}{2.310363in}}%
\pgfpathlineto{\pgfqpoint{2.808882in}{2.321956in}}%
\pgfpathlineto{\pgfqpoint{2.809645in}{2.428831in}}%
\pgfpathlineto{\pgfqpoint{2.809300in}{2.315549in}}%
\pgfpathlineto{\pgfqpoint{2.810100in}{2.364988in}}%
\pgfpathlineto{\pgfqpoint{2.811073in}{2.302177in}}%
\pgfpathlineto{\pgfqpoint{2.810888in}{2.426461in}}%
\pgfpathlineto{\pgfqpoint{2.811196in}{2.377732in}}%
\pgfpathlineto{\pgfqpoint{2.812131in}{2.436848in}}%
\pgfpathlineto{\pgfqpoint{2.811897in}{2.300687in}}%
\pgfpathlineto{\pgfqpoint{2.812279in}{2.374439in}}%
\pgfpathlineto{\pgfqpoint{2.813128in}{2.278557in}}%
\pgfpathlineto{\pgfqpoint{2.813362in}{2.444383in}}%
\pgfpathlineto{\pgfqpoint{2.814359in}{2.267469in}}%
\pgfpathlineto{\pgfqpoint{2.814568in}{2.397681in}}%
\pgfpathlineto{\pgfqpoint{2.814605in}{2.450909in}}%
\pgfpathlineto{\pgfqpoint{2.815183in}{2.260771in}}%
\pgfpathlineto{\pgfqpoint{2.815663in}{2.391204in}}%
\pgfpathlineto{\pgfqpoint{2.816414in}{2.259349in}}%
\pgfpathlineto{\pgfqpoint{2.816672in}{2.453848in}}%
\pgfpathlineto{\pgfqpoint{2.816759in}{2.385983in}}%
\pgfpathlineto{\pgfqpoint{2.817079in}{2.457385in}}%
\pgfpathlineto{\pgfqpoint{2.817239in}{2.256088in}}%
\pgfpathlineto{\pgfqpoint{2.817903in}{2.447354in}}%
\pgfpathlineto{\pgfqpoint{2.818063in}{2.267498in}}%
\pgfpathlineto{\pgfqpoint{2.818297in}{2.452798in}}%
\pgfpathlineto{\pgfqpoint{2.819097in}{2.424196in}}%
\pgfpathlineto{\pgfqpoint{2.819528in}{2.459040in}}%
\pgfpathlineto{\pgfqpoint{2.819294in}{2.270267in}}%
\pgfpathlineto{\pgfqpoint{2.820106in}{2.270536in}}%
\pgfpathlineto{\pgfqpoint{2.820525in}{2.259110in}}%
\pgfpathlineto{\pgfqpoint{2.820759in}{2.460926in}}%
\pgfpathlineto{\pgfqpoint{2.821066in}{2.352029in}}%
\pgfpathlineto{\pgfqpoint{2.822002in}{2.474131in}}%
\pgfpathlineto{\pgfqpoint{2.821349in}{2.247022in}}%
\pgfpathlineto{\pgfqpoint{2.822149in}{2.326212in}}%
\pgfpathlineto{\pgfqpoint{2.822174in}{2.261998in}}%
\pgfpathlineto{\pgfqpoint{2.822408in}{2.468074in}}%
\pgfpathlineto{\pgfqpoint{2.823220in}{2.433394in}}%
\pgfpathlineto{\pgfqpoint{2.823232in}{2.462733in}}%
\pgfpathlineto{\pgfqpoint{2.823405in}{2.286783in}}%
\pgfpathlineto{\pgfqpoint{2.824303in}{2.371881in}}%
\pgfpathlineto{\pgfqpoint{2.825300in}{2.451724in}}%
\pgfpathlineto{\pgfqpoint{2.825054in}{2.295954in}}%
\pgfpathlineto{\pgfqpoint{2.825423in}{2.391823in}}%
\pgfpathlineto{\pgfqpoint{2.825879in}{2.293839in}}%
\pgfpathlineto{\pgfqpoint{2.826125in}{2.453764in}}%
\pgfpathlineto{\pgfqpoint{2.826506in}{2.399552in}}%
\pgfpathlineto{\pgfqpoint{2.827356in}{2.450537in}}%
\pgfpathlineto{\pgfqpoint{2.826703in}{2.298098in}}%
\pgfpathlineto{\pgfqpoint{2.827589in}{2.345497in}}%
\pgfpathlineto{\pgfqpoint{2.827774in}{2.451827in}}%
\pgfpathlineto{\pgfqpoint{2.827946in}{2.310753in}}%
\pgfpathlineto{\pgfqpoint{2.828722in}{2.387669in}}%
\pgfpathlineto{\pgfqpoint{2.828771in}{2.306616in}}%
\pgfpathlineto{\pgfqpoint{2.829423in}{2.442204in}}%
\pgfpathlineto{\pgfqpoint{2.829805in}{2.404133in}}%
\pgfpathlineto{\pgfqpoint{2.830248in}{2.444051in}}%
\pgfpathlineto{\pgfqpoint{2.830826in}{2.302095in}}%
\pgfpathlineto{\pgfqpoint{2.830851in}{2.327168in}}%
\pgfpathlineto{\pgfqpoint{2.831479in}{2.448679in}}%
\pgfpathlineto{\pgfqpoint{2.831651in}{2.293486in}}%
\pgfpathlineto{\pgfqpoint{2.831971in}{2.338821in}}%
\pgfpathlineto{\pgfqpoint{2.832303in}{2.448374in}}%
\pgfpathlineto{\pgfqpoint{2.832476in}{2.285551in}}%
\pgfpathlineto{\pgfqpoint{2.832869in}{2.319070in}}%
\pgfpathlineto{\pgfqpoint{2.833706in}{2.277406in}}%
\pgfpathlineto{\pgfqpoint{2.833534in}{2.440406in}}%
\pgfpathlineto{\pgfqpoint{2.833928in}{2.425218in}}%
\pgfpathlineto{\pgfqpoint{2.833952in}{2.436813in}}%
\pgfpathlineto{\pgfqpoint{2.834531in}{2.273196in}}%
\pgfpathlineto{\pgfqpoint{2.834912in}{2.368924in}}%
\pgfpathlineto{\pgfqpoint{2.835762in}{2.261033in}}%
\pgfpathlineto{\pgfqpoint{2.835589in}{2.436930in}}%
\pgfpathlineto{\pgfqpoint{2.836008in}{2.435136in}}%
\pgfpathlineto{\pgfqpoint{2.836992in}{2.249651in}}%
\pgfpathlineto{\pgfqpoint{2.836857in}{2.445855in}}%
\pgfpathlineto{\pgfqpoint{2.837152in}{2.405703in}}%
\pgfpathlineto{\pgfqpoint{2.838088in}{2.457737in}}%
\pgfpathlineto{\pgfqpoint{2.837817in}{2.236828in}}%
\pgfpathlineto{\pgfqpoint{2.838199in}{2.362872in}}%
\pgfpathlineto{\pgfqpoint{2.839048in}{2.227787in}}%
\pgfpathlineto{\pgfqpoint{2.838494in}{2.467143in}}%
\pgfpathlineto{\pgfqpoint{2.839294in}{2.457021in}}%
\pgfpathlineto{\pgfqpoint{2.839725in}{2.480022in}}%
\pgfpathlineto{\pgfqpoint{2.839454in}{2.230301in}}%
\pgfpathlineto{\pgfqpoint{2.839860in}{2.242478in}}%
\pgfpathlineto{\pgfqpoint{2.840279in}{2.215690in}}%
\pgfpathlineto{\pgfqpoint{2.840549in}{2.491565in}}%
\pgfpathlineto{\pgfqpoint{2.840845in}{2.422970in}}%
\pgfpathlineto{\pgfqpoint{2.841780in}{2.509178in}}%
\pgfpathlineto{\pgfqpoint{2.841509in}{2.208551in}}%
\pgfpathlineto{\pgfqpoint{2.841891in}{2.397094in}}%
\pgfpathlineto{\pgfqpoint{2.842740in}{2.198061in}}%
\pgfpathlineto{\pgfqpoint{2.842186in}{2.522076in}}%
\pgfpathlineto{\pgfqpoint{2.842986in}{2.483470in}}%
\pgfpathlineto{\pgfqpoint{2.843417in}{2.536003in}}%
\pgfpathlineto{\pgfqpoint{2.843565in}{2.191751in}}%
\pgfpathlineto{\pgfqpoint{2.843959in}{2.243832in}}%
\pgfpathlineto{\pgfqpoint{2.844795in}{2.169856in}}%
\pgfpathlineto{\pgfqpoint{2.844648in}{2.549270in}}%
\pgfpathlineto{\pgfqpoint{2.845017in}{2.424065in}}%
\pgfpathlineto{\pgfqpoint{2.845472in}{2.550517in}}%
\pgfpathlineto{\pgfqpoint{2.846026in}{2.155704in}}%
\pgfpathlineto{\pgfqpoint{2.846112in}{2.370972in}}%
\pgfpathlineto{\pgfqpoint{2.846432in}{2.156009in}}%
\pgfpathlineto{\pgfqpoint{2.847109in}{2.547642in}}%
\pgfpathlineto{\pgfqpoint{2.847195in}{2.425970in}}%
\pgfpathlineto{\pgfqpoint{2.847934in}{2.550072in}}%
\pgfpathlineto{\pgfqpoint{2.847257in}{2.147690in}}%
\pgfpathlineto{\pgfqpoint{2.848315in}{2.481421in}}%
\pgfpathlineto{\pgfqpoint{2.849165in}{2.560229in}}%
\pgfpathlineto{\pgfqpoint{2.848488in}{2.146920in}}%
\pgfpathlineto{\pgfqpoint{2.849288in}{2.321602in}}%
\pgfpathlineto{\pgfqpoint{2.849312in}{2.140204in}}%
\pgfpathlineto{\pgfqpoint{2.849989in}{2.559919in}}%
\pgfpathlineto{\pgfqpoint{2.850371in}{2.497266in}}%
\pgfpathlineto{\pgfqpoint{2.851220in}{2.571884in}}%
\pgfpathlineto{\pgfqpoint{2.850543in}{2.137152in}}%
\pgfpathlineto{\pgfqpoint{2.851343in}{2.308290in}}%
\pgfpathlineto{\pgfqpoint{2.851774in}{2.135681in}}%
\pgfpathlineto{\pgfqpoint{2.851626in}{2.569082in}}%
\pgfpathlineto{\pgfqpoint{2.852426in}{2.504649in}}%
\pgfpathlineto{\pgfqpoint{2.852451in}{2.579890in}}%
\pgfpathlineto{\pgfqpoint{2.852599in}{2.129981in}}%
\pgfpathlineto{\pgfqpoint{2.853399in}{2.283206in}}%
\pgfpathlineto{\pgfqpoint{2.853829in}{2.130068in}}%
\pgfpathlineto{\pgfqpoint{2.853682in}{2.576222in}}%
\pgfpathlineto{\pgfqpoint{2.854482in}{2.503890in}}%
\pgfpathlineto{\pgfqpoint{2.854506in}{2.576438in}}%
\pgfpathlineto{\pgfqpoint{2.855060in}{2.118813in}}%
\pgfpathlineto{\pgfqpoint{2.855454in}{2.247341in}}%
\pgfpathlineto{\pgfqpoint{2.856291in}{2.109824in}}%
\pgfpathlineto{\pgfqpoint{2.855737in}{2.572169in}}%
\pgfpathlineto{\pgfqpoint{2.856525in}{2.502333in}}%
\pgfpathlineto{\pgfqpoint{2.856968in}{2.578976in}}%
\pgfpathlineto{\pgfqpoint{2.856709in}{2.120980in}}%
\pgfpathlineto{\pgfqpoint{2.857103in}{2.149471in}}%
\pgfpathlineto{\pgfqpoint{2.857522in}{2.101267in}}%
\pgfpathlineto{\pgfqpoint{2.857792in}{2.573879in}}%
\pgfpathlineto{\pgfqpoint{2.858137in}{2.419908in}}%
\pgfpathlineto{\pgfqpoint{2.858198in}{2.575847in}}%
\pgfpathlineto{\pgfqpoint{2.858248in}{2.271911in}}%
\pgfpathlineto{\pgfqpoint{2.858322in}{2.305199in}}%
\pgfpathlineto{\pgfqpoint{2.858346in}{2.095202in}}%
\pgfpathlineto{\pgfqpoint{2.859023in}{2.571783in}}%
\pgfpathlineto{\pgfqpoint{2.859405in}{2.509399in}}%
\pgfpathlineto{\pgfqpoint{2.859429in}{2.577024in}}%
\pgfpathlineto{\pgfqpoint{2.859577in}{2.091846in}}%
\pgfpathlineto{\pgfqpoint{2.859971in}{2.218976in}}%
\pgfpathlineto{\pgfqpoint{2.860402in}{2.097914in}}%
\pgfpathlineto{\pgfqpoint{2.860254in}{2.577552in}}%
\pgfpathlineto{\pgfqpoint{2.861042in}{2.510723in}}%
\pgfpathlineto{\pgfqpoint{2.861066in}{2.560177in}}%
\pgfpathlineto{\pgfqpoint{2.861485in}{2.587711in}}%
\pgfpathlineto{\pgfqpoint{2.861632in}{2.095190in}}%
\pgfpathlineto{\pgfqpoint{2.861928in}{2.397241in}}%
\pgfpathlineto{\pgfqpoint{2.862038in}{2.091767in}}%
\pgfpathlineto{\pgfqpoint{2.862715in}{2.592470in}}%
\pgfpathlineto{\pgfqpoint{2.863023in}{2.471826in}}%
\pgfpathlineto{\pgfqpoint{2.863269in}{2.082578in}}%
\pgfpathlineto{\pgfqpoint{2.863946in}{2.600633in}}%
\pgfpathlineto{\pgfqpoint{2.864143in}{2.427762in}}%
\pgfpathlineto{\pgfqpoint{2.865177in}{2.603311in}}%
\pgfpathlineto{\pgfqpoint{2.864500in}{2.083159in}}%
\pgfpathlineto{\pgfqpoint{2.865214in}{2.382302in}}%
\pgfpathlineto{\pgfqpoint{2.865731in}{2.077400in}}%
\pgfpathlineto{\pgfqpoint{2.866002in}{2.605926in}}%
\pgfpathlineto{\pgfqpoint{2.866309in}{2.472395in}}%
\pgfpathlineto{\pgfqpoint{2.866962in}{2.067068in}}%
\pgfpathlineto{\pgfqpoint{2.866408in}{2.602848in}}%
\pgfpathlineto{\pgfqpoint{2.867417in}{2.376243in}}%
\pgfpathlineto{\pgfqpoint{2.867638in}{2.601173in}}%
\pgfpathlineto{\pgfqpoint{2.868192in}{2.062576in}}%
\pgfpathlineto{\pgfqpoint{2.868500in}{2.319845in}}%
\pgfpathlineto{\pgfqpoint{2.869423in}{2.058621in}}%
\pgfpathlineto{\pgfqpoint{2.868869in}{2.599858in}}%
\pgfpathlineto{\pgfqpoint{2.869583in}{2.457093in}}%
\pgfpathlineto{\pgfqpoint{2.869694in}{2.586784in}}%
\pgfpathlineto{\pgfqpoint{2.869817in}{2.195363in}}%
\pgfpathlineto{\pgfqpoint{2.870654in}{2.055147in}}%
\pgfpathlineto{\pgfqpoint{2.870100in}{2.601932in}}%
\pgfpathlineto{\pgfqpoint{2.870888in}{2.540792in}}%
\pgfpathlineto{\pgfqpoint{2.871885in}{2.046961in}}%
\pgfpathlineto{\pgfqpoint{2.871737in}{2.602524in}}%
\pgfpathlineto{\pgfqpoint{2.872032in}{2.420655in}}%
\pgfpathlineto{\pgfqpoint{2.872968in}{2.600154in}}%
\pgfpathlineto{\pgfqpoint{2.872291in}{2.041241in}}%
\pgfpathlineto{\pgfqpoint{2.873091in}{2.246778in}}%
\pgfpathlineto{\pgfqpoint{2.873522in}{2.033937in}}%
\pgfpathlineto{\pgfqpoint{2.873374in}{2.598270in}}%
\pgfpathlineto{\pgfqpoint{2.874174in}{2.526980in}}%
\pgfpathlineto{\pgfqpoint{2.874605in}{2.600575in}}%
\pgfpathlineto{\pgfqpoint{2.874752in}{2.040552in}}%
\pgfpathlineto{\pgfqpoint{2.875134in}{2.291161in}}%
\pgfpathlineto{\pgfqpoint{2.875158in}{2.038124in}}%
\pgfpathlineto{\pgfqpoint{2.875835in}{2.592594in}}%
\pgfpathlineto{\pgfqpoint{2.876217in}{2.542861in}}%
\pgfpathlineto{\pgfqpoint{2.876242in}{2.592347in}}%
\pgfpathlineto{\pgfqpoint{2.876266in}{2.473755in}}%
\pgfpathlineto{\pgfqpoint{2.876389in}{2.041553in}}%
\pgfpathlineto{\pgfqpoint{2.877066in}{2.582988in}}%
\pgfpathlineto{\pgfqpoint{2.877361in}{2.452157in}}%
\pgfpathlineto{\pgfqpoint{2.877472in}{2.593204in}}%
\pgfpathlineto{\pgfqpoint{2.878026in}{2.038415in}}%
\pgfpathlineto{\pgfqpoint{2.878408in}{2.339037in}}%
\pgfpathlineto{\pgfqpoint{2.879257in}{2.039220in}}%
\pgfpathlineto{\pgfqpoint{2.878703in}{2.591146in}}%
\pgfpathlineto{\pgfqpoint{2.879503in}{2.538104in}}%
\pgfpathlineto{\pgfqpoint{2.880340in}{2.585770in}}%
\pgfpathlineto{\pgfqpoint{2.879663in}{2.048322in}}%
\pgfpathlineto{\pgfqpoint{2.880475in}{2.113174in}}%
\pgfpathlineto{\pgfqpoint{2.880488in}{2.051300in}}%
\pgfpathlineto{\pgfqpoint{2.881165in}{2.580444in}}%
\pgfpathlineto{\pgfqpoint{2.881509in}{2.441149in}}%
\pgfpathlineto{\pgfqpoint{2.881571in}{2.584775in}}%
\pgfpathlineto{\pgfqpoint{2.881632in}{2.263765in}}%
\pgfpathlineto{\pgfqpoint{2.881694in}{2.271832in}}%
\pgfpathlineto{\pgfqpoint{2.881718in}{2.065770in}}%
\pgfpathlineto{\pgfqpoint{2.882395in}{2.577099in}}%
\pgfpathlineto{\pgfqpoint{2.882777in}{2.538768in}}%
\pgfpathlineto{\pgfqpoint{2.882801in}{2.584758in}}%
\pgfpathlineto{\pgfqpoint{2.883355in}{2.063867in}}%
\pgfpathlineto{\pgfqpoint{2.883663in}{2.339013in}}%
\pgfpathlineto{\pgfqpoint{2.884586in}{2.065852in}}%
\pgfpathlineto{\pgfqpoint{2.884032in}{2.583238in}}%
\pgfpathlineto{\pgfqpoint{2.884746in}{2.453745in}}%
\pgfpathlineto{\pgfqpoint{2.885263in}{2.579337in}}%
\pgfpathlineto{\pgfqpoint{2.885411in}{2.065906in}}%
\pgfpathlineto{\pgfqpoint{2.885792in}{2.305266in}}%
\pgfpathlineto{\pgfqpoint{2.885817in}{2.061677in}}%
\pgfpathlineto{\pgfqpoint{2.886494in}{2.573338in}}%
\pgfpathlineto{\pgfqpoint{2.886875in}{2.525219in}}%
\pgfpathlineto{\pgfqpoint{2.886900in}{2.567170in}}%
\pgfpathlineto{\pgfqpoint{2.886937in}{2.387826in}}%
\pgfpathlineto{\pgfqpoint{2.887048in}{2.061520in}}%
\pgfpathlineto{\pgfqpoint{2.887725in}{2.566491in}}%
\pgfpathlineto{\pgfqpoint{2.888032in}{2.488689in}}%
\pgfpathlineto{\pgfqpoint{2.888278in}{2.066634in}}%
\pgfpathlineto{\pgfqpoint{2.888131in}{2.565920in}}%
\pgfpathlineto{\pgfqpoint{2.889152in}{2.457219in}}%
\pgfpathlineto{\pgfqpoint{2.889361in}{2.562382in}}%
\pgfpathlineto{\pgfqpoint{2.889509in}{2.073670in}}%
\pgfpathlineto{\pgfqpoint{2.890223in}{2.337660in}}%
\pgfpathlineto{\pgfqpoint{2.890740in}{2.083512in}}%
\pgfpathlineto{\pgfqpoint{2.890592in}{2.565761in}}%
\pgfpathlineto{\pgfqpoint{2.891306in}{2.439495in}}%
\pgfpathlineto{\pgfqpoint{2.891823in}{2.564609in}}%
\pgfpathlineto{\pgfqpoint{2.891971in}{2.092415in}}%
\pgfpathlineto{\pgfqpoint{2.892352in}{2.292393in}}%
\pgfpathlineto{\pgfqpoint{2.892377in}{2.091885in}}%
\pgfpathlineto{\pgfqpoint{2.893054in}{2.566103in}}%
\pgfpathlineto{\pgfqpoint{2.893435in}{2.527151in}}%
\pgfpathlineto{\pgfqpoint{2.893608in}{2.093421in}}%
\pgfpathlineto{\pgfqpoint{2.893460in}{2.567191in}}%
\pgfpathlineto{\pgfqpoint{2.894641in}{2.442562in}}%
\pgfpathlineto{\pgfqpoint{2.894691in}{2.569076in}}%
\pgfpathlineto{\pgfqpoint{2.894838in}{2.091400in}}%
\pgfpathlineto{\pgfqpoint{2.895737in}{2.474060in}}%
\pgfpathlineto{\pgfqpoint{2.896475in}{2.090437in}}%
\pgfpathlineto{\pgfqpoint{2.896328in}{2.565882in}}%
\pgfpathlineto{\pgfqpoint{2.896881in}{2.097491in}}%
\pgfpathlineto{\pgfqpoint{2.897558in}{2.564257in}}%
\pgfpathlineto{\pgfqpoint{2.897706in}{2.082234in}}%
\pgfpathlineto{\pgfqpoint{2.898100in}{2.190914in}}%
\pgfpathlineto{\pgfqpoint{2.898937in}{2.075569in}}%
\pgfpathlineto{\pgfqpoint{2.898789in}{2.558918in}}%
\pgfpathlineto{\pgfqpoint{2.899158in}{2.496414in}}%
\pgfpathlineto{\pgfqpoint{2.899195in}{2.554129in}}%
\pgfpathlineto{\pgfqpoint{2.899343in}{2.082293in}}%
\pgfpathlineto{\pgfqpoint{2.900143in}{2.271898in}}%
\pgfpathlineto{\pgfqpoint{2.900168in}{2.074900in}}%
\pgfpathlineto{\pgfqpoint{2.900426in}{2.549577in}}%
\pgfpathlineto{\pgfqpoint{2.901226in}{2.519722in}}%
\pgfpathlineto{\pgfqpoint{2.901251in}{2.552115in}}%
\pgfpathlineto{\pgfqpoint{2.901288in}{2.357837in}}%
\pgfpathlineto{\pgfqpoint{2.901398in}{2.076144in}}%
\pgfpathlineto{\pgfqpoint{2.901657in}{2.547369in}}%
\pgfpathlineto{\pgfqpoint{2.902383in}{2.485164in}}%
\pgfpathlineto{\pgfqpoint{2.902629in}{2.075173in}}%
\pgfpathlineto{\pgfqpoint{2.902481in}{2.549348in}}%
\pgfpathlineto{\pgfqpoint{2.903503in}{2.453793in}}%
\pgfpathlineto{\pgfqpoint{2.904537in}{2.552755in}}%
\pgfpathlineto{\pgfqpoint{2.903860in}{2.076565in}}%
\pgfpathlineto{\pgfqpoint{2.904574in}{2.352400in}}%
\pgfpathlineto{\pgfqpoint{2.905091in}{2.075676in}}%
\pgfpathlineto{\pgfqpoint{2.904943in}{2.551609in}}%
\pgfpathlineto{\pgfqpoint{2.905657in}{2.434099in}}%
\pgfpathlineto{\pgfqpoint{2.905768in}{2.554340in}}%
\pgfpathlineto{\pgfqpoint{2.906321in}{2.079315in}}%
\pgfpathlineto{\pgfqpoint{2.906715in}{2.212771in}}%
\pgfpathlineto{\pgfqpoint{2.907146in}{2.083246in}}%
\pgfpathlineto{\pgfqpoint{2.906998in}{2.558060in}}%
\pgfpathlineto{\pgfqpoint{2.907786in}{2.502785in}}%
\pgfpathlineto{\pgfqpoint{2.907823in}{2.554187in}}%
\pgfpathlineto{\pgfqpoint{2.907946in}{2.227003in}}%
\pgfpathlineto{\pgfqpoint{2.908377in}{2.083163in}}%
\pgfpathlineto{\pgfqpoint{2.908229in}{2.558957in}}%
\pgfpathlineto{\pgfqpoint{2.909029in}{2.507305in}}%
\pgfpathlineto{\pgfqpoint{2.909460in}{2.557043in}}%
\pgfpathlineto{\pgfqpoint{2.909608in}{2.083910in}}%
\pgfpathlineto{\pgfqpoint{2.910001in}{2.197295in}}%
\pgfpathlineto{\pgfqpoint{2.910838in}{2.084708in}}%
\pgfpathlineto{\pgfqpoint{2.910691in}{2.556144in}}%
\pgfpathlineto{\pgfqpoint{2.911072in}{2.523624in}}%
\pgfpathlineto{\pgfqpoint{2.912069in}{2.087538in}}%
\pgfpathlineto{\pgfqpoint{2.911921in}{2.555537in}}%
\pgfpathlineto{\pgfqpoint{2.912278in}{2.430678in}}%
\pgfpathlineto{\pgfqpoint{2.912328in}{2.549608in}}%
\pgfpathlineto{\pgfqpoint{2.913300in}{2.087120in}}%
\pgfpathlineto{\pgfqpoint{2.913558in}{2.546748in}}%
\pgfpathlineto{\pgfqpoint{2.914494in}{2.404677in}}%
\pgfpathlineto{\pgfqpoint{2.914937in}{2.085388in}}%
\pgfpathlineto{\pgfqpoint{2.914789in}{2.547299in}}%
\pgfpathlineto{\pgfqpoint{2.915589in}{2.512859in}}%
\pgfpathlineto{\pgfqpoint{2.916020in}{2.548370in}}%
\pgfpathlineto{\pgfqpoint{2.916168in}{2.078138in}}%
\pgfpathlineto{\pgfqpoint{2.916549in}{2.345428in}}%
\pgfpathlineto{\pgfqpoint{2.917398in}{2.072910in}}%
\pgfpathlineto{\pgfqpoint{2.917251in}{2.547952in}}%
\pgfpathlineto{\pgfqpoint{2.917644in}{2.514468in}}%
\pgfpathlineto{\pgfqpoint{2.917657in}{2.547720in}}%
\pgfpathlineto{\pgfqpoint{2.917804in}{2.070475in}}%
\pgfpathlineto{\pgfqpoint{2.918604in}{2.259989in}}%
\pgfpathlineto{\pgfqpoint{2.919035in}{2.071956in}}%
\pgfpathlineto{\pgfqpoint{2.918887in}{2.548476in}}%
\pgfpathlineto{\pgfqpoint{2.919687in}{2.505704in}}%
\pgfpathlineto{\pgfqpoint{2.920118in}{2.548215in}}%
\pgfpathlineto{\pgfqpoint{2.920266in}{2.077672in}}%
\pgfpathlineto{\pgfqpoint{2.920647in}{2.321531in}}%
\pgfpathlineto{\pgfqpoint{2.921078in}{2.071613in}}%
\pgfpathlineto{\pgfqpoint{2.921349in}{2.548552in}}%
\pgfpathlineto{\pgfqpoint{2.921743in}{2.503930in}}%
\pgfpathlineto{\pgfqpoint{2.921755in}{2.551692in}}%
\pgfpathlineto{\pgfqpoint{2.922309in}{2.064120in}}%
\pgfpathlineto{\pgfqpoint{2.922801in}{2.421898in}}%
\pgfpathlineto{\pgfqpoint{2.923392in}{2.558197in}}%
\pgfpathlineto{\pgfqpoint{2.923946in}{2.062331in}}%
\pgfpathlineto{\pgfqpoint{2.925029in}{2.556350in}}%
\pgfpathlineto{\pgfqpoint{2.925140in}{2.431986in}}%
\pgfpathlineto{\pgfqpoint{2.925583in}{2.070696in}}%
\pgfpathlineto{\pgfqpoint{2.925435in}{2.549639in}}%
\pgfpathlineto{\pgfqpoint{2.926235in}{2.493023in}}%
\pgfpathlineto{\pgfqpoint{2.926666in}{2.548715in}}%
\pgfpathlineto{\pgfqpoint{2.926814in}{2.074684in}}%
\pgfpathlineto{\pgfqpoint{2.927207in}{2.159019in}}%
\pgfpathlineto{\pgfqpoint{2.927220in}{2.068578in}}%
\pgfpathlineto{\pgfqpoint{2.927478in}{2.538891in}}%
\pgfpathlineto{\pgfqpoint{2.928266in}{2.520129in}}%
\pgfpathlineto{\pgfqpoint{2.928857in}{2.067540in}}%
\pgfpathlineto{\pgfqpoint{2.928709in}{2.552690in}}%
\pgfpathlineto{\pgfqpoint{2.929447in}{2.425761in}}%
\pgfpathlineto{\pgfqpoint{2.930346in}{2.550424in}}%
\pgfpathlineto{\pgfqpoint{2.929669in}{2.086296in}}%
\pgfpathlineto{\pgfqpoint{2.930481in}{2.143253in}}%
\pgfpathlineto{\pgfqpoint{2.930900in}{2.079337in}}%
\pgfpathlineto{\pgfqpoint{2.931134in}{2.552462in}}%
\pgfpathlineto{\pgfqpoint{2.931527in}{2.456711in}}%
\pgfpathlineto{\pgfqpoint{2.932364in}{2.563328in}}%
\pgfpathlineto{\pgfqpoint{2.932537in}{2.070466in}}%
\pgfpathlineto{\pgfqpoint{2.932623in}{2.420831in}}%
\pgfpathlineto{\pgfqpoint{2.932943in}{2.074945in}}%
\pgfpathlineto{\pgfqpoint{2.932771in}{2.570938in}}%
\pgfpathlineto{\pgfqpoint{2.933706in}{2.498721in}}%
\pgfpathlineto{\pgfqpoint{2.934001in}{2.573105in}}%
\pgfpathlineto{\pgfqpoint{2.934174in}{2.061652in}}%
\pgfpathlineto{\pgfqpoint{2.934826in}{2.559464in}}%
\pgfpathlineto{\pgfqpoint{2.935404in}{2.053889in}}%
\pgfpathlineto{\pgfqpoint{2.935232in}{2.568161in}}%
\pgfpathlineto{\pgfqpoint{2.936020in}{2.445981in}}%
\pgfpathlineto{\pgfqpoint{2.936894in}{2.564607in}}%
\pgfpathlineto{\pgfqpoint{2.936635in}{2.059373in}}%
\pgfpathlineto{\pgfqpoint{2.937017in}{2.366497in}}%
\pgfpathlineto{\pgfqpoint{2.937460in}{2.068205in}}%
\pgfpathlineto{\pgfqpoint{2.937300in}{2.565596in}}%
\pgfpathlineto{\pgfqpoint{2.938112in}{2.546875in}}%
\pgfpathlineto{\pgfqpoint{2.938949in}{2.568414in}}%
\pgfpathlineto{\pgfqpoint{2.938691in}{2.084633in}}%
\pgfpathlineto{\pgfqpoint{2.939072in}{2.347247in}}%
\pgfpathlineto{\pgfqpoint{2.939515in}{2.097146in}}%
\pgfpathlineto{\pgfqpoint{2.939355in}{2.560814in}}%
\pgfpathlineto{\pgfqpoint{2.940167in}{2.521936in}}%
\pgfpathlineto{\pgfqpoint{2.940180in}{2.552103in}}%
\pgfpathlineto{\pgfqpoint{2.940340in}{2.117047in}}%
\pgfpathlineto{\pgfqpoint{2.941140in}{2.266696in}}%
\pgfpathlineto{\pgfqpoint{2.941164in}{2.124836in}}%
\pgfpathlineto{\pgfqpoint{2.941410in}{2.535163in}}%
\pgfpathlineto{\pgfqpoint{2.942210in}{2.480398in}}%
\pgfpathlineto{\pgfqpoint{2.942235in}{2.530133in}}%
\pgfpathlineto{\pgfqpoint{2.942814in}{2.145403in}}%
\pgfpathlineto{\pgfqpoint{2.943207in}{2.214718in}}%
\pgfpathlineto{\pgfqpoint{2.943638in}{2.155085in}}%
\pgfpathlineto{\pgfqpoint{2.943466in}{2.529812in}}%
\pgfpathlineto{\pgfqpoint{2.944254in}{2.468183in}}%
\pgfpathlineto{\pgfqpoint{2.944697in}{2.524373in}}%
\pgfpathlineto{\pgfqpoint{2.944463in}{2.167178in}}%
\pgfpathlineto{\pgfqpoint{2.945250in}{2.326432in}}%
\pgfpathlineto{\pgfqpoint{2.945694in}{2.178829in}}%
\pgfpathlineto{\pgfqpoint{2.945927in}{2.520409in}}%
\pgfpathlineto{\pgfqpoint{2.946321in}{2.480917in}}%
\pgfpathlineto{\pgfqpoint{2.947158in}{2.516734in}}%
\pgfpathlineto{\pgfqpoint{2.946518in}{2.198062in}}%
\pgfpathlineto{\pgfqpoint{2.947330in}{2.218843in}}%
\pgfpathlineto{\pgfqpoint{2.947343in}{2.217466in}}%
\pgfpathlineto{\pgfqpoint{2.947367in}{2.329353in}}%
\pgfpathlineto{\pgfqpoint{2.948389in}{2.513769in}}%
\pgfpathlineto{\pgfqpoint{2.947749in}{2.222080in}}%
\pgfpathlineto{\pgfqpoint{2.948487in}{2.365860in}}%
\pgfpathlineto{\pgfqpoint{2.948561in}{2.237744in}}%
\pgfpathlineto{\pgfqpoint{2.948795in}{2.504832in}}%
\pgfpathlineto{\pgfqpoint{2.949189in}{2.447013in}}%
\pgfpathlineto{\pgfqpoint{2.949238in}{2.506484in}}%
\pgfpathlineto{\pgfqpoint{2.949386in}{2.244350in}}%
\pgfpathlineto{\pgfqpoint{2.950260in}{2.345718in}}%
\pgfpathlineto{\pgfqpoint{2.950617in}{2.270117in}}%
\pgfpathlineto{\pgfqpoint{2.950469in}{2.506196in}}%
\pgfpathlineto{\pgfqpoint{2.951257in}{2.474975in}}%
\pgfpathlineto{\pgfqpoint{2.951294in}{2.503078in}}%
\pgfpathlineto{\pgfqpoint{2.951318in}{2.449592in}}%
\pgfpathlineto{\pgfqpoint{2.951429in}{2.288961in}}%
\pgfpathlineto{\pgfqpoint{2.951712in}{2.500216in}}%
\pgfpathlineto{\pgfqpoint{2.952426in}{2.420655in}}%
\pgfpathlineto{\pgfqpoint{2.952524in}{2.497180in}}%
\pgfpathlineto{\pgfqpoint{2.953164in}{2.287680in}}%
\pgfpathlineto{\pgfqpoint{2.953460in}{2.319768in}}%
\pgfpathlineto{\pgfqpoint{2.953989in}{2.288331in}}%
\pgfpathlineto{\pgfqpoint{2.953755in}{2.493831in}}%
\pgfpathlineto{\pgfqpoint{2.954469in}{2.405333in}}%
\pgfpathlineto{\pgfqpoint{2.954580in}{2.488549in}}%
\pgfpathlineto{\pgfqpoint{2.955195in}{2.286471in}}%
\pgfpathlineto{\pgfqpoint{2.955515in}{2.316113in}}%
\pgfpathlineto{\pgfqpoint{2.955835in}{2.481641in}}%
\pgfpathlineto{\pgfqpoint{2.956020in}{2.283852in}}%
\pgfpathlineto{\pgfqpoint{2.956414in}{2.293748in}}%
\pgfpathlineto{\pgfqpoint{2.956844in}{2.288311in}}%
\pgfpathlineto{\pgfqpoint{2.956660in}{2.484033in}}%
\pgfpathlineto{\pgfqpoint{2.957054in}{2.470526in}}%
\pgfpathlineto{\pgfqpoint{2.957484in}{2.482376in}}%
\pgfpathlineto{\pgfqpoint{2.957250in}{2.287930in}}%
\pgfpathlineto{\pgfqpoint{2.957644in}{2.298575in}}%
\pgfpathlineto{\pgfqpoint{2.958063in}{2.279044in}}%
\pgfpathlineto{\pgfqpoint{2.957890in}{2.482560in}}%
\pgfpathlineto{\pgfqpoint{2.958580in}{2.407793in}}%
\pgfpathlineto{\pgfqpoint{2.959121in}{2.481437in}}%
\pgfpathlineto{\pgfqpoint{2.958875in}{2.275612in}}%
\pgfpathlineto{\pgfqpoint{2.959207in}{2.334156in}}%
\pgfpathlineto{\pgfqpoint{2.959294in}{2.282745in}}%
\pgfpathlineto{\pgfqpoint{2.959540in}{2.458088in}}%
\pgfpathlineto{\pgfqpoint{2.960266in}{2.398843in}}%
\pgfpathlineto{\pgfqpoint{2.960364in}{2.475539in}}%
\pgfpathlineto{\pgfqpoint{2.960524in}{2.297259in}}%
\pgfpathlineto{\pgfqpoint{2.961373in}{2.268140in}}%
\pgfpathlineto{\pgfqpoint{2.961226in}{2.461336in}}%
\pgfpathlineto{\pgfqpoint{2.961509in}{2.415136in}}%
\pgfpathlineto{\pgfqpoint{2.962457in}{2.517858in}}%
\pgfpathlineto{\pgfqpoint{2.961792in}{2.227471in}}%
\pgfpathlineto{\pgfqpoint{2.962580in}{2.379371in}}%
\pgfpathlineto{\pgfqpoint{2.962617in}{2.266341in}}%
\pgfpathlineto{\pgfqpoint{2.963146in}{2.456420in}}%
\pgfpathlineto{\pgfqpoint{2.963700in}{2.321985in}}%
\pgfpathlineto{\pgfqpoint{2.964647in}{2.224398in}}%
\pgfpathlineto{\pgfqpoint{2.964180in}{2.470197in}}%
\pgfpathlineto{\pgfqpoint{2.964758in}{2.327625in}}%
\pgfpathlineto{\pgfqpoint{2.965410in}{2.496085in}}%
\pgfpathlineto{\pgfqpoint{2.965066in}{2.238040in}}%
\pgfpathlineto{\pgfqpoint{2.965853in}{2.324399in}}%
\pgfpathlineto{\pgfqpoint{2.965890in}{2.212219in}}%
\pgfpathlineto{\pgfqpoint{2.966063in}{2.479607in}}%
\pgfpathlineto{\pgfqpoint{2.966949in}{2.375664in}}%
\pgfpathlineto{\pgfqpoint{2.967983in}{2.489124in}}%
\pgfpathlineto{\pgfqpoint{2.967121in}{2.275632in}}%
\pgfpathlineto{\pgfqpoint{2.968118in}{2.452617in}}%
\pgfpathlineto{\pgfqpoint{2.969152in}{2.290957in}}%
\pgfpathlineto{\pgfqpoint{2.969226in}{2.451134in}}%
\pgfpathlineto{\pgfqpoint{2.970284in}{2.269284in}}%
\pgfpathlineto{\pgfqpoint{2.969275in}{2.493530in}}%
\pgfpathlineto{\pgfqpoint{2.970407in}{2.395723in}}%
\pgfpathlineto{\pgfqpoint{2.970518in}{2.479102in}}%
\pgfpathlineto{\pgfqpoint{2.970826in}{2.280999in}}%
\pgfpathlineto{\pgfqpoint{2.971478in}{2.342030in}}%
\pgfpathlineto{\pgfqpoint{2.971515in}{2.309297in}}%
\pgfpathlineto{\pgfqpoint{2.972204in}{2.479493in}}%
\pgfpathlineto{\pgfqpoint{2.972573in}{2.352894in}}%
\pgfpathlineto{\pgfqpoint{2.973533in}{2.479788in}}%
\pgfpathlineto{\pgfqpoint{2.973275in}{2.263930in}}%
\pgfpathlineto{\pgfqpoint{2.973669in}{2.358797in}}%
\pgfpathlineto{\pgfqpoint{2.974518in}{2.264140in}}%
\pgfpathlineto{\pgfqpoint{2.974284in}{2.530871in}}%
\pgfpathlineto{\pgfqpoint{2.974752in}{2.350619in}}%
\pgfpathlineto{\pgfqpoint{2.975404in}{2.500003in}}%
\pgfpathlineto{\pgfqpoint{2.975663in}{2.271360in}}%
\pgfpathlineto{\pgfqpoint{2.975847in}{2.350062in}}%
\pgfpathlineto{\pgfqpoint{2.976475in}{2.231307in}}%
\pgfpathlineto{\pgfqpoint{2.976647in}{2.485873in}}%
\pgfpathlineto{\pgfqpoint{2.976930in}{2.388248in}}%
\pgfpathlineto{\pgfqpoint{2.977964in}{2.476427in}}%
\pgfpathlineto{\pgfqpoint{2.977103in}{2.264141in}}%
\pgfpathlineto{\pgfqpoint{2.978026in}{2.386223in}}%
\pgfpathlineto{\pgfqpoint{2.978333in}{2.250171in}}%
\pgfpathlineto{\pgfqpoint{2.978518in}{2.481295in}}%
\pgfpathlineto{\pgfqpoint{2.979084in}{2.480336in}}%
\pgfpathlineto{\pgfqpoint{2.979109in}{2.500304in}}%
\pgfpathlineto{\pgfqpoint{2.979564in}{2.252509in}}%
\pgfpathlineto{\pgfqpoint{2.980057in}{2.378898in}}%
\pgfpathlineto{\pgfqpoint{2.980204in}{2.234710in}}%
\pgfpathlineto{\pgfqpoint{2.980881in}{2.499415in}}%
\pgfpathlineto{\pgfqpoint{2.981201in}{2.283968in}}%
\pgfpathlineto{\pgfqpoint{2.981681in}{2.519985in}}%
\pgfpathlineto{\pgfqpoint{2.982026in}{2.267078in}}%
\pgfpathlineto{\pgfqpoint{2.982518in}{2.401729in}}%
\pgfpathlineto{\pgfqpoint{2.983133in}{2.242495in}}%
\pgfpathlineto{\pgfqpoint{2.983478in}{2.498356in}}%
\pgfpathlineto{\pgfqpoint{2.983576in}{2.474380in}}%
\pgfpathlineto{\pgfqpoint{2.983601in}{2.492657in}}%
\pgfpathlineto{\pgfqpoint{2.983933in}{2.221439in}}%
\pgfpathlineto{\pgfqpoint{2.984635in}{2.412892in}}%
\pgfpathlineto{\pgfqpoint{2.985164in}{2.199165in}}%
\pgfpathlineto{\pgfqpoint{2.985336in}{2.554442in}}%
\pgfpathlineto{\pgfqpoint{2.985829in}{2.372937in}}%
\pgfpathlineto{\pgfqpoint{2.986456in}{2.490640in}}%
\pgfpathlineto{\pgfqpoint{2.986813in}{2.249757in}}%
\pgfpathlineto{\pgfqpoint{2.986900in}{2.333565in}}%
\pgfpathlineto{\pgfqpoint{2.987638in}{2.212985in}}%
\pgfpathlineto{\pgfqpoint{2.987293in}{2.518897in}}%
\pgfpathlineto{\pgfqpoint{2.987970in}{2.398202in}}%
\pgfpathlineto{\pgfqpoint{2.989041in}{2.516811in}}%
\pgfpathlineto{\pgfqpoint{2.988869in}{2.245319in}}%
\pgfpathlineto{\pgfqpoint{2.989090in}{2.417541in}}%
\pgfpathlineto{\pgfqpoint{2.989164in}{2.518959in}}%
\pgfpathlineto{\pgfqpoint{2.989386in}{2.249218in}}%
\pgfpathlineto{\pgfqpoint{2.990186in}{2.437237in}}%
\pgfpathlineto{\pgfqpoint{2.990629in}{2.260540in}}%
\pgfpathlineto{\pgfqpoint{2.990998in}{2.517434in}}%
\pgfpathlineto{\pgfqpoint{2.991355in}{2.291304in}}%
\pgfpathlineto{\pgfqpoint{2.991527in}{2.462807in}}%
\pgfpathlineto{\pgfqpoint{2.992389in}{2.272309in}}%
\pgfpathlineto{\pgfqpoint{2.992487in}{2.332255in}}%
\pgfpathlineto{\pgfqpoint{2.993213in}{2.272961in}}%
\pgfpathlineto{\pgfqpoint{2.992746in}{2.479565in}}%
\pgfpathlineto{\pgfqpoint{2.993521in}{2.434391in}}%
\pgfpathlineto{\pgfqpoint{2.993533in}{2.444787in}}%
\pgfpathlineto{\pgfqpoint{2.994420in}{2.280951in}}%
\pgfpathlineto{\pgfqpoint{2.994444in}{2.254429in}}%
\pgfpathlineto{\pgfqpoint{2.994690in}{2.498004in}}%
\pgfpathlineto{\pgfqpoint{2.995429in}{2.405370in}}%
\pgfpathlineto{\pgfqpoint{2.995638in}{2.465952in}}%
\pgfpathlineto{\pgfqpoint{2.996192in}{2.282042in}}%
\pgfpathlineto{\pgfqpoint{2.996549in}{2.439206in}}%
\pgfpathlineto{\pgfqpoint{2.996807in}{2.266043in}}%
\pgfpathlineto{\pgfqpoint{2.996610in}{2.461869in}}%
\pgfpathlineto{\pgfqpoint{2.997792in}{2.327765in}}%
\pgfpathlineto{\pgfqpoint{2.998358in}{2.496047in}}%
\pgfpathlineto{\pgfqpoint{2.998678in}{2.269162in}}%
\pgfpathlineto{\pgfqpoint{2.998924in}{2.392270in}}%
\pgfpathlineto{\pgfqpoint{2.999416in}{2.297426in}}%
\pgfpathlineto{\pgfqpoint{2.999072in}{2.441913in}}%
\pgfpathlineto{\pgfqpoint{3.000044in}{2.336179in}}%
\pgfpathlineto{\pgfqpoint{3.000229in}{2.484511in}}%
\pgfpathlineto{\pgfqpoint{3.000672in}{2.274059in}}%
\pgfpathlineto{\pgfqpoint{3.001189in}{2.433485in}}%
\pgfpathlineto{\pgfqpoint{3.002370in}{2.246069in}}%
\pgfpathlineto{\pgfqpoint{3.002050in}{2.528281in}}%
\pgfpathlineto{\pgfqpoint{3.002432in}{2.337360in}}%
\pgfpathlineto{\pgfqpoint{3.002961in}{2.471288in}}%
\pgfpathlineto{\pgfqpoint{3.002518in}{2.222550in}}%
\pgfpathlineto{\pgfqpoint{3.003552in}{2.389323in}}%
\pgfpathlineto{\pgfqpoint{3.004376in}{2.237673in}}%
\pgfpathlineto{\pgfqpoint{3.003909in}{2.494540in}}%
\pgfpathlineto{\pgfqpoint{3.004635in}{2.391745in}}%
\pgfpathlineto{\pgfqpoint{3.005755in}{2.534118in}}%
\pgfpathlineto{\pgfqpoint{3.005509in}{2.273185in}}%
\pgfpathlineto{\pgfqpoint{3.005792in}{2.451758in}}%
\pgfpathlineto{\pgfqpoint{3.005829in}{2.467458in}}%
\pgfpathlineto{\pgfqpoint{3.005915in}{2.417769in}}%
\pgfpathlineto{\pgfqpoint{3.006223in}{2.251029in}}%
\pgfpathlineto{\pgfqpoint{3.006690in}{2.473956in}}%
\pgfpathlineto{\pgfqpoint{3.007072in}{2.342269in}}%
\pgfpathlineto{\pgfqpoint{3.007650in}{2.502302in}}%
\pgfpathlineto{\pgfqpoint{3.007970in}{2.219276in}}%
\pgfpathlineto{\pgfqpoint{3.008056in}{2.258332in}}%
\pgfpathlineto{\pgfqpoint{3.008069in}{2.226886in}}%
\pgfpathlineto{\pgfqpoint{3.008549in}{2.485021in}}%
\pgfpathlineto{\pgfqpoint{3.009115in}{2.350368in}}%
\pgfpathlineto{\pgfqpoint{3.009509in}{2.511787in}}%
\pgfpathlineto{\pgfqpoint{3.009730in}{2.244624in}}%
\pgfpathlineto{\pgfqpoint{3.010247in}{2.404360in}}%
\pgfpathlineto{\pgfqpoint{3.011416in}{2.511334in}}%
\pgfpathlineto{\pgfqpoint{3.010543in}{2.287889in}}%
\pgfpathlineto{\pgfqpoint{3.011441in}{2.461792in}}%
\pgfpathlineto{\pgfqpoint{3.011786in}{2.238838in}}%
\pgfpathlineto{\pgfqpoint{3.012598in}{2.296414in}}%
\pgfpathlineto{\pgfqpoint{3.012647in}{2.386623in}}%
\pgfpathlineto{\pgfqpoint{3.013176in}{2.544746in}}%
\pgfpathlineto{\pgfqpoint{3.013546in}{2.255916in}}%
\pgfpathlineto{\pgfqpoint{3.013779in}{2.451708in}}%
\pgfpathlineto{\pgfqpoint{3.014259in}{2.277274in}}%
\pgfpathlineto{\pgfqpoint{3.013964in}{2.453872in}}%
\pgfpathlineto{\pgfqpoint{3.014912in}{2.380124in}}%
\pgfpathlineto{\pgfqpoint{3.015010in}{2.496995in}}%
\pgfpathlineto{\pgfqpoint{3.015392in}{2.244572in}}%
\pgfpathlineto{\pgfqpoint{3.016056in}{2.419597in}}%
\pgfpathlineto{\pgfqpoint{3.017139in}{2.286091in}}%
\pgfpathlineto{\pgfqpoint{3.016906in}{2.556164in}}%
\pgfpathlineto{\pgfqpoint{3.017238in}{2.289749in}}%
\pgfpathlineto{\pgfqpoint{3.017250in}{2.278515in}}%
\pgfpathlineto{\pgfqpoint{3.017681in}{2.448953in}}%
\pgfpathlineto{\pgfqpoint{3.018272in}{2.345921in}}%
\pgfpathlineto{\pgfqpoint{3.018764in}{2.506866in}}%
\pgfpathlineto{\pgfqpoint{3.019121in}{2.254882in}}%
\pgfpathlineto{\pgfqpoint{3.019195in}{2.290216in}}%
\pgfpathlineto{\pgfqpoint{3.019207in}{2.263868in}}%
\pgfpathlineto{\pgfqpoint{3.019687in}{2.443946in}}%
\pgfpathlineto{\pgfqpoint{3.020266in}{2.337446in}}%
\pgfpathlineto{\pgfqpoint{3.020623in}{2.485919in}}%
\pgfpathlineto{\pgfqpoint{3.020955in}{2.262857in}}%
\pgfpathlineto{\pgfqpoint{3.021386in}{2.365583in}}%
\pgfpathlineto{\pgfqpoint{3.022186in}{2.276794in}}%
\pgfpathlineto{\pgfqpoint{3.022370in}{2.467312in}}%
\pgfpathlineto{\pgfqpoint{3.022432in}{2.441464in}}%
\pgfpathlineto{\pgfqpoint{3.022469in}{2.496300in}}%
\pgfpathlineto{\pgfqpoint{3.022924in}{2.265368in}}%
\pgfpathlineto{\pgfqpoint{3.023490in}{2.380318in}}%
\pgfpathlineto{\pgfqpoint{3.023552in}{2.273842in}}%
\pgfpathlineto{\pgfqpoint{3.024302in}{2.478652in}}%
\pgfpathlineto{\pgfqpoint{3.024684in}{2.311814in}}%
\pgfpathlineto{\pgfqpoint{3.024930in}{2.439337in}}%
\pgfpathlineto{\pgfqpoint{3.024782in}{2.266045in}}%
\pgfpathlineto{\pgfqpoint{3.025804in}{2.337950in}}%
\pgfpathlineto{\pgfqpoint{3.025915in}{2.282580in}}%
\pgfpathlineto{\pgfqpoint{3.026161in}{2.465087in}}%
\pgfpathlineto{\pgfqpoint{3.026813in}{2.413520in}}%
\pgfpathlineto{\pgfqpoint{3.027761in}{2.298336in}}%
\pgfpathlineto{\pgfqpoint{3.027096in}{2.434219in}}%
\pgfpathlineto{\pgfqpoint{3.027921in}{2.385427in}}%
\pgfpathlineto{\pgfqpoint{3.028056in}{2.468519in}}%
\pgfpathlineto{\pgfqpoint{3.028450in}{2.304313in}}%
\pgfpathlineto{\pgfqpoint{3.029004in}{2.341131in}}%
\pgfpathlineto{\pgfqpoint{3.029742in}{2.290178in}}%
\pgfpathlineto{\pgfqpoint{3.029299in}{2.444211in}}%
\pgfpathlineto{\pgfqpoint{3.029853in}{2.405026in}}%
\pgfpathlineto{\pgfqpoint{3.029902in}{2.477019in}}%
\pgfpathlineto{\pgfqpoint{3.030358in}{2.276434in}}%
\pgfpathlineto{\pgfqpoint{3.030936in}{2.366968in}}%
\pgfpathlineto{\pgfqpoint{3.031589in}{2.292913in}}%
\pgfpathlineto{\pgfqpoint{3.031773in}{2.501163in}}%
\pgfpathlineto{\pgfqpoint{3.032032in}{2.371934in}}%
\pgfpathlineto{\pgfqpoint{3.032487in}{2.442102in}}%
\pgfpathlineto{\pgfqpoint{3.032216in}{2.282835in}}%
\pgfpathlineto{\pgfqpoint{3.033139in}{2.381505in}}%
\pgfpathlineto{\pgfqpoint{3.033459in}{2.253141in}}%
\pgfpathlineto{\pgfqpoint{3.033607in}{2.464844in}}%
\pgfpathlineto{\pgfqpoint{3.034222in}{2.393486in}}%
\pgfpathlineto{\pgfqpoint{3.034702in}{2.287757in}}%
\pgfpathlineto{\pgfqpoint{3.034850in}{2.432003in}}%
\pgfpathlineto{\pgfqpoint{3.035601in}{2.481487in}}%
\pgfpathlineto{\pgfqpoint{3.035219in}{2.289135in}}%
\pgfpathlineto{\pgfqpoint{3.035749in}{2.339074in}}%
\pgfpathlineto{\pgfqpoint{3.036573in}{2.278332in}}%
\pgfpathlineto{\pgfqpoint{3.036389in}{2.437258in}}%
\pgfpathlineto{\pgfqpoint{3.036684in}{2.399395in}}%
\pgfpathlineto{\pgfqpoint{3.037250in}{2.449773in}}%
\pgfpathlineto{\pgfqpoint{3.037595in}{2.301241in}}%
\pgfpathlineto{\pgfqpoint{3.037767in}{2.372116in}}%
\pgfpathlineto{\pgfqpoint{3.038936in}{2.244729in}}%
\pgfpathlineto{\pgfqpoint{3.038604in}{2.461673in}}%
\pgfpathlineto{\pgfqpoint{3.038949in}{2.264966in}}%
\pgfpathlineto{\pgfqpoint{3.039835in}{2.437067in}}%
\pgfpathlineto{\pgfqpoint{3.040093in}{2.402676in}}%
\pgfpathlineto{\pgfqpoint{3.041053in}{2.508857in}}%
\pgfpathlineto{\pgfqpoint{3.040893in}{2.283560in}}%
\pgfpathlineto{\pgfqpoint{3.041201in}{2.417485in}}%
\pgfpathlineto{\pgfqpoint{3.041521in}{2.279979in}}%
\pgfpathlineto{\pgfqpoint{3.041890in}{2.458043in}}%
\pgfpathlineto{\pgfqpoint{3.042309in}{2.408353in}}%
\pgfpathlineto{\pgfqpoint{3.042776in}{2.301125in}}%
\pgfpathlineto{\pgfqpoint{3.042432in}{2.438667in}}%
\pgfpathlineto{\pgfqpoint{3.043416in}{2.405281in}}%
\pgfpathlineto{\pgfqpoint{3.043453in}{2.464141in}}%
\pgfpathlineto{\pgfqpoint{3.044007in}{2.273695in}}%
\pgfpathlineto{\pgfqpoint{3.044487in}{2.333260in}}%
\pgfpathlineto{\pgfqpoint{3.045238in}{2.280109in}}%
\pgfpathlineto{\pgfqpoint{3.044807in}{2.450741in}}%
\pgfpathlineto{\pgfqpoint{3.045545in}{2.389166in}}%
\pgfpathlineto{\pgfqpoint{3.046542in}{2.476985in}}%
\pgfpathlineto{\pgfqpoint{3.045878in}{2.312719in}}%
\pgfpathlineto{\pgfqpoint{3.046678in}{2.419249in}}%
\pgfpathlineto{\pgfqpoint{3.047724in}{2.264854in}}%
\pgfpathlineto{\pgfqpoint{3.047256in}{2.444121in}}%
\pgfpathlineto{\pgfqpoint{3.047785in}{2.392873in}}%
\pgfpathlineto{\pgfqpoint{3.048007in}{2.479820in}}%
\pgfpathlineto{\pgfqpoint{3.048352in}{2.276805in}}%
\pgfpathlineto{\pgfqpoint{3.048869in}{2.375876in}}%
\pgfpathlineto{\pgfqpoint{3.048967in}{2.298930in}}%
\pgfpathlineto{\pgfqpoint{3.049644in}{2.457814in}}%
\pgfpathlineto{\pgfqpoint{3.050025in}{2.345646in}}%
\pgfpathlineto{\pgfqpoint{3.050370in}{2.491292in}}%
\pgfpathlineto{\pgfqpoint{3.050825in}{2.282492in}}%
\pgfpathlineto{\pgfqpoint{3.051182in}{2.415646in}}%
\pgfpathlineto{\pgfqpoint{3.051441in}{2.272762in}}%
\pgfpathlineto{\pgfqpoint{3.051736in}{2.429077in}}%
\pgfpathlineto{\pgfqpoint{3.052376in}{2.373129in}}%
\pgfpathlineto{\pgfqpoint{3.053472in}{2.471659in}}%
\pgfpathlineto{\pgfqpoint{3.053299in}{2.287757in}}%
\pgfpathlineto{\pgfqpoint{3.053509in}{2.428463in}}%
\pgfpathlineto{\pgfqpoint{3.053693in}{2.301204in}}%
\pgfpathlineto{\pgfqpoint{3.054210in}{2.451036in}}%
\pgfpathlineto{\pgfqpoint{3.054641in}{2.360397in}}%
\pgfpathlineto{\pgfqpoint{3.055342in}{2.460361in}}%
\pgfpathlineto{\pgfqpoint{3.055170in}{2.263856in}}%
\pgfpathlineto{\pgfqpoint{3.055761in}{2.376251in}}%
\pgfpathlineto{\pgfqpoint{3.056795in}{2.289922in}}%
\pgfpathlineto{\pgfqpoint{3.056585in}{2.467281in}}%
\pgfpathlineto{\pgfqpoint{3.056905in}{2.291546in}}%
\pgfpathlineto{\pgfqpoint{3.057829in}{2.454256in}}%
\pgfpathlineto{\pgfqpoint{3.057521in}{2.274632in}}%
\pgfpathlineto{\pgfqpoint{3.058087in}{2.349631in}}%
\pgfpathlineto{\pgfqpoint{3.058875in}{2.283493in}}%
\pgfpathlineto{\pgfqpoint{3.058579in}{2.435434in}}%
\pgfpathlineto{\pgfqpoint{3.059022in}{2.411327in}}%
\pgfpathlineto{\pgfqpoint{3.059552in}{2.432430in}}%
\pgfpathlineto{\pgfqpoint{3.059958in}{2.321664in}}%
\pgfpathlineto{\pgfqpoint{3.059982in}{2.274984in}}%
\pgfpathlineto{\pgfqpoint{3.060302in}{2.481418in}}%
\pgfpathlineto{\pgfqpoint{3.061016in}{2.421145in}}%
\pgfpathlineto{\pgfqpoint{3.061545in}{2.444772in}}%
\pgfpathlineto{\pgfqpoint{3.061349in}{2.280217in}}%
\pgfpathlineto{\pgfqpoint{3.062099in}{2.402895in}}%
\pgfpathlineto{\pgfqpoint{3.062592in}{2.288154in}}%
\pgfpathlineto{\pgfqpoint{3.062173in}{2.443383in}}%
\pgfpathlineto{\pgfqpoint{3.063244in}{2.361061in}}%
\pgfpathlineto{\pgfqpoint{3.064019in}{2.462372in}}%
\pgfpathlineto{\pgfqpoint{3.063699in}{2.297429in}}%
\pgfpathlineto{\pgfqpoint{3.064315in}{2.306187in}}%
\pgfpathlineto{\pgfqpoint{3.064450in}{2.281992in}}%
\pgfpathlineto{\pgfqpoint{3.065262in}{2.461223in}}%
\pgfpathlineto{\pgfqpoint{3.065361in}{2.376755in}}%
\pgfpathlineto{\pgfqpoint{3.066493in}{2.451775in}}%
\pgfpathlineto{\pgfqpoint{3.066308in}{2.298052in}}%
\pgfpathlineto{\pgfqpoint{3.066518in}{2.421888in}}%
\pgfpathlineto{\pgfqpoint{3.067552in}{2.279531in}}%
\pgfpathlineto{\pgfqpoint{3.067121in}{2.446491in}}%
\pgfpathlineto{\pgfqpoint{3.067625in}{2.397079in}}%
\pgfpathlineto{\pgfqpoint{3.067736in}{2.452101in}}%
\pgfpathlineto{\pgfqpoint{3.068179in}{2.296628in}}%
\pgfpathlineto{\pgfqpoint{3.068659in}{2.334706in}}%
\pgfpathlineto{\pgfqpoint{3.069422in}{2.307227in}}%
\pgfpathlineto{\pgfqpoint{3.068905in}{2.444127in}}%
\pgfpathlineto{\pgfqpoint{3.069570in}{2.434334in}}%
\pgfpathlineto{\pgfqpoint{3.069607in}{2.462690in}}%
\pgfpathlineto{\pgfqpoint{3.070038in}{2.291997in}}%
\pgfpathlineto{\pgfqpoint{3.070505in}{2.365838in}}%
\pgfpathlineto{\pgfqpoint{3.070665in}{2.304404in}}%
\pgfpathlineto{\pgfqpoint{3.070850in}{2.439398in}}%
\pgfpathlineto{\pgfqpoint{3.071601in}{2.383216in}}%
\pgfpathlineto{\pgfqpoint{3.072081in}{2.440193in}}%
\pgfpathlineto{\pgfqpoint{3.071896in}{2.302665in}}%
\pgfpathlineto{\pgfqpoint{3.072512in}{2.337034in}}%
\pgfpathlineto{\pgfqpoint{3.073152in}{2.276416in}}%
\pgfpathlineto{\pgfqpoint{3.073324in}{2.460582in}}%
\pgfpathlineto{\pgfqpoint{3.073545in}{2.383730in}}%
\pgfpathlineto{\pgfqpoint{3.073619in}{2.352742in}}%
\pgfpathlineto{\pgfqpoint{3.073767in}{2.310933in}}%
\pgfpathlineto{\pgfqpoint{3.074567in}{2.456151in}}%
\pgfpathlineto{\pgfqpoint{3.074665in}{2.417497in}}%
\pgfpathlineto{\pgfqpoint{3.075195in}{2.448401in}}%
\pgfpathlineto{\pgfqpoint{3.075010in}{2.292970in}}%
\pgfpathlineto{\pgfqpoint{3.075736in}{2.398563in}}%
\pgfpathlineto{\pgfqpoint{3.076253in}{2.299694in}}%
\pgfpathlineto{\pgfqpoint{3.076438in}{2.437325in}}%
\pgfpathlineto{\pgfqpoint{3.076905in}{2.340372in}}%
\pgfpathlineto{\pgfqpoint{3.077053in}{2.447006in}}%
\pgfpathlineto{\pgfqpoint{3.077496in}{2.325224in}}%
\pgfpathlineto{\pgfqpoint{3.078013in}{2.368850in}}%
\pgfpathlineto{\pgfqpoint{3.078124in}{2.287775in}}%
\pgfpathlineto{\pgfqpoint{3.078296in}{2.450150in}}%
\pgfpathlineto{\pgfqpoint{3.079096in}{2.385269in}}%
\pgfpathlineto{\pgfqpoint{3.080142in}{2.446585in}}%
\pgfpathlineto{\pgfqpoint{3.079367in}{2.303736in}}%
\pgfpathlineto{\pgfqpoint{3.080179in}{2.391984in}}%
\pgfpathlineto{\pgfqpoint{3.081188in}{2.312303in}}%
\pgfpathlineto{\pgfqpoint{3.080770in}{2.458163in}}%
\pgfpathlineto{\pgfqpoint{3.081287in}{2.392158in}}%
\pgfpathlineto{\pgfqpoint{3.081385in}{2.449501in}}%
\pgfpathlineto{\pgfqpoint{3.081816in}{2.300457in}}%
\pgfpathlineto{\pgfqpoint{3.082358in}{2.373473in}}%
\pgfpathlineto{\pgfqpoint{3.082444in}{2.313553in}}%
\pgfpathlineto{\pgfqpoint{3.082862in}{2.425049in}}%
\pgfpathlineto{\pgfqpoint{3.083453in}{2.379732in}}%
\pgfpathlineto{\pgfqpoint{3.084487in}{2.447688in}}%
\pgfpathlineto{\pgfqpoint{3.084339in}{2.298613in}}%
\pgfpathlineto{\pgfqpoint{3.084598in}{2.417576in}}%
\pgfpathlineto{\pgfqpoint{3.085668in}{2.291269in}}%
\pgfpathlineto{\pgfqpoint{3.085311in}{2.448404in}}%
\pgfpathlineto{\pgfqpoint{3.085718in}{2.394446in}}%
\pgfpathlineto{\pgfqpoint{3.086124in}{2.440498in}}%
\pgfpathlineto{\pgfqpoint{3.086247in}{2.320048in}}%
\pgfpathlineto{\pgfqpoint{3.086801in}{2.385167in}}%
\pgfpathlineto{\pgfqpoint{3.087527in}{2.293394in}}%
\pgfpathlineto{\pgfqpoint{3.087798in}{2.436234in}}%
\pgfpathlineto{\pgfqpoint{3.087908in}{2.376710in}}%
\pgfpathlineto{\pgfqpoint{3.088770in}{2.292469in}}%
\pgfpathlineto{\pgfqpoint{3.089041in}{2.449905in}}%
\pgfpathlineto{\pgfqpoint{3.089496in}{2.282405in}}%
\pgfpathlineto{\pgfqpoint{3.090161in}{2.384738in}}%
\pgfpathlineto{\pgfqpoint{3.090875in}{2.448073in}}%
\pgfpathlineto{\pgfqpoint{3.090616in}{2.306396in}}%
\pgfpathlineto{\pgfqpoint{3.091231in}{2.318834in}}%
\pgfpathlineto{\pgfqpoint{3.091330in}{2.284292in}}%
\pgfpathlineto{\pgfqpoint{3.091293in}{2.356152in}}%
\pgfpathlineto{\pgfqpoint{3.091404in}{2.341753in}}%
\pgfpathlineto{\pgfqpoint{3.092265in}{2.465179in}}%
\pgfpathlineto{\pgfqpoint{3.091995in}{2.248142in}}%
\pgfpathlineto{\pgfqpoint{3.092499in}{2.348768in}}%
\pgfpathlineto{\pgfqpoint{3.093238in}{2.235472in}}%
\pgfpathlineto{\pgfqpoint{3.092893in}{2.468406in}}%
\pgfpathlineto{\pgfqpoint{3.093582in}{2.397277in}}%
\pgfpathlineto{\pgfqpoint{3.094628in}{2.484319in}}%
\pgfpathlineto{\pgfqpoint{3.094468in}{2.249934in}}%
\pgfpathlineto{\pgfqpoint{3.094739in}{2.451544in}}%
\pgfpathlineto{\pgfqpoint{3.095687in}{2.255026in}}%
\pgfpathlineto{\pgfqpoint{3.095428in}{2.456383in}}%
\pgfpathlineto{\pgfqpoint{3.095921in}{2.415157in}}%
\pgfpathlineto{\pgfqpoint{3.096622in}{2.450768in}}%
\pgfpathlineto{\pgfqpoint{3.096351in}{2.280375in}}%
\pgfpathlineto{\pgfqpoint{3.096942in}{2.325727in}}%
\pgfpathlineto{\pgfqpoint{3.097582in}{2.267775in}}%
\pgfpathlineto{\pgfqpoint{3.097755in}{2.472148in}}%
\pgfpathlineto{\pgfqpoint{3.097828in}{2.445581in}}%
\pgfpathlineto{\pgfqpoint{3.097841in}{2.482690in}}%
\pgfpathlineto{\pgfqpoint{3.098813in}{2.207330in}}%
\pgfpathlineto{\pgfqpoint{3.098887in}{2.334498in}}%
\pgfpathlineto{\pgfqpoint{3.100007in}{2.271122in}}%
\pgfpathlineto{\pgfqpoint{3.098973in}{2.481004in}}%
\pgfpathlineto{\pgfqpoint{3.100056in}{2.272272in}}%
\pgfpathlineto{\pgfqpoint{3.100942in}{2.476501in}}%
\pgfpathlineto{\pgfqpoint{3.100684in}{2.260926in}}%
\pgfpathlineto{\pgfqpoint{3.101188in}{2.289803in}}%
\pgfpathlineto{\pgfqpoint{3.101459in}{2.460187in}}%
\pgfpathlineto{\pgfqpoint{3.101902in}{2.218856in}}%
\pgfpathlineto{\pgfqpoint{3.102468in}{2.360778in}}%
\pgfpathlineto{\pgfqpoint{3.103158in}{2.218084in}}%
\pgfpathlineto{\pgfqpoint{3.103318in}{2.502870in}}%
\pgfpathlineto{\pgfqpoint{3.103576in}{2.351393in}}%
\pgfpathlineto{\pgfqpoint{3.104105in}{2.467865in}}%
\pgfpathlineto{\pgfqpoint{3.104401in}{2.244968in}}%
\pgfpathlineto{\pgfqpoint{3.104708in}{2.407654in}}%
\pgfpathlineto{\pgfqpoint{3.105631in}{2.256824in}}%
\pgfpathlineto{\pgfqpoint{3.105299in}{2.485480in}}%
\pgfpathlineto{\pgfqpoint{3.105767in}{2.393634in}}%
\pgfpathlineto{\pgfqpoint{3.106087in}{2.463732in}}%
\pgfpathlineto{\pgfqpoint{3.106247in}{2.224677in}}%
\pgfpathlineto{\pgfqpoint{3.106838in}{2.322329in}}%
\pgfpathlineto{\pgfqpoint{3.107502in}{2.230934in}}%
\pgfpathlineto{\pgfqpoint{3.107662in}{2.457845in}}%
\pgfpathlineto{\pgfqpoint{3.107921in}{2.364853in}}%
\pgfpathlineto{\pgfqpoint{3.108450in}{2.455496in}}%
\pgfpathlineto{\pgfqpoint{3.108105in}{2.248858in}}%
\pgfpathlineto{\pgfqpoint{3.109041in}{2.414896in}}%
\pgfpathlineto{\pgfqpoint{3.109976in}{2.257626in}}%
\pgfpathlineto{\pgfqpoint{3.109644in}{2.472905in}}%
\pgfpathlineto{\pgfqpoint{3.110222in}{2.400105in}}%
\pgfpathlineto{\pgfqpoint{3.110431in}{2.449180in}}%
\pgfpathlineto{\pgfqpoint{3.110591in}{2.248991in}}%
\pgfpathlineto{\pgfqpoint{3.111281in}{2.361740in}}%
\pgfpathlineto{\pgfqpoint{3.111847in}{2.258844in}}%
\pgfpathlineto{\pgfqpoint{3.111613in}{2.469673in}}%
\pgfpathlineto{\pgfqpoint{3.112376in}{2.368671in}}%
\pgfpathlineto{\pgfqpoint{3.113201in}{2.452195in}}%
\pgfpathlineto{\pgfqpoint{3.112450in}{2.258786in}}%
\pgfpathlineto{\pgfqpoint{3.113496in}{2.423888in}}%
\pgfpathlineto{\pgfqpoint{3.113681in}{2.274427in}}%
\pgfpathlineto{\pgfqpoint{3.113988in}{2.433486in}}%
\pgfpathlineto{\pgfqpoint{3.114690in}{2.397265in}}%
\pgfpathlineto{\pgfqpoint{3.115071in}{2.430989in}}%
\pgfpathlineto{\pgfqpoint{3.114924in}{2.243952in}}%
\pgfpathlineto{\pgfqpoint{3.115773in}{2.395754in}}%
\pgfpathlineto{\pgfqpoint{3.116154in}{2.285351in}}%
\pgfpathlineto{\pgfqpoint{3.115958in}{2.451769in}}%
\pgfpathlineto{\pgfqpoint{3.116905in}{2.361327in}}%
\pgfpathlineto{\pgfqpoint{3.117533in}{2.449946in}}%
\pgfpathlineto{\pgfqpoint{3.117385in}{2.251318in}}%
\pgfpathlineto{\pgfqpoint{3.117988in}{2.296583in}}%
\pgfpathlineto{\pgfqpoint{3.118001in}{2.267232in}}%
\pgfpathlineto{\pgfqpoint{3.118321in}{2.440609in}}%
\pgfpathlineto{\pgfqpoint{3.119047in}{2.419632in}}%
\pgfpathlineto{\pgfqpoint{3.119244in}{2.272023in}}%
\pgfpathlineto{\pgfqpoint{3.119908in}{2.444720in}}%
\pgfpathlineto{\pgfqpoint{3.120191in}{2.380906in}}%
\pgfpathlineto{\pgfqpoint{3.120462in}{2.294458in}}%
\pgfpathlineto{\pgfqpoint{3.120302in}{2.453256in}}%
\pgfpathlineto{\pgfqpoint{3.121250in}{2.368654in}}%
\pgfpathlineto{\pgfqpoint{3.121890in}{2.447048in}}%
\pgfpathlineto{\pgfqpoint{3.121718in}{2.275889in}}%
\pgfpathlineto{\pgfqpoint{3.122333in}{2.311759in}}%
\pgfpathlineto{\pgfqpoint{3.122776in}{2.428143in}}%
\pgfpathlineto{\pgfqpoint{3.122924in}{2.291587in}}%
\pgfpathlineto{\pgfqpoint{3.123527in}{2.344556in}}%
\pgfpathlineto{\pgfqpoint{3.124093in}{2.317251in}}%
\pgfpathlineto{\pgfqpoint{3.123871in}{2.413475in}}%
\pgfpathlineto{\pgfqpoint{3.124598in}{2.385480in}}%
\pgfpathlineto{\pgfqpoint{3.124647in}{2.454251in}}%
\pgfpathlineto{\pgfqpoint{3.124794in}{2.294100in}}%
\pgfpathlineto{\pgfqpoint{3.125681in}{2.333800in}}%
\pgfpathlineto{\pgfqpoint{3.126222in}{2.430650in}}%
\pgfpathlineto{\pgfqpoint{3.125988in}{2.323948in}}%
\pgfpathlineto{\pgfqpoint{3.126825in}{2.399385in}}%
\pgfpathlineto{\pgfqpoint{3.127256in}{2.273712in}}%
\pgfpathlineto{\pgfqpoint{3.127404in}{2.469792in}}%
\pgfpathlineto{\pgfqpoint{3.127957in}{2.338440in}}%
\pgfpathlineto{\pgfqpoint{3.128979in}{2.443439in}}%
\pgfpathlineto{\pgfqpoint{3.128831in}{2.292016in}}%
\pgfpathlineto{\pgfqpoint{3.129065in}{2.351996in}}%
\pgfpathlineto{\pgfqpoint{3.129730in}{2.306184in}}%
\pgfpathlineto{\pgfqpoint{3.129779in}{2.463643in}}%
\pgfpathlineto{\pgfqpoint{3.130136in}{2.373554in}}%
\pgfpathlineto{\pgfqpoint{3.130567in}{2.435019in}}%
\pgfpathlineto{\pgfqpoint{3.130333in}{2.293929in}}%
\pgfpathlineto{\pgfqpoint{3.131244in}{2.391922in}}%
\pgfpathlineto{\pgfqpoint{3.131588in}{2.275041in}}%
\pgfpathlineto{\pgfqpoint{3.131748in}{2.464270in}}%
\pgfpathlineto{\pgfqpoint{3.132339in}{2.407821in}}%
\pgfpathlineto{\pgfqpoint{3.133422in}{2.307415in}}%
\pgfpathlineto{\pgfqpoint{3.132942in}{2.455685in}}%
\pgfpathlineto{\pgfqpoint{3.133471in}{2.369890in}}%
\pgfpathlineto{\pgfqpoint{3.134099in}{2.448182in}}%
\pgfpathlineto{\pgfqpoint{3.134050in}{2.306375in}}%
\pgfpathlineto{\pgfqpoint{3.134567in}{2.358017in}}%
\pgfpathlineto{\pgfqpoint{3.134665in}{2.302580in}}%
\pgfpathlineto{\pgfqpoint{3.134911in}{2.440837in}}%
\pgfpathlineto{\pgfqpoint{3.135317in}{2.381112in}}%
\pgfpathlineto{\pgfqpoint{3.136081in}{2.448638in}}%
\pgfpathlineto{\pgfqpoint{3.135908in}{2.261383in}}%
\pgfpathlineto{\pgfqpoint{3.136413in}{2.363715in}}%
\pgfpathlineto{\pgfqpoint{3.136524in}{2.288480in}}%
\pgfpathlineto{\pgfqpoint{3.136770in}{2.424235in}}%
\pgfpathlineto{\pgfqpoint{3.137508in}{2.358882in}}%
\pgfpathlineto{\pgfqpoint{3.138530in}{2.424663in}}%
\pgfpathlineto{\pgfqpoint{3.137754in}{2.291650in}}%
\pgfpathlineto{\pgfqpoint{3.138641in}{2.386735in}}%
\pgfpathlineto{\pgfqpoint{3.138997in}{2.304114in}}%
\pgfpathlineto{\pgfqpoint{3.139145in}{2.431684in}}%
\pgfpathlineto{\pgfqpoint{3.139736in}{2.373165in}}%
\pgfpathlineto{\pgfqpoint{3.140376in}{2.447180in}}%
\pgfpathlineto{\pgfqpoint{3.140216in}{2.286254in}}%
\pgfpathlineto{\pgfqpoint{3.140807in}{2.339221in}}%
\pgfpathlineto{\pgfqpoint{3.140831in}{2.303806in}}%
\pgfpathlineto{\pgfqpoint{3.141644in}{2.430155in}}%
\pgfpathlineto{\pgfqpoint{3.141890in}{2.386491in}}%
\pgfpathlineto{\pgfqpoint{3.142074in}{2.297240in}}%
\pgfpathlineto{\pgfqpoint{3.142333in}{2.442251in}}%
\pgfpathlineto{\pgfqpoint{3.143022in}{2.368481in}}%
\pgfpathlineto{\pgfqpoint{3.143490in}{2.451349in}}%
\pgfpathlineto{\pgfqpoint{3.143305in}{2.299794in}}%
\pgfpathlineto{\pgfqpoint{3.143908in}{2.341389in}}%
\pgfpathlineto{\pgfqpoint{3.143933in}{2.287002in}}%
\pgfpathlineto{\pgfqpoint{3.144179in}{2.426327in}}%
\pgfpathlineto{\pgfqpoint{3.144991in}{2.390943in}}%
\pgfpathlineto{\pgfqpoint{3.145348in}{2.420179in}}%
\pgfpathlineto{\pgfqpoint{3.145151in}{2.310857in}}%
\pgfpathlineto{\pgfqpoint{3.146062in}{2.366112in}}%
\pgfpathlineto{\pgfqpoint{3.146394in}{2.322405in}}%
\pgfpathlineto{\pgfqpoint{3.146640in}{2.435110in}}%
\pgfpathlineto{\pgfqpoint{3.147157in}{2.375877in}}%
\pgfpathlineto{\pgfqpoint{3.147810in}{2.439626in}}%
\pgfpathlineto{\pgfqpoint{3.147637in}{2.294234in}}%
\pgfpathlineto{\pgfqpoint{3.148228in}{2.344969in}}%
\pgfpathlineto{\pgfqpoint{3.148253in}{2.303926in}}%
\pgfpathlineto{\pgfqpoint{3.148917in}{2.413624in}}%
\pgfpathlineto{\pgfqpoint{3.149324in}{2.372222in}}%
\pgfpathlineto{\pgfqpoint{3.149742in}{2.430102in}}%
\pgfpathlineto{\pgfqpoint{3.149484in}{2.301817in}}%
\pgfpathlineto{\pgfqpoint{3.150431in}{2.370482in}}%
\pgfpathlineto{\pgfqpoint{3.150924in}{2.418518in}}%
\pgfpathlineto{\pgfqpoint{3.150714in}{2.315398in}}%
\pgfpathlineto{\pgfqpoint{3.151600in}{2.397187in}}%
\pgfpathlineto{\pgfqpoint{3.151945in}{2.308390in}}%
\pgfpathlineto{\pgfqpoint{3.152105in}{2.435028in}}%
\pgfpathlineto{\pgfqpoint{3.152733in}{2.379068in}}%
\pgfpathlineto{\pgfqpoint{3.153373in}{2.420867in}}%
\pgfpathlineto{\pgfqpoint{3.153188in}{2.313102in}}%
\pgfpathlineto{\pgfqpoint{3.153791in}{2.335362in}}%
\pgfpathlineto{\pgfqpoint{3.153816in}{2.289267in}}%
\pgfpathlineto{\pgfqpoint{3.154062in}{2.438008in}}%
\pgfpathlineto{\pgfqpoint{3.154887in}{2.355923in}}%
\pgfpathlineto{\pgfqpoint{3.155231in}{2.453689in}}%
\pgfpathlineto{\pgfqpoint{3.155059in}{2.320567in}}%
\pgfpathlineto{\pgfqpoint{3.155650in}{2.341475in}}%
\pgfpathlineto{\pgfqpoint{3.156265in}{2.303020in}}%
\pgfpathlineto{\pgfqpoint{3.156425in}{2.435532in}}%
\pgfpathlineto{\pgfqpoint{3.156745in}{2.364819in}}%
\pgfpathlineto{\pgfqpoint{3.157693in}{2.424115in}}%
\pgfpathlineto{\pgfqpoint{3.157520in}{2.314034in}}%
\pgfpathlineto{\pgfqpoint{3.157816in}{2.358657in}}%
\pgfpathlineto{\pgfqpoint{3.158136in}{2.295002in}}%
\pgfpathlineto{\pgfqpoint{3.158382in}{2.447553in}}%
\pgfpathlineto{\pgfqpoint{3.158874in}{2.408864in}}%
\pgfpathlineto{\pgfqpoint{3.159551in}{2.450073in}}%
\pgfpathlineto{\pgfqpoint{3.159797in}{2.318646in}}%
\pgfpathlineto{\pgfqpoint{3.159957in}{2.389347in}}%
\pgfpathlineto{\pgfqpoint{3.160585in}{2.281770in}}%
\pgfpathlineto{\pgfqpoint{3.160745in}{2.457008in}}%
\pgfpathlineto{\pgfqpoint{3.161077in}{2.339443in}}%
\pgfpathlineto{\pgfqpoint{3.161533in}{2.435952in}}%
\pgfpathlineto{\pgfqpoint{3.161840in}{2.313295in}}%
\pgfpathlineto{\pgfqpoint{3.162197in}{2.350369in}}%
\pgfpathlineto{\pgfqpoint{3.162948in}{2.292949in}}%
\pgfpathlineto{\pgfqpoint{3.162702in}{2.444671in}}%
\pgfpathlineto{\pgfqpoint{3.163096in}{2.401721in}}%
\pgfpathlineto{\pgfqpoint{3.163884in}{2.450716in}}%
\pgfpathlineto{\pgfqpoint{3.164130in}{2.306926in}}%
\pgfpathlineto{\pgfqpoint{3.164179in}{2.340951in}}%
\pgfpathlineto{\pgfqpoint{3.164930in}{2.253178in}}%
\pgfpathlineto{\pgfqpoint{3.164474in}{2.446390in}}%
\pgfpathlineto{\pgfqpoint{3.165065in}{2.440223in}}%
\pgfpathlineto{\pgfqpoint{3.165090in}{2.473576in}}%
\pgfpathlineto{\pgfqpoint{3.166111in}{2.299679in}}%
\pgfpathlineto{\pgfqpoint{3.166124in}{2.297054in}}%
\pgfpathlineto{\pgfqpoint{3.166345in}{2.440208in}}%
\pgfpathlineto{\pgfqpoint{3.166825in}{2.375336in}}%
\pgfpathlineto{\pgfqpoint{3.167563in}{2.466959in}}%
\pgfpathlineto{\pgfqpoint{3.167391in}{2.276815in}}%
\pgfpathlineto{\pgfqpoint{3.167883in}{2.356729in}}%
\pgfpathlineto{\pgfqpoint{3.168007in}{2.255062in}}%
\pgfpathlineto{\pgfqpoint{3.168253in}{2.468959in}}%
\pgfpathlineto{\pgfqpoint{3.168991in}{2.332244in}}%
\pgfpathlineto{\pgfqpoint{3.169434in}{2.493232in}}%
\pgfpathlineto{\pgfqpoint{3.169262in}{2.279690in}}%
\pgfpathlineto{\pgfqpoint{3.170136in}{2.424741in}}%
\pgfpathlineto{\pgfqpoint{3.170468in}{2.254013in}}%
\pgfpathlineto{\pgfqpoint{3.170628in}{2.469232in}}%
\pgfpathlineto{\pgfqpoint{3.171268in}{2.386147in}}%
\pgfpathlineto{\pgfqpoint{3.171416in}{2.460844in}}%
\pgfpathlineto{\pgfqpoint{3.172339in}{2.277658in}}%
\pgfpathlineto{\pgfqpoint{3.172363in}{2.310385in}}%
\pgfpathlineto{\pgfqpoint{3.172597in}{2.469805in}}%
\pgfpathlineto{\pgfqpoint{3.172843in}{2.283486in}}%
\pgfpathlineto{\pgfqpoint{3.173471in}{2.323820in}}%
\pgfpathlineto{\pgfqpoint{3.173779in}{2.463786in}}%
\pgfpathlineto{\pgfqpoint{3.173557in}{2.271242in}}%
\pgfpathlineto{\pgfqpoint{3.174603in}{2.386634in}}%
\pgfpathlineto{\pgfqpoint{3.174813in}{2.262006in}}%
\pgfpathlineto{\pgfqpoint{3.174973in}{2.481612in}}%
\pgfpathlineto{\pgfqpoint{3.175699in}{2.407631in}}%
\pgfpathlineto{\pgfqpoint{3.176007in}{2.299820in}}%
\pgfpathlineto{\pgfqpoint{3.175760in}{2.450472in}}%
\pgfpathlineto{\pgfqpoint{3.176807in}{2.385062in}}%
\pgfpathlineto{\pgfqpoint{3.176942in}{2.463232in}}%
\pgfpathlineto{\pgfqpoint{3.177188in}{2.285697in}}%
\pgfpathlineto{\pgfqpoint{3.177877in}{2.287761in}}%
\pgfpathlineto{\pgfqpoint{3.177890in}{2.266851in}}%
\pgfpathlineto{\pgfqpoint{3.178136in}{2.460355in}}%
\pgfpathlineto{\pgfqpoint{3.178899in}{2.399284in}}%
\pgfpathlineto{\pgfqpoint{3.179317in}{2.477674in}}%
\pgfpathlineto{\pgfqpoint{3.179133in}{2.276483in}}%
\pgfpathlineto{\pgfqpoint{3.180007in}{2.430550in}}%
\pgfpathlineto{\pgfqpoint{3.180979in}{2.286666in}}%
\pgfpathlineto{\pgfqpoint{3.180499in}{2.448750in}}%
\pgfpathlineto{\pgfqpoint{3.181139in}{2.406984in}}%
\pgfpathlineto{\pgfqpoint{3.181287in}{2.434740in}}%
\pgfpathlineto{\pgfqpoint{3.181508in}{2.303178in}}%
\pgfpathlineto{\pgfqpoint{3.182185in}{2.337589in}}%
\pgfpathlineto{\pgfqpoint{3.182210in}{2.286798in}}%
\pgfpathlineto{\pgfqpoint{3.182456in}{2.446409in}}%
\pgfpathlineto{\pgfqpoint{3.183280in}{2.375411in}}%
\pgfpathlineto{\pgfqpoint{3.183453in}{2.279114in}}%
\pgfpathlineto{\pgfqpoint{3.183625in}{2.449783in}}%
\pgfpathlineto{\pgfqpoint{3.184363in}{2.405808in}}%
\pgfpathlineto{\pgfqpoint{3.185299in}{2.288034in}}%
\pgfpathlineto{\pgfqpoint{3.184843in}{2.420142in}}%
\pgfpathlineto{\pgfqpoint{3.185447in}{2.394239in}}%
\pgfpathlineto{\pgfqpoint{3.185570in}{2.430545in}}%
\pgfpathlineto{\pgfqpoint{3.185816in}{2.309464in}}%
\pgfpathlineto{\pgfqpoint{3.186505in}{2.317353in}}%
\pgfpathlineto{\pgfqpoint{3.186542in}{2.295631in}}%
\pgfpathlineto{\pgfqpoint{3.186702in}{2.442924in}}%
\pgfpathlineto{\pgfqpoint{3.187330in}{2.406001in}}%
\pgfpathlineto{\pgfqpoint{3.187379in}{2.387903in}}%
\pgfpathlineto{\pgfqpoint{3.187416in}{2.409891in}}%
\pgfpathlineto{\pgfqpoint{3.187440in}{2.396168in}}%
\pgfpathlineto{\pgfqpoint{3.188388in}{2.293335in}}%
\pgfpathlineto{\pgfqpoint{3.187933in}{2.456222in}}%
\pgfpathlineto{\pgfqpoint{3.188536in}{2.396515in}}%
\pgfpathlineto{\pgfqpoint{3.188708in}{2.434511in}}%
\pgfpathlineto{\pgfqpoint{3.189016in}{2.315391in}}%
\pgfpathlineto{\pgfqpoint{3.189594in}{2.360672in}}%
\pgfpathlineto{\pgfqpoint{3.189631in}{2.302635in}}%
\pgfpathlineto{\pgfqpoint{3.189877in}{2.450307in}}%
\pgfpathlineto{\pgfqpoint{3.190714in}{2.342055in}}%
\pgfpathlineto{\pgfqpoint{3.191046in}{2.462734in}}%
\pgfpathlineto{\pgfqpoint{3.190862in}{2.284570in}}%
\pgfpathlineto{\pgfqpoint{3.191871in}{2.369598in}}%
\pgfpathlineto{\pgfqpoint{3.192093in}{2.308040in}}%
\pgfpathlineto{\pgfqpoint{3.192228in}{2.426973in}}%
\pgfpathlineto{\pgfqpoint{3.192954in}{2.389134in}}%
\pgfpathlineto{\pgfqpoint{3.193496in}{2.436124in}}%
\pgfpathlineto{\pgfqpoint{3.193951in}{2.299185in}}%
\pgfpathlineto{\pgfqpoint{3.194037in}{2.364767in}}%
\pgfpathlineto{\pgfqpoint{3.194579in}{2.327787in}}%
\pgfpathlineto{\pgfqpoint{3.194111in}{2.437991in}}%
\pgfpathlineto{\pgfqpoint{3.194825in}{2.404755in}}%
\pgfpathlineto{\pgfqpoint{3.194948in}{2.407354in}}%
\pgfpathlineto{\pgfqpoint{3.195133in}{2.335688in}}%
\pgfpathlineto{\pgfqpoint{3.195157in}{2.338490in}}%
\pgfpathlineto{\pgfqpoint{3.195810in}{2.291650in}}%
\pgfpathlineto{\pgfqpoint{3.195342in}{2.450471in}}%
\pgfpathlineto{\pgfqpoint{3.196240in}{2.354585in}}%
\pgfpathlineto{\pgfqpoint{3.197225in}{2.429774in}}%
\pgfpathlineto{\pgfqpoint{3.197028in}{2.303829in}}%
\pgfpathlineto{\pgfqpoint{3.197373in}{2.373480in}}%
\pgfpathlineto{\pgfqpoint{3.198283in}{2.297823in}}%
\pgfpathlineto{\pgfqpoint{3.197828in}{2.424672in}}%
\pgfpathlineto{\pgfqpoint{3.198419in}{2.387284in}}%
\pgfpathlineto{\pgfqpoint{3.198456in}{2.454164in}}%
\pgfpathlineto{\pgfqpoint{3.198899in}{2.293719in}}%
\pgfpathlineto{\pgfqpoint{3.199502in}{2.332131in}}%
\pgfpathlineto{\pgfqpoint{3.200142in}{2.307788in}}%
\pgfpathlineto{\pgfqpoint{3.200290in}{2.437288in}}%
\pgfpathlineto{\pgfqpoint{3.200585in}{2.354999in}}%
\pgfpathlineto{\pgfqpoint{3.200905in}{2.432463in}}%
\pgfpathlineto{\pgfqpoint{3.200757in}{2.304164in}}%
\pgfpathlineto{\pgfqpoint{3.201348in}{2.319091in}}%
\pgfpathlineto{\pgfqpoint{3.201373in}{2.291696in}}%
\pgfpathlineto{\pgfqpoint{3.201570in}{2.427972in}}%
\pgfpathlineto{\pgfqpoint{3.202431in}{2.352954in}}%
\pgfpathlineto{\pgfqpoint{3.202763in}{2.459131in}}%
\pgfpathlineto{\pgfqpoint{3.202591in}{2.283270in}}%
\pgfpathlineto{\pgfqpoint{3.203563in}{2.386870in}}%
\pgfpathlineto{\pgfqpoint{3.203834in}{2.305144in}}%
\pgfpathlineto{\pgfqpoint{3.204610in}{2.427791in}}%
\pgfpathlineto{\pgfqpoint{3.204671in}{2.388199in}}%
\pgfpathlineto{\pgfqpoint{3.204696in}{2.410466in}}%
\pgfpathlineto{\pgfqpoint{3.204720in}{2.448483in}}%
\pgfpathlineto{\pgfqpoint{3.205065in}{2.301732in}}%
\pgfpathlineto{\pgfqpoint{3.205791in}{2.371960in}}%
\pgfpathlineto{\pgfqpoint{3.205890in}{2.444856in}}%
\pgfpathlineto{\pgfqpoint{3.206136in}{2.322913in}}%
\pgfpathlineto{\pgfqpoint{3.206874in}{2.368997in}}%
\pgfpathlineto{\pgfqpoint{3.206923in}{2.253353in}}%
\pgfpathlineto{\pgfqpoint{3.207083in}{2.475720in}}%
\pgfpathlineto{\pgfqpoint{3.207969in}{2.378087in}}%
\pgfpathlineto{\pgfqpoint{3.209053in}{2.456990in}}%
\pgfpathlineto{\pgfqpoint{3.208105in}{2.296360in}}%
\pgfpathlineto{\pgfqpoint{3.209102in}{2.390382in}}%
\pgfpathlineto{\pgfqpoint{3.209397in}{2.274105in}}%
\pgfpathlineto{\pgfqpoint{3.209446in}{2.442208in}}%
\pgfpathlineto{\pgfqpoint{3.210185in}{2.400536in}}%
\pgfpathlineto{\pgfqpoint{3.210246in}{2.441657in}}%
\pgfpathlineto{\pgfqpoint{3.210493in}{2.313100in}}%
\pgfpathlineto{\pgfqpoint{3.211169in}{2.350022in}}%
\pgfpathlineto{\pgfqpoint{3.211256in}{2.279019in}}%
\pgfpathlineto{\pgfqpoint{3.211428in}{2.470536in}}%
\pgfpathlineto{\pgfqpoint{3.212216in}{2.411845in}}%
\pgfpathlineto{\pgfqpoint{3.212228in}{2.412601in}}%
\pgfpathlineto{\pgfqpoint{3.212351in}{2.369604in}}%
\pgfpathlineto{\pgfqpoint{3.212474in}{2.246574in}}%
\pgfpathlineto{\pgfqpoint{3.212634in}{2.465900in}}%
\pgfpathlineto{\pgfqpoint{3.213446in}{2.397015in}}%
\pgfpathlineto{\pgfqpoint{3.213656in}{2.290968in}}%
\pgfpathlineto{\pgfqpoint{3.213902in}{2.451622in}}%
\pgfpathlineto{\pgfqpoint{3.214554in}{2.368029in}}%
\pgfpathlineto{\pgfqpoint{3.214603in}{2.461039in}}%
\pgfpathlineto{\pgfqpoint{3.215563in}{2.266941in}}%
\pgfpathlineto{\pgfqpoint{3.215637in}{2.323603in}}%
\pgfpathlineto{\pgfqpoint{3.216043in}{2.309002in}}%
\pgfpathlineto{\pgfqpoint{3.215809in}{2.440949in}}%
\pgfpathlineto{\pgfqpoint{3.216659in}{2.383623in}}%
\pgfpathlineto{\pgfqpoint{3.216819in}{2.266182in}}%
\pgfpathlineto{\pgfqpoint{3.216991in}{2.473130in}}%
\pgfpathlineto{\pgfqpoint{3.217557in}{2.398835in}}%
\pgfpathlineto{\pgfqpoint{3.218185in}{2.467546in}}%
\pgfpathlineto{\pgfqpoint{3.218025in}{2.260540in}}%
\pgfpathlineto{\pgfqpoint{3.218640in}{2.319364in}}%
\pgfpathlineto{\pgfqpoint{3.219206in}{2.291827in}}%
\pgfpathlineto{\pgfqpoint{3.218973in}{2.446686in}}%
\pgfpathlineto{\pgfqpoint{3.219662in}{2.381433in}}%
\pgfpathlineto{\pgfqpoint{3.219908in}{2.263687in}}%
\pgfpathlineto{\pgfqpoint{3.220154in}{2.465181in}}%
\pgfpathlineto{\pgfqpoint{3.220634in}{2.395363in}}%
\pgfpathlineto{\pgfqpoint{3.221323in}{2.450124in}}%
\pgfpathlineto{\pgfqpoint{3.221102in}{2.287626in}}%
\pgfpathlineto{\pgfqpoint{3.221729in}{2.365271in}}%
\pgfpathlineto{\pgfqpoint{3.222369in}{2.247412in}}%
\pgfpathlineto{\pgfqpoint{3.222529in}{2.472023in}}%
\pgfpathlineto{\pgfqpoint{3.222813in}{2.385318in}}%
\pgfpathlineto{\pgfqpoint{3.222849in}{2.362880in}}%
\pgfpathlineto{\pgfqpoint{3.222985in}{2.285011in}}%
\pgfpathlineto{\pgfqpoint{3.223120in}{2.428595in}}%
\pgfpathlineto{\pgfqpoint{3.223969in}{2.348676in}}%
\pgfpathlineto{\pgfqpoint{3.224499in}{2.455249in}}%
\pgfpathlineto{\pgfqpoint{3.224843in}{2.285729in}}%
\pgfpathlineto{\pgfqpoint{3.225102in}{2.428196in}}%
\pgfpathlineto{\pgfqpoint{3.225446in}{2.285252in}}%
\pgfpathlineto{\pgfqpoint{3.225680in}{2.435027in}}%
\pgfpathlineto{\pgfqpoint{3.226222in}{2.406497in}}%
\pgfpathlineto{\pgfqpoint{3.226714in}{2.281588in}}%
\pgfpathlineto{\pgfqpoint{3.226862in}{2.473783in}}%
\pgfpathlineto{\pgfqpoint{3.227354in}{2.365318in}}%
\pgfpathlineto{\pgfqpoint{3.227465in}{2.437233in}}%
\pgfpathlineto{\pgfqpoint{3.227896in}{2.308248in}}%
\pgfpathlineto{\pgfqpoint{3.228462in}{2.368658in}}%
\pgfpathlineto{\pgfqpoint{3.229176in}{2.306120in}}%
\pgfpathlineto{\pgfqpoint{3.228720in}{2.450599in}}%
\pgfpathlineto{\pgfqpoint{3.229569in}{2.367995in}}%
\pgfpathlineto{\pgfqpoint{3.229779in}{2.308067in}}%
\pgfpathlineto{\pgfqpoint{3.230579in}{2.442138in}}%
\pgfpathlineto{\pgfqpoint{3.230652in}{2.382474in}}%
\pgfpathlineto{\pgfqpoint{3.231206in}{2.467506in}}%
\pgfpathlineto{\pgfqpoint{3.231034in}{2.282171in}}%
\pgfpathlineto{\pgfqpoint{3.231760in}{2.387533in}}%
\pgfpathlineto{\pgfqpoint{3.232412in}{2.440019in}}%
\pgfpathlineto{\pgfqpoint{3.232265in}{2.321663in}}%
\pgfpathlineto{\pgfqpoint{3.232806in}{2.362536in}}%
\pgfpathlineto{\pgfqpoint{3.232892in}{2.300576in}}%
\pgfpathlineto{\pgfqpoint{3.233052in}{2.440774in}}%
\pgfpathlineto{\pgfqpoint{3.233902in}{2.364455in}}%
\pgfpathlineto{\pgfqpoint{3.234308in}{2.441233in}}%
\pgfpathlineto{\pgfqpoint{3.234136in}{2.310627in}}%
\pgfpathlineto{\pgfqpoint{3.235034in}{2.396495in}}%
\pgfpathlineto{\pgfqpoint{3.235046in}{2.397172in}}%
\pgfpathlineto{\pgfqpoint{3.235206in}{2.358828in}}%
\pgfpathlineto{\pgfqpoint{3.235354in}{2.306361in}}%
\pgfpathlineto{\pgfqpoint{3.235502in}{2.455255in}}%
\pgfpathlineto{\pgfqpoint{3.236265in}{2.419950in}}%
\pgfpathlineto{\pgfqpoint{3.237200in}{2.300180in}}%
\pgfpathlineto{\pgfqpoint{3.236757in}{2.452684in}}%
\pgfpathlineto{\pgfqpoint{3.237348in}{2.414572in}}%
\pgfpathlineto{\pgfqpoint{3.237385in}{2.435504in}}%
\pgfpathlineto{\pgfqpoint{3.238369in}{2.333583in}}%
\pgfpathlineto{\pgfqpoint{3.238628in}{2.473197in}}%
\pgfpathlineto{\pgfqpoint{3.238456in}{2.296999in}}%
\pgfpathlineto{\pgfqpoint{3.239551in}{2.371525in}}%
\pgfpathlineto{\pgfqpoint{3.240314in}{2.311712in}}%
\pgfpathlineto{\pgfqpoint{3.240474in}{2.445530in}}%
\pgfpathlineto{\pgfqpoint{3.240659in}{2.368851in}}%
\pgfpathlineto{\pgfqpoint{3.241557in}{2.292526in}}%
\pgfpathlineto{\pgfqpoint{3.241114in}{2.447830in}}%
\pgfpathlineto{\pgfqpoint{3.241656in}{2.386714in}}%
\pgfpathlineto{\pgfqpoint{3.241729in}{2.446991in}}%
\pgfpathlineto{\pgfqpoint{3.242172in}{2.314014in}}%
\pgfpathlineto{\pgfqpoint{3.242726in}{2.332161in}}%
\pgfpathlineto{\pgfqpoint{3.242849in}{2.417112in}}%
\pgfpathlineto{\pgfqpoint{3.242788in}{2.303163in}}%
\pgfpathlineto{\pgfqpoint{3.242899in}{2.404035in}}%
\pgfpathlineto{\pgfqpoint{3.242936in}{2.447328in}}%
\pgfpathlineto{\pgfqpoint{3.243416in}{2.301185in}}%
\pgfpathlineto{\pgfqpoint{3.243969in}{2.342719in}}%
\pgfpathlineto{\pgfqpoint{3.244646in}{2.306656in}}%
\pgfpathlineto{\pgfqpoint{3.244794in}{2.442894in}}%
\pgfpathlineto{\pgfqpoint{3.244880in}{2.415811in}}%
\pgfpathlineto{\pgfqpoint{3.245409in}{2.445986in}}%
\pgfpathlineto{\pgfqpoint{3.245274in}{2.330516in}}%
\pgfpathlineto{\pgfqpoint{3.245754in}{2.331961in}}%
\pgfpathlineto{\pgfqpoint{3.246505in}{2.301020in}}%
\pgfpathlineto{\pgfqpoint{3.246037in}{2.473789in}}%
\pgfpathlineto{\pgfqpoint{3.246763in}{2.398564in}}%
\pgfpathlineto{\pgfqpoint{3.247736in}{2.293438in}}%
\pgfpathlineto{\pgfqpoint{3.247280in}{2.428702in}}%
\pgfpathlineto{\pgfqpoint{3.247846in}{2.394249in}}%
\pgfpathlineto{\pgfqpoint{3.247982in}{2.450212in}}%
\pgfpathlineto{\pgfqpoint{3.248339in}{2.316405in}}%
\pgfpathlineto{\pgfqpoint{3.248905in}{2.319251in}}%
\pgfpathlineto{\pgfqpoint{3.249151in}{2.466770in}}%
\pgfpathlineto{\pgfqpoint{3.248979in}{2.294698in}}%
\pgfpathlineto{\pgfqpoint{3.250074in}{2.349263in}}%
\pgfpathlineto{\pgfqpoint{3.250222in}{2.311182in}}%
\pgfpathlineto{\pgfqpoint{3.250345in}{2.454842in}}%
\pgfpathlineto{\pgfqpoint{3.251108in}{2.435887in}}%
\pgfpathlineto{\pgfqpoint{3.252068in}{2.299733in}}%
\pgfpathlineto{\pgfqpoint{3.251612in}{2.453719in}}%
\pgfpathlineto{\pgfqpoint{3.252215in}{2.420739in}}%
\pgfpathlineto{\pgfqpoint{3.252868in}{2.442187in}}%
\pgfpathlineto{\pgfqpoint{3.252609in}{2.330995in}}%
\pgfpathlineto{\pgfqpoint{3.253188in}{2.362686in}}%
\pgfpathlineto{\pgfqpoint{3.253311in}{2.269916in}}%
\pgfpathlineto{\pgfqpoint{3.253471in}{2.480489in}}%
\pgfpathlineto{\pgfqpoint{3.254246in}{2.397709in}}%
\pgfpathlineto{\pgfqpoint{3.254677in}{2.458164in}}%
\pgfpathlineto{\pgfqpoint{3.254505in}{2.293141in}}%
\pgfpathlineto{\pgfqpoint{3.255354in}{2.405397in}}%
\pgfpathlineto{\pgfqpoint{3.255686in}{2.274532in}}%
\pgfpathlineto{\pgfqpoint{3.255846in}{2.458922in}}%
\pgfpathlineto{\pgfqpoint{3.256499in}{2.347177in}}%
\pgfpathlineto{\pgfqpoint{3.256646in}{2.452509in}}%
\pgfpathlineto{\pgfqpoint{3.256880in}{2.297895in}}%
\pgfpathlineto{\pgfqpoint{3.257606in}{2.365664in}}%
\pgfpathlineto{\pgfqpoint{3.257655in}{2.258552in}}%
\pgfpathlineto{\pgfqpoint{3.257815in}{2.471648in}}%
\pgfpathlineto{\pgfqpoint{3.258702in}{2.372949in}}%
\pgfpathlineto{\pgfqpoint{3.259022in}{2.475094in}}%
\pgfpathlineto{\pgfqpoint{3.258862in}{2.264111in}}%
\pgfpathlineto{\pgfqpoint{3.259834in}{2.400371in}}%
\pgfpathlineto{\pgfqpoint{3.260745in}{2.277565in}}%
\pgfpathlineto{\pgfqpoint{3.260289in}{2.457328in}}%
\pgfpathlineto{\pgfqpoint{3.260942in}{2.374364in}}%
\pgfpathlineto{\pgfqpoint{3.260991in}{2.473503in}}%
\pgfpathlineto{\pgfqpoint{3.261237in}{2.287098in}}%
\pgfpathlineto{\pgfqpoint{3.262062in}{2.394949in}}%
\pgfpathlineto{\pgfqpoint{3.263206in}{2.246568in}}%
\pgfpathlineto{\pgfqpoint{3.262172in}{2.460617in}}%
\pgfpathlineto{\pgfqpoint{3.263231in}{2.306106in}}%
\pgfpathlineto{\pgfqpoint{3.263366in}{2.483644in}}%
\pgfpathlineto{\pgfqpoint{3.263612in}{2.295700in}}%
\pgfpathlineto{\pgfqpoint{3.264339in}{2.347070in}}%
\pgfpathlineto{\pgfqpoint{3.265225in}{2.454904in}}%
\pgfpathlineto{\pgfqpoint{3.264400in}{2.282895in}}%
\pgfpathlineto{\pgfqpoint{3.265459in}{2.362734in}}%
\pgfpathlineto{\pgfqpoint{3.266295in}{2.272038in}}%
\pgfpathlineto{\pgfqpoint{3.265742in}{2.456199in}}%
\pgfpathlineto{\pgfqpoint{3.266517in}{2.424632in}}%
\pgfpathlineto{\pgfqpoint{3.266542in}{2.446412in}}%
\pgfpathlineto{\pgfqpoint{3.267477in}{2.312815in}}%
\pgfpathlineto{\pgfqpoint{3.267526in}{2.343382in}}%
\pgfpathlineto{\pgfqpoint{3.267551in}{2.273654in}}%
\pgfpathlineto{\pgfqpoint{3.267723in}{2.486950in}}%
\pgfpathlineto{\pgfqpoint{3.268622in}{2.383693in}}%
\pgfpathlineto{\pgfqpoint{3.268757in}{2.281661in}}%
\pgfpathlineto{\pgfqpoint{3.268917in}{2.464095in}}%
\pgfpathlineto{\pgfqpoint{3.269742in}{2.370174in}}%
\pgfpathlineto{\pgfqpoint{3.270640in}{2.276405in}}%
\pgfpathlineto{\pgfqpoint{3.270086in}{2.445753in}}%
\pgfpathlineto{\pgfqpoint{3.270751in}{2.391230in}}%
\pgfpathlineto{\pgfqpoint{3.270886in}{2.447994in}}%
\pgfpathlineto{\pgfqpoint{3.271120in}{2.312411in}}%
\pgfpathlineto{\pgfqpoint{3.271846in}{2.343514in}}%
\pgfpathlineto{\pgfqpoint{3.272068in}{2.477188in}}%
\pgfpathlineto{\pgfqpoint{3.271895in}{2.287320in}}%
\pgfpathlineto{\pgfqpoint{3.272978in}{2.378245in}}%
\pgfpathlineto{\pgfqpoint{3.273102in}{2.291404in}}%
\pgfpathlineto{\pgfqpoint{3.273262in}{2.469912in}}%
\pgfpathlineto{\pgfqpoint{3.274098in}{2.350803in}}%
\pgfpathlineto{\pgfqpoint{3.274985in}{2.301002in}}%
\pgfpathlineto{\pgfqpoint{3.274431in}{2.442858in}}%
\pgfpathlineto{\pgfqpoint{3.275083in}{2.379288in}}%
\pgfpathlineto{\pgfqpoint{3.275231in}{2.454998in}}%
\pgfpathlineto{\pgfqpoint{3.275588in}{2.305160in}}%
\pgfpathlineto{\pgfqpoint{3.276178in}{2.357149in}}%
\pgfpathlineto{\pgfqpoint{3.276228in}{2.279416in}}%
\pgfpathlineto{\pgfqpoint{3.276400in}{2.468596in}}%
\pgfpathlineto{\pgfqpoint{3.277286in}{2.358008in}}%
\pgfpathlineto{\pgfqpoint{3.277594in}{2.467004in}}%
\pgfpathlineto{\pgfqpoint{3.278074in}{2.303951in}}%
\pgfpathlineto{\pgfqpoint{3.278406in}{2.390056in}}%
\pgfpathlineto{\pgfqpoint{3.278702in}{2.313323in}}%
\pgfpathlineto{\pgfqpoint{3.279477in}{2.452133in}}%
\pgfpathlineto{\pgfqpoint{3.279502in}{2.412785in}}%
\pgfpathlineto{\pgfqpoint{3.280560in}{2.290749in}}%
\pgfpathlineto{\pgfqpoint{3.280092in}{2.446867in}}%
\pgfpathlineto{\pgfqpoint{3.280609in}{2.404985in}}%
\pgfpathlineto{\pgfqpoint{3.280720in}{2.470180in}}%
\pgfpathlineto{\pgfqpoint{3.281175in}{2.302647in}}%
\pgfpathlineto{\pgfqpoint{3.281545in}{2.358009in}}%
\pgfpathlineto{\pgfqpoint{3.282406in}{2.289366in}}%
\pgfpathlineto{\pgfqpoint{3.281951in}{2.439971in}}%
\pgfpathlineto{\pgfqpoint{3.282542in}{2.418782in}}%
\pgfpathlineto{\pgfqpoint{3.282578in}{2.444248in}}%
\pgfpathlineto{\pgfqpoint{3.282800in}{2.328711in}}%
\pgfpathlineto{\pgfqpoint{3.283600in}{2.347407in}}%
\pgfpathlineto{\pgfqpoint{3.283649in}{2.288136in}}%
\pgfpathlineto{\pgfqpoint{3.283822in}{2.462979in}}%
\pgfpathlineto{\pgfqpoint{3.284695in}{2.346841in}}%
\pgfpathlineto{\pgfqpoint{3.285668in}{2.451889in}}%
\pgfpathlineto{\pgfqpoint{3.285520in}{2.314516in}}%
\pgfpathlineto{\pgfqpoint{3.285815in}{2.376191in}}%
\pgfpathlineto{\pgfqpoint{3.286751in}{2.279300in}}%
\pgfpathlineto{\pgfqpoint{3.286283in}{2.451996in}}%
\pgfpathlineto{\pgfqpoint{3.286849in}{2.392816in}}%
\pgfpathlineto{\pgfqpoint{3.286898in}{2.445402in}}%
\pgfpathlineto{\pgfqpoint{3.287157in}{2.328976in}}%
\pgfpathlineto{\pgfqpoint{3.287920in}{2.345856in}}%
\pgfpathlineto{\pgfqpoint{3.288597in}{2.283824in}}%
\pgfpathlineto{\pgfqpoint{3.288129in}{2.485676in}}%
\pgfpathlineto{\pgfqpoint{3.289003in}{2.358610in}}%
\pgfpathlineto{\pgfqpoint{3.290000in}{2.453123in}}%
\pgfpathlineto{\pgfqpoint{3.289840in}{2.311276in}}%
\pgfpathlineto{\pgfqpoint{3.290123in}{2.399285in}}%
\pgfpathlineto{\pgfqpoint{3.291083in}{2.295066in}}%
\pgfpathlineto{\pgfqpoint{3.290615in}{2.447051in}}%
\pgfpathlineto{\pgfqpoint{3.291218in}{2.427779in}}%
\pgfpathlineto{\pgfqpoint{3.291255in}{2.484883in}}%
\pgfpathlineto{\pgfqpoint{3.291711in}{2.300107in}}%
\pgfpathlineto{\pgfqpoint{3.292289in}{2.343305in}}%
\pgfpathlineto{\pgfqpoint{3.292941in}{2.285610in}}%
\pgfpathlineto{\pgfqpoint{3.292461in}{2.459848in}}%
\pgfpathlineto{\pgfqpoint{3.293372in}{2.375351in}}%
\pgfpathlineto{\pgfqpoint{3.293385in}{2.375422in}}%
\pgfpathlineto{\pgfqpoint{3.293397in}{2.370234in}}%
\pgfpathlineto{\pgfqpoint{3.293557in}{2.289399in}}%
\pgfpathlineto{\pgfqpoint{3.293606in}{2.459228in}}%
\pgfpathlineto{\pgfqpoint{3.294529in}{2.333435in}}%
\pgfpathlineto{\pgfqpoint{3.295563in}{2.498166in}}%
\pgfpathlineto{\pgfqpoint{3.295403in}{2.284474in}}%
\pgfpathlineto{\pgfqpoint{3.295674in}{2.404846in}}%
\pgfpathlineto{\pgfqpoint{3.296031in}{2.287649in}}%
\pgfpathlineto{\pgfqpoint{3.296720in}{2.435385in}}%
\pgfpathlineto{\pgfqpoint{3.296769in}{2.420132in}}%
\pgfpathlineto{\pgfqpoint{3.297532in}{2.465141in}}%
\pgfpathlineto{\pgfqpoint{3.297274in}{2.271169in}}%
\pgfpathlineto{\pgfqpoint{3.297803in}{2.353945in}}%
\pgfpathlineto{\pgfqpoint{3.298689in}{2.475272in}}%
\pgfpathlineto{\pgfqpoint{3.298517in}{2.289925in}}%
\pgfpathlineto{\pgfqpoint{3.298874in}{2.347727in}}%
\pgfpathlineto{\pgfqpoint{3.299735in}{2.276381in}}%
\pgfpathlineto{\pgfqpoint{3.299280in}{2.453553in}}%
\pgfpathlineto{\pgfqpoint{3.299846in}{2.394233in}}%
\pgfpathlineto{\pgfqpoint{3.299883in}{2.510812in}}%
\pgfpathlineto{\pgfqpoint{3.300880in}{2.278234in}}%
\pgfpathlineto{\pgfqpoint{3.300941in}{2.346808in}}%
\pgfpathlineto{\pgfqpoint{3.301594in}{2.273213in}}%
\pgfpathlineto{\pgfqpoint{3.301040in}{2.487562in}}%
\pgfpathlineto{\pgfqpoint{3.302061in}{2.304300in}}%
\pgfpathlineto{\pgfqpoint{3.302074in}{2.304420in}}%
\pgfpathlineto{\pgfqpoint{3.303009in}{2.543111in}}%
\pgfpathlineto{\pgfqpoint{3.302837in}{2.212512in}}%
\pgfpathlineto{\pgfqpoint{3.303194in}{2.354978in}}%
\pgfpathlineto{\pgfqpoint{3.304068in}{2.262474in}}%
\pgfpathlineto{\pgfqpoint{3.304191in}{2.501483in}}%
\pgfpathlineto{\pgfqpoint{3.304289in}{2.395248in}}%
\pgfpathlineto{\pgfqpoint{3.304683in}{2.260692in}}%
\pgfpathlineto{\pgfqpoint{3.304855in}{2.481721in}}%
\pgfpathlineto{\pgfqpoint{3.304941in}{2.457625in}}%
\pgfpathlineto{\pgfqpoint{3.305360in}{2.514766in}}%
\pgfpathlineto{\pgfqpoint{3.305200in}{2.256718in}}%
\pgfpathlineto{\pgfqpoint{3.306000in}{2.364578in}}%
\pgfpathlineto{\pgfqpoint{3.306086in}{2.511118in}}%
\pgfpathlineto{\pgfqpoint{3.306357in}{2.301068in}}%
\pgfpathlineto{\pgfqpoint{3.307034in}{2.334256in}}%
\pgfpathlineto{\pgfqpoint{3.307157in}{2.175025in}}%
\pgfpathlineto{\pgfqpoint{3.307317in}{2.565749in}}%
\pgfpathlineto{\pgfqpoint{3.308117in}{2.405207in}}%
\pgfpathlineto{\pgfqpoint{3.309003in}{2.216054in}}%
\pgfpathlineto{\pgfqpoint{3.308548in}{2.496420in}}%
\pgfpathlineto{\pgfqpoint{3.309163in}{2.490459in}}%
\pgfpathlineto{\pgfqpoint{3.309803in}{2.511285in}}%
\pgfpathlineto{\pgfqpoint{3.309631in}{2.262301in}}%
\pgfpathlineto{\pgfqpoint{3.310135in}{2.307662in}}%
\pgfpathlineto{\pgfqpoint{3.310258in}{2.200758in}}%
\pgfpathlineto{\pgfqpoint{3.310406in}{2.524953in}}%
\pgfpathlineto{\pgfqpoint{3.310418in}{2.560156in}}%
\pgfpathlineto{\pgfqpoint{3.310874in}{2.238867in}}%
\pgfpathlineto{\pgfqpoint{3.311440in}{2.311352in}}%
\pgfpathlineto{\pgfqpoint{3.311489in}{2.203140in}}%
\pgfpathlineto{\pgfqpoint{3.311649in}{2.544539in}}%
\pgfpathlineto{\pgfqpoint{3.312535in}{2.352576in}}%
\pgfpathlineto{\pgfqpoint{3.313508in}{2.527339in}}%
\pgfpathlineto{\pgfqpoint{3.313348in}{2.198185in}}%
\pgfpathlineto{\pgfqpoint{3.313643in}{2.368160in}}%
\pgfpathlineto{\pgfqpoint{3.314591in}{2.225962in}}%
\pgfpathlineto{\pgfqpoint{3.314123in}{2.483854in}}%
\pgfpathlineto{\pgfqpoint{3.314701in}{2.386391in}}%
\pgfpathlineto{\pgfqpoint{3.314751in}{2.545642in}}%
\pgfpathlineto{\pgfqpoint{3.315206in}{2.220239in}}%
\pgfpathlineto{\pgfqpoint{3.315784in}{2.304971in}}%
\pgfpathlineto{\pgfqpoint{3.315821in}{2.238028in}}%
\pgfpathlineto{\pgfqpoint{3.315981in}{2.483639in}}%
\pgfpathlineto{\pgfqpoint{3.316868in}{2.349469in}}%
\pgfpathlineto{\pgfqpoint{3.317852in}{2.487533in}}%
\pgfpathlineto{\pgfqpoint{3.317052in}{2.252936in}}%
\pgfpathlineto{\pgfqpoint{3.317988in}{2.382345in}}%
\pgfpathlineto{\pgfqpoint{3.318911in}{2.262118in}}%
\pgfpathlineto{\pgfqpoint{3.318468in}{2.476140in}}%
\pgfpathlineto{\pgfqpoint{3.319034in}{2.410335in}}%
\pgfpathlineto{\pgfqpoint{3.319071in}{2.474710in}}%
\pgfpathlineto{\pgfqpoint{3.319526in}{2.293508in}}%
\pgfpathlineto{\pgfqpoint{3.320104in}{2.341293in}}%
\pgfpathlineto{\pgfqpoint{3.320769in}{2.271958in}}%
\pgfpathlineto{\pgfqpoint{3.320314in}{2.475010in}}%
\pgfpathlineto{\pgfqpoint{3.321188in}{2.354500in}}%
\pgfpathlineto{\pgfqpoint{3.322184in}{2.462865in}}%
\pgfpathlineto{\pgfqpoint{3.321384in}{2.301759in}}%
\pgfpathlineto{\pgfqpoint{3.322308in}{2.383675in}}%
\pgfpathlineto{\pgfqpoint{3.323243in}{2.285729in}}%
\pgfpathlineto{\pgfqpoint{3.322788in}{2.453137in}}%
\pgfpathlineto{\pgfqpoint{3.323366in}{2.399144in}}%
\pgfpathlineto{\pgfqpoint{3.323428in}{2.457908in}}%
\pgfpathlineto{\pgfqpoint{3.323871in}{2.312226in}}%
\pgfpathlineto{\pgfqpoint{3.324437in}{2.346776in}}%
\pgfpathlineto{\pgfqpoint{3.324486in}{2.296087in}}%
\pgfpathlineto{\pgfqpoint{3.324646in}{2.446592in}}%
\pgfpathlineto{\pgfqpoint{3.325532in}{2.365241in}}%
\pgfpathlineto{\pgfqpoint{3.325729in}{2.324656in}}%
\pgfpathlineto{\pgfqpoint{3.325840in}{2.396938in}}%
\pgfpathlineto{\pgfqpoint{3.325889in}{2.450883in}}%
\pgfpathlineto{\pgfqpoint{3.326357in}{2.302722in}}%
\pgfpathlineto{\pgfqpoint{3.326923in}{2.366181in}}%
\pgfpathlineto{\pgfqpoint{3.327588in}{2.301325in}}%
\pgfpathlineto{\pgfqpoint{3.327735in}{2.449480in}}%
\pgfpathlineto{\pgfqpoint{3.327748in}{2.455491in}}%
\pgfpathlineto{\pgfqpoint{3.327969in}{2.317105in}}%
\pgfpathlineto{\pgfqpoint{3.328658in}{2.364569in}}%
\pgfpathlineto{\pgfqpoint{3.329458in}{2.321888in}}%
\pgfpathlineto{\pgfqpoint{3.329003in}{2.451794in}}%
\pgfpathlineto{\pgfqpoint{3.329729in}{2.403531in}}%
\pgfpathlineto{\pgfqpoint{3.330197in}{2.428231in}}%
\pgfpathlineto{\pgfqpoint{3.329926in}{2.333493in}}%
\pgfpathlineto{\pgfqpoint{3.330640in}{2.350254in}}%
\pgfpathlineto{\pgfqpoint{3.330689in}{2.310669in}}%
\pgfpathlineto{\pgfqpoint{3.330861in}{2.436375in}}%
\pgfpathlineto{\pgfqpoint{3.331698in}{2.384142in}}%
\pgfpathlineto{\pgfqpoint{3.332104in}{2.436642in}}%
\pgfpathlineto{\pgfqpoint{3.331920in}{2.316036in}}%
\pgfpathlineto{\pgfqpoint{3.332806in}{2.390660in}}%
\pgfpathlineto{\pgfqpoint{3.333311in}{2.434603in}}%
\pgfpathlineto{\pgfqpoint{3.333052in}{2.344791in}}%
\pgfpathlineto{\pgfqpoint{3.333495in}{2.348629in}}%
\pgfpathlineto{\pgfqpoint{3.334184in}{2.338867in}}%
\pgfpathlineto{\pgfqpoint{3.333963in}{2.414947in}}%
\pgfpathlineto{\pgfqpoint{3.334492in}{2.388961in}}%
\pgfpathlineto{\pgfqpoint{3.335255in}{2.418031in}}%
\pgfpathlineto{\pgfqpoint{3.335403in}{2.327150in}}%
\pgfpathlineto{\pgfqpoint{3.335600in}{2.390570in}}%
\pgfpathlineto{\pgfqpoint{3.336424in}{2.425658in}}%
\pgfpathlineto{\pgfqpoint{3.336154in}{2.344562in}}%
\pgfpathlineto{\pgfqpoint{3.336523in}{2.366586in}}%
\pgfpathlineto{\pgfqpoint{3.336634in}{2.336354in}}%
\pgfpathlineto{\pgfqpoint{3.337101in}{2.408391in}}%
\pgfpathlineto{\pgfqpoint{3.337594in}{2.393526in}}%
\pgfpathlineto{\pgfqpoint{3.338283in}{2.412289in}}%
\pgfpathlineto{\pgfqpoint{3.338504in}{2.335432in}}%
\pgfpathlineto{\pgfqpoint{3.338677in}{2.384970in}}%
\pgfpathlineto{\pgfqpoint{3.339747in}{2.343850in}}%
\pgfpathlineto{\pgfqpoint{3.339526in}{2.417877in}}%
\pgfpathlineto{\pgfqpoint{3.339821in}{2.375158in}}%
\pgfpathlineto{\pgfqpoint{3.340757in}{2.410251in}}%
\pgfpathlineto{\pgfqpoint{3.340474in}{2.353358in}}%
\pgfpathlineto{\pgfqpoint{3.340904in}{2.368104in}}%
\pgfpathlineto{\pgfqpoint{3.341606in}{2.342632in}}%
\pgfpathlineto{\pgfqpoint{3.341409in}{2.404899in}}%
\pgfpathlineto{\pgfqpoint{3.341914in}{2.379100in}}%
\pgfpathlineto{\pgfqpoint{3.342640in}{2.409347in}}%
\pgfpathlineto{\pgfqpoint{3.342824in}{2.346282in}}%
\pgfpathlineto{\pgfqpoint{3.343034in}{2.386870in}}%
\pgfpathlineto{\pgfqpoint{3.343969in}{2.346021in}}%
\pgfpathlineto{\pgfqpoint{3.343821in}{2.411459in}}%
\pgfpathlineto{\pgfqpoint{3.344191in}{2.381649in}}%
\pgfpathlineto{\pgfqpoint{3.345052in}{2.405238in}}%
\pgfpathlineto{\pgfqpoint{3.344744in}{2.343103in}}%
\pgfpathlineto{\pgfqpoint{3.345187in}{2.368927in}}%
\pgfpathlineto{\pgfqpoint{3.345926in}{2.334040in}}%
\pgfpathlineto{\pgfqpoint{3.345717in}{2.409299in}}%
\pgfpathlineto{\pgfqpoint{3.346246in}{2.393422in}}%
\pgfpathlineto{\pgfqpoint{3.346258in}{2.393988in}}%
\pgfpathlineto{\pgfqpoint{3.346431in}{2.366030in}}%
\pgfpathlineto{\pgfqpoint{3.346652in}{2.368060in}}%
\pgfpathlineto{\pgfqpoint{3.346701in}{2.343862in}}%
\pgfpathlineto{\pgfqpoint{3.346947in}{2.413938in}}%
\pgfpathlineto{\pgfqpoint{3.347747in}{2.376745in}}%
\pgfpathlineto{\pgfqpoint{3.348387in}{2.347294in}}%
\pgfpathlineto{\pgfqpoint{3.348178in}{2.414560in}}%
\pgfpathlineto{\pgfqpoint{3.348794in}{2.399951in}}%
\pgfpathlineto{\pgfqpoint{3.349360in}{2.414828in}}%
\pgfpathlineto{\pgfqpoint{3.349040in}{2.343003in}}%
\pgfpathlineto{\pgfqpoint{3.349778in}{2.358091in}}%
\pgfpathlineto{\pgfqpoint{3.349815in}{2.344843in}}%
\pgfpathlineto{\pgfqpoint{3.350012in}{2.414046in}}%
\pgfpathlineto{\pgfqpoint{3.350886in}{2.359647in}}%
\pgfpathlineto{\pgfqpoint{3.351255in}{2.414774in}}%
\pgfpathlineto{\pgfqpoint{3.351501in}{2.341534in}}%
\pgfpathlineto{\pgfqpoint{3.352043in}{2.370157in}}%
\pgfpathlineto{\pgfqpoint{3.352067in}{2.371372in}}%
\pgfpathlineto{\pgfqpoint{3.352264in}{2.340826in}}%
\pgfpathlineto{\pgfqpoint{3.352277in}{2.336407in}}%
\pgfpathlineto{\pgfqpoint{3.353138in}{2.423902in}}%
\pgfpathlineto{\pgfqpoint{3.353347in}{2.341694in}}%
\pgfpathlineto{\pgfqpoint{3.354320in}{2.422346in}}%
\pgfpathlineto{\pgfqpoint{3.354135in}{2.338170in}}%
\pgfpathlineto{\pgfqpoint{3.354492in}{2.357423in}}%
\pgfpathlineto{\pgfqpoint{3.355390in}{2.339090in}}%
\pgfpathlineto{\pgfqpoint{3.354960in}{2.411056in}}%
\pgfpathlineto{\pgfqpoint{3.355501in}{2.398667in}}%
\pgfpathlineto{\pgfqpoint{3.355612in}{2.421457in}}%
\pgfpathlineto{\pgfqpoint{3.355809in}{2.348813in}}%
\pgfpathlineto{\pgfqpoint{3.356510in}{2.356896in}}%
\pgfpathlineto{\pgfqpoint{3.357446in}{2.423296in}}%
\pgfpathlineto{\pgfqpoint{3.356609in}{2.338888in}}%
\pgfpathlineto{\pgfqpoint{3.357643in}{2.364268in}}%
\pgfpathlineto{\pgfqpoint{3.358480in}{2.346543in}}%
\pgfpathlineto{\pgfqpoint{3.358074in}{2.409045in}}%
\pgfpathlineto{\pgfqpoint{3.358590in}{2.387488in}}%
\pgfpathlineto{\pgfqpoint{3.358689in}{2.413505in}}%
\pgfpathlineto{\pgfqpoint{3.359095in}{2.350089in}}%
\pgfpathlineto{\pgfqpoint{3.359661in}{2.359402in}}%
\pgfpathlineto{\pgfqpoint{3.359710in}{2.336141in}}%
\pgfpathlineto{\pgfqpoint{3.360560in}{2.417192in}}%
\pgfpathlineto{\pgfqpoint{3.360757in}{2.362264in}}%
\pgfpathlineto{\pgfqpoint{3.361766in}{2.417590in}}%
\pgfpathlineto{\pgfqpoint{3.361581in}{2.345537in}}%
\pgfpathlineto{\pgfqpoint{3.361914in}{2.374102in}}%
\pgfpathlineto{\pgfqpoint{3.362824in}{2.336647in}}%
\pgfpathlineto{\pgfqpoint{3.362394in}{2.402161in}}%
\pgfpathlineto{\pgfqpoint{3.362972in}{2.399950in}}%
\pgfpathlineto{\pgfqpoint{3.363034in}{2.412686in}}%
\pgfpathlineto{\pgfqpoint{3.363218in}{2.347523in}}%
\pgfpathlineto{\pgfqpoint{3.364043in}{2.343105in}}%
\pgfpathlineto{\pgfqpoint{3.363686in}{2.397614in}}%
\pgfpathlineto{\pgfqpoint{3.364092in}{2.369992in}}%
\pgfpathlineto{\pgfqpoint{3.364252in}{2.412367in}}%
\pgfpathlineto{\pgfqpoint{3.364424in}{2.353899in}}%
\pgfpathlineto{\pgfqpoint{3.365163in}{2.356391in}}%
\pgfpathlineto{\pgfqpoint{3.365914in}{2.348605in}}%
\pgfpathlineto{\pgfqpoint{3.366086in}{2.406670in}}%
\pgfpathlineto{\pgfqpoint{3.366160in}{2.393243in}}%
\pgfpathlineto{\pgfqpoint{3.367144in}{2.348594in}}%
\pgfpathlineto{\pgfqpoint{3.366726in}{2.400822in}}%
\pgfpathlineto{\pgfqpoint{3.367280in}{2.390422in}}%
\pgfpathlineto{\pgfqpoint{3.367366in}{2.412681in}}%
\pgfpathlineto{\pgfqpoint{3.367563in}{2.354828in}}%
\pgfpathlineto{\pgfqpoint{3.368240in}{2.360685in}}%
\pgfpathlineto{\pgfqpoint{3.369015in}{2.352962in}}%
\pgfpathlineto{\pgfqpoint{3.369212in}{2.411403in}}%
\pgfpathlineto{\pgfqpoint{3.369298in}{2.380775in}}%
\pgfpathlineto{\pgfqpoint{3.370492in}{2.407363in}}%
\pgfpathlineto{\pgfqpoint{3.369434in}{2.350442in}}%
\pgfpathlineto{\pgfqpoint{3.370504in}{2.403478in}}%
\pgfpathlineto{\pgfqpoint{3.370664in}{2.350921in}}%
\pgfpathlineto{\pgfqpoint{3.371637in}{2.392147in}}%
\pgfpathlineto{\pgfqpoint{3.371698in}{2.415731in}}%
\pgfpathlineto{\pgfqpoint{3.371895in}{2.347624in}}%
\pgfpathlineto{\pgfqpoint{3.372695in}{2.364431in}}%
\pgfpathlineto{\pgfqpoint{3.373089in}{2.353196in}}%
\pgfpathlineto{\pgfqpoint{3.372929in}{2.405077in}}%
\pgfpathlineto{\pgfqpoint{3.373507in}{2.387766in}}%
\pgfpathlineto{\pgfqpoint{3.373544in}{2.398987in}}%
\pgfpathlineto{\pgfqpoint{3.374357in}{2.348896in}}%
\pgfpathlineto{\pgfqpoint{3.374553in}{2.368085in}}%
\pgfpathlineto{\pgfqpoint{3.374984in}{2.347348in}}%
\pgfpathlineto{\pgfqpoint{3.374800in}{2.411411in}}%
\pgfpathlineto{\pgfqpoint{3.375661in}{2.363163in}}%
\pgfpathlineto{\pgfqpoint{3.375993in}{2.401588in}}%
\pgfpathlineto{\pgfqpoint{3.376240in}{2.345924in}}%
\pgfpathlineto{\pgfqpoint{3.376781in}{2.373740in}}%
\pgfpathlineto{\pgfqpoint{3.377483in}{2.343894in}}%
\pgfpathlineto{\pgfqpoint{3.377249in}{2.401191in}}%
\pgfpathlineto{\pgfqpoint{3.377864in}{2.389555in}}%
\pgfpathlineto{\pgfqpoint{3.377926in}{2.403222in}}%
\pgfpathlineto{\pgfqpoint{3.378098in}{2.341112in}}%
\pgfpathlineto{\pgfqpoint{3.378849in}{2.372656in}}%
\pgfpathlineto{\pgfqpoint{3.379341in}{2.347573in}}%
\pgfpathlineto{\pgfqpoint{3.379107in}{2.405880in}}%
\pgfpathlineto{\pgfqpoint{3.380043in}{2.356457in}}%
\pgfpathlineto{\pgfqpoint{3.380363in}{2.402306in}}%
\pgfpathlineto{\pgfqpoint{3.380572in}{2.347760in}}%
\pgfpathlineto{\pgfqpoint{3.381150in}{2.363313in}}%
\pgfpathlineto{\pgfqpoint{3.381200in}{2.348289in}}%
\pgfpathlineto{\pgfqpoint{3.381446in}{2.395252in}}%
\pgfpathlineto{\pgfqpoint{3.382184in}{2.386616in}}%
\pgfpathlineto{\pgfqpoint{3.382233in}{2.403149in}}%
\pgfpathlineto{\pgfqpoint{3.382430in}{2.350550in}}%
\pgfpathlineto{\pgfqpoint{3.383267in}{2.377447in}}%
\pgfpathlineto{\pgfqpoint{3.383477in}{2.397041in}}%
\pgfpathlineto{\pgfqpoint{3.383698in}{2.352594in}}%
\pgfpathlineto{\pgfqpoint{3.384203in}{2.371520in}}%
\pgfpathlineto{\pgfqpoint{3.384892in}{2.349650in}}%
\pgfpathlineto{\pgfqpoint{3.384695in}{2.396526in}}%
\pgfpathlineto{\pgfqpoint{3.385273in}{2.384151in}}%
\pgfpathlineto{\pgfqpoint{3.385347in}{2.401804in}}%
\pgfpathlineto{\pgfqpoint{3.385520in}{2.352413in}}%
\pgfpathlineto{\pgfqpoint{3.386307in}{2.374006in}}%
\pgfpathlineto{\pgfqpoint{3.386750in}{2.349442in}}%
\pgfpathlineto{\pgfqpoint{3.386529in}{2.400615in}}%
\pgfpathlineto{\pgfqpoint{3.387489in}{2.365610in}}%
\pgfpathlineto{\pgfqpoint{3.388400in}{2.399509in}}%
\pgfpathlineto{\pgfqpoint{3.388006in}{2.357521in}}%
\pgfpathlineto{\pgfqpoint{3.388523in}{2.369905in}}%
\pgfpathlineto{\pgfqpoint{3.388646in}{2.345821in}}%
\pgfpathlineto{\pgfqpoint{3.388880in}{2.401700in}}%
\pgfpathlineto{\pgfqpoint{3.389606in}{2.377902in}}%
\pgfpathlineto{\pgfqpoint{3.389680in}{2.400903in}}%
\pgfpathlineto{\pgfqpoint{3.389926in}{2.348640in}}%
\pgfpathlineto{\pgfqpoint{3.390627in}{2.353171in}}%
\pgfpathlineto{\pgfqpoint{3.391046in}{2.321604in}}%
\pgfpathlineto{\pgfqpoint{3.391353in}{2.416617in}}%
\pgfpathlineto{\pgfqpoint{3.391723in}{2.353519in}}%
\pgfpathlineto{\pgfqpoint{3.392240in}{2.410901in}}%
\pgfpathlineto{\pgfqpoint{3.392067in}{2.337127in}}%
\pgfpathlineto{\pgfqpoint{3.392830in}{2.355167in}}%
\pgfpathlineto{\pgfqpoint{3.392855in}{2.312947in}}%
\pgfpathlineto{\pgfqpoint{3.393876in}{2.434991in}}%
\pgfpathlineto{\pgfqpoint{3.393938in}{2.359485in}}%
\pgfpathlineto{\pgfqpoint{3.394098in}{2.395874in}}%
\pgfpathlineto{\pgfqpoint{3.394566in}{2.430557in}}%
\pgfpathlineto{\pgfqpoint{3.394172in}{2.304048in}}%
\pgfpathlineto{\pgfqpoint{3.394787in}{2.319099in}}%
\pgfpathlineto{\pgfqpoint{3.394836in}{2.316629in}}%
\pgfpathlineto{\pgfqpoint{3.394873in}{2.353782in}}%
\pgfpathlineto{\pgfqpoint{3.395280in}{2.447053in}}%
\pgfpathlineto{\pgfqpoint{3.394996in}{2.291054in}}%
\pgfpathlineto{\pgfqpoint{3.395956in}{2.328010in}}%
\pgfpathlineto{\pgfqpoint{3.396670in}{2.272039in}}%
\pgfpathlineto{\pgfqpoint{3.396486in}{2.436280in}}%
\pgfpathlineto{\pgfqpoint{3.396990in}{2.405407in}}%
\pgfpathlineto{\pgfqpoint{3.397064in}{2.467443in}}%
\pgfpathlineto{\pgfqpoint{3.397175in}{2.303364in}}%
\pgfpathlineto{\pgfqpoint{3.397864in}{2.309746in}}%
\pgfpathlineto{\pgfqpoint{3.397876in}{2.291520in}}%
\pgfpathlineto{\pgfqpoint{3.398640in}{2.434759in}}%
\pgfpathlineto{\pgfqpoint{3.398947in}{2.343773in}}%
\pgfpathlineto{\pgfqpoint{3.399427in}{2.460682in}}%
\pgfpathlineto{\pgfqpoint{3.399760in}{2.205762in}}%
\pgfpathlineto{\pgfqpoint{3.400080in}{2.394506in}}%
\pgfpathlineto{\pgfqpoint{3.400966in}{2.291620in}}%
\pgfpathlineto{\pgfqpoint{3.401027in}{2.455515in}}%
\pgfpathlineto{\pgfqpoint{3.401187in}{2.368689in}}%
\pgfpathlineto{\pgfqpoint{3.402221in}{2.485459in}}%
\pgfpathlineto{\pgfqpoint{3.402160in}{2.243992in}}%
\pgfpathlineto{\pgfqpoint{3.402307in}{2.404784in}}%
\pgfpathlineto{\pgfqpoint{3.403009in}{2.471048in}}%
\pgfpathlineto{\pgfqpoint{3.402369in}{2.275516in}}%
\pgfpathlineto{\pgfqpoint{3.403292in}{2.375769in}}%
\pgfpathlineto{\pgfqpoint{3.403846in}{2.279495in}}%
\pgfpathlineto{\pgfqpoint{3.404203in}{2.464295in}}%
\pgfpathlineto{\pgfqpoint{3.404387in}{2.404669in}}%
\pgfpathlineto{\pgfqpoint{3.404535in}{2.282706in}}%
\pgfpathlineto{\pgfqpoint{3.404596in}{2.485090in}}%
\pgfpathlineto{\pgfqpoint{3.405360in}{2.408501in}}%
\pgfpathlineto{\pgfqpoint{3.405803in}{2.469034in}}%
\pgfpathlineto{\pgfqpoint{3.405507in}{2.293197in}}%
\pgfpathlineto{\pgfqpoint{3.406443in}{2.337707in}}%
\pgfpathlineto{\pgfqpoint{3.406504in}{2.484471in}}%
\pgfpathlineto{\pgfqpoint{3.407144in}{2.300679in}}%
\pgfpathlineto{\pgfqpoint{3.407587in}{2.385628in}}%
\pgfpathlineto{\pgfqpoint{3.408609in}{2.280008in}}%
\pgfpathlineto{\pgfqpoint{3.408166in}{2.437561in}}%
\pgfpathlineto{\pgfqpoint{3.408683in}{2.398146in}}%
\pgfpathlineto{\pgfqpoint{3.409323in}{2.313183in}}%
\pgfpathlineto{\pgfqpoint{3.408953in}{2.463338in}}%
\pgfpathlineto{\pgfqpoint{3.409618in}{2.418158in}}%
\pgfpathlineto{\pgfqpoint{3.410418in}{2.464312in}}%
\pgfpathlineto{\pgfqpoint{3.409778in}{2.298208in}}%
\pgfpathlineto{\pgfqpoint{3.410689in}{2.402032in}}%
\pgfpathlineto{\pgfqpoint{3.410984in}{2.230961in}}%
\pgfpathlineto{\pgfqpoint{3.411144in}{2.451867in}}%
\pgfpathlineto{\pgfqpoint{3.411809in}{2.362083in}}%
\pgfpathlineto{\pgfqpoint{3.412695in}{2.442058in}}%
\pgfpathlineto{\pgfqpoint{3.412153in}{2.299922in}}%
\pgfpathlineto{\pgfqpoint{3.412867in}{2.314721in}}%
\pgfpathlineto{\pgfqpoint{3.413347in}{2.251602in}}%
\pgfpathlineto{\pgfqpoint{3.413520in}{2.462299in}}%
\pgfpathlineto{\pgfqpoint{3.413901in}{2.412941in}}%
\pgfpathlineto{\pgfqpoint{3.414541in}{2.286923in}}%
\pgfpathlineto{\pgfqpoint{3.414332in}{2.448326in}}%
\pgfpathlineto{\pgfqpoint{3.415033in}{2.389812in}}%
\pgfpathlineto{\pgfqpoint{3.415883in}{2.462594in}}%
\pgfpathlineto{\pgfqpoint{3.415255in}{2.279970in}}%
\pgfpathlineto{\pgfqpoint{3.416129in}{2.392594in}}%
\pgfpathlineto{\pgfqpoint{3.416449in}{2.294302in}}%
\pgfpathlineto{\pgfqpoint{3.416596in}{2.436309in}}%
\pgfpathlineto{\pgfqpoint{3.417249in}{2.353963in}}%
\pgfpathlineto{\pgfqpoint{3.418221in}{2.484918in}}%
\pgfpathlineto{\pgfqpoint{3.417606in}{2.296850in}}%
\pgfpathlineto{\pgfqpoint{3.418344in}{2.344965in}}%
\pgfpathlineto{\pgfqpoint{3.419267in}{2.291100in}}%
\pgfpathlineto{\pgfqpoint{3.418923in}{2.445502in}}%
\pgfpathlineto{\pgfqpoint{3.419366in}{2.379095in}}%
\pgfpathlineto{\pgfqpoint{3.419427in}{2.456245in}}%
\pgfpathlineto{\pgfqpoint{3.419563in}{2.313407in}}%
\pgfpathlineto{\pgfqpoint{3.420399in}{2.366500in}}%
\pgfpathlineto{\pgfqpoint{3.420916in}{2.297992in}}%
\pgfpathlineto{\pgfqpoint{3.421286in}{2.493458in}}%
\pgfpathlineto{\pgfqpoint{3.421507in}{2.368565in}}%
\pgfpathlineto{\pgfqpoint{3.421778in}{2.477612in}}%
\pgfpathlineto{\pgfqpoint{3.421618in}{2.296799in}}%
\pgfpathlineto{\pgfqpoint{3.422627in}{2.406629in}}%
\pgfpathlineto{\pgfqpoint{3.423046in}{2.290860in}}%
\pgfpathlineto{\pgfqpoint{3.423624in}{2.470200in}}%
\pgfpathlineto{\pgfqpoint{3.423784in}{2.333749in}}%
\pgfpathlineto{\pgfqpoint{3.424338in}{2.471605in}}%
\pgfpathlineto{\pgfqpoint{3.423932in}{2.287641in}}%
\pgfpathlineto{\pgfqpoint{3.424879in}{2.346280in}}%
\pgfpathlineto{\pgfqpoint{3.424904in}{2.308810in}}%
\pgfpathlineto{\pgfqpoint{3.425187in}{2.483681in}}%
\pgfpathlineto{\pgfqpoint{3.425926in}{2.422597in}}%
\pgfpathlineto{\pgfqpoint{3.425963in}{2.495047in}}%
\pgfpathlineto{\pgfqpoint{3.426123in}{2.304883in}}%
\pgfpathlineto{\pgfqpoint{3.427009in}{2.368417in}}%
\pgfpathlineto{\pgfqpoint{3.427316in}{2.234548in}}%
\pgfpathlineto{\pgfqpoint{3.427538in}{2.486918in}}%
\pgfpathlineto{\pgfqpoint{3.428104in}{2.370903in}}%
\pgfpathlineto{\pgfqpoint{3.428719in}{2.474903in}}%
\pgfpathlineto{\pgfqpoint{3.428867in}{2.279378in}}%
\pgfpathlineto{\pgfqpoint{3.429199in}{2.342507in}}%
\pgfpathlineto{\pgfqpoint{3.429729in}{2.283451in}}%
\pgfpathlineto{\pgfqpoint{3.429864in}{2.514686in}}%
\pgfpathlineto{\pgfqpoint{3.430295in}{2.373570in}}%
\pgfpathlineto{\pgfqpoint{3.430639in}{2.469844in}}%
\pgfpathlineto{\pgfqpoint{3.430418in}{2.263469in}}%
\pgfpathlineto{\pgfqpoint{3.430861in}{2.315718in}}%
\pgfpathlineto{\pgfqpoint{3.430886in}{2.274148in}}%
\pgfpathlineto{\pgfqpoint{3.431390in}{2.473112in}}%
\pgfpathlineto{\pgfqpoint{3.431969in}{2.294250in}}%
\pgfpathlineto{\pgfqpoint{3.432203in}{2.484217in}}%
\pgfpathlineto{\pgfqpoint{3.432067in}{2.281615in}}%
\pgfpathlineto{\pgfqpoint{3.433089in}{2.322984in}}%
\pgfpathlineto{\pgfqpoint{3.433249in}{2.295052in}}%
\pgfpathlineto{\pgfqpoint{3.433753in}{2.496902in}}%
\pgfpathlineto{\pgfqpoint{3.434172in}{2.339258in}}%
\pgfpathlineto{\pgfqpoint{3.435255in}{2.451810in}}%
\pgfpathlineto{\pgfqpoint{3.434307in}{2.296334in}}%
\pgfpathlineto{\pgfqpoint{3.435304in}{2.409876in}}%
\pgfpathlineto{\pgfqpoint{3.435575in}{2.300451in}}%
\pgfpathlineto{\pgfqpoint{3.436079in}{2.436041in}}%
\pgfpathlineto{\pgfqpoint{3.436104in}{2.472293in}}%
\pgfpathlineto{\pgfqpoint{3.436227in}{2.301133in}}%
\pgfpathlineto{\pgfqpoint{3.437138in}{2.335976in}}%
\pgfpathlineto{\pgfqpoint{3.437593in}{2.467392in}}%
\pgfpathlineto{\pgfqpoint{3.437926in}{2.304103in}}%
\pgfpathlineto{\pgfqpoint{3.438516in}{2.414129in}}%
\pgfpathlineto{\pgfqpoint{3.439304in}{2.274295in}}%
\pgfpathlineto{\pgfqpoint{3.439206in}{2.475749in}}%
\pgfpathlineto{\pgfqpoint{3.439624in}{2.392442in}}%
\pgfpathlineto{\pgfqpoint{3.439673in}{2.474923in}}%
\pgfpathlineto{\pgfqpoint{3.440510in}{2.280423in}}%
\pgfpathlineto{\pgfqpoint{3.440732in}{2.393323in}}%
\pgfpathlineto{\pgfqpoint{3.441286in}{2.276457in}}%
\pgfpathlineto{\pgfqpoint{3.441556in}{2.491135in}}%
\pgfpathlineto{\pgfqpoint{3.441889in}{2.336206in}}%
\pgfpathlineto{\pgfqpoint{3.442824in}{2.259265in}}%
\pgfpathlineto{\pgfqpoint{3.442024in}{2.490367in}}%
\pgfpathlineto{\pgfqpoint{3.442910in}{2.357026in}}%
\pgfpathlineto{\pgfqpoint{3.442972in}{2.472656in}}%
\pgfpathlineto{\pgfqpoint{3.443599in}{2.283983in}}%
\pgfpathlineto{\pgfqpoint{3.443981in}{2.326526in}}%
\pgfpathlineto{\pgfqpoint{3.444202in}{2.297603in}}%
\pgfpathlineto{\pgfqpoint{3.444596in}{2.484945in}}%
\pgfpathlineto{\pgfqpoint{3.445064in}{2.363756in}}%
\pgfpathlineto{\pgfqpoint{3.446135in}{2.502507in}}%
\pgfpathlineto{\pgfqpoint{3.445150in}{2.282641in}}%
\pgfpathlineto{\pgfqpoint{3.446258in}{2.428052in}}%
\pgfpathlineto{\pgfqpoint{3.446430in}{2.291683in}}%
\pgfpathlineto{\pgfqpoint{3.446836in}{2.476537in}}%
\pgfpathlineto{\pgfqpoint{3.447390in}{2.368751in}}%
\pgfpathlineto{\pgfqpoint{3.447402in}{2.369921in}}%
\pgfpathlineto{\pgfqpoint{3.447452in}{2.282487in}}%
\pgfpathlineto{\pgfqpoint{3.447464in}{2.267808in}}%
\pgfpathlineto{\pgfqpoint{3.447612in}{2.451242in}}%
\pgfpathlineto{\pgfqpoint{3.448436in}{2.429116in}}%
\pgfpathlineto{\pgfqpoint{3.449236in}{2.483211in}}%
\pgfpathlineto{\pgfqpoint{3.448855in}{2.278051in}}%
\pgfpathlineto{\pgfqpoint{3.449507in}{2.378499in}}%
\pgfpathlineto{\pgfqpoint{3.449790in}{2.250644in}}%
\pgfpathlineto{\pgfqpoint{3.449950in}{2.464391in}}%
\pgfpathlineto{\pgfqpoint{3.450615in}{2.375608in}}%
\pgfpathlineto{\pgfqpoint{3.451255in}{2.293682in}}%
\pgfpathlineto{\pgfqpoint{3.451562in}{2.488459in}}%
\pgfpathlineto{\pgfqpoint{3.451710in}{2.388835in}}%
\pgfpathlineto{\pgfqpoint{3.451722in}{2.389519in}}%
\pgfpathlineto{\pgfqpoint{3.451747in}{2.361678in}}%
\pgfpathlineto{\pgfqpoint{3.452116in}{2.248212in}}%
\pgfpathlineto{\pgfqpoint{3.452276in}{2.494505in}}%
\pgfpathlineto{\pgfqpoint{3.452842in}{2.384205in}}%
\pgfpathlineto{\pgfqpoint{3.452892in}{2.322330in}}%
\pgfpathlineto{\pgfqpoint{3.453322in}{2.301414in}}%
\pgfpathlineto{\pgfqpoint{3.453101in}{2.456288in}}%
\pgfpathlineto{\pgfqpoint{3.453790in}{2.414932in}}%
\pgfpathlineto{\pgfqpoint{3.454639in}{2.522162in}}%
\pgfpathlineto{\pgfqpoint{3.454442in}{2.273792in}}%
\pgfpathlineto{\pgfqpoint{3.454775in}{2.353565in}}%
\pgfpathlineto{\pgfqpoint{3.454922in}{2.297974in}}%
\pgfpathlineto{\pgfqpoint{3.455439in}{2.468753in}}%
\pgfpathlineto{\pgfqpoint{3.455882in}{2.342574in}}%
\pgfpathlineto{\pgfqpoint{3.456953in}{2.500614in}}%
\pgfpathlineto{\pgfqpoint{3.456781in}{2.279487in}}%
\pgfpathlineto{\pgfqpoint{3.457089in}{2.407600in}}%
\pgfpathlineto{\pgfqpoint{3.457261in}{2.289519in}}%
\pgfpathlineto{\pgfqpoint{3.457753in}{2.448310in}}%
\pgfpathlineto{\pgfqpoint{3.458209in}{2.363316in}}%
\pgfpathlineto{\pgfqpoint{3.459292in}{2.484302in}}%
\pgfpathlineto{\pgfqpoint{3.459107in}{2.284250in}}%
\pgfpathlineto{\pgfqpoint{3.459415in}{2.435021in}}%
\pgfpathlineto{\pgfqpoint{3.459562in}{2.285547in}}%
\pgfpathlineto{\pgfqpoint{3.460092in}{2.458137in}}%
\pgfpathlineto{\pgfqpoint{3.460547in}{2.361564in}}%
\pgfpathlineto{\pgfqpoint{3.461618in}{2.493015in}}%
\pgfpathlineto{\pgfqpoint{3.460756in}{2.300824in}}%
\pgfpathlineto{\pgfqpoint{3.461679in}{2.391075in}}%
\pgfpathlineto{\pgfqpoint{3.461925in}{2.307153in}}%
\pgfpathlineto{\pgfqpoint{3.462405in}{2.472688in}}%
\pgfpathlineto{\pgfqpoint{3.462787in}{2.377011in}}%
\pgfpathlineto{\pgfqpoint{3.463944in}{2.502146in}}%
\pgfpathlineto{\pgfqpoint{3.463759in}{2.297331in}}%
\pgfpathlineto{\pgfqpoint{3.463981in}{2.435090in}}%
\pgfpathlineto{\pgfqpoint{3.464239in}{2.305696in}}%
\pgfpathlineto{\pgfqpoint{3.464744in}{2.457376in}}%
\pgfpathlineto{\pgfqpoint{3.465113in}{2.372269in}}%
\pgfpathlineto{\pgfqpoint{3.465519in}{2.451727in}}%
\pgfpathlineto{\pgfqpoint{3.466073in}{2.305638in}}%
\pgfpathlineto{\pgfqpoint{3.466085in}{2.293339in}}%
\pgfpathlineto{\pgfqpoint{3.466258in}{2.481269in}}%
\pgfpathlineto{\pgfqpoint{3.467033in}{2.440770in}}%
\pgfpathlineto{\pgfqpoint{3.467845in}{2.473017in}}%
\pgfpathlineto{\pgfqpoint{3.467279in}{2.312272in}}%
\pgfpathlineto{\pgfqpoint{3.467993in}{2.345059in}}%
\pgfpathlineto{\pgfqpoint{3.468424in}{2.295447in}}%
\pgfpathlineto{\pgfqpoint{3.468584in}{2.476197in}}%
\pgfpathlineto{\pgfqpoint{3.469113in}{2.327895in}}%
\pgfpathlineto{\pgfqpoint{3.469125in}{2.313924in}}%
\pgfpathlineto{\pgfqpoint{3.469384in}{2.455903in}}%
\pgfpathlineto{\pgfqpoint{3.470122in}{2.425566in}}%
\pgfpathlineto{\pgfqpoint{3.470159in}{2.475826in}}%
\pgfpathlineto{\pgfqpoint{3.470332in}{2.328018in}}%
\pgfpathlineto{\pgfqpoint{3.471169in}{2.358361in}}%
\pgfpathlineto{\pgfqpoint{3.471205in}{2.310153in}}%
\pgfpathlineto{\pgfqpoint{3.471759in}{2.452674in}}%
\pgfpathlineto{\pgfqpoint{3.472239in}{2.388495in}}%
\pgfpathlineto{\pgfqpoint{3.472485in}{2.466826in}}%
\pgfpathlineto{\pgfqpoint{3.472842in}{2.315842in}}%
\pgfpathlineto{\pgfqpoint{3.473396in}{2.430784in}}%
\pgfpathlineto{\pgfqpoint{3.474393in}{2.295143in}}%
\pgfpathlineto{\pgfqpoint{3.474085in}{2.452901in}}%
\pgfpathlineto{\pgfqpoint{3.474565in}{2.392491in}}%
\pgfpathlineto{\pgfqpoint{3.474812in}{2.454233in}}%
\pgfpathlineto{\pgfqpoint{3.475169in}{2.292173in}}%
\pgfpathlineto{\pgfqpoint{3.475636in}{2.371962in}}%
\pgfpathlineto{\pgfqpoint{3.476104in}{2.320326in}}%
\pgfpathlineto{\pgfqpoint{3.475735in}{2.448470in}}%
\pgfpathlineto{\pgfqpoint{3.476387in}{2.422801in}}%
\pgfpathlineto{\pgfqpoint{3.476424in}{2.449052in}}%
\pgfpathlineto{\pgfqpoint{3.476719in}{2.315444in}}%
\pgfpathlineto{\pgfqpoint{3.477408in}{2.350725in}}%
\pgfpathlineto{\pgfqpoint{3.477495in}{2.306694in}}%
\pgfpathlineto{\pgfqpoint{3.477827in}{2.433305in}}%
\pgfpathlineto{\pgfqpoint{3.478504in}{2.357983in}}%
\pgfpathlineto{\pgfqpoint{3.478528in}{2.363951in}}%
\pgfpathlineto{\pgfqpoint{3.478541in}{2.354702in}}%
\pgfpathlineto{\pgfqpoint{3.479033in}{2.325532in}}%
\pgfpathlineto{\pgfqpoint{3.478750in}{2.442038in}}%
\pgfpathlineto{\pgfqpoint{3.479612in}{2.402379in}}%
\pgfpathlineto{\pgfqpoint{3.479968in}{2.297842in}}%
\pgfpathlineto{\pgfqpoint{3.480252in}{2.447050in}}%
\pgfpathlineto{\pgfqpoint{3.480781in}{2.346146in}}%
\pgfpathlineto{\pgfqpoint{3.481076in}{2.448371in}}%
\pgfpathlineto{\pgfqpoint{3.481359in}{2.324558in}}%
\pgfpathlineto{\pgfqpoint{3.482012in}{2.422240in}}%
\pgfpathlineto{\pgfqpoint{3.482295in}{2.296697in}}%
\pgfpathlineto{\pgfqpoint{3.482578in}{2.437656in}}%
\pgfpathlineto{\pgfqpoint{3.483255in}{2.374089in}}%
\pgfpathlineto{\pgfqpoint{3.483685in}{2.328693in}}%
\pgfpathlineto{\pgfqpoint{3.483402in}{2.451788in}}%
\pgfpathlineto{\pgfqpoint{3.484079in}{2.380586in}}%
\pgfpathlineto{\pgfqpoint{3.484338in}{2.437589in}}%
\pgfpathlineto{\pgfqpoint{3.484608in}{2.317252in}}%
\pgfpathlineto{\pgfqpoint{3.485175in}{2.354349in}}%
\pgfpathlineto{\pgfqpoint{3.485532in}{2.328605in}}%
\pgfpathlineto{\pgfqpoint{3.485728in}{2.436411in}}%
\pgfpathlineto{\pgfqpoint{3.486258in}{2.357102in}}%
\pgfpathlineto{\pgfqpoint{3.486652in}{2.432602in}}%
\pgfpathlineto{\pgfqpoint{3.486935in}{2.316878in}}%
\pgfpathlineto{\pgfqpoint{3.487365in}{2.381615in}}%
\pgfpathlineto{\pgfqpoint{3.487845in}{2.330831in}}%
\pgfpathlineto{\pgfqpoint{3.488067in}{2.423465in}}%
\pgfpathlineto{\pgfqpoint{3.488485in}{2.362559in}}%
\pgfpathlineto{\pgfqpoint{3.488633in}{2.348540in}}%
\pgfpathlineto{\pgfqpoint{3.488719in}{2.380213in}}%
\pgfpathlineto{\pgfqpoint{3.488965in}{2.447963in}}%
\pgfpathlineto{\pgfqpoint{3.489372in}{2.323813in}}%
\pgfpathlineto{\pgfqpoint{3.489802in}{2.365620in}}%
\pgfpathlineto{\pgfqpoint{3.490172in}{2.334715in}}%
\pgfpathlineto{\pgfqpoint{3.490516in}{2.406028in}}%
\pgfpathlineto{\pgfqpoint{3.490910in}{2.364801in}}%
\pgfpathlineto{\pgfqpoint{3.491685in}{2.324914in}}%
\pgfpathlineto{\pgfqpoint{3.491292in}{2.442483in}}%
\pgfpathlineto{\pgfqpoint{3.491956in}{2.396195in}}%
\pgfpathlineto{\pgfqpoint{3.492682in}{2.414668in}}%
\pgfpathlineto{\pgfqpoint{3.492485in}{2.325797in}}%
\pgfpathlineto{\pgfqpoint{3.493052in}{2.378195in}}%
\pgfpathlineto{\pgfqpoint{3.493261in}{2.335189in}}%
\pgfpathlineto{\pgfqpoint{3.493618in}{2.444404in}}%
\pgfpathlineto{\pgfqpoint{3.494135in}{2.389386in}}%
\pgfpathlineto{\pgfqpoint{3.494996in}{2.418258in}}%
\pgfpathlineto{\pgfqpoint{3.494812in}{2.329612in}}%
\pgfpathlineto{\pgfqpoint{3.495230in}{2.389859in}}%
\pgfpathlineto{\pgfqpoint{3.495587in}{2.324152in}}%
\pgfpathlineto{\pgfqpoint{3.495932in}{2.425936in}}%
\pgfpathlineto{\pgfqpoint{3.496350in}{2.347841in}}%
\pgfpathlineto{\pgfqpoint{3.497322in}{2.425487in}}%
\pgfpathlineto{\pgfqpoint{3.497138in}{2.315468in}}%
\pgfpathlineto{\pgfqpoint{3.497482in}{2.384106in}}%
\pgfpathlineto{\pgfqpoint{3.497913in}{2.323000in}}%
\pgfpathlineto{\pgfqpoint{3.498270in}{2.422388in}}%
\pgfpathlineto{\pgfqpoint{3.498590in}{2.378514in}}%
\pgfpathlineto{\pgfqpoint{3.499070in}{2.411927in}}%
\pgfpathlineto{\pgfqpoint{3.499451in}{2.333025in}}%
\pgfpathlineto{\pgfqpoint{3.499735in}{2.398271in}}%
\pgfpathlineto{\pgfqpoint{3.500239in}{2.325989in}}%
\pgfpathlineto{\pgfqpoint{3.500584in}{2.414239in}}%
\pgfpathlineto{\pgfqpoint{3.500855in}{2.357523in}}%
\pgfpathlineto{\pgfqpoint{3.501950in}{2.417727in}}%
\pgfpathlineto{\pgfqpoint{3.501778in}{2.343700in}}%
\pgfpathlineto{\pgfqpoint{3.501987in}{2.393707in}}%
\pgfpathlineto{\pgfqpoint{3.502602in}{2.325076in}}%
\pgfpathlineto{\pgfqpoint{3.502910in}{2.411354in}}%
\pgfpathlineto{\pgfqpoint{3.503107in}{2.374239in}}%
\pgfpathlineto{\pgfqpoint{3.503993in}{2.342517in}}%
\pgfpathlineto{\pgfqpoint{3.503710in}{2.420422in}}%
\pgfpathlineto{\pgfqpoint{3.504227in}{2.371168in}}%
\pgfpathlineto{\pgfqpoint{3.504276in}{2.423495in}}%
\pgfpathlineto{\pgfqpoint{3.504916in}{2.327462in}}%
\pgfpathlineto{\pgfqpoint{3.505322in}{2.362187in}}%
\pgfpathlineto{\pgfqpoint{3.506307in}{2.334114in}}%
\pgfpathlineto{\pgfqpoint{3.506024in}{2.419041in}}%
\pgfpathlineto{\pgfqpoint{3.506455in}{2.352680in}}%
\pgfpathlineto{\pgfqpoint{3.506590in}{2.418638in}}%
\pgfpathlineto{\pgfqpoint{3.507230in}{2.335821in}}%
\pgfpathlineto{\pgfqpoint{3.507624in}{2.388331in}}%
\pgfpathlineto{\pgfqpoint{3.507784in}{2.338150in}}%
\pgfpathlineto{\pgfqpoint{3.508338in}{2.407005in}}%
\pgfpathlineto{\pgfqpoint{3.508756in}{2.360089in}}%
\pgfpathlineto{\pgfqpoint{3.509630in}{2.332879in}}%
\pgfpathlineto{\pgfqpoint{3.508904in}{2.413574in}}%
\pgfpathlineto{\pgfqpoint{3.509790in}{2.376357in}}%
\pgfpathlineto{\pgfqpoint{3.510282in}{2.411230in}}%
\pgfpathlineto{\pgfqpoint{3.510098in}{2.331028in}}%
\pgfpathlineto{\pgfqpoint{3.510898in}{2.373649in}}%
\pgfpathlineto{\pgfqpoint{3.510910in}{2.373626in}}%
\pgfpathlineto{\pgfqpoint{3.511944in}{2.328124in}}%
\pgfpathlineto{\pgfqpoint{3.511439in}{2.415492in}}%
\pgfpathlineto{\pgfqpoint{3.512005in}{2.379999in}}%
\pgfpathlineto{\pgfqpoint{3.513051in}{2.413545in}}%
\pgfpathlineto{\pgfqpoint{3.512731in}{2.342109in}}%
\pgfpathlineto{\pgfqpoint{3.513113in}{2.377236in}}%
\pgfpathlineto{\pgfqpoint{3.513753in}{2.414820in}}%
\pgfpathlineto{\pgfqpoint{3.513335in}{2.328266in}}%
\pgfpathlineto{\pgfqpoint{3.513925in}{2.350934in}}%
\pgfpathlineto{\pgfqpoint{3.514258in}{2.327183in}}%
\pgfpathlineto{\pgfqpoint{3.514356in}{2.410186in}}%
\pgfpathlineto{\pgfqpoint{3.514885in}{2.375847in}}%
\pgfpathlineto{\pgfqpoint{3.515378in}{2.421119in}}%
\pgfpathlineto{\pgfqpoint{3.515784in}{2.324645in}}%
\pgfpathlineto{\pgfqpoint{3.515993in}{2.381566in}}%
\pgfpathlineto{\pgfqpoint{3.516571in}{2.324355in}}%
\pgfpathlineto{\pgfqpoint{3.516768in}{2.411177in}}%
\pgfpathlineto{\pgfqpoint{3.517113in}{2.372854in}}%
\pgfpathlineto{\pgfqpoint{3.517691in}{2.417761in}}%
\pgfpathlineto{\pgfqpoint{3.518098in}{2.317964in}}%
\pgfpathlineto{\pgfqpoint{3.518221in}{2.375396in}}%
\pgfpathlineto{\pgfqpoint{3.519316in}{2.411798in}}%
\pgfpathlineto{\pgfqpoint{3.518898in}{2.323248in}}%
\pgfpathlineto{\pgfqpoint{3.519328in}{2.391445in}}%
\pgfpathlineto{\pgfqpoint{3.520424in}{2.324939in}}%
\pgfpathlineto{\pgfqpoint{3.520018in}{2.415762in}}%
\pgfpathlineto{\pgfqpoint{3.520448in}{2.359907in}}%
\pgfpathlineto{\pgfqpoint{3.521396in}{2.408685in}}%
\pgfpathlineto{\pgfqpoint{3.521211in}{2.324694in}}%
\pgfpathlineto{\pgfqpoint{3.521568in}{2.383388in}}%
\pgfpathlineto{\pgfqpoint{3.521814in}{2.343778in}}%
\pgfpathlineto{\pgfqpoint{3.522331in}{2.415998in}}%
\pgfpathlineto{\pgfqpoint{3.522676in}{2.380015in}}%
\pgfpathlineto{\pgfqpoint{3.523710in}{2.411681in}}%
\pgfpathlineto{\pgfqpoint{3.523538in}{2.329233in}}%
\pgfpathlineto{\pgfqpoint{3.523771in}{2.380294in}}%
\pgfpathlineto{\pgfqpoint{3.524288in}{2.346211in}}%
\pgfpathlineto{\pgfqpoint{3.524645in}{2.423149in}}%
\pgfpathlineto{\pgfqpoint{3.524867in}{2.386154in}}%
\pgfpathlineto{\pgfqpoint{3.525876in}{2.329361in}}%
\pgfpathlineto{\pgfqpoint{3.525568in}{2.418889in}}%
\pgfpathlineto{\pgfqpoint{3.525999in}{2.375809in}}%
\pgfpathlineto{\pgfqpoint{3.526959in}{2.421899in}}%
\pgfpathlineto{\pgfqpoint{3.526676in}{2.342680in}}%
\pgfpathlineto{\pgfqpoint{3.527119in}{2.388735in}}%
\pgfpathlineto{\pgfqpoint{3.528202in}{2.331157in}}%
\pgfpathlineto{\pgfqpoint{3.527894in}{2.414759in}}%
\pgfpathlineto{\pgfqpoint{3.528239in}{2.367536in}}%
\pgfpathlineto{\pgfqpoint{3.529285in}{2.420909in}}%
\pgfpathlineto{\pgfqpoint{3.529076in}{2.337950in}}%
\pgfpathlineto{\pgfqpoint{3.529371in}{2.381530in}}%
\pgfpathlineto{\pgfqpoint{3.530516in}{2.339637in}}%
\pgfpathlineto{\pgfqpoint{3.530221in}{2.420204in}}%
\pgfpathlineto{\pgfqpoint{3.530528in}{2.341782in}}%
\pgfpathlineto{\pgfqpoint{3.531611in}{2.416130in}}%
\pgfpathlineto{\pgfqpoint{3.531402in}{2.341261in}}%
\pgfpathlineto{\pgfqpoint{3.531673in}{2.393280in}}%
\pgfpathlineto{\pgfqpoint{3.531858in}{2.342933in}}%
\pgfpathlineto{\pgfqpoint{3.532534in}{2.416774in}}%
\pgfpathlineto{\pgfqpoint{3.532842in}{2.346984in}}%
\pgfpathlineto{\pgfqpoint{3.533925in}{2.414830in}}%
\pgfpathlineto{\pgfqpoint{3.533753in}{2.340203in}}%
\pgfpathlineto{\pgfqpoint{3.533999in}{2.382403in}}%
\pgfpathlineto{\pgfqpoint{3.535082in}{2.348500in}}%
\pgfpathlineto{\pgfqpoint{3.534848in}{2.414291in}}%
\pgfpathlineto{\pgfqpoint{3.535156in}{2.352022in}}%
\pgfpathlineto{\pgfqpoint{3.536239in}{2.410363in}}%
\pgfpathlineto{\pgfqpoint{3.536079in}{2.346379in}}%
\pgfpathlineto{\pgfqpoint{3.536313in}{2.384422in}}%
\pgfpathlineto{\pgfqpoint{3.537458in}{2.342871in}}%
\pgfpathlineto{\pgfqpoint{3.537162in}{2.416835in}}%
\pgfpathlineto{\pgfqpoint{3.537470in}{2.345246in}}%
\pgfpathlineto{\pgfqpoint{3.538553in}{2.404625in}}%
\pgfpathlineto{\pgfqpoint{3.538614in}{2.387205in}}%
\pgfpathlineto{\pgfqpoint{3.539501in}{2.401889in}}%
\pgfpathlineto{\pgfqpoint{3.539771in}{2.349670in}}%
\pgfpathlineto{\pgfqpoint{3.540842in}{2.402961in}}%
\pgfpathlineto{\pgfqpoint{3.539857in}{2.341638in}}%
\pgfpathlineto{\pgfqpoint{3.541076in}{2.387862in}}%
\pgfpathlineto{\pgfqpoint{3.542171in}{2.339200in}}%
\pgfpathlineto{\pgfqpoint{3.541765in}{2.400548in}}%
\pgfpathlineto{\pgfqpoint{3.542196in}{2.355738in}}%
\pgfpathlineto{\pgfqpoint{3.543156in}{2.407244in}}%
\pgfpathlineto{\pgfqpoint{3.542959in}{2.346467in}}%
\pgfpathlineto{\pgfqpoint{3.543316in}{2.382532in}}%
\pgfpathlineto{\pgfqpoint{3.543747in}{2.349200in}}%
\pgfpathlineto{\pgfqpoint{3.544079in}{2.400809in}}%
\pgfpathlineto{\pgfqpoint{3.544436in}{2.366569in}}%
\pgfpathlineto{\pgfqpoint{3.545482in}{2.403297in}}%
\pgfpathlineto{\pgfqpoint{3.544497in}{2.342636in}}%
\pgfpathlineto{\pgfqpoint{3.545581in}{2.386889in}}%
\pgfpathlineto{\pgfqpoint{3.546061in}{2.348676in}}%
\pgfpathlineto{\pgfqpoint{3.546405in}{2.407740in}}%
\pgfpathlineto{\pgfqpoint{3.546725in}{2.362612in}}%
\pgfpathlineto{\pgfqpoint{3.547599in}{2.349566in}}%
\pgfpathlineto{\pgfqpoint{3.547328in}{2.402798in}}%
\pgfpathlineto{\pgfqpoint{3.547759in}{2.369739in}}%
\pgfpathlineto{\pgfqpoint{3.548731in}{2.413338in}}%
\pgfpathlineto{\pgfqpoint{3.548387in}{2.351410in}}%
\pgfpathlineto{\pgfqpoint{3.548879in}{2.380479in}}%
\pgfpathlineto{\pgfqpoint{3.549839in}{2.349838in}}%
\pgfpathlineto{\pgfqpoint{3.549654in}{2.406565in}}%
\pgfpathlineto{\pgfqpoint{3.549987in}{2.365731in}}%
\pgfpathlineto{\pgfqpoint{3.551045in}{2.410127in}}%
\pgfpathlineto{\pgfqpoint{3.550762in}{2.349074in}}%
\pgfpathlineto{\pgfqpoint{3.551131in}{2.385910in}}%
\pgfpathlineto{\pgfqpoint{3.551451in}{2.351434in}}%
\pgfpathlineto{\pgfqpoint{3.551944in}{2.393889in}}%
\pgfpathlineto{\pgfqpoint{3.551981in}{2.411448in}}%
\pgfpathlineto{\pgfqpoint{3.552153in}{2.343906in}}%
\pgfpathlineto{\pgfqpoint{3.553002in}{2.366248in}}%
\pgfpathlineto{\pgfqpoint{3.553076in}{2.346714in}}%
\pgfpathlineto{\pgfqpoint{3.553371in}{2.402257in}}%
\pgfpathlineto{\pgfqpoint{3.554097in}{2.366553in}}%
\pgfpathlineto{\pgfqpoint{3.554294in}{2.407215in}}%
\pgfpathlineto{\pgfqpoint{3.554479in}{2.347676in}}%
\pgfpathlineto{\pgfqpoint{3.555254in}{2.384136in}}%
\pgfpathlineto{\pgfqpoint{3.555402in}{2.341639in}}%
\pgfpathlineto{\pgfqpoint{3.555685in}{2.401911in}}%
\pgfpathlineto{\pgfqpoint{3.556350in}{2.379576in}}%
\pgfpathlineto{\pgfqpoint{3.556608in}{2.405796in}}%
\pgfpathlineto{\pgfqpoint{3.556879in}{2.350321in}}%
\pgfpathlineto{\pgfqpoint{3.557457in}{2.378031in}}%
\pgfpathlineto{\pgfqpoint{3.557716in}{2.345495in}}%
\pgfpathlineto{\pgfqpoint{3.557999in}{2.403785in}}%
\pgfpathlineto{\pgfqpoint{3.558590in}{2.361457in}}%
\pgfpathlineto{\pgfqpoint{3.558922in}{2.403721in}}%
\pgfpathlineto{\pgfqpoint{3.559205in}{2.351634in}}%
\pgfpathlineto{\pgfqpoint{3.559784in}{2.369983in}}%
\pgfpathlineto{\pgfqpoint{3.560042in}{2.346599in}}%
\pgfpathlineto{\pgfqpoint{3.559857in}{2.399667in}}%
\pgfpathlineto{\pgfqpoint{3.560879in}{2.374601in}}%
\pgfpathlineto{\pgfqpoint{3.561519in}{2.349868in}}%
\pgfpathlineto{\pgfqpoint{3.561248in}{2.403752in}}%
\pgfpathlineto{\pgfqpoint{3.561913in}{2.375064in}}%
\pgfpathlineto{\pgfqpoint{3.562860in}{2.401593in}}%
\pgfpathlineto{\pgfqpoint{3.562356in}{2.353508in}}%
\pgfpathlineto{\pgfqpoint{3.563020in}{2.374146in}}%
\pgfpathlineto{\pgfqpoint{3.563833in}{2.352618in}}%
\pgfpathlineto{\pgfqpoint{3.563562in}{2.403115in}}%
\pgfpathlineto{\pgfqpoint{3.564116in}{2.379866in}}%
\pgfpathlineto{\pgfqpoint{3.564485in}{2.398672in}}%
\pgfpathlineto{\pgfqpoint{3.564300in}{2.353103in}}%
\pgfpathlineto{\pgfqpoint{3.565125in}{2.367699in}}%
\pgfpathlineto{\pgfqpoint{3.565691in}{2.352332in}}%
\pgfpathlineto{\pgfqpoint{3.565876in}{2.404404in}}%
\pgfpathlineto{\pgfqpoint{3.566097in}{2.391070in}}%
\pgfpathlineto{\pgfqpoint{3.566799in}{2.403326in}}%
\pgfpathlineto{\pgfqpoint{3.566282in}{2.349241in}}%
\pgfpathlineto{\pgfqpoint{3.567057in}{2.362948in}}%
\pgfpathlineto{\pgfqpoint{3.567070in}{2.354541in}}%
\pgfpathlineto{\pgfqpoint{3.567956in}{2.400307in}}%
\pgfpathlineto{\pgfqpoint{3.568153in}{2.365718in}}%
\pgfpathlineto{\pgfqpoint{3.569113in}{2.401945in}}%
\pgfpathlineto{\pgfqpoint{3.568596in}{2.352152in}}%
\pgfpathlineto{\pgfqpoint{3.569260in}{2.375299in}}%
\pgfpathlineto{\pgfqpoint{3.569310in}{2.351644in}}%
\pgfpathlineto{\pgfqpoint{3.569950in}{2.396187in}}%
\pgfpathlineto{\pgfqpoint{3.570257in}{2.382598in}}%
\pgfpathlineto{\pgfqpoint{3.571193in}{2.392993in}}%
\pgfpathlineto{\pgfqpoint{3.570910in}{2.358695in}}%
\pgfpathlineto{\pgfqpoint{3.571353in}{2.378800in}}%
\pgfpathlineto{\pgfqpoint{3.571624in}{2.355576in}}%
\pgfpathlineto{\pgfqpoint{3.572251in}{2.393989in}}%
\pgfpathlineto{\pgfqpoint{3.572473in}{2.359239in}}%
\pgfpathlineto{\pgfqpoint{3.573039in}{2.394452in}}%
\pgfpathlineto{\pgfqpoint{3.572547in}{2.355684in}}%
\pgfpathlineto{\pgfqpoint{3.573593in}{2.373487in}}%
\pgfpathlineto{\pgfqpoint{3.573999in}{2.354257in}}%
\pgfpathlineto{\pgfqpoint{3.574196in}{2.392038in}}%
\pgfpathlineto{\pgfqpoint{3.574651in}{2.386531in}}%
\pgfpathlineto{\pgfqpoint{3.574664in}{2.390635in}}%
\pgfpathlineto{\pgfqpoint{3.574774in}{2.353141in}}%
\pgfpathlineto{\pgfqpoint{3.575685in}{2.360888in}}%
\pgfpathlineto{\pgfqpoint{3.576313in}{2.354052in}}%
\pgfpathlineto{\pgfqpoint{3.576510in}{2.391582in}}%
\pgfpathlineto{\pgfqpoint{3.576780in}{2.363202in}}%
\pgfpathlineto{\pgfqpoint{3.577654in}{2.388990in}}%
\pgfpathlineto{\pgfqpoint{3.577088in}{2.356298in}}%
\pgfpathlineto{\pgfqpoint{3.577913in}{2.378370in}}%
\pgfpathlineto{\pgfqpoint{3.578011in}{2.357205in}}%
\pgfpathlineto{\pgfqpoint{3.578356in}{2.393058in}}%
\pgfpathlineto{\pgfqpoint{3.579008in}{2.378709in}}%
\pgfpathlineto{\pgfqpoint{3.579624in}{2.393670in}}%
\pgfpathlineto{\pgfqpoint{3.579845in}{2.361074in}}%
\pgfpathlineto{\pgfqpoint{3.580103in}{2.378173in}}%
\pgfpathlineto{\pgfqpoint{3.580916in}{2.360362in}}%
\pgfpathlineto{\pgfqpoint{3.580411in}{2.389949in}}%
\pgfpathlineto{\pgfqpoint{3.581236in}{2.363148in}}%
\pgfpathlineto{\pgfqpoint{3.582245in}{2.389335in}}%
\pgfpathlineto{\pgfqpoint{3.582147in}{2.358407in}}%
\pgfpathlineto{\pgfqpoint{3.582393in}{2.385242in}}%
\pgfpathlineto{\pgfqpoint{3.583217in}{2.357460in}}%
\pgfpathlineto{\pgfqpoint{3.582713in}{2.393565in}}%
\pgfpathlineto{\pgfqpoint{3.583537in}{2.363258in}}%
\pgfpathlineto{\pgfqpoint{3.584559in}{2.394795in}}%
\pgfpathlineto{\pgfqpoint{3.583993in}{2.354185in}}%
\pgfpathlineto{\pgfqpoint{3.584657in}{2.379385in}}%
\pgfpathlineto{\pgfqpoint{3.585519in}{2.355752in}}%
\pgfpathlineto{\pgfqpoint{3.585482in}{2.398372in}}%
\pgfpathlineto{\pgfqpoint{3.585765in}{2.375893in}}%
\pgfpathlineto{\pgfqpoint{3.586860in}{2.394152in}}%
\pgfpathlineto{\pgfqpoint{3.586442in}{2.350897in}}%
\pgfpathlineto{\pgfqpoint{3.586873in}{2.390490in}}%
\pgfpathlineto{\pgfqpoint{3.587217in}{2.354483in}}%
\pgfpathlineto{\pgfqpoint{3.587783in}{2.401275in}}%
\pgfpathlineto{\pgfqpoint{3.587993in}{2.369619in}}%
\pgfpathlineto{\pgfqpoint{3.588251in}{2.390480in}}%
\pgfpathlineto{\pgfqpoint{3.588756in}{2.348586in}}%
\pgfpathlineto{\pgfqpoint{3.589100in}{2.376403in}}%
\pgfpathlineto{\pgfqpoint{3.589531in}{2.356757in}}%
\pgfpathlineto{\pgfqpoint{3.590097in}{2.395244in}}%
\pgfpathlineto{\pgfqpoint{3.590196in}{2.378141in}}%
\pgfpathlineto{\pgfqpoint{3.591267in}{2.391970in}}%
\pgfpathlineto{\pgfqpoint{3.591057in}{2.356427in}}%
\pgfpathlineto{\pgfqpoint{3.591291in}{2.378741in}}%
\pgfpathlineto{\pgfqpoint{3.591907in}{2.359228in}}%
\pgfpathlineto{\pgfqpoint{3.592190in}{2.394866in}}%
\pgfpathlineto{\pgfqpoint{3.592387in}{2.373154in}}%
\pgfpathlineto{\pgfqpoint{3.593113in}{2.394174in}}%
\pgfpathlineto{\pgfqpoint{3.592916in}{2.358096in}}%
\pgfpathlineto{\pgfqpoint{3.593494in}{2.375028in}}%
\pgfpathlineto{\pgfqpoint{3.594220in}{2.354457in}}%
\pgfpathlineto{\pgfqpoint{3.594503in}{2.390682in}}%
\pgfpathlineto{\pgfqpoint{3.594614in}{2.364234in}}%
\pgfpathlineto{\pgfqpoint{3.595427in}{2.393374in}}%
\pgfpathlineto{\pgfqpoint{3.594676in}{2.358178in}}%
\pgfpathlineto{\pgfqpoint{3.595734in}{2.370722in}}%
\pgfpathlineto{\pgfqpoint{3.596350in}{2.393759in}}%
\pgfpathlineto{\pgfqpoint{3.596534in}{2.353707in}}%
\pgfpathlineto{\pgfqpoint{3.596854in}{2.374644in}}%
\pgfpathlineto{\pgfqpoint{3.597457in}{2.360100in}}%
\pgfpathlineto{\pgfqpoint{3.597740in}{2.392841in}}%
\pgfpathlineto{\pgfqpoint{3.598011in}{2.361066in}}%
\pgfpathlineto{\pgfqpoint{3.598663in}{2.395272in}}%
\pgfpathlineto{\pgfqpoint{3.598934in}{2.354444in}}%
\pgfpathlineto{\pgfqpoint{3.599180in}{2.374376in}}%
\pgfpathlineto{\pgfqpoint{3.599574in}{2.392888in}}%
\pgfpathlineto{\pgfqpoint{3.599328in}{2.359054in}}%
\pgfpathlineto{\pgfqpoint{3.599759in}{2.365082in}}%
\pgfpathlineto{\pgfqpoint{3.599796in}{2.354580in}}%
\pgfpathlineto{\pgfqpoint{3.600054in}{2.397626in}}%
\pgfpathlineto{\pgfqpoint{3.600830in}{2.381219in}}%
\pgfpathlineto{\pgfqpoint{3.600965in}{2.394288in}}%
\pgfpathlineto{\pgfqpoint{3.601174in}{2.354335in}}%
\pgfpathlineto{\pgfqpoint{3.601937in}{2.386370in}}%
\pgfpathlineto{\pgfqpoint{3.602097in}{2.352418in}}%
\pgfpathlineto{\pgfqpoint{3.602331in}{2.394912in}}%
\pgfpathlineto{\pgfqpoint{3.603070in}{2.370437in}}%
\pgfpathlineto{\pgfqpoint{3.603488in}{2.351906in}}%
\pgfpathlineto{\pgfqpoint{3.603254in}{2.396745in}}%
\pgfpathlineto{\pgfqpoint{3.604066in}{2.376743in}}%
\pgfpathlineto{\pgfqpoint{3.604694in}{2.394212in}}%
\pgfpathlineto{\pgfqpoint{3.604423in}{2.351215in}}%
\pgfpathlineto{\pgfqpoint{3.605186in}{2.381609in}}%
\pgfpathlineto{\pgfqpoint{3.605826in}{2.345229in}}%
\pgfpathlineto{\pgfqpoint{3.605568in}{2.399798in}}%
\pgfpathlineto{\pgfqpoint{3.606306in}{2.368688in}}%
\pgfpathlineto{\pgfqpoint{3.606922in}{2.400981in}}%
\pgfpathlineto{\pgfqpoint{3.606737in}{2.348234in}}%
\pgfpathlineto{\pgfqpoint{3.607439in}{2.386458in}}%
\pgfpathlineto{\pgfqpoint{3.607845in}{2.408141in}}%
\pgfpathlineto{\pgfqpoint{3.608128in}{2.342481in}}%
\pgfpathlineto{\pgfqpoint{3.608485in}{2.375184in}}%
\pgfpathlineto{\pgfqpoint{3.608583in}{2.345531in}}%
\pgfpathlineto{\pgfqpoint{3.608768in}{2.403644in}}%
\pgfpathlineto{\pgfqpoint{3.609605in}{2.369647in}}%
\pgfpathlineto{\pgfqpoint{3.610233in}{2.400813in}}%
\pgfpathlineto{\pgfqpoint{3.610442in}{2.342776in}}%
\pgfpathlineto{\pgfqpoint{3.610725in}{2.382955in}}%
\pgfpathlineto{\pgfqpoint{3.611353in}{2.344963in}}%
\pgfpathlineto{\pgfqpoint{3.611082in}{2.401605in}}%
\pgfpathlineto{\pgfqpoint{3.611845in}{2.371720in}}%
\pgfpathlineto{\pgfqpoint{3.612460in}{2.396898in}}%
\pgfpathlineto{\pgfqpoint{3.612276in}{2.347454in}}%
\pgfpathlineto{\pgfqpoint{3.612953in}{2.382296in}}%
\pgfpathlineto{\pgfqpoint{3.613666in}{2.343771in}}%
\pgfpathlineto{\pgfqpoint{3.613383in}{2.403476in}}%
\pgfpathlineto{\pgfqpoint{3.614134in}{2.355326in}}%
\pgfpathlineto{\pgfqpoint{3.614762in}{2.403419in}}%
\pgfpathlineto{\pgfqpoint{3.614577in}{2.346928in}}%
\pgfpathlineto{\pgfqpoint{3.615266in}{2.380093in}}%
\pgfpathlineto{\pgfqpoint{3.615968in}{2.349177in}}%
\pgfpathlineto{\pgfqpoint{3.615685in}{2.405940in}}%
\pgfpathlineto{\pgfqpoint{3.616116in}{2.384098in}}%
\pgfpathlineto{\pgfqpoint{3.616596in}{2.402678in}}%
\pgfpathlineto{\pgfqpoint{3.616879in}{2.346658in}}%
\pgfpathlineto{\pgfqpoint{3.617211in}{2.380629in}}%
\pgfpathlineto{\pgfqpoint{3.618331in}{2.347525in}}%
\pgfpathlineto{\pgfqpoint{3.617974in}{2.399581in}}%
\pgfpathlineto{\pgfqpoint{3.618393in}{2.368367in}}%
\pgfpathlineto{\pgfqpoint{3.619353in}{2.399004in}}%
\pgfpathlineto{\pgfqpoint{3.619254in}{2.350300in}}%
\pgfpathlineto{\pgfqpoint{3.619537in}{2.384686in}}%
\pgfpathlineto{\pgfqpoint{3.620633in}{2.351163in}}%
\pgfpathlineto{\pgfqpoint{3.619820in}{2.392899in}}%
\pgfpathlineto{\pgfqpoint{3.620682in}{2.364143in}}%
\pgfpathlineto{\pgfqpoint{3.620903in}{2.393147in}}%
\pgfpathlineto{\pgfqpoint{3.621543in}{2.343021in}}%
\pgfpathlineto{\pgfqpoint{3.621851in}{2.373587in}}%
\pgfpathlineto{\pgfqpoint{3.622011in}{2.348845in}}%
\pgfpathlineto{\pgfqpoint{3.622737in}{2.392575in}}%
\pgfpathlineto{\pgfqpoint{3.622983in}{2.365386in}}%
\pgfpathlineto{\pgfqpoint{3.624116in}{2.392965in}}%
\pgfpathlineto{\pgfqpoint{3.623845in}{2.347307in}}%
\pgfpathlineto{\pgfqpoint{3.624165in}{2.378174in}}%
\pgfpathlineto{\pgfqpoint{3.625211in}{2.353029in}}%
\pgfpathlineto{\pgfqpoint{3.624928in}{2.392812in}}%
\pgfpathlineto{\pgfqpoint{3.625285in}{2.368476in}}%
\pgfpathlineto{\pgfqpoint{3.625629in}{2.396055in}}%
\pgfpathlineto{\pgfqpoint{3.626134in}{2.353872in}}%
\pgfpathlineto{\pgfqpoint{3.626417in}{2.390678in}}%
\pgfpathlineto{\pgfqpoint{3.627045in}{2.354811in}}%
\pgfpathlineto{\pgfqpoint{3.626540in}{2.397950in}}%
\pgfpathlineto{\pgfqpoint{3.627574in}{2.357186in}}%
\pgfpathlineto{\pgfqpoint{3.627931in}{2.398123in}}%
\pgfpathlineto{\pgfqpoint{3.628485in}{2.356308in}}%
\pgfpathlineto{\pgfqpoint{3.628731in}{2.379102in}}%
\pgfpathlineto{\pgfqpoint{3.629876in}{2.354576in}}%
\pgfpathlineto{\pgfqpoint{3.629703in}{2.399583in}}%
\pgfpathlineto{\pgfqpoint{3.629900in}{2.362274in}}%
\pgfpathlineto{\pgfqpoint{3.630233in}{2.397566in}}%
\pgfpathlineto{\pgfqpoint{3.630799in}{2.354593in}}%
\pgfpathlineto{\pgfqpoint{3.631033in}{2.379955in}}%
\pgfpathlineto{\pgfqpoint{3.632017in}{2.398905in}}%
\pgfpathlineto{\pgfqpoint{3.632189in}{2.356421in}}%
\pgfpathlineto{\pgfqpoint{3.632940in}{2.394882in}}%
\pgfpathlineto{\pgfqpoint{3.633346in}{2.380330in}}%
\pgfpathlineto{\pgfqpoint{3.634503in}{2.355985in}}%
\pgfpathlineto{\pgfqpoint{3.634331in}{2.400985in}}%
\pgfpathlineto{\pgfqpoint{3.634516in}{2.358763in}}%
\pgfpathlineto{\pgfqpoint{3.635254in}{2.398119in}}%
\pgfpathlineto{\pgfqpoint{3.635426in}{2.354096in}}%
\pgfpathlineto{\pgfqpoint{3.635673in}{2.378194in}}%
\pgfpathlineto{\pgfqpoint{3.636362in}{2.357021in}}%
\pgfpathlineto{\pgfqpoint{3.635722in}{2.399508in}}%
\pgfpathlineto{\pgfqpoint{3.636903in}{2.357147in}}%
\pgfpathlineto{\pgfqpoint{3.636965in}{2.379032in}}%
\pgfpathlineto{\pgfqpoint{3.637580in}{2.397411in}}%
\pgfpathlineto{\pgfqpoint{3.637753in}{2.356628in}}%
\pgfpathlineto{\pgfqpoint{3.638085in}{2.384106in}}%
\pgfpathlineto{\pgfqpoint{3.639168in}{2.355007in}}%
\pgfpathlineto{\pgfqpoint{3.638934in}{2.393428in}}%
\pgfpathlineto{\pgfqpoint{3.639279in}{2.370713in}}%
\pgfpathlineto{\pgfqpoint{3.639771in}{2.393113in}}%
\pgfpathlineto{\pgfqpoint{3.640103in}{2.347262in}}%
\pgfpathlineto{\pgfqpoint{3.640399in}{2.385901in}}%
\pgfpathlineto{\pgfqpoint{3.641026in}{2.349421in}}%
\pgfpathlineto{\pgfqpoint{3.641174in}{2.395324in}}%
\pgfpathlineto{\pgfqpoint{3.641543in}{2.367076in}}%
\pgfpathlineto{\pgfqpoint{3.642602in}{2.402967in}}%
\pgfpathlineto{\pgfqpoint{3.642429in}{2.344165in}}%
\pgfpathlineto{\pgfqpoint{3.642737in}{2.384124in}}%
\pgfpathlineto{\pgfqpoint{3.642885in}{2.349091in}}%
\pgfpathlineto{\pgfqpoint{3.643537in}{2.404827in}}%
\pgfpathlineto{\pgfqpoint{3.643882in}{2.369753in}}%
\pgfpathlineto{\pgfqpoint{3.644460in}{2.406631in}}%
\pgfpathlineto{\pgfqpoint{3.644768in}{2.351714in}}%
\pgfpathlineto{\pgfqpoint{3.645026in}{2.380189in}}%
\pgfpathlineto{\pgfqpoint{3.645703in}{2.356742in}}%
\pgfpathlineto{\pgfqpoint{3.645863in}{2.408125in}}%
\pgfpathlineto{\pgfqpoint{3.646159in}{2.356789in}}%
\pgfpathlineto{\pgfqpoint{3.646171in}{2.356698in}}%
\pgfpathlineto{\pgfqpoint{3.646183in}{2.359981in}}%
\pgfpathlineto{\pgfqpoint{3.646786in}{2.402755in}}%
\pgfpathlineto{\pgfqpoint{3.647094in}{2.354279in}}%
\pgfpathlineto{\pgfqpoint{3.647316in}{2.383408in}}%
\pgfpathlineto{\pgfqpoint{3.647906in}{2.357219in}}%
\pgfpathlineto{\pgfqpoint{3.647722in}{2.400588in}}%
\pgfpathlineto{\pgfqpoint{3.648177in}{2.398736in}}%
\pgfpathlineto{\pgfqpoint{3.648189in}{2.403342in}}%
\pgfpathlineto{\pgfqpoint{3.648780in}{2.357704in}}%
\pgfpathlineto{\pgfqpoint{3.649199in}{2.368525in}}%
\pgfpathlineto{\pgfqpoint{3.650171in}{2.355921in}}%
\pgfpathlineto{\pgfqpoint{3.649605in}{2.398187in}}%
\pgfpathlineto{\pgfqpoint{3.650282in}{2.369915in}}%
\pgfpathlineto{\pgfqpoint{3.650528in}{2.401057in}}%
\pgfpathlineto{\pgfqpoint{3.651106in}{2.354445in}}%
\pgfpathlineto{\pgfqpoint{3.651463in}{2.393906in}}%
\pgfpathlineto{\pgfqpoint{3.652497in}{2.353415in}}%
\pgfpathlineto{\pgfqpoint{3.652608in}{2.377112in}}%
\pgfpathlineto{\pgfqpoint{3.652756in}{2.395184in}}%
\pgfpathlineto{\pgfqpoint{3.653420in}{2.352349in}}%
\pgfpathlineto{\pgfqpoint{3.653789in}{2.384869in}}%
\pgfpathlineto{\pgfqpoint{3.654811in}{2.351946in}}%
\pgfpathlineto{\pgfqpoint{3.654602in}{2.389596in}}%
\pgfpathlineto{\pgfqpoint{3.654909in}{2.375158in}}%
\pgfpathlineto{\pgfqpoint{3.655980in}{2.392797in}}%
\pgfpathlineto{\pgfqpoint{3.655734in}{2.352202in}}%
\pgfpathlineto{\pgfqpoint{3.656017in}{2.382220in}}%
\pgfpathlineto{\pgfqpoint{3.656202in}{2.352642in}}%
\pgfpathlineto{\pgfqpoint{3.656768in}{2.394825in}}%
\pgfpathlineto{\pgfqpoint{3.657199in}{2.363432in}}%
\pgfpathlineto{\pgfqpoint{3.658269in}{2.404399in}}%
\pgfpathlineto{\pgfqpoint{3.658036in}{2.348343in}}%
\pgfpathlineto{\pgfqpoint{3.658319in}{2.383330in}}%
\pgfpathlineto{\pgfqpoint{3.659414in}{2.353726in}}%
\pgfpathlineto{\pgfqpoint{3.659192in}{2.405650in}}%
\pgfpathlineto{\pgfqpoint{3.659488in}{2.361939in}}%
\pgfpathlineto{\pgfqpoint{3.659648in}{2.404416in}}%
\pgfpathlineto{\pgfqpoint{3.660374in}{2.350770in}}%
\pgfpathlineto{\pgfqpoint{3.660620in}{2.377506in}}%
\pgfpathlineto{\pgfqpoint{3.660842in}{2.353123in}}%
\pgfpathlineto{\pgfqpoint{3.661039in}{2.398538in}}%
\pgfpathlineto{\pgfqpoint{3.661765in}{2.364308in}}%
\pgfpathlineto{\pgfqpoint{3.662269in}{2.397663in}}%
\pgfpathlineto{\pgfqpoint{3.662651in}{2.358192in}}%
\pgfpathlineto{\pgfqpoint{3.662897in}{2.377767in}}%
\pgfpathlineto{\pgfqpoint{3.663451in}{2.357500in}}%
\pgfpathlineto{\pgfqpoint{3.663635in}{2.397383in}}%
\pgfpathlineto{\pgfqpoint{3.664029in}{2.358863in}}%
\pgfpathlineto{\pgfqpoint{3.664546in}{2.400196in}}%
\pgfpathlineto{\pgfqpoint{3.664940in}{2.358057in}}%
\pgfpathlineto{\pgfqpoint{3.665174in}{2.373563in}}%
\pgfpathlineto{\pgfqpoint{3.665395in}{2.360550in}}%
\pgfpathlineto{\pgfqpoint{3.665912in}{2.407453in}}%
\pgfpathlineto{\pgfqpoint{3.666257in}{2.375808in}}%
\pgfpathlineto{\pgfqpoint{3.666368in}{2.406342in}}%
\pgfpathlineto{\pgfqpoint{3.666319in}{2.361743in}}%
\pgfpathlineto{\pgfqpoint{3.667352in}{2.372023in}}%
\pgfpathlineto{\pgfqpoint{3.667734in}{2.399994in}}%
\pgfpathlineto{\pgfqpoint{3.668115in}{2.359997in}}%
\pgfpathlineto{\pgfqpoint{3.668448in}{2.371154in}}%
\pgfpathlineto{\pgfqpoint{3.669494in}{2.357759in}}%
\pgfpathlineto{\pgfqpoint{3.669100in}{2.399842in}}%
\pgfpathlineto{\pgfqpoint{3.669519in}{2.373133in}}%
\pgfpathlineto{\pgfqpoint{3.670011in}{2.401283in}}%
\pgfpathlineto{\pgfqpoint{3.670195in}{2.354068in}}%
\pgfpathlineto{\pgfqpoint{3.670626in}{2.369458in}}%
\pgfpathlineto{\pgfqpoint{3.671106in}{2.349882in}}%
\pgfpathlineto{\pgfqpoint{3.671377in}{2.403915in}}%
\pgfpathlineto{\pgfqpoint{3.671722in}{2.375733in}}%
\pgfpathlineto{\pgfqpoint{3.671832in}{2.400573in}}%
\pgfpathlineto{\pgfqpoint{3.672017in}{2.355224in}}%
\pgfpathlineto{\pgfqpoint{3.672829in}{2.377428in}}%
\pgfpathlineto{\pgfqpoint{3.672842in}{2.377269in}}%
\pgfpathlineto{\pgfqpoint{3.672866in}{2.384534in}}%
\pgfpathlineto{\pgfqpoint{3.673211in}{2.399719in}}%
\pgfpathlineto{\pgfqpoint{3.673395in}{2.357849in}}%
\pgfpathlineto{\pgfqpoint{3.673949in}{2.370353in}}%
\pgfpathlineto{\pgfqpoint{3.674122in}{2.395799in}}%
\pgfpathlineto{\pgfqpoint{3.674319in}{2.360595in}}%
\pgfpathlineto{\pgfqpoint{3.675057in}{2.379051in}}%
\pgfpathlineto{\pgfqpoint{3.675771in}{2.360572in}}%
\pgfpathlineto{\pgfqpoint{3.675955in}{2.394236in}}%
\pgfpathlineto{\pgfqpoint{3.676165in}{2.371627in}}%
\pgfpathlineto{\pgfqpoint{3.676879in}{2.394595in}}%
\pgfpathlineto{\pgfqpoint{3.676239in}{2.361336in}}%
\pgfpathlineto{\pgfqpoint{3.677297in}{2.380874in}}%
\pgfpathlineto{\pgfqpoint{3.678097in}{2.361918in}}%
\pgfpathlineto{\pgfqpoint{3.677802in}{2.393943in}}%
\pgfpathlineto{\pgfqpoint{3.678257in}{2.390174in}}%
\pgfpathlineto{\pgfqpoint{3.679192in}{2.394699in}}%
\pgfpathlineto{\pgfqpoint{3.679020in}{2.361886in}}%
\pgfpathlineto{\pgfqpoint{3.679315in}{2.374970in}}%
\pgfpathlineto{\pgfqpoint{3.680300in}{2.361835in}}%
\pgfpathlineto{\pgfqpoint{3.679660in}{2.392697in}}%
\pgfpathlineto{\pgfqpoint{3.680435in}{2.369500in}}%
\pgfpathlineto{\pgfqpoint{3.681445in}{2.389974in}}%
\pgfpathlineto{\pgfqpoint{3.681235in}{2.358404in}}%
\pgfpathlineto{\pgfqpoint{3.681568in}{2.379904in}}%
\pgfpathlineto{\pgfqpoint{3.682085in}{2.357985in}}%
\pgfpathlineto{\pgfqpoint{3.682368in}{2.392448in}}%
\pgfpathlineto{\pgfqpoint{3.682675in}{2.377924in}}%
\pgfpathlineto{\pgfqpoint{3.682909in}{2.391730in}}%
\pgfpathlineto{\pgfqpoint{3.683475in}{2.358987in}}%
\pgfpathlineto{\pgfqpoint{3.683832in}{2.389499in}}%
\pgfpathlineto{\pgfqpoint{3.683845in}{2.390867in}}%
\pgfpathlineto{\pgfqpoint{3.684411in}{2.355369in}}%
\pgfpathlineto{\pgfqpoint{3.684829in}{2.376802in}}%
\pgfpathlineto{\pgfqpoint{3.685814in}{2.349693in}}%
\pgfpathlineto{\pgfqpoint{3.685506in}{2.392180in}}%
\pgfpathlineto{\pgfqpoint{3.685925in}{2.371631in}}%
\pgfpathlineto{\pgfqpoint{3.686897in}{2.394862in}}%
\pgfpathlineto{\pgfqpoint{3.686737in}{2.345812in}}%
\pgfpathlineto{\pgfqpoint{3.687045in}{2.384912in}}%
\pgfpathlineto{\pgfqpoint{3.687205in}{2.346965in}}%
\pgfpathlineto{\pgfqpoint{3.687832in}{2.397257in}}%
\pgfpathlineto{\pgfqpoint{3.688214in}{2.365187in}}%
\pgfpathlineto{\pgfqpoint{3.689235in}{2.401292in}}%
\pgfpathlineto{\pgfqpoint{3.689063in}{2.355677in}}%
\pgfpathlineto{\pgfqpoint{3.689371in}{2.383141in}}%
\pgfpathlineto{\pgfqpoint{3.689543in}{2.352007in}}%
\pgfpathlineto{\pgfqpoint{3.689703in}{2.399541in}}%
\pgfpathlineto{\pgfqpoint{3.690540in}{2.369908in}}%
\pgfpathlineto{\pgfqpoint{3.691094in}{2.398648in}}%
\pgfpathlineto{\pgfqpoint{3.691389in}{2.351589in}}%
\pgfpathlineto{\pgfqpoint{3.691660in}{2.373263in}}%
\pgfpathlineto{\pgfqpoint{3.691857in}{2.353515in}}%
\pgfpathlineto{\pgfqpoint{3.692485in}{2.398192in}}%
\pgfpathlineto{\pgfqpoint{3.692817in}{2.365827in}}%
\pgfpathlineto{\pgfqpoint{3.693888in}{2.399476in}}%
\pgfpathlineto{\pgfqpoint{3.693728in}{2.356762in}}%
\pgfpathlineto{\pgfqpoint{3.693962in}{2.378709in}}%
\pgfpathlineto{\pgfqpoint{3.694195in}{2.357231in}}%
\pgfpathlineto{\pgfqpoint{3.694811in}{2.397387in}}%
\pgfpathlineto{\pgfqpoint{3.695143in}{2.368052in}}%
\pgfpathlineto{\pgfqpoint{3.695746in}{2.396967in}}%
\pgfpathlineto{\pgfqpoint{3.695857in}{2.361814in}}%
\pgfpathlineto{\pgfqpoint{3.696263in}{2.377300in}}%
\pgfpathlineto{\pgfqpoint{3.696275in}{2.377417in}}%
\pgfpathlineto{\pgfqpoint{3.696300in}{2.368148in}}%
\pgfpathlineto{\pgfqpoint{3.697248in}{2.356040in}}%
\pgfpathlineto{\pgfqpoint{3.696657in}{2.393066in}}%
\pgfpathlineto{\pgfqpoint{3.697408in}{2.367976in}}%
\pgfpathlineto{\pgfqpoint{3.697703in}{2.355898in}}%
\pgfpathlineto{\pgfqpoint{3.698048in}{2.389153in}}%
\pgfpathlineto{\pgfqpoint{3.698405in}{2.377981in}}%
\pgfpathlineto{\pgfqpoint{3.698946in}{2.389148in}}%
\pgfpathlineto{\pgfqpoint{3.699094in}{2.357762in}}%
\pgfpathlineto{\pgfqpoint{3.699488in}{2.377942in}}%
\pgfpathlineto{\pgfqpoint{3.699562in}{2.358065in}}%
\pgfpathlineto{\pgfqpoint{3.700337in}{2.388800in}}%
\pgfpathlineto{\pgfqpoint{3.700583in}{2.381430in}}%
\pgfpathlineto{\pgfqpoint{3.701260in}{2.389307in}}%
\pgfpathlineto{\pgfqpoint{3.700952in}{2.359448in}}%
\pgfpathlineto{\pgfqpoint{3.701383in}{2.370723in}}%
\pgfpathlineto{\pgfqpoint{3.701408in}{2.359213in}}%
\pgfpathlineto{\pgfqpoint{3.701728in}{2.390011in}}%
\pgfpathlineto{\pgfqpoint{3.702491in}{2.368494in}}%
\pgfpathlineto{\pgfqpoint{3.702577in}{2.393188in}}%
\pgfpathlineto{\pgfqpoint{3.702860in}{2.362291in}}%
\pgfpathlineto{\pgfqpoint{3.703611in}{2.378604in}}%
\pgfpathlineto{\pgfqpoint{3.704694in}{2.359885in}}%
\pgfpathlineto{\pgfqpoint{3.704411in}{2.395203in}}%
\pgfpathlineto{\pgfqpoint{3.704718in}{2.374463in}}%
\pgfpathlineto{\pgfqpoint{3.704878in}{2.394537in}}%
\pgfpathlineto{\pgfqpoint{3.705026in}{2.360474in}}%
\pgfpathlineto{\pgfqpoint{3.705826in}{2.381008in}}%
\pgfpathlineto{\pgfqpoint{3.706909in}{2.363357in}}%
\pgfpathlineto{\pgfqpoint{3.706257in}{2.393613in}}%
\pgfpathlineto{\pgfqpoint{3.706983in}{2.367231in}}%
\pgfpathlineto{\pgfqpoint{3.707943in}{2.393621in}}%
\pgfpathlineto{\pgfqpoint{3.707980in}{2.363339in}}%
\pgfpathlineto{\pgfqpoint{3.708128in}{2.383371in}}%
\pgfpathlineto{\pgfqpoint{3.708152in}{2.384035in}}%
\pgfpathlineto{\pgfqpoint{3.708189in}{2.373530in}}%
\pgfpathlineto{\pgfqpoint{3.709235in}{2.359956in}}%
\pgfpathlineto{\pgfqpoint{3.708854in}{2.401904in}}%
\pgfpathlineto{\pgfqpoint{3.709272in}{2.375791in}}%
\pgfpathlineto{\pgfqpoint{3.710220in}{2.404563in}}%
\pgfpathlineto{\pgfqpoint{3.710158in}{2.357291in}}%
\pgfpathlineto{\pgfqpoint{3.710380in}{2.381959in}}%
\pgfpathlineto{\pgfqpoint{3.711537in}{2.353594in}}%
\pgfpathlineto{\pgfqpoint{3.711131in}{2.404857in}}%
\pgfpathlineto{\pgfqpoint{3.711549in}{2.355618in}}%
\pgfpathlineto{\pgfqpoint{3.712054in}{2.402703in}}%
\pgfpathlineto{\pgfqpoint{3.712669in}{2.372271in}}%
\pgfpathlineto{\pgfqpoint{3.713358in}{2.360962in}}%
\pgfpathlineto{\pgfqpoint{3.712965in}{2.398977in}}%
\pgfpathlineto{\pgfqpoint{3.713765in}{2.376666in}}%
\pgfpathlineto{\pgfqpoint{3.713777in}{2.376788in}}%
\pgfpathlineto{\pgfqpoint{3.713789in}{2.369992in}}%
\pgfpathlineto{\pgfqpoint{3.714725in}{2.360150in}}%
\pgfpathlineto{\pgfqpoint{3.714786in}{2.398977in}}%
\pgfpathlineto{\pgfqpoint{3.714885in}{2.377903in}}%
\pgfpathlineto{\pgfqpoint{3.714897in}{2.377899in}}%
\pgfpathlineto{\pgfqpoint{3.715241in}{2.401352in}}%
\pgfpathlineto{\pgfqpoint{3.715525in}{2.360055in}}%
\pgfpathlineto{\pgfqpoint{3.715955in}{2.371678in}}%
\pgfpathlineto{\pgfqpoint{3.716435in}{2.360098in}}%
\pgfpathlineto{\pgfqpoint{3.716152in}{2.401241in}}%
\pgfpathlineto{\pgfqpoint{3.717038in}{2.380361in}}%
\pgfpathlineto{\pgfqpoint{3.717063in}{2.400249in}}%
\pgfpathlineto{\pgfqpoint{3.717814in}{2.359196in}}%
\pgfpathlineto{\pgfqpoint{3.718134in}{2.385639in}}%
\pgfpathlineto{\pgfqpoint{3.718269in}{2.362103in}}%
\pgfpathlineto{\pgfqpoint{3.718897in}{2.397218in}}%
\pgfpathlineto{\pgfqpoint{3.719254in}{2.367554in}}%
\pgfpathlineto{\pgfqpoint{3.719352in}{2.396617in}}%
\pgfpathlineto{\pgfqpoint{3.720091in}{2.361575in}}%
\pgfpathlineto{\pgfqpoint{3.720423in}{2.384036in}}%
\pgfpathlineto{\pgfqpoint{3.721469in}{2.356049in}}%
\pgfpathlineto{\pgfqpoint{3.720731in}{2.397211in}}%
\pgfpathlineto{\pgfqpoint{3.721568in}{2.364525in}}%
\pgfpathlineto{\pgfqpoint{3.722577in}{2.398717in}}%
\pgfpathlineto{\pgfqpoint{3.722392in}{2.355172in}}%
\pgfpathlineto{\pgfqpoint{3.722737in}{2.371793in}}%
\pgfpathlineto{\pgfqpoint{3.723328in}{2.357619in}}%
\pgfpathlineto{\pgfqpoint{3.723032in}{2.399651in}}%
\pgfpathlineto{\pgfqpoint{3.723869in}{2.365146in}}%
\pgfpathlineto{\pgfqpoint{3.724423in}{2.401192in}}%
\pgfpathlineto{\pgfqpoint{3.724251in}{2.358033in}}%
\pgfpathlineto{\pgfqpoint{3.724989in}{2.377330in}}%
\pgfpathlineto{\pgfqpoint{3.725346in}{2.401694in}}%
\pgfpathlineto{\pgfqpoint{3.725641in}{2.360812in}}%
\pgfpathlineto{\pgfqpoint{3.726072in}{2.378167in}}%
\pgfpathlineto{\pgfqpoint{3.726921in}{2.359015in}}%
\pgfpathlineto{\pgfqpoint{3.726281in}{2.397605in}}%
\pgfpathlineto{\pgfqpoint{3.727168in}{2.381422in}}%
\pgfpathlineto{\pgfqpoint{3.727204in}{2.396105in}}%
\pgfpathlineto{\pgfqpoint{3.727844in}{2.357410in}}%
\pgfpathlineto{\pgfqpoint{3.728201in}{2.376835in}}%
\pgfpathlineto{\pgfqpoint{3.728312in}{2.357144in}}%
\pgfpathlineto{\pgfqpoint{3.728608in}{2.393212in}}%
\pgfpathlineto{\pgfqpoint{3.729309in}{2.378343in}}%
\pgfpathlineto{\pgfqpoint{3.729321in}{2.378407in}}%
\pgfpathlineto{\pgfqpoint{3.729334in}{2.375559in}}%
\pgfpathlineto{\pgfqpoint{3.730171in}{2.358121in}}%
\pgfpathlineto{\pgfqpoint{3.729998in}{2.392094in}}%
\pgfpathlineto{\pgfqpoint{3.730417in}{2.381298in}}%
\pgfpathlineto{\pgfqpoint{3.730466in}{2.390808in}}%
\pgfpathlineto{\pgfqpoint{3.731032in}{2.357693in}}%
\pgfpathlineto{\pgfqpoint{3.731438in}{2.375523in}}%
\pgfpathlineto{\pgfqpoint{3.731955in}{2.354872in}}%
\pgfpathlineto{\pgfqpoint{3.731857in}{2.388874in}}%
\pgfpathlineto{\pgfqpoint{3.732534in}{2.373497in}}%
\pgfpathlineto{\pgfqpoint{3.732792in}{2.386462in}}%
\pgfpathlineto{\pgfqpoint{3.732952in}{2.356279in}}%
\pgfpathlineto{\pgfqpoint{3.733654in}{2.383254in}}%
\pgfpathlineto{\pgfqpoint{3.733875in}{2.356835in}}%
\pgfpathlineto{\pgfqpoint{3.734454in}{2.386611in}}%
\pgfpathlineto{\pgfqpoint{3.734860in}{2.375606in}}%
\pgfpathlineto{\pgfqpoint{3.735844in}{2.392347in}}%
\pgfpathlineto{\pgfqpoint{3.735697in}{2.358514in}}%
\pgfpathlineto{\pgfqpoint{3.736004in}{2.383258in}}%
\pgfpathlineto{\pgfqpoint{3.737075in}{2.355335in}}%
\pgfpathlineto{\pgfqpoint{3.736312in}{2.392176in}}%
\pgfpathlineto{\pgfqpoint{3.737174in}{2.369863in}}%
\pgfpathlineto{\pgfqpoint{3.737235in}{2.388642in}}%
\pgfpathlineto{\pgfqpoint{3.737543in}{2.356323in}}%
\pgfpathlineto{\pgfqpoint{3.738294in}{2.377569in}}%
\pgfpathlineto{\pgfqpoint{3.739106in}{2.389597in}}%
\pgfpathlineto{\pgfqpoint{3.738466in}{2.361184in}}%
\pgfpathlineto{\pgfqpoint{3.739328in}{2.372809in}}%
\pgfpathlineto{\pgfqpoint{3.740337in}{2.361962in}}%
\pgfpathlineto{\pgfqpoint{3.739561in}{2.389031in}}%
\pgfpathlineto{\pgfqpoint{3.740423in}{2.376724in}}%
\pgfpathlineto{\pgfqpoint{3.741395in}{2.389073in}}%
\pgfpathlineto{\pgfqpoint{3.741198in}{2.360683in}}%
\pgfpathlineto{\pgfqpoint{3.741506in}{2.374467in}}%
\pgfpathlineto{\pgfqpoint{3.742454in}{2.360778in}}%
\pgfpathlineto{\pgfqpoint{3.742355in}{2.388535in}}%
\pgfpathlineto{\pgfqpoint{3.742626in}{2.367641in}}%
\pgfpathlineto{\pgfqpoint{3.743377in}{2.360559in}}%
\pgfpathlineto{\pgfqpoint{3.742811in}{2.388681in}}%
\pgfpathlineto{\pgfqpoint{3.743598in}{2.379267in}}%
\pgfpathlineto{\pgfqpoint{3.743684in}{2.389617in}}%
\pgfpathlineto{\pgfqpoint{3.744448in}{2.358801in}}%
\pgfpathlineto{\pgfqpoint{3.744718in}{2.381575in}}%
\pgfpathlineto{\pgfqpoint{3.745826in}{2.357394in}}%
\pgfpathlineto{\pgfqpoint{3.745543in}{2.389802in}}%
\pgfpathlineto{\pgfqpoint{3.745863in}{2.371436in}}%
\pgfpathlineto{\pgfqpoint{3.746921in}{2.391427in}}%
\pgfpathlineto{\pgfqpoint{3.746294in}{2.358111in}}%
\pgfpathlineto{\pgfqpoint{3.746971in}{2.377409in}}%
\pgfpathlineto{\pgfqpoint{3.747204in}{2.357793in}}%
\pgfpathlineto{\pgfqpoint{3.747389in}{2.390437in}}%
\pgfpathlineto{\pgfqpoint{3.748066in}{2.370529in}}%
\pgfpathlineto{\pgfqpoint{3.748546in}{2.386759in}}%
\pgfpathlineto{\pgfqpoint{3.748127in}{2.359926in}}%
\pgfpathlineto{\pgfqpoint{3.749186in}{2.380255in}}%
\pgfpathlineto{\pgfqpoint{3.750269in}{2.362418in}}%
\pgfpathlineto{\pgfqpoint{3.749617in}{2.387981in}}%
\pgfpathlineto{\pgfqpoint{3.750355in}{2.373192in}}%
\pgfpathlineto{\pgfqpoint{3.750540in}{2.388423in}}%
\pgfpathlineto{\pgfqpoint{3.751327in}{2.358972in}}%
\pgfpathlineto{\pgfqpoint{3.751475in}{2.381609in}}%
\pgfpathlineto{\pgfqpoint{3.752214in}{2.390360in}}%
\pgfpathlineto{\pgfqpoint{3.751795in}{2.359196in}}%
\pgfpathlineto{\pgfqpoint{3.752521in}{2.374413in}}%
\pgfpathlineto{\pgfqpoint{3.753124in}{2.395140in}}%
\pgfpathlineto{\pgfqpoint{3.752706in}{2.362440in}}%
\pgfpathlineto{\pgfqpoint{3.753506in}{2.365686in}}%
\pgfpathlineto{\pgfqpoint{3.754441in}{2.362594in}}%
\pgfpathlineto{\pgfqpoint{3.753580in}{2.397328in}}%
\pgfpathlineto{\pgfqpoint{3.754478in}{2.383690in}}%
\pgfpathlineto{\pgfqpoint{3.754503in}{2.396957in}}%
\pgfpathlineto{\pgfqpoint{3.754897in}{2.361204in}}%
\pgfpathlineto{\pgfqpoint{3.755574in}{2.377454in}}%
\pgfpathlineto{\pgfqpoint{3.755820in}{2.364134in}}%
\pgfpathlineto{\pgfqpoint{3.756337in}{2.395787in}}%
\pgfpathlineto{\pgfqpoint{3.756706in}{2.367959in}}%
\pgfpathlineto{\pgfqpoint{3.757518in}{2.365884in}}%
\pgfpathlineto{\pgfqpoint{3.756792in}{2.395701in}}%
\pgfpathlineto{\pgfqpoint{3.757678in}{2.380088in}}%
\pgfpathlineto{\pgfqpoint{3.758614in}{2.395668in}}%
\pgfpathlineto{\pgfqpoint{3.758441in}{2.363713in}}%
\pgfpathlineto{\pgfqpoint{3.758786in}{2.378020in}}%
\pgfpathlineto{\pgfqpoint{3.759820in}{2.360969in}}%
\pgfpathlineto{\pgfqpoint{3.759537in}{2.399459in}}%
\pgfpathlineto{\pgfqpoint{3.759931in}{2.366961in}}%
\pgfpathlineto{\pgfqpoint{3.759992in}{2.399451in}}%
\pgfpathlineto{\pgfqpoint{3.760287in}{2.360452in}}%
\pgfpathlineto{\pgfqpoint{3.761063in}{2.379903in}}%
\pgfpathlineto{\pgfqpoint{3.762121in}{2.358354in}}%
\pgfpathlineto{\pgfqpoint{3.761826in}{2.396566in}}%
\pgfpathlineto{\pgfqpoint{3.762195in}{2.369101in}}%
\pgfpathlineto{\pgfqpoint{3.762220in}{2.367207in}}%
\pgfpathlineto{\pgfqpoint{3.762244in}{2.371047in}}%
\pgfpathlineto{\pgfqpoint{3.762281in}{2.395289in}}%
\pgfpathlineto{\pgfqpoint{3.763032in}{2.356291in}}%
\pgfpathlineto{\pgfqpoint{3.763364in}{2.377636in}}%
\pgfpathlineto{\pgfqpoint{3.764411in}{2.358262in}}%
\pgfpathlineto{\pgfqpoint{3.763660in}{2.395561in}}%
\pgfpathlineto{\pgfqpoint{3.764497in}{2.364020in}}%
\pgfpathlineto{\pgfqpoint{3.765038in}{2.398659in}}%
\pgfpathlineto{\pgfqpoint{3.764878in}{2.358512in}}%
\pgfpathlineto{\pgfqpoint{3.765654in}{2.379665in}}%
\pgfpathlineto{\pgfqpoint{3.766712in}{2.359141in}}%
\pgfpathlineto{\pgfqpoint{3.766417in}{2.400520in}}%
\pgfpathlineto{\pgfqpoint{3.766798in}{2.363778in}}%
\pgfpathlineto{\pgfqpoint{3.767807in}{2.403336in}}%
\pgfpathlineto{\pgfqpoint{3.767241in}{2.358912in}}%
\pgfpathlineto{\pgfqpoint{3.767955in}{2.377875in}}%
\pgfpathlineto{\pgfqpoint{3.769087in}{2.359184in}}%
\pgfpathlineto{\pgfqpoint{3.768730in}{2.403305in}}%
\pgfpathlineto{\pgfqpoint{3.769100in}{2.363354in}}%
\pgfpathlineto{\pgfqpoint{3.769654in}{2.401328in}}%
\pgfpathlineto{\pgfqpoint{3.769949in}{2.359754in}}%
\pgfpathlineto{\pgfqpoint{3.770232in}{2.381399in}}%
\pgfpathlineto{\pgfqpoint{3.770417in}{2.359522in}}%
\pgfpathlineto{\pgfqpoint{3.771044in}{2.397465in}}%
\pgfpathlineto{\pgfqpoint{3.771414in}{2.368988in}}%
\pgfpathlineto{\pgfqpoint{3.771980in}{2.395676in}}%
\pgfpathlineto{\pgfqpoint{3.771807in}{2.361031in}}%
\pgfpathlineto{\pgfqpoint{3.772521in}{2.369680in}}%
\pgfpathlineto{\pgfqpoint{3.773370in}{2.393984in}}%
\pgfpathlineto{\pgfqpoint{3.773555in}{2.360849in}}%
\pgfpathlineto{\pgfqpoint{3.773629in}{2.374211in}}%
\pgfpathlineto{\pgfqpoint{3.774466in}{2.359303in}}%
\pgfpathlineto{\pgfqpoint{3.773838in}{2.393980in}}%
\pgfpathlineto{\pgfqpoint{3.774712in}{2.384597in}}%
\pgfpathlineto{\pgfqpoint{3.774774in}{2.391819in}}%
\pgfpathlineto{\pgfqpoint{3.774860in}{2.363279in}}%
\pgfpathlineto{\pgfqpoint{3.774909in}{2.368593in}}%
\pgfpathlineto{\pgfqpoint{3.775401in}{2.360550in}}%
\pgfpathlineto{\pgfqpoint{3.775623in}{2.392799in}}%
\pgfpathlineto{\pgfqpoint{3.776004in}{2.373619in}}%
\pgfpathlineto{\pgfqpoint{3.776090in}{2.393004in}}%
\pgfpathlineto{\pgfqpoint{3.776792in}{2.358734in}}%
\pgfpathlineto{\pgfqpoint{3.777112in}{2.380278in}}%
\pgfpathlineto{\pgfqpoint{3.777260in}{2.359594in}}%
\pgfpathlineto{\pgfqpoint{3.777949in}{2.389593in}}%
\pgfpathlineto{\pgfqpoint{3.778232in}{2.373451in}}%
\pgfpathlineto{\pgfqpoint{3.778884in}{2.390620in}}%
\pgfpathlineto{\pgfqpoint{3.779069in}{2.361433in}}%
\pgfpathlineto{\pgfqpoint{3.779364in}{2.385027in}}%
\pgfpathlineto{\pgfqpoint{3.780460in}{2.358080in}}%
\pgfpathlineto{\pgfqpoint{3.780152in}{2.389755in}}%
\pgfpathlineto{\pgfqpoint{3.780509in}{2.369637in}}%
\pgfpathlineto{\pgfqpoint{3.781075in}{2.390905in}}%
\pgfpathlineto{\pgfqpoint{3.781383in}{2.356464in}}%
\pgfpathlineto{\pgfqpoint{3.781641in}{2.375095in}}%
\pgfpathlineto{\pgfqpoint{3.781850in}{2.358521in}}%
\pgfpathlineto{\pgfqpoint{3.782478in}{2.392931in}}%
\pgfpathlineto{\pgfqpoint{3.782786in}{2.361174in}}%
\pgfpathlineto{\pgfqpoint{3.783869in}{2.394499in}}%
\pgfpathlineto{\pgfqpoint{3.783955in}{2.379134in}}%
\pgfpathlineto{\pgfqpoint{3.784915in}{2.361162in}}%
\pgfpathlineto{\pgfqpoint{3.784804in}{2.393108in}}%
\pgfpathlineto{\pgfqpoint{3.785087in}{2.365745in}}%
\pgfpathlineto{\pgfqpoint{3.785838in}{2.358463in}}%
\pgfpathlineto{\pgfqpoint{3.785272in}{2.393294in}}%
\pgfpathlineto{\pgfqpoint{3.786097in}{2.379684in}}%
\pgfpathlineto{\pgfqpoint{3.786306in}{2.358476in}}%
\pgfpathlineto{\pgfqpoint{3.786195in}{2.393379in}}%
\pgfpathlineto{\pgfqpoint{3.787093in}{2.386973in}}%
\pgfpathlineto{\pgfqpoint{3.787118in}{2.390427in}}%
\pgfpathlineto{\pgfqpoint{3.787697in}{2.357978in}}%
\pgfpathlineto{\pgfqpoint{3.788127in}{2.372980in}}%
\pgfpathlineto{\pgfqpoint{3.788620in}{2.356530in}}%
\pgfpathlineto{\pgfqpoint{3.788484in}{2.386915in}}%
\pgfpathlineto{\pgfqpoint{3.789235in}{2.368753in}}%
\pgfpathlineto{\pgfqpoint{3.790244in}{2.387986in}}%
\pgfpathlineto{\pgfqpoint{3.789543in}{2.358065in}}%
\pgfpathlineto{\pgfqpoint{3.790367in}{2.383324in}}%
\pgfpathlineto{\pgfqpoint{3.791450in}{2.355014in}}%
\pgfpathlineto{\pgfqpoint{3.790577in}{2.388673in}}%
\pgfpathlineto{\pgfqpoint{3.791475in}{2.362742in}}%
\pgfpathlineto{\pgfqpoint{3.792423in}{2.390722in}}%
\pgfpathlineto{\pgfqpoint{3.791918in}{2.353585in}}%
\pgfpathlineto{\pgfqpoint{3.792583in}{2.376907in}}%
\pgfpathlineto{\pgfqpoint{3.792841in}{2.355299in}}%
\pgfpathlineto{\pgfqpoint{3.793346in}{2.395959in}}%
\pgfpathlineto{\pgfqpoint{3.793715in}{2.367862in}}%
\pgfpathlineto{\pgfqpoint{3.793937in}{2.398642in}}%
\pgfpathlineto{\pgfqpoint{3.793764in}{2.359244in}}%
\pgfpathlineto{\pgfqpoint{3.794884in}{2.386294in}}%
\pgfpathlineto{\pgfqpoint{3.795610in}{2.357898in}}%
\pgfpathlineto{\pgfqpoint{3.795192in}{2.396147in}}%
\pgfpathlineto{\pgfqpoint{3.796017in}{2.368054in}}%
\pgfpathlineto{\pgfqpoint{3.796238in}{2.394132in}}%
\pgfpathlineto{\pgfqpoint{3.796521in}{2.357633in}}%
\pgfpathlineto{\pgfqpoint{3.797272in}{2.378196in}}%
\pgfpathlineto{\pgfqpoint{3.797444in}{2.361438in}}%
\pgfpathlineto{\pgfqpoint{3.797949in}{2.392785in}}%
\pgfpathlineto{\pgfqpoint{3.798380in}{2.370664in}}%
\pgfpathlineto{\pgfqpoint{3.798404in}{2.394114in}}%
\pgfpathlineto{\pgfqpoint{3.799217in}{2.359964in}}%
\pgfpathlineto{\pgfqpoint{3.799487in}{2.380538in}}%
\pgfpathlineto{\pgfqpoint{3.799684in}{2.359301in}}%
\pgfpathlineto{\pgfqpoint{3.800238in}{2.395778in}}%
\pgfpathlineto{\pgfqpoint{3.800644in}{2.371547in}}%
\pgfpathlineto{\pgfqpoint{3.801617in}{2.401302in}}%
\pgfpathlineto{\pgfqpoint{3.800743in}{2.360843in}}%
\pgfpathlineto{\pgfqpoint{3.801777in}{2.379537in}}%
\pgfpathlineto{\pgfqpoint{3.802010in}{2.362011in}}%
\pgfpathlineto{\pgfqpoint{3.802540in}{2.401219in}}%
\pgfpathlineto{\pgfqpoint{3.802921in}{2.365767in}}%
\pgfpathlineto{\pgfqpoint{3.803500in}{2.362683in}}%
\pgfpathlineto{\pgfqpoint{3.802995in}{2.402854in}}%
\pgfpathlineto{\pgfqpoint{3.803893in}{2.384409in}}%
\pgfpathlineto{\pgfqpoint{3.804373in}{2.405240in}}%
\pgfpathlineto{\pgfqpoint{3.803955in}{2.365202in}}%
\pgfpathlineto{\pgfqpoint{3.804989in}{2.378559in}}%
\pgfpathlineto{\pgfqpoint{3.805937in}{2.363659in}}%
\pgfpathlineto{\pgfqpoint{3.805297in}{2.404089in}}%
\pgfpathlineto{\pgfqpoint{3.806109in}{2.367839in}}%
\pgfpathlineto{\pgfqpoint{3.806121in}{2.367661in}}%
\pgfpathlineto{\pgfqpoint{3.806195in}{2.381337in}}%
\pgfpathlineto{\pgfqpoint{3.806220in}{2.395513in}}%
\pgfpathlineto{\pgfqpoint{3.806404in}{2.363045in}}%
\pgfpathlineto{\pgfqpoint{3.807290in}{2.372191in}}%
\pgfpathlineto{\pgfqpoint{3.807869in}{2.365310in}}%
\pgfpathlineto{\pgfqpoint{3.808053in}{2.396792in}}%
\pgfpathlineto{\pgfqpoint{3.808361in}{2.377812in}}%
\pgfpathlineto{\pgfqpoint{3.809432in}{2.398932in}}%
\pgfpathlineto{\pgfqpoint{3.809149in}{2.364794in}}%
\pgfpathlineto{\pgfqpoint{3.809456in}{2.379207in}}%
\pgfpathlineto{\pgfqpoint{3.810527in}{2.361154in}}%
\pgfpathlineto{\pgfqpoint{3.810355in}{2.399126in}}%
\pgfpathlineto{\pgfqpoint{3.810576in}{2.372439in}}%
\pgfpathlineto{\pgfqpoint{3.810995in}{2.360773in}}%
\pgfpathlineto{\pgfqpoint{3.810823in}{2.398440in}}%
\pgfpathlineto{\pgfqpoint{3.811598in}{2.377601in}}%
\pgfpathlineto{\pgfqpoint{3.811746in}{2.397874in}}%
\pgfpathlineto{\pgfqpoint{3.811918in}{2.361291in}}%
\pgfpathlineto{\pgfqpoint{3.812693in}{2.377616in}}%
\pgfpathlineto{\pgfqpoint{3.813690in}{2.359373in}}%
\pgfpathlineto{\pgfqpoint{3.813136in}{2.395022in}}%
\pgfpathlineto{\pgfqpoint{3.813801in}{2.368250in}}%
\pgfpathlineto{\pgfqpoint{3.814453in}{2.392521in}}%
\pgfpathlineto{\pgfqpoint{3.814613in}{2.356136in}}%
\pgfpathlineto{\pgfqpoint{3.814933in}{2.382103in}}%
\pgfpathlineto{\pgfqpoint{3.815536in}{2.352052in}}%
\pgfpathlineto{\pgfqpoint{3.815844in}{2.395352in}}%
\pgfpathlineto{\pgfqpoint{3.816066in}{2.369450in}}%
\pgfpathlineto{\pgfqpoint{3.816927in}{2.350369in}}%
\pgfpathlineto{\pgfqpoint{3.816632in}{2.398087in}}%
\pgfpathlineto{\pgfqpoint{3.817087in}{2.397023in}}%
\pgfpathlineto{\pgfqpoint{3.818035in}{2.400614in}}%
\pgfpathlineto{\pgfqpoint{3.817395in}{2.350831in}}%
\pgfpathlineto{\pgfqpoint{3.818072in}{2.375297in}}%
\pgfpathlineto{\pgfqpoint{3.818109in}{2.381617in}}%
\pgfpathlineto{\pgfqpoint{3.818958in}{2.403953in}}%
\pgfpathlineto{\pgfqpoint{3.818330in}{2.351660in}}%
\pgfpathlineto{\pgfqpoint{3.819192in}{2.375712in}}%
\pgfpathlineto{\pgfqpoint{3.819733in}{2.354420in}}%
\pgfpathlineto{\pgfqpoint{3.819426in}{2.406276in}}%
\pgfpathlineto{\pgfqpoint{3.820287in}{2.375602in}}%
\pgfpathlineto{\pgfqpoint{3.820349in}{2.406745in}}%
\pgfpathlineto{\pgfqpoint{3.820669in}{2.356416in}}%
\pgfpathlineto{\pgfqpoint{3.821383in}{2.379512in}}%
\pgfpathlineto{\pgfqpoint{3.821961in}{2.357957in}}%
\pgfpathlineto{\pgfqpoint{3.822220in}{2.403677in}}%
\pgfpathlineto{\pgfqpoint{3.822527in}{2.361193in}}%
\pgfpathlineto{\pgfqpoint{3.823278in}{2.357129in}}%
\pgfpathlineto{\pgfqpoint{3.823155in}{2.403212in}}%
\pgfpathlineto{\pgfqpoint{3.823500in}{2.363852in}}%
\pgfpathlineto{\pgfqpoint{3.824558in}{2.403961in}}%
\pgfpathlineto{\pgfqpoint{3.823746in}{2.354387in}}%
\pgfpathlineto{\pgfqpoint{3.824620in}{2.374347in}}%
\pgfpathlineto{\pgfqpoint{3.825136in}{2.353782in}}%
\pgfpathlineto{\pgfqpoint{3.825481in}{2.406274in}}%
\pgfpathlineto{\pgfqpoint{3.825703in}{2.367550in}}%
\pgfpathlineto{\pgfqpoint{3.826404in}{2.404721in}}%
\pgfpathlineto{\pgfqpoint{3.826540in}{2.352211in}}%
\pgfpathlineto{\pgfqpoint{3.826847in}{2.396326in}}%
\pgfpathlineto{\pgfqpoint{3.827352in}{2.402787in}}%
\pgfpathlineto{\pgfqpoint{3.827475in}{2.347942in}}%
\pgfpathlineto{\pgfqpoint{3.827856in}{2.375936in}}%
\pgfpathlineto{\pgfqpoint{3.828866in}{2.342613in}}%
\pgfpathlineto{\pgfqpoint{3.828287in}{2.399408in}}%
\pgfpathlineto{\pgfqpoint{3.828964in}{2.375744in}}%
\pgfpathlineto{\pgfqpoint{3.829210in}{2.396743in}}%
\pgfpathlineto{\pgfqpoint{3.829789in}{2.342530in}}%
\pgfpathlineto{\pgfqpoint{3.830109in}{2.390217in}}%
\pgfpathlineto{\pgfqpoint{3.830133in}{2.394051in}}%
\pgfpathlineto{\pgfqpoint{3.830256in}{2.342768in}}%
\pgfpathlineto{\pgfqpoint{3.831106in}{2.376013in}}%
\pgfpathlineto{\pgfqpoint{3.832115in}{2.343277in}}%
\pgfpathlineto{\pgfqpoint{3.831758in}{2.390438in}}%
\pgfpathlineto{\pgfqpoint{3.832201in}{2.376438in}}%
\pgfpathlineto{\pgfqpoint{3.833149in}{2.391525in}}%
\pgfpathlineto{\pgfqpoint{3.833038in}{2.343319in}}%
\pgfpathlineto{\pgfqpoint{3.833309in}{2.383039in}}%
\pgfpathlineto{\pgfqpoint{3.834072in}{2.396169in}}%
\pgfpathlineto{\pgfqpoint{3.833506in}{2.344984in}}%
\pgfpathlineto{\pgfqpoint{3.834330in}{2.376729in}}%
\pgfpathlineto{\pgfqpoint{3.834429in}{2.346072in}}%
\pgfpathlineto{\pgfqpoint{3.834995in}{2.399367in}}%
\pgfpathlineto{\pgfqpoint{3.835426in}{2.365311in}}%
\pgfpathlineto{\pgfqpoint{3.835918in}{2.399162in}}%
\pgfpathlineto{\pgfqpoint{3.836275in}{2.345626in}}%
\pgfpathlineto{\pgfqpoint{3.836546in}{2.376430in}}%
\pgfpathlineto{\pgfqpoint{3.836743in}{2.346593in}}%
\pgfpathlineto{\pgfqpoint{3.837309in}{2.402922in}}%
\pgfpathlineto{\pgfqpoint{3.837506in}{2.383431in}}%
\pgfpathlineto{\pgfqpoint{3.838232in}{2.405396in}}%
\pgfpathlineto{\pgfqpoint{3.837666in}{2.349829in}}%
\pgfpathlineto{\pgfqpoint{3.838564in}{2.362711in}}%
\pgfpathlineto{\pgfqpoint{3.838589in}{2.352525in}}%
\pgfpathlineto{\pgfqpoint{3.839155in}{2.403894in}}%
\pgfpathlineto{\pgfqpoint{3.839610in}{2.400072in}}%
\pgfpathlineto{\pgfqpoint{3.840078in}{2.402788in}}%
\pgfpathlineto{\pgfqpoint{3.840029in}{2.352033in}}%
\pgfpathlineto{\pgfqpoint{3.840484in}{2.354979in}}%
\pgfpathlineto{\pgfqpoint{3.840496in}{2.351282in}}%
\pgfpathlineto{\pgfqpoint{3.841001in}{2.402906in}}%
\pgfpathlineto{\pgfqpoint{3.841444in}{2.378429in}}%
\pgfpathlineto{\pgfqpoint{3.841924in}{2.403946in}}%
\pgfpathlineto{\pgfqpoint{3.842343in}{2.354906in}}%
\pgfpathlineto{\pgfqpoint{3.842552in}{2.379587in}}%
\pgfpathlineto{\pgfqpoint{3.843266in}{2.356480in}}%
\pgfpathlineto{\pgfqpoint{3.842847in}{2.400798in}}%
\pgfpathlineto{\pgfqpoint{3.843684in}{2.369884in}}%
\pgfpathlineto{\pgfqpoint{3.844570in}{2.354024in}}%
\pgfpathlineto{\pgfqpoint{3.843770in}{2.394667in}}%
\pgfpathlineto{\pgfqpoint{3.844669in}{2.377914in}}%
\pgfpathlineto{\pgfqpoint{3.845604in}{2.396733in}}%
\pgfpathlineto{\pgfqpoint{3.845493in}{2.354172in}}%
\pgfpathlineto{\pgfqpoint{3.845776in}{2.383865in}}%
\pgfpathlineto{\pgfqpoint{3.846527in}{2.397235in}}%
\pgfpathlineto{\pgfqpoint{3.846416in}{2.353577in}}%
\pgfpathlineto{\pgfqpoint{3.846810in}{2.370349in}}%
\pgfpathlineto{\pgfqpoint{3.847487in}{2.352099in}}%
\pgfpathlineto{\pgfqpoint{3.847450in}{2.397226in}}%
\pgfpathlineto{\pgfqpoint{3.847647in}{2.385265in}}%
\pgfpathlineto{\pgfqpoint{3.848373in}{2.396622in}}%
\pgfpathlineto{\pgfqpoint{3.848410in}{2.350934in}}%
\pgfpathlineto{\pgfqpoint{3.848706in}{2.371669in}}%
\pgfpathlineto{\pgfqpoint{3.849813in}{2.351420in}}%
\pgfpathlineto{\pgfqpoint{3.849752in}{2.403117in}}%
\pgfpathlineto{\pgfqpoint{3.849826in}{2.353535in}}%
\pgfpathlineto{\pgfqpoint{3.850675in}{2.402802in}}%
\pgfpathlineto{\pgfqpoint{3.850749in}{2.348632in}}%
\pgfpathlineto{\pgfqpoint{3.850958in}{2.369974in}}%
\pgfpathlineto{\pgfqpoint{3.851672in}{2.348329in}}%
\pgfpathlineto{\pgfqpoint{3.851598in}{2.404884in}}%
\pgfpathlineto{\pgfqpoint{3.852029in}{2.369108in}}%
\pgfpathlineto{\pgfqpoint{3.852521in}{2.409432in}}%
\pgfpathlineto{\pgfqpoint{3.852139in}{2.348855in}}%
\pgfpathlineto{\pgfqpoint{3.853149in}{2.383239in}}%
\pgfpathlineto{\pgfqpoint{3.853998in}{2.348633in}}%
\pgfpathlineto{\pgfqpoint{3.853235in}{2.406923in}}%
\pgfpathlineto{\pgfqpoint{3.854281in}{2.370587in}}%
\pgfpathlineto{\pgfqpoint{3.855081in}{2.402261in}}%
\pgfpathlineto{\pgfqpoint{3.854909in}{2.346923in}}%
\pgfpathlineto{\pgfqpoint{3.855364in}{2.349794in}}%
\pgfpathlineto{\pgfqpoint{3.855376in}{2.349452in}}%
\pgfpathlineto{\pgfqpoint{3.855426in}{2.376055in}}%
\pgfpathlineto{\pgfqpoint{3.856472in}{2.408570in}}%
\pgfpathlineto{\pgfqpoint{3.855856in}{2.350264in}}%
\pgfpathlineto{\pgfqpoint{3.856533in}{2.377255in}}%
\pgfpathlineto{\pgfqpoint{3.857222in}{2.351207in}}%
\pgfpathlineto{\pgfqpoint{3.857382in}{2.408510in}}%
\pgfpathlineto{\pgfqpoint{3.857604in}{2.385487in}}%
\pgfpathlineto{\pgfqpoint{3.857616in}{2.385600in}}%
\pgfpathlineto{\pgfqpoint{3.857690in}{2.350376in}}%
\pgfpathlineto{\pgfqpoint{3.858306in}{2.408293in}}%
\pgfpathlineto{\pgfqpoint{3.858724in}{2.375412in}}%
\pgfpathlineto{\pgfqpoint{3.859696in}{2.413905in}}%
\pgfpathlineto{\pgfqpoint{3.859044in}{2.353135in}}%
\pgfpathlineto{\pgfqpoint{3.859819in}{2.360224in}}%
\pgfpathlineto{\pgfqpoint{3.860164in}{2.414028in}}%
\pgfpathlineto{\pgfqpoint{3.859967in}{2.354670in}}%
\pgfpathlineto{\pgfqpoint{3.860915in}{2.361789in}}%
\pgfpathlineto{\pgfqpoint{3.861826in}{2.352297in}}%
\pgfpathlineto{\pgfqpoint{3.861087in}{2.412591in}}%
\pgfpathlineto{\pgfqpoint{3.861961in}{2.378105in}}%
\pgfpathlineto{\pgfqpoint{3.862946in}{2.410163in}}%
\pgfpathlineto{\pgfqpoint{3.862749in}{2.354173in}}%
\pgfpathlineto{\pgfqpoint{3.863044in}{2.371509in}}%
\pgfpathlineto{\pgfqpoint{3.864066in}{2.352529in}}%
\pgfpathlineto{\pgfqpoint{3.863413in}{2.409421in}}%
\pgfpathlineto{\pgfqpoint{3.864176in}{2.359367in}}%
\pgfpathlineto{\pgfqpoint{3.864336in}{2.409803in}}%
\pgfpathlineto{\pgfqpoint{3.865001in}{2.349899in}}%
\pgfpathlineto{\pgfqpoint{3.865370in}{2.366862in}}%
\pgfpathlineto{\pgfqpoint{3.865924in}{2.342734in}}%
\pgfpathlineto{\pgfqpoint{3.866195in}{2.409404in}}%
\pgfpathlineto{\pgfqpoint{3.866478in}{2.361027in}}%
\pgfpathlineto{\pgfqpoint{3.867586in}{2.411713in}}%
\pgfpathlineto{\pgfqpoint{3.867315in}{2.341619in}}%
\pgfpathlineto{\pgfqpoint{3.867635in}{2.368080in}}%
\pgfpathlineto{\pgfqpoint{3.868238in}{2.338705in}}%
\pgfpathlineto{\pgfqpoint{3.868053in}{2.411465in}}%
\pgfpathlineto{\pgfqpoint{3.868718in}{2.361126in}}%
\pgfpathlineto{\pgfqpoint{3.869444in}{2.415076in}}%
\pgfpathlineto{\pgfqpoint{3.869161in}{2.341100in}}%
\pgfpathlineto{\pgfqpoint{3.869838in}{2.384415in}}%
\pgfpathlineto{\pgfqpoint{3.870084in}{2.342975in}}%
\pgfpathlineto{\pgfqpoint{3.869912in}{2.416185in}}%
\pgfpathlineto{\pgfqpoint{3.870785in}{2.387147in}}%
\pgfpathlineto{\pgfqpoint{3.870835in}{2.416795in}}%
\pgfpathlineto{\pgfqpoint{3.871007in}{2.346783in}}%
\pgfpathlineto{\pgfqpoint{3.871856in}{2.362279in}}%
\pgfpathlineto{\pgfqpoint{3.871881in}{2.347545in}}%
\pgfpathlineto{\pgfqpoint{3.872225in}{2.410236in}}%
\pgfpathlineto{\pgfqpoint{3.872952in}{2.364943in}}%
\pgfpathlineto{\pgfqpoint{3.873616in}{2.405941in}}%
\pgfpathlineto{\pgfqpoint{3.873272in}{2.347637in}}%
\pgfpathlineto{\pgfqpoint{3.874096in}{2.398785in}}%
\pgfpathlineto{\pgfqpoint{3.874195in}{2.350957in}}%
\pgfpathlineto{\pgfqpoint{3.874552in}{2.399827in}}%
\pgfpathlineto{\pgfqpoint{3.875253in}{2.367813in}}%
\pgfpathlineto{\pgfqpoint{3.875499in}{2.400385in}}%
\pgfpathlineto{\pgfqpoint{3.875635in}{2.356914in}}%
\pgfpathlineto{\pgfqpoint{3.876361in}{2.371422in}}%
\pgfpathlineto{\pgfqpoint{3.877025in}{2.347261in}}%
\pgfpathlineto{\pgfqpoint{3.877149in}{2.398016in}}%
\pgfpathlineto{\pgfqpoint{3.877161in}{2.398247in}}%
\pgfpathlineto{\pgfqpoint{3.877222in}{2.378664in}}%
\pgfpathlineto{\pgfqpoint{3.877493in}{2.344557in}}%
\pgfpathlineto{\pgfqpoint{3.878133in}{2.403566in}}%
\pgfpathlineto{\pgfqpoint{3.878342in}{2.364769in}}%
\pgfpathlineto{\pgfqpoint{3.878601in}{2.404486in}}%
\pgfpathlineto{\pgfqpoint{3.878896in}{2.341173in}}%
\pgfpathlineto{\pgfqpoint{3.879315in}{2.358636in}}%
\pgfpathlineto{\pgfqpoint{3.879807in}{2.340279in}}%
\pgfpathlineto{\pgfqpoint{3.879536in}{2.414560in}}%
\pgfpathlineto{\pgfqpoint{3.880385in}{2.377114in}}%
\pgfpathlineto{\pgfqpoint{3.880878in}{2.404057in}}%
\pgfpathlineto{\pgfqpoint{3.880718in}{2.341440in}}%
\pgfpathlineto{\pgfqpoint{3.881481in}{2.374070in}}%
\pgfpathlineto{\pgfqpoint{3.882601in}{2.335490in}}%
\pgfpathlineto{\pgfqpoint{3.881887in}{2.415309in}}%
\pgfpathlineto{\pgfqpoint{3.882699in}{2.346003in}}%
\pgfpathlineto{\pgfqpoint{3.882761in}{2.414636in}}%
\pgfpathlineto{\pgfqpoint{3.883081in}{2.344749in}}%
\pgfpathlineto{\pgfqpoint{3.883832in}{2.383276in}}%
\pgfpathlineto{\pgfqpoint{3.884804in}{2.329014in}}%
\pgfpathlineto{\pgfqpoint{3.884607in}{2.428687in}}%
\pgfpathlineto{\pgfqpoint{3.884927in}{2.357161in}}%
\pgfpathlineto{\pgfqpoint{3.885653in}{2.436998in}}%
\pgfpathlineto{\pgfqpoint{3.885752in}{2.308768in}}%
\pgfpathlineto{\pgfqpoint{3.886047in}{2.406849in}}%
\pgfpathlineto{\pgfqpoint{3.886109in}{2.421056in}}%
\pgfpathlineto{\pgfqpoint{3.886232in}{2.362359in}}%
\pgfpathlineto{\pgfqpoint{3.886281in}{2.291446in}}%
\pgfpathlineto{\pgfqpoint{3.887056in}{2.430157in}}%
\pgfpathlineto{\pgfqpoint{3.887364in}{2.340266in}}%
\pgfpathlineto{\pgfqpoint{3.888287in}{2.441427in}}%
\pgfpathlineto{\pgfqpoint{3.888385in}{2.303448in}}%
\pgfpathlineto{\pgfqpoint{3.888459in}{2.341924in}}%
\pgfpathlineto{\pgfqpoint{3.889444in}{2.282942in}}%
\pgfpathlineto{\pgfqpoint{3.888804in}{2.454980in}}%
\pgfpathlineto{\pgfqpoint{3.889567in}{2.339407in}}%
\pgfpathlineto{\pgfqpoint{3.890207in}{2.449787in}}%
\pgfpathlineto{\pgfqpoint{3.890490in}{2.277069in}}%
\pgfpathlineto{\pgfqpoint{3.890687in}{2.369303in}}%
\pgfpathlineto{\pgfqpoint{3.890736in}{2.450020in}}%
\pgfpathlineto{\pgfqpoint{3.891019in}{2.279859in}}%
\pgfpathlineto{\pgfqpoint{3.891832in}{2.407856in}}%
\pgfpathlineto{\pgfqpoint{3.892078in}{2.286301in}}%
\pgfpathlineto{\pgfqpoint{3.892828in}{2.431421in}}%
\pgfpathlineto{\pgfqpoint{3.893887in}{2.459009in}}%
\pgfpathlineto{\pgfqpoint{3.893653in}{2.276390in}}%
\pgfpathlineto{\pgfqpoint{3.893936in}{2.437028in}}%
\pgfpathlineto{\pgfqpoint{3.894170in}{2.279898in}}%
\pgfpathlineto{\pgfqpoint{3.894416in}{2.455859in}}%
\pgfpathlineto{\pgfqpoint{3.895352in}{2.366650in}}%
\pgfpathlineto{\pgfqpoint{3.896004in}{2.438954in}}%
\pgfpathlineto{\pgfqpoint{3.895758in}{2.294034in}}%
\pgfpathlineto{\pgfqpoint{3.896459in}{2.385279in}}%
\pgfpathlineto{\pgfqpoint{3.897333in}{2.268729in}}%
\pgfpathlineto{\pgfqpoint{3.896521in}{2.458920in}}%
\pgfpathlineto{\pgfqpoint{3.897542in}{2.399094in}}%
\pgfpathlineto{\pgfqpoint{3.898108in}{2.456378in}}%
\pgfpathlineto{\pgfqpoint{3.897862in}{2.278225in}}%
\pgfpathlineto{\pgfqpoint{3.898662in}{2.429191in}}%
\pgfpathlineto{\pgfqpoint{3.898908in}{2.287658in}}%
\pgfpathlineto{\pgfqpoint{3.899684in}{2.434440in}}%
\pgfpathlineto{\pgfqpoint{3.899807in}{2.369709in}}%
\pgfpathlineto{\pgfqpoint{3.900213in}{2.447480in}}%
\pgfpathlineto{\pgfqpoint{3.900496in}{2.292291in}}%
\pgfpathlineto{\pgfqpoint{3.900927in}{2.398656in}}%
\pgfpathlineto{\pgfqpoint{3.901013in}{2.270345in}}%
\pgfpathlineto{\pgfqpoint{3.901259in}{2.458209in}}%
\pgfpathlineto{\pgfqpoint{3.902072in}{2.291368in}}%
\pgfpathlineto{\pgfqpoint{3.902305in}{2.464916in}}%
\pgfpathlineto{\pgfqpoint{3.902588in}{2.282843in}}%
\pgfpathlineto{\pgfqpoint{3.903204in}{2.326167in}}%
\pgfpathlineto{\pgfqpoint{3.903881in}{2.445929in}}%
\pgfpathlineto{\pgfqpoint{3.904164in}{2.289360in}}%
\pgfpathlineto{\pgfqpoint{3.904250in}{2.298349in}}%
\pgfpathlineto{\pgfqpoint{3.904275in}{2.308803in}}%
\pgfpathlineto{\pgfqpoint{3.905210in}{2.269912in}}%
\pgfpathlineto{\pgfqpoint{3.905456in}{2.463957in}}%
\pgfpathlineto{\pgfqpoint{3.905493in}{2.446375in}}%
\pgfpathlineto{\pgfqpoint{3.905739in}{2.267342in}}%
\pgfpathlineto{\pgfqpoint{3.905985in}{2.461695in}}%
\pgfpathlineto{\pgfqpoint{3.906650in}{2.385732in}}%
\pgfpathlineto{\pgfqpoint{3.907561in}{2.460635in}}%
\pgfpathlineto{\pgfqpoint{3.906785in}{2.276958in}}%
\pgfpathlineto{\pgfqpoint{3.907745in}{2.410483in}}%
\pgfpathlineto{\pgfqpoint{3.908890in}{2.258503in}}%
\pgfpathlineto{\pgfqpoint{3.908607in}{2.472120in}}%
\pgfpathlineto{\pgfqpoint{3.908902in}{2.269918in}}%
\pgfpathlineto{\pgfqpoint{3.909124in}{2.476789in}}%
\pgfpathlineto{\pgfqpoint{3.909407in}{2.259810in}}%
\pgfpathlineto{\pgfqpoint{3.910096in}{2.409032in}}%
\pgfpathlineto{\pgfqpoint{3.910453in}{2.275148in}}%
\pgfpathlineto{\pgfqpoint{3.910182in}{2.468232in}}%
\pgfpathlineto{\pgfqpoint{3.911191in}{2.383589in}}%
\pgfpathlineto{\pgfqpoint{3.912275in}{2.477346in}}%
\pgfpathlineto{\pgfqpoint{3.912028in}{2.265110in}}%
\pgfpathlineto{\pgfqpoint{3.912311in}{2.447792in}}%
\pgfpathlineto{\pgfqpoint{3.913075in}{2.248720in}}%
\pgfpathlineto{\pgfqpoint{3.913321in}{2.485965in}}%
\pgfpathlineto{\pgfqpoint{3.913407in}{2.452571in}}%
\pgfpathlineto{\pgfqpoint{3.913431in}{2.427958in}}%
\pgfpathlineto{\pgfqpoint{3.913604in}{2.246346in}}%
\pgfpathlineto{\pgfqpoint{3.913850in}{2.478649in}}%
\pgfpathlineto{\pgfqpoint{3.914539in}{2.410747in}}%
\pgfpathlineto{\pgfqpoint{3.914551in}{2.412297in}}%
\pgfpathlineto{\pgfqpoint{3.914625in}{2.299094in}}%
\pgfpathlineto{\pgfqpoint{3.914650in}{2.261593in}}%
\pgfpathlineto{\pgfqpoint{3.914896in}{2.474176in}}%
\pgfpathlineto{\pgfqpoint{3.915721in}{2.293419in}}%
\pgfpathlineto{\pgfqpoint{3.916471in}{2.481783in}}%
\pgfpathlineto{\pgfqpoint{3.916742in}{2.254252in}}%
\pgfpathlineto{\pgfqpoint{3.916816in}{2.302964in}}%
\pgfpathlineto{\pgfqpoint{3.917271in}{2.245164in}}%
\pgfpathlineto{\pgfqpoint{3.916988in}{2.489910in}}%
\pgfpathlineto{\pgfqpoint{3.917899in}{2.305547in}}%
\pgfpathlineto{\pgfqpoint{3.918035in}{2.486430in}}%
\pgfpathlineto{\pgfqpoint{3.918318in}{2.251074in}}%
\pgfpathlineto{\pgfqpoint{3.919019in}{2.378833in}}%
\pgfpathlineto{\pgfqpoint{3.919893in}{2.251695in}}%
\pgfpathlineto{\pgfqpoint{3.919610in}{2.486314in}}%
\pgfpathlineto{\pgfqpoint{3.920102in}{2.390787in}}%
\pgfpathlineto{\pgfqpoint{3.921185in}{2.505096in}}%
\pgfpathlineto{\pgfqpoint{3.920939in}{2.240601in}}%
\pgfpathlineto{\pgfqpoint{3.921222in}{2.437955in}}%
\pgfpathlineto{\pgfqpoint{3.921271in}{2.474074in}}%
\pgfpathlineto{\pgfqpoint{3.921370in}{2.385326in}}%
\pgfpathlineto{\pgfqpoint{3.921456in}{2.235324in}}%
\pgfpathlineto{\pgfqpoint{3.922231in}{2.509845in}}%
\pgfpathlineto{\pgfqpoint{3.922515in}{2.251155in}}%
\pgfpathlineto{\pgfqpoint{3.922761in}{2.506614in}}%
\pgfpathlineto{\pgfqpoint{3.923548in}{2.232396in}}%
\pgfpathlineto{\pgfqpoint{3.923622in}{2.297665in}}%
\pgfpathlineto{\pgfqpoint{3.924595in}{2.218725in}}%
\pgfpathlineto{\pgfqpoint{3.924324in}{2.509649in}}%
\pgfpathlineto{\pgfqpoint{3.924718in}{2.322724in}}%
\pgfpathlineto{\pgfqpoint{3.925370in}{2.520900in}}%
\pgfpathlineto{\pgfqpoint{3.925641in}{2.209159in}}%
\pgfpathlineto{\pgfqpoint{3.925838in}{2.369102in}}%
\pgfpathlineto{\pgfqpoint{3.925899in}{2.524582in}}%
\pgfpathlineto{\pgfqpoint{3.926170in}{2.213110in}}%
\pgfpathlineto{\pgfqpoint{3.927031in}{2.467555in}}%
\pgfpathlineto{\pgfqpoint{3.928262in}{2.202435in}}%
\pgfpathlineto{\pgfqpoint{3.927991in}{2.516025in}}%
\pgfpathlineto{\pgfqpoint{3.928275in}{2.218514in}}%
\pgfpathlineto{\pgfqpoint{3.929038in}{2.539520in}}%
\pgfpathlineto{\pgfqpoint{3.929308in}{2.188236in}}%
\pgfpathlineto{\pgfqpoint{3.929395in}{2.263147in}}%
\pgfpathlineto{\pgfqpoint{3.929468in}{2.451946in}}%
\pgfpathlineto{\pgfqpoint{3.929530in}{2.442066in}}%
\pgfpathlineto{\pgfqpoint{3.929567in}{2.539626in}}%
\pgfpathlineto{\pgfqpoint{3.929838in}{2.186048in}}%
\pgfpathlineto{\pgfqpoint{3.930638in}{2.473725in}}%
\pgfpathlineto{\pgfqpoint{3.931413in}{2.203310in}}%
\pgfpathlineto{\pgfqpoint{3.931659in}{2.527480in}}%
\pgfpathlineto{\pgfqpoint{3.931733in}{2.492936in}}%
\pgfpathlineto{\pgfqpoint{3.931745in}{2.494517in}}%
\pgfpathlineto{\pgfqpoint{3.931844in}{2.396460in}}%
\pgfpathlineto{\pgfqpoint{3.932976in}{2.173250in}}%
\pgfpathlineto{\pgfqpoint{3.932705in}{2.557137in}}%
\pgfpathlineto{\pgfqpoint{3.933001in}{2.214667in}}%
\pgfpathlineto{\pgfqpoint{3.933234in}{2.558610in}}%
\pgfpathlineto{\pgfqpoint{3.933505in}{2.171784in}}%
\pgfpathlineto{\pgfqpoint{3.934121in}{2.275298in}}%
\pgfpathlineto{\pgfqpoint{3.934281in}{2.541639in}}%
\pgfpathlineto{\pgfqpoint{3.935081in}{2.185260in}}%
\pgfpathlineto{\pgfqpoint{3.935364in}{2.459688in}}%
\pgfpathlineto{\pgfqpoint{3.936127in}{2.164581in}}%
\pgfpathlineto{\pgfqpoint{3.935844in}{2.547505in}}%
\pgfpathlineto{\pgfqpoint{3.936348in}{2.503131in}}%
\pgfpathlineto{\pgfqpoint{3.937419in}{2.583330in}}%
\pgfpathlineto{\pgfqpoint{3.937173in}{2.152143in}}%
\pgfpathlineto{\pgfqpoint{3.937444in}{2.513413in}}%
\pgfpathlineto{\pgfqpoint{3.937690in}{2.159085in}}%
\pgfpathlineto{\pgfqpoint{3.937936in}{2.568347in}}%
\pgfpathlineto{\pgfqpoint{3.938601in}{2.378874in}}%
\pgfpathlineto{\pgfqpoint{3.939511in}{2.569033in}}%
\pgfpathlineto{\pgfqpoint{3.939265in}{2.163050in}}%
\pgfpathlineto{\pgfqpoint{3.939696in}{2.392311in}}%
\pgfpathlineto{\pgfqpoint{3.940828in}{2.135152in}}%
\pgfpathlineto{\pgfqpoint{3.940558in}{2.604974in}}%
\pgfpathlineto{\pgfqpoint{3.940853in}{2.190118in}}%
\pgfpathlineto{\pgfqpoint{3.941087in}{2.605163in}}%
\pgfpathlineto{\pgfqpoint{3.941358in}{2.136157in}}%
\pgfpathlineto{\pgfqpoint{3.941973in}{2.249026in}}%
\pgfpathlineto{\pgfqpoint{3.943179in}{2.595299in}}%
\pgfpathlineto{\pgfqpoint{3.942921in}{2.153896in}}%
\pgfpathlineto{\pgfqpoint{3.943204in}{2.510183in}}%
\pgfpathlineto{\pgfqpoint{3.943967in}{2.133291in}}%
\pgfpathlineto{\pgfqpoint{3.944225in}{2.621805in}}%
\pgfpathlineto{\pgfqpoint{3.944299in}{2.548147in}}%
\pgfpathlineto{\pgfqpoint{3.944496in}{2.124224in}}%
\pgfpathlineto{\pgfqpoint{3.945271in}{2.625667in}}%
\pgfpathlineto{\pgfqpoint{3.945665in}{2.345997in}}%
\pgfpathlineto{\pgfqpoint{3.946318in}{2.619687in}}%
\pgfpathlineto{\pgfqpoint{3.946588in}{2.134174in}}%
\pgfpathlineto{\pgfqpoint{3.946773in}{2.373625in}}%
\pgfpathlineto{\pgfqpoint{3.947634in}{2.116638in}}%
\pgfpathlineto{\pgfqpoint{3.947364in}{2.620853in}}%
\pgfpathlineto{\pgfqpoint{3.947844in}{2.385232in}}%
\pgfpathlineto{\pgfqpoint{3.948410in}{2.645544in}}%
\pgfpathlineto{\pgfqpoint{3.948681in}{2.106403in}}%
\pgfpathlineto{\pgfqpoint{3.948964in}{2.527218in}}%
\pgfpathlineto{\pgfqpoint{3.949198in}{2.109032in}}%
\pgfpathlineto{\pgfqpoint{3.949456in}{2.632050in}}%
\pgfpathlineto{\pgfqpoint{3.950121in}{2.415215in}}%
\pgfpathlineto{\pgfqpoint{3.950133in}{2.415968in}}%
\pgfpathlineto{\pgfqpoint{3.950145in}{2.394647in}}%
\pgfpathlineto{\pgfqpoint{3.951290in}{2.097335in}}%
\pgfpathlineto{\pgfqpoint{3.951019in}{2.623369in}}%
\pgfpathlineto{\pgfqpoint{3.951302in}{2.118881in}}%
\pgfpathlineto{\pgfqpoint{3.952065in}{2.643988in}}%
\pgfpathlineto{\pgfqpoint{3.952336in}{2.090246in}}%
\pgfpathlineto{\pgfqpoint{3.952422in}{2.212143in}}%
\pgfpathlineto{\pgfqpoint{3.952853in}{2.090101in}}%
\pgfpathlineto{\pgfqpoint{3.952582in}{2.643403in}}%
\pgfpathlineto{\pgfqpoint{3.953001in}{2.455985in}}%
\pgfpathlineto{\pgfqpoint{3.953111in}{2.636624in}}%
\pgfpathlineto{\pgfqpoint{3.953382in}{2.092094in}}%
\pgfpathlineto{\pgfqpoint{3.954071in}{2.438898in}}%
\pgfpathlineto{\pgfqpoint{3.954945in}{2.085683in}}%
\pgfpathlineto{\pgfqpoint{3.954674in}{2.635120in}}%
\pgfpathlineto{\pgfqpoint{3.955167in}{2.484234in}}%
\pgfpathlineto{\pgfqpoint{3.956237in}{2.638500in}}%
\pgfpathlineto{\pgfqpoint{3.955991in}{2.075793in}}%
\pgfpathlineto{\pgfqpoint{3.956274in}{2.547851in}}%
\pgfpathlineto{\pgfqpoint{3.956508in}{2.069707in}}%
\pgfpathlineto{\pgfqpoint{3.956754in}{2.635793in}}%
\pgfpathlineto{\pgfqpoint{3.957271in}{2.612247in}}%
\pgfpathlineto{\pgfqpoint{3.957284in}{2.623944in}}%
\pgfpathlineto{\pgfqpoint{3.957554in}{2.067476in}}%
\pgfpathlineto{\pgfqpoint{3.958010in}{2.276139in}}%
\pgfpathlineto{\pgfqpoint{3.959117in}{2.072133in}}%
\pgfpathlineto{\pgfqpoint{3.958317in}{2.612265in}}%
\pgfpathlineto{\pgfqpoint{3.959154in}{2.208036in}}%
\pgfpathlineto{\pgfqpoint{3.959364in}{2.603979in}}%
\pgfpathlineto{\pgfqpoint{3.960164in}{2.060373in}}%
\pgfpathlineto{\pgfqpoint{3.960324in}{2.435088in}}%
\pgfpathlineto{\pgfqpoint{3.961210in}{2.050798in}}%
\pgfpathlineto{\pgfqpoint{3.960410in}{2.603097in}}%
\pgfpathlineto{\pgfqpoint{3.961407in}{2.429897in}}%
\pgfpathlineto{\pgfqpoint{3.962514in}{2.603129in}}%
\pgfpathlineto{\pgfqpoint{3.961727in}{2.055207in}}%
\pgfpathlineto{\pgfqpoint{3.962551in}{2.578824in}}%
\pgfpathlineto{\pgfqpoint{3.963290in}{2.060529in}}%
\pgfpathlineto{\pgfqpoint{3.963561in}{2.616064in}}%
\pgfpathlineto{\pgfqpoint{3.963942in}{2.310732in}}%
\pgfpathlineto{\pgfqpoint{3.965124in}{2.629582in}}%
\pgfpathlineto{\pgfqpoint{3.964853in}{2.060902in}}%
\pgfpathlineto{\pgfqpoint{3.965148in}{2.585172in}}%
\pgfpathlineto{\pgfqpoint{3.965259in}{2.463576in}}%
\pgfpathlineto{\pgfqpoint{3.965899in}{2.060941in}}%
\pgfpathlineto{\pgfqpoint{3.965641in}{2.626861in}}%
\pgfpathlineto{\pgfqpoint{3.966477in}{2.181303in}}%
\pgfpathlineto{\pgfqpoint{3.966490in}{2.181293in}}%
\pgfpathlineto{\pgfqpoint{3.966687in}{2.621198in}}%
\pgfpathlineto{\pgfqpoint{3.966945in}{2.062552in}}%
\pgfpathlineto{\pgfqpoint{3.967462in}{2.064156in}}%
\pgfpathlineto{\pgfqpoint{3.967474in}{2.063886in}}%
\pgfpathlineto{\pgfqpoint{3.967991in}{2.052062in}}%
\pgfpathlineto{\pgfqpoint{3.968779in}{2.634378in}}%
\pgfpathlineto{\pgfqpoint{3.969554in}{2.048148in}}%
\pgfpathlineto{\pgfqpoint{3.969813in}{2.637441in}}%
\pgfpathlineto{\pgfqpoint{3.969936in}{2.517128in}}%
\pgfpathlineto{\pgfqpoint{3.971117in}{2.050712in}}%
\pgfpathlineto{\pgfqpoint{3.970859in}{2.637459in}}%
\pgfpathlineto{\pgfqpoint{3.971228in}{2.226468in}}%
\pgfpathlineto{\pgfqpoint{3.972422in}{2.642770in}}%
\pgfpathlineto{\pgfqpoint{3.972164in}{2.039245in}}%
\pgfpathlineto{\pgfqpoint{3.972434in}{2.630499in}}%
\pgfpathlineto{\pgfqpoint{3.973210in}{2.030174in}}%
\pgfpathlineto{\pgfqpoint{3.973468in}{2.645422in}}%
\pgfpathlineto{\pgfqpoint{3.973542in}{2.624181in}}%
\pgfpathlineto{\pgfqpoint{3.973727in}{2.034332in}}%
\pgfpathlineto{\pgfqpoint{3.974514in}{2.647972in}}%
\pgfpathlineto{\pgfqpoint{3.974982in}{2.425688in}}%
\pgfpathlineto{\pgfqpoint{3.976077in}{2.652860in}}%
\pgfpathlineto{\pgfqpoint{3.975819in}{2.032793in}}%
\pgfpathlineto{\pgfqpoint{3.976114in}{2.557469in}}%
\pgfpathlineto{\pgfqpoint{3.976594in}{2.656452in}}%
\pgfpathlineto{\pgfqpoint{3.976853in}{2.027752in}}%
\pgfpathlineto{\pgfqpoint{3.977222in}{2.609039in}}%
\pgfpathlineto{\pgfqpoint{3.977899in}{2.025797in}}%
\pgfpathlineto{\pgfqpoint{3.978157in}{2.653459in}}%
\pgfpathlineto{\pgfqpoint{3.978490in}{2.109977in}}%
\pgfpathlineto{\pgfqpoint{3.978637in}{2.480541in}}%
\pgfpathlineto{\pgfqpoint{3.978674in}{2.654098in}}%
\pgfpathlineto{\pgfqpoint{3.979462in}{2.038053in}}%
\pgfpathlineto{\pgfqpoint{3.979757in}{2.563111in}}%
\pgfpathlineto{\pgfqpoint{3.980237in}{2.660615in}}%
\pgfpathlineto{\pgfqpoint{3.979979in}{2.035306in}}%
\pgfpathlineto{\pgfqpoint{3.980484in}{2.075331in}}%
\pgfpathlineto{\pgfqpoint{3.981542in}{2.034481in}}%
\pgfpathlineto{\pgfqpoint{3.980754in}{2.661333in}}%
\pgfpathlineto{\pgfqpoint{3.981579in}{2.105292in}}%
\pgfpathlineto{\pgfqpoint{3.981616in}{2.083462in}}%
\pgfpathlineto{\pgfqpoint{3.981665in}{2.233097in}}%
\pgfpathlineto{\pgfqpoint{3.982317in}{2.661875in}}%
\pgfpathlineto{\pgfqpoint{3.982059in}{2.042889in}}%
\pgfpathlineto{\pgfqpoint{3.982847in}{2.654167in}}%
\pgfpathlineto{\pgfqpoint{3.983105in}{2.043704in}}%
\pgfpathlineto{\pgfqpoint{3.983880in}{2.665181in}}%
\pgfpathlineto{\pgfqpoint{3.983954in}{2.623514in}}%
\pgfpathlineto{\pgfqpoint{3.984397in}{2.666453in}}%
\pgfpathlineto{\pgfqpoint{3.984151in}{2.057467in}}%
\pgfpathlineto{\pgfqpoint{3.984668in}{2.059986in}}%
\pgfpathlineto{\pgfqpoint{3.985444in}{2.673697in}}%
\pgfpathlineto{\pgfqpoint{3.985259in}{2.050969in}}%
\pgfpathlineto{\pgfqpoint{3.985764in}{2.072950in}}%
\pgfpathlineto{\pgfqpoint{3.985776in}{2.053001in}}%
\pgfpathlineto{\pgfqpoint{3.985960in}{2.670781in}}%
\pgfpathlineto{\pgfqpoint{3.986539in}{2.587202in}}%
\pgfpathlineto{\pgfqpoint{3.987007in}{2.672571in}}%
\pgfpathlineto{\pgfqpoint{3.987339in}{2.044130in}}%
\pgfpathlineto{\pgfqpoint{3.987659in}{2.610759in}}%
\pgfpathlineto{\pgfqpoint{3.988902in}{2.030050in}}%
\pgfpathlineto{\pgfqpoint{3.988040in}{2.663653in}}%
\pgfpathlineto{\pgfqpoint{3.988927in}{2.080991in}}%
\pgfpathlineto{\pgfqpoint{3.989087in}{2.664450in}}%
\pgfpathlineto{\pgfqpoint{3.989936in}{2.028483in}}%
\pgfpathlineto{\pgfqpoint{3.990157in}{2.582943in}}%
\pgfpathlineto{\pgfqpoint{3.990982in}{2.029520in}}%
\pgfpathlineto{\pgfqpoint{3.990231in}{2.663559in}}%
\pgfpathlineto{\pgfqpoint{3.991142in}{2.589956in}}%
\pgfpathlineto{\pgfqpoint{3.991782in}{2.666188in}}%
\pgfpathlineto{\pgfqpoint{3.991499in}{2.030399in}}%
\pgfpathlineto{\pgfqpoint{3.992250in}{2.574444in}}%
\pgfpathlineto{\pgfqpoint{3.992299in}{2.665153in}}%
\pgfpathlineto{\pgfqpoint{3.993062in}{2.030374in}}%
\pgfpathlineto{\pgfqpoint{3.993382in}{2.625077in}}%
\pgfpathlineto{\pgfqpoint{3.993579in}{2.028050in}}%
\pgfpathlineto{\pgfqpoint{3.994379in}{2.660613in}}%
\pgfpathlineto{\pgfqpoint{3.994662in}{2.155205in}}%
\pgfpathlineto{\pgfqpoint{3.994896in}{2.656284in}}%
\pgfpathlineto{\pgfqpoint{3.995659in}{2.037469in}}%
\pgfpathlineto{\pgfqpoint{3.995868in}{2.559448in}}%
\pgfpathlineto{\pgfqpoint{3.995942in}{2.653643in}}%
\pgfpathlineto{\pgfqpoint{3.996040in}{2.394459in}}%
\pgfpathlineto{\pgfqpoint{3.996176in}{2.034249in}}%
\pgfpathlineto{\pgfqpoint{3.996976in}{2.660283in}}%
\pgfpathlineto{\pgfqpoint{3.997210in}{2.041273in}}%
\pgfpathlineto{\pgfqpoint{3.997739in}{2.028251in}}%
\pgfpathlineto{\pgfqpoint{3.997493in}{2.653734in}}%
\pgfpathlineto{\pgfqpoint{3.997862in}{2.342241in}}%
\pgfpathlineto{\pgfqpoint{3.998010in}{2.650206in}}%
\pgfpathlineto{\pgfqpoint{3.998256in}{2.029986in}}%
\pgfpathlineto{\pgfqpoint{3.999007in}{2.569507in}}%
\pgfpathlineto{\pgfqpoint{3.999056in}{2.650114in}}%
\pgfpathlineto{\pgfqpoint{3.999302in}{2.034435in}}%
\pgfpathlineto{\pgfqpoint{4.000127in}{2.629414in}}%
\pgfpathlineto{\pgfqpoint{4.000336in}{2.036542in}}%
\pgfpathlineto{\pgfqpoint{4.000607in}{2.647043in}}%
\pgfpathlineto{\pgfqpoint{4.001493in}{2.292933in}}%
\pgfpathlineto{\pgfqpoint{4.002170in}{2.649350in}}%
\pgfpathlineto{\pgfqpoint{4.001899in}{2.043536in}}%
\pgfpathlineto{\pgfqpoint{4.002674in}{2.624009in}}%
\pgfpathlineto{\pgfqpoint{4.002699in}{2.642732in}}%
\pgfpathlineto{\pgfqpoint{4.002945in}{2.060718in}}%
\pgfpathlineto{\pgfqpoint{4.003425in}{2.133086in}}%
\pgfpathlineto{\pgfqpoint{4.003462in}{2.064454in}}%
\pgfpathlineto{\pgfqpoint{4.003745in}{2.635429in}}%
\pgfpathlineto{\pgfqpoint{4.004520in}{2.112149in}}%
\pgfpathlineto{\pgfqpoint{4.004767in}{2.626496in}}%
\pgfpathlineto{\pgfqpoint{4.005542in}{2.067898in}}%
\pgfpathlineto{\pgfqpoint{4.005776in}{2.585344in}}%
\pgfpathlineto{\pgfqpoint{4.006330in}{2.632123in}}%
\pgfpathlineto{\pgfqpoint{4.006059in}{2.073878in}}%
\pgfpathlineto{\pgfqpoint{4.006883in}{2.597903in}}%
\pgfpathlineto{\pgfqpoint{4.008139in}{2.066987in}}%
\pgfpathlineto{\pgfqpoint{4.008287in}{2.439009in}}%
\pgfpathlineto{\pgfqpoint{4.008939in}{2.642713in}}%
\pgfpathlineto{\pgfqpoint{4.008656in}{2.088624in}}%
\pgfpathlineto{\pgfqpoint{4.009419in}{2.565065in}}%
\pgfpathlineto{\pgfqpoint{4.009468in}{2.621192in}}%
\pgfpathlineto{\pgfqpoint{4.010219in}{2.103977in}}%
\pgfpathlineto{\pgfqpoint{4.010527in}{2.570198in}}%
\pgfpathlineto{\pgfqpoint{4.010563in}{2.574257in}}%
\pgfpathlineto{\pgfqpoint{4.010600in}{2.476987in}}%
\pgfpathlineto{\pgfqpoint{4.010748in}{2.098885in}}%
\pgfpathlineto{\pgfqpoint{4.011031in}{2.626749in}}%
\pgfpathlineto{\pgfqpoint{4.011782in}{2.129436in}}%
\pgfpathlineto{\pgfqpoint{4.013099in}{2.602303in}}%
\pgfpathlineto{\pgfqpoint{4.012816in}{2.118122in}}%
\pgfpathlineto{\pgfqpoint{4.013222in}{2.367592in}}%
\pgfpathlineto{\pgfqpoint{4.013333in}{2.169539in}}%
\pgfpathlineto{\pgfqpoint{4.013603in}{2.612879in}}%
\pgfpathlineto{\pgfqpoint{4.014366in}{2.199638in}}%
\pgfpathlineto{\pgfqpoint{4.015400in}{2.155987in}}%
\pgfpathlineto{\pgfqpoint{4.015228in}{2.561563in}}%
\pgfpathlineto{\pgfqpoint{4.015437in}{2.247636in}}%
\pgfpathlineto{\pgfqpoint{4.015708in}{2.569491in}}%
\pgfpathlineto{\pgfqpoint{4.016508in}{2.184815in}}%
\pgfpathlineto{\pgfqpoint{4.017813in}{2.551443in}}%
\pgfpathlineto{\pgfqpoint{4.017850in}{2.479023in}}%
\pgfpathlineto{\pgfqpoint{4.017973in}{2.238184in}}%
\pgfpathlineto{\pgfqpoint{4.018317in}{2.522912in}}%
\pgfpathlineto{\pgfqpoint{4.019019in}{2.263770in}}%
\pgfpathlineto{\pgfqpoint{4.019043in}{2.233428in}}%
\pgfpathlineto{\pgfqpoint{4.019893in}{2.491063in}}%
\pgfpathlineto{\pgfqpoint{4.020102in}{2.282861in}}%
\pgfpathlineto{\pgfqpoint{4.020397in}{2.485772in}}%
\pgfpathlineto{\pgfqpoint{4.021185in}{2.282500in}}%
\pgfpathlineto{\pgfqpoint{4.021271in}{2.347381in}}%
\pgfpathlineto{\pgfqpoint{4.021456in}{2.494595in}}%
\pgfpathlineto{\pgfqpoint{4.021628in}{2.282570in}}%
\pgfpathlineto{\pgfqpoint{4.022502in}{2.465506in}}%
\pgfpathlineto{\pgfqpoint{4.022662in}{2.310543in}}%
\pgfpathlineto{\pgfqpoint{4.023733in}{2.329247in}}%
\pgfpathlineto{\pgfqpoint{4.024582in}{2.446597in}}%
\pgfpathlineto{\pgfqpoint{4.024853in}{2.342951in}}%
\pgfpathlineto{\pgfqpoint{4.024939in}{2.323815in}}%
\pgfpathlineto{\pgfqpoint{4.025628in}{2.414115in}}%
\pgfpathlineto{\pgfqpoint{4.025973in}{2.339979in}}%
\pgfpathlineto{\pgfqpoint{4.026010in}{2.336454in}}%
\pgfpathlineto{\pgfqpoint{4.026083in}{2.383700in}}%
\pgfpathlineto{\pgfqpoint{4.026133in}{2.378755in}}%
\pgfpathlineto{\pgfqpoint{4.027056in}{2.319464in}}%
\pgfpathlineto{\pgfqpoint{4.027314in}{2.421452in}}%
\pgfpathlineto{\pgfqpoint{4.028090in}{2.320439in}}%
\pgfpathlineto{\pgfqpoint{4.028656in}{2.366611in}}%
\pgfpathlineto{\pgfqpoint{4.029357in}{2.433022in}}%
\pgfpathlineto{\pgfqpoint{4.029099in}{2.315019in}}%
\pgfpathlineto{\pgfqpoint{4.029751in}{2.349293in}}%
\pgfpathlineto{\pgfqpoint{4.030453in}{2.427592in}}%
\pgfpathlineto{\pgfqpoint{4.030170in}{2.332391in}}%
\pgfpathlineto{\pgfqpoint{4.030970in}{2.394578in}}%
\pgfpathlineto{\pgfqpoint{4.031708in}{2.281304in}}%
\pgfpathlineto{\pgfqpoint{4.031499in}{2.486070in}}%
\pgfpathlineto{\pgfqpoint{4.032040in}{2.429262in}}%
\pgfpathlineto{\pgfqpoint{4.032926in}{2.472228in}}%
\pgfpathlineto{\pgfqpoint{4.032742in}{2.286880in}}%
\pgfpathlineto{\pgfqpoint{4.033111in}{2.415400in}}%
\pgfpathlineto{\pgfqpoint{4.033271in}{2.210351in}}%
\pgfpathlineto{\pgfqpoint{4.033591in}{2.515063in}}%
\pgfpathlineto{\pgfqpoint{4.034293in}{2.337381in}}%
\pgfpathlineto{\pgfqpoint{4.034748in}{2.255587in}}%
\pgfpathlineto{\pgfqpoint{4.034526in}{2.437688in}}%
\pgfpathlineto{\pgfqpoint{4.034969in}{2.397521in}}%
\pgfpathlineto{\pgfqpoint{4.035622in}{2.497211in}}%
\pgfpathlineto{\pgfqpoint{4.035413in}{2.269634in}}%
\pgfpathlineto{\pgfqpoint{4.036053in}{2.369494in}}%
\pgfpathlineto{\pgfqpoint{4.036853in}{2.209692in}}%
\pgfpathlineto{\pgfqpoint{4.036508in}{2.470425in}}%
\pgfpathlineto{\pgfqpoint{4.037099in}{2.458248in}}%
\pgfpathlineto{\pgfqpoint{4.037148in}{2.513456in}}%
\pgfpathlineto{\pgfqpoint{4.037382in}{2.230034in}}%
\pgfpathlineto{\pgfqpoint{4.037936in}{2.356362in}}%
\pgfpathlineto{\pgfqpoint{4.038908in}{2.217220in}}%
\pgfpathlineto{\pgfqpoint{4.038600in}{2.486535in}}%
\pgfpathlineto{\pgfqpoint{4.039019in}{2.340790in}}%
\pgfpathlineto{\pgfqpoint{4.039831in}{2.553811in}}%
\pgfpathlineto{\pgfqpoint{4.039523in}{2.256151in}}%
\pgfpathlineto{\pgfqpoint{4.040114in}{2.297594in}}%
\pgfpathlineto{\pgfqpoint{4.040631in}{2.462747in}}%
\pgfpathlineto{\pgfqpoint{4.041049in}{2.227960in}}%
\pgfpathlineto{\pgfqpoint{4.041579in}{2.180069in}}%
\pgfpathlineto{\pgfqpoint{4.041222in}{2.490681in}}%
\pgfpathlineto{\pgfqpoint{4.041825in}{2.441499in}}%
\pgfpathlineto{\pgfqpoint{4.041911in}{2.536490in}}%
\pgfpathlineto{\pgfqpoint{4.042329in}{2.290151in}}%
\pgfpathlineto{\pgfqpoint{4.042859in}{2.387730in}}%
\pgfpathlineto{\pgfqpoint{4.043696in}{2.229518in}}%
\pgfpathlineto{\pgfqpoint{4.043376in}{2.481953in}}%
\pgfpathlineto{\pgfqpoint{4.043929in}{2.450706in}}%
\pgfpathlineto{\pgfqpoint{4.044016in}{2.514715in}}%
\pgfpathlineto{\pgfqpoint{4.044311in}{2.279123in}}%
\pgfpathlineto{\pgfqpoint{4.044988in}{2.367067in}}%
\pgfpathlineto{\pgfqpoint{4.045209in}{2.286906in}}%
\pgfpathlineto{\pgfqpoint{4.045320in}{2.420747in}}%
\pgfpathlineto{\pgfqpoint{4.046108in}{2.514047in}}%
\pgfpathlineto{\pgfqpoint{4.045763in}{2.217037in}}%
\pgfpathlineto{\pgfqpoint{4.046366in}{2.345838in}}%
\pgfpathlineto{\pgfqpoint{4.046465in}{2.284067in}}%
\pgfpathlineto{\pgfqpoint{4.046637in}{2.463319in}}%
\pgfpathlineto{\pgfqpoint{4.047400in}{2.411181in}}%
\pgfpathlineto{\pgfqpoint{4.048176in}{2.503298in}}%
\pgfpathlineto{\pgfqpoint{4.047843in}{2.247663in}}%
\pgfpathlineto{\pgfqpoint{4.048373in}{2.265111in}}%
\pgfpathlineto{\pgfqpoint{4.048385in}{2.256507in}}%
\pgfpathlineto{\pgfqpoint{4.048693in}{2.506656in}}%
\pgfpathlineto{\pgfqpoint{4.049246in}{2.407825in}}%
\pgfpathlineto{\pgfqpoint{4.050256in}{2.467709in}}%
\pgfpathlineto{\pgfqpoint{4.049911in}{2.263340in}}%
\pgfpathlineto{\pgfqpoint{4.050329in}{2.396130in}}%
\pgfpathlineto{\pgfqpoint{4.050416in}{2.254813in}}%
\pgfpathlineto{\pgfqpoint{4.050785in}{2.511660in}}%
\pgfpathlineto{\pgfqpoint{4.051449in}{2.372926in}}%
\pgfpathlineto{\pgfqpoint{4.052569in}{2.250226in}}%
\pgfpathlineto{\pgfqpoint{4.052336in}{2.451839in}}%
\pgfpathlineto{\pgfqpoint{4.052619in}{2.324564in}}%
\pgfpathlineto{\pgfqpoint{4.052877in}{2.530060in}}%
\pgfpathlineto{\pgfqpoint{4.053209in}{2.254410in}}%
\pgfpathlineto{\pgfqpoint{4.053788in}{2.374745in}}%
\pgfpathlineto{\pgfqpoint{4.054588in}{2.221816in}}%
\pgfpathlineto{\pgfqpoint{4.054440in}{2.462067in}}%
\pgfpathlineto{\pgfqpoint{4.054809in}{2.446069in}}%
\pgfpathlineto{\pgfqpoint{4.054969in}{2.513743in}}%
\pgfpathlineto{\pgfqpoint{4.055179in}{2.272015in}}%
\pgfpathlineto{\pgfqpoint{4.055757in}{2.355761in}}%
\pgfpathlineto{\pgfqpoint{4.056717in}{2.265104in}}%
\pgfpathlineto{\pgfqpoint{4.056212in}{2.419442in}}%
\pgfpathlineto{\pgfqpoint{4.056803in}{2.392351in}}%
\pgfpathlineto{\pgfqpoint{4.057591in}{2.527717in}}%
\pgfpathlineto{\pgfqpoint{4.057443in}{2.287620in}}%
\pgfpathlineto{\pgfqpoint{4.057812in}{2.365911in}}%
\pgfpathlineto{\pgfqpoint{4.057923in}{2.301334in}}%
\pgfpathlineto{\pgfqpoint{4.058083in}{2.420676in}}%
\pgfpathlineto{\pgfqpoint{4.058354in}{2.415724in}}%
\pgfpathlineto{\pgfqpoint{4.059117in}{2.489473in}}%
\pgfpathlineto{\pgfqpoint{4.058834in}{2.243301in}}%
\pgfpathlineto{\pgfqpoint{4.059252in}{2.348499in}}%
\pgfpathlineto{\pgfqpoint{4.059289in}{2.220549in}}%
\pgfpathlineto{\pgfqpoint{4.059683in}{2.481047in}}%
\pgfpathlineto{\pgfqpoint{4.060348in}{2.373098in}}%
\pgfpathlineto{\pgfqpoint{4.060889in}{2.274777in}}%
\pgfpathlineto{\pgfqpoint{4.061136in}{2.462735in}}%
\pgfpathlineto{\pgfqpoint{4.061726in}{2.498172in}}%
\pgfpathlineto{\pgfqpoint{4.061505in}{2.264989in}}%
\pgfpathlineto{\pgfqpoint{4.061874in}{2.379566in}}%
\pgfpathlineto{\pgfqpoint{4.063006in}{2.261440in}}%
\pgfpathlineto{\pgfqpoint{4.062797in}{2.439869in}}%
\pgfpathlineto{\pgfqpoint{4.063031in}{2.298728in}}%
\pgfpathlineto{\pgfqpoint{4.063782in}{2.470830in}}%
\pgfpathlineto{\pgfqpoint{4.063474in}{2.237287in}}%
\pgfpathlineto{\pgfqpoint{4.064151in}{2.359101in}}%
\pgfpathlineto{\pgfqpoint{4.064754in}{2.430974in}}%
\pgfpathlineto{\pgfqpoint{4.065000in}{2.302544in}}%
\pgfpathlineto{\pgfqpoint{4.065271in}{2.384125in}}%
\pgfpathlineto{\pgfqpoint{4.065652in}{2.304775in}}%
\pgfpathlineto{\pgfqpoint{4.065923in}{2.490069in}}%
\pgfpathlineto{\pgfqpoint{4.066391in}{2.369112in}}%
\pgfpathlineto{\pgfqpoint{4.067302in}{2.467745in}}%
\pgfpathlineto{\pgfqpoint{4.067179in}{2.275850in}}%
\pgfpathlineto{\pgfqpoint{4.067511in}{2.406605in}}%
\pgfpathlineto{\pgfqpoint{4.067659in}{2.285218in}}%
\pgfpathlineto{\pgfqpoint{4.068016in}{2.452142in}}%
\pgfpathlineto{\pgfqpoint{4.068680in}{2.382451in}}%
\pgfpathlineto{\pgfqpoint{4.069542in}{2.445463in}}%
\pgfpathlineto{\pgfqpoint{4.069246in}{2.297587in}}%
\pgfpathlineto{\pgfqpoint{4.069751in}{2.355467in}}%
\pgfpathlineto{\pgfqpoint{4.069800in}{2.314167in}}%
\pgfpathlineto{\pgfqpoint{4.070108in}{2.477034in}}%
\pgfpathlineto{\pgfqpoint{4.070859in}{2.349902in}}%
\pgfpathlineto{\pgfqpoint{4.071499in}{2.440498in}}%
\pgfpathlineto{\pgfqpoint{4.071831in}{2.283515in}}%
\pgfpathlineto{\pgfqpoint{4.071929in}{2.291611in}}%
\pgfpathlineto{\pgfqpoint{4.072102in}{2.457506in}}%
\pgfpathlineto{\pgfqpoint{4.073259in}{2.376861in}}%
\pgfpathlineto{\pgfqpoint{4.073320in}{2.304449in}}%
\pgfpathlineto{\pgfqpoint{4.074256in}{2.449175in}}%
\pgfpathlineto{\pgfqpoint{4.074342in}{2.404058in}}%
\pgfpathlineto{\pgfqpoint{4.074859in}{2.442631in}}%
\pgfpathlineto{\pgfqpoint{4.074440in}{2.300432in}}%
\pgfpathlineto{\pgfqpoint{4.075388in}{2.352493in}}%
\pgfpathlineto{\pgfqpoint{4.075991in}{2.285601in}}%
\pgfpathlineto{\pgfqpoint{4.076262in}{2.457193in}}%
\pgfpathlineto{\pgfqpoint{4.076422in}{2.398274in}}%
\pgfpathlineto{\pgfqpoint{4.076459in}{2.441493in}}%
\pgfpathlineto{\pgfqpoint{4.076557in}{2.276679in}}%
\pgfpathlineto{\pgfqpoint{4.077480in}{2.332219in}}%
\pgfpathlineto{\pgfqpoint{4.078034in}{2.312988in}}%
\pgfpathlineto{\pgfqpoint{4.078354in}{2.441053in}}%
\pgfpathlineto{\pgfqpoint{4.078428in}{2.427720in}}%
\pgfpathlineto{\pgfqpoint{4.079055in}{2.483532in}}%
\pgfpathlineto{\pgfqpoint{4.078809in}{2.307006in}}%
\pgfpathlineto{\pgfqpoint{4.079351in}{2.312211in}}%
\pgfpathlineto{\pgfqpoint{4.080225in}{2.270857in}}%
\pgfpathlineto{\pgfqpoint{4.080003in}{2.424053in}}%
\pgfpathlineto{\pgfqpoint{4.080335in}{2.361367in}}%
\pgfpathlineto{\pgfqpoint{4.080446in}{2.505263in}}%
\pgfpathlineto{\pgfqpoint{4.080779in}{2.255760in}}%
\pgfpathlineto{\pgfqpoint{4.081443in}{2.378141in}}%
\pgfpathlineto{\pgfqpoint{4.082219in}{2.285275in}}%
\pgfpathlineto{\pgfqpoint{4.081837in}{2.466324in}}%
\pgfpathlineto{\pgfqpoint{4.082502in}{2.432837in}}%
\pgfpathlineto{\pgfqpoint{4.082612in}{2.501353in}}%
\pgfpathlineto{\pgfqpoint{4.082895in}{2.276960in}}%
\pgfpathlineto{\pgfqpoint{4.083548in}{2.341901in}}%
\pgfpathlineto{\pgfqpoint{4.084360in}{2.235981in}}%
\pgfpathlineto{\pgfqpoint{4.084028in}{2.476795in}}%
\pgfpathlineto{\pgfqpoint{4.084569in}{2.447321in}}%
\pgfpathlineto{\pgfqpoint{4.084680in}{2.527233in}}%
\pgfpathlineto{\pgfqpoint{4.084926in}{2.276687in}}%
\pgfpathlineto{\pgfqpoint{4.085591in}{2.354864in}}%
\pgfpathlineto{\pgfqpoint{4.085763in}{2.281590in}}%
\pgfpathlineto{\pgfqpoint{4.086145in}{2.490594in}}%
\pgfpathlineto{\pgfqpoint{4.086625in}{2.380310in}}%
\pgfpathlineto{\pgfqpoint{4.087363in}{2.505114in}}%
\pgfpathlineto{\pgfqpoint{4.087166in}{2.273510in}}%
\pgfpathlineto{\pgfqpoint{4.087708in}{2.292460in}}%
\pgfpathlineto{\pgfqpoint{4.088779in}{2.524775in}}%
\pgfpathlineto{\pgfqpoint{4.088532in}{2.224175in}}%
\pgfpathlineto{\pgfqpoint{4.088975in}{2.417452in}}%
\pgfpathlineto{\pgfqpoint{4.089099in}{2.280121in}}%
\pgfpathlineto{\pgfqpoint{4.089135in}{2.235092in}}%
\pgfpathlineto{\pgfqpoint{4.089480in}{2.465240in}}%
\pgfpathlineto{\pgfqpoint{4.090083in}{2.387609in}}%
\pgfpathlineto{\pgfqpoint{4.091006in}{2.512430in}}%
\pgfpathlineto{\pgfqpoint{4.090539in}{2.266637in}}%
\pgfpathlineto{\pgfqpoint{4.091154in}{2.335852in}}%
\pgfpathlineto{\pgfqpoint{4.091388in}{2.230207in}}%
\pgfpathlineto{\pgfqpoint{4.091597in}{2.539609in}}%
\pgfpathlineto{\pgfqpoint{4.092249in}{2.340640in}}%
\pgfpathlineto{\pgfqpoint{4.092988in}{2.546656in}}%
\pgfpathlineto{\pgfqpoint{4.092754in}{2.250810in}}%
\pgfpathlineto{\pgfqpoint{4.093308in}{2.275146in}}%
\pgfpathlineto{\pgfqpoint{4.093345in}{2.246574in}}%
\pgfpathlineto{\pgfqpoint{4.093714in}{2.469044in}}%
\pgfpathlineto{\pgfqpoint{4.094292in}{2.378531in}}%
\pgfpathlineto{\pgfqpoint{4.094563in}{2.487603in}}%
\pgfpathlineto{\pgfqpoint{4.094932in}{2.287133in}}%
\pgfpathlineto{\pgfqpoint{4.095375in}{2.370089in}}%
\pgfpathlineto{\pgfqpoint{4.096311in}{2.238084in}}%
\pgfpathlineto{\pgfqpoint{4.095819in}{2.466158in}}%
\pgfpathlineto{\pgfqpoint{4.096471in}{2.377330in}}%
\pgfpathlineto{\pgfqpoint{4.097209in}{2.505880in}}%
\pgfpathlineto{\pgfqpoint{4.096902in}{2.223064in}}%
\pgfpathlineto{\pgfqpoint{4.097554in}{2.360789in}}%
\pgfpathlineto{\pgfqpoint{4.097591in}{2.299115in}}%
\pgfpathlineto{\pgfqpoint{4.098132in}{2.471828in}}%
\pgfpathlineto{\pgfqpoint{4.098612in}{2.423486in}}%
\pgfpathlineto{\pgfqpoint{4.099326in}{2.497421in}}%
\pgfpathlineto{\pgfqpoint{4.099117in}{2.268760in}}%
\pgfpathlineto{\pgfqpoint{4.099658in}{2.367688in}}%
\pgfpathlineto{\pgfqpoint{4.099855in}{2.254354in}}%
\pgfpathlineto{\pgfqpoint{4.100225in}{2.467742in}}%
\pgfpathlineto{\pgfqpoint{4.100742in}{2.431247in}}%
\pgfpathlineto{\pgfqpoint{4.100778in}{2.441742in}}%
\pgfpathlineto{\pgfqpoint{4.100803in}{2.438333in}}%
\pgfpathlineto{\pgfqpoint{4.101418in}{2.482204in}}%
\pgfpathlineto{\pgfqpoint{4.101049in}{2.277501in}}%
\pgfpathlineto{\pgfqpoint{4.101800in}{2.357199in}}%
\pgfpathlineto{\pgfqpoint{4.102662in}{2.255753in}}%
\pgfpathlineto{\pgfqpoint{4.102292in}{2.460198in}}%
\pgfpathlineto{\pgfqpoint{4.102846in}{2.403944in}}%
\pgfpathlineto{\pgfqpoint{4.103769in}{2.481552in}}%
\pgfpathlineto{\pgfqpoint{4.103412in}{2.241150in}}%
\pgfpathlineto{\pgfqpoint{4.103892in}{2.337075in}}%
\pgfpathlineto{\pgfqpoint{4.104015in}{2.272777in}}%
\pgfpathlineto{\pgfqpoint{4.104938in}{2.443317in}}%
\pgfpathlineto{\pgfqpoint{4.105148in}{2.478291in}}%
\pgfpathlineto{\pgfqpoint{4.105308in}{2.293832in}}%
\pgfpathlineto{\pgfqpoint{4.105985in}{2.387507in}}%
\pgfpathlineto{\pgfqpoint{4.106292in}{2.270300in}}%
\pgfpathlineto{\pgfqpoint{4.106538in}{2.472301in}}%
\pgfpathlineto{\pgfqpoint{4.107018in}{2.401526in}}%
\pgfpathlineto{\pgfqpoint{4.107929in}{2.485623in}}%
\pgfpathlineto{\pgfqpoint{4.107560in}{2.234162in}}%
\pgfpathlineto{\pgfqpoint{4.108102in}{2.359253in}}%
\pgfpathlineto{\pgfqpoint{4.109025in}{2.271505in}}%
\pgfpathlineto{\pgfqpoint{4.108705in}{2.478613in}}%
\pgfpathlineto{\pgfqpoint{4.109111in}{2.381246in}}%
\pgfpathlineto{\pgfqpoint{4.109382in}{2.449537in}}%
\pgfpathlineto{\pgfqpoint{4.109529in}{2.313738in}}%
\pgfpathlineto{\pgfqpoint{4.110218in}{2.380910in}}%
\pgfpathlineto{\pgfqpoint{4.110748in}{2.466797in}}%
\pgfpathlineto{\pgfqpoint{4.110366in}{2.259265in}}%
\pgfpathlineto{\pgfqpoint{4.111351in}{2.407310in}}%
\pgfpathlineto{\pgfqpoint{4.111794in}{2.298888in}}%
\pgfpathlineto{\pgfqpoint{4.111474in}{2.479572in}}%
\pgfpathlineto{\pgfqpoint{4.112495in}{2.325339in}}%
\pgfpathlineto{\pgfqpoint{4.112852in}{2.475228in}}%
\pgfpathlineto{\pgfqpoint{4.113185in}{2.249583in}}%
\pgfpathlineto{\pgfqpoint{4.113652in}{2.393538in}}%
\pgfpathlineto{\pgfqpoint{4.114588in}{2.250944in}}%
\pgfpathlineto{\pgfqpoint{4.114305in}{2.462878in}}%
\pgfpathlineto{\pgfqpoint{4.114760in}{2.392520in}}%
\pgfpathlineto{\pgfqpoint{4.115018in}{2.456935in}}%
\pgfpathlineto{\pgfqpoint{4.115252in}{2.278252in}}%
\pgfpathlineto{\pgfqpoint{4.115806in}{2.390653in}}%
\pgfpathlineto{\pgfqpoint{4.115843in}{2.285331in}}%
\pgfpathlineto{\pgfqpoint{4.116397in}{2.452150in}}%
\pgfpathlineto{\pgfqpoint{4.116914in}{2.379116in}}%
\pgfpathlineto{\pgfqpoint{4.117357in}{2.318665in}}%
\pgfpathlineto{\pgfqpoint{4.117074in}{2.486298in}}%
\pgfpathlineto{\pgfqpoint{4.117972in}{2.396558in}}%
\pgfpathlineto{\pgfqpoint{4.118514in}{2.477228in}}%
\pgfpathlineto{\pgfqpoint{4.118797in}{2.246341in}}%
\pgfpathlineto{\pgfqpoint{4.119080in}{2.394927in}}%
\pgfpathlineto{\pgfqpoint{4.119178in}{2.469371in}}%
\pgfpathlineto{\pgfqpoint{4.119375in}{2.302379in}}%
\pgfpathlineto{\pgfqpoint{4.120138in}{2.375999in}}%
\pgfpathlineto{\pgfqpoint{4.120237in}{2.323394in}}%
\pgfpathlineto{\pgfqpoint{4.120606in}{2.448898in}}%
\pgfpathlineto{\pgfqpoint{4.121234in}{2.393442in}}%
\pgfpathlineto{\pgfqpoint{4.122120in}{2.451397in}}%
\pgfpathlineto{\pgfqpoint{4.121751in}{2.299233in}}%
\pgfpathlineto{\pgfqpoint{4.122292in}{2.362578in}}%
\pgfpathlineto{\pgfqpoint{4.122341in}{2.243991in}}%
\pgfpathlineto{\pgfqpoint{4.122711in}{2.494857in}}%
\pgfpathlineto{\pgfqpoint{4.123375in}{2.386358in}}%
\pgfpathlineto{\pgfqpoint{4.124151in}{2.447294in}}%
\pgfpathlineto{\pgfqpoint{4.123732in}{2.328273in}}%
\pgfpathlineto{\pgfqpoint{4.124434in}{2.328910in}}%
\pgfpathlineto{\pgfqpoint{4.125283in}{2.286464in}}%
\pgfpathlineto{\pgfqpoint{4.125061in}{2.447880in}}%
\pgfpathlineto{\pgfqpoint{4.125320in}{2.390438in}}%
\pgfpathlineto{\pgfqpoint{4.126243in}{2.468523in}}%
\pgfpathlineto{\pgfqpoint{4.125874in}{2.291741in}}%
\pgfpathlineto{\pgfqpoint{4.126428in}{2.394174in}}%
\pgfpathlineto{\pgfqpoint{4.126945in}{2.461273in}}%
\pgfpathlineto{\pgfqpoint{4.126563in}{2.295476in}}%
\pgfpathlineto{\pgfqpoint{4.127252in}{2.342831in}}%
\pgfpathlineto{\pgfqpoint{4.128089in}{2.289275in}}%
\pgfpathlineto{\pgfqpoint{4.127695in}{2.445275in}}%
\pgfpathlineto{\pgfqpoint{4.128274in}{2.406899in}}%
\pgfpathlineto{\pgfqpoint{4.129061in}{2.459928in}}%
\pgfpathlineto{\pgfqpoint{4.128668in}{2.294047in}}%
\pgfpathlineto{\pgfqpoint{4.129332in}{2.344500in}}%
\pgfpathlineto{\pgfqpoint{4.129406in}{2.269327in}}%
\pgfpathlineto{\pgfqpoint{4.129775in}{2.502819in}}%
\pgfpathlineto{\pgfqpoint{4.130403in}{2.408297in}}%
\pgfpathlineto{\pgfqpoint{4.131621in}{2.273431in}}%
\pgfpathlineto{\pgfqpoint{4.130477in}{2.460290in}}%
\pgfpathlineto{\pgfqpoint{4.131634in}{2.288206in}}%
\pgfpathlineto{\pgfqpoint{4.131880in}{2.460122in}}%
\pgfpathlineto{\pgfqpoint{4.132212in}{2.266066in}}%
\pgfpathlineto{\pgfqpoint{4.132766in}{2.390608in}}%
\pgfpathlineto{\pgfqpoint{4.133738in}{2.286586in}}%
\pgfpathlineto{\pgfqpoint{4.133332in}{2.456720in}}%
\pgfpathlineto{\pgfqpoint{4.133874in}{2.383022in}}%
\pgfpathlineto{\pgfqpoint{4.133935in}{2.471472in}}%
\pgfpathlineto{\pgfqpoint{4.134329in}{2.291209in}}%
\pgfpathlineto{\pgfqpoint{4.134957in}{2.360941in}}%
\pgfpathlineto{\pgfqpoint{4.135757in}{2.281310in}}%
\pgfpathlineto{\pgfqpoint{4.135375in}{2.456429in}}%
\pgfpathlineto{\pgfqpoint{4.136003in}{2.385758in}}%
\pgfpathlineto{\pgfqpoint{4.136077in}{2.462365in}}%
\pgfpathlineto{\pgfqpoint{4.136508in}{2.276643in}}%
\pgfpathlineto{\pgfqpoint{4.136938in}{2.344515in}}%
\pgfpathlineto{\pgfqpoint{4.137098in}{2.287676in}}%
\pgfpathlineto{\pgfqpoint{4.137480in}{2.473162in}}%
\pgfpathlineto{\pgfqpoint{4.137984in}{2.379980in}}%
\pgfpathlineto{\pgfqpoint{4.138908in}{2.470587in}}%
\pgfpathlineto{\pgfqpoint{4.138551in}{2.305372in}}%
\pgfpathlineto{\pgfqpoint{4.139068in}{2.329900in}}%
\pgfpathlineto{\pgfqpoint{4.139831in}{2.451665in}}%
\pgfpathlineto{\pgfqpoint{4.140040in}{2.270197in}}%
\pgfpathlineto{\pgfqpoint{4.140212in}{2.377162in}}%
\pgfpathlineto{\pgfqpoint{4.140631in}{2.271084in}}%
\pgfpathlineto{\pgfqpoint{4.140323in}{2.440293in}}%
\pgfpathlineto{\pgfqpoint{4.140975in}{2.428509in}}%
\pgfpathlineto{\pgfqpoint{4.141012in}{2.468428in}}%
\pgfpathlineto{\pgfqpoint{4.141443in}{2.295270in}}%
\pgfpathlineto{\pgfqpoint{4.142009in}{2.324617in}}%
\pgfpathlineto{\pgfqpoint{4.142846in}{2.305300in}}%
\pgfpathlineto{\pgfqpoint{4.142452in}{2.438456in}}%
\pgfpathlineto{\pgfqpoint{4.143043in}{2.401318in}}%
\pgfpathlineto{\pgfqpoint{4.143351in}{2.433772in}}%
\pgfpathlineto{\pgfqpoint{4.143560in}{2.303193in}}%
\pgfpathlineto{\pgfqpoint{4.144089in}{2.363389in}}%
\pgfpathlineto{\pgfqpoint{4.144175in}{2.294312in}}%
\pgfpathlineto{\pgfqpoint{4.144544in}{2.433635in}}%
\pgfpathlineto{\pgfqpoint{4.145135in}{2.393450in}}%
\pgfpathlineto{\pgfqpoint{4.145997in}{2.435949in}}%
\pgfpathlineto{\pgfqpoint{4.145652in}{2.308998in}}%
\pgfpathlineto{\pgfqpoint{4.146194in}{2.374147in}}%
\pgfpathlineto{\pgfqpoint{4.146354in}{2.310622in}}%
\pgfpathlineto{\pgfqpoint{4.146588in}{2.425007in}}%
\pgfpathlineto{\pgfqpoint{4.147289in}{2.383488in}}%
\pgfpathlineto{\pgfqpoint{4.147720in}{2.295085in}}%
\pgfpathlineto{\pgfqpoint{4.147523in}{2.429576in}}%
\pgfpathlineto{\pgfqpoint{4.147991in}{2.408035in}}%
\pgfpathlineto{\pgfqpoint{4.148089in}{2.446175in}}%
\pgfpathlineto{\pgfqpoint{4.148311in}{2.314270in}}%
\pgfpathlineto{\pgfqpoint{4.149061in}{2.365568in}}%
\pgfpathlineto{\pgfqpoint{4.149160in}{2.306719in}}%
\pgfpathlineto{\pgfqpoint{4.149529in}{2.436758in}}%
\pgfpathlineto{\pgfqpoint{4.150095in}{2.399915in}}%
\pgfpathlineto{\pgfqpoint{4.150120in}{2.432972in}}%
\pgfpathlineto{\pgfqpoint{4.150563in}{2.318184in}}%
\pgfpathlineto{\pgfqpoint{4.151191in}{2.376982in}}%
\pgfpathlineto{\pgfqpoint{4.152224in}{2.455608in}}%
\pgfpathlineto{\pgfqpoint{4.151252in}{2.286078in}}%
\pgfpathlineto{\pgfqpoint{4.152348in}{2.395372in}}%
\pgfpathlineto{\pgfqpoint{4.153357in}{2.311764in}}%
\pgfpathlineto{\pgfqpoint{4.153172in}{2.423861in}}%
\pgfpathlineto{\pgfqpoint{4.153541in}{2.360476in}}%
\pgfpathlineto{\pgfqpoint{4.153652in}{2.443465in}}%
\pgfpathlineto{\pgfqpoint{4.154071in}{2.287013in}}%
\pgfpathlineto{\pgfqpoint{4.154624in}{2.347202in}}%
\pgfpathlineto{\pgfqpoint{4.155474in}{2.290378in}}%
\pgfpathlineto{\pgfqpoint{4.155166in}{2.444181in}}%
\pgfpathlineto{\pgfqpoint{4.155683in}{2.392637in}}%
\pgfpathlineto{\pgfqpoint{4.155757in}{2.455214in}}%
\pgfpathlineto{\pgfqpoint{4.156175in}{2.294018in}}%
\pgfpathlineto{\pgfqpoint{4.156741in}{2.360290in}}%
\pgfpathlineto{\pgfqpoint{4.157591in}{2.290017in}}%
\pgfpathlineto{\pgfqpoint{4.157258in}{2.430109in}}%
\pgfpathlineto{\pgfqpoint{4.157800in}{2.417638in}}%
\pgfpathlineto{\pgfqpoint{4.157837in}{2.429118in}}%
\pgfpathlineto{\pgfqpoint{4.157849in}{2.435418in}}%
\pgfpathlineto{\pgfqpoint{4.158169in}{2.311786in}}%
\pgfpathlineto{\pgfqpoint{4.158821in}{2.368110in}}%
\pgfpathlineto{\pgfqpoint{4.159695in}{2.282016in}}%
\pgfpathlineto{\pgfqpoint{4.159301in}{2.428735in}}%
\pgfpathlineto{\pgfqpoint{4.159880in}{2.410219in}}%
\pgfpathlineto{\pgfqpoint{4.160052in}{2.426946in}}%
\pgfpathlineto{\pgfqpoint{4.160286in}{2.309608in}}%
\pgfpathlineto{\pgfqpoint{4.160926in}{2.376189in}}%
\pgfpathlineto{\pgfqpoint{4.161787in}{2.305810in}}%
\pgfpathlineto{\pgfqpoint{4.161443in}{2.429723in}}%
\pgfpathlineto{\pgfqpoint{4.161997in}{2.395315in}}%
\pgfpathlineto{\pgfqpoint{4.162034in}{2.438653in}}%
\pgfpathlineto{\pgfqpoint{4.162378in}{2.318699in}}%
\pgfpathlineto{\pgfqpoint{4.163055in}{2.331778in}}%
\pgfpathlineto{\pgfqpoint{4.163437in}{2.427256in}}%
\pgfpathlineto{\pgfqpoint{4.163904in}{2.296196in}}%
\pgfpathlineto{\pgfqpoint{4.164298in}{2.401018in}}%
\pgfpathlineto{\pgfqpoint{4.164495in}{2.311705in}}%
\pgfpathlineto{\pgfqpoint{4.164864in}{2.415449in}}%
\pgfpathlineto{\pgfqpoint{4.165431in}{2.347527in}}%
\pgfpathlineto{\pgfqpoint{4.166378in}{2.422773in}}%
\pgfpathlineto{\pgfqpoint{4.166009in}{2.296828in}}%
\pgfpathlineto{\pgfqpoint{4.166551in}{2.380820in}}%
\pgfpathlineto{\pgfqpoint{4.166587in}{2.304295in}}%
\pgfpathlineto{\pgfqpoint{4.166969in}{2.414704in}}%
\pgfpathlineto{\pgfqpoint{4.167646in}{2.411880in}}%
\pgfpathlineto{\pgfqpoint{4.167671in}{2.423284in}}%
\pgfpathlineto{\pgfqpoint{4.168003in}{2.308514in}}%
\pgfpathlineto{\pgfqpoint{4.168667in}{2.351630in}}%
\pgfpathlineto{\pgfqpoint{4.169529in}{2.306156in}}%
\pgfpathlineto{\pgfqpoint{4.169320in}{2.425787in}}%
\pgfpathlineto{\pgfqpoint{4.169738in}{2.389108in}}%
\pgfpathlineto{\pgfqpoint{4.169911in}{2.419985in}}%
\pgfpathlineto{\pgfqpoint{4.170120in}{2.321972in}}%
\pgfpathlineto{\pgfqpoint{4.170772in}{2.371715in}}%
\pgfpathlineto{\pgfqpoint{4.170920in}{2.317294in}}%
\pgfpathlineto{\pgfqpoint{4.171264in}{2.418976in}}%
\pgfpathlineto{\pgfqpoint{4.171843in}{2.413279in}}%
\pgfpathlineto{\pgfqpoint{4.171855in}{2.420949in}}%
\pgfpathlineto{\pgfqpoint{4.172323in}{2.312826in}}%
\pgfpathlineto{\pgfqpoint{4.172877in}{2.368523in}}%
\pgfpathlineto{\pgfqpoint{4.172914in}{2.328788in}}%
\pgfpathlineto{\pgfqpoint{4.173455in}{2.415587in}}%
\pgfpathlineto{\pgfqpoint{4.173960in}{2.391299in}}%
\pgfpathlineto{\pgfqpoint{4.174883in}{2.423431in}}%
\pgfpathlineto{\pgfqpoint{4.174427in}{2.327419in}}%
\pgfpathlineto{\pgfqpoint{4.175006in}{2.354372in}}%
\pgfpathlineto{\pgfqpoint{4.175117in}{2.307616in}}%
\pgfpathlineto{\pgfqpoint{4.175474in}{2.427113in}}%
\pgfpathlineto{\pgfqpoint{4.176015in}{2.364973in}}%
\pgfpathlineto{\pgfqpoint{4.176987in}{2.427586in}}%
\pgfpathlineto{\pgfqpoint{4.176532in}{2.327588in}}%
\pgfpathlineto{\pgfqpoint{4.177098in}{2.352796in}}%
\pgfpathlineto{\pgfqpoint{4.177923in}{2.316951in}}%
\pgfpathlineto{\pgfqpoint{4.177578in}{2.429198in}}%
\pgfpathlineto{\pgfqpoint{4.178120in}{2.365529in}}%
\pgfpathlineto{\pgfqpoint{4.179006in}{2.414901in}}%
\pgfpathlineto{\pgfqpoint{4.178637in}{2.320399in}}%
\pgfpathlineto{\pgfqpoint{4.179203in}{2.347651in}}%
\pgfpathlineto{\pgfqpoint{4.180015in}{2.328529in}}%
\pgfpathlineto{\pgfqpoint{4.179707in}{2.410211in}}%
\pgfpathlineto{\pgfqpoint{4.180175in}{2.368545in}}%
\pgfpathlineto{\pgfqpoint{4.180520in}{2.428361in}}%
\pgfpathlineto{\pgfqpoint{4.180729in}{2.317987in}}%
\pgfpathlineto{\pgfqpoint{4.181270in}{2.373796in}}%
\pgfpathlineto{\pgfqpoint{4.182120in}{2.322652in}}%
\pgfpathlineto{\pgfqpoint{4.181874in}{2.413930in}}%
\pgfpathlineto{\pgfqpoint{4.182366in}{2.378655in}}%
\pgfpathlineto{\pgfqpoint{4.182575in}{2.413525in}}%
\pgfpathlineto{\pgfqpoint{4.182944in}{2.328571in}}%
\pgfpathlineto{\pgfqpoint{4.183474in}{2.386067in}}%
\pgfpathlineto{\pgfqpoint{4.183523in}{2.314759in}}%
\pgfpathlineto{\pgfqpoint{4.183904in}{2.402313in}}%
\pgfpathlineto{\pgfqpoint{4.184569in}{2.387055in}}%
\pgfpathlineto{\pgfqpoint{4.184667in}{2.424422in}}%
\pgfpathlineto{\pgfqpoint{4.185061in}{2.333319in}}%
\pgfpathlineto{\pgfqpoint{4.185615in}{2.338047in}}%
\pgfpathlineto{\pgfqpoint{4.186083in}{2.423455in}}%
\pgfpathlineto{\pgfqpoint{4.186317in}{2.318996in}}%
\pgfpathlineto{\pgfqpoint{4.186858in}{2.377707in}}%
\pgfpathlineto{\pgfqpoint{4.187720in}{2.335219in}}%
\pgfpathlineto{\pgfqpoint{4.187264in}{2.402568in}}%
\pgfpathlineto{\pgfqpoint{4.187954in}{2.385994in}}%
\pgfpathlineto{\pgfqpoint{4.188187in}{2.428654in}}%
\pgfpathlineto{\pgfqpoint{4.188397in}{2.323084in}}%
\pgfpathlineto{\pgfqpoint{4.188975in}{2.344880in}}%
\pgfpathlineto{\pgfqpoint{4.189270in}{2.324226in}}%
\pgfpathlineto{\pgfqpoint{4.189615in}{2.427266in}}%
\pgfpathlineto{\pgfqpoint{4.190021in}{2.359339in}}%
\pgfpathlineto{\pgfqpoint{4.190981in}{2.409116in}}%
\pgfpathlineto{\pgfqpoint{4.190674in}{2.341378in}}%
\pgfpathlineto{\pgfqpoint{4.191154in}{2.399425in}}%
\pgfpathlineto{\pgfqpoint{4.191917in}{2.327151in}}%
\pgfpathlineto{\pgfqpoint{4.191720in}{2.436661in}}%
\pgfpathlineto{\pgfqpoint{4.192274in}{2.380727in}}%
\pgfpathlineto{\pgfqpoint{4.192409in}{2.413431in}}%
\pgfpathlineto{\pgfqpoint{4.192790in}{2.331096in}}%
\pgfpathlineto{\pgfqpoint{4.193295in}{2.343884in}}%
\pgfpathlineto{\pgfqpoint{4.193997in}{2.327269in}}%
\pgfpathlineto{\pgfqpoint{4.193800in}{2.422160in}}%
\pgfpathlineto{\pgfqpoint{4.194329in}{2.410006in}}%
\pgfpathlineto{\pgfqpoint{4.194723in}{2.328071in}}%
\pgfpathlineto{\pgfqpoint{4.195240in}{2.416695in}}%
\pgfpathlineto{\pgfqpoint{4.195547in}{2.362902in}}%
\pgfpathlineto{\pgfqpoint{4.196495in}{2.346815in}}%
\pgfpathlineto{\pgfqpoint{4.195867in}{2.428035in}}%
\pgfpathlineto{\pgfqpoint{4.196544in}{2.389361in}}%
\pgfpathlineto{\pgfqpoint{4.197283in}{2.428559in}}%
\pgfpathlineto{\pgfqpoint{4.197517in}{2.322938in}}%
\pgfpathlineto{\pgfqpoint{4.197640in}{2.381861in}}%
\pgfpathlineto{\pgfqpoint{4.197677in}{2.327376in}}%
\pgfpathlineto{\pgfqpoint{4.198464in}{2.409801in}}%
\pgfpathlineto{\pgfqpoint{4.198735in}{2.399823in}}%
\pgfpathlineto{\pgfqpoint{4.199597in}{2.321246in}}%
\pgfpathlineto{\pgfqpoint{4.199400in}{2.438346in}}%
\pgfpathlineto{\pgfqpoint{4.199892in}{2.366427in}}%
\pgfpathlineto{\pgfqpoint{4.200827in}{2.423598in}}%
\pgfpathlineto{\pgfqpoint{4.200187in}{2.333142in}}%
\pgfpathlineto{\pgfqpoint{4.200975in}{2.361275in}}%
\pgfpathlineto{\pgfqpoint{4.201701in}{2.330558in}}%
\pgfpathlineto{\pgfqpoint{4.201763in}{2.414048in}}%
\pgfpathlineto{\pgfqpoint{4.202070in}{2.376450in}}%
\pgfpathlineto{\pgfqpoint{4.202403in}{2.317880in}}%
\pgfpathlineto{\pgfqpoint{4.202353in}{2.410762in}}%
\pgfpathlineto{\pgfqpoint{4.202883in}{2.391430in}}%
\pgfpathlineto{\pgfqpoint{4.202932in}{2.435837in}}%
\pgfpathlineto{\pgfqpoint{4.203129in}{2.332623in}}%
\pgfpathlineto{\pgfqpoint{4.203953in}{2.361063in}}%
\pgfpathlineto{\pgfqpoint{4.204113in}{2.405857in}}%
\pgfpathlineto{\pgfqpoint{4.204507in}{2.332946in}}%
\pgfpathlineto{\pgfqpoint{4.205073in}{2.364294in}}%
\pgfpathlineto{\pgfqpoint{4.205197in}{2.337776in}}%
\pgfpathlineto{\pgfqpoint{4.205295in}{2.413158in}}%
\pgfpathlineto{\pgfqpoint{4.206107in}{2.377922in}}%
\pgfpathlineto{\pgfqpoint{4.207067in}{2.418114in}}%
\pgfpathlineto{\pgfqpoint{4.206957in}{2.347946in}}%
\pgfpathlineto{\pgfqpoint{4.207227in}{2.396023in}}%
\pgfpathlineto{\pgfqpoint{4.208298in}{2.344798in}}%
\pgfpathlineto{\pgfqpoint{4.208249in}{2.418091in}}%
\pgfpathlineto{\pgfqpoint{4.208384in}{2.377035in}}%
\pgfpathlineto{\pgfqpoint{4.208495in}{2.424942in}}%
\pgfpathlineto{\pgfqpoint{4.208877in}{2.344199in}}%
\pgfpathlineto{\pgfqpoint{4.209467in}{2.371992in}}%
\pgfpathlineto{\pgfqpoint{4.210083in}{2.336080in}}%
\pgfpathlineto{\pgfqpoint{4.210021in}{2.420012in}}%
\pgfpathlineto{\pgfqpoint{4.210563in}{2.384637in}}%
\pgfpathlineto{\pgfqpoint{4.210612in}{2.421727in}}%
\pgfpathlineto{\pgfqpoint{4.210797in}{2.337457in}}%
\pgfpathlineto{\pgfqpoint{4.211646in}{2.353165in}}%
\pgfpathlineto{\pgfqpoint{4.211670in}{2.339608in}}%
\pgfpathlineto{\pgfqpoint{4.212027in}{2.410529in}}%
\pgfpathlineto{\pgfqpoint{4.212680in}{2.396357in}}%
\pgfpathlineto{\pgfqpoint{4.212963in}{2.411733in}}%
\pgfpathlineto{\pgfqpoint{4.213603in}{2.339129in}}%
\pgfpathlineto{\pgfqpoint{4.213713in}{2.376820in}}%
\pgfpathlineto{\pgfqpoint{4.213763in}{2.338586in}}%
\pgfpathlineto{\pgfqpoint{4.214132in}{2.413028in}}%
\pgfpathlineto{\pgfqpoint{4.214809in}{2.377076in}}%
\pgfpathlineto{\pgfqpoint{4.215929in}{2.414879in}}%
\pgfpathlineto{\pgfqpoint{4.215387in}{2.343155in}}%
\pgfpathlineto{\pgfqpoint{4.215941in}{2.402309in}}%
\pgfpathlineto{\pgfqpoint{4.216569in}{2.339447in}}%
\pgfpathlineto{\pgfqpoint{4.216175in}{2.407536in}}%
\pgfpathlineto{\pgfqpoint{4.217061in}{2.369492in}}%
\pgfpathlineto{\pgfqpoint{4.217443in}{2.345482in}}%
\pgfpathlineto{\pgfqpoint{4.217344in}{2.407372in}}%
\pgfpathlineto{\pgfqpoint{4.217689in}{2.406736in}}%
\pgfpathlineto{\pgfqpoint{4.218280in}{2.423235in}}%
\pgfpathlineto{\pgfqpoint{4.218489in}{2.323062in}}%
\pgfpathlineto{\pgfqpoint{4.218747in}{2.382360in}}%
\pgfpathlineto{\pgfqpoint{4.219523in}{2.332314in}}%
\pgfpathlineto{\pgfqpoint{4.219461in}{2.415789in}}%
\pgfpathlineto{\pgfqpoint{4.219892in}{2.347911in}}%
\pgfpathlineto{\pgfqpoint{4.220643in}{2.432301in}}%
\pgfpathlineto{\pgfqpoint{4.220704in}{2.337925in}}%
\pgfpathlineto{\pgfqpoint{4.221024in}{2.382435in}}%
\pgfpathlineto{\pgfqpoint{4.221344in}{2.421902in}}%
\pgfpathlineto{\pgfqpoint{4.221283in}{2.325612in}}%
\pgfpathlineto{\pgfqpoint{4.222095in}{2.373016in}}%
\pgfpathlineto{\pgfqpoint{4.223067in}{2.326313in}}%
\pgfpathlineto{\pgfqpoint{4.222427in}{2.418299in}}%
\pgfpathlineto{\pgfqpoint{4.223215in}{2.352833in}}%
\pgfpathlineto{\pgfqpoint{4.223855in}{2.411548in}}%
\pgfpathlineto{\pgfqpoint{4.224076in}{2.330882in}}%
\pgfpathlineto{\pgfqpoint{4.224360in}{2.380279in}}%
\pgfpathlineto{\pgfqpoint{4.225160in}{2.344532in}}%
\pgfpathlineto{\pgfqpoint{4.225036in}{2.408734in}}%
\pgfpathlineto{\pgfqpoint{4.225480in}{2.370986in}}%
\pgfpathlineto{\pgfqpoint{4.226218in}{2.421260in}}%
\pgfpathlineto{\pgfqpoint{4.225861in}{2.332635in}}%
\pgfpathlineto{\pgfqpoint{4.226575in}{2.381562in}}%
\pgfpathlineto{\pgfqpoint{4.226870in}{2.332943in}}%
\pgfpathlineto{\pgfqpoint{4.226661in}{2.423289in}}%
\pgfpathlineto{\pgfqpoint{4.227670in}{2.394968in}}%
\pgfpathlineto{\pgfqpoint{4.227953in}{2.334407in}}%
\pgfpathlineto{\pgfqpoint{4.228335in}{2.429174in}}%
\pgfpathlineto{\pgfqpoint{4.228864in}{2.360004in}}%
\pgfpathlineto{\pgfqpoint{4.229036in}{2.428601in}}%
\pgfpathlineto{\pgfqpoint{4.228975in}{2.326225in}}%
\pgfpathlineto{\pgfqpoint{4.229972in}{2.371286in}}%
\pgfpathlineto{\pgfqpoint{4.231055in}{2.333968in}}%
\pgfpathlineto{\pgfqpoint{4.230809in}{2.401370in}}%
\pgfpathlineto{\pgfqpoint{4.231092in}{2.359363in}}%
\pgfpathlineto{\pgfqpoint{4.231843in}{2.424557in}}%
\pgfpathlineto{\pgfqpoint{4.231633in}{2.338087in}}%
\pgfpathlineto{\pgfqpoint{4.232187in}{2.340410in}}%
\pgfpathlineto{\pgfqpoint{4.233246in}{2.407644in}}%
\pgfpathlineto{\pgfqpoint{4.233344in}{2.375888in}}%
\pgfpathlineto{\pgfqpoint{4.233566in}{2.342382in}}%
\pgfpathlineto{\pgfqpoint{4.233923in}{2.416981in}}%
\pgfpathlineto{\pgfqpoint{4.234464in}{2.356652in}}%
\pgfpathlineto{\pgfqpoint{4.234636in}{2.410410in}}%
\pgfpathlineto{\pgfqpoint{4.234563in}{2.337871in}}%
\pgfpathlineto{\pgfqpoint{4.235584in}{2.372195in}}%
\pgfpathlineto{\pgfqpoint{4.235658in}{2.336956in}}%
\pgfpathlineto{\pgfqpoint{4.236027in}{2.429300in}}%
\pgfpathlineto{\pgfqpoint{4.236593in}{2.401324in}}%
\pgfpathlineto{\pgfqpoint{4.236741in}{2.427122in}}%
\pgfpathlineto{\pgfqpoint{4.236680in}{2.346410in}}%
\pgfpathlineto{\pgfqpoint{4.237504in}{2.357118in}}%
\pgfpathlineto{\pgfqpoint{4.238452in}{2.340902in}}%
\pgfpathlineto{\pgfqpoint{4.237590in}{2.398784in}}%
\pgfpathlineto{\pgfqpoint{4.238563in}{2.384632in}}%
\pgfpathlineto{\pgfqpoint{4.239560in}{2.414297in}}%
\pgfpathlineto{\pgfqpoint{4.239449in}{2.349706in}}%
\pgfpathlineto{\pgfqpoint{4.239621in}{2.371134in}}%
\pgfpathlineto{\pgfqpoint{4.239867in}{2.341162in}}%
\pgfpathlineto{\pgfqpoint{4.239695in}{2.418986in}}%
\pgfpathlineto{\pgfqpoint{4.240729in}{2.370150in}}%
\pgfpathlineto{\pgfqpoint{4.241615in}{2.401844in}}%
\pgfpathlineto{\pgfqpoint{4.241418in}{2.344315in}}%
\pgfpathlineto{\pgfqpoint{4.241849in}{2.375879in}}%
\pgfpathlineto{\pgfqpoint{4.242243in}{2.337285in}}%
\pgfpathlineto{\pgfqpoint{4.242058in}{2.405734in}}%
\pgfpathlineto{\pgfqpoint{4.242969in}{2.356389in}}%
\pgfpathlineto{\pgfqpoint{4.243732in}{2.419393in}}%
\pgfpathlineto{\pgfqpoint{4.243646in}{2.341254in}}%
\pgfpathlineto{\pgfqpoint{4.244076in}{2.352916in}}%
\pgfpathlineto{\pgfqpoint{4.244433in}{2.429973in}}%
\pgfpathlineto{\pgfqpoint{4.245049in}{2.351287in}}%
\pgfpathlineto{\pgfqpoint{4.245184in}{2.363100in}}%
\pgfpathlineto{\pgfqpoint{4.245738in}{2.348089in}}%
\pgfpathlineto{\pgfqpoint{4.246206in}{2.400572in}}%
\pgfpathlineto{\pgfqpoint{4.246255in}{2.385325in}}%
\pgfpathlineto{\pgfqpoint{4.246698in}{2.403313in}}%
\pgfpathlineto{\pgfqpoint{4.246439in}{2.340461in}}%
\pgfpathlineto{\pgfqpoint{4.247313in}{2.371727in}}%
\pgfpathlineto{\pgfqpoint{4.248212in}{2.346496in}}%
\pgfpathlineto{\pgfqpoint{4.247387in}{2.415544in}}%
\pgfpathlineto{\pgfqpoint{4.248421in}{2.363355in}}%
\pgfpathlineto{\pgfqpoint{4.249061in}{2.408285in}}%
\pgfpathlineto{\pgfqpoint{4.249246in}{2.347307in}}%
\pgfpathlineto{\pgfqpoint{4.249553in}{2.369287in}}%
\pgfpathlineto{\pgfqpoint{4.250636in}{2.342648in}}%
\pgfpathlineto{\pgfqpoint{4.249763in}{2.408765in}}%
\pgfpathlineto{\pgfqpoint{4.250661in}{2.355540in}}%
\pgfpathlineto{\pgfqpoint{4.251424in}{2.410468in}}%
\pgfpathlineto{\pgfqpoint{4.251338in}{2.341896in}}%
\pgfpathlineto{\pgfqpoint{4.251769in}{2.360024in}}%
\pgfpathlineto{\pgfqpoint{4.252126in}{2.426988in}}%
\pgfpathlineto{\pgfqpoint{4.252372in}{2.347018in}}%
\pgfpathlineto{\pgfqpoint{4.252876in}{2.363716in}}%
\pgfpathlineto{\pgfqpoint{4.253430in}{2.328842in}}%
\pgfpathlineto{\pgfqpoint{4.253209in}{2.406606in}}%
\pgfpathlineto{\pgfqpoint{4.253873in}{2.380885in}}%
\pgfpathlineto{\pgfqpoint{4.253910in}{2.407475in}}%
\pgfpathlineto{\pgfqpoint{4.254132in}{2.338260in}}%
\pgfpathlineto{\pgfqpoint{4.254981in}{2.383792in}}%
\pgfpathlineto{\pgfqpoint{4.255203in}{2.346251in}}%
\pgfpathlineto{\pgfqpoint{4.255092in}{2.410029in}}%
\pgfpathlineto{\pgfqpoint{4.256126in}{2.363193in}}%
\pgfpathlineto{\pgfqpoint{4.256286in}{2.415147in}}%
\pgfpathlineto{\pgfqpoint{4.256396in}{2.348646in}}%
\pgfpathlineto{\pgfqpoint{4.257258in}{2.374530in}}%
\pgfpathlineto{\pgfqpoint{4.257652in}{2.345132in}}%
\pgfpathlineto{\pgfqpoint{4.257467in}{2.411250in}}%
\pgfpathlineto{\pgfqpoint{4.258366in}{2.375586in}}%
\pgfpathlineto{\pgfqpoint{4.259252in}{2.410937in}}%
\pgfpathlineto{\pgfqpoint{4.259030in}{2.346924in}}%
\pgfpathlineto{\pgfqpoint{4.259424in}{2.360878in}}%
\pgfpathlineto{\pgfqpoint{4.260064in}{2.351041in}}%
\pgfpathlineto{\pgfqpoint{4.259830in}{2.418400in}}%
\pgfpathlineto{\pgfqpoint{4.260335in}{2.378191in}}%
\pgfpathlineto{\pgfqpoint{4.260913in}{2.401093in}}%
\pgfpathlineto{\pgfqpoint{4.261123in}{2.330143in}}%
\pgfpathlineto{\pgfqpoint{4.261443in}{2.381240in}}%
\pgfpathlineto{\pgfqpoint{4.261824in}{2.324903in}}%
\pgfpathlineto{\pgfqpoint{4.261615in}{2.417005in}}%
\pgfpathlineto{\pgfqpoint{4.262550in}{2.370851in}}%
\pgfpathlineto{\pgfqpoint{4.262796in}{2.410282in}}%
\pgfpathlineto{\pgfqpoint{4.262907in}{2.340121in}}%
\pgfpathlineto{\pgfqpoint{4.263670in}{2.389019in}}%
\pgfpathlineto{\pgfqpoint{4.264790in}{2.346592in}}%
\pgfpathlineto{\pgfqpoint{4.263990in}{2.418743in}}%
\pgfpathlineto{\pgfqpoint{4.264815in}{2.361378in}}%
\pgfpathlineto{\pgfqpoint{4.265762in}{2.416708in}}%
\pgfpathlineto{\pgfqpoint{4.265332in}{2.345324in}}%
\pgfpathlineto{\pgfqpoint{4.265910in}{2.358624in}}%
\pgfpathlineto{\pgfqpoint{4.265996in}{2.337263in}}%
\pgfpathlineto{\pgfqpoint{4.266255in}{2.404618in}}%
\pgfpathlineto{\pgfqpoint{4.266907in}{2.381988in}}%
\pgfpathlineto{\pgfqpoint{4.267522in}{2.416476in}}%
\pgfpathlineto{\pgfqpoint{4.267756in}{2.355188in}}%
\pgfpathlineto{\pgfqpoint{4.268002in}{2.371445in}}%
\pgfpathlineto{\pgfqpoint{4.268384in}{2.399353in}}%
\pgfpathlineto{\pgfqpoint{4.268815in}{2.336538in}}%
\pgfpathlineto{\pgfqpoint{4.269110in}{2.374370in}}%
\pgfpathlineto{\pgfqpoint{4.269516in}{2.324404in}}%
\pgfpathlineto{\pgfqpoint{4.269307in}{2.416873in}}%
\pgfpathlineto{\pgfqpoint{4.270242in}{2.365060in}}%
\pgfpathlineto{\pgfqpoint{4.270489in}{2.414308in}}%
\pgfpathlineto{\pgfqpoint{4.270587in}{2.346107in}}%
\pgfpathlineto{\pgfqpoint{4.271375in}{2.389419in}}%
\pgfpathlineto{\pgfqpoint{4.272310in}{2.340583in}}%
\pgfpathlineto{\pgfqpoint{4.271670in}{2.413784in}}%
\pgfpathlineto{\pgfqpoint{4.272507in}{2.371672in}}%
\pgfpathlineto{\pgfqpoint{4.273442in}{2.410995in}}%
\pgfpathlineto{\pgfqpoint{4.273012in}{2.345415in}}%
\pgfpathlineto{\pgfqpoint{4.273602in}{2.363507in}}%
\pgfpathlineto{\pgfqpoint{4.273676in}{2.337932in}}%
\pgfpathlineto{\pgfqpoint{4.274513in}{2.412773in}}%
\pgfpathlineto{\pgfqpoint{4.274587in}{2.382322in}}%
\pgfpathlineto{\pgfqpoint{4.275215in}{2.420608in}}%
\pgfpathlineto{\pgfqpoint{4.274809in}{2.352850in}}%
\pgfpathlineto{\pgfqpoint{4.275695in}{2.381193in}}%
\pgfpathlineto{\pgfqpoint{4.276076in}{2.396403in}}%
\pgfpathlineto{\pgfqpoint{4.275769in}{2.362183in}}%
\pgfpathlineto{\pgfqpoint{4.276175in}{2.363997in}}%
\pgfpathlineto{\pgfqpoint{4.277209in}{2.335682in}}%
\pgfpathlineto{\pgfqpoint{4.276987in}{2.413042in}}%
\pgfpathlineto{\pgfqpoint{4.277246in}{2.373625in}}%
\pgfpathlineto{\pgfqpoint{4.278169in}{2.417476in}}%
\pgfpathlineto{\pgfqpoint{4.277898in}{2.331773in}}%
\pgfpathlineto{\pgfqpoint{4.278353in}{2.370306in}}%
\pgfpathlineto{\pgfqpoint{4.278439in}{2.376802in}}%
\pgfpathlineto{\pgfqpoint{4.278575in}{2.347581in}}%
\pgfpathlineto{\pgfqpoint{4.278587in}{2.345882in}}%
\pgfpathlineto{\pgfqpoint{4.278673in}{2.409080in}}%
\pgfpathlineto{\pgfqpoint{4.279326in}{2.379116in}}%
\pgfpathlineto{\pgfqpoint{4.279362in}{2.419801in}}%
\pgfpathlineto{\pgfqpoint{4.280002in}{2.335556in}}%
\pgfpathlineto{\pgfqpoint{4.280433in}{2.392244in}}%
\pgfpathlineto{\pgfqpoint{4.280655in}{2.355836in}}%
\pgfpathlineto{\pgfqpoint{4.280532in}{2.402290in}}%
\pgfpathlineto{\pgfqpoint{4.280679in}{2.361449in}}%
\pgfpathlineto{\pgfqpoint{4.281369in}{2.342325in}}%
\pgfpathlineto{\pgfqpoint{4.281135in}{2.408959in}}%
\pgfpathlineto{\pgfqpoint{4.281787in}{2.359008in}}%
\pgfpathlineto{\pgfqpoint{4.282895in}{2.428673in}}%
\pgfpathlineto{\pgfqpoint{4.282095in}{2.333956in}}%
\pgfpathlineto{\pgfqpoint{4.282932in}{2.375153in}}%
\pgfpathlineto{\pgfqpoint{4.283141in}{2.349502in}}%
\pgfpathlineto{\pgfqpoint{4.283965in}{2.403466in}}%
\pgfpathlineto{\pgfqpoint{4.284039in}{2.366192in}}%
\pgfpathlineto{\pgfqpoint{4.284679in}{2.408578in}}%
\pgfpathlineto{\pgfqpoint{4.284187in}{2.330441in}}%
\pgfpathlineto{\pgfqpoint{4.285172in}{2.393082in}}%
\pgfpathlineto{\pgfqpoint{4.285578in}{2.340975in}}%
\pgfpathlineto{\pgfqpoint{4.285849in}{2.412740in}}%
\pgfpathlineto{\pgfqpoint{4.286304in}{2.366071in}}%
\pgfpathlineto{\pgfqpoint{4.287042in}{2.424093in}}%
\pgfpathlineto{\pgfqpoint{4.286981in}{2.346156in}}%
\pgfpathlineto{\pgfqpoint{4.287412in}{2.368025in}}%
\pgfpathlineto{\pgfqpoint{4.287769in}{2.408419in}}%
\pgfpathlineto{\pgfqpoint{4.287682in}{2.347118in}}%
\pgfpathlineto{\pgfqpoint{4.288519in}{2.370578in}}%
\pgfpathlineto{\pgfqpoint{4.289061in}{2.344548in}}%
\pgfpathlineto{\pgfqpoint{4.288815in}{2.403172in}}%
\pgfpathlineto{\pgfqpoint{4.289639in}{2.356427in}}%
\pgfpathlineto{\pgfqpoint{4.290575in}{2.422589in}}%
\pgfpathlineto{\pgfqpoint{4.289775in}{2.343941in}}%
\pgfpathlineto{\pgfqpoint{4.290796in}{2.364777in}}%
\pgfpathlineto{\pgfqpoint{4.291867in}{2.336566in}}%
\pgfpathlineto{\pgfqpoint{4.291769in}{2.406590in}}%
\pgfpathlineto{\pgfqpoint{4.291892in}{2.362279in}}%
\pgfpathlineto{\pgfqpoint{4.292359in}{2.410405in}}%
\pgfpathlineto{\pgfqpoint{4.292901in}{2.350730in}}%
\pgfpathlineto{\pgfqpoint{4.292999in}{2.369307in}}%
\pgfpathlineto{\pgfqpoint{4.294033in}{2.419099in}}%
\pgfpathlineto{\pgfqpoint{4.293947in}{2.350901in}}%
\pgfpathlineto{\pgfqpoint{4.294156in}{2.388120in}}%
\pgfpathlineto{\pgfqpoint{4.294821in}{2.345959in}}%
\pgfpathlineto{\pgfqpoint{4.294735in}{2.423739in}}%
\pgfpathlineto{\pgfqpoint{4.295276in}{2.374260in}}%
\pgfpathlineto{\pgfqpoint{4.295449in}{2.409746in}}%
\pgfpathlineto{\pgfqpoint{4.296015in}{2.346447in}}%
\pgfpathlineto{\pgfqpoint{4.296396in}{2.389598in}}%
\pgfpathlineto{\pgfqpoint{4.296593in}{2.349267in}}%
\pgfpathlineto{\pgfqpoint{4.296975in}{2.410475in}}%
\pgfpathlineto{\pgfqpoint{4.297504in}{2.376308in}}%
\pgfpathlineto{\pgfqpoint{4.297553in}{2.415898in}}%
\pgfpathlineto{\pgfqpoint{4.297787in}{2.348888in}}%
\pgfpathlineto{\pgfqpoint{4.298612in}{2.375524in}}%
\pgfpathlineto{\pgfqpoint{4.299547in}{2.348721in}}%
\pgfpathlineto{\pgfqpoint{4.299461in}{2.415190in}}%
\pgfpathlineto{\pgfqpoint{4.299744in}{2.367123in}}%
\pgfpathlineto{\pgfqpoint{4.300519in}{2.407224in}}%
\pgfpathlineto{\pgfqpoint{4.300236in}{2.349076in}}%
\pgfpathlineto{\pgfqpoint{4.300864in}{2.378069in}}%
\pgfpathlineto{\pgfqpoint{4.301332in}{2.340547in}}%
\pgfpathlineto{\pgfqpoint{4.301713in}{2.420525in}}%
\pgfpathlineto{\pgfqpoint{4.302009in}{2.354880in}}%
\pgfpathlineto{\pgfqpoint{4.302415in}{2.426957in}}%
\pgfpathlineto{\pgfqpoint{4.302722in}{2.347149in}}%
\pgfpathlineto{\pgfqpoint{4.303165in}{2.378721in}}%
\pgfpathlineto{\pgfqpoint{4.303485in}{2.410048in}}%
\pgfpathlineto{\pgfqpoint{4.304285in}{2.345604in}}%
\pgfpathlineto{\pgfqpoint{4.305233in}{2.407221in}}%
\pgfpathlineto{\pgfqpoint{4.305418in}{2.374935in}}%
\pgfpathlineto{\pgfqpoint{4.305467in}{2.340200in}}%
\pgfpathlineto{\pgfqpoint{4.306427in}{2.414009in}}%
\pgfpathlineto{\pgfqpoint{4.306525in}{2.365967in}}%
\pgfpathlineto{\pgfqpoint{4.307141in}{2.409052in}}%
\pgfpathlineto{\pgfqpoint{4.306907in}{2.344775in}}%
\pgfpathlineto{\pgfqpoint{4.307645in}{2.387070in}}%
\pgfpathlineto{\pgfqpoint{4.308285in}{2.348368in}}%
\pgfpathlineto{\pgfqpoint{4.308199in}{2.405719in}}%
\pgfpathlineto{\pgfqpoint{4.308765in}{2.373394in}}%
\pgfpathlineto{\pgfqpoint{4.309393in}{2.427525in}}%
\pgfpathlineto{\pgfqpoint{4.309012in}{2.344100in}}%
\pgfpathlineto{\pgfqpoint{4.309848in}{2.354239in}}%
\pgfpathlineto{\pgfqpoint{4.309861in}{2.353535in}}%
\pgfpathlineto{\pgfqpoint{4.310070in}{2.388926in}}%
\pgfpathlineto{\pgfqpoint{4.310095in}{2.413526in}}%
\pgfpathlineto{\pgfqpoint{4.311067in}{2.346437in}}%
\pgfpathlineto{\pgfqpoint{4.311178in}{2.397273in}}%
\pgfpathlineto{\pgfqpoint{4.311965in}{2.346856in}}%
\pgfpathlineto{\pgfqpoint{4.312199in}{2.399012in}}%
\pgfpathlineto{\pgfqpoint{4.312298in}{2.378243in}}%
\pgfpathlineto{\pgfqpoint{4.312347in}{2.412229in}}%
\pgfpathlineto{\pgfqpoint{4.313147in}{2.347875in}}%
\pgfpathlineto{\pgfqpoint{4.313405in}{2.377192in}}%
\pgfpathlineto{\pgfqpoint{4.314181in}{2.341142in}}%
\pgfpathlineto{\pgfqpoint{4.314107in}{2.408532in}}%
\pgfpathlineto{\pgfqpoint{4.314267in}{2.396323in}}%
\pgfpathlineto{\pgfqpoint{4.314968in}{2.410289in}}%
\pgfpathlineto{\pgfqpoint{4.314759in}{2.346486in}}%
\pgfpathlineto{\pgfqpoint{4.315239in}{2.362200in}}%
\pgfpathlineto{\pgfqpoint{4.315953in}{2.347513in}}%
\pgfpathlineto{\pgfqpoint{4.315682in}{2.407009in}}%
\pgfpathlineto{\pgfqpoint{4.316322in}{2.373915in}}%
\pgfpathlineto{\pgfqpoint{4.317073in}{2.428588in}}%
\pgfpathlineto{\pgfqpoint{4.316679in}{2.344551in}}%
\pgfpathlineto{\pgfqpoint{4.317442in}{2.387115in}}%
\pgfpathlineto{\pgfqpoint{4.318070in}{2.344875in}}%
\pgfpathlineto{\pgfqpoint{4.317775in}{2.418652in}}%
\pgfpathlineto{\pgfqpoint{4.318464in}{2.397020in}}%
\pgfpathlineto{\pgfqpoint{4.318488in}{2.414801in}}%
\pgfpathlineto{\pgfqpoint{4.319473in}{2.342392in}}%
\pgfpathlineto{\pgfqpoint{4.319559in}{2.382599in}}%
\pgfpathlineto{\pgfqpoint{4.319768in}{2.347499in}}%
\pgfpathlineto{\pgfqpoint{4.320593in}{2.419681in}}%
\pgfpathlineto{\pgfqpoint{4.320679in}{2.376601in}}%
\pgfpathlineto{\pgfqpoint{4.321282in}{2.409469in}}%
\pgfpathlineto{\pgfqpoint{4.321024in}{2.342042in}}%
\pgfpathlineto{\pgfqpoint{4.321799in}{2.398726in}}%
\pgfpathlineto{\pgfqpoint{4.322575in}{2.342098in}}%
\pgfpathlineto{\pgfqpoint{4.321959in}{2.405238in}}%
\pgfpathlineto{\pgfqpoint{4.322932in}{2.364429in}}%
\pgfpathlineto{\pgfqpoint{4.323350in}{2.409312in}}%
\pgfpathlineto{\pgfqpoint{4.323965in}{2.341241in}}%
\pgfpathlineto{\pgfqpoint{4.324088in}{2.399216in}}%
\pgfpathlineto{\pgfqpoint{4.325221in}{2.341769in}}%
\pgfpathlineto{\pgfqpoint{4.324741in}{2.412703in}}%
\pgfpathlineto{\pgfqpoint{4.325233in}{2.350950in}}%
\pgfpathlineto{\pgfqpoint{4.326156in}{2.417288in}}%
\pgfpathlineto{\pgfqpoint{4.325368in}{2.339130in}}%
\pgfpathlineto{\pgfqpoint{4.326365in}{2.377631in}}%
\pgfpathlineto{\pgfqpoint{4.326771in}{2.343587in}}%
\pgfpathlineto{\pgfqpoint{4.326870in}{2.421304in}}%
\pgfpathlineto{\pgfqpoint{4.327498in}{2.363383in}}%
\pgfpathlineto{\pgfqpoint{4.328273in}{2.421210in}}%
\pgfpathlineto{\pgfqpoint{4.328162in}{2.329910in}}%
\pgfpathlineto{\pgfqpoint{4.328642in}{2.389372in}}%
\pgfpathlineto{\pgfqpoint{4.328741in}{2.338834in}}%
\pgfpathlineto{\pgfqpoint{4.328962in}{2.412572in}}%
\pgfpathlineto{\pgfqpoint{4.329775in}{2.372407in}}%
\pgfpathlineto{\pgfqpoint{4.329811in}{2.405539in}}%
\pgfpathlineto{\pgfqpoint{4.330242in}{2.342941in}}%
\pgfpathlineto{\pgfqpoint{4.330882in}{2.385924in}}%
\pgfpathlineto{\pgfqpoint{4.330956in}{2.340731in}}%
\pgfpathlineto{\pgfqpoint{4.331768in}{2.403496in}}%
\pgfpathlineto{\pgfqpoint{4.332015in}{2.364132in}}%
\pgfpathlineto{\pgfqpoint{4.333147in}{2.417235in}}%
\pgfpathlineto{\pgfqpoint{4.333048in}{2.345706in}}%
\pgfpathlineto{\pgfqpoint{4.333159in}{2.408583in}}%
\pgfpathlineto{\pgfqpoint{4.333455in}{2.338161in}}%
\pgfpathlineto{\pgfqpoint{4.333861in}{2.408699in}}%
\pgfpathlineto{\pgfqpoint{4.334304in}{2.356377in}}%
\pgfpathlineto{\pgfqpoint{4.334538in}{2.411422in}}%
\pgfpathlineto{\pgfqpoint{4.334636in}{2.346504in}}%
\pgfpathlineto{\pgfqpoint{4.335448in}{2.382703in}}%
\pgfpathlineto{\pgfqpoint{4.336408in}{2.329119in}}%
\pgfpathlineto{\pgfqpoint{4.335941in}{2.407928in}}%
\pgfpathlineto{\pgfqpoint{4.336581in}{2.366952in}}%
\pgfpathlineto{\pgfqpoint{4.337344in}{2.415278in}}%
\pgfpathlineto{\pgfqpoint{4.337233in}{2.340729in}}%
\pgfpathlineto{\pgfqpoint{4.337701in}{2.392133in}}%
\pgfpathlineto{\pgfqpoint{4.338624in}{2.341558in}}%
\pgfpathlineto{\pgfqpoint{4.338058in}{2.403818in}}%
\pgfpathlineto{\pgfqpoint{4.338833in}{2.372206in}}%
\pgfpathlineto{\pgfqpoint{4.339436in}{2.415073in}}%
\pgfpathlineto{\pgfqpoint{4.339350in}{2.318759in}}%
\pgfpathlineto{\pgfqpoint{4.339916in}{2.351067in}}%
\pgfpathlineto{\pgfqpoint{4.339941in}{2.337813in}}%
\pgfpathlineto{\pgfqpoint{4.340285in}{2.415703in}}%
\pgfpathlineto{\pgfqpoint{4.340975in}{2.393849in}}%
\pgfpathlineto{\pgfqpoint{4.341541in}{2.407033in}}%
\pgfpathlineto{\pgfqpoint{4.341430in}{2.337162in}}%
\pgfpathlineto{\pgfqpoint{4.342008in}{2.344211in}}%
\pgfpathlineto{\pgfqpoint{4.342144in}{2.337995in}}%
\pgfpathlineto{\pgfqpoint{4.342378in}{2.408953in}}%
\pgfpathlineto{\pgfqpoint{4.343079in}{2.411113in}}%
\pgfpathlineto{\pgfqpoint{4.342845in}{2.345184in}}%
\pgfpathlineto{\pgfqpoint{4.343239in}{2.374561in}}%
\pgfpathlineto{\pgfqpoint{4.343522in}{2.329523in}}%
\pgfpathlineto{\pgfqpoint{4.343805in}{2.408232in}}%
\pgfpathlineto{\pgfqpoint{4.344310in}{2.388058in}}%
\pgfpathlineto{\pgfqpoint{4.345048in}{2.410170in}}%
\pgfpathlineto{\pgfqpoint{4.344950in}{2.328480in}}%
\pgfpathlineto{\pgfqpoint{4.345405in}{2.383815in}}%
\pgfpathlineto{\pgfqpoint{4.345442in}{2.388679in}}%
\pgfpathlineto{\pgfqpoint{4.345467in}{2.367489in}}%
\pgfpathlineto{\pgfqpoint{4.346341in}{2.316736in}}%
\pgfpathlineto{\pgfqpoint{4.345713in}{2.405309in}}%
\pgfpathlineto{\pgfqpoint{4.346550in}{2.378812in}}%
\pgfpathlineto{\pgfqpoint{4.346599in}{2.411663in}}%
\pgfpathlineto{\pgfqpoint{4.347042in}{2.327027in}}%
\pgfpathlineto{\pgfqpoint{4.347608in}{2.339454in}}%
\pgfpathlineto{\pgfqpoint{4.347633in}{2.319545in}}%
\pgfpathlineto{\pgfqpoint{4.348064in}{2.419662in}}%
\pgfpathlineto{\pgfqpoint{4.348642in}{2.393594in}}%
\pgfpathlineto{\pgfqpoint{4.349245in}{2.427252in}}%
\pgfpathlineto{\pgfqpoint{4.349134in}{2.329340in}}%
\pgfpathlineto{\pgfqpoint{4.349701in}{2.361646in}}%
\pgfpathlineto{\pgfqpoint{4.350525in}{2.325697in}}%
\pgfpathlineto{\pgfqpoint{4.350094in}{2.426928in}}%
\pgfpathlineto{\pgfqpoint{4.350771in}{2.398738in}}%
\pgfpathlineto{\pgfqpoint{4.351522in}{2.423246in}}%
\pgfpathlineto{\pgfqpoint{4.351830in}{2.309587in}}%
\pgfpathlineto{\pgfqpoint{4.352187in}{2.440643in}}%
\pgfpathlineto{\pgfqpoint{4.352999in}{2.373082in}}%
\pgfpathlineto{\pgfqpoint{4.353639in}{2.423325in}}%
\pgfpathlineto{\pgfqpoint{4.353233in}{2.314220in}}%
\pgfpathlineto{\pgfqpoint{4.353898in}{2.370205in}}%
\pgfpathlineto{\pgfqpoint{4.354045in}{2.298765in}}%
\pgfpathlineto{\pgfqpoint{4.354304in}{2.423803in}}%
\pgfpathlineto{\pgfqpoint{4.354968in}{2.410504in}}%
\pgfpathlineto{\pgfqpoint{4.355448in}{2.303884in}}%
\pgfpathlineto{\pgfqpoint{4.355707in}{2.440658in}}%
\pgfpathlineto{\pgfqpoint{4.356199in}{2.356718in}}%
\pgfpathlineto{\pgfqpoint{4.357159in}{2.431906in}}%
\pgfpathlineto{\pgfqpoint{4.356765in}{2.325114in}}%
\pgfpathlineto{\pgfqpoint{4.357319in}{2.382683in}}%
\pgfpathlineto{\pgfqpoint{4.358267in}{2.298316in}}%
\pgfpathlineto{\pgfqpoint{4.357824in}{2.427910in}}%
\pgfpathlineto{\pgfqpoint{4.358427in}{2.384702in}}%
\pgfpathlineto{\pgfqpoint{4.358858in}{2.287762in}}%
\pgfpathlineto{\pgfqpoint{4.359214in}{2.453540in}}%
\pgfpathlineto{\pgfqpoint{4.359596in}{2.335653in}}%
\pgfpathlineto{\pgfqpoint{4.359793in}{2.438432in}}%
\pgfpathlineto{\pgfqpoint{4.359694in}{2.313840in}}%
\pgfpathlineto{\pgfqpoint{4.360765in}{2.391076in}}%
\pgfpathlineto{\pgfqpoint{4.361787in}{2.289681in}}%
\pgfpathlineto{\pgfqpoint{4.361331in}{2.429563in}}%
\pgfpathlineto{\pgfqpoint{4.361848in}{2.407209in}}%
\pgfpathlineto{\pgfqpoint{4.362734in}{2.444222in}}%
\pgfpathlineto{\pgfqpoint{4.362378in}{2.287448in}}%
\pgfpathlineto{\pgfqpoint{4.362882in}{2.359996in}}%
\pgfpathlineto{\pgfqpoint{4.363202in}{2.302826in}}%
\pgfpathlineto{\pgfqpoint{4.363313in}{2.431791in}}%
\pgfpathlineto{\pgfqpoint{4.363941in}{2.360840in}}%
\pgfpathlineto{\pgfqpoint{4.364851in}{2.438444in}}%
\pgfpathlineto{\pgfqpoint{4.364605in}{2.296246in}}%
\pgfpathlineto{\pgfqpoint{4.365061in}{2.385474in}}%
\pgfpathlineto{\pgfqpoint{4.365196in}{2.297299in}}%
\pgfpathlineto{\pgfqpoint{4.365417in}{2.433772in}}%
\pgfpathlineto{\pgfqpoint{4.366107in}{2.420945in}}%
\pgfpathlineto{\pgfqpoint{4.366254in}{2.452481in}}%
\pgfpathlineto{\pgfqpoint{4.366722in}{2.295384in}}%
\pgfpathlineto{\pgfqpoint{4.367116in}{2.380576in}}%
\pgfpathlineto{\pgfqpoint{4.368125in}{2.284372in}}%
\pgfpathlineto{\pgfqpoint{4.367670in}{2.440833in}}%
\pgfpathlineto{\pgfqpoint{4.368187in}{2.388759in}}%
\pgfpathlineto{\pgfqpoint{4.368371in}{2.448112in}}%
\pgfpathlineto{\pgfqpoint{4.368716in}{2.290798in}}%
\pgfpathlineto{\pgfqpoint{4.369270in}{2.351643in}}%
\pgfpathlineto{\pgfqpoint{4.369282in}{2.352103in}}%
\pgfpathlineto{\pgfqpoint{4.369381in}{2.319362in}}%
\pgfpathlineto{\pgfqpoint{4.370242in}{2.293115in}}%
\pgfpathlineto{\pgfqpoint{4.369774in}{2.438047in}}%
\pgfpathlineto{\pgfqpoint{4.370427in}{2.393663in}}%
\pgfpathlineto{\pgfqpoint{4.371190in}{2.441053in}}%
\pgfpathlineto{\pgfqpoint{4.370821in}{2.297409in}}%
\pgfpathlineto{\pgfqpoint{4.371424in}{2.346669in}}%
\pgfpathlineto{\pgfqpoint{4.372347in}{2.288749in}}%
\pgfpathlineto{\pgfqpoint{4.371891in}{2.455736in}}%
\pgfpathlineto{\pgfqpoint{4.372396in}{2.374007in}}%
\pgfpathlineto{\pgfqpoint{4.373294in}{2.452915in}}%
\pgfpathlineto{\pgfqpoint{4.372925in}{2.281219in}}%
\pgfpathlineto{\pgfqpoint{4.373491in}{2.340313in}}%
\pgfpathlineto{\pgfqpoint{4.373861in}{2.442030in}}%
\pgfpathlineto{\pgfqpoint{4.373750in}{2.286464in}}%
\pgfpathlineto{\pgfqpoint{4.374304in}{2.324778in}}%
\pgfpathlineto{\pgfqpoint{4.375030in}{2.280689in}}%
\pgfpathlineto{\pgfqpoint{4.374722in}{2.448562in}}%
\pgfpathlineto{\pgfqpoint{4.375350in}{2.415072in}}%
\pgfpathlineto{\pgfqpoint{4.375399in}{2.456315in}}%
\pgfpathlineto{\pgfqpoint{4.375744in}{2.278044in}}%
\pgfpathlineto{\pgfqpoint{4.376359in}{2.338066in}}%
\pgfpathlineto{\pgfqpoint{4.376433in}{2.267974in}}%
\pgfpathlineto{\pgfqpoint{4.376814in}{2.455583in}}%
\pgfpathlineto{\pgfqpoint{4.377331in}{2.411833in}}%
\pgfpathlineto{\pgfqpoint{4.378082in}{2.490336in}}%
\pgfpathlineto{\pgfqpoint{4.377848in}{2.245417in}}%
\pgfpathlineto{\pgfqpoint{4.378390in}{2.324973in}}%
\pgfpathlineto{\pgfqpoint{4.379239in}{2.271491in}}%
\pgfpathlineto{\pgfqpoint{4.378907in}{2.459931in}}%
\pgfpathlineto{\pgfqpoint{4.379399in}{2.351227in}}%
\pgfpathlineto{\pgfqpoint{4.379461in}{2.492907in}}%
\pgfpathlineto{\pgfqpoint{4.379928in}{2.295959in}}%
\pgfpathlineto{\pgfqpoint{4.380507in}{2.351356in}}%
\pgfpathlineto{\pgfqpoint{4.381565in}{2.455958in}}%
\pgfpathlineto{\pgfqpoint{4.380630in}{2.308073in}}%
\pgfpathlineto{\pgfqpoint{4.381701in}{2.399533in}}%
\pgfpathlineto{\pgfqpoint{4.382008in}{2.305614in}}%
\pgfpathlineto{\pgfqpoint{4.382279in}{2.436020in}}%
\pgfpathlineto{\pgfqpoint{4.382882in}{2.351808in}}%
\pgfpathlineto{\pgfqpoint{4.383042in}{2.435683in}}%
\pgfpathlineto{\pgfqpoint{4.383387in}{2.311561in}}%
\pgfpathlineto{\pgfqpoint{4.383990in}{2.353998in}}%
\pgfpathlineto{\pgfqpoint{4.384421in}{2.413176in}}%
\pgfpathlineto{\pgfqpoint{4.384076in}{2.306672in}}%
\pgfpathlineto{\pgfqpoint{4.385147in}{2.383392in}}%
\pgfpathlineto{\pgfqpoint{4.385479in}{2.336468in}}%
\pgfpathlineto{\pgfqpoint{4.385639in}{2.397266in}}%
\pgfpathlineto{\pgfqpoint{4.386217in}{2.378309in}}%
\pgfpathlineto{\pgfqpoint{4.386784in}{2.399847in}}%
\pgfpathlineto{\pgfqpoint{4.386624in}{2.343941in}}%
\pgfpathlineto{\pgfqpoint{4.387313in}{2.371038in}}%
\pgfpathlineto{\pgfqpoint{4.387547in}{2.357110in}}%
\pgfpathlineto{\pgfqpoint{4.387473in}{2.392038in}}%
\pgfpathlineto{\pgfqpoint{4.388420in}{2.368213in}}%
\pgfpathlineto{\pgfqpoint{4.389159in}{2.396665in}}%
\pgfpathlineto{\pgfqpoint{4.389233in}{2.354501in}}%
\pgfpathlineto{\pgfqpoint{4.389553in}{2.382136in}}%
\pgfpathlineto{\pgfqpoint{4.390414in}{2.425954in}}%
\pgfpathlineto{\pgfqpoint{4.390205in}{2.341479in}}%
\pgfpathlineto{\pgfqpoint{4.390599in}{2.380051in}}%
\pgfpathlineto{\pgfqpoint{4.391325in}{2.323099in}}%
\pgfpathlineto{\pgfqpoint{4.390980in}{2.430186in}}%
\pgfpathlineto{\pgfqpoint{4.391707in}{2.371025in}}%
\pgfpathlineto{\pgfqpoint{4.391879in}{2.337489in}}%
\pgfpathlineto{\pgfqpoint{4.392445in}{2.406275in}}%
\pgfpathlineto{\pgfqpoint{4.392839in}{2.349086in}}%
\pgfpathlineto{\pgfqpoint{4.393147in}{2.428936in}}%
\pgfpathlineto{\pgfqpoint{4.393454in}{2.327762in}}%
\pgfpathlineto{\pgfqpoint{4.393947in}{2.349831in}}%
\pgfpathlineto{\pgfqpoint{4.394882in}{2.299410in}}%
\pgfpathlineto{\pgfqpoint{4.394537in}{2.461305in}}%
\pgfpathlineto{\pgfqpoint{4.395005in}{2.356792in}}%
\pgfpathlineto{\pgfqpoint{4.395042in}{2.421546in}}%
\pgfpathlineto{\pgfqpoint{4.395424in}{2.331010in}}%
\pgfpathlineto{\pgfqpoint{4.396088in}{2.341716in}}%
\pgfpathlineto{\pgfqpoint{4.396974in}{2.311860in}}%
\pgfpathlineto{\pgfqpoint{4.396359in}{2.426095in}}%
\pgfpathlineto{\pgfqpoint{4.397171in}{2.370382in}}%
\pgfpathlineto{\pgfqpoint{4.398070in}{2.473938in}}%
\pgfpathlineto{\pgfqpoint{4.397824in}{2.332259in}}%
\pgfpathlineto{\pgfqpoint{4.398217in}{2.371734in}}%
\pgfpathlineto{\pgfqpoint{4.399067in}{2.312082in}}%
\pgfpathlineto{\pgfqpoint{4.398820in}{2.442361in}}%
\pgfpathlineto{\pgfqpoint{4.399288in}{2.392810in}}%
\pgfpathlineto{\pgfqpoint{4.400137in}{2.447271in}}%
\pgfpathlineto{\pgfqpoint{4.399793in}{2.321670in}}%
\pgfpathlineto{\pgfqpoint{4.400371in}{2.380312in}}%
\pgfpathlineto{\pgfqpoint{4.401171in}{2.322402in}}%
\pgfpathlineto{\pgfqpoint{4.400580in}{2.435072in}}%
\pgfpathlineto{\pgfqpoint{4.401467in}{2.392163in}}%
\pgfpathlineto{\pgfqpoint{4.402427in}{2.321007in}}%
\pgfpathlineto{\pgfqpoint{4.402057in}{2.419864in}}%
\pgfpathlineto{\pgfqpoint{4.402562in}{2.396818in}}%
\pgfpathlineto{\pgfqpoint{4.403091in}{2.427506in}}%
\pgfpathlineto{\pgfqpoint{4.402956in}{2.331022in}}%
\pgfpathlineto{\pgfqpoint{4.403510in}{2.351802in}}%
\pgfpathlineto{\pgfqpoint{4.403522in}{2.351447in}}%
\pgfpathlineto{\pgfqpoint{4.403584in}{2.379407in}}%
\pgfpathlineto{\pgfqpoint{4.403707in}{2.371480in}}%
\pgfpathlineto{\pgfqpoint{4.404334in}{2.432509in}}%
\pgfpathlineto{\pgfqpoint{4.404076in}{2.325679in}}%
\pgfpathlineto{\pgfqpoint{4.404777in}{2.337800in}}%
\pgfpathlineto{\pgfqpoint{4.404913in}{2.320285in}}%
\pgfpathlineto{\pgfqpoint{4.405590in}{2.446220in}}%
\pgfpathlineto{\pgfqpoint{4.405774in}{2.382260in}}%
\pgfpathlineto{\pgfqpoint{4.406402in}{2.455610in}}%
\pgfpathlineto{\pgfqpoint{4.406057in}{2.309362in}}%
\pgfpathlineto{\pgfqpoint{4.406882in}{2.379104in}}%
\pgfpathlineto{\pgfqpoint{4.407436in}{2.307480in}}%
\pgfpathlineto{\pgfqpoint{4.407645in}{2.445844in}}%
\pgfpathlineto{\pgfqpoint{4.408027in}{2.342860in}}%
\pgfpathlineto{\pgfqpoint{4.408088in}{2.432117in}}%
\pgfpathlineto{\pgfqpoint{4.408679in}{2.330468in}}%
\pgfpathlineto{\pgfqpoint{4.409159in}{2.371566in}}%
\pgfpathlineto{\pgfqpoint{4.409208in}{2.362068in}}%
\pgfpathlineto{\pgfqpoint{4.409319in}{2.438841in}}%
\pgfpathlineto{\pgfqpoint{4.409331in}{2.451007in}}%
\pgfpathlineto{\pgfqpoint{4.409491in}{2.324467in}}%
\pgfpathlineto{\pgfqpoint{4.410303in}{2.327033in}}%
\pgfpathlineto{\pgfqpoint{4.410328in}{2.318869in}}%
\pgfpathlineto{\pgfqpoint{4.410587in}{2.453115in}}%
\pgfpathlineto{\pgfqpoint{4.411288in}{2.392084in}}%
\pgfpathlineto{\pgfqpoint{4.411325in}{2.432571in}}%
\pgfpathlineto{\pgfqpoint{4.411707in}{2.325941in}}%
\pgfpathlineto{\pgfqpoint{4.412359in}{2.352911in}}%
\pgfpathlineto{\pgfqpoint{4.412433in}{2.321461in}}%
\pgfpathlineto{\pgfqpoint{4.412642in}{2.434803in}}%
\pgfpathlineto{\pgfqpoint{4.413282in}{2.373068in}}%
\pgfpathlineto{\pgfqpoint{4.413897in}{2.433431in}}%
\pgfpathlineto{\pgfqpoint{4.413688in}{2.309161in}}%
\pgfpathlineto{\pgfqpoint{4.414390in}{2.385380in}}%
\pgfpathlineto{\pgfqpoint{4.415350in}{2.324119in}}%
\pgfpathlineto{\pgfqpoint{4.415153in}{2.441569in}}%
\pgfpathlineto{\pgfqpoint{4.415510in}{2.361194in}}%
\pgfpathlineto{\pgfqpoint{4.416396in}{2.455502in}}%
\pgfpathlineto{\pgfqpoint{4.416199in}{2.314495in}}%
\pgfpathlineto{\pgfqpoint{4.416593in}{2.336248in}}%
\pgfpathlineto{\pgfqpoint{4.417454in}{2.305014in}}%
\pgfpathlineto{\pgfqpoint{4.417085in}{2.426953in}}%
\pgfpathlineto{\pgfqpoint{4.417503in}{2.378568in}}%
\pgfpathlineto{\pgfqpoint{4.417651in}{2.467803in}}%
\pgfpathlineto{\pgfqpoint{4.417971in}{2.321008in}}%
\pgfpathlineto{\pgfqpoint{4.418537in}{2.365909in}}%
\pgfpathlineto{\pgfqpoint{4.418697in}{2.308965in}}%
\pgfpathlineto{\pgfqpoint{4.418919in}{2.457878in}}%
\pgfpathlineto{\pgfqpoint{4.419596in}{2.427717in}}%
\pgfpathlineto{\pgfqpoint{4.419953in}{2.305012in}}%
\pgfpathlineto{\pgfqpoint{4.420162in}{2.448007in}}%
\pgfpathlineto{\pgfqpoint{4.420790in}{2.368943in}}%
\pgfpathlineto{\pgfqpoint{4.421417in}{2.449755in}}%
\pgfpathlineto{\pgfqpoint{4.421602in}{2.320216in}}%
\pgfpathlineto{\pgfqpoint{4.421910in}{2.393518in}}%
\pgfpathlineto{\pgfqpoint{4.422439in}{2.309491in}}%
\pgfpathlineto{\pgfqpoint{4.422673in}{2.454660in}}%
\pgfpathlineto{\pgfqpoint{4.423067in}{2.355866in}}%
\pgfpathlineto{\pgfqpoint{4.423928in}{2.456028in}}%
\pgfpathlineto{\pgfqpoint{4.423719in}{2.299257in}}%
\pgfpathlineto{\pgfqpoint{4.424174in}{2.352998in}}%
\pgfpathlineto{\pgfqpoint{4.424593in}{2.442844in}}%
\pgfpathlineto{\pgfqpoint{4.424839in}{2.324737in}}%
\pgfpathlineto{\pgfqpoint{4.424962in}{2.297597in}}%
\pgfpathlineto{\pgfqpoint{4.425171in}{2.457701in}}%
\pgfpathlineto{\pgfqpoint{4.425614in}{2.417035in}}%
\pgfpathlineto{\pgfqpoint{4.425860in}{2.435220in}}%
\pgfpathlineto{\pgfqpoint{4.425787in}{2.350803in}}%
\pgfpathlineto{\pgfqpoint{4.426057in}{2.362998in}}%
\pgfpathlineto{\pgfqpoint{4.426217in}{2.308990in}}%
\pgfpathlineto{\pgfqpoint{4.426427in}{2.460999in}}%
\pgfpathlineto{\pgfqpoint{4.427103in}{2.428519in}}%
\pgfpathlineto{\pgfqpoint{4.427460in}{2.299122in}}%
\pgfpathlineto{\pgfqpoint{4.427670in}{2.459782in}}%
\pgfpathlineto{\pgfqpoint{4.428334in}{2.406753in}}%
\pgfpathlineto{\pgfqpoint{4.428925in}{2.470088in}}%
\pgfpathlineto{\pgfqpoint{4.428716in}{2.303039in}}%
\pgfpathlineto{\pgfqpoint{4.429430in}{2.397678in}}%
\pgfpathlineto{\pgfqpoint{4.429971in}{2.294041in}}%
\pgfpathlineto{\pgfqpoint{4.430180in}{2.470765in}}%
\pgfpathlineto{\pgfqpoint{4.430586in}{2.380113in}}%
\pgfpathlineto{\pgfqpoint{4.431423in}{2.448102in}}%
\pgfpathlineto{\pgfqpoint{4.431226in}{2.298249in}}%
\pgfpathlineto{\pgfqpoint{4.431682in}{2.354688in}}%
\pgfpathlineto{\pgfqpoint{4.431743in}{2.332382in}}%
\pgfpathlineto{\pgfqpoint{4.431866in}{2.437670in}}%
\pgfpathlineto{\pgfqpoint{4.432679in}{2.450286in}}%
\pgfpathlineto{\pgfqpoint{4.432470in}{2.308112in}}%
\pgfpathlineto{\pgfqpoint{4.432851in}{2.339038in}}%
\pgfpathlineto{\pgfqpoint{4.433713in}{2.295975in}}%
\pgfpathlineto{\pgfqpoint{4.433134in}{2.443510in}}%
\pgfpathlineto{\pgfqpoint{4.433848in}{2.426853in}}%
\pgfpathlineto{\pgfqpoint{4.433922in}{2.477985in}}%
\pgfpathlineto{\pgfqpoint{4.434254in}{2.306035in}}%
\pgfpathlineto{\pgfqpoint{4.434882in}{2.341871in}}%
\pgfpathlineto{\pgfqpoint{4.434956in}{2.291886in}}%
\pgfpathlineto{\pgfqpoint{4.435177in}{2.478483in}}%
\pgfpathlineto{\pgfqpoint{4.435830in}{2.401407in}}%
\pgfpathlineto{\pgfqpoint{4.436433in}{2.463203in}}%
\pgfpathlineto{\pgfqpoint{4.436211in}{2.288026in}}%
\pgfpathlineto{\pgfqpoint{4.436937in}{2.400338in}}%
\pgfpathlineto{\pgfqpoint{4.437466in}{2.305961in}}%
\pgfpathlineto{\pgfqpoint{4.437676in}{2.465036in}}%
\pgfpathlineto{\pgfqpoint{4.438082in}{2.381172in}}%
\pgfpathlineto{\pgfqpoint{4.438931in}{2.470046in}}%
\pgfpathlineto{\pgfqpoint{4.438710in}{2.301697in}}%
\pgfpathlineto{\pgfqpoint{4.439165in}{2.351007in}}%
\pgfpathlineto{\pgfqpoint{4.440174in}{2.483453in}}%
\pgfpathlineto{\pgfqpoint{4.439965in}{2.296715in}}%
\pgfpathlineto{\pgfqpoint{4.440334in}{2.367204in}}%
\pgfpathlineto{\pgfqpoint{4.441208in}{2.291203in}}%
\pgfpathlineto{\pgfqpoint{4.440876in}{2.438434in}}%
\pgfpathlineto{\pgfqpoint{4.441393in}{2.435464in}}%
\pgfpathlineto{\pgfqpoint{4.441430in}{2.482867in}}%
\pgfpathlineto{\pgfqpoint{4.441762in}{2.305099in}}%
\pgfpathlineto{\pgfqpoint{4.442402in}{2.343698in}}%
\pgfpathlineto{\pgfqpoint{4.442463in}{2.284649in}}%
\pgfpathlineto{\pgfqpoint{4.442685in}{2.472263in}}%
\pgfpathlineto{\pgfqpoint{4.443325in}{2.392659in}}%
\pgfpathlineto{\pgfqpoint{4.443928in}{2.459629in}}%
\pgfpathlineto{\pgfqpoint{4.443719in}{2.294875in}}%
\pgfpathlineto{\pgfqpoint{4.444433in}{2.394032in}}%
\pgfpathlineto{\pgfqpoint{4.444457in}{2.395430in}}%
\pgfpathlineto{\pgfqpoint{4.444519in}{2.346838in}}%
\pgfpathlineto{\pgfqpoint{4.444962in}{2.305753in}}%
\pgfpathlineto{\pgfqpoint{4.445183in}{2.464017in}}%
\pgfpathlineto{\pgfqpoint{4.445577in}{2.369746in}}%
\pgfpathlineto{\pgfqpoint{4.446439in}{2.466830in}}%
\pgfpathlineto{\pgfqpoint{4.446106in}{2.309373in}}%
\pgfpathlineto{\pgfqpoint{4.446673in}{2.348293in}}%
\pgfpathlineto{\pgfqpoint{4.447103in}{2.436708in}}%
\pgfpathlineto{\pgfqpoint{4.446759in}{2.311499in}}%
\pgfpathlineto{\pgfqpoint{4.447423in}{2.321605in}}%
\pgfpathlineto{\pgfqpoint{4.447460in}{2.295834in}}%
\pgfpathlineto{\pgfqpoint{4.447694in}{2.466525in}}%
\pgfpathlineto{\pgfqpoint{4.448396in}{2.418729in}}%
\pgfpathlineto{\pgfqpoint{4.448716in}{2.294576in}}%
\pgfpathlineto{\pgfqpoint{4.448937in}{2.462298in}}%
\pgfpathlineto{\pgfqpoint{4.449565in}{2.373648in}}%
\pgfpathlineto{\pgfqpoint{4.450193in}{2.436457in}}%
\pgfpathlineto{\pgfqpoint{4.449959in}{2.311319in}}%
\pgfpathlineto{\pgfqpoint{4.450685in}{2.396292in}}%
\pgfpathlineto{\pgfqpoint{4.451153in}{2.317265in}}%
\pgfpathlineto{\pgfqpoint{4.451448in}{2.446318in}}%
\pgfpathlineto{\pgfqpoint{4.451829in}{2.377022in}}%
\pgfpathlineto{\pgfqpoint{4.452666in}{2.452013in}}%
\pgfpathlineto{\pgfqpoint{4.452457in}{2.308756in}}%
\pgfpathlineto{\pgfqpoint{4.452925in}{2.366736in}}%
\pgfpathlineto{\pgfqpoint{4.453011in}{2.302761in}}%
\pgfpathlineto{\pgfqpoint{4.453897in}{2.441044in}}%
\pgfpathlineto{\pgfqpoint{4.453922in}{2.458894in}}%
\pgfpathlineto{\pgfqpoint{4.454869in}{2.309870in}}%
\pgfpathlineto{\pgfqpoint{4.454894in}{2.317019in}}%
\pgfpathlineto{\pgfqpoint{4.454943in}{2.299504in}}%
\pgfpathlineto{\pgfqpoint{4.455177in}{2.452456in}}%
\pgfpathlineto{\pgfqpoint{4.455682in}{2.393134in}}%
\pgfpathlineto{\pgfqpoint{4.456408in}{2.437638in}}%
\pgfpathlineto{\pgfqpoint{4.456186in}{2.290340in}}%
\pgfpathlineto{\pgfqpoint{4.456703in}{2.336448in}}%
\pgfpathlineto{\pgfqpoint{4.457466in}{2.298369in}}%
\pgfpathlineto{\pgfqpoint{4.457097in}{2.429954in}}%
\pgfpathlineto{\pgfqpoint{4.457639in}{2.426373in}}%
\pgfpathlineto{\pgfqpoint{4.457676in}{2.450449in}}%
\pgfpathlineto{\pgfqpoint{4.458008in}{2.312495in}}%
\pgfpathlineto{\pgfqpoint{4.458660in}{2.319683in}}%
\pgfpathlineto{\pgfqpoint{4.458685in}{2.303730in}}%
\pgfpathlineto{\pgfqpoint{4.458919in}{2.452480in}}%
\pgfpathlineto{\pgfqpoint{4.459694in}{2.384888in}}%
\pgfpathlineto{\pgfqpoint{4.460211in}{2.433138in}}%
\pgfpathlineto{\pgfqpoint{4.459916in}{2.302560in}}%
\pgfpathlineto{\pgfqpoint{4.460433in}{2.333889in}}%
\pgfpathlineto{\pgfqpoint{4.461208in}{2.299538in}}%
\pgfpathlineto{\pgfqpoint{4.460839in}{2.432806in}}%
\pgfpathlineto{\pgfqpoint{4.461319in}{2.398050in}}%
\pgfpathlineto{\pgfqpoint{4.461405in}{2.448506in}}%
\pgfpathlineto{\pgfqpoint{4.461749in}{2.308632in}}%
\pgfpathlineto{\pgfqpoint{4.462316in}{2.312588in}}%
\pgfpathlineto{\pgfqpoint{4.462414in}{2.305000in}}%
\pgfpathlineto{\pgfqpoint{4.462599in}{2.406744in}}%
\pgfpathlineto{\pgfqpoint{4.462685in}{2.469017in}}%
\pgfpathlineto{\pgfqpoint{4.462956in}{2.323882in}}%
\pgfpathlineto{\pgfqpoint{4.463608in}{2.333020in}}%
\pgfpathlineto{\pgfqpoint{4.463682in}{2.292043in}}%
\pgfpathlineto{\pgfqpoint{4.463805in}{2.376802in}}%
\pgfpathlineto{\pgfqpoint{4.463940in}{2.442187in}}%
\pgfpathlineto{\pgfqpoint{4.464211in}{2.325469in}}%
\pgfpathlineto{\pgfqpoint{4.464876in}{2.351375in}}%
\pgfpathlineto{\pgfqpoint{4.464937in}{2.292529in}}%
\pgfpathlineto{\pgfqpoint{4.465183in}{2.448137in}}%
\pgfpathlineto{\pgfqpoint{4.465823in}{2.422439in}}%
\pgfpathlineto{\pgfqpoint{4.465836in}{2.423924in}}%
\pgfpathlineto{\pgfqpoint{4.466106in}{2.330111in}}%
\pgfpathlineto{\pgfqpoint{4.466168in}{2.339231in}}%
\pgfpathlineto{\pgfqpoint{4.466709in}{2.321902in}}%
\pgfpathlineto{\pgfqpoint{4.466451in}{2.458776in}}%
\pgfpathlineto{\pgfqpoint{4.467239in}{2.371387in}}%
\pgfpathlineto{\pgfqpoint{4.467349in}{2.328222in}}%
\pgfpathlineto{\pgfqpoint{4.467719in}{2.425499in}}%
\pgfpathlineto{\pgfqpoint{4.468260in}{2.393889in}}%
\pgfpathlineto{\pgfqpoint{4.468334in}{2.428770in}}%
\pgfpathlineto{\pgfqpoint{4.468568in}{2.331984in}}%
\pgfpathlineto{\pgfqpoint{4.468605in}{2.301241in}}%
\pgfpathlineto{\pgfqpoint{4.468962in}{2.433093in}}%
\pgfpathlineto{\pgfqpoint{4.469614in}{2.399792in}}%
\pgfpathlineto{\pgfqpoint{4.470685in}{2.313169in}}%
\pgfpathlineto{\pgfqpoint{4.470205in}{2.418964in}}%
\pgfpathlineto{\pgfqpoint{4.470771in}{2.387565in}}%
\pgfpathlineto{\pgfqpoint{4.470857in}{2.424899in}}%
\pgfpathlineto{\pgfqpoint{4.471189in}{2.341353in}}%
\pgfpathlineto{\pgfqpoint{4.471497in}{2.349397in}}%
\pgfpathlineto{\pgfqpoint{4.471522in}{2.326939in}}%
\pgfpathlineto{\pgfqpoint{4.472100in}{2.417443in}}%
\pgfpathlineto{\pgfqpoint{4.472592in}{2.355821in}}%
\pgfpathlineto{\pgfqpoint{4.473479in}{2.421716in}}%
\pgfpathlineto{\pgfqpoint{4.473196in}{2.315622in}}%
\pgfpathlineto{\pgfqpoint{4.473602in}{2.332620in}}%
\pgfpathlineto{\pgfqpoint{4.473626in}{2.312130in}}%
\pgfpathlineto{\pgfqpoint{4.473959in}{2.412436in}}%
\pgfpathlineto{\pgfqpoint{4.474697in}{2.335018in}}%
\pgfpathlineto{\pgfqpoint{4.474820in}{2.427796in}}%
\pgfpathlineto{\pgfqpoint{4.475300in}{2.315955in}}%
\pgfpathlineto{\pgfqpoint{4.475817in}{2.364307in}}%
\pgfpathlineto{\pgfqpoint{4.475891in}{2.421610in}}%
\pgfpathlineto{\pgfqpoint{4.476125in}{2.320124in}}%
\pgfpathlineto{\pgfqpoint{4.476925in}{2.373289in}}%
\pgfpathlineto{\pgfqpoint{4.476949in}{2.379162in}}%
\pgfpathlineto{\pgfqpoint{4.477614in}{2.414534in}}%
\pgfpathlineto{\pgfqpoint{4.477220in}{2.341323in}}%
\pgfpathlineto{\pgfqpoint{4.478045in}{2.382433in}}%
\pgfpathlineto{\pgfqpoint{4.478599in}{2.313954in}}%
\pgfpathlineto{\pgfqpoint{4.478956in}{2.425125in}}%
\pgfpathlineto{\pgfqpoint{4.479165in}{2.365242in}}%
\pgfpathlineto{\pgfqpoint{4.479657in}{2.422167in}}%
\pgfpathlineto{\pgfqpoint{4.479916in}{2.326941in}}%
\pgfpathlineto{\pgfqpoint{4.480272in}{2.365341in}}%
\pgfpathlineto{\pgfqpoint{4.481072in}{2.312332in}}%
\pgfpathlineto{\pgfqpoint{4.480777in}{2.421066in}}%
\pgfpathlineto{\pgfqpoint{4.481368in}{2.375562in}}%
\pgfpathlineto{\pgfqpoint{4.481429in}{2.450140in}}%
\pgfpathlineto{\pgfqpoint{4.482389in}{2.321073in}}%
\pgfpathlineto{\pgfqpoint{4.482562in}{2.430657in}}%
\pgfpathlineto{\pgfqpoint{4.482648in}{2.412461in}}%
\pgfpathlineto{\pgfqpoint{4.482672in}{2.462698in}}%
\pgfpathlineto{\pgfqpoint{4.483546in}{2.323751in}}%
\pgfpathlineto{\pgfqpoint{4.483743in}{2.374802in}}%
\pgfpathlineto{\pgfqpoint{4.483903in}{2.471079in}}%
\pgfpathlineto{\pgfqpoint{4.484076in}{2.310519in}}%
\pgfpathlineto{\pgfqpoint{4.484703in}{2.372459in}}%
\pgfpathlineto{\pgfqpoint{4.485405in}{2.298716in}}%
\pgfpathlineto{\pgfqpoint{4.485146in}{2.450717in}}%
\pgfpathlineto{\pgfqpoint{4.485774in}{2.430984in}}%
\pgfpathlineto{\pgfqpoint{4.486648in}{2.316964in}}%
\pgfpathlineto{\pgfqpoint{4.486266in}{2.440023in}}%
\pgfpathlineto{\pgfqpoint{4.486980in}{2.369179in}}%
\pgfpathlineto{\pgfqpoint{4.487620in}{2.449647in}}%
\pgfpathlineto{\pgfqpoint{4.487756in}{2.320444in}}%
\pgfpathlineto{\pgfqpoint{4.488088in}{2.378610in}}%
\pgfpathlineto{\pgfqpoint{4.488863in}{2.423687in}}%
\pgfpathlineto{\pgfqpoint{4.489023in}{2.311059in}}%
\pgfpathlineto{\pgfqpoint{4.489995in}{2.443459in}}%
\pgfpathlineto{\pgfqpoint{4.489602in}{2.302891in}}%
\pgfpathlineto{\pgfqpoint{4.490254in}{2.333448in}}%
\pgfpathlineto{\pgfqpoint{4.490365in}{2.304417in}}%
\pgfpathlineto{\pgfqpoint{4.490734in}{2.425665in}}%
\pgfpathlineto{\pgfqpoint{4.491103in}{2.369250in}}%
\pgfpathlineto{\pgfqpoint{4.491768in}{2.463492in}}%
\pgfpathlineto{\pgfqpoint{4.492075in}{2.313795in}}%
\pgfpathlineto{\pgfqpoint{4.492162in}{2.327627in}}%
\pgfpathlineto{\pgfqpoint{4.492186in}{2.334024in}}%
\pgfpathlineto{\pgfqpoint{4.492568in}{2.454420in}}%
\pgfpathlineto{\pgfqpoint{4.492765in}{2.301768in}}%
\pgfpathlineto{\pgfqpoint{4.493306in}{2.349419in}}%
\pgfpathlineto{\pgfqpoint{4.494242in}{2.445995in}}%
\pgfpathlineto{\pgfqpoint{4.494069in}{2.292249in}}%
\pgfpathlineto{\pgfqpoint{4.494500in}{2.386898in}}%
\pgfpathlineto{\pgfqpoint{4.495312in}{2.296356in}}%
\pgfpathlineto{\pgfqpoint{4.494931in}{2.450081in}}%
\pgfpathlineto{\pgfqpoint{4.495620in}{2.360035in}}%
\pgfpathlineto{\pgfqpoint{4.496309in}{2.449845in}}%
\pgfpathlineto{\pgfqpoint{4.495866in}{2.299712in}}%
\pgfpathlineto{\pgfqpoint{4.496802in}{2.418326in}}%
\pgfpathlineto{\pgfqpoint{4.497712in}{2.311448in}}%
\pgfpathlineto{\pgfqpoint{4.497540in}{2.422697in}}%
\pgfpathlineto{\pgfqpoint{4.498008in}{2.398635in}}%
\pgfpathlineto{\pgfqpoint{4.498155in}{2.433302in}}%
\pgfpathlineto{\pgfqpoint{4.498352in}{2.314829in}}%
\pgfpathlineto{\pgfqpoint{4.499017in}{2.337856in}}%
\pgfpathlineto{\pgfqpoint{4.499583in}{2.307191in}}%
\pgfpathlineto{\pgfqpoint{4.500014in}{2.429867in}}%
\pgfpathlineto{\pgfqpoint{4.500075in}{2.393929in}}%
\pgfpathlineto{\pgfqpoint{4.500826in}{2.314842in}}%
\pgfpathlineto{\pgfqpoint{4.500445in}{2.443177in}}%
\pgfpathlineto{\pgfqpoint{4.501195in}{2.370803in}}%
\pgfpathlineto{\pgfqpoint{4.501749in}{2.437228in}}%
\pgfpathlineto{\pgfqpoint{4.501515in}{2.314576in}}%
\pgfpathlineto{\pgfqpoint{4.502303in}{2.392830in}}%
\pgfpathlineto{\pgfqpoint{4.503288in}{2.312671in}}%
\pgfpathlineto{\pgfqpoint{4.502919in}{2.426401in}}%
\pgfpathlineto{\pgfqpoint{4.503435in}{2.373666in}}%
\pgfpathlineto{\pgfqpoint{4.504162in}{2.431075in}}%
\pgfpathlineto{\pgfqpoint{4.504014in}{2.323103in}}%
\pgfpathlineto{\pgfqpoint{4.504445in}{2.357388in}}%
\pgfpathlineto{\pgfqpoint{4.504531in}{2.326020in}}%
\pgfpathlineto{\pgfqpoint{4.504974in}{2.423996in}}%
\pgfpathlineto{\pgfqpoint{4.505442in}{2.401416in}}%
\pgfpathlineto{\pgfqpoint{4.506020in}{2.421926in}}%
\pgfpathlineto{\pgfqpoint{4.505835in}{2.334189in}}%
\pgfpathlineto{\pgfqpoint{4.506463in}{2.356433in}}%
\pgfpathlineto{\pgfqpoint{4.506500in}{2.352123in}}%
\pgfpathlineto{\pgfqpoint{4.506512in}{2.361542in}}%
\pgfpathlineto{\pgfqpoint{4.506709in}{2.427165in}}%
\pgfpathlineto{\pgfqpoint{4.506931in}{2.318627in}}%
\pgfpathlineto{\pgfqpoint{4.507595in}{2.360086in}}%
\pgfpathlineto{\pgfqpoint{4.508235in}{2.317167in}}%
\pgfpathlineto{\pgfqpoint{4.508075in}{2.429351in}}%
\pgfpathlineto{\pgfqpoint{4.508654in}{2.380909in}}%
\pgfpathlineto{\pgfqpoint{4.509712in}{2.424569in}}%
\pgfpathlineto{\pgfqpoint{4.509491in}{2.323320in}}%
\pgfpathlineto{\pgfqpoint{4.509762in}{2.377404in}}%
\pgfpathlineto{\pgfqpoint{4.509774in}{2.377485in}}%
\pgfpathlineto{\pgfqpoint{4.510426in}{2.433940in}}%
\pgfpathlineto{\pgfqpoint{4.510722in}{2.320930in}}%
\pgfpathlineto{\pgfqpoint{4.510832in}{2.342663in}}%
\pgfpathlineto{\pgfqpoint{4.511349in}{2.333774in}}%
\pgfpathlineto{\pgfqpoint{4.511152in}{2.412426in}}%
\pgfpathlineto{\pgfqpoint{4.511632in}{2.390112in}}%
\pgfpathlineto{\pgfqpoint{4.512223in}{2.437004in}}%
\pgfpathlineto{\pgfqpoint{4.511965in}{2.328748in}}%
\pgfpathlineto{\pgfqpoint{4.512703in}{2.356093in}}%
\pgfpathlineto{\pgfqpoint{4.513195in}{2.330106in}}%
\pgfpathlineto{\pgfqpoint{4.513023in}{2.459694in}}%
\pgfpathlineto{\pgfqpoint{4.513503in}{2.399200in}}%
\pgfpathlineto{\pgfqpoint{4.514266in}{2.429231in}}%
\pgfpathlineto{\pgfqpoint{4.513983in}{2.331789in}}%
\pgfpathlineto{\pgfqpoint{4.514488in}{2.355615in}}%
\pgfpathlineto{\pgfqpoint{4.514512in}{2.336628in}}%
\pgfpathlineto{\pgfqpoint{4.515374in}{2.429392in}}%
\pgfpathlineto{\pgfqpoint{4.515558in}{2.382054in}}%
\pgfpathlineto{\pgfqpoint{4.515780in}{2.323669in}}%
\pgfpathlineto{\pgfqpoint{4.516137in}{2.432802in}}%
\pgfpathlineto{\pgfqpoint{4.516617in}{2.388333in}}%
\pgfpathlineto{\pgfqpoint{4.516740in}{2.439039in}}%
\pgfpathlineto{\pgfqpoint{4.517085in}{2.334146in}}%
\pgfpathlineto{\pgfqpoint{4.517602in}{2.370934in}}%
\pgfpathlineto{\pgfqpoint{4.517651in}{2.337873in}}%
\pgfpathlineto{\pgfqpoint{4.517983in}{2.425982in}}%
\pgfpathlineto{\pgfqpoint{4.518709in}{2.352409in}}%
\pgfpathlineto{\pgfqpoint{4.519842in}{2.425327in}}%
\pgfpathlineto{\pgfqpoint{4.518906in}{2.339261in}}%
\pgfpathlineto{\pgfqpoint{4.519866in}{2.408262in}}%
\pgfpathlineto{\pgfqpoint{4.520752in}{2.334925in}}%
\pgfpathlineto{\pgfqpoint{4.520063in}{2.420406in}}%
\pgfpathlineto{\pgfqpoint{4.521011in}{2.354551in}}%
\pgfpathlineto{\pgfqpoint{4.521085in}{2.421415in}}%
\pgfpathlineto{\pgfqpoint{4.521811in}{2.337272in}}%
\pgfpathlineto{\pgfqpoint{4.522217in}{2.408932in}}%
\pgfpathlineto{\pgfqpoint{4.523263in}{2.341563in}}%
\pgfpathlineto{\pgfqpoint{4.522943in}{2.427360in}}%
\pgfpathlineto{\pgfqpoint{4.523337in}{2.374641in}}%
\pgfpathlineto{\pgfqpoint{4.523977in}{2.425220in}}%
\pgfpathlineto{\pgfqpoint{4.523878in}{2.334676in}}%
\pgfpathlineto{\pgfqpoint{4.524420in}{2.362344in}}%
\pgfpathlineto{\pgfqpoint{4.524912in}{2.327617in}}%
\pgfpathlineto{\pgfqpoint{4.524789in}{2.417007in}}%
\pgfpathlineto{\pgfqpoint{4.525515in}{2.376177in}}%
\pgfpathlineto{\pgfqpoint{4.526217in}{2.322503in}}%
\pgfpathlineto{\pgfqpoint{4.526045in}{2.418610in}}%
\pgfpathlineto{\pgfqpoint{4.526438in}{2.391526in}}%
\pgfpathlineto{\pgfqpoint{4.526660in}{2.435098in}}%
\pgfpathlineto{\pgfqpoint{4.527005in}{2.335892in}}%
\pgfpathlineto{\pgfqpoint{4.527435in}{2.360526in}}%
\pgfpathlineto{\pgfqpoint{4.528014in}{2.335783in}}%
\pgfpathlineto{\pgfqpoint{4.527891in}{2.429255in}}%
\pgfpathlineto{\pgfqpoint{4.528482in}{2.410855in}}%
\pgfpathlineto{\pgfqpoint{4.528518in}{2.431450in}}%
\pgfpathlineto{\pgfqpoint{4.528826in}{2.339612in}}%
\pgfpathlineto{\pgfqpoint{4.529565in}{2.396359in}}%
\pgfpathlineto{\pgfqpoint{4.529934in}{2.331364in}}%
\pgfpathlineto{\pgfqpoint{4.529762in}{2.434333in}}%
\pgfpathlineto{\pgfqpoint{4.530721in}{2.351228in}}%
\pgfpathlineto{\pgfqpoint{4.531005in}{2.431083in}}%
\pgfpathlineto{\pgfqpoint{4.531177in}{2.326367in}}%
\pgfpathlineto{\pgfqpoint{4.531854in}{2.370037in}}%
\pgfpathlineto{\pgfqpoint{4.532518in}{2.328536in}}%
\pgfpathlineto{\pgfqpoint{4.532863in}{2.431163in}}%
\pgfpathlineto{\pgfqpoint{4.532974in}{2.343144in}}%
\pgfpathlineto{\pgfqpoint{4.533971in}{2.423098in}}%
\pgfpathlineto{\pgfqpoint{4.533577in}{2.316765in}}%
\pgfpathlineto{\pgfqpoint{4.534155in}{2.383555in}}%
\pgfpathlineto{\pgfqpoint{4.534377in}{2.340539in}}%
\pgfpathlineto{\pgfqpoint{4.534672in}{2.408688in}}%
\pgfpathlineto{\pgfqpoint{4.534709in}{2.443401in}}%
\pgfpathlineto{\pgfqpoint{4.534894in}{2.306701in}}%
\pgfpathlineto{\pgfqpoint{4.535768in}{2.410411in}}%
\pgfpathlineto{\pgfqpoint{4.536137in}{2.325215in}}%
\pgfpathlineto{\pgfqpoint{4.536358in}{2.419765in}}%
\pgfpathlineto{\pgfqpoint{4.536900in}{2.366258in}}%
\pgfpathlineto{\pgfqpoint{4.537589in}{2.428947in}}%
\pgfpathlineto{\pgfqpoint{4.537380in}{2.321902in}}%
\pgfpathlineto{\pgfqpoint{4.537897in}{2.361071in}}%
\pgfpathlineto{\pgfqpoint{4.538537in}{2.315744in}}%
\pgfpathlineto{\pgfqpoint{4.538918in}{2.421091in}}%
\pgfpathlineto{\pgfqpoint{4.538980in}{2.374350in}}%
\pgfpathlineto{\pgfqpoint{4.539669in}{2.429234in}}%
\pgfpathlineto{\pgfqpoint{4.539226in}{2.328319in}}%
\pgfpathlineto{\pgfqpoint{4.540088in}{2.387204in}}%
\pgfpathlineto{\pgfqpoint{4.540395in}{2.321405in}}%
\pgfpathlineto{\pgfqpoint{4.540285in}{2.427212in}}%
\pgfpathlineto{\pgfqpoint{4.541232in}{2.350549in}}%
\pgfpathlineto{\pgfqpoint{4.541515in}{2.425750in}}%
\pgfpathlineto{\pgfqpoint{4.541700in}{2.324877in}}%
\pgfpathlineto{\pgfqpoint{4.542328in}{2.345224in}}%
\pgfpathlineto{\pgfqpoint{4.542943in}{2.331199in}}%
\pgfpathlineto{\pgfqpoint{4.542648in}{2.422455in}}%
\pgfpathlineto{\pgfqpoint{4.543238in}{2.381563in}}%
\pgfpathlineto{\pgfqpoint{4.543374in}{2.421828in}}%
\pgfpathlineto{\pgfqpoint{4.543571in}{2.323697in}}%
\pgfpathlineto{\pgfqpoint{4.544297in}{2.348969in}}%
\pgfpathlineto{\pgfqpoint{4.544321in}{2.347963in}}%
\pgfpathlineto{\pgfqpoint{4.544346in}{2.364620in}}%
\pgfpathlineto{\pgfqpoint{4.545232in}{2.432919in}}%
\pgfpathlineto{\pgfqpoint{4.544728in}{2.319204in}}%
\pgfpathlineto{\pgfqpoint{4.545417in}{2.334930in}}%
\pgfpathlineto{\pgfqpoint{4.545848in}{2.424152in}}%
\pgfpathlineto{\pgfqpoint{4.545958in}{2.325792in}}%
\pgfpathlineto{\pgfqpoint{4.546561in}{2.344340in}}%
\pgfpathlineto{\pgfqpoint{4.546660in}{2.332669in}}%
\pgfpathlineto{\pgfqpoint{4.546968in}{2.417580in}}%
\pgfpathlineto{\pgfqpoint{4.547558in}{2.392174in}}%
\pgfpathlineto{\pgfqpoint{4.547595in}{2.433175in}}%
\pgfpathlineto{\pgfqpoint{4.547817in}{2.316789in}}%
\pgfpathlineto{\pgfqpoint{4.548617in}{2.347453in}}%
\pgfpathlineto{\pgfqpoint{4.548826in}{2.418903in}}%
\pgfpathlineto{\pgfqpoint{4.549048in}{2.319671in}}%
\pgfpathlineto{\pgfqpoint{4.549737in}{2.354404in}}%
\pgfpathlineto{\pgfqpoint{4.549761in}{2.354608in}}%
\pgfpathlineto{\pgfqpoint{4.549798in}{2.351541in}}%
\pgfpathlineto{\pgfqpoint{4.550278in}{2.305265in}}%
\pgfpathlineto{\pgfqpoint{4.550069in}{2.424769in}}%
\pgfpathlineto{\pgfqpoint{4.550746in}{2.403402in}}%
\pgfpathlineto{\pgfqpoint{4.551300in}{2.423288in}}%
\pgfpathlineto{\pgfqpoint{4.551054in}{2.327446in}}%
\pgfpathlineto{\pgfqpoint{4.551829in}{2.388442in}}%
\pgfpathlineto{\pgfqpoint{4.552752in}{2.323487in}}%
\pgfpathlineto{\pgfqpoint{4.552432in}{2.427257in}}%
\pgfpathlineto{\pgfqpoint{4.552974in}{2.364072in}}%
\pgfpathlineto{\pgfqpoint{4.553380in}{2.323304in}}%
\pgfpathlineto{\pgfqpoint{4.553134in}{2.430408in}}%
\pgfpathlineto{\pgfqpoint{4.553638in}{2.398703in}}%
\pgfpathlineto{\pgfqpoint{4.554278in}{2.425376in}}%
\pgfpathlineto{\pgfqpoint{4.554611in}{2.316080in}}%
\pgfpathlineto{\pgfqpoint{4.554709in}{2.363912in}}%
\pgfpathlineto{\pgfqpoint{4.555854in}{2.327865in}}%
\pgfpathlineto{\pgfqpoint{4.555521in}{2.419460in}}%
\pgfpathlineto{\pgfqpoint{4.555866in}{2.336211in}}%
\pgfpathlineto{\pgfqpoint{4.556248in}{2.428191in}}%
\pgfpathlineto{\pgfqpoint{4.556469in}{2.296224in}}%
\pgfpathlineto{\pgfqpoint{4.557048in}{2.393626in}}%
\pgfpathlineto{\pgfqpoint{4.557700in}{2.307853in}}%
\pgfpathlineto{\pgfqpoint{4.557478in}{2.417433in}}%
\pgfpathlineto{\pgfqpoint{4.558155in}{2.381263in}}%
\pgfpathlineto{\pgfqpoint{4.558611in}{2.434488in}}%
\pgfpathlineto{\pgfqpoint{4.558931in}{2.311557in}}%
\pgfpathlineto{\pgfqpoint{4.559275in}{2.401550in}}%
\pgfpathlineto{\pgfqpoint{4.559558in}{2.303054in}}%
\pgfpathlineto{\pgfqpoint{4.559952in}{2.430969in}}%
\pgfpathlineto{\pgfqpoint{4.560444in}{2.385770in}}%
\pgfpathlineto{\pgfqpoint{4.560531in}{2.416194in}}%
\pgfpathlineto{\pgfqpoint{4.561417in}{2.320405in}}%
\pgfpathlineto{\pgfqpoint{4.561515in}{2.352358in}}%
\pgfpathlineto{\pgfqpoint{4.561737in}{2.432002in}}%
\pgfpathlineto{\pgfqpoint{4.562032in}{2.310332in}}%
\pgfpathlineto{\pgfqpoint{4.562623in}{2.355252in}}%
\pgfpathlineto{\pgfqpoint{4.562648in}{2.310801in}}%
\pgfpathlineto{\pgfqpoint{4.563029in}{2.425797in}}%
\pgfpathlineto{\pgfqpoint{4.563718in}{2.379112in}}%
\pgfpathlineto{\pgfqpoint{4.564863in}{2.427038in}}%
\pgfpathlineto{\pgfqpoint{4.564506in}{2.314282in}}%
\pgfpathlineto{\pgfqpoint{4.564875in}{2.425359in}}%
\pgfpathlineto{\pgfqpoint{4.565121in}{2.307045in}}%
\pgfpathlineto{\pgfqpoint{4.565441in}{2.432449in}}%
\pgfpathlineto{\pgfqpoint{4.566057in}{2.408980in}}%
\pgfpathlineto{\pgfqpoint{4.566229in}{2.421325in}}%
\pgfpathlineto{\pgfqpoint{4.566352in}{2.318849in}}%
\pgfpathlineto{\pgfqpoint{4.567029in}{2.352417in}}%
\pgfpathlineto{\pgfqpoint{4.567595in}{2.316248in}}%
\pgfpathlineto{\pgfqpoint{4.567977in}{2.430944in}}%
\pgfpathlineto{\pgfqpoint{4.568063in}{2.405730in}}%
\pgfpathlineto{\pgfqpoint{4.568592in}{2.433022in}}%
\pgfpathlineto{\pgfqpoint{4.568211in}{2.304761in}}%
\pgfpathlineto{\pgfqpoint{4.568814in}{2.330491in}}%
\pgfpathlineto{\pgfqpoint{4.568900in}{2.307920in}}%
\pgfpathlineto{\pgfqpoint{4.569208in}{2.421269in}}%
\pgfpathlineto{\pgfqpoint{4.569835in}{2.415019in}}%
\pgfpathlineto{\pgfqpoint{4.570684in}{2.316830in}}%
\pgfpathlineto{\pgfqpoint{4.570401in}{2.420654in}}%
\pgfpathlineto{\pgfqpoint{4.570955in}{2.408633in}}%
\pgfpathlineto{\pgfqpoint{4.570980in}{2.407376in}}%
\pgfpathlineto{\pgfqpoint{4.571041in}{2.417183in}}%
\pgfpathlineto{\pgfqpoint{4.571681in}{2.435228in}}%
\pgfpathlineto{\pgfqpoint{4.571374in}{2.307512in}}%
\pgfpathlineto{\pgfqpoint{4.571964in}{2.333938in}}%
\pgfpathlineto{\pgfqpoint{4.572014in}{2.305856in}}%
\pgfpathlineto{\pgfqpoint{4.572223in}{2.424545in}}%
\pgfpathlineto{\pgfqpoint{4.572974in}{2.386553in}}%
\pgfpathlineto{\pgfqpoint{4.573454in}{2.426427in}}%
\pgfpathlineto{\pgfqpoint{4.573847in}{2.301987in}}%
\pgfpathlineto{\pgfqpoint{4.574106in}{2.416634in}}%
\pgfpathlineto{\pgfqpoint{4.574389in}{2.302242in}}%
\pgfpathlineto{\pgfqpoint{4.574771in}{2.428967in}}%
\pgfpathlineto{\pgfqpoint{4.575275in}{2.396453in}}%
\pgfpathlineto{\pgfqpoint{4.575349in}{2.436188in}}%
\pgfpathlineto{\pgfqpoint{4.575718in}{2.310246in}}%
\pgfpathlineto{\pgfqpoint{4.576223in}{2.345092in}}%
\pgfpathlineto{\pgfqpoint{4.576974in}{2.315005in}}%
\pgfpathlineto{\pgfqpoint{4.576580in}{2.439925in}}%
\pgfpathlineto{\pgfqpoint{4.577306in}{2.383123in}}%
\pgfpathlineto{\pgfqpoint{4.577355in}{2.431065in}}%
\pgfpathlineto{\pgfqpoint{4.577552in}{2.301649in}}%
\pgfpathlineto{\pgfqpoint{4.578451in}{2.420978in}}%
\pgfpathlineto{\pgfqpoint{4.578487in}{2.428455in}}%
\pgfpathlineto{\pgfqpoint{4.578820in}{2.317982in}}%
\pgfpathlineto{\pgfqpoint{4.579287in}{2.400072in}}%
\pgfpathlineto{\pgfqpoint{4.579398in}{2.304418in}}%
\pgfpathlineto{\pgfqpoint{4.580309in}{2.443969in}}%
\pgfpathlineto{\pgfqpoint{4.580395in}{2.383982in}}%
\pgfpathlineto{\pgfqpoint{4.581072in}{2.428244in}}%
\pgfpathlineto{\pgfqpoint{4.580641in}{2.314458in}}%
\pgfpathlineto{\pgfqpoint{4.581171in}{2.339898in}}%
\pgfpathlineto{\pgfqpoint{4.581257in}{2.321506in}}%
\pgfpathlineto{\pgfqpoint{4.581552in}{2.429060in}}%
\pgfpathlineto{\pgfqpoint{4.582241in}{2.381371in}}%
\pgfpathlineto{\pgfqpoint{4.582512in}{2.302947in}}%
\pgfpathlineto{\pgfqpoint{4.582352in}{2.416165in}}%
\pgfpathlineto{\pgfqpoint{4.582967in}{2.405488in}}%
\pgfpathlineto{\pgfqpoint{4.583423in}{2.438123in}}%
\pgfpathlineto{\pgfqpoint{4.583731in}{2.313284in}}%
\pgfpathlineto{\pgfqpoint{4.584051in}{2.402257in}}%
\pgfpathlineto{\pgfqpoint{4.584371in}{2.296702in}}%
\pgfpathlineto{\pgfqpoint{4.584211in}{2.425552in}}%
\pgfpathlineto{\pgfqpoint{4.585171in}{2.376409in}}%
\pgfpathlineto{\pgfqpoint{4.585244in}{2.437677in}}%
\pgfpathlineto{\pgfqpoint{4.586204in}{2.302925in}}%
\pgfpathlineto{\pgfqpoint{4.586217in}{2.301255in}}%
\pgfpathlineto{\pgfqpoint{4.586487in}{2.410516in}}%
\pgfpathlineto{\pgfqpoint{4.586524in}{2.424634in}}%
\pgfpathlineto{\pgfqpoint{4.586746in}{2.314664in}}%
\pgfpathlineto{\pgfqpoint{4.587435in}{2.328937in}}%
\pgfpathlineto{\pgfqpoint{4.588051in}{2.313144in}}%
\pgfpathlineto{\pgfqpoint{4.587718in}{2.433167in}}%
\pgfpathlineto{\pgfqpoint{4.588334in}{2.400705in}}%
\pgfpathlineto{\pgfqpoint{4.588974in}{2.439797in}}%
\pgfpathlineto{\pgfqpoint{4.588691in}{2.314011in}}%
\pgfpathlineto{\pgfqpoint{4.589404in}{2.360661in}}%
\pgfpathlineto{\pgfqpoint{4.590217in}{2.422124in}}%
\pgfpathlineto{\pgfqpoint{4.589909in}{2.302338in}}%
\pgfpathlineto{\pgfqpoint{4.590426in}{2.351698in}}%
\pgfpathlineto{\pgfqpoint{4.591152in}{2.305474in}}%
\pgfpathlineto{\pgfqpoint{4.590832in}{2.425576in}}%
\pgfpathlineto{\pgfqpoint{4.591497in}{2.379016in}}%
\pgfpathlineto{\pgfqpoint{4.592063in}{2.440449in}}%
\pgfpathlineto{\pgfqpoint{4.591755in}{2.304977in}}%
\pgfpathlineto{\pgfqpoint{4.592617in}{2.393846in}}%
\pgfpathlineto{\pgfqpoint{4.592998in}{2.315823in}}%
\pgfpathlineto{\pgfqpoint{4.593294in}{2.422554in}}%
\pgfpathlineto{\pgfqpoint{4.593810in}{2.376267in}}%
\pgfpathlineto{\pgfqpoint{4.593921in}{2.440903in}}%
\pgfpathlineto{\pgfqpoint{4.594229in}{2.305830in}}%
\pgfpathlineto{\pgfqpoint{4.594844in}{2.324228in}}%
\pgfpathlineto{\pgfqpoint{4.594906in}{2.316371in}}%
\pgfpathlineto{\pgfqpoint{4.595152in}{2.421991in}}%
\pgfpathlineto{\pgfqpoint{4.595644in}{2.372707in}}%
\pgfpathlineto{\pgfqpoint{4.596395in}{2.433285in}}%
\pgfpathlineto{\pgfqpoint{4.596137in}{2.326780in}}%
\pgfpathlineto{\pgfqpoint{4.596666in}{2.345848in}}%
\pgfpathlineto{\pgfqpoint{4.597318in}{2.311361in}}%
\pgfpathlineto{\pgfqpoint{4.597626in}{2.439420in}}%
\pgfpathlineto{\pgfqpoint{4.597724in}{2.392638in}}%
\pgfpathlineto{\pgfqpoint{4.598414in}{2.421836in}}%
\pgfpathlineto{\pgfqpoint{4.597934in}{2.320938in}}%
\pgfpathlineto{\pgfqpoint{4.598537in}{2.325743in}}%
\pgfpathlineto{\pgfqpoint{4.598561in}{2.312900in}}%
\pgfpathlineto{\pgfqpoint{4.599447in}{2.432226in}}%
\pgfpathlineto{\pgfqpoint{4.599558in}{2.400635in}}%
\pgfpathlineto{\pgfqpoint{4.600420in}{2.292424in}}%
\pgfpathlineto{\pgfqpoint{4.600223in}{2.431647in}}%
\pgfpathlineto{\pgfqpoint{4.600654in}{2.404732in}}%
\pgfpathlineto{\pgfqpoint{4.600727in}{2.431599in}}%
\pgfpathlineto{\pgfqpoint{4.601047in}{2.323966in}}%
\pgfpathlineto{\pgfqpoint{4.601589in}{2.360061in}}%
\pgfpathlineto{\pgfqpoint{4.601897in}{2.416253in}}%
\pgfpathlineto{\pgfqpoint{4.601663in}{2.307021in}}%
\pgfpathlineto{\pgfqpoint{4.602241in}{2.336501in}}%
\pgfpathlineto{\pgfqpoint{4.602881in}{2.310837in}}%
\pgfpathlineto{\pgfqpoint{4.602574in}{2.450083in}}%
\pgfpathlineto{\pgfqpoint{4.603300in}{2.404438in}}%
\pgfpathlineto{\pgfqpoint{4.603361in}{2.435893in}}%
\pgfpathlineto{\pgfqpoint{4.603509in}{2.309412in}}%
\pgfpathlineto{\pgfqpoint{4.604420in}{2.414436in}}%
\pgfpathlineto{\pgfqpoint{4.605035in}{2.423723in}}%
\pgfpathlineto{\pgfqpoint{4.604752in}{2.321369in}}%
\pgfpathlineto{\pgfqpoint{4.605244in}{2.385079in}}%
\pgfpathlineto{\pgfqpoint{4.605970in}{2.303582in}}%
\pgfpathlineto{\pgfqpoint{4.606278in}{2.431826in}}%
\pgfpathlineto{\pgfqpoint{4.606352in}{2.386196in}}%
\pgfpathlineto{\pgfqpoint{4.607017in}{2.429153in}}%
\pgfpathlineto{\pgfqpoint{4.606610in}{2.306308in}}%
\pgfpathlineto{\pgfqpoint{4.607164in}{2.343614in}}%
\pgfpathlineto{\pgfqpoint{4.607201in}{2.306204in}}%
\pgfpathlineto{\pgfqpoint{4.608124in}{2.425060in}}%
\pgfpathlineto{\pgfqpoint{4.608223in}{2.385688in}}%
\pgfpathlineto{\pgfqpoint{4.608826in}{2.413648in}}%
\pgfpathlineto{\pgfqpoint{4.608444in}{2.300286in}}%
\pgfpathlineto{\pgfqpoint{4.609195in}{2.339256in}}%
\pgfpathlineto{\pgfqpoint{4.609724in}{2.314110in}}%
\pgfpathlineto{\pgfqpoint{4.609367in}{2.439399in}}%
\pgfpathlineto{\pgfqpoint{4.609897in}{2.387170in}}%
\pgfpathlineto{\pgfqpoint{4.610118in}{2.425873in}}%
\pgfpathlineto{\pgfqpoint{4.610303in}{2.306636in}}%
\pgfpathlineto{\pgfqpoint{4.610943in}{2.338611in}}%
\pgfpathlineto{\pgfqpoint{4.611090in}{2.375058in}}%
\pgfpathlineto{\pgfqpoint{4.611226in}{2.426016in}}%
\pgfpathlineto{\pgfqpoint{4.611546in}{2.304509in}}%
\pgfpathlineto{\pgfqpoint{4.612124in}{2.326243in}}%
\pgfpathlineto{\pgfqpoint{4.612161in}{2.312207in}}%
\pgfpathlineto{\pgfqpoint{4.612469in}{2.438505in}}%
\pgfpathlineto{\pgfqpoint{4.613146in}{2.389262in}}%
\pgfpathlineto{\pgfqpoint{4.613810in}{2.422735in}}%
\pgfpathlineto{\pgfqpoint{4.613392in}{2.307234in}}%
\pgfpathlineto{\pgfqpoint{4.613958in}{2.347991in}}%
\pgfpathlineto{\pgfqpoint{4.614647in}{2.303149in}}%
\pgfpathlineto{\pgfqpoint{4.614253in}{2.425828in}}%
\pgfpathlineto{\pgfqpoint{4.614893in}{2.416781in}}%
\pgfpathlineto{\pgfqpoint{4.615029in}{2.426364in}}%
\pgfpathlineto{\pgfqpoint{4.615226in}{2.304596in}}%
\pgfpathlineto{\pgfqpoint{4.615817in}{2.373447in}}%
\pgfpathlineto{\pgfqpoint{4.615866in}{2.306904in}}%
\pgfpathlineto{\pgfqpoint{4.616161in}{2.448593in}}%
\pgfpathlineto{\pgfqpoint{4.616900in}{2.412486in}}%
\pgfpathlineto{\pgfqpoint{4.617367in}{2.439105in}}%
\pgfpathlineto{\pgfqpoint{4.617097in}{2.317001in}}%
\pgfpathlineto{\pgfqpoint{4.618020in}{2.427394in}}%
\pgfpathlineto{\pgfqpoint{4.618327in}{2.286545in}}%
\pgfpathlineto{\pgfqpoint{4.618143in}{2.433966in}}%
\pgfpathlineto{\pgfqpoint{4.619189in}{2.395263in}}%
\pgfpathlineto{\pgfqpoint{4.619263in}{2.451647in}}%
\pgfpathlineto{\pgfqpoint{4.619570in}{2.321404in}}%
\pgfpathlineto{\pgfqpoint{4.620149in}{2.332772in}}%
\pgfpathlineto{\pgfqpoint{4.620186in}{2.308462in}}%
\pgfpathlineto{\pgfqpoint{4.621109in}{2.438914in}}%
\pgfpathlineto{\pgfqpoint{4.621195in}{2.393674in}}%
\pgfpathlineto{\pgfqpoint{4.621257in}{2.421845in}}%
\pgfpathlineto{\pgfqpoint{4.621392in}{2.331216in}}%
\pgfpathlineto{\pgfqpoint{4.621429in}{2.292032in}}%
\pgfpathlineto{\pgfqpoint{4.622364in}{2.433773in}}%
\pgfpathlineto{\pgfqpoint{4.622463in}{2.401063in}}%
\pgfpathlineto{\pgfqpoint{4.622475in}{2.401865in}}%
\pgfpathlineto{\pgfqpoint{4.622647in}{2.342771in}}%
\pgfpathlineto{\pgfqpoint{4.623300in}{2.307792in}}%
\pgfpathlineto{\pgfqpoint{4.623583in}{2.417068in}}%
\pgfpathlineto{\pgfqpoint{4.624210in}{2.444533in}}%
\pgfpathlineto{\pgfqpoint{4.623890in}{2.314923in}}%
\pgfpathlineto{\pgfqpoint{4.624506in}{2.321045in}}%
\pgfpathlineto{\pgfqpoint{4.625121in}{2.305506in}}%
\pgfpathlineto{\pgfqpoint{4.624998in}{2.425369in}}%
\pgfpathlineto{\pgfqpoint{4.625417in}{2.398990in}}%
\pgfpathlineto{\pgfqpoint{4.625527in}{2.424736in}}%
\pgfpathlineto{\pgfqpoint{4.626377in}{2.312790in}}%
\pgfpathlineto{\pgfqpoint{4.626450in}{2.333903in}}%
\pgfpathlineto{\pgfqpoint{4.627373in}{2.428329in}}%
\pgfpathlineto{\pgfqpoint{4.626980in}{2.300536in}}%
\pgfpathlineto{\pgfqpoint{4.627570in}{2.353590in}}%
\pgfpathlineto{\pgfqpoint{4.627607in}{2.311394in}}%
\pgfpathlineto{\pgfqpoint{4.627903in}{2.434564in}}%
\pgfpathlineto{\pgfqpoint{4.628629in}{2.416675in}}%
\pgfpathlineto{\pgfqpoint{4.629133in}{2.436893in}}%
\pgfpathlineto{\pgfqpoint{4.628826in}{2.303365in}}%
\pgfpathlineto{\pgfqpoint{4.629577in}{2.339301in}}%
\pgfpathlineto{\pgfqpoint{4.630093in}{2.302433in}}%
\pgfpathlineto{\pgfqpoint{4.629724in}{2.440431in}}%
\pgfpathlineto{\pgfqpoint{4.630500in}{2.438080in}}%
\pgfpathlineto{\pgfqpoint{4.630512in}{2.438609in}}%
\pgfpathlineto{\pgfqpoint{4.630573in}{2.396759in}}%
\pgfpathlineto{\pgfqpoint{4.630623in}{2.319995in}}%
\pgfpathlineto{\pgfqpoint{4.631607in}{2.433127in}}%
\pgfpathlineto{\pgfqpoint{4.631693in}{2.376552in}}%
\pgfpathlineto{\pgfqpoint{4.631927in}{2.296146in}}%
\pgfpathlineto{\pgfqpoint{4.632850in}{2.448205in}}%
\pgfpathlineto{\pgfqpoint{4.632863in}{2.450093in}}%
\pgfpathlineto{\pgfqpoint{4.633133in}{2.325267in}}%
\pgfpathlineto{\pgfqpoint{4.633786in}{2.286299in}}%
\pgfpathlineto{\pgfqpoint{4.633638in}{2.427198in}}%
\pgfpathlineto{\pgfqpoint{4.634056in}{2.403276in}}%
\pgfpathlineto{\pgfqpoint{4.634093in}{2.436148in}}%
\pgfpathlineto{\pgfqpoint{4.635041in}{2.302584in}}%
\pgfpathlineto{\pgfqpoint{4.635103in}{2.341029in}}%
\pgfpathlineto{\pgfqpoint{4.635952in}{2.431342in}}%
\pgfpathlineto{\pgfqpoint{4.635632in}{2.306467in}}%
\pgfpathlineto{\pgfqpoint{4.636223in}{2.353798in}}%
\pgfpathlineto{\pgfqpoint{4.636272in}{2.316463in}}%
\pgfpathlineto{\pgfqpoint{4.636567in}{2.428511in}}%
\pgfpathlineto{\pgfqpoint{4.637281in}{2.408606in}}%
\pgfpathlineto{\pgfqpoint{4.638106in}{2.301998in}}%
\pgfpathlineto{\pgfqpoint{4.637798in}{2.442318in}}%
\pgfpathlineto{\pgfqpoint{4.638352in}{2.411809in}}%
\pgfpathlineto{\pgfqpoint{4.638413in}{2.439215in}}%
\pgfpathlineto{\pgfqpoint{4.638721in}{2.323917in}}%
\pgfpathlineto{\pgfqpoint{4.639287in}{2.331314in}}%
\pgfpathlineto{\pgfqpoint{4.639336in}{2.313146in}}%
\pgfpathlineto{\pgfqpoint{4.639632in}{2.419994in}}%
\pgfpathlineto{\pgfqpoint{4.640223in}{2.406405in}}%
\pgfpathlineto{\pgfqpoint{4.640272in}{2.436237in}}%
\pgfpathlineto{\pgfqpoint{4.640580in}{2.305514in}}%
\pgfpathlineto{\pgfqpoint{4.641109in}{2.363400in}}%
\pgfpathlineto{\pgfqpoint{4.641810in}{2.314166in}}%
\pgfpathlineto{\pgfqpoint{4.641515in}{2.432616in}}%
\pgfpathlineto{\pgfqpoint{4.642192in}{2.389849in}}%
\pgfpathlineto{\pgfqpoint{4.642253in}{2.420683in}}%
\pgfpathlineto{\pgfqpoint{4.642438in}{2.307261in}}%
\pgfpathlineto{\pgfqpoint{4.643041in}{2.322449in}}%
\pgfpathlineto{\pgfqpoint{4.643669in}{2.310022in}}%
\pgfpathlineto{\pgfqpoint{4.643324in}{2.420673in}}%
\pgfpathlineto{\pgfqpoint{4.643878in}{2.389380in}}%
\pgfpathlineto{\pgfqpoint{4.644100in}{2.430552in}}%
\pgfpathlineto{\pgfqpoint{4.644296in}{2.306288in}}%
\pgfpathlineto{\pgfqpoint{4.644875in}{2.352566in}}%
\pgfpathlineto{\pgfqpoint{4.644912in}{2.318140in}}%
\pgfpathlineto{\pgfqpoint{4.645835in}{2.427018in}}%
\pgfpathlineto{\pgfqpoint{4.645921in}{2.391847in}}%
\pgfpathlineto{\pgfqpoint{4.646450in}{2.442940in}}%
\pgfpathlineto{\pgfqpoint{4.646758in}{2.301047in}}%
\pgfpathlineto{\pgfqpoint{4.647090in}{2.422879in}}%
\pgfpathlineto{\pgfqpoint{4.647373in}{2.315574in}}%
\pgfpathlineto{\pgfqpoint{4.647816in}{2.429610in}}%
\pgfpathlineto{\pgfqpoint{4.648223in}{2.383552in}}%
\pgfpathlineto{\pgfqpoint{4.648887in}{2.418404in}}%
\pgfpathlineto{\pgfqpoint{4.649183in}{2.325270in}}%
\pgfpathlineto{\pgfqpoint{4.649195in}{2.326605in}}%
\pgfpathlineto{\pgfqpoint{4.649232in}{2.312009in}}%
\pgfpathlineto{\pgfqpoint{4.649540in}{2.431396in}}%
\pgfpathlineto{\pgfqpoint{4.649552in}{2.438627in}}%
\pgfpathlineto{\pgfqpoint{4.649847in}{2.305686in}}%
\pgfpathlineto{\pgfqpoint{4.650450in}{2.325642in}}%
\pgfpathlineto{\pgfqpoint{4.650475in}{2.307975in}}%
\pgfpathlineto{\pgfqpoint{4.650906in}{2.426318in}}%
\pgfpathlineto{\pgfqpoint{4.651484in}{2.393276in}}%
\pgfpathlineto{\pgfqpoint{4.652641in}{2.428510in}}%
\pgfpathlineto{\pgfqpoint{4.652321in}{2.308896in}}%
\pgfpathlineto{\pgfqpoint{4.652653in}{2.426861in}}%
\pgfpathlineto{\pgfqpoint{4.653564in}{2.306373in}}%
\pgfpathlineto{\pgfqpoint{4.653244in}{2.435211in}}%
\pgfpathlineto{\pgfqpoint{4.653786in}{2.396040in}}%
\pgfpathlineto{\pgfqpoint{4.654007in}{2.425452in}}%
\pgfpathlineto{\pgfqpoint{4.654167in}{2.313338in}}%
\pgfpathlineto{\pgfqpoint{4.654709in}{2.351417in}}%
\pgfpathlineto{\pgfqpoint{4.655410in}{2.306262in}}%
\pgfpathlineto{\pgfqpoint{4.655743in}{2.431891in}}%
\pgfpathlineto{\pgfqpoint{4.655779in}{2.395134in}}%
\pgfpathlineto{\pgfqpoint{4.656666in}{2.308158in}}%
\pgfpathlineto{\pgfqpoint{4.656346in}{2.427230in}}%
\pgfpathlineto{\pgfqpoint{4.656863in}{2.394585in}}%
\pgfpathlineto{\pgfqpoint{4.656973in}{2.424403in}}%
\pgfpathlineto{\pgfqpoint{4.657884in}{2.307408in}}%
\pgfpathlineto{\pgfqpoint{4.658832in}{2.431421in}}%
\pgfpathlineto{\pgfqpoint{4.659041in}{2.331677in}}%
\pgfpathlineto{\pgfqpoint{4.659755in}{2.307355in}}%
\pgfpathlineto{\pgfqpoint{4.660038in}{2.428787in}}%
\pgfpathlineto{\pgfqpoint{4.660063in}{2.437904in}}%
\pgfpathlineto{\pgfqpoint{4.660358in}{2.314030in}}%
\pgfpathlineto{\pgfqpoint{4.660936in}{2.336335in}}%
\pgfpathlineto{\pgfqpoint{4.660986in}{2.304820in}}%
\pgfpathlineto{\pgfqpoint{4.661281in}{2.426015in}}%
\pgfpathlineto{\pgfqpoint{4.661884in}{2.418888in}}%
\pgfpathlineto{\pgfqpoint{4.661921in}{2.426923in}}%
\pgfpathlineto{\pgfqpoint{4.662216in}{2.305283in}}%
\pgfpathlineto{\pgfqpoint{4.662770in}{2.337023in}}%
\pgfpathlineto{\pgfqpoint{4.663447in}{2.303916in}}%
\pgfpathlineto{\pgfqpoint{4.663127in}{2.434751in}}%
\pgfpathlineto{\pgfqpoint{4.663755in}{2.417771in}}%
\pgfpathlineto{\pgfqpoint{4.663890in}{2.433055in}}%
\pgfpathlineto{\pgfqpoint{4.664075in}{2.306395in}}%
\pgfpathlineto{\pgfqpoint{4.664616in}{2.345762in}}%
\pgfpathlineto{\pgfqpoint{4.664998in}{2.419967in}}%
\pgfpathlineto{\pgfqpoint{4.665244in}{2.313780in}}%
\pgfpathlineto{\pgfqpoint{4.665293in}{2.304057in}}%
\pgfpathlineto{\pgfqpoint{4.665576in}{2.429356in}}%
\pgfpathlineto{\pgfqpoint{4.666093in}{2.359750in}}%
\pgfpathlineto{\pgfqpoint{4.666229in}{2.440818in}}%
\pgfpathlineto{\pgfqpoint{4.666549in}{2.309449in}}%
\pgfpathlineto{\pgfqpoint{4.667152in}{2.323533in}}%
\pgfpathlineto{\pgfqpoint{4.667176in}{2.312180in}}%
\pgfpathlineto{\pgfqpoint{4.667410in}{2.426424in}}%
\pgfpathlineto{\pgfqpoint{4.668161in}{2.402152in}}%
\pgfpathlineto{\pgfqpoint{4.668703in}{2.436473in}}%
\pgfpathlineto{\pgfqpoint{4.668395in}{2.287996in}}%
\pgfpathlineto{\pgfqpoint{4.669306in}{2.435986in}}%
\pgfpathlineto{\pgfqpoint{4.669318in}{2.439761in}}%
\pgfpathlineto{\pgfqpoint{4.669638in}{2.306066in}}%
\pgfpathlineto{\pgfqpoint{4.670167in}{2.363854in}}%
\pgfpathlineto{\pgfqpoint{4.670253in}{2.326172in}}%
\pgfpathlineto{\pgfqpoint{4.670524in}{2.437221in}}%
\pgfpathlineto{\pgfqpoint{4.671176in}{2.421947in}}%
\pgfpathlineto{\pgfqpoint{4.671312in}{2.436768in}}%
\pgfpathlineto{\pgfqpoint{4.671386in}{2.376327in}}%
\pgfpathlineto{\pgfqpoint{4.671484in}{2.287438in}}%
\pgfpathlineto{\pgfqpoint{4.672432in}{2.442523in}}%
\pgfpathlineto{\pgfqpoint{4.672481in}{2.388144in}}%
\pgfpathlineto{\pgfqpoint{4.672518in}{2.409850in}}%
\pgfpathlineto{\pgfqpoint{4.672727in}{2.314195in}}%
\pgfpathlineto{\pgfqpoint{4.673650in}{2.443754in}}%
\pgfpathlineto{\pgfqpoint{4.674573in}{2.296422in}}%
\pgfpathlineto{\pgfqpoint{4.674819in}{2.415510in}}%
\pgfpathlineto{\pgfqpoint{4.674893in}{2.441878in}}%
\pgfpathlineto{\pgfqpoint{4.675189in}{2.314367in}}%
\pgfpathlineto{\pgfqpoint{4.675718in}{2.387845in}}%
\pgfpathlineto{\pgfqpoint{4.676419in}{2.310134in}}%
\pgfpathlineto{\pgfqpoint{4.676112in}{2.422058in}}%
\pgfpathlineto{\pgfqpoint{4.676813in}{2.403428in}}%
\pgfpathlineto{\pgfqpoint{4.677662in}{2.300817in}}%
\pgfpathlineto{\pgfqpoint{4.677367in}{2.426303in}}%
\pgfpathlineto{\pgfqpoint{4.677909in}{2.407926in}}%
\pgfpathlineto{\pgfqpoint{4.678733in}{2.436651in}}%
\pgfpathlineto{\pgfqpoint{4.678290in}{2.298840in}}%
\pgfpathlineto{\pgfqpoint{4.678979in}{2.361680in}}%
\pgfpathlineto{\pgfqpoint{4.679964in}{2.442584in}}%
\pgfpathlineto{\pgfqpoint{4.679533in}{2.324629in}}%
\pgfpathlineto{\pgfqpoint{4.680062in}{2.358238in}}%
\pgfpathlineto{\pgfqpoint{4.680136in}{2.291927in}}%
\pgfpathlineto{\pgfqpoint{4.681084in}{2.437756in}}%
\pgfpathlineto{\pgfqpoint{4.681146in}{2.380074in}}%
\pgfpathlineto{\pgfqpoint{4.682315in}{2.452412in}}%
\pgfpathlineto{\pgfqpoint{4.681995in}{2.308205in}}%
\pgfpathlineto{\pgfqpoint{4.682327in}{2.443229in}}%
\pgfpathlineto{\pgfqpoint{4.682610in}{2.303569in}}%
\pgfpathlineto{\pgfqpoint{4.683447in}{2.399526in}}%
\pgfpathlineto{\pgfqpoint{4.683681in}{2.438416in}}%
\pgfpathlineto{\pgfqpoint{4.683829in}{2.301018in}}%
\pgfpathlineto{\pgfqpoint{4.684456in}{2.332390in}}%
\pgfpathlineto{\pgfqpoint{4.685084in}{2.294327in}}%
\pgfpathlineto{\pgfqpoint{4.684924in}{2.435117in}}%
\pgfpathlineto{\pgfqpoint{4.685330in}{2.398371in}}%
\pgfpathlineto{\pgfqpoint{4.686019in}{2.431947in}}%
\pgfpathlineto{\pgfqpoint{4.685650in}{2.312763in}}%
\pgfpathlineto{\pgfqpoint{4.686266in}{2.349469in}}%
\pgfpathlineto{\pgfqpoint{4.686942in}{2.301313in}}%
\pgfpathlineto{\pgfqpoint{4.686586in}{2.434335in}}%
\pgfpathlineto{\pgfqpoint{4.687336in}{2.385156in}}%
\pgfpathlineto{\pgfqpoint{4.687816in}{2.439900in}}%
\pgfpathlineto{\pgfqpoint{4.688198in}{2.314930in}}%
\pgfpathlineto{\pgfqpoint{4.688456in}{2.409814in}}%
\pgfpathlineto{\pgfqpoint{4.688789in}{2.304117in}}%
\pgfpathlineto{\pgfqpoint{4.688604in}{2.435213in}}%
\pgfpathlineto{\pgfqpoint{4.689613in}{2.379751in}}%
\pgfpathlineto{\pgfqpoint{4.689724in}{2.441134in}}%
\pgfpathlineto{\pgfqpoint{4.690032in}{2.314111in}}%
\pgfpathlineto{\pgfqpoint{4.690696in}{2.355612in}}%
\pgfpathlineto{\pgfqpoint{4.691718in}{2.445577in}}%
\pgfpathlineto{\pgfqpoint{4.691238in}{2.322901in}}%
\pgfpathlineto{\pgfqpoint{4.691804in}{2.365331in}}%
\pgfpathlineto{\pgfqpoint{4.691890in}{2.310008in}}%
\pgfpathlineto{\pgfqpoint{4.692826in}{2.436521in}}%
\pgfpathlineto{\pgfqpoint{4.692899in}{2.377375in}}%
\pgfpathlineto{\pgfqpoint{4.694056in}{2.438140in}}%
\pgfpathlineto{\pgfqpoint{4.693121in}{2.318273in}}%
\pgfpathlineto{\pgfqpoint{4.694069in}{2.428540in}}%
\pgfpathlineto{\pgfqpoint{4.694979in}{2.305503in}}%
\pgfpathlineto{\pgfqpoint{4.694819in}{2.430317in}}%
\pgfpathlineto{\pgfqpoint{4.695201in}{2.399414in}}%
\pgfpathlineto{\pgfqpoint{4.695275in}{2.417317in}}%
\pgfpathlineto{\pgfqpoint{4.695546in}{2.316057in}}%
\pgfpathlineto{\pgfqpoint{4.695730in}{2.334328in}}%
\pgfpathlineto{\pgfqpoint{4.696198in}{2.318754in}}%
\pgfpathlineto{\pgfqpoint{4.695927in}{2.432027in}}%
\pgfpathlineto{\pgfqpoint{4.696641in}{2.406123in}}%
\pgfpathlineto{\pgfqpoint{4.697145in}{2.433885in}}%
\pgfpathlineto{\pgfqpoint{4.697416in}{2.318742in}}%
\pgfpathlineto{\pgfqpoint{4.697749in}{2.410732in}}%
\pgfpathlineto{\pgfqpoint{4.697909in}{2.429725in}}%
\pgfpathlineto{\pgfqpoint{4.698069in}{2.301418in}}%
\pgfpathlineto{\pgfqpoint{4.698585in}{2.381053in}}%
\pgfpathlineto{\pgfqpoint{4.698635in}{2.307046in}}%
\pgfpathlineto{\pgfqpoint{4.699029in}{2.432112in}}%
\pgfpathlineto{\pgfqpoint{4.699693in}{2.376736in}}%
\pgfpathlineto{\pgfqpoint{4.700235in}{2.434918in}}%
\pgfpathlineto{\pgfqpoint{4.700518in}{2.315669in}}%
\pgfpathlineto{\pgfqpoint{4.700850in}{2.417546in}}%
\pgfpathlineto{\pgfqpoint{4.701170in}{2.300773in}}%
\pgfpathlineto{\pgfqpoint{4.701441in}{2.434815in}}%
\pgfpathlineto{\pgfqpoint{4.702007in}{2.386793in}}%
\pgfpathlineto{\pgfqpoint{4.702229in}{2.436506in}}%
\pgfpathlineto{\pgfqpoint{4.702376in}{2.324815in}}%
\pgfpathlineto{\pgfqpoint{4.703078in}{2.363930in}}%
\pgfpathlineto{\pgfqpoint{4.703152in}{2.353019in}}%
\pgfpathlineto{\pgfqpoint{4.703275in}{2.393251in}}%
\pgfpathlineto{\pgfqpoint{4.704087in}{2.436780in}}%
\pgfpathlineto{\pgfqpoint{4.704272in}{2.297254in}}%
\pgfpathlineto{\pgfqpoint{4.704321in}{2.337304in}}%
\pgfpathlineto{\pgfqpoint{4.704530in}{2.454547in}}%
\pgfpathlineto{\pgfqpoint{4.704825in}{2.305226in}}%
\pgfpathlineto{\pgfqpoint{4.705429in}{2.341217in}}%
\pgfpathlineto{\pgfqpoint{4.705502in}{2.323951in}}%
\pgfpathlineto{\pgfqpoint{4.706413in}{2.431729in}}%
\pgfpathlineto{\pgfqpoint{4.706499in}{2.371678in}}%
\pgfpathlineto{\pgfqpoint{4.707632in}{2.454289in}}%
\pgfpathlineto{\pgfqpoint{4.707349in}{2.290662in}}%
\pgfpathlineto{\pgfqpoint{4.707705in}{2.409313in}}%
\pgfpathlineto{\pgfqpoint{4.707915in}{2.309790in}}%
\pgfpathlineto{\pgfqpoint{4.708419in}{2.449675in}}%
\pgfpathlineto{\pgfqpoint{4.708825in}{2.392889in}}%
\pgfpathlineto{\pgfqpoint{4.709502in}{2.431587in}}%
\pgfpathlineto{\pgfqpoint{4.709145in}{2.335347in}}%
\pgfpathlineto{\pgfqpoint{4.709724in}{2.363851in}}%
\pgfpathlineto{\pgfqpoint{4.710450in}{2.285437in}}%
\pgfpathlineto{\pgfqpoint{4.710721in}{2.444860in}}%
\pgfpathlineto{\pgfqpoint{4.710745in}{2.434332in}}%
\pgfpathlineto{\pgfqpoint{4.711004in}{2.317611in}}%
\pgfpathlineto{\pgfqpoint{4.711509in}{2.438409in}}%
\pgfpathlineto{\pgfqpoint{4.712062in}{2.359612in}}%
\pgfpathlineto{\pgfqpoint{4.713182in}{2.436976in}}%
\pgfpathlineto{\pgfqpoint{4.712862in}{2.306380in}}%
\pgfpathlineto{\pgfqpoint{4.713232in}{2.422664in}}%
\pgfpathlineto{\pgfqpoint{4.713527in}{2.293625in}}%
\pgfpathlineto{\pgfqpoint{4.713810in}{2.440794in}}%
\pgfpathlineto{\pgfqpoint{4.714438in}{2.398978in}}%
\pgfpathlineto{\pgfqpoint{4.714598in}{2.435481in}}%
\pgfpathlineto{\pgfqpoint{4.715336in}{2.331339in}}%
\pgfpathlineto{\pgfqpoint{4.715373in}{2.350083in}}%
\pgfpathlineto{\pgfqpoint{4.715964in}{2.304480in}}%
\pgfpathlineto{\pgfqpoint{4.716296in}{2.434965in}}%
\pgfpathlineto{\pgfqpoint{4.716407in}{2.389640in}}%
\pgfpathlineto{\pgfqpoint{4.717072in}{2.431515in}}%
\pgfpathlineto{\pgfqpoint{4.716628in}{2.300653in}}%
\pgfpathlineto{\pgfqpoint{4.717367in}{2.351729in}}%
\pgfpathlineto{\pgfqpoint{4.718462in}{2.325813in}}%
\pgfpathlineto{\pgfqpoint{4.717699in}{2.418817in}}%
\pgfpathlineto{\pgfqpoint{4.718487in}{2.337334in}}%
\pgfpathlineto{\pgfqpoint{4.719422in}{2.433792in}}%
\pgfpathlineto{\pgfqpoint{4.719041in}{2.295477in}}%
\pgfpathlineto{\pgfqpoint{4.719632in}{2.367557in}}%
\pgfpathlineto{\pgfqpoint{4.719668in}{2.317542in}}%
\pgfpathlineto{\pgfqpoint{4.719976in}{2.437382in}}%
\pgfpathlineto{\pgfqpoint{4.720715in}{2.384429in}}%
\pgfpathlineto{\pgfqpoint{4.721822in}{2.433052in}}%
\pgfpathlineto{\pgfqpoint{4.721576in}{2.317986in}}%
\pgfpathlineto{\pgfqpoint{4.721847in}{2.426962in}}%
\pgfpathlineto{\pgfqpoint{4.722142in}{2.299915in}}%
\pgfpathlineto{\pgfqpoint{4.722622in}{2.428562in}}%
\pgfpathlineto{\pgfqpoint{4.723028in}{2.390605in}}%
\pgfpathlineto{\pgfqpoint{4.723090in}{2.431245in}}%
\pgfpathlineto{\pgfqpoint{4.723435in}{2.328470in}}%
\pgfpathlineto{\pgfqpoint{4.724001in}{2.329488in}}%
\pgfpathlineto{\pgfqpoint{4.724616in}{2.315500in}}%
\pgfpathlineto{\pgfqpoint{4.724321in}{2.434801in}}%
\pgfpathlineto{\pgfqpoint{4.724850in}{2.379459in}}%
\pgfpathlineto{\pgfqpoint{4.724936in}{2.450028in}}%
\pgfpathlineto{\pgfqpoint{4.725293in}{2.305491in}}%
\pgfpathlineto{\pgfqpoint{4.725908in}{2.341419in}}%
\pgfpathlineto{\pgfqpoint{4.726807in}{2.422001in}}%
\pgfpathlineto{\pgfqpoint{4.725958in}{2.328331in}}%
\pgfpathlineto{\pgfqpoint{4.727053in}{2.359586in}}%
\pgfpathlineto{\pgfqpoint{4.727693in}{2.306152in}}%
\pgfpathlineto{\pgfqpoint{4.728075in}{2.444978in}}%
\pgfpathlineto{\pgfqpoint{4.728136in}{2.396295in}}%
\pgfpathlineto{\pgfqpoint{4.728628in}{2.435565in}}%
\pgfpathlineto{\pgfqpoint{4.728321in}{2.304308in}}%
\pgfpathlineto{\pgfqpoint{4.728961in}{2.321972in}}%
\pgfpathlineto{\pgfqpoint{4.728985in}{2.301798in}}%
\pgfpathlineto{\pgfqpoint{4.729847in}{2.429549in}}%
\pgfpathlineto{\pgfqpoint{4.730007in}{2.376915in}}%
\pgfpathlineto{\pgfqpoint{4.730228in}{2.311185in}}%
\pgfpathlineto{\pgfqpoint{4.730438in}{2.420307in}}%
\pgfpathlineto{\pgfqpoint{4.730487in}{2.437058in}}%
\pgfpathlineto{\pgfqpoint{4.730832in}{2.293068in}}%
\pgfpathlineto{\pgfqpoint{4.731435in}{2.341156in}}%
\pgfpathlineto{\pgfqpoint{4.731472in}{2.305956in}}%
\pgfpathlineto{\pgfqpoint{4.731742in}{2.433747in}}%
\pgfpathlineto{\pgfqpoint{4.732493in}{2.399200in}}%
\pgfpathlineto{\pgfqpoint{4.733588in}{2.448331in}}%
\pgfpathlineto{\pgfqpoint{4.732690in}{2.317220in}}%
\pgfpathlineto{\pgfqpoint{4.733625in}{2.429594in}}%
\pgfpathlineto{\pgfqpoint{4.733945in}{2.298978in}}%
\pgfpathlineto{\pgfqpoint{4.734376in}{2.444390in}}%
\pgfpathlineto{\pgfqpoint{4.734782in}{2.395279in}}%
\pgfpathlineto{\pgfqpoint{4.734819in}{2.425995in}}%
\pgfpathlineto{\pgfqpoint{4.735755in}{2.315560in}}%
\pgfpathlineto{\pgfqpoint{4.735853in}{2.350884in}}%
\pgfpathlineto{\pgfqpoint{4.735865in}{2.350791in}}%
\pgfpathlineto{\pgfqpoint{4.735878in}{2.352768in}}%
\pgfpathlineto{\pgfqpoint{4.736727in}{2.433353in}}%
\pgfpathlineto{\pgfqpoint{4.736345in}{2.320984in}}%
\pgfpathlineto{\pgfqpoint{4.736948in}{2.353234in}}%
\pgfpathlineto{\pgfqpoint{4.737625in}{2.312028in}}%
\pgfpathlineto{\pgfqpoint{4.737478in}{2.434744in}}%
\pgfpathlineto{\pgfqpoint{4.738031in}{2.399326in}}%
\pgfpathlineto{\pgfqpoint{4.738721in}{2.421846in}}%
\pgfpathlineto{\pgfqpoint{4.738881in}{2.304381in}}%
\pgfpathlineto{\pgfqpoint{4.739139in}{2.406156in}}%
\pgfpathlineto{\pgfqpoint{4.739336in}{2.409169in}}%
\pgfpathlineto{\pgfqpoint{4.739250in}{2.390077in}}%
\pgfpathlineto{\pgfqpoint{4.739361in}{2.395596in}}%
\pgfpathlineto{\pgfqpoint{4.739447in}{2.306640in}}%
\pgfpathlineto{\pgfqpoint{4.740382in}{2.441613in}}%
\pgfpathlineto{\pgfqpoint{4.740468in}{2.397900in}}%
\pgfpathlineto{\pgfqpoint{4.740739in}{2.307146in}}%
\pgfpathlineto{\pgfqpoint{4.741170in}{2.439681in}}%
\pgfpathlineto{\pgfqpoint{4.741576in}{2.399716in}}%
\pgfpathlineto{\pgfqpoint{4.741613in}{2.431576in}}%
\pgfpathlineto{\pgfqpoint{4.742536in}{2.310195in}}%
\pgfpathlineto{\pgfqpoint{4.742647in}{2.352947in}}%
\pgfpathlineto{\pgfqpoint{4.743484in}{2.439599in}}%
\pgfpathlineto{\pgfqpoint{4.743213in}{2.320133in}}%
\pgfpathlineto{\pgfqpoint{4.743742in}{2.368865in}}%
\pgfpathlineto{\pgfqpoint{4.743841in}{2.310648in}}%
\pgfpathlineto{\pgfqpoint{4.744271in}{2.441344in}}%
\pgfpathlineto{\pgfqpoint{4.744850in}{2.369416in}}%
\pgfpathlineto{\pgfqpoint{4.746007in}{2.430690in}}%
\pgfpathlineto{\pgfqpoint{4.745662in}{2.300871in}}%
\pgfpathlineto{\pgfqpoint{4.746031in}{2.421199in}}%
\pgfpathlineto{\pgfqpoint{4.746942in}{2.314421in}}%
\pgfpathlineto{\pgfqpoint{4.746598in}{2.439190in}}%
\pgfpathlineto{\pgfqpoint{4.747164in}{2.383595in}}%
\pgfpathlineto{\pgfqpoint{4.747521in}{2.328068in}}%
\pgfpathlineto{\pgfqpoint{4.747373in}{2.444890in}}%
\pgfpathlineto{\pgfqpoint{4.748333in}{2.372844in}}%
\pgfpathlineto{\pgfqpoint{4.749219in}{2.423641in}}%
\pgfpathlineto{\pgfqpoint{4.748764in}{2.299099in}}%
\pgfpathlineto{\pgfqpoint{4.749391in}{2.312809in}}%
\pgfpathlineto{\pgfqpoint{4.749404in}{2.309373in}}%
\pgfpathlineto{\pgfqpoint{4.749699in}{2.445646in}}%
\pgfpathlineto{\pgfqpoint{4.750216in}{2.388071in}}%
\pgfpathlineto{\pgfqpoint{4.750462in}{2.442879in}}%
\pgfpathlineto{\pgfqpoint{4.750622in}{2.331780in}}%
\pgfpathlineto{\pgfqpoint{4.751188in}{2.334742in}}%
\pgfpathlineto{\pgfqpoint{4.751533in}{2.427114in}}%
\pgfpathlineto{\pgfqpoint{4.751853in}{2.303227in}}%
\pgfpathlineto{\pgfqpoint{4.752395in}{2.361065in}}%
\pgfpathlineto{\pgfqpoint{4.752493in}{2.310668in}}%
\pgfpathlineto{\pgfqpoint{4.752801in}{2.437384in}}%
\pgfpathlineto{\pgfqpoint{4.753478in}{2.365762in}}%
\pgfpathlineto{\pgfqpoint{4.753995in}{2.424605in}}%
\pgfpathlineto{\pgfqpoint{4.753711in}{2.330383in}}%
\pgfpathlineto{\pgfqpoint{4.754635in}{2.423089in}}%
\pgfpathlineto{\pgfqpoint{4.754881in}{2.327615in}}%
\pgfpathlineto{\pgfqpoint{4.754955in}{2.301991in}}%
\pgfpathlineto{\pgfqpoint{4.755878in}{2.433061in}}%
\pgfpathlineto{\pgfqpoint{4.755939in}{2.394617in}}%
\pgfpathlineto{\pgfqpoint{4.757084in}{2.432147in}}%
\pgfpathlineto{\pgfqpoint{4.756198in}{2.331072in}}%
\pgfpathlineto{\pgfqpoint{4.757096in}{2.432120in}}%
\pgfpathlineto{\pgfqpoint{4.758007in}{2.298876in}}%
\pgfpathlineto{\pgfqpoint{4.758228in}{2.395345in}}%
\pgfpathlineto{\pgfqpoint{4.758327in}{2.437238in}}%
\pgfpathlineto{\pgfqpoint{4.758684in}{2.318397in}}%
\pgfpathlineto{\pgfqpoint{4.759238in}{2.337961in}}%
\pgfpathlineto{\pgfqpoint{4.760185in}{2.439985in}}%
\pgfpathlineto{\pgfqpoint{4.759287in}{2.327421in}}%
\pgfpathlineto{\pgfqpoint{4.760419in}{2.363408in}}%
\pgfpathlineto{\pgfqpoint{4.761145in}{2.290394in}}%
\pgfpathlineto{\pgfqpoint{4.760961in}{2.432682in}}%
\pgfpathlineto{\pgfqpoint{4.761453in}{2.427179in}}%
\pgfpathlineto{\pgfqpoint{4.761465in}{2.427748in}}%
\pgfpathlineto{\pgfqpoint{4.761613in}{2.392304in}}%
\pgfpathlineto{\pgfqpoint{4.761908in}{2.323211in}}%
\pgfpathlineto{\pgfqpoint{4.762093in}{2.430490in}}%
\pgfpathlineto{\pgfqpoint{4.762745in}{2.357031in}}%
\pgfpathlineto{\pgfqpoint{4.763311in}{2.440003in}}%
\pgfpathlineto{\pgfqpoint{4.763607in}{2.307577in}}%
\pgfpathlineto{\pgfqpoint{4.763927in}{2.429585in}}%
\pgfpathlineto{\pgfqpoint{4.764247in}{2.291216in}}%
\pgfpathlineto{\pgfqpoint{4.765108in}{2.404399in}}%
\pgfpathlineto{\pgfqpoint{4.765748in}{2.414440in}}%
\pgfpathlineto{\pgfqpoint{4.766068in}{2.314181in}}%
\pgfpathlineto{\pgfqpoint{4.766130in}{2.344109in}}%
\pgfpathlineto{\pgfqpoint{4.766979in}{2.433279in}}%
\pgfpathlineto{\pgfqpoint{4.766647in}{2.310000in}}%
\pgfpathlineto{\pgfqpoint{4.767250in}{2.355995in}}%
\pgfpathlineto{\pgfqpoint{4.767336in}{2.311748in}}%
\pgfpathlineto{\pgfqpoint{4.767582in}{2.433917in}}%
\pgfpathlineto{\pgfqpoint{4.768333in}{2.396999in}}%
\pgfpathlineto{\pgfqpoint{4.769428in}{2.430300in}}%
\pgfpathlineto{\pgfqpoint{4.769194in}{2.316600in}}%
\pgfpathlineto{\pgfqpoint{4.769453in}{2.418157in}}%
\pgfpathlineto{\pgfqpoint{4.769785in}{2.294402in}}%
\pgfpathlineto{\pgfqpoint{4.770142in}{2.429824in}}%
\pgfpathlineto{\pgfqpoint{4.770610in}{2.375529in}}%
\pgfpathlineto{\pgfqpoint{4.770721in}{2.423033in}}%
\pgfpathlineto{\pgfqpoint{4.771644in}{2.310452in}}%
\pgfpathlineto{\pgfqpoint{4.771705in}{2.359897in}}%
\pgfpathlineto{\pgfqpoint{4.772542in}{2.444741in}}%
\pgfpathlineto{\pgfqpoint{4.772259in}{2.314753in}}%
\pgfpathlineto{\pgfqpoint{4.772788in}{2.354981in}}%
\pgfpathlineto{\pgfqpoint{4.772899in}{2.304760in}}%
\pgfpathlineto{\pgfqpoint{4.773342in}{2.442584in}}%
\pgfpathlineto{\pgfqpoint{4.773822in}{2.405039in}}%
\pgfpathlineto{\pgfqpoint{4.774573in}{2.425010in}}%
\pgfpathlineto{\pgfqpoint{4.774241in}{2.325598in}}%
\pgfpathlineto{\pgfqpoint{4.774671in}{2.357948in}}%
\pgfpathlineto{\pgfqpoint{4.774721in}{2.307215in}}%
\pgfpathlineto{\pgfqpoint{4.775705in}{2.441877in}}%
\pgfpathlineto{\pgfqpoint{4.775754in}{2.396574in}}%
\pgfpathlineto{\pgfqpoint{4.776616in}{2.295866in}}%
\pgfpathlineto{\pgfqpoint{4.776431in}{2.429205in}}%
\pgfpathlineto{\pgfqpoint{4.776850in}{2.397949in}}%
\pgfpathlineto{\pgfqpoint{4.777478in}{2.441866in}}%
\pgfpathlineto{\pgfqpoint{4.777207in}{2.324955in}}%
\pgfpathlineto{\pgfqpoint{4.777773in}{2.335085in}}%
\pgfpathlineto{\pgfqpoint{4.778450in}{2.308163in}}%
\pgfpathlineto{\pgfqpoint{4.778278in}{2.437940in}}%
\pgfpathlineto{\pgfqpoint{4.778770in}{2.404049in}}%
\pgfpathlineto{\pgfqpoint{4.778905in}{2.426441in}}%
\pgfpathlineto{\pgfqpoint{4.779102in}{2.318024in}}%
\pgfpathlineto{\pgfqpoint{4.779767in}{2.331422in}}%
\pgfpathlineto{\pgfqpoint{4.779779in}{2.330545in}}%
\pgfpathlineto{\pgfqpoint{4.779927in}{2.394050in}}%
\pgfpathlineto{\pgfqpoint{4.780641in}{2.430987in}}%
\pgfpathlineto{\pgfqpoint{4.780321in}{2.310502in}}%
\pgfpathlineto{\pgfqpoint{4.780997in}{2.354037in}}%
\pgfpathlineto{\pgfqpoint{4.781564in}{2.317944in}}%
\pgfpathlineto{\pgfqpoint{4.781207in}{2.432877in}}%
\pgfpathlineto{\pgfqpoint{4.781994in}{2.427448in}}%
\pgfpathlineto{\pgfqpoint{4.782007in}{2.437708in}}%
\pgfpathlineto{\pgfqpoint{4.782142in}{2.315715in}}%
\pgfpathlineto{\pgfqpoint{4.783077in}{2.407059in}}%
\pgfpathlineto{\pgfqpoint{4.783385in}{2.313773in}}%
\pgfpathlineto{\pgfqpoint{4.783250in}{2.421431in}}%
\pgfpathlineto{\pgfqpoint{4.784259in}{2.390334in}}%
\pgfpathlineto{\pgfqpoint{4.784911in}{2.427086in}}%
\pgfpathlineto{\pgfqpoint{4.785244in}{2.314481in}}%
\pgfpathlineto{\pgfqpoint{4.785342in}{2.375498in}}%
\pgfpathlineto{\pgfqpoint{4.786524in}{2.316676in}}%
\pgfpathlineto{\pgfqpoint{4.786142in}{2.429579in}}%
\pgfpathlineto{\pgfqpoint{4.786536in}{2.324101in}}%
\pgfpathlineto{\pgfqpoint{4.787373in}{2.419537in}}%
\pgfpathlineto{\pgfqpoint{4.787114in}{2.321672in}}%
\pgfpathlineto{\pgfqpoint{4.787668in}{2.364601in}}%
\pgfpathlineto{\pgfqpoint{4.787730in}{2.325964in}}%
\pgfpathlineto{\pgfqpoint{4.788382in}{2.316954in}}%
\pgfpathlineto{\pgfqpoint{4.788025in}{2.423596in}}%
\pgfpathlineto{\pgfqpoint{4.788677in}{2.388925in}}%
\pgfpathlineto{\pgfqpoint{4.789256in}{2.430236in}}%
\pgfpathlineto{\pgfqpoint{4.788985in}{2.324098in}}%
\pgfpathlineto{\pgfqpoint{4.789711in}{2.356707in}}%
\pgfpathlineto{\pgfqpoint{4.790191in}{2.325262in}}%
\pgfpathlineto{\pgfqpoint{4.790487in}{2.430037in}}%
\pgfpathlineto{\pgfqpoint{4.790671in}{2.406588in}}%
\pgfpathlineto{\pgfqpoint{4.791127in}{2.422551in}}%
\pgfpathlineto{\pgfqpoint{4.791484in}{2.328078in}}%
\pgfpathlineto{\pgfqpoint{4.791754in}{2.394558in}}%
\pgfpathlineto{\pgfqpoint{4.792185in}{2.334355in}}%
\pgfpathlineto{\pgfqpoint{4.791927in}{2.435756in}}%
\pgfpathlineto{\pgfqpoint{4.792887in}{2.379031in}}%
\pgfpathlineto{\pgfqpoint{4.793588in}{2.426324in}}%
\pgfpathlineto{\pgfqpoint{4.793293in}{2.319988in}}%
\pgfpathlineto{\pgfqpoint{4.793896in}{2.350514in}}%
\pgfpathlineto{\pgfqpoint{4.793908in}{2.350688in}}%
\pgfpathlineto{\pgfqpoint{4.793933in}{2.339417in}}%
\pgfpathlineto{\pgfqpoint{4.794056in}{2.323342in}}%
\pgfpathlineto{\pgfqpoint{4.794228in}{2.426970in}}%
\pgfpathlineto{\pgfqpoint{4.794979in}{2.394944in}}%
\pgfpathlineto{\pgfqpoint{4.795028in}{2.439682in}}%
\pgfpathlineto{\pgfqpoint{4.795287in}{2.325619in}}%
\pgfpathlineto{\pgfqpoint{4.796087in}{2.396701in}}%
\pgfpathlineto{\pgfqpoint{4.797170in}{2.312134in}}%
\pgfpathlineto{\pgfqpoint{4.796690in}{2.428974in}}%
\pgfpathlineto{\pgfqpoint{4.797281in}{2.384251in}}%
\pgfpathlineto{\pgfqpoint{4.798130in}{2.449780in}}%
\pgfpathlineto{\pgfqpoint{4.797687in}{2.321521in}}%
\pgfpathlineto{\pgfqpoint{4.798364in}{2.354483in}}%
\pgfpathlineto{\pgfqpoint{4.799496in}{2.315860in}}%
\pgfpathlineto{\pgfqpoint{4.798548in}{2.435545in}}%
\pgfpathlineto{\pgfqpoint{4.799508in}{2.316249in}}%
\pgfpathlineto{\pgfqpoint{4.800616in}{2.442312in}}%
\pgfpathlineto{\pgfqpoint{4.800271in}{2.311704in}}%
\pgfpathlineto{\pgfqpoint{4.800714in}{2.338053in}}%
\pgfpathlineto{\pgfqpoint{4.800788in}{2.323237in}}%
\pgfpathlineto{\pgfqpoint{4.801133in}{2.403248in}}%
\pgfpathlineto{\pgfqpoint{4.801182in}{2.395462in}}%
\pgfpathlineto{\pgfqpoint{4.801231in}{2.442147in}}%
\pgfpathlineto{\pgfqpoint{4.801502in}{2.326808in}}%
\pgfpathlineto{\pgfqpoint{4.802290in}{2.401651in}}%
\pgfpathlineto{\pgfqpoint{4.803360in}{2.309729in}}%
\pgfpathlineto{\pgfqpoint{4.802979in}{2.430179in}}%
\pgfpathlineto{\pgfqpoint{4.803447in}{2.371518in}}%
\pgfpathlineto{\pgfqpoint{4.804333in}{2.439887in}}%
\pgfpathlineto{\pgfqpoint{4.803877in}{2.317286in}}%
\pgfpathlineto{\pgfqpoint{4.804493in}{2.346340in}}%
\pgfpathlineto{\pgfqpoint{4.804505in}{2.346337in}}%
\pgfpathlineto{\pgfqpoint{4.804751in}{2.426044in}}%
\pgfpathlineto{\pgfqpoint{4.804616in}{2.329883in}}%
\pgfpathlineto{\pgfqpoint{4.805625in}{2.353991in}}%
\pgfpathlineto{\pgfqpoint{4.806462in}{2.303881in}}%
\pgfpathlineto{\pgfqpoint{4.806634in}{2.426688in}}%
\pgfpathlineto{\pgfqpoint{4.806647in}{2.425934in}}%
\pgfpathlineto{\pgfqpoint{4.806991in}{2.316244in}}%
\pgfpathlineto{\pgfqpoint{4.807434in}{2.438115in}}%
\pgfpathlineto{\pgfqpoint{4.807840in}{2.411583in}}%
\pgfpathlineto{\pgfqpoint{4.807865in}{2.427064in}}%
\pgfpathlineto{\pgfqpoint{4.808850in}{2.303597in}}%
\pgfpathlineto{\pgfqpoint{4.808899in}{2.356811in}}%
\pgfpathlineto{\pgfqpoint{4.809576in}{2.299692in}}%
\pgfpathlineto{\pgfqpoint{4.809157in}{2.441044in}}%
\pgfpathlineto{\pgfqpoint{4.809871in}{2.400252in}}%
\pgfpathlineto{\pgfqpoint{4.810967in}{2.433903in}}%
\pgfpathlineto{\pgfqpoint{4.810093in}{2.321441in}}%
\pgfpathlineto{\pgfqpoint{4.810991in}{2.407671in}}%
\pgfpathlineto{\pgfqpoint{4.811939in}{2.294050in}}%
\pgfpathlineto{\pgfqpoint{4.811754in}{2.427887in}}%
\pgfpathlineto{\pgfqpoint{4.812099in}{2.400126in}}%
\pgfpathlineto{\pgfqpoint{4.812271in}{2.446034in}}%
\pgfpathlineto{\pgfqpoint{4.812677in}{2.305465in}}%
\pgfpathlineto{\pgfqpoint{4.813120in}{2.357464in}}%
\pgfpathlineto{\pgfqpoint{4.813194in}{2.321637in}}%
\pgfpathlineto{\pgfqpoint{4.814093in}{2.427748in}}%
\pgfpathlineto{\pgfqpoint{4.814191in}{2.382883in}}%
\pgfpathlineto{\pgfqpoint{4.815373in}{2.433529in}}%
\pgfpathlineto{\pgfqpoint{4.815040in}{2.290824in}}%
\pgfpathlineto{\pgfqpoint{4.815385in}{2.423946in}}%
\pgfpathlineto{\pgfqpoint{4.815791in}{2.322901in}}%
\pgfpathlineto{\pgfqpoint{4.815988in}{2.441120in}}%
\pgfpathlineto{\pgfqpoint{4.816517in}{2.402597in}}%
\pgfpathlineto{\pgfqpoint{4.817219in}{2.438282in}}%
\pgfpathlineto{\pgfqpoint{4.816911in}{2.326157in}}%
\pgfpathlineto{\pgfqpoint{4.817576in}{2.377021in}}%
\pgfpathlineto{\pgfqpoint{4.818142in}{2.295623in}}%
\pgfpathlineto{\pgfqpoint{4.817847in}{2.442376in}}%
\pgfpathlineto{\pgfqpoint{4.818720in}{2.342008in}}%
\pgfpathlineto{\pgfqpoint{4.818757in}{2.334461in}}%
\pgfpathlineto{\pgfqpoint{4.819077in}{2.428775in}}%
\pgfpathlineto{\pgfqpoint{4.819533in}{2.382626in}}%
\pgfpathlineto{\pgfqpoint{4.820320in}{2.422902in}}%
\pgfpathlineto{\pgfqpoint{4.820567in}{2.321302in}}%
\pgfpathlineto{\pgfqpoint{4.820603in}{2.334465in}}%
\pgfpathlineto{\pgfqpoint{4.820616in}{2.334336in}}%
\pgfpathlineto{\pgfqpoint{4.820628in}{2.335804in}}%
\pgfpathlineto{\pgfqpoint{4.820936in}{2.443431in}}%
\pgfpathlineto{\pgfqpoint{4.821243in}{2.306396in}}%
\pgfpathlineto{\pgfqpoint{4.821773in}{2.373115in}}%
\pgfpathlineto{\pgfqpoint{4.821859in}{2.318760in}}%
\pgfpathlineto{\pgfqpoint{4.822179in}{2.434120in}}%
\pgfpathlineto{\pgfqpoint{4.822880in}{2.361158in}}%
\pgfpathlineto{\pgfqpoint{4.824037in}{2.443104in}}%
\pgfpathlineto{\pgfqpoint{4.823705in}{2.311902in}}%
\pgfpathlineto{\pgfqpoint{4.824173in}{2.400489in}}%
\pgfpathlineto{\pgfqpoint{4.824960in}{2.323909in}}%
\pgfpathlineto{\pgfqpoint{4.824579in}{2.428469in}}%
\pgfpathlineto{\pgfqpoint{4.825268in}{2.398351in}}%
\pgfpathlineto{\pgfqpoint{4.825391in}{2.427861in}}%
\pgfpathlineto{\pgfqpoint{4.825563in}{2.332135in}}%
\pgfpathlineto{\pgfqpoint{4.826376in}{2.395663in}}%
\pgfpathlineto{\pgfqpoint{4.826437in}{2.438868in}}%
\pgfpathlineto{\pgfqpoint{4.826807in}{2.298802in}}%
\pgfpathlineto{\pgfqpoint{4.827373in}{2.360294in}}%
\pgfpathlineto{\pgfqpoint{4.827422in}{2.323343in}}%
\pgfpathlineto{\pgfqpoint{4.827754in}{2.433455in}}%
\pgfpathlineto{\pgfqpoint{4.828431in}{2.373323in}}%
\pgfpathlineto{\pgfqpoint{4.828493in}{2.437590in}}%
\pgfpathlineto{\pgfqpoint{4.828665in}{2.326025in}}%
\pgfpathlineto{\pgfqpoint{4.829600in}{2.418776in}}%
\pgfpathlineto{\pgfqpoint{4.830351in}{2.439503in}}%
\pgfpathlineto{\pgfqpoint{4.829908in}{2.305273in}}%
\pgfpathlineto{\pgfqpoint{4.830474in}{2.352642in}}%
\pgfpathlineto{\pgfqpoint{4.830487in}{2.352690in}}%
\pgfpathlineto{\pgfqpoint{4.830499in}{2.348545in}}%
\pgfpathlineto{\pgfqpoint{4.830523in}{2.326715in}}%
\pgfpathlineto{\pgfqpoint{4.830856in}{2.444555in}}%
\pgfpathlineto{\pgfqpoint{4.831557in}{2.378172in}}%
\pgfpathlineto{\pgfqpoint{4.831594in}{2.440749in}}%
\pgfpathlineto{\pgfqpoint{4.832271in}{2.324708in}}%
\pgfpathlineto{\pgfqpoint{4.832690in}{2.426819in}}%
\pgfpathlineto{\pgfqpoint{4.833453in}{2.443667in}}%
\pgfpathlineto{\pgfqpoint{4.832997in}{2.299954in}}%
\pgfpathlineto{\pgfqpoint{4.833563in}{2.349633in}}%
\pgfpathlineto{\pgfqpoint{4.833625in}{2.334815in}}%
\pgfpathlineto{\pgfqpoint{4.833957in}{2.450657in}}%
\pgfpathlineto{\pgfqpoint{4.834450in}{2.397669in}}%
\pgfpathlineto{\pgfqpoint{4.834696in}{2.430164in}}%
\pgfpathlineto{\pgfqpoint{4.835360in}{2.324769in}}%
\pgfpathlineto{\pgfqpoint{4.835533in}{2.383686in}}%
\pgfpathlineto{\pgfqpoint{4.836099in}{2.297527in}}%
\pgfpathlineto{\pgfqpoint{4.835803in}{2.441847in}}%
\pgfpathlineto{\pgfqpoint{4.836702in}{2.327952in}}%
\pgfpathlineto{\pgfqpoint{4.837059in}{2.444932in}}%
\pgfpathlineto{\pgfqpoint{4.837896in}{2.371512in}}%
\pgfpathlineto{\pgfqpoint{4.838462in}{2.319016in}}%
\pgfpathlineto{\pgfqpoint{4.838917in}{2.453412in}}%
\pgfpathlineto{\pgfqpoint{4.838991in}{2.362357in}}%
\pgfpathlineto{\pgfqpoint{4.839422in}{2.444482in}}%
\pgfpathlineto{\pgfqpoint{4.839200in}{2.309345in}}%
\pgfpathlineto{\pgfqpoint{4.840123in}{2.412518in}}%
\pgfpathlineto{\pgfqpoint{4.840886in}{2.428601in}}%
\pgfpathlineto{\pgfqpoint{4.841046in}{2.316230in}}%
\pgfpathlineto{\pgfqpoint{4.841120in}{2.348098in}}%
\pgfpathlineto{\pgfqpoint{4.841637in}{2.307271in}}%
\pgfpathlineto{\pgfqpoint{4.841268in}{2.440448in}}%
\pgfpathlineto{\pgfqpoint{4.841970in}{2.407580in}}%
\pgfpathlineto{\pgfqpoint{4.842006in}{2.453749in}}%
\pgfpathlineto{\pgfqpoint{4.842893in}{2.308438in}}%
\pgfpathlineto{\pgfqpoint{4.843028in}{2.354435in}}%
\pgfpathlineto{\pgfqpoint{4.843348in}{2.455062in}}%
\pgfpathlineto{\pgfqpoint{4.843508in}{2.323728in}}%
\pgfpathlineto{\pgfqpoint{4.844123in}{2.345216in}}%
\pgfpathlineto{\pgfqpoint{4.844751in}{2.288335in}}%
\pgfpathlineto{\pgfqpoint{4.845108in}{2.449025in}}%
\pgfpathlineto{\pgfqpoint{4.845170in}{2.397229in}}%
\pgfpathlineto{\pgfqpoint{4.845231in}{2.407113in}}%
\pgfpathlineto{\pgfqpoint{4.845366in}{2.294769in}}%
\pgfpathlineto{\pgfqpoint{4.845834in}{2.445397in}}%
\pgfpathlineto{\pgfqpoint{4.846326in}{2.393107in}}%
\pgfpathlineto{\pgfqpoint{4.846450in}{2.473795in}}%
\pgfpathlineto{\pgfqpoint{4.846597in}{2.312077in}}%
\pgfpathlineto{\pgfqpoint{4.847446in}{2.418724in}}%
\pgfpathlineto{\pgfqpoint{4.848308in}{2.466292in}}%
\pgfpathlineto{\pgfqpoint{4.847853in}{2.286722in}}%
\pgfpathlineto{\pgfqpoint{4.848443in}{2.331750in}}%
\pgfpathlineto{\pgfqpoint{4.848468in}{2.308554in}}%
\pgfpathlineto{\pgfqpoint{4.848813in}{2.448421in}}%
\pgfpathlineto{\pgfqpoint{4.849502in}{2.381166in}}%
\pgfpathlineto{\pgfqpoint{4.849551in}{2.466086in}}%
\pgfpathlineto{\pgfqpoint{4.849699in}{2.300309in}}%
\pgfpathlineto{\pgfqpoint{4.850634in}{2.429094in}}%
\pgfpathlineto{\pgfqpoint{4.850930in}{2.281262in}}%
\pgfpathlineto{\pgfqpoint{4.851410in}{2.489784in}}%
\pgfpathlineto{\pgfqpoint{4.851791in}{2.373771in}}%
\pgfpathlineto{\pgfqpoint{4.851926in}{2.469595in}}%
\pgfpathlineto{\pgfqpoint{4.852788in}{2.305770in}}%
\pgfpathlineto{\pgfqpoint{4.852874in}{2.348549in}}%
\pgfpathlineto{\pgfqpoint{4.853317in}{2.309745in}}%
\pgfpathlineto{\pgfqpoint{4.853157in}{2.439116in}}%
\pgfpathlineto{\pgfqpoint{4.853674in}{2.418653in}}%
\pgfpathlineto{\pgfqpoint{4.854511in}{2.475728in}}%
\pgfpathlineto{\pgfqpoint{4.854031in}{2.282775in}}%
\pgfpathlineto{\pgfqpoint{4.854683in}{2.340927in}}%
\pgfpathlineto{\pgfqpoint{4.854806in}{2.324115in}}%
\pgfpathlineto{\pgfqpoint{4.854905in}{2.377549in}}%
\pgfpathlineto{\pgfqpoint{4.855028in}{2.465713in}}%
\pgfpathlineto{\pgfqpoint{4.855890in}{2.307806in}}%
\pgfpathlineto{\pgfqpoint{4.855988in}{2.344234in}}%
\pgfpathlineto{\pgfqpoint{4.856000in}{2.343788in}}%
\pgfpathlineto{\pgfqpoint{4.856062in}{2.377452in}}%
\pgfpathlineto{\pgfqpoint{4.856874in}{2.474231in}}%
\pgfpathlineto{\pgfqpoint{4.856419in}{2.308169in}}%
\pgfpathlineto{\pgfqpoint{4.857096in}{2.330441in}}%
\pgfpathlineto{\pgfqpoint{4.857170in}{2.295618in}}%
\pgfpathlineto{\pgfqpoint{4.857600in}{2.461959in}}%
\pgfpathlineto{\pgfqpoint{4.858093in}{2.402186in}}%
\pgfpathlineto{\pgfqpoint{4.858130in}{2.449812in}}%
\pgfpathlineto{\pgfqpoint{4.859003in}{2.317711in}}%
\pgfpathlineto{\pgfqpoint{4.859200in}{2.416881in}}%
\pgfpathlineto{\pgfqpoint{4.859348in}{2.448503in}}%
\pgfpathlineto{\pgfqpoint{4.859496in}{2.352528in}}%
\pgfpathlineto{\pgfqpoint{4.860333in}{2.299706in}}%
\pgfpathlineto{\pgfqpoint{4.859976in}{2.473297in}}%
\pgfpathlineto{\pgfqpoint{4.860566in}{2.418235in}}%
\pgfpathlineto{\pgfqpoint{4.860702in}{2.455033in}}%
\pgfpathlineto{\pgfqpoint{4.860837in}{2.311682in}}%
\pgfpathlineto{\pgfqpoint{4.861588in}{2.335994in}}%
\pgfpathlineto{\pgfqpoint{4.862117in}{2.315258in}}%
\pgfpathlineto{\pgfqpoint{4.862326in}{2.427846in}}%
\pgfpathlineto{\pgfqpoint{4.862523in}{2.396249in}}%
\pgfpathlineto{\pgfqpoint{4.863594in}{2.460145in}}%
\pgfpathlineto{\pgfqpoint{4.863385in}{2.291722in}}%
\pgfpathlineto{\pgfqpoint{4.863631in}{2.421706in}}%
\pgfpathlineto{\pgfqpoint{4.864579in}{2.311602in}}%
\pgfpathlineto{\pgfqpoint{4.864406in}{2.468384in}}%
\pgfpathlineto{\pgfqpoint{4.864776in}{2.373352in}}%
\pgfpathlineto{\pgfqpoint{4.865551in}{2.427903in}}%
\pgfpathlineto{\pgfqpoint{4.865723in}{2.316334in}}%
\pgfpathlineto{\pgfqpoint{4.865822in}{2.326102in}}%
\pgfpathlineto{\pgfqpoint{4.866449in}{2.265700in}}%
\pgfpathlineto{\pgfqpoint{4.865969in}{2.439587in}}%
\pgfpathlineto{\pgfqpoint{4.866671in}{2.402992in}}%
\pgfpathlineto{\pgfqpoint{4.867533in}{2.479306in}}%
\pgfpathlineto{\pgfqpoint{4.867176in}{2.283341in}}%
\pgfpathlineto{\pgfqpoint{4.867754in}{2.344385in}}%
\pgfpathlineto{\pgfqpoint{4.868271in}{2.425363in}}%
\pgfpathlineto{\pgfqpoint{4.868443in}{2.323772in}}%
\pgfpathlineto{\pgfqpoint{4.868456in}{2.320940in}}%
\pgfpathlineto{\pgfqpoint{4.868800in}{2.414573in}}%
\pgfpathlineto{\pgfqpoint{4.869280in}{2.375896in}}%
\pgfpathlineto{\pgfqpoint{4.870351in}{2.413058in}}%
\pgfpathlineto{\pgfqpoint{4.869576in}{2.325000in}}%
\pgfpathlineto{\pgfqpoint{4.870388in}{2.382869in}}%
\pgfpathlineto{\pgfqpoint{4.871225in}{2.335276in}}%
\pgfpathlineto{\pgfqpoint{4.870880in}{2.413218in}}%
\pgfpathlineto{\pgfqpoint{4.871471in}{2.405697in}}%
\pgfpathlineto{\pgfqpoint{4.872222in}{2.422898in}}%
\pgfpathlineto{\pgfqpoint{4.871816in}{2.342907in}}%
\pgfpathlineto{\pgfqpoint{4.872431in}{2.361484in}}%
\pgfpathlineto{\pgfqpoint{4.872542in}{2.354337in}}%
\pgfpathlineto{\pgfqpoint{4.873145in}{2.337972in}}%
\pgfpathlineto{\pgfqpoint{4.872985in}{2.411991in}}%
\pgfpathlineto{\pgfqpoint{4.873613in}{2.365908in}}%
\pgfpathlineto{\pgfqpoint{4.874511in}{2.420175in}}%
\pgfpathlineto{\pgfqpoint{4.874166in}{2.331494in}}%
\pgfpathlineto{\pgfqpoint{4.874696in}{2.360929in}}%
\pgfpathlineto{\pgfqpoint{4.875693in}{2.298483in}}%
\pgfpathlineto{\pgfqpoint{4.875225in}{2.425403in}}%
\pgfpathlineto{\pgfqpoint{4.875742in}{2.369294in}}%
\pgfpathlineto{\pgfqpoint{4.876456in}{2.439963in}}%
\pgfpathlineto{\pgfqpoint{4.876308in}{2.330559in}}%
\pgfpathlineto{\pgfqpoint{4.876776in}{2.332738in}}%
\pgfpathlineto{\pgfqpoint{4.876800in}{2.310094in}}%
\pgfpathlineto{\pgfqpoint{4.877354in}{2.430021in}}%
\pgfpathlineto{\pgfqpoint{4.877834in}{2.358194in}}%
\pgfpathlineto{\pgfqpoint{4.878573in}{2.439538in}}%
\pgfpathlineto{\pgfqpoint{4.878917in}{2.305393in}}%
\pgfpathlineto{\pgfqpoint{4.879274in}{2.448763in}}%
\pgfpathlineto{\pgfqpoint{4.879889in}{2.273705in}}%
\pgfpathlineto{\pgfqpoint{4.880271in}{2.387889in}}%
\pgfpathlineto{\pgfqpoint{4.881133in}{2.268675in}}%
\pgfpathlineto{\pgfqpoint{4.880665in}{2.478555in}}%
\pgfpathlineto{\pgfqpoint{4.881354in}{2.466285in}}%
\pgfpathlineto{\pgfqpoint{4.881366in}{2.492062in}}%
\pgfpathlineto{\pgfqpoint{4.881834in}{2.231261in}}%
\pgfpathlineto{\pgfqpoint{4.882412in}{2.382640in}}%
\pgfpathlineto{\pgfqpoint{4.882511in}{2.289097in}}%
\pgfpathlineto{\pgfqpoint{4.883372in}{2.448824in}}%
\pgfpathlineto{\pgfqpoint{4.883409in}{2.459455in}}%
\pgfpathlineto{\pgfqpoint{4.883754in}{2.265605in}}%
\pgfpathlineto{\pgfqpoint{4.884406in}{2.418490in}}%
\pgfpathlineto{\pgfqpoint{4.884665in}{2.268890in}}%
\pgfpathlineto{\pgfqpoint{4.885403in}{2.451042in}}%
\pgfpathlineto{\pgfqpoint{4.885514in}{2.393221in}}%
\pgfpathlineto{\pgfqpoint{4.886376in}{2.543935in}}%
\pgfpathlineto{\pgfqpoint{4.885772in}{2.277908in}}%
\pgfpathlineto{\pgfqpoint{4.886585in}{2.356749in}}%
\pgfpathlineto{\pgfqpoint{4.886732in}{2.167032in}}%
\pgfpathlineto{\pgfqpoint{4.887102in}{2.466885in}}%
\pgfpathlineto{\pgfqpoint{4.887717in}{2.316421in}}%
\pgfpathlineto{\pgfqpoint{4.888308in}{2.573402in}}%
\pgfpathlineto{\pgfqpoint{4.888628in}{2.244715in}}%
\pgfpathlineto{\pgfqpoint{4.888665in}{2.185105in}}%
\pgfpathlineto{\pgfqpoint{4.889256in}{2.472481in}}%
\pgfpathlineto{\pgfqpoint{4.889686in}{2.333128in}}%
\pgfpathlineto{\pgfqpoint{4.890449in}{2.456466in}}%
\pgfpathlineto{\pgfqpoint{4.890634in}{2.288372in}}%
\pgfpathlineto{\pgfqpoint{4.890819in}{2.371262in}}%
\pgfpathlineto{\pgfqpoint{4.891569in}{2.198523in}}%
\pgfpathlineto{\pgfqpoint{4.891200in}{2.531931in}}%
\pgfpathlineto{\pgfqpoint{4.891902in}{2.417663in}}%
\pgfpathlineto{\pgfqpoint{4.892197in}{2.472512in}}%
\pgfpathlineto{\pgfqpoint{4.892517in}{2.324884in}}%
\pgfpathlineto{\pgfqpoint{4.893551in}{2.166366in}}%
\pgfpathlineto{\pgfqpoint{4.893194in}{2.575045in}}%
\pgfpathlineto{\pgfqpoint{4.893649in}{2.263252in}}%
\pgfpathlineto{\pgfqpoint{4.894117in}{2.537425in}}%
\pgfpathlineto{\pgfqpoint{4.894523in}{2.220901in}}%
\pgfpathlineto{\pgfqpoint{4.894905in}{2.311971in}}%
\pgfpathlineto{\pgfqpoint{4.895496in}{2.228565in}}%
\pgfpathlineto{\pgfqpoint{4.895311in}{2.516423in}}%
\pgfpathlineto{\pgfqpoint{4.895939in}{2.364041in}}%
\pgfpathlineto{\pgfqpoint{4.896062in}{2.528756in}}%
\pgfpathlineto{\pgfqpoint{4.896468in}{2.202244in}}%
\pgfpathlineto{\pgfqpoint{4.897083in}{2.470003in}}%
\pgfpathlineto{\pgfqpoint{4.897096in}{2.470623in}}%
\pgfpathlineto{\pgfqpoint{4.897182in}{2.432298in}}%
\pgfpathlineto{\pgfqpoint{4.898449in}{2.214277in}}%
\pgfpathlineto{\pgfqpoint{4.898142in}{2.533874in}}%
\pgfpathlineto{\pgfqpoint{4.898462in}{2.214793in}}%
\pgfpathlineto{\pgfqpoint{4.898991in}{2.556700in}}%
\pgfpathlineto{\pgfqpoint{4.899336in}{2.191328in}}%
\pgfpathlineto{\pgfqpoint{4.899705in}{2.402089in}}%
\pgfpathlineto{\pgfqpoint{4.900332in}{2.255743in}}%
\pgfpathlineto{\pgfqpoint{4.900086in}{2.486159in}}%
\pgfpathlineto{\pgfqpoint{4.900825in}{2.376975in}}%
\pgfpathlineto{\pgfqpoint{4.900936in}{2.563054in}}%
\pgfpathlineto{\pgfqpoint{4.901329in}{2.234314in}}%
\pgfpathlineto{\pgfqpoint{4.901969in}{2.451601in}}%
\pgfpathlineto{\pgfqpoint{4.903015in}{2.494143in}}%
\pgfpathlineto{\pgfqpoint{4.902240in}{2.229622in}}%
\pgfpathlineto{\pgfqpoint{4.903102in}{2.474335in}}%
\pgfpathlineto{\pgfqpoint{4.903175in}{2.380317in}}%
\pgfpathlineto{\pgfqpoint{4.904222in}{2.166672in}}%
\pgfpathlineto{\pgfqpoint{4.903865in}{2.549569in}}%
\pgfpathlineto{\pgfqpoint{4.904332in}{2.288697in}}%
\pgfpathlineto{\pgfqpoint{4.905071in}{2.471564in}}%
\pgfpathlineto{\pgfqpoint{4.905305in}{2.279348in}}%
\pgfpathlineto{\pgfqpoint{4.905440in}{2.308677in}}%
\pgfpathlineto{\pgfqpoint{4.905822in}{2.540126in}}%
\pgfpathlineto{\pgfqpoint{4.906154in}{2.207818in}}%
\pgfpathlineto{\pgfqpoint{4.906929in}{2.420364in}}%
\pgfpathlineto{\pgfqpoint{4.907902in}{2.471563in}}%
\pgfpathlineto{\pgfqpoint{4.908148in}{2.269960in}}%
\pgfpathlineto{\pgfqpoint{4.908751in}{2.542734in}}%
\pgfpathlineto{\pgfqpoint{4.909059in}{2.251137in}}%
\pgfpathlineto{\pgfqpoint{4.909108in}{2.201544in}}%
\pgfpathlineto{\pgfqpoint{4.909280in}{2.446569in}}%
\pgfpathlineto{\pgfqpoint{4.910105in}{2.317448in}}%
\pgfpathlineto{\pgfqpoint{4.910695in}{2.548896in}}%
\pgfpathlineto{\pgfqpoint{4.911052in}{2.218890in}}%
\pgfpathlineto{\pgfqpoint{4.911274in}{2.363982in}}%
\pgfpathlineto{\pgfqpoint{4.912025in}{2.262308in}}%
\pgfpathlineto{\pgfqpoint{4.911668in}{2.465918in}}%
\pgfpathlineto{\pgfqpoint{4.912455in}{2.307553in}}%
\pgfpathlineto{\pgfqpoint{4.913637in}{2.521511in}}%
\pgfpathlineto{\pgfqpoint{4.913034in}{2.292348in}}%
\pgfpathlineto{\pgfqpoint{4.913649in}{2.511272in}}%
\pgfpathlineto{\pgfqpoint{4.913994in}{2.191941in}}%
\pgfpathlineto{\pgfqpoint{4.914843in}{2.365281in}}%
\pgfpathlineto{\pgfqpoint{4.915594in}{2.548127in}}%
\pgfpathlineto{\pgfqpoint{4.915385in}{2.279845in}}%
\pgfpathlineto{\pgfqpoint{4.915852in}{2.342661in}}%
\pgfpathlineto{\pgfqpoint{4.915951in}{2.193840in}}%
\pgfpathlineto{\pgfqpoint{4.916554in}{2.472969in}}%
\pgfpathlineto{\pgfqpoint{4.916972in}{2.312695in}}%
\pgfpathlineto{\pgfqpoint{4.917699in}{2.463507in}}%
\pgfpathlineto{\pgfqpoint{4.917895in}{2.296176in}}%
\pgfpathlineto{\pgfqpoint{4.918105in}{2.388376in}}%
\pgfpathlineto{\pgfqpoint{4.918117in}{2.388822in}}%
\pgfpathlineto{\pgfqpoint{4.918154in}{2.355398in}}%
\pgfpathlineto{\pgfqpoint{4.918905in}{2.213510in}}%
\pgfpathlineto{\pgfqpoint{4.918535in}{2.508370in}}%
\pgfpathlineto{\pgfqpoint{4.919249in}{2.383127in}}%
\pgfpathlineto{\pgfqpoint{4.919520in}{2.459490in}}%
\pgfpathlineto{\pgfqpoint{4.920185in}{2.297997in}}%
\pgfpathlineto{\pgfqpoint{4.920259in}{2.332132in}}%
\pgfpathlineto{\pgfqpoint{4.920886in}{2.231653in}}%
\pgfpathlineto{\pgfqpoint{4.920529in}{2.515013in}}%
\pgfpathlineto{\pgfqpoint{4.921305in}{2.394944in}}%
\pgfpathlineto{\pgfqpoint{4.921452in}{2.477613in}}%
\pgfpathlineto{\pgfqpoint{4.921809in}{2.255673in}}%
\pgfpathlineto{\pgfqpoint{4.922252in}{2.331123in}}%
\pgfpathlineto{\pgfqpoint{4.923151in}{2.300685in}}%
\pgfpathlineto{\pgfqpoint{4.922622in}{2.458704in}}%
\pgfpathlineto{\pgfqpoint{4.923298in}{2.368071in}}%
\pgfpathlineto{\pgfqpoint{4.923409in}{2.483477in}}%
\pgfpathlineto{\pgfqpoint{4.923828in}{2.235148in}}%
\pgfpathlineto{\pgfqpoint{4.924418in}{2.411486in}}%
\pgfpathlineto{\pgfqpoint{4.924455in}{2.435372in}}%
\pgfpathlineto{\pgfqpoint{4.925194in}{2.310095in}}%
\pgfpathlineto{\pgfqpoint{4.925785in}{2.262153in}}%
\pgfpathlineto{\pgfqpoint{4.925452in}{2.482875in}}%
\pgfpathlineto{\pgfqpoint{4.926129in}{2.385282in}}%
\pgfpathlineto{\pgfqpoint{4.926351in}{2.486661in}}%
\pgfpathlineto{\pgfqpoint{4.926695in}{2.266377in}}%
\pgfpathlineto{\pgfqpoint{4.927151in}{2.330406in}}%
\pgfpathlineto{\pgfqpoint{4.927163in}{2.329402in}}%
\pgfpathlineto{\pgfqpoint{4.927274in}{2.407590in}}%
\pgfpathlineto{\pgfqpoint{4.927569in}{2.461130in}}%
\pgfpathlineto{\pgfqpoint{4.928074in}{2.300125in}}%
\pgfpathlineto{\pgfqpoint{4.928788in}{2.232464in}}%
\pgfpathlineto{\pgfqpoint{4.928480in}{2.495350in}}%
\pgfpathlineto{\pgfqpoint{4.929083in}{2.369145in}}%
\pgfpathlineto{\pgfqpoint{4.929625in}{2.275456in}}%
\pgfpathlineto{\pgfqpoint{4.929206in}{2.464570in}}%
\pgfpathlineto{\pgfqpoint{4.930215in}{2.356419in}}%
\pgfpathlineto{\pgfqpoint{4.931274in}{2.494226in}}%
\pgfpathlineto{\pgfqpoint{4.930757in}{2.251925in}}%
\pgfpathlineto{\pgfqpoint{4.931360in}{2.421040in}}%
\pgfpathlineto{\pgfqpoint{4.931409in}{2.429075in}}%
\pgfpathlineto{\pgfqpoint{4.931520in}{2.391363in}}%
\pgfpathlineto{\pgfqpoint{4.931618in}{2.217756in}}%
\pgfpathlineto{\pgfqpoint{4.932492in}{2.453510in}}%
\pgfpathlineto{\pgfqpoint{4.932665in}{2.323361in}}%
\pgfpathlineto{\pgfqpoint{4.933391in}{2.513662in}}%
\pgfpathlineto{\pgfqpoint{4.933588in}{2.213725in}}%
\pgfpathlineto{\pgfqpoint{4.933797in}{2.362522in}}%
\pgfpathlineto{\pgfqpoint{4.934560in}{2.249800in}}%
\pgfpathlineto{\pgfqpoint{4.934105in}{2.457093in}}%
\pgfpathlineto{\pgfqpoint{4.934745in}{2.418439in}}%
\pgfpathlineto{\pgfqpoint{4.935348in}{2.518460in}}%
\pgfpathlineto{\pgfqpoint{4.934991in}{2.249761in}}%
\pgfpathlineto{\pgfqpoint{4.935766in}{2.350367in}}%
\pgfpathlineto{\pgfqpoint{4.936542in}{2.222890in}}%
\pgfpathlineto{\pgfqpoint{4.936209in}{2.500855in}}%
\pgfpathlineto{\pgfqpoint{4.936837in}{2.402476in}}%
\pgfpathlineto{\pgfqpoint{4.937945in}{2.276646in}}%
\pgfpathlineto{\pgfqpoint{4.937157in}{2.441150in}}%
\pgfpathlineto{\pgfqpoint{4.938031in}{2.380980in}}%
\pgfpathlineto{\pgfqpoint{4.938314in}{2.505317in}}%
\pgfpathlineto{\pgfqpoint{4.938498in}{2.226658in}}%
\pgfpathlineto{\pgfqpoint{4.939175in}{2.411374in}}%
\pgfpathlineto{\pgfqpoint{4.939471in}{2.237348in}}%
\pgfpathlineto{\pgfqpoint{4.940222in}{2.482434in}}%
\pgfpathlineto{\pgfqpoint{4.940246in}{2.516525in}}%
\pgfpathlineto{\pgfqpoint{4.940480in}{2.304155in}}%
\pgfpathlineto{\pgfqpoint{4.941280in}{2.391897in}}%
\pgfpathlineto{\pgfqpoint{4.941440in}{2.213804in}}%
\pgfpathlineto{\pgfqpoint{4.941649in}{2.436444in}}%
\pgfpathlineto{\pgfqpoint{4.942425in}{2.320245in}}%
\pgfpathlineto{\pgfqpoint{4.943052in}{2.507170in}}%
\pgfpathlineto{\pgfqpoint{4.943360in}{2.272458in}}%
\pgfpathlineto{\pgfqpoint{4.943409in}{2.238857in}}%
\pgfpathlineto{\pgfqpoint{4.944025in}{2.468358in}}%
\pgfpathlineto{\pgfqpoint{4.944418in}{2.303952in}}%
\pgfpathlineto{\pgfqpoint{4.945157in}{2.505871in}}%
\pgfpathlineto{\pgfqpoint{4.944812in}{2.286941in}}%
\pgfpathlineto{\pgfqpoint{4.945538in}{2.335926in}}%
\pgfpathlineto{\pgfqpoint{4.946031in}{2.492571in}}%
\pgfpathlineto{\pgfqpoint{4.946314in}{2.228769in}}%
\pgfpathlineto{\pgfqpoint{4.946831in}{2.408852in}}%
\pgfpathlineto{\pgfqpoint{4.947717in}{2.283729in}}%
\pgfpathlineto{\pgfqpoint{4.946978in}{2.420014in}}%
\pgfpathlineto{\pgfqpoint{4.947914in}{2.419254in}}%
\pgfpathlineto{\pgfqpoint{4.947963in}{2.508842in}}%
\pgfpathlineto{\pgfqpoint{4.948320in}{2.253013in}}%
\pgfpathlineto{\pgfqpoint{4.949009in}{2.397412in}}%
\pgfpathlineto{\pgfqpoint{4.949206in}{2.256098in}}%
\pgfpathlineto{\pgfqpoint{4.950018in}{2.429488in}}%
\pgfpathlineto{\pgfqpoint{4.950055in}{2.497136in}}%
\pgfpathlineto{\pgfqpoint{4.950585in}{2.291508in}}%
\pgfpathlineto{\pgfqpoint{4.951101in}{2.376952in}}%
\pgfpathlineto{\pgfqpoint{4.951212in}{2.233815in}}%
\pgfpathlineto{\pgfqpoint{4.951520in}{2.438711in}}%
\pgfpathlineto{\pgfqpoint{4.952209in}{2.356725in}}%
\pgfpathlineto{\pgfqpoint{4.952898in}{2.490337in}}%
\pgfpathlineto{\pgfqpoint{4.952677in}{2.279135in}}%
\pgfpathlineto{\pgfqpoint{4.953145in}{2.298748in}}%
\pgfpathlineto{\pgfqpoint{4.954092in}{2.266805in}}%
\pgfpathlineto{\pgfqpoint{4.953735in}{2.461907in}}%
\pgfpathlineto{\pgfqpoint{4.954203in}{2.303520in}}%
\pgfpathlineto{\pgfqpoint{4.954954in}{2.489436in}}%
\pgfpathlineto{\pgfqpoint{4.955311in}{2.323958in}}%
\pgfpathlineto{\pgfqpoint{4.955471in}{2.276953in}}%
\pgfpathlineto{\pgfqpoint{4.955692in}{2.455180in}}%
\pgfpathlineto{\pgfqpoint{4.955803in}{2.495581in}}%
\pgfpathlineto{\pgfqpoint{4.956160in}{2.237181in}}%
\pgfpathlineto{\pgfqpoint{4.956603in}{2.372480in}}%
\pgfpathlineto{\pgfqpoint{4.957034in}{2.422068in}}%
\pgfpathlineto{\pgfqpoint{4.957378in}{2.332471in}}%
\pgfpathlineto{\pgfqpoint{4.957551in}{2.240111in}}%
\pgfpathlineto{\pgfqpoint{4.957895in}{2.514834in}}%
\pgfpathlineto{\pgfqpoint{4.958400in}{2.391265in}}%
\pgfpathlineto{\pgfqpoint{4.958597in}{2.442364in}}%
\pgfpathlineto{\pgfqpoint{4.959052in}{2.299464in}}%
\pgfpathlineto{\pgfqpoint{4.959458in}{2.315942in}}%
\pgfpathlineto{\pgfqpoint{4.960345in}{2.265174in}}%
\pgfpathlineto{\pgfqpoint{4.959815in}{2.506530in}}%
\pgfpathlineto{\pgfqpoint{4.960505in}{2.404656in}}%
\pgfpathlineto{\pgfqpoint{4.960689in}{2.482090in}}%
\pgfpathlineto{\pgfqpoint{4.961034in}{2.276468in}}%
\pgfpathlineto{\pgfqpoint{4.961563in}{2.318630in}}%
\pgfpathlineto{\pgfqpoint{4.962437in}{2.248280in}}%
\pgfpathlineto{\pgfqpoint{4.961908in}{2.472143in}}%
\pgfpathlineto{\pgfqpoint{4.962560in}{2.332244in}}%
\pgfpathlineto{\pgfqpoint{4.962781in}{2.487702in}}%
\pgfpathlineto{\pgfqpoint{4.963138in}{2.292169in}}%
\pgfpathlineto{\pgfqpoint{4.963680in}{2.361355in}}%
\pgfpathlineto{\pgfqpoint{4.964701in}{2.487285in}}%
\pgfpathlineto{\pgfqpoint{4.964517in}{2.267280in}}%
\pgfpathlineto{\pgfqpoint{4.964874in}{2.422758in}}%
\pgfpathlineto{\pgfqpoint{4.965883in}{2.289439in}}%
\pgfpathlineto{\pgfqpoint{4.965575in}{2.442219in}}%
\pgfpathlineto{\pgfqpoint{4.966031in}{2.354944in}}%
\pgfpathlineto{\pgfqpoint{4.966781in}{2.478294in}}%
\pgfpathlineto{\pgfqpoint{4.966449in}{2.285174in}}%
\pgfpathlineto{\pgfqpoint{4.967126in}{2.342038in}}%
\pgfpathlineto{\pgfqpoint{4.967311in}{2.264750in}}%
\pgfpathlineto{\pgfqpoint{4.967643in}{2.463016in}}%
\pgfpathlineto{\pgfqpoint{4.968148in}{2.402613in}}%
\pgfpathlineto{\pgfqpoint{4.969034in}{2.465109in}}%
\pgfpathlineto{\pgfqpoint{4.969218in}{2.322912in}}%
\pgfpathlineto{\pgfqpoint{4.969341in}{2.279942in}}%
\pgfpathlineto{\pgfqpoint{4.969563in}{2.476506in}}%
\pgfpathlineto{\pgfqpoint{4.970252in}{2.410939in}}%
\pgfpathlineto{\pgfqpoint{4.970966in}{2.476884in}}%
\pgfpathlineto{\pgfqpoint{4.970744in}{2.307410in}}%
\pgfpathlineto{\pgfqpoint{4.971274in}{2.329936in}}%
\pgfpathlineto{\pgfqpoint{4.971323in}{2.285293in}}%
\pgfpathlineto{\pgfqpoint{4.971828in}{2.444169in}}%
\pgfpathlineto{\pgfqpoint{4.972320in}{2.342074in}}%
\pgfpathlineto{\pgfqpoint{4.972529in}{2.445608in}}%
\pgfpathlineto{\pgfqpoint{4.972837in}{2.309239in}}%
\pgfpathlineto{\pgfqpoint{4.973415in}{2.328318in}}%
\pgfpathlineto{\pgfqpoint{4.973920in}{2.448927in}}%
\pgfpathlineto{\pgfqpoint{4.974252in}{2.288484in}}%
\pgfpathlineto{\pgfqpoint{4.974671in}{2.363607in}}%
\pgfpathlineto{\pgfqpoint{4.975064in}{2.320852in}}%
\pgfpathlineto{\pgfqpoint{4.975126in}{2.411225in}}%
\pgfpathlineto{\pgfqpoint{4.975532in}{2.378686in}}%
\pgfpathlineto{\pgfqpoint{4.975815in}{2.444194in}}%
\pgfpathlineto{\pgfqpoint{4.975655in}{2.291142in}}%
\pgfpathlineto{\pgfqpoint{4.976628in}{2.384280in}}%
\pgfpathlineto{\pgfqpoint{4.977046in}{2.308070in}}%
\pgfpathlineto{\pgfqpoint{4.977391in}{2.431558in}}%
\pgfpathlineto{\pgfqpoint{4.977760in}{2.360066in}}%
\pgfpathlineto{\pgfqpoint{4.977908in}{2.447738in}}%
\pgfpathlineto{\pgfqpoint{4.978437in}{2.314594in}}%
\pgfpathlineto{\pgfqpoint{4.978892in}{2.397985in}}%
\pgfpathlineto{\pgfqpoint{4.979237in}{2.314354in}}%
\pgfpathlineto{\pgfqpoint{4.979298in}{2.437335in}}%
\pgfpathlineto{\pgfqpoint{4.979963in}{2.381647in}}%
\pgfpathlineto{\pgfqpoint{4.979988in}{2.437165in}}%
\pgfpathlineto{\pgfqpoint{4.980628in}{2.286801in}}%
\pgfpathlineto{\pgfqpoint{4.981046in}{2.325421in}}%
\pgfpathlineto{\pgfqpoint{4.981378in}{2.430290in}}%
\pgfpathlineto{\pgfqpoint{4.982018in}{2.314655in}}%
\pgfpathlineto{\pgfqpoint{4.982277in}{2.410456in}}%
\pgfpathlineto{\pgfqpoint{4.982597in}{2.304327in}}%
\pgfpathlineto{\pgfqpoint{4.982769in}{2.457538in}}%
\pgfpathlineto{\pgfqpoint{4.983446in}{2.362776in}}%
\pgfpathlineto{\pgfqpoint{4.984160in}{2.439429in}}%
\pgfpathlineto{\pgfqpoint{4.984000in}{2.318326in}}%
\pgfpathlineto{\pgfqpoint{4.984541in}{2.329699in}}%
\pgfpathlineto{\pgfqpoint{4.985551in}{2.431954in}}%
\pgfpathlineto{\pgfqpoint{4.985391in}{2.306348in}}%
\pgfpathlineto{\pgfqpoint{4.985698in}{2.405827in}}%
\pgfpathlineto{\pgfqpoint{4.986031in}{2.336035in}}%
\pgfpathlineto{\pgfqpoint{4.986252in}{2.413198in}}%
\pgfpathlineto{\pgfqpoint{4.986658in}{2.402967in}}%
\pgfpathlineto{\pgfqpoint{4.987643in}{2.438005in}}%
\pgfpathlineto{\pgfqpoint{4.987434in}{2.314170in}}%
\pgfpathlineto{\pgfqpoint{4.987741in}{2.392425in}}%
\pgfpathlineto{\pgfqpoint{4.988824in}{2.331270in}}%
\pgfpathlineto{\pgfqpoint{4.988332in}{2.425537in}}%
\pgfpathlineto{\pgfqpoint{4.988886in}{2.367377in}}%
\pgfpathlineto{\pgfqpoint{4.989034in}{2.432036in}}%
\pgfpathlineto{\pgfqpoint{4.989391in}{2.315546in}}%
\pgfpathlineto{\pgfqpoint{4.989994in}{2.374746in}}%
\pgfpathlineto{\pgfqpoint{4.990424in}{2.416238in}}%
\pgfpathlineto{\pgfqpoint{4.990203in}{2.314238in}}%
\pgfpathlineto{\pgfqpoint{4.991040in}{2.370291in}}%
\pgfpathlineto{\pgfqpoint{4.992172in}{2.315084in}}%
\pgfpathlineto{\pgfqpoint{4.991114in}{2.436537in}}%
\pgfpathlineto{\pgfqpoint{4.992197in}{2.327770in}}%
\pgfpathlineto{\pgfqpoint{4.992504in}{2.454403in}}%
\pgfpathlineto{\pgfqpoint{4.992332in}{2.315871in}}%
\pgfpathlineto{\pgfqpoint{4.993341in}{2.374739in}}%
\pgfpathlineto{\pgfqpoint{4.993895in}{2.450304in}}%
\pgfpathlineto{\pgfqpoint{4.994252in}{2.335904in}}%
\pgfpathlineto{\pgfqpoint{4.994351in}{2.348042in}}%
\pgfpathlineto{\pgfqpoint{4.995077in}{2.315694in}}%
\pgfpathlineto{\pgfqpoint{4.994584in}{2.435199in}}%
\pgfpathlineto{\pgfqpoint{4.995421in}{2.399966in}}%
\pgfpathlineto{\pgfqpoint{4.995975in}{2.421651in}}%
\pgfpathlineto{\pgfqpoint{4.995766in}{2.323650in}}%
\pgfpathlineto{\pgfqpoint{4.996443in}{2.367952in}}%
\pgfpathlineto{\pgfqpoint{4.997157in}{2.308079in}}%
\pgfpathlineto{\pgfqpoint{4.996677in}{2.456314in}}%
\pgfpathlineto{\pgfqpoint{4.997514in}{2.401176in}}%
\pgfpathlineto{\pgfqpoint{4.997526in}{2.402106in}}%
\pgfpathlineto{\pgfqpoint{4.997723in}{2.338497in}}%
\pgfpathlineto{\pgfqpoint{4.997834in}{2.351805in}}%
\pgfpathlineto{\pgfqpoint{4.998006in}{2.326291in}}%
\pgfpathlineto{\pgfqpoint{4.998757in}{2.451357in}}%
\pgfpathlineto{\pgfqpoint{4.998904in}{2.379275in}}%
\pgfpathlineto{\pgfqpoint{4.998917in}{2.379402in}}%
\pgfpathlineto{\pgfqpoint{4.998941in}{2.368691in}}%
\pgfpathlineto{\pgfqpoint{4.999963in}{2.292749in}}%
\pgfpathlineto{\pgfqpoint{4.999446in}{2.459416in}}%
\pgfpathlineto{\pgfqpoint{5.000024in}{2.375825in}}%
\pgfpathlineto{\pgfqpoint{5.000837in}{2.439648in}}%
\pgfpathlineto{\pgfqpoint{5.001083in}{2.330019in}}%
\pgfpathlineto{\pgfqpoint{5.001107in}{2.344319in}}%
\pgfpathlineto{\pgfqpoint{5.002227in}{2.460770in}}%
\pgfpathlineto{\pgfqpoint{5.001895in}{2.290000in}}%
\pgfpathlineto{\pgfqpoint{5.002387in}{2.411225in}}%
\pgfpathlineto{\pgfqpoint{5.002744in}{2.329912in}}%
\pgfpathlineto{\pgfqpoint{5.003507in}{2.399909in}}%
\pgfpathlineto{\pgfqpoint{5.003618in}{2.463475in}}%
\pgfpathlineto{\pgfqpoint{5.003987in}{2.301955in}}%
\pgfpathlineto{\pgfqpoint{5.004591in}{2.383290in}}%
\pgfpathlineto{\pgfqpoint{5.005021in}{2.425469in}}%
\pgfpathlineto{\pgfqpoint{5.004849in}{2.316443in}}%
\pgfpathlineto{\pgfqpoint{5.005354in}{2.361703in}}%
\pgfpathlineto{\pgfqpoint{5.006080in}{2.317481in}}%
\pgfpathlineto{\pgfqpoint{5.005710in}{2.434118in}}%
\pgfpathlineto{\pgfqpoint{5.006387in}{2.425519in}}%
\pgfpathlineto{\pgfqpoint{5.006412in}{2.457470in}}%
\pgfpathlineto{\pgfqpoint{5.006769in}{2.305243in}}%
\pgfpathlineto{\pgfqpoint{5.007446in}{2.361426in}}%
\pgfpathlineto{\pgfqpoint{5.007470in}{2.338047in}}%
\pgfpathlineto{\pgfqpoint{5.008492in}{2.442770in}}%
\pgfpathlineto{\pgfqpoint{5.008504in}{2.445539in}}%
\pgfpathlineto{\pgfqpoint{5.008861in}{2.299443in}}%
\pgfpathlineto{\pgfqpoint{5.009390in}{2.395918in}}%
\pgfpathlineto{\pgfqpoint{5.009723in}{2.321808in}}%
\pgfpathlineto{\pgfqpoint{5.009895in}{2.403278in}}%
\pgfpathlineto{\pgfqpoint{5.010461in}{2.396915in}}%
\pgfpathlineto{\pgfqpoint{5.011286in}{2.430853in}}%
\pgfpathlineto{\pgfqpoint{5.010941in}{2.324669in}}%
\pgfpathlineto{\pgfqpoint{5.011544in}{2.374128in}}%
\pgfpathlineto{\pgfqpoint{5.011766in}{2.313064in}}%
\pgfpathlineto{\pgfqpoint{5.011987in}{2.420828in}}%
\pgfpathlineto{\pgfqpoint{5.012554in}{2.391992in}}%
\pgfpathlineto{\pgfqpoint{5.013366in}{2.440881in}}%
\pgfpathlineto{\pgfqpoint{5.013034in}{2.344287in}}%
\pgfpathlineto{\pgfqpoint{5.013600in}{2.349349in}}%
\pgfpathlineto{\pgfqpoint{5.013723in}{2.303652in}}%
\pgfpathlineto{\pgfqpoint{5.014067in}{2.425107in}}%
\pgfpathlineto{\pgfqpoint{5.014707in}{2.337939in}}%
\pgfpathlineto{\pgfqpoint{5.015446in}{2.431305in}}%
\pgfpathlineto{\pgfqpoint{5.015803in}{2.334519in}}%
\pgfpathlineto{\pgfqpoint{5.016664in}{2.310175in}}%
\pgfpathlineto{\pgfqpoint{5.016147in}{2.429646in}}%
\pgfpathlineto{\pgfqpoint{5.016726in}{2.366426in}}%
\pgfpathlineto{\pgfqpoint{5.016837in}{2.419676in}}%
\pgfpathlineto{\pgfqpoint{5.017206in}{2.353464in}}%
\pgfpathlineto{\pgfqpoint{5.017834in}{2.365044in}}%
\pgfpathlineto{\pgfqpoint{5.018597in}{2.305167in}}%
\pgfpathlineto{\pgfqpoint{5.018227in}{2.449072in}}%
\pgfpathlineto{\pgfqpoint{5.018892in}{2.384929in}}%
\pgfpathlineto{\pgfqpoint{5.018929in}{2.437310in}}%
\pgfpathlineto{\pgfqpoint{5.019446in}{2.325762in}}%
\pgfpathlineto{\pgfqpoint{5.019975in}{2.339031in}}%
\pgfpathlineto{\pgfqpoint{5.020677in}{2.321177in}}%
\pgfpathlineto{\pgfqpoint{5.020320in}{2.451428in}}%
\pgfpathlineto{\pgfqpoint{5.020972in}{2.374082in}}%
\pgfpathlineto{\pgfqpoint{5.021009in}{2.434362in}}%
\pgfpathlineto{\pgfqpoint{5.021538in}{2.311552in}}%
\pgfpathlineto{\pgfqpoint{5.022067in}{2.343690in}}%
\pgfpathlineto{\pgfqpoint{5.023101in}{2.450723in}}%
\pgfpathlineto{\pgfqpoint{5.022929in}{2.324436in}}%
\pgfpathlineto{\pgfqpoint{5.023310in}{2.387003in}}%
\pgfpathlineto{\pgfqpoint{5.023470in}{2.305857in}}%
\pgfpathlineto{\pgfqpoint{5.023803in}{2.428541in}}%
\pgfpathlineto{\pgfqpoint{5.024455in}{2.350083in}}%
\pgfpathlineto{\pgfqpoint{5.025181in}{2.449592in}}%
\pgfpathlineto{\pgfqpoint{5.025550in}{2.330338in}}%
\pgfpathlineto{\pgfqpoint{5.026572in}{2.430680in}}%
\pgfpathlineto{\pgfqpoint{5.026400in}{2.310865in}}%
\pgfpathlineto{\pgfqpoint{5.026757in}{2.392909in}}%
\pgfpathlineto{\pgfqpoint{5.027741in}{2.328398in}}%
\pgfpathlineto{\pgfqpoint{5.027261in}{2.414645in}}%
\pgfpathlineto{\pgfqpoint{5.027901in}{2.368315in}}%
\pgfpathlineto{\pgfqpoint{5.027963in}{2.443518in}}%
\pgfpathlineto{\pgfqpoint{5.028320in}{2.319112in}}%
\pgfpathlineto{\pgfqpoint{5.028997in}{2.372955in}}%
\pgfpathlineto{\pgfqpoint{5.029710in}{2.330936in}}%
\pgfpathlineto{\pgfqpoint{5.030055in}{2.434957in}}%
\pgfpathlineto{\pgfqpoint{5.030067in}{2.427382in}}%
\pgfpathlineto{\pgfqpoint{5.031212in}{2.317173in}}%
\pgfpathlineto{\pgfqpoint{5.031273in}{2.326949in}}%
\pgfpathlineto{\pgfqpoint{5.032430in}{2.408889in}}%
\pgfpathlineto{\pgfqpoint{5.032455in}{2.399862in}}%
\pgfpathlineto{\pgfqpoint{5.033021in}{2.335910in}}%
\pgfpathlineto{\pgfqpoint{5.033403in}{2.407407in}}%
\pgfpathlineto{\pgfqpoint{5.033637in}{2.347561in}}%
\pgfpathlineto{\pgfqpoint{5.034080in}{2.423691in}}%
\pgfpathlineto{\pgfqpoint{5.033784in}{2.335445in}}%
\pgfpathlineto{\pgfqpoint{5.034793in}{2.409038in}}%
\pgfpathlineto{\pgfqpoint{5.035089in}{2.336766in}}%
\pgfpathlineto{\pgfqpoint{5.034855in}{2.432193in}}%
\pgfpathlineto{\pgfqpoint{5.035987in}{2.359151in}}%
\pgfpathlineto{\pgfqpoint{5.036307in}{2.408729in}}%
\pgfpathlineto{\pgfqpoint{5.036123in}{2.343292in}}%
\pgfpathlineto{\pgfqpoint{5.037144in}{2.369456in}}%
\pgfpathlineto{\pgfqpoint{5.037243in}{2.342727in}}%
\pgfpathlineto{\pgfqpoint{5.037833in}{2.400407in}}%
\pgfpathlineto{\pgfqpoint{5.038215in}{2.371591in}}%
\pgfpathlineto{\pgfqpoint{5.038732in}{2.411411in}}%
\pgfpathlineto{\pgfqpoint{5.038461in}{2.334103in}}%
\pgfpathlineto{\pgfqpoint{5.039310in}{2.362152in}}%
\pgfpathlineto{\pgfqpoint{5.040012in}{2.319816in}}%
\pgfpathlineto{\pgfqpoint{5.039397in}{2.412714in}}%
\pgfpathlineto{\pgfqpoint{5.040381in}{2.384927in}}%
\pgfpathlineto{\pgfqpoint{5.041083in}{2.405782in}}%
\pgfpathlineto{\pgfqpoint{5.040541in}{2.347425in}}%
\pgfpathlineto{\pgfqpoint{5.041427in}{2.351752in}}%
\pgfpathlineto{\pgfqpoint{5.041735in}{2.423656in}}%
\pgfpathlineto{\pgfqpoint{5.042006in}{2.337429in}}%
\pgfpathlineto{\pgfqpoint{5.042646in}{2.390524in}}%
\pgfpathlineto{\pgfqpoint{5.043470in}{2.338590in}}%
\pgfpathlineto{\pgfqpoint{5.043273in}{2.423845in}}%
\pgfpathlineto{\pgfqpoint{5.043766in}{2.370440in}}%
\pgfpathlineto{\pgfqpoint{5.044073in}{2.397016in}}%
\pgfpathlineto{\pgfqpoint{5.044000in}{2.338524in}}%
\pgfpathlineto{\pgfqpoint{5.044873in}{2.375454in}}%
\pgfpathlineto{\pgfqpoint{5.045021in}{2.330878in}}%
\pgfpathlineto{\pgfqpoint{5.045637in}{2.403712in}}%
\pgfpathlineto{\pgfqpoint{5.046006in}{2.364125in}}%
\pgfpathlineto{\pgfqpoint{5.046387in}{2.401162in}}%
\pgfpathlineto{\pgfqpoint{5.046880in}{2.344429in}}%
\pgfpathlineto{\pgfqpoint{5.047101in}{2.365896in}}%
\pgfpathlineto{\pgfqpoint{5.047815in}{2.345983in}}%
\pgfpathlineto{\pgfqpoint{5.047175in}{2.415027in}}%
\pgfpathlineto{\pgfqpoint{5.048184in}{2.374034in}}%
\pgfpathlineto{\pgfqpoint{5.048750in}{2.412663in}}%
\pgfpathlineto{\pgfqpoint{5.048935in}{2.346825in}}%
\pgfpathlineto{\pgfqpoint{5.049292in}{2.372658in}}%
\pgfpathlineto{\pgfqpoint{5.049526in}{2.416767in}}%
\pgfpathlineto{\pgfqpoint{5.049735in}{2.346828in}}%
\pgfpathlineto{\pgfqpoint{5.050387in}{2.372067in}}%
\pgfpathlineto{\pgfqpoint{5.050781in}{2.344826in}}%
\pgfpathlineto{\pgfqpoint{5.051101in}{2.406528in}}%
\pgfpathlineto{\pgfqpoint{5.051409in}{2.383178in}}%
\pgfpathlineto{\pgfqpoint{5.052135in}{2.395262in}}%
\pgfpathlineto{\pgfqpoint{5.051753in}{2.341986in}}%
\pgfpathlineto{\pgfqpoint{5.052492in}{2.377132in}}%
\pgfpathlineto{\pgfqpoint{5.052541in}{2.352093in}}%
\pgfpathlineto{\pgfqpoint{5.053021in}{2.398333in}}%
\pgfpathlineto{\pgfqpoint{5.053587in}{2.378144in}}%
\pgfpathlineto{\pgfqpoint{5.054400in}{2.400577in}}%
\pgfpathlineto{\pgfqpoint{5.053920in}{2.350247in}}%
\pgfpathlineto{\pgfqpoint{5.054646in}{2.356556in}}%
\pgfpathlineto{\pgfqpoint{5.054707in}{2.345843in}}%
\pgfpathlineto{\pgfqpoint{5.054929in}{2.388153in}}%
\pgfpathlineto{\pgfqpoint{5.055040in}{2.413339in}}%
\pgfpathlineto{\pgfqpoint{5.055667in}{2.345704in}}%
\pgfpathlineto{\pgfqpoint{5.056012in}{2.358239in}}%
\pgfpathlineto{\pgfqpoint{5.056270in}{2.347154in}}%
\pgfpathlineto{\pgfqpoint{5.056381in}{2.394119in}}%
\pgfpathlineto{\pgfqpoint{5.057181in}{2.407038in}}%
\pgfpathlineto{\pgfqpoint{5.056566in}{2.347590in}}%
\pgfpathlineto{\pgfqpoint{5.057452in}{2.377809in}}%
\pgfpathlineto{\pgfqpoint{5.058141in}{2.339216in}}%
\pgfpathlineto{\pgfqpoint{5.057969in}{2.402102in}}%
\pgfpathlineto{\pgfqpoint{5.058547in}{2.383000in}}%
\pgfpathlineto{\pgfqpoint{5.059618in}{2.327588in}}%
\pgfpathlineto{\pgfqpoint{5.059212in}{2.407232in}}%
\pgfpathlineto{\pgfqpoint{5.059655in}{2.377665in}}%
\pgfpathlineto{\pgfqpoint{5.060000in}{2.408084in}}%
\pgfpathlineto{\pgfqpoint{5.059716in}{2.335111in}}%
\pgfpathlineto{\pgfqpoint{5.060800in}{2.404639in}}%
\pgfpathlineto{\pgfqpoint{5.061181in}{2.341537in}}%
\pgfpathlineto{\pgfqpoint{5.061341in}{2.418599in}}%
\pgfpathlineto{\pgfqpoint{5.061969in}{2.353532in}}%
\pgfpathlineto{\pgfqpoint{5.062904in}{2.412928in}}%
\pgfpathlineto{\pgfqpoint{5.062584in}{2.343775in}}%
\pgfpathlineto{\pgfqpoint{5.063187in}{2.368009in}}%
\pgfpathlineto{\pgfqpoint{5.063544in}{2.347059in}}%
\pgfpathlineto{\pgfqpoint{5.063938in}{2.398292in}}%
\pgfpathlineto{\pgfqpoint{5.064196in}{2.393143in}}%
\pgfpathlineto{\pgfqpoint{5.064209in}{2.393400in}}%
\pgfpathlineto{\pgfqpoint{5.064295in}{2.372865in}}%
\pgfpathlineto{\pgfqpoint{5.065120in}{2.338615in}}%
\pgfpathlineto{\pgfqpoint{5.064726in}{2.401748in}}%
\pgfpathlineto{\pgfqpoint{5.065390in}{2.382812in}}%
\pgfpathlineto{\pgfqpoint{5.066289in}{2.401661in}}%
\pgfpathlineto{\pgfqpoint{5.065907in}{2.342094in}}%
\pgfpathlineto{\pgfqpoint{5.066461in}{2.371270in}}%
\pgfpathlineto{\pgfqpoint{5.067470in}{2.340376in}}%
\pgfpathlineto{\pgfqpoint{5.067076in}{2.400756in}}%
\pgfpathlineto{\pgfqpoint{5.067581in}{2.364329in}}%
\pgfpathlineto{\pgfqpoint{5.068406in}{2.407589in}}%
\pgfpathlineto{\pgfqpoint{5.068258in}{2.342391in}}%
\pgfpathlineto{\pgfqpoint{5.068676in}{2.369847in}}%
\pgfpathlineto{\pgfqpoint{5.069046in}{2.346915in}}%
\pgfpathlineto{\pgfqpoint{5.069193in}{2.410857in}}%
\pgfpathlineto{\pgfqpoint{5.069759in}{2.392651in}}%
\pgfpathlineto{\pgfqpoint{5.069883in}{2.399066in}}%
\pgfpathlineto{\pgfqpoint{5.070264in}{2.349625in}}%
\pgfpathlineto{\pgfqpoint{5.070683in}{2.371237in}}%
\pgfpathlineto{\pgfqpoint{5.071396in}{2.339882in}}%
\pgfpathlineto{\pgfqpoint{5.071544in}{2.398964in}}%
\pgfpathlineto{\pgfqpoint{5.071766in}{2.389498in}}%
\pgfpathlineto{\pgfqpoint{5.072566in}{2.398669in}}%
\pgfpathlineto{\pgfqpoint{5.071827in}{2.347433in}}%
\pgfpathlineto{\pgfqpoint{5.072849in}{2.373713in}}%
\pgfpathlineto{\pgfqpoint{5.073292in}{2.355498in}}%
\pgfpathlineto{\pgfqpoint{5.073095in}{2.397736in}}%
\pgfpathlineto{\pgfqpoint{5.073759in}{2.379080in}}%
\pgfpathlineto{\pgfqpoint{5.074646in}{2.406983in}}%
\pgfpathlineto{\pgfqpoint{5.074178in}{2.347214in}}%
\pgfpathlineto{\pgfqpoint{5.074830in}{2.356287in}}%
\pgfpathlineto{\pgfqpoint{5.074855in}{2.346900in}}%
\pgfpathlineto{\pgfqpoint{5.075421in}{2.409799in}}%
\pgfpathlineto{\pgfqpoint{5.075913in}{2.372494in}}%
\pgfpathlineto{\pgfqpoint{5.076196in}{2.404459in}}%
\pgfpathlineto{\pgfqpoint{5.076393in}{2.347713in}}%
\pgfpathlineto{\pgfqpoint{5.077033in}{2.384078in}}%
\pgfpathlineto{\pgfqpoint{5.077181in}{2.345948in}}%
\pgfpathlineto{\pgfqpoint{5.077759in}{2.400370in}}%
\pgfpathlineto{\pgfqpoint{5.078166in}{2.370071in}}%
\pgfpathlineto{\pgfqpoint{5.078535in}{2.421440in}}%
\pgfpathlineto{\pgfqpoint{5.078732in}{2.341662in}}%
\pgfpathlineto{\pgfqpoint{5.079249in}{2.361055in}}%
\pgfpathlineto{\pgfqpoint{5.079298in}{2.403912in}}%
\pgfpathlineto{\pgfqpoint{5.079323in}{2.414017in}}%
\pgfpathlineto{\pgfqpoint{5.080283in}{2.339651in}}%
\pgfpathlineto{\pgfqpoint{5.080356in}{2.377989in}}%
\pgfpathlineto{\pgfqpoint{5.081058in}{2.335198in}}%
\pgfpathlineto{\pgfqpoint{5.080886in}{2.410464in}}%
\pgfpathlineto{\pgfqpoint{5.081476in}{2.369293in}}%
\pgfpathlineto{\pgfqpoint{5.082621in}{2.321512in}}%
\pgfpathlineto{\pgfqpoint{5.082424in}{2.407949in}}%
\pgfpathlineto{\pgfqpoint{5.082633in}{2.326538in}}%
\pgfpathlineto{\pgfqpoint{5.083199in}{2.403023in}}%
\pgfpathlineto{\pgfqpoint{5.083778in}{2.387771in}}%
\pgfpathlineto{\pgfqpoint{5.084184in}{2.329043in}}%
\pgfpathlineto{\pgfqpoint{5.084775in}{2.407619in}}%
\pgfpathlineto{\pgfqpoint{5.084996in}{2.358015in}}%
\pgfpathlineto{\pgfqpoint{5.085550in}{2.402146in}}%
\pgfpathlineto{\pgfqpoint{5.085747in}{2.342653in}}%
\pgfpathlineto{\pgfqpoint{5.086116in}{2.374353in}}%
\pgfpathlineto{\pgfqpoint{5.087113in}{2.409010in}}%
\pgfpathlineto{\pgfqpoint{5.086535in}{2.343095in}}%
\pgfpathlineto{\pgfqpoint{5.087261in}{2.382831in}}%
\pgfpathlineto{\pgfqpoint{5.088098in}{2.336136in}}%
\pgfpathlineto{\pgfqpoint{5.087901in}{2.403564in}}%
\pgfpathlineto{\pgfqpoint{5.088381in}{2.363011in}}%
\pgfpathlineto{\pgfqpoint{5.088689in}{2.401325in}}%
\pgfpathlineto{\pgfqpoint{5.088886in}{2.335828in}}%
\pgfpathlineto{\pgfqpoint{5.089526in}{2.376603in}}%
\pgfpathlineto{\pgfqpoint{5.089673in}{2.348591in}}%
\pgfpathlineto{\pgfqpoint{5.090252in}{2.401929in}}%
\pgfpathlineto{\pgfqpoint{5.090584in}{2.380233in}}%
\pgfpathlineto{\pgfqpoint{5.091039in}{2.407681in}}%
\pgfpathlineto{\pgfqpoint{5.091482in}{2.353779in}}%
\pgfpathlineto{\pgfqpoint{5.092282in}{2.346665in}}%
\pgfpathlineto{\pgfqpoint{5.091827in}{2.404343in}}%
\pgfpathlineto{\pgfqpoint{5.092504in}{2.376011in}}%
\pgfpathlineto{\pgfqpoint{5.093193in}{2.396187in}}%
\pgfpathlineto{\pgfqpoint{5.093070in}{2.343141in}}%
\pgfpathlineto{\pgfqpoint{5.093587in}{2.352383in}}%
\pgfpathlineto{\pgfqpoint{5.093599in}{2.346187in}}%
\pgfpathlineto{\pgfqpoint{5.094104in}{2.398287in}}%
\pgfpathlineto{\pgfqpoint{5.094670in}{2.361946in}}%
\pgfpathlineto{\pgfqpoint{5.094892in}{2.395524in}}%
\pgfpathlineto{\pgfqpoint{5.095433in}{2.347497in}}%
\pgfpathlineto{\pgfqpoint{5.095815in}{2.384866in}}%
\pgfpathlineto{\pgfqpoint{5.096221in}{2.346688in}}%
\pgfpathlineto{\pgfqpoint{5.096590in}{2.401395in}}%
\pgfpathlineto{\pgfqpoint{5.096947in}{2.366156in}}%
\pgfpathlineto{\pgfqpoint{5.098042in}{2.406537in}}%
\pgfpathlineto{\pgfqpoint{5.097796in}{2.341561in}}%
\pgfpathlineto{\pgfqpoint{5.098067in}{2.375538in}}%
\pgfpathlineto{\pgfqpoint{5.098584in}{2.343808in}}%
\pgfpathlineto{\pgfqpoint{5.098141in}{2.409279in}}%
\pgfpathlineto{\pgfqpoint{5.099162in}{2.385537in}}%
\pgfpathlineto{\pgfqpoint{5.099606in}{2.399325in}}%
\pgfpathlineto{\pgfqpoint{5.099901in}{2.345980in}}%
\pgfpathlineto{\pgfqpoint{5.100122in}{2.365691in}}%
\pgfpathlineto{\pgfqpoint{5.100689in}{2.344518in}}%
\pgfpathlineto{\pgfqpoint{5.100492in}{2.405774in}}%
\pgfpathlineto{\pgfqpoint{5.101156in}{2.374833in}}%
\pgfpathlineto{\pgfqpoint{5.102055in}{2.402780in}}%
\pgfpathlineto{\pgfqpoint{5.101230in}{2.347915in}}%
\pgfpathlineto{\pgfqpoint{5.102252in}{2.365487in}}%
\pgfpathlineto{\pgfqpoint{5.102842in}{2.406247in}}%
\pgfpathlineto{\pgfqpoint{5.102707in}{2.345347in}}%
\pgfpathlineto{\pgfqpoint{5.103298in}{2.359889in}}%
\pgfpathlineto{\pgfqpoint{5.103482in}{2.344439in}}%
\pgfpathlineto{\pgfqpoint{5.103630in}{2.407322in}}%
\pgfpathlineto{\pgfqpoint{5.104196in}{2.377086in}}%
\pgfpathlineto{\pgfqpoint{5.104418in}{2.400872in}}%
\pgfpathlineto{\pgfqpoint{5.104602in}{2.352703in}}%
\pgfpathlineto{\pgfqpoint{5.105255in}{2.371423in}}%
\pgfpathlineto{\pgfqpoint{5.106178in}{2.352445in}}%
\pgfpathlineto{\pgfqpoint{5.106227in}{2.395598in}}%
\pgfpathlineto{\pgfqpoint{5.106326in}{2.384182in}}%
\pgfpathlineto{\pgfqpoint{5.106633in}{2.345421in}}%
\pgfpathlineto{\pgfqpoint{5.107015in}{2.400452in}}%
\pgfpathlineto{\pgfqpoint{5.107433in}{2.370509in}}%
\pgfpathlineto{\pgfqpoint{5.107802in}{2.403798in}}%
\pgfpathlineto{\pgfqpoint{5.108184in}{2.345754in}}%
\pgfpathlineto{\pgfqpoint{5.108553in}{2.384673in}}%
\pgfpathlineto{\pgfqpoint{5.109107in}{2.406510in}}%
\pgfpathlineto{\pgfqpoint{5.108972in}{2.343220in}}%
\pgfpathlineto{\pgfqpoint{5.109636in}{2.370976in}}%
\pgfpathlineto{\pgfqpoint{5.109747in}{2.348969in}}%
\pgfpathlineto{\pgfqpoint{5.109895in}{2.406479in}}%
\pgfpathlineto{\pgfqpoint{5.110559in}{2.379091in}}%
\pgfpathlineto{\pgfqpoint{5.110670in}{2.402842in}}%
\pgfpathlineto{\pgfqpoint{5.110867in}{2.348737in}}%
\pgfpathlineto{\pgfqpoint{5.111642in}{2.356264in}}%
\pgfpathlineto{\pgfqpoint{5.112233in}{2.404332in}}%
\pgfpathlineto{\pgfqpoint{5.112085in}{2.350855in}}%
\pgfpathlineto{\pgfqpoint{5.112787in}{2.374416in}}%
\pgfpathlineto{\pgfqpoint{5.113021in}{2.406001in}}%
\pgfpathlineto{\pgfqpoint{5.113205in}{2.350903in}}%
\pgfpathlineto{\pgfqpoint{5.113858in}{2.376068in}}%
\pgfpathlineto{\pgfqpoint{5.113969in}{2.345330in}}%
\pgfpathlineto{\pgfqpoint{5.114559in}{2.400340in}}%
\pgfpathlineto{\pgfqpoint{5.114965in}{2.376961in}}%
\pgfpathlineto{\pgfqpoint{5.115335in}{2.394574in}}%
\pgfpathlineto{\pgfqpoint{5.115187in}{2.350136in}}%
\pgfpathlineto{\pgfqpoint{5.115495in}{2.364591in}}%
\pgfpathlineto{\pgfqpoint{5.115962in}{2.347334in}}%
\pgfpathlineto{\pgfqpoint{5.116110in}{2.399511in}}%
\pgfpathlineto{\pgfqpoint{5.116602in}{2.364477in}}%
\pgfpathlineto{\pgfqpoint{5.116885in}{2.398485in}}%
\pgfpathlineto{\pgfqpoint{5.117082in}{2.345434in}}%
\pgfpathlineto{\pgfqpoint{5.117747in}{2.378000in}}%
\pgfpathlineto{\pgfqpoint{5.117858in}{2.346914in}}%
\pgfpathlineto{\pgfqpoint{5.118239in}{2.398891in}}%
\pgfpathlineto{\pgfqpoint{5.118867in}{2.375408in}}%
\pgfpathlineto{\pgfqpoint{5.119409in}{2.356883in}}%
\pgfpathlineto{\pgfqpoint{5.119027in}{2.403728in}}%
\pgfpathlineto{\pgfqpoint{5.119532in}{2.377196in}}%
\pgfpathlineto{\pgfqpoint{5.120381in}{2.403424in}}%
\pgfpathlineto{\pgfqpoint{5.120184in}{2.347359in}}%
\pgfpathlineto{\pgfqpoint{5.120639in}{2.377104in}}%
\pgfpathlineto{\pgfqpoint{5.121095in}{2.421907in}}%
\pgfpathlineto{\pgfqpoint{5.120898in}{2.349768in}}%
\pgfpathlineto{\pgfqpoint{5.121661in}{2.362878in}}%
\pgfpathlineto{\pgfqpoint{5.122522in}{2.329402in}}%
\pgfpathlineto{\pgfqpoint{5.121919in}{2.422579in}}%
\pgfpathlineto{\pgfqpoint{5.122719in}{2.403622in}}%
\pgfpathlineto{\pgfqpoint{5.122732in}{2.408951in}}%
\pgfpathlineto{\pgfqpoint{5.123335in}{2.339231in}}%
\pgfpathlineto{\pgfqpoint{5.123765in}{2.382436in}}%
\pgfpathlineto{\pgfqpoint{5.124849in}{2.337348in}}%
\pgfpathlineto{\pgfqpoint{5.124381in}{2.403607in}}%
\pgfpathlineto{\pgfqpoint{5.124885in}{2.362262in}}%
\pgfpathlineto{\pgfqpoint{5.125193in}{2.406590in}}%
\pgfpathlineto{\pgfqpoint{5.125661in}{2.333803in}}%
\pgfpathlineto{\pgfqpoint{5.126042in}{2.388553in}}%
\pgfpathlineto{\pgfqpoint{5.126473in}{2.337272in}}%
\pgfpathlineto{\pgfqpoint{5.126842in}{2.411917in}}%
\pgfpathlineto{\pgfqpoint{5.127199in}{2.349933in}}%
\pgfpathlineto{\pgfqpoint{5.127655in}{2.415348in}}%
\pgfpathlineto{\pgfqpoint{5.127962in}{2.346118in}}%
\pgfpathlineto{\pgfqpoint{5.128369in}{2.382403in}}%
\pgfpathlineto{\pgfqpoint{5.128787in}{2.344420in}}%
\pgfpathlineto{\pgfqpoint{5.128467in}{2.410970in}}%
\pgfpathlineto{\pgfqpoint{5.129267in}{2.393904in}}%
\pgfpathlineto{\pgfqpoint{5.129710in}{2.402487in}}%
\pgfpathlineto{\pgfqpoint{5.129587in}{2.331218in}}%
\pgfpathlineto{\pgfqpoint{5.130313in}{2.379932in}}%
\pgfpathlineto{\pgfqpoint{5.130387in}{2.331318in}}%
\pgfpathlineto{\pgfqpoint{5.130941in}{2.409123in}}%
\pgfpathlineto{\pgfqpoint{5.131433in}{2.373906in}}%
\pgfpathlineto{\pgfqpoint{5.131753in}{2.412440in}}%
\pgfpathlineto{\pgfqpoint{5.132024in}{2.339232in}}%
\pgfpathlineto{\pgfqpoint{5.132565in}{2.399733in}}%
\pgfpathlineto{\pgfqpoint{5.132578in}{2.399955in}}%
\pgfpathlineto{\pgfqpoint{5.132615in}{2.380963in}}%
\pgfpathlineto{\pgfqpoint{5.132824in}{2.349578in}}%
\pgfpathlineto{\pgfqpoint{5.133402in}{2.404730in}}%
\pgfpathlineto{\pgfqpoint{5.133772in}{2.366338in}}%
\pgfpathlineto{\pgfqpoint{5.134215in}{2.403528in}}%
\pgfpathlineto{\pgfqpoint{5.134818in}{2.340040in}}%
\pgfpathlineto{\pgfqpoint{5.134879in}{2.374138in}}%
\pgfpathlineto{\pgfqpoint{5.135310in}{2.340464in}}%
\pgfpathlineto{\pgfqpoint{5.135458in}{2.403036in}}%
\pgfpathlineto{\pgfqpoint{5.135950in}{2.380934in}}%
\pgfpathlineto{\pgfqpoint{5.135987in}{2.410559in}}%
\pgfpathlineto{\pgfqpoint{5.136947in}{2.338575in}}%
\pgfpathlineto{\pgfqpoint{5.137045in}{2.361096in}}%
\pgfpathlineto{\pgfqpoint{5.137513in}{2.412605in}}%
\pgfpathlineto{\pgfqpoint{5.137772in}{2.346205in}}%
\pgfpathlineto{\pgfqpoint{5.138141in}{2.354664in}}%
\pgfpathlineto{\pgfqpoint{5.138325in}{2.424236in}}%
\pgfpathlineto{\pgfqpoint{5.138596in}{2.339298in}}%
\pgfpathlineto{\pgfqpoint{5.139359in}{2.379106in}}%
\pgfpathlineto{\pgfqpoint{5.139421in}{2.334957in}}%
\pgfpathlineto{\pgfqpoint{5.139975in}{2.415951in}}%
\pgfpathlineto{\pgfqpoint{5.140467in}{2.372715in}}%
\pgfpathlineto{\pgfqpoint{5.140787in}{2.414812in}}%
\pgfpathlineto{\pgfqpoint{5.141390in}{2.331102in}}%
\pgfpathlineto{\pgfqpoint{5.141624in}{2.402692in}}%
\pgfpathlineto{\pgfqpoint{5.141882in}{2.330411in}}%
\pgfpathlineto{\pgfqpoint{5.142436in}{2.417957in}}%
\pgfpathlineto{\pgfqpoint{5.142781in}{2.360276in}}%
\pgfpathlineto{\pgfqpoint{5.142793in}{2.360323in}}%
\pgfpathlineto{\pgfqpoint{5.143248in}{2.411262in}}%
\pgfpathlineto{\pgfqpoint{5.143519in}{2.342064in}}%
\pgfpathlineto{\pgfqpoint{5.143925in}{2.370890in}}%
\pgfpathlineto{\pgfqpoint{5.144073in}{2.413433in}}%
\pgfpathlineto{\pgfqpoint{5.144356in}{2.334364in}}%
\pgfpathlineto{\pgfqpoint{5.145058in}{2.372661in}}%
\pgfpathlineto{\pgfqpoint{5.145095in}{2.385536in}}%
\pgfpathlineto{\pgfqpoint{5.145132in}{2.352476in}}%
\pgfpathlineto{\pgfqpoint{5.145168in}{2.329472in}}%
\pgfpathlineto{\pgfqpoint{5.145722in}{2.414076in}}%
\pgfpathlineto{\pgfqpoint{5.146215in}{2.374984in}}%
\pgfpathlineto{\pgfqpoint{5.146535in}{2.413994in}}%
\pgfpathlineto{\pgfqpoint{5.146818in}{2.330659in}}%
\pgfpathlineto{\pgfqpoint{5.147359in}{2.409489in}}%
\pgfpathlineto{\pgfqpoint{5.147630in}{2.321900in}}%
\pgfpathlineto{\pgfqpoint{5.147778in}{2.411219in}}%
\pgfpathlineto{\pgfqpoint{5.148565in}{2.382826in}}%
\pgfpathlineto{\pgfqpoint{5.148602in}{2.415458in}}%
\pgfpathlineto{\pgfqpoint{5.149279in}{2.332393in}}%
\pgfpathlineto{\pgfqpoint{5.149661in}{2.367058in}}%
\pgfpathlineto{\pgfqpoint{5.150092in}{2.329979in}}%
\pgfpathlineto{\pgfqpoint{5.149833in}{2.416872in}}%
\pgfpathlineto{\pgfqpoint{5.150227in}{2.400136in}}%
\pgfpathlineto{\pgfqpoint{5.150252in}{2.411667in}}%
\pgfpathlineto{\pgfqpoint{5.150928in}{2.334921in}}%
\pgfpathlineto{\pgfqpoint{5.151285in}{2.362525in}}%
\pgfpathlineto{\pgfqpoint{5.151753in}{2.333072in}}%
\pgfpathlineto{\pgfqpoint{5.151888in}{2.411469in}}%
\pgfpathlineto{\pgfqpoint{5.152258in}{2.379270in}}%
\pgfpathlineto{\pgfqpoint{5.152713in}{2.418960in}}%
\pgfpathlineto{\pgfqpoint{5.152565in}{2.332738in}}%
\pgfpathlineto{\pgfqpoint{5.153341in}{2.374842in}}%
\pgfpathlineto{\pgfqpoint{5.153390in}{2.331416in}}%
\pgfpathlineto{\pgfqpoint{5.153538in}{2.411626in}}%
\pgfpathlineto{\pgfqpoint{5.154448in}{2.376228in}}%
\pgfpathlineto{\pgfqpoint{5.154768in}{2.411121in}}%
\pgfpathlineto{\pgfqpoint{5.155027in}{2.333708in}}%
\pgfpathlineto{\pgfqpoint{5.155359in}{2.343736in}}%
\pgfpathlineto{\pgfqpoint{5.155864in}{2.336560in}}%
\pgfpathlineto{\pgfqpoint{5.155593in}{2.420855in}}%
\pgfpathlineto{\pgfqpoint{5.156344in}{2.397194in}}%
\pgfpathlineto{\pgfqpoint{5.156405in}{2.410859in}}%
\pgfpathlineto{\pgfqpoint{5.156676in}{2.326299in}}%
\pgfpathlineto{\pgfqpoint{5.157291in}{2.365250in}}%
\pgfpathlineto{\pgfqpoint{5.157365in}{2.389501in}}%
\pgfpathlineto{\pgfqpoint{5.157464in}{2.345777in}}%
\pgfpathlineto{\pgfqpoint{5.157488in}{2.326655in}}%
\pgfpathlineto{\pgfqpoint{5.157648in}{2.417186in}}%
\pgfpathlineto{\pgfqpoint{5.158547in}{2.369715in}}%
\pgfpathlineto{\pgfqpoint{5.159138in}{2.336460in}}%
\pgfpathlineto{\pgfqpoint{5.159704in}{2.407795in}}%
\pgfpathlineto{\pgfqpoint{5.159950in}{2.333808in}}%
\pgfpathlineto{\pgfqpoint{5.160516in}{2.416370in}}%
\pgfpathlineto{\pgfqpoint{5.160885in}{2.366235in}}%
\pgfpathlineto{\pgfqpoint{5.161328in}{2.406711in}}%
\pgfpathlineto{\pgfqpoint{5.161599in}{2.338337in}}%
\pgfpathlineto{\pgfqpoint{5.161981in}{2.359448in}}%
\pgfpathlineto{\pgfqpoint{5.162571in}{2.411690in}}%
\pgfpathlineto{\pgfqpoint{5.162424in}{2.338589in}}%
\pgfpathlineto{\pgfqpoint{5.163187in}{2.385761in}}%
\pgfpathlineto{\pgfqpoint{5.163248in}{2.331958in}}%
\pgfpathlineto{\pgfqpoint{5.163396in}{2.420677in}}%
\pgfpathlineto{\pgfqpoint{5.164295in}{2.380448in}}%
\pgfpathlineto{\pgfqpoint{5.165033in}{2.415027in}}%
\pgfpathlineto{\pgfqpoint{5.164898in}{2.334759in}}%
\pgfpathlineto{\pgfqpoint{5.165451in}{2.410134in}}%
\pgfpathlineto{\pgfqpoint{5.165710in}{2.338914in}}%
\pgfpathlineto{\pgfqpoint{5.165858in}{2.413657in}}%
\pgfpathlineto{\pgfqpoint{5.166608in}{2.366377in}}%
\pgfpathlineto{\pgfqpoint{5.166621in}{2.365956in}}%
\pgfpathlineto{\pgfqpoint{5.166658in}{2.400371in}}%
\pgfpathlineto{\pgfqpoint{5.166682in}{2.417180in}}%
\pgfpathlineto{\pgfqpoint{5.167691in}{2.335282in}}%
\pgfpathlineto{\pgfqpoint{5.167704in}{2.337147in}}%
\pgfpathlineto{\pgfqpoint{5.168331in}{2.408590in}}%
\pgfpathlineto{\pgfqpoint{5.168947in}{2.383573in}}%
\pgfpathlineto{\pgfqpoint{5.169833in}{2.340591in}}%
\pgfpathlineto{\pgfqpoint{5.169144in}{2.410971in}}%
\pgfpathlineto{\pgfqpoint{5.170055in}{2.371227in}}%
\pgfpathlineto{\pgfqpoint{5.171199in}{2.408105in}}%
\pgfpathlineto{\pgfqpoint{5.170645in}{2.337861in}}%
\pgfpathlineto{\pgfqpoint{5.171211in}{2.408059in}}%
\pgfpathlineto{\pgfqpoint{5.172294in}{2.336524in}}%
\pgfpathlineto{\pgfqpoint{5.172024in}{2.414856in}}%
\pgfpathlineto{\pgfqpoint{5.172393in}{2.358476in}}%
\pgfpathlineto{\pgfqpoint{5.173254in}{2.408380in}}%
\pgfpathlineto{\pgfqpoint{5.173119in}{2.333099in}}%
\pgfpathlineto{\pgfqpoint{5.173501in}{2.355334in}}%
\pgfpathlineto{\pgfqpoint{5.174079in}{2.413082in}}%
\pgfpathlineto{\pgfqpoint{5.173931in}{2.333838in}}%
\pgfpathlineto{\pgfqpoint{5.174658in}{2.377720in}}%
\pgfpathlineto{\pgfqpoint{5.174904in}{2.411225in}}%
\pgfpathlineto{\pgfqpoint{5.174756in}{2.334686in}}%
\pgfpathlineto{\pgfqpoint{5.175531in}{2.371402in}}%
\pgfpathlineto{\pgfqpoint{5.175581in}{2.335744in}}%
\pgfpathlineto{\pgfqpoint{5.176553in}{2.408652in}}%
\pgfpathlineto{\pgfqpoint{5.176627in}{2.376166in}}%
\pgfpathlineto{\pgfqpoint{5.177365in}{2.405440in}}%
\pgfpathlineto{\pgfqpoint{5.177230in}{2.342217in}}%
\pgfpathlineto{\pgfqpoint{5.177759in}{2.389818in}}%
\pgfpathlineto{\pgfqpoint{5.178202in}{2.407781in}}%
\pgfpathlineto{\pgfqpoint{5.178054in}{2.336301in}}%
\pgfpathlineto{\pgfqpoint{5.178818in}{2.375027in}}%
\pgfpathlineto{\pgfqpoint{5.179691in}{2.334066in}}%
\pgfpathlineto{\pgfqpoint{5.179839in}{2.419350in}}%
\pgfpathlineto{\pgfqpoint{5.179913in}{2.388906in}}%
\pgfpathlineto{\pgfqpoint{5.180664in}{2.419253in}}%
\pgfpathlineto{\pgfqpoint{5.180516in}{2.333762in}}%
\pgfpathlineto{\pgfqpoint{5.180996in}{2.382253in}}%
\pgfpathlineto{\pgfqpoint{5.181328in}{2.341653in}}%
\pgfpathlineto{\pgfqpoint{5.181476in}{2.424675in}}%
\pgfpathlineto{\pgfqpoint{5.182202in}{2.360037in}}%
\pgfpathlineto{\pgfqpoint{5.182301in}{2.424081in}}%
\pgfpathlineto{\pgfqpoint{5.182978in}{2.345732in}}%
\pgfpathlineto{\pgfqpoint{5.183285in}{2.357766in}}%
\pgfpathlineto{\pgfqpoint{5.184159in}{2.338076in}}%
\pgfpathlineto{\pgfqpoint{5.183950in}{2.420857in}}%
\pgfpathlineto{\pgfqpoint{5.184319in}{2.377915in}}%
\pgfpathlineto{\pgfqpoint{5.184774in}{2.424175in}}%
\pgfpathlineto{\pgfqpoint{5.184971in}{2.335598in}}%
\pgfpathlineto{\pgfqpoint{5.185402in}{2.380585in}}%
\pgfpathlineto{\pgfqpoint{5.185796in}{2.342019in}}%
\pgfpathlineto{\pgfqpoint{5.186411in}{2.417550in}}%
\pgfpathlineto{\pgfqpoint{5.186510in}{2.373323in}}%
\pgfpathlineto{\pgfqpoint{5.187101in}{2.344454in}}%
\pgfpathlineto{\pgfqpoint{5.186953in}{2.403953in}}%
\pgfpathlineto{\pgfqpoint{5.187211in}{2.397368in}}%
\pgfpathlineto{\pgfqpoint{5.187236in}{2.415635in}}%
\pgfpathlineto{\pgfqpoint{5.188258in}{2.338846in}}%
\pgfpathlineto{\pgfqpoint{5.188270in}{2.338178in}}%
\pgfpathlineto{\pgfqpoint{5.188454in}{2.385851in}}%
\pgfpathlineto{\pgfqpoint{5.188885in}{2.409512in}}%
\pgfpathlineto{\pgfqpoint{5.189082in}{2.332944in}}%
\pgfpathlineto{\pgfqpoint{5.189525in}{2.371380in}}%
\pgfpathlineto{\pgfqpoint{5.189907in}{2.336619in}}%
\pgfpathlineto{\pgfqpoint{5.190534in}{2.421803in}}%
\pgfpathlineto{\pgfqpoint{5.190633in}{2.372943in}}%
\pgfpathlineto{\pgfqpoint{5.190645in}{2.372953in}}%
\pgfpathlineto{\pgfqpoint{5.191211in}{2.338725in}}%
\pgfpathlineto{\pgfqpoint{5.191347in}{2.415148in}}%
\pgfpathlineto{\pgfqpoint{5.191679in}{2.386785in}}%
\pgfpathlineto{\pgfqpoint{5.192184in}{2.409901in}}%
\pgfpathlineto{\pgfqpoint{5.192036in}{2.334445in}}%
\pgfpathlineto{\pgfqpoint{5.192750in}{2.377428in}}%
\pgfpathlineto{\pgfqpoint{5.193205in}{2.337958in}}%
\pgfpathlineto{\pgfqpoint{5.193008in}{2.411001in}}%
\pgfpathlineto{\pgfqpoint{5.193796in}{2.384737in}}%
\pgfpathlineto{\pgfqpoint{5.194645in}{2.412585in}}%
\pgfpathlineto{\pgfqpoint{5.194510in}{2.333219in}}%
\pgfpathlineto{\pgfqpoint{5.194879in}{2.353746in}}%
\pgfpathlineto{\pgfqpoint{5.195064in}{2.397028in}}%
\pgfpathlineto{\pgfqpoint{5.195334in}{2.329924in}}%
\pgfpathlineto{\pgfqpoint{5.196036in}{2.379523in}}%
\pgfpathlineto{\pgfqpoint{5.196147in}{2.342522in}}%
\pgfpathlineto{\pgfqpoint{5.196627in}{2.391899in}}%
\pgfpathlineto{\pgfqpoint{5.197131in}{2.364728in}}%
\pgfpathlineto{\pgfqpoint{5.197168in}{2.419386in}}%
\pgfpathlineto{\pgfqpoint{5.197710in}{2.350734in}}%
\pgfpathlineto{\pgfqpoint{5.198227in}{2.358887in}}%
\pgfpathlineto{\pgfqpoint{5.198682in}{2.337872in}}%
\pgfpathlineto{\pgfqpoint{5.198830in}{2.419606in}}%
\pgfpathlineto{\pgfqpoint{5.199310in}{2.365067in}}%
\pgfpathlineto{\pgfqpoint{5.200454in}{2.419405in}}%
\pgfpathlineto{\pgfqpoint{5.199470in}{2.329250in}}%
\pgfpathlineto{\pgfqpoint{5.200467in}{2.411951in}}%
\pgfpathlineto{\pgfqpoint{5.200590in}{2.338621in}}%
\pgfpathlineto{\pgfqpoint{5.201267in}{2.413655in}}%
\pgfpathlineto{\pgfqpoint{5.201636in}{2.345186in}}%
\pgfpathlineto{\pgfqpoint{5.201648in}{2.345139in}}%
\pgfpathlineto{\pgfqpoint{5.201661in}{2.348440in}}%
\pgfpathlineto{\pgfqpoint{5.202104in}{2.427849in}}%
\pgfpathlineto{\pgfqpoint{5.202436in}{2.337198in}}%
\pgfpathlineto{\pgfqpoint{5.202891in}{2.416411in}}%
\pgfpathlineto{\pgfqpoint{5.203384in}{2.342632in}}%
\pgfpathlineto{\pgfqpoint{5.204085in}{2.356336in}}%
\pgfpathlineto{\pgfqpoint{5.205107in}{2.413451in}}%
\pgfpathlineto{\pgfqpoint{5.204947in}{2.338270in}}%
\pgfpathlineto{\pgfqpoint{5.205254in}{2.396223in}}%
\pgfpathlineto{\pgfqpoint{5.205734in}{2.326790in}}%
\pgfpathlineto{\pgfqpoint{5.205931in}{2.411292in}}%
\pgfpathlineto{\pgfqpoint{5.206387in}{2.365332in}}%
\pgfpathlineto{\pgfqpoint{5.206916in}{2.410711in}}%
\pgfpathlineto{\pgfqpoint{5.206534in}{2.333589in}}%
\pgfpathlineto{\pgfqpoint{5.207285in}{2.350062in}}%
\pgfpathlineto{\pgfqpoint{5.208085in}{2.333075in}}%
\pgfpathlineto{\pgfqpoint{5.207704in}{2.408231in}}%
\pgfpathlineto{\pgfqpoint{5.208344in}{2.373784in}}%
\pgfpathlineto{\pgfqpoint{5.209267in}{2.411445in}}%
\pgfpathlineto{\pgfqpoint{5.208873in}{2.331207in}}%
\pgfpathlineto{\pgfqpoint{5.209439in}{2.378205in}}%
\pgfpathlineto{\pgfqpoint{5.209661in}{2.326152in}}%
\pgfpathlineto{\pgfqpoint{5.210054in}{2.406811in}}%
\pgfpathlineto{\pgfqpoint{5.210559in}{2.356535in}}%
\pgfpathlineto{\pgfqpoint{5.210596in}{2.404149in}}%
\pgfpathlineto{\pgfqpoint{5.211236in}{2.330647in}}%
\pgfpathlineto{\pgfqpoint{5.211679in}{2.377405in}}%
\pgfpathlineto{\pgfqpoint{5.212688in}{2.401017in}}%
\pgfpathlineto{\pgfqpoint{5.212024in}{2.331960in}}%
\pgfpathlineto{\pgfqpoint{5.212774in}{2.373688in}}%
\pgfpathlineto{\pgfqpoint{5.212811in}{2.332745in}}%
\pgfpathlineto{\pgfqpoint{5.213537in}{2.404503in}}%
\pgfpathlineto{\pgfqpoint{5.213870in}{2.395925in}}%
\pgfpathlineto{\pgfqpoint{5.214325in}{2.409066in}}%
\pgfpathlineto{\pgfqpoint{5.214485in}{2.340595in}}%
\pgfpathlineto{\pgfqpoint{5.214793in}{2.375852in}}%
\pgfpathlineto{\pgfqpoint{5.215273in}{2.344149in}}%
\pgfpathlineto{\pgfqpoint{5.215113in}{2.410067in}}%
\pgfpathlineto{\pgfqpoint{5.215876in}{2.391548in}}%
\pgfpathlineto{\pgfqpoint{5.216799in}{2.409255in}}%
\pgfpathlineto{\pgfqpoint{5.216848in}{2.348725in}}%
\pgfpathlineto{\pgfqpoint{5.216934in}{2.384524in}}%
\pgfpathlineto{\pgfqpoint{5.217882in}{2.340093in}}%
\pgfpathlineto{\pgfqpoint{5.217587in}{2.416439in}}%
\pgfpathlineto{\pgfqpoint{5.218079in}{2.353082in}}%
\pgfpathlineto{\pgfqpoint{5.219162in}{2.409541in}}%
\pgfpathlineto{\pgfqpoint{5.218916in}{2.339179in}}%
\pgfpathlineto{\pgfqpoint{5.219322in}{2.371027in}}%
\pgfpathlineto{\pgfqpoint{5.219704in}{2.339097in}}%
\pgfpathlineto{\pgfqpoint{5.219950in}{2.412166in}}%
\pgfpathlineto{\pgfqpoint{5.220504in}{2.345163in}}%
\pgfpathlineto{\pgfqpoint{5.221611in}{2.405838in}}%
\pgfpathlineto{\pgfqpoint{5.221279in}{2.343475in}}%
\pgfpathlineto{\pgfqpoint{5.221660in}{2.389592in}}%
\pgfpathlineto{\pgfqpoint{5.222534in}{2.412415in}}%
\pgfpathlineto{\pgfqpoint{5.222854in}{2.332498in}}%
\pgfpathlineto{\pgfqpoint{5.223322in}{2.414896in}}%
\pgfpathlineto{\pgfqpoint{5.224048in}{2.385262in}}%
\pgfpathlineto{\pgfqpoint{5.224159in}{2.375830in}}%
\pgfpathlineto{\pgfqpoint{5.224110in}{2.414663in}}%
\pgfpathlineto{\pgfqpoint{5.224196in}{2.408021in}}%
\pgfpathlineto{\pgfqpoint{5.224996in}{2.418255in}}%
\pgfpathlineto{\pgfqpoint{5.224614in}{2.334554in}}%
\pgfpathlineto{\pgfqpoint{5.225193in}{2.351699in}}%
\pgfpathlineto{\pgfqpoint{5.226177in}{2.335577in}}%
\pgfpathlineto{\pgfqpoint{5.225784in}{2.420031in}}%
\pgfpathlineto{\pgfqpoint{5.226300in}{2.344085in}}%
\pgfpathlineto{\pgfqpoint{5.227359in}{2.420900in}}%
\pgfpathlineto{\pgfqpoint{5.226977in}{2.325477in}}%
\pgfpathlineto{\pgfqpoint{5.227445in}{2.400856in}}%
\pgfpathlineto{\pgfqpoint{5.227457in}{2.401501in}}%
\pgfpathlineto{\pgfqpoint{5.227556in}{2.354718in}}%
\pgfpathlineto{\pgfqpoint{5.227765in}{2.329730in}}%
\pgfpathlineto{\pgfqpoint{5.228147in}{2.414870in}}%
\pgfpathlineto{\pgfqpoint{5.228664in}{2.351657in}}%
\pgfpathlineto{\pgfqpoint{5.229722in}{2.408776in}}%
\pgfpathlineto{\pgfqpoint{5.229439in}{2.334685in}}%
\pgfpathlineto{\pgfqpoint{5.229796in}{2.392011in}}%
\pgfpathlineto{\pgfqpoint{5.230510in}{2.401612in}}%
\pgfpathlineto{\pgfqpoint{5.230128in}{2.336675in}}%
\pgfpathlineto{\pgfqpoint{5.230867in}{2.383985in}}%
\pgfpathlineto{\pgfqpoint{5.231802in}{2.341603in}}%
\pgfpathlineto{\pgfqpoint{5.231568in}{2.399152in}}%
\pgfpathlineto{\pgfqpoint{5.231962in}{2.390406in}}%
\pgfpathlineto{\pgfqpoint{5.232417in}{2.407292in}}%
\pgfpathlineto{\pgfqpoint{5.232590in}{2.339430in}}%
\pgfpathlineto{\pgfqpoint{5.233033in}{2.373252in}}%
\pgfpathlineto{\pgfqpoint{5.233377in}{2.342827in}}%
\pgfpathlineto{\pgfqpoint{5.233205in}{2.412637in}}%
\pgfpathlineto{\pgfqpoint{5.233956in}{2.387300in}}%
\pgfpathlineto{\pgfqpoint{5.234891in}{2.411029in}}%
\pgfpathlineto{\pgfqpoint{5.234940in}{2.346752in}}%
\pgfpathlineto{\pgfqpoint{5.235039in}{2.372435in}}%
\pgfpathlineto{\pgfqpoint{5.235174in}{2.346914in}}%
\pgfpathlineto{\pgfqpoint{5.235679in}{2.421853in}}%
\pgfpathlineto{\pgfqpoint{5.236171in}{2.360918in}}%
\pgfpathlineto{\pgfqpoint{5.236467in}{2.414673in}}%
\pgfpathlineto{\pgfqpoint{5.236737in}{2.351808in}}%
\pgfpathlineto{\pgfqpoint{5.237279in}{2.369359in}}%
\pgfpathlineto{\pgfqpoint{5.237980in}{2.348106in}}%
\pgfpathlineto{\pgfqpoint{5.238030in}{2.407960in}}%
\pgfpathlineto{\pgfqpoint{5.238350in}{2.387341in}}%
\pgfpathlineto{\pgfqpoint{5.238817in}{2.401078in}}%
\pgfpathlineto{\pgfqpoint{5.238768in}{2.349211in}}%
\pgfpathlineto{\pgfqpoint{5.239433in}{2.374321in}}%
\pgfpathlineto{\pgfqpoint{5.240343in}{2.341748in}}%
\pgfpathlineto{\pgfqpoint{5.239605in}{2.402487in}}%
\pgfpathlineto{\pgfqpoint{5.240380in}{2.393190in}}%
\pgfpathlineto{\pgfqpoint{5.241402in}{2.410494in}}%
\pgfpathlineto{\pgfqpoint{5.241119in}{2.340962in}}%
\pgfpathlineto{\pgfqpoint{5.241439in}{2.374783in}}%
\pgfpathlineto{\pgfqpoint{5.241907in}{2.342538in}}%
\pgfpathlineto{\pgfqpoint{5.242288in}{2.410890in}}%
\pgfpathlineto{\pgfqpoint{5.242534in}{2.377624in}}%
\pgfpathlineto{\pgfqpoint{5.243076in}{2.417777in}}%
\pgfpathlineto{\pgfqpoint{5.242694in}{2.342865in}}%
\pgfpathlineto{\pgfqpoint{5.243556in}{2.360249in}}%
\pgfpathlineto{\pgfqpoint{5.244257in}{2.342489in}}%
\pgfpathlineto{\pgfqpoint{5.243863in}{2.415013in}}%
\pgfpathlineto{\pgfqpoint{5.244614in}{2.383169in}}%
\pgfpathlineto{\pgfqpoint{5.245439in}{2.412907in}}%
\pgfpathlineto{\pgfqpoint{5.245045in}{2.340821in}}%
\pgfpathlineto{\pgfqpoint{5.245710in}{2.376523in}}%
\pgfpathlineto{\pgfqpoint{5.246608in}{2.346982in}}%
\pgfpathlineto{\pgfqpoint{5.246214in}{2.409891in}}%
\pgfpathlineto{\pgfqpoint{5.246645in}{2.382948in}}%
\pgfpathlineto{\pgfqpoint{5.247002in}{2.410814in}}%
\pgfpathlineto{\pgfqpoint{5.247396in}{2.343718in}}%
\pgfpathlineto{\pgfqpoint{5.247703in}{2.362572in}}%
\pgfpathlineto{\pgfqpoint{5.248183in}{2.343386in}}%
\pgfpathlineto{\pgfqpoint{5.247790in}{2.407919in}}%
\pgfpathlineto{\pgfqpoint{5.248787in}{2.380902in}}%
\pgfpathlineto{\pgfqpoint{5.249353in}{2.402740in}}%
\pgfpathlineto{\pgfqpoint{5.249747in}{2.341567in}}%
\pgfpathlineto{\pgfqpoint{5.249907in}{2.388498in}}%
\pgfpathlineto{\pgfqpoint{5.249931in}{2.382780in}}%
\pgfpathlineto{\pgfqpoint{5.250534in}{2.344133in}}%
\pgfpathlineto{\pgfqpoint{5.250140in}{2.401038in}}%
\pgfpathlineto{\pgfqpoint{5.251063in}{2.362962in}}%
\pgfpathlineto{\pgfqpoint{5.251703in}{2.400078in}}%
\pgfpathlineto{\pgfqpoint{5.251642in}{2.352454in}}%
\pgfpathlineto{\pgfqpoint{5.252085in}{2.360201in}}%
\pgfpathlineto{\pgfqpoint{5.252885in}{2.353218in}}%
\pgfpathlineto{\pgfqpoint{5.252934in}{2.413608in}}%
\pgfpathlineto{\pgfqpoint{5.253156in}{2.376702in}}%
\pgfpathlineto{\pgfqpoint{5.253217in}{2.358262in}}%
\pgfpathlineto{\pgfqpoint{5.253254in}{2.386168in}}%
\pgfpathlineto{\pgfqpoint{5.253710in}{2.407627in}}%
\pgfpathlineto{\pgfqpoint{5.253660in}{2.347903in}}%
\pgfpathlineto{\pgfqpoint{5.254350in}{2.380003in}}%
\pgfpathlineto{\pgfqpoint{5.254448in}{2.344725in}}%
\pgfpathlineto{\pgfqpoint{5.254497in}{2.404063in}}%
\pgfpathlineto{\pgfqpoint{5.255260in}{2.392070in}}%
\pgfpathlineto{\pgfqpoint{5.256060in}{2.408733in}}%
\pgfpathlineto{\pgfqpoint{5.256011in}{2.344986in}}%
\pgfpathlineto{\pgfqpoint{5.256331in}{2.355974in}}%
\pgfpathlineto{\pgfqpoint{5.256380in}{2.381672in}}%
\pgfpathlineto{\pgfqpoint{5.256836in}{2.407360in}}%
\pgfpathlineto{\pgfqpoint{5.256786in}{2.342546in}}%
\pgfpathlineto{\pgfqpoint{5.257488in}{2.383571in}}%
\pgfpathlineto{\pgfqpoint{5.257574in}{2.339973in}}%
\pgfpathlineto{\pgfqpoint{5.257623in}{2.409400in}}%
\pgfpathlineto{\pgfqpoint{5.258497in}{2.397743in}}%
\pgfpathlineto{\pgfqpoint{5.258534in}{2.404859in}}%
\pgfpathlineto{\pgfqpoint{5.259125in}{2.348273in}}%
\pgfpathlineto{\pgfqpoint{5.259925in}{2.340393in}}%
\pgfpathlineto{\pgfqpoint{5.259543in}{2.410433in}}%
\pgfpathlineto{\pgfqpoint{5.260183in}{2.372191in}}%
\pgfpathlineto{\pgfqpoint{5.260319in}{2.407826in}}%
\pgfpathlineto{\pgfqpoint{5.260700in}{2.337127in}}%
\pgfpathlineto{\pgfqpoint{5.261217in}{2.353380in}}%
\pgfpathlineto{\pgfqpoint{5.261488in}{2.333855in}}%
\pgfpathlineto{\pgfqpoint{5.261537in}{2.405453in}}%
\pgfpathlineto{\pgfqpoint{5.262288in}{2.351959in}}%
\pgfpathlineto{\pgfqpoint{5.263100in}{2.407543in}}%
\pgfpathlineto{\pgfqpoint{5.263051in}{2.341907in}}%
\pgfpathlineto{\pgfqpoint{5.263396in}{2.351062in}}%
\pgfpathlineto{\pgfqpoint{5.263888in}{2.412803in}}%
\pgfpathlineto{\pgfqpoint{5.263839in}{2.339867in}}%
\pgfpathlineto{\pgfqpoint{5.264528in}{2.370148in}}%
\pgfpathlineto{\pgfqpoint{5.265451in}{2.410906in}}%
\pgfpathlineto{\pgfqpoint{5.265402in}{2.344324in}}%
\pgfpathlineto{\pgfqpoint{5.265623in}{2.382580in}}%
\pgfpathlineto{\pgfqpoint{5.266522in}{2.335267in}}%
\pgfpathlineto{\pgfqpoint{5.266239in}{2.409255in}}%
\pgfpathlineto{\pgfqpoint{5.266743in}{2.369394in}}%
\pgfpathlineto{\pgfqpoint{5.267310in}{2.343518in}}%
\pgfpathlineto{\pgfqpoint{5.267703in}{2.417157in}}%
\pgfpathlineto{\pgfqpoint{5.267802in}{2.409502in}}%
\pgfpathlineto{\pgfqpoint{5.268786in}{2.336807in}}%
\pgfpathlineto{\pgfqpoint{5.268590in}{2.419247in}}%
\pgfpathlineto{\pgfqpoint{5.268934in}{2.394016in}}%
\pgfpathlineto{\pgfqpoint{5.269377in}{2.427089in}}%
\pgfpathlineto{\pgfqpoint{5.269586in}{2.328319in}}%
\pgfpathlineto{\pgfqpoint{5.270005in}{2.377499in}}%
\pgfpathlineto{\pgfqpoint{5.270423in}{2.334799in}}%
\pgfpathlineto{\pgfqpoint{5.270165in}{2.427825in}}%
\pgfpathlineto{\pgfqpoint{5.270817in}{2.397470in}}%
\pgfpathlineto{\pgfqpoint{5.271740in}{2.403401in}}%
\pgfpathlineto{\pgfqpoint{5.271211in}{2.339777in}}%
\pgfpathlineto{\pgfqpoint{5.271814in}{2.358239in}}%
\pgfpathlineto{\pgfqpoint{5.271937in}{2.336829in}}%
\pgfpathlineto{\pgfqpoint{5.272503in}{2.395759in}}%
\pgfpathlineto{\pgfqpoint{5.273303in}{2.406650in}}%
\pgfpathlineto{\pgfqpoint{5.272713in}{2.342508in}}%
\pgfpathlineto{\pgfqpoint{5.273439in}{2.365036in}}%
\pgfpathlineto{\pgfqpoint{5.273500in}{2.345742in}}%
\pgfpathlineto{\pgfqpoint{5.274091in}{2.399348in}}%
\pgfpathlineto{\pgfqpoint{5.274546in}{2.363083in}}%
\pgfpathlineto{\pgfqpoint{5.275642in}{2.393533in}}%
\pgfpathlineto{\pgfqpoint{5.275063in}{2.355352in}}%
\pgfpathlineto{\pgfqpoint{5.275679in}{2.379020in}}%
\pgfpathlineto{\pgfqpoint{5.276602in}{2.328367in}}%
\pgfpathlineto{\pgfqpoint{5.276429in}{2.399466in}}%
\pgfpathlineto{\pgfqpoint{5.276749in}{2.389456in}}%
\pgfpathlineto{\pgfqpoint{5.276786in}{2.394639in}}%
\pgfpathlineto{\pgfqpoint{5.277414in}{2.348784in}}%
\pgfpathlineto{\pgfqpoint{5.277648in}{2.353435in}}%
\pgfpathlineto{\pgfqpoint{5.278189in}{2.331941in}}%
\pgfpathlineto{\pgfqpoint{5.278399in}{2.411073in}}%
\pgfpathlineto{\pgfqpoint{5.278620in}{2.382207in}}%
\pgfpathlineto{\pgfqpoint{5.278916in}{2.401430in}}%
\pgfpathlineto{\pgfqpoint{5.279642in}{2.338833in}}%
\pgfpathlineto{\pgfqpoint{5.280491in}{2.328672in}}%
\pgfpathlineto{\pgfqpoint{5.280220in}{2.428009in}}%
\pgfpathlineto{\pgfqpoint{5.280639in}{2.386331in}}%
\pgfpathlineto{\pgfqpoint{5.281082in}{2.419011in}}%
\pgfpathlineto{\pgfqpoint{5.281316in}{2.321593in}}%
\pgfpathlineto{\pgfqpoint{5.281783in}{2.416571in}}%
\pgfpathlineto{\pgfqpoint{5.282817in}{2.321966in}}%
\pgfpathlineto{\pgfqpoint{5.282977in}{2.392989in}}%
\pgfpathlineto{\pgfqpoint{5.283002in}{2.433745in}}%
\pgfpathlineto{\pgfqpoint{5.283236in}{2.319582in}}%
\pgfpathlineto{\pgfqpoint{5.284060in}{2.349001in}}%
\pgfpathlineto{\pgfqpoint{5.284540in}{2.430980in}}%
\pgfpathlineto{\pgfqpoint{5.284183in}{2.316986in}}%
\pgfpathlineto{\pgfqpoint{5.285168in}{2.355056in}}%
\pgfpathlineto{\pgfqpoint{5.285993in}{2.288570in}}%
\pgfpathlineto{\pgfqpoint{5.285648in}{2.430512in}}%
\pgfpathlineto{\pgfqpoint{5.286276in}{2.336962in}}%
\pgfpathlineto{\pgfqpoint{5.286460in}{2.464270in}}%
\pgfpathlineto{\pgfqpoint{5.286953in}{2.316409in}}%
\pgfpathlineto{\pgfqpoint{5.287420in}{2.376273in}}%
\pgfpathlineto{\pgfqpoint{5.288060in}{2.319733in}}%
\pgfpathlineto{\pgfqpoint{5.288245in}{2.398203in}}%
\pgfpathlineto{\pgfqpoint{5.288393in}{2.383706in}}%
\pgfpathlineto{\pgfqpoint{5.288516in}{2.482511in}}%
\pgfpathlineto{\pgfqpoint{5.288774in}{2.272506in}}%
\pgfpathlineto{\pgfqpoint{5.289476in}{2.335937in}}%
\pgfpathlineto{\pgfqpoint{5.289586in}{2.310075in}}%
\pgfpathlineto{\pgfqpoint{5.289906in}{2.395697in}}%
\pgfpathlineto{\pgfqpoint{5.290066in}{2.458020in}}%
\pgfpathlineto{\pgfqpoint{5.290829in}{2.297665in}}%
\pgfpathlineto{\pgfqpoint{5.291002in}{2.376493in}}%
\pgfpathlineto{\pgfqpoint{5.291297in}{2.466458in}}%
\pgfpathlineto{\pgfqpoint{5.291556in}{2.290394in}}%
\pgfpathlineto{\pgfqpoint{5.292183in}{2.413308in}}%
\pgfpathlineto{\pgfqpoint{5.292909in}{2.333687in}}%
\pgfpathlineto{\pgfqpoint{5.293254in}{2.426341in}}%
\pgfpathlineto{\pgfqpoint{5.293303in}{2.393551in}}%
\pgfpathlineto{\pgfqpoint{5.293599in}{2.286535in}}%
\pgfpathlineto{\pgfqpoint{5.294066in}{2.469797in}}%
\pgfpathlineto{\pgfqpoint{5.294497in}{2.316896in}}%
\pgfpathlineto{\pgfqpoint{5.294768in}{2.453600in}}%
\pgfpathlineto{\pgfqpoint{5.295666in}{2.355749in}}%
\pgfpathlineto{\pgfqpoint{5.296380in}{2.299429in}}%
\pgfpathlineto{\pgfqpoint{5.296725in}{2.423596in}}%
\pgfpathlineto{\pgfqpoint{5.296848in}{2.463414in}}%
\pgfpathlineto{\pgfqpoint{5.297020in}{2.360304in}}%
\pgfpathlineto{\pgfqpoint{5.297229in}{2.302379in}}%
\pgfpathlineto{\pgfqpoint{5.297525in}{2.446768in}}%
\pgfpathlineto{\pgfqpoint{5.298116in}{2.379375in}}%
\pgfpathlineto{\pgfqpoint{5.299186in}{2.292253in}}%
\pgfpathlineto{\pgfqpoint{5.298817in}{2.423339in}}%
\pgfpathlineto{\pgfqpoint{5.299309in}{2.344577in}}%
\pgfpathlineto{\pgfqpoint{5.299629in}{2.459276in}}%
\pgfpathlineto{\pgfqpoint{5.300011in}{2.300531in}}%
\pgfpathlineto{\pgfqpoint{5.300479in}{2.410548in}}%
\pgfpathlineto{\pgfqpoint{5.301254in}{2.335520in}}%
\pgfpathlineto{\pgfqpoint{5.301574in}{2.418278in}}%
\pgfpathlineto{\pgfqpoint{5.302399in}{2.461612in}}%
\pgfpathlineto{\pgfqpoint{5.301919in}{2.300806in}}%
\pgfpathlineto{\pgfqpoint{5.302571in}{2.364881in}}%
\pgfpathlineto{\pgfqpoint{5.302620in}{2.306857in}}%
\pgfpathlineto{\pgfqpoint{5.303076in}{2.443811in}}%
\pgfpathlineto{\pgfqpoint{5.303666in}{2.394673in}}%
\pgfpathlineto{\pgfqpoint{5.304368in}{2.429604in}}%
\pgfpathlineto{\pgfqpoint{5.304122in}{2.341782in}}%
\pgfpathlineto{\pgfqpoint{5.304552in}{2.371445in}}%
\pgfpathlineto{\pgfqpoint{5.304688in}{2.291592in}}%
\pgfpathlineto{\pgfqpoint{5.305192in}{2.438558in}}%
\pgfpathlineto{\pgfqpoint{5.305685in}{2.357399in}}%
\pgfpathlineto{\pgfqpoint{5.305845in}{2.452472in}}%
\pgfpathlineto{\pgfqpoint{5.306276in}{2.327280in}}%
\pgfpathlineto{\pgfqpoint{5.306780in}{2.348604in}}%
\pgfpathlineto{\pgfqpoint{5.307482in}{2.304413in}}%
\pgfpathlineto{\pgfqpoint{5.307149in}{2.434928in}}%
\pgfpathlineto{\pgfqpoint{5.307679in}{2.383531in}}%
\pgfpathlineto{\pgfqpoint{5.307962in}{2.451820in}}%
\pgfpathlineto{\pgfqpoint{5.308343in}{2.286559in}}%
\pgfpathlineto{\pgfqpoint{5.308811in}{2.415468in}}%
\pgfpathlineto{\pgfqpoint{5.309143in}{2.331136in}}%
\pgfpathlineto{\pgfqpoint{5.309512in}{2.424346in}}%
\pgfpathlineto{\pgfqpoint{5.309906in}{2.413402in}}%
\pgfpathlineto{\pgfqpoint{5.310731in}{2.446994in}}%
\pgfpathlineto{\pgfqpoint{5.310263in}{2.294543in}}%
\pgfpathlineto{\pgfqpoint{5.310916in}{2.341597in}}%
\pgfpathlineto{\pgfqpoint{5.310952in}{2.281024in}}%
\pgfpathlineto{\pgfqpoint{5.311445in}{2.459648in}}%
\pgfpathlineto{\pgfqpoint{5.311999in}{2.392688in}}%
\pgfpathlineto{\pgfqpoint{5.313032in}{2.284259in}}%
\pgfpathlineto{\pgfqpoint{5.312700in}{2.438992in}}%
\pgfpathlineto{\pgfqpoint{5.313205in}{2.361844in}}%
\pgfpathlineto{\pgfqpoint{5.314226in}{2.463145in}}%
\pgfpathlineto{\pgfqpoint{5.313894in}{2.287137in}}%
\pgfpathlineto{\pgfqpoint{5.314374in}{2.430863in}}%
\pgfpathlineto{\pgfqpoint{5.314706in}{2.325923in}}%
\pgfpathlineto{\pgfqpoint{5.315063in}{2.438968in}}%
\pgfpathlineto{\pgfqpoint{5.315482in}{2.433700in}}%
\pgfpathlineto{\pgfqpoint{5.316515in}{2.282254in}}%
\pgfpathlineto{\pgfqpoint{5.316294in}{2.446588in}}%
\pgfpathlineto{\pgfqpoint{5.316712in}{2.337925in}}%
\pgfpathlineto{\pgfqpoint{5.316995in}{2.466319in}}%
\pgfpathlineto{\pgfqpoint{5.317205in}{2.327425in}}%
\pgfpathlineto{\pgfqpoint{5.317869in}{2.383489in}}%
\pgfpathlineto{\pgfqpoint{5.318595in}{2.273766in}}%
\pgfpathlineto{\pgfqpoint{5.318263in}{2.439701in}}%
\pgfpathlineto{\pgfqpoint{5.318940in}{2.427116in}}%
\pgfpathlineto{\pgfqpoint{5.319765in}{2.468371in}}%
\pgfpathlineto{\pgfqpoint{5.319445in}{2.273424in}}%
\pgfpathlineto{\pgfqpoint{5.319974in}{2.339576in}}%
\pgfpathlineto{\pgfqpoint{5.320245in}{2.320882in}}%
\pgfpathlineto{\pgfqpoint{5.320626in}{2.436888in}}%
\pgfpathlineto{\pgfqpoint{5.320995in}{2.390454in}}%
\pgfpathlineto{\pgfqpoint{5.321857in}{2.448933in}}%
\pgfpathlineto{\pgfqpoint{5.321365in}{2.280140in}}%
\pgfpathlineto{\pgfqpoint{5.322017in}{2.381021in}}%
\pgfpathlineto{\pgfqpoint{5.322066in}{2.280076in}}%
\pgfpathlineto{\pgfqpoint{5.322546in}{2.458718in}}%
\pgfpathlineto{\pgfqpoint{5.323125in}{2.374320in}}%
\pgfpathlineto{\pgfqpoint{5.324146in}{2.274873in}}%
\pgfpathlineto{\pgfqpoint{5.323814in}{2.434085in}}%
\pgfpathlineto{\pgfqpoint{5.324306in}{2.350399in}}%
\pgfpathlineto{\pgfqpoint{5.325328in}{2.469751in}}%
\pgfpathlineto{\pgfqpoint{5.324848in}{2.277800in}}%
\pgfpathlineto{\pgfqpoint{5.325500in}{2.394533in}}%
\pgfpathlineto{\pgfqpoint{5.325697in}{2.328937in}}%
\pgfpathlineto{\pgfqpoint{5.326165in}{2.435347in}}%
\pgfpathlineto{\pgfqpoint{5.326595in}{2.422657in}}%
\pgfpathlineto{\pgfqpoint{5.327617in}{2.290583in}}%
\pgfpathlineto{\pgfqpoint{5.327395in}{2.448690in}}%
\pgfpathlineto{\pgfqpoint{5.327851in}{2.358586in}}%
\pgfpathlineto{\pgfqpoint{5.327888in}{2.348827in}}%
\pgfpathlineto{\pgfqpoint{5.328011in}{2.415153in}}%
\pgfpathlineto{\pgfqpoint{5.328048in}{2.408653in}}%
\pgfpathlineto{\pgfqpoint{5.328109in}{2.449310in}}%
\pgfpathlineto{\pgfqpoint{5.328319in}{2.313735in}}%
\pgfpathlineto{\pgfqpoint{5.329106in}{2.355430in}}%
\pgfpathlineto{\pgfqpoint{5.330189in}{2.452281in}}%
\pgfpathlineto{\pgfqpoint{5.329697in}{2.287780in}}%
\pgfpathlineto{\pgfqpoint{5.330288in}{2.377933in}}%
\pgfpathlineto{\pgfqpoint{5.330325in}{2.392982in}}%
\pgfpathlineto{\pgfqpoint{5.330362in}{2.351592in}}%
\pgfpathlineto{\pgfqpoint{5.330399in}{2.284407in}}%
\pgfpathlineto{\pgfqpoint{5.330891in}{2.464614in}}%
\pgfpathlineto{\pgfqpoint{5.331457in}{2.377301in}}%
\pgfpathlineto{\pgfqpoint{5.331469in}{2.377416in}}%
\pgfpathlineto{\pgfqpoint{5.331482in}{2.374970in}}%
\pgfpathlineto{\pgfqpoint{5.332479in}{2.298833in}}%
\pgfpathlineto{\pgfqpoint{5.331592in}{2.430445in}}%
\pgfpathlineto{\pgfqpoint{5.332602in}{2.364929in}}%
\pgfpathlineto{\pgfqpoint{5.332639in}{2.353838in}}%
\pgfpathlineto{\pgfqpoint{5.332675in}{2.368163in}}%
\pgfpathlineto{\pgfqpoint{5.333672in}{2.460566in}}%
\pgfpathlineto{\pgfqpoint{5.333180in}{2.310094in}}%
\pgfpathlineto{\pgfqpoint{5.333808in}{2.406646in}}%
\pgfpathlineto{\pgfqpoint{5.334571in}{2.312562in}}%
\pgfpathlineto{\pgfqpoint{5.334374in}{2.425454in}}%
\pgfpathlineto{\pgfqpoint{5.334903in}{2.400508in}}%
\pgfpathlineto{\pgfqpoint{5.335752in}{2.458520in}}%
\pgfpathlineto{\pgfqpoint{5.335260in}{2.288779in}}%
\pgfpathlineto{\pgfqpoint{5.335925in}{2.345149in}}%
\pgfpathlineto{\pgfqpoint{5.336651in}{2.295714in}}%
\pgfpathlineto{\pgfqpoint{5.336442in}{2.468649in}}%
\pgfpathlineto{\pgfqpoint{5.337008in}{2.390183in}}%
\pgfpathlineto{\pgfqpoint{5.337143in}{2.442720in}}%
\pgfpathlineto{\pgfqpoint{5.337340in}{2.318734in}}%
\pgfpathlineto{\pgfqpoint{5.337894in}{2.377323in}}%
\pgfpathlineto{\pgfqpoint{5.338029in}{2.287559in}}%
\pgfpathlineto{\pgfqpoint{5.338522in}{2.444091in}}%
\pgfpathlineto{\pgfqpoint{5.339014in}{2.345999in}}%
\pgfpathlineto{\pgfqpoint{5.339223in}{2.463385in}}%
\pgfpathlineto{\pgfqpoint{5.339420in}{2.312259in}}%
\pgfpathlineto{\pgfqpoint{5.340097in}{2.338640in}}%
\pgfpathlineto{\pgfqpoint{5.340811in}{2.282541in}}%
\pgfpathlineto{\pgfqpoint{5.340602in}{2.432485in}}%
\pgfpathlineto{\pgfqpoint{5.341131in}{2.410932in}}%
\pgfpathlineto{\pgfqpoint{5.341303in}{2.446722in}}%
\pgfpathlineto{\pgfqpoint{5.341377in}{2.365595in}}%
\pgfpathlineto{\pgfqpoint{5.341463in}{2.372150in}}%
\pgfpathlineto{\pgfqpoint{5.342189in}{2.299948in}}%
\pgfpathlineto{\pgfqpoint{5.341992in}{2.470541in}}%
\pgfpathlineto{\pgfqpoint{5.342558in}{2.402549in}}%
\pgfpathlineto{\pgfqpoint{5.342694in}{2.439709in}}%
\pgfpathlineto{\pgfqpoint{5.342878in}{2.314744in}}%
\pgfpathlineto{\pgfqpoint{5.343026in}{2.336610in}}%
\pgfpathlineto{\pgfqpoint{5.343580in}{2.290367in}}%
\pgfpathlineto{\pgfqpoint{5.343371in}{2.451353in}}%
\pgfpathlineto{\pgfqpoint{5.344023in}{2.405831in}}%
\pgfpathlineto{\pgfqpoint{5.344774in}{2.463049in}}%
\pgfpathlineto{\pgfqpoint{5.344429in}{2.297005in}}%
\pgfpathlineto{\pgfqpoint{5.345082in}{2.353395in}}%
\pgfpathlineto{\pgfqpoint{5.345648in}{2.320718in}}%
\pgfpathlineto{\pgfqpoint{5.345463in}{2.436898in}}%
\pgfpathlineto{\pgfqpoint{5.345968in}{2.391871in}}%
\pgfpathlineto{\pgfqpoint{5.346842in}{2.449726in}}%
\pgfpathlineto{\pgfqpoint{5.346349in}{2.284607in}}%
\pgfpathlineto{\pgfqpoint{5.347014in}{2.356246in}}%
\pgfpathlineto{\pgfqpoint{5.347075in}{2.301611in}}%
\pgfpathlineto{\pgfqpoint{5.347543in}{2.469086in}}%
\pgfpathlineto{\pgfqpoint{5.348097in}{2.406831in}}%
\pgfpathlineto{\pgfqpoint{5.349131in}{2.282359in}}%
\pgfpathlineto{\pgfqpoint{5.348922in}{2.454644in}}%
\pgfpathlineto{\pgfqpoint{5.349315in}{2.347240in}}%
\pgfpathlineto{\pgfqpoint{5.350325in}{2.462116in}}%
\pgfpathlineto{\pgfqpoint{5.349980in}{2.304564in}}%
\pgfpathlineto{\pgfqpoint{5.350460in}{2.397280in}}%
\pgfpathlineto{\pgfqpoint{5.350509in}{2.308569in}}%
\pgfpathlineto{\pgfqpoint{5.351014in}{2.438444in}}%
\pgfpathlineto{\pgfqpoint{5.351555in}{2.415301in}}%
\pgfpathlineto{\pgfqpoint{5.351691in}{2.459696in}}%
\pgfpathlineto{\pgfqpoint{5.351900in}{2.301959in}}%
\pgfpathlineto{\pgfqpoint{5.352565in}{2.352731in}}%
\pgfpathlineto{\pgfqpoint{5.353315in}{2.300192in}}%
\pgfpathlineto{\pgfqpoint{5.353106in}{2.443020in}}%
\pgfpathlineto{\pgfqpoint{5.353635in}{2.415685in}}%
\pgfpathlineto{\pgfqpoint{5.354706in}{2.295333in}}%
\pgfpathlineto{\pgfqpoint{5.354460in}{2.443718in}}%
\pgfpathlineto{\pgfqpoint{5.354940in}{2.364819in}}%
\pgfpathlineto{\pgfqpoint{5.355162in}{2.453901in}}%
\pgfpathlineto{\pgfqpoint{5.355445in}{2.309225in}}%
\pgfpathlineto{\pgfqpoint{5.356011in}{2.346006in}}%
\pgfpathlineto{\pgfqpoint{5.356060in}{2.315272in}}%
\pgfpathlineto{\pgfqpoint{5.356589in}{2.418802in}}%
\pgfpathlineto{\pgfqpoint{5.357057in}{2.391320in}}%
\pgfpathlineto{\pgfqpoint{5.357081in}{2.413315in}}%
\pgfpathlineto{\pgfqpoint{5.357463in}{2.323240in}}%
\pgfpathlineto{\pgfqpoint{5.358128in}{2.361006in}}%
\pgfpathlineto{\pgfqpoint{5.358165in}{2.324649in}}%
\pgfpathlineto{\pgfqpoint{5.358558in}{2.420137in}}%
\pgfpathlineto{\pgfqpoint{5.359211in}{2.376444in}}%
\pgfpathlineto{\pgfqpoint{5.359334in}{2.412378in}}%
\pgfpathlineto{\pgfqpoint{5.360171in}{2.336066in}}%
\pgfpathlineto{\pgfqpoint{5.360343in}{2.411797in}}%
\pgfpathlineto{\pgfqpoint{5.360823in}{2.342206in}}%
\pgfpathlineto{\pgfqpoint{5.360651in}{2.418000in}}%
\pgfpathlineto{\pgfqpoint{5.361451in}{2.404300in}}%
\pgfpathlineto{\pgfqpoint{5.361894in}{2.422918in}}%
\pgfpathlineto{\pgfqpoint{5.361721in}{2.339032in}}%
\pgfpathlineto{\pgfqpoint{5.362485in}{2.342954in}}%
\pgfpathlineto{\pgfqpoint{5.362731in}{2.426029in}}%
\pgfpathlineto{\pgfqpoint{5.363691in}{2.369724in}}%
\pgfpathlineto{\pgfqpoint{5.363728in}{2.338038in}}%
\pgfpathlineto{\pgfqpoint{5.364614in}{2.415843in}}%
\pgfpathlineto{\pgfqpoint{5.364761in}{2.401960in}}%
\pgfpathlineto{\pgfqpoint{5.364786in}{2.415179in}}%
\pgfpathlineto{\pgfqpoint{5.365697in}{2.330563in}}%
\pgfpathlineto{\pgfqpoint{5.365771in}{2.335173in}}%
\pgfpathlineto{\pgfqpoint{5.365783in}{2.331130in}}%
\pgfpathlineto{\pgfqpoint{5.366041in}{2.421592in}}%
\pgfpathlineto{\pgfqpoint{5.366755in}{2.369361in}}%
\pgfpathlineto{\pgfqpoint{5.366866in}{2.429098in}}%
\pgfpathlineto{\pgfqpoint{5.367457in}{2.328580in}}%
\pgfpathlineto{\pgfqpoint{5.367838in}{2.361301in}}%
\pgfpathlineto{\pgfqpoint{5.368257in}{2.344907in}}%
\pgfpathlineto{\pgfqpoint{5.368749in}{2.413098in}}%
\pgfpathlineto{\pgfqpoint{5.368872in}{2.401461in}}%
\pgfpathlineto{\pgfqpoint{5.369352in}{2.417632in}}%
\pgfpathlineto{\pgfqpoint{5.369094in}{2.322934in}}%
\pgfpathlineto{\pgfqpoint{5.369795in}{2.387218in}}%
\pgfpathlineto{\pgfqpoint{5.369918in}{2.323579in}}%
\pgfpathlineto{\pgfqpoint{5.370177in}{2.424376in}}%
\pgfpathlineto{\pgfqpoint{5.370915in}{2.370364in}}%
\pgfpathlineto{\pgfqpoint{5.371161in}{2.319109in}}%
\pgfpathlineto{\pgfqpoint{5.371014in}{2.421433in}}%
\pgfpathlineto{\pgfqpoint{5.372011in}{2.365079in}}%
\pgfpathlineto{\pgfqpoint{5.372651in}{2.419917in}}%
\pgfpathlineto{\pgfqpoint{5.372798in}{2.329398in}}%
\pgfpathlineto{\pgfqpoint{5.373106in}{2.358411in}}%
\pgfpathlineto{\pgfqpoint{5.373217in}{2.315400in}}%
\pgfpathlineto{\pgfqpoint{5.373463in}{2.426023in}}%
\pgfpathlineto{\pgfqpoint{5.374189in}{2.366957in}}%
\pgfpathlineto{\pgfqpoint{5.374300in}{2.424261in}}%
\pgfpathlineto{\pgfqpoint{5.374866in}{2.330020in}}%
\pgfpathlineto{\pgfqpoint{5.375248in}{2.359417in}}%
\pgfpathlineto{\pgfqpoint{5.376097in}{2.315367in}}%
\pgfpathlineto{\pgfqpoint{5.375937in}{2.410487in}}%
\pgfpathlineto{\pgfqpoint{5.376331in}{2.401915in}}%
\pgfpathlineto{\pgfqpoint{5.376343in}{2.402251in}}%
\pgfpathlineto{\pgfqpoint{5.376392in}{2.377966in}}%
\pgfpathlineto{\pgfqpoint{5.376921in}{2.324921in}}%
\pgfpathlineto{\pgfqpoint{5.376749in}{2.415555in}}%
\pgfpathlineto{\pgfqpoint{5.377525in}{2.361948in}}%
\pgfpathlineto{\pgfqpoint{5.378398in}{2.443883in}}%
\pgfpathlineto{\pgfqpoint{5.378152in}{2.319813in}}%
\pgfpathlineto{\pgfqpoint{5.378632in}{2.372821in}}%
\pgfpathlineto{\pgfqpoint{5.378977in}{2.331822in}}%
\pgfpathlineto{\pgfqpoint{5.379223in}{2.422532in}}%
\pgfpathlineto{\pgfqpoint{5.379789in}{2.338386in}}%
\pgfpathlineto{\pgfqpoint{5.380626in}{2.318463in}}%
\pgfpathlineto{\pgfqpoint{5.380060in}{2.419319in}}%
\pgfpathlineto{\pgfqpoint{5.380835in}{2.369269in}}%
\pgfpathlineto{\pgfqpoint{5.381192in}{2.417959in}}%
\pgfpathlineto{\pgfqpoint{5.381438in}{2.318871in}}%
\pgfpathlineto{\pgfqpoint{5.381943in}{2.370302in}}%
\pgfpathlineto{\pgfqpoint{5.382263in}{2.322304in}}%
\pgfpathlineto{\pgfqpoint{5.382608in}{2.419907in}}%
\pgfpathlineto{\pgfqpoint{5.383026in}{2.375052in}}%
\pgfpathlineto{\pgfqpoint{5.383063in}{2.353619in}}%
\pgfpathlineto{\pgfqpoint{5.383912in}{2.320457in}}%
\pgfpathlineto{\pgfqpoint{5.383346in}{2.418341in}}%
\pgfpathlineto{\pgfqpoint{5.384134in}{2.393266in}}%
\pgfpathlineto{\pgfqpoint{5.384983in}{2.422630in}}%
\pgfpathlineto{\pgfqpoint{5.384737in}{2.327057in}}%
\pgfpathlineto{\pgfqpoint{5.385217in}{2.378342in}}%
\pgfpathlineto{\pgfqpoint{5.385561in}{2.314356in}}%
\pgfpathlineto{\pgfqpoint{5.385808in}{2.425484in}}%
\pgfpathlineto{\pgfqpoint{5.386324in}{2.372474in}}%
\pgfpathlineto{\pgfqpoint{5.386386in}{2.314076in}}%
\pgfpathlineto{\pgfqpoint{5.386632in}{2.430472in}}%
\pgfpathlineto{\pgfqpoint{5.387408in}{2.377066in}}%
\pgfpathlineto{\pgfqpoint{5.387457in}{2.434253in}}%
\pgfpathlineto{\pgfqpoint{5.388441in}{2.323591in}}%
\pgfpathlineto{\pgfqpoint{5.388515in}{2.380819in}}%
\pgfpathlineto{\pgfqpoint{5.389266in}{2.326780in}}%
\pgfpathlineto{\pgfqpoint{5.389106in}{2.428136in}}%
\pgfpathlineto{\pgfqpoint{5.389660in}{2.338233in}}%
\pgfpathlineto{\pgfqpoint{5.389672in}{2.328854in}}%
\pgfpathlineto{\pgfqpoint{5.389931in}{2.431381in}}%
\pgfpathlineto{\pgfqpoint{5.390706in}{2.373184in}}%
\pgfpathlineto{\pgfqpoint{5.390755in}{2.429779in}}%
\pgfpathlineto{\pgfqpoint{5.391728in}{2.324084in}}%
\pgfpathlineto{\pgfqpoint{5.391801in}{2.368970in}}%
\pgfpathlineto{\pgfqpoint{5.392552in}{2.322029in}}%
\pgfpathlineto{\pgfqpoint{5.392392in}{2.416133in}}%
\pgfpathlineto{\pgfqpoint{5.392909in}{2.366022in}}%
\pgfpathlineto{\pgfqpoint{5.393217in}{2.421435in}}%
\pgfpathlineto{\pgfqpoint{5.393364in}{2.334795in}}%
\pgfpathlineto{\pgfqpoint{5.393377in}{2.326587in}}%
\pgfpathlineto{\pgfqpoint{5.394054in}{2.435410in}}%
\pgfpathlineto{\pgfqpoint{5.394398in}{2.384439in}}%
\pgfpathlineto{\pgfqpoint{5.395432in}{2.336645in}}%
\pgfpathlineto{\pgfqpoint{5.394866in}{2.432008in}}%
\pgfpathlineto{\pgfqpoint{5.395568in}{2.364385in}}%
\pgfpathlineto{\pgfqpoint{5.395691in}{2.427144in}}%
\pgfpathlineto{\pgfqpoint{5.395838in}{2.333966in}}%
\pgfpathlineto{\pgfqpoint{5.396638in}{2.351708in}}%
\pgfpathlineto{\pgfqpoint{5.396663in}{2.322908in}}%
\pgfpathlineto{\pgfqpoint{5.397697in}{2.408010in}}%
\pgfpathlineto{\pgfqpoint{5.397709in}{2.405278in}}%
\pgfpathlineto{\pgfqpoint{5.397906in}{2.321586in}}%
\pgfpathlineto{\pgfqpoint{5.398238in}{2.417601in}}%
\pgfpathlineto{\pgfqpoint{5.398915in}{2.358390in}}%
\pgfpathlineto{\pgfqpoint{5.399801in}{2.433979in}}%
\pgfpathlineto{\pgfqpoint{5.399961in}{2.322087in}}%
\pgfpathlineto{\pgfqpoint{5.400023in}{2.375445in}}%
\pgfpathlineto{\pgfqpoint{5.400774in}{2.319783in}}%
\pgfpathlineto{\pgfqpoint{5.400626in}{2.431128in}}%
\pgfpathlineto{\pgfqpoint{5.401131in}{2.372664in}}%
\pgfpathlineto{\pgfqpoint{5.401451in}{2.428047in}}%
\pgfpathlineto{\pgfqpoint{5.401598in}{2.333943in}}%
\pgfpathlineto{\pgfqpoint{5.402103in}{2.343669in}}%
\pgfpathlineto{\pgfqpoint{5.402841in}{2.323180in}}%
\pgfpathlineto{\pgfqpoint{5.402263in}{2.434190in}}%
\pgfpathlineto{\pgfqpoint{5.403063in}{2.411371in}}%
\pgfpathlineto{\pgfqpoint{5.403912in}{2.440965in}}%
\pgfpathlineto{\pgfqpoint{5.403567in}{2.325150in}}%
\pgfpathlineto{\pgfqpoint{5.404146in}{2.366815in}}%
\pgfpathlineto{\pgfqpoint{5.404884in}{2.308114in}}%
\pgfpathlineto{\pgfqpoint{5.404737in}{2.436804in}}%
\pgfpathlineto{\pgfqpoint{5.405241in}{2.383700in}}%
\pgfpathlineto{\pgfqpoint{5.405254in}{2.383620in}}%
\pgfpathlineto{\pgfqpoint{5.405709in}{2.316366in}}%
\pgfpathlineto{\pgfqpoint{5.405549in}{2.417444in}}%
\pgfpathlineto{\pgfqpoint{5.406349in}{2.389235in}}%
\pgfpathlineto{\pgfqpoint{5.407198in}{2.439701in}}%
\pgfpathlineto{\pgfqpoint{5.407346in}{2.322299in}}%
\pgfpathlineto{\pgfqpoint{5.407432in}{2.381548in}}%
\pgfpathlineto{\pgfqpoint{5.407666in}{2.320850in}}%
\pgfpathlineto{\pgfqpoint{5.408011in}{2.451906in}}%
\pgfpathlineto{\pgfqpoint{5.408564in}{2.339267in}}%
\pgfpathlineto{\pgfqpoint{5.408995in}{2.308635in}}%
\pgfpathlineto{\pgfqpoint{5.408835in}{2.451199in}}%
\pgfpathlineto{\pgfqpoint{5.409537in}{2.364990in}}%
\pgfpathlineto{\pgfqpoint{5.409660in}{2.430766in}}%
\pgfpathlineto{\pgfqpoint{5.409807in}{2.317257in}}%
\pgfpathlineto{\pgfqpoint{5.410607in}{2.356764in}}%
\pgfpathlineto{\pgfqpoint{5.410632in}{2.324161in}}%
\pgfpathlineto{\pgfqpoint{5.411297in}{2.431146in}}%
\pgfpathlineto{\pgfqpoint{5.411691in}{2.396532in}}%
\pgfpathlineto{\pgfqpoint{5.412121in}{2.446990in}}%
\pgfpathlineto{\pgfqpoint{5.411863in}{2.328465in}}%
\pgfpathlineto{\pgfqpoint{5.412577in}{2.348913in}}%
\pgfpathlineto{\pgfqpoint{5.413094in}{2.313514in}}%
\pgfpathlineto{\pgfqpoint{5.412946in}{2.443523in}}%
\pgfpathlineto{\pgfqpoint{5.413635in}{2.361132in}}%
\pgfpathlineto{\pgfqpoint{5.413758in}{2.430153in}}%
\pgfpathlineto{\pgfqpoint{5.413918in}{2.311801in}}%
\pgfpathlineto{\pgfqpoint{5.414718in}{2.338107in}}%
\pgfpathlineto{\pgfqpoint{5.414743in}{2.317179in}}%
\pgfpathlineto{\pgfqpoint{5.415407in}{2.436301in}}%
\pgfpathlineto{\pgfqpoint{5.415789in}{2.394610in}}%
\pgfpathlineto{\pgfqpoint{5.416232in}{2.450814in}}%
\pgfpathlineto{\pgfqpoint{5.415974in}{2.322427in}}%
\pgfpathlineto{\pgfqpoint{5.416687in}{2.336979in}}%
\pgfpathlineto{\pgfqpoint{5.416786in}{2.330009in}}%
\pgfpathlineto{\pgfqpoint{5.417032in}{2.430145in}}%
\pgfpathlineto{\pgfqpoint{5.417044in}{2.443581in}}%
\pgfpathlineto{\pgfqpoint{5.418029in}{2.324035in}}%
\pgfpathlineto{\pgfqpoint{5.418091in}{2.387972in}}%
\pgfpathlineto{\pgfqpoint{5.418841in}{2.321337in}}%
\pgfpathlineto{\pgfqpoint{5.418694in}{2.428565in}}%
\pgfpathlineto{\pgfqpoint{5.419210in}{2.372562in}}%
\pgfpathlineto{\pgfqpoint{5.420084in}{2.328146in}}%
\pgfpathlineto{\pgfqpoint{5.419518in}{2.429561in}}%
\pgfpathlineto{\pgfqpoint{5.420294in}{2.369167in}}%
\pgfpathlineto{\pgfqpoint{5.421155in}{2.445981in}}%
\pgfpathlineto{\pgfqpoint{5.420897in}{2.328507in}}%
\pgfpathlineto{\pgfqpoint{5.421389in}{2.362782in}}%
\pgfpathlineto{\pgfqpoint{5.422127in}{2.319064in}}%
\pgfpathlineto{\pgfqpoint{5.421980in}{2.443789in}}%
\pgfpathlineto{\pgfqpoint{5.422497in}{2.367679in}}%
\pgfpathlineto{\pgfqpoint{5.422952in}{2.318294in}}%
\pgfpathlineto{\pgfqpoint{5.422792in}{2.432886in}}%
\pgfpathlineto{\pgfqpoint{5.423580in}{2.370418in}}%
\pgfpathlineto{\pgfqpoint{5.424441in}{2.431004in}}%
\pgfpathlineto{\pgfqpoint{5.423777in}{2.331169in}}%
\pgfpathlineto{\pgfqpoint{5.424675in}{2.369586in}}%
\pgfpathlineto{\pgfqpoint{5.425414in}{2.328428in}}%
\pgfpathlineto{\pgfqpoint{5.425266in}{2.437351in}}%
\pgfpathlineto{\pgfqpoint{5.425783in}{2.361532in}}%
\pgfpathlineto{\pgfqpoint{5.426238in}{2.321571in}}%
\pgfpathlineto{\pgfqpoint{5.426078in}{2.440879in}}%
\pgfpathlineto{\pgfqpoint{5.426854in}{2.366199in}}%
\pgfpathlineto{\pgfqpoint{5.426903in}{2.439283in}}%
\pgfpathlineto{\pgfqpoint{5.427050in}{2.318432in}}%
\pgfpathlineto{\pgfqpoint{5.427949in}{2.379425in}}%
\pgfpathlineto{\pgfqpoint{5.428700in}{2.327340in}}%
\pgfpathlineto{\pgfqpoint{5.428552in}{2.434848in}}%
\pgfpathlineto{\pgfqpoint{5.429069in}{2.355541in}}%
\pgfpathlineto{\pgfqpoint{5.429524in}{2.319050in}}%
\pgfpathlineto{\pgfqpoint{5.429364in}{2.441852in}}%
\pgfpathlineto{\pgfqpoint{5.430140in}{2.365205in}}%
\pgfpathlineto{\pgfqpoint{5.430189in}{2.445616in}}%
\pgfpathlineto{\pgfqpoint{5.430349in}{2.315075in}}%
\pgfpathlineto{\pgfqpoint{5.431235in}{2.373024in}}%
\pgfpathlineto{\pgfqpoint{5.431986in}{2.315313in}}%
\pgfpathlineto{\pgfqpoint{5.431838in}{2.434534in}}%
\pgfpathlineto{\pgfqpoint{5.432355in}{2.359733in}}%
\pgfpathlineto{\pgfqpoint{5.432810in}{2.315126in}}%
\pgfpathlineto{\pgfqpoint{5.432663in}{2.435505in}}%
\pgfpathlineto{\pgfqpoint{5.433426in}{2.356574in}}%
\pgfpathlineto{\pgfqpoint{5.434312in}{2.444386in}}%
\pgfpathlineto{\pgfqpoint{5.433635in}{2.306414in}}%
\pgfpathlineto{\pgfqpoint{5.434534in}{2.373459in}}%
\pgfpathlineto{\pgfqpoint{5.435284in}{2.309117in}}%
\pgfpathlineto{\pgfqpoint{5.435137in}{2.440215in}}%
\pgfpathlineto{\pgfqpoint{5.435641in}{2.362319in}}%
\pgfpathlineto{\pgfqpoint{5.436786in}{2.445263in}}%
\pgfpathlineto{\pgfqpoint{5.436109in}{2.312636in}}%
\pgfpathlineto{\pgfqpoint{5.436835in}{2.397920in}}%
\pgfpathlineto{\pgfqpoint{5.436946in}{2.306395in}}%
\pgfpathlineto{\pgfqpoint{5.437610in}{2.451995in}}%
\pgfpathlineto{\pgfqpoint{5.437906in}{2.394955in}}%
\pgfpathlineto{\pgfqpoint{5.438435in}{2.444815in}}%
\pgfpathlineto{\pgfqpoint{5.438090in}{2.336421in}}%
\pgfpathlineto{\pgfqpoint{5.438952in}{2.354498in}}%
\pgfpathlineto{\pgfqpoint{5.439272in}{2.424881in}}%
\pgfpathlineto{\pgfqpoint{5.439654in}{2.350976in}}%
\pgfpathlineto{\pgfqpoint{5.440146in}{2.373400in}}%
\pgfpathlineto{\pgfqpoint{5.441217in}{2.330962in}}%
\pgfpathlineto{\pgfqpoint{5.441007in}{2.414592in}}%
\pgfpathlineto{\pgfqpoint{5.441253in}{2.360287in}}%
\pgfpathlineto{\pgfqpoint{5.441783in}{2.419261in}}%
\pgfpathlineto{\pgfqpoint{5.441930in}{2.317761in}}%
\pgfpathlineto{\pgfqpoint{5.442398in}{2.391941in}}%
\pgfpathlineto{\pgfqpoint{5.442570in}{2.413509in}}%
\pgfpathlineto{\pgfqpoint{5.442718in}{2.337818in}}%
\pgfpathlineto{\pgfqpoint{5.442755in}{2.310983in}}%
\pgfpathlineto{\pgfqpoint{5.443186in}{2.405171in}}%
\pgfpathlineto{\pgfqpoint{5.443764in}{2.375934in}}%
\pgfpathlineto{\pgfqpoint{5.444109in}{2.413798in}}%
\pgfpathlineto{\pgfqpoint{5.444318in}{2.326980in}}%
\pgfpathlineto{\pgfqpoint{5.444921in}{2.388976in}}%
\pgfpathlineto{\pgfqpoint{5.445844in}{2.334090in}}%
\pgfpathlineto{\pgfqpoint{5.445660in}{2.414957in}}%
\pgfpathlineto{\pgfqpoint{5.446078in}{2.360221in}}%
\pgfpathlineto{\pgfqpoint{5.446435in}{2.414166in}}%
\pgfpathlineto{\pgfqpoint{5.446632in}{2.329714in}}%
\pgfpathlineto{\pgfqpoint{5.447247in}{2.393721in}}%
\pgfpathlineto{\pgfqpoint{5.448183in}{2.332490in}}%
\pgfpathlineto{\pgfqpoint{5.447838in}{2.407202in}}%
\pgfpathlineto{\pgfqpoint{5.448404in}{2.369078in}}%
\pgfpathlineto{\pgfqpoint{5.448453in}{2.353073in}}%
\pgfpathlineto{\pgfqpoint{5.448613in}{2.397519in}}%
\pgfpathlineto{\pgfqpoint{5.448737in}{2.394418in}}%
\pgfpathlineto{\pgfqpoint{5.449537in}{2.405948in}}%
\pgfpathlineto{\pgfqpoint{5.448970in}{2.339140in}}%
\pgfpathlineto{\pgfqpoint{5.449795in}{2.370192in}}%
\pgfpathlineto{\pgfqpoint{5.450533in}{2.344188in}}%
\pgfpathlineto{\pgfqpoint{5.450324in}{2.412749in}}%
\pgfpathlineto{\pgfqpoint{5.450878in}{2.373694in}}%
\pgfpathlineto{\pgfqpoint{5.451100in}{2.408022in}}%
\pgfpathlineto{\pgfqpoint{5.451309in}{2.339324in}}%
\pgfpathlineto{\pgfqpoint{5.451998in}{2.376977in}}%
\pgfpathlineto{\pgfqpoint{5.452097in}{2.326926in}}%
\pgfpathlineto{\pgfqpoint{5.453007in}{2.405812in}}%
\pgfpathlineto{\pgfqpoint{5.453130in}{2.358329in}}%
\pgfpathlineto{\pgfqpoint{5.453167in}{2.380862in}}%
\pgfpathlineto{\pgfqpoint{5.453807in}{2.422740in}}%
\pgfpathlineto{\pgfqpoint{5.453660in}{2.341035in}}%
\pgfpathlineto{\pgfqpoint{5.454250in}{2.374159in}}%
\pgfpathlineto{\pgfqpoint{5.454940in}{2.346764in}}%
\pgfpathlineto{\pgfqpoint{5.454595in}{2.412923in}}%
\pgfpathlineto{\pgfqpoint{5.455321in}{2.379851in}}%
\pgfpathlineto{\pgfqpoint{5.456441in}{2.413357in}}%
\pgfpathlineto{\pgfqpoint{5.456035in}{2.350301in}}%
\pgfpathlineto{\pgfqpoint{5.456466in}{2.397648in}}%
\pgfpathlineto{\pgfqpoint{5.457512in}{2.338827in}}%
\pgfpathlineto{\pgfqpoint{5.457229in}{2.421872in}}%
\pgfpathlineto{\pgfqpoint{5.457610in}{2.351813in}}%
\pgfpathlineto{\pgfqpoint{5.458780in}{2.439796in}}%
\pgfpathlineto{\pgfqpoint{5.458312in}{2.335008in}}%
\pgfpathlineto{\pgfqpoint{5.458829in}{2.392261in}}%
\pgfpathlineto{\pgfqpoint{5.459173in}{2.335965in}}%
\pgfpathlineto{\pgfqpoint{5.459567in}{2.434287in}}%
\pgfpathlineto{\pgfqpoint{5.459986in}{2.362014in}}%
\pgfpathlineto{\pgfqpoint{5.460355in}{2.427384in}}%
\pgfpathlineto{\pgfqpoint{5.460232in}{2.333324in}}%
\pgfpathlineto{\pgfqpoint{5.461167in}{2.418277in}}%
\pgfpathlineto{\pgfqpoint{5.461807in}{2.322977in}}%
\pgfpathlineto{\pgfqpoint{5.461943in}{2.430883in}}%
\pgfpathlineto{\pgfqpoint{5.462620in}{2.334769in}}%
\pgfpathlineto{\pgfqpoint{5.463518in}{2.433461in}}%
\pgfpathlineto{\pgfqpoint{5.463395in}{2.333952in}}%
\pgfpathlineto{\pgfqpoint{5.463752in}{2.381247in}}%
\pgfpathlineto{\pgfqpoint{5.464183in}{2.341422in}}%
\pgfpathlineto{\pgfqpoint{5.464318in}{2.432291in}}%
\pgfpathlineto{\pgfqpoint{5.464884in}{2.365466in}}%
\pgfpathlineto{\pgfqpoint{5.465770in}{2.338265in}}%
\pgfpathlineto{\pgfqpoint{5.465106in}{2.419452in}}%
\pgfpathlineto{\pgfqpoint{5.465856in}{2.381049in}}%
\pgfpathlineto{\pgfqpoint{5.465893in}{2.408362in}}%
\pgfpathlineto{\pgfqpoint{5.466546in}{2.334519in}}%
\pgfpathlineto{\pgfqpoint{5.466976in}{2.390843in}}%
\pgfpathlineto{\pgfqpoint{5.467346in}{2.328314in}}%
\pgfpathlineto{\pgfqpoint{5.467715in}{2.408719in}}%
\pgfpathlineto{\pgfqpoint{5.468146in}{2.346531in}}%
\pgfpathlineto{\pgfqpoint{5.468478in}{2.407407in}}%
\pgfpathlineto{\pgfqpoint{5.468909in}{2.338498in}}%
\pgfpathlineto{\pgfqpoint{5.469303in}{2.397795in}}%
\pgfpathlineto{\pgfqpoint{5.469709in}{2.341214in}}%
\pgfpathlineto{\pgfqpoint{5.470066in}{2.400528in}}%
\pgfpathlineto{\pgfqpoint{5.470533in}{2.359821in}}%
\pgfpathlineto{\pgfqpoint{5.471063in}{2.394095in}}%
\pgfpathlineto{\pgfqpoint{5.471284in}{2.346145in}}%
\pgfpathlineto{\pgfqpoint{5.471456in}{2.358534in}}%
\pgfpathlineto{\pgfqpoint{5.471469in}{2.344288in}}%
\pgfpathlineto{\pgfqpoint{5.472380in}{2.399098in}}%
\pgfpathlineto{\pgfqpoint{5.472540in}{2.387549in}}%
\pgfpathlineto{\pgfqpoint{5.473044in}{2.337032in}}%
\pgfpathlineto{\pgfqpoint{5.473155in}{2.395487in}}%
\pgfpathlineto{\pgfqpoint{5.473660in}{2.376065in}}%
\pgfpathlineto{\pgfqpoint{5.474386in}{2.398509in}}%
\pgfpathlineto{\pgfqpoint{5.473832in}{2.338491in}}%
\pgfpathlineto{\pgfqpoint{5.474743in}{2.366809in}}%
\pgfpathlineto{\pgfqpoint{5.475420in}{2.344710in}}%
\pgfpathlineto{\pgfqpoint{5.475173in}{2.399252in}}%
\pgfpathlineto{\pgfqpoint{5.475740in}{2.377060in}}%
\pgfpathlineto{\pgfqpoint{5.476010in}{2.404311in}}%
\pgfpathlineto{\pgfqpoint{5.476404in}{2.337485in}}%
\pgfpathlineto{\pgfqpoint{5.476884in}{2.393827in}}%
\pgfpathlineto{\pgfqpoint{5.477955in}{2.334737in}}%
\pgfpathlineto{\pgfqpoint{5.477672in}{2.404970in}}%
\pgfpathlineto{\pgfqpoint{5.478066in}{2.374561in}}%
\pgfpathlineto{\pgfqpoint{5.478336in}{2.409834in}}%
\pgfpathlineto{\pgfqpoint{5.478767in}{2.330631in}}%
\pgfpathlineto{\pgfqpoint{5.479210in}{2.399078in}}%
\pgfpathlineto{\pgfqpoint{5.479223in}{2.399847in}}%
\pgfpathlineto{\pgfqpoint{5.479469in}{2.356329in}}%
\pgfpathlineto{\pgfqpoint{5.480330in}{2.322815in}}%
\pgfpathlineto{\pgfqpoint{5.479998in}{2.408801in}}%
\pgfpathlineto{\pgfqpoint{5.480552in}{2.369918in}}%
\pgfpathlineto{\pgfqpoint{5.481672in}{2.415536in}}%
\pgfpathlineto{\pgfqpoint{5.481106in}{2.324848in}}%
\pgfpathlineto{\pgfqpoint{5.481696in}{2.396552in}}%
\pgfpathlineto{\pgfqpoint{5.482644in}{2.316016in}}%
\pgfpathlineto{\pgfqpoint{5.482447in}{2.416203in}}%
\pgfpathlineto{\pgfqpoint{5.482804in}{2.399024in}}%
\pgfpathlineto{\pgfqpoint{5.483210in}{2.412414in}}%
\pgfpathlineto{\pgfqpoint{5.483419in}{2.330795in}}%
\pgfpathlineto{\pgfqpoint{5.483789in}{2.363247in}}%
\pgfpathlineto{\pgfqpoint{5.484195in}{2.328660in}}%
\pgfpathlineto{\pgfqpoint{5.484503in}{2.406787in}}%
\pgfpathlineto{\pgfqpoint{5.484712in}{2.382173in}}%
\pgfpathlineto{\pgfqpoint{5.485512in}{2.421011in}}%
\pgfpathlineto{\pgfqpoint{5.485647in}{2.325717in}}%
\pgfpathlineto{\pgfqpoint{5.485807in}{2.369283in}}%
\pgfpathlineto{\pgfqpoint{5.486287in}{2.426623in}}%
\pgfpathlineto{\pgfqpoint{5.486410in}{2.333891in}}%
\pgfpathlineto{\pgfqpoint{5.486459in}{2.318704in}}%
\pgfpathlineto{\pgfqpoint{5.486743in}{2.414097in}}%
\pgfpathlineto{\pgfqpoint{5.487395in}{2.393156in}}%
\pgfpathlineto{\pgfqpoint{5.488023in}{2.327757in}}%
\pgfpathlineto{\pgfqpoint{5.487530in}{2.419027in}}%
\pgfpathlineto{\pgfqpoint{5.488256in}{2.401489in}}%
\pgfpathlineto{\pgfqpoint{5.488355in}{2.428075in}}%
\pgfpathlineto{\pgfqpoint{5.488724in}{2.325846in}}%
\pgfpathlineto{\pgfqpoint{5.489290in}{2.361641in}}%
\pgfpathlineto{\pgfqpoint{5.489524in}{2.313478in}}%
\pgfpathlineto{\pgfqpoint{5.489819in}{2.413798in}}%
\pgfpathlineto{\pgfqpoint{5.490398in}{2.350973in}}%
\pgfpathlineto{\pgfqpoint{5.491506in}{2.413184in}}%
\pgfpathlineto{\pgfqpoint{5.491149in}{2.318661in}}%
\pgfpathlineto{\pgfqpoint{5.491567in}{2.401812in}}%
\pgfpathlineto{\pgfqpoint{5.492699in}{2.312364in}}%
\pgfpathlineto{\pgfqpoint{5.492281in}{2.415394in}}%
\pgfpathlineto{\pgfqpoint{5.492736in}{2.347356in}}%
\pgfpathlineto{\pgfqpoint{5.492773in}{2.339159in}}%
\pgfpathlineto{\pgfqpoint{5.492810in}{2.377868in}}%
\pgfpathlineto{\pgfqpoint{5.493819in}{2.412426in}}%
\pgfpathlineto{\pgfqpoint{5.493475in}{2.313011in}}%
\pgfpathlineto{\pgfqpoint{5.493918in}{2.384559in}}%
\pgfpathlineto{\pgfqpoint{5.494263in}{2.311092in}}%
\pgfpathlineto{\pgfqpoint{5.494607in}{2.419423in}}%
\pgfpathlineto{\pgfqpoint{5.495063in}{2.327594in}}%
\pgfpathlineto{\pgfqpoint{5.495432in}{2.415826in}}%
\pgfpathlineto{\pgfqpoint{5.495826in}{2.318811in}}%
\pgfpathlineto{\pgfqpoint{5.496244in}{2.409836in}}%
\pgfpathlineto{\pgfqpoint{5.496613in}{2.319414in}}%
\pgfpathlineto{\pgfqpoint{5.496995in}{2.410335in}}%
\pgfpathlineto{\pgfqpoint{5.497438in}{2.359059in}}%
\pgfpathlineto{\pgfqpoint{5.497795in}{2.411401in}}%
\pgfpathlineto{\pgfqpoint{5.498115in}{2.330988in}}%
\pgfpathlineto{\pgfqpoint{5.498595in}{2.401697in}}%
\pgfpathlineto{\pgfqpoint{5.498903in}{2.327713in}}%
\pgfpathlineto{\pgfqpoint{5.499346in}{2.404379in}}%
\pgfpathlineto{\pgfqpoint{5.499801in}{2.370328in}}%
\pgfpathlineto{\pgfqpoint{5.500158in}{2.416342in}}%
\pgfpathlineto{\pgfqpoint{5.500478in}{2.344878in}}%
\pgfpathlineto{\pgfqpoint{5.500958in}{2.413191in}}%
\pgfpathlineto{\pgfqpoint{5.501253in}{2.347430in}}%
\pgfpathlineto{\pgfqpoint{5.502115in}{2.372954in}}%
\pgfpathlineto{\pgfqpoint{5.502398in}{2.347322in}}%
\pgfpathlineto{\pgfqpoint{5.502521in}{2.410792in}}%
\pgfpathlineto{\pgfqpoint{5.503210in}{2.370070in}}%
\pgfpathlineto{\pgfqpoint{5.503309in}{2.410096in}}%
\pgfpathlineto{\pgfqpoint{5.503973in}{2.352229in}}%
\pgfpathlineto{\pgfqpoint{5.504318in}{2.377412in}}%
\pgfpathlineto{\pgfqpoint{5.504773in}{2.351654in}}%
\pgfpathlineto{\pgfqpoint{5.504896in}{2.404307in}}%
\pgfpathlineto{\pgfqpoint{5.505413in}{2.380978in}}%
\pgfpathlineto{\pgfqpoint{5.506472in}{2.407749in}}%
\pgfpathlineto{\pgfqpoint{5.506275in}{2.352832in}}%
\pgfpathlineto{\pgfqpoint{5.506509in}{2.385855in}}%
\pgfpathlineto{\pgfqpoint{5.507136in}{2.356601in}}%
\pgfpathlineto{\pgfqpoint{5.507259in}{2.408199in}}%
\pgfpathlineto{\pgfqpoint{5.507616in}{2.374933in}}%
\pgfpathlineto{\pgfqpoint{5.508047in}{2.404044in}}%
\pgfpathlineto{\pgfqpoint{5.508121in}{2.352090in}}%
\pgfpathlineto{\pgfqpoint{5.508687in}{2.368562in}}%
\pgfpathlineto{\pgfqpoint{5.509499in}{2.352591in}}%
\pgfpathlineto{\pgfqpoint{5.508847in}{2.401541in}}%
\pgfpathlineto{\pgfqpoint{5.509770in}{2.372265in}}%
\pgfpathlineto{\pgfqpoint{5.510386in}{2.397381in}}%
\pgfpathlineto{\pgfqpoint{5.510730in}{2.352008in}}%
\pgfpathlineto{\pgfqpoint{5.510853in}{2.365851in}}%
\pgfpathlineto{\pgfqpoint{5.511272in}{2.348005in}}%
\pgfpathlineto{\pgfqpoint{5.511173in}{2.399207in}}%
\pgfpathlineto{\pgfqpoint{5.511912in}{2.394781in}}%
\pgfpathlineto{\pgfqpoint{5.512059in}{2.346341in}}%
\pgfpathlineto{\pgfqpoint{5.512736in}{2.403958in}}%
\pgfpathlineto{\pgfqpoint{5.513204in}{2.366162in}}%
\pgfpathlineto{\pgfqpoint{5.513524in}{2.406896in}}%
\pgfpathlineto{\pgfqpoint{5.513622in}{2.348719in}}%
\pgfpathlineto{\pgfqpoint{5.514373in}{2.383979in}}%
\pgfpathlineto{\pgfqpoint{5.515099in}{2.399723in}}%
\pgfpathlineto{\pgfqpoint{5.515506in}{2.357557in}}%
\pgfpathlineto{\pgfqpoint{5.515875in}{2.396983in}}%
\pgfpathlineto{\pgfqpoint{5.516466in}{2.347038in}}%
\pgfpathlineto{\pgfqpoint{5.516724in}{2.384415in}}%
\pgfpathlineto{\pgfqpoint{5.517253in}{2.355923in}}%
\pgfpathlineto{\pgfqpoint{5.516872in}{2.397614in}}%
\pgfpathlineto{\pgfqpoint{5.517869in}{2.369662in}}%
\pgfpathlineto{\pgfqpoint{5.518422in}{2.394005in}}%
\pgfpathlineto{\pgfqpoint{5.518546in}{2.355617in}}%
\pgfpathlineto{\pgfqpoint{5.518989in}{2.383764in}}%
\pgfpathlineto{\pgfqpoint{5.519038in}{2.389784in}}%
\pgfpathlineto{\pgfqpoint{5.519050in}{2.383934in}}%
\pgfpathlineto{\pgfqpoint{5.520096in}{2.358969in}}%
\pgfpathlineto{\pgfqpoint{5.519813in}{2.391566in}}%
\pgfpathlineto{\pgfqpoint{5.520182in}{2.362545in}}%
\pgfpathlineto{\pgfqpoint{5.520872in}{2.357284in}}%
\pgfpathlineto{\pgfqpoint{5.521364in}{2.388988in}}%
\pgfpathlineto{\pgfqpoint{5.522435in}{2.358067in}}%
\pgfpathlineto{\pgfqpoint{5.522595in}{2.382283in}}%
\pgfpathlineto{\pgfqpoint{5.523419in}{2.393488in}}%
\pgfpathlineto{\pgfqpoint{5.523210in}{2.357148in}}%
\pgfpathlineto{\pgfqpoint{5.523715in}{2.388107in}}%
\pgfpathlineto{\pgfqpoint{5.523998in}{2.357960in}}%
\pgfpathlineto{\pgfqpoint{5.524195in}{2.392870in}}%
\pgfpathlineto{\pgfqpoint{5.524872in}{2.366687in}}%
\pgfpathlineto{\pgfqpoint{5.524982in}{2.391800in}}%
\pgfpathlineto{\pgfqpoint{5.525549in}{2.362924in}}%
\pgfpathlineto{\pgfqpoint{5.525992in}{2.374229in}}%
\pgfpathlineto{\pgfqpoint{5.526545in}{2.390265in}}%
\pgfpathlineto{\pgfqpoint{5.526336in}{2.364657in}}%
\pgfpathlineto{\pgfqpoint{5.526878in}{2.371923in}}%
\pgfpathlineto{\pgfqpoint{5.527173in}{2.365879in}}%
\pgfpathlineto{\pgfqpoint{5.527321in}{2.389180in}}%
\pgfpathlineto{\pgfqpoint{5.527973in}{2.370155in}}%
\pgfpathlineto{\pgfqpoint{5.528096in}{2.388291in}}%
\pgfpathlineto{\pgfqpoint{5.528724in}{2.368974in}}%
\pgfpathlineto{\pgfqpoint{5.529093in}{2.373539in}}%
\pgfpathlineto{\pgfqpoint{5.529499in}{2.369683in}}%
\pgfpathlineto{\pgfqpoint{5.529647in}{2.385358in}}%
\pgfpathlineto{\pgfqpoint{5.530152in}{2.374796in}}%
\pgfpathlineto{\pgfqpoint{5.530422in}{2.384444in}}%
\pgfpathlineto{\pgfqpoint{5.530275in}{2.371129in}}%
\pgfpathlineto{\pgfqpoint{5.531382in}{2.378437in}}%
\pgfpathlineto{\pgfqpoint{5.531665in}{2.371403in}}%
\pgfpathlineto{\pgfqpoint{5.531973in}{2.382610in}}%
\pgfpathlineto{\pgfqpoint{5.532527in}{2.374341in}}%
\pgfpathlineto{\pgfqpoint{5.533130in}{2.373406in}}%
\pgfpathlineto{\pgfqpoint{5.532749in}{2.381335in}}%
\pgfpathlineto{\pgfqpoint{5.533475in}{2.377748in}}%
\pgfpathlineto{\pgfqpoint{5.533536in}{2.378991in}}%
\pgfpathlineto{\pgfqpoint{5.533893in}{2.374699in}}%
\pgfpathlineto{\pgfqpoint{5.534484in}{2.375928in}}%
\pgfpathlineto{\pgfqpoint{5.534545in}{2.375992in}}%
\pgfpathlineto{\pgfqpoint{5.534545in}{2.375992in}}%
\pgfusepath{stroke}%
\end{pgfscope}%
\begin{pgfscope}%
\pgfsetrectcap%
\pgfsetmiterjoin%
\pgfsetlinewidth{0.803000pt}%
\definecolor{currentstroke}{rgb}{0.000000,0.000000,0.000000}%
\pgfsetstrokecolor{currentstroke}%
\pgfsetdash{}{0pt}%
\pgfpathmoveto{\pgfqpoint{0.800000in}{0.528000in}}%
\pgfpathlineto{\pgfqpoint{0.800000in}{4.224000in}}%
\pgfusepath{stroke}%
\end{pgfscope}%
\begin{pgfscope}%
\pgfsetrectcap%
\pgfsetmiterjoin%
\pgfsetlinewidth{0.803000pt}%
\definecolor{currentstroke}{rgb}{0.000000,0.000000,0.000000}%
\pgfsetstrokecolor{currentstroke}%
\pgfsetdash{}{0pt}%
\pgfpathmoveto{\pgfqpoint{5.760000in}{0.528000in}}%
\pgfpathlineto{\pgfqpoint{5.760000in}{4.224000in}}%
\pgfusepath{stroke}%
\end{pgfscope}%
\begin{pgfscope}%
\pgfsetrectcap%
\pgfsetmiterjoin%
\pgfsetlinewidth{0.803000pt}%
\definecolor{currentstroke}{rgb}{0.000000,0.000000,0.000000}%
\pgfsetstrokecolor{currentstroke}%
\pgfsetdash{}{0pt}%
\pgfpathmoveto{\pgfqpoint{0.800000in}{0.528000in}}%
\pgfpathlineto{\pgfqpoint{5.760000in}{0.528000in}}%
\pgfusepath{stroke}%
\end{pgfscope}%
\begin{pgfscope}%
\pgfsetrectcap%
\pgfsetmiterjoin%
\pgfsetlinewidth{0.803000pt}%
\definecolor{currentstroke}{rgb}{0.000000,0.000000,0.000000}%
\pgfsetstrokecolor{currentstroke}%
\pgfsetdash{}{0pt}%
\pgfpathmoveto{\pgfqpoint{0.800000in}{4.224000in}}%
\pgfpathlineto{\pgfqpoint{5.760000in}{4.224000in}}%
\pgfusepath{stroke}%
\end{pgfscope}%
\end{pgfpicture}%
\makeatother%
\endgroup%
}
    \end{center}
    \caption{A PGF histogram from \texttt{matplotlib}.}
\end{figure}



\section{Misure dell'ampiezza}



Intuitivamente, osservare quanto il tracciato del grafico si distanzia dall'asse orizzontale dà un'idea approssimativa di ``quanto forte'' è il suono, o della sua dinamica (non uso il termine ``intensità'', che ha un significato specifico definito più avanti). È utile definire alcune maniere diverse di misurare l'ampiezza di un fenomeno sonoro:

\subsection{Ampiezza istantanea}

L'\emph{ampiezza istantanea} è l'ampiezza misurata a un istante specifico del fenomeno sonoro. È un valore reale%
\footnote{I numeri reali, appartenenti all'insieme $\mathbb{R}$, sono tutti i numeri rappresentabili su una retta. Per chi non ha mai incontrato i numeri complessi, appartenenti all'insieme $\mathbb{C}$ e rappresentabili su un piano, l'insieme $\mathbb{R}$ è, intuitivamente, l'insieme di \emph{tutti i numeri}. Purtroppo o per fortuna, la matematica moderna è un po' più complicata di così. Dovremo accennare altre volte a $\mathbb{C}$, perché i numeri complessi sono molto utili per rappresentare certi aspetti dei segnali audio.}
\emph{segnato}%
\footnote{Cioè, potenzialmente negativo.}
: convenzionalmente, è positiva se rappresenta una pressione nella direzione nella direzione dall'emittente al recettore; negativa se rappresenta una pressione nella direzione dal recettore al ricevente (cioè, dal punto di vista del ricevente, una depressione). 

Le ragioni per la scelta dell'istante a cui compiere la misurazione possono essere varie, ma in generale l'ampiezza istantanea viene misurata ripetutamente, a intervalli di tempo regolari, nelle operazioni di campionamento del segnale: se le misure sono abbastanza frequenti, la successione delle ampiezze istantanee può costituire una buona approssimazione numerica dell'andamento del segnale. Di fatto, virtualmente tutte le tecniche di campionamento digitale del segnale audio sono basate su questo principio. 

Di per sé, l'ampiezza istantanea non ci dà un'informazione affidabile sul livello dinamico percepito, perché qualsiasi fenomeno sonoro è costituito dalla rapida alternanza di pressioni relativamente alte (cioè pressioni che, in una direzione o nell'altra, deviano in maniera importante dalla pressione atmosferica media) e pressioni relativamente basse (cioè pressioni prossime alla pressione atmosferica media). 


\subsection{Ampiezza di picco}

L'\emph{ampiezza di picco} è riferita non a un istante ma a un intervallo di tempo, ed è l'ampiezza di valore assoluto massimo entro tale intervallo. Se l'intervallo è ragionevolmente ampio, può costituire una prima stima del livello dinamico percepito. L'ampiezza di picco è, in linea di principio, un valore segnato, per le stesse ragioni per cui lo è l'ampiezza istantanea: è, a tutti gli effetti, l'ampiezza istantanea rilevata all'istante in cui questa è massima in valore assoluto.

Un caso tipico in cui è importante conoscere l'ampiezza di picco è la regolazione del guadagno in un sistema di registrazione: si producono segnali forti e si regola il livello di premplificazione in maniera che l'ampiezza di picco ricevuta dal registratore (o, in un sistema digitale, dal convertitore) non ecceda il livello massimo che questo può trattare. In questo caso, come in molti altri, non è interessante sapere in quale direzione è stata rilevata la massima ampiezza: per questa ragione, viene normalmente considerato il valore assoluto dell'ampiezza di picco. 

L'ampiezza di picco può essere raggiunta più volte nell'intervallo di tempo considerato, anche con segni diversi: per esempio, una sinusoide correttamente simmetrica rispetto allo 0 raggiunge la sua ampiezza di picco due volte per ogni ciclo, una volta dal lato positivo e una dal lato negativo.



\subsection{Ampiezza picco-picco}

L'\emph{ampiezza picco-picco} è il valore assoluto della differenza tra la massima e la minima ampiezza entro un dato intervallo di tempo. In questo caso, ``massima'' e ``minima'' vanno intese in senso numerico, considerando il segno, e non in valore assoluto: cioè, le ampiezze negative saranno tutte considerate più piccole delle ampiezze positive (e quindi, per esempio, un'ampiezza di -0.5 è considerata più piccola di un'ampiezza di 1; e un'ampiezza di 0.1 è considerata più grande di una di -0.5). 

Intuitivamente, l'ampiezza picco-picco calcola la distanza tra i valori estremi di ampiezza istantanea che il segnale tocca entro l'intervallo temporale. Nel caso di una forma d'onda perfettamente centrata sullo 0, essa coincide con il doppio del valore assoluto dell'ampiezza di picco. Per esempio, il segnale nell'intervallo considerato potrebbe avere ampiezza massima a 0.7 e minima a -0.7: entrambi i valori possono essere considerati ampiezza di picco, visto che il loro valore assoluto 0.7 è identico. In questo caso, l'ampiezza picco-picco, cioè il valore assoluto della differenza tra massimo e minimo è $0.7 - (-0.7) = 0.7 + 0.7 = 1.4$. 

Se però la forma d'onda non è centrata sullo 0, o se la posizione dello 0 non è nota con esattezza (per esempio perché esiste una componente di tensione continua),%
\footnote{Una componente di tensione continua provoca una traslazione sull'asse verticale della rappresentazione nel dominio del tempo.}
l'ampiezza picco-picco fornisce l'unica possibile misura affidabile. Ad esempio, se il segnale descritto sopra si trovasse traslato di -0.5 sull'asse verticale, il suo massimo sarebbe $0.7 - 0.5 = 0.2$ e il minimo $-0.7 - 0.5 = -1.2$. Questo però non modificherebbe l'ampiezza picco-picco, che sarebbe $0.2 - (-1.2) = 1.4$. L'ampiezza picco-picco, secondo la definizione data sopra, non è mai negativa. Inoltre, in un segnale bipolare (cioè, un segnale che attraversa la linea dello 0) è sempre maggiore dell'ampiezza di picco.




\subsection{Ampiezza RMS}

L'\emph{ampiezza RMS} (da \emph{root-mean-square}: vedremo tra poco perché) è una misura presa rispetto a un intervallo di tempo, e tiene conto dell'andamento dell'ampiezza nell'intervallo. Di conseguenza, può rappresentare in maniera più accurata il livello dinamico percepito, che a parità di ampiezza di picco (o picco-picco) sarà maggiore se, su scale di durata di frazioni di secondo, il segnale rimane forte più a lungo. L'ampiezza RMS non è mai maggiore dell'ampiezza di picco, e in generale --- salvo i casi di forme d'onda rettangolari e di segnali costanti --- è minore di essa.

La definizione di ampiezza RMS per un segnale campionato in ampiezza a intervalli di tempo regolari, come i segnali digitali che trattiamo abitualmente, è
\begin{equation}\label{eq:rms}
A_{\text{RMS}} = \sqrt{\frac{A_1^2+A_2^2+\ldots+A_n^2}{n}}
\end{equation}
dove $n$ è il numero totale di campioni e $A_1$ ... $A_n$ sono i valori di ampiezza istantanea dei singoli campioni entro la finestra temporale considerata. L'ampiezza RMS è quindi la radice quadrata della media dei quadrati delle ampiezze istantanee, da cui il nome. 

Se consideriamo invece un segnale continuo nel tempo, com'è il caso per segnali 
la definizione di RMS deve fare ricorso al concetto di integrale:%
\footnote{Un integrale può essere visto come la somma di infiniti campioni di durata infinitesimale: ci serve quindi a calcolare il valore totalizzato da un fenomeno continuo. Se non sei familiare con questo formalismo ignora tranquillamente i dettagli di questa formula.}
\begin{equation*}
A_{\textrm{RMS}} = \sqrt{\frac{1}{t_2-t_1}\int_{t_1}^{t_2}[A(t)]^2\textrm{d}t}
\end{equation*}
dove $t$ indica il tempo, $t_1$ e $t_2$ sono rispettivamente il tempo iniziale e finale considerati e $A(t)$ è l'ampiezza nel tempo.

L'ampiezza RMS non è mai negativa.%
\footnote{Volendo essere pignoli dovremmo dire che ogni numero reale positivo ha due radici quadrate, una positiva e una negativa: ma per il calcolo della RMS consideriamo solo quella positiva. D'altra parte, solo numeri reali non negativi hanno una radice quadrata reale: ma la formula della RMS mette sotto radice una somma di quadrati, e il quadrato di qualsiasi numero reale è non negativo. Tutto questo per dire che la RMS è ben definita.}


\section{Intensità sonora}

Dal punto di vista percettivo, più importante del concetto di pressione è quello di \emph{intensità sonora}. Per definirla correttamente abbiamo bisogno di introdurre alcune nozioni:


\subsection{Energia e lavoro}
Energia e lavoro sono due grandezze strettamente correlate, rappresentate dalle stesse unità di misura come il Joule e la caloria. Il lavoro è definito come la forza che sposta un corpo moltiplicata per lo spostamento effettuato:
\begin{equation}
W = F \cdot s
\end{equation}
L'energia può essere vista come la capacità di un corpo di svolgere un lavoro: si misura in termini del lavoro che in virtù di essa un corpo è in grado di svolgere. È una grandezza fisica fondamentale che obbedisce alla legge di conservazione: viene trasferita, non si crea e non si distrugge.%
\footnote{Questa è una semplificazione: nelle reazioni nucleari, la materia viene trasformata in energia; ed è teoricamente possibile trasformare energia in materia. Nella meccanica classica, però, l'energia è sempre conservata.}
In senso proprio, quindi, ciò che viene trasferito da un corpo vibrante alla membrana del nostro timpano è l'energia sonora, non la pressione sonora. In altri termini, il fatto che una pressione sonora si applichi al timpano è concettualmente una conseguenza del trasferimento di energia, e non viceversa.

L'energia è proporzionale al quadrato della pressione sonora e, di conseguenza, al quadrato dell'ampiezza e di tutti gli altri fenomeni direttamente proporzionali all'ampiezza:%
\footnote{La ragione di questa relazione discende dalle definizioni formali di pressione e di energia. Potresti chiederti: ma \emph{perché} è così? La fisica dovrebbe rispondere che la scienza si occupa del come, non del perché che è invece appannaggio della filosofia; ma qui puoi trovare alcuni tentativi, più o meno fantasiosi, di spiegazioni intuitive: \url{https://languagelog.ldc.upenn.edu/nll/?p=6508}. E, a dire il vero, anche il concetto di \emph{conseguenza}, impiegato in maniera un po' infingarda qualche riga sopra, sarebbe una questione filosofica più che scientifica.}
\begin{equation}
E \propto \Delta p^2 \propto \Delta A^2
\end{equation}



\subsection{Potenza}

La \emph{potenza} (misurata in Watt e indicata qui dalla lettera $P$ maiuscola, da non confondersi con la $p$ minuscola con cui indichiamo la pressione) è una misura della quantità di lavoro svolto, o di energia dispiegata, per unità di tempo: 
\begin{equation}
P = \frac{W}{t}
\end{equation}
A parità di lavoro o di energia, la potenza è maggiore se il tempo è più breve. La stessa quantità di energia sonora dispiegata in un secondo o in un'ora darà luogo a fenomeni sonori di potenza diversa, maggiore nel primo caso (perché il lavoro è concentrato in un tempo più breve), minore nel secondo. A parità di tempo, la potenza è proporzionale all'energia e quindi al quadrato dell'ampiezza:
\begin{equation}
P \propto E \propto \Delta p^2 \propto A^2
\end{equation}



\subsection{Intensità}

L'intensità sonora è una misura della potenza riferita all'area su cui questa è distribuita:
\begin{equation}
I = \frac{P}{a}
\end{equation}

L'intensità sonora non ha una sua specifica unità di misura, ma è misurata in Watt per metro quadrato ($\frac{W}{m^2}$ o $W \cdot m^{-2}$).

Possiamo assumere che il suono, in assenza di ostacoli, si propaghi su un fronte d'onda di forma corrispondente a una superficie sferica a partire dall'emittente, che consideriamo puntiforme. La superficie sferica naturalmente sarà sempre più ampia via via che il fronte d'onda si allontana dalla sorgente, secondo la formula
\begin{equation}
a = 4 \pi r^2
\end{equation}
dove $a$ è l'area della sfera e $r$ il raggio, che in questo caso corrisponde alla distanza tra l'emittente e il fronte d'onda che stiamo considerando. Di conseguenza, possiamo dire che l'energia dispiegata dal fenomeno sonoro in un certo lasso di tempo si distribuisce su una superficie la cui area aumenta in maniera proporzionale al quadrato della distanza. In altri termini, posso pensare l'intensità sonora come la potenza portata dall'onda sonora su una certa superficie, in senso perpendicolare a essa, e a una certa distanza dall'emittente. E dunque, a parità di tutti gli altri parametri, l'intensità sonora è inversamente proporzionale al quadrato della distanza:
\begin{equation}
I \propto \frac{1}{r^2}
\end{equation}

Questa relazione è chiamata \emph{legge dell'inverso del quadrato} e si applica a un vasto insieme di altri fenomeni, tra cui l'intensità luminosa e l'attrazione gravitazionale.

D'altra parte, l'intensità sonora è direttamente proporzionale alla potenza, all'energia e al quadrato della pressione e dell'ampiezza:
\begin{equation}
I \propto P \propto E \propto \Delta p^2 \propto A^2
\end{equation}

Di conseguenza,
\begin{equation}
\frac{1}{r^2} \propto \Delta p^2 \propto A^2
\end{equation}
e quindi
\begin{equation}
\frac{1}{r} \propto \Delta p \propto A
\end{equation}

Questo è un risultato molto importante: pressione, ampiezza e spostamento sono inversamente proporzionali alla distanza, non al suo quadrato. Si dice quindi che seguono la \emph{legge dell'inverso della distanza}.



\section{Riepilogo e classificazione delle grandezze}

Abbiamo quindi definito due categorie di grandezze fisiche nelle quali ci imbatteremo ancora, e sulle quali non dev'esserci confusione:

\subsubsection{Quantità proporzionali alla potenza (\emph{power quantities})}

La prima categoria include l'energia, il lavoro, la potenza e l'intensità sonora. È importante ricordare che da un punto di vista concettuale la quantità fondamentale tra queste è l'energia, che gode del principio di conservazione. L'energia, il lavoro e la potenza non variano con la distanza: la quantità di energia (e quindi di lavoro e di potenza) in gioco è quella trasferita o trasferibile dall'emittente ai corpi circostanti, e rispetto a essa la distanza non è un parametro rilevante. La distanza entra in gioco quando si considera l'intensità sonora, che segue la \emph{legge dell'inverso del quadrato}.


\subsubsection{Quantità proporzionali alla radice quadrata della potenza (\emph{root-power quantities})}

La seconda categoria include la pressione sonora e la tensione elettrica. La potenza è proporzionale al quadrato di queste grandezze: di conseguenza, queste saranno proporzionali alla radice quadrata della potenza, da cui il nome:
\begin{equation}
W \propto \Delta p^2 \implies \Delta p \propto \sqrt{W}
\end{equation}

Si tratta di grandezze a cui non si applica principio di conservazione e che variano in maniera inversamente proporzionale alla distanza. Concettualmente, possiamo considerare i fenomeni rappresentati da queste grandezze come conseguenze del trasferimento di energia. D'altra parte la pressione sonora è il fenomeno che viene direttamente misurato da una membrana di microfono o dal nostro timpano.

L'ampiezza, in senso proprio, non è una grandezza fisica perché non descrive un fenomeno specifico; possiamo però considerarla un'astrazione delle \emph{root-power quantities}, utile quando vogliamo considerare il fenomeno sonoro in maniera indipendente dal fenomeno fisico che l'ha prodotto; questo è particolarmente rilevante nel momento in cui trattiamo il suono dal punto di vista elettroacustico e, di conseguenza, numerico.



\section{Intermezzo: logaritmi --- un ripasso}

Il prossimo concetto che introdurremo richiede uno strumento matematico che dovresti già conoscere, ma che potrebbe meritare un veloce ripasso: il \emph{logaritmo}.

Il logaritmo è definito come l'operazione inversa rispetto all'elevamento a potenza: cioè, se $b^x = y$ allora $\log _b y = x$. Chiamiamo $b$ la \emph{base} del logaritmo.

I logaritmi hanno alcune proprietà molto utili, che sono facilmente dimostrabili ma che mi limiterò a enunciare:

\begin{itemize}

\item Il logaritmo di 1 in qualsiasi base è 0, perché qualsiasi numero elevato a potenza 0 dà 1.

\item Il logaritmo di qualsiasi numero maggiore di 1 è positivo in qualsiasi base; il logaritmo di qualsiasi numero maggiore di 0 e minore di 1 è negativo in qualsiasi base.

\item Il logaritmo di 0 e di numeri negativi non esiste.

\item La somma di due logaritmi di uguale base è uguale al logaritmo (nella stessa base) del prodotto: $\log _b x + \log _b y = \log _b (xy)$. Questa identità rende tra l'altro evidente il fatto che il logaritmo non è una funzione lineare. 

\item È possibile trasformare un esponente dentro il logaritmo in un prodotto fuori dal logaritmo: $\log _b x^y = y \log _b x$. Queste due ultime affermazioni possono essere intese intuitivamente pensando che il logaritmo ci permette di ``scendere di livello'' con le operazioni aritmetiche: trasforma l'elevamento a potenza in moltiplicazione e la moltiplicazione in addizione. Per lungo tempo, questa semplificazione dei calcoli è stata la prima ragione per usare i logaritmi.

\item I logaritmi in tutte le basi sono direttamente proporzionali tra loro: cioè, è possibile passare da una base all'altra applicando al logaritmo un coefficiente --- più specificatamente, $\log _b x = \frac{\log _c b}{\log _c a}$, dove $c$ è una base scelta arbitrariamente. Questa formula, detta \emph{formula del cambio di base}, ci dice tra l'altro che se conosciamo i logaritmi in una base qualsiasi possiamo calcolare logaritmi in qualsiasi altra base.

\item Il logaritmo ci dà una misura dei rapporti tra numeri: la differenza dei logaritmi di coppie di numeri in rapporto 2:1 sarà sempre la stessa (il valore esatto della differenza dipende dalla base del logaritmo). Quindi $\frac{x}{y} = \frac{z}{w} \implies \log _b x - \log _b y = \log _b z - \log _b w$. Per esempio, in base 10 un rapporto di 2:1 corrisponderà a una differenza tra i logaritmi di circa 0.301 (incontreremo di nuovo molto presto questo valore). 

\item Il logaritmo del reciproco di un numero è il negativo del logaritmo di quel numero: $log _b \frac{1}{x} = -log _b x$.

\item Il logaritmo in base 10 ci dà una misura della ``lunghezza tipografica'' di un numero intero: $log_{10}10 = 1$, $log_{10}100 = 2$, $log_{10}1000 = 3$ e così via; per argomenti intermedi avremo valori di logaritmo intermedi (ma, ovviamente, non \emph{linearmente} intermedi: $log_{10}55 \approx 1,74$, che è molto più vicino a $log_{10}100$ che a $log_{10}10$ anche se 55 è la media esatta tra 10 e 100). Generalizzando ulteriormente, il logaritmo in base $b$ ci dà una misura della ``lunghezza tipografica'' di un numero rappresentato in base di numerazione $b$.

\end{itemize}




\section{Decibel}

Le pressioni (o, dovremmo dire, i $\Delta p$) che possiamo percepire come fenomeni sonori --- abbastanza forti da essere udite, non così forti da danneggiare le nostre orecchie --- si situano approssimativamente in un ambito compreso tra i \qty{20}{\micro\pascal} e i \qty{20}{\Pa}. La nostra scala percettiva della dinamica, però, è tutt'altro che lineare rispetto all'ambito di pressione così misurato; la distanza percettiva tra un suono A di pressione \qty{100}{\micro\pascal} e un suono B di \qty{200}{\micro\pascal} sarà analoga a quella tra un suono C di pressione \qty{1000}{\micro\pascal} e un suono D di \qty{1100}{\micro\pascal}, nonostante la differenza tra A e B sia la stessa che tra C e D, e cioè di \qty{100}{\micro\pascal}.

Invece, la distanza percettiva tra un suono A di pressione \qty{100}{\micro\pascal} e un suono B di \qty{200}{\micro\pascal} sarà decisamente diversa rispetto a quella tra un suono C di pressione \qty{1000}{\micro\pascal} e un suono D di \qty{2000}{\micro\pascal}: l'osservazione rilevante è che il rapporto di pressione tra A e B è lo stesso che tra C e D, e cioè $\frac{1}{2}$. 

Il fatto che tendiamo a graduare molti fenomeni secondo i rapporti moltiplicativi tra le grandezze fisiche coinvolte piuttosto che secondo le loro differenze è talvolta considerata una legge fondamentale della percezione umana, chiamata \emph{legge di Fechner}. Guardando questa relazione moltiplicativa da un punto di vista leggermente diverso, possiamo dire che la nostra scala percettiva è tendenzialmente \emph{logaritmica} rispetto alla grandezza fisica osservata, dal momento che lo stesso rapporto moltiplicativo corrisponde alla stessa differenza tra i logaritmi. Tornando all'ultimo esempio,
\begin{equation}
\begin{aligned}
log_{10}100 = 2\\
log_{10}200 \approx 2.301\\
log_{10}1000 = 3\\
log_{10}2000 \approx 3.301\\
log_{10}200 - log_{10}100 = log_{10}2000 - log_{10}1000 \approx 0.301
\end{aligned}
\end{equation}
e anche, per la definizione stessa di logaritmo,
\begin{equation}
log_{10}\frac{200}{100} = log_{10}\frac{2000}{1000} = log_{10}2 \approx 0.301 
\end{equation}
e, già che ci siamo,
\begin{equation}
log_{10}\frac{100}{200} = log_{10}\frac{1000}{2000} = log_{10}\frac{1}{2} \approx -0.301 
\end{equation}

Da questa constatazione discende la definizione di bel (\unit{B}), un'unità di misura adimensionale che, in una delle sue possibili definizioni, esprime il logaritmo decimale del rapporto tra due intensità sonore:
\begin{equation}
L_{B} = log_{10}\frac{I}{I_0}
\end{equation}
dove $I$ è l'intensità sonora che vogliamo misurare rispetto a un'intensità di riferimento $I_0$. 

Molto più spesso del bel, in realtà, è usato il decibel, corrispondente alla decima parte del bel:
\begin{equation}
L_{dB} = 10 \cdot log_{10}\frac{I}{I_0}
\end{equation}

Formule analoghe possono essere usate per esprimere la relazione tra due livelli di energia sonora o di potenza sonora, dal momento che si tratta di grandezze proporzionali tra loro:
\begin{equation}
L_{dB} = 10 \cdot log_{10}\frac{E}{E_0}
\end{equation}
\begin{equation}
L_{dB} = 10 \cdot log_{10}\frac{P}{P_0}
\end{equation}

Una proprietà fondamentale di questa definizione di decibel è la sua coerenza con un'altra definizione equivalente, ma riferita al rapporto tra due pressioni sonore:
\begin{equation}
L_{dB} = 20 \cdot log_{10}\frac{\Delta p}{\Delta p_0}
\end{equation}

L'equivalenza tra queste due definizioni discende dalla relazione quadratica tra pressione e intensità sonora e dalla proprietà che permette di ``tramutare'' un esponente dentro il logaritmo in un coefficiente fuori dal logaritmo:
\begin{equation}
log x^a = a \cdot log x
\end{equation}
e quindi:
\begin{equation}
\begin{aligned}
L_{dB} = 10 \cdot log_{10}\frac{I}{I_0}\\
= 10 \cdot log_{10}\frac{\Delta p^2}{\Delta p_0^2}\\
= 10 \cdot log_{10}(\frac{\Delta p}{\Delta p_0})^2\\
= 2 \cdot 10 log_{10}\frac{\Delta p}{\Delta p_0}\\
= 20 log_{10}\frac{\Delta p}{\Delta p_0}
\end{aligned}
\end{equation}

Formule analoghe a quella che definisce il decibel in riferimento alla pressione sonora possono essere usate considerando altre grandezze a essa proporzionali, come l'ampiezza e la tensione elettrica:
\begin{equation}
L_{dB} = 20 \cdot log_{10}\frac{A}{A_0}
\end{equation}
\begin{equation}
L_{dB} = 20 \cdot log_{10}\frac{V}{V_0}
\end{equation}

Quale che sia il valore di riferimento di pressione sonora, intensità, potenza, voltaggio, ampiezza o altro, ci sono alcune considerazioni generali che possono essere fatte:

\begin{itemize}

\item A un suono al livello di riferimento corrisponde sempre un livello di \qty{0}{dB}: se per esempio $I$ è uguale a $I_0$, e poiché il logaritmo di 1 in qualsiasi base è 0, allora
\begin{equation}
L_{dB} = 10 \cdot log_{10}\frac{I}{I_0} = 10 \cdot log_{10}1 = 10 \cdot 0 = 0
\end{equation}
e lo stesso vale se invece dell'intensità si considera la pressione acustica, o l'ampiezza, o qualsiasi altra delle grandezze che abbiamo citato.

\item A suoni più forti del valore di riferimento (cioè con ampiezza, pressione, intensità ecc. superiori a esso) corrispondono livelli in decibel positivi; a suoni più deboli del valore di riferimento corrispondono livelli in decibel negativi. Infatti nel primo caso il rapporto tra valore misurato e valore di riferimento sarà maggiore di 1, e quindi il suo logaritmo maggiore di 0; nel secondo caso il rapporto sarà maggiore di 0 e minore di 1, e quindi il suo logaritmo minore di 0.

\item Il valore di riferimento non può essere il silenzio perfetto (corrispondente a un valore di intensità sonora, pressione sonora, ampiezza ecc. pari a 0), perché in questo caso metteremmo uno 0 sotto la linea di frazione e la divisione per 0 ha risultato indefinito.

\item In senso proprio, neppure il valore misurato può essere il silenzio perfetto, perché questo vorrebbe dire calcolare il logaritmo di 0 che pure non è definito. Con un leggero abuso di terminologia, si usa però spesso scrivere che il silenzio perfetto corrisponde a $-\infty$ dB. Ad ogni modo, il silenzio perfetto non esiste nel mondo fisico, ma solo nel contesto astratto delle rappresentazioni numeriche.

\item In ogni caso, anche quando il valore misurato o il valore di riferimento derivano da misurazioni con segno come la misurazione dell'ampiezza istantanea, nel calcolo del livello in dB si considera sempre il loro valore assoluto, perché il logaritmo dei numeri negativi non è definito.

\end{itemize}

La scala dei decibel, come dicevamo, rappresenta una modellizzazione semplice della maniera in cui la percezione umana gradua l'intensità del fenomeno sonoro. In maniera estremamente approssimativa, ma non completamente sbagliata, possiamo considerare che una differenza di 1, o 10, o 50 dB tra due suoni altrimenti identici abbia sempre lo stesso ``peso'' percettivo. Questo è particolarmente importante in elettroacustica e informatica musicale, contesti nei quali è semplice controllare in maniera precisa l'ampiezza di un suono senza modificare gli altri suoi parametri. 



\section{Livelli di riferimento convenzionali per i decibel}

Il decibel è una misura relativa, che però può essere impiegata in maniera assoluta se viene fissato un valore di riferimento per $A_0$, $\Delta p_0$, $V_0$, $W_0$ e così via. Esistono alcuni valori di riferimento convenzionali che danno luogo a scale assolute utili in contesti diversi. Queste comprendono, tra le altre:

\begin{itemize}

\item{$dB_{SPL}$}, dove SPL sta per \emph{sound pressure level}: misura la pressione acustica rispetto a una $P_0$ di \qty{20}{\micro\pascal}, corrispondenti approssimativamente alla soglia di udibilità umana. Quando nel linguaggio comune si dice ``in quel locale la musica era a 90 dB'' si sottintende che si sta parlando di $dB_{SPL}$. Tutti i suoni udibili avranno livelli non negativi in $dB_{SPL}$. La pressione sonora di \qty{20}{\pascal} che viene spesso indicata come la massima che l'orecchio umano può tollerare senza subire danni permanenti corrisponderà quindi a un livello sonoro di 120 dB\ped{SPL}:
\begin{equation}
L_{dB} = 20 \cdot log_{10}\frac{20 \unit{Pa}}{20 \unit{\micro\pascal}} = 20 \cdot log_{10}10^6 = 20 \cdot 6 = 120
\end{equation}

\item{$dB_{FS}$}, dove FS sta per \emph{full scale}: misura l'ampiezza rispetto al massimo livello rappresentabile in un sistema audio digitale a rappresentazione intera o in virgola fissa.%
\footnote{Maggiori dettagli su queste rappresentazioni saranno dati più avanti. Per il momento, possiamo considerare come minimo che si tratta delle rappresentazioni usate dalle interfacce audio e dalla maggior parte dei formati di file audio.}
Tutti i suoni correttamente rappresentabili in un tale sistema avranno livelli non positivi in $dB_{FS}$.

\end{itemize}

Esistono numerose altre scale di riferimento usate in elettroacustica e psicoacustica, ma non le approfondiremo qui.




\section{Piccola aritmetica dei decibel}

Succede spesso di dover convertire in decibel un rapporto di ampiezze, o viceversa. Questo può essere facilmente fatto con Max tramite gli oggetti \emph{atodb} e \emph{dbtoa}, ma è utile avere in mente almeno alcune identità semplici:

\begin{itemize}

\item Se un dato rapporto di ampiezza corrisponde a un guadagno pari a $L_{dB}$, allora l'inverso di tale rapporto corrisponderà a un guadagno pari a $-L_{dB}$: se

\begin{equation}
20 \cdot log_{10} \frac{A}{A_0} = L_{dB}
\end{equation}
allora
\begin{equation}
\begin{aligned}
20 \cdot log_{10} \frac{A_0}{A} =\\
20 \cdot log_{10} \frac{1}{\frac{A_0}{A}} =\\
20 \cdot log_{10} (\frac{A}{A_0})^{-1} =\\
-1 \cdot 20 \cdot log_{10} \frac{A}{A_0} =\\
-1 \cdot L_{dB} =\\
-L_{dB}
\end{aligned}
\end{equation}

\item Raddoppiare l'ampiezza corrisponde a un guadagno di circa \qty{6}{dB}, dal momento che $log_{10} 2 \approx 0.301$ e quindi $20 \cdot log_{10} 2 \approx 6.02$. Poiché uno scarto di \qty{0.02}{dB} è nella maggior parte dei casi del tutto irrilevante, spesso si considera l'identità come esatta anziché approssimata.

\item Dimezzare l'ampiezza corrisponde a un guadagno di circa \qty{-6}{dB}.

\item Moltiplicare per 10 l'ampiezza corrisponde a un guadagno di esattamente \qty{20}{dB}, poiché $log_{10} 10 = 1$ e quindi $20 \cdot log_{10} 10 = 20$.

\item Dividere per 10 l'ampiezza corrisponde a un guadagno di esattamente \qty{-20}{dB}.

\end{itemize}

Considerando che, in virtù delle proprietà dei logaritmi che abbiamo visto sopra, moltiplicazioni dell'ampiezza corrispondono sempre a somme algebriche di decibel ed elevamenti a potenza dell'ampiezza corrispondono a moltiplicazioni di decibel, le identità riportate sopra possono essere utili per fare alcuni calcoli in maniera semplice. Per esempio:

\begin{itemize}

\item Un guadagno di 100, cioè $10 \cdot 10$, corrisponde a $20 + 20 = 40$ dB.

\item Un guadagno di 20, cioè $2 \cdot 10$, corrisponde a circa $6 + 20 = 26$ dB.

\item Un'attenuazione di 1000, cioè un guadagno di $\frac{1}{1000}$ o $10^{-3}$ corrisponde a $20 \cdot -3 = -60$ dB.

\item \qty{8}{dB}, cioè $20 - 2 \cdot 6$ dB, corrispondono a un guadagno di circa $10 \cdot 2^{-2} = 10 \cdot \frac{1}{4} = 2.5$

\item \qty{10}{dB}, cioè $20 \cdot \frac{1}{2}$ dB, corrispondono a un guadagno di $10^\frac{1}{2} = \sqrt{10} \approx {3.16}$

\end{itemize}

Queste operazioni corrispondono, tra l'altro, al collegamento in serie di amplificatori o attenuatori, come avviene normalmente nel percorso del segnale lungo una catena elettroacustica o all'interno di un mixer. Se per esempio lo stadio di guadagno di un mixer è regolato a \qty{40}{dB}, il fader del corrispondente canale a \qty{-10}{dB} e il livello di uscita a \qty{-20}{dB} (assumendo che non ci siano altre trasformazioni del segnale, come equalizzazioni o altro) sapremo che il segnale in uscita dal mixer sarà $40-10-20 = 10$ dB più forte rispetto a quello in entrata, e quindi che il voltaggio in uscita sarà circa pari al voltaggio in entrata moltiplicato per 3.16.



  