
\chapter{Pressione, ampiezza, intensità}

\section{Pressione sonora}

Il suono è costituito da (o associato a) onde di pressione longitudinali che si propagano in un mezzo elastico (tipicamente l'aria).

L'onda longitudinale è tale perché lo spostamento di energia avviene lungo lo stesso asse dello spostamento di materia (come per esempio nel caso di una molla che viene colpita a un'estremità). Il caso opposto, che non riguarda il suono, è l'onda trasversale, in cui lo spostamento di energia avviene perpendicolarmente rispetto all'asse dello spostamento di materia: l'esempio classico è l'onda del mare, in cui l'energia si sposta orizzontalmente (dal mare aperto verso la riva) mentre la materia si sposta verticalmente (la singola molecola d'acqua si alza e si abbassa a causa del moto ondoso).

Il suono sarà in generale prodotto da un corpo emittente, che per semplicità immaginiamo puntiforme. Nel nostro modello, consideriamo anche la presenza di un recettore, pure puntiforme. Il corpo emittente oscilla in maniera più o meno regolare attorno a un suo punto di riposo. Così facendo, crea nell'aria un'alternanza di compressioni nella direzione del recettore e compressioni nella direzione opposta a quella del recettore (che possiamo considerare decompressioni nella direzione del recettore). Questi fronti di compressioni e decompressioni si propagano in forma di superfici sferiche. 

La pressione si misura in pascal (\unit{Pa}) e i suoi multipli e sottomultipli, come 
\begin{itemize}
\item il millipascal (\unit{mPa}: $\frac{1}{1000}$\unit{Pa}), 
\item il micropascal (\unit{\micro\pascal}: $\frac{1}{1000000}$\unit{Pa}), 
\item il kilopascal (\unit{\kilo\pascal}: \qty{1000}{Pa}),
\item il \unit{bar} (\unit{bar}: \qty{100000}{Pa}).
\end{itemize}

La pressione atmosferica standard al livello del mare è poco più di \qty{100000}{Pa}. Il fenomeno sonoro provoca piccole variazioni nella pressione atmosferica: le variazioni tipiche dei suoni udibili avvengono in un ambito compreso all'incirca tra \qty{20}{\micro\pascal} e \qty{20}{\pascal}. 

Il recettore del fenomeno sonoro (ad esempio la membrana del timpano, o una membrana microfonica) è immerso nell'atmosfera, che esercita eguale pressione da entrambi i lati: di conseguenza, possiamo considerare in generale nulla la pressione atmosferica complessiva a cui la membrana è sottoposta.%
\footnote{Questo non è vero ad esempio nel caso di sbalzi d'altitudine o di immersioni subacquee: in questi casi infatti bisogna compensare le variazioni di pressione a cui il timpano è sottoposto.}
Il fenomeno sonoro provoca invece l'applicazione di una pressione direzionata sul recettore, che ne sarà compresso verso l'interno o verso l'esterno: ha senso quindi dire che il recettore è sottoposto a una pressione sonora dell'ambito descritto sopra (tra \qty{20}{\micro\pascal} e \qty{20}{\pascal}). Per convenzione, parleremo di pressione positiva quando la pressione esterna è superiore alla pressione interna (e quindi la membrana del timpano sarà spinta verso l'interno del cranio); di pressione negativa nel caso opposto (in cui la membrana del timpano è ``risucchiata'' verso l'esterno del cranio).

L'onda sonora impiega tempo a propagarsi: al livello del mare, con aria secca e temperatura di \qty{20}{\celsius}, la velocità di propagazione è di circa \qty{343}{m / s}. Questo vuol dire che, approssimativamente, il suono in queste condizioni percorre \qty{30}{cm} in \qty{1}{ms}; \qty{1}{m} in \qty{3}{ms}; e \qty{1}{km} in \qty{3}{s}.



\section{Ampiezza e rappresentazione nel dominio del tempo}

Ci interessa individuare alcune grandezze che variano in maniera idealmente proporzionale%
\footnote{Nel mondo reale, questa proporzionalità è sempre approssimativa. Rispetto alle definizioni della fisica classica, e in un contesto ideale in cui non esistono fenomeni parassiti che disturbano l'emissione, la trasmissione e la ricezione del suono (come ad esempio rumori d'ambiente, correnti d'aria, superfici riflettenti, l'elasticità non perfetta dei materiali e così via), è spesso utile considerarla esatta per semplicità.}
tra loro, per lo stesso fenomeno sonoro:

\begin{itemize}
\item lo spostamento dell'emittente $s_e$
\item lo spostamento del recettore $s_r$ (se il mezzo elastico non induce distorsioni: in generale possiamo considerare che le distorsioni indotte dall'aria siano estremamente ridotte)
\item le variazioni di pressione atmosferica $\Delta p$ (in assenza di altri fenomeni perturbativi)
\item la tensione elettrica nella catena elettroacustica $V$ (assumendo che trasduttori, conduttori e amplificatori si comportino in maniera ideale)
\end{itemize}

Possiamo allora scrivere che
\begin{equation}
s_e \propto s_r \propto \Delta p \propto V
\end{equation}
dove il simbolo $\propto$ indica una relazione di proporzionalità diretta.

Poiché la percezione e il trattamento del fenomeno sonoro avvengono in generale in termini astratti rispetto alla materialità delle grandezze fisiche coinvolte, spesso utilizziamo il concetto di \emph{ampiezza} del segnale sonoro. L'ampiezza è una misura adimensionale proporzionale alle grandezze elencate sopra; le rappresenta tutte, e non coincide con nessuna in particolare. Il suo ambito è arbitrario e convenzionale; in condizioni normali, si adotta talvolta un ambito compreso tra -1 e 1, ma ci sono moltissime eccezioni. Il concetto di ampiezza è cruciale nella rappresentazione digitale del suono. Quindi
\begin{equation}
A \propto s_e \propto s_r \propto \Delta p \propto V
\end{equation}

Se si traccia l'andamento di una qualsiasi di queste grandezze rispetto al tempo si ottiene un grafico che rappresenta il fenomeno sonoro \emph{nel dominio del tempo}. Il fatto che si tratti di grandezze tutte proporzionali tra loro fa sì che, cambiando l'unità di misura sull'asse verticale, il grafico mantenga la stessa forma. Possiamo quindi parlare di \emph{forma d'onda} (in inglese \emph{waveform}). Questo tipo di grafico è talvolta chiamato \emph{audiogramma}.


\begin{figure}
    \begin{center}
       \scalebox{0.6} {%% Creator: Matplotlib, PGF backend
%%
%% To include the figure in your LaTeX document, write
%%   \input{<filename>.pgf}
%%
%% Make sure the required packages are loaded in your preamble
%%   \usepackage{pgf}
%%
%% Also ensure that all the required font packages are loaded; for instance,
%% the lmodern package is sometimes necessary when using math font.
%%   \usepackage{lmodern}
%%
%% Figures using additional raster images can only be included by \input if
%% they are in the same directory as the main LaTeX file. For loading figures
%% from other directories you can use the `import` package
%%   \usepackage{import}
%%
%% and then include the figures with
%%   \import{<path to file>}{<filename>.pgf}
%%
%% Matplotlib used the following preamble
%%   
%%   \usepackage{fontspec}
%%   \setmainfont{DejaVuSerif.ttf}[Path=\detokenize{/opt/homebrew/Caskroom/miniconda/base/envs/label-studio/lib/python3.9/site-packages/matplotlib/mpl-data/fonts/ttf/}]
%%   \setsansfont{DejaVuSans.ttf}[Path=\detokenize{/opt/homebrew/Caskroom/miniconda/base/envs/label-studio/lib/python3.9/site-packages/matplotlib/mpl-data/fonts/ttf/}]
%%   \setmonofont{DejaVuSansMono.ttf}[Path=\detokenize{/opt/homebrew/Caskroom/miniconda/base/envs/label-studio/lib/python3.9/site-packages/matplotlib/mpl-data/fonts/ttf/}]
%%   \makeatletter\@ifpackageloaded{underscore}{}{\usepackage[strings]{underscore}}\makeatother
%%
\begingroup%
\makeatletter%
\begin{pgfpicture}%
\pgfpathrectangle{\pgfpointorigin}{\pgfqpoint{7.000000in}{4.000000in}}%
\pgfusepath{use as bounding box, clip}%
\begin{pgfscope}%
\pgfsetbuttcap%
\pgfsetmiterjoin%
\definecolor{currentfill}{rgb}{1.000000,1.000000,1.000000}%
\pgfsetfillcolor{currentfill}%
\pgfsetlinewidth{0.000000pt}%
\definecolor{currentstroke}{rgb}{1.000000,1.000000,1.000000}%
\pgfsetstrokecolor{currentstroke}%
\pgfsetdash{}{0pt}%
\pgfpathmoveto{\pgfqpoint{0.000000in}{0.000000in}}%
\pgfpathlineto{\pgfqpoint{7.000000in}{0.000000in}}%
\pgfpathlineto{\pgfqpoint{7.000000in}{4.000000in}}%
\pgfpathlineto{\pgfqpoint{0.000000in}{4.000000in}}%
\pgfpathlineto{\pgfqpoint{0.000000in}{0.000000in}}%
\pgfpathclose%
\pgfusepath{fill}%
\end{pgfscope}%
\begin{pgfscope}%
\pgfsetbuttcap%
\pgfsetmiterjoin%
\definecolor{currentfill}{rgb}{1.000000,1.000000,1.000000}%
\pgfsetfillcolor{currentfill}%
\pgfsetlinewidth{0.000000pt}%
\definecolor{currentstroke}{rgb}{0.000000,0.000000,0.000000}%
\pgfsetstrokecolor{currentstroke}%
\pgfsetstrokeopacity{0.000000}%
\pgfsetdash{}{0pt}%
\pgfpathmoveto{\pgfqpoint{0.746130in}{2.463273in}}%
\pgfpathlineto{\pgfqpoint{6.958330in}{2.463273in}}%
\pgfpathlineto{\pgfqpoint{6.958330in}{3.748369in}}%
\pgfpathlineto{\pgfqpoint{0.746130in}{3.748369in}}%
\pgfpathlineto{\pgfqpoint{0.746130in}{2.463273in}}%
\pgfpathclose%
\pgfusepath{fill}%
\end{pgfscope}%
\begin{pgfscope}%
\pgfsetbuttcap%
\pgfsetroundjoin%
\definecolor{currentfill}{rgb}{0.000000,0.000000,0.000000}%
\pgfsetfillcolor{currentfill}%
\pgfsetlinewidth{0.803000pt}%
\definecolor{currentstroke}{rgb}{0.000000,0.000000,0.000000}%
\pgfsetstrokecolor{currentstroke}%
\pgfsetdash{}{0pt}%
\pgfsys@defobject{currentmarker}{\pgfqpoint{0.000000in}{-0.048611in}}{\pgfqpoint{0.000000in}{0.000000in}}{%
\pgfpathmoveto{\pgfqpoint{0.000000in}{0.000000in}}%
\pgfpathlineto{\pgfqpoint{0.000000in}{-0.048611in}}%
\pgfusepath{stroke,fill}%
}%
\begin{pgfscope}%
\pgfsys@transformshift{1.028503in}{2.463273in}%
\pgfsys@useobject{currentmarker}{}%
\end{pgfscope}%
\end{pgfscope}%
\begin{pgfscope}%
\definecolor{textcolor}{rgb}{0.000000,0.000000,0.000000}%
\pgfsetstrokecolor{textcolor}%
\pgfsetfillcolor{textcolor}%
\pgftext[x=1.028503in,y=2.366051in,,top]{\color{textcolor}\sffamily\fontsize{10.000000}{12.000000}\selectfont 0}%
\end{pgfscope}%
\begin{pgfscope}%
\pgfsetbuttcap%
\pgfsetroundjoin%
\definecolor{currentfill}{rgb}{0.000000,0.000000,0.000000}%
\pgfsetfillcolor{currentfill}%
\pgfsetlinewidth{0.803000pt}%
\definecolor{currentstroke}{rgb}{0.000000,0.000000,0.000000}%
\pgfsetstrokecolor{currentstroke}%
\pgfsetdash{}{0pt}%
\pgfsys@defobject{currentmarker}{\pgfqpoint{0.000000in}{-0.048611in}}{\pgfqpoint{0.000000in}{0.000000in}}{%
\pgfpathmoveto{\pgfqpoint{0.000000in}{0.000000in}}%
\pgfpathlineto{\pgfqpoint{0.000000in}{-0.048611in}}%
\pgfusepath{stroke,fill}%
}%
\begin{pgfscope}%
\pgfsys@transformshift{1.799245in}{2.463273in}%
\pgfsys@useobject{currentmarker}{}%
\end{pgfscope}%
\end{pgfscope}%
\begin{pgfscope}%
\definecolor{textcolor}{rgb}{0.000000,0.000000,0.000000}%
\pgfsetstrokecolor{textcolor}%
\pgfsetfillcolor{textcolor}%
\pgftext[x=1.799245in,y=2.366051in,,top]{\color{textcolor}\sffamily\fontsize{10.000000}{12.000000}\selectfont 50000}%
\end{pgfscope}%
\begin{pgfscope}%
\pgfsetbuttcap%
\pgfsetroundjoin%
\definecolor{currentfill}{rgb}{0.000000,0.000000,0.000000}%
\pgfsetfillcolor{currentfill}%
\pgfsetlinewidth{0.803000pt}%
\definecolor{currentstroke}{rgb}{0.000000,0.000000,0.000000}%
\pgfsetstrokecolor{currentstroke}%
\pgfsetdash{}{0pt}%
\pgfsys@defobject{currentmarker}{\pgfqpoint{0.000000in}{-0.048611in}}{\pgfqpoint{0.000000in}{0.000000in}}{%
\pgfpathmoveto{\pgfqpoint{0.000000in}{0.000000in}}%
\pgfpathlineto{\pgfqpoint{0.000000in}{-0.048611in}}%
\pgfusepath{stroke,fill}%
}%
\begin{pgfscope}%
\pgfsys@transformshift{2.569986in}{2.463273in}%
\pgfsys@useobject{currentmarker}{}%
\end{pgfscope}%
\end{pgfscope}%
\begin{pgfscope}%
\definecolor{textcolor}{rgb}{0.000000,0.000000,0.000000}%
\pgfsetstrokecolor{textcolor}%
\pgfsetfillcolor{textcolor}%
\pgftext[x=2.569986in,y=2.366051in,,top]{\color{textcolor}\sffamily\fontsize{10.000000}{12.000000}\selectfont 100000}%
\end{pgfscope}%
\begin{pgfscope}%
\pgfsetbuttcap%
\pgfsetroundjoin%
\definecolor{currentfill}{rgb}{0.000000,0.000000,0.000000}%
\pgfsetfillcolor{currentfill}%
\pgfsetlinewidth{0.803000pt}%
\definecolor{currentstroke}{rgb}{0.000000,0.000000,0.000000}%
\pgfsetstrokecolor{currentstroke}%
\pgfsetdash{}{0pt}%
\pgfsys@defobject{currentmarker}{\pgfqpoint{0.000000in}{-0.048611in}}{\pgfqpoint{0.000000in}{0.000000in}}{%
\pgfpathmoveto{\pgfqpoint{0.000000in}{0.000000in}}%
\pgfpathlineto{\pgfqpoint{0.000000in}{-0.048611in}}%
\pgfusepath{stroke,fill}%
}%
\begin{pgfscope}%
\pgfsys@transformshift{3.340728in}{2.463273in}%
\pgfsys@useobject{currentmarker}{}%
\end{pgfscope}%
\end{pgfscope}%
\begin{pgfscope}%
\definecolor{textcolor}{rgb}{0.000000,0.000000,0.000000}%
\pgfsetstrokecolor{textcolor}%
\pgfsetfillcolor{textcolor}%
\pgftext[x=3.340728in,y=2.366051in,,top]{\color{textcolor}\sffamily\fontsize{10.000000}{12.000000}\selectfont 150000}%
\end{pgfscope}%
\begin{pgfscope}%
\pgfsetbuttcap%
\pgfsetroundjoin%
\definecolor{currentfill}{rgb}{0.000000,0.000000,0.000000}%
\pgfsetfillcolor{currentfill}%
\pgfsetlinewidth{0.803000pt}%
\definecolor{currentstroke}{rgb}{0.000000,0.000000,0.000000}%
\pgfsetstrokecolor{currentstroke}%
\pgfsetdash{}{0pt}%
\pgfsys@defobject{currentmarker}{\pgfqpoint{0.000000in}{-0.048611in}}{\pgfqpoint{0.000000in}{0.000000in}}{%
\pgfpathmoveto{\pgfqpoint{0.000000in}{0.000000in}}%
\pgfpathlineto{\pgfqpoint{0.000000in}{-0.048611in}}%
\pgfusepath{stroke,fill}%
}%
\begin{pgfscope}%
\pgfsys@transformshift{4.111469in}{2.463273in}%
\pgfsys@useobject{currentmarker}{}%
\end{pgfscope}%
\end{pgfscope}%
\begin{pgfscope}%
\definecolor{textcolor}{rgb}{0.000000,0.000000,0.000000}%
\pgfsetstrokecolor{textcolor}%
\pgfsetfillcolor{textcolor}%
\pgftext[x=4.111469in,y=2.366051in,,top]{\color{textcolor}\sffamily\fontsize{10.000000}{12.000000}\selectfont 200000}%
\end{pgfscope}%
\begin{pgfscope}%
\pgfsetbuttcap%
\pgfsetroundjoin%
\definecolor{currentfill}{rgb}{0.000000,0.000000,0.000000}%
\pgfsetfillcolor{currentfill}%
\pgfsetlinewidth{0.803000pt}%
\definecolor{currentstroke}{rgb}{0.000000,0.000000,0.000000}%
\pgfsetstrokecolor{currentstroke}%
\pgfsetdash{}{0pt}%
\pgfsys@defobject{currentmarker}{\pgfqpoint{0.000000in}{-0.048611in}}{\pgfqpoint{0.000000in}{0.000000in}}{%
\pgfpathmoveto{\pgfqpoint{0.000000in}{0.000000in}}%
\pgfpathlineto{\pgfqpoint{0.000000in}{-0.048611in}}%
\pgfusepath{stroke,fill}%
}%
\begin{pgfscope}%
\pgfsys@transformshift{4.882211in}{2.463273in}%
\pgfsys@useobject{currentmarker}{}%
\end{pgfscope}%
\end{pgfscope}%
\begin{pgfscope}%
\definecolor{textcolor}{rgb}{0.000000,0.000000,0.000000}%
\pgfsetstrokecolor{textcolor}%
\pgfsetfillcolor{textcolor}%
\pgftext[x=4.882211in,y=2.366051in,,top]{\color{textcolor}\sffamily\fontsize{10.000000}{12.000000}\selectfont 250000}%
\end{pgfscope}%
\begin{pgfscope}%
\pgfsetbuttcap%
\pgfsetroundjoin%
\definecolor{currentfill}{rgb}{0.000000,0.000000,0.000000}%
\pgfsetfillcolor{currentfill}%
\pgfsetlinewidth{0.803000pt}%
\definecolor{currentstroke}{rgb}{0.000000,0.000000,0.000000}%
\pgfsetstrokecolor{currentstroke}%
\pgfsetdash{}{0pt}%
\pgfsys@defobject{currentmarker}{\pgfqpoint{0.000000in}{-0.048611in}}{\pgfqpoint{0.000000in}{0.000000in}}{%
\pgfpathmoveto{\pgfqpoint{0.000000in}{0.000000in}}%
\pgfpathlineto{\pgfqpoint{0.000000in}{-0.048611in}}%
\pgfusepath{stroke,fill}%
}%
\begin{pgfscope}%
\pgfsys@transformshift{5.652952in}{2.463273in}%
\pgfsys@useobject{currentmarker}{}%
\end{pgfscope}%
\end{pgfscope}%
\begin{pgfscope}%
\definecolor{textcolor}{rgb}{0.000000,0.000000,0.000000}%
\pgfsetstrokecolor{textcolor}%
\pgfsetfillcolor{textcolor}%
\pgftext[x=5.652952in,y=2.366051in,,top]{\color{textcolor}\sffamily\fontsize{10.000000}{12.000000}\selectfont 300000}%
\end{pgfscope}%
\begin{pgfscope}%
\pgfsetbuttcap%
\pgfsetroundjoin%
\definecolor{currentfill}{rgb}{0.000000,0.000000,0.000000}%
\pgfsetfillcolor{currentfill}%
\pgfsetlinewidth{0.803000pt}%
\definecolor{currentstroke}{rgb}{0.000000,0.000000,0.000000}%
\pgfsetstrokecolor{currentstroke}%
\pgfsetdash{}{0pt}%
\pgfsys@defobject{currentmarker}{\pgfqpoint{0.000000in}{-0.048611in}}{\pgfqpoint{0.000000in}{0.000000in}}{%
\pgfpathmoveto{\pgfqpoint{0.000000in}{0.000000in}}%
\pgfpathlineto{\pgfqpoint{0.000000in}{-0.048611in}}%
\pgfusepath{stroke,fill}%
}%
\begin{pgfscope}%
\pgfsys@transformshift{6.423694in}{2.463273in}%
\pgfsys@useobject{currentmarker}{}%
\end{pgfscope}%
\end{pgfscope}%
\begin{pgfscope}%
\definecolor{textcolor}{rgb}{0.000000,0.000000,0.000000}%
\pgfsetstrokecolor{textcolor}%
\pgfsetfillcolor{textcolor}%
\pgftext[x=6.423694in,y=2.366051in,,top]{\color{textcolor}\sffamily\fontsize{10.000000}{12.000000}\selectfont 350000}%
\end{pgfscope}%
\begin{pgfscope}%
\definecolor{textcolor}{rgb}{0.000000,0.000000,0.000000}%
\pgfsetstrokecolor{textcolor}%
\pgfsetfillcolor{textcolor}%
\pgftext[x=3.852230in,y=2.176083in,,top]{\color{textcolor}\sffamily\fontsize{10.000000}{12.000000}\selectfont ms}%
\end{pgfscope}%
\begin{pgfscope}%
\pgfsetbuttcap%
\pgfsetroundjoin%
\definecolor{currentfill}{rgb}{0.000000,0.000000,0.000000}%
\pgfsetfillcolor{currentfill}%
\pgfsetlinewidth{0.803000pt}%
\definecolor{currentstroke}{rgb}{0.000000,0.000000,0.000000}%
\pgfsetstrokecolor{currentstroke}%
\pgfsetdash{}{0pt}%
\pgfsys@defobject{currentmarker}{\pgfqpoint{-0.048611in}{0.000000in}}{\pgfqpoint{-0.000000in}{0.000000in}}{%
\pgfpathmoveto{\pgfqpoint{-0.000000in}{0.000000in}}%
\pgfpathlineto{\pgfqpoint{-0.048611in}{0.000000in}}%
\pgfusepath{stroke,fill}%
}%
\begin{pgfscope}%
\pgfsys@transformshift{0.746130in}{2.788636in}%
\pgfsys@useobject{currentmarker}{}%
\end{pgfscope}%
\end{pgfscope}%
\begin{pgfscope}%
\definecolor{textcolor}{rgb}{0.000000,0.000000,0.000000}%
\pgfsetstrokecolor{textcolor}%
\pgfsetfillcolor{textcolor}%
\pgftext[x=0.231638in, y=2.735875in, left, base]{\color{textcolor}\sffamily\fontsize{10.000000}{12.000000}\selectfont \ensuremath{-}0.25}%
\end{pgfscope}%
\begin{pgfscope}%
\pgfsetbuttcap%
\pgfsetroundjoin%
\definecolor{currentfill}{rgb}{0.000000,0.000000,0.000000}%
\pgfsetfillcolor{currentfill}%
\pgfsetlinewidth{0.803000pt}%
\definecolor{currentstroke}{rgb}{0.000000,0.000000,0.000000}%
\pgfsetstrokecolor{currentstroke}%
\pgfsetdash{}{0pt}%
\pgfsys@defobject{currentmarker}{\pgfqpoint{-0.048611in}{0.000000in}}{\pgfqpoint{-0.000000in}{0.000000in}}{%
\pgfpathmoveto{\pgfqpoint{-0.000000in}{0.000000in}}%
\pgfpathlineto{\pgfqpoint{-0.048611in}{0.000000in}}%
\pgfusepath{stroke,fill}%
}%
\begin{pgfscope}%
\pgfsys@transformshift{0.746130in}{3.171652in}%
\pgfsys@useobject{currentmarker}{}%
\end{pgfscope}%
\end{pgfscope}%
\begin{pgfscope}%
\definecolor{textcolor}{rgb}{0.000000,0.000000,0.000000}%
\pgfsetstrokecolor{textcolor}%
\pgfsetfillcolor{textcolor}%
\pgftext[x=0.339663in, y=3.118891in, left, base]{\color{textcolor}\sffamily\fontsize{10.000000}{12.000000}\selectfont 0.00}%
\end{pgfscope}%
\begin{pgfscope}%
\pgfsetbuttcap%
\pgfsetroundjoin%
\definecolor{currentfill}{rgb}{0.000000,0.000000,0.000000}%
\pgfsetfillcolor{currentfill}%
\pgfsetlinewidth{0.803000pt}%
\definecolor{currentstroke}{rgb}{0.000000,0.000000,0.000000}%
\pgfsetstrokecolor{currentstroke}%
\pgfsetdash{}{0pt}%
\pgfsys@defobject{currentmarker}{\pgfqpoint{-0.048611in}{0.000000in}}{\pgfqpoint{-0.000000in}{0.000000in}}{%
\pgfpathmoveto{\pgfqpoint{-0.000000in}{0.000000in}}%
\pgfpathlineto{\pgfqpoint{-0.048611in}{0.000000in}}%
\pgfusepath{stroke,fill}%
}%
\begin{pgfscope}%
\pgfsys@transformshift{0.746130in}{3.554669in}%
\pgfsys@useobject{currentmarker}{}%
\end{pgfscope}%
\end{pgfscope}%
\begin{pgfscope}%
\definecolor{textcolor}{rgb}{0.000000,0.000000,0.000000}%
\pgfsetstrokecolor{textcolor}%
\pgfsetfillcolor{textcolor}%
\pgftext[x=0.339663in, y=3.501907in, left, base]{\color{textcolor}\sffamily\fontsize{10.000000}{12.000000}\selectfont 0.25}%
\end{pgfscope}%
\begin{pgfscope}%
\definecolor{textcolor}{rgb}{0.000000,0.000000,0.000000}%
\pgfsetstrokecolor{textcolor}%
\pgfsetfillcolor{textcolor}%
\pgftext[x=0.176083in,y=3.105821in,,bottom,rotate=90.000000]{\color{textcolor}\sffamily\fontsize{10.000000}{12.000000}\selectfont Amplitude}%
\end{pgfscope}%
\begin{pgfscope}%
\pgfpathrectangle{\pgfqpoint{0.746130in}{2.463273in}}{\pgfqpoint{6.212200in}{1.285096in}}%
\pgfusepath{clip}%
\pgfsetrectcap%
\pgfsetroundjoin%
\pgfsetlinewidth{1.505625pt}%
\definecolor{currentstroke}{rgb}{0.121569,0.466667,0.705882}%
\pgfsetstrokecolor{currentstroke}%
\pgfsetdash{}{0pt}%
\pgfpathmoveto{\pgfqpoint{1.028503in}{3.171555in}}%
\pgfpathlineto{\pgfqpoint{1.029551in}{3.171457in}}%
\pgfpathlineto{\pgfqpoint{1.028873in}{3.171785in}}%
\pgfpathlineto{\pgfqpoint{1.029613in}{3.171553in}}%
\pgfpathlineto{\pgfqpoint{1.032989in}{3.171672in}}%
\pgfpathlineto{\pgfqpoint{1.033405in}{3.171904in}}%
\pgfpathlineto{\pgfqpoint{1.033883in}{3.171354in}}%
\pgfpathlineto{\pgfqpoint{1.034083in}{3.171478in}}%
\pgfpathlineto{\pgfqpoint{1.035270in}{3.171716in}}%
\pgfpathlineto{\pgfqpoint{1.035301in}{3.171684in}}%
\pgfpathlineto{\pgfqpoint{1.036503in}{3.171600in}}%
\pgfpathlineto{\pgfqpoint{1.046061in}{3.171996in}}%
\pgfpathlineto{\pgfqpoint{1.047232in}{3.172625in}}%
\pgfpathlineto{\pgfqpoint{1.046646in}{3.170931in}}%
\pgfpathlineto{\pgfqpoint{1.047248in}{3.172621in}}%
\pgfpathlineto{\pgfqpoint{1.048619in}{3.169519in}}%
\pgfpathlineto{\pgfqpoint{1.048681in}{3.169675in}}%
\pgfpathlineto{\pgfqpoint{1.050223in}{3.168516in}}%
\pgfpathlineto{\pgfqpoint{1.050531in}{3.169496in}}%
\pgfpathlineto{\pgfqpoint{1.051903in}{3.173677in}}%
\pgfpathlineto{\pgfqpoint{1.052427in}{3.172486in}}%
\pgfpathlineto{\pgfqpoint{1.052088in}{3.174108in}}%
\pgfpathlineto{\pgfqpoint{1.053013in}{3.173806in}}%
\pgfpathlineto{\pgfqpoint{1.053090in}{3.173910in}}%
\pgfpathlineto{\pgfqpoint{1.053229in}{3.172927in}}%
\pgfpathlineto{\pgfqpoint{1.053552in}{3.173583in}}%
\pgfpathlineto{\pgfqpoint{1.053583in}{3.171487in}}%
\pgfpathlineto{\pgfqpoint{1.054678in}{3.171630in}}%
\pgfpathlineto{\pgfqpoint{1.056142in}{3.171666in}}%
\pgfpathlineto{\pgfqpoint{1.056157in}{3.171645in}}%
\pgfpathlineto{\pgfqpoint{1.136176in}{3.171665in}}%
\pgfpathlineto{\pgfqpoint{1.137409in}{3.171634in}}%
\pgfpathlineto{\pgfqpoint{1.138596in}{3.171668in}}%
\pgfpathlineto{\pgfqpoint{1.139814in}{3.171641in}}%
\pgfpathlineto{\pgfqpoint{1.139829in}{3.171659in}}%
\pgfpathlineto{\pgfqpoint{1.143405in}{3.171656in}}%
\pgfpathlineto{\pgfqpoint{1.144669in}{3.171640in}}%
\pgfpathlineto{\pgfqpoint{1.144685in}{3.171656in}}%
\pgfpathlineto{\pgfqpoint{1.147074in}{3.171650in}}%
\pgfpathlineto{\pgfqpoint{1.148508in}{3.171676in}}%
\pgfpathlineto{\pgfqpoint{1.149710in}{3.171641in}}%
\pgfpathlineto{\pgfqpoint{1.152223in}{3.171663in}}%
\pgfpathlineto{\pgfqpoint{1.153594in}{3.171625in}}%
\pgfpathlineto{\pgfqpoint{1.154781in}{3.171677in}}%
\pgfpathlineto{\pgfqpoint{1.155953in}{3.171643in}}%
\pgfpathlineto{\pgfqpoint{1.158404in}{3.171654in}}%
\pgfpathlineto{\pgfqpoint{1.160161in}{3.171656in}}%
\pgfpathlineto{\pgfqpoint{1.174543in}{3.171648in}}%
\pgfpathlineto{\pgfqpoint{1.175977in}{3.171681in}}%
\pgfpathlineto{\pgfqpoint{1.177164in}{3.171644in}}%
\pgfpathlineto{\pgfqpoint{1.188447in}{3.171639in}}%
\pgfpathlineto{\pgfqpoint{1.189711in}{3.171678in}}%
\pgfpathlineto{\pgfqpoint{1.190883in}{3.171634in}}%
\pgfpathlineto{\pgfqpoint{1.193134in}{3.171664in}}%
\pgfpathlineto{\pgfqpoint{1.194459in}{3.171637in}}%
\pgfpathlineto{\pgfqpoint{1.194475in}{3.171656in}}%
\pgfpathlineto{\pgfqpoint{1.196340in}{3.171667in}}%
\pgfpathlineto{\pgfqpoint{1.197604in}{3.171639in}}%
\pgfpathlineto{\pgfqpoint{1.197635in}{3.171659in}}%
\pgfpathlineto{\pgfqpoint{1.199577in}{3.171663in}}%
\pgfpathlineto{\pgfqpoint{1.200826in}{3.171617in}}%
\pgfpathlineto{\pgfqpoint{1.200856in}{3.171663in}}%
\pgfpathlineto{\pgfqpoint{1.203153in}{3.171664in}}%
\pgfpathlineto{\pgfqpoint{1.204510in}{3.171636in}}%
\pgfpathlineto{\pgfqpoint{1.205743in}{3.171652in}}%
\pgfpathlineto{\pgfqpoint{1.254377in}{3.171658in}}%
\pgfpathlineto{\pgfqpoint{1.255641in}{3.171640in}}%
\pgfpathlineto{\pgfqpoint{1.257013in}{3.171648in}}%
\pgfpathlineto{\pgfqpoint{1.259772in}{3.171642in}}%
\pgfpathlineto{\pgfqpoint{1.261360in}{3.171674in}}%
\pgfpathlineto{\pgfqpoint{1.262516in}{3.171642in}}%
\pgfpathlineto{\pgfqpoint{1.262531in}{3.171651in}}%
\pgfpathlineto{\pgfqpoint{1.265306in}{3.171661in}}%
\pgfpathlineto{\pgfqpoint{1.266832in}{3.171650in}}%
\pgfpathlineto{\pgfqpoint{1.268481in}{3.171654in}}%
\pgfpathlineto{\pgfqpoint{1.269853in}{3.171667in}}%
\pgfpathlineto{\pgfqpoint{1.271718in}{3.171648in}}%
\pgfpathlineto{\pgfqpoint{1.274539in}{3.171643in}}%
\pgfpathlineto{\pgfqpoint{1.277175in}{3.171633in}}%
\pgfpathlineto{\pgfqpoint{1.278316in}{3.171662in}}%
\pgfpathlineto{\pgfqpoint{1.278331in}{3.171637in}}%
\pgfpathlineto{\pgfqpoint{1.279564in}{3.171663in}}%
\pgfpathlineto{\pgfqpoint{1.280890in}{3.171650in}}%
\pgfpathlineto{\pgfqpoint{1.282493in}{3.171660in}}%
\pgfpathlineto{\pgfqpoint{1.283711in}{3.171642in}}%
\pgfpathlineto{\pgfqpoint{1.283726in}{3.171658in}}%
\pgfpathlineto{\pgfqpoint{1.285653in}{3.171649in}}%
\pgfpathlineto{\pgfqpoint{1.287688in}{3.171670in}}%
\pgfpathlineto{\pgfqpoint{1.287703in}{3.171644in}}%
\pgfpathlineto{\pgfqpoint{1.289291in}{3.171652in}}%
\pgfpathlineto{\pgfqpoint{1.290786in}{3.171643in}}%
\pgfpathlineto{\pgfqpoint{1.319612in}{3.171657in}}%
\pgfpathlineto{\pgfqpoint{1.321061in}{3.171637in}}%
\pgfpathlineto{\pgfqpoint{1.322371in}{3.171653in}}%
\pgfpathlineto{\pgfqpoint{1.322402in}{3.171670in}}%
\pgfpathlineto{\pgfqpoint{1.323574in}{3.171643in}}%
\pgfpathlineto{\pgfqpoint{1.325979in}{3.171672in}}%
\pgfpathlineto{\pgfqpoint{1.327212in}{3.171634in}}%
\pgfpathlineto{\pgfqpoint{1.327227in}{3.171666in}}%
\pgfpathlineto{\pgfqpoint{1.329231in}{3.171643in}}%
\pgfpathlineto{\pgfqpoint{1.330803in}{3.171650in}}%
\pgfpathlineto{\pgfqpoint{1.340037in}{3.171652in}}%
\pgfpathlineto{\pgfqpoint{1.342087in}{3.171665in}}%
\pgfpathlineto{\pgfqpoint{1.344677in}{3.171649in}}%
\pgfpathlineto{\pgfqpoint{1.349733in}{3.171651in}}%
\pgfpathlineto{\pgfqpoint{1.351583in}{3.171617in}}%
\pgfpathlineto{\pgfqpoint{1.352739in}{3.171665in}}%
\pgfpathlineto{\pgfqpoint{1.377541in}{3.171644in}}%
\pgfpathlineto{\pgfqpoint{1.378836in}{3.171663in}}%
\pgfpathlineto{\pgfqpoint{1.380300in}{3.171633in}}%
\pgfpathlineto{\pgfqpoint{1.381549in}{3.171677in}}%
\pgfpathlineto{\pgfqpoint{1.382828in}{3.171647in}}%
\pgfpathlineto{\pgfqpoint{1.384771in}{3.171668in}}%
\pgfpathlineto{\pgfqpoint{1.386035in}{3.171644in}}%
\pgfpathlineto{\pgfqpoint{1.386050in}{3.171659in}}%
\pgfpathlineto{\pgfqpoint{1.392956in}{3.171649in}}%
\pgfpathlineto{\pgfqpoint{1.394867in}{3.174334in}}%
\pgfpathlineto{\pgfqpoint{1.395330in}{3.173362in}}%
\pgfpathlineto{\pgfqpoint{1.397010in}{3.169955in}}%
\pgfpathlineto{\pgfqpoint{1.397164in}{3.170452in}}%
\pgfpathlineto{\pgfqpoint{1.397611in}{3.170682in}}%
\pgfpathlineto{\pgfqpoint{1.398135in}{3.169326in}}%
\pgfpathlineto{\pgfqpoint{1.399985in}{3.173252in}}%
\pgfpathlineto{\pgfqpoint{1.400201in}{3.172775in}}%
\pgfpathlineto{\pgfqpoint{1.401357in}{3.170164in}}%
\pgfpathlineto{\pgfqpoint{1.401419in}{3.170301in}}%
\pgfpathlineto{\pgfqpoint{1.402189in}{3.170301in}}%
\pgfpathlineto{\pgfqpoint{1.401727in}{3.168498in}}%
\pgfpathlineto{\pgfqpoint{1.402374in}{3.169735in}}%
\pgfpathlineto{\pgfqpoint{1.402899in}{3.169085in}}%
\pgfpathlineto{\pgfqpoint{1.403299in}{3.170706in}}%
\pgfpathlineto{\pgfqpoint{1.405087in}{3.173007in}}%
\pgfpathlineto{\pgfqpoint{1.405195in}{3.172585in}}%
\pgfpathlineto{\pgfqpoint{1.405488in}{3.171501in}}%
\pgfpathlineto{\pgfqpoint{1.405951in}{3.173569in}}%
\pgfpathlineto{\pgfqpoint{1.406351in}{3.172024in}}%
\pgfpathlineto{\pgfqpoint{1.407554in}{3.174195in}}%
\pgfpathlineto{\pgfqpoint{1.407878in}{3.172706in}}%
\pgfpathlineto{\pgfqpoint{1.409080in}{3.170498in}}%
\pgfpathlineto{\pgfqpoint{1.408001in}{3.172791in}}%
\pgfpathlineto{\pgfqpoint{1.409095in}{3.170556in}}%
\pgfpathlineto{\pgfqpoint{1.409142in}{3.170818in}}%
\pgfpathlineto{\pgfqpoint{1.409434in}{3.169477in}}%
\pgfpathlineto{\pgfqpoint{1.410143in}{3.169934in}}%
\pgfpathlineto{\pgfqpoint{1.410236in}{3.169734in}}%
\pgfpathlineto{\pgfqpoint{1.411038in}{3.170912in}}%
\pgfpathlineto{\pgfqpoint{1.411068in}{3.170843in}}%
\pgfpathlineto{\pgfqpoint{1.411623in}{3.172154in}}%
\pgfpathlineto{\pgfqpoint{1.411685in}{3.172026in}}%
\pgfpathlineto{\pgfqpoint{1.412302in}{3.172942in}}%
\pgfpathlineto{\pgfqpoint{1.412826in}{3.171940in}}%
\pgfpathlineto{\pgfqpoint{1.414105in}{3.169575in}}%
\pgfpathlineto{\pgfqpoint{1.412918in}{3.172033in}}%
\pgfpathlineto{\pgfqpoint{1.414151in}{3.169680in}}%
\pgfpathlineto{\pgfqpoint{1.415924in}{3.172831in}}%
\pgfpathlineto{\pgfqpoint{1.415955in}{3.172764in}}%
\pgfpathlineto{\pgfqpoint{1.418067in}{3.170174in}}%
\pgfpathlineto{\pgfqpoint{1.416063in}{3.173328in}}%
\pgfpathlineto{\pgfqpoint{1.418514in}{3.171276in}}%
\pgfpathlineto{\pgfqpoint{1.420101in}{3.173746in}}%
\pgfpathlineto{\pgfqpoint{1.418575in}{3.170929in}}%
\pgfpathlineto{\pgfqpoint{1.420394in}{3.172789in}}%
\pgfpathlineto{\pgfqpoint{1.420703in}{3.172850in}}%
\pgfpathlineto{\pgfqpoint{1.421550in}{3.171879in}}%
\pgfpathlineto{\pgfqpoint{1.421612in}{3.172330in}}%
\pgfpathlineto{\pgfqpoint{1.421997in}{3.171238in}}%
\pgfpathlineto{\pgfqpoint{1.422583in}{3.171305in}}%
\pgfpathlineto{\pgfqpoint{1.423215in}{3.171027in}}%
\pgfpathlineto{\pgfqpoint{1.423508in}{3.173242in}}%
\pgfpathlineto{\pgfqpoint{1.423539in}{3.172752in}}%
\pgfpathlineto{\pgfqpoint{1.424387in}{3.173264in}}%
\pgfpathlineto{\pgfqpoint{1.423878in}{3.170515in}}%
\pgfpathlineto{\pgfqpoint{1.424510in}{3.171665in}}%
\pgfpathlineto{\pgfqpoint{1.425204in}{3.166849in}}%
\pgfpathlineto{\pgfqpoint{1.425497in}{3.174301in}}%
\pgfpathlineto{\pgfqpoint{1.426252in}{3.197360in}}%
\pgfpathlineto{\pgfqpoint{1.425820in}{3.138103in}}%
\pgfpathlineto{\pgfqpoint{1.426437in}{3.161657in}}%
\pgfpathlineto{\pgfqpoint{1.426576in}{3.145733in}}%
\pgfpathlineto{\pgfqpoint{1.427362in}{3.192501in}}%
\pgfpathlineto{\pgfqpoint{1.428210in}{3.222361in}}%
\pgfpathlineto{\pgfqpoint{1.427871in}{3.131825in}}%
\pgfpathlineto{\pgfqpoint{1.428379in}{3.170478in}}%
\pgfpathlineto{\pgfqpoint{1.428564in}{3.102703in}}%
\pgfpathlineto{\pgfqpoint{1.428888in}{3.216349in}}%
\pgfpathlineto{\pgfqpoint{1.429458in}{3.191560in}}%
\pgfpathlineto{\pgfqpoint{1.429936in}{3.111507in}}%
\pgfpathlineto{\pgfqpoint{1.430321in}{3.222927in}}%
\pgfpathlineto{\pgfqpoint{1.430414in}{3.218848in}}%
\pgfpathlineto{\pgfqpoint{1.430769in}{3.240858in}}%
\pgfpathlineto{\pgfqpoint{1.430491in}{3.122243in}}%
\pgfpathlineto{\pgfqpoint{1.431031in}{3.144660in}}%
\pgfpathlineto{\pgfqpoint{1.431154in}{3.024909in}}%
\pgfpathlineto{\pgfqpoint{1.431508in}{3.280950in}}%
\pgfpathlineto{\pgfqpoint{1.432110in}{3.219616in}}%
\pgfpathlineto{\pgfqpoint{1.432187in}{3.152553in}}%
\pgfpathlineto{\pgfqpoint{1.432387in}{3.264904in}}%
\pgfpathlineto{\pgfqpoint{1.432418in}{3.293907in}}%
\pgfpathlineto{\pgfqpoint{1.432603in}{3.083409in}}%
\pgfpathlineto{\pgfqpoint{1.433420in}{3.141268in}}%
\pgfpathlineto{\pgfqpoint{1.433482in}{3.239331in}}%
\pgfpathlineto{\pgfqpoint{1.434160in}{3.292932in}}%
\pgfpathlineto{\pgfqpoint{1.434345in}{3.057140in}}%
\pgfpathlineto{\pgfqpoint{1.434530in}{3.169950in}}%
\pgfpathlineto{\pgfqpoint{1.434992in}{3.054683in}}%
\pgfpathlineto{\pgfqpoint{1.434792in}{3.290212in}}%
\pgfpathlineto{\pgfqpoint{1.435670in}{3.127345in}}%
\pgfpathlineto{\pgfqpoint{1.436380in}{3.297305in}}%
\pgfpathlineto{\pgfqpoint{1.436318in}{3.083576in}}%
\pgfpathlineto{\pgfqpoint{1.436857in}{3.157448in}}%
\pgfpathlineto{\pgfqpoint{1.436996in}{3.038575in}}%
\pgfpathlineto{\pgfqpoint{1.437274in}{3.314224in}}%
\pgfpathlineto{\pgfqpoint{1.437921in}{3.214121in}}%
\pgfpathlineto{\pgfqpoint{1.438830in}{3.276634in}}%
\pgfpathlineto{\pgfqpoint{1.438106in}{3.063770in}}%
\pgfpathlineto{\pgfqpoint{1.438969in}{3.169132in}}%
\pgfpathlineto{\pgfqpoint{1.439278in}{3.037788in}}%
\pgfpathlineto{\pgfqpoint{1.440002in}{3.299286in}}%
\pgfpathlineto{\pgfqpoint{1.440064in}{3.153703in}}%
\pgfpathlineto{\pgfqpoint{1.441266in}{3.359276in}}%
\pgfpathlineto{\pgfqpoint{1.440449in}{3.020418in}}%
\pgfpathlineto{\pgfqpoint{1.441281in}{3.335388in}}%
\pgfpathlineto{\pgfqpoint{1.441744in}{2.980226in}}%
\pgfpathlineto{\pgfqpoint{1.442407in}{3.188828in}}%
\pgfpathlineto{\pgfqpoint{1.442453in}{3.332059in}}%
\pgfpathlineto{\pgfqpoint{1.442761in}{3.021404in}}%
\pgfpathlineto{\pgfqpoint{1.443517in}{3.240220in}}%
\pgfpathlineto{\pgfqpoint{1.443964in}{3.007735in}}%
\pgfpathlineto{\pgfqpoint{1.443609in}{3.313335in}}%
\pgfpathlineto{\pgfqpoint{1.444611in}{3.242034in}}%
\pgfpathlineto{\pgfqpoint{1.444796in}{3.260628in}}%
\pgfpathlineto{\pgfqpoint{1.445120in}{3.083587in}}%
\pgfpathlineto{\pgfqpoint{1.445382in}{3.169813in}}%
\pgfpathlineto{\pgfqpoint{1.445675in}{3.042744in}}%
\pgfpathlineto{\pgfqpoint{1.446045in}{3.327252in}}%
\pgfpathlineto{\pgfqpoint{1.446492in}{3.151885in}}%
\pgfpathlineto{\pgfqpoint{1.446970in}{3.275528in}}%
\pgfpathlineto{\pgfqpoint{1.446661in}{3.020223in}}%
\pgfpathlineto{\pgfqpoint{1.447648in}{3.196601in}}%
\pgfpathlineto{\pgfqpoint{1.448681in}{2.974899in}}%
\pgfpathlineto{\pgfqpoint{1.448496in}{3.321412in}}%
\pgfpathlineto{\pgfqpoint{1.448788in}{3.123049in}}%
\pgfpathlineto{\pgfqpoint{1.448943in}{3.021311in}}%
\pgfpathlineto{\pgfqpoint{1.449420in}{3.293052in}}%
\pgfpathlineto{\pgfqpoint{1.449898in}{3.115741in}}%
\pgfpathlineto{\pgfqpoint{1.450823in}{3.300286in}}%
\pgfpathlineto{\pgfqpoint{1.450237in}{3.051080in}}%
\pgfpathlineto{\pgfqpoint{1.450977in}{3.121712in}}%
\pgfpathlineto{\pgfqpoint{1.451316in}{3.024672in}}%
\pgfpathlineto{\pgfqpoint{1.451733in}{3.293378in}}%
\pgfpathlineto{\pgfqpoint{1.452087in}{3.121006in}}%
\pgfpathlineto{\pgfqpoint{1.452673in}{3.015488in}}%
\pgfpathlineto{\pgfqpoint{1.453043in}{3.343357in}}%
\pgfpathlineto{\pgfqpoint{1.453105in}{3.285621in}}%
\pgfpathlineto{\pgfqpoint{1.453151in}{3.378892in}}%
\pgfpathlineto{\pgfqpoint{1.453644in}{2.947051in}}%
\pgfpathlineto{\pgfqpoint{1.454184in}{3.258534in}}%
\pgfpathlineto{\pgfqpoint{1.454523in}{3.072606in}}%
\pgfpathlineto{\pgfqpoint{1.454708in}{3.261832in}}%
\pgfpathlineto{\pgfqpoint{1.455294in}{3.242093in}}%
\pgfpathlineto{\pgfqpoint{1.455355in}{3.392914in}}%
\pgfpathlineto{\pgfqpoint{1.455818in}{3.008366in}}%
\pgfpathlineto{\pgfqpoint{1.456419in}{3.297374in}}%
\pgfpathlineto{\pgfqpoint{1.457267in}{3.046390in}}%
\pgfpathlineto{\pgfqpoint{1.457606in}{3.218574in}}%
\pgfpathlineto{\pgfqpoint{1.457698in}{3.362440in}}%
\pgfpathlineto{\pgfqpoint{1.458284in}{2.962135in}}%
\pgfpathlineto{\pgfqpoint{1.458731in}{3.273810in}}%
\pgfpathlineto{\pgfqpoint{1.459687in}{3.017028in}}%
\pgfpathlineto{\pgfqpoint{1.459872in}{3.133008in}}%
\pgfpathlineto{\pgfqpoint{1.460041in}{3.470951in}}%
\pgfpathlineto{\pgfqpoint{1.460457in}{2.979921in}}%
\pgfpathlineto{\pgfqpoint{1.461028in}{3.353260in}}%
\pgfpathlineto{\pgfqpoint{1.462184in}{3.006777in}}%
\pgfpathlineto{\pgfqpoint{1.462230in}{3.241956in}}%
\pgfpathlineto{\pgfqpoint{1.462384in}{3.406358in}}%
\pgfpathlineto{\pgfqpoint{1.462955in}{2.969386in}}%
\pgfpathlineto{\pgfqpoint{1.463355in}{3.328997in}}%
\pgfpathlineto{\pgfqpoint{1.463725in}{3.007226in}}%
\pgfpathlineto{\pgfqpoint{1.464542in}{3.203757in}}%
\pgfpathlineto{\pgfqpoint{1.464758in}{3.355390in}}%
\pgfpathlineto{\pgfqpoint{1.465282in}{2.956087in}}%
\pgfpathlineto{\pgfqpoint{1.465652in}{3.205184in}}%
\pgfpathlineto{\pgfqpoint{1.465837in}{3.280873in}}%
\pgfpathlineto{\pgfqpoint{1.466639in}{3.040750in}}%
\pgfpathlineto{\pgfqpoint{1.467610in}{2.917088in}}%
\pgfpathlineto{\pgfqpoint{1.467024in}{3.370883in}}%
\pgfpathlineto{\pgfqpoint{1.467702in}{3.111797in}}%
\pgfpathlineto{\pgfqpoint{1.468381in}{3.047543in}}%
\pgfpathlineto{\pgfqpoint{1.467903in}{3.319546in}}%
\pgfpathlineto{\pgfqpoint{1.468643in}{3.189614in}}%
\pgfpathlineto{\pgfqpoint{1.469414in}{3.336265in}}%
\pgfpathlineto{\pgfqpoint{1.469028in}{3.014693in}}%
\pgfpathlineto{\pgfqpoint{1.469706in}{3.097776in}}%
\pgfpathlineto{\pgfqpoint{1.469938in}{3.010346in}}%
\pgfpathlineto{\pgfqpoint{1.469861in}{3.290136in}}%
\pgfpathlineto{\pgfqpoint{1.470739in}{3.178105in}}%
\pgfpathlineto{\pgfqpoint{1.471710in}{3.441622in}}%
\pgfpathlineto{\pgfqpoint{1.471340in}{2.981849in}}%
\pgfpathlineto{\pgfqpoint{1.471864in}{3.242273in}}%
\pgfpathlineto{\pgfqpoint{1.472681in}{3.351927in}}%
\pgfpathlineto{\pgfqpoint{1.472281in}{2.952914in}}%
\pgfpathlineto{\pgfqpoint{1.472836in}{3.227003in}}%
\pgfpathlineto{\pgfqpoint{1.473514in}{2.961563in}}%
\pgfpathlineto{\pgfqpoint{1.473899in}{3.263527in}}%
\pgfpathlineto{\pgfqpoint{1.473945in}{3.192295in}}%
\pgfpathlineto{\pgfqpoint{1.474023in}{3.379731in}}%
\pgfpathlineto{\pgfqpoint{1.474423in}{2.966093in}}%
\pgfpathlineto{\pgfqpoint{1.475040in}{3.181699in}}%
\pgfpathlineto{\pgfqpoint{1.475857in}{3.021397in}}%
\pgfpathlineto{\pgfqpoint{1.475734in}{3.286465in}}%
\pgfpathlineto{\pgfqpoint{1.476165in}{3.118856in}}%
\pgfpathlineto{\pgfqpoint{1.476396in}{3.350595in}}%
\pgfpathlineto{\pgfqpoint{1.476782in}{2.976356in}}%
\pgfpathlineto{\pgfqpoint{1.477321in}{3.278624in}}%
\pgfpathlineto{\pgfqpoint{1.478169in}{2.994357in}}%
\pgfpathlineto{\pgfqpoint{1.477614in}{3.302297in}}%
\pgfpathlineto{\pgfqpoint{1.478493in}{3.109406in}}%
\pgfpathlineto{\pgfqpoint{1.478662in}{3.406533in}}%
\pgfpathlineto{\pgfqpoint{1.479094in}{2.963484in}}%
\pgfpathlineto{\pgfqpoint{1.479634in}{3.327410in}}%
\pgfpathlineto{\pgfqpoint{1.479649in}{3.340433in}}%
\pgfpathlineto{\pgfqpoint{1.480019in}{3.088491in}}%
\pgfpathlineto{\pgfqpoint{1.480451in}{3.094825in}}%
\pgfpathlineto{\pgfqpoint{1.481422in}{2.934244in}}%
\pgfpathlineto{\pgfqpoint{1.481021in}{3.375909in}}%
\pgfpathlineto{\pgfqpoint{1.481483in}{3.213117in}}%
\pgfpathlineto{\pgfqpoint{1.481992in}{3.309284in}}%
\pgfpathlineto{\pgfqpoint{1.481684in}{3.093977in}}%
\pgfpathlineto{\pgfqpoint{1.482177in}{3.151748in}}%
\pgfpathlineto{\pgfqpoint{1.482824in}{3.004252in}}%
\pgfpathlineto{\pgfqpoint{1.482470in}{3.300018in}}%
\pgfpathlineto{\pgfqpoint{1.483241in}{3.210113in}}%
\pgfpathlineto{\pgfqpoint{1.483318in}{3.384819in}}%
\pgfpathlineto{\pgfqpoint{1.483734in}{2.919117in}}%
\pgfpathlineto{\pgfqpoint{1.484335in}{3.194136in}}%
\pgfpathlineto{\pgfqpoint{1.484674in}{2.992156in}}%
\pgfpathlineto{\pgfqpoint{1.484443in}{3.353885in}}%
\pgfpathlineto{\pgfqpoint{1.485460in}{3.090625in}}%
\pgfpathlineto{\pgfqpoint{1.486462in}{3.341254in}}%
\pgfpathlineto{\pgfqpoint{1.486092in}{2.981509in}}%
\pgfpathlineto{\pgfqpoint{1.486570in}{3.115814in}}%
\pgfpathlineto{\pgfqpoint{1.486586in}{3.115233in}}%
\pgfpathlineto{\pgfqpoint{1.486601in}{3.124384in}}%
\pgfpathlineto{\pgfqpoint{1.487218in}{3.277970in}}%
\pgfpathlineto{\pgfqpoint{1.487618in}{2.929791in}}%
\pgfpathlineto{\pgfqpoint{1.487711in}{3.132524in}}%
\pgfpathlineto{\pgfqpoint{1.488543in}{2.972823in}}%
\pgfpathlineto{\pgfqpoint{1.488019in}{3.370679in}}%
\pgfpathlineto{\pgfqpoint{1.488744in}{3.196023in}}%
\pgfpathlineto{\pgfqpoint{1.489114in}{3.249777in}}%
\pgfpathlineto{\pgfqpoint{1.489669in}{3.091880in}}%
\pgfpathlineto{\pgfqpoint{1.489807in}{3.120113in}}%
\pgfpathlineto{\pgfqpoint{1.489884in}{3.125242in}}%
\pgfpathlineto{\pgfqpoint{1.489915in}{3.049404in}}%
\pgfpathlineto{\pgfqpoint{1.490886in}{2.961125in}}%
\pgfpathlineto{\pgfqpoint{1.490301in}{3.414241in}}%
\pgfpathlineto{\pgfqpoint{1.490933in}{3.158949in}}%
\pgfpathlineto{\pgfqpoint{1.491272in}{3.305580in}}%
\pgfpathlineto{\pgfqpoint{1.491503in}{3.034979in}}%
\pgfpathlineto{\pgfqpoint{1.492042in}{3.171364in}}%
\pgfpathlineto{\pgfqpoint{1.492674in}{3.361291in}}%
\pgfpathlineto{\pgfqpoint{1.492443in}{3.002490in}}%
\pgfpathlineto{\pgfqpoint{1.493122in}{3.147042in}}%
\pgfpathlineto{\pgfqpoint{1.493831in}{3.053101in}}%
\pgfpathlineto{\pgfqpoint{1.494154in}{3.255551in}}%
\pgfpathlineto{\pgfqpoint{1.494201in}{3.414177in}}%
\pgfpathlineto{\pgfqpoint{1.494771in}{3.015213in}}%
\pgfpathlineto{\pgfqpoint{1.495233in}{3.167180in}}%
\pgfpathlineto{\pgfqpoint{1.496235in}{3.285330in}}%
\pgfpathlineto{\pgfqpoint{1.495542in}{3.013756in}}%
\pgfpathlineto{\pgfqpoint{1.496297in}{3.152970in}}%
\pgfpathlineto{\pgfqpoint{1.496759in}{3.028762in}}%
\pgfpathlineto{\pgfqpoint{1.496574in}{3.325794in}}%
\pgfpathlineto{\pgfqpoint{1.497407in}{3.146953in}}%
\pgfpathlineto{\pgfqpoint{1.497869in}{3.010101in}}%
\pgfpathlineto{\pgfqpoint{1.498116in}{3.327058in}}%
\pgfpathlineto{\pgfqpoint{1.498517in}{3.144550in}}%
\pgfpathlineto{\pgfqpoint{1.498887in}{3.309332in}}%
\pgfpathlineto{\pgfqpoint{1.499287in}{3.007395in}}%
\pgfpathlineto{\pgfqpoint{1.499642in}{3.266487in}}%
\pgfpathlineto{\pgfqpoint{1.500305in}{3.059874in}}%
\pgfpathlineto{\pgfqpoint{1.500844in}{3.119045in}}%
\pgfpathlineto{\pgfqpoint{1.501168in}{3.356293in}}%
\pgfpathlineto{\pgfqpoint{1.501600in}{2.997566in}}%
\pgfpathlineto{\pgfqpoint{1.502031in}{3.354645in}}%
\pgfpathlineto{\pgfqpoint{1.502370in}{3.049908in}}%
\pgfpathlineto{\pgfqpoint{1.503218in}{3.117096in}}%
\pgfpathlineto{\pgfqpoint{1.503912in}{2.983860in}}%
\pgfpathlineto{\pgfqpoint{1.503511in}{3.341203in}}%
\pgfpathlineto{\pgfqpoint{1.504266in}{3.161213in}}%
\pgfpathlineto{\pgfqpoint{1.505053in}{3.326238in}}%
\pgfpathlineto{\pgfqpoint{1.505191in}{3.057706in}}%
\pgfpathlineto{\pgfqpoint{1.505315in}{3.073123in}}%
\pgfpathlineto{\pgfqpoint{1.505330in}{3.063956in}}%
\pgfpathlineto{\pgfqpoint{1.505916in}{3.333293in}}%
\pgfpathlineto{\pgfqpoint{1.506008in}{3.313079in}}%
\pgfpathlineto{\pgfqpoint{1.506024in}{3.340211in}}%
\pgfpathlineto{\pgfqpoint{1.506224in}{2.960316in}}%
\pgfpathlineto{\pgfqpoint{1.507026in}{3.165538in}}%
\pgfpathlineto{\pgfqpoint{1.507812in}{3.018320in}}%
\pgfpathlineto{\pgfqpoint{1.507426in}{3.269717in}}%
\pgfpathlineto{\pgfqpoint{1.508105in}{3.181509in}}%
\pgfpathlineto{\pgfqpoint{1.508583in}{3.036954in}}%
\pgfpathlineto{\pgfqpoint{1.508197in}{3.275809in}}%
\pgfpathlineto{\pgfqpoint{1.508829in}{3.194657in}}%
\pgfpathlineto{\pgfqpoint{1.508953in}{3.369480in}}%
\pgfpathlineto{\pgfqpoint{1.509245in}{3.071647in}}%
\pgfpathlineto{\pgfqpoint{1.509955in}{3.264774in}}%
\pgfpathlineto{\pgfqpoint{1.510124in}{2.975309in}}%
\pgfpathlineto{\pgfqpoint{1.510509in}{3.352674in}}%
\pgfpathlineto{\pgfqpoint{1.511126in}{3.119397in}}%
\pgfpathlineto{\pgfqpoint{1.511912in}{3.309839in}}%
\pgfpathlineto{\pgfqpoint{1.512174in}{3.086990in}}%
\pgfpathlineto{\pgfqpoint{1.512251in}{3.168682in}}%
\pgfpathlineto{\pgfqpoint{1.513392in}{3.001824in}}%
\pgfpathlineto{\pgfqpoint{1.512852in}{3.356987in}}%
\pgfpathlineto{\pgfqpoint{1.513407in}{3.051696in}}%
\pgfpathlineto{\pgfqpoint{1.513762in}{3.280931in}}%
\pgfpathlineto{\pgfqpoint{1.514009in}{3.007096in}}%
\pgfpathlineto{\pgfqpoint{1.514548in}{3.192202in}}%
\pgfpathlineto{\pgfqpoint{1.515566in}{3.022547in}}%
\pgfpathlineto{\pgfqpoint{1.515165in}{3.291084in}}%
\pgfpathlineto{\pgfqpoint{1.515766in}{3.139601in}}%
\pgfpathlineto{\pgfqpoint{1.516691in}{3.313884in}}%
\pgfpathlineto{\pgfqpoint{1.516336in}{3.022523in}}%
\pgfpathlineto{\pgfqpoint{1.516891in}{3.181298in}}%
\pgfpathlineto{\pgfqpoint{1.517277in}{3.022535in}}%
\pgfpathlineto{\pgfqpoint{1.517492in}{3.283211in}}%
\pgfpathlineto{\pgfqpoint{1.517770in}{3.260121in}}%
\pgfpathlineto{\pgfqpoint{1.517785in}{3.265413in}}%
\pgfpathlineto{\pgfqpoint{1.517909in}{2.987128in}}%
\pgfpathlineto{\pgfqpoint{1.518571in}{3.188163in}}%
\pgfpathlineto{\pgfqpoint{1.519465in}{3.040299in}}%
\pgfpathlineto{\pgfqpoint{1.519003in}{3.258360in}}%
\pgfpathlineto{\pgfqpoint{1.519666in}{3.194827in}}%
\pgfpathlineto{\pgfqpoint{1.520606in}{3.303129in}}%
\pgfpathlineto{\pgfqpoint{1.520236in}{2.966381in}}%
\pgfpathlineto{\pgfqpoint{1.520760in}{3.228190in}}%
\pgfpathlineto{\pgfqpoint{1.521809in}{3.028168in}}%
\pgfpathlineto{\pgfqpoint{1.521269in}{3.289837in}}%
\pgfpathlineto{\pgfqpoint{1.521901in}{3.125060in}}%
\pgfpathlineto{\pgfqpoint{1.522040in}{3.277537in}}%
\pgfpathlineto{\pgfqpoint{1.522733in}{3.051723in}}%
\pgfpathlineto{\pgfqpoint{1.523103in}{3.209063in}}%
\pgfpathlineto{\pgfqpoint{1.524136in}{3.006252in}}%
\pgfpathlineto{\pgfqpoint{1.523674in}{3.307503in}}%
\pgfpathlineto{\pgfqpoint{1.524290in}{3.068316in}}%
\pgfpathlineto{\pgfqpoint{1.524522in}{3.308512in}}%
\pgfpathlineto{\pgfqpoint{1.524922in}{3.054506in}}%
\pgfpathlineto{\pgfqpoint{1.525477in}{3.181347in}}%
\pgfpathlineto{\pgfqpoint{1.525523in}{3.251565in}}%
\pgfpathlineto{\pgfqpoint{1.525662in}{3.097487in}}%
\pgfpathlineto{\pgfqpoint{1.525693in}{3.064519in}}%
\pgfpathlineto{\pgfqpoint{1.525940in}{3.301805in}}%
\pgfpathlineto{\pgfqpoint{1.526695in}{3.186557in}}%
\pgfpathlineto{\pgfqpoint{1.527604in}{3.310763in}}%
\pgfpathlineto{\pgfqpoint{1.527420in}{3.065316in}}%
\pgfpathlineto{\pgfqpoint{1.527805in}{3.220872in}}%
\pgfpathlineto{\pgfqpoint{1.528745in}{3.048312in}}%
\pgfpathlineto{\pgfqpoint{1.528545in}{3.298576in}}%
\pgfpathlineto{\pgfqpoint{1.528992in}{3.100220in}}%
\pgfpathlineto{\pgfqpoint{1.530102in}{3.280364in}}%
\pgfpathlineto{\pgfqpoint{1.529716in}{3.027484in}}%
\pgfpathlineto{\pgfqpoint{1.530133in}{3.206141in}}%
\pgfpathlineto{\pgfqpoint{1.531242in}{3.079284in}}%
\pgfpathlineto{\pgfqpoint{1.530872in}{3.268370in}}%
\pgfpathlineto{\pgfqpoint{1.531319in}{3.106407in}}%
\pgfpathlineto{\pgfqpoint{1.531504in}{3.351367in}}%
\pgfpathlineto{\pgfqpoint{1.531905in}{3.098662in}}%
\pgfpathlineto{\pgfqpoint{1.532537in}{3.184846in}}%
\pgfpathlineto{\pgfqpoint{1.533632in}{3.029363in}}%
\pgfpathlineto{\pgfqpoint{1.533061in}{3.302270in}}%
\pgfpathlineto{\pgfqpoint{1.533678in}{3.102909in}}%
\pgfpathlineto{\pgfqpoint{1.533817in}{3.283485in}}%
\pgfpathlineto{\pgfqpoint{1.534202in}{3.082901in}}%
\pgfpathlineto{\pgfqpoint{1.534803in}{3.213810in}}%
\pgfpathlineto{\pgfqpoint{1.535189in}{3.067591in}}%
\pgfpathlineto{\pgfqpoint{1.535404in}{3.306730in}}%
\pgfpathlineto{\pgfqpoint{1.536021in}{3.162899in}}%
\pgfpathlineto{\pgfqpoint{1.536961in}{3.275544in}}%
\pgfpathlineto{\pgfqpoint{1.536561in}{3.077417in}}%
\pgfpathlineto{\pgfqpoint{1.537131in}{3.175603in}}%
\pgfpathlineto{\pgfqpoint{1.537717in}{3.258411in}}%
\pgfpathlineto{\pgfqpoint{1.537547in}{3.043107in}}%
\pgfpathlineto{\pgfqpoint{1.538210in}{3.153498in}}%
\pgfpathlineto{\pgfqpoint{1.539104in}{3.072087in}}%
\pgfpathlineto{\pgfqpoint{1.538395in}{3.256886in}}%
\pgfpathlineto{\pgfqpoint{1.539227in}{3.207594in}}%
\pgfpathlineto{\pgfqpoint{1.539258in}{3.269295in}}%
\pgfpathlineto{\pgfqpoint{1.539628in}{3.074705in}}%
\pgfpathlineto{\pgfqpoint{1.540337in}{3.208487in}}%
\pgfpathlineto{\pgfqpoint{1.541432in}{3.058360in}}%
\pgfpathlineto{\pgfqpoint{1.540815in}{3.259149in}}%
\pgfpathlineto{\pgfqpoint{1.541478in}{3.094429in}}%
\pgfpathlineto{\pgfqpoint{1.542295in}{3.280318in}}%
\pgfpathlineto{\pgfqpoint{1.542233in}{3.093570in}}%
\pgfpathlineto{\pgfqpoint{1.542619in}{3.197903in}}%
\pgfpathlineto{\pgfqpoint{1.543173in}{3.261897in}}%
\pgfpathlineto{\pgfqpoint{1.543775in}{3.094450in}}%
\pgfpathlineto{\pgfqpoint{1.544700in}{3.270597in}}%
\pgfpathlineto{\pgfqpoint{1.544545in}{3.064059in}}%
\pgfpathlineto{\pgfqpoint{1.544992in}{3.152802in}}%
\pgfpathlineto{\pgfqpoint{1.545332in}{3.077440in}}%
\pgfpathlineto{\pgfqpoint{1.545486in}{3.255298in}}%
\pgfpathlineto{\pgfqpoint{1.546133in}{3.116927in}}%
\pgfpathlineto{\pgfqpoint{1.546241in}{3.256584in}}%
\pgfpathlineto{\pgfqpoint{1.546888in}{3.108697in}}%
\pgfpathlineto{\pgfqpoint{1.547289in}{3.177871in}}%
\pgfpathlineto{\pgfqpoint{1.547397in}{3.048328in}}%
\pgfpathlineto{\pgfqpoint{1.547783in}{3.271398in}}%
\pgfpathlineto{\pgfqpoint{1.548476in}{3.077979in}}%
\pgfpathlineto{\pgfqpoint{1.548599in}{3.239641in}}%
\pgfpathlineto{\pgfqpoint{1.549617in}{3.209896in}}%
\pgfpathlineto{\pgfqpoint{1.550002in}{3.074159in}}%
\pgfpathlineto{\pgfqpoint{1.550156in}{3.292429in}}%
\pgfpathlineto{\pgfqpoint{1.550788in}{3.101966in}}%
\pgfpathlineto{\pgfqpoint{1.551698in}{3.280466in}}%
\pgfpathlineto{\pgfqpoint{1.551297in}{3.080589in}}%
\pgfpathlineto{\pgfqpoint{1.551960in}{3.180553in}}%
\pgfpathlineto{\pgfqpoint{1.552314in}{3.075288in}}%
\pgfpathlineto{\pgfqpoint{1.552484in}{3.256620in}}%
\pgfpathlineto{\pgfqpoint{1.553070in}{3.157342in}}%
\pgfpathlineto{\pgfqpoint{1.554056in}{3.268498in}}%
\pgfpathlineto{\pgfqpoint{1.553918in}{3.092237in}}%
\pgfpathlineto{\pgfqpoint{1.554180in}{3.163481in}}%
\pgfpathlineto{\pgfqpoint{1.554688in}{3.097590in}}%
\pgfpathlineto{\pgfqpoint{1.554765in}{3.251075in}}%
\pgfpathlineto{\pgfqpoint{1.555043in}{3.231833in}}%
\pgfpathlineto{\pgfqpoint{1.555058in}{3.235090in}}%
\pgfpathlineto{\pgfqpoint{1.555197in}{3.104162in}}%
\pgfpathlineto{\pgfqpoint{1.555767in}{3.153429in}}%
\pgfpathlineto{\pgfqpoint{1.555783in}{3.153382in}}%
\pgfpathlineto{\pgfqpoint{1.556384in}{3.269629in}}%
\pgfpathlineto{\pgfqpoint{1.556230in}{3.059524in}}%
\pgfpathlineto{\pgfqpoint{1.556877in}{3.144861in}}%
\pgfpathlineto{\pgfqpoint{1.557771in}{3.083434in}}%
\pgfpathlineto{\pgfqpoint{1.557355in}{3.241937in}}%
\pgfpathlineto{\pgfqpoint{1.557895in}{3.217792in}}%
\pgfpathlineto{\pgfqpoint{1.558372in}{3.240722in}}%
\pgfpathlineto{\pgfqpoint{1.558295in}{3.104971in}}%
\pgfpathlineto{\pgfqpoint{1.558527in}{3.128336in}}%
\pgfpathlineto{\pgfqpoint{1.559082in}{3.090859in}}%
\pgfpathlineto{\pgfqpoint{1.558681in}{3.276918in}}%
\pgfpathlineto{\pgfqpoint{1.559529in}{3.198978in}}%
\pgfpathlineto{\pgfqpoint{1.559667in}{3.246597in}}%
\pgfpathlineto{\pgfqpoint{1.560145in}{3.034536in}}%
\pgfpathlineto{\pgfqpoint{1.560608in}{3.181052in}}%
\pgfpathlineto{\pgfqpoint{1.561687in}{3.045551in}}%
\pgfpathlineto{\pgfqpoint{1.560746in}{3.234131in}}%
\pgfpathlineto{\pgfqpoint{1.561748in}{3.173757in}}%
\pgfpathlineto{\pgfqpoint{1.562272in}{3.253235in}}%
\pgfpathlineto{\pgfqpoint{1.562473in}{3.076314in}}%
\pgfpathlineto{\pgfqpoint{1.562843in}{3.159023in}}%
\pgfpathlineto{\pgfqpoint{1.562982in}{3.119846in}}%
\pgfpathlineto{\pgfqpoint{1.562920in}{3.229911in}}%
\pgfpathlineto{\pgfqpoint{1.563305in}{3.182738in}}%
\pgfpathlineto{\pgfqpoint{1.563336in}{3.253912in}}%
\pgfpathlineto{\pgfqpoint{1.564045in}{3.043495in}}%
\pgfpathlineto{\pgfqpoint{1.564400in}{3.152069in}}%
\pgfpathlineto{\pgfqpoint{1.564893in}{3.271248in}}%
\pgfpathlineto{\pgfqpoint{1.565063in}{3.075169in}}%
\pgfpathlineto{\pgfqpoint{1.565556in}{3.165655in}}%
\pgfpathlineto{\pgfqpoint{1.566373in}{3.049227in}}%
\pgfpathlineto{\pgfqpoint{1.566450in}{3.250111in}}%
\pgfpathlineto{\pgfqpoint{1.566650in}{3.189824in}}%
\pgfpathlineto{\pgfqpoint{1.567236in}{3.241439in}}%
\pgfpathlineto{\pgfqpoint{1.567144in}{3.085968in}}%
\pgfpathlineto{\pgfqpoint{1.567591in}{3.133922in}}%
\pgfpathlineto{\pgfqpoint{1.568700in}{3.052735in}}%
\pgfpathlineto{\pgfqpoint{1.567745in}{3.242514in}}%
\pgfpathlineto{\pgfqpoint{1.568731in}{3.089972in}}%
\pgfpathlineto{\pgfqpoint{1.568824in}{3.253641in}}%
\pgfpathlineto{\pgfqpoint{1.569872in}{3.171557in}}%
\pgfpathlineto{\pgfqpoint{1.570057in}{3.261077in}}%
\pgfpathlineto{\pgfqpoint{1.569980in}{3.052642in}}%
\pgfpathlineto{\pgfqpoint{1.570889in}{3.157613in}}%
\pgfpathlineto{\pgfqpoint{1.570966in}{3.053404in}}%
\pgfpathlineto{\pgfqpoint{1.571383in}{3.241480in}}%
\pgfpathlineto{\pgfqpoint{1.571984in}{3.163845in}}%
\pgfpathlineto{\pgfqpoint{1.572415in}{3.269831in}}%
\pgfpathlineto{\pgfqpoint{1.572616in}{3.060730in}}%
\pgfpathlineto{\pgfqpoint{1.573063in}{3.147395in}}%
\pgfpathlineto{\pgfqpoint{1.573171in}{3.254110in}}%
\pgfpathlineto{\pgfqpoint{1.573371in}{3.078339in}}%
\pgfpathlineto{\pgfqpoint{1.573849in}{3.128877in}}%
\pgfpathlineto{\pgfqpoint{1.574913in}{3.063630in}}%
\pgfpathlineto{\pgfqpoint{1.574728in}{3.267116in}}%
\pgfpathlineto{\pgfqpoint{1.574943in}{3.107453in}}%
\pgfpathlineto{\pgfqpoint{1.575945in}{3.256561in}}%
\pgfpathlineto{\pgfqpoint{1.575437in}{3.095875in}}%
\pgfpathlineto{\pgfqpoint{1.576115in}{3.230005in}}%
\pgfpathlineto{\pgfqpoint{1.576469in}{3.062635in}}%
\pgfpathlineto{\pgfqpoint{1.576285in}{3.264994in}}%
\pgfpathlineto{\pgfqpoint{1.577286in}{3.121653in}}%
\pgfpathlineto{\pgfqpoint{1.577672in}{3.239872in}}%
\pgfpathlineto{\pgfqpoint{1.578057in}{3.088014in}}%
\pgfpathlineto{\pgfqpoint{1.578489in}{3.168825in}}%
\pgfpathlineto{\pgfqpoint{1.578828in}{3.014666in}}%
\pgfpathlineto{\pgfqpoint{1.578643in}{3.260457in}}%
\pgfpathlineto{\pgfqpoint{1.579614in}{3.107844in}}%
\pgfpathlineto{\pgfqpoint{1.580616in}{3.256956in}}%
\pgfpathlineto{\pgfqpoint{1.580369in}{3.009269in}}%
\pgfpathlineto{\pgfqpoint{1.580739in}{3.187351in}}%
\pgfpathlineto{\pgfqpoint{1.581140in}{3.066023in}}%
\pgfpathlineto{\pgfqpoint{1.580971in}{3.247531in}}%
\pgfpathlineto{\pgfqpoint{1.581818in}{3.191355in}}%
\pgfpathlineto{\pgfqpoint{1.582713in}{3.019267in}}%
\pgfpathlineto{\pgfqpoint{1.582944in}{3.244297in}}%
\pgfpathlineto{\pgfqpoint{1.583499in}{3.093717in}}%
\pgfpathlineto{\pgfqpoint{1.584115in}{3.210955in}}%
\pgfpathlineto{\pgfqpoint{1.584131in}{3.212517in}}%
\pgfpathlineto{\pgfqpoint{1.584239in}{3.115625in}}%
\pgfpathlineto{\pgfqpoint{1.585040in}{3.028818in}}%
\pgfpathlineto{\pgfqpoint{1.584547in}{3.229875in}}%
\pgfpathlineto{\pgfqpoint{1.585318in}{3.165389in}}%
\pgfpathlineto{\pgfqpoint{1.585426in}{3.240824in}}%
\pgfpathlineto{\pgfqpoint{1.585826in}{3.073276in}}%
\pgfpathlineto{\pgfqpoint{1.586458in}{3.191889in}}%
\pgfpathlineto{\pgfqpoint{1.586597in}{3.051486in}}%
\pgfpathlineto{\pgfqpoint{1.586751in}{3.239491in}}%
\pgfpathlineto{\pgfqpoint{1.587476in}{3.221912in}}%
\pgfpathlineto{\pgfqpoint{1.587738in}{3.282386in}}%
\pgfpathlineto{\pgfqpoint{1.588154in}{3.078370in}}%
\pgfpathlineto{\pgfqpoint{1.588508in}{3.225478in}}%
\pgfpathlineto{\pgfqpoint{1.588940in}{3.082089in}}%
\pgfpathlineto{\pgfqpoint{1.588724in}{3.241164in}}%
\pgfpathlineto{\pgfqpoint{1.589649in}{3.103979in}}%
\pgfpathlineto{\pgfqpoint{1.590050in}{3.245190in}}%
\pgfpathlineto{\pgfqpoint{1.590497in}{3.053428in}}%
\pgfpathlineto{\pgfqpoint{1.590913in}{3.143029in}}%
\pgfpathlineto{\pgfqpoint{1.592038in}{3.063219in}}%
\pgfpathlineto{\pgfqpoint{1.591422in}{3.245150in}}%
\pgfpathlineto{\pgfqpoint{1.592069in}{3.094669in}}%
\pgfpathlineto{\pgfqpoint{1.592393in}{3.264145in}}%
\pgfpathlineto{\pgfqpoint{1.592825in}{3.082918in}}%
\pgfpathlineto{\pgfqpoint{1.593210in}{3.195888in}}%
\pgfpathlineto{\pgfqpoint{1.593580in}{3.066024in}}%
\pgfpathlineto{\pgfqpoint{1.593950in}{3.239061in}}%
\pgfpathlineto{\pgfqpoint{1.594397in}{3.098572in}}%
\pgfpathlineto{\pgfqpoint{1.594613in}{3.260815in}}%
\pgfpathlineto{\pgfqpoint{1.595121in}{3.033670in}}%
\pgfpathlineto{\pgfqpoint{1.595646in}{3.175471in}}%
\pgfpathlineto{\pgfqpoint{1.596709in}{3.075953in}}%
\pgfpathlineto{\pgfqpoint{1.596586in}{3.239955in}}%
\pgfpathlineto{\pgfqpoint{1.596771in}{3.151425in}}%
\pgfpathlineto{\pgfqpoint{1.597572in}{3.243787in}}%
\pgfpathlineto{\pgfqpoint{1.597480in}{3.034866in}}%
\pgfpathlineto{\pgfqpoint{1.597912in}{3.173310in}}%
\pgfpathlineto{\pgfqpoint{1.598266in}{3.101114in}}%
\pgfpathlineto{\pgfqpoint{1.598528in}{3.224223in}}%
\pgfpathlineto{\pgfqpoint{1.599530in}{3.261385in}}%
\pgfpathlineto{\pgfqpoint{1.599037in}{3.008698in}}%
\pgfpathlineto{\pgfqpoint{1.599607in}{3.196587in}}%
\pgfpathlineto{\pgfqpoint{1.600501in}{3.253832in}}%
\pgfpathlineto{\pgfqpoint{1.599808in}{3.064328in}}%
\pgfpathlineto{\pgfqpoint{1.600547in}{3.142624in}}%
\pgfpathlineto{\pgfqpoint{1.601380in}{3.040645in}}%
\pgfpathlineto{\pgfqpoint{1.601503in}{3.234232in}}%
\pgfpathlineto{\pgfqpoint{1.601627in}{3.195061in}}%
\pgfpathlineto{\pgfqpoint{1.602166in}{3.122282in}}%
\pgfpathlineto{\pgfqpoint{1.602490in}{3.255005in}}%
\pgfpathlineto{\pgfqpoint{1.602767in}{3.163923in}}%
\pgfpathlineto{\pgfqpoint{1.603122in}{3.259536in}}%
\pgfpathlineto{\pgfqpoint{1.603708in}{3.043213in}}%
\pgfpathlineto{\pgfqpoint{1.603892in}{3.204831in}}%
\pgfpathlineto{\pgfqpoint{1.604494in}{3.098036in}}%
\pgfpathlineto{\pgfqpoint{1.604062in}{3.233747in}}%
\pgfpathlineto{\pgfqpoint{1.605049in}{3.151323in}}%
\pgfpathlineto{\pgfqpoint{1.605434in}{3.288622in}}%
\pgfpathlineto{\pgfqpoint{1.605264in}{3.060570in}}%
\pgfpathlineto{\pgfqpoint{1.606189in}{3.193080in}}%
\pgfpathlineto{\pgfqpoint{1.606821in}{3.090792in}}%
\pgfpathlineto{\pgfqpoint{1.606390in}{3.277274in}}%
\pgfpathlineto{\pgfqpoint{1.607130in}{3.191514in}}%
\pgfpathlineto{\pgfqpoint{1.607947in}{3.240913in}}%
\pgfpathlineto{\pgfqpoint{1.607592in}{3.086372in}}%
\pgfpathlineto{\pgfqpoint{1.608193in}{3.131418in}}%
\pgfpathlineto{\pgfqpoint{1.608317in}{3.105235in}}%
\pgfpathlineto{\pgfqpoint{1.608702in}{3.222847in}}%
\pgfpathlineto{\pgfqpoint{1.609349in}{3.269856in}}%
\pgfpathlineto{\pgfqpoint{1.609164in}{3.073020in}}%
\pgfpathlineto{\pgfqpoint{1.609766in}{3.173382in}}%
\pgfpathlineto{\pgfqpoint{1.610105in}{3.256503in}}%
\pgfpathlineto{\pgfqpoint{1.609935in}{3.109142in}}%
\pgfpathlineto{\pgfqpoint{1.610690in}{3.111963in}}%
\pgfpathlineto{\pgfqpoint{1.610721in}{3.085436in}}%
\pgfpathlineto{\pgfqpoint{1.611662in}{3.245970in}}%
\pgfpathlineto{\pgfqpoint{1.611769in}{3.146570in}}%
\pgfpathlineto{\pgfqpoint{1.612232in}{3.100702in}}%
\pgfpathlineto{\pgfqpoint{1.612432in}{3.231291in}}%
\pgfpathlineto{\pgfqpoint{1.612833in}{3.147346in}}%
\pgfpathlineto{\pgfqpoint{1.613388in}{3.236389in}}%
\pgfpathlineto{\pgfqpoint{1.613789in}{3.064276in}}%
\pgfpathlineto{\pgfqpoint{1.613958in}{3.206048in}}%
\pgfpathlineto{\pgfqpoint{1.614020in}{3.232095in}}%
\pgfpathlineto{\pgfqpoint{1.614560in}{3.095586in}}%
\pgfpathlineto{\pgfqpoint{1.614868in}{3.164389in}}%
\pgfpathlineto{\pgfqpoint{1.615392in}{3.076668in}}%
\pgfpathlineto{\pgfqpoint{1.614945in}{3.242313in}}%
\pgfpathlineto{\pgfqpoint{1.615947in}{3.165839in}}%
\pgfpathlineto{\pgfqpoint{1.616240in}{3.244497in}}%
\pgfpathlineto{\pgfqpoint{1.616163in}{3.084261in}}%
\pgfpathlineto{\pgfqpoint{1.617026in}{3.175224in}}%
\pgfpathlineto{\pgfqpoint{1.617720in}{3.035329in}}%
\pgfpathlineto{\pgfqpoint{1.617118in}{3.236710in}}%
\pgfpathlineto{\pgfqpoint{1.618136in}{3.174236in}}%
\pgfpathlineto{\pgfqpoint{1.618197in}{3.246780in}}%
\pgfpathlineto{\pgfqpoint{1.618490in}{3.070483in}}%
\pgfpathlineto{\pgfqpoint{1.619215in}{3.168859in}}%
\pgfpathlineto{\pgfqpoint{1.620016in}{3.081992in}}%
\pgfpathlineto{\pgfqpoint{1.619600in}{3.228441in}}%
\pgfpathlineto{\pgfqpoint{1.620325in}{3.151973in}}%
\pgfpathlineto{\pgfqpoint{1.621173in}{3.264454in}}%
\pgfpathlineto{\pgfqpoint{1.620787in}{3.099093in}}%
\pgfpathlineto{\pgfqpoint{1.621404in}{3.137315in}}%
\pgfpathlineto{\pgfqpoint{1.622390in}{3.073109in}}%
\pgfpathlineto{\pgfqpoint{1.621805in}{3.228833in}}%
\pgfpathlineto{\pgfqpoint{1.622437in}{3.124247in}}%
\pgfpathlineto{\pgfqpoint{1.622714in}{3.232545in}}%
\pgfpathlineto{\pgfqpoint{1.623176in}{3.098444in}}%
\pgfpathlineto{\pgfqpoint{1.623562in}{3.171057in}}%
\pgfpathlineto{\pgfqpoint{1.623932in}{3.068097in}}%
\pgfpathlineto{\pgfqpoint{1.624117in}{3.282520in}}%
\pgfpathlineto{\pgfqpoint{1.624718in}{3.077491in}}%
\pgfpathlineto{\pgfqpoint{1.625088in}{3.264600in}}%
\pgfpathlineto{\pgfqpoint{1.625874in}{3.187055in}}%
\pgfpathlineto{\pgfqpoint{1.626229in}{3.093075in}}%
\pgfpathlineto{\pgfqpoint{1.626429in}{3.246523in}}%
\pgfpathlineto{\pgfqpoint{1.627015in}{3.099087in}}%
\pgfpathlineto{\pgfqpoint{1.628032in}{3.269595in}}%
\pgfpathlineto{\pgfqpoint{1.627816in}{3.093744in}}%
\pgfpathlineto{\pgfqpoint{1.628217in}{3.190531in}}%
\pgfpathlineto{\pgfqpoint{1.628541in}{3.094539in}}%
\pgfpathlineto{\pgfqpoint{1.628787in}{3.256890in}}%
\pgfpathlineto{\pgfqpoint{1.629373in}{3.108471in}}%
\pgfpathlineto{\pgfqpoint{1.630344in}{3.259204in}}%
\pgfpathlineto{\pgfqpoint{1.630129in}{3.066013in}}%
\pgfpathlineto{\pgfqpoint{1.630545in}{3.210514in}}%
\pgfpathlineto{\pgfqpoint{1.630915in}{3.063787in}}%
\pgfpathlineto{\pgfqpoint{1.631130in}{3.237057in}}%
\pgfpathlineto{\pgfqpoint{1.631701in}{3.094462in}}%
\pgfpathlineto{\pgfqpoint{1.632687in}{3.262577in}}%
\pgfpathlineto{\pgfqpoint{1.632456in}{3.044147in}}%
\pgfpathlineto{\pgfqpoint{1.632842in}{3.219738in}}%
\pgfpathlineto{\pgfqpoint{1.633227in}{3.073853in}}%
\pgfpathlineto{\pgfqpoint{1.633921in}{3.261732in}}%
\pgfpathlineto{\pgfqpoint{1.634799in}{3.072647in}}%
\pgfpathlineto{\pgfqpoint{1.635138in}{3.184891in}}%
\pgfpathlineto{\pgfqpoint{1.635770in}{3.241647in}}%
\pgfpathlineto{\pgfqpoint{1.635385in}{3.091378in}}%
\pgfpathlineto{\pgfqpoint{1.636264in}{3.221367in}}%
\pgfpathlineto{\pgfqpoint{1.636356in}{3.053655in}}%
\pgfpathlineto{\pgfqpoint{1.636541in}{3.248001in}}%
\pgfpathlineto{\pgfqpoint{1.637389in}{3.176236in}}%
\pgfpathlineto{\pgfqpoint{1.637898in}{3.078543in}}%
\pgfpathlineto{\pgfqpoint{1.637497in}{3.243361in}}%
\pgfpathlineto{\pgfqpoint{1.638530in}{3.143942in}}%
\pgfpathlineto{\pgfqpoint{1.638591in}{3.253249in}}%
\pgfpathlineto{\pgfqpoint{1.638668in}{3.068063in}}%
\pgfpathlineto{\pgfqpoint{1.639686in}{3.239239in}}%
\pgfpathlineto{\pgfqpoint{1.640056in}{3.094091in}}%
\pgfpathlineto{\pgfqpoint{1.639824in}{3.263442in}}%
\pgfpathlineto{\pgfqpoint{1.640857in}{3.132174in}}%
\pgfpathlineto{\pgfqpoint{1.641227in}{3.258819in}}%
\pgfpathlineto{\pgfqpoint{1.641011in}{3.076321in}}%
\pgfpathlineto{\pgfqpoint{1.642013in}{3.229357in}}%
\pgfpathlineto{\pgfqpoint{1.642568in}{3.076189in}}%
\pgfpathlineto{\pgfqpoint{1.642769in}{3.286860in}}%
\pgfpathlineto{\pgfqpoint{1.643154in}{3.140596in}}%
\pgfpathlineto{\pgfqpoint{1.643524in}{3.254345in}}%
\pgfpathlineto{\pgfqpoint{1.643339in}{3.086560in}}%
\pgfpathlineto{\pgfqpoint{1.644094in}{3.106994in}}%
\pgfpathlineto{\pgfqpoint{1.644125in}{3.087791in}}%
\pgfpathlineto{\pgfqpoint{1.644480in}{3.262556in}}%
\pgfpathlineto{\pgfqpoint{1.645081in}{3.241731in}}%
\pgfpathlineto{\pgfqpoint{1.645112in}{3.265060in}}%
\pgfpathlineto{\pgfqpoint{1.645651in}{3.078046in}}%
\pgfpathlineto{\pgfqpoint{1.646098in}{3.168971in}}%
\pgfpathlineto{\pgfqpoint{1.646453in}{3.119093in}}%
\pgfpathlineto{\pgfqpoint{1.646669in}{3.285045in}}%
\pgfpathlineto{\pgfqpoint{1.647100in}{3.184609in}}%
\pgfpathlineto{\pgfqpoint{1.647439in}{3.288038in}}%
\pgfpathlineto{\pgfqpoint{1.648010in}{3.088231in}}%
\pgfpathlineto{\pgfqpoint{1.648241in}{3.236598in}}%
\pgfpathlineto{\pgfqpoint{1.648780in}{3.077869in}}%
\pgfpathlineto{\pgfqpoint{1.648996in}{3.262700in}}%
\pgfpathlineto{\pgfqpoint{1.649382in}{3.145832in}}%
\pgfpathlineto{\pgfqpoint{1.649782in}{3.247638in}}%
\pgfpathlineto{\pgfqpoint{1.649567in}{3.062974in}}%
\pgfpathlineto{\pgfqpoint{1.650584in}{3.220179in}}%
\pgfpathlineto{\pgfqpoint{1.651108in}{3.062003in}}%
\pgfpathlineto{\pgfqpoint{1.651339in}{3.269769in}}%
\pgfpathlineto{\pgfqpoint{1.651756in}{3.158522in}}%
\pgfpathlineto{\pgfqpoint{1.652573in}{3.261543in}}%
\pgfpathlineto{\pgfqpoint{1.651879in}{3.085988in}}%
\pgfpathlineto{\pgfqpoint{1.652896in}{3.246770in}}%
\pgfpathlineto{\pgfqpoint{1.653451in}{3.066622in}}%
\pgfpathlineto{\pgfqpoint{1.653667in}{3.264754in}}%
\pgfpathlineto{\pgfqpoint{1.654068in}{3.108970in}}%
\pgfpathlineto{\pgfqpoint{1.655208in}{3.245636in}}%
\pgfpathlineto{\pgfqpoint{1.655008in}{3.076721in}}%
\pgfpathlineto{\pgfqpoint{1.655224in}{3.234670in}}%
\pgfpathlineto{\pgfqpoint{1.655779in}{3.059735in}}%
\pgfpathlineto{\pgfqpoint{1.655979in}{3.242203in}}%
\pgfpathlineto{\pgfqpoint{1.656411in}{3.114402in}}%
\pgfpathlineto{\pgfqpoint{1.657243in}{3.253415in}}%
\pgfpathlineto{\pgfqpoint{1.657336in}{3.071999in}}%
\pgfpathlineto{\pgfqpoint{1.657582in}{3.216904in}}%
\pgfpathlineto{\pgfqpoint{1.658708in}{3.089304in}}%
\pgfpathlineto{\pgfqpoint{1.658476in}{3.242659in}}%
\pgfpathlineto{\pgfqpoint{1.658738in}{3.112305in}}%
\pgfpathlineto{\pgfqpoint{1.659108in}{3.261917in}}%
\pgfpathlineto{\pgfqpoint{1.659663in}{3.083041in}}%
\pgfpathlineto{\pgfqpoint{1.659895in}{3.232101in}}%
\pgfpathlineto{\pgfqpoint{1.660265in}{3.105935in}}%
\pgfpathlineto{\pgfqpoint{1.660650in}{3.242708in}}%
\pgfpathlineto{\pgfqpoint{1.661051in}{3.123741in}}%
\pgfpathlineto{\pgfqpoint{1.661421in}{3.273473in}}%
\pgfpathlineto{\pgfqpoint{1.661991in}{3.085125in}}%
\pgfpathlineto{\pgfqpoint{1.662207in}{3.236781in}}%
\pgfpathlineto{\pgfqpoint{1.662762in}{3.089495in}}%
\pgfpathlineto{\pgfqpoint{1.662392in}{3.241764in}}%
\pgfpathlineto{\pgfqpoint{1.663378in}{3.149911in}}%
\pgfpathlineto{\pgfqpoint{1.663548in}{3.095351in}}%
\pgfpathlineto{\pgfqpoint{1.663455in}{3.229145in}}%
\pgfpathlineto{\pgfqpoint{1.663733in}{3.221301in}}%
\pgfpathlineto{\pgfqpoint{1.663764in}{3.243174in}}%
\pgfpathlineto{\pgfqpoint{1.664319in}{3.090319in}}%
\pgfpathlineto{\pgfqpoint{1.664766in}{3.164548in}}%
\pgfpathlineto{\pgfqpoint{1.664935in}{3.117328in}}%
\pgfpathlineto{\pgfqpoint{1.665305in}{3.247041in}}%
\pgfpathlineto{\pgfqpoint{1.666091in}{3.274990in}}%
\pgfpathlineto{\pgfqpoint{1.665876in}{3.104021in}}%
\pgfpathlineto{\pgfqpoint{1.666353in}{3.156127in}}%
\pgfpathlineto{\pgfqpoint{1.666862in}{3.245736in}}%
\pgfpathlineto{\pgfqpoint{1.666646in}{3.102449in}}%
\pgfpathlineto{\pgfqpoint{1.667402in}{3.141621in}}%
\pgfpathlineto{\pgfqpoint{1.668203in}{3.062946in}}%
\pgfpathlineto{\pgfqpoint{1.667648in}{3.248770in}}%
\pgfpathlineto{\pgfqpoint{1.668511in}{3.116597in}}%
\pgfpathlineto{\pgfqpoint{1.668573in}{3.182517in}}%
\pgfpathlineto{\pgfqpoint{1.668897in}{3.230967in}}%
\pgfpathlineto{\pgfqpoint{1.668989in}{3.088324in}}%
\pgfpathlineto{\pgfqpoint{1.669683in}{3.218479in}}%
\pgfpathlineto{\pgfqpoint{1.670531in}{3.076473in}}%
\pgfpathlineto{\pgfqpoint{1.669991in}{3.253056in}}%
\pgfpathlineto{\pgfqpoint{1.670839in}{3.120468in}}%
\pgfpathlineto{\pgfqpoint{1.671224in}{3.255538in}}%
\pgfpathlineto{\pgfqpoint{1.671147in}{3.091571in}}%
\pgfpathlineto{\pgfqpoint{1.672011in}{3.224699in}}%
\pgfpathlineto{\pgfqpoint{1.672103in}{3.077700in}}%
\pgfpathlineto{\pgfqpoint{1.672303in}{3.256957in}}%
\pgfpathlineto{\pgfqpoint{1.673151in}{3.137981in}}%
\pgfpathlineto{\pgfqpoint{1.673845in}{3.243010in}}%
\pgfpathlineto{\pgfqpoint{1.673475in}{3.100138in}}%
\pgfpathlineto{\pgfqpoint{1.674354in}{3.208933in}}%
\pgfpathlineto{\pgfqpoint{1.674431in}{3.071458in}}%
\pgfpathlineto{\pgfqpoint{1.674631in}{3.238967in}}%
\pgfpathlineto{\pgfqpoint{1.675510in}{3.132556in}}%
\pgfpathlineto{\pgfqpoint{1.675895in}{3.244054in}}%
\pgfpathlineto{\pgfqpoint{1.675988in}{3.095053in}}%
\pgfpathlineto{\pgfqpoint{1.676604in}{3.127958in}}%
\pgfpathlineto{\pgfqpoint{1.677760in}{3.244137in}}%
\pgfpathlineto{\pgfqpoint{1.677375in}{3.095336in}}%
\pgfpathlineto{\pgfqpoint{1.677807in}{3.148641in}}%
\pgfpathlineto{\pgfqpoint{1.678315in}{3.113326in}}%
\pgfpathlineto{\pgfqpoint{1.678531in}{3.239334in}}%
\pgfpathlineto{\pgfqpoint{1.678932in}{3.125603in}}%
\pgfpathlineto{\pgfqpoint{1.680073in}{3.261780in}}%
\pgfpathlineto{\pgfqpoint{1.679872in}{3.116419in}}%
\pgfpathlineto{\pgfqpoint{1.680150in}{3.162843in}}%
\pgfpathlineto{\pgfqpoint{1.680643in}{3.104826in}}%
\pgfpathlineto{\pgfqpoint{1.680843in}{3.233025in}}%
\pgfpathlineto{\pgfqpoint{1.681290in}{3.137748in}}%
\pgfpathlineto{\pgfqpoint{1.682416in}{3.243080in}}%
\pgfpathlineto{\pgfqpoint{1.681429in}{3.110238in}}%
\pgfpathlineto{\pgfqpoint{1.682462in}{3.187309in}}%
\pgfpathlineto{\pgfqpoint{1.682955in}{3.100177in}}%
\pgfpathlineto{\pgfqpoint{1.683186in}{3.242992in}}%
\pgfpathlineto{\pgfqpoint{1.683587in}{3.130115in}}%
\pgfpathlineto{\pgfqpoint{1.683757in}{3.122451in}}%
\pgfpathlineto{\pgfqpoint{1.683664in}{3.227445in}}%
\pgfpathlineto{\pgfqpoint{1.683957in}{3.226034in}}%
\pgfpathlineto{\pgfqpoint{1.683988in}{3.249923in}}%
\pgfpathlineto{\pgfqpoint{1.684512in}{3.120701in}}%
\pgfpathlineto{\pgfqpoint{1.685036in}{3.178681in}}%
\pgfpathlineto{\pgfqpoint{1.685884in}{3.102613in}}%
\pgfpathlineto{\pgfqpoint{1.685514in}{3.234155in}}%
\pgfpathlineto{\pgfqpoint{1.685976in}{3.208309in}}%
\pgfpathlineto{\pgfqpoint{1.686300in}{3.238088in}}%
\pgfpathlineto{\pgfqpoint{1.686855in}{3.073948in}}%
\pgfpathlineto{\pgfqpoint{1.687102in}{3.231925in}}%
\pgfpathlineto{\pgfqpoint{1.687641in}{3.111056in}}%
\pgfpathlineto{\pgfqpoint{1.688258in}{3.156745in}}%
\pgfpathlineto{\pgfqpoint{1.688643in}{3.231281in}}%
\pgfpathlineto{\pgfqpoint{1.689183in}{3.089187in}}%
\pgfpathlineto{\pgfqpoint{1.689445in}{3.205215in}}%
\pgfpathlineto{\pgfqpoint{1.689799in}{3.093690in}}%
\pgfpathlineto{\pgfqpoint{1.689876in}{3.249736in}}%
\pgfpathlineto{\pgfqpoint{1.690570in}{3.131864in}}%
\pgfpathlineto{\pgfqpoint{1.690955in}{3.248756in}}%
\pgfpathlineto{\pgfqpoint{1.690755in}{3.082833in}}%
\pgfpathlineto{\pgfqpoint{1.691757in}{3.178058in}}%
\pgfpathlineto{\pgfqpoint{1.692297in}{3.102371in}}%
\pgfpathlineto{\pgfqpoint{1.692497in}{3.233902in}}%
\pgfpathlineto{\pgfqpoint{1.692882in}{3.152840in}}%
\pgfpathlineto{\pgfqpoint{1.693453in}{3.232041in}}%
\pgfpathlineto{\pgfqpoint{1.693083in}{3.081364in}}%
\pgfpathlineto{\pgfqpoint{1.694100in}{3.193007in}}%
\pgfpathlineto{\pgfqpoint{1.694640in}{3.112227in}}%
\pgfpathlineto{\pgfqpoint{1.694855in}{3.244554in}}%
\pgfpathlineto{\pgfqpoint{1.695272in}{3.140458in}}%
\pgfpathlineto{\pgfqpoint{1.695780in}{3.252895in}}%
\pgfpathlineto{\pgfqpoint{1.696012in}{3.108527in}}%
\pgfpathlineto{\pgfqpoint{1.696428in}{3.229258in}}%
\pgfpathlineto{\pgfqpoint{1.696983in}{3.108687in}}%
\pgfpathlineto{\pgfqpoint{1.697568in}{3.139511in}}%
\pgfpathlineto{\pgfqpoint{1.697584in}{3.139118in}}%
\pgfpathlineto{\pgfqpoint{1.697645in}{3.163094in}}%
\pgfpathlineto{\pgfqpoint{1.698725in}{3.257524in}}%
\pgfpathlineto{\pgfqpoint{1.698324in}{3.116370in}}%
\pgfpathlineto{\pgfqpoint{1.698786in}{3.195564in}}%
\pgfpathlineto{\pgfqpoint{1.699295in}{3.101338in}}%
\pgfpathlineto{\pgfqpoint{1.699680in}{3.235181in}}%
\pgfpathlineto{\pgfqpoint{1.699927in}{3.132807in}}%
\pgfpathlineto{\pgfqpoint{1.700436in}{3.248370in}}%
\pgfpathlineto{\pgfqpoint{1.700605in}{3.110751in}}%
\pgfpathlineto{\pgfqpoint{1.701114in}{3.181250in}}%
\pgfpathlineto{\pgfqpoint{1.701607in}{3.096898in}}%
\pgfpathlineto{\pgfqpoint{1.701838in}{3.236372in}}%
\pgfpathlineto{\pgfqpoint{1.702239in}{3.123866in}}%
\pgfpathlineto{\pgfqpoint{1.702640in}{3.250701in}}%
\pgfpathlineto{\pgfqpoint{1.703210in}{3.118570in}}%
\pgfpathlineto{\pgfqpoint{1.703441in}{3.192214in}}%
\pgfpathlineto{\pgfqpoint{1.704536in}{3.098563in}}%
\pgfpathlineto{\pgfqpoint{1.704166in}{3.234704in}}%
\pgfpathlineto{\pgfqpoint{1.704567in}{3.138364in}}%
\pgfpathlineto{\pgfqpoint{1.704952in}{3.230763in}}%
\pgfpathlineto{\pgfqpoint{1.705507in}{3.082396in}}%
\pgfpathlineto{\pgfqpoint{1.705723in}{3.226600in}}%
\pgfpathlineto{\pgfqpoint{1.706663in}{3.247851in}}%
\pgfpathlineto{\pgfqpoint{1.706062in}{3.125618in}}%
\pgfpathlineto{\pgfqpoint{1.706709in}{3.177327in}}%
\pgfpathlineto{\pgfqpoint{1.707835in}{3.102581in}}%
\pgfpathlineto{\pgfqpoint{1.707295in}{3.241862in}}%
\pgfpathlineto{\pgfqpoint{1.707866in}{3.125233in}}%
\pgfpathlineto{\pgfqpoint{1.708205in}{3.238614in}}%
\pgfpathlineto{\pgfqpoint{1.708436in}{3.095112in}}%
\pgfpathlineto{\pgfqpoint{1.709037in}{3.175873in}}%
\pgfpathlineto{\pgfqpoint{1.709422in}{3.090729in}}%
\pgfpathlineto{\pgfqpoint{1.709607in}{3.235802in}}%
\pgfpathlineto{\pgfqpoint{1.710101in}{3.178826in}}%
\pgfpathlineto{\pgfqpoint{1.710548in}{3.241219in}}%
\pgfpathlineto{\pgfqpoint{1.710964in}{3.110310in}}%
\pgfpathlineto{\pgfqpoint{1.711226in}{3.193176in}}%
\pgfpathlineto{\pgfqpoint{1.711750in}{3.107138in}}%
\pgfpathlineto{\pgfqpoint{1.712105in}{3.246718in}}%
\pgfpathlineto{\pgfqpoint{1.712382in}{3.139332in}}%
\pgfpathlineto{\pgfqpoint{1.713507in}{3.237435in}}%
\pgfpathlineto{\pgfqpoint{1.713338in}{3.096387in}}%
\pgfpathlineto{\pgfqpoint{1.713538in}{3.207392in}}%
\pgfpathlineto{\pgfqpoint{1.714663in}{3.119212in}}%
\pgfpathlineto{\pgfqpoint{1.714432in}{3.235231in}}%
\pgfpathlineto{\pgfqpoint{1.714679in}{3.131436in}}%
\pgfpathlineto{\pgfqpoint{1.715049in}{3.230582in}}%
\pgfpathlineto{\pgfqpoint{1.714879in}{3.109197in}}%
\pgfpathlineto{\pgfqpoint{1.715835in}{3.199720in}}%
\pgfpathlineto{\pgfqpoint{1.716960in}{3.124548in}}%
\pgfpathlineto{\pgfqpoint{1.716005in}{3.224753in}}%
\pgfpathlineto{\pgfqpoint{1.716991in}{3.155710in}}%
\pgfpathlineto{\pgfqpoint{1.717546in}{3.231616in}}%
\pgfpathlineto{\pgfqpoint{1.717238in}{3.113806in}}%
\pgfpathlineto{\pgfqpoint{1.718132in}{3.192713in}}%
\pgfpathlineto{\pgfqpoint{1.718332in}{3.231949in}}%
\pgfpathlineto{\pgfqpoint{1.718779in}{3.124671in}}%
\pgfpathlineto{\pgfqpoint{1.719165in}{3.167376in}}%
\pgfpathlineto{\pgfqpoint{1.719565in}{3.128671in}}%
\pgfpathlineto{\pgfqpoint{1.719396in}{3.212556in}}%
\pgfpathlineto{\pgfqpoint{1.720274in}{3.161732in}}%
\pgfpathlineto{\pgfqpoint{1.720490in}{3.216769in}}%
\pgfpathlineto{\pgfqpoint{1.720876in}{3.123705in}}%
\pgfpathlineto{\pgfqpoint{1.721477in}{3.189359in}}%
\pgfpathlineto{\pgfqpoint{1.721847in}{3.126370in}}%
\pgfpathlineto{\pgfqpoint{1.722032in}{3.208739in}}%
\pgfpathlineto{\pgfqpoint{1.722633in}{3.142994in}}%
\pgfpathlineto{\pgfqpoint{1.723172in}{3.109101in}}%
\pgfpathlineto{\pgfqpoint{1.722987in}{3.227436in}}%
\pgfpathlineto{\pgfqpoint{1.723619in}{3.203188in}}%
\pgfpathlineto{\pgfqpoint{1.724144in}{3.115910in}}%
\pgfpathlineto{\pgfqpoint{1.723758in}{3.214712in}}%
\pgfpathlineto{\pgfqpoint{1.724822in}{3.184211in}}%
\pgfpathlineto{\pgfqpoint{1.725932in}{3.221820in}}%
\pgfpathlineto{\pgfqpoint{1.725762in}{3.113202in}}%
\pgfpathlineto{\pgfqpoint{1.725947in}{3.216044in}}%
\pgfpathlineto{\pgfqpoint{1.727072in}{3.092943in}}%
\pgfpathlineto{\pgfqpoint{1.726887in}{3.219057in}}%
\pgfpathlineto{\pgfqpoint{1.727119in}{3.148186in}}%
\pgfpathlineto{\pgfqpoint{1.728059in}{3.101149in}}%
\pgfpathlineto{\pgfqpoint{1.728259in}{3.210438in}}%
\pgfpathlineto{\pgfqpoint{1.728275in}{3.211057in}}%
\pgfpathlineto{\pgfqpoint{1.728352in}{3.176680in}}%
\pgfpathlineto{\pgfqpoint{1.728629in}{3.132633in}}%
\pgfpathlineto{\pgfqpoint{1.729200in}{3.228647in}}%
\pgfpathlineto{\pgfqpoint{1.729477in}{3.167289in}}%
\pgfpathlineto{\pgfqpoint{1.729847in}{3.227646in}}%
\pgfpathlineto{\pgfqpoint{1.730433in}{3.122775in}}%
\pgfpathlineto{\pgfqpoint{1.730633in}{3.209414in}}%
\pgfpathlineto{\pgfqpoint{1.730988in}{3.096714in}}%
\pgfpathlineto{\pgfqpoint{1.730757in}{3.217999in}}%
\pgfpathlineto{\pgfqpoint{1.731774in}{3.153433in}}%
\pgfpathlineto{\pgfqpoint{1.732144in}{3.226931in}}%
\pgfpathlineto{\pgfqpoint{1.731990in}{3.088227in}}%
\pgfpathlineto{\pgfqpoint{1.732884in}{3.169301in}}%
\pgfpathlineto{\pgfqpoint{1.733516in}{3.123020in}}%
\pgfpathlineto{\pgfqpoint{1.733130in}{3.230801in}}%
\pgfpathlineto{\pgfqpoint{1.733655in}{3.192669in}}%
\pgfpathlineto{\pgfqpoint{1.734656in}{3.233231in}}%
\pgfpathlineto{\pgfqpoint{1.734287in}{3.124240in}}%
\pgfpathlineto{\pgfqpoint{1.734764in}{3.192962in}}%
\pgfpathlineto{\pgfqpoint{1.734841in}{3.131206in}}%
\pgfpathlineto{\pgfqpoint{1.735890in}{3.083289in}}%
\pgfpathlineto{\pgfqpoint{1.735304in}{3.206153in}}%
\pgfpathlineto{\pgfqpoint{1.735936in}{3.140408in}}%
\pgfpathlineto{\pgfqpoint{1.736075in}{3.227085in}}%
\pgfpathlineto{\pgfqpoint{1.736445in}{3.127132in}}%
\pgfpathlineto{\pgfqpoint{1.737061in}{3.195694in}}%
\pgfpathlineto{\pgfqpoint{1.737431in}{3.104408in}}%
\pgfpathlineto{\pgfqpoint{1.737601in}{3.229349in}}%
\pgfpathlineto{\pgfqpoint{1.738233in}{3.129391in}}%
\pgfpathlineto{\pgfqpoint{1.738572in}{3.229226in}}%
\pgfpathlineto{\pgfqpoint{1.739373in}{3.163475in}}%
\pgfpathlineto{\pgfqpoint{1.739805in}{3.090193in}}%
\pgfpathlineto{\pgfqpoint{1.739975in}{3.224697in}}%
\pgfpathlineto{\pgfqpoint{1.740499in}{3.142965in}}%
\pgfpathlineto{\pgfqpoint{1.741531in}{3.228380in}}%
\pgfpathlineto{\pgfqpoint{1.741331in}{3.108712in}}%
\pgfpathlineto{\pgfqpoint{1.741686in}{3.201011in}}%
\pgfpathlineto{\pgfqpoint{1.742117in}{3.115445in}}%
\pgfpathlineto{\pgfqpoint{1.742472in}{3.214166in}}%
\pgfpathlineto{\pgfqpoint{1.742826in}{3.158848in}}%
\pgfpathlineto{\pgfqpoint{1.743705in}{3.100437in}}%
\pgfpathlineto{\pgfqpoint{1.743227in}{3.212391in}}%
\pgfpathlineto{\pgfqpoint{1.743859in}{3.201287in}}%
\pgfpathlineto{\pgfqpoint{1.744799in}{3.214601in}}%
\pgfpathlineto{\pgfqpoint{1.744399in}{3.122670in}}%
\pgfpathlineto{\pgfqpoint{1.744923in}{3.189286in}}%
\pgfpathlineto{\pgfqpoint{1.746002in}{3.105880in}}%
\pgfpathlineto{\pgfqpoint{1.745555in}{3.220554in}}%
\pgfpathlineto{\pgfqpoint{1.746063in}{3.138248in}}%
\pgfpathlineto{\pgfqpoint{1.747127in}{3.215799in}}%
\pgfpathlineto{\pgfqpoint{1.746788in}{3.116028in}}%
\pgfpathlineto{\pgfqpoint{1.747189in}{3.168171in}}%
\pgfpathlineto{\pgfqpoint{1.748329in}{3.105371in}}%
\pgfpathlineto{\pgfqpoint{1.747898in}{3.215982in}}%
\pgfpathlineto{\pgfqpoint{1.748360in}{3.132324in}}%
\pgfpathlineto{\pgfqpoint{1.749455in}{3.230226in}}%
\pgfpathlineto{\pgfqpoint{1.749116in}{3.119877in}}%
\pgfpathlineto{\pgfqpoint{1.749501in}{3.181576in}}%
\pgfpathlineto{\pgfqpoint{1.749640in}{3.098982in}}%
\pgfpathlineto{\pgfqpoint{1.749794in}{3.212275in}}%
\pgfpathlineto{\pgfqpoint{1.750688in}{3.121020in}}%
\pgfpathlineto{\pgfqpoint{1.751782in}{3.228569in}}%
\pgfpathlineto{\pgfqpoint{1.751474in}{3.116477in}}%
\pgfpathlineto{\pgfqpoint{1.751967in}{3.153225in}}%
\pgfpathlineto{\pgfqpoint{1.752245in}{3.099396in}}%
\pgfpathlineto{\pgfqpoint{1.752399in}{3.221398in}}%
\pgfpathlineto{\pgfqpoint{1.753077in}{3.153667in}}%
\pgfpathlineto{\pgfqpoint{1.753355in}{3.237222in}}%
\pgfpathlineto{\pgfqpoint{1.753540in}{3.096560in}}%
\pgfpathlineto{\pgfqpoint{1.754187in}{3.169613in}}%
\pgfpathlineto{\pgfqpoint{1.754526in}{3.094949in}}%
\pgfpathlineto{\pgfqpoint{1.754912in}{3.220334in}}%
\pgfpathlineto{\pgfqpoint{1.755235in}{3.204784in}}%
\pgfpathlineto{\pgfqpoint{1.755682in}{3.239005in}}%
\pgfpathlineto{\pgfqpoint{1.755389in}{3.122440in}}%
\pgfpathlineto{\pgfqpoint{1.755836in}{3.151943in}}%
\pgfpathlineto{\pgfqpoint{1.756160in}{3.097112in}}%
\pgfpathlineto{\pgfqpoint{1.756330in}{3.227307in}}%
\pgfpathlineto{\pgfqpoint{1.756946in}{3.126863in}}%
\pgfpathlineto{\pgfqpoint{1.757239in}{3.225420in}}%
\pgfpathlineto{\pgfqpoint{1.757440in}{3.111457in}}%
\pgfpathlineto{\pgfqpoint{1.758087in}{3.179044in}}%
\pgfpathlineto{\pgfqpoint{1.758457in}{3.066984in}}%
\pgfpathlineto{\pgfqpoint{1.758627in}{3.216841in}}%
\pgfpathlineto{\pgfqpoint{1.759243in}{3.130430in}}%
\pgfpathlineto{\pgfqpoint{1.759598in}{3.233873in}}%
\pgfpathlineto{\pgfqpoint{1.760014in}{3.105484in}}%
\pgfpathlineto{\pgfqpoint{1.760384in}{3.163410in}}%
\pgfpathlineto{\pgfqpoint{1.760800in}{3.116939in}}%
\pgfpathlineto{\pgfqpoint{1.761124in}{3.234699in}}%
\pgfpathlineto{\pgfqpoint{1.761509in}{3.142849in}}%
\pgfpathlineto{\pgfqpoint{1.762218in}{3.214202in}}%
\pgfpathlineto{\pgfqpoint{1.762311in}{3.112612in}}%
\pgfpathlineto{\pgfqpoint{1.762357in}{3.052677in}}%
\pgfpathlineto{\pgfqpoint{1.762557in}{3.217580in}}%
\pgfpathlineto{\pgfqpoint{1.763374in}{3.158628in}}%
\pgfpathlineto{\pgfqpoint{1.763498in}{3.218464in}}%
\pgfpathlineto{\pgfqpoint{1.763914in}{3.090752in}}%
\pgfpathlineto{\pgfqpoint{1.764515in}{3.195966in}}%
\pgfpathlineto{\pgfqpoint{1.764561in}{3.199712in}}%
\pgfpathlineto{\pgfqpoint{1.764592in}{3.180920in}}%
\pgfpathlineto{\pgfqpoint{1.764685in}{3.096111in}}%
\pgfpathlineto{\pgfqpoint{1.765039in}{3.230776in}}%
\pgfpathlineto{\pgfqpoint{1.765671in}{3.208221in}}%
\pgfpathlineto{\pgfqpoint{1.766457in}{3.222474in}}%
\pgfpathlineto{\pgfqpoint{1.766272in}{3.061786in}}%
\pgfpathlineto{\pgfqpoint{1.766704in}{3.189248in}}%
\pgfpathlineto{\pgfqpoint{1.767814in}{3.082190in}}%
\pgfpathlineto{\pgfqpoint{1.767351in}{3.213630in}}%
\pgfpathlineto{\pgfqpoint{1.767875in}{3.160946in}}%
\pgfpathlineto{\pgfqpoint{1.767999in}{3.222868in}}%
\pgfpathlineto{\pgfqpoint{1.768585in}{3.082010in}}%
\pgfpathlineto{\pgfqpoint{1.769047in}{3.208830in}}%
\pgfpathlineto{\pgfqpoint{1.770172in}{3.071632in}}%
\pgfpathlineto{\pgfqpoint{1.769694in}{3.226723in}}%
\pgfpathlineto{\pgfqpoint{1.770249in}{3.168122in}}%
\pgfpathlineto{\pgfqpoint{1.770357in}{3.217950in}}%
\pgfpathlineto{\pgfqpoint{1.770866in}{3.113697in}}%
\pgfpathlineto{\pgfqpoint{1.771375in}{3.196454in}}%
\pgfpathlineto{\pgfqpoint{1.772484in}{3.076214in}}%
\pgfpathlineto{\pgfqpoint{1.772022in}{3.234070in}}%
\pgfpathlineto{\pgfqpoint{1.772546in}{3.168625in}}%
\pgfpathlineto{\pgfqpoint{1.773594in}{3.227974in}}%
\pgfpathlineto{\pgfqpoint{1.773255in}{3.100073in}}%
\pgfpathlineto{\pgfqpoint{1.773656in}{3.189040in}}%
\pgfpathlineto{\pgfqpoint{1.774072in}{3.097439in}}%
\pgfpathlineto{\pgfqpoint{1.774365in}{3.227234in}}%
\pgfpathlineto{\pgfqpoint{1.774828in}{3.115390in}}%
\pgfpathlineto{\pgfqpoint{1.775922in}{3.238695in}}%
\pgfpathlineto{\pgfqpoint{1.775629in}{3.109229in}}%
\pgfpathlineto{\pgfqpoint{1.775999in}{3.180300in}}%
\pgfpathlineto{\pgfqpoint{1.776400in}{3.082669in}}%
\pgfpathlineto{\pgfqpoint{1.776569in}{3.214490in}}%
\pgfpathlineto{\pgfqpoint{1.777186in}{3.107402in}}%
\pgfpathlineto{\pgfqpoint{1.777494in}{3.228074in}}%
\pgfpathlineto{\pgfqpoint{1.777957in}{3.103984in}}%
\pgfpathlineto{\pgfqpoint{1.778327in}{3.182112in}}%
\pgfpathlineto{\pgfqpoint{1.778712in}{3.081366in}}%
\pgfpathlineto{\pgfqpoint{1.778866in}{3.217179in}}%
\pgfpathlineto{\pgfqpoint{1.779514in}{3.117807in}}%
\pgfpathlineto{\pgfqpoint{1.779529in}{3.116471in}}%
\pgfpathlineto{\pgfqpoint{1.779729in}{3.194924in}}%
\pgfpathlineto{\pgfqpoint{1.779745in}{3.193879in}}%
\pgfpathlineto{\pgfqpoint{1.779837in}{3.239285in}}%
\pgfpathlineto{\pgfqpoint{1.779991in}{3.120163in}}%
\pgfpathlineto{\pgfqpoint{1.780993in}{3.085046in}}%
\pgfpathlineto{\pgfqpoint{1.780469in}{3.215463in}}%
\pgfpathlineto{\pgfqpoint{1.781086in}{3.100660in}}%
\pgfpathlineto{\pgfqpoint{1.781394in}{3.226186in}}%
\pgfpathlineto{\pgfqpoint{1.782242in}{3.184696in}}%
\pgfpathlineto{\pgfqpoint{1.782627in}{3.070382in}}%
\pgfpathlineto{\pgfqpoint{1.782797in}{3.215623in}}%
\pgfpathlineto{\pgfqpoint{1.783429in}{3.130853in}}%
\pgfpathlineto{\pgfqpoint{1.783737in}{3.236936in}}%
\pgfpathlineto{\pgfqpoint{1.783922in}{3.101044in}}%
\pgfpathlineto{\pgfqpoint{1.784554in}{3.181893in}}%
\pgfpathlineto{\pgfqpoint{1.784955in}{3.085098in}}%
\pgfpathlineto{\pgfqpoint{1.785310in}{3.217768in}}%
\pgfpathlineto{\pgfqpoint{1.785680in}{3.145793in}}%
\pgfpathlineto{\pgfqpoint{1.786065in}{3.229765in}}%
\pgfpathlineto{\pgfqpoint{1.786527in}{3.074661in}}%
\pgfpathlineto{\pgfqpoint{1.786836in}{3.172814in}}%
\pgfpathlineto{\pgfqpoint{1.787298in}{3.113663in}}%
\pgfpathlineto{\pgfqpoint{1.787653in}{3.228631in}}%
\pgfpathlineto{\pgfqpoint{1.787961in}{3.151854in}}%
\pgfpathlineto{\pgfqpoint{1.788655in}{3.218538in}}%
\pgfpathlineto{\pgfqpoint{1.788840in}{3.065645in}}%
\pgfpathlineto{\pgfqpoint{1.789102in}{3.182956in}}%
\pgfpathlineto{\pgfqpoint{1.789626in}{3.125546in}}%
\pgfpathlineto{\pgfqpoint{1.789965in}{3.228640in}}%
\pgfpathlineto{\pgfqpoint{1.790196in}{3.187982in}}%
\pgfpathlineto{\pgfqpoint{1.790628in}{3.220701in}}%
\pgfpathlineto{\pgfqpoint{1.790427in}{3.090429in}}%
\pgfpathlineto{\pgfqpoint{1.791291in}{3.196239in}}%
\pgfpathlineto{\pgfqpoint{1.791969in}{3.103138in}}%
\pgfpathlineto{\pgfqpoint{1.791599in}{3.223074in}}%
\pgfpathlineto{\pgfqpoint{1.792431in}{3.153372in}}%
\pgfpathlineto{\pgfqpoint{1.792570in}{3.217808in}}%
\pgfpathlineto{\pgfqpoint{1.792755in}{3.049220in}}%
\pgfpathlineto{\pgfqpoint{1.793495in}{3.138916in}}%
\pgfpathlineto{\pgfqpoint{1.794312in}{3.092175in}}%
\pgfpathlineto{\pgfqpoint{1.793880in}{3.222569in}}%
\pgfpathlineto{\pgfqpoint{1.794528in}{3.210771in}}%
\pgfpathlineto{\pgfqpoint{1.795083in}{3.084898in}}%
\pgfpathlineto{\pgfqpoint{1.795499in}{3.218307in}}%
\pgfpathlineto{\pgfqpoint{1.795884in}{3.109770in}}%
\pgfpathlineto{\pgfqpoint{1.796192in}{3.212718in}}%
\pgfpathlineto{\pgfqpoint{1.796655in}{3.047111in}}%
\pgfpathlineto{\pgfqpoint{1.797056in}{3.198467in}}%
\pgfpathlineto{\pgfqpoint{1.797349in}{3.133908in}}%
\pgfpathlineto{\pgfqpoint{1.797395in}{3.137008in}}%
\pgfpathlineto{\pgfqpoint{1.798196in}{3.091690in}}%
\pgfpathlineto{\pgfqpoint{1.797503in}{3.222805in}}%
\pgfpathlineto{\pgfqpoint{1.798443in}{3.208951in}}%
\pgfpathlineto{\pgfqpoint{1.798520in}{3.213911in}}%
\pgfpathlineto{\pgfqpoint{1.798952in}{3.106127in}}%
\pgfpathlineto{\pgfqpoint{1.798983in}{3.076788in}}%
\pgfpathlineto{\pgfqpoint{1.799337in}{3.219944in}}%
\pgfpathlineto{\pgfqpoint{1.800000in}{3.182595in}}%
\pgfpathlineto{\pgfqpoint{1.800092in}{3.224661in}}%
\pgfpathlineto{\pgfqpoint{1.800555in}{3.060652in}}%
\pgfpathlineto{\pgfqpoint{1.801110in}{3.183388in}}%
\pgfpathlineto{\pgfqpoint{1.801403in}{3.219064in}}%
\pgfpathlineto{\pgfqpoint{1.801326in}{3.106315in}}%
\pgfpathlineto{\pgfqpoint{1.802050in}{3.150688in}}%
\pgfpathlineto{\pgfqpoint{1.802883in}{3.077405in}}%
\pgfpathlineto{\pgfqpoint{1.802420in}{3.224919in}}%
\pgfpathlineto{\pgfqpoint{1.803083in}{3.187478in}}%
\pgfpathlineto{\pgfqpoint{1.803992in}{3.229632in}}%
\pgfpathlineto{\pgfqpoint{1.803653in}{3.095878in}}%
\pgfpathlineto{\pgfqpoint{1.804131in}{3.177089in}}%
\pgfpathlineto{\pgfqpoint{1.804455in}{3.078492in}}%
\pgfpathlineto{\pgfqpoint{1.804763in}{3.214786in}}%
\pgfpathlineto{\pgfqpoint{1.805256in}{3.141875in}}%
\pgfpathlineto{\pgfqpoint{1.806335in}{3.232454in}}%
\pgfpathlineto{\pgfqpoint{1.805996in}{3.098654in}}%
\pgfpathlineto{\pgfqpoint{1.806382in}{3.199835in}}%
\pgfpathlineto{\pgfqpoint{1.806782in}{3.076630in}}%
\pgfpathlineto{\pgfqpoint{1.807152in}{3.204712in}}%
\pgfpathlineto{\pgfqpoint{1.807584in}{3.117364in}}%
\pgfpathlineto{\pgfqpoint{1.807908in}{3.232300in}}%
\pgfpathlineto{\pgfqpoint{1.808355in}{3.094397in}}%
\pgfpathlineto{\pgfqpoint{1.808725in}{3.187272in}}%
\pgfpathlineto{\pgfqpoint{1.809110in}{3.091952in}}%
\pgfpathlineto{\pgfqpoint{1.809203in}{3.216219in}}%
\pgfpathlineto{\pgfqpoint{1.809943in}{3.146083in}}%
\pgfpathlineto{\pgfqpoint{1.810251in}{3.234912in}}%
\pgfpathlineto{\pgfqpoint{1.810682in}{3.084467in}}%
\pgfpathlineto{\pgfqpoint{1.811083in}{3.208447in}}%
\pgfpathlineto{\pgfqpoint{1.811453in}{3.070268in}}%
\pgfpathlineto{\pgfqpoint{1.811823in}{3.228785in}}%
\pgfpathlineto{\pgfqpoint{1.812301in}{3.153782in}}%
\pgfpathlineto{\pgfqpoint{1.812563in}{3.226949in}}%
\pgfpathlineto{\pgfqpoint{1.813010in}{3.080193in}}%
\pgfpathlineto{\pgfqpoint{1.813503in}{3.183892in}}%
\pgfpathlineto{\pgfqpoint{1.814567in}{3.093556in}}%
\pgfpathlineto{\pgfqpoint{1.814151in}{3.239202in}}%
\pgfpathlineto{\pgfqpoint{1.814629in}{3.146017in}}%
\pgfpathlineto{\pgfqpoint{1.815708in}{3.228101in}}%
\pgfpathlineto{\pgfqpoint{1.815353in}{3.067701in}}%
\pgfpathlineto{\pgfqpoint{1.815769in}{3.200394in}}%
\pgfpathlineto{\pgfqpoint{1.816910in}{3.071445in}}%
\pgfpathlineto{\pgfqpoint{1.816463in}{3.237908in}}%
\pgfpathlineto{\pgfqpoint{1.816956in}{3.151156in}}%
\pgfpathlineto{\pgfqpoint{1.818051in}{3.234735in}}%
\pgfpathlineto{\pgfqpoint{1.817681in}{3.095101in}}%
\pgfpathlineto{\pgfqpoint{1.818097in}{3.210774in}}%
\pgfpathlineto{\pgfqpoint{1.819253in}{3.062711in}}%
\pgfpathlineto{\pgfqpoint{1.818791in}{3.232348in}}%
\pgfpathlineto{\pgfqpoint{1.819284in}{3.094748in}}%
\pgfpathlineto{\pgfqpoint{1.820378in}{3.237353in}}%
\pgfpathlineto{\pgfqpoint{1.820425in}{3.206811in}}%
\pgfpathlineto{\pgfqpoint{1.820810in}{3.075700in}}%
\pgfpathlineto{\pgfqpoint{1.821026in}{3.223510in}}%
\pgfpathlineto{\pgfqpoint{1.821627in}{3.140824in}}%
\pgfpathlineto{\pgfqpoint{1.821920in}{3.237819in}}%
\pgfpathlineto{\pgfqpoint{1.822367in}{3.101284in}}%
\pgfpathlineto{\pgfqpoint{1.822752in}{3.182178in}}%
\pgfpathlineto{\pgfqpoint{1.823153in}{3.070898in}}%
\pgfpathlineto{\pgfqpoint{1.823508in}{3.219472in}}%
\pgfpathlineto{\pgfqpoint{1.823955in}{3.123718in}}%
\pgfpathlineto{\pgfqpoint{1.824294in}{3.235111in}}%
\pgfpathlineto{\pgfqpoint{1.824710in}{3.075817in}}%
\pgfpathlineto{\pgfqpoint{1.825095in}{3.175694in}}%
\pgfpathlineto{\pgfqpoint{1.825280in}{3.211169in}}%
\pgfpathlineto{\pgfqpoint{1.825388in}{3.144170in}}%
\pgfpathlineto{\pgfqpoint{1.825496in}{3.089130in}}%
\pgfpathlineto{\pgfqpoint{1.825851in}{3.240017in}}%
\pgfpathlineto{\pgfqpoint{1.826421in}{3.167427in}}%
\pgfpathlineto{\pgfqpoint{1.826606in}{3.242851in}}%
\pgfpathlineto{\pgfqpoint{1.827053in}{3.063257in}}%
\pgfpathlineto{\pgfqpoint{1.827531in}{3.166767in}}%
\pgfpathlineto{\pgfqpoint{1.828194in}{3.234031in}}%
\pgfpathlineto{\pgfqpoint{1.827824in}{3.107068in}}%
\pgfpathlineto{\pgfqpoint{1.828517in}{3.153937in}}%
\pgfpathlineto{\pgfqpoint{1.829396in}{3.087477in}}%
\pgfpathlineto{\pgfqpoint{1.828826in}{3.230788in}}%
\pgfpathlineto{\pgfqpoint{1.829550in}{3.193311in}}%
\pgfpathlineto{\pgfqpoint{1.830537in}{3.244328in}}%
\pgfpathlineto{\pgfqpoint{1.830167in}{3.099286in}}%
\pgfpathlineto{\pgfqpoint{1.830614in}{3.169936in}}%
\pgfpathlineto{\pgfqpoint{1.830953in}{3.057860in}}%
\pgfpathlineto{\pgfqpoint{1.831169in}{3.223063in}}%
\pgfpathlineto{\pgfqpoint{1.831754in}{3.123778in}}%
\pgfpathlineto{\pgfqpoint{1.832726in}{3.224409in}}%
\pgfpathlineto{\pgfqpoint{1.832510in}{3.105526in}}%
\pgfpathlineto{\pgfqpoint{1.832895in}{3.178429in}}%
\pgfpathlineto{\pgfqpoint{1.833281in}{3.087429in}}%
\pgfpathlineto{\pgfqpoint{1.833651in}{3.227451in}}%
\pgfpathlineto{\pgfqpoint{1.834098in}{3.121638in}}%
\pgfpathlineto{\pgfqpoint{1.834437in}{3.240966in}}%
\pgfpathlineto{\pgfqpoint{1.834853in}{3.057526in}}%
\pgfpathlineto{\pgfqpoint{1.835238in}{3.202388in}}%
\pgfpathlineto{\pgfqpoint{1.835624in}{3.098851in}}%
\pgfpathlineto{\pgfqpoint{1.835994in}{3.217183in}}%
\pgfpathlineto{\pgfqpoint{1.836471in}{3.155761in}}%
\pgfpathlineto{\pgfqpoint{1.837520in}{3.226977in}}%
\pgfpathlineto{\pgfqpoint{1.837181in}{3.084434in}}%
\pgfpathlineto{\pgfqpoint{1.837612in}{3.201844in}}%
\pgfpathlineto{\pgfqpoint{1.838753in}{3.067382in}}%
\pgfpathlineto{\pgfqpoint{1.838337in}{3.228312in}}%
\pgfpathlineto{\pgfqpoint{1.838922in}{3.181155in}}%
\pgfpathlineto{\pgfqpoint{1.838969in}{3.218078in}}%
\pgfpathlineto{\pgfqpoint{1.839524in}{3.094077in}}%
\pgfpathlineto{\pgfqpoint{1.840017in}{3.178432in}}%
\pgfpathlineto{\pgfqpoint{1.841080in}{3.079248in}}%
\pgfpathlineto{\pgfqpoint{1.840633in}{3.227369in}}%
\pgfpathlineto{\pgfqpoint{1.841142in}{3.166251in}}%
\pgfpathlineto{\pgfqpoint{1.841296in}{3.228328in}}%
\pgfpathlineto{\pgfqpoint{1.841867in}{3.082655in}}%
\pgfpathlineto{\pgfqpoint{1.842267in}{3.207573in}}%
\pgfpathlineto{\pgfqpoint{1.842653in}{3.079044in}}%
\pgfpathlineto{\pgfqpoint{1.842961in}{3.222184in}}%
\pgfpathlineto{\pgfqpoint{1.843470in}{3.160047in}}%
\pgfpathlineto{\pgfqpoint{1.844533in}{3.231445in}}%
\pgfpathlineto{\pgfqpoint{1.844194in}{3.089533in}}%
\pgfpathlineto{\pgfqpoint{1.844595in}{3.189840in}}%
\pgfpathlineto{\pgfqpoint{1.844980in}{3.073687in}}%
\pgfpathlineto{\pgfqpoint{1.845335in}{3.225780in}}%
\pgfpathlineto{\pgfqpoint{1.845782in}{3.095284in}}%
\pgfpathlineto{\pgfqpoint{1.846861in}{3.231723in}}%
\pgfpathlineto{\pgfqpoint{1.846537in}{3.093342in}}%
\pgfpathlineto{\pgfqpoint{1.846938in}{3.187714in}}%
\pgfpathlineto{\pgfqpoint{1.848094in}{3.092705in}}%
\pgfpathlineto{\pgfqpoint{1.847678in}{3.207207in}}%
\pgfpathlineto{\pgfqpoint{1.848125in}{3.125506in}}%
\pgfpathlineto{\pgfqpoint{1.848449in}{3.229254in}}%
\pgfpathlineto{\pgfqpoint{1.848880in}{3.088056in}}%
\pgfpathlineto{\pgfqpoint{1.849266in}{3.204191in}}%
\pgfpathlineto{\pgfqpoint{1.849651in}{3.077010in}}%
\pgfpathlineto{\pgfqpoint{1.849975in}{3.218057in}}%
\pgfpathlineto{\pgfqpoint{1.850484in}{3.146966in}}%
\pgfpathlineto{\pgfqpoint{1.850792in}{3.225620in}}%
\pgfpathlineto{\pgfqpoint{1.851193in}{3.099452in}}%
\pgfpathlineto{\pgfqpoint{1.851640in}{3.180789in}}%
\pgfpathlineto{\pgfqpoint{1.851979in}{3.092438in}}%
\pgfpathlineto{\pgfqpoint{1.852333in}{3.230420in}}%
\pgfpathlineto{\pgfqpoint{1.852780in}{3.123536in}}%
\pgfpathlineto{\pgfqpoint{1.853073in}{3.226999in}}%
\pgfpathlineto{\pgfqpoint{1.853520in}{3.084400in}}%
\pgfpathlineto{\pgfqpoint{1.853952in}{3.190630in}}%
\pgfpathlineto{\pgfqpoint{1.855077in}{3.103065in}}%
\pgfpathlineto{\pgfqpoint{1.854676in}{3.225176in}}%
\pgfpathlineto{\pgfqpoint{1.855123in}{3.151528in}}%
\pgfpathlineto{\pgfqpoint{1.856218in}{3.231220in}}%
\pgfpathlineto{\pgfqpoint{1.855848in}{3.101501in}}%
\pgfpathlineto{\pgfqpoint{1.856264in}{3.212432in}}%
\pgfpathlineto{\pgfqpoint{1.857405in}{3.072248in}}%
\pgfpathlineto{\pgfqpoint{1.857004in}{3.227486in}}%
\pgfpathlineto{\pgfqpoint{1.857451in}{3.140205in}}%
\pgfpathlineto{\pgfqpoint{1.857775in}{3.219638in}}%
\pgfpathlineto{\pgfqpoint{1.858176in}{3.103987in}}%
\pgfpathlineto{\pgfqpoint{1.858607in}{3.197990in}}%
\pgfpathlineto{\pgfqpoint{1.859717in}{3.094217in}}%
\pgfpathlineto{\pgfqpoint{1.859162in}{3.211643in}}%
\pgfpathlineto{\pgfqpoint{1.859763in}{3.149780in}}%
\pgfpathlineto{\pgfqpoint{1.860873in}{3.223744in}}%
\pgfpathlineto{\pgfqpoint{1.860503in}{3.085044in}}%
\pgfpathlineto{\pgfqpoint{1.860919in}{3.205601in}}%
\pgfpathlineto{\pgfqpoint{1.861274in}{3.074024in}}%
\pgfpathlineto{\pgfqpoint{1.861675in}{3.212873in}}%
\pgfpathlineto{\pgfqpoint{1.862075in}{3.131637in}}%
\pgfpathlineto{\pgfqpoint{1.863062in}{3.207045in}}%
\pgfpathlineto{\pgfqpoint{1.862831in}{3.093349in}}%
\pgfpathlineto{\pgfqpoint{1.863216in}{3.199326in}}%
\pgfpathlineto{\pgfqpoint{1.863602in}{3.078984in}}%
\pgfpathlineto{\pgfqpoint{1.864002in}{3.219836in}}%
\pgfpathlineto{\pgfqpoint{1.864419in}{3.158066in}}%
\pgfpathlineto{\pgfqpoint{1.864773in}{3.221905in}}%
\pgfpathlineto{\pgfqpoint{1.865158in}{3.090320in}}%
\pgfpathlineto{\pgfqpoint{1.865544in}{3.213071in}}%
\pgfpathlineto{\pgfqpoint{1.866700in}{3.096851in}}%
\pgfpathlineto{\pgfqpoint{1.866315in}{3.217266in}}%
\pgfpathlineto{\pgfqpoint{1.866746in}{3.167081in}}%
\pgfpathlineto{\pgfqpoint{1.867486in}{3.098452in}}%
\pgfpathlineto{\pgfqpoint{1.867871in}{3.227627in}}%
\pgfpathlineto{\pgfqpoint{1.868257in}{3.092983in}}%
\pgfpathlineto{\pgfqpoint{1.869058in}{3.137636in}}%
\pgfpathlineto{\pgfqpoint{1.869413in}{3.227489in}}%
\pgfpathlineto{\pgfqpoint{1.869798in}{3.114584in}}%
\pgfpathlineto{\pgfqpoint{1.870230in}{3.185698in}}%
\pgfpathlineto{\pgfqpoint{1.870569in}{3.107565in}}%
\pgfpathlineto{\pgfqpoint{1.870970in}{3.229945in}}%
\pgfpathlineto{\pgfqpoint{1.871355in}{3.122272in}}%
\pgfpathlineto{\pgfqpoint{1.871741in}{3.227928in}}%
\pgfpathlineto{\pgfqpoint{1.872126in}{3.092688in}}%
\pgfpathlineto{\pgfqpoint{1.872527in}{3.224969in}}%
\pgfpathlineto{\pgfqpoint{1.873498in}{3.120385in}}%
\pgfpathlineto{\pgfqpoint{1.873297in}{3.241219in}}%
\pgfpathlineto{\pgfqpoint{1.873698in}{3.158242in}}%
\pgfpathlineto{\pgfqpoint{1.874854in}{3.235954in}}%
\pgfpathlineto{\pgfqpoint{1.874454in}{3.123978in}}%
\pgfpathlineto{\pgfqpoint{1.874870in}{3.218207in}}%
\pgfpathlineto{\pgfqpoint{1.875055in}{3.101107in}}%
\pgfpathlineto{\pgfqpoint{1.875625in}{3.231050in}}%
\pgfpathlineto{\pgfqpoint{1.876026in}{3.139760in}}%
\pgfpathlineto{\pgfqpoint{1.877182in}{3.224953in}}%
\pgfpathlineto{\pgfqpoint{1.876612in}{3.124352in}}%
\pgfpathlineto{\pgfqpoint{1.877213in}{3.193728in}}%
\pgfpathlineto{\pgfqpoint{1.877382in}{3.119501in}}%
\pgfpathlineto{\pgfqpoint{1.877799in}{3.208969in}}%
\pgfpathlineto{\pgfqpoint{1.878338in}{3.141766in}}%
\pgfpathlineto{\pgfqpoint{1.878723in}{3.222294in}}%
\pgfpathlineto{\pgfqpoint{1.878939in}{3.099361in}}%
\pgfpathlineto{\pgfqpoint{1.879494in}{3.218352in}}%
\pgfpathlineto{\pgfqpoint{1.880481in}{3.124128in}}%
\pgfpathlineto{\pgfqpoint{1.880666in}{3.158552in}}%
\pgfpathlineto{\pgfqpoint{1.881668in}{3.213193in}}%
\pgfpathlineto{\pgfqpoint{1.881236in}{3.118645in}}%
\pgfpathlineto{\pgfqpoint{1.881791in}{3.181753in}}%
\pgfpathlineto{\pgfqpoint{1.882593in}{3.214850in}}%
\pgfpathlineto{\pgfqpoint{1.882808in}{3.114400in}}%
\pgfpathlineto{\pgfqpoint{1.883363in}{3.214095in}}%
\pgfpathlineto{\pgfqpoint{1.884180in}{3.176849in}}%
\pgfpathlineto{\pgfqpoint{1.885136in}{3.116713in}}%
\pgfpathlineto{\pgfqpoint{1.884920in}{3.205990in}}%
\pgfpathlineto{\pgfqpoint{1.885306in}{3.144805in}}%
\pgfpathlineto{\pgfqpoint{1.886308in}{3.226067in}}%
\pgfpathlineto{\pgfqpoint{1.885876in}{3.107100in}}%
\pgfpathlineto{\pgfqpoint{1.886431in}{3.188542in}}%
\pgfpathlineto{\pgfqpoint{1.886462in}{3.216899in}}%
\pgfpathlineto{\pgfqpoint{1.886678in}{3.114437in}}%
\pgfpathlineto{\pgfqpoint{1.887417in}{3.138849in}}%
\pgfpathlineto{\pgfqpoint{1.888204in}{3.118976in}}%
\pgfpathlineto{\pgfqpoint{1.887864in}{3.218206in}}%
\pgfpathlineto{\pgfqpoint{1.888450in}{3.180295in}}%
\pgfpathlineto{\pgfqpoint{1.889406in}{3.229396in}}%
\pgfpathlineto{\pgfqpoint{1.889005in}{3.124638in}}%
\pgfpathlineto{\pgfqpoint{1.889483in}{3.175422in}}%
\pgfpathlineto{\pgfqpoint{1.889776in}{3.105550in}}%
\pgfpathlineto{\pgfqpoint{1.890192in}{3.227474in}}%
\pgfpathlineto{\pgfqpoint{1.890593in}{3.141447in}}%
\pgfpathlineto{\pgfqpoint{1.890963in}{3.227490in}}%
\pgfpathlineto{\pgfqpoint{1.890716in}{3.122884in}}%
\pgfpathlineto{\pgfqpoint{1.891749in}{3.208817in}}%
\pgfpathlineto{\pgfqpoint{1.892104in}{3.114647in}}%
\pgfpathlineto{\pgfqpoint{1.892504in}{3.225517in}}%
\pgfpathlineto{\pgfqpoint{1.893090in}{3.159423in}}%
\pgfpathlineto{\pgfqpoint{1.893275in}{3.233773in}}%
\pgfpathlineto{\pgfqpoint{1.893645in}{3.109529in}}%
\pgfpathlineto{\pgfqpoint{1.894231in}{3.188481in}}%
\pgfpathlineto{\pgfqpoint{1.894400in}{3.121016in}}%
\pgfpathlineto{\pgfqpoint{1.894832in}{3.235420in}}%
\pgfpathlineto{\pgfqpoint{1.895387in}{3.160448in}}%
\pgfpathlineto{\pgfqpoint{1.896389in}{3.231512in}}%
\pgfpathlineto{\pgfqpoint{1.895973in}{3.120783in}}%
\pgfpathlineto{\pgfqpoint{1.896543in}{3.193812in}}%
\pgfpathlineto{\pgfqpoint{1.897499in}{3.113302in}}%
\pgfpathlineto{\pgfqpoint{1.897160in}{3.234356in}}%
\pgfpathlineto{\pgfqpoint{1.897699in}{3.135375in}}%
\pgfpathlineto{\pgfqpoint{1.898701in}{3.232925in}}%
\pgfpathlineto{\pgfqpoint{1.898269in}{3.118113in}}%
\pgfpathlineto{\pgfqpoint{1.898824in}{3.187906in}}%
\pgfpathlineto{\pgfqpoint{1.899826in}{3.123546in}}%
\pgfpathlineto{\pgfqpoint{1.899472in}{3.227918in}}%
\pgfpathlineto{\pgfqpoint{1.900027in}{3.140434in}}%
\pgfpathlineto{\pgfqpoint{1.901029in}{3.238674in}}%
\pgfpathlineto{\pgfqpoint{1.900597in}{3.116157in}}%
\pgfpathlineto{\pgfqpoint{1.901152in}{3.192467in}}%
\pgfpathlineto{\pgfqpoint{1.901368in}{3.115341in}}%
\pgfpathlineto{\pgfqpoint{1.901799in}{3.234608in}}%
\pgfpathlineto{\pgfqpoint{1.902354in}{3.132367in}}%
\pgfpathlineto{\pgfqpoint{1.903341in}{3.231196in}}%
\pgfpathlineto{\pgfqpoint{1.902925in}{3.120195in}}%
\pgfpathlineto{\pgfqpoint{1.903511in}{3.186123in}}%
\pgfpathlineto{\pgfqpoint{1.904466in}{3.107910in}}%
\pgfpathlineto{\pgfqpoint{1.904127in}{3.236647in}}%
\pgfpathlineto{\pgfqpoint{1.904667in}{3.121644in}}%
\pgfpathlineto{\pgfqpoint{1.905669in}{3.242965in}}%
\pgfpathlineto{\pgfqpoint{1.905437in}{3.104271in}}%
\pgfpathlineto{\pgfqpoint{1.905807in}{3.184594in}}%
\pgfpathlineto{\pgfqpoint{1.906455in}{3.230005in}}%
\pgfpathlineto{\pgfqpoint{1.906208in}{3.115793in}}%
\pgfpathlineto{\pgfqpoint{1.906640in}{3.173693in}}%
\pgfpathlineto{\pgfqpoint{1.907750in}{3.113350in}}%
\pgfpathlineto{\pgfqpoint{1.907226in}{3.240971in}}%
\pgfpathlineto{\pgfqpoint{1.907765in}{3.113598in}}%
\pgfpathlineto{\pgfqpoint{1.907996in}{3.251993in}}%
\pgfpathlineto{\pgfqpoint{1.908351in}{3.104879in}}%
\pgfpathlineto{\pgfqpoint{1.908937in}{3.212207in}}%
\pgfpathlineto{\pgfqpoint{1.910077in}{3.109253in}}%
\pgfpathlineto{\pgfqpoint{1.909538in}{3.250375in}}%
\pgfpathlineto{\pgfqpoint{1.910093in}{3.116381in}}%
\pgfpathlineto{\pgfqpoint{1.911095in}{3.243306in}}%
\pgfpathlineto{\pgfqpoint{1.910694in}{3.103935in}}%
\pgfpathlineto{\pgfqpoint{1.911249in}{3.200164in}}%
\pgfpathlineto{\pgfqpoint{1.912251in}{3.099823in}}%
\pgfpathlineto{\pgfqpoint{1.911865in}{3.253229in}}%
\pgfpathlineto{\pgfqpoint{1.912451in}{3.164231in}}%
\pgfpathlineto{\pgfqpoint{1.913422in}{3.255349in}}%
\pgfpathlineto{\pgfqpoint{1.913021in}{3.102094in}}%
\pgfpathlineto{\pgfqpoint{1.913576in}{3.208923in}}%
\pgfpathlineto{\pgfqpoint{1.914563in}{3.104766in}}%
\pgfpathlineto{\pgfqpoint{1.914193in}{3.247248in}}%
\pgfpathlineto{\pgfqpoint{1.914748in}{3.144142in}}%
\pgfpathlineto{\pgfqpoint{1.915750in}{3.255593in}}%
\pgfpathlineto{\pgfqpoint{1.915334in}{3.102051in}}%
\pgfpathlineto{\pgfqpoint{1.915904in}{3.204133in}}%
\pgfpathlineto{\pgfqpoint{1.916891in}{3.096984in}}%
\pgfpathlineto{\pgfqpoint{1.916521in}{3.247999in}}%
\pgfpathlineto{\pgfqpoint{1.917060in}{3.118584in}}%
\pgfpathlineto{\pgfqpoint{1.918062in}{3.251595in}}%
\pgfpathlineto{\pgfqpoint{1.917661in}{3.109416in}}%
\pgfpathlineto{\pgfqpoint{1.918201in}{3.182132in}}%
\pgfpathlineto{\pgfqpoint{1.918848in}{3.243848in}}%
\pgfpathlineto{\pgfqpoint{1.919218in}{3.095925in}}%
\pgfpathlineto{\pgfqpoint{1.919295in}{3.167146in}}%
\pgfpathlineto{\pgfqpoint{1.919989in}{3.099762in}}%
\pgfpathlineto{\pgfqpoint{1.919619in}{3.256283in}}%
\pgfpathlineto{\pgfqpoint{1.920282in}{3.211625in}}%
\pgfpathlineto{\pgfqpoint{1.920390in}{3.256290in}}%
\pgfpathlineto{\pgfqpoint{1.920760in}{3.103430in}}%
\pgfpathlineto{\pgfqpoint{1.921345in}{3.191380in}}%
\pgfpathlineto{\pgfqpoint{1.921530in}{3.104132in}}%
\pgfpathlineto{\pgfqpoint{1.921947in}{3.252094in}}%
\pgfpathlineto{\pgfqpoint{1.922486in}{3.130118in}}%
\pgfpathlineto{\pgfqpoint{1.922717in}{3.252646in}}%
\pgfpathlineto{\pgfqpoint{1.923087in}{3.093234in}}%
\pgfpathlineto{\pgfqpoint{1.923642in}{3.213863in}}%
\pgfpathlineto{\pgfqpoint{1.923858in}{3.100113in}}%
\pgfpathlineto{\pgfqpoint{1.924259in}{3.258322in}}%
\pgfpathlineto{\pgfqpoint{1.924814in}{3.139802in}}%
\pgfpathlineto{\pgfqpoint{1.925816in}{3.251756in}}%
\pgfpathlineto{\pgfqpoint{1.925400in}{3.097952in}}%
\pgfpathlineto{\pgfqpoint{1.925970in}{3.203798in}}%
\pgfpathlineto{\pgfqpoint{1.926956in}{3.094455in}}%
\pgfpathlineto{\pgfqpoint{1.926587in}{3.260917in}}%
\pgfpathlineto{\pgfqpoint{1.927126in}{3.130411in}}%
\pgfpathlineto{\pgfqpoint{1.928143in}{3.259297in}}%
\pgfpathlineto{\pgfqpoint{1.927727in}{3.096137in}}%
\pgfpathlineto{\pgfqpoint{1.928267in}{3.183731in}}%
\pgfpathlineto{\pgfqpoint{1.928914in}{3.246720in}}%
\pgfpathlineto{\pgfqpoint{1.929284in}{3.092905in}}%
\pgfpathlineto{\pgfqpoint{1.929346in}{3.160910in}}%
\pgfpathlineto{\pgfqpoint{1.930055in}{3.097883in}}%
\pgfpathlineto{\pgfqpoint{1.929685in}{3.250493in}}%
\pgfpathlineto{\pgfqpoint{1.930394in}{3.224145in}}%
\pgfpathlineto{\pgfqpoint{1.930471in}{3.261549in}}%
\pgfpathlineto{\pgfqpoint{1.930826in}{3.096794in}}%
\pgfpathlineto{\pgfqpoint{1.931427in}{3.169088in}}%
\pgfpathlineto{\pgfqpoint{1.931596in}{3.087156in}}%
\pgfpathlineto{\pgfqpoint{1.932013in}{3.256693in}}%
\pgfpathlineto{\pgfqpoint{1.932552in}{3.133839in}}%
\pgfpathlineto{\pgfqpoint{1.932783in}{3.254124in}}%
\pgfpathlineto{\pgfqpoint{1.933153in}{3.091778in}}%
\pgfpathlineto{\pgfqpoint{1.933708in}{3.212767in}}%
\pgfpathlineto{\pgfqpoint{1.933924in}{3.083196in}}%
\pgfpathlineto{\pgfqpoint{1.934340in}{3.264219in}}%
\pgfpathlineto{\pgfqpoint{1.934895in}{3.153638in}}%
\pgfpathlineto{\pgfqpoint{1.935111in}{3.256262in}}%
\pgfpathlineto{\pgfqpoint{1.935481in}{3.080859in}}%
\pgfpathlineto{\pgfqpoint{1.936051in}{3.197989in}}%
\pgfpathlineto{\pgfqpoint{1.936252in}{3.083585in}}%
\pgfpathlineto{\pgfqpoint{1.936668in}{3.264214in}}%
\pgfpathlineto{\pgfqpoint{1.937192in}{3.121315in}}%
\pgfpathlineto{\pgfqpoint{1.938225in}{3.272783in}}%
\pgfpathlineto{\pgfqpoint{1.937793in}{3.072507in}}%
\pgfpathlineto{\pgfqpoint{1.938333in}{3.190978in}}%
\pgfpathlineto{\pgfqpoint{1.938995in}{3.276801in}}%
\pgfpathlineto{\pgfqpoint{1.939350in}{3.086238in}}%
\pgfpathlineto{\pgfqpoint{1.939412in}{3.147763in}}%
\pgfpathlineto{\pgfqpoint{1.940121in}{3.075702in}}%
\pgfpathlineto{\pgfqpoint{1.939766in}{3.262631in}}%
\pgfpathlineto{\pgfqpoint{1.940460in}{3.212996in}}%
\pgfpathlineto{\pgfqpoint{1.940537in}{3.271641in}}%
\pgfpathlineto{\pgfqpoint{1.940891in}{3.082780in}}%
\pgfpathlineto{\pgfqpoint{1.941493in}{3.165184in}}%
\pgfpathlineto{\pgfqpoint{1.941662in}{3.067748in}}%
\pgfpathlineto{\pgfqpoint{1.942094in}{3.269449in}}%
\pgfpathlineto{\pgfqpoint{1.942603in}{3.111642in}}%
\pgfpathlineto{\pgfqpoint{1.942865in}{3.281067in}}%
\pgfpathlineto{\pgfqpoint{1.943219in}{3.082624in}}%
\pgfpathlineto{\pgfqpoint{1.943759in}{3.198688in}}%
\pgfpathlineto{\pgfqpoint{1.944421in}{3.274987in}}%
\pgfpathlineto{\pgfqpoint{1.943990in}{3.068290in}}%
\pgfpathlineto{\pgfqpoint{1.944745in}{3.084376in}}%
\pgfpathlineto{\pgfqpoint{1.945531in}{3.073714in}}%
\pgfpathlineto{\pgfqpoint{1.945192in}{3.276654in}}%
\pgfpathlineto{\pgfqpoint{1.945716in}{3.147997in}}%
\pgfpathlineto{\pgfqpoint{1.946734in}{3.280267in}}%
\pgfpathlineto{\pgfqpoint{1.946302in}{3.067704in}}%
\pgfpathlineto{\pgfqpoint{1.946872in}{3.213419in}}%
\pgfpathlineto{\pgfqpoint{1.947859in}{3.065081in}}%
\pgfpathlineto{\pgfqpoint{1.947504in}{3.269662in}}%
\pgfpathlineto{\pgfqpoint{1.948044in}{3.155132in}}%
\pgfpathlineto{\pgfqpoint{1.949061in}{3.277194in}}%
\pgfpathlineto{\pgfqpoint{1.948630in}{3.061603in}}%
\pgfpathlineto{\pgfqpoint{1.949200in}{3.207657in}}%
\pgfpathlineto{\pgfqpoint{1.950171in}{3.065073in}}%
\pgfpathlineto{\pgfqpoint{1.949832in}{3.272648in}}%
\pgfpathlineto{\pgfqpoint{1.950356in}{3.144226in}}%
\pgfpathlineto{\pgfqpoint{1.951389in}{3.278910in}}%
\pgfpathlineto{\pgfqpoint{1.950957in}{3.070361in}}%
\pgfpathlineto{\pgfqpoint{1.951512in}{3.211769in}}%
\pgfpathlineto{\pgfqpoint{1.952499in}{3.056574in}}%
\pgfpathlineto{\pgfqpoint{1.952160in}{3.275077in}}%
\pgfpathlineto{\pgfqpoint{1.952699in}{3.163417in}}%
\pgfpathlineto{\pgfqpoint{1.952930in}{3.285498in}}%
\pgfpathlineto{\pgfqpoint{1.953270in}{3.064058in}}%
\pgfpathlineto{\pgfqpoint{1.953855in}{3.200640in}}%
\pgfpathlineto{\pgfqpoint{1.954826in}{3.065716in}}%
\pgfpathlineto{\pgfqpoint{1.954487in}{3.273524in}}%
\pgfpathlineto{\pgfqpoint{1.954996in}{3.124364in}}%
\pgfpathlineto{\pgfqpoint{1.955243in}{3.284396in}}%
\pgfpathlineto{\pgfqpoint{1.955736in}{3.068904in}}%
\pgfpathlineto{\pgfqpoint{1.956152in}{3.211327in}}%
\pgfpathlineto{\pgfqpoint{1.956368in}{3.061049in}}%
\pgfpathlineto{\pgfqpoint{1.956800in}{3.283193in}}%
\pgfpathlineto{\pgfqpoint{1.957401in}{3.161874in}}%
\pgfpathlineto{\pgfqpoint{1.957540in}{3.291687in}}%
\pgfpathlineto{\pgfqpoint{1.958064in}{3.066258in}}%
\pgfpathlineto{\pgfqpoint{1.958511in}{3.195068in}}%
\pgfpathlineto{\pgfqpoint{1.958696in}{3.065392in}}%
\pgfpathlineto{\pgfqpoint{1.959096in}{3.288769in}}%
\pgfpathlineto{\pgfqpoint{1.959667in}{3.114659in}}%
\pgfpathlineto{\pgfqpoint{1.959867in}{3.299467in}}%
\pgfpathlineto{\pgfqpoint{1.960253in}{3.074630in}}%
\pgfpathlineto{\pgfqpoint{1.960838in}{3.194159in}}%
\pgfpathlineto{\pgfqpoint{1.961948in}{3.071141in}}%
\pgfpathlineto{\pgfqpoint{1.961424in}{3.303737in}}%
\pgfpathlineto{\pgfqpoint{1.961994in}{3.105147in}}%
\pgfpathlineto{\pgfqpoint{1.962195in}{3.297626in}}%
\pgfpathlineto{\pgfqpoint{1.962734in}{3.078289in}}%
\pgfpathlineto{\pgfqpoint{1.963151in}{3.200689in}}%
\pgfpathlineto{\pgfqpoint{1.964276in}{3.064742in}}%
\pgfpathlineto{\pgfqpoint{1.963736in}{3.299368in}}%
\pgfpathlineto{\pgfqpoint{1.964322in}{3.094206in}}%
\pgfpathlineto{\pgfqpoint{1.965293in}{3.305865in}}%
\pgfpathlineto{\pgfqpoint{1.965062in}{3.065826in}}%
\pgfpathlineto{\pgfqpoint{1.965478in}{3.197556in}}%
\pgfpathlineto{\pgfqpoint{1.965833in}{3.064991in}}%
\pgfpathlineto{\pgfqpoint{1.966064in}{3.307561in}}%
\pgfpathlineto{\pgfqpoint{1.966650in}{3.091894in}}%
\pgfpathlineto{\pgfqpoint{1.967621in}{3.313643in}}%
\pgfpathlineto{\pgfqpoint{1.967390in}{3.067518in}}%
\pgfpathlineto{\pgfqpoint{1.967806in}{3.195815in}}%
\pgfpathlineto{\pgfqpoint{1.968946in}{3.051738in}}%
\pgfpathlineto{\pgfqpoint{1.968392in}{3.310992in}}%
\pgfpathlineto{\pgfqpoint{1.968993in}{3.116310in}}%
\pgfpathlineto{\pgfqpoint{1.969948in}{3.314839in}}%
\pgfpathlineto{\pgfqpoint{1.969717in}{3.053660in}}%
\pgfpathlineto{\pgfqpoint{1.970118in}{3.202198in}}%
\pgfpathlineto{\pgfqpoint{1.970503in}{3.050310in}}%
\pgfpathlineto{\pgfqpoint{1.970719in}{3.311811in}}%
\pgfpathlineto{\pgfqpoint{1.971336in}{3.162235in}}%
\pgfpathlineto{\pgfqpoint{1.972276in}{3.319439in}}%
\pgfpathlineto{\pgfqpoint{1.972045in}{3.043142in}}%
\pgfpathlineto{\pgfqpoint{1.972461in}{3.207631in}}%
\pgfpathlineto{\pgfqpoint{1.972831in}{3.040840in}}%
\pgfpathlineto{\pgfqpoint{1.973047in}{3.310260in}}%
\pgfpathlineto{\pgfqpoint{1.973633in}{3.080274in}}%
\pgfpathlineto{\pgfqpoint{1.974604in}{3.322143in}}%
\pgfpathlineto{\pgfqpoint{1.974373in}{3.038168in}}%
\pgfpathlineto{\pgfqpoint{1.974789in}{3.213878in}}%
\pgfpathlineto{\pgfqpoint{1.975929in}{3.029379in}}%
\pgfpathlineto{\pgfqpoint{1.975374in}{3.315266in}}%
\pgfpathlineto{\pgfqpoint{1.975960in}{3.072390in}}%
\pgfpathlineto{\pgfqpoint{1.976161in}{3.322935in}}%
\pgfpathlineto{\pgfqpoint{1.976716in}{3.029454in}}%
\pgfpathlineto{\pgfqpoint{1.977116in}{3.221660in}}%
\pgfpathlineto{\pgfqpoint{1.978257in}{3.023678in}}%
\pgfpathlineto{\pgfqpoint{1.977702in}{3.315883in}}%
\pgfpathlineto{\pgfqpoint{1.978288in}{3.074057in}}%
\pgfpathlineto{\pgfqpoint{1.978488in}{3.323779in}}%
\pgfpathlineto{\pgfqpoint{1.979043in}{3.028989in}}%
\pgfpathlineto{\pgfqpoint{1.979444in}{3.224800in}}%
\pgfpathlineto{\pgfqpoint{1.979814in}{3.021744in}}%
\pgfpathlineto{\pgfqpoint{1.980045in}{3.318206in}}%
\pgfpathlineto{\pgfqpoint{1.980616in}{3.067836in}}%
\pgfpathlineto{\pgfqpoint{1.980816in}{3.314812in}}%
\pgfpathlineto{\pgfqpoint{1.981371in}{3.033466in}}%
\pgfpathlineto{\pgfqpoint{1.981772in}{3.220527in}}%
\pgfpathlineto{\pgfqpoint{1.982142in}{3.030325in}}%
\pgfpathlineto{\pgfqpoint{1.982373in}{3.318423in}}%
\pgfpathlineto{\pgfqpoint{1.982943in}{3.076706in}}%
\pgfpathlineto{\pgfqpoint{1.983144in}{3.315740in}}%
\pgfpathlineto{\pgfqpoint{1.983698in}{3.047006in}}%
\pgfpathlineto{\pgfqpoint{1.984099in}{3.219063in}}%
\pgfpathlineto{\pgfqpoint{1.984469in}{3.052105in}}%
\pgfpathlineto{\pgfqpoint{1.984700in}{3.306053in}}%
\pgfpathlineto{\pgfqpoint{1.985271in}{3.088911in}}%
\pgfpathlineto{\pgfqpoint{1.985471in}{3.297946in}}%
\pgfpathlineto{\pgfqpoint{1.986026in}{3.067593in}}%
\pgfpathlineto{\pgfqpoint{1.986442in}{3.181180in}}%
\pgfpathlineto{\pgfqpoint{1.986797in}{3.078256in}}%
\pgfpathlineto{\pgfqpoint{1.987028in}{3.279858in}}%
\pgfpathlineto{\pgfqpoint{1.987614in}{3.100785in}}%
\pgfpathlineto{\pgfqpoint{1.987799in}{3.263011in}}%
\pgfpathlineto{\pgfqpoint{1.988862in}{3.201739in}}%
\pgfpathlineto{\pgfqpoint{1.989726in}{3.113456in}}%
\pgfpathlineto{\pgfqpoint{1.989356in}{3.252718in}}%
\pgfpathlineto{\pgfqpoint{1.989988in}{3.163058in}}%
\pgfpathlineto{\pgfqpoint{1.990126in}{3.247174in}}%
\pgfpathlineto{\pgfqpoint{1.990974in}{3.097748in}}%
\pgfpathlineto{\pgfqpoint{1.991098in}{3.180681in}}%
\pgfpathlineto{\pgfqpoint{1.991745in}{3.099095in}}%
\pgfpathlineto{\pgfqpoint{1.991668in}{3.241967in}}%
\pgfpathlineto{\pgfqpoint{1.992285in}{3.157815in}}%
\pgfpathlineto{\pgfqpoint{1.992439in}{3.253362in}}%
\pgfpathlineto{\pgfqpoint{1.993302in}{3.098327in}}%
\pgfpathlineto{\pgfqpoint{1.993425in}{3.186279in}}%
\pgfpathlineto{\pgfqpoint{1.994073in}{3.103328in}}%
\pgfpathlineto{\pgfqpoint{1.994011in}{3.235124in}}%
\pgfpathlineto{\pgfqpoint{1.994597in}{3.125555in}}%
\pgfpathlineto{\pgfqpoint{1.995059in}{3.222538in}}%
\pgfpathlineto{\pgfqpoint{1.995630in}{3.066888in}}%
\pgfpathlineto{\pgfqpoint{1.995737in}{3.202727in}}%
\pgfpathlineto{\pgfqpoint{1.996400in}{3.079548in}}%
\pgfpathlineto{\pgfqpoint{1.995830in}{3.239752in}}%
\pgfpathlineto{\pgfqpoint{1.996909in}{3.100938in}}%
\pgfpathlineto{\pgfqpoint{1.998158in}{3.245417in}}%
\pgfpathlineto{\pgfqpoint{1.997957in}{3.085427in}}%
\pgfpathlineto{\pgfqpoint{1.998343in}{3.203514in}}%
\pgfpathlineto{\pgfqpoint{1.998728in}{3.078823in}}%
\pgfpathlineto{\pgfqpoint{1.999067in}{3.248274in}}%
\pgfpathlineto{\pgfqpoint{1.999530in}{3.119223in}}%
\pgfpathlineto{\pgfqpoint{1.999714in}{3.257074in}}%
\pgfpathlineto{\pgfqpoint{2.000285in}{3.076373in}}%
\pgfpathlineto{\pgfqpoint{2.000670in}{3.214155in}}%
\pgfpathlineto{\pgfqpoint{2.001826in}{3.085821in}}%
\pgfpathlineto{\pgfqpoint{2.001379in}{3.262267in}}%
\pgfpathlineto{\pgfqpoint{2.001857in}{3.125377in}}%
\pgfpathlineto{\pgfqpoint{2.002936in}{3.275465in}}%
\pgfpathlineto{\pgfqpoint{2.002597in}{3.080766in}}%
\pgfpathlineto{\pgfqpoint{2.003013in}{3.217792in}}%
\pgfpathlineto{\pgfqpoint{2.004154in}{3.073050in}}%
\pgfpathlineto{\pgfqpoint{2.003707in}{3.271421in}}%
\pgfpathlineto{\pgfqpoint{2.004216in}{3.157478in}}%
\pgfpathlineto{\pgfqpoint{2.005264in}{3.276487in}}%
\pgfpathlineto{\pgfqpoint{2.004925in}{3.083255in}}%
\pgfpathlineto{\pgfqpoint{2.005372in}{3.219832in}}%
\pgfpathlineto{\pgfqpoint{2.006482in}{3.073528in}}%
\pgfpathlineto{\pgfqpoint{2.006035in}{3.279584in}}%
\pgfpathlineto{\pgfqpoint{2.006512in}{3.121891in}}%
\pgfpathlineto{\pgfqpoint{2.006805in}{3.285878in}}%
\pgfpathlineto{\pgfqpoint{2.007252in}{3.077700in}}%
\pgfpathlineto{\pgfqpoint{2.007699in}{3.210225in}}%
\pgfpathlineto{\pgfqpoint{2.008748in}{3.074663in}}%
\pgfpathlineto{\pgfqpoint{2.008362in}{3.293827in}}%
\pgfpathlineto{\pgfqpoint{2.008840in}{3.130237in}}%
\pgfpathlineto{\pgfqpoint{2.009904in}{3.300888in}}%
\pgfpathlineto{\pgfqpoint{2.009518in}{3.071529in}}%
\pgfpathlineto{\pgfqpoint{2.010027in}{3.199587in}}%
\pgfpathlineto{\pgfqpoint{2.011075in}{3.060223in}}%
\pgfpathlineto{\pgfqpoint{2.010690in}{3.305558in}}%
\pgfpathlineto{\pgfqpoint{2.011152in}{3.130040in}}%
\pgfpathlineto{\pgfqpoint{2.011461in}{3.309795in}}%
\pgfpathlineto{\pgfqpoint{2.011846in}{3.061715in}}%
\pgfpathlineto{\pgfqpoint{2.012339in}{3.211193in}}%
\pgfpathlineto{\pgfqpoint{2.013403in}{3.058305in}}%
\pgfpathlineto{\pgfqpoint{2.013017in}{3.306314in}}%
\pgfpathlineto{\pgfqpoint{2.013480in}{3.131345in}}%
\pgfpathlineto{\pgfqpoint{2.013788in}{3.314139in}}%
\pgfpathlineto{\pgfqpoint{2.014189in}{3.059456in}}%
\pgfpathlineto{\pgfqpoint{2.014667in}{3.221051in}}%
\pgfpathlineto{\pgfqpoint{2.014975in}{3.066432in}}%
\pgfpathlineto{\pgfqpoint{2.015345in}{3.302865in}}%
\pgfpathlineto{\pgfqpoint{2.015808in}{3.148745in}}%
\pgfpathlineto{\pgfqpoint{2.016887in}{3.304887in}}%
\pgfpathlineto{\pgfqpoint{2.016532in}{3.062822in}}%
\pgfpathlineto{\pgfqpoint{2.016964in}{3.212171in}}%
\pgfpathlineto{\pgfqpoint{2.016979in}{3.213234in}}%
\pgfpathlineto{\pgfqpoint{2.017010in}{3.161954in}}%
\pgfpathlineto{\pgfqpoint{2.017303in}{3.061752in}}%
\pgfpathlineto{\pgfqpoint{2.017657in}{3.302990in}}%
\pgfpathlineto{\pgfqpoint{2.018120in}{3.138424in}}%
\pgfpathlineto{\pgfqpoint{2.019214in}{3.297279in}}%
\pgfpathlineto{\pgfqpoint{2.018860in}{3.062662in}}%
\pgfpathlineto{\pgfqpoint{2.019276in}{3.224185in}}%
\pgfpathlineto{\pgfqpoint{2.020401in}{3.059942in}}%
\pgfpathlineto{\pgfqpoint{2.019985in}{3.293836in}}%
\pgfpathlineto{\pgfqpoint{2.020447in}{3.143386in}}%
\pgfpathlineto{\pgfqpoint{2.020756in}{3.291910in}}%
\pgfpathlineto{\pgfqpoint{2.021172in}{3.058171in}}%
\pgfpathlineto{\pgfqpoint{2.021604in}{3.219547in}}%
\pgfpathlineto{\pgfqpoint{2.022729in}{3.051019in}}%
\pgfpathlineto{\pgfqpoint{2.022313in}{3.284562in}}%
\pgfpathlineto{\pgfqpoint{2.022775in}{3.150725in}}%
\pgfpathlineto{\pgfqpoint{2.023715in}{3.289803in}}%
\pgfpathlineto{\pgfqpoint{2.023500in}{3.051107in}}%
\pgfpathlineto{\pgfqpoint{2.023916in}{3.213943in}}%
\pgfpathlineto{\pgfqpoint{2.025056in}{3.044428in}}%
\pgfpathlineto{\pgfqpoint{2.024486in}{3.294253in}}%
\pgfpathlineto{\pgfqpoint{2.025103in}{3.153539in}}%
\pgfpathlineto{\pgfqpoint{2.026043in}{3.294808in}}%
\pgfpathlineto{\pgfqpoint{2.025827in}{3.041865in}}%
\pgfpathlineto{\pgfqpoint{2.026228in}{3.224269in}}%
\pgfpathlineto{\pgfqpoint{2.026598in}{3.048083in}}%
\pgfpathlineto{\pgfqpoint{2.026952in}{3.303063in}}%
\pgfpathlineto{\pgfqpoint{2.027400in}{3.081545in}}%
\pgfpathlineto{\pgfqpoint{2.028509in}{3.307339in}}%
\pgfpathlineto{\pgfqpoint{2.028155in}{3.049222in}}%
\pgfpathlineto{\pgfqpoint{2.028556in}{3.228260in}}%
\pgfpathlineto{\pgfqpoint{2.029712in}{3.052882in}}%
\pgfpathlineto{\pgfqpoint{2.029280in}{3.318116in}}%
\pgfpathlineto{\pgfqpoint{2.029727in}{3.087650in}}%
\pgfpathlineto{\pgfqpoint{2.030837in}{3.329272in}}%
\pgfpathlineto{\pgfqpoint{2.030482in}{3.053410in}}%
\pgfpathlineto{\pgfqpoint{2.030883in}{3.240894in}}%
\pgfpathlineto{\pgfqpoint{2.032024in}{3.043482in}}%
\pgfpathlineto{\pgfqpoint{2.031608in}{3.333992in}}%
\pgfpathlineto{\pgfqpoint{2.032055in}{3.088821in}}%
\pgfpathlineto{\pgfqpoint{2.032379in}{3.327145in}}%
\pgfpathlineto{\pgfqpoint{2.032810in}{3.046060in}}%
\pgfpathlineto{\pgfqpoint{2.033211in}{3.245883in}}%
\pgfpathlineto{\pgfqpoint{2.034182in}{3.037689in}}%
\pgfpathlineto{\pgfqpoint{2.033935in}{3.327192in}}%
\pgfpathlineto{\pgfqpoint{2.034382in}{3.096115in}}%
\pgfpathlineto{\pgfqpoint{2.035477in}{3.330364in}}%
\pgfpathlineto{\pgfqpoint{2.034953in}{3.033185in}}%
\pgfpathlineto{\pgfqpoint{2.035539in}{3.247457in}}%
\pgfpathlineto{\pgfqpoint{2.036510in}{3.024358in}}%
\pgfpathlineto{\pgfqpoint{2.036263in}{3.330337in}}%
\pgfpathlineto{\pgfqpoint{2.036710in}{3.102719in}}%
\pgfpathlineto{\pgfqpoint{2.037805in}{3.341748in}}%
\pgfpathlineto{\pgfqpoint{2.037280in}{3.020307in}}%
\pgfpathlineto{\pgfqpoint{2.037851in}{3.280178in}}%
\pgfpathlineto{\pgfqpoint{2.038837in}{3.017930in}}%
\pgfpathlineto{\pgfqpoint{2.038575in}{3.346765in}}%
\pgfpathlineto{\pgfqpoint{2.039053in}{3.153662in}}%
\pgfpathlineto{\pgfqpoint{2.040132in}{3.353613in}}%
\pgfpathlineto{\pgfqpoint{2.039608in}{3.013859in}}%
\pgfpathlineto{\pgfqpoint{2.040194in}{3.268474in}}%
\pgfpathlineto{\pgfqpoint{2.041319in}{3.011003in}}%
\pgfpathlineto{\pgfqpoint{2.040903in}{3.358075in}}%
\pgfpathlineto{\pgfqpoint{2.041381in}{3.157542in}}%
\pgfpathlineto{\pgfqpoint{2.042460in}{3.363656in}}%
\pgfpathlineto{\pgfqpoint{2.042090in}{3.008990in}}%
\pgfpathlineto{\pgfqpoint{2.042521in}{3.266785in}}%
\pgfpathlineto{\pgfqpoint{2.043647in}{2.996789in}}%
\pgfpathlineto{\pgfqpoint{2.043231in}{3.376020in}}%
\pgfpathlineto{\pgfqpoint{2.043708in}{3.156565in}}%
\pgfpathlineto{\pgfqpoint{2.044772in}{3.379986in}}%
\pgfpathlineto{\pgfqpoint{2.044417in}{2.985625in}}%
\pgfpathlineto{\pgfqpoint{2.044849in}{3.269345in}}%
\pgfpathlineto{\pgfqpoint{2.045188in}{2.985837in}}%
\pgfpathlineto{\pgfqpoint{2.045558in}{3.377546in}}%
\pgfpathlineto{\pgfqpoint{2.046036in}{3.151634in}}%
\pgfpathlineto{\pgfqpoint{2.047100in}{3.389916in}}%
\pgfpathlineto{\pgfqpoint{2.046745in}{2.983577in}}%
\pgfpathlineto{\pgfqpoint{2.047192in}{3.255768in}}%
\pgfpathlineto{\pgfqpoint{2.047516in}{2.982235in}}%
\pgfpathlineto{\pgfqpoint{2.047870in}{3.386820in}}%
\pgfpathlineto{\pgfqpoint{2.048333in}{3.072769in}}%
\pgfpathlineto{\pgfqpoint{2.049427in}{3.393179in}}%
\pgfpathlineto{\pgfqpoint{2.049057in}{2.987802in}}%
\pgfpathlineto{\pgfqpoint{2.049504in}{3.287484in}}%
\pgfpathlineto{\pgfqpoint{2.050614in}{2.974050in}}%
\pgfpathlineto{\pgfqpoint{2.050198in}{3.392923in}}%
\pgfpathlineto{\pgfqpoint{2.050676in}{3.140104in}}%
\pgfpathlineto{\pgfqpoint{2.051755in}{3.398702in}}%
\pgfpathlineto{\pgfqpoint{2.051385in}{2.967220in}}%
\pgfpathlineto{\pgfqpoint{2.051832in}{3.291118in}}%
\pgfpathlineto{\pgfqpoint{2.052942in}{2.957942in}}%
\pgfpathlineto{\pgfqpoint{2.052526in}{3.401357in}}%
\pgfpathlineto{\pgfqpoint{2.052988in}{3.087460in}}%
\pgfpathlineto{\pgfqpoint{2.054083in}{3.404904in}}%
\pgfpathlineto{\pgfqpoint{2.053713in}{2.951475in}}%
\pgfpathlineto{\pgfqpoint{2.054144in}{3.321209in}}%
\pgfpathlineto{\pgfqpoint{2.055254in}{2.950430in}}%
\pgfpathlineto{\pgfqpoint{2.054853in}{3.401068in}}%
\pgfpathlineto{\pgfqpoint{2.055362in}{3.183941in}}%
\pgfpathlineto{\pgfqpoint{2.056410in}{3.397110in}}%
\pgfpathlineto{\pgfqpoint{2.056025in}{2.949791in}}%
\pgfpathlineto{\pgfqpoint{2.056503in}{3.250775in}}%
\pgfpathlineto{\pgfqpoint{2.057582in}{2.930796in}}%
\pgfpathlineto{\pgfqpoint{2.057181in}{3.398523in}}%
\pgfpathlineto{\pgfqpoint{2.057643in}{3.103216in}}%
\pgfpathlineto{\pgfqpoint{2.057952in}{3.395209in}}%
\pgfpathlineto{\pgfqpoint{2.058352in}{2.928592in}}%
\pgfpathlineto{\pgfqpoint{2.058800in}{3.311936in}}%
\pgfpathlineto{\pgfqpoint{2.059909in}{2.920302in}}%
\pgfpathlineto{\pgfqpoint{2.059509in}{3.399682in}}%
\pgfpathlineto{\pgfqpoint{2.059986in}{3.157819in}}%
\pgfpathlineto{\pgfqpoint{2.061050in}{3.406990in}}%
\pgfpathlineto{\pgfqpoint{2.060680in}{2.918598in}}%
\pgfpathlineto{\pgfqpoint{2.061158in}{3.239160in}}%
\pgfpathlineto{\pgfqpoint{2.062237in}{2.920587in}}%
\pgfpathlineto{\pgfqpoint{2.061836in}{3.408827in}}%
\pgfpathlineto{\pgfqpoint{2.062299in}{3.128751in}}%
\pgfpathlineto{\pgfqpoint{2.062607in}{3.408844in}}%
\pgfpathlineto{\pgfqpoint{2.063008in}{2.916817in}}%
\pgfpathlineto{\pgfqpoint{2.063470in}{3.292061in}}%
\pgfpathlineto{\pgfqpoint{2.063779in}{2.923383in}}%
\pgfpathlineto{\pgfqpoint{2.064148in}{3.397631in}}%
\pgfpathlineto{\pgfqpoint{2.064626in}{3.137789in}}%
\pgfpathlineto{\pgfqpoint{2.065551in}{3.408845in}}%
\pgfpathlineto{\pgfqpoint{2.065335in}{2.925118in}}%
\pgfpathlineto{\pgfqpoint{2.065798in}{3.295176in}}%
\pgfpathlineto{\pgfqpoint{2.066892in}{2.927450in}}%
\pgfpathlineto{\pgfqpoint{2.066322in}{3.404972in}}%
\pgfpathlineto{\pgfqpoint{2.066954in}{3.147346in}}%
\pgfpathlineto{\pgfqpoint{2.067879in}{3.406537in}}%
\pgfpathlineto{\pgfqpoint{2.067663in}{2.925797in}}%
\pgfpathlineto{\pgfqpoint{2.068110in}{3.315372in}}%
\pgfpathlineto{\pgfqpoint{2.069220in}{2.926599in}}%
\pgfpathlineto{\pgfqpoint{2.068650in}{3.417972in}}%
\pgfpathlineto{\pgfqpoint{2.069266in}{3.101803in}}%
\pgfpathlineto{\pgfqpoint{2.069420in}{3.423040in}}%
\pgfpathlineto{\pgfqpoint{2.069991in}{2.919778in}}%
\pgfpathlineto{\pgfqpoint{2.070422in}{3.337409in}}%
\pgfpathlineto{\pgfqpoint{2.070761in}{2.926766in}}%
\pgfpathlineto{\pgfqpoint{2.070962in}{3.418498in}}%
\pgfpathlineto{\pgfqpoint{2.071609in}{3.159686in}}%
\pgfpathlineto{\pgfqpoint{2.071748in}{3.418085in}}%
\pgfpathlineto{\pgfqpoint{2.072303in}{2.931681in}}%
\pgfpathlineto{\pgfqpoint{2.072765in}{3.298013in}}%
\pgfpathlineto{\pgfqpoint{2.073860in}{2.930692in}}%
\pgfpathlineto{\pgfqpoint{2.073289in}{3.419166in}}%
\pgfpathlineto{\pgfqpoint{2.073906in}{3.098585in}}%
\pgfpathlineto{\pgfqpoint{2.074060in}{3.417793in}}%
\pgfpathlineto{\pgfqpoint{2.074631in}{2.935235in}}%
\pgfpathlineto{\pgfqpoint{2.075078in}{3.325901in}}%
\pgfpathlineto{\pgfqpoint{2.076172in}{2.932457in}}%
\pgfpathlineto{\pgfqpoint{2.075602in}{3.406467in}}%
\pgfpathlineto{\pgfqpoint{2.076234in}{3.135739in}}%
\pgfpathlineto{\pgfqpoint{2.077159in}{3.399785in}}%
\pgfpathlineto{\pgfqpoint{2.076958in}{2.936218in}}%
\pgfpathlineto{\pgfqpoint{2.077405in}{3.302663in}}%
\pgfpathlineto{\pgfqpoint{2.078346in}{2.923215in}}%
\pgfpathlineto{\pgfqpoint{2.077929in}{3.398536in}}%
\pgfpathlineto{\pgfqpoint{2.078546in}{3.102594in}}%
\pgfpathlineto{\pgfqpoint{2.078700in}{3.388502in}}%
\pgfpathlineto{\pgfqpoint{2.079116in}{2.920126in}}%
\pgfpathlineto{\pgfqpoint{2.079702in}{3.356845in}}%
\pgfpathlineto{\pgfqpoint{2.080673in}{2.907703in}}%
\pgfpathlineto{\pgfqpoint{2.080257in}{3.384442in}}%
\pgfpathlineto{\pgfqpoint{2.080904in}{3.178955in}}%
\pgfpathlineto{\pgfqpoint{2.081028in}{3.383718in}}%
\pgfpathlineto{\pgfqpoint{2.081444in}{2.903395in}}%
\pgfpathlineto{\pgfqpoint{2.082061in}{3.258273in}}%
\pgfpathlineto{\pgfqpoint{2.083001in}{2.896228in}}%
\pgfpathlineto{\pgfqpoint{2.082585in}{3.377970in}}%
\pgfpathlineto{\pgfqpoint{2.083201in}{3.121376in}}%
\pgfpathlineto{\pgfqpoint{2.083355in}{3.392737in}}%
\pgfpathlineto{\pgfqpoint{2.083772in}{2.889434in}}%
\pgfpathlineto{\pgfqpoint{2.084373in}{3.290329in}}%
\pgfpathlineto{\pgfqpoint{2.085313in}{2.894858in}}%
\pgfpathlineto{\pgfqpoint{2.084897in}{3.395113in}}%
\pgfpathlineto{\pgfqpoint{2.085513in}{3.083498in}}%
\pgfpathlineto{\pgfqpoint{2.086454in}{3.399132in}}%
\pgfpathlineto{\pgfqpoint{2.086099in}{2.892480in}}%
\pgfpathlineto{\pgfqpoint{2.086685in}{3.319475in}}%
\pgfpathlineto{\pgfqpoint{2.087641in}{2.885228in}}%
\pgfpathlineto{\pgfqpoint{2.087224in}{3.399148in}}%
\pgfpathlineto{\pgfqpoint{2.087841in}{3.094276in}}%
\pgfpathlineto{\pgfqpoint{2.088781in}{3.404209in}}%
\pgfpathlineto{\pgfqpoint{2.088411in}{2.891023in}}%
\pgfpathlineto{\pgfqpoint{2.089013in}{3.301520in}}%
\pgfpathlineto{\pgfqpoint{2.089968in}{2.886835in}}%
\pgfpathlineto{\pgfqpoint{2.089722in}{3.410600in}}%
\pgfpathlineto{\pgfqpoint{2.090169in}{3.108693in}}%
\pgfpathlineto{\pgfqpoint{2.091263in}{3.413569in}}%
\pgfpathlineto{\pgfqpoint{2.090739in}{2.888579in}}%
\pgfpathlineto{\pgfqpoint{2.091325in}{3.344464in}}%
\pgfpathlineto{\pgfqpoint{2.091510in}{2.885676in}}%
\pgfpathlineto{\pgfqpoint{2.092049in}{3.417113in}}%
\pgfpathlineto{\pgfqpoint{2.092481in}{3.057543in}}%
\pgfpathlineto{\pgfqpoint{2.092820in}{3.426982in}}%
\pgfpathlineto{\pgfqpoint{2.093051in}{2.891443in}}%
\pgfpathlineto{\pgfqpoint{2.093637in}{3.357422in}}%
\pgfpathlineto{\pgfqpoint{2.094608in}{2.886388in}}%
\pgfpathlineto{\pgfqpoint{2.094362in}{3.432599in}}%
\pgfpathlineto{\pgfqpoint{2.094824in}{3.150264in}}%
\pgfpathlineto{\pgfqpoint{2.095903in}{3.433435in}}%
\pgfpathlineto{\pgfqpoint{2.095379in}{2.879776in}}%
\pgfpathlineto{\pgfqpoint{2.095980in}{3.325559in}}%
\pgfpathlineto{\pgfqpoint{2.096920in}{2.877328in}}%
\pgfpathlineto{\pgfqpoint{2.096689in}{3.437089in}}%
\pgfpathlineto{\pgfqpoint{2.097136in}{3.104549in}}%
\pgfpathlineto{\pgfqpoint{2.097460in}{3.445450in}}%
\pgfpathlineto{\pgfqpoint{2.097691in}{2.879252in}}%
\pgfpathlineto{\pgfqpoint{2.098292in}{3.355041in}}%
\pgfpathlineto{\pgfqpoint{2.099248in}{2.871803in}}%
\pgfpathlineto{\pgfqpoint{2.099001in}{3.441634in}}%
\pgfpathlineto{\pgfqpoint{2.099479in}{3.193939in}}%
\pgfpathlineto{\pgfqpoint{2.100558in}{3.443590in}}%
\pgfpathlineto{\pgfqpoint{2.100019in}{2.873543in}}%
\pgfpathlineto{\pgfqpoint{2.100635in}{3.301457in}}%
\pgfpathlineto{\pgfqpoint{2.101576in}{2.871954in}}%
\pgfpathlineto{\pgfqpoint{2.101329in}{3.448636in}}%
\pgfpathlineto{\pgfqpoint{2.101776in}{3.079218in}}%
\pgfpathlineto{\pgfqpoint{2.102716in}{3.441142in}}%
\pgfpathlineto{\pgfqpoint{2.102346in}{2.872032in}}%
\pgfpathlineto{\pgfqpoint{2.102932in}{3.371506in}}%
\pgfpathlineto{\pgfqpoint{2.103888in}{2.866547in}}%
\pgfpathlineto{\pgfqpoint{2.103657in}{3.443184in}}%
\pgfpathlineto{\pgfqpoint{2.104135in}{3.208601in}}%
\pgfpathlineto{\pgfqpoint{2.105214in}{3.447773in}}%
\pgfpathlineto{\pgfqpoint{2.104659in}{2.867927in}}%
\pgfpathlineto{\pgfqpoint{2.105291in}{3.260809in}}%
\pgfpathlineto{\pgfqpoint{2.105429in}{2.871656in}}%
\pgfpathlineto{\pgfqpoint{2.105984in}{3.454486in}}%
\pgfpathlineto{\pgfqpoint{2.106431in}{3.121174in}}%
\pgfpathlineto{\pgfqpoint{2.107526in}{3.453355in}}%
\pgfpathlineto{\pgfqpoint{2.106986in}{2.868044in}}%
\pgfpathlineto{\pgfqpoint{2.107587in}{3.335532in}}%
\pgfpathlineto{\pgfqpoint{2.107757in}{2.866841in}}%
\pgfpathlineto{\pgfqpoint{2.108312in}{3.450823in}}%
\pgfpathlineto{\pgfqpoint{2.108759in}{3.133658in}}%
\pgfpathlineto{\pgfqpoint{2.109853in}{3.453301in}}%
\pgfpathlineto{\pgfqpoint{2.109314in}{2.865600in}}%
\pgfpathlineto{\pgfqpoint{2.109915in}{3.327928in}}%
\pgfpathlineto{\pgfqpoint{2.110085in}{2.867121in}}%
\pgfpathlineto{\pgfqpoint{2.110640in}{3.444485in}}%
\pgfpathlineto{\pgfqpoint{2.111102in}{3.201479in}}%
\pgfpathlineto{\pgfqpoint{2.111410in}{3.443765in}}%
\pgfpathlineto{\pgfqpoint{2.111642in}{2.866745in}}%
\pgfpathlineto{\pgfqpoint{2.112258in}{3.299043in}}%
\pgfpathlineto{\pgfqpoint{2.112412in}{2.864457in}}%
\pgfpathlineto{\pgfqpoint{2.112952in}{3.432423in}}%
\pgfpathlineto{\pgfqpoint{2.113399in}{3.119391in}}%
\pgfpathlineto{\pgfqpoint{2.113738in}{3.438395in}}%
\pgfpathlineto{\pgfqpoint{2.113969in}{2.866878in}}%
\pgfpathlineto{\pgfqpoint{2.114570in}{3.328510in}}%
\pgfpathlineto{\pgfqpoint{2.115511in}{2.857177in}}%
\pgfpathlineto{\pgfqpoint{2.115279in}{3.425168in}}%
\pgfpathlineto{\pgfqpoint{2.115726in}{3.122371in}}%
\pgfpathlineto{\pgfqpoint{2.116066in}{3.431631in}}%
\pgfpathlineto{\pgfqpoint{2.116281in}{2.861180in}}%
\pgfpathlineto{\pgfqpoint{2.116883in}{3.338870in}}%
\pgfpathlineto{\pgfqpoint{2.117838in}{2.857464in}}%
\pgfpathlineto{\pgfqpoint{2.117607in}{3.428963in}}%
\pgfpathlineto{\pgfqpoint{2.118054in}{3.138221in}}%
\pgfpathlineto{\pgfqpoint{2.119164in}{3.428261in}}%
\pgfpathlineto{\pgfqpoint{2.118609in}{2.855983in}}%
\pgfpathlineto{\pgfqpoint{2.119210in}{3.332601in}}%
\pgfpathlineto{\pgfqpoint{2.119380in}{2.852958in}}%
\pgfpathlineto{\pgfqpoint{2.119935in}{3.433234in}}%
\pgfpathlineto{\pgfqpoint{2.120382in}{3.140998in}}%
\pgfpathlineto{\pgfqpoint{2.121492in}{3.435927in}}%
\pgfpathlineto{\pgfqpoint{2.120937in}{2.853272in}}%
\pgfpathlineto{\pgfqpoint{2.121538in}{3.327620in}}%
\pgfpathlineto{\pgfqpoint{2.121707in}{2.859213in}}%
\pgfpathlineto{\pgfqpoint{2.122262in}{3.445844in}}%
\pgfpathlineto{\pgfqpoint{2.122694in}{3.103787in}}%
\pgfpathlineto{\pgfqpoint{2.123819in}{3.447342in}}%
\pgfpathlineto{\pgfqpoint{2.123264in}{2.865640in}}%
\pgfpathlineto{\pgfqpoint{2.123850in}{3.366977in}}%
\pgfpathlineto{\pgfqpoint{2.124035in}{2.862479in}}%
\pgfpathlineto{\pgfqpoint{2.124590in}{3.446996in}}%
\pgfpathlineto{\pgfqpoint{2.125022in}{3.115373in}}%
\pgfpathlineto{\pgfqpoint{2.126132in}{3.452784in}}%
\pgfpathlineto{\pgfqpoint{2.125577in}{2.866098in}}%
\pgfpathlineto{\pgfqpoint{2.126178in}{3.362201in}}%
\pgfpathlineto{\pgfqpoint{2.126363in}{2.867937in}}%
\pgfpathlineto{\pgfqpoint{2.126918in}{3.454615in}}%
\pgfpathlineto{\pgfqpoint{2.127349in}{3.129452in}}%
\pgfpathlineto{\pgfqpoint{2.127688in}{3.458050in}}%
\pgfpathlineto{\pgfqpoint{2.127904in}{2.870741in}}%
\pgfpathlineto{\pgfqpoint{2.128505in}{3.376775in}}%
\pgfpathlineto{\pgfqpoint{2.128690in}{2.871772in}}%
\pgfpathlineto{\pgfqpoint{2.129245in}{3.440258in}}%
\pgfpathlineto{\pgfqpoint{2.129677in}{3.142898in}}%
\pgfpathlineto{\pgfqpoint{2.130787in}{3.430790in}}%
\pgfpathlineto{\pgfqpoint{2.130386in}{2.903945in}}%
\pgfpathlineto{\pgfqpoint{2.130833in}{3.354818in}}%
\pgfpathlineto{\pgfqpoint{2.131943in}{2.907374in}}%
\pgfpathlineto{\pgfqpoint{2.131573in}{3.449358in}}%
\pgfpathlineto{\pgfqpoint{2.132005in}{3.098177in}}%
\pgfpathlineto{\pgfqpoint{2.133114in}{3.463387in}}%
\pgfpathlineto{\pgfqpoint{2.132559in}{2.904989in}}%
\pgfpathlineto{\pgfqpoint{2.133161in}{3.300222in}}%
\pgfpathlineto{\pgfqpoint{2.133330in}{2.913924in}}%
\pgfpathlineto{\pgfqpoint{2.133885in}{3.464387in}}%
\pgfpathlineto{\pgfqpoint{2.134332in}{3.102774in}}%
\pgfpathlineto{\pgfqpoint{2.134671in}{3.481887in}}%
\pgfpathlineto{\pgfqpoint{2.134872in}{2.921008in}}%
\pgfpathlineto{\pgfqpoint{2.135488in}{3.387542in}}%
\pgfpathlineto{\pgfqpoint{2.136598in}{2.898276in}}%
\pgfpathlineto{\pgfqpoint{2.136228in}{3.400525in}}%
\pgfpathlineto{\pgfqpoint{2.136644in}{3.082889in}}%
\pgfpathlineto{\pgfqpoint{2.137785in}{3.397782in}}%
\pgfpathlineto{\pgfqpoint{2.137369in}{2.913412in}}%
\pgfpathlineto{\pgfqpoint{2.137801in}{3.389866in}}%
\pgfpathlineto{\pgfqpoint{2.138001in}{2.906606in}}%
\pgfpathlineto{\pgfqpoint{2.139003in}{3.192940in}}%
\pgfpathlineto{\pgfqpoint{2.139327in}{3.457757in}}%
\pgfpathlineto{\pgfqpoint{2.139527in}{2.944283in}}%
\pgfpathlineto{\pgfqpoint{2.140144in}{3.250457in}}%
\pgfpathlineto{\pgfqpoint{2.140298in}{2.952308in}}%
\pgfpathlineto{\pgfqpoint{2.140714in}{3.331434in}}%
\pgfpathlineto{\pgfqpoint{2.140822in}{3.327513in}}%
\pgfpathlineto{\pgfqpoint{2.140868in}{3.456753in}}%
\pgfpathlineto{\pgfqpoint{2.141253in}{2.948831in}}%
\pgfpathlineto{\pgfqpoint{2.141839in}{2.991805in}}%
\pgfpathlineto{\pgfqpoint{2.142394in}{3.392116in}}%
\pgfpathlineto{\pgfqpoint{2.142055in}{2.989891in}}%
\pgfpathlineto{\pgfqpoint{2.143042in}{3.261040in}}%
\pgfpathlineto{\pgfqpoint{2.143350in}{2.937220in}}%
\pgfpathlineto{\pgfqpoint{2.143781in}{3.362761in}}%
\pgfpathlineto{\pgfqpoint{2.144244in}{3.010909in}}%
\pgfpathlineto{\pgfqpoint{2.144568in}{3.320390in}}%
\pgfpathlineto{\pgfqpoint{2.144891in}{2.959358in}}%
\pgfpathlineto{\pgfqpoint{2.145400in}{3.172827in}}%
\pgfpathlineto{\pgfqpoint{2.145554in}{3.349684in}}%
\pgfpathlineto{\pgfqpoint{2.145755in}{2.942694in}}%
\pgfpathlineto{\pgfqpoint{2.146433in}{3.073675in}}%
\pgfpathlineto{\pgfqpoint{2.146525in}{3.020898in}}%
\pgfpathlineto{\pgfqpoint{2.147096in}{3.330343in}}%
\pgfpathlineto{\pgfqpoint{2.147373in}{3.176308in}}%
\pgfpathlineto{\pgfqpoint{2.147635in}{3.325041in}}%
\pgfpathlineto{\pgfqpoint{2.147450in}{3.004761in}}%
\pgfpathlineto{\pgfqpoint{2.148206in}{3.080456in}}%
\pgfpathlineto{\pgfqpoint{2.148838in}{3.047666in}}%
\pgfpathlineto{\pgfqpoint{2.148653in}{3.284599in}}%
\pgfpathlineto{\pgfqpoint{2.149254in}{3.183274in}}%
\pgfpathlineto{\pgfqpoint{2.149408in}{3.294526in}}%
\pgfpathlineto{\pgfqpoint{2.149608in}{3.029825in}}%
\pgfpathlineto{\pgfqpoint{2.150302in}{3.082055in}}%
\pgfpathlineto{\pgfqpoint{2.150379in}{3.009479in}}%
\pgfpathlineto{\pgfqpoint{2.150718in}{3.268070in}}%
\pgfpathlineto{\pgfqpoint{2.151335in}{3.137593in}}%
\pgfpathlineto{\pgfqpoint{2.152260in}{3.307170in}}%
\pgfpathlineto{\pgfqpoint{2.151905in}{3.013085in}}%
\pgfpathlineto{\pgfqpoint{2.152522in}{3.244711in}}%
\pgfpathlineto{\pgfqpoint{2.152691in}{2.997307in}}%
\pgfpathlineto{\pgfqpoint{2.153030in}{3.305701in}}%
\pgfpathlineto{\pgfqpoint{2.153647in}{3.174264in}}%
\pgfpathlineto{\pgfqpoint{2.154587in}{3.302920in}}%
\pgfpathlineto{\pgfqpoint{2.154202in}{2.996096in}}%
\pgfpathlineto{\pgfqpoint{2.154757in}{3.180967in}}%
\pgfpathlineto{\pgfqpoint{2.154772in}{3.180619in}}%
\pgfpathlineto{\pgfqpoint{2.154788in}{3.198824in}}%
\pgfpathlineto{\pgfqpoint{2.155343in}{3.272709in}}%
\pgfpathlineto{\pgfqpoint{2.155774in}{2.968442in}}%
\pgfpathlineto{\pgfqpoint{2.155851in}{3.098527in}}%
\pgfpathlineto{\pgfqpoint{2.156252in}{3.360827in}}%
\pgfpathlineto{\pgfqpoint{2.156637in}{2.960904in}}%
\pgfpathlineto{\pgfqpoint{2.157131in}{3.285436in}}%
\pgfpathlineto{\pgfqpoint{2.158071in}{2.919427in}}%
\pgfpathlineto{\pgfqpoint{2.157747in}{3.321205in}}%
\pgfpathlineto{\pgfqpoint{2.158379in}{3.060263in}}%
\pgfpathlineto{\pgfqpoint{2.158718in}{3.322981in}}%
\pgfpathlineto{\pgfqpoint{2.159243in}{3.035094in}}%
\pgfpathlineto{\pgfqpoint{2.159289in}{2.978486in}}%
\pgfpathlineto{\pgfqpoint{2.160013in}{3.386410in}}%
\pgfpathlineto{\pgfqpoint{2.160306in}{3.130805in}}%
\pgfpathlineto{\pgfqpoint{2.161324in}{3.405664in}}%
\pgfpathlineto{\pgfqpoint{2.160414in}{2.954378in}}%
\pgfpathlineto{\pgfqpoint{2.161555in}{3.244486in}}%
\pgfpathlineto{\pgfqpoint{2.162788in}{2.819867in}}%
\pgfpathlineto{\pgfqpoint{2.162372in}{3.532122in}}%
\pgfpathlineto{\pgfqpoint{2.162803in}{2.826013in}}%
\pgfpathlineto{\pgfqpoint{2.163913in}{3.398626in}}%
\pgfpathlineto{\pgfqpoint{2.164067in}{3.221806in}}%
\pgfpathlineto{\pgfqpoint{2.164283in}{2.810113in}}%
\pgfpathlineto{\pgfqpoint{2.164807in}{3.554598in}}%
\pgfpathlineto{\pgfqpoint{2.165331in}{3.002500in}}%
\pgfpathlineto{\pgfqpoint{2.166534in}{3.311336in}}%
\pgfpathlineto{\pgfqpoint{2.165408in}{3.001965in}}%
\pgfpathlineto{\pgfqpoint{2.166549in}{3.287546in}}%
\pgfpathlineto{\pgfqpoint{2.166611in}{3.023615in}}%
\pgfpathlineto{\pgfqpoint{2.167197in}{3.355866in}}%
\pgfpathlineto{\pgfqpoint{2.167674in}{3.180189in}}%
\pgfpathlineto{\pgfqpoint{2.168384in}{3.414344in}}%
\pgfpathlineto{\pgfqpoint{2.167998in}{3.004997in}}%
\pgfpathlineto{\pgfqpoint{2.168723in}{3.068620in}}%
\pgfpathlineto{\pgfqpoint{2.168985in}{2.898427in}}%
\pgfpathlineto{\pgfqpoint{2.169386in}{3.444850in}}%
\pgfpathlineto{\pgfqpoint{2.169879in}{2.969255in}}%
\pgfpathlineto{\pgfqpoint{2.170896in}{3.474119in}}%
\pgfpathlineto{\pgfqpoint{2.171035in}{3.171716in}}%
\pgfpathlineto{\pgfqpoint{2.171297in}{2.850633in}}%
\pgfpathlineto{\pgfqpoint{2.171698in}{3.469557in}}%
\pgfpathlineto{\pgfqpoint{2.171775in}{3.587018in}}%
\pgfpathlineto{\pgfqpoint{2.172237in}{2.872089in}}%
\pgfpathlineto{\pgfqpoint{2.172684in}{3.273325in}}%
\pgfpathlineto{\pgfqpoint{2.173363in}{3.007694in}}%
\pgfpathlineto{\pgfqpoint{2.173070in}{3.292681in}}%
\pgfpathlineto{\pgfqpoint{2.173871in}{3.098515in}}%
\pgfpathlineto{\pgfqpoint{2.174180in}{3.360399in}}%
\pgfpathlineto{\pgfqpoint{2.174519in}{3.014372in}}%
\pgfpathlineto{\pgfqpoint{2.174950in}{2.990291in}}%
\pgfpathlineto{\pgfqpoint{2.174704in}{3.187994in}}%
\pgfpathlineto{\pgfqpoint{2.175135in}{3.148609in}}%
\pgfpathlineto{\pgfqpoint{2.175382in}{3.507946in}}%
\pgfpathlineto{\pgfqpoint{2.175767in}{2.860511in}}%
\pgfpathlineto{\pgfqpoint{2.176338in}{3.454917in}}%
\pgfpathlineto{\pgfqpoint{2.176862in}{2.887974in}}%
\pgfpathlineto{\pgfqpoint{2.177632in}{3.149303in}}%
\pgfpathlineto{\pgfqpoint{2.178758in}{3.566582in}}%
\pgfpathlineto{\pgfqpoint{2.178280in}{2.822278in}}%
\pgfpathlineto{\pgfqpoint{2.178819in}{3.517528in}}%
\pgfpathlineto{\pgfqpoint{2.178850in}{3.532578in}}%
\pgfpathlineto{\pgfqpoint{2.179205in}{2.860060in}}%
\pgfpathlineto{\pgfqpoint{2.179236in}{2.802458in}}%
\pgfpathlineto{\pgfqpoint{2.180053in}{3.379396in}}%
\pgfpathlineto{\pgfqpoint{2.180191in}{3.160515in}}%
\pgfpathlineto{\pgfqpoint{2.181178in}{3.423370in}}%
\pgfpathlineto{\pgfqpoint{2.180376in}{3.006696in}}%
\pgfpathlineto{\pgfqpoint{2.181409in}{3.295393in}}%
\pgfpathlineto{\pgfqpoint{2.181995in}{2.922919in}}%
\pgfpathlineto{\pgfqpoint{2.182334in}{3.486074in}}%
\pgfpathlineto{\pgfqpoint{2.182365in}{3.593399in}}%
\pgfpathlineto{\pgfqpoint{2.182827in}{2.809037in}}%
\pgfpathlineto{\pgfqpoint{2.183413in}{3.359035in}}%
\pgfpathlineto{\pgfqpoint{2.183428in}{3.369198in}}%
\pgfpathlineto{\pgfqpoint{2.183845in}{2.950598in}}%
\pgfpathlineto{\pgfqpoint{2.184384in}{2.872603in}}%
\pgfpathlineto{\pgfqpoint{2.184692in}{3.422569in}}%
\pgfpathlineto{\pgfqpoint{2.184770in}{3.407934in}}%
\pgfpathlineto{\pgfqpoint{2.185756in}{3.540462in}}%
\pgfpathlineto{\pgfqpoint{2.185278in}{2.770084in}}%
\pgfpathlineto{\pgfqpoint{2.185879in}{3.462991in}}%
\pgfpathlineto{\pgfqpoint{2.186219in}{2.922521in}}%
\pgfpathlineto{\pgfqpoint{2.187035in}{3.344596in}}%
\pgfpathlineto{\pgfqpoint{2.187051in}{3.374455in}}%
\pgfpathlineto{\pgfqpoint{2.187390in}{2.941541in}}%
\pgfpathlineto{\pgfqpoint{2.188068in}{3.207651in}}%
\pgfpathlineto{\pgfqpoint{2.188192in}{3.332352in}}%
\pgfpathlineto{\pgfqpoint{2.188500in}{3.172226in}}%
\pgfpathlineto{\pgfqpoint{2.188978in}{2.858289in}}%
\pgfpathlineto{\pgfqpoint{2.189394in}{3.585886in}}%
\pgfpathlineto{\pgfqpoint{2.189594in}{3.229083in}}%
\pgfpathlineto{\pgfqpoint{2.189841in}{2.792461in}}%
\pgfpathlineto{\pgfqpoint{2.190303in}{3.466602in}}%
\pgfpathlineto{\pgfqpoint{2.190781in}{3.045157in}}%
\pgfpathlineto{\pgfqpoint{2.190812in}{3.053472in}}%
\pgfpathlineto{\pgfqpoint{2.190843in}{2.988650in}}%
\pgfpathlineto{\pgfqpoint{2.191413in}{2.883421in}}%
\pgfpathlineto{\pgfqpoint{2.191706in}{3.450344in}}%
\pgfpathlineto{\pgfqpoint{2.191814in}{3.408683in}}%
\pgfpathlineto{\pgfqpoint{2.192770in}{3.545508in}}%
\pgfpathlineto{\pgfqpoint{2.192307in}{2.838615in}}%
\pgfpathlineto{\pgfqpoint{2.192909in}{3.414999in}}%
\pgfpathlineto{\pgfqpoint{2.193186in}{2.893447in}}%
\pgfpathlineto{\pgfqpoint{2.194034in}{3.324136in}}%
\pgfpathlineto{\pgfqpoint{2.194095in}{3.345148in}}%
\pgfpathlineto{\pgfqpoint{2.194188in}{3.202045in}}%
\pgfpathlineto{\pgfqpoint{2.194404in}{2.893608in}}%
\pgfpathlineto{\pgfqpoint{2.195205in}{3.285263in}}%
\pgfpathlineto{\pgfqpoint{2.195267in}{3.237312in}}%
\pgfpathlineto{\pgfqpoint{2.195360in}{3.323581in}}%
\pgfpathlineto{\pgfqpoint{2.195822in}{2.974247in}}%
\pgfpathlineto{\pgfqpoint{2.195899in}{3.058345in}}%
\pgfpathlineto{\pgfqpoint{2.196870in}{2.808096in}}%
\pgfpathlineto{\pgfqpoint{2.196392in}{3.530423in}}%
\pgfpathlineto{\pgfqpoint{2.197024in}{2.986986in}}%
\pgfpathlineto{\pgfqpoint{2.198103in}{3.372242in}}%
\pgfpathlineto{\pgfqpoint{2.198165in}{3.186394in}}%
\pgfpathlineto{\pgfqpoint{2.198843in}{3.478456in}}%
\pgfpathlineto{\pgfqpoint{2.198442in}{2.903098in}}%
\pgfpathlineto{\pgfqpoint{2.199074in}{3.080438in}}%
\pgfpathlineto{\pgfqpoint{2.199337in}{2.932959in}}%
\pgfpathlineto{\pgfqpoint{2.199737in}{3.407007in}}%
\pgfpathlineto{\pgfqpoint{2.200200in}{3.051084in}}%
\pgfpathlineto{\pgfqpoint{2.201125in}{3.323120in}}%
\pgfpathlineto{\pgfqpoint{2.200338in}{2.971048in}}%
\pgfpathlineto{\pgfqpoint{2.201387in}{3.108284in}}%
\pgfpathlineto{\pgfqpoint{2.201433in}{3.038080in}}%
\pgfpathlineto{\pgfqpoint{2.202342in}{3.284168in}}%
\pgfpathlineto{\pgfqpoint{2.203344in}{3.447168in}}%
\pgfpathlineto{\pgfqpoint{2.202974in}{2.904268in}}%
\pgfpathlineto{\pgfqpoint{2.203483in}{3.392031in}}%
\pgfpathlineto{\pgfqpoint{2.203884in}{2.843443in}}%
\pgfpathlineto{\pgfqpoint{2.204809in}{3.102673in}}%
\pgfpathlineto{\pgfqpoint{2.205857in}{3.412390in}}%
\pgfpathlineto{\pgfqpoint{2.205425in}{2.955482in}}%
\pgfpathlineto{\pgfqpoint{2.205949in}{3.258844in}}%
\pgfpathlineto{\pgfqpoint{2.206304in}{2.945388in}}%
\pgfpathlineto{\pgfqpoint{2.206982in}{3.328951in}}%
\pgfpathlineto{\pgfqpoint{2.207136in}{3.096773in}}%
\pgfpathlineto{\pgfqpoint{2.208061in}{3.325029in}}%
\pgfpathlineto{\pgfqpoint{2.207460in}{3.014879in}}%
\pgfpathlineto{\pgfqpoint{2.208246in}{3.102617in}}%
\pgfpathlineto{\pgfqpoint{2.209094in}{3.001440in}}%
\pgfpathlineto{\pgfqpoint{2.208493in}{3.330229in}}%
\pgfpathlineto{\pgfqpoint{2.209264in}{3.277323in}}%
\pgfpathlineto{\pgfqpoint{2.209372in}{3.376400in}}%
\pgfpathlineto{\pgfqpoint{2.209849in}{2.933279in}}%
\pgfpathlineto{\pgfqpoint{2.210389in}{3.308222in}}%
\pgfpathlineto{\pgfqpoint{2.210728in}{3.022412in}}%
\pgfpathlineto{\pgfqpoint{2.211530in}{3.167323in}}%
\pgfpathlineto{\pgfqpoint{2.211684in}{3.332680in}}%
\pgfpathlineto{\pgfqpoint{2.212254in}{3.009563in}}%
\pgfpathlineto{\pgfqpoint{2.212624in}{3.155459in}}%
\pgfpathlineto{\pgfqpoint{2.212670in}{3.089193in}}%
\pgfpathlineto{\pgfqpoint{2.213056in}{3.288555in}}%
\pgfpathlineto{\pgfqpoint{2.213133in}{3.251069in}}%
\pgfpathlineto{\pgfqpoint{2.213164in}{3.356455in}}%
\pgfpathlineto{\pgfqpoint{2.213641in}{3.001073in}}%
\pgfpathlineto{\pgfqpoint{2.214243in}{3.310912in}}%
\pgfpathlineto{\pgfqpoint{2.214505in}{2.980217in}}%
\pgfpathlineto{\pgfqpoint{2.214320in}{3.319605in}}%
\pgfpathlineto{\pgfqpoint{2.215383in}{3.145642in}}%
\pgfpathlineto{\pgfqpoint{2.215538in}{3.400731in}}%
\pgfpathlineto{\pgfqpoint{2.216108in}{2.950663in}}%
\pgfpathlineto{\pgfqpoint{2.216509in}{3.216957in}}%
\pgfpathlineto{\pgfqpoint{2.216971in}{2.975966in}}%
\pgfpathlineto{\pgfqpoint{2.216694in}{3.501584in}}%
\pgfpathlineto{\pgfqpoint{2.217634in}{3.189645in}}%
\pgfpathlineto{\pgfqpoint{2.218281in}{3.378504in}}%
\pgfpathlineto{\pgfqpoint{2.217942in}{3.017246in}}%
\pgfpathlineto{\pgfqpoint{2.218528in}{3.124817in}}%
\pgfpathlineto{\pgfqpoint{2.219407in}{2.932310in}}%
\pgfpathlineto{\pgfqpoint{2.218929in}{3.462377in}}%
\pgfpathlineto{\pgfqpoint{2.219638in}{3.003536in}}%
\pgfpathlineto{\pgfqpoint{2.219977in}{3.355758in}}%
\pgfpathlineto{\pgfqpoint{2.220594in}{2.936996in}}%
\pgfpathlineto{\pgfqpoint{2.220763in}{3.142972in}}%
\pgfpathlineto{\pgfqpoint{2.221642in}{2.906880in}}%
\pgfpathlineto{\pgfqpoint{2.221380in}{3.435659in}}%
\pgfpathlineto{\pgfqpoint{2.221858in}{3.162392in}}%
\pgfpathlineto{\pgfqpoint{2.221904in}{3.480084in}}%
\pgfpathlineto{\pgfqpoint{2.222166in}{2.978377in}}%
\pgfpathlineto{\pgfqpoint{2.222967in}{3.316387in}}%
\pgfpathlineto{\pgfqpoint{2.224062in}{2.809885in}}%
\pgfpathlineto{\pgfqpoint{2.223599in}{3.477387in}}%
\pgfpathlineto{\pgfqpoint{2.224093in}{3.116798in}}%
\pgfpathlineto{\pgfqpoint{2.224509in}{3.419376in}}%
\pgfpathlineto{\pgfqpoint{2.224586in}{2.987108in}}%
\pgfpathlineto{\pgfqpoint{2.225203in}{3.184126in}}%
\pgfpathlineto{\pgfqpoint{2.225357in}{2.955641in}}%
\pgfpathlineto{\pgfqpoint{2.226050in}{3.444644in}}%
\pgfpathlineto{\pgfqpoint{2.226312in}{3.060812in}}%
\pgfpathlineto{\pgfqpoint{2.227237in}{3.477945in}}%
\pgfpathlineto{\pgfqpoint{2.226837in}{2.921078in}}%
\pgfpathlineto{\pgfqpoint{2.227453in}{3.172331in}}%
\pgfpathlineto{\pgfqpoint{2.228147in}{3.448460in}}%
\pgfpathlineto{\pgfqpoint{2.227684in}{2.940475in}}%
\pgfpathlineto{\pgfqpoint{2.228471in}{3.049566in}}%
\pgfpathlineto{\pgfqpoint{2.228501in}{2.985240in}}%
\pgfpathlineto{\pgfqpoint{2.228794in}{3.327224in}}%
\pgfpathlineto{\pgfqpoint{2.229164in}{3.292681in}}%
\pgfpathlineto{\pgfqpoint{2.229704in}{3.493913in}}%
\pgfpathlineto{\pgfqpoint{2.229257in}{2.863183in}}%
\pgfpathlineto{\pgfqpoint{2.230243in}{3.236754in}}%
\pgfpathlineto{\pgfqpoint{2.231076in}{2.950536in}}%
\pgfpathlineto{\pgfqpoint{2.230721in}{3.404852in}}%
\pgfpathlineto{\pgfqpoint{2.231353in}{3.153490in}}%
\pgfpathlineto{\pgfqpoint{2.231785in}{3.470505in}}%
\pgfpathlineto{\pgfqpoint{2.231476in}{2.970929in}}%
\pgfpathlineto{\pgfqpoint{2.232463in}{3.162606in}}%
\pgfpathlineto{\pgfqpoint{2.232571in}{3.026205in}}%
\pgfpathlineto{\pgfqpoint{2.233342in}{3.391369in}}%
\pgfpathlineto{\pgfqpoint{2.233573in}{3.128218in}}%
\pgfpathlineto{\pgfqpoint{2.233712in}{2.981795in}}%
\pgfpathlineto{\pgfqpoint{2.234390in}{3.463745in}}%
\pgfpathlineto{\pgfqpoint{2.234498in}{3.339040in}}%
\pgfpathlineto{\pgfqpoint{2.235407in}{3.399913in}}%
\pgfpathlineto{\pgfqpoint{2.234698in}{2.956203in}}%
\pgfpathlineto{\pgfqpoint{2.235561in}{3.272196in}}%
\pgfpathlineto{\pgfqpoint{2.235746in}{2.950635in}}%
\pgfpathlineto{\pgfqpoint{2.235916in}{3.332684in}}%
\pgfpathlineto{\pgfqpoint{2.236702in}{3.030848in}}%
\pgfpathlineto{\pgfqpoint{2.236964in}{3.428216in}}%
\pgfpathlineto{\pgfqpoint{2.237226in}{2.881541in}}%
\pgfpathlineto{\pgfqpoint{2.237889in}{3.334712in}}%
\pgfpathlineto{\pgfqpoint{2.238074in}{2.970023in}}%
\pgfpathlineto{\pgfqpoint{2.238012in}{3.409044in}}%
\pgfpathlineto{\pgfqpoint{2.238999in}{3.331608in}}%
\pgfpathlineto{\pgfqpoint{2.240047in}{3.390935in}}%
\pgfpathlineto{\pgfqpoint{2.239631in}{2.959962in}}%
\pgfpathlineto{\pgfqpoint{2.240093in}{3.335426in}}%
\pgfpathlineto{\pgfqpoint{2.240926in}{2.909895in}}%
\pgfpathlineto{\pgfqpoint{2.240571in}{3.475829in}}%
\pgfpathlineto{\pgfqpoint{2.241203in}{3.231765in}}%
\pgfpathlineto{\pgfqpoint{2.241604in}{3.469644in}}%
\pgfpathlineto{\pgfqpoint{2.241866in}{2.889095in}}%
\pgfpathlineto{\pgfqpoint{2.242313in}{3.234586in}}%
\pgfpathlineto{\pgfqpoint{2.243361in}{2.899112in}}%
\pgfpathlineto{\pgfqpoint{2.242652in}{3.419095in}}%
\pgfpathlineto{\pgfqpoint{2.243454in}{3.137218in}}%
\pgfpathlineto{\pgfqpoint{2.243885in}{2.941798in}}%
\pgfpathlineto{\pgfqpoint{2.243808in}{3.486007in}}%
\pgfpathlineto{\pgfqpoint{2.244564in}{3.134664in}}%
\pgfpathlineto{\pgfqpoint{2.245226in}{3.420143in}}%
\pgfpathlineto{\pgfqpoint{2.245442in}{3.004884in}}%
\pgfpathlineto{\pgfqpoint{2.246321in}{2.928335in}}%
\pgfpathlineto{\pgfqpoint{2.245874in}{3.432885in}}%
\pgfpathlineto{\pgfqpoint{2.246367in}{3.261695in}}%
\pgfpathlineto{\pgfqpoint{2.246768in}{3.472097in}}%
\pgfpathlineto{\pgfqpoint{2.246845in}{2.901632in}}%
\pgfpathlineto{\pgfqpoint{2.247462in}{3.275461in}}%
\pgfpathlineto{\pgfqpoint{2.248525in}{2.994941in}}%
\pgfpathlineto{\pgfqpoint{2.248464in}{3.394581in}}%
\pgfpathlineto{\pgfqpoint{2.248587in}{3.143442in}}%
\pgfpathlineto{\pgfqpoint{2.248972in}{3.427143in}}%
\pgfpathlineto{\pgfqpoint{2.249065in}{2.950041in}}%
\pgfpathlineto{\pgfqpoint{2.249743in}{3.232172in}}%
\pgfpathlineto{\pgfqpoint{2.250730in}{2.982337in}}%
\pgfpathlineto{\pgfqpoint{2.249866in}{3.405644in}}%
\pgfpathlineto{\pgfqpoint{2.250853in}{3.188858in}}%
\pgfpathlineto{\pgfqpoint{2.251947in}{3.514189in}}%
\pgfpathlineto{\pgfqpoint{2.251485in}{2.923190in}}%
\pgfpathlineto{\pgfqpoint{2.251963in}{3.435034in}}%
\pgfpathlineto{\pgfqpoint{2.252764in}{3.011078in}}%
\pgfpathlineto{\pgfqpoint{2.253088in}{3.276598in}}%
\pgfpathlineto{\pgfqpoint{2.253998in}{3.380953in}}%
\pgfpathlineto{\pgfqpoint{2.253705in}{2.976956in}}%
\pgfpathlineto{\pgfqpoint{2.254044in}{3.081847in}}%
\pgfpathlineto{\pgfqpoint{2.254244in}{2.908503in}}%
\pgfpathlineto{\pgfqpoint{2.254506in}{3.494951in}}%
\pgfpathlineto{\pgfqpoint{2.255123in}{3.042031in}}%
\pgfpathlineto{\pgfqpoint{2.255539in}{3.403235in}}%
\pgfpathlineto{\pgfqpoint{2.255909in}{2.991995in}}%
\pgfpathlineto{\pgfqpoint{2.256233in}{3.227602in}}%
\pgfpathlineto{\pgfqpoint{2.256664in}{2.950587in}}%
\pgfpathlineto{\pgfqpoint{2.257111in}{3.475363in}}%
\pgfpathlineto{\pgfqpoint{2.257358in}{3.119407in}}%
\pgfpathlineto{\pgfqpoint{2.257420in}{2.990027in}}%
\pgfpathlineto{\pgfqpoint{2.257589in}{3.313536in}}%
\pgfpathlineto{\pgfqpoint{2.258113in}{3.404414in}}%
\pgfpathlineto{\pgfqpoint{2.258375in}{3.012687in}}%
\pgfpathlineto{\pgfqpoint{2.258668in}{3.254023in}}%
\pgfpathlineto{\pgfqpoint{2.259408in}{2.923634in}}%
\pgfpathlineto{\pgfqpoint{2.259686in}{3.465205in}}%
\pgfpathlineto{\pgfqpoint{2.259778in}{3.143835in}}%
\pgfpathlineto{\pgfqpoint{2.260842in}{3.382224in}}%
\pgfpathlineto{\pgfqpoint{2.260271in}{2.992222in}}%
\pgfpathlineto{\pgfqpoint{2.260873in}{3.165450in}}%
\pgfpathlineto{\pgfqpoint{2.260919in}{2.976899in}}%
\pgfpathlineto{\pgfqpoint{2.261720in}{3.344649in}}%
\pgfpathlineto{\pgfqpoint{2.261998in}{3.053017in}}%
\pgfpathlineto{\pgfqpoint{2.262784in}{3.394513in}}%
\pgfpathlineto{\pgfqpoint{2.262584in}{3.008129in}}%
\pgfpathlineto{\pgfqpoint{2.263092in}{3.045644in}}%
\pgfpathlineto{\pgfqpoint{2.264063in}{2.990928in}}%
\pgfpathlineto{\pgfqpoint{2.263801in}{3.409295in}}%
\pgfpathlineto{\pgfqpoint{2.264156in}{3.127048in}}%
\pgfpathlineto{\pgfqpoint{2.264341in}{3.413452in}}%
\pgfpathlineto{\pgfqpoint{2.264588in}{2.990000in}}%
\pgfpathlineto{\pgfqpoint{2.265297in}{3.227810in}}%
\pgfpathlineto{\pgfqpoint{2.265867in}{3.405903in}}%
\pgfpathlineto{\pgfqpoint{2.265559in}{2.981579in}}%
\pgfpathlineto{\pgfqpoint{2.266268in}{3.133432in}}%
\pgfpathlineto{\pgfqpoint{2.267116in}{3.020569in}}%
\pgfpathlineto{\pgfqpoint{2.266884in}{3.403716in}}%
\pgfpathlineto{\pgfqpoint{2.267362in}{3.134911in}}%
\pgfpathlineto{\pgfqpoint{2.268441in}{3.360129in}}%
\pgfpathlineto{\pgfqpoint{2.268287in}{2.995890in}}%
\pgfpathlineto{\pgfqpoint{2.268472in}{3.300674in}}%
\pgfpathlineto{\pgfqpoint{2.269227in}{2.980480in}}%
\pgfpathlineto{\pgfqpoint{2.268965in}{3.479394in}}%
\pgfpathlineto{\pgfqpoint{2.269582in}{3.153872in}}%
\pgfpathlineto{\pgfqpoint{2.270538in}{3.337354in}}%
\pgfpathlineto{\pgfqpoint{2.269798in}{3.023916in}}%
\pgfpathlineto{\pgfqpoint{2.270676in}{3.167224in}}%
\pgfpathlineto{\pgfqpoint{2.270723in}{2.965980in}}%
\pgfpathlineto{\pgfqpoint{2.271170in}{3.388289in}}%
\pgfpathlineto{\pgfqpoint{2.271802in}{3.016631in}}%
\pgfpathlineto{\pgfqpoint{2.272064in}{3.423938in}}%
\pgfpathlineto{\pgfqpoint{2.272912in}{3.048112in}}%
\pgfpathlineto{\pgfqpoint{2.272927in}{3.013252in}}%
\pgfpathlineto{\pgfqpoint{2.273621in}{3.375934in}}%
\pgfpathlineto{\pgfqpoint{2.273960in}{3.127613in}}%
\pgfpathlineto{\pgfqpoint{2.274145in}{3.408316in}}%
\pgfpathlineto{\pgfqpoint{2.274962in}{2.968668in}}%
\pgfpathlineto{\pgfqpoint{2.275070in}{3.158473in}}%
\pgfpathlineto{\pgfqpoint{2.275640in}{3.363515in}}%
\pgfpathlineto{\pgfqpoint{2.275902in}{2.992689in}}%
\pgfpathlineto{\pgfqpoint{2.276226in}{3.280182in}}%
\pgfpathlineto{\pgfqpoint{2.276442in}{2.950734in}}%
\pgfpathlineto{\pgfqpoint{2.277243in}{3.396318in}}%
\pgfpathlineto{\pgfqpoint{2.277336in}{3.272831in}}%
\pgfpathlineto{\pgfqpoint{2.278245in}{3.340908in}}%
\pgfpathlineto{\pgfqpoint{2.277937in}{2.984963in}}%
\pgfpathlineto{\pgfqpoint{2.278415in}{3.249648in}}%
\pgfpathlineto{\pgfqpoint{2.279309in}{3.379870in}}%
\pgfpathlineto{\pgfqpoint{2.279602in}{2.980911in}}%
\pgfpathlineto{\pgfqpoint{2.280819in}{3.399472in}}%
\pgfpathlineto{\pgfqpoint{2.281081in}{3.028464in}}%
\pgfpathlineto{\pgfqpoint{2.281975in}{3.249956in}}%
\pgfpathlineto{\pgfqpoint{2.283024in}{3.383571in}}%
\pgfpathlineto{\pgfqpoint{2.282561in}{3.012192in}}%
\pgfpathlineto{\pgfqpoint{2.283054in}{3.268606in}}%
\pgfpathlineto{\pgfqpoint{2.283101in}{2.989890in}}%
\pgfpathlineto{\pgfqpoint{2.283918in}{3.304582in}}%
\pgfpathlineto{\pgfqpoint{2.284180in}{3.102382in}}%
\pgfpathlineto{\pgfqpoint{2.284442in}{3.354327in}}%
\pgfpathlineto{\pgfqpoint{2.284766in}{2.989929in}}%
\pgfpathlineto{\pgfqpoint{2.285290in}{3.127167in}}%
\pgfpathlineto{\pgfqpoint{2.285798in}{3.029123in}}%
\pgfpathlineto{\pgfqpoint{2.285999in}{3.365435in}}%
\pgfpathlineto{\pgfqpoint{2.286369in}{3.153516in}}%
\pgfpathlineto{\pgfqpoint{2.287047in}{3.309736in}}%
\pgfpathlineto{\pgfqpoint{2.287232in}{3.016423in}}%
\pgfpathlineto{\pgfqpoint{2.287509in}{3.252380in}}%
\pgfpathlineto{\pgfqpoint{2.288280in}{2.992725in}}%
\pgfpathlineto{\pgfqpoint{2.288542in}{3.304508in}}%
\pgfpathlineto{\pgfqpoint{2.288665in}{3.135223in}}%
\pgfpathlineto{\pgfqpoint{2.289082in}{3.342787in}}%
\pgfpathlineto{\pgfqpoint{2.289313in}{3.044664in}}%
\pgfpathlineto{\pgfqpoint{2.289760in}{3.120605in}}%
\pgfpathlineto{\pgfqpoint{2.289960in}{3.011051in}}%
\pgfpathlineto{\pgfqpoint{2.290762in}{3.316818in}}%
\pgfpathlineto{\pgfqpoint{2.290885in}{3.074431in}}%
\pgfpathlineto{\pgfqpoint{2.291147in}{3.337211in}}%
\pgfpathlineto{\pgfqpoint{2.291240in}{3.048708in}}%
\pgfpathlineto{\pgfqpoint{2.291964in}{3.091398in}}%
\pgfpathlineto{\pgfqpoint{2.292920in}{2.998351in}}%
\pgfpathlineto{\pgfqpoint{2.292673in}{3.358033in}}%
\pgfpathlineto{\pgfqpoint{2.293059in}{3.096524in}}%
\pgfpathlineto{\pgfqpoint{2.293722in}{3.400712in}}%
\pgfpathlineto{\pgfqpoint{2.293459in}{3.013883in}}%
\pgfpathlineto{\pgfqpoint{2.294169in}{3.158008in}}%
\pgfpathlineto{\pgfqpoint{2.294970in}{3.056908in}}%
\pgfpathlineto{\pgfqpoint{2.294893in}{3.342444in}}%
\pgfpathlineto{\pgfqpoint{2.295248in}{3.244506in}}%
\pgfpathlineto{\pgfqpoint{2.296327in}{3.339310in}}%
\pgfpathlineto{\pgfqpoint{2.295494in}{3.024326in}}%
\pgfpathlineto{\pgfqpoint{2.296357in}{3.258006in}}%
\pgfpathlineto{\pgfqpoint{2.296635in}{3.010650in}}%
\pgfpathlineto{\pgfqpoint{2.296820in}{3.312891in}}%
\pgfpathlineto{\pgfqpoint{2.297467in}{3.224041in}}%
\pgfpathlineto{\pgfqpoint{2.297483in}{3.226211in}}%
\pgfpathlineto{\pgfqpoint{2.297560in}{3.083382in}}%
\pgfpathlineto{\pgfqpoint{2.297575in}{3.088216in}}%
\pgfpathlineto{\pgfqpoint{2.298099in}{3.024563in}}%
\pgfpathlineto{\pgfqpoint{2.297853in}{3.359114in}}%
\pgfpathlineto{\pgfqpoint{2.298346in}{3.248199in}}%
\pgfpathlineto{\pgfqpoint{2.298392in}{3.354971in}}%
\pgfpathlineto{\pgfqpoint{2.299101in}{3.049150in}}%
\pgfpathlineto{\pgfqpoint{2.299440in}{3.281428in}}%
\pgfpathlineto{\pgfqpoint{2.300134in}{2.966258in}}%
\pgfpathlineto{\pgfqpoint{2.300057in}{3.315681in}}%
\pgfpathlineto{\pgfqpoint{2.300550in}{3.225976in}}%
\pgfpathlineto{\pgfqpoint{2.301460in}{3.348822in}}%
\pgfpathlineto{\pgfqpoint{2.301167in}{3.034437in}}%
\pgfpathlineto{\pgfqpoint{2.301629in}{3.159328in}}%
\pgfpathlineto{\pgfqpoint{2.301984in}{3.311551in}}%
\pgfpathlineto{\pgfqpoint{2.301799in}{3.019614in}}%
\pgfpathlineto{\pgfqpoint{2.302539in}{3.101762in}}%
\pgfpathlineto{\pgfqpoint{2.302708in}{3.024937in}}%
\pgfpathlineto{\pgfqpoint{2.303017in}{3.434055in}}%
\pgfpathlineto{\pgfqpoint{2.303633in}{3.125258in}}%
\pgfpathlineto{\pgfqpoint{2.304065in}{3.313291in}}%
\pgfpathlineto{\pgfqpoint{2.304265in}{3.057704in}}%
\pgfpathlineto{\pgfqpoint{2.304728in}{3.170297in}}%
\pgfpathlineto{\pgfqpoint{2.304774in}{2.982188in}}%
\pgfpathlineto{\pgfqpoint{2.305576in}{3.347155in}}%
\pgfpathlineto{\pgfqpoint{2.305853in}{3.090749in}}%
\pgfpathlineto{\pgfqpoint{2.306115in}{3.363494in}}%
\pgfpathlineto{\pgfqpoint{2.306331in}{3.053155in}}%
\pgfpathlineto{\pgfqpoint{2.306947in}{3.107922in}}%
\pgfpathlineto{\pgfqpoint{2.307348in}{3.019367in}}%
\pgfpathlineto{\pgfqpoint{2.307657in}{3.361606in}}%
\pgfpathlineto{\pgfqpoint{2.308011in}{3.126729in}}%
\pgfpathlineto{\pgfqpoint{2.308181in}{3.390709in}}%
\pgfpathlineto{\pgfqpoint{2.308258in}{3.026272in}}%
\pgfpathlineto{\pgfqpoint{2.309167in}{3.259656in}}%
\pgfpathlineto{\pgfqpoint{2.310231in}{3.359205in}}%
\pgfpathlineto{\pgfqpoint{2.309938in}{2.985901in}}%
\pgfpathlineto{\pgfqpoint{2.310262in}{3.301278in}}%
\pgfpathlineto{\pgfqpoint{2.310477in}{3.010438in}}%
\pgfpathlineto{\pgfqpoint{2.311279in}{3.360691in}}%
\pgfpathlineto{\pgfqpoint{2.311372in}{3.178682in}}%
\pgfpathlineto{\pgfqpoint{2.311911in}{3.326229in}}%
\pgfpathlineto{\pgfqpoint{2.311988in}{3.034184in}}%
\pgfpathlineto{\pgfqpoint{2.312481in}{3.173714in}}%
\pgfpathlineto{\pgfqpoint{2.312898in}{3.010556in}}%
\pgfpathlineto{\pgfqpoint{2.312821in}{3.354840in}}%
\pgfpathlineto{\pgfqpoint{2.313591in}{3.156314in}}%
\pgfpathlineto{\pgfqpoint{2.313838in}{3.325362in}}%
\pgfpathlineto{\pgfqpoint{2.314177in}{3.000244in}}%
\pgfpathlineto{\pgfqpoint{2.314670in}{3.160283in}}%
\pgfpathlineto{\pgfqpoint{2.315657in}{3.026226in}}%
\pgfpathlineto{\pgfqpoint{2.314871in}{3.388689in}}%
\pgfpathlineto{\pgfqpoint{2.315765in}{3.154076in}}%
\pgfpathlineto{\pgfqpoint{2.315919in}{3.353653in}}%
\pgfpathlineto{\pgfqpoint{2.316628in}{3.023760in}}%
\pgfpathlineto{\pgfqpoint{2.316844in}{3.122291in}}%
\pgfpathlineto{\pgfqpoint{2.317676in}{3.026571in}}%
\pgfpathlineto{\pgfqpoint{2.317075in}{3.381513in}}%
\pgfpathlineto{\pgfqpoint{2.317923in}{3.166190in}}%
\pgfpathlineto{\pgfqpoint{2.319002in}{3.342182in}}%
\pgfpathlineto{\pgfqpoint{2.318817in}{3.025934in}}%
\pgfpathlineto{\pgfqpoint{2.319033in}{3.274204in}}%
\pgfpathlineto{\pgfqpoint{2.319233in}{3.041974in}}%
\pgfpathlineto{\pgfqpoint{2.320035in}{3.390123in}}%
\pgfpathlineto{\pgfqpoint{2.320143in}{3.164122in}}%
\pgfpathlineto{\pgfqpoint{2.320543in}{3.314923in}}%
\pgfpathlineto{\pgfqpoint{2.320281in}{3.023857in}}%
\pgfpathlineto{\pgfqpoint{2.321237in}{3.181959in}}%
\pgfpathlineto{\pgfqpoint{2.321299in}{3.021651in}}%
\pgfpathlineto{\pgfqpoint{2.322239in}{3.347554in}}%
\pgfpathlineto{\pgfqpoint{2.322362in}{3.125159in}}%
\pgfpathlineto{\pgfqpoint{2.323133in}{3.375137in}}%
\pgfpathlineto{\pgfqpoint{2.322840in}{3.035720in}}%
\pgfpathlineto{\pgfqpoint{2.323457in}{3.078161in}}%
\pgfpathlineto{\pgfqpoint{2.323888in}{3.067987in}}%
\pgfpathlineto{\pgfqpoint{2.323642in}{3.283419in}}%
\pgfpathlineto{\pgfqpoint{2.324027in}{3.172931in}}%
\pgfpathlineto{\pgfqpoint{2.324675in}{3.348838in}}%
\pgfpathlineto{\pgfqpoint{2.324998in}{3.042356in}}%
\pgfpathlineto{\pgfqpoint{2.325122in}{3.137751in}}%
\pgfpathlineto{\pgfqpoint{2.325199in}{3.373773in}}%
\pgfpathlineto{\pgfqpoint{2.326031in}{3.049767in}}%
\pgfpathlineto{\pgfqpoint{2.326262in}{3.237532in}}%
\pgfpathlineto{\pgfqpoint{2.326478in}{3.048698in}}%
\pgfpathlineto{\pgfqpoint{2.326725in}{3.320965in}}%
\pgfpathlineto{\pgfqpoint{2.327372in}{3.158011in}}%
\pgfpathlineto{\pgfqpoint{2.327757in}{3.351791in}}%
\pgfpathlineto{\pgfqpoint{2.327495in}{3.040382in}}%
\pgfpathlineto{\pgfqpoint{2.328482in}{3.167440in}}%
\pgfpathlineto{\pgfqpoint{2.329160in}{3.076461in}}%
\pgfpathlineto{\pgfqpoint{2.329314in}{3.317922in}}%
\pgfpathlineto{\pgfqpoint{2.329607in}{3.140116in}}%
\pgfpathlineto{\pgfqpoint{2.330363in}{3.343646in}}%
\pgfpathlineto{\pgfqpoint{2.330162in}{3.078732in}}%
\pgfpathlineto{\pgfqpoint{2.330640in}{3.099508in}}%
\pgfpathlineto{\pgfqpoint{2.331195in}{3.038254in}}%
\pgfpathlineto{\pgfqpoint{2.331349in}{3.284222in}}%
\pgfpathlineto{\pgfqpoint{2.331735in}{3.115196in}}%
\pgfpathlineto{\pgfqpoint{2.331889in}{3.332929in}}%
\pgfpathlineto{\pgfqpoint{2.332675in}{3.039797in}}%
\pgfpathlineto{\pgfqpoint{2.332844in}{3.152657in}}%
\pgfpathlineto{\pgfqpoint{2.333261in}{3.080228in}}%
\pgfpathlineto{\pgfqpoint{2.332921in}{3.337373in}}%
\pgfpathlineto{\pgfqpoint{2.333923in}{3.247234in}}%
\pgfpathlineto{\pgfqpoint{2.333954in}{3.319010in}}%
\pgfpathlineto{\pgfqpoint{2.334170in}{3.052321in}}%
\pgfpathlineto{\pgfqpoint{2.335018in}{3.282758in}}%
\pgfpathlineto{\pgfqpoint{2.335835in}{3.056365in}}%
\pgfpathlineto{\pgfqpoint{2.335527in}{3.312060in}}%
\pgfpathlineto{\pgfqpoint{2.336143in}{3.212136in}}%
\pgfpathlineto{\pgfqpoint{2.337053in}{3.369964in}}%
\pgfpathlineto{\pgfqpoint{2.336791in}{3.061239in}}%
\pgfpathlineto{\pgfqpoint{2.337238in}{3.213474in}}%
\pgfpathlineto{\pgfqpoint{2.337299in}{3.056844in}}%
\pgfpathlineto{\pgfqpoint{2.338085in}{3.317653in}}%
\pgfpathlineto{\pgfqpoint{2.338378in}{3.149009in}}%
\pgfpathlineto{\pgfqpoint{2.338933in}{3.055822in}}%
\pgfpathlineto{\pgfqpoint{2.339257in}{3.325958in}}%
\pgfpathlineto{\pgfqpoint{2.339473in}{3.146939in}}%
\pgfpathlineto{\pgfqpoint{2.340136in}{3.310339in}}%
\pgfpathlineto{\pgfqpoint{2.340367in}{3.057851in}}%
\pgfpathlineto{\pgfqpoint{2.340475in}{3.084099in}}%
\pgfpathlineto{\pgfqpoint{2.341430in}{3.070599in}}%
\pgfpathlineto{\pgfqpoint{2.340660in}{3.294515in}}%
\pgfpathlineto{\pgfqpoint{2.341446in}{3.102492in}}%
\pgfpathlineto{\pgfqpoint{2.341692in}{3.364615in}}%
\pgfpathlineto{\pgfqpoint{2.342479in}{3.064368in}}%
\pgfpathlineto{\pgfqpoint{2.342556in}{3.122535in}}%
\pgfpathlineto{\pgfqpoint{2.343049in}{3.065909in}}%
\pgfpathlineto{\pgfqpoint{2.343249in}{3.294911in}}%
\pgfpathlineto{\pgfqpoint{2.343635in}{3.138053in}}%
\pgfpathlineto{\pgfqpoint{2.344267in}{3.313487in}}%
\pgfpathlineto{\pgfqpoint{2.344513in}{3.052466in}}%
\pgfpathlineto{\pgfqpoint{2.344683in}{3.133473in}}%
\pgfpathlineto{\pgfqpoint{2.345562in}{3.070260in}}%
\pgfpathlineto{\pgfqpoint{2.345300in}{3.346786in}}%
\pgfpathlineto{\pgfqpoint{2.345762in}{3.142239in}}%
\pgfpathlineto{\pgfqpoint{2.346856in}{3.336769in}}%
\pgfpathlineto{\pgfqpoint{2.346594in}{3.063260in}}%
\pgfpathlineto{\pgfqpoint{2.346887in}{3.238883in}}%
\pgfpathlineto{\pgfqpoint{2.347458in}{3.077777in}}%
\pgfpathlineto{\pgfqpoint{2.347381in}{3.325321in}}%
\pgfpathlineto{\pgfqpoint{2.347997in}{3.243852in}}%
\pgfpathlineto{\pgfqpoint{2.348891in}{3.328220in}}%
\pgfpathlineto{\pgfqpoint{2.348213in}{3.043310in}}%
\pgfpathlineto{\pgfqpoint{2.349076in}{3.228844in}}%
\pgfpathlineto{\pgfqpoint{2.349677in}{3.065971in}}%
\pgfpathlineto{\pgfqpoint{2.349939in}{3.327010in}}%
\pgfpathlineto{\pgfqpoint{2.350217in}{3.131400in}}%
\pgfpathlineto{\pgfqpoint{2.350972in}{3.314223in}}%
\pgfpathlineto{\pgfqpoint{2.351173in}{3.082605in}}%
\pgfpathlineto{\pgfqpoint{2.351311in}{3.143765in}}%
\pgfpathlineto{\pgfqpoint{2.352082in}{3.076165in}}%
\pgfpathlineto{\pgfqpoint{2.352005in}{3.317167in}}%
\pgfpathlineto{\pgfqpoint{2.352421in}{3.139092in}}%
\pgfpathlineto{\pgfqpoint{2.352837in}{3.068195in}}%
\pgfpathlineto{\pgfqpoint{2.353022in}{3.291915in}}%
\pgfpathlineto{\pgfqpoint{2.353423in}{3.206713in}}%
\pgfpathlineto{\pgfqpoint{2.354055in}{3.361409in}}%
\pgfpathlineto{\pgfqpoint{2.353809in}{3.101533in}}%
\pgfpathlineto{\pgfqpoint{2.354286in}{3.123919in}}%
\pgfpathlineto{\pgfqpoint{2.355304in}{3.083481in}}%
\pgfpathlineto{\pgfqpoint{2.355088in}{3.306413in}}%
\pgfpathlineto{\pgfqpoint{2.355381in}{3.141193in}}%
\pgfpathlineto{\pgfqpoint{2.355751in}{3.290521in}}%
\pgfpathlineto{\pgfqpoint{2.356337in}{3.068178in}}%
\pgfpathlineto{\pgfqpoint{2.356506in}{3.159017in}}%
\pgfpathlineto{\pgfqpoint{2.357076in}{3.088029in}}%
\pgfpathlineto{\pgfqpoint{2.357154in}{3.307286in}}%
\pgfpathlineto{\pgfqpoint{2.357616in}{3.133963in}}%
\pgfpathlineto{\pgfqpoint{2.358695in}{3.322123in}}%
\pgfpathlineto{\pgfqpoint{2.358263in}{3.081261in}}%
\pgfpathlineto{\pgfqpoint{2.358741in}{3.180945in}}%
\pgfpathlineto{\pgfqpoint{2.359296in}{3.085417in}}%
\pgfpathlineto{\pgfqpoint{2.359219in}{3.349140in}}%
\pgfpathlineto{\pgfqpoint{2.359851in}{3.180452in}}%
\pgfpathlineto{\pgfqpoint{2.360252in}{3.329844in}}%
\pgfpathlineto{\pgfqpoint{2.360514in}{3.086989in}}%
\pgfpathlineto{\pgfqpoint{2.360946in}{3.161214in}}%
\pgfpathlineto{\pgfqpoint{2.361501in}{3.083053in}}%
\pgfpathlineto{\pgfqpoint{2.361285in}{3.278531in}}%
\pgfpathlineto{\pgfqpoint{2.362071in}{3.110614in}}%
\pgfpathlineto{\pgfqpoint{2.362302in}{3.301360in}}%
\pgfpathlineto{\pgfqpoint{2.362240in}{3.078658in}}%
\pgfpathlineto{\pgfqpoint{2.363242in}{3.198049in}}%
\pgfpathlineto{\pgfqpoint{2.364075in}{3.101445in}}%
\pgfpathlineto{\pgfqpoint{2.363859in}{3.331889in}}%
\pgfpathlineto{\pgfqpoint{2.364337in}{3.186180in}}%
\pgfpathlineto{\pgfqpoint{2.364383in}{3.321323in}}%
\pgfpathlineto{\pgfqpoint{2.364460in}{3.068371in}}%
\pgfpathlineto{\pgfqpoint{2.365447in}{3.239745in}}%
\pgfpathlineto{\pgfqpoint{2.365663in}{3.098272in}}%
\pgfpathlineto{\pgfqpoint{2.365909in}{3.277625in}}%
\pgfpathlineto{\pgfqpoint{2.366557in}{3.185517in}}%
\pgfpathlineto{\pgfqpoint{2.367466in}{3.300653in}}%
\pgfpathlineto{\pgfqpoint{2.366695in}{3.086741in}}%
\pgfpathlineto{\pgfqpoint{2.367651in}{3.187195in}}%
\pgfpathlineto{\pgfqpoint{2.367728in}{3.095361in}}%
\pgfpathlineto{\pgfqpoint{2.368499in}{3.277876in}}%
\pgfpathlineto{\pgfqpoint{2.368792in}{3.135550in}}%
\pgfpathlineto{\pgfqpoint{2.369023in}{3.296489in}}%
\pgfpathlineto{\pgfqpoint{2.369100in}{3.090658in}}%
\pgfpathlineto{\pgfqpoint{2.369902in}{3.170275in}}%
\pgfpathlineto{\pgfqpoint{2.370379in}{3.094354in}}%
\pgfpathlineto{\pgfqpoint{2.370549in}{3.275020in}}%
\pgfpathlineto{\pgfqpoint{2.371011in}{3.144276in}}%
\pgfpathlineto{\pgfqpoint{2.372106in}{3.315885in}}%
\pgfpathlineto{\pgfqpoint{2.371859in}{3.083303in}}%
\pgfpathlineto{\pgfqpoint{2.372137in}{3.237141in}}%
\pgfpathlineto{\pgfqpoint{2.372892in}{3.093457in}}%
\pgfpathlineto{\pgfqpoint{2.373139in}{3.274035in}}%
\pgfpathlineto{\pgfqpoint{2.373247in}{3.189495in}}%
\pgfpathlineto{\pgfqpoint{2.374172in}{3.295444in}}%
\pgfpathlineto{\pgfqpoint{2.373355in}{3.095969in}}%
\pgfpathlineto{\pgfqpoint{2.374357in}{3.190665in}}%
\pgfpathlineto{\pgfqpoint{2.375281in}{3.095767in}}%
\pgfpathlineto{\pgfqpoint{2.374680in}{3.277296in}}%
\pgfpathlineto{\pgfqpoint{2.375466in}{3.136316in}}%
\pgfpathlineto{\pgfqpoint{2.376237in}{3.306624in}}%
\pgfpathlineto{\pgfqpoint{2.376299in}{3.061361in}}%
\pgfpathlineto{\pgfqpoint{2.376576in}{3.165823in}}%
\pgfpathlineto{\pgfqpoint{2.377347in}{3.092732in}}%
\pgfpathlineto{\pgfqpoint{2.377270in}{3.295658in}}%
\pgfpathlineto{\pgfqpoint{2.377686in}{3.166475in}}%
\pgfpathlineto{\pgfqpoint{2.378796in}{3.282906in}}%
\pgfpathlineto{\pgfqpoint{2.378519in}{3.085838in}}%
\pgfpathlineto{\pgfqpoint{2.378827in}{3.221665in}}%
\pgfpathlineto{\pgfqpoint{2.379258in}{3.083564in}}%
\pgfpathlineto{\pgfqpoint{2.379320in}{3.315163in}}%
\pgfpathlineto{\pgfqpoint{2.379937in}{3.173593in}}%
\pgfpathlineto{\pgfqpoint{2.380862in}{3.313048in}}%
\pgfpathlineto{\pgfqpoint{2.380939in}{3.089080in}}%
\pgfpathlineto{\pgfqpoint{2.381031in}{3.160548in}}%
\pgfpathlineto{\pgfqpoint{2.381478in}{3.089136in}}%
\pgfpathlineto{\pgfqpoint{2.381386in}{3.329467in}}%
\pgfpathlineto{\pgfqpoint{2.382126in}{3.156824in}}%
\pgfpathlineto{\pgfqpoint{2.382927in}{3.315325in}}%
\pgfpathlineto{\pgfqpoint{2.382218in}{3.077159in}}%
\pgfpathlineto{\pgfqpoint{2.383220in}{3.162622in}}%
\pgfpathlineto{\pgfqpoint{2.383698in}{3.078058in}}%
\pgfpathlineto{\pgfqpoint{2.383960in}{3.336690in}}%
\pgfpathlineto{\pgfqpoint{2.384314in}{3.168557in}}%
\pgfpathlineto{\pgfqpoint{2.384993in}{3.303464in}}%
\pgfpathlineto{\pgfqpoint{2.385193in}{3.097441in}}%
\pgfpathlineto{\pgfqpoint{2.385424in}{3.167905in}}%
\pgfpathlineto{\pgfqpoint{2.386103in}{3.079375in}}%
\pgfpathlineto{\pgfqpoint{2.386026in}{3.311703in}}%
\pgfpathlineto{\pgfqpoint{2.386488in}{3.186743in}}%
\pgfpathlineto{\pgfqpoint{2.386534in}{3.315838in}}%
\pgfpathlineto{\pgfqpoint{2.386843in}{3.084164in}}%
\pgfpathlineto{\pgfqpoint{2.387598in}{3.195043in}}%
\pgfpathlineto{\pgfqpoint{2.388153in}{3.074728in}}%
\pgfpathlineto{\pgfqpoint{2.388076in}{3.343025in}}%
\pgfpathlineto{\pgfqpoint{2.388708in}{3.182267in}}%
\pgfpathlineto{\pgfqpoint{2.389109in}{3.333008in}}%
\pgfpathlineto{\pgfqpoint{2.389186in}{3.076475in}}%
\pgfpathlineto{\pgfqpoint{2.389787in}{3.192785in}}%
\pgfpathlineto{\pgfqpoint{2.390357in}{3.047218in}}%
\pgfpathlineto{\pgfqpoint{2.390650in}{3.289746in}}%
\pgfpathlineto{\pgfqpoint{2.390912in}{3.112957in}}%
\pgfpathlineto{\pgfqpoint{2.391174in}{3.327022in}}%
\pgfpathlineto{\pgfqpoint{2.391097in}{3.079242in}}%
\pgfpathlineto{\pgfqpoint{2.392006in}{3.111402in}}%
\pgfpathlineto{\pgfqpoint{2.392130in}{3.094274in}}%
\pgfpathlineto{\pgfqpoint{2.392716in}{3.350192in}}%
\pgfpathlineto{\pgfqpoint{2.393055in}{3.123911in}}%
\pgfpathlineto{\pgfqpoint{2.393240in}{3.348306in}}%
\pgfpathlineto{\pgfqpoint{2.393301in}{3.072158in}}%
\pgfpathlineto{\pgfqpoint{2.394180in}{3.153086in}}%
\pgfpathlineto{\pgfqpoint{2.394272in}{3.327977in}}%
\pgfpathlineto{\pgfqpoint{2.394982in}{3.092055in}}%
\pgfpathlineto{\pgfqpoint{2.395336in}{3.181743in}}%
\pgfpathlineto{\pgfqpoint{2.396261in}{3.074001in}}%
\pgfpathlineto{\pgfqpoint{2.396323in}{3.346328in}}%
\pgfpathlineto{\pgfqpoint{2.396431in}{3.203771in}}%
\pgfpathlineto{\pgfqpoint{2.396847in}{3.327872in}}%
\pgfpathlineto{\pgfqpoint{2.396585in}{3.096216in}}%
\pgfpathlineto{\pgfqpoint{2.397402in}{3.170161in}}%
\pgfpathlineto{\pgfqpoint{2.397941in}{3.075875in}}%
\pgfpathlineto{\pgfqpoint{2.397880in}{3.333828in}}%
\pgfpathlineto{\pgfqpoint{2.398496in}{3.143326in}}%
\pgfpathlineto{\pgfqpoint{2.398897in}{3.331879in}}%
\pgfpathlineto{\pgfqpoint{2.399221in}{3.063804in}}%
\pgfpathlineto{\pgfqpoint{2.399606in}{3.157781in}}%
\pgfpathlineto{\pgfqpoint{2.400700in}{3.069451in}}%
\pgfpathlineto{\pgfqpoint{2.399914in}{3.339998in}}%
\pgfpathlineto{\pgfqpoint{2.400731in}{3.116402in}}%
\pgfpathlineto{\pgfqpoint{2.400963in}{3.351613in}}%
\pgfpathlineto{\pgfqpoint{2.400885in}{3.059363in}}%
\pgfpathlineto{\pgfqpoint{2.401872in}{3.177299in}}%
\pgfpathlineto{\pgfqpoint{2.402936in}{3.082553in}}%
\pgfpathlineto{\pgfqpoint{2.401995in}{3.316183in}}%
\pgfpathlineto{\pgfqpoint{2.402966in}{3.187584in}}%
\pgfpathlineto{\pgfqpoint{2.404045in}{3.343103in}}%
\pgfpathlineto{\pgfqpoint{2.403968in}{3.070013in}}%
\pgfpathlineto{\pgfqpoint{2.404076in}{3.228138in}}%
\pgfpathlineto{\pgfqpoint{2.405140in}{3.062600in}}%
\pgfpathlineto{\pgfqpoint{2.405078in}{3.366984in}}%
\pgfpathlineto{\pgfqpoint{2.405186in}{3.227042in}}%
\pgfpathlineto{\pgfqpoint{2.406111in}{3.340996in}}%
\pgfpathlineto{\pgfqpoint{2.406173in}{3.073735in}}%
\pgfpathlineto{\pgfqpoint{2.406265in}{3.201451in}}%
\pgfpathlineto{\pgfqpoint{2.407344in}{3.075628in}}%
\pgfpathlineto{\pgfqpoint{2.407128in}{3.285136in}}%
\pgfpathlineto{\pgfqpoint{2.407390in}{3.140276in}}%
\pgfpathlineto{\pgfqpoint{2.408161in}{3.353411in}}%
\pgfpathlineto{\pgfqpoint{2.408100in}{3.066999in}}%
\pgfpathlineto{\pgfqpoint{2.408470in}{3.129463in}}%
\pgfpathlineto{\pgfqpoint{2.408608in}{3.050460in}}%
\pgfpathlineto{\pgfqpoint{2.409194in}{3.341028in}}%
\pgfpathlineto{\pgfqpoint{2.409533in}{3.153793in}}%
\pgfpathlineto{\pgfqpoint{2.409703in}{3.358399in}}%
\pgfpathlineto{\pgfqpoint{2.410150in}{3.077050in}}%
\pgfpathlineto{\pgfqpoint{2.410628in}{3.154349in}}%
\pgfpathlineto{\pgfqpoint{2.411059in}{3.069121in}}%
\pgfpathlineto{\pgfqpoint{2.410736in}{3.330251in}}%
\pgfpathlineto{\pgfqpoint{2.411707in}{3.126953in}}%
\pgfpathlineto{\pgfqpoint{2.412786in}{3.350544in}}%
\pgfpathlineto{\pgfqpoint{2.412724in}{3.068303in}}%
\pgfpathlineto{\pgfqpoint{2.412817in}{3.274351in}}%
\pgfpathlineto{\pgfqpoint{2.413757in}{3.058244in}}%
\pgfpathlineto{\pgfqpoint{2.413310in}{3.338452in}}%
\pgfpathlineto{\pgfqpoint{2.413926in}{3.182428in}}%
\pgfpathlineto{\pgfqpoint{2.414867in}{3.308961in}}%
\pgfpathlineto{\pgfqpoint{2.414928in}{3.068756in}}%
\pgfpathlineto{\pgfqpoint{2.415021in}{3.194698in}}%
\pgfpathlineto{\pgfqpoint{2.415807in}{3.060319in}}%
\pgfpathlineto{\pgfqpoint{2.415884in}{3.332818in}}%
\pgfpathlineto{\pgfqpoint{2.416131in}{3.142688in}}%
\pgfpathlineto{\pgfqpoint{2.416917in}{3.350232in}}%
\pgfpathlineto{\pgfqpoint{2.416208in}{3.078192in}}%
\pgfpathlineto{\pgfqpoint{2.416963in}{3.113123in}}%
\pgfpathlineto{\pgfqpoint{2.417873in}{3.069384in}}%
\pgfpathlineto{\pgfqpoint{2.417950in}{3.348204in}}%
\pgfpathlineto{\pgfqpoint{2.418042in}{3.127555in}}%
\pgfpathlineto{\pgfqpoint{2.418458in}{3.301522in}}%
\pgfpathlineto{\pgfqpoint{2.418782in}{3.062067in}}%
\pgfpathlineto{\pgfqpoint{2.419152in}{3.143907in}}%
\pgfpathlineto{\pgfqpoint{2.419923in}{3.059022in}}%
\pgfpathlineto{\pgfqpoint{2.420000in}{3.341389in}}%
\pgfpathlineto{\pgfqpoint{2.420277in}{3.096200in}}%
\pgfpathlineto{\pgfqpoint{2.420509in}{3.337681in}}%
\pgfpathlineto{\pgfqpoint{2.420447in}{3.055312in}}%
\pgfpathlineto{\pgfqpoint{2.421433in}{3.203121in}}%
\pgfpathlineto{\pgfqpoint{2.422497in}{3.065162in}}%
\pgfpathlineto{\pgfqpoint{2.422065in}{3.358463in}}%
\pgfpathlineto{\pgfqpoint{2.422528in}{3.184087in}}%
\pgfpathlineto{\pgfqpoint{2.423098in}{3.345728in}}%
\pgfpathlineto{\pgfqpoint{2.423407in}{3.059605in}}%
\pgfpathlineto{\pgfqpoint{2.423622in}{3.243928in}}%
\pgfpathlineto{\pgfqpoint{2.423931in}{3.063776in}}%
\pgfpathlineto{\pgfqpoint{2.424624in}{3.366501in}}%
\pgfpathlineto{\pgfqpoint{2.424732in}{3.176957in}}%
\pgfpathlineto{\pgfqpoint{2.425148in}{3.365439in}}%
\pgfpathlineto{\pgfqpoint{2.425595in}{3.047173in}}%
\pgfpathlineto{\pgfqpoint{2.425827in}{3.169030in}}%
\pgfpathlineto{\pgfqpoint{2.425981in}{3.082109in}}%
\pgfpathlineto{\pgfqpoint{2.426690in}{3.352942in}}%
\pgfpathlineto{\pgfqpoint{2.426936in}{3.129274in}}%
\pgfpathlineto{\pgfqpoint{2.427707in}{3.351429in}}%
\pgfpathlineto{\pgfqpoint{2.427646in}{3.040945in}}%
\pgfpathlineto{\pgfqpoint{2.428016in}{3.105960in}}%
\pgfpathlineto{\pgfqpoint{2.428555in}{3.042839in}}%
\pgfpathlineto{\pgfqpoint{2.428740in}{3.352171in}}%
\pgfpathlineto{\pgfqpoint{2.429095in}{3.146985in}}%
\pgfpathlineto{\pgfqpoint{2.429773in}{3.374353in}}%
\pgfpathlineto{\pgfqpoint{2.429850in}{3.073100in}}%
\pgfpathlineto{\pgfqpoint{2.430189in}{3.157957in}}%
\pgfpathlineto{\pgfqpoint{2.430605in}{3.037425in}}%
\pgfpathlineto{\pgfqpoint{2.430806in}{3.329539in}}%
\pgfpathlineto{\pgfqpoint{2.431268in}{3.170010in}}%
\pgfpathlineto{\pgfqpoint{2.431823in}{3.351710in}}%
\pgfpathlineto{\pgfqpoint{2.431761in}{3.043509in}}%
\pgfpathlineto{\pgfqpoint{2.432378in}{3.209049in}}%
\pgfpathlineto{\pgfqpoint{2.432794in}{3.053877in}}%
\pgfpathlineto{\pgfqpoint{2.432856in}{3.344655in}}%
\pgfpathlineto{\pgfqpoint{2.433472in}{3.192051in}}%
\pgfpathlineto{\pgfqpoint{2.433889in}{3.358145in}}%
\pgfpathlineto{\pgfqpoint{2.433704in}{3.063397in}}%
\pgfpathlineto{\pgfqpoint{2.434567in}{3.210330in}}%
\pgfpathlineto{\pgfqpoint{2.434721in}{3.037725in}}%
\pgfpathlineto{\pgfqpoint{2.434921in}{3.346257in}}%
\pgfpathlineto{\pgfqpoint{2.435692in}{3.153589in}}%
\pgfpathlineto{\pgfqpoint{2.435754in}{3.029606in}}%
\pgfpathlineto{\pgfqpoint{2.436463in}{3.348764in}}%
\pgfpathlineto{\pgfqpoint{2.436802in}{3.108018in}}%
\pgfpathlineto{\pgfqpoint{2.436987in}{3.354019in}}%
\pgfpathlineto{\pgfqpoint{2.437295in}{3.055191in}}%
\pgfpathlineto{\pgfqpoint{2.437912in}{3.154630in}}%
\pgfpathlineto{\pgfqpoint{2.438852in}{3.054035in}}%
\pgfpathlineto{\pgfqpoint{2.438513in}{3.356636in}}%
\pgfpathlineto{\pgfqpoint{2.438991in}{3.174418in}}%
\pgfpathlineto{\pgfqpoint{2.439037in}{3.355868in}}%
\pgfpathlineto{\pgfqpoint{2.439484in}{3.040139in}}%
\pgfpathlineto{\pgfqpoint{2.440101in}{3.205121in}}%
\pgfpathlineto{\pgfqpoint{2.440394in}{3.025586in}}%
\pgfpathlineto{\pgfqpoint{2.440579in}{3.376591in}}%
\pgfpathlineto{\pgfqpoint{2.441195in}{3.171164in}}%
\pgfpathlineto{\pgfqpoint{2.441611in}{3.372250in}}%
\pgfpathlineto{\pgfqpoint{2.441935in}{3.046096in}}%
\pgfpathlineto{\pgfqpoint{2.442290in}{3.195582in}}%
\pgfpathlineto{\pgfqpoint{2.442444in}{3.030885in}}%
\pgfpathlineto{\pgfqpoint{2.443153in}{3.339402in}}%
\pgfpathlineto{\pgfqpoint{2.443415in}{3.138035in}}%
\pgfpathlineto{\pgfqpoint{2.443477in}{3.051598in}}%
\pgfpathlineto{\pgfqpoint{2.443662in}{3.369904in}}%
\pgfpathlineto{\pgfqpoint{2.444525in}{3.118902in}}%
\pgfpathlineto{\pgfqpoint{2.445203in}{3.374132in}}%
\pgfpathlineto{\pgfqpoint{2.445542in}{3.028785in}}%
\pgfpathlineto{\pgfqpoint{2.445619in}{3.191950in}}%
\pgfpathlineto{\pgfqpoint{2.446560in}{3.035399in}}%
\pgfpathlineto{\pgfqpoint{2.445727in}{3.356005in}}%
\pgfpathlineto{\pgfqpoint{2.446714in}{3.190960in}}%
\pgfpathlineto{\pgfqpoint{2.446760in}{3.372109in}}%
\pgfpathlineto{\pgfqpoint{2.447592in}{3.037858in}}%
\pgfpathlineto{\pgfqpoint{2.447808in}{3.251881in}}%
\pgfpathlineto{\pgfqpoint{2.448101in}{3.054821in}}%
\pgfpathlineto{\pgfqpoint{2.448810in}{3.360848in}}%
\pgfpathlineto{\pgfqpoint{2.448918in}{3.197389in}}%
\pgfpathlineto{\pgfqpoint{2.449334in}{3.361327in}}%
\pgfpathlineto{\pgfqpoint{2.449134in}{3.045411in}}%
\pgfpathlineto{\pgfqpoint{2.450012in}{3.156335in}}%
\pgfpathlineto{\pgfqpoint{2.450675in}{3.042933in}}%
\pgfpathlineto{\pgfqpoint{2.450352in}{3.369967in}}%
\pgfpathlineto{\pgfqpoint{2.451107in}{3.186158in}}%
\pgfpathlineto{\pgfqpoint{2.451323in}{3.051628in}}%
\pgfpathlineto{\pgfqpoint{2.451893in}{3.364765in}}%
\pgfpathlineto{\pgfqpoint{2.452217in}{3.036908in}}%
\pgfpathlineto{\pgfqpoint{2.452417in}{3.367046in}}%
\pgfpathlineto{\pgfqpoint{2.453296in}{3.225207in}}%
\pgfpathlineto{\pgfqpoint{2.453959in}{3.360545in}}%
\pgfpathlineto{\pgfqpoint{2.454282in}{3.028953in}}%
\pgfpathlineto{\pgfqpoint{2.454359in}{3.210504in}}%
\pgfpathlineto{\pgfqpoint{2.454406in}{3.050458in}}%
\pgfpathlineto{\pgfqpoint{2.454483in}{3.339054in}}%
\pgfpathlineto{\pgfqpoint{2.455454in}{3.127804in}}%
\pgfpathlineto{\pgfqpoint{2.455500in}{3.363604in}}%
\pgfpathlineto{\pgfqpoint{2.455824in}{3.046624in}}%
\pgfpathlineto{\pgfqpoint{2.456564in}{3.196428in}}%
\pgfpathlineto{\pgfqpoint{2.457365in}{3.036482in}}%
\pgfpathlineto{\pgfqpoint{2.457566in}{3.369229in}}%
\pgfpathlineto{\pgfqpoint{2.457658in}{3.167194in}}%
\pgfpathlineto{\pgfqpoint{2.458599in}{3.365350in}}%
\pgfpathlineto{\pgfqpoint{2.458398in}{3.051683in}}%
\pgfpathlineto{\pgfqpoint{2.458753in}{3.205162in}}%
\pgfpathlineto{\pgfqpoint{2.459431in}{3.031965in}}%
\pgfpathlineto{\pgfqpoint{2.459107in}{3.351046in}}%
\pgfpathlineto{\pgfqpoint{2.459863in}{3.173870in}}%
\pgfpathlineto{\pgfqpoint{2.460464in}{3.054215in}}%
\pgfpathlineto{\pgfqpoint{2.460649in}{3.350141in}}%
\pgfpathlineto{\pgfqpoint{2.460988in}{3.089818in}}%
\pgfpathlineto{\pgfqpoint{2.461173in}{3.351757in}}%
\pgfpathlineto{\pgfqpoint{2.462005in}{3.048923in}}%
\pgfpathlineto{\pgfqpoint{2.462098in}{3.130214in}}%
\pgfpathlineto{\pgfqpoint{2.462514in}{3.046319in}}%
\pgfpathlineto{\pgfqpoint{2.462190in}{3.362318in}}%
\pgfpathlineto{\pgfqpoint{2.463177in}{3.120153in}}%
\pgfpathlineto{\pgfqpoint{2.463747in}{3.355138in}}%
\pgfpathlineto{\pgfqpoint{2.464055in}{3.036509in}}%
\pgfpathlineto{\pgfqpoint{2.464287in}{3.214050in}}%
\pgfpathlineto{\pgfqpoint{2.465088in}{3.057217in}}%
\pgfpathlineto{\pgfqpoint{2.465289in}{3.362263in}}%
\pgfpathlineto{\pgfqpoint{2.465396in}{3.198239in}}%
\pgfpathlineto{\pgfqpoint{2.465797in}{3.357568in}}%
\pgfpathlineto{\pgfqpoint{2.466121in}{3.039797in}}%
\pgfpathlineto{\pgfqpoint{2.466491in}{3.142060in}}%
\pgfpathlineto{\pgfqpoint{2.467154in}{3.048479in}}%
\pgfpathlineto{\pgfqpoint{2.467354in}{3.349207in}}%
\pgfpathlineto{\pgfqpoint{2.467601in}{3.142350in}}%
\pgfpathlineto{\pgfqpoint{2.467662in}{3.048251in}}%
\pgfpathlineto{\pgfqpoint{2.467832in}{3.278134in}}%
\pgfpathlineto{\pgfqpoint{2.468896in}{3.364551in}}%
\pgfpathlineto{\pgfqpoint{2.468695in}{3.047277in}}%
\pgfpathlineto{\pgfqpoint{2.468911in}{3.299187in}}%
\pgfpathlineto{\pgfqpoint{2.469219in}{3.038179in}}%
\pgfpathlineto{\pgfqpoint{2.469404in}{3.357584in}}%
\pgfpathlineto{\pgfqpoint{2.470021in}{3.156180in}}%
\pgfpathlineto{\pgfqpoint{2.470437in}{3.380129in}}%
\pgfpathlineto{\pgfqpoint{2.470761in}{3.042503in}}%
\pgfpathlineto{\pgfqpoint{2.471115in}{3.205086in}}%
\pgfpathlineto{\pgfqpoint{2.471270in}{3.036343in}}%
\pgfpathlineto{\pgfqpoint{2.471979in}{3.362305in}}%
\pgfpathlineto{\pgfqpoint{2.472225in}{3.156398in}}%
\pgfpathlineto{\pgfqpoint{2.472302in}{3.050883in}}%
\pgfpathlineto{\pgfqpoint{2.473011in}{3.368052in}}%
\pgfpathlineto{\pgfqpoint{2.473351in}{3.107465in}}%
\pgfpathlineto{\pgfqpoint{2.474044in}{3.378777in}}%
\pgfpathlineto{\pgfqpoint{2.474368in}{3.035717in}}%
\pgfpathlineto{\pgfqpoint{2.474460in}{3.110824in}}%
\pgfpathlineto{\pgfqpoint{2.475385in}{3.045720in}}%
\pgfpathlineto{\pgfqpoint{2.475062in}{3.357631in}}%
\pgfpathlineto{\pgfqpoint{2.475539in}{3.159395in}}%
\pgfpathlineto{\pgfqpoint{2.475586in}{3.370549in}}%
\pgfpathlineto{\pgfqpoint{2.475909in}{3.025106in}}%
\pgfpathlineto{\pgfqpoint{2.476649in}{3.192006in}}%
\pgfpathlineto{\pgfqpoint{2.476927in}{3.043655in}}%
\pgfpathlineto{\pgfqpoint{2.477127in}{3.367991in}}%
\pgfpathlineto{\pgfqpoint{2.477744in}{3.172856in}}%
\pgfpathlineto{\pgfqpoint{2.478669in}{3.357773in}}%
\pgfpathlineto{\pgfqpoint{2.477960in}{3.042482in}}%
\pgfpathlineto{\pgfqpoint{2.478838in}{3.167926in}}%
\pgfpathlineto{\pgfqpoint{2.478992in}{3.036701in}}%
\pgfpathlineto{\pgfqpoint{2.479701in}{3.365760in}}%
\pgfpathlineto{\pgfqpoint{2.479933in}{3.166082in}}%
\pgfpathlineto{\pgfqpoint{2.480734in}{3.372898in}}%
\pgfpathlineto{\pgfqpoint{2.480534in}{3.031690in}}%
\pgfpathlineto{\pgfqpoint{2.481012in}{3.135106in}}%
\pgfpathlineto{\pgfqpoint{2.481058in}{3.025924in}}%
\pgfpathlineto{\pgfqpoint{2.481243in}{3.358749in}}%
\pgfpathlineto{\pgfqpoint{2.482106in}{3.116256in}}%
\pgfpathlineto{\pgfqpoint{2.482276in}{3.367397in}}%
\pgfpathlineto{\pgfqpoint{2.483108in}{3.028536in}}%
\pgfpathlineto{\pgfqpoint{2.483201in}{3.200168in}}%
\pgfpathlineto{\pgfqpoint{2.483617in}{3.037552in}}%
\pgfpathlineto{\pgfqpoint{2.483817in}{3.362466in}}%
\pgfpathlineto{\pgfqpoint{2.484295in}{3.167071in}}%
\pgfpathlineto{\pgfqpoint{2.484850in}{3.374719in}}%
\pgfpathlineto{\pgfqpoint{2.485174in}{3.044462in}}%
\pgfpathlineto{\pgfqpoint{2.485405in}{3.205630in}}%
\pgfpathlineto{\pgfqpoint{2.485682in}{3.030950in}}%
\pgfpathlineto{\pgfqpoint{2.485883in}{3.377414in}}%
\pgfpathlineto{\pgfqpoint{2.486499in}{3.167990in}}%
\pgfpathlineto{\pgfqpoint{2.487424in}{3.373911in}}%
\pgfpathlineto{\pgfqpoint{2.486715in}{3.035968in}}%
\pgfpathlineto{\pgfqpoint{2.487594in}{3.179650in}}%
\pgfpathlineto{\pgfqpoint{2.487748in}{3.015846in}}%
\pgfpathlineto{\pgfqpoint{2.487933in}{3.359832in}}%
\pgfpathlineto{\pgfqpoint{2.488688in}{3.170167in}}%
\pgfpathlineto{\pgfqpoint{2.488966in}{3.369960in}}%
\pgfpathlineto{\pgfqpoint{2.488781in}{3.022436in}}%
\pgfpathlineto{\pgfqpoint{2.489767in}{3.131498in}}%
\pgfpathlineto{\pgfqpoint{2.490831in}{3.016266in}}%
\pgfpathlineto{\pgfqpoint{2.489999in}{3.362100in}}%
\pgfpathlineto{\pgfqpoint{2.490862in}{3.106042in}}%
\pgfpathlineto{\pgfqpoint{2.491031in}{3.373866in}}%
\pgfpathlineto{\pgfqpoint{2.491864in}{3.022399in}}%
\pgfpathlineto{\pgfqpoint{2.491956in}{3.192947in}}%
\pgfpathlineto{\pgfqpoint{2.492897in}{3.000229in}}%
\pgfpathlineto{\pgfqpoint{2.492573in}{3.385999in}}%
\pgfpathlineto{\pgfqpoint{2.493051in}{3.163393in}}%
\pgfpathlineto{\pgfqpoint{2.494114in}{3.387543in}}%
\pgfpathlineto{\pgfqpoint{2.493405in}{3.015033in}}%
\pgfpathlineto{\pgfqpoint{2.494161in}{3.188893in}}%
\pgfpathlineto{\pgfqpoint{2.494438in}{3.004116in}}%
\pgfpathlineto{\pgfqpoint{2.494638in}{3.377263in}}%
\pgfpathlineto{\pgfqpoint{2.495255in}{3.167968in}}%
\pgfpathlineto{\pgfqpoint{2.496180in}{3.385523in}}%
\pgfpathlineto{\pgfqpoint{2.495980in}{3.008650in}}%
\pgfpathlineto{\pgfqpoint{2.496349in}{3.156177in}}%
\pgfpathlineto{\pgfqpoint{2.497012in}{3.005743in}}%
\pgfpathlineto{\pgfqpoint{2.496689in}{3.393972in}}%
\pgfpathlineto{\pgfqpoint{2.497444in}{3.187626in}}%
\pgfpathlineto{\pgfqpoint{2.497721in}{3.395008in}}%
\pgfpathlineto{\pgfqpoint{2.498045in}{3.003022in}}%
\pgfpathlineto{\pgfqpoint{2.498415in}{3.113420in}}%
\pgfpathlineto{\pgfqpoint{2.499078in}{3.006593in}}%
\pgfpathlineto{\pgfqpoint{2.499263in}{3.395502in}}%
\pgfpathlineto{\pgfqpoint{2.499494in}{3.154403in}}%
\pgfpathlineto{\pgfqpoint{2.500296in}{3.384649in}}%
\pgfpathlineto{\pgfqpoint{2.499587in}{2.985941in}}%
\pgfpathlineto{\pgfqpoint{2.500573in}{3.172348in}}%
\pgfpathlineto{\pgfqpoint{2.501128in}{2.988710in}}%
\pgfpathlineto{\pgfqpoint{2.500804in}{3.395916in}}%
\pgfpathlineto{\pgfqpoint{2.501668in}{3.102597in}}%
\pgfpathlineto{\pgfqpoint{2.501837in}{3.396854in}}%
\pgfpathlineto{\pgfqpoint{2.502670in}{2.980164in}}%
\pgfpathlineto{\pgfqpoint{2.502762in}{3.171593in}}%
\pgfpathlineto{\pgfqpoint{2.503702in}{2.982979in}}%
\pgfpathlineto{\pgfqpoint{2.503379in}{3.404221in}}%
\pgfpathlineto{\pgfqpoint{2.503856in}{3.216460in}}%
\pgfpathlineto{\pgfqpoint{2.504411in}{3.404995in}}%
\pgfpathlineto{\pgfqpoint{2.504735in}{2.971279in}}%
\pgfpathlineto{\pgfqpoint{2.504951in}{3.267264in}}%
\pgfpathlineto{\pgfqpoint{2.505244in}{2.975792in}}%
\pgfpathlineto{\pgfqpoint{2.505953in}{3.408706in}}%
\pgfpathlineto{\pgfqpoint{2.506061in}{3.214092in}}%
\pgfpathlineto{\pgfqpoint{2.506462in}{3.410648in}}%
\pgfpathlineto{\pgfqpoint{2.506785in}{2.962119in}}%
\pgfpathlineto{\pgfqpoint{2.507140in}{3.174672in}}%
\pgfpathlineto{\pgfqpoint{2.507818in}{2.963046in}}%
\pgfpathlineto{\pgfqpoint{2.508003in}{3.423854in}}%
\pgfpathlineto{\pgfqpoint{2.508234in}{3.133663in}}%
\pgfpathlineto{\pgfqpoint{2.509036in}{3.424619in}}%
\pgfpathlineto{\pgfqpoint{2.508327in}{2.969762in}}%
\pgfpathlineto{\pgfqpoint{2.509329in}{3.115239in}}%
\pgfpathlineto{\pgfqpoint{2.509360in}{2.945035in}}%
\pgfpathlineto{\pgfqpoint{2.509545in}{3.428656in}}%
\pgfpathlineto{\pgfqpoint{2.510423in}{3.181308in}}%
\pgfpathlineto{\pgfqpoint{2.510577in}{3.434355in}}%
\pgfpathlineto{\pgfqpoint{2.510901in}{2.942047in}}%
\pgfpathlineto{\pgfqpoint{2.511502in}{3.193517in}}%
\pgfpathlineto{\pgfqpoint{2.511934in}{2.942695in}}%
\pgfpathlineto{\pgfqpoint{2.512119in}{3.431440in}}%
\pgfpathlineto{\pgfqpoint{2.512597in}{3.223138in}}%
\pgfpathlineto{\pgfqpoint{2.513660in}{3.438372in}}%
\pgfpathlineto{\pgfqpoint{2.513475in}{2.943214in}}%
\pgfpathlineto{\pgfqpoint{2.513691in}{3.294493in}}%
\pgfpathlineto{\pgfqpoint{2.514508in}{2.934245in}}%
\pgfpathlineto{\pgfqpoint{2.514693in}{3.453497in}}%
\pgfpathlineto{\pgfqpoint{2.514801in}{3.243919in}}%
\pgfpathlineto{\pgfqpoint{2.515202in}{3.455620in}}%
\pgfpathlineto{\pgfqpoint{2.515017in}{2.943024in}}%
\pgfpathlineto{\pgfqpoint{2.515880in}{3.187003in}}%
\pgfpathlineto{\pgfqpoint{2.516050in}{2.925653in}}%
\pgfpathlineto{\pgfqpoint{2.516235in}{3.466152in}}%
\pgfpathlineto{\pgfqpoint{2.516990in}{3.192364in}}%
\pgfpathlineto{\pgfqpoint{2.517267in}{3.471499in}}%
\pgfpathlineto{\pgfqpoint{2.517082in}{2.930201in}}%
\pgfpathlineto{\pgfqpoint{2.517961in}{3.033944in}}%
\pgfpathlineto{\pgfqpoint{2.518624in}{2.922976in}}%
\pgfpathlineto{\pgfqpoint{2.518809in}{3.463808in}}%
\pgfpathlineto{\pgfqpoint{2.519040in}{3.141665in}}%
\pgfpathlineto{\pgfqpoint{2.519842in}{3.467404in}}%
\pgfpathlineto{\pgfqpoint{2.519657in}{2.929900in}}%
\pgfpathlineto{\pgfqpoint{2.520135in}{3.124504in}}%
\pgfpathlineto{\pgfqpoint{2.521198in}{2.925356in}}%
\pgfpathlineto{\pgfqpoint{2.520350in}{3.453134in}}%
\pgfpathlineto{\pgfqpoint{2.521229in}{3.117332in}}%
\pgfpathlineto{\pgfqpoint{2.521383in}{3.451258in}}%
\pgfpathlineto{\pgfqpoint{2.522231in}{2.935664in}}%
\pgfpathlineto{\pgfqpoint{2.522323in}{3.081338in}}%
\pgfpathlineto{\pgfqpoint{2.523264in}{2.939012in}}%
\pgfpathlineto{\pgfqpoint{2.522416in}{3.439627in}}%
\pgfpathlineto{\pgfqpoint{2.523403in}{3.135306in}}%
\pgfpathlineto{\pgfqpoint{2.523449in}{3.408372in}}%
\pgfpathlineto{\pgfqpoint{2.523772in}{2.946049in}}%
\pgfpathlineto{\pgfqpoint{2.524512in}{3.245004in}}%
\pgfpathlineto{\pgfqpoint{2.524805in}{2.963355in}}%
\pgfpathlineto{\pgfqpoint{2.524990in}{3.385554in}}%
\pgfpathlineto{\pgfqpoint{2.525622in}{3.250418in}}%
\pgfpathlineto{\pgfqpoint{2.526023in}{3.360826in}}%
\pgfpathlineto{\pgfqpoint{2.525838in}{2.976505in}}%
\pgfpathlineto{\pgfqpoint{2.526670in}{3.242761in}}%
\pgfpathlineto{\pgfqpoint{2.527380in}{2.982130in}}%
\pgfpathlineto{\pgfqpoint{2.527056in}{3.353063in}}%
\pgfpathlineto{\pgfqpoint{2.527796in}{3.155751in}}%
\pgfpathlineto{\pgfqpoint{2.528089in}{3.346328in}}%
\pgfpathlineto{\pgfqpoint{2.528412in}{2.984782in}}%
\pgfpathlineto{\pgfqpoint{2.528890in}{3.145636in}}%
\pgfpathlineto{\pgfqpoint{2.528921in}{3.002571in}}%
\pgfpathlineto{\pgfqpoint{2.529121in}{3.332900in}}%
\pgfpathlineto{\pgfqpoint{2.529985in}{3.104207in}}%
\pgfpathlineto{\pgfqpoint{2.530154in}{3.326354in}}%
\pgfpathlineto{\pgfqpoint{2.530987in}{3.023435in}}%
\pgfpathlineto{\pgfqpoint{2.531095in}{3.150876in}}%
\pgfpathlineto{\pgfqpoint{2.531187in}{3.317635in}}%
\pgfpathlineto{\pgfqpoint{2.531495in}{3.037155in}}%
\pgfpathlineto{\pgfqpoint{2.532004in}{3.046206in}}%
\pgfpathlineto{\pgfqpoint{2.532528in}{3.038762in}}%
\pgfpathlineto{\pgfqpoint{2.532220in}{3.310224in}}%
\pgfpathlineto{\pgfqpoint{2.532698in}{3.230351in}}%
\pgfpathlineto{\pgfqpoint{2.532728in}{3.302947in}}%
\pgfpathlineto{\pgfqpoint{2.533037in}{3.041446in}}%
\pgfpathlineto{\pgfqpoint{2.533792in}{3.233283in}}%
\pgfpathlineto{\pgfqpoint{2.534070in}{3.048369in}}%
\pgfpathlineto{\pgfqpoint{2.534270in}{3.289570in}}%
\pgfpathlineto{\pgfqpoint{2.534979in}{3.109900in}}%
\pgfpathlineto{\pgfqpoint{2.535827in}{3.282426in}}%
\pgfpathlineto{\pgfqpoint{2.535102in}{3.058360in}}%
\pgfpathlineto{\pgfqpoint{2.536104in}{3.138022in}}%
\pgfpathlineto{\pgfqpoint{2.536135in}{3.070963in}}%
\pgfpathlineto{\pgfqpoint{2.536860in}{3.276744in}}%
\pgfpathlineto{\pgfqpoint{2.537199in}{3.124865in}}%
\pgfpathlineto{\pgfqpoint{2.537368in}{3.281537in}}%
\pgfpathlineto{\pgfqpoint{2.537677in}{3.083619in}}%
\pgfpathlineto{\pgfqpoint{2.538417in}{3.267534in}}%
\pgfpathlineto{\pgfqpoint{2.539218in}{3.091718in}}%
\pgfpathlineto{\pgfqpoint{2.539434in}{3.270871in}}%
\pgfpathlineto{\pgfqpoint{2.539650in}{3.101151in}}%
\pgfpathlineto{\pgfqpoint{2.539958in}{3.264629in}}%
\pgfpathlineto{\pgfqpoint{2.540251in}{3.092645in}}%
\pgfpathlineto{\pgfqpoint{2.540744in}{3.115907in}}%
\pgfpathlineto{\pgfqpoint{2.541191in}{3.088937in}}%
\pgfpathlineto{\pgfqpoint{2.540991in}{3.258827in}}%
\pgfpathlineto{\pgfqpoint{2.541823in}{3.129149in}}%
\pgfpathlineto{\pgfqpoint{2.542024in}{3.258620in}}%
\pgfpathlineto{\pgfqpoint{2.542224in}{3.090636in}}%
\pgfpathlineto{\pgfqpoint{2.543087in}{3.238610in}}%
\pgfpathlineto{\pgfqpoint{2.543241in}{3.087623in}}%
\pgfpathlineto{\pgfqpoint{2.543565in}{3.253958in}}%
\pgfpathlineto{\pgfqpoint{2.544398in}{3.130072in}}%
\pgfpathlineto{\pgfqpoint{2.544598in}{3.255202in}}%
\pgfpathlineto{\pgfqpoint{2.544798in}{3.096104in}}%
\pgfpathlineto{\pgfqpoint{2.545677in}{3.230140in}}%
\pgfpathlineto{\pgfqpoint{2.546340in}{3.090628in}}%
\pgfpathlineto{\pgfqpoint{2.546139in}{3.242954in}}%
\pgfpathlineto{\pgfqpoint{2.546864in}{3.102754in}}%
\pgfpathlineto{\pgfqpoint{2.547172in}{3.247340in}}%
\pgfpathlineto{\pgfqpoint{2.547881in}{3.094511in}}%
\pgfpathlineto{\pgfqpoint{2.547989in}{3.126544in}}%
\pgfpathlineto{\pgfqpoint{2.548205in}{3.244345in}}%
\pgfpathlineto{\pgfqpoint{2.548390in}{3.096613in}}%
\pgfpathlineto{\pgfqpoint{2.549222in}{3.240312in}}%
\pgfpathlineto{\pgfqpoint{2.549238in}{3.241027in}}%
\pgfpathlineto{\pgfqpoint{2.549299in}{3.211734in}}%
\pgfpathlineto{\pgfqpoint{2.549423in}{3.102854in}}%
\pgfpathlineto{\pgfqpoint{2.549746in}{3.241208in}}%
\pgfpathlineto{\pgfqpoint{2.550456in}{3.104621in}}%
\pgfpathlineto{\pgfqpoint{2.551288in}{3.239596in}}%
\pgfpathlineto{\pgfqpoint{2.551565in}{3.114253in}}%
\pgfpathlineto{\pgfqpoint{2.552321in}{3.237140in}}%
\pgfpathlineto{\pgfqpoint{2.552521in}{3.109179in}}%
\pgfpathlineto{\pgfqpoint{2.552984in}{3.178089in}}%
\pgfpathlineto{\pgfqpoint{2.554063in}{3.104113in}}%
\pgfpathlineto{\pgfqpoint{2.553878in}{3.236453in}}%
\pgfpathlineto{\pgfqpoint{2.554109in}{3.143502in}}%
\pgfpathlineto{\pgfqpoint{2.554571in}{3.110863in}}%
\pgfpathlineto{\pgfqpoint{2.554386in}{3.239101in}}%
\pgfpathlineto{\pgfqpoint{2.555219in}{3.129408in}}%
\pgfpathlineto{\pgfqpoint{2.556452in}{3.238343in}}%
\pgfpathlineto{\pgfqpoint{2.555604in}{3.109464in}}%
\pgfpathlineto{\pgfqpoint{2.556514in}{3.198571in}}%
\pgfpathlineto{\pgfqpoint{2.556637in}{3.111667in}}%
\pgfpathlineto{\pgfqpoint{2.556961in}{3.235884in}}%
\pgfpathlineto{\pgfqpoint{2.557670in}{3.124830in}}%
\pgfpathlineto{\pgfqpoint{2.557993in}{3.237176in}}%
\pgfpathlineto{\pgfqpoint{2.557747in}{3.116759in}}%
\pgfpathlineto{\pgfqpoint{2.558764in}{3.122817in}}%
\pgfpathlineto{\pgfqpoint{2.558780in}{3.117908in}}%
\pgfpathlineto{\pgfqpoint{2.559026in}{3.238066in}}%
\pgfpathlineto{\pgfqpoint{2.559519in}{3.219745in}}%
\pgfpathlineto{\pgfqpoint{2.559535in}{3.237842in}}%
\pgfpathlineto{\pgfqpoint{2.560336in}{3.118981in}}%
\pgfpathlineto{\pgfqpoint{2.560614in}{3.203174in}}%
\pgfpathlineto{\pgfqpoint{2.561354in}{3.118067in}}%
\pgfpathlineto{\pgfqpoint{2.561600in}{3.237069in}}%
\pgfpathlineto{\pgfqpoint{2.561909in}{3.126291in}}%
\pgfpathlineto{\pgfqpoint{2.562633in}{3.229139in}}%
\pgfpathlineto{\pgfqpoint{2.562911in}{3.124013in}}%
\pgfpathlineto{\pgfqpoint{2.563049in}{3.159532in}}%
\pgfpathlineto{\pgfqpoint{2.563142in}{3.230471in}}%
\pgfpathlineto{\pgfqpoint{2.563419in}{3.124331in}}%
\pgfpathlineto{\pgfqpoint{2.563913in}{3.137663in}}%
\pgfpathlineto{\pgfqpoint{2.564452in}{3.121897in}}%
\pgfpathlineto{\pgfqpoint{2.564699in}{3.228585in}}%
\pgfpathlineto{\pgfqpoint{2.565007in}{3.142991in}}%
\pgfpathlineto{\pgfqpoint{2.565732in}{3.230001in}}%
\pgfpathlineto{\pgfqpoint{2.565485in}{3.123760in}}%
\pgfpathlineto{\pgfqpoint{2.566132in}{3.160330in}}%
\pgfpathlineto{\pgfqpoint{2.567026in}{3.125697in}}%
\pgfpathlineto{\pgfqpoint{2.566764in}{3.228288in}}%
\pgfpathlineto{\pgfqpoint{2.567211in}{3.166322in}}%
\pgfpathlineto{\pgfqpoint{2.568306in}{3.223559in}}%
\pgfpathlineto{\pgfqpoint{2.568059in}{3.118698in}}%
\pgfpathlineto{\pgfqpoint{2.568352in}{3.194585in}}%
\pgfpathlineto{\pgfqpoint{2.569616in}{3.116996in}}%
\pgfpathlineto{\pgfqpoint{2.569339in}{3.216289in}}%
\pgfpathlineto{\pgfqpoint{2.569632in}{3.123281in}}%
\pgfpathlineto{\pgfqpoint{2.570371in}{3.214360in}}%
\pgfpathlineto{\pgfqpoint{2.570649in}{3.120463in}}%
\pgfpathlineto{\pgfqpoint{2.570772in}{3.154346in}}%
\pgfpathlineto{\pgfqpoint{2.571158in}{3.115406in}}%
\pgfpathlineto{\pgfqpoint{2.570880in}{3.218576in}}%
\pgfpathlineto{\pgfqpoint{2.571867in}{3.165203in}}%
\pgfpathlineto{\pgfqpoint{2.571913in}{3.218725in}}%
\pgfpathlineto{\pgfqpoint{2.572190in}{3.121475in}}%
\pgfpathlineto{\pgfqpoint{2.572992in}{3.199492in}}%
\pgfpathlineto{\pgfqpoint{2.573223in}{3.116479in}}%
\pgfpathlineto{\pgfqpoint{2.573470in}{3.221729in}}%
\pgfpathlineto{\pgfqpoint{2.574148in}{3.183202in}}%
\pgfpathlineto{\pgfqpoint{2.575011in}{3.208029in}}%
\pgfpathlineto{\pgfqpoint{2.574765in}{3.113427in}}%
\pgfpathlineto{\pgfqpoint{2.575196in}{3.183989in}}%
\pgfpathlineto{\pgfqpoint{2.576322in}{3.117632in}}%
\pgfpathlineto{\pgfqpoint{2.576060in}{3.212540in}}%
\pgfpathlineto{\pgfqpoint{2.576337in}{3.126890in}}%
\pgfpathlineto{\pgfqpoint{2.577092in}{3.212264in}}%
\pgfpathlineto{\pgfqpoint{2.577354in}{3.119197in}}%
\pgfpathlineto{\pgfqpoint{2.577478in}{3.154114in}}%
\pgfpathlineto{\pgfqpoint{2.578387in}{3.123494in}}%
\pgfpathlineto{\pgfqpoint{2.577601in}{3.210556in}}%
\pgfpathlineto{\pgfqpoint{2.578557in}{3.156486in}}%
\pgfpathlineto{\pgfqpoint{2.578634in}{3.212279in}}%
\pgfpathlineto{\pgfqpoint{2.578896in}{3.121310in}}%
\pgfpathlineto{\pgfqpoint{2.579713in}{3.203924in}}%
\pgfpathlineto{\pgfqpoint{2.579898in}{3.137943in}}%
\pgfpathlineto{\pgfqpoint{2.579929in}{3.121466in}}%
\pgfpathlineto{\pgfqpoint{2.580746in}{3.205521in}}%
\pgfpathlineto{\pgfqpoint{2.580992in}{3.140424in}}%
\pgfpathlineto{\pgfqpoint{2.581254in}{3.206257in}}%
\pgfpathlineto{\pgfqpoint{2.581994in}{3.125986in}}%
\pgfpathlineto{\pgfqpoint{2.582118in}{3.162247in}}%
\pgfpathlineto{\pgfqpoint{2.582503in}{3.123773in}}%
\pgfpathlineto{\pgfqpoint{2.582303in}{3.208128in}}%
\pgfpathlineto{\pgfqpoint{2.583197in}{3.163774in}}%
\pgfpathlineto{\pgfqpoint{2.583829in}{3.203864in}}%
\pgfpathlineto{\pgfqpoint{2.583551in}{3.122402in}}%
\pgfpathlineto{\pgfqpoint{2.584368in}{3.195685in}}%
\pgfpathlineto{\pgfqpoint{2.585601in}{3.126594in}}%
\pgfpathlineto{\pgfqpoint{2.585386in}{3.204876in}}%
\pgfpathlineto{\pgfqpoint{2.585617in}{3.132311in}}%
\pgfpathlineto{\pgfqpoint{2.586418in}{3.209589in}}%
\pgfpathlineto{\pgfqpoint{2.586773in}{3.167901in}}%
\pgfpathlineto{\pgfqpoint{2.587451in}{3.206243in}}%
\pgfpathlineto{\pgfqpoint{2.587143in}{3.134788in}}%
\pgfpathlineto{\pgfqpoint{2.587636in}{3.149523in}}%
\pgfpathlineto{\pgfqpoint{2.588176in}{3.134380in}}%
\pgfpathlineto{\pgfqpoint{2.588484in}{3.199502in}}%
\pgfpathlineto{\pgfqpoint{2.588731in}{3.145276in}}%
\pgfpathlineto{\pgfqpoint{2.588993in}{3.203400in}}%
\pgfpathlineto{\pgfqpoint{2.589208in}{3.138402in}}%
\pgfpathlineto{\pgfqpoint{2.589856in}{3.167508in}}%
\pgfpathlineto{\pgfqpoint{2.590750in}{3.138262in}}%
\pgfpathlineto{\pgfqpoint{2.590550in}{3.196659in}}%
\pgfpathlineto{\pgfqpoint{2.590842in}{3.181240in}}%
\pgfpathlineto{\pgfqpoint{2.591582in}{3.198007in}}%
\pgfpathlineto{\pgfqpoint{2.591274in}{3.144718in}}%
\pgfpathlineto{\pgfqpoint{2.591752in}{3.158309in}}%
\pgfpathlineto{\pgfqpoint{2.592291in}{3.143025in}}%
\pgfpathlineto{\pgfqpoint{2.592091in}{3.195613in}}%
\pgfpathlineto{\pgfqpoint{2.592846in}{3.152468in}}%
\pgfpathlineto{\pgfqpoint{2.593124in}{3.193501in}}%
\pgfpathlineto{\pgfqpoint{2.593324in}{3.141232in}}%
\pgfpathlineto{\pgfqpoint{2.593972in}{3.163739in}}%
\pgfpathlineto{\pgfqpoint{2.594357in}{3.145940in}}%
\pgfpathlineto{\pgfqpoint{2.594157in}{3.195087in}}%
\pgfpathlineto{\pgfqpoint{2.595051in}{3.171998in}}%
\pgfpathlineto{\pgfqpoint{2.595174in}{3.188035in}}%
\pgfpathlineto{\pgfqpoint{2.595390in}{3.147064in}}%
\pgfpathlineto{\pgfqpoint{2.596160in}{3.171643in}}%
\pgfpathlineto{\pgfqpoint{2.596222in}{3.187078in}}%
\pgfpathlineto{\pgfqpoint{2.596931in}{3.147844in}}%
\pgfpathlineto{\pgfqpoint{2.597301in}{3.177663in}}%
\pgfpathlineto{\pgfqpoint{2.597455in}{3.148640in}}%
\pgfpathlineto{\pgfqpoint{2.598288in}{3.186042in}}%
\pgfpathlineto{\pgfqpoint{2.598519in}{3.160613in}}%
\pgfpathlineto{\pgfqpoint{2.599336in}{3.188595in}}%
\pgfpathlineto{\pgfqpoint{2.599151in}{3.155395in}}%
\pgfpathlineto{\pgfqpoint{2.599629in}{3.166810in}}%
\pgfpathlineto{\pgfqpoint{2.600168in}{3.153698in}}%
\pgfpathlineto{\pgfqpoint{2.600631in}{3.188748in}}%
\pgfpathlineto{\pgfqpoint{2.600723in}{3.171874in}}%
\pgfpathlineto{\pgfqpoint{2.601664in}{3.187547in}}%
\pgfpathlineto{\pgfqpoint{2.601571in}{3.150583in}}%
\pgfpathlineto{\pgfqpoint{2.601833in}{3.179195in}}%
\pgfpathlineto{\pgfqpoint{2.602234in}{3.143237in}}%
\pgfpathlineto{\pgfqpoint{2.602681in}{3.191212in}}%
\pgfpathlineto{\pgfqpoint{2.602928in}{3.177840in}}%
\pgfpathlineto{\pgfqpoint{2.603852in}{3.195570in}}%
\pgfpathlineto{\pgfqpoint{2.603775in}{3.143004in}}%
\pgfpathlineto{\pgfqpoint{2.603991in}{3.182060in}}%
\pgfpathlineto{\pgfqpoint{2.604115in}{3.136166in}}%
\pgfpathlineto{\pgfqpoint{2.604500in}{3.216833in}}%
\pgfpathlineto{\pgfqpoint{2.605055in}{3.209651in}}%
\pgfpathlineto{\pgfqpoint{2.605086in}{3.213850in}}%
\pgfpathlineto{\pgfqpoint{2.605671in}{3.137426in}}%
\pgfpathlineto{\pgfqpoint{2.606041in}{3.196306in}}%
\pgfpathlineto{\pgfqpoint{2.606982in}{3.229023in}}%
\pgfpathlineto{\pgfqpoint{2.607213in}{3.111566in}}%
\pgfpathlineto{\pgfqpoint{2.607567in}{3.215083in}}%
\pgfpathlineto{\pgfqpoint{2.608354in}{3.159268in}}%
\pgfpathlineto{\pgfqpoint{2.608415in}{3.267051in}}%
\pgfpathlineto{\pgfqpoint{2.608770in}{3.058608in}}%
\pgfpathlineto{\pgfqpoint{2.609494in}{3.194876in}}%
\pgfpathlineto{\pgfqpoint{2.610311in}{3.023549in}}%
\pgfpathlineto{\pgfqpoint{2.610234in}{3.301575in}}%
\pgfpathlineto{\pgfqpoint{2.610589in}{3.175471in}}%
\pgfpathlineto{\pgfqpoint{2.611406in}{3.331108in}}%
\pgfpathlineto{\pgfqpoint{2.611205in}{3.068429in}}%
\pgfpathlineto{\pgfqpoint{2.611699in}{3.189256in}}%
\pgfpathlineto{\pgfqpoint{2.611853in}{2.967397in}}%
\pgfpathlineto{\pgfqpoint{2.612639in}{3.268128in}}%
\pgfpathlineto{\pgfqpoint{2.612778in}{3.243738in}}%
\pgfpathlineto{\pgfqpoint{2.612824in}{3.300192in}}%
\pgfpathlineto{\pgfqpoint{2.613610in}{3.036116in}}%
\pgfpathlineto{\pgfqpoint{2.613841in}{3.218290in}}%
\pgfpathlineto{\pgfqpoint{2.614427in}{2.856749in}}%
\pgfpathlineto{\pgfqpoint{2.613965in}{3.354060in}}%
\pgfpathlineto{\pgfqpoint{2.614967in}{3.194967in}}%
\pgfpathlineto{\pgfqpoint{2.616046in}{3.306135in}}%
\pgfpathlineto{\pgfqpoint{2.615676in}{3.004456in}}%
\pgfpathlineto{\pgfqpoint{2.616076in}{3.246996in}}%
\pgfpathlineto{\pgfqpoint{2.616878in}{3.027835in}}%
\pgfpathlineto{\pgfqpoint{2.616416in}{3.268168in}}%
\pgfpathlineto{\pgfqpoint{2.617186in}{3.213973in}}%
\pgfpathlineto{\pgfqpoint{2.617279in}{3.251866in}}%
\pgfpathlineto{\pgfqpoint{2.617340in}{3.142372in}}%
\pgfpathlineto{\pgfqpoint{2.618173in}{2.912940in}}%
\pgfpathlineto{\pgfqpoint{2.617818in}{3.396859in}}%
\pgfpathlineto{\pgfqpoint{2.618420in}{3.150767in}}%
\pgfpathlineto{\pgfqpoint{2.618481in}{3.321410in}}%
\pgfpathlineto{\pgfqpoint{2.619452in}{3.017833in}}%
\pgfpathlineto{\pgfqpoint{2.619514in}{3.114660in}}%
\pgfpathlineto{\pgfqpoint{2.620223in}{3.349675in}}%
\pgfpathlineto{\pgfqpoint{2.619976in}{3.039429in}}%
\pgfpathlineto{\pgfqpoint{2.620578in}{3.063977in}}%
\pgfpathlineto{\pgfqpoint{2.620732in}{2.850142in}}%
\pgfpathlineto{\pgfqpoint{2.621456in}{3.400741in}}%
\pgfpathlineto{\pgfqpoint{2.621580in}{3.282799in}}%
\pgfpathlineto{\pgfqpoint{2.622659in}{3.345642in}}%
\pgfpathlineto{\pgfqpoint{2.621950in}{2.834369in}}%
\pgfpathlineto{\pgfqpoint{2.622736in}{3.321708in}}%
\pgfpathlineto{\pgfqpoint{2.623106in}{2.932662in}}%
\pgfpathlineto{\pgfqpoint{2.623383in}{3.324378in}}%
\pgfpathlineto{\pgfqpoint{2.623861in}{3.192333in}}%
\pgfpathlineto{\pgfqpoint{2.624031in}{3.517562in}}%
\pgfpathlineto{\pgfqpoint{2.624370in}{2.736209in}}%
\pgfpathlineto{\pgfqpoint{2.624986in}{3.302865in}}%
\pgfpathlineto{\pgfqpoint{2.625048in}{3.443465in}}%
\pgfpathlineto{\pgfqpoint{2.625603in}{2.844963in}}%
\pgfpathlineto{\pgfqpoint{2.626004in}{3.148843in}}%
\pgfpathlineto{\pgfqpoint{2.626975in}{2.802001in}}%
\pgfpathlineto{\pgfqpoint{2.626543in}{3.522519in}}%
\pgfpathlineto{\pgfqpoint{2.627113in}{3.147596in}}%
\pgfpathlineto{\pgfqpoint{2.627129in}{3.147607in}}%
\pgfpathlineto{\pgfqpoint{2.628100in}{2.691358in}}%
\pgfpathlineto{\pgfqpoint{2.627668in}{3.518972in}}%
\pgfpathlineto{\pgfqpoint{2.628239in}{3.018527in}}%
\pgfpathlineto{\pgfqpoint{2.629025in}{3.407528in}}%
\pgfpathlineto{\pgfqpoint{2.629318in}{2.856732in}}%
\pgfpathlineto{\pgfqpoint{2.629333in}{2.847285in}}%
\pgfpathlineto{\pgfqpoint{2.629549in}{3.241605in}}%
\pgfpathlineto{\pgfqpoint{2.629950in}{3.114207in}}%
\pgfpathlineto{\pgfqpoint{2.630119in}{3.638125in}}%
\pgfpathlineto{\pgfqpoint{2.630597in}{2.658862in}}%
\pgfpathlineto{\pgfqpoint{2.631106in}{3.251734in}}%
\pgfpathlineto{\pgfqpoint{2.632015in}{3.401291in}}%
\pgfpathlineto{\pgfqpoint{2.631707in}{2.770150in}}%
\pgfpathlineto{\pgfqpoint{2.632139in}{3.116159in}}%
\pgfpathlineto{\pgfqpoint{2.632355in}{2.915964in}}%
\pgfpathlineto{\pgfqpoint{2.632632in}{3.617667in}}%
\pgfpathlineto{\pgfqpoint{2.632694in}{3.616052in}}%
\pgfpathlineto{\pgfqpoint{2.632724in}{3.682120in}}%
\pgfpathlineto{\pgfqpoint{2.633017in}{2.681933in}}%
\pgfpathlineto{\pgfqpoint{2.633788in}{3.615343in}}%
\pgfpathlineto{\pgfqpoint{2.634204in}{2.624614in}}%
\pgfpathlineto{\pgfqpoint{2.635083in}{3.321989in}}%
\pgfpathlineto{\pgfqpoint{2.635468in}{2.831803in}}%
\pgfpathlineto{\pgfqpoint{2.636239in}{3.602267in}}%
\pgfpathlineto{\pgfqpoint{2.636671in}{2.702668in}}%
\pgfpathlineto{\pgfqpoint{2.636717in}{2.527414in}}%
\pgfpathlineto{\pgfqpoint{2.637518in}{3.422832in}}%
\pgfpathlineto{\pgfqpoint{2.637703in}{3.251854in}}%
\pgfpathlineto{\pgfqpoint{2.638798in}{3.674293in}}%
\pgfpathlineto{\pgfqpoint{2.638459in}{2.837219in}}%
\pgfpathlineto{\pgfqpoint{2.638906in}{3.636931in}}%
\pgfpathlineto{\pgfqpoint{2.639168in}{2.608689in}}%
\pgfpathlineto{\pgfqpoint{2.640093in}{3.333433in}}%
\pgfpathlineto{\pgfqpoint{2.640632in}{3.428688in}}%
\pgfpathlineto{\pgfqpoint{2.640401in}{2.672340in}}%
\pgfpathlineto{\pgfqpoint{2.641033in}{2.984843in}}%
\pgfpathlineto{\pgfqpoint{2.641095in}{3.125621in}}%
\pgfpathlineto{\pgfqpoint{2.641388in}{3.645903in}}%
\pgfpathlineto{\pgfqpoint{2.641711in}{2.787912in}}%
\pgfpathlineto{\pgfqpoint{2.642235in}{3.246903in}}%
\pgfpathlineto{\pgfqpoint{2.642390in}{3.583544in}}%
\pgfpathlineto{\pgfqpoint{2.642867in}{2.521687in}}%
\pgfpathlineto{\pgfqpoint{2.643314in}{3.201722in}}%
\pgfpathlineto{\pgfqpoint{2.643823in}{3.377679in}}%
\pgfpathlineto{\pgfqpoint{2.643407in}{3.062545in}}%
\pgfpathlineto{\pgfqpoint{2.644024in}{3.158648in}}%
\pgfpathlineto{\pgfqpoint{2.644594in}{2.801887in}}%
\pgfpathlineto{\pgfqpoint{2.644887in}{3.659934in}}%
\pgfpathlineto{\pgfqpoint{2.645056in}{3.509874in}}%
\pgfpathlineto{\pgfqpoint{2.645103in}{3.516004in}}%
\pgfpathlineto{\pgfqpoint{2.645257in}{2.959900in}}%
\pgfpathlineto{\pgfqpoint{2.645365in}{2.635043in}}%
\pgfpathlineto{\pgfqpoint{2.645781in}{3.443452in}}%
\pgfpathlineto{\pgfqpoint{2.646351in}{3.028112in}}%
\pgfpathlineto{\pgfqpoint{2.646505in}{2.768837in}}%
\pgfpathlineto{\pgfqpoint{2.646814in}{3.398211in}}%
\pgfpathlineto{\pgfqpoint{2.647307in}{3.363405in}}%
\pgfpathlineto{\pgfqpoint{2.647492in}{3.672722in}}%
\pgfpathlineto{\pgfqpoint{2.648062in}{2.819723in}}%
\pgfpathlineto{\pgfqpoint{2.648401in}{3.338007in}}%
\pgfpathlineto{\pgfqpoint{2.648956in}{2.625567in}}%
\pgfpathlineto{\pgfqpoint{2.648602in}{3.500747in}}%
\pgfpathlineto{\pgfqpoint{2.649650in}{3.134527in}}%
\pgfpathlineto{\pgfqpoint{2.650020in}{3.489095in}}%
\pgfpathlineto{\pgfqpoint{2.650359in}{2.869489in}}%
\pgfpathlineto{\pgfqpoint{2.650744in}{3.043326in}}%
\pgfpathlineto{\pgfqpoint{2.651053in}{3.590217in}}%
\pgfpathlineto{\pgfqpoint{2.651515in}{2.629784in}}%
\pgfpathlineto{\pgfqpoint{2.651885in}{3.300947in}}%
\pgfpathlineto{\pgfqpoint{2.652394in}{2.950057in}}%
\pgfpathlineto{\pgfqpoint{2.652255in}{3.355569in}}%
\pgfpathlineto{\pgfqpoint{2.653026in}{3.132317in}}%
\pgfpathlineto{\pgfqpoint{2.653951in}{2.680558in}}%
\pgfpathlineto{\pgfqpoint{2.653550in}{3.685152in}}%
\pgfpathlineto{\pgfqpoint{2.654167in}{2.993048in}}%
\pgfpathlineto{\pgfqpoint{2.654444in}{3.419012in}}%
\pgfpathlineto{\pgfqpoint{2.655014in}{2.867965in}}%
\pgfpathlineto{\pgfqpoint{2.655338in}{3.310339in}}%
\pgfpathlineto{\pgfqpoint{2.656556in}{2.715547in}}%
\pgfpathlineto{\pgfqpoint{2.656093in}{3.560276in}}%
\pgfpathlineto{\pgfqpoint{2.656571in}{2.810952in}}%
\pgfpathlineto{\pgfqpoint{2.657157in}{3.432086in}}%
\pgfpathlineto{\pgfqpoint{2.657697in}{3.065068in}}%
\pgfpathlineto{\pgfqpoint{2.658806in}{3.370110in}}%
\pgfpathlineto{\pgfqpoint{2.658097in}{2.903391in}}%
\pgfpathlineto{\pgfqpoint{2.658822in}{3.242933in}}%
\pgfpathlineto{\pgfqpoint{2.659361in}{2.705731in}}%
\pgfpathlineto{\pgfqpoint{2.659793in}{3.651943in}}%
\pgfpathlineto{\pgfqpoint{2.659916in}{3.349187in}}%
\pgfpathlineto{\pgfqpoint{2.659932in}{3.350372in}}%
\pgfpathlineto{\pgfqpoint{2.659962in}{3.291508in}}%
\pgfpathlineto{\pgfqpoint{2.660271in}{2.771068in}}%
\pgfpathlineto{\pgfqpoint{2.660641in}{3.432643in}}%
\pgfpathlineto{\pgfqpoint{2.661103in}{3.045042in}}%
\pgfpathlineto{\pgfqpoint{2.662213in}{3.639063in}}%
\pgfpathlineto{\pgfqpoint{2.661951in}{2.670868in}}%
\pgfpathlineto{\pgfqpoint{2.662275in}{3.545279in}}%
\pgfpathlineto{\pgfqpoint{2.662737in}{2.648120in}}%
\pgfpathlineto{\pgfqpoint{2.663523in}{3.033229in}}%
\pgfpathlineto{\pgfqpoint{2.664263in}{2.720955in}}%
\pgfpathlineto{\pgfqpoint{2.664679in}{3.380657in}}%
\pgfpathlineto{\pgfqpoint{2.664818in}{3.461001in}}%
\pgfpathlineto{\pgfqpoint{2.665419in}{2.845478in}}%
\pgfpathlineto{\pgfqpoint{2.665651in}{3.152450in}}%
\pgfpathlineto{\pgfqpoint{2.666437in}{2.785181in}}%
\pgfpathlineto{\pgfqpoint{2.665743in}{3.556304in}}%
\pgfpathlineto{\pgfqpoint{2.666745in}{3.202170in}}%
\pgfpathlineto{\pgfqpoint{2.667408in}{3.478331in}}%
\pgfpathlineto{\pgfqpoint{2.667762in}{2.951383in}}%
\pgfpathlineto{\pgfqpoint{2.667824in}{3.045837in}}%
\pgfpathlineto{\pgfqpoint{2.667994in}{2.631921in}}%
\pgfpathlineto{\pgfqpoint{2.668364in}{3.689956in}}%
\pgfpathlineto{\pgfqpoint{2.668934in}{2.942440in}}%
\pgfpathlineto{\pgfqpoint{2.670028in}{3.462412in}}%
\pgfpathlineto{\pgfqpoint{2.670075in}{3.157059in}}%
\pgfpathlineto{\pgfqpoint{2.670414in}{2.741317in}}%
\pgfpathlineto{\pgfqpoint{2.670892in}{3.411053in}}%
\pgfpathlineto{\pgfqpoint{2.671184in}{2.958260in}}%
\pgfpathlineto{\pgfqpoint{2.672001in}{3.423659in}}%
\pgfpathlineto{\pgfqpoint{2.671693in}{2.832103in}}%
\pgfpathlineto{\pgfqpoint{2.672310in}{3.220423in}}%
\pgfpathlineto{\pgfqpoint{2.672757in}{2.921392in}}%
\pgfpathlineto{\pgfqpoint{2.672526in}{3.316108in}}%
\pgfpathlineto{\pgfqpoint{2.673420in}{3.167818in}}%
\pgfpathlineto{\pgfqpoint{2.674437in}{3.551738in}}%
\pgfpathlineto{\pgfqpoint{2.674144in}{2.728501in}}%
\pgfpathlineto{\pgfqpoint{2.674622in}{3.403008in}}%
\pgfpathlineto{\pgfqpoint{2.675038in}{2.803410in}}%
\pgfpathlineto{\pgfqpoint{2.675747in}{3.225699in}}%
\pgfpathlineto{\pgfqpoint{2.676009in}{3.533050in}}%
\pgfpathlineto{\pgfqpoint{2.676595in}{2.709578in}}%
\pgfpathlineto{\pgfqpoint{2.676919in}{3.440485in}}%
\pgfpathlineto{\pgfqpoint{2.676934in}{3.447907in}}%
\pgfpathlineto{\pgfqpoint{2.677243in}{3.077529in}}%
\pgfpathlineto{\pgfqpoint{2.677335in}{3.172352in}}%
\pgfpathlineto{\pgfqpoint{2.677381in}{2.944572in}}%
\pgfpathlineto{\pgfqpoint{2.678337in}{3.414601in}}%
\pgfpathlineto{\pgfqpoint{2.678429in}{3.231718in}}%
\pgfpathlineto{\pgfqpoint{2.678476in}{3.361651in}}%
\pgfpathlineto{\pgfqpoint{2.678938in}{2.804295in}}%
\pgfpathlineto{\pgfqpoint{2.679539in}{3.246835in}}%
\pgfpathlineto{\pgfqpoint{2.680341in}{2.849836in}}%
\pgfpathlineto{\pgfqpoint{2.680572in}{3.484472in}}%
\pgfpathlineto{\pgfqpoint{2.680634in}{3.294705in}}%
\pgfpathlineto{\pgfqpoint{2.680695in}{3.490764in}}%
\pgfpathlineto{\pgfqpoint{2.681250in}{2.862563in}}%
\pgfpathlineto{\pgfqpoint{2.681697in}{3.140685in}}%
\pgfpathlineto{\pgfqpoint{2.682252in}{3.511840in}}%
\pgfpathlineto{\pgfqpoint{2.682037in}{2.967017in}}%
\pgfpathlineto{\pgfqpoint{2.682638in}{3.036355in}}%
\pgfpathlineto{\pgfqpoint{2.682807in}{2.826215in}}%
\pgfpathlineto{\pgfqpoint{2.683008in}{3.505180in}}%
\pgfpathlineto{\pgfqpoint{2.683732in}{3.075537in}}%
\pgfpathlineto{\pgfqpoint{2.684441in}{3.426422in}}%
\pgfpathlineto{\pgfqpoint{2.683963in}{2.942654in}}%
\pgfpathlineto{\pgfqpoint{2.684842in}{3.119625in}}%
\pgfpathlineto{\pgfqpoint{2.685351in}{3.363538in}}%
\pgfpathlineto{\pgfqpoint{2.685135in}{2.858782in}}%
\pgfpathlineto{\pgfqpoint{2.685906in}{3.050415in}}%
\pgfpathlineto{\pgfqpoint{2.686337in}{2.942688in}}%
\pgfpathlineto{\pgfqpoint{2.685983in}{3.495997in}}%
\pgfpathlineto{\pgfqpoint{2.686892in}{3.392177in}}%
\pgfpathlineto{\pgfqpoint{2.686908in}{3.428209in}}%
\pgfpathlineto{\pgfqpoint{2.687463in}{3.003421in}}%
\pgfpathlineto{\pgfqpoint{2.687910in}{3.201943in}}%
\pgfpathlineto{\pgfqpoint{2.687971in}{3.178716in}}%
\pgfpathlineto{\pgfqpoint{2.687987in}{3.232678in}}%
\pgfpathlineto{\pgfqpoint{2.689081in}{3.476583in}}%
\pgfpathlineto{\pgfqpoint{2.688896in}{2.876425in}}%
\pgfpathlineto{\pgfqpoint{2.689112in}{3.396160in}}%
\pgfpathlineto{\pgfqpoint{2.690191in}{2.860630in}}%
\pgfpathlineto{\pgfqpoint{2.689204in}{3.462535in}}%
\pgfpathlineto{\pgfqpoint{2.690237in}{3.141832in}}%
\pgfpathlineto{\pgfqpoint{2.690499in}{3.340828in}}%
\pgfpathlineto{\pgfqpoint{2.690299in}{3.003762in}}%
\pgfpathlineto{\pgfqpoint{2.690546in}{3.135333in}}%
\pgfpathlineto{\pgfqpoint{2.691347in}{2.792536in}}%
\pgfpathlineto{\pgfqpoint{2.690638in}{3.562015in}}%
\pgfpathlineto{\pgfqpoint{2.691625in}{3.325396in}}%
\pgfpathlineto{\pgfqpoint{2.692179in}{3.531154in}}%
\pgfpathlineto{\pgfqpoint{2.692627in}{2.942246in}}%
\pgfpathlineto{\pgfqpoint{2.692719in}{3.304531in}}%
\pgfpathlineto{\pgfqpoint{2.693659in}{2.814802in}}%
\pgfpathlineto{\pgfqpoint{2.693212in}{3.442965in}}%
\pgfpathlineto{\pgfqpoint{2.693829in}{3.165750in}}%
\pgfpathlineto{\pgfqpoint{2.694245in}{3.475851in}}%
\pgfpathlineto{\pgfqpoint{2.694445in}{2.874432in}}%
\pgfpathlineto{\pgfqpoint{2.694923in}{3.052219in}}%
\pgfpathlineto{\pgfqpoint{2.696002in}{2.801770in}}%
\pgfpathlineto{\pgfqpoint{2.695787in}{3.505053in}}%
\pgfpathlineto{\pgfqpoint{2.696033in}{3.024809in}}%
\pgfpathlineto{\pgfqpoint{2.696958in}{3.508790in}}%
\pgfpathlineto{\pgfqpoint{2.696758in}{2.935264in}}%
\pgfpathlineto{\pgfqpoint{2.697143in}{3.024388in}}%
\pgfpathlineto{\pgfqpoint{2.697852in}{3.435107in}}%
\pgfpathlineto{\pgfqpoint{2.697544in}{2.749000in}}%
\pgfpathlineto{\pgfqpoint{2.698284in}{3.099464in}}%
\pgfpathlineto{\pgfqpoint{2.699085in}{2.903265in}}%
\pgfpathlineto{\pgfqpoint{2.698376in}{3.472975in}}%
\pgfpathlineto{\pgfqpoint{2.699363in}{3.337408in}}%
\pgfpathlineto{\pgfqpoint{2.700426in}{3.437415in}}%
\pgfpathlineto{\pgfqpoint{2.699856in}{2.828535in}}%
\pgfpathlineto{\pgfqpoint{2.700457in}{3.311129in}}%
\pgfpathlineto{\pgfqpoint{2.701274in}{2.814742in}}%
\pgfpathlineto{\pgfqpoint{2.700935in}{3.416171in}}%
\pgfpathlineto{\pgfqpoint{2.701552in}{3.199876in}}%
\pgfpathlineto{\pgfqpoint{2.701968in}{3.504233in}}%
\pgfpathlineto{\pgfqpoint{2.702184in}{2.779121in}}%
\pgfpathlineto{\pgfqpoint{2.702646in}{3.240683in}}%
\pgfpathlineto{\pgfqpoint{2.703725in}{2.784142in}}%
\pgfpathlineto{\pgfqpoint{2.703139in}{3.492767in}}%
\pgfpathlineto{\pgfqpoint{2.703771in}{3.025590in}}%
\pgfpathlineto{\pgfqpoint{2.704558in}{3.491277in}}%
\pgfpathlineto{\pgfqpoint{2.704866in}{2.942079in}}%
\pgfpathlineto{\pgfqpoint{2.705575in}{3.465628in}}%
\pgfpathlineto{\pgfqpoint{2.705899in}{2.915216in}}%
\pgfpathlineto{\pgfqpoint{2.706037in}{2.885530in}}%
\pgfpathlineto{\pgfqpoint{2.705976in}{3.207915in}}%
\pgfpathlineto{\pgfqpoint{2.706068in}{3.149829in}}%
\pgfpathlineto{\pgfqpoint{2.707116in}{3.468111in}}%
\pgfpathlineto{\pgfqpoint{2.706824in}{2.910414in}}%
\pgfpathlineto{\pgfqpoint{2.707163in}{3.158898in}}%
\pgfpathlineto{\pgfqpoint{2.707456in}{2.846171in}}%
\pgfpathlineto{\pgfqpoint{2.707779in}{3.456814in}}%
\pgfpathlineto{\pgfqpoint{2.708257in}{3.176664in}}%
\pgfpathlineto{\pgfqpoint{2.709197in}{3.474004in}}%
\pgfpathlineto{\pgfqpoint{2.708365in}{2.850143in}}%
\pgfpathlineto{\pgfqpoint{2.709367in}{3.322873in}}%
\pgfpathlineto{\pgfqpoint{2.709907in}{2.790234in}}%
\pgfpathlineto{\pgfqpoint{2.710215in}{3.472717in}}%
\pgfpathlineto{\pgfqpoint{2.710492in}{3.217115in}}%
\pgfpathlineto{\pgfqpoint{2.711448in}{2.871773in}}%
\pgfpathlineto{\pgfqpoint{2.710862in}{3.421989in}}%
\pgfpathlineto{\pgfqpoint{2.711602in}{3.181462in}}%
\pgfpathlineto{\pgfqpoint{2.711756in}{3.518001in}}%
\pgfpathlineto{\pgfqpoint{2.712095in}{2.872394in}}%
\pgfpathlineto{\pgfqpoint{2.712697in}{3.110137in}}%
\pgfpathlineto{\pgfqpoint{2.713637in}{2.814108in}}%
\pgfpathlineto{\pgfqpoint{2.713313in}{3.452481in}}%
\pgfpathlineto{\pgfqpoint{2.713791in}{3.087978in}}%
\pgfpathlineto{\pgfqpoint{2.713961in}{3.511351in}}%
\pgfpathlineto{\pgfqpoint{2.714546in}{2.784937in}}%
\pgfpathlineto{\pgfqpoint{2.714901in}{3.093091in}}%
\pgfpathlineto{\pgfqpoint{2.715194in}{2.887780in}}%
\pgfpathlineto{\pgfqpoint{2.715518in}{3.530503in}}%
\pgfpathlineto{\pgfqpoint{2.715980in}{3.059062in}}%
\pgfpathlineto{\pgfqpoint{2.717059in}{3.438117in}}%
\pgfpathlineto{\pgfqpoint{2.716088in}{2.729130in}}%
\pgfpathlineto{\pgfqpoint{2.717105in}{3.196051in}}%
\pgfpathlineto{\pgfqpoint{2.717645in}{2.796596in}}%
\pgfpathlineto{\pgfqpoint{2.717938in}{3.465897in}}%
\pgfpathlineto{\pgfqpoint{2.718215in}{3.191942in}}%
\pgfpathlineto{\pgfqpoint{2.718970in}{3.412651in}}%
\pgfpathlineto{\pgfqpoint{2.719186in}{2.902825in}}%
\pgfpathlineto{\pgfqpoint{2.719294in}{3.176803in}}%
\pgfpathlineto{\pgfqpoint{2.719942in}{2.927059in}}%
\pgfpathlineto{\pgfqpoint{2.720157in}{3.465692in}}%
\pgfpathlineto{\pgfqpoint{2.720404in}{3.158779in}}%
\pgfpathlineto{\pgfqpoint{2.720743in}{2.832655in}}%
\pgfpathlineto{\pgfqpoint{2.720527in}{3.407898in}}%
\pgfpathlineto{\pgfqpoint{2.721514in}{3.114352in}}%
\pgfpathlineto{\pgfqpoint{2.721699in}{3.511386in}}%
\pgfpathlineto{\pgfqpoint{2.722285in}{2.754925in}}%
\pgfpathlineto{\pgfqpoint{2.722624in}{3.242625in}}%
\pgfpathlineto{\pgfqpoint{2.723687in}{2.980854in}}%
\pgfpathlineto{\pgfqpoint{2.723256in}{3.421708in}}%
\pgfpathlineto{\pgfqpoint{2.723734in}{3.200201in}}%
\pgfpathlineto{\pgfqpoint{2.723749in}{3.205754in}}%
\pgfpathlineto{\pgfqpoint{2.723811in}{2.877304in}}%
\pgfpathlineto{\pgfqpoint{2.723826in}{2.768833in}}%
\pgfpathlineto{\pgfqpoint{2.724134in}{3.476991in}}%
\pgfpathlineto{\pgfqpoint{2.724890in}{3.063119in}}%
\pgfpathlineto{\pgfqpoint{2.725167in}{3.473294in}}%
\pgfpathlineto{\pgfqpoint{2.725383in}{2.826130in}}%
\pgfpathlineto{\pgfqpoint{2.725984in}{3.073034in}}%
\pgfpathlineto{\pgfqpoint{2.726940in}{2.852408in}}%
\pgfpathlineto{\pgfqpoint{2.726724in}{3.437413in}}%
\pgfpathlineto{\pgfqpoint{2.727063in}{3.107124in}}%
\pgfpathlineto{\pgfqpoint{2.727865in}{3.435415in}}%
\pgfpathlineto{\pgfqpoint{2.727433in}{2.909059in}}%
\pgfpathlineto{\pgfqpoint{2.728173in}{3.116580in}}%
\pgfpathlineto{\pgfqpoint{2.728481in}{2.755240in}}%
\pgfpathlineto{\pgfqpoint{2.728728in}{3.411474in}}%
\pgfpathlineto{\pgfqpoint{2.729268in}{3.112949in}}%
\pgfpathlineto{\pgfqpoint{2.730270in}{3.437079in}}%
\pgfpathlineto{\pgfqpoint{2.730023in}{2.776330in}}%
\pgfpathlineto{\pgfqpoint{2.730377in}{3.172940in}}%
\pgfpathlineto{\pgfqpoint{2.730670in}{2.929163in}}%
\pgfpathlineto{\pgfqpoint{2.731364in}{3.420903in}}%
\pgfpathlineto{\pgfqpoint{2.731456in}{3.192879in}}%
\pgfpathlineto{\pgfqpoint{2.731857in}{3.407481in}}%
\pgfpathlineto{\pgfqpoint{2.731580in}{2.782212in}}%
\pgfpathlineto{\pgfqpoint{2.732566in}{3.282326in}}%
\pgfpathlineto{\pgfqpoint{2.733121in}{2.826525in}}%
\pgfpathlineto{\pgfqpoint{2.732905in}{3.422144in}}%
\pgfpathlineto{\pgfqpoint{2.733676in}{3.159021in}}%
\pgfpathlineto{\pgfqpoint{2.734062in}{3.393590in}}%
\pgfpathlineto{\pgfqpoint{2.734678in}{2.778887in}}%
\pgfpathlineto{\pgfqpoint{2.734771in}{3.133295in}}%
\pgfpathlineto{\pgfqpoint{2.734786in}{3.132654in}}%
\pgfpathlineto{\pgfqpoint{2.734801in}{3.151671in}}%
\pgfpathlineto{\pgfqpoint{2.735618in}{3.408805in}}%
\pgfpathlineto{\pgfqpoint{2.735187in}{2.950073in}}%
\pgfpathlineto{\pgfqpoint{2.735896in}{3.110755in}}%
\pgfpathlineto{\pgfqpoint{2.736220in}{2.828517in}}%
\pgfpathlineto{\pgfqpoint{2.736497in}{3.456029in}}%
\pgfpathlineto{\pgfqpoint{2.737006in}{3.069712in}}%
\pgfpathlineto{\pgfqpoint{2.738054in}{3.452160in}}%
\pgfpathlineto{\pgfqpoint{2.737777in}{2.829790in}}%
\pgfpathlineto{\pgfqpoint{2.738116in}{3.113475in}}%
\pgfpathlineto{\pgfqpoint{2.738285in}{2.941986in}}%
\pgfpathlineto{\pgfqpoint{2.738717in}{3.368104in}}%
\pgfpathlineto{\pgfqpoint{2.739071in}{3.332171in}}%
\pgfpathlineto{\pgfqpoint{2.739102in}{3.424357in}}%
\pgfpathlineto{\pgfqpoint{2.739318in}{2.830802in}}%
\pgfpathlineto{\pgfqpoint{2.740135in}{3.239581in}}%
\pgfpathlineto{\pgfqpoint{2.740875in}{2.840001in}}%
\pgfpathlineto{\pgfqpoint{2.740258in}{3.403817in}}%
\pgfpathlineto{\pgfqpoint{2.741137in}{3.386552in}}%
\pgfpathlineto{\pgfqpoint{2.741183in}{3.208213in}}%
\pgfpathlineto{\pgfqpoint{2.741384in}{2.976460in}}%
\pgfpathlineto{\pgfqpoint{2.741800in}{3.377279in}}%
\pgfpathlineto{\pgfqpoint{2.742293in}{3.128564in}}%
\pgfpathlineto{\pgfqpoint{2.742694in}{3.418594in}}%
\pgfpathlineto{\pgfqpoint{2.742416in}{2.894412in}}%
\pgfpathlineto{\pgfqpoint{2.743434in}{3.189074in}}%
\pgfpathlineto{\pgfqpoint{2.743973in}{2.931808in}}%
\pgfpathlineto{\pgfqpoint{2.744235in}{3.436528in}}%
\pgfpathlineto{\pgfqpoint{2.744605in}{2.978264in}}%
\pgfpathlineto{\pgfqpoint{2.745777in}{3.428224in}}%
\pgfpathlineto{\pgfqpoint{2.745515in}{2.881919in}}%
\pgfpathlineto{\pgfqpoint{2.745792in}{3.395986in}}%
\pgfpathlineto{\pgfqpoint{2.747072in}{2.903530in}}%
\pgfpathlineto{\pgfqpoint{2.747164in}{3.127968in}}%
\pgfpathlineto{\pgfqpoint{2.747318in}{3.407761in}}%
\pgfpathlineto{\pgfqpoint{2.747580in}{3.006443in}}%
\pgfpathlineto{\pgfqpoint{2.748274in}{3.194030in}}%
\pgfpathlineto{\pgfqpoint{2.748613in}{2.948598in}}%
\pgfpathlineto{\pgfqpoint{2.748875in}{3.474469in}}%
\pgfpathlineto{\pgfqpoint{2.749368in}{3.177292in}}%
\pgfpathlineto{\pgfqpoint{2.750417in}{3.463482in}}%
\pgfpathlineto{\pgfqpoint{2.750170in}{2.980392in}}%
\pgfpathlineto{\pgfqpoint{2.750478in}{3.194421in}}%
\pgfpathlineto{\pgfqpoint{2.750679in}{2.965154in}}%
\pgfpathlineto{\pgfqpoint{2.751080in}{3.398268in}}%
\pgfpathlineto{\pgfqpoint{2.751619in}{3.068217in}}%
\pgfpathlineto{\pgfqpoint{2.751712in}{2.917792in}}%
\pgfpathlineto{\pgfqpoint{2.751974in}{3.485602in}}%
\pgfpathlineto{\pgfqpoint{2.752621in}{3.378326in}}%
\pgfpathlineto{\pgfqpoint{2.753530in}{3.463936in}}%
\pgfpathlineto{\pgfqpoint{2.753268in}{2.942312in}}%
\pgfpathlineto{\pgfqpoint{2.753577in}{3.227276in}}%
\pgfpathlineto{\pgfqpoint{2.753777in}{2.996212in}}%
\pgfpathlineto{\pgfqpoint{2.754178in}{3.334189in}}%
\pgfpathlineto{\pgfqpoint{2.754733in}{3.031366in}}%
\pgfpathlineto{\pgfqpoint{2.754810in}{2.968362in}}%
\pgfpathlineto{\pgfqpoint{2.755041in}{3.407063in}}%
\pgfpathlineto{\pgfqpoint{2.755072in}{3.486893in}}%
\pgfpathlineto{\pgfqpoint{2.755334in}{2.973266in}}%
\pgfpathlineto{\pgfqpoint{2.756089in}{3.255641in}}%
\pgfpathlineto{\pgfqpoint{2.756382in}{3.008246in}}%
\pgfpathlineto{\pgfqpoint{2.756583in}{3.332135in}}%
\pgfpathlineto{\pgfqpoint{2.756629in}{3.436382in}}%
\pgfpathlineto{\pgfqpoint{2.756876in}{2.980297in}}%
\pgfpathlineto{\pgfqpoint{2.757662in}{3.243632in}}%
\pgfpathlineto{\pgfqpoint{2.757939in}{2.971365in}}%
\pgfpathlineto{\pgfqpoint{2.758170in}{3.482573in}}%
\pgfpathlineto{\pgfqpoint{2.758772in}{3.163608in}}%
\pgfpathlineto{\pgfqpoint{2.759727in}{3.472853in}}%
\pgfpathlineto{\pgfqpoint{2.759527in}{2.955156in}}%
\pgfpathlineto{\pgfqpoint{2.759881in}{3.187817in}}%
\pgfpathlineto{\pgfqpoint{2.759897in}{3.188245in}}%
\pgfpathlineto{\pgfqpoint{2.759912in}{3.180409in}}%
\pgfpathlineto{\pgfqpoint{2.761068in}{2.981731in}}%
\pgfpathlineto{\pgfqpoint{2.760359in}{3.334010in}}%
\pgfpathlineto{\pgfqpoint{2.761084in}{3.002644in}}%
\pgfpathlineto{\pgfqpoint{2.761269in}{3.444346in}}%
\pgfpathlineto{\pgfqpoint{2.762209in}{3.136121in}}%
\pgfpathlineto{\pgfqpoint{2.762826in}{3.439923in}}%
\pgfpathlineto{\pgfqpoint{2.762625in}{2.959497in}}%
\pgfpathlineto{\pgfqpoint{2.763226in}{3.061930in}}%
\pgfpathlineto{\pgfqpoint{2.764167in}{2.975323in}}%
\pgfpathlineto{\pgfqpoint{2.763458in}{3.379062in}}%
\pgfpathlineto{\pgfqpoint{2.764275in}{3.270781in}}%
\pgfpathlineto{\pgfqpoint{2.764814in}{3.007099in}}%
\pgfpathlineto{\pgfqpoint{2.764367in}{3.449737in}}%
\pgfpathlineto{\pgfqpoint{2.765369in}{3.273293in}}%
\pgfpathlineto{\pgfqpoint{2.765924in}{3.459481in}}%
\pgfpathlineto{\pgfqpoint{2.765724in}{2.927153in}}%
\pgfpathlineto{\pgfqpoint{2.766217in}{3.055261in}}%
\pgfpathlineto{\pgfqpoint{2.767265in}{2.942809in}}%
\pgfpathlineto{\pgfqpoint{2.766556in}{3.344749in}}%
\pgfpathlineto{\pgfqpoint{2.767296in}{3.070498in}}%
\pgfpathlineto{\pgfqpoint{2.767466in}{3.426800in}}%
\pgfpathlineto{\pgfqpoint{2.767913in}{3.016768in}}%
\pgfpathlineto{\pgfqpoint{2.768406in}{3.142939in}}%
\pgfpathlineto{\pgfqpoint{2.769022in}{3.415163in}}%
\pgfpathlineto{\pgfqpoint{2.768822in}{2.911580in}}%
\pgfpathlineto{\pgfqpoint{2.769423in}{3.109619in}}%
\pgfpathlineto{\pgfqpoint{2.770363in}{2.976256in}}%
\pgfpathlineto{\pgfqpoint{2.769670in}{3.376681in}}%
\pgfpathlineto{\pgfqpoint{2.770487in}{3.266773in}}%
\pgfpathlineto{\pgfqpoint{2.771150in}{3.002982in}}%
\pgfpathlineto{\pgfqpoint{2.770564in}{3.438302in}}%
\pgfpathlineto{\pgfqpoint{2.771581in}{3.300296in}}%
\pgfpathlineto{\pgfqpoint{2.771920in}{2.924507in}}%
\pgfpathlineto{\pgfqpoint{2.772121in}{3.446393in}}%
\pgfpathlineto{\pgfqpoint{2.772737in}{3.231759in}}%
\pgfpathlineto{\pgfqpoint{2.773662in}{3.392526in}}%
\pgfpathlineto{\pgfqpoint{2.773477in}{2.927772in}}%
\pgfpathlineto{\pgfqpoint{2.773832in}{3.166544in}}%
\pgfpathlineto{\pgfqpoint{2.775019in}{2.879623in}}%
\pgfpathlineto{\pgfqpoint{2.774325in}{3.369045in}}%
\pgfpathlineto{\pgfqpoint{2.775034in}{2.915809in}}%
\pgfpathlineto{\pgfqpoint{2.775219in}{3.423101in}}%
\pgfpathlineto{\pgfqpoint{2.776159in}{3.119762in}}%
\pgfpathlineto{\pgfqpoint{2.776576in}{2.918412in}}%
\pgfpathlineto{\pgfqpoint{2.776776in}{3.403442in}}%
\pgfpathlineto{\pgfqpoint{2.777239in}{3.140395in}}%
\pgfpathlineto{\pgfqpoint{2.778318in}{3.418470in}}%
\pgfpathlineto{\pgfqpoint{2.778117in}{2.934292in}}%
\pgfpathlineto{\pgfqpoint{2.778364in}{3.280281in}}%
\pgfpathlineto{\pgfqpoint{2.778641in}{2.941786in}}%
\pgfpathlineto{\pgfqpoint{2.778965in}{3.365286in}}%
\pgfpathlineto{\pgfqpoint{2.779474in}{3.271572in}}%
\pgfpathlineto{\pgfqpoint{2.779674in}{2.900853in}}%
\pgfpathlineto{\pgfqpoint{2.779890in}{3.406959in}}%
\pgfpathlineto{\pgfqpoint{2.780614in}{3.185605in}}%
\pgfpathlineto{\pgfqpoint{2.781431in}{3.427994in}}%
\pgfpathlineto{\pgfqpoint{2.781231in}{2.906512in}}%
\pgfpathlineto{\pgfqpoint{2.781678in}{3.130538in}}%
\pgfpathlineto{\pgfqpoint{2.782772in}{2.923902in}}%
\pgfpathlineto{\pgfqpoint{2.782079in}{3.333575in}}%
\pgfpathlineto{\pgfqpoint{2.782788in}{2.987847in}}%
\pgfpathlineto{\pgfqpoint{2.782988in}{3.408319in}}%
\pgfpathlineto{\pgfqpoint{2.783559in}{2.970172in}}%
\pgfpathlineto{\pgfqpoint{2.783913in}{3.161983in}}%
\pgfpathlineto{\pgfqpoint{2.784329in}{2.952983in}}%
\pgfpathlineto{\pgfqpoint{2.784530in}{3.403870in}}%
\pgfpathlineto{\pgfqpoint{2.784992in}{3.178806in}}%
\pgfpathlineto{\pgfqpoint{2.786087in}{3.390587in}}%
\pgfpathlineto{\pgfqpoint{2.785871in}{2.928332in}}%
\pgfpathlineto{\pgfqpoint{2.786102in}{3.318944in}}%
\pgfpathlineto{\pgfqpoint{2.786380in}{2.979377in}}%
\pgfpathlineto{\pgfqpoint{2.786719in}{3.352720in}}%
\pgfpathlineto{\pgfqpoint{2.787227in}{3.207845in}}%
\pgfpathlineto{\pgfqpoint{2.787304in}{2.946418in}}%
\pgfpathlineto{\pgfqpoint{2.787628in}{3.418432in}}%
\pgfpathlineto{\pgfqpoint{2.788337in}{3.168787in}}%
\pgfpathlineto{\pgfqpoint{2.788969in}{2.957627in}}%
\pgfpathlineto{\pgfqpoint{2.788645in}{3.273441in}}%
\pgfpathlineto{\pgfqpoint{2.789015in}{3.175471in}}%
\pgfpathlineto{\pgfqpoint{2.789185in}{3.399593in}}%
\pgfpathlineto{\pgfqpoint{2.789755in}{2.988210in}}%
\pgfpathlineto{\pgfqpoint{2.790125in}{3.176000in}}%
\pgfpathlineto{\pgfqpoint{2.790726in}{3.364551in}}%
\pgfpathlineto{\pgfqpoint{2.790526in}{2.964404in}}%
\pgfpathlineto{\pgfqpoint{2.791143in}{3.083652in}}%
\pgfpathlineto{\pgfqpoint{2.792068in}{2.981040in}}%
\pgfpathlineto{\pgfqpoint{2.791374in}{3.315124in}}%
\pgfpathlineto{\pgfqpoint{2.792191in}{3.202398in}}%
\pgfpathlineto{\pgfqpoint{2.792283in}{3.346409in}}%
\pgfpathlineto{\pgfqpoint{2.792592in}{2.993573in}}%
\pgfpathlineto{\pgfqpoint{2.793085in}{3.136417in}}%
\pgfpathlineto{\pgfqpoint{2.794133in}{2.981697in}}%
\pgfpathlineto{\pgfqpoint{2.793809in}{3.354508in}}%
\pgfpathlineto{\pgfqpoint{2.794179in}{3.156846in}}%
\pgfpathlineto{\pgfqpoint{2.795181in}{3.005645in}}%
\pgfpathlineto{\pgfqpoint{2.795351in}{3.352521in}}%
\pgfpathlineto{\pgfqpoint{2.795382in}{3.364699in}}%
\pgfpathlineto{\pgfqpoint{2.795659in}{3.060029in}}%
\pgfpathlineto{\pgfqpoint{2.795690in}{2.994326in}}%
\pgfpathlineto{\pgfqpoint{2.796029in}{3.302745in}}%
\pgfpathlineto{\pgfqpoint{2.796754in}{3.086955in}}%
\pgfpathlineto{\pgfqpoint{2.796939in}{3.329144in}}%
\pgfpathlineto{\pgfqpoint{2.797509in}{3.022670in}}%
\pgfpathlineto{\pgfqpoint{2.797879in}{3.183764in}}%
\pgfpathlineto{\pgfqpoint{2.798280in}{3.005434in}}%
\pgfpathlineto{\pgfqpoint{2.798480in}{3.320271in}}%
\pgfpathlineto{\pgfqpoint{2.798943in}{3.153527in}}%
\pgfpathlineto{\pgfqpoint{2.798989in}{3.336760in}}%
\pgfpathlineto{\pgfqpoint{2.799066in}{3.038482in}}%
\pgfpathlineto{\pgfqpoint{2.800068in}{3.229650in}}%
\pgfpathlineto{\pgfqpoint{2.800345in}{3.005375in}}%
\pgfpathlineto{\pgfqpoint{2.800546in}{3.329557in}}%
\pgfpathlineto{\pgfqpoint{2.801178in}{3.218951in}}%
\pgfpathlineto{\pgfqpoint{2.801579in}{3.372017in}}%
\pgfpathlineto{\pgfqpoint{2.801902in}{2.997627in}}%
\pgfpathlineto{\pgfqpoint{2.802149in}{3.118463in}}%
\pgfpathlineto{\pgfqpoint{2.802935in}{3.040077in}}%
\pgfpathlineto{\pgfqpoint{2.803151in}{3.345087in}}%
\pgfpathlineto{\pgfqpoint{2.803228in}{3.194758in}}%
\pgfpathlineto{\pgfqpoint{2.803660in}{3.314149in}}%
\pgfpathlineto{\pgfqpoint{2.803444in}{3.024281in}}%
\pgfpathlineto{\pgfqpoint{2.804322in}{3.159923in}}%
\pgfpathlineto{\pgfqpoint{2.805001in}{3.025136in}}%
\pgfpathlineto{\pgfqpoint{2.804692in}{3.338511in}}%
\pgfpathlineto{\pgfqpoint{2.805417in}{3.191647in}}%
\pgfpathlineto{\pgfqpoint{2.806558in}{3.021945in}}%
\pgfpathlineto{\pgfqpoint{2.806234in}{3.321064in}}%
\pgfpathlineto{\pgfqpoint{2.806588in}{3.114030in}}%
\pgfpathlineto{\pgfqpoint{2.807344in}{3.037932in}}%
\pgfpathlineto{\pgfqpoint{2.807775in}{3.344132in}}%
\pgfpathlineto{\pgfqpoint{2.807991in}{3.083927in}}%
\pgfpathlineto{\pgfqpoint{2.808022in}{3.089547in}}%
\pgfpathlineto{\pgfqpoint{2.808099in}{3.000490in}}%
\pgfpathlineto{\pgfqpoint{2.808315in}{3.321044in}}%
\pgfpathlineto{\pgfqpoint{2.809101in}{3.121691in}}%
\pgfpathlineto{\pgfqpoint{2.809147in}{3.060191in}}%
\pgfpathlineto{\pgfqpoint{2.809209in}{3.219678in}}%
\pgfpathlineto{\pgfqpoint{2.809332in}{3.354769in}}%
\pgfpathlineto{\pgfqpoint{2.809656in}{3.006452in}}%
\pgfpathlineto{\pgfqpoint{2.810288in}{3.173746in}}%
\pgfpathlineto{\pgfqpoint{2.811197in}{3.031420in}}%
\pgfpathlineto{\pgfqpoint{2.810889in}{3.357536in}}%
\pgfpathlineto{\pgfqpoint{2.811367in}{3.151585in}}%
\pgfpathlineto{\pgfqpoint{2.812431in}{3.353461in}}%
\pgfpathlineto{\pgfqpoint{2.811490in}{3.057654in}}%
\pgfpathlineto{\pgfqpoint{2.812477in}{3.249234in}}%
\pgfpathlineto{\pgfqpoint{2.812754in}{3.016109in}}%
\pgfpathlineto{\pgfqpoint{2.812970in}{3.332181in}}%
\pgfpathlineto{\pgfqpoint{2.813587in}{3.219904in}}%
\pgfpathlineto{\pgfqpoint{2.813987in}{3.349057in}}%
\pgfpathlineto{\pgfqpoint{2.814311in}{3.024740in}}%
\pgfpathlineto{\pgfqpoint{2.814697in}{3.217982in}}%
\pgfpathlineto{\pgfqpoint{2.815097in}{3.041191in}}%
\pgfpathlineto{\pgfqpoint{2.815514in}{3.322488in}}%
\pgfpathlineto{\pgfqpoint{2.815544in}{3.357167in}}%
\pgfpathlineto{\pgfqpoint{2.815868in}{3.025487in}}%
\pgfpathlineto{\pgfqpoint{2.816593in}{3.280511in}}%
\pgfpathlineto{\pgfqpoint{2.817410in}{3.033327in}}%
\pgfpathlineto{\pgfqpoint{2.817086in}{3.358527in}}%
\pgfpathlineto{\pgfqpoint{2.817718in}{3.162796in}}%
\pgfpathlineto{\pgfqpoint{2.818643in}{3.350503in}}%
\pgfpathlineto{\pgfqpoint{2.818365in}{3.053163in}}%
\pgfpathlineto{\pgfqpoint{2.818812in}{3.150688in}}%
\pgfpathlineto{\pgfqpoint{2.818966in}{3.027044in}}%
\pgfpathlineto{\pgfqpoint{2.819676in}{3.314184in}}%
\pgfpathlineto{\pgfqpoint{2.819953in}{3.109740in}}%
\pgfpathlineto{\pgfqpoint{2.820200in}{3.343877in}}%
\pgfpathlineto{\pgfqpoint{2.820523in}{3.046471in}}%
\pgfpathlineto{\pgfqpoint{2.821109in}{3.181700in}}%
\pgfpathlineto{\pgfqpoint{2.822065in}{3.047032in}}%
\pgfpathlineto{\pgfqpoint{2.821741in}{3.362943in}}%
\pgfpathlineto{\pgfqpoint{2.822219in}{3.128468in}}%
\pgfpathlineto{\pgfqpoint{2.823298in}{3.352693in}}%
\pgfpathlineto{\pgfqpoint{2.823021in}{3.063893in}}%
\pgfpathlineto{\pgfqpoint{2.823344in}{3.216628in}}%
\pgfpathlineto{\pgfqpoint{2.823622in}{3.050047in}}%
\pgfpathlineto{\pgfqpoint{2.824331in}{3.329867in}}%
\pgfpathlineto{\pgfqpoint{2.824454in}{3.214643in}}%
\pgfpathlineto{\pgfqpoint{2.824855in}{3.348240in}}%
\pgfpathlineto{\pgfqpoint{2.825179in}{3.041652in}}%
\pgfpathlineto{\pgfqpoint{2.825441in}{3.122731in}}%
\pgfpathlineto{\pgfqpoint{2.826134in}{3.061375in}}%
\pgfpathlineto{\pgfqpoint{2.826396in}{3.330559in}}%
\pgfpathlineto{\pgfqpoint{2.826489in}{3.173550in}}%
\pgfpathlineto{\pgfqpoint{2.827445in}{3.306817in}}%
\pgfpathlineto{\pgfqpoint{2.826736in}{3.059341in}}%
\pgfpathlineto{\pgfqpoint{2.827583in}{3.204442in}}%
\pgfpathlineto{\pgfqpoint{2.828277in}{3.046829in}}%
\pgfpathlineto{\pgfqpoint{2.827953in}{3.351950in}}%
\pgfpathlineto{\pgfqpoint{2.828724in}{3.130194in}}%
\pgfpathlineto{\pgfqpoint{2.829834in}{3.055077in}}%
\pgfpathlineto{\pgfqpoint{2.829510in}{3.332553in}}%
\pgfpathlineto{\pgfqpoint{2.829849in}{3.074793in}}%
\pgfpathlineto{\pgfqpoint{2.830543in}{3.332635in}}%
\pgfpathlineto{\pgfqpoint{2.830790in}{3.053344in}}%
\pgfpathlineto{\pgfqpoint{2.831052in}{3.329919in}}%
\pgfpathlineto{\pgfqpoint{2.832347in}{3.059779in}}%
\pgfpathlineto{\pgfqpoint{2.832485in}{3.152531in}}%
\pgfpathlineto{\pgfqpoint{2.832609in}{3.318737in}}%
\pgfpathlineto{\pgfqpoint{2.832932in}{3.072279in}}%
\pgfpathlineto{\pgfqpoint{2.833657in}{3.287025in}}%
\pgfpathlineto{\pgfqpoint{2.834489in}{3.055706in}}%
\pgfpathlineto{\pgfqpoint{2.834150in}{3.325492in}}%
\pgfpathlineto{\pgfqpoint{2.834797in}{3.144267in}}%
\pgfpathlineto{\pgfqpoint{2.835707in}{3.315737in}}%
\pgfpathlineto{\pgfqpoint{2.835445in}{3.062920in}}%
\pgfpathlineto{\pgfqpoint{2.835907in}{3.153386in}}%
\pgfpathlineto{\pgfqpoint{2.837002in}{3.054752in}}%
\pgfpathlineto{\pgfqpoint{2.836755in}{3.313090in}}%
\pgfpathlineto{\pgfqpoint{2.837033in}{3.127517in}}%
\pgfpathlineto{\pgfqpoint{2.837264in}{3.314831in}}%
\pgfpathlineto{\pgfqpoint{2.837588in}{3.078804in}}%
\pgfpathlineto{\pgfqpoint{2.838158in}{3.187480in}}%
\pgfpathlineto{\pgfqpoint{2.838559in}{3.059089in}}%
\pgfpathlineto{\pgfqpoint{2.838312in}{3.307803in}}%
\pgfpathlineto{\pgfqpoint{2.838790in}{3.274766in}}%
\pgfpathlineto{\pgfqpoint{2.838821in}{3.304828in}}%
\pgfpathlineto{\pgfqpoint{2.839144in}{3.076789in}}%
\pgfpathlineto{\pgfqpoint{2.839869in}{3.284477in}}%
\pgfpathlineto{\pgfqpoint{2.839946in}{3.066029in}}%
\pgfpathlineto{\pgfqpoint{2.840378in}{3.305456in}}%
\pgfpathlineto{\pgfqpoint{2.841010in}{3.123782in}}%
\pgfpathlineto{\pgfqpoint{2.841919in}{3.308073in}}%
\pgfpathlineto{\pgfqpoint{2.841657in}{3.074407in}}%
\pgfpathlineto{\pgfqpoint{2.842120in}{3.154144in}}%
\pgfpathlineto{\pgfqpoint{2.842258in}{3.069063in}}%
\pgfpathlineto{\pgfqpoint{2.842967in}{3.292414in}}%
\pgfpathlineto{\pgfqpoint{2.843229in}{3.092864in}}%
\pgfpathlineto{\pgfqpoint{2.843476in}{3.303402in}}%
\pgfpathlineto{\pgfqpoint{2.843800in}{3.079236in}}%
\pgfpathlineto{\pgfqpoint{2.844370in}{3.184902in}}%
\pgfpathlineto{\pgfqpoint{2.844771in}{3.068751in}}%
\pgfpathlineto{\pgfqpoint{2.845033in}{3.311509in}}%
\pgfpathlineto{\pgfqpoint{2.845511in}{3.158059in}}%
\pgfpathlineto{\pgfqpoint{2.846590in}{3.295960in}}%
\pgfpathlineto{\pgfqpoint{2.846158in}{3.077333in}}%
\pgfpathlineto{\pgfqpoint{2.846636in}{3.213432in}}%
\pgfpathlineto{\pgfqpoint{2.847206in}{3.081607in}}%
\pgfpathlineto{\pgfqpoint{2.847623in}{3.285151in}}%
\pgfpathlineto{\pgfqpoint{2.847746in}{3.195234in}}%
\pgfpathlineto{\pgfqpoint{2.848147in}{3.296349in}}%
\pgfpathlineto{\pgfqpoint{2.847869in}{3.074171in}}%
\pgfpathlineto{\pgfqpoint{2.848856in}{3.228762in}}%
\pgfpathlineto{\pgfqpoint{2.849426in}{3.080354in}}%
\pgfpathlineto{\pgfqpoint{2.849688in}{3.297159in}}%
\pgfpathlineto{\pgfqpoint{2.850027in}{3.087549in}}%
\pgfpathlineto{\pgfqpoint{2.851245in}{3.298993in}}%
\pgfpathlineto{\pgfqpoint{2.850983in}{3.065477in}}%
\pgfpathlineto{\pgfqpoint{2.851276in}{3.260338in}}%
\pgfpathlineto{\pgfqpoint{2.851569in}{3.081829in}}%
\pgfpathlineto{\pgfqpoint{2.851800in}{3.278664in}}%
\pgfpathlineto{\pgfqpoint{2.852401in}{3.167424in}}%
\pgfpathlineto{\pgfqpoint{2.852802in}{3.299303in}}%
\pgfpathlineto{\pgfqpoint{2.852525in}{3.074093in}}%
\pgfpathlineto{\pgfqpoint{2.853511in}{3.214477in}}%
\pgfpathlineto{\pgfqpoint{2.854081in}{3.075212in}}%
\pgfpathlineto{\pgfqpoint{2.854359in}{3.303404in}}%
\pgfpathlineto{\pgfqpoint{2.854683in}{3.087818in}}%
\pgfpathlineto{\pgfqpoint{2.855900in}{3.282992in}}%
\pgfpathlineto{\pgfqpoint{2.855638in}{3.077305in}}%
\pgfpathlineto{\pgfqpoint{2.855947in}{3.236531in}}%
\pgfpathlineto{\pgfqpoint{2.856517in}{3.085045in}}%
\pgfpathlineto{\pgfqpoint{2.856455in}{3.282335in}}%
\pgfpathlineto{\pgfqpoint{2.857072in}{3.188157in}}%
\pgfpathlineto{\pgfqpoint{2.858012in}{3.296346in}}%
\pgfpathlineto{\pgfqpoint{2.857195in}{3.060796in}}%
\pgfpathlineto{\pgfqpoint{2.858182in}{3.199871in}}%
\pgfpathlineto{\pgfqpoint{2.858752in}{3.070764in}}%
\pgfpathlineto{\pgfqpoint{2.859014in}{3.293375in}}%
\pgfpathlineto{\pgfqpoint{2.859353in}{3.097634in}}%
\pgfpathlineto{\pgfqpoint{2.859569in}{3.284096in}}%
\pgfpathlineto{\pgfqpoint{2.860294in}{3.064424in}}%
\pgfpathlineto{\pgfqpoint{2.860540in}{3.277909in}}%
\pgfpathlineto{\pgfqpoint{2.860571in}{3.288173in}}%
\pgfpathlineto{\pgfqpoint{2.860787in}{3.122661in}}%
\pgfpathlineto{\pgfqpoint{2.861851in}{3.052707in}}%
\pgfpathlineto{\pgfqpoint{2.861619in}{3.287851in}}%
\pgfpathlineto{\pgfqpoint{2.861881in}{3.097627in}}%
\pgfpathlineto{\pgfqpoint{2.862667in}{3.286418in}}%
\pgfpathlineto{\pgfqpoint{2.862729in}{3.082836in}}%
\pgfpathlineto{\pgfqpoint{2.863007in}{3.203183in}}%
\pgfpathlineto{\pgfqpoint{2.863407in}{3.053633in}}%
\pgfpathlineto{\pgfqpoint{2.863176in}{3.281654in}}%
\pgfpathlineto{\pgfqpoint{2.864101in}{3.199447in}}%
\pgfpathlineto{\pgfqpoint{2.864224in}{3.298318in}}%
\pgfpathlineto{\pgfqpoint{2.864949in}{3.066758in}}%
\pgfpathlineto{\pgfqpoint{2.865242in}{3.275805in}}%
\pgfpathlineto{\pgfqpoint{2.865843in}{3.083717in}}%
\pgfpathlineto{\pgfqpoint{2.866275in}{3.292471in}}%
\pgfpathlineto{\pgfqpoint{2.866382in}{3.173683in}}%
\pgfpathlineto{\pgfqpoint{2.867338in}{3.295652in}}%
\pgfpathlineto{\pgfqpoint{2.866506in}{3.073418in}}%
\pgfpathlineto{\pgfqpoint{2.867492in}{3.218067in}}%
\pgfpathlineto{\pgfqpoint{2.868047in}{3.056894in}}%
\pgfpathlineto{\pgfqpoint{2.867831in}{3.285957in}}%
\pgfpathlineto{\pgfqpoint{2.868679in}{3.107825in}}%
\pgfpathlineto{\pgfqpoint{2.868880in}{3.295428in}}%
\pgfpathlineto{\pgfqpoint{2.869604in}{3.043726in}}%
\pgfpathlineto{\pgfqpoint{2.869912in}{3.254393in}}%
\pgfpathlineto{\pgfqpoint{2.870514in}{3.069161in}}%
\pgfpathlineto{\pgfqpoint{2.870437in}{3.286122in}}%
\pgfpathlineto{\pgfqpoint{2.871053in}{3.168125in}}%
\pgfpathlineto{\pgfqpoint{2.871993in}{3.287613in}}%
\pgfpathlineto{\pgfqpoint{2.871161in}{3.050813in}}%
\pgfpathlineto{\pgfqpoint{2.872163in}{3.191737in}}%
\pgfpathlineto{\pgfqpoint{2.872718in}{3.058413in}}%
\pgfpathlineto{\pgfqpoint{2.872502in}{3.276221in}}%
\pgfpathlineto{\pgfqpoint{2.873304in}{3.123155in}}%
\pgfpathlineto{\pgfqpoint{2.873350in}{3.095407in}}%
\pgfpathlineto{\pgfqpoint{2.873396in}{3.160994in}}%
\pgfpathlineto{\pgfqpoint{2.873550in}{3.295268in}}%
\pgfpathlineto{\pgfqpoint{2.874259in}{3.051891in}}%
\pgfpathlineto{\pgfqpoint{2.874521in}{3.224982in}}%
\pgfpathlineto{\pgfqpoint{2.875184in}{3.076261in}}%
\pgfpathlineto{\pgfqpoint{2.875107in}{3.309279in}}%
\pgfpathlineto{\pgfqpoint{2.875601in}{3.244096in}}%
\pgfpathlineto{\pgfqpoint{2.876048in}{3.267648in}}%
\pgfpathlineto{\pgfqpoint{2.875816in}{3.037313in}}%
\pgfpathlineto{\pgfqpoint{2.876587in}{3.137885in}}%
\pgfpathlineto{\pgfqpoint{2.877373in}{3.034584in}}%
\pgfpathlineto{\pgfqpoint{2.876664in}{3.302946in}}%
\pgfpathlineto{\pgfqpoint{2.877589in}{3.260118in}}%
\pgfpathlineto{\pgfqpoint{2.878221in}{3.304382in}}%
\pgfpathlineto{\pgfqpoint{2.878298in}{3.052220in}}%
\pgfpathlineto{\pgfqpoint{2.878560in}{3.137042in}}%
\pgfpathlineto{\pgfqpoint{2.878591in}{3.167682in}}%
\pgfpathlineto{\pgfqpoint{2.878730in}{3.282034in}}%
\pgfpathlineto{\pgfqpoint{2.878930in}{3.040149in}}%
\pgfpathlineto{\pgfqpoint{2.879685in}{3.178211in}}%
\pgfpathlineto{\pgfqpoint{2.880487in}{3.042546in}}%
\pgfpathlineto{\pgfqpoint{2.879778in}{3.311002in}}%
\pgfpathlineto{\pgfqpoint{2.880780in}{3.189952in}}%
\pgfpathlineto{\pgfqpoint{2.881335in}{3.313630in}}%
\pgfpathlineto{\pgfqpoint{2.881412in}{3.070305in}}%
\pgfpathlineto{\pgfqpoint{2.881874in}{3.208924in}}%
\pgfpathlineto{\pgfqpoint{2.882044in}{3.045005in}}%
\pgfpathlineto{\pgfqpoint{2.882892in}{3.302694in}}%
\pgfpathlineto{\pgfqpoint{2.882984in}{3.114516in}}%
\pgfpathlineto{\pgfqpoint{2.883940in}{3.283893in}}%
\pgfpathlineto{\pgfqpoint{2.883601in}{3.061683in}}%
\pgfpathlineto{\pgfqpoint{2.884094in}{3.173740in}}%
\pgfpathlineto{\pgfqpoint{2.884526in}{3.061943in}}%
\pgfpathlineto{\pgfqpoint{2.884449in}{3.286454in}}%
\pgfpathlineto{\pgfqpoint{2.885204in}{3.164720in}}%
\pgfpathlineto{\pgfqpoint{2.885219in}{3.164647in}}%
\pgfpathlineto{\pgfqpoint{2.886083in}{3.061309in}}%
\pgfpathlineto{\pgfqpoint{2.885990in}{3.282727in}}%
\pgfpathlineto{\pgfqpoint{2.886345in}{3.121430in}}%
\pgfpathlineto{\pgfqpoint{2.887054in}{3.271546in}}%
\pgfpathlineto{\pgfqpoint{2.886838in}{3.057458in}}%
\pgfpathlineto{\pgfqpoint{2.887455in}{3.172447in}}%
\pgfpathlineto{\pgfqpoint{2.887624in}{3.061040in}}%
\pgfpathlineto{\pgfqpoint{2.887547in}{3.285988in}}%
\pgfpathlineto{\pgfqpoint{2.888549in}{3.168075in}}%
\pgfpathlineto{\pgfqpoint{2.889104in}{3.295981in}}%
\pgfpathlineto{\pgfqpoint{2.889181in}{3.065759in}}%
\pgfpathlineto{\pgfqpoint{2.889659in}{3.209791in}}%
\pgfpathlineto{\pgfqpoint{2.890738in}{3.057886in}}%
\pgfpathlineto{\pgfqpoint{2.890168in}{3.292756in}}%
\pgfpathlineto{\pgfqpoint{2.890769in}{3.166561in}}%
\pgfpathlineto{\pgfqpoint{2.891724in}{3.288663in}}%
\pgfpathlineto{\pgfqpoint{2.891509in}{3.079681in}}%
\pgfpathlineto{\pgfqpoint{2.891879in}{3.164223in}}%
\pgfpathlineto{\pgfqpoint{2.892295in}{3.073697in}}%
\pgfpathlineto{\pgfqpoint{2.892218in}{3.295163in}}%
\pgfpathlineto{\pgfqpoint{2.892973in}{3.168291in}}%
\pgfpathlineto{\pgfqpoint{2.892988in}{3.169101in}}%
\pgfpathlineto{\pgfqpoint{2.893019in}{3.113637in}}%
\pgfpathlineto{\pgfqpoint{2.893066in}{3.062119in}}%
\pgfpathlineto{\pgfqpoint{2.893775in}{3.285345in}}%
\pgfpathlineto{\pgfqpoint{2.894114in}{3.122883in}}%
\pgfpathlineto{\pgfqpoint{2.894823in}{3.285244in}}%
\pgfpathlineto{\pgfqpoint{2.894622in}{3.076120in}}%
\pgfpathlineto{\pgfqpoint{2.895239in}{3.142328in}}%
\pgfpathlineto{\pgfqpoint{2.895409in}{3.069955in}}%
\pgfpathlineto{\pgfqpoint{2.895332in}{3.293612in}}%
\pgfpathlineto{\pgfqpoint{2.896318in}{3.157750in}}%
\pgfpathlineto{\pgfqpoint{2.896380in}{3.292220in}}%
\pgfpathlineto{\pgfqpoint{2.896965in}{3.079073in}}%
\pgfpathlineto{\pgfqpoint{2.897428in}{3.218956in}}%
\pgfpathlineto{\pgfqpoint{2.898507in}{3.088560in}}%
\pgfpathlineto{\pgfqpoint{2.898430in}{3.294929in}}%
\pgfpathlineto{\pgfqpoint{2.898538in}{3.138394in}}%
\pgfpathlineto{\pgfqpoint{2.899494in}{3.288788in}}%
\pgfpathlineto{\pgfqpoint{2.899278in}{3.073780in}}%
\pgfpathlineto{\pgfqpoint{2.899648in}{3.177664in}}%
\pgfpathlineto{\pgfqpoint{2.900696in}{3.085291in}}%
\pgfpathlineto{\pgfqpoint{2.899987in}{3.286792in}}%
\pgfpathlineto{\pgfqpoint{2.900742in}{3.174261in}}%
\pgfpathlineto{\pgfqpoint{2.901035in}{3.282883in}}%
\pgfpathlineto{\pgfqpoint{2.901636in}{3.082512in}}%
\pgfpathlineto{\pgfqpoint{2.901744in}{3.111614in}}%
\pgfpathlineto{\pgfqpoint{2.902592in}{3.281909in}}%
\pgfpathlineto{\pgfqpoint{2.902392in}{3.083959in}}%
\pgfpathlineto{\pgfqpoint{2.902839in}{3.132650in}}%
\pgfpathlineto{\pgfqpoint{2.903948in}{3.092405in}}%
\pgfpathlineto{\pgfqpoint{2.903101in}{3.271795in}}%
\pgfpathlineto{\pgfqpoint{2.903964in}{3.099701in}}%
\pgfpathlineto{\pgfqpoint{2.904657in}{3.279108in}}%
\pgfpathlineto{\pgfqpoint{2.904735in}{3.091667in}}%
\pgfpathlineto{\pgfqpoint{2.905105in}{3.202980in}}%
\pgfpathlineto{\pgfqpoint{2.905505in}{3.077400in}}%
\pgfpathlineto{\pgfqpoint{2.905706in}{3.279698in}}%
\pgfpathlineto{\pgfqpoint{2.906184in}{3.227612in}}%
\pgfpathlineto{\pgfqpoint{2.907263in}{3.280708in}}%
\pgfpathlineto{\pgfqpoint{2.907062in}{3.085345in}}%
\pgfpathlineto{\pgfqpoint{2.907293in}{3.227559in}}%
\pgfpathlineto{\pgfqpoint{2.907848in}{3.076069in}}%
\pgfpathlineto{\pgfqpoint{2.907771in}{3.269921in}}%
\pgfpathlineto{\pgfqpoint{2.908403in}{3.174668in}}%
\pgfpathlineto{\pgfqpoint{2.908819in}{3.292581in}}%
\pgfpathlineto{\pgfqpoint{2.908619in}{3.080146in}}%
\pgfpathlineto{\pgfqpoint{2.909498in}{3.174588in}}%
\pgfpathlineto{\pgfqpoint{2.910176in}{3.077755in}}%
\pgfpathlineto{\pgfqpoint{2.910376in}{3.279864in}}%
\pgfpathlineto{\pgfqpoint{2.910623in}{3.118422in}}%
\pgfpathlineto{\pgfqpoint{2.911717in}{3.061637in}}%
\pgfpathlineto{\pgfqpoint{2.910870in}{3.267664in}}%
\pgfpathlineto{\pgfqpoint{2.911733in}{3.079275in}}%
\pgfpathlineto{\pgfqpoint{2.911933in}{3.289717in}}%
\pgfpathlineto{\pgfqpoint{2.912504in}{3.076331in}}%
\pgfpathlineto{\pgfqpoint{2.912858in}{3.184170in}}%
\pgfpathlineto{\pgfqpoint{2.913274in}{3.054119in}}%
\pgfpathlineto{\pgfqpoint{2.913475in}{3.280671in}}%
\pgfpathlineto{\pgfqpoint{2.913937in}{3.167293in}}%
\pgfpathlineto{\pgfqpoint{2.915032in}{3.294918in}}%
\pgfpathlineto{\pgfqpoint{2.914061in}{3.055169in}}%
\pgfpathlineto{\pgfqpoint{2.915063in}{3.226385in}}%
\pgfpathlineto{\pgfqpoint{2.915617in}{3.070418in}}%
\pgfpathlineto{\pgfqpoint{2.915679in}{3.280708in}}%
\pgfpathlineto{\pgfqpoint{2.916172in}{3.189800in}}%
\pgfpathlineto{\pgfqpoint{2.916589in}{3.296315in}}%
\pgfpathlineto{\pgfqpoint{2.916388in}{3.061881in}}%
\pgfpathlineto{\pgfqpoint{2.917267in}{3.170893in}}%
\pgfpathlineto{\pgfqpoint{2.917930in}{3.061612in}}%
\pgfpathlineto{\pgfqpoint{2.918145in}{3.305939in}}%
\pgfpathlineto{\pgfqpoint{2.918392in}{3.122372in}}%
\pgfpathlineto{\pgfqpoint{2.918716in}{3.073581in}}%
\pgfpathlineto{\pgfqpoint{2.918793in}{3.282966in}}%
\pgfpathlineto{\pgfqpoint{2.919394in}{3.166882in}}%
\pgfpathlineto{\pgfqpoint{2.919687in}{3.299649in}}%
\pgfpathlineto{\pgfqpoint{2.919487in}{3.054008in}}%
\pgfpathlineto{\pgfqpoint{2.920489in}{3.168793in}}%
\pgfpathlineto{\pgfqpoint{2.921043in}{3.055198in}}%
\pgfpathlineto{\pgfqpoint{2.921244in}{3.295320in}}%
\pgfpathlineto{\pgfqpoint{2.921598in}{3.160337in}}%
\pgfpathlineto{\pgfqpoint{2.921891in}{3.271582in}}%
\pgfpathlineto{\pgfqpoint{2.922585in}{3.071120in}}%
\pgfpathlineto{\pgfqpoint{2.922600in}{3.060856in}}%
\pgfpathlineto{\pgfqpoint{2.922801in}{3.295661in}}%
\pgfpathlineto{\pgfqpoint{2.923433in}{3.243989in}}%
\pgfpathlineto{\pgfqpoint{2.924358in}{3.299576in}}%
\pgfpathlineto{\pgfqpoint{2.924142in}{3.067748in}}%
\pgfpathlineto{\pgfqpoint{2.924512in}{3.207271in}}%
\pgfpathlineto{\pgfqpoint{2.924943in}{3.097014in}}%
\pgfpathlineto{\pgfqpoint{2.925005in}{3.276320in}}%
\pgfpathlineto{\pgfqpoint{2.925622in}{3.187957in}}%
\pgfpathlineto{\pgfqpoint{2.925699in}{3.065665in}}%
\pgfpathlineto{\pgfqpoint{2.925915in}{3.301507in}}%
\pgfpathlineto{\pgfqpoint{2.926531in}{3.182663in}}%
\pgfpathlineto{\pgfqpoint{2.927471in}{3.295671in}}%
\pgfpathlineto{\pgfqpoint{2.927256in}{3.070224in}}%
\pgfpathlineto{\pgfqpoint{2.927626in}{3.203191in}}%
\pgfpathlineto{\pgfqpoint{2.928042in}{3.102515in}}%
\pgfpathlineto{\pgfqpoint{2.928119in}{3.260356in}}%
\pgfpathlineto{\pgfqpoint{2.928735in}{3.190641in}}%
\pgfpathlineto{\pgfqpoint{2.928797in}{3.074401in}}%
\pgfpathlineto{\pgfqpoint{2.929028in}{3.295659in}}%
\pgfpathlineto{\pgfqpoint{2.929876in}{3.114215in}}%
\pgfpathlineto{\pgfqpoint{2.930585in}{3.296343in}}%
\pgfpathlineto{\pgfqpoint{2.930354in}{3.076383in}}%
\pgfpathlineto{\pgfqpoint{2.930986in}{3.136651in}}%
\pgfpathlineto{\pgfqpoint{2.931233in}{3.258256in}}%
\pgfpathlineto{\pgfqpoint{2.931418in}{3.105130in}}%
\pgfpathlineto{\pgfqpoint{2.931880in}{3.127760in}}%
\pgfpathlineto{\pgfqpoint{2.931911in}{3.079207in}}%
\pgfpathlineto{\pgfqpoint{2.932142in}{3.294311in}}%
\pgfpathlineto{\pgfqpoint{2.932990in}{3.120513in}}%
\pgfpathlineto{\pgfqpoint{2.933684in}{3.296134in}}%
\pgfpathlineto{\pgfqpoint{2.933468in}{3.081526in}}%
\pgfpathlineto{\pgfqpoint{2.934100in}{3.138572in}}%
\pgfpathlineto{\pgfqpoint{2.935241in}{3.290287in}}%
\pgfpathlineto{\pgfqpoint{2.935009in}{3.081119in}}%
\pgfpathlineto{\pgfqpoint{2.935287in}{3.186344in}}%
\pgfpathlineto{\pgfqpoint{2.935318in}{3.108035in}}%
\pgfpathlineto{\pgfqpoint{2.935903in}{3.257250in}}%
\pgfpathlineto{\pgfqpoint{2.936397in}{3.143062in}}%
\pgfpathlineto{\pgfqpoint{2.936797in}{3.291896in}}%
\pgfpathlineto{\pgfqpoint{2.936566in}{3.084937in}}%
\pgfpathlineto{\pgfqpoint{2.937491in}{3.173691in}}%
\pgfpathlineto{\pgfqpoint{2.938123in}{3.090820in}}%
\pgfpathlineto{\pgfqpoint{2.938354in}{3.286271in}}%
\pgfpathlineto{\pgfqpoint{2.938616in}{3.131362in}}%
\pgfpathlineto{\pgfqpoint{2.939665in}{3.094239in}}%
\pgfpathlineto{\pgfqpoint{2.939017in}{3.256570in}}%
\pgfpathlineto{\pgfqpoint{2.939695in}{3.127366in}}%
\pgfpathlineto{\pgfqpoint{2.939911in}{3.281821in}}%
\pgfpathlineto{\pgfqpoint{2.940250in}{3.110380in}}%
\pgfpathlineto{\pgfqpoint{2.940852in}{3.191009in}}%
\pgfpathlineto{\pgfqpoint{2.941221in}{3.092076in}}%
\pgfpathlineto{\pgfqpoint{2.941453in}{3.270995in}}%
\pgfpathlineto{\pgfqpoint{2.941946in}{3.189608in}}%
\pgfpathlineto{\pgfqpoint{2.943010in}{3.267434in}}%
\pgfpathlineto{\pgfqpoint{2.942593in}{3.086211in}}%
\pgfpathlineto{\pgfqpoint{2.943040in}{3.228835in}}%
\pgfpathlineto{\pgfqpoint{2.944150in}{3.084600in}}%
\pgfpathlineto{\pgfqpoint{2.943672in}{3.247317in}}%
\pgfpathlineto{\pgfqpoint{2.944166in}{3.118287in}}%
\pgfpathlineto{\pgfqpoint{2.944566in}{3.266965in}}%
\pgfpathlineto{\pgfqpoint{2.944921in}{3.093622in}}%
\pgfpathlineto{\pgfqpoint{2.945276in}{3.150115in}}%
\pgfpathlineto{\pgfqpoint{2.946123in}{3.274784in}}%
\pgfpathlineto{\pgfqpoint{2.945707in}{3.085109in}}%
\pgfpathlineto{\pgfqpoint{2.946185in}{3.132662in}}%
\pgfpathlineto{\pgfqpoint{2.947249in}{3.083242in}}%
\pgfpathlineto{\pgfqpoint{2.946786in}{3.258540in}}%
\pgfpathlineto{\pgfqpoint{2.947279in}{3.138122in}}%
\pgfpathlineto{\pgfqpoint{2.947680in}{3.274345in}}%
\pgfpathlineto{\pgfqpoint{2.948019in}{3.083040in}}%
\pgfpathlineto{\pgfqpoint{2.948389in}{3.155727in}}%
\pgfpathlineto{\pgfqpoint{2.949237in}{3.270819in}}%
\pgfpathlineto{\pgfqpoint{2.948806in}{3.081161in}}%
\pgfpathlineto{\pgfqpoint{2.949407in}{3.174730in}}%
\pgfpathlineto{\pgfqpoint{2.949576in}{3.078670in}}%
\pgfpathlineto{\pgfqpoint{2.949900in}{3.253404in}}%
\pgfpathlineto{\pgfqpoint{2.950532in}{3.126065in}}%
\pgfpathlineto{\pgfqpoint{2.950794in}{3.260718in}}%
\pgfpathlineto{\pgfqpoint{2.951133in}{3.077323in}}%
\pgfpathlineto{\pgfqpoint{2.951626in}{3.114095in}}%
\pgfpathlineto{\pgfqpoint{2.951642in}{3.112361in}}%
\pgfpathlineto{\pgfqpoint{2.951811in}{3.211389in}}%
\pgfpathlineto{\pgfqpoint{2.952336in}{3.262119in}}%
\pgfpathlineto{\pgfqpoint{2.952675in}{3.079254in}}%
\pgfpathlineto{\pgfqpoint{2.952906in}{3.224331in}}%
\pgfpathlineto{\pgfqpoint{2.953461in}{3.076781in}}%
\pgfpathlineto{\pgfqpoint{2.953892in}{3.265703in}}%
\pgfpathlineto{\pgfqpoint{2.954016in}{3.210779in}}%
\pgfpathlineto{\pgfqpoint{2.954555in}{3.252611in}}%
\pgfpathlineto{\pgfqpoint{2.954232in}{3.066672in}}%
\pgfpathlineto{\pgfqpoint{2.955095in}{3.214612in}}%
\pgfpathlineto{\pgfqpoint{2.955788in}{3.065465in}}%
\pgfpathlineto{\pgfqpoint{2.955449in}{3.260494in}}%
\pgfpathlineto{\pgfqpoint{2.956205in}{3.176808in}}%
\pgfpathlineto{\pgfqpoint{2.957006in}{3.256180in}}%
\pgfpathlineto{\pgfqpoint{2.956575in}{3.080302in}}%
\pgfpathlineto{\pgfqpoint{2.957191in}{3.124449in}}%
\pgfpathlineto{\pgfqpoint{2.957345in}{3.069033in}}%
\pgfpathlineto{\pgfqpoint{2.957546in}{3.258919in}}%
\pgfpathlineto{\pgfqpoint{2.958286in}{3.135691in}}%
\pgfpathlineto{\pgfqpoint{2.959087in}{3.271066in}}%
\pgfpathlineto{\pgfqpoint{2.958887in}{3.065958in}}%
\pgfpathlineto{\pgfqpoint{2.959380in}{3.126513in}}%
\pgfpathlineto{\pgfqpoint{2.960444in}{3.058304in}}%
\pgfpathlineto{\pgfqpoint{2.960120in}{3.265704in}}%
\pgfpathlineto{\pgfqpoint{2.960475in}{3.115162in}}%
\pgfpathlineto{\pgfqpoint{2.960644in}{3.276110in}}%
\pgfpathlineto{\pgfqpoint{2.961230in}{3.073613in}}%
\pgfpathlineto{\pgfqpoint{2.961600in}{3.176955in}}%
\pgfpathlineto{\pgfqpoint{2.962001in}{3.057702in}}%
\pgfpathlineto{\pgfqpoint{2.961662in}{3.267883in}}%
\pgfpathlineto{\pgfqpoint{2.962170in}{3.211932in}}%
\pgfpathlineto{\pgfqpoint{2.962201in}{3.272549in}}%
\pgfpathlineto{\pgfqpoint{2.962787in}{3.072472in}}%
\pgfpathlineto{\pgfqpoint{2.963265in}{3.166146in}}%
\pgfpathlineto{\pgfqpoint{2.963558in}{3.063040in}}%
\pgfpathlineto{\pgfqpoint{2.963743in}{3.286847in}}%
\pgfpathlineto{\pgfqpoint{2.964359in}{3.123841in}}%
\pgfpathlineto{\pgfqpoint{2.965299in}{3.292556in}}%
\pgfpathlineto{\pgfqpoint{2.965099in}{3.059803in}}%
\pgfpathlineto{\pgfqpoint{2.965469in}{3.184561in}}%
\pgfpathlineto{\pgfqpoint{2.965885in}{3.063446in}}%
\pgfpathlineto{\pgfqpoint{2.966332in}{3.278839in}}%
\pgfpathlineto{\pgfqpoint{2.966579in}{3.165870in}}%
\pgfpathlineto{\pgfqpoint{2.966656in}{3.052792in}}%
\pgfpathlineto{\pgfqpoint{2.966856in}{3.293797in}}%
\pgfpathlineto{\pgfqpoint{2.967334in}{3.194305in}}%
\pgfpathlineto{\pgfqpoint{2.968398in}{3.291656in}}%
\pgfpathlineto{\pgfqpoint{2.968213in}{3.049605in}}%
\pgfpathlineto{\pgfqpoint{2.968429in}{3.247946in}}%
\pgfpathlineto{\pgfqpoint{2.968984in}{3.059649in}}%
\pgfpathlineto{\pgfqpoint{2.969431in}{3.280522in}}%
\pgfpathlineto{\pgfqpoint{2.969539in}{3.186882in}}%
\pgfpathlineto{\pgfqpoint{2.969955in}{3.302130in}}%
\pgfpathlineto{\pgfqpoint{2.969754in}{3.057039in}}%
\pgfpathlineto{\pgfqpoint{2.970633in}{3.179806in}}%
\pgfpathlineto{\pgfqpoint{2.971311in}{3.059812in}}%
\pgfpathlineto{\pgfqpoint{2.971512in}{3.303665in}}%
\pgfpathlineto{\pgfqpoint{2.971743in}{3.164666in}}%
\pgfpathlineto{\pgfqpoint{2.972020in}{3.275723in}}%
\pgfpathlineto{\pgfqpoint{2.972082in}{3.065840in}}%
\pgfpathlineto{\pgfqpoint{2.972714in}{3.135547in}}%
\pgfpathlineto{\pgfqpoint{2.973639in}{3.059542in}}%
\pgfpathlineto{\pgfqpoint{2.973053in}{3.302173in}}%
\pgfpathlineto{\pgfqpoint{2.973808in}{3.152510in}}%
\pgfpathlineto{\pgfqpoint{2.974610in}{3.308766in}}%
\pgfpathlineto{\pgfqpoint{2.974410in}{3.056872in}}%
\pgfpathlineto{\pgfqpoint{2.974903in}{3.140328in}}%
\pgfpathlineto{\pgfqpoint{2.975966in}{3.054033in}}%
\pgfpathlineto{\pgfqpoint{2.975643in}{3.277663in}}%
\pgfpathlineto{\pgfqpoint{2.975997in}{3.147286in}}%
\pgfpathlineto{\pgfqpoint{2.976167in}{3.300429in}}%
\pgfpathlineto{\pgfqpoint{2.976753in}{3.057508in}}%
\pgfpathlineto{\pgfqpoint{2.977107in}{3.179902in}}%
\pgfpathlineto{\pgfqpoint{2.977523in}{3.065984in}}%
\pgfpathlineto{\pgfqpoint{2.977708in}{3.299999in}}%
\pgfpathlineto{\pgfqpoint{2.978186in}{3.168759in}}%
\pgfpathlineto{\pgfqpoint{2.979265in}{3.306106in}}%
\pgfpathlineto{\pgfqpoint{2.978294in}{3.061160in}}%
\pgfpathlineto{\pgfqpoint{2.979296in}{3.232397in}}%
\pgfpathlineto{\pgfqpoint{2.979851in}{3.059811in}}%
\pgfpathlineto{\pgfqpoint{2.979389in}{3.277418in}}%
\pgfpathlineto{\pgfqpoint{2.980406in}{3.198601in}}%
\pgfpathlineto{\pgfqpoint{2.980822in}{3.305440in}}%
\pgfpathlineto{\pgfqpoint{2.980622in}{3.055508in}}%
\pgfpathlineto{\pgfqpoint{2.981485in}{3.191719in}}%
\pgfpathlineto{\pgfqpoint{2.982163in}{3.059055in}}%
\pgfpathlineto{\pgfqpoint{2.982364in}{3.309214in}}%
\pgfpathlineto{\pgfqpoint{2.982595in}{3.158116in}}%
\pgfpathlineto{\pgfqpoint{2.983396in}{3.280722in}}%
\pgfpathlineto{\pgfqpoint{2.982949in}{3.054951in}}%
\pgfpathlineto{\pgfqpoint{2.983566in}{3.150865in}}%
\pgfpathlineto{\pgfqpoint{2.983720in}{3.058867in}}%
\pgfpathlineto{\pgfqpoint{2.983921in}{3.306765in}}%
\pgfpathlineto{\pgfqpoint{2.984660in}{3.161867in}}%
\pgfpathlineto{\pgfqpoint{2.985462in}{3.304661in}}%
\pgfpathlineto{\pgfqpoint{2.985277in}{3.061238in}}%
\pgfpathlineto{\pgfqpoint{2.985739in}{3.147392in}}%
\pgfpathlineto{\pgfqpoint{2.986819in}{3.060268in}}%
\pgfpathlineto{\pgfqpoint{2.986495in}{3.272891in}}%
\pgfpathlineto{\pgfqpoint{2.986849in}{3.145117in}}%
\pgfpathlineto{\pgfqpoint{2.987019in}{3.306967in}}%
\pgfpathlineto{\pgfqpoint{2.987605in}{3.066043in}}%
\pgfpathlineto{\pgfqpoint{2.987959in}{3.184692in}}%
\pgfpathlineto{\pgfqpoint{2.988375in}{3.058921in}}%
\pgfpathlineto{\pgfqpoint{2.988560in}{3.306917in}}%
\pgfpathlineto{\pgfqpoint{2.989054in}{3.209441in}}%
\pgfpathlineto{\pgfqpoint{2.990117in}{3.306277in}}%
\pgfpathlineto{\pgfqpoint{2.989146in}{3.059677in}}%
\pgfpathlineto{\pgfqpoint{2.990148in}{3.232314in}}%
\pgfpathlineto{\pgfqpoint{2.990703in}{3.069070in}}%
\pgfpathlineto{\pgfqpoint{2.990256in}{3.280918in}}%
\pgfpathlineto{\pgfqpoint{2.991258in}{3.177195in}}%
\pgfpathlineto{\pgfqpoint{2.991674in}{3.294390in}}%
\pgfpathlineto{\pgfqpoint{2.991474in}{3.069524in}}%
\pgfpathlineto{\pgfqpoint{2.992337in}{3.185750in}}%
\pgfpathlineto{\pgfqpoint{2.993015in}{3.071281in}}%
\pgfpathlineto{\pgfqpoint{2.993216in}{3.298121in}}%
\pgfpathlineto{\pgfqpoint{2.993447in}{3.166338in}}%
\pgfpathlineto{\pgfqpoint{2.993524in}{3.115633in}}%
\pgfpathlineto{\pgfqpoint{2.993709in}{3.216071in}}%
\pgfpathlineto{\pgfqpoint{2.994773in}{3.299188in}}%
\pgfpathlineto{\pgfqpoint{2.994572in}{3.071505in}}%
\pgfpathlineto{\pgfqpoint{2.994803in}{3.234735in}}%
\pgfpathlineto{\pgfqpoint{2.995358in}{3.088097in}}%
\pgfpathlineto{\pgfqpoint{2.994911in}{3.275017in}}%
\pgfpathlineto{\pgfqpoint{2.995929in}{3.185805in}}%
\pgfpathlineto{\pgfqpoint{2.996314in}{3.288340in}}%
\pgfpathlineto{\pgfqpoint{2.996114in}{3.070590in}}%
\pgfpathlineto{\pgfqpoint{2.996884in}{3.116853in}}%
\pgfpathlineto{\pgfqpoint{2.997671in}{3.071439in}}%
\pgfpathlineto{\pgfqpoint{2.997871in}{3.285318in}}%
\pgfpathlineto{\pgfqpoint{2.997963in}{3.166996in}}%
\pgfpathlineto{\pgfqpoint{2.998010in}{3.259439in}}%
\pgfpathlineto{\pgfqpoint{2.998457in}{3.091523in}}%
\pgfpathlineto{\pgfqpoint{2.999058in}{3.184821in}}%
\pgfpathlineto{\pgfqpoint{2.999212in}{3.068648in}}%
\pgfpathlineto{\pgfqpoint{2.999428in}{3.282330in}}%
\pgfpathlineto{\pgfqpoint{3.000152in}{3.157116in}}%
\pgfpathlineto{\pgfqpoint{3.000969in}{3.283402in}}%
\pgfpathlineto{\pgfqpoint{3.000754in}{3.075304in}}%
\pgfpathlineto{\pgfqpoint{3.001247in}{3.128994in}}%
\pgfpathlineto{\pgfqpoint{3.002310in}{3.069498in}}%
\pgfpathlineto{\pgfqpoint{3.001879in}{3.254540in}}%
\pgfpathlineto{\pgfqpoint{3.002341in}{3.097823in}}%
\pgfpathlineto{\pgfqpoint{3.002526in}{3.286803in}}%
\pgfpathlineto{\pgfqpoint{3.003482in}{3.179278in}}%
\pgfpathlineto{\pgfqpoint{3.003867in}{3.075107in}}%
\pgfpathlineto{\pgfqpoint{3.004083in}{3.279656in}}%
\pgfpathlineto{\pgfqpoint{3.004561in}{3.196930in}}%
\pgfpathlineto{\pgfqpoint{3.005625in}{3.273000in}}%
\pgfpathlineto{\pgfqpoint{3.005409in}{3.067721in}}%
\pgfpathlineto{\pgfqpoint{3.005671in}{3.222484in}}%
\pgfpathlineto{\pgfqpoint{3.006210in}{3.115068in}}%
\pgfpathlineto{\pgfqpoint{3.005763in}{3.249400in}}%
\pgfpathlineto{\pgfqpoint{3.006796in}{3.173134in}}%
\pgfpathlineto{\pgfqpoint{3.007182in}{3.274292in}}%
\pgfpathlineto{\pgfqpoint{3.006966in}{3.075484in}}%
\pgfpathlineto{\pgfqpoint{3.007752in}{3.134880in}}%
\pgfpathlineto{\pgfqpoint{3.008507in}{3.070928in}}%
\pgfpathlineto{\pgfqpoint{3.008738in}{3.270394in}}%
\pgfpathlineto{\pgfqpoint{3.008815in}{3.168480in}}%
\pgfpathlineto{\pgfqpoint{3.008877in}{3.242820in}}%
\pgfpathlineto{\pgfqpoint{3.009016in}{3.115845in}}%
\pgfpathlineto{\pgfqpoint{3.009848in}{3.131257in}}%
\pgfpathlineto{\pgfqpoint{3.010064in}{3.071293in}}%
\pgfpathlineto{\pgfqpoint{3.010295in}{3.260217in}}%
\pgfpathlineto{\pgfqpoint{3.010912in}{3.136958in}}%
\pgfpathlineto{\pgfqpoint{3.011837in}{3.263563in}}%
\pgfpathlineto{\pgfqpoint{3.011621in}{3.075473in}}%
\pgfpathlineto{\pgfqpoint{3.012037in}{3.172468in}}%
\pgfpathlineto{\pgfqpoint{3.012130in}{3.122330in}}%
\pgfpathlineto{\pgfqpoint{3.012592in}{3.214734in}}%
\pgfpathlineto{\pgfqpoint{3.012700in}{3.178939in}}%
\pgfpathlineto{\pgfqpoint{3.013394in}{3.261394in}}%
\pgfpathlineto{\pgfqpoint{3.013162in}{3.078456in}}%
\pgfpathlineto{\pgfqpoint{3.013794in}{3.169223in}}%
\pgfpathlineto{\pgfqpoint{3.014719in}{3.080750in}}%
\pgfpathlineto{\pgfqpoint{3.014288in}{3.243061in}}%
\pgfpathlineto{\pgfqpoint{3.014874in}{3.160468in}}%
\pgfpathlineto{\pgfqpoint{3.014951in}{3.264394in}}%
\pgfpathlineto{\pgfqpoint{3.015228in}{3.128259in}}%
\pgfpathlineto{\pgfqpoint{3.016014in}{3.209999in}}%
\pgfpathlineto{\pgfqpoint{3.016261in}{3.077257in}}%
\pgfpathlineto{\pgfqpoint{3.016492in}{3.259870in}}%
\pgfpathlineto{\pgfqpoint{3.017155in}{3.139888in}}%
\pgfpathlineto{\pgfqpoint{3.017818in}{3.076709in}}%
\pgfpathlineto{\pgfqpoint{3.017402in}{3.241210in}}%
\pgfpathlineto{\pgfqpoint{3.018003in}{3.207981in}}%
\pgfpathlineto{\pgfqpoint{3.018049in}{3.262032in}}%
\pgfpathlineto{\pgfqpoint{3.018388in}{3.119355in}}%
\pgfpathlineto{\pgfqpoint{3.019113in}{3.213715in}}%
\pgfpathlineto{\pgfqpoint{3.019359in}{3.073806in}}%
\pgfpathlineto{\pgfqpoint{3.019606in}{3.262229in}}%
\pgfpathlineto{\pgfqpoint{3.020269in}{3.134639in}}%
\pgfpathlineto{\pgfqpoint{3.021147in}{3.267082in}}%
\pgfpathlineto{\pgfqpoint{3.020916in}{3.080436in}}%
\pgfpathlineto{\pgfqpoint{3.021394in}{3.151773in}}%
\pgfpathlineto{\pgfqpoint{3.022458in}{3.076449in}}%
\pgfpathlineto{\pgfqpoint{3.022041in}{3.238649in}}%
\pgfpathlineto{\pgfqpoint{3.022519in}{3.119371in}}%
\pgfpathlineto{\pgfqpoint{3.022704in}{3.267045in}}%
\pgfpathlineto{\pgfqpoint{3.023043in}{3.111827in}}%
\pgfpathlineto{\pgfqpoint{3.023660in}{3.171906in}}%
\pgfpathlineto{\pgfqpoint{3.024015in}{3.077573in}}%
\pgfpathlineto{\pgfqpoint{3.024261in}{3.273405in}}%
\pgfpathlineto{\pgfqpoint{3.024739in}{3.177160in}}%
\pgfpathlineto{\pgfqpoint{3.025818in}{3.279922in}}%
\pgfpathlineto{\pgfqpoint{3.025556in}{3.082184in}}%
\pgfpathlineto{\pgfqpoint{3.025864in}{3.200486in}}%
\pgfpathlineto{\pgfqpoint{3.026928in}{3.104005in}}%
\pgfpathlineto{\pgfqpoint{3.026697in}{3.241833in}}%
\pgfpathlineto{\pgfqpoint{3.026990in}{3.182362in}}%
\pgfpathlineto{\pgfqpoint{3.027360in}{3.279795in}}%
\pgfpathlineto{\pgfqpoint{3.027113in}{3.082241in}}%
\pgfpathlineto{\pgfqpoint{3.027668in}{3.137662in}}%
\pgfpathlineto{\pgfqpoint{3.028670in}{3.077112in}}%
\pgfpathlineto{\pgfqpoint{3.028408in}{3.243818in}}%
\pgfpathlineto{\pgfqpoint{3.028762in}{3.147161in}}%
\pgfpathlineto{\pgfqpoint{3.029256in}{3.105767in}}%
\pgfpathlineto{\pgfqpoint{3.028916in}{3.276849in}}%
\pgfpathlineto{\pgfqpoint{3.029764in}{3.170713in}}%
\pgfpathlineto{\pgfqpoint{3.030473in}{3.279399in}}%
\pgfpathlineto{\pgfqpoint{3.030211in}{3.077377in}}%
\pgfpathlineto{\pgfqpoint{3.030874in}{3.177362in}}%
\pgfpathlineto{\pgfqpoint{3.031768in}{3.079001in}}%
\pgfpathlineto{\pgfqpoint{3.031506in}{3.250070in}}%
\pgfpathlineto{\pgfqpoint{3.031953in}{3.179836in}}%
\pgfpathlineto{\pgfqpoint{3.032015in}{3.284283in}}%
\pgfpathlineto{\pgfqpoint{3.032354in}{3.101114in}}%
\pgfpathlineto{\pgfqpoint{3.033078in}{3.233781in}}%
\pgfpathlineto{\pgfqpoint{3.033310in}{3.082608in}}%
\pgfpathlineto{\pgfqpoint{3.033572in}{3.287797in}}%
\pgfpathlineto{\pgfqpoint{3.034219in}{3.132851in}}%
\pgfpathlineto{\pgfqpoint{3.035129in}{3.289718in}}%
\pgfpathlineto{\pgfqpoint{3.034867in}{3.073721in}}%
\pgfpathlineto{\pgfqpoint{3.035344in}{3.152821in}}%
\pgfpathlineto{\pgfqpoint{3.036423in}{3.079406in}}%
\pgfpathlineto{\pgfqpoint{3.036161in}{3.257536in}}%
\pgfpathlineto{\pgfqpoint{3.036501in}{3.124425in}}%
\pgfpathlineto{\pgfqpoint{3.036670in}{3.289215in}}%
\pgfpathlineto{\pgfqpoint{3.037271in}{3.099962in}}%
\pgfpathlineto{\pgfqpoint{3.037718in}{3.260677in}}%
\pgfpathlineto{\pgfqpoint{3.037965in}{3.076191in}}%
\pgfpathlineto{\pgfqpoint{3.038227in}{3.294260in}}%
\pgfpathlineto{\pgfqpoint{3.038890in}{3.155025in}}%
\pgfpathlineto{\pgfqpoint{3.039784in}{3.294691in}}%
\pgfpathlineto{\pgfqpoint{3.039522in}{3.083817in}}%
\pgfpathlineto{\pgfqpoint{3.039969in}{3.160040in}}%
\pgfpathlineto{\pgfqpoint{3.041063in}{3.079388in}}%
\pgfpathlineto{\pgfqpoint{3.040817in}{3.263206in}}%
\pgfpathlineto{\pgfqpoint{3.041140in}{3.100617in}}%
\pgfpathlineto{\pgfqpoint{3.041325in}{3.295861in}}%
\pgfpathlineto{\pgfqpoint{3.041649in}{3.093192in}}%
\pgfpathlineto{\pgfqpoint{3.042281in}{3.160023in}}%
\pgfpathlineto{\pgfqpoint{3.042620in}{3.081826in}}%
\pgfpathlineto{\pgfqpoint{3.042882in}{3.294715in}}%
\pgfpathlineto{\pgfqpoint{3.043345in}{3.151511in}}%
\pgfpathlineto{\pgfqpoint{3.044439in}{3.298693in}}%
\pgfpathlineto{\pgfqpoint{3.044162in}{3.089426in}}%
\pgfpathlineto{\pgfqpoint{3.044470in}{3.229169in}}%
\pgfpathlineto{\pgfqpoint{3.045025in}{3.097328in}}%
\pgfpathlineto{\pgfqpoint{3.045472in}{3.267152in}}%
\pgfpathlineto{\pgfqpoint{3.045595in}{3.195342in}}%
\pgfpathlineto{\pgfqpoint{3.045657in}{3.138685in}}%
\pgfpathlineto{\pgfqpoint{3.046582in}{3.092289in}}%
\pgfpathlineto{\pgfqpoint{3.045981in}{3.294090in}}%
\pgfpathlineto{\pgfqpoint{3.046721in}{3.181629in}}%
\pgfpathlineto{\pgfqpoint{3.047538in}{3.297066in}}%
\pgfpathlineto{\pgfqpoint{3.047275in}{3.093260in}}%
\pgfpathlineto{\pgfqpoint{3.047723in}{3.146387in}}%
\pgfpathlineto{\pgfqpoint{3.048139in}{3.092788in}}%
\pgfpathlineto{\pgfqpoint{3.048555in}{3.245233in}}%
\pgfpathlineto{\pgfqpoint{3.049094in}{3.296406in}}%
\pgfpathlineto{\pgfqpoint{3.048817in}{3.097661in}}%
\pgfpathlineto{\pgfqpoint{3.049634in}{3.203903in}}%
\pgfpathlineto{\pgfqpoint{3.049696in}{3.093475in}}%
\pgfpathlineto{\pgfqpoint{3.050636in}{3.293112in}}%
\pgfpathlineto{\pgfqpoint{3.050728in}{3.182183in}}%
\pgfpathlineto{\pgfqpoint{3.051684in}{3.262112in}}%
\pgfpathlineto{\pgfqpoint{3.051237in}{3.087709in}}%
\pgfpathlineto{\pgfqpoint{3.051823in}{3.192341in}}%
\pgfpathlineto{\pgfqpoint{3.052794in}{3.087392in}}%
\pgfpathlineto{\pgfqpoint{3.052193in}{3.295320in}}%
\pgfpathlineto{\pgfqpoint{3.052933in}{3.181883in}}%
\pgfpathlineto{\pgfqpoint{3.053750in}{3.299169in}}%
\pgfpathlineto{\pgfqpoint{3.053303in}{3.097855in}}%
\pgfpathlineto{\pgfqpoint{3.053919in}{3.151342in}}%
\pgfpathlineto{\pgfqpoint{3.054351in}{3.091282in}}%
\pgfpathlineto{\pgfqpoint{3.054783in}{3.271674in}}%
\pgfpathlineto{\pgfqpoint{3.055060in}{3.120961in}}%
\pgfpathlineto{\pgfqpoint{3.055307in}{3.297249in}}%
\pgfpathlineto{\pgfqpoint{3.055908in}{3.089648in}}%
\pgfpathlineto{\pgfqpoint{3.056370in}{3.190399in}}%
\pgfpathlineto{\pgfqpoint{3.057449in}{3.088178in}}%
\pgfpathlineto{\pgfqpoint{3.056848in}{3.293263in}}%
\pgfpathlineto{\pgfqpoint{3.057480in}{3.129030in}}%
\pgfpathlineto{\pgfqpoint{3.058405in}{3.299522in}}%
\pgfpathlineto{\pgfqpoint{3.058220in}{3.102477in}}%
\pgfpathlineto{\pgfqpoint{3.058590in}{3.143400in}}%
\pgfpathlineto{\pgfqpoint{3.059006in}{3.081749in}}%
\pgfpathlineto{\pgfqpoint{3.058929in}{3.245199in}}%
\pgfpathlineto{\pgfqpoint{3.059407in}{3.201935in}}%
\pgfpathlineto{\pgfqpoint{3.059962in}{3.294623in}}%
\pgfpathlineto{\pgfqpoint{3.060270in}{3.092885in}}%
\pgfpathlineto{\pgfqpoint{3.060501in}{3.227365in}}%
\pgfpathlineto{\pgfqpoint{3.060563in}{3.074609in}}%
\pgfpathlineto{\pgfqpoint{3.061519in}{3.292675in}}%
\pgfpathlineto{\pgfqpoint{3.061611in}{3.172884in}}%
\pgfpathlineto{\pgfqpoint{3.062552in}{3.270476in}}%
\pgfpathlineto{\pgfqpoint{3.062120in}{3.082736in}}%
\pgfpathlineto{\pgfqpoint{3.062706in}{3.177350in}}%
\pgfpathlineto{\pgfqpoint{3.063661in}{3.076755in}}%
\pgfpathlineto{\pgfqpoint{3.063060in}{3.296999in}}%
\pgfpathlineto{\pgfqpoint{3.063800in}{3.178043in}}%
\pgfpathlineto{\pgfqpoint{3.064617in}{3.293077in}}%
\pgfpathlineto{\pgfqpoint{3.064432in}{3.082375in}}%
\pgfpathlineto{\pgfqpoint{3.064864in}{3.154536in}}%
\pgfpathlineto{\pgfqpoint{3.065218in}{3.072141in}}%
\pgfpathlineto{\pgfqpoint{3.065650in}{3.268368in}}%
\pgfpathlineto{\pgfqpoint{3.066005in}{3.117735in}}%
\pgfpathlineto{\pgfqpoint{3.066174in}{3.295751in}}%
\pgfpathlineto{\pgfqpoint{3.066760in}{3.077602in}}%
\pgfpathlineto{\pgfqpoint{3.067130in}{3.157510in}}%
\pgfpathlineto{\pgfqpoint{3.067531in}{3.081708in}}%
\pgfpathlineto{\pgfqpoint{3.067731in}{3.299477in}}%
\pgfpathlineto{\pgfqpoint{3.068101in}{3.204115in}}%
\pgfpathlineto{\pgfqpoint{3.068748in}{3.265566in}}%
\pgfpathlineto{\pgfqpoint{3.068317in}{3.075395in}}%
\pgfpathlineto{\pgfqpoint{3.069057in}{3.115585in}}%
\pgfpathlineto{\pgfqpoint{3.069087in}{3.075355in}}%
\pgfpathlineto{\pgfqpoint{3.069272in}{3.296306in}}%
\pgfpathlineto{\pgfqpoint{3.070136in}{3.154779in}}%
\pgfpathlineto{\pgfqpoint{3.070829in}{3.298453in}}%
\pgfpathlineto{\pgfqpoint{3.070644in}{3.075235in}}%
\pgfpathlineto{\pgfqpoint{3.071246in}{3.159353in}}%
\pgfpathlineto{\pgfqpoint{3.072186in}{3.075847in}}%
\pgfpathlineto{\pgfqpoint{3.071862in}{3.259932in}}%
\pgfpathlineto{\pgfqpoint{3.072325in}{3.182022in}}%
\pgfpathlineto{\pgfqpoint{3.072386in}{3.299388in}}%
\pgfpathlineto{\pgfqpoint{3.072972in}{3.076989in}}%
\pgfpathlineto{\pgfqpoint{3.073434in}{3.222876in}}%
\pgfpathlineto{\pgfqpoint{3.073743in}{3.071521in}}%
\pgfpathlineto{\pgfqpoint{3.073943in}{3.296727in}}%
\pgfpathlineto{\pgfqpoint{3.074560in}{3.162513in}}%
\pgfpathlineto{\pgfqpoint{3.075500in}{3.296921in}}%
\pgfpathlineto{\pgfqpoint{3.075300in}{3.076082in}}%
\pgfpathlineto{\pgfqpoint{3.075654in}{3.178305in}}%
\pgfpathlineto{\pgfqpoint{3.076070in}{3.096967in}}%
\pgfpathlineto{\pgfqpoint{3.076009in}{3.259467in}}%
\pgfpathlineto{\pgfqpoint{3.076795in}{3.131650in}}%
\pgfpathlineto{\pgfqpoint{3.076841in}{3.082850in}}%
\pgfpathlineto{\pgfqpoint{3.077042in}{3.297961in}}%
\pgfpathlineto{\pgfqpoint{3.077889in}{3.133984in}}%
\pgfpathlineto{\pgfqpoint{3.078598in}{3.295711in}}%
\pgfpathlineto{\pgfqpoint{3.078398in}{3.079754in}}%
\pgfpathlineto{\pgfqpoint{3.078999in}{3.171356in}}%
\pgfpathlineto{\pgfqpoint{3.079955in}{3.084727in}}%
\pgfpathlineto{\pgfqpoint{3.079107in}{3.243903in}}%
\pgfpathlineto{\pgfqpoint{3.080094in}{3.196602in}}%
\pgfpathlineto{\pgfqpoint{3.080155in}{3.292315in}}%
\pgfpathlineto{\pgfqpoint{3.080448in}{3.107353in}}%
\pgfpathlineto{\pgfqpoint{3.081204in}{3.218252in}}%
\pgfpathlineto{\pgfqpoint{3.081496in}{3.092939in}}%
\pgfpathlineto{\pgfqpoint{3.081712in}{3.285869in}}%
\pgfpathlineto{\pgfqpoint{3.082329in}{3.170424in}}%
\pgfpathlineto{\pgfqpoint{3.083269in}{3.285205in}}%
\pgfpathlineto{\pgfqpoint{3.083038in}{3.090554in}}%
\pgfpathlineto{\pgfqpoint{3.083423in}{3.183253in}}%
\pgfpathlineto{\pgfqpoint{3.084579in}{3.092082in}}%
\pgfpathlineto{\pgfqpoint{3.083778in}{3.221969in}}%
\pgfpathlineto{\pgfqpoint{3.084595in}{3.094826in}}%
\pgfpathlineto{\pgfqpoint{3.084811in}{3.285846in}}%
\pgfpathlineto{\pgfqpoint{3.085874in}{3.214403in}}%
\pgfpathlineto{\pgfqpoint{3.086136in}{3.095174in}}%
\pgfpathlineto{\pgfqpoint{3.086368in}{3.287968in}}%
\pgfpathlineto{\pgfqpoint{3.087000in}{3.183728in}}%
\pgfpathlineto{\pgfqpoint{3.087924in}{3.280442in}}%
\pgfpathlineto{\pgfqpoint{3.087678in}{3.095023in}}%
\pgfpathlineto{\pgfqpoint{3.088094in}{3.184700in}}%
\pgfpathlineto{\pgfqpoint{3.089235in}{3.094493in}}%
\pgfpathlineto{\pgfqpoint{3.088587in}{3.224736in}}%
\pgfpathlineto{\pgfqpoint{3.089250in}{3.106925in}}%
\pgfpathlineto{\pgfqpoint{3.089481in}{3.280151in}}%
\pgfpathlineto{\pgfqpoint{3.090422in}{3.184583in}}%
\pgfpathlineto{\pgfqpoint{3.090437in}{3.184592in}}%
\pgfpathlineto{\pgfqpoint{3.090792in}{3.100896in}}%
\pgfpathlineto{\pgfqpoint{3.091038in}{3.278229in}}%
\pgfpathlineto{\pgfqpoint{3.091531in}{3.196355in}}%
\pgfpathlineto{\pgfqpoint{3.092580in}{3.272509in}}%
\pgfpathlineto{\pgfqpoint{3.092333in}{3.101582in}}%
\pgfpathlineto{\pgfqpoint{3.092626in}{3.210210in}}%
\pgfpathlineto{\pgfqpoint{3.093720in}{3.114342in}}%
\pgfpathlineto{\pgfqpoint{3.093628in}{3.229801in}}%
\pgfpathlineto{\pgfqpoint{3.093751in}{3.160104in}}%
\pgfpathlineto{\pgfqpoint{3.094137in}{3.275766in}}%
\pgfpathlineto{\pgfqpoint{3.093890in}{3.099760in}}%
\pgfpathlineto{\pgfqpoint{3.094861in}{3.173727in}}%
\pgfpathlineto{\pgfqpoint{3.094876in}{3.174284in}}%
\pgfpathlineto{\pgfqpoint{3.094969in}{3.144481in}}%
\pgfpathlineto{\pgfqpoint{3.095431in}{3.103908in}}%
\pgfpathlineto{\pgfqpoint{3.095693in}{3.274878in}}%
\pgfpathlineto{\pgfqpoint{3.096063in}{3.141180in}}%
\pgfpathlineto{\pgfqpoint{3.097235in}{3.269200in}}%
\pgfpathlineto{\pgfqpoint{3.096988in}{3.098918in}}%
\pgfpathlineto{\pgfqpoint{3.097281in}{3.207095in}}%
\pgfpathlineto{\pgfqpoint{3.097482in}{3.109857in}}%
\pgfpathlineto{\pgfqpoint{3.098283in}{3.235739in}}%
\pgfpathlineto{\pgfqpoint{3.098406in}{3.168234in}}%
\pgfpathlineto{\pgfqpoint{3.098792in}{3.278351in}}%
\pgfpathlineto{\pgfqpoint{3.098530in}{3.098550in}}%
\pgfpathlineto{\pgfqpoint{3.099516in}{3.179170in}}%
\pgfpathlineto{\pgfqpoint{3.099640in}{3.136478in}}%
\pgfpathlineto{\pgfqpoint{3.099809in}{3.217590in}}%
\pgfpathlineto{\pgfqpoint{3.100349in}{3.274814in}}%
\pgfpathlineto{\pgfqpoint{3.100087in}{3.098915in}}%
\pgfpathlineto{\pgfqpoint{3.100888in}{3.189060in}}%
\pgfpathlineto{\pgfqpoint{3.101644in}{3.093023in}}%
\pgfpathlineto{\pgfqpoint{3.101875in}{3.267476in}}%
\pgfpathlineto{\pgfqpoint{3.101890in}{3.272740in}}%
\pgfpathlineto{\pgfqpoint{3.102137in}{3.108254in}}%
\pgfpathlineto{\pgfqpoint{3.102646in}{3.162241in}}%
\pgfpathlineto{\pgfqpoint{3.103185in}{3.090106in}}%
\pgfpathlineto{\pgfqpoint{3.103432in}{3.267626in}}%
\pgfpathlineto{\pgfqpoint{3.103786in}{3.133592in}}%
\pgfpathlineto{\pgfqpoint{3.104742in}{3.087755in}}%
\pgfpathlineto{\pgfqpoint{3.104989in}{3.270048in}}%
\pgfpathlineto{\pgfqpoint{3.105235in}{3.108040in}}%
\pgfpathlineto{\pgfqpoint{3.106160in}{3.185155in}}%
\pgfpathlineto{\pgfqpoint{3.106530in}{3.268559in}}%
\pgfpathlineto{\pgfqpoint{3.106283in}{3.088400in}}%
\pgfpathlineto{\pgfqpoint{3.107116in}{3.156745in}}%
\pgfpathlineto{\pgfqpoint{3.107840in}{3.089860in}}%
\pgfpathlineto{\pgfqpoint{3.107578in}{3.235380in}}%
\pgfpathlineto{\pgfqpoint{3.108025in}{3.193213in}}%
\pgfpathlineto{\pgfqpoint{3.108087in}{3.274969in}}%
\pgfpathlineto{\pgfqpoint{3.108349in}{3.113994in}}%
\pgfpathlineto{\pgfqpoint{3.109151in}{3.218473in}}%
\pgfpathlineto{\pgfqpoint{3.109382in}{3.095050in}}%
\pgfpathlineto{\pgfqpoint{3.109628in}{3.271214in}}%
\pgfpathlineto{\pgfqpoint{3.110276in}{3.184509in}}%
\pgfpathlineto{\pgfqpoint{3.111185in}{3.273079in}}%
\pgfpathlineto{\pgfqpoint{3.110939in}{3.091983in}}%
\pgfpathlineto{\pgfqpoint{3.111370in}{3.167044in}}%
\pgfpathlineto{\pgfqpoint{3.112496in}{3.099266in}}%
\pgfpathlineto{\pgfqpoint{3.112234in}{3.239776in}}%
\pgfpathlineto{\pgfqpoint{3.112511in}{3.125501in}}%
\pgfpathlineto{\pgfqpoint{3.112742in}{3.267575in}}%
\pgfpathlineto{\pgfqpoint{3.113004in}{3.115511in}}%
\pgfpathlineto{\pgfqpoint{3.113667in}{3.163307in}}%
\pgfpathlineto{\pgfqpoint{3.114037in}{3.101328in}}%
\pgfpathlineto{\pgfqpoint{3.114284in}{3.267209in}}%
\pgfpathlineto{\pgfqpoint{3.114746in}{3.166110in}}%
\pgfpathlineto{\pgfqpoint{3.115841in}{3.265881in}}%
\pgfpathlineto{\pgfqpoint{3.115594in}{3.103728in}}%
\pgfpathlineto{\pgfqpoint{3.115871in}{3.214377in}}%
\pgfpathlineto{\pgfqpoint{3.116103in}{3.113436in}}%
\pgfpathlineto{\pgfqpoint{3.116873in}{3.243411in}}%
\pgfpathlineto{\pgfqpoint{3.116981in}{3.168965in}}%
\pgfpathlineto{\pgfqpoint{3.117382in}{3.267313in}}%
\pgfpathlineto{\pgfqpoint{3.117136in}{3.107503in}}%
\pgfpathlineto{\pgfqpoint{3.118091in}{3.176074in}}%
\pgfpathlineto{\pgfqpoint{3.118692in}{3.112588in}}%
\pgfpathlineto{\pgfqpoint{3.118939in}{3.271236in}}%
\pgfpathlineto{\pgfqpoint{3.119247in}{3.127419in}}%
\pgfpathlineto{\pgfqpoint{3.119771in}{3.117932in}}%
\pgfpathlineto{\pgfqpoint{3.119972in}{3.247322in}}%
\pgfpathlineto{\pgfqpoint{3.120311in}{3.139885in}}%
\pgfpathlineto{\pgfqpoint{3.120481in}{3.270161in}}%
\pgfpathlineto{\pgfqpoint{3.121328in}{3.111102in}}%
\pgfpathlineto{\pgfqpoint{3.121421in}{3.156021in}}%
\pgfpathlineto{\pgfqpoint{3.121452in}{3.151028in}}%
\pgfpathlineto{\pgfqpoint{3.121498in}{3.229501in}}%
\pgfpathlineto{\pgfqpoint{3.122037in}{3.270158in}}%
\pgfpathlineto{\pgfqpoint{3.122361in}{3.109389in}}%
\pgfpathlineto{\pgfqpoint{3.122577in}{3.224042in}}%
\pgfpathlineto{\pgfqpoint{3.122870in}{3.111033in}}%
\pgfpathlineto{\pgfqpoint{3.123579in}{3.269884in}}%
\pgfpathlineto{\pgfqpoint{3.123702in}{3.193117in}}%
\pgfpathlineto{\pgfqpoint{3.124103in}{3.247065in}}%
\pgfpathlineto{\pgfqpoint{3.124427in}{3.106695in}}%
\pgfpathlineto{\pgfqpoint{3.124689in}{3.136224in}}%
\pgfpathlineto{\pgfqpoint{3.125460in}{3.105439in}}%
\pgfpathlineto{\pgfqpoint{3.125136in}{3.267570in}}%
\pgfpathlineto{\pgfqpoint{3.125614in}{3.201816in}}%
\pgfpathlineto{\pgfqpoint{3.126677in}{3.269494in}}%
\pgfpathlineto{\pgfqpoint{3.125968in}{3.107806in}}%
\pgfpathlineto{\pgfqpoint{3.126724in}{3.212075in}}%
\pgfpathlineto{\pgfqpoint{3.127001in}{3.110215in}}%
\pgfpathlineto{\pgfqpoint{3.127201in}{3.249221in}}%
\pgfpathlineto{\pgfqpoint{3.127833in}{3.185045in}}%
\pgfpathlineto{\pgfqpoint{3.128234in}{3.267010in}}%
\pgfpathlineto{\pgfqpoint{3.128558in}{3.109951in}}%
\pgfpathlineto{\pgfqpoint{3.128912in}{3.183612in}}%
\pgfpathlineto{\pgfqpoint{3.130053in}{3.113401in}}%
\pgfpathlineto{\pgfqpoint{3.129776in}{3.264705in}}%
\pgfpathlineto{\pgfqpoint{3.130069in}{3.116757in}}%
\pgfpathlineto{\pgfqpoint{3.130300in}{3.248871in}}%
\pgfpathlineto{\pgfqpoint{3.130115in}{3.113071in}}%
\pgfpathlineto{\pgfqpoint{3.131225in}{3.176908in}}%
\pgfpathlineto{\pgfqpoint{3.131610in}{3.111090in}}%
\pgfpathlineto{\pgfqpoint{3.131333in}{3.265608in}}%
\pgfpathlineto{\pgfqpoint{3.132319in}{3.178142in}}%
\pgfpathlineto{\pgfqpoint{3.132889in}{3.261035in}}%
\pgfpathlineto{\pgfqpoint{3.133167in}{3.111072in}}%
\pgfpathlineto{\pgfqpoint{3.133444in}{3.214791in}}%
\pgfpathlineto{\pgfqpoint{3.133722in}{3.114569in}}%
\pgfpathlineto{\pgfqpoint{3.134431in}{3.254349in}}%
\pgfpathlineto{\pgfqpoint{3.134554in}{3.181182in}}%
\pgfpathlineto{\pgfqpoint{3.134955in}{3.237621in}}%
\pgfpathlineto{\pgfqpoint{3.134708in}{3.104723in}}%
\pgfpathlineto{\pgfqpoint{3.135633in}{3.165426in}}%
\pgfpathlineto{\pgfqpoint{3.136265in}{3.109510in}}%
\pgfpathlineto{\pgfqpoint{3.135988in}{3.254557in}}%
\pgfpathlineto{\pgfqpoint{3.136774in}{3.123886in}}%
\pgfpathlineto{\pgfqpoint{3.137529in}{3.258051in}}%
\pgfpathlineto{\pgfqpoint{3.137807in}{3.111323in}}%
\pgfpathlineto{\pgfqpoint{3.137838in}{3.125864in}}%
\pgfpathlineto{\pgfqpoint{3.137853in}{3.112772in}}%
\pgfpathlineto{\pgfqpoint{3.138053in}{3.246209in}}%
\pgfpathlineto{\pgfqpoint{3.138901in}{3.125773in}}%
\pgfpathlineto{\pgfqpoint{3.139086in}{3.256929in}}%
\pgfpathlineto{\pgfqpoint{3.139364in}{3.110666in}}%
\pgfpathlineto{\pgfqpoint{3.140011in}{3.157226in}}%
\pgfpathlineto{\pgfqpoint{3.140951in}{3.110727in}}%
\pgfpathlineto{\pgfqpoint{3.140643in}{3.256966in}}%
\pgfpathlineto{\pgfqpoint{3.141090in}{3.163923in}}%
\pgfpathlineto{\pgfqpoint{3.142185in}{3.257507in}}%
\pgfpathlineto{\pgfqpoint{3.141476in}{3.114420in}}%
\pgfpathlineto{\pgfqpoint{3.142215in}{3.236647in}}%
\pgfpathlineto{\pgfqpoint{3.142508in}{3.108888in}}%
\pgfpathlineto{\pgfqpoint{3.142709in}{3.246569in}}%
\pgfpathlineto{\pgfqpoint{3.143341in}{3.181819in}}%
\pgfpathlineto{\pgfqpoint{3.143742in}{3.253166in}}%
\pgfpathlineto{\pgfqpoint{3.144050in}{3.108621in}}%
\pgfpathlineto{\pgfqpoint{3.144435in}{3.162972in}}%
\pgfpathlineto{\pgfqpoint{3.144574in}{3.113321in}}%
\pgfpathlineto{\pgfqpoint{3.145283in}{3.256386in}}%
\pgfpathlineto{\pgfqpoint{3.145560in}{3.116709in}}%
\pgfpathlineto{\pgfqpoint{3.145807in}{3.246597in}}%
\pgfpathlineto{\pgfqpoint{3.145607in}{3.108520in}}%
\pgfpathlineto{\pgfqpoint{3.146717in}{3.185888in}}%
\pgfpathlineto{\pgfqpoint{3.147148in}{3.113244in}}%
\pgfpathlineto{\pgfqpoint{3.146840in}{3.254069in}}%
\pgfpathlineto{\pgfqpoint{3.147811in}{3.165245in}}%
\pgfpathlineto{\pgfqpoint{3.148381in}{3.254200in}}%
\pgfpathlineto{\pgfqpoint{3.148705in}{3.109666in}}%
\pgfpathlineto{\pgfqpoint{3.148936in}{3.233820in}}%
\pgfpathlineto{\pgfqpoint{3.149229in}{3.117277in}}%
\pgfpathlineto{\pgfqpoint{3.149938in}{3.253385in}}%
\pgfpathlineto{\pgfqpoint{3.150062in}{3.195453in}}%
\pgfpathlineto{\pgfqpoint{3.150447in}{3.240305in}}%
\pgfpathlineto{\pgfqpoint{3.150262in}{3.114614in}}%
\pgfpathlineto{\pgfqpoint{3.151125in}{3.193964in}}%
\pgfpathlineto{\pgfqpoint{3.151803in}{3.112497in}}%
\pgfpathlineto{\pgfqpoint{3.151480in}{3.253034in}}%
\pgfpathlineto{\pgfqpoint{3.152250in}{3.136022in}}%
\pgfpathlineto{\pgfqpoint{3.152328in}{3.116522in}}%
\pgfpathlineto{\pgfqpoint{3.152374in}{3.180363in}}%
\pgfpathlineto{\pgfqpoint{3.152466in}{3.173060in}}%
\pgfpathlineto{\pgfqpoint{3.153037in}{3.251691in}}%
\pgfpathlineto{\pgfqpoint{3.153360in}{3.112376in}}%
\pgfpathlineto{\pgfqpoint{3.153592in}{3.232329in}}%
\pgfpathlineto{\pgfqpoint{3.154393in}{3.118950in}}%
\pgfpathlineto{\pgfqpoint{3.154578in}{3.254456in}}%
\pgfpathlineto{\pgfqpoint{3.154717in}{3.198495in}}%
\pgfpathlineto{\pgfqpoint{3.155102in}{3.243723in}}%
\pgfpathlineto{\pgfqpoint{3.154902in}{3.116334in}}%
\pgfpathlineto{\pgfqpoint{3.155780in}{3.182531in}}%
\pgfpathlineto{\pgfqpoint{3.156459in}{3.110886in}}%
\pgfpathlineto{\pgfqpoint{3.156135in}{3.251576in}}%
\pgfpathlineto{\pgfqpoint{3.156921in}{3.130581in}}%
\pgfpathlineto{\pgfqpoint{3.157677in}{3.250168in}}%
\pgfpathlineto{\pgfqpoint{3.157939in}{3.114155in}}%
\pgfpathlineto{\pgfqpoint{3.157985in}{3.133103in}}%
\pgfpathlineto{\pgfqpoint{3.158000in}{3.114230in}}%
\pgfpathlineto{\pgfqpoint{3.158201in}{3.241148in}}%
\pgfpathlineto{\pgfqpoint{3.159064in}{3.149896in}}%
\pgfpathlineto{\pgfqpoint{3.159233in}{3.252877in}}%
\pgfpathlineto{\pgfqpoint{3.159557in}{3.112808in}}%
\pgfpathlineto{\pgfqpoint{3.160158in}{3.154942in}}%
\pgfpathlineto{\pgfqpoint{3.161052in}{3.114294in}}%
\pgfpathlineto{\pgfqpoint{3.160775in}{3.252443in}}%
\pgfpathlineto{\pgfqpoint{3.161237in}{3.149088in}}%
\pgfpathlineto{\pgfqpoint{3.162332in}{3.250446in}}%
\pgfpathlineto{\pgfqpoint{3.161623in}{3.116726in}}%
\pgfpathlineto{\pgfqpoint{3.162378in}{3.217492in}}%
\pgfpathlineto{\pgfqpoint{3.162655in}{3.112337in}}%
\pgfpathlineto{\pgfqpoint{3.162840in}{3.239787in}}%
\pgfpathlineto{\pgfqpoint{3.163488in}{3.193328in}}%
\pgfpathlineto{\pgfqpoint{3.163873in}{3.251266in}}%
\pgfpathlineto{\pgfqpoint{3.164151in}{3.118011in}}%
\pgfpathlineto{\pgfqpoint{3.164567in}{3.180401in}}%
\pgfpathlineto{\pgfqpoint{3.165415in}{3.247450in}}%
\pgfpathlineto{\pgfqpoint{3.165754in}{3.116186in}}%
\pgfpathlineto{\pgfqpoint{3.165939in}{3.235187in}}%
\pgfpathlineto{\pgfqpoint{3.166879in}{3.145895in}}%
\pgfpathlineto{\pgfqpoint{3.166956in}{3.245771in}}%
\pgfpathlineto{\pgfqpoint{3.167234in}{3.113291in}}%
\pgfpathlineto{\pgfqpoint{3.168066in}{3.187837in}}%
\pgfpathlineto{\pgfqpoint{3.168791in}{3.113625in}}%
\pgfpathlineto{\pgfqpoint{3.168513in}{3.247261in}}%
\pgfpathlineto{\pgfqpoint{3.169161in}{3.196682in}}%
\pgfpathlineto{\pgfqpoint{3.170055in}{3.246600in}}%
\pgfpathlineto{\pgfqpoint{3.169823in}{3.114980in}}%
\pgfpathlineto{\pgfqpoint{3.170224in}{3.197923in}}%
\pgfpathlineto{\pgfqpoint{3.170332in}{3.110838in}}%
\pgfpathlineto{\pgfqpoint{3.170579in}{3.236034in}}%
\pgfpathlineto{\pgfqpoint{3.171365in}{3.116955in}}%
\pgfpathlineto{\pgfqpoint{3.171581in}{3.227673in}}%
\pgfpathlineto{\pgfqpoint{3.171596in}{3.246587in}}%
\pgfpathlineto{\pgfqpoint{3.171889in}{3.107855in}}%
\pgfpathlineto{\pgfqpoint{3.172660in}{3.207591in}}%
\pgfpathlineto{\pgfqpoint{3.173430in}{3.108873in}}%
\pgfpathlineto{\pgfqpoint{3.173153in}{3.246106in}}%
\pgfpathlineto{\pgfqpoint{3.173785in}{3.181252in}}%
\pgfpathlineto{\pgfqpoint{3.174694in}{3.250743in}}%
\pgfpathlineto{\pgfqpoint{3.174463in}{3.114880in}}%
\pgfpathlineto{\pgfqpoint{3.174879in}{3.180418in}}%
\pgfpathlineto{\pgfqpoint{3.174972in}{3.110644in}}%
\pgfpathlineto{\pgfqpoint{3.175203in}{3.247025in}}%
\pgfpathlineto{\pgfqpoint{3.176036in}{3.129374in}}%
\pgfpathlineto{\pgfqpoint{3.176128in}{3.182482in}}%
\pgfpathlineto{\pgfqpoint{3.176066in}{3.125674in}}%
\pgfpathlineto{\pgfqpoint{3.176205in}{3.182145in}}%
\pgfpathlineto{\pgfqpoint{3.176236in}{3.253012in}}%
\pgfpathlineto{\pgfqpoint{3.176529in}{3.108734in}}%
\pgfpathlineto{\pgfqpoint{3.177300in}{3.204735in}}%
\pgfpathlineto{\pgfqpoint{3.178070in}{3.104458in}}%
\pgfpathlineto{\pgfqpoint{3.177793in}{3.261577in}}%
\pgfpathlineto{\pgfqpoint{3.178409in}{3.187628in}}%
\pgfpathlineto{\pgfqpoint{3.179334in}{3.265992in}}%
\pgfpathlineto{\pgfqpoint{3.179103in}{3.108077in}}%
\pgfpathlineto{\pgfqpoint{3.179504in}{3.187649in}}%
\pgfpathlineto{\pgfqpoint{3.179627in}{3.100572in}}%
\pgfpathlineto{\pgfqpoint{3.180367in}{3.265188in}}%
\pgfpathlineto{\pgfqpoint{3.180691in}{3.113535in}}%
\pgfpathlineto{\pgfqpoint{3.181400in}{3.278203in}}%
\pgfpathlineto{\pgfqpoint{3.181169in}{3.103357in}}%
\pgfpathlineto{\pgfqpoint{3.181832in}{3.138088in}}%
\pgfpathlineto{\pgfqpoint{3.182756in}{3.105346in}}%
\pgfpathlineto{\pgfqpoint{3.182433in}{3.288366in}}%
\pgfpathlineto{\pgfqpoint{3.182895in}{3.148123in}}%
\pgfpathlineto{\pgfqpoint{3.183974in}{3.299347in}}%
\pgfpathlineto{\pgfqpoint{3.183743in}{3.106414in}}%
\pgfpathlineto{\pgfqpoint{3.184005in}{3.217727in}}%
\pgfpathlineto{\pgfqpoint{3.184298in}{3.100253in}}%
\pgfpathlineto{\pgfqpoint{3.185007in}{3.294601in}}%
\pgfpathlineto{\pgfqpoint{3.185115in}{3.202957in}}%
\pgfpathlineto{\pgfqpoint{3.185516in}{3.303520in}}%
\pgfpathlineto{\pgfqpoint{3.185839in}{3.096988in}}%
\pgfpathlineto{\pgfqpoint{3.186209in}{3.181886in}}%
\pgfpathlineto{\pgfqpoint{3.187396in}{3.092249in}}%
\pgfpathlineto{\pgfqpoint{3.187073in}{3.302437in}}%
\pgfpathlineto{\pgfqpoint{3.187412in}{3.122575in}}%
\pgfpathlineto{\pgfqpoint{3.187581in}{3.307581in}}%
\pgfpathlineto{\pgfqpoint{3.188429in}{3.096531in}}%
\pgfpathlineto{\pgfqpoint{3.188522in}{3.129975in}}%
\pgfpathlineto{\pgfqpoint{3.188938in}{3.085478in}}%
\pgfpathlineto{\pgfqpoint{3.188614in}{3.311119in}}%
\pgfpathlineto{\pgfqpoint{3.189585in}{3.118312in}}%
\pgfpathlineto{\pgfqpoint{3.190156in}{3.310432in}}%
\pgfpathlineto{\pgfqpoint{3.190479in}{3.086680in}}%
\pgfpathlineto{\pgfqpoint{3.190710in}{3.185237in}}%
\pgfpathlineto{\pgfqpoint{3.191512in}{3.090560in}}%
\pgfpathlineto{\pgfqpoint{3.191697in}{3.311883in}}%
\pgfpathlineto{\pgfqpoint{3.191805in}{3.183695in}}%
\pgfpathlineto{\pgfqpoint{3.192221in}{3.310055in}}%
\pgfpathlineto{\pgfqpoint{3.192021in}{3.086388in}}%
\pgfpathlineto{\pgfqpoint{3.192899in}{3.193166in}}%
\pgfpathlineto{\pgfqpoint{3.193054in}{3.089049in}}%
\pgfpathlineto{\pgfqpoint{3.193763in}{3.314107in}}%
\pgfpathlineto{\pgfqpoint{3.194056in}{3.102780in}}%
\pgfpathlineto{\pgfqpoint{3.194086in}{3.093058in}}%
\pgfpathlineto{\pgfqpoint{3.194117in}{3.172434in}}%
\pgfpathlineto{\pgfqpoint{3.194795in}{3.310340in}}%
\pgfpathlineto{\pgfqpoint{3.194595in}{3.091883in}}%
\pgfpathlineto{\pgfqpoint{3.195196in}{3.171029in}}%
\pgfpathlineto{\pgfqpoint{3.196137in}{3.094130in}}%
\pgfpathlineto{\pgfqpoint{3.195304in}{3.315571in}}%
\pgfpathlineto{\pgfqpoint{3.196291in}{3.163181in}}%
\pgfpathlineto{\pgfqpoint{3.196337in}{3.312449in}}%
\pgfpathlineto{\pgfqpoint{3.196661in}{3.086844in}}%
\pgfpathlineto{\pgfqpoint{3.197401in}{3.202993in}}%
\pgfpathlineto{\pgfqpoint{3.198202in}{3.083776in}}%
\pgfpathlineto{\pgfqpoint{3.197878in}{3.308253in}}%
\pgfpathlineto{\pgfqpoint{3.198510in}{3.202425in}}%
\pgfpathlineto{\pgfqpoint{3.198911in}{3.307755in}}%
\pgfpathlineto{\pgfqpoint{3.199235in}{3.082803in}}%
\pgfpathlineto{\pgfqpoint{3.199605in}{3.179236in}}%
\pgfpathlineto{\pgfqpoint{3.199744in}{3.084982in}}%
\pgfpathlineto{\pgfqpoint{3.199944in}{3.307955in}}%
\pgfpathlineto{\pgfqpoint{3.200792in}{3.101802in}}%
\pgfpathlineto{\pgfqpoint{3.201485in}{3.312140in}}%
\pgfpathlineto{\pgfqpoint{3.201408in}{3.091006in}}%
\pgfpathlineto{\pgfqpoint{3.201902in}{3.142655in}}%
\pgfpathlineto{\pgfqpoint{3.202950in}{3.088202in}}%
\pgfpathlineto{\pgfqpoint{3.201994in}{3.306365in}}%
\pgfpathlineto{\pgfqpoint{3.202981in}{3.138528in}}%
\pgfpathlineto{\pgfqpoint{3.203027in}{3.309530in}}%
\pgfpathlineto{\pgfqpoint{3.203983in}{3.091354in}}%
\pgfpathlineto{\pgfqpoint{3.204091in}{3.234104in}}%
\pgfpathlineto{\pgfqpoint{3.205015in}{3.082583in}}%
\pgfpathlineto{\pgfqpoint{3.205093in}{3.308204in}}%
\pgfpathlineto{\pgfqpoint{3.205200in}{3.176163in}}%
\pgfpathlineto{\pgfqpoint{3.205601in}{3.303507in}}%
\pgfpathlineto{\pgfqpoint{3.206048in}{3.074945in}}%
\pgfpathlineto{\pgfqpoint{3.206295in}{3.182365in}}%
\pgfpathlineto{\pgfqpoint{3.206557in}{3.077501in}}%
\pgfpathlineto{\pgfqpoint{3.206634in}{3.310173in}}%
\pgfpathlineto{\pgfqpoint{3.207482in}{3.092437in}}%
\pgfpathlineto{\pgfqpoint{3.208175in}{3.316664in}}%
\pgfpathlineto{\pgfqpoint{3.207590in}{3.072954in}}%
\pgfpathlineto{\pgfqpoint{3.208592in}{3.157867in}}%
\pgfpathlineto{\pgfqpoint{3.209131in}{3.074194in}}%
\pgfpathlineto{\pgfqpoint{3.209208in}{3.315551in}}%
\pgfpathlineto{\pgfqpoint{3.209686in}{3.185670in}}%
\pgfpathlineto{\pgfqpoint{3.210750in}{3.318297in}}%
\pgfpathlineto{\pgfqpoint{3.210688in}{3.073488in}}%
\pgfpathlineto{\pgfqpoint{3.210796in}{3.191876in}}%
\pgfpathlineto{\pgfqpoint{3.211197in}{3.072241in}}%
\pgfpathlineto{\pgfqpoint{3.211274in}{3.320130in}}%
\pgfpathlineto{\pgfqpoint{3.211890in}{3.173309in}}%
\pgfpathlineto{\pgfqpoint{3.212815in}{3.325153in}}%
\pgfpathlineto{\pgfqpoint{3.212738in}{3.068690in}}%
\pgfpathlineto{\pgfqpoint{3.212985in}{3.172663in}}%
\pgfpathlineto{\pgfqpoint{3.213771in}{3.064652in}}%
\pgfpathlineto{\pgfqpoint{3.213848in}{3.321519in}}%
\pgfpathlineto{\pgfqpoint{3.214156in}{3.085873in}}%
\pgfpathlineto{\pgfqpoint{3.214804in}{3.066761in}}%
\pgfpathlineto{\pgfqpoint{3.214357in}{3.328514in}}%
\pgfpathlineto{\pgfqpoint{3.214835in}{3.184159in}}%
\pgfpathlineto{\pgfqpoint{3.215898in}{3.333994in}}%
\pgfpathlineto{\pgfqpoint{3.215837in}{3.062186in}}%
\pgfpathlineto{\pgfqpoint{3.215945in}{3.186927in}}%
\pgfpathlineto{\pgfqpoint{3.216854in}{3.060394in}}%
\pgfpathlineto{\pgfqpoint{3.216931in}{3.330710in}}%
\pgfpathlineto{\pgfqpoint{3.217039in}{3.173182in}}%
\pgfpathlineto{\pgfqpoint{3.217440in}{3.334593in}}%
\pgfpathlineto{\pgfqpoint{3.217378in}{3.057371in}}%
\pgfpathlineto{\pgfqpoint{3.218133in}{3.162292in}}%
\pgfpathlineto{\pgfqpoint{3.218920in}{3.055484in}}%
\pgfpathlineto{\pgfqpoint{3.218981in}{3.340373in}}%
\pgfpathlineto{\pgfqpoint{3.219290in}{3.090734in}}%
\pgfpathlineto{\pgfqpoint{3.219952in}{3.053655in}}%
\pgfpathlineto{\pgfqpoint{3.219505in}{3.337607in}}%
\pgfpathlineto{\pgfqpoint{3.219983in}{3.206003in}}%
\pgfpathlineto{\pgfqpoint{3.220014in}{3.341192in}}%
\pgfpathlineto{\pgfqpoint{3.220461in}{3.048875in}}%
\pgfpathlineto{\pgfqpoint{3.221078in}{3.251462in}}%
\pgfpathlineto{\pgfqpoint{3.221494in}{3.046698in}}%
\pgfpathlineto{\pgfqpoint{3.222080in}{3.345371in}}%
\pgfpathlineto{\pgfqpoint{3.222188in}{3.168135in}}%
\pgfpathlineto{\pgfqpoint{3.222588in}{3.348195in}}%
\pgfpathlineto{\pgfqpoint{3.223035in}{3.049715in}}%
\pgfpathlineto{\pgfqpoint{3.223282in}{3.162940in}}%
\pgfpathlineto{\pgfqpoint{3.224068in}{3.050685in}}%
\pgfpathlineto{\pgfqpoint{3.223621in}{3.356825in}}%
\pgfpathlineto{\pgfqpoint{3.224423in}{3.099568in}}%
\pgfpathlineto{\pgfqpoint{3.224592in}{3.062434in}}%
\pgfpathlineto{\pgfqpoint{3.224654in}{3.345873in}}%
\pgfpathlineto{\pgfqpoint{3.225502in}{3.076250in}}%
\pgfpathlineto{\pgfqpoint{3.226195in}{3.367163in}}%
\pgfpathlineto{\pgfqpoint{3.226134in}{3.050385in}}%
\pgfpathlineto{\pgfqpoint{3.226612in}{3.168694in}}%
\pgfpathlineto{\pgfqpoint{3.226642in}{3.045942in}}%
\pgfpathlineto{\pgfqpoint{3.227228in}{3.367969in}}%
\pgfpathlineto{\pgfqpoint{3.227706in}{3.191203in}}%
\pgfpathlineto{\pgfqpoint{3.228770in}{3.382698in}}%
\pgfpathlineto{\pgfqpoint{3.228708in}{3.042229in}}%
\pgfpathlineto{\pgfqpoint{3.228816in}{3.209077in}}%
\pgfpathlineto{\pgfqpoint{3.229741in}{3.037057in}}%
\pgfpathlineto{\pgfqpoint{3.229802in}{3.389645in}}%
\pgfpathlineto{\pgfqpoint{3.229926in}{3.205398in}}%
\pgfpathlineto{\pgfqpoint{3.230835in}{3.394375in}}%
\pgfpathlineto{\pgfqpoint{3.230250in}{3.039686in}}%
\pgfpathlineto{\pgfqpoint{3.231005in}{3.180329in}}%
\pgfpathlineto{\pgfqpoint{3.231282in}{3.031326in}}%
\pgfpathlineto{\pgfqpoint{3.231868in}{3.402857in}}%
\pgfpathlineto{\pgfqpoint{3.232130in}{3.103859in}}%
\pgfpathlineto{\pgfqpoint{3.232824in}{3.032279in}}%
\pgfpathlineto{\pgfqpoint{3.232377in}{3.406940in}}%
\pgfpathlineto{\pgfqpoint{3.233240in}{3.099176in}}%
\pgfpathlineto{\pgfqpoint{3.233410in}{3.414109in}}%
\pgfpathlineto{\pgfqpoint{3.233857in}{3.027511in}}%
\pgfpathlineto{\pgfqpoint{3.234350in}{3.099412in}}%
\pgfpathlineto{\pgfqpoint{3.234889in}{3.020426in}}%
\pgfpathlineto{\pgfqpoint{3.234951in}{3.424995in}}%
\pgfpathlineto{\pgfqpoint{3.235429in}{3.152614in}}%
\pgfpathlineto{\pgfqpoint{3.236493in}{3.433860in}}%
\pgfpathlineto{\pgfqpoint{3.235922in}{3.012204in}}%
\pgfpathlineto{\pgfqpoint{3.236539in}{3.246086in}}%
\pgfpathlineto{\pgfqpoint{3.237464in}{3.006489in}}%
\pgfpathlineto{\pgfqpoint{3.237525in}{3.445138in}}%
\pgfpathlineto{\pgfqpoint{3.237649in}{3.203021in}}%
\pgfpathlineto{\pgfqpoint{3.238558in}{3.459919in}}%
\pgfpathlineto{\pgfqpoint{3.238496in}{3.006061in}}%
\pgfpathlineto{\pgfqpoint{3.238728in}{3.171228in}}%
\pgfpathlineto{\pgfqpoint{3.239005in}{3.000438in}}%
\pgfpathlineto{\pgfqpoint{3.239591in}{3.473694in}}%
\pgfpathlineto{\pgfqpoint{3.239853in}{3.104158in}}%
\pgfpathlineto{\pgfqpoint{3.240547in}{2.995396in}}%
\pgfpathlineto{\pgfqpoint{3.240100in}{3.474294in}}%
\pgfpathlineto{\pgfqpoint{3.240963in}{3.078567in}}%
\pgfpathlineto{\pgfqpoint{3.241132in}{3.490041in}}%
\pgfpathlineto{\pgfqpoint{3.241579in}{2.990678in}}%
\pgfpathlineto{\pgfqpoint{3.242057in}{3.195998in}}%
\pgfpathlineto{\pgfqpoint{3.242612in}{2.982564in}}%
\pgfpathlineto{\pgfqpoint{3.242674in}{3.506274in}}%
\pgfpathlineto{\pgfqpoint{3.243152in}{3.204185in}}%
\pgfpathlineto{\pgfqpoint{3.244215in}{3.524459in}}%
\pgfpathlineto{\pgfqpoint{3.244154in}{2.968915in}}%
\pgfpathlineto{\pgfqpoint{3.244262in}{3.226095in}}%
\pgfpathlineto{\pgfqpoint{3.245186in}{2.966794in}}%
\pgfpathlineto{\pgfqpoint{3.245248in}{3.531207in}}%
\pgfpathlineto{\pgfqpoint{3.245356in}{3.169096in}}%
\pgfpathlineto{\pgfqpoint{3.246281in}{3.538776in}}%
\pgfpathlineto{\pgfqpoint{3.245695in}{2.961351in}}%
\pgfpathlineto{\pgfqpoint{3.246451in}{3.127290in}}%
\pgfpathlineto{\pgfqpoint{3.247237in}{2.949969in}}%
\pgfpathlineto{\pgfqpoint{3.246790in}{3.538995in}}%
\pgfpathlineto{\pgfqpoint{3.247545in}{3.184708in}}%
\pgfpathlineto{\pgfqpoint{3.248269in}{2.944586in}}%
\pgfpathlineto{\pgfqpoint{3.247822in}{3.541987in}}%
\pgfpathlineto{\pgfqpoint{3.248670in}{2.998718in}}%
\pgfpathlineto{\pgfqpoint{3.248855in}{3.526391in}}%
\pgfpathlineto{\pgfqpoint{3.248778in}{2.949810in}}%
\pgfpathlineto{\pgfqpoint{3.249780in}{3.113895in}}%
\pgfpathlineto{\pgfqpoint{3.249811in}{2.941104in}}%
\pgfpathlineto{\pgfqpoint{3.249873in}{3.502911in}}%
\pgfpathlineto{\pgfqpoint{3.250859in}{3.068358in}}%
\pgfpathlineto{\pgfqpoint{3.250905in}{3.495962in}}%
\pgfpathlineto{\pgfqpoint{3.251352in}{2.940679in}}%
\pgfpathlineto{\pgfqpoint{3.251969in}{3.267141in}}%
\pgfpathlineto{\pgfqpoint{3.252385in}{2.952195in}}%
\pgfpathlineto{\pgfqpoint{3.252447in}{3.447681in}}%
\pgfpathlineto{\pgfqpoint{3.253079in}{3.250767in}}%
\pgfpathlineto{\pgfqpoint{3.253480in}{3.393323in}}%
\pgfpathlineto{\pgfqpoint{3.253403in}{2.969852in}}%
\pgfpathlineto{\pgfqpoint{3.254158in}{3.185799in}}%
\pgfpathlineto{\pgfqpoint{3.254435in}{2.978435in}}%
\pgfpathlineto{\pgfqpoint{3.254497in}{3.343256in}}%
\pgfpathlineto{\pgfqpoint{3.255252in}{3.221891in}}%
\pgfpathlineto{\pgfqpoint{3.255530in}{3.283069in}}%
\pgfpathlineto{\pgfqpoint{3.255468in}{3.005134in}}%
\pgfpathlineto{\pgfqpoint{3.256331in}{3.166595in}}%
\pgfpathlineto{\pgfqpoint{3.256486in}{3.023315in}}%
\pgfpathlineto{\pgfqpoint{3.256671in}{3.273192in}}%
\pgfpathlineto{\pgfqpoint{3.257426in}{3.174684in}}%
\pgfpathlineto{\pgfqpoint{3.258227in}{3.304594in}}%
\pgfpathlineto{\pgfqpoint{3.257518in}{3.032128in}}%
\pgfpathlineto{\pgfqpoint{3.258505in}{3.167336in}}%
\pgfpathlineto{\pgfqpoint{3.259584in}{3.029706in}}%
\pgfpathlineto{\pgfqpoint{3.258752in}{3.262972in}}%
\pgfpathlineto{\pgfqpoint{3.259615in}{3.147831in}}%
\pgfpathlineto{\pgfqpoint{3.259646in}{3.253411in}}%
\pgfpathlineto{\pgfqpoint{3.260108in}{3.069293in}}%
\pgfpathlineto{\pgfqpoint{3.260725in}{3.179142in}}%
\pgfpathlineto{\pgfqpoint{3.261141in}{3.061738in}}%
\pgfpathlineto{\pgfqpoint{3.261588in}{3.271634in}}%
\pgfpathlineto{\pgfqpoint{3.261835in}{3.158894in}}%
\pgfpathlineto{\pgfqpoint{3.262097in}{3.254783in}}%
\pgfpathlineto{\pgfqpoint{3.262019in}{3.110140in}}%
\pgfpathlineto{\pgfqpoint{3.262158in}{3.113748in}}%
\pgfpathlineto{\pgfqpoint{3.262698in}{3.070423in}}%
\pgfpathlineto{\pgfqpoint{3.263129in}{3.260122in}}%
\pgfpathlineto{\pgfqpoint{3.263253in}{3.104798in}}%
\pgfpathlineto{\pgfqpoint{3.263653in}{3.250493in}}%
\pgfpathlineto{\pgfqpoint{3.263731in}{3.062730in}}%
\pgfpathlineto{\pgfqpoint{3.264378in}{3.135574in}}%
\pgfpathlineto{\pgfqpoint{3.264917in}{3.048030in}}%
\pgfpathlineto{\pgfqpoint{3.265195in}{3.256456in}}%
\pgfpathlineto{\pgfqpoint{3.265472in}{3.134760in}}%
\pgfpathlineto{\pgfqpoint{3.266243in}{3.273546in}}%
\pgfpathlineto{\pgfqpoint{3.266459in}{3.019737in}}%
\pgfpathlineto{\pgfqpoint{3.266598in}{3.192219in}}%
\pgfpathlineto{\pgfqpoint{3.267492in}{3.008478in}}%
\pgfpathlineto{\pgfqpoint{3.267261in}{3.285620in}}%
\pgfpathlineto{\pgfqpoint{3.267723in}{3.160358in}}%
\pgfpathlineto{\pgfqpoint{3.267785in}{3.286164in}}%
\pgfpathlineto{\pgfqpoint{3.268525in}{2.995022in}}%
\pgfpathlineto{\pgfqpoint{3.268848in}{3.243718in}}%
\pgfpathlineto{\pgfqpoint{3.269557in}{2.982910in}}%
\pgfpathlineto{\pgfqpoint{3.269342in}{3.297094in}}%
\pgfpathlineto{\pgfqpoint{3.269958in}{3.098227in}}%
\pgfpathlineto{\pgfqpoint{3.270883in}{3.299175in}}%
\pgfpathlineto{\pgfqpoint{3.270066in}{2.978693in}}%
\pgfpathlineto{\pgfqpoint{3.271068in}{3.104467in}}%
\pgfpathlineto{\pgfqpoint{3.271608in}{2.976313in}}%
\pgfpathlineto{\pgfqpoint{3.271931in}{3.302014in}}%
\pgfpathlineto{\pgfqpoint{3.272162in}{3.120272in}}%
\pgfpathlineto{\pgfqpoint{3.272440in}{3.307938in}}%
\pgfpathlineto{\pgfqpoint{3.272640in}{2.970852in}}%
\pgfpathlineto{\pgfqpoint{3.273272in}{3.167714in}}%
\pgfpathlineto{\pgfqpoint{3.273673in}{2.989958in}}%
\pgfpathlineto{\pgfqpoint{3.273966in}{3.300256in}}%
\pgfpathlineto{\pgfqpoint{3.274367in}{3.173346in}}%
\pgfpathlineto{\pgfqpoint{3.275507in}{3.310708in}}%
\pgfpathlineto{\pgfqpoint{3.275215in}{2.994595in}}%
\pgfpathlineto{\pgfqpoint{3.275523in}{3.292356in}}%
\pgfpathlineto{\pgfqpoint{3.276756in}{2.975913in}}%
\pgfpathlineto{\pgfqpoint{3.276540in}{3.313579in}}%
\pgfpathlineto{\pgfqpoint{3.276802in}{3.137960in}}%
\pgfpathlineto{\pgfqpoint{3.277573in}{3.330630in}}%
\pgfpathlineto{\pgfqpoint{3.277789in}{2.955669in}}%
\pgfpathlineto{\pgfqpoint{3.277912in}{3.151610in}}%
\pgfpathlineto{\pgfqpoint{3.278298in}{2.971920in}}%
\pgfpathlineto{\pgfqpoint{3.278606in}{3.335980in}}%
\pgfpathlineto{\pgfqpoint{3.279022in}{3.138723in}}%
\pgfpathlineto{\pgfqpoint{3.279115in}{3.325837in}}%
\pgfpathlineto{\pgfqpoint{3.279330in}{2.995151in}}%
\pgfpathlineto{\pgfqpoint{3.280194in}{3.266162in}}%
\pgfpathlineto{\pgfqpoint{3.280363in}{3.022251in}}%
\pgfpathlineto{\pgfqpoint{3.280671in}{3.308064in}}%
\pgfpathlineto{\pgfqpoint{3.281303in}{3.196878in}}%
\pgfpathlineto{\pgfqpoint{3.281704in}{3.292148in}}%
\pgfpathlineto{\pgfqpoint{3.281396in}{3.038864in}}%
\pgfpathlineto{\pgfqpoint{3.282382in}{3.178307in}}%
\pgfpathlineto{\pgfqpoint{3.283462in}{3.034067in}}%
\pgfpathlineto{\pgfqpoint{3.282737in}{3.298458in}}%
\pgfpathlineto{\pgfqpoint{3.283492in}{3.122358in}}%
\pgfpathlineto{\pgfqpoint{3.283770in}{3.301873in}}%
\pgfpathlineto{\pgfqpoint{3.284494in}{3.041201in}}%
\pgfpathlineto{\pgfqpoint{3.284618in}{3.164230in}}%
\pgfpathlineto{\pgfqpoint{3.285311in}{3.296469in}}%
\pgfpathlineto{\pgfqpoint{3.285018in}{3.051656in}}%
\pgfpathlineto{\pgfqpoint{3.285512in}{3.073502in}}%
\pgfpathlineto{\pgfqpoint{3.285527in}{3.045166in}}%
\pgfpathlineto{\pgfqpoint{3.285835in}{3.298630in}}%
\pgfpathlineto{\pgfqpoint{3.286591in}{3.112786in}}%
\pgfpathlineto{\pgfqpoint{3.286868in}{3.291280in}}%
\pgfpathlineto{\pgfqpoint{3.287084in}{3.055463in}}%
\pgfpathlineto{\pgfqpoint{3.287762in}{3.160935in}}%
\pgfpathlineto{\pgfqpoint{3.287901in}{3.282516in}}%
\pgfpathlineto{\pgfqpoint{3.288117in}{3.055870in}}%
\pgfpathlineto{\pgfqpoint{3.288595in}{3.150803in}}%
\pgfpathlineto{\pgfqpoint{3.289658in}{3.047893in}}%
\pgfpathlineto{\pgfqpoint{3.288934in}{3.285608in}}%
\pgfpathlineto{\pgfqpoint{3.289705in}{3.106453in}}%
\pgfpathlineto{\pgfqpoint{3.290475in}{3.293359in}}%
\pgfpathlineto{\pgfqpoint{3.290691in}{3.033477in}}%
\pgfpathlineto{\pgfqpoint{3.290861in}{3.157291in}}%
\pgfpathlineto{\pgfqpoint{3.291508in}{3.292848in}}%
\pgfpathlineto{\pgfqpoint{3.291724in}{3.020189in}}%
\pgfpathlineto{\pgfqpoint{3.292063in}{3.263202in}}%
\pgfpathlineto{\pgfqpoint{3.292757in}{3.016731in}}%
\pgfpathlineto{\pgfqpoint{3.293050in}{3.279504in}}%
\pgfpathlineto{\pgfqpoint{3.293188in}{3.186480in}}%
\pgfpathlineto{\pgfqpoint{3.293574in}{3.273487in}}%
\pgfpathlineto{\pgfqpoint{3.293265in}{3.006550in}}%
\pgfpathlineto{\pgfqpoint{3.294252in}{3.203791in}}%
\pgfpathlineto{\pgfqpoint{3.295331in}{2.985820in}}%
\pgfpathlineto{\pgfqpoint{3.294591in}{3.272640in}}%
\pgfpathlineto{\pgfqpoint{3.295362in}{3.079455in}}%
\pgfpathlineto{\pgfqpoint{3.296179in}{3.280635in}}%
\pgfpathlineto{\pgfqpoint{3.296364in}{2.976233in}}%
\pgfpathlineto{\pgfqpoint{3.296487in}{3.142305in}}%
\pgfpathlineto{\pgfqpoint{3.297381in}{2.960072in}}%
\pgfpathlineto{\pgfqpoint{3.297212in}{3.288629in}}%
\pgfpathlineto{\pgfqpoint{3.297551in}{3.161066in}}%
\pgfpathlineto{\pgfqpoint{3.298229in}{3.305200in}}%
\pgfpathlineto{\pgfqpoint{3.298414in}{2.938600in}}%
\pgfpathlineto{\pgfqpoint{3.298691in}{3.258476in}}%
\pgfpathlineto{\pgfqpoint{3.299262in}{3.324278in}}%
\pgfpathlineto{\pgfqpoint{3.299447in}{2.928734in}}%
\pgfpathlineto{\pgfqpoint{3.299786in}{3.301932in}}%
\pgfpathlineto{\pgfqpoint{3.299955in}{2.923459in}}%
\pgfpathlineto{\pgfqpoint{3.300803in}{3.345844in}}%
\pgfpathlineto{\pgfqpoint{3.300896in}{3.141240in}}%
\pgfpathlineto{\pgfqpoint{3.301836in}{3.365174in}}%
\pgfpathlineto{\pgfqpoint{3.301497in}{2.903202in}}%
\pgfpathlineto{\pgfqpoint{3.301990in}{3.037932in}}%
\pgfpathlineto{\pgfqpoint{3.303038in}{2.891247in}}%
\pgfpathlineto{\pgfqpoint{3.302345in}{3.388465in}}%
\pgfpathlineto{\pgfqpoint{3.303085in}{3.106845in}}%
\pgfpathlineto{\pgfqpoint{3.303886in}{3.416266in}}%
\pgfpathlineto{\pgfqpoint{3.304071in}{2.876703in}}%
\pgfpathlineto{\pgfqpoint{3.304194in}{3.098307in}}%
\pgfpathlineto{\pgfqpoint{3.304580in}{2.873682in}}%
\pgfpathlineto{\pgfqpoint{3.304919in}{3.418063in}}%
\pgfpathlineto{\pgfqpoint{3.305258in}{3.109775in}}%
\pgfpathlineto{\pgfqpoint{3.305428in}{3.439589in}}%
\pgfpathlineto{\pgfqpoint{3.306121in}{2.850798in}}%
\pgfpathlineto{\pgfqpoint{3.306383in}{3.214964in}}%
\pgfpathlineto{\pgfqpoint{3.306969in}{3.461806in}}%
\pgfpathlineto{\pgfqpoint{3.307154in}{2.826450in}}%
\pgfpathlineto{\pgfqpoint{3.307509in}{3.308863in}}%
\pgfpathlineto{\pgfqpoint{3.307663in}{2.814378in}}%
\pgfpathlineto{\pgfqpoint{3.308002in}{3.463894in}}%
\pgfpathlineto{\pgfqpoint{3.308619in}{3.255797in}}%
\pgfpathlineto{\pgfqpoint{3.309543in}{3.457034in}}%
\pgfpathlineto{\pgfqpoint{3.308696in}{2.802752in}}%
\pgfpathlineto{\pgfqpoint{3.309682in}{3.247600in}}%
\pgfpathlineto{\pgfqpoint{3.310237in}{2.789332in}}%
\pgfpathlineto{\pgfqpoint{3.310052in}{3.459079in}}%
\pgfpathlineto{\pgfqpoint{3.310792in}{3.033331in}}%
\pgfpathlineto{\pgfqpoint{3.311594in}{3.470428in}}%
\pgfpathlineto{\pgfqpoint{3.311779in}{2.788042in}}%
\pgfpathlineto{\pgfqpoint{3.311902in}{3.095527in}}%
\pgfpathlineto{\pgfqpoint{3.312811in}{2.776796in}}%
\pgfpathlineto{\pgfqpoint{3.312626in}{3.480156in}}%
\pgfpathlineto{\pgfqpoint{3.312981in}{3.101844in}}%
\pgfpathlineto{\pgfqpoint{3.313659in}{3.479638in}}%
\pgfpathlineto{\pgfqpoint{3.313844in}{2.781721in}}%
\pgfpathlineto{\pgfqpoint{3.314137in}{3.374720in}}%
\pgfpathlineto{\pgfqpoint{3.315201in}{3.499674in}}%
\pgfpathlineto{\pgfqpoint{3.314353in}{2.771684in}}%
\pgfpathlineto{\pgfqpoint{3.315216in}{3.388043in}}%
\pgfpathlineto{\pgfqpoint{3.315894in}{2.769221in}}%
\pgfpathlineto{\pgfqpoint{3.315709in}{3.494982in}}%
\pgfpathlineto{\pgfqpoint{3.316326in}{3.160493in}}%
\pgfpathlineto{\pgfqpoint{3.316742in}{3.513080in}}%
\pgfpathlineto{\pgfqpoint{3.316927in}{2.759676in}}%
\pgfpathlineto{\pgfqpoint{3.317405in}{3.164669in}}%
\pgfpathlineto{\pgfqpoint{3.318469in}{2.759822in}}%
\pgfpathlineto{\pgfqpoint{3.317775in}{3.508430in}}%
\pgfpathlineto{\pgfqpoint{3.318515in}{3.081810in}}%
\pgfpathlineto{\pgfqpoint{3.319316in}{3.507299in}}%
\pgfpathlineto{\pgfqpoint{3.319501in}{2.755495in}}%
\pgfpathlineto{\pgfqpoint{3.319609in}{3.122792in}}%
\pgfpathlineto{\pgfqpoint{3.320010in}{2.740974in}}%
\pgfpathlineto{\pgfqpoint{3.319825in}{3.506787in}}%
\pgfpathlineto{\pgfqpoint{3.320704in}{3.162444in}}%
\pgfpathlineto{\pgfqpoint{3.320858in}{3.500152in}}%
\pgfpathlineto{\pgfqpoint{3.321552in}{2.725921in}}%
\pgfpathlineto{\pgfqpoint{3.321875in}{3.467342in}}%
\pgfpathlineto{\pgfqpoint{3.322399in}{3.511550in}}%
\pgfpathlineto{\pgfqpoint{3.322076in}{2.744604in}}%
\pgfpathlineto{\pgfqpoint{3.322569in}{2.792313in}}%
\pgfpathlineto{\pgfqpoint{3.323093in}{2.711593in}}%
\pgfpathlineto{\pgfqpoint{3.323432in}{3.503015in}}%
\pgfpathlineto{\pgfqpoint{3.323648in}{2.939488in}}%
\pgfpathlineto{\pgfqpoint{3.323941in}{3.506311in}}%
\pgfpathlineto{\pgfqpoint{3.324126in}{2.701435in}}%
\pgfpathlineto{\pgfqpoint{3.324758in}{3.084048in}}%
\pgfpathlineto{\pgfqpoint{3.325667in}{2.695817in}}%
\pgfpathlineto{\pgfqpoint{3.325482in}{3.508282in}}%
\pgfpathlineto{\pgfqpoint{3.325837in}{3.133617in}}%
\pgfpathlineto{\pgfqpoint{3.326515in}{3.509166in}}%
\pgfpathlineto{\pgfqpoint{3.326700in}{2.705977in}}%
\pgfpathlineto{\pgfqpoint{3.326993in}{3.402883in}}%
\pgfpathlineto{\pgfqpoint{3.327024in}{3.506589in}}%
\pgfpathlineto{\pgfqpoint{3.327101in}{2.983405in}}%
\pgfpathlineto{\pgfqpoint{3.327178in}{3.102610in}}%
\pgfpathlineto{\pgfqpoint{3.328242in}{2.701416in}}%
\pgfpathlineto{\pgfqpoint{3.328057in}{3.526178in}}%
\pgfpathlineto{\pgfqpoint{3.328288in}{3.118260in}}%
\pgfpathlineto{\pgfqpoint{3.329089in}{3.518272in}}%
\pgfpathlineto{\pgfqpoint{3.328750in}{2.695685in}}%
\pgfpathlineto{\pgfqpoint{3.329382in}{3.068752in}}%
\pgfpathlineto{\pgfqpoint{3.330292in}{2.680297in}}%
\pgfpathlineto{\pgfqpoint{3.329598in}{3.534148in}}%
\pgfpathlineto{\pgfqpoint{3.330461in}{3.125913in}}%
\pgfpathlineto{\pgfqpoint{3.331140in}{3.547817in}}%
\pgfpathlineto{\pgfqpoint{3.331325in}{2.687523in}}%
\pgfpathlineto{\pgfqpoint{3.331617in}{3.411840in}}%
\pgfpathlineto{\pgfqpoint{3.331833in}{2.681269in}}%
\pgfpathlineto{\pgfqpoint{3.332172in}{3.551267in}}%
\pgfpathlineto{\pgfqpoint{3.332666in}{3.423353in}}%
\pgfpathlineto{\pgfqpoint{3.333714in}{3.556681in}}%
\pgfpathlineto{\pgfqpoint{3.333375in}{2.671626in}}%
\pgfpathlineto{\pgfqpoint{3.333745in}{3.320832in}}%
\pgfpathlineto{\pgfqpoint{3.334408in}{2.666039in}}%
\pgfpathlineto{\pgfqpoint{3.334223in}{3.551525in}}%
\pgfpathlineto{\pgfqpoint{3.334839in}{3.213482in}}%
\pgfpathlineto{\pgfqpoint{3.335255in}{3.550623in}}%
\pgfpathlineto{\pgfqpoint{3.334916in}{2.654324in}}%
\pgfpathlineto{\pgfqpoint{3.335918in}{3.005540in}}%
\pgfpathlineto{\pgfqpoint{3.336458in}{2.646802in}}%
\pgfpathlineto{\pgfqpoint{3.336797in}{3.540748in}}%
\pgfpathlineto{\pgfqpoint{3.337013in}{2.931183in}}%
\pgfpathlineto{\pgfqpoint{3.337306in}{3.546519in}}%
\pgfpathlineto{\pgfqpoint{3.337999in}{2.640180in}}%
\pgfpathlineto{\pgfqpoint{3.338123in}{3.022343in}}%
\pgfpathlineto{\pgfqpoint{3.338508in}{2.649759in}}%
\pgfpathlineto{\pgfqpoint{3.338847in}{3.549991in}}%
\pgfpathlineto{\pgfqpoint{3.339094in}{3.283626in}}%
\pgfpathlineto{\pgfqpoint{3.339356in}{3.546784in}}%
\pgfpathlineto{\pgfqpoint{3.340049in}{2.625047in}}%
\pgfpathlineto{\pgfqpoint{3.340157in}{3.228882in}}%
\pgfpathlineto{\pgfqpoint{3.341082in}{2.620653in}}%
\pgfpathlineto{\pgfqpoint{3.340897in}{3.550983in}}%
\pgfpathlineto{\pgfqpoint{3.341252in}{3.149656in}}%
\pgfpathlineto{\pgfqpoint{3.341930in}{3.533020in}}%
\pgfpathlineto{\pgfqpoint{3.341591in}{2.611074in}}%
\pgfpathlineto{\pgfqpoint{3.342408in}{3.433820in}}%
\pgfpathlineto{\pgfqpoint{3.342439in}{3.547013in}}%
\pgfpathlineto{\pgfqpoint{3.342624in}{2.623952in}}%
\pgfpathlineto{\pgfqpoint{3.343009in}{2.976544in}}%
\pgfpathlineto{\pgfqpoint{3.343132in}{2.598844in}}%
\pgfpathlineto{\pgfqpoint{3.343456in}{3.520553in}}%
\pgfpathlineto{\pgfqpoint{3.343965in}{3.483416in}}%
\pgfpathlineto{\pgfqpoint{3.344489in}{3.547719in}}%
\pgfpathlineto{\pgfqpoint{3.344674in}{2.609921in}}%
\pgfpathlineto{\pgfqpoint{3.345028in}{3.355125in}}%
\pgfpathlineto{\pgfqpoint{3.345183in}{2.605855in}}%
\pgfpathlineto{\pgfqpoint{3.346030in}{3.534355in}}%
\pgfpathlineto{\pgfqpoint{3.346138in}{3.299980in}}%
\pgfpathlineto{\pgfqpoint{3.346539in}{3.533941in}}%
\pgfpathlineto{\pgfqpoint{3.346724in}{2.611597in}}%
\pgfpathlineto{\pgfqpoint{3.347094in}{3.246669in}}%
\pgfpathlineto{\pgfqpoint{3.347233in}{2.613432in}}%
\pgfpathlineto{\pgfqpoint{3.348080in}{3.535376in}}%
\pgfpathlineto{\pgfqpoint{3.348188in}{3.269540in}}%
\pgfpathlineto{\pgfqpoint{3.348589in}{3.533460in}}%
\pgfpathlineto{\pgfqpoint{3.348774in}{2.606343in}}%
\pgfpathlineto{\pgfqpoint{3.349252in}{3.109755in}}%
\pgfpathlineto{\pgfqpoint{3.350316in}{2.607691in}}%
\pgfpathlineto{\pgfqpoint{3.349622in}{3.531929in}}%
\pgfpathlineto{\pgfqpoint{3.350362in}{3.046556in}}%
\pgfpathlineto{\pgfqpoint{3.351163in}{3.516977in}}%
\pgfpathlineto{\pgfqpoint{3.350824in}{2.622933in}}%
\pgfpathlineto{\pgfqpoint{3.351456in}{3.162595in}}%
\pgfpathlineto{\pgfqpoint{3.351857in}{2.627919in}}%
\pgfpathlineto{\pgfqpoint{3.351672in}{3.522927in}}%
\pgfpathlineto{\pgfqpoint{3.352551in}{3.134976in}}%
\pgfpathlineto{\pgfqpoint{3.353214in}{3.521261in}}%
\pgfpathlineto{\pgfqpoint{3.353399in}{2.652150in}}%
\pgfpathlineto{\pgfqpoint{3.353676in}{3.355115in}}%
\pgfpathlineto{\pgfqpoint{3.354755in}{3.521232in}}%
\pgfpathlineto{\pgfqpoint{3.353907in}{2.660992in}}%
\pgfpathlineto{\pgfqpoint{3.354771in}{3.428376in}}%
\pgfpathlineto{\pgfqpoint{3.355449in}{2.648963in}}%
\pgfpathlineto{\pgfqpoint{3.355264in}{3.513976in}}%
\pgfpathlineto{\pgfqpoint{3.355880in}{3.139706in}}%
\pgfpathlineto{\pgfqpoint{3.356297in}{3.518687in}}%
\pgfpathlineto{\pgfqpoint{3.356482in}{2.656132in}}%
\pgfpathlineto{\pgfqpoint{3.356959in}{3.066190in}}%
\pgfpathlineto{\pgfqpoint{3.356990in}{2.652287in}}%
\pgfpathlineto{\pgfqpoint{3.357838in}{3.512155in}}%
\pgfpathlineto{\pgfqpoint{3.358054in}{2.844918in}}%
\pgfpathlineto{\pgfqpoint{3.358347in}{3.506007in}}%
\pgfpathlineto{\pgfqpoint{3.358532in}{2.645296in}}%
\pgfpathlineto{\pgfqpoint{3.359164in}{3.149075in}}%
\pgfpathlineto{\pgfqpoint{3.360073in}{2.645034in}}%
\pgfpathlineto{\pgfqpoint{3.359380in}{3.502109in}}%
\pgfpathlineto{\pgfqpoint{3.360258in}{3.138784in}}%
\pgfpathlineto{\pgfqpoint{3.360921in}{3.490643in}}%
\pgfpathlineto{\pgfqpoint{3.360582in}{2.661459in}}%
\pgfpathlineto{\pgfqpoint{3.361383in}{3.374010in}}%
\pgfpathlineto{\pgfqpoint{3.361430in}{3.489686in}}%
\pgfpathlineto{\pgfqpoint{3.361615in}{2.653598in}}%
\pgfpathlineto{\pgfqpoint{3.362478in}{3.347707in}}%
\pgfpathlineto{\pgfqpoint{3.363156in}{2.665379in}}%
\pgfpathlineto{\pgfqpoint{3.362971in}{3.483763in}}%
\pgfpathlineto{\pgfqpoint{3.363588in}{3.219892in}}%
\pgfpathlineto{\pgfqpoint{3.364513in}{3.489421in}}%
\pgfpathlineto{\pgfqpoint{3.363665in}{2.670663in}}%
\pgfpathlineto{\pgfqpoint{3.364667in}{2.986751in}}%
\pgfpathlineto{\pgfqpoint{3.364698in}{2.681860in}}%
\pgfpathlineto{\pgfqpoint{3.365021in}{3.480598in}}%
\pgfpathlineto{\pgfqpoint{3.365761in}{2.930287in}}%
\pgfpathlineto{\pgfqpoint{3.366054in}{3.487491in}}%
\pgfpathlineto{\pgfqpoint{3.366748in}{2.695881in}}%
\pgfpathlineto{\pgfqpoint{3.366871in}{3.083095in}}%
\pgfpathlineto{\pgfqpoint{3.367257in}{2.709425in}}%
\pgfpathlineto{\pgfqpoint{3.367596in}{3.489993in}}%
\pgfpathlineto{\pgfqpoint{3.367950in}{3.109998in}}%
\pgfpathlineto{\pgfqpoint{3.368104in}{3.491815in}}%
\pgfpathlineto{\pgfqpoint{3.368289in}{2.698455in}}%
\pgfpathlineto{\pgfqpoint{3.369106in}{3.413946in}}%
\pgfpathlineto{\pgfqpoint{3.369646in}{3.494972in}}%
\pgfpathlineto{\pgfqpoint{3.369831in}{2.695070in}}%
\pgfpathlineto{\pgfqpoint{3.370170in}{3.426455in}}%
\pgfpathlineto{\pgfqpoint{3.370340in}{2.696204in}}%
\pgfpathlineto{\pgfqpoint{3.371187in}{3.487033in}}%
\pgfpathlineto{\pgfqpoint{3.371280in}{3.122605in}}%
\pgfpathlineto{\pgfqpoint{3.371696in}{3.489624in}}%
\pgfpathlineto{\pgfqpoint{3.371881in}{2.693458in}}%
\pgfpathlineto{\pgfqpoint{3.372359in}{3.092745in}}%
\pgfpathlineto{\pgfqpoint{3.373422in}{2.679721in}}%
\pgfpathlineto{\pgfqpoint{3.373237in}{3.486902in}}%
\pgfpathlineto{\pgfqpoint{3.373469in}{3.083045in}}%
\pgfpathlineto{\pgfqpoint{3.373746in}{3.471877in}}%
\pgfpathlineto{\pgfqpoint{3.374455in}{2.683587in}}%
\pgfpathlineto{\pgfqpoint{3.374563in}{3.157321in}}%
\pgfpathlineto{\pgfqpoint{3.374964in}{2.668560in}}%
\pgfpathlineto{\pgfqpoint{3.374779in}{3.477961in}}%
\pgfpathlineto{\pgfqpoint{3.375658in}{3.137871in}}%
\pgfpathlineto{\pgfqpoint{3.376320in}{3.467159in}}%
\pgfpathlineto{\pgfqpoint{3.376505in}{2.667440in}}%
\pgfpathlineto{\pgfqpoint{3.376798in}{3.427177in}}%
\pgfpathlineto{\pgfqpoint{3.377538in}{2.680271in}}%
\pgfpathlineto{\pgfqpoint{3.377862in}{3.466569in}}%
\pgfpathlineto{\pgfqpoint{3.377970in}{3.339982in}}%
\pgfpathlineto{\pgfqpoint{3.378371in}{3.458622in}}%
\pgfpathlineto{\pgfqpoint{3.378047in}{2.669522in}}%
\pgfpathlineto{\pgfqpoint{3.378941in}{3.079590in}}%
\pgfpathlineto{\pgfqpoint{3.379588in}{2.667897in}}%
\pgfpathlineto{\pgfqpoint{3.379403in}{3.461935in}}%
\pgfpathlineto{\pgfqpoint{3.380020in}{3.276979in}}%
\pgfpathlineto{\pgfqpoint{3.380945in}{3.461025in}}%
\pgfpathlineto{\pgfqpoint{3.380621in}{2.676749in}}%
\pgfpathlineto{\pgfqpoint{3.381084in}{3.206319in}}%
\pgfpathlineto{\pgfqpoint{3.381130in}{2.670227in}}%
\pgfpathlineto{\pgfqpoint{3.381978in}{3.467641in}}%
\pgfpathlineto{\pgfqpoint{3.382194in}{2.936890in}}%
\pgfpathlineto{\pgfqpoint{3.382486in}{3.465723in}}%
\pgfpathlineto{\pgfqpoint{3.382671in}{2.668739in}}%
\pgfpathlineto{\pgfqpoint{3.383303in}{3.065440in}}%
\pgfpathlineto{\pgfqpoint{3.384213in}{2.674833in}}%
\pgfpathlineto{\pgfqpoint{3.383519in}{3.470296in}}%
\pgfpathlineto{\pgfqpoint{3.384382in}{3.070218in}}%
\pgfpathlineto{\pgfqpoint{3.385061in}{3.476525in}}%
\pgfpathlineto{\pgfqpoint{3.385246in}{2.681416in}}%
\pgfpathlineto{\pgfqpoint{3.385539in}{3.394791in}}%
\pgfpathlineto{\pgfqpoint{3.385569in}{3.469510in}}%
\pgfpathlineto{\pgfqpoint{3.385600in}{3.353851in}}%
\pgfpathlineto{\pgfqpoint{3.385754in}{2.683241in}}%
\pgfpathlineto{\pgfqpoint{3.386602in}{3.478026in}}%
\pgfpathlineto{\pgfqpoint{3.386710in}{3.316043in}}%
\pgfpathlineto{\pgfqpoint{3.387111in}{3.468357in}}%
\pgfpathlineto{\pgfqpoint{3.386787in}{2.681276in}}%
\pgfpathlineto{\pgfqpoint{3.387666in}{3.280514in}}%
\pgfpathlineto{\pgfqpoint{3.388329in}{2.682526in}}%
\pgfpathlineto{\pgfqpoint{3.388144in}{3.474822in}}%
\pgfpathlineto{\pgfqpoint{3.388760in}{3.254722in}}%
\pgfpathlineto{\pgfqpoint{3.389685in}{3.473317in}}%
\pgfpathlineto{\pgfqpoint{3.389361in}{2.697883in}}%
\pgfpathlineto{\pgfqpoint{3.389824in}{3.211119in}}%
\pgfpathlineto{\pgfqpoint{3.389870in}{2.683864in}}%
\pgfpathlineto{\pgfqpoint{3.390194in}{3.465131in}}%
\pgfpathlineto{\pgfqpoint{3.390934in}{2.966448in}}%
\pgfpathlineto{\pgfqpoint{3.391227in}{3.472299in}}%
\pgfpathlineto{\pgfqpoint{3.391412in}{2.688602in}}%
\pgfpathlineto{\pgfqpoint{3.392044in}{3.032010in}}%
\pgfpathlineto{\pgfqpoint{3.392953in}{2.687902in}}%
\pgfpathlineto{\pgfqpoint{3.392768in}{3.461752in}}%
\pgfpathlineto{\pgfqpoint{3.393123in}{3.093620in}}%
\pgfpathlineto{\pgfqpoint{3.393277in}{3.457582in}}%
\pgfpathlineto{\pgfqpoint{3.393462in}{2.694726in}}%
\pgfpathlineto{\pgfqpoint{3.394263in}{3.394908in}}%
\pgfpathlineto{\pgfqpoint{3.394310in}{3.457412in}}%
\pgfpathlineto{\pgfqpoint{3.394371in}{2.971486in}}%
\pgfpathlineto{\pgfqpoint{3.394464in}{3.016962in}}%
\pgfpathlineto{\pgfqpoint{3.395003in}{2.685002in}}%
\pgfpathlineto{\pgfqpoint{3.394818in}{3.458504in}}%
\pgfpathlineto{\pgfqpoint{3.395558in}{3.026233in}}%
\pgfpathlineto{\pgfqpoint{3.396360in}{3.460298in}}%
\pgfpathlineto{\pgfqpoint{3.396545in}{2.672862in}}%
\pgfpathlineto{\pgfqpoint{3.396653in}{3.145378in}}%
\pgfpathlineto{\pgfqpoint{3.397053in}{2.677476in}}%
\pgfpathlineto{\pgfqpoint{3.396868in}{3.454265in}}%
\pgfpathlineto{\pgfqpoint{3.397747in}{3.114948in}}%
\pgfpathlineto{\pgfqpoint{3.397901in}{3.459599in}}%
\pgfpathlineto{\pgfqpoint{3.398595in}{2.660029in}}%
\pgfpathlineto{\pgfqpoint{3.398872in}{3.315364in}}%
\pgfpathlineto{\pgfqpoint{3.399951in}{3.460476in}}%
\pgfpathlineto{\pgfqpoint{3.399104in}{2.673240in}}%
\pgfpathlineto{\pgfqpoint{3.399982in}{3.313801in}}%
\pgfpathlineto{\pgfqpoint{3.400136in}{2.662509in}}%
\pgfpathlineto{\pgfqpoint{3.400460in}{3.457723in}}%
\pgfpathlineto{\pgfqpoint{3.401077in}{3.193059in}}%
\pgfpathlineto{\pgfqpoint{3.401493in}{3.460038in}}%
\pgfpathlineto{\pgfqpoint{3.401678in}{2.672081in}}%
\pgfpathlineto{\pgfqpoint{3.402156in}{3.080440in}}%
\pgfpathlineto{\pgfqpoint{3.402695in}{2.661935in}}%
\pgfpathlineto{\pgfqpoint{3.403034in}{3.460602in}}%
\pgfpathlineto{\pgfqpoint{3.403250in}{2.961064in}}%
\pgfpathlineto{\pgfqpoint{3.403543in}{3.465861in}}%
\pgfpathlineto{\pgfqpoint{3.404237in}{2.649387in}}%
\pgfpathlineto{\pgfqpoint{3.404360in}{3.024320in}}%
\pgfpathlineto{\pgfqpoint{3.404745in}{2.656243in}}%
\pgfpathlineto{\pgfqpoint{3.405085in}{3.469231in}}%
\pgfpathlineto{\pgfqpoint{3.405439in}{3.054417in}}%
\pgfpathlineto{\pgfqpoint{3.405593in}{3.476754in}}%
\pgfpathlineto{\pgfqpoint{3.406287in}{2.646392in}}%
\pgfpathlineto{\pgfqpoint{3.406580in}{3.430096in}}%
\pgfpathlineto{\pgfqpoint{3.406796in}{2.653208in}}%
\pgfpathlineto{\pgfqpoint{3.407643in}{3.473661in}}%
\pgfpathlineto{\pgfqpoint{3.407736in}{3.115271in}}%
\pgfpathlineto{\pgfqpoint{3.408152in}{3.462423in}}%
\pgfpathlineto{\pgfqpoint{3.408337in}{2.660400in}}%
\pgfpathlineto{\pgfqpoint{3.408815in}{3.172485in}}%
\pgfpathlineto{\pgfqpoint{3.408846in}{2.665858in}}%
\pgfpathlineto{\pgfqpoint{3.409694in}{3.460877in}}%
\pgfpathlineto{\pgfqpoint{3.409909in}{2.949681in}}%
\pgfpathlineto{\pgfqpoint{3.410202in}{3.458706in}}%
\pgfpathlineto{\pgfqpoint{3.410387in}{2.656852in}}%
\pgfpathlineto{\pgfqpoint{3.411019in}{3.116192in}}%
\pgfpathlineto{\pgfqpoint{3.411929in}{2.670829in}}%
\pgfpathlineto{\pgfqpoint{3.411744in}{3.467106in}}%
\pgfpathlineto{\pgfqpoint{3.412114in}{3.062736in}}%
\pgfpathlineto{\pgfqpoint{3.412252in}{3.467533in}}%
\pgfpathlineto{\pgfqpoint{3.412437in}{2.655115in}}%
\pgfpathlineto{\pgfqpoint{3.413239in}{3.442012in}}%
\pgfpathlineto{\pgfqpoint{3.413455in}{2.686523in}}%
\pgfpathlineto{\pgfqpoint{3.414303in}{3.463738in}}%
\pgfpathlineto{\pgfqpoint{3.414395in}{3.165665in}}%
\pgfpathlineto{\pgfqpoint{3.415289in}{3.467151in}}%
\pgfpathlineto{\pgfqpoint{3.414996in}{2.674870in}}%
\pgfpathlineto{\pgfqpoint{3.415474in}{3.235555in}}%
\pgfpathlineto{\pgfqpoint{3.416538in}{2.672613in}}%
\pgfpathlineto{\pgfqpoint{3.416476in}{3.476977in}}%
\pgfpathlineto{\pgfqpoint{3.416584in}{3.157717in}}%
\pgfpathlineto{\pgfqpoint{3.417339in}{3.498090in}}%
\pgfpathlineto{\pgfqpoint{3.417046in}{2.660013in}}%
\pgfpathlineto{\pgfqpoint{3.417678in}{3.163695in}}%
\pgfpathlineto{\pgfqpoint{3.418588in}{2.647273in}}%
\pgfpathlineto{\pgfqpoint{3.418372in}{3.484925in}}%
\pgfpathlineto{\pgfqpoint{3.418788in}{3.142070in}}%
\pgfpathlineto{\pgfqpoint{3.418881in}{3.501719in}}%
\pgfpathlineto{\pgfqpoint{3.419097in}{2.645255in}}%
\pgfpathlineto{\pgfqpoint{3.419944in}{3.460303in}}%
\pgfpathlineto{\pgfqpoint{3.420638in}{2.632255in}}%
\pgfpathlineto{\pgfqpoint{3.420422in}{3.493439in}}%
\pgfpathlineto{\pgfqpoint{3.421054in}{3.180016in}}%
\pgfpathlineto{\pgfqpoint{3.421470in}{3.483407in}}%
\pgfpathlineto{\pgfqpoint{3.421147in}{2.641779in}}%
\pgfpathlineto{\pgfqpoint{3.422149in}{3.057885in}}%
\pgfpathlineto{\pgfqpoint{3.422180in}{2.641438in}}%
\pgfpathlineto{\pgfqpoint{3.423012in}{3.489144in}}%
\pgfpathlineto{\pgfqpoint{3.423243in}{2.947162in}}%
\pgfpathlineto{\pgfqpoint{3.424045in}{3.492617in}}%
\pgfpathlineto{\pgfqpoint{3.423721in}{2.669973in}}%
\pgfpathlineto{\pgfqpoint{3.424353in}{3.092472in}}%
\pgfpathlineto{\pgfqpoint{3.424754in}{2.683738in}}%
\pgfpathlineto{\pgfqpoint{3.425078in}{3.493863in}}%
\pgfpathlineto{\pgfqpoint{3.425448in}{3.074516in}}%
\pgfpathlineto{\pgfqpoint{3.425586in}{3.481137in}}%
\pgfpathlineto{\pgfqpoint{3.425787in}{2.704691in}}%
\pgfpathlineto{\pgfqpoint{3.426588in}{3.358083in}}%
\pgfpathlineto{\pgfqpoint{3.426619in}{3.466550in}}%
\pgfpathlineto{\pgfqpoint{3.426819in}{2.738018in}}%
\pgfpathlineto{\pgfqpoint{3.427667in}{3.379347in}}%
\pgfpathlineto{\pgfqpoint{3.427852in}{2.751061in}}%
\pgfpathlineto{\pgfqpoint{3.428161in}{3.438183in}}%
\pgfpathlineto{\pgfqpoint{3.428777in}{3.194429in}}%
\pgfpathlineto{\pgfqpoint{3.429193in}{3.429759in}}%
\pgfpathlineto{\pgfqpoint{3.428885in}{2.766679in}}%
\pgfpathlineto{\pgfqpoint{3.429872in}{3.149511in}}%
\pgfpathlineto{\pgfqpoint{3.429918in}{2.785502in}}%
\pgfpathlineto{\pgfqpoint{3.430735in}{3.429221in}}%
\pgfpathlineto{\pgfqpoint{3.430981in}{3.047297in}}%
\pgfpathlineto{\pgfqpoint{3.431243in}{3.421474in}}%
\pgfpathlineto{\pgfqpoint{3.431459in}{2.815938in}}%
\pgfpathlineto{\pgfqpoint{3.432091in}{3.065449in}}%
\pgfpathlineto{\pgfqpoint{3.432492in}{2.825043in}}%
\pgfpathlineto{\pgfqpoint{3.432276in}{3.420113in}}%
\pgfpathlineto{\pgfqpoint{3.433170in}{3.063578in}}%
\pgfpathlineto{\pgfqpoint{3.433818in}{3.413475in}}%
\pgfpathlineto{\pgfqpoint{3.433525in}{2.841475in}}%
\pgfpathlineto{\pgfqpoint{3.434342in}{3.401145in}}%
\pgfpathlineto{\pgfqpoint{3.434558in}{2.873683in}}%
\pgfpathlineto{\pgfqpoint{3.435359in}{3.407321in}}%
\pgfpathlineto{\pgfqpoint{3.435606in}{2.985099in}}%
\pgfpathlineto{\pgfqpoint{3.436392in}{3.398701in}}%
\pgfpathlineto{\pgfqpoint{3.436099in}{2.913903in}}%
\pgfpathlineto{\pgfqpoint{3.436716in}{3.073183in}}%
\pgfpathlineto{\pgfqpoint{3.437117in}{2.940133in}}%
\pgfpathlineto{\pgfqpoint{3.436901in}{3.402356in}}%
\pgfpathlineto{\pgfqpoint{3.437795in}{3.088207in}}%
\pgfpathlineto{\pgfqpoint{3.437964in}{3.390157in}}%
\pgfpathlineto{\pgfqpoint{3.438149in}{2.951196in}}%
\pgfpathlineto{\pgfqpoint{3.438951in}{3.363755in}}%
\pgfpathlineto{\pgfqpoint{3.438997in}{3.387522in}}%
\pgfpathlineto{\pgfqpoint{3.439043in}{3.160715in}}%
\pgfpathlineto{\pgfqpoint{3.439182in}{2.979009in}}%
\pgfpathlineto{\pgfqpoint{3.439506in}{3.389675in}}%
\pgfpathlineto{\pgfqpoint{3.440215in}{3.013659in}}%
\pgfpathlineto{\pgfqpoint{3.440539in}{3.384454in}}%
\pgfpathlineto{\pgfqpoint{3.441340in}{3.041913in}}%
\pgfpathlineto{\pgfqpoint{3.441356in}{3.039032in}}%
\pgfpathlineto{\pgfqpoint{3.441433in}{3.222740in}}%
\pgfpathlineto{\pgfqpoint{3.442080in}{3.374577in}}%
\pgfpathlineto{\pgfqpoint{3.442373in}{3.034850in}}%
\pgfpathlineto{\pgfqpoint{3.442604in}{3.368393in}}%
\pgfpathlineto{\pgfqpoint{3.442882in}{3.023754in}}%
\pgfpathlineto{\pgfqpoint{3.443622in}{3.368968in}}%
\pgfpathlineto{\pgfqpoint{3.443837in}{3.092326in}}%
\pgfpathlineto{\pgfqpoint{3.443914in}{3.024845in}}%
\pgfpathlineto{\pgfqpoint{3.443992in}{3.187238in}}%
\pgfpathlineto{\pgfqpoint{3.444654in}{3.360123in}}%
\pgfpathlineto{\pgfqpoint{3.444423in}{3.029846in}}%
\pgfpathlineto{\pgfqpoint{3.445163in}{3.344575in}}%
\pgfpathlineto{\pgfqpoint{3.445225in}{3.266208in}}%
\pgfpathlineto{\pgfqpoint{3.445425in}{3.021730in}}%
\pgfpathlineto{\pgfqpoint{3.446227in}{3.348556in}}%
\pgfpathlineto{\pgfqpoint{3.446365in}{3.065276in}}%
\pgfpathlineto{\pgfqpoint{3.446458in}{3.017345in}}%
\pgfpathlineto{\pgfqpoint{3.446566in}{3.194458in}}%
\pgfpathlineto{\pgfqpoint{3.447260in}{3.352561in}}%
\pgfpathlineto{\pgfqpoint{3.447491in}{3.024811in}}%
\pgfpathlineto{\pgfqpoint{3.447737in}{3.331052in}}%
\pgfpathlineto{\pgfqpoint{3.447768in}{3.336884in}}%
\pgfpathlineto{\pgfqpoint{3.447784in}{3.326013in}}%
\pgfpathlineto{\pgfqpoint{3.447999in}{3.024172in}}%
\pgfpathlineto{\pgfqpoint{3.448801in}{3.350094in}}%
\pgfpathlineto{\pgfqpoint{3.448940in}{3.033814in}}%
\pgfpathlineto{\pgfqpoint{3.449834in}{3.344403in}}%
\pgfpathlineto{\pgfqpoint{3.450034in}{3.003546in}}%
\pgfpathlineto{\pgfqpoint{3.450142in}{3.119507in}}%
\pgfpathlineto{\pgfqpoint{3.450342in}{3.348214in}}%
\pgfpathlineto{\pgfqpoint{3.450558in}{3.015490in}}%
\pgfpathlineto{\pgfqpoint{3.451283in}{3.214746in}}%
\pgfpathlineto{\pgfqpoint{3.451375in}{3.284421in}}%
\pgfpathlineto{\pgfqpoint{3.451483in}{3.156024in}}%
\pgfpathlineto{\pgfqpoint{3.452115in}{3.015818in}}%
\pgfpathlineto{\pgfqpoint{3.451899in}{3.338338in}}%
\pgfpathlineto{\pgfqpoint{3.452547in}{3.149941in}}%
\pgfpathlineto{\pgfqpoint{3.452978in}{3.314553in}}%
\pgfpathlineto{\pgfqpoint{3.453163in}{2.991034in}}%
\pgfpathlineto{\pgfqpoint{3.453626in}{3.114050in}}%
\pgfpathlineto{\pgfqpoint{3.453687in}{2.922930in}}%
\pgfpathlineto{\pgfqpoint{3.454520in}{3.409203in}}%
\pgfpathlineto{\pgfqpoint{3.454736in}{3.065500in}}%
\pgfpathlineto{\pgfqpoint{3.455383in}{3.306322in}}%
\pgfpathlineto{\pgfqpoint{3.455738in}{3.027465in}}%
\pgfpathlineto{\pgfqpoint{3.455846in}{3.141043in}}%
\pgfpathlineto{\pgfqpoint{3.456678in}{3.329392in}}%
\pgfpathlineto{\pgfqpoint{3.456740in}{2.966874in}}%
\pgfpathlineto{\pgfqpoint{3.457002in}{3.240443in}}%
\pgfpathlineto{\pgfqpoint{3.457264in}{2.917785in}}%
\pgfpathlineto{\pgfqpoint{3.457449in}{3.338832in}}%
\pgfpathlineto{\pgfqpoint{3.458081in}{3.270301in}}%
\pgfpathlineto{\pgfqpoint{3.458219in}{3.372742in}}%
\pgfpathlineto{\pgfqpoint{3.458821in}{2.897389in}}%
\pgfpathlineto{\pgfqpoint{3.459160in}{3.233808in}}%
\pgfpathlineto{\pgfqpoint{3.459329in}{2.998194in}}%
\pgfpathlineto{\pgfqpoint{3.460239in}{3.277893in}}%
\pgfpathlineto{\pgfqpoint{3.460593in}{3.321170in}}%
\pgfpathlineto{\pgfqpoint{3.460362in}{3.003579in}}%
\pgfpathlineto{\pgfqpoint{3.461148in}{3.076528in}}%
\pgfpathlineto{\pgfqpoint{3.461179in}{3.014636in}}%
\pgfpathlineto{\pgfqpoint{3.461441in}{3.361086in}}%
\pgfpathlineto{\pgfqpoint{3.462212in}{3.164941in}}%
\pgfpathlineto{\pgfqpoint{3.462906in}{3.029242in}}%
\pgfpathlineto{\pgfqpoint{3.462613in}{3.255092in}}%
\pgfpathlineto{\pgfqpoint{3.462967in}{3.212834in}}%
\pgfpathlineto{\pgfqpoint{3.463060in}{3.368464in}}%
\pgfpathlineto{\pgfqpoint{3.463892in}{3.010501in}}%
\pgfpathlineto{\pgfqpoint{3.464077in}{3.258070in}}%
\pgfpathlineto{\pgfqpoint{3.464324in}{2.992948in}}%
\pgfpathlineto{\pgfqpoint{3.464617in}{3.344304in}}%
\pgfpathlineto{\pgfqpoint{3.465218in}{3.221565in}}%
\pgfpathlineto{\pgfqpoint{3.466066in}{3.316652in}}%
\pgfpathlineto{\pgfqpoint{3.465865in}{3.059954in}}%
\pgfpathlineto{\pgfqpoint{3.466266in}{3.179282in}}%
\pgfpathlineto{\pgfqpoint{3.467407in}{3.022202in}}%
\pgfpathlineto{\pgfqpoint{3.466728in}{3.344958in}}%
\pgfpathlineto{\pgfqpoint{3.467422in}{3.050509in}}%
\pgfpathlineto{\pgfqpoint{3.468393in}{3.345452in}}%
\pgfpathlineto{\pgfqpoint{3.468070in}{2.983983in}}%
\pgfpathlineto{\pgfqpoint{3.468563in}{3.142845in}}%
\pgfpathlineto{\pgfqpoint{3.469626in}{2.984335in}}%
\pgfpathlineto{\pgfqpoint{3.469334in}{3.430994in}}%
\pgfpathlineto{\pgfqpoint{3.469719in}{3.092716in}}%
\pgfpathlineto{\pgfqpoint{3.469781in}{3.027804in}}%
\pgfpathlineto{\pgfqpoint{3.469996in}{3.303018in}}%
\pgfpathlineto{\pgfqpoint{3.470459in}{3.212968in}}%
\pgfpathlineto{\pgfqpoint{3.470736in}{3.379305in}}%
\pgfpathlineto{\pgfqpoint{3.471060in}{2.996425in}}%
\pgfpathlineto{\pgfqpoint{3.471553in}{3.215483in}}%
\pgfpathlineto{\pgfqpoint{3.472077in}{2.929353in}}%
\pgfpathlineto{\pgfqpoint{3.472293in}{3.355642in}}%
\pgfpathlineto{\pgfqpoint{3.472725in}{3.084932in}}%
\pgfpathlineto{\pgfqpoint{3.473835in}{3.305216in}}%
\pgfpathlineto{\pgfqpoint{3.472864in}{2.984337in}}%
\pgfpathlineto{\pgfqpoint{3.473881in}{3.260360in}}%
\pgfpathlineto{\pgfqpoint{3.473943in}{3.339824in}}%
\pgfpathlineto{\pgfqpoint{3.474050in}{3.120414in}}%
\pgfpathlineto{\pgfqpoint{3.474405in}{2.960942in}}%
\pgfpathlineto{\pgfqpoint{3.474636in}{3.347976in}}%
\pgfpathlineto{\pgfqpoint{3.475099in}{3.193172in}}%
\pgfpathlineto{\pgfqpoint{3.475376in}{3.379809in}}%
\pgfpathlineto{\pgfqpoint{3.475947in}{2.964857in}}%
\pgfpathlineto{\pgfqpoint{3.476301in}{3.322205in}}%
\pgfpathlineto{\pgfqpoint{3.476748in}{2.935052in}}%
\pgfpathlineto{\pgfqpoint{3.476394in}{3.333954in}}%
\pgfpathlineto{\pgfqpoint{3.477550in}{3.191428in}}%
\pgfpathlineto{\pgfqpoint{3.478598in}{3.412765in}}%
\pgfpathlineto{\pgfqpoint{3.477997in}{3.017539in}}%
\pgfpathlineto{\pgfqpoint{3.478675in}{3.285316in}}%
\pgfpathlineto{\pgfqpoint{3.479029in}{2.989255in}}%
\pgfpathlineto{\pgfqpoint{3.479893in}{3.147377in}}%
\pgfpathlineto{\pgfqpoint{3.480848in}{3.376547in}}%
\pgfpathlineto{\pgfqpoint{3.480417in}{2.948089in}}%
\pgfpathlineto{\pgfqpoint{3.481033in}{3.318248in}}%
\pgfpathlineto{\pgfqpoint{3.481419in}{2.912829in}}%
\pgfpathlineto{\pgfqpoint{3.482174in}{3.219850in}}%
\pgfpathlineto{\pgfqpoint{3.482960in}{2.875529in}}%
\pgfpathlineto{\pgfqpoint{3.483145in}{3.373782in}}%
\pgfpathlineto{\pgfqpoint{3.483176in}{3.470467in}}%
\pgfpathlineto{\pgfqpoint{3.483731in}{2.892579in}}%
\pgfpathlineto{\pgfqpoint{3.484209in}{3.237089in}}%
\pgfpathlineto{\pgfqpoint{3.485026in}{2.960250in}}%
\pgfpathlineto{\pgfqpoint{3.484579in}{3.363625in}}%
\pgfpathlineto{\pgfqpoint{3.485350in}{3.192135in}}%
\pgfpathlineto{\pgfqpoint{3.485627in}{3.410943in}}%
\pgfpathlineto{\pgfqpoint{3.486059in}{2.898672in}}%
\pgfpathlineto{\pgfqpoint{3.486444in}{3.211007in}}%
\pgfpathlineto{\pgfqpoint{3.487600in}{2.952817in}}%
\pgfpathlineto{\pgfqpoint{3.486922in}{3.284353in}}%
\pgfpathlineto{\pgfqpoint{3.487631in}{3.016908in}}%
\pgfpathlineto{\pgfqpoint{3.487970in}{3.411048in}}%
\pgfpathlineto{\pgfqpoint{3.488248in}{2.959347in}}%
\pgfpathlineto{\pgfqpoint{3.488772in}{3.170575in}}%
\pgfpathlineto{\pgfqpoint{3.489804in}{2.978306in}}%
\pgfpathlineto{\pgfqpoint{3.488972in}{3.288238in}}%
\pgfpathlineto{\pgfqpoint{3.489974in}{3.060381in}}%
\pgfpathlineto{\pgfqpoint{3.490267in}{3.408493in}}%
\pgfpathlineto{\pgfqpoint{3.490575in}{2.980752in}}%
\pgfpathlineto{\pgfqpoint{3.491099in}{3.160661in}}%
\pgfpathlineto{\pgfqpoint{3.491423in}{3.292073in}}%
\pgfpathlineto{\pgfqpoint{3.492009in}{2.998014in}}%
\pgfpathlineto{\pgfqpoint{3.492132in}{3.098399in}}%
\pgfpathlineto{\pgfqpoint{3.493042in}{2.999106in}}%
\pgfpathlineto{\pgfqpoint{3.492456in}{3.345080in}}%
\pgfpathlineto{\pgfqpoint{3.493211in}{3.142206in}}%
\pgfpathlineto{\pgfqpoint{3.493442in}{3.286841in}}%
\pgfpathlineto{\pgfqpoint{3.494275in}{3.075715in}}%
\pgfpathlineto{\pgfqpoint{3.494306in}{3.121694in}}%
\pgfpathlineto{\pgfqpoint{3.494891in}{3.375956in}}%
\pgfpathlineto{\pgfqpoint{3.494583in}{2.968073in}}%
\pgfpathlineto{\pgfqpoint{3.495431in}{3.145867in}}%
\pgfpathlineto{\pgfqpoint{3.496140in}{3.060418in}}%
\pgfpathlineto{\pgfqpoint{3.496078in}{3.322284in}}%
\pgfpathlineto{\pgfqpoint{3.496494in}{3.190491in}}%
\pgfpathlineto{\pgfqpoint{3.497296in}{3.315447in}}%
\pgfpathlineto{\pgfqpoint{3.497527in}{2.994868in}}%
\pgfpathlineto{\pgfqpoint{3.497543in}{2.987521in}}%
\pgfpathlineto{\pgfqpoint{3.498051in}{3.263864in}}%
\pgfpathlineto{\pgfqpoint{3.498067in}{3.288647in}}%
\pgfpathlineto{\pgfqpoint{3.498915in}{3.035348in}}%
\pgfpathlineto{\pgfqpoint{3.499069in}{3.159190in}}%
\pgfpathlineto{\pgfqpoint{3.499886in}{2.992745in}}%
\pgfpathlineto{\pgfqpoint{3.499485in}{3.372680in}}%
\pgfpathlineto{\pgfqpoint{3.500163in}{3.167814in}}%
\pgfpathlineto{\pgfqpoint{3.500379in}{3.282029in}}%
\pgfpathlineto{\pgfqpoint{3.500811in}{3.040075in}}%
\pgfpathlineto{\pgfqpoint{3.501227in}{3.093846in}}%
\pgfpathlineto{\pgfqpoint{3.502198in}{3.003425in}}%
\pgfpathlineto{\pgfqpoint{3.501828in}{3.353362in}}%
\pgfpathlineto{\pgfqpoint{3.502290in}{3.130684in}}%
\pgfpathlineto{\pgfqpoint{3.502321in}{3.153450in}}%
\pgfpathlineto{\pgfqpoint{3.502337in}{3.116184in}}%
\pgfpathlineto{\pgfqpoint{3.502383in}{3.000945in}}%
\pgfpathlineto{\pgfqpoint{3.502583in}{3.280214in}}%
\pgfpathlineto{\pgfqpoint{3.503431in}{3.079704in}}%
\pgfpathlineto{\pgfqpoint{3.504109in}{3.426659in}}%
\pgfpathlineto{\pgfqpoint{3.504495in}{2.983670in}}%
\pgfpathlineto{\pgfqpoint{3.504695in}{2.914689in}}%
\pgfpathlineto{\pgfqpoint{3.505250in}{3.331218in}}%
\pgfpathlineto{\pgfqpoint{3.505512in}{3.160160in}}%
\pgfpathlineto{\pgfqpoint{3.506437in}{3.370156in}}%
\pgfpathlineto{\pgfqpoint{3.505851in}{3.042795in}}%
\pgfpathlineto{\pgfqpoint{3.506637in}{3.197174in}}%
\pgfpathlineto{\pgfqpoint{3.507208in}{3.323415in}}%
\pgfpathlineto{\pgfqpoint{3.507023in}{2.940014in}}%
\pgfpathlineto{\pgfqpoint{3.507747in}{3.197922in}}%
\pgfpathlineto{\pgfqpoint{3.508441in}{2.999482in}}%
\pgfpathlineto{\pgfqpoint{3.508749in}{3.436431in}}%
\pgfpathlineto{\pgfqpoint{3.508826in}{3.317018in}}%
\pgfpathlineto{\pgfqpoint{3.508842in}{3.324805in}}%
\pgfpathlineto{\pgfqpoint{3.509119in}{2.985244in}}%
\pgfpathlineto{\pgfqpoint{3.509304in}{3.088515in}}%
\pgfpathlineto{\pgfqpoint{3.509335in}{2.962379in}}%
\pgfpathlineto{\pgfqpoint{3.509921in}{3.335686in}}%
\pgfpathlineto{\pgfqpoint{3.510399in}{3.115167in}}%
\pgfpathlineto{\pgfqpoint{3.511123in}{3.383155in}}%
\pgfpathlineto{\pgfqpoint{3.511416in}{3.042862in}}%
\pgfpathlineto{\pgfqpoint{3.511524in}{2.909206in}}%
\pgfpathlineto{\pgfqpoint{3.511832in}{3.296019in}}%
\pgfpathlineto{\pgfqpoint{3.512202in}{3.227254in}}%
\pgfpathlineto{\pgfqpoint{3.512248in}{3.354216in}}%
\pgfpathlineto{\pgfqpoint{3.513065in}{3.003798in}}%
\pgfpathlineto{\pgfqpoint{3.513312in}{3.259787in}}%
\pgfpathlineto{\pgfqpoint{3.513451in}{3.399038in}}%
\pgfpathlineto{\pgfqpoint{3.513728in}{2.951654in}}%
\pgfpathlineto{\pgfqpoint{3.514422in}{3.267844in}}%
\pgfpathlineto{\pgfqpoint{3.514746in}{3.024104in}}%
\pgfpathlineto{\pgfqpoint{3.514669in}{3.306414in}}%
\pgfpathlineto{\pgfqpoint{3.515578in}{3.180064in}}%
\pgfpathlineto{\pgfqpoint{3.515840in}{3.398278in}}%
\pgfpathlineto{\pgfqpoint{3.516303in}{2.941965in}}%
\pgfpathlineto{\pgfqpoint{3.516657in}{3.129184in}}%
\pgfpathlineto{\pgfqpoint{3.517690in}{3.004457in}}%
\pgfpathlineto{\pgfqpoint{3.516857in}{3.294728in}}%
\pgfpathlineto{\pgfqpoint{3.517752in}{3.109584in}}%
\pgfpathlineto{\pgfqpoint{3.518044in}{3.454230in}}%
\pgfpathlineto{\pgfqpoint{3.518507in}{2.970563in}}%
\pgfpathlineto{\pgfqpoint{3.518877in}{3.155320in}}%
\pgfpathlineto{\pgfqpoint{3.519031in}{3.302054in}}%
\pgfpathlineto{\pgfqpoint{3.519401in}{3.006328in}}%
\pgfpathlineto{\pgfqpoint{3.519833in}{3.077098in}}%
\pgfpathlineto{\pgfqpoint{3.520819in}{2.951567in}}%
\pgfpathlineto{\pgfqpoint{3.520341in}{3.374266in}}%
\pgfpathlineto{\pgfqpoint{3.520958in}{3.037240in}}%
\pgfpathlineto{\pgfqpoint{3.521528in}{3.301965in}}%
\pgfpathlineto{\pgfqpoint{3.522099in}{3.171906in}}%
\pgfpathlineto{\pgfqpoint{3.522129in}{3.180769in}}%
\pgfpathlineto{\pgfqpoint{3.522145in}{3.160425in}}%
\pgfpathlineto{\pgfqpoint{3.523147in}{3.008407in}}%
\pgfpathlineto{\pgfqpoint{3.522715in}{3.473349in}}%
\pgfpathlineto{\pgfqpoint{3.523301in}{3.034563in}}%
\pgfpathlineto{\pgfqpoint{3.523686in}{3.293817in}}%
\pgfpathlineto{\pgfqpoint{3.524457in}{3.123552in}}%
\pgfpathlineto{\pgfqpoint{3.525490in}{2.968831in}}%
\pgfpathlineto{\pgfqpoint{3.525043in}{3.390797in}}%
\pgfpathlineto{\pgfqpoint{3.525613in}{3.005145in}}%
\pgfpathlineto{\pgfqpoint{3.526199in}{3.285432in}}%
\pgfpathlineto{\pgfqpoint{3.526738in}{3.179567in}}%
\pgfpathlineto{\pgfqpoint{3.527787in}{2.982186in}}%
\pgfpathlineto{\pgfqpoint{3.527370in}{3.355719in}}%
\pgfpathlineto{\pgfqpoint{3.527848in}{3.171396in}}%
\pgfpathlineto{\pgfqpoint{3.527894in}{3.092246in}}%
\pgfpathlineto{\pgfqpoint{3.527925in}{3.010665in}}%
\pgfpathlineto{\pgfqpoint{3.528912in}{3.306473in}}%
\pgfpathlineto{\pgfqpoint{3.528974in}{3.142540in}}%
\pgfpathlineto{\pgfqpoint{3.529683in}{3.373103in}}%
\pgfpathlineto{\pgfqpoint{3.529328in}{3.005525in}}%
\pgfpathlineto{\pgfqpoint{3.529975in}{3.098254in}}%
\pgfpathlineto{\pgfqpoint{3.530253in}{2.986391in}}%
\pgfpathlineto{\pgfqpoint{3.530823in}{3.283997in}}%
\pgfpathlineto{\pgfqpoint{3.531070in}{3.125442in}}%
\pgfpathlineto{\pgfqpoint{3.531979in}{3.343550in}}%
\pgfpathlineto{\pgfqpoint{3.531687in}{3.006289in}}%
\pgfpathlineto{\pgfqpoint{3.532195in}{3.204763in}}%
\pgfpathlineto{\pgfqpoint{3.532766in}{3.277715in}}%
\pgfpathlineto{\pgfqpoint{3.532581in}{2.987524in}}%
\pgfpathlineto{\pgfqpoint{3.533213in}{3.133092in}}%
\pgfpathlineto{\pgfqpoint{3.533999in}{3.015214in}}%
\pgfpathlineto{\pgfqpoint{3.533521in}{3.271240in}}%
\pgfpathlineto{\pgfqpoint{3.534245in}{3.187771in}}%
\pgfpathlineto{\pgfqpoint{3.534307in}{3.320836in}}%
\pgfpathlineto{\pgfqpoint{3.534908in}{3.016267in}}%
\pgfpathlineto{\pgfqpoint{3.535324in}{3.154846in}}%
\pgfpathlineto{\pgfqpoint{3.536311in}{3.041599in}}%
\pgfpathlineto{\pgfqpoint{3.535479in}{3.269144in}}%
\pgfpathlineto{\pgfqpoint{3.536496in}{3.122220in}}%
\pgfpathlineto{\pgfqpoint{3.536681in}{3.326582in}}%
\pgfpathlineto{\pgfqpoint{3.537174in}{3.051608in}}%
\pgfpathlineto{\pgfqpoint{3.537606in}{3.169627in}}%
\pgfpathlineto{\pgfqpoint{3.538669in}{3.073346in}}%
\pgfpathlineto{\pgfqpoint{3.538238in}{3.285877in}}%
\pgfpathlineto{\pgfqpoint{3.538731in}{3.150261in}}%
\pgfpathlineto{\pgfqpoint{3.538993in}{3.340815in}}%
\pgfpathlineto{\pgfqpoint{3.538793in}{3.027937in}}%
\pgfpathlineto{\pgfqpoint{3.539533in}{3.044490in}}%
\pgfpathlineto{\pgfqpoint{3.539564in}{3.004923in}}%
\pgfpathlineto{\pgfqpoint{3.539779in}{3.279189in}}%
\pgfpathlineto{\pgfqpoint{3.540535in}{3.192936in}}%
\pgfpathlineto{\pgfqpoint{3.541336in}{3.381247in}}%
\pgfpathlineto{\pgfqpoint{3.541105in}{3.032517in}}%
\pgfpathlineto{\pgfqpoint{3.541598in}{3.149816in}}%
\pgfpathlineto{\pgfqpoint{3.541891in}{3.015641in}}%
\pgfpathlineto{\pgfqpoint{3.542230in}{3.282344in}}%
\pgfpathlineto{\pgfqpoint{3.542693in}{3.111634in}}%
\pgfpathlineto{\pgfqpoint{3.543633in}{3.320427in}}%
\pgfpathlineto{\pgfqpoint{3.543448in}{2.965916in}}%
\pgfpathlineto{\pgfqpoint{3.543818in}{3.201494in}}%
\pgfpathlineto{\pgfqpoint{3.544111in}{3.067484in}}%
\pgfpathlineto{\pgfqpoint{3.544296in}{3.249079in}}%
\pgfpathlineto{\pgfqpoint{3.544928in}{3.196530in}}%
\pgfpathlineto{\pgfqpoint{3.545005in}{3.023884in}}%
\pgfpathlineto{\pgfqpoint{3.545221in}{3.285961in}}%
\pgfpathlineto{\pgfqpoint{3.545884in}{3.199231in}}%
\pgfpathlineto{\pgfqpoint{3.546130in}{3.348298in}}%
\pgfpathlineto{\pgfqpoint{3.546408in}{3.008866in}}%
\pgfpathlineto{\pgfqpoint{3.546993in}{3.202846in}}%
\pgfpathlineto{\pgfqpoint{3.547348in}{3.008100in}}%
\pgfpathlineto{\pgfqpoint{3.547117in}{3.274234in}}%
\pgfpathlineto{\pgfqpoint{3.548134in}{3.092311in}}%
\pgfpathlineto{\pgfqpoint{3.549059in}{3.308455in}}%
\pgfpathlineto{\pgfqpoint{3.548627in}{3.046463in}}%
\pgfpathlineto{\pgfqpoint{3.549244in}{3.176937in}}%
\pgfpathlineto{\pgfqpoint{3.550308in}{2.951830in}}%
\pgfpathlineto{\pgfqpoint{3.549891in}{3.315117in}}%
\pgfpathlineto{\pgfqpoint{3.550369in}{3.126061in}}%
\pgfpathlineto{\pgfqpoint{3.551217in}{3.027212in}}%
\pgfpathlineto{\pgfqpoint{3.550770in}{3.259186in}}%
\pgfpathlineto{\pgfqpoint{3.551387in}{3.193255in}}%
\pgfpathlineto{\pgfqpoint{3.552080in}{3.292715in}}%
\pgfpathlineto{\pgfqpoint{3.551495in}{3.108016in}}%
\pgfpathlineto{\pgfqpoint{3.552481in}{3.166003in}}%
\pgfpathlineto{\pgfqpoint{3.553545in}{3.010859in}}%
\pgfpathlineto{\pgfqpoint{3.552959in}{3.394131in}}%
\pgfpathlineto{\pgfqpoint{3.553591in}{3.135990in}}%
\pgfpathlineto{\pgfqpoint{3.554007in}{3.309039in}}%
\pgfpathlineto{\pgfqpoint{3.554208in}{3.027990in}}%
\pgfpathlineto{\pgfqpoint{3.554701in}{3.220168in}}%
\pgfpathlineto{\pgfqpoint{3.555117in}{3.046270in}}%
\pgfpathlineto{\pgfqpoint{3.555317in}{3.253443in}}%
\pgfpathlineto{\pgfqpoint{3.555826in}{3.145097in}}%
\pgfpathlineto{\pgfqpoint{3.555965in}{3.319250in}}%
\pgfpathlineto{\pgfqpoint{3.556659in}{3.000336in}}%
\pgfpathlineto{\pgfqpoint{3.557029in}{3.266857in}}%
\pgfpathlineto{\pgfqpoint{3.558200in}{3.011076in}}%
\pgfpathlineto{\pgfqpoint{3.557661in}{3.296811in}}%
\pgfpathlineto{\pgfqpoint{3.558215in}{3.024661in}}%
\pgfpathlineto{\pgfqpoint{3.558632in}{3.265143in}}%
\pgfpathlineto{\pgfqpoint{3.559372in}{3.134772in}}%
\pgfpathlineto{\pgfqpoint{3.560404in}{3.025321in}}%
\pgfpathlineto{\pgfqpoint{3.559834in}{3.340759in}}%
\pgfpathlineto{\pgfqpoint{3.560451in}{3.154242in}}%
\pgfpathlineto{\pgfqpoint{3.561530in}{3.342791in}}%
\pgfpathlineto{\pgfqpoint{3.561314in}{2.985530in}}%
\pgfpathlineto{\pgfqpoint{3.561591in}{3.194102in}}%
\pgfpathlineto{\pgfqpoint{3.561668in}{3.345506in}}%
\pgfpathlineto{\pgfqpoint{3.562100in}{3.005544in}}%
\pgfpathlineto{\pgfqpoint{3.562717in}{3.222028in}}%
\pgfpathlineto{\pgfqpoint{3.562871in}{3.042593in}}%
\pgfpathlineto{\pgfqpoint{3.563719in}{3.308656in}}%
\pgfpathlineto{\pgfqpoint{3.563826in}{3.200862in}}%
\pgfpathlineto{\pgfqpoint{3.564628in}{3.364716in}}%
\pgfpathlineto{\pgfqpoint{3.564273in}{3.044115in}}%
\pgfpathlineto{\pgfqpoint{3.564828in}{3.168929in}}%
\pgfpathlineto{\pgfqpoint{3.565969in}{2.998773in}}%
\pgfpathlineto{\pgfqpoint{3.565599in}{3.309742in}}%
\pgfpathlineto{\pgfqpoint{3.566000in}{3.069393in}}%
\pgfpathlineto{\pgfqpoint{3.566339in}{3.260533in}}%
\pgfpathlineto{\pgfqpoint{3.566755in}{3.055746in}}%
\pgfpathlineto{\pgfqpoint{3.567141in}{3.166845in}}%
\pgfpathlineto{\pgfqpoint{3.567449in}{3.246529in}}%
\pgfpathlineto{\pgfqpoint{3.567233in}{3.063599in}}%
\pgfpathlineto{\pgfqpoint{3.568235in}{3.154213in}}%
\pgfpathlineto{\pgfqpoint{3.568297in}{3.023883in}}%
\pgfpathlineto{\pgfqpoint{3.568513in}{3.331841in}}%
\pgfpathlineto{\pgfqpoint{3.569268in}{3.229617in}}%
\pgfpathlineto{\pgfqpoint{3.569437in}{3.297306in}}%
\pgfpathlineto{\pgfqpoint{3.569699in}{3.047479in}}%
\pgfpathlineto{\pgfqpoint{3.570347in}{3.189755in}}%
\pgfpathlineto{\pgfqpoint{3.570856in}{3.312921in}}%
\pgfpathlineto{\pgfqpoint{3.570640in}{2.983859in}}%
\pgfpathlineto{\pgfqpoint{3.571395in}{3.133840in}}%
\pgfpathlineto{\pgfqpoint{3.572181in}{3.035700in}}%
\pgfpathlineto{\pgfqpoint{3.571472in}{3.276709in}}%
\pgfpathlineto{\pgfqpoint{3.572243in}{3.234219in}}%
\pgfpathlineto{\pgfqpoint{3.572413in}{3.324510in}}%
\pgfpathlineto{\pgfqpoint{3.572675in}{3.027508in}}%
\pgfpathlineto{\pgfqpoint{3.573353in}{3.255741in}}%
\pgfpathlineto{\pgfqpoint{3.573569in}{3.007599in}}%
\pgfpathlineto{\pgfqpoint{3.573584in}{3.001904in}}%
\pgfpathlineto{\pgfqpoint{3.573877in}{3.268394in}}%
\pgfpathlineto{\pgfqpoint{3.573939in}{3.263048in}}%
\pgfpathlineto{\pgfqpoint{3.573969in}{3.302697in}}%
\pgfpathlineto{\pgfqpoint{3.574509in}{3.048779in}}%
\pgfpathlineto{\pgfqpoint{3.574971in}{3.194219in}}%
\pgfpathlineto{\pgfqpoint{3.575280in}{3.016743in}}%
\pgfpathlineto{\pgfqpoint{3.575480in}{3.242296in}}%
\pgfpathlineto{\pgfqpoint{3.576081in}{3.149657in}}%
\pgfpathlineto{\pgfqpoint{3.577068in}{3.348182in}}%
\pgfpathlineto{\pgfqpoint{3.576667in}{3.002493in}}%
\pgfpathlineto{\pgfqpoint{3.577222in}{3.225718in}}%
\pgfpathlineto{\pgfqpoint{3.577854in}{3.263235in}}%
\pgfpathlineto{\pgfqpoint{3.577453in}{3.004152in}}%
\pgfpathlineto{\pgfqpoint{3.578208in}{3.090574in}}%
\pgfpathlineto{\pgfqpoint{3.578378in}{3.011257in}}%
\pgfpathlineto{\pgfqpoint{3.578625in}{3.286815in}}%
\pgfpathlineto{\pgfqpoint{3.579257in}{3.180361in}}%
\pgfpathlineto{\pgfqpoint{3.579411in}{3.284489in}}%
\pgfpathlineto{\pgfqpoint{3.579935in}{3.024548in}}%
\pgfpathlineto{\pgfqpoint{3.580351in}{3.195725in}}%
\pgfpathlineto{\pgfqpoint{3.581322in}{3.040080in}}%
\pgfpathlineto{\pgfqpoint{3.580952in}{3.258118in}}%
\pgfpathlineto{\pgfqpoint{3.581523in}{3.089304in}}%
\pgfpathlineto{\pgfqpoint{3.581723in}{3.316289in}}%
\pgfpathlineto{\pgfqpoint{3.582263in}{3.014229in}}%
\pgfpathlineto{\pgfqpoint{3.582663in}{3.221363in}}%
\pgfpathlineto{\pgfqpoint{3.583049in}{3.078857in}}%
\pgfpathlineto{\pgfqpoint{3.583265in}{3.286715in}}%
\pgfpathlineto{\pgfqpoint{3.583280in}{3.314365in}}%
\pgfpathlineto{\pgfqpoint{3.583665in}{3.043519in}}%
\pgfpathlineto{\pgfqpoint{3.584313in}{3.182912in}}%
\pgfpathlineto{\pgfqpoint{3.584590in}{3.041123in}}%
\pgfpathlineto{\pgfqpoint{3.584821in}{3.298543in}}%
\pgfpathlineto{\pgfqpoint{3.585423in}{3.158599in}}%
\pgfpathlineto{\pgfqpoint{3.586378in}{3.299090in}}%
\pgfpathlineto{\pgfqpoint{3.586147in}{3.010108in}}%
\pgfpathlineto{\pgfqpoint{3.586548in}{3.239823in}}%
\pgfpathlineto{\pgfqpoint{3.586933in}{3.038738in}}%
\pgfpathlineto{\pgfqpoint{3.587165in}{3.269181in}}%
\pgfpathlineto{\pgfqpoint{3.587750in}{3.169238in}}%
\pgfpathlineto{\pgfqpoint{3.588721in}{3.316821in}}%
\pgfpathlineto{\pgfqpoint{3.588490in}{3.056487in}}%
\pgfpathlineto{\pgfqpoint{3.588876in}{3.237703in}}%
\pgfpathlineto{\pgfqpoint{3.589261in}{3.030984in}}%
\pgfpathlineto{\pgfqpoint{3.589492in}{3.274492in}}%
\pgfpathlineto{\pgfqpoint{3.590093in}{3.159955in}}%
\pgfpathlineto{\pgfqpoint{3.590278in}{3.277818in}}%
\pgfpathlineto{\pgfqpoint{3.590818in}{3.062093in}}%
\pgfpathlineto{\pgfqpoint{3.591203in}{3.186220in}}%
\pgfpathlineto{\pgfqpoint{3.591589in}{3.048848in}}%
\pgfpathlineto{\pgfqpoint{3.591804in}{3.263321in}}%
\pgfpathlineto{\pgfqpoint{3.592606in}{3.282053in}}%
\pgfpathlineto{\pgfqpoint{3.592205in}{3.082505in}}%
\pgfpathlineto{\pgfqpoint{3.592806in}{3.152009in}}%
\pgfpathlineto{\pgfqpoint{3.593161in}{3.004891in}}%
\pgfpathlineto{\pgfqpoint{3.593377in}{3.313292in}}%
\pgfpathlineto{\pgfqpoint{3.593962in}{3.099426in}}%
\pgfpathlineto{\pgfqpoint{3.594934in}{3.305870in}}%
\pgfpathlineto{\pgfqpoint{3.594718in}{3.083154in}}%
\pgfpathlineto{\pgfqpoint{3.595088in}{3.215046in}}%
\pgfpathlineto{\pgfqpoint{3.595489in}{3.032613in}}%
\pgfpathlineto{\pgfqpoint{3.595720in}{3.292893in}}%
\pgfpathlineto{\pgfqpoint{3.596244in}{3.076513in}}%
\pgfpathlineto{\pgfqpoint{3.596275in}{3.070149in}}%
\pgfpathlineto{\pgfqpoint{3.596290in}{3.083224in}}%
\pgfpathlineto{\pgfqpoint{3.597277in}{3.274345in}}%
\pgfpathlineto{\pgfqpoint{3.597045in}{3.043873in}}%
\pgfpathlineto{\pgfqpoint{3.597431in}{3.165708in}}%
\pgfpathlineto{\pgfqpoint{3.598047in}{3.290557in}}%
\pgfpathlineto{\pgfqpoint{3.597770in}{3.045853in}}%
\pgfpathlineto{\pgfqpoint{3.598417in}{3.125524in}}%
\pgfpathlineto{\pgfqpoint{3.599388in}{3.023914in}}%
\pgfpathlineto{\pgfqpoint{3.598818in}{3.277082in}}%
\pgfpathlineto{\pgfqpoint{3.599435in}{3.140834in}}%
\pgfpathlineto{\pgfqpoint{3.599604in}{3.295822in}}%
\pgfpathlineto{\pgfqpoint{3.600159in}{3.050658in}}%
\pgfpathlineto{\pgfqpoint{3.600545in}{3.180383in}}%
\pgfpathlineto{\pgfqpoint{3.600945in}{3.050641in}}%
\pgfpathlineto{\pgfqpoint{3.601130in}{3.270806in}}%
\pgfpathlineto{\pgfqpoint{3.601747in}{3.113574in}}%
\pgfpathlineto{\pgfqpoint{3.602703in}{3.309241in}}%
\pgfpathlineto{\pgfqpoint{3.602502in}{3.070848in}}%
\pgfpathlineto{\pgfqpoint{3.602857in}{3.189026in}}%
\pgfpathlineto{\pgfqpoint{3.603473in}{3.294736in}}%
\pgfpathlineto{\pgfqpoint{3.604013in}{3.045151in}}%
\pgfpathlineto{\pgfqpoint{3.604383in}{3.274877in}}%
\pgfpathlineto{\pgfqpoint{3.605215in}{3.145725in}}%
\pgfpathlineto{\pgfqpoint{3.605585in}{3.069164in}}%
\pgfpathlineto{\pgfqpoint{3.605323in}{3.253789in}}%
\pgfpathlineto{\pgfqpoint{3.606325in}{3.141328in}}%
\pgfpathlineto{\pgfqpoint{3.607358in}{3.291699in}}%
\pgfpathlineto{\pgfqpoint{3.607173in}{3.042062in}}%
\pgfpathlineto{\pgfqpoint{3.607481in}{3.233341in}}%
\pgfpathlineto{\pgfqpoint{3.608391in}{3.292898in}}%
\pgfpathlineto{\pgfqpoint{3.607975in}{3.049763in}}%
\pgfpathlineto{\pgfqpoint{3.608545in}{3.211085in}}%
\pgfpathlineto{\pgfqpoint{3.608838in}{3.029765in}}%
\pgfpathlineto{\pgfqpoint{3.609408in}{3.279660in}}%
\pgfpathlineto{\pgfqpoint{3.609670in}{3.157400in}}%
\pgfpathlineto{\pgfqpoint{3.610580in}{3.233366in}}%
\pgfpathlineto{\pgfqpoint{3.610379in}{3.078313in}}%
\pgfpathlineto{\pgfqpoint{3.610780in}{3.164818in}}%
\pgfpathlineto{\pgfqpoint{3.611165in}{3.033323in}}%
\pgfpathlineto{\pgfqpoint{3.611505in}{3.272519in}}%
\pgfpathlineto{\pgfqpoint{3.611982in}{3.119940in}}%
\pgfpathlineto{\pgfqpoint{3.613061in}{3.295412in}}%
\pgfpathlineto{\pgfqpoint{3.612722in}{3.031774in}}%
\pgfpathlineto{\pgfqpoint{3.613108in}{3.166745in}}%
\pgfpathlineto{\pgfqpoint{3.613632in}{3.014921in}}%
\pgfpathlineto{\pgfqpoint{3.613277in}{3.278369in}}%
\pgfpathlineto{\pgfqpoint{3.613817in}{3.233390in}}%
\pgfpathlineto{\pgfqpoint{3.614063in}{3.283481in}}%
\pgfpathlineto{\pgfqpoint{3.614403in}{3.037330in}}%
\pgfpathlineto{\pgfqpoint{3.614880in}{3.175455in}}%
\pgfpathlineto{\pgfqpoint{3.615929in}{3.018081in}}%
\pgfpathlineto{\pgfqpoint{3.615358in}{3.292343in}}%
\pgfpathlineto{\pgfqpoint{3.616006in}{3.084344in}}%
\pgfpathlineto{\pgfqpoint{3.617100in}{3.320990in}}%
\pgfpathlineto{\pgfqpoint{3.616761in}{2.957545in}}%
\pgfpathlineto{\pgfqpoint{3.617193in}{3.269330in}}%
\pgfpathlineto{\pgfqpoint{3.618318in}{2.936328in}}%
\pgfpathlineto{\pgfqpoint{3.617886in}{3.326393in}}%
\pgfpathlineto{\pgfqpoint{3.618364in}{3.150300in}}%
\pgfpathlineto{\pgfqpoint{3.619412in}{3.313005in}}%
\pgfpathlineto{\pgfqpoint{3.619089in}{2.985685in}}%
\pgfpathlineto{\pgfqpoint{3.619459in}{3.198399in}}%
\pgfpathlineto{\pgfqpoint{3.619859in}{2.960546in}}%
\pgfpathlineto{\pgfqpoint{3.620060in}{3.353041in}}%
\pgfpathlineto{\pgfqpoint{3.620615in}{2.989650in}}%
\pgfpathlineto{\pgfqpoint{3.621062in}{3.306260in}}%
\pgfpathlineto{\pgfqpoint{3.621385in}{2.969072in}}%
\pgfpathlineto{\pgfqpoint{3.622064in}{3.158543in}}%
\pgfpathlineto{\pgfqpoint{3.622218in}{3.011522in}}%
\pgfpathlineto{\pgfqpoint{3.622557in}{3.296857in}}%
\pgfpathlineto{\pgfqpoint{3.623143in}{3.188483in}}%
\pgfpathlineto{\pgfqpoint{3.624083in}{3.350312in}}%
\pgfpathlineto{\pgfqpoint{3.623759in}{2.990422in}}%
\pgfpathlineto{\pgfqpoint{3.624253in}{3.217074in}}%
\pgfpathlineto{\pgfqpoint{3.625301in}{2.889203in}}%
\pgfpathlineto{\pgfqpoint{3.624992in}{3.329640in}}%
\pgfpathlineto{\pgfqpoint{3.625440in}{3.097968in}}%
\pgfpathlineto{\pgfqpoint{3.625501in}{3.347489in}}%
\pgfpathlineto{\pgfqpoint{3.626087in}{3.023698in}}%
\pgfpathlineto{\pgfqpoint{3.626565in}{3.197336in}}%
\pgfpathlineto{\pgfqpoint{3.627058in}{3.296701in}}%
\pgfpathlineto{\pgfqpoint{3.626796in}{2.996027in}}%
\pgfpathlineto{\pgfqpoint{3.627582in}{3.093523in}}%
\pgfpathlineto{\pgfqpoint{3.628399in}{2.975738in}}%
\pgfpathlineto{\pgfqpoint{3.627968in}{3.339949in}}%
\pgfpathlineto{\pgfqpoint{3.628584in}{3.263481in}}%
\pgfpathlineto{\pgfqpoint{3.628615in}{3.312629in}}%
\pgfpathlineto{\pgfqpoint{3.629170in}{2.908504in}}%
\pgfpathlineto{\pgfqpoint{3.629648in}{3.167536in}}%
\pgfpathlineto{\pgfqpoint{3.630742in}{2.907210in}}%
\pgfpathlineto{\pgfqpoint{3.630465in}{3.307257in}}%
\pgfpathlineto{\pgfqpoint{3.630773in}{3.017053in}}%
\pgfpathlineto{\pgfqpoint{3.630943in}{3.384105in}}%
\pgfpathlineto{\pgfqpoint{3.631513in}{2.958986in}}%
\pgfpathlineto{\pgfqpoint{3.631914in}{3.265832in}}%
\pgfpathlineto{\pgfqpoint{3.631929in}{3.325487in}}%
\pgfpathlineto{\pgfqpoint{3.632299in}{2.952230in}}%
\pgfpathlineto{\pgfqpoint{3.632977in}{3.139590in}}%
\pgfpathlineto{\pgfqpoint{3.633841in}{2.972083in}}%
\pgfpathlineto{\pgfqpoint{3.633424in}{3.354985in}}%
\pgfpathlineto{\pgfqpoint{3.633902in}{3.237384in}}%
\pgfpathlineto{\pgfqpoint{3.634411in}{3.318566in}}%
\pgfpathlineto{\pgfqpoint{3.634611in}{2.918251in}}%
\pgfpathlineto{\pgfqpoint{3.634997in}{3.206465in}}%
\pgfpathlineto{\pgfqpoint{3.635891in}{3.265225in}}%
\pgfpathlineto{\pgfqpoint{3.635382in}{3.041155in}}%
\pgfpathlineto{\pgfqpoint{3.636014in}{3.161559in}}%
\pgfpathlineto{\pgfqpoint{3.636184in}{2.928728in}}%
\pgfpathlineto{\pgfqpoint{3.636384in}{3.308707in}}%
\pgfpathlineto{\pgfqpoint{3.637093in}{3.103438in}}%
\pgfpathlineto{\pgfqpoint{3.637371in}{3.304775in}}%
\pgfpathlineto{\pgfqpoint{3.637710in}{3.021810in}}%
\pgfpathlineto{\pgfqpoint{3.638218in}{3.204279in}}%
\pgfpathlineto{\pgfqpoint{3.638357in}{3.215528in}}%
\pgfpathlineto{\pgfqpoint{3.638403in}{3.154659in}}%
\pgfpathlineto{\pgfqpoint{3.639282in}{2.973427in}}%
\pgfpathlineto{\pgfqpoint{3.638866in}{3.333927in}}%
\pgfpathlineto{\pgfqpoint{3.639482in}{3.224402in}}%
\pgfpathlineto{\pgfqpoint{3.639852in}{3.294198in}}%
\pgfpathlineto{\pgfqpoint{3.640053in}{2.958967in}}%
\pgfpathlineto{\pgfqpoint{3.640500in}{3.183895in}}%
\pgfpathlineto{\pgfqpoint{3.641625in}{2.975467in}}%
\pgfpathlineto{\pgfqpoint{3.641332in}{3.328515in}}%
\pgfpathlineto{\pgfqpoint{3.641641in}{3.045236in}}%
\pgfpathlineto{\pgfqpoint{3.641825in}{3.288279in}}%
\pgfpathlineto{\pgfqpoint{3.642380in}{2.975369in}}%
\pgfpathlineto{\pgfqpoint{3.642766in}{3.212399in}}%
\pgfpathlineto{\pgfqpoint{3.643321in}{3.299246in}}%
\pgfpathlineto{\pgfqpoint{3.643120in}{2.975815in}}%
\pgfpathlineto{\pgfqpoint{3.643845in}{3.197438in}}%
\pgfpathlineto{\pgfqpoint{3.643922in}{3.001562in}}%
\pgfpathlineto{\pgfqpoint{3.644307in}{3.267917in}}%
\pgfpathlineto{\pgfqpoint{3.644986in}{3.106036in}}%
\pgfpathlineto{\pgfqpoint{3.645664in}{3.263736in}}%
\pgfpathlineto{\pgfqpoint{3.645479in}{2.950529in}}%
\pgfpathlineto{\pgfqpoint{3.646080in}{3.091003in}}%
\pgfpathlineto{\pgfqpoint{3.646157in}{3.184043in}}%
\pgfpathlineto{\pgfqpoint{3.646774in}{3.298533in}}%
\pgfpathlineto{\pgfqpoint{3.647020in}{3.019854in}}%
\pgfpathlineto{\pgfqpoint{3.647282in}{3.240898in}}%
\pgfpathlineto{\pgfqpoint{3.647806in}{3.029473in}}%
\pgfpathlineto{\pgfqpoint{3.648269in}{3.245954in}}%
\pgfpathlineto{\pgfqpoint{3.648438in}{3.117534in}}%
\pgfpathlineto{\pgfqpoint{3.648747in}{3.295480in}}%
\pgfpathlineto{\pgfqpoint{3.648562in}{2.962863in}}%
\pgfpathlineto{\pgfqpoint{3.649641in}{3.219761in}}%
\pgfpathlineto{\pgfqpoint{3.650103in}{3.061897in}}%
\pgfpathlineto{\pgfqpoint{3.649733in}{3.279845in}}%
\pgfpathlineto{\pgfqpoint{3.650704in}{3.243072in}}%
\pgfpathlineto{\pgfqpoint{3.651460in}{3.269949in}}%
\pgfpathlineto{\pgfqpoint{3.650889in}{2.997535in}}%
\pgfpathlineto{\pgfqpoint{3.651629in}{3.106900in}}%
\pgfpathlineto{\pgfqpoint{3.652415in}{3.035104in}}%
\pgfpathlineto{\pgfqpoint{3.652215in}{3.301023in}}%
\pgfpathlineto{\pgfqpoint{3.652693in}{3.251508in}}%
\pgfpathlineto{\pgfqpoint{3.652724in}{3.276129in}}%
\pgfpathlineto{\pgfqpoint{3.652893in}{3.080515in}}%
\pgfpathlineto{\pgfqpoint{3.653757in}{3.216407in}}%
\pgfpathlineto{\pgfqpoint{3.653988in}{3.004009in}}%
\pgfpathlineto{\pgfqpoint{3.654204in}{3.290628in}}%
\pgfpathlineto{\pgfqpoint{3.654897in}{3.155470in}}%
\pgfpathlineto{\pgfqpoint{3.655313in}{3.258970in}}%
\pgfpathlineto{\pgfqpoint{3.655498in}{3.030297in}}%
\pgfpathlineto{\pgfqpoint{3.655961in}{3.090401in}}%
\pgfpathlineto{\pgfqpoint{3.655976in}{3.058292in}}%
\pgfpathlineto{\pgfqpoint{3.656069in}{3.246547in}}%
\pgfpathlineto{\pgfqpoint{3.657040in}{3.150147in}}%
\pgfpathlineto{\pgfqpoint{3.657841in}{3.034505in}}%
\pgfpathlineto{\pgfqpoint{3.657657in}{3.302690in}}%
\pgfpathlineto{\pgfqpoint{3.658104in}{3.178130in}}%
\pgfpathlineto{\pgfqpoint{3.658150in}{3.290913in}}%
\pgfpathlineto{\pgfqpoint{3.658581in}{3.081658in}}%
\pgfpathlineto{\pgfqpoint{3.659213in}{3.203789in}}%
\pgfpathlineto{\pgfqpoint{3.659337in}{3.084488in}}%
\pgfpathlineto{\pgfqpoint{3.659630in}{3.263168in}}%
\pgfpathlineto{\pgfqpoint{3.660323in}{3.160701in}}%
\pgfpathlineto{\pgfqpoint{3.661109in}{3.328714in}}%
\pgfpathlineto{\pgfqpoint{3.660924in}{3.000364in}}%
\pgfpathlineto{\pgfqpoint{3.661387in}{3.087864in}}%
\pgfpathlineto{\pgfqpoint{3.661695in}{3.032233in}}%
\pgfpathlineto{\pgfqpoint{3.661618in}{3.257807in}}%
\pgfpathlineto{\pgfqpoint{3.662481in}{3.103510in}}%
\pgfpathlineto{\pgfqpoint{3.663576in}{3.292005in}}%
\pgfpathlineto{\pgfqpoint{3.662898in}{3.031016in}}%
\pgfpathlineto{\pgfqpoint{3.663622in}{3.178572in}}%
\pgfpathlineto{\pgfqpoint{3.664778in}{3.034218in}}%
\pgfpathlineto{\pgfqpoint{3.664085in}{3.318417in}}%
\pgfpathlineto{\pgfqpoint{3.664794in}{3.073179in}}%
\pgfpathlineto{\pgfqpoint{3.665071in}{3.270484in}}%
\pgfpathlineto{\pgfqpoint{3.665256in}{3.058669in}}%
\pgfpathlineto{\pgfqpoint{3.665934in}{3.177345in}}%
\pgfpathlineto{\pgfqpoint{3.666551in}{3.319467in}}%
\pgfpathlineto{\pgfqpoint{3.666350in}{3.002590in}}%
\pgfpathlineto{\pgfqpoint{3.666967in}{3.171022in}}%
\pgfpathlineto{\pgfqpoint{3.667121in}{3.057754in}}%
\pgfpathlineto{\pgfqpoint{3.667044in}{3.252664in}}%
\pgfpathlineto{\pgfqpoint{3.668015in}{3.227025in}}%
\pgfpathlineto{\pgfqpoint{3.668046in}{3.305091in}}%
\pgfpathlineto{\pgfqpoint{3.668647in}{3.056803in}}%
\pgfpathlineto{\pgfqpoint{3.669094in}{3.156688in}}%
\pgfpathlineto{\pgfqpoint{3.670204in}{3.048705in}}%
\pgfpathlineto{\pgfqpoint{3.669495in}{3.292526in}}%
\pgfpathlineto{\pgfqpoint{3.670235in}{3.119038in}}%
\pgfpathlineto{\pgfqpoint{3.670512in}{3.280227in}}%
\pgfpathlineto{\pgfqpoint{3.670775in}{3.075564in}}%
\pgfpathlineto{\pgfqpoint{3.671360in}{3.195901in}}%
\pgfpathlineto{\pgfqpoint{3.671977in}{3.293290in}}%
\pgfpathlineto{\pgfqpoint{3.671761in}{2.979719in}}%
\pgfpathlineto{\pgfqpoint{3.672409in}{3.115747in}}%
\pgfpathlineto{\pgfqpoint{3.672840in}{3.252425in}}%
\pgfpathlineto{\pgfqpoint{3.672532in}{3.025094in}}%
\pgfpathlineto{\pgfqpoint{3.673657in}{3.197026in}}%
\pgfpathlineto{\pgfqpoint{3.674073in}{3.030402in}}%
\pgfpathlineto{\pgfqpoint{3.674412in}{3.252327in}}%
\pgfpathlineto{\pgfqpoint{3.674767in}{3.154599in}}%
\pgfpathlineto{\pgfqpoint{3.675815in}{3.264899in}}%
\pgfpathlineto{\pgfqpoint{3.675630in}{3.051274in}}%
\pgfpathlineto{\pgfqpoint{3.675939in}{3.223983in}}%
\pgfpathlineto{\pgfqpoint{3.677156in}{3.021367in}}%
\pgfpathlineto{\pgfqpoint{3.676586in}{3.226395in}}%
\pgfpathlineto{\pgfqpoint{3.677172in}{3.026159in}}%
\pgfpathlineto{\pgfqpoint{3.677357in}{3.290849in}}%
\pgfpathlineto{\pgfqpoint{3.678343in}{3.245039in}}%
\pgfpathlineto{\pgfqpoint{3.679484in}{3.039763in}}%
\pgfpathlineto{\pgfqpoint{3.678944in}{3.262339in}}%
\pgfpathlineto{\pgfqpoint{3.679499in}{3.059214in}}%
\pgfpathlineto{\pgfqpoint{3.679808in}{3.282594in}}%
\pgfpathlineto{\pgfqpoint{3.680640in}{3.164632in}}%
\pgfpathlineto{\pgfqpoint{3.680779in}{3.220823in}}%
\pgfpathlineto{\pgfqpoint{3.681010in}{3.059136in}}%
\pgfpathlineto{\pgfqpoint{3.681812in}{3.022619in}}%
\pgfpathlineto{\pgfqpoint{3.681257in}{3.297830in}}%
\pgfpathlineto{\pgfqpoint{3.681997in}{3.178501in}}%
\pgfpathlineto{\pgfqpoint{3.682120in}{3.255928in}}%
\pgfpathlineto{\pgfqpoint{3.682567in}{3.069835in}}%
\pgfpathlineto{\pgfqpoint{3.683106in}{3.202676in}}%
\pgfpathlineto{\pgfqpoint{3.683338in}{3.062566in}}%
\pgfpathlineto{\pgfqpoint{3.683584in}{3.245633in}}%
\pgfpathlineto{\pgfqpoint{3.684170in}{3.183456in}}%
\pgfpathlineto{\pgfqpoint{3.685203in}{3.270636in}}%
\pgfpathlineto{\pgfqpoint{3.684895in}{3.081905in}}%
\pgfpathlineto{\pgfqpoint{3.685264in}{3.127516in}}%
\pgfpathlineto{\pgfqpoint{3.685773in}{3.106017in}}%
\pgfpathlineto{\pgfqpoint{3.685604in}{3.194064in}}%
\pgfpathlineto{\pgfqpoint{3.685850in}{3.171446in}}%
\pgfpathlineto{\pgfqpoint{3.686667in}{3.278198in}}%
\pgfpathlineto{\pgfqpoint{3.686451in}{3.034730in}}%
\pgfpathlineto{\pgfqpoint{3.686929in}{3.134013in}}%
\pgfpathlineto{\pgfqpoint{3.687222in}{3.050959in}}%
\pgfpathlineto{\pgfqpoint{3.687145in}{3.244265in}}%
\pgfpathlineto{\pgfqpoint{3.688008in}{3.122831in}}%
\pgfpathlineto{\pgfqpoint{3.689087in}{3.262250in}}%
\pgfpathlineto{\pgfqpoint{3.688764in}{3.047428in}}%
\pgfpathlineto{\pgfqpoint{3.689134in}{3.187334in}}%
\pgfpathlineto{\pgfqpoint{3.690305in}{3.070170in}}%
\pgfpathlineto{\pgfqpoint{3.689596in}{3.256913in}}%
\pgfpathlineto{\pgfqpoint{3.690321in}{3.077562in}}%
\pgfpathlineto{\pgfqpoint{3.690567in}{3.242851in}}%
\pgfpathlineto{\pgfqpoint{3.691446in}{3.170984in}}%
\pgfpathlineto{\pgfqpoint{3.691847in}{3.058434in}}%
\pgfpathlineto{\pgfqpoint{3.692047in}{3.270498in}}%
\pgfpathlineto{\pgfqpoint{3.692525in}{3.202339in}}%
\pgfpathlineto{\pgfqpoint{3.692956in}{3.243357in}}%
\pgfpathlineto{\pgfqpoint{3.692633in}{3.073141in}}%
\pgfpathlineto{\pgfqpoint{3.693388in}{3.096201in}}%
\pgfpathlineto{\pgfqpoint{3.694190in}{3.026412in}}%
\pgfpathlineto{\pgfqpoint{3.693635in}{3.246785in}}%
\pgfpathlineto{\pgfqpoint{3.694452in}{3.181527in}}%
\pgfpathlineto{\pgfqpoint{3.694498in}{3.275489in}}%
\pgfpathlineto{\pgfqpoint{3.694945in}{3.091454in}}%
\pgfpathlineto{\pgfqpoint{3.695531in}{3.138032in}}%
\pgfpathlineto{\pgfqpoint{3.695747in}{3.078827in}}%
\pgfpathlineto{\pgfqpoint{3.695947in}{3.275539in}}%
\pgfpathlineto{\pgfqpoint{3.695962in}{3.301749in}}%
\pgfpathlineto{\pgfqpoint{3.696502in}{3.058864in}}%
\pgfpathlineto{\pgfqpoint{3.696980in}{3.167812in}}%
\pgfpathlineto{\pgfqpoint{3.697257in}{3.049442in}}%
\pgfpathlineto{\pgfqpoint{3.697458in}{3.271343in}}%
\pgfpathlineto{\pgfqpoint{3.698074in}{3.142696in}}%
\pgfpathlineto{\pgfqpoint{3.699045in}{3.252225in}}%
\pgfpathlineto{\pgfqpoint{3.698830in}{3.067886in}}%
\pgfpathlineto{\pgfqpoint{3.699184in}{3.199624in}}%
\pgfpathlineto{\pgfqpoint{3.699600in}{3.036015in}}%
\pgfpathlineto{\pgfqpoint{3.699909in}{3.291473in}}%
\pgfpathlineto{\pgfqpoint{3.700294in}{3.162351in}}%
\pgfpathlineto{\pgfqpoint{3.700356in}{3.069737in}}%
\pgfpathlineto{\pgfqpoint{3.700556in}{3.247329in}}%
\pgfpathlineto{\pgfqpoint{3.701311in}{3.193104in}}%
\pgfpathlineto{\pgfqpoint{3.701373in}{3.295692in}}%
\pgfpathlineto{\pgfqpoint{3.701928in}{3.066749in}}%
\pgfpathlineto{\pgfqpoint{3.702390in}{3.174042in}}%
\pgfpathlineto{\pgfqpoint{3.702668in}{3.013858in}}%
\pgfpathlineto{\pgfqpoint{3.702868in}{3.307306in}}%
\pgfpathlineto{\pgfqpoint{3.703485in}{3.138534in}}%
\pgfpathlineto{\pgfqpoint{3.703855in}{3.272046in}}%
\pgfpathlineto{\pgfqpoint{3.704240in}{3.066648in}}%
\pgfpathlineto{\pgfqpoint{3.704595in}{3.199600in}}%
\pgfpathlineto{\pgfqpoint{3.705627in}{3.032578in}}%
\pgfpathlineto{\pgfqpoint{3.705319in}{3.286648in}}%
\pgfpathlineto{\pgfqpoint{3.705720in}{3.159127in}}%
\pgfpathlineto{\pgfqpoint{3.705766in}{3.065230in}}%
\pgfpathlineto{\pgfqpoint{3.705967in}{3.276538in}}%
\pgfpathlineto{\pgfqpoint{3.706768in}{3.272217in}}%
\pgfpathlineto{\pgfqpoint{3.706799in}{3.296770in}}%
\pgfpathlineto{\pgfqpoint{3.707107in}{3.055984in}}%
\pgfpathlineto{\pgfqpoint{3.707816in}{3.215311in}}%
\pgfpathlineto{\pgfqpoint{3.708109in}{2.965979in}}%
\pgfpathlineto{\pgfqpoint{3.708310in}{3.335051in}}%
\pgfpathlineto{\pgfqpoint{3.708911in}{3.182131in}}%
\pgfpathlineto{\pgfqpoint{3.709882in}{3.279174in}}%
\pgfpathlineto{\pgfqpoint{3.709604in}{3.039451in}}%
\pgfpathlineto{\pgfqpoint{3.710021in}{3.206955in}}%
\pgfpathlineto{\pgfqpoint{3.710761in}{3.300839in}}%
\pgfpathlineto{\pgfqpoint{3.710098in}{3.050589in}}%
\pgfpathlineto{\pgfqpoint{3.710946in}{3.115093in}}%
\pgfpathlineto{\pgfqpoint{3.711963in}{2.969132in}}%
\pgfpathlineto{\pgfqpoint{3.711408in}{3.323969in}}%
\pgfpathlineto{\pgfqpoint{3.712009in}{3.176173in}}%
\pgfpathlineto{\pgfqpoint{3.712271in}{3.327319in}}%
\pgfpathlineto{\pgfqpoint{3.712564in}{3.044297in}}%
\pgfpathlineto{\pgfqpoint{3.713134in}{3.216755in}}%
\pgfpathlineto{\pgfqpoint{3.713535in}{3.010374in}}%
\pgfpathlineto{\pgfqpoint{3.713751in}{3.367965in}}%
\pgfpathlineto{\pgfqpoint{3.714229in}{3.215100in}}%
\pgfpathlineto{\pgfqpoint{3.714491in}{3.285249in}}%
\pgfpathlineto{\pgfqpoint{3.714938in}{3.026779in}}%
\pgfpathlineto{\pgfqpoint{3.715031in}{3.032516in}}%
\pgfpathlineto{\pgfqpoint{3.715046in}{2.967377in}}%
\pgfpathlineto{\pgfqpoint{3.715246in}{3.327776in}}%
\pgfpathlineto{\pgfqpoint{3.716079in}{3.270014in}}%
\pgfpathlineto{\pgfqpoint{3.716603in}{3.057781in}}%
\pgfpathlineto{\pgfqpoint{3.716233in}{3.313729in}}%
\pgfpathlineto{\pgfqpoint{3.717204in}{3.218300in}}%
\pgfpathlineto{\pgfqpoint{3.717713in}{3.328736in}}%
\pgfpathlineto{\pgfqpoint{3.717389in}{3.006971in}}%
\pgfpathlineto{\pgfqpoint{3.718006in}{3.054382in}}%
\pgfpathlineto{\pgfqpoint{3.718915in}{2.996227in}}%
\pgfpathlineto{\pgfqpoint{3.718221in}{3.306617in}}%
\pgfpathlineto{\pgfqpoint{3.719054in}{3.156654in}}%
\pgfpathlineto{\pgfqpoint{3.719193in}{3.318657in}}%
\pgfpathlineto{\pgfqpoint{3.719501in}{3.057065in}}%
\pgfpathlineto{\pgfqpoint{3.720194in}{3.252967in}}%
\pgfpathlineto{\pgfqpoint{3.720487in}{2.980762in}}%
\pgfpathlineto{\pgfqpoint{3.720688in}{3.348508in}}%
\pgfpathlineto{\pgfqpoint{3.721381in}{3.089244in}}%
\pgfpathlineto{\pgfqpoint{3.721674in}{3.296360in}}%
\pgfpathlineto{\pgfqpoint{3.721983in}{3.044083in}}%
\pgfpathlineto{\pgfqpoint{3.722507in}{3.176491in}}%
\pgfpathlineto{\pgfqpoint{3.723154in}{3.317729in}}%
\pgfpathlineto{\pgfqpoint{3.723462in}{3.020434in}}%
\pgfpathlineto{\pgfqpoint{3.723555in}{3.083213in}}%
\pgfpathlineto{\pgfqpoint{3.724341in}{2.988874in}}%
\pgfpathlineto{\pgfqpoint{3.723647in}{3.283188in}}%
\pgfpathlineto{\pgfqpoint{3.724603in}{3.231599in}}%
\pgfpathlineto{\pgfqpoint{3.724649in}{3.312911in}}%
\pgfpathlineto{\pgfqpoint{3.724942in}{3.074533in}}%
\pgfpathlineto{\pgfqpoint{3.725667in}{3.164955in}}%
\pgfpathlineto{\pgfqpoint{3.725898in}{3.005004in}}%
\pgfpathlineto{\pgfqpoint{3.726129in}{3.341913in}}%
\pgfpathlineto{\pgfqpoint{3.726792in}{3.114715in}}%
\pgfpathlineto{\pgfqpoint{3.727609in}{3.293478in}}%
\pgfpathlineto{\pgfqpoint{3.727424in}{3.030579in}}%
\pgfpathlineto{\pgfqpoint{3.727886in}{3.105096in}}%
\pgfpathlineto{\pgfqpoint{3.728210in}{3.022056in}}%
\pgfpathlineto{\pgfqpoint{3.728441in}{3.267923in}}%
\pgfpathlineto{\pgfqpoint{3.728565in}{3.253802in}}%
\pgfpathlineto{\pgfqpoint{3.728596in}{3.270017in}}%
\pgfpathlineto{\pgfqpoint{3.728873in}{3.049706in}}%
\pgfpathlineto{\pgfqpoint{3.729505in}{3.190468in}}%
\pgfpathlineto{\pgfqpoint{3.729752in}{3.022277in}}%
\pgfpathlineto{\pgfqpoint{3.730060in}{3.289558in}}%
\pgfpathlineto{\pgfqpoint{3.730646in}{3.102635in}}%
\pgfpathlineto{\pgfqpoint{3.731524in}{3.295207in}}%
\pgfpathlineto{\pgfqpoint{3.731309in}{3.009409in}}%
\pgfpathlineto{\pgfqpoint{3.731802in}{3.126464in}}%
\pgfpathlineto{\pgfqpoint{3.732079in}{3.047151in}}%
\pgfpathlineto{\pgfqpoint{3.732403in}{3.264957in}}%
\pgfpathlineto{\pgfqpoint{3.732881in}{3.122980in}}%
\pgfpathlineto{\pgfqpoint{3.733960in}{3.262991in}}%
\pgfpathlineto{\pgfqpoint{3.733621in}{3.024346in}}%
\pgfpathlineto{\pgfqpoint{3.734006in}{3.205582in}}%
\pgfpathlineto{\pgfqpoint{3.734469in}{3.261328in}}%
\pgfpathlineto{\pgfqpoint{3.734268in}{3.060234in}}%
\pgfpathlineto{\pgfqpoint{3.734993in}{3.160151in}}%
\pgfpathlineto{\pgfqpoint{3.735178in}{3.037069in}}%
\pgfpathlineto{\pgfqpoint{3.735378in}{3.283721in}}%
\pgfpathlineto{\pgfqpoint{3.736103in}{3.156699in}}%
\pgfpathlineto{\pgfqpoint{3.736920in}{3.305990in}}%
\pgfpathlineto{\pgfqpoint{3.736719in}{3.034072in}}%
\pgfpathlineto{\pgfqpoint{3.737212in}{3.166829in}}%
\pgfpathlineto{\pgfqpoint{3.737397in}{3.194275in}}%
\pgfpathlineto{\pgfqpoint{3.737459in}{3.080784in}}%
\pgfpathlineto{\pgfqpoint{3.737490in}{3.033224in}}%
\pgfpathlineto{\pgfqpoint{3.737891in}{3.269634in}}%
\pgfpathlineto{\pgfqpoint{3.738492in}{3.259272in}}%
\pgfpathlineto{\pgfqpoint{3.739355in}{3.296085in}}%
\pgfpathlineto{\pgfqpoint{3.739047in}{3.048797in}}%
\pgfpathlineto{\pgfqpoint{3.739401in}{3.187561in}}%
\pgfpathlineto{\pgfqpoint{3.740033in}{3.255379in}}%
\pgfpathlineto{\pgfqpoint{3.740573in}{3.019458in}}%
\pgfpathlineto{\pgfqpoint{3.740588in}{3.018546in}}%
\pgfpathlineto{\pgfqpoint{3.740604in}{3.047021in}}%
\pgfpathlineto{\pgfqpoint{3.740820in}{3.316895in}}%
\pgfpathlineto{\pgfqpoint{3.741374in}{3.026283in}}%
\pgfpathlineto{\pgfqpoint{3.741775in}{3.212295in}}%
\pgfpathlineto{\pgfqpoint{3.742299in}{3.257010in}}%
\pgfpathlineto{\pgfqpoint{3.742130in}{3.057849in}}%
\pgfpathlineto{\pgfqpoint{3.742793in}{3.168815in}}%
\pgfpathlineto{\pgfqpoint{3.742885in}{3.075012in}}%
\pgfpathlineto{\pgfqpoint{3.743255in}{3.268945in}}%
\pgfpathlineto{\pgfqpoint{3.743733in}{3.218829in}}%
\pgfpathlineto{\pgfqpoint{3.744658in}{3.275461in}}%
\pgfpathlineto{\pgfqpoint{3.744457in}{3.043020in}}%
\pgfpathlineto{\pgfqpoint{3.744827in}{3.168808in}}%
\pgfpathlineto{\pgfqpoint{3.745413in}{3.241371in}}%
\pgfpathlineto{\pgfqpoint{3.745244in}{3.090916in}}%
\pgfpathlineto{\pgfqpoint{3.745860in}{3.114727in}}%
\pgfpathlineto{\pgfqpoint{3.746014in}{3.042661in}}%
\pgfpathlineto{\pgfqpoint{3.746168in}{3.225666in}}%
\pgfpathlineto{\pgfqpoint{3.746199in}{3.296359in}}%
\pgfpathlineto{\pgfqpoint{3.746785in}{3.030403in}}%
\pgfpathlineto{\pgfqpoint{3.747232in}{3.138080in}}%
\pgfpathlineto{\pgfqpoint{3.747556in}{3.053154in}}%
\pgfpathlineto{\pgfqpoint{3.747355in}{3.183274in}}%
\pgfpathlineto{\pgfqpoint{3.747602in}{3.163329in}}%
\pgfpathlineto{\pgfqpoint{3.748558in}{3.261700in}}%
\pgfpathlineto{\pgfqpoint{3.748311in}{3.050796in}}%
\pgfpathlineto{\pgfqpoint{3.748712in}{3.180979in}}%
\pgfpathlineto{\pgfqpoint{3.749082in}{3.070265in}}%
\pgfpathlineto{\pgfqpoint{3.749298in}{3.233335in}}%
\pgfpathlineto{\pgfqpoint{3.750099in}{3.302543in}}%
\pgfpathlineto{\pgfqpoint{3.749883in}{3.040740in}}%
\pgfpathlineto{\pgfqpoint{3.750315in}{3.141863in}}%
\pgfpathlineto{\pgfqpoint{3.750654in}{3.033867in}}%
\pgfpathlineto{\pgfqpoint{3.750870in}{3.240520in}}%
\pgfpathlineto{\pgfqpoint{3.751456in}{3.120479in}}%
\pgfpathlineto{\pgfqpoint{3.752396in}{3.274283in}}%
\pgfpathlineto{\pgfqpoint{3.752211in}{3.057427in}}%
\pgfpathlineto{\pgfqpoint{3.752581in}{3.215620in}}%
\pgfpathlineto{\pgfqpoint{3.753753in}{3.030479in}}%
\pgfpathlineto{\pgfqpoint{3.753167in}{3.266203in}}%
\pgfpathlineto{\pgfqpoint{3.753768in}{3.071093in}}%
\pgfpathlineto{\pgfqpoint{3.753999in}{3.258682in}}%
\pgfpathlineto{\pgfqpoint{3.754909in}{3.196803in}}%
\pgfpathlineto{\pgfqpoint{3.755279in}{3.016370in}}%
\pgfpathlineto{\pgfqpoint{3.755494in}{3.310862in}}%
\pgfpathlineto{\pgfqpoint{3.756096in}{3.129323in}}%
\pgfpathlineto{\pgfqpoint{3.756466in}{3.268856in}}%
\pgfpathlineto{\pgfqpoint{3.756836in}{3.052999in}}%
\pgfpathlineto{\pgfqpoint{3.757205in}{3.180722in}}%
\pgfpathlineto{\pgfqpoint{3.758377in}{3.047286in}}%
\pgfpathlineto{\pgfqpoint{3.757945in}{3.293031in}}%
\pgfpathlineto{\pgfqpoint{3.758408in}{3.123634in}}%
\pgfpathlineto{\pgfqpoint{3.759410in}{3.286956in}}%
\pgfpathlineto{\pgfqpoint{3.759163in}{3.065438in}}%
\pgfpathlineto{\pgfqpoint{3.759518in}{3.200791in}}%
\pgfpathlineto{\pgfqpoint{3.760705in}{2.966271in}}%
\pgfpathlineto{\pgfqpoint{3.760150in}{3.279325in}}%
\pgfpathlineto{\pgfqpoint{3.760735in}{3.114120in}}%
\pgfpathlineto{\pgfqpoint{3.760905in}{3.338641in}}%
\pgfpathlineto{\pgfqpoint{3.761475in}{3.025701in}}%
\pgfpathlineto{\pgfqpoint{3.761876in}{3.267284in}}%
\pgfpathlineto{\pgfqpoint{3.762400in}{3.276262in}}%
\pgfpathlineto{\pgfqpoint{3.762184in}{3.038289in}}%
\pgfpathlineto{\pgfqpoint{3.762647in}{3.187422in}}%
\pgfpathlineto{\pgfqpoint{3.763371in}{3.307276in}}%
\pgfpathlineto{\pgfqpoint{3.763803in}{3.001022in}}%
\pgfpathlineto{\pgfqpoint{3.763865in}{3.282522in}}%
\pgfpathlineto{\pgfqpoint{3.764959in}{3.219090in}}%
\pgfpathlineto{\pgfqpoint{3.765591in}{3.266956in}}%
\pgfpathlineto{\pgfqpoint{3.765175in}{3.066322in}}%
\pgfpathlineto{\pgfqpoint{3.765930in}{3.175389in}}%
\pgfpathlineto{\pgfqpoint{3.766131in}{3.009251in}}%
\pgfpathlineto{\pgfqpoint{3.766346in}{3.329960in}}%
\pgfpathlineto{\pgfqpoint{3.767040in}{3.161456in}}%
\pgfpathlineto{\pgfqpoint{3.767857in}{3.322197in}}%
\pgfpathlineto{\pgfqpoint{3.767657in}{2.954919in}}%
\pgfpathlineto{\pgfqpoint{3.768119in}{3.099571in}}%
\pgfpathlineto{\pgfqpoint{3.769137in}{3.029260in}}%
\pgfpathlineto{\pgfqpoint{3.768844in}{3.280239in}}%
\pgfpathlineto{\pgfqpoint{3.769229in}{3.059104in}}%
\pgfpathlineto{\pgfqpoint{3.770324in}{3.314056in}}%
\pgfpathlineto{\pgfqpoint{3.770000in}{3.029657in}}%
\pgfpathlineto{\pgfqpoint{3.770431in}{3.208512in}}%
\pgfpathlineto{\pgfqpoint{3.770447in}{3.211747in}}%
\pgfpathlineto{\pgfqpoint{3.770616in}{3.037942in}}%
\pgfpathlineto{\pgfqpoint{3.771526in}{2.989025in}}%
\pgfpathlineto{\pgfqpoint{3.770940in}{3.295906in}}%
\pgfpathlineto{\pgfqpoint{3.771649in}{3.118703in}}%
\pgfpathlineto{\pgfqpoint{3.771834in}{3.280413in}}%
\pgfpathlineto{\pgfqpoint{3.772127in}{3.059460in}}%
\pgfpathlineto{\pgfqpoint{3.772774in}{3.197343in}}%
\pgfpathlineto{\pgfqpoint{3.773314in}{3.334304in}}%
\pgfpathlineto{\pgfqpoint{3.773098in}{2.987754in}}%
\pgfpathlineto{\pgfqpoint{3.773823in}{3.173511in}}%
\pgfpathlineto{\pgfqpoint{3.774609in}{2.978307in}}%
\pgfpathlineto{\pgfqpoint{3.774809in}{3.324953in}}%
\pgfpathlineto{\pgfqpoint{3.774902in}{3.227144in}}%
\pgfpathlineto{\pgfqpoint{3.775796in}{3.290022in}}%
\pgfpathlineto{\pgfqpoint{3.775395in}{3.061918in}}%
\pgfpathlineto{\pgfqpoint{3.775904in}{3.203443in}}%
\pgfpathlineto{\pgfqpoint{3.776967in}{2.983576in}}%
\pgfpathlineto{\pgfqpoint{3.776397in}{3.281991in}}%
\pgfpathlineto{\pgfqpoint{3.776998in}{3.126785in}}%
\pgfpathlineto{\pgfqpoint{3.777276in}{3.320992in}}%
\pgfpathlineto{\pgfqpoint{3.777584in}{3.041615in}}%
\pgfpathlineto{\pgfqpoint{3.778123in}{3.222979in}}%
\pgfpathlineto{\pgfqpoint{3.778463in}{3.023664in}}%
\pgfpathlineto{\pgfqpoint{3.778740in}{3.295778in}}%
\pgfpathlineto{\pgfqpoint{3.779310in}{3.092969in}}%
\pgfpathlineto{\pgfqpoint{3.780251in}{3.332450in}}%
\pgfpathlineto{\pgfqpoint{3.780050in}{2.956322in}}%
\pgfpathlineto{\pgfqpoint{3.780420in}{3.149879in}}%
\pgfpathlineto{\pgfqpoint{3.780821in}{3.019285in}}%
\pgfpathlineto{\pgfqpoint{3.780991in}{3.259726in}}%
\pgfpathlineto{\pgfqpoint{3.781561in}{3.077424in}}%
\pgfpathlineto{\pgfqpoint{3.782717in}{3.304361in}}%
\pgfpathlineto{\pgfqpoint{3.782393in}{3.051495in}}%
\pgfpathlineto{\pgfqpoint{3.782763in}{3.180889in}}%
\pgfpathlineto{\pgfqpoint{3.783904in}{3.019688in}}%
\pgfpathlineto{\pgfqpoint{3.783349in}{3.298440in}}%
\pgfpathlineto{\pgfqpoint{3.783935in}{3.073752in}}%
\pgfpathlineto{\pgfqpoint{3.784921in}{3.276955in}}%
\pgfpathlineto{\pgfqpoint{3.785076in}{3.207034in}}%
\pgfpathlineto{\pgfqpoint{3.785492in}{3.013552in}}%
\pgfpathlineto{\pgfqpoint{3.785677in}{3.335397in}}%
\pgfpathlineto{\pgfqpoint{3.786278in}{3.096813in}}%
\pgfpathlineto{\pgfqpoint{3.786432in}{3.274192in}}%
\pgfpathlineto{\pgfqpoint{3.786972in}{3.058196in}}%
\pgfpathlineto{\pgfqpoint{3.787403in}{3.203089in}}%
\pgfpathlineto{\pgfqpoint{3.788575in}{3.054634in}}%
\pgfpathlineto{\pgfqpoint{3.788004in}{3.296574in}}%
\pgfpathlineto{\pgfqpoint{3.788605in}{3.099036in}}%
\pgfpathlineto{\pgfqpoint{3.788790in}{3.290199in}}%
\pgfpathlineto{\pgfqpoint{3.789330in}{3.057894in}}%
\pgfpathlineto{\pgfqpoint{3.789715in}{3.157378in}}%
\pgfpathlineto{\pgfqpoint{3.790085in}{3.084809in}}%
\pgfpathlineto{\pgfqpoint{3.790317in}{3.271050in}}%
\pgfpathlineto{\pgfqpoint{3.791118in}{3.324886in}}%
\pgfpathlineto{\pgfqpoint{3.790902in}{3.014529in}}%
\pgfpathlineto{\pgfqpoint{3.791303in}{3.199546in}}%
\pgfpathlineto{\pgfqpoint{3.791673in}{3.030065in}}%
\pgfpathlineto{\pgfqpoint{3.791997in}{3.248426in}}%
\pgfpathlineto{\pgfqpoint{3.792475in}{3.117617in}}%
\pgfpathlineto{\pgfqpoint{3.793430in}{3.280122in}}%
\pgfpathlineto{\pgfqpoint{3.793230in}{3.045350in}}%
\pgfpathlineto{\pgfqpoint{3.793615in}{3.208879in}}%
\pgfpathlineto{\pgfqpoint{3.794001in}{3.053377in}}%
\pgfpathlineto{\pgfqpoint{3.794232in}{3.247109in}}%
\pgfpathlineto{\pgfqpoint{3.794787in}{3.062180in}}%
\pgfpathlineto{\pgfqpoint{3.795003in}{3.280890in}}%
\pgfpathlineto{\pgfqpoint{3.796020in}{3.194241in}}%
\pgfpathlineto{\pgfqpoint{3.796313in}{3.055037in}}%
\pgfpathlineto{\pgfqpoint{3.796498in}{3.304371in}}%
\pgfpathlineto{\pgfqpoint{3.797145in}{3.117265in}}%
\pgfpathlineto{\pgfqpoint{3.798070in}{3.300065in}}%
\pgfpathlineto{\pgfqpoint{3.797870in}{3.063352in}}%
\pgfpathlineto{\pgfqpoint{3.798255in}{3.159127in}}%
\pgfpathlineto{\pgfqpoint{3.798856in}{3.271295in}}%
\pgfpathlineto{\pgfqpoint{3.798625in}{3.044686in}}%
\pgfpathlineto{\pgfqpoint{3.799195in}{3.129252in}}%
\pgfpathlineto{\pgfqpoint{3.799380in}{3.103185in}}%
\pgfpathlineto{\pgfqpoint{3.799458in}{3.148490in}}%
\pgfpathlineto{\pgfqpoint{3.800413in}{3.334416in}}%
\pgfpathlineto{\pgfqpoint{3.800197in}{3.039360in}}%
\pgfpathlineto{\pgfqpoint{3.800567in}{3.181912in}}%
\pgfpathlineto{\pgfqpoint{3.800984in}{3.056392in}}%
\pgfpathlineto{\pgfqpoint{3.801307in}{3.246603in}}%
\pgfpathlineto{\pgfqpoint{3.801770in}{3.097404in}}%
\pgfpathlineto{\pgfqpoint{3.802725in}{3.288085in}}%
\pgfpathlineto{\pgfqpoint{3.802525in}{3.063998in}}%
\pgfpathlineto{\pgfqpoint{3.802895in}{3.202555in}}%
\pgfpathlineto{\pgfqpoint{3.803527in}{3.291937in}}%
\pgfpathlineto{\pgfqpoint{3.804082in}{3.031869in}}%
\pgfpathlineto{\pgfqpoint{3.804298in}{3.290533in}}%
\pgfpathlineto{\pgfqpoint{3.805254in}{3.184084in}}%
\pgfpathlineto{\pgfqpoint{3.805808in}{3.291096in}}%
\pgfpathlineto{\pgfqpoint{3.805623in}{3.049681in}}%
\pgfpathlineto{\pgfqpoint{3.806271in}{3.154515in}}%
\pgfpathlineto{\pgfqpoint{3.806410in}{3.046369in}}%
\pgfpathlineto{\pgfqpoint{3.806733in}{3.264072in}}%
\pgfpathlineto{\pgfqpoint{3.807227in}{3.233697in}}%
\pgfpathlineto{\pgfqpoint{3.808136in}{3.283671in}}%
\pgfpathlineto{\pgfqpoint{3.807951in}{3.055532in}}%
\pgfpathlineto{\pgfqpoint{3.808306in}{3.172208in}}%
\pgfpathlineto{\pgfqpoint{3.808907in}{3.288849in}}%
\pgfpathlineto{\pgfqpoint{3.809493in}{3.062598in}}%
\pgfpathlineto{\pgfqpoint{3.809693in}{3.335407in}}%
\pgfpathlineto{\pgfqpoint{3.810279in}{3.046093in}}%
\pgfpathlineto{\pgfqpoint{3.810803in}{3.118994in}}%
\pgfpathlineto{\pgfqpoint{3.811820in}{3.033396in}}%
\pgfpathlineto{\pgfqpoint{3.811250in}{3.259906in}}%
\pgfpathlineto{\pgfqpoint{3.811851in}{3.098780in}}%
\pgfpathlineto{\pgfqpoint{3.812129in}{3.295926in}}%
\pgfpathlineto{\pgfqpoint{3.812576in}{3.071856in}}%
\pgfpathlineto{\pgfqpoint{3.812976in}{3.174555in}}%
\pgfpathlineto{\pgfqpoint{3.813377in}{3.035506in}}%
\pgfpathlineto{\pgfqpoint{3.813562in}{3.261780in}}%
\pgfpathlineto{\pgfqpoint{3.813593in}{3.323653in}}%
\pgfpathlineto{\pgfqpoint{3.814163in}{3.044946in}}%
\pgfpathlineto{\pgfqpoint{3.814626in}{3.153843in}}%
\pgfpathlineto{\pgfqpoint{3.814934in}{3.063110in}}%
\pgfpathlineto{\pgfqpoint{3.815088in}{3.303678in}}%
\pgfpathlineto{\pgfqpoint{3.815736in}{3.110604in}}%
\pgfpathlineto{\pgfqpoint{3.816676in}{3.301798in}}%
\pgfpathlineto{\pgfqpoint{3.816476in}{3.048753in}}%
\pgfpathlineto{\pgfqpoint{3.816861in}{3.163539in}}%
\pgfpathlineto{\pgfqpoint{3.817555in}{3.277822in}}%
\pgfpathlineto{\pgfqpoint{3.817246in}{3.043938in}}%
\pgfpathlineto{\pgfqpoint{3.817940in}{3.101458in}}%
\pgfpathlineto{\pgfqpoint{3.819004in}{3.346627in}}%
\pgfpathlineto{\pgfqpoint{3.818803in}{2.994007in}}%
\pgfpathlineto{\pgfqpoint{3.819173in}{3.178131in}}%
\pgfpathlineto{\pgfqpoint{3.820298in}{3.032899in}}%
\pgfpathlineto{\pgfqpoint{3.819743in}{3.297453in}}%
\pgfpathlineto{\pgfqpoint{3.820360in}{3.042024in}}%
\pgfpathlineto{\pgfqpoint{3.820514in}{3.309243in}}%
\pgfpathlineto{\pgfqpoint{3.821547in}{3.172044in}}%
\pgfpathlineto{\pgfqpoint{3.821717in}{3.211159in}}%
\pgfpathlineto{\pgfqpoint{3.821747in}{3.113406in}}%
\pgfpathlineto{\pgfqpoint{3.821778in}{3.001737in}}%
\pgfpathlineto{\pgfqpoint{3.821979in}{3.310511in}}%
\pgfpathlineto{\pgfqpoint{3.822826in}{3.228865in}}%
\pgfpathlineto{\pgfqpoint{3.822981in}{3.299773in}}%
\pgfpathlineto{\pgfqpoint{3.823273in}{3.040860in}}%
\pgfpathlineto{\pgfqpoint{3.823720in}{3.200468in}}%
\pgfpathlineto{\pgfqpoint{3.824245in}{2.974978in}}%
\pgfpathlineto{\pgfqpoint{3.824445in}{3.331822in}}%
\pgfpathlineto{\pgfqpoint{3.824846in}{3.115300in}}%
\pgfpathlineto{\pgfqpoint{3.825755in}{2.984287in}}%
\pgfpathlineto{\pgfqpoint{3.825185in}{3.296863in}}%
\pgfpathlineto{\pgfqpoint{3.825894in}{3.162595in}}%
\pgfpathlineto{\pgfqpoint{3.825956in}{3.337592in}}%
\pgfpathlineto{\pgfqpoint{3.826249in}{3.038309in}}%
\pgfpathlineto{\pgfqpoint{3.827004in}{3.168640in}}%
\pgfpathlineto{\pgfqpoint{3.827235in}{3.014477in}}%
\pgfpathlineto{\pgfqpoint{3.827543in}{3.307842in}}%
\pgfpathlineto{\pgfqpoint{3.827913in}{3.231294in}}%
\pgfpathlineto{\pgfqpoint{3.828422in}{3.334928in}}%
\pgfpathlineto{\pgfqpoint{3.828114in}{3.006816in}}%
\pgfpathlineto{\pgfqpoint{3.828823in}{3.102083in}}%
\pgfpathlineto{\pgfqpoint{3.829609in}{3.043913in}}%
\pgfpathlineto{\pgfqpoint{3.829054in}{3.316573in}}%
\pgfpathlineto{\pgfqpoint{3.829840in}{3.219277in}}%
\pgfpathlineto{\pgfqpoint{3.829902in}{3.313350in}}%
\pgfpathlineto{\pgfqpoint{3.830210in}{3.053180in}}%
\pgfpathlineto{\pgfqpoint{3.830919in}{3.206041in}}%
\pgfpathlineto{\pgfqpoint{3.831197in}{2.954909in}}%
\pgfpathlineto{\pgfqpoint{3.831397in}{3.351911in}}%
\pgfpathlineto{\pgfqpoint{3.832014in}{3.171164in}}%
\pgfpathlineto{\pgfqpoint{3.832923in}{3.300676in}}%
\pgfpathlineto{\pgfqpoint{3.832692in}{3.015741in}}%
\pgfpathlineto{\pgfqpoint{3.833124in}{3.215262in}}%
\pgfpathlineto{\pgfqpoint{3.834187in}{3.014925in}}%
\pgfpathlineto{\pgfqpoint{3.833725in}{3.303783in}}%
\pgfpathlineto{\pgfqpoint{3.834310in}{3.027917in}}%
\pgfpathlineto{\pgfqpoint{3.834372in}{3.305952in}}%
\pgfpathlineto{\pgfqpoint{3.835066in}{2.997561in}}%
\pgfpathlineto{\pgfqpoint{3.835451in}{3.191099in}}%
\pgfpathlineto{\pgfqpoint{3.836099in}{3.286595in}}%
\pgfpathlineto{\pgfqpoint{3.835667in}{3.067969in}}%
\pgfpathlineto{\pgfqpoint{3.836268in}{3.146506in}}%
\pgfpathlineto{\pgfqpoint{3.836638in}{3.000266in}}%
\pgfpathlineto{\pgfqpoint{3.836854in}{3.357447in}}%
\pgfpathlineto{\pgfqpoint{3.837316in}{3.163943in}}%
\pgfpathlineto{\pgfqpoint{3.838349in}{3.319173in}}%
\pgfpathlineto{\pgfqpoint{3.838149in}{3.013674in}}%
\pgfpathlineto{\pgfqpoint{3.838442in}{3.206894in}}%
\pgfpathlineto{\pgfqpoint{3.839629in}{3.033802in}}%
\pgfpathlineto{\pgfqpoint{3.839182in}{3.307608in}}%
\pgfpathlineto{\pgfqpoint{3.839659in}{3.129321in}}%
\pgfpathlineto{\pgfqpoint{3.840816in}{3.292212in}}%
\pgfpathlineto{\pgfqpoint{3.840507in}{3.004874in}}%
\pgfpathlineto{\pgfqpoint{3.840831in}{3.280487in}}%
\pgfpathlineto{\pgfqpoint{3.841108in}{3.065168in}}%
\pgfpathlineto{\pgfqpoint{3.842018in}{3.117253in}}%
\pgfpathlineto{\pgfqpoint{3.842295in}{3.341099in}}%
\pgfpathlineto{\pgfqpoint{3.842080in}{3.023152in}}%
\pgfpathlineto{\pgfqpoint{3.843189in}{3.204716in}}%
\pgfpathlineto{\pgfqpoint{3.843590in}{3.029991in}}%
\pgfpathlineto{\pgfqpoint{3.843791in}{3.328915in}}%
\pgfpathlineto{\pgfqpoint{3.844330in}{3.109355in}}%
\pgfpathlineto{\pgfqpoint{3.844392in}{3.038159in}}%
\pgfpathlineto{\pgfqpoint{3.844608in}{3.293750in}}%
\pgfpathlineto{\pgfqpoint{3.845332in}{3.219961in}}%
\pgfpathlineto{\pgfqpoint{3.846257in}{3.303940in}}%
\pgfpathlineto{\pgfqpoint{3.845949in}{3.046064in}}%
\pgfpathlineto{\pgfqpoint{3.846396in}{3.174644in}}%
\pgfpathlineto{\pgfqpoint{3.847506in}{3.009916in}}%
\pgfpathlineto{\pgfqpoint{3.846904in}{3.293336in}}%
\pgfpathlineto{\pgfqpoint{3.847536in}{3.112327in}}%
\pgfpathlineto{\pgfqpoint{3.847721in}{3.326710in}}%
\pgfpathlineto{\pgfqpoint{3.848276in}{3.044574in}}%
\pgfpathlineto{\pgfqpoint{3.848646in}{3.165445in}}%
\pgfpathlineto{\pgfqpoint{3.849217in}{3.324046in}}%
\pgfpathlineto{\pgfqpoint{3.849818in}{3.051002in}}%
\pgfpathlineto{\pgfqpoint{3.850034in}{3.291347in}}%
\pgfpathlineto{\pgfqpoint{3.851020in}{3.177267in}}%
\pgfpathlineto{\pgfqpoint{3.851375in}{3.077546in}}%
\pgfpathlineto{\pgfqpoint{3.851575in}{3.299143in}}%
\pgfpathlineto{\pgfqpoint{3.852176in}{3.137152in}}%
\pgfpathlineto{\pgfqpoint{3.853132in}{3.329363in}}%
\pgfpathlineto{\pgfqpoint{3.852932in}{3.028893in}}%
\pgfpathlineto{\pgfqpoint{3.853286in}{3.147854in}}%
\pgfpathlineto{\pgfqpoint{3.853918in}{3.251254in}}%
\pgfpathlineto{\pgfqpoint{3.853702in}{3.048818in}}%
\pgfpathlineto{\pgfqpoint{3.854242in}{3.130365in}}%
\pgfpathlineto{\pgfqpoint{3.855244in}{3.026577in}}%
\pgfpathlineto{\pgfqpoint{3.854673in}{3.278776in}}%
\pgfpathlineto{\pgfqpoint{3.855305in}{3.176919in}}%
\pgfpathlineto{\pgfqpoint{3.855460in}{3.285939in}}%
\pgfpathlineto{\pgfqpoint{3.855737in}{3.092463in}}%
\pgfpathlineto{\pgfqpoint{3.856400in}{3.174602in}}%
\pgfpathlineto{\pgfqpoint{3.856801in}{3.024517in}}%
\pgfpathlineto{\pgfqpoint{3.857017in}{3.317305in}}%
\pgfpathlineto{\pgfqpoint{3.857494in}{3.178581in}}%
\pgfpathlineto{\pgfqpoint{3.858543in}{3.290945in}}%
\pgfpathlineto{\pgfqpoint{3.857587in}{3.055825in}}%
\pgfpathlineto{\pgfqpoint{3.858620in}{3.196094in}}%
\pgfpathlineto{\pgfqpoint{3.858681in}{3.229520in}}%
\pgfpathlineto{\pgfqpoint{3.858743in}{3.100687in}}%
\pgfpathlineto{\pgfqpoint{3.859067in}{3.114862in}}%
\pgfpathlineto{\pgfqpoint{3.859915in}{3.061228in}}%
\pgfpathlineto{\pgfqpoint{3.859329in}{3.298734in}}%
\pgfpathlineto{\pgfqpoint{3.860069in}{3.231285in}}%
\pgfpathlineto{\pgfqpoint{3.860099in}{3.298913in}}%
\pgfpathlineto{\pgfqpoint{3.860685in}{3.009721in}}%
\pgfpathlineto{\pgfqpoint{3.861132in}{3.156728in}}%
\pgfpathlineto{\pgfqpoint{3.862227in}{3.032779in}}%
\pgfpathlineto{\pgfqpoint{3.861626in}{3.281491in}}%
\pgfpathlineto{\pgfqpoint{3.862258in}{3.109369in}}%
\pgfpathlineto{\pgfqpoint{3.862412in}{3.355312in}}%
\pgfpathlineto{\pgfqpoint{3.862997in}{3.017297in}}%
\pgfpathlineto{\pgfqpoint{3.863383in}{3.224197in}}%
\pgfpathlineto{\pgfqpoint{3.863876in}{3.286351in}}%
\pgfpathlineto{\pgfqpoint{3.863475in}{3.066408in}}%
\pgfpathlineto{\pgfqpoint{3.864385in}{3.196305in}}%
\pgfpathlineto{\pgfqpoint{3.864554in}{3.063268in}}%
\pgfpathlineto{\pgfqpoint{3.864755in}{3.300800in}}%
\pgfpathlineto{\pgfqpoint{3.865479in}{3.209482in}}%
\pgfpathlineto{\pgfqpoint{3.866327in}{3.353984in}}%
\pgfpathlineto{\pgfqpoint{3.866111in}{3.036123in}}%
\pgfpathlineto{\pgfqpoint{3.866543in}{3.135699in}}%
\pgfpathlineto{\pgfqpoint{3.866897in}{3.044565in}}%
\pgfpathlineto{\pgfqpoint{3.867082in}{3.267397in}}%
\pgfpathlineto{\pgfqpoint{3.867684in}{3.098941in}}%
\pgfpathlineto{\pgfqpoint{3.867838in}{3.312062in}}%
\pgfpathlineto{\pgfqpoint{3.868439in}{3.020289in}}%
\pgfpathlineto{\pgfqpoint{3.868809in}{3.193347in}}%
\pgfpathlineto{\pgfqpoint{3.869179in}{3.097303in}}%
\pgfpathlineto{\pgfqpoint{3.869210in}{3.026632in}}%
\pgfpathlineto{\pgfqpoint{3.869271in}{3.311024in}}%
\pgfpathlineto{\pgfqpoint{3.870212in}{3.276285in}}%
\pgfpathlineto{\pgfqpoint{3.870227in}{3.278269in}}%
\pgfpathlineto{\pgfqpoint{3.870335in}{3.169683in}}%
\pgfpathlineto{\pgfqpoint{3.871399in}{3.061566in}}%
\pgfpathlineto{\pgfqpoint{3.870952in}{3.295852in}}%
\pgfpathlineto{\pgfqpoint{3.871445in}{3.161630in}}%
\pgfpathlineto{\pgfqpoint{3.871522in}{3.018386in}}%
\pgfpathlineto{\pgfqpoint{3.871691in}{3.279061in}}%
\pgfpathlineto{\pgfqpoint{3.871722in}{3.376228in}}%
\pgfpathlineto{\pgfqpoint{3.872308in}{3.023703in}}%
\pgfpathlineto{\pgfqpoint{3.872755in}{3.144367in}}%
\pgfpathlineto{\pgfqpoint{3.873865in}{2.996105in}}%
\pgfpathlineto{\pgfqpoint{3.873310in}{3.281708in}}%
\pgfpathlineto{\pgfqpoint{3.873880in}{3.065570in}}%
\pgfpathlineto{\pgfqpoint{3.874189in}{3.320926in}}%
\pgfpathlineto{\pgfqpoint{3.874620in}{3.039344in}}%
\pgfpathlineto{\pgfqpoint{3.875006in}{3.145603in}}%
\pgfpathlineto{\pgfqpoint{3.875638in}{3.337891in}}%
\pgfpathlineto{\pgfqpoint{3.875422in}{3.027514in}}%
\pgfpathlineto{\pgfqpoint{3.876023in}{3.118754in}}%
\pgfpathlineto{\pgfqpoint{3.876948in}{3.004834in}}%
\pgfpathlineto{\pgfqpoint{3.876378in}{3.301520in}}%
\pgfpathlineto{\pgfqpoint{3.876994in}{3.133577in}}%
\pgfpathlineto{\pgfqpoint{3.877133in}{3.397405in}}%
\pgfpathlineto{\pgfqpoint{3.877719in}{3.018720in}}%
\pgfpathlineto{\pgfqpoint{3.878119in}{3.257556in}}%
\pgfpathlineto{\pgfqpoint{3.878582in}{3.358471in}}%
\pgfpathlineto{\pgfqpoint{3.879276in}{2.999528in}}%
\pgfpathlineto{\pgfqpoint{3.879584in}{3.345970in}}%
\pgfpathlineto{\pgfqpoint{3.880416in}{3.165319in}}%
\pgfpathlineto{\pgfqpoint{3.880832in}{2.897881in}}%
\pgfpathlineto{\pgfqpoint{3.881048in}{3.451491in}}%
\pgfpathlineto{\pgfqpoint{3.881511in}{3.186591in}}%
\pgfpathlineto{\pgfqpoint{3.882528in}{3.381782in}}%
\pgfpathlineto{\pgfqpoint{3.882374in}{2.981545in}}%
\pgfpathlineto{\pgfqpoint{3.882621in}{3.217879in}}%
\pgfpathlineto{\pgfqpoint{3.882713in}{3.261669in}}%
\pgfpathlineto{\pgfqpoint{3.882806in}{3.031063in}}%
\pgfpathlineto{\pgfqpoint{3.883114in}{3.054234in}}%
\pgfpathlineto{\pgfqpoint{3.883792in}{2.971906in}}%
\pgfpathlineto{\pgfqpoint{3.883992in}{3.404026in}}%
\pgfpathlineto{\pgfqpoint{3.884100in}{3.318471in}}%
\pgfpathlineto{\pgfqpoint{3.884902in}{3.397916in}}%
\pgfpathlineto{\pgfqpoint{3.884702in}{2.979385in}}%
\pgfpathlineto{\pgfqpoint{3.885133in}{3.097570in}}%
\pgfpathlineto{\pgfqpoint{3.885673in}{3.391767in}}%
\pgfpathlineto{\pgfqpoint{3.886120in}{3.029766in}}%
\pgfpathlineto{\pgfqpoint{3.886181in}{3.099832in}}%
\pgfpathlineto{\pgfqpoint{3.886243in}{2.835106in}}%
\pgfpathlineto{\pgfqpoint{3.886443in}{3.489401in}}%
\pgfpathlineto{\pgfqpoint{3.887245in}{3.292349in}}%
\pgfpathlineto{\pgfqpoint{3.887785in}{2.940049in}}%
\pgfpathlineto{\pgfqpoint{3.887985in}{3.373305in}}%
\pgfpathlineto{\pgfqpoint{3.888370in}{3.248818in}}%
\pgfpathlineto{\pgfqpoint{3.888786in}{3.394871in}}%
\pgfpathlineto{\pgfqpoint{3.888555in}{2.903811in}}%
\pgfpathlineto{\pgfqpoint{3.889156in}{3.084155in}}%
\pgfpathlineto{\pgfqpoint{3.890128in}{2.878197in}}%
\pgfpathlineto{\pgfqpoint{3.889557in}{3.398198in}}%
\pgfpathlineto{\pgfqpoint{3.890205in}{3.231166in}}%
\pgfpathlineto{\pgfqpoint{3.890898in}{2.942013in}}%
\pgfpathlineto{\pgfqpoint{3.890328in}{3.480034in}}%
\pgfpathlineto{\pgfqpoint{3.891299in}{3.202192in}}%
\pgfpathlineto{\pgfqpoint{3.891869in}{3.453883in}}%
\pgfpathlineto{\pgfqpoint{3.891669in}{2.882187in}}%
\pgfpathlineto{\pgfqpoint{3.892363in}{3.140639in}}%
\pgfpathlineto{\pgfqpoint{3.892440in}{2.895641in}}%
\pgfpathlineto{\pgfqpoint{3.892656in}{3.387649in}}%
\pgfpathlineto{\pgfqpoint{3.893396in}{3.297805in}}%
\pgfpathlineto{\pgfqpoint{3.894197in}{3.425081in}}%
\pgfpathlineto{\pgfqpoint{3.893997in}{2.873888in}}%
\pgfpathlineto{\pgfqpoint{3.894459in}{3.137106in}}%
\pgfpathlineto{\pgfqpoint{3.895554in}{2.920403in}}%
\pgfpathlineto{\pgfqpoint{3.894968in}{3.352261in}}%
\pgfpathlineto{\pgfqpoint{3.895584in}{3.051720in}}%
\pgfpathlineto{\pgfqpoint{3.895754in}{3.455730in}}%
\pgfpathlineto{\pgfqpoint{3.896324in}{2.910819in}}%
\pgfpathlineto{\pgfqpoint{3.896710in}{3.221634in}}%
\pgfpathlineto{\pgfqpoint{3.896725in}{3.225228in}}%
\pgfpathlineto{\pgfqpoint{3.897049in}{3.052710in}}%
\pgfpathlineto{\pgfqpoint{3.897095in}{2.940609in}}%
\pgfpathlineto{\pgfqpoint{3.897295in}{3.351902in}}%
\pgfpathlineto{\pgfqpoint{3.898082in}{3.344743in}}%
\pgfpathlineto{\pgfqpoint{3.898637in}{2.965573in}}%
\pgfpathlineto{\pgfqpoint{3.899315in}{3.149152in}}%
\pgfpathlineto{\pgfqpoint{3.899392in}{2.998172in}}%
\pgfpathlineto{\pgfqpoint{3.899407in}{2.974470in}}%
\pgfpathlineto{\pgfqpoint{3.899639in}{3.358423in}}%
\pgfpathlineto{\pgfqpoint{3.900348in}{3.186247in}}%
\pgfpathlineto{\pgfqpoint{3.900409in}{3.339344in}}%
\pgfpathlineto{\pgfqpoint{3.900964in}{2.980949in}}%
\pgfpathlineto{\pgfqpoint{3.901427in}{3.147876in}}%
\pgfpathlineto{\pgfqpoint{3.902521in}{3.013863in}}%
\pgfpathlineto{\pgfqpoint{3.901951in}{3.294451in}}%
\pgfpathlineto{\pgfqpoint{3.902552in}{3.056897in}}%
\pgfpathlineto{\pgfqpoint{3.902721in}{3.337452in}}%
\pgfpathlineto{\pgfqpoint{3.903292in}{2.997426in}}%
\pgfpathlineto{\pgfqpoint{3.903708in}{3.207197in}}%
\pgfpathlineto{\pgfqpoint{3.904063in}{3.047331in}}%
\pgfpathlineto{\pgfqpoint{3.904294in}{3.288561in}}%
\pgfpathlineto{\pgfqpoint{3.904910in}{3.111285in}}%
\pgfpathlineto{\pgfqpoint{3.905065in}{3.317114in}}%
\pgfpathlineto{\pgfqpoint{3.905619in}{3.052490in}}%
\pgfpathlineto{\pgfqpoint{3.906051in}{3.186082in}}%
\pgfpathlineto{\pgfqpoint{3.906390in}{3.020487in}}%
\pgfpathlineto{\pgfqpoint{3.906621in}{3.308812in}}%
\pgfpathlineto{\pgfqpoint{3.907207in}{3.089494in}}%
\pgfpathlineto{\pgfqpoint{3.908148in}{3.289864in}}%
\pgfpathlineto{\pgfqpoint{3.907947in}{3.037832in}}%
\pgfpathlineto{\pgfqpoint{3.908348in}{3.151711in}}%
\pgfpathlineto{\pgfqpoint{3.908363in}{3.152348in}}%
\pgfpathlineto{\pgfqpoint{3.908394in}{3.122089in}}%
\pgfpathlineto{\pgfqpoint{3.909504in}{3.085672in}}%
\pgfpathlineto{\pgfqpoint{3.908949in}{3.255411in}}%
\pgfpathlineto{\pgfqpoint{3.909519in}{3.091332in}}%
\pgfpathlineto{\pgfqpoint{3.909704in}{3.297050in}}%
\pgfpathlineto{\pgfqpoint{3.910290in}{3.048944in}}%
\pgfpathlineto{\pgfqpoint{3.910691in}{3.212584in}}%
\pgfpathlineto{\pgfqpoint{3.911832in}{3.046604in}}%
\pgfpathlineto{\pgfqpoint{3.911246in}{3.224970in}}%
\pgfpathlineto{\pgfqpoint{3.911847in}{3.056390in}}%
\pgfpathlineto{\pgfqpoint{3.912032in}{3.304766in}}%
\pgfpathlineto{\pgfqpoint{3.913049in}{3.175450in}}%
\pgfpathlineto{\pgfqpoint{3.914175in}{3.081038in}}%
\pgfpathlineto{\pgfqpoint{3.913604in}{3.298575in}}%
\pgfpathlineto{\pgfqpoint{3.914236in}{3.105354in}}%
\pgfpathlineto{\pgfqpoint{3.914267in}{3.098220in}}%
\pgfpathlineto{\pgfqpoint{3.914375in}{3.240197in}}%
\pgfpathlineto{\pgfqpoint{3.915053in}{3.210800in}}%
\pgfpathlineto{\pgfqpoint{3.915932in}{3.272754in}}%
\pgfpathlineto{\pgfqpoint{3.915716in}{3.062252in}}%
\pgfpathlineto{\pgfqpoint{3.916117in}{3.174244in}}%
\pgfpathlineto{\pgfqpoint{3.917258in}{3.071239in}}%
\pgfpathlineto{\pgfqpoint{3.917026in}{3.228846in}}%
\pgfpathlineto{\pgfqpoint{3.917273in}{3.073872in}}%
\pgfpathlineto{\pgfqpoint{3.917489in}{3.273201in}}%
\pgfpathlineto{\pgfqpoint{3.918475in}{3.193319in}}%
\pgfpathlineto{\pgfqpoint{3.919246in}{3.118293in}}%
\pgfpathlineto{\pgfqpoint{3.919000in}{3.269788in}}%
\pgfpathlineto{\pgfqpoint{3.919447in}{3.210674in}}%
\pgfpathlineto{\pgfqpoint{3.919817in}{3.236872in}}%
\pgfpathlineto{\pgfqpoint{3.920094in}{3.109470in}}%
\pgfpathlineto{\pgfqpoint{3.920541in}{3.210250in}}%
\pgfpathlineto{\pgfqpoint{3.921620in}{3.089850in}}%
\pgfpathlineto{\pgfqpoint{3.921435in}{3.242036in}}%
\pgfpathlineto{\pgfqpoint{3.921774in}{3.174513in}}%
\pgfpathlineto{\pgfqpoint{3.922899in}{3.254808in}}%
\pgfpathlineto{\pgfqpoint{3.922560in}{3.119007in}}%
\pgfpathlineto{\pgfqpoint{3.922977in}{3.203262in}}%
\pgfpathlineto{\pgfqpoint{3.923162in}{3.105263in}}%
\pgfpathlineto{\pgfqpoint{3.923747in}{3.225894in}}%
\pgfpathlineto{\pgfqpoint{3.924148in}{3.143008in}}%
\pgfpathlineto{\pgfqpoint{3.925227in}{3.232421in}}%
\pgfpathlineto{\pgfqpoint{3.925027in}{3.119678in}}%
\pgfpathlineto{\pgfqpoint{3.925335in}{3.219730in}}%
\pgfpathlineto{\pgfqpoint{3.925505in}{3.103719in}}%
\pgfpathlineto{\pgfqpoint{3.926507in}{3.142882in}}%
\pgfpathlineto{\pgfqpoint{3.926784in}{3.241778in}}%
\pgfpathlineto{\pgfqpoint{3.927062in}{3.117816in}}%
\pgfpathlineto{\pgfqpoint{3.927786in}{3.182057in}}%
\pgfpathlineto{\pgfqpoint{3.927971in}{3.133738in}}%
\pgfpathlineto{\pgfqpoint{3.928326in}{3.229009in}}%
\pgfpathlineto{\pgfqpoint{3.928927in}{3.139281in}}%
\pgfpathlineto{\pgfqpoint{3.929389in}{3.115776in}}%
\pgfpathlineto{\pgfqpoint{3.929143in}{3.220045in}}%
\pgfpathlineto{\pgfqpoint{3.929775in}{3.176844in}}%
\pgfpathlineto{\pgfqpoint{3.930684in}{3.227494in}}%
\pgfpathlineto{\pgfqpoint{3.930376in}{3.135754in}}%
\pgfpathlineto{\pgfqpoint{3.930854in}{3.143298in}}%
\pgfpathlineto{\pgfqpoint{3.930915in}{3.121888in}}%
\pgfpathlineto{\pgfqpoint{3.931455in}{3.196701in}}%
\pgfpathlineto{\pgfqpoint{3.931933in}{3.150986in}}%
\pgfpathlineto{\pgfqpoint{3.932164in}{3.231031in}}%
\pgfpathlineto{\pgfqpoint{3.932827in}{3.120473in}}%
\pgfpathlineto{\pgfqpoint{3.933166in}{3.179857in}}%
\pgfpathlineto{\pgfqpoint{3.933320in}{3.116564in}}%
\pgfpathlineto{\pgfqpoint{3.933705in}{3.220614in}}%
\pgfpathlineto{\pgfqpoint{3.934384in}{3.150048in}}%
\pgfpathlineto{\pgfqpoint{3.934538in}{3.227414in}}%
\pgfpathlineto{\pgfqpoint{3.934800in}{3.101387in}}%
\pgfpathlineto{\pgfqpoint{3.935540in}{3.169341in}}%
\pgfpathlineto{\pgfqpoint{3.935771in}{3.117835in}}%
\pgfpathlineto{\pgfqpoint{3.936079in}{3.235183in}}%
\pgfpathlineto{\pgfqpoint{3.936603in}{3.183393in}}%
\pgfpathlineto{\pgfqpoint{3.937621in}{3.236224in}}%
\pgfpathlineto{\pgfqpoint{3.937343in}{3.131364in}}%
\pgfpathlineto{\pgfqpoint{3.937698in}{3.184154in}}%
\pgfpathlineto{\pgfqpoint{3.938700in}{3.116397in}}%
\pgfpathlineto{\pgfqpoint{3.938422in}{3.222783in}}%
\pgfpathlineto{\pgfqpoint{3.938869in}{3.122831in}}%
\pgfpathlineto{\pgfqpoint{3.939100in}{3.236673in}}%
\pgfpathlineto{\pgfqpoint{3.939671in}{3.119478in}}%
\pgfpathlineto{\pgfqpoint{3.940072in}{3.188458in}}%
\pgfpathlineto{\pgfqpoint{3.940287in}{3.123231in}}%
\pgfpathlineto{\pgfqpoint{3.940719in}{3.214621in}}%
\pgfpathlineto{\pgfqpoint{3.941228in}{3.134514in}}%
\pgfpathlineto{\pgfqpoint{3.941474in}{3.236583in}}%
\pgfpathlineto{\pgfqpoint{3.941783in}{3.113937in}}%
\pgfpathlineto{\pgfqpoint{3.942461in}{3.161869in}}%
\pgfpathlineto{\pgfqpoint{3.942492in}{3.163903in}}%
\pgfpathlineto{\pgfqpoint{3.942615in}{3.117625in}}%
\pgfpathlineto{\pgfqpoint{3.942723in}{3.126823in}}%
\pgfpathlineto{\pgfqpoint{3.942754in}{3.105351in}}%
\pgfpathlineto{\pgfqpoint{3.943016in}{3.243336in}}%
\pgfpathlineto{\pgfqpoint{3.943740in}{3.192125in}}%
\pgfpathlineto{\pgfqpoint{3.943833in}{3.251867in}}%
\pgfpathlineto{\pgfqpoint{3.944095in}{3.114204in}}%
\pgfpathlineto{\pgfqpoint{3.944742in}{3.144253in}}%
\pgfpathlineto{\pgfqpoint{3.945081in}{3.108303in}}%
\pgfpathlineto{\pgfqpoint{3.945313in}{3.249261in}}%
\pgfpathlineto{\pgfqpoint{3.945343in}{3.244916in}}%
\pgfpathlineto{\pgfqpoint{3.946654in}{3.109844in}}%
\pgfpathlineto{\pgfqpoint{3.946777in}{3.199756in}}%
\pgfpathlineto{\pgfqpoint{3.946931in}{3.247773in}}%
\pgfpathlineto{\pgfqpoint{3.947178in}{3.126125in}}%
\pgfpathlineto{\pgfqpoint{3.947841in}{3.175694in}}%
\pgfpathlineto{\pgfqpoint{3.948180in}{3.109506in}}%
\pgfpathlineto{\pgfqpoint{3.948411in}{3.238400in}}%
\pgfpathlineto{\pgfqpoint{3.948997in}{3.126993in}}%
\pgfpathlineto{\pgfqpoint{3.949228in}{3.250852in}}%
\pgfpathlineto{\pgfqpoint{3.949613in}{3.123964in}}%
\pgfpathlineto{\pgfqpoint{3.950184in}{3.171301in}}%
\pgfpathlineto{\pgfqpoint{3.950523in}{3.122324in}}%
\pgfpathlineto{\pgfqpoint{3.950662in}{3.190891in}}%
\pgfpathlineto{\pgfqpoint{3.950785in}{3.234458in}}%
\pgfpathlineto{\pgfqpoint{3.951294in}{3.128262in}}%
\pgfpathlineto{\pgfqpoint{3.951741in}{3.159726in}}%
\pgfpathlineto{\pgfqpoint{3.952064in}{3.104905in}}%
\pgfpathlineto{\pgfqpoint{3.952326in}{3.239291in}}%
\pgfpathlineto{\pgfqpoint{3.952897in}{3.146956in}}%
\pgfpathlineto{\pgfqpoint{3.953128in}{3.240631in}}%
\pgfpathlineto{\pgfqpoint{3.953513in}{3.126638in}}%
\pgfpathlineto{\pgfqpoint{3.954007in}{3.153653in}}%
\pgfpathlineto{\pgfqpoint{3.954407in}{3.120640in}}%
\pgfpathlineto{\pgfqpoint{3.954577in}{3.201635in}}%
\pgfpathlineto{\pgfqpoint{3.954639in}{3.241298in}}%
\pgfpathlineto{\pgfqpoint{3.955178in}{3.122580in}}%
\pgfpathlineto{\pgfqpoint{3.955641in}{3.157803in}}%
\pgfpathlineto{\pgfqpoint{3.955964in}{3.105753in}}%
\pgfpathlineto{\pgfqpoint{3.956226in}{3.233086in}}%
\pgfpathlineto{\pgfqpoint{3.956750in}{3.150224in}}%
\pgfpathlineto{\pgfqpoint{3.957752in}{3.232551in}}%
\pgfpathlineto{\pgfqpoint{3.957490in}{3.116567in}}%
\pgfpathlineto{\pgfqpoint{3.957907in}{3.164694in}}%
\pgfpathlineto{\pgfqpoint{3.958924in}{3.131342in}}%
\pgfpathlineto{\pgfqpoint{3.958539in}{3.232163in}}%
\pgfpathlineto{\pgfqpoint{3.959032in}{3.152758in}}%
\pgfpathlineto{\pgfqpoint{3.959833in}{3.125778in}}%
\pgfpathlineto{\pgfqpoint{3.959278in}{3.220410in}}%
\pgfpathlineto{\pgfqpoint{3.959988in}{3.185648in}}%
\pgfpathlineto{\pgfqpoint{3.960049in}{3.223011in}}%
\pgfpathlineto{\pgfqpoint{3.960604in}{3.131557in}}%
\pgfpathlineto{\pgfqpoint{3.961067in}{3.156357in}}%
\pgfpathlineto{\pgfqpoint{3.961375in}{3.125758in}}%
\pgfpathlineto{\pgfqpoint{3.961652in}{3.233078in}}%
\pgfpathlineto{\pgfqpoint{3.962161in}{3.158702in}}%
\pgfpathlineto{\pgfqpoint{3.963163in}{3.230572in}}%
\pgfpathlineto{\pgfqpoint{3.962916in}{3.134385in}}%
\pgfpathlineto{\pgfqpoint{3.963333in}{3.166152in}}%
\pgfpathlineto{\pgfqpoint{3.964242in}{3.128854in}}%
\pgfpathlineto{\pgfqpoint{3.963965in}{3.230937in}}%
\pgfpathlineto{\pgfqpoint{3.964442in}{3.166032in}}%
\pgfpathlineto{\pgfqpoint{3.965244in}{3.134221in}}%
\pgfpathlineto{\pgfqpoint{3.965444in}{3.210983in}}%
\pgfpathlineto{\pgfqpoint{3.965568in}{3.224171in}}%
\pgfpathlineto{\pgfqpoint{3.965784in}{3.129656in}}%
\pgfpathlineto{\pgfqpoint{3.966431in}{3.180906in}}%
\pgfpathlineto{\pgfqpoint{3.967325in}{3.124135in}}%
\pgfpathlineto{\pgfqpoint{3.967078in}{3.238184in}}%
\pgfpathlineto{\pgfqpoint{3.967525in}{3.184309in}}%
\pgfpathlineto{\pgfqpoint{3.968127in}{3.139807in}}%
\pgfpathlineto{\pgfqpoint{3.967865in}{3.213013in}}%
\pgfpathlineto{\pgfqpoint{3.968512in}{3.197187in}}%
\pgfpathlineto{\pgfqpoint{3.968620in}{3.220345in}}%
\pgfpathlineto{\pgfqpoint{3.968820in}{3.133466in}}%
\pgfpathlineto{\pgfqpoint{3.969560in}{3.167389in}}%
\pgfpathlineto{\pgfqpoint{3.970408in}{3.126264in}}%
\pgfpathlineto{\pgfqpoint{3.970192in}{3.206032in}}%
\pgfpathlineto{\pgfqpoint{3.970685in}{3.153890in}}%
\pgfpathlineto{\pgfqpoint{3.970701in}{3.152908in}}%
\pgfpathlineto{\pgfqpoint{3.970917in}{3.198666in}}%
\pgfpathlineto{\pgfqpoint{3.970963in}{3.230950in}}%
\pgfpathlineto{\pgfqpoint{3.971194in}{3.123673in}}%
\pgfpathlineto{\pgfqpoint{3.971965in}{3.149261in}}%
\pgfpathlineto{\pgfqpoint{3.972766in}{3.121288in}}%
\pgfpathlineto{\pgfqpoint{3.972458in}{3.214501in}}%
\pgfpathlineto{\pgfqpoint{3.973029in}{3.173086in}}%
\pgfpathlineto{\pgfqpoint{3.974030in}{3.213836in}}%
\pgfpathlineto{\pgfqpoint{3.973599in}{3.132859in}}%
\pgfpathlineto{\pgfqpoint{3.974169in}{3.181248in}}%
\pgfpathlineto{\pgfqpoint{3.975094in}{3.113230in}}%
\pgfpathlineto{\pgfqpoint{3.974878in}{3.217238in}}%
\pgfpathlineto{\pgfqpoint{3.975310in}{3.159906in}}%
\pgfpathlineto{\pgfqpoint{3.976358in}{3.221688in}}%
\pgfpathlineto{\pgfqpoint{3.975788in}{3.129465in}}%
\pgfpathlineto{\pgfqpoint{3.976512in}{3.171877in}}%
\pgfpathlineto{\pgfqpoint{3.976651in}{3.124050in}}%
\pgfpathlineto{\pgfqpoint{3.977160in}{3.204062in}}%
\pgfpathlineto{\pgfqpoint{3.977622in}{3.168725in}}%
\pgfpathlineto{\pgfqpoint{3.978192in}{3.124363in}}%
\pgfpathlineto{\pgfqpoint{3.977930in}{3.215704in}}%
\pgfpathlineto{\pgfqpoint{3.978593in}{3.181528in}}%
\pgfpathlineto{\pgfqpoint{3.978778in}{3.209184in}}%
\pgfpathlineto{\pgfqpoint{3.978979in}{3.125248in}}%
\pgfpathlineto{\pgfqpoint{3.979611in}{3.166564in}}%
\pgfpathlineto{\pgfqpoint{3.979719in}{3.128124in}}%
\pgfpathlineto{\pgfqpoint{3.980273in}{3.217115in}}%
\pgfpathlineto{\pgfqpoint{3.980705in}{3.163052in}}%
\pgfpathlineto{\pgfqpoint{3.981830in}{3.206887in}}%
\pgfpathlineto{\pgfqpoint{3.981291in}{3.133686in}}%
\pgfpathlineto{\pgfqpoint{3.981877in}{3.192856in}}%
\pgfpathlineto{\pgfqpoint{3.982108in}{3.132457in}}%
\pgfpathlineto{\pgfqpoint{3.982570in}{3.199341in}}%
\pgfpathlineto{\pgfqpoint{3.983033in}{3.166089in}}%
\pgfpathlineto{\pgfqpoint{3.983079in}{3.164491in}}%
\pgfpathlineto{\pgfqpoint{3.983141in}{3.178296in}}%
\pgfpathlineto{\pgfqpoint{3.984173in}{3.214863in}}%
\pgfpathlineto{\pgfqpoint{3.983603in}{3.127527in}}%
\pgfpathlineto{\pgfqpoint{3.984251in}{3.179928in}}%
\pgfpathlineto{\pgfqpoint{3.984389in}{3.132154in}}%
\pgfpathlineto{\pgfqpoint{3.984883in}{3.196721in}}%
\pgfpathlineto{\pgfqpoint{3.985391in}{3.166202in}}%
\pgfpathlineto{\pgfqpoint{3.985931in}{3.127179in}}%
\pgfpathlineto{\pgfqpoint{3.985653in}{3.212871in}}%
\pgfpathlineto{\pgfqpoint{3.986285in}{3.178180in}}%
\pgfpathlineto{\pgfqpoint{3.986470in}{3.203838in}}%
\pgfpathlineto{\pgfqpoint{3.986717in}{3.138844in}}%
\pgfpathlineto{\pgfqpoint{3.987349in}{3.175027in}}%
\pgfpathlineto{\pgfqpoint{3.988305in}{3.121116in}}%
\pgfpathlineto{\pgfqpoint{3.987996in}{3.211019in}}%
\pgfpathlineto{\pgfqpoint{3.988459in}{3.172168in}}%
\pgfpathlineto{\pgfqpoint{3.988598in}{3.214689in}}%
\pgfpathlineto{\pgfqpoint{3.989029in}{3.125018in}}%
\pgfpathlineto{\pgfqpoint{3.989615in}{3.194993in}}%
\pgfpathlineto{\pgfqpoint{3.989908in}{3.125837in}}%
\pgfpathlineto{\pgfqpoint{3.990401in}{3.208837in}}%
\pgfpathlineto{\pgfqpoint{3.990709in}{3.181094in}}%
\pgfpathlineto{\pgfqpoint{3.991696in}{3.239669in}}%
\pgfpathlineto{\pgfqpoint{3.991311in}{3.080562in}}%
\pgfpathlineto{\pgfqpoint{3.991804in}{3.164051in}}%
\pgfpathlineto{\pgfqpoint{3.992806in}{3.230097in}}%
\pgfpathlineto{\pgfqpoint{3.992590in}{3.106557in}}%
\pgfpathlineto{\pgfqpoint{3.993022in}{3.204519in}}%
\pgfpathlineto{\pgfqpoint{3.993576in}{3.066066in}}%
\pgfpathlineto{\pgfqpoint{3.994101in}{3.258822in}}%
\pgfpathlineto{\pgfqpoint{3.994856in}{3.270437in}}%
\pgfpathlineto{\pgfqpoint{3.994501in}{3.086022in}}%
\pgfpathlineto{\pgfqpoint{3.995072in}{3.184683in}}%
\pgfpathlineto{\pgfqpoint{3.996259in}{3.029404in}}%
\pgfpathlineto{\pgfqpoint{3.995719in}{3.263013in}}%
\pgfpathlineto{\pgfqpoint{3.996305in}{3.105696in}}%
\pgfpathlineto{\pgfqpoint{3.996613in}{3.290637in}}%
\pgfpathlineto{\pgfqpoint{3.996891in}{3.039152in}}%
\pgfpathlineto{\pgfqpoint{3.997430in}{3.164151in}}%
\pgfpathlineto{\pgfqpoint{3.998355in}{2.997562in}}%
\pgfpathlineto{\pgfqpoint{3.998124in}{3.272596in}}%
\pgfpathlineto{\pgfqpoint{3.998525in}{3.168541in}}%
\pgfpathlineto{\pgfqpoint{3.998848in}{3.324781in}}%
\pgfpathlineto{\pgfqpoint{3.998633in}{3.037293in}}%
\pgfpathlineto{\pgfqpoint{3.999480in}{3.104165in}}%
\pgfpathlineto{\pgfqpoint{3.999866in}{3.030185in}}%
\pgfpathlineto{\pgfqpoint{4.000390in}{3.263244in}}%
\pgfpathlineto{\pgfqpoint{4.000559in}{3.184784in}}%
\pgfpathlineto{\pgfqpoint{4.001346in}{3.060564in}}%
\pgfpathlineto{\pgfqpoint{4.000821in}{3.270049in}}%
\pgfpathlineto{\pgfqpoint{4.001669in}{3.179888in}}%
\pgfpathlineto{\pgfqpoint{4.001808in}{3.313459in}}%
\pgfpathlineto{\pgfqpoint{4.002224in}{2.886577in}}%
\pgfpathlineto{\pgfqpoint{4.002810in}{3.211831in}}%
\pgfpathlineto{\pgfqpoint{4.003735in}{3.030353in}}%
\pgfpathlineto{\pgfqpoint{4.003812in}{3.304806in}}%
\pgfpathlineto{\pgfqpoint{4.003920in}{3.161979in}}%
\pgfpathlineto{\pgfqpoint{4.004490in}{3.073213in}}%
\pgfpathlineto{\pgfqpoint{4.004059in}{3.274091in}}%
\pgfpathlineto{\pgfqpoint{4.004999in}{3.172596in}}%
\pgfpathlineto{\pgfqpoint{4.005307in}{3.354949in}}%
\pgfpathlineto{\pgfqpoint{4.005230in}{2.950596in}}%
\pgfpathlineto{\pgfqpoint{4.006078in}{3.139811in}}%
\pgfpathlineto{\pgfqpoint{4.006217in}{3.027912in}}%
\pgfpathlineto{\pgfqpoint{4.006294in}{3.330817in}}%
\pgfpathlineto{\pgfqpoint{4.007203in}{3.058217in}}%
\pgfpathlineto{\pgfqpoint{4.008282in}{3.354331in}}%
\pgfpathlineto{\pgfqpoint{4.007342in}{3.010049in}}%
\pgfpathlineto{\pgfqpoint{4.008421in}{3.131167in}}%
\pgfpathlineto{\pgfqpoint{4.008467in}{3.031486in}}%
\pgfpathlineto{\pgfqpoint{4.009269in}{3.295065in}}%
\pgfpathlineto{\pgfqpoint{4.009515in}{3.156672in}}%
\pgfpathlineto{\pgfqpoint{4.010302in}{3.073715in}}%
\pgfpathlineto{\pgfqpoint{4.009793in}{3.327444in}}%
\pgfpathlineto{\pgfqpoint{4.010502in}{3.137861in}}%
\pgfpathlineto{\pgfqpoint{4.010672in}{3.353295in}}%
\pgfpathlineto{\pgfqpoint{4.011473in}{3.045522in}}%
\pgfpathlineto{\pgfqpoint{4.011658in}{3.290900in}}%
\pgfpathlineto{\pgfqpoint{4.011689in}{3.309721in}}%
\pgfpathlineto{\pgfqpoint{4.012367in}{3.016961in}}%
\pgfpathlineto{\pgfqpoint{4.012629in}{3.218446in}}%
\pgfpathlineto{\pgfqpoint{4.013307in}{3.010907in}}%
\pgfpathlineto{\pgfqpoint{4.013508in}{3.307947in}}%
\pgfpathlineto{\pgfqpoint{4.013708in}{3.270166in}}%
\pgfpathlineto{\pgfqpoint{4.013739in}{3.317905in}}%
\pgfpathlineto{\pgfqpoint{4.014202in}{3.066462in}}%
\pgfpathlineto{\pgfqpoint{4.014741in}{3.139095in}}%
\pgfpathlineto{\pgfqpoint{4.014772in}{3.041384in}}%
\pgfpathlineto{\pgfqpoint{4.015573in}{3.319536in}}%
\pgfpathlineto{\pgfqpoint{4.015836in}{3.178136in}}%
\pgfpathlineto{\pgfqpoint{4.016483in}{3.298697in}}%
\pgfpathlineto{\pgfqpoint{4.016283in}{2.928774in}}%
\pgfpathlineto{\pgfqpoint{4.016915in}{3.133417in}}%
\pgfpathlineto{\pgfqpoint{4.017161in}{3.041581in}}%
\pgfpathlineto{\pgfqpoint{4.017547in}{3.270238in}}%
\pgfpathlineto{\pgfqpoint{4.017932in}{3.188734in}}%
\pgfpathlineto{\pgfqpoint{4.018425in}{3.282272in}}%
\pgfpathlineto{\pgfqpoint{4.018672in}{3.033957in}}%
\pgfpathlineto{\pgfqpoint{4.019042in}{3.195958in}}%
\pgfpathlineto{\pgfqpoint{4.019458in}{3.316166in}}%
\pgfpathlineto{\pgfqpoint{4.019242in}{2.963339in}}%
\pgfpathlineto{\pgfqpoint{4.020059in}{3.181576in}}%
\pgfpathlineto{\pgfqpoint{4.020737in}{3.022487in}}%
\pgfpathlineto{\pgfqpoint{4.020475in}{3.292768in}}%
\pgfpathlineto{\pgfqpoint{4.021169in}{3.119363in}}%
\pgfpathlineto{\pgfqpoint{4.021431in}{3.315429in}}%
\pgfpathlineto{\pgfqpoint{4.021631in}{3.010844in}}%
\pgfpathlineto{\pgfqpoint{4.022171in}{3.049549in}}%
\pgfpathlineto{\pgfqpoint{4.023127in}{3.034843in}}%
\pgfpathlineto{\pgfqpoint{4.022418in}{3.316661in}}%
\pgfpathlineto{\pgfqpoint{4.023173in}{3.121406in}}%
\pgfpathlineto{\pgfqpoint{4.024375in}{3.319535in}}%
\pgfpathlineto{\pgfqpoint{4.023682in}{3.044948in}}%
\pgfpathlineto{\pgfqpoint{4.024483in}{3.299310in}}%
\pgfpathlineto{\pgfqpoint{4.024576in}{3.039109in}}%
\pgfpathlineto{\pgfqpoint{4.025346in}{3.354043in}}%
\pgfpathlineto{\pgfqpoint{4.025609in}{3.197276in}}%
\pgfpathlineto{\pgfqpoint{4.026225in}{3.288039in}}%
\pgfpathlineto{\pgfqpoint{4.025809in}{3.054571in}}%
\pgfpathlineto{\pgfqpoint{4.026626in}{3.098095in}}%
\pgfpathlineto{\pgfqpoint{4.026657in}{3.029481in}}%
\pgfpathlineto{\pgfqpoint{4.026857in}{3.306028in}}%
\pgfpathlineto{\pgfqpoint{4.027705in}{3.144049in}}%
\pgfpathlineto{\pgfqpoint{4.028337in}{3.302397in}}%
\pgfpathlineto{\pgfqpoint{4.028722in}{3.041023in}}%
\pgfpathlineto{\pgfqpoint{4.028815in}{3.150014in}}%
\pgfpathlineto{\pgfqpoint{4.029185in}{3.368344in}}%
\pgfpathlineto{\pgfqpoint{4.029601in}{3.039025in}}%
\pgfpathlineto{\pgfqpoint{4.029863in}{3.126795in}}%
\pgfpathlineto{\pgfqpoint{4.030927in}{3.065307in}}%
\pgfpathlineto{\pgfqpoint{4.030094in}{3.279482in}}%
\pgfpathlineto{\pgfqpoint{4.030942in}{3.101848in}}%
\pgfpathlineto{\pgfqpoint{4.031389in}{3.029079in}}%
\pgfpathlineto{\pgfqpoint{4.032083in}{3.299229in}}%
\pgfpathlineto{\pgfqpoint{4.033008in}{3.331750in}}%
\pgfpathlineto{\pgfqpoint{4.032499in}{3.023689in}}%
\pgfpathlineto{\pgfqpoint{4.033116in}{3.216832in}}%
\pgfpathlineto{\pgfqpoint{4.033717in}{3.059138in}}%
\pgfpathlineto{\pgfqpoint{4.034071in}{3.351971in}}%
\pgfpathlineto{\pgfqpoint{4.034148in}{3.251659in}}%
\pgfpathlineto{\pgfqpoint{4.035042in}{3.371005in}}%
\pgfpathlineto{\pgfqpoint{4.034487in}{3.063411in}}%
\pgfpathlineto{\pgfqpoint{4.035212in}{3.079112in}}%
\pgfpathlineto{\pgfqpoint{4.035243in}{3.052562in}}%
\pgfpathlineto{\pgfqpoint{4.035921in}{3.307179in}}%
\pgfpathlineto{\pgfqpoint{4.036106in}{3.196469in}}%
\pgfpathlineto{\pgfqpoint{4.037015in}{3.357392in}}%
\pgfpathlineto{\pgfqpoint{4.036738in}{2.934781in}}%
\pgfpathlineto{\pgfqpoint{4.037216in}{3.197963in}}%
\pgfpathlineto{\pgfqpoint{4.038141in}{3.031378in}}%
\pgfpathlineto{\pgfqpoint{4.037925in}{3.289253in}}%
\pgfpathlineto{\pgfqpoint{4.038341in}{3.099771in}}%
\pgfpathlineto{\pgfqpoint{4.038495in}{3.337273in}}%
\pgfpathlineto{\pgfqpoint{4.038680in}{3.009852in}}%
\pgfpathlineto{\pgfqpoint{4.039497in}{3.213354in}}%
\pgfpathlineto{\pgfqpoint{4.040623in}{2.983212in}}%
\pgfpathlineto{\pgfqpoint{4.039929in}{3.403891in}}%
\pgfpathlineto{\pgfqpoint{4.040653in}{3.123637in}}%
\pgfpathlineto{\pgfqpoint{4.040900in}{3.328800in}}%
\pgfpathlineto{\pgfqpoint{4.041208in}{3.001094in}}%
\pgfpathlineto{\pgfqpoint{4.041779in}{3.268425in}}%
\pgfpathlineto{\pgfqpoint{4.041840in}{3.334273in}}%
\pgfpathlineto{\pgfqpoint{4.042565in}{3.034757in}}%
\pgfpathlineto{\pgfqpoint{4.042657in}{3.108300in}}%
\pgfpathlineto{\pgfqpoint{4.042688in}{3.013597in}}%
\pgfpathlineto{\pgfqpoint{4.042858in}{3.352870in}}%
\pgfpathlineto{\pgfqpoint{4.043736in}{3.205721in}}%
\pgfpathlineto{\pgfqpoint{4.044800in}{3.374112in}}%
\pgfpathlineto{\pgfqpoint{4.044168in}{3.036100in}}%
\pgfpathlineto{\pgfqpoint{4.044846in}{3.222595in}}%
\pgfpathlineto{\pgfqpoint{4.045494in}{3.038246in}}%
\pgfpathlineto{\pgfqpoint{4.045848in}{3.297951in}}%
\pgfpathlineto{\pgfqpoint{4.045971in}{3.148726in}}%
\pgfpathlineto{\pgfqpoint{4.046681in}{3.298602in}}%
\pgfpathlineto{\pgfqpoint{4.046927in}{3.047418in}}%
\pgfpathlineto{\pgfqpoint{4.047051in}{3.088444in}}%
\pgfpathlineto{\pgfqpoint{4.047081in}{3.045140in}}%
\pgfpathlineto{\pgfqpoint{4.047744in}{3.332901in}}%
\pgfpathlineto{\pgfqpoint{4.048114in}{3.153014in}}%
\pgfpathlineto{\pgfqpoint{4.048638in}{3.272805in}}%
\pgfpathlineto{\pgfqpoint{4.049101in}{3.051840in}}%
\pgfpathlineto{\pgfqpoint{4.049255in}{3.234701in}}%
\pgfpathlineto{\pgfqpoint{4.050026in}{3.051256in}}%
\pgfpathlineto{\pgfqpoint{4.049609in}{3.324695in}}%
\pgfpathlineto{\pgfqpoint{4.050380in}{3.175574in}}%
\pgfpathlineto{\pgfqpoint{4.051398in}{3.297693in}}%
\pgfpathlineto{\pgfqpoint{4.051320in}{3.015138in}}%
\pgfpathlineto{\pgfqpoint{4.051505in}{3.273189in}}%
\pgfpathlineto{\pgfqpoint{4.051629in}{3.338689in}}%
\pgfpathlineto{\pgfqpoint{4.051752in}{3.001340in}}%
\pgfpathlineto{\pgfqpoint{4.052584in}{3.254888in}}%
\pgfpathlineto{\pgfqpoint{4.053263in}{3.011602in}}%
\pgfpathlineto{\pgfqpoint{4.053340in}{3.319285in}}%
\pgfpathlineto{\pgfqpoint{4.053710in}{3.206071in}}%
\pgfpathlineto{\pgfqpoint{4.054234in}{3.004961in}}%
\pgfpathlineto{\pgfqpoint{4.054558in}{3.344533in}}%
\pgfpathlineto{\pgfqpoint{4.054573in}{3.364454in}}%
\pgfpathlineto{\pgfqpoint{4.054727in}{3.036439in}}%
\pgfpathlineto{\pgfqpoint{4.055513in}{3.182322in}}%
\pgfpathlineto{\pgfqpoint{4.056161in}{2.976172in}}%
\pgfpathlineto{\pgfqpoint{4.056346in}{3.333510in}}%
\pgfpathlineto{\pgfqpoint{4.056608in}{3.173752in}}%
\pgfpathlineto{\pgfqpoint{4.057394in}{3.331339in}}%
\pgfpathlineto{\pgfqpoint{4.057132in}{3.017563in}}%
\pgfpathlineto{\pgfqpoint{4.057687in}{3.122852in}}%
\pgfpathlineto{\pgfqpoint{4.057887in}{3.040370in}}%
\pgfpathlineto{\pgfqpoint{4.058380in}{3.354088in}}%
\pgfpathlineto{\pgfqpoint{4.058781in}{3.136710in}}%
\pgfpathlineto{\pgfqpoint{4.059274in}{3.303736in}}%
\pgfpathlineto{\pgfqpoint{4.059074in}{3.015316in}}%
\pgfpathlineto{\pgfqpoint{4.059891in}{3.147365in}}%
\pgfpathlineto{\pgfqpoint{4.060045in}{3.059531in}}%
\pgfpathlineto{\pgfqpoint{4.060261in}{3.274282in}}%
\pgfpathlineto{\pgfqpoint{4.060307in}{3.383498in}}%
\pgfpathlineto{\pgfqpoint{4.060677in}{3.030458in}}%
\pgfpathlineto{\pgfqpoint{4.061340in}{3.174648in}}%
\pgfpathlineto{\pgfqpoint{4.061972in}{2.990477in}}%
\pgfpathlineto{\pgfqpoint{4.062157in}{3.297651in}}%
\pgfpathlineto{\pgfqpoint{4.062435in}{3.188942in}}%
\pgfpathlineto{\pgfqpoint{4.063221in}{3.316621in}}%
\pgfpathlineto{\pgfqpoint{4.062527in}{3.022680in}}%
\pgfpathlineto{\pgfqpoint{4.063452in}{3.159642in}}%
\pgfpathlineto{\pgfqpoint{4.063714in}{3.007630in}}%
\pgfpathlineto{\pgfqpoint{4.064192in}{3.351184in}}%
\pgfpathlineto{\pgfqpoint{4.064577in}{3.066515in}}%
\pgfpathlineto{\pgfqpoint{4.065086in}{3.319670in}}%
\pgfpathlineto{\pgfqpoint{4.064885in}{2.961734in}}%
\pgfpathlineto{\pgfqpoint{4.065718in}{3.151585in}}%
\pgfpathlineto{\pgfqpoint{4.066720in}{3.033804in}}%
\pgfpathlineto{\pgfqpoint{4.066134in}{3.302900in}}%
\pgfpathlineto{\pgfqpoint{4.066843in}{3.096320in}}%
\pgfpathlineto{\pgfqpoint{4.067799in}{2.957663in}}%
\pgfpathlineto{\pgfqpoint{4.067999in}{3.370098in}}%
\pgfpathlineto{\pgfqpoint{4.068092in}{3.298136in}}%
\pgfpathlineto{\pgfqpoint{4.068338in}{3.015930in}}%
\pgfpathlineto{\pgfqpoint{4.069032in}{3.306101in}}%
\pgfpathlineto{\pgfqpoint{4.069325in}{3.051159in}}%
\pgfpathlineto{\pgfqpoint{4.070034in}{3.337025in}}%
\pgfpathlineto{\pgfqpoint{4.070450in}{3.140189in}}%
\pgfpathlineto{\pgfqpoint{4.070959in}{3.416410in}}%
\pgfpathlineto{\pgfqpoint{4.070712in}{3.000497in}}%
\pgfpathlineto{\pgfqpoint{4.071545in}{3.185385in}}%
\pgfpathlineto{\pgfqpoint{4.072655in}{3.040535in}}%
\pgfpathlineto{\pgfqpoint{4.071961in}{3.326974in}}%
\pgfpathlineto{\pgfqpoint{4.072670in}{3.064380in}}%
\pgfpathlineto{\pgfqpoint{4.073641in}{3.010034in}}%
\pgfpathlineto{\pgfqpoint{4.073857in}{3.380327in}}%
\pgfpathlineto{\pgfqpoint{4.074242in}{3.026834in}}%
\pgfpathlineto{\pgfqpoint{4.075475in}{3.150046in}}%
\pgfpathlineto{\pgfqpoint{4.075861in}{3.332493in}}%
\pgfpathlineto{\pgfqpoint{4.075707in}{3.075837in}}%
\pgfpathlineto{\pgfqpoint{4.076508in}{3.115023in}}%
\pgfpathlineto{\pgfqpoint{4.076555in}{3.018011in}}%
\pgfpathlineto{\pgfqpoint{4.076786in}{3.353012in}}%
\pgfpathlineto{\pgfqpoint{4.077587in}{3.190583in}}%
\pgfpathlineto{\pgfqpoint{4.077788in}{3.309197in}}%
\pgfpathlineto{\pgfqpoint{4.078065in}{3.072229in}}%
\pgfpathlineto{\pgfqpoint{4.078574in}{3.109527in}}%
\pgfpathlineto{\pgfqpoint{4.078620in}{3.045765in}}%
\pgfpathlineto{\pgfqpoint{4.078774in}{3.277675in}}%
\pgfpathlineto{\pgfqpoint{4.079622in}{3.210488in}}%
\pgfpathlineto{\pgfqpoint{4.079699in}{3.367602in}}%
\pgfpathlineto{\pgfqpoint{4.080085in}{3.056364in}}%
\pgfpathlineto{\pgfqpoint{4.080747in}{3.234290in}}%
\pgfpathlineto{\pgfqpoint{4.080979in}{3.085848in}}%
\pgfpathlineto{\pgfqpoint{4.081672in}{3.266263in}}%
\pgfpathlineto{\pgfqpoint{4.081965in}{3.164072in}}%
\pgfpathlineto{\pgfqpoint{4.082613in}{3.382893in}}%
\pgfpathlineto{\pgfqpoint{4.082381in}{3.039916in}}%
\pgfpathlineto{\pgfqpoint{4.083044in}{3.119280in}}%
\pgfpathlineto{\pgfqpoint{4.083615in}{3.307922in}}%
\pgfpathlineto{\pgfqpoint{4.083229in}{3.078464in}}%
\pgfpathlineto{\pgfqpoint{4.084123in}{3.121841in}}%
\pgfpathlineto{\pgfqpoint{4.084801in}{3.088165in}}%
\pgfpathlineto{\pgfqpoint{4.084586in}{3.298463in}}%
\pgfpathlineto{\pgfqpoint{4.085218in}{3.134243in}}%
\pgfpathlineto{\pgfqpoint{4.085295in}{3.033231in}}%
\pgfpathlineto{\pgfqpoint{4.085511in}{3.347932in}}%
\pgfpathlineto{\pgfqpoint{4.086297in}{3.109367in}}%
\pgfpathlineto{\pgfqpoint{4.086790in}{3.064935in}}%
\pgfpathlineto{\pgfqpoint{4.087484in}{3.320317in}}%
\pgfpathlineto{\pgfqpoint{4.088424in}{3.339440in}}%
\pgfpathlineto{\pgfqpoint{4.088224in}{3.036761in}}%
\pgfpathlineto{\pgfqpoint{4.088501in}{3.279224in}}%
\pgfpathlineto{\pgfqpoint{4.089102in}{3.067701in}}%
\pgfpathlineto{\pgfqpoint{4.089426in}{3.305456in}}%
\pgfpathlineto{\pgfqpoint{4.089657in}{3.130168in}}%
\pgfpathlineto{\pgfqpoint{4.089703in}{3.073715in}}%
\pgfpathlineto{\pgfqpoint{4.090397in}{3.338818in}}%
\pgfpathlineto{\pgfqpoint{4.090767in}{3.093859in}}%
\pgfpathlineto{\pgfqpoint{4.091384in}{3.331836in}}%
\pgfpathlineto{\pgfqpoint{4.091707in}{3.061386in}}%
\pgfpathlineto{\pgfqpoint{4.091923in}{3.173455in}}%
\pgfpathlineto{\pgfqpoint{4.092016in}{3.079125in}}%
\pgfpathlineto{\pgfqpoint{4.092324in}{3.268661in}}%
\pgfpathlineto{\pgfqpoint{4.093310in}{3.323747in}}%
\pgfpathlineto{\pgfqpoint{4.092678in}{3.073015in}}%
\pgfpathlineto{\pgfqpoint{4.093403in}{3.214924in}}%
\pgfpathlineto{\pgfqpoint{4.093465in}{3.235278in}}%
\pgfpathlineto{\pgfqpoint{4.093495in}{3.202272in}}%
\pgfpathlineto{\pgfqpoint{4.093757in}{3.070914in}}%
\pgfpathlineto{\pgfqpoint{4.094297in}{3.301580in}}%
\pgfpathlineto{\pgfqpoint{4.094636in}{3.087847in}}%
\pgfpathlineto{\pgfqpoint{4.095314in}{3.300429in}}%
\pgfpathlineto{\pgfqpoint{4.095669in}{3.077285in}}%
\pgfpathlineto{\pgfqpoint{4.096671in}{3.031277in}}%
\pgfpathlineto{\pgfqpoint{4.096224in}{3.302660in}}%
\pgfpathlineto{\pgfqpoint{4.096748in}{3.103856in}}%
\pgfpathlineto{\pgfqpoint{4.097380in}{3.293009in}}%
\pgfpathlineto{\pgfqpoint{4.097842in}{3.078423in}}%
\pgfpathlineto{\pgfqpoint{4.098243in}{3.293984in}}%
\pgfpathlineto{\pgfqpoint{4.098613in}{3.070247in}}%
\pgfpathlineto{\pgfqpoint{4.098999in}{3.140842in}}%
\pgfpathlineto{\pgfqpoint{4.099584in}{3.055594in}}%
\pgfpathlineto{\pgfqpoint{4.099137in}{3.275617in}}%
\pgfpathlineto{\pgfqpoint{4.099939in}{3.182131in}}%
\pgfpathlineto{\pgfqpoint{4.100000in}{3.267613in}}%
\pgfpathlineto{\pgfqpoint{4.100756in}{3.071342in}}%
\pgfpathlineto{\pgfqpoint{4.101018in}{3.140464in}}%
\pgfpathlineto{\pgfqpoint{4.101157in}{3.282237in}}%
\pgfpathlineto{\pgfqpoint{4.101511in}{3.087140in}}%
\pgfpathlineto{\pgfqpoint{4.102220in}{3.242224in}}%
\pgfpathlineto{\pgfqpoint{4.102683in}{3.040772in}}%
\pgfpathlineto{\pgfqpoint{4.103037in}{3.290631in}}%
\pgfpathlineto{\pgfqpoint{4.103376in}{3.152807in}}%
\pgfpathlineto{\pgfqpoint{4.103669in}{3.071400in}}%
\pgfpathlineto{\pgfqpoint{4.104070in}{3.292843in}}%
\pgfpathlineto{\pgfqpoint{4.104471in}{3.148873in}}%
\pgfpathlineto{\pgfqpoint{4.105242in}{3.249084in}}%
\pgfpathlineto{\pgfqpoint{4.105411in}{3.091217in}}%
\pgfpathlineto{\pgfqpoint{4.105550in}{3.158307in}}%
\pgfpathlineto{\pgfqpoint{4.105596in}{3.038854in}}%
\pgfpathlineto{\pgfqpoint{4.105951in}{3.274899in}}%
\pgfpathlineto{\pgfqpoint{4.106660in}{3.161412in}}%
\pgfpathlineto{\pgfqpoint{4.106983in}{3.298564in}}%
\pgfpathlineto{\pgfqpoint{4.106737in}{3.090145in}}%
\pgfpathlineto{\pgfqpoint{4.107754in}{3.152306in}}%
\pgfpathlineto{\pgfqpoint{4.107800in}{3.122242in}}%
\pgfpathlineto{\pgfqpoint{4.107831in}{3.179332in}}%
\pgfpathlineto{\pgfqpoint{4.108155in}{3.274787in}}%
\pgfpathlineto{\pgfqpoint{4.108494in}{3.073275in}}%
\pgfpathlineto{\pgfqpoint{4.108941in}{3.199792in}}%
\pgfpathlineto{\pgfqpoint{4.109650in}{3.092286in}}%
\pgfpathlineto{\pgfqpoint{4.109018in}{3.252003in}}%
\pgfpathlineto{\pgfqpoint{4.109851in}{3.229673in}}%
\pgfpathlineto{\pgfqpoint{4.109897in}{3.272816in}}%
\pgfpathlineto{\pgfqpoint{4.110236in}{3.107495in}}%
\pgfpathlineto{\pgfqpoint{4.110945in}{3.212527in}}%
\pgfpathlineto{\pgfqpoint{4.111407in}{3.072648in}}%
\pgfpathlineto{\pgfqpoint{4.111053in}{3.266437in}}%
\pgfpathlineto{\pgfqpoint{4.112070in}{3.179624in}}%
\pgfpathlineto{\pgfqpoint{4.112548in}{3.096013in}}%
\pgfpathlineto{\pgfqpoint{4.112826in}{3.251136in}}%
\pgfpathlineto{\pgfqpoint{4.113226in}{3.150749in}}%
\pgfpathlineto{\pgfqpoint{4.113951in}{3.292160in}}%
\pgfpathlineto{\pgfqpoint{4.114290in}{3.105137in}}%
\pgfpathlineto{\pgfqpoint{4.114460in}{3.084262in}}%
\pgfpathlineto{\pgfqpoint{4.114552in}{3.230417in}}%
\pgfpathlineto{\pgfqpoint{4.115354in}{3.142312in}}%
\pgfpathlineto{\pgfqpoint{4.115893in}{3.221936in}}%
\pgfpathlineto{\pgfqpoint{4.115462in}{3.102518in}}%
\pgfpathlineto{\pgfqpoint{4.116464in}{3.150117in}}%
\pgfpathlineto{\pgfqpoint{4.116494in}{3.148216in}}%
\pgfpathlineto{\pgfqpoint{4.116525in}{3.171008in}}%
\pgfpathlineto{\pgfqpoint{4.116864in}{3.282984in}}%
\pgfpathlineto{\pgfqpoint{4.117358in}{3.086105in}}%
\pgfpathlineto{\pgfqpoint{4.117635in}{3.199828in}}%
\pgfpathlineto{\pgfqpoint{4.118360in}{3.087585in}}%
\pgfpathlineto{\pgfqpoint{4.118606in}{3.236405in}}%
\pgfpathlineto{\pgfqpoint{4.118760in}{3.161193in}}%
\pgfpathlineto{\pgfqpoint{4.119778in}{3.286199in}}%
\pgfpathlineto{\pgfqpoint{4.119331in}{3.103311in}}%
\pgfpathlineto{\pgfqpoint{4.119870in}{3.169391in}}%
\pgfpathlineto{\pgfqpoint{4.120271in}{3.105261in}}%
\pgfpathlineto{\pgfqpoint{4.120626in}{3.224216in}}%
\pgfpathlineto{\pgfqpoint{4.120918in}{3.204228in}}%
\pgfpathlineto{\pgfqpoint{4.120934in}{3.204381in}}%
\pgfpathlineto{\pgfqpoint{4.120949in}{3.194710in}}%
\pgfpathlineto{\pgfqpoint{4.121273in}{3.093973in}}%
\pgfpathlineto{\pgfqpoint{4.121504in}{3.242416in}}%
\pgfpathlineto{\pgfqpoint{4.122075in}{3.182610in}}%
\pgfpathlineto{\pgfqpoint{4.122676in}{3.255274in}}%
\pgfpathlineto{\pgfqpoint{4.122244in}{3.084829in}}%
\pgfpathlineto{\pgfqpoint{4.122799in}{3.154103in}}%
\pgfpathlineto{\pgfqpoint{4.123169in}{3.110054in}}%
\pgfpathlineto{\pgfqpoint{4.123662in}{3.225991in}}%
\pgfpathlineto{\pgfqpoint{4.123893in}{3.152773in}}%
\pgfpathlineto{\pgfqpoint{4.124418in}{3.254523in}}%
\pgfpathlineto{\pgfqpoint{4.124186in}{3.070288in}}%
\pgfpathlineto{\pgfqpoint{4.125034in}{3.177701in}}%
\pgfpathlineto{\pgfqpoint{4.125157in}{3.082899in}}%
\pgfpathlineto{\pgfqpoint{4.125605in}{3.249332in}}%
\pgfpathlineto{\pgfqpoint{4.126144in}{3.155542in}}%
\pgfpathlineto{\pgfqpoint{4.126606in}{3.231814in}}%
\pgfpathlineto{\pgfqpoint{4.127084in}{3.099687in}}%
\pgfpathlineto{\pgfqpoint{4.127346in}{3.209069in}}%
\pgfpathlineto{\pgfqpoint{4.128071in}{3.087905in}}%
\pgfpathlineto{\pgfqpoint{4.127593in}{3.227904in}}%
\pgfpathlineto{\pgfqpoint{4.128441in}{3.204147in}}%
\pgfpathlineto{\pgfqpoint{4.129504in}{3.241392in}}%
\pgfpathlineto{\pgfqpoint{4.129119in}{3.119538in}}%
\pgfpathlineto{\pgfqpoint{4.129535in}{3.209873in}}%
\pgfpathlineto{\pgfqpoint{4.130568in}{3.111411in}}%
\pgfpathlineto{\pgfqpoint{4.130214in}{3.241528in}}%
\pgfpathlineto{\pgfqpoint{4.130661in}{3.174544in}}%
\pgfpathlineto{\pgfqpoint{4.131031in}{3.086377in}}%
\pgfpathlineto{\pgfqpoint{4.130799in}{3.218973in}}%
\pgfpathlineto{\pgfqpoint{4.131339in}{3.191504in}}%
\pgfpathlineto{\pgfqpoint{4.132418in}{3.246041in}}%
\pgfpathlineto{\pgfqpoint{4.131724in}{3.117303in}}%
\pgfpathlineto{\pgfqpoint{4.132449in}{3.200924in}}%
\pgfpathlineto{\pgfqpoint{4.133466in}{3.115224in}}%
\pgfpathlineto{\pgfqpoint{4.133127in}{3.251186in}}%
\pgfpathlineto{\pgfqpoint{4.133574in}{3.172613in}}%
\pgfpathlineto{\pgfqpoint{4.133929in}{3.090372in}}%
\pgfpathlineto{\pgfqpoint{4.134283in}{3.217097in}}%
\pgfpathlineto{\pgfqpoint{4.134684in}{3.159095in}}%
\pgfpathlineto{\pgfqpoint{4.135316in}{3.243728in}}%
\pgfpathlineto{\pgfqpoint{4.135655in}{3.112272in}}%
\pgfpathlineto{\pgfqpoint{4.135670in}{3.101511in}}%
\pgfpathlineto{\pgfqpoint{4.136025in}{3.243053in}}%
\pgfpathlineto{\pgfqpoint{4.136688in}{3.153338in}}%
\pgfpathlineto{\pgfqpoint{4.137751in}{3.219617in}}%
\pgfpathlineto{\pgfqpoint{4.136827in}{3.104370in}}%
\pgfpathlineto{\pgfqpoint{4.137798in}{3.167248in}}%
\pgfpathlineto{\pgfqpoint{4.138568in}{3.114207in}}%
\pgfpathlineto{\pgfqpoint{4.138214in}{3.223573in}}%
\pgfpathlineto{\pgfqpoint{4.138877in}{3.170201in}}%
\pgfpathlineto{\pgfqpoint{4.138923in}{3.234572in}}%
\pgfpathlineto{\pgfqpoint{4.139832in}{3.099443in}}%
\pgfpathlineto{\pgfqpoint{4.139971in}{3.184869in}}%
\pgfpathlineto{\pgfqpoint{4.140418in}{3.096344in}}%
\pgfpathlineto{\pgfqpoint{4.140649in}{3.230647in}}%
\pgfpathlineto{\pgfqpoint{4.141066in}{3.168212in}}%
\pgfpathlineto{\pgfqpoint{4.142098in}{3.237785in}}%
\pgfpathlineto{\pgfqpoint{4.141744in}{3.110634in}}%
\pgfpathlineto{\pgfqpoint{4.142160in}{3.152261in}}%
\pgfpathlineto{\pgfqpoint{4.142838in}{3.216805in}}%
\pgfpathlineto{\pgfqpoint{4.142730in}{3.091480in}}%
\pgfpathlineto{\pgfqpoint{4.143285in}{3.156246in}}%
\pgfpathlineto{\pgfqpoint{4.143717in}{3.114900in}}%
\pgfpathlineto{\pgfqpoint{4.144118in}{3.234523in}}%
\pgfpathlineto{\pgfqpoint{4.144380in}{3.151896in}}%
\pgfpathlineto{\pgfqpoint{4.144996in}{3.236660in}}%
\pgfpathlineto{\pgfqpoint{4.144472in}{3.091718in}}%
\pgfpathlineto{\pgfqpoint{4.145505in}{3.169589in}}%
\pgfpathlineto{\pgfqpoint{4.145628in}{3.089904in}}%
\pgfpathlineto{\pgfqpoint{4.146461in}{3.232760in}}%
\pgfpathlineto{\pgfqpoint{4.146661in}{3.131100in}}%
\pgfpathlineto{\pgfqpoint{4.146677in}{3.129885in}}%
\pgfpathlineto{\pgfqpoint{4.146723in}{3.203663in}}%
\pgfpathlineto{\pgfqpoint{4.147031in}{3.247207in}}%
\pgfpathlineto{\pgfqpoint{4.147540in}{3.085655in}}%
\pgfpathlineto{\pgfqpoint{4.147802in}{3.180972in}}%
\pgfpathlineto{\pgfqpoint{4.148526in}{3.085170in}}%
\pgfpathlineto{\pgfqpoint{4.148773in}{3.230559in}}%
\pgfpathlineto{\pgfqpoint{4.148881in}{3.190478in}}%
\pgfpathlineto{\pgfqpoint{4.149636in}{3.219716in}}%
\pgfpathlineto{\pgfqpoint{4.149128in}{3.110119in}}%
\pgfpathlineto{\pgfqpoint{4.149837in}{3.158559in}}%
\pgfpathlineto{\pgfqpoint{4.150438in}{3.074468in}}%
\pgfpathlineto{\pgfqpoint{4.149929in}{3.241584in}}%
\pgfpathlineto{\pgfqpoint{4.150916in}{3.171298in}}%
\pgfpathlineto{\pgfqpoint{4.151964in}{3.231598in}}%
\pgfpathlineto{\pgfqpoint{4.151440in}{3.083316in}}%
\pgfpathlineto{\pgfqpoint{4.151995in}{3.163049in}}%
\pgfpathlineto{\pgfqpoint{4.152180in}{3.115707in}}%
\pgfpathlineto{\pgfqpoint{4.152843in}{3.238237in}}%
\pgfpathlineto{\pgfqpoint{4.153089in}{3.169217in}}%
\pgfpathlineto{\pgfqpoint{4.153983in}{3.225395in}}%
\pgfpathlineto{\pgfqpoint{4.153351in}{3.086148in}}%
\pgfpathlineto{\pgfqpoint{4.154184in}{3.177497in}}%
\pgfpathlineto{\pgfqpoint{4.154338in}{3.085737in}}%
\pgfpathlineto{\pgfqpoint{4.154862in}{3.232532in}}%
\pgfpathlineto{\pgfqpoint{4.155309in}{3.140877in}}%
\pgfpathlineto{\pgfqpoint{4.155741in}{3.238632in}}%
\pgfpathlineto{\pgfqpoint{4.156249in}{3.094130in}}%
\pgfpathlineto{\pgfqpoint{4.156434in}{3.171726in}}%
\pgfpathlineto{\pgfqpoint{4.156897in}{3.230609in}}%
\pgfpathlineto{\pgfqpoint{4.157220in}{3.113478in}}%
\pgfpathlineto{\pgfqpoint{4.157251in}{3.093337in}}%
\pgfpathlineto{\pgfqpoint{4.157467in}{3.231402in}}%
\pgfpathlineto{\pgfqpoint{4.158299in}{3.135742in}}%
\pgfpathlineto{\pgfqpoint{4.158638in}{3.250607in}}%
\pgfpathlineto{\pgfqpoint{4.159147in}{3.113118in}}%
\pgfpathlineto{\pgfqpoint{4.159425in}{3.189201in}}%
\pgfpathlineto{\pgfqpoint{4.160180in}{3.093553in}}%
\pgfpathlineto{\pgfqpoint{4.160380in}{3.243492in}}%
\pgfpathlineto{\pgfqpoint{4.160535in}{3.175703in}}%
\pgfpathlineto{\pgfqpoint{4.161536in}{3.248514in}}%
\pgfpathlineto{\pgfqpoint{4.161182in}{3.115855in}}%
\pgfpathlineto{\pgfqpoint{4.161644in}{3.185560in}}%
\pgfpathlineto{\pgfqpoint{4.162045in}{3.128828in}}%
\pgfpathlineto{\pgfqpoint{4.162400in}{3.217147in}}%
\pgfpathlineto{\pgfqpoint{4.162662in}{3.195790in}}%
\pgfpathlineto{\pgfqpoint{4.163294in}{3.241777in}}%
\pgfpathlineto{\pgfqpoint{4.163093in}{3.096560in}}%
\pgfpathlineto{\pgfqpoint{4.163741in}{3.176829in}}%
\pgfpathlineto{\pgfqpoint{4.164188in}{3.107936in}}%
\pgfpathlineto{\pgfqpoint{4.164450in}{3.246856in}}%
\pgfpathlineto{\pgfqpoint{4.164866in}{3.158387in}}%
\pgfpathlineto{\pgfqpoint{4.165621in}{3.245676in}}%
\pgfpathlineto{\pgfqpoint{4.165144in}{3.122096in}}%
\pgfpathlineto{\pgfqpoint{4.165945in}{3.133833in}}%
\pgfpathlineto{\pgfqpoint{4.165991in}{3.110759in}}%
\pgfpathlineto{\pgfqpoint{4.166207in}{3.223507in}}%
\pgfpathlineto{\pgfqpoint{4.166747in}{3.223489in}}%
\pgfpathlineto{\pgfqpoint{4.167363in}{3.238854in}}%
\pgfpathlineto{\pgfqpoint{4.167101in}{3.113479in}}%
\pgfpathlineto{\pgfqpoint{4.167687in}{3.118758in}}%
\pgfpathlineto{\pgfqpoint{4.168519in}{3.239931in}}%
\pgfpathlineto{\pgfqpoint{4.168828in}{3.136477in}}%
\pgfpathlineto{\pgfqpoint{4.168889in}{3.122077in}}%
\pgfpathlineto{\pgfqpoint{4.169090in}{3.199427in}}%
\pgfpathlineto{\pgfqpoint{4.169676in}{3.221891in}}%
\pgfpathlineto{\pgfqpoint{4.170030in}{3.115522in}}%
\pgfpathlineto{\pgfqpoint{4.170045in}{3.111708in}}%
\pgfpathlineto{\pgfqpoint{4.170261in}{3.236676in}}%
\pgfpathlineto{\pgfqpoint{4.170801in}{3.195481in}}%
\pgfpathlineto{\pgfqpoint{4.171417in}{3.235773in}}%
\pgfpathlineto{\pgfqpoint{4.171032in}{3.137664in}}%
\pgfpathlineto{\pgfqpoint{4.171679in}{3.167454in}}%
\pgfpathlineto{\pgfqpoint{4.171710in}{3.125602in}}%
\pgfpathlineto{\pgfqpoint{4.172573in}{3.211557in}}%
\pgfpathlineto{\pgfqpoint{4.172789in}{3.144585in}}%
\pgfpathlineto{\pgfqpoint{4.173159in}{3.229196in}}%
\pgfpathlineto{\pgfqpoint{4.172959in}{3.122051in}}%
\pgfpathlineto{\pgfqpoint{4.173915in}{3.164797in}}%
\pgfpathlineto{\pgfqpoint{4.174685in}{3.116175in}}%
\pgfpathlineto{\pgfqpoint{4.174315in}{3.240034in}}%
\pgfpathlineto{\pgfqpoint{4.175009in}{3.172257in}}%
\pgfpathlineto{\pgfqpoint{4.176057in}{3.219587in}}%
\pgfpathlineto{\pgfqpoint{4.175857in}{3.129386in}}%
\pgfpathlineto{\pgfqpoint{4.176134in}{3.190415in}}%
\pgfpathlineto{\pgfqpoint{4.176751in}{3.132646in}}%
\pgfpathlineto{\pgfqpoint{4.177213in}{3.211791in}}%
\pgfpathlineto{\pgfqpoint{4.177244in}{3.215006in}}%
\pgfpathlineto{\pgfqpoint{4.177691in}{3.114112in}}%
\pgfpathlineto{\pgfqpoint{4.178061in}{3.189410in}}%
\pgfpathlineto{\pgfqpoint{4.178678in}{3.120448in}}%
\pgfpathlineto{\pgfqpoint{4.178924in}{3.216800in}}%
\pgfpathlineto{\pgfqpoint{4.179171in}{3.164645in}}%
\pgfpathlineto{\pgfqpoint{4.180081in}{3.212760in}}%
\pgfpathlineto{\pgfqpoint{4.179649in}{3.127245in}}%
\pgfpathlineto{\pgfqpoint{4.180281in}{3.171402in}}%
\pgfpathlineto{\pgfqpoint{4.180589in}{3.110028in}}%
\pgfpathlineto{\pgfqpoint{4.180697in}{3.212688in}}%
\pgfpathlineto{\pgfqpoint{4.181406in}{3.154141in}}%
\pgfpathlineto{\pgfqpoint{4.181822in}{3.223972in}}%
\pgfpathlineto{\pgfqpoint{4.181576in}{3.122198in}}%
\pgfpathlineto{\pgfqpoint{4.182516in}{3.164502in}}%
\pgfpathlineto{\pgfqpoint{4.183503in}{3.115783in}}%
\pgfpathlineto{\pgfqpoint{4.182979in}{3.213196in}}%
\pgfpathlineto{\pgfqpoint{4.183580in}{3.190314in}}%
\pgfpathlineto{\pgfqpoint{4.183842in}{3.205307in}}%
\pgfpathlineto{\pgfqpoint{4.184489in}{3.121062in}}%
\pgfpathlineto{\pgfqpoint{4.184612in}{3.159835in}}%
\pgfpathlineto{\pgfqpoint{4.184659in}{3.119762in}}%
\pgfpathlineto{\pgfqpoint{4.184736in}{3.217363in}}%
\pgfpathlineto{\pgfqpoint{4.185707in}{3.160805in}}%
\pgfpathlineto{\pgfqpoint{4.185892in}{3.224803in}}%
\pgfpathlineto{\pgfqpoint{4.186401in}{3.129110in}}%
\pgfpathlineto{\pgfqpoint{4.186848in}{3.172896in}}%
\pgfpathlineto{\pgfqpoint{4.187387in}{3.127386in}}%
\pgfpathlineto{\pgfqpoint{4.187048in}{3.216527in}}%
\pgfpathlineto{\pgfqpoint{4.187896in}{3.184969in}}%
\pgfpathlineto{\pgfqpoint{4.188805in}{3.234177in}}%
\pgfpathlineto{\pgfqpoint{4.188374in}{3.130475in}}%
\pgfpathlineto{\pgfqpoint{4.188990in}{3.179152in}}%
\pgfpathlineto{\pgfqpoint{4.189299in}{3.131322in}}%
\pgfpathlineto{\pgfqpoint{4.189961in}{3.222835in}}%
\pgfpathlineto{\pgfqpoint{4.190116in}{3.144948in}}%
\pgfpathlineto{\pgfqpoint{4.190547in}{3.214124in}}%
\pgfpathlineto{\pgfqpoint{4.190193in}{3.127843in}}%
\pgfpathlineto{\pgfqpoint{4.191241in}{3.160094in}}%
\pgfpathlineto{\pgfqpoint{4.191349in}{3.126563in}}%
\pgfpathlineto{\pgfqpoint{4.191703in}{3.228800in}}%
\pgfpathlineto{\pgfqpoint{4.192304in}{3.193608in}}%
\pgfpathlineto{\pgfqpoint{4.193091in}{3.117909in}}%
\pgfpathlineto{\pgfqpoint{4.192875in}{3.231012in}}%
\pgfpathlineto{\pgfqpoint{4.193414in}{3.187687in}}%
\pgfpathlineto{\pgfqpoint{4.193461in}{3.219893in}}%
\pgfpathlineto{\pgfqpoint{4.194247in}{3.122611in}}%
\pgfpathlineto{\pgfqpoint{4.194493in}{3.188521in}}%
\pgfpathlineto{\pgfqpoint{4.195110in}{3.135729in}}%
\pgfpathlineto{\pgfqpoint{4.194617in}{3.215622in}}%
\pgfpathlineto{\pgfqpoint{4.195603in}{3.166877in}}%
\pgfpathlineto{\pgfqpoint{4.195773in}{3.223924in}}%
\pgfpathlineto{\pgfqpoint{4.196004in}{3.124221in}}%
\pgfpathlineto{\pgfqpoint{4.196559in}{3.165106in}}%
\pgfpathlineto{\pgfqpoint{4.197160in}{3.114113in}}%
\pgfpathlineto{\pgfqpoint{4.197515in}{3.215043in}}%
\pgfpathlineto{\pgfqpoint{4.197653in}{3.172148in}}%
\pgfpathlineto{\pgfqpoint{4.198023in}{3.140371in}}%
\pgfpathlineto{\pgfqpoint{4.198378in}{3.210094in}}%
\pgfpathlineto{\pgfqpoint{4.198640in}{3.207723in}}%
\pgfpathlineto{\pgfqpoint{4.198671in}{3.221548in}}%
\pgfpathlineto{\pgfqpoint{4.199010in}{3.128651in}}%
\pgfpathlineto{\pgfqpoint{4.199719in}{3.191279in}}%
\pgfpathlineto{\pgfqpoint{4.200058in}{3.120570in}}%
\pgfpathlineto{\pgfqpoint{4.200413in}{3.218180in}}%
\pgfpathlineto{\pgfqpoint{4.200844in}{3.152907in}}%
\pgfpathlineto{\pgfqpoint{4.201569in}{3.218074in}}%
\pgfpathlineto{\pgfqpoint{4.201908in}{3.131521in}}%
\pgfpathlineto{\pgfqpoint{4.201923in}{3.130850in}}%
\pgfpathlineto{\pgfqpoint{4.202108in}{3.169846in}}%
\pgfpathlineto{\pgfqpoint{4.202740in}{3.211285in}}%
\pgfpathlineto{\pgfqpoint{4.202972in}{3.122418in}}%
\pgfpathlineto{\pgfqpoint{4.203218in}{3.171966in}}%
\pgfpathlineto{\pgfqpoint{4.204128in}{3.137434in}}%
\pgfpathlineto{\pgfqpoint{4.203604in}{3.210677in}}%
\pgfpathlineto{\pgfqpoint{4.204328in}{3.172184in}}%
\pgfpathlineto{\pgfqpoint{4.204482in}{3.218126in}}%
\pgfpathlineto{\pgfqpoint{4.204821in}{3.127892in}}%
\pgfpathlineto{\pgfqpoint{4.205376in}{3.154313in}}%
\pgfpathlineto{\pgfqpoint{4.205870in}{3.133988in}}%
\pgfpathlineto{\pgfqpoint{4.205638in}{3.212310in}}%
\pgfpathlineto{\pgfqpoint{4.206471in}{3.171038in}}%
\pgfpathlineto{\pgfqpoint{4.207380in}{3.217058in}}%
\pgfpathlineto{\pgfqpoint{4.207149in}{3.136362in}}%
\pgfpathlineto{\pgfqpoint{4.207581in}{3.177806in}}%
\pgfpathlineto{\pgfqpoint{4.207719in}{3.132498in}}%
\pgfpathlineto{\pgfqpoint{4.207966in}{3.211328in}}%
\pgfpathlineto{\pgfqpoint{4.208706in}{3.152987in}}%
\pgfpathlineto{\pgfqpoint{4.209400in}{3.212494in}}%
\pgfpathlineto{\pgfqpoint{4.208875in}{3.135590in}}%
\pgfpathlineto{\pgfqpoint{4.209769in}{3.145119in}}%
\pgfpathlineto{\pgfqpoint{4.210787in}{3.126842in}}%
\pgfpathlineto{\pgfqpoint{4.210278in}{3.219216in}}%
\pgfpathlineto{\pgfqpoint{4.210802in}{3.137168in}}%
\pgfpathlineto{\pgfqpoint{4.211434in}{3.217412in}}%
\pgfpathlineto{\pgfqpoint{4.211681in}{3.133725in}}%
\pgfpathlineto{\pgfqpoint{4.211928in}{3.167640in}}%
\pgfpathlineto{\pgfqpoint{4.212698in}{3.143387in}}%
\pgfpathlineto{\pgfqpoint{4.212883in}{3.212357in}}%
\pgfpathlineto{\pgfqpoint{4.213022in}{3.162226in}}%
\pgfpathlineto{\pgfqpoint{4.213176in}{3.202348in}}%
\pgfpathlineto{\pgfqpoint{4.213685in}{3.131718in}}%
\pgfpathlineto{\pgfqpoint{4.214101in}{3.137205in}}%
\pgfpathlineto{\pgfqpoint{4.214332in}{3.215099in}}%
\pgfpathlineto{\pgfqpoint{4.214579in}{3.130867in}}%
\pgfpathlineto{\pgfqpoint{4.215226in}{3.168197in}}%
\pgfpathlineto{\pgfqpoint{4.215735in}{3.136903in}}%
\pgfpathlineto{\pgfqpoint{4.215380in}{3.205456in}}%
\pgfpathlineto{\pgfqpoint{4.216321in}{3.153129in}}%
\pgfpathlineto{\pgfqpoint{4.216937in}{3.200109in}}%
\pgfpathlineto{\pgfqpoint{4.216583in}{3.142673in}}%
\pgfpathlineto{\pgfqpoint{4.217431in}{3.169820in}}%
\pgfpathlineto{\pgfqpoint{4.217477in}{3.137451in}}%
\pgfpathlineto{\pgfqpoint{4.218263in}{3.201777in}}%
\pgfpathlineto{\pgfqpoint{4.218541in}{3.143584in}}%
\pgfpathlineto{\pgfqpoint{4.219250in}{3.202551in}}%
\pgfpathlineto{\pgfqpoint{4.218633in}{3.137632in}}%
\pgfpathlineto{\pgfqpoint{4.219650in}{3.169737in}}%
\pgfpathlineto{\pgfqpoint{4.220452in}{3.135241in}}%
\pgfpathlineto{\pgfqpoint{4.220699in}{3.198510in}}%
\pgfpathlineto{\pgfqpoint{4.220745in}{3.174963in}}%
\pgfpathlineto{\pgfqpoint{4.221423in}{3.133373in}}%
\pgfpathlineto{\pgfqpoint{4.221284in}{3.196159in}}%
\pgfpathlineto{\pgfqpoint{4.221824in}{3.174000in}}%
\pgfpathlineto{\pgfqpoint{4.222148in}{3.194085in}}%
\pgfpathlineto{\pgfqpoint{4.222579in}{3.136282in}}%
\pgfpathlineto{\pgfqpoint{4.222934in}{3.176246in}}%
\pgfpathlineto{\pgfqpoint{4.223350in}{3.134899in}}%
\pgfpathlineto{\pgfqpoint{4.223597in}{3.197745in}}%
\pgfpathlineto{\pgfqpoint{4.224028in}{3.175846in}}%
\pgfpathlineto{\pgfqpoint{4.224444in}{3.191827in}}%
\pgfpathlineto{\pgfqpoint{4.224321in}{3.138660in}}%
\pgfpathlineto{\pgfqpoint{4.224876in}{3.151392in}}%
\pgfpathlineto{\pgfqpoint{4.225477in}{3.140178in}}%
\pgfpathlineto{\pgfqpoint{4.225909in}{3.200218in}}%
\pgfpathlineto{\pgfqpoint{4.225940in}{3.182785in}}%
\pgfpathlineto{\pgfqpoint{4.226233in}{3.140924in}}%
\pgfpathlineto{\pgfqpoint{4.226757in}{3.194183in}}%
\pgfpathlineto{\pgfqpoint{4.227034in}{3.185011in}}%
\pgfpathlineto{\pgfqpoint{4.227497in}{3.201243in}}%
\pgfpathlineto{\pgfqpoint{4.227774in}{3.146658in}}%
\pgfpathlineto{\pgfqpoint{4.228082in}{3.184200in}}%
\pgfpathlineto{\pgfqpoint{4.229115in}{3.145465in}}%
\pgfpathlineto{\pgfqpoint{4.228483in}{3.195012in}}%
\pgfpathlineto{\pgfqpoint{4.229208in}{3.158172in}}%
\pgfpathlineto{\pgfqpoint{4.229809in}{3.193150in}}%
\pgfpathlineto{\pgfqpoint{4.230071in}{3.142276in}}%
\pgfpathlineto{\pgfqpoint{4.230317in}{3.169130in}}%
\pgfpathlineto{\pgfqpoint{4.230657in}{3.142191in}}%
\pgfpathlineto{\pgfqpoint{4.230780in}{3.193982in}}%
\pgfpathlineto{\pgfqpoint{4.231335in}{3.168609in}}%
\pgfpathlineto{\pgfqpoint{4.231366in}{3.201066in}}%
\pgfpathlineto{\pgfqpoint{4.231998in}{3.140606in}}%
\pgfpathlineto{\pgfqpoint{4.232429in}{3.168384in}}%
\pgfpathlineto{\pgfqpoint{4.232969in}{3.135122in}}%
\pgfpathlineto{\pgfqpoint{4.232522in}{3.198960in}}%
\pgfpathlineto{\pgfqpoint{4.233555in}{3.153930in}}%
\pgfpathlineto{\pgfqpoint{4.233678in}{3.203126in}}%
\pgfpathlineto{\pgfqpoint{4.234125in}{3.139205in}}%
\pgfpathlineto{\pgfqpoint{4.234664in}{3.183288in}}%
\pgfpathlineto{\pgfqpoint{4.234880in}{3.137746in}}%
\pgfpathlineto{\pgfqpoint{4.234834in}{3.209115in}}%
\pgfpathlineto{\pgfqpoint{4.235774in}{3.183517in}}%
\pgfpathlineto{\pgfqpoint{4.236560in}{3.202050in}}%
\pgfpathlineto{\pgfqpoint{4.236036in}{3.129616in}}%
\pgfpathlineto{\pgfqpoint{4.236869in}{3.183038in}}%
\pgfpathlineto{\pgfqpoint{4.237007in}{3.135621in}}%
\pgfpathlineto{\pgfqpoint{4.237717in}{3.213978in}}%
\pgfpathlineto{\pgfqpoint{4.237994in}{3.171634in}}%
\pgfpathlineto{\pgfqpoint{4.238302in}{3.195900in}}%
\pgfpathlineto{\pgfqpoint{4.238934in}{3.125746in}}%
\pgfpathlineto{\pgfqpoint{4.239088in}{3.163273in}}%
\pgfpathlineto{\pgfqpoint{4.239905in}{3.139428in}}%
\pgfpathlineto{\pgfqpoint{4.240029in}{3.197124in}}%
\pgfpathlineto{\pgfqpoint{4.240152in}{3.183686in}}%
\pgfpathlineto{\pgfqpoint{4.240615in}{3.203878in}}%
\pgfpathlineto{\pgfqpoint{4.240568in}{3.148990in}}%
\pgfpathlineto{\pgfqpoint{4.241046in}{3.158536in}}%
\pgfpathlineto{\pgfqpoint{4.241817in}{3.138877in}}%
\pgfpathlineto{\pgfqpoint{4.241493in}{3.198048in}}%
\pgfpathlineto{\pgfqpoint{4.242033in}{3.172803in}}%
\pgfpathlineto{\pgfqpoint{4.242079in}{3.198395in}}%
\pgfpathlineto{\pgfqpoint{4.242881in}{3.143567in}}%
\pgfpathlineto{\pgfqpoint{4.243127in}{3.180154in}}%
\pgfpathlineto{\pgfqpoint{4.244145in}{3.141671in}}%
\pgfpathlineto{\pgfqpoint{4.243235in}{3.203245in}}%
\pgfpathlineto{\pgfqpoint{4.244222in}{3.183654in}}%
\pgfpathlineto{\pgfqpoint{4.244391in}{3.202086in}}%
\pgfpathlineto{\pgfqpoint{4.245116in}{3.145144in}}%
\pgfpathlineto{\pgfqpoint{4.245270in}{3.161961in}}%
\pgfpathlineto{\pgfqpoint{4.245779in}{3.135578in}}%
\pgfpathlineto{\pgfqpoint{4.246133in}{3.196238in}}%
\pgfpathlineto{\pgfqpoint{4.246380in}{3.162865in}}%
\pgfpathlineto{\pgfqpoint{4.247289in}{3.200747in}}%
\pgfpathlineto{\pgfqpoint{4.246935in}{3.144064in}}%
\pgfpathlineto{\pgfqpoint{4.247474in}{3.174853in}}%
\pgfpathlineto{\pgfqpoint{4.248091in}{3.143941in}}%
\pgfpathlineto{\pgfqpoint{4.248445in}{3.201392in}}%
\pgfpathlineto{\pgfqpoint{4.248599in}{3.152380in}}%
\pgfpathlineto{\pgfqpoint{4.249031in}{3.198412in}}%
\pgfpathlineto{\pgfqpoint{4.248677in}{3.134322in}}%
\pgfpathlineto{\pgfqpoint{4.249725in}{3.169906in}}%
\pgfpathlineto{\pgfqpoint{4.249833in}{3.145028in}}%
\pgfpathlineto{\pgfqpoint{4.250187in}{3.199855in}}%
\pgfpathlineto{\pgfqpoint{4.250835in}{3.162769in}}%
\pgfpathlineto{\pgfqpoint{4.251343in}{3.203924in}}%
\pgfpathlineto{\pgfqpoint{4.251682in}{3.135556in}}%
\pgfpathlineto{\pgfqpoint{4.251975in}{3.178156in}}%
\pgfpathlineto{\pgfqpoint{4.252761in}{3.135783in}}%
\pgfpathlineto{\pgfqpoint{4.252484in}{3.199932in}}%
\pgfpathlineto{\pgfqpoint{4.253039in}{3.193040in}}%
\pgfpathlineto{\pgfqpoint{4.253085in}{3.207866in}}%
\pgfpathlineto{\pgfqpoint{4.253332in}{3.136296in}}%
\pgfpathlineto{\pgfqpoint{4.254133in}{3.189248in}}%
\pgfpathlineto{\pgfqpoint{4.254488in}{3.135373in}}%
\pgfpathlineto{\pgfqpoint{4.254226in}{3.202278in}}%
\pgfpathlineto{\pgfqpoint{4.255259in}{3.182165in}}%
\pgfpathlineto{\pgfqpoint{4.255289in}{3.183367in}}%
\pgfpathlineto{\pgfqpoint{4.255305in}{3.180535in}}%
\pgfpathlineto{\pgfqpoint{4.255644in}{3.132163in}}%
\pgfpathlineto{\pgfqpoint{4.255937in}{3.203321in}}%
\pgfpathlineto{\pgfqpoint{4.256415in}{3.177496in}}%
\pgfpathlineto{\pgfqpoint{4.256507in}{3.192924in}}%
\pgfpathlineto{\pgfqpoint{4.256646in}{3.156162in}}%
\pgfpathlineto{\pgfqpoint{4.256723in}{3.156740in}}%
\pgfpathlineto{\pgfqpoint{4.257386in}{3.131305in}}%
\pgfpathlineto{\pgfqpoint{4.257093in}{3.206391in}}%
\pgfpathlineto{\pgfqpoint{4.257818in}{3.164890in}}%
\pgfpathlineto{\pgfqpoint{4.258897in}{3.202150in}}%
\pgfpathlineto{\pgfqpoint{4.258557in}{3.130148in}}%
\pgfpathlineto{\pgfqpoint{4.258943in}{3.181767in}}%
\pgfpathlineto{\pgfqpoint{4.259729in}{3.127296in}}%
\pgfpathlineto{\pgfqpoint{4.259991in}{3.211504in}}%
\pgfpathlineto{\pgfqpoint{4.260022in}{3.206007in}}%
\pgfpathlineto{\pgfqpoint{4.260037in}{3.207408in}}%
\pgfpathlineto{\pgfqpoint{4.260222in}{3.124838in}}%
\pgfpathlineto{\pgfqpoint{4.260268in}{3.138843in}}%
\pgfpathlineto{\pgfqpoint{4.260315in}{3.120123in}}%
\pgfpathlineto{\pgfqpoint{4.260531in}{3.208106in}}%
\pgfpathlineto{\pgfqpoint{4.261317in}{3.184517in}}%
\pgfpathlineto{\pgfqpoint{4.261687in}{3.213484in}}%
\pgfpathlineto{\pgfqpoint{4.261455in}{3.125156in}}%
\pgfpathlineto{\pgfqpoint{4.262396in}{3.186218in}}%
\pgfpathlineto{\pgfqpoint{4.263197in}{3.115523in}}%
\pgfpathlineto{\pgfqpoint{4.262843in}{3.225476in}}%
\pgfpathlineto{\pgfqpoint{4.263475in}{3.194267in}}%
\pgfpathlineto{\pgfqpoint{4.263506in}{3.198837in}}%
\pgfpathlineto{\pgfqpoint{4.263660in}{3.160914in}}%
\pgfpathlineto{\pgfqpoint{4.263768in}{3.120630in}}%
\pgfpathlineto{\pgfqpoint{4.263999in}{3.217944in}}%
\pgfpathlineto{\pgfqpoint{4.264770in}{3.159096in}}%
\pgfpathlineto{\pgfqpoint{4.265833in}{3.213203in}}%
\pgfpathlineto{\pgfqpoint{4.265510in}{3.122899in}}%
\pgfpathlineto{\pgfqpoint{4.265895in}{3.173010in}}%
\pgfpathlineto{\pgfqpoint{4.266095in}{3.116017in}}%
\pgfpathlineto{\pgfqpoint{4.266897in}{3.214530in}}%
\pgfpathlineto{\pgfqpoint{4.266974in}{3.183599in}}%
\pgfpathlineto{\pgfqpoint{4.267467in}{3.209007in}}%
\pgfpathlineto{\pgfqpoint{4.267236in}{3.119679in}}%
\pgfpathlineto{\pgfqpoint{4.268084in}{3.185889in}}%
\pgfpathlineto{\pgfqpoint{4.268392in}{3.123851in}}%
\pgfpathlineto{\pgfqpoint{4.268623in}{3.206647in}}%
\pgfpathlineto{\pgfqpoint{4.269178in}{3.181533in}}%
\pgfpathlineto{\pgfqpoint{4.269779in}{3.217664in}}%
\pgfpathlineto{\pgfqpoint{4.270134in}{3.117682in}}%
\pgfpathlineto{\pgfqpoint{4.270257in}{3.157492in}}%
\pgfpathlineto{\pgfqpoint{4.270365in}{3.216558in}}%
\pgfpathlineto{\pgfqpoint{4.271275in}{3.122969in}}%
\pgfpathlineto{\pgfqpoint{4.271352in}{3.169929in}}%
\pgfpathlineto{\pgfqpoint{4.271860in}{3.129404in}}%
\pgfpathlineto{\pgfqpoint{4.271506in}{3.217567in}}%
\pgfpathlineto{\pgfqpoint{4.272462in}{3.157120in}}%
\pgfpathlineto{\pgfqpoint{4.272662in}{3.221790in}}%
\pgfpathlineto{\pgfqpoint{4.273017in}{3.126735in}}%
\pgfpathlineto{\pgfqpoint{4.273556in}{3.137492in}}%
\pgfpathlineto{\pgfqpoint{4.273587in}{3.134781in}}%
\pgfpathlineto{\pgfqpoint{4.273772in}{3.186853in}}%
\pgfpathlineto{\pgfqpoint{4.273803in}{3.216327in}}%
\pgfpathlineto{\pgfqpoint{4.274157in}{3.122517in}}%
\pgfpathlineto{\pgfqpoint{4.274820in}{3.139290in}}%
\pgfpathlineto{\pgfqpoint{4.274835in}{3.132264in}}%
\pgfpathlineto{\pgfqpoint{4.275529in}{3.211140in}}%
\pgfpathlineto{\pgfqpoint{4.275899in}{3.140552in}}%
\pgfpathlineto{\pgfqpoint{4.276685in}{3.206951in}}%
\pgfpathlineto{\pgfqpoint{4.275976in}{3.123969in}}%
\pgfpathlineto{\pgfqpoint{4.276994in}{3.141280in}}%
\pgfpathlineto{\pgfqpoint{4.277132in}{3.128615in}}%
\pgfpathlineto{\pgfqpoint{4.277256in}{3.210174in}}%
\pgfpathlineto{\pgfqpoint{4.277996in}{3.166537in}}%
\pgfpathlineto{\pgfqpoint{4.278412in}{3.198917in}}%
\pgfpathlineto{\pgfqpoint{4.278859in}{3.130060in}}%
\pgfpathlineto{\pgfqpoint{4.279121in}{3.189135in}}%
\pgfpathlineto{\pgfqpoint{4.279999in}{3.116426in}}%
\pgfpathlineto{\pgfqpoint{4.279198in}{3.200367in}}%
\pgfpathlineto{\pgfqpoint{4.280215in}{3.192222in}}%
\pgfpathlineto{\pgfqpoint{4.280354in}{3.194978in}}%
\pgfpathlineto{\pgfqpoint{4.280585in}{3.126180in}}%
\pgfpathlineto{\pgfqpoint{4.280955in}{3.158970in}}%
\pgfpathlineto{\pgfqpoint{4.281726in}{3.128605in}}%
\pgfpathlineto{\pgfqpoint{4.281495in}{3.199408in}}%
\pgfpathlineto{\pgfqpoint{4.282019in}{3.167973in}}%
\pgfpathlineto{\pgfqpoint{4.282080in}{3.199942in}}%
\pgfpathlineto{\pgfqpoint{4.282882in}{3.123604in}}%
\pgfpathlineto{\pgfqpoint{4.283144in}{3.181626in}}%
\pgfpathlineto{\pgfqpoint{4.283452in}{3.133252in}}%
\pgfpathlineto{\pgfqpoint{4.283221in}{3.200061in}}%
\pgfpathlineto{\pgfqpoint{4.284223in}{3.185507in}}%
\pgfpathlineto{\pgfqpoint{4.285117in}{3.205237in}}%
\pgfpathlineto{\pgfqpoint{4.284593in}{3.133186in}}%
\pgfpathlineto{\pgfqpoint{4.285318in}{3.170604in}}%
\pgfpathlineto{\pgfqpoint{4.285749in}{3.134597in}}%
\pgfpathlineto{\pgfqpoint{4.286258in}{3.208409in}}%
\pgfpathlineto{\pgfqpoint{4.286427in}{3.164926in}}%
\pgfpathlineto{\pgfqpoint{4.287414in}{3.203163in}}%
\pgfpathlineto{\pgfqpoint{4.286890in}{3.136169in}}%
\pgfpathlineto{\pgfqpoint{4.287491in}{3.140693in}}%
\pgfpathlineto{\pgfqpoint{4.288000in}{3.208700in}}%
\pgfpathlineto{\pgfqpoint{4.287537in}{3.138177in}}%
\pgfpathlineto{\pgfqpoint{4.288616in}{3.148348in}}%
\pgfpathlineto{\pgfqpoint{4.288693in}{3.138677in}}%
\pgfpathlineto{\pgfqpoint{4.289156in}{3.208426in}}%
\pgfpathlineto{\pgfqpoint{4.289526in}{3.181225in}}%
\pgfpathlineto{\pgfqpoint{4.290219in}{3.211144in}}%
\pgfpathlineto{\pgfqpoint{4.290435in}{3.135776in}}%
\pgfpathlineto{\pgfqpoint{4.290589in}{3.163515in}}%
\pgfpathlineto{\pgfqpoint{4.291591in}{3.135805in}}%
\pgfpathlineto{\pgfqpoint{4.290882in}{3.207766in}}%
\pgfpathlineto{\pgfqpoint{4.291730in}{3.156864in}}%
\pgfpathlineto{\pgfqpoint{4.291961in}{3.209753in}}%
\pgfpathlineto{\pgfqpoint{4.292748in}{3.142712in}}%
\pgfpathlineto{\pgfqpoint{4.292840in}{3.157886in}}%
\pgfpathlineto{\pgfqpoint{4.293333in}{3.138867in}}%
\pgfpathlineto{\pgfqpoint{4.293117in}{3.210009in}}%
\pgfpathlineto{\pgfqpoint{4.293688in}{3.200625in}}%
\pgfpathlineto{\pgfqpoint{4.294274in}{3.203271in}}%
\pgfpathlineto{\pgfqpoint{4.293904in}{3.146686in}}%
\pgfpathlineto{\pgfqpoint{4.294381in}{3.181322in}}%
\pgfpathlineto{\pgfqpoint{4.294489in}{3.139398in}}%
\pgfpathlineto{\pgfqpoint{4.294859in}{3.210870in}}%
\pgfpathlineto{\pgfqpoint{4.295491in}{3.171450in}}%
\pgfpathlineto{\pgfqpoint{4.296015in}{3.213491in}}%
\pgfpathlineto{\pgfqpoint{4.296231in}{3.138137in}}%
\pgfpathlineto{\pgfqpoint{4.296617in}{3.197574in}}%
\pgfpathlineto{\pgfqpoint{4.297387in}{3.134972in}}%
\pgfpathlineto{\pgfqpoint{4.297172in}{3.208692in}}%
\pgfpathlineto{\pgfqpoint{4.297727in}{3.179795in}}%
\pgfpathlineto{\pgfqpoint{4.297757in}{3.211017in}}%
\pgfpathlineto{\pgfqpoint{4.298559in}{3.139870in}}%
\pgfpathlineto{\pgfqpoint{4.298836in}{3.178726in}}%
\pgfpathlineto{\pgfqpoint{4.298913in}{3.210173in}}%
\pgfpathlineto{\pgfqpoint{4.299237in}{3.140081in}}%
\pgfpathlineto{\pgfqpoint{4.299823in}{3.153733in}}%
\pgfpathlineto{\pgfqpoint{4.300085in}{3.207506in}}%
\pgfpathlineto{\pgfqpoint{4.300301in}{3.139213in}}%
\pgfpathlineto{\pgfqpoint{4.300871in}{3.150732in}}%
\pgfpathlineto{\pgfqpoint{4.301457in}{3.140358in}}%
\pgfpathlineto{\pgfqpoint{4.301657in}{3.197558in}}%
\pgfpathlineto{\pgfqpoint{4.301781in}{3.200837in}}%
\pgfpathlineto{\pgfqpoint{4.301996in}{3.153478in}}%
\pgfpathlineto{\pgfqpoint{4.302012in}{3.154192in}}%
\pgfpathlineto{\pgfqpoint{4.302073in}{3.136498in}}%
\pgfpathlineto{\pgfqpoint{4.302829in}{3.200310in}}%
\pgfpathlineto{\pgfqpoint{4.303091in}{3.162911in}}%
\pgfpathlineto{\pgfqpoint{4.304000in}{3.203045in}}%
\pgfpathlineto{\pgfqpoint{4.303245in}{3.123529in}}%
\pgfpathlineto{\pgfqpoint{4.304232in}{3.170293in}}%
\pgfpathlineto{\pgfqpoint{4.304401in}{3.127145in}}%
\pgfpathlineto{\pgfqpoint{4.304586in}{3.204012in}}%
\pgfpathlineto{\pgfqpoint{4.305295in}{3.181129in}}%
\pgfpathlineto{\pgfqpoint{4.305311in}{3.181064in}}%
\pgfpathlineto{\pgfqpoint{4.306158in}{3.118342in}}%
\pgfpathlineto{\pgfqpoint{4.305788in}{3.212653in}}%
\pgfpathlineto{\pgfqpoint{4.306343in}{3.207228in}}%
\pgfpathlineto{\pgfqpoint{4.306374in}{3.216811in}}%
\pgfpathlineto{\pgfqpoint{4.306729in}{3.126591in}}%
\pgfpathlineto{\pgfqpoint{4.307268in}{3.146754in}}%
\pgfpathlineto{\pgfqpoint{4.307315in}{3.127281in}}%
\pgfpathlineto{\pgfqpoint{4.307546in}{3.219925in}}%
\pgfpathlineto{\pgfqpoint{4.308301in}{3.168266in}}%
\pgfpathlineto{\pgfqpoint{4.309087in}{3.130983in}}%
\pgfpathlineto{\pgfqpoint{4.308702in}{3.222946in}}%
\pgfpathlineto{\pgfqpoint{4.309226in}{3.178581in}}%
\pgfpathlineto{\pgfqpoint{4.309873in}{3.221967in}}%
\pgfpathlineto{\pgfqpoint{4.310259in}{3.139404in}}%
\pgfpathlineto{\pgfqpoint{4.310336in}{3.186921in}}%
\pgfpathlineto{\pgfqpoint{4.310351in}{3.186846in}}%
\pgfpathlineto{\pgfqpoint{4.310382in}{3.189258in}}%
\pgfpathlineto{\pgfqpoint{4.310459in}{3.225448in}}%
\pgfpathlineto{\pgfqpoint{4.310845in}{3.139331in}}%
\pgfpathlineto{\pgfqpoint{4.311353in}{3.163226in}}%
\pgfpathlineto{\pgfqpoint{4.312001in}{3.135280in}}%
\pgfpathlineto{\pgfqpoint{4.311615in}{3.216456in}}%
\pgfpathlineto{\pgfqpoint{4.312186in}{3.202096in}}%
\pgfpathlineto{\pgfqpoint{4.312216in}{3.214197in}}%
\pgfpathlineto{\pgfqpoint{4.313018in}{3.140202in}}%
\pgfpathlineto{\pgfqpoint{4.313280in}{3.195904in}}%
\pgfpathlineto{\pgfqpoint{4.314112in}{3.141015in}}%
\pgfpathlineto{\pgfqpoint{4.313373in}{3.217438in}}%
\pgfpathlineto{\pgfqpoint{4.314405in}{3.186512in}}%
\pgfpathlineto{\pgfqpoint{4.314544in}{3.209578in}}%
\pgfpathlineto{\pgfqpoint{4.314698in}{3.141155in}}%
\pgfpathlineto{\pgfqpoint{4.315438in}{3.166191in}}%
\pgfpathlineto{\pgfqpoint{4.315854in}{3.138028in}}%
\pgfpathlineto{\pgfqpoint{4.316301in}{3.213612in}}%
\pgfpathlineto{\pgfqpoint{4.316563in}{3.159676in}}%
\pgfpathlineto{\pgfqpoint{4.317457in}{3.204713in}}%
\pgfpathlineto{\pgfqpoint{4.317026in}{3.135557in}}%
\pgfpathlineto{\pgfqpoint{4.317658in}{3.153270in}}%
\pgfpathlineto{\pgfqpoint{4.318768in}{3.133832in}}%
\pgfpathlineto{\pgfqpoint{4.318506in}{3.200599in}}%
\pgfpathlineto{\pgfqpoint{4.318783in}{3.140007in}}%
\pgfpathlineto{\pgfqpoint{4.319091in}{3.203777in}}%
\pgfpathlineto{\pgfqpoint{4.319354in}{3.137540in}}%
\pgfpathlineto{\pgfqpoint{4.319893in}{3.142382in}}%
\pgfpathlineto{\pgfqpoint{4.319924in}{3.132048in}}%
\pgfpathlineto{\pgfqpoint{4.320248in}{3.200726in}}%
\pgfpathlineto{\pgfqpoint{4.320818in}{3.187711in}}%
\pgfpathlineto{\pgfqpoint{4.320833in}{3.188342in}}%
\pgfpathlineto{\pgfqpoint{4.321049in}{3.152428in}}%
\pgfpathlineto{\pgfqpoint{4.321666in}{3.131373in}}%
\pgfpathlineto{\pgfqpoint{4.321989in}{3.198004in}}%
\pgfpathlineto{\pgfqpoint{4.322067in}{3.177525in}}%
\pgfpathlineto{\pgfqpoint{4.322560in}{3.193141in}}%
\pgfpathlineto{\pgfqpoint{4.322251in}{3.134711in}}%
\pgfpathlineto{\pgfqpoint{4.322806in}{3.141190in}}%
\pgfpathlineto{\pgfqpoint{4.322822in}{3.131802in}}%
\pgfpathlineto{\pgfqpoint{4.323130in}{3.199781in}}%
\pgfpathlineto{\pgfqpoint{4.323885in}{3.151567in}}%
\pgfpathlineto{\pgfqpoint{4.324117in}{3.203175in}}%
\pgfpathlineto{\pgfqpoint{4.324564in}{3.135347in}}%
\pgfpathlineto{\pgfqpoint{4.325103in}{3.165218in}}%
\pgfpathlineto{\pgfqpoint{4.325704in}{3.125338in}}%
\pgfpathlineto{\pgfqpoint{4.325997in}{3.219208in}}%
\pgfpathlineto{\pgfqpoint{4.326213in}{3.161665in}}%
\pgfpathlineto{\pgfqpoint{4.326275in}{3.134961in}}%
\pgfpathlineto{\pgfqpoint{4.327153in}{3.221303in}}%
\pgfpathlineto{\pgfqpoint{4.327261in}{3.189968in}}%
\pgfpathlineto{\pgfqpoint{4.327724in}{3.219237in}}%
\pgfpathlineto{\pgfqpoint{4.327431in}{3.134353in}}%
\pgfpathlineto{\pgfqpoint{4.328340in}{3.193984in}}%
\pgfpathlineto{\pgfqpoint{4.328633in}{3.129403in}}%
\pgfpathlineto{\pgfqpoint{4.328895in}{3.213246in}}%
\pgfpathlineto{\pgfqpoint{4.329435in}{3.196774in}}%
\pgfpathlineto{\pgfqpoint{4.329466in}{3.209395in}}%
\pgfpathlineto{\pgfqpoint{4.330283in}{3.145096in}}%
\pgfpathlineto{\pgfqpoint{4.330498in}{3.160436in}}%
\pgfpathlineto{\pgfqpoint{4.331007in}{3.207928in}}%
\pgfpathlineto{\pgfqpoint{4.331485in}{3.141832in}}%
\pgfpathlineto{\pgfqpoint{4.331608in}{3.173754in}}%
\pgfpathlineto{\pgfqpoint{4.332487in}{3.140673in}}%
\pgfpathlineto{\pgfqpoint{4.332148in}{3.202354in}}%
\pgfpathlineto{\pgfqpoint{4.332687in}{3.189693in}}%
\pgfpathlineto{\pgfqpoint{4.332718in}{3.207460in}}%
\pgfpathlineto{\pgfqpoint{4.333196in}{3.141254in}}%
\pgfpathlineto{\pgfqpoint{4.333751in}{3.150273in}}%
\pgfpathlineto{\pgfqpoint{4.334352in}{3.141606in}}%
\pgfpathlineto{\pgfqpoint{4.333859in}{3.212170in}}%
\pgfpathlineto{\pgfqpoint{4.334799in}{3.159509in}}%
\pgfpathlineto{\pgfqpoint{4.335570in}{3.224322in}}%
\pgfpathlineto{\pgfqpoint{4.334922in}{3.145781in}}%
\pgfpathlineto{\pgfqpoint{4.335894in}{3.171619in}}%
\pgfpathlineto{\pgfqpoint{4.336079in}{3.147777in}}%
\pgfpathlineto{\pgfqpoint{4.336140in}{3.222462in}}%
\pgfpathlineto{\pgfqpoint{4.337003in}{3.171569in}}%
\pgfpathlineto{\pgfqpoint{4.337851in}{3.211831in}}%
\pgfpathlineto{\pgfqpoint{4.337774in}{3.153146in}}%
\pgfpathlineto{\pgfqpoint{4.338067in}{3.163708in}}%
\pgfpathlineto{\pgfqpoint{4.338915in}{3.143675in}}%
\pgfpathlineto{\pgfqpoint{4.338422in}{3.206836in}}%
\pgfpathlineto{\pgfqpoint{4.339146in}{3.172188in}}%
\pgfpathlineto{\pgfqpoint{4.339562in}{3.211578in}}%
\pgfpathlineto{\pgfqpoint{4.339485in}{3.145761in}}%
\pgfpathlineto{\pgfqpoint{4.340025in}{3.158182in}}%
\pgfpathlineto{\pgfqpoint{4.340934in}{3.134926in}}%
\pgfpathlineto{\pgfqpoint{4.340703in}{3.213991in}}%
\pgfpathlineto{\pgfqpoint{4.341119in}{3.167891in}}%
\pgfpathlineto{\pgfqpoint{4.341150in}{3.171694in}}%
\pgfpathlineto{\pgfqpoint{4.341165in}{3.160835in}}%
\pgfpathlineto{\pgfqpoint{4.342075in}{3.127915in}}%
\pgfpathlineto{\pgfqpoint{4.341844in}{3.212781in}}%
\pgfpathlineto{\pgfqpoint{4.342260in}{3.171730in}}%
\pgfpathlineto{\pgfqpoint{4.342414in}{3.218398in}}%
\pgfpathlineto{\pgfqpoint{4.342645in}{3.130078in}}%
\pgfpathlineto{\pgfqpoint{4.343200in}{3.149189in}}%
\pgfpathlineto{\pgfqpoint{4.343216in}{3.136862in}}%
\pgfpathlineto{\pgfqpoint{4.343555in}{3.212143in}}%
\pgfpathlineto{\pgfqpoint{4.344264in}{3.174493in}}%
\pgfpathlineto{\pgfqpoint{4.344711in}{3.211371in}}%
\pgfpathlineto{\pgfqpoint{4.344942in}{3.141257in}}%
\pgfpathlineto{\pgfqpoint{4.345374in}{3.177062in}}%
\pgfpathlineto{\pgfqpoint{4.345852in}{3.204807in}}%
\pgfpathlineto{\pgfqpoint{4.345512in}{3.145802in}}%
\pgfpathlineto{\pgfqpoint{4.346453in}{3.177031in}}%
\pgfpathlineto{\pgfqpoint{4.346761in}{3.150768in}}%
\pgfpathlineto{\pgfqpoint{4.346992in}{3.198640in}}%
\pgfpathlineto{\pgfqpoint{4.347547in}{3.185587in}}%
\pgfpathlineto{\pgfqpoint{4.348148in}{3.202190in}}%
\pgfpathlineto{\pgfqpoint{4.347917in}{3.145817in}}%
\pgfpathlineto{\pgfqpoint{4.348626in}{3.166733in}}%
\pgfpathlineto{\pgfqpoint{4.349305in}{3.202792in}}%
\pgfpathlineto{\pgfqpoint{4.349073in}{3.148251in}}%
\pgfpathlineto{\pgfqpoint{4.349736in}{3.168585in}}%
\pgfpathlineto{\pgfqpoint{4.350831in}{3.148071in}}%
\pgfpathlineto{\pgfqpoint{4.350461in}{3.201699in}}%
\pgfpathlineto{\pgfqpoint{4.350846in}{3.153464in}}%
\pgfpathlineto{\pgfqpoint{4.351046in}{3.202222in}}%
\pgfpathlineto{\pgfqpoint{4.351416in}{3.148598in}}%
\pgfpathlineto{\pgfqpoint{4.351956in}{3.165033in}}%
\pgfpathlineto{\pgfqpoint{4.351987in}{3.148017in}}%
\pgfpathlineto{\pgfqpoint{4.352203in}{3.202966in}}%
\pgfpathlineto{\pgfqpoint{4.353066in}{3.163947in}}%
\pgfpathlineto{\pgfqpoint{4.353359in}{3.198156in}}%
\pgfpathlineto{\pgfqpoint{4.353590in}{3.147932in}}%
\pgfpathlineto{\pgfqpoint{4.354145in}{3.155207in}}%
\pgfpathlineto{\pgfqpoint{4.354761in}{3.142187in}}%
\pgfpathlineto{\pgfqpoint{4.355023in}{3.195052in}}%
\pgfpathlineto{\pgfqpoint{4.355100in}{3.188565in}}%
\pgfpathlineto{\pgfqpoint{4.355609in}{3.195684in}}%
\pgfpathlineto{\pgfqpoint{4.355825in}{3.141485in}}%
\pgfpathlineto{\pgfqpoint{4.356133in}{3.168214in}}%
\pgfpathlineto{\pgfqpoint{4.356149in}{3.168421in}}%
\pgfpathlineto{\pgfqpoint{4.356180in}{3.199196in}}%
\pgfpathlineto{\pgfqpoint{4.356411in}{3.142501in}}%
\pgfpathlineto{\pgfqpoint{4.357259in}{3.179513in}}%
\pgfpathlineto{\pgfqpoint{4.358153in}{3.141585in}}%
\pgfpathlineto{\pgfqpoint{4.357351in}{3.197391in}}%
\pgfpathlineto{\pgfqpoint{4.358353in}{3.185105in}}%
\pgfpathlineto{\pgfqpoint{4.359324in}{3.134603in}}%
\pgfpathlineto{\pgfqpoint{4.358523in}{3.194555in}}%
\pgfpathlineto{\pgfqpoint{4.359478in}{3.172828in}}%
\pgfpathlineto{\pgfqpoint{4.360110in}{3.198748in}}%
\pgfpathlineto{\pgfqpoint{4.360496in}{3.127599in}}%
\pgfpathlineto{\pgfqpoint{4.360557in}{3.151786in}}%
\pgfpathlineto{\pgfqpoint{4.361066in}{3.127312in}}%
\pgfpathlineto{\pgfqpoint{4.361266in}{3.198126in}}%
\pgfpathlineto{\pgfqpoint{4.361513in}{3.172490in}}%
\pgfpathlineto{\pgfqpoint{4.362438in}{3.203785in}}%
\pgfpathlineto{\pgfqpoint{4.361652in}{3.121101in}}%
\pgfpathlineto{\pgfqpoint{4.362623in}{3.179143in}}%
\pgfpathlineto{\pgfqpoint{4.362808in}{3.128809in}}%
\pgfpathlineto{\pgfqpoint{4.363024in}{3.207249in}}%
\pgfpathlineto{\pgfqpoint{4.363733in}{3.176684in}}%
\pgfpathlineto{\pgfqpoint{4.363779in}{3.187400in}}%
\pgfpathlineto{\pgfqpoint{4.363933in}{3.148352in}}%
\pgfpathlineto{\pgfqpoint{4.364565in}{3.137620in}}%
\pgfpathlineto{\pgfqpoint{4.364195in}{3.210111in}}%
\pgfpathlineto{\pgfqpoint{4.364750in}{3.205248in}}%
\pgfpathlineto{\pgfqpoint{4.364781in}{3.214005in}}%
\pgfpathlineto{\pgfqpoint{4.365166in}{3.131474in}}%
\pgfpathlineto{\pgfqpoint{4.365706in}{3.151482in}}%
\pgfpathlineto{\pgfqpoint{4.366322in}{3.133075in}}%
\pgfpathlineto{\pgfqpoint{4.365937in}{3.209423in}}%
\pgfpathlineto{\pgfqpoint{4.366800in}{3.158527in}}%
\pgfpathlineto{\pgfqpoint{4.367479in}{3.130774in}}%
\pgfpathlineto{\pgfqpoint{4.367109in}{3.209579in}}%
\pgfpathlineto{\pgfqpoint{4.367849in}{3.177526in}}%
\pgfpathlineto{\pgfqpoint{4.368851in}{3.208815in}}%
\pgfpathlineto{\pgfqpoint{4.368064in}{3.133999in}}%
\pgfpathlineto{\pgfqpoint{4.368958in}{3.177269in}}%
\pgfpathlineto{\pgfqpoint{4.369220in}{3.139949in}}%
\pgfpathlineto{\pgfqpoint{4.369436in}{3.205548in}}%
\pgfpathlineto{\pgfqpoint{4.369976in}{3.193297in}}%
\pgfpathlineto{\pgfqpoint{4.370608in}{3.210965in}}%
\pgfpathlineto{\pgfqpoint{4.370407in}{3.139438in}}%
\pgfpathlineto{\pgfqpoint{4.370978in}{3.142602in}}%
\pgfpathlineto{\pgfqpoint{4.370993in}{3.140222in}}%
\pgfpathlineto{\pgfqpoint{4.371194in}{3.206438in}}%
\pgfpathlineto{\pgfqpoint{4.371733in}{3.194510in}}%
\pgfpathlineto{\pgfqpoint{4.371764in}{3.207466in}}%
\pgfpathlineto{\pgfqpoint{4.372149in}{3.145620in}}%
\pgfpathlineto{\pgfqpoint{4.372828in}{3.184908in}}%
\pgfpathlineto{\pgfqpoint{4.373074in}{3.147898in}}%
\pgfpathlineto{\pgfqpoint{4.372935in}{3.206764in}}%
\pgfpathlineto{\pgfqpoint{4.373475in}{3.194542in}}%
\pgfpathlineto{\pgfqpoint{4.373506in}{3.204949in}}%
\pgfpathlineto{\pgfqpoint{4.374230in}{3.139622in}}%
\pgfpathlineto{\pgfqpoint{4.374554in}{3.179955in}}%
\pgfpathlineto{\pgfqpoint{4.374677in}{3.194880in}}%
\pgfpathlineto{\pgfqpoint{4.374816in}{3.138228in}}%
\pgfpathlineto{\pgfqpoint{4.375356in}{3.156473in}}%
\pgfpathlineto{\pgfqpoint{4.375386in}{3.137990in}}%
\pgfpathlineto{\pgfqpoint{4.375818in}{3.193678in}}%
\pgfpathlineto{\pgfqpoint{4.376404in}{3.191652in}}%
\pgfpathlineto{\pgfqpoint{4.376419in}{3.191978in}}%
\pgfpathlineto{\pgfqpoint{4.376481in}{3.172787in}}%
\pgfpathlineto{\pgfqpoint{4.377128in}{3.141111in}}%
\pgfpathlineto{\pgfqpoint{4.376943in}{3.193669in}}%
\pgfpathlineto{\pgfqpoint{4.377498in}{3.188006in}}%
\pgfpathlineto{\pgfqpoint{4.377529in}{3.192312in}}%
\pgfpathlineto{\pgfqpoint{4.377714in}{3.141618in}}%
\pgfpathlineto{\pgfqpoint{4.378454in}{3.156723in}}%
\pgfpathlineto{\pgfqpoint{4.379456in}{3.143936in}}%
\pgfpathlineto{\pgfqpoint{4.378685in}{3.193087in}}%
\pgfpathlineto{\pgfqpoint{4.379548in}{3.160602in}}%
\pgfpathlineto{\pgfqpoint{4.380427in}{3.195115in}}%
\pgfpathlineto{\pgfqpoint{4.380026in}{3.143542in}}%
\pgfpathlineto{\pgfqpoint{4.380643in}{3.158389in}}%
\pgfpathlineto{\pgfqpoint{4.380689in}{3.146434in}}%
\pgfpathlineto{\pgfqpoint{4.381475in}{3.197127in}}%
\pgfpathlineto{\pgfqpoint{4.381491in}{3.200435in}}%
\pgfpathlineto{\pgfqpoint{4.381845in}{3.148695in}}%
\pgfpathlineto{\pgfqpoint{4.382385in}{3.156620in}}%
\pgfpathlineto{\pgfqpoint{4.382986in}{3.146841in}}%
\pgfpathlineto{\pgfqpoint{4.382631in}{3.199664in}}%
\pgfpathlineto{\pgfqpoint{4.383202in}{3.197892in}}%
\pgfpathlineto{\pgfqpoint{4.383787in}{3.203810in}}%
\pgfpathlineto{\pgfqpoint{4.383988in}{3.144841in}}%
\pgfpathlineto{\pgfqpoint{4.384096in}{3.150365in}}%
\pgfpathlineto{\pgfqpoint{4.384142in}{3.144666in}}%
\pgfpathlineto{\pgfqpoint{4.384373in}{3.202693in}}%
\pgfpathlineto{\pgfqpoint{4.384928in}{3.196169in}}%
\pgfpathlineto{\pgfqpoint{4.385529in}{3.199793in}}%
\pgfpathlineto{\pgfqpoint{4.385129in}{3.147298in}}%
\pgfpathlineto{\pgfqpoint{4.385822in}{3.153512in}}%
\pgfpathlineto{\pgfqpoint{4.386547in}{3.151514in}}%
\pgfpathlineto{\pgfqpoint{4.386100in}{3.201147in}}%
\pgfpathlineto{\pgfqpoint{4.386624in}{3.172110in}}%
\pgfpathlineto{\pgfqpoint{4.386670in}{3.200863in}}%
\pgfpathlineto{\pgfqpoint{4.386917in}{3.150481in}}%
\pgfpathlineto{\pgfqpoint{4.387718in}{3.158723in}}%
\pgfpathlineto{\pgfqpoint{4.388782in}{3.208202in}}%
\pgfpathlineto{\pgfqpoint{4.388258in}{3.150451in}}%
\pgfpathlineto{\pgfqpoint{4.388813in}{3.164986in}}%
\pgfpathlineto{\pgfqpoint{4.389830in}{3.144785in}}%
\pgfpathlineto{\pgfqpoint{4.389352in}{3.215031in}}%
\pgfpathlineto{\pgfqpoint{4.389892in}{3.188472in}}%
\pgfpathlineto{\pgfqpoint{4.390493in}{3.218374in}}%
\pgfpathlineto{\pgfqpoint{4.390416in}{3.143728in}}%
\pgfpathlineto{\pgfqpoint{4.390955in}{3.161831in}}%
\pgfpathlineto{\pgfqpoint{4.391557in}{3.136756in}}%
\pgfpathlineto{\pgfqpoint{4.391063in}{3.219483in}}%
\pgfpathlineto{\pgfqpoint{4.392050in}{3.175379in}}%
\pgfpathlineto{\pgfqpoint{4.392713in}{3.134132in}}%
\pgfpathlineto{\pgfqpoint{4.392204in}{3.219976in}}%
\pgfpathlineto{\pgfqpoint{4.392774in}{3.211710in}}%
\pgfpathlineto{\pgfqpoint{4.392790in}{3.212477in}}%
\pgfpathlineto{\pgfqpoint{4.392821in}{3.165597in}}%
\pgfpathlineto{\pgfqpoint{4.393283in}{3.141275in}}%
\pgfpathlineto{\pgfqpoint{4.393360in}{3.216368in}}%
\pgfpathlineto{\pgfqpoint{4.393900in}{3.178691in}}%
\pgfpathlineto{\pgfqpoint{4.393930in}{3.213151in}}%
\pgfpathlineto{\pgfqpoint{4.394424in}{3.146675in}}%
\pgfpathlineto{\pgfqpoint{4.394979in}{3.149935in}}%
\pgfpathlineto{\pgfqpoint{4.394994in}{3.146470in}}%
\pgfpathlineto{\pgfqpoint{4.395071in}{3.206523in}}%
\pgfpathlineto{\pgfqpoint{4.396011in}{3.161706in}}%
\pgfpathlineto{\pgfqpoint{4.396782in}{3.210129in}}%
\pgfpathlineto{\pgfqpoint{4.396705in}{3.145110in}}%
\pgfpathlineto{\pgfqpoint{4.397106in}{3.161456in}}%
\pgfpathlineto{\pgfqpoint{4.397707in}{3.144951in}}%
\pgfpathlineto{\pgfqpoint{4.397353in}{3.214106in}}%
\pgfpathlineto{\pgfqpoint{4.398200in}{3.169848in}}%
\pgfpathlineto{\pgfqpoint{4.398493in}{3.213920in}}%
\pgfpathlineto{\pgfqpoint{4.398848in}{3.145023in}}%
\pgfpathlineto{\pgfqpoint{4.399279in}{3.163071in}}%
\pgfpathlineto{\pgfqpoint{4.399434in}{3.146928in}}%
\pgfpathlineto{\pgfqpoint{4.399634in}{3.212259in}}%
\pgfpathlineto{\pgfqpoint{4.400343in}{3.177783in}}%
\pgfpathlineto{\pgfqpoint{4.400775in}{3.207533in}}%
\pgfpathlineto{\pgfqpoint{4.400574in}{3.143513in}}%
\pgfpathlineto{\pgfqpoint{4.401437in}{3.172763in}}%
\pgfpathlineto{\pgfqpoint{4.401715in}{3.148476in}}%
\pgfpathlineto{\pgfqpoint{4.401931in}{3.207184in}}%
\pgfpathlineto{\pgfqpoint{4.402470in}{3.181172in}}%
\pgfpathlineto{\pgfqpoint{4.402501in}{3.206177in}}%
\pgfpathlineto{\pgfqpoint{4.403426in}{3.147497in}}%
\pgfpathlineto{\pgfqpoint{4.403549in}{3.157634in}}%
\pgfpathlineto{\pgfqpoint{4.403996in}{3.143399in}}%
\pgfpathlineto{\pgfqpoint{4.404228in}{3.207172in}}%
\pgfpathlineto{\pgfqpoint{4.404490in}{3.173458in}}%
\pgfpathlineto{\pgfqpoint{4.404798in}{3.207171in}}%
\pgfpathlineto{\pgfqpoint{4.405152in}{3.138243in}}%
\pgfpathlineto{\pgfqpoint{4.405569in}{3.169137in}}%
\pgfpathlineto{\pgfqpoint{4.406309in}{3.136775in}}%
\pgfpathlineto{\pgfqpoint{4.405954in}{3.208893in}}%
\pgfpathlineto{\pgfqpoint{4.406524in}{3.205411in}}%
\pgfpathlineto{\pgfqpoint{4.407110in}{3.211257in}}%
\pgfpathlineto{\pgfqpoint{4.406894in}{3.141407in}}%
\pgfpathlineto{\pgfqpoint{4.407449in}{3.155403in}}%
\pgfpathlineto{\pgfqpoint{4.407480in}{3.140872in}}%
\pgfpathlineto{\pgfqpoint{4.408266in}{3.209766in}}%
\pgfpathlineto{\pgfqpoint{4.408528in}{3.171107in}}%
\pgfpathlineto{\pgfqpoint{4.408636in}{3.141566in}}%
\pgfpathlineto{\pgfqpoint{4.408852in}{3.213839in}}%
\pgfpathlineto{\pgfqpoint{4.409407in}{3.189234in}}%
\pgfpathlineto{\pgfqpoint{4.410008in}{3.214678in}}%
\pgfpathlineto{\pgfqpoint{4.410378in}{3.146219in}}%
\pgfpathlineto{\pgfqpoint{4.410501in}{3.188635in}}%
\pgfpathlineto{\pgfqpoint{4.411411in}{3.147199in}}%
\pgfpathlineto{\pgfqpoint{4.410594in}{3.210744in}}%
\pgfpathlineto{\pgfqpoint{4.411642in}{3.175002in}}%
\pgfpathlineto{\pgfqpoint{4.412336in}{3.205320in}}%
\pgfpathlineto{\pgfqpoint{4.412567in}{3.140941in}}%
\pgfpathlineto{\pgfqpoint{4.412737in}{3.161435in}}%
\pgfpathlineto{\pgfqpoint{4.413507in}{3.203833in}}%
\pgfpathlineto{\pgfqpoint{4.413723in}{3.140077in}}%
\pgfpathlineto{\pgfqpoint{4.413846in}{3.162059in}}%
\pgfpathlineto{\pgfqpoint{4.414309in}{3.141825in}}%
\pgfpathlineto{\pgfqpoint{4.414093in}{3.200475in}}%
\pgfpathlineto{\pgfqpoint{4.414941in}{3.164808in}}%
\pgfpathlineto{\pgfqpoint{4.415835in}{3.198603in}}%
\pgfpathlineto{\pgfqpoint{4.415465in}{3.143311in}}%
\pgfpathlineto{\pgfqpoint{4.416004in}{3.150487in}}%
\pgfpathlineto{\pgfqpoint{4.416051in}{3.141712in}}%
\pgfpathlineto{\pgfqpoint{4.416421in}{3.196450in}}%
\pgfpathlineto{\pgfqpoint{4.416960in}{3.182078in}}%
\pgfpathlineto{\pgfqpoint{4.417577in}{3.196095in}}%
\pgfpathlineto{\pgfqpoint{4.417700in}{3.138826in}}%
\pgfpathlineto{\pgfqpoint{4.418024in}{3.181900in}}%
\pgfpathlineto{\pgfqpoint{4.418286in}{3.136272in}}%
\pgfpathlineto{\pgfqpoint{4.418163in}{3.193210in}}%
\pgfpathlineto{\pgfqpoint{4.419134in}{3.180067in}}%
\pgfpathlineto{\pgfqpoint{4.419334in}{3.189172in}}%
\pgfpathlineto{\pgfqpoint{4.419534in}{3.138628in}}%
\pgfpathlineto{\pgfqpoint{4.420105in}{3.141830in}}%
\pgfpathlineto{\pgfqpoint{4.420691in}{3.139559in}}%
\pgfpathlineto{\pgfqpoint{4.420490in}{3.185854in}}%
\pgfpathlineto{\pgfqpoint{4.420814in}{3.179642in}}%
\pgfpathlineto{\pgfqpoint{4.421415in}{3.189421in}}%
\pgfpathlineto{\pgfqpoint{4.421261in}{3.141223in}}%
\pgfpathlineto{\pgfqpoint{4.421739in}{3.157608in}}%
\pgfpathlineto{\pgfqpoint{4.421831in}{3.141447in}}%
\pgfpathlineto{\pgfqpoint{4.422001in}{3.190806in}}%
\pgfpathlineto{\pgfqpoint{4.422556in}{3.189920in}}%
\pgfpathlineto{\pgfqpoint{4.423157in}{3.199026in}}%
\pgfpathlineto{\pgfqpoint{4.422972in}{3.142370in}}%
\pgfpathlineto{\pgfqpoint{4.423527in}{3.147396in}}%
\pgfpathlineto{\pgfqpoint{4.424128in}{3.140012in}}%
\pgfpathlineto{\pgfqpoint{4.423743in}{3.198741in}}%
\pgfpathlineto{\pgfqpoint{4.424575in}{3.167455in}}%
\pgfpathlineto{\pgfqpoint{4.424591in}{3.167608in}}%
\pgfpathlineto{\pgfqpoint{4.424621in}{3.160061in}}%
\pgfpathlineto{\pgfqpoint{4.424652in}{3.165389in}}%
\pgfpathlineto{\pgfqpoint{4.424698in}{3.137048in}}%
\pgfpathlineto{\pgfqpoint{4.424899in}{3.192822in}}%
\pgfpathlineto{\pgfqpoint{4.425747in}{3.167242in}}%
\pgfpathlineto{\pgfqpoint{4.426656in}{3.192066in}}%
\pgfpathlineto{\pgfqpoint{4.425870in}{3.139421in}}%
\pgfpathlineto{\pgfqpoint{4.426826in}{3.169657in}}%
\pgfpathlineto{\pgfqpoint{4.427011in}{3.147021in}}%
\pgfpathlineto{\pgfqpoint{4.427242in}{3.194421in}}%
\pgfpathlineto{\pgfqpoint{4.427874in}{3.184233in}}%
\pgfpathlineto{\pgfqpoint{4.428398in}{3.193148in}}%
\pgfpathlineto{\pgfqpoint{4.428768in}{3.150281in}}%
\pgfpathlineto{\pgfqpoint{4.429693in}{3.148053in}}%
\pgfpathlineto{\pgfqpoint{4.428984in}{3.192777in}}%
\pgfpathlineto{\pgfqpoint{4.429785in}{3.160063in}}%
\pgfpathlineto{\pgfqpoint{4.430109in}{3.193545in}}%
\pgfpathlineto{\pgfqpoint{4.429862in}{3.146004in}}%
\pgfpathlineto{\pgfqpoint{4.430849in}{3.148626in}}%
\pgfpathlineto{\pgfqpoint{4.431435in}{3.146163in}}%
\pgfpathlineto{\pgfqpoint{4.431311in}{3.192642in}}%
\pgfpathlineto{\pgfqpoint{4.431697in}{3.167020in}}%
\pgfpathlineto{\pgfqpoint{4.431882in}{3.192887in}}%
\pgfpathlineto{\pgfqpoint{4.432591in}{3.145795in}}%
\pgfpathlineto{\pgfqpoint{4.432791in}{3.157287in}}%
\pgfpathlineto{\pgfqpoint{4.432976in}{3.194456in}}%
\pgfpathlineto{\pgfqpoint{4.433747in}{3.145201in}}%
\pgfpathlineto{\pgfqpoint{4.433901in}{3.160697in}}%
\pgfpathlineto{\pgfqpoint{4.434502in}{3.142662in}}%
\pgfpathlineto{\pgfqpoint{4.434132in}{3.193882in}}%
\pgfpathlineto{\pgfqpoint{4.434996in}{3.154536in}}%
\pgfpathlineto{\pgfqpoint{4.435304in}{3.194766in}}%
\pgfpathlineto{\pgfqpoint{4.435658in}{3.140495in}}%
\pgfpathlineto{\pgfqpoint{4.436105in}{3.159518in}}%
\pgfpathlineto{\pgfqpoint{4.436244in}{3.141696in}}%
\pgfpathlineto{\pgfqpoint{4.437030in}{3.197486in}}%
\pgfpathlineto{\pgfqpoint{4.437215in}{3.153133in}}%
\pgfpathlineto{\pgfqpoint{4.437616in}{3.195829in}}%
\pgfpathlineto{\pgfqpoint{4.437385in}{3.141163in}}%
\pgfpathlineto{\pgfqpoint{4.438341in}{3.158251in}}%
\pgfpathlineto{\pgfqpoint{4.438541in}{3.144736in}}%
\pgfpathlineto{\pgfqpoint{4.439065in}{3.189669in}}%
\pgfpathlineto{\pgfqpoint{4.439358in}{3.185678in}}%
\pgfpathlineto{\pgfqpoint{4.439820in}{3.189523in}}%
\pgfpathlineto{\pgfqpoint{4.439682in}{3.150158in}}%
\pgfpathlineto{\pgfqpoint{4.440144in}{3.153951in}}%
\pgfpathlineto{\pgfqpoint{4.440175in}{3.159668in}}%
\pgfpathlineto{\pgfqpoint{4.440406in}{3.191716in}}%
\pgfpathlineto{\pgfqpoint{4.441223in}{3.148909in}}%
\pgfpathlineto{\pgfqpoint{4.441269in}{3.160054in}}%
\pgfpathlineto{\pgfqpoint{4.442379in}{3.145263in}}%
\pgfpathlineto{\pgfqpoint{4.441562in}{3.192456in}}%
\pgfpathlineto{\pgfqpoint{4.442395in}{3.152729in}}%
\pgfpathlineto{\pgfqpoint{4.443073in}{3.193964in}}%
\pgfpathlineto{\pgfqpoint{4.442549in}{3.143137in}}%
\pgfpathlineto{\pgfqpoint{4.443505in}{3.156802in}}%
\pgfpathlineto{\pgfqpoint{4.443705in}{3.144126in}}%
\pgfpathlineto{\pgfqpoint{4.444229in}{3.201610in}}%
\pgfpathlineto{\pgfqpoint{4.444584in}{3.161644in}}%
\pgfpathlineto{\pgfqpoint{4.445370in}{3.207368in}}%
\pgfpathlineto{\pgfqpoint{4.444861in}{3.150075in}}%
\pgfpathlineto{\pgfqpoint{4.445709in}{3.164181in}}%
\pgfpathlineto{\pgfqpoint{4.446449in}{3.149203in}}%
\pgfpathlineto{\pgfqpoint{4.446526in}{3.206746in}}%
\pgfpathlineto{\pgfqpoint{4.446772in}{3.171048in}}%
\pgfpathlineto{\pgfqpoint{4.447096in}{3.205283in}}%
\pgfpathlineto{\pgfqpoint{4.447019in}{3.146875in}}%
\pgfpathlineto{\pgfqpoint{4.447867in}{3.174088in}}%
\pgfpathlineto{\pgfqpoint{4.448175in}{3.151782in}}%
\pgfpathlineto{\pgfqpoint{4.448823in}{3.204788in}}%
\pgfpathlineto{\pgfqpoint{4.448961in}{3.166754in}}%
\pgfpathlineto{\pgfqpoint{4.449393in}{3.204643in}}%
\pgfpathlineto{\pgfqpoint{4.449316in}{3.154953in}}%
\pgfpathlineto{\pgfqpoint{4.450071in}{3.174964in}}%
\pgfpathlineto{\pgfqpoint{4.450888in}{3.154039in}}%
\pgfpathlineto{\pgfqpoint{4.450534in}{3.203392in}}%
\pgfpathlineto{\pgfqpoint{4.451104in}{3.201398in}}%
\pgfpathlineto{\pgfqpoint{4.452029in}{3.148082in}}%
\pgfpathlineto{\pgfqpoint{4.451674in}{3.204587in}}%
\pgfpathlineto{\pgfqpoint{4.452229in}{3.183706in}}%
\pgfpathlineto{\pgfqpoint{4.452831in}{3.210936in}}%
\pgfpathlineto{\pgfqpoint{4.453185in}{3.146481in}}%
\pgfpathlineto{\pgfqpoint{4.453308in}{3.157442in}}%
\pgfpathlineto{\pgfqpoint{4.453771in}{3.145616in}}%
\pgfpathlineto{\pgfqpoint{4.453401in}{3.210922in}}%
\pgfpathlineto{\pgfqpoint{4.453940in}{3.175152in}}%
\pgfpathlineto{\pgfqpoint{4.454557in}{3.208803in}}%
\pgfpathlineto{\pgfqpoint{4.454912in}{3.143453in}}%
\pgfpathlineto{\pgfqpoint{4.455035in}{3.158742in}}%
\pgfpathlineto{\pgfqpoint{4.456068in}{3.142102in}}%
\pgfpathlineto{\pgfqpoint{4.455698in}{3.206092in}}%
\pgfpathlineto{\pgfqpoint{4.456145in}{3.158178in}}%
\pgfpathlineto{\pgfqpoint{4.456268in}{3.203953in}}%
\pgfpathlineto{\pgfqpoint{4.456638in}{3.138795in}}%
\pgfpathlineto{\pgfqpoint{4.457178in}{3.153040in}}%
\pgfpathlineto{\pgfqpoint{4.457208in}{3.138649in}}%
\pgfpathlineto{\pgfqpoint{4.457994in}{3.204409in}}%
\pgfpathlineto{\pgfqpoint{4.458257in}{3.170458in}}%
\pgfpathlineto{\pgfqpoint{4.458935in}{3.141948in}}%
\pgfpathlineto{\pgfqpoint{4.458565in}{3.204226in}}%
\pgfpathlineto{\pgfqpoint{4.459120in}{3.181667in}}%
\pgfpathlineto{\pgfqpoint{4.459721in}{3.209596in}}%
\pgfpathlineto{\pgfqpoint{4.459521in}{3.142367in}}%
\pgfpathlineto{\pgfqpoint{4.460199in}{3.153699in}}%
\pgfpathlineto{\pgfqpoint{4.460877in}{3.210855in}}%
\pgfpathlineto{\pgfqpoint{4.460754in}{3.144621in}}%
\pgfpathlineto{\pgfqpoint{4.461293in}{3.157072in}}%
\pgfpathlineto{\pgfqpoint{4.461817in}{3.143421in}}%
\pgfpathlineto{\pgfqpoint{4.461447in}{3.212713in}}%
\pgfpathlineto{\pgfqpoint{4.462403in}{3.145148in}}%
\pgfpathlineto{\pgfqpoint{4.463189in}{3.217429in}}%
\pgfpathlineto{\pgfqpoint{4.462480in}{3.143038in}}%
\pgfpathlineto{\pgfqpoint{4.463513in}{3.158418in}}%
\pgfpathlineto{\pgfqpoint{4.463636in}{3.144929in}}%
\pgfpathlineto{\pgfqpoint{4.464345in}{3.217377in}}%
\pgfpathlineto{\pgfqpoint{4.464592in}{3.174684in}}%
\pgfpathlineto{\pgfqpoint{4.465502in}{3.214066in}}%
\pgfpathlineto{\pgfqpoint{4.464792in}{3.143492in}}%
\pgfpathlineto{\pgfqpoint{4.465671in}{3.176161in}}%
\pgfpathlineto{\pgfqpoint{4.466457in}{3.144059in}}%
\pgfpathlineto{\pgfqpoint{4.466087in}{3.210146in}}%
\pgfpathlineto{\pgfqpoint{4.466781in}{3.166710in}}%
\pgfpathlineto{\pgfqpoint{4.467243in}{3.207597in}}%
\pgfpathlineto{\pgfqpoint{4.467613in}{3.144801in}}%
\pgfpathlineto{\pgfqpoint{4.467906in}{3.176056in}}%
\pgfpathlineto{\pgfqpoint{4.468199in}{3.146585in}}%
\pgfpathlineto{\pgfqpoint{4.468415in}{3.204600in}}%
\pgfpathlineto{\pgfqpoint{4.468985in}{3.200461in}}%
\pgfpathlineto{\pgfqpoint{4.469001in}{3.202478in}}%
\pgfpathlineto{\pgfqpoint{4.469355in}{3.147652in}}%
\pgfpathlineto{\pgfqpoint{4.469756in}{3.159550in}}%
\pgfpathlineto{\pgfqpoint{4.470388in}{3.146281in}}%
\pgfpathlineto{\pgfqpoint{4.470157in}{3.201767in}}%
\pgfpathlineto{\pgfqpoint{4.470866in}{3.153339in}}%
\pgfpathlineto{\pgfqpoint{4.471328in}{3.199088in}}%
\pgfpathlineto{\pgfqpoint{4.471529in}{3.143692in}}%
\pgfpathlineto{\pgfqpoint{4.471991in}{3.160365in}}%
\pgfpathlineto{\pgfqpoint{4.472700in}{3.145781in}}%
\pgfpathlineto{\pgfqpoint{4.472978in}{3.199783in}}%
\pgfpathlineto{\pgfqpoint{4.473039in}{3.185913in}}%
\pgfpathlineto{\pgfqpoint{4.473563in}{3.200128in}}%
\pgfpathlineto{\pgfqpoint{4.473856in}{3.145521in}}%
\pgfpathlineto{\pgfqpoint{4.474165in}{3.191190in}}%
\pgfpathlineto{\pgfqpoint{4.474442in}{3.142740in}}%
\pgfpathlineto{\pgfqpoint{4.474735in}{3.197803in}}%
\pgfpathlineto{\pgfqpoint{4.475290in}{3.178634in}}%
\pgfpathlineto{\pgfqpoint{4.475891in}{3.194415in}}%
\pgfpathlineto{\pgfqpoint{4.475598in}{3.144601in}}%
\pgfpathlineto{\pgfqpoint{4.476400in}{3.186414in}}%
\pgfpathlineto{\pgfqpoint{4.477294in}{3.147258in}}%
\pgfpathlineto{\pgfqpoint{4.477063in}{3.196134in}}%
\pgfpathlineto{\pgfqpoint{4.477540in}{3.180128in}}%
\pgfpathlineto{\pgfqpoint{4.478650in}{3.194686in}}%
\pgfpathlineto{\pgfqpoint{4.478450in}{3.142822in}}%
\pgfpathlineto{\pgfqpoint{4.478666in}{3.186633in}}%
\pgfpathlineto{\pgfqpoint{4.479606in}{3.141214in}}%
\pgfpathlineto{\pgfqpoint{4.479221in}{3.194265in}}%
\pgfpathlineto{\pgfqpoint{4.479760in}{3.179756in}}%
\pgfpathlineto{\pgfqpoint{4.479806in}{3.196612in}}%
\pgfpathlineto{\pgfqpoint{4.480192in}{3.138939in}}%
\pgfpathlineto{\pgfqpoint{4.480839in}{3.166501in}}%
\pgfpathlineto{\pgfqpoint{4.481348in}{3.143826in}}%
\pgfpathlineto{\pgfqpoint{4.480978in}{3.196831in}}%
\pgfpathlineto{\pgfqpoint{4.481533in}{3.195106in}}%
\pgfpathlineto{\pgfqpoint{4.482150in}{3.200680in}}%
\pgfpathlineto{\pgfqpoint{4.481934in}{3.145133in}}%
\pgfpathlineto{\pgfqpoint{4.482442in}{3.160180in}}%
\pgfpathlineto{\pgfqpoint{4.483105in}{3.149757in}}%
\pgfpathlineto{\pgfqpoint{4.483306in}{3.202630in}}%
\pgfpathlineto{\pgfqpoint{4.483537in}{3.160061in}}%
\pgfpathlineto{\pgfqpoint{4.484477in}{3.200301in}}%
\pgfpathlineto{\pgfqpoint{4.484200in}{3.149207in}}%
\pgfpathlineto{\pgfqpoint{4.484585in}{3.166406in}}%
\pgfpathlineto{\pgfqpoint{4.485202in}{3.143454in}}%
\pgfpathlineto{\pgfqpoint{4.485063in}{3.200612in}}%
\pgfpathlineto{\pgfqpoint{4.485664in}{3.190609in}}%
\pgfpathlineto{\pgfqpoint{4.485772in}{3.142285in}}%
\pgfpathlineto{\pgfqpoint{4.486219in}{3.200755in}}%
\pgfpathlineto{\pgfqpoint{4.486774in}{3.190732in}}%
\pgfpathlineto{\pgfqpoint{4.486805in}{3.196934in}}%
\pgfpathlineto{\pgfqpoint{4.487514in}{3.142343in}}%
\pgfpathlineto{\pgfqpoint{4.487761in}{3.163567in}}%
\pgfpathlineto{\pgfqpoint{4.487961in}{3.192997in}}%
\pgfpathlineto{\pgfqpoint{4.488100in}{3.141473in}}%
\pgfpathlineto{\pgfqpoint{4.488639in}{3.166595in}}%
\pgfpathlineto{\pgfqpoint{4.489256in}{3.139049in}}%
\pgfpathlineto{\pgfqpoint{4.489086in}{3.189931in}}%
\pgfpathlineto{\pgfqpoint{4.489734in}{3.172590in}}%
\pgfpathlineto{\pgfqpoint{4.490412in}{3.141619in}}%
\pgfpathlineto{\pgfqpoint{4.490242in}{3.190573in}}%
\pgfpathlineto{\pgfqpoint{4.490782in}{3.182235in}}%
\pgfpathlineto{\pgfqpoint{4.491707in}{3.192874in}}%
\pgfpathlineto{\pgfqpoint{4.491568in}{3.142026in}}%
\pgfpathlineto{\pgfqpoint{4.491892in}{3.183459in}}%
\pgfpathlineto{\pgfqpoint{4.492801in}{3.136510in}}%
\pgfpathlineto{\pgfqpoint{4.492863in}{3.192289in}}%
\pgfpathlineto{\pgfqpoint{4.492986in}{3.178394in}}%
\pgfpathlineto{\pgfqpoint{4.494019in}{3.196306in}}%
\pgfpathlineto{\pgfqpoint{4.493387in}{3.134117in}}%
\pgfpathlineto{\pgfqpoint{4.494034in}{3.181388in}}%
\pgfpathlineto{\pgfqpoint{4.494543in}{3.136986in}}%
\pgfpathlineto{\pgfqpoint{4.494759in}{3.200270in}}%
\pgfpathlineto{\pgfqpoint{4.495144in}{3.155945in}}%
\pgfpathlineto{\pgfqpoint{4.495915in}{3.209568in}}%
\pgfpathlineto{\pgfqpoint{4.495699in}{3.143594in}}%
\pgfpathlineto{\pgfqpoint{4.496239in}{3.162643in}}%
\pgfpathlineto{\pgfqpoint{4.497271in}{3.142658in}}%
\pgfpathlineto{\pgfqpoint{4.496331in}{3.208686in}}%
\pgfpathlineto{\pgfqpoint{4.497349in}{3.153735in}}%
\pgfpathlineto{\pgfqpoint{4.497487in}{3.205389in}}%
\pgfpathlineto{\pgfqpoint{4.498011in}{3.141340in}}%
\pgfpathlineto{\pgfqpoint{4.498412in}{3.147668in}}%
\pgfpathlineto{\pgfqpoint{4.499152in}{3.140895in}}%
\pgfpathlineto{\pgfqpoint{4.498798in}{3.202016in}}%
\pgfpathlineto{\pgfqpoint{4.499352in}{3.183825in}}%
\pgfpathlineto{\pgfqpoint{4.499954in}{3.201258in}}%
\pgfpathlineto{\pgfqpoint{4.499738in}{3.144765in}}%
\pgfpathlineto{\pgfqpoint{4.500447in}{3.168837in}}%
\pgfpathlineto{\pgfqpoint{4.501511in}{3.201986in}}%
\pgfpathlineto{\pgfqpoint{4.501372in}{3.148767in}}%
\pgfpathlineto{\pgfqpoint{4.501526in}{3.187155in}}%
\pgfpathlineto{\pgfqpoint{4.502528in}{3.144799in}}%
\pgfpathlineto{\pgfqpoint{4.502081in}{3.200107in}}%
\pgfpathlineto{\pgfqpoint{4.502636in}{3.171918in}}%
\pgfpathlineto{\pgfqpoint{4.503237in}{3.203378in}}%
\pgfpathlineto{\pgfqpoint{4.503114in}{3.143689in}}%
\pgfpathlineto{\pgfqpoint{4.503730in}{3.161530in}}%
\pgfpathlineto{\pgfqpoint{4.503807in}{3.204771in}}%
\pgfpathlineto{\pgfqpoint{4.504439in}{3.146271in}}%
\pgfpathlineto{\pgfqpoint{4.504809in}{3.169335in}}%
\pgfpathlineto{\pgfqpoint{4.505010in}{3.147321in}}%
\pgfpathlineto{\pgfqpoint{4.505534in}{3.214023in}}%
\pgfpathlineto{\pgfqpoint{4.505919in}{3.168907in}}%
\pgfpathlineto{\pgfqpoint{4.506690in}{3.213883in}}%
\pgfpathlineto{\pgfqpoint{4.506027in}{3.148227in}}%
\pgfpathlineto{\pgfqpoint{4.506998in}{3.168283in}}%
\pgfpathlineto{\pgfqpoint{4.507892in}{3.149352in}}%
\pgfpathlineto{\pgfqpoint{4.507260in}{3.216621in}}%
\pgfpathlineto{\pgfqpoint{4.508093in}{3.168077in}}%
\pgfpathlineto{\pgfqpoint{4.508987in}{3.220617in}}%
\pgfpathlineto{\pgfqpoint{4.508463in}{3.153570in}}%
\pgfpathlineto{\pgfqpoint{4.509187in}{3.173202in}}%
\pgfpathlineto{\pgfqpoint{4.509804in}{3.154852in}}%
\pgfpathlineto{\pgfqpoint{4.510143in}{3.218689in}}%
\pgfpathlineto{\pgfqpoint{4.510282in}{3.164480in}}%
\pgfpathlineto{\pgfqpoint{4.510713in}{3.210958in}}%
\pgfpathlineto{\pgfqpoint{4.510944in}{3.150987in}}%
\pgfpathlineto{\pgfqpoint{4.511391in}{3.164307in}}%
\pgfpathlineto{\pgfqpoint{4.511530in}{3.149958in}}%
\pgfpathlineto{\pgfqpoint{4.511469in}{3.177614in}}%
\pgfpathlineto{\pgfqpoint{4.511838in}{3.175290in}}%
\pgfpathlineto{\pgfqpoint{4.512440in}{3.203677in}}%
\pgfpathlineto{\pgfqpoint{4.512101in}{3.152964in}}%
\pgfpathlineto{\pgfqpoint{4.512933in}{3.165685in}}%
\pgfpathlineto{\pgfqpoint{4.513365in}{3.153751in}}%
\pgfpathlineto{\pgfqpoint{4.513010in}{3.203075in}}%
\pgfpathlineto{\pgfqpoint{4.513565in}{3.181167in}}%
\pgfpathlineto{\pgfqpoint{4.514166in}{3.208964in}}%
\pgfpathlineto{\pgfqpoint{4.514397in}{3.155376in}}%
\pgfpathlineto{\pgfqpoint{4.514659in}{3.179723in}}%
\pgfpathlineto{\pgfqpoint{4.515538in}{3.149509in}}%
\pgfpathlineto{\pgfqpoint{4.515322in}{3.210053in}}%
\pgfpathlineto{\pgfqpoint{4.515785in}{3.162880in}}%
\pgfpathlineto{\pgfqpoint{4.516478in}{3.210378in}}%
\pgfpathlineto{\pgfqpoint{4.516694in}{3.146792in}}%
\pgfpathlineto{\pgfqpoint{4.516895in}{3.181313in}}%
\pgfpathlineto{\pgfqpoint{4.517280in}{3.146154in}}%
\pgfpathlineto{\pgfqpoint{4.517064in}{3.209230in}}%
\pgfpathlineto{\pgfqpoint{4.518020in}{3.166669in}}%
\pgfpathlineto{\pgfqpoint{4.518220in}{3.208283in}}%
\pgfpathlineto{\pgfqpoint{4.518436in}{3.147021in}}%
\pgfpathlineto{\pgfqpoint{4.519145in}{3.173750in}}%
\pgfpathlineto{\pgfqpoint{4.520178in}{3.145477in}}%
\pgfpathlineto{\pgfqpoint{4.519376in}{3.203533in}}%
\pgfpathlineto{\pgfqpoint{4.520255in}{3.172810in}}%
\pgfpathlineto{\pgfqpoint{4.521026in}{3.199075in}}%
\pgfpathlineto{\pgfqpoint{4.521226in}{3.140841in}}%
\pgfpathlineto{\pgfqpoint{4.521303in}{3.160620in}}%
\pgfpathlineto{\pgfqpoint{4.522382in}{3.134592in}}%
\pgfpathlineto{\pgfqpoint{4.521611in}{3.199318in}}%
\pgfpathlineto{\pgfqpoint{4.522413in}{3.144839in}}%
\pgfpathlineto{\pgfqpoint{4.523353in}{3.204059in}}%
\pgfpathlineto{\pgfqpoint{4.522968in}{3.131550in}}%
\pgfpathlineto{\pgfqpoint{4.523523in}{3.146223in}}%
\pgfpathlineto{\pgfqpoint{4.524124in}{3.131577in}}%
\pgfpathlineto{\pgfqpoint{4.524340in}{3.208638in}}%
\pgfpathlineto{\pgfqpoint{4.524587in}{3.162844in}}%
\pgfpathlineto{\pgfqpoint{4.524926in}{3.210406in}}%
\pgfpathlineto{\pgfqpoint{4.524710in}{3.128732in}}%
\pgfpathlineto{\pgfqpoint{4.525712in}{3.173723in}}%
\pgfpathlineto{\pgfqpoint{4.525881in}{3.130677in}}%
\pgfpathlineto{\pgfqpoint{4.526683in}{3.216445in}}%
\pgfpathlineto{\pgfqpoint{4.526822in}{3.169851in}}%
\pgfpathlineto{\pgfqpoint{4.527839in}{3.222352in}}%
\pgfpathlineto{\pgfqpoint{4.527053in}{3.132931in}}%
\pgfpathlineto{\pgfqpoint{4.527947in}{3.190717in}}%
\pgfpathlineto{\pgfqpoint{4.528224in}{3.135515in}}%
\pgfpathlineto{\pgfqpoint{4.528425in}{3.220937in}}%
\pgfpathlineto{\pgfqpoint{4.528980in}{3.215031in}}%
\pgfpathlineto{\pgfqpoint{4.528995in}{3.223137in}}%
\pgfpathlineto{\pgfqpoint{4.529396in}{3.138858in}}%
\pgfpathlineto{\pgfqpoint{4.529951in}{3.146094in}}%
\pgfpathlineto{\pgfqpoint{4.529966in}{3.139449in}}%
\pgfpathlineto{\pgfqpoint{4.530167in}{3.219306in}}%
\pgfpathlineto{\pgfqpoint{4.531030in}{3.152469in}}%
\pgfpathlineto{\pgfqpoint{4.531338in}{3.218000in}}%
\pgfpathlineto{\pgfqpoint{4.531153in}{3.141971in}}%
\pgfpathlineto{\pgfqpoint{4.532140in}{3.154438in}}%
\pgfpathlineto{\pgfqpoint{4.532664in}{3.140052in}}%
\pgfpathlineto{\pgfqpoint{4.532510in}{3.217221in}}%
\pgfpathlineto{\pgfqpoint{4.533080in}{3.214642in}}%
\pgfpathlineto{\pgfqpoint{4.533096in}{3.217977in}}%
\pgfpathlineto{\pgfqpoint{4.533250in}{3.135459in}}%
\pgfpathlineto{\pgfqpoint{4.533820in}{3.140398in}}%
\pgfpathlineto{\pgfqpoint{4.534421in}{3.136437in}}%
\pgfpathlineto{\pgfqpoint{4.534267in}{3.218476in}}%
\pgfpathlineto{\pgfqpoint{4.534745in}{3.187991in}}%
\pgfpathlineto{\pgfqpoint{4.534776in}{3.187691in}}%
\pgfpathlineto{\pgfqpoint{4.534791in}{3.193485in}}%
\pgfpathlineto{\pgfqpoint{4.535423in}{3.222349in}}%
\pgfpathlineto{\pgfqpoint{4.534992in}{3.134447in}}%
\pgfpathlineto{\pgfqpoint{4.535793in}{3.161420in}}%
\pgfpathlineto{\pgfqpoint{4.536749in}{3.131816in}}%
\pgfpathlineto{\pgfqpoint{4.536579in}{3.219747in}}%
\pgfpathlineto{\pgfqpoint{4.536872in}{3.171892in}}%
\pgfpathlineto{\pgfqpoint{4.537766in}{3.216510in}}%
\pgfpathlineto{\pgfqpoint{4.537335in}{3.127295in}}%
\pgfpathlineto{\pgfqpoint{4.537890in}{3.144683in}}%
\pgfpathlineto{\pgfqpoint{4.538491in}{3.121433in}}%
\pgfpathlineto{\pgfqpoint{4.538352in}{3.212337in}}%
\pgfpathlineto{\pgfqpoint{4.538922in}{3.209007in}}%
\pgfpathlineto{\pgfqpoint{4.538938in}{3.210851in}}%
\pgfpathlineto{\pgfqpoint{4.539076in}{3.118668in}}%
\pgfpathlineto{\pgfqpoint{4.539570in}{3.161953in}}%
\pgfpathlineto{\pgfqpoint{4.539662in}{3.115743in}}%
\pgfpathlineto{\pgfqpoint{4.540094in}{3.206389in}}%
\pgfpathlineto{\pgfqpoint{4.540603in}{3.193818in}}%
\pgfpathlineto{\pgfqpoint{4.540680in}{3.203970in}}%
\pgfpathlineto{\pgfqpoint{4.540818in}{3.115605in}}%
\pgfpathlineto{\pgfqpoint{4.541342in}{3.155312in}}%
\pgfpathlineto{\pgfqpoint{4.541404in}{3.116003in}}%
\pgfpathlineto{\pgfqpoint{4.541836in}{3.199355in}}%
\pgfpathlineto{\pgfqpoint{4.542360in}{3.193614in}}%
\pgfpathlineto{\pgfqpoint{4.542406in}{3.198373in}}%
\pgfpathlineto{\pgfqpoint{4.542560in}{3.118494in}}%
\pgfpathlineto{\pgfqpoint{4.543069in}{3.157249in}}%
\pgfpathlineto{\pgfqpoint{4.543732in}{3.116856in}}%
\pgfpathlineto{\pgfqpoint{4.543285in}{3.195830in}}%
\pgfpathlineto{\pgfqpoint{4.544086in}{3.188312in}}%
\pgfpathlineto{\pgfqpoint{4.544441in}{3.197318in}}%
\pgfpathlineto{\pgfqpoint{4.544302in}{3.119198in}}%
\pgfpathlineto{\pgfqpoint{4.544826in}{3.148585in}}%
\pgfpathlineto{\pgfqpoint{4.544888in}{3.116925in}}%
\pgfpathlineto{\pgfqpoint{4.545597in}{3.200225in}}%
\pgfpathlineto{\pgfqpoint{4.545751in}{3.178315in}}%
\pgfpathlineto{\pgfqpoint{4.546753in}{3.205431in}}%
\pgfpathlineto{\pgfqpoint{4.546044in}{3.119974in}}%
\pgfpathlineto{\pgfqpoint{4.546861in}{3.183265in}}%
\pgfpathlineto{\pgfqpoint{4.547786in}{3.123308in}}%
\pgfpathlineto{\pgfqpoint{4.547339in}{3.210781in}}%
\pgfpathlineto{\pgfqpoint{4.547894in}{3.177407in}}%
\pgfpathlineto{\pgfqpoint{4.548495in}{3.210439in}}%
\pgfpathlineto{\pgfqpoint{4.548942in}{3.120789in}}%
\pgfpathlineto{\pgfqpoint{4.548988in}{3.152767in}}%
\pgfpathlineto{\pgfqpoint{4.549528in}{3.122409in}}%
\pgfpathlineto{\pgfqpoint{4.549651in}{3.210670in}}%
\pgfpathlineto{\pgfqpoint{4.550114in}{3.131036in}}%
\pgfpathlineto{\pgfqpoint{4.550237in}{3.216735in}}%
\pgfpathlineto{\pgfqpoint{4.550684in}{3.127827in}}%
\pgfpathlineto{\pgfqpoint{4.551223in}{3.157919in}}%
\pgfpathlineto{\pgfqpoint{4.551840in}{3.132342in}}%
\pgfpathlineto{\pgfqpoint{4.551393in}{3.220878in}}%
\pgfpathlineto{\pgfqpoint{4.552333in}{3.138289in}}%
\pgfpathlineto{\pgfqpoint{4.552549in}{3.218364in}}%
\pgfpathlineto{\pgfqpoint{4.552410in}{3.133554in}}%
\pgfpathlineto{\pgfqpoint{4.553459in}{3.161500in}}%
\pgfpathlineto{\pgfqpoint{4.554229in}{3.130260in}}%
\pgfpathlineto{\pgfqpoint{4.553705in}{3.216511in}}%
\pgfpathlineto{\pgfqpoint{4.554538in}{3.168190in}}%
\pgfpathlineto{\pgfqpoint{4.554861in}{3.216709in}}%
\pgfpathlineto{\pgfqpoint{4.554800in}{3.130658in}}%
\pgfpathlineto{\pgfqpoint{4.555647in}{3.171997in}}%
\pgfpathlineto{\pgfqpoint{4.556541in}{3.136329in}}%
\pgfpathlineto{\pgfqpoint{4.556017in}{3.218449in}}%
\pgfpathlineto{\pgfqpoint{4.556726in}{3.183642in}}%
\pgfpathlineto{\pgfqpoint{4.557174in}{3.213178in}}%
\pgfpathlineto{\pgfqpoint{4.557698in}{3.138966in}}%
\pgfpathlineto{\pgfqpoint{4.557806in}{3.160462in}}%
\pgfpathlineto{\pgfqpoint{4.558268in}{3.139042in}}%
\pgfpathlineto{\pgfqpoint{4.557929in}{3.204106in}}%
\pgfpathlineto{\pgfqpoint{4.558484in}{3.194361in}}%
\pgfpathlineto{\pgfqpoint{4.558900in}{3.201776in}}%
\pgfpathlineto{\pgfqpoint{4.559332in}{3.134852in}}%
\pgfpathlineto{\pgfqpoint{4.559517in}{3.159456in}}%
\pgfpathlineto{\pgfqpoint{4.559902in}{3.136415in}}%
\pgfpathlineto{\pgfqpoint{4.559655in}{3.199045in}}%
\pgfpathlineto{\pgfqpoint{4.560041in}{3.193479in}}%
\pgfpathlineto{\pgfqpoint{4.560626in}{3.206371in}}%
\pgfpathlineto{\pgfqpoint{4.560488in}{3.135100in}}%
\pgfpathlineto{\pgfqpoint{4.561043in}{3.159705in}}%
\pgfpathlineto{\pgfqpoint{4.561644in}{3.134104in}}%
\pgfpathlineto{\pgfqpoint{4.561783in}{3.207212in}}%
\pgfpathlineto{\pgfqpoint{4.562152in}{3.160991in}}%
\pgfpathlineto{\pgfqpoint{4.562939in}{3.207196in}}%
\pgfpathlineto{\pgfqpoint{4.562985in}{3.131628in}}%
\pgfpathlineto{\pgfqpoint{4.563262in}{3.178528in}}%
\pgfpathlineto{\pgfqpoint{4.564141in}{3.129677in}}%
\pgfpathlineto{\pgfqpoint{4.564095in}{3.206186in}}%
\pgfpathlineto{\pgfqpoint{4.564341in}{3.187990in}}%
\pgfpathlineto{\pgfqpoint{4.564665in}{3.213753in}}%
\pgfpathlineto{\pgfqpoint{4.564727in}{3.130804in}}%
\pgfpathlineto{\pgfqpoint{4.565436in}{3.180645in}}%
\pgfpathlineto{\pgfqpoint{4.566484in}{3.126427in}}%
\pgfpathlineto{\pgfqpoint{4.565821in}{3.217063in}}%
\pgfpathlineto{\pgfqpoint{4.566546in}{3.176699in}}%
\pgfpathlineto{\pgfqpoint{4.566977in}{3.216535in}}%
\pgfpathlineto{\pgfqpoint{4.567070in}{3.125822in}}%
\pgfpathlineto{\pgfqpoint{4.567579in}{3.180171in}}%
\pgfpathlineto{\pgfqpoint{4.568226in}{3.125315in}}%
\pgfpathlineto{\pgfqpoint{4.568133in}{3.220021in}}%
\pgfpathlineto{\pgfqpoint{4.568673in}{3.160112in}}%
\pgfpathlineto{\pgfqpoint{4.569290in}{3.227636in}}%
\pgfpathlineto{\pgfqpoint{4.568812in}{3.126197in}}%
\pgfpathlineto{\pgfqpoint{4.569783in}{3.184079in}}%
\pgfpathlineto{\pgfqpoint{4.570554in}{3.129257in}}%
\pgfpathlineto{\pgfqpoint{4.570184in}{3.223435in}}%
\pgfpathlineto{\pgfqpoint{4.570908in}{3.154795in}}%
\pgfpathlineto{\pgfqpoint{4.571031in}{3.216412in}}%
\pgfpathlineto{\pgfqpoint{4.571725in}{3.124673in}}%
\pgfpathlineto{\pgfqpoint{4.572018in}{3.160564in}}%
\pgfpathlineto{\pgfqpoint{4.572280in}{3.122962in}}%
\pgfpathlineto{\pgfqpoint{4.573066in}{3.217519in}}%
\pgfpathlineto{\pgfqpoint{4.573082in}{3.217531in}}%
\pgfpathlineto{\pgfqpoint{4.573467in}{3.128556in}}%
\pgfpathlineto{\pgfqpoint{4.573652in}{3.223027in}}%
\pgfpathlineto{\pgfqpoint{4.574191in}{3.202979in}}%
\pgfpathlineto{\pgfqpoint{4.574238in}{3.226193in}}%
\pgfpathlineto{\pgfqpoint{4.575178in}{3.130134in}}%
\pgfpathlineto{\pgfqpoint{4.575286in}{3.186159in}}%
\pgfpathlineto{\pgfqpoint{4.575764in}{3.128743in}}%
\pgfpathlineto{\pgfqpoint{4.575378in}{3.226093in}}%
\pgfpathlineto{\pgfqpoint{4.576396in}{3.156982in}}%
\pgfpathlineto{\pgfqpoint{4.577120in}{3.229845in}}%
\pgfpathlineto{\pgfqpoint{4.576920in}{3.130417in}}%
\pgfpathlineto{\pgfqpoint{4.577490in}{3.138847in}}%
\pgfpathlineto{\pgfqpoint{4.577506in}{3.133979in}}%
\pgfpathlineto{\pgfqpoint{4.578276in}{3.235128in}}%
\pgfpathlineto{\pgfqpoint{4.578523in}{3.146055in}}%
\pgfpathlineto{\pgfqpoint{4.578862in}{3.235333in}}%
\pgfpathlineto{\pgfqpoint{4.578616in}{3.135934in}}%
\pgfpathlineto{\pgfqpoint{4.579633in}{3.171414in}}%
\pgfpathlineto{\pgfqpoint{4.580357in}{3.133088in}}%
\pgfpathlineto{\pgfqpoint{4.580018in}{3.232927in}}%
\pgfpathlineto{\pgfqpoint{4.580758in}{3.139205in}}%
\pgfpathlineto{\pgfqpoint{4.581760in}{3.226434in}}%
\pgfpathlineto{\pgfqpoint{4.580943in}{3.131960in}}%
\pgfpathlineto{\pgfqpoint{4.581883in}{3.164831in}}%
\pgfpathlineto{\pgfqpoint{4.582099in}{3.135101in}}%
\pgfpathlineto{\pgfqpoint{4.582346in}{3.228860in}}%
\pgfpathlineto{\pgfqpoint{4.582870in}{3.189620in}}%
\pgfpathlineto{\pgfqpoint{4.582932in}{3.227619in}}%
\pgfpathlineto{\pgfqpoint{4.583749in}{3.132349in}}%
\pgfpathlineto{\pgfqpoint{4.583980in}{3.193534in}}%
\pgfpathlineto{\pgfqpoint{4.584920in}{3.127945in}}%
\pgfpathlineto{\pgfqpoint{4.584088in}{3.228257in}}%
\pgfpathlineto{\pgfqpoint{4.585105in}{3.178024in}}%
\pgfpathlineto{\pgfqpoint{4.585259in}{3.225505in}}%
\pgfpathlineto{\pgfqpoint{4.586076in}{3.115947in}}%
\pgfpathlineto{\pgfqpoint{4.586184in}{3.143604in}}%
\pgfpathlineto{\pgfqpoint{4.587001in}{3.228944in}}%
\pgfpathlineto{\pgfqpoint{4.587232in}{3.115868in}}%
\pgfpathlineto{\pgfqpoint{4.587309in}{3.157880in}}%
\pgfpathlineto{\pgfqpoint{4.588389in}{3.111922in}}%
\pgfpathlineto{\pgfqpoint{4.588157in}{3.231457in}}%
\pgfpathlineto{\pgfqpoint{4.588419in}{3.159604in}}%
\pgfpathlineto{\pgfqpoint{4.588743in}{3.231042in}}%
\pgfpathlineto{\pgfqpoint{4.588974in}{3.109199in}}%
\pgfpathlineto{\pgfqpoint{4.589514in}{3.143136in}}%
\pgfpathlineto{\pgfqpoint{4.589545in}{3.111964in}}%
\pgfpathlineto{\pgfqpoint{4.590485in}{3.237088in}}%
\pgfpathlineto{\pgfqpoint{4.590639in}{3.122589in}}%
\pgfpathlineto{\pgfqpoint{4.591071in}{3.238945in}}%
\pgfpathlineto{\pgfqpoint{4.590716in}{3.116327in}}%
\pgfpathlineto{\pgfqpoint{4.591764in}{3.151905in}}%
\pgfpathlineto{\pgfqpoint{4.591795in}{3.120501in}}%
\pgfpathlineto{\pgfqpoint{4.592227in}{3.239966in}}%
\pgfpathlineto{\pgfqpoint{4.592797in}{3.233246in}}%
\pgfpathlineto{\pgfqpoint{4.592813in}{3.237195in}}%
\pgfpathlineto{\pgfqpoint{4.592951in}{3.123777in}}%
\pgfpathlineto{\pgfqpoint{4.593506in}{3.148675in}}%
\pgfpathlineto{\pgfqpoint{4.593537in}{3.124003in}}%
\pgfpathlineto{\pgfqpoint{4.593969in}{3.228984in}}%
\pgfpathlineto{\pgfqpoint{4.594539in}{3.222985in}}%
\pgfpathlineto{\pgfqpoint{4.594554in}{3.228347in}}%
\pgfpathlineto{\pgfqpoint{4.595279in}{3.124157in}}%
\pgfpathlineto{\pgfqpoint{4.595433in}{3.165181in}}%
\pgfpathlineto{\pgfqpoint{4.596435in}{3.129717in}}%
\pgfpathlineto{\pgfqpoint{4.595711in}{3.221791in}}%
\pgfpathlineto{\pgfqpoint{4.596528in}{3.152483in}}%
\pgfpathlineto{\pgfqpoint{4.596882in}{3.211553in}}%
\pgfpathlineto{\pgfqpoint{4.597021in}{3.133531in}}%
\pgfpathlineto{\pgfqpoint{4.597622in}{3.136952in}}%
\pgfpathlineto{\pgfqpoint{4.598069in}{3.212487in}}%
\pgfpathlineto{\pgfqpoint{4.597668in}{3.136552in}}%
\pgfpathlineto{\pgfqpoint{4.598747in}{3.145960in}}%
\pgfpathlineto{\pgfqpoint{4.599410in}{3.134375in}}%
\pgfpathlineto{\pgfqpoint{4.599256in}{3.204639in}}%
\pgfpathlineto{\pgfqpoint{4.599533in}{3.195275in}}%
\pgfpathlineto{\pgfqpoint{4.600150in}{3.208906in}}%
\pgfpathlineto{\pgfqpoint{4.599980in}{3.123526in}}%
\pgfpathlineto{\pgfqpoint{4.600535in}{3.140505in}}%
\pgfpathlineto{\pgfqpoint{4.600566in}{3.118998in}}%
\pgfpathlineto{\pgfqpoint{4.601368in}{3.217814in}}%
\pgfpathlineto{\pgfqpoint{4.601630in}{3.152845in}}%
\pgfpathlineto{\pgfqpoint{4.601954in}{3.219354in}}%
\pgfpathlineto{\pgfqpoint{4.601722in}{3.118965in}}%
\pgfpathlineto{\pgfqpoint{4.602308in}{3.120218in}}%
\pgfpathlineto{\pgfqpoint{4.602894in}{3.112860in}}%
\pgfpathlineto{\pgfqpoint{4.603125in}{3.236224in}}%
\pgfpathlineto{\pgfqpoint{4.603202in}{3.197132in}}%
\pgfpathlineto{\pgfqpoint{4.603233in}{3.225292in}}%
\pgfpathlineto{\pgfqpoint{4.603464in}{3.111834in}}%
\pgfpathlineto{\pgfqpoint{4.604297in}{3.186451in}}%
\pgfpathlineto{\pgfqpoint{4.604805in}{3.218637in}}%
\pgfpathlineto{\pgfqpoint{4.604605in}{3.113779in}}%
\pgfpathlineto{\pgfqpoint{4.605021in}{3.154824in}}%
\pgfpathlineto{\pgfqpoint{4.605684in}{3.116304in}}%
\pgfpathlineto{\pgfqpoint{4.605407in}{3.217977in}}%
\pgfpathlineto{\pgfqpoint{4.605961in}{3.216593in}}%
\pgfpathlineto{\pgfqpoint{4.606069in}{3.237478in}}%
\pgfpathlineto{\pgfqpoint{4.606963in}{3.103816in}}%
\pgfpathlineto{\pgfqpoint{4.607010in}{3.150357in}}%
\pgfpathlineto{\pgfqpoint{4.607565in}{3.119320in}}%
\pgfpathlineto{\pgfqpoint{4.607164in}{3.236352in}}%
\pgfpathlineto{\pgfqpoint{4.607857in}{3.183229in}}%
\pgfpathlineto{\pgfqpoint{4.607919in}{3.222897in}}%
\pgfpathlineto{\pgfqpoint{4.608397in}{3.128002in}}%
\pgfpathlineto{\pgfqpoint{4.608921in}{3.148435in}}%
\pgfpathlineto{\pgfqpoint{4.609723in}{3.092971in}}%
\pgfpathlineto{\pgfqpoint{4.609476in}{3.259881in}}%
\pgfpathlineto{\pgfqpoint{4.609985in}{3.208029in}}%
\pgfpathlineto{\pgfqpoint{4.610786in}{3.273799in}}%
\pgfpathlineto{\pgfqpoint{4.610262in}{3.087587in}}%
\pgfpathlineto{\pgfqpoint{4.610848in}{3.152923in}}%
\pgfpathlineto{\pgfqpoint{4.611572in}{3.030061in}}%
\pgfpathlineto{\pgfqpoint{4.611249in}{3.267436in}}%
\pgfpathlineto{\pgfqpoint{4.611912in}{3.244075in}}%
\pgfpathlineto{\pgfqpoint{4.612235in}{3.049982in}}%
\pgfpathlineto{\pgfqpoint{4.612544in}{3.262342in}}%
\pgfpathlineto{\pgfqpoint{4.613068in}{3.174291in}}%
\pgfpathlineto{\pgfqpoint{4.614085in}{3.281215in}}%
\pgfpathlineto{\pgfqpoint{4.613669in}{3.078804in}}%
\pgfpathlineto{\pgfqpoint{4.614147in}{3.169596in}}%
\pgfpathlineto{\pgfqpoint{4.614995in}{3.034850in}}%
\pgfpathlineto{\pgfqpoint{4.614732in}{3.303909in}}%
\pgfpathlineto{\pgfqpoint{4.615195in}{3.258832in}}%
\pgfpathlineto{\pgfqpoint{4.615395in}{3.286727in}}%
\pgfpathlineto{\pgfqpoint{4.615488in}{3.115752in}}%
\pgfpathlineto{\pgfqpoint{4.615534in}{3.015821in}}%
\pgfpathlineto{\pgfqpoint{4.616490in}{3.295215in}}%
\pgfpathlineto{\pgfqpoint{4.616521in}{3.266706in}}%
\pgfpathlineto{\pgfqpoint{4.616613in}{3.272050in}}%
\pgfpathlineto{\pgfqpoint{4.616752in}{3.164795in}}%
\pgfpathlineto{\pgfqpoint{4.616844in}{3.005985in}}%
\pgfpathlineto{\pgfqpoint{4.617153in}{3.295604in}}%
\pgfpathlineto{\pgfqpoint{4.617800in}{3.289281in}}%
\pgfpathlineto{\pgfqpoint{4.617954in}{3.202450in}}%
\pgfpathlineto{\pgfqpoint{4.618833in}{3.021445in}}%
\pgfpathlineto{\pgfqpoint{4.618478in}{3.259341in}}%
\pgfpathlineto{\pgfqpoint{4.619079in}{3.167848in}}%
\pgfpathlineto{\pgfqpoint{4.619789in}{3.280942in}}%
\pgfpathlineto{\pgfqpoint{4.619480in}{3.033441in}}%
\pgfpathlineto{\pgfqpoint{4.620128in}{3.049215in}}%
\pgfpathlineto{\pgfqpoint{4.620806in}{3.004848in}}%
\pgfpathlineto{\pgfqpoint{4.621099in}{3.310657in}}%
\pgfpathlineto{\pgfqpoint{4.621453in}{3.010722in}}%
\pgfpathlineto{\pgfqpoint{4.622394in}{3.244356in}}%
\pgfpathlineto{\pgfqpoint{4.622424in}{3.302296in}}%
\pgfpathlineto{\pgfqpoint{4.622779in}{3.030048in}}%
\pgfpathlineto{\pgfqpoint{4.623411in}{3.078399in}}%
\pgfpathlineto{\pgfqpoint{4.623442in}{3.034394in}}%
\pgfpathlineto{\pgfqpoint{4.624382in}{3.277095in}}%
\pgfpathlineto{\pgfqpoint{4.624398in}{3.310508in}}%
\pgfpathlineto{\pgfqpoint{4.624752in}{3.016556in}}%
\pgfpathlineto{\pgfqpoint{4.625384in}{3.045090in}}%
\pgfpathlineto{\pgfqpoint{4.625415in}{2.992020in}}%
\pgfpathlineto{\pgfqpoint{4.625723in}{3.304621in}}%
\pgfpathlineto{\pgfqpoint{4.626355in}{3.257421in}}%
\pgfpathlineto{\pgfqpoint{4.626386in}{3.306252in}}%
\pgfpathlineto{\pgfqpoint{4.626741in}{3.023844in}}%
\pgfpathlineto{\pgfqpoint{4.627373in}{3.049529in}}%
\pgfpathlineto{\pgfqpoint{4.627388in}{3.023717in}}%
\pgfpathlineto{\pgfqpoint{4.627696in}{3.269122in}}%
\pgfpathlineto{\pgfqpoint{4.628344in}{3.259657in}}%
\pgfpathlineto{\pgfqpoint{4.629022in}{3.291351in}}%
\pgfpathlineto{\pgfqpoint{4.628698in}{3.047940in}}%
\pgfpathlineto{\pgfqpoint{4.629269in}{3.147239in}}%
\pgfpathlineto{\pgfqpoint{4.630024in}{2.994726in}}%
\pgfpathlineto{\pgfqpoint{4.629669in}{3.287411in}}%
\pgfpathlineto{\pgfqpoint{4.630301in}{3.245498in}}%
\pgfpathlineto{\pgfqpoint{4.630332in}{3.309317in}}%
\pgfpathlineto{\pgfqpoint{4.630687in}{2.998355in}}%
\pgfpathlineto{\pgfqpoint{4.631334in}{3.018013in}}%
\pgfpathlineto{\pgfqpoint{4.631643in}{3.320548in}}%
\pgfpathlineto{\pgfqpoint{4.631997in}{3.015655in}}%
\pgfpathlineto{\pgfqpoint{4.632567in}{3.116945in}}%
\pgfpathlineto{\pgfqpoint{4.632660in}{3.026243in}}%
\pgfpathlineto{\pgfqpoint{4.632968in}{3.274079in}}%
\pgfpathlineto{\pgfqpoint{4.633585in}{3.203287in}}%
\pgfpathlineto{\pgfqpoint{4.634278in}{3.316313in}}%
\pgfpathlineto{\pgfqpoint{4.634633in}{3.000595in}}%
\pgfpathlineto{\pgfqpoint{4.635589in}{3.318943in}}%
\pgfpathlineto{\pgfqpoint{4.635280in}{2.994001in}}%
\pgfpathlineto{\pgfqpoint{4.635835in}{3.168055in}}%
\pgfpathlineto{\pgfqpoint{4.635943in}{2.989696in}}%
\pgfpathlineto{\pgfqpoint{4.636252in}{3.315154in}}%
\pgfpathlineto{\pgfqpoint{4.636899in}{3.314555in}}%
\pgfpathlineto{\pgfqpoint{4.637254in}{3.005800in}}%
\pgfpathlineto{\pgfqpoint{4.638194in}{3.251604in}}%
\pgfpathlineto{\pgfqpoint{4.638888in}{3.315495in}}%
\pgfpathlineto{\pgfqpoint{4.638564in}{2.993163in}}%
\pgfpathlineto{\pgfqpoint{4.639211in}{3.010256in}}%
\pgfpathlineto{\pgfqpoint{4.639889in}{2.974896in}}%
\pgfpathlineto{\pgfqpoint{4.639535in}{3.332612in}}%
\pgfpathlineto{\pgfqpoint{4.640152in}{3.241956in}}%
\pgfpathlineto{\pgfqpoint{4.640182in}{3.340431in}}%
\pgfpathlineto{\pgfqpoint{4.640537in}{2.977085in}}%
\pgfpathlineto{\pgfqpoint{4.641200in}{2.984980in}}%
\pgfpathlineto{\pgfqpoint{4.641508in}{3.326102in}}%
\pgfpathlineto{\pgfqpoint{4.642402in}{3.204253in}}%
\pgfpathlineto{\pgfqpoint{4.643173in}{2.998764in}}%
\pgfpathlineto{\pgfqpoint{4.643466in}{3.325982in}}%
\pgfpathlineto{\pgfqpoint{4.643728in}{3.163070in}}%
\pgfpathlineto{\pgfqpoint{4.644483in}{2.970788in}}%
\pgfpathlineto{\pgfqpoint{4.644129in}{3.341364in}}%
\pgfpathlineto{\pgfqpoint{4.644761in}{3.285475in}}%
\pgfpathlineto{\pgfqpoint{4.645439in}{3.355797in}}%
\pgfpathlineto{\pgfqpoint{4.645131in}{2.958514in}}%
\pgfpathlineto{\pgfqpoint{4.645778in}{2.976020in}}%
\pgfpathlineto{\pgfqpoint{4.645793in}{2.954537in}}%
\pgfpathlineto{\pgfqpoint{4.646102in}{3.343546in}}%
\pgfpathlineto{\pgfqpoint{4.646703in}{3.176630in}}%
\pgfpathlineto{\pgfqpoint{4.646749in}{3.342458in}}%
\pgfpathlineto{\pgfqpoint{4.647104in}{2.980070in}}%
\pgfpathlineto{\pgfqpoint{4.647751in}{2.998223in}}%
\pgfpathlineto{\pgfqpoint{4.648414in}{2.980576in}}%
\pgfpathlineto{\pgfqpoint{4.648059in}{3.318637in}}%
\pgfpathlineto{\pgfqpoint{4.648676in}{3.163360in}}%
\pgfpathlineto{\pgfqpoint{4.649385in}{3.348794in}}%
\pgfpathlineto{\pgfqpoint{4.649724in}{2.967777in}}%
\pgfpathlineto{\pgfqpoint{4.649740in}{2.968663in}}%
\pgfpathlineto{\pgfqpoint{4.650032in}{3.362402in}}%
\pgfpathlineto{\pgfqpoint{4.650387in}{2.952559in}}%
\pgfpathlineto{\pgfqpoint{4.650926in}{3.208938in}}%
\pgfpathlineto{\pgfqpoint{4.651050in}{2.959473in}}%
\pgfpathlineto{\pgfqpoint{4.651343in}{3.356575in}}%
\pgfpathlineto{\pgfqpoint{4.651990in}{3.333984in}}%
\pgfpathlineto{\pgfqpoint{4.652006in}{3.350424in}}%
\pgfpathlineto{\pgfqpoint{4.652360in}{2.968213in}}%
\pgfpathlineto{\pgfqpoint{4.652900in}{3.195682in}}%
\pgfpathlineto{\pgfqpoint{4.653670in}{2.963494in}}%
\pgfpathlineto{\pgfqpoint{4.653963in}{3.357397in}}%
\pgfpathlineto{\pgfqpoint{4.654626in}{3.383619in}}%
\pgfpathlineto{\pgfqpoint{4.654318in}{2.950296in}}%
\pgfpathlineto{\pgfqpoint{4.654857in}{3.232517in}}%
\pgfpathlineto{\pgfqpoint{4.655628in}{2.936080in}}%
\pgfpathlineto{\pgfqpoint{4.655289in}{3.387834in}}%
\pgfpathlineto{\pgfqpoint{4.655921in}{3.358588in}}%
\pgfpathlineto{\pgfqpoint{4.656599in}{3.395786in}}%
\pgfpathlineto{\pgfqpoint{4.656291in}{2.939468in}}%
\pgfpathlineto{\pgfqpoint{4.656877in}{3.044133in}}%
\pgfpathlineto{\pgfqpoint{4.656938in}{2.940396in}}%
\pgfpathlineto{\pgfqpoint{4.657262in}{3.390375in}}%
\pgfpathlineto{\pgfqpoint{4.657863in}{3.197526in}}%
\pgfpathlineto{\pgfqpoint{4.657909in}{3.386636in}}%
\pgfpathlineto{\pgfqpoint{4.658911in}{2.926486in}}%
\pgfpathlineto{\pgfqpoint{4.658942in}{3.023475in}}%
\pgfpathlineto{\pgfqpoint{4.659883in}{3.412954in}}%
\pgfpathlineto{\pgfqpoint{4.659559in}{2.908284in}}%
\pgfpathlineto{\pgfqpoint{4.660098in}{3.224595in}}%
\pgfpathlineto{\pgfqpoint{4.660869in}{2.892265in}}%
\pgfpathlineto{\pgfqpoint{4.660530in}{3.414297in}}%
\pgfpathlineto{\pgfqpoint{4.661162in}{3.364535in}}%
\pgfpathlineto{\pgfqpoint{4.661193in}{3.420464in}}%
\pgfpathlineto{\pgfqpoint{4.661532in}{2.898881in}}%
\pgfpathlineto{\pgfqpoint{4.662133in}{3.009394in}}%
\pgfpathlineto{\pgfqpoint{4.662842in}{2.900607in}}%
\pgfpathlineto{\pgfqpoint{4.662503in}{3.411023in}}%
\pgfpathlineto{\pgfqpoint{4.663104in}{3.185111in}}%
\pgfpathlineto{\pgfqpoint{4.663813in}{3.406135in}}%
\pgfpathlineto{\pgfqpoint{4.664152in}{2.881006in}}%
\pgfpathlineto{\pgfqpoint{4.664168in}{2.907931in}}%
\pgfpathlineto{\pgfqpoint{4.665124in}{3.445478in}}%
\pgfpathlineto{\pgfqpoint{4.664800in}{2.874586in}}%
\pgfpathlineto{\pgfqpoint{4.665339in}{3.240413in}}%
\pgfpathlineto{\pgfqpoint{4.666126in}{2.853564in}}%
\pgfpathlineto{\pgfqpoint{4.665786in}{3.445656in}}%
\pgfpathlineto{\pgfqpoint{4.666403in}{3.320523in}}%
\pgfpathlineto{\pgfqpoint{4.666434in}{3.437263in}}%
\pgfpathlineto{\pgfqpoint{4.666773in}{2.876579in}}%
\pgfpathlineto{\pgfqpoint{4.667420in}{2.902478in}}%
\pgfpathlineto{\pgfqpoint{4.668099in}{2.882471in}}%
\pgfpathlineto{\pgfqpoint{4.667759in}{3.420104in}}%
\pgfpathlineto{\pgfqpoint{4.668253in}{3.146258in}}%
\pgfpathlineto{\pgfqpoint{4.669070in}{3.450770in}}%
\pgfpathlineto{\pgfqpoint{4.668746in}{2.855439in}}%
\pgfpathlineto{\pgfqpoint{4.669332in}{3.040240in}}%
\pgfpathlineto{\pgfqpoint{4.670056in}{2.832133in}}%
\pgfpathlineto{\pgfqpoint{4.669717in}{3.474979in}}%
\pgfpathlineto{\pgfqpoint{4.670334in}{3.294816in}}%
\pgfpathlineto{\pgfqpoint{4.670380in}{3.477445in}}%
\pgfpathlineto{\pgfqpoint{4.670719in}{2.829678in}}%
\pgfpathlineto{\pgfqpoint{4.671367in}{2.845471in}}%
\pgfpathlineto{\pgfqpoint{4.671552in}{3.284663in}}%
\pgfpathlineto{\pgfqpoint{4.671690in}{3.449026in}}%
\pgfpathlineto{\pgfqpoint{4.672029in}{2.857068in}}%
\pgfpathlineto{\pgfqpoint{4.672584in}{3.146780in}}%
\pgfpathlineto{\pgfqpoint{4.673340in}{2.829121in}}%
\pgfpathlineto{\pgfqpoint{4.673001in}{3.444241in}}%
\pgfpathlineto{\pgfqpoint{4.673617in}{3.304397in}}%
\pgfpathlineto{\pgfqpoint{4.674311in}{3.503088in}}%
\pgfpathlineto{\pgfqpoint{4.674650in}{2.801300in}}%
\pgfpathlineto{\pgfqpoint{4.675621in}{3.518841in}}%
\pgfpathlineto{\pgfqpoint{4.675313in}{2.796788in}}%
\pgfpathlineto{\pgfqpoint{4.675899in}{2.973783in}}%
\pgfpathlineto{\pgfqpoint{4.675960in}{2.808412in}}%
\pgfpathlineto{\pgfqpoint{4.676268in}{3.493752in}}%
\pgfpathlineto{\pgfqpoint{4.676808in}{3.309127in}}%
\pgfpathlineto{\pgfqpoint{4.677933in}{2.815052in}}%
\pgfpathlineto{\pgfqpoint{4.676931in}{3.492312in}}%
\pgfpathlineto{\pgfqpoint{4.678087in}{3.132698in}}%
\pgfpathlineto{\pgfqpoint{4.678904in}{3.519250in}}%
\pgfpathlineto{\pgfqpoint{4.678581in}{2.794681in}}%
\pgfpathlineto{\pgfqpoint{4.679166in}{3.011033in}}%
\pgfpathlineto{\pgfqpoint{4.679891in}{2.768335in}}%
\pgfpathlineto{\pgfqpoint{4.679552in}{3.555085in}}%
\pgfpathlineto{\pgfqpoint{4.680168in}{3.301057in}}%
\pgfpathlineto{\pgfqpoint{4.680215in}{3.555402in}}%
\pgfpathlineto{\pgfqpoint{4.680554in}{2.770018in}}%
\pgfpathlineto{\pgfqpoint{4.681232in}{2.885858in}}%
\pgfpathlineto{\pgfqpoint{4.681525in}{3.535743in}}%
\pgfpathlineto{\pgfqpoint{4.681864in}{2.806998in}}%
\pgfpathlineto{\pgfqpoint{4.682388in}{3.227180in}}%
\pgfpathlineto{\pgfqpoint{4.683174in}{2.782141in}}%
\pgfpathlineto{\pgfqpoint{4.682835in}{3.538885in}}%
\pgfpathlineto{\pgfqpoint{4.683452in}{3.325011in}}%
\pgfpathlineto{\pgfqpoint{4.684145in}{3.583271in}}%
\pgfpathlineto{\pgfqpoint{4.684485in}{2.750036in}}%
\pgfpathlineto{\pgfqpoint{4.684515in}{2.898570in}}%
\pgfpathlineto{\pgfqpoint{4.685456in}{3.589738in}}%
\pgfpathlineto{\pgfqpoint{4.685132in}{2.752277in}}%
\pgfpathlineto{\pgfqpoint{4.685672in}{3.176637in}}%
\pgfpathlineto{\pgfqpoint{4.685795in}{2.751487in}}%
\pgfpathlineto{\pgfqpoint{4.686103in}{3.568705in}}%
\pgfpathlineto{\pgfqpoint{4.686720in}{3.242309in}}%
\pgfpathlineto{\pgfqpoint{4.686766in}{3.579724in}}%
\pgfpathlineto{\pgfqpoint{4.687753in}{2.758867in}}%
\pgfpathlineto{\pgfqpoint{4.687799in}{2.949322in}}%
\pgfpathlineto{\pgfqpoint{4.688739in}{3.587809in}}%
\pgfpathlineto{\pgfqpoint{4.688415in}{2.737333in}}%
\pgfpathlineto{\pgfqpoint{4.688955in}{3.155708in}}%
\pgfpathlineto{\pgfqpoint{4.689726in}{2.720194in}}%
\pgfpathlineto{\pgfqpoint{4.689386in}{3.623023in}}%
\pgfpathlineto{\pgfqpoint{4.690003in}{3.362427in}}%
\pgfpathlineto{\pgfqpoint{4.690034in}{3.600498in}}%
\pgfpathlineto{\pgfqpoint{4.690373in}{2.724595in}}%
\pgfpathlineto{\pgfqpoint{4.691067in}{2.909036in}}%
\pgfpathlineto{\pgfqpoint{4.691344in}{3.592692in}}%
\pgfpathlineto{\pgfqpoint{4.691683in}{2.734237in}}%
\pgfpathlineto{\pgfqpoint{4.692207in}{3.243166in}}%
\pgfpathlineto{\pgfqpoint{4.692994in}{2.705009in}}%
\pgfpathlineto{\pgfqpoint{4.692654in}{3.585890in}}%
\pgfpathlineto{\pgfqpoint{4.693271in}{3.312317in}}%
\pgfpathlineto{\pgfqpoint{4.693965in}{3.620418in}}%
\pgfpathlineto{\pgfqpoint{4.694304in}{2.693137in}}%
\pgfpathlineto{\pgfqpoint{4.694335in}{2.839816in}}%
\pgfpathlineto{\pgfqpoint{4.694612in}{3.619439in}}%
\pgfpathlineto{\pgfqpoint{4.694951in}{2.692895in}}%
\pgfpathlineto{\pgfqpoint{4.695475in}{3.233316in}}%
\pgfpathlineto{\pgfqpoint{4.695614in}{2.696232in}}%
\pgfpathlineto{\pgfqpoint{4.695922in}{3.603332in}}%
\pgfpathlineto{\pgfqpoint{4.696539in}{3.309383in}}%
\pgfpathlineto{\pgfqpoint{4.697233in}{3.605568in}}%
\pgfpathlineto{\pgfqpoint{4.697572in}{2.685496in}}%
\pgfpathlineto{\pgfqpoint{4.697587in}{2.721424in}}%
\pgfpathlineto{\pgfqpoint{4.698543in}{3.601521in}}%
\pgfpathlineto{\pgfqpoint{4.698219in}{2.679879in}}%
\pgfpathlineto{\pgfqpoint{4.698759in}{3.157240in}}%
\pgfpathlineto{\pgfqpoint{4.699529in}{2.658743in}}%
\pgfpathlineto{\pgfqpoint{4.699190in}{3.611227in}}%
\pgfpathlineto{\pgfqpoint{4.699807in}{3.365461in}}%
\pgfpathlineto{\pgfqpoint{4.699838in}{3.606694in}}%
\pgfpathlineto{\pgfqpoint{4.700840in}{2.655008in}}%
\pgfpathlineto{\pgfqpoint{4.701148in}{3.585403in}}%
\pgfpathlineto{\pgfqpoint{4.702057in}{3.040427in}}%
\pgfpathlineto{\pgfqpoint{4.702797in}{2.662806in}}%
\pgfpathlineto{\pgfqpoint{4.702458in}{3.556390in}}%
\pgfpathlineto{\pgfqpoint{4.703059in}{3.252762in}}%
\pgfpathlineto{\pgfqpoint{4.703106in}{3.553420in}}%
\pgfpathlineto{\pgfqpoint{4.703460in}{2.661049in}}%
\pgfpathlineto{\pgfqpoint{4.704092in}{2.690400in}}%
\pgfpathlineto{\pgfqpoint{4.704108in}{2.643113in}}%
\pgfpathlineto{\pgfqpoint{4.704416in}{3.551943in}}%
\pgfpathlineto{\pgfqpoint{4.705002in}{3.229697in}}%
\pgfpathlineto{\pgfqpoint{4.705742in}{3.547769in}}%
\pgfpathlineto{\pgfqpoint{4.705418in}{2.627079in}}%
\pgfpathlineto{\pgfqpoint{4.706004in}{2.968143in}}%
\pgfpathlineto{\pgfqpoint{4.706065in}{2.634462in}}%
\pgfpathlineto{\pgfqpoint{4.706404in}{3.548710in}}%
\pgfpathlineto{\pgfqpoint{4.707006in}{3.442316in}}%
\pgfpathlineto{\pgfqpoint{4.707699in}{3.552744in}}%
\pgfpathlineto{\pgfqpoint{4.707376in}{2.646520in}}%
\pgfpathlineto{\pgfqpoint{4.707977in}{2.903351in}}%
\pgfpathlineto{\pgfqpoint{4.708023in}{2.643373in}}%
\pgfpathlineto{\pgfqpoint{4.708362in}{3.573657in}}%
\pgfpathlineto{\pgfqpoint{4.708994in}{3.549399in}}%
\pgfpathlineto{\pgfqpoint{4.709672in}{3.591132in}}%
\pgfpathlineto{\pgfqpoint{4.709333in}{2.644998in}}%
\pgfpathlineto{\pgfqpoint{4.709888in}{3.151455in}}%
\pgfpathlineto{\pgfqpoint{4.709981in}{2.643999in}}%
\pgfpathlineto{\pgfqpoint{4.710320in}{3.596294in}}%
\pgfpathlineto{\pgfqpoint{4.710921in}{3.369257in}}%
\pgfpathlineto{\pgfqpoint{4.710967in}{3.591736in}}%
\pgfpathlineto{\pgfqpoint{4.711291in}{2.644063in}}%
\pgfpathlineto{\pgfqpoint{4.711923in}{2.733226in}}%
\pgfpathlineto{\pgfqpoint{4.712601in}{2.646762in}}%
\pgfpathlineto{\pgfqpoint{4.712278in}{3.582254in}}%
\pgfpathlineto{\pgfqpoint{4.712879in}{3.324436in}}%
\pgfpathlineto{\pgfqpoint{4.713588in}{3.590721in}}%
\pgfpathlineto{\pgfqpoint{4.713264in}{2.648996in}}%
\pgfpathlineto{\pgfqpoint{4.713896in}{2.663392in}}%
\pgfpathlineto{\pgfqpoint{4.713911in}{2.629195in}}%
\pgfpathlineto{\pgfqpoint{4.714235in}{3.597999in}}%
\pgfpathlineto{\pgfqpoint{4.714744in}{3.152058in}}%
\pgfpathlineto{\pgfqpoint{4.714898in}{3.604324in}}%
\pgfpathlineto{\pgfqpoint{4.715222in}{2.638522in}}%
\pgfpathlineto{\pgfqpoint{4.715808in}{3.006647in}}%
\pgfpathlineto{\pgfqpoint{4.715869in}{2.622642in}}%
\pgfpathlineto{\pgfqpoint{4.716193in}{3.609455in}}%
\pgfpathlineto{\pgfqpoint{4.716825in}{3.520254in}}%
\pgfpathlineto{\pgfqpoint{4.717503in}{3.609484in}}%
\pgfpathlineto{\pgfqpoint{4.717179in}{2.629482in}}%
\pgfpathlineto{\pgfqpoint{4.717781in}{2.936676in}}%
\pgfpathlineto{\pgfqpoint{4.718490in}{2.615998in}}%
\pgfpathlineto{\pgfqpoint{4.718151in}{3.601976in}}%
\pgfpathlineto{\pgfqpoint{4.718798in}{3.590656in}}%
\pgfpathlineto{\pgfqpoint{4.719461in}{3.618377in}}%
\pgfpathlineto{\pgfqpoint{4.719137in}{2.607732in}}%
\pgfpathlineto{\pgfqpoint{4.719677in}{3.237056in}}%
\pgfpathlineto{\pgfqpoint{4.720447in}{2.592542in}}%
\pgfpathlineto{\pgfqpoint{4.720124in}{3.615139in}}%
\pgfpathlineto{\pgfqpoint{4.720725in}{3.384979in}}%
\pgfpathlineto{\pgfqpoint{4.721419in}{3.623203in}}%
\pgfpathlineto{\pgfqpoint{4.721095in}{2.599504in}}%
\pgfpathlineto{\pgfqpoint{4.721727in}{2.696251in}}%
\pgfpathlineto{\pgfqpoint{4.722405in}{2.601052in}}%
\pgfpathlineto{\pgfqpoint{4.722081in}{3.627090in}}%
\pgfpathlineto{\pgfqpoint{4.722683in}{3.372507in}}%
\pgfpathlineto{\pgfqpoint{4.723376in}{3.634865in}}%
\pgfpathlineto{\pgfqpoint{4.723052in}{2.610036in}}%
\pgfpathlineto{\pgfqpoint{4.723700in}{2.614835in}}%
\pgfpathlineto{\pgfqpoint{4.723715in}{2.596927in}}%
\pgfpathlineto{\pgfqpoint{4.723993in}{3.425823in}}%
\pgfpathlineto{\pgfqpoint{4.724686in}{3.641289in}}%
\pgfpathlineto{\pgfqpoint{4.724363in}{2.589185in}}%
\pgfpathlineto{\pgfqpoint{4.724995in}{2.650302in}}%
\pgfpathlineto{\pgfqpoint{4.725010in}{2.588487in}}%
\pgfpathlineto{\pgfqpoint{4.725334in}{3.637285in}}%
\pgfpathlineto{\pgfqpoint{4.725935in}{3.306204in}}%
\pgfpathlineto{\pgfqpoint{4.726644in}{3.636277in}}%
\pgfpathlineto{\pgfqpoint{4.726320in}{2.585213in}}%
\pgfpathlineto{\pgfqpoint{4.726968in}{2.602173in}}%
\pgfpathlineto{\pgfqpoint{4.727292in}{3.637347in}}%
\pgfpathlineto{\pgfqpoint{4.728232in}{2.852153in}}%
\pgfpathlineto{\pgfqpoint{4.728926in}{2.601135in}}%
\pgfpathlineto{\pgfqpoint{4.729249in}{3.648261in}}%
\pgfpathlineto{\pgfqpoint{4.730236in}{2.600567in}}%
\pgfpathlineto{\pgfqpoint{4.729897in}{3.649463in}}%
\pgfpathlineto{\pgfqpoint{4.730498in}{3.277658in}}%
\pgfpathlineto{\pgfqpoint{4.731207in}{3.649007in}}%
\pgfpathlineto{\pgfqpoint{4.730883in}{2.599755in}}%
\pgfpathlineto{\pgfqpoint{4.731531in}{2.613834in}}%
\pgfpathlineto{\pgfqpoint{4.731854in}{3.650371in}}%
\pgfpathlineto{\pgfqpoint{4.732795in}{2.866120in}}%
\pgfpathlineto{\pgfqpoint{4.732841in}{2.615200in}}%
\pgfpathlineto{\pgfqpoint{4.733812in}{3.655907in}}%
\pgfpathlineto{\pgfqpoint{4.734891in}{2.633882in}}%
\pgfpathlineto{\pgfqpoint{4.734459in}{3.658036in}}%
\pgfpathlineto{\pgfqpoint{4.735061in}{3.259649in}}%
\pgfpathlineto{\pgfqpoint{4.735770in}{3.670167in}}%
\pgfpathlineto{\pgfqpoint{4.735539in}{2.627365in}}%
\pgfpathlineto{\pgfqpoint{4.736093in}{2.654935in}}%
\pgfpathlineto{\pgfqpoint{4.736186in}{2.630768in}}%
\pgfpathlineto{\pgfqpoint{4.736232in}{2.765301in}}%
\pgfpathlineto{\pgfqpoint{4.736417in}{3.665285in}}%
\pgfpathlineto{\pgfqpoint{4.736849in}{2.632628in}}%
\pgfpathlineto{\pgfqpoint{4.737357in}{2.885348in}}%
\pgfpathlineto{\pgfqpoint{4.738144in}{2.615913in}}%
\pgfpathlineto{\pgfqpoint{4.737727in}{3.668282in}}%
\pgfpathlineto{\pgfqpoint{4.738359in}{3.600657in}}%
\pgfpathlineto{\pgfqpoint{4.738375in}{3.663272in}}%
\pgfpathlineto{\pgfqpoint{4.738791in}{2.625994in}}%
\pgfpathlineto{\pgfqpoint{4.739300in}{3.002435in}}%
\pgfpathlineto{\pgfqpoint{4.740101in}{2.592335in}}%
\pgfpathlineto{\pgfqpoint{4.739685in}{3.651383in}}%
\pgfpathlineto{\pgfqpoint{4.740317in}{3.626769in}}%
\pgfpathlineto{\pgfqpoint{4.740333in}{3.654682in}}%
\pgfpathlineto{\pgfqpoint{4.740749in}{2.590514in}}%
\pgfpathlineto{\pgfqpoint{4.741211in}{3.310352in}}%
\pgfpathlineto{\pgfqpoint{4.741396in}{2.589710in}}%
\pgfpathlineto{\pgfqpoint{4.741766in}{3.653190in}}%
\pgfpathlineto{\pgfqpoint{4.742259in}{3.552556in}}%
\pgfpathlineto{\pgfqpoint{4.743061in}{3.654738in}}%
\pgfpathlineto{\pgfqpoint{4.742706in}{2.591448in}}%
\pgfpathlineto{\pgfqpoint{4.743231in}{2.846515in}}%
\pgfpathlineto{\pgfqpoint{4.743354in}{2.592919in}}%
\pgfpathlineto{\pgfqpoint{4.743708in}{3.657593in}}%
\pgfpathlineto{\pgfqpoint{4.744202in}{3.465311in}}%
\pgfpathlineto{\pgfqpoint{4.744356in}{3.655859in}}%
\pgfpathlineto{\pgfqpoint{4.744649in}{2.606107in}}%
\pgfpathlineto{\pgfqpoint{4.745219in}{2.729597in}}%
\pgfpathlineto{\pgfqpoint{4.745312in}{2.592877in}}%
\pgfpathlineto{\pgfqpoint{4.745512in}{3.545817in}}%
\pgfpathlineto{\pgfqpoint{4.746313in}{3.644035in}}%
\pgfpathlineto{\pgfqpoint{4.745959in}{2.588986in}}%
\pgfpathlineto{\pgfqpoint{4.746468in}{2.974669in}}%
\pgfpathlineto{\pgfqpoint{4.746606in}{2.592076in}}%
\pgfpathlineto{\pgfqpoint{4.746961in}{3.648257in}}%
\pgfpathlineto{\pgfqpoint{4.747485in}{3.608995in}}%
\pgfpathlineto{\pgfqpoint{4.748564in}{2.604758in}}%
\pgfpathlineto{\pgfqpoint{4.747608in}{3.641008in}}%
\pgfpathlineto{\pgfqpoint{4.748734in}{3.245822in}}%
\pgfpathlineto{\pgfqpoint{4.749566in}{3.645156in}}%
\pgfpathlineto{\pgfqpoint{4.749211in}{2.599367in}}%
\pgfpathlineto{\pgfqpoint{4.749782in}{2.755526in}}%
\pgfpathlineto{\pgfqpoint{4.749859in}{2.600303in}}%
\pgfpathlineto{\pgfqpoint{4.750044in}{3.317303in}}%
\pgfpathlineto{\pgfqpoint{4.750213in}{3.647705in}}%
\pgfpathlineto{\pgfqpoint{4.750522in}{2.594039in}}%
\pgfpathlineto{\pgfqpoint{4.751092in}{2.770063in}}%
\pgfpathlineto{\pgfqpoint{4.751107in}{2.770343in}}%
\pgfpathlineto{\pgfqpoint{4.751169in}{2.589322in}}%
\pgfpathlineto{\pgfqpoint{4.751508in}{3.630830in}}%
\pgfpathlineto{\pgfqpoint{4.752002in}{3.374280in}}%
\pgfpathlineto{\pgfqpoint{4.752819in}{3.630675in}}%
\pgfpathlineto{\pgfqpoint{4.752464in}{2.599454in}}%
\pgfpathlineto{\pgfqpoint{4.753050in}{2.797933in}}%
\pgfpathlineto{\pgfqpoint{4.753127in}{2.599678in}}%
\pgfpathlineto{\pgfqpoint{4.753312in}{3.447191in}}%
\pgfpathlineto{\pgfqpoint{4.753466in}{3.630556in}}%
\pgfpathlineto{\pgfqpoint{4.753774in}{2.602549in}}%
\pgfpathlineto{\pgfqpoint{4.754314in}{2.801273in}}%
\pgfpathlineto{\pgfqpoint{4.754422in}{2.603206in}}%
\pgfpathlineto{\pgfqpoint{4.754591in}{3.194761in}}%
\pgfpathlineto{\pgfqpoint{4.754761in}{3.625533in}}%
\pgfpathlineto{\pgfqpoint{4.755085in}{2.613753in}}%
\pgfpathlineto{\pgfqpoint{4.755655in}{2.824571in}}%
\pgfpathlineto{\pgfqpoint{4.756379in}{2.614918in}}%
\pgfpathlineto{\pgfqpoint{4.756071in}{3.628841in}}%
\pgfpathlineto{\pgfqpoint{4.756533in}{3.106212in}}%
\pgfpathlineto{\pgfqpoint{4.756718in}{3.629396in}}%
\pgfpathlineto{\pgfqpoint{4.757027in}{2.626522in}}%
\pgfpathlineto{\pgfqpoint{4.757613in}{2.838175in}}%
\pgfpathlineto{\pgfqpoint{4.757690in}{2.643691in}}%
\pgfpathlineto{\pgfqpoint{4.758029in}{3.615003in}}%
\pgfpathlineto{\pgfqpoint{4.758507in}{3.255237in}}%
\pgfpathlineto{\pgfqpoint{4.758692in}{3.606085in}}%
\pgfpathlineto{\pgfqpoint{4.758984in}{2.657635in}}%
\pgfpathlineto{\pgfqpoint{4.759570in}{2.858837in}}%
\pgfpathlineto{\pgfqpoint{4.759632in}{2.656649in}}%
\pgfpathlineto{\pgfqpoint{4.759971in}{3.591126in}}%
\pgfpathlineto{\pgfqpoint{4.760480in}{3.386651in}}%
\pgfpathlineto{\pgfqpoint{4.760634in}{3.584615in}}%
\pgfpathlineto{\pgfqpoint{4.760942in}{2.655714in}}%
\pgfpathlineto{\pgfqpoint{4.761497in}{2.848388in}}%
\pgfpathlineto{\pgfqpoint{4.761929in}{3.600548in}}%
\pgfpathlineto{\pgfqpoint{4.761590in}{2.665728in}}%
\pgfpathlineto{\pgfqpoint{4.762730in}{3.129116in}}%
\pgfpathlineto{\pgfqpoint{4.762900in}{2.679587in}}%
\pgfpathlineto{\pgfqpoint{4.763239in}{3.541555in}}%
\pgfpathlineto{\pgfqpoint{4.763794in}{3.398238in}}%
\pgfpathlineto{\pgfqpoint{4.763871in}{3.542037in}}%
\pgfpathlineto{\pgfqpoint{4.764041in}{3.164703in}}%
\pgfpathlineto{\pgfqpoint{4.764195in}{2.654188in}}%
\pgfpathlineto{\pgfqpoint{4.764580in}{3.593871in}}%
\pgfpathlineto{\pgfqpoint{4.765120in}{3.402085in}}%
\pgfpathlineto{\pgfqpoint{4.765197in}{3.618282in}}%
\pgfpathlineto{\pgfqpoint{4.765505in}{2.754090in}}%
\pgfpathlineto{\pgfqpoint{4.766014in}{2.984734in}}%
\pgfpathlineto{\pgfqpoint{4.766800in}{2.716130in}}%
\pgfpathlineto{\pgfqpoint{4.766491in}{3.539933in}}%
\pgfpathlineto{\pgfqpoint{4.767046in}{3.399344in}}%
\pgfpathlineto{\pgfqpoint{4.767463in}{2.707604in}}%
\pgfpathlineto{\pgfqpoint{4.767817in}{3.591550in}}%
\pgfpathlineto{\pgfqpoint{4.768264in}{3.026480in}}%
\pgfpathlineto{\pgfqpoint{4.768449in}{3.581235in}}%
\pgfpathlineto{\pgfqpoint{4.768757in}{2.758764in}}%
\pgfpathlineto{\pgfqpoint{4.769359in}{2.940927in}}%
\pgfpathlineto{\pgfqpoint{4.770052in}{2.739818in}}%
\pgfpathlineto{\pgfqpoint{4.769744in}{3.490312in}}%
\pgfpathlineto{\pgfqpoint{4.770345in}{3.364215in}}%
\pgfpathlineto{\pgfqpoint{4.771039in}{3.568323in}}%
\pgfpathlineto{\pgfqpoint{4.770700in}{2.825918in}}%
\pgfpathlineto{\pgfqpoint{4.771301in}{2.951058in}}%
\pgfpathlineto{\pgfqpoint{4.771378in}{2.829687in}}%
\pgfpathlineto{\pgfqpoint{4.771532in}{3.104814in}}%
\pgfpathlineto{\pgfqpoint{4.771686in}{3.497549in}}%
\pgfpathlineto{\pgfqpoint{4.772025in}{2.838171in}}%
\pgfpathlineto{\pgfqpoint{4.772611in}{2.951046in}}%
\pgfpathlineto{\pgfqpoint{4.773289in}{2.803224in}}%
\pgfpathlineto{\pgfqpoint{4.773074in}{3.482390in}}%
\pgfpathlineto{\pgfqpoint{4.773551in}{3.228901in}}%
\pgfpathlineto{\pgfqpoint{4.773675in}{3.495666in}}%
\pgfpathlineto{\pgfqpoint{4.773937in}{2.916942in}}%
\pgfpathlineto{\pgfqpoint{4.774569in}{2.926080in}}%
\pgfpathlineto{\pgfqpoint{4.774677in}{2.851499in}}%
\pgfpathlineto{\pgfqpoint{4.775016in}{3.428785in}}%
\pgfpathlineto{\pgfqpoint{4.775417in}{3.128700in}}%
\pgfpathlineto{\pgfqpoint{4.776311in}{3.465445in}}%
\pgfpathlineto{\pgfqpoint{4.775864in}{2.855168in}}%
\pgfpathlineto{\pgfqpoint{4.776465in}{3.036676in}}%
\pgfpathlineto{\pgfqpoint{4.776511in}{2.940869in}}%
\pgfpathlineto{\pgfqpoint{4.776943in}{3.417668in}}%
\pgfpathlineto{\pgfqpoint{4.777467in}{3.307401in}}%
\pgfpathlineto{\pgfqpoint{4.777498in}{3.381933in}}%
\pgfpathlineto{\pgfqpoint{4.777852in}{2.932906in}}%
\pgfpathlineto{\pgfqpoint{4.778453in}{3.001014in}}%
\pgfpathlineto{\pgfqpoint{4.778469in}{2.980617in}}%
\pgfpathlineto{\pgfqpoint{4.778916in}{3.364334in}}%
\pgfpathlineto{\pgfqpoint{4.779409in}{3.168908in}}%
\pgfpathlineto{\pgfqpoint{4.779548in}{3.355473in}}%
\pgfpathlineto{\pgfqpoint{4.779902in}{3.041291in}}%
\pgfpathlineto{\pgfqpoint{4.780473in}{3.068304in}}%
\pgfpathlineto{\pgfqpoint{4.780534in}{3.015080in}}%
\pgfpathlineto{\pgfqpoint{4.780735in}{3.234896in}}%
\pgfpathlineto{\pgfqpoint{4.780874in}{3.370247in}}%
\pgfpathlineto{\pgfqpoint{4.781089in}{3.015197in}}%
\pgfpathlineto{\pgfqpoint{4.781721in}{3.055748in}}%
\pgfpathlineto{\pgfqpoint{4.781737in}{3.044462in}}%
\pgfpathlineto{\pgfqpoint{4.782168in}{3.326368in}}%
\pgfpathlineto{\pgfqpoint{4.782677in}{3.143454in}}%
\pgfpathlineto{\pgfqpoint{4.783479in}{3.285759in}}%
\pgfpathlineto{\pgfqpoint{4.783062in}{3.066716in}}%
\pgfpathlineto{\pgfqpoint{4.783772in}{3.123140in}}%
\pgfpathlineto{\pgfqpoint{4.783972in}{3.117842in}}%
\pgfpathlineto{\pgfqpoint{4.784126in}{3.236055in}}%
\pgfpathlineto{\pgfqpoint{4.784789in}{3.289873in}}%
\pgfpathlineto{\pgfqpoint{4.784357in}{3.105269in}}%
\pgfpathlineto{\pgfqpoint{4.785066in}{3.141756in}}%
\pgfpathlineto{\pgfqpoint{4.785236in}{3.084266in}}%
\pgfpathlineto{\pgfqpoint{4.785452in}{3.231411in}}%
\pgfpathlineto{\pgfqpoint{4.785976in}{3.190433in}}%
\pgfpathlineto{\pgfqpoint{4.786099in}{3.235479in}}%
\pgfpathlineto{\pgfqpoint{4.786469in}{3.100690in}}%
\pgfpathlineto{\pgfqpoint{4.787070in}{3.191366in}}%
\pgfpathlineto{\pgfqpoint{4.787887in}{3.076978in}}%
\pgfpathlineto{\pgfqpoint{4.788103in}{3.222396in}}%
\pgfpathlineto{\pgfqpoint{4.788211in}{3.247765in}}%
\pgfpathlineto{\pgfqpoint{4.788488in}{3.108390in}}%
\pgfpathlineto{\pgfqpoint{4.789013in}{3.156166in}}%
\pgfpathlineto{\pgfqpoint{4.789182in}{3.078611in}}%
\pgfpathlineto{\pgfqpoint{4.789506in}{3.236605in}}%
\pgfpathlineto{\pgfqpoint{4.790015in}{3.218294in}}%
\pgfpathlineto{\pgfqpoint{4.790770in}{3.267140in}}%
\pgfpathlineto{\pgfqpoint{4.790446in}{3.069535in}}%
\pgfpathlineto{\pgfqpoint{4.790986in}{3.197227in}}%
\pgfpathlineto{\pgfqpoint{4.791155in}{3.082831in}}%
\pgfpathlineto{\pgfqpoint{4.791479in}{3.251973in}}%
\pgfpathlineto{\pgfqpoint{4.792096in}{3.174883in}}%
\pgfpathlineto{\pgfqpoint{4.792142in}{3.258047in}}%
\pgfpathlineto{\pgfqpoint{4.793051in}{3.013649in}}%
\pgfpathlineto{\pgfqpoint{4.793175in}{3.094798in}}%
\pgfpathlineto{\pgfqpoint{4.793452in}{3.355972in}}%
\pgfpathlineto{\pgfqpoint{4.793714in}{3.013077in}}%
\pgfpathlineto{\pgfqpoint{4.794300in}{3.145981in}}%
\pgfpathlineto{\pgfqpoint{4.795009in}{3.022415in}}%
\pgfpathlineto{\pgfqpoint{4.795240in}{3.332793in}}%
\pgfpathlineto{\pgfqpoint{4.795286in}{3.306140in}}%
\pgfpathlineto{\pgfqpoint{4.795672in}{2.894262in}}%
\pgfpathlineto{\pgfqpoint{4.796073in}{3.404524in}}%
\pgfpathlineto{\pgfqpoint{4.796658in}{3.203671in}}%
\pgfpathlineto{\pgfqpoint{4.797244in}{3.274954in}}%
\pgfpathlineto{\pgfqpoint{4.797522in}{2.970013in}}%
\pgfpathlineto{\pgfqpoint{4.797753in}{3.205132in}}%
\pgfpathlineto{\pgfqpoint{4.798354in}{2.993535in}}%
\pgfpathlineto{\pgfqpoint{4.798616in}{3.374630in}}%
\pgfpathlineto{\pgfqpoint{4.798955in}{3.024201in}}%
\pgfpathlineto{\pgfqpoint{4.799726in}{3.329774in}}%
\pgfpathlineto{\pgfqpoint{4.800065in}{3.048486in}}%
\pgfpathlineto{\pgfqpoint{4.800157in}{2.893158in}}%
\pgfpathlineto{\pgfqpoint{4.800527in}{3.401833in}}%
\pgfpathlineto{\pgfqpoint{4.801098in}{3.198661in}}%
\pgfpathlineto{\pgfqpoint{4.801267in}{3.379010in}}%
\pgfpathlineto{\pgfqpoint{4.801853in}{3.053305in}}%
\pgfpathlineto{\pgfqpoint{4.802177in}{3.172210in}}%
\pgfpathlineto{\pgfqpoint{4.802732in}{2.905764in}}%
\pgfpathlineto{\pgfqpoint{4.802346in}{3.356751in}}%
\pgfpathlineto{\pgfqpoint{4.803148in}{3.336954in}}%
\pgfpathlineto{\pgfqpoint{4.803888in}{3.469410in}}%
\pgfpathlineto{\pgfqpoint{4.803502in}{2.970956in}}%
\pgfpathlineto{\pgfqpoint{4.804165in}{3.159349in}}%
\pgfpathlineto{\pgfqpoint{4.804890in}{3.316916in}}%
\pgfpathlineto{\pgfqpoint{4.805368in}{2.971042in}}%
\pgfpathlineto{\pgfqpoint{4.806077in}{2.843553in}}%
\pgfpathlineto{\pgfqpoint{4.805630in}{3.363694in}}%
\pgfpathlineto{\pgfqpoint{4.806262in}{3.107707in}}%
\pgfpathlineto{\pgfqpoint{4.806493in}{3.440404in}}%
\pgfpathlineto{\pgfqpoint{4.807017in}{3.027892in}}%
\pgfpathlineto{\pgfqpoint{4.807510in}{3.257745in}}%
\pgfpathlineto{\pgfqpoint{4.808728in}{2.926359in}}%
\pgfpathlineto{\pgfqpoint{4.808327in}{3.349079in}}%
\pgfpathlineto{\pgfqpoint{4.808851in}{3.110030in}}%
\pgfpathlineto{\pgfqpoint{4.809129in}{3.403940in}}%
\pgfpathlineto{\pgfqpoint{4.809499in}{3.009426in}}%
\pgfpathlineto{\pgfqpoint{4.810085in}{3.247601in}}%
\pgfpathlineto{\pgfqpoint{4.811318in}{2.905458in}}%
\pgfpathlineto{\pgfqpoint{4.810917in}{3.344231in}}%
\pgfpathlineto{\pgfqpoint{4.811379in}{3.003112in}}%
\pgfpathlineto{\pgfqpoint{4.811749in}{3.402822in}}%
\pgfpathlineto{\pgfqpoint{4.812536in}{3.146745in}}%
\pgfpathlineto{\pgfqpoint{4.812921in}{3.079961in}}%
\pgfpathlineto{\pgfqpoint{4.813491in}{3.274393in}}%
\pgfpathlineto{\pgfqpoint{4.813584in}{3.214725in}}%
\pgfpathlineto{\pgfqpoint{4.813645in}{3.265749in}}%
\pgfpathlineto{\pgfqpoint{4.813892in}{2.997602in}}%
\pgfpathlineto{\pgfqpoint{4.813923in}{2.956743in}}%
\pgfpathlineto{\pgfqpoint{4.814339in}{3.384821in}}%
\pgfpathlineto{\pgfqpoint{4.814909in}{3.161199in}}%
\pgfpathlineto{\pgfqpoint{4.814987in}{3.390445in}}%
\pgfpathlineto{\pgfqpoint{4.815418in}{3.042440in}}%
\pgfpathlineto{\pgfqpoint{4.816019in}{3.182721in}}%
\pgfpathlineto{\pgfqpoint{4.816513in}{2.982995in}}%
\pgfpathlineto{\pgfqpoint{4.816127in}{3.310334in}}%
\pgfpathlineto{\pgfqpoint{4.816821in}{3.264436in}}%
\pgfpathlineto{\pgfqpoint{4.817607in}{3.398825in}}%
\pgfpathlineto{\pgfqpoint{4.817145in}{2.968717in}}%
\pgfpathlineto{\pgfqpoint{4.817854in}{3.123963in}}%
\pgfpathlineto{\pgfqpoint{4.818023in}{3.005085in}}%
\pgfpathlineto{\pgfqpoint{4.818254in}{3.318184in}}%
\pgfpathlineto{\pgfqpoint{4.818933in}{3.161269in}}%
\pgfpathlineto{\pgfqpoint{4.819549in}{3.298651in}}%
\pgfpathlineto{\pgfqpoint{4.819750in}{2.974775in}}%
\pgfpathlineto{\pgfqpoint{4.819811in}{3.001430in}}%
\pgfpathlineto{\pgfqpoint{4.819842in}{2.961035in}}%
\pgfpathlineto{\pgfqpoint{4.820228in}{3.429636in}}%
\pgfpathlineto{\pgfqpoint{4.820767in}{3.203381in}}%
\pgfpathlineto{\pgfqpoint{4.820875in}{3.360964in}}%
\pgfpathlineto{\pgfqpoint{4.821723in}{3.058137in}}%
\pgfpathlineto{\pgfqpoint{4.821846in}{3.137044in}}%
\pgfpathlineto{\pgfqpoint{4.822848in}{3.402312in}}%
\pgfpathlineto{\pgfqpoint{4.822370in}{2.913460in}}%
\pgfpathlineto{\pgfqpoint{4.822956in}{3.146564in}}%
\pgfpathlineto{\pgfqpoint{4.823110in}{2.997522in}}%
\pgfpathlineto{\pgfqpoint{4.823511in}{3.375853in}}%
\pgfpathlineto{\pgfqpoint{4.824050in}{3.164622in}}%
\pgfpathlineto{\pgfqpoint{4.824405in}{3.244399in}}%
\pgfpathlineto{\pgfqpoint{4.824929in}{3.096982in}}%
\pgfpathlineto{\pgfqpoint{4.825037in}{2.985949in}}%
\pgfpathlineto{\pgfqpoint{4.825392in}{3.331573in}}%
\pgfpathlineto{\pgfqpoint{4.825993in}{3.155770in}}%
\pgfpathlineto{\pgfqpoint{4.826131in}{3.425713in}}%
\pgfpathlineto{\pgfqpoint{4.826548in}{3.046618in}}%
\pgfpathlineto{\pgfqpoint{4.827118in}{3.229721in}}%
\pgfpathlineto{\pgfqpoint{4.828259in}{2.911339in}}%
\pgfpathlineto{\pgfqpoint{4.828043in}{3.361671in}}%
\pgfpathlineto{\pgfqpoint{4.828351in}{2.999573in}}%
\pgfpathlineto{\pgfqpoint{4.828567in}{3.129797in}}%
\pgfpathlineto{\pgfqpoint{4.828752in}{3.347562in}}%
\pgfpathlineto{\pgfqpoint{4.829446in}{3.044334in}}%
\pgfpathlineto{\pgfqpoint{4.829708in}{3.224598in}}%
\pgfpathlineto{\pgfqpoint{4.829908in}{3.291368in}}%
\pgfpathlineto{\pgfqpoint{4.830124in}{3.068901in}}%
\pgfpathlineto{\pgfqpoint{4.831033in}{2.985756in}}%
\pgfpathlineto{\pgfqpoint{4.830648in}{3.350506in}}%
\pgfpathlineto{\pgfqpoint{4.831203in}{3.087371in}}%
\pgfpathlineto{\pgfqpoint{4.831311in}{3.376237in}}%
\pgfpathlineto{\pgfqpoint{4.831742in}{3.059885in}}%
\pgfpathlineto{\pgfqpoint{4.832344in}{3.255805in}}%
\pgfpathlineto{\pgfqpoint{4.832359in}{3.256876in}}%
\pgfpathlineto{\pgfqpoint{4.832529in}{3.191499in}}%
\pgfpathlineto{\pgfqpoint{4.833500in}{2.939367in}}%
\pgfpathlineto{\pgfqpoint{4.833114in}{3.306079in}}%
\pgfpathlineto{\pgfqpoint{4.833777in}{3.135776in}}%
\pgfpathlineto{\pgfqpoint{4.833885in}{3.330453in}}%
\pgfpathlineto{\pgfqpoint{4.834671in}{3.060490in}}%
\pgfpathlineto{\pgfqpoint{4.834949in}{3.181892in}}%
\pgfpathlineto{\pgfqpoint{4.835874in}{3.281359in}}%
\pgfpathlineto{\pgfqpoint{4.835411in}{3.048646in}}%
\pgfpathlineto{\pgfqpoint{4.836012in}{3.174460in}}%
\pgfpathlineto{\pgfqpoint{4.836228in}{3.052381in}}%
\pgfpathlineto{\pgfqpoint{4.836567in}{3.362669in}}%
\pgfpathlineto{\pgfqpoint{4.837138in}{3.122011in}}%
\pgfpathlineto{\pgfqpoint{4.838294in}{3.325285in}}%
\pgfpathlineto{\pgfqpoint{4.838140in}{3.003944in}}%
\pgfpathlineto{\pgfqpoint{4.838309in}{3.323908in}}%
\pgfpathlineto{\pgfqpoint{4.838741in}{3.019631in}}%
\pgfpathlineto{\pgfqpoint{4.839512in}{3.171480in}}%
\pgfpathlineto{\pgfqpoint{4.840729in}{3.040345in}}%
\pgfpathlineto{\pgfqpoint{4.839604in}{3.255047in}}%
\pgfpathlineto{\pgfqpoint{4.840760in}{3.105034in}}%
\pgfpathlineto{\pgfqpoint{4.841808in}{3.340840in}}%
\pgfpathlineto{\pgfqpoint{4.841423in}{3.068108in}}%
\pgfpathlineto{\pgfqpoint{4.841901in}{3.199736in}}%
\pgfpathlineto{\pgfqpoint{4.842348in}{3.090208in}}%
\pgfpathlineto{\pgfqpoint{4.842564in}{3.269776in}}%
\pgfpathlineto{\pgfqpoint{4.843011in}{3.177525in}}%
\pgfpathlineto{\pgfqpoint{4.843550in}{3.279660in}}%
\pgfpathlineto{\pgfqpoint{4.843966in}{3.016779in}}%
\pgfpathlineto{\pgfqpoint{4.844059in}{3.043721in}}%
\pgfpathlineto{\pgfqpoint{4.844090in}{3.030337in}}%
\pgfpathlineto{\pgfqpoint{4.844306in}{3.308140in}}%
\pgfpathlineto{\pgfqpoint{4.844814in}{3.206204in}}%
\pgfpathlineto{\pgfqpoint{4.845061in}{3.256010in}}%
\pgfpathlineto{\pgfqpoint{4.845816in}{3.061681in}}%
\pgfpathlineto{\pgfqpoint{4.845832in}{3.051835in}}%
\pgfpathlineto{\pgfqpoint{4.846263in}{3.270465in}}%
\pgfpathlineto{\pgfqpoint{4.846741in}{3.145340in}}%
\pgfpathlineto{\pgfqpoint{4.847003in}{3.294188in}}%
\pgfpathlineto{\pgfqpoint{4.847234in}{3.045108in}}%
\pgfpathlineto{\pgfqpoint{4.847851in}{3.172905in}}%
\pgfpathlineto{\pgfqpoint{4.848128in}{3.097746in}}%
\pgfpathlineto{\pgfqpoint{4.848329in}{3.275665in}}%
\pgfpathlineto{\pgfqpoint{4.848961in}{3.166223in}}%
\pgfpathlineto{\pgfqpoint{4.849886in}{3.005333in}}%
\pgfpathlineto{\pgfqpoint{4.849516in}{3.307616in}}%
\pgfpathlineto{\pgfqpoint{4.850055in}{3.131283in}}%
\pgfpathlineto{\pgfqpoint{4.850286in}{3.269451in}}%
\pgfpathlineto{\pgfqpoint{4.851088in}{3.075682in}}%
\pgfpathlineto{\pgfqpoint{4.851165in}{3.152982in}}%
\pgfpathlineto{\pgfqpoint{4.852275in}{3.285577in}}%
\pgfpathlineto{\pgfqpoint{4.851735in}{3.066134in}}%
\pgfpathlineto{\pgfqpoint{4.852398in}{3.189614in}}%
\pgfpathlineto{\pgfqpoint{4.853400in}{3.055392in}}%
\pgfpathlineto{\pgfqpoint{4.853015in}{3.351723in}}%
\pgfpathlineto{\pgfqpoint{4.853524in}{3.166509in}}%
\pgfpathlineto{\pgfqpoint{4.854202in}{3.252120in}}%
\pgfpathlineto{\pgfqpoint{4.854479in}{2.995584in}}%
\pgfpathlineto{\pgfqpoint{4.854618in}{3.147148in}}%
\pgfpathlineto{\pgfqpoint{4.854757in}{3.388113in}}%
\pgfpathlineto{\pgfqpoint{4.855173in}{2.970302in}}%
\pgfpathlineto{\pgfqpoint{4.855774in}{3.185374in}}%
\pgfpathlineto{\pgfqpoint{4.856977in}{3.019728in}}%
\pgfpathlineto{\pgfqpoint{4.856499in}{3.322906in}}%
\pgfpathlineto{\pgfqpoint{4.856992in}{3.037141in}}%
\pgfpathlineto{\pgfqpoint{4.857470in}{3.381565in}}%
\pgfpathlineto{\pgfqpoint{4.857824in}{3.005803in}}%
\pgfpathlineto{\pgfqpoint{4.858318in}{3.318413in}}%
\pgfpathlineto{\pgfqpoint{4.859659in}{2.937180in}}%
\pgfpathlineto{\pgfqpoint{4.859243in}{3.340440in}}%
\pgfpathlineto{\pgfqpoint{4.859844in}{3.155563in}}%
\pgfpathlineto{\pgfqpoint{4.860059in}{3.424903in}}%
\pgfpathlineto{\pgfqpoint{4.860368in}{3.005346in}}%
\pgfpathlineto{\pgfqpoint{4.861000in}{3.292288in}}%
\pgfpathlineto{\pgfqpoint{4.861416in}{3.013556in}}%
\pgfpathlineto{\pgfqpoint{4.861894in}{3.363548in}}%
\pgfpathlineto{\pgfqpoint{4.862233in}{3.053489in}}%
\pgfpathlineto{\pgfqpoint{4.863420in}{3.387863in}}%
\pgfpathlineto{\pgfqpoint{4.863173in}{3.000026in}}%
\pgfpathlineto{\pgfqpoint{4.863636in}{3.256728in}}%
\pgfpathlineto{\pgfqpoint{4.864884in}{2.917411in}}%
\pgfpathlineto{\pgfqpoint{4.864468in}{3.368804in}}%
\pgfpathlineto{\pgfqpoint{4.865008in}{3.013684in}}%
\pgfpathlineto{\pgfqpoint{4.865193in}{3.420786in}}%
\pgfpathlineto{\pgfqpoint{4.865640in}{2.935692in}}%
\pgfpathlineto{\pgfqpoint{4.866318in}{3.257806in}}%
\pgfpathlineto{\pgfqpoint{4.866719in}{2.983792in}}%
\pgfpathlineto{\pgfqpoint{4.867212in}{3.357734in}}%
\pgfpathlineto{\pgfqpoint{4.867474in}{3.198707in}}%
\pgfpathlineto{\pgfqpoint{4.867983in}{3.400115in}}%
\pgfpathlineto{\pgfqpoint{4.868291in}{2.990811in}}%
\pgfpathlineto{\pgfqpoint{4.868414in}{3.019643in}}%
\pgfpathlineto{\pgfqpoint{4.868461in}{2.927512in}}%
\pgfpathlineto{\pgfqpoint{4.868723in}{3.445627in}}%
\pgfpathlineto{\pgfqpoint{4.869463in}{3.175387in}}%
\pgfpathlineto{\pgfqpoint{4.870465in}{3.457428in}}%
\pgfpathlineto{\pgfqpoint{4.870172in}{2.962014in}}%
\pgfpathlineto{\pgfqpoint{4.870588in}{3.233088in}}%
\pgfpathlineto{\pgfqpoint{4.870912in}{2.954919in}}%
\pgfpathlineto{\pgfqpoint{4.871374in}{3.327461in}}%
\pgfpathlineto{\pgfqpoint{4.871852in}{3.105825in}}%
\pgfpathlineto{\pgfqpoint{4.872438in}{3.358540in}}%
\pgfpathlineto{\pgfqpoint{4.871991in}{3.009330in}}%
\pgfpathlineto{\pgfqpoint{4.872900in}{3.022838in}}%
\pgfpathlineto{\pgfqpoint{4.873008in}{3.112824in}}%
\pgfpathlineto{\pgfqpoint{4.873054in}{3.102706in}}%
\pgfpathlineto{\pgfqpoint{4.873193in}{3.348575in}}%
\pgfpathlineto{\pgfqpoint{4.873763in}{2.973668in}}%
\pgfpathlineto{\pgfqpoint{4.874210in}{3.283104in}}%
\pgfpathlineto{\pgfqpoint{4.875366in}{2.915550in}}%
\pgfpathlineto{\pgfqpoint{4.875089in}{3.374753in}}%
\pgfpathlineto{\pgfqpoint{4.875551in}{3.091923in}}%
\pgfpathlineto{\pgfqpoint{4.875752in}{3.389145in}}%
\pgfpathlineto{\pgfqpoint{4.876106in}{2.964354in}}%
\pgfpathlineto{\pgfqpoint{4.876692in}{3.195936in}}%
\pgfpathlineto{\pgfqpoint{4.877093in}{3.071403in}}%
\pgfpathlineto{\pgfqpoint{4.876908in}{3.332122in}}%
\pgfpathlineto{\pgfqpoint{4.877709in}{3.246566in}}%
\pgfpathlineto{\pgfqpoint{4.878403in}{3.374981in}}%
\pgfpathlineto{\pgfqpoint{4.878141in}{2.992071in}}%
\pgfpathlineto{\pgfqpoint{4.878742in}{3.206539in}}%
\pgfpathlineto{\pgfqpoint{4.879066in}{2.967948in}}%
\pgfpathlineto{\pgfqpoint{4.879528in}{3.325281in}}%
\pgfpathlineto{\pgfqpoint{4.879852in}{3.181216in}}%
\pgfpathlineto{\pgfqpoint{4.880392in}{3.314482in}}%
\pgfpathlineto{\pgfqpoint{4.880561in}{3.006708in}}%
\pgfpathlineto{\pgfqpoint{4.880762in}{3.050155in}}%
\pgfpathlineto{\pgfqpoint{4.880777in}{3.020880in}}%
\pgfpathlineto{\pgfqpoint{4.881024in}{3.349498in}}%
\pgfpathlineto{\pgfqpoint{4.881856in}{3.065322in}}%
\pgfpathlineto{\pgfqpoint{4.882118in}{3.312649in}}%
\pgfpathlineto{\pgfqpoint{4.882581in}{2.970290in}}%
\pgfpathlineto{\pgfqpoint{4.883012in}{3.253489in}}%
\pgfpathlineto{\pgfqpoint{4.883968in}{3.348407in}}%
\pgfpathlineto{\pgfqpoint{4.883521in}{2.945837in}}%
\pgfpathlineto{\pgfqpoint{4.884014in}{3.208117in}}%
\pgfpathlineto{\pgfqpoint{4.884276in}{2.998797in}}%
\pgfpathlineto{\pgfqpoint{4.884739in}{3.278293in}}%
\pgfpathlineto{\pgfqpoint{4.885155in}{3.098143in}}%
\pgfpathlineto{\pgfqpoint{4.885247in}{2.998992in}}%
\pgfpathlineto{\pgfqpoint{4.885401in}{3.187529in}}%
\pgfpathlineto{\pgfqpoint{4.885694in}{3.342946in}}%
\pgfpathlineto{\pgfqpoint{4.885895in}{3.034056in}}%
\pgfpathlineto{\pgfqpoint{4.886481in}{3.101180in}}%
\pgfpathlineto{\pgfqpoint{4.887128in}{2.994649in}}%
\pgfpathlineto{\pgfqpoint{4.886558in}{3.282075in}}%
\pgfpathlineto{\pgfqpoint{4.887390in}{3.261089in}}%
\pgfpathlineto{\pgfqpoint{4.887436in}{3.332915in}}%
\pgfpathlineto{\pgfqpoint{4.887806in}{3.011962in}}%
\pgfpathlineto{\pgfqpoint{4.888469in}{3.233250in}}%
\pgfpathlineto{\pgfqpoint{4.888716in}{2.934135in}}%
\pgfpathlineto{\pgfqpoint{4.889178in}{3.355224in}}%
\pgfpathlineto{\pgfqpoint{4.889625in}{3.117529in}}%
\pgfpathlineto{\pgfqpoint{4.890149in}{3.343485in}}%
\pgfpathlineto{\pgfqpoint{4.890550in}{2.996669in}}%
\pgfpathlineto{\pgfqpoint{4.890797in}{3.232855in}}%
\pgfpathlineto{\pgfqpoint{4.890997in}{3.294796in}}%
\pgfpathlineto{\pgfqpoint{4.891182in}{3.067390in}}%
\pgfpathlineto{\pgfqpoint{4.891475in}{3.144151in}}%
\pgfpathlineto{\pgfqpoint{4.892230in}{2.976171in}}%
\pgfpathlineto{\pgfqpoint{4.892138in}{3.305097in}}%
\pgfpathlineto{\pgfqpoint{4.892554in}{3.172922in}}%
\pgfpathlineto{\pgfqpoint{4.893618in}{3.345092in}}%
\pgfpathlineto{\pgfqpoint{4.893155in}{3.002671in}}%
\pgfpathlineto{\pgfqpoint{4.893695in}{3.219374in}}%
\pgfpathlineto{\pgfqpoint{4.894018in}{3.042522in}}%
\pgfpathlineto{\pgfqpoint{4.894496in}{3.274051in}}%
\pgfpathlineto{\pgfqpoint{4.894882in}{3.072641in}}%
\pgfpathlineto{\pgfqpoint{4.895344in}{3.337816in}}%
\pgfpathlineto{\pgfqpoint{4.895760in}{2.959959in}}%
\pgfpathlineto{\pgfqpoint{4.896099in}{3.200953in}}%
\pgfpathlineto{\pgfqpoint{4.896778in}{3.018140in}}%
\pgfpathlineto{\pgfqpoint{4.897163in}{3.317136in}}%
\pgfpathlineto{\pgfqpoint{4.897178in}{3.310109in}}%
\pgfpathlineto{\pgfqpoint{4.897518in}{2.962237in}}%
\pgfpathlineto{\pgfqpoint{4.898427in}{3.087970in}}%
\pgfpathlineto{\pgfqpoint{4.898828in}{3.301766in}}%
\pgfpathlineto{\pgfqpoint{4.899090in}{3.019822in}}%
\pgfpathlineto{\pgfqpoint{4.899614in}{3.131533in}}%
\pgfpathlineto{\pgfqpoint{4.900184in}{3.028253in}}%
\pgfpathlineto{\pgfqpoint{4.899784in}{3.299171in}}%
\pgfpathlineto{\pgfqpoint{4.900385in}{3.180973in}}%
\pgfpathlineto{\pgfqpoint{4.900631in}{3.356354in}}%
\pgfpathlineto{\pgfqpoint{4.900986in}{3.075641in}}%
\pgfpathlineto{\pgfqpoint{4.901525in}{3.260119in}}%
\pgfpathlineto{\pgfqpoint{4.902789in}{2.954529in}}%
\pgfpathlineto{\pgfqpoint{4.902435in}{3.341166in}}%
\pgfpathlineto{\pgfqpoint{4.902820in}{3.004374in}}%
\pgfpathlineto{\pgfqpoint{4.903267in}{3.328009in}}%
\pgfpathlineto{\pgfqpoint{4.903976in}{3.198627in}}%
\pgfpathlineto{\pgfqpoint{4.905055in}{3.293725in}}%
\pgfpathlineto{\pgfqpoint{4.904593in}{3.083559in}}%
\pgfpathlineto{\pgfqpoint{4.905102in}{3.212612in}}%
\pgfpathlineto{\pgfqpoint{4.905441in}{3.094856in}}%
\pgfpathlineto{\pgfqpoint{4.905934in}{3.276504in}}%
\pgfpathlineto{\pgfqpoint{4.906181in}{3.194966in}}%
\pgfpathlineto{\pgfqpoint{4.906951in}{3.297910in}}%
\pgfpathlineto{\pgfqpoint{4.907198in}{3.038114in}}%
\pgfpathlineto{\pgfqpoint{4.907229in}{2.950594in}}%
\pgfpathlineto{\pgfqpoint{4.907691in}{3.370686in}}%
\pgfpathlineto{\pgfqpoint{4.908277in}{3.138527in}}%
\pgfpathlineto{\pgfqpoint{4.909402in}{3.247718in}}%
\pgfpathlineto{\pgfqpoint{4.908971in}{3.091731in}}%
\pgfpathlineto{\pgfqpoint{4.909433in}{3.220944in}}%
\pgfpathlineto{\pgfqpoint{4.909495in}{3.291040in}}%
\pgfpathlineto{\pgfqpoint{4.910189in}{3.066624in}}%
\pgfpathlineto{\pgfqpoint{4.910497in}{3.210345in}}%
\pgfpathlineto{\pgfqpoint{4.910913in}{3.021719in}}%
\pgfpathlineto{\pgfqpoint{4.911221in}{3.295595in}}%
\pgfpathlineto{\pgfqpoint{4.911668in}{3.038685in}}%
\pgfpathlineto{\pgfqpoint{4.912115in}{3.326588in}}%
\pgfpathlineto{\pgfqpoint{4.912516in}{3.036809in}}%
\pgfpathlineto{\pgfqpoint{4.912809in}{3.138361in}}%
\pgfpathlineto{\pgfqpoint{4.912994in}{3.314448in}}%
\pgfpathlineto{\pgfqpoint{4.913549in}{3.064136in}}%
\pgfpathlineto{\pgfqpoint{4.914011in}{3.261740in}}%
\pgfpathlineto{\pgfqpoint{4.914428in}{3.026425in}}%
\pgfpathlineto{\pgfqpoint{4.914751in}{3.297570in}}%
\pgfpathlineto{\pgfqpoint{4.915537in}{3.169829in}}%
\pgfpathlineto{\pgfqpoint{4.916539in}{3.384019in}}%
\pgfpathlineto{\pgfqpoint{4.916077in}{2.993022in}}%
\pgfpathlineto{\pgfqpoint{4.916678in}{3.232220in}}%
\pgfpathlineto{\pgfqpoint{4.917079in}{3.039212in}}%
\pgfpathlineto{\pgfqpoint{4.917418in}{3.312803in}}%
\pgfpathlineto{\pgfqpoint{4.917850in}{3.103808in}}%
\pgfpathlineto{\pgfqpoint{4.918266in}{3.311482in}}%
\pgfpathlineto{\pgfqpoint{4.918836in}{3.027995in}}%
\pgfpathlineto{\pgfqpoint{4.919592in}{2.987561in}}%
\pgfpathlineto{\pgfqpoint{4.919175in}{3.312520in}}%
\pgfpathlineto{\pgfqpoint{4.919838in}{3.222168in}}%
\pgfpathlineto{\pgfqpoint{4.920532in}{3.025410in}}%
\pgfpathlineto{\pgfqpoint{4.920054in}{3.298446in}}%
\pgfpathlineto{\pgfqpoint{4.920902in}{3.264450in}}%
\pgfpathlineto{\pgfqpoint{4.921750in}{3.331527in}}%
\pgfpathlineto{\pgfqpoint{4.921503in}{3.021922in}}%
\pgfpathlineto{\pgfqpoint{4.921965in}{3.221545in}}%
\pgfpathlineto{\pgfqpoint{4.922243in}{3.029664in}}%
\pgfpathlineto{\pgfqpoint{4.922690in}{3.309347in}}%
\pgfpathlineto{\pgfqpoint{4.923122in}{3.069926in}}%
\pgfpathlineto{\pgfqpoint{4.924432in}{3.316277in}}%
\pgfpathlineto{\pgfqpoint{4.924031in}{3.013087in}}%
\pgfpathlineto{\pgfqpoint{4.924493in}{3.258992in}}%
\pgfpathlineto{\pgfqpoint{4.924509in}{3.259609in}}%
\pgfpathlineto{\pgfqpoint{4.924586in}{3.218465in}}%
\pgfpathlineto{\pgfqpoint{4.924971in}{3.005273in}}%
\pgfpathlineto{\pgfqpoint{4.925341in}{3.282218in}}%
\pgfpathlineto{\pgfqpoint{4.925881in}{3.085570in}}%
\pgfpathlineto{\pgfqpoint{4.926189in}{3.334357in}}%
\pgfpathlineto{\pgfqpoint{4.926436in}{3.043343in}}%
\pgfpathlineto{\pgfqpoint{4.927145in}{3.235544in}}%
\pgfpathlineto{\pgfqpoint{4.927530in}{3.053381in}}%
\pgfpathlineto{\pgfqpoint{4.927977in}{3.330044in}}%
\pgfpathlineto{\pgfqpoint{4.928270in}{3.191107in}}%
\pgfpathlineto{\pgfqpoint{4.929395in}{2.994478in}}%
\pgfpathlineto{\pgfqpoint{4.929133in}{3.298360in}}%
\pgfpathlineto{\pgfqpoint{4.929442in}{3.139203in}}%
\pgfpathlineto{\pgfqpoint{4.930135in}{2.995963in}}%
\pgfpathlineto{\pgfqpoint{4.930582in}{3.290849in}}%
\pgfpathlineto{\pgfqpoint{4.930613in}{3.326429in}}%
\pgfpathlineto{\pgfqpoint{4.931153in}{3.036465in}}%
\pgfpathlineto{\pgfqpoint{4.931615in}{3.162331in}}%
\pgfpathlineto{\pgfqpoint{4.932031in}{3.055997in}}%
\pgfpathlineto{\pgfqpoint{4.932417in}{3.276240in}}%
\pgfpathlineto{\pgfqpoint{4.932633in}{3.184143in}}%
\pgfpathlineto{\pgfqpoint{4.933542in}{3.268395in}}%
\pgfpathlineto{\pgfqpoint{4.932910in}{3.053259in}}%
\pgfpathlineto{\pgfqpoint{4.933634in}{3.069253in}}%
\pgfpathlineto{\pgfqpoint{4.933804in}{3.049732in}}%
\pgfpathlineto{\pgfqpoint{4.933850in}{3.161369in}}%
\pgfpathlineto{\pgfqpoint{4.934143in}{3.254533in}}%
\pgfpathlineto{\pgfqpoint{4.934575in}{3.034861in}}%
\pgfpathlineto{\pgfqpoint{4.934991in}{3.243489in}}%
\pgfpathlineto{\pgfqpoint{4.935037in}{3.268167in}}%
\pgfpathlineto{\pgfqpoint{4.935546in}{3.051218in}}%
\pgfpathlineto{\pgfqpoint{4.936008in}{3.235637in}}%
\pgfpathlineto{\pgfqpoint{4.936425in}{3.059453in}}%
\pgfpathlineto{\pgfqpoint{4.936856in}{3.272042in}}%
\pgfpathlineto{\pgfqpoint{4.937164in}{3.089151in}}%
\pgfpathlineto{\pgfqpoint{4.937596in}{3.253719in}}%
\pgfpathlineto{\pgfqpoint{4.937303in}{3.062172in}}%
\pgfpathlineto{\pgfqpoint{4.938382in}{3.128405in}}%
\pgfpathlineto{\pgfqpoint{4.939014in}{3.036155in}}%
\pgfpathlineto{\pgfqpoint{4.938768in}{3.261370in}}%
\pgfpathlineto{\pgfqpoint{4.939323in}{3.201443in}}%
\pgfpathlineto{\pgfqpoint{4.939477in}{3.289166in}}%
\pgfpathlineto{\pgfqpoint{4.939754in}{3.068280in}}%
\pgfpathlineto{\pgfqpoint{4.940417in}{3.208119in}}%
\pgfpathlineto{\pgfqpoint{4.940818in}{3.055636in}}%
\pgfpathlineto{\pgfqpoint{4.941280in}{3.273395in}}%
\pgfpathlineto{\pgfqpoint{4.941712in}{3.096894in}}%
\pgfpathlineto{\pgfqpoint{4.942020in}{3.267056in}}%
\pgfpathlineto{\pgfqpoint{4.942575in}{3.074836in}}%
\pgfpathlineto{\pgfqpoint{4.942914in}{3.214719in}}%
\pgfpathlineto{\pgfqpoint{4.943438in}{3.021072in}}%
\pgfpathlineto{\pgfqpoint{4.943916in}{3.304634in}}%
\pgfpathlineto{\pgfqpoint{4.944009in}{3.200391in}}%
\pgfpathlineto{\pgfqpoint{4.944656in}{3.304962in}}%
\pgfpathlineto{\pgfqpoint{4.944332in}{3.026088in}}%
\pgfpathlineto{\pgfqpoint{4.945041in}{3.125715in}}%
\pgfpathlineto{\pgfqpoint{4.946074in}{3.064084in}}%
\pgfpathlineto{\pgfqpoint{4.945843in}{3.251800in}}%
\pgfpathlineto{\pgfqpoint{4.946120in}{3.126991in}}%
\pgfpathlineto{\pgfqpoint{4.946444in}{3.284628in}}%
\pgfpathlineto{\pgfqpoint{4.946968in}{3.022637in}}%
\pgfpathlineto{\pgfqpoint{4.947292in}{3.236272in}}%
\pgfpathlineto{\pgfqpoint{4.948340in}{3.285826in}}%
\pgfpathlineto{\pgfqpoint{4.947862in}{3.047363in}}%
\pgfpathlineto{\pgfqpoint{4.948371in}{3.236525in}}%
\pgfpathlineto{\pgfqpoint{4.948726in}{3.028272in}}%
\pgfpathlineto{\pgfqpoint{4.949080in}{3.304302in}}%
\pgfpathlineto{\pgfqpoint{4.949527in}{3.128067in}}%
\pgfpathlineto{\pgfqpoint{4.949820in}{3.250748in}}%
\pgfpathlineto{\pgfqpoint{4.949604in}{3.034367in}}%
\pgfpathlineto{\pgfqpoint{4.950622in}{3.119245in}}%
\pgfpathlineto{\pgfqpoint{4.951377in}{3.027669in}}%
\pgfpathlineto{\pgfqpoint{4.950961in}{3.262261in}}%
\pgfpathlineto{\pgfqpoint{4.951577in}{3.141749in}}%
\pgfpathlineto{\pgfqpoint{4.951701in}{3.271153in}}%
\pgfpathlineto{\pgfqpoint{4.952101in}{3.064121in}}%
\pgfpathlineto{\pgfqpoint{4.952703in}{3.199609in}}%
\pgfpathlineto{\pgfqpoint{4.953134in}{3.083154in}}%
\pgfpathlineto{\pgfqpoint{4.953520in}{3.259960in}}%
\pgfpathlineto{\pgfqpoint{4.953705in}{3.238653in}}%
\pgfpathlineto{\pgfqpoint{4.953720in}{3.257156in}}%
\pgfpathlineto{\pgfqpoint{4.954013in}{3.014270in}}%
\pgfpathlineto{\pgfqpoint{4.954691in}{3.106727in}}%
\pgfpathlineto{\pgfqpoint{4.954753in}{3.060474in}}%
\pgfpathlineto{\pgfqpoint{4.955200in}{3.255231in}}%
\pgfpathlineto{\pgfqpoint{4.955693in}{3.156164in}}%
\pgfpathlineto{\pgfqpoint{4.956202in}{3.261615in}}%
\pgfpathlineto{\pgfqpoint{4.956633in}{3.054114in}}%
\pgfpathlineto{\pgfqpoint{4.956818in}{3.182650in}}%
\pgfpathlineto{\pgfqpoint{4.957373in}{3.075698in}}%
\pgfpathlineto{\pgfqpoint{4.956942in}{3.276569in}}%
\pgfpathlineto{\pgfqpoint{4.957836in}{3.213548in}}%
\pgfpathlineto{\pgfqpoint{4.958699in}{3.257484in}}%
\pgfpathlineto{\pgfqpoint{4.958221in}{3.097600in}}%
\pgfpathlineto{\pgfqpoint{4.958915in}{3.221306in}}%
\pgfpathlineto{\pgfqpoint{4.959285in}{3.038015in}}%
\pgfpathlineto{\pgfqpoint{4.959624in}{3.250677in}}%
\pgfpathlineto{\pgfqpoint{4.960055in}{3.121383in}}%
\pgfpathlineto{\pgfqpoint{4.960487in}{3.237713in}}%
\pgfpathlineto{\pgfqpoint{4.961057in}{3.067492in}}%
\pgfpathlineto{\pgfqpoint{4.961150in}{3.141662in}}%
\pgfpathlineto{\pgfqpoint{4.961921in}{3.039073in}}%
\pgfpathlineto{\pgfqpoint{4.961366in}{3.246732in}}%
\pgfpathlineto{\pgfqpoint{4.962229in}{3.218559in}}%
\pgfpathlineto{\pgfqpoint{4.962645in}{3.051578in}}%
\pgfpathlineto{\pgfqpoint{4.962383in}{3.249977in}}%
\pgfpathlineto{\pgfqpoint{4.963401in}{3.167828in}}%
\pgfpathlineto{\pgfqpoint{4.964002in}{3.250832in}}%
\pgfpathlineto{\pgfqpoint{4.964418in}{3.058642in}}%
\pgfpathlineto{\pgfqpoint{4.964495in}{3.154884in}}%
\pgfpathlineto{\pgfqpoint{4.964927in}{3.225078in}}%
\pgfpathlineto{\pgfqpoint{4.964557in}{3.092583in}}%
\pgfpathlineto{\pgfqpoint{4.965420in}{3.109585in}}%
\pgfpathlineto{\pgfqpoint{4.966329in}{3.054694in}}%
\pgfpathlineto{\pgfqpoint{4.966067in}{3.255024in}}%
\pgfpathlineto{\pgfqpoint{4.966360in}{3.131567in}}%
\pgfpathlineto{\pgfqpoint{4.966807in}{3.245309in}}%
\pgfpathlineto{\pgfqpoint{4.967069in}{3.081179in}}%
\pgfpathlineto{\pgfqpoint{4.967485in}{3.212760in}}%
\pgfpathlineto{\pgfqpoint{4.967532in}{3.234465in}}%
\pgfpathlineto{\pgfqpoint{4.968056in}{3.100691in}}%
\pgfpathlineto{\pgfqpoint{4.968071in}{3.073345in}}%
\pgfpathlineto{\pgfqpoint{4.968503in}{3.243619in}}%
\pgfpathlineto{\pgfqpoint{4.969119in}{3.149765in}}%
\pgfpathlineto{\pgfqpoint{4.969243in}{3.246923in}}%
\pgfpathlineto{\pgfqpoint{4.969829in}{3.065863in}}%
\pgfpathlineto{\pgfqpoint{4.970276in}{3.213898in}}%
\pgfpathlineto{\pgfqpoint{4.970568in}{3.092593in}}%
\pgfpathlineto{\pgfqpoint{4.971247in}{3.237943in}}%
\pgfpathlineto{\pgfqpoint{4.971478in}{3.141430in}}%
\pgfpathlineto{\pgfqpoint{4.971987in}{3.240049in}}%
\pgfpathlineto{\pgfqpoint{4.972464in}{3.090301in}}%
\pgfpathlineto{\pgfqpoint{4.972557in}{3.150989in}}%
\pgfpathlineto{\pgfqpoint{4.973328in}{3.057139in}}%
\pgfpathlineto{\pgfqpoint{4.973035in}{3.251078in}}%
\pgfpathlineto{\pgfqpoint{4.973636in}{3.188791in}}%
\pgfpathlineto{\pgfqpoint{4.973775in}{3.257245in}}%
\pgfpathlineto{\pgfqpoint{4.974052in}{3.090693in}}%
\pgfpathlineto{\pgfqpoint{4.974746in}{3.193806in}}%
\pgfpathlineto{\pgfqpoint{4.975671in}{3.258036in}}%
\pgfpathlineto{\pgfqpoint{4.975100in}{3.090583in}}%
\pgfpathlineto{\pgfqpoint{4.975717in}{3.184648in}}%
\pgfpathlineto{\pgfqpoint{4.976842in}{3.072770in}}%
\pgfpathlineto{\pgfqpoint{4.976411in}{3.260736in}}%
\pgfpathlineto{\pgfqpoint{4.976858in}{3.088069in}}%
\pgfpathlineto{\pgfqpoint{4.977289in}{3.233506in}}%
\pgfpathlineto{\pgfqpoint{4.977736in}{3.078545in}}%
\pgfpathlineto{\pgfqpoint{4.977998in}{3.183147in}}%
\pgfpathlineto{\pgfqpoint{4.978199in}{3.236795in}}%
\pgfpathlineto{\pgfqpoint{4.978584in}{3.098929in}}%
\pgfpathlineto{\pgfqpoint{4.979124in}{3.204397in}}%
\pgfpathlineto{\pgfqpoint{4.979170in}{3.222579in}}%
\pgfpathlineto{\pgfqpoint{4.979232in}{3.164475in}}%
\pgfpathlineto{\pgfqpoint{4.980357in}{3.074506in}}%
\pgfpathlineto{\pgfqpoint{4.980095in}{3.259335in}}%
\pgfpathlineto{\pgfqpoint{4.980372in}{3.101861in}}%
\pgfpathlineto{\pgfqpoint{4.980835in}{3.247640in}}%
\pgfpathlineto{\pgfqpoint{4.980526in}{3.097302in}}%
\pgfpathlineto{\pgfqpoint{4.981498in}{3.184525in}}%
\pgfpathlineto{\pgfqpoint{4.981790in}{3.235169in}}%
\pgfpathlineto{\pgfqpoint{4.981929in}{3.155435in}}%
\pgfpathlineto{\pgfqpoint{4.982099in}{3.082317in}}%
\pgfpathlineto{\pgfqpoint{4.982669in}{3.234491in}}%
\pgfpathlineto{\pgfqpoint{4.983008in}{3.146483in}}%
\pgfpathlineto{\pgfqpoint{4.983548in}{3.231761in}}%
\pgfpathlineto{\pgfqpoint{4.983856in}{3.069099in}}%
\pgfpathlineto{\pgfqpoint{4.984149in}{3.211980in}}%
\pgfpathlineto{\pgfqpoint{4.984580in}{3.108386in}}%
\pgfpathlineto{\pgfqpoint{4.985228in}{3.220151in}}%
\pgfpathlineto{\pgfqpoint{4.985290in}{3.252738in}}%
\pgfpathlineto{\pgfqpoint{4.985783in}{3.100180in}}%
\pgfpathlineto{\pgfqpoint{4.986261in}{3.167605in}}%
\pgfpathlineto{\pgfqpoint{4.987355in}{3.076195in}}%
\pgfpathlineto{\pgfqpoint{4.987062in}{3.251120in}}%
\pgfpathlineto{\pgfqpoint{4.987386in}{3.139321in}}%
\pgfpathlineto{\pgfqpoint{4.987956in}{3.230646in}}%
\pgfpathlineto{\pgfqpoint{4.987540in}{3.087766in}}%
\pgfpathlineto{\pgfqpoint{4.988542in}{3.216143in}}%
\pgfpathlineto{\pgfqpoint{4.989112in}{3.103362in}}%
\pgfpathlineto{\pgfqpoint{4.989652in}{3.200843in}}%
\pgfpathlineto{\pgfqpoint{4.989698in}{3.259825in}}%
\pgfpathlineto{\pgfqpoint{4.989960in}{3.083040in}}%
\pgfpathlineto{\pgfqpoint{4.990746in}{3.184381in}}%
\pgfpathlineto{\pgfqpoint{4.991486in}{3.257502in}}%
\pgfpathlineto{\pgfqpoint{4.991055in}{3.084953in}}%
\pgfpathlineto{\pgfqpoint{4.991718in}{3.107560in}}%
\pgfpathlineto{\pgfqpoint{4.991733in}{3.098806in}}%
\pgfpathlineto{\pgfqpoint{4.992365in}{3.225171in}}%
\pgfpathlineto{\pgfqpoint{4.992642in}{3.195830in}}%
\pgfpathlineto{\pgfqpoint{4.993197in}{3.227108in}}%
\pgfpathlineto{\pgfqpoint{4.993475in}{3.090338in}}%
\pgfpathlineto{\pgfqpoint{4.993706in}{3.143681in}}%
\pgfpathlineto{\pgfqpoint{4.993721in}{3.141926in}}%
\pgfpathlineto{\pgfqpoint{4.994061in}{3.219376in}}%
\pgfpathlineto{\pgfqpoint{4.994122in}{3.273233in}}%
\pgfpathlineto{\pgfqpoint{4.994369in}{3.089851in}}%
\pgfpathlineto{\pgfqpoint{4.995078in}{3.146656in}}%
\pgfpathlineto{\pgfqpoint{4.995463in}{3.096457in}}%
\pgfpathlineto{\pgfqpoint{4.995910in}{3.232060in}}%
\pgfpathlineto{\pgfqpoint{4.996219in}{3.108834in}}%
\pgfpathlineto{\pgfqpoint{4.996234in}{3.106213in}}%
\pgfpathlineto{\pgfqpoint{4.996650in}{3.221197in}}%
\pgfpathlineto{\pgfqpoint{4.996727in}{3.248951in}}%
\pgfpathlineto{\pgfqpoint{4.996974in}{3.090049in}}%
\pgfpathlineto{\pgfqpoint{4.997652in}{3.196960in}}%
\pgfpathlineto{\pgfqpoint{4.997883in}{3.091391in}}%
\pgfpathlineto{\pgfqpoint{4.998531in}{3.239799in}}%
\pgfpathlineto{\pgfqpoint{4.998778in}{3.143386in}}%
\pgfpathlineto{\pgfqpoint{4.999317in}{3.258788in}}%
\pgfpathlineto{\pgfqpoint{4.998978in}{3.124549in}}%
\pgfpathlineto{\pgfqpoint{4.999872in}{3.130306in}}%
\pgfpathlineto{\pgfqpoint{5.001090in}{3.259666in}}%
\pgfpathlineto{\pgfqpoint{5.000643in}{3.118192in}}%
\pgfpathlineto{\pgfqpoint{5.001182in}{3.197377in}}%
\pgfpathlineto{\pgfqpoint{5.001383in}{3.082797in}}%
\pgfpathlineto{\pgfqpoint{5.001984in}{3.227561in}}%
\pgfpathlineto{\pgfqpoint{5.002292in}{3.191592in}}%
\pgfpathlineto{\pgfqpoint{5.003248in}{3.116191in}}%
\pgfpathlineto{\pgfqpoint{5.002570in}{3.228254in}}%
\pgfpathlineto{\pgfqpoint{5.003417in}{3.170942in}}%
\pgfpathlineto{\pgfqpoint{5.003741in}{3.276055in}}%
\pgfpathlineto{\pgfqpoint{5.003988in}{3.079964in}}%
\pgfpathlineto{\pgfqpoint{5.004512in}{3.175875in}}%
\pgfpathlineto{\pgfqpoint{5.004728in}{3.099884in}}%
\pgfpathlineto{\pgfqpoint{5.005529in}{3.251359in}}%
\pgfpathlineto{\pgfqpoint{5.005622in}{3.146087in}}%
\pgfpathlineto{\pgfqpoint{5.006346in}{3.221648in}}%
\pgfpathlineto{\pgfqpoint{5.006023in}{3.108498in}}%
\pgfpathlineto{\pgfqpoint{5.006593in}{3.136965in}}%
\pgfpathlineto{\pgfqpoint{5.007502in}{3.074326in}}%
\pgfpathlineto{\pgfqpoint{5.006701in}{3.235367in}}%
\pgfpathlineto{\pgfqpoint{5.007703in}{3.138693in}}%
\pgfpathlineto{\pgfqpoint{5.008165in}{3.271854in}}%
\pgfpathlineto{\pgfqpoint{5.008412in}{3.099015in}}%
\pgfpathlineto{\pgfqpoint{5.008813in}{3.149604in}}%
\pgfpathlineto{\pgfqpoint{5.008843in}{3.153276in}}%
\pgfpathlineto{\pgfqpoint{5.009028in}{3.235746in}}%
\pgfpathlineto{\pgfqpoint{5.009260in}{3.125357in}}%
\pgfpathlineto{\pgfqpoint{5.009969in}{3.180194in}}%
\pgfpathlineto{\pgfqpoint{5.010138in}{3.099555in}}%
\pgfpathlineto{\pgfqpoint{5.010770in}{3.219054in}}%
\pgfpathlineto{\pgfqpoint{5.011063in}{3.173749in}}%
\pgfpathlineto{\pgfqpoint{5.011125in}{3.233876in}}%
\pgfpathlineto{\pgfqpoint{5.011186in}{3.110193in}}%
\pgfpathlineto{\pgfqpoint{5.012188in}{3.212152in}}%
\pgfpathlineto{\pgfqpoint{5.013206in}{3.124675in}}%
\pgfpathlineto{\pgfqpoint{5.013298in}{3.181824in}}%
\pgfpathlineto{\pgfqpoint{5.013345in}{3.242176in}}%
\pgfpathlineto{\pgfqpoint{5.013838in}{3.121470in}}%
\pgfpathlineto{\pgfqpoint{5.014408in}{3.192802in}}%
\pgfpathlineto{\pgfqpoint{5.014886in}{3.119403in}}%
\pgfpathlineto{\pgfqpoint{5.014824in}{3.242136in}}%
\pgfpathlineto{\pgfqpoint{5.015117in}{3.239206in}}%
\pgfpathlineto{\pgfqpoint{5.015133in}{3.253609in}}%
\pgfpathlineto{\pgfqpoint{5.015611in}{3.118400in}}%
\pgfpathlineto{\pgfqpoint{5.016135in}{3.159237in}}%
\pgfpathlineto{\pgfqpoint{5.017044in}{3.245354in}}%
\pgfpathlineto{\pgfqpoint{5.016243in}{3.132697in}}%
\pgfpathlineto{\pgfqpoint{5.017075in}{3.190381in}}%
\pgfpathlineto{\pgfqpoint{5.017121in}{3.104803in}}%
\pgfpathlineto{\pgfqpoint{5.017784in}{3.248225in}}%
\pgfpathlineto{\pgfqpoint{5.018185in}{3.159243in}}%
\pgfpathlineto{\pgfqpoint{5.019110in}{3.110711in}}%
\pgfpathlineto{\pgfqpoint{5.018647in}{3.240256in}}%
\pgfpathlineto{\pgfqpoint{5.019248in}{3.203315in}}%
\pgfpathlineto{\pgfqpoint{5.019326in}{3.116427in}}%
\pgfpathlineto{\pgfqpoint{5.019557in}{3.229473in}}%
\pgfpathlineto{\pgfqpoint{5.020358in}{3.200231in}}%
\pgfpathlineto{\pgfqpoint{5.020728in}{3.231490in}}%
\pgfpathlineto{\pgfqpoint{5.020620in}{3.110510in}}%
\pgfpathlineto{\pgfqpoint{5.021329in}{3.152991in}}%
\pgfpathlineto{\pgfqpoint{5.021730in}{3.109001in}}%
\pgfpathlineto{\pgfqpoint{5.022193in}{3.233659in}}%
\pgfpathlineto{\pgfqpoint{5.022424in}{3.154911in}}%
\pgfpathlineto{\pgfqpoint{5.022963in}{3.229092in}}%
\pgfpathlineto{\pgfqpoint{5.022840in}{3.119717in}}%
\pgfpathlineto{\pgfqpoint{5.023549in}{3.163633in}}%
\pgfpathlineto{\pgfqpoint{5.023827in}{3.213027in}}%
\pgfpathlineto{\pgfqpoint{5.023765in}{3.116652in}}%
\pgfpathlineto{\pgfqpoint{5.024104in}{3.157299in}}%
\pgfpathlineto{\pgfqpoint{5.025029in}{3.116397in}}%
\pgfpathlineto{\pgfqpoint{5.024443in}{3.236758in}}%
\pgfpathlineto{\pgfqpoint{5.025168in}{3.214281in}}%
\pgfpathlineto{\pgfqpoint{5.025183in}{3.215632in}}%
\pgfpathlineto{\pgfqpoint{5.025214in}{3.155577in}}%
\pgfpathlineto{\pgfqpoint{5.025245in}{3.110441in}}%
\pgfpathlineto{\pgfqpoint{5.026216in}{3.224187in}}%
\pgfpathlineto{\pgfqpoint{5.026339in}{3.120549in}}%
\pgfpathlineto{\pgfqpoint{5.027387in}{3.250751in}}%
\pgfpathlineto{\pgfqpoint{5.027603in}{3.159423in}}%
\pgfpathlineto{\pgfqpoint{5.027650in}{3.083005in}}%
\pgfpathlineto{\pgfqpoint{5.028451in}{3.226569in}}%
\pgfpathlineto{\pgfqpoint{5.028713in}{3.148473in}}%
\pgfpathlineto{\pgfqpoint{5.028944in}{3.098497in}}%
\pgfpathlineto{\pgfqpoint{5.028867in}{3.238283in}}%
\pgfpathlineto{\pgfqpoint{5.029715in}{3.191792in}}%
\pgfpathlineto{\pgfqpoint{5.030347in}{3.265932in}}%
\pgfpathlineto{\pgfqpoint{5.030424in}{3.107893in}}%
\pgfpathlineto{\pgfqpoint{5.030810in}{3.175518in}}%
\pgfpathlineto{\pgfqpoint{5.031226in}{3.248519in}}%
\pgfpathlineto{\pgfqpoint{5.031149in}{3.087275in}}%
\pgfpathlineto{\pgfqpoint{5.031858in}{3.170425in}}%
\pgfpathlineto{\pgfqpoint{5.032474in}{3.121182in}}%
\pgfpathlineto{\pgfqpoint{5.032582in}{3.242486in}}%
\pgfpathlineto{\pgfqpoint{5.032952in}{3.180126in}}%
\pgfpathlineto{\pgfqpoint{5.034047in}{3.230896in}}%
\pgfpathlineto{\pgfqpoint{5.033384in}{3.088448in}}%
\pgfpathlineto{\pgfqpoint{5.034078in}{3.208057in}}%
\pgfpathlineto{\pgfqpoint{5.034648in}{3.096100in}}%
\pgfpathlineto{\pgfqpoint{5.034370in}{3.231180in}}%
\pgfpathlineto{\pgfqpoint{5.035203in}{3.167183in}}%
\pgfpathlineto{\pgfqpoint{5.035850in}{3.226467in}}%
\pgfpathlineto{\pgfqpoint{5.035974in}{3.123768in}}%
\pgfpathlineto{\pgfqpoint{5.036004in}{3.118957in}}%
\pgfpathlineto{\pgfqpoint{5.036097in}{3.181668in}}%
\pgfpathlineto{\pgfqpoint{5.037006in}{3.238511in}}%
\pgfpathlineto{\pgfqpoint{5.036883in}{3.099035in}}%
\pgfpathlineto{\pgfqpoint{5.037068in}{3.143709in}}%
\pgfpathlineto{\pgfqpoint{5.038147in}{3.099550in}}%
\pgfpathlineto{\pgfqpoint{5.037885in}{3.250842in}}%
\pgfpathlineto{\pgfqpoint{5.038178in}{3.125598in}}%
\pgfpathlineto{\pgfqpoint{5.038471in}{3.243885in}}%
\pgfpathlineto{\pgfqpoint{5.038872in}{3.112687in}}%
\pgfpathlineto{\pgfqpoint{5.039303in}{3.156236in}}%
\pgfpathlineto{\pgfqpoint{5.039981in}{3.260696in}}%
\pgfpathlineto{\pgfqpoint{5.039504in}{3.102002in}}%
\pgfpathlineto{\pgfqpoint{5.040351in}{3.130654in}}%
\pgfpathlineto{\pgfqpoint{5.040783in}{3.088300in}}%
\pgfpathlineto{\pgfqpoint{5.040860in}{3.259737in}}%
\pgfpathlineto{\pgfqpoint{5.041353in}{3.155977in}}%
\pgfpathlineto{\pgfqpoint{5.041770in}{3.234583in}}%
\pgfpathlineto{\pgfqpoint{5.041646in}{3.112562in}}%
\pgfpathlineto{\pgfqpoint{5.042463in}{3.173713in}}%
\pgfpathlineto{\pgfqpoint{5.043388in}{3.101267in}}%
\pgfpathlineto{\pgfqpoint{5.043080in}{3.214136in}}%
\pgfpathlineto{\pgfqpoint{5.043588in}{3.144055in}}%
\pgfpathlineto{\pgfqpoint{5.044375in}{3.252965in}}%
\pgfpathlineto{\pgfqpoint{5.044113in}{3.108165in}}%
\pgfpathlineto{\pgfqpoint{5.044683in}{3.141174in}}%
\pgfpathlineto{\pgfqpoint{5.044791in}{3.111503in}}%
\pgfpathlineto{\pgfqpoint{5.045300in}{3.214832in}}%
\pgfpathlineto{\pgfqpoint{5.045747in}{3.173085in}}%
\pgfpathlineto{\pgfqpoint{5.046132in}{3.224642in}}%
\pgfpathlineto{\pgfqpoint{5.046517in}{3.122421in}}%
\pgfpathlineto{\pgfqpoint{5.046533in}{3.115356in}}%
\pgfpathlineto{\pgfqpoint{5.046980in}{3.240278in}}%
\pgfpathlineto{\pgfqpoint{5.047488in}{3.158693in}}%
\pgfpathlineto{\pgfqpoint{5.047874in}{3.229274in}}%
\pgfpathlineto{\pgfqpoint{5.047781in}{3.107803in}}%
\pgfpathlineto{\pgfqpoint{5.048598in}{3.173696in}}%
\pgfpathlineto{\pgfqpoint{5.049153in}{3.106271in}}%
\pgfpathlineto{\pgfqpoint{5.049600in}{3.248736in}}%
\pgfpathlineto{\pgfqpoint{5.049616in}{3.260906in}}%
\pgfpathlineto{\pgfqpoint{5.050016in}{3.109118in}}%
\pgfpathlineto{\pgfqpoint{5.050587in}{3.157758in}}%
\pgfpathlineto{\pgfqpoint{5.050895in}{3.126363in}}%
\pgfpathlineto{\pgfqpoint{5.051404in}{3.230759in}}%
\pgfpathlineto{\pgfqpoint{5.051681in}{3.166036in}}%
\pgfpathlineto{\pgfqpoint{5.052267in}{3.209408in}}%
\pgfpathlineto{\pgfqpoint{5.051758in}{3.138461in}}%
\pgfpathlineto{\pgfqpoint{5.052622in}{3.149026in}}%
\pgfpathlineto{\pgfqpoint{5.052652in}{3.112878in}}%
\pgfpathlineto{\pgfqpoint{5.053331in}{3.218294in}}%
\pgfpathlineto{\pgfqpoint{5.053747in}{3.129713in}}%
\pgfpathlineto{\pgfqpoint{5.053901in}{3.127622in}}%
\pgfpathlineto{\pgfqpoint{5.054040in}{3.235783in}}%
\pgfpathlineto{\pgfqpoint{5.054178in}{3.199107in}}%
\pgfpathlineto{\pgfqpoint{5.054209in}{3.243635in}}%
\pgfpathlineto{\pgfqpoint{5.054425in}{3.113314in}}%
\pgfpathlineto{\pgfqpoint{5.055242in}{3.131974in}}%
\pgfpathlineto{\pgfqpoint{5.055643in}{3.121113in}}%
\pgfpathlineto{\pgfqpoint{5.055705in}{3.201461in}}%
\pgfpathlineto{\pgfqpoint{5.055751in}{3.201423in}}%
\pgfpathlineto{\pgfqpoint{5.056614in}{3.214931in}}%
\pgfpathlineto{\pgfqpoint{5.056367in}{3.118594in}}%
\pgfpathlineto{\pgfqpoint{5.056768in}{3.163272in}}%
\pgfpathlineto{\pgfqpoint{5.057400in}{3.106822in}}%
\pgfpathlineto{\pgfqpoint{5.057169in}{3.221444in}}%
\pgfpathlineto{\pgfqpoint{5.057878in}{3.148844in}}%
\pgfpathlineto{\pgfqpoint{5.058649in}{3.222447in}}%
\pgfpathlineto{\pgfqpoint{5.058140in}{3.115580in}}%
\pgfpathlineto{\pgfqpoint{5.058972in}{3.157443in}}%
\pgfpathlineto{\pgfqpoint{5.059157in}{3.113468in}}%
\pgfpathlineto{\pgfqpoint{5.059265in}{3.244316in}}%
\pgfpathlineto{\pgfqpoint{5.060067in}{3.148453in}}%
\pgfpathlineto{\pgfqpoint{5.060144in}{3.262035in}}%
\pgfpathlineto{\pgfqpoint{5.061115in}{3.129285in}}%
\pgfpathlineto{\pgfqpoint{5.061208in}{3.193953in}}%
\pgfpathlineto{\pgfqpoint{5.061778in}{3.124913in}}%
\pgfpathlineto{\pgfqpoint{5.061886in}{3.210287in}}%
\pgfpathlineto{\pgfqpoint{5.062333in}{3.170892in}}%
\pgfpathlineto{\pgfqpoint{5.062980in}{3.217390in}}%
\pgfpathlineto{\pgfqpoint{5.062657in}{3.112140in}}%
\pgfpathlineto{\pgfqpoint{5.063427in}{3.151485in}}%
\pgfpathlineto{\pgfqpoint{5.063844in}{3.237872in}}%
\pgfpathlineto{\pgfqpoint{5.063551in}{3.131467in}}%
\pgfpathlineto{\pgfqpoint{5.064599in}{3.216188in}}%
\pgfpathlineto{\pgfqpoint{5.064876in}{3.122247in}}%
\pgfpathlineto{\pgfqpoint{5.065339in}{3.221585in}}%
\pgfpathlineto{\pgfqpoint{5.065724in}{3.152098in}}%
\pgfpathlineto{\pgfqpoint{5.066819in}{3.226519in}}%
\pgfpathlineto{\pgfqpoint{5.066171in}{3.123603in}}%
\pgfpathlineto{\pgfqpoint{5.066849in}{3.185633in}}%
\pgfpathlineto{\pgfqpoint{5.067913in}{3.115803in}}%
\pgfpathlineto{\pgfqpoint{5.067559in}{3.214374in}}%
\pgfpathlineto{\pgfqpoint{5.067959in}{3.159442in}}%
\pgfpathlineto{\pgfqpoint{5.068900in}{3.229371in}}%
\pgfpathlineto{\pgfqpoint{5.068792in}{3.114543in}}%
\pgfpathlineto{\pgfqpoint{5.069085in}{3.203974in}}%
\pgfpathlineto{\pgfqpoint{5.070087in}{3.123120in}}%
\pgfpathlineto{\pgfqpoint{5.069778in}{3.257036in}}%
\pgfpathlineto{\pgfqpoint{5.070287in}{3.174366in}}%
\pgfpathlineto{\pgfqpoint{5.070626in}{3.196638in}}%
\pgfpathlineto{\pgfqpoint{5.070564in}{3.158538in}}%
\pgfpathlineto{\pgfqpoint{5.070703in}{3.162900in}}%
\pgfpathlineto{\pgfqpoint{5.071412in}{3.092682in}}%
\pgfpathlineto{\pgfqpoint{5.071135in}{3.222905in}}%
\pgfpathlineto{\pgfqpoint{5.071798in}{3.173446in}}%
\pgfpathlineto{\pgfqpoint{5.072291in}{3.108455in}}%
\pgfpathlineto{\pgfqpoint{5.072013in}{3.224360in}}%
\pgfpathlineto{\pgfqpoint{5.072907in}{3.158908in}}%
\pgfpathlineto{\pgfqpoint{5.073493in}{3.228636in}}%
\pgfpathlineto{\pgfqpoint{5.073632in}{3.121836in}}%
\pgfpathlineto{\pgfqpoint{5.074002in}{3.142092in}}%
\pgfpathlineto{\pgfqpoint{5.074511in}{3.122077in}}%
\pgfpathlineto{\pgfqpoint{5.074249in}{3.208878in}}%
\pgfpathlineto{\pgfqpoint{5.074958in}{3.186189in}}%
\pgfpathlineto{\pgfqpoint{5.074988in}{3.237206in}}%
\pgfpathlineto{\pgfqpoint{5.075127in}{3.125846in}}%
\pgfpathlineto{\pgfqpoint{5.076052in}{3.170915in}}%
\pgfpathlineto{\pgfqpoint{5.076468in}{3.230681in}}%
\pgfpathlineto{\pgfqpoint{5.076700in}{3.119961in}}%
\pgfpathlineto{\pgfqpoint{5.077131in}{3.153405in}}%
\pgfpathlineto{\pgfqpoint{5.077501in}{3.131374in}}%
\pgfpathlineto{\pgfqpoint{5.077208in}{3.225171in}}%
\pgfpathlineto{\pgfqpoint{5.078102in}{3.156366in}}%
\pgfpathlineto{\pgfqpoint{5.078703in}{3.230156in}}%
\pgfpathlineto{\pgfqpoint{5.078426in}{3.122962in}}%
\pgfpathlineto{\pgfqpoint{5.079212in}{3.155733in}}%
\pgfpathlineto{\pgfqpoint{5.079428in}{3.242653in}}%
\pgfpathlineto{\pgfqpoint{5.079721in}{3.129857in}}%
\pgfpathlineto{\pgfqpoint{5.080368in}{3.168025in}}%
\pgfpathlineto{\pgfqpoint{5.081046in}{3.094862in}}%
\pgfpathlineto{\pgfqpoint{5.080784in}{3.213672in}}%
\pgfpathlineto{\pgfqpoint{5.081478in}{3.167576in}}%
\pgfpathlineto{\pgfqpoint{5.081509in}{3.177797in}}%
\pgfpathlineto{\pgfqpoint{5.081663in}{3.240318in}}%
\pgfpathlineto{\pgfqpoint{5.081925in}{3.086088in}}%
\pgfpathlineto{\pgfqpoint{5.082588in}{3.183380in}}%
\pgfpathlineto{\pgfqpoint{5.083282in}{3.111570in}}%
\pgfpathlineto{\pgfqpoint{5.083143in}{3.229060in}}%
\pgfpathlineto{\pgfqpoint{5.083713in}{3.157788in}}%
\pgfpathlineto{\pgfqpoint{5.084638in}{3.243229in}}%
\pgfpathlineto{\pgfqpoint{5.084160in}{3.121699in}}%
\pgfpathlineto{\pgfqpoint{5.084839in}{3.190831in}}%
\pgfpathlineto{\pgfqpoint{5.085640in}{3.122406in}}%
\pgfpathlineto{\pgfqpoint{5.085363in}{3.219402in}}%
\pgfpathlineto{\pgfqpoint{5.085948in}{3.181912in}}%
\pgfpathlineto{\pgfqpoint{5.086858in}{3.239821in}}%
\pgfpathlineto{\pgfqpoint{5.086318in}{3.120283in}}%
\pgfpathlineto{\pgfqpoint{5.087027in}{3.151557in}}%
\pgfpathlineto{\pgfqpoint{5.087151in}{3.106784in}}%
\pgfpathlineto{\pgfqpoint{5.087474in}{3.219575in}}%
\pgfpathlineto{\pgfqpoint{5.088091in}{3.163776in}}%
\pgfpathlineto{\pgfqpoint{5.089062in}{3.239433in}}%
\pgfpathlineto{\pgfqpoint{5.088554in}{3.137150in}}%
\pgfpathlineto{\pgfqpoint{5.089186in}{3.153785in}}%
\pgfpathlineto{\pgfqpoint{5.090141in}{3.210758in}}%
\pgfpathlineto{\pgfqpoint{5.089355in}{3.136801in}}%
\pgfpathlineto{\pgfqpoint{5.090265in}{3.148461in}}%
\pgfpathlineto{\pgfqpoint{5.090681in}{3.105570in}}%
\pgfpathlineto{\pgfqpoint{5.090557in}{3.207403in}}%
\pgfpathlineto{\pgfqpoint{5.091236in}{3.187808in}}%
\pgfpathlineto{\pgfqpoint{5.091297in}{3.240097in}}%
\pgfpathlineto{\pgfqpoint{5.091559in}{3.085251in}}%
\pgfpathlineto{\pgfqpoint{5.092268in}{3.122209in}}%
\pgfpathlineto{\pgfqpoint{5.092438in}{3.103653in}}%
\pgfpathlineto{\pgfqpoint{5.092777in}{3.235803in}}%
\pgfpathlineto{\pgfqpoint{5.093085in}{3.167344in}}%
\pgfpathlineto{\pgfqpoint{5.093409in}{3.219928in}}%
\pgfpathlineto{\pgfqpoint{5.093795in}{3.126987in}}%
\pgfpathlineto{\pgfqpoint{5.094165in}{3.131183in}}%
\pgfpathlineto{\pgfqpoint{5.095059in}{3.112344in}}%
\pgfpathlineto{\pgfqpoint{5.094257in}{3.234924in}}%
\pgfpathlineto{\pgfqpoint{5.095259in}{3.139679in}}%
\pgfpathlineto{\pgfqpoint{5.095274in}{3.138605in}}%
\pgfpathlineto{\pgfqpoint{5.095321in}{3.195417in}}%
\pgfpathlineto{\pgfqpoint{5.095737in}{3.221887in}}%
\pgfpathlineto{\pgfqpoint{5.095937in}{3.120435in}}%
\pgfpathlineto{\pgfqpoint{5.096384in}{3.162993in}}%
\pgfpathlineto{\pgfqpoint{5.096770in}{3.107905in}}%
\pgfpathlineto{\pgfqpoint{5.096477in}{3.230254in}}%
\pgfpathlineto{\pgfqpoint{5.097525in}{3.133109in}}%
\pgfpathlineto{\pgfqpoint{5.097679in}{3.120841in}}%
\pgfpathlineto{\pgfqpoint{5.098696in}{3.246352in}}%
\pgfpathlineto{\pgfqpoint{5.098774in}{3.137756in}}%
\pgfpathlineto{\pgfqpoint{5.099853in}{3.157538in}}%
\pgfpathlineto{\pgfqpoint{5.100916in}{3.233682in}}%
\pgfpathlineto{\pgfqpoint{5.100315in}{3.104964in}}%
\pgfpathlineto{\pgfqpoint{5.100978in}{3.164705in}}%
\pgfpathlineto{\pgfqpoint{5.102057in}{3.097591in}}%
\pgfpathlineto{\pgfqpoint{5.101517in}{3.210495in}}%
\pgfpathlineto{\pgfqpoint{5.102103in}{3.153020in}}%
\pgfpathlineto{\pgfqpoint{5.102396in}{3.241107in}}%
\pgfpathlineto{\pgfqpoint{5.102920in}{3.121218in}}%
\pgfpathlineto{\pgfqpoint{5.103244in}{3.176337in}}%
\pgfpathlineto{\pgfqpoint{5.103429in}{3.121569in}}%
\pgfpathlineto{\pgfqpoint{5.103891in}{3.245001in}}%
\pgfpathlineto{\pgfqpoint{5.104354in}{3.172771in}}%
\pgfpathlineto{\pgfqpoint{5.104801in}{3.220382in}}%
\pgfpathlineto{\pgfqpoint{5.104693in}{3.103926in}}%
\pgfpathlineto{\pgfqpoint{5.105402in}{3.154860in}}%
\pgfpathlineto{\pgfqpoint{5.106404in}{3.115261in}}%
\pgfpathlineto{\pgfqpoint{5.106111in}{3.226845in}}%
\pgfpathlineto{\pgfqpoint{5.106496in}{3.164151in}}%
\pgfpathlineto{\pgfqpoint{5.107591in}{3.244564in}}%
\pgfpathlineto{\pgfqpoint{5.107313in}{3.101246in}}%
\pgfpathlineto{\pgfqpoint{5.107622in}{3.200949in}}%
\pgfpathlineto{\pgfqpoint{5.108624in}{3.127279in}}%
\pgfpathlineto{\pgfqpoint{5.108315in}{3.259858in}}%
\pgfpathlineto{\pgfqpoint{5.108747in}{3.172293in}}%
\pgfpathlineto{\pgfqpoint{5.109656in}{3.217646in}}%
\pgfpathlineto{\pgfqpoint{5.109549in}{3.140941in}}%
\pgfpathlineto{\pgfqpoint{5.109733in}{3.157565in}}%
\pgfpathlineto{\pgfqpoint{5.109934in}{3.095360in}}%
\pgfpathlineto{\pgfqpoint{5.109811in}{3.208232in}}%
\pgfpathlineto{\pgfqpoint{5.110504in}{3.188618in}}%
\pgfpathlineto{\pgfqpoint{5.110550in}{3.226207in}}%
\pgfpathlineto{\pgfqpoint{5.110828in}{3.119603in}}%
\pgfpathlineto{\pgfqpoint{5.111583in}{3.155908in}}%
\pgfpathlineto{\pgfqpoint{5.112015in}{3.233176in}}%
\pgfpathlineto{\pgfqpoint{5.111676in}{3.113000in}}%
\pgfpathlineto{\pgfqpoint{5.112786in}{3.196662in}}%
\pgfpathlineto{\pgfqpoint{5.113433in}{3.121677in}}%
\pgfpathlineto{\pgfqpoint{5.113510in}{3.252187in}}%
\pgfpathlineto{\pgfqpoint{5.113942in}{3.146743in}}%
\pgfpathlineto{\pgfqpoint{5.114420in}{3.225941in}}%
\pgfpathlineto{\pgfqpoint{5.114312in}{3.123287in}}%
\pgfpathlineto{\pgfqpoint{5.115082in}{3.165995in}}%
\pgfpathlineto{\pgfqpoint{5.116038in}{3.118983in}}%
\pgfpathlineto{\pgfqpoint{5.115730in}{3.217154in}}%
\pgfpathlineto{\pgfqpoint{5.116146in}{3.195210in}}%
\pgfpathlineto{\pgfqpoint{5.117056in}{3.241471in}}%
\pgfpathlineto{\pgfqpoint{5.116932in}{3.117968in}}%
\pgfpathlineto{\pgfqpoint{5.117102in}{3.185910in}}%
\pgfpathlineto{\pgfqpoint{5.117348in}{3.131859in}}%
\pgfpathlineto{\pgfqpoint{5.117934in}{3.249668in}}%
\pgfpathlineto{\pgfqpoint{5.118227in}{3.146199in}}%
\pgfpathlineto{\pgfqpoint{5.119167in}{3.131110in}}%
\pgfpathlineto{\pgfqpoint{5.118566in}{3.210562in}}%
\pgfpathlineto{\pgfqpoint{5.119244in}{3.190605in}}%
\pgfpathlineto{\pgfqpoint{5.120169in}{3.229265in}}%
\pgfpathlineto{\pgfqpoint{5.119553in}{3.105617in}}%
\pgfpathlineto{\pgfqpoint{5.120339in}{3.194074in}}%
\pgfpathlineto{\pgfqpoint{5.120848in}{3.129337in}}%
\pgfpathlineto{\pgfqpoint{5.120771in}{3.220550in}}%
\pgfpathlineto{\pgfqpoint{5.121418in}{3.205536in}}%
\pgfpathlineto{\pgfqpoint{5.122158in}{3.129622in}}%
\pgfpathlineto{\pgfqpoint{5.122266in}{3.243825in}}%
\pgfpathlineto{\pgfqpoint{5.122697in}{3.145193in}}%
\pgfpathlineto{\pgfqpoint{5.123144in}{3.251594in}}%
\pgfpathlineto{\pgfqpoint{5.123252in}{3.121347in}}%
\pgfpathlineto{\pgfqpoint{5.123869in}{3.209202in}}%
\pgfpathlineto{\pgfqpoint{5.124748in}{3.122163in}}%
\pgfpathlineto{\pgfqpoint{5.124038in}{3.228162in}}%
\pgfpathlineto{\pgfqpoint{5.125040in}{3.152282in}}%
\pgfpathlineto{\pgfqpoint{5.125950in}{3.229383in}}%
\pgfpathlineto{\pgfqpoint{5.125472in}{3.126885in}}%
\pgfpathlineto{\pgfqpoint{5.126135in}{3.158760in}}%
\pgfpathlineto{\pgfqpoint{5.126967in}{3.126252in}}%
\pgfpathlineto{\pgfqpoint{5.126674in}{3.238465in}}%
\pgfpathlineto{\pgfqpoint{5.127245in}{3.145664in}}%
\pgfpathlineto{\pgfqpoint{5.127568in}{3.231129in}}%
\pgfpathlineto{\pgfqpoint{5.128262in}{3.127062in}}%
\pgfpathlineto{\pgfqpoint{5.128416in}{3.195744in}}%
\pgfpathlineto{\pgfqpoint{5.129172in}{3.125972in}}%
\pgfpathlineto{\pgfqpoint{5.129064in}{3.237280in}}%
\pgfpathlineto{\pgfqpoint{5.129557in}{3.149105in}}%
\pgfpathlineto{\pgfqpoint{5.130035in}{3.126567in}}%
\pgfpathlineto{\pgfqpoint{5.129804in}{3.212888in}}%
\pgfpathlineto{\pgfqpoint{5.130328in}{3.177836in}}%
\pgfpathlineto{\pgfqpoint{5.130389in}{3.223940in}}%
\pgfpathlineto{\pgfqpoint{5.130482in}{3.129778in}}%
\pgfpathlineto{\pgfqpoint{5.131361in}{3.142139in}}%
\pgfpathlineto{\pgfqpoint{5.131407in}{3.112283in}}%
\pgfpathlineto{\pgfqpoint{5.131885in}{3.246213in}}%
\pgfpathlineto{\pgfqpoint{5.132470in}{3.138456in}}%
\pgfpathlineto{\pgfqpoint{5.132763in}{3.256984in}}%
\pgfpathlineto{\pgfqpoint{5.133149in}{3.123339in}}%
\pgfpathlineto{\pgfqpoint{5.133704in}{3.176209in}}%
\pgfpathlineto{\pgfqpoint{5.134366in}{3.121833in}}%
\pgfpathlineto{\pgfqpoint{5.134104in}{3.228669in}}%
\pgfpathlineto{\pgfqpoint{5.134798in}{3.176634in}}%
\pgfpathlineto{\pgfqpoint{5.134983in}{3.221597in}}%
\pgfpathlineto{\pgfqpoint{5.135106in}{3.120752in}}%
\pgfpathlineto{\pgfqpoint{5.135908in}{3.184524in}}%
\pgfpathlineto{\pgfqpoint{5.136586in}{3.111703in}}%
\pgfpathlineto{\pgfqpoint{5.136293in}{3.223935in}}%
\pgfpathlineto{\pgfqpoint{5.137018in}{3.150822in}}%
\pgfpathlineto{\pgfqpoint{5.137788in}{3.235301in}}%
\pgfpathlineto{\pgfqpoint{5.137881in}{3.119516in}}%
\pgfpathlineto{\pgfqpoint{5.138143in}{3.179066in}}%
\pgfpathlineto{\pgfqpoint{5.138390in}{3.119364in}}%
\pgfpathlineto{\pgfqpoint{5.138683in}{3.227000in}}%
\pgfpathlineto{\pgfqpoint{5.139253in}{3.170163in}}%
\pgfpathlineto{\pgfqpoint{5.140008in}{3.221418in}}%
\pgfpathlineto{\pgfqpoint{5.140116in}{3.125381in}}%
\pgfpathlineto{\pgfqpoint{5.140347in}{3.171323in}}%
\pgfpathlineto{\pgfqpoint{5.141026in}{3.118234in}}%
\pgfpathlineto{\pgfqpoint{5.140764in}{3.214775in}}%
\pgfpathlineto{\pgfqpoint{5.141426in}{3.171296in}}%
\pgfpathlineto{\pgfqpoint{5.141503in}{3.257935in}}%
\pgfpathlineto{\pgfqpoint{5.141873in}{3.127585in}}%
\pgfpathlineto{\pgfqpoint{5.142490in}{3.143958in}}%
\pgfpathlineto{\pgfqpoint{5.142767in}{3.128468in}}%
\pgfpathlineto{\pgfqpoint{5.143276in}{3.214168in}}%
\pgfpathlineto{\pgfqpoint{5.143292in}{3.219383in}}%
\pgfpathlineto{\pgfqpoint{5.143600in}{3.122147in}}%
\pgfpathlineto{\pgfqpoint{5.144201in}{3.162938in}}%
\pgfpathlineto{\pgfqpoint{5.144725in}{3.122848in}}%
\pgfpathlineto{\pgfqpoint{5.145018in}{3.210187in}}%
\pgfpathlineto{\pgfqpoint{5.145172in}{3.209906in}}%
\pgfpathlineto{\pgfqpoint{5.145203in}{3.232321in}}%
\pgfpathlineto{\pgfqpoint{5.146205in}{3.124555in}}%
\pgfpathlineto{\pgfqpoint{5.147407in}{3.226130in}}%
\pgfpathlineto{\pgfqpoint{5.147454in}{3.174726in}}%
\pgfpathlineto{\pgfqpoint{5.147500in}{3.113281in}}%
\pgfpathlineto{\pgfqpoint{5.148486in}{3.229071in}}%
\pgfpathlineto{\pgfqpoint{5.148563in}{3.167890in}}%
\pgfpathlineto{\pgfqpoint{5.149257in}{3.127556in}}%
\pgfpathlineto{\pgfqpoint{5.149380in}{3.223580in}}%
\pgfpathlineto{\pgfqpoint{5.149612in}{3.207488in}}%
\pgfpathlineto{\pgfqpoint{5.149627in}{3.221291in}}%
\pgfpathlineto{\pgfqpoint{5.149720in}{3.123949in}}%
\pgfpathlineto{\pgfqpoint{5.150614in}{3.131059in}}%
\pgfpathlineto{\pgfqpoint{5.150629in}{3.118988in}}%
\pgfpathlineto{\pgfqpoint{5.151122in}{3.259715in}}%
\pgfpathlineto{\pgfqpoint{5.151723in}{3.123012in}}%
\pgfpathlineto{\pgfqpoint{5.152001in}{3.243076in}}%
\pgfpathlineto{\pgfqpoint{5.152371in}{3.119532in}}%
\pgfpathlineto{\pgfqpoint{5.152910in}{3.224206in}}%
\pgfpathlineto{\pgfqpoint{5.153203in}{3.125441in}}%
\pgfpathlineto{\pgfqpoint{5.154036in}{3.177763in}}%
\pgfpathlineto{\pgfqpoint{5.154806in}{3.233341in}}%
\pgfpathlineto{\pgfqpoint{5.154128in}{3.115374in}}%
\pgfpathlineto{\pgfqpoint{5.155130in}{3.187066in}}%
\pgfpathlineto{\pgfqpoint{5.156070in}{3.114787in}}%
\pgfpathlineto{\pgfqpoint{5.155531in}{3.244800in}}%
\pgfpathlineto{\pgfqpoint{5.156255in}{3.151453in}}%
\pgfpathlineto{\pgfqpoint{5.156394in}{3.227698in}}%
\pgfpathlineto{\pgfqpoint{5.157103in}{3.123250in}}%
\pgfpathlineto{\pgfqpoint{5.157350in}{3.159186in}}%
\pgfpathlineto{\pgfqpoint{5.158013in}{3.114882in}}%
\pgfpathlineto{\pgfqpoint{5.158136in}{3.217982in}}%
\pgfpathlineto{\pgfqpoint{5.158460in}{3.152276in}}%
\pgfpathlineto{\pgfqpoint{5.158984in}{3.227436in}}%
\pgfpathlineto{\pgfqpoint{5.158891in}{3.128207in}}%
\pgfpathlineto{\pgfqpoint{5.159554in}{3.170409in}}%
\pgfpathlineto{\pgfqpoint{5.159755in}{3.113447in}}%
\pgfpathlineto{\pgfqpoint{5.159893in}{3.227085in}}%
\pgfpathlineto{\pgfqpoint{5.160664in}{3.146378in}}%
\pgfpathlineto{\pgfqpoint{5.160726in}{3.233114in}}%
\pgfpathlineto{\pgfqpoint{5.161512in}{3.109911in}}%
\pgfpathlineto{\pgfqpoint{5.161789in}{3.184939in}}%
\pgfpathlineto{\pgfqpoint{5.162498in}{3.240791in}}%
\pgfpathlineto{\pgfqpoint{5.162236in}{3.126961in}}%
\pgfpathlineto{\pgfqpoint{5.162838in}{3.159765in}}%
\pgfpathlineto{\pgfqpoint{5.163269in}{3.117375in}}%
\pgfpathlineto{\pgfqpoint{5.163393in}{3.247517in}}%
\pgfpathlineto{\pgfqpoint{5.163855in}{3.179621in}}%
\pgfpathlineto{\pgfqpoint{5.164425in}{3.223992in}}%
\pgfpathlineto{\pgfqpoint{5.164102in}{3.139097in}}%
\pgfpathlineto{\pgfqpoint{5.164795in}{3.148063in}}%
\pgfpathlineto{\pgfqpoint{5.165011in}{3.094472in}}%
\pgfpathlineto{\pgfqpoint{5.165150in}{3.247360in}}%
\pgfpathlineto{\pgfqpoint{5.165890in}{3.153880in}}%
\pgfpathlineto{\pgfqpoint{5.166768in}{3.128417in}}%
\pgfpathlineto{\pgfqpoint{5.166013in}{3.232895in}}%
\pgfpathlineto{\pgfqpoint{5.166815in}{3.183250in}}%
\pgfpathlineto{\pgfqpoint{5.166907in}{3.224834in}}%
\pgfpathlineto{\pgfqpoint{5.167616in}{3.116293in}}%
\pgfpathlineto{\pgfqpoint{5.167909in}{3.173922in}}%
\pgfpathlineto{\pgfqpoint{5.168618in}{3.217426in}}%
\pgfpathlineto{\pgfqpoint{5.168510in}{3.112592in}}%
\pgfpathlineto{\pgfqpoint{5.168973in}{3.163986in}}%
\pgfpathlineto{\pgfqpoint{5.169389in}{3.113967in}}%
\pgfpathlineto{\pgfqpoint{5.169528in}{3.217697in}}%
\pgfpathlineto{\pgfqpoint{5.170098in}{3.143070in}}%
\pgfpathlineto{\pgfqpoint{5.170252in}{3.122237in}}%
\pgfpathlineto{\pgfqpoint{5.170607in}{3.216971in}}%
\pgfpathlineto{\pgfqpoint{5.171007in}{3.163950in}}%
\pgfpathlineto{\pgfqpoint{5.171254in}{3.240703in}}%
\pgfpathlineto{\pgfqpoint{5.171639in}{3.108288in}}%
\pgfpathlineto{\pgfqpoint{5.172164in}{3.207097in}}%
\pgfpathlineto{\pgfqpoint{5.173119in}{3.122259in}}%
\pgfpathlineto{\pgfqpoint{5.172996in}{3.230968in}}%
\pgfpathlineto{\pgfqpoint{5.173273in}{3.204124in}}%
\pgfpathlineto{\pgfqpoint{5.173767in}{3.129988in}}%
\pgfpathlineto{\pgfqpoint{5.173905in}{3.205882in}}%
\pgfpathlineto{\pgfqpoint{5.174430in}{3.144865in}}%
\pgfpathlineto{\pgfqpoint{5.174753in}{3.225119in}}%
\pgfpathlineto{\pgfqpoint{5.175339in}{3.093146in}}%
\pgfpathlineto{\pgfqpoint{5.175493in}{3.131469in}}%
\pgfpathlineto{\pgfqpoint{5.176372in}{3.112589in}}%
\pgfpathlineto{\pgfqpoint{5.176495in}{3.223841in}}%
\pgfpathlineto{\pgfqpoint{5.176511in}{3.237426in}}%
\pgfpathlineto{\pgfqpoint{5.177281in}{3.117786in}}%
\pgfpathlineto{\pgfqpoint{5.177528in}{3.156775in}}%
\pgfpathlineto{\pgfqpoint{5.178114in}{3.113976in}}%
\pgfpathlineto{\pgfqpoint{5.178237in}{3.209984in}}%
\pgfpathlineto{\pgfqpoint{5.178407in}{3.208700in}}%
\pgfpathlineto{\pgfqpoint{5.179131in}{3.237083in}}%
\pgfpathlineto{\pgfqpoint{5.179023in}{3.075798in}}%
\pgfpathlineto{\pgfqpoint{5.179470in}{3.181139in}}%
\pgfpathlineto{\pgfqpoint{5.179763in}{3.107706in}}%
\pgfpathlineto{\pgfqpoint{5.180195in}{3.238138in}}%
\pgfpathlineto{\pgfqpoint{5.180595in}{3.150069in}}%
\pgfpathlineto{\pgfqpoint{5.180919in}{3.225946in}}%
\pgfpathlineto{\pgfqpoint{5.180780in}{3.109705in}}%
\pgfpathlineto{\pgfqpoint{5.181597in}{3.152976in}}%
\pgfpathlineto{\pgfqpoint{5.181628in}{3.106616in}}%
\pgfpathlineto{\pgfqpoint{5.181767in}{3.223619in}}%
\pgfpathlineto{\pgfqpoint{5.182723in}{3.114273in}}%
\pgfpathlineto{\pgfqpoint{5.183694in}{3.230452in}}%
\pgfpathlineto{\pgfqpoint{5.184064in}{3.181516in}}%
\pgfpathlineto{\pgfqpoint{5.184249in}{3.093824in}}%
\pgfpathlineto{\pgfqpoint{5.184603in}{3.225627in}}%
\pgfpathlineto{\pgfqpoint{5.185174in}{3.142752in}}%
\pgfpathlineto{\pgfqpoint{5.186160in}{3.228872in}}%
\pgfpathlineto{\pgfqpoint{5.186037in}{3.092077in}}%
\pgfpathlineto{\pgfqpoint{5.186361in}{3.198954in}}%
\pgfpathlineto{\pgfqpoint{5.187131in}{3.115228in}}%
\pgfpathlineto{\pgfqpoint{5.186993in}{3.220732in}}%
\pgfpathlineto{\pgfqpoint{5.187486in}{3.183154in}}%
\pgfpathlineto{\pgfqpoint{5.187779in}{3.072410in}}%
\pgfpathlineto{\pgfqpoint{5.188102in}{3.231373in}}%
\pgfpathlineto{\pgfqpoint{5.188673in}{3.090213in}}%
\pgfpathlineto{\pgfqpoint{5.189397in}{3.077114in}}%
\pgfpathlineto{\pgfqpoint{5.189844in}{3.234504in}}%
\pgfpathlineto{\pgfqpoint{5.190430in}{3.091485in}}%
\pgfpathlineto{\pgfqpoint{5.189937in}{3.244767in}}%
\pgfpathlineto{\pgfqpoint{5.191309in}{3.124772in}}%
\pgfpathlineto{\pgfqpoint{5.191417in}{3.257477in}}%
\pgfpathlineto{\pgfqpoint{5.192187in}{3.092699in}}%
\pgfpathlineto{\pgfqpoint{5.192480in}{3.256935in}}%
\pgfpathlineto{\pgfqpoint{5.193760in}{3.075860in}}%
\pgfpathlineto{\pgfqpoint{5.193837in}{3.148614in}}%
\pgfpathlineto{\pgfqpoint{5.193914in}{3.072819in}}%
\pgfpathlineto{\pgfqpoint{5.194068in}{3.208878in}}%
\pgfpathlineto{\pgfqpoint{5.194161in}{3.198341in}}%
\pgfpathlineto{\pgfqpoint{5.195101in}{3.279903in}}%
\pgfpathlineto{\pgfqpoint{5.194654in}{3.060440in}}%
\pgfpathlineto{\pgfqpoint{5.195255in}{3.188258in}}%
\pgfpathlineto{\pgfqpoint{5.195687in}{3.061590in}}%
\pgfpathlineto{\pgfqpoint{5.195995in}{3.251414in}}%
\pgfpathlineto{\pgfqpoint{5.196427in}{3.085447in}}%
\pgfpathlineto{\pgfqpoint{5.196920in}{3.250901in}}%
\pgfpathlineto{\pgfqpoint{5.197428in}{3.042317in}}%
\pgfpathlineto{\pgfqpoint{5.197567in}{3.172404in}}%
\pgfpathlineto{\pgfqpoint{5.197752in}{3.251702in}}%
\pgfpathlineto{\pgfqpoint{5.198153in}{3.049209in}}%
\pgfpathlineto{\pgfqpoint{5.198739in}{3.216980in}}%
\pgfpathlineto{\pgfqpoint{5.198754in}{3.217501in}}%
\pgfpathlineto{\pgfqpoint{5.198800in}{3.188390in}}%
\pgfpathlineto{\pgfqpoint{5.199186in}{3.050889in}}%
\pgfpathlineto{\pgfqpoint{5.199509in}{3.279927in}}%
\pgfpathlineto{\pgfqpoint{5.200018in}{3.138646in}}%
\pgfpathlineto{\pgfqpoint{5.201328in}{3.265271in}}%
\pgfpathlineto{\pgfqpoint{5.200835in}{3.086440in}}%
\pgfpathlineto{\pgfqpoint{5.201344in}{3.254895in}}%
\pgfpathlineto{\pgfqpoint{5.201837in}{3.083527in}}%
\pgfpathlineto{\pgfqpoint{5.202161in}{3.258580in}}%
\pgfpathlineto{\pgfqpoint{5.202485in}{3.180846in}}%
\pgfpathlineto{\pgfqpoint{5.203456in}{3.023892in}}%
\pgfpathlineto{\pgfqpoint{5.203039in}{3.248543in}}%
\pgfpathlineto{\pgfqpoint{5.203656in}{3.128089in}}%
\pgfpathlineto{\pgfqpoint{5.203903in}{3.301499in}}%
\pgfpathlineto{\pgfqpoint{5.204334in}{3.044574in}}%
\pgfpathlineto{\pgfqpoint{5.204843in}{3.231521in}}%
\pgfpathlineto{\pgfqpoint{5.205228in}{3.069212in}}%
\pgfpathlineto{\pgfqpoint{5.205722in}{3.254768in}}%
\pgfpathlineto{\pgfqpoint{5.205999in}{3.155586in}}%
\pgfpathlineto{\pgfqpoint{5.206554in}{3.261347in}}%
\pgfpathlineto{\pgfqpoint{5.206246in}{3.054753in}}%
\pgfpathlineto{\pgfqpoint{5.206909in}{3.138705in}}%
\pgfpathlineto{\pgfqpoint{5.207864in}{3.023366in}}%
\pgfpathlineto{\pgfqpoint{5.207556in}{3.262664in}}%
\pgfpathlineto{\pgfqpoint{5.208034in}{3.086882in}}%
\pgfpathlineto{\pgfqpoint{5.208311in}{3.285895in}}%
\pgfpathlineto{\pgfqpoint{5.208897in}{3.049118in}}%
\pgfpathlineto{\pgfqpoint{5.209205in}{3.241337in}}%
\pgfpathlineto{\pgfqpoint{5.210654in}{3.038099in}}%
\pgfpathlineto{\pgfqpoint{5.210115in}{3.265671in}}%
\pgfpathlineto{\pgfqpoint{5.210701in}{3.110908in}}%
\pgfpathlineto{\pgfqpoint{5.210963in}{3.276219in}}%
\pgfpathlineto{\pgfqpoint{5.211394in}{3.039861in}}%
\pgfpathlineto{\pgfqpoint{5.211857in}{3.258795in}}%
\pgfpathlineto{\pgfqpoint{5.212273in}{3.050385in}}%
\pgfpathlineto{\pgfqpoint{5.212720in}{3.299726in}}%
\pgfpathlineto{\pgfqpoint{5.213059in}{3.148664in}}%
\pgfpathlineto{\pgfqpoint{5.213444in}{3.291334in}}%
\pgfpathlineto{\pgfqpoint{5.213306in}{3.036656in}}%
\pgfpathlineto{\pgfqpoint{5.213999in}{3.093368in}}%
\pgfpathlineto{\pgfqpoint{5.215063in}{3.018214in}}%
\pgfpathlineto{\pgfqpoint{5.214493in}{3.280220in}}%
\pgfpathlineto{\pgfqpoint{5.215109in}{3.090495in}}%
\pgfpathlineto{\pgfqpoint{5.215371in}{3.292409in}}%
\pgfpathlineto{\pgfqpoint{5.215803in}{3.028975in}}%
\pgfpathlineto{\pgfqpoint{5.216250in}{3.254873in}}%
\pgfpathlineto{\pgfqpoint{5.216420in}{3.190626in}}%
\pgfpathlineto{\pgfqpoint{5.216666in}{3.038356in}}%
\pgfpathlineto{\pgfqpoint{5.217129in}{3.275555in}}%
\pgfpathlineto{\pgfqpoint{5.217576in}{3.074936in}}%
\pgfpathlineto{\pgfqpoint{5.217853in}{3.268255in}}%
\pgfpathlineto{\pgfqpoint{5.217714in}{3.032856in}}%
\pgfpathlineto{\pgfqpoint{5.218763in}{3.217653in}}%
\pgfpathlineto{\pgfqpoint{5.219472in}{3.031104in}}%
\pgfpathlineto{\pgfqpoint{5.219765in}{3.290863in}}%
\pgfpathlineto{\pgfqpoint{5.219780in}{3.305176in}}%
\pgfpathlineto{\pgfqpoint{5.220350in}{3.025545in}}%
\pgfpathlineto{\pgfqpoint{5.220797in}{3.234525in}}%
\pgfpathlineto{\pgfqpoint{5.221075in}{3.012934in}}%
\pgfpathlineto{\pgfqpoint{5.221537in}{3.300451in}}%
\pgfpathlineto{\pgfqpoint{5.222031in}{3.121258in}}%
\pgfpathlineto{\pgfqpoint{5.223325in}{3.293162in}}%
\pgfpathlineto{\pgfqpoint{5.222108in}{3.021718in}}%
\pgfpathlineto{\pgfqpoint{5.223356in}{3.277414in}}%
\pgfpathlineto{\pgfqpoint{5.224605in}{3.007618in}}%
\pgfpathlineto{\pgfqpoint{5.224173in}{3.306146in}}%
\pgfpathlineto{\pgfqpoint{5.224651in}{3.072682in}}%
\pgfpathlineto{\pgfqpoint{5.225052in}{3.302354in}}%
\pgfpathlineto{\pgfqpoint{5.225468in}{2.990756in}}%
\pgfpathlineto{\pgfqpoint{5.225823in}{3.263858in}}%
\pgfpathlineto{\pgfqpoint{5.227241in}{2.952982in}}%
\pgfpathlineto{\pgfqpoint{5.225946in}{3.304919in}}%
\pgfpathlineto{\pgfqpoint{5.227287in}{3.029801in}}%
\pgfpathlineto{\pgfqpoint{5.227534in}{3.363116in}}%
\pgfpathlineto{\pgfqpoint{5.228119in}{3.015366in}}%
\pgfpathlineto{\pgfqpoint{5.228474in}{3.258618in}}%
\pgfpathlineto{\pgfqpoint{5.228983in}{2.996645in}}%
\pgfpathlineto{\pgfqpoint{5.229260in}{3.367421in}}%
\pgfpathlineto{\pgfqpoint{5.229707in}{3.092341in}}%
\pgfpathlineto{\pgfqpoint{5.229846in}{3.037617in}}%
\pgfpathlineto{\pgfqpoint{5.230092in}{3.223738in}}%
\pgfpathlineto{\pgfqpoint{5.230139in}{3.296545in}}%
\pgfpathlineto{\pgfqpoint{5.230724in}{3.057904in}}%
\pgfpathlineto{\pgfqpoint{5.231187in}{3.193946in}}%
\pgfpathlineto{\pgfqpoint{5.231896in}{3.305548in}}%
\pgfpathlineto{\pgfqpoint{5.231588in}{3.060105in}}%
\pgfpathlineto{\pgfqpoint{5.232235in}{3.155161in}}%
\pgfpathlineto{\pgfqpoint{5.232451in}{3.053786in}}%
\pgfpathlineto{\pgfqpoint{5.232790in}{3.272160in}}%
\pgfpathlineto{\pgfqpoint{5.233376in}{3.117412in}}%
\pgfpathlineto{\pgfqpoint{5.233746in}{3.271596in}}%
\pgfpathlineto{\pgfqpoint{5.234177in}{3.063745in}}%
\pgfpathlineto{\pgfqpoint{5.234594in}{3.229341in}}%
\pgfpathlineto{\pgfqpoint{5.235041in}{3.055558in}}%
\pgfpathlineto{\pgfqpoint{5.234779in}{3.235237in}}%
\pgfpathlineto{\pgfqpoint{5.235765in}{3.143581in}}%
\pgfpathlineto{\pgfqpoint{5.236798in}{3.105454in}}%
\pgfpathlineto{\pgfqpoint{5.236258in}{3.233331in}}%
\pgfpathlineto{\pgfqpoint{5.236860in}{3.144399in}}%
\pgfpathlineto{\pgfqpoint{5.237754in}{3.209917in}}%
\pgfpathlineto{\pgfqpoint{5.237661in}{3.123296in}}%
\pgfpathlineto{\pgfqpoint{5.238031in}{3.203624in}}%
\pgfpathlineto{\pgfqpoint{5.238232in}{3.117967in}}%
\pgfpathlineto{\pgfqpoint{5.238432in}{3.211585in}}%
\pgfpathlineto{\pgfqpoint{5.239172in}{3.192154in}}%
\pgfpathlineto{\pgfqpoint{5.239203in}{3.185758in}}%
\pgfpathlineto{\pgfqpoint{5.239388in}{3.140020in}}%
\pgfpathlineto{\pgfqpoint{5.239295in}{3.198509in}}%
\pgfpathlineto{\pgfqpoint{5.240451in}{3.145839in}}%
\pgfpathlineto{\pgfqpoint{5.241407in}{3.206257in}}%
\pgfpathlineto{\pgfqpoint{5.241499in}{3.135651in}}%
\pgfpathlineto{\pgfqpoint{5.241592in}{3.162464in}}%
\pgfpathlineto{\pgfqpoint{5.242717in}{3.113845in}}%
\pgfpathlineto{\pgfqpoint{5.242270in}{3.226241in}}%
\pgfpathlineto{\pgfqpoint{5.242748in}{3.125683in}}%
\pgfpathlineto{\pgfqpoint{5.243688in}{3.262391in}}%
\pgfpathlineto{\pgfqpoint{5.243411in}{3.084691in}}%
\pgfpathlineto{\pgfqpoint{5.243889in}{3.195859in}}%
\pgfpathlineto{\pgfqpoint{5.244120in}{3.083066in}}%
\pgfpathlineto{\pgfqpoint{5.244505in}{3.229875in}}%
\pgfpathlineto{\pgfqpoint{5.244999in}{3.180914in}}%
\pgfpathlineto{\pgfqpoint{5.245523in}{3.222349in}}%
\pgfpathlineto{\pgfqpoint{5.245985in}{3.112516in}}%
\pgfpathlineto{\pgfqpoint{5.246016in}{3.126582in}}%
\pgfpathlineto{\pgfqpoint{5.246401in}{3.260298in}}%
\pgfpathlineto{\pgfqpoint{5.246787in}{3.090875in}}%
\pgfpathlineto{\pgfqpoint{5.247234in}{3.257810in}}%
\pgfpathlineto{\pgfqpoint{5.248575in}{3.043397in}}%
\pgfpathlineto{\pgfqpoint{5.248143in}{3.314502in}}%
\pgfpathlineto{\pgfqpoint{5.248621in}{3.127365in}}%
\pgfpathlineto{\pgfqpoint{5.248775in}{3.247923in}}%
\pgfpathlineto{\pgfqpoint{5.249253in}{3.096314in}}%
\pgfpathlineto{\pgfqpoint{5.249746in}{3.172064in}}%
\pgfpathlineto{\pgfqpoint{5.250116in}{3.078102in}}%
\pgfpathlineto{\pgfqpoint{5.250425in}{3.255540in}}%
\pgfpathlineto{\pgfqpoint{5.250872in}{3.135497in}}%
\pgfpathlineto{\pgfqpoint{5.250980in}{3.254629in}}%
\pgfpathlineto{\pgfqpoint{5.251195in}{3.064245in}}%
\pgfpathlineto{\pgfqpoint{5.252059in}{3.211213in}}%
\pgfpathlineto{\pgfqpoint{5.252999in}{3.092213in}}%
\pgfpathlineto{\pgfqpoint{5.252567in}{3.335656in}}%
\pgfpathlineto{\pgfqpoint{5.253215in}{3.121473in}}%
\pgfpathlineto{\pgfqpoint{5.254155in}{3.290636in}}%
\pgfpathlineto{\pgfqpoint{5.253816in}{3.064616in}}%
\pgfpathlineto{\pgfqpoint{5.254463in}{3.207028in}}%
\pgfpathlineto{\pgfqpoint{5.254725in}{3.080674in}}%
\pgfpathlineto{\pgfqpoint{5.255157in}{3.291001in}}%
\pgfpathlineto{\pgfqpoint{5.255635in}{3.116982in}}%
\pgfpathlineto{\pgfqpoint{5.255712in}{3.270572in}}%
\pgfpathlineto{\pgfqpoint{5.256452in}{3.081899in}}%
\pgfpathlineto{\pgfqpoint{5.256837in}{3.188361in}}%
\pgfpathlineto{\pgfqpoint{5.257562in}{3.245105in}}%
\pgfpathlineto{\pgfqpoint{5.257115in}{3.106016in}}%
\pgfpathlineto{\pgfqpoint{5.257824in}{3.180813in}}%
\pgfpathlineto{\pgfqpoint{5.258024in}{3.079562in}}%
\pgfpathlineto{\pgfqpoint{5.258857in}{3.257904in}}%
\pgfpathlineto{\pgfqpoint{5.258918in}{3.196703in}}%
\pgfpathlineto{\pgfqpoint{5.259042in}{3.104988in}}%
\pgfpathlineto{\pgfqpoint{5.259766in}{3.217045in}}%
\pgfpathlineto{\pgfqpoint{5.260413in}{3.266281in}}%
\pgfpathlineto{\pgfqpoint{5.260090in}{3.087387in}}%
\pgfpathlineto{\pgfqpoint{5.260799in}{3.143115in}}%
\pgfpathlineto{\pgfqpoint{5.261338in}{3.281194in}}%
\pgfpathlineto{\pgfqpoint{5.261138in}{3.078354in}}%
\pgfpathlineto{\pgfqpoint{5.261616in}{3.133995in}}%
\pgfpathlineto{\pgfqpoint{5.262572in}{3.060063in}}%
\pgfpathlineto{\pgfqpoint{5.261986in}{3.289241in}}%
\pgfpathlineto{\pgfqpoint{5.262741in}{3.105489in}}%
\pgfpathlineto{\pgfqpoint{5.263003in}{3.304965in}}%
\pgfpathlineto{\pgfqpoint{5.263219in}{3.099240in}}%
\pgfpathlineto{\pgfqpoint{5.263897in}{3.204008in}}%
\pgfpathlineto{\pgfqpoint{5.264298in}{3.056910in}}%
\pgfpathlineto{\pgfqpoint{5.264560in}{3.288611in}}%
\pgfpathlineto{\pgfqpoint{5.265038in}{3.116156in}}%
\pgfpathlineto{\pgfqpoint{5.265115in}{3.265625in}}%
\pgfpathlineto{\pgfqpoint{5.265855in}{3.095406in}}%
\pgfpathlineto{\pgfqpoint{5.266179in}{3.201042in}}%
\pgfpathlineto{\pgfqpoint{5.266872in}{3.085357in}}%
\pgfpathlineto{\pgfqpoint{5.266672in}{3.297257in}}%
\pgfpathlineto{\pgfqpoint{5.267335in}{3.133277in}}%
\pgfpathlineto{\pgfqpoint{5.267427in}{3.111226in}}%
\pgfpathlineto{\pgfqpoint{5.267381in}{3.142824in}}%
\pgfpathlineto{\pgfqpoint{5.267458in}{3.141054in}}%
\pgfpathlineto{\pgfqpoint{5.268244in}{3.300788in}}%
\pgfpathlineto{\pgfqpoint{5.267920in}{3.075983in}}%
\pgfpathlineto{\pgfqpoint{5.268568in}{3.182029in}}%
\pgfpathlineto{\pgfqpoint{5.269647in}{3.087826in}}%
\pgfpathlineto{\pgfqpoint{5.269169in}{3.266385in}}%
\pgfpathlineto{\pgfqpoint{5.269709in}{3.120470in}}%
\pgfpathlineto{\pgfqpoint{5.270818in}{3.270122in}}%
\pgfpathlineto{\pgfqpoint{5.270556in}{3.080323in}}%
\pgfpathlineto{\pgfqpoint{5.270896in}{3.207242in}}%
\pgfpathlineto{\pgfqpoint{5.271666in}{3.251935in}}%
\pgfpathlineto{\pgfqpoint{5.272067in}{3.102739in}}%
\pgfpathlineto{\pgfqpoint{5.272129in}{3.059725in}}%
\pgfpathlineto{\pgfqpoint{5.272391in}{3.267824in}}%
\pgfpathlineto{\pgfqpoint{5.273162in}{3.104508in}}%
\pgfpathlineto{\pgfqpoint{5.273963in}{3.281453in}}%
\pgfpathlineto{\pgfqpoint{5.274210in}{3.084774in}}%
\pgfpathlineto{\pgfqpoint{5.274302in}{3.152131in}}%
\pgfpathlineto{\pgfqpoint{5.275273in}{3.068658in}}%
\pgfpathlineto{\pgfqpoint{5.274518in}{3.261431in}}%
\pgfpathlineto{\pgfqpoint{5.275381in}{3.160398in}}%
\pgfpathlineto{\pgfqpoint{5.275520in}{3.304784in}}%
\pgfpathlineto{\pgfqpoint{5.275782in}{3.100510in}}%
\pgfpathlineto{\pgfqpoint{5.276491in}{3.163245in}}%
\pgfpathlineto{\pgfqpoint{5.277092in}{3.325383in}}%
\pgfpathlineto{\pgfqpoint{5.276846in}{3.052781in}}%
\pgfpathlineto{\pgfqpoint{5.277447in}{3.117160in}}%
\pgfpathlineto{\pgfqpoint{5.278403in}{3.059398in}}%
\pgfpathlineto{\pgfqpoint{5.277956in}{3.284883in}}%
\pgfpathlineto{\pgfqpoint{5.278526in}{3.149682in}}%
\pgfpathlineto{\pgfqpoint{5.278680in}{3.308764in}}%
\pgfpathlineto{\pgfqpoint{5.279065in}{3.095431in}}%
\pgfpathlineto{\pgfqpoint{5.279636in}{3.175219in}}%
\pgfpathlineto{\pgfqpoint{5.279667in}{3.185069in}}%
\pgfpathlineto{\pgfqpoint{5.280237in}{3.292234in}}%
\pgfpathlineto{\pgfqpoint{5.279975in}{3.052778in}}%
\pgfpathlineto{\pgfqpoint{5.280823in}{3.247386in}}%
\pgfpathlineto{\pgfqpoint{5.281809in}{3.295160in}}%
\pgfpathlineto{\pgfqpoint{5.282040in}{3.078239in}}%
\pgfpathlineto{\pgfqpoint{5.282380in}{3.260962in}}%
\pgfpathlineto{\pgfqpoint{5.283073in}{3.061673in}}%
\pgfpathlineto{\pgfqpoint{5.283089in}{3.060279in}}%
\pgfpathlineto{\pgfqpoint{5.283166in}{3.136209in}}%
\pgfpathlineto{\pgfqpoint{5.283382in}{3.303374in}}%
\pgfpathlineto{\pgfqpoint{5.283628in}{3.085000in}}%
\pgfpathlineto{\pgfqpoint{5.284306in}{3.205764in}}%
\pgfpathlineto{\pgfqpoint{5.284692in}{3.043141in}}%
\pgfpathlineto{\pgfqpoint{5.284954in}{3.305666in}}%
\pgfpathlineto{\pgfqpoint{5.285447in}{3.139638in}}%
\pgfpathlineto{\pgfqpoint{5.286511in}{3.308467in}}%
\pgfpathlineto{\pgfqpoint{5.286249in}{3.040361in}}%
\pgfpathlineto{\pgfqpoint{5.286588in}{3.213276in}}%
\pgfpathlineto{\pgfqpoint{5.286912in}{3.089062in}}%
\pgfpathlineto{\pgfqpoint{5.287374in}{3.270821in}}%
\pgfpathlineto{\pgfqpoint{5.287759in}{3.101570in}}%
\pgfpathlineto{\pgfqpoint{5.287821in}{3.059440in}}%
\pgfpathlineto{\pgfqpoint{5.288083in}{3.313989in}}%
\pgfpathlineto{\pgfqpoint{5.288623in}{3.240531in}}%
\pgfpathlineto{\pgfqpoint{5.289640in}{3.311951in}}%
\pgfpathlineto{\pgfqpoint{5.289378in}{3.042914in}}%
\pgfpathlineto{\pgfqpoint{5.289702in}{3.237106in}}%
\pgfpathlineto{\pgfqpoint{5.289902in}{3.081856in}}%
\pgfpathlineto{\pgfqpoint{5.290210in}{3.279125in}}%
\pgfpathlineto{\pgfqpoint{5.290889in}{3.086933in}}%
\pgfpathlineto{\pgfqpoint{5.290950in}{3.049474in}}%
\pgfpathlineto{\pgfqpoint{5.291212in}{3.329209in}}%
\pgfpathlineto{\pgfqpoint{5.291736in}{3.187746in}}%
\pgfpathlineto{\pgfqpoint{5.292785in}{3.330343in}}%
\pgfpathlineto{\pgfqpoint{5.292523in}{3.034406in}}%
\pgfpathlineto{\pgfqpoint{5.292846in}{3.213476in}}%
\pgfpathlineto{\pgfqpoint{5.293185in}{3.081535in}}%
\pgfpathlineto{\pgfqpoint{5.293355in}{3.278399in}}%
\pgfpathlineto{\pgfqpoint{5.294002in}{3.118850in}}%
\pgfpathlineto{\pgfqpoint{5.294095in}{3.041452in}}%
\pgfpathlineto{\pgfqpoint{5.294342in}{3.292392in}}%
\pgfpathlineto{\pgfqpoint{5.295097in}{3.124420in}}%
\pgfpathlineto{\pgfqpoint{5.295652in}{3.057969in}}%
\pgfpathlineto{\pgfqpoint{5.295205in}{3.251939in}}%
\pgfpathlineto{\pgfqpoint{5.295790in}{3.192268in}}%
\pgfpathlineto{\pgfqpoint{5.295914in}{3.296050in}}%
\pgfpathlineto{\pgfqpoint{5.296315in}{3.082383in}}%
\pgfpathlineto{\pgfqpoint{5.296854in}{3.156741in}}%
\pgfpathlineto{\pgfqpoint{5.297209in}{3.037644in}}%
\pgfpathlineto{\pgfqpoint{5.297471in}{3.342434in}}%
\pgfpathlineto{\pgfqpoint{5.297964in}{3.124362in}}%
\pgfpathlineto{\pgfqpoint{5.299043in}{3.343268in}}%
\pgfpathlineto{\pgfqpoint{5.298766in}{3.030798in}}%
\pgfpathlineto{\pgfqpoint{5.299120in}{3.231206in}}%
\pgfpathlineto{\pgfqpoint{5.299613in}{3.270764in}}%
\pgfpathlineto{\pgfqpoint{5.300322in}{3.037031in}}%
\pgfpathlineto{\pgfqpoint{5.300338in}{3.024333in}}%
\pgfpathlineto{\pgfqpoint{5.300615in}{3.317681in}}%
\pgfpathlineto{\pgfqpoint{5.301155in}{3.239016in}}%
\pgfpathlineto{\pgfqpoint{5.302172in}{3.320749in}}%
\pgfpathlineto{\pgfqpoint{5.301910in}{3.054367in}}%
\pgfpathlineto{\pgfqpoint{5.302234in}{3.236231in}}%
\pgfpathlineto{\pgfqpoint{5.302588in}{3.082208in}}%
\pgfpathlineto{\pgfqpoint{5.302743in}{3.275047in}}%
\pgfpathlineto{\pgfqpoint{5.303375in}{3.144098in}}%
\pgfpathlineto{\pgfqpoint{5.303467in}{3.047227in}}%
\pgfpathlineto{\pgfqpoint{5.303714in}{3.302092in}}%
\pgfpathlineto{\pgfqpoint{5.303745in}{3.329139in}}%
\pgfpathlineto{\pgfqpoint{5.304161in}{3.055944in}}%
\pgfpathlineto{\pgfqpoint{5.304700in}{3.145481in}}%
\pgfpathlineto{\pgfqpoint{5.305039in}{3.038884in}}%
\pgfpathlineto{\pgfqpoint{5.305209in}{3.234306in}}%
\pgfpathlineto{\pgfqpoint{5.305301in}{3.351590in}}%
\pgfpathlineto{\pgfqpoint{5.305718in}{3.040137in}}%
\pgfpathlineto{\pgfqpoint{5.306273in}{3.166386in}}%
\pgfpathlineto{\pgfqpoint{5.306334in}{3.206056in}}%
\pgfpathlineto{\pgfqpoint{5.306581in}{3.043103in}}%
\pgfpathlineto{\pgfqpoint{5.306596in}{3.029654in}}%
\pgfpathlineto{\pgfqpoint{5.306874in}{3.350608in}}%
\pgfpathlineto{\pgfqpoint{5.307506in}{3.204029in}}%
\pgfpathlineto{\pgfqpoint{5.307737in}{3.278803in}}%
\pgfpathlineto{\pgfqpoint{5.307999in}{3.119415in}}%
\pgfpathlineto{\pgfqpoint{5.308169in}{3.018679in}}%
\pgfpathlineto{\pgfqpoint{5.308446in}{3.332852in}}%
\pgfpathlineto{\pgfqpoint{5.309063in}{3.195733in}}%
\pgfpathlineto{\pgfqpoint{5.309741in}{3.035803in}}%
\pgfpathlineto{\pgfqpoint{5.309294in}{3.262892in}}%
\pgfpathlineto{\pgfqpoint{5.309972in}{3.262775in}}%
\pgfpathlineto{\pgfqpoint{5.310003in}{3.311696in}}%
\pgfpathlineto{\pgfqpoint{5.310419in}{3.077212in}}%
\pgfpathlineto{\pgfqpoint{5.311051in}{3.223579in}}%
\pgfpathlineto{\pgfqpoint{5.311298in}{3.054019in}}%
\pgfpathlineto{\pgfqpoint{5.311575in}{3.319044in}}%
\pgfpathlineto{\pgfqpoint{5.312099in}{3.226073in}}%
\pgfpathlineto{\pgfqpoint{5.313148in}{3.323753in}}%
\pgfpathlineto{\pgfqpoint{5.312731in}{3.060080in}}%
\pgfpathlineto{\pgfqpoint{5.313209in}{3.249410in}}%
\pgfpathlineto{\pgfqpoint{5.313980in}{3.273313in}}%
\pgfpathlineto{\pgfqpoint{5.314412in}{3.049877in}}%
\pgfpathlineto{\pgfqpoint{5.314427in}{3.037409in}}%
\pgfpathlineto{\pgfqpoint{5.314720in}{3.323242in}}%
\pgfpathlineto{\pgfqpoint{5.315352in}{3.169520in}}%
\pgfpathlineto{\pgfqpoint{5.316277in}{3.316165in}}%
\pgfpathlineto{\pgfqpoint{5.315999in}{3.035302in}}%
\pgfpathlineto{\pgfqpoint{5.316462in}{3.180660in}}%
\pgfpathlineto{\pgfqpoint{5.317556in}{3.063340in}}%
\pgfpathlineto{\pgfqpoint{5.317109in}{3.241135in}}%
\pgfpathlineto{\pgfqpoint{5.317602in}{3.105341in}}%
\pgfpathlineto{\pgfqpoint{5.317849in}{3.272892in}}%
\pgfpathlineto{\pgfqpoint{5.318234in}{3.078375in}}%
\pgfpathlineto{\pgfqpoint{5.318759in}{3.205613in}}%
\pgfpathlineto{\pgfqpoint{5.319051in}{3.073296in}}%
\pgfpathlineto{\pgfqpoint{5.319421in}{3.289405in}}%
\pgfpathlineto{\pgfqpoint{5.319899in}{3.173365in}}%
\pgfpathlineto{\pgfqpoint{5.320947in}{3.298941in}}%
\pgfpathlineto{\pgfqpoint{5.320685in}{3.059047in}}%
\pgfpathlineto{\pgfqpoint{5.321025in}{3.236598in}}%
\pgfpathlineto{\pgfqpoint{5.321379in}{3.049008in}}%
\pgfpathlineto{\pgfqpoint{5.321811in}{3.270939in}}%
\pgfpathlineto{\pgfqpoint{5.322242in}{3.049171in}}%
\pgfpathlineto{\pgfqpoint{5.322520in}{3.310465in}}%
\pgfpathlineto{\pgfqpoint{5.323630in}{3.142327in}}%
\pgfpathlineto{\pgfqpoint{5.323799in}{3.043555in}}%
\pgfpathlineto{\pgfqpoint{5.324092in}{3.299684in}}%
\pgfpathlineto{\pgfqpoint{5.324709in}{3.185120in}}%
\pgfpathlineto{\pgfqpoint{5.325634in}{3.274870in}}%
\pgfpathlineto{\pgfqpoint{5.325356in}{3.028209in}}%
\pgfpathlineto{\pgfqpoint{5.325803in}{3.188540in}}%
\pgfpathlineto{\pgfqpoint{5.326959in}{3.041654in}}%
\pgfpathlineto{\pgfqpoint{5.326497in}{3.262003in}}%
\pgfpathlineto{\pgfqpoint{5.327006in}{3.101915in}}%
\pgfpathlineto{\pgfqpoint{5.327221in}{3.296323in}}%
\pgfpathlineto{\pgfqpoint{5.327638in}{3.065309in}}%
\pgfpathlineto{\pgfqpoint{5.328254in}{3.228676in}}%
\pgfpathlineto{\pgfqpoint{5.328485in}{3.050631in}}%
\pgfpathlineto{\pgfqpoint{5.328778in}{3.299724in}}%
\pgfpathlineto{\pgfqpoint{5.329395in}{3.199022in}}%
\pgfpathlineto{\pgfqpoint{5.330397in}{3.267334in}}%
\pgfpathlineto{\pgfqpoint{5.330027in}{3.048671in}}%
\pgfpathlineto{\pgfqpoint{5.330489in}{3.197409in}}%
\pgfpathlineto{\pgfqpoint{5.331645in}{3.043611in}}%
\pgfpathlineto{\pgfqpoint{5.331183in}{3.266779in}}%
\pgfpathlineto{\pgfqpoint{5.331676in}{3.096545in}}%
\pgfpathlineto{\pgfqpoint{5.331892in}{3.293069in}}%
\pgfpathlineto{\pgfqpoint{5.332324in}{3.058840in}}%
\pgfpathlineto{\pgfqpoint{5.332848in}{3.205454in}}%
\pgfpathlineto{\pgfqpoint{5.333156in}{3.052758in}}%
\pgfpathlineto{\pgfqpoint{5.333434in}{3.279279in}}%
\pgfpathlineto{\pgfqpoint{5.333495in}{3.327416in}}%
\pgfpathlineto{\pgfqpoint{5.333834in}{3.084377in}}%
\pgfpathlineto{\pgfqpoint{5.334451in}{3.208425in}}%
\pgfpathlineto{\pgfqpoint{5.334744in}{3.031060in}}%
\pgfpathlineto{\pgfqpoint{5.335067in}{3.282487in}}%
\pgfpathlineto{\pgfqpoint{5.335545in}{3.188019in}}%
\pgfpathlineto{\pgfqpoint{5.336624in}{3.292451in}}%
\pgfpathlineto{\pgfqpoint{5.336316in}{3.031875in}}%
\pgfpathlineto{\pgfqpoint{5.336671in}{3.233805in}}%
\pgfpathlineto{\pgfqpoint{5.336964in}{3.078514in}}%
\pgfpathlineto{\pgfqpoint{5.337441in}{3.251905in}}%
\pgfpathlineto{\pgfqpoint{5.337888in}{3.098427in}}%
\pgfpathlineto{\pgfqpoint{5.338212in}{3.310267in}}%
\pgfpathlineto{\pgfqpoint{5.338536in}{3.081062in}}%
\pgfpathlineto{\pgfqpoint{5.339045in}{3.218181in}}%
\pgfpathlineto{\pgfqpoint{5.339337in}{3.091644in}}%
\pgfpathlineto{\pgfqpoint{5.339800in}{3.254542in}}%
\pgfpathlineto{\pgfqpoint{5.340216in}{3.142628in}}%
\pgfpathlineto{\pgfqpoint{5.340910in}{3.046462in}}%
\pgfpathlineto{\pgfqpoint{5.340571in}{3.260019in}}%
\pgfpathlineto{\pgfqpoint{5.341233in}{3.223440in}}%
\pgfpathlineto{\pgfqpoint{5.341973in}{3.092328in}}%
\pgfpathlineto{\pgfqpoint{5.341357in}{3.267259in}}%
\pgfpathlineto{\pgfqpoint{5.342390in}{3.184879in}}%
\pgfpathlineto{\pgfqpoint{5.342405in}{3.184761in}}%
\pgfpathlineto{\pgfqpoint{5.343515in}{3.066438in}}%
\pgfpathlineto{\pgfqpoint{5.342914in}{3.243599in}}%
\pgfpathlineto{\pgfqpoint{5.343561in}{3.122781in}}%
\pgfpathlineto{\pgfqpoint{5.343731in}{3.253537in}}%
\pgfpathlineto{\pgfqpoint{5.344563in}{3.089496in}}%
\pgfpathlineto{\pgfqpoint{5.344671in}{3.128575in}}%
\pgfpathlineto{\pgfqpoint{5.344686in}{3.118097in}}%
\pgfpathlineto{\pgfqpoint{5.345288in}{3.241051in}}%
\pgfpathlineto{\pgfqpoint{5.345688in}{3.191481in}}%
\pgfpathlineto{\pgfqpoint{5.346659in}{3.070546in}}%
\pgfpathlineto{\pgfqpoint{5.346228in}{3.245033in}}%
\pgfpathlineto{\pgfqpoint{5.346814in}{3.179182in}}%
\pgfpathlineto{\pgfqpoint{5.347014in}{3.248207in}}%
\pgfpathlineto{\pgfqpoint{5.347199in}{3.064698in}}%
\pgfpathlineto{\pgfqpoint{5.347846in}{3.100648in}}%
\pgfpathlineto{\pgfqpoint{5.348694in}{3.258389in}}%
\pgfpathlineto{\pgfqpoint{5.348525in}{3.080060in}}%
\pgfpathlineto{\pgfqpoint{5.349141in}{3.154220in}}%
\pgfpathlineto{\pgfqpoint{5.349295in}{3.071103in}}%
\pgfpathlineto{\pgfqpoint{5.350035in}{3.248030in}}%
\pgfpathlineto{\pgfqpoint{5.350205in}{3.199235in}}%
\pgfpathlineto{\pgfqpoint{5.350328in}{3.078084in}}%
\pgfpathlineto{\pgfqpoint{5.350729in}{3.239569in}}%
\pgfpathlineto{\pgfqpoint{5.351361in}{3.176947in}}%
\pgfpathlineto{\pgfqpoint{5.352193in}{3.236180in}}%
\pgfpathlineto{\pgfqpoint{5.351700in}{3.113584in}}%
\pgfpathlineto{\pgfqpoint{5.352409in}{3.150508in}}%
\pgfpathlineto{\pgfqpoint{5.353427in}{3.067752in}}%
\pgfpathlineto{\pgfqpoint{5.353211in}{3.244627in}}%
\pgfpathlineto{\pgfqpoint{5.353550in}{3.112389in}}%
\pgfpathlineto{\pgfqpoint{5.354244in}{3.099187in}}%
\pgfpathlineto{\pgfqpoint{5.353874in}{3.253915in}}%
\pgfpathlineto{\pgfqpoint{5.354459in}{3.162207in}}%
\pgfpathlineto{\pgfqpoint{5.354752in}{3.248962in}}%
\pgfpathlineto{\pgfqpoint{5.355076in}{3.089500in}}%
\pgfpathlineto{\pgfqpoint{5.355538in}{3.135598in}}%
\pgfpathlineto{\pgfqpoint{5.356525in}{3.065036in}}%
\pgfpathlineto{\pgfqpoint{5.356155in}{3.247119in}}%
\pgfpathlineto{\pgfqpoint{5.356648in}{3.112966in}}%
\pgfpathlineto{\pgfqpoint{5.356972in}{3.295805in}}%
\pgfpathlineto{\pgfqpoint{5.357188in}{3.106237in}}%
\pgfpathlineto{\pgfqpoint{5.357851in}{3.213183in}}%
\pgfpathlineto{\pgfqpoint{5.358174in}{3.079673in}}%
\pgfpathlineto{\pgfqpoint{5.358390in}{3.263179in}}%
\pgfpathlineto{\pgfqpoint{5.358498in}{3.232709in}}%
\pgfpathlineto{\pgfqpoint{5.358529in}{3.316835in}}%
\pgfpathlineto{\pgfqpoint{5.359454in}{3.104414in}}%
\pgfpathlineto{\pgfqpoint{5.359577in}{3.130513in}}%
\pgfpathlineto{\pgfqpoint{5.360286in}{3.061999in}}%
\pgfpathlineto{\pgfqpoint{5.360070in}{3.330869in}}%
\pgfpathlineto{\pgfqpoint{5.360625in}{3.184288in}}%
\pgfpathlineto{\pgfqpoint{5.360856in}{3.279877in}}%
\pgfpathlineto{\pgfqpoint{5.361088in}{3.128137in}}%
\pgfpathlineto{\pgfqpoint{5.361149in}{3.131958in}}%
\pgfpathlineto{\pgfqpoint{5.361951in}{3.042235in}}%
\pgfpathlineto{\pgfqpoint{5.361627in}{3.296771in}}%
\pgfpathlineto{\pgfqpoint{5.362228in}{3.161220in}}%
\pgfpathlineto{\pgfqpoint{5.363030in}{3.278864in}}%
\pgfpathlineto{\pgfqpoint{5.362783in}{3.114105in}}%
\pgfpathlineto{\pgfqpoint{5.363307in}{3.167623in}}%
\pgfpathlineto{\pgfqpoint{5.363508in}{3.072792in}}%
\pgfpathlineto{\pgfqpoint{5.363970in}{3.274020in}}%
\pgfpathlineto{\pgfqpoint{5.364417in}{3.151023in}}%
\pgfpathlineto{\pgfqpoint{5.364726in}{3.294980in}}%
\pgfpathlineto{\pgfqpoint{5.364895in}{3.078619in}}%
\pgfpathlineto{\pgfqpoint{5.365558in}{3.178241in}}%
\pgfpathlineto{\pgfqpoint{5.366483in}{3.062903in}}%
\pgfpathlineto{\pgfqpoint{5.366283in}{3.251508in}}%
\pgfpathlineto{\pgfqpoint{5.366668in}{3.176325in}}%
\pgfpathlineto{\pgfqpoint{5.367701in}{3.284618in}}%
\pgfpathlineto{\pgfqpoint{5.367207in}{3.049227in}}%
\pgfpathlineto{\pgfqpoint{5.367793in}{3.230002in}}%
\pgfpathlineto{\pgfqpoint{5.367809in}{3.230055in}}%
\pgfpathlineto{\pgfqpoint{5.367932in}{3.268150in}}%
\pgfpathlineto{\pgfqpoint{5.368163in}{3.051782in}}%
\pgfpathlineto{\pgfqpoint{5.368795in}{3.134487in}}%
\pgfpathlineto{\pgfqpoint{5.369535in}{3.090399in}}%
\pgfpathlineto{\pgfqpoint{5.369381in}{3.276483in}}%
\pgfpathlineto{\pgfqpoint{5.369843in}{3.120712in}}%
\pgfpathlineto{\pgfqpoint{5.369920in}{3.318164in}}%
\pgfpathlineto{\pgfqpoint{5.370306in}{3.067486in}}%
\pgfpathlineto{\pgfqpoint{5.370969in}{3.212464in}}%
\pgfpathlineto{\pgfqpoint{5.371169in}{3.047345in}}%
\pgfpathlineto{\pgfqpoint{5.371709in}{3.262839in}}%
\pgfpathlineto{\pgfqpoint{5.372125in}{3.089161in}}%
\pgfpathlineto{\pgfqpoint{5.373019in}{3.288864in}}%
\pgfpathlineto{\pgfqpoint{5.372803in}{3.031405in}}%
\pgfpathlineto{\pgfqpoint{5.373358in}{3.166496in}}%
\pgfpathlineto{\pgfqpoint{5.374360in}{3.038283in}}%
\pgfpathlineto{\pgfqpoint{5.373882in}{3.295707in}}%
\pgfpathlineto{\pgfqpoint{5.374468in}{3.129989in}}%
\pgfpathlineto{\pgfqpoint{5.375608in}{3.295311in}}%
\pgfpathlineto{\pgfqpoint{5.375054in}{3.043903in}}%
\pgfpathlineto{\pgfqpoint{5.375670in}{3.223959in}}%
\pgfpathlineto{\pgfqpoint{5.375917in}{3.064524in}}%
\pgfpathlineto{\pgfqpoint{5.376225in}{3.242530in}}%
\pgfpathlineto{\pgfqpoint{5.376795in}{3.159645in}}%
\pgfpathlineto{\pgfqpoint{5.377921in}{3.267608in}}%
\pgfpathlineto{\pgfqpoint{5.377366in}{3.063556in}}%
\pgfpathlineto{\pgfqpoint{5.377952in}{3.252459in}}%
\pgfpathlineto{\pgfqpoint{5.379061in}{3.056566in}}%
\pgfpathlineto{\pgfqpoint{5.378784in}{3.257572in}}%
\pgfpathlineto{\pgfqpoint{5.379138in}{3.123438in}}%
\pgfpathlineto{\pgfqpoint{5.379709in}{3.056427in}}%
\pgfpathlineto{\pgfqpoint{5.379478in}{3.256126in}}%
\pgfpathlineto{\pgfqpoint{5.380079in}{3.195989in}}%
\pgfpathlineto{\pgfqpoint{5.380788in}{3.284145in}}%
\pgfpathlineto{\pgfqpoint{5.380618in}{3.089153in}}%
\pgfpathlineto{\pgfqpoint{5.381127in}{3.134024in}}%
\pgfpathlineto{\pgfqpoint{5.382129in}{3.068794in}}%
\pgfpathlineto{\pgfqpoint{5.381790in}{3.270449in}}%
\pgfpathlineto{\pgfqpoint{5.382237in}{3.135210in}}%
\pgfpathlineto{\pgfqpoint{5.382807in}{3.091380in}}%
\pgfpathlineto{\pgfqpoint{5.382422in}{3.274183in}}%
\pgfpathlineto{\pgfqpoint{5.383300in}{3.168710in}}%
\pgfpathlineto{\pgfqpoint{5.383886in}{3.256053in}}%
\pgfpathlineto{\pgfqpoint{5.384256in}{3.078451in}}%
\pgfpathlineto{\pgfqpoint{5.384333in}{3.097521in}}%
\pgfpathlineto{\pgfqpoint{5.384349in}{3.065604in}}%
\pgfpathlineto{\pgfqpoint{5.384749in}{3.252011in}}%
\pgfpathlineto{\pgfqpoint{5.385381in}{3.167349in}}%
\pgfpathlineto{\pgfqpoint{5.385443in}{3.263880in}}%
\pgfpathlineto{\pgfqpoint{5.385906in}{3.087957in}}%
\pgfpathlineto{\pgfqpoint{5.386491in}{3.207840in}}%
\pgfpathlineto{\pgfqpoint{5.387540in}{3.101637in}}%
\pgfpathlineto{\pgfqpoint{5.387077in}{3.238986in}}%
\pgfpathlineto{\pgfqpoint{5.387632in}{3.156691in}}%
\pgfpathlineto{\pgfqpoint{5.388634in}{3.257333in}}%
\pgfpathlineto{\pgfqpoint{5.388218in}{3.108376in}}%
\pgfpathlineto{\pgfqpoint{5.388819in}{3.221312in}}%
\pgfpathlineto{\pgfqpoint{5.388911in}{3.075577in}}%
\pgfpathlineto{\pgfqpoint{5.389389in}{3.249790in}}%
\pgfpathlineto{\pgfqpoint{5.389991in}{3.129566in}}%
\pgfpathlineto{\pgfqpoint{5.390545in}{3.073133in}}%
\pgfpathlineto{\pgfqpoint{5.390345in}{3.260992in}}%
\pgfpathlineto{\pgfqpoint{5.390761in}{3.180427in}}%
\pgfpathlineto{\pgfqpoint{5.391624in}{3.247084in}}%
\pgfpathlineto{\pgfqpoint{5.391486in}{3.097508in}}%
\pgfpathlineto{\pgfqpoint{5.391902in}{3.220752in}}%
\pgfpathlineto{\pgfqpoint{5.392118in}{3.083436in}}%
\pgfpathlineto{\pgfqpoint{5.392395in}{3.252985in}}%
\pgfpathlineto{\pgfqpoint{5.393120in}{3.117815in}}%
\pgfpathlineto{\pgfqpoint{5.393289in}{3.268676in}}%
\pgfpathlineto{\pgfqpoint{5.393659in}{3.079434in}}%
\pgfpathlineto{\pgfqpoint{5.394276in}{3.218698in}}%
\pgfpathlineto{\pgfqpoint{5.395216in}{3.092525in}}%
\pgfpathlineto{\pgfqpoint{5.395000in}{3.249678in}}%
\pgfpathlineto{\pgfqpoint{5.395463in}{3.161222in}}%
\pgfpathlineto{\pgfqpoint{5.396542in}{3.311804in}}%
\pgfpathlineto{\pgfqpoint{5.396187in}{3.116181in}}%
\pgfpathlineto{\pgfqpoint{5.396588in}{3.207926in}}%
\pgfpathlineto{\pgfqpoint{5.396758in}{3.094799in}}%
\pgfpathlineto{\pgfqpoint{5.397313in}{3.246847in}}%
\pgfpathlineto{\pgfqpoint{5.397790in}{3.134860in}}%
\pgfpathlineto{\pgfqpoint{5.398099in}{3.260791in}}%
\pgfpathlineto{\pgfqpoint{5.398315in}{3.105498in}}%
\pgfpathlineto{\pgfqpoint{5.398916in}{3.181729in}}%
\pgfpathlineto{\pgfqpoint{5.398931in}{3.182732in}}%
\pgfpathlineto{\pgfqpoint{5.398993in}{3.122924in}}%
\pgfpathlineto{\pgfqpoint{5.399995in}{3.084020in}}%
\pgfpathlineto{\pgfqpoint{5.399486in}{3.261060in}}%
\pgfpathlineto{\pgfqpoint{5.400087in}{3.129144in}}%
\pgfpathlineto{\pgfqpoint{5.401197in}{3.277216in}}%
\pgfpathlineto{\pgfqpoint{5.400843in}{3.115215in}}%
\pgfpathlineto{\pgfqpoint{5.401274in}{3.193189in}}%
\pgfpathlineto{\pgfqpoint{5.401629in}{3.101565in}}%
\pgfpathlineto{\pgfqpoint{5.401737in}{3.221034in}}%
\pgfpathlineto{\pgfqpoint{5.402399in}{3.158973in}}%
\pgfpathlineto{\pgfqpoint{5.402754in}{3.255350in}}%
\pgfpathlineto{\pgfqpoint{5.402970in}{3.114400in}}%
\pgfpathlineto{\pgfqpoint{5.403556in}{3.212287in}}%
\pgfpathlineto{\pgfqpoint{5.403910in}{3.110130in}}%
\pgfpathlineto{\pgfqpoint{5.404373in}{3.252534in}}%
\pgfpathlineto{\pgfqpoint{5.404696in}{3.136253in}}%
\pgfpathlineto{\pgfqpoint{5.405220in}{3.117280in}}%
\pgfpathlineto{\pgfqpoint{5.405082in}{3.254254in}}%
\pgfpathlineto{\pgfqpoint{5.405344in}{3.217954in}}%
\pgfpathlineto{\pgfqpoint{5.405359in}{3.246013in}}%
\pgfpathlineto{\pgfqpoint{5.406222in}{3.102870in}}%
\pgfpathlineto{\pgfqpoint{5.406438in}{3.193571in}}%
\pgfpathlineto{\pgfqpoint{5.406639in}{3.247703in}}%
\pgfpathlineto{\pgfqpoint{5.406854in}{3.110201in}}%
\pgfpathlineto{\pgfqpoint{5.407502in}{3.164958in}}%
\pgfpathlineto{\pgfqpoint{5.407548in}{3.106800in}}%
\pgfpathlineto{\pgfqpoint{5.408041in}{3.245497in}}%
\pgfpathlineto{\pgfqpoint{5.408627in}{3.139160in}}%
\pgfpathlineto{\pgfqpoint{5.408966in}{3.257659in}}%
\pgfpathlineto{\pgfqpoint{5.409367in}{3.113985in}}%
\pgfpathlineto{\pgfqpoint{5.409783in}{3.191026in}}%
\pgfpathlineto{\pgfqpoint{5.410138in}{3.102452in}}%
\pgfpathlineto{\pgfqpoint{5.410261in}{3.254074in}}%
\pgfpathlineto{\pgfqpoint{5.410939in}{3.136841in}}%
\pgfpathlineto{\pgfqpoint{5.411278in}{3.240322in}}%
\pgfpathlineto{\pgfqpoint{5.411433in}{3.090631in}}%
\pgfpathlineto{\pgfqpoint{5.412095in}{3.196574in}}%
\pgfpathlineto{\pgfqpoint{5.413067in}{3.082067in}}%
\pgfpathlineto{\pgfqpoint{5.412851in}{3.243005in}}%
\pgfpathlineto{\pgfqpoint{5.413251in}{3.135650in}}%
\pgfpathlineto{\pgfqpoint{5.413267in}{3.133887in}}%
\pgfpathlineto{\pgfqpoint{5.413575in}{3.210985in}}%
\pgfpathlineto{\pgfqpoint{5.413621in}{3.270616in}}%
\pgfpathlineto{\pgfqpoint{5.414053in}{3.104489in}}%
\pgfpathlineto{\pgfqpoint{5.414623in}{3.114041in}}%
\pgfpathlineto{\pgfqpoint{5.415949in}{3.264507in}}%
\pgfpathlineto{\pgfqpoint{5.415317in}{3.104306in}}%
\pgfpathlineto{\pgfqpoint{5.415965in}{3.243461in}}%
\pgfpathlineto{\pgfqpoint{5.416334in}{3.110718in}}%
\pgfpathlineto{\pgfqpoint{5.417121in}{3.146872in}}%
\pgfpathlineto{\pgfqpoint{5.417151in}{3.140248in}}%
\pgfpathlineto{\pgfqpoint{5.417244in}{3.217239in}}%
\pgfpathlineto{\pgfqpoint{5.417444in}{3.213504in}}%
\pgfpathlineto{\pgfqpoint{5.417506in}{3.269336in}}%
\pgfpathlineto{\pgfqpoint{5.417722in}{3.096906in}}%
\pgfpathlineto{\pgfqpoint{5.418477in}{3.152963in}}%
\pgfpathlineto{\pgfqpoint{5.419279in}{3.088539in}}%
\pgfpathlineto{\pgfqpoint{5.419063in}{3.263893in}}%
\pgfpathlineto{\pgfqpoint{5.419510in}{3.182294in}}%
\pgfpathlineto{\pgfqpoint{5.419834in}{3.263878in}}%
\pgfpathlineto{\pgfqpoint{5.419972in}{3.092364in}}%
\pgfpathlineto{\pgfqpoint{5.420635in}{3.207757in}}%
\pgfpathlineto{\pgfqpoint{5.420959in}{3.092171in}}%
\pgfpathlineto{\pgfqpoint{5.421391in}{3.264027in}}%
\pgfpathlineto{\pgfqpoint{5.421760in}{3.163308in}}%
\pgfpathlineto{\pgfqpoint{5.421791in}{3.160125in}}%
\pgfpathlineto{\pgfqpoint{5.421992in}{3.207012in}}%
\pgfpathlineto{\pgfqpoint{5.422778in}{3.250521in}}%
\pgfpathlineto{\pgfqpoint{5.422285in}{3.072459in}}%
\pgfpathlineto{\pgfqpoint{5.423055in}{3.148242in}}%
\pgfpathlineto{\pgfqpoint{5.423934in}{3.055606in}}%
\pgfpathlineto{\pgfqpoint{5.423703in}{3.284520in}}%
\pgfpathlineto{\pgfqpoint{5.424150in}{3.171401in}}%
\pgfpathlineto{\pgfqpoint{5.424335in}{3.280231in}}%
\pgfpathlineto{\pgfqpoint{5.424612in}{3.092671in}}%
\pgfpathlineto{\pgfqpoint{5.425290in}{3.231243in}}%
\pgfpathlineto{\pgfqpoint{5.425491in}{3.086610in}}%
\pgfpathlineto{\pgfqpoint{5.425768in}{3.244939in}}%
\pgfpathlineto{\pgfqpoint{5.426524in}{3.157314in}}%
\pgfpathlineto{\pgfqpoint{5.427310in}{3.260316in}}%
\pgfpathlineto{\pgfqpoint{5.427048in}{3.081062in}}%
\pgfpathlineto{\pgfqpoint{5.427649in}{3.185017in}}%
\pgfpathlineto{\pgfqpoint{5.428497in}{3.070750in}}%
\pgfpathlineto{\pgfqpoint{5.428358in}{3.237686in}}%
\pgfpathlineto{\pgfqpoint{5.428820in}{3.156872in}}%
\pgfpathlineto{\pgfqpoint{5.429915in}{3.260797in}}%
\pgfpathlineto{\pgfqpoint{5.429360in}{3.091807in}}%
\pgfpathlineto{\pgfqpoint{5.429961in}{3.209925in}}%
\pgfpathlineto{\pgfqpoint{5.430824in}{3.080229in}}%
\pgfpathlineto{\pgfqpoint{5.430686in}{3.257410in}}%
\pgfpathlineto{\pgfqpoint{5.431102in}{3.151195in}}%
\pgfpathlineto{\pgfqpoint{5.432227in}{3.254962in}}%
\pgfpathlineto{\pgfqpoint{5.431842in}{3.108519in}}%
\pgfpathlineto{\pgfqpoint{5.432273in}{3.183814in}}%
\pgfpathlineto{\pgfqpoint{5.432458in}{3.086044in}}%
\pgfpathlineto{\pgfqpoint{5.432875in}{3.228206in}}%
\pgfpathlineto{\pgfqpoint{5.433414in}{3.099947in}}%
\pgfpathlineto{\pgfqpoint{5.433645in}{3.249445in}}%
\pgfpathlineto{\pgfqpoint{5.434015in}{3.096631in}}%
\pgfpathlineto{\pgfqpoint{5.434632in}{3.197499in}}%
\pgfpathlineto{\pgfqpoint{5.434801in}{3.084067in}}%
\pgfpathlineto{\pgfqpoint{5.435187in}{3.248382in}}%
\pgfpathlineto{\pgfqpoint{5.435773in}{3.152597in}}%
\pgfpathlineto{\pgfqpoint{5.436882in}{3.266967in}}%
\pgfpathlineto{\pgfqpoint{5.436250in}{3.076543in}}%
\pgfpathlineto{\pgfqpoint{5.436913in}{3.227806in}}%
\pgfpathlineto{\pgfqpoint{5.437792in}{3.087576in}}%
\pgfpathlineto{\pgfqpoint{5.437653in}{3.252286in}}%
\pgfpathlineto{\pgfqpoint{5.438054in}{3.139775in}}%
\pgfpathlineto{\pgfqpoint{5.438671in}{3.099091in}}%
\pgfpathlineto{\pgfqpoint{5.438424in}{3.248648in}}%
\pgfpathlineto{\pgfqpoint{5.439010in}{3.209677in}}%
\pgfpathlineto{\pgfqpoint{5.439842in}{3.267395in}}%
\pgfpathlineto{\pgfqpoint{5.439333in}{3.100372in}}%
\pgfpathlineto{\pgfqpoint{5.440042in}{3.176419in}}%
\pgfpathlineto{\pgfqpoint{5.440120in}{3.072500in}}%
\pgfpathlineto{\pgfqpoint{5.440613in}{3.244219in}}%
\pgfpathlineto{\pgfqpoint{5.441183in}{3.122878in}}%
\pgfpathlineto{\pgfqpoint{5.441384in}{3.243496in}}%
\pgfpathlineto{\pgfqpoint{5.441661in}{3.077325in}}%
\pgfpathlineto{\pgfqpoint{5.442370in}{3.170806in}}%
\pgfpathlineto{\pgfqpoint{5.443203in}{3.053201in}}%
\pgfpathlineto{\pgfqpoint{5.442940in}{3.253320in}}%
\pgfpathlineto{\pgfqpoint{5.443495in}{3.151734in}}%
\pgfpathlineto{\pgfqpoint{5.444482in}{3.250839in}}%
\pgfpathlineto{\pgfqpoint{5.444174in}{3.090345in}}%
\pgfpathlineto{\pgfqpoint{5.444652in}{3.200141in}}%
\pgfpathlineto{\pgfqpoint{5.444759in}{3.096072in}}%
\pgfpathlineto{\pgfqpoint{5.445253in}{3.240115in}}%
\pgfpathlineto{\pgfqpoint{5.445761in}{3.174690in}}%
\pgfpathlineto{\pgfqpoint{5.446779in}{3.262763in}}%
\pgfpathlineto{\pgfqpoint{5.446301in}{3.083716in}}%
\pgfpathlineto{\pgfqpoint{5.446933in}{3.232955in}}%
\pgfpathlineto{\pgfqpoint{5.447997in}{3.073954in}}%
\pgfpathlineto{\pgfqpoint{5.447457in}{3.250361in}}%
\pgfpathlineto{\pgfqpoint{5.448135in}{3.127177in}}%
\pgfpathlineto{\pgfqpoint{5.448212in}{3.254336in}}%
\pgfpathlineto{\pgfqpoint{5.448629in}{3.071312in}}%
\pgfpathlineto{\pgfqpoint{5.449291in}{3.210742in}}%
\pgfpathlineto{\pgfqpoint{5.450185in}{3.091048in}}%
\pgfpathlineto{\pgfqpoint{5.449769in}{3.244429in}}%
\pgfpathlineto{\pgfqpoint{5.450432in}{3.169962in}}%
\pgfpathlineto{\pgfqpoint{5.450956in}{3.038062in}}%
\pgfpathlineto{\pgfqpoint{5.450679in}{3.259051in}}%
\pgfpathlineto{\pgfqpoint{5.451280in}{3.197565in}}%
\pgfpathlineto{\pgfqpoint{5.451434in}{3.248172in}}%
\pgfpathlineto{\pgfqpoint{5.451727in}{3.084426in}}%
\pgfpathlineto{\pgfqpoint{5.452390in}{3.202108in}}%
\pgfpathlineto{\pgfqpoint{5.452498in}{3.057536in}}%
\pgfpathlineto{\pgfqpoint{5.452868in}{3.227753in}}%
\pgfpathlineto{\pgfqpoint{5.453546in}{3.141655in}}%
\pgfpathlineto{\pgfqpoint{5.453638in}{3.269595in}}%
\pgfpathlineto{\pgfqpoint{5.454039in}{3.063738in}}%
\pgfpathlineto{\pgfqpoint{5.454717in}{3.182261in}}%
\pgfpathlineto{\pgfqpoint{5.454748in}{3.191298in}}%
\pgfpathlineto{\pgfqpoint{5.454779in}{3.157574in}}%
\pgfpathlineto{\pgfqpoint{5.454825in}{3.049499in}}%
\pgfpathlineto{\pgfqpoint{5.455319in}{3.263701in}}%
\pgfpathlineto{\pgfqpoint{5.455889in}{3.139907in}}%
\pgfpathlineto{\pgfqpoint{5.456043in}{3.238960in}}%
\pgfpathlineto{\pgfqpoint{5.456382in}{3.090923in}}%
\pgfpathlineto{\pgfqpoint{5.457045in}{3.187811in}}%
\pgfpathlineto{\pgfqpoint{5.457924in}{3.061687in}}%
\pgfpathlineto{\pgfqpoint{5.457554in}{3.265433in}}%
\pgfpathlineto{\pgfqpoint{5.458170in}{3.164706in}}%
\pgfpathlineto{\pgfqpoint{5.458540in}{3.262532in}}%
\pgfpathlineto{\pgfqpoint{5.458694in}{3.062472in}}%
\pgfpathlineto{\pgfqpoint{5.459357in}{3.182558in}}%
\pgfpathlineto{\pgfqpoint{5.459465in}{3.081689in}}%
\pgfpathlineto{\pgfqpoint{5.459943in}{3.248119in}}%
\pgfpathlineto{\pgfqpoint{5.460467in}{3.169827in}}%
\pgfpathlineto{\pgfqpoint{5.461469in}{3.257120in}}%
\pgfpathlineto{\pgfqpoint{5.461022in}{3.068301in}}%
\pgfpathlineto{\pgfqpoint{5.461608in}{3.213779in}}%
\pgfpathlineto{\pgfqpoint{5.461793in}{3.056183in}}%
\pgfpathlineto{\pgfqpoint{5.462194in}{3.266181in}}%
\pgfpathlineto{\pgfqpoint{5.462826in}{3.143610in}}%
\pgfpathlineto{\pgfqpoint{5.463180in}{3.247553in}}%
\pgfpathlineto{\pgfqpoint{5.463334in}{3.075950in}}%
\pgfpathlineto{\pgfqpoint{5.463982in}{3.207370in}}%
\pgfpathlineto{\pgfqpoint{5.464891in}{3.071593in}}%
\pgfpathlineto{\pgfqpoint{5.464521in}{3.236757in}}%
\pgfpathlineto{\pgfqpoint{5.465122in}{3.168754in}}%
\pgfpathlineto{\pgfqpoint{5.466140in}{3.267139in}}%
\pgfpathlineto{\pgfqpoint{5.465662in}{3.052358in}}%
\pgfpathlineto{\pgfqpoint{5.466356in}{3.220662in}}%
\pgfpathlineto{\pgfqpoint{5.466525in}{3.057647in}}%
\pgfpathlineto{\pgfqpoint{5.466911in}{3.247458in}}%
\pgfpathlineto{\pgfqpoint{5.467527in}{3.170115in}}%
\pgfpathlineto{\pgfqpoint{5.468406in}{3.246429in}}%
\pgfpathlineto{\pgfqpoint{5.467990in}{3.084294in}}%
\pgfpathlineto{\pgfqpoint{5.468637in}{3.180528in}}%
\pgfpathlineto{\pgfqpoint{5.469624in}{3.056965in}}%
\pgfpathlineto{\pgfqpoint{5.469254in}{3.267405in}}%
\pgfpathlineto{\pgfqpoint{5.469793in}{3.129087in}}%
\pgfpathlineto{\pgfqpoint{5.470009in}{3.270833in}}%
\pgfpathlineto{\pgfqpoint{5.470425in}{3.054190in}}%
\pgfpathlineto{\pgfqpoint{5.470965in}{3.216745in}}%
\pgfpathlineto{\pgfqpoint{5.471165in}{3.072959in}}%
\pgfpathlineto{\pgfqpoint{5.471674in}{3.235027in}}%
\pgfpathlineto{\pgfqpoint{5.472152in}{3.170795in}}%
\pgfpathlineto{\pgfqpoint{5.473200in}{3.258917in}}%
\pgfpathlineto{\pgfqpoint{5.472722in}{3.047713in}}%
\pgfpathlineto{\pgfqpoint{5.473277in}{3.213117in}}%
\pgfpathlineto{\pgfqpoint{5.473400in}{3.048139in}}%
\pgfpathlineto{\pgfqpoint{5.473878in}{3.260349in}}%
\pgfpathlineto{\pgfqpoint{5.474448in}{3.125782in}}%
\pgfpathlineto{\pgfqpoint{5.474603in}{3.272442in}}%
\pgfpathlineto{\pgfqpoint{5.475065in}{3.061543in}}%
\pgfpathlineto{\pgfqpoint{5.475604in}{3.229580in}}%
\pgfpathlineto{\pgfqpoint{5.476637in}{3.069511in}}%
\pgfpathlineto{\pgfqpoint{5.476144in}{3.278699in}}%
\pgfpathlineto{\pgfqpoint{5.476761in}{3.135081in}}%
\pgfpathlineto{\pgfqpoint{5.476930in}{3.265472in}}%
\pgfpathlineto{\pgfqpoint{5.477362in}{3.047146in}}%
\pgfpathlineto{\pgfqpoint{5.477932in}{3.221056in}}%
\pgfpathlineto{\pgfqpoint{5.478950in}{3.074498in}}%
\pgfpathlineto{\pgfqpoint{5.478533in}{3.259491in}}%
\pgfpathlineto{\pgfqpoint{5.479088in}{3.164450in}}%
\pgfpathlineto{\pgfqpoint{5.480075in}{3.268884in}}%
\pgfpathlineto{\pgfqpoint{5.479674in}{3.051783in}}%
\pgfpathlineto{\pgfqpoint{5.480244in}{3.210981in}}%
\pgfpathlineto{\pgfqpoint{5.481231in}{3.068596in}}%
\pgfpathlineto{\pgfqpoint{5.480815in}{3.285472in}}%
\pgfpathlineto{\pgfqpoint{5.481431in}{3.159355in}}%
\pgfpathlineto{\pgfqpoint{5.482372in}{3.260506in}}%
\pgfpathlineto{\pgfqpoint{5.482002in}{3.080399in}}%
\pgfpathlineto{\pgfqpoint{5.482557in}{3.187991in}}%
\pgfpathlineto{\pgfqpoint{5.483574in}{3.049320in}}%
\pgfpathlineto{\pgfqpoint{5.483374in}{3.238911in}}%
\pgfpathlineto{\pgfqpoint{5.483728in}{3.159284in}}%
\pgfpathlineto{\pgfqpoint{5.484715in}{3.275682in}}%
\pgfpathlineto{\pgfqpoint{5.484252in}{3.076088in}}%
\pgfpathlineto{\pgfqpoint{5.484869in}{3.198401in}}%
\pgfpathlineto{\pgfqpoint{5.485902in}{3.038862in}}%
\pgfpathlineto{\pgfqpoint{5.485701in}{3.254631in}}%
\pgfpathlineto{\pgfqpoint{5.485994in}{3.180252in}}%
\pgfpathlineto{\pgfqpoint{5.486009in}{3.180525in}}%
\pgfpathlineto{\pgfqpoint{5.486025in}{3.165781in}}%
\pgfpathlineto{\pgfqpoint{5.486549in}{3.071225in}}%
\pgfpathlineto{\pgfqpoint{5.486996in}{3.274935in}}%
\pgfpathlineto{\pgfqpoint{5.487104in}{3.206644in}}%
\pgfpathlineto{\pgfqpoint{5.487736in}{3.255524in}}%
\pgfpathlineto{\pgfqpoint{5.487458in}{3.067470in}}%
\pgfpathlineto{\pgfqpoint{5.488137in}{3.151706in}}%
\pgfpathlineto{\pgfqpoint{5.488214in}{3.046486in}}%
\pgfpathlineto{\pgfqpoint{5.488599in}{3.253094in}}%
\pgfpathlineto{\pgfqpoint{5.489216in}{3.169803in}}%
\pgfpathlineto{\pgfqpoint{5.490094in}{3.267383in}}%
\pgfpathlineto{\pgfqpoint{5.489755in}{3.084561in}}%
\pgfpathlineto{\pgfqpoint{5.490356in}{3.221730in}}%
\pgfpathlineto{\pgfqpoint{5.490511in}{3.066395in}}%
\pgfpathlineto{\pgfqpoint{5.490911in}{3.265364in}}%
\pgfpathlineto{\pgfqpoint{5.491497in}{3.135422in}}%
\pgfpathlineto{\pgfqpoint{5.491667in}{3.278484in}}%
\pgfpathlineto{\pgfqpoint{5.492129in}{3.090177in}}%
\pgfpathlineto{\pgfqpoint{5.492669in}{3.247132in}}%
\pgfpathlineto{\pgfqpoint{5.492838in}{3.048300in}}%
\pgfpathlineto{\pgfqpoint{5.493224in}{3.248891in}}%
\pgfpathlineto{\pgfqpoint{5.493840in}{3.154222in}}%
\pgfpathlineto{\pgfqpoint{5.493994in}{3.254671in}}%
\pgfpathlineto{\pgfqpoint{5.494395in}{3.053552in}}%
\pgfpathlineto{\pgfqpoint{5.494981in}{3.225374in}}%
\pgfpathlineto{\pgfqpoint{5.494996in}{3.228102in}}%
\pgfpathlineto{\pgfqpoint{5.495135in}{3.071319in}}%
\pgfpathlineto{\pgfqpoint{5.495150in}{3.052719in}}%
\pgfpathlineto{\pgfqpoint{5.495536in}{3.279578in}}%
\pgfpathlineto{\pgfqpoint{5.496168in}{3.161375in}}%
\pgfpathlineto{\pgfqpoint{5.497077in}{3.249611in}}%
\pgfpathlineto{\pgfqpoint{5.496707in}{3.070881in}}%
\pgfpathlineto{\pgfqpoint{5.497309in}{3.218094in}}%
\pgfpathlineto{\pgfqpoint{5.498249in}{3.054147in}}%
\pgfpathlineto{\pgfqpoint{5.497863in}{3.280338in}}%
\pgfpathlineto{\pgfqpoint{5.498465in}{3.147042in}}%
\pgfpathlineto{\pgfqpoint{5.498634in}{3.262111in}}%
\pgfpathlineto{\pgfqpoint{5.499097in}{3.071799in}}%
\pgfpathlineto{\pgfqpoint{5.499621in}{3.236267in}}%
\pgfpathlineto{\pgfqpoint{5.499790in}{3.072707in}}%
\pgfpathlineto{\pgfqpoint{5.500792in}{3.163080in}}%
\pgfpathlineto{\pgfqpoint{5.500962in}{3.267580in}}%
\pgfpathlineto{\pgfqpoint{5.501347in}{3.064338in}}%
\pgfpathlineto{\pgfqpoint{5.501933in}{3.232274in}}%
\pgfpathlineto{\pgfqpoint{5.502503in}{3.277853in}}%
\pgfpathlineto{\pgfqpoint{5.502118in}{3.063410in}}%
\pgfpathlineto{\pgfqpoint{5.502858in}{3.120574in}}%
\pgfpathlineto{\pgfqpoint{5.503675in}{3.065987in}}%
\pgfpathlineto{\pgfqpoint{5.503490in}{3.248409in}}%
\pgfpathlineto{\pgfqpoint{5.503922in}{3.204993in}}%
\pgfpathlineto{\pgfqpoint{5.504785in}{3.265806in}}%
\pgfpathlineto{\pgfqpoint{5.504446in}{3.097587in}}%
\pgfpathlineto{\pgfqpoint{5.504985in}{3.200890in}}%
\pgfpathlineto{\pgfqpoint{5.506003in}{3.031699in}}%
\pgfpathlineto{\pgfqpoint{5.505756in}{3.264837in}}%
\pgfpathlineto{\pgfqpoint{5.506157in}{3.129040in}}%
\pgfpathlineto{\pgfqpoint{5.506388in}{3.264757in}}%
\pgfpathlineto{\pgfqpoint{5.506789in}{3.084517in}}%
\pgfpathlineto{\pgfqpoint{5.507390in}{3.231114in}}%
\pgfpathlineto{\pgfqpoint{5.507559in}{3.056142in}}%
\pgfpathlineto{\pgfqpoint{5.507852in}{3.239059in}}%
\pgfpathlineto{\pgfqpoint{5.508531in}{3.153270in}}%
\pgfpathlineto{\pgfqpoint{5.508700in}{3.295710in}}%
\pgfpathlineto{\pgfqpoint{5.509085in}{3.062532in}}%
\pgfpathlineto{\pgfqpoint{5.509718in}{3.209806in}}%
\pgfpathlineto{\pgfqpoint{5.509872in}{3.060145in}}%
\pgfpathlineto{\pgfqpoint{5.510242in}{3.261162in}}%
\pgfpathlineto{\pgfqpoint{5.510843in}{3.146158in}}%
\pgfpathlineto{\pgfqpoint{5.511783in}{3.251569in}}%
\pgfpathlineto{\pgfqpoint{5.511429in}{3.080169in}}%
\pgfpathlineto{\pgfqpoint{5.512030in}{3.204479in}}%
\pgfpathlineto{\pgfqpoint{5.512955in}{3.050383in}}%
\pgfpathlineto{\pgfqpoint{5.512585in}{3.264066in}}%
\pgfpathlineto{\pgfqpoint{5.513170in}{3.166423in}}%
\pgfpathlineto{\pgfqpoint{5.513340in}{3.265138in}}%
\pgfpathlineto{\pgfqpoint{5.513756in}{3.054949in}}%
\pgfpathlineto{\pgfqpoint{5.514342in}{3.228512in}}%
\pgfpathlineto{\pgfqpoint{5.514496in}{3.054774in}}%
\pgfpathlineto{\pgfqpoint{5.515498in}{3.177425in}}%
\pgfpathlineto{\pgfqpoint{5.515652in}{3.253808in}}%
\pgfpathlineto{\pgfqpoint{5.516053in}{3.044864in}}%
\pgfpathlineto{\pgfqpoint{5.516623in}{3.222467in}}%
\pgfpathlineto{\pgfqpoint{5.516839in}{3.057124in}}%
\pgfpathlineto{\pgfqpoint{5.517209in}{3.277818in}}%
\pgfpathlineto{\pgfqpoint{5.517872in}{3.190357in}}%
\pgfpathlineto{\pgfqpoint{5.518149in}{3.255169in}}%
\pgfpathlineto{\pgfqpoint{5.518381in}{3.055498in}}%
\pgfpathlineto{\pgfqpoint{5.518982in}{3.192028in}}%
\pgfpathlineto{\pgfqpoint{5.519167in}{3.109019in}}%
\pgfpathlineto{\pgfqpoint{5.519444in}{3.234112in}}%
\pgfpathlineto{\pgfqpoint{5.519537in}{3.255408in}}%
\pgfpathlineto{\pgfqpoint{5.519922in}{3.055563in}}%
\pgfpathlineto{\pgfqpoint{5.519938in}{3.051935in}}%
\pgfpathlineto{\pgfqpoint{5.520277in}{3.231532in}}%
\pgfpathlineto{\pgfqpoint{5.521094in}{3.276321in}}%
\pgfpathlineto{\pgfqpoint{5.520708in}{3.064826in}}%
\pgfpathlineto{\pgfqpoint{5.521294in}{3.203683in}}%
\pgfpathlineto{\pgfqpoint{5.522250in}{3.056500in}}%
\pgfpathlineto{\pgfqpoint{5.521803in}{3.256967in}}%
\pgfpathlineto{\pgfqpoint{5.522450in}{3.167898in}}%
\pgfpathlineto{\pgfqpoint{5.523329in}{3.255093in}}%
\pgfpathlineto{\pgfqpoint{5.523021in}{3.084637in}}%
\pgfpathlineto{\pgfqpoint{5.523560in}{3.180309in}}%
\pgfpathlineto{\pgfqpoint{5.524300in}{3.255990in}}%
\pgfpathlineto{\pgfqpoint{5.523822in}{3.049658in}}%
\pgfpathlineto{\pgfqpoint{5.524485in}{3.093404in}}%
\pgfpathlineto{\pgfqpoint{5.524547in}{3.052081in}}%
\pgfpathlineto{\pgfqpoint{5.524947in}{3.248374in}}%
\pgfpathlineto{\pgfqpoint{5.525533in}{3.112379in}}%
\pgfpathlineto{\pgfqpoint{5.525718in}{3.293214in}}%
\pgfpathlineto{\pgfqpoint{5.526134in}{3.072488in}}%
\pgfpathlineto{\pgfqpoint{5.526689in}{3.254714in}}%
\pgfpathlineto{\pgfqpoint{5.526890in}{3.072855in}}%
\pgfpathlineto{\pgfqpoint{5.527229in}{3.277326in}}%
\pgfpathlineto{\pgfqpoint{5.527907in}{3.196921in}}%
\pgfpathlineto{\pgfqpoint{5.528200in}{3.268720in}}%
\pgfpathlineto{\pgfqpoint{5.528431in}{3.021854in}}%
\pgfpathlineto{\pgfqpoint{5.529048in}{3.211127in}}%
\pgfpathlineto{\pgfqpoint{5.529233in}{3.078560in}}%
\pgfpathlineto{\pgfqpoint{5.529603in}{3.298329in}}%
\pgfpathlineto{\pgfqpoint{5.530188in}{3.163079in}}%
\pgfpathlineto{\pgfqpoint{5.531144in}{3.275970in}}%
\pgfpathlineto{\pgfqpoint{5.530759in}{3.058554in}}%
\pgfpathlineto{\pgfqpoint{5.531329in}{3.181021in}}%
\pgfpathlineto{\pgfqpoint{5.532316in}{3.031042in}}%
\pgfpathlineto{\pgfqpoint{5.531915in}{3.277006in}}%
\pgfpathlineto{\pgfqpoint{5.532485in}{3.135138in}}%
\pgfpathlineto{\pgfqpoint{5.533487in}{3.268397in}}%
\pgfpathlineto{\pgfqpoint{5.533071in}{3.092927in}}%
\pgfpathlineto{\pgfqpoint{5.533641in}{3.212922in}}%
\pgfpathlineto{\pgfqpoint{5.534659in}{3.057433in}}%
\pgfpathlineto{\pgfqpoint{5.534196in}{3.237194in}}%
\pgfpathlineto{\pgfqpoint{5.534813in}{3.163074in}}%
\pgfpathlineto{\pgfqpoint{5.535799in}{3.286416in}}%
\pgfpathlineto{\pgfqpoint{5.535399in}{3.069374in}}%
\pgfpathlineto{\pgfqpoint{5.535954in}{3.203271in}}%
\pgfpathlineto{\pgfqpoint{5.536015in}{3.220108in}}%
\pgfpathlineto{\pgfqpoint{5.536046in}{3.189716in}}%
\pgfpathlineto{\pgfqpoint{5.536940in}{3.053606in}}%
\pgfpathlineto{\pgfqpoint{5.536786in}{3.254324in}}%
\pgfpathlineto{\pgfqpoint{5.537171in}{3.155573in}}%
\pgfpathlineto{\pgfqpoint{5.537449in}{3.253264in}}%
\pgfpathlineto{\pgfqpoint{5.537711in}{3.083738in}}%
\pgfpathlineto{\pgfqpoint{5.538358in}{3.197057in}}%
\pgfpathlineto{\pgfqpoint{5.539268in}{3.045282in}}%
\pgfpathlineto{\pgfqpoint{5.539006in}{3.242728in}}%
\pgfpathlineto{\pgfqpoint{5.539499in}{3.151255in}}%
\pgfpathlineto{\pgfqpoint{5.540424in}{3.269722in}}%
\pgfpathlineto{\pgfqpoint{5.540054in}{3.063465in}}%
\pgfpathlineto{\pgfqpoint{5.540701in}{3.200853in}}%
\pgfpathlineto{\pgfqpoint{5.541580in}{3.050020in}}%
\pgfpathlineto{\pgfqpoint{5.541364in}{3.254017in}}%
\pgfpathlineto{\pgfqpoint{5.541857in}{3.153996in}}%
\pgfpathlineto{\pgfqpoint{5.542705in}{3.279547in}}%
\pgfpathlineto{\pgfqpoint{5.542459in}{3.092986in}}%
\pgfpathlineto{\pgfqpoint{5.542998in}{3.178834in}}%
\pgfpathlineto{\pgfqpoint{5.543168in}{3.048458in}}%
\pgfpathlineto{\pgfqpoint{5.543692in}{3.276495in}}%
\pgfpathlineto{\pgfqpoint{5.544123in}{3.139702in}}%
\pgfpathlineto{\pgfqpoint{5.544832in}{3.095597in}}%
\pgfpathlineto{\pgfqpoint{5.544247in}{3.253840in}}%
\pgfpathlineto{\pgfqpoint{5.545002in}{3.199250in}}%
\pgfpathlineto{\pgfqpoint{5.545064in}{3.267315in}}%
\pgfpathlineto{\pgfqpoint{5.545464in}{3.037931in}}%
\pgfpathlineto{\pgfqpoint{5.546081in}{3.210670in}}%
\pgfpathlineto{\pgfqpoint{5.547006in}{3.074696in}}%
\pgfpathlineto{\pgfqpoint{5.546636in}{3.295727in}}%
\pgfpathlineto{\pgfqpoint{5.547206in}{3.169086in}}%
\pgfpathlineto{\pgfqpoint{5.547607in}{3.257387in}}%
\pgfpathlineto{\pgfqpoint{5.547746in}{3.094643in}}%
\pgfpathlineto{\pgfqpoint{5.547792in}{3.021441in}}%
\pgfpathlineto{\pgfqpoint{5.548178in}{3.272374in}}%
\pgfpathlineto{\pgfqpoint{5.548810in}{3.151310in}}%
\pgfpathlineto{\pgfqpoint{5.549873in}{3.263658in}}%
\pgfpathlineto{\pgfqpoint{5.549364in}{3.048712in}}%
\pgfpathlineto{\pgfqpoint{5.549935in}{3.190625in}}%
\pgfpathlineto{\pgfqpoint{5.550505in}{3.264327in}}%
\pgfpathlineto{\pgfqpoint{5.550104in}{3.055214in}}%
\pgfpathlineto{\pgfqpoint{5.550752in}{3.138239in}}%
\pgfpathlineto{\pgfqpoint{5.550906in}{3.071953in}}%
\pgfpathlineto{\pgfqpoint{5.551276in}{3.259585in}}%
\pgfpathlineto{\pgfqpoint{5.551831in}{3.168297in}}%
\pgfpathlineto{\pgfqpoint{5.552817in}{3.282707in}}%
\pgfpathlineto{\pgfqpoint{5.552432in}{3.063103in}}%
\pgfpathlineto{\pgfqpoint{5.553049in}{3.195205in}}%
\pgfpathlineto{\pgfqpoint{5.553203in}{3.047731in}}%
\pgfpathlineto{\pgfqpoint{5.553588in}{3.277510in}}%
\pgfpathlineto{\pgfqpoint{5.554189in}{3.126146in}}%
\pgfpathlineto{\pgfqpoint{5.554528in}{3.253190in}}%
\pgfpathlineto{\pgfqpoint{5.554744in}{3.066399in}}%
\pgfpathlineto{\pgfqpoint{5.555361in}{3.202261in}}%
\pgfpathlineto{\pgfqpoint{5.556301in}{3.053618in}}%
\pgfpathlineto{\pgfqpoint{5.555916in}{3.272524in}}%
\pgfpathlineto{\pgfqpoint{5.556502in}{3.162625in}}%
\pgfpathlineto{\pgfqpoint{5.557473in}{3.266460in}}%
\pgfpathlineto{\pgfqpoint{5.557072in}{3.073304in}}%
\pgfpathlineto{\pgfqpoint{5.557642in}{3.211963in}}%
\pgfpathlineto{\pgfqpoint{5.558629in}{3.056544in}}%
\pgfpathlineto{\pgfqpoint{5.558398in}{3.246478in}}%
\pgfpathlineto{\pgfqpoint{5.558814in}{3.124334in}}%
\pgfpathlineto{\pgfqpoint{5.559739in}{3.246461in}}%
\pgfpathlineto{\pgfqpoint{5.559400in}{3.076352in}}%
\pgfpathlineto{\pgfqpoint{5.559970in}{3.203907in}}%
\pgfpathlineto{\pgfqpoint{5.560956in}{3.054914in}}%
\pgfpathlineto{\pgfqpoint{5.560710in}{3.263003in}}%
\pgfpathlineto{\pgfqpoint{5.561157in}{3.141873in}}%
\pgfpathlineto{\pgfqpoint{5.561342in}{3.254837in}}%
\pgfpathlineto{\pgfqpoint{5.561727in}{3.074762in}}%
\pgfpathlineto{\pgfqpoint{5.562359in}{3.192942in}}%
\pgfpathlineto{\pgfqpoint{5.563269in}{3.071572in}}%
\pgfpathlineto{\pgfqpoint{5.562883in}{3.257086in}}%
\pgfpathlineto{\pgfqpoint{5.563484in}{3.155163in}}%
\pgfpathlineto{\pgfqpoint{5.563654in}{3.283749in}}%
\pgfpathlineto{\pgfqpoint{5.564039in}{3.046139in}}%
\pgfpathlineto{\pgfqpoint{5.564671in}{3.191174in}}%
\pgfpathlineto{\pgfqpoint{5.564810in}{3.070465in}}%
\pgfpathlineto{\pgfqpoint{5.565365in}{3.261426in}}%
\pgfpathlineto{\pgfqpoint{5.565797in}{3.148466in}}%
\pgfpathlineto{\pgfqpoint{5.566706in}{3.242661in}}%
\pgfpathlineto{\pgfqpoint{5.566367in}{3.097032in}}%
\pgfpathlineto{\pgfqpoint{5.566953in}{3.198946in}}%
\pgfpathlineto{\pgfqpoint{5.567138in}{3.064494in}}%
\pgfpathlineto{\pgfqpoint{5.567508in}{3.242753in}}%
\pgfpathlineto{\pgfqpoint{5.567539in}{3.276526in}}%
\pgfpathlineto{\pgfqpoint{5.567908in}{3.053906in}}%
\pgfpathlineto{\pgfqpoint{5.568556in}{3.190504in}}%
\pgfpathlineto{\pgfqpoint{5.568695in}{3.057740in}}%
\pgfpathlineto{\pgfqpoint{5.569234in}{3.255914in}}%
\pgfpathlineto{\pgfqpoint{5.569681in}{3.141958in}}%
\pgfpathlineto{\pgfqpoint{5.570606in}{3.247820in}}%
\pgfpathlineto{\pgfqpoint{5.570236in}{3.084232in}}%
\pgfpathlineto{\pgfqpoint{5.570868in}{3.191463in}}%
\pgfpathlineto{\pgfqpoint{5.571007in}{3.059282in}}%
\pgfpathlineto{\pgfqpoint{5.571408in}{3.259584in}}%
\pgfpathlineto{\pgfqpoint{5.572009in}{3.153457in}}%
\pgfpathlineto{\pgfqpoint{5.572163in}{3.270805in}}%
\pgfpathlineto{\pgfqpoint{5.572564in}{3.055057in}}%
\pgfpathlineto{\pgfqpoint{5.573196in}{3.208631in}}%
\pgfpathlineto{\pgfqpoint{5.573319in}{3.066720in}}%
\pgfpathlineto{\pgfqpoint{5.573720in}{3.250777in}}%
\pgfpathlineto{\pgfqpoint{5.574321in}{3.173808in}}%
\pgfpathlineto{\pgfqpoint{5.575292in}{3.265245in}}%
\pgfpathlineto{\pgfqpoint{5.574876in}{3.054872in}}%
\pgfpathlineto{\pgfqpoint{5.575446in}{3.232757in}}%
\pgfpathlineto{\pgfqpoint{5.576448in}{3.058046in}}%
\pgfpathlineto{\pgfqpoint{5.576048in}{3.257440in}}%
\pgfpathlineto{\pgfqpoint{5.576618in}{3.129718in}}%
\pgfpathlineto{\pgfqpoint{5.576834in}{3.252707in}}%
\pgfpathlineto{\pgfqpoint{5.577204in}{3.058372in}}%
\pgfpathlineto{\pgfqpoint{5.577836in}{3.206367in}}%
\pgfpathlineto{\pgfqpoint{5.577974in}{3.056791in}}%
\pgfpathlineto{\pgfqpoint{5.578360in}{3.252656in}}%
\pgfpathlineto{\pgfqpoint{5.578976in}{3.154756in}}%
\pgfpathlineto{\pgfqpoint{5.579161in}{3.264459in}}%
\pgfpathlineto{\pgfqpoint{5.579531in}{3.072191in}}%
\pgfpathlineto{\pgfqpoint{5.580148in}{3.238827in}}%
\pgfpathlineto{\pgfqpoint{5.580317in}{3.056702in}}%
\pgfpathlineto{\pgfqpoint{5.580703in}{3.275314in}}%
\pgfpathlineto{\pgfqpoint{5.581396in}{3.188260in}}%
\pgfpathlineto{\pgfqpoint{5.581612in}{3.258782in}}%
\pgfpathlineto{\pgfqpoint{5.581859in}{3.052456in}}%
\pgfpathlineto{\pgfqpoint{5.582491in}{3.190174in}}%
\pgfpathlineto{\pgfqpoint{5.583400in}{3.053232in}}%
\pgfpathlineto{\pgfqpoint{5.583030in}{3.256927in}}%
\pgfpathlineto{\pgfqpoint{5.583601in}{3.154668in}}%
\pgfpathlineto{\pgfqpoint{5.584541in}{3.270035in}}%
\pgfpathlineto{\pgfqpoint{5.584187in}{3.066886in}}%
\pgfpathlineto{\pgfqpoint{5.584741in}{3.217787in}}%
\pgfpathlineto{\pgfqpoint{5.584942in}{3.050942in}}%
\pgfpathlineto{\pgfqpoint{5.585497in}{3.267195in}}%
\pgfpathlineto{\pgfqpoint{5.585928in}{3.131627in}}%
\pgfpathlineto{\pgfqpoint{5.586113in}{3.253752in}}%
\pgfpathlineto{\pgfqpoint{5.586499in}{3.087151in}}%
\pgfpathlineto{\pgfqpoint{5.587085in}{3.214538in}}%
\pgfpathlineto{\pgfqpoint{5.587254in}{3.051179in}}%
\pgfpathlineto{\pgfqpoint{5.587609in}{3.261002in}}%
\pgfpathlineto{\pgfqpoint{5.588256in}{3.144440in}}%
\pgfpathlineto{\pgfqpoint{5.588426in}{3.280195in}}%
\pgfpathlineto{\pgfqpoint{5.588826in}{3.060209in}}%
\pgfpathlineto{\pgfqpoint{5.589428in}{3.225827in}}%
\pgfpathlineto{\pgfqpoint{5.589551in}{3.126334in}}%
\pgfpathlineto{\pgfqpoint{5.589613in}{3.064782in}}%
\pgfpathlineto{\pgfqpoint{5.589905in}{3.256091in}}%
\pgfpathlineto{\pgfqpoint{5.590615in}{3.156853in}}%
\pgfpathlineto{\pgfqpoint{5.591524in}{3.272918in}}%
\pgfpathlineto{\pgfqpoint{5.591139in}{3.024283in}}%
\pgfpathlineto{\pgfqpoint{5.591755in}{3.190868in}}%
\pgfpathlineto{\pgfqpoint{5.591786in}{3.179208in}}%
\pgfpathlineto{\pgfqpoint{5.591909in}{3.049013in}}%
\pgfpathlineto{\pgfqpoint{5.592295in}{3.278425in}}%
\pgfpathlineto{\pgfqpoint{5.592911in}{3.148310in}}%
\pgfpathlineto{\pgfqpoint{5.593805in}{3.274171in}}%
\pgfpathlineto{\pgfqpoint{5.593466in}{3.088212in}}%
\pgfpathlineto{\pgfqpoint{5.594052in}{3.181446in}}%
\pgfpathlineto{\pgfqpoint{5.594175in}{3.089786in}}%
\pgfpathlineto{\pgfqpoint{5.594622in}{3.248594in}}%
\pgfpathlineto{\pgfqpoint{5.594746in}{3.235377in}}%
\pgfpathlineto{\pgfqpoint{5.595409in}{3.281620in}}%
\pgfpathlineto{\pgfqpoint{5.595008in}{3.023348in}}%
\pgfpathlineto{\pgfqpoint{5.595748in}{3.123290in}}%
\pgfpathlineto{\pgfqpoint{5.595809in}{3.065089in}}%
\pgfpathlineto{\pgfqpoint{5.596195in}{3.283050in}}%
\pgfpathlineto{\pgfqpoint{5.596811in}{3.151699in}}%
\pgfpathlineto{\pgfqpoint{5.597721in}{3.285111in}}%
\pgfpathlineto{\pgfqpoint{5.597335in}{3.088817in}}%
\pgfpathlineto{\pgfqpoint{5.597937in}{3.179449in}}%
\pgfpathlineto{\pgfqpoint{5.598877in}{3.038393in}}%
\pgfpathlineto{\pgfqpoint{5.598492in}{3.267289in}}%
\pgfpathlineto{\pgfqpoint{5.599093in}{3.135241in}}%
\pgfpathlineto{\pgfqpoint{5.599278in}{3.281969in}}%
\pgfpathlineto{\pgfqpoint{5.599648in}{3.068443in}}%
\pgfpathlineto{\pgfqpoint{5.600264in}{3.229560in}}%
\pgfpathlineto{\pgfqpoint{5.601189in}{3.061356in}}%
\pgfpathlineto{\pgfqpoint{5.600804in}{3.248779in}}%
\pgfpathlineto{\pgfqpoint{5.601451in}{3.189853in}}%
\pgfpathlineto{\pgfqpoint{5.602376in}{3.255888in}}%
\pgfpathlineto{\pgfqpoint{5.601914in}{3.063535in}}%
\pgfpathlineto{\pgfqpoint{5.602561in}{3.204497in}}%
\pgfpathlineto{\pgfqpoint{5.603532in}{3.042443in}}%
\pgfpathlineto{\pgfqpoint{5.603101in}{3.266844in}}%
\pgfpathlineto{\pgfqpoint{5.603733in}{3.124753in}}%
\pgfpathlineto{\pgfqpoint{5.604627in}{3.277871in}}%
\pgfpathlineto{\pgfqpoint{5.604303in}{3.063806in}}%
\pgfpathlineto{\pgfqpoint{5.604889in}{3.211018in}}%
\pgfpathlineto{\pgfqpoint{5.605844in}{3.030866in}}%
\pgfpathlineto{\pgfqpoint{5.605629in}{3.283153in}}%
\pgfpathlineto{\pgfqpoint{5.606106in}{3.168173in}}%
\pgfpathlineto{\pgfqpoint{5.607031in}{3.275067in}}%
\pgfpathlineto{\pgfqpoint{5.606646in}{3.065777in}}%
\pgfpathlineto{\pgfqpoint{5.607247in}{3.210822in}}%
\pgfpathlineto{\pgfqpoint{5.608172in}{3.058124in}}%
\pgfpathlineto{\pgfqpoint{5.607725in}{3.249586in}}%
\pgfpathlineto{\pgfqpoint{5.608434in}{3.157564in}}%
\pgfpathlineto{\pgfqpoint{5.608573in}{3.299610in}}%
\pgfpathlineto{\pgfqpoint{5.608943in}{3.050362in}}%
\pgfpathlineto{\pgfqpoint{5.609590in}{3.209747in}}%
\pgfpathlineto{\pgfqpoint{5.610469in}{3.046090in}}%
\pgfpathlineto{\pgfqpoint{5.610284in}{3.276172in}}%
\pgfpathlineto{\pgfqpoint{5.610715in}{3.145829in}}%
\pgfpathlineto{\pgfqpoint{5.611440in}{3.087234in}}%
\pgfpathlineto{\pgfqpoint{5.610854in}{3.259235in}}%
\pgfpathlineto{\pgfqpoint{5.611502in}{3.178822in}}%
\pgfpathlineto{\pgfqpoint{5.611841in}{3.270648in}}%
\pgfpathlineto{\pgfqpoint{5.612041in}{3.034885in}}%
\pgfpathlineto{\pgfqpoint{5.612642in}{3.234504in}}%
\pgfpathlineto{\pgfqpoint{5.612750in}{3.065757in}}%
\pgfpathlineto{\pgfqpoint{5.613213in}{3.265339in}}%
\pgfpathlineto{\pgfqpoint{5.613798in}{3.160242in}}%
\pgfpathlineto{\pgfqpoint{5.614924in}{3.270198in}}%
\pgfpathlineto{\pgfqpoint{5.614369in}{3.046584in}}%
\pgfpathlineto{\pgfqpoint{5.614955in}{3.214986in}}%
\pgfpathlineto{\pgfqpoint{5.615941in}{3.069386in}}%
\pgfpathlineto{\pgfqpoint{5.615463in}{3.278658in}}%
\pgfpathlineto{\pgfqpoint{5.616095in}{3.125352in}}%
\pgfpathlineto{\pgfqpoint{5.616450in}{3.270810in}}%
\pgfpathlineto{\pgfqpoint{5.616681in}{3.051279in}}%
\pgfpathlineto{\pgfqpoint{5.617313in}{3.181245in}}%
\pgfpathlineto{\pgfqpoint{5.617483in}{3.066974in}}%
\pgfpathlineto{\pgfqpoint{5.617853in}{3.280724in}}%
\pgfpathlineto{\pgfqpoint{5.618469in}{3.145989in}}%
\pgfpathlineto{\pgfqpoint{5.619379in}{3.283139in}}%
\pgfpathlineto{\pgfqpoint{5.619009in}{3.080254in}}%
\pgfpathlineto{\pgfqpoint{5.619610in}{3.194996in}}%
\pgfpathlineto{\pgfqpoint{5.620566in}{3.061145in}}%
\pgfpathlineto{\pgfqpoint{5.620350in}{3.288165in}}%
\pgfpathlineto{\pgfqpoint{5.620751in}{3.145033in}}%
\pgfpathlineto{\pgfqpoint{5.621737in}{3.272999in}}%
\pgfpathlineto{\pgfqpoint{5.621259in}{3.079813in}}%
\pgfpathlineto{\pgfqpoint{5.621907in}{3.240071in}}%
\pgfpathlineto{\pgfqpoint{5.622107in}{3.074985in}}%
\pgfpathlineto{\pgfqpoint{5.622677in}{3.245951in}}%
\pgfpathlineto{\pgfqpoint{5.623109in}{3.144242in}}%
\pgfpathlineto{\pgfqpoint{5.623279in}{3.275710in}}%
\pgfpathlineto{\pgfqpoint{5.623633in}{3.087726in}}%
\pgfpathlineto{\pgfqpoint{5.624296in}{3.195714in}}%
\pgfpathlineto{\pgfqpoint{5.624435in}{3.053599in}}%
\pgfpathlineto{\pgfqpoint{5.624805in}{3.240840in}}%
\pgfpathlineto{\pgfqpoint{5.625452in}{3.144806in}}%
\pgfpathlineto{\pgfqpoint{5.625622in}{3.265474in}}%
\pgfpathlineto{\pgfqpoint{5.625961in}{3.075790in}}%
\pgfpathlineto{\pgfqpoint{5.626624in}{3.223162in}}%
\pgfpathlineto{\pgfqpoint{5.627487in}{3.075771in}}%
\pgfpathlineto{\pgfqpoint{5.627148in}{3.268585in}}%
\pgfpathlineto{\pgfqpoint{5.627764in}{3.162364in}}%
\pgfpathlineto{\pgfqpoint{5.628103in}{3.261618in}}%
\pgfpathlineto{\pgfqpoint{5.628304in}{3.046760in}}%
\pgfpathlineto{\pgfqpoint{5.628905in}{3.197541in}}%
\pgfpathlineto{\pgfqpoint{5.629013in}{3.056184in}}%
\pgfpathlineto{\pgfqpoint{5.629506in}{3.265615in}}%
\pgfpathlineto{\pgfqpoint{5.630061in}{3.170909in}}%
\pgfpathlineto{\pgfqpoint{5.630570in}{3.100999in}}%
\pgfpathlineto{\pgfqpoint{5.630477in}{3.243359in}}%
\pgfpathlineto{\pgfqpoint{5.630924in}{3.200984in}}%
\pgfpathlineto{\pgfqpoint{5.631017in}{3.270316in}}%
\pgfpathlineto{\pgfqpoint{5.631371in}{3.070624in}}%
\pgfpathlineto{\pgfqpoint{5.632019in}{3.211518in}}%
\pgfpathlineto{\pgfqpoint{5.632188in}{3.045679in}}%
\pgfpathlineto{\pgfqpoint{5.632527in}{3.270142in}}%
\pgfpathlineto{\pgfqpoint{5.633190in}{3.117988in}}%
\pgfpathlineto{\pgfqpoint{5.633514in}{3.272974in}}%
\pgfpathlineto{\pgfqpoint{5.633699in}{3.085940in}}%
\pgfpathlineto{\pgfqpoint{5.634362in}{3.217082in}}%
\pgfpathlineto{\pgfqpoint{5.635225in}{3.068732in}}%
\pgfpathlineto{\pgfqpoint{5.634886in}{3.263233in}}%
\pgfpathlineto{\pgfqpoint{5.635503in}{3.165149in}}%
\pgfpathlineto{\pgfqpoint{5.636397in}{3.303186in}}%
\pgfpathlineto{\pgfqpoint{5.636073in}{3.039787in}}%
\pgfpathlineto{\pgfqpoint{5.636659in}{3.207210in}}%
\pgfpathlineto{\pgfqpoint{5.636705in}{3.178420in}}%
\pgfpathlineto{\pgfqpoint{5.636767in}{3.053137in}}%
\pgfpathlineto{\pgfqpoint{5.637383in}{3.299906in}}%
\pgfpathlineto{\pgfqpoint{5.637815in}{3.152102in}}%
\pgfpathlineto{\pgfqpoint{5.638401in}{3.113299in}}%
\pgfpathlineto{\pgfqpoint{5.638185in}{3.250839in}}%
\pgfpathlineto{\pgfqpoint{5.638724in}{3.240218in}}%
\pgfpathlineto{\pgfqpoint{5.639726in}{3.271483in}}%
\pgfpathlineto{\pgfqpoint{5.639094in}{3.057495in}}%
\pgfpathlineto{\pgfqpoint{5.639772in}{3.187294in}}%
\pgfpathlineto{\pgfqpoint{5.639927in}{3.028748in}}%
\pgfpathlineto{\pgfqpoint{5.640281in}{3.302752in}}%
\pgfpathlineto{\pgfqpoint{5.640959in}{3.158419in}}%
\pgfpathlineto{\pgfqpoint{5.641268in}{3.295027in}}%
\pgfpathlineto{\pgfqpoint{5.641483in}{3.085822in}}%
\pgfpathlineto{\pgfqpoint{5.642100in}{3.222267in}}%
\pgfpathlineto{\pgfqpoint{5.642979in}{3.058307in}}%
\pgfpathlineto{\pgfqpoint{5.642624in}{3.264737in}}%
\pgfpathlineto{\pgfqpoint{5.643256in}{3.186987in}}%
\pgfpathlineto{\pgfqpoint{5.644150in}{3.286963in}}%
\pgfpathlineto{\pgfqpoint{5.643811in}{3.019998in}}%
\pgfpathlineto{\pgfqpoint{5.644381in}{3.230140in}}%
\pgfpathlineto{\pgfqpoint{5.644505in}{3.073876in}}%
\pgfpathlineto{\pgfqpoint{5.645137in}{3.276161in}}%
\pgfpathlineto{\pgfqpoint{5.645538in}{3.160145in}}%
\pgfpathlineto{\pgfqpoint{5.646046in}{3.098801in}}%
\pgfpathlineto{\pgfqpoint{5.645923in}{3.230430in}}%
\pgfpathlineto{\pgfqpoint{5.646447in}{3.228274in}}%
\pgfpathlineto{\pgfqpoint{5.647233in}{3.273761in}}%
\pgfpathlineto{\pgfqpoint{5.646832in}{3.055069in}}%
\pgfpathlineto{\pgfqpoint{5.647526in}{3.194070in}}%
\pgfpathlineto{\pgfqpoint{5.647665in}{3.033709in}}%
\pgfpathlineto{\pgfqpoint{5.648019in}{3.280155in}}%
\pgfpathlineto{\pgfqpoint{5.648682in}{3.114936in}}%
\pgfpathlineto{\pgfqpoint{5.649006in}{3.271258in}}%
\pgfpathlineto{\pgfqpoint{5.649222in}{3.105323in}}%
\pgfpathlineto{\pgfqpoint{5.649869in}{3.202972in}}%
\pgfpathlineto{\pgfqpoint{5.650717in}{3.051887in}}%
\pgfpathlineto{\pgfqpoint{5.650393in}{3.269822in}}%
\pgfpathlineto{\pgfqpoint{5.650979in}{3.166917in}}%
\pgfpathlineto{\pgfqpoint{5.651133in}{3.270394in}}%
\pgfpathlineto{\pgfqpoint{5.651549in}{3.045478in}}%
\pgfpathlineto{\pgfqpoint{5.652166in}{3.216767in}}%
\pgfpathlineto{\pgfqpoint{5.652258in}{3.075436in}}%
\pgfpathlineto{\pgfqpoint{5.652890in}{3.243353in}}%
\pgfpathlineto{\pgfqpoint{5.653307in}{3.169974in}}%
\pgfpathlineto{\pgfqpoint{5.654201in}{3.243194in}}%
\pgfpathlineto{\pgfqpoint{5.653846in}{3.087611in}}%
\pgfpathlineto{\pgfqpoint{5.654432in}{3.228868in}}%
\pgfpathlineto{\pgfqpoint{5.654571in}{3.036812in}}%
\pgfpathlineto{\pgfqpoint{5.655049in}{3.268429in}}%
\pgfpathlineto{\pgfqpoint{5.655619in}{3.122189in}}%
\pgfpathlineto{\pgfqpoint{5.655742in}{3.274441in}}%
\pgfpathlineto{\pgfqpoint{5.656189in}{3.077466in}}%
\pgfpathlineto{\pgfqpoint{5.656837in}{3.194145in}}%
\pgfpathlineto{\pgfqpoint{5.657746in}{3.074504in}}%
\pgfpathlineto{\pgfqpoint{5.657284in}{3.257990in}}%
\pgfpathlineto{\pgfqpoint{5.657947in}{3.164508in}}%
\pgfpathlineto{\pgfqpoint{5.658054in}{3.267190in}}%
\pgfpathlineto{\pgfqpoint{5.658455in}{3.044242in}}%
\pgfpathlineto{\pgfqpoint{5.659103in}{3.212142in}}%
\pgfpathlineto{\pgfqpoint{5.659288in}{3.065103in}}%
\pgfpathlineto{\pgfqpoint{5.659642in}{3.264163in}}%
\pgfpathlineto{\pgfqpoint{5.660274in}{3.134689in}}%
\pgfpathlineto{\pgfqpoint{5.661184in}{3.270119in}}%
\pgfpathlineto{\pgfqpoint{5.660829in}{3.080694in}}%
\pgfpathlineto{\pgfqpoint{5.661415in}{3.194522in}}%
\pgfpathlineto{\pgfqpoint{5.662401in}{3.053579in}}%
\pgfpathlineto{\pgfqpoint{5.661954in}{3.295618in}}%
\pgfpathlineto{\pgfqpoint{5.662556in}{3.123121in}}%
\pgfpathlineto{\pgfqpoint{5.662941in}{3.295595in}}%
\pgfpathlineto{\pgfqpoint{5.663095in}{3.067486in}}%
\pgfpathlineto{\pgfqpoint{5.663758in}{3.192808in}}%
\pgfpathlineto{\pgfqpoint{5.664667in}{3.070785in}}%
\pgfpathlineto{\pgfqpoint{5.664297in}{3.248685in}}%
\pgfpathlineto{\pgfqpoint{5.664883in}{3.178228in}}%
\pgfpathlineto{\pgfqpoint{5.665885in}{3.287161in}}%
\pgfpathlineto{\pgfqpoint{5.665407in}{3.054688in}}%
\pgfpathlineto{\pgfqpoint{5.666024in}{3.213130in}}%
\pgfpathlineto{\pgfqpoint{5.667026in}{3.047397in}}%
\pgfpathlineto{\pgfqpoint{5.666579in}{3.271398in}}%
\pgfpathlineto{\pgfqpoint{5.667242in}{3.161242in}}%
\pgfpathlineto{\pgfqpoint{5.668105in}{3.261324in}}%
\pgfpathlineto{\pgfqpoint{5.667735in}{3.080169in}}%
\pgfpathlineto{\pgfqpoint{5.668382in}{3.216831in}}%
\pgfpathlineto{\pgfqpoint{5.669338in}{3.032777in}}%
\pgfpathlineto{\pgfqpoint{5.668906in}{3.273899in}}%
\pgfpathlineto{\pgfqpoint{5.669600in}{3.157816in}}%
\pgfpathlineto{\pgfqpoint{5.670479in}{3.268354in}}%
\pgfpathlineto{\pgfqpoint{5.670140in}{3.054358in}}%
\pgfpathlineto{\pgfqpoint{5.670741in}{3.213918in}}%
\pgfpathlineto{\pgfqpoint{5.671666in}{3.073221in}}%
\pgfpathlineto{\pgfqpoint{5.671465in}{3.262775in}}%
\pgfpathlineto{\pgfqpoint{5.671959in}{3.174750in}}%
\pgfpathlineto{\pgfqpoint{5.672791in}{3.292775in}}%
\pgfpathlineto{\pgfqpoint{5.672406in}{3.086547in}}%
\pgfpathlineto{\pgfqpoint{5.673068in}{3.195457in}}%
\pgfpathlineto{\pgfqpoint{5.673238in}{3.042673in}}%
\pgfpathlineto{\pgfqpoint{5.673778in}{3.286177in}}%
\pgfpathlineto{\pgfqpoint{5.674209in}{3.117779in}}%
\pgfpathlineto{\pgfqpoint{5.674332in}{3.255372in}}%
\pgfpathlineto{\pgfqpoint{5.674887in}{3.095735in}}%
\pgfpathlineto{\pgfqpoint{5.675427in}{3.191937in}}%
\pgfpathlineto{\pgfqpoint{5.675504in}{3.070442in}}%
\pgfpathlineto{\pgfqpoint{5.675966in}{3.256463in}}%
\pgfpathlineto{\pgfqpoint{5.676537in}{3.161235in}}%
\pgfpathlineto{\pgfqpoint{5.677662in}{3.270024in}}%
\pgfpathlineto{\pgfqpoint{5.677123in}{3.070402in}}%
\pgfpathlineto{\pgfqpoint{5.677708in}{3.229514in}}%
\pgfpathlineto{\pgfqpoint{5.677847in}{3.064527in}}%
\pgfpathlineto{\pgfqpoint{5.678263in}{3.244056in}}%
\pgfpathlineto{\pgfqpoint{5.678864in}{3.132035in}}%
\pgfpathlineto{\pgfqpoint{5.679219in}{3.248424in}}%
\pgfpathlineto{\pgfqpoint{5.679419in}{3.051721in}}%
\pgfpathlineto{\pgfqpoint{5.680005in}{3.221822in}}%
\pgfpathlineto{\pgfqpoint{5.680128in}{3.055503in}}%
\pgfpathlineto{\pgfqpoint{5.680606in}{3.252234in}}%
\pgfpathlineto{\pgfqpoint{5.681177in}{3.141436in}}%
\pgfpathlineto{\pgfqpoint{5.681300in}{3.281525in}}%
\pgfpathlineto{\pgfqpoint{5.681747in}{3.056352in}}%
\pgfpathlineto{\pgfqpoint{5.682333in}{3.221576in}}%
\pgfpathlineto{\pgfqpoint{5.683319in}{3.083600in}}%
\pgfpathlineto{\pgfqpoint{5.682841in}{3.264718in}}%
\pgfpathlineto{\pgfqpoint{5.683473in}{3.149288in}}%
\pgfpathlineto{\pgfqpoint{5.683581in}{3.219758in}}%
\pgfpathlineto{\pgfqpoint{5.684460in}{3.263352in}}%
\pgfpathlineto{\pgfqpoint{5.683998in}{3.061457in}}%
\pgfpathlineto{\pgfqpoint{5.684645in}{3.223632in}}%
\pgfpathlineto{\pgfqpoint{5.685632in}{3.062216in}}%
\pgfpathlineto{\pgfqpoint{5.685185in}{3.278154in}}%
\pgfpathlineto{\pgfqpoint{5.685801in}{3.115868in}}%
\pgfpathlineto{\pgfqpoint{5.686171in}{3.281075in}}%
\pgfpathlineto{\pgfqpoint{5.686325in}{3.102882in}}%
\pgfpathlineto{\pgfqpoint{5.687050in}{3.167337in}}%
\pgfpathlineto{\pgfqpoint{5.687913in}{3.045843in}}%
\pgfpathlineto{\pgfqpoint{5.687697in}{3.242329in}}%
\pgfpathlineto{\pgfqpoint{5.688144in}{3.159087in}}%
\pgfpathlineto{\pgfqpoint{5.689084in}{3.277468in}}%
\pgfpathlineto{\pgfqpoint{5.688715in}{3.070473in}}%
\pgfpathlineto{\pgfqpoint{5.689331in}{3.219904in}}%
\pgfpathlineto{\pgfqpoint{5.689516in}{3.068534in}}%
\pgfpathlineto{\pgfqpoint{5.690056in}{3.287014in}}%
\pgfpathlineto{\pgfqpoint{5.690549in}{3.189927in}}%
\pgfpathlineto{\pgfqpoint{5.690611in}{3.251861in}}%
\pgfpathlineto{\pgfqpoint{5.690965in}{3.114915in}}%
\pgfpathlineto{\pgfqpoint{5.691643in}{3.177436in}}%
\pgfpathlineto{\pgfqpoint{5.691797in}{3.042876in}}%
\pgfpathlineto{\pgfqpoint{5.692152in}{3.239972in}}%
\pgfpathlineto{\pgfqpoint{5.692198in}{3.236068in}}%
\pgfpathlineto{\pgfqpoint{5.692969in}{3.288280in}}%
\pgfpathlineto{\pgfqpoint{5.692599in}{3.060081in}}%
\pgfpathlineto{\pgfqpoint{5.693231in}{3.197250in}}%
\pgfpathlineto{\pgfqpoint{5.693431in}{3.087183in}}%
\pgfpathlineto{\pgfqpoint{5.693925in}{3.283646in}}%
\pgfpathlineto{\pgfqpoint{5.694356in}{3.153852in}}%
\pgfpathlineto{\pgfqpoint{5.695266in}{3.257246in}}%
\pgfpathlineto{\pgfqpoint{5.694834in}{3.102563in}}%
\pgfpathlineto{\pgfqpoint{5.695497in}{3.190829in}}%
\pgfpathlineto{\pgfqpoint{5.695667in}{3.049789in}}%
\pgfpathlineto{\pgfqpoint{5.696237in}{3.253396in}}%
\pgfpathlineto{\pgfqpoint{5.696669in}{3.105374in}}%
\pgfpathlineto{\pgfqpoint{5.696854in}{3.274444in}}%
\pgfpathlineto{\pgfqpoint{5.697254in}{3.094850in}}%
\pgfpathlineto{\pgfqpoint{5.697840in}{3.229636in}}%
\pgfpathlineto{\pgfqpoint{5.697994in}{3.095263in}}%
\pgfpathlineto{\pgfqpoint{5.698349in}{3.253045in}}%
\pgfpathlineto{\pgfqpoint{5.699073in}{3.202162in}}%
\pgfpathlineto{\pgfqpoint{5.699875in}{3.266426in}}%
\pgfpathlineto{\pgfqpoint{5.699551in}{3.047719in}}%
\pgfpathlineto{\pgfqpoint{5.700152in}{3.203083in}}%
\pgfpathlineto{\pgfqpoint{5.700337in}{3.064302in}}%
\pgfpathlineto{\pgfqpoint{5.700707in}{3.267204in}}%
\pgfpathlineto{\pgfqpoint{5.701324in}{3.135978in}}%
\pgfpathlineto{\pgfqpoint{5.702218in}{3.265675in}}%
\pgfpathlineto{\pgfqpoint{5.701879in}{3.102358in}}%
\pgfpathlineto{\pgfqpoint{5.702449in}{3.177993in}}%
\pgfpathlineto{\pgfqpoint{5.703374in}{3.042503in}}%
\pgfpathlineto{\pgfqpoint{5.703189in}{3.258959in}}%
\pgfpathlineto{\pgfqpoint{5.703590in}{3.148462in}}%
\pgfpathlineto{\pgfqpoint{5.703775in}{3.274200in}}%
\pgfpathlineto{\pgfqpoint{5.704222in}{3.075192in}}%
\pgfpathlineto{\pgfqpoint{5.704361in}{3.090938in}}%
\pgfpathlineto{\pgfqpoint{5.704376in}{3.086578in}}%
\pgfpathlineto{\pgfqpoint{5.704592in}{3.268122in}}%
\pgfpathlineto{\pgfqpoint{5.705208in}{3.158710in}}%
\pgfpathlineto{\pgfqpoint{5.706102in}{3.278800in}}%
\pgfpathlineto{\pgfqpoint{5.705748in}{3.111927in}}%
\pgfpathlineto{\pgfqpoint{5.706318in}{3.171492in}}%
\pgfpathlineto{\pgfqpoint{5.707305in}{3.028300in}}%
\pgfpathlineto{\pgfqpoint{5.707074in}{3.266571in}}%
\pgfpathlineto{\pgfqpoint{5.707459in}{3.122616in}}%
\pgfpathlineto{\pgfqpoint{5.707644in}{3.265712in}}%
\pgfpathlineto{\pgfqpoint{5.708261in}{3.083253in}}%
\pgfpathlineto{\pgfqpoint{5.708723in}{3.201141in}}%
\pgfpathlineto{\pgfqpoint{5.708862in}{3.091850in}}%
\pgfpathlineto{\pgfqpoint{5.709478in}{3.221035in}}%
\pgfpathlineto{\pgfqpoint{5.709833in}{3.167696in}}%
\pgfpathlineto{\pgfqpoint{5.710018in}{3.278829in}}%
\pgfpathlineto{\pgfqpoint{5.710388in}{3.057073in}}%
\pgfpathlineto{\pgfqpoint{5.711004in}{3.230268in}}%
\pgfpathlineto{\pgfqpoint{5.711189in}{3.029675in}}%
\pgfpathlineto{\pgfqpoint{5.711729in}{3.255165in}}%
\pgfpathlineto{\pgfqpoint{5.712176in}{3.118172in}}%
\pgfpathlineto{\pgfqpoint{5.713070in}{3.236023in}}%
\pgfpathlineto{\pgfqpoint{5.712731in}{3.101677in}}%
\pgfpathlineto{\pgfqpoint{5.713378in}{3.184752in}}%
\pgfpathlineto{\pgfqpoint{5.714195in}{3.061130in}}%
\pgfpathlineto{\pgfqpoint{5.713918in}{3.265457in}}%
\pgfpathlineto{\pgfqpoint{5.714488in}{3.170896in}}%
\pgfpathlineto{\pgfqpoint{5.715367in}{3.268639in}}%
\pgfpathlineto{\pgfqpoint{5.715058in}{3.064058in}}%
\pgfpathlineto{\pgfqpoint{5.715644in}{3.232863in}}%
\pgfpathlineto{\pgfqpoint{5.715798in}{3.072377in}}%
\pgfpathlineto{\pgfqpoint{5.716369in}{3.252942in}}%
\pgfpathlineto{\pgfqpoint{5.716816in}{3.168067in}}%
\pgfpathlineto{\pgfqpoint{5.717679in}{3.262582in}}%
\pgfpathlineto{\pgfqpoint{5.717386in}{3.072182in}}%
\pgfpathlineto{\pgfqpoint{5.717941in}{3.215285in}}%
\pgfpathlineto{\pgfqpoint{5.718126in}{3.035010in}}%
\pgfpathlineto{\pgfqpoint{5.718573in}{3.261785in}}%
\pgfpathlineto{\pgfqpoint{5.719128in}{3.119001in}}%
\pgfpathlineto{\pgfqpoint{5.719298in}{3.250412in}}%
\pgfpathlineto{\pgfqpoint{5.719683in}{3.078830in}}%
\pgfpathlineto{\pgfqpoint{5.720284in}{3.230224in}}%
\pgfpathlineto{\pgfqpoint{5.720454in}{3.061888in}}%
\pgfpathlineto{\pgfqpoint{5.720839in}{3.270733in}}%
\pgfpathlineto{\pgfqpoint{5.721456in}{3.138928in}}%
\pgfpathlineto{\pgfqpoint{5.721579in}{3.286764in}}%
\pgfpathlineto{\pgfqpoint{5.722026in}{3.052355in}}%
\pgfpathlineto{\pgfqpoint{5.722612in}{3.239382in}}%
\pgfpathlineto{\pgfqpoint{5.722735in}{3.055003in}}%
\pgfpathlineto{\pgfqpoint{5.723120in}{3.260424in}}%
\pgfpathlineto{\pgfqpoint{5.723768in}{3.164265in}}%
\pgfpathlineto{\pgfqpoint{5.724770in}{3.256731in}}%
\pgfpathlineto{\pgfqpoint{5.724307in}{3.056468in}}%
\pgfpathlineto{\pgfqpoint{5.724909in}{3.242191in}}%
\pgfpathlineto{\pgfqpoint{5.725864in}{3.054855in}}%
\pgfpathlineto{\pgfqpoint{5.725541in}{3.281969in}}%
\pgfpathlineto{\pgfqpoint{5.726157in}{3.186788in}}%
\pgfpathlineto{\pgfqpoint{5.726450in}{3.260748in}}%
\pgfpathlineto{\pgfqpoint{5.726681in}{3.037463in}}%
\pgfpathlineto{\pgfqpoint{5.727252in}{3.210937in}}%
\pgfpathlineto{\pgfqpoint{5.728238in}{3.076773in}}%
\pgfpathlineto{\pgfqpoint{5.727760in}{3.281950in}}%
\pgfpathlineto{\pgfqpoint{5.728392in}{3.127994in}}%
\pgfpathlineto{\pgfqpoint{5.728762in}{3.275375in}}%
\pgfpathlineto{\pgfqpoint{5.728978in}{3.058054in}}%
\pgfpathlineto{\pgfqpoint{5.729595in}{3.188088in}}%
\pgfpathlineto{\pgfqpoint{5.729795in}{3.074568in}}%
\pgfpathlineto{\pgfqpoint{5.730165in}{3.243417in}}%
\pgfpathlineto{\pgfqpoint{5.730705in}{3.160735in}}%
\pgfpathlineto{\pgfqpoint{5.731722in}{3.263731in}}%
\pgfpathlineto{\pgfqpoint{5.731321in}{3.061972in}}%
\pgfpathlineto{\pgfqpoint{5.731907in}{3.212047in}}%
\pgfpathlineto{\pgfqpoint{5.732076in}{3.069717in}}%
\pgfpathlineto{\pgfqpoint{5.732431in}{3.266897in}}%
\pgfpathlineto{\pgfqpoint{5.733078in}{3.145970in}}%
\pgfpathlineto{\pgfqpoint{5.733433in}{3.274988in}}%
\pgfpathlineto{\pgfqpoint{5.733602in}{3.070700in}}%
\pgfpathlineto{\pgfqpoint{5.734235in}{3.227949in}}%
\pgfpathlineto{\pgfqpoint{5.735159in}{3.067449in}}%
\pgfpathlineto{\pgfqpoint{5.734990in}{3.247730in}}%
\pgfpathlineto{\pgfqpoint{5.735375in}{3.143962in}}%
\pgfpathlineto{\pgfqpoint{5.735761in}{3.245415in}}%
\pgfpathlineto{\pgfqpoint{5.735930in}{3.088679in}}%
\pgfpathlineto{\pgfqpoint{5.736562in}{3.215960in}}%
\pgfpathlineto{\pgfqpoint{5.737487in}{3.068635in}}%
\pgfpathlineto{\pgfqpoint{5.737071in}{3.257200in}}%
\pgfpathlineto{\pgfqpoint{5.737734in}{3.160371in}}%
\pgfpathlineto{\pgfqpoint{5.738612in}{3.261374in}}%
\pgfpathlineto{\pgfqpoint{5.738258in}{3.090221in}}%
\pgfpathlineto{\pgfqpoint{5.738890in}{3.226076in}}%
\pgfpathlineto{\pgfqpoint{5.739090in}{3.072310in}}%
\pgfpathlineto{\pgfqpoint{5.739383in}{3.243660in}}%
\pgfpathlineto{\pgfqpoint{5.740077in}{3.138687in}}%
\pgfpathlineto{\pgfqpoint{5.740971in}{3.251356in}}%
\pgfpathlineto{\pgfqpoint{5.740616in}{3.079389in}}%
\pgfpathlineto{\pgfqpoint{5.741233in}{3.207288in}}%
\pgfpathlineto{\pgfqpoint{5.741418in}{3.072776in}}%
\pgfpathlineto{\pgfqpoint{5.741973in}{3.255413in}}%
\pgfpathlineto{\pgfqpoint{5.742389in}{3.133824in}}%
\pgfpathlineto{\pgfqpoint{5.742512in}{3.262474in}}%
\pgfpathlineto{\pgfqpoint{5.742975in}{3.088777in}}%
\pgfpathlineto{\pgfqpoint{5.743576in}{3.187970in}}%
\pgfpathlineto{\pgfqpoint{5.743684in}{3.086688in}}%
\pgfpathlineto{\pgfqpoint{5.744054in}{3.262142in}}%
\pgfpathlineto{\pgfqpoint{5.744701in}{3.143249in}}%
\pgfpathlineto{\pgfqpoint{5.745857in}{3.271718in}}%
\pgfpathlineto{\pgfqpoint{5.745302in}{3.091403in}}%
\pgfpathlineto{\pgfqpoint{5.745888in}{3.224961in}}%
\pgfpathlineto{\pgfqpoint{5.746181in}{3.101915in}}%
\pgfpathlineto{\pgfqpoint{5.746397in}{3.258610in}}%
\pgfpathlineto{\pgfqpoint{5.747029in}{3.145994in}}%
\pgfpathlineto{\pgfqpoint{5.747938in}{3.255924in}}%
\pgfpathlineto{\pgfqpoint{5.747568in}{3.077856in}}%
\pgfpathlineto{\pgfqpoint{5.748200in}{3.241862in}}%
\pgfpathlineto{\pgfqpoint{5.748524in}{3.083474in}}%
\pgfpathlineto{\pgfqpoint{5.748740in}{3.257005in}}%
\pgfpathlineto{\pgfqpoint{5.749372in}{3.138913in}}%
\pgfpathlineto{\pgfqpoint{5.749742in}{3.278292in}}%
\pgfpathlineto{\pgfqpoint{5.750066in}{3.087287in}}%
\pgfpathlineto{\pgfqpoint{5.750543in}{3.227039in}}%
\pgfpathlineto{\pgfqpoint{5.751453in}{3.067072in}}%
\pgfpathlineto{\pgfqpoint{5.751268in}{3.239105in}}%
\pgfpathlineto{\pgfqpoint{5.751669in}{3.170834in}}%
\pgfpathlineto{\pgfqpoint{5.752624in}{3.266238in}}%
\pgfpathlineto{\pgfqpoint{5.752424in}{3.064703in}}%
\pgfpathlineto{\pgfqpoint{5.752871in}{3.247453in}}%
\pgfpathlineto{\pgfqpoint{5.753071in}{3.080423in}}%
\pgfpathlineto{\pgfqpoint{5.753626in}{3.295203in}}%
\pgfpathlineto{\pgfqpoint{5.754027in}{3.133794in}}%
\pgfpathlineto{\pgfqpoint{5.754150in}{3.271365in}}%
\pgfpathlineto{\pgfqpoint{5.754644in}{3.115925in}}%
\pgfpathlineto{\pgfqpoint{5.755214in}{3.187993in}}%
\pgfpathlineto{\pgfqpoint{5.756309in}{3.063984in}}%
\pgfpathlineto{\pgfqpoint{5.755800in}{3.252443in}}%
\pgfpathlineto{\pgfqpoint{5.756355in}{3.119399in}}%
\pgfpathlineto{\pgfqpoint{5.756740in}{3.282697in}}%
\pgfpathlineto{\pgfqpoint{5.756956in}{3.083296in}}%
\pgfpathlineto{\pgfqpoint{5.757542in}{3.238381in}}%
\pgfpathlineto{\pgfqpoint{5.757850in}{3.089277in}}%
\pgfpathlineto{\pgfqpoint{5.758035in}{3.257691in}}%
\pgfpathlineto{\pgfqpoint{5.758667in}{3.142268in}}%
\pgfpathlineto{\pgfqpoint{5.759700in}{3.262379in}}%
\pgfpathlineto{\pgfqpoint{5.759284in}{3.067680in}}%
\pgfpathlineto{\pgfqpoint{5.759808in}{3.209291in}}%
\pgfpathlineto{\pgfqpoint{5.760625in}{3.272738in}}%
\pgfpathlineto{\pgfqpoint{5.760178in}{3.060678in}}%
\pgfpathlineto{\pgfqpoint{5.760779in}{3.106846in}}%
\pgfpathlineto{\pgfqpoint{5.760825in}{3.073331in}}%
\pgfpathlineto{\pgfqpoint{5.761395in}{3.278635in}}%
\pgfpathlineto{\pgfqpoint{5.761827in}{3.168201in}}%
\pgfpathlineto{\pgfqpoint{5.761920in}{3.255455in}}%
\pgfpathlineto{\pgfqpoint{5.762397in}{3.109160in}}%
\pgfpathlineto{\pgfqpoint{5.762968in}{3.200299in}}%
\pgfpathlineto{\pgfqpoint{5.764062in}{3.050885in}}%
\pgfpathlineto{\pgfqpoint{5.763569in}{3.256153in}}%
\pgfpathlineto{\pgfqpoint{5.764139in}{3.131249in}}%
\pgfpathlineto{\pgfqpoint{5.765280in}{3.275669in}}%
\pgfpathlineto{\pgfqpoint{5.764725in}{3.071586in}}%
\pgfpathlineto{\pgfqpoint{5.765311in}{3.224891in}}%
\pgfpathlineto{\pgfqpoint{5.766390in}{3.111727in}}%
\pgfpathlineto{\pgfqpoint{5.765819in}{3.257163in}}%
\pgfpathlineto{\pgfqpoint{5.766436in}{3.162383in}}%
\pgfpathlineto{\pgfqpoint{5.767438in}{3.280574in}}%
\pgfpathlineto{\pgfqpoint{5.767053in}{3.050409in}}%
\pgfpathlineto{\pgfqpoint{5.767577in}{3.236442in}}%
\pgfpathlineto{\pgfqpoint{5.768394in}{3.265766in}}%
\pgfpathlineto{\pgfqpoint{5.767962in}{3.043869in}}%
\pgfpathlineto{\pgfqpoint{5.768486in}{3.145935in}}%
\pgfpathlineto{\pgfqpoint{5.768610in}{3.080290in}}%
\pgfpathlineto{\pgfqpoint{5.769165in}{3.268472in}}%
\pgfpathlineto{\pgfqpoint{5.769550in}{3.150779in}}%
\pgfpathlineto{\pgfqpoint{5.769704in}{3.268616in}}%
\pgfpathlineto{\pgfqpoint{5.770274in}{3.105520in}}%
\pgfpathlineto{\pgfqpoint{5.770721in}{3.216541in}}%
\pgfpathlineto{\pgfqpoint{5.770922in}{3.034421in}}%
\pgfpathlineto{\pgfqpoint{5.771338in}{3.288929in}}%
\pgfpathlineto{\pgfqpoint{5.771878in}{3.082134in}}%
\pgfpathlineto{\pgfqpoint{5.772109in}{3.279301in}}%
\pgfpathlineto{\pgfqpoint{5.772494in}{3.080618in}}%
\pgfpathlineto{\pgfqpoint{5.773064in}{3.243536in}}%
\pgfpathlineto{\pgfqpoint{5.773881in}{3.099621in}}%
\pgfpathlineto{\pgfqpoint{5.773619in}{3.258307in}}%
\pgfpathlineto{\pgfqpoint{5.774205in}{3.152357in}}%
\pgfpathlineto{\pgfqpoint{5.775223in}{3.267989in}}%
\pgfpathlineto{\pgfqpoint{5.774806in}{3.029018in}}%
\pgfpathlineto{\pgfqpoint{5.775361in}{3.250986in}}%
\pgfpathlineto{\pgfqpoint{5.775747in}{3.082735in}}%
\pgfpathlineto{\pgfqpoint{5.775993in}{3.280701in}}%
\pgfpathlineto{\pgfqpoint{5.776641in}{3.186532in}}%
\pgfpathlineto{\pgfqpoint{5.777535in}{3.275948in}}%
\pgfpathlineto{\pgfqpoint{5.777149in}{3.088187in}}%
\pgfpathlineto{\pgfqpoint{5.777720in}{3.146515in}}%
\pgfpathlineto{\pgfqpoint{5.778691in}{3.037055in}}%
\pgfpathlineto{\pgfqpoint{5.778321in}{3.282803in}}%
\pgfpathlineto{\pgfqpoint{5.778830in}{3.116372in}}%
\pgfpathlineto{\pgfqpoint{5.779862in}{3.260028in}}%
\pgfpathlineto{\pgfqpoint{5.779462in}{3.102093in}}%
\pgfpathlineto{\pgfqpoint{5.780032in}{3.224619in}}%
\pgfpathlineto{\pgfqpoint{5.781019in}{3.088959in}}%
\pgfpathlineto{\pgfqpoint{5.780803in}{3.227441in}}%
\pgfpathlineto{\pgfqpoint{5.781173in}{3.155612in}}%
\pgfpathlineto{\pgfqpoint{5.782190in}{3.284570in}}%
\pgfpathlineto{\pgfqpoint{5.781728in}{3.080056in}}%
\pgfpathlineto{\pgfqpoint{5.782298in}{3.185356in}}%
\pgfpathlineto{\pgfqpoint{5.782976in}{3.253139in}}%
\pgfpathlineto{\pgfqpoint{5.782575in}{3.055095in}}%
\pgfpathlineto{\pgfqpoint{5.783238in}{3.166822in}}%
\pgfpathlineto{\pgfqpoint{5.783346in}{3.075800in}}%
\pgfpathlineto{\pgfqpoint{5.783747in}{3.268978in}}%
\pgfpathlineto{\pgfqpoint{5.784333in}{3.180555in}}%
\pgfpathlineto{\pgfqpoint{5.785196in}{3.257099in}}%
\pgfpathlineto{\pgfqpoint{5.784903in}{3.076972in}}%
\pgfpathlineto{\pgfqpoint{5.785473in}{3.205552in}}%
\pgfpathlineto{\pgfqpoint{5.785658in}{3.064316in}}%
\pgfpathlineto{\pgfqpoint{5.786075in}{3.284024in}}%
\pgfpathlineto{\pgfqpoint{5.786614in}{3.107447in}}%
\pgfpathlineto{\pgfqpoint{5.786753in}{3.259516in}}%
\pgfpathlineto{\pgfqpoint{5.787231in}{3.084422in}}%
\pgfpathlineto{\pgfqpoint{5.787801in}{3.218790in}}%
\pgfpathlineto{\pgfqpoint{5.787986in}{3.098197in}}%
\pgfpathlineto{\pgfqpoint{5.788402in}{3.256312in}}%
\pgfpathlineto{\pgfqpoint{5.788926in}{3.138654in}}%
\pgfpathlineto{\pgfqpoint{5.789080in}{3.276929in}}%
\pgfpathlineto{\pgfqpoint{5.789543in}{3.042378in}}%
\pgfpathlineto{\pgfqpoint{5.790113in}{3.239646in}}%
\pgfpathlineto{\pgfqpoint{5.790314in}{3.083475in}}%
\pgfpathlineto{\pgfqpoint{5.790730in}{3.267864in}}%
\pgfpathlineto{\pgfqpoint{5.791254in}{3.160987in}}%
\pgfpathlineto{\pgfqpoint{5.791871in}{3.087966in}}%
\pgfpathlineto{\pgfqpoint{5.791655in}{3.274790in}}%
\pgfpathlineto{\pgfqpoint{5.792256in}{3.213197in}}%
\pgfpathlineto{\pgfqpoint{5.792950in}{3.251391in}}%
\pgfpathlineto{\pgfqpoint{5.792503in}{3.090467in}}%
\pgfpathlineto{\pgfqpoint{5.793304in}{3.141794in}}%
\pgfpathlineto{\pgfqpoint{5.793320in}{3.142724in}}%
\pgfpathlineto{\pgfqpoint{5.793381in}{3.080421in}}%
\pgfpathlineto{\pgfqpoint{5.793427in}{3.053215in}}%
\pgfpathlineto{\pgfqpoint{5.793982in}{3.277993in}}%
\pgfpathlineto{\pgfqpoint{5.794414in}{3.126942in}}%
\pgfpathlineto{\pgfqpoint{5.794584in}{3.264396in}}%
\pgfpathlineto{\pgfqpoint{5.794614in}{3.286452in}}%
\pgfpathlineto{\pgfqpoint{5.794984in}{3.105556in}}%
\pgfpathlineto{\pgfqpoint{5.795601in}{3.157850in}}%
\pgfpathlineto{\pgfqpoint{5.796387in}{3.085761in}}%
\pgfpathlineto{\pgfqpoint{5.796171in}{3.238806in}}%
\pgfpathlineto{\pgfqpoint{5.796695in}{3.161953in}}%
\pgfpathlineto{\pgfqpoint{5.797004in}{3.150200in}}%
\pgfpathlineto{\pgfqpoint{5.796927in}{3.263502in}}%
\pgfpathlineto{\pgfqpoint{5.797081in}{3.258923in}}%
\pgfpathlineto{\pgfqpoint{5.797096in}{3.261827in}}%
\pgfpathlineto{\pgfqpoint{5.797250in}{3.093578in}}%
\pgfpathlineto{\pgfqpoint{5.797297in}{3.044308in}}%
\pgfpathlineto{\pgfqpoint{5.797867in}{3.284966in}}%
\pgfpathlineto{\pgfqpoint{5.798314in}{3.140061in}}%
\pgfpathlineto{\pgfqpoint{5.798499in}{3.296671in}}%
\pgfpathlineto{\pgfqpoint{5.798961in}{3.108095in}}%
\pgfpathlineto{\pgfqpoint{5.799455in}{3.213115in}}%
\pgfpathlineto{\pgfqpoint{5.800256in}{3.085863in}}%
\pgfpathlineto{\pgfqpoint{5.800025in}{3.256817in}}%
\pgfpathlineto{\pgfqpoint{5.800595in}{3.155362in}}%
\pgfpathlineto{\pgfqpoint{5.800811in}{3.281917in}}%
\pgfpathlineto{\pgfqpoint{5.801181in}{3.040245in}}%
\pgfpathlineto{\pgfqpoint{5.801782in}{3.213961in}}%
\pgfpathlineto{\pgfqpoint{5.801921in}{3.090753in}}%
\pgfpathlineto{\pgfqpoint{5.802383in}{3.287084in}}%
\pgfpathlineto{\pgfqpoint{5.802908in}{3.153793in}}%
\pgfpathlineto{\pgfqpoint{5.803925in}{3.266382in}}%
\pgfpathlineto{\pgfqpoint{5.803493in}{3.085379in}}%
\pgfpathlineto{\pgfqpoint{5.804079in}{3.220427in}}%
\pgfpathlineto{\pgfqpoint{5.805066in}{3.060034in}}%
\pgfpathlineto{\pgfqpoint{5.804711in}{3.301284in}}%
\pgfpathlineto{\pgfqpoint{5.805204in}{3.134372in}}%
\pgfpathlineto{\pgfqpoint{5.805343in}{3.286330in}}%
\pgfpathlineto{\pgfqpoint{5.805821in}{3.062159in}}%
\pgfpathlineto{\pgfqpoint{5.806376in}{3.242796in}}%
\pgfpathlineto{\pgfqpoint{5.807177in}{3.259736in}}%
\pgfpathlineto{\pgfqpoint{5.806715in}{3.095052in}}%
\pgfpathlineto{\pgfqpoint{5.807224in}{3.139245in}}%
\pgfpathlineto{\pgfqpoint{5.808118in}{3.056561in}}%
\pgfpathlineto{\pgfqpoint{5.807655in}{3.279575in}}%
\pgfpathlineto{\pgfqpoint{5.808287in}{3.196222in}}%
\pgfpathlineto{\pgfqpoint{5.808580in}{3.301849in}}%
\pgfpathlineto{\pgfqpoint{5.808781in}{3.099521in}}%
\pgfpathlineto{\pgfqpoint{5.808873in}{3.101937in}}%
\pgfpathlineto{\pgfqpoint{5.809690in}{3.058516in}}%
\pgfpathlineto{\pgfqpoint{5.809243in}{3.279911in}}%
\pgfpathlineto{\pgfqpoint{5.809937in}{3.139391in}}%
\pgfpathlineto{\pgfqpoint{5.810260in}{3.304047in}}%
\pgfpathlineto{\pgfqpoint{5.810461in}{3.084119in}}%
\pgfpathlineto{\pgfqpoint{5.811077in}{3.194832in}}%
\pgfpathlineto{\pgfqpoint{5.812018in}{3.024852in}}%
\pgfpathlineto{\pgfqpoint{5.811540in}{3.291377in}}%
\pgfpathlineto{\pgfqpoint{5.812172in}{3.190836in}}%
\pgfpathlineto{\pgfqpoint{5.812465in}{3.293937in}}%
\pgfpathlineto{\pgfqpoint{5.812788in}{3.035625in}}%
\pgfpathlineto{\pgfqpoint{5.813266in}{3.226511in}}%
\pgfpathlineto{\pgfqpoint{5.813575in}{3.050580in}}%
\pgfpathlineto{\pgfqpoint{5.814145in}{3.335416in}}%
\pgfpathlineto{\pgfqpoint{5.814392in}{3.132130in}}%
\pgfpathlineto{\pgfqpoint{5.815455in}{3.288663in}}%
\pgfpathlineto{\pgfqpoint{5.815224in}{3.094412in}}%
\pgfpathlineto{\pgfqpoint{5.815579in}{3.238350in}}%
\pgfpathlineto{\pgfqpoint{5.815902in}{3.022149in}}%
\pgfpathlineto{\pgfqpoint{5.816473in}{3.322853in}}%
\pgfpathlineto{\pgfqpoint{5.816750in}{3.099638in}}%
\pgfpathlineto{\pgfqpoint{5.816796in}{3.064401in}}%
\pgfpathlineto{\pgfqpoint{5.817105in}{3.292926in}}%
\pgfpathlineto{\pgfqpoint{5.817721in}{3.201841in}}%
\pgfpathlineto{\pgfqpoint{5.818030in}{3.322508in}}%
\pgfpathlineto{\pgfqpoint{5.818214in}{3.044903in}}%
\pgfpathlineto{\pgfqpoint{5.818723in}{3.149472in}}%
\pgfpathlineto{\pgfqpoint{5.819756in}{3.013006in}}%
\pgfpathlineto{\pgfqpoint{5.819586in}{3.311588in}}%
\pgfpathlineto{\pgfqpoint{5.819864in}{3.063402in}}%
\pgfpathlineto{\pgfqpoint{5.820357in}{3.362192in}}%
\pgfpathlineto{\pgfqpoint{5.821035in}{3.223295in}}%
\pgfpathlineto{\pgfqpoint{5.822084in}{3.054047in}}%
\pgfpathlineto{\pgfqpoint{5.821899in}{3.307507in}}%
\pgfpathlineto{\pgfqpoint{5.822176in}{3.119089in}}%
\pgfpathlineto{\pgfqpoint{5.823317in}{3.301244in}}%
\pgfpathlineto{\pgfqpoint{5.822746in}{3.060704in}}%
\pgfpathlineto{\pgfqpoint{5.823378in}{3.158648in}}%
\pgfpathlineto{\pgfqpoint{5.823641in}{3.015540in}}%
\pgfpathlineto{\pgfqpoint{5.823456in}{3.315924in}}%
\pgfpathlineto{\pgfqpoint{5.824195in}{3.237282in}}%
\pgfpathlineto{\pgfqpoint{5.824242in}{3.338654in}}%
\pgfpathlineto{\pgfqpoint{5.824411in}{3.068833in}}%
\pgfpathlineto{\pgfqpoint{5.825274in}{3.131036in}}%
\pgfpathlineto{\pgfqpoint{5.825968in}{3.057456in}}%
\pgfpathlineto{\pgfqpoint{5.825783in}{3.287991in}}%
\pgfpathlineto{\pgfqpoint{5.826292in}{3.213944in}}%
\pgfpathlineto{\pgfqpoint{5.827201in}{3.336147in}}%
\pgfpathlineto{\pgfqpoint{5.826631in}{3.058064in}}%
\pgfpathlineto{\pgfqpoint{5.827371in}{3.218415in}}%
\pgfpathlineto{\pgfqpoint{5.827571in}{3.037046in}}%
\pgfpathlineto{\pgfqpoint{5.828111in}{3.315597in}}%
\pgfpathlineto{\pgfqpoint{5.828512in}{3.117889in}}%
\pgfpathlineto{\pgfqpoint{5.829236in}{3.092691in}}%
\pgfpathlineto{\pgfqpoint{5.828774in}{3.295257in}}%
\pgfpathlineto{\pgfqpoint{5.829390in}{3.199645in}}%
\pgfpathlineto{\pgfqpoint{5.830300in}{3.293063in}}%
\pgfpathlineto{\pgfqpoint{5.829868in}{3.074044in}}%
\pgfpathlineto{\pgfqpoint{5.830469in}{3.192269in}}%
\pgfpathlineto{\pgfqpoint{5.831533in}{3.043893in}}%
\pgfpathlineto{\pgfqpoint{5.831086in}{3.334583in}}%
\pgfpathlineto{\pgfqpoint{5.831595in}{3.111758in}}%
\pgfpathlineto{\pgfqpoint{5.831733in}{3.306524in}}%
\pgfpathlineto{\pgfqpoint{5.832165in}{3.063948in}}%
\pgfpathlineto{\pgfqpoint{5.832797in}{3.228929in}}%
\pgfpathlineto{\pgfqpoint{5.833768in}{3.069935in}}%
\pgfpathlineto{\pgfqpoint{5.833938in}{3.146278in}}%
\pgfpathlineto{\pgfqpoint{5.834955in}{3.295136in}}%
\pgfpathlineto{\pgfqpoint{5.834477in}{3.034029in}}%
\pgfpathlineto{\pgfqpoint{5.835078in}{3.244582in}}%
\pgfpathlineto{\pgfqpoint{5.835094in}{3.245467in}}%
\pgfpathlineto{\pgfqpoint{5.835125in}{3.200450in}}%
\pgfpathlineto{\pgfqpoint{5.835356in}{3.030523in}}%
\pgfpathlineto{\pgfqpoint{5.835618in}{3.312560in}}%
\pgfpathlineto{\pgfqpoint{5.836296in}{3.113193in}}%
\pgfpathlineto{\pgfqpoint{5.836851in}{3.063813in}}%
\pgfpathlineto{\pgfqpoint{5.836635in}{3.326355in}}%
\pgfpathlineto{\pgfqpoint{5.837375in}{3.105258in}}%
\pgfpathlineto{\pgfqpoint{5.838069in}{3.258567in}}%
\pgfpathlineto{\pgfqpoint{5.838285in}{3.071738in}}%
\pgfpathlineto{\pgfqpoint{5.838500in}{3.161558in}}%
\pgfpathlineto{\pgfqpoint{5.838593in}{3.278134in}}%
\pgfpathlineto{\pgfqpoint{5.839132in}{3.060909in}}%
\pgfpathlineto{\pgfqpoint{5.839194in}{2.986948in}}%
\pgfpathlineto{\pgfqpoint{5.839518in}{3.320963in}}%
\pgfpathlineto{\pgfqpoint{5.840165in}{3.131315in}}%
\pgfpathlineto{\pgfqpoint{5.840551in}{3.344646in}}%
\pgfpathlineto{\pgfqpoint{5.840751in}{3.020836in}}%
\pgfpathlineto{\pgfqpoint{5.841306in}{3.201148in}}%
\pgfpathlineto{\pgfqpoint{5.841475in}{3.254315in}}%
\pgfpathlineto{\pgfqpoint{5.841707in}{3.079451in}}%
\pgfpathlineto{\pgfqpoint{5.842400in}{3.211465in}}%
\pgfpathlineto{\pgfqpoint{5.843109in}{3.086249in}}%
\pgfpathlineto{\pgfqpoint{5.842924in}{3.232566in}}%
\pgfpathlineto{\pgfqpoint{5.843526in}{3.188437in}}%
\pgfpathlineto{\pgfqpoint{5.844081in}{3.233709in}}%
\pgfpathlineto{\pgfqpoint{5.844281in}{3.120449in}}%
\pgfpathlineto{\pgfqpoint{5.844605in}{3.192469in}}%
\pgfpathlineto{\pgfqpoint{5.845175in}{3.103457in}}%
\pgfpathlineto{\pgfqpoint{5.844743in}{3.233977in}}%
\pgfpathlineto{\pgfqpoint{5.845730in}{3.161177in}}%
\pgfpathlineto{\pgfqpoint{5.846424in}{3.250187in}}%
\pgfpathlineto{\pgfqpoint{5.845915in}{3.116235in}}%
\pgfpathlineto{\pgfqpoint{5.846747in}{3.145799in}}%
\pgfpathlineto{\pgfqpoint{5.846871in}{3.114178in}}%
\pgfpathlineto{\pgfqpoint{5.847318in}{3.206443in}}%
\pgfpathlineto{\pgfqpoint{5.847379in}{3.231921in}}%
\pgfpathlineto{\pgfqpoint{5.847580in}{3.107972in}}%
\pgfpathlineto{\pgfqpoint{5.848381in}{3.176331in}}%
\pgfpathlineto{\pgfqpoint{5.848566in}{3.162633in}}%
\pgfpathlineto{\pgfqpoint{5.848628in}{3.201991in}}%
\pgfpathlineto{\pgfqpoint{5.849291in}{3.245626in}}%
\pgfpathlineto{\pgfqpoint{5.848859in}{3.097125in}}%
\pgfpathlineto{\pgfqpoint{5.849630in}{3.119251in}}%
\pgfpathlineto{\pgfqpoint{5.849645in}{3.118860in}}%
\pgfpathlineto{\pgfqpoint{5.849676in}{3.143228in}}%
\pgfpathlineto{\pgfqpoint{5.850185in}{3.254382in}}%
\pgfpathlineto{\pgfqpoint{5.849861in}{3.090045in}}%
\pgfpathlineto{\pgfqpoint{5.850724in}{3.095380in}}%
\pgfpathlineto{\pgfqpoint{5.850771in}{3.041844in}}%
\pgfpathlineto{\pgfqpoint{5.851726in}{3.278764in}}%
\pgfpathlineto{\pgfqpoint{5.851803in}{3.155795in}}%
\pgfpathlineto{\pgfqpoint{5.852852in}{3.262115in}}%
\pgfpathlineto{\pgfqpoint{5.852158in}{3.061288in}}%
\pgfpathlineto{\pgfqpoint{5.852913in}{3.168735in}}%
\pgfpathlineto{\pgfqpoint{5.853067in}{3.095400in}}%
\pgfpathlineto{\pgfqpoint{5.853576in}{3.217373in}}%
\pgfpathlineto{\pgfqpoint{5.854378in}{3.278051in}}%
\pgfpathlineto{\pgfqpoint{5.854085in}{3.093272in}}%
\pgfpathlineto{\pgfqpoint{5.854624in}{3.125926in}}%
\pgfpathlineto{\pgfqpoint{5.855256in}{3.293498in}}%
\pgfpathlineto{\pgfqpoint{5.854809in}{3.053416in}}%
\pgfpathlineto{\pgfqpoint{5.855719in}{3.121316in}}%
\pgfpathlineto{\pgfqpoint{5.856027in}{3.000352in}}%
\pgfpathlineto{\pgfqpoint{5.856243in}{3.288278in}}%
\pgfpathlineto{\pgfqpoint{5.856798in}{3.138443in}}%
\pgfpathlineto{\pgfqpoint{5.857877in}{3.366006in}}%
\pgfpathlineto{\pgfqpoint{5.857584in}{2.991929in}}%
\pgfpathlineto{\pgfqpoint{5.857954in}{3.288565in}}%
\pgfpathlineto{\pgfqpoint{5.858463in}{2.929276in}}%
\pgfpathlineto{\pgfqpoint{5.858046in}{3.333756in}}%
\pgfpathlineto{\pgfqpoint{5.859403in}{3.137576in}}%
\pgfpathlineto{\pgfqpoint{5.859511in}{3.221151in}}%
\pgfpathlineto{\pgfqpoint{5.859973in}{3.056341in}}%
\pgfpathlineto{\pgfqpoint{5.860174in}{3.073798in}}%
\pgfpathlineto{\pgfqpoint{5.860867in}{2.986788in}}%
\pgfpathlineto{\pgfqpoint{5.860436in}{3.311403in}}%
\pgfpathlineto{\pgfqpoint{5.861160in}{3.214037in}}%
\pgfpathlineto{\pgfqpoint{5.862008in}{2.992289in}}%
\pgfpathlineto{\pgfqpoint{5.861561in}{3.291245in}}%
\pgfpathlineto{\pgfqpoint{5.862270in}{3.204830in}}%
\pgfpathlineto{\pgfqpoint{5.863149in}{3.313153in}}%
\pgfpathlineto{\pgfqpoint{5.862717in}{3.082750in}}%
\pgfpathlineto{\pgfqpoint{5.863318in}{3.145767in}}%
\pgfpathlineto{\pgfqpoint{5.863395in}{3.007390in}}%
\pgfpathlineto{\pgfqpoint{5.864151in}{3.452871in}}%
\pgfpathlineto{\pgfqpoint{5.864459in}{3.058224in}}%
\pgfpathlineto{\pgfqpoint{5.864598in}{2.821720in}}%
\pgfpathlineto{\pgfqpoint{5.864814in}{3.311129in}}%
\pgfpathlineto{\pgfqpoint{5.864891in}{3.289788in}}%
\pgfpathlineto{\pgfqpoint{5.865060in}{3.323845in}}%
\pgfpathlineto{\pgfqpoint{5.865461in}{3.025501in}}%
\pgfpathlineto{\pgfqpoint{5.865769in}{3.146797in}}%
\pgfpathlineto{\pgfqpoint{5.866340in}{3.025540in}}%
\pgfpathlineto{\pgfqpoint{5.866571in}{3.502216in}}%
\pgfpathlineto{\pgfqpoint{5.866817in}{3.293932in}}%
\pgfpathlineto{\pgfqpoint{5.867018in}{2.851984in}}%
\pgfpathlineto{\pgfqpoint{5.867758in}{3.333216in}}%
\pgfpathlineto{\pgfqpoint{5.868066in}{3.135896in}}%
\pgfpathlineto{\pgfqpoint{5.868220in}{2.981246in}}%
\pgfpathlineto{\pgfqpoint{5.868976in}{3.259743in}}%
\pgfpathlineto{\pgfqpoint{5.869253in}{3.306398in}}%
\pgfpathlineto{\pgfqpoint{5.869484in}{3.024914in}}%
\pgfpathlineto{\pgfqpoint{5.869947in}{3.156728in}}%
\pgfpathlineto{\pgfqpoint{5.870656in}{2.874455in}}%
\pgfpathlineto{\pgfqpoint{5.870193in}{3.432770in}}%
\pgfpathlineto{\pgfqpoint{5.870979in}{3.253205in}}%
\pgfpathlineto{\pgfqpoint{5.871442in}{3.333268in}}%
\pgfpathlineto{\pgfqpoint{5.871874in}{3.014124in}}%
\pgfpathlineto{\pgfqpoint{5.872089in}{3.270060in}}%
\pgfpathlineto{\pgfqpoint{5.872166in}{3.197803in}}%
\pgfpathlineto{\pgfqpoint{5.873138in}{2.820605in}}%
\pgfpathlineto{\pgfqpoint{5.872691in}{3.504967in}}%
\pgfpathlineto{\pgfqpoint{5.873353in}{3.107334in}}%
\pgfpathlineto{\pgfqpoint{5.873400in}{3.088183in}}%
\pgfpathlineto{\pgfqpoint{5.873461in}{3.159696in}}%
\pgfpathlineto{\pgfqpoint{5.873507in}{3.158057in}}%
\pgfpathlineto{\pgfqpoint{5.873847in}{3.441970in}}%
\pgfpathlineto{\pgfqpoint{5.874355in}{2.911928in}}%
\pgfpathlineto{\pgfqpoint{5.874648in}{3.220247in}}%
\pgfpathlineto{\pgfqpoint{5.874664in}{3.223360in}}%
\pgfpathlineto{\pgfqpoint{5.874849in}{3.042288in}}%
\pgfpathlineto{\pgfqpoint{5.875573in}{2.924761in}}%
\pgfpathlineto{\pgfqpoint{5.875342in}{3.406801in}}%
\pgfpathlineto{\pgfqpoint{5.875820in}{3.166955in}}%
\pgfpathlineto{\pgfqpoint{5.876282in}{3.427454in}}%
\pgfpathlineto{\pgfqpoint{5.876698in}{3.036613in}}%
\pgfpathlineto{\pgfqpoint{5.876791in}{2.880686in}}%
\pgfpathlineto{\pgfqpoint{5.877577in}{3.330105in}}%
\pgfpathlineto{\pgfqpoint{5.877700in}{3.240056in}}%
\pgfpathlineto{\pgfqpoint{5.877978in}{2.996805in}}%
\pgfpathlineto{\pgfqpoint{5.878718in}{3.334684in}}%
\pgfpathlineto{\pgfqpoint{5.878887in}{3.436024in}}%
\pgfpathlineto{\pgfqpoint{5.879273in}{2.900836in}}%
\pgfpathlineto{\pgfqpoint{5.879797in}{3.272703in}}%
\pgfpathlineto{\pgfqpoint{5.879951in}{3.474247in}}%
\pgfpathlineto{\pgfqpoint{5.880383in}{2.862405in}}%
\pgfpathlineto{\pgfqpoint{5.880830in}{3.221919in}}%
\pgfpathlineto{\pgfqpoint{5.881631in}{2.970274in}}%
\pgfpathlineto{\pgfqpoint{5.881323in}{3.356122in}}%
\pgfpathlineto{\pgfqpoint{5.882186in}{3.106183in}}%
\pgfpathlineto{\pgfqpoint{5.882386in}{3.484888in}}%
\pgfpathlineto{\pgfqpoint{5.882880in}{2.934388in}}%
\pgfpathlineto{\pgfqpoint{5.883311in}{3.124070in}}%
\pgfpathlineto{\pgfqpoint{5.884020in}{2.926533in}}%
\pgfpathlineto{\pgfqpoint{5.883620in}{3.353256in}}%
\pgfpathlineto{\pgfqpoint{5.884329in}{3.261633in}}%
\pgfpathlineto{\pgfqpoint{5.885315in}{2.960127in}}%
\pgfpathlineto{\pgfqpoint{5.884992in}{3.369491in}}%
\pgfpathlineto{\pgfqpoint{5.885654in}{3.068657in}}%
\pgfpathlineto{\pgfqpoint{5.886055in}{3.462307in}}%
\pgfpathlineto{\pgfqpoint{5.886425in}{2.992178in}}%
\pgfpathlineto{\pgfqpoint{5.886502in}{2.821117in}}%
\pgfpathlineto{\pgfqpoint{5.886718in}{3.329645in}}%
\pgfpathlineto{\pgfqpoint{5.887427in}{3.291836in}}%
\pgfpathlineto{\pgfqpoint{5.888506in}{3.446493in}}%
\pgfpathlineto{\pgfqpoint{5.887859in}{3.009803in}}%
\pgfpathlineto{\pgfqpoint{5.888568in}{3.337944in}}%
\pgfpathlineto{\pgfqpoint{5.888922in}{2.890020in}}%
\pgfpathlineto{\pgfqpoint{5.888645in}{3.343571in}}%
\pgfpathlineto{\pgfqpoint{5.889693in}{3.290189in}}%
\pgfpathlineto{\pgfqpoint{5.889724in}{3.318562in}}%
\pgfpathlineto{\pgfqpoint{5.890140in}{3.011713in}}%
\pgfpathlineto{\pgfqpoint{5.890649in}{3.086469in}}%
\pgfpathlineto{\pgfqpoint{5.891420in}{2.994081in}}%
\pgfpathlineto{\pgfqpoint{5.891111in}{3.331680in}}%
\pgfpathlineto{\pgfqpoint{5.891605in}{3.234133in}}%
\pgfpathlineto{\pgfqpoint{5.892175in}{3.450861in}}%
\pgfpathlineto{\pgfqpoint{5.891882in}{3.027926in}}%
\pgfpathlineto{\pgfqpoint{5.892468in}{3.168794in}}%
\pgfpathlineto{\pgfqpoint{5.892622in}{2.879513in}}%
\pgfpathlineto{\pgfqpoint{5.892838in}{3.289826in}}%
\pgfpathlineto{\pgfqpoint{5.893547in}{3.209606in}}%
\pgfpathlineto{\pgfqpoint{5.894610in}{3.461178in}}%
\pgfpathlineto{\pgfqpoint{5.894348in}{2.983625in}}%
\pgfpathlineto{\pgfqpoint{5.894765in}{3.343525in}}%
\pgfpathlineto{\pgfqpoint{5.895057in}{2.908561in}}%
\pgfpathlineto{\pgfqpoint{5.896013in}{3.262890in}}%
\pgfpathlineto{\pgfqpoint{5.896244in}{3.002527in}}%
\pgfpathlineto{\pgfqpoint{5.896275in}{2.981268in}}%
\pgfpathlineto{\pgfqpoint{5.897077in}{3.333206in}}%
\pgfpathlineto{\pgfqpoint{5.897092in}{3.334773in}}%
\pgfpathlineto{\pgfqpoint{5.897293in}{3.248223in}}%
\pgfpathlineto{\pgfqpoint{5.897539in}{3.031572in}}%
\pgfpathlineto{\pgfqpoint{5.898295in}{3.415321in}}%
\pgfpathlineto{\pgfqpoint{5.898341in}{3.349089in}}%
\pgfpathlineto{\pgfqpoint{5.898742in}{2.863432in}}%
\pgfpathlineto{\pgfqpoint{5.899497in}{3.316924in}}%
\pgfpathlineto{\pgfqpoint{5.899512in}{3.324812in}}%
\pgfpathlineto{\pgfqpoint{5.899975in}{3.055903in}}%
\pgfpathlineto{\pgfqpoint{5.900252in}{3.156735in}}%
\pgfpathlineto{\pgfqpoint{5.901193in}{2.866613in}}%
\pgfpathlineto{\pgfqpoint{5.900745in}{3.459892in}}%
\pgfpathlineto{\pgfqpoint{5.901378in}{3.115194in}}%
\pgfpathlineto{\pgfqpoint{5.901948in}{3.334034in}}%
\pgfpathlineto{\pgfqpoint{5.902410in}{2.968389in}}%
\pgfpathlineto{\pgfqpoint{5.902426in}{2.961865in}}%
\pgfpathlineto{\pgfqpoint{5.902657in}{3.230000in}}%
\pgfpathlineto{\pgfqpoint{5.902996in}{3.165054in}}%
\pgfpathlineto{\pgfqpoint{5.903381in}{3.318189in}}%
\pgfpathlineto{\pgfqpoint{5.903628in}{3.037982in}}%
\pgfpathlineto{\pgfqpoint{5.904060in}{3.065088in}}%
\pgfpathlineto{\pgfqpoint{5.904892in}{2.899552in}}%
\pgfpathlineto{\pgfqpoint{5.904430in}{3.393316in}}%
\pgfpathlineto{\pgfqpoint{5.905077in}{3.137193in}}%
\pgfpathlineto{\pgfqpoint{5.905663in}{3.311463in}}%
\pgfpathlineto{\pgfqpoint{5.906079in}{3.052344in}}%
\pgfpathlineto{\pgfqpoint{5.906202in}{3.161065in}}%
\pgfpathlineto{\pgfqpoint{5.906218in}{3.160252in}}%
\pgfpathlineto{\pgfqpoint{5.906249in}{3.190460in}}%
\pgfpathlineto{\pgfqpoint{5.906927in}{3.404440in}}%
\pgfpathlineto{\pgfqpoint{5.906634in}{3.032951in}}%
\pgfpathlineto{\pgfqpoint{5.907266in}{3.052344in}}%
\pgfpathlineto{\pgfqpoint{5.907374in}{2.929933in}}%
\pgfpathlineto{\pgfqpoint{5.908083in}{3.341811in}}%
\pgfpathlineto{\pgfqpoint{5.908268in}{3.189697in}}%
\pgfpathlineto{\pgfqpoint{5.909224in}{3.303830in}}%
\pgfpathlineto{\pgfqpoint{5.908530in}{2.970156in}}%
\pgfpathlineto{\pgfqpoint{5.909378in}{3.227625in}}%
\pgfpathlineto{\pgfqpoint{5.910210in}{3.045532in}}%
\pgfpathlineto{\pgfqpoint{5.909547in}{3.310147in}}%
\pgfpathlineto{\pgfqpoint{5.910426in}{3.265164in}}%
\pgfpathlineto{\pgfqpoint{5.910534in}{3.351631in}}%
\pgfpathlineto{\pgfqpoint{5.911058in}{2.935785in}}%
\pgfpathlineto{\pgfqpoint{5.911505in}{3.190422in}}%
\pgfpathlineto{\pgfqpoint{5.912784in}{3.050763in}}%
\pgfpathlineto{\pgfqpoint{5.911675in}{3.293188in}}%
\pgfpathlineto{\pgfqpoint{5.912862in}{3.120786in}}%
\pgfpathlineto{\pgfqpoint{5.913093in}{3.350622in}}%
\pgfpathlineto{\pgfqpoint{5.913509in}{2.981008in}}%
\pgfpathlineto{\pgfqpoint{5.914002in}{3.181485in}}%
\pgfpathlineto{\pgfqpoint{5.914064in}{3.238333in}}%
\pgfpathlineto{\pgfqpoint{5.914218in}{3.356961in}}%
\pgfpathlineto{\pgfqpoint{5.914650in}{2.988080in}}%
\pgfpathlineto{\pgfqpoint{5.915112in}{3.217332in}}%
\pgfpathlineto{\pgfqpoint{5.916422in}{3.016300in}}%
\pgfpathlineto{\pgfqpoint{5.915744in}{3.314209in}}%
\pgfpathlineto{\pgfqpoint{5.916453in}{3.029602in}}%
\pgfpathlineto{\pgfqpoint{5.916885in}{3.371513in}}%
\pgfpathlineto{\pgfqpoint{5.917270in}{2.931290in}}%
\pgfpathlineto{\pgfqpoint{5.917640in}{3.160173in}}%
\pgfpathlineto{\pgfqpoint{5.918318in}{3.003284in}}%
\pgfpathlineto{\pgfqpoint{5.917794in}{3.319969in}}%
\pgfpathlineto{\pgfqpoint{5.918565in}{3.223937in}}%
\pgfpathlineto{\pgfqpoint{5.919336in}{3.357491in}}%
\pgfpathlineto{\pgfqpoint{5.918873in}{3.016912in}}%
\pgfpathlineto{\pgfqpoint{5.919582in}{3.035637in}}%
\pgfpathlineto{\pgfqpoint{5.919737in}{2.963879in}}%
\pgfpathlineto{\pgfqpoint{5.920384in}{3.369630in}}%
\pgfpathlineto{\pgfqpoint{5.920507in}{3.246353in}}%
\pgfpathlineto{\pgfqpoint{5.921062in}{3.275226in}}%
\pgfpathlineto{\pgfqpoint{5.920816in}{2.906661in}}%
\pgfpathlineto{\pgfqpoint{5.921340in}{3.100940in}}%
\pgfpathlineto{\pgfqpoint{5.922064in}{3.014430in}}%
\pgfpathlineto{\pgfqpoint{5.921910in}{3.301449in}}%
\pgfpathlineto{\pgfqpoint{5.922419in}{3.133980in}}%
\pgfpathlineto{\pgfqpoint{5.923035in}{3.402177in}}%
\pgfpathlineto{\pgfqpoint{5.923282in}{2.899911in}}%
\pgfpathlineto{\pgfqpoint{5.923529in}{3.155795in}}%
\pgfpathlineto{\pgfqpoint{5.924500in}{2.960321in}}%
\pgfpathlineto{\pgfqpoint{5.923929in}{3.307449in}}%
\pgfpathlineto{\pgfqpoint{5.924654in}{3.143454in}}%
\pgfpathlineto{\pgfqpoint{5.925486in}{3.410211in}}%
\pgfpathlineto{\pgfqpoint{5.925039in}{2.960257in}}%
\pgfpathlineto{\pgfqpoint{5.925687in}{3.125017in}}%
\pgfpathlineto{\pgfqpoint{5.925887in}{3.002947in}}%
\pgfpathlineto{\pgfqpoint{5.926565in}{3.380732in}}%
\pgfpathlineto{\pgfqpoint{5.926781in}{3.163426in}}%
\pgfpathlineto{\pgfqpoint{5.926982in}{2.915258in}}%
\pgfpathlineto{\pgfqpoint{5.927213in}{3.291502in}}%
\pgfpathlineto{\pgfqpoint{5.927845in}{3.211368in}}%
\pgfpathlineto{\pgfqpoint{5.927953in}{3.277371in}}%
\pgfpathlineto{\pgfqpoint{5.928215in}{3.038856in}}%
\pgfpathlineto{\pgfqpoint{5.928523in}{3.086001in}}%
\pgfpathlineto{\pgfqpoint{5.928739in}{3.005278in}}%
\pgfpathlineto{\pgfqpoint{5.928985in}{3.366824in}}%
\pgfpathlineto{\pgfqpoint{5.929140in}{3.318489in}}%
\pgfpathlineto{\pgfqpoint{5.929201in}{3.388203in}}%
\pgfpathlineto{\pgfqpoint{5.929432in}{2.921569in}}%
\pgfpathlineto{\pgfqpoint{5.930234in}{3.333083in}}%
\pgfpathlineto{\pgfqpoint{5.930650in}{2.939470in}}%
\pgfpathlineto{\pgfqpoint{5.931406in}{3.186510in}}%
\pgfpathlineto{\pgfqpoint{5.931621in}{3.406972in}}%
\pgfpathlineto{\pgfqpoint{5.931914in}{3.051343in}}%
\pgfpathlineto{\pgfqpoint{5.932546in}{3.303935in}}%
\pgfpathlineto{\pgfqpoint{5.933117in}{2.900044in}}%
\pgfpathlineto{\pgfqpoint{5.932716in}{3.374222in}}%
\pgfpathlineto{\pgfqpoint{5.933718in}{3.195699in}}%
\pgfpathlineto{\pgfqpoint{5.934088in}{3.272017in}}%
\pgfpathlineto{\pgfqpoint{5.934319in}{3.072175in}}%
\pgfpathlineto{\pgfqpoint{5.934704in}{3.077397in}}%
\pgfpathlineto{\pgfqpoint{5.935583in}{2.941997in}}%
\pgfpathlineto{\pgfqpoint{5.935136in}{3.391307in}}%
\pgfpathlineto{\pgfqpoint{5.935783in}{3.069471in}}%
\pgfpathlineto{\pgfqpoint{5.936354in}{3.326313in}}%
\pgfpathlineto{\pgfqpoint{5.936801in}{2.957219in}}%
\pgfpathlineto{\pgfqpoint{5.937772in}{3.389130in}}%
\pgfpathlineto{\pgfqpoint{5.938080in}{3.098157in}}%
\pgfpathlineto{\pgfqpoint{5.938419in}{3.042224in}}%
\pgfpathlineto{\pgfqpoint{5.938743in}{3.347028in}}%
\pgfpathlineto{\pgfqpoint{5.938866in}{3.366859in}}%
\pgfpathlineto{\pgfqpoint{5.939175in}{2.979727in}}%
\pgfpathlineto{\pgfqpoint{5.939221in}{2.925103in}}%
\pgfpathlineto{\pgfqpoint{5.939514in}{3.268457in}}%
\pgfpathlineto{\pgfqpoint{5.940115in}{3.197340in}}%
\pgfpathlineto{\pgfqpoint{5.940978in}{3.017138in}}%
\pgfpathlineto{\pgfqpoint{5.940238in}{3.244637in}}%
\pgfpathlineto{\pgfqpoint{5.941194in}{3.186920in}}%
\pgfpathlineto{\pgfqpoint{5.941286in}{3.394107in}}%
\pgfpathlineto{\pgfqpoint{5.941734in}{2.965702in}}%
\pgfpathlineto{\pgfqpoint{5.942366in}{3.268714in}}%
\pgfpathlineto{\pgfqpoint{5.942473in}{3.308469in}}%
\pgfpathlineto{\pgfqpoint{5.942843in}{2.970868in}}%
\pgfpathlineto{\pgfqpoint{5.943352in}{3.201057in}}%
\pgfpathlineto{\pgfqpoint{5.944570in}{3.030164in}}%
\pgfpathlineto{\pgfqpoint{5.943907in}{3.374503in}}%
\pgfpathlineto{\pgfqpoint{5.944585in}{3.050027in}}%
\pgfpathlineto{\pgfqpoint{5.945001in}{3.355055in}}%
\pgfpathlineto{\pgfqpoint{5.945356in}{2.933553in}}%
\pgfpathlineto{\pgfqpoint{5.945803in}{3.216907in}}%
\pgfpathlineto{\pgfqpoint{5.945849in}{3.224592in}}%
\pgfpathlineto{\pgfqpoint{5.945896in}{3.210798in}}%
\pgfpathlineto{\pgfqpoint{5.947190in}{3.009446in}}%
\pgfpathlineto{\pgfqpoint{5.946790in}{3.238300in}}%
\pgfpathlineto{\pgfqpoint{5.947221in}{3.078719in}}%
\pgfpathlineto{\pgfqpoint{5.947468in}{3.363117in}}%
\pgfpathlineto{\pgfqpoint{5.947869in}{3.005586in}}%
\pgfpathlineto{\pgfqpoint{5.948377in}{3.232536in}}%
\pgfpathlineto{\pgfqpoint{5.948516in}{3.315510in}}%
\pgfpathlineto{\pgfqpoint{5.948963in}{2.988798in}}%
\pgfpathlineto{\pgfqpoint{5.949441in}{3.221054in}}%
\pgfpathlineto{\pgfqpoint{5.950690in}{3.005791in}}%
\pgfpathlineto{\pgfqpoint{5.950042in}{3.361608in}}%
\pgfpathlineto{\pgfqpoint{5.950736in}{3.085849in}}%
\pgfpathlineto{\pgfqpoint{5.951106in}{3.371898in}}%
\pgfpathlineto{\pgfqpoint{5.951553in}{2.939190in}}%
\pgfpathlineto{\pgfqpoint{5.951984in}{3.252626in}}%
\pgfpathlineto{\pgfqpoint{5.953295in}{2.944097in}}%
\pgfpathlineto{\pgfqpoint{5.953726in}{3.404139in}}%
\pgfpathlineto{\pgfqpoint{5.954744in}{3.269254in}}%
\pgfpathlineto{\pgfqpoint{5.955715in}{3.010115in}}%
\pgfpathlineto{\pgfqpoint{5.955468in}{3.281870in}}%
\pgfpathlineto{\pgfqpoint{5.955946in}{3.074593in}}%
\pgfpathlineto{\pgfqpoint{5.956131in}{3.390235in}}%
\pgfpathlineto{\pgfqpoint{5.956794in}{2.986066in}}%
\pgfpathlineto{\pgfqpoint{5.957102in}{3.253788in}}%
\pgfpathlineto{\pgfqpoint{5.957225in}{3.349307in}}%
\pgfpathlineto{\pgfqpoint{5.957580in}{3.044081in}}%
\pgfpathlineto{\pgfqpoint{5.957657in}{3.004980in}}%
\pgfpathlineto{\pgfqpoint{5.957888in}{3.299103in}}%
\pgfpathlineto{\pgfqpoint{5.958582in}{3.157075in}}%
\pgfpathlineto{\pgfqpoint{5.958751in}{3.332652in}}%
\pgfpathlineto{\pgfqpoint{5.959414in}{2.957776in}}%
\pgfpathlineto{\pgfqpoint{5.959738in}{3.325022in}}%
\pgfpathlineto{\pgfqpoint{5.960293in}{3.031271in}}%
\pgfpathlineto{\pgfqpoint{5.959846in}{3.358706in}}%
\pgfpathlineto{\pgfqpoint{5.960987in}{3.159630in}}%
\pgfpathlineto{\pgfqpoint{5.961372in}{3.306746in}}%
\pgfpathlineto{\pgfqpoint{5.961989in}{3.002266in}}%
\pgfpathlineto{\pgfqpoint{5.962019in}{2.989593in}}%
\pgfpathlineto{\pgfqpoint{5.962097in}{3.119581in}}%
\pgfpathlineto{\pgfqpoint{5.962251in}{3.358008in}}%
\pgfpathlineto{\pgfqpoint{5.962898in}{3.038112in}}%
\pgfpathlineto{\pgfqpoint{5.963237in}{3.192719in}}%
\pgfpathlineto{\pgfqpoint{5.964440in}{3.019558in}}%
\pgfpathlineto{\pgfqpoint{5.963993in}{3.319388in}}%
\pgfpathlineto{\pgfqpoint{5.964578in}{3.073760in}}%
\pgfpathlineto{\pgfqpoint{5.964856in}{3.342952in}}%
\pgfpathlineto{\pgfqpoint{5.965519in}{2.985356in}}%
\pgfpathlineto{\pgfqpoint{5.965642in}{3.039368in}}%
\pgfpathlineto{\pgfqpoint{5.965657in}{3.039328in}}%
\pgfpathlineto{\pgfqpoint{5.965935in}{3.317367in}}%
\pgfpathlineto{\pgfqpoint{5.966860in}{3.172841in}}%
\pgfpathlineto{\pgfqpoint{5.968062in}{3.010797in}}%
\pgfpathlineto{\pgfqpoint{5.967677in}{3.320871in}}%
\pgfpathlineto{\pgfqpoint{5.968077in}{3.017959in}}%
\pgfpathlineto{\pgfqpoint{5.968340in}{3.339958in}}%
\pgfpathlineto{\pgfqpoint{5.968124in}{3.013143in}}%
\pgfpathlineto{\pgfqpoint{5.969465in}{3.184775in}}%
\pgfpathlineto{\pgfqpoint{5.970544in}{3.019757in}}%
\pgfpathlineto{\pgfqpoint{5.970097in}{3.340590in}}%
\pgfpathlineto{\pgfqpoint{5.970729in}{3.059182in}}%
\pgfpathlineto{\pgfqpoint{5.972054in}{3.288217in}}%
\pgfpathlineto{\pgfqpoint{5.971762in}{3.053451in}}%
\pgfpathlineto{\pgfqpoint{5.972162in}{3.204376in}}%
\pgfpathlineto{\pgfqpoint{5.972440in}{3.059856in}}%
\pgfpathlineto{\pgfqpoint{5.972702in}{3.277982in}}%
\pgfpathlineto{\pgfqpoint{5.973380in}{3.108017in}}%
\pgfpathlineto{\pgfqpoint{5.973796in}{3.293774in}}%
\pgfpathlineto{\pgfqpoint{5.974213in}{3.025101in}}%
\pgfpathlineto{\pgfqpoint{5.974706in}{3.186620in}}%
\pgfpathlineto{\pgfqpoint{5.975230in}{3.079304in}}%
\pgfpathlineto{\pgfqpoint{5.975307in}{3.230639in}}%
\pgfpathlineto{\pgfqpoint{5.975816in}{3.176151in}}%
\pgfpathlineto{\pgfqpoint{5.976170in}{3.285849in}}%
\pgfpathlineto{\pgfqpoint{5.975970in}{3.029552in}}%
\pgfpathlineto{\pgfqpoint{5.976926in}{3.189512in}}%
\pgfpathlineto{\pgfqpoint{5.977712in}{3.057899in}}%
\pgfpathlineto{\pgfqpoint{5.977049in}{3.269970in}}%
\pgfpathlineto{\pgfqpoint{5.978005in}{3.224076in}}%
\pgfpathlineto{\pgfqpoint{5.978791in}{3.291783in}}%
\pgfpathlineto{\pgfqpoint{5.978390in}{3.081718in}}%
\pgfpathlineto{\pgfqpoint{5.979068in}{3.174538in}}%
\pgfpathlineto{\pgfqpoint{5.979454in}{3.068823in}}%
\pgfpathlineto{\pgfqpoint{5.979885in}{3.257902in}}%
\pgfpathlineto{\pgfqpoint{5.980163in}{3.167908in}}%
\pgfpathlineto{\pgfqpoint{5.980533in}{3.274363in}}%
\pgfpathlineto{\pgfqpoint{5.980456in}{3.068422in}}%
\pgfpathlineto{\pgfqpoint{5.981273in}{3.196159in}}%
\pgfpathlineto{\pgfqpoint{5.982197in}{3.022283in}}%
\pgfpathlineto{\pgfqpoint{5.981396in}{3.274078in}}%
\pgfpathlineto{\pgfqpoint{5.982336in}{3.208401in}}%
\pgfpathlineto{\pgfqpoint{5.982382in}{3.270377in}}%
\pgfpathlineto{\pgfqpoint{5.982937in}{3.070379in}}%
\pgfpathlineto{\pgfqpoint{5.983415in}{3.164504in}}%
\pgfpathlineto{\pgfqpoint{5.983939in}{3.068926in}}%
\pgfpathlineto{\pgfqpoint{5.984016in}{3.254027in}}%
\pgfpathlineto{\pgfqpoint{5.984510in}{3.135865in}}%
\pgfpathlineto{\pgfqpoint{5.984880in}{3.308193in}}%
\pgfpathlineto{\pgfqpoint{5.984664in}{3.051630in}}%
\pgfpathlineto{\pgfqpoint{5.985635in}{3.195353in}}%
\pgfpathlineto{\pgfqpoint{5.985697in}{3.065257in}}%
\pgfpathlineto{\pgfqpoint{5.986622in}{3.277869in}}%
\pgfpathlineto{\pgfqpoint{5.986729in}{3.222558in}}%
\pgfpathlineto{\pgfqpoint{5.987084in}{3.077086in}}%
\pgfpathlineto{\pgfqpoint{5.987500in}{3.248692in}}%
\pgfpathlineto{\pgfqpoint{5.987978in}{3.133593in}}%
\pgfpathlineto{\pgfqpoint{5.988363in}{3.265352in}}%
\pgfpathlineto{\pgfqpoint{5.988163in}{3.055016in}}%
\pgfpathlineto{\pgfqpoint{5.989119in}{3.212689in}}%
\pgfpathlineto{\pgfqpoint{5.989905in}{3.089003in}}%
\pgfpathlineto{\pgfqpoint{5.990121in}{3.272426in}}%
\pgfpathlineto{\pgfqpoint{5.990213in}{3.221768in}}%
\pgfpathlineto{\pgfqpoint{5.990722in}{3.068114in}}%
\pgfpathlineto{\pgfqpoint{5.990968in}{3.256740in}}%
\pgfpathlineto{\pgfqpoint{5.990984in}{3.275483in}}%
\pgfpathlineto{\pgfqpoint{5.991431in}{3.108919in}}%
\pgfpathlineto{\pgfqpoint{5.992017in}{3.180392in}}%
\pgfpathlineto{\pgfqpoint{5.992726in}{3.265489in}}%
\pgfpathlineto{\pgfqpoint{5.992464in}{3.096749in}}%
\pgfpathlineto{\pgfqpoint{5.993065in}{3.149052in}}%
\pgfpathlineto{\pgfqpoint{5.993173in}{3.070419in}}%
\pgfpathlineto{\pgfqpoint{5.993589in}{3.257767in}}%
\pgfpathlineto{\pgfqpoint{5.994098in}{3.175059in}}%
\pgfpathlineto{\pgfqpoint{5.994468in}{3.239034in}}%
\pgfpathlineto{\pgfqpoint{5.994190in}{3.068228in}}%
\pgfpathlineto{\pgfqpoint{5.995208in}{3.194891in}}%
\pgfpathlineto{\pgfqpoint{5.995947in}{3.103421in}}%
\pgfpathlineto{\pgfqpoint{5.995331in}{3.273027in}}%
\pgfpathlineto{\pgfqpoint{5.996271in}{3.220570in}}%
\pgfpathlineto{\pgfqpoint{5.996302in}{3.224258in}}%
\pgfpathlineto{\pgfqpoint{5.996626in}{3.106191in}}%
\pgfpathlineto{\pgfqpoint{5.996657in}{3.069644in}}%
\pgfpathlineto{\pgfqpoint{5.997073in}{3.302944in}}%
\pgfpathlineto{\pgfqpoint{5.997720in}{3.099869in}}%
\pgfpathlineto{\pgfqpoint{5.998815in}{3.296080in}}%
\pgfpathlineto{\pgfqpoint{5.997874in}{3.098645in}}%
\pgfpathlineto{\pgfqpoint{5.998923in}{3.204790in}}%
\pgfpathlineto{\pgfqpoint{5.999416in}{3.087168in}}%
\pgfpathlineto{\pgfqpoint{5.999678in}{3.270786in}}%
\pgfpathlineto{\pgfqpoint{6.000079in}{3.159042in}}%
\pgfpathlineto{\pgfqpoint{6.000294in}{3.070666in}}%
\pgfpathlineto{\pgfqpoint{6.000557in}{3.268906in}}%
\pgfpathlineto{\pgfqpoint{6.001204in}{3.109771in}}%
\pgfpathlineto{\pgfqpoint{6.002298in}{3.306144in}}%
\pgfpathlineto{\pgfqpoint{6.001867in}{3.104722in}}%
\pgfpathlineto{\pgfqpoint{6.002360in}{3.220641in}}%
\pgfpathlineto{\pgfqpoint{6.002900in}{3.057914in}}%
\pgfpathlineto{\pgfqpoint{6.003162in}{3.285569in}}%
\pgfpathlineto{\pgfqpoint{6.003501in}{3.198641in}}%
\pgfpathlineto{\pgfqpoint{6.003963in}{3.088411in}}%
\pgfpathlineto{\pgfqpoint{6.004040in}{3.250773in}}%
\pgfpathlineto{\pgfqpoint{6.004534in}{3.192159in}}%
\pgfpathlineto{\pgfqpoint{6.004904in}{3.297843in}}%
\pgfpathlineto{\pgfqpoint{6.005351in}{3.064583in}}%
\pgfpathlineto{\pgfqpoint{6.005659in}{3.223609in}}%
\pgfpathlineto{\pgfqpoint{6.006414in}{3.032243in}}%
\pgfpathlineto{\pgfqpoint{6.005767in}{3.311338in}}%
\pgfpathlineto{\pgfqpoint{6.006784in}{3.196769in}}%
\pgfpathlineto{\pgfqpoint{6.007509in}{3.278236in}}%
\pgfpathlineto{\pgfqpoint{6.007092in}{3.097554in}}%
\pgfpathlineto{\pgfqpoint{6.007786in}{3.120465in}}%
\pgfpathlineto{\pgfqpoint{6.008834in}{3.027640in}}%
\pgfpathlineto{\pgfqpoint{6.008387in}{3.299997in}}%
\pgfpathlineto{\pgfqpoint{6.008896in}{3.109267in}}%
\pgfpathlineto{\pgfqpoint{6.009250in}{3.313606in}}%
\pgfpathlineto{\pgfqpoint{6.009035in}{3.076618in}}%
\pgfpathlineto{\pgfqpoint{6.010021in}{3.175033in}}%
\pgfpathlineto{\pgfqpoint{6.010576in}{3.109924in}}%
\pgfpathlineto{\pgfqpoint{6.010992in}{3.318135in}}%
\pgfpathlineto{\pgfqpoint{6.011116in}{3.188098in}}%
\pgfpathlineto{\pgfqpoint{6.011455in}{3.047658in}}%
\pgfpathlineto{\pgfqpoint{6.011871in}{3.305346in}}%
\pgfpathlineto{\pgfqpoint{6.012226in}{3.185492in}}%
\pgfpathlineto{\pgfqpoint{6.012534in}{3.071921in}}%
\pgfpathlineto{\pgfqpoint{6.012750in}{3.254491in}}%
\pgfpathlineto{\pgfqpoint{6.013243in}{3.180893in}}%
\pgfpathlineto{\pgfqpoint{6.013613in}{3.268974in}}%
\pgfpathlineto{\pgfqpoint{6.014075in}{3.073658in}}%
\pgfpathlineto{\pgfqpoint{6.014368in}{3.236099in}}%
\pgfpathlineto{\pgfqpoint{6.014939in}{3.053164in}}%
\pgfpathlineto{\pgfqpoint{6.014492in}{3.308079in}}%
\pgfpathlineto{\pgfqpoint{6.015509in}{3.201537in}}%
\pgfpathlineto{\pgfqpoint{6.015586in}{3.218569in}}%
\pgfpathlineto{\pgfqpoint{6.015786in}{3.147247in}}%
\pgfpathlineto{\pgfqpoint{6.015817in}{3.108098in}}%
\pgfpathlineto{\pgfqpoint{6.015894in}{3.224562in}}%
\pgfpathlineto{\pgfqpoint{6.016881in}{3.160765in}}%
\pgfpathlineto{\pgfqpoint{6.017112in}{3.288101in}}%
\pgfpathlineto{\pgfqpoint{6.017420in}{3.101559in}}%
\pgfpathlineto{\pgfqpoint{6.017559in}{3.043452in}}%
\pgfpathlineto{\pgfqpoint{6.017975in}{3.280275in}}%
\pgfpathlineto{\pgfqpoint{6.018345in}{3.171608in}}%
\pgfpathlineto{\pgfqpoint{6.018854in}{3.217331in}}%
\pgfpathlineto{\pgfqpoint{6.018638in}{3.080905in}}%
\pgfpathlineto{\pgfqpoint{6.019424in}{3.159661in}}%
\pgfpathlineto{\pgfqpoint{6.020164in}{3.085695in}}%
\pgfpathlineto{\pgfqpoint{6.019717in}{3.252378in}}%
\pgfpathlineto{\pgfqpoint{6.020411in}{3.176369in}}%
\pgfpathlineto{\pgfqpoint{6.020596in}{3.263508in}}%
\pgfpathlineto{\pgfqpoint{6.021197in}{3.066262in}}%
\pgfpathlineto{\pgfqpoint{6.021552in}{3.205115in}}%
\pgfpathlineto{\pgfqpoint{6.021675in}{3.216021in}}%
\pgfpathlineto{\pgfqpoint{6.021891in}{3.149171in}}%
\pgfpathlineto{\pgfqpoint{6.022785in}{3.118547in}}%
\pgfpathlineto{\pgfqpoint{6.021983in}{3.226354in}}%
\pgfpathlineto{\pgfqpoint{6.022985in}{3.150049in}}%
\pgfpathlineto{\pgfqpoint{6.023201in}{3.280301in}}%
\pgfpathlineto{\pgfqpoint{6.023648in}{3.050500in}}%
\pgfpathlineto{\pgfqpoint{6.024141in}{3.202031in}}%
\pgfpathlineto{\pgfqpoint{6.024681in}{3.102721in}}%
\pgfpathlineto{\pgfqpoint{6.024280in}{3.217056in}}%
\pgfpathlineto{\pgfqpoint{6.025421in}{3.128286in}}%
\pgfpathlineto{\pgfqpoint{6.025806in}{3.264264in}}%
\pgfpathlineto{\pgfqpoint{6.026268in}{3.098147in}}%
\pgfpathlineto{\pgfqpoint{6.026561in}{3.215525in}}%
\pgfpathlineto{\pgfqpoint{6.027332in}{3.061423in}}%
\pgfpathlineto{\pgfqpoint{6.026685in}{3.261487in}}%
\pgfpathlineto{\pgfqpoint{6.027687in}{3.196472in}}%
\pgfpathlineto{\pgfqpoint{6.028087in}{3.219994in}}%
\pgfpathlineto{\pgfqpoint{6.028010in}{3.133915in}}%
\pgfpathlineto{\pgfqpoint{6.028504in}{3.149639in}}%
\pgfpathlineto{\pgfqpoint{6.028874in}{3.092671in}}%
\pgfpathlineto{\pgfqpoint{6.029182in}{3.214143in}}%
\pgfpathlineto{\pgfqpoint{6.029244in}{3.207080in}}%
\pgfpathlineto{\pgfqpoint{6.029290in}{3.294017in}}%
\pgfpathlineto{\pgfqpoint{6.029752in}{3.053038in}}%
\pgfpathlineto{\pgfqpoint{6.030353in}{3.205673in}}%
\pgfpathlineto{\pgfqpoint{6.031032in}{3.231016in}}%
\pgfpathlineto{\pgfqpoint{6.030816in}{3.087525in}}%
\pgfpathlineto{\pgfqpoint{6.031386in}{3.168712in}}%
\pgfpathlineto{\pgfqpoint{6.031910in}{3.297961in}}%
\pgfpathlineto{\pgfqpoint{6.031494in}{3.100476in}}%
\pgfpathlineto{\pgfqpoint{6.032296in}{3.128684in}}%
\pgfpathlineto{\pgfqpoint{6.033436in}{3.063730in}}%
\pgfpathlineto{\pgfqpoint{6.032774in}{3.269384in}}%
\pgfpathlineto{\pgfqpoint{6.033452in}{3.084388in}}%
\pgfpathlineto{\pgfqpoint{6.033652in}{3.268480in}}%
\pgfpathlineto{\pgfqpoint{6.034592in}{3.158708in}}%
\pgfpathlineto{\pgfqpoint{6.035178in}{3.085305in}}%
\pgfpathlineto{\pgfqpoint{6.035379in}{3.281035in}}%
\pgfpathlineto{\pgfqpoint{6.035394in}{3.296781in}}%
\pgfpathlineto{\pgfqpoint{6.035856in}{3.054193in}}%
\pgfpathlineto{\pgfqpoint{6.036396in}{3.185935in}}%
\pgfpathlineto{\pgfqpoint{6.036920in}{3.075654in}}%
\pgfpathlineto{\pgfqpoint{6.037136in}{3.237587in}}%
\pgfpathlineto{\pgfqpoint{6.037336in}{3.219306in}}%
\pgfpathlineto{\pgfqpoint{6.037999in}{3.294888in}}%
\pgfpathlineto{\pgfqpoint{6.037583in}{3.096632in}}%
\pgfpathlineto{\pgfqpoint{6.038292in}{3.135484in}}%
\pgfpathlineto{\pgfqpoint{6.038462in}{3.095189in}}%
\pgfpathlineto{\pgfqpoint{6.038754in}{3.196316in}}%
\pgfpathlineto{\pgfqpoint{6.038832in}{3.176495in}}%
\pgfpathlineto{\pgfqpoint{6.039741in}{3.263218in}}%
\pgfpathlineto{\pgfqpoint{6.039525in}{3.062578in}}%
\pgfpathlineto{\pgfqpoint{6.039941in}{3.191370in}}%
\pgfpathlineto{\pgfqpoint{6.040604in}{3.236367in}}%
\pgfpathlineto{\pgfqpoint{6.040203in}{3.120305in}}%
\pgfpathlineto{\pgfqpoint{6.040990in}{3.167939in}}%
\pgfpathlineto{\pgfqpoint{6.041930in}{3.076390in}}%
\pgfpathlineto{\pgfqpoint{6.041483in}{3.284716in}}%
\pgfpathlineto{\pgfqpoint{6.042177in}{3.132464in}}%
\pgfpathlineto{\pgfqpoint{6.042346in}{3.251116in}}%
\pgfpathlineto{\pgfqpoint{6.043009in}{3.110517in}}%
\pgfpathlineto{\pgfqpoint{6.043317in}{3.185628in}}%
\pgfpathlineto{\pgfqpoint{6.043672in}{3.096190in}}%
\pgfpathlineto{\pgfqpoint{6.044088in}{3.263048in}}%
\pgfpathlineto{\pgfqpoint{6.044103in}{3.270379in}}%
\pgfpathlineto{\pgfqpoint{6.044550in}{3.095392in}}%
\pgfpathlineto{\pgfqpoint{6.045028in}{3.196021in}}%
\pgfpathlineto{\pgfqpoint{6.045552in}{3.073142in}}%
\pgfpathlineto{\pgfqpoint{6.045244in}{3.244916in}}%
\pgfpathlineto{\pgfqpoint{6.046138in}{3.183401in}}%
\pgfpathlineto{\pgfqpoint{6.047078in}{3.226728in}}%
\pgfpathlineto{\pgfqpoint{6.046940in}{3.101082in}}%
\pgfpathlineto{\pgfqpoint{6.047233in}{3.184408in}}%
\pgfpathlineto{\pgfqpoint{6.047818in}{3.104519in}}%
\pgfpathlineto{\pgfqpoint{6.048296in}{3.224246in}}%
\pgfpathlineto{\pgfqpoint{6.048774in}{3.103740in}}%
\pgfpathlineto{\pgfqpoint{6.049144in}{3.251515in}}%
\pgfpathlineto{\pgfqpoint{6.049576in}{3.140223in}}%
\pgfpathlineto{\pgfqpoint{6.050115in}{3.265751in}}%
\pgfpathlineto{\pgfqpoint{6.050408in}{3.105952in}}%
\pgfpathlineto{\pgfqpoint{6.050701in}{3.162439in}}%
\pgfpathlineto{\pgfqpoint{6.051163in}{3.116544in}}%
\pgfpathlineto{\pgfqpoint{6.050901in}{3.240039in}}%
\pgfpathlineto{\pgfqpoint{6.051749in}{3.177517in}}%
\pgfpathlineto{\pgfqpoint{6.051934in}{3.226459in}}%
\pgfpathlineto{\pgfqpoint{6.052258in}{3.124754in}}%
\pgfpathlineto{\pgfqpoint{6.052859in}{3.184636in}}%
\pgfpathlineto{\pgfqpoint{6.053106in}{3.115935in}}%
\pgfpathlineto{\pgfqpoint{6.053846in}{3.212524in}}%
\pgfpathlineto{\pgfqpoint{6.053969in}{3.172879in}}%
\pgfpathlineto{\pgfqpoint{6.054971in}{3.230951in}}%
\pgfpathlineto{\pgfqpoint{6.054632in}{3.101493in}}%
\pgfpathlineto{\pgfqpoint{6.055033in}{3.165916in}}%
\pgfpathlineto{\pgfqpoint{6.055603in}{3.110917in}}%
\pgfpathlineto{\pgfqpoint{6.055803in}{3.233133in}}%
\pgfpathlineto{\pgfqpoint{6.056081in}{3.186997in}}%
\pgfpathlineto{\pgfqpoint{6.056944in}{3.232366in}}%
\pgfpathlineto{\pgfqpoint{6.056574in}{3.077569in}}%
\pgfpathlineto{\pgfqpoint{6.057175in}{3.185303in}}%
\pgfpathlineto{\pgfqpoint{6.057237in}{3.123802in}}%
\pgfpathlineto{\pgfqpoint{6.057915in}{3.221525in}}%
\pgfpathlineto{\pgfqpoint{6.058347in}{3.131047in}}%
\pgfpathlineto{\pgfqpoint{6.058732in}{3.251456in}}%
\pgfpathlineto{\pgfqpoint{6.059071in}{3.107063in}}%
\pgfpathlineto{\pgfqpoint{6.059564in}{3.188559in}}%
\pgfpathlineto{\pgfqpoint{6.060227in}{3.125194in}}%
\pgfpathlineto{\pgfqpoint{6.059688in}{3.249031in}}%
\pgfpathlineto{\pgfqpoint{6.060628in}{3.220768in}}%
\pgfpathlineto{\pgfqpoint{6.060659in}{3.251772in}}%
\pgfpathlineto{\pgfqpoint{6.061568in}{3.108896in}}%
\pgfpathlineto{\pgfqpoint{6.061707in}{3.189558in}}%
\pgfpathlineto{\pgfqpoint{6.062848in}{3.096093in}}%
\pgfpathlineto{\pgfqpoint{6.062293in}{3.202561in}}%
\pgfpathlineto{\pgfqpoint{6.062894in}{3.162887in}}%
\pgfpathlineto{\pgfqpoint{6.063619in}{3.218059in}}%
\pgfpathlineto{\pgfqpoint{6.063834in}{3.117193in}}%
\pgfpathlineto{\pgfqpoint{6.063989in}{3.158190in}}%
\pgfpathlineto{\pgfqpoint{6.064559in}{3.213788in}}%
\pgfpathlineto{\pgfqpoint{6.064343in}{3.133972in}}%
\pgfpathlineto{\pgfqpoint{6.065021in}{3.157429in}}%
\pgfpathlineto{\pgfqpoint{6.065175in}{3.118785in}}%
\pgfpathlineto{\pgfqpoint{6.065545in}{3.237006in}}%
\pgfpathlineto{\pgfqpoint{6.066147in}{3.145980in}}%
\pgfpathlineto{\pgfqpoint{6.066532in}{3.211458in}}%
\pgfpathlineto{\pgfqpoint{6.066347in}{3.121387in}}%
\pgfpathlineto{\pgfqpoint{6.067303in}{3.164316in}}%
\pgfpathlineto{\pgfqpoint{6.067750in}{3.122797in}}%
\pgfpathlineto{\pgfqpoint{6.067519in}{3.233047in}}%
\pgfpathlineto{\pgfqpoint{6.068336in}{3.169667in}}%
\pgfpathlineto{\pgfqpoint{6.068490in}{3.239919in}}%
\pgfpathlineto{\pgfqpoint{6.068752in}{3.122802in}}%
\pgfpathlineto{\pgfqpoint{6.069476in}{3.221345in}}%
\pgfpathlineto{\pgfqpoint{6.070062in}{3.119449in}}%
\pgfpathlineto{\pgfqpoint{6.070463in}{3.222774in}}%
\pgfpathlineto{\pgfqpoint{6.070725in}{3.148910in}}%
\pgfpathlineto{\pgfqpoint{6.071758in}{3.203907in}}%
\pgfpathlineto{\pgfqpoint{6.071280in}{3.114694in}}%
\pgfpathlineto{\pgfqpoint{6.071866in}{3.194615in}}%
\pgfpathlineto{\pgfqpoint{6.072867in}{3.209050in}}%
\pgfpathlineto{\pgfqpoint{6.072266in}{3.131619in}}%
\pgfpathlineto{\pgfqpoint{6.072914in}{3.174979in}}%
\pgfpathlineto{\pgfqpoint{6.073993in}{3.128527in}}%
\pgfpathlineto{\pgfqpoint{6.073469in}{3.205555in}}%
\pgfpathlineto{\pgfqpoint{6.074070in}{3.164125in}}%
\pgfpathlineto{\pgfqpoint{6.074594in}{3.212808in}}%
\pgfpathlineto{\pgfqpoint{6.074979in}{3.121152in}}%
\pgfpathlineto{\pgfqpoint{6.075164in}{3.155397in}}%
\pgfpathlineto{\pgfqpoint{6.076182in}{3.120920in}}%
\pgfpathlineto{\pgfqpoint{6.075396in}{3.234179in}}%
\pgfpathlineto{\pgfqpoint{6.076243in}{3.178832in}}%
\pgfpathlineto{\pgfqpoint{6.077107in}{3.217578in}}%
\pgfpathlineto{\pgfqpoint{6.076937in}{3.123347in}}%
\pgfpathlineto{\pgfqpoint{6.077369in}{3.202145in}}%
\pgfpathlineto{\pgfqpoint{6.077492in}{3.125117in}}%
\pgfpathlineto{\pgfqpoint{6.078078in}{3.223629in}}%
\pgfpathlineto{\pgfqpoint{6.078540in}{3.168573in}}%
\pgfpathlineto{\pgfqpoint{6.079280in}{3.110055in}}%
\pgfpathlineto{\pgfqpoint{6.079064in}{3.215363in}}%
\pgfpathlineto{\pgfqpoint{6.079326in}{3.194080in}}%
\pgfpathlineto{\pgfqpoint{6.080328in}{3.222950in}}%
\pgfpathlineto{\pgfqpoint{6.080128in}{3.109845in}}%
\pgfpathlineto{\pgfqpoint{6.080375in}{3.186210in}}%
\pgfpathlineto{\pgfqpoint{6.081130in}{3.090583in}}%
\pgfpathlineto{\pgfqpoint{6.080621in}{3.223952in}}%
\pgfpathlineto{\pgfqpoint{6.081469in}{3.199967in}}%
\pgfpathlineto{\pgfqpoint{6.081608in}{3.225379in}}%
\pgfpathlineto{\pgfqpoint{6.081885in}{3.107676in}}%
\pgfpathlineto{\pgfqpoint{6.082517in}{3.171387in}}%
\pgfpathlineto{\pgfqpoint{6.083088in}{3.113942in}}%
\pgfpathlineto{\pgfqpoint{6.083288in}{3.242987in}}%
\pgfpathlineto{\pgfqpoint{6.083658in}{3.138111in}}%
\pgfpathlineto{\pgfqpoint{6.084259in}{3.225584in}}%
\pgfpathlineto{\pgfqpoint{6.083858in}{3.130056in}}%
\pgfpathlineto{\pgfqpoint{6.084783in}{3.156723in}}%
\pgfpathlineto{\pgfqpoint{6.084845in}{3.117689in}}%
\pgfpathlineto{\pgfqpoint{6.085246in}{3.233490in}}%
\pgfpathlineto{\pgfqpoint{6.085862in}{3.172034in}}%
\pgfpathlineto{\pgfqpoint{6.086540in}{3.208982in}}%
\pgfpathlineto{\pgfqpoint{6.086047in}{3.123189in}}%
\pgfpathlineto{\pgfqpoint{6.086972in}{3.180458in}}%
\pgfpathlineto{\pgfqpoint{6.088020in}{3.109048in}}%
\pgfpathlineto{\pgfqpoint{6.087527in}{3.214770in}}%
\pgfpathlineto{\pgfqpoint{6.088067in}{3.173683in}}%
\pgfpathlineto{\pgfqpoint{6.088514in}{3.210247in}}%
\pgfpathlineto{\pgfqpoint{6.089007in}{3.114874in}}%
\pgfpathlineto{\pgfqpoint{6.089176in}{3.184937in}}%
\pgfpathlineto{\pgfqpoint{6.089485in}{3.214624in}}%
\pgfpathlineto{\pgfqpoint{6.089762in}{3.123602in}}%
\pgfpathlineto{\pgfqpoint{6.090240in}{3.177377in}}%
\pgfpathlineto{\pgfqpoint{6.090964in}{3.111998in}}%
\pgfpathlineto{\pgfqpoint{6.091149in}{3.221016in}}%
\pgfpathlineto{\pgfqpoint{6.091304in}{3.197211in}}%
\pgfpathlineto{\pgfqpoint{6.092136in}{3.224550in}}%
\pgfpathlineto{\pgfqpoint{6.091951in}{3.115371in}}%
\pgfpathlineto{\pgfqpoint{6.092336in}{3.175031in}}%
\pgfpathlineto{\pgfqpoint{6.092938in}{3.122948in}}%
\pgfpathlineto{\pgfqpoint{6.093123in}{3.230023in}}%
\pgfpathlineto{\pgfqpoint{6.093477in}{3.134506in}}%
\pgfpathlineto{\pgfqpoint{6.093493in}{3.134261in}}%
\pgfpathlineto{\pgfqpoint{6.093508in}{3.142248in}}%
\pgfpathlineto{\pgfqpoint{6.093986in}{3.210277in}}%
\pgfpathlineto{\pgfqpoint{6.094464in}{3.127486in}}%
\pgfpathlineto{\pgfqpoint{6.094649in}{3.166891in}}%
\pgfpathlineto{\pgfqpoint{6.095450in}{3.128143in}}%
\pgfpathlineto{\pgfqpoint{6.095080in}{3.207669in}}%
\pgfpathlineto{\pgfqpoint{6.095759in}{3.165019in}}%
\pgfpathlineto{\pgfqpoint{6.096067in}{3.210107in}}%
\pgfpathlineto{\pgfqpoint{6.095882in}{3.111171in}}%
\pgfpathlineto{\pgfqpoint{6.096822in}{3.160660in}}%
\pgfpathlineto{\pgfqpoint{6.096868in}{3.126767in}}%
\pgfpathlineto{\pgfqpoint{6.097346in}{3.209613in}}%
\pgfpathlineto{\pgfqpoint{6.097917in}{3.173795in}}%
\pgfpathlineto{\pgfqpoint{6.098256in}{3.137320in}}%
\pgfpathlineto{\pgfqpoint{6.098009in}{3.208051in}}%
\pgfpathlineto{\pgfqpoint{6.098980in}{3.205683in}}%
\pgfpathlineto{\pgfqpoint{6.098996in}{3.206206in}}%
\pgfpathlineto{\pgfqpoint{6.099119in}{3.173025in}}%
\pgfpathlineto{\pgfqpoint{6.099181in}{3.185129in}}%
\pgfpathlineto{\pgfqpoint{6.100213in}{3.122922in}}%
\pgfpathlineto{\pgfqpoint{6.099951in}{3.223536in}}%
\pgfpathlineto{\pgfqpoint{6.100321in}{3.153163in}}%
\pgfpathlineto{\pgfqpoint{6.101185in}{3.129434in}}%
\pgfpathlineto{\pgfqpoint{6.100922in}{3.228250in}}%
\pgfpathlineto{\pgfqpoint{6.101416in}{3.144354in}}%
\pgfpathlineto{\pgfqpoint{6.101894in}{3.219308in}}%
\pgfpathlineto{\pgfqpoint{6.102140in}{3.124284in}}%
\pgfpathlineto{\pgfqpoint{6.102556in}{3.188799in}}%
\pgfpathlineto{\pgfqpoint{6.102880in}{3.217019in}}%
\pgfpathlineto{\pgfqpoint{6.103127in}{3.121328in}}%
\pgfpathlineto{\pgfqpoint{6.103605in}{3.174507in}}%
\pgfpathlineto{\pgfqpoint{6.104098in}{3.129463in}}%
\pgfpathlineto{\pgfqpoint{6.103851in}{3.212462in}}%
\pgfpathlineto{\pgfqpoint{6.104668in}{3.172234in}}%
\pgfpathlineto{\pgfqpoint{6.104822in}{3.247744in}}%
\pgfpathlineto{\pgfqpoint{6.105069in}{3.114151in}}%
\pgfpathlineto{\pgfqpoint{6.105809in}{3.235314in}}%
\pgfpathlineto{\pgfqpoint{6.107011in}{3.110783in}}%
\pgfpathlineto{\pgfqpoint{6.107027in}{3.122264in}}%
\pgfpathlineto{\pgfqpoint{6.107767in}{3.229364in}}%
\pgfpathlineto{\pgfqpoint{6.107982in}{3.103327in}}%
\pgfpathlineto{\pgfqpoint{6.108167in}{3.182495in}}%
\pgfpathlineto{\pgfqpoint{6.108969in}{3.088862in}}%
\pgfpathlineto{\pgfqpoint{6.108753in}{3.221592in}}%
\pgfpathlineto{\pgfqpoint{6.109293in}{3.171846in}}%
\pgfpathlineto{\pgfqpoint{6.109940in}{3.080409in}}%
\pgfpathlineto{\pgfqpoint{6.109694in}{3.225153in}}%
\pgfpathlineto{\pgfqpoint{6.110372in}{3.184726in}}%
\pgfpathlineto{\pgfqpoint{6.110665in}{3.216904in}}%
\pgfpathlineto{\pgfqpoint{6.110927in}{3.104679in}}%
\pgfpathlineto{\pgfqpoint{6.111420in}{3.173344in}}%
\pgfpathlineto{\pgfqpoint{6.111898in}{3.093020in}}%
\pgfpathlineto{\pgfqpoint{6.111651in}{3.216550in}}%
\pgfpathlineto{\pgfqpoint{6.112561in}{3.141450in}}%
\pgfpathlineto{\pgfqpoint{6.112638in}{3.224601in}}%
\pgfpathlineto{\pgfqpoint{6.112869in}{3.097777in}}%
\pgfpathlineto{\pgfqpoint{6.113701in}{3.172891in}}%
\pgfpathlineto{\pgfqpoint{6.113856in}{3.115810in}}%
\pgfpathlineto{\pgfqpoint{6.114580in}{3.223016in}}%
\pgfpathlineto{\pgfqpoint{6.114857in}{3.130461in}}%
\pgfpathlineto{\pgfqpoint{6.115567in}{3.226930in}}%
\pgfpathlineto{\pgfqpoint{6.115813in}{3.121031in}}%
\pgfpathlineto{\pgfqpoint{6.115983in}{3.171775in}}%
\pgfpathlineto{\pgfqpoint{6.116553in}{3.217810in}}%
\pgfpathlineto{\pgfqpoint{6.116784in}{3.110675in}}%
\pgfpathlineto{\pgfqpoint{6.117786in}{3.104382in}}%
\pgfpathlineto{\pgfqpoint{6.117540in}{3.214061in}}%
\pgfpathlineto{\pgfqpoint{6.117817in}{3.139379in}}%
\pgfpathlineto{\pgfqpoint{6.118511in}{3.203631in}}%
\pgfpathlineto{\pgfqpoint{6.118773in}{3.125754in}}%
\pgfpathlineto{\pgfqpoint{6.118958in}{3.180495in}}%
\pgfpathlineto{\pgfqpoint{6.119497in}{3.215073in}}%
\pgfpathlineto{\pgfqpoint{6.119081in}{3.137087in}}%
\pgfpathlineto{\pgfqpoint{6.119713in}{3.153389in}}%
\pgfpathlineto{\pgfqpoint{6.119759in}{3.134231in}}%
\pgfpathlineto{\pgfqpoint{6.120484in}{3.224705in}}%
\pgfpathlineto{\pgfqpoint{6.120777in}{3.171144in}}%
\pgfpathlineto{\pgfqpoint{6.121470in}{3.219115in}}%
\pgfpathlineto{\pgfqpoint{6.121054in}{3.127597in}}%
\pgfpathlineto{\pgfqpoint{6.121902in}{3.193289in}}%
\pgfpathlineto{\pgfqpoint{6.123027in}{3.116628in}}%
\pgfpathlineto{\pgfqpoint{6.122349in}{3.203140in}}%
\pgfpathlineto{\pgfqpoint{6.123074in}{3.169176in}}%
\pgfpathlineto{\pgfqpoint{6.123181in}{3.205457in}}%
\pgfpathlineto{\pgfqpoint{6.123690in}{3.121729in}}%
\pgfpathlineto{\pgfqpoint{6.124183in}{3.186060in}}%
\pgfpathlineto{\pgfqpoint{6.124677in}{3.123480in}}%
\pgfpathlineto{\pgfqpoint{6.124322in}{3.208973in}}%
\pgfpathlineto{\pgfqpoint{6.125278in}{3.163333in}}%
\pgfpathlineto{\pgfqpoint{6.125309in}{3.204347in}}%
\pgfpathlineto{\pgfqpoint{6.125987in}{3.123923in}}%
\pgfpathlineto{\pgfqpoint{6.126403in}{3.189099in}}%
\pgfpathlineto{\pgfqpoint{6.126449in}{3.190870in}}%
\pgfpathlineto{\pgfqpoint{6.126480in}{3.174375in}}%
\pgfpathlineto{\pgfqpoint{6.126974in}{3.122567in}}%
\pgfpathlineto{\pgfqpoint{6.127436in}{3.214177in}}%
\pgfpathlineto{\pgfqpoint{6.127575in}{3.185367in}}%
\pgfpathlineto{\pgfqpoint{6.128207in}{3.116910in}}%
\pgfpathlineto{\pgfqpoint{6.128423in}{3.218631in}}%
\pgfpathlineto{\pgfqpoint{6.128669in}{3.185717in}}%
\pgfpathlineto{\pgfqpoint{6.129378in}{3.227380in}}%
\pgfpathlineto{\pgfqpoint{6.128947in}{3.113981in}}%
\pgfpathlineto{\pgfqpoint{6.129748in}{3.171895in}}%
\pgfpathlineto{\pgfqpoint{6.129933in}{3.117744in}}%
\pgfpathlineto{\pgfqpoint{6.130349in}{3.220454in}}%
\pgfpathlineto{\pgfqpoint{6.130920in}{3.138164in}}%
\pgfpathlineto{\pgfqpoint{6.131213in}{3.210712in}}%
\pgfpathlineto{\pgfqpoint{6.131583in}{3.121381in}}%
\pgfpathlineto{\pgfqpoint{6.132076in}{3.180951in}}%
\pgfpathlineto{\pgfqpoint{6.132569in}{3.118933in}}%
\pgfpathlineto{\pgfqpoint{6.132322in}{3.221510in}}%
\pgfpathlineto{\pgfqpoint{6.133170in}{3.197467in}}%
\pgfpathlineto{\pgfqpoint{6.133617in}{3.215576in}}%
\pgfpathlineto{\pgfqpoint{6.133247in}{3.124622in}}%
\pgfpathlineto{\pgfqpoint{6.134203in}{3.151978in}}%
\pgfpathlineto{\pgfqpoint{6.135097in}{3.120322in}}%
\pgfpathlineto{\pgfqpoint{6.134280in}{3.216497in}}%
\pgfpathlineto{\pgfqpoint{6.135251in}{3.204517in}}%
\pgfpathlineto{\pgfqpoint{6.136253in}{3.224103in}}%
\pgfpathlineto{\pgfqpoint{6.136068in}{3.118801in}}%
\pgfpathlineto{\pgfqpoint{6.136330in}{3.170533in}}%
\pgfpathlineto{\pgfqpoint{6.137471in}{3.132640in}}%
\pgfpathlineto{\pgfqpoint{6.137240in}{3.213302in}}%
\pgfpathlineto{\pgfqpoint{6.137486in}{3.133167in}}%
\pgfpathlineto{\pgfqpoint{6.138226in}{3.201735in}}%
\pgfpathlineto{\pgfqpoint{6.138627in}{3.169180in}}%
\pgfpathlineto{\pgfqpoint{6.138689in}{3.146517in}}%
\pgfpathlineto{\pgfqpoint{6.138843in}{3.184941in}}%
\pgfpathlineto{\pgfqpoint{6.139506in}{3.204470in}}%
\pgfpathlineto{\pgfqpoint{6.139444in}{3.132208in}}%
\pgfpathlineto{\pgfqpoint{6.139937in}{3.179421in}}%
\pgfpathlineto{\pgfqpoint{6.140014in}{3.120445in}}%
\pgfpathlineto{\pgfqpoint{6.140492in}{3.212599in}}%
\pgfpathlineto{\pgfqpoint{6.141032in}{3.188515in}}%
\pgfpathlineto{\pgfqpoint{6.141479in}{3.218202in}}%
\pgfpathlineto{\pgfqpoint{6.141957in}{3.121003in}}%
\pgfpathlineto{\pgfqpoint{6.142049in}{3.170846in}}%
\pgfpathlineto{\pgfqpoint{6.142943in}{3.116759in}}%
\pgfpathlineto{\pgfqpoint{6.143113in}{3.222743in}}%
\pgfpathlineto{\pgfqpoint{6.143914in}{3.126386in}}%
\pgfpathlineto{\pgfqpoint{6.144793in}{3.166288in}}%
\pgfpathlineto{\pgfqpoint{6.145071in}{3.216601in}}%
\pgfpathlineto{\pgfqpoint{6.145317in}{3.125999in}}%
\pgfpathlineto{\pgfqpoint{6.145857in}{3.147901in}}%
\pgfpathlineto{\pgfqpoint{6.146843in}{3.129545in}}%
\pgfpathlineto{\pgfqpoint{6.146057in}{3.213137in}}%
\pgfpathlineto{\pgfqpoint{6.146951in}{3.148704in}}%
\pgfpathlineto{\pgfqpoint{6.148015in}{3.221891in}}%
\pgfpathlineto{\pgfqpoint{6.147259in}{3.129827in}}%
\pgfpathlineto{\pgfqpoint{6.148076in}{3.189252in}}%
\pgfpathlineto{\pgfqpoint{6.149202in}{3.120293in}}%
\pgfpathlineto{\pgfqpoint{6.148955in}{3.218207in}}%
\pgfpathlineto{\pgfqpoint{6.149233in}{3.143432in}}%
\pgfpathlineto{\pgfqpoint{6.149942in}{3.212411in}}%
\pgfpathlineto{\pgfqpoint{6.150173in}{3.120332in}}%
\pgfpathlineto{\pgfqpoint{6.150358in}{3.176137in}}%
\pgfpathlineto{\pgfqpoint{6.151144in}{3.123962in}}%
\pgfpathlineto{\pgfqpoint{6.150913in}{3.202757in}}%
\pgfpathlineto{\pgfqpoint{6.151468in}{3.172766in}}%
\pgfpathlineto{\pgfqpoint{6.151699in}{3.123649in}}%
\pgfpathlineto{\pgfqpoint{6.151853in}{3.196846in}}%
\pgfpathlineto{\pgfqpoint{6.151884in}{3.211024in}}%
\pgfpathlineto{\pgfqpoint{6.152670in}{3.124147in}}%
\pgfpathlineto{\pgfqpoint{6.152948in}{3.188110in}}%
\pgfpathlineto{\pgfqpoint{6.153102in}{3.120468in}}%
\pgfpathlineto{\pgfqpoint{6.153718in}{3.205211in}}%
\pgfpathlineto{\pgfqpoint{6.154119in}{3.165401in}}%
\pgfpathlineto{\pgfqpoint{6.154551in}{3.209985in}}%
\pgfpathlineto{\pgfqpoint{6.155044in}{3.137443in}}%
\pgfpathlineto{\pgfqpoint{6.155244in}{3.192845in}}%
\pgfpathlineto{\pgfqpoint{6.156431in}{3.131408in}}%
\pgfpathlineto{\pgfqpoint{6.155537in}{3.218085in}}%
\pgfpathlineto{\pgfqpoint{6.156462in}{3.143841in}}%
\pgfpathlineto{\pgfqpoint{6.157233in}{3.217575in}}%
\pgfpathlineto{\pgfqpoint{6.156986in}{3.123692in}}%
\pgfpathlineto{\pgfqpoint{6.157618in}{3.198506in}}%
\pgfpathlineto{\pgfqpoint{6.157880in}{3.127726in}}%
\pgfpathlineto{\pgfqpoint{6.158127in}{3.248527in}}%
\pgfpathlineto{\pgfqpoint{6.158759in}{3.153628in}}%
\pgfpathlineto{\pgfqpoint{6.159160in}{3.249651in}}%
\pgfpathlineto{\pgfqpoint{6.158913in}{3.095299in}}%
\pgfpathlineto{\pgfqpoint{6.159853in}{3.158289in}}%
\pgfpathlineto{\pgfqpoint{6.159915in}{3.093620in}}%
\pgfpathlineto{\pgfqpoint{6.160177in}{3.226830in}}%
\pgfpathlineto{\pgfqpoint{6.160979in}{3.141931in}}%
\pgfpathlineto{\pgfqpoint{6.161210in}{3.224466in}}%
\pgfpathlineto{\pgfqpoint{6.161549in}{3.124804in}}%
\pgfpathlineto{\pgfqpoint{6.162258in}{3.215192in}}%
\pgfpathlineto{\pgfqpoint{6.162828in}{3.106926in}}%
\pgfpathlineto{\pgfqpoint{6.163260in}{3.222877in}}%
\pgfpathlineto{\pgfqpoint{6.163368in}{3.195351in}}%
\pgfpathlineto{\pgfqpoint{6.163399in}{3.220318in}}%
\pgfpathlineto{\pgfqpoint{6.163846in}{3.100991in}}%
\pgfpathlineto{\pgfqpoint{6.164462in}{3.180040in}}%
\pgfpathlineto{\pgfqpoint{6.164863in}{3.106799in}}%
\pgfpathlineto{\pgfqpoint{6.165326in}{3.231797in}}%
\pgfpathlineto{\pgfqpoint{6.165649in}{3.124516in}}%
\pgfpathlineto{\pgfqpoint{6.165665in}{3.118167in}}%
\pgfpathlineto{\pgfqpoint{6.166343in}{3.237544in}}%
\pgfpathlineto{\pgfqpoint{6.166744in}{3.122454in}}%
\pgfpathlineto{\pgfqpoint{6.167360in}{3.230213in}}%
\pgfpathlineto{\pgfqpoint{6.167761in}{3.118769in}}%
\pgfpathlineto{\pgfqpoint{6.168069in}{3.181199in}}%
\pgfpathlineto{\pgfqpoint{6.168763in}{3.096662in}}%
\pgfpathlineto{\pgfqpoint{6.168917in}{3.216008in}}%
\pgfpathlineto{\pgfqpoint{6.169195in}{3.161131in}}%
\pgfpathlineto{\pgfqpoint{6.169950in}{3.224947in}}%
\pgfpathlineto{\pgfqpoint{6.169765in}{3.096830in}}%
\pgfpathlineto{\pgfqpoint{6.170366in}{3.181584in}}%
\pgfpathlineto{\pgfqpoint{6.170752in}{3.103967in}}%
\pgfpathlineto{\pgfqpoint{6.170459in}{3.227119in}}%
\pgfpathlineto{\pgfqpoint{6.171430in}{3.192881in}}%
\pgfpathlineto{\pgfqpoint{6.171476in}{3.232674in}}%
\pgfpathlineto{\pgfqpoint{6.171815in}{3.110082in}}%
\pgfpathlineto{\pgfqpoint{6.172524in}{3.203589in}}%
\pgfpathlineto{\pgfqpoint{6.172817in}{3.127407in}}%
\pgfpathlineto{\pgfqpoint{6.173542in}{3.219763in}}%
\pgfpathlineto{\pgfqpoint{6.173665in}{3.165729in}}%
\pgfpathlineto{\pgfqpoint{6.174559in}{3.217749in}}%
\pgfpathlineto{\pgfqpoint{6.174282in}{3.112874in}}%
\pgfpathlineto{\pgfqpoint{6.174837in}{3.200798in}}%
\pgfpathlineto{\pgfqpoint{6.175314in}{3.111434in}}%
\pgfpathlineto{\pgfqpoint{6.175083in}{3.215459in}}%
\pgfpathlineto{\pgfqpoint{6.175993in}{3.120494in}}%
\pgfpathlineto{\pgfqpoint{6.176779in}{3.229525in}}%
\pgfpathlineto{\pgfqpoint{6.176964in}{3.110174in}}%
\pgfpathlineto{\pgfqpoint{6.177180in}{3.172449in}}%
\pgfpathlineto{\pgfqpoint{6.177981in}{3.108981in}}%
\pgfpathlineto{\pgfqpoint{6.177781in}{3.225313in}}%
\pgfpathlineto{\pgfqpoint{6.178274in}{3.174564in}}%
\pgfpathlineto{\pgfqpoint{6.178690in}{3.232951in}}%
\pgfpathlineto{\pgfqpoint{6.179014in}{3.121758in}}%
\pgfpathlineto{\pgfqpoint{6.179369in}{3.184759in}}%
\pgfpathlineto{\pgfqpoint{6.180047in}{3.110193in}}%
\pgfpathlineto{\pgfqpoint{6.179708in}{3.252426in}}%
\pgfpathlineto{\pgfqpoint{6.180509in}{3.149381in}}%
\pgfpathlineto{\pgfqpoint{6.180555in}{3.146796in}}%
\pgfpathlineto{\pgfqpoint{6.180710in}{3.236268in}}%
\pgfpathlineto{\pgfqpoint{6.180725in}{3.248686in}}%
\pgfpathlineto{\pgfqpoint{6.181080in}{3.102923in}}%
\pgfpathlineto{\pgfqpoint{6.181789in}{3.223516in}}%
\pgfpathlineto{\pgfqpoint{6.182529in}{3.100708in}}%
\pgfpathlineto{\pgfqpoint{6.182791in}{3.236645in}}%
\pgfpathlineto{\pgfqpoint{6.182914in}{3.198352in}}%
\pgfpathlineto{\pgfqpoint{6.183315in}{3.227525in}}%
\pgfpathlineto{\pgfqpoint{6.183546in}{3.096468in}}%
\pgfpathlineto{\pgfqpoint{6.184008in}{3.188411in}}%
\pgfpathlineto{\pgfqpoint{6.184163in}{3.095311in}}%
\pgfpathlineto{\pgfqpoint{6.184856in}{3.241912in}}%
\pgfpathlineto{\pgfqpoint{6.185211in}{3.115441in}}%
\pgfpathlineto{\pgfqpoint{6.185874in}{3.230702in}}%
\pgfpathlineto{\pgfqpoint{6.186213in}{3.114824in}}%
\pgfpathlineto{\pgfqpoint{6.186398in}{3.215686in}}%
\pgfpathlineto{\pgfqpoint{6.187261in}{3.101930in}}%
\pgfpathlineto{\pgfqpoint{6.186906in}{3.234336in}}%
\pgfpathlineto{\pgfqpoint{6.187538in}{3.187176in}}%
\pgfpathlineto{\pgfqpoint{6.187939in}{3.234269in}}%
\pgfpathlineto{\pgfqpoint{6.188278in}{3.093737in}}%
\pgfpathlineto{\pgfqpoint{6.188617in}{3.184810in}}%
\pgfpathlineto{\pgfqpoint{6.189311in}{3.096819in}}%
\pgfpathlineto{\pgfqpoint{6.188972in}{3.235413in}}%
\pgfpathlineto{\pgfqpoint{6.189758in}{3.155089in}}%
\pgfpathlineto{\pgfqpoint{6.189989in}{3.235276in}}%
\pgfpathlineto{\pgfqpoint{6.190344in}{3.095726in}}%
\pgfpathlineto{\pgfqpoint{6.191038in}{3.224601in}}%
\pgfpathlineto{\pgfqpoint{6.191361in}{3.081059in}}%
\pgfpathlineto{\pgfqpoint{6.191546in}{3.230629in}}%
\pgfpathlineto{\pgfqpoint{6.192163in}{3.181176in}}%
\pgfpathlineto{\pgfqpoint{6.192579in}{3.237727in}}%
\pgfpathlineto{\pgfqpoint{6.192379in}{3.097955in}}%
\pgfpathlineto{\pgfqpoint{6.193257in}{3.178360in}}%
\pgfpathlineto{\pgfqpoint{6.193427in}{3.098630in}}%
\pgfpathlineto{\pgfqpoint{6.194121in}{3.240095in}}%
\pgfpathlineto{\pgfqpoint{6.194352in}{3.177641in}}%
\pgfpathlineto{\pgfqpoint{6.194645in}{3.231380in}}%
\pgfpathlineto{\pgfqpoint{6.194444in}{3.094588in}}%
\pgfpathlineto{\pgfqpoint{6.195415in}{3.166874in}}%
\pgfpathlineto{\pgfqpoint{6.196525in}{3.099767in}}%
\pgfpathlineto{\pgfqpoint{6.195662in}{3.222239in}}%
\pgfpathlineto{\pgfqpoint{6.196556in}{3.118449in}}%
\pgfpathlineto{\pgfqpoint{6.197728in}{3.243591in}}%
\pgfpathlineto{\pgfqpoint{6.197543in}{3.099207in}}%
\pgfpathlineto{\pgfqpoint{6.197774in}{3.187349in}}%
\pgfpathlineto{\pgfqpoint{6.198575in}{3.096993in}}%
\pgfpathlineto{\pgfqpoint{6.198760in}{3.231311in}}%
\pgfpathlineto{\pgfqpoint{6.198868in}{3.186426in}}%
\pgfpathlineto{\pgfqpoint{6.199269in}{3.223847in}}%
\pgfpathlineto{\pgfqpoint{6.199593in}{3.098122in}}%
\pgfpathlineto{\pgfqpoint{6.199932in}{3.171672in}}%
\pgfpathlineto{\pgfqpoint{6.200626in}{3.100832in}}%
\pgfpathlineto{\pgfqpoint{6.200302in}{3.230465in}}%
\pgfpathlineto{\pgfqpoint{6.201119in}{3.127949in}}%
\pgfpathlineto{\pgfqpoint{6.201335in}{3.246765in}}%
\pgfpathlineto{\pgfqpoint{6.201674in}{3.105607in}}%
\pgfpathlineto{\pgfqpoint{6.202367in}{3.226299in}}%
\pgfpathlineto{\pgfqpoint{6.202691in}{3.088425in}}%
\pgfpathlineto{\pgfqpoint{6.203385in}{3.227719in}}%
\pgfpathlineto{\pgfqpoint{6.203508in}{3.175969in}}%
\pgfpathlineto{\pgfqpoint{6.203909in}{3.240622in}}%
\pgfpathlineto{\pgfqpoint{6.203709in}{3.089021in}}%
\pgfpathlineto{\pgfqpoint{6.204618in}{3.179318in}}%
\pgfpathlineto{\pgfqpoint{6.204741in}{3.097625in}}%
\pgfpathlineto{\pgfqpoint{6.204926in}{3.234310in}}%
\pgfpathlineto{\pgfqpoint{6.205805in}{3.113955in}}%
\pgfpathlineto{\pgfqpoint{6.206483in}{3.224896in}}%
\pgfpathlineto{\pgfqpoint{6.206792in}{3.100998in}}%
\pgfpathlineto{\pgfqpoint{6.207007in}{3.208495in}}%
\pgfpathlineto{\pgfqpoint{6.207840in}{3.110555in}}%
\pgfpathlineto{\pgfqpoint{6.207501in}{3.239254in}}%
\pgfpathlineto{\pgfqpoint{6.208133in}{3.188153in}}%
\pgfpathlineto{\pgfqpoint{6.208518in}{3.223081in}}%
\pgfpathlineto{\pgfqpoint{6.208857in}{3.108584in}}%
\pgfpathlineto{\pgfqpoint{6.209212in}{3.182303in}}%
\pgfpathlineto{\pgfqpoint{6.209890in}{3.109005in}}%
\pgfpathlineto{\pgfqpoint{6.210075in}{3.231418in}}%
\pgfpathlineto{\pgfqpoint{6.210368in}{3.133791in}}%
\pgfpathlineto{\pgfqpoint{6.211108in}{3.246467in}}%
\pgfpathlineto{\pgfqpoint{6.210923in}{3.097901in}}%
\pgfpathlineto{\pgfqpoint{6.211632in}{3.205665in}}%
\pgfpathlineto{\pgfqpoint{6.211940in}{3.107170in}}%
\pgfpathlineto{\pgfqpoint{6.212125in}{3.244349in}}%
\pgfpathlineto{\pgfqpoint{6.212757in}{3.166589in}}%
\pgfpathlineto{\pgfqpoint{6.213158in}{3.237006in}}%
\pgfpathlineto{\pgfqpoint{6.212988in}{3.102591in}}%
\pgfpathlineto{\pgfqpoint{6.213867in}{3.187349in}}%
\pgfpathlineto{\pgfqpoint{6.214006in}{3.109550in}}%
\pgfpathlineto{\pgfqpoint{6.214191in}{3.234712in}}%
\pgfpathlineto{\pgfqpoint{6.215054in}{3.132624in}}%
\pgfpathlineto{\pgfqpoint{6.215223in}{3.240612in}}%
\pgfpathlineto{\pgfqpoint{6.215455in}{3.104574in}}%
\pgfpathlineto{\pgfqpoint{6.216272in}{3.208731in}}%
\pgfpathlineto{\pgfqpoint{6.216487in}{3.103468in}}%
\pgfpathlineto{\pgfqpoint{6.217289in}{3.226226in}}%
\pgfpathlineto{\pgfqpoint{6.217397in}{3.174697in}}%
\pgfpathlineto{\pgfqpoint{6.217798in}{3.216458in}}%
\pgfpathlineto{\pgfqpoint{6.217505in}{3.112797in}}%
\pgfpathlineto{\pgfqpoint{6.218075in}{3.157080in}}%
\pgfpathlineto{\pgfqpoint{6.219170in}{3.112357in}}%
\pgfpathlineto{\pgfqpoint{6.218306in}{3.230213in}}%
\pgfpathlineto{\pgfqpoint{6.219200in}{3.123056in}}%
\pgfpathlineto{\pgfqpoint{6.219848in}{3.219717in}}%
\pgfpathlineto{\pgfqpoint{6.220187in}{3.107785in}}%
\pgfpathlineto{\pgfqpoint{6.220403in}{3.179741in}}%
\pgfpathlineto{\pgfqpoint{6.221220in}{3.112604in}}%
\pgfpathlineto{\pgfqpoint{6.220881in}{3.225415in}}%
\pgfpathlineto{\pgfqpoint{6.221497in}{3.174025in}}%
\pgfpathlineto{\pgfqpoint{6.221913in}{3.236719in}}%
\pgfpathlineto{\pgfqpoint{6.222253in}{3.105548in}}%
\pgfpathlineto{\pgfqpoint{6.222592in}{3.180219in}}%
\pgfpathlineto{\pgfqpoint{6.223285in}{3.099812in}}%
\pgfpathlineto{\pgfqpoint{6.223455in}{3.225875in}}%
\pgfpathlineto{\pgfqpoint{6.223748in}{3.129200in}}%
\pgfpathlineto{\pgfqpoint{6.224488in}{3.233748in}}%
\pgfpathlineto{\pgfqpoint{6.224303in}{3.101050in}}%
\pgfpathlineto{\pgfqpoint{6.225012in}{3.215998in}}%
\pgfpathlineto{\pgfqpoint{6.225336in}{3.102469in}}%
\pgfpathlineto{\pgfqpoint{6.225521in}{3.230640in}}%
\pgfpathlineto{\pgfqpoint{6.226137in}{3.172282in}}%
\pgfpathlineto{\pgfqpoint{6.226553in}{3.223631in}}%
\pgfpathlineto{\pgfqpoint{6.226368in}{3.104240in}}%
\pgfpathlineto{\pgfqpoint{6.227247in}{3.186217in}}%
\pgfpathlineto{\pgfqpoint{6.227386in}{3.107088in}}%
\pgfpathlineto{\pgfqpoint{6.227586in}{3.226331in}}%
\pgfpathlineto{\pgfqpoint{6.228419in}{3.118796in}}%
\pgfpathlineto{\pgfqpoint{6.228434in}{3.115080in}}%
\pgfpathlineto{\pgfqpoint{6.228604in}{3.220952in}}%
\pgfpathlineto{\pgfqpoint{6.229236in}{3.172241in}}%
\pgfpathlineto{\pgfqpoint{6.229652in}{3.224872in}}%
\pgfpathlineto{\pgfqpoint{6.229467in}{3.105174in}}%
\pgfpathlineto{\pgfqpoint{6.230345in}{3.184889in}}%
\pgfpathlineto{\pgfqpoint{6.231517in}{3.101431in}}%
\pgfpathlineto{\pgfqpoint{6.230669in}{3.240354in}}%
\pgfpathlineto{\pgfqpoint{6.231563in}{3.121854in}}%
\pgfpathlineto{\pgfqpoint{6.231702in}{3.244244in}}%
\pgfpathlineto{\pgfqpoint{6.232550in}{3.100922in}}%
\pgfpathlineto{\pgfqpoint{6.232781in}{3.194160in}}%
\pgfpathlineto{\pgfqpoint{6.233567in}{3.114136in}}%
\pgfpathlineto{\pgfqpoint{6.233752in}{3.253161in}}%
\pgfpathlineto{\pgfqpoint{6.233983in}{3.132546in}}%
\pgfpathlineto{\pgfqpoint{6.234615in}{3.122319in}}%
\pgfpathlineto{\pgfqpoint{6.234785in}{3.252168in}}%
\pgfpathlineto{\pgfqpoint{6.235078in}{3.132165in}}%
\pgfpathlineto{\pgfqpoint{6.235818in}{3.248764in}}%
\pgfpathlineto{\pgfqpoint{6.235633in}{3.120966in}}%
\pgfpathlineto{\pgfqpoint{6.236265in}{3.179829in}}%
\pgfpathlineto{\pgfqpoint{6.237112in}{3.108147in}}%
\pgfpathlineto{\pgfqpoint{6.236850in}{3.246768in}}%
\pgfpathlineto{\pgfqpoint{6.237344in}{3.185489in}}%
\pgfpathlineto{\pgfqpoint{6.237883in}{3.252324in}}%
\pgfpathlineto{\pgfqpoint{6.238130in}{3.103996in}}%
\pgfpathlineto{\pgfqpoint{6.238423in}{3.183389in}}%
\pgfpathlineto{\pgfqpoint{6.239163in}{3.114748in}}%
\pgfpathlineto{\pgfqpoint{6.238916in}{3.239418in}}%
\pgfpathlineto{\pgfqpoint{6.239533in}{3.179382in}}%
\pgfpathlineto{\pgfqpoint{6.239933in}{3.241231in}}%
\pgfpathlineto{\pgfqpoint{6.239764in}{3.117460in}}%
\pgfpathlineto{\pgfqpoint{6.240642in}{3.193977in}}%
\pgfpathlineto{\pgfqpoint{6.240966in}{3.238024in}}%
\pgfpathlineto{\pgfqpoint{6.241829in}{3.113442in}}%
\pgfpathlineto{\pgfqpoint{6.241999in}{3.234711in}}%
\pgfpathlineto{\pgfqpoint{6.242261in}{3.108317in}}%
\pgfpathlineto{\pgfqpoint{6.243063in}{3.198007in}}%
\pgfpathlineto{\pgfqpoint{6.243278in}{3.099552in}}%
\pgfpathlineto{\pgfqpoint{6.244065in}{3.233847in}}%
\pgfpathlineto{\pgfqpoint{6.244188in}{3.172098in}}%
\pgfpathlineto{\pgfqpoint{6.244311in}{3.105707in}}%
\pgfpathlineto{\pgfqpoint{6.244727in}{3.221750in}}%
\pgfpathlineto{\pgfqpoint{6.245067in}{3.217437in}}%
\pgfpathlineto{\pgfqpoint{6.245097in}{3.248353in}}%
\pgfpathlineto{\pgfqpoint{6.245945in}{3.109232in}}%
\pgfpathlineto{\pgfqpoint{6.246161in}{3.197883in}}%
\pgfpathlineto{\pgfqpoint{6.246978in}{3.102065in}}%
\pgfpathlineto{\pgfqpoint{6.247163in}{3.228423in}}%
\pgfpathlineto{\pgfqpoint{6.247286in}{3.173628in}}%
\pgfpathlineto{\pgfqpoint{6.247302in}{3.174447in}}%
\pgfpathlineto{\pgfqpoint{6.247379in}{3.125003in}}%
\pgfpathlineto{\pgfqpoint{6.248442in}{3.107949in}}%
\pgfpathlineto{\pgfqpoint{6.248196in}{3.230264in}}%
\pgfpathlineto{\pgfqpoint{6.248473in}{3.128942in}}%
\pgfpathlineto{\pgfqpoint{6.248720in}{3.227749in}}%
\pgfpathlineto{\pgfqpoint{6.249475in}{3.112230in}}%
\pgfpathlineto{\pgfqpoint{6.249660in}{3.196513in}}%
\pgfpathlineto{\pgfqpoint{6.250076in}{3.100013in}}%
\pgfpathlineto{\pgfqpoint{6.250246in}{3.232917in}}%
\pgfpathlineto{\pgfqpoint{6.250755in}{3.199518in}}%
\pgfpathlineto{\pgfqpoint{6.250770in}{3.206865in}}%
\pgfpathlineto{\pgfqpoint{6.251109in}{3.094494in}}%
\pgfpathlineto{\pgfqpoint{6.251803in}{3.173807in}}%
\pgfpathlineto{\pgfqpoint{6.252127in}{3.115591in}}%
\pgfpathlineto{\pgfqpoint{6.251942in}{3.201515in}}%
\pgfpathlineto{\pgfqpoint{6.252913in}{3.171263in}}%
\pgfpathlineto{\pgfqpoint{6.253406in}{3.244305in}}%
\pgfpathlineto{\pgfqpoint{6.253468in}{3.138048in}}%
\pgfpathlineto{\pgfqpoint{6.253976in}{3.160371in}}%
\pgfpathlineto{\pgfqpoint{6.254747in}{3.129055in}}%
\pgfpathlineto{\pgfqpoint{6.254516in}{3.229492in}}%
\pgfpathlineto{\pgfqpoint{6.255102in}{3.146324in}}%
\pgfpathlineto{\pgfqpoint{6.255487in}{3.244674in}}%
\pgfpathlineto{\pgfqpoint{6.255302in}{3.107805in}}%
\pgfpathlineto{\pgfqpoint{6.256227in}{3.176548in}}%
\pgfpathlineto{\pgfqpoint{6.256289in}{3.093366in}}%
\pgfpathlineto{\pgfqpoint{6.256504in}{3.243195in}}%
\pgfpathlineto{\pgfqpoint{6.257321in}{3.175794in}}%
\pgfpathlineto{\pgfqpoint{6.257522in}{3.244337in}}%
\pgfpathlineto{\pgfqpoint{6.257691in}{3.109059in}}%
\pgfpathlineto{\pgfqpoint{6.258508in}{3.232352in}}%
\pgfpathlineto{\pgfqpoint{6.258539in}{3.234709in}}%
\pgfpathlineto{\pgfqpoint{6.258585in}{3.229776in}}%
\pgfpathlineto{\pgfqpoint{6.258678in}{3.115024in}}%
\pgfpathlineto{\pgfqpoint{6.259587in}{3.258478in}}%
\pgfpathlineto{\pgfqpoint{6.259896in}{3.147473in}}%
\pgfpathlineto{\pgfqpoint{6.260574in}{3.239324in}}%
\pgfpathlineto{\pgfqpoint{6.260004in}{3.106675in}}%
\pgfpathlineto{\pgfqpoint{6.260959in}{3.142416in}}%
\pgfpathlineto{\pgfqpoint{6.261190in}{3.115775in}}%
\pgfpathlineto{\pgfqpoint{6.261545in}{3.230226in}}%
\pgfpathlineto{\pgfqpoint{6.262069in}{3.138724in}}%
\pgfpathlineto{\pgfqpoint{6.262362in}{3.229160in}}%
\pgfpathlineto{\pgfqpoint{6.262162in}{3.112778in}}%
\pgfpathlineto{\pgfqpoint{6.263133in}{3.113515in}}%
\pgfpathlineto{\pgfqpoint{6.263148in}{3.108471in}}%
\pgfpathlineto{\pgfqpoint{6.263349in}{3.234367in}}%
\pgfpathlineto{\pgfqpoint{6.264027in}{3.170657in}}%
\pgfpathlineto{\pgfqpoint{6.264381in}{3.230751in}}%
\pgfpathlineto{\pgfqpoint{6.264135in}{3.089248in}}%
\pgfpathlineto{\pgfqpoint{6.265075in}{3.164806in}}%
\pgfpathlineto{\pgfqpoint{6.265137in}{3.100632in}}%
\pgfpathlineto{\pgfqpoint{6.265615in}{3.229778in}}%
\pgfpathlineto{\pgfqpoint{6.266216in}{3.114082in}}%
\pgfpathlineto{\pgfqpoint{6.266601in}{3.225626in}}%
\pgfpathlineto{\pgfqpoint{6.267079in}{3.099772in}}%
\pgfpathlineto{\pgfqpoint{6.267495in}{3.160313in}}%
\pgfpathlineto{\pgfqpoint{6.268065in}{3.096644in}}%
\pgfpathlineto{\pgfqpoint{6.267665in}{3.218880in}}%
\pgfpathlineto{\pgfqpoint{6.268513in}{3.162419in}}%
\pgfpathlineto{\pgfqpoint{6.268559in}{3.231008in}}%
\pgfpathlineto{\pgfqpoint{6.269052in}{3.088178in}}%
\pgfpathlineto{\pgfqpoint{6.269638in}{3.198545in}}%
\pgfpathlineto{\pgfqpoint{6.270039in}{3.097582in}}%
\pgfpathlineto{\pgfqpoint{6.270224in}{3.218789in}}%
\pgfpathlineto{\pgfqpoint{6.270809in}{3.165455in}}%
\pgfpathlineto{\pgfqpoint{6.271226in}{3.212246in}}%
\pgfpathlineto{\pgfqpoint{6.271025in}{3.095706in}}%
\pgfpathlineto{\pgfqpoint{6.271950in}{3.186287in}}%
\pgfpathlineto{\pgfqpoint{6.272012in}{3.097904in}}%
\pgfpathlineto{\pgfqpoint{6.272844in}{3.213545in}}%
\pgfpathlineto{\pgfqpoint{6.273044in}{3.171971in}}%
\pgfpathlineto{\pgfqpoint{6.273908in}{3.219383in}}%
\pgfpathlineto{\pgfqpoint{6.273122in}{3.112497in}}%
\pgfpathlineto{\pgfqpoint{6.274154in}{3.185692in}}%
\pgfpathlineto{\pgfqpoint{6.274894in}{3.227024in}}%
\pgfpathlineto{\pgfqpoint{6.274971in}{3.116382in}}%
\pgfpathlineto{\pgfqpoint{6.275079in}{3.126791in}}%
\pgfpathlineto{\pgfqpoint{6.275095in}{3.112365in}}%
\pgfpathlineto{\pgfqpoint{6.275881in}{3.228700in}}%
\pgfpathlineto{\pgfqpoint{6.276127in}{3.183988in}}%
\pgfpathlineto{\pgfqpoint{6.277006in}{3.224369in}}%
\pgfpathlineto{\pgfqpoint{6.276713in}{3.123631in}}%
\pgfpathlineto{\pgfqpoint{6.277206in}{3.167947in}}%
\pgfpathlineto{\pgfqpoint{6.278054in}{3.125978in}}%
\pgfpathlineto{\pgfqpoint{6.277993in}{3.227341in}}%
\pgfpathlineto{\pgfqpoint{6.278255in}{3.188513in}}%
\pgfpathlineto{\pgfqpoint{6.278979in}{3.239370in}}%
\pgfpathlineto{\pgfqpoint{6.278347in}{3.112888in}}%
\pgfpathlineto{\pgfqpoint{6.279303in}{3.172551in}}%
\pgfpathlineto{\pgfqpoint{6.279349in}{3.111524in}}%
\pgfpathlineto{\pgfqpoint{6.279966in}{3.221835in}}%
\pgfpathlineto{\pgfqpoint{6.280413in}{3.167709in}}%
\pgfpathlineto{\pgfqpoint{6.280644in}{3.109994in}}%
\pgfpathlineto{\pgfqpoint{6.280952in}{3.227818in}}%
\pgfpathlineto{\pgfqpoint{6.281538in}{3.135137in}}%
\pgfpathlineto{\pgfqpoint{6.281939in}{3.232215in}}%
\pgfpathlineto{\pgfqpoint{6.281631in}{3.109855in}}%
\pgfpathlineto{\pgfqpoint{6.282602in}{3.120659in}}%
\pgfpathlineto{\pgfqpoint{6.282617in}{3.111262in}}%
\pgfpathlineto{\pgfqpoint{6.283342in}{3.217160in}}%
\pgfpathlineto{\pgfqpoint{6.283650in}{3.156187in}}%
\pgfpathlineto{\pgfqpoint{6.284313in}{3.223995in}}%
\pgfpathlineto{\pgfqpoint{6.284590in}{3.107582in}}%
\pgfpathlineto{\pgfqpoint{6.284760in}{3.164094in}}%
\pgfpathlineto{\pgfqpoint{6.285577in}{3.098806in}}%
\pgfpathlineto{\pgfqpoint{6.285176in}{3.232633in}}%
\pgfpathlineto{\pgfqpoint{6.285854in}{3.172667in}}%
\pgfpathlineto{\pgfqpoint{6.286162in}{3.236787in}}%
\pgfpathlineto{\pgfqpoint{6.286563in}{3.100324in}}%
\pgfpathlineto{\pgfqpoint{6.286979in}{3.214204in}}%
\pgfpathlineto{\pgfqpoint{6.287781in}{3.102249in}}%
\pgfpathlineto{\pgfqpoint{6.287272in}{3.238562in}}%
\pgfpathlineto{\pgfqpoint{6.288105in}{3.208870in}}%
\pgfpathlineto{\pgfqpoint{6.288259in}{3.242411in}}%
\pgfpathlineto{\pgfqpoint{6.288768in}{3.112775in}}%
\pgfpathlineto{\pgfqpoint{6.289184in}{3.172890in}}%
\pgfpathlineto{\pgfqpoint{6.289245in}{3.245385in}}%
\pgfpathlineto{\pgfqpoint{6.289739in}{3.103962in}}%
\pgfpathlineto{\pgfqpoint{6.290355in}{3.202528in}}%
\pgfpathlineto{\pgfqpoint{6.290741in}{3.087048in}}%
\pgfpathlineto{\pgfqpoint{6.291219in}{3.246841in}}%
\pgfpathlineto{\pgfqpoint{6.291511in}{3.138348in}}%
\pgfpathlineto{\pgfqpoint{6.292205in}{3.236744in}}%
\pgfpathlineto{\pgfqpoint{6.291727in}{3.094170in}}%
\pgfpathlineto{\pgfqpoint{6.292637in}{3.184839in}}%
\pgfpathlineto{\pgfqpoint{6.292714in}{3.100815in}}%
\pgfpathlineto{\pgfqpoint{6.293192in}{3.220955in}}%
\pgfpathlineto{\pgfqpoint{6.293731in}{3.154702in}}%
\pgfpathlineto{\pgfqpoint{6.294178in}{3.226538in}}%
\pgfpathlineto{\pgfqpoint{6.293824in}{3.102468in}}%
\pgfpathlineto{\pgfqpoint{6.294826in}{3.125457in}}%
\pgfpathlineto{\pgfqpoint{6.295165in}{3.214541in}}%
\pgfpathlineto{\pgfqpoint{6.295673in}{3.115742in}}%
\pgfpathlineto{\pgfqpoint{6.295951in}{3.155927in}}%
\pgfpathlineto{\pgfqpoint{6.296490in}{3.210422in}}%
\pgfpathlineto{\pgfqpoint{6.296783in}{3.114052in}}%
\pgfpathlineto{\pgfqpoint{6.297030in}{3.175952in}}%
\pgfpathlineto{\pgfqpoint{6.297770in}{3.110414in}}%
\pgfpathlineto{\pgfqpoint{6.297554in}{3.224054in}}%
\pgfpathlineto{\pgfqpoint{6.298140in}{3.176304in}}%
\pgfpathlineto{\pgfqpoint{6.298756in}{3.116101in}}%
\pgfpathlineto{\pgfqpoint{6.298541in}{3.233003in}}%
\pgfpathlineto{\pgfqpoint{6.299250in}{3.174353in}}%
\pgfpathlineto{\pgfqpoint{6.300020in}{3.126707in}}%
\pgfpathlineto{\pgfqpoint{6.299527in}{3.225644in}}%
\pgfpathlineto{\pgfqpoint{6.300375in}{3.152492in}}%
\pgfpathlineto{\pgfqpoint{6.300652in}{3.230311in}}%
\pgfpathlineto{\pgfqpoint{6.300714in}{3.122675in}}%
\pgfpathlineto{\pgfqpoint{6.301516in}{3.208621in}}%
\pgfpathlineto{\pgfqpoint{6.301701in}{3.125293in}}%
\pgfpathlineto{\pgfqpoint{6.301639in}{3.248437in}}%
\pgfpathlineto{\pgfqpoint{6.302610in}{3.223378in}}%
\pgfpathlineto{\pgfqpoint{6.302626in}{3.236414in}}%
\pgfpathlineto{\pgfqpoint{6.302965in}{3.131141in}}%
\pgfpathlineto{\pgfqpoint{6.303643in}{3.160532in}}%
\pgfpathlineto{\pgfqpoint{6.304522in}{3.124941in}}%
\pgfpathlineto{\pgfqpoint{6.304583in}{3.225172in}}%
\pgfpathlineto{\pgfqpoint{6.304707in}{3.201791in}}%
\pgfpathlineto{\pgfqpoint{6.305015in}{3.209778in}}%
\pgfpathlineto{\pgfqpoint{6.304938in}{3.131087in}}%
\pgfpathlineto{\pgfqpoint{6.305477in}{3.150464in}}%
\pgfpathlineto{\pgfqpoint{6.306495in}{3.118220in}}%
\pgfpathlineto{\pgfqpoint{6.305570in}{3.213647in}}%
\pgfpathlineto{\pgfqpoint{6.306525in}{3.169660in}}%
\pgfpathlineto{\pgfqpoint{6.306556in}{3.216007in}}%
\pgfpathlineto{\pgfqpoint{6.307481in}{3.114294in}}%
\pgfpathlineto{\pgfqpoint{6.307635in}{3.171849in}}%
\pgfpathlineto{\pgfqpoint{6.307820in}{3.229244in}}%
\pgfpathlineto{\pgfqpoint{6.308452in}{3.112979in}}%
\pgfpathlineto{\pgfqpoint{6.308730in}{3.169599in}}%
\pgfpathlineto{\pgfqpoint{6.308761in}{3.179592in}}%
\pgfpathlineto{\pgfqpoint{6.308807in}{3.229415in}}%
\pgfpathlineto{\pgfqpoint{6.309439in}{3.115618in}}%
\pgfpathlineto{\pgfqpoint{6.309855in}{3.158494in}}%
\pgfpathlineto{\pgfqpoint{6.310903in}{3.241611in}}%
\pgfpathlineto{\pgfqpoint{6.310425in}{3.116166in}}%
\pgfpathlineto{\pgfqpoint{6.310950in}{3.172833in}}%
\pgfpathlineto{\pgfqpoint{6.311397in}{3.124298in}}%
\pgfpathlineto{\pgfqpoint{6.311890in}{3.236982in}}%
\pgfpathlineto{\pgfqpoint{6.312059in}{3.154748in}}%
\pgfpathlineto{\pgfqpoint{6.312876in}{3.231366in}}%
\pgfpathlineto{\pgfqpoint{6.312383in}{3.115535in}}%
\pgfpathlineto{\pgfqpoint{6.313216in}{3.172790in}}%
\pgfpathlineto{\pgfqpoint{6.313370in}{3.112743in}}%
\pgfpathlineto{\pgfqpoint{6.313863in}{3.233456in}}%
\pgfpathlineto{\pgfqpoint{6.314233in}{3.176568in}}%
\pgfpathlineto{\pgfqpoint{6.314834in}{3.228406in}}%
\pgfpathlineto{\pgfqpoint{6.314356in}{3.125653in}}%
\pgfpathlineto{\pgfqpoint{6.315281in}{3.166049in}}%
\pgfpathlineto{\pgfqpoint{6.316314in}{3.117595in}}%
\pgfpathlineto{\pgfqpoint{6.315821in}{3.229950in}}%
\pgfpathlineto{\pgfqpoint{6.316360in}{3.197619in}}%
\pgfpathlineto{\pgfqpoint{6.316807in}{3.225102in}}%
\pgfpathlineto{\pgfqpoint{6.317300in}{3.117037in}}%
\pgfpathlineto{\pgfqpoint{6.317393in}{3.163936in}}%
\pgfpathlineto{\pgfqpoint{6.318272in}{3.120528in}}%
\pgfpathlineto{\pgfqpoint{6.317778in}{3.213692in}}%
\pgfpathlineto{\pgfqpoint{6.318472in}{3.184208in}}%
\pgfpathlineto{\pgfqpoint{6.318765in}{3.216430in}}%
\pgfpathlineto{\pgfqpoint{6.318688in}{3.136082in}}%
\pgfpathlineto{\pgfqpoint{6.319212in}{3.168244in}}%
\pgfpathlineto{\pgfqpoint{6.319258in}{3.113992in}}%
\pgfpathlineto{\pgfqpoint{6.319751in}{3.213580in}}%
\pgfpathlineto{\pgfqpoint{6.320291in}{3.196374in}}%
\pgfpathlineto{\pgfqpoint{6.320723in}{3.208381in}}%
\pgfpathlineto{\pgfqpoint{6.320645in}{3.133879in}}%
\pgfpathlineto{\pgfqpoint{6.321200in}{3.156504in}}%
\pgfpathlineto{\pgfqpoint{6.321632in}{3.132223in}}%
\pgfpathlineto{\pgfqpoint{6.322249in}{3.224011in}}%
\pgfpathlineto{\pgfqpoint{6.322264in}{3.230272in}}%
\pgfpathlineto{\pgfqpoint{6.322619in}{3.133794in}}%
\pgfpathlineto{\pgfqpoint{6.323158in}{3.155189in}}%
\pgfpathlineto{\pgfqpoint{6.324160in}{3.124602in}}%
\pgfpathlineto{\pgfqpoint{6.323251in}{3.234629in}}%
\pgfpathlineto{\pgfqpoint{6.324206in}{3.209085in}}%
\pgfpathlineto{\pgfqpoint{6.324222in}{3.224614in}}%
\pgfpathlineto{\pgfqpoint{6.325147in}{3.119280in}}%
\pgfpathlineto{\pgfqpoint{6.325316in}{3.212623in}}%
\pgfpathlineto{\pgfqpoint{6.325640in}{3.217348in}}%
\pgfpathlineto{\pgfqpoint{6.325563in}{3.127562in}}%
\pgfpathlineto{\pgfqpoint{6.326087in}{3.152882in}}%
\pgfpathlineto{\pgfqpoint{6.327104in}{3.119717in}}%
\pgfpathlineto{\pgfqpoint{6.326334in}{3.221628in}}%
\pgfpathlineto{\pgfqpoint{6.327135in}{3.178995in}}%
\pgfpathlineto{\pgfqpoint{6.327166in}{3.226467in}}%
\pgfpathlineto{\pgfqpoint{6.328075in}{3.115632in}}%
\pgfpathlineto{\pgfqpoint{6.328245in}{3.198179in}}%
\pgfpathlineto{\pgfqpoint{6.329124in}{3.227583in}}%
\pgfpathlineto{\pgfqpoint{6.329062in}{3.111322in}}%
\pgfpathlineto{\pgfqpoint{6.329324in}{3.189672in}}%
\pgfpathlineto{\pgfqpoint{6.330033in}{3.118727in}}%
\pgfpathlineto{\pgfqpoint{6.330110in}{3.220142in}}%
\pgfpathlineto{\pgfqpoint{6.330465in}{3.150202in}}%
\pgfpathlineto{\pgfqpoint{6.331528in}{3.229313in}}%
\pgfpathlineto{\pgfqpoint{6.331020in}{3.112056in}}%
\pgfpathlineto{\pgfqpoint{6.331590in}{3.162675in}}%
\pgfpathlineto{\pgfqpoint{6.332068in}{3.224757in}}%
\pgfpathlineto{\pgfqpoint{6.332006in}{3.112026in}}%
\pgfpathlineto{\pgfqpoint{6.332392in}{3.146947in}}%
\pgfpathlineto{\pgfqpoint{6.332977in}{3.106557in}}%
\pgfpathlineto{\pgfqpoint{6.332499in}{3.224947in}}%
\pgfpathlineto{\pgfqpoint{6.333440in}{3.168590in}}%
\pgfpathlineto{\pgfqpoint{6.334026in}{3.220974in}}%
\pgfpathlineto{\pgfqpoint{6.333964in}{3.101078in}}%
\pgfpathlineto{\pgfqpoint{6.334550in}{3.175635in}}%
\pgfpathlineto{\pgfqpoint{6.334950in}{3.103331in}}%
\pgfpathlineto{\pgfqpoint{6.335012in}{3.216079in}}%
\pgfpathlineto{\pgfqpoint{6.335721in}{3.145906in}}%
\pgfpathlineto{\pgfqpoint{6.335983in}{3.224474in}}%
\pgfpathlineto{\pgfqpoint{6.335922in}{3.114561in}}%
\pgfpathlineto{\pgfqpoint{6.336846in}{3.179022in}}%
\pgfpathlineto{\pgfqpoint{6.336908in}{3.111145in}}%
\pgfpathlineto{\pgfqpoint{6.336970in}{3.233282in}}%
\pgfpathlineto{\pgfqpoint{6.337925in}{3.199733in}}%
\pgfpathlineto{\pgfqpoint{6.338927in}{3.230105in}}%
\pgfpathlineto{\pgfqpoint{6.338866in}{3.118609in}}%
\pgfpathlineto{\pgfqpoint{6.339035in}{3.204536in}}%
\pgfpathlineto{\pgfqpoint{6.339297in}{3.112066in}}%
\pgfpathlineto{\pgfqpoint{6.339914in}{3.227340in}}%
\pgfpathlineto{\pgfqpoint{6.340084in}{3.218922in}}%
\pgfpathlineto{\pgfqpoint{6.340762in}{3.224855in}}%
\pgfpathlineto{\pgfqpoint{6.340269in}{3.103442in}}%
\pgfpathlineto{\pgfqpoint{6.340808in}{3.159073in}}%
\pgfpathlineto{\pgfqpoint{6.341255in}{3.117260in}}%
\pgfpathlineto{\pgfqpoint{6.341748in}{3.240573in}}%
\pgfpathlineto{\pgfqpoint{6.341872in}{3.227755in}}%
\pgfpathlineto{\pgfqpoint{6.342550in}{3.107793in}}%
\pgfpathlineto{\pgfqpoint{6.342858in}{3.244073in}}%
\pgfpathlineto{\pgfqpoint{6.343012in}{3.190584in}}%
\pgfpathlineto{\pgfqpoint{6.343845in}{3.257204in}}%
\pgfpathlineto{\pgfqpoint{6.343105in}{3.106021in}}%
\pgfpathlineto{\pgfqpoint{6.344045in}{3.184762in}}%
\pgfpathlineto{\pgfqpoint{6.344107in}{3.091807in}}%
\pgfpathlineto{\pgfqpoint{6.344831in}{3.258437in}}%
\pgfpathlineto{\pgfqpoint{6.345170in}{3.111080in}}%
\pgfpathlineto{\pgfqpoint{6.345818in}{3.217520in}}%
\pgfpathlineto{\pgfqpoint{6.346142in}{3.110994in}}%
\pgfpathlineto{\pgfqpoint{6.346327in}{3.165584in}}%
\pgfpathlineto{\pgfqpoint{6.347051in}{3.106058in}}%
\pgfpathlineto{\pgfqpoint{6.346804in}{3.217537in}}%
\pgfpathlineto{\pgfqpoint{6.347452in}{3.149437in}}%
\pgfpathlineto{\pgfqpoint{6.348022in}{3.115568in}}%
\pgfpathlineto{\pgfqpoint{6.347776in}{3.208574in}}%
\pgfpathlineto{\pgfqpoint{6.348223in}{3.182812in}}%
\pgfpathlineto{\pgfqpoint{6.348762in}{3.222978in}}%
\pgfpathlineto{\pgfqpoint{6.349009in}{3.120984in}}%
\pgfpathlineto{\pgfqpoint{6.349286in}{3.175903in}}%
\pgfpathlineto{\pgfqpoint{6.349995in}{3.145486in}}%
\pgfpathlineto{\pgfqpoint{6.349749in}{3.210751in}}%
\pgfpathlineto{\pgfqpoint{6.350411in}{3.163832in}}%
\pgfpathlineto{\pgfqpoint{6.350966in}{3.137077in}}%
\pgfpathlineto{\pgfqpoint{6.351198in}{3.197064in}}%
\pgfpathlineto{\pgfqpoint{6.351475in}{3.168190in}}%
\pgfpathlineto{\pgfqpoint{6.352184in}{3.205167in}}%
\pgfpathlineto{\pgfqpoint{6.351953in}{3.113023in}}%
\pgfpathlineto{\pgfqpoint{6.352600in}{3.185729in}}%
\pgfpathlineto{\pgfqpoint{6.352893in}{3.091887in}}%
\pgfpathlineto{\pgfqpoint{6.352677in}{3.210949in}}%
\pgfpathlineto{\pgfqpoint{6.353726in}{3.167730in}}%
\pgfpathlineto{\pgfqpoint{6.354805in}{3.218394in}}%
\pgfpathlineto{\pgfqpoint{6.353911in}{3.126077in}}%
\pgfpathlineto{\pgfqpoint{6.354820in}{3.201129in}}%
\pgfpathlineto{\pgfqpoint{6.354882in}{3.097873in}}%
\pgfpathlineto{\pgfqpoint{6.355144in}{3.230385in}}%
\pgfpathlineto{\pgfqpoint{6.355945in}{3.161698in}}%
\pgfpathlineto{\pgfqpoint{6.356978in}{3.228545in}}%
\pgfpathlineto{\pgfqpoint{6.356716in}{3.099010in}}%
\pgfpathlineto{\pgfqpoint{6.357071in}{3.182637in}}%
\pgfpathlineto{\pgfqpoint{6.357764in}{3.092398in}}%
\pgfpathlineto{\pgfqpoint{6.357425in}{3.258745in}}%
\pgfpathlineto{\pgfqpoint{6.358273in}{3.152995in}}%
\pgfpathlineto{\pgfqpoint{6.358504in}{3.243677in}}%
\pgfpathlineto{\pgfqpoint{6.358797in}{3.080544in}}%
\pgfpathlineto{\pgfqpoint{6.359445in}{3.195319in}}%
\pgfpathlineto{\pgfqpoint{6.359799in}{3.095479in}}%
\pgfpathlineto{\pgfqpoint{6.359568in}{3.230101in}}%
\pgfpathlineto{\pgfqpoint{6.360554in}{3.174909in}}%
\pgfpathlineto{\pgfqpoint{6.360909in}{3.268351in}}%
\pgfpathlineto{\pgfqpoint{6.361202in}{3.077176in}}%
\pgfpathlineto{\pgfqpoint{6.361664in}{3.183955in}}%
\pgfpathlineto{\pgfqpoint{6.362389in}{3.072829in}}%
\pgfpathlineto{\pgfqpoint{6.361988in}{3.249692in}}%
\pgfpathlineto{\pgfqpoint{6.362728in}{3.230461in}}%
\pgfpathlineto{\pgfqpoint{6.362836in}{3.263720in}}%
\pgfpathlineto{\pgfqpoint{6.363437in}{3.117059in}}%
\pgfpathlineto{\pgfqpoint{6.363730in}{3.145407in}}%
\pgfpathlineto{\pgfqpoint{6.364655in}{3.025244in}}%
\pgfpathlineto{\pgfqpoint{6.364223in}{3.262937in}}%
\pgfpathlineto{\pgfqpoint{6.364840in}{3.122936in}}%
\pgfpathlineto{\pgfqpoint{6.364901in}{3.096424in}}%
\pgfpathlineto{\pgfqpoint{6.365194in}{3.244896in}}%
\pgfpathlineto{\pgfqpoint{6.365241in}{3.319468in}}%
\pgfpathlineto{\pgfqpoint{6.365857in}{3.071862in}}%
\pgfpathlineto{\pgfqpoint{6.366320in}{3.283260in}}%
\pgfpathlineto{\pgfqpoint{6.367244in}{3.077429in}}%
\pgfpathlineto{\pgfqpoint{6.367460in}{3.205029in}}%
\pgfpathlineto{\pgfqpoint{6.367815in}{3.350012in}}%
\pgfpathlineto{\pgfqpoint{6.368077in}{3.118109in}}%
\pgfpathlineto{\pgfqpoint{6.368139in}{2.998345in}}%
\pgfpathlineto{\pgfqpoint{6.368724in}{3.331580in}}%
\pgfpathlineto{\pgfqpoint{6.369187in}{3.097523in}}%
\pgfpathlineto{\pgfqpoint{6.369757in}{3.309001in}}%
\pgfpathlineto{\pgfqpoint{6.369295in}{3.083349in}}%
\pgfpathlineto{\pgfqpoint{6.370451in}{3.239621in}}%
\pgfpathlineto{\pgfqpoint{6.371622in}{3.028298in}}%
\pgfpathlineto{\pgfqpoint{6.371299in}{3.323131in}}%
\pgfpathlineto{\pgfqpoint{6.371638in}{3.035350in}}%
\pgfpathlineto{\pgfqpoint{6.372177in}{3.304783in}}%
\pgfpathlineto{\pgfqpoint{6.372809in}{3.131722in}}%
\pgfpathlineto{\pgfqpoint{6.373904in}{3.259731in}}%
\pgfpathlineto{\pgfqpoint{6.373318in}{3.100796in}}%
\pgfpathlineto{\pgfqpoint{6.373981in}{3.178198in}}%
\pgfpathlineto{\pgfqpoint{6.373996in}{3.178403in}}%
\pgfpathlineto{\pgfqpoint{6.374027in}{3.164652in}}%
\pgfpathlineto{\pgfqpoint{6.374181in}{3.021837in}}%
\pgfpathlineto{\pgfqpoint{6.374767in}{3.328723in}}%
\pgfpathlineto{\pgfqpoint{6.375260in}{3.038488in}}%
\pgfpathlineto{\pgfqpoint{6.375646in}{3.301598in}}%
\pgfpathlineto{\pgfqpoint{6.376755in}{3.176088in}}%
\pgfpathlineto{\pgfqpoint{6.377665in}{3.043430in}}%
\pgfpathlineto{\pgfqpoint{6.377218in}{3.233411in}}%
\pgfpathlineto{\pgfqpoint{6.377865in}{3.153647in}}%
\pgfpathlineto{\pgfqpoint{6.378251in}{3.318034in}}%
\pgfpathlineto{\pgfqpoint{6.378713in}{3.051524in}}%
\pgfpathlineto{\pgfqpoint{6.378729in}{3.048369in}}%
\pgfpathlineto{\pgfqpoint{6.378991in}{3.217424in}}%
\pgfpathlineto{\pgfqpoint{6.379098in}{3.290158in}}%
\pgfpathlineto{\pgfqpoint{6.379623in}{3.098592in}}%
\pgfpathlineto{\pgfqpoint{6.380085in}{3.207246in}}%
\pgfpathlineto{\pgfqpoint{6.381179in}{3.031412in}}%
\pgfpathlineto{\pgfqpoint{6.380717in}{3.250925in}}%
\pgfpathlineto{\pgfqpoint{6.381272in}{3.082815in}}%
\pgfpathlineto{\pgfqpoint{6.382212in}{3.045274in}}%
\pgfpathlineto{\pgfqpoint{6.381734in}{3.311104in}}%
\pgfpathlineto{\pgfqpoint{6.382289in}{3.107484in}}%
\pgfpathlineto{\pgfqpoint{6.382582in}{3.298200in}}%
\pgfpathlineto{\pgfqpoint{6.383384in}{3.105331in}}%
\pgfpathlineto{\pgfqpoint{6.383415in}{3.123795in}}%
\pgfpathlineto{\pgfqpoint{6.384185in}{3.255110in}}%
\pgfpathlineto{\pgfqpoint{6.383769in}{3.103866in}}%
\pgfpathlineto{\pgfqpoint{6.384525in}{3.140311in}}%
\pgfpathlineto{\pgfqpoint{6.384602in}{3.045735in}}%
\pgfpathlineto{\pgfqpoint{6.385203in}{3.315015in}}%
\pgfpathlineto{\pgfqpoint{6.385696in}{3.057160in}}%
\pgfpathlineto{\pgfqpoint{6.385958in}{3.205823in}}%
\pgfpathlineto{\pgfqpoint{6.386051in}{3.285207in}}%
\pgfpathlineto{\pgfqpoint{6.386590in}{3.108153in}}%
\pgfpathlineto{\pgfqpoint{6.387053in}{3.187936in}}%
\pgfpathlineto{\pgfqpoint{6.388070in}{3.030305in}}%
\pgfpathlineto{\pgfqpoint{6.387669in}{3.261416in}}%
\pgfpathlineto{\pgfqpoint{6.388193in}{3.114547in}}%
\pgfpathlineto{\pgfqpoint{6.388255in}{3.094564in}}%
\pgfpathlineto{\pgfqpoint{6.388347in}{3.200129in}}%
\pgfpathlineto{\pgfqpoint{6.389519in}{3.299711in}}%
\pgfpathlineto{\pgfqpoint{6.389180in}{3.033217in}}%
\pgfpathlineto{\pgfqpoint{6.389534in}{3.298649in}}%
\pgfpathlineto{\pgfqpoint{6.390058in}{3.090068in}}%
\pgfpathlineto{\pgfqpoint{6.390953in}{3.163575in}}%
\pgfpathlineto{\pgfqpoint{6.392170in}{3.298618in}}%
\pgfpathlineto{\pgfqpoint{6.391569in}{3.051775in}}%
\pgfpathlineto{\pgfqpoint{6.392201in}{3.287339in}}%
\pgfpathlineto{\pgfqpoint{6.392648in}{3.021877in}}%
\pgfpathlineto{\pgfqpoint{6.393018in}{3.298295in}}%
\pgfpathlineto{\pgfqpoint{6.393465in}{3.159946in}}%
\pgfpathlineto{\pgfqpoint{6.393650in}{3.096525in}}%
\pgfpathlineto{\pgfqpoint{6.394113in}{3.252611in}}%
\pgfpathlineto{\pgfqpoint{6.394421in}{3.175494in}}%
\pgfpathlineto{\pgfqpoint{6.395531in}{3.278018in}}%
\pgfpathlineto{\pgfqpoint{6.395053in}{3.035248in}}%
\pgfpathlineto{\pgfqpoint{6.395562in}{3.229645in}}%
\pgfpathlineto{\pgfqpoint{6.395916in}{3.012608in}}%
\pgfpathlineto{\pgfqpoint{6.396533in}{3.311727in}}%
\pgfpathlineto{\pgfqpoint{6.396671in}{3.196937in}}%
\pgfpathlineto{\pgfqpoint{6.397442in}{3.258514in}}%
\pgfpathlineto{\pgfqpoint{6.397134in}{3.089837in}}%
\pgfpathlineto{\pgfqpoint{6.397735in}{3.113298in}}%
\pgfpathlineto{\pgfqpoint{6.398521in}{3.018026in}}%
\pgfpathlineto{\pgfqpoint{6.398105in}{3.277138in}}%
\pgfpathlineto{\pgfqpoint{6.398752in}{3.167259in}}%
\pgfpathlineto{\pgfqpoint{6.399600in}{3.022845in}}%
\pgfpathlineto{\pgfqpoint{6.399970in}{3.301447in}}%
\pgfpathlineto{\pgfqpoint{6.400016in}{3.317583in}}%
\pgfpathlineto{\pgfqpoint{6.400618in}{3.087795in}}%
\pgfpathlineto{\pgfqpoint{6.400941in}{3.227413in}}%
\pgfpathlineto{\pgfqpoint{6.402005in}{3.031069in}}%
\pgfpathlineto{\pgfqpoint{6.401065in}{3.277097in}}%
\pgfpathlineto{\pgfqpoint{6.402128in}{3.112764in}}%
\pgfpathlineto{\pgfqpoint{6.402606in}{3.289857in}}%
\pgfpathlineto{\pgfqpoint{6.402884in}{3.014668in}}%
\pgfpathlineto{\pgfqpoint{6.403038in}{3.039081in}}%
\pgfpathlineto{\pgfqpoint{6.403069in}{3.022254in}}%
\pgfpathlineto{\pgfqpoint{6.403485in}{3.322898in}}%
\pgfpathlineto{\pgfqpoint{6.404055in}{3.113518in}}%
\pgfpathlineto{\pgfqpoint{6.404071in}{3.103404in}}%
\pgfpathlineto{\pgfqpoint{6.404548in}{3.263206in}}%
\pgfpathlineto{\pgfqpoint{6.405011in}{3.209699in}}%
\pgfpathlineto{\pgfqpoint{6.406090in}{3.319175in}}%
\pgfpathlineto{\pgfqpoint{6.405489in}{3.000454in}}%
\pgfpathlineto{\pgfqpoint{6.406152in}{3.269298in}}%
\pgfpathlineto{\pgfqpoint{6.406552in}{2.999881in}}%
\pgfpathlineto{\pgfqpoint{6.406953in}{3.326334in}}%
\pgfpathlineto{\pgfqpoint{6.407508in}{3.107757in}}%
\pgfpathlineto{\pgfqpoint{6.407554in}{3.079353in}}%
\pgfpathlineto{\pgfqpoint{6.408032in}{3.273614in}}%
\pgfpathlineto{\pgfqpoint{6.408464in}{3.161799in}}%
\pgfpathlineto{\pgfqpoint{6.409574in}{3.293783in}}%
\pgfpathlineto{\pgfqpoint{6.408957in}{3.011128in}}%
\pgfpathlineto{\pgfqpoint{6.409635in}{3.240580in}}%
\pgfpathlineto{\pgfqpoint{6.409836in}{3.011021in}}%
\pgfpathlineto{\pgfqpoint{6.410437in}{3.310170in}}%
\pgfpathlineto{\pgfqpoint{6.410838in}{3.134188in}}%
\pgfpathlineto{\pgfqpoint{6.412025in}{3.268920in}}%
\pgfpathlineto{\pgfqpoint{6.410915in}{3.114487in}}%
\pgfpathlineto{\pgfqpoint{6.412256in}{3.166318in}}%
\pgfpathlineto{\pgfqpoint{6.412441in}{3.002309in}}%
\pgfpathlineto{\pgfqpoint{6.413057in}{3.312254in}}%
\pgfpathlineto{\pgfqpoint{6.413427in}{3.034866in}}%
\pgfpathlineto{\pgfqpoint{6.413921in}{3.328646in}}%
\pgfpathlineto{\pgfqpoint{6.414876in}{3.192448in}}%
\pgfpathlineto{\pgfqpoint{6.415909in}{3.045369in}}%
\pgfpathlineto{\pgfqpoint{6.414969in}{3.271033in}}%
\pgfpathlineto{\pgfqpoint{6.416094in}{3.130618in}}%
\pgfpathlineto{\pgfqpoint{6.416110in}{3.130041in}}%
\pgfpathlineto{\pgfqpoint{6.416156in}{3.166077in}}%
\pgfpathlineto{\pgfqpoint{6.416510in}{3.293376in}}%
\pgfpathlineto{\pgfqpoint{6.416788in}{3.028616in}}%
\pgfpathlineto{\pgfqpoint{6.417296in}{3.226456in}}%
\pgfpathlineto{\pgfqpoint{6.417666in}{3.067385in}}%
\pgfpathlineto{\pgfqpoint{6.417404in}{3.294415in}}%
\pgfpathlineto{\pgfqpoint{6.418422in}{3.209515in}}%
\pgfpathlineto{\pgfqpoint{6.418992in}{3.254582in}}%
\pgfpathlineto{\pgfqpoint{6.419393in}{3.023921in}}%
\pgfpathlineto{\pgfqpoint{6.419439in}{3.066086in}}%
\pgfpathlineto{\pgfqpoint{6.420009in}{3.299390in}}%
\pgfpathlineto{\pgfqpoint{6.420272in}{3.018273in}}%
\pgfpathlineto{\pgfqpoint{6.420595in}{3.101436in}}%
\pgfpathlineto{\pgfqpoint{6.420611in}{3.100207in}}%
\pgfpathlineto{\pgfqpoint{6.420672in}{3.182037in}}%
\pgfpathlineto{\pgfqpoint{6.420888in}{3.320043in}}%
\pgfpathlineto{\pgfqpoint{6.421150in}{3.077665in}}%
\pgfpathlineto{\pgfqpoint{6.421813in}{3.228733in}}%
\pgfpathlineto{\pgfqpoint{6.422877in}{3.042431in}}%
\pgfpathlineto{\pgfqpoint{6.421936in}{3.250594in}}%
\pgfpathlineto{\pgfqpoint{6.423000in}{3.129703in}}%
\pgfpathlineto{\pgfqpoint{6.423493in}{3.278512in}}%
\pgfpathlineto{\pgfqpoint{6.423755in}{3.061288in}}%
\pgfpathlineto{\pgfqpoint{6.424156in}{3.185401in}}%
\pgfpathlineto{\pgfqpoint{6.424372in}{3.313264in}}%
\pgfpathlineto{\pgfqpoint{6.424634in}{3.079720in}}%
\pgfpathlineto{\pgfqpoint{6.425297in}{3.220026in}}%
\pgfpathlineto{\pgfqpoint{6.426360in}{3.025595in}}%
\pgfpathlineto{\pgfqpoint{6.425975in}{3.262119in}}%
\pgfpathlineto{\pgfqpoint{6.426499in}{3.110972in}}%
\pgfpathlineto{\pgfqpoint{6.426545in}{3.100877in}}%
\pgfpathlineto{\pgfqpoint{6.426592in}{3.141297in}}%
\pgfpathlineto{\pgfqpoint{6.426977in}{3.309838in}}%
\pgfpathlineto{\pgfqpoint{6.427239in}{3.041388in}}%
\pgfpathlineto{\pgfqpoint{6.427748in}{3.228434in}}%
\pgfpathlineto{\pgfqpoint{6.427779in}{3.250512in}}%
\pgfpathlineto{\pgfqpoint{6.427840in}{3.326800in}}%
\pgfpathlineto{\pgfqpoint{6.428102in}{3.037207in}}%
\pgfpathlineto{\pgfqpoint{6.428873in}{3.233716in}}%
\pgfpathlineto{\pgfqpoint{6.429582in}{3.269856in}}%
\pgfpathlineto{\pgfqpoint{6.428965in}{3.075757in}}%
\pgfpathlineto{\pgfqpoint{6.429782in}{3.141814in}}%
\pgfpathlineto{\pgfqpoint{6.429829in}{3.023552in}}%
\pgfpathlineto{\pgfqpoint{6.430445in}{3.285675in}}%
\pgfpathlineto{\pgfqpoint{6.430908in}{3.067403in}}%
\pgfpathlineto{\pgfqpoint{6.431324in}{3.317985in}}%
\pgfpathlineto{\pgfqpoint{6.431571in}{3.064914in}}%
\pgfpathlineto{\pgfqpoint{6.432264in}{3.194727in}}%
\pgfpathlineto{\pgfqpoint{6.433312in}{3.015149in}}%
\pgfpathlineto{\pgfqpoint{6.433050in}{3.266241in}}%
\pgfpathlineto{\pgfqpoint{6.433420in}{3.116553in}}%
\pgfpathlineto{\pgfqpoint{6.433929in}{3.290081in}}%
\pgfpathlineto{\pgfqpoint{6.434376in}{3.047860in}}%
\pgfpathlineto{\pgfqpoint{6.434623in}{3.247604in}}%
\pgfpathlineto{\pgfqpoint{6.434792in}{3.329969in}}%
\pgfpathlineto{\pgfqpoint{6.435008in}{3.103622in}}%
\pgfpathlineto{\pgfqpoint{6.435039in}{3.044297in}}%
\pgfpathlineto{\pgfqpoint{6.435671in}{3.278337in}}%
\pgfpathlineto{\pgfqpoint{6.436118in}{3.085977in}}%
\pgfpathlineto{\pgfqpoint{6.437397in}{3.298908in}}%
\pgfpathlineto{\pgfqpoint{6.436781in}{3.028253in}}%
\pgfpathlineto{\pgfqpoint{6.437428in}{3.279173in}}%
\pgfpathlineto{\pgfqpoint{6.437844in}{3.039370in}}%
\pgfpathlineto{\pgfqpoint{6.438276in}{3.317422in}}%
\pgfpathlineto{\pgfqpoint{6.438646in}{3.128581in}}%
\pgfpathlineto{\pgfqpoint{6.440002in}{3.291783in}}%
\pgfpathlineto{\pgfqpoint{6.439370in}{3.079079in}}%
\pgfpathlineto{\pgfqpoint{6.440049in}{3.241027in}}%
\pgfpathlineto{\pgfqpoint{6.440249in}{3.018608in}}%
\pgfpathlineto{\pgfqpoint{6.440866in}{3.295111in}}%
\pgfpathlineto{\pgfqpoint{6.441236in}{3.086327in}}%
\pgfpathlineto{\pgfqpoint{6.441744in}{3.327532in}}%
\pgfpathlineto{\pgfqpoint{6.441991in}{3.057296in}}%
\pgfpathlineto{\pgfqpoint{6.442469in}{3.208936in}}%
\pgfpathlineto{\pgfqpoint{6.442854in}{3.052958in}}%
\pgfpathlineto{\pgfqpoint{6.442623in}{3.287511in}}%
\pgfpathlineto{\pgfqpoint{6.443440in}{3.273930in}}%
\pgfpathlineto{\pgfqpoint{6.444349in}{3.307646in}}%
\pgfpathlineto{\pgfqpoint{6.443733in}{3.014844in}}%
\pgfpathlineto{\pgfqpoint{6.444519in}{3.241430in}}%
\pgfpathlineto{\pgfqpoint{6.444797in}{3.052027in}}%
\pgfpathlineto{\pgfqpoint{6.445228in}{3.315860in}}%
\pgfpathlineto{\pgfqpoint{6.445675in}{3.110127in}}%
\pgfpathlineto{\pgfqpoint{6.446939in}{3.311808in}}%
\pgfpathlineto{\pgfqpoint{6.446338in}{3.067385in}}%
\pgfpathlineto{\pgfqpoint{6.446985in}{3.274933in}}%
\pgfpathlineto{\pgfqpoint{6.447201in}{3.047666in}}%
\pgfpathlineto{\pgfqpoint{6.447818in}{3.297059in}}%
\pgfpathlineto{\pgfqpoint{6.448172in}{3.082849in}}%
\pgfpathlineto{\pgfqpoint{6.448712in}{3.283882in}}%
\pgfpathlineto{\pgfqpoint{6.448974in}{3.044707in}}%
\pgfpathlineto{\pgfqpoint{6.449406in}{3.228405in}}%
\pgfpathlineto{\pgfqpoint{6.449837in}{3.068375in}}%
\pgfpathlineto{\pgfqpoint{6.449544in}{3.282208in}}%
\pgfpathlineto{\pgfqpoint{6.450377in}{3.270686in}}%
\pgfpathlineto{\pgfqpoint{6.451286in}{3.302102in}}%
\pgfpathlineto{\pgfqpoint{6.450716in}{3.036570in}}%
\pgfpathlineto{\pgfqpoint{6.451332in}{3.217536in}}%
\pgfpathlineto{\pgfqpoint{6.451641in}{3.059833in}}%
\pgfpathlineto{\pgfqpoint{6.452134in}{3.269685in}}%
\pgfpathlineto{\pgfqpoint{6.452489in}{3.104158in}}%
\pgfpathlineto{\pgfqpoint{6.453352in}{3.076525in}}%
\pgfpathlineto{\pgfqpoint{6.453074in}{3.243327in}}%
\pgfpathlineto{\pgfqpoint{6.453413in}{3.142492in}}%
\pgfpathlineto{\pgfqpoint{6.453691in}{3.234139in}}%
\pgfpathlineto{\pgfqpoint{6.454169in}{3.083301in}}%
\pgfpathlineto{\pgfqpoint{6.454570in}{3.198202in}}%
\pgfpathlineto{\pgfqpoint{6.455541in}{3.245563in}}%
\pgfpathlineto{\pgfqpoint{6.455047in}{3.085661in}}%
\pgfpathlineto{\pgfqpoint{6.455695in}{3.215602in}}%
\pgfpathlineto{\pgfqpoint{6.455895in}{3.086814in}}%
\pgfpathlineto{\pgfqpoint{6.456512in}{3.232569in}}%
\pgfpathlineto{\pgfqpoint{6.456835in}{3.141442in}}%
\pgfpathlineto{\pgfqpoint{6.457776in}{3.231597in}}%
\pgfpathlineto{\pgfqpoint{6.457560in}{3.104780in}}%
\pgfpathlineto{\pgfqpoint{6.457961in}{3.191939in}}%
\pgfpathlineto{\pgfqpoint{6.458377in}{3.115062in}}%
\pgfpathlineto{\pgfqpoint{6.458161in}{3.241984in}}%
\pgfpathlineto{\pgfqpoint{6.459055in}{3.217390in}}%
\pgfpathlineto{\pgfqpoint{6.459718in}{3.250221in}}%
\pgfpathlineto{\pgfqpoint{6.459502in}{3.109747in}}%
\pgfpathlineto{\pgfqpoint{6.460073in}{3.158108in}}%
\pgfpathlineto{\pgfqpoint{6.460088in}{3.157756in}}%
\pgfpathlineto{\pgfqpoint{6.460103in}{3.176416in}}%
\pgfpathlineto{\pgfqpoint{6.460766in}{3.255429in}}%
\pgfpathlineto{\pgfqpoint{6.460458in}{3.116315in}}%
\pgfpathlineto{\pgfqpoint{6.461182in}{3.162793in}}%
\pgfpathlineto{\pgfqpoint{6.462015in}{3.108082in}}%
\pgfpathlineto{\pgfqpoint{6.461275in}{3.226997in}}%
\pgfpathlineto{\pgfqpoint{6.462262in}{3.166820in}}%
\pgfpathlineto{\pgfqpoint{6.463125in}{3.238373in}}%
\pgfpathlineto{\pgfqpoint{6.462770in}{3.110355in}}%
\pgfpathlineto{\pgfqpoint{6.463387in}{3.196243in}}%
\pgfpathlineto{\pgfqpoint{6.464481in}{3.095565in}}%
\pgfpathlineto{\pgfqpoint{6.464296in}{3.222218in}}%
\pgfpathlineto{\pgfqpoint{6.464527in}{3.127537in}}%
\pgfpathlineto{\pgfqpoint{6.464589in}{3.096515in}}%
\pgfpathlineto{\pgfqpoint{6.464697in}{3.167204in}}%
\pgfpathlineto{\pgfqpoint{6.464759in}{3.160507in}}%
\pgfpathlineto{\pgfqpoint{6.464913in}{3.248000in}}%
\pgfpathlineto{\pgfqpoint{6.465637in}{3.108068in}}%
\pgfpathlineto{\pgfqpoint{6.465869in}{3.166505in}}%
\pgfpathlineto{\pgfqpoint{6.465946in}{3.260569in}}%
\pgfpathlineto{\pgfqpoint{6.466146in}{3.104561in}}%
\pgfpathlineto{\pgfqpoint{6.466655in}{3.119763in}}%
\pgfpathlineto{\pgfqpoint{6.466686in}{3.092244in}}%
\pgfpathlineto{\pgfqpoint{6.467271in}{3.233152in}}%
\pgfpathlineto{\pgfqpoint{6.467734in}{3.135705in}}%
\pgfpathlineto{\pgfqpoint{6.468304in}{3.233775in}}%
\pgfpathlineto{\pgfqpoint{6.468736in}{3.082790in}}%
\pgfpathlineto{\pgfqpoint{6.468828in}{3.124781in}}%
\pgfpathlineto{\pgfqpoint{6.469059in}{3.241369in}}%
\pgfpathlineto{\pgfqpoint{6.469260in}{3.099580in}}%
\pgfpathlineto{\pgfqpoint{6.469738in}{3.102782in}}%
\pgfpathlineto{\pgfqpoint{6.469769in}{3.083870in}}%
\pgfpathlineto{\pgfqpoint{6.470092in}{3.252661in}}%
\pgfpathlineto{\pgfqpoint{6.470817in}{3.115038in}}%
\pgfpathlineto{\pgfqpoint{6.471140in}{3.247734in}}%
\pgfpathlineto{\pgfqpoint{6.471325in}{3.076384in}}%
\pgfpathlineto{\pgfqpoint{6.471942in}{3.161060in}}%
\pgfpathlineto{\pgfqpoint{6.472358in}{3.079165in}}%
\pgfpathlineto{\pgfqpoint{6.472142in}{3.241197in}}%
\pgfpathlineto{\pgfqpoint{6.473036in}{3.143618in}}%
\pgfpathlineto{\pgfqpoint{6.473900in}{3.070173in}}%
\pgfpathlineto{\pgfqpoint{6.474208in}{3.255419in}}%
\pgfpathlineto{\pgfqpoint{6.474932in}{3.075666in}}%
\pgfpathlineto{\pgfqpoint{6.476012in}{3.166676in}}%
\pgfpathlineto{\pgfqpoint{6.476397in}{3.239681in}}%
\pgfpathlineto{\pgfqpoint{6.476474in}{3.088615in}}%
\pgfpathlineto{\pgfqpoint{6.477091in}{3.166816in}}%
\pgfpathlineto{\pgfqpoint{6.477507in}{3.070118in}}%
\pgfpathlineto{\pgfqpoint{6.477306in}{3.229404in}}%
\pgfpathlineto{\pgfqpoint{6.478216in}{3.146288in}}%
\pgfpathlineto{\pgfqpoint{6.478540in}{3.086117in}}%
\pgfpathlineto{\pgfqpoint{6.478324in}{3.237890in}}%
\pgfpathlineto{\pgfqpoint{6.479295in}{3.148121in}}%
\pgfpathlineto{\pgfqpoint{6.480389in}{3.285328in}}%
\pgfpathlineto{\pgfqpoint{6.480081in}{3.077563in}}%
\pgfpathlineto{\pgfqpoint{6.480466in}{3.231805in}}%
\pgfpathlineto{\pgfqpoint{6.480513in}{3.250539in}}%
\pgfpathlineto{\pgfqpoint{6.480605in}{3.095455in}}%
\pgfpathlineto{\pgfqpoint{6.481068in}{3.128789in}}%
\pgfpathlineto{\pgfqpoint{6.481114in}{3.097673in}}%
\pgfpathlineto{\pgfqpoint{6.481422in}{3.249573in}}%
\pgfpathlineto{\pgfqpoint{6.482162in}{3.108920in}}%
\pgfpathlineto{\pgfqpoint{6.482470in}{3.244194in}}%
\pgfpathlineto{\pgfqpoint{6.483179in}{3.075303in}}%
\pgfpathlineto{\pgfqpoint{6.483272in}{3.147002in}}%
\pgfpathlineto{\pgfqpoint{6.484197in}{3.075985in}}%
\pgfpathlineto{\pgfqpoint{6.483889in}{3.241915in}}%
\pgfpathlineto{\pgfqpoint{6.484382in}{3.131971in}}%
\pgfpathlineto{\pgfqpoint{6.484921in}{3.236386in}}%
\pgfpathlineto{\pgfqpoint{6.485230in}{3.081735in}}%
\pgfpathlineto{\pgfqpoint{6.485553in}{3.224723in}}%
\pgfpathlineto{\pgfqpoint{6.485661in}{3.245178in}}%
\pgfpathlineto{\pgfqpoint{6.486262in}{3.084841in}}%
\pgfpathlineto{\pgfqpoint{6.486632in}{3.210786in}}%
\pgfpathlineto{\pgfqpoint{6.487295in}{3.078642in}}%
\pgfpathlineto{\pgfqpoint{6.487603in}{3.235986in}}%
\pgfpathlineto{\pgfqpoint{6.487696in}{3.212504in}}%
\pgfpathlineto{\pgfqpoint{6.488636in}{3.249738in}}%
\pgfpathlineto{\pgfqpoint{6.488328in}{3.089693in}}%
\pgfpathlineto{\pgfqpoint{6.488790in}{3.191286in}}%
\pgfpathlineto{\pgfqpoint{6.489361in}{3.068424in}}%
\pgfpathlineto{\pgfqpoint{6.489669in}{3.254516in}}%
\pgfpathlineto{\pgfqpoint{6.489900in}{3.140458in}}%
\pgfpathlineto{\pgfqpoint{6.490702in}{3.262870in}}%
\pgfpathlineto{\pgfqpoint{6.490394in}{3.067957in}}%
\pgfpathlineto{\pgfqpoint{6.491010in}{3.142303in}}%
\pgfpathlineto{\pgfqpoint{6.491426in}{3.088620in}}%
\pgfpathlineto{\pgfqpoint{6.491735in}{3.269201in}}%
\pgfpathlineto{\pgfqpoint{6.492089in}{3.163811in}}%
\pgfpathlineto{\pgfqpoint{6.492767in}{3.249839in}}%
\pgfpathlineto{\pgfqpoint{6.492968in}{3.083889in}}%
\pgfpathlineto{\pgfqpoint{6.493199in}{3.181947in}}%
\pgfpathlineto{\pgfqpoint{6.494001in}{3.089230in}}%
\pgfpathlineto{\pgfqpoint{6.493800in}{3.258958in}}%
\pgfpathlineto{\pgfqpoint{6.494155in}{3.185569in}}%
\pgfpathlineto{\pgfqpoint{6.494833in}{3.264392in}}%
\pgfpathlineto{\pgfqpoint{6.494509in}{3.092703in}}%
\pgfpathlineto{\pgfqpoint{6.495249in}{3.186391in}}%
\pgfpathlineto{\pgfqpoint{6.495434in}{3.106288in}}%
\pgfpathlineto{\pgfqpoint{6.495866in}{3.261709in}}%
\pgfpathlineto{\pgfqpoint{6.496359in}{3.181796in}}%
\pgfpathlineto{\pgfqpoint{6.496899in}{3.254841in}}%
\pgfpathlineto{\pgfqpoint{6.497084in}{3.084716in}}%
\pgfpathlineto{\pgfqpoint{6.497454in}{3.166760in}}%
\pgfpathlineto{\pgfqpoint{6.498116in}{3.081273in}}%
\pgfpathlineto{\pgfqpoint{6.497916in}{3.238859in}}%
\pgfpathlineto{\pgfqpoint{6.498563in}{3.154943in}}%
\pgfpathlineto{\pgfqpoint{6.498949in}{3.247737in}}%
\pgfpathlineto{\pgfqpoint{6.498779in}{3.125351in}}%
\pgfpathlineto{\pgfqpoint{6.499118in}{3.133002in}}%
\pgfpathlineto{\pgfqpoint{6.499149in}{3.088906in}}%
\pgfpathlineto{\pgfqpoint{6.499997in}{3.271139in}}%
\pgfpathlineto{\pgfqpoint{6.500213in}{3.163676in}}%
\pgfpathlineto{\pgfqpoint{6.501014in}{3.265441in}}%
\pgfpathlineto{\pgfqpoint{6.500691in}{3.106325in}}%
\pgfpathlineto{\pgfqpoint{6.501307in}{3.155504in}}%
\pgfpathlineto{\pgfqpoint{6.502232in}{3.101263in}}%
\pgfpathlineto{\pgfqpoint{6.502047in}{3.257297in}}%
\pgfpathlineto{\pgfqpoint{6.502402in}{3.164348in}}%
\pgfpathlineto{\pgfqpoint{6.503080in}{3.238617in}}%
\pgfpathlineto{\pgfqpoint{6.503265in}{3.082745in}}%
\pgfpathlineto{\pgfqpoint{6.503527in}{3.212627in}}%
\pgfpathlineto{\pgfqpoint{6.503789in}{3.088509in}}%
\pgfpathlineto{\pgfqpoint{6.504498in}{3.222454in}}%
\pgfpathlineto{\pgfqpoint{6.504544in}{3.219396in}}%
\pgfpathlineto{\pgfqpoint{6.505238in}{3.241316in}}%
\pgfpathlineto{\pgfqpoint{6.504822in}{3.080533in}}%
\pgfpathlineto{\pgfqpoint{6.505608in}{3.201541in}}%
\pgfpathlineto{\pgfqpoint{6.505855in}{3.087213in}}%
\pgfpathlineto{\pgfqpoint{6.506148in}{3.245502in}}%
\pgfpathlineto{\pgfqpoint{6.506163in}{3.253920in}}%
\pgfpathlineto{\pgfqpoint{6.506749in}{3.098878in}}%
\pgfpathlineto{\pgfqpoint{6.507088in}{3.177711in}}%
\pgfpathlineto{\pgfqpoint{6.507396in}{3.081372in}}%
\pgfpathlineto{\pgfqpoint{6.507196in}{3.268743in}}%
\pgfpathlineto{\pgfqpoint{6.508167in}{3.172981in}}%
\pgfpathlineto{\pgfqpoint{6.508229in}{3.263968in}}%
\pgfpathlineto{\pgfqpoint{6.508414in}{3.077514in}}%
\pgfpathlineto{\pgfqpoint{6.509292in}{3.219649in}}%
\pgfpathlineto{\pgfqpoint{6.510109in}{3.099586in}}%
\pgfpathlineto{\pgfqpoint{6.510279in}{3.269096in}}%
\pgfpathlineto{\pgfqpoint{6.510371in}{3.230325in}}%
\pgfpathlineto{\pgfqpoint{6.511311in}{3.275018in}}%
\pgfpathlineto{\pgfqpoint{6.511003in}{3.083202in}}%
\pgfpathlineto{\pgfqpoint{6.511450in}{3.230015in}}%
\pgfpathlineto{\pgfqpoint{6.511913in}{3.086500in}}%
\pgfpathlineto{\pgfqpoint{6.512344in}{3.280441in}}%
\pgfpathlineto{\pgfqpoint{6.512576in}{3.179866in}}%
\pgfpathlineto{\pgfqpoint{6.513377in}{3.273473in}}%
\pgfpathlineto{\pgfqpoint{6.513562in}{3.057972in}}%
\pgfpathlineto{\pgfqpoint{6.513655in}{3.161854in}}%
\pgfpathlineto{\pgfqpoint{6.514595in}{3.071790in}}%
\pgfpathlineto{\pgfqpoint{6.514394in}{3.241053in}}%
\pgfpathlineto{\pgfqpoint{6.514749in}{3.165248in}}%
\pgfpathlineto{\pgfqpoint{6.515427in}{3.236270in}}%
\pgfpathlineto{\pgfqpoint{6.515612in}{3.087254in}}%
\pgfpathlineto{\pgfqpoint{6.515874in}{3.210010in}}%
\pgfpathlineto{\pgfqpoint{6.517046in}{3.079299in}}%
\pgfpathlineto{\pgfqpoint{6.516460in}{3.278324in}}%
\pgfpathlineto{\pgfqpoint{6.517292in}{3.137369in}}%
\pgfpathlineto{\pgfqpoint{6.517431in}{3.200989in}}%
\pgfpathlineto{\pgfqpoint{6.517477in}{3.298763in}}%
\pgfpathlineto{\pgfqpoint{6.517678in}{3.080256in}}%
\pgfpathlineto{\pgfqpoint{6.518541in}{3.257871in}}%
\pgfpathlineto{\pgfqpoint{6.518711in}{3.058845in}}%
\pgfpathlineto{\pgfqpoint{6.519543in}{3.263362in}}%
\pgfpathlineto{\pgfqpoint{6.519651in}{3.236980in}}%
\pgfpathlineto{\pgfqpoint{6.519666in}{3.240133in}}%
\pgfpathlineto{\pgfqpoint{6.519728in}{3.073283in}}%
\pgfpathlineto{\pgfqpoint{6.520237in}{3.131021in}}%
\pgfpathlineto{\pgfqpoint{6.520761in}{3.084845in}}%
\pgfpathlineto{\pgfqpoint{6.520576in}{3.249653in}}%
\pgfpathlineto{\pgfqpoint{6.521393in}{3.102812in}}%
\pgfpathlineto{\pgfqpoint{6.521408in}{3.100105in}}%
\pgfpathlineto{\pgfqpoint{6.521578in}{3.240989in}}%
\pgfpathlineto{\pgfqpoint{6.522626in}{3.290529in}}%
\pgfpathlineto{\pgfqpoint{6.522302in}{3.092052in}}%
\pgfpathlineto{\pgfqpoint{6.522672in}{3.223448in}}%
\pgfpathlineto{\pgfqpoint{6.522826in}{3.099815in}}%
\pgfpathlineto{\pgfqpoint{6.523659in}{3.284725in}}%
\pgfpathlineto{\pgfqpoint{6.523751in}{3.235373in}}%
\pgfpathlineto{\pgfqpoint{6.524676in}{3.262336in}}%
\pgfpathlineto{\pgfqpoint{6.523844in}{3.067016in}}%
\pgfpathlineto{\pgfqpoint{6.524815in}{3.225721in}}%
\pgfpathlineto{\pgfqpoint{6.524877in}{3.064147in}}%
\pgfpathlineto{\pgfqpoint{6.525709in}{3.256061in}}%
\pgfpathlineto{\pgfqpoint{6.525940in}{3.150951in}}%
\pgfpathlineto{\pgfqpoint{6.526742in}{3.272630in}}%
\pgfpathlineto{\pgfqpoint{6.526418in}{3.090417in}}%
\pgfpathlineto{\pgfqpoint{6.527035in}{3.166728in}}%
\pgfpathlineto{\pgfqpoint{6.527451in}{3.081941in}}%
\pgfpathlineto{\pgfqpoint{6.527775in}{3.296933in}}%
\pgfpathlineto{\pgfqpoint{6.528129in}{3.164745in}}%
\pgfpathlineto{\pgfqpoint{6.528792in}{3.284822in}}%
\pgfpathlineto{\pgfqpoint{6.528468in}{3.094637in}}%
\pgfpathlineto{\pgfqpoint{6.529239in}{3.188078in}}%
\pgfpathlineto{\pgfqpoint{6.530025in}{3.084633in}}%
\pgfpathlineto{\pgfqpoint{6.529825in}{3.273089in}}%
\pgfpathlineto{\pgfqpoint{6.530303in}{3.201054in}}%
\pgfpathlineto{\pgfqpoint{6.530857in}{3.259676in}}%
\pgfpathlineto{\pgfqpoint{6.531042in}{3.080115in}}%
\pgfpathlineto{\pgfqpoint{6.531397in}{3.213900in}}%
\pgfpathlineto{\pgfqpoint{6.532075in}{3.098335in}}%
\pgfpathlineto{\pgfqpoint{6.531890in}{3.261343in}}%
\pgfpathlineto{\pgfqpoint{6.532522in}{3.155111in}}%
\pgfpathlineto{\pgfqpoint{6.532599in}{3.091517in}}%
\pgfpathlineto{\pgfqpoint{6.532877in}{3.203409in}}%
\pgfpathlineto{\pgfqpoint{6.533940in}{3.288840in}}%
\pgfpathlineto{\pgfqpoint{6.533617in}{3.092122in}}%
\pgfpathlineto{\pgfqpoint{6.533987in}{3.218450in}}%
\pgfpathlineto{\pgfqpoint{6.534264in}{3.091028in}}%
\pgfpathlineto{\pgfqpoint{6.534973in}{3.285170in}}%
\pgfpathlineto{\pgfqpoint{6.535066in}{3.226153in}}%
\pgfpathlineto{\pgfqpoint{6.535991in}{3.266911in}}%
\pgfpathlineto{\pgfqpoint{6.535158in}{3.076310in}}%
\pgfpathlineto{\pgfqpoint{6.536129in}{3.227424in}}%
\pgfpathlineto{\pgfqpoint{6.536191in}{3.075020in}}%
\pgfpathlineto{\pgfqpoint{6.537023in}{3.263690in}}%
\pgfpathlineto{\pgfqpoint{6.537255in}{3.166927in}}%
\pgfpathlineto{\pgfqpoint{6.538056in}{3.263760in}}%
\pgfpathlineto{\pgfqpoint{6.537733in}{3.096901in}}%
\pgfpathlineto{\pgfqpoint{6.538349in}{3.160911in}}%
\pgfpathlineto{\pgfqpoint{6.539274in}{3.091990in}}%
\pgfpathlineto{\pgfqpoint{6.539089in}{3.274389in}}%
\pgfpathlineto{\pgfqpoint{6.539444in}{3.185479in}}%
\pgfpathlineto{\pgfqpoint{6.540106in}{3.280297in}}%
\pgfpathlineto{\pgfqpoint{6.540307in}{3.080507in}}%
\pgfpathlineto{\pgfqpoint{6.540553in}{3.197672in}}%
\pgfpathlineto{\pgfqpoint{6.541324in}{3.075251in}}%
\pgfpathlineto{\pgfqpoint{6.541139in}{3.277625in}}%
\pgfpathlineto{\pgfqpoint{6.541632in}{3.202838in}}%
\pgfpathlineto{\pgfqpoint{6.542172in}{3.264454in}}%
\pgfpathlineto{\pgfqpoint{6.542357in}{3.080684in}}%
\pgfpathlineto{\pgfqpoint{6.542712in}{3.200400in}}%
\pgfpathlineto{\pgfqpoint{6.543390in}{3.090168in}}%
\pgfpathlineto{\pgfqpoint{6.543205in}{3.270197in}}%
\pgfpathlineto{\pgfqpoint{6.543852in}{3.137392in}}%
\pgfpathlineto{\pgfqpoint{6.544423in}{3.076286in}}%
\pgfpathlineto{\pgfqpoint{6.544222in}{3.281927in}}%
\pgfpathlineto{\pgfqpoint{6.544962in}{3.125758in}}%
\pgfpathlineto{\pgfqpoint{6.545255in}{3.288229in}}%
\pgfpathlineto{\pgfqpoint{6.545455in}{3.069630in}}%
\pgfpathlineto{\pgfqpoint{6.546072in}{3.141918in}}%
\pgfpathlineto{\pgfqpoint{6.546473in}{3.074888in}}%
\pgfpathlineto{\pgfqpoint{6.546288in}{3.273977in}}%
\pgfpathlineto{\pgfqpoint{6.547151in}{3.139385in}}%
\pgfpathlineto{\pgfqpoint{6.547321in}{3.269672in}}%
\pgfpathlineto{\pgfqpoint{6.547506in}{3.070028in}}%
\pgfpathlineto{\pgfqpoint{6.548245in}{3.144583in}}%
\pgfpathlineto{\pgfqpoint{6.548538in}{3.069715in}}%
\pgfpathlineto{\pgfqpoint{6.548353in}{3.271298in}}%
\pgfpathlineto{\pgfqpoint{6.549324in}{3.166890in}}%
\pgfpathlineto{\pgfqpoint{6.550419in}{3.286170in}}%
\pgfpathlineto{\pgfqpoint{6.549571in}{3.055125in}}%
\pgfpathlineto{\pgfqpoint{6.550450in}{3.242087in}}%
\pgfpathlineto{\pgfqpoint{6.550604in}{3.055835in}}%
\pgfpathlineto{\pgfqpoint{6.551452in}{3.279185in}}%
\pgfpathlineto{\pgfqpoint{6.551560in}{3.242769in}}%
\pgfpathlineto{\pgfqpoint{6.551637in}{3.059651in}}%
\pgfpathlineto{\pgfqpoint{6.552485in}{3.274251in}}%
\pgfpathlineto{\pgfqpoint{6.552824in}{3.156237in}}%
\pgfpathlineto{\pgfqpoint{6.553517in}{3.287639in}}%
\pgfpathlineto{\pgfqpoint{6.553718in}{3.055093in}}%
\pgfpathlineto{\pgfqpoint{6.553949in}{3.205582in}}%
\pgfpathlineto{\pgfqpoint{6.554750in}{3.058197in}}%
\pgfpathlineto{\pgfqpoint{6.554550in}{3.298912in}}%
\pgfpathlineto{\pgfqpoint{6.555043in}{3.194491in}}%
\pgfpathlineto{\pgfqpoint{6.555583in}{3.286888in}}%
\pgfpathlineto{\pgfqpoint{6.555151in}{3.105374in}}%
\pgfpathlineto{\pgfqpoint{6.556107in}{3.168820in}}%
\pgfpathlineto{\pgfqpoint{6.556184in}{3.129694in}}%
\pgfpathlineto{\pgfqpoint{6.556631in}{3.253508in}}%
\pgfpathlineto{\pgfqpoint{6.557248in}{3.144253in}}%
\pgfpathlineto{\pgfqpoint{6.557648in}{3.224303in}}%
\pgfpathlineto{\pgfqpoint{6.558080in}{3.104646in}}%
\pgfpathlineto{\pgfqpoint{6.558373in}{3.164748in}}%
\pgfpathlineto{\pgfqpoint{6.559067in}{3.096233in}}%
\pgfpathlineto{\pgfqpoint{6.558805in}{3.236277in}}%
\pgfpathlineto{\pgfqpoint{6.559437in}{3.187887in}}%
\pgfpathlineto{\pgfqpoint{6.559776in}{3.244096in}}%
\pgfpathlineto{\pgfqpoint{6.559961in}{3.074126in}}%
\pgfpathlineto{\pgfqpoint{6.560546in}{3.198346in}}%
\pgfpathlineto{\pgfqpoint{6.560762in}{3.234464in}}%
\pgfpathlineto{\pgfqpoint{6.560932in}{3.119426in}}%
\pgfpathlineto{\pgfqpoint{6.560993in}{3.062777in}}%
\pgfpathlineto{\pgfqpoint{6.561533in}{3.220502in}}%
\pgfpathlineto{\pgfqpoint{6.562042in}{3.117327in}}%
\pgfpathlineto{\pgfqpoint{6.562689in}{3.234949in}}%
\pgfpathlineto{\pgfqpoint{6.562951in}{3.089565in}}%
\pgfpathlineto{\pgfqpoint{6.563198in}{3.175351in}}%
\pgfpathlineto{\pgfqpoint{6.563907in}{3.114851in}}%
\pgfpathlineto{\pgfqpoint{6.563660in}{3.226691in}}%
\pgfpathlineto{\pgfqpoint{6.564277in}{3.181721in}}%
\pgfpathlineto{\pgfqpoint{6.564631in}{3.236889in}}%
\pgfpathlineto{\pgfqpoint{6.564863in}{3.101471in}}%
\pgfpathlineto{\pgfqpoint{6.565418in}{3.209545in}}%
\pgfpathlineto{\pgfqpoint{6.565849in}{3.094144in}}%
\pgfpathlineto{\pgfqpoint{6.565603in}{3.235564in}}%
\pgfpathlineto{\pgfqpoint{6.566527in}{3.183134in}}%
\pgfpathlineto{\pgfqpoint{6.566574in}{3.233564in}}%
\pgfpathlineto{\pgfqpoint{6.566820in}{3.102322in}}%
\pgfpathlineto{\pgfqpoint{6.567637in}{3.184598in}}%
\pgfpathlineto{\pgfqpoint{6.567791in}{3.098792in}}%
\pgfpathlineto{\pgfqpoint{6.568516in}{3.220248in}}%
\pgfpathlineto{\pgfqpoint{6.568809in}{3.133726in}}%
\pgfpathlineto{\pgfqpoint{6.569487in}{3.221802in}}%
\pgfpathlineto{\pgfqpoint{6.569749in}{3.119372in}}%
\pgfpathlineto{\pgfqpoint{6.569950in}{3.185769in}}%
\pgfpathlineto{\pgfqpoint{6.570736in}{3.118381in}}%
\pgfpathlineto{\pgfqpoint{6.570474in}{3.233191in}}%
\pgfpathlineto{\pgfqpoint{6.571136in}{3.161926in}}%
\pgfpathlineto{\pgfqpoint{6.571152in}{3.160888in}}%
\pgfpathlineto{\pgfqpoint{6.571414in}{3.214375in}}%
\pgfpathlineto{\pgfqpoint{6.571445in}{3.225276in}}%
\pgfpathlineto{\pgfqpoint{6.571707in}{3.110236in}}%
\pgfpathlineto{\pgfqpoint{6.572462in}{3.190692in}}%
\pgfpathlineto{\pgfqpoint{6.572693in}{3.089474in}}%
\pgfpathlineto{\pgfqpoint{6.573402in}{3.204859in}}%
\pgfpathlineto{\pgfqpoint{6.573680in}{3.101132in}}%
\pgfpathlineto{\pgfqpoint{6.574836in}{3.249921in}}%
\pgfpathlineto{\pgfqpoint{6.574882in}{3.204603in}}%
\pgfpathlineto{\pgfqpoint{6.575298in}{3.117772in}}%
\pgfpathlineto{\pgfqpoint{6.575823in}{3.233483in}}%
\pgfpathlineto{\pgfqpoint{6.576054in}{3.162021in}}%
\pgfpathlineto{\pgfqpoint{6.577164in}{3.220793in}}%
\pgfpathlineto{\pgfqpoint{6.576254in}{3.122695in}}%
\pgfpathlineto{\pgfqpoint{6.577194in}{3.180886in}}%
\pgfpathlineto{\pgfqpoint{6.577626in}{3.128618in}}%
\pgfpathlineto{\pgfqpoint{6.578135in}{3.234209in}}%
\pgfpathlineto{\pgfqpoint{6.578304in}{3.170722in}}%
\pgfpathlineto{\pgfqpoint{6.579121in}{3.248468in}}%
\pgfpathlineto{\pgfqpoint{6.578613in}{3.116391in}}%
\pgfpathlineto{\pgfqpoint{6.579383in}{3.176984in}}%
\pgfpathlineto{\pgfqpoint{6.580478in}{3.103008in}}%
\pgfpathlineto{\pgfqpoint{6.580077in}{3.263206in}}%
\pgfpathlineto{\pgfqpoint{6.580570in}{3.122933in}}%
\pgfpathlineto{\pgfqpoint{6.581064in}{3.278484in}}%
\pgfpathlineto{\pgfqpoint{6.581557in}{3.104612in}}%
\pgfpathlineto{\pgfqpoint{6.581757in}{3.195636in}}%
\pgfpathlineto{\pgfqpoint{6.582050in}{3.269259in}}%
\pgfpathlineto{\pgfqpoint{6.581896in}{3.119020in}}%
\pgfpathlineto{\pgfqpoint{6.582775in}{3.177182in}}%
\pgfpathlineto{\pgfqpoint{6.582883in}{3.100188in}}%
\pgfpathlineto{\pgfqpoint{6.583037in}{3.257699in}}%
\pgfpathlineto{\pgfqpoint{6.583915in}{3.151475in}}%
\pgfpathlineto{\pgfqpoint{6.585025in}{3.263557in}}%
\pgfpathlineto{\pgfqpoint{6.584856in}{3.082862in}}%
\pgfpathlineto{\pgfqpoint{6.585071in}{3.245210in}}%
\pgfpathlineto{\pgfqpoint{6.585858in}{3.085442in}}%
\pgfpathlineto{\pgfqpoint{6.586027in}{3.267209in}}%
\pgfpathlineto{\pgfqpoint{6.586228in}{3.195642in}}%
\pgfpathlineto{\pgfqpoint{6.586243in}{3.197174in}}%
\pgfpathlineto{\pgfqpoint{6.586520in}{3.123686in}}%
\pgfpathlineto{\pgfqpoint{6.586783in}{3.141185in}}%
\pgfpathlineto{\pgfqpoint{6.586844in}{3.101241in}}%
\pgfpathlineto{\pgfqpoint{6.586998in}{3.267875in}}%
\pgfpathlineto{\pgfqpoint{6.587862in}{3.142212in}}%
\pgfpathlineto{\pgfqpoint{6.588000in}{3.265916in}}%
\pgfpathlineto{\pgfqpoint{6.588817in}{3.120383in}}%
\pgfpathlineto{\pgfqpoint{6.589002in}{3.242545in}}%
\pgfpathlineto{\pgfqpoint{6.589819in}{3.108462in}}%
\pgfpathlineto{\pgfqpoint{6.590174in}{3.198352in}}%
\pgfpathlineto{\pgfqpoint{6.590282in}{3.215287in}}%
\pgfpathlineto{\pgfqpoint{6.590775in}{3.113597in}}%
\pgfpathlineto{\pgfqpoint{6.591792in}{3.091800in}}%
\pgfpathlineto{\pgfqpoint{6.591237in}{3.218868in}}%
\pgfpathlineto{\pgfqpoint{6.591823in}{3.139859in}}%
\pgfpathlineto{\pgfqpoint{6.592255in}{3.226443in}}%
\pgfpathlineto{\pgfqpoint{6.592763in}{3.109389in}}%
\pgfpathlineto{\pgfqpoint{6.592964in}{3.208371in}}%
\pgfpathlineto{\pgfqpoint{6.593750in}{3.108853in}}%
\pgfpathlineto{\pgfqpoint{6.593210in}{3.224245in}}%
\pgfpathlineto{\pgfqpoint{6.594089in}{3.188942in}}%
\pgfpathlineto{\pgfqpoint{6.594182in}{3.218013in}}%
\pgfpathlineto{\pgfqpoint{6.594752in}{3.113401in}}%
\pgfpathlineto{\pgfqpoint{6.595230in}{3.202884in}}%
\pgfpathlineto{\pgfqpoint{6.595739in}{3.127744in}}%
\pgfpathlineto{\pgfqpoint{6.595261in}{3.203398in}}%
\pgfpathlineto{\pgfqpoint{6.596432in}{3.193480in}}%
\pgfpathlineto{\pgfqpoint{6.596448in}{3.201954in}}%
\pgfpathlineto{\pgfqpoint{6.596956in}{3.118548in}}%
\pgfpathlineto{\pgfqpoint{6.597527in}{3.184867in}}%
\pgfpathlineto{\pgfqpoint{6.598097in}{3.210332in}}%
\pgfpathlineto{\pgfqpoint{6.597943in}{3.122693in}}%
\pgfpathlineto{\pgfqpoint{6.598452in}{3.160489in}}%
\pgfpathlineto{\pgfqpoint{6.598929in}{3.106397in}}%
\pgfpathlineto{\pgfqpoint{6.599068in}{3.204285in}}%
\pgfpathlineto{\pgfqpoint{6.599376in}{3.179679in}}%
\pgfpathlineto{\pgfqpoint{6.600378in}{3.201699in}}%
\pgfpathlineto{\pgfqpoint{6.599916in}{3.108840in}}%
\pgfpathlineto{\pgfqpoint{6.600486in}{3.183075in}}%
\pgfpathlineto{\pgfqpoint{6.600918in}{3.113089in}}%
\pgfpathlineto{\pgfqpoint{6.600610in}{3.209345in}}%
\pgfpathlineto{\pgfqpoint{6.601565in}{3.195067in}}%
\pgfpathlineto{\pgfqpoint{6.602644in}{3.219061in}}%
\pgfpathlineto{\pgfqpoint{6.601904in}{3.119256in}}%
\pgfpathlineto{\pgfqpoint{6.602660in}{3.209839in}}%
\pgfpathlineto{\pgfqpoint{6.603138in}{3.107156in}}%
\pgfpathlineto{\pgfqpoint{6.603615in}{3.211361in}}%
\pgfpathlineto{\pgfqpoint{6.603816in}{3.153047in}}%
\pgfpathlineto{\pgfqpoint{6.604109in}{3.102809in}}%
\pgfpathlineto{\pgfqpoint{6.603939in}{3.200548in}}%
\pgfpathlineto{\pgfqpoint{6.604463in}{3.177011in}}%
\pgfpathlineto{\pgfqpoint{6.605558in}{3.228309in}}%
\pgfpathlineto{\pgfqpoint{6.605080in}{3.102554in}}%
\pgfpathlineto{\pgfqpoint{6.605604in}{3.207411in}}%
\pgfpathlineto{\pgfqpoint{6.606097in}{3.095678in}}%
\pgfpathlineto{\pgfqpoint{6.606560in}{3.218526in}}%
\pgfpathlineto{\pgfqpoint{6.606729in}{3.187172in}}%
\pgfpathlineto{\pgfqpoint{6.607099in}{3.088998in}}%
\pgfpathlineto{\pgfqpoint{6.607639in}{3.226580in}}%
\pgfpathlineto{\pgfqpoint{6.607777in}{3.188913in}}%
\pgfpathlineto{\pgfqpoint{6.607808in}{3.224967in}}%
\pgfpathlineto{\pgfqpoint{6.608055in}{3.082591in}}%
\pgfpathlineto{\pgfqpoint{6.608872in}{3.175748in}}%
\pgfpathlineto{\pgfqpoint{6.609982in}{3.076431in}}%
\pgfpathlineto{\pgfqpoint{6.609735in}{3.237858in}}%
\pgfpathlineto{\pgfqpoint{6.610028in}{3.128354in}}%
\pgfpathlineto{\pgfqpoint{6.610706in}{3.238975in}}%
\pgfpathlineto{\pgfqpoint{6.610953in}{3.071205in}}%
\pgfpathlineto{\pgfqpoint{6.611169in}{3.216923in}}%
\pgfpathlineto{\pgfqpoint{6.611924in}{3.095953in}}%
\pgfpathlineto{\pgfqpoint{6.611662in}{3.232631in}}%
\pgfpathlineto{\pgfqpoint{6.612294in}{3.195430in}}%
\pgfpathlineto{\pgfqpoint{6.613281in}{3.223208in}}%
\pgfpathlineto{\pgfqpoint{6.612895in}{3.092379in}}%
\pgfpathlineto{\pgfqpoint{6.613327in}{3.164531in}}%
\pgfpathlineto{\pgfqpoint{6.613743in}{3.100381in}}%
\pgfpathlineto{\pgfqpoint{6.614252in}{3.226578in}}%
\pgfpathlineto{\pgfqpoint{6.614421in}{3.163524in}}%
\pgfpathlineto{\pgfqpoint{6.615516in}{3.256424in}}%
\pgfpathlineto{\pgfqpoint{6.614714in}{3.087451in}}%
\pgfpathlineto{\pgfqpoint{6.615562in}{3.181712in}}%
\pgfpathlineto{\pgfqpoint{6.615732in}{3.075707in}}%
\pgfpathlineto{\pgfqpoint{6.616086in}{3.235449in}}%
\pgfpathlineto{\pgfqpoint{6.616749in}{3.084830in}}%
\pgfpathlineto{\pgfqpoint{6.617073in}{3.243705in}}%
\pgfpathlineto{\pgfqpoint{6.617982in}{3.214336in}}%
\pgfpathlineto{\pgfqpoint{6.618105in}{3.258855in}}%
\pgfpathlineto{\pgfqpoint{6.618568in}{3.087666in}}%
\pgfpathlineto{\pgfqpoint{6.619092in}{3.227238in}}%
\pgfpathlineto{\pgfqpoint{6.619570in}{3.066955in}}%
\pgfpathlineto{\pgfqpoint{6.619940in}{3.234948in}}%
\pgfpathlineto{\pgfqpoint{6.620264in}{3.144183in}}%
\pgfpathlineto{\pgfqpoint{6.621019in}{3.231046in}}%
\pgfpathlineto{\pgfqpoint{6.620556in}{3.073968in}}%
\pgfpathlineto{\pgfqpoint{6.621435in}{3.159689in}}%
\pgfpathlineto{\pgfqpoint{6.621605in}{3.075634in}}%
\pgfpathlineto{\pgfqpoint{6.622052in}{3.233919in}}%
\pgfpathlineto{\pgfqpoint{6.622607in}{3.109262in}}%
\pgfpathlineto{\pgfqpoint{6.623023in}{3.237620in}}%
\pgfpathlineto{\pgfqpoint{6.623547in}{3.065090in}}%
\pgfpathlineto{\pgfqpoint{6.623840in}{3.208870in}}%
\pgfpathlineto{\pgfqpoint{6.624518in}{3.066173in}}%
\pgfpathlineto{\pgfqpoint{6.623994in}{3.231770in}}%
\pgfpathlineto{\pgfqpoint{6.624919in}{3.226046in}}%
\pgfpathlineto{\pgfqpoint{6.624950in}{3.232651in}}%
\pgfpathlineto{\pgfqpoint{6.625505in}{3.062960in}}%
\pgfpathlineto{\pgfqpoint{6.625813in}{3.161131in}}%
\pgfpathlineto{\pgfqpoint{6.625828in}{3.160007in}}%
\pgfpathlineto{\pgfqpoint{6.625905in}{3.231989in}}%
\pgfpathlineto{\pgfqpoint{6.625936in}{3.244368in}}%
\pgfpathlineto{\pgfqpoint{6.626491in}{3.075315in}}%
\pgfpathlineto{\pgfqpoint{6.626984in}{3.233041in}}%
\pgfpathlineto{\pgfqpoint{6.627462in}{3.075885in}}%
\pgfpathlineto{\pgfqpoint{6.627940in}{3.238120in}}%
\pgfpathlineto{\pgfqpoint{6.628125in}{3.200065in}}%
\pgfpathlineto{\pgfqpoint{6.628927in}{3.229149in}}%
\pgfpathlineto{\pgfqpoint{6.628449in}{3.076896in}}%
\pgfpathlineto{\pgfqpoint{6.629204in}{3.193895in}}%
\pgfpathlineto{\pgfqpoint{6.629343in}{3.095594in}}%
\pgfpathlineto{\pgfqpoint{6.629929in}{3.230935in}}%
\pgfpathlineto{\pgfqpoint{6.630360in}{3.115377in}}%
\pgfpathlineto{\pgfqpoint{6.630591in}{3.221693in}}%
\pgfpathlineto{\pgfqpoint{6.631316in}{3.090793in}}%
\pgfpathlineto{\pgfqpoint{6.631609in}{3.183890in}}%
\pgfpathlineto{\pgfqpoint{6.632302in}{3.099189in}}%
\pgfpathlineto{\pgfqpoint{6.631871in}{3.219176in}}%
\pgfpathlineto{\pgfqpoint{6.632734in}{3.138924in}}%
\pgfpathlineto{\pgfqpoint{6.632888in}{3.239208in}}%
\pgfpathlineto{\pgfqpoint{6.633289in}{3.119536in}}%
\pgfpathlineto{\pgfqpoint{6.633906in}{3.218040in}}%
\pgfpathlineto{\pgfqpoint{6.634260in}{3.123811in}}%
\pgfpathlineto{\pgfqpoint{6.634861in}{3.233149in}}%
\pgfpathlineto{\pgfqpoint{6.635046in}{3.164107in}}%
\pgfpathlineto{\pgfqpoint{6.635848in}{3.229914in}}%
\pgfpathlineto{\pgfqpoint{6.635694in}{3.123629in}}%
\pgfpathlineto{\pgfqpoint{6.636141in}{3.166192in}}%
\pgfpathlineto{\pgfqpoint{6.636680in}{3.127519in}}%
\pgfpathlineto{\pgfqpoint{6.636834in}{3.228748in}}%
\pgfpathlineto{\pgfqpoint{6.637235in}{3.175559in}}%
\pgfpathlineto{\pgfqpoint{6.637667in}{3.131846in}}%
\pgfpathlineto{\pgfqpoint{6.637821in}{3.225917in}}%
\pgfpathlineto{\pgfqpoint{6.638438in}{3.155911in}}%
\pgfpathlineto{\pgfqpoint{6.638823in}{3.219054in}}%
\pgfpathlineto{\pgfqpoint{6.638669in}{3.130883in}}%
\pgfpathlineto{\pgfqpoint{6.639532in}{3.160927in}}%
\pgfpathlineto{\pgfqpoint{6.640549in}{3.132855in}}%
\pgfpathlineto{\pgfqpoint{6.639810in}{3.221500in}}%
\pgfpathlineto{\pgfqpoint{6.640642in}{3.137164in}}%
\pgfpathlineto{\pgfqpoint{6.641783in}{3.225573in}}%
\pgfpathlineto{\pgfqpoint{6.641798in}{3.222980in}}%
\pgfpathlineto{\pgfqpoint{6.642862in}{3.131613in}}%
\pgfpathlineto{\pgfqpoint{6.642970in}{3.163458in}}%
\pgfpathlineto{\pgfqpoint{6.643771in}{3.214422in}}%
\pgfpathlineto{\pgfqpoint{6.643848in}{3.135758in}}%
\pgfpathlineto{\pgfqpoint{6.644049in}{3.159913in}}%
\pgfpathlineto{\pgfqpoint{6.644588in}{3.132453in}}%
\pgfpathlineto{\pgfqpoint{6.644634in}{3.194743in}}%
\pgfpathlineto{\pgfqpoint{6.644727in}{3.208690in}}%
\pgfpathlineto{\pgfqpoint{6.645251in}{3.133379in}}%
\pgfpathlineto{\pgfqpoint{6.645744in}{3.204233in}}%
\pgfpathlineto{\pgfqpoint{6.646808in}{3.124773in}}%
\pgfpathlineto{\pgfqpoint{6.646685in}{3.210514in}}%
\pgfpathlineto{\pgfqpoint{6.646870in}{3.176391in}}%
\pgfpathlineto{\pgfqpoint{6.647918in}{3.211630in}}%
\pgfpathlineto{\pgfqpoint{6.647794in}{3.121986in}}%
\pgfpathlineto{\pgfqpoint{6.647979in}{3.182913in}}%
\pgfpathlineto{\pgfqpoint{6.648781in}{3.126261in}}%
\pgfpathlineto{\pgfqpoint{6.648642in}{3.218471in}}%
\pgfpathlineto{\pgfqpoint{6.649105in}{3.175450in}}%
\pgfpathlineto{\pgfqpoint{6.649629in}{3.223390in}}%
\pgfpathlineto{\pgfqpoint{6.649752in}{3.125968in}}%
\pgfpathlineto{\pgfqpoint{6.650122in}{3.173544in}}%
\pgfpathlineto{\pgfqpoint{6.650400in}{3.138797in}}%
\pgfpathlineto{\pgfqpoint{6.650600in}{3.216093in}}%
\pgfpathlineto{\pgfqpoint{6.651232in}{3.157161in}}%
\pgfpathlineto{\pgfqpoint{6.651602in}{3.211378in}}%
\pgfpathlineto{\pgfqpoint{6.652111in}{3.140768in}}%
\pgfpathlineto{\pgfqpoint{6.652296in}{3.162953in}}%
\pgfpathlineto{\pgfqpoint{6.653313in}{3.123153in}}%
\pgfpathlineto{\pgfqpoint{6.652573in}{3.206790in}}%
\pgfpathlineto{\pgfqpoint{6.653405in}{3.159465in}}%
\pgfpathlineto{\pgfqpoint{6.653822in}{3.207847in}}%
\pgfpathlineto{\pgfqpoint{6.653960in}{3.142836in}}%
\pgfpathlineto{\pgfqpoint{6.654269in}{3.144670in}}%
\pgfpathlineto{\pgfqpoint{6.654299in}{3.138031in}}%
\pgfpathlineto{\pgfqpoint{6.654793in}{3.206791in}}%
\pgfpathlineto{\pgfqpoint{6.655301in}{3.160953in}}%
\pgfpathlineto{\pgfqpoint{6.655764in}{3.201803in}}%
\pgfpathlineto{\pgfqpoint{6.655918in}{3.137520in}}%
\pgfpathlineto{\pgfqpoint{6.656442in}{3.185141in}}%
\pgfpathlineto{\pgfqpoint{6.657506in}{3.197718in}}%
\pgfpathlineto{\pgfqpoint{6.656889in}{3.143525in}}%
\pgfpathlineto{\pgfqpoint{6.657521in}{3.195505in}}%
\pgfpathlineto{\pgfqpoint{6.657860in}{3.143132in}}%
\pgfpathlineto{\pgfqpoint{6.658662in}{3.165932in}}%
\pgfpathlineto{\pgfqpoint{6.659448in}{3.193402in}}%
\pgfpathlineto{\pgfqpoint{6.658831in}{3.140312in}}%
\pgfpathlineto{\pgfqpoint{6.659756in}{3.167842in}}%
\pgfpathlineto{\pgfqpoint{6.660789in}{3.141623in}}%
\pgfpathlineto{\pgfqpoint{6.660435in}{3.193044in}}%
\pgfpathlineto{\pgfqpoint{6.660897in}{3.142830in}}%
\pgfpathlineto{\pgfqpoint{6.662022in}{3.200937in}}%
\pgfpathlineto{\pgfqpoint{6.661760in}{3.140083in}}%
\pgfpathlineto{\pgfqpoint{6.662053in}{3.181320in}}%
\pgfpathlineto{\pgfqpoint{6.662747in}{3.141443in}}%
\pgfpathlineto{\pgfqpoint{6.662993in}{3.199903in}}%
\pgfpathlineto{\pgfqpoint{6.663163in}{3.171775in}}%
\pgfpathlineto{\pgfqpoint{6.663363in}{3.187959in}}%
\pgfpathlineto{\pgfqpoint{6.663440in}{3.156979in}}%
\pgfpathlineto{\pgfqpoint{6.663687in}{3.157939in}}%
\pgfpathlineto{\pgfqpoint{6.663718in}{3.146374in}}%
\pgfpathlineto{\pgfqpoint{6.663980in}{3.198111in}}%
\pgfpathlineto{\pgfqpoint{6.664797in}{3.152238in}}%
\pgfpathlineto{\pgfqpoint{6.664951in}{3.197722in}}%
\pgfpathlineto{\pgfqpoint{6.665984in}{3.177465in}}%
\pgfpathlineto{\pgfqpoint{6.666724in}{3.154705in}}%
\pgfpathlineto{\pgfqpoint{6.666909in}{3.193724in}}%
\pgfpathlineto{\pgfqpoint{6.667094in}{3.175811in}}%
\pgfpathlineto{\pgfqpoint{6.667880in}{3.192234in}}%
\pgfpathlineto{\pgfqpoint{6.667695in}{3.156285in}}%
\pgfpathlineto{\pgfqpoint{6.668111in}{3.174029in}}%
\pgfpathlineto{\pgfqpoint{6.668666in}{3.159887in}}%
\pgfpathlineto{\pgfqpoint{6.668851in}{3.191492in}}%
\pgfpathlineto{\pgfqpoint{6.669267in}{3.165912in}}%
\pgfpathlineto{\pgfqpoint{6.669637in}{3.161074in}}%
\pgfpathlineto{\pgfqpoint{6.669822in}{3.187323in}}%
\pgfpathlineto{\pgfqpoint{6.670315in}{3.167359in}}%
\pgfpathlineto{\pgfqpoint{6.670793in}{3.185792in}}%
\pgfpathlineto{\pgfqpoint{6.670608in}{3.163495in}}%
\pgfpathlineto{\pgfqpoint{6.671441in}{3.170016in}}%
\pgfpathlineto{\pgfqpoint{6.672350in}{3.163954in}}%
\pgfpathlineto{\pgfqpoint{6.671764in}{3.184684in}}%
\pgfpathlineto{\pgfqpoint{6.672551in}{3.168849in}}%
\pgfpathlineto{\pgfqpoint{6.672736in}{3.182722in}}%
\pgfpathlineto{\pgfqpoint{6.673213in}{3.164673in}}%
\pgfpathlineto{\pgfqpoint{6.673830in}{3.175014in}}%
\pgfpathlineto{\pgfqpoint{6.674185in}{3.167309in}}%
\pgfpathlineto{\pgfqpoint{6.674693in}{3.176661in}}%
\pgfpathlineto{\pgfqpoint{6.675002in}{3.169721in}}%
\pgfpathlineto{\pgfqpoint{6.675695in}{3.172637in}}%
\pgfpathlineto{\pgfqpoint{6.675140in}{3.169474in}}%
\pgfpathlineto{\pgfqpoint{6.675957in}{3.171639in}}%
\pgfusepath{stroke}%
\end{pgfscope}%
\begin{pgfscope}%
\pgfsetrectcap%
\pgfsetmiterjoin%
\pgfsetlinewidth{0.803000pt}%
\definecolor{currentstroke}{rgb}{0.000000,0.000000,0.000000}%
\pgfsetstrokecolor{currentstroke}%
\pgfsetdash{}{0pt}%
\pgfpathmoveto{\pgfqpoint{0.746130in}{2.463273in}}%
\pgfpathlineto{\pgfqpoint{0.746130in}{3.748369in}}%
\pgfusepath{stroke}%
\end{pgfscope}%
\begin{pgfscope}%
\pgfsetrectcap%
\pgfsetmiterjoin%
\pgfsetlinewidth{0.803000pt}%
\definecolor{currentstroke}{rgb}{0.000000,0.000000,0.000000}%
\pgfsetstrokecolor{currentstroke}%
\pgfsetdash{}{0pt}%
\pgfpathmoveto{\pgfqpoint{6.958330in}{2.463273in}}%
\pgfpathlineto{\pgfqpoint{6.958330in}{3.748369in}}%
\pgfusepath{stroke}%
\end{pgfscope}%
\begin{pgfscope}%
\pgfsetrectcap%
\pgfsetmiterjoin%
\pgfsetlinewidth{0.803000pt}%
\definecolor{currentstroke}{rgb}{0.000000,0.000000,0.000000}%
\pgfsetstrokecolor{currentstroke}%
\pgfsetdash{}{0pt}%
\pgfpathmoveto{\pgfqpoint{0.746130in}{2.463273in}}%
\pgfpathlineto{\pgfqpoint{6.958330in}{2.463273in}}%
\pgfusepath{stroke}%
\end{pgfscope}%
\begin{pgfscope}%
\pgfsetrectcap%
\pgfsetmiterjoin%
\pgfsetlinewidth{0.803000pt}%
\definecolor{currentstroke}{rgb}{0.000000,0.000000,0.000000}%
\pgfsetstrokecolor{currentstroke}%
\pgfsetdash{}{0pt}%
\pgfpathmoveto{\pgfqpoint{0.746130in}{3.748369in}}%
\pgfpathlineto{\pgfqpoint{6.958330in}{3.748369in}}%
\pgfusepath{stroke}%
\end{pgfscope}%
\begin{pgfscope}%
\definecolor{textcolor}{rgb}{0.000000,0.000000,0.000000}%
\pgfsetstrokecolor{textcolor}%
\pgfsetfillcolor{textcolor}%
\pgftext[x=3.852230in,y=3.831702in,,base]{\color{textcolor}\sffamily\fontsize{12.000000}{14.400000}\selectfont Violin Partita No.2 in D minor, BWV 1004 (Bach, Johann Sebastian) - V Ciaccona}%
\end{pgfscope}%
\begin{pgfscope}%
\pgfsetbuttcap%
\pgfsetmiterjoin%
\definecolor{currentfill}{rgb}{1.000000,1.000000,1.000000}%
\pgfsetfillcolor{currentfill}%
\pgfsetlinewidth{0.000000pt}%
\definecolor{currentstroke}{rgb}{0.000000,0.000000,0.000000}%
\pgfsetstrokecolor{currentstroke}%
\pgfsetstrokeopacity{0.000000}%
\pgfsetdash{}{0pt}%
\pgfpathmoveto{\pgfqpoint{0.746130in}{0.463273in}}%
\pgfpathlineto{\pgfqpoint{6.958330in}{0.463273in}}%
\pgfpathlineto{\pgfqpoint{6.958330in}{1.748369in}}%
\pgfpathlineto{\pgfqpoint{0.746130in}{1.748369in}}%
\pgfpathlineto{\pgfqpoint{0.746130in}{0.463273in}}%
\pgfpathclose%
\pgfusepath{fill}%
\end{pgfscope}%
\begin{pgfscope}%
\pgfsetbuttcap%
\pgfsetroundjoin%
\definecolor{currentfill}{rgb}{0.000000,0.000000,0.000000}%
\pgfsetfillcolor{currentfill}%
\pgfsetlinewidth{0.803000pt}%
\definecolor{currentstroke}{rgb}{0.000000,0.000000,0.000000}%
\pgfsetstrokecolor{currentstroke}%
\pgfsetdash{}{0pt}%
\pgfsys@defobject{currentmarker}{\pgfqpoint{0.000000in}{-0.048611in}}{\pgfqpoint{0.000000in}{0.000000in}}{%
\pgfpathmoveto{\pgfqpoint{0.000000in}{0.000000in}}%
\pgfpathlineto{\pgfqpoint{0.000000in}{-0.048611in}}%
\pgfusepath{stroke,fill}%
}%
\begin{pgfscope}%
\pgfsys@transformshift{1.028503in}{0.463273in}%
\pgfsys@useobject{currentmarker}{}%
\end{pgfscope}%
\end{pgfscope}%
\begin{pgfscope}%
\definecolor{textcolor}{rgb}{0.000000,0.000000,0.000000}%
\pgfsetstrokecolor{textcolor}%
\pgfsetfillcolor{textcolor}%
\pgftext[x=1.028503in,y=0.366051in,,top]{\color{textcolor}\sffamily\fontsize{10.000000}{12.000000}\selectfont 0}%
\end{pgfscope}%
\begin{pgfscope}%
\pgfsetbuttcap%
\pgfsetroundjoin%
\definecolor{currentfill}{rgb}{0.000000,0.000000,0.000000}%
\pgfsetfillcolor{currentfill}%
\pgfsetlinewidth{0.803000pt}%
\definecolor{currentstroke}{rgb}{0.000000,0.000000,0.000000}%
\pgfsetstrokecolor{currentstroke}%
\pgfsetdash{}{0pt}%
\pgfsys@defobject{currentmarker}{\pgfqpoint{0.000000in}{-0.048611in}}{\pgfqpoint{0.000000in}{0.000000in}}{%
\pgfpathmoveto{\pgfqpoint{0.000000in}{0.000000in}}%
\pgfpathlineto{\pgfqpoint{0.000000in}{-0.048611in}}%
\pgfusepath{stroke,fill}%
}%
\begin{pgfscope}%
\pgfsys@transformshift{1.920541in}{0.463273in}%
\pgfsys@useobject{currentmarker}{}%
\end{pgfscope}%
\end{pgfscope}%
\begin{pgfscope}%
\definecolor{textcolor}{rgb}{0.000000,0.000000,0.000000}%
\pgfsetstrokecolor{textcolor}%
\pgfsetfillcolor{textcolor}%
\pgftext[x=1.920541in,y=0.366051in,,top]{\color{textcolor}\sffamily\fontsize{10.000000}{12.000000}\selectfont 50000}%
\end{pgfscope}%
\begin{pgfscope}%
\pgfsetbuttcap%
\pgfsetroundjoin%
\definecolor{currentfill}{rgb}{0.000000,0.000000,0.000000}%
\pgfsetfillcolor{currentfill}%
\pgfsetlinewidth{0.803000pt}%
\definecolor{currentstroke}{rgb}{0.000000,0.000000,0.000000}%
\pgfsetstrokecolor{currentstroke}%
\pgfsetdash{}{0pt}%
\pgfsys@defobject{currentmarker}{\pgfqpoint{0.000000in}{-0.048611in}}{\pgfqpoint{0.000000in}{0.000000in}}{%
\pgfpathmoveto{\pgfqpoint{0.000000in}{0.000000in}}%
\pgfpathlineto{\pgfqpoint{0.000000in}{-0.048611in}}%
\pgfusepath{stroke,fill}%
}%
\begin{pgfscope}%
\pgfsys@transformshift{2.812578in}{0.463273in}%
\pgfsys@useobject{currentmarker}{}%
\end{pgfscope}%
\end{pgfscope}%
\begin{pgfscope}%
\definecolor{textcolor}{rgb}{0.000000,0.000000,0.000000}%
\pgfsetstrokecolor{textcolor}%
\pgfsetfillcolor{textcolor}%
\pgftext[x=2.812578in,y=0.366051in,,top]{\color{textcolor}\sffamily\fontsize{10.000000}{12.000000}\selectfont 100000}%
\end{pgfscope}%
\begin{pgfscope}%
\pgfsetbuttcap%
\pgfsetroundjoin%
\definecolor{currentfill}{rgb}{0.000000,0.000000,0.000000}%
\pgfsetfillcolor{currentfill}%
\pgfsetlinewidth{0.803000pt}%
\definecolor{currentstroke}{rgb}{0.000000,0.000000,0.000000}%
\pgfsetstrokecolor{currentstroke}%
\pgfsetdash{}{0pt}%
\pgfsys@defobject{currentmarker}{\pgfqpoint{0.000000in}{-0.048611in}}{\pgfqpoint{0.000000in}{0.000000in}}{%
\pgfpathmoveto{\pgfqpoint{0.000000in}{0.000000in}}%
\pgfpathlineto{\pgfqpoint{0.000000in}{-0.048611in}}%
\pgfusepath{stroke,fill}%
}%
\begin{pgfscope}%
\pgfsys@transformshift{3.704616in}{0.463273in}%
\pgfsys@useobject{currentmarker}{}%
\end{pgfscope}%
\end{pgfscope}%
\begin{pgfscope}%
\definecolor{textcolor}{rgb}{0.000000,0.000000,0.000000}%
\pgfsetstrokecolor{textcolor}%
\pgfsetfillcolor{textcolor}%
\pgftext[x=3.704616in,y=0.366051in,,top]{\color{textcolor}\sffamily\fontsize{10.000000}{12.000000}\selectfont 150000}%
\end{pgfscope}%
\begin{pgfscope}%
\pgfsetbuttcap%
\pgfsetroundjoin%
\definecolor{currentfill}{rgb}{0.000000,0.000000,0.000000}%
\pgfsetfillcolor{currentfill}%
\pgfsetlinewidth{0.803000pt}%
\definecolor{currentstroke}{rgb}{0.000000,0.000000,0.000000}%
\pgfsetstrokecolor{currentstroke}%
\pgfsetdash{}{0pt}%
\pgfsys@defobject{currentmarker}{\pgfqpoint{0.000000in}{-0.048611in}}{\pgfqpoint{0.000000in}{0.000000in}}{%
\pgfpathmoveto{\pgfqpoint{0.000000in}{0.000000in}}%
\pgfpathlineto{\pgfqpoint{0.000000in}{-0.048611in}}%
\pgfusepath{stroke,fill}%
}%
\begin{pgfscope}%
\pgfsys@transformshift{4.596653in}{0.463273in}%
\pgfsys@useobject{currentmarker}{}%
\end{pgfscope}%
\end{pgfscope}%
\begin{pgfscope}%
\definecolor{textcolor}{rgb}{0.000000,0.000000,0.000000}%
\pgfsetstrokecolor{textcolor}%
\pgfsetfillcolor{textcolor}%
\pgftext[x=4.596653in,y=0.366051in,,top]{\color{textcolor}\sffamily\fontsize{10.000000}{12.000000}\selectfont 200000}%
\end{pgfscope}%
\begin{pgfscope}%
\pgfsetbuttcap%
\pgfsetroundjoin%
\definecolor{currentfill}{rgb}{0.000000,0.000000,0.000000}%
\pgfsetfillcolor{currentfill}%
\pgfsetlinewidth{0.803000pt}%
\definecolor{currentstroke}{rgb}{0.000000,0.000000,0.000000}%
\pgfsetstrokecolor{currentstroke}%
\pgfsetdash{}{0pt}%
\pgfsys@defobject{currentmarker}{\pgfqpoint{0.000000in}{-0.048611in}}{\pgfqpoint{0.000000in}{0.000000in}}{%
\pgfpathmoveto{\pgfqpoint{0.000000in}{0.000000in}}%
\pgfpathlineto{\pgfqpoint{0.000000in}{-0.048611in}}%
\pgfusepath{stroke,fill}%
}%
\begin{pgfscope}%
\pgfsys@transformshift{5.488691in}{0.463273in}%
\pgfsys@useobject{currentmarker}{}%
\end{pgfscope}%
\end{pgfscope}%
\begin{pgfscope}%
\definecolor{textcolor}{rgb}{0.000000,0.000000,0.000000}%
\pgfsetstrokecolor{textcolor}%
\pgfsetfillcolor{textcolor}%
\pgftext[x=5.488691in,y=0.366051in,,top]{\color{textcolor}\sffamily\fontsize{10.000000}{12.000000}\selectfont 250000}%
\end{pgfscope}%
\begin{pgfscope}%
\pgfsetbuttcap%
\pgfsetroundjoin%
\definecolor{currentfill}{rgb}{0.000000,0.000000,0.000000}%
\pgfsetfillcolor{currentfill}%
\pgfsetlinewidth{0.803000pt}%
\definecolor{currentstroke}{rgb}{0.000000,0.000000,0.000000}%
\pgfsetstrokecolor{currentstroke}%
\pgfsetdash{}{0pt}%
\pgfsys@defobject{currentmarker}{\pgfqpoint{0.000000in}{-0.048611in}}{\pgfqpoint{0.000000in}{0.000000in}}{%
\pgfpathmoveto{\pgfqpoint{0.000000in}{0.000000in}}%
\pgfpathlineto{\pgfqpoint{0.000000in}{-0.048611in}}%
\pgfusepath{stroke,fill}%
}%
\begin{pgfscope}%
\pgfsys@transformshift{6.380729in}{0.463273in}%
\pgfsys@useobject{currentmarker}{}%
\end{pgfscope}%
\end{pgfscope}%
\begin{pgfscope}%
\definecolor{textcolor}{rgb}{0.000000,0.000000,0.000000}%
\pgfsetstrokecolor{textcolor}%
\pgfsetfillcolor{textcolor}%
\pgftext[x=6.380729in,y=0.366051in,,top]{\color{textcolor}\sffamily\fontsize{10.000000}{12.000000}\selectfont 300000}%
\end{pgfscope}%
\begin{pgfscope}%
\definecolor{textcolor}{rgb}{0.000000,0.000000,0.000000}%
\pgfsetstrokecolor{textcolor}%
\pgfsetfillcolor{textcolor}%
\pgftext[x=3.852230in,y=0.176083in,,top]{\color{textcolor}\sffamily\fontsize{10.000000}{12.000000}\selectfont ms}%
\end{pgfscope}%
\begin{pgfscope}%
\pgfsetbuttcap%
\pgfsetroundjoin%
\definecolor{currentfill}{rgb}{0.000000,0.000000,0.000000}%
\pgfsetfillcolor{currentfill}%
\pgfsetlinewidth{0.803000pt}%
\definecolor{currentstroke}{rgb}{0.000000,0.000000,0.000000}%
\pgfsetstrokecolor{currentstroke}%
\pgfsetdash{}{0pt}%
\pgfsys@defobject{currentmarker}{\pgfqpoint{-0.048611in}{0.000000in}}{\pgfqpoint{-0.000000in}{0.000000in}}{%
\pgfpathmoveto{\pgfqpoint{-0.000000in}{0.000000in}}%
\pgfpathlineto{\pgfqpoint{-0.048611in}{0.000000in}}%
\pgfusepath{stroke,fill}%
}%
\begin{pgfscope}%
\pgfsys@transformshift{0.746130in}{0.506522in}%
\pgfsys@useobject{currentmarker}{}%
\end{pgfscope}%
\end{pgfscope}%
\begin{pgfscope}%
\definecolor{textcolor}{rgb}{0.000000,0.000000,0.000000}%
\pgfsetstrokecolor{textcolor}%
\pgfsetfillcolor{textcolor}%
\pgftext[x=0.320004in, y=0.453760in, left, base]{\color{textcolor}\sffamily\fontsize{10.000000}{12.000000}\selectfont \ensuremath{-}0.5}%
\end{pgfscope}%
\begin{pgfscope}%
\pgfsetbuttcap%
\pgfsetroundjoin%
\definecolor{currentfill}{rgb}{0.000000,0.000000,0.000000}%
\pgfsetfillcolor{currentfill}%
\pgfsetlinewidth{0.803000pt}%
\definecolor{currentstroke}{rgb}{0.000000,0.000000,0.000000}%
\pgfsetstrokecolor{currentstroke}%
\pgfsetdash{}{0pt}%
\pgfsys@defobject{currentmarker}{\pgfqpoint{-0.048611in}{0.000000in}}{\pgfqpoint{-0.000000in}{0.000000in}}{%
\pgfpathmoveto{\pgfqpoint{-0.000000in}{0.000000in}}%
\pgfpathlineto{\pgfqpoint{-0.048611in}{0.000000in}}%
\pgfusepath{stroke,fill}%
}%
\begin{pgfscope}%
\pgfsys@transformshift{0.746130in}{1.121764in}%
\pgfsys@useobject{currentmarker}{}%
\end{pgfscope}%
\end{pgfscope}%
\begin{pgfscope}%
\definecolor{textcolor}{rgb}{0.000000,0.000000,0.000000}%
\pgfsetstrokecolor{textcolor}%
\pgfsetfillcolor{textcolor}%
\pgftext[x=0.428029in, y=1.069002in, left, base]{\color{textcolor}\sffamily\fontsize{10.000000}{12.000000}\selectfont 0.0}%
\end{pgfscope}%
\begin{pgfscope}%
\pgfsetbuttcap%
\pgfsetroundjoin%
\definecolor{currentfill}{rgb}{0.000000,0.000000,0.000000}%
\pgfsetfillcolor{currentfill}%
\pgfsetlinewidth{0.803000pt}%
\definecolor{currentstroke}{rgb}{0.000000,0.000000,0.000000}%
\pgfsetstrokecolor{currentstroke}%
\pgfsetdash{}{0pt}%
\pgfsys@defobject{currentmarker}{\pgfqpoint{-0.048611in}{0.000000in}}{\pgfqpoint{-0.000000in}{0.000000in}}{%
\pgfpathmoveto{\pgfqpoint{-0.000000in}{0.000000in}}%
\pgfpathlineto{\pgfqpoint{-0.048611in}{0.000000in}}%
\pgfusepath{stroke,fill}%
}%
\begin{pgfscope}%
\pgfsys@transformshift{0.746130in}{1.737006in}%
\pgfsys@useobject{currentmarker}{}%
\end{pgfscope}%
\end{pgfscope}%
\begin{pgfscope}%
\definecolor{textcolor}{rgb}{0.000000,0.000000,0.000000}%
\pgfsetstrokecolor{textcolor}%
\pgfsetfillcolor{textcolor}%
\pgftext[x=0.428029in, y=1.684245in, left, base]{\color{textcolor}\sffamily\fontsize{10.000000}{12.000000}\selectfont 0.5}%
\end{pgfscope}%
\begin{pgfscope}%
\definecolor{textcolor}{rgb}{0.000000,0.000000,0.000000}%
\pgfsetstrokecolor{textcolor}%
\pgfsetfillcolor{textcolor}%
\pgftext[x=0.264448in,y=1.105821in,,bottom,rotate=90.000000]{\color{textcolor}\sffamily\fontsize{10.000000}{12.000000}\selectfont Amplitude}%
\end{pgfscope}%
\begin{pgfscope}%
\pgfpathrectangle{\pgfqpoint{0.746130in}{0.463273in}}{\pgfqpoint{6.212200in}{1.285096in}}%
\pgfusepath{clip}%
\pgfsetrectcap%
\pgfsetroundjoin%
\pgfsetlinewidth{1.505625pt}%
\definecolor{currentstroke}{rgb}{0.121569,0.466667,0.705882}%
\pgfsetstrokecolor{currentstroke}%
\pgfsetdash{}{0pt}%
\pgfpathmoveto{\pgfqpoint{1.028503in}{1.121766in}}%
\pgfpathlineto{\pgfqpoint{1.029199in}{1.120129in}}%
\pgfpathlineto{\pgfqpoint{1.029484in}{1.122405in}}%
\pgfpathlineto{\pgfqpoint{1.029574in}{1.123673in}}%
\pgfpathlineto{\pgfqpoint{1.030394in}{1.120534in}}%
\pgfpathlineto{\pgfqpoint{1.031054in}{1.119396in}}%
\pgfpathlineto{\pgfqpoint{1.031304in}{1.121649in}}%
\pgfpathlineto{\pgfqpoint{1.031429in}{1.121428in}}%
\pgfpathlineto{\pgfqpoint{1.031750in}{1.122661in}}%
\pgfpathlineto{\pgfqpoint{1.032232in}{1.120794in}}%
\pgfpathlineto{\pgfqpoint{1.032303in}{1.120320in}}%
\pgfpathlineto{\pgfqpoint{1.033035in}{1.124945in}}%
\pgfpathlineto{\pgfqpoint{1.033588in}{1.126584in}}%
\pgfpathlineto{\pgfqpoint{1.033855in}{1.120205in}}%
\pgfpathlineto{\pgfqpoint{1.034569in}{1.116157in}}%
\pgfpathlineto{\pgfqpoint{1.034712in}{1.122729in}}%
\pgfpathlineto{\pgfqpoint{1.034979in}{1.125122in}}%
\pgfpathlineto{\pgfqpoint{1.035550in}{1.121572in}}%
\pgfpathlineto{\pgfqpoint{1.036317in}{1.116797in}}%
\pgfpathlineto{\pgfqpoint{1.036531in}{1.124741in}}%
\pgfpathlineto{\pgfqpoint{1.036710in}{1.128686in}}%
\pgfpathlineto{\pgfqpoint{1.037424in}{1.120215in}}%
\pgfpathlineto{\pgfqpoint{1.038030in}{1.114006in}}%
\pgfpathlineto{\pgfqpoint{1.038351in}{1.123824in}}%
\pgfpathlineto{\pgfqpoint{1.038387in}{1.123823in}}%
\pgfpathlineto{\pgfqpoint{1.038405in}{1.123219in}}%
\pgfpathlineto{\pgfqpoint{1.038512in}{1.119557in}}%
\pgfpathlineto{\pgfqpoint{1.039011in}{1.124974in}}%
\pgfpathlineto{\pgfqpoint{1.039529in}{1.121687in}}%
\pgfpathlineto{\pgfqpoint{1.040189in}{1.124783in}}%
\pgfpathlineto{\pgfqpoint{1.039868in}{1.117818in}}%
\pgfpathlineto{\pgfqpoint{1.040421in}{1.117939in}}%
\pgfpathlineto{\pgfqpoint{1.040456in}{1.116984in}}%
\pgfpathlineto{\pgfqpoint{1.040795in}{1.127144in}}%
\pgfpathlineto{\pgfqpoint{1.041402in}{1.122153in}}%
\pgfpathlineto{\pgfqpoint{1.042455in}{1.126409in}}%
\pgfpathlineto{\pgfqpoint{1.041848in}{1.116303in}}%
\pgfpathlineto{\pgfqpoint{1.042651in}{1.125079in}}%
\pgfpathlineto{\pgfqpoint{1.043454in}{1.117185in}}%
\pgfpathlineto{\pgfqpoint{1.043043in}{1.125895in}}%
\pgfpathlineto{\pgfqpoint{1.043935in}{1.118735in}}%
\pgfpathlineto{\pgfqpoint{1.044435in}{1.124961in}}%
\pgfpathlineto{\pgfqpoint{1.045042in}{1.119153in}}%
\pgfpathlineto{\pgfqpoint{1.046005in}{1.092823in}}%
\pgfpathlineto{\pgfqpoint{1.045755in}{1.127546in}}%
\pgfpathlineto{\pgfqpoint{1.046058in}{1.121486in}}%
\pgfpathlineto{\pgfqpoint{1.046344in}{1.243922in}}%
\pgfpathlineto{\pgfqpoint{1.046594in}{0.868475in}}%
\pgfpathlineto{\pgfqpoint{1.047182in}{1.143456in}}%
\pgfpathlineto{\pgfqpoint{1.047200in}{1.143368in}}%
\pgfpathlineto{\pgfqpoint{1.047611in}{1.420823in}}%
\pgfpathlineto{\pgfqpoint{1.048217in}{1.121222in}}%
\pgfpathlineto{\pgfqpoint{1.048913in}{0.862150in}}%
\pgfpathlineto{\pgfqpoint{1.049252in}{1.247848in}}%
\pgfpathlineto{\pgfqpoint{1.049270in}{1.247824in}}%
\pgfpathlineto{\pgfqpoint{1.049288in}{1.246461in}}%
\pgfpathlineto{\pgfqpoint{1.049377in}{1.319483in}}%
\pgfpathlineto{\pgfqpoint{1.049644in}{1.410761in}}%
\pgfpathlineto{\pgfqpoint{1.049912in}{1.001203in}}%
\pgfpathlineto{\pgfqpoint{1.050358in}{0.603083in}}%
\pgfpathlineto{\pgfqpoint{1.050911in}{1.286071in}}%
\pgfpathlineto{\pgfqpoint{1.051357in}{1.593279in}}%
\pgfpathlineto{\pgfqpoint{1.051892in}{1.002932in}}%
\pgfpathlineto{\pgfqpoint{1.052303in}{0.743895in}}%
\pgfpathlineto{\pgfqpoint{1.052784in}{1.259057in}}%
\pgfpathlineto{\pgfqpoint{1.052998in}{1.457986in}}%
\pgfpathlineto{\pgfqpoint{1.053569in}{0.755269in}}%
\pgfpathlineto{\pgfqpoint{1.053730in}{0.521687in}}%
\pgfpathlineto{\pgfqpoint{1.054372in}{1.288023in}}%
\pgfpathlineto{\pgfqpoint{1.054890in}{1.689956in}}%
\pgfpathlineto{\pgfqpoint{1.055389in}{1.211123in}}%
\pgfpathlineto{\pgfqpoint{1.055853in}{0.763635in}}%
\pgfpathlineto{\pgfqpoint{1.056335in}{1.335737in}}%
\pgfpathlineto{\pgfqpoint{1.056388in}{1.410116in}}%
\pgfpathlineto{\pgfqpoint{1.057120in}{0.580213in}}%
\pgfpathlineto{\pgfqpoint{1.058261in}{1.650276in}}%
\pgfpathlineto{\pgfqpoint{1.058993in}{1.090017in}}%
\pgfpathlineto{\pgfqpoint{1.059261in}{0.890326in}}%
\pgfpathlineto{\pgfqpoint{1.059796in}{1.296107in}}%
\pgfpathlineto{\pgfqpoint{1.060188in}{0.921407in}}%
\pgfpathlineto{\pgfqpoint{1.060492in}{0.634765in}}%
\pgfpathlineto{\pgfqpoint{1.061098in}{0.982246in}}%
\pgfpathlineto{\pgfqpoint{1.061830in}{1.535247in}}%
\pgfpathlineto{\pgfqpoint{1.062365in}{1.217818in}}%
\pgfpathlineto{\pgfqpoint{1.063863in}{0.603522in}}%
\pgfpathlineto{\pgfqpoint{1.063150in}{1.286745in}}%
\pgfpathlineto{\pgfqpoint{1.063899in}{0.636414in}}%
\pgfpathlineto{\pgfqpoint{1.065202in}{1.396117in}}%
\pgfpathlineto{\pgfqpoint{1.065487in}{1.382644in}}%
\pgfpathlineto{\pgfqpoint{1.065630in}{1.439764in}}%
\pgfpathlineto{\pgfqpoint{1.065808in}{1.220282in}}%
\pgfpathlineto{\pgfqpoint{1.067235in}{0.561313in}}%
\pgfpathlineto{\pgfqpoint{1.066504in}{1.410109in}}%
\pgfpathlineto{\pgfqpoint{1.067485in}{0.798785in}}%
\pgfpathlineto{\pgfqpoint{1.069876in}{1.536201in}}%
\pgfpathlineto{\pgfqpoint{1.070072in}{1.308828in}}%
\pgfpathlineto{\pgfqpoint{1.070607in}{0.581656in}}%
\pgfpathlineto{\pgfqpoint{1.071250in}{1.042261in}}%
\pgfpathlineto{\pgfqpoint{1.071767in}{1.346571in}}%
\pgfpathlineto{\pgfqpoint{1.072427in}{1.139095in}}%
\pgfpathlineto{\pgfqpoint{1.072659in}{1.026955in}}%
\pgfpathlineto{\pgfqpoint{1.073248in}{1.569246in}}%
\pgfpathlineto{\pgfqpoint{1.073283in}{1.542143in}}%
\pgfpathlineto{\pgfqpoint{1.073979in}{0.661902in}}%
\pgfpathlineto{\pgfqpoint{1.074729in}{1.120832in}}%
\pgfpathlineto{\pgfqpoint{1.075103in}{1.343042in}}%
\pgfpathlineto{\pgfqpoint{1.075781in}{1.117646in}}%
\pgfpathlineto{\pgfqpoint{1.076013in}{0.957078in}}%
\pgfpathlineto{\pgfqpoint{1.076620in}{1.532929in}}%
\pgfpathlineto{\pgfqpoint{1.077333in}{0.726297in}}%
\pgfpathlineto{\pgfqpoint{1.078332in}{1.238321in}}%
\pgfpathlineto{\pgfqpoint{1.078457in}{1.326129in}}%
\pgfpathlineto{\pgfqpoint{1.079135in}{1.122457in}}%
\pgfpathlineto{\pgfqpoint{1.079367in}{0.944666in}}%
\pgfpathlineto{\pgfqpoint{1.079974in}{1.466617in}}%
\pgfpathlineto{\pgfqpoint{1.080152in}{1.382442in}}%
\pgfpathlineto{\pgfqpoint{1.080705in}{0.748408in}}%
\pgfpathlineto{\pgfqpoint{1.081347in}{1.144469in}}%
\pgfpathlineto{\pgfqpoint{1.081811in}{1.338233in}}%
\pgfpathlineto{\pgfqpoint{1.082347in}{1.102796in}}%
\pgfpathlineto{\pgfqpoint{1.082454in}{1.164739in}}%
\pgfpathlineto{\pgfqpoint{1.082739in}{0.940967in}}%
\pgfpathlineto{\pgfqpoint{1.083310in}{1.400051in}}%
\pgfpathlineto{\pgfqpoint{1.083346in}{1.416424in}}%
\pgfpathlineto{\pgfqpoint{1.083827in}{0.953199in}}%
\pgfpathlineto{\pgfqpoint{1.084059in}{0.724679in}}%
\pgfpathlineto{\pgfqpoint{1.084666in}{1.134992in}}%
\pgfpathlineto{\pgfqpoint{1.085165in}{1.371799in}}%
\pgfpathlineto{\pgfqpoint{1.085718in}{1.098165in}}%
\pgfpathlineto{\pgfqpoint{1.085790in}{1.166186in}}%
\pgfpathlineto{\pgfqpoint{1.086057in}{0.957670in}}%
\pgfpathlineto{\pgfqpoint{1.086093in}{0.937457in}}%
\pgfpathlineto{\pgfqpoint{1.086628in}{1.314193in}}%
\pgfpathlineto{\pgfqpoint{1.086717in}{1.385246in}}%
\pgfpathlineto{\pgfqpoint{1.087342in}{0.805262in}}%
\pgfpathlineto{\pgfqpoint{1.087431in}{0.692689in}}%
\pgfpathlineto{\pgfqpoint{1.088127in}{1.244889in}}%
\pgfpathlineto{\pgfqpoint{1.088555in}{1.408807in}}%
\pgfpathlineto{\pgfqpoint{1.089072in}{1.131637in}}%
\pgfpathlineto{\pgfqpoint{1.089162in}{1.190985in}}%
\pgfpathlineto{\pgfqpoint{1.089465in}{0.957921in}}%
\pgfpathlineto{\pgfqpoint{1.089965in}{1.262655in}}%
\pgfpathlineto{\pgfqpoint{1.090089in}{1.353277in}}%
\pgfpathlineto{\pgfqpoint{1.090589in}{0.860430in}}%
\pgfpathlineto{\pgfqpoint{1.090803in}{0.673947in}}%
\pgfpathlineto{\pgfqpoint{1.091374in}{1.092707in}}%
\pgfpathlineto{\pgfqpoint{1.092034in}{1.422929in}}%
\pgfpathlineto{\pgfqpoint{1.092587in}{1.208949in}}%
\pgfpathlineto{\pgfqpoint{1.092837in}{0.995618in}}%
\pgfpathlineto{\pgfqpoint{1.093443in}{1.341005in}}%
\pgfpathlineto{\pgfqpoint{1.093533in}{1.334469in}}%
\pgfpathlineto{\pgfqpoint{1.093658in}{1.180408in}}%
\pgfpathlineto{\pgfqpoint{1.094157in}{0.658004in}}%
\pgfpathlineto{\pgfqpoint{1.094782in}{1.103561in}}%
\pgfpathlineto{\pgfqpoint{1.095388in}{1.418890in}}%
\pgfpathlineto{\pgfqpoint{1.095977in}{1.223032in}}%
\pgfpathlineto{\pgfqpoint{1.097529in}{0.650604in}}%
\pgfpathlineto{\pgfqpoint{1.096905in}{1.350352in}}%
\pgfpathlineto{\pgfqpoint{1.097600in}{0.666823in}}%
\pgfpathlineto{\pgfqpoint{1.098100in}{1.033769in}}%
\pgfpathlineto{\pgfqpoint{1.098760in}{1.399932in}}%
\pgfpathlineto{\pgfqpoint{1.099384in}{1.199375in}}%
\pgfpathlineto{\pgfqpoint{1.100919in}{0.653140in}}%
\pgfpathlineto{\pgfqpoint{1.100259in}{1.370017in}}%
\pgfpathlineto{\pgfqpoint{1.101026in}{0.693111in}}%
\pgfpathlineto{\pgfqpoint{1.102132in}{1.371359in}}%
\pgfpathlineto{\pgfqpoint{1.102667in}{1.250068in}}%
\pgfpathlineto{\pgfqpoint{1.103006in}{1.073003in}}%
\pgfpathlineto{\pgfqpoint{1.103399in}{1.302728in}}%
\pgfpathlineto{\pgfqpoint{1.103631in}{1.396113in}}%
\pgfpathlineto{\pgfqpoint{1.103898in}{1.012706in}}%
\pgfpathlineto{\pgfqpoint{1.104308in}{0.666958in}}%
\pgfpathlineto{\pgfqpoint{1.104915in}{1.080274in}}%
\pgfpathlineto{\pgfqpoint{1.105504in}{1.342961in}}%
\pgfpathlineto{\pgfqpoint{1.106110in}{1.192806in}}%
\pgfpathlineto{\pgfqpoint{1.106360in}{1.073159in}}%
\pgfpathlineto{\pgfqpoint{1.106860in}{1.381059in}}%
\pgfpathlineto{\pgfqpoint{1.106985in}{1.413545in}}%
\pgfpathlineto{\pgfqpoint{1.107145in}{1.231160in}}%
\pgfpathlineto{\pgfqpoint{1.107680in}{0.689978in}}%
\pgfpathlineto{\pgfqpoint{1.108305in}{1.110985in}}%
\pgfpathlineto{\pgfqpoint{1.108876in}{1.317964in}}%
\pgfpathlineto{\pgfqpoint{1.109482in}{1.178846in}}%
\pgfpathlineto{\pgfqpoint{1.109732in}{1.067144in}}%
\pgfpathlineto{\pgfqpoint{1.110178in}{1.338721in}}%
\pgfpathlineto{\pgfqpoint{1.110356in}{1.409577in}}%
\pgfpathlineto{\pgfqpoint{1.110803in}{0.893154in}}%
\pgfpathlineto{\pgfqpoint{1.111052in}{0.715488in}}%
\pgfpathlineto{\pgfqpoint{1.111534in}{1.021642in}}%
\pgfpathlineto{\pgfqpoint{1.112230in}{1.304437in}}%
\pgfpathlineto{\pgfqpoint{1.112872in}{1.156758in}}%
\pgfpathlineto{\pgfqpoint{1.113086in}{1.055424in}}%
\pgfpathlineto{\pgfqpoint{1.113657in}{1.361163in}}%
\pgfpathlineto{\pgfqpoint{1.113728in}{1.400372in}}%
\pgfpathlineto{\pgfqpoint{1.114103in}{0.950654in}}%
\pgfpathlineto{\pgfqpoint{1.114424in}{0.728963in}}%
\pgfpathlineto{\pgfqpoint{1.115013in}{1.114551in}}%
\pgfpathlineto{\pgfqpoint{1.115584in}{1.304716in}}%
\pgfpathlineto{\pgfqpoint{1.116173in}{1.168969in}}%
\pgfpathlineto{\pgfqpoint{1.116458in}{1.045823in}}%
\pgfpathlineto{\pgfqpoint{1.116868in}{1.262194in}}%
\pgfpathlineto{\pgfqpoint{1.117100in}{1.379990in}}%
\pgfpathlineto{\pgfqpoint{1.117564in}{0.905259in}}%
\pgfpathlineto{\pgfqpoint{1.117796in}{0.738759in}}%
\pgfpathlineto{\pgfqpoint{1.118331in}{1.085665in}}%
\pgfpathlineto{\pgfqpoint{1.118956in}{1.313174in}}%
\pgfpathlineto{\pgfqpoint{1.119544in}{1.164561in}}%
\pgfpathlineto{\pgfqpoint{1.119848in}{1.038680in}}%
\pgfpathlineto{\pgfqpoint{1.120151in}{1.222257in}}%
\pgfpathlineto{\pgfqpoint{1.120472in}{1.358386in}}%
\pgfpathlineto{\pgfqpoint{1.120651in}{1.164451in}}%
\pgfpathlineto{\pgfqpoint{1.121150in}{0.740721in}}%
\pgfpathlineto{\pgfqpoint{1.121757in}{1.147384in}}%
\pgfpathlineto{\pgfqpoint{1.122310in}{1.325768in}}%
\pgfpathlineto{\pgfqpoint{1.122756in}{1.123988in}}%
\pgfpathlineto{\pgfqpoint{1.122881in}{1.169546in}}%
\pgfpathlineto{\pgfqpoint{1.123095in}{1.071589in}}%
\pgfpathlineto{\pgfqpoint{1.123220in}{1.035722in}}%
\pgfpathlineto{\pgfqpoint{1.123773in}{1.306885in}}%
\pgfpathlineto{\pgfqpoint{1.123844in}{1.343458in}}%
\pgfpathlineto{\pgfqpoint{1.124308in}{0.886930in}}%
\pgfpathlineto{\pgfqpoint{1.124522in}{0.725972in}}%
\pgfpathlineto{\pgfqpoint{1.125075in}{1.103315in}}%
\pgfpathlineto{\pgfqpoint{1.125664in}{1.339233in}}%
\pgfpathlineto{\pgfqpoint{1.126253in}{1.174668in}}%
\pgfpathlineto{\pgfqpoint{1.126288in}{1.176253in}}%
\pgfpathlineto{\pgfqpoint{1.126342in}{1.156619in}}%
\pgfpathlineto{\pgfqpoint{1.126592in}{1.043036in}}%
\pgfpathlineto{\pgfqpoint{1.127216in}{1.336809in}}%
\pgfpathlineto{\pgfqpoint{1.127359in}{1.201130in}}%
\pgfpathlineto{\pgfqpoint{1.127894in}{0.707617in}}%
\pgfpathlineto{\pgfqpoint{1.128483in}{1.144007in}}%
\pgfpathlineto{\pgfqpoint{1.129018in}{1.341631in}}%
\pgfpathlineto{\pgfqpoint{1.129482in}{1.135363in}}%
\pgfpathlineto{\pgfqpoint{1.129625in}{1.176081in}}%
\pgfpathlineto{\pgfqpoint{1.129660in}{1.180461in}}%
\pgfpathlineto{\pgfqpoint{1.129928in}{1.059818in}}%
\pgfpathlineto{\pgfqpoint{1.129963in}{1.056440in}}%
\pgfpathlineto{\pgfqpoint{1.130178in}{1.174740in}}%
\pgfpathlineto{\pgfqpoint{1.130588in}{1.342069in}}%
\pgfpathlineto{\pgfqpoint{1.130980in}{0.882593in}}%
\pgfpathlineto{\pgfqpoint{1.131266in}{0.690768in}}%
\pgfpathlineto{\pgfqpoint{1.131605in}{0.995131in}}%
\pgfpathlineto{\pgfqpoint{1.132372in}{1.339729in}}%
\pgfpathlineto{\pgfqpoint{1.132943in}{1.166053in}}%
\pgfpathlineto{\pgfqpoint{1.133960in}{1.341889in}}%
\pgfpathlineto{\pgfqpoint{1.133335in}{1.074020in}}%
\pgfpathlineto{\pgfqpoint{1.134103in}{1.194403in}}%
\pgfpathlineto{\pgfqpoint{1.134638in}{0.686498in}}%
\pgfpathlineto{\pgfqpoint{1.135226in}{1.144051in}}%
\pgfpathlineto{\pgfqpoint{1.135744in}{1.326510in}}%
\pgfpathlineto{\pgfqpoint{1.136226in}{1.134607in}}%
\pgfpathlineto{\pgfqpoint{1.136368in}{1.182549in}}%
\pgfpathlineto{\pgfqpoint{1.137314in}{1.338102in}}%
\pgfpathlineto{\pgfqpoint{1.136707in}{1.094510in}}%
\pgfpathlineto{\pgfqpoint{1.137474in}{1.185443in}}%
\pgfpathlineto{\pgfqpoint{1.137992in}{0.688027in}}%
\pgfpathlineto{\pgfqpoint{1.138598in}{1.149470in}}%
\pgfpathlineto{\pgfqpoint{1.139098in}{1.303862in}}%
\pgfpathlineto{\pgfqpoint{1.139597in}{1.129273in}}%
\pgfpathlineto{\pgfqpoint{1.139740in}{1.179916in}}%
\pgfpathlineto{\pgfqpoint{1.140686in}{1.340087in}}%
\pgfpathlineto{\pgfqpoint{1.140079in}{1.109984in}}%
\pgfpathlineto{\pgfqpoint{1.140846in}{1.182077in}}%
\pgfpathlineto{\pgfqpoint{1.141364in}{0.697958in}}%
\pgfpathlineto{\pgfqpoint{1.141952in}{1.142084in}}%
\pgfpathlineto{\pgfqpoint{1.142452in}{1.287256in}}%
\pgfpathlineto{\pgfqpoint{1.142969in}{1.125187in}}%
\pgfpathlineto{\pgfqpoint{1.143094in}{1.173507in}}%
\pgfpathlineto{\pgfqpoint{1.144058in}{1.329203in}}%
\pgfpathlineto{\pgfqpoint{1.143451in}{1.115944in}}%
\pgfpathlineto{\pgfqpoint{1.144218in}{1.178385in}}%
\pgfpathlineto{\pgfqpoint{1.144736in}{0.713010in}}%
\pgfpathlineto{\pgfqpoint{1.145324in}{1.150574in}}%
\pgfpathlineto{\pgfqpoint{1.145824in}{1.272508in}}%
\pgfpathlineto{\pgfqpoint{1.146341in}{1.122127in}}%
\pgfpathlineto{\pgfqpoint{1.146448in}{1.164896in}}%
\pgfpathlineto{\pgfqpoint{1.147447in}{1.319935in}}%
\pgfpathlineto{\pgfqpoint{1.146805in}{1.116945in}}%
\pgfpathlineto{\pgfqpoint{1.147572in}{1.204900in}}%
\pgfpathlineto{\pgfqpoint{1.148108in}{0.729651in}}%
\pgfpathlineto{\pgfqpoint{1.148696in}{1.155778in}}%
\pgfpathlineto{\pgfqpoint{1.149178in}{1.268048in}}%
\pgfpathlineto{\pgfqpoint{1.149713in}{1.122001in}}%
\pgfpathlineto{\pgfqpoint{1.149802in}{1.156332in}}%
\pgfpathlineto{\pgfqpoint{1.150819in}{1.311230in}}%
\pgfpathlineto{\pgfqpoint{1.150177in}{1.111686in}}%
\pgfpathlineto{\pgfqpoint{1.150944in}{1.202086in}}%
\pgfpathlineto{\pgfqpoint{1.151479in}{0.744092in}}%
\pgfpathlineto{\pgfqpoint{1.152068in}{1.158216in}}%
\pgfpathlineto{\pgfqpoint{1.152550in}{1.272644in}}%
\pgfpathlineto{\pgfqpoint{1.153085in}{1.125684in}}%
\pgfpathlineto{\pgfqpoint{1.153174in}{1.155981in}}%
\pgfpathlineto{\pgfqpoint{1.154209in}{1.300374in}}%
\pgfpathlineto{\pgfqpoint{1.153549in}{1.104642in}}%
\pgfpathlineto{\pgfqpoint{1.154316in}{1.195571in}}%
\pgfpathlineto{\pgfqpoint{1.154851in}{0.753529in}}%
\pgfpathlineto{\pgfqpoint{1.155440in}{1.164055in}}%
\pgfpathlineto{\pgfqpoint{1.155922in}{1.279636in}}%
\pgfpathlineto{\pgfqpoint{1.156457in}{1.127610in}}%
\pgfpathlineto{\pgfqpoint{1.156511in}{1.140570in}}%
\pgfpathlineto{\pgfqpoint{1.157581in}{1.298101in}}%
\pgfpathlineto{\pgfqpoint{1.156903in}{1.099531in}}%
\pgfpathlineto{\pgfqpoint{1.157688in}{1.191591in}}%
\pgfpathlineto{\pgfqpoint{1.158223in}{0.748696in}}%
\pgfpathlineto{\pgfqpoint{1.158812in}{1.172989in}}%
\pgfpathlineto{\pgfqpoint{1.159276in}{1.290839in}}%
\pgfpathlineto{\pgfqpoint{1.159829in}{1.127822in}}%
\pgfpathlineto{\pgfqpoint{1.159847in}{1.128631in}}%
\pgfpathlineto{\pgfqpoint{1.160953in}{1.299134in}}%
\pgfpathlineto{\pgfqpoint{1.160275in}{1.098036in}}%
\pgfpathlineto{\pgfqpoint{1.161078in}{1.168214in}}%
\pgfpathlineto{\pgfqpoint{1.161595in}{0.738088in}}%
\pgfpathlineto{\pgfqpoint{1.162166in}{1.166219in}}%
\pgfpathlineto{\pgfqpoint{1.162648in}{1.299564in}}%
\pgfpathlineto{\pgfqpoint{1.163201in}{1.126788in}}%
\pgfpathlineto{\pgfqpoint{1.163219in}{1.128095in}}%
\pgfpathlineto{\pgfqpoint{1.164325in}{1.296468in}}%
\pgfpathlineto{\pgfqpoint{1.163754in}{1.100821in}}%
\pgfpathlineto{\pgfqpoint{1.164450in}{1.164554in}}%
\pgfpathlineto{\pgfqpoint{1.164949in}{0.733592in}}%
\pgfpathlineto{\pgfqpoint{1.165538in}{1.174631in}}%
\pgfpathlineto{\pgfqpoint{1.166020in}{1.298962in}}%
\pgfpathlineto{\pgfqpoint{1.166573in}{1.124733in}}%
\pgfpathlineto{\pgfqpoint{1.167679in}{1.294372in}}%
\pgfpathlineto{\pgfqpoint{1.167126in}{1.113306in}}%
\pgfpathlineto{\pgfqpoint{1.167822in}{1.156516in}}%
\pgfpathlineto{\pgfqpoint{1.168321in}{0.724393in}}%
\pgfpathlineto{\pgfqpoint{1.168892in}{1.171878in}}%
\pgfpathlineto{\pgfqpoint{1.169392in}{1.294438in}}%
\pgfpathlineto{\pgfqpoint{1.169927in}{1.125288in}}%
\pgfpathlineto{\pgfqpoint{1.169945in}{1.124350in}}%
\pgfpathlineto{\pgfqpoint{1.170159in}{1.165218in}}%
\pgfpathlineto{\pgfqpoint{1.170533in}{1.140438in}}%
\pgfpathlineto{\pgfqpoint{1.171051in}{1.290587in}}%
\pgfpathlineto{\pgfqpoint{1.171336in}{0.959971in}}%
\pgfpathlineto{\pgfqpoint{1.171693in}{0.722945in}}%
\pgfpathlineto{\pgfqpoint{1.172264in}{1.179245in}}%
\pgfpathlineto{\pgfqpoint{1.172746in}{1.283250in}}%
\pgfpathlineto{\pgfqpoint{1.173281in}{1.125328in}}%
\pgfpathlineto{\pgfqpoint{1.173299in}{1.121486in}}%
\pgfpathlineto{\pgfqpoint{1.173977in}{1.206097in}}%
\pgfpathlineto{\pgfqpoint{1.174423in}{1.286361in}}%
\pgfpathlineto{\pgfqpoint{1.174690in}{0.981012in}}%
\pgfpathlineto{\pgfqpoint{1.175065in}{0.729294in}}%
\pgfpathlineto{\pgfqpoint{1.175636in}{1.185802in}}%
\pgfpathlineto{\pgfqpoint{1.176118in}{1.272163in}}%
\pgfpathlineto{\pgfqpoint{1.176635in}{1.132234in}}%
\pgfpathlineto{\pgfqpoint{1.176671in}{1.122737in}}%
\pgfpathlineto{\pgfqpoint{1.177438in}{1.245527in}}%
\pgfpathlineto{\pgfqpoint{1.177795in}{1.279173in}}%
\pgfpathlineto{\pgfqpoint{1.177991in}{1.073773in}}%
\pgfpathlineto{\pgfqpoint{1.178437in}{0.737568in}}%
\pgfpathlineto{\pgfqpoint{1.179008in}{1.189822in}}%
\pgfpathlineto{\pgfqpoint{1.179489in}{1.263522in}}%
\pgfpathlineto{\pgfqpoint{1.179989in}{1.141469in}}%
\pgfpathlineto{\pgfqpoint{1.180042in}{1.124742in}}%
\pgfpathlineto{\pgfqpoint{1.180863in}{1.260305in}}%
\pgfpathlineto{\pgfqpoint{1.181166in}{1.271188in}}%
\pgfpathlineto{\pgfqpoint{1.181273in}{1.177191in}}%
\pgfpathlineto{\pgfqpoint{1.181809in}{0.748101in}}%
\pgfpathlineto{\pgfqpoint{1.182362in}{1.180423in}}%
\pgfpathlineto{\pgfqpoint{1.182861in}{1.254300in}}%
\pgfpathlineto{\pgfqpoint{1.183379in}{1.134708in}}%
\pgfpathlineto{\pgfqpoint{1.183414in}{1.128246in}}%
\pgfpathlineto{\pgfqpoint{1.184146in}{1.226272in}}%
\pgfpathlineto{\pgfqpoint{1.184538in}{1.268215in}}%
\pgfpathlineto{\pgfqpoint{1.184735in}{1.074092in}}%
\pgfpathlineto{\pgfqpoint{1.185181in}{0.758212in}}%
\pgfpathlineto{\pgfqpoint{1.185752in}{1.192289in}}%
\pgfpathlineto{\pgfqpoint{1.186215in}{1.252914in}}%
\pgfpathlineto{\pgfqpoint{1.186715in}{1.155933in}}%
\pgfpathlineto{\pgfqpoint{1.186786in}{1.131129in}}%
\pgfpathlineto{\pgfqpoint{1.187625in}{1.248453in}}%
\pgfpathlineto{\pgfqpoint{1.187660in}{1.248213in}}%
\pgfpathlineto{\pgfqpoint{1.187678in}{1.248491in}}%
\pgfpathlineto{\pgfqpoint{1.187750in}{1.235142in}}%
\pgfpathlineto{\pgfqpoint{1.188553in}{0.763590in}}%
\pgfpathlineto{\pgfqpoint{1.187910in}{1.261008in}}%
\pgfpathlineto{\pgfqpoint{1.189052in}{1.153281in}}%
\pgfpathlineto{\pgfqpoint{1.189587in}{1.253606in}}%
\pgfpathlineto{\pgfqpoint{1.190122in}{1.139908in}}%
\pgfpathlineto{\pgfqpoint{1.190176in}{1.131961in}}%
\pgfpathlineto{\pgfqpoint{1.190907in}{1.222367in}}%
\pgfpathlineto{\pgfqpoint{1.191282in}{1.256871in}}%
\pgfpathlineto{\pgfqpoint{1.191514in}{1.030974in}}%
\pgfpathlineto{\pgfqpoint{1.191924in}{0.767797in}}%
\pgfpathlineto{\pgfqpoint{1.192495in}{1.193976in}}%
\pgfpathlineto{\pgfqpoint{1.192959in}{1.257725in}}%
\pgfpathlineto{\pgfqpoint{1.193459in}{1.157131in}}%
\pgfpathlineto{\pgfqpoint{1.193548in}{1.130546in}}%
\pgfpathlineto{\pgfqpoint{1.194386in}{1.243079in}}%
\pgfpathlineto{\pgfqpoint{1.194654in}{1.251935in}}%
\pgfpathlineto{\pgfqpoint{1.194743in}{1.190730in}}%
\pgfpathlineto{\pgfqpoint{1.195296in}{0.769249in}}%
\pgfpathlineto{\pgfqpoint{1.195849in}{1.189293in}}%
\pgfpathlineto{\pgfqpoint{1.196331in}{1.258791in}}%
\pgfpathlineto{\pgfqpoint{1.196848in}{1.149701in}}%
\pgfpathlineto{\pgfqpoint{1.196938in}{1.129922in}}%
\pgfpathlineto{\pgfqpoint{1.197740in}{1.241696in}}%
\pgfpathlineto{\pgfqpoint{1.198008in}{1.245761in}}%
\pgfpathlineto{\pgfqpoint{1.198115in}{1.185060in}}%
\pgfpathlineto{\pgfqpoint{1.198668in}{0.769415in}}%
\pgfpathlineto{\pgfqpoint{1.199203in}{1.186773in}}%
\pgfpathlineto{\pgfqpoint{1.199685in}{1.258254in}}%
\pgfpathlineto{\pgfqpoint{1.200220in}{1.151089in}}%
\pgfpathlineto{\pgfqpoint{1.200292in}{1.130511in}}%
\pgfpathlineto{\pgfqpoint{1.201112in}{1.242434in}}%
\pgfpathlineto{\pgfqpoint{1.201166in}{1.246729in}}%
\pgfpathlineto{\pgfqpoint{1.201505in}{1.164154in}}%
\pgfpathlineto{\pgfqpoint{1.202040in}{0.768078in}}%
\pgfpathlineto{\pgfqpoint{1.202575in}{1.189808in}}%
\pgfpathlineto{\pgfqpoint{1.202879in}{1.257690in}}%
\pgfpathlineto{\pgfqpoint{1.203592in}{1.153205in}}%
\pgfpathlineto{\pgfqpoint{1.203664in}{1.132957in}}%
\pgfpathlineto{\pgfqpoint{1.204484in}{1.242138in}}%
\pgfpathlineto{\pgfqpoint{1.204520in}{1.245784in}}%
\pgfpathlineto{\pgfqpoint{1.204877in}{1.161097in}}%
\pgfpathlineto{\pgfqpoint{1.205412in}{0.768058in}}%
\pgfpathlineto{\pgfqpoint{1.205947in}{1.191199in}}%
\pgfpathlineto{\pgfqpoint{1.206233in}{1.256919in}}%
\pgfpathlineto{\pgfqpoint{1.206964in}{1.153961in}}%
\pgfpathlineto{\pgfqpoint{1.207036in}{1.134046in}}%
\pgfpathlineto{\pgfqpoint{1.207856in}{1.242201in}}%
\pgfpathlineto{\pgfqpoint{1.207892in}{1.245782in}}%
\pgfpathlineto{\pgfqpoint{1.208249in}{1.158455in}}%
\pgfpathlineto{\pgfqpoint{1.208784in}{0.772570in}}%
\pgfpathlineto{\pgfqpoint{1.209319in}{1.191692in}}%
\pgfpathlineto{\pgfqpoint{1.209605in}{1.256621in}}%
\pgfpathlineto{\pgfqpoint{1.210336in}{1.152069in}}%
\pgfpathlineto{\pgfqpoint{1.210407in}{1.133404in}}%
\pgfpathlineto{\pgfqpoint{1.211228in}{1.243466in}}%
\pgfpathlineto{\pgfqpoint{1.211264in}{1.247165in}}%
\pgfpathlineto{\pgfqpoint{1.211621in}{1.155584in}}%
\pgfpathlineto{\pgfqpoint{1.212156in}{0.773799in}}%
\pgfpathlineto{\pgfqpoint{1.212691in}{1.191253in}}%
\pgfpathlineto{\pgfqpoint{1.212976in}{1.254033in}}%
\pgfpathlineto{\pgfqpoint{1.213708in}{1.154148in}}%
\pgfpathlineto{\pgfqpoint{1.213779in}{1.135453in}}%
\pgfpathlineto{\pgfqpoint{1.214600in}{1.245061in}}%
\pgfpathlineto{\pgfqpoint{1.214636in}{1.248050in}}%
\pgfpathlineto{\pgfqpoint{1.214992in}{1.154308in}}%
\pgfpathlineto{\pgfqpoint{1.215528in}{0.777205in}}%
\pgfpathlineto{\pgfqpoint{1.216063in}{1.193241in}}%
\pgfpathlineto{\pgfqpoint{1.216348in}{1.254183in}}%
\pgfpathlineto{\pgfqpoint{1.217080in}{1.151949in}}%
\pgfpathlineto{\pgfqpoint{1.217151in}{1.134075in}}%
\pgfpathlineto{\pgfqpoint{1.217954in}{1.243683in}}%
\pgfpathlineto{\pgfqpoint{1.218008in}{1.249591in}}%
\pgfpathlineto{\pgfqpoint{1.218382in}{1.136703in}}%
\pgfpathlineto{\pgfqpoint{1.218917in}{0.780222in}}%
\pgfpathlineto{\pgfqpoint{1.219417in}{1.190173in}}%
\pgfpathlineto{\pgfqpoint{1.219702in}{1.256016in}}%
\pgfpathlineto{\pgfqpoint{1.220434in}{1.156996in}}%
\pgfpathlineto{\pgfqpoint{1.220523in}{1.132058in}}%
\pgfpathlineto{\pgfqpoint{1.221344in}{1.246465in}}%
\pgfpathlineto{\pgfqpoint{1.221362in}{1.248378in}}%
\pgfpathlineto{\pgfqpoint{1.221701in}{1.180927in}}%
\pgfpathlineto{\pgfqpoint{1.222289in}{0.781638in}}%
\pgfpathlineto{\pgfqpoint{1.222789in}{1.194842in}}%
\pgfpathlineto{\pgfqpoint{1.223074in}{1.259510in}}%
\pgfpathlineto{\pgfqpoint{1.223806in}{1.154128in}}%
\pgfpathlineto{\pgfqpoint{1.223895in}{1.130557in}}%
\pgfpathlineto{\pgfqpoint{1.224698in}{1.242538in}}%
\pgfpathlineto{\pgfqpoint{1.224734in}{1.246583in}}%
\pgfpathlineto{\pgfqpoint{1.225108in}{1.144205in}}%
\pgfpathlineto{\pgfqpoint{1.225661in}{0.785217in}}%
\pgfpathlineto{\pgfqpoint{1.226161in}{1.197941in}}%
\pgfpathlineto{\pgfqpoint{1.226446in}{1.258959in}}%
\pgfpathlineto{\pgfqpoint{1.227160in}{1.160664in}}%
\pgfpathlineto{\pgfqpoint{1.227267in}{1.131021in}}%
\pgfpathlineto{\pgfqpoint{1.228070in}{1.241473in}}%
\pgfpathlineto{\pgfqpoint{1.228105in}{1.244960in}}%
\pgfpathlineto{\pgfqpoint{1.228462in}{1.153027in}}%
\pgfpathlineto{\pgfqpoint{1.229033in}{0.792607in}}%
\pgfpathlineto{\pgfqpoint{1.229515in}{1.194379in}}%
\pgfpathlineto{\pgfqpoint{1.229818in}{1.255431in}}%
\pgfpathlineto{\pgfqpoint{1.230532in}{1.162088in}}%
\pgfpathlineto{\pgfqpoint{1.230639in}{1.134593in}}%
\pgfpathlineto{\pgfqpoint{1.231442in}{1.237383in}}%
\pgfpathlineto{\pgfqpoint{1.231477in}{1.239465in}}%
\pgfpathlineto{\pgfqpoint{1.231798in}{1.172044in}}%
\pgfpathlineto{\pgfqpoint{1.232405in}{0.797214in}}%
\pgfpathlineto{\pgfqpoint{1.232869in}{1.187462in}}%
\pgfpathlineto{\pgfqpoint{1.233172in}{1.254228in}}%
\pgfpathlineto{\pgfqpoint{1.233922in}{1.160842in}}%
\pgfpathlineto{\pgfqpoint{1.234011in}{1.140600in}}%
\pgfpathlineto{\pgfqpoint{1.234831in}{1.233012in}}%
\pgfpathlineto{\pgfqpoint{1.234849in}{1.232984in}}%
\pgfpathlineto{\pgfqpoint{1.235367in}{0.991626in}}%
\pgfpathlineto{\pgfqpoint{1.235777in}{0.803627in}}%
\pgfpathlineto{\pgfqpoint{1.236277in}{1.201548in}}%
\pgfpathlineto{\pgfqpoint{1.236544in}{1.248008in}}%
\pgfpathlineto{\pgfqpoint{1.237293in}{1.166773in}}%
\pgfpathlineto{\pgfqpoint{1.237400in}{1.148195in}}%
\pgfpathlineto{\pgfqpoint{1.238203in}{1.227688in}}%
\pgfpathlineto{\pgfqpoint{1.238221in}{1.227186in}}%
\pgfpathlineto{\pgfqpoint{1.238792in}{0.939383in}}%
\pgfpathlineto{\pgfqpoint{1.239149in}{0.808835in}}%
\pgfpathlineto{\pgfqpoint{1.239595in}{1.171261in}}%
\pgfpathlineto{\pgfqpoint{1.239916in}{1.244483in}}%
\pgfpathlineto{\pgfqpoint{1.240683in}{1.164978in}}%
\pgfpathlineto{\pgfqpoint{1.240790in}{1.150412in}}%
\pgfpathlineto{\pgfqpoint{1.241575in}{1.223970in}}%
\pgfpathlineto{\pgfqpoint{1.241593in}{1.223391in}}%
\pgfpathlineto{\pgfqpoint{1.242200in}{0.918266in}}%
\pgfpathlineto{\pgfqpoint{1.242521in}{0.811991in}}%
\pgfpathlineto{\pgfqpoint{1.242913in}{1.129577in}}%
\pgfpathlineto{\pgfqpoint{1.243288in}{1.243932in}}%
\pgfpathlineto{\pgfqpoint{1.244055in}{1.168122in}}%
\pgfpathlineto{\pgfqpoint{1.244180in}{1.149512in}}%
\pgfpathlineto{\pgfqpoint{1.244947in}{1.223465in}}%
\pgfpathlineto{\pgfqpoint{1.245126in}{1.175295in}}%
\pgfpathlineto{\pgfqpoint{1.245143in}{1.175180in}}%
\pgfpathlineto{\pgfqpoint{1.245197in}{1.180844in}}%
\pgfpathlineto{\pgfqpoint{1.245215in}{1.182403in}}%
\pgfpathlineto{\pgfqpoint{1.245375in}{1.104671in}}%
\pgfpathlineto{\pgfqpoint{1.245893in}{0.812483in}}%
\pgfpathlineto{\pgfqpoint{1.246374in}{1.181938in}}%
\pgfpathlineto{\pgfqpoint{1.246660in}{1.243747in}}%
\pgfpathlineto{\pgfqpoint{1.247445in}{1.163208in}}%
\pgfpathlineto{\pgfqpoint{1.247552in}{1.147695in}}%
\pgfpathlineto{\pgfqpoint{1.248283in}{1.221339in}}%
\pgfpathlineto{\pgfqpoint{1.248319in}{1.226531in}}%
\pgfpathlineto{\pgfqpoint{1.248747in}{1.106890in}}%
\pgfpathlineto{\pgfqpoint{1.249265in}{0.809127in}}%
\pgfpathlineto{\pgfqpoint{1.249746in}{1.183426in}}%
\pgfpathlineto{\pgfqpoint{1.250032in}{1.247966in}}%
\pgfpathlineto{\pgfqpoint{1.250799in}{1.163732in}}%
\pgfpathlineto{\pgfqpoint{1.250924in}{1.139615in}}%
\pgfpathlineto{\pgfqpoint{1.251691in}{1.230486in}}%
\pgfpathlineto{\pgfqpoint{1.252012in}{1.175632in}}%
\pgfpathlineto{\pgfqpoint{1.252636in}{0.808133in}}%
\pgfpathlineto{\pgfqpoint{1.253100in}{1.177912in}}%
\pgfpathlineto{\pgfqpoint{1.253404in}{1.254259in}}%
\pgfpathlineto{\pgfqpoint{1.254171in}{1.158398in}}%
\pgfpathlineto{\pgfqpoint{1.254296in}{1.135146in}}%
\pgfpathlineto{\pgfqpoint{1.255009in}{1.224952in}}%
\pgfpathlineto{\pgfqpoint{1.255063in}{1.234307in}}%
\pgfpathlineto{\pgfqpoint{1.255527in}{1.073939in}}%
\pgfpathlineto{\pgfqpoint{1.256008in}{0.807685in}}%
\pgfpathlineto{\pgfqpoint{1.256490in}{1.193751in}}%
\pgfpathlineto{\pgfqpoint{1.256758in}{1.257561in}}%
\pgfpathlineto{\pgfqpoint{1.257525in}{1.161581in}}%
\pgfpathlineto{\pgfqpoint{1.257668in}{1.134323in}}%
\pgfpathlineto{\pgfqpoint{1.258381in}{1.227706in}}%
\pgfpathlineto{\pgfqpoint{1.258435in}{1.235578in}}%
\pgfpathlineto{\pgfqpoint{1.258863in}{1.094603in}}%
\pgfpathlineto{\pgfqpoint{1.259380in}{0.811411in}}%
\pgfpathlineto{\pgfqpoint{1.259844in}{1.189897in}}%
\pgfpathlineto{\pgfqpoint{1.260112in}{1.259987in}}%
\pgfpathlineto{\pgfqpoint{1.260897in}{1.158568in}}%
\pgfpathlineto{\pgfqpoint{1.261039in}{1.134267in}}%
\pgfpathlineto{\pgfqpoint{1.261717in}{1.218490in}}%
\pgfpathlineto{\pgfqpoint{1.261807in}{1.233797in}}%
\pgfpathlineto{\pgfqpoint{1.262253in}{1.074558in}}%
\pgfpathlineto{\pgfqpoint{1.262734in}{0.817279in}}%
\pgfpathlineto{\pgfqpoint{1.263198in}{1.183542in}}%
\pgfpathlineto{\pgfqpoint{1.263430in}{1.259405in}}%
\pgfpathlineto{\pgfqpoint{1.264001in}{1.152325in}}%
\pgfpathlineto{\pgfqpoint{1.264269in}{1.161252in}}%
\pgfpathlineto{\pgfqpoint{1.264394in}{1.140037in}}%
\pgfpathlineto{\pgfqpoint{1.264750in}{1.217664in}}%
\pgfpathlineto{\pgfqpoint{1.265089in}{1.214105in}}%
\pgfpathlineto{\pgfqpoint{1.265161in}{1.227134in}}%
\pgfpathlineto{\pgfqpoint{1.265607in}{1.081662in}}%
\pgfpathlineto{\pgfqpoint{1.266106in}{0.825704in}}%
\pgfpathlineto{\pgfqpoint{1.266570in}{1.184522in}}%
\pgfpathlineto{\pgfqpoint{1.266784in}{1.257128in}}%
\pgfpathlineto{\pgfqpoint{1.267373in}{1.150494in}}%
\pgfpathlineto{\pgfqpoint{1.267641in}{1.166799in}}%
\pgfpathlineto{\pgfqpoint{1.267765in}{1.147242in}}%
\pgfpathlineto{\pgfqpoint{1.268104in}{1.221716in}}%
\pgfpathlineto{\pgfqpoint{1.268461in}{1.207592in}}%
\pgfpathlineto{\pgfqpoint{1.268533in}{1.218679in}}%
\pgfpathlineto{\pgfqpoint{1.268961in}{1.089819in}}%
\pgfpathlineto{\pgfqpoint{1.269478in}{0.833838in}}%
\pgfpathlineto{\pgfqpoint{1.269924in}{1.171658in}}%
\pgfpathlineto{\pgfqpoint{1.270156in}{1.252192in}}%
\pgfpathlineto{\pgfqpoint{1.270745in}{1.149229in}}%
\pgfpathlineto{\pgfqpoint{1.271030in}{1.170466in}}%
\pgfpathlineto{\pgfqpoint{1.271137in}{1.151990in}}%
\pgfpathlineto{\pgfqpoint{1.271476in}{1.222744in}}%
\pgfpathlineto{\pgfqpoint{1.271869in}{1.209262in}}%
\pgfpathlineto{\pgfqpoint{1.271904in}{1.212941in}}%
\pgfpathlineto{\pgfqpoint{1.272297in}{1.111847in}}%
\pgfpathlineto{\pgfqpoint{1.272850in}{0.841949in}}%
\pgfpathlineto{\pgfqpoint{1.273296in}{1.167416in}}%
\pgfpathlineto{\pgfqpoint{1.273528in}{1.246284in}}%
\pgfpathlineto{\pgfqpoint{1.274099in}{1.152762in}}%
\pgfpathlineto{\pgfqpoint{1.274402in}{1.176652in}}%
\pgfpathlineto{\pgfqpoint{1.276222in}{0.845494in}}%
\pgfpathlineto{\pgfqpoint{1.274848in}{1.218808in}}%
\pgfpathlineto{\pgfqpoint{1.276507in}{1.023446in}}%
\pgfpathlineto{\pgfqpoint{1.276918in}{1.245424in}}%
\pgfpathlineto{\pgfqpoint{1.277774in}{1.181036in}}%
\pgfpathlineto{\pgfqpoint{1.279612in}{0.846211in}}%
\pgfpathlineto{\pgfqpoint{1.278220in}{1.215500in}}%
\pgfpathlineto{\pgfqpoint{1.279844in}{0.988051in}}%
\pgfpathlineto{\pgfqpoint{1.280307in}{1.244928in}}%
\pgfpathlineto{\pgfqpoint{1.281164in}{1.177227in}}%
\pgfpathlineto{\pgfqpoint{1.282984in}{0.846198in}}%
\pgfpathlineto{\pgfqpoint{1.281610in}{1.213859in}}%
\pgfpathlineto{\pgfqpoint{1.283162in}{0.936591in}}%
\pgfpathlineto{\pgfqpoint{1.283697in}{1.248390in}}%
\pgfpathlineto{\pgfqpoint{1.284536in}{1.174077in}}%
\pgfpathlineto{\pgfqpoint{1.286356in}{0.846354in}}%
\pgfpathlineto{\pgfqpoint{1.284982in}{1.211571in}}%
\pgfpathlineto{\pgfqpoint{1.286516in}{0.921794in}}%
\pgfpathlineto{\pgfqpoint{1.287069in}{1.251870in}}%
\pgfpathlineto{\pgfqpoint{1.287908in}{1.171583in}}%
\pgfpathlineto{\pgfqpoint{1.289727in}{0.847270in}}%
\pgfpathlineto{\pgfqpoint{1.288354in}{1.213451in}}%
\pgfpathlineto{\pgfqpoint{1.289870in}{0.906850in}}%
\pgfpathlineto{\pgfqpoint{1.290423in}{1.254540in}}%
\pgfpathlineto{\pgfqpoint{1.291280in}{1.170199in}}%
\pgfpathlineto{\pgfqpoint{1.293099in}{0.848297in}}%
\pgfpathlineto{\pgfqpoint{1.291726in}{1.217582in}}%
\pgfpathlineto{\pgfqpoint{1.293260in}{0.924370in}}%
\pgfpathlineto{\pgfqpoint{1.293795in}{1.257162in}}%
\pgfpathlineto{\pgfqpoint{1.294651in}{1.169032in}}%
\pgfpathlineto{\pgfqpoint{1.296471in}{0.853454in}}%
\pgfpathlineto{\pgfqpoint{1.295097in}{1.220362in}}%
\pgfpathlineto{\pgfqpoint{1.296632in}{0.928720in}}%
\pgfpathlineto{\pgfqpoint{1.297149in}{1.258064in}}%
\pgfpathlineto{\pgfqpoint{1.298023in}{1.170025in}}%
\pgfpathlineto{\pgfqpoint{1.299843in}{0.858570in}}%
\pgfpathlineto{\pgfqpoint{1.298469in}{1.225139in}}%
\pgfpathlineto{\pgfqpoint{1.300022in}{0.946872in}}%
\pgfpathlineto{\pgfqpoint{1.300521in}{1.256338in}}%
\pgfpathlineto{\pgfqpoint{1.301395in}{1.172416in}}%
\pgfpathlineto{\pgfqpoint{1.303215in}{0.863926in}}%
\pgfpathlineto{\pgfqpoint{1.301841in}{1.228490in}}%
\pgfpathlineto{\pgfqpoint{1.303393in}{0.950398in}}%
\pgfpathlineto{\pgfqpoint{1.303893in}{1.254903in}}%
\pgfpathlineto{\pgfqpoint{1.304767in}{1.174327in}}%
\pgfpathlineto{\pgfqpoint{1.306587in}{0.869272in}}%
\pgfpathlineto{\pgfqpoint{1.305213in}{1.228024in}}%
\pgfpathlineto{\pgfqpoint{1.306783in}{0.968674in}}%
\pgfpathlineto{\pgfqpoint{1.307265in}{1.253205in}}%
\pgfpathlineto{\pgfqpoint{1.308139in}{1.177849in}}%
\pgfpathlineto{\pgfqpoint{1.309959in}{0.871898in}}%
\pgfpathlineto{\pgfqpoint{1.308585in}{1.228624in}}%
\pgfpathlineto{\pgfqpoint{1.310173in}{0.983567in}}%
\pgfpathlineto{\pgfqpoint{1.310637in}{1.253646in}}%
\pgfpathlineto{\pgfqpoint{1.311511in}{1.179778in}}%
\pgfpathlineto{\pgfqpoint{1.313331in}{0.874031in}}%
\pgfpathlineto{\pgfqpoint{1.311957in}{1.228844in}}%
\pgfpathlineto{\pgfqpoint{1.313563in}{1.000080in}}%
\pgfpathlineto{\pgfqpoint{1.314009in}{1.253557in}}%
\pgfpathlineto{\pgfqpoint{1.314865in}{1.182328in}}%
\pgfpathlineto{\pgfqpoint{1.316720in}{0.877949in}}%
\pgfpathlineto{\pgfqpoint{1.315329in}{1.229351in}}%
\pgfpathlineto{\pgfqpoint{1.316970in}{1.032381in}}%
\pgfpathlineto{\pgfqpoint{1.317381in}{1.250770in}}%
\pgfpathlineto{\pgfqpoint{1.318237in}{1.183507in}}%
\pgfpathlineto{\pgfqpoint{1.320092in}{0.881855in}}%
\pgfpathlineto{\pgfqpoint{1.318701in}{1.229277in}}%
\pgfpathlineto{\pgfqpoint{1.320360in}{1.050221in}}%
\pgfpathlineto{\pgfqpoint{1.320752in}{1.251430in}}%
\pgfpathlineto{\pgfqpoint{1.321609in}{1.185569in}}%
\pgfpathlineto{\pgfqpoint{1.323464in}{0.884454in}}%
\pgfpathlineto{\pgfqpoint{1.322073in}{1.228590in}}%
\pgfpathlineto{\pgfqpoint{1.323750in}{1.065485in}}%
\pgfpathlineto{\pgfqpoint{1.324124in}{1.248808in}}%
\pgfpathlineto{\pgfqpoint{1.324999in}{1.185851in}}%
\pgfpathlineto{\pgfqpoint{1.326836in}{0.887020in}}%
\pgfpathlineto{\pgfqpoint{1.325445in}{1.228232in}}%
\pgfpathlineto{\pgfqpoint{1.327104in}{1.048022in}}%
\pgfpathlineto{\pgfqpoint{1.327496in}{1.247616in}}%
\pgfpathlineto{\pgfqpoint{1.328370in}{1.187187in}}%
\pgfpathlineto{\pgfqpoint{1.330208in}{0.889108in}}%
\pgfpathlineto{\pgfqpoint{1.328816in}{1.226266in}}%
\pgfpathlineto{\pgfqpoint{1.330494in}{1.062259in}}%
\pgfpathlineto{\pgfqpoint{1.330868in}{1.247432in}}%
\pgfpathlineto{\pgfqpoint{1.331760in}{1.184463in}}%
\pgfpathlineto{\pgfqpoint{1.333580in}{0.889345in}}%
\pgfpathlineto{\pgfqpoint{1.332188in}{1.224960in}}%
\pgfpathlineto{\pgfqpoint{1.333883in}{1.072586in}}%
\pgfpathlineto{\pgfqpoint{1.334240in}{1.246909in}}%
\pgfpathlineto{\pgfqpoint{1.335132in}{1.184357in}}%
\pgfpathlineto{\pgfqpoint{1.335560in}{1.224462in}}%
\pgfpathlineto{\pgfqpoint{1.336666in}{0.932951in}}%
\pgfpathlineto{\pgfqpoint{1.336952in}{0.890266in}}%
\pgfpathlineto{\pgfqpoint{1.337273in}{1.085088in}}%
\pgfpathlineto{\pgfqpoint{1.337630in}{1.248603in}}%
\pgfpathlineto{\pgfqpoint{1.338504in}{1.182903in}}%
\pgfpathlineto{\pgfqpoint{1.338932in}{1.224448in}}%
\pgfpathlineto{\pgfqpoint{1.340003in}{0.948194in}}%
\pgfpathlineto{\pgfqpoint{1.340324in}{0.892631in}}%
\pgfpathlineto{\pgfqpoint{1.340698in}{1.126784in}}%
\pgfpathlineto{\pgfqpoint{1.341002in}{1.250894in}}%
\pgfpathlineto{\pgfqpoint{1.341876in}{1.181431in}}%
\pgfpathlineto{\pgfqpoint{1.342322in}{1.226562in}}%
\pgfpathlineto{\pgfqpoint{1.343285in}{0.998665in}}%
\pgfpathlineto{\pgfqpoint{1.343678in}{0.893897in}}%
\pgfpathlineto{\pgfqpoint{1.344124in}{1.167952in}}%
\pgfpathlineto{\pgfqpoint{1.344374in}{1.251802in}}%
\pgfpathlineto{\pgfqpoint{1.344927in}{1.124354in}}%
\pgfpathlineto{\pgfqpoint{1.345248in}{1.181967in}}%
\pgfpathlineto{\pgfqpoint{1.345337in}{1.178269in}}%
\pgfpathlineto{\pgfqpoint{1.345605in}{1.220707in}}%
\pgfpathlineto{\pgfqpoint{1.345694in}{1.231217in}}%
\pgfpathlineto{\pgfqpoint{1.346283in}{1.128980in}}%
\pgfpathlineto{\pgfqpoint{1.347050in}{0.898497in}}%
\pgfpathlineto{\pgfqpoint{1.347424in}{1.117319in}}%
\pgfpathlineto{\pgfqpoint{1.347728in}{1.251381in}}%
\pgfpathlineto{\pgfqpoint{1.348620in}{1.183250in}}%
\pgfpathlineto{\pgfqpoint{1.348691in}{1.181144in}}%
\pgfpathlineto{\pgfqpoint{1.348905in}{1.206024in}}%
\pgfpathlineto{\pgfqpoint{1.349066in}{1.233979in}}%
\pgfpathlineto{\pgfqpoint{1.349654in}{1.125183in}}%
\pgfpathlineto{\pgfqpoint{1.350422in}{0.905893in}}%
\pgfpathlineto{\pgfqpoint{1.350778in}{1.107168in}}%
\pgfpathlineto{\pgfqpoint{1.351100in}{1.248794in}}%
\pgfpathlineto{\pgfqpoint{1.351992in}{1.187598in}}%
\pgfpathlineto{\pgfqpoint{1.352063in}{1.186116in}}%
\pgfpathlineto{\pgfqpoint{1.352224in}{1.200418in}}%
\pgfpathlineto{\pgfqpoint{1.352438in}{1.237497in}}%
\pgfpathlineto{\pgfqpoint{1.352955in}{1.146917in}}%
\pgfpathlineto{\pgfqpoint{1.353794in}{0.911782in}}%
\pgfpathlineto{\pgfqpoint{1.354150in}{1.106848in}}%
\pgfpathlineto{\pgfqpoint{1.354471in}{1.245325in}}%
\pgfpathlineto{\pgfqpoint{1.355381in}{1.193737in}}%
\pgfpathlineto{\pgfqpoint{1.355435in}{1.192946in}}%
\pgfpathlineto{\pgfqpoint{1.355613in}{1.206981in}}%
\pgfpathlineto{\pgfqpoint{1.355792in}{1.237881in}}%
\pgfpathlineto{\pgfqpoint{1.356309in}{1.146732in}}%
\pgfpathlineto{\pgfqpoint{1.357165in}{0.918468in}}%
\pgfpathlineto{\pgfqpoint{1.357522in}{1.104396in}}%
\pgfpathlineto{\pgfqpoint{1.357843in}{1.239696in}}%
\pgfpathlineto{\pgfqpoint{1.358753in}{1.197646in}}%
\pgfpathlineto{\pgfqpoint{1.358807in}{1.196689in}}%
\pgfpathlineto{\pgfqpoint{1.359021in}{1.213479in}}%
\pgfpathlineto{\pgfqpoint{1.359164in}{1.236184in}}%
\pgfpathlineto{\pgfqpoint{1.359717in}{1.132278in}}%
\pgfpathlineto{\pgfqpoint{1.360537in}{0.924327in}}%
\pgfpathlineto{\pgfqpoint{1.360894in}{1.100523in}}%
\pgfpathlineto{\pgfqpoint{1.361233in}{1.234398in}}%
\pgfpathlineto{\pgfqpoint{1.362143in}{1.201215in}}%
\pgfpathlineto{\pgfqpoint{1.362197in}{1.200161in}}%
\pgfpathlineto{\pgfqpoint{1.362446in}{1.222576in}}%
\pgfpathlineto{\pgfqpoint{1.362536in}{1.232339in}}%
\pgfpathlineto{\pgfqpoint{1.363071in}{1.135772in}}%
\pgfpathlineto{\pgfqpoint{1.363909in}{0.926130in}}%
\pgfpathlineto{\pgfqpoint{1.364266in}{1.096418in}}%
\pgfpathlineto{\pgfqpoint{1.364605in}{1.233824in}}%
\pgfpathlineto{\pgfqpoint{1.365533in}{1.201160in}}%
\pgfpathlineto{\pgfqpoint{1.365586in}{1.200153in}}%
\pgfpathlineto{\pgfqpoint{1.365890in}{1.226511in}}%
\pgfpathlineto{\pgfqpoint{1.365907in}{1.226935in}}%
\pgfpathlineto{\pgfqpoint{1.366050in}{1.208749in}}%
\pgfpathlineto{\pgfqpoint{1.367281in}{0.926213in}}%
\pgfpathlineto{\pgfqpoint{1.367620in}{1.080923in}}%
\pgfpathlineto{\pgfqpoint{1.367977in}{1.233353in}}%
\pgfpathlineto{\pgfqpoint{1.368905in}{1.200086in}}%
\pgfpathlineto{\pgfqpoint{1.369297in}{1.224752in}}%
\pgfpathlineto{\pgfqpoint{1.369886in}{1.120610in}}%
\pgfpathlineto{\pgfqpoint{1.370653in}{0.924770in}}%
\pgfpathlineto{\pgfqpoint{1.371028in}{1.101995in}}%
\pgfpathlineto{\pgfqpoint{1.371349in}{1.236398in}}%
\pgfpathlineto{\pgfqpoint{1.372277in}{1.197151in}}%
\pgfpathlineto{\pgfqpoint{1.372330in}{1.196111in}}%
\pgfpathlineto{\pgfqpoint{1.372616in}{1.220501in}}%
\pgfpathlineto{\pgfqpoint{1.372669in}{1.223306in}}%
\pgfpathlineto{\pgfqpoint{1.373008in}{1.158696in}}%
\pgfpathlineto{\pgfqpoint{1.373062in}{1.160774in}}%
\pgfpathlineto{\pgfqpoint{1.373151in}{1.149935in}}%
\pgfpathlineto{\pgfqpoint{1.374025in}{0.924409in}}%
\pgfpathlineto{\pgfqpoint{1.374400in}{1.103566in}}%
\pgfpathlineto{\pgfqpoint{1.374739in}{1.240655in}}%
\pgfpathlineto{\pgfqpoint{1.375631in}{1.193204in}}%
\pgfpathlineto{\pgfqpoint{1.375684in}{1.192441in}}%
\pgfpathlineto{\pgfqpoint{1.375916in}{1.208570in}}%
\pgfpathlineto{\pgfqpoint{1.376041in}{1.224980in}}%
\pgfpathlineto{\pgfqpoint{1.376612in}{1.128452in}}%
\pgfpathlineto{\pgfqpoint{1.377397in}{0.925214in}}%
\pgfpathlineto{\pgfqpoint{1.377772in}{1.105302in}}%
\pgfpathlineto{\pgfqpoint{1.378110in}{1.240988in}}%
\pgfpathlineto{\pgfqpoint{1.379003in}{1.193370in}}%
\pgfpathlineto{\pgfqpoint{1.379038in}{1.193426in}}%
\pgfpathlineto{\pgfqpoint{1.379074in}{1.194135in}}%
\pgfpathlineto{\pgfqpoint{1.379431in}{1.228643in}}%
\pgfpathlineto{\pgfqpoint{1.379680in}{1.176239in}}%
\pgfpathlineto{\pgfqpoint{1.380769in}{0.928257in}}%
\pgfpathlineto{\pgfqpoint{1.381108in}{1.085759in}}%
\pgfpathlineto{\pgfqpoint{1.381465in}{1.241473in}}%
\pgfpathlineto{\pgfqpoint{1.382374in}{1.193323in}}%
\pgfpathlineto{\pgfqpoint{1.382678in}{1.218287in}}%
\pgfpathlineto{\pgfqpoint{1.382785in}{1.231353in}}%
\pgfpathlineto{\pgfqpoint{1.383320in}{1.134590in}}%
\pgfpathlineto{\pgfqpoint{1.384123in}{0.934264in}}%
\pgfpathlineto{\pgfqpoint{1.384497in}{1.102298in}}%
\pgfpathlineto{\pgfqpoint{1.384836in}{1.240593in}}%
\pgfpathlineto{\pgfqpoint{1.385746in}{1.196664in}}%
\pgfpathlineto{\pgfqpoint{1.386157in}{1.233314in}}%
\pgfpathlineto{\pgfqpoint{1.386406in}{1.181755in}}%
\pgfpathlineto{\pgfqpoint{1.387495in}{0.939822in}}%
\pgfpathlineto{\pgfqpoint{1.387816in}{1.068091in}}%
\pgfpathlineto{\pgfqpoint{1.388208in}{1.236675in}}%
\pgfpathlineto{\pgfqpoint{1.389136in}{1.202644in}}%
\pgfpathlineto{\pgfqpoint{1.389529in}{1.235955in}}%
\pgfpathlineto{\pgfqpoint{1.389778in}{1.179668in}}%
\pgfpathlineto{\pgfqpoint{1.390867in}{0.946819in}}%
\pgfpathlineto{\pgfqpoint{1.391188in}{1.068859in}}%
\pgfpathlineto{\pgfqpoint{1.391580in}{1.232324in}}%
\pgfpathlineto{\pgfqpoint{1.392508in}{1.205443in}}%
\pgfpathlineto{\pgfqpoint{1.392883in}{1.234447in}}%
\pgfpathlineto{\pgfqpoint{1.393132in}{1.183916in}}%
\pgfpathlineto{\pgfqpoint{1.394060in}{0.953309in}}%
\pgfpathlineto{\pgfqpoint{1.394542in}{1.057151in}}%
\pgfpathlineto{\pgfqpoint{1.394952in}{1.227791in}}%
\pgfpathlineto{\pgfqpoint{1.395880in}{1.207742in}}%
\pgfpathlineto{\pgfqpoint{1.396255in}{1.232288in}}%
\pgfpathlineto{\pgfqpoint{1.396504in}{1.181330in}}%
\pgfpathlineto{\pgfqpoint{1.397610in}{0.955377in}}%
\pgfpathlineto{\pgfqpoint{1.397914in}{1.054386in}}%
\pgfpathlineto{\pgfqpoint{1.398342in}{1.225067in}}%
\pgfpathlineto{\pgfqpoint{1.399270in}{1.211850in}}%
\pgfpathlineto{\pgfqpoint{1.399626in}{1.229488in}}%
\pgfpathlineto{\pgfqpoint{1.399841in}{1.192299in}}%
\pgfpathlineto{\pgfqpoint{1.400982in}{0.957796in}}%
\pgfpathlineto{\pgfqpoint{1.401286in}{1.052615in}}%
\pgfpathlineto{\pgfqpoint{1.401714in}{1.223930in}}%
\pgfpathlineto{\pgfqpoint{1.402659in}{1.212033in}}%
\pgfpathlineto{\pgfqpoint{1.402998in}{1.227571in}}%
\pgfpathlineto{\pgfqpoint{1.403195in}{1.198380in}}%
\pgfpathlineto{\pgfqpoint{1.404372in}{0.958821in}}%
\pgfpathlineto{\pgfqpoint{1.404675in}{1.060600in}}%
\pgfpathlineto{\pgfqpoint{1.405086in}{1.223654in}}%
\pgfpathlineto{\pgfqpoint{1.406031in}{1.211639in}}%
\pgfpathlineto{\pgfqpoint{1.406370in}{1.226427in}}%
\pgfpathlineto{\pgfqpoint{1.406566in}{1.198354in}}%
\pgfpathlineto{\pgfqpoint{1.407726in}{0.959483in}}%
\pgfpathlineto{\pgfqpoint{1.408047in}{1.059567in}}%
\pgfpathlineto{\pgfqpoint{1.408458in}{1.224719in}}%
\pgfpathlineto{\pgfqpoint{1.409421in}{1.211747in}}%
\pgfpathlineto{\pgfqpoint{1.409742in}{1.224849in}}%
\pgfpathlineto{\pgfqpoint{1.409921in}{1.203863in}}%
\pgfpathlineto{\pgfqpoint{1.411080in}{0.961226in}}%
\pgfpathlineto{\pgfqpoint{1.411455in}{1.079953in}}%
\pgfpathlineto{\pgfqpoint{1.411830in}{1.223663in}}%
\pgfpathlineto{\pgfqpoint{1.412793in}{1.211597in}}%
\pgfpathlineto{\pgfqpoint{1.413114in}{1.225549in}}%
\pgfpathlineto{\pgfqpoint{1.413310in}{1.201336in}}%
\pgfpathlineto{\pgfqpoint{1.414434in}{0.962176in}}%
\pgfpathlineto{\pgfqpoint{1.414809in}{1.069923in}}%
\pgfpathlineto{\pgfqpoint{1.415201in}{1.222936in}}%
\pgfpathlineto{\pgfqpoint{1.416165in}{1.212241in}}%
\pgfpathlineto{\pgfqpoint{1.416486in}{1.224241in}}%
\pgfpathlineto{\pgfqpoint{1.416700in}{1.195398in}}%
\pgfpathlineto{\pgfqpoint{1.417770in}{0.963675in}}%
\pgfpathlineto{\pgfqpoint{1.418163in}{1.061159in}}%
\pgfpathlineto{\pgfqpoint{1.418573in}{1.223987in}}%
\pgfpathlineto{\pgfqpoint{1.419537in}{1.212846in}}%
\pgfpathlineto{\pgfqpoint{1.419858in}{1.225496in}}%
\pgfpathlineto{\pgfqpoint{1.420072in}{1.196280in}}%
\pgfpathlineto{\pgfqpoint{1.421125in}{0.965829in}}%
\pgfpathlineto{\pgfqpoint{1.421535in}{1.061186in}}%
\pgfpathlineto{\pgfqpoint{1.421945in}{1.221026in}}%
\pgfpathlineto{\pgfqpoint{1.422909in}{1.214862in}}%
\pgfpathlineto{\pgfqpoint{1.423230in}{1.226593in}}%
\pgfpathlineto{\pgfqpoint{1.423462in}{1.191806in}}%
\pgfpathlineto{\pgfqpoint{1.424443in}{0.966752in}}%
\pgfpathlineto{\pgfqpoint{1.424889in}{1.051413in}}%
\pgfpathlineto{\pgfqpoint{1.425317in}{1.220734in}}%
\pgfpathlineto{\pgfqpoint{1.426281in}{1.214686in}}%
\pgfpathlineto{\pgfqpoint{1.426602in}{1.226555in}}%
\pgfpathlineto{\pgfqpoint{1.426851in}{1.184306in}}%
\pgfpathlineto{\pgfqpoint{1.427797in}{0.969568in}}%
\pgfpathlineto{\pgfqpoint{1.428261in}{1.054531in}}%
\pgfpathlineto{\pgfqpoint{1.428707in}{1.219684in}}%
\pgfpathlineto{\pgfqpoint{1.429652in}{1.216087in}}%
\pgfpathlineto{\pgfqpoint{1.429974in}{1.225845in}}%
\pgfpathlineto{\pgfqpoint{1.430223in}{1.184781in}}%
\pgfpathlineto{\pgfqpoint{1.431169in}{0.970680in}}%
\pgfpathlineto{\pgfqpoint{1.431633in}{1.054738in}}%
\pgfpathlineto{\pgfqpoint{1.433024in}{1.218487in}}%
\pgfpathlineto{\pgfqpoint{1.433328in}{1.228302in}}%
\pgfpathlineto{\pgfqpoint{1.433613in}{1.178513in}}%
\pgfpathlineto{\pgfqpoint{1.434541in}{0.972691in}}%
\pgfpathlineto{\pgfqpoint{1.435005in}{1.056342in}}%
\pgfpathlineto{\pgfqpoint{1.436414in}{1.221334in}}%
\pgfpathlineto{\pgfqpoint{1.436700in}{1.227757in}}%
\pgfpathlineto{\pgfqpoint{1.436967in}{1.184204in}}%
\pgfpathlineto{\pgfqpoint{1.437913in}{0.976024in}}%
\pgfpathlineto{\pgfqpoint{1.438377in}{1.057214in}}%
\pgfpathlineto{\pgfqpoint{1.439786in}{1.223759in}}%
\pgfpathlineto{\pgfqpoint{1.440071in}{1.228388in}}%
\pgfpathlineto{\pgfqpoint{1.440339in}{1.183107in}}%
\pgfpathlineto{\pgfqpoint{1.441285in}{0.978652in}}%
\pgfpathlineto{\pgfqpoint{1.441748in}{1.056795in}}%
\pgfpathlineto{\pgfqpoint{1.443176in}{1.228230in}}%
\pgfpathlineto{\pgfqpoint{1.443229in}{1.229580in}}%
\pgfpathlineto{\pgfqpoint{1.443657in}{1.200173in}}%
\pgfpathlineto{\pgfqpoint{1.444639in}{0.981390in}}%
\pgfpathlineto{\pgfqpoint{1.445138in}{1.064774in}}%
\pgfpathlineto{\pgfqpoint{1.446583in}{1.229463in}}%
\pgfpathlineto{\pgfqpoint{1.446851in}{1.223829in}}%
\pgfpathlineto{\pgfqpoint{1.447547in}{1.068186in}}%
\pgfpathlineto{\pgfqpoint{1.448028in}{0.984758in}}%
\pgfpathlineto{\pgfqpoint{1.448546in}{1.078039in}}%
\pgfpathlineto{\pgfqpoint{1.449955in}{1.229993in}}%
\pgfpathlineto{\pgfqpoint{1.449973in}{1.229722in}}%
\pgfpathlineto{\pgfqpoint{1.450490in}{1.162641in}}%
\pgfpathlineto{\pgfqpoint{1.451400in}{0.985944in}}%
\pgfpathlineto{\pgfqpoint{1.451864in}{1.050630in}}%
\pgfpathlineto{\pgfqpoint{1.453327in}{1.228743in}}%
\pgfpathlineto{\pgfqpoint{1.453345in}{1.228405in}}%
\pgfpathlineto{\pgfqpoint{1.453898in}{1.149369in}}%
\pgfpathlineto{\pgfqpoint{1.454790in}{0.986487in}}%
\pgfpathlineto{\pgfqpoint{1.455254in}{1.057739in}}%
\pgfpathlineto{\pgfqpoint{1.456699in}{1.226845in}}%
\pgfpathlineto{\pgfqpoint{1.456717in}{1.226493in}}%
\pgfpathlineto{\pgfqpoint{1.457288in}{1.145777in}}%
\pgfpathlineto{\pgfqpoint{1.458162in}{0.985561in}}%
\pgfpathlineto{\pgfqpoint{1.458644in}{1.064656in}}%
\pgfpathlineto{\pgfqpoint{1.460071in}{1.225779in}}%
\pgfpathlineto{\pgfqpoint{1.460089in}{1.225544in}}%
\pgfpathlineto{\pgfqpoint{1.460642in}{1.153383in}}%
\pgfpathlineto{\pgfqpoint{1.461534in}{0.984087in}}%
\pgfpathlineto{\pgfqpoint{1.461998in}{1.057606in}}%
\pgfpathlineto{\pgfqpoint{1.463443in}{1.225735in}}%
\pgfpathlineto{\pgfqpoint{1.463461in}{1.225718in}}%
\pgfpathlineto{\pgfqpoint{1.463942in}{1.182405in}}%
\pgfpathlineto{\pgfqpoint{1.464906in}{0.983628in}}%
\pgfpathlineto{\pgfqpoint{1.465370in}{1.059956in}}%
\pgfpathlineto{\pgfqpoint{1.466833in}{1.228761in}}%
\pgfpathlineto{\pgfqpoint{1.467279in}{1.199265in}}%
\pgfpathlineto{\pgfqpoint{1.468260in}{0.984131in}}%
\pgfpathlineto{\pgfqpoint{1.468759in}{1.070664in}}%
\pgfpathlineto{\pgfqpoint{1.470204in}{1.231355in}}%
\pgfpathlineto{\pgfqpoint{1.470222in}{1.231115in}}%
\pgfpathlineto{\pgfqpoint{1.470740in}{1.164211in}}%
\pgfpathlineto{\pgfqpoint{1.471632in}{0.987425in}}%
\pgfpathlineto{\pgfqpoint{1.472096in}{1.059123in}}%
\pgfpathlineto{\pgfqpoint{1.473576in}{1.235165in}}%
\pgfpathlineto{\pgfqpoint{1.473594in}{1.234822in}}%
\pgfpathlineto{\pgfqpoint{1.474112in}{1.162995in}}%
\pgfpathlineto{\pgfqpoint{1.475004in}{0.990515in}}%
\pgfpathlineto{\pgfqpoint{1.475467in}{1.059629in}}%
\pgfpathlineto{\pgfqpoint{1.476948in}{1.240338in}}%
\pgfpathlineto{\pgfqpoint{1.476984in}{1.238975in}}%
\pgfpathlineto{\pgfqpoint{1.477573in}{1.127886in}}%
\pgfpathlineto{\pgfqpoint{1.478376in}{0.994531in}}%
\pgfpathlineto{\pgfqpoint{1.478875in}{1.073439in}}%
\pgfpathlineto{\pgfqpoint{1.480320in}{1.241125in}}%
\pgfpathlineto{\pgfqpoint{1.480374in}{1.238098in}}%
\pgfpathlineto{\pgfqpoint{1.481159in}{1.081518in}}%
\pgfpathlineto{\pgfqpoint{1.481747in}{1.000669in}}%
\pgfpathlineto{\pgfqpoint{1.482265in}{1.078893in}}%
\pgfpathlineto{\pgfqpoint{1.483692in}{1.241145in}}%
\pgfpathlineto{\pgfqpoint{1.483763in}{1.236152in}}%
\pgfpathlineto{\pgfqpoint{1.484959in}{1.028045in}}%
\pgfpathlineto{\pgfqpoint{1.485119in}{1.002073in}}%
\pgfpathlineto{\pgfqpoint{1.485726in}{1.110457in}}%
\pgfpathlineto{\pgfqpoint{1.487064in}{1.240177in}}%
\pgfpathlineto{\pgfqpoint{1.487135in}{1.234864in}}%
\pgfpathlineto{\pgfqpoint{1.488509in}{1.002971in}}%
\pgfpathlineto{\pgfqpoint{1.489187in}{1.142267in}}%
\pgfpathlineto{\pgfqpoint{1.490436in}{1.237462in}}%
\pgfpathlineto{\pgfqpoint{1.489954in}{1.120998in}}%
\pgfpathlineto{\pgfqpoint{1.490561in}{1.227475in}}%
\pgfpathlineto{\pgfqpoint{1.491274in}{1.086099in}}%
\pgfpathlineto{\pgfqpoint{1.491881in}{1.002599in}}%
\pgfpathlineto{\pgfqpoint{1.492398in}{1.077354in}}%
\pgfpathlineto{\pgfqpoint{1.493808in}{1.236090in}}%
\pgfpathlineto{\pgfqpoint{1.493879in}{1.231524in}}%
\pgfpathlineto{\pgfqpoint{1.495128in}{1.016961in}}%
\pgfpathlineto{\pgfqpoint{1.495253in}{1.000723in}}%
\pgfpathlineto{\pgfqpoint{1.495842in}{1.107572in}}%
\pgfpathlineto{\pgfqpoint{1.497180in}{1.234510in}}%
\pgfpathlineto{\pgfqpoint{1.497233in}{1.232268in}}%
\pgfpathlineto{\pgfqpoint{1.498036in}{1.083264in}}%
\pgfpathlineto{\pgfqpoint{1.498625in}{1.000386in}}%
\pgfpathlineto{\pgfqpoint{1.499142in}{1.078376in}}%
\pgfpathlineto{\pgfqpoint{1.500569in}{1.234833in}}%
\pgfpathlineto{\pgfqpoint{1.500623in}{1.232389in}}%
\pgfpathlineto{\pgfqpoint{1.501212in}{1.121580in}}%
\pgfpathlineto{\pgfqpoint{1.501997in}{1.000200in}}%
\pgfpathlineto{\pgfqpoint{1.502496in}{1.073013in}}%
\pgfpathlineto{\pgfqpoint{1.503941in}{1.236568in}}%
\pgfpathlineto{\pgfqpoint{1.504013in}{1.233084in}}%
\pgfpathlineto{\pgfqpoint{1.504584in}{1.120712in}}%
\pgfpathlineto{\pgfqpoint{1.505369in}{1.000358in}}%
\pgfpathlineto{\pgfqpoint{1.505868in}{1.075712in}}%
\pgfpathlineto{\pgfqpoint{1.507313in}{1.238439in}}%
\pgfpathlineto{\pgfqpoint{1.507385in}{1.235495in}}%
\pgfpathlineto{\pgfqpoint{1.507920in}{1.129250in}}%
\pgfpathlineto{\pgfqpoint{1.508740in}{1.002739in}}%
\pgfpathlineto{\pgfqpoint{1.509204in}{1.063819in}}%
\pgfpathlineto{\pgfqpoint{1.510685in}{1.241130in}}%
\pgfpathlineto{\pgfqpoint{1.510792in}{1.235624in}}%
\pgfpathlineto{\pgfqpoint{1.511310in}{1.122404in}}%
\pgfpathlineto{\pgfqpoint{1.512112in}{1.005087in}}%
\pgfpathlineto{\pgfqpoint{1.512594in}{1.069570in}}%
\pgfpathlineto{\pgfqpoint{1.514057in}{1.243144in}}%
\pgfpathlineto{\pgfqpoint{1.514164in}{1.237238in}}%
\pgfpathlineto{\pgfqpoint{1.514806in}{1.092268in}}%
\pgfpathlineto{\pgfqpoint{1.515484in}{1.008499in}}%
\pgfpathlineto{\pgfqpoint{1.515984in}{1.076623in}}%
\pgfpathlineto{\pgfqpoint{1.517429in}{1.243088in}}%
\pgfpathlineto{\pgfqpoint{1.517554in}{1.235401in}}%
\pgfpathlineto{\pgfqpoint{1.518303in}{1.068758in}}%
\pgfpathlineto{\pgfqpoint{1.518856in}{1.010356in}}%
\pgfpathlineto{\pgfqpoint{1.519356in}{1.074315in}}%
\pgfpathlineto{\pgfqpoint{1.520801in}{1.244929in}}%
\pgfpathlineto{\pgfqpoint{1.520944in}{1.234404in}}%
\pgfpathlineto{\pgfqpoint{1.522246in}{1.011841in}}%
\pgfpathlineto{\pgfqpoint{1.522906in}{1.134675in}}%
\pgfpathlineto{\pgfqpoint{1.524173in}{1.243490in}}%
\pgfpathlineto{\pgfqpoint{1.523637in}{1.117353in}}%
\pgfpathlineto{\pgfqpoint{1.524333in}{1.230847in}}%
\pgfpathlineto{\pgfqpoint{1.525618in}{1.014542in}}%
\pgfpathlineto{\pgfqpoint{1.526260in}{1.127584in}}%
\pgfpathlineto{\pgfqpoint{1.527562in}{1.243305in}}%
\pgfpathlineto{\pgfqpoint{1.527009in}{1.117497in}}%
\pgfpathlineto{\pgfqpoint{1.527705in}{1.231079in}}%
\pgfpathlineto{\pgfqpoint{1.528990in}{1.014729in}}%
\pgfpathlineto{\pgfqpoint{1.529632in}{1.125619in}}%
\pgfpathlineto{\pgfqpoint{1.530934in}{1.243883in}}%
\pgfpathlineto{\pgfqpoint{1.530381in}{1.119858in}}%
\pgfpathlineto{\pgfqpoint{1.531077in}{1.231208in}}%
\pgfpathlineto{\pgfqpoint{1.532362in}{1.014947in}}%
\pgfpathlineto{\pgfqpoint{1.532986in}{1.118722in}}%
\pgfpathlineto{\pgfqpoint{1.534306in}{1.242777in}}%
\pgfpathlineto{\pgfqpoint{1.534431in}{1.232868in}}%
\pgfpathlineto{\pgfqpoint{1.535734in}{1.016887in}}%
\pgfpathlineto{\pgfqpoint{1.536376in}{1.123625in}}%
\pgfpathlineto{\pgfqpoint{1.537678in}{1.242116in}}%
\pgfpathlineto{\pgfqpoint{1.537196in}{1.119783in}}%
\pgfpathlineto{\pgfqpoint{1.537821in}{1.230397in}}%
\pgfpathlineto{\pgfqpoint{1.539105in}{1.016289in}}%
\pgfpathlineto{\pgfqpoint{1.539712in}{1.111782in}}%
\pgfpathlineto{\pgfqpoint{1.541050in}{1.241335in}}%
\pgfpathlineto{\pgfqpoint{1.541175in}{1.232750in}}%
\pgfpathlineto{\pgfqpoint{1.542495in}{1.016542in}}%
\pgfpathlineto{\pgfqpoint{1.543120in}{1.122582in}}%
\pgfpathlineto{\pgfqpoint{1.544440in}{1.241232in}}%
\pgfpathlineto{\pgfqpoint{1.543958in}{1.118604in}}%
\pgfpathlineto{\pgfqpoint{1.544565in}{1.231506in}}%
\pgfpathlineto{\pgfqpoint{1.545867in}{1.016200in}}%
\pgfpathlineto{\pgfqpoint{1.546474in}{1.117397in}}%
\pgfpathlineto{\pgfqpoint{1.547812in}{1.241297in}}%
\pgfpathlineto{\pgfqpoint{1.547937in}{1.232494in}}%
\pgfpathlineto{\pgfqpoint{1.549221in}{1.016041in}}%
\pgfpathlineto{\pgfqpoint{1.549846in}{1.118659in}}%
\pgfpathlineto{\pgfqpoint{1.551184in}{1.242257in}}%
\pgfpathlineto{\pgfqpoint{1.550702in}{1.116470in}}%
\pgfpathlineto{\pgfqpoint{1.551309in}{1.234006in}}%
\pgfpathlineto{\pgfqpoint{1.552611in}{1.017078in}}%
\pgfpathlineto{\pgfqpoint{1.553200in}{1.112941in}}%
\pgfpathlineto{\pgfqpoint{1.554573in}{1.244655in}}%
\pgfpathlineto{\pgfqpoint{1.554680in}{1.237080in}}%
\pgfpathlineto{\pgfqpoint{1.555965in}{1.018752in}}%
\pgfpathlineto{\pgfqpoint{1.556589in}{1.117232in}}%
\pgfpathlineto{\pgfqpoint{1.557945in}{1.246263in}}%
\pgfpathlineto{\pgfqpoint{1.557428in}{1.114907in}}%
\pgfpathlineto{\pgfqpoint{1.558070in}{1.235593in}}%
\pgfpathlineto{\pgfqpoint{1.559337in}{1.021408in}}%
\pgfpathlineto{\pgfqpoint{1.559926in}{1.106268in}}%
\pgfpathlineto{\pgfqpoint{1.561317in}{1.246636in}}%
\pgfpathlineto{\pgfqpoint{1.561460in}{1.232699in}}%
\pgfpathlineto{\pgfqpoint{1.562709in}{1.024643in}}%
\pgfpathlineto{\pgfqpoint{1.563262in}{1.093576in}}%
\pgfpathlineto{\pgfqpoint{1.564689in}{1.248132in}}%
\pgfpathlineto{\pgfqpoint{1.564850in}{1.229201in}}%
\pgfpathlineto{\pgfqpoint{1.566098in}{1.026554in}}%
\pgfpathlineto{\pgfqpoint{1.566616in}{1.087047in}}%
\pgfpathlineto{\pgfqpoint{1.568061in}{1.247041in}}%
\pgfpathlineto{\pgfqpoint{1.568239in}{1.223463in}}%
\pgfpathlineto{\pgfqpoint{1.569470in}{1.030085in}}%
\pgfpathlineto{\pgfqpoint{1.569970in}{1.080461in}}%
\pgfpathlineto{\pgfqpoint{1.571433in}{1.245689in}}%
\pgfpathlineto{\pgfqpoint{1.571629in}{1.217680in}}%
\pgfpathlineto{\pgfqpoint{1.572860in}{1.029978in}}%
\pgfpathlineto{\pgfqpoint{1.573342in}{1.077693in}}%
\pgfpathlineto{\pgfqpoint{1.574805in}{1.243836in}}%
\pgfpathlineto{\pgfqpoint{1.575001in}{1.216230in}}%
\pgfpathlineto{\pgfqpoint{1.576232in}{1.029649in}}%
\pgfpathlineto{\pgfqpoint{1.576714in}{1.076829in}}%
\pgfpathlineto{\pgfqpoint{1.578177in}{1.240742in}}%
\pgfpathlineto{\pgfqpoint{1.578373in}{1.215112in}}%
\pgfpathlineto{\pgfqpoint{1.579604in}{1.029103in}}%
\pgfpathlineto{\pgfqpoint{1.580103in}{1.081295in}}%
\pgfpathlineto{\pgfqpoint{1.581549in}{1.240192in}}%
\pgfpathlineto{\pgfqpoint{1.581745in}{1.215875in}}%
\pgfpathlineto{\pgfqpoint{1.582976in}{1.027271in}}%
\pgfpathlineto{\pgfqpoint{1.583475in}{1.082708in}}%
\pgfpathlineto{\pgfqpoint{1.584938in}{1.240831in}}%
\pgfpathlineto{\pgfqpoint{1.585117in}{1.218093in}}%
\pgfpathlineto{\pgfqpoint{1.586348in}{1.027073in}}%
\pgfpathlineto{\pgfqpoint{1.586847in}{1.084281in}}%
\pgfpathlineto{\pgfqpoint{1.588310in}{1.242268in}}%
\pgfpathlineto{\pgfqpoint{1.588489in}{1.220691in}}%
\pgfpathlineto{\pgfqpoint{1.589720in}{1.028103in}}%
\pgfpathlineto{\pgfqpoint{1.590219in}{1.085806in}}%
\pgfpathlineto{\pgfqpoint{1.591700in}{1.243676in}}%
\pgfpathlineto{\pgfqpoint{1.591878in}{1.218843in}}%
\pgfpathlineto{\pgfqpoint{1.593092in}{1.030265in}}%
\pgfpathlineto{\pgfqpoint{1.593573in}{1.083838in}}%
\pgfpathlineto{\pgfqpoint{1.595072in}{1.245432in}}%
\pgfpathlineto{\pgfqpoint{1.595268in}{1.215129in}}%
\pgfpathlineto{\pgfqpoint{1.596463in}{1.032717in}}%
\pgfpathlineto{\pgfqpoint{1.596927in}{1.080119in}}%
\pgfpathlineto{\pgfqpoint{1.598444in}{1.247186in}}%
\pgfpathlineto{\pgfqpoint{1.598676in}{1.207339in}}%
\pgfpathlineto{\pgfqpoint{1.599835in}{1.035657in}}%
\pgfpathlineto{\pgfqpoint{1.600281in}{1.075878in}}%
\pgfpathlineto{\pgfqpoint{1.601798in}{1.246836in}}%
\pgfpathlineto{\pgfqpoint{1.602083in}{1.195417in}}%
\pgfpathlineto{\pgfqpoint{1.603207in}{1.038715in}}%
\pgfpathlineto{\pgfqpoint{1.603653in}{1.075317in}}%
\pgfpathlineto{\pgfqpoint{1.605188in}{1.244836in}}%
\pgfpathlineto{\pgfqpoint{1.605491in}{1.184759in}}%
\pgfpathlineto{\pgfqpoint{1.606597in}{1.040169in}}%
\pgfpathlineto{\pgfqpoint{1.607043in}{1.075219in}}%
\pgfpathlineto{\pgfqpoint{1.608559in}{1.243826in}}%
\pgfpathlineto{\pgfqpoint{1.608845in}{1.188607in}}%
\pgfpathlineto{\pgfqpoint{1.609969in}{1.039871in}}%
\pgfpathlineto{\pgfqpoint{1.610397in}{1.071562in}}%
\pgfpathlineto{\pgfqpoint{1.611931in}{1.241132in}}%
\pgfpathlineto{\pgfqpoint{1.612253in}{1.178526in}}%
\pgfpathlineto{\pgfqpoint{1.613359in}{1.040628in}}%
\pgfpathlineto{\pgfqpoint{1.613787in}{1.073991in}}%
\pgfpathlineto{\pgfqpoint{1.615303in}{1.239876in}}%
\pgfpathlineto{\pgfqpoint{1.615607in}{1.183285in}}%
\pgfpathlineto{\pgfqpoint{1.616731in}{1.039351in}}%
\pgfpathlineto{\pgfqpoint{1.617159in}{1.073601in}}%
\pgfpathlineto{\pgfqpoint{1.618675in}{1.239611in}}%
\pgfpathlineto{\pgfqpoint{1.618978in}{1.184343in}}%
\pgfpathlineto{\pgfqpoint{1.620102in}{1.039006in}}%
\pgfpathlineto{\pgfqpoint{1.620531in}{1.074578in}}%
\pgfpathlineto{\pgfqpoint{1.622047in}{1.238833in}}%
\pgfpathlineto{\pgfqpoint{1.622368in}{1.181386in}}%
\pgfpathlineto{\pgfqpoint{1.623474in}{1.039876in}}%
\pgfpathlineto{\pgfqpoint{1.623903in}{1.075410in}}%
\pgfpathlineto{\pgfqpoint{1.625437in}{1.239516in}}%
\pgfpathlineto{\pgfqpoint{1.625740in}{1.182310in}}%
\pgfpathlineto{\pgfqpoint{1.626846in}{1.039856in}}%
\pgfpathlineto{\pgfqpoint{1.627274in}{1.077086in}}%
\pgfpathlineto{\pgfqpoint{1.628809in}{1.240176in}}%
\pgfpathlineto{\pgfqpoint{1.629112in}{1.182696in}}%
\pgfpathlineto{\pgfqpoint{1.630218in}{1.041343in}}%
\pgfpathlineto{\pgfqpoint{1.630628in}{1.075592in}}%
\pgfpathlineto{\pgfqpoint{1.632181in}{1.240504in}}%
\pgfpathlineto{\pgfqpoint{1.632537in}{1.170115in}}%
\pgfpathlineto{\pgfqpoint{1.633590in}{1.043213in}}%
\pgfpathlineto{\pgfqpoint{1.634000in}{1.076294in}}%
\pgfpathlineto{\pgfqpoint{1.635553in}{1.240847in}}%
\pgfpathlineto{\pgfqpoint{1.635927in}{1.164580in}}%
\pgfpathlineto{\pgfqpoint{1.636962in}{1.044321in}}%
\pgfpathlineto{\pgfqpoint{1.637354in}{1.075308in}}%
\pgfpathlineto{\pgfqpoint{1.638924in}{1.240580in}}%
\pgfpathlineto{\pgfqpoint{1.639335in}{1.154303in}}%
\pgfpathlineto{\pgfqpoint{1.640334in}{1.045767in}}%
\pgfpathlineto{\pgfqpoint{1.640709in}{1.074236in}}%
\pgfpathlineto{\pgfqpoint{1.642296in}{1.241183in}}%
\pgfpathlineto{\pgfqpoint{1.642742in}{1.143724in}}%
\pgfpathlineto{\pgfqpoint{1.643724in}{1.047354in}}%
\pgfpathlineto{\pgfqpoint{1.644063in}{1.072496in}}%
\pgfpathlineto{\pgfqpoint{1.645668in}{1.240130in}}%
\pgfpathlineto{\pgfqpoint{1.646132in}{1.136739in}}%
\pgfpathlineto{\pgfqpoint{1.647095in}{1.049447in}}%
\pgfpathlineto{\pgfqpoint{1.647434in}{1.073497in}}%
\pgfpathlineto{\pgfqpoint{1.648880in}{1.210483in}}%
\pgfpathlineto{\pgfqpoint{1.649058in}{1.239362in}}%
\pgfpathlineto{\pgfqpoint{1.649629in}{1.099169in}}%
\pgfpathlineto{\pgfqpoint{1.650467in}{1.050250in}}%
\pgfpathlineto{\pgfqpoint{1.650824in}{1.072739in}}%
\pgfpathlineto{\pgfqpoint{1.652287in}{1.219959in}}%
\pgfpathlineto{\pgfqpoint{1.652430in}{1.238458in}}%
\pgfpathlineto{\pgfqpoint{1.652947in}{1.114343in}}%
\pgfpathlineto{\pgfqpoint{1.653839in}{1.050261in}}%
\pgfpathlineto{\pgfqpoint{1.654196in}{1.072362in}}%
\pgfpathlineto{\pgfqpoint{1.655623in}{1.207002in}}%
\pgfpathlineto{\pgfqpoint{1.655802in}{1.236185in}}%
\pgfpathlineto{\pgfqpoint{1.656373in}{1.099619in}}%
\pgfpathlineto{\pgfqpoint{1.657229in}{1.051042in}}%
\pgfpathlineto{\pgfqpoint{1.657586in}{1.073328in}}%
\pgfpathlineto{\pgfqpoint{1.659174in}{1.234722in}}%
\pgfpathlineto{\pgfqpoint{1.659620in}{1.136300in}}%
\pgfpathlineto{\pgfqpoint{1.660601in}{1.050059in}}%
\pgfpathlineto{\pgfqpoint{1.660940in}{1.070986in}}%
\pgfpathlineto{\pgfqpoint{1.662385in}{1.208276in}}%
\pgfpathlineto{\pgfqpoint{1.662546in}{1.233748in}}%
\pgfpathlineto{\pgfqpoint{1.663116in}{1.102367in}}%
\pgfpathlineto{\pgfqpoint{1.663973in}{1.049681in}}%
\pgfpathlineto{\pgfqpoint{1.664330in}{1.072100in}}%
\pgfpathlineto{\pgfqpoint{1.665935in}{1.232748in}}%
\pgfpathlineto{\pgfqpoint{1.666381in}{1.133928in}}%
\pgfpathlineto{\pgfqpoint{1.667345in}{1.050126in}}%
\pgfpathlineto{\pgfqpoint{1.667702in}{1.073217in}}%
\pgfpathlineto{\pgfqpoint{1.669307in}{1.232311in}}%
\pgfpathlineto{\pgfqpoint{1.669735in}{1.140114in}}%
\pgfpathlineto{\pgfqpoint{1.670717in}{1.049555in}}%
\pgfpathlineto{\pgfqpoint{1.671056in}{1.073677in}}%
\pgfpathlineto{\pgfqpoint{1.672679in}{1.233290in}}%
\pgfpathlineto{\pgfqpoint{1.673125in}{1.134974in}}%
\pgfpathlineto{\pgfqpoint{1.674089in}{1.050954in}}%
\pgfpathlineto{\pgfqpoint{1.674428in}{1.075642in}}%
\pgfpathlineto{\pgfqpoint{1.676051in}{1.234595in}}%
\pgfpathlineto{\pgfqpoint{1.676497in}{1.135891in}}%
\pgfpathlineto{\pgfqpoint{1.677460in}{1.053113in}}%
\pgfpathlineto{\pgfqpoint{1.677799in}{1.076852in}}%
\pgfpathlineto{\pgfqpoint{1.679298in}{1.219109in}}%
\pgfpathlineto{\pgfqpoint{1.679423in}{1.235855in}}%
\pgfpathlineto{\pgfqpoint{1.679940in}{1.112508in}}%
\pgfpathlineto{\pgfqpoint{1.680850in}{1.055694in}}%
\pgfpathlineto{\pgfqpoint{1.681153in}{1.077560in}}%
\pgfpathlineto{\pgfqpoint{1.682634in}{1.209118in}}%
\pgfpathlineto{\pgfqpoint{1.682795in}{1.235053in}}%
\pgfpathlineto{\pgfqpoint{1.683348in}{1.100923in}}%
\pgfpathlineto{\pgfqpoint{1.684222in}{1.058152in}}%
\pgfpathlineto{\pgfqpoint{1.684543in}{1.078396in}}%
\pgfpathlineto{\pgfqpoint{1.685917in}{1.179894in}}%
\pgfpathlineto{\pgfqpoint{1.686167in}{1.234946in}}%
\pgfpathlineto{\pgfqpoint{1.686755in}{1.091734in}}%
\pgfpathlineto{\pgfqpoint{1.687059in}{1.059666in}}%
\pgfpathlineto{\pgfqpoint{1.687915in}{1.077089in}}%
\pgfpathlineto{\pgfqpoint{1.688433in}{1.138130in}}%
\pgfpathlineto{\pgfqpoint{1.689539in}{1.232671in}}%
\pgfpathlineto{\pgfqpoint{1.688914in}{1.135102in}}%
\pgfpathlineto{\pgfqpoint{1.689842in}{1.170120in}}%
\pgfpathlineto{\pgfqpoint{1.690984in}{1.060619in}}%
\pgfpathlineto{\pgfqpoint{1.691287in}{1.075946in}}%
\pgfpathlineto{\pgfqpoint{1.691751in}{1.131730in}}%
\pgfpathlineto{\pgfqpoint{1.692911in}{1.230427in}}%
\pgfpathlineto{\pgfqpoint{1.693178in}{1.179290in}}%
\pgfpathlineto{\pgfqpoint{1.694373in}{1.060139in}}%
\pgfpathlineto{\pgfqpoint{1.694641in}{1.073291in}}%
\pgfpathlineto{\pgfqpoint{1.695158in}{1.136174in}}%
\pgfpathlineto{\pgfqpoint{1.696300in}{1.228420in}}%
\pgfpathlineto{\pgfqpoint{1.696568in}{1.174247in}}%
\pgfpathlineto{\pgfqpoint{1.697745in}{1.059083in}}%
\pgfpathlineto{\pgfqpoint{1.698031in}{1.073039in}}%
\pgfpathlineto{\pgfqpoint{1.698602in}{1.144508in}}%
\pgfpathlineto{\pgfqpoint{1.699672in}{1.226046in}}%
\pgfpathlineto{\pgfqpoint{1.699083in}{1.138350in}}%
\pgfpathlineto{\pgfqpoint{1.699958in}{1.169562in}}%
\pgfpathlineto{\pgfqpoint{1.701117in}{1.058793in}}%
\pgfpathlineto{\pgfqpoint{1.701421in}{1.074260in}}%
\pgfpathlineto{\pgfqpoint{1.702901in}{1.205462in}}%
\pgfpathlineto{\pgfqpoint{1.703044in}{1.225210in}}%
\pgfpathlineto{\pgfqpoint{1.703579in}{1.105789in}}%
\pgfpathlineto{\pgfqpoint{1.704489in}{1.057732in}}%
\pgfpathlineto{\pgfqpoint{1.704810in}{1.075461in}}%
\pgfpathlineto{\pgfqpoint{1.706434in}{1.226134in}}%
\pgfpathlineto{\pgfqpoint{1.706844in}{1.135953in}}%
\pgfpathlineto{\pgfqpoint{1.707861in}{1.058262in}}%
\pgfpathlineto{\pgfqpoint{1.708164in}{1.075968in}}%
\pgfpathlineto{\pgfqpoint{1.709806in}{1.225832in}}%
\pgfpathlineto{\pgfqpoint{1.710216in}{1.135993in}}%
\pgfpathlineto{\pgfqpoint{1.711233in}{1.059821in}}%
\pgfpathlineto{\pgfqpoint{1.711536in}{1.077841in}}%
\pgfpathlineto{\pgfqpoint{1.713178in}{1.226509in}}%
\pgfpathlineto{\pgfqpoint{1.713588in}{1.136395in}}%
\pgfpathlineto{\pgfqpoint{1.714605in}{1.061117in}}%
\pgfpathlineto{\pgfqpoint{1.714890in}{1.077891in}}%
\pgfpathlineto{\pgfqpoint{1.716478in}{1.222809in}}%
\pgfpathlineto{\pgfqpoint{1.716550in}{1.227525in}}%
\pgfpathlineto{\pgfqpoint{1.716906in}{1.150315in}}%
\pgfpathlineto{\pgfqpoint{1.717424in}{1.062073in}}%
\pgfpathlineto{\pgfqpoint{1.718262in}{1.079097in}}%
\pgfpathlineto{\pgfqpoint{1.719814in}{1.216418in}}%
\pgfpathlineto{\pgfqpoint{1.719921in}{1.226956in}}%
\pgfpathlineto{\pgfqpoint{1.720367in}{1.124489in}}%
\pgfpathlineto{\pgfqpoint{1.720778in}{1.062604in}}%
\pgfpathlineto{\pgfqpoint{1.721634in}{1.078837in}}%
\pgfpathlineto{\pgfqpoint{1.723079in}{1.188212in}}%
\pgfpathlineto{\pgfqpoint{1.723293in}{1.225925in}}%
\pgfpathlineto{\pgfqpoint{1.723846in}{1.094926in}}%
\pgfpathlineto{\pgfqpoint{1.724150in}{1.062561in}}%
\pgfpathlineto{\pgfqpoint{1.725024in}{1.079697in}}%
\pgfpathlineto{\pgfqpoint{1.726433in}{1.181851in}}%
\pgfpathlineto{\pgfqpoint{1.726665in}{1.223813in}}%
\pgfpathlineto{\pgfqpoint{1.727218in}{1.095073in}}%
\pgfpathlineto{\pgfqpoint{1.727504in}{1.064173in}}%
\pgfpathlineto{\pgfqpoint{1.728396in}{1.078838in}}%
\pgfpathlineto{\pgfqpoint{1.728967in}{1.138405in}}%
\pgfpathlineto{\pgfqpoint{1.730037in}{1.223019in}}%
\pgfpathlineto{\pgfqpoint{1.730340in}{1.162873in}}%
\pgfpathlineto{\pgfqpoint{1.730876in}{1.065834in}}%
\pgfpathlineto{\pgfqpoint{1.731750in}{1.076611in}}%
\pgfpathlineto{\pgfqpoint{1.733213in}{1.188625in}}%
\pgfpathlineto{\pgfqpoint{1.733409in}{1.220256in}}%
\pgfpathlineto{\pgfqpoint{1.733962in}{1.096331in}}%
\pgfpathlineto{\pgfqpoint{1.734890in}{1.066308in}}%
\pgfpathlineto{\pgfqpoint{1.735157in}{1.078113in}}%
\pgfpathlineto{\pgfqpoint{1.736567in}{1.181102in}}%
\pgfpathlineto{\pgfqpoint{1.736799in}{1.218669in}}%
\pgfpathlineto{\pgfqpoint{1.737352in}{1.093588in}}%
\pgfpathlineto{\pgfqpoint{1.738280in}{1.065569in}}%
\pgfpathlineto{\pgfqpoint{1.738529in}{1.077166in}}%
\pgfpathlineto{\pgfqpoint{1.739974in}{1.189285in}}%
\pgfpathlineto{\pgfqpoint{1.740171in}{1.218475in}}%
\pgfpathlineto{\pgfqpoint{1.740724in}{1.095234in}}%
\pgfpathlineto{\pgfqpoint{1.741651in}{1.065663in}}%
\pgfpathlineto{\pgfqpoint{1.741919in}{1.078101in}}%
\pgfpathlineto{\pgfqpoint{1.743364in}{1.191728in}}%
\pgfpathlineto{\pgfqpoint{1.743543in}{1.216926in}}%
\pgfpathlineto{\pgfqpoint{1.744078in}{1.099777in}}%
\pgfpathlineto{\pgfqpoint{1.745023in}{1.066038in}}%
\pgfpathlineto{\pgfqpoint{1.745273in}{1.078093in}}%
\pgfpathlineto{\pgfqpoint{1.746772in}{1.199177in}}%
\pgfpathlineto{\pgfqpoint{1.746932in}{1.216608in}}%
\pgfpathlineto{\pgfqpoint{1.747432in}{1.104698in}}%
\pgfpathlineto{\pgfqpoint{1.748395in}{1.065953in}}%
\pgfpathlineto{\pgfqpoint{1.748645in}{1.078138in}}%
\pgfpathlineto{\pgfqpoint{1.750304in}{1.216560in}}%
\pgfpathlineto{\pgfqpoint{1.750697in}{1.132159in}}%
\pgfpathlineto{\pgfqpoint{1.751767in}{1.066407in}}%
\pgfpathlineto{\pgfqpoint{1.751999in}{1.077700in}}%
\pgfpathlineto{\pgfqpoint{1.753676in}{1.216169in}}%
\pgfpathlineto{\pgfqpoint{1.754069in}{1.132559in}}%
\pgfpathlineto{\pgfqpoint{1.755139in}{1.067928in}}%
\pgfpathlineto{\pgfqpoint{1.755371in}{1.078711in}}%
\pgfpathlineto{\pgfqpoint{1.757048in}{1.216151in}}%
\pgfpathlineto{\pgfqpoint{1.757441in}{1.132868in}}%
\pgfpathlineto{\pgfqpoint{1.757869in}{1.067542in}}%
\pgfpathlineto{\pgfqpoint{1.758743in}{1.078697in}}%
\pgfpathlineto{\pgfqpoint{1.760420in}{1.216103in}}%
\pgfpathlineto{\pgfqpoint{1.760812in}{1.131784in}}%
\pgfpathlineto{\pgfqpoint{1.761223in}{1.066762in}}%
\pgfpathlineto{\pgfqpoint{1.762115in}{1.079785in}}%
\pgfpathlineto{\pgfqpoint{1.763667in}{1.202485in}}%
\pgfpathlineto{\pgfqpoint{1.763792in}{1.215216in}}%
\pgfpathlineto{\pgfqpoint{1.764256in}{1.113272in}}%
\pgfpathlineto{\pgfqpoint{1.764595in}{1.067541in}}%
\pgfpathlineto{\pgfqpoint{1.765487in}{1.079498in}}%
\pgfpathlineto{\pgfqpoint{1.766950in}{1.181810in}}%
\pgfpathlineto{\pgfqpoint{1.767164in}{1.214452in}}%
\pgfpathlineto{\pgfqpoint{1.767699in}{1.095425in}}%
\pgfpathlineto{\pgfqpoint{1.767967in}{1.067192in}}%
\pgfpathlineto{\pgfqpoint{1.768876in}{1.080369in}}%
\pgfpathlineto{\pgfqpoint{1.770304in}{1.176772in}}%
\pgfpathlineto{\pgfqpoint{1.770553in}{1.213009in}}%
\pgfpathlineto{\pgfqpoint{1.771053in}{1.099463in}}%
\pgfpathlineto{\pgfqpoint{1.771338in}{1.068225in}}%
\pgfpathlineto{\pgfqpoint{1.772248in}{1.079035in}}%
\pgfpathlineto{\pgfqpoint{1.772784in}{1.130420in}}%
\pgfpathlineto{\pgfqpoint{1.773925in}{1.211905in}}%
\pgfpathlineto{\pgfqpoint{1.774175in}{1.164445in}}%
\pgfpathlineto{\pgfqpoint{1.774693in}{1.069628in}}%
\pgfpathlineto{\pgfqpoint{1.775585in}{1.076691in}}%
\pgfpathlineto{\pgfqpoint{1.776209in}{1.136555in}}%
\pgfpathlineto{\pgfqpoint{1.777297in}{1.209421in}}%
\pgfpathlineto{\pgfqpoint{1.777547in}{1.163517in}}%
\pgfpathlineto{\pgfqpoint{1.778064in}{1.070947in}}%
\pgfpathlineto{\pgfqpoint{1.778956in}{1.076434in}}%
\pgfpathlineto{\pgfqpoint{1.779599in}{1.139211in}}%
\pgfpathlineto{\pgfqpoint{1.780669in}{1.207366in}}%
\pgfpathlineto{\pgfqpoint{1.780919in}{1.163570in}}%
\pgfpathlineto{\pgfqpoint{1.782168in}{1.070477in}}%
\pgfpathlineto{\pgfqpoint{1.782346in}{1.075599in}}%
\pgfpathlineto{\pgfqpoint{1.783006in}{1.142989in}}%
\pgfpathlineto{\pgfqpoint{1.784059in}{1.207007in}}%
\pgfpathlineto{\pgfqpoint{1.784309in}{1.160486in}}%
\pgfpathlineto{\pgfqpoint{1.785540in}{1.069863in}}%
\pgfpathlineto{\pgfqpoint{1.785736in}{1.076026in}}%
\pgfpathlineto{\pgfqpoint{1.787306in}{1.196186in}}%
\pgfpathlineto{\pgfqpoint{1.787431in}{1.205681in}}%
\pgfpathlineto{\pgfqpoint{1.787841in}{1.120700in}}%
\pgfpathlineto{\pgfqpoint{1.788912in}{1.070795in}}%
\pgfpathlineto{\pgfqpoint{1.789126in}{1.078260in}}%
\pgfpathlineto{\pgfqpoint{1.790803in}{1.205839in}}%
\pgfpathlineto{\pgfqpoint{1.791195in}{1.126467in}}%
\pgfpathlineto{\pgfqpoint{1.792284in}{1.070994in}}%
\pgfpathlineto{\pgfqpoint{1.792498in}{1.078835in}}%
\pgfpathlineto{\pgfqpoint{1.794175in}{1.206553in}}%
\pgfpathlineto{\pgfqpoint{1.794549in}{1.131642in}}%
\pgfpathlineto{\pgfqpoint{1.794960in}{1.071877in}}%
\pgfpathlineto{\pgfqpoint{1.795852in}{1.078841in}}%
\pgfpathlineto{\pgfqpoint{1.797547in}{1.206539in}}%
\pgfpathlineto{\pgfqpoint{1.797921in}{1.131608in}}%
\pgfpathlineto{\pgfqpoint{1.798332in}{1.071237in}}%
\pgfpathlineto{\pgfqpoint{1.799224in}{1.080167in}}%
\pgfpathlineto{\pgfqpoint{1.800918in}{1.206700in}}%
\pgfpathlineto{\pgfqpoint{1.801293in}{1.130707in}}%
\pgfpathlineto{\pgfqpoint{1.801686in}{1.070020in}}%
\pgfpathlineto{\pgfqpoint{1.802595in}{1.080057in}}%
\pgfpathlineto{\pgfqpoint{1.804219in}{1.201871in}}%
\pgfpathlineto{\pgfqpoint{1.804290in}{1.205894in}}%
\pgfpathlineto{\pgfqpoint{1.804629in}{1.138704in}}%
\pgfpathlineto{\pgfqpoint{1.805058in}{1.070065in}}%
\pgfpathlineto{\pgfqpoint{1.805950in}{1.079889in}}%
\pgfpathlineto{\pgfqpoint{1.807448in}{1.176707in}}%
\pgfpathlineto{\pgfqpoint{1.807662in}{1.204627in}}%
\pgfpathlineto{\pgfqpoint{1.808162in}{1.099556in}}%
\pgfpathlineto{\pgfqpoint{1.808429in}{1.071522in}}%
\pgfpathlineto{\pgfqpoint{1.809357in}{1.079561in}}%
\pgfpathlineto{\pgfqpoint{1.809964in}{1.135348in}}%
\pgfpathlineto{\pgfqpoint{1.811034in}{1.202826in}}%
\pgfpathlineto{\pgfqpoint{1.811284in}{1.160865in}}%
\pgfpathlineto{\pgfqpoint{1.811801in}{1.072073in}}%
\pgfpathlineto{\pgfqpoint{1.812693in}{1.077484in}}%
\pgfpathlineto{\pgfqpoint{1.813246in}{1.125522in}}%
\pgfpathlineto{\pgfqpoint{1.814424in}{1.200604in}}%
\pgfpathlineto{\pgfqpoint{1.814638in}{1.165074in}}%
\pgfpathlineto{\pgfqpoint{1.815173in}{1.073500in}}%
\pgfpathlineto{\pgfqpoint{1.816083in}{1.076544in}}%
\pgfpathlineto{\pgfqpoint{1.816636in}{1.127786in}}%
\pgfpathlineto{\pgfqpoint{1.817796in}{1.198994in}}%
\pgfpathlineto{\pgfqpoint{1.818010in}{1.165459in}}%
\pgfpathlineto{\pgfqpoint{1.819348in}{1.074029in}}%
\pgfpathlineto{\pgfqpoint{1.819455in}{1.075948in}}%
\pgfpathlineto{\pgfqpoint{1.820044in}{1.132773in}}%
\pgfpathlineto{\pgfqpoint{1.821168in}{1.197585in}}%
\pgfpathlineto{\pgfqpoint{1.821382in}{1.165787in}}%
\pgfpathlineto{\pgfqpoint{1.822720in}{1.074122in}}%
\pgfpathlineto{\pgfqpoint{1.822845in}{1.076930in}}%
\pgfpathlineto{\pgfqpoint{1.823558in}{1.147882in}}%
\pgfpathlineto{\pgfqpoint{1.824540in}{1.196826in}}%
\pgfpathlineto{\pgfqpoint{1.824789in}{1.158037in}}%
\pgfpathlineto{\pgfqpoint{1.826092in}{1.073578in}}%
\pgfpathlineto{\pgfqpoint{1.826217in}{1.076720in}}%
\pgfpathlineto{\pgfqpoint{1.827804in}{1.188368in}}%
\pgfpathlineto{\pgfqpoint{1.827929in}{1.197126in}}%
\pgfpathlineto{\pgfqpoint{1.828340in}{1.116648in}}%
\pgfpathlineto{\pgfqpoint{1.829464in}{1.074054in}}%
\pgfpathlineto{\pgfqpoint{1.829606in}{1.078217in}}%
\pgfpathlineto{\pgfqpoint{1.831301in}{1.196715in}}%
\pgfpathlineto{\pgfqpoint{1.831676in}{1.125095in}}%
\pgfpathlineto{\pgfqpoint{1.832051in}{1.075253in}}%
\pgfpathlineto{\pgfqpoint{1.832978in}{1.079777in}}%
\pgfpathlineto{\pgfqpoint{1.834673in}{1.196592in}}%
\pgfpathlineto{\pgfqpoint{1.835048in}{1.125596in}}%
\pgfpathlineto{\pgfqpoint{1.835422in}{1.074630in}}%
\pgfpathlineto{\pgfqpoint{1.836350in}{1.080023in}}%
\pgfpathlineto{\pgfqpoint{1.837938in}{1.189733in}}%
\pgfpathlineto{\pgfqpoint{1.838045in}{1.196954in}}%
\pgfpathlineto{\pgfqpoint{1.838438in}{1.121083in}}%
\pgfpathlineto{\pgfqpoint{1.838794in}{1.074025in}}%
\pgfpathlineto{\pgfqpoint{1.839704in}{1.079624in}}%
\pgfpathlineto{\pgfqpoint{1.841256in}{1.181206in}}%
\pgfpathlineto{\pgfqpoint{1.841417in}{1.195735in}}%
\pgfpathlineto{\pgfqpoint{1.841863in}{1.108466in}}%
\pgfpathlineto{\pgfqpoint{1.842166in}{1.074203in}}%
\pgfpathlineto{\pgfqpoint{1.843094in}{1.080936in}}%
\pgfpathlineto{\pgfqpoint{1.843754in}{1.139692in}}%
\pgfpathlineto{\pgfqpoint{1.844789in}{1.194884in}}%
\pgfpathlineto{\pgfqpoint{1.845039in}{1.155543in}}%
\pgfpathlineto{\pgfqpoint{1.845520in}{1.074124in}}%
\pgfpathlineto{\pgfqpoint{1.846448in}{1.079524in}}%
\pgfpathlineto{\pgfqpoint{1.847072in}{1.134484in}}%
\pgfpathlineto{\pgfqpoint{1.848161in}{1.193483in}}%
\pgfpathlineto{\pgfqpoint{1.848393in}{1.159267in}}%
\pgfpathlineto{\pgfqpoint{1.848892in}{1.074452in}}%
\pgfpathlineto{\pgfqpoint{1.849820in}{1.079152in}}%
\pgfpathlineto{\pgfqpoint{1.850373in}{1.125290in}}%
\pgfpathlineto{\pgfqpoint{1.851533in}{1.192376in}}%
\pgfpathlineto{\pgfqpoint{1.851747in}{1.163688in}}%
\pgfpathlineto{\pgfqpoint{1.852264in}{1.075675in}}%
\pgfpathlineto{\pgfqpoint{1.853192in}{1.078193in}}%
\pgfpathlineto{\pgfqpoint{1.853674in}{1.115732in}}%
\pgfpathlineto{\pgfqpoint{1.854905in}{1.190756in}}%
\pgfpathlineto{\pgfqpoint{1.855119in}{1.163212in}}%
\pgfpathlineto{\pgfqpoint{1.855636in}{1.075955in}}%
\pgfpathlineto{\pgfqpoint{1.856582in}{1.078499in}}%
\pgfpathlineto{\pgfqpoint{1.857117in}{1.125773in}}%
\pgfpathlineto{\pgfqpoint{1.858294in}{1.189095in}}%
\pgfpathlineto{\pgfqpoint{1.858491in}{1.162734in}}%
\pgfpathlineto{\pgfqpoint{1.859864in}{1.076535in}}%
\pgfpathlineto{\pgfqpoint{1.859954in}{1.077767in}}%
\pgfpathlineto{\pgfqpoint{1.860471in}{1.122997in}}%
\pgfpathlineto{\pgfqpoint{1.861666in}{1.188529in}}%
\pgfpathlineto{\pgfqpoint{1.861862in}{1.163698in}}%
\pgfpathlineto{\pgfqpoint{1.863254in}{1.075975in}}%
\pgfpathlineto{\pgfqpoint{1.863343in}{1.077639in}}%
\pgfpathlineto{\pgfqpoint{1.863861in}{1.125328in}}%
\pgfpathlineto{\pgfqpoint{1.865038in}{1.187580in}}%
\pgfpathlineto{\pgfqpoint{1.865234in}{1.163921in}}%
\pgfpathlineto{\pgfqpoint{1.866626in}{1.076441in}}%
\pgfpathlineto{\pgfqpoint{1.866715in}{1.078213in}}%
\pgfpathlineto{\pgfqpoint{1.867250in}{1.128393in}}%
\pgfpathlineto{\pgfqpoint{1.868428in}{1.186715in}}%
\pgfpathlineto{\pgfqpoint{1.868606in}{1.164256in}}%
\pgfpathlineto{\pgfqpoint{1.869980in}{1.076593in}}%
\pgfpathlineto{\pgfqpoint{1.870087in}{1.078209in}}%
\pgfpathlineto{\pgfqpoint{1.870622in}{1.127853in}}%
\pgfpathlineto{\pgfqpoint{1.871800in}{1.187168in}}%
\pgfpathlineto{\pgfqpoint{1.871996in}{1.161681in}}%
\pgfpathlineto{\pgfqpoint{1.873370in}{1.077115in}}%
\pgfpathlineto{\pgfqpoint{1.873459in}{1.078675in}}%
\pgfpathlineto{\pgfqpoint{1.873958in}{1.123317in}}%
\pgfpathlineto{\pgfqpoint{1.875172in}{1.186463in}}%
\pgfpathlineto{\pgfqpoint{1.875368in}{1.162091in}}%
\pgfpathlineto{\pgfqpoint{1.875903in}{1.078221in}}%
\pgfpathlineto{\pgfqpoint{1.876831in}{1.079985in}}%
\pgfpathlineto{\pgfqpoint{1.877420in}{1.133060in}}%
\pgfpathlineto{\pgfqpoint{1.878544in}{1.186508in}}%
\pgfpathlineto{\pgfqpoint{1.878758in}{1.158871in}}%
\pgfpathlineto{\pgfqpoint{1.879275in}{1.077317in}}%
\pgfpathlineto{\pgfqpoint{1.880203in}{1.080077in}}%
\pgfpathlineto{\pgfqpoint{1.880667in}{1.117709in}}%
\pgfpathlineto{\pgfqpoint{1.881915in}{1.186950in}}%
\pgfpathlineto{\pgfqpoint{1.882112in}{1.163128in}}%
\pgfpathlineto{\pgfqpoint{1.882647in}{1.076711in}}%
\pgfpathlineto{\pgfqpoint{1.883575in}{1.080720in}}%
\pgfpathlineto{\pgfqpoint{1.884074in}{1.120679in}}%
\pgfpathlineto{\pgfqpoint{1.885287in}{1.185361in}}%
\pgfpathlineto{\pgfqpoint{1.885484in}{1.162389in}}%
\pgfpathlineto{\pgfqpoint{1.886001in}{1.076966in}}%
\pgfpathlineto{\pgfqpoint{1.886947in}{1.081338in}}%
\pgfpathlineto{\pgfqpoint{1.887303in}{1.103340in}}%
\pgfpathlineto{\pgfqpoint{1.888659in}{1.184702in}}%
\pgfpathlineto{\pgfqpoint{1.888873in}{1.157960in}}%
\pgfpathlineto{\pgfqpoint{1.889373in}{1.076919in}}%
\pgfpathlineto{\pgfqpoint{1.890301in}{1.080360in}}%
\pgfpathlineto{\pgfqpoint{1.890586in}{1.093562in}}%
\pgfpathlineto{\pgfqpoint{1.892031in}{1.183037in}}%
\pgfpathlineto{\pgfqpoint{1.892281in}{1.148592in}}%
\pgfpathlineto{\pgfqpoint{1.892745in}{1.077069in}}%
\pgfpathlineto{\pgfqpoint{1.893673in}{1.079547in}}%
\pgfpathlineto{\pgfqpoint{1.893922in}{1.089209in}}%
\pgfpathlineto{\pgfqpoint{1.895421in}{1.181752in}}%
\pgfpathlineto{\pgfqpoint{1.895671in}{1.144605in}}%
\pgfpathlineto{\pgfqpoint{1.897027in}{1.078526in}}%
\pgfpathlineto{\pgfqpoint{1.897062in}{1.078635in}}%
\pgfpathlineto{\pgfqpoint{1.897312in}{1.089970in}}%
\pgfpathlineto{\pgfqpoint{1.898793in}{1.179646in}}%
\pgfpathlineto{\pgfqpoint{1.899025in}{1.148242in}}%
\pgfpathlineto{\pgfqpoint{1.900398in}{1.078192in}}%
\pgfpathlineto{\pgfqpoint{1.900434in}{1.078320in}}%
\pgfpathlineto{\pgfqpoint{1.900702in}{1.092066in}}%
\pgfpathlineto{\pgfqpoint{1.902165in}{1.178798in}}%
\pgfpathlineto{\pgfqpoint{1.902379in}{1.152722in}}%
\pgfpathlineto{\pgfqpoint{1.903770in}{1.077272in}}%
\pgfpathlineto{\pgfqpoint{1.903824in}{1.077641in}}%
\pgfpathlineto{\pgfqpoint{1.904163in}{1.101524in}}%
\pgfpathlineto{\pgfqpoint{1.905554in}{1.178613in}}%
\pgfpathlineto{\pgfqpoint{1.905733in}{1.158117in}}%
\pgfpathlineto{\pgfqpoint{1.907160in}{1.077389in}}%
\pgfpathlineto{\pgfqpoint{1.907231in}{1.078573in}}%
\pgfpathlineto{\pgfqpoint{1.907749in}{1.129038in}}%
\pgfpathlineto{\pgfqpoint{1.908926in}{1.177861in}}%
\pgfpathlineto{\pgfqpoint{1.909087in}{1.161653in}}%
\pgfpathlineto{\pgfqpoint{1.910514in}{1.078724in}}%
\pgfpathlineto{\pgfqpoint{1.910603in}{1.080044in}}%
\pgfpathlineto{\pgfqpoint{1.911174in}{1.136017in}}%
\pgfpathlineto{\pgfqpoint{1.912298in}{1.178133in}}%
\pgfpathlineto{\pgfqpoint{1.912477in}{1.159254in}}%
\pgfpathlineto{\pgfqpoint{1.913886in}{1.079181in}}%
\pgfpathlineto{\pgfqpoint{1.913975in}{1.080382in}}%
\pgfpathlineto{\pgfqpoint{1.914493in}{1.128154in}}%
\pgfpathlineto{\pgfqpoint{1.915670in}{1.178702in}}%
\pgfpathlineto{\pgfqpoint{1.915849in}{1.160494in}}%
\pgfpathlineto{\pgfqpoint{1.917276in}{1.080069in}}%
\pgfpathlineto{\pgfqpoint{1.917347in}{1.081084in}}%
\pgfpathlineto{\pgfqpoint{1.917847in}{1.125626in}}%
\pgfpathlineto{\pgfqpoint{1.919042in}{1.178434in}}%
\pgfpathlineto{\pgfqpoint{1.919220in}{1.160262in}}%
\pgfpathlineto{\pgfqpoint{1.919756in}{1.079358in}}%
\pgfpathlineto{\pgfqpoint{1.920719in}{1.082078in}}%
\pgfpathlineto{\pgfqpoint{1.921201in}{1.122219in}}%
\pgfpathlineto{\pgfqpoint{1.922414in}{1.178066in}}%
\pgfpathlineto{\pgfqpoint{1.922592in}{1.160236in}}%
\pgfpathlineto{\pgfqpoint{1.923110in}{1.078655in}}%
\pgfpathlineto{\pgfqpoint{1.924091in}{1.081757in}}%
\pgfpathlineto{\pgfqpoint{1.924483in}{1.111058in}}%
\pgfpathlineto{\pgfqpoint{1.925786in}{1.177454in}}%
\pgfpathlineto{\pgfqpoint{1.925982in}{1.156076in}}%
\pgfpathlineto{\pgfqpoint{1.926482in}{1.078807in}}%
\pgfpathlineto{\pgfqpoint{1.927445in}{1.081512in}}%
\pgfpathlineto{\pgfqpoint{1.927713in}{1.095456in}}%
\pgfpathlineto{\pgfqpoint{1.929176in}{1.175660in}}%
\pgfpathlineto{\pgfqpoint{1.929390in}{1.148207in}}%
\pgfpathlineto{\pgfqpoint{1.929854in}{1.079916in}}%
\pgfpathlineto{\pgfqpoint{1.930817in}{1.080814in}}%
\pgfpathlineto{\pgfqpoint{1.931013in}{1.088146in}}%
\pgfpathlineto{\pgfqpoint{1.932548in}{1.174497in}}%
\pgfpathlineto{\pgfqpoint{1.932797in}{1.140321in}}%
\pgfpathlineto{\pgfqpoint{1.934189in}{1.080108in}}%
\pgfpathlineto{\pgfqpoint{1.934385in}{1.087634in}}%
\pgfpathlineto{\pgfqpoint{1.935919in}{1.172678in}}%
\pgfpathlineto{\pgfqpoint{1.936151in}{1.143922in}}%
\pgfpathlineto{\pgfqpoint{1.937561in}{1.079621in}}%
\pgfpathlineto{\pgfqpoint{1.937686in}{1.082628in}}%
\pgfpathlineto{\pgfqpoint{1.939291in}{1.171630in}}%
\pgfpathlineto{\pgfqpoint{1.939595in}{1.130177in}}%
\pgfpathlineto{\pgfqpoint{1.940933in}{1.079297in}}%
\pgfpathlineto{\pgfqpoint{1.940951in}{1.079384in}}%
\pgfpathlineto{\pgfqpoint{1.941182in}{1.091732in}}%
\pgfpathlineto{\pgfqpoint{1.942663in}{1.170818in}}%
\pgfpathlineto{\pgfqpoint{1.942877in}{1.148937in}}%
\pgfpathlineto{\pgfqpoint{1.944287in}{1.079517in}}%
\pgfpathlineto{\pgfqpoint{1.944340in}{1.079838in}}%
\pgfpathlineto{\pgfqpoint{1.944733in}{1.112472in}}%
\pgfpathlineto{\pgfqpoint{1.946053in}{1.170648in}}%
\pgfpathlineto{\pgfqpoint{1.946196in}{1.159057in}}%
\pgfpathlineto{\pgfqpoint{1.947659in}{1.079718in}}%
\pgfpathlineto{\pgfqpoint{1.947784in}{1.082358in}}%
\pgfpathlineto{\pgfqpoint{1.949425in}{1.170569in}}%
\pgfpathlineto{\pgfqpoint{1.949746in}{1.126503in}}%
\pgfpathlineto{\pgfqpoint{1.951031in}{1.080446in}}%
\pgfpathlineto{\pgfqpoint{1.951084in}{1.080868in}}%
\pgfpathlineto{\pgfqpoint{1.951477in}{1.112636in}}%
\pgfpathlineto{\pgfqpoint{1.952797in}{1.170205in}}%
\pgfpathlineto{\pgfqpoint{1.952957in}{1.157124in}}%
\pgfpathlineto{\pgfqpoint{1.954420in}{1.081552in}}%
\pgfpathlineto{\pgfqpoint{1.954492in}{1.082741in}}%
\pgfpathlineto{\pgfqpoint{1.955080in}{1.138561in}}%
\pgfpathlineto{\pgfqpoint{1.956169in}{1.169824in}}%
\pgfpathlineto{\pgfqpoint{1.956347in}{1.154478in}}%
\pgfpathlineto{\pgfqpoint{1.957792in}{1.081777in}}%
\pgfpathlineto{\pgfqpoint{1.957864in}{1.082827in}}%
\pgfpathlineto{\pgfqpoint{1.958399in}{1.132228in}}%
\pgfpathlineto{\pgfqpoint{1.959558in}{1.170525in}}%
\pgfpathlineto{\pgfqpoint{1.959719in}{1.155587in}}%
\pgfpathlineto{\pgfqpoint{1.960236in}{1.081655in}}%
\pgfpathlineto{\pgfqpoint{1.961235in}{1.082999in}}%
\pgfpathlineto{\pgfqpoint{1.961717in}{1.125779in}}%
\pgfpathlineto{\pgfqpoint{1.962930in}{1.169164in}}%
\pgfpathlineto{\pgfqpoint{1.963073in}{1.157275in}}%
\pgfpathlineto{\pgfqpoint{1.963608in}{1.081130in}}%
\pgfpathlineto{\pgfqpoint{1.964625in}{1.083992in}}%
\pgfpathlineto{\pgfqpoint{1.965196in}{1.136095in}}%
\pgfpathlineto{\pgfqpoint{1.966302in}{1.168618in}}%
\pgfpathlineto{\pgfqpoint{1.966463in}{1.154435in}}%
\pgfpathlineto{\pgfqpoint{1.966980in}{1.081505in}}%
\pgfpathlineto{\pgfqpoint{1.967979in}{1.083074in}}%
\pgfpathlineto{\pgfqpoint{1.968354in}{1.112190in}}%
\pgfpathlineto{\pgfqpoint{1.969674in}{1.168272in}}%
\pgfpathlineto{\pgfqpoint{1.969835in}{1.154303in}}%
\pgfpathlineto{\pgfqpoint{1.970352in}{1.081988in}}%
\pgfpathlineto{\pgfqpoint{1.971351in}{1.082808in}}%
\pgfpathlineto{\pgfqpoint{1.971726in}{1.111531in}}%
\pgfpathlineto{\pgfqpoint{1.973046in}{1.166834in}}%
\pgfpathlineto{\pgfqpoint{1.973189in}{1.156538in}}%
\pgfpathlineto{\pgfqpoint{1.974687in}{1.081748in}}%
\pgfpathlineto{\pgfqpoint{1.974777in}{1.083035in}}%
\pgfpathlineto{\pgfqpoint{1.975365in}{1.141007in}}%
\pgfpathlineto{\pgfqpoint{1.976418in}{1.165492in}}%
\pgfpathlineto{\pgfqpoint{1.976578in}{1.153188in}}%
\pgfpathlineto{\pgfqpoint{1.978077in}{1.081587in}}%
\pgfpathlineto{\pgfqpoint{1.978131in}{1.082405in}}%
\pgfpathlineto{\pgfqpoint{1.978612in}{1.128091in}}%
\pgfpathlineto{\pgfqpoint{1.979808in}{1.164401in}}%
\pgfpathlineto{\pgfqpoint{1.979933in}{1.155622in}}%
\pgfpathlineto{\pgfqpoint{1.981449in}{1.081352in}}%
\pgfpathlineto{\pgfqpoint{1.981538in}{1.083219in}}%
\pgfpathlineto{\pgfqpoint{1.983180in}{1.163786in}}%
\pgfpathlineto{\pgfqpoint{1.983501in}{1.122675in}}%
\pgfpathlineto{\pgfqpoint{1.984821in}{1.081397in}}%
\pgfpathlineto{\pgfqpoint{1.984839in}{1.081501in}}%
\pgfpathlineto{\pgfqpoint{1.985106in}{1.099343in}}%
\pgfpathlineto{\pgfqpoint{1.986551in}{1.163405in}}%
\pgfpathlineto{\pgfqpoint{1.986694in}{1.153982in}}%
\pgfpathlineto{\pgfqpoint{1.988175in}{1.081628in}}%
\pgfpathlineto{\pgfqpoint{1.988282in}{1.083794in}}%
\pgfpathlineto{\pgfqpoint{1.989941in}{1.162881in}}%
\pgfpathlineto{\pgfqpoint{1.990245in}{1.124105in}}%
\pgfpathlineto{\pgfqpoint{1.991565in}{1.082451in}}%
\pgfpathlineto{\pgfqpoint{1.991583in}{1.082567in}}%
\pgfpathlineto{\pgfqpoint{1.991886in}{1.104598in}}%
\pgfpathlineto{\pgfqpoint{1.993313in}{1.163289in}}%
\pgfpathlineto{\pgfqpoint{1.993438in}{1.155314in}}%
\pgfpathlineto{\pgfqpoint{1.994937in}{1.082868in}}%
\pgfpathlineto{\pgfqpoint{1.995062in}{1.086382in}}%
\pgfpathlineto{\pgfqpoint{1.996685in}{1.163099in}}%
\pgfpathlineto{\pgfqpoint{1.996953in}{1.132383in}}%
\pgfpathlineto{\pgfqpoint{1.998309in}{1.084077in}}%
\pgfpathlineto{\pgfqpoint{1.998326in}{1.084170in}}%
\pgfpathlineto{\pgfqpoint{1.998576in}{1.099182in}}%
\pgfpathlineto{\pgfqpoint{2.000057in}{1.163070in}}%
\pgfpathlineto{\pgfqpoint{2.000182in}{1.155818in}}%
\pgfpathlineto{\pgfqpoint{2.000735in}{1.083654in}}%
\pgfpathlineto{\pgfqpoint{2.001823in}{1.088198in}}%
\pgfpathlineto{\pgfqpoint{2.003429in}{1.163311in}}%
\pgfpathlineto{\pgfqpoint{2.003679in}{1.136449in}}%
\pgfpathlineto{\pgfqpoint{2.004107in}{1.083189in}}%
\pgfpathlineto{\pgfqpoint{2.005052in}{1.084872in}}%
\pgfpathlineto{\pgfqpoint{2.005070in}{1.084825in}}%
\pgfpathlineto{\pgfqpoint{2.005177in}{1.087301in}}%
\pgfpathlineto{\pgfqpoint{2.006801in}{1.161785in}}%
\pgfpathlineto{\pgfqpoint{2.007086in}{1.129049in}}%
\pgfpathlineto{\pgfqpoint{2.007479in}{1.083084in}}%
\pgfpathlineto{\pgfqpoint{2.008424in}{1.085602in}}%
\pgfpathlineto{\pgfqpoint{2.008460in}{1.085422in}}%
\pgfpathlineto{\pgfqpoint{2.008638in}{1.092706in}}%
\pgfpathlineto{\pgfqpoint{2.010173in}{1.161455in}}%
\pgfpathlineto{\pgfqpoint{2.010369in}{1.144950in}}%
\pgfpathlineto{\pgfqpoint{2.010851in}{1.083665in}}%
\pgfpathlineto{\pgfqpoint{2.011832in}{1.084253in}}%
\pgfpathlineto{\pgfqpoint{2.011992in}{1.090140in}}%
\pgfpathlineto{\pgfqpoint{2.013545in}{1.160313in}}%
\pgfpathlineto{\pgfqpoint{2.013759in}{1.141401in}}%
\pgfpathlineto{\pgfqpoint{2.015204in}{1.083997in}}%
\pgfpathlineto{\pgfqpoint{2.015329in}{1.087338in}}%
\pgfpathlineto{\pgfqpoint{2.016934in}{1.158803in}}%
\pgfpathlineto{\pgfqpoint{2.017166in}{1.135249in}}%
\pgfpathlineto{\pgfqpoint{2.018576in}{1.083236in}}%
\pgfpathlineto{\pgfqpoint{2.018593in}{1.083224in}}%
\pgfpathlineto{\pgfqpoint{2.018647in}{1.084253in}}%
\pgfpathlineto{\pgfqpoint{2.019271in}{1.145225in}}%
\pgfpathlineto{\pgfqpoint{2.020306in}{1.157581in}}%
\pgfpathlineto{\pgfqpoint{2.020431in}{1.150739in}}%
\pgfpathlineto{\pgfqpoint{2.021948in}{1.083083in}}%
\pgfpathlineto{\pgfqpoint{2.022090in}{1.087775in}}%
\pgfpathlineto{\pgfqpoint{2.023678in}{1.157162in}}%
\pgfpathlineto{\pgfqpoint{2.023892in}{1.139351in}}%
\pgfpathlineto{\pgfqpoint{2.025319in}{1.083229in}}%
\pgfpathlineto{\pgfqpoint{2.025355in}{1.083524in}}%
\pgfpathlineto{\pgfqpoint{2.025712in}{1.114113in}}%
\pgfpathlineto{\pgfqpoint{2.027068in}{1.157138in}}%
\pgfpathlineto{\pgfqpoint{2.027157in}{1.153337in}}%
\pgfpathlineto{\pgfqpoint{2.028691in}{1.083663in}}%
\pgfpathlineto{\pgfqpoint{2.028905in}{1.094722in}}%
\pgfpathlineto{\pgfqpoint{2.030440in}{1.157220in}}%
\pgfpathlineto{\pgfqpoint{2.030565in}{1.150487in}}%
\pgfpathlineto{\pgfqpoint{2.032063in}{1.084615in}}%
\pgfpathlineto{\pgfqpoint{2.032206in}{1.089656in}}%
\pgfpathlineto{\pgfqpoint{2.033812in}{1.157158in}}%
\pgfpathlineto{\pgfqpoint{2.034008in}{1.141738in}}%
\pgfpathlineto{\pgfqpoint{2.035435in}{1.085456in}}%
\pgfpathlineto{\pgfqpoint{2.035471in}{1.085634in}}%
\pgfpathlineto{\pgfqpoint{2.035792in}{1.110814in}}%
\pgfpathlineto{\pgfqpoint{2.037184in}{1.157973in}}%
\pgfpathlineto{\pgfqpoint{2.037273in}{1.155177in}}%
\pgfpathlineto{\pgfqpoint{2.038825in}{1.085898in}}%
\pgfpathlineto{\pgfqpoint{2.039075in}{1.100565in}}%
\pgfpathlineto{\pgfqpoint{2.040555in}{1.157446in}}%
\pgfpathlineto{\pgfqpoint{2.040680in}{1.151605in}}%
\pgfpathlineto{\pgfqpoint{2.041233in}{1.085064in}}%
\pgfpathlineto{\pgfqpoint{2.042340in}{1.091566in}}%
\pgfpathlineto{\pgfqpoint{2.043927in}{1.156710in}}%
\pgfpathlineto{\pgfqpoint{2.044124in}{1.142358in}}%
\pgfpathlineto{\pgfqpoint{2.044605in}{1.085083in}}%
\pgfpathlineto{\pgfqpoint{2.045604in}{1.086761in}}%
\pgfpathlineto{\pgfqpoint{2.045872in}{1.105055in}}%
\pgfpathlineto{\pgfqpoint{2.047299in}{1.156392in}}%
\pgfpathlineto{\pgfqpoint{2.047406in}{1.152484in}}%
\pgfpathlineto{\pgfqpoint{2.047977in}{1.085356in}}%
\pgfpathlineto{\pgfqpoint{2.049155in}{1.095554in}}%
\pgfpathlineto{\pgfqpoint{2.050689in}{1.154991in}}%
\pgfpathlineto{\pgfqpoint{2.050814in}{1.148329in}}%
\pgfpathlineto{\pgfqpoint{2.051349in}{1.085755in}}%
\pgfpathlineto{\pgfqpoint{2.052437in}{1.088852in}}%
\pgfpathlineto{\pgfqpoint{2.054061in}{1.154134in}}%
\pgfpathlineto{\pgfqpoint{2.054311in}{1.131068in}}%
\pgfpathlineto{\pgfqpoint{2.055702in}{1.085617in}}%
\pgfpathlineto{\pgfqpoint{2.055774in}{1.086606in}}%
\pgfpathlineto{\pgfqpoint{2.056594in}{1.154350in}}%
\pgfpathlineto{\pgfqpoint{2.057754in}{1.120226in}}%
\pgfpathlineto{\pgfqpoint{2.059074in}{1.085689in}}%
\pgfpathlineto{\pgfqpoint{2.059163in}{1.087212in}}%
\pgfpathlineto{\pgfqpoint{2.059966in}{1.154176in}}%
\pgfpathlineto{\pgfqpoint{2.061090in}{1.126793in}}%
\pgfpathlineto{\pgfqpoint{2.062446in}{1.085675in}}%
\pgfpathlineto{\pgfqpoint{2.062464in}{1.085707in}}%
\pgfpathlineto{\pgfqpoint{2.062678in}{1.097980in}}%
\pgfpathlineto{\pgfqpoint{2.063338in}{1.153865in}}%
\pgfpathlineto{\pgfqpoint{2.064284in}{1.149253in}}%
\pgfpathlineto{\pgfqpoint{2.065818in}{1.085691in}}%
\pgfpathlineto{\pgfqpoint{2.066068in}{1.099957in}}%
\pgfpathlineto{\pgfqpoint{2.066710in}{1.154274in}}%
\pgfpathlineto{\pgfqpoint{2.067638in}{1.150444in}}%
\pgfpathlineto{\pgfqpoint{2.069190in}{1.086419in}}%
\pgfpathlineto{\pgfqpoint{2.069493in}{1.106370in}}%
\pgfpathlineto{\pgfqpoint{2.070082in}{1.153300in}}%
\pgfpathlineto{\pgfqpoint{2.070992in}{1.151350in}}%
\pgfpathlineto{\pgfqpoint{2.071456in}{1.096271in}}%
\pgfpathlineto{\pgfqpoint{2.072580in}{1.087018in}}%
\pgfpathlineto{\pgfqpoint{2.072134in}{1.112501in}}%
\pgfpathlineto{\pgfqpoint{2.072651in}{1.088513in}}%
\pgfpathlineto{\pgfqpoint{2.073454in}{1.152878in}}%
\pgfpathlineto{\pgfqpoint{2.074596in}{1.127159in}}%
\pgfpathlineto{\pgfqpoint{2.075951in}{1.087797in}}%
\pgfpathlineto{\pgfqpoint{2.076094in}{1.093183in}}%
\pgfpathlineto{\pgfqpoint{2.076826in}{1.152397in}}%
\pgfpathlineto{\pgfqpoint{2.077860in}{1.142205in}}%
\pgfpathlineto{\pgfqpoint{2.078360in}{1.087669in}}%
\pgfpathlineto{\pgfqpoint{2.079395in}{1.089206in}}%
\pgfpathlineto{\pgfqpoint{2.081072in}{1.152219in}}%
\pgfpathlineto{\pgfqpoint{2.081357in}{1.125494in}}%
\pgfpathlineto{\pgfqpoint{2.081732in}{1.087380in}}%
\pgfpathlineto{\pgfqpoint{2.082695in}{1.088406in}}%
\pgfpathlineto{\pgfqpoint{2.082713in}{1.088390in}}%
\pgfpathlineto{\pgfqpoint{2.082767in}{1.089420in}}%
\pgfpathlineto{\pgfqpoint{2.083587in}{1.151776in}}%
\pgfpathlineto{\pgfqpoint{2.084747in}{1.122346in}}%
\pgfpathlineto{\pgfqpoint{2.085104in}{1.087239in}}%
\pgfpathlineto{\pgfqpoint{2.086067in}{1.089353in}}%
\pgfpathlineto{\pgfqpoint{2.086085in}{1.089294in}}%
\pgfpathlineto{\pgfqpoint{2.086192in}{1.092346in}}%
\pgfpathlineto{\pgfqpoint{2.086959in}{1.151755in}}%
\pgfpathlineto{\pgfqpoint{2.088012in}{1.137791in}}%
\pgfpathlineto{\pgfqpoint{2.088476in}{1.087299in}}%
\pgfpathlineto{\pgfqpoint{2.089493in}{1.088757in}}%
\pgfpathlineto{\pgfqpoint{2.089832in}{1.116657in}}%
\pgfpathlineto{\pgfqpoint{2.090349in}{1.151872in}}%
\pgfpathlineto{\pgfqpoint{2.091241in}{1.149790in}}%
\pgfpathlineto{\pgfqpoint{2.091865in}{1.087661in}}%
\pgfpathlineto{\pgfqpoint{2.093186in}{1.114108in}}%
\pgfpathlineto{\pgfqpoint{2.093721in}{1.152396in}}%
\pgfpathlineto{\pgfqpoint{2.094595in}{1.149056in}}%
\pgfpathlineto{\pgfqpoint{2.094881in}{1.119859in}}%
\pgfpathlineto{\pgfqpoint{2.095219in}{1.088127in}}%
\pgfpathlineto{\pgfqpoint{2.096201in}{1.088733in}}%
\pgfpathlineto{\pgfqpoint{2.096219in}{1.088653in}}%
\pgfpathlineto{\pgfqpoint{2.096343in}{1.092813in}}%
\pgfpathlineto{\pgfqpoint{2.097093in}{1.152930in}}%
\pgfpathlineto{\pgfqpoint{2.098092in}{1.141171in}}%
\pgfpathlineto{\pgfqpoint{2.099590in}{1.087920in}}%
\pgfpathlineto{\pgfqpoint{2.099680in}{1.090099in}}%
\pgfpathlineto{\pgfqpoint{2.100465in}{1.153294in}}%
\pgfpathlineto{\pgfqpoint{2.101553in}{1.131190in}}%
\pgfpathlineto{\pgfqpoint{2.102962in}{1.087878in}}%
\pgfpathlineto{\pgfqpoint{2.102980in}{1.087920in}}%
\pgfpathlineto{\pgfqpoint{2.103230in}{1.105790in}}%
\pgfpathlineto{\pgfqpoint{2.103837in}{1.153015in}}%
\pgfpathlineto{\pgfqpoint{2.104693in}{1.147455in}}%
\pgfpathlineto{\pgfqpoint{2.104711in}{1.147481in}}%
\pgfpathlineto{\pgfqpoint{2.104764in}{1.146327in}}%
\pgfpathlineto{\pgfqpoint{2.106334in}{1.088133in}}%
\pgfpathlineto{\pgfqpoint{2.106638in}{1.110192in}}%
\pgfpathlineto{\pgfqpoint{2.107208in}{1.152784in}}%
\pgfpathlineto{\pgfqpoint{2.108047in}{1.147290in}}%
\pgfpathlineto{\pgfqpoint{2.108083in}{1.147626in}}%
\pgfpathlineto{\pgfqpoint{2.108279in}{1.134944in}}%
\pgfpathlineto{\pgfqpoint{2.109706in}{1.088441in}}%
\pgfpathlineto{\pgfqpoint{2.109760in}{1.089254in}}%
\pgfpathlineto{\pgfqpoint{2.110402in}{1.147535in}}%
\pgfpathlineto{\pgfqpoint{2.110580in}{1.152791in}}%
\pgfpathlineto{\pgfqpoint{2.111151in}{1.140436in}}%
\pgfpathlineto{\pgfqpoint{2.111490in}{1.147126in}}%
\pgfpathlineto{\pgfqpoint{2.111829in}{1.111754in}}%
\pgfpathlineto{\pgfqpoint{2.113078in}{1.089544in}}%
\pgfpathlineto{\pgfqpoint{2.112632in}{1.112642in}}%
\pgfpathlineto{\pgfqpoint{2.113114in}{1.089924in}}%
\pgfpathlineto{\pgfqpoint{2.113471in}{1.121427in}}%
\pgfpathlineto{\pgfqpoint{2.113952in}{1.151612in}}%
\pgfpathlineto{\pgfqpoint{2.114809in}{1.147540in}}%
\pgfpathlineto{\pgfqpoint{2.114844in}{1.147764in}}%
\pgfpathlineto{\pgfqpoint{2.115005in}{1.138446in}}%
\pgfpathlineto{\pgfqpoint{2.116450in}{1.089890in}}%
\pgfpathlineto{\pgfqpoint{2.116521in}{1.091017in}}%
\pgfpathlineto{\pgfqpoint{2.117324in}{1.150897in}}%
\pgfpathlineto{\pgfqpoint{2.118484in}{1.125008in}}%
\pgfpathlineto{\pgfqpoint{2.118876in}{1.089059in}}%
\pgfpathlineto{\pgfqpoint{2.119822in}{1.090917in}}%
\pgfpathlineto{\pgfqpoint{2.119840in}{1.090911in}}%
\pgfpathlineto{\pgfqpoint{2.119893in}{1.091906in}}%
\pgfpathlineto{\pgfqpoint{2.120696in}{1.150635in}}%
\pgfpathlineto{\pgfqpoint{2.121856in}{1.125069in}}%
\pgfpathlineto{\pgfqpoint{2.122248in}{1.088879in}}%
\pgfpathlineto{\pgfqpoint{2.123194in}{1.091341in}}%
\pgfpathlineto{\pgfqpoint{2.123212in}{1.091283in}}%
\pgfpathlineto{\pgfqpoint{2.123319in}{1.094322in}}%
\pgfpathlineto{\pgfqpoint{2.124086in}{1.150416in}}%
\pgfpathlineto{\pgfqpoint{2.125138in}{1.136815in}}%
\pgfpathlineto{\pgfqpoint{2.125620in}{1.088739in}}%
\pgfpathlineto{\pgfqpoint{2.126619in}{1.091423in}}%
\pgfpathlineto{\pgfqpoint{2.126994in}{1.123190in}}%
\pgfpathlineto{\pgfqpoint{2.127458in}{1.150379in}}%
\pgfpathlineto{\pgfqpoint{2.128332in}{1.146858in}}%
\pgfpathlineto{\pgfqpoint{2.128510in}{1.136189in}}%
\pgfpathlineto{\pgfqpoint{2.128992in}{1.088988in}}%
\pgfpathlineto{\pgfqpoint{2.129991in}{1.091712in}}%
\pgfpathlineto{\pgfqpoint{2.130294in}{1.114871in}}%
\pgfpathlineto{\pgfqpoint{2.130847in}{1.150790in}}%
\pgfpathlineto{\pgfqpoint{2.131686in}{1.146052in}}%
\pgfpathlineto{\pgfqpoint{2.131704in}{1.146101in}}%
\pgfpathlineto{\pgfqpoint{2.131793in}{1.143462in}}%
\pgfpathlineto{\pgfqpoint{2.132364in}{1.089480in}}%
\pgfpathlineto{\pgfqpoint{2.133559in}{1.102531in}}%
\pgfpathlineto{\pgfqpoint{2.134219in}{1.151330in}}%
\pgfpathlineto{\pgfqpoint{2.135076in}{1.145578in}}%
\pgfpathlineto{\pgfqpoint{2.135218in}{1.139215in}}%
\pgfpathlineto{\pgfqpoint{2.135736in}{1.089998in}}%
\pgfpathlineto{\pgfqpoint{2.136824in}{1.093511in}}%
\pgfpathlineto{\pgfqpoint{2.137591in}{1.151747in}}%
\pgfpathlineto{\pgfqpoint{2.138608in}{1.137191in}}%
\pgfpathlineto{\pgfqpoint{2.140089in}{1.090526in}}%
\pgfpathlineto{\pgfqpoint{2.140160in}{1.091802in}}%
\pgfpathlineto{\pgfqpoint{2.140945in}{1.151707in}}%
\pgfpathlineto{\pgfqpoint{2.142069in}{1.128496in}}%
\pgfpathlineto{\pgfqpoint{2.143461in}{1.090277in}}%
\pgfpathlineto{\pgfqpoint{2.143497in}{1.090575in}}%
\pgfpathlineto{\pgfqpoint{2.143853in}{1.122581in}}%
\pgfpathlineto{\pgfqpoint{2.144317in}{1.151995in}}%
\pgfpathlineto{\pgfqpoint{2.145138in}{1.143021in}}%
\pgfpathlineto{\pgfqpoint{2.145209in}{1.144143in}}%
\pgfpathlineto{\pgfqpoint{2.145477in}{1.125302in}}%
\pgfpathlineto{\pgfqpoint{2.146833in}{1.090578in}}%
\pgfpathlineto{\pgfqpoint{2.146851in}{1.090623in}}%
\pgfpathlineto{\pgfqpoint{2.147100in}{1.108429in}}%
\pgfpathlineto{\pgfqpoint{2.147689in}{1.151514in}}%
\pgfpathlineto{\pgfqpoint{2.148510in}{1.142446in}}%
\pgfpathlineto{\pgfqpoint{2.148581in}{1.143782in}}%
\pgfpathlineto{\pgfqpoint{2.148867in}{1.123686in}}%
\pgfpathlineto{\pgfqpoint{2.150205in}{1.091638in}}%
\pgfpathlineto{\pgfqpoint{2.150223in}{1.091672in}}%
\pgfpathlineto{\pgfqpoint{2.150472in}{1.108890in}}%
\pgfpathlineto{\pgfqpoint{2.151061in}{1.150909in}}%
\pgfpathlineto{\pgfqpoint{2.151864in}{1.142058in}}%
\pgfpathlineto{\pgfqpoint{2.151971in}{1.144195in}}%
\pgfpathlineto{\pgfqpoint{2.152274in}{1.119924in}}%
\pgfpathlineto{\pgfqpoint{2.152631in}{1.091301in}}%
\pgfpathlineto{\pgfqpoint{2.153594in}{1.092200in}}%
\pgfpathlineto{\pgfqpoint{2.153809in}{1.105260in}}%
\pgfpathlineto{\pgfqpoint{2.154433in}{1.150572in}}%
\pgfpathlineto{\pgfqpoint{2.155271in}{1.143236in}}%
\pgfpathlineto{\pgfqpoint{2.155343in}{1.144300in}}%
\pgfpathlineto{\pgfqpoint{2.155610in}{1.124636in}}%
\pgfpathlineto{\pgfqpoint{2.156003in}{1.090874in}}%
\pgfpathlineto{\pgfqpoint{2.156966in}{1.092806in}}%
\pgfpathlineto{\pgfqpoint{2.157145in}{1.101524in}}%
\pgfpathlineto{\pgfqpoint{2.157805in}{1.149477in}}%
\pgfpathlineto{\pgfqpoint{2.158715in}{1.144527in}}%
\pgfpathlineto{\pgfqpoint{2.158857in}{1.138709in}}%
\pgfpathlineto{\pgfqpoint{2.159375in}{1.090887in}}%
\pgfpathlineto{\pgfqpoint{2.160445in}{1.096171in}}%
\pgfpathlineto{\pgfqpoint{2.161195in}{1.149532in}}%
\pgfpathlineto{\pgfqpoint{2.162229in}{1.138484in}}%
\pgfpathlineto{\pgfqpoint{2.162747in}{1.090361in}}%
\pgfpathlineto{\pgfqpoint{2.163817in}{1.096668in}}%
\pgfpathlineto{\pgfqpoint{2.164566in}{1.149656in}}%
\pgfpathlineto{\pgfqpoint{2.165619in}{1.136472in}}%
\pgfpathlineto{\pgfqpoint{2.166119in}{1.090528in}}%
\pgfpathlineto{\pgfqpoint{2.167171in}{1.095360in}}%
\pgfpathlineto{\pgfqpoint{2.167938in}{1.149402in}}%
\pgfpathlineto{\pgfqpoint{2.169062in}{1.129768in}}%
\pgfpathlineto{\pgfqpoint{2.169491in}{1.091257in}}%
\pgfpathlineto{\pgfqpoint{2.170472in}{1.093629in}}%
\pgfpathlineto{\pgfqpoint{2.170597in}{1.097709in}}%
\pgfpathlineto{\pgfqpoint{2.171328in}{1.149787in}}%
\pgfpathlineto{\pgfqpoint{2.172309in}{1.139663in}}%
\pgfpathlineto{\pgfqpoint{2.172880in}{1.090982in}}%
\pgfpathlineto{\pgfqpoint{2.174040in}{1.102990in}}%
\pgfpathlineto{\pgfqpoint{2.174700in}{1.150177in}}%
\pgfpathlineto{\pgfqpoint{2.175556in}{1.141574in}}%
\pgfpathlineto{\pgfqpoint{2.175592in}{1.141736in}}%
\pgfpathlineto{\pgfqpoint{2.175735in}{1.136122in}}%
\pgfpathlineto{\pgfqpoint{2.176252in}{1.091766in}}%
\pgfpathlineto{\pgfqpoint{2.177305in}{1.095218in}}%
\pgfpathlineto{\pgfqpoint{2.178072in}{1.150465in}}%
\pgfpathlineto{\pgfqpoint{2.179142in}{1.133259in}}%
\pgfpathlineto{\pgfqpoint{2.179624in}{1.092350in}}%
\pgfpathlineto{\pgfqpoint{2.180659in}{1.094421in}}%
\pgfpathlineto{\pgfqpoint{2.181444in}{1.150328in}}%
\pgfpathlineto{\pgfqpoint{2.182568in}{1.128183in}}%
\pgfpathlineto{\pgfqpoint{2.182996in}{1.092885in}}%
\pgfpathlineto{\pgfqpoint{2.183995in}{1.093810in}}%
\pgfpathlineto{\pgfqpoint{2.184316in}{1.120452in}}%
\pgfpathlineto{\pgfqpoint{2.184798in}{1.150010in}}%
\pgfpathlineto{\pgfqpoint{2.185565in}{1.137853in}}%
\pgfpathlineto{\pgfqpoint{2.185726in}{1.140717in}}%
\pgfpathlineto{\pgfqpoint{2.185993in}{1.123346in}}%
\pgfpathlineto{\pgfqpoint{2.186386in}{1.092942in}}%
\pgfpathlineto{\pgfqpoint{2.187367in}{1.093855in}}%
\pgfpathlineto{\pgfqpoint{2.187706in}{1.122708in}}%
\pgfpathlineto{\pgfqpoint{2.188170in}{1.150439in}}%
\pgfpathlineto{\pgfqpoint{2.188937in}{1.137900in}}%
\pgfpathlineto{\pgfqpoint{2.189098in}{1.140931in}}%
\pgfpathlineto{\pgfqpoint{2.189365in}{1.123857in}}%
\pgfpathlineto{\pgfqpoint{2.189758in}{1.092555in}}%
\pgfpathlineto{\pgfqpoint{2.190721in}{1.094356in}}%
\pgfpathlineto{\pgfqpoint{2.190935in}{1.106899in}}%
\pgfpathlineto{\pgfqpoint{2.191542in}{1.149766in}}%
\pgfpathlineto{\pgfqpoint{2.192345in}{1.138355in}}%
\pgfpathlineto{\pgfqpoint{2.192487in}{1.140793in}}%
\pgfpathlineto{\pgfqpoint{2.192755in}{1.122575in}}%
\pgfpathlineto{\pgfqpoint{2.193130in}{1.093004in}}%
\pgfpathlineto{\pgfqpoint{2.194093in}{1.094992in}}%
\pgfpathlineto{\pgfqpoint{2.194236in}{1.100544in}}%
\pgfpathlineto{\pgfqpoint{2.194914in}{1.149033in}}%
\pgfpathlineto{\pgfqpoint{2.195859in}{1.141016in}}%
\pgfpathlineto{\pgfqpoint{2.196055in}{1.130541in}}%
\pgfpathlineto{\pgfqpoint{2.196519in}{1.092462in}}%
\pgfpathlineto{\pgfqpoint{2.197501in}{1.095773in}}%
\pgfpathlineto{\pgfqpoint{2.197911in}{1.131886in}}%
\pgfpathlineto{\pgfqpoint{2.198286in}{1.148924in}}%
\pgfpathlineto{\pgfqpoint{2.199088in}{1.138828in}}%
\pgfpathlineto{\pgfqpoint{2.199213in}{1.141209in}}%
\pgfpathlineto{\pgfqpoint{2.199517in}{1.121095in}}%
\pgfpathlineto{\pgfqpoint{2.199891in}{1.092252in}}%
\pgfpathlineto{\pgfqpoint{2.200837in}{1.096046in}}%
\pgfpathlineto{\pgfqpoint{2.200926in}{1.097870in}}%
\pgfpathlineto{\pgfqpoint{2.201657in}{1.148472in}}%
\pgfpathlineto{\pgfqpoint{2.202781in}{1.132681in}}%
\pgfpathlineto{\pgfqpoint{2.203263in}{1.091976in}}%
\pgfpathlineto{\pgfqpoint{2.204262in}{1.096462in}}%
\pgfpathlineto{\pgfqpoint{2.204690in}{1.133731in}}%
\pgfpathlineto{\pgfqpoint{2.205047in}{1.148022in}}%
\pgfpathlineto{\pgfqpoint{2.205850in}{1.138691in}}%
\pgfpathlineto{\pgfqpoint{2.205975in}{1.140505in}}%
\pgfpathlineto{\pgfqpoint{2.206242in}{1.123047in}}%
\pgfpathlineto{\pgfqpoint{2.206635in}{1.091867in}}%
\pgfpathlineto{\pgfqpoint{2.207581in}{1.096889in}}%
\pgfpathlineto{\pgfqpoint{2.207598in}{1.096790in}}%
\pgfpathlineto{\pgfqpoint{2.207741in}{1.101950in}}%
\pgfpathlineto{\pgfqpoint{2.208419in}{1.148397in}}%
\pgfpathlineto{\pgfqpoint{2.209365in}{1.139886in}}%
\pgfpathlineto{\pgfqpoint{2.209597in}{1.125165in}}%
\pgfpathlineto{\pgfqpoint{2.210007in}{1.092296in}}%
\pgfpathlineto{\pgfqpoint{2.210970in}{1.096326in}}%
\pgfpathlineto{\pgfqpoint{2.211059in}{1.098014in}}%
\pgfpathlineto{\pgfqpoint{2.211791in}{1.148202in}}%
\pgfpathlineto{\pgfqpoint{2.212897in}{1.132328in}}%
\pgfpathlineto{\pgfqpoint{2.213379in}{1.092842in}}%
\pgfpathlineto{\pgfqpoint{2.214396in}{1.096401in}}%
\pgfpathlineto{\pgfqpoint{2.214895in}{1.139780in}}%
\pgfpathlineto{\pgfqpoint{2.215181in}{1.148828in}}%
\pgfpathlineto{\pgfqpoint{2.215841in}{1.136011in}}%
\pgfpathlineto{\pgfqpoint{2.215948in}{1.136828in}}%
\pgfpathlineto{\pgfqpoint{2.216091in}{1.138887in}}%
\pgfpathlineto{\pgfqpoint{2.216340in}{1.125210in}}%
\pgfpathlineto{\pgfqpoint{2.216751in}{1.092906in}}%
\pgfpathlineto{\pgfqpoint{2.217714in}{1.096470in}}%
\pgfpathlineto{\pgfqpoint{2.217732in}{1.096412in}}%
\pgfpathlineto{\pgfqpoint{2.217839in}{1.099557in}}%
\pgfpathlineto{\pgfqpoint{2.218535in}{1.149178in}}%
\pgfpathlineto{\pgfqpoint{2.219552in}{1.136953in}}%
\pgfpathlineto{\pgfqpoint{2.220140in}{1.093736in}}%
\pgfpathlineto{\pgfqpoint{2.221318in}{1.107678in}}%
\pgfpathlineto{\pgfqpoint{2.221925in}{1.149071in}}%
\pgfpathlineto{\pgfqpoint{2.222692in}{1.135718in}}%
\pgfpathlineto{\pgfqpoint{2.222852in}{1.138108in}}%
\pgfpathlineto{\pgfqpoint{2.223102in}{1.124704in}}%
\pgfpathlineto{\pgfqpoint{2.223512in}{1.094319in}}%
\pgfpathlineto{\pgfqpoint{2.224494in}{1.096318in}}%
\pgfpathlineto{\pgfqpoint{2.224726in}{1.111679in}}%
\pgfpathlineto{\pgfqpoint{2.225279in}{1.149268in}}%
\pgfpathlineto{\pgfqpoint{2.226028in}{1.134697in}}%
\pgfpathlineto{\pgfqpoint{2.226242in}{1.137929in}}%
\pgfpathlineto{\pgfqpoint{2.226456in}{1.126857in}}%
\pgfpathlineto{\pgfqpoint{2.226884in}{1.094532in}}%
\pgfpathlineto{\pgfqpoint{2.227865in}{1.096732in}}%
\pgfpathlineto{\pgfqpoint{2.228151in}{1.117993in}}%
\pgfpathlineto{\pgfqpoint{2.228650in}{1.148804in}}%
\pgfpathlineto{\pgfqpoint{2.229400in}{1.134412in}}%
\pgfpathlineto{\pgfqpoint{2.229614in}{1.137935in}}%
\pgfpathlineto{\pgfqpoint{2.229846in}{1.125924in}}%
\pgfpathlineto{\pgfqpoint{2.230274in}{1.094280in}}%
\pgfpathlineto{\pgfqpoint{2.231237in}{1.097176in}}%
\pgfpathlineto{\pgfqpoint{2.231505in}{1.116239in}}%
\pgfpathlineto{\pgfqpoint{2.232022in}{1.148966in}}%
\pgfpathlineto{\pgfqpoint{2.232772in}{1.134859in}}%
\pgfpathlineto{\pgfqpoint{2.232986in}{1.138458in}}%
\pgfpathlineto{\pgfqpoint{2.233218in}{1.126596in}}%
\pgfpathlineto{\pgfqpoint{2.233646in}{1.094091in}}%
\pgfpathlineto{\pgfqpoint{2.234609in}{1.098066in}}%
\pgfpathlineto{\pgfqpoint{2.234859in}{1.114314in}}%
\pgfpathlineto{\pgfqpoint{2.235394in}{1.147880in}}%
\pgfpathlineto{\pgfqpoint{2.236144in}{1.134566in}}%
\pgfpathlineto{\pgfqpoint{2.236358in}{1.138289in}}%
\pgfpathlineto{\pgfqpoint{2.236590in}{1.126859in}}%
\pgfpathlineto{\pgfqpoint{2.237018in}{1.093918in}}%
\pgfpathlineto{\pgfqpoint{2.237981in}{1.098284in}}%
\pgfpathlineto{\pgfqpoint{2.238195in}{1.110048in}}%
\pgfpathlineto{\pgfqpoint{2.238766in}{1.147098in}}%
\pgfpathlineto{\pgfqpoint{2.239569in}{1.135993in}}%
\pgfpathlineto{\pgfqpoint{2.239730in}{1.138615in}}%
\pgfpathlineto{\pgfqpoint{2.239997in}{1.123620in}}%
\pgfpathlineto{\pgfqpoint{2.240408in}{1.093488in}}%
\pgfpathlineto{\pgfqpoint{2.241335in}{1.099039in}}%
\pgfpathlineto{\pgfqpoint{2.241353in}{1.099020in}}%
\pgfpathlineto{\pgfqpoint{2.241424in}{1.100479in}}%
\pgfpathlineto{\pgfqpoint{2.242138in}{1.147448in}}%
\pgfpathlineto{\pgfqpoint{2.243280in}{1.131584in}}%
\pgfpathlineto{\pgfqpoint{2.243762in}{1.093264in}}%
\pgfpathlineto{\pgfqpoint{2.244761in}{1.099639in}}%
\pgfpathlineto{\pgfqpoint{2.245135in}{1.129978in}}%
\pgfpathlineto{\pgfqpoint{2.245528in}{1.146973in}}%
\pgfpathlineto{\pgfqpoint{2.246295in}{1.135647in}}%
\pgfpathlineto{\pgfqpoint{2.246473in}{1.138086in}}%
\pgfpathlineto{\pgfqpoint{2.246705in}{1.127045in}}%
\pgfpathlineto{\pgfqpoint{2.247151in}{1.093510in}}%
\pgfpathlineto{\pgfqpoint{2.248097in}{1.099176in}}%
\pgfpathlineto{\pgfqpoint{2.248168in}{1.100097in}}%
\pgfpathlineto{\pgfqpoint{2.248900in}{1.146889in}}%
\pgfpathlineto{\pgfqpoint{2.250042in}{1.129700in}}%
\pgfpathlineto{\pgfqpoint{2.250523in}{1.093637in}}%
\pgfpathlineto{\pgfqpoint{2.251504in}{1.099217in}}%
\pgfpathlineto{\pgfqpoint{2.251790in}{1.119378in}}%
\pgfpathlineto{\pgfqpoint{2.252289in}{1.147265in}}%
\pgfpathlineto{\pgfqpoint{2.253039in}{1.134517in}}%
\pgfpathlineto{\pgfqpoint{2.253217in}{1.136552in}}%
\pgfpathlineto{\pgfqpoint{2.253431in}{1.128042in}}%
\pgfpathlineto{\pgfqpoint{2.253895in}{1.093950in}}%
\pgfpathlineto{\pgfqpoint{2.254876in}{1.098798in}}%
\pgfpathlineto{\pgfqpoint{2.255144in}{1.117077in}}%
\pgfpathlineto{\pgfqpoint{2.255661in}{1.147752in}}%
\pgfpathlineto{\pgfqpoint{2.256411in}{1.134577in}}%
\pgfpathlineto{\pgfqpoint{2.256607in}{1.136691in}}%
\pgfpathlineto{\pgfqpoint{2.256821in}{1.127091in}}%
\pgfpathlineto{\pgfqpoint{2.257267in}{1.094851in}}%
\pgfpathlineto{\pgfqpoint{2.258248in}{1.098889in}}%
\pgfpathlineto{\pgfqpoint{2.258498in}{1.115087in}}%
\pgfpathlineto{\pgfqpoint{2.259015in}{1.148187in}}%
\pgfpathlineto{\pgfqpoint{2.259765in}{1.133296in}}%
\pgfpathlineto{\pgfqpoint{2.259979in}{1.135688in}}%
\pgfpathlineto{\pgfqpoint{2.260157in}{1.129590in}}%
\pgfpathlineto{\pgfqpoint{2.260639in}{1.095005in}}%
\pgfpathlineto{\pgfqpoint{2.261656in}{1.099757in}}%
\pgfpathlineto{\pgfqpoint{2.262369in}{1.148184in}}%
\pgfpathlineto{\pgfqpoint{2.263583in}{1.126103in}}%
\pgfpathlineto{\pgfqpoint{2.264029in}{1.095631in}}%
\pgfpathlineto{\pgfqpoint{2.265010in}{1.099291in}}%
\pgfpathlineto{\pgfqpoint{2.265367in}{1.128703in}}%
\pgfpathlineto{\pgfqpoint{2.265759in}{1.148167in}}%
\pgfpathlineto{\pgfqpoint{2.266509in}{1.133110in}}%
\pgfpathlineto{\pgfqpoint{2.266740in}{1.136065in}}%
\pgfpathlineto{\pgfqpoint{2.266937in}{1.128516in}}%
\pgfpathlineto{\pgfqpoint{2.267401in}{1.095436in}}%
\pgfpathlineto{\pgfqpoint{2.268400in}{1.099890in}}%
\pgfpathlineto{\pgfqpoint{2.269113in}{1.147893in}}%
\pgfpathlineto{\pgfqpoint{2.270344in}{1.125633in}}%
\pgfpathlineto{\pgfqpoint{2.270790in}{1.095322in}}%
\pgfpathlineto{\pgfqpoint{2.271736in}{1.100363in}}%
\pgfpathlineto{\pgfqpoint{2.271968in}{1.113907in}}%
\pgfpathlineto{\pgfqpoint{2.272485in}{1.147402in}}%
\pgfpathlineto{\pgfqpoint{2.273252in}{1.132735in}}%
\pgfpathlineto{\pgfqpoint{2.273502in}{1.136143in}}%
\pgfpathlineto{\pgfqpoint{2.273681in}{1.129293in}}%
\pgfpathlineto{\pgfqpoint{2.274162in}{1.095303in}}%
\pgfpathlineto{\pgfqpoint{2.275143in}{1.101116in}}%
\pgfpathlineto{\pgfqpoint{2.275625in}{1.141096in}}%
\pgfpathlineto{\pgfqpoint{2.275857in}{1.147231in}}%
\pgfpathlineto{\pgfqpoint{2.276571in}{1.132953in}}%
\pgfpathlineto{\pgfqpoint{2.276589in}{1.133004in}}%
\pgfpathlineto{\pgfqpoint{2.276856in}{1.136327in}}%
\pgfpathlineto{\pgfqpoint{2.277035in}{1.130724in}}%
\pgfpathlineto{\pgfqpoint{2.277534in}{1.094763in}}%
\pgfpathlineto{\pgfqpoint{2.278533in}{1.101913in}}%
\pgfpathlineto{\pgfqpoint{2.279229in}{1.146576in}}%
\pgfpathlineto{\pgfqpoint{2.280442in}{1.128385in}}%
\pgfpathlineto{\pgfqpoint{2.280906in}{1.095090in}}%
\pgfpathlineto{\pgfqpoint{2.281869in}{1.101849in}}%
\pgfpathlineto{\pgfqpoint{2.282119in}{1.116770in}}%
\pgfpathlineto{\pgfqpoint{2.282619in}{1.146622in}}%
\pgfpathlineto{\pgfqpoint{2.283386in}{1.133584in}}%
\pgfpathlineto{\pgfqpoint{2.283600in}{1.136059in}}%
\pgfpathlineto{\pgfqpoint{2.283796in}{1.129423in}}%
\pgfpathlineto{\pgfqpoint{2.284278in}{1.094572in}}%
\pgfpathlineto{\pgfqpoint{2.285259in}{1.102079in}}%
\pgfpathlineto{\pgfqpoint{2.285580in}{1.125572in}}%
\pgfpathlineto{\pgfqpoint{2.285991in}{1.146453in}}%
\pgfpathlineto{\pgfqpoint{2.286758in}{1.133389in}}%
\pgfpathlineto{\pgfqpoint{2.286990in}{1.135395in}}%
\pgfpathlineto{\pgfqpoint{2.287150in}{1.130304in}}%
\pgfpathlineto{\pgfqpoint{2.287650in}{1.094847in}}%
\pgfpathlineto{\pgfqpoint{2.288649in}{1.102250in}}%
\pgfpathlineto{\pgfqpoint{2.289041in}{1.133483in}}%
\pgfpathlineto{\pgfqpoint{2.289363in}{1.146218in}}%
\pgfpathlineto{\pgfqpoint{2.290094in}{1.133181in}}%
\pgfpathlineto{\pgfqpoint{2.290148in}{1.133341in}}%
\pgfpathlineto{\pgfqpoint{2.290344in}{1.135192in}}%
\pgfpathlineto{\pgfqpoint{2.290558in}{1.127962in}}%
\pgfpathlineto{\pgfqpoint{2.291022in}{1.095309in}}%
\pgfpathlineto{\pgfqpoint{2.292003in}{1.101579in}}%
\pgfpathlineto{\pgfqpoint{2.292235in}{1.114949in}}%
\pgfpathlineto{\pgfqpoint{2.292734in}{1.146364in}}%
\pgfpathlineto{\pgfqpoint{2.293519in}{1.132982in}}%
\pgfpathlineto{\pgfqpoint{2.293734in}{1.134619in}}%
\pgfpathlineto{\pgfqpoint{2.293912in}{1.128882in}}%
\pgfpathlineto{\pgfqpoint{2.294411in}{1.095143in}}%
\pgfpathlineto{\pgfqpoint{2.295411in}{1.102266in}}%
\pgfpathlineto{\pgfqpoint{2.296106in}{1.147105in}}%
\pgfpathlineto{\pgfqpoint{2.297337in}{1.124942in}}%
\pgfpathlineto{\pgfqpoint{2.297783in}{1.095815in}}%
\pgfpathlineto{\pgfqpoint{2.298729in}{1.101679in}}%
\pgfpathlineto{\pgfqpoint{2.298836in}{1.104054in}}%
\pgfpathlineto{\pgfqpoint{2.299478in}{1.147039in}}%
\pgfpathlineto{\pgfqpoint{2.300513in}{1.133627in}}%
\pgfpathlineto{\pgfqpoint{2.300763in}{1.120929in}}%
\pgfpathlineto{\pgfqpoint{2.301155in}{1.096072in}}%
\pgfpathlineto{\pgfqpoint{2.302101in}{1.101459in}}%
\pgfpathlineto{\pgfqpoint{2.302190in}{1.103006in}}%
\pgfpathlineto{\pgfqpoint{2.302850in}{1.147202in}}%
\pgfpathlineto{\pgfqpoint{2.303974in}{1.131805in}}%
\pgfpathlineto{\pgfqpoint{2.304527in}{1.096576in}}%
\pgfpathlineto{\pgfqpoint{2.305651in}{1.108607in}}%
\pgfpathlineto{\pgfqpoint{2.306204in}{1.147318in}}%
\pgfpathlineto{\pgfqpoint{2.307043in}{1.131594in}}%
\pgfpathlineto{\pgfqpoint{2.307239in}{1.133514in}}%
\pgfpathlineto{\pgfqpoint{2.307453in}{1.126155in}}%
\pgfpathlineto{\pgfqpoint{2.307917in}{1.096419in}}%
\pgfpathlineto{\pgfqpoint{2.308880in}{1.102296in}}%
\pgfpathlineto{\pgfqpoint{2.309166in}{1.121966in}}%
\pgfpathlineto{\pgfqpoint{2.309576in}{1.147367in}}%
\pgfpathlineto{\pgfqpoint{2.310361in}{1.131378in}}%
\pgfpathlineto{\pgfqpoint{2.310629in}{1.133871in}}%
\pgfpathlineto{\pgfqpoint{2.310789in}{1.129125in}}%
\pgfpathlineto{\pgfqpoint{2.311289in}{1.096071in}}%
\pgfpathlineto{\pgfqpoint{2.312288in}{1.103408in}}%
\pgfpathlineto{\pgfqpoint{2.312948in}{1.147020in}}%
\pgfpathlineto{\pgfqpoint{2.314161in}{1.129412in}}%
\pgfpathlineto{\pgfqpoint{2.314679in}{1.096414in}}%
\pgfpathlineto{\pgfqpoint{2.315678in}{1.104652in}}%
\pgfpathlineto{\pgfqpoint{2.316302in}{1.146415in}}%
\pgfpathlineto{\pgfqpoint{2.317497in}{1.131566in}}%
\pgfpathlineto{\pgfqpoint{2.318050in}{1.095983in}}%
\pgfpathlineto{\pgfqpoint{2.319103in}{1.106700in}}%
\pgfpathlineto{\pgfqpoint{2.319692in}{1.146407in}}%
\pgfpathlineto{\pgfqpoint{2.320673in}{1.134075in}}%
\pgfpathlineto{\pgfqpoint{2.320744in}{1.134415in}}%
\pgfpathlineto{\pgfqpoint{2.320941in}{1.127584in}}%
\pgfpathlineto{\pgfqpoint{2.321422in}{1.095740in}}%
\pgfpathlineto{\pgfqpoint{2.322368in}{1.104181in}}%
\pgfpathlineto{\pgfqpoint{2.322564in}{1.113088in}}%
\pgfpathlineto{\pgfqpoint{2.323064in}{1.145916in}}%
\pgfpathlineto{\pgfqpoint{2.323884in}{1.132018in}}%
\pgfpathlineto{\pgfqpoint{2.324116in}{1.133992in}}%
\pgfpathlineto{\pgfqpoint{2.324313in}{1.127433in}}%
\pgfpathlineto{\pgfqpoint{2.324794in}{1.095912in}}%
\pgfpathlineto{\pgfqpoint{2.325740in}{1.104481in}}%
\pgfpathlineto{\pgfqpoint{2.325865in}{1.107867in}}%
\pgfpathlineto{\pgfqpoint{2.326453in}{1.145771in}}%
\pgfpathlineto{\pgfqpoint{2.327417in}{1.133471in}}%
\pgfpathlineto{\pgfqpoint{2.327488in}{1.133817in}}%
\pgfpathlineto{\pgfqpoint{2.327684in}{1.127370in}}%
\pgfpathlineto{\pgfqpoint{2.328166in}{1.096074in}}%
\pgfpathlineto{\pgfqpoint{2.329112in}{1.104580in}}%
\pgfpathlineto{\pgfqpoint{2.329237in}{1.107679in}}%
\pgfpathlineto{\pgfqpoint{2.329825in}{1.145935in}}%
\pgfpathlineto{\pgfqpoint{2.330789in}{1.133024in}}%
\pgfpathlineto{\pgfqpoint{2.330842in}{1.133300in}}%
\pgfpathlineto{\pgfqpoint{2.331056in}{1.127066in}}%
\pgfpathlineto{\pgfqpoint{2.331538in}{1.096158in}}%
\pgfpathlineto{\pgfqpoint{2.332501in}{1.104219in}}%
\pgfpathlineto{\pgfqpoint{2.332662in}{1.110264in}}%
\pgfpathlineto{\pgfqpoint{2.333215in}{1.145799in}}%
\pgfpathlineto{\pgfqpoint{2.334054in}{1.131800in}}%
\pgfpathlineto{\pgfqpoint{2.334232in}{1.132937in}}%
\pgfpathlineto{\pgfqpoint{2.334410in}{1.128183in}}%
\pgfpathlineto{\pgfqpoint{2.334910in}{1.096676in}}%
\pgfpathlineto{\pgfqpoint{2.335909in}{1.104278in}}%
\pgfpathlineto{\pgfqpoint{2.336337in}{1.137773in}}%
\pgfpathlineto{\pgfqpoint{2.336587in}{1.146178in}}%
\pgfpathlineto{\pgfqpoint{2.337354in}{1.131161in}}%
\pgfpathlineto{\pgfqpoint{2.337586in}{1.132314in}}%
\pgfpathlineto{\pgfqpoint{2.337747in}{1.129651in}}%
\pgfpathlineto{\pgfqpoint{2.338300in}{1.096904in}}%
\pgfpathlineto{\pgfqpoint{2.339334in}{1.106040in}}%
\pgfpathlineto{\pgfqpoint{2.339959in}{1.146691in}}%
\pgfpathlineto{\pgfqpoint{2.341029in}{1.131636in}}%
\pgfpathlineto{\pgfqpoint{2.341350in}{1.113773in}}%
\pgfpathlineto{\pgfqpoint{2.341672in}{1.097420in}}%
\pgfpathlineto{\pgfqpoint{2.342207in}{1.114452in}}%
\pgfpathlineto{\pgfqpoint{2.342564in}{1.104533in}}%
\pgfpathlineto{\pgfqpoint{2.342617in}{1.104071in}}%
\pgfpathlineto{\pgfqpoint{2.342849in}{1.115598in}}%
\pgfpathlineto{\pgfqpoint{2.343313in}{1.146572in}}%
\pgfpathlineto{\pgfqpoint{2.344116in}{1.130438in}}%
\pgfpathlineto{\pgfqpoint{2.344366in}{1.132082in}}%
\pgfpathlineto{\pgfqpoint{2.344526in}{1.128554in}}%
\pgfpathlineto{\pgfqpoint{2.345044in}{1.097855in}}%
\pgfpathlineto{\pgfqpoint{2.346078in}{1.106102in}}%
\pgfpathlineto{\pgfqpoint{2.346685in}{1.146834in}}%
\pgfpathlineto{\pgfqpoint{2.347773in}{1.131744in}}%
\pgfpathlineto{\pgfqpoint{2.348023in}{1.120576in}}%
\pgfpathlineto{\pgfqpoint{2.348433in}{1.097461in}}%
\pgfpathlineto{\pgfqpoint{2.349325in}{1.104879in}}%
\pgfpathlineto{\pgfqpoint{2.349361in}{1.104693in}}%
\pgfpathlineto{\pgfqpoint{2.349539in}{1.111571in}}%
\pgfpathlineto{\pgfqpoint{2.350057in}{1.146663in}}%
\pgfpathlineto{\pgfqpoint{2.350877in}{1.129943in}}%
\pgfpathlineto{\pgfqpoint{2.351127in}{1.131905in}}%
\pgfpathlineto{\pgfqpoint{2.351306in}{1.127177in}}%
\pgfpathlineto{\pgfqpoint{2.351805in}{1.097493in}}%
\pgfpathlineto{\pgfqpoint{2.352769in}{1.105617in}}%
\pgfpathlineto{\pgfqpoint{2.353125in}{1.131627in}}%
\pgfpathlineto{\pgfqpoint{2.353429in}{1.146613in}}%
\pgfpathlineto{\pgfqpoint{2.354160in}{1.130209in}}%
\pgfpathlineto{\pgfqpoint{2.354214in}{1.130281in}}%
\pgfpathlineto{\pgfqpoint{2.354499in}{1.132628in}}%
\pgfpathlineto{\pgfqpoint{2.354678in}{1.127890in}}%
\pgfpathlineto{\pgfqpoint{2.355177in}{1.097348in}}%
\pgfpathlineto{\pgfqpoint{2.356140in}{1.106011in}}%
\pgfpathlineto{\pgfqpoint{2.356462in}{1.127821in}}%
\pgfpathlineto{\pgfqpoint{2.356783in}{1.145946in}}%
\pgfpathlineto{\pgfqpoint{2.357586in}{1.130114in}}%
\pgfpathlineto{\pgfqpoint{2.357818in}{1.132010in}}%
\pgfpathlineto{\pgfqpoint{2.357871in}{1.132225in}}%
\pgfpathlineto{\pgfqpoint{2.358067in}{1.126634in}}%
\pgfpathlineto{\pgfqpoint{2.358549in}{1.097071in}}%
\pgfpathlineto{\pgfqpoint{2.359495in}{1.106420in}}%
\pgfpathlineto{\pgfqpoint{2.359673in}{1.113228in}}%
\pgfpathlineto{\pgfqpoint{2.360172in}{1.145460in}}%
\pgfpathlineto{\pgfqpoint{2.361011in}{1.130695in}}%
\pgfpathlineto{\pgfqpoint{2.361243in}{1.132454in}}%
\pgfpathlineto{\pgfqpoint{2.361457in}{1.125477in}}%
\pgfpathlineto{\pgfqpoint{2.361939in}{1.096597in}}%
\pgfpathlineto{\pgfqpoint{2.362849in}{1.106431in}}%
\pgfpathlineto{\pgfqpoint{2.362866in}{1.106380in}}%
\pgfpathlineto{\pgfqpoint{2.362973in}{1.108740in}}%
\pgfpathlineto{\pgfqpoint{2.363544in}{1.145632in}}%
\pgfpathlineto{\pgfqpoint{2.364561in}{1.132239in}}%
\pgfpathlineto{\pgfqpoint{2.364597in}{1.132282in}}%
\pgfpathlineto{\pgfqpoint{2.364722in}{1.130594in}}%
\pgfpathlineto{\pgfqpoint{2.365311in}{1.096825in}}%
\pgfpathlineto{\pgfqpoint{2.366381in}{1.110579in}}%
\pgfpathlineto{\pgfqpoint{2.366916in}{1.145405in}}%
\pgfpathlineto{\pgfqpoint{2.367844in}{1.131232in}}%
\pgfpathlineto{\pgfqpoint{2.367969in}{1.131866in}}%
\pgfpathlineto{\pgfqpoint{2.368183in}{1.126177in}}%
\pgfpathlineto{\pgfqpoint{2.368683in}{1.097038in}}%
\pgfpathlineto{\pgfqpoint{2.369610in}{1.106635in}}%
\pgfpathlineto{\pgfqpoint{2.369682in}{1.107304in}}%
\pgfpathlineto{\pgfqpoint{2.370306in}{1.145314in}}%
\pgfpathlineto{\pgfqpoint{2.371537in}{1.126906in}}%
\pgfpathlineto{\pgfqpoint{2.372054in}{1.097239in}}%
\pgfpathlineto{\pgfqpoint{2.373036in}{1.106633in}}%
\pgfpathlineto{\pgfqpoint{2.373428in}{1.135153in}}%
\pgfpathlineto{\pgfqpoint{2.373678in}{1.145752in}}%
\pgfpathlineto{\pgfqpoint{2.374481in}{1.130400in}}%
\pgfpathlineto{\pgfqpoint{2.374695in}{1.131031in}}%
\pgfpathlineto{\pgfqpoint{2.374891in}{1.127672in}}%
\pgfpathlineto{\pgfqpoint{2.375426in}{1.097409in}}%
\pgfpathlineto{\pgfqpoint{2.376443in}{1.107261in}}%
\pgfpathlineto{\pgfqpoint{2.377068in}{1.145823in}}%
\pgfpathlineto{\pgfqpoint{2.378227in}{1.129137in}}%
\pgfpathlineto{\pgfqpoint{2.378798in}{1.098146in}}%
\pgfpathlineto{\pgfqpoint{2.379887in}{1.110251in}}%
\pgfpathlineto{\pgfqpoint{2.380440in}{1.146020in}}%
\pgfpathlineto{\pgfqpoint{2.381296in}{1.129553in}}%
\pgfpathlineto{\pgfqpoint{2.381474in}{1.130397in}}%
\pgfpathlineto{\pgfqpoint{2.381635in}{1.127759in}}%
\pgfpathlineto{\pgfqpoint{2.382188in}{1.098306in}}%
\pgfpathlineto{\pgfqpoint{2.383205in}{1.107977in}}%
\pgfpathlineto{\pgfqpoint{2.383811in}{1.146461in}}%
\pgfpathlineto{\pgfqpoint{2.384882in}{1.130451in}}%
\pgfpathlineto{\pgfqpoint{2.385132in}{1.121328in}}%
\pgfpathlineto{\pgfqpoint{2.385560in}{1.098587in}}%
\pgfpathlineto{\pgfqpoint{2.386452in}{1.106874in}}%
\pgfpathlineto{\pgfqpoint{2.386488in}{1.106724in}}%
\pgfpathlineto{\pgfqpoint{2.386648in}{1.111450in}}%
\pgfpathlineto{\pgfqpoint{2.387183in}{1.146138in}}%
\pgfpathlineto{\pgfqpoint{2.388022in}{1.129465in}}%
\pgfpathlineto{\pgfqpoint{2.388236in}{1.130732in}}%
\pgfpathlineto{\pgfqpoint{2.388414in}{1.127202in}}%
\pgfpathlineto{\pgfqpoint{2.388932in}{1.098574in}}%
\pgfpathlineto{\pgfqpoint{2.389913in}{1.107431in}}%
\pgfpathlineto{\pgfqpoint{2.390359in}{1.140442in}}%
\pgfpathlineto{\pgfqpoint{2.390537in}{1.146214in}}%
\pgfpathlineto{\pgfqpoint{2.391287in}{1.129122in}}%
\pgfpathlineto{\pgfqpoint{2.391537in}{1.130378in}}%
\pgfpathlineto{\pgfqpoint{2.391608in}{1.130727in}}%
\pgfpathlineto{\pgfqpoint{2.391840in}{1.124390in}}%
\pgfpathlineto{\pgfqpoint{2.392322in}{1.098335in}}%
\pgfpathlineto{\pgfqpoint{2.393214in}{1.107950in}}%
\pgfpathlineto{\pgfqpoint{2.393231in}{1.107916in}}%
\pgfpathlineto{\pgfqpoint{2.393338in}{1.109783in}}%
\pgfpathlineto{\pgfqpoint{2.393909in}{1.145902in}}%
\pgfpathlineto{\pgfqpoint{2.394962in}{1.130662in}}%
\pgfpathlineto{\pgfqpoint{2.394998in}{1.130741in}}%
\pgfpathlineto{\pgfqpoint{2.395140in}{1.128151in}}%
\pgfpathlineto{\pgfqpoint{2.395693in}{1.098182in}}%
\pgfpathlineto{\pgfqpoint{2.396693in}{1.109011in}}%
\pgfpathlineto{\pgfqpoint{2.397299in}{1.145495in}}%
\pgfpathlineto{\pgfqpoint{2.398459in}{1.130188in}}%
\pgfpathlineto{\pgfqpoint{2.399065in}{1.097957in}}%
\pgfpathlineto{\pgfqpoint{2.400171in}{1.114092in}}%
\pgfpathlineto{\pgfqpoint{2.400671in}{1.145130in}}%
\pgfpathlineto{\pgfqpoint{2.401510in}{1.129873in}}%
\pgfpathlineto{\pgfqpoint{2.401724in}{1.130852in}}%
\pgfpathlineto{\pgfqpoint{2.401902in}{1.127533in}}%
\pgfpathlineto{\pgfqpoint{2.402437in}{1.097952in}}%
\pgfpathlineto{\pgfqpoint{2.403401in}{1.108906in}}%
\pgfpathlineto{\pgfqpoint{2.403686in}{1.125317in}}%
\pgfpathlineto{\pgfqpoint{2.404043in}{1.144888in}}%
\pgfpathlineto{\pgfqpoint{2.404846in}{1.129698in}}%
\pgfpathlineto{\pgfqpoint{2.405096in}{1.130668in}}%
\pgfpathlineto{\pgfqpoint{2.405274in}{1.127231in}}%
\pgfpathlineto{\pgfqpoint{2.405809in}{1.097833in}}%
\pgfpathlineto{\pgfqpoint{2.406773in}{1.108816in}}%
\pgfpathlineto{\pgfqpoint{2.407022in}{1.121643in}}%
\pgfpathlineto{\pgfqpoint{2.407415in}{1.144973in}}%
\pgfpathlineto{\pgfqpoint{2.408218in}{1.129916in}}%
\pgfpathlineto{\pgfqpoint{2.408592in}{1.128950in}}%
\pgfpathlineto{\pgfqpoint{2.409181in}{1.098153in}}%
\pgfpathlineto{\pgfqpoint{2.410251in}{1.111572in}}%
\pgfpathlineto{\pgfqpoint{2.410805in}{1.144867in}}%
\pgfpathlineto{\pgfqpoint{2.411732in}{1.130047in}}%
\pgfpathlineto{\pgfqpoint{2.411821in}{1.130218in}}%
\pgfpathlineto{\pgfqpoint{2.411964in}{1.128771in}}%
\pgfpathlineto{\pgfqpoint{2.412553in}{1.098646in}}%
\pgfpathlineto{\pgfqpoint{2.413659in}{1.112775in}}%
\pgfpathlineto{\pgfqpoint{2.414176in}{1.144752in}}%
\pgfpathlineto{\pgfqpoint{2.415051in}{1.129451in}}%
\pgfpathlineto{\pgfqpoint{2.415193in}{1.129750in}}%
\pgfpathlineto{\pgfqpoint{2.415372in}{1.127316in}}%
\pgfpathlineto{\pgfqpoint{2.415943in}{1.098621in}}%
\pgfpathlineto{\pgfqpoint{2.416942in}{1.109346in}}%
\pgfpathlineto{\pgfqpoint{2.417548in}{1.145541in}}%
\pgfpathlineto{\pgfqpoint{2.418708in}{1.127968in}}%
\pgfpathlineto{\pgfqpoint{2.419315in}{1.098978in}}%
\pgfpathlineto{\pgfqpoint{2.420421in}{1.113835in}}%
\pgfpathlineto{\pgfqpoint{2.420920in}{1.145480in}}%
\pgfpathlineto{\pgfqpoint{2.421759in}{1.128789in}}%
\pgfpathlineto{\pgfqpoint{2.422098in}{1.127838in}}%
\pgfpathlineto{\pgfqpoint{2.422686in}{1.099555in}}%
\pgfpathlineto{\pgfqpoint{2.423775in}{1.112548in}}%
\pgfpathlineto{\pgfqpoint{2.424292in}{1.145443in}}%
\pgfpathlineto{\pgfqpoint{2.425149in}{1.128561in}}%
\pgfpathlineto{\pgfqpoint{2.425345in}{1.129201in}}%
\pgfpathlineto{\pgfqpoint{2.425505in}{1.126979in}}%
\pgfpathlineto{\pgfqpoint{2.426058in}{1.099501in}}%
\pgfpathlineto{\pgfqpoint{2.427075in}{1.110070in}}%
\pgfpathlineto{\pgfqpoint{2.427664in}{1.145903in}}%
\pgfpathlineto{\pgfqpoint{2.428752in}{1.129096in}}%
\pgfpathlineto{\pgfqpoint{2.429038in}{1.119083in}}%
\pgfpathlineto{\pgfqpoint{2.429448in}{1.099268in}}%
\pgfpathlineto{\pgfqpoint{2.430287in}{1.109888in}}%
\pgfpathlineto{\pgfqpoint{2.430358in}{1.109075in}}%
\pgfpathlineto{\pgfqpoint{2.430643in}{1.121710in}}%
\pgfpathlineto{\pgfqpoint{2.431036in}{1.145532in}}%
\pgfpathlineto{\pgfqpoint{2.431821in}{1.128341in}}%
\pgfpathlineto{\pgfqpoint{2.432106in}{1.129501in}}%
\pgfpathlineto{\pgfqpoint{2.432267in}{1.127007in}}%
\pgfpathlineto{\pgfqpoint{2.432820in}{1.099383in}}%
\pgfpathlineto{\pgfqpoint{2.433819in}{1.110430in}}%
\pgfpathlineto{\pgfqpoint{2.434408in}{1.145271in}}%
\pgfpathlineto{\pgfqpoint{2.435568in}{1.128693in}}%
\pgfpathlineto{\pgfqpoint{2.436192in}{1.099102in}}%
\pgfpathlineto{\pgfqpoint{2.437316in}{1.116663in}}%
\pgfpathlineto{\pgfqpoint{2.437780in}{1.144923in}}%
\pgfpathlineto{\pgfqpoint{2.438600in}{1.128475in}}%
\pgfpathlineto{\pgfqpoint{2.438850in}{1.129683in}}%
\pgfpathlineto{\pgfqpoint{2.439046in}{1.125811in}}%
\pgfpathlineto{\pgfqpoint{2.439582in}{1.098962in}}%
\pgfpathlineto{\pgfqpoint{2.440492in}{1.110290in}}%
\pgfpathlineto{\pgfqpoint{2.440509in}{1.110283in}}%
\pgfpathlineto{\pgfqpoint{2.440581in}{1.111273in}}%
\pgfpathlineto{\pgfqpoint{2.441170in}{1.144641in}}%
\pgfpathlineto{\pgfqpoint{2.442276in}{1.129715in}}%
\pgfpathlineto{\pgfqpoint{2.442632in}{1.112736in}}%
\pgfpathlineto{\pgfqpoint{2.442954in}{1.098857in}}%
\pgfpathlineto{\pgfqpoint{2.443524in}{1.115781in}}%
\pgfpathlineto{\pgfqpoint{2.443739in}{1.112549in}}%
\pgfpathlineto{\pgfqpoint{2.443863in}{1.110809in}}%
\pgfpathlineto{\pgfqpoint{2.444185in}{1.125597in}}%
\pgfpathlineto{\pgfqpoint{2.444541in}{1.144658in}}%
\pgfpathlineto{\pgfqpoint{2.445326in}{1.128982in}}%
\pgfpathlineto{\pgfqpoint{2.445576in}{1.129574in}}%
\pgfpathlineto{\pgfqpoint{2.445719in}{1.127880in}}%
\pgfpathlineto{\pgfqpoint{2.446325in}{1.098898in}}%
\pgfpathlineto{\pgfqpoint{2.447360in}{1.112318in}}%
\pgfpathlineto{\pgfqpoint{2.447913in}{1.144102in}}%
\pgfpathlineto{\pgfqpoint{2.448912in}{1.129344in}}%
\pgfpathlineto{\pgfqpoint{2.448966in}{1.129250in}}%
\pgfpathlineto{\pgfqpoint{2.449055in}{1.128387in}}%
\pgfpathlineto{\pgfqpoint{2.449697in}{1.099240in}}%
\pgfpathlineto{\pgfqpoint{2.450821in}{1.116114in}}%
\pgfpathlineto{\pgfqpoint{2.451303in}{1.144138in}}%
\pgfpathlineto{\pgfqpoint{2.452124in}{1.129090in}}%
\pgfpathlineto{\pgfqpoint{2.452266in}{1.129207in}}%
\pgfpathlineto{\pgfqpoint{2.452445in}{1.127845in}}%
\pgfpathlineto{\pgfqpoint{2.453069in}{1.099347in}}%
\pgfpathlineto{\pgfqpoint{2.454175in}{1.114909in}}%
\pgfpathlineto{\pgfqpoint{2.454675in}{1.144461in}}%
\pgfpathlineto{\pgfqpoint{2.455513in}{1.128760in}}%
\pgfpathlineto{\pgfqpoint{2.455638in}{1.128790in}}%
\pgfpathlineto{\pgfqpoint{2.455817in}{1.127575in}}%
\pgfpathlineto{\pgfqpoint{2.456441in}{1.099976in}}%
\pgfpathlineto{\pgfqpoint{2.457547in}{1.114719in}}%
\pgfpathlineto{\pgfqpoint{2.458047in}{1.144621in}}%
\pgfpathlineto{\pgfqpoint{2.458885in}{1.128324in}}%
\pgfpathlineto{\pgfqpoint{2.459028in}{1.128415in}}%
\pgfpathlineto{\pgfqpoint{2.459189in}{1.127335in}}%
\pgfpathlineto{\pgfqpoint{2.459813in}{1.100132in}}%
\pgfpathlineto{\pgfqpoint{2.460919in}{1.114942in}}%
\pgfpathlineto{\pgfqpoint{2.461419in}{1.145126in}}%
\pgfpathlineto{\pgfqpoint{2.462239in}{1.127901in}}%
\pgfpathlineto{\pgfqpoint{2.462436in}{1.128109in}}%
\pgfpathlineto{\pgfqpoint{2.462614in}{1.126212in}}%
\pgfpathlineto{\pgfqpoint{2.463185in}{1.100533in}}%
\pgfpathlineto{\pgfqpoint{2.464238in}{1.112480in}}%
\pgfpathlineto{\pgfqpoint{2.464791in}{1.144853in}}%
\pgfpathlineto{\pgfqpoint{2.465736in}{1.128087in}}%
\pgfpathlineto{\pgfqpoint{2.465808in}{1.128180in}}%
\pgfpathlineto{\pgfqpoint{2.465968in}{1.126809in}}%
\pgfpathlineto{\pgfqpoint{2.466575in}{1.100501in}}%
\pgfpathlineto{\pgfqpoint{2.467627in}{1.113168in}}%
\pgfpathlineto{\pgfqpoint{2.468163in}{1.145188in}}%
\pgfpathlineto{\pgfqpoint{2.469090in}{1.128311in}}%
\pgfpathlineto{\pgfqpoint{2.469162in}{1.128489in}}%
\pgfpathlineto{\pgfqpoint{2.469376in}{1.126310in}}%
\pgfpathlineto{\pgfqpoint{2.469947in}{1.100361in}}%
\pgfpathlineto{\pgfqpoint{2.470946in}{1.111997in}}%
\pgfpathlineto{\pgfqpoint{2.471534in}{1.144969in}}%
\pgfpathlineto{\pgfqpoint{2.472694in}{1.127280in}}%
\pgfpathlineto{\pgfqpoint{2.473336in}{1.100144in}}%
\pgfpathlineto{\pgfqpoint{2.474425in}{1.116181in}}%
\pgfpathlineto{\pgfqpoint{2.474906in}{1.144554in}}%
\pgfpathlineto{\pgfqpoint{2.475727in}{1.127995in}}%
\pgfpathlineto{\pgfqpoint{2.475959in}{1.128798in}}%
\pgfpathlineto{\pgfqpoint{2.476173in}{1.125051in}}%
\pgfpathlineto{\pgfqpoint{2.476708in}{1.100137in}}%
\pgfpathlineto{\pgfqpoint{2.477582in}{1.111936in}}%
\pgfpathlineto{\pgfqpoint{2.477618in}{1.111698in}}%
\pgfpathlineto{\pgfqpoint{2.477850in}{1.119271in}}%
\pgfpathlineto{\pgfqpoint{2.478278in}{1.143978in}}%
\pgfpathlineto{\pgfqpoint{2.479081in}{1.127916in}}%
\pgfpathlineto{\pgfqpoint{2.479295in}{1.128707in}}%
\pgfpathlineto{\pgfqpoint{2.479545in}{1.124732in}}%
\pgfpathlineto{\pgfqpoint{2.480080in}{1.099950in}}%
\pgfpathlineto{\pgfqpoint{2.480937in}{1.112932in}}%
\pgfpathlineto{\pgfqpoint{2.481008in}{1.112423in}}%
\pgfpathlineto{\pgfqpoint{2.481258in}{1.121708in}}%
\pgfpathlineto{\pgfqpoint{2.481668in}{1.143946in}}%
\pgfpathlineto{\pgfqpoint{2.482453in}{1.128001in}}%
\pgfpathlineto{\pgfqpoint{2.482667in}{1.128698in}}%
\pgfpathlineto{\pgfqpoint{2.482917in}{1.124529in}}%
\pgfpathlineto{\pgfqpoint{2.483452in}{1.099673in}}%
\pgfpathlineto{\pgfqpoint{2.484308in}{1.112747in}}%
\pgfpathlineto{\pgfqpoint{2.484380in}{1.112147in}}%
\pgfpathlineto{\pgfqpoint{2.484630in}{1.121078in}}%
\pgfpathlineto{\pgfqpoint{2.485040in}{1.143808in}}%
\pgfpathlineto{\pgfqpoint{2.485825in}{1.128358in}}%
\pgfpathlineto{\pgfqpoint{2.486003in}{1.128802in}}%
\pgfpathlineto{\pgfqpoint{2.486253in}{1.125694in}}%
\pgfpathlineto{\pgfqpoint{2.486824in}{1.099936in}}%
\pgfpathlineto{\pgfqpoint{2.487734in}{1.112640in}}%
\pgfpathlineto{\pgfqpoint{2.487770in}{1.112496in}}%
\pgfpathlineto{\pgfqpoint{2.487930in}{1.116547in}}%
\pgfpathlineto{\pgfqpoint{2.488412in}{1.143726in}}%
\pgfpathlineto{\pgfqpoint{2.489250in}{1.128278in}}%
\pgfpathlineto{\pgfqpoint{2.489340in}{1.128489in}}%
\pgfpathlineto{\pgfqpoint{2.489571in}{1.126806in}}%
\pgfpathlineto{\pgfqpoint{2.490196in}{1.100238in}}%
\pgfpathlineto{\pgfqpoint{2.491231in}{1.113410in}}%
\pgfpathlineto{\pgfqpoint{2.491802in}{1.143645in}}%
\pgfpathlineto{\pgfqpoint{2.492836in}{1.128104in}}%
\pgfpathlineto{\pgfqpoint{2.493140in}{1.119223in}}%
\pgfpathlineto{\pgfqpoint{2.493568in}{1.100551in}}%
\pgfpathlineto{\pgfqpoint{2.494317in}{1.115039in}}%
\pgfpathlineto{\pgfqpoint{2.494513in}{1.111958in}}%
\pgfpathlineto{\pgfqpoint{2.494834in}{1.126585in}}%
\pgfpathlineto{\pgfqpoint{2.495173in}{1.143954in}}%
\pgfpathlineto{\pgfqpoint{2.495958in}{1.128002in}}%
\pgfpathlineto{\pgfqpoint{2.496083in}{1.128111in}}%
\pgfpathlineto{\pgfqpoint{2.496297in}{1.126727in}}%
\pgfpathlineto{\pgfqpoint{2.496940in}{1.100576in}}%
\pgfpathlineto{\pgfqpoint{2.498028in}{1.114782in}}%
\pgfpathlineto{\pgfqpoint{2.498545in}{1.143998in}}%
\pgfpathlineto{\pgfqpoint{2.499402in}{1.127537in}}%
\pgfpathlineto{\pgfqpoint{2.499580in}{1.127279in}}%
\pgfpathlineto{\pgfqpoint{2.499669in}{1.126459in}}%
\pgfpathlineto{\pgfqpoint{2.500329in}{1.101128in}}%
\pgfpathlineto{\pgfqpoint{2.501418in}{1.115636in}}%
\pgfpathlineto{\pgfqpoint{2.501917in}{1.143924in}}%
\pgfpathlineto{\pgfqpoint{2.502756in}{1.127336in}}%
\pgfpathlineto{\pgfqpoint{2.502845in}{1.127475in}}%
\pgfpathlineto{\pgfqpoint{2.503095in}{1.125586in}}%
\pgfpathlineto{\pgfqpoint{2.503701in}{1.101246in}}%
\pgfpathlineto{\pgfqpoint{2.504718in}{1.113301in}}%
\pgfpathlineto{\pgfqpoint{2.505289in}{1.144430in}}%
\pgfpathlineto{\pgfqpoint{2.506377in}{1.127173in}}%
\pgfpathlineto{\pgfqpoint{2.506716in}{1.115252in}}%
\pgfpathlineto{\pgfqpoint{2.507073in}{1.101267in}}%
\pgfpathlineto{\pgfqpoint{2.507662in}{1.116773in}}%
\pgfpathlineto{\pgfqpoint{2.507823in}{1.114955in}}%
\pgfpathlineto{\pgfqpoint{2.508001in}{1.112673in}}%
\pgfpathlineto{\pgfqpoint{2.508304in}{1.124954in}}%
\pgfpathlineto{\pgfqpoint{2.508661in}{1.144153in}}%
\pgfpathlineto{\pgfqpoint{2.509428in}{1.127081in}}%
\pgfpathlineto{\pgfqpoint{2.509660in}{1.127700in}}%
\pgfpathlineto{\pgfqpoint{2.509892in}{1.124790in}}%
\pgfpathlineto{\pgfqpoint{2.510463in}{1.101198in}}%
\pgfpathlineto{\pgfqpoint{2.511337in}{1.112958in}}%
\pgfpathlineto{\pgfqpoint{2.511391in}{1.112754in}}%
\pgfpathlineto{\pgfqpoint{2.511587in}{1.118563in}}%
\pgfpathlineto{\pgfqpoint{2.512033in}{1.143958in}}%
\pgfpathlineto{\pgfqpoint{2.512836in}{1.127124in}}%
\pgfpathlineto{\pgfqpoint{2.513014in}{1.127694in}}%
\pgfpathlineto{\pgfqpoint{2.513282in}{1.124169in}}%
\pgfpathlineto{\pgfqpoint{2.513835in}{1.100990in}}%
\pgfpathlineto{\pgfqpoint{2.514673in}{1.113891in}}%
\pgfpathlineto{\pgfqpoint{2.514763in}{1.113244in}}%
\pgfpathlineto{\pgfqpoint{2.515030in}{1.123462in}}%
\pgfpathlineto{\pgfqpoint{2.515405in}{1.143688in}}%
\pgfpathlineto{\pgfqpoint{2.516172in}{1.126824in}}%
\pgfpathlineto{\pgfqpoint{2.516368in}{1.127703in}}%
\pgfpathlineto{\pgfqpoint{2.516689in}{1.122970in}}%
\pgfpathlineto{\pgfqpoint{2.517207in}{1.100897in}}%
\pgfpathlineto{\pgfqpoint{2.518010in}{1.114487in}}%
\pgfpathlineto{\pgfqpoint{2.518135in}{1.113274in}}%
\pgfpathlineto{\pgfqpoint{2.518420in}{1.124085in}}%
\pgfpathlineto{\pgfqpoint{2.518795in}{1.143203in}}%
\pgfpathlineto{\pgfqpoint{2.519562in}{1.127437in}}%
\pgfpathlineto{\pgfqpoint{2.519758in}{1.128281in}}%
\pgfpathlineto{\pgfqpoint{2.520061in}{1.123396in}}%
\pgfpathlineto{\pgfqpoint{2.520579in}{1.100949in}}%
\pgfpathlineto{\pgfqpoint{2.521346in}{1.115528in}}%
\pgfpathlineto{\pgfqpoint{2.521506in}{1.113731in}}%
\pgfpathlineto{\pgfqpoint{2.521810in}{1.125377in}}%
\pgfpathlineto{\pgfqpoint{2.522167in}{1.143355in}}%
\pgfpathlineto{\pgfqpoint{2.522952in}{1.127707in}}%
\pgfpathlineto{\pgfqpoint{2.523112in}{1.128224in}}%
\pgfpathlineto{\pgfqpoint{2.523398in}{1.124489in}}%
\pgfpathlineto{\pgfqpoint{2.523951in}{1.101099in}}%
\pgfpathlineto{\pgfqpoint{2.524753in}{1.115661in}}%
\pgfpathlineto{\pgfqpoint{2.524896in}{1.114194in}}%
\pgfpathlineto{\pgfqpoint{2.525199in}{1.126456in}}%
\pgfpathlineto{\pgfqpoint{2.525538in}{1.143174in}}%
\pgfpathlineto{\pgfqpoint{2.526323in}{1.127542in}}%
\pgfpathlineto{\pgfqpoint{2.526466in}{1.128085in}}%
\pgfpathlineto{\pgfqpoint{2.526787in}{1.123490in}}%
\pgfpathlineto{\pgfqpoint{2.527322in}{1.100951in}}%
\pgfpathlineto{\pgfqpoint{2.528107in}{1.115730in}}%
\pgfpathlineto{\pgfqpoint{2.528268in}{1.113736in}}%
\pgfpathlineto{\pgfqpoint{2.528571in}{1.125452in}}%
\pgfpathlineto{\pgfqpoint{2.528910in}{1.142705in}}%
\pgfpathlineto{\pgfqpoint{2.529695in}{1.127484in}}%
\pgfpathlineto{\pgfqpoint{2.529820in}{1.127914in}}%
\pgfpathlineto{\pgfqpoint{2.530123in}{1.124456in}}%
\pgfpathlineto{\pgfqpoint{2.530694in}{1.101113in}}%
\pgfpathlineto{\pgfqpoint{2.531551in}{1.114569in}}%
\pgfpathlineto{\pgfqpoint{2.531658in}{1.113685in}}%
\pgfpathlineto{\pgfqpoint{2.531925in}{1.123977in}}%
\pgfpathlineto{\pgfqpoint{2.532300in}{1.142993in}}%
\pgfpathlineto{\pgfqpoint{2.533067in}{1.127357in}}%
\pgfpathlineto{\pgfqpoint{2.533228in}{1.127733in}}%
\pgfpathlineto{\pgfqpoint{2.533478in}{1.124880in}}%
\pgfpathlineto{\pgfqpoint{2.534066in}{1.101383in}}%
\pgfpathlineto{\pgfqpoint{2.534958in}{1.114181in}}%
\pgfpathlineto{\pgfqpoint{2.535030in}{1.113731in}}%
\pgfpathlineto{\pgfqpoint{2.535279in}{1.122538in}}%
\pgfpathlineto{\pgfqpoint{2.535672in}{1.143129in}}%
\pgfpathlineto{\pgfqpoint{2.536457in}{1.127424in}}%
\pgfpathlineto{\pgfqpoint{2.536564in}{1.127695in}}%
\pgfpathlineto{\pgfqpoint{2.536849in}{1.124776in}}%
\pgfpathlineto{\pgfqpoint{2.537456in}{1.101672in}}%
\pgfpathlineto{\pgfqpoint{2.538366in}{1.113798in}}%
\pgfpathlineto{\pgfqpoint{2.538402in}{1.113667in}}%
\pgfpathlineto{\pgfqpoint{2.538598in}{1.119122in}}%
\pgfpathlineto{\pgfqpoint{2.539044in}{1.143361in}}%
\pgfpathlineto{\pgfqpoint{2.539847in}{1.127225in}}%
\pgfpathlineto{\pgfqpoint{2.539936in}{1.127363in}}%
\pgfpathlineto{\pgfqpoint{2.540239in}{1.124279in}}%
\pgfpathlineto{\pgfqpoint{2.540828in}{1.101896in}}%
\pgfpathlineto{\pgfqpoint{2.541702in}{1.114215in}}%
\pgfpathlineto{\pgfqpoint{2.541773in}{1.113732in}}%
\pgfpathlineto{\pgfqpoint{2.542023in}{1.122352in}}%
\pgfpathlineto{\pgfqpoint{2.542416in}{1.143282in}}%
\pgfpathlineto{\pgfqpoint{2.543183in}{1.126594in}}%
\pgfpathlineto{\pgfqpoint{2.543343in}{1.127072in}}%
\pgfpathlineto{\pgfqpoint{2.543629in}{1.123888in}}%
\pgfpathlineto{\pgfqpoint{2.544200in}{1.102423in}}%
\pgfpathlineto{\pgfqpoint{2.545056in}{1.114354in}}%
\pgfpathlineto{\pgfqpoint{2.545145in}{1.113686in}}%
\pgfpathlineto{\pgfqpoint{2.545395in}{1.122089in}}%
\pgfpathlineto{\pgfqpoint{2.545806in}{1.143245in}}%
\pgfpathlineto{\pgfqpoint{2.546573in}{1.126610in}}%
\pgfpathlineto{\pgfqpoint{2.546715in}{1.127039in}}%
\pgfpathlineto{\pgfqpoint{2.547001in}{1.124101in}}%
\pgfpathlineto{\pgfqpoint{2.547572in}{1.102157in}}%
\pgfpathlineto{\pgfqpoint{2.548464in}{1.114237in}}%
\pgfpathlineto{\pgfqpoint{2.548517in}{1.113984in}}%
\pgfpathlineto{\pgfqpoint{2.548749in}{1.121147in}}%
\pgfpathlineto{\pgfqpoint{2.549177in}{1.143521in}}%
\pgfpathlineto{\pgfqpoint{2.549945in}{1.126496in}}%
\pgfpathlineto{\pgfqpoint{2.550105in}{1.127098in}}%
\pgfpathlineto{\pgfqpoint{2.550408in}{1.123032in}}%
\pgfpathlineto{\pgfqpoint{2.550961in}{1.101864in}}%
\pgfpathlineto{\pgfqpoint{2.551729in}{1.115906in}}%
\pgfpathlineto{\pgfqpoint{2.551889in}{1.114354in}}%
\pgfpathlineto{\pgfqpoint{2.552175in}{1.124565in}}%
\pgfpathlineto{\pgfqpoint{2.552549in}{1.143026in}}%
\pgfpathlineto{\pgfqpoint{2.553299in}{1.126397in}}%
\pgfpathlineto{\pgfqpoint{2.553513in}{1.127183in}}%
\pgfpathlineto{\pgfqpoint{2.553798in}{1.122906in}}%
\pgfpathlineto{\pgfqpoint{2.554333in}{1.102050in}}%
\pgfpathlineto{\pgfqpoint{2.555083in}{1.116438in}}%
\pgfpathlineto{\pgfqpoint{2.555261in}{1.114459in}}%
\pgfpathlineto{\pgfqpoint{2.555564in}{1.125459in}}%
\pgfpathlineto{\pgfqpoint{2.555921in}{1.142745in}}%
\pgfpathlineto{\pgfqpoint{2.556688in}{1.126619in}}%
\pgfpathlineto{\pgfqpoint{2.556867in}{1.127419in}}%
\pgfpathlineto{\pgfqpoint{2.557170in}{1.122880in}}%
\pgfpathlineto{\pgfqpoint{2.557705in}{1.101743in}}%
\pgfpathlineto{\pgfqpoint{2.558437in}{1.116990in}}%
\pgfpathlineto{\pgfqpoint{2.558651in}{1.114926in}}%
\pgfpathlineto{\pgfqpoint{2.558936in}{1.125336in}}%
\pgfpathlineto{\pgfqpoint{2.559293in}{1.142533in}}%
\pgfpathlineto{\pgfqpoint{2.560060in}{1.126799in}}%
\pgfpathlineto{\pgfqpoint{2.560221in}{1.127554in}}%
\pgfpathlineto{\pgfqpoint{2.560542in}{1.123070in}}%
\pgfpathlineto{\pgfqpoint{2.561077in}{1.101781in}}%
\pgfpathlineto{\pgfqpoint{2.561791in}{1.117611in}}%
\pgfpathlineto{\pgfqpoint{2.562023in}{1.115039in}}%
\pgfpathlineto{\pgfqpoint{2.562308in}{1.124839in}}%
\pgfpathlineto{\pgfqpoint{2.562665in}{1.142158in}}%
\pgfpathlineto{\pgfqpoint{2.563450in}{1.127097in}}%
\pgfpathlineto{\pgfqpoint{2.563593in}{1.127757in}}%
\pgfpathlineto{\pgfqpoint{2.563950in}{1.121726in}}%
\pgfpathlineto{\pgfqpoint{2.564449in}{1.101782in}}%
\pgfpathlineto{\pgfqpoint{2.565145in}{1.117910in}}%
\pgfpathlineto{\pgfqpoint{2.565412in}{1.115189in}}%
\pgfpathlineto{\pgfqpoint{2.565662in}{1.123443in}}%
\pgfpathlineto{\pgfqpoint{2.566055in}{1.142176in}}%
\pgfpathlineto{\pgfqpoint{2.566822in}{1.127177in}}%
\pgfpathlineto{\pgfqpoint{2.566947in}{1.127626in}}%
\pgfpathlineto{\pgfqpoint{2.567268in}{1.123403in}}%
\pgfpathlineto{\pgfqpoint{2.567821in}{1.102116in}}%
\pgfpathlineto{\pgfqpoint{2.568517in}{1.118675in}}%
\pgfpathlineto{\pgfqpoint{2.568802in}{1.115498in}}%
\pgfpathlineto{\pgfqpoint{2.569052in}{1.124252in}}%
\pgfpathlineto{\pgfqpoint{2.569427in}{1.142075in}}%
\pgfpathlineto{\pgfqpoint{2.570194in}{1.127022in}}%
\pgfpathlineto{\pgfqpoint{2.570319in}{1.127499in}}%
\pgfpathlineto{\pgfqpoint{2.570640in}{1.123205in}}%
\pgfpathlineto{\pgfqpoint{2.571193in}{1.102305in}}%
\pgfpathlineto{\pgfqpoint{2.571889in}{1.118250in}}%
\pgfpathlineto{\pgfqpoint{2.572156in}{1.114924in}}%
\pgfpathlineto{\pgfqpoint{2.572424in}{1.123531in}}%
\pgfpathlineto{\pgfqpoint{2.572799in}{1.142002in}}%
\pgfpathlineto{\pgfqpoint{2.573584in}{1.127092in}}%
\pgfpathlineto{\pgfqpoint{2.573691in}{1.127360in}}%
\pgfpathlineto{\pgfqpoint{2.573976in}{1.123951in}}%
\pgfpathlineto{\pgfqpoint{2.574583in}{1.102227in}}%
\pgfpathlineto{\pgfqpoint{2.575314in}{1.117512in}}%
\pgfpathlineto{\pgfqpoint{2.575528in}{1.114923in}}%
\pgfpathlineto{\pgfqpoint{2.575831in}{1.125873in}}%
\pgfpathlineto{\pgfqpoint{2.576188in}{1.142435in}}%
\pgfpathlineto{\pgfqpoint{2.576938in}{1.126612in}}%
\pgfpathlineto{\pgfqpoint{2.577045in}{1.126999in}}%
\pgfpathlineto{\pgfqpoint{2.577348in}{1.123851in}}%
\pgfpathlineto{\pgfqpoint{2.577955in}{1.102570in}}%
\pgfpathlineto{\pgfqpoint{2.578704in}{1.117271in}}%
\pgfpathlineto{\pgfqpoint{2.578900in}{1.114896in}}%
\pgfpathlineto{\pgfqpoint{2.579221in}{1.126774in}}%
\pgfpathlineto{\pgfqpoint{2.579560in}{1.142257in}}%
\pgfpathlineto{\pgfqpoint{2.580202in}{1.125779in}}%
\pgfpathlineto{\pgfqpoint{2.580310in}{1.126085in}}%
\pgfpathlineto{\pgfqpoint{2.580434in}{1.126571in}}%
\pgfpathlineto{\pgfqpoint{2.580756in}{1.122740in}}%
\pgfpathlineto{\pgfqpoint{2.581326in}{1.102702in}}%
\pgfpathlineto{\pgfqpoint{2.582040in}{1.117578in}}%
\pgfpathlineto{\pgfqpoint{2.582272in}{1.114832in}}%
\pgfpathlineto{\pgfqpoint{2.582575in}{1.125455in}}%
\pgfpathlineto{\pgfqpoint{2.582932in}{1.142464in}}%
\pgfpathlineto{\pgfqpoint{2.583699in}{1.126362in}}%
\pgfpathlineto{\pgfqpoint{2.583842in}{1.126811in}}%
\pgfpathlineto{\pgfqpoint{2.584181in}{1.121682in}}%
\pgfpathlineto{\pgfqpoint{2.584698in}{1.103101in}}%
\pgfpathlineto{\pgfqpoint{2.585394in}{1.117981in}}%
\pgfpathlineto{\pgfqpoint{2.585644in}{1.115161in}}%
\pgfpathlineto{\pgfqpoint{2.585912in}{1.123287in}}%
\pgfpathlineto{\pgfqpoint{2.586304in}{1.142451in}}%
\pgfpathlineto{\pgfqpoint{2.587053in}{1.125950in}}%
\pgfpathlineto{\pgfqpoint{2.587214in}{1.126648in}}%
\pgfpathlineto{\pgfqpoint{2.587535in}{1.122372in}}%
\pgfpathlineto{\pgfqpoint{2.588088in}{1.103116in}}%
\pgfpathlineto{\pgfqpoint{2.588784in}{1.117667in}}%
\pgfpathlineto{\pgfqpoint{2.589034in}{1.115271in}}%
\pgfpathlineto{\pgfqpoint{2.589283in}{1.123106in}}%
\pgfpathlineto{\pgfqpoint{2.589676in}{1.142342in}}%
\pgfpathlineto{\pgfqpoint{2.590425in}{1.125913in}}%
\pgfpathlineto{\pgfqpoint{2.590586in}{1.126799in}}%
\pgfpathlineto{\pgfqpoint{2.590943in}{1.121368in}}%
\pgfpathlineto{\pgfqpoint{2.591460in}{1.102912in}}%
\pgfpathlineto{\pgfqpoint{2.592138in}{1.118229in}}%
\pgfpathlineto{\pgfqpoint{2.592388in}{1.115811in}}%
\pgfpathlineto{\pgfqpoint{2.592655in}{1.123457in}}%
\pgfpathlineto{\pgfqpoint{2.593048in}{1.142516in}}%
\pgfpathlineto{\pgfqpoint{2.593797in}{1.125993in}}%
\pgfpathlineto{\pgfqpoint{2.593993in}{1.126946in}}%
\pgfpathlineto{\pgfqpoint{2.594315in}{1.121413in}}%
\pgfpathlineto{\pgfqpoint{2.594814in}{1.102860in}}%
\pgfpathlineto{\pgfqpoint{2.595510in}{1.118472in}}%
\pgfpathlineto{\pgfqpoint{2.595777in}{1.116013in}}%
\pgfpathlineto{\pgfqpoint{2.596009in}{1.122101in}}%
\pgfpathlineto{\pgfqpoint{2.596438in}{1.141730in}}%
\pgfpathlineto{\pgfqpoint{2.597187in}{1.126022in}}%
\pgfpathlineto{\pgfqpoint{2.597365in}{1.126946in}}%
\pgfpathlineto{\pgfqpoint{2.597704in}{1.121035in}}%
\pgfpathlineto{\pgfqpoint{2.598204in}{1.102767in}}%
\pgfpathlineto{\pgfqpoint{2.598864in}{1.118701in}}%
\pgfpathlineto{\pgfqpoint{2.599149in}{1.115924in}}%
\pgfpathlineto{\pgfqpoint{2.599381in}{1.121736in}}%
\pgfpathlineto{\pgfqpoint{2.599809in}{1.141470in}}%
\pgfpathlineto{\pgfqpoint{2.600577in}{1.126563in}}%
\pgfpathlineto{\pgfqpoint{2.600701in}{1.127250in}}%
\pgfpathlineto{\pgfqpoint{2.601023in}{1.122824in}}%
\pgfpathlineto{\pgfqpoint{2.601576in}{1.102662in}}%
\pgfpathlineto{\pgfqpoint{2.602236in}{1.118992in}}%
\pgfpathlineto{\pgfqpoint{2.602539in}{1.116448in}}%
\pgfpathlineto{\pgfqpoint{2.602753in}{1.121779in}}%
\pgfpathlineto{\pgfqpoint{2.603181in}{1.141551in}}%
\pgfpathlineto{\pgfqpoint{2.603949in}{1.126519in}}%
\pgfpathlineto{\pgfqpoint{2.604091in}{1.127127in}}%
\pgfpathlineto{\pgfqpoint{2.604395in}{1.122731in}}%
\pgfpathlineto{\pgfqpoint{2.604948in}{1.102692in}}%
\pgfpathlineto{\pgfqpoint{2.605608in}{1.119261in}}%
\pgfpathlineto{\pgfqpoint{2.605911in}{1.116299in}}%
\pgfpathlineto{\pgfqpoint{2.606125in}{1.121319in}}%
\pgfpathlineto{\pgfqpoint{2.606553in}{1.141243in}}%
\pgfpathlineto{\pgfqpoint{2.607338in}{1.126779in}}%
\pgfpathlineto{\pgfqpoint{2.607463in}{1.127279in}}%
\pgfpathlineto{\pgfqpoint{2.607802in}{1.121464in}}%
\pgfpathlineto{\pgfqpoint{2.608320in}{1.102952in}}%
\pgfpathlineto{\pgfqpoint{2.608962in}{1.119414in}}%
\pgfpathlineto{\pgfqpoint{2.609194in}{1.116998in}}%
\pgfpathlineto{\pgfqpoint{2.609283in}{1.116319in}}%
\pgfpathlineto{\pgfqpoint{2.609568in}{1.124865in}}%
\pgfpathlineto{\pgfqpoint{2.609943in}{1.141193in}}%
\pgfpathlineto{\pgfqpoint{2.610692in}{1.126421in}}%
\pgfpathlineto{\pgfqpoint{2.610835in}{1.126948in}}%
\pgfpathlineto{\pgfqpoint{2.611138in}{1.122396in}}%
\pgfpathlineto{\pgfqpoint{2.611709in}{1.103063in}}%
\pgfpathlineto{\pgfqpoint{2.612334in}{1.119588in}}%
\pgfpathlineto{\pgfqpoint{2.612673in}{1.116197in}}%
\pgfpathlineto{\pgfqpoint{2.612869in}{1.120774in}}%
\pgfpathlineto{\pgfqpoint{2.613315in}{1.141304in}}%
\pgfpathlineto{\pgfqpoint{2.614100in}{1.126651in}}%
\pgfpathlineto{\pgfqpoint{2.614207in}{1.126943in}}%
\pgfpathlineto{\pgfqpoint{2.614492in}{1.122947in}}%
\pgfpathlineto{\pgfqpoint{2.615081in}{1.103335in}}%
\pgfpathlineto{\pgfqpoint{2.615723in}{1.119185in}}%
\pgfpathlineto{\pgfqpoint{2.616062in}{1.115939in}}%
\pgfpathlineto{\pgfqpoint{2.616259in}{1.121258in}}%
\pgfpathlineto{\pgfqpoint{2.616687in}{1.141363in}}%
\pgfpathlineto{\pgfqpoint{2.617454in}{1.126307in}}%
\pgfpathlineto{\pgfqpoint{2.617561in}{1.126701in}}%
\pgfpathlineto{\pgfqpoint{2.617882in}{1.122324in}}%
\pgfpathlineto{\pgfqpoint{2.618453in}{1.103476in}}%
\pgfpathlineto{\pgfqpoint{2.619077in}{1.119339in}}%
\pgfpathlineto{\pgfqpoint{2.619238in}{1.117885in}}%
\pgfpathlineto{\pgfqpoint{2.619416in}{1.116197in}}%
\pgfpathlineto{\pgfqpoint{2.619702in}{1.125390in}}%
\pgfpathlineto{\pgfqpoint{2.620077in}{1.141645in}}%
\pgfpathlineto{\pgfqpoint{2.620683in}{1.125299in}}%
\pgfpathlineto{\pgfqpoint{2.620808in}{1.125777in}}%
\pgfpathlineto{\pgfqpoint{2.620933in}{1.126272in}}%
\pgfpathlineto{\pgfqpoint{2.621236in}{1.122739in}}%
\pgfpathlineto{\pgfqpoint{2.621825in}{1.103734in}}%
\pgfpathlineto{\pgfqpoint{2.622467in}{1.119193in}}%
\pgfpathlineto{\pgfqpoint{2.622788in}{1.116107in}}%
\pgfpathlineto{\pgfqpoint{2.623002in}{1.121215in}}%
\pgfpathlineto{\pgfqpoint{2.623431in}{1.141294in}}%
\pgfpathlineto{\pgfqpoint{2.624198in}{1.125707in}}%
\pgfpathlineto{\pgfqpoint{2.624358in}{1.126234in}}%
\pgfpathlineto{\pgfqpoint{2.624644in}{1.122063in}}%
\pgfpathlineto{\pgfqpoint{2.625215in}{1.103977in}}%
\pgfpathlineto{\pgfqpoint{2.625857in}{1.118968in}}%
\pgfpathlineto{\pgfqpoint{2.626160in}{1.116237in}}%
\pgfpathlineto{\pgfqpoint{2.626392in}{1.122120in}}%
\pgfpathlineto{\pgfqpoint{2.626803in}{1.141604in}}%
\pgfpathlineto{\pgfqpoint{2.627570in}{1.125802in}}%
\pgfpathlineto{\pgfqpoint{2.627712in}{1.126384in}}%
\pgfpathlineto{\pgfqpoint{2.628051in}{1.121204in}}%
\pgfpathlineto{\pgfqpoint{2.628569in}{1.103845in}}%
\pgfpathlineto{\pgfqpoint{2.629229in}{1.118931in}}%
\pgfpathlineto{\pgfqpoint{2.629532in}{1.116442in}}%
\pgfpathlineto{\pgfqpoint{2.629764in}{1.122118in}}%
\pgfpathlineto{\pgfqpoint{2.630174in}{1.141274in}}%
\pgfpathlineto{\pgfqpoint{2.630942in}{1.125601in}}%
\pgfpathlineto{\pgfqpoint{2.631084in}{1.126384in}}%
\pgfpathlineto{\pgfqpoint{2.631459in}{1.120149in}}%
\pgfpathlineto{\pgfqpoint{2.631958in}{1.104000in}}%
\pgfpathlineto{\pgfqpoint{2.632583in}{1.119190in}}%
\pgfpathlineto{\pgfqpoint{2.632779in}{1.117715in}}%
\pgfpathlineto{\pgfqpoint{2.632904in}{1.116791in}}%
\pgfpathlineto{\pgfqpoint{2.633190in}{1.124891in}}%
\pgfpathlineto{\pgfqpoint{2.633564in}{1.141093in}}%
\pgfpathlineto{\pgfqpoint{2.634189in}{1.124680in}}%
\pgfpathlineto{\pgfqpoint{2.634296in}{1.125360in}}%
\pgfpathlineto{\pgfqpoint{2.634492in}{1.126411in}}%
\pgfpathlineto{\pgfqpoint{2.634813in}{1.120975in}}%
\pgfpathlineto{\pgfqpoint{2.635330in}{1.103673in}}%
\pgfpathlineto{\pgfqpoint{2.635955in}{1.119356in}}%
\pgfpathlineto{\pgfqpoint{2.635991in}{1.119366in}}%
\pgfpathlineto{\pgfqpoint{2.636133in}{1.118199in}}%
\pgfpathlineto{\pgfqpoint{2.636294in}{1.117109in}}%
\pgfpathlineto{\pgfqpoint{2.636544in}{1.123980in}}%
\pgfpathlineto{\pgfqpoint{2.636936in}{1.141069in}}%
\pgfpathlineto{\pgfqpoint{2.637685in}{1.125748in}}%
\pgfpathlineto{\pgfqpoint{2.637828in}{1.126614in}}%
\pgfpathlineto{\pgfqpoint{2.638185in}{1.121006in}}%
\pgfpathlineto{\pgfqpoint{2.638702in}{1.103662in}}%
\pgfpathlineto{\pgfqpoint{2.639327in}{1.119560in}}%
\pgfpathlineto{\pgfqpoint{2.639345in}{1.119579in}}%
\pgfpathlineto{\pgfqpoint{2.639523in}{1.118191in}}%
\pgfpathlineto{\pgfqpoint{2.639666in}{1.117180in}}%
\pgfpathlineto{\pgfqpoint{2.639951in}{1.125376in}}%
\pgfpathlineto{\pgfqpoint{2.640308in}{1.140545in}}%
\pgfpathlineto{\pgfqpoint{2.640915in}{1.124988in}}%
\pgfpathlineto{\pgfqpoint{2.641057in}{1.125866in}}%
\pgfpathlineto{\pgfqpoint{2.641236in}{1.126833in}}%
\pgfpathlineto{\pgfqpoint{2.641575in}{1.120636in}}%
\pgfpathlineto{\pgfqpoint{2.642074in}{1.103750in}}%
\pgfpathlineto{\pgfqpoint{2.642699in}{1.119751in}}%
\pgfpathlineto{\pgfqpoint{2.642716in}{1.119766in}}%
\pgfpathlineto{\pgfqpoint{2.642877in}{1.118568in}}%
\pgfpathlineto{\pgfqpoint{2.643055in}{1.117300in}}%
\pgfpathlineto{\pgfqpoint{2.643305in}{1.124270in}}%
\pgfpathlineto{\pgfqpoint{2.643680in}{1.140527in}}%
\pgfpathlineto{\pgfqpoint{2.644447in}{1.126112in}}%
\pgfpathlineto{\pgfqpoint{2.644572in}{1.126784in}}%
\pgfpathlineto{\pgfqpoint{2.644911in}{1.121467in}}%
\pgfpathlineto{\pgfqpoint{2.645446in}{1.103522in}}%
\pgfpathlineto{\pgfqpoint{2.646053in}{1.119836in}}%
\pgfpathlineto{\pgfqpoint{2.646427in}{1.117535in}}%
\pgfpathlineto{\pgfqpoint{2.646784in}{1.130064in}}%
\pgfpathlineto{\pgfqpoint{2.647052in}{1.140469in}}%
\pgfpathlineto{\pgfqpoint{2.647694in}{1.125082in}}%
\pgfpathlineto{\pgfqpoint{2.647819in}{1.125896in}}%
\pgfpathlineto{\pgfqpoint{2.647944in}{1.126586in}}%
\pgfpathlineto{\pgfqpoint{2.648283in}{1.121359in}}%
\pgfpathlineto{\pgfqpoint{2.648836in}{1.103790in}}%
\pgfpathlineto{\pgfqpoint{2.649425in}{1.119966in}}%
\pgfpathlineto{\pgfqpoint{2.649799in}{1.117234in}}%
\pgfpathlineto{\pgfqpoint{2.650192in}{1.131591in}}%
\pgfpathlineto{\pgfqpoint{2.650442in}{1.140276in}}%
\pgfpathlineto{\pgfqpoint{2.651048in}{1.125401in}}%
\pgfpathlineto{\pgfqpoint{2.651209in}{1.126377in}}%
\pgfpathlineto{\pgfqpoint{2.651316in}{1.126915in}}%
\pgfpathlineto{\pgfqpoint{2.651637in}{1.122153in}}%
\pgfpathlineto{\pgfqpoint{2.652190in}{1.104084in}}%
\pgfpathlineto{\pgfqpoint{2.652814in}{1.120023in}}%
\pgfpathlineto{\pgfqpoint{2.652868in}{1.120021in}}%
\pgfpathlineto{\pgfqpoint{2.652993in}{1.118848in}}%
\pgfpathlineto{\pgfqpoint{2.653189in}{1.117144in}}%
\pgfpathlineto{\pgfqpoint{2.653474in}{1.126055in}}%
\pgfpathlineto{\pgfqpoint{2.653813in}{1.140370in}}%
\pgfpathlineto{\pgfqpoint{2.654420in}{1.125328in}}%
\pgfpathlineto{\pgfqpoint{2.654581in}{1.126011in}}%
\pgfpathlineto{\pgfqpoint{2.654705in}{1.126487in}}%
\pgfpathlineto{\pgfqpoint{2.655009in}{1.121812in}}%
\pgfpathlineto{\pgfqpoint{2.655580in}{1.104157in}}%
\pgfpathlineto{\pgfqpoint{2.656168in}{1.120046in}}%
\pgfpathlineto{\pgfqpoint{2.656543in}{1.117264in}}%
\pgfpathlineto{\pgfqpoint{2.656953in}{1.132382in}}%
\pgfpathlineto{\pgfqpoint{2.657185in}{1.140600in}}%
\pgfpathlineto{\pgfqpoint{2.657828in}{1.125062in}}%
\pgfpathlineto{\pgfqpoint{2.657935in}{1.125573in}}%
\pgfpathlineto{\pgfqpoint{2.658077in}{1.126191in}}%
\pgfpathlineto{\pgfqpoint{2.658398in}{1.121180in}}%
\pgfpathlineto{\pgfqpoint{2.658952in}{1.104360in}}%
\pgfpathlineto{\pgfqpoint{2.659540in}{1.119601in}}%
\pgfpathlineto{\pgfqpoint{2.659915in}{1.116961in}}%
\pgfpathlineto{\pgfqpoint{2.660272in}{1.128808in}}%
\pgfpathlineto{\pgfqpoint{2.660575in}{1.140503in}}%
\pgfpathlineto{\pgfqpoint{2.661182in}{1.124953in}}%
\pgfpathlineto{\pgfqpoint{2.661324in}{1.125557in}}%
\pgfpathlineto{\pgfqpoint{2.661449in}{1.126080in}}%
\pgfpathlineto{\pgfqpoint{2.661770in}{1.121427in}}%
\pgfpathlineto{\pgfqpoint{2.662323in}{1.104657in}}%
\pgfpathlineto{\pgfqpoint{2.662930in}{1.119806in}}%
\pgfpathlineto{\pgfqpoint{2.662984in}{1.119849in}}%
\pgfpathlineto{\pgfqpoint{2.663215in}{1.117581in}}%
\pgfpathlineto{\pgfqpoint{2.663305in}{1.117176in}}%
\pgfpathlineto{\pgfqpoint{2.663554in}{1.123792in}}%
\pgfpathlineto{\pgfqpoint{2.663947in}{1.140534in}}%
\pgfpathlineto{\pgfqpoint{2.664678in}{1.125116in}}%
\pgfpathlineto{\pgfqpoint{2.664839in}{1.125935in}}%
\pgfpathlineto{\pgfqpoint{2.665178in}{1.120554in}}%
\pgfpathlineto{\pgfqpoint{2.665695in}{1.104902in}}%
\pgfpathlineto{\pgfqpoint{2.666302in}{1.119773in}}%
\pgfpathlineto{\pgfqpoint{2.666659in}{1.117405in}}%
\pgfpathlineto{\pgfqpoint{2.666980in}{1.126756in}}%
\pgfpathlineto{\pgfqpoint{2.667319in}{1.140557in}}%
\pgfpathlineto{\pgfqpoint{2.667943in}{1.124724in}}%
\pgfpathlineto{\pgfqpoint{2.668050in}{1.125227in}}%
\pgfpathlineto{\pgfqpoint{2.668229in}{1.126125in}}%
\pgfpathlineto{\pgfqpoint{2.668532in}{1.121537in}}%
\pgfpathlineto{\pgfqpoint{2.669085in}{1.104824in}}%
\pgfpathlineto{\pgfqpoint{2.669692in}{1.119697in}}%
\pgfpathlineto{\pgfqpoint{2.670031in}{1.117592in}}%
\pgfpathlineto{\pgfqpoint{2.670334in}{1.125699in}}%
\pgfpathlineto{\pgfqpoint{2.670691in}{1.140574in}}%
\pgfpathlineto{\pgfqpoint{2.671315in}{1.124485in}}%
\pgfpathlineto{\pgfqpoint{2.671422in}{1.125010in}}%
\pgfpathlineto{\pgfqpoint{2.671583in}{1.126058in}}%
\pgfpathlineto{\pgfqpoint{2.671940in}{1.120256in}}%
\pgfpathlineto{\pgfqpoint{2.672457in}{1.104534in}}%
\pgfpathlineto{\pgfqpoint{2.673064in}{1.119654in}}%
\pgfpathlineto{\pgfqpoint{2.673420in}{1.117819in}}%
\pgfpathlineto{\pgfqpoint{2.673759in}{1.128634in}}%
\pgfpathlineto{\pgfqpoint{2.674063in}{1.140237in}}%
\pgfpathlineto{\pgfqpoint{2.674687in}{1.124354in}}%
\pgfpathlineto{\pgfqpoint{2.674794in}{1.124964in}}%
\pgfpathlineto{\pgfqpoint{2.674973in}{1.126086in}}%
\pgfpathlineto{\pgfqpoint{2.675329in}{1.120043in}}%
\pgfpathlineto{\pgfqpoint{2.675829in}{1.104731in}}%
\pgfpathlineto{\pgfqpoint{2.676435in}{1.119787in}}%
\pgfpathlineto{\pgfqpoint{2.676792in}{1.117856in}}%
\pgfpathlineto{\pgfqpoint{2.677149in}{1.128970in}}%
\pgfpathlineto{\pgfqpoint{2.677435in}{1.139675in}}%
\pgfpathlineto{\pgfqpoint{2.678041in}{1.124316in}}%
\pgfpathlineto{\pgfqpoint{2.678184in}{1.125150in}}%
\pgfpathlineto{\pgfqpoint{2.678344in}{1.126201in}}%
\pgfpathlineto{\pgfqpoint{2.678683in}{1.120658in}}%
\pgfpathlineto{\pgfqpoint{2.679219in}{1.104613in}}%
\pgfpathlineto{\pgfqpoint{2.679807in}{1.119873in}}%
\pgfpathlineto{\pgfqpoint{2.680164in}{1.118003in}}%
\pgfpathlineto{\pgfqpoint{2.680539in}{1.129867in}}%
\pgfpathlineto{\pgfqpoint{2.680824in}{1.139683in}}%
\pgfpathlineto{\pgfqpoint{2.681449in}{1.124682in}}%
\pgfpathlineto{\pgfqpoint{2.681574in}{1.125584in}}%
\pgfpathlineto{\pgfqpoint{2.681716in}{1.126424in}}%
\pgfpathlineto{\pgfqpoint{2.682037in}{1.121310in}}%
\pgfpathlineto{\pgfqpoint{2.682573in}{1.104548in}}%
\pgfpathlineto{\pgfqpoint{2.683179in}{1.120278in}}%
\pgfpathlineto{\pgfqpoint{2.683554in}{1.118272in}}%
\pgfpathlineto{\pgfqpoint{2.683964in}{1.132419in}}%
\pgfpathlineto{\pgfqpoint{2.684196in}{1.139512in}}%
\pgfpathlineto{\pgfqpoint{2.684821in}{1.124798in}}%
\pgfpathlineto{\pgfqpoint{2.684946in}{1.125653in}}%
\pgfpathlineto{\pgfqpoint{2.685106in}{1.126541in}}%
\pgfpathlineto{\pgfqpoint{2.685427in}{1.120771in}}%
\pgfpathlineto{\pgfqpoint{2.685945in}{1.104699in}}%
\pgfpathlineto{\pgfqpoint{2.686533in}{1.120202in}}%
\pgfpathlineto{\pgfqpoint{2.687568in}{1.139304in}}%
\pgfpathlineto{\pgfqpoint{2.686944in}{1.118255in}}%
\pgfpathlineto{\pgfqpoint{2.688103in}{1.125045in}}%
\pgfpathlineto{\pgfqpoint{2.688460in}{1.126081in}}%
\pgfpathlineto{\pgfqpoint{2.689245in}{1.105085in}}%
\pgfpathlineto{\pgfqpoint{2.689317in}{1.104547in}}%
\pgfpathlineto{\pgfqpoint{2.689655in}{1.113298in}}%
\pgfpathlineto{\pgfqpoint{2.690940in}{1.139447in}}%
\pgfpathlineto{\pgfqpoint{2.691243in}{1.132088in}}%
\pgfpathlineto{\pgfqpoint{2.692688in}{1.104946in}}%
\pgfpathlineto{\pgfqpoint{2.692938in}{1.110308in}}%
\pgfpathlineto{\pgfqpoint{2.694330in}{1.139226in}}%
\pgfpathlineto{\pgfqpoint{2.694633in}{1.131370in}}%
\pgfpathlineto{\pgfqpoint{2.696060in}{1.105093in}}%
\pgfpathlineto{\pgfqpoint{2.696292in}{1.109558in}}%
\pgfpathlineto{\pgfqpoint{2.697702in}{1.139648in}}%
\pgfpathlineto{\pgfqpoint{2.698023in}{1.131160in}}%
\pgfpathlineto{\pgfqpoint{2.699432in}{1.105343in}}%
\pgfpathlineto{\pgfqpoint{2.699682in}{1.110386in}}%
\pgfpathlineto{\pgfqpoint{2.701074in}{1.139654in}}%
\pgfpathlineto{\pgfqpoint{2.701377in}{1.131774in}}%
\pgfpathlineto{\pgfqpoint{2.702822in}{1.105658in}}%
\pgfpathlineto{\pgfqpoint{2.703036in}{1.109703in}}%
\pgfpathlineto{\pgfqpoint{2.704463in}{1.139477in}}%
\pgfpathlineto{\pgfqpoint{2.704784in}{1.130397in}}%
\pgfpathlineto{\pgfqpoint{2.706194in}{1.105588in}}%
\pgfpathlineto{\pgfqpoint{2.706426in}{1.110102in}}%
\pgfpathlineto{\pgfqpoint{2.707817in}{1.139562in}}%
\pgfpathlineto{\pgfqpoint{2.708139in}{1.131036in}}%
\pgfpathlineto{\pgfqpoint{2.709584in}{1.105642in}}%
\pgfpathlineto{\pgfqpoint{2.709780in}{1.109349in}}%
\pgfpathlineto{\pgfqpoint{2.711207in}{1.139212in}}%
\pgfpathlineto{\pgfqpoint{2.711528in}{1.130247in}}%
\pgfpathlineto{\pgfqpoint{2.712938in}{1.105546in}}%
\pgfpathlineto{\pgfqpoint{2.713152in}{1.109201in}}%
\pgfpathlineto{\pgfqpoint{2.714579in}{1.139392in}}%
\pgfpathlineto{\pgfqpoint{2.714900in}{1.130438in}}%
\pgfpathlineto{\pgfqpoint{2.716327in}{1.105652in}}%
\pgfpathlineto{\pgfqpoint{2.716541in}{1.109696in}}%
\pgfpathlineto{\pgfqpoint{2.717951in}{1.138818in}}%
\pgfpathlineto{\pgfqpoint{2.718272in}{1.130226in}}%
\pgfpathlineto{\pgfqpoint{2.719699in}{1.105610in}}%
\pgfpathlineto{\pgfqpoint{2.719913in}{1.109607in}}%
\pgfpathlineto{\pgfqpoint{2.721323in}{1.138628in}}%
\pgfpathlineto{\pgfqpoint{2.721644in}{1.130417in}}%
\pgfpathlineto{\pgfqpoint{2.723071in}{1.105366in}}%
\pgfpathlineto{\pgfqpoint{2.723303in}{1.110008in}}%
\pgfpathlineto{\pgfqpoint{2.724695in}{1.138808in}}%
\pgfpathlineto{\pgfqpoint{2.725016in}{1.130770in}}%
\pgfpathlineto{\pgfqpoint{2.726425in}{1.105388in}}%
\pgfpathlineto{\pgfqpoint{2.726675in}{1.109991in}}%
\pgfpathlineto{\pgfqpoint{2.728084in}{1.138262in}}%
\pgfpathlineto{\pgfqpoint{2.728388in}{1.130623in}}%
\pgfpathlineto{\pgfqpoint{2.729815in}{1.105838in}}%
\pgfpathlineto{\pgfqpoint{2.730047in}{1.110399in}}%
\pgfpathlineto{\pgfqpoint{2.731456in}{1.138290in}}%
\pgfpathlineto{\pgfqpoint{2.731760in}{1.130969in}}%
\pgfpathlineto{\pgfqpoint{2.733187in}{1.105892in}}%
\pgfpathlineto{\pgfqpoint{2.733419in}{1.110382in}}%
\pgfpathlineto{\pgfqpoint{2.734846in}{1.138737in}}%
\pgfpathlineto{\pgfqpoint{2.735132in}{1.131411in}}%
\pgfpathlineto{\pgfqpoint{2.736559in}{1.106025in}}%
\pgfpathlineto{\pgfqpoint{2.736791in}{1.110346in}}%
\pgfpathlineto{\pgfqpoint{2.738218in}{1.138586in}}%
\pgfpathlineto{\pgfqpoint{2.738539in}{1.130074in}}%
\pgfpathlineto{\pgfqpoint{2.739949in}{1.106402in}}%
\pgfpathlineto{\pgfqpoint{2.740163in}{1.110412in}}%
\pgfpathlineto{\pgfqpoint{2.741590in}{1.138680in}}%
\pgfpathlineto{\pgfqpoint{2.741893in}{1.130735in}}%
\pgfpathlineto{\pgfqpoint{2.743303in}{1.106372in}}%
\pgfpathlineto{\pgfqpoint{2.743517in}{1.109774in}}%
\pgfpathlineto{\pgfqpoint{2.744962in}{1.138407in}}%
\pgfpathlineto{\pgfqpoint{2.745265in}{1.130604in}}%
\pgfpathlineto{\pgfqpoint{2.746692in}{1.106578in}}%
\pgfpathlineto{\pgfqpoint{2.746906in}{1.110160in}}%
\pgfpathlineto{\pgfqpoint{2.748334in}{1.138550in}}%
\pgfpathlineto{\pgfqpoint{2.748655in}{1.130245in}}%
\pgfpathlineto{\pgfqpoint{2.750064in}{1.106626in}}%
\pgfpathlineto{\pgfqpoint{2.750278in}{1.110159in}}%
\pgfpathlineto{\pgfqpoint{2.751706in}{1.138476in}}%
\pgfpathlineto{\pgfqpoint{2.752027in}{1.130060in}}%
\pgfpathlineto{\pgfqpoint{2.753454in}{1.106518in}}%
\pgfpathlineto{\pgfqpoint{2.753668in}{1.110375in}}%
\pgfpathlineto{\pgfqpoint{2.755078in}{1.137993in}}%
\pgfpathlineto{\pgfqpoint{2.755381in}{1.130442in}}%
\pgfpathlineto{\pgfqpoint{2.756808in}{1.106370in}}%
\pgfpathlineto{\pgfqpoint{2.757022in}{1.109696in}}%
\pgfpathlineto{\pgfqpoint{2.758467in}{1.137847in}}%
\pgfpathlineto{\pgfqpoint{2.758788in}{1.129315in}}%
\pgfpathlineto{\pgfqpoint{2.760198in}{1.106395in}}%
\pgfpathlineto{\pgfqpoint{2.760430in}{1.110643in}}%
\pgfpathlineto{\pgfqpoint{2.761839in}{1.137739in}}%
\pgfpathlineto{\pgfqpoint{2.762160in}{1.129568in}}%
\pgfpathlineto{\pgfqpoint{2.763570in}{1.106449in}}%
\pgfpathlineto{\pgfqpoint{2.763802in}{1.110767in}}%
\pgfpathlineto{\pgfqpoint{2.765211in}{1.137652in}}%
\pgfpathlineto{\pgfqpoint{2.765514in}{1.130286in}}%
\pgfpathlineto{\pgfqpoint{2.766942in}{1.106474in}}%
\pgfpathlineto{\pgfqpoint{2.767174in}{1.110894in}}%
\pgfpathlineto{\pgfqpoint{2.768601in}{1.137682in}}%
\pgfpathlineto{\pgfqpoint{2.768886in}{1.130630in}}%
\pgfpathlineto{\pgfqpoint{2.770296in}{1.106566in}}%
\pgfpathlineto{\pgfqpoint{2.770545in}{1.111054in}}%
\pgfpathlineto{\pgfqpoint{2.771973in}{1.137527in}}%
\pgfpathlineto{\pgfqpoint{2.772258in}{1.130700in}}%
\pgfpathlineto{\pgfqpoint{2.773685in}{1.106847in}}%
\pgfpathlineto{\pgfqpoint{2.773900in}{1.110493in}}%
\pgfpathlineto{\pgfqpoint{2.775345in}{1.137504in}}%
\pgfpathlineto{\pgfqpoint{2.775666in}{1.129653in}}%
\pgfpathlineto{\pgfqpoint{2.777057in}{1.107036in}}%
\pgfpathlineto{\pgfqpoint{2.777271in}{1.110576in}}%
\pgfpathlineto{\pgfqpoint{2.778717in}{1.137838in}}%
\pgfpathlineto{\pgfqpoint{2.779038in}{1.129842in}}%
\pgfpathlineto{\pgfqpoint{2.780429in}{1.107180in}}%
\pgfpathlineto{\pgfqpoint{2.780661in}{1.111098in}}%
\pgfpathlineto{\pgfqpoint{2.782106in}{1.137637in}}%
\pgfpathlineto{\pgfqpoint{2.782392in}{1.130459in}}%
\pgfpathlineto{\pgfqpoint{2.783801in}{1.107467in}}%
\pgfpathlineto{\pgfqpoint{2.784033in}{1.111065in}}%
\pgfpathlineto{\pgfqpoint{2.785460in}{1.137572in}}%
\pgfpathlineto{\pgfqpoint{2.785764in}{1.130348in}}%
\pgfpathlineto{\pgfqpoint{2.787191in}{1.107339in}}%
\pgfpathlineto{\pgfqpoint{2.787405in}{1.110917in}}%
\pgfpathlineto{\pgfqpoint{2.788850in}{1.137614in}}%
\pgfpathlineto{\pgfqpoint{2.789171in}{1.129221in}}%
\pgfpathlineto{\pgfqpoint{2.790563in}{1.107292in}}%
\pgfpathlineto{\pgfqpoint{2.790777in}{1.110572in}}%
\pgfpathlineto{\pgfqpoint{2.792222in}{1.137359in}}%
\pgfpathlineto{\pgfqpoint{2.792525in}{1.129687in}}%
\pgfpathlineto{\pgfqpoint{2.793953in}{1.107263in}}%
\pgfpathlineto{\pgfqpoint{2.794149in}{1.110474in}}%
\pgfpathlineto{\pgfqpoint{2.795594in}{1.137069in}}%
\pgfpathlineto{\pgfqpoint{2.795897in}{1.129560in}}%
\pgfpathlineto{\pgfqpoint{2.797307in}{1.107455in}}%
\pgfpathlineto{\pgfqpoint{2.797521in}{1.110634in}}%
\pgfpathlineto{\pgfqpoint{2.798966in}{1.136936in}}%
\pgfpathlineto{\pgfqpoint{2.799287in}{1.129019in}}%
\pgfpathlineto{\pgfqpoint{2.800696in}{1.107252in}}%
\pgfpathlineto{\pgfqpoint{2.800910in}{1.110837in}}%
\pgfpathlineto{\pgfqpoint{2.802338in}{1.136789in}}%
\pgfpathlineto{\pgfqpoint{2.802641in}{1.129866in}}%
\pgfpathlineto{\pgfqpoint{2.804050in}{1.107120in}}%
\pgfpathlineto{\pgfqpoint{2.804282in}{1.110833in}}%
\pgfpathlineto{\pgfqpoint{2.805727in}{1.136755in}}%
\pgfpathlineto{\pgfqpoint{2.806031in}{1.129492in}}%
\pgfpathlineto{\pgfqpoint{2.807422in}{1.107392in}}%
\pgfpathlineto{\pgfqpoint{2.807654in}{1.111143in}}%
\pgfpathlineto{\pgfqpoint{2.809099in}{1.136455in}}%
\pgfpathlineto{\pgfqpoint{2.809385in}{1.130058in}}%
\pgfpathlineto{\pgfqpoint{2.810794in}{1.107386in}}%
\pgfpathlineto{\pgfqpoint{2.811044in}{1.111559in}}%
\pgfpathlineto{\pgfqpoint{2.812471in}{1.136768in}}%
\pgfpathlineto{\pgfqpoint{2.812739in}{1.131104in}}%
\pgfpathlineto{\pgfqpoint{2.814184in}{1.107604in}}%
\pgfpathlineto{\pgfqpoint{2.814398in}{1.111196in}}%
\pgfpathlineto{\pgfqpoint{2.815861in}{1.136587in}}%
\pgfpathlineto{\pgfqpoint{2.816146in}{1.129836in}}%
\pgfpathlineto{\pgfqpoint{2.817556in}{1.107845in}}%
\pgfpathlineto{\pgfqpoint{2.817806in}{1.112169in}}%
\pgfpathlineto{\pgfqpoint{2.819215in}{1.136933in}}%
\pgfpathlineto{\pgfqpoint{2.819518in}{1.130270in}}%
\pgfpathlineto{\pgfqpoint{2.820928in}{1.108207in}}%
\pgfpathlineto{\pgfqpoint{2.821160in}{1.111846in}}%
\pgfpathlineto{\pgfqpoint{2.822605in}{1.136714in}}%
\pgfpathlineto{\pgfqpoint{2.822872in}{1.130588in}}%
\pgfpathlineto{\pgfqpoint{2.824300in}{1.108129in}}%
\pgfpathlineto{\pgfqpoint{2.824514in}{1.111133in}}%
\pgfpathlineto{\pgfqpoint{2.825977in}{1.136275in}}%
\pgfpathlineto{\pgfqpoint{2.826298in}{1.128518in}}%
\pgfpathlineto{\pgfqpoint{2.827689in}{1.108108in}}%
\pgfpathlineto{\pgfqpoint{2.827903in}{1.111399in}}%
\pgfpathlineto{\pgfqpoint{2.829349in}{1.136769in}}%
\pgfpathlineto{\pgfqpoint{2.829670in}{1.128969in}}%
\pgfpathlineto{\pgfqpoint{2.831061in}{1.108311in}}%
\pgfpathlineto{\pgfqpoint{2.831275in}{1.111563in}}%
\pgfpathlineto{\pgfqpoint{2.832720in}{1.136559in}}%
\pgfpathlineto{\pgfqpoint{2.833042in}{1.128836in}}%
\pgfpathlineto{\pgfqpoint{2.834433in}{1.108147in}}%
\pgfpathlineto{\pgfqpoint{2.834647in}{1.111224in}}%
\pgfpathlineto{\pgfqpoint{2.836110in}{1.136373in}}%
\pgfpathlineto{\pgfqpoint{2.836414in}{1.128921in}}%
\pgfpathlineto{\pgfqpoint{2.837805in}{1.108224in}}%
\pgfpathlineto{\pgfqpoint{2.838019in}{1.111242in}}%
\pgfpathlineto{\pgfqpoint{2.839464in}{1.135972in}}%
\pgfpathlineto{\pgfqpoint{2.839785in}{1.128713in}}%
\pgfpathlineto{\pgfqpoint{2.841177in}{1.108135in}}%
\pgfpathlineto{\pgfqpoint{2.841391in}{1.111176in}}%
\pgfpathlineto{\pgfqpoint{2.842854in}{1.135821in}}%
\pgfpathlineto{\pgfqpoint{2.843157in}{1.128881in}}%
\pgfpathlineto{\pgfqpoint{2.844567in}{1.108061in}}%
\pgfpathlineto{\pgfqpoint{2.844763in}{1.111081in}}%
\pgfpathlineto{\pgfqpoint{2.846226in}{1.136028in}}%
\pgfpathlineto{\pgfqpoint{2.846529in}{1.129398in}}%
\pgfpathlineto{\pgfqpoint{2.847921in}{1.108208in}}%
\pgfpathlineto{\pgfqpoint{2.848153in}{1.111766in}}%
\pgfpathlineto{\pgfqpoint{2.849598in}{1.135726in}}%
\pgfpathlineto{\pgfqpoint{2.849883in}{1.129873in}}%
\pgfpathlineto{\pgfqpoint{2.851311in}{1.108431in}}%
\pgfpathlineto{\pgfqpoint{2.851507in}{1.111384in}}%
\pgfpathlineto{\pgfqpoint{2.852970in}{1.135639in}}%
\pgfpathlineto{\pgfqpoint{2.853273in}{1.129476in}}%
\pgfpathlineto{\pgfqpoint{2.854665in}{1.108480in}}%
\pgfpathlineto{\pgfqpoint{2.854897in}{1.111668in}}%
\pgfpathlineto{\pgfqpoint{2.856377in}{1.135884in}}%
\pgfpathlineto{\pgfqpoint{2.856645in}{1.129726in}}%
\pgfpathlineto{\pgfqpoint{2.858036in}{1.108568in}}%
\pgfpathlineto{\pgfqpoint{2.858268in}{1.111736in}}%
\pgfpathlineto{\pgfqpoint{2.859731in}{1.135659in}}%
\pgfpathlineto{\pgfqpoint{2.860017in}{1.129581in}}%
\pgfpathlineto{\pgfqpoint{2.861426in}{1.108963in}}%
\pgfpathlineto{\pgfqpoint{2.861658in}{1.112482in}}%
\pgfpathlineto{\pgfqpoint{2.863103in}{1.135957in}}%
\pgfpathlineto{\pgfqpoint{2.863389in}{1.129899in}}%
\pgfpathlineto{\pgfqpoint{2.864798in}{1.109069in}}%
\pgfpathlineto{\pgfqpoint{2.864994in}{1.111529in}}%
\pgfpathlineto{\pgfqpoint{2.866475in}{1.135816in}}%
\pgfpathlineto{\pgfqpoint{2.866778in}{1.129062in}}%
\pgfpathlineto{\pgfqpoint{2.868188in}{1.109113in}}%
\pgfpathlineto{\pgfqpoint{2.868384in}{1.111726in}}%
\pgfpathlineto{\pgfqpoint{2.869865in}{1.135549in}}%
\pgfpathlineto{\pgfqpoint{2.870168in}{1.128368in}}%
\pgfpathlineto{\pgfqpoint{2.871560in}{1.108966in}}%
\pgfpathlineto{\pgfqpoint{2.871774in}{1.112086in}}%
\pgfpathlineto{\pgfqpoint{2.873219in}{1.135626in}}%
\pgfpathlineto{\pgfqpoint{2.873522in}{1.129076in}}%
\pgfpathlineto{\pgfqpoint{2.874932in}{1.108899in}}%
\pgfpathlineto{\pgfqpoint{2.875128in}{1.111486in}}%
\pgfpathlineto{\pgfqpoint{2.876609in}{1.135216in}}%
\pgfpathlineto{\pgfqpoint{2.876912in}{1.128295in}}%
\pgfpathlineto{\pgfqpoint{2.878304in}{1.109014in}}%
\pgfpathlineto{\pgfqpoint{2.878500in}{1.111485in}}%
\pgfpathlineto{\pgfqpoint{2.879981in}{1.135112in}}%
\pgfpathlineto{\pgfqpoint{2.880302in}{1.127996in}}%
\pgfpathlineto{\pgfqpoint{2.881675in}{1.108941in}}%
\pgfpathlineto{\pgfqpoint{2.881890in}{1.111737in}}%
\pgfpathlineto{\pgfqpoint{2.883370in}{1.135082in}}%
\pgfpathlineto{\pgfqpoint{2.883656in}{1.128709in}}%
\pgfpathlineto{\pgfqpoint{2.885047in}{1.108943in}}%
\pgfpathlineto{\pgfqpoint{2.885261in}{1.111770in}}%
\pgfpathlineto{\pgfqpoint{2.886742in}{1.134812in}}%
\pgfpathlineto{\pgfqpoint{2.887028in}{1.128667in}}%
\pgfpathlineto{\pgfqpoint{2.888419in}{1.109028in}}%
\pgfpathlineto{\pgfqpoint{2.888633in}{1.111926in}}%
\pgfpathlineto{\pgfqpoint{2.890114in}{1.135017in}}%
\pgfpathlineto{\pgfqpoint{2.890382in}{1.129708in}}%
\pgfpathlineto{\pgfqpoint{2.891791in}{1.109223in}}%
\pgfpathlineto{\pgfqpoint{2.892023in}{1.112526in}}%
\pgfpathlineto{\pgfqpoint{2.893486in}{1.134833in}}%
\pgfpathlineto{\pgfqpoint{2.893754in}{1.129686in}}%
\pgfpathlineto{\pgfqpoint{2.895163in}{1.109494in}}%
\pgfpathlineto{\pgfqpoint{2.895377in}{1.112242in}}%
\pgfpathlineto{\pgfqpoint{2.896876in}{1.134997in}}%
\pgfpathlineto{\pgfqpoint{2.897161in}{1.128807in}}%
\pgfpathlineto{\pgfqpoint{2.898535in}{1.109540in}}%
\pgfpathlineto{\pgfqpoint{2.898749in}{1.112143in}}%
\pgfpathlineto{\pgfqpoint{2.900248in}{1.135132in}}%
\pgfpathlineto{\pgfqpoint{2.900533in}{1.128876in}}%
\pgfpathlineto{\pgfqpoint{2.901925in}{1.109667in}}%
\pgfpathlineto{\pgfqpoint{2.902139in}{1.112619in}}%
\pgfpathlineto{\pgfqpoint{2.903620in}{1.134874in}}%
\pgfpathlineto{\pgfqpoint{2.903887in}{1.129291in}}%
\pgfpathlineto{\pgfqpoint{2.905297in}{1.109881in}}%
\pgfpathlineto{\pgfqpoint{2.905511in}{1.112558in}}%
\pgfpathlineto{\pgfqpoint{2.906992in}{1.134872in}}%
\pgfpathlineto{\pgfqpoint{2.907295in}{1.128164in}}%
\pgfpathlineto{\pgfqpoint{2.908669in}{1.109807in}}%
\pgfpathlineto{\pgfqpoint{2.908883in}{1.112457in}}%
\pgfpathlineto{\pgfqpoint{2.910363in}{1.135028in}}%
\pgfpathlineto{\pgfqpoint{2.910649in}{1.129011in}}%
\pgfpathlineto{\pgfqpoint{2.912058in}{1.109868in}}%
\pgfpathlineto{\pgfqpoint{2.912255in}{1.112295in}}%
\pgfpathlineto{\pgfqpoint{2.913735in}{1.134567in}}%
\pgfpathlineto{\pgfqpoint{2.914039in}{1.128068in}}%
\pgfpathlineto{\pgfqpoint{2.915412in}{1.109857in}}%
\pgfpathlineto{\pgfqpoint{2.915626in}{1.112293in}}%
\pgfpathlineto{\pgfqpoint{2.917125in}{1.134422in}}%
\pgfpathlineto{\pgfqpoint{2.917411in}{1.128154in}}%
\pgfpathlineto{\pgfqpoint{2.918802in}{1.109741in}}%
\pgfpathlineto{\pgfqpoint{2.918998in}{1.112132in}}%
\pgfpathlineto{\pgfqpoint{2.920479in}{1.134267in}}%
\pgfpathlineto{\pgfqpoint{2.920765in}{1.128753in}}%
\pgfpathlineto{\pgfqpoint{2.922174in}{1.109712in}}%
\pgfpathlineto{\pgfqpoint{2.922388in}{1.112484in}}%
\pgfpathlineto{\pgfqpoint{2.923869in}{1.134183in}}%
\pgfpathlineto{\pgfqpoint{2.924172in}{1.127903in}}%
\pgfpathlineto{\pgfqpoint{2.925564in}{1.109744in}}%
\pgfpathlineto{\pgfqpoint{2.925760in}{1.112463in}}%
\pgfpathlineto{\pgfqpoint{2.927259in}{1.134267in}}%
\pgfpathlineto{\pgfqpoint{2.927526in}{1.128641in}}%
\pgfpathlineto{\pgfqpoint{2.928918in}{1.109926in}}%
\pgfpathlineto{\pgfqpoint{2.929132in}{1.112715in}}%
\pgfpathlineto{\pgfqpoint{2.930631in}{1.133934in}}%
\pgfpathlineto{\pgfqpoint{2.930898in}{1.128542in}}%
\pgfpathlineto{\pgfqpoint{2.932308in}{1.110073in}}%
\pgfpathlineto{\pgfqpoint{2.932504in}{1.112816in}}%
\pgfpathlineto{\pgfqpoint{2.934002in}{1.134127in}}%
\pgfpathlineto{\pgfqpoint{2.934270in}{1.128913in}}%
\pgfpathlineto{\pgfqpoint{2.935662in}{1.110167in}}%
\pgfpathlineto{\pgfqpoint{2.935876in}{1.112733in}}%
\pgfpathlineto{\pgfqpoint{2.937374in}{1.134189in}}%
\pgfpathlineto{\pgfqpoint{2.937642in}{1.128984in}}%
\pgfpathlineto{\pgfqpoint{2.939051in}{1.110349in}}%
\pgfpathlineto{\pgfqpoint{2.939248in}{1.112846in}}%
\pgfpathlineto{\pgfqpoint{2.940764in}{1.134248in}}%
\pgfpathlineto{\pgfqpoint{2.941032in}{1.128544in}}%
\pgfpathlineto{\pgfqpoint{2.942423in}{1.110530in}}%
\pgfpathlineto{\pgfqpoint{2.942619in}{1.112912in}}%
\pgfpathlineto{\pgfqpoint{2.944118in}{1.134325in}}%
\pgfpathlineto{\pgfqpoint{2.944404in}{1.128573in}}%
\pgfpathlineto{\pgfqpoint{2.945795in}{1.110534in}}%
\pgfpathlineto{\pgfqpoint{2.945991in}{1.112743in}}%
\pgfpathlineto{\pgfqpoint{2.947490in}{1.133986in}}%
\pgfpathlineto{\pgfqpoint{2.947775in}{1.128427in}}%
\pgfpathlineto{\pgfqpoint{2.949167in}{1.110613in}}%
\pgfpathlineto{\pgfqpoint{2.949363in}{1.112673in}}%
\pgfpathlineto{\pgfqpoint{2.950880in}{1.133990in}}%
\pgfpathlineto{\pgfqpoint{2.951165in}{1.127909in}}%
\pgfpathlineto{\pgfqpoint{2.952557in}{1.110459in}}%
\pgfpathlineto{\pgfqpoint{2.952753in}{1.112837in}}%
\pgfpathlineto{\pgfqpoint{2.954252in}{1.133807in}}%
\pgfpathlineto{\pgfqpoint{2.954555in}{1.127314in}}%
\pgfpathlineto{\pgfqpoint{2.955929in}{1.110553in}}%
\pgfpathlineto{\pgfqpoint{2.956125in}{1.112866in}}%
\pgfpathlineto{\pgfqpoint{2.957624in}{1.133564in}}%
\pgfpathlineto{\pgfqpoint{2.957927in}{1.127337in}}%
\pgfpathlineto{\pgfqpoint{2.959283in}{1.110796in}}%
\pgfpathlineto{\pgfqpoint{2.959479in}{1.112614in}}%
\pgfpathlineto{\pgfqpoint{2.960995in}{1.133516in}}%
\pgfpathlineto{\pgfqpoint{2.961299in}{1.127518in}}%
\pgfpathlineto{\pgfqpoint{2.962690in}{1.110534in}}%
\pgfpathlineto{\pgfqpoint{2.962869in}{1.112790in}}%
\pgfpathlineto{\pgfqpoint{2.964385in}{1.133404in}}%
\pgfpathlineto{\pgfqpoint{2.964653in}{1.128067in}}%
\pgfpathlineto{\pgfqpoint{2.966044in}{1.110503in}}%
\pgfpathlineto{\pgfqpoint{2.966258in}{1.113169in}}%
\pgfpathlineto{\pgfqpoint{2.967757in}{1.133338in}}%
\pgfpathlineto{\pgfqpoint{2.968007in}{1.128811in}}%
\pgfpathlineto{\pgfqpoint{2.969416in}{1.110691in}}%
\pgfpathlineto{\pgfqpoint{2.969613in}{1.112911in}}%
\pgfpathlineto{\pgfqpoint{2.971129in}{1.133372in}}%
\pgfpathlineto{\pgfqpoint{2.971379in}{1.128963in}}%
\pgfpathlineto{\pgfqpoint{2.972788in}{1.110823in}}%
\pgfpathlineto{\pgfqpoint{2.973002in}{1.113346in}}%
\pgfpathlineto{\pgfqpoint{2.974519in}{1.133029in}}%
\pgfpathlineto{\pgfqpoint{2.974751in}{1.128912in}}%
\pgfpathlineto{\pgfqpoint{2.976160in}{1.111051in}}%
\pgfpathlineto{\pgfqpoint{2.976356in}{1.113166in}}%
\pgfpathlineto{\pgfqpoint{2.977891in}{1.133450in}}%
\pgfpathlineto{\pgfqpoint{2.978158in}{1.128237in}}%
\pgfpathlineto{\pgfqpoint{2.979532in}{1.111004in}}%
\pgfpathlineto{\pgfqpoint{2.979728in}{1.113059in}}%
\pgfpathlineto{\pgfqpoint{2.981263in}{1.133260in}}%
\pgfpathlineto{\pgfqpoint{2.981530in}{1.128036in}}%
\pgfpathlineto{\pgfqpoint{2.982886in}{1.111315in}}%
\pgfpathlineto{\pgfqpoint{2.983100in}{1.113227in}}%
\pgfpathlineto{\pgfqpoint{2.984617in}{1.133130in}}%
\pgfpathlineto{\pgfqpoint{2.984920in}{1.127502in}}%
\pgfpathlineto{\pgfqpoint{2.986294in}{1.111326in}}%
\pgfpathlineto{\pgfqpoint{2.986472in}{1.113039in}}%
\pgfpathlineto{\pgfqpoint{2.988006in}{1.133299in}}%
\pgfpathlineto{\pgfqpoint{2.988310in}{1.127055in}}%
\pgfpathlineto{\pgfqpoint{2.989666in}{1.111227in}}%
\pgfpathlineto{\pgfqpoint{2.989862in}{1.113146in}}%
\pgfpathlineto{\pgfqpoint{2.991396in}{1.132983in}}%
\pgfpathlineto{\pgfqpoint{2.991682in}{1.127051in}}%
\pgfpathlineto{\pgfqpoint{2.993055in}{1.111455in}}%
\pgfpathlineto{\pgfqpoint{2.993234in}{1.113355in}}%
\pgfpathlineto{\pgfqpoint{2.994750in}{1.133031in}}%
\pgfpathlineto{\pgfqpoint{2.995036in}{1.127627in}}%
\pgfpathlineto{\pgfqpoint{2.996427in}{1.111314in}}%
\pgfpathlineto{\pgfqpoint{2.996606in}{1.113208in}}%
\pgfpathlineto{\pgfqpoint{2.998122in}{1.132767in}}%
\pgfpathlineto{\pgfqpoint{2.998408in}{1.127544in}}%
\pgfpathlineto{\pgfqpoint{2.999799in}{1.111320in}}%
\pgfpathlineto{\pgfqpoint{2.999995in}{1.113490in}}%
\pgfpathlineto{\pgfqpoint{3.001512in}{1.132578in}}%
\pgfpathlineto{\pgfqpoint{3.001779in}{1.127678in}}%
\pgfpathlineto{\pgfqpoint{3.003171in}{1.111437in}}%
\pgfpathlineto{\pgfqpoint{3.003385in}{1.113913in}}%
\pgfpathlineto{\pgfqpoint{3.004884in}{1.132546in}}%
\pgfpathlineto{\pgfqpoint{3.005151in}{1.127748in}}%
\pgfpathlineto{\pgfqpoint{3.006543in}{1.111309in}}%
\pgfpathlineto{\pgfqpoint{3.006757in}{1.113903in}}%
\pgfpathlineto{\pgfqpoint{3.008256in}{1.132694in}}%
\pgfpathlineto{\pgfqpoint{3.008523in}{1.128035in}}%
\pgfpathlineto{\pgfqpoint{3.009915in}{1.111579in}}%
\pgfpathlineto{\pgfqpoint{3.010111in}{1.113693in}}%
\pgfpathlineto{\pgfqpoint{3.011645in}{1.132611in}}%
\pgfpathlineto{\pgfqpoint{3.011895in}{1.128208in}}%
\pgfpathlineto{\pgfqpoint{3.013287in}{1.111596in}}%
\pgfpathlineto{\pgfqpoint{3.013501in}{1.114027in}}%
\pgfpathlineto{\pgfqpoint{3.015017in}{1.132541in}}%
\pgfpathlineto{\pgfqpoint{3.015249in}{1.128751in}}%
\pgfpathlineto{\pgfqpoint{3.016659in}{1.111735in}}%
\pgfpathlineto{\pgfqpoint{3.016855in}{1.113811in}}%
\pgfpathlineto{\pgfqpoint{3.018389in}{1.132509in}}%
\pgfpathlineto{\pgfqpoint{3.018657in}{1.127765in}}%
\pgfpathlineto{\pgfqpoint{3.020031in}{1.111719in}}%
\pgfpathlineto{\pgfqpoint{3.020262in}{1.114418in}}%
\pgfpathlineto{\pgfqpoint{3.021761in}{1.132472in}}%
\pgfpathlineto{\pgfqpoint{3.022029in}{1.127727in}}%
\pgfpathlineto{\pgfqpoint{3.023402in}{1.111926in}}%
\pgfpathlineto{\pgfqpoint{3.023617in}{1.114166in}}%
\pgfpathlineto{\pgfqpoint{3.025133in}{1.132532in}}%
\pgfpathlineto{\pgfqpoint{3.025401in}{1.127810in}}%
\pgfpathlineto{\pgfqpoint{3.026792in}{1.112272in}}%
\pgfpathlineto{\pgfqpoint{3.026971in}{1.113907in}}%
\pgfpathlineto{\pgfqpoint{3.028505in}{1.132356in}}%
\pgfpathlineto{\pgfqpoint{3.028808in}{1.126755in}}%
\pgfpathlineto{\pgfqpoint{3.030146in}{1.112118in}}%
\pgfpathlineto{\pgfqpoint{3.030342in}{1.113635in}}%
\pgfpathlineto{\pgfqpoint{3.031877in}{1.132466in}}%
\pgfpathlineto{\pgfqpoint{3.032198in}{1.126403in}}%
\pgfpathlineto{\pgfqpoint{3.033518in}{1.112087in}}%
\pgfpathlineto{\pgfqpoint{3.033732in}{1.113935in}}%
\pgfpathlineto{\pgfqpoint{3.035267in}{1.132142in}}%
\pgfpathlineto{\pgfqpoint{3.035534in}{1.127287in}}%
\pgfpathlineto{\pgfqpoint{3.036908in}{1.111974in}}%
\pgfpathlineto{\pgfqpoint{3.037104in}{1.113808in}}%
\pgfpathlineto{\pgfqpoint{3.038621in}{1.131892in}}%
\pgfpathlineto{\pgfqpoint{3.038888in}{1.127714in}}%
\pgfpathlineto{\pgfqpoint{3.040298in}{1.111933in}}%
\pgfpathlineto{\pgfqpoint{3.040476in}{1.113707in}}%
\pgfpathlineto{\pgfqpoint{3.042028in}{1.131861in}}%
\pgfpathlineto{\pgfqpoint{3.042278in}{1.127348in}}%
\pgfpathlineto{\pgfqpoint{3.043670in}{1.111966in}}%
\pgfpathlineto{\pgfqpoint{3.043848in}{1.113778in}}%
\pgfpathlineto{\pgfqpoint{3.045400in}{1.131863in}}%
\pgfpathlineto{\pgfqpoint{3.045686in}{1.126571in}}%
\pgfpathlineto{\pgfqpoint{3.047024in}{1.112076in}}%
\pgfpathlineto{\pgfqpoint{3.047220in}{1.113860in}}%
\pgfpathlineto{\pgfqpoint{3.048772in}{1.131946in}}%
\pgfpathlineto{\pgfqpoint{3.049075in}{1.126276in}}%
\pgfpathlineto{\pgfqpoint{3.050413in}{1.112322in}}%
\pgfpathlineto{\pgfqpoint{3.050592in}{1.114069in}}%
\pgfpathlineto{\pgfqpoint{3.052144in}{1.131641in}}%
\pgfpathlineto{\pgfqpoint{3.052394in}{1.127532in}}%
\pgfpathlineto{\pgfqpoint{3.053785in}{1.112218in}}%
\pgfpathlineto{\pgfqpoint{3.053981in}{1.114241in}}%
\pgfpathlineto{\pgfqpoint{3.055534in}{1.131720in}}%
\pgfpathlineto{\pgfqpoint{3.055766in}{1.127901in}}%
\pgfpathlineto{\pgfqpoint{3.057139in}{1.112313in}}%
\pgfpathlineto{\pgfqpoint{3.057371in}{1.114554in}}%
\pgfpathlineto{\pgfqpoint{3.058906in}{1.131867in}}%
\pgfpathlineto{\pgfqpoint{3.059137in}{1.127985in}}%
\pgfpathlineto{\pgfqpoint{3.060529in}{1.112551in}}%
\pgfpathlineto{\pgfqpoint{3.060725in}{1.114464in}}%
\pgfpathlineto{\pgfqpoint{3.062260in}{1.131632in}}%
\pgfpathlineto{\pgfqpoint{3.062527in}{1.127303in}}%
\pgfpathlineto{\pgfqpoint{3.063919in}{1.112666in}}%
\pgfpathlineto{\pgfqpoint{3.064115in}{1.114747in}}%
\pgfpathlineto{\pgfqpoint{3.065631in}{1.131718in}}%
\pgfpathlineto{\pgfqpoint{3.065899in}{1.127490in}}%
\pgfpathlineto{\pgfqpoint{3.067291in}{1.112748in}}%
\pgfpathlineto{\pgfqpoint{3.067469in}{1.114377in}}%
\pgfpathlineto{\pgfqpoint{3.069021in}{1.131635in}}%
\pgfpathlineto{\pgfqpoint{3.069307in}{1.126439in}}%
\pgfpathlineto{\pgfqpoint{3.070680in}{1.112861in}}%
\pgfpathlineto{\pgfqpoint{3.070841in}{1.114323in}}%
\pgfpathlineto{\pgfqpoint{3.072393in}{1.131570in}}%
\pgfpathlineto{\pgfqpoint{3.072661in}{1.126949in}}%
\pgfpathlineto{\pgfqpoint{3.074017in}{1.112768in}}%
\pgfpathlineto{\pgfqpoint{3.074231in}{1.114542in}}%
\pgfpathlineto{\pgfqpoint{3.075747in}{1.131591in}}%
\pgfpathlineto{\pgfqpoint{3.075997in}{1.128019in}}%
\pgfpathlineto{\pgfqpoint{3.077406in}{1.112661in}}%
\pgfpathlineto{\pgfqpoint{3.077603in}{1.114425in}}%
\pgfpathlineto{\pgfqpoint{3.079137in}{1.131146in}}%
\pgfpathlineto{\pgfqpoint{3.079405in}{1.126890in}}%
\pgfpathlineto{\pgfqpoint{3.080778in}{1.112676in}}%
\pgfpathlineto{\pgfqpoint{3.080975in}{1.114356in}}%
\pgfpathlineto{\pgfqpoint{3.082509in}{1.131151in}}%
\pgfpathlineto{\pgfqpoint{3.082776in}{1.127171in}}%
\pgfpathlineto{\pgfqpoint{3.084168in}{1.112737in}}%
\pgfpathlineto{\pgfqpoint{3.084364in}{1.114740in}}%
\pgfpathlineto{\pgfqpoint{3.085899in}{1.131105in}}%
\pgfpathlineto{\pgfqpoint{3.086166in}{1.126706in}}%
\pgfpathlineto{\pgfqpoint{3.087522in}{1.112825in}}%
\pgfpathlineto{\pgfqpoint{3.087718in}{1.114570in}}%
\pgfpathlineto{\pgfqpoint{3.089270in}{1.131144in}}%
\pgfpathlineto{\pgfqpoint{3.089574in}{1.126068in}}%
\pgfpathlineto{\pgfqpoint{3.090912in}{1.112982in}}%
\pgfpathlineto{\pgfqpoint{3.091108in}{1.114949in}}%
\pgfpathlineto{\pgfqpoint{3.092642in}{1.131014in}}%
\pgfpathlineto{\pgfqpoint{3.092910in}{1.126901in}}%
\pgfpathlineto{\pgfqpoint{3.094284in}{1.113036in}}%
\pgfpathlineto{\pgfqpoint{3.094480in}{1.114975in}}%
\pgfpathlineto{\pgfqpoint{3.096014in}{1.130947in}}%
\pgfpathlineto{\pgfqpoint{3.096264in}{1.127428in}}%
\pgfpathlineto{\pgfqpoint{3.097638in}{1.112984in}}%
\pgfpathlineto{\pgfqpoint{3.097852in}{1.114818in}}%
\pgfpathlineto{\pgfqpoint{3.099404in}{1.131106in}}%
\pgfpathlineto{\pgfqpoint{3.099654in}{1.127189in}}%
\pgfpathlineto{\pgfqpoint{3.101028in}{1.113126in}}%
\pgfpathlineto{\pgfqpoint{3.101224in}{1.114935in}}%
\pgfpathlineto{\pgfqpoint{3.102758in}{1.131064in}}%
\pgfpathlineto{\pgfqpoint{3.102990in}{1.128016in}}%
\pgfpathlineto{\pgfqpoint{3.104417in}{1.113358in}}%
\pgfpathlineto{\pgfqpoint{3.104596in}{1.115132in}}%
\pgfpathlineto{\pgfqpoint{3.106148in}{1.131026in}}%
\pgfpathlineto{\pgfqpoint{3.106398in}{1.127237in}}%
\pgfpathlineto{\pgfqpoint{3.107789in}{1.113391in}}%
\pgfpathlineto{\pgfqpoint{3.107950in}{1.114634in}}%
\pgfpathlineto{\pgfqpoint{3.109520in}{1.130840in}}%
\pgfpathlineto{\pgfqpoint{3.109787in}{1.126627in}}%
\pgfpathlineto{\pgfqpoint{3.111161in}{1.113256in}}%
\pgfpathlineto{\pgfqpoint{3.111339in}{1.114753in}}%
\pgfpathlineto{\pgfqpoint{3.112892in}{1.130743in}}%
\pgfpathlineto{\pgfqpoint{3.113159in}{1.126530in}}%
\pgfpathlineto{\pgfqpoint{3.114551in}{1.113276in}}%
\pgfpathlineto{\pgfqpoint{3.114711in}{1.114698in}}%
\pgfpathlineto{\pgfqpoint{3.116264in}{1.130631in}}%
\pgfpathlineto{\pgfqpoint{3.116549in}{1.126106in}}%
\pgfpathlineto{\pgfqpoint{3.117905in}{1.113298in}}%
\pgfpathlineto{\pgfqpoint{3.118083in}{1.114703in}}%
\pgfpathlineto{\pgfqpoint{3.119635in}{1.130441in}}%
\pgfpathlineto{\pgfqpoint{3.119903in}{1.126659in}}%
\pgfpathlineto{\pgfqpoint{3.121259in}{1.113319in}}%
\pgfpathlineto{\pgfqpoint{3.121473in}{1.114934in}}%
\pgfpathlineto{\pgfqpoint{3.123007in}{1.130385in}}%
\pgfpathlineto{\pgfqpoint{3.123257in}{1.127150in}}%
\pgfpathlineto{\pgfqpoint{3.124631in}{1.113422in}}%
\pgfpathlineto{\pgfqpoint{3.124863in}{1.115389in}}%
\pgfpathlineto{\pgfqpoint{3.126397in}{1.130643in}}%
\pgfpathlineto{\pgfqpoint{3.126629in}{1.127583in}}%
\pgfpathlineto{\pgfqpoint{3.128038in}{1.113500in}}%
\pgfpathlineto{\pgfqpoint{3.128217in}{1.115163in}}%
\pgfpathlineto{\pgfqpoint{3.129769in}{1.130438in}}%
\pgfpathlineto{\pgfqpoint{3.130037in}{1.126591in}}%
\pgfpathlineto{\pgfqpoint{3.131392in}{1.113390in}}%
\pgfpathlineto{\pgfqpoint{3.131571in}{1.114801in}}%
\pgfpathlineto{\pgfqpoint{3.133159in}{1.130179in}}%
\pgfpathlineto{\pgfqpoint{3.133426in}{1.126083in}}%
\pgfpathlineto{\pgfqpoint{3.134782in}{1.113431in}}%
\pgfpathlineto{\pgfqpoint{3.134978in}{1.115267in}}%
\pgfpathlineto{\pgfqpoint{3.136531in}{1.130284in}}%
\pgfpathlineto{\pgfqpoint{3.136780in}{1.126682in}}%
\pgfpathlineto{\pgfqpoint{3.138136in}{1.113726in}}%
\pgfpathlineto{\pgfqpoint{3.138333in}{1.115314in}}%
\pgfpathlineto{\pgfqpoint{3.139903in}{1.130553in}}%
\pgfpathlineto{\pgfqpoint{3.140170in}{1.126508in}}%
\pgfpathlineto{\pgfqpoint{3.141508in}{1.113989in}}%
\pgfpathlineto{\pgfqpoint{3.141704in}{1.115374in}}%
\pgfpathlineto{\pgfqpoint{3.143292in}{1.130430in}}%
\pgfpathlineto{\pgfqpoint{3.143560in}{1.125960in}}%
\pgfpathlineto{\pgfqpoint{3.144898in}{1.113832in}}%
\pgfpathlineto{\pgfqpoint{3.145076in}{1.115210in}}%
\pgfpathlineto{\pgfqpoint{3.146646in}{1.130177in}}%
\pgfpathlineto{\pgfqpoint{3.146896in}{1.126738in}}%
\pgfpathlineto{\pgfqpoint{3.148288in}{1.114103in}}%
\pgfpathlineto{\pgfqpoint{3.148430in}{1.115118in}}%
\pgfpathlineto{\pgfqpoint{3.150018in}{1.130320in}}%
\pgfpathlineto{\pgfqpoint{3.150304in}{1.126030in}}%
\pgfpathlineto{\pgfqpoint{3.151642in}{1.113912in}}%
\pgfpathlineto{\pgfqpoint{3.151838in}{1.115266in}}%
\pgfpathlineto{\pgfqpoint{3.153408in}{1.130213in}}%
\pgfpathlineto{\pgfqpoint{3.153676in}{1.126084in}}%
\pgfpathlineto{\pgfqpoint{3.155031in}{1.113978in}}%
\pgfpathlineto{\pgfqpoint{3.155210in}{1.115336in}}%
\pgfpathlineto{\pgfqpoint{3.156762in}{1.129932in}}%
\pgfpathlineto{\pgfqpoint{3.157030in}{1.126334in}}%
\pgfpathlineto{\pgfqpoint{3.158403in}{1.113897in}}%
\pgfpathlineto{\pgfqpoint{3.158582in}{1.115293in}}%
\pgfpathlineto{\pgfqpoint{3.160134in}{1.129655in}}%
\pgfpathlineto{\pgfqpoint{3.160384in}{1.126639in}}%
\pgfpathlineto{\pgfqpoint{3.161775in}{1.113843in}}%
\pgfpathlineto{\pgfqpoint{3.161972in}{1.115424in}}%
\pgfpathlineto{\pgfqpoint{3.163542in}{1.129719in}}%
\pgfpathlineto{\pgfqpoint{3.163756in}{1.126852in}}%
\pgfpathlineto{\pgfqpoint{3.165147in}{1.113950in}}%
\pgfpathlineto{\pgfqpoint{3.165361in}{1.115827in}}%
\pgfpathlineto{\pgfqpoint{3.166896in}{1.129734in}}%
\pgfpathlineto{\pgfqpoint{3.167074in}{1.127991in}}%
\pgfpathlineto{\pgfqpoint{3.168519in}{1.113856in}}%
\pgfpathlineto{\pgfqpoint{3.168769in}{1.116422in}}%
\pgfpathlineto{\pgfqpoint{3.170285in}{1.129719in}}%
\pgfpathlineto{\pgfqpoint{3.170464in}{1.127851in}}%
\pgfpathlineto{\pgfqpoint{3.171891in}{1.114164in}}%
\pgfpathlineto{\pgfqpoint{3.172141in}{1.116707in}}%
\pgfpathlineto{\pgfqpoint{3.173675in}{1.129690in}}%
\pgfpathlineto{\pgfqpoint{3.173836in}{1.127864in}}%
\pgfpathlineto{\pgfqpoint{3.175281in}{1.114139in}}%
\pgfpathlineto{\pgfqpoint{3.175495in}{1.116194in}}%
\pgfpathlineto{\pgfqpoint{3.177029in}{1.129503in}}%
\pgfpathlineto{\pgfqpoint{3.177225in}{1.127412in}}%
\pgfpathlineto{\pgfqpoint{3.178635in}{1.114066in}}%
\pgfpathlineto{\pgfqpoint{3.178867in}{1.116043in}}%
\pgfpathlineto{\pgfqpoint{3.180419in}{1.129715in}}%
\pgfpathlineto{\pgfqpoint{3.180615in}{1.127229in}}%
\pgfpathlineto{\pgfqpoint{3.182007in}{1.114401in}}%
\pgfpathlineto{\pgfqpoint{3.182256in}{1.116641in}}%
\pgfpathlineto{\pgfqpoint{3.183773in}{1.129505in}}%
\pgfpathlineto{\pgfqpoint{3.183951in}{1.127777in}}%
\pgfpathlineto{\pgfqpoint{3.185396in}{1.114361in}}%
\pgfpathlineto{\pgfqpoint{3.185628in}{1.116479in}}%
\pgfpathlineto{\pgfqpoint{3.187163in}{1.129537in}}%
\pgfpathlineto{\pgfqpoint{3.187323in}{1.127956in}}%
\pgfpathlineto{\pgfqpoint{3.188768in}{1.114441in}}%
\pgfpathlineto{\pgfqpoint{3.189018in}{1.116738in}}%
\pgfpathlineto{\pgfqpoint{3.190517in}{1.129584in}}%
\pgfpathlineto{\pgfqpoint{3.190731in}{1.127273in}}%
\pgfpathlineto{\pgfqpoint{3.192158in}{1.114661in}}%
\pgfpathlineto{\pgfqpoint{3.192372in}{1.116584in}}%
\pgfpathlineto{\pgfqpoint{3.193906in}{1.129540in}}%
\pgfpathlineto{\pgfqpoint{3.194138in}{1.126500in}}%
\pgfpathlineto{\pgfqpoint{3.195530in}{1.114492in}}%
\pgfpathlineto{\pgfqpoint{3.195708in}{1.115820in}}%
\pgfpathlineto{\pgfqpoint{3.197278in}{1.129453in}}%
\pgfpathlineto{\pgfqpoint{3.197510in}{1.126599in}}%
\pgfpathlineto{\pgfqpoint{3.198902in}{1.114535in}}%
\pgfpathlineto{\pgfqpoint{3.199080in}{1.115840in}}%
\pgfpathlineto{\pgfqpoint{3.200650in}{1.129135in}}%
\pgfpathlineto{\pgfqpoint{3.200882in}{1.126445in}}%
\pgfpathlineto{\pgfqpoint{3.202292in}{1.114623in}}%
\pgfpathlineto{\pgfqpoint{3.202452in}{1.115905in}}%
\pgfpathlineto{\pgfqpoint{3.204040in}{1.128977in}}%
\pgfpathlineto{\pgfqpoint{3.204236in}{1.126794in}}%
\pgfpathlineto{\pgfqpoint{3.205664in}{1.114416in}}%
\pgfpathlineto{\pgfqpoint{3.205860in}{1.116268in}}%
\pgfpathlineto{\pgfqpoint{3.207412in}{1.129231in}}%
\pgfpathlineto{\pgfqpoint{3.207626in}{1.126717in}}%
\pgfpathlineto{\pgfqpoint{3.209035in}{1.114515in}}%
\pgfpathlineto{\pgfqpoint{3.209232in}{1.116353in}}%
\pgfpathlineto{\pgfqpoint{3.210784in}{1.129020in}}%
\pgfpathlineto{\pgfqpoint{3.210962in}{1.127347in}}%
\pgfpathlineto{\pgfqpoint{3.212390in}{1.114814in}}%
\pgfpathlineto{\pgfqpoint{3.212639in}{1.117073in}}%
\pgfpathlineto{\pgfqpoint{3.214138in}{1.129079in}}%
\pgfpathlineto{\pgfqpoint{3.214316in}{1.127809in}}%
\pgfpathlineto{\pgfqpoint{3.215779in}{1.114612in}}%
\pgfpathlineto{\pgfqpoint{3.216029in}{1.117195in}}%
\pgfpathlineto{\pgfqpoint{3.217528in}{1.129088in}}%
\pgfpathlineto{\pgfqpoint{3.217706in}{1.127499in}}%
\pgfpathlineto{\pgfqpoint{3.219169in}{1.114815in}}%
\pgfpathlineto{\pgfqpoint{3.219401in}{1.117272in}}%
\pgfpathlineto{\pgfqpoint{3.220900in}{1.129011in}}%
\pgfpathlineto{\pgfqpoint{3.221078in}{1.127468in}}%
\pgfpathlineto{\pgfqpoint{3.222541in}{1.114816in}}%
\pgfpathlineto{\pgfqpoint{3.222755in}{1.116886in}}%
\pgfpathlineto{\pgfqpoint{3.224271in}{1.128940in}}%
\pgfpathlineto{\pgfqpoint{3.224450in}{1.127430in}}%
\pgfpathlineto{\pgfqpoint{3.225913in}{1.114936in}}%
\pgfpathlineto{\pgfqpoint{3.226145in}{1.117286in}}%
\pgfpathlineto{\pgfqpoint{3.227661in}{1.128804in}}%
\pgfpathlineto{\pgfqpoint{3.227840in}{1.127132in}}%
\pgfpathlineto{\pgfqpoint{3.229267in}{1.115054in}}%
\pgfpathlineto{\pgfqpoint{3.229499in}{1.116892in}}%
\pgfpathlineto{\pgfqpoint{3.231033in}{1.128817in}}%
\pgfpathlineto{\pgfqpoint{3.231265in}{1.126055in}}%
\pgfpathlineto{\pgfqpoint{3.232657in}{1.114741in}}%
\pgfpathlineto{\pgfqpoint{3.232835in}{1.116030in}}%
\pgfpathlineto{\pgfqpoint{3.234387in}{1.128686in}}%
\pgfpathlineto{\pgfqpoint{3.234601in}{1.126718in}}%
\pgfpathlineto{\pgfqpoint{3.236028in}{1.115113in}}%
\pgfpathlineto{\pgfqpoint{3.236225in}{1.116549in}}%
\pgfpathlineto{\pgfqpoint{3.237777in}{1.128527in}}%
\pgfpathlineto{\pgfqpoint{3.237955in}{1.127004in}}%
\pgfpathlineto{\pgfqpoint{3.239418in}{1.114878in}}%
\pgfpathlineto{\pgfqpoint{3.239632in}{1.116916in}}%
\pgfpathlineto{\pgfqpoint{3.241149in}{1.128592in}}%
\pgfpathlineto{\pgfqpoint{3.241309in}{1.127396in}}%
\pgfpathlineto{\pgfqpoint{3.242808in}{1.114960in}}%
\pgfpathlineto{\pgfqpoint{3.243058in}{1.117888in}}%
\pgfpathlineto{\pgfqpoint{3.244539in}{1.128617in}}%
\pgfpathlineto{\pgfqpoint{3.244681in}{1.127469in}}%
\pgfpathlineto{\pgfqpoint{3.246144in}{1.115113in}}%
\pgfpathlineto{\pgfqpoint{3.246430in}{1.118086in}}%
\pgfpathlineto{\pgfqpoint{3.247910in}{1.128396in}}%
\pgfpathlineto{\pgfqpoint{3.248053in}{1.127373in}}%
\pgfpathlineto{\pgfqpoint{3.249516in}{1.114999in}}%
\pgfpathlineto{\pgfqpoint{3.249802in}{1.117863in}}%
\pgfpathlineto{\pgfqpoint{3.251282in}{1.128445in}}%
\pgfpathlineto{\pgfqpoint{3.251425in}{1.127479in}}%
\pgfpathlineto{\pgfqpoint{3.252906in}{1.115225in}}%
\pgfpathlineto{\pgfqpoint{3.253156in}{1.117743in}}%
\pgfpathlineto{\pgfqpoint{3.254654in}{1.128349in}}%
\pgfpathlineto{\pgfqpoint{3.254833in}{1.126947in}}%
\pgfpathlineto{\pgfqpoint{3.256260in}{1.115341in}}%
\pgfpathlineto{\pgfqpoint{3.256528in}{1.117721in}}%
\pgfpathlineto{\pgfqpoint{3.258026in}{1.128439in}}%
\pgfpathlineto{\pgfqpoint{3.258205in}{1.127124in}}%
\pgfpathlineto{\pgfqpoint{3.259632in}{1.115201in}}%
\pgfpathlineto{\pgfqpoint{3.259899in}{1.117615in}}%
\pgfpathlineto{\pgfqpoint{3.261416in}{1.128454in}}%
\pgfpathlineto{\pgfqpoint{3.261576in}{1.127089in}}%
\pgfpathlineto{\pgfqpoint{3.263004in}{1.115445in}}%
\pgfpathlineto{\pgfqpoint{3.263253in}{1.117421in}}%
\pgfpathlineto{\pgfqpoint{3.264788in}{1.128416in}}%
\pgfpathlineto{\pgfqpoint{3.264931in}{1.127342in}}%
\pgfpathlineto{\pgfqpoint{3.266411in}{1.115413in}}%
\pgfpathlineto{\pgfqpoint{3.266661in}{1.117899in}}%
\pgfpathlineto{\pgfqpoint{3.268160in}{1.128189in}}%
\pgfpathlineto{\pgfqpoint{3.268302in}{1.127176in}}%
\pgfpathlineto{\pgfqpoint{3.269783in}{1.115465in}}%
\pgfpathlineto{\pgfqpoint{3.270051in}{1.118117in}}%
\pgfpathlineto{\pgfqpoint{3.271514in}{1.128247in}}%
\pgfpathlineto{\pgfqpoint{3.271692in}{1.127039in}}%
\pgfpathlineto{\pgfqpoint{3.273155in}{1.115453in}}%
\pgfpathlineto{\pgfqpoint{3.273441in}{1.118337in}}%
\pgfpathlineto{\pgfqpoint{3.274903in}{1.128110in}}%
\pgfpathlineto{\pgfqpoint{3.275100in}{1.126395in}}%
\pgfpathlineto{\pgfqpoint{3.276545in}{1.115376in}}%
\pgfpathlineto{\pgfqpoint{3.276723in}{1.116816in}}%
\pgfpathlineto{\pgfqpoint{3.278275in}{1.128049in}}%
\pgfpathlineto{\pgfqpoint{3.278472in}{1.126421in}}%
\pgfpathlineto{\pgfqpoint{3.279899in}{1.115562in}}%
\pgfpathlineto{\pgfqpoint{3.280113in}{1.117246in}}%
\pgfpathlineto{\pgfqpoint{3.281647in}{1.127979in}}%
\pgfpathlineto{\pgfqpoint{3.281826in}{1.126751in}}%
\pgfpathlineto{\pgfqpoint{3.283289in}{1.115408in}}%
\pgfpathlineto{\pgfqpoint{3.283503in}{1.117267in}}%
\pgfpathlineto{\pgfqpoint{3.285037in}{1.127929in}}%
\pgfpathlineto{\pgfqpoint{3.285144in}{1.127367in}}%
\pgfpathlineto{\pgfqpoint{3.286661in}{1.115486in}}%
\pgfpathlineto{\pgfqpoint{3.287000in}{1.119492in}}%
\pgfpathlineto{\pgfqpoint{3.288427in}{1.127891in}}%
\pgfpathlineto{\pgfqpoint{3.288534in}{1.127247in}}%
\pgfpathlineto{\pgfqpoint{3.290032in}{1.115563in}}%
\pgfpathlineto{\pgfqpoint{3.290354in}{1.119255in}}%
\pgfpathlineto{\pgfqpoint{3.291799in}{1.127802in}}%
\pgfpathlineto{\pgfqpoint{3.291906in}{1.127171in}}%
\pgfpathlineto{\pgfqpoint{3.293404in}{1.115574in}}%
\pgfpathlineto{\pgfqpoint{3.293725in}{1.119177in}}%
\pgfpathlineto{\pgfqpoint{3.295171in}{1.127786in}}%
\pgfpathlineto{\pgfqpoint{3.295278in}{1.127136in}}%
\pgfpathlineto{\pgfqpoint{3.296758in}{1.115527in}}%
\pgfpathlineto{\pgfqpoint{3.297080in}{1.118776in}}%
\pgfpathlineto{\pgfqpoint{3.298542in}{1.127810in}}%
\pgfpathlineto{\pgfqpoint{3.298650in}{1.127225in}}%
\pgfpathlineto{\pgfqpoint{3.300148in}{1.115789in}}%
\pgfpathlineto{\pgfqpoint{3.300505in}{1.119760in}}%
\pgfpathlineto{\pgfqpoint{3.301914in}{1.127698in}}%
\pgfpathlineto{\pgfqpoint{3.302057in}{1.126773in}}%
\pgfpathlineto{\pgfqpoint{3.303502in}{1.115845in}}%
\pgfpathlineto{\pgfqpoint{3.303788in}{1.118299in}}%
\pgfpathlineto{\pgfqpoint{3.305268in}{1.127662in}}%
\pgfpathlineto{\pgfqpoint{3.305411in}{1.126928in}}%
\pgfpathlineto{\pgfqpoint{3.306910in}{1.115940in}}%
\pgfpathlineto{\pgfqpoint{3.307195in}{1.118869in}}%
\pgfpathlineto{\pgfqpoint{3.308658in}{1.127814in}}%
\pgfpathlineto{\pgfqpoint{3.308801in}{1.126901in}}%
\pgfpathlineto{\pgfqpoint{3.310300in}{1.116062in}}%
\pgfpathlineto{\pgfqpoint{3.310567in}{1.118861in}}%
\pgfpathlineto{\pgfqpoint{3.312048in}{1.127602in}}%
\pgfpathlineto{\pgfqpoint{3.312173in}{1.126714in}}%
\pgfpathlineto{\pgfqpoint{3.313689in}{1.115829in}}%
\pgfpathlineto{\pgfqpoint{3.313921in}{1.118270in}}%
\pgfpathlineto{\pgfqpoint{3.315420in}{1.127577in}}%
\pgfpathlineto{\pgfqpoint{3.315527in}{1.126930in}}%
\pgfpathlineto{\pgfqpoint{3.317043in}{1.115921in}}%
\pgfpathlineto{\pgfqpoint{3.317364in}{1.119538in}}%
\pgfpathlineto{\pgfqpoint{3.318774in}{1.127429in}}%
\pgfpathlineto{\pgfqpoint{3.318899in}{1.126834in}}%
\pgfpathlineto{\pgfqpoint{3.320397in}{1.116043in}}%
\pgfpathlineto{\pgfqpoint{3.320754in}{1.119829in}}%
\pgfpathlineto{\pgfqpoint{3.322164in}{1.127331in}}%
\pgfpathlineto{\pgfqpoint{3.322289in}{1.126582in}}%
\pgfpathlineto{\pgfqpoint{3.323787in}{1.115897in}}%
\pgfpathlineto{\pgfqpoint{3.324037in}{1.118248in}}%
\pgfpathlineto{\pgfqpoint{3.325536in}{1.127361in}}%
\pgfpathlineto{\pgfqpoint{3.325643in}{1.126874in}}%
\pgfpathlineto{\pgfqpoint{3.327159in}{1.115888in}}%
\pgfpathlineto{\pgfqpoint{3.327516in}{1.119995in}}%
\pgfpathlineto{\pgfqpoint{3.328907in}{1.127439in}}%
\pgfpathlineto{\pgfqpoint{3.329032in}{1.126886in}}%
\pgfpathlineto{\pgfqpoint{3.330549in}{1.116129in}}%
\pgfpathlineto{\pgfqpoint{3.330888in}{1.120252in}}%
\pgfpathlineto{\pgfqpoint{3.332297in}{1.127347in}}%
\pgfpathlineto{\pgfqpoint{3.332386in}{1.126884in}}%
\pgfpathlineto{\pgfqpoint{3.333921in}{1.116064in}}%
\pgfpathlineto{\pgfqpoint{3.334224in}{1.119499in}}%
\pgfpathlineto{\pgfqpoint{3.335651in}{1.127309in}}%
\pgfpathlineto{\pgfqpoint{3.335776in}{1.126794in}}%
\pgfpathlineto{\pgfqpoint{3.337275in}{1.116193in}}%
\pgfpathlineto{\pgfqpoint{3.337614in}{1.119869in}}%
\pgfpathlineto{\pgfqpoint{3.339005in}{1.127381in}}%
\pgfpathlineto{\pgfqpoint{3.339130in}{1.127011in}}%
\pgfpathlineto{\pgfqpoint{3.340647in}{1.116084in}}%
\pgfpathlineto{\pgfqpoint{3.341021in}{1.120263in}}%
\pgfpathlineto{\pgfqpoint{3.342413in}{1.127174in}}%
\pgfpathlineto{\pgfqpoint{3.342502in}{1.126818in}}%
\pgfpathlineto{\pgfqpoint{3.344054in}{1.116388in}}%
\pgfpathlineto{\pgfqpoint{3.344447in}{1.121144in}}%
\pgfpathlineto{\pgfqpoint{3.345785in}{1.127286in}}%
\pgfpathlineto{\pgfqpoint{3.345945in}{1.126308in}}%
\pgfpathlineto{\pgfqpoint{3.347408in}{1.116351in}}%
\pgfpathlineto{\pgfqpoint{3.347640in}{1.118286in}}%
\pgfpathlineto{\pgfqpoint{3.349157in}{1.127024in}}%
\pgfpathlineto{\pgfqpoint{3.349282in}{1.126420in}}%
\pgfpathlineto{\pgfqpoint{3.350798in}{1.116297in}}%
\pgfpathlineto{\pgfqpoint{3.351048in}{1.118714in}}%
\pgfpathlineto{\pgfqpoint{3.352529in}{1.127079in}}%
\pgfpathlineto{\pgfqpoint{3.352636in}{1.126646in}}%
\pgfpathlineto{\pgfqpoint{3.354170in}{1.116302in}}%
\pgfpathlineto{\pgfqpoint{3.354509in}{1.120048in}}%
\pgfpathlineto{\pgfqpoint{3.355918in}{1.126974in}}%
\pgfpathlineto{\pgfqpoint{3.356025in}{1.126381in}}%
\pgfpathlineto{\pgfqpoint{3.357524in}{1.116250in}}%
\pgfpathlineto{\pgfqpoint{3.357809in}{1.118849in}}%
\pgfpathlineto{\pgfqpoint{3.359290in}{1.126846in}}%
\pgfpathlineto{\pgfqpoint{3.359379in}{1.126459in}}%
\pgfpathlineto{\pgfqpoint{3.360896in}{1.116165in}}%
\pgfpathlineto{\pgfqpoint{3.361253in}{1.119858in}}%
\pgfpathlineto{\pgfqpoint{3.362680in}{1.126815in}}%
\pgfpathlineto{\pgfqpoint{3.362769in}{1.126419in}}%
\pgfpathlineto{\pgfqpoint{3.364286in}{1.116198in}}%
\pgfpathlineto{\pgfqpoint{3.364642in}{1.120146in}}%
\pgfpathlineto{\pgfqpoint{3.366034in}{1.126759in}}%
\pgfpathlineto{\pgfqpoint{3.366159in}{1.126160in}}%
\pgfpathlineto{\pgfqpoint{3.367658in}{1.116300in}}%
\pgfpathlineto{\pgfqpoint{3.367961in}{1.119432in}}%
\pgfpathlineto{\pgfqpoint{3.369388in}{1.126724in}}%
\pgfpathlineto{\pgfqpoint{3.369459in}{1.126638in}}%
\pgfpathlineto{\pgfqpoint{3.371029in}{1.116315in}}%
\pgfpathlineto{\pgfqpoint{3.371529in}{1.122213in}}%
\pgfpathlineto{\pgfqpoint{3.372796in}{1.126834in}}%
\pgfpathlineto{\pgfqpoint{3.372974in}{1.125603in}}%
\pgfpathlineto{\pgfqpoint{3.374401in}{1.116549in}}%
\pgfpathlineto{\pgfqpoint{3.374651in}{1.118649in}}%
\pgfpathlineto{\pgfqpoint{3.376150in}{1.126799in}}%
\pgfpathlineto{\pgfqpoint{3.376221in}{1.126685in}}%
\pgfpathlineto{\pgfqpoint{3.377791in}{1.116504in}}%
\pgfpathlineto{\pgfqpoint{3.378237in}{1.121734in}}%
\pgfpathlineto{\pgfqpoint{3.379522in}{1.126864in}}%
\pgfpathlineto{\pgfqpoint{3.379682in}{1.126084in}}%
\pgfpathlineto{\pgfqpoint{3.381163in}{1.116728in}}%
\pgfpathlineto{\pgfqpoint{3.381413in}{1.118957in}}%
\pgfpathlineto{\pgfqpoint{3.382911in}{1.126824in}}%
\pgfpathlineto{\pgfqpoint{3.383001in}{1.126501in}}%
\pgfpathlineto{\pgfqpoint{3.384553in}{1.116579in}}%
\pgfpathlineto{\pgfqpoint{3.384892in}{1.120381in}}%
\pgfpathlineto{\pgfqpoint{3.386265in}{1.126647in}}%
\pgfpathlineto{\pgfqpoint{3.386390in}{1.126164in}}%
\pgfpathlineto{\pgfqpoint{3.387925in}{1.116842in}}%
\pgfpathlineto{\pgfqpoint{3.388228in}{1.119900in}}%
\pgfpathlineto{\pgfqpoint{3.389655in}{1.126723in}}%
\pgfpathlineto{\pgfqpoint{3.389762in}{1.126277in}}%
\pgfpathlineto{\pgfqpoint{3.391297in}{1.116679in}}%
\pgfpathlineto{\pgfqpoint{3.391636in}{1.120353in}}%
\pgfpathlineto{\pgfqpoint{3.393027in}{1.126618in}}%
\pgfpathlineto{\pgfqpoint{3.393134in}{1.126211in}}%
\pgfpathlineto{\pgfqpoint{3.394668in}{1.116689in}}%
\pgfpathlineto{\pgfqpoint{3.394954in}{1.119465in}}%
\pgfpathlineto{\pgfqpoint{3.396381in}{1.126611in}}%
\pgfpathlineto{\pgfqpoint{3.396506in}{1.126212in}}%
\pgfpathlineto{\pgfqpoint{3.398058in}{1.116632in}}%
\pgfpathlineto{\pgfqpoint{3.398379in}{1.120126in}}%
\pgfpathlineto{\pgfqpoint{3.399789in}{1.126423in}}%
\pgfpathlineto{\pgfqpoint{3.399842in}{1.126263in}}%
\pgfpathlineto{\pgfqpoint{3.401430in}{1.116705in}}%
\pgfpathlineto{\pgfqpoint{3.401840in}{1.121596in}}%
\pgfpathlineto{\pgfqpoint{3.403143in}{1.126377in}}%
\pgfpathlineto{\pgfqpoint{3.403268in}{1.125954in}}%
\pgfpathlineto{\pgfqpoint{3.404784in}{1.116664in}}%
\pgfpathlineto{\pgfqpoint{3.405070in}{1.119272in}}%
\pgfpathlineto{\pgfqpoint{3.406533in}{1.126337in}}%
\pgfpathlineto{\pgfqpoint{3.406604in}{1.126118in}}%
\pgfpathlineto{\pgfqpoint{3.408138in}{1.116642in}}%
\pgfpathlineto{\pgfqpoint{3.408602in}{1.121711in}}%
\pgfpathlineto{\pgfqpoint{3.409922in}{1.126388in}}%
\pgfpathlineto{\pgfqpoint{3.410029in}{1.125898in}}%
\pgfpathlineto{\pgfqpoint{3.411510in}{1.116943in}}%
\pgfpathlineto{\pgfqpoint{3.411813in}{1.119526in}}%
\pgfpathlineto{\pgfqpoint{3.413241in}{1.126276in}}%
\pgfpathlineto{\pgfqpoint{3.413366in}{1.126006in}}%
\pgfpathlineto{\pgfqpoint{3.414936in}{1.116808in}}%
\pgfpathlineto{\pgfqpoint{3.415364in}{1.121885in}}%
\pgfpathlineto{\pgfqpoint{3.416666in}{1.126232in}}%
\pgfpathlineto{\pgfqpoint{3.416791in}{1.125679in}}%
\pgfpathlineto{\pgfqpoint{3.418290in}{1.116867in}}%
\pgfpathlineto{\pgfqpoint{3.418539in}{1.119065in}}%
\pgfpathlineto{\pgfqpoint{3.420056in}{1.126211in}}%
\pgfpathlineto{\pgfqpoint{3.420127in}{1.125901in}}%
\pgfpathlineto{\pgfqpoint{3.421662in}{1.116945in}}%
\pgfpathlineto{\pgfqpoint{3.421983in}{1.120114in}}%
\pgfpathlineto{\pgfqpoint{3.423392in}{1.126273in}}%
\pgfpathlineto{\pgfqpoint{3.423517in}{1.125842in}}%
\pgfpathlineto{\pgfqpoint{3.425033in}{1.116853in}}%
\pgfpathlineto{\pgfqpoint{3.425301in}{1.119104in}}%
\pgfpathlineto{\pgfqpoint{3.426764in}{1.126260in}}%
\pgfpathlineto{\pgfqpoint{3.426853in}{1.126049in}}%
\pgfpathlineto{\pgfqpoint{3.428423in}{1.116957in}}%
\pgfpathlineto{\pgfqpoint{3.428780in}{1.120690in}}%
\pgfpathlineto{\pgfqpoint{3.430154in}{1.125896in}}%
\pgfpathlineto{\pgfqpoint{3.430225in}{1.125726in}}%
\pgfpathlineto{\pgfqpoint{3.431795in}{1.116955in}}%
\pgfpathlineto{\pgfqpoint{3.432188in}{1.121222in}}%
\pgfpathlineto{\pgfqpoint{3.433561in}{1.126244in}}%
\pgfpathlineto{\pgfqpoint{3.433633in}{1.125890in}}%
\pgfpathlineto{\pgfqpoint{3.435167in}{1.117294in}}%
\pgfpathlineto{\pgfqpoint{3.435506in}{1.120684in}}%
\pgfpathlineto{\pgfqpoint{3.436898in}{1.126217in}}%
\pgfpathlineto{\pgfqpoint{3.436987in}{1.125936in}}%
\pgfpathlineto{\pgfqpoint{3.438557in}{1.117155in}}%
\pgfpathlineto{\pgfqpoint{3.438896in}{1.120697in}}%
\pgfpathlineto{\pgfqpoint{3.440287in}{1.125896in}}%
\pgfpathlineto{\pgfqpoint{3.440305in}{1.125873in}}%
\pgfpathlineto{\pgfqpoint{3.440840in}{1.120056in}}%
\pgfpathlineto{\pgfqpoint{3.441929in}{1.117030in}}%
\pgfpathlineto{\pgfqpoint{3.441376in}{1.121329in}}%
\pgfpathlineto{\pgfqpoint{3.442125in}{1.118444in}}%
\pgfpathlineto{\pgfqpoint{3.443641in}{1.125972in}}%
\pgfpathlineto{\pgfqpoint{3.443766in}{1.125558in}}%
\pgfpathlineto{\pgfqpoint{3.445301in}{1.117179in}}%
\pgfpathlineto{\pgfqpoint{3.445639in}{1.120727in}}%
\pgfpathlineto{\pgfqpoint{3.447031in}{1.125949in}}%
\pgfpathlineto{\pgfqpoint{3.447085in}{1.125841in}}%
\pgfpathlineto{\pgfqpoint{3.448655in}{1.117102in}}%
\pgfpathlineto{\pgfqpoint{3.449101in}{1.121871in}}%
\pgfpathlineto{\pgfqpoint{3.450403in}{1.126089in}}%
\pgfpathlineto{\pgfqpoint{3.450510in}{1.125763in}}%
\pgfpathlineto{\pgfqpoint{3.452062in}{1.117248in}}%
\pgfpathlineto{\pgfqpoint{3.452383in}{1.120655in}}%
\pgfpathlineto{\pgfqpoint{3.453793in}{1.125923in}}%
\pgfpathlineto{\pgfqpoint{3.453828in}{1.125839in}}%
\pgfpathlineto{\pgfqpoint{3.454239in}{1.121188in}}%
\pgfpathlineto{\pgfqpoint{3.455416in}{1.117183in}}%
\pgfpathlineto{\pgfqpoint{3.454845in}{1.121239in}}%
\pgfpathlineto{\pgfqpoint{3.455595in}{1.118246in}}%
\pgfpathlineto{\pgfqpoint{3.457147in}{1.125759in}}%
\pgfpathlineto{\pgfqpoint{3.457289in}{1.125235in}}%
\pgfpathlineto{\pgfqpoint{3.458788in}{1.117328in}}%
\pgfpathlineto{\pgfqpoint{3.459074in}{1.119817in}}%
\pgfpathlineto{\pgfqpoint{3.460501in}{1.126080in}}%
\pgfpathlineto{\pgfqpoint{3.460572in}{1.125978in}}%
\pgfpathlineto{\pgfqpoint{3.462178in}{1.117368in}}%
\pgfpathlineto{\pgfqpoint{3.462553in}{1.121324in}}%
\pgfpathlineto{\pgfqpoint{3.463908in}{1.125743in}}%
\pgfpathlineto{\pgfqpoint{3.463962in}{1.125603in}}%
\pgfpathlineto{\pgfqpoint{3.465550in}{1.117347in}}%
\pgfpathlineto{\pgfqpoint{3.465871in}{1.120461in}}%
\pgfpathlineto{\pgfqpoint{3.467280in}{1.125795in}}%
\pgfpathlineto{\pgfqpoint{3.467316in}{1.125716in}}%
\pgfpathlineto{\pgfqpoint{3.467744in}{1.121066in}}%
\pgfpathlineto{\pgfqpoint{3.468940in}{1.117408in}}%
\pgfpathlineto{\pgfqpoint{3.468386in}{1.121290in}}%
\pgfpathlineto{\pgfqpoint{3.469082in}{1.118241in}}%
\pgfpathlineto{\pgfqpoint{3.470634in}{1.125771in}}%
\pgfpathlineto{\pgfqpoint{3.470795in}{1.125182in}}%
\pgfpathlineto{\pgfqpoint{3.472294in}{1.117369in}}%
\pgfpathlineto{\pgfqpoint{3.472525in}{1.119048in}}%
\pgfpathlineto{\pgfqpoint{3.474024in}{1.125796in}}%
\pgfpathlineto{\pgfqpoint{3.474095in}{1.125662in}}%
\pgfpathlineto{\pgfqpoint{3.475665in}{1.117553in}}%
\pgfpathlineto{\pgfqpoint{3.476111in}{1.122196in}}%
\pgfpathlineto{\pgfqpoint{3.477360in}{1.125707in}}%
\pgfpathlineto{\pgfqpoint{3.477467in}{1.125498in}}%
\pgfpathlineto{\pgfqpoint{3.479055in}{1.117444in}}%
\pgfpathlineto{\pgfqpoint{3.479483in}{1.121985in}}%
\pgfpathlineto{\pgfqpoint{3.480768in}{1.125442in}}%
\pgfpathlineto{\pgfqpoint{3.480839in}{1.125268in}}%
\pgfpathlineto{\pgfqpoint{3.482427in}{1.117373in}}%
\pgfpathlineto{\pgfqpoint{3.482784in}{1.120929in}}%
\pgfpathlineto{\pgfqpoint{3.484122in}{1.125435in}}%
\pgfpathlineto{\pgfqpoint{3.484176in}{1.125380in}}%
\pgfpathlineto{\pgfqpoint{3.484675in}{1.120413in}}%
\pgfpathlineto{\pgfqpoint{3.485799in}{1.117195in}}%
\pgfpathlineto{\pgfqpoint{3.485264in}{1.121167in}}%
\pgfpathlineto{\pgfqpoint{3.485977in}{1.118283in}}%
\pgfpathlineto{\pgfqpoint{3.487530in}{1.125459in}}%
\pgfpathlineto{\pgfqpoint{3.487690in}{1.124715in}}%
\pgfpathlineto{\pgfqpoint{3.489153in}{1.117405in}}%
\pgfpathlineto{\pgfqpoint{3.489403in}{1.119173in}}%
\pgfpathlineto{\pgfqpoint{3.490901in}{1.125559in}}%
\pgfpathlineto{\pgfqpoint{3.490955in}{1.125436in}}%
\pgfpathlineto{\pgfqpoint{3.492543in}{1.117480in}}%
\pgfpathlineto{\pgfqpoint{3.492953in}{1.121701in}}%
\pgfpathlineto{\pgfqpoint{3.494273in}{1.125407in}}%
\pgfpathlineto{\pgfqpoint{3.494345in}{1.125238in}}%
\pgfpathlineto{\pgfqpoint{3.494755in}{1.120873in}}%
\pgfpathlineto{\pgfqpoint{3.495933in}{1.117368in}}%
\pgfpathlineto{\pgfqpoint{3.495397in}{1.121091in}}%
\pgfpathlineto{\pgfqpoint{3.496075in}{1.118171in}}%
\pgfpathlineto{\pgfqpoint{3.497645in}{1.125521in}}%
\pgfpathlineto{\pgfqpoint{3.497788in}{1.125038in}}%
\pgfpathlineto{\pgfqpoint{3.499304in}{1.117746in}}%
\pgfpathlineto{\pgfqpoint{3.499554in}{1.119748in}}%
\pgfpathlineto{\pgfqpoint{3.501017in}{1.125408in}}%
\pgfpathlineto{\pgfqpoint{3.501071in}{1.125304in}}%
\pgfpathlineto{\pgfqpoint{3.502694in}{1.117667in}}%
\pgfpathlineto{\pgfqpoint{3.503087in}{1.121825in}}%
\pgfpathlineto{\pgfqpoint{3.504353in}{1.125538in}}%
\pgfpathlineto{\pgfqpoint{3.504478in}{1.125333in}}%
\pgfpathlineto{\pgfqpoint{3.506048in}{1.117679in}}%
\pgfpathlineto{\pgfqpoint{3.506405in}{1.121105in}}%
\pgfpathlineto{\pgfqpoint{3.507761in}{1.125495in}}%
\pgfpathlineto{\pgfqpoint{3.507797in}{1.125478in}}%
\pgfpathlineto{\pgfqpoint{3.508100in}{1.122997in}}%
\pgfpathlineto{\pgfqpoint{3.509438in}{1.117866in}}%
\pgfpathlineto{\pgfqpoint{3.509527in}{1.118199in}}%
\pgfpathlineto{\pgfqpoint{3.511133in}{1.125353in}}%
\pgfpathlineto{\pgfqpoint{3.511418in}{1.123559in}}%
\pgfpathlineto{\pgfqpoint{3.512792in}{1.117682in}}%
\pgfpathlineto{\pgfqpoint{3.512917in}{1.118133in}}%
\pgfpathlineto{\pgfqpoint{3.514505in}{1.125610in}}%
\pgfpathlineto{\pgfqpoint{3.514755in}{1.124128in}}%
\pgfpathlineto{\pgfqpoint{3.516164in}{1.117748in}}%
\pgfpathlineto{\pgfqpoint{3.516378in}{1.119024in}}%
\pgfpathlineto{\pgfqpoint{3.517877in}{1.125333in}}%
\pgfpathlineto{\pgfqpoint{3.517966in}{1.125122in}}%
\pgfpathlineto{\pgfqpoint{3.519572in}{1.117853in}}%
\pgfpathlineto{\pgfqpoint{3.519946in}{1.121706in}}%
\pgfpathlineto{\pgfqpoint{3.521249in}{1.125240in}}%
\pgfpathlineto{\pgfqpoint{3.521302in}{1.125183in}}%
\pgfpathlineto{\pgfqpoint{3.521766in}{1.120802in}}%
\pgfpathlineto{\pgfqpoint{3.522943in}{1.117657in}}%
\pgfpathlineto{\pgfqpoint{3.522408in}{1.121216in}}%
\pgfpathlineto{\pgfqpoint{3.523086in}{1.118460in}}%
\pgfpathlineto{\pgfqpoint{3.524638in}{1.125296in}}%
\pgfpathlineto{\pgfqpoint{3.524781in}{1.124802in}}%
\pgfpathlineto{\pgfqpoint{3.526333in}{1.117892in}}%
\pgfpathlineto{\pgfqpoint{3.526565in}{1.119879in}}%
\pgfpathlineto{\pgfqpoint{3.527992in}{1.125065in}}%
\pgfpathlineto{\pgfqpoint{3.528064in}{1.124983in}}%
\pgfpathlineto{\pgfqpoint{3.529687in}{1.117670in}}%
\pgfpathlineto{\pgfqpoint{3.530098in}{1.121956in}}%
\pgfpathlineto{\pgfqpoint{3.531400in}{1.125123in}}%
\pgfpathlineto{\pgfqpoint{3.531418in}{1.125080in}}%
\pgfpathlineto{\pgfqpoint{3.533041in}{1.117870in}}%
\pgfpathlineto{\pgfqpoint{3.533470in}{1.122116in}}%
\pgfpathlineto{\pgfqpoint{3.534754in}{1.125221in}}%
\pgfpathlineto{\pgfqpoint{3.534825in}{1.125122in}}%
\pgfpathlineto{\pgfqpoint{3.536449in}{1.117894in}}%
\pgfpathlineto{\pgfqpoint{3.536752in}{1.120727in}}%
\pgfpathlineto{\pgfqpoint{3.538126in}{1.125224in}}%
\pgfpathlineto{\pgfqpoint{3.538162in}{1.125185in}}%
\pgfpathlineto{\pgfqpoint{3.538376in}{1.123921in}}%
\pgfpathlineto{\pgfqpoint{3.539803in}{1.117875in}}%
\pgfpathlineto{\pgfqpoint{3.539981in}{1.118913in}}%
\pgfpathlineto{\pgfqpoint{3.541498in}{1.125354in}}%
\pgfpathlineto{\pgfqpoint{3.541641in}{1.124913in}}%
\pgfpathlineto{\pgfqpoint{3.543193in}{1.118121in}}%
\pgfpathlineto{\pgfqpoint{3.543442in}{1.120020in}}%
\pgfpathlineto{\pgfqpoint{3.544888in}{1.125034in}}%
\pgfpathlineto{\pgfqpoint{3.544923in}{1.124988in}}%
\pgfpathlineto{\pgfqpoint{3.545227in}{1.122658in}}%
\pgfpathlineto{\pgfqpoint{3.546565in}{1.118131in}}%
\pgfpathlineto{\pgfqpoint{3.546690in}{1.118656in}}%
\pgfpathlineto{\pgfqpoint{3.548259in}{1.125106in}}%
\pgfpathlineto{\pgfqpoint{3.548438in}{1.124285in}}%
\pgfpathlineto{\pgfqpoint{3.549954in}{1.117944in}}%
\pgfpathlineto{\pgfqpoint{3.550151in}{1.119357in}}%
\pgfpathlineto{\pgfqpoint{3.551596in}{1.125000in}}%
\pgfpathlineto{\pgfqpoint{3.551721in}{1.124792in}}%
\pgfpathlineto{\pgfqpoint{3.553326in}{1.117820in}}%
\pgfpathlineto{\pgfqpoint{3.553701in}{1.121555in}}%
\pgfpathlineto{\pgfqpoint{3.555003in}{1.124808in}}%
\pgfpathlineto{\pgfqpoint{3.555057in}{1.124747in}}%
\pgfpathlineto{\pgfqpoint{3.556680in}{1.118001in}}%
\pgfpathlineto{\pgfqpoint{3.556984in}{1.120532in}}%
\pgfpathlineto{\pgfqpoint{3.558340in}{1.125104in}}%
\pgfpathlineto{\pgfqpoint{3.558393in}{1.125052in}}%
\pgfpathlineto{\pgfqpoint{3.558678in}{1.123157in}}%
\pgfpathlineto{\pgfqpoint{3.560088in}{1.118008in}}%
\pgfpathlineto{\pgfqpoint{3.560177in}{1.118436in}}%
\pgfpathlineto{\pgfqpoint{3.561747in}{1.124924in}}%
\pgfpathlineto{\pgfqpoint{3.561997in}{1.123671in}}%
\pgfpathlineto{\pgfqpoint{3.563424in}{1.118058in}}%
\pgfpathlineto{\pgfqpoint{3.563603in}{1.118921in}}%
\pgfpathlineto{\pgfqpoint{3.565119in}{1.124842in}}%
\pgfpathlineto{\pgfqpoint{3.565262in}{1.124428in}}%
\pgfpathlineto{\pgfqpoint{3.566796in}{1.118018in}}%
\pgfpathlineto{\pgfqpoint{3.567099in}{1.120452in}}%
\pgfpathlineto{\pgfqpoint{3.568473in}{1.125032in}}%
\pgfpathlineto{\pgfqpoint{3.568544in}{1.124949in}}%
\pgfpathlineto{\pgfqpoint{3.568883in}{1.122411in}}%
\pgfpathlineto{\pgfqpoint{3.570186in}{1.118120in}}%
\pgfpathlineto{\pgfqpoint{3.570293in}{1.118509in}}%
\pgfpathlineto{\pgfqpoint{3.571898in}{1.124921in}}%
\pgfpathlineto{\pgfqpoint{3.572130in}{1.123679in}}%
\pgfpathlineto{\pgfqpoint{3.573576in}{1.118421in}}%
\pgfpathlineto{\pgfqpoint{3.573736in}{1.119334in}}%
\pgfpathlineto{\pgfqpoint{3.575270in}{1.124922in}}%
\pgfpathlineto{\pgfqpoint{3.575306in}{1.124823in}}%
\pgfpathlineto{\pgfqpoint{3.575663in}{1.121901in}}%
\pgfpathlineto{\pgfqpoint{3.576947in}{1.118295in}}%
\pgfpathlineto{\pgfqpoint{3.577054in}{1.118727in}}%
\pgfpathlineto{\pgfqpoint{3.578642in}{1.124993in}}%
\pgfpathlineto{\pgfqpoint{3.578803in}{1.124340in}}%
\pgfpathlineto{\pgfqpoint{3.580301in}{1.118265in}}%
\pgfpathlineto{\pgfqpoint{3.580551in}{1.119912in}}%
\pgfpathlineto{\pgfqpoint{3.581996in}{1.124893in}}%
\pgfpathlineto{\pgfqpoint{3.582050in}{1.124850in}}%
\pgfpathlineto{\pgfqpoint{3.582496in}{1.120971in}}%
\pgfpathlineto{\pgfqpoint{3.583673in}{1.118239in}}%
\pgfpathlineto{\pgfqpoint{3.583192in}{1.121094in}}%
\pgfpathlineto{\pgfqpoint{3.583816in}{1.118742in}}%
\pgfpathlineto{\pgfqpoint{3.585386in}{1.124896in}}%
\pgfpathlineto{\pgfqpoint{3.585529in}{1.124375in}}%
\pgfpathlineto{\pgfqpoint{3.587081in}{1.118285in}}%
\pgfpathlineto{\pgfqpoint{3.587259in}{1.119368in}}%
\pgfpathlineto{\pgfqpoint{3.588704in}{1.124860in}}%
\pgfpathlineto{\pgfqpoint{3.588865in}{1.124517in}}%
\pgfpathlineto{\pgfqpoint{3.590453in}{1.118256in}}%
\pgfpathlineto{\pgfqpoint{3.590667in}{1.119678in}}%
\pgfpathlineto{\pgfqpoint{3.592112in}{1.124747in}}%
\pgfpathlineto{\pgfqpoint{3.592183in}{1.124652in}}%
\pgfpathlineto{\pgfqpoint{3.592398in}{1.123234in}}%
\pgfpathlineto{\pgfqpoint{3.593825in}{1.118172in}}%
\pgfpathlineto{\pgfqpoint{3.593967in}{1.118858in}}%
\pgfpathlineto{\pgfqpoint{3.595484in}{1.124756in}}%
\pgfpathlineto{\pgfqpoint{3.595609in}{1.124414in}}%
\pgfpathlineto{\pgfqpoint{3.597197in}{1.118248in}}%
\pgfpathlineto{\pgfqpoint{3.597500in}{1.120785in}}%
\pgfpathlineto{\pgfqpoint{3.598820in}{1.124599in}}%
\pgfpathlineto{\pgfqpoint{3.598909in}{1.124542in}}%
\pgfpathlineto{\pgfqpoint{3.599159in}{1.123048in}}%
\pgfpathlineto{\pgfqpoint{3.600551in}{1.118273in}}%
\pgfpathlineto{\pgfqpoint{3.600693in}{1.118738in}}%
\pgfpathlineto{\pgfqpoint{3.602210in}{1.124724in}}%
\pgfpathlineto{\pgfqpoint{3.602424in}{1.124076in}}%
\pgfpathlineto{\pgfqpoint{3.603940in}{1.118353in}}%
\pgfpathlineto{\pgfqpoint{3.604155in}{1.119655in}}%
\pgfpathlineto{\pgfqpoint{3.605582in}{1.124660in}}%
\pgfpathlineto{\pgfqpoint{3.605689in}{1.124544in}}%
\pgfpathlineto{\pgfqpoint{3.607330in}{1.118353in}}%
\pgfpathlineto{\pgfqpoint{3.607723in}{1.122046in}}%
\pgfpathlineto{\pgfqpoint{3.609007in}{1.124646in}}%
\pgfpathlineto{\pgfqpoint{3.609061in}{1.124549in}}%
\pgfpathlineto{\pgfqpoint{3.610702in}{1.118336in}}%
\pgfpathlineto{\pgfqpoint{3.611059in}{1.121658in}}%
\pgfpathlineto{\pgfqpoint{3.612361in}{1.124597in}}%
\pgfpathlineto{\pgfqpoint{3.612397in}{1.124579in}}%
\pgfpathlineto{\pgfqpoint{3.612629in}{1.123330in}}%
\pgfpathlineto{\pgfqpoint{3.614092in}{1.118378in}}%
\pgfpathlineto{\pgfqpoint{3.614252in}{1.119352in}}%
\pgfpathlineto{\pgfqpoint{3.615769in}{1.124546in}}%
\pgfpathlineto{\pgfqpoint{3.615822in}{1.124419in}}%
\pgfpathlineto{\pgfqpoint{3.616197in}{1.121466in}}%
\pgfpathlineto{\pgfqpoint{3.617428in}{1.118529in}}%
\pgfpathlineto{\pgfqpoint{3.617571in}{1.119014in}}%
\pgfpathlineto{\pgfqpoint{3.619123in}{1.124510in}}%
\pgfpathlineto{\pgfqpoint{3.619355in}{1.123484in}}%
\pgfpathlineto{\pgfqpoint{3.620818in}{1.118555in}}%
\pgfpathlineto{\pgfqpoint{3.620961in}{1.119153in}}%
\pgfpathlineto{\pgfqpoint{3.622495in}{1.124696in}}%
\pgfpathlineto{\pgfqpoint{3.622673in}{1.124060in}}%
\pgfpathlineto{\pgfqpoint{3.624190in}{1.118537in}}%
\pgfpathlineto{\pgfqpoint{3.624386in}{1.119542in}}%
\pgfpathlineto{\pgfqpoint{3.625867in}{1.124671in}}%
\pgfpathlineto{\pgfqpoint{3.625938in}{1.124534in}}%
\pgfpathlineto{\pgfqpoint{3.627579in}{1.118578in}}%
\pgfpathlineto{\pgfqpoint{3.627972in}{1.122018in}}%
\pgfpathlineto{\pgfqpoint{3.629221in}{1.124557in}}%
\pgfpathlineto{\pgfqpoint{3.629256in}{1.124537in}}%
\pgfpathlineto{\pgfqpoint{3.629667in}{1.121690in}}%
\pgfpathlineto{\pgfqpoint{3.630951in}{1.118658in}}%
\pgfpathlineto{\pgfqpoint{3.631041in}{1.118880in}}%
\pgfpathlineto{\pgfqpoint{3.632611in}{1.124569in}}%
\pgfpathlineto{\pgfqpoint{3.632932in}{1.122786in}}%
\pgfpathlineto{\pgfqpoint{3.634341in}{1.118571in}}%
\pgfpathlineto{\pgfqpoint{3.634430in}{1.118879in}}%
\pgfpathlineto{\pgfqpoint{3.635965in}{1.124509in}}%
\pgfpathlineto{\pgfqpoint{3.636214in}{1.123592in}}%
\pgfpathlineto{\pgfqpoint{3.637731in}{1.118684in}}%
\pgfpathlineto{\pgfqpoint{3.637891in}{1.119732in}}%
\pgfpathlineto{\pgfqpoint{3.639354in}{1.124575in}}%
\pgfpathlineto{\pgfqpoint{3.639444in}{1.124435in}}%
\pgfpathlineto{\pgfqpoint{3.641067in}{1.118654in}}%
\pgfpathlineto{\pgfqpoint{3.641370in}{1.120890in}}%
\pgfpathlineto{\pgfqpoint{3.642744in}{1.124447in}}%
\pgfpathlineto{\pgfqpoint{3.642762in}{1.124437in}}%
\pgfpathlineto{\pgfqpoint{3.644439in}{1.118635in}}%
\pgfpathlineto{\pgfqpoint{3.644831in}{1.121931in}}%
\pgfpathlineto{\pgfqpoint{3.646080in}{1.124525in}}%
\pgfpathlineto{\pgfqpoint{3.646134in}{1.124496in}}%
\pgfpathlineto{\pgfqpoint{3.646491in}{1.122207in}}%
\pgfpathlineto{\pgfqpoint{3.647811in}{1.118678in}}%
\pgfpathlineto{\pgfqpoint{3.647954in}{1.119228in}}%
\pgfpathlineto{\pgfqpoint{3.649470in}{1.124516in}}%
\pgfpathlineto{\pgfqpoint{3.649702in}{1.123655in}}%
\pgfpathlineto{\pgfqpoint{3.651201in}{1.118680in}}%
\pgfpathlineto{\pgfqpoint{3.651361in}{1.119416in}}%
\pgfpathlineto{\pgfqpoint{3.652860in}{1.124385in}}%
\pgfpathlineto{\pgfqpoint{3.652985in}{1.124072in}}%
\pgfpathlineto{\pgfqpoint{3.654555in}{1.118734in}}%
\pgfpathlineto{\pgfqpoint{3.654804in}{1.120215in}}%
\pgfpathlineto{\pgfqpoint{3.656232in}{1.124506in}}%
\pgfpathlineto{\pgfqpoint{3.656267in}{1.124456in}}%
\pgfpathlineto{\pgfqpoint{3.657944in}{1.118705in}}%
\pgfpathlineto{\pgfqpoint{3.658266in}{1.121043in}}%
\pgfpathlineto{\pgfqpoint{3.659586in}{1.124359in}}%
\pgfpathlineto{\pgfqpoint{3.659657in}{1.124315in}}%
\pgfpathlineto{\pgfqpoint{3.660139in}{1.120855in}}%
\pgfpathlineto{\pgfqpoint{3.661334in}{1.118759in}}%
\pgfpathlineto{\pgfqpoint{3.660835in}{1.121091in}}%
\pgfpathlineto{\pgfqpoint{3.661441in}{1.119115in}}%
\pgfpathlineto{\pgfqpoint{3.662993in}{1.124393in}}%
\pgfpathlineto{\pgfqpoint{3.663207in}{1.123464in}}%
\pgfpathlineto{\pgfqpoint{3.664688in}{1.118870in}}%
\pgfpathlineto{\pgfqpoint{3.664813in}{1.119242in}}%
\pgfpathlineto{\pgfqpoint{3.666365in}{1.124433in}}%
\pgfpathlineto{\pgfqpoint{3.666597in}{1.123392in}}%
\pgfpathlineto{\pgfqpoint{3.668096in}{1.118758in}}%
\pgfpathlineto{\pgfqpoint{3.668203in}{1.119158in}}%
\pgfpathlineto{\pgfqpoint{3.669755in}{1.124258in}}%
\pgfpathlineto{\pgfqpoint{3.669951in}{1.123379in}}%
\pgfpathlineto{\pgfqpoint{3.671450in}{1.118896in}}%
\pgfpathlineto{\pgfqpoint{3.671575in}{1.119327in}}%
\pgfpathlineto{\pgfqpoint{3.673109in}{1.124362in}}%
\pgfpathlineto{\pgfqpoint{3.673359in}{1.123174in}}%
\pgfpathlineto{\pgfqpoint{3.674840in}{1.118766in}}%
\pgfpathlineto{\pgfqpoint{3.675036in}{1.119882in}}%
\pgfpathlineto{\pgfqpoint{3.676481in}{1.124199in}}%
\pgfpathlineto{\pgfqpoint{3.676606in}{1.123874in}}%
\pgfpathlineto{\pgfqpoint{3.678212in}{1.118794in}}%
\pgfpathlineto{\pgfqpoint{3.678497in}{1.120828in}}%
\pgfpathlineto{\pgfqpoint{3.679853in}{1.124280in}}%
\pgfpathlineto{\pgfqpoint{3.679889in}{1.124242in}}%
\pgfpathlineto{\pgfqpoint{3.680352in}{1.121058in}}%
\pgfpathlineto{\pgfqpoint{3.681583in}{1.118898in}}%
\pgfpathlineto{\pgfqpoint{3.681066in}{1.121183in}}%
\pgfpathlineto{\pgfqpoint{3.681673in}{1.119126in}}%
\pgfpathlineto{\pgfqpoint{3.683207in}{1.124238in}}%
\pgfpathlineto{\pgfqpoint{3.683439in}{1.123434in}}%
\pgfpathlineto{\pgfqpoint{3.684937in}{1.118904in}}%
\pgfpathlineto{\pgfqpoint{3.685169in}{1.120000in}}%
\pgfpathlineto{\pgfqpoint{3.686561in}{1.124198in}}%
\pgfpathlineto{\pgfqpoint{3.686704in}{1.124012in}}%
\pgfpathlineto{\pgfqpoint{3.688345in}{1.118834in}}%
\pgfpathlineto{\pgfqpoint{3.688613in}{1.120787in}}%
\pgfpathlineto{\pgfqpoint{3.689933in}{1.124265in}}%
\pgfpathlineto{\pgfqpoint{3.689986in}{1.124248in}}%
\pgfpathlineto{\pgfqpoint{3.690272in}{1.122863in}}%
\pgfpathlineto{\pgfqpoint{3.691699in}{1.118952in}}%
\pgfpathlineto{\pgfqpoint{3.691788in}{1.119143in}}%
\pgfpathlineto{\pgfqpoint{3.693340in}{1.124133in}}%
\pgfpathlineto{\pgfqpoint{3.693572in}{1.123375in}}%
\pgfpathlineto{\pgfqpoint{3.695089in}{1.119099in}}%
\pgfpathlineto{\pgfqpoint{3.695196in}{1.119466in}}%
\pgfpathlineto{\pgfqpoint{3.696695in}{1.124356in}}%
\pgfpathlineto{\pgfqpoint{3.696962in}{1.123471in}}%
\pgfpathlineto{\pgfqpoint{3.698479in}{1.119059in}}%
\pgfpathlineto{\pgfqpoint{3.698639in}{1.119934in}}%
\pgfpathlineto{\pgfqpoint{3.700084in}{1.124239in}}%
\pgfpathlineto{\pgfqpoint{3.700245in}{1.123908in}}%
\pgfpathlineto{\pgfqpoint{3.701815in}{1.119058in}}%
\pgfpathlineto{\pgfqpoint{3.702118in}{1.120973in}}%
\pgfpathlineto{\pgfqpoint{3.703510in}{1.124281in}}%
\pgfpathlineto{\pgfqpoint{3.703813in}{1.122618in}}%
\pgfpathlineto{\pgfqpoint{3.705205in}{1.119080in}}%
\pgfpathlineto{\pgfqpoint{3.705276in}{1.119223in}}%
\pgfpathlineto{\pgfqpoint{3.706828in}{1.124208in}}%
\pgfpathlineto{\pgfqpoint{3.707238in}{1.121983in}}%
\pgfpathlineto{\pgfqpoint{3.708576in}{1.119159in}}%
\pgfpathlineto{\pgfqpoint{3.708666in}{1.119322in}}%
\pgfpathlineto{\pgfqpoint{3.710236in}{1.124208in}}%
\pgfpathlineto{\pgfqpoint{3.710468in}{1.123210in}}%
\pgfpathlineto{\pgfqpoint{3.711966in}{1.119174in}}%
\pgfpathlineto{\pgfqpoint{3.712055in}{1.119395in}}%
\pgfpathlineto{\pgfqpoint{3.713625in}{1.124243in}}%
\pgfpathlineto{\pgfqpoint{3.713947in}{1.122419in}}%
\pgfpathlineto{\pgfqpoint{3.715338in}{1.119069in}}%
\pgfpathlineto{\pgfqpoint{3.715445in}{1.119377in}}%
\pgfpathlineto{\pgfqpoint{3.716979in}{1.124019in}}%
\pgfpathlineto{\pgfqpoint{3.717194in}{1.123214in}}%
\pgfpathlineto{\pgfqpoint{3.718728in}{1.119029in}}%
\pgfpathlineto{\pgfqpoint{3.718888in}{1.119827in}}%
\pgfpathlineto{\pgfqpoint{3.720334in}{1.124094in}}%
\pgfpathlineto{\pgfqpoint{3.720441in}{1.123896in}}%
\pgfpathlineto{\pgfqpoint{3.722082in}{1.119178in}}%
\pgfpathlineto{\pgfqpoint{3.722350in}{1.120763in}}%
\pgfpathlineto{\pgfqpoint{3.723723in}{1.124171in}}%
\pgfpathlineto{\pgfqpoint{3.723741in}{1.124165in}}%
\pgfpathlineto{\pgfqpoint{3.723991in}{1.123048in}}%
\pgfpathlineto{\pgfqpoint{3.725472in}{1.118967in}}%
\pgfpathlineto{\pgfqpoint{3.725561in}{1.119233in}}%
\pgfpathlineto{\pgfqpoint{3.727077in}{1.123848in}}%
\pgfpathlineto{\pgfqpoint{3.727381in}{1.122603in}}%
\pgfpathlineto{\pgfqpoint{3.728826in}{1.118940in}}%
\pgfpathlineto{\pgfqpoint{3.728968in}{1.119468in}}%
\pgfpathlineto{\pgfqpoint{3.730503in}{1.124056in}}%
\pgfpathlineto{\pgfqpoint{3.730735in}{1.123003in}}%
\pgfpathlineto{\pgfqpoint{3.732215in}{1.119246in}}%
\pgfpathlineto{\pgfqpoint{3.732323in}{1.119537in}}%
\pgfpathlineto{\pgfqpoint{3.733839in}{1.124109in}}%
\pgfpathlineto{\pgfqpoint{3.734196in}{1.122380in}}%
\pgfpathlineto{\pgfqpoint{3.735605in}{1.119289in}}%
\pgfpathlineto{\pgfqpoint{3.735677in}{1.119475in}}%
\pgfpathlineto{\pgfqpoint{3.737229in}{1.123937in}}%
\pgfpathlineto{\pgfqpoint{3.737532in}{1.122505in}}%
\pgfpathlineto{\pgfqpoint{3.738959in}{1.119223in}}%
\pgfpathlineto{\pgfqpoint{3.739084in}{1.119590in}}%
\pgfpathlineto{\pgfqpoint{3.740601in}{1.124070in}}%
\pgfpathlineto{\pgfqpoint{3.740743in}{1.123707in}}%
\pgfpathlineto{\pgfqpoint{3.742349in}{1.119199in}}%
\pgfpathlineto{\pgfqpoint{3.742599in}{1.120712in}}%
\pgfpathlineto{\pgfqpoint{3.743973in}{1.123981in}}%
\pgfpathlineto{\pgfqpoint{3.744008in}{1.123945in}}%
\pgfpathlineto{\pgfqpoint{3.744294in}{1.122482in}}%
\pgfpathlineto{\pgfqpoint{3.745703in}{1.119317in}}%
\pgfpathlineto{\pgfqpoint{3.745792in}{1.119485in}}%
\pgfpathlineto{\pgfqpoint{3.747344in}{1.123975in}}%
\pgfpathlineto{\pgfqpoint{3.747630in}{1.122702in}}%
\pgfpathlineto{\pgfqpoint{3.749093in}{1.119263in}}%
\pgfpathlineto{\pgfqpoint{3.749146in}{1.119346in}}%
\pgfpathlineto{\pgfqpoint{3.749414in}{1.121372in}}%
\pgfpathlineto{\pgfqpoint{3.750734in}{1.124018in}}%
\pgfpathlineto{\pgfqpoint{3.750788in}{1.123963in}}%
\pgfpathlineto{\pgfqpoint{3.752447in}{1.119343in}}%
\pgfpathlineto{\pgfqpoint{3.752804in}{1.121578in}}%
\pgfpathlineto{\pgfqpoint{3.754088in}{1.123917in}}%
\pgfpathlineto{\pgfqpoint{3.754160in}{1.123814in}}%
\pgfpathlineto{\pgfqpoint{3.755854in}{1.119434in}}%
\pgfpathlineto{\pgfqpoint{3.756176in}{1.121593in}}%
\pgfpathlineto{\pgfqpoint{3.757478in}{1.123995in}}%
\pgfpathlineto{\pgfqpoint{3.757496in}{1.123987in}}%
\pgfpathlineto{\pgfqpoint{3.757924in}{1.121415in}}%
\pgfpathlineto{\pgfqpoint{3.759244in}{1.119320in}}%
\pgfpathlineto{\pgfqpoint{3.759262in}{1.119366in}}%
\pgfpathlineto{\pgfqpoint{3.760850in}{1.123948in}}%
\pgfpathlineto{\pgfqpoint{3.761135in}{1.122682in}}%
\pgfpathlineto{\pgfqpoint{3.762580in}{1.119444in}}%
\pgfpathlineto{\pgfqpoint{3.762616in}{1.119483in}}%
\pgfpathlineto{\pgfqpoint{3.762884in}{1.121253in}}%
\pgfpathlineto{\pgfqpoint{3.764222in}{1.124002in}}%
\pgfpathlineto{\pgfqpoint{3.764257in}{1.123943in}}%
\pgfpathlineto{\pgfqpoint{3.764561in}{1.122355in}}%
\pgfpathlineto{\pgfqpoint{3.765952in}{1.119364in}}%
\pgfpathlineto{\pgfqpoint{3.766202in}{1.120602in}}%
\pgfpathlineto{\pgfqpoint{3.767594in}{1.124061in}}%
\pgfpathlineto{\pgfqpoint{3.767629in}{1.124019in}}%
\pgfpathlineto{\pgfqpoint{3.768040in}{1.121608in}}%
\pgfpathlineto{\pgfqpoint{3.769342in}{1.119434in}}%
\pgfpathlineto{\pgfqpoint{3.769396in}{1.119494in}}%
\pgfpathlineto{\pgfqpoint{3.771001in}{1.123986in}}%
\pgfpathlineto{\pgfqpoint{3.771287in}{1.122625in}}%
\pgfpathlineto{\pgfqpoint{3.772714in}{1.119508in}}%
\pgfpathlineto{\pgfqpoint{3.772768in}{1.119625in}}%
\pgfpathlineto{\pgfqpoint{3.774373in}{1.123939in}}%
\pgfpathlineto{\pgfqpoint{3.774766in}{1.121649in}}%
\pgfpathlineto{\pgfqpoint{3.776068in}{1.119415in}}%
\pgfpathlineto{\pgfqpoint{3.776157in}{1.119533in}}%
\pgfpathlineto{\pgfqpoint{3.777692in}{1.123975in}}%
\pgfpathlineto{\pgfqpoint{3.778048in}{1.122433in}}%
\pgfpathlineto{\pgfqpoint{3.779458in}{1.119475in}}%
\pgfpathlineto{\pgfqpoint{3.779529in}{1.119559in}}%
\pgfpathlineto{\pgfqpoint{3.779850in}{1.122108in}}%
\pgfpathlineto{\pgfqpoint{3.781117in}{1.123797in}}%
\pgfpathlineto{\pgfqpoint{3.781153in}{1.123733in}}%
\pgfpathlineto{\pgfqpoint{3.781634in}{1.120794in}}%
\pgfpathlineto{\pgfqpoint{3.782830in}{1.119560in}}%
\pgfpathlineto{\pgfqpoint{3.782366in}{1.121150in}}%
\pgfpathlineto{\pgfqpoint{3.782937in}{1.119756in}}%
\pgfpathlineto{\pgfqpoint{3.784489in}{1.123728in}}%
\pgfpathlineto{\pgfqpoint{3.784685in}{1.123034in}}%
\pgfpathlineto{\pgfqpoint{3.786202in}{1.119469in}}%
\pgfpathlineto{\pgfqpoint{3.786326in}{1.119761in}}%
\pgfpathlineto{\pgfqpoint{3.787861in}{1.123932in}}%
\pgfpathlineto{\pgfqpoint{3.788075in}{1.123138in}}%
\pgfpathlineto{\pgfqpoint{3.789591in}{1.119430in}}%
\pgfpathlineto{\pgfqpoint{3.789698in}{1.119645in}}%
\pgfpathlineto{\pgfqpoint{3.791233in}{1.123658in}}%
\pgfpathlineto{\pgfqpoint{3.791572in}{1.122051in}}%
\pgfpathlineto{\pgfqpoint{3.792981in}{1.119471in}}%
\pgfpathlineto{\pgfqpoint{3.793017in}{1.119536in}}%
\pgfpathlineto{\pgfqpoint{3.793374in}{1.122210in}}%
\pgfpathlineto{\pgfqpoint{3.794605in}{1.123828in}}%
\pgfpathlineto{\pgfqpoint{3.794640in}{1.123776in}}%
\pgfpathlineto{\pgfqpoint{3.795104in}{1.120888in}}%
\pgfpathlineto{\pgfqpoint{3.796353in}{1.119431in}}%
\pgfpathlineto{\pgfqpoint{3.796442in}{1.119619in}}%
\pgfpathlineto{\pgfqpoint{3.797994in}{1.123892in}}%
\pgfpathlineto{\pgfqpoint{3.798208in}{1.123016in}}%
\pgfpathlineto{\pgfqpoint{3.799761in}{1.119648in}}%
\pgfpathlineto{\pgfqpoint{3.799814in}{1.119827in}}%
\pgfpathlineto{\pgfqpoint{3.801331in}{1.123769in}}%
\pgfpathlineto{\pgfqpoint{3.801616in}{1.122786in}}%
\pgfpathlineto{\pgfqpoint{3.803079in}{1.119602in}}%
\pgfpathlineto{\pgfqpoint{3.803150in}{1.119662in}}%
\pgfpathlineto{\pgfqpoint{3.803578in}{1.123092in}}%
\pgfpathlineto{\pgfqpoint{3.804720in}{1.123845in}}%
\pgfpathlineto{\pgfqpoint{3.804203in}{1.122524in}}%
\pgfpathlineto{\pgfqpoint{3.804792in}{1.123761in}}%
\pgfpathlineto{\pgfqpoint{3.806487in}{1.119624in}}%
\pgfpathlineto{\pgfqpoint{3.806790in}{1.121510in}}%
\pgfpathlineto{\pgfqpoint{3.807129in}{1.123731in}}%
\pgfpathlineto{\pgfqpoint{3.808110in}{1.123609in}}%
\pgfpathlineto{\pgfqpoint{3.808413in}{1.122282in}}%
\pgfpathlineto{\pgfqpoint{3.809841in}{1.119514in}}%
\pgfpathlineto{\pgfqpoint{3.809983in}{1.119911in}}%
\pgfpathlineto{\pgfqpoint{3.811446in}{1.123755in}}%
\pgfpathlineto{\pgfqpoint{3.811625in}{1.123344in}}%
\pgfpathlineto{\pgfqpoint{3.813212in}{1.119574in}}%
\pgfpathlineto{\pgfqpoint{3.813373in}{1.120076in}}%
\pgfpathlineto{\pgfqpoint{3.814872in}{1.123755in}}%
\pgfpathlineto{\pgfqpoint{3.814961in}{1.123565in}}%
\pgfpathlineto{\pgfqpoint{3.816602in}{1.119658in}}%
\pgfpathlineto{\pgfqpoint{3.816816in}{1.120593in}}%
\pgfpathlineto{\pgfqpoint{3.818226in}{1.123683in}}%
\pgfpathlineto{\pgfqpoint{3.818261in}{1.123648in}}%
\pgfpathlineto{\pgfqpoint{3.818583in}{1.122048in}}%
\pgfpathlineto{\pgfqpoint{3.819956in}{1.119741in}}%
\pgfpathlineto{\pgfqpoint{3.820063in}{1.119880in}}%
\pgfpathlineto{\pgfqpoint{3.820206in}{1.120774in}}%
\pgfpathlineto{\pgfqpoint{3.821580in}{1.123897in}}%
\pgfpathlineto{\pgfqpoint{3.821633in}{1.123853in}}%
\pgfpathlineto{\pgfqpoint{3.821883in}{1.122766in}}%
\pgfpathlineto{\pgfqpoint{3.823364in}{1.119769in}}%
\pgfpathlineto{\pgfqpoint{3.823400in}{1.119831in}}%
\pgfpathlineto{\pgfqpoint{3.824970in}{1.123722in}}%
\pgfpathlineto{\pgfqpoint{3.825362in}{1.121855in}}%
\pgfpathlineto{\pgfqpoint{3.826736in}{1.119958in}}%
\pgfpathlineto{\pgfqpoint{3.827200in}{1.123088in}}%
\pgfpathlineto{\pgfqpoint{3.828341in}{1.123832in}}%
\pgfpathlineto{\pgfqpoint{3.827842in}{1.122511in}}%
\pgfpathlineto{\pgfqpoint{3.828395in}{1.123771in}}%
\pgfpathlineto{\pgfqpoint{3.830126in}{1.119806in}}%
\pgfpathlineto{\pgfqpoint{3.830411in}{1.121490in}}%
\pgfpathlineto{\pgfqpoint{3.831713in}{1.123810in}}%
\pgfpathlineto{\pgfqpoint{3.831785in}{1.123762in}}%
\pgfpathlineto{\pgfqpoint{3.833480in}{1.119804in}}%
\pgfpathlineto{\pgfqpoint{3.833765in}{1.121301in}}%
\pgfpathlineto{\pgfqpoint{3.835121in}{1.123718in}}%
\pgfpathlineto{\pgfqpoint{3.835157in}{1.123675in}}%
\pgfpathlineto{\pgfqpoint{3.836851in}{1.119733in}}%
\pgfpathlineto{\pgfqpoint{3.837190in}{1.121787in}}%
\pgfpathlineto{\pgfqpoint{3.837547in}{1.123806in}}%
\pgfpathlineto{\pgfqpoint{3.838529in}{1.123627in}}%
\pgfpathlineto{\pgfqpoint{3.840241in}{1.119775in}}%
\pgfpathlineto{\pgfqpoint{3.840437in}{1.120606in}}%
\pgfpathlineto{\pgfqpoint{3.840937in}{1.123668in}}%
\pgfpathlineto{\pgfqpoint{3.841918in}{1.123438in}}%
\pgfpathlineto{\pgfqpoint{3.843577in}{1.119806in}}%
\pgfpathlineto{\pgfqpoint{3.843881in}{1.121216in}}%
\pgfpathlineto{\pgfqpoint{3.845219in}{1.123712in}}%
\pgfpathlineto{\pgfqpoint{3.845290in}{1.123613in}}%
\pgfpathlineto{\pgfqpoint{3.846949in}{1.119867in}}%
\pgfpathlineto{\pgfqpoint{3.847270in}{1.121438in}}%
\pgfpathlineto{\pgfqpoint{3.847663in}{1.123834in}}%
\pgfpathlineto{\pgfqpoint{3.848662in}{1.123583in}}%
\pgfpathlineto{\pgfqpoint{3.848983in}{1.121852in}}%
\pgfpathlineto{\pgfqpoint{3.850357in}{1.119879in}}%
\pgfpathlineto{\pgfqpoint{3.850732in}{1.122147in}}%
\pgfpathlineto{\pgfqpoint{3.851053in}{1.123681in}}%
\pgfpathlineto{\pgfqpoint{3.851963in}{1.123454in}}%
\pgfpathlineto{\pgfqpoint{3.852248in}{1.122442in}}%
\pgfpathlineto{\pgfqpoint{3.853711in}{1.119700in}}%
\pgfpathlineto{\pgfqpoint{3.853765in}{1.119705in}}%
\pgfpathlineto{\pgfqpoint{3.853979in}{1.120841in}}%
\pgfpathlineto{\pgfqpoint{3.855352in}{1.123494in}}%
\pgfpathlineto{\pgfqpoint{3.855745in}{1.121639in}}%
\pgfpathlineto{\pgfqpoint{3.857119in}{1.119772in}}%
\pgfpathlineto{\pgfqpoint{3.857136in}{1.119783in}}%
\pgfpathlineto{\pgfqpoint{3.858724in}{1.123626in}}%
\pgfpathlineto{\pgfqpoint{3.859135in}{1.121651in}}%
\pgfpathlineto{\pgfqpoint{3.860490in}{1.119846in}}%
\pgfpathlineto{\pgfqpoint{3.860508in}{1.119869in}}%
\pgfpathlineto{\pgfqpoint{3.860829in}{1.121792in}}%
\pgfpathlineto{\pgfqpoint{3.862096in}{1.123611in}}%
\pgfpathlineto{\pgfqpoint{3.862114in}{1.123585in}}%
\pgfpathlineto{\pgfqpoint{3.862524in}{1.121486in}}%
\pgfpathlineto{\pgfqpoint{3.863845in}{1.119883in}}%
\pgfpathlineto{\pgfqpoint{3.863862in}{1.119891in}}%
\pgfpathlineto{\pgfqpoint{3.863898in}{1.119898in}}%
\pgfpathlineto{\pgfqpoint{3.864094in}{1.120925in}}%
\pgfpathlineto{\pgfqpoint{3.865486in}{1.123757in}}%
\pgfpathlineto{\pgfqpoint{3.865504in}{1.123729in}}%
\pgfpathlineto{\pgfqpoint{3.866039in}{1.120807in}}%
\pgfpathlineto{\pgfqpoint{3.867252in}{1.120030in}}%
\pgfpathlineto{\pgfqpoint{3.866735in}{1.121241in}}%
\pgfpathlineto{\pgfqpoint{3.867270in}{1.120058in}}%
\pgfpathlineto{\pgfqpoint{3.868858in}{1.123567in}}%
\pgfpathlineto{\pgfqpoint{3.869161in}{1.122251in}}%
\pgfpathlineto{\pgfqpoint{3.870588in}{1.119956in}}%
\pgfpathlineto{\pgfqpoint{3.870892in}{1.121273in}}%
\pgfpathlineto{\pgfqpoint{3.872248in}{1.123604in}}%
\pgfpathlineto{\pgfqpoint{3.872283in}{1.123564in}}%
\pgfpathlineto{\pgfqpoint{3.873978in}{1.119945in}}%
\pgfpathlineto{\pgfqpoint{3.874246in}{1.121113in}}%
\pgfpathlineto{\pgfqpoint{3.875602in}{1.123572in}}%
\pgfpathlineto{\pgfqpoint{3.875637in}{1.123518in}}%
\pgfpathlineto{\pgfqpoint{3.875958in}{1.121893in}}%
\pgfpathlineto{\pgfqpoint{3.877350in}{1.119881in}}%
\pgfpathlineto{\pgfqpoint{3.877386in}{1.119888in}}%
\pgfpathlineto{\pgfqpoint{3.878973in}{1.123547in}}%
\pgfpathlineto{\pgfqpoint{3.879259in}{1.122436in}}%
\pgfpathlineto{\pgfqpoint{3.879758in}{1.120061in}}%
\pgfpathlineto{\pgfqpoint{3.880633in}{1.120206in}}%
\pgfpathlineto{\pgfqpoint{3.880740in}{1.120076in}}%
\pgfpathlineto{\pgfqpoint{3.881114in}{1.122295in}}%
\pgfpathlineto{\pgfqpoint{3.881400in}{1.123757in}}%
\pgfpathlineto{\pgfqpoint{3.882381in}{1.123569in}}%
\pgfpathlineto{\pgfqpoint{3.882649in}{1.122477in}}%
\pgfpathlineto{\pgfqpoint{3.884112in}{1.119999in}}%
\pgfpathlineto{\pgfqpoint{3.884361in}{1.121091in}}%
\pgfpathlineto{\pgfqpoint{3.884790in}{1.123643in}}%
\pgfpathlineto{\pgfqpoint{3.885753in}{1.123493in}}%
\pgfpathlineto{\pgfqpoint{3.887519in}{1.120074in}}%
\pgfpathlineto{\pgfqpoint{3.887769in}{1.121428in}}%
\pgfpathlineto{\pgfqpoint{3.888179in}{1.123722in}}%
\pgfpathlineto{\pgfqpoint{3.889196in}{1.123423in}}%
\pgfpathlineto{\pgfqpoint{3.890873in}{1.119909in}}%
\pgfpathlineto{\pgfqpoint{3.891105in}{1.120971in}}%
\pgfpathlineto{\pgfqpoint{3.891551in}{1.123594in}}%
\pgfpathlineto{\pgfqpoint{3.892515in}{1.123382in}}%
\pgfpathlineto{\pgfqpoint{3.894281in}{1.120175in}}%
\pgfpathlineto{\pgfqpoint{3.894531in}{1.121559in}}%
\pgfpathlineto{\pgfqpoint{3.894905in}{1.123751in}}%
\pgfpathlineto{\pgfqpoint{3.895851in}{1.123483in}}%
\pgfpathlineto{\pgfqpoint{3.896475in}{1.120346in}}%
\pgfpathlineto{\pgfqpoint{3.897617in}{1.120031in}}%
\pgfpathlineto{\pgfqpoint{3.897100in}{1.121089in}}%
\pgfpathlineto{\pgfqpoint{3.897688in}{1.120148in}}%
\pgfpathlineto{\pgfqpoint{3.899258in}{1.123541in}}%
\pgfpathlineto{\pgfqpoint{3.899544in}{1.122280in}}%
\pgfpathlineto{\pgfqpoint{3.900971in}{1.120188in}}%
\pgfpathlineto{\pgfqpoint{3.901114in}{1.120482in}}%
\pgfpathlineto{\pgfqpoint{3.901239in}{1.121235in}}%
\pgfpathlineto{\pgfqpoint{3.902577in}{1.123593in}}%
\pgfpathlineto{\pgfqpoint{3.902684in}{1.123469in}}%
\pgfpathlineto{\pgfqpoint{3.904361in}{1.120105in}}%
\pgfpathlineto{\pgfqpoint{3.904682in}{1.121703in}}%
\pgfpathlineto{\pgfqpoint{3.906002in}{1.123554in}}%
\pgfpathlineto{\pgfqpoint{3.906038in}{1.123522in}}%
\pgfpathlineto{\pgfqpoint{3.907733in}{1.120083in}}%
\pgfpathlineto{\pgfqpoint{3.908054in}{1.121582in}}%
\pgfpathlineto{\pgfqpoint{3.908446in}{1.123569in}}%
\pgfpathlineto{\pgfqpoint{3.909374in}{1.123416in}}%
\pgfpathlineto{\pgfqpoint{3.909731in}{1.121739in}}%
\pgfpathlineto{\pgfqpoint{3.911123in}{1.120150in}}%
\pgfpathlineto{\pgfqpoint{3.911676in}{1.123371in}}%
\pgfpathlineto{\pgfqpoint{3.912764in}{1.123520in}}%
\pgfpathlineto{\pgfqpoint{3.912282in}{1.122348in}}%
\pgfpathlineto{\pgfqpoint{3.912817in}{1.123424in}}%
\pgfpathlineto{\pgfqpoint{3.914494in}{1.120111in}}%
\pgfpathlineto{\pgfqpoint{3.914584in}{1.120247in}}%
\pgfpathlineto{\pgfqpoint{3.916100in}{1.123401in}}%
\pgfpathlineto{\pgfqpoint{3.916350in}{1.122528in}}%
\pgfpathlineto{\pgfqpoint{3.916903in}{1.119998in}}%
\pgfpathlineto{\pgfqpoint{3.917777in}{1.120209in}}%
\pgfpathlineto{\pgfqpoint{3.917920in}{1.120113in}}%
\pgfpathlineto{\pgfqpoint{3.918170in}{1.121491in}}%
\pgfpathlineto{\pgfqpoint{3.918544in}{1.123621in}}%
\pgfpathlineto{\pgfqpoint{3.919543in}{1.123295in}}%
\pgfpathlineto{\pgfqpoint{3.920079in}{1.120437in}}%
\pgfpathlineto{\pgfqpoint{3.921238in}{1.120033in}}%
\pgfpathlineto{\pgfqpoint{3.920757in}{1.121053in}}%
\pgfpathlineto{\pgfqpoint{3.921256in}{1.120042in}}%
\pgfpathlineto{\pgfqpoint{3.921702in}{1.122737in}}%
\pgfpathlineto{\pgfqpoint{3.922844in}{1.123418in}}%
\pgfpathlineto{\pgfqpoint{3.922398in}{1.122286in}}%
\pgfpathlineto{\pgfqpoint{3.922915in}{1.123340in}}%
\pgfpathlineto{\pgfqpoint{3.924646in}{1.120047in}}%
\pgfpathlineto{\pgfqpoint{3.924878in}{1.121133in}}%
\pgfpathlineto{\pgfqpoint{3.926234in}{1.123418in}}%
\pgfpathlineto{\pgfqpoint{3.926287in}{1.123356in}}%
\pgfpathlineto{\pgfqpoint{3.927964in}{1.120114in}}%
\pgfpathlineto{\pgfqpoint{3.928285in}{1.121449in}}%
\pgfpathlineto{\pgfqpoint{3.928696in}{1.123635in}}%
\pgfpathlineto{\pgfqpoint{3.929606in}{1.123367in}}%
\pgfpathlineto{\pgfqpoint{3.929909in}{1.122262in}}%
\pgfpathlineto{\pgfqpoint{3.931318in}{1.120193in}}%
\pgfpathlineto{\pgfqpoint{3.931746in}{1.122220in}}%
\pgfpathlineto{\pgfqpoint{3.932068in}{1.123607in}}%
\pgfpathlineto{\pgfqpoint{3.933013in}{1.123378in}}%
\pgfpathlineto{\pgfqpoint{3.934779in}{1.120262in}}%
\pgfpathlineto{\pgfqpoint{3.935011in}{1.121358in}}%
\pgfpathlineto{\pgfqpoint{3.935439in}{1.123645in}}%
\pgfpathlineto{\pgfqpoint{3.936367in}{1.123392in}}%
\pgfpathlineto{\pgfqpoint{3.936706in}{1.121965in}}%
\pgfpathlineto{\pgfqpoint{3.938098in}{1.120299in}}%
\pgfpathlineto{\pgfqpoint{3.938383in}{1.121374in}}%
\pgfpathlineto{\pgfqpoint{3.938758in}{1.123567in}}%
\pgfpathlineto{\pgfqpoint{3.939846in}{1.123181in}}%
\pgfpathlineto{\pgfqpoint{3.941523in}{1.120163in}}%
\pgfpathlineto{\pgfqpoint{3.941719in}{1.120943in}}%
\pgfpathlineto{\pgfqpoint{3.943111in}{1.123460in}}%
\pgfpathlineto{\pgfqpoint{3.943182in}{1.123350in}}%
\pgfpathlineto{\pgfqpoint{3.943539in}{1.121491in}}%
\pgfpathlineto{\pgfqpoint{3.943914in}{1.120171in}}%
\pgfpathlineto{\pgfqpoint{3.944824in}{1.120410in}}%
\pgfpathlineto{\pgfqpoint{3.945163in}{1.121669in}}%
\pgfpathlineto{\pgfqpoint{3.945591in}{1.123655in}}%
\pgfpathlineto{\pgfqpoint{3.946501in}{1.123435in}}%
\pgfpathlineto{\pgfqpoint{3.947000in}{1.120945in}}%
\pgfpathlineto{\pgfqpoint{3.947821in}{1.121200in}}%
\pgfpathlineto{\pgfqpoint{3.948231in}{1.120259in}}%
\pgfpathlineto{\pgfqpoint{3.949873in}{1.123338in}}%
\pgfpathlineto{\pgfqpoint{3.950229in}{1.121793in}}%
\pgfpathlineto{\pgfqpoint{3.951585in}{1.120304in}}%
\pgfpathlineto{\pgfqpoint{3.951728in}{1.120433in}}%
\pgfpathlineto{\pgfqpoint{3.951889in}{1.121314in}}%
\pgfpathlineto{\pgfqpoint{3.953262in}{1.123380in}}%
\pgfpathlineto{\pgfqpoint{3.953298in}{1.123326in}}%
\pgfpathlineto{\pgfqpoint{3.955011in}{1.120370in}}%
\pgfpathlineto{\pgfqpoint{3.955278in}{1.121458in}}%
\pgfpathlineto{\pgfqpoint{3.955689in}{1.123605in}}%
\pgfpathlineto{\pgfqpoint{3.956599in}{1.123306in}}%
\pgfpathlineto{\pgfqpoint{3.956670in}{1.123267in}}%
\pgfpathlineto{\pgfqpoint{3.956920in}{1.122171in}}%
\pgfpathlineto{\pgfqpoint{3.957384in}{1.120147in}}%
\pgfpathlineto{\pgfqpoint{3.958240in}{1.120448in}}%
\pgfpathlineto{\pgfqpoint{3.958401in}{1.120310in}}%
\pgfpathlineto{\pgfqpoint{3.958722in}{1.121894in}}%
\pgfpathlineto{\pgfqpoint{3.959078in}{1.123561in}}%
\pgfpathlineto{\pgfqpoint{3.959988in}{1.123304in}}%
\pgfpathlineto{\pgfqpoint{3.960274in}{1.122256in}}%
\pgfpathlineto{\pgfqpoint{3.961719in}{1.120410in}}%
\pgfpathlineto{\pgfqpoint{3.961755in}{1.120427in}}%
\pgfpathlineto{\pgfqpoint{3.961826in}{1.120487in}}%
\pgfpathlineto{\pgfqpoint{3.962022in}{1.121482in}}%
\pgfpathlineto{\pgfqpoint{3.962433in}{1.123682in}}%
\pgfpathlineto{\pgfqpoint{3.963360in}{1.123381in}}%
\pgfpathlineto{\pgfqpoint{3.963664in}{1.122206in}}%
\pgfpathlineto{\pgfqpoint{3.965055in}{1.120342in}}%
\pgfpathlineto{\pgfqpoint{3.965162in}{1.120313in}}%
\pgfpathlineto{\pgfqpoint{3.965394in}{1.121354in}}%
\pgfpathlineto{\pgfqpoint{3.965804in}{1.123492in}}%
\pgfpathlineto{\pgfqpoint{3.966750in}{1.123190in}}%
\pgfpathlineto{\pgfqpoint{3.967035in}{1.122205in}}%
\pgfpathlineto{\pgfqpoint{3.968427in}{1.120263in}}%
\pgfpathlineto{\pgfqpoint{3.968552in}{1.120266in}}%
\pgfpathlineto{\pgfqpoint{3.968766in}{1.121316in}}%
\pgfpathlineto{\pgfqpoint{3.969194in}{1.123527in}}%
\pgfpathlineto{\pgfqpoint{3.970158in}{1.123263in}}%
\pgfpathlineto{\pgfqpoint{3.971870in}{1.120333in}}%
\pgfpathlineto{\pgfqpoint{3.972209in}{1.121915in}}%
\pgfpathlineto{\pgfqpoint{3.972548in}{1.123561in}}%
\pgfpathlineto{\pgfqpoint{3.973512in}{1.123228in}}%
\pgfpathlineto{\pgfqpoint{3.973779in}{1.122180in}}%
\pgfpathlineto{\pgfqpoint{3.975224in}{1.120332in}}%
\pgfpathlineto{\pgfqpoint{3.975260in}{1.120350in}}%
\pgfpathlineto{\pgfqpoint{3.975278in}{1.120337in}}%
\pgfpathlineto{\pgfqpoint{3.975545in}{1.121524in}}%
\pgfpathlineto{\pgfqpoint{3.975938in}{1.123432in}}%
\pgfpathlineto{\pgfqpoint{3.976919in}{1.123207in}}%
\pgfpathlineto{\pgfqpoint{3.977312in}{1.121290in}}%
\pgfpathlineto{\pgfqpoint{3.978596in}{1.120418in}}%
\pgfpathlineto{\pgfqpoint{3.978917in}{1.121509in}}%
\pgfpathlineto{\pgfqpoint{3.979310in}{1.123423in}}%
\pgfpathlineto{\pgfqpoint{3.980309in}{1.123159in}}%
\pgfpathlineto{\pgfqpoint{3.982040in}{1.120414in}}%
\pgfpathlineto{\pgfqpoint{3.982236in}{1.121146in}}%
\pgfpathlineto{\pgfqpoint{3.982700in}{1.123380in}}%
\pgfpathlineto{\pgfqpoint{3.983752in}{1.122912in}}%
\pgfpathlineto{\pgfqpoint{3.984430in}{1.120072in}}%
\pgfpathlineto{\pgfqpoint{3.985501in}{1.120594in}}%
\pgfpathlineto{\pgfqpoint{3.987017in}{1.123191in}}%
\pgfpathlineto{\pgfqpoint{3.987124in}{1.122958in}}%
\pgfpathlineto{\pgfqpoint{3.987749in}{1.120147in}}%
\pgfpathlineto{\pgfqpoint{3.988890in}{1.120769in}}%
\pgfpathlineto{\pgfqpoint{3.990389in}{1.123181in}}%
\pgfpathlineto{\pgfqpoint{3.990550in}{1.122771in}}%
\pgfpathlineto{\pgfqpoint{3.991120in}{1.120186in}}%
\pgfpathlineto{\pgfqpoint{3.992084in}{1.120423in}}%
\pgfpathlineto{\pgfqpoint{3.992155in}{1.120420in}}%
\pgfpathlineto{\pgfqpoint{3.992423in}{1.121553in}}%
\pgfpathlineto{\pgfqpoint{3.992815in}{1.123452in}}%
\pgfpathlineto{\pgfqpoint{3.993725in}{1.123222in}}%
\pgfpathlineto{\pgfqpoint{3.995527in}{1.120444in}}%
\pgfpathlineto{\pgfqpoint{3.995813in}{1.121739in}}%
\pgfpathlineto{\pgfqpoint{3.996187in}{1.123517in}}%
\pgfpathlineto{\pgfqpoint{3.997151in}{1.123267in}}%
\pgfpathlineto{\pgfqpoint{3.997454in}{1.121954in}}%
\pgfpathlineto{\pgfqpoint{3.997900in}{1.120259in}}%
\pgfpathlineto{\pgfqpoint{3.998774in}{1.120555in}}%
\pgfpathlineto{\pgfqpoint{3.998881in}{1.120432in}}%
\pgfpathlineto{\pgfqpoint{3.999238in}{1.122076in}}%
\pgfpathlineto{\pgfqpoint{3.999577in}{1.123494in}}%
\pgfpathlineto{\pgfqpoint{4.000487in}{1.123153in}}%
\pgfpathlineto{\pgfqpoint{4.000558in}{1.123123in}}%
\pgfpathlineto{\pgfqpoint{4.000808in}{1.122052in}}%
\pgfpathlineto{\pgfqpoint{4.001254in}{1.120360in}}%
\pgfpathlineto{\pgfqpoint{4.002164in}{1.120584in}}%
\pgfpathlineto{\pgfqpoint{4.002253in}{1.120473in}}%
\pgfpathlineto{\pgfqpoint{4.002628in}{1.122182in}}%
\pgfpathlineto{\pgfqpoint{4.002949in}{1.123456in}}%
\pgfpathlineto{\pgfqpoint{4.003894in}{1.123116in}}%
\pgfpathlineto{\pgfqpoint{4.004216in}{1.121773in}}%
\pgfpathlineto{\pgfqpoint{4.004644in}{1.120223in}}%
\pgfpathlineto{\pgfqpoint{4.005554in}{1.120523in}}%
\pgfpathlineto{\pgfqpoint{4.005928in}{1.121648in}}%
\pgfpathlineto{\pgfqpoint{4.006339in}{1.123457in}}%
\pgfpathlineto{\pgfqpoint{4.007266in}{1.123195in}}%
\pgfpathlineto{\pgfqpoint{4.007784in}{1.120825in}}%
\pgfpathlineto{\pgfqpoint{4.008533in}{1.121202in}}%
\pgfpathlineto{\pgfqpoint{4.008979in}{1.120517in}}%
\pgfpathlineto{\pgfqpoint{4.009122in}{1.120660in}}%
\pgfpathlineto{\pgfqpoint{4.009282in}{1.121495in}}%
\pgfpathlineto{\pgfqpoint{4.009711in}{1.123381in}}%
\pgfpathlineto{\pgfqpoint{4.010674in}{1.123142in}}%
\pgfpathlineto{\pgfqpoint{4.011227in}{1.120579in}}%
\pgfpathlineto{\pgfqpoint{4.012387in}{1.120479in}}%
\pgfpathlineto{\pgfqpoint{4.011905in}{1.121228in}}%
\pgfpathlineto{\pgfqpoint{4.012404in}{1.120509in}}%
\pgfpathlineto{\pgfqpoint{4.013992in}{1.123157in}}%
\pgfpathlineto{\pgfqpoint{4.014313in}{1.121970in}}%
\pgfpathlineto{\pgfqpoint{4.014742in}{1.120221in}}%
\pgfpathlineto{\pgfqpoint{4.015651in}{1.120569in}}%
\pgfpathlineto{\pgfqpoint{4.015848in}{1.120623in}}%
\pgfpathlineto{\pgfqpoint{4.016044in}{1.121557in}}%
\pgfpathlineto{\pgfqpoint{4.016436in}{1.123396in}}%
\pgfpathlineto{\pgfqpoint{4.017436in}{1.123119in}}%
\pgfpathlineto{\pgfqpoint{4.019148in}{1.120535in}}%
\pgfpathlineto{\pgfqpoint{4.019487in}{1.122031in}}%
\pgfpathlineto{\pgfqpoint{4.019826in}{1.123344in}}%
\pgfpathlineto{\pgfqpoint{4.020772in}{1.123112in}}%
\pgfpathlineto{\pgfqpoint{4.022520in}{1.120448in}}%
\pgfpathlineto{\pgfqpoint{4.022877in}{1.122003in}}%
\pgfpathlineto{\pgfqpoint{4.023234in}{1.123268in}}%
\pgfpathlineto{\pgfqpoint{4.024197in}{1.123036in}}%
\pgfpathlineto{\pgfqpoint{4.025910in}{1.120758in}}%
\pgfpathlineto{\pgfqpoint{4.026124in}{1.121558in}}%
\pgfpathlineto{\pgfqpoint{4.026552in}{1.123596in}}%
\pgfpathlineto{\pgfqpoint{4.027533in}{1.123294in}}%
\pgfpathlineto{\pgfqpoint{4.027926in}{1.121470in}}%
\pgfpathlineto{\pgfqpoint{4.028265in}{1.120433in}}%
\pgfpathlineto{\pgfqpoint{4.029228in}{1.120674in}}%
\pgfpathlineto{\pgfqpoint{4.029549in}{1.121716in}}%
\pgfpathlineto{\pgfqpoint{4.029978in}{1.123419in}}%
\pgfpathlineto{\pgfqpoint{4.030905in}{1.123136in}}%
\pgfpathlineto{\pgfqpoint{4.032654in}{1.120507in}}%
\pgfpathlineto{\pgfqpoint{4.032975in}{1.121854in}}%
\pgfpathlineto{\pgfqpoint{4.033332in}{1.123260in}}%
\pgfpathlineto{\pgfqpoint{4.034295in}{1.122934in}}%
\pgfpathlineto{\pgfqpoint{4.036008in}{1.120446in}}%
\pgfpathlineto{\pgfqpoint{4.036258in}{1.121255in}}%
\pgfpathlineto{\pgfqpoint{4.036721in}{1.123230in}}%
\pgfpathlineto{\pgfqpoint{4.037685in}{1.122946in}}%
\pgfpathlineto{\pgfqpoint{4.038131in}{1.120856in}}%
\pgfpathlineto{\pgfqpoint{4.038416in}{1.120270in}}%
\pgfpathlineto{\pgfqpoint{4.038934in}{1.121205in}}%
\pgfpathlineto{\pgfqpoint{4.039344in}{1.120575in}}%
\pgfpathlineto{\pgfqpoint{4.039487in}{1.120681in}}%
\pgfpathlineto{\pgfqpoint{4.039683in}{1.121637in}}%
\pgfpathlineto{\pgfqpoint{4.040075in}{1.123420in}}%
\pgfpathlineto{\pgfqpoint{4.040985in}{1.123123in}}%
\pgfpathlineto{\pgfqpoint{4.041521in}{1.120914in}}%
\pgfpathlineto{\pgfqpoint{4.041824in}{1.120445in}}%
\pgfpathlineto{\pgfqpoint{4.042288in}{1.121345in}}%
\pgfpathlineto{\pgfqpoint{4.042680in}{1.120713in}}%
\pgfpathlineto{\pgfqpoint{4.043055in}{1.121744in}}%
\pgfpathlineto{\pgfqpoint{4.043412in}{1.123435in}}%
\pgfpathlineto{\pgfqpoint{4.044393in}{1.123021in}}%
\pgfpathlineto{\pgfqpoint{4.044946in}{1.120637in}}%
\pgfpathlineto{\pgfqpoint{4.045660in}{1.121165in}}%
\pgfpathlineto{\pgfqpoint{4.046123in}{1.120559in}}%
\pgfpathlineto{\pgfqpoint{4.046445in}{1.121704in}}%
\pgfpathlineto{\pgfqpoint{4.046837in}{1.123308in}}%
\pgfpathlineto{\pgfqpoint{4.047747in}{1.123118in}}%
\pgfpathlineto{\pgfqpoint{4.048104in}{1.121870in}}%
\pgfpathlineto{\pgfqpoint{4.048550in}{1.120487in}}%
\pgfpathlineto{\pgfqpoint{4.049442in}{1.120759in}}%
\pgfpathlineto{\pgfqpoint{4.049549in}{1.120747in}}%
\pgfpathlineto{\pgfqpoint{4.049817in}{1.121845in}}%
\pgfpathlineto{\pgfqpoint{4.050173in}{1.123471in}}%
\pgfpathlineto{\pgfqpoint{4.051119in}{1.123051in}}%
\pgfpathlineto{\pgfqpoint{4.051279in}{1.122772in}}%
\pgfpathlineto{\pgfqpoint{4.051422in}{1.122057in}}%
\pgfpathlineto{\pgfqpoint{4.051904in}{1.120271in}}%
\pgfpathlineto{\pgfqpoint{4.052814in}{1.120637in}}%
\pgfpathlineto{\pgfqpoint{4.053206in}{1.121818in}}%
\pgfpathlineto{\pgfqpoint{4.053563in}{1.123311in}}%
\pgfpathlineto{\pgfqpoint{4.054473in}{1.122912in}}%
\pgfpathlineto{\pgfqpoint{4.054509in}{1.122973in}}%
\pgfpathlineto{\pgfqpoint{4.054919in}{1.121263in}}%
\pgfpathlineto{\pgfqpoint{4.055311in}{1.120220in}}%
\pgfpathlineto{\pgfqpoint{4.056150in}{1.120571in}}%
\pgfpathlineto{\pgfqpoint{4.056257in}{1.120490in}}%
\pgfpathlineto{\pgfqpoint{4.056560in}{1.121655in}}%
\pgfpathlineto{\pgfqpoint{4.056935in}{1.123400in}}%
\pgfpathlineto{\pgfqpoint{4.057916in}{1.123135in}}%
\pgfpathlineto{\pgfqpoint{4.058416in}{1.120905in}}%
\pgfpathlineto{\pgfqpoint{4.059165in}{1.121284in}}%
\pgfpathlineto{\pgfqpoint{4.059665in}{1.120733in}}%
\pgfpathlineto{\pgfqpoint{4.061342in}{1.123076in}}%
\pgfpathlineto{\pgfqpoint{4.061574in}{1.121975in}}%
\pgfpathlineto{\pgfqpoint{4.062037in}{1.120290in}}%
\pgfpathlineto{\pgfqpoint{4.062965in}{1.120654in}}%
\pgfpathlineto{\pgfqpoint{4.063054in}{1.120675in}}%
\pgfpathlineto{\pgfqpoint{4.063322in}{1.121780in}}%
\pgfpathlineto{\pgfqpoint{4.063732in}{1.123347in}}%
\pgfpathlineto{\pgfqpoint{4.064660in}{1.123193in}}%
\pgfpathlineto{\pgfqpoint{4.064963in}{1.122019in}}%
\pgfpathlineto{\pgfqpoint{4.065409in}{1.120484in}}%
\pgfpathlineto{\pgfqpoint{4.066319in}{1.120773in}}%
\pgfpathlineto{\pgfqpoint{4.066444in}{1.120792in}}%
\pgfpathlineto{\pgfqpoint{4.066694in}{1.121743in}}%
\pgfpathlineto{\pgfqpoint{4.067069in}{1.123377in}}%
\pgfpathlineto{\pgfqpoint{4.068014in}{1.123069in}}%
\pgfpathlineto{\pgfqpoint{4.068085in}{1.123024in}}%
\pgfpathlineto{\pgfqpoint{4.068317in}{1.122062in}}%
\pgfpathlineto{\pgfqpoint{4.068763in}{1.120408in}}%
\pgfpathlineto{\pgfqpoint{4.069673in}{1.120735in}}%
\pgfpathlineto{\pgfqpoint{4.069762in}{1.120709in}}%
\pgfpathlineto{\pgfqpoint{4.070101in}{1.121974in}}%
\pgfpathlineto{\pgfqpoint{4.070476in}{1.123306in}}%
\pgfpathlineto{\pgfqpoint{4.071404in}{1.123003in}}%
\pgfpathlineto{\pgfqpoint{4.071725in}{1.121724in}}%
\pgfpathlineto{\pgfqpoint{4.072135in}{1.120337in}}%
\pgfpathlineto{\pgfqpoint{4.073045in}{1.120753in}}%
\pgfpathlineto{\pgfqpoint{4.073134in}{1.120700in}}%
\pgfpathlineto{\pgfqpoint{4.073527in}{1.122205in}}%
\pgfpathlineto{\pgfqpoint{4.073848in}{1.123307in}}%
\pgfpathlineto{\pgfqpoint{4.074740in}{1.123043in}}%
\pgfpathlineto{\pgfqpoint{4.075079in}{1.121836in}}%
\pgfpathlineto{\pgfqpoint{4.075472in}{1.120320in}}%
\pgfpathlineto{\pgfqpoint{4.076364in}{1.120816in}}%
\pgfpathlineto{\pgfqpoint{4.076542in}{1.120637in}}%
\pgfpathlineto{\pgfqpoint{4.076917in}{1.122282in}}%
\pgfpathlineto{\pgfqpoint{4.077202in}{1.123242in}}%
\pgfpathlineto{\pgfqpoint{4.077684in}{1.122044in}}%
\pgfpathlineto{\pgfqpoint{4.078112in}{1.122852in}}%
\pgfpathlineto{\pgfqpoint{4.078451in}{1.121830in}}%
\pgfpathlineto{\pgfqpoint{4.078897in}{1.120359in}}%
\pgfpathlineto{\pgfqpoint{4.079771in}{1.120810in}}%
\pgfpathlineto{\pgfqpoint{4.079878in}{1.120677in}}%
\pgfpathlineto{\pgfqpoint{4.080324in}{1.122477in}}%
\pgfpathlineto{\pgfqpoint{4.080592in}{1.123231in}}%
\pgfpathlineto{\pgfqpoint{4.081074in}{1.122035in}}%
\pgfpathlineto{\pgfqpoint{4.081520in}{1.122894in}}%
\pgfpathlineto{\pgfqpoint{4.082019in}{1.120792in}}%
\pgfpathlineto{\pgfqpoint{4.082251in}{1.120289in}}%
\pgfpathlineto{\pgfqpoint{4.082751in}{1.121127in}}%
\pgfpathlineto{\pgfqpoint{4.083214in}{1.120523in}}%
\pgfpathlineto{\pgfqpoint{4.084838in}{1.122856in}}%
\pgfpathlineto{\pgfqpoint{4.085230in}{1.121619in}}%
\pgfpathlineto{\pgfqpoint{4.085623in}{1.120295in}}%
\pgfpathlineto{\pgfqpoint{4.086497in}{1.120746in}}%
\pgfpathlineto{\pgfqpoint{4.086658in}{1.120638in}}%
\pgfpathlineto{\pgfqpoint{4.086997in}{1.121997in}}%
\pgfpathlineto{\pgfqpoint{4.087353in}{1.123230in}}%
\pgfpathlineto{\pgfqpoint{4.088246in}{1.122965in}}%
\pgfpathlineto{\pgfqpoint{4.088584in}{1.121921in}}%
\pgfpathlineto{\pgfqpoint{4.089031in}{1.120451in}}%
\pgfpathlineto{\pgfqpoint{4.089923in}{1.120705in}}%
\pgfpathlineto{\pgfqpoint{4.090030in}{1.120628in}}%
\pgfpathlineto{\pgfqpoint{4.090386in}{1.122097in}}%
\pgfpathlineto{\pgfqpoint{4.090743in}{1.123269in}}%
\pgfpathlineto{\pgfqpoint{4.091671in}{1.122927in}}%
\pgfpathlineto{\pgfqpoint{4.091974in}{1.121786in}}%
\pgfpathlineto{\pgfqpoint{4.092385in}{1.120407in}}%
\pgfpathlineto{\pgfqpoint{4.093366in}{1.120784in}}%
\pgfpathlineto{\pgfqpoint{4.095079in}{1.123016in}}%
\pgfpathlineto{\pgfqpoint{4.095275in}{1.122289in}}%
\pgfpathlineto{\pgfqpoint{4.095774in}{1.120481in}}%
\pgfpathlineto{\pgfqpoint{4.096791in}{1.120869in}}%
\pgfpathlineto{\pgfqpoint{4.098415in}{1.123006in}}%
\pgfpathlineto{\pgfqpoint{4.098700in}{1.121998in}}%
\pgfpathlineto{\pgfqpoint{4.099128in}{1.120407in}}%
\pgfpathlineto{\pgfqpoint{4.100003in}{1.120863in}}%
\pgfpathlineto{\pgfqpoint{4.100234in}{1.120861in}}%
\pgfpathlineto{\pgfqpoint{4.100466in}{1.121865in}}%
\pgfpathlineto{\pgfqpoint{4.100841in}{1.123231in}}%
\pgfpathlineto{\pgfqpoint{4.101787in}{1.122967in}}%
\pgfpathlineto{\pgfqpoint{4.102143in}{1.121606in}}%
\pgfpathlineto{\pgfqpoint{4.102572in}{1.120448in}}%
\pgfpathlineto{\pgfqpoint{4.103410in}{1.120801in}}%
\pgfpathlineto{\pgfqpoint{4.103553in}{1.120754in}}%
\pgfpathlineto{\pgfqpoint{4.103838in}{1.121825in}}%
\pgfpathlineto{\pgfqpoint{4.104213in}{1.123268in}}%
\pgfpathlineto{\pgfqpoint{4.105105in}{1.122924in}}%
\pgfpathlineto{\pgfqpoint{4.105319in}{1.122594in}}%
\pgfpathlineto{\pgfqpoint{4.105462in}{1.121870in}}%
\pgfpathlineto{\pgfqpoint{4.105908in}{1.120413in}}%
\pgfpathlineto{\pgfqpoint{4.106800in}{1.120859in}}%
\pgfpathlineto{\pgfqpoint{4.107264in}{1.122112in}}%
\pgfpathlineto{\pgfqpoint{4.107603in}{1.123232in}}%
\pgfpathlineto{\pgfqpoint{4.108495in}{1.122950in}}%
\pgfpathlineto{\pgfqpoint{4.108584in}{1.122926in}}%
\pgfpathlineto{\pgfqpoint{4.108834in}{1.121895in}}%
\pgfpathlineto{\pgfqpoint{4.109280in}{1.120484in}}%
\pgfpathlineto{\pgfqpoint{4.110190in}{1.120842in}}%
\pgfpathlineto{\pgfqpoint{4.110279in}{1.120822in}}%
\pgfpathlineto{\pgfqpoint{4.110618in}{1.122002in}}%
\pgfpathlineto{\pgfqpoint{4.110957in}{1.123219in}}%
\pgfpathlineto{\pgfqpoint{4.111885in}{1.122891in}}%
\pgfpathlineto{\pgfqpoint{4.112206in}{1.121837in}}%
\pgfpathlineto{\pgfqpoint{4.112598in}{1.120370in}}%
\pgfpathlineto{\pgfqpoint{4.113562in}{1.120751in}}%
\pgfpathlineto{\pgfqpoint{4.113936in}{1.121639in}}%
\pgfpathlineto{\pgfqpoint{4.113954in}{1.121740in}}%
\pgfpathlineto{\pgfqpoint{4.114329in}{1.123184in}}%
\pgfpathlineto{\pgfqpoint{4.115256in}{1.122893in}}%
\pgfpathlineto{\pgfqpoint{4.115702in}{1.121279in}}%
\pgfpathlineto{\pgfqpoint{4.116059in}{1.120487in}}%
\pgfpathlineto{\pgfqpoint{4.116898in}{1.120894in}}%
\pgfpathlineto{\pgfqpoint{4.117023in}{1.120816in}}%
\pgfpathlineto{\pgfqpoint{4.117397in}{1.122156in}}%
\pgfpathlineto{\pgfqpoint{4.117701in}{1.123208in}}%
\pgfpathlineto{\pgfqpoint{4.118628in}{1.122815in}}%
\pgfpathlineto{\pgfqpoint{4.118967in}{1.121758in}}%
\pgfpathlineto{\pgfqpoint{4.119395in}{1.120399in}}%
\pgfpathlineto{\pgfqpoint{4.120198in}{1.120976in}}%
\pgfpathlineto{\pgfqpoint{4.120484in}{1.120907in}}%
\pgfpathlineto{\pgfqpoint{4.120716in}{1.121889in}}%
\pgfpathlineto{\pgfqpoint{4.121108in}{1.123269in}}%
\pgfpathlineto{\pgfqpoint{4.121947in}{1.122779in}}%
\pgfpathlineto{\pgfqpoint{4.122357in}{1.121719in}}%
\pgfpathlineto{\pgfqpoint{4.122785in}{1.120486in}}%
\pgfpathlineto{\pgfqpoint{4.123642in}{1.120941in}}%
\pgfpathlineto{\pgfqpoint{4.123749in}{1.120866in}}%
\pgfpathlineto{\pgfqpoint{4.124177in}{1.122338in}}%
\pgfpathlineto{\pgfqpoint{4.124480in}{1.123233in}}%
\pgfpathlineto{\pgfqpoint{4.124926in}{1.122064in}}%
\pgfpathlineto{\pgfqpoint{4.125390in}{1.122851in}}%
\pgfpathlineto{\pgfqpoint{4.125729in}{1.121808in}}%
\pgfpathlineto{\pgfqpoint{4.126139in}{1.120480in}}%
\pgfpathlineto{\pgfqpoint{4.127085in}{1.120795in}}%
\pgfpathlineto{\pgfqpoint{4.127245in}{1.120944in}}%
\pgfpathlineto{\pgfqpoint{4.127459in}{1.121821in}}%
\pgfpathlineto{\pgfqpoint{4.127852in}{1.123265in}}%
\pgfpathlineto{\pgfqpoint{4.128798in}{1.122939in}}%
\pgfpathlineto{\pgfqpoint{4.129190in}{1.121394in}}%
\pgfpathlineto{\pgfqpoint{4.129565in}{1.120503in}}%
\pgfpathlineto{\pgfqpoint{4.130385in}{1.120903in}}%
\pgfpathlineto{\pgfqpoint{4.130564in}{1.120874in}}%
\pgfpathlineto{\pgfqpoint{4.130849in}{1.121919in}}%
\pgfpathlineto{\pgfqpoint{4.131224in}{1.123229in}}%
\pgfpathlineto{\pgfqpoint{4.132116in}{1.122899in}}%
\pgfpathlineto{\pgfqpoint{4.132169in}{1.122932in}}%
\pgfpathlineto{\pgfqpoint{4.132544in}{1.121481in}}%
\pgfpathlineto{\pgfqpoint{4.132937in}{1.120498in}}%
\pgfpathlineto{\pgfqpoint{4.133811in}{1.120813in}}%
\pgfpathlineto{\pgfqpoint{4.134364in}{1.122615in}}%
\pgfpathlineto{\pgfqpoint{4.134578in}{1.123158in}}%
\pgfpathlineto{\pgfqpoint{4.135077in}{1.122020in}}%
\pgfpathlineto{\pgfqpoint{4.135559in}{1.122816in}}%
\pgfpathlineto{\pgfqpoint{4.137290in}{1.120886in}}%
\pgfpathlineto{\pgfqpoint{4.137504in}{1.121457in}}%
\pgfpathlineto{\pgfqpoint{4.137986in}{1.123228in}}%
\pgfpathlineto{\pgfqpoint{4.138913in}{1.122894in}}%
\pgfpathlineto{\pgfqpoint{4.139341in}{1.121253in}}%
\pgfpathlineto{\pgfqpoint{4.139645in}{1.120520in}}%
\pgfpathlineto{\pgfqpoint{4.140590in}{1.120864in}}%
\pgfpathlineto{\pgfqpoint{4.140965in}{1.121779in}}%
\pgfpathlineto{\pgfqpoint{4.140983in}{1.121877in}}%
\pgfpathlineto{\pgfqpoint{4.141357in}{1.123094in}}%
\pgfpathlineto{\pgfqpoint{4.142267in}{1.122834in}}%
\pgfpathlineto{\pgfqpoint{4.144087in}{1.120871in}}%
\pgfpathlineto{\pgfqpoint{4.144355in}{1.121869in}}%
\pgfpathlineto{\pgfqpoint{4.144676in}{1.123045in}}%
\pgfpathlineto{\pgfqpoint{4.145621in}{1.122800in}}%
\pgfpathlineto{\pgfqpoint{4.145711in}{1.122780in}}%
\pgfpathlineto{\pgfqpoint{4.145978in}{1.121766in}}%
\pgfpathlineto{\pgfqpoint{4.146389in}{1.120495in}}%
\pgfpathlineto{\pgfqpoint{4.147298in}{1.120926in}}%
\pgfpathlineto{\pgfqpoint{4.147727in}{1.121939in}}%
\pgfpathlineto{\pgfqpoint{4.148119in}{1.123197in}}%
\pgfpathlineto{\pgfqpoint{4.148993in}{1.122739in}}%
\pgfpathlineto{\pgfqpoint{4.149082in}{1.122738in}}%
\pgfpathlineto{\pgfqpoint{4.149350in}{1.121716in}}%
\pgfpathlineto{\pgfqpoint{4.149760in}{1.120411in}}%
\pgfpathlineto{\pgfqpoint{4.150617in}{1.120848in}}%
\pgfpathlineto{\pgfqpoint{4.150849in}{1.120794in}}%
\pgfpathlineto{\pgfqpoint{4.151134in}{1.121933in}}%
\pgfpathlineto{\pgfqpoint{4.151473in}{1.123043in}}%
\pgfpathlineto{\pgfqpoint{4.152401in}{1.122741in}}%
\pgfpathlineto{\pgfqpoint{4.152740in}{1.121671in}}%
\pgfpathlineto{\pgfqpoint{4.153150in}{1.120483in}}%
\pgfpathlineto{\pgfqpoint{4.153989in}{1.120993in}}%
\pgfpathlineto{\pgfqpoint{4.154167in}{1.120896in}}%
\pgfpathlineto{\pgfqpoint{4.154488in}{1.122030in}}%
\pgfpathlineto{\pgfqpoint{4.154845in}{1.123319in}}%
\pgfpathlineto{\pgfqpoint{4.155808in}{1.122882in}}%
\pgfpathlineto{\pgfqpoint{4.157539in}{1.120976in}}%
\pgfpathlineto{\pgfqpoint{4.157753in}{1.121490in}}%
\pgfpathlineto{\pgfqpoint{4.158217in}{1.123234in}}%
\pgfpathlineto{\pgfqpoint{4.159252in}{1.122748in}}%
\pgfpathlineto{\pgfqpoint{4.160911in}{1.120978in}}%
\pgfpathlineto{\pgfqpoint{4.161161in}{1.121662in}}%
\pgfpathlineto{\pgfqpoint{4.161625in}{1.123269in}}%
\pgfpathlineto{\pgfqpoint{4.162481in}{1.122847in}}%
\pgfpathlineto{\pgfqpoint{4.162570in}{1.122895in}}%
\pgfpathlineto{\pgfqpoint{4.162963in}{1.121316in}}%
\pgfpathlineto{\pgfqpoint{4.163284in}{1.120558in}}%
\pgfpathlineto{\pgfqpoint{4.164140in}{1.121003in}}%
\pgfpathlineto{\pgfqpoint{4.164283in}{1.120942in}}%
\pgfpathlineto{\pgfqpoint{4.164640in}{1.122116in}}%
\pgfpathlineto{\pgfqpoint{4.164996in}{1.123117in}}%
\pgfpathlineto{\pgfqpoint{4.165906in}{1.122665in}}%
\pgfpathlineto{\pgfqpoint{4.166263in}{1.121505in}}%
\pgfpathlineto{\pgfqpoint{4.166656in}{1.120404in}}%
\pgfpathlineto{\pgfqpoint{4.167512in}{1.120828in}}%
\pgfpathlineto{\pgfqpoint{4.167994in}{1.121909in}}%
\pgfpathlineto{\pgfqpoint{4.168333in}{1.123079in}}%
\pgfpathlineto{\pgfqpoint{4.169260in}{1.122702in}}%
\pgfpathlineto{\pgfqpoint{4.169617in}{1.121695in}}%
\pgfpathlineto{\pgfqpoint{4.170028in}{1.120521in}}%
\pgfpathlineto{\pgfqpoint{4.170866in}{1.120978in}}%
\pgfpathlineto{\pgfqpoint{4.171098in}{1.120953in}}%
\pgfpathlineto{\pgfqpoint{4.171366in}{1.121961in}}%
\pgfpathlineto{\pgfqpoint{4.171705in}{1.123123in}}%
\pgfpathlineto{\pgfqpoint{4.172668in}{1.122701in}}%
\pgfpathlineto{\pgfqpoint{4.173007in}{1.121648in}}%
\pgfpathlineto{\pgfqpoint{4.173382in}{1.120525in}}%
\pgfpathlineto{\pgfqpoint{4.174309in}{1.120901in}}%
\pgfpathlineto{\pgfqpoint{4.174737in}{1.121943in}}%
\pgfpathlineto{\pgfqpoint{4.175112in}{1.123096in}}%
\pgfpathlineto{\pgfqpoint{4.176022in}{1.122718in}}%
\pgfpathlineto{\pgfqpoint{4.177770in}{1.121050in}}%
\pgfpathlineto{\pgfqpoint{4.178145in}{1.122181in}}%
\pgfpathlineto{\pgfqpoint{4.178502in}{1.123154in}}%
\pgfpathlineto{\pgfqpoint{4.179376in}{1.122858in}}%
\pgfpathlineto{\pgfqpoint{4.179804in}{1.121521in}}%
\pgfpathlineto{\pgfqpoint{4.180143in}{1.120634in}}%
\pgfpathlineto{\pgfqpoint{4.181017in}{1.121000in}}%
\pgfpathlineto{\pgfqpoint{4.181142in}{1.120894in}}%
\pgfpathlineto{\pgfqpoint{4.181535in}{1.122085in}}%
\pgfpathlineto{\pgfqpoint{4.181856in}{1.123004in}}%
\pgfpathlineto{\pgfqpoint{4.182302in}{1.121912in}}%
\pgfpathlineto{\pgfqpoint{4.182766in}{1.122632in}}%
\pgfpathlineto{\pgfqpoint{4.183194in}{1.121214in}}%
\pgfpathlineto{\pgfqpoint{4.183515in}{1.120424in}}%
\pgfpathlineto{\pgfqpoint{4.184354in}{1.121043in}}%
\pgfpathlineto{\pgfqpoint{4.184532in}{1.120968in}}%
\pgfpathlineto{\pgfqpoint{4.184889in}{1.122066in}}%
\pgfpathlineto{\pgfqpoint{4.185210in}{1.123137in}}%
\pgfpathlineto{\pgfqpoint{4.186138in}{1.122854in}}%
\pgfpathlineto{\pgfqpoint{4.186298in}{1.122657in}}%
\pgfpathlineto{\pgfqpoint{4.186495in}{1.121845in}}%
\pgfpathlineto{\pgfqpoint{4.186923in}{1.120693in}}%
\pgfpathlineto{\pgfqpoint{4.187833in}{1.121043in}}%
\pgfpathlineto{\pgfqpoint{4.187940in}{1.121049in}}%
\pgfpathlineto{\pgfqpoint{4.188314in}{1.122347in}}%
\pgfpathlineto{\pgfqpoint{4.188618in}{1.123083in}}%
\pgfpathlineto{\pgfqpoint{4.189064in}{1.122028in}}%
\pgfpathlineto{\pgfqpoint{4.189492in}{1.122630in}}%
\pgfpathlineto{\pgfqpoint{4.189599in}{1.122653in}}%
\pgfpathlineto{\pgfqpoint{4.189902in}{1.121529in}}%
\pgfpathlineto{\pgfqpoint{4.190241in}{1.120558in}}%
\pgfpathlineto{\pgfqpoint{4.191115in}{1.121205in}}%
\pgfpathlineto{\pgfqpoint{4.191258in}{1.121070in}}%
\pgfpathlineto{\pgfqpoint{4.191651in}{1.122250in}}%
\pgfpathlineto{\pgfqpoint{4.191972in}{1.123118in}}%
\pgfpathlineto{\pgfqpoint{4.192453in}{1.122051in}}%
\pgfpathlineto{\pgfqpoint{4.192882in}{1.122677in}}%
\pgfpathlineto{\pgfqpoint{4.192953in}{1.122674in}}%
\pgfpathlineto{\pgfqpoint{4.193274in}{1.121506in}}%
\pgfpathlineto{\pgfqpoint{4.193613in}{1.120564in}}%
\pgfpathlineto{\pgfqpoint{4.194541in}{1.121048in}}%
\pgfpathlineto{\pgfqpoint{4.195040in}{1.122272in}}%
\pgfpathlineto{\pgfqpoint{4.195326in}{1.123114in}}%
\pgfpathlineto{\pgfqpoint{4.196325in}{1.122777in}}%
\pgfpathlineto{\pgfqpoint{4.198073in}{1.121050in}}%
\pgfpathlineto{\pgfqpoint{4.198252in}{1.121499in}}%
\pgfpathlineto{\pgfqpoint{4.198733in}{1.123055in}}%
\pgfpathlineto{\pgfqpoint{4.199643in}{1.122616in}}%
\pgfpathlineto{\pgfqpoint{4.199679in}{1.122618in}}%
\pgfpathlineto{\pgfqpoint{4.200018in}{1.121568in}}%
\pgfpathlineto{\pgfqpoint{4.200375in}{1.120485in}}%
\pgfpathlineto{\pgfqpoint{4.201213in}{1.121079in}}%
\pgfpathlineto{\pgfqpoint{4.201463in}{1.121048in}}%
\pgfpathlineto{\pgfqpoint{4.201766in}{1.122053in}}%
\pgfpathlineto{\pgfqpoint{4.202141in}{1.123043in}}%
\pgfpathlineto{\pgfqpoint{4.203104in}{1.122674in}}%
\pgfpathlineto{\pgfqpoint{4.204799in}{1.120946in}}%
\pgfpathlineto{\pgfqpoint{4.205085in}{1.121765in}}%
\pgfpathlineto{\pgfqpoint{4.205459in}{1.123001in}}%
\pgfpathlineto{\pgfqpoint{4.206351in}{1.122557in}}%
\pgfpathlineto{\pgfqpoint{4.206869in}{1.121147in}}%
\pgfpathlineto{\pgfqpoint{4.207172in}{1.120595in}}%
\pgfpathlineto{\pgfqpoint{4.208064in}{1.120882in}}%
\pgfpathlineto{\pgfqpoint{4.208349in}{1.121291in}}%
\pgfpathlineto{\pgfqpoint{4.208492in}{1.121884in}}%
\pgfpathlineto{\pgfqpoint{4.208867in}{1.122952in}}%
\pgfpathlineto{\pgfqpoint{4.209795in}{1.122630in}}%
\pgfpathlineto{\pgfqpoint{4.210187in}{1.121381in}}%
\pgfpathlineto{\pgfqpoint{4.210508in}{1.120593in}}%
\pgfpathlineto{\pgfqpoint{4.211400in}{1.120946in}}%
\pgfpathlineto{\pgfqpoint{4.211668in}{1.121088in}}%
\pgfpathlineto{\pgfqpoint{4.211882in}{1.121906in}}%
\pgfpathlineto{\pgfqpoint{4.212239in}{1.122851in}}%
\pgfpathlineto{\pgfqpoint{4.213113in}{1.122404in}}%
\pgfpathlineto{\pgfqpoint{4.213202in}{1.122441in}}%
\pgfpathlineto{\pgfqpoint{4.213541in}{1.121335in}}%
\pgfpathlineto{\pgfqpoint{4.213880in}{1.120423in}}%
\pgfpathlineto{\pgfqpoint{4.214754in}{1.121027in}}%
\pgfpathlineto{\pgfqpoint{4.214951in}{1.121023in}}%
\pgfpathlineto{\pgfqpoint{4.215254in}{1.122004in}}%
\pgfpathlineto{\pgfqpoint{4.215593in}{1.122994in}}%
\pgfpathlineto{\pgfqpoint{4.216521in}{1.122608in}}%
\pgfpathlineto{\pgfqpoint{4.216681in}{1.122468in}}%
\pgfpathlineto{\pgfqpoint{4.216895in}{1.121628in}}%
\pgfpathlineto{\pgfqpoint{4.217270in}{1.120621in}}%
\pgfpathlineto{\pgfqpoint{4.218108in}{1.121155in}}%
\pgfpathlineto{\pgfqpoint{4.218287in}{1.121049in}}%
\pgfpathlineto{\pgfqpoint{4.218697in}{1.122283in}}%
\pgfpathlineto{\pgfqpoint{4.218965in}{1.122997in}}%
\pgfpathlineto{\pgfqpoint{4.219518in}{1.121912in}}%
\pgfpathlineto{\pgfqpoint{4.219875in}{1.122562in}}%
\pgfpathlineto{\pgfqpoint{4.220053in}{1.122446in}}%
\pgfpathlineto{\pgfqpoint{4.220285in}{1.121524in}}%
\pgfpathlineto{\pgfqpoint{4.220660in}{1.120635in}}%
\pgfpathlineto{\pgfqpoint{4.221534in}{1.121163in}}%
\pgfpathlineto{\pgfqpoint{4.222015in}{1.122218in}}%
\pgfpathlineto{\pgfqpoint{4.222372in}{1.123157in}}%
\pgfpathlineto{\pgfqpoint{4.223318in}{1.122821in}}%
\pgfpathlineto{\pgfqpoint{4.225066in}{1.121044in}}%
\pgfpathlineto{\pgfqpoint{4.225262in}{1.121474in}}%
\pgfpathlineto{\pgfqpoint{4.226690in}{1.122632in}}%
\pgfpathlineto{\pgfqpoint{4.226761in}{1.122540in}}%
\pgfpathlineto{\pgfqpoint{4.227171in}{1.120967in}}%
\pgfpathlineto{\pgfqpoint{4.227350in}{1.120640in}}%
\pgfpathlineto{\pgfqpoint{4.227885in}{1.121410in}}%
\pgfpathlineto{\pgfqpoint{4.228295in}{1.121091in}}%
\pgfpathlineto{\pgfqpoint{4.228670in}{1.121643in}}%
\pgfpathlineto{\pgfqpoint{4.228759in}{1.122056in}}%
\pgfpathlineto{\pgfqpoint{4.229098in}{1.123009in}}%
\pgfpathlineto{\pgfqpoint{4.230008in}{1.122629in}}%
\pgfpathlineto{\pgfqpoint{4.230436in}{1.121431in}}%
\pgfpathlineto{\pgfqpoint{4.230740in}{1.120658in}}%
\pgfpathlineto{\pgfqpoint{4.231614in}{1.121156in}}%
\pgfpathlineto{\pgfqpoint{4.231792in}{1.121052in}}%
\pgfpathlineto{\pgfqpoint{4.232203in}{1.122261in}}%
\pgfpathlineto{\pgfqpoint{4.232488in}{1.122900in}}%
\pgfpathlineto{\pgfqpoint{4.232970in}{1.121947in}}%
\pgfpathlineto{\pgfqpoint{4.233398in}{1.122549in}}%
\pgfpathlineto{\pgfqpoint{4.233434in}{1.122593in}}%
\pgfpathlineto{\pgfqpoint{4.233808in}{1.121389in}}%
\pgfpathlineto{\pgfqpoint{4.234147in}{1.120561in}}%
\pgfpathlineto{\pgfqpoint{4.235039in}{1.120998in}}%
\pgfpathlineto{\pgfqpoint{4.235182in}{1.121015in}}%
\pgfpathlineto{\pgfqpoint{4.235539in}{1.122097in}}%
\pgfpathlineto{\pgfqpoint{4.235860in}{1.122932in}}%
\pgfpathlineto{\pgfqpoint{4.236823in}{1.122544in}}%
\pgfpathlineto{\pgfqpoint{4.238572in}{1.121033in}}%
\pgfpathlineto{\pgfqpoint{4.238786in}{1.121568in}}%
\pgfpathlineto{\pgfqpoint{4.239250in}{1.122972in}}%
\pgfpathlineto{\pgfqpoint{4.240106in}{1.122596in}}%
\pgfpathlineto{\pgfqpoint{4.240641in}{1.121088in}}%
\pgfpathlineto{\pgfqpoint{4.240837in}{1.120635in}}%
\pgfpathlineto{\pgfqpoint{4.241373in}{1.121430in}}%
\pgfpathlineto{\pgfqpoint{4.241819in}{1.121015in}}%
\pgfpathlineto{\pgfqpoint{4.241944in}{1.120981in}}%
\pgfpathlineto{\pgfqpoint{4.242283in}{1.122006in}}%
\pgfpathlineto{\pgfqpoint{4.242604in}{1.122881in}}%
\pgfpathlineto{\pgfqpoint{4.243514in}{1.122539in}}%
\pgfpathlineto{\pgfqpoint{4.243585in}{1.122550in}}%
\pgfpathlineto{\pgfqpoint{4.243924in}{1.121437in}}%
\pgfpathlineto{\pgfqpoint{4.244263in}{1.120677in}}%
\pgfpathlineto{\pgfqpoint{4.245084in}{1.121294in}}%
\pgfpathlineto{\pgfqpoint{4.245351in}{1.121273in}}%
\pgfpathlineto{\pgfqpoint{4.245672in}{1.122276in}}%
\pgfpathlineto{\pgfqpoint{4.246029in}{1.123048in}}%
\pgfpathlineto{\pgfqpoint{4.246921in}{1.122658in}}%
\pgfpathlineto{\pgfqpoint{4.247028in}{1.122555in}}%
\pgfpathlineto{\pgfqpoint{4.247296in}{1.121620in}}%
\pgfpathlineto{\pgfqpoint{4.247653in}{1.120739in}}%
\pgfpathlineto{\pgfqpoint{4.248580in}{1.121017in}}%
\pgfpathlineto{\pgfqpoint{4.250347in}{1.122622in}}%
\pgfpathlineto{\pgfqpoint{4.250489in}{1.122325in}}%
\pgfpathlineto{\pgfqpoint{4.251025in}{1.120696in}}%
\pgfpathlineto{\pgfqpoint{4.252113in}{1.121228in}}%
\pgfpathlineto{\pgfqpoint{4.253701in}{1.122630in}}%
\pgfpathlineto{\pgfqpoint{4.253968in}{1.121959in}}%
\pgfpathlineto{\pgfqpoint{4.254361in}{1.120822in}}%
\pgfpathlineto{\pgfqpoint{4.255271in}{1.121288in}}%
\pgfpathlineto{\pgfqpoint{4.255503in}{1.121320in}}%
\pgfpathlineto{\pgfqpoint{4.255788in}{1.122263in}}%
\pgfpathlineto{\pgfqpoint{4.256127in}{1.123010in}}%
\pgfpathlineto{\pgfqpoint{4.256627in}{1.121971in}}%
\pgfpathlineto{\pgfqpoint{4.257037in}{1.122595in}}%
\pgfpathlineto{\pgfqpoint{4.257501in}{1.121162in}}%
\pgfpathlineto{\pgfqpoint{4.257750in}{1.120651in}}%
\pgfpathlineto{\pgfqpoint{4.258304in}{1.121419in}}%
\pgfpathlineto{\pgfqpoint{4.258660in}{1.121130in}}%
\pgfpathlineto{\pgfqpoint{4.258946in}{1.121347in}}%
\pgfpathlineto{\pgfqpoint{4.259142in}{1.122097in}}%
\pgfpathlineto{\pgfqpoint{4.259535in}{1.122963in}}%
\pgfpathlineto{\pgfqpoint{4.260391in}{1.122594in}}%
\pgfpathlineto{\pgfqpoint{4.260534in}{1.122529in}}%
\pgfpathlineto{\pgfqpoint{4.260801in}{1.121561in}}%
\pgfpathlineto{\pgfqpoint{4.261140in}{1.120709in}}%
\pgfpathlineto{\pgfqpoint{4.261997in}{1.121198in}}%
\pgfpathlineto{\pgfqpoint{4.262282in}{1.121297in}}%
\pgfpathlineto{\pgfqpoint{4.262532in}{1.122147in}}%
\pgfpathlineto{\pgfqpoint{4.262871in}{1.122973in}}%
\pgfpathlineto{\pgfqpoint{4.263745in}{1.122519in}}%
\pgfpathlineto{\pgfqpoint{4.263852in}{1.122554in}}%
\pgfpathlineto{\pgfqpoint{4.264209in}{1.121304in}}%
\pgfpathlineto{\pgfqpoint{4.264548in}{1.120633in}}%
\pgfpathlineto{\pgfqpoint{4.265404in}{1.121064in}}%
\pgfpathlineto{\pgfqpoint{4.267224in}{1.122518in}}%
\pgfpathlineto{\pgfqpoint{4.267492in}{1.121708in}}%
\pgfpathlineto{\pgfqpoint{4.267902in}{1.120609in}}%
\pgfpathlineto{\pgfqpoint{4.268740in}{1.121146in}}%
\pgfpathlineto{\pgfqpoint{4.269436in}{1.122628in}}%
\pgfpathlineto{\pgfqpoint{4.269650in}{1.122872in}}%
\pgfpathlineto{\pgfqpoint{4.270132in}{1.121982in}}%
\pgfpathlineto{\pgfqpoint{4.270578in}{1.122552in}}%
\pgfpathlineto{\pgfqpoint{4.270917in}{1.121499in}}%
\pgfpathlineto{\pgfqpoint{4.271220in}{1.120669in}}%
\pgfpathlineto{\pgfqpoint{4.272148in}{1.121116in}}%
\pgfpathlineto{\pgfqpoint{4.272683in}{1.122142in}}%
\pgfpathlineto{\pgfqpoint{4.273004in}{1.122786in}}%
\pgfpathlineto{\pgfqpoint{4.273486in}{1.121842in}}%
\pgfpathlineto{\pgfqpoint{4.273932in}{1.122438in}}%
\pgfpathlineto{\pgfqpoint{4.274307in}{1.121432in}}%
\pgfpathlineto{\pgfqpoint{4.274646in}{1.120635in}}%
\pgfpathlineto{\pgfqpoint{4.275556in}{1.121035in}}%
\pgfpathlineto{\pgfqpoint{4.275698in}{1.121011in}}%
\pgfpathlineto{\pgfqpoint{4.276109in}{1.122251in}}%
\pgfpathlineto{\pgfqpoint{4.276394in}{1.122818in}}%
\pgfpathlineto{\pgfqpoint{4.276876in}{1.121891in}}%
\pgfpathlineto{\pgfqpoint{4.277304in}{1.122506in}}%
\pgfpathlineto{\pgfqpoint{4.279035in}{1.121018in}}%
\pgfpathlineto{\pgfqpoint{4.279195in}{1.121271in}}%
\pgfpathlineto{\pgfqpoint{4.280729in}{1.122383in}}%
\pgfpathlineto{\pgfqpoint{4.280908in}{1.121982in}}%
\pgfpathlineto{\pgfqpoint{4.281372in}{1.120697in}}%
\pgfpathlineto{\pgfqpoint{4.282174in}{1.121274in}}%
\pgfpathlineto{\pgfqpoint{4.282906in}{1.122585in}}%
\pgfpathlineto{\pgfqpoint{4.283138in}{1.122965in}}%
\pgfpathlineto{\pgfqpoint{4.283620in}{1.121948in}}%
\pgfpathlineto{\pgfqpoint{4.284048in}{1.122512in}}%
\pgfpathlineto{\pgfqpoint{4.284315in}{1.122002in}}%
\pgfpathlineto{\pgfqpoint{4.284422in}{1.121498in}}%
\pgfpathlineto{\pgfqpoint{4.284761in}{1.120709in}}%
\pgfpathlineto{\pgfqpoint{4.285671in}{1.121188in}}%
\pgfpathlineto{\pgfqpoint{4.285832in}{1.121181in}}%
\pgfpathlineto{\pgfqpoint{4.286171in}{1.122193in}}%
\pgfpathlineto{\pgfqpoint{4.286510in}{1.122932in}}%
\pgfpathlineto{\pgfqpoint{4.287384in}{1.122526in}}%
\pgfpathlineto{\pgfqpoint{4.287491in}{1.122554in}}%
\pgfpathlineto{\pgfqpoint{4.287830in}{1.121494in}}%
\pgfpathlineto{\pgfqpoint{4.288115in}{1.120794in}}%
\pgfpathlineto{\pgfqpoint{4.289025in}{1.121273in}}%
\pgfpathlineto{\pgfqpoint{4.289168in}{1.121235in}}%
\pgfpathlineto{\pgfqpoint{4.289578in}{1.122373in}}%
\pgfpathlineto{\pgfqpoint{4.289882in}{1.122971in}}%
\pgfpathlineto{\pgfqpoint{4.290346in}{1.121948in}}%
\pgfpathlineto{\pgfqpoint{4.290792in}{1.122551in}}%
\pgfpathlineto{\pgfqpoint{4.291184in}{1.121553in}}%
\pgfpathlineto{\pgfqpoint{4.291559in}{1.120785in}}%
\pgfpathlineto{\pgfqpoint{4.292415in}{1.121150in}}%
\pgfpathlineto{\pgfqpoint{4.292593in}{1.121181in}}%
\pgfpathlineto{\pgfqpoint{4.292897in}{1.122115in}}%
\pgfpathlineto{\pgfqpoint{4.293254in}{1.122928in}}%
\pgfpathlineto{\pgfqpoint{4.294146in}{1.122594in}}%
\pgfpathlineto{\pgfqpoint{4.294181in}{1.122634in}}%
\pgfpathlineto{\pgfqpoint{4.294609in}{1.121351in}}%
\pgfpathlineto{\pgfqpoint{4.295858in}{1.121191in}}%
\pgfpathlineto{\pgfqpoint{4.295948in}{1.121210in}}%
\pgfpathlineto{\pgfqpoint{4.296322in}{1.122224in}}%
\pgfpathlineto{\pgfqpoint{4.297571in}{1.122444in}}%
\pgfpathlineto{\pgfqpoint{4.297107in}{1.121814in}}%
\pgfpathlineto{\pgfqpoint{4.297607in}{1.122411in}}%
\pgfpathlineto{\pgfqpoint{4.297785in}{1.122080in}}%
\pgfpathlineto{\pgfqpoint{4.297946in}{1.121409in}}%
\pgfpathlineto{\pgfqpoint{4.298303in}{1.120710in}}%
\pgfpathlineto{\pgfqpoint{4.299105in}{1.121270in}}%
\pgfpathlineto{\pgfqpoint{4.301050in}{1.122451in}}%
\pgfpathlineto{\pgfqpoint{4.301353in}{1.121314in}}%
\pgfpathlineto{\pgfqpoint{4.301621in}{1.120799in}}%
\pgfpathlineto{\pgfqpoint{4.302566in}{1.121237in}}%
\pgfpathlineto{\pgfqpoint{4.303155in}{1.122458in}}%
\pgfpathlineto{\pgfqpoint{4.303351in}{1.122844in}}%
\pgfpathlineto{\pgfqpoint{4.303887in}{1.121895in}}%
\pgfpathlineto{\pgfqpoint{4.304279in}{1.122390in}}%
\pgfpathlineto{\pgfqpoint{4.304404in}{1.122383in}}%
\pgfpathlineto{\pgfqpoint{4.304690in}{1.121418in}}%
\pgfpathlineto{\pgfqpoint{4.305011in}{1.120674in}}%
\pgfpathlineto{\pgfqpoint{4.305849in}{1.121277in}}%
\pgfpathlineto{\pgfqpoint{4.306242in}{1.121527in}}%
\pgfpathlineto{\pgfqpoint{4.306438in}{1.122219in}}%
\pgfpathlineto{\pgfqpoint{4.306723in}{1.122906in}}%
\pgfpathlineto{\pgfqpoint{4.307312in}{1.121912in}}%
\pgfpathlineto{\pgfqpoint{4.307598in}{1.122375in}}%
\pgfpathlineto{\pgfqpoint{4.307722in}{1.122442in}}%
\pgfpathlineto{\pgfqpoint{4.308151in}{1.121139in}}%
\pgfpathlineto{\pgfqpoint{4.308383in}{1.120793in}}%
\pgfpathlineto{\pgfqpoint{4.308864in}{1.121502in}}%
\pgfpathlineto{\pgfqpoint{4.309310in}{1.121240in}}%
\pgfpathlineto{\pgfqpoint{4.309828in}{1.122267in}}%
\pgfpathlineto{\pgfqpoint{4.310113in}{1.122881in}}%
\pgfpathlineto{\pgfqpoint{4.310648in}{1.121956in}}%
\pgfpathlineto{\pgfqpoint{4.311005in}{1.122456in}}%
\pgfpathlineto{\pgfqpoint{4.311076in}{1.122496in}}%
\pgfpathlineto{\pgfqpoint{4.311505in}{1.121191in}}%
\pgfpathlineto{\pgfqpoint{4.311719in}{1.120680in}}%
\pgfpathlineto{\pgfqpoint{4.312325in}{1.121512in}}%
\pgfpathlineto{\pgfqpoint{4.312629in}{1.121275in}}%
\pgfpathlineto{\pgfqpoint{4.313271in}{1.122485in}}%
\pgfpathlineto{\pgfqpoint{4.313521in}{1.122856in}}%
\pgfpathlineto{\pgfqpoint{4.314002in}{1.122002in}}%
\pgfpathlineto{\pgfqpoint{4.314448in}{1.122555in}}%
\pgfpathlineto{\pgfqpoint{4.316250in}{1.121255in}}%
\pgfpathlineto{\pgfqpoint{4.316625in}{1.122337in}}%
\pgfpathlineto{\pgfqpoint{4.316857in}{1.122765in}}%
\pgfpathlineto{\pgfqpoint{4.317374in}{1.121801in}}%
\pgfpathlineto{\pgfqpoint{4.317749in}{1.122288in}}%
\pgfpathlineto{\pgfqpoint{4.317856in}{1.122331in}}%
\pgfpathlineto{\pgfqpoint{4.318320in}{1.120935in}}%
\pgfpathlineto{\pgfqpoint{4.318534in}{1.120705in}}%
\pgfpathlineto{\pgfqpoint{4.319051in}{1.121450in}}%
\pgfpathlineto{\pgfqpoint{4.319408in}{1.121125in}}%
\pgfpathlineto{\pgfqpoint{4.319979in}{1.122246in}}%
\pgfpathlineto{\pgfqpoint{4.320264in}{1.122819in}}%
\pgfpathlineto{\pgfqpoint{4.320764in}{1.121912in}}%
\pgfpathlineto{\pgfqpoint{4.321157in}{1.122438in}}%
\pgfpathlineto{\pgfqpoint{4.321442in}{1.121962in}}%
\pgfpathlineto{\pgfqpoint{4.321585in}{1.121404in}}%
\pgfpathlineto{\pgfqpoint{4.321924in}{1.120823in}}%
\pgfpathlineto{\pgfqpoint{4.322423in}{1.121627in}}%
\pgfpathlineto{\pgfqpoint{4.322727in}{1.121387in}}%
\pgfpathlineto{\pgfqpoint{4.323404in}{1.122606in}}%
\pgfpathlineto{\pgfqpoint{4.323619in}{1.122976in}}%
\pgfpathlineto{\pgfqpoint{4.324118in}{1.122061in}}%
\pgfpathlineto{\pgfqpoint{4.324511in}{1.122514in}}%
\pgfpathlineto{\pgfqpoint{4.324618in}{1.122531in}}%
\pgfpathlineto{\pgfqpoint{4.324974in}{1.121423in}}%
\pgfpathlineto{\pgfqpoint{4.325260in}{1.120864in}}%
\pgfpathlineto{\pgfqpoint{4.326152in}{1.121263in}}%
\pgfpathlineto{\pgfqpoint{4.326455in}{1.121409in}}%
\pgfpathlineto{\pgfqpoint{4.326705in}{1.122166in}}%
\pgfpathlineto{\pgfqpoint{4.326973in}{1.122761in}}%
\pgfpathlineto{\pgfqpoint{4.327508in}{1.121826in}}%
\pgfpathlineto{\pgfqpoint{4.327865in}{1.122287in}}%
\pgfpathlineto{\pgfqpoint{4.328079in}{1.122233in}}%
\pgfpathlineto{\pgfqpoint{4.328328in}{1.121356in}}%
\pgfpathlineto{\pgfqpoint{4.328650in}{1.120677in}}%
\pgfpathlineto{\pgfqpoint{4.329524in}{1.121012in}}%
\pgfpathlineto{\pgfqpoint{4.330113in}{1.122150in}}%
\pgfpathlineto{\pgfqpoint{4.330362in}{1.122657in}}%
\pgfpathlineto{\pgfqpoint{4.330880in}{1.121763in}}%
\pgfpathlineto{\pgfqpoint{4.331308in}{1.122364in}}%
\pgfpathlineto{\pgfqpoint{4.331736in}{1.121281in}}%
\pgfpathlineto{\pgfqpoint{4.332075in}{1.120753in}}%
\pgfpathlineto{\pgfqpoint{4.332485in}{1.121496in}}%
\pgfpathlineto{\pgfqpoint{4.332860in}{1.121281in}}%
\pgfpathlineto{\pgfqpoint{4.333449in}{1.122253in}}%
\pgfpathlineto{\pgfqpoint{4.333788in}{1.122812in}}%
\pgfpathlineto{\pgfqpoint{4.334305in}{1.121871in}}%
\pgfpathlineto{\pgfqpoint{4.334644in}{1.122386in}}%
\pgfpathlineto{\pgfqpoint{4.334751in}{1.122407in}}%
\pgfpathlineto{\pgfqpoint{4.335126in}{1.121229in}}%
\pgfpathlineto{\pgfqpoint{4.335411in}{1.120711in}}%
\pgfpathlineto{\pgfqpoint{4.335982in}{1.121495in}}%
\pgfpathlineto{\pgfqpoint{4.336285in}{1.121198in}}%
\pgfpathlineto{\pgfqpoint{4.336892in}{1.122428in}}%
\pgfpathlineto{\pgfqpoint{4.337641in}{1.121946in}}%
\pgfpathlineto{\pgfqpoint{4.338070in}{1.122450in}}%
\pgfpathlineto{\pgfqpoint{4.338141in}{1.122453in}}%
\pgfpathlineto{\pgfqpoint{4.338462in}{1.121436in}}%
\pgfpathlineto{\pgfqpoint{4.338783in}{1.120748in}}%
\pgfpathlineto{\pgfqpoint{4.339640in}{1.121236in}}%
\pgfpathlineto{\pgfqpoint{4.340282in}{1.122420in}}%
\pgfpathlineto{\pgfqpoint{4.340478in}{1.122739in}}%
\pgfpathlineto{\pgfqpoint{4.340995in}{1.121801in}}%
\pgfpathlineto{\pgfqpoint{4.341388in}{1.122310in}}%
\pgfpathlineto{\pgfqpoint{4.341531in}{1.122296in}}%
\pgfpathlineto{\pgfqpoint{4.341870in}{1.121232in}}%
\pgfpathlineto{\pgfqpoint{4.342102in}{1.120721in}}%
\pgfpathlineto{\pgfqpoint{4.342672in}{1.121416in}}%
\pgfpathlineto{\pgfqpoint{4.343029in}{1.121124in}}%
\pgfpathlineto{\pgfqpoint{4.343743in}{1.122506in}}%
\pgfpathlineto{\pgfqpoint{4.344349in}{1.121873in}}%
\pgfpathlineto{\pgfqpoint{4.344867in}{1.122440in}}%
\pgfpathlineto{\pgfqpoint{4.345313in}{1.121119in}}%
\pgfpathlineto{\pgfqpoint{4.345509in}{1.120866in}}%
\pgfpathlineto{\pgfqpoint{4.346027in}{1.121655in}}%
\pgfpathlineto{\pgfqpoint{4.346383in}{1.121417in}}%
\pgfpathlineto{\pgfqpoint{4.346615in}{1.121431in}}%
\pgfpathlineto{\pgfqpoint{4.346990in}{1.122400in}}%
\pgfpathlineto{\pgfqpoint{4.347293in}{1.122821in}}%
\pgfpathlineto{\pgfqpoint{4.347739in}{1.121930in}}%
\pgfpathlineto{\pgfqpoint{4.348150in}{1.122368in}}%
\pgfpathlineto{\pgfqpoint{4.348685in}{1.121036in}}%
\pgfpathlineto{\pgfqpoint{4.348863in}{1.120747in}}%
\pgfpathlineto{\pgfqpoint{4.349327in}{1.121505in}}%
\pgfpathlineto{\pgfqpoint{4.349755in}{1.121316in}}%
\pgfpathlineto{\pgfqpoint{4.350344in}{1.122362in}}%
\pgfpathlineto{\pgfqpoint{4.350647in}{1.122817in}}%
\pgfpathlineto{\pgfqpoint{4.351183in}{1.121909in}}%
\pgfpathlineto{\pgfqpoint{4.351557in}{1.122443in}}%
\pgfpathlineto{\pgfqpoint{4.352075in}{1.121101in}}%
\pgfpathlineto{\pgfqpoint{4.352217in}{1.120860in}}%
\pgfpathlineto{\pgfqpoint{4.352752in}{1.121568in}}%
\pgfpathlineto{\pgfqpoint{4.353199in}{1.121240in}}%
\pgfpathlineto{\pgfqpoint{4.353288in}{1.121202in}}%
\pgfpathlineto{\pgfqpoint{4.353752in}{1.122229in}}%
\pgfpathlineto{\pgfqpoint{4.354055in}{1.122591in}}%
\pgfpathlineto{\pgfqpoint{4.354519in}{1.121793in}}%
\pgfpathlineto{\pgfqpoint{4.354911in}{1.122287in}}%
\pgfpathlineto{\pgfqpoint{4.355339in}{1.121314in}}%
\pgfpathlineto{\pgfqpoint{4.355643in}{1.120749in}}%
\pgfpathlineto{\pgfqpoint{4.356160in}{1.121499in}}%
\pgfpathlineto{\pgfqpoint{4.356463in}{1.121312in}}%
\pgfpathlineto{\pgfqpoint{4.357213in}{1.122632in}}%
\pgfpathlineto{\pgfqpoint{4.357355in}{1.122854in}}%
\pgfpathlineto{\pgfqpoint{4.357855in}{1.121988in}}%
\pgfpathlineto{\pgfqpoint{4.358319in}{1.122443in}}%
\pgfpathlineto{\pgfqpoint{4.358801in}{1.121149in}}%
\pgfpathlineto{\pgfqpoint{4.359032in}{1.120839in}}%
\pgfpathlineto{\pgfqpoint{4.359496in}{1.121543in}}%
\pgfpathlineto{\pgfqpoint{4.359942in}{1.121294in}}%
\pgfpathlineto{\pgfqpoint{4.360192in}{1.121461in}}%
\pgfpathlineto{\pgfqpoint{4.360478in}{1.122258in}}%
\pgfpathlineto{\pgfqpoint{4.360799in}{1.122695in}}%
\pgfpathlineto{\pgfqpoint{4.361245in}{1.121869in}}%
\pgfpathlineto{\pgfqpoint{4.361637in}{1.122296in}}%
\pgfpathlineto{\pgfqpoint{4.362137in}{1.121275in}}%
\pgfpathlineto{\pgfqpoint{4.362369in}{1.120783in}}%
\pgfpathlineto{\pgfqpoint{4.362904in}{1.121544in}}%
\pgfpathlineto{\pgfqpoint{4.363279in}{1.121317in}}%
\pgfpathlineto{\pgfqpoint{4.363510in}{1.121336in}}%
\pgfpathlineto{\pgfqpoint{4.363849in}{1.122263in}}%
\pgfpathlineto{\pgfqpoint{4.364171in}{1.122738in}}%
\pgfpathlineto{\pgfqpoint{4.364617in}{1.121912in}}%
\pgfpathlineto{\pgfqpoint{4.365045in}{1.122404in}}%
\pgfpathlineto{\pgfqpoint{4.365098in}{1.122444in}}%
\pgfpathlineto{\pgfqpoint{4.365616in}{1.121051in}}%
\pgfpathlineto{\pgfqpoint{4.366240in}{1.121621in}}%
\pgfpathlineto{\pgfqpoint{4.366704in}{1.121385in}}%
\pgfpathlineto{\pgfqpoint{4.368559in}{1.122238in}}%
\pgfpathlineto{\pgfqpoint{4.368881in}{1.121286in}}%
\pgfpathlineto{\pgfqpoint{4.369095in}{1.120777in}}%
\pgfpathlineto{\pgfqpoint{4.369612in}{1.121510in}}%
\pgfpathlineto{\pgfqpoint{4.370022in}{1.121276in}}%
\pgfpathlineto{\pgfqpoint{4.370647in}{1.122389in}}%
\pgfpathlineto{\pgfqpoint{4.370879in}{1.122748in}}%
\pgfpathlineto{\pgfqpoint{4.371414in}{1.121874in}}%
\pgfpathlineto{\pgfqpoint{4.371771in}{1.122325in}}%
\pgfpathlineto{\pgfqpoint{4.372270in}{1.121246in}}%
\pgfpathlineto{\pgfqpoint{4.372520in}{1.120860in}}%
\pgfpathlineto{\pgfqpoint{4.373020in}{1.121541in}}%
\pgfpathlineto{\pgfqpoint{4.373430in}{1.121262in}}%
\pgfpathlineto{\pgfqpoint{4.373537in}{1.121236in}}%
\pgfpathlineto{\pgfqpoint{4.374019in}{1.122393in}}%
\pgfpathlineto{\pgfqpoint{4.374768in}{1.121763in}}%
\pgfpathlineto{\pgfqpoint{4.375178in}{1.122227in}}%
\pgfpathlineto{\pgfqpoint{4.375660in}{1.121130in}}%
\pgfpathlineto{\pgfqpoint{4.375856in}{1.120785in}}%
\pgfpathlineto{\pgfqpoint{4.376391in}{1.121499in}}%
\pgfpathlineto{\pgfqpoint{4.376784in}{1.121315in}}%
\pgfpathlineto{\pgfqpoint{4.377052in}{1.121463in}}%
\pgfpathlineto{\pgfqpoint{4.377337in}{1.122238in}}%
\pgfpathlineto{\pgfqpoint{4.377640in}{1.122695in}}%
\pgfpathlineto{\pgfqpoint{4.378122in}{1.121837in}}%
\pgfpathlineto{\pgfqpoint{4.378532in}{1.122286in}}%
\pgfpathlineto{\pgfqpoint{4.378639in}{1.122290in}}%
\pgfpathlineto{\pgfqpoint{4.379014in}{1.121196in}}%
\pgfpathlineto{\pgfqpoint{4.379264in}{1.120746in}}%
\pgfpathlineto{\pgfqpoint{4.379799in}{1.121502in}}%
\pgfpathlineto{\pgfqpoint{4.380138in}{1.121256in}}%
\pgfpathlineto{\pgfqpoint{4.382118in}{1.122172in}}%
\pgfpathlineto{\pgfqpoint{4.382439in}{1.121105in}}%
\pgfpathlineto{\pgfqpoint{4.382654in}{1.120821in}}%
\pgfpathlineto{\pgfqpoint{4.383207in}{1.121618in}}%
\pgfpathlineto{\pgfqpoint{4.383510in}{1.121389in}}%
\pgfpathlineto{\pgfqpoint{4.383778in}{1.121424in}}%
\pgfpathlineto{\pgfqpoint{4.384117in}{1.122319in}}%
\pgfpathlineto{\pgfqpoint{4.384402in}{1.122781in}}%
\pgfpathlineto{\pgfqpoint{4.384919in}{1.121943in}}%
\pgfpathlineto{\pgfqpoint{4.385294in}{1.122372in}}%
\pgfpathlineto{\pgfqpoint{4.385348in}{1.122396in}}%
\pgfpathlineto{\pgfqpoint{4.385776in}{1.121248in}}%
\pgfpathlineto{\pgfqpoint{4.386061in}{1.120897in}}%
\pgfpathlineto{\pgfqpoint{4.386561in}{1.121545in}}%
\pgfpathlineto{\pgfqpoint{4.386953in}{1.121237in}}%
\pgfpathlineto{\pgfqpoint{4.387524in}{1.122377in}}%
\pgfpathlineto{\pgfqpoint{4.388256in}{1.121911in}}%
\pgfpathlineto{\pgfqpoint{4.388719in}{1.122407in}}%
\pgfpathlineto{\pgfqpoint{4.390557in}{1.121441in}}%
\pgfpathlineto{\pgfqpoint{4.390825in}{1.122135in}}%
\pgfpathlineto{\pgfqpoint{4.391146in}{1.122648in}}%
\pgfpathlineto{\pgfqpoint{4.391645in}{1.121823in}}%
\pgfpathlineto{\pgfqpoint{4.392038in}{1.122218in}}%
\pgfpathlineto{\pgfqpoint{4.392644in}{1.120868in}}%
\pgfpathlineto{\pgfqpoint{4.392787in}{1.120783in}}%
\pgfpathlineto{\pgfqpoint{4.393287in}{1.121572in}}%
\pgfpathlineto{\pgfqpoint{4.393643in}{1.121351in}}%
\pgfpathlineto{\pgfqpoint{4.393840in}{1.121359in}}%
\pgfpathlineto{\pgfqpoint{4.394268in}{1.122386in}}%
\pgfpathlineto{\pgfqpoint{4.395481in}{1.122409in}}%
\pgfpathlineto{\pgfqpoint{4.395035in}{1.121916in}}%
\pgfpathlineto{\pgfqpoint{4.395499in}{1.122379in}}%
\pgfpathlineto{\pgfqpoint{4.397265in}{1.121362in}}%
\pgfpathlineto{\pgfqpoint{4.397461in}{1.121816in}}%
\pgfpathlineto{\pgfqpoint{4.397890in}{1.122642in}}%
\pgfpathlineto{\pgfqpoint{4.398710in}{1.122155in}}%
\pgfpathlineto{\pgfqpoint{4.398889in}{1.122258in}}%
\pgfpathlineto{\pgfqpoint{4.399317in}{1.121128in}}%
\pgfpathlineto{\pgfqpoint{4.399549in}{1.120857in}}%
\pgfpathlineto{\pgfqpoint{4.400013in}{1.121543in}}%
\pgfpathlineto{\pgfqpoint{4.400459in}{1.121275in}}%
\pgfpathlineto{\pgfqpoint{4.400976in}{1.122306in}}%
\pgfpathlineto{\pgfqpoint{4.401244in}{1.122740in}}%
\pgfpathlineto{\pgfqpoint{4.401815in}{1.121930in}}%
\pgfpathlineto{\pgfqpoint{4.402082in}{1.122208in}}%
\pgfpathlineto{\pgfqpoint{4.402278in}{1.122328in}}%
\pgfpathlineto{\pgfqpoint{4.402671in}{1.121169in}}%
\pgfpathlineto{\pgfqpoint{4.402867in}{1.120832in}}%
\pgfpathlineto{\pgfqpoint{4.403385in}{1.121579in}}%
\pgfpathlineto{\pgfqpoint{4.403777in}{1.121354in}}%
\pgfpathlineto{\pgfqpoint{4.404187in}{1.121729in}}%
\pgfpathlineto{\pgfqpoint{4.404366in}{1.122263in}}%
\pgfpathlineto{\pgfqpoint{4.404669in}{1.122683in}}%
\pgfpathlineto{\pgfqpoint{4.405133in}{1.121854in}}%
\pgfpathlineto{\pgfqpoint{4.405561in}{1.122328in}}%
\pgfpathlineto{\pgfqpoint{4.406078in}{1.121218in}}%
\pgfpathlineto{\pgfqpoint{4.406739in}{1.121611in}}%
\pgfpathlineto{\pgfqpoint{4.407256in}{1.121312in}}%
\pgfpathlineto{\pgfqpoint{4.408969in}{1.122254in}}%
\pgfpathlineto{\pgfqpoint{4.409147in}{1.121994in}}%
\pgfpathlineto{\pgfqpoint{4.409629in}{1.120823in}}%
\pgfpathlineto{\pgfqpoint{4.410521in}{1.121314in}}%
\pgfpathlineto{\pgfqpoint{4.411145in}{1.122317in}}%
\pgfpathlineto{\pgfqpoint{4.412341in}{1.122293in}}%
\pgfpathlineto{\pgfqpoint{4.411895in}{1.121843in}}%
\pgfpathlineto{\pgfqpoint{4.412394in}{1.122260in}}%
\pgfpathlineto{\pgfqpoint{4.414071in}{1.121328in}}%
\pgfpathlineto{\pgfqpoint{4.414303in}{1.121774in}}%
\pgfpathlineto{\pgfqpoint{4.414803in}{1.122592in}}%
\pgfpathlineto{\pgfqpoint{4.415588in}{1.122122in}}%
\pgfpathlineto{\pgfqpoint{4.415766in}{1.122213in}}%
\pgfpathlineto{\pgfqpoint{4.416176in}{1.121177in}}%
\pgfpathlineto{\pgfqpoint{4.416480in}{1.120896in}}%
\pgfpathlineto{\pgfqpoint{4.416890in}{1.121612in}}%
\pgfpathlineto{\pgfqpoint{4.417265in}{1.121387in}}%
\pgfpathlineto{\pgfqpoint{4.418014in}{1.122583in}}%
\pgfpathlineto{\pgfqpoint{4.419084in}{1.122271in}}%
\pgfpathlineto{\pgfqpoint{4.418674in}{1.121850in}}%
\pgfpathlineto{\pgfqpoint{4.419102in}{1.122253in}}%
\pgfpathlineto{\pgfqpoint{4.419584in}{1.121127in}}%
\pgfpathlineto{\pgfqpoint{4.419780in}{1.120824in}}%
\pgfpathlineto{\pgfqpoint{4.420280in}{1.121599in}}%
\pgfpathlineto{\pgfqpoint{4.420654in}{1.121385in}}%
\pgfpathlineto{\pgfqpoint{4.423009in}{1.120874in}}%
\pgfpathlineto{\pgfqpoint{4.423152in}{1.120782in}}%
\pgfpathlineto{\pgfqpoint{4.423616in}{1.121503in}}%
\pgfpathlineto{\pgfqpoint{4.424026in}{1.121303in}}%
\pgfpathlineto{\pgfqpoint{4.424330in}{1.121445in}}%
\pgfpathlineto{\pgfqpoint{4.424615in}{1.122217in}}%
\pgfpathlineto{\pgfqpoint{4.424883in}{1.122626in}}%
\pgfpathlineto{\pgfqpoint{4.425364in}{1.121872in}}%
\pgfpathlineto{\pgfqpoint{4.425810in}{1.122289in}}%
\pgfpathlineto{\pgfqpoint{4.426310in}{1.121224in}}%
\pgfpathlineto{\pgfqpoint{4.427541in}{1.121395in}}%
\pgfpathlineto{\pgfqpoint{4.429289in}{1.122167in}}%
\pgfpathlineto{\pgfqpoint{4.429432in}{1.121891in}}%
\pgfpathlineto{\pgfqpoint{4.429896in}{1.120822in}}%
\pgfpathlineto{\pgfqpoint{4.430663in}{1.121539in}}%
\pgfpathlineto{\pgfqpoint{4.430984in}{1.121470in}}%
\pgfpathlineto{\pgfqpoint{4.431377in}{1.122349in}}%
\pgfpathlineto{\pgfqpoint{4.432126in}{1.122026in}}%
\pgfpathlineto{\pgfqpoint{4.432572in}{1.122403in}}%
\pgfpathlineto{\pgfqpoint{4.433107in}{1.121249in}}%
\pgfpathlineto{\pgfqpoint{4.434320in}{1.121352in}}%
\pgfpathlineto{\pgfqpoint{4.433696in}{1.121649in}}%
\pgfpathlineto{\pgfqpoint{4.434338in}{1.121380in}}%
\pgfpathlineto{\pgfqpoint{4.436051in}{1.122172in}}%
\pgfpathlineto{\pgfqpoint{4.436158in}{1.121989in}}%
\pgfpathlineto{\pgfqpoint{4.436640in}{1.120934in}}%
\pgfpathlineto{\pgfqpoint{4.437585in}{1.121407in}}%
\pgfpathlineto{\pgfqpoint{4.439441in}{1.122199in}}%
\pgfpathlineto{\pgfqpoint{4.439619in}{1.121756in}}%
\pgfpathlineto{\pgfqpoint{4.440029in}{1.120855in}}%
\pgfpathlineto{\pgfqpoint{4.440975in}{1.121350in}}%
\pgfpathlineto{\pgfqpoint{4.446684in}{1.120898in}}%
\pgfpathlineto{\pgfqpoint{4.447808in}{1.121275in}}%
\pgfpathlineto{\pgfqpoint{4.447166in}{1.121431in}}%
\pgfpathlineto{\pgfqpoint{4.447826in}{1.121303in}}%
\pgfpathlineto{\pgfqpoint{4.449574in}{1.122174in}}%
\pgfpathlineto{\pgfqpoint{4.449842in}{1.121522in}}%
\pgfpathlineto{\pgfqpoint{4.450163in}{1.120969in}}%
\pgfpathlineto{\pgfqpoint{4.451019in}{1.121370in}}%
\pgfpathlineto{\pgfqpoint{4.452946in}{1.122201in}}%
\pgfpathlineto{\pgfqpoint{4.453125in}{1.121799in}}%
\pgfpathlineto{\pgfqpoint{4.453571in}{1.120906in}}%
\pgfpathlineto{\pgfqpoint{4.454409in}{1.121333in}}%
\pgfpathlineto{\pgfqpoint{4.454873in}{1.121883in}}%
\pgfpathlineto{\pgfqpoint{4.455034in}{1.122250in}}%
\pgfpathlineto{\pgfqpoint{4.456211in}{1.122286in}}%
\pgfpathlineto{\pgfqpoint{4.455729in}{1.121837in}}%
\pgfpathlineto{\pgfqpoint{4.456229in}{1.122252in}}%
\pgfpathlineto{\pgfqpoint{4.456728in}{1.121191in}}%
\pgfpathlineto{\pgfqpoint{4.456889in}{1.121035in}}%
\pgfpathlineto{\pgfqpoint{4.457371in}{1.121733in}}%
\pgfpathlineto{\pgfqpoint{4.457799in}{1.121498in}}%
\pgfpathlineto{\pgfqpoint{4.457995in}{1.121477in}}%
\pgfpathlineto{\pgfqpoint{4.458441in}{1.122371in}}%
\pgfpathlineto{\pgfqpoint{4.458655in}{1.122547in}}%
\pgfpathlineto{\pgfqpoint{4.459083in}{1.121782in}}%
\pgfpathlineto{\pgfqpoint{4.459512in}{1.122065in}}%
\pgfpathlineto{\pgfqpoint{4.459619in}{1.122107in}}%
\pgfpathlineto{\pgfqpoint{4.460136in}{1.120975in}}%
\pgfpathlineto{\pgfqpoint{4.460743in}{1.121571in}}%
\pgfpathlineto{\pgfqpoint{4.461278in}{1.121381in}}%
\pgfpathlineto{\pgfqpoint{4.463062in}{1.122183in}}%
\pgfpathlineto{\pgfqpoint{4.463294in}{1.121630in}}%
\pgfpathlineto{\pgfqpoint{4.463615in}{1.120897in}}%
\pgfpathlineto{\pgfqpoint{4.464471in}{1.121537in}}%
\pgfpathlineto{\pgfqpoint{4.899018in}{1.121218in}}%
\pgfpathlineto{\pgfqpoint{4.900178in}{1.121613in}}%
\pgfpathlineto{\pgfqpoint{4.915789in}{1.121474in}}%
\pgfpathlineto{\pgfqpoint{4.917145in}{1.121771in}}%
\pgfpathlineto{\pgfqpoint{4.917180in}{1.121825in}}%
\pgfpathlineto{\pgfqpoint{4.918768in}{1.121866in}}%
\pgfpathlineto{\pgfqpoint{4.918786in}{1.121844in}}%
\pgfpathlineto{\pgfqpoint{4.920873in}{1.122063in}}%
\pgfpathlineto{\pgfqpoint{4.921480in}{1.121651in}}%
\pgfpathlineto{\pgfqpoint{4.921962in}{1.121721in}}%
\pgfpathlineto{\pgfqpoint{4.923496in}{1.121836in}}%
\pgfpathlineto{\pgfqpoint{4.925851in}{1.121483in}}%
\pgfpathlineto{\pgfqpoint{4.927171in}{1.121609in}}%
\pgfpathlineto{\pgfqpoint{4.927189in}{1.121629in}}%
\pgfpathlineto{\pgfqpoint{4.928723in}{1.121839in}}%
\pgfpathlineto{\pgfqpoint{4.954646in}{1.122062in}}%
\pgfpathlineto{\pgfqpoint{4.955948in}{1.121799in}}%
\pgfpathlineto{\pgfqpoint{4.957251in}{1.121729in}}%
\pgfpathlineto{\pgfqpoint{4.960266in}{1.121811in}}%
\pgfpathlineto{\pgfqpoint{4.961568in}{1.122143in}}%
\pgfpathlineto{\pgfqpoint{4.961639in}{1.122065in}}%
\pgfpathlineto{\pgfqpoint{4.963763in}{1.121838in}}%
\pgfpathlineto{\pgfqpoint{5.233069in}{1.121497in}}%
\pgfpathlineto{\pgfqpoint{5.234228in}{1.121650in}}%
\pgfpathlineto{\pgfqpoint{5.236066in}{1.121802in}}%
\pgfpathlineto{\pgfqpoint{5.258296in}{1.121911in}}%
\pgfpathlineto{\pgfqpoint{5.259562in}{1.121764in}}%
\pgfpathlineto{\pgfqpoint{5.817086in}{1.121691in}}%
\pgfpathlineto{\pgfqpoint{5.818424in}{1.121726in}}%
\pgfpathlineto{\pgfqpoint{5.883453in}{1.121763in}}%
\pgfpathlineto{\pgfqpoint{5.884738in}{1.121689in}}%
\pgfpathlineto{\pgfqpoint{5.918903in}{1.121801in}}%
\pgfpathlineto{\pgfqpoint{5.927324in}{1.121732in}}%
\pgfpathlineto{\pgfqpoint{5.930035in}{1.121800in}}%
\pgfpathlineto{\pgfqpoint{5.940008in}{1.121766in}}%
\pgfpathlineto{\pgfqpoint{5.945379in}{1.121726in}}%
\pgfpathlineto{\pgfqpoint{5.985752in}{1.121765in}}%
\pgfpathlineto{\pgfqpoint{5.987037in}{1.121763in}}%
\pgfpathlineto{\pgfqpoint{5.997652in}{1.121764in}}%
\pgfpathlineto{\pgfqpoint{6.377143in}{1.121771in}}%
\pgfpathlineto{\pgfqpoint{6.378409in}{1.121800in}}%
\pgfpathlineto{\pgfqpoint{6.386723in}{1.121764in}}%
\pgfpathlineto{\pgfqpoint{6.675957in}{1.121764in}}%
\pgfpathlineto{\pgfqpoint{6.675957in}{1.121764in}}%
\pgfusepath{stroke}%
\end{pgfscope}%
\begin{pgfscope}%
\pgfsetrectcap%
\pgfsetmiterjoin%
\pgfsetlinewidth{0.803000pt}%
\definecolor{currentstroke}{rgb}{0.000000,0.000000,0.000000}%
\pgfsetstrokecolor{currentstroke}%
\pgfsetdash{}{0pt}%
\pgfpathmoveto{\pgfqpoint{0.746130in}{0.463273in}}%
\pgfpathlineto{\pgfqpoint{0.746130in}{1.748369in}}%
\pgfusepath{stroke}%
\end{pgfscope}%
\begin{pgfscope}%
\pgfsetrectcap%
\pgfsetmiterjoin%
\pgfsetlinewidth{0.803000pt}%
\definecolor{currentstroke}{rgb}{0.000000,0.000000,0.000000}%
\pgfsetstrokecolor{currentstroke}%
\pgfsetdash{}{0pt}%
\pgfpathmoveto{\pgfqpoint{6.958330in}{0.463273in}}%
\pgfpathlineto{\pgfqpoint{6.958330in}{1.748369in}}%
\pgfusepath{stroke}%
\end{pgfscope}%
\begin{pgfscope}%
\pgfsetrectcap%
\pgfsetmiterjoin%
\pgfsetlinewidth{0.803000pt}%
\definecolor{currentstroke}{rgb}{0.000000,0.000000,0.000000}%
\pgfsetstrokecolor{currentstroke}%
\pgfsetdash{}{0pt}%
\pgfpathmoveto{\pgfqpoint{0.746130in}{0.463273in}}%
\pgfpathlineto{\pgfqpoint{6.958330in}{0.463273in}}%
\pgfusepath{stroke}%
\end{pgfscope}%
\begin{pgfscope}%
\pgfsetrectcap%
\pgfsetmiterjoin%
\pgfsetlinewidth{0.803000pt}%
\definecolor{currentstroke}{rgb}{0.000000,0.000000,0.000000}%
\pgfsetstrokecolor{currentstroke}%
\pgfsetdash{}{0pt}%
\pgfpathmoveto{\pgfqpoint{0.746130in}{1.748369in}}%
\pgfpathlineto{\pgfqpoint{6.958330in}{1.748369in}}%
\pgfusepath{stroke}%
\end{pgfscope}%
\begin{pgfscope}%
\definecolor{textcolor}{rgb}{0.000000,0.000000,0.000000}%
\pgfsetstrokecolor{textcolor}%
\pgfsetfillcolor{textcolor}%
\pgftext[x=3.852230in,y=1.831702in,,base]{\color{textcolor}\sffamily\fontsize{12.000000}{14.400000}\selectfont Ordinario chitarra La2 Fortissimo}%
\end{pgfscope}%
\end{pgfpicture}%
\makeatother%
\endgroup%
}
    \end{center}
    \caption{\emph{Forme d'onda} o \emph{Audiogrammi} : nell'asse delle ascisse viene rappresentata la durata in millisecondi mentre nell'asse delle ordinate l'ampiezza del segnale.}
\end{figure}



\section{Misure dell'ampiezza}

\begin{figure}
    \begin{center}
       \scalebox{0.6} {%% Creator: Matplotlib, PGF backend
%%
%% To include the figure in your LaTeX document, write
%%   \input{<filename>.pgf}
%%
%% Make sure the required packages are loaded in your preamble
%%   \usepackage{pgf}
%%
%% Also ensure that all the required font packages are loaded; for instance,
%% the lmodern package is sometimes necessary when using math font.
%%   \usepackage{lmodern}
%%
%% Figures using additional raster images can only be included by \input if
%% they are in the same directory as the main LaTeX file. For loading figures
%% from other directories you can use the `import` package
%%   \usepackage{import}
%%
%% and then include the figures with
%%   \import{<path to file>}{<filename>.pgf}
%%
%% Matplotlib used the following preamble
%%   
%%   \makeatletter\@ifpackageloaded{underscore}{}{\usepackage[strings]{underscore}}\makeatother
%%
\begingroup%
\makeatletter%
\begin{pgfpicture}%
\pgfpathrectangle{\pgfpointorigin}{\pgfqpoint{7.000000in}{4.000000in}}%
\pgfusepath{use as bounding box, clip}%
\begin{pgfscope}%
\pgfsetbuttcap%
\pgfsetmiterjoin%
\definecolor{currentfill}{rgb}{1.000000,1.000000,1.000000}%
\pgfsetfillcolor{currentfill}%
\pgfsetlinewidth{0.000000pt}%
\definecolor{currentstroke}{rgb}{1.000000,1.000000,1.000000}%
\pgfsetstrokecolor{currentstroke}%
\pgfsetdash{}{0pt}%
\pgfpathmoveto{\pgfqpoint{0.000000in}{0.000000in}}%
\pgfpathlineto{\pgfqpoint{7.000000in}{0.000000in}}%
\pgfpathlineto{\pgfqpoint{7.000000in}{4.000000in}}%
\pgfpathlineto{\pgfqpoint{0.000000in}{4.000000in}}%
\pgfpathlineto{\pgfqpoint{0.000000in}{0.000000in}}%
\pgfpathclose%
\pgfusepath{fill}%
\end{pgfscope}%
\begin{pgfscope}%
\pgfsetbuttcap%
\pgfsetmiterjoin%
\definecolor{currentfill}{rgb}{1.000000,1.000000,1.000000}%
\pgfsetfillcolor{currentfill}%
\pgfsetlinewidth{0.000000pt}%
\definecolor{currentstroke}{rgb}{0.000000,0.000000,0.000000}%
\pgfsetstrokecolor{currentstroke}%
\pgfsetstrokeopacity{0.000000}%
\pgfsetdash{}{0pt}%
\pgfpathmoveto{\pgfqpoint{0.875000in}{0.440000in}}%
\pgfpathlineto{\pgfqpoint{6.300000in}{0.440000in}}%
\pgfpathlineto{\pgfqpoint{6.300000in}{3.520000in}}%
\pgfpathlineto{\pgfqpoint{0.875000in}{3.520000in}}%
\pgfpathlineto{\pgfqpoint{0.875000in}{0.440000in}}%
\pgfpathclose%
\pgfusepath{fill}%
\end{pgfscope}%
\begin{pgfscope}%
\pgfsetbuttcap%
\pgfsetroundjoin%
\definecolor{currentfill}{rgb}{0.000000,0.000000,0.000000}%
\pgfsetfillcolor{currentfill}%
\pgfsetlinewidth{0.803000pt}%
\definecolor{currentstroke}{rgb}{0.000000,0.000000,0.000000}%
\pgfsetstrokecolor{currentstroke}%
\pgfsetdash{}{0pt}%
\pgfsys@defobject{currentmarker}{\pgfqpoint{0.000000in}{-0.048611in}}{\pgfqpoint{0.000000in}{0.000000in}}{%
\pgfpathmoveto{\pgfqpoint{0.000000in}{0.000000in}}%
\pgfpathlineto{\pgfqpoint{0.000000in}{-0.048611in}}%
\pgfusepath{stroke,fill}%
}%
\begin{pgfscope}%
\pgfsys@transformshift{1.121591in}{1.980000in}%
\pgfsys@useobject{currentmarker}{}%
\end{pgfscope}%
\end{pgfscope}%
\begin{pgfscope}%
\definecolor{textcolor}{rgb}{0.000000,0.000000,0.000000}%
\pgfsetstrokecolor{textcolor}%
\pgfsetfillcolor{textcolor}%
\pgftext[x=1.121591in,y=1.882778in,,top]{\color{textcolor}\rmfamily\fontsize{10.000000}{12.000000}\selectfont \(\displaystyle {0.0}\)}%
\end{pgfscope}%
\begin{pgfscope}%
\pgfsetbuttcap%
\pgfsetroundjoin%
\definecolor{currentfill}{rgb}{0.000000,0.000000,0.000000}%
\pgfsetfillcolor{currentfill}%
\pgfsetlinewidth{0.803000pt}%
\definecolor{currentstroke}{rgb}{0.000000,0.000000,0.000000}%
\pgfsetstrokecolor{currentstroke}%
\pgfsetdash{}{0pt}%
\pgfsys@defobject{currentmarker}{\pgfqpoint{0.000000in}{-0.048611in}}{\pgfqpoint{0.000000in}{0.000000in}}{%
\pgfpathmoveto{\pgfqpoint{0.000000in}{0.000000in}}%
\pgfpathlineto{\pgfqpoint{0.000000in}{-0.048611in}}%
\pgfusepath{stroke,fill}%
}%
\begin{pgfscope}%
\pgfsys@transformshift{2.107955in}{1.980000in}%
\pgfsys@useobject{currentmarker}{}%
\end{pgfscope}%
\end{pgfscope}%
\begin{pgfscope}%
\definecolor{textcolor}{rgb}{0.000000,0.000000,0.000000}%
\pgfsetstrokecolor{textcolor}%
\pgfsetfillcolor{textcolor}%
\pgftext[x=2.107955in,y=1.882778in,,top]{\color{textcolor}\rmfamily\fontsize{10.000000}{12.000000}\selectfont \(\displaystyle {0.2}\)}%
\end{pgfscope}%
\begin{pgfscope}%
\pgfsetbuttcap%
\pgfsetroundjoin%
\definecolor{currentfill}{rgb}{0.000000,0.000000,0.000000}%
\pgfsetfillcolor{currentfill}%
\pgfsetlinewidth{0.803000pt}%
\definecolor{currentstroke}{rgb}{0.000000,0.000000,0.000000}%
\pgfsetstrokecolor{currentstroke}%
\pgfsetdash{}{0pt}%
\pgfsys@defobject{currentmarker}{\pgfqpoint{0.000000in}{-0.048611in}}{\pgfqpoint{0.000000in}{0.000000in}}{%
\pgfpathmoveto{\pgfqpoint{0.000000in}{0.000000in}}%
\pgfpathlineto{\pgfqpoint{0.000000in}{-0.048611in}}%
\pgfusepath{stroke,fill}%
}%
\begin{pgfscope}%
\pgfsys@transformshift{3.094318in}{1.980000in}%
\pgfsys@useobject{currentmarker}{}%
\end{pgfscope}%
\end{pgfscope}%
\begin{pgfscope}%
\definecolor{textcolor}{rgb}{0.000000,0.000000,0.000000}%
\pgfsetstrokecolor{textcolor}%
\pgfsetfillcolor{textcolor}%
\pgftext[x=3.094318in,y=1.882778in,,top]{\color{textcolor}\rmfamily\fontsize{10.000000}{12.000000}\selectfont \(\displaystyle {0.4}\)}%
\end{pgfscope}%
\begin{pgfscope}%
\pgfsetbuttcap%
\pgfsetroundjoin%
\definecolor{currentfill}{rgb}{0.000000,0.000000,0.000000}%
\pgfsetfillcolor{currentfill}%
\pgfsetlinewidth{0.803000pt}%
\definecolor{currentstroke}{rgb}{0.000000,0.000000,0.000000}%
\pgfsetstrokecolor{currentstroke}%
\pgfsetdash{}{0pt}%
\pgfsys@defobject{currentmarker}{\pgfqpoint{0.000000in}{-0.048611in}}{\pgfqpoint{0.000000in}{0.000000in}}{%
\pgfpathmoveto{\pgfqpoint{0.000000in}{0.000000in}}%
\pgfpathlineto{\pgfqpoint{0.000000in}{-0.048611in}}%
\pgfusepath{stroke,fill}%
}%
\begin{pgfscope}%
\pgfsys@transformshift{4.080682in}{1.980000in}%
\pgfsys@useobject{currentmarker}{}%
\end{pgfscope}%
\end{pgfscope}%
\begin{pgfscope}%
\definecolor{textcolor}{rgb}{0.000000,0.000000,0.000000}%
\pgfsetstrokecolor{textcolor}%
\pgfsetfillcolor{textcolor}%
\pgftext[x=4.080682in,y=1.882778in,,top]{\color{textcolor}\rmfamily\fontsize{10.000000}{12.000000}\selectfont \(\displaystyle {0.6}\)}%
\end{pgfscope}%
\begin{pgfscope}%
\pgfsetbuttcap%
\pgfsetroundjoin%
\definecolor{currentfill}{rgb}{0.000000,0.000000,0.000000}%
\pgfsetfillcolor{currentfill}%
\pgfsetlinewidth{0.803000pt}%
\definecolor{currentstroke}{rgb}{0.000000,0.000000,0.000000}%
\pgfsetstrokecolor{currentstroke}%
\pgfsetdash{}{0pt}%
\pgfsys@defobject{currentmarker}{\pgfqpoint{0.000000in}{-0.048611in}}{\pgfqpoint{0.000000in}{0.000000in}}{%
\pgfpathmoveto{\pgfqpoint{0.000000in}{0.000000in}}%
\pgfpathlineto{\pgfqpoint{0.000000in}{-0.048611in}}%
\pgfusepath{stroke,fill}%
}%
\begin{pgfscope}%
\pgfsys@transformshift{5.067045in}{1.980000in}%
\pgfsys@useobject{currentmarker}{}%
\end{pgfscope}%
\end{pgfscope}%
\begin{pgfscope}%
\definecolor{textcolor}{rgb}{0.000000,0.000000,0.000000}%
\pgfsetstrokecolor{textcolor}%
\pgfsetfillcolor{textcolor}%
\pgftext[x=5.067045in,y=1.882778in,,top]{\color{textcolor}\rmfamily\fontsize{10.000000}{12.000000}\selectfont \(\displaystyle {0.8}\)}%
\end{pgfscope}%
\begin{pgfscope}%
\pgfsetbuttcap%
\pgfsetroundjoin%
\definecolor{currentfill}{rgb}{0.000000,0.000000,0.000000}%
\pgfsetfillcolor{currentfill}%
\pgfsetlinewidth{0.803000pt}%
\definecolor{currentstroke}{rgb}{0.000000,0.000000,0.000000}%
\pgfsetstrokecolor{currentstroke}%
\pgfsetdash{}{0pt}%
\pgfsys@defobject{currentmarker}{\pgfqpoint{0.000000in}{-0.048611in}}{\pgfqpoint{0.000000in}{0.000000in}}{%
\pgfpathmoveto{\pgfqpoint{0.000000in}{0.000000in}}%
\pgfpathlineto{\pgfqpoint{0.000000in}{-0.048611in}}%
\pgfusepath{stroke,fill}%
}%
\begin{pgfscope}%
\pgfsys@transformshift{6.053409in}{1.980000in}%
\pgfsys@useobject{currentmarker}{}%
\end{pgfscope}%
\end{pgfscope}%
\begin{pgfscope}%
\definecolor{textcolor}{rgb}{0.000000,0.000000,0.000000}%
\pgfsetstrokecolor{textcolor}%
\pgfsetfillcolor{textcolor}%
\pgftext[x=6.053409in,y=1.882778in,,top]{\color{textcolor}\rmfamily\fontsize{10.000000}{12.000000}\selectfont \(\displaystyle {1.0}\)}%
\end{pgfscope}%
\begin{pgfscope}%
\pgfsetbuttcap%
\pgfsetroundjoin%
\definecolor{currentfill}{rgb}{0.000000,0.000000,0.000000}%
\pgfsetfillcolor{currentfill}%
\pgfsetlinewidth{0.803000pt}%
\definecolor{currentstroke}{rgb}{0.000000,0.000000,0.000000}%
\pgfsetstrokecolor{currentstroke}%
\pgfsetdash{}{0pt}%
\pgfsys@defobject{currentmarker}{\pgfqpoint{-0.048611in}{0.000000in}}{\pgfqpoint{-0.000000in}{0.000000in}}{%
\pgfpathmoveto{\pgfqpoint{-0.000000in}{0.000000in}}%
\pgfpathlineto{\pgfqpoint{-0.048611in}{0.000000in}}%
\pgfusepath{stroke,fill}%
}%
\begin{pgfscope}%
\pgfsys@transformshift{1.121591in}{0.580000in}%
\pgfsys@useobject{currentmarker}{}%
\end{pgfscope}%
\end{pgfscope}%
\begin{pgfscope}%
\definecolor{textcolor}{rgb}{0.000000,0.000000,0.000000}%
\pgfsetstrokecolor{textcolor}%
\pgfsetfillcolor{textcolor}%
\pgftext[x=0.669429in, y=0.531775in, left, base]{\color{textcolor}\rmfamily\fontsize{10.000000}{12.000000}\selectfont \(\displaystyle {\ensuremath{-}1.00}\)}%
\end{pgfscope}%
\begin{pgfscope}%
\pgfsetbuttcap%
\pgfsetroundjoin%
\definecolor{currentfill}{rgb}{0.000000,0.000000,0.000000}%
\pgfsetfillcolor{currentfill}%
\pgfsetlinewidth{0.803000pt}%
\definecolor{currentstroke}{rgb}{0.000000,0.000000,0.000000}%
\pgfsetstrokecolor{currentstroke}%
\pgfsetdash{}{0pt}%
\pgfsys@defobject{currentmarker}{\pgfqpoint{-0.048611in}{0.000000in}}{\pgfqpoint{-0.000000in}{0.000000in}}{%
\pgfpathmoveto{\pgfqpoint{-0.000000in}{0.000000in}}%
\pgfpathlineto{\pgfqpoint{-0.048611in}{0.000000in}}%
\pgfusepath{stroke,fill}%
}%
\begin{pgfscope}%
\pgfsys@transformshift{1.121591in}{0.930000in}%
\pgfsys@useobject{currentmarker}{}%
\end{pgfscope}%
\end{pgfscope}%
\begin{pgfscope}%
\definecolor{textcolor}{rgb}{0.000000,0.000000,0.000000}%
\pgfsetstrokecolor{textcolor}%
\pgfsetfillcolor{textcolor}%
\pgftext[x=0.669429in, y=0.881775in, left, base]{\color{textcolor}\rmfamily\fontsize{10.000000}{12.000000}\selectfont \(\displaystyle {\ensuremath{-}0.75}\)}%
\end{pgfscope}%
\begin{pgfscope}%
\pgfsetbuttcap%
\pgfsetroundjoin%
\definecolor{currentfill}{rgb}{0.000000,0.000000,0.000000}%
\pgfsetfillcolor{currentfill}%
\pgfsetlinewidth{0.803000pt}%
\definecolor{currentstroke}{rgb}{0.000000,0.000000,0.000000}%
\pgfsetstrokecolor{currentstroke}%
\pgfsetdash{}{0pt}%
\pgfsys@defobject{currentmarker}{\pgfqpoint{-0.048611in}{0.000000in}}{\pgfqpoint{-0.000000in}{0.000000in}}{%
\pgfpathmoveto{\pgfqpoint{-0.000000in}{0.000000in}}%
\pgfpathlineto{\pgfqpoint{-0.048611in}{0.000000in}}%
\pgfusepath{stroke,fill}%
}%
\begin{pgfscope}%
\pgfsys@transformshift{1.121591in}{1.280000in}%
\pgfsys@useobject{currentmarker}{}%
\end{pgfscope}%
\end{pgfscope}%
\begin{pgfscope}%
\definecolor{textcolor}{rgb}{0.000000,0.000000,0.000000}%
\pgfsetstrokecolor{textcolor}%
\pgfsetfillcolor{textcolor}%
\pgftext[x=0.669429in, y=1.231775in, left, base]{\color{textcolor}\rmfamily\fontsize{10.000000}{12.000000}\selectfont \(\displaystyle {\ensuremath{-}0.50}\)}%
\end{pgfscope}%
\begin{pgfscope}%
\pgfsetbuttcap%
\pgfsetroundjoin%
\definecolor{currentfill}{rgb}{0.000000,0.000000,0.000000}%
\pgfsetfillcolor{currentfill}%
\pgfsetlinewidth{0.803000pt}%
\definecolor{currentstroke}{rgb}{0.000000,0.000000,0.000000}%
\pgfsetstrokecolor{currentstroke}%
\pgfsetdash{}{0pt}%
\pgfsys@defobject{currentmarker}{\pgfqpoint{-0.048611in}{0.000000in}}{\pgfqpoint{-0.000000in}{0.000000in}}{%
\pgfpathmoveto{\pgfqpoint{-0.000000in}{0.000000in}}%
\pgfpathlineto{\pgfqpoint{-0.048611in}{0.000000in}}%
\pgfusepath{stroke,fill}%
}%
\begin{pgfscope}%
\pgfsys@transformshift{1.121591in}{1.630000in}%
\pgfsys@useobject{currentmarker}{}%
\end{pgfscope}%
\end{pgfscope}%
\begin{pgfscope}%
\definecolor{textcolor}{rgb}{0.000000,0.000000,0.000000}%
\pgfsetstrokecolor{textcolor}%
\pgfsetfillcolor{textcolor}%
\pgftext[x=0.669429in, y=1.581775in, left, base]{\color{textcolor}\rmfamily\fontsize{10.000000}{12.000000}\selectfont \(\displaystyle {\ensuremath{-}0.25}\)}%
\end{pgfscope}%
\begin{pgfscope}%
\pgfsetbuttcap%
\pgfsetroundjoin%
\definecolor{currentfill}{rgb}{0.000000,0.000000,0.000000}%
\pgfsetfillcolor{currentfill}%
\pgfsetlinewidth{0.803000pt}%
\definecolor{currentstroke}{rgb}{0.000000,0.000000,0.000000}%
\pgfsetstrokecolor{currentstroke}%
\pgfsetdash{}{0pt}%
\pgfsys@defobject{currentmarker}{\pgfqpoint{-0.048611in}{0.000000in}}{\pgfqpoint{-0.000000in}{0.000000in}}{%
\pgfpathmoveto{\pgfqpoint{-0.000000in}{0.000000in}}%
\pgfpathlineto{\pgfqpoint{-0.048611in}{0.000000in}}%
\pgfusepath{stroke,fill}%
}%
\begin{pgfscope}%
\pgfsys@transformshift{1.121591in}{1.980000in}%
\pgfsys@useobject{currentmarker}{}%
\end{pgfscope}%
\end{pgfscope}%
\begin{pgfscope}%
\definecolor{textcolor}{rgb}{0.000000,0.000000,0.000000}%
\pgfsetstrokecolor{textcolor}%
\pgfsetfillcolor{textcolor}%
\pgftext[x=0.777454in, y=1.931775in, left, base]{\color{textcolor}\rmfamily\fontsize{10.000000}{12.000000}\selectfont \(\displaystyle {0.00}\)}%
\end{pgfscope}%
\begin{pgfscope}%
\pgfsetbuttcap%
\pgfsetroundjoin%
\definecolor{currentfill}{rgb}{0.000000,0.000000,0.000000}%
\pgfsetfillcolor{currentfill}%
\pgfsetlinewidth{0.803000pt}%
\definecolor{currentstroke}{rgb}{0.000000,0.000000,0.000000}%
\pgfsetstrokecolor{currentstroke}%
\pgfsetdash{}{0pt}%
\pgfsys@defobject{currentmarker}{\pgfqpoint{-0.048611in}{0.000000in}}{\pgfqpoint{-0.000000in}{0.000000in}}{%
\pgfpathmoveto{\pgfqpoint{-0.000000in}{0.000000in}}%
\pgfpathlineto{\pgfqpoint{-0.048611in}{0.000000in}}%
\pgfusepath{stroke,fill}%
}%
\begin{pgfscope}%
\pgfsys@transformshift{1.121591in}{2.330000in}%
\pgfsys@useobject{currentmarker}{}%
\end{pgfscope}%
\end{pgfscope}%
\begin{pgfscope}%
\definecolor{textcolor}{rgb}{0.000000,0.000000,0.000000}%
\pgfsetstrokecolor{textcolor}%
\pgfsetfillcolor{textcolor}%
\pgftext[x=0.777454in, y=2.281775in, left, base]{\color{textcolor}\rmfamily\fontsize{10.000000}{12.000000}\selectfont \(\displaystyle {0.25}\)}%
\end{pgfscope}%
\begin{pgfscope}%
\pgfsetbuttcap%
\pgfsetroundjoin%
\definecolor{currentfill}{rgb}{0.000000,0.000000,0.000000}%
\pgfsetfillcolor{currentfill}%
\pgfsetlinewidth{0.803000pt}%
\definecolor{currentstroke}{rgb}{0.000000,0.000000,0.000000}%
\pgfsetstrokecolor{currentstroke}%
\pgfsetdash{}{0pt}%
\pgfsys@defobject{currentmarker}{\pgfqpoint{-0.048611in}{0.000000in}}{\pgfqpoint{-0.000000in}{0.000000in}}{%
\pgfpathmoveto{\pgfqpoint{-0.000000in}{0.000000in}}%
\pgfpathlineto{\pgfqpoint{-0.048611in}{0.000000in}}%
\pgfusepath{stroke,fill}%
}%
\begin{pgfscope}%
\pgfsys@transformshift{1.121591in}{2.680000in}%
\pgfsys@useobject{currentmarker}{}%
\end{pgfscope}%
\end{pgfscope}%
\begin{pgfscope}%
\definecolor{textcolor}{rgb}{0.000000,0.000000,0.000000}%
\pgfsetstrokecolor{textcolor}%
\pgfsetfillcolor{textcolor}%
\pgftext[x=0.777454in, y=2.631775in, left, base]{\color{textcolor}\rmfamily\fontsize{10.000000}{12.000000}\selectfont \(\displaystyle {0.50}\)}%
\end{pgfscope}%
\begin{pgfscope}%
\pgfsetbuttcap%
\pgfsetroundjoin%
\definecolor{currentfill}{rgb}{0.000000,0.000000,0.000000}%
\pgfsetfillcolor{currentfill}%
\pgfsetlinewidth{0.803000pt}%
\definecolor{currentstroke}{rgb}{0.000000,0.000000,0.000000}%
\pgfsetstrokecolor{currentstroke}%
\pgfsetdash{}{0pt}%
\pgfsys@defobject{currentmarker}{\pgfqpoint{-0.048611in}{0.000000in}}{\pgfqpoint{-0.000000in}{0.000000in}}{%
\pgfpathmoveto{\pgfqpoint{-0.000000in}{0.000000in}}%
\pgfpathlineto{\pgfqpoint{-0.048611in}{0.000000in}}%
\pgfusepath{stroke,fill}%
}%
\begin{pgfscope}%
\pgfsys@transformshift{1.121591in}{3.030000in}%
\pgfsys@useobject{currentmarker}{}%
\end{pgfscope}%
\end{pgfscope}%
\begin{pgfscope}%
\definecolor{textcolor}{rgb}{0.000000,0.000000,0.000000}%
\pgfsetstrokecolor{textcolor}%
\pgfsetfillcolor{textcolor}%
\pgftext[x=0.777454in, y=2.981775in, left, base]{\color{textcolor}\rmfamily\fontsize{10.000000}{12.000000}\selectfont \(\displaystyle {0.75}\)}%
\end{pgfscope}%
\begin{pgfscope}%
\pgfsetbuttcap%
\pgfsetroundjoin%
\definecolor{currentfill}{rgb}{0.000000,0.000000,0.000000}%
\pgfsetfillcolor{currentfill}%
\pgfsetlinewidth{0.803000pt}%
\definecolor{currentstroke}{rgb}{0.000000,0.000000,0.000000}%
\pgfsetstrokecolor{currentstroke}%
\pgfsetdash{}{0pt}%
\pgfsys@defobject{currentmarker}{\pgfqpoint{-0.048611in}{0.000000in}}{\pgfqpoint{-0.000000in}{0.000000in}}{%
\pgfpathmoveto{\pgfqpoint{-0.000000in}{0.000000in}}%
\pgfpathlineto{\pgfqpoint{-0.048611in}{0.000000in}}%
\pgfusepath{stroke,fill}%
}%
\begin{pgfscope}%
\pgfsys@transformshift{1.121591in}{3.380000in}%
\pgfsys@useobject{currentmarker}{}%
\end{pgfscope}%
\end{pgfscope}%
\begin{pgfscope}%
\definecolor{textcolor}{rgb}{0.000000,0.000000,0.000000}%
\pgfsetstrokecolor{textcolor}%
\pgfsetfillcolor{textcolor}%
\pgftext[x=0.777454in, y=3.331775in, left, base]{\color{textcolor}\rmfamily\fontsize{10.000000}{12.000000}\selectfont \(\displaystyle {1.00}\)}%
\end{pgfscope}%
\begin{pgfscope}%
\pgfsetbuttcap%
\pgfsetmiterjoin%
\definecolor{currentfill}{rgb}{0.000000,0.000000,0.000000}%
\pgfsetfillcolor{currentfill}%
\pgfsetlinewidth{1.003750pt}%
\definecolor{currentstroke}{rgb}{0.000000,0.000000,0.000000}%
\pgfsetstrokecolor{currentstroke}%
\pgfsetdash{}{0pt}%
\pgfsys@defobject{currentmarker}{\pgfqpoint{-0.041667in}{-0.041667in}}{\pgfqpoint{0.041667in}{0.041667in}}{%
\pgfpathmoveto{\pgfqpoint{0.041667in}{-0.000000in}}%
\pgfpathlineto{\pgfqpoint{-0.041667in}{0.041667in}}%
\pgfpathlineto{\pgfqpoint{-0.041667in}{-0.041667in}}%
\pgfpathlineto{\pgfqpoint{0.041667in}{-0.000000in}}%
\pgfpathclose%
\pgfusepath{stroke,fill}%
}%
\begin{pgfscope}%
\pgfsys@transformshift{6.300000in}{1.980000in}%
\pgfsys@useobject{currentmarker}{}%
\end{pgfscope}%
\end{pgfscope}%
\begin{pgfscope}%
\pgfsetbuttcap%
\pgfsetmiterjoin%
\definecolor{currentfill}{rgb}{0.000000,0.000000,0.000000}%
\pgfsetfillcolor{currentfill}%
\pgfsetlinewidth{1.003750pt}%
\definecolor{currentstroke}{rgb}{0.000000,0.000000,0.000000}%
\pgfsetstrokecolor{currentstroke}%
\pgfsetdash{}{0pt}%
\pgfsys@defobject{currentmarker}{\pgfqpoint{-0.041667in}{-0.041667in}}{\pgfqpoint{0.041667in}{0.041667in}}{%
\pgfpathmoveto{\pgfqpoint{0.000000in}{0.041667in}}%
\pgfpathlineto{\pgfqpoint{-0.041667in}{-0.041667in}}%
\pgfpathlineto{\pgfqpoint{0.041667in}{-0.041667in}}%
\pgfpathlineto{\pgfqpoint{0.000000in}{0.041667in}}%
\pgfpathclose%
\pgfusepath{stroke,fill}%
}%
\begin{pgfscope}%
\pgfsys@transformshift{1.121591in}{3.520000in}%
\pgfsys@useobject{currentmarker}{}%
\end{pgfscope}%
\end{pgfscope}%
\begin{pgfscope}%
\pgfpathrectangle{\pgfqpoint{0.875000in}{0.440000in}}{\pgfqpoint{5.425000in}{3.080000in}}%
\pgfusepath{clip}%
\pgfsetrectcap%
\pgfsetroundjoin%
\pgfsetlinewidth{1.505625pt}%
\definecolor{currentstroke}{rgb}{0.121569,0.466667,0.705882}%
\pgfsetstrokecolor{currentstroke}%
\pgfsetdash{}{0pt}%
\pgfpathmoveto{\pgfqpoint{1.121591in}{1.980000in}}%
\pgfpathlineto{\pgfqpoint{1.289124in}{2.276537in}}%
\pgfpathlineto{\pgfqpoint{1.378147in}{2.429472in}}%
\pgfpathlineto{\pgfqpoint{1.452183in}{2.552345in}}%
\pgfpathlineto{\pgfqpoint{1.517720in}{2.656901in}}%
\pgfpathlineto{\pgfqpoint{1.577665in}{2.748424in}}%
\pgfpathlineto{\pgfqpoint{1.633136in}{2.829137in}}%
\pgfpathlineto{\pgfqpoint{1.685253in}{2.901114in}}%
\pgfpathlineto{\pgfqpoint{1.734461in}{2.965355in}}%
\pgfpathlineto{\pgfqpoint{1.781433in}{3.023068in}}%
\pgfpathlineto{\pgfqpoint{1.826168in}{3.074564in}}%
\pgfpathlineto{\pgfqpoint{1.869114in}{3.120659in}}%
\pgfpathlineto{\pgfqpoint{1.910494in}{3.161845in}}%
\pgfpathlineto{\pgfqpoint{1.950308in}{3.198375in}}%
\pgfpathlineto{\pgfqpoint{1.988780in}{3.230698in}}%
\pgfpathlineto{\pgfqpoint{2.026134in}{3.259207in}}%
\pgfpathlineto{\pgfqpoint{2.062370in}{3.284096in}}%
\pgfpathlineto{\pgfqpoint{2.097710in}{3.305694in}}%
\pgfpathlineto{\pgfqpoint{2.132156in}{3.324160in}}%
\pgfpathlineto{\pgfqpoint{2.165931in}{3.339754in}}%
\pgfpathlineto{\pgfqpoint{2.199035in}{3.352596in}}%
\pgfpathlineto{\pgfqpoint{2.231468in}{3.362810in}}%
\pgfpathlineto{\pgfqpoint{2.263454in}{3.370572in}}%
\pgfpathlineto{\pgfqpoint{2.295216in}{3.375995in}}%
\pgfpathlineto{\pgfqpoint{2.326754in}{3.379119in}}%
\pgfpathlineto{\pgfqpoint{2.358068in}{3.379986in}}%
\pgfpathlineto{\pgfqpoint{2.389383in}{3.378626in}}%
\pgfpathlineto{\pgfqpoint{2.420697in}{3.375040in}}%
\pgfpathlineto{\pgfqpoint{2.452236in}{3.369184in}}%
\pgfpathlineto{\pgfqpoint{2.483998in}{3.361021in}}%
\pgfpathlineto{\pgfqpoint{2.516207in}{3.350435in}}%
\pgfpathlineto{\pgfqpoint{2.548863in}{3.337346in}}%
\pgfpathlineto{\pgfqpoint{2.582191in}{3.321566in}}%
\pgfpathlineto{\pgfqpoint{2.616190in}{3.302977in}}%
\pgfpathlineto{\pgfqpoint{2.650859in}{3.281467in}}%
\pgfpathlineto{\pgfqpoint{2.686424in}{3.256764in}}%
\pgfpathlineto{\pgfqpoint{2.723107in}{3.228539in}}%
\pgfpathlineto{\pgfqpoint{2.760908in}{3.196603in}}%
\pgfpathlineto{\pgfqpoint{2.799827in}{3.160775in}}%
\pgfpathlineto{\pgfqpoint{2.840089in}{3.120659in}}%
\pgfpathlineto{\pgfqpoint{2.881916in}{3.075807in}}%
\pgfpathlineto{\pgfqpoint{2.925533in}{3.025725in}}%
\pgfpathlineto{\pgfqpoint{2.970939in}{2.970161in}}%
\pgfpathlineto{\pgfqpoint{3.018582in}{2.908303in}}%
\pgfpathlineto{\pgfqpoint{3.068909in}{2.839251in}}%
\pgfpathlineto{\pgfqpoint{3.122143in}{2.762374in}}%
\pgfpathlineto{\pgfqpoint{3.179181in}{2.676023in}}%
\pgfpathlineto{\pgfqpoint{3.240915in}{2.578436in}}%
\pgfpathlineto{\pgfqpoint{3.308912in}{2.466692in}}%
\pgfpathlineto{\pgfqpoint{3.386528in}{2.334731in}}%
\pgfpathlineto{\pgfqpoint{3.482037in}{2.167729in}}%
\pgfpathlineto{\pgfqpoint{3.688489in}{1.800576in}}%
\pgfpathlineto{\pgfqpoint{3.795630in}{1.613320in}}%
\pgfpathlineto{\pgfqpoint{3.876824in}{1.475772in}}%
\pgfpathlineto{\pgfqpoint{3.946834in}{1.361444in}}%
\pgfpathlineto{\pgfqpoint{4.009687in}{1.262969in}}%
\pgfpathlineto{\pgfqpoint{4.067619in}{1.176258in}}%
\pgfpathlineto{\pgfqpoint{4.121749in}{1.099184in}}%
\pgfpathlineto{\pgfqpoint{4.172523in}{1.030686in}}%
\pgfpathlineto{\pgfqpoint{4.220837in}{0.969191in}}%
\pgfpathlineto{\pgfqpoint{4.266690in}{0.914364in}}%
\pgfpathlineto{\pgfqpoint{4.310754in}{0.865099in}}%
\pgfpathlineto{\pgfqpoint{4.353029in}{0.821135in}}%
\pgfpathlineto{\pgfqpoint{4.393738in}{0.781974in}}%
\pgfpathlineto{\pgfqpoint{4.432881in}{0.747355in}}%
\pgfpathlineto{\pgfqpoint{4.470906in}{0.716660in}}%
\pgfpathlineto{\pgfqpoint{4.507812in}{0.689701in}}%
\pgfpathlineto{\pgfqpoint{4.543600in}{0.666282in}}%
\pgfpathlineto{\pgfqpoint{4.578494in}{0.646077in}}%
\pgfpathlineto{\pgfqpoint{4.612492in}{0.628925in}}%
\pgfpathlineto{\pgfqpoint{4.645820in}{0.614570in}}%
\pgfpathlineto{\pgfqpoint{4.678700in}{0.602820in}}%
\pgfpathlineto{\pgfqpoint{4.710910in}{0.593652in}}%
\pgfpathlineto{\pgfqpoint{4.742671in}{0.586897in}}%
\pgfpathlineto{\pgfqpoint{4.774210in}{0.582447in}}%
\pgfpathlineto{\pgfqpoint{4.805524in}{0.580259in}}%
\pgfpathlineto{\pgfqpoint{4.836839in}{0.580299in}}%
\pgfpathlineto{\pgfqpoint{4.868153in}{0.582566in}}%
\pgfpathlineto{\pgfqpoint{4.899468in}{0.587057in}}%
\pgfpathlineto{\pgfqpoint{4.931006in}{0.593820in}}%
\pgfpathlineto{\pgfqpoint{4.962992in}{0.602964in}}%
\pgfpathlineto{\pgfqpoint{4.995425in}{0.614570in}}%
\pgfpathlineto{\pgfqpoint{5.028305in}{0.628716in}}%
\pgfpathlineto{\pgfqpoint{5.061856in}{0.645593in}}%
\pgfpathlineto{\pgfqpoint{5.096079in}{0.665319in}}%
\pgfpathlineto{\pgfqpoint{5.131196in}{0.688158in}}%
\pgfpathlineto{\pgfqpoint{5.167207in}{0.714263in}}%
\pgfpathlineto{\pgfqpoint{5.204337in}{0.743967in}}%
\pgfpathlineto{\pgfqpoint{5.242586in}{0.777456in}}%
\pgfpathlineto{\pgfqpoint{5.281953in}{0.814905in}}%
\pgfpathlineto{\pgfqpoint{5.322885in}{0.856948in}}%
\pgfpathlineto{\pgfqpoint{5.365384in}{0.903829in}}%
\pgfpathlineto{\pgfqpoint{5.409671in}{0.956037in}}%
\pgfpathlineto{\pgfqpoint{5.455972in}{1.014099in}}%
\pgfpathlineto{\pgfqpoint{5.504733in}{1.078876in}}%
\pgfpathlineto{\pgfqpoint{5.556179in}{1.150981in}}%
\pgfpathlineto{\pgfqpoint{5.610756in}{1.231359in}}%
\pgfpathlineto{\pgfqpoint{5.669358in}{1.321683in}}%
\pgfpathlineto{\pgfqpoint{5.733330in}{1.424451in}}%
\pgfpathlineto{\pgfqpoint{5.804682in}{1.543394in}}%
\pgfpathlineto{\pgfqpoint{5.887889in}{1.686583in}}%
\pgfpathlineto{\pgfqpoint{5.997490in}{1.879953in}}%
\pgfpathlineto{\pgfqpoint{6.053409in}{1.979601in}}%
\pgfpathlineto{\pgfqpoint{6.053409in}{1.979601in}}%
\pgfusepath{stroke}%
\end{pgfscope}%
\begin{pgfscope}%
\pgfpathrectangle{\pgfqpoint{0.875000in}{0.440000in}}{\pgfqpoint{5.425000in}{3.080000in}}%
\pgfusepath{clip}%
\pgfsetbuttcap%
\pgfsetroundjoin%
\pgfsetlinewidth{1.505625pt}%
\definecolor{currentstroke}{rgb}{0.501961,0.501961,0.501961}%
\pgfsetstrokecolor{currentstroke}%
\pgfsetdash{{5.550000pt}{2.400000pt}}{0.000000pt}%
\pgfpathmoveto{\pgfqpoint{1.146250in}{3.030000in}}%
\pgfpathlineto{\pgfqpoint{3.316250in}{3.030000in}}%
\pgfusepath{stroke}%
\end{pgfscope}%
\begin{pgfscope}%
\pgfpathrectangle{\pgfqpoint{0.875000in}{0.440000in}}{\pgfqpoint{5.425000in}{3.080000in}}%
\pgfusepath{clip}%
\pgfsetbuttcap%
\pgfsetroundjoin%
\pgfsetlinewidth{1.505625pt}%
\definecolor{currentstroke}{rgb}{0.501961,0.501961,0.501961}%
\pgfsetstrokecolor{currentstroke}%
\pgfsetdash{{5.550000pt}{2.400000pt}}{0.000000pt}%
\pgfpathmoveto{\pgfqpoint{2.285500in}{3.380000in}}%
\pgfpathlineto{\pgfqpoint{4.943750in}{3.380000in}}%
\pgfusepath{stroke}%
\end{pgfscope}%
\begin{pgfscope}%
\pgfsetrectcap%
\pgfsetmiterjoin%
\pgfsetlinewidth{0.803000pt}%
\definecolor{currentstroke}{rgb}{0.000000,0.000000,0.000000}%
\pgfsetstrokecolor{currentstroke}%
\pgfsetdash{}{0pt}%
\pgfpathmoveto{\pgfqpoint{1.121591in}{0.440000in}}%
\pgfpathlineto{\pgfqpoint{1.121591in}{3.520000in}}%
\pgfusepath{stroke}%
\end{pgfscope}%
\begin{pgfscope}%
\pgfsetrectcap%
\pgfsetmiterjoin%
\pgfsetlinewidth{0.803000pt}%
\definecolor{currentstroke}{rgb}{0.000000,0.000000,0.000000}%
\pgfsetstrokecolor{currentstroke}%
\pgfsetdash{}{0pt}%
\pgfpathmoveto{\pgfqpoint{0.875000in}{1.980000in}}%
\pgfpathlineto{\pgfqpoint{6.300000in}{1.980000in}}%
\pgfusepath{stroke}%
\end{pgfscope}%
\begin{pgfscope}%
\pgfsetroundcap%
\pgfsetroundjoin%
\pgfsetlinewidth{1.003750pt}%
\definecolor{currentstroke}{rgb}{0.000000,0.000000,0.000000}%
\pgfsetstrokecolor{currentstroke}%
\pgfsetdash{}{0pt}%
\pgfpathmoveto{\pgfqpoint{3.587500in}{3.336658in}}%
\pgfpathquadraticcurveto{\pgfqpoint{3.587500in}{2.679975in}}{\pgfqpoint{3.587500in}{2.023292in}}%
\pgfusepath{stroke}%
\end{pgfscope}%
\begin{pgfscope}%
\pgfsetroundcap%
\pgfsetroundjoin%
\pgfsetlinewidth{1.003750pt}%
\definecolor{currentstroke}{rgb}{0.000000,0.000000,0.000000}%
\pgfsetstrokecolor{currentstroke}%
\pgfsetdash{}{0pt}%
\pgfpathmoveto{\pgfqpoint{3.559722in}{3.281102in}}%
\pgfpathlineto{\pgfqpoint{3.587500in}{3.336658in}}%
\pgfpathlineto{\pgfqpoint{3.615278in}{3.281102in}}%
\pgfusepath{stroke}%
\end{pgfscope}%
\begin{pgfscope}%
\pgfsetroundcap%
\pgfsetroundjoin%
\pgfsetlinewidth{1.003750pt}%
\definecolor{currentstroke}{rgb}{0.000000,0.000000,0.000000}%
\pgfsetstrokecolor{currentstroke}%
\pgfsetdash{}{0pt}%
\pgfpathmoveto{\pgfqpoint{3.615278in}{2.078847in}}%
\pgfpathlineto{\pgfqpoint{3.587500in}{2.023292in}}%
\pgfpathlineto{\pgfqpoint{3.559722in}{2.078847in}}%
\pgfusepath{stroke}%
\end{pgfscope}%
\begin{pgfscope}%
\pgfsetbuttcap%
\pgfsetmiterjoin%
\definecolor{currentfill}{rgb}{0.800000,0.800000,0.800000}%
\pgfsetfillcolor{currentfill}%
\pgfsetlinewidth{1.003750pt}%
\definecolor{currentstroke}{rgb}{0.000000,0.000000,0.000000}%
\pgfsetstrokecolor{currentstroke}%
\pgfsetdash{}{0pt}%
\pgfpathmoveto{\pgfqpoint{3.636818in}{2.611327in}}%
\pgfpathlineto{\pgfqpoint{4.140291in}{2.611327in}}%
\pgfpathquadraticcurveto{\pgfqpoint{4.181958in}{2.611327in}}{\pgfqpoint{4.181958in}{2.652994in}}%
\pgfpathlineto{\pgfqpoint{4.181958in}{2.776451in}}%
\pgfpathquadraticcurveto{\pgfqpoint{4.181958in}{2.818117in}}{\pgfqpoint{4.140291in}{2.818117in}}%
\pgfpathlineto{\pgfqpoint{3.636818in}{2.818117in}}%
\pgfpathquadraticcurveto{\pgfqpoint{3.595152in}{2.818117in}}{\pgfqpoint{3.595152in}{2.776451in}}%
\pgfpathlineto{\pgfqpoint{3.595152in}{2.652994in}}%
\pgfpathquadraticcurveto{\pgfqpoint{3.595152in}{2.611327in}}{\pgfqpoint{3.636818in}{2.611327in}}%
\pgfpathlineto{\pgfqpoint{3.636818in}{2.611327in}}%
\pgfpathclose%
\pgfusepath{stroke,fill}%
\end{pgfscope}%
\begin{pgfscope}%
\definecolor{textcolor}{rgb}{0.000000,0.000000,0.000000}%
\pgfsetstrokecolor{textcolor}%
\pgfsetfillcolor{textcolor}%
\pgftext[x=3.636818in,y=2.680000in,left,base]{\color{textcolor}\rmfamily\fontsize{10.000000}{12.000000}\selectfont 1 - Peak}%
\end{pgfscope}%
\begin{pgfscope}%
\pgfsetroundcap%
\pgfsetroundjoin%
\pgfsetlinewidth{1.003750pt}%
\definecolor{currentstroke}{rgb}{0.000000,0.000000,0.000000}%
\pgfsetstrokecolor{currentstroke}%
\pgfsetdash{}{0pt}%
\pgfpathmoveto{\pgfqpoint{1.861364in}{2.986690in}}%
\pgfpathquadraticcurveto{\pgfqpoint{1.861364in}{2.505007in}}{\pgfqpoint{1.861364in}{2.023324in}}%
\pgfusepath{stroke}%
\end{pgfscope}%
\begin{pgfscope}%
\pgfsetroundcap%
\pgfsetroundjoin%
\pgfsetlinewidth{1.003750pt}%
\definecolor{currentstroke}{rgb}{0.000000,0.000000,0.000000}%
\pgfsetstrokecolor{currentstroke}%
\pgfsetdash{}{0pt}%
\pgfpathmoveto{\pgfqpoint{1.833586in}{2.931135in}}%
\pgfpathlineto{\pgfqpoint{1.861364in}{2.986690in}}%
\pgfpathlineto{\pgfqpoint{1.889141in}{2.931135in}}%
\pgfusepath{stroke}%
\end{pgfscope}%
\begin{pgfscope}%
\pgfsetroundcap%
\pgfsetroundjoin%
\pgfsetlinewidth{1.003750pt}%
\definecolor{currentstroke}{rgb}{0.000000,0.000000,0.000000}%
\pgfsetstrokecolor{currentstroke}%
\pgfsetdash{}{0pt}%
\pgfpathmoveto{\pgfqpoint{1.889141in}{2.078879in}}%
\pgfpathlineto{\pgfqpoint{1.861364in}{2.023324in}}%
\pgfpathlineto{\pgfqpoint{1.833586in}{2.078879in}}%
\pgfusepath{stroke}%
\end{pgfscope}%
\begin{pgfscope}%
\pgfsetbuttcap%
\pgfsetmiterjoin%
\definecolor{currentfill}{rgb}{0.800000,0.800000,0.800000}%
\pgfsetfillcolor{currentfill}%
\pgfsetlinewidth{1.003750pt}%
\definecolor{currentstroke}{rgb}{0.000000,0.000000,0.000000}%
\pgfsetstrokecolor{currentstroke}%
\pgfsetdash{}{0pt}%
\pgfpathmoveto{\pgfqpoint{1.910682in}{2.261327in}}%
\pgfpathlineto{\pgfqpoint{2.425729in}{2.261327in}}%
\pgfpathquadraticcurveto{\pgfqpoint{2.467395in}{2.261327in}}{\pgfqpoint{2.467395in}{2.302994in}}%
\pgfpathlineto{\pgfqpoint{2.467395in}{2.426451in}}%
\pgfpathquadraticcurveto{\pgfqpoint{2.467395in}{2.468117in}}{\pgfqpoint{2.425729in}{2.468117in}}%
\pgfpathlineto{\pgfqpoint{1.910682in}{2.468117in}}%
\pgfpathquadraticcurveto{\pgfqpoint{1.869015in}{2.468117in}}{\pgfqpoint{1.869015in}{2.426451in}}%
\pgfpathlineto{\pgfqpoint{1.869015in}{2.302994in}}%
\pgfpathquadraticcurveto{\pgfqpoint{1.869015in}{2.261327in}}{\pgfqpoint{1.910682in}{2.261327in}}%
\pgfpathlineto{\pgfqpoint{1.910682in}{2.261327in}}%
\pgfpathclose%
\pgfusepath{stroke,fill}%
\end{pgfscope}%
\begin{pgfscope}%
\definecolor{textcolor}{rgb}{0.000000,0.000000,0.000000}%
\pgfsetstrokecolor{textcolor}%
\pgfsetfillcolor{textcolor}%
\pgftext[x=1.910682in,y=2.330000in,left,base]{\color{textcolor}\rmfamily\fontsize{10.000000}{12.000000}\selectfont 3 - RMS}%
\end{pgfscope}%
\begin{pgfscope}%
\pgfsetroundcap%
\pgfsetroundjoin%
\pgfsetlinewidth{1.003750pt}%
\definecolor{currentstroke}{rgb}{0.000000,0.000000,0.000000}%
\pgfsetstrokecolor{currentstroke}%
\pgfsetdash{}{0pt}%
\pgfpathmoveto{\pgfqpoint{4.820455in}{0.623342in}}%
\pgfpathquadraticcurveto{\pgfqpoint{4.820455in}{1.980011in}}{\pgfqpoint{4.820455in}{3.336681in}}%
\pgfusepath{stroke}%
\end{pgfscope}%
\begin{pgfscope}%
\pgfsetroundcap%
\pgfsetroundjoin%
\pgfsetlinewidth{1.003750pt}%
\definecolor{currentstroke}{rgb}{0.000000,0.000000,0.000000}%
\pgfsetstrokecolor{currentstroke}%
\pgfsetdash{}{0pt}%
\pgfpathmoveto{\pgfqpoint{4.848232in}{0.678898in}}%
\pgfpathlineto{\pgfqpoint{4.820455in}{0.623342in}}%
\pgfpathlineto{\pgfqpoint{4.792677in}{0.678898in}}%
\pgfusepath{stroke}%
\end{pgfscope}%
\begin{pgfscope}%
\pgfsetroundcap%
\pgfsetroundjoin%
\pgfsetlinewidth{1.003750pt}%
\definecolor{currentstroke}{rgb}{0.000000,0.000000,0.000000}%
\pgfsetstrokecolor{currentstroke}%
\pgfsetdash{}{0pt}%
\pgfpathmoveto{\pgfqpoint{4.792677in}{3.281125in}}%
\pgfpathlineto{\pgfqpoint{4.820455in}{3.336681in}}%
\pgfpathlineto{\pgfqpoint{4.848232in}{3.281125in}}%
\pgfusepath{stroke}%
\end{pgfscope}%
\begin{pgfscope}%
\pgfsetbuttcap%
\pgfsetmiterjoin%
\definecolor{currentfill}{rgb}{0.800000,0.800000,0.800000}%
\pgfsetfillcolor{currentfill}%
\pgfsetlinewidth{1.003750pt}%
\definecolor{currentstroke}{rgb}{0.000000,0.000000,0.000000}%
\pgfsetstrokecolor{currentstroke}%
\pgfsetdash{}{0pt}%
\pgfpathmoveto{\pgfqpoint{4.919091in}{2.051327in}}%
\pgfpathlineto{\pgfqpoint{5.764000in}{2.051327in}}%
\pgfpathquadraticcurveto{\pgfqpoint{5.805666in}{2.051327in}}{\pgfqpoint{5.805666in}{2.092994in}}%
\pgfpathlineto{\pgfqpoint{5.805666in}{2.216451in}}%
\pgfpathquadraticcurveto{\pgfqpoint{5.805666in}{2.258117in}}{\pgfqpoint{5.764000in}{2.258117in}}%
\pgfpathlineto{\pgfqpoint{4.919091in}{2.258117in}}%
\pgfpathquadraticcurveto{\pgfqpoint{4.877424in}{2.258117in}}{\pgfqpoint{4.877424in}{2.216451in}}%
\pgfpathlineto{\pgfqpoint{4.877424in}{2.092994in}}%
\pgfpathquadraticcurveto{\pgfqpoint{4.877424in}{2.051327in}}{\pgfqpoint{4.919091in}{2.051327in}}%
\pgfpathlineto{\pgfqpoint{4.919091in}{2.051327in}}%
\pgfpathclose%
\pgfusepath{stroke,fill}%
\end{pgfscope}%
\begin{pgfscope}%
\definecolor{textcolor}{rgb}{0.000000,0.000000,0.000000}%
\pgfsetstrokecolor{textcolor}%
\pgfsetfillcolor{textcolor}%
\pgftext[x=4.919091in,y=2.120000in,left,base]{\color{textcolor}\rmfamily\fontsize{10.000000}{12.000000}\selectfont 2 - Peak-Peak}%
\end{pgfscope}%
\begin{pgfscope}%
\definecolor{textcolor}{rgb}{0.000000,0.000000,0.000000}%
\pgfsetstrokecolor{textcolor}%
\pgfsetfillcolor{textcolor}%
\pgftext[x=3.587500in,y=3.603333in,,base]{\color{textcolor}\rmfamily\fontsize{12.000000}{14.400000}\selectfont Measures of amplitude. (1) Peak amplitude. (2) Peak-to-peak amplitude (3)RMS amplitude}%
\end{pgfscope}%
\end{pgfpicture}%
\makeatother%
\endgroup%
}
    \end{center}
    \caption{Measures of amplitude. (1) Peak amplitude. (2) Peak-to-peak amplitude (3)RMS amplitude.}
\end{figure}

Intuitivamente, osservare quanto il tracciato del grafico si distanzia dall'asse orizzontale dà un'idea approssimativa di ``quanto forte'' è il suono, o della sua dinamica (non uso il termine ``intensità'', che ha un significato specifico definito più avanti). È utile definire alcune maniere diverse di misurare l'ampiezza di un fenomeno sonoro:

\subsection{Ampiezza istantanea}

\begin{figure}
    \begin{center}
       \scalebox{0.6} {%% Creator: Matplotlib, PGF backend
%%
%% To include the figure in your LaTeX document, write
%%   \input{<filename>.pgf}
%%
%% Make sure the required packages are loaded in your preamble
%%   \usepackage{pgf}
%%
%% Also ensure that all the required font packages are loaded; for instance,
%% the lmodern package is sometimes necessary when using math font.
%%   \usepackage{lmodern}
%%
%% Figures using additional raster images can only be included by \input if
%% they are in the same directory as the main LaTeX file. For loading figures
%% from other directories you can use the `import` package
%%   \usepackage{import}
%%
%% and then include the figures with
%%   \import{<path to file>}{<filename>.pgf}
%%
%% Matplotlib used the following preamble
%%   
%%   \usepackage{fontspec}
%%   \setmainfont{DejaVuSerif.ttf}[Path=\detokenize{/opt/homebrew/Caskroom/miniconda/base/envs/label-studio/lib/python3.9/site-packages/matplotlib/mpl-data/fonts/ttf/}]
%%   \setsansfont{DejaVuSans.ttf}[Path=\detokenize{/opt/homebrew/Caskroom/miniconda/base/envs/label-studio/lib/python3.9/site-packages/matplotlib/mpl-data/fonts/ttf/}]
%%   \setmonofont{DejaVuSansMono.ttf}[Path=\detokenize{/opt/homebrew/Caskroom/miniconda/base/envs/label-studio/lib/python3.9/site-packages/matplotlib/mpl-data/fonts/ttf/}]
%%   \makeatletter\@ifpackageloaded{underscore}{}{\usepackage[strings]{underscore}}\makeatother
%%
\begingroup%
\makeatletter%
\begin{pgfpicture}%
\pgfpathrectangle{\pgfpointorigin}{\pgfqpoint{7.000000in}{4.000000in}}%
\pgfusepath{use as bounding box, clip}%
\begin{pgfscope}%
\pgfsetbuttcap%
\pgfsetmiterjoin%
\definecolor{currentfill}{rgb}{1.000000,1.000000,1.000000}%
\pgfsetfillcolor{currentfill}%
\pgfsetlinewidth{0.000000pt}%
\definecolor{currentstroke}{rgb}{1.000000,1.000000,1.000000}%
\pgfsetstrokecolor{currentstroke}%
\pgfsetdash{}{0pt}%
\pgfpathmoveto{\pgfqpoint{0.000000in}{0.000000in}}%
\pgfpathlineto{\pgfqpoint{7.000000in}{0.000000in}}%
\pgfpathlineto{\pgfqpoint{7.000000in}{4.000000in}}%
\pgfpathlineto{\pgfqpoint{0.000000in}{4.000000in}}%
\pgfpathlineto{\pgfqpoint{0.000000in}{0.000000in}}%
\pgfpathclose%
\pgfusepath{fill}%
\end{pgfscope}%
\begin{pgfscope}%
\pgfsetbuttcap%
\pgfsetmiterjoin%
\definecolor{currentfill}{rgb}{1.000000,1.000000,1.000000}%
\pgfsetfillcolor{currentfill}%
\pgfsetlinewidth{0.000000pt}%
\definecolor{currentstroke}{rgb}{0.000000,0.000000,0.000000}%
\pgfsetstrokecolor{currentstroke}%
\pgfsetstrokeopacity{0.000000}%
\pgfsetdash{}{0pt}%
\pgfpathmoveto{\pgfqpoint{0.875000in}{0.440000in}}%
\pgfpathlineto{\pgfqpoint{6.300000in}{0.440000in}}%
\pgfpathlineto{\pgfqpoint{6.300000in}{3.520000in}}%
\pgfpathlineto{\pgfqpoint{0.875000in}{3.520000in}}%
\pgfpathlineto{\pgfqpoint{0.875000in}{0.440000in}}%
\pgfpathclose%
\pgfusepath{fill}%
\end{pgfscope}%
\begin{pgfscope}%
\pgfsetbuttcap%
\pgfsetroundjoin%
\definecolor{currentfill}{rgb}{0.000000,0.000000,0.000000}%
\pgfsetfillcolor{currentfill}%
\pgfsetlinewidth{0.803000pt}%
\definecolor{currentstroke}{rgb}{0.000000,0.000000,0.000000}%
\pgfsetstrokecolor{currentstroke}%
\pgfsetdash{}{0pt}%
\pgfsys@defobject{currentmarker}{\pgfqpoint{0.000000in}{-0.048611in}}{\pgfqpoint{0.000000in}{0.000000in}}{%
\pgfpathmoveto{\pgfqpoint{0.000000in}{0.000000in}}%
\pgfpathlineto{\pgfqpoint{0.000000in}{-0.048611in}}%
\pgfusepath{stroke,fill}%
}%
\begin{pgfscope}%
\pgfsys@transformshift{1.121591in}{0.440000in}%
\pgfsys@useobject{currentmarker}{}%
\end{pgfscope}%
\end{pgfscope}%
\begin{pgfscope}%
\definecolor{textcolor}{rgb}{0.000000,0.000000,0.000000}%
\pgfsetstrokecolor{textcolor}%
\pgfsetfillcolor{textcolor}%
\pgftext[x=1.121591in,y=0.342778in,,top]{\color{textcolor}\sffamily\fontsize{10.000000}{12.000000}\selectfont 0.000}%
\end{pgfscope}%
\begin{pgfscope}%
\pgfsetbuttcap%
\pgfsetroundjoin%
\definecolor{currentfill}{rgb}{0.000000,0.000000,0.000000}%
\pgfsetfillcolor{currentfill}%
\pgfsetlinewidth{0.803000pt}%
\definecolor{currentstroke}{rgb}{0.000000,0.000000,0.000000}%
\pgfsetstrokecolor{currentstroke}%
\pgfsetdash{}{0pt}%
\pgfsys@defobject{currentmarker}{\pgfqpoint{0.000000in}{-0.048611in}}{\pgfqpoint{0.000000in}{0.000000in}}{%
\pgfpathmoveto{\pgfqpoint{0.000000in}{0.000000in}}%
\pgfpathlineto{\pgfqpoint{0.000000in}{-0.048611in}}%
\pgfusepath{stroke,fill}%
}%
\begin{pgfscope}%
\pgfsys@transformshift{1.863045in}{0.440000in}%
\pgfsys@useobject{currentmarker}{}%
\end{pgfscope}%
\end{pgfscope}%
\begin{pgfscope}%
\definecolor{textcolor}{rgb}{0.000000,0.000000,0.000000}%
\pgfsetstrokecolor{textcolor}%
\pgfsetfillcolor{textcolor}%
\pgftext[x=1.863045in,y=0.342778in,,top]{\color{textcolor}\sffamily\fontsize{10.000000}{12.000000}\selectfont 0.002}%
\end{pgfscope}%
\begin{pgfscope}%
\pgfsetbuttcap%
\pgfsetroundjoin%
\definecolor{currentfill}{rgb}{0.000000,0.000000,0.000000}%
\pgfsetfillcolor{currentfill}%
\pgfsetlinewidth{0.803000pt}%
\definecolor{currentstroke}{rgb}{0.000000,0.000000,0.000000}%
\pgfsetstrokecolor{currentstroke}%
\pgfsetdash{}{0pt}%
\pgfsys@defobject{currentmarker}{\pgfqpoint{0.000000in}{-0.048611in}}{\pgfqpoint{0.000000in}{0.000000in}}{%
\pgfpathmoveto{\pgfqpoint{0.000000in}{0.000000in}}%
\pgfpathlineto{\pgfqpoint{0.000000in}{-0.048611in}}%
\pgfusepath{stroke,fill}%
}%
\begin{pgfscope}%
\pgfsys@transformshift{2.604499in}{0.440000in}%
\pgfsys@useobject{currentmarker}{}%
\end{pgfscope}%
\end{pgfscope}%
\begin{pgfscope}%
\definecolor{textcolor}{rgb}{0.000000,0.000000,0.000000}%
\pgfsetstrokecolor{textcolor}%
\pgfsetfillcolor{textcolor}%
\pgftext[x=2.604499in,y=0.342778in,,top]{\color{textcolor}\sffamily\fontsize{10.000000}{12.000000}\selectfont 0.003}%
\end{pgfscope}%
\begin{pgfscope}%
\pgfsetbuttcap%
\pgfsetroundjoin%
\definecolor{currentfill}{rgb}{0.000000,0.000000,0.000000}%
\pgfsetfillcolor{currentfill}%
\pgfsetlinewidth{0.803000pt}%
\definecolor{currentstroke}{rgb}{0.000000,0.000000,0.000000}%
\pgfsetstrokecolor{currentstroke}%
\pgfsetdash{}{0pt}%
\pgfsys@defobject{currentmarker}{\pgfqpoint{0.000000in}{-0.048611in}}{\pgfqpoint{0.000000in}{0.000000in}}{%
\pgfpathmoveto{\pgfqpoint{0.000000in}{0.000000in}}%
\pgfpathlineto{\pgfqpoint{0.000000in}{-0.048611in}}%
\pgfusepath{stroke,fill}%
}%
\begin{pgfscope}%
\pgfsys@transformshift{3.345953in}{0.440000in}%
\pgfsys@useobject{currentmarker}{}%
\end{pgfscope}%
\end{pgfscope}%
\begin{pgfscope}%
\definecolor{textcolor}{rgb}{0.000000,0.000000,0.000000}%
\pgfsetstrokecolor{textcolor}%
\pgfsetfillcolor{textcolor}%
\pgftext[x=3.345953in,y=0.342778in,,top]{\color{textcolor}\sffamily\fontsize{10.000000}{12.000000}\selectfont 0.005}%
\end{pgfscope}%
\begin{pgfscope}%
\pgfsetbuttcap%
\pgfsetroundjoin%
\definecolor{currentfill}{rgb}{0.000000,0.000000,0.000000}%
\pgfsetfillcolor{currentfill}%
\pgfsetlinewidth{0.803000pt}%
\definecolor{currentstroke}{rgb}{0.000000,0.000000,0.000000}%
\pgfsetstrokecolor{currentstroke}%
\pgfsetdash{}{0pt}%
\pgfsys@defobject{currentmarker}{\pgfqpoint{0.000000in}{-0.048611in}}{\pgfqpoint{0.000000in}{0.000000in}}{%
\pgfpathmoveto{\pgfqpoint{0.000000in}{0.000000in}}%
\pgfpathlineto{\pgfqpoint{0.000000in}{-0.048611in}}%
\pgfusepath{stroke,fill}%
}%
\begin{pgfscope}%
\pgfsys@transformshift{4.087407in}{0.440000in}%
\pgfsys@useobject{currentmarker}{}%
\end{pgfscope}%
\end{pgfscope}%
\begin{pgfscope}%
\definecolor{textcolor}{rgb}{0.000000,0.000000,0.000000}%
\pgfsetstrokecolor{textcolor}%
\pgfsetfillcolor{textcolor}%
\pgftext[x=4.087407in,y=0.342778in,,top]{\color{textcolor}\sffamily\fontsize{10.000000}{12.000000}\selectfont 0.006}%
\end{pgfscope}%
\begin{pgfscope}%
\pgfsetbuttcap%
\pgfsetroundjoin%
\definecolor{currentfill}{rgb}{0.000000,0.000000,0.000000}%
\pgfsetfillcolor{currentfill}%
\pgfsetlinewidth{0.803000pt}%
\definecolor{currentstroke}{rgb}{0.000000,0.000000,0.000000}%
\pgfsetstrokecolor{currentstroke}%
\pgfsetdash{}{0pt}%
\pgfsys@defobject{currentmarker}{\pgfqpoint{0.000000in}{-0.048611in}}{\pgfqpoint{0.000000in}{0.000000in}}{%
\pgfpathmoveto{\pgfqpoint{0.000000in}{0.000000in}}%
\pgfpathlineto{\pgfqpoint{0.000000in}{-0.048611in}}%
\pgfusepath{stroke,fill}%
}%
\begin{pgfscope}%
\pgfsys@transformshift{4.828861in}{0.440000in}%
\pgfsys@useobject{currentmarker}{}%
\end{pgfscope}%
\end{pgfscope}%
\begin{pgfscope}%
\definecolor{textcolor}{rgb}{0.000000,0.000000,0.000000}%
\pgfsetstrokecolor{textcolor}%
\pgfsetfillcolor{textcolor}%
\pgftext[x=4.828861in,y=0.342778in,,top]{\color{textcolor}\sffamily\fontsize{10.000000}{12.000000}\selectfont 0.008}%
\end{pgfscope}%
\begin{pgfscope}%
\pgfsetbuttcap%
\pgfsetroundjoin%
\definecolor{currentfill}{rgb}{0.000000,0.000000,0.000000}%
\pgfsetfillcolor{currentfill}%
\pgfsetlinewidth{0.803000pt}%
\definecolor{currentstroke}{rgb}{0.000000,0.000000,0.000000}%
\pgfsetstrokecolor{currentstroke}%
\pgfsetdash{}{0pt}%
\pgfsys@defobject{currentmarker}{\pgfqpoint{0.000000in}{-0.048611in}}{\pgfqpoint{0.000000in}{0.000000in}}{%
\pgfpathmoveto{\pgfqpoint{0.000000in}{0.000000in}}%
\pgfpathlineto{\pgfqpoint{0.000000in}{-0.048611in}}%
\pgfusepath{stroke,fill}%
}%
\begin{pgfscope}%
\pgfsys@transformshift{5.570315in}{0.440000in}%
\pgfsys@useobject{currentmarker}{}%
\end{pgfscope}%
\end{pgfscope}%
\begin{pgfscope}%
\definecolor{textcolor}{rgb}{0.000000,0.000000,0.000000}%
\pgfsetstrokecolor{textcolor}%
\pgfsetfillcolor{textcolor}%
\pgftext[x=5.570315in,y=0.342778in,,top]{\color{textcolor}\sffamily\fontsize{10.000000}{12.000000}\selectfont 0.009}%
\end{pgfscope}%
\begin{pgfscope}%
\definecolor{textcolor}{rgb}{0.000000,0.000000,0.000000}%
\pgfsetstrokecolor{textcolor}%
\pgfsetfillcolor{textcolor}%
\pgftext[x=3.587500in,y=0.152809in,,top]{\color{textcolor}\sffamily\fontsize{10.000000}{12.000000}\selectfont Time}%
\end{pgfscope}%
\begin{pgfscope}%
\pgfsetbuttcap%
\pgfsetroundjoin%
\definecolor{currentfill}{rgb}{0.000000,0.000000,0.000000}%
\pgfsetfillcolor{currentfill}%
\pgfsetlinewidth{0.803000pt}%
\definecolor{currentstroke}{rgb}{0.000000,0.000000,0.000000}%
\pgfsetstrokecolor{currentstroke}%
\pgfsetdash{}{0pt}%
\pgfsys@defobject{currentmarker}{\pgfqpoint{-0.048611in}{0.000000in}}{\pgfqpoint{-0.000000in}{0.000000in}}{%
\pgfpathmoveto{\pgfqpoint{-0.000000in}{0.000000in}}%
\pgfpathlineto{\pgfqpoint{-0.048611in}{0.000000in}}%
\pgfusepath{stroke,fill}%
}%
\begin{pgfscope}%
\pgfsys@transformshift{0.875000in}{0.470530in}%
\pgfsys@useobject{currentmarker}{}%
\end{pgfscope}%
\end{pgfscope}%
\begin{pgfscope}%
\definecolor{textcolor}{rgb}{0.000000,0.000000,0.000000}%
\pgfsetstrokecolor{textcolor}%
\pgfsetfillcolor{textcolor}%
\pgftext[x=0.360508in, y=0.417768in, left, base]{\color{textcolor}\sffamily\fontsize{10.000000}{12.000000}\selectfont \ensuremath{-}0.15}%
\end{pgfscope}%
\begin{pgfscope}%
\pgfsetbuttcap%
\pgfsetroundjoin%
\definecolor{currentfill}{rgb}{0.000000,0.000000,0.000000}%
\pgfsetfillcolor{currentfill}%
\pgfsetlinewidth{0.803000pt}%
\definecolor{currentstroke}{rgb}{0.000000,0.000000,0.000000}%
\pgfsetstrokecolor{currentstroke}%
\pgfsetdash{}{0pt}%
\pgfsys@defobject{currentmarker}{\pgfqpoint{-0.048611in}{0.000000in}}{\pgfqpoint{-0.000000in}{0.000000in}}{%
\pgfpathmoveto{\pgfqpoint{-0.000000in}{0.000000in}}%
\pgfpathlineto{\pgfqpoint{-0.048611in}{0.000000in}}%
\pgfusepath{stroke,fill}%
}%
\begin{pgfscope}%
\pgfsys@transformshift{0.875000in}{0.973686in}%
\pgfsys@useobject{currentmarker}{}%
\end{pgfscope}%
\end{pgfscope}%
\begin{pgfscope}%
\definecolor{textcolor}{rgb}{0.000000,0.000000,0.000000}%
\pgfsetstrokecolor{textcolor}%
\pgfsetfillcolor{textcolor}%
\pgftext[x=0.360508in, y=0.920925in, left, base]{\color{textcolor}\sffamily\fontsize{10.000000}{12.000000}\selectfont \ensuremath{-}0.10}%
\end{pgfscope}%
\begin{pgfscope}%
\pgfsetbuttcap%
\pgfsetroundjoin%
\definecolor{currentfill}{rgb}{0.000000,0.000000,0.000000}%
\pgfsetfillcolor{currentfill}%
\pgfsetlinewidth{0.803000pt}%
\definecolor{currentstroke}{rgb}{0.000000,0.000000,0.000000}%
\pgfsetstrokecolor{currentstroke}%
\pgfsetdash{}{0pt}%
\pgfsys@defobject{currentmarker}{\pgfqpoint{-0.048611in}{0.000000in}}{\pgfqpoint{-0.000000in}{0.000000in}}{%
\pgfpathmoveto{\pgfqpoint{-0.000000in}{0.000000in}}%
\pgfpathlineto{\pgfqpoint{-0.048611in}{0.000000in}}%
\pgfusepath{stroke,fill}%
}%
\begin{pgfscope}%
\pgfsys@transformshift{0.875000in}{1.476843in}%
\pgfsys@useobject{currentmarker}{}%
\end{pgfscope}%
\end{pgfscope}%
\begin{pgfscope}%
\definecolor{textcolor}{rgb}{0.000000,0.000000,0.000000}%
\pgfsetstrokecolor{textcolor}%
\pgfsetfillcolor{textcolor}%
\pgftext[x=0.360508in, y=1.424082in, left, base]{\color{textcolor}\sffamily\fontsize{10.000000}{12.000000}\selectfont \ensuremath{-}0.05}%
\end{pgfscope}%
\begin{pgfscope}%
\pgfsetbuttcap%
\pgfsetroundjoin%
\definecolor{currentfill}{rgb}{0.000000,0.000000,0.000000}%
\pgfsetfillcolor{currentfill}%
\pgfsetlinewidth{0.803000pt}%
\definecolor{currentstroke}{rgb}{0.000000,0.000000,0.000000}%
\pgfsetstrokecolor{currentstroke}%
\pgfsetdash{}{0pt}%
\pgfsys@defobject{currentmarker}{\pgfqpoint{-0.048611in}{0.000000in}}{\pgfqpoint{-0.000000in}{0.000000in}}{%
\pgfpathmoveto{\pgfqpoint{-0.000000in}{0.000000in}}%
\pgfpathlineto{\pgfqpoint{-0.048611in}{0.000000in}}%
\pgfusepath{stroke,fill}%
}%
\begin{pgfscope}%
\pgfsys@transformshift{0.875000in}{1.980000in}%
\pgfsys@useobject{currentmarker}{}%
\end{pgfscope}%
\end{pgfscope}%
\begin{pgfscope}%
\definecolor{textcolor}{rgb}{0.000000,0.000000,0.000000}%
\pgfsetstrokecolor{textcolor}%
\pgfsetfillcolor{textcolor}%
\pgftext[x=0.468533in, y=1.927238in, left, base]{\color{textcolor}\sffamily\fontsize{10.000000}{12.000000}\selectfont 0.00}%
\end{pgfscope}%
\begin{pgfscope}%
\pgfsetbuttcap%
\pgfsetroundjoin%
\definecolor{currentfill}{rgb}{0.000000,0.000000,0.000000}%
\pgfsetfillcolor{currentfill}%
\pgfsetlinewidth{0.803000pt}%
\definecolor{currentstroke}{rgb}{0.000000,0.000000,0.000000}%
\pgfsetstrokecolor{currentstroke}%
\pgfsetdash{}{0pt}%
\pgfsys@defobject{currentmarker}{\pgfqpoint{-0.048611in}{0.000000in}}{\pgfqpoint{-0.000000in}{0.000000in}}{%
\pgfpathmoveto{\pgfqpoint{-0.000000in}{0.000000in}}%
\pgfpathlineto{\pgfqpoint{-0.048611in}{0.000000in}}%
\pgfusepath{stroke,fill}%
}%
\begin{pgfscope}%
\pgfsys@transformshift{0.875000in}{2.483157in}%
\pgfsys@useobject{currentmarker}{}%
\end{pgfscope}%
\end{pgfscope}%
\begin{pgfscope}%
\definecolor{textcolor}{rgb}{0.000000,0.000000,0.000000}%
\pgfsetstrokecolor{textcolor}%
\pgfsetfillcolor{textcolor}%
\pgftext[x=0.468533in, y=2.430395in, left, base]{\color{textcolor}\sffamily\fontsize{10.000000}{12.000000}\selectfont 0.05}%
\end{pgfscope}%
\begin{pgfscope}%
\pgfsetbuttcap%
\pgfsetroundjoin%
\definecolor{currentfill}{rgb}{0.000000,0.000000,0.000000}%
\pgfsetfillcolor{currentfill}%
\pgfsetlinewidth{0.803000pt}%
\definecolor{currentstroke}{rgb}{0.000000,0.000000,0.000000}%
\pgfsetstrokecolor{currentstroke}%
\pgfsetdash{}{0pt}%
\pgfsys@defobject{currentmarker}{\pgfqpoint{-0.048611in}{0.000000in}}{\pgfqpoint{-0.000000in}{0.000000in}}{%
\pgfpathmoveto{\pgfqpoint{-0.000000in}{0.000000in}}%
\pgfpathlineto{\pgfqpoint{-0.048611in}{0.000000in}}%
\pgfusepath{stroke,fill}%
}%
\begin{pgfscope}%
\pgfsys@transformshift{0.875000in}{2.986314in}%
\pgfsys@useobject{currentmarker}{}%
\end{pgfscope}%
\end{pgfscope}%
\begin{pgfscope}%
\definecolor{textcolor}{rgb}{0.000000,0.000000,0.000000}%
\pgfsetstrokecolor{textcolor}%
\pgfsetfillcolor{textcolor}%
\pgftext[x=0.468533in, y=2.933552in, left, base]{\color{textcolor}\sffamily\fontsize{10.000000}{12.000000}\selectfont 0.10}%
\end{pgfscope}%
\begin{pgfscope}%
\pgfsetbuttcap%
\pgfsetroundjoin%
\definecolor{currentfill}{rgb}{0.000000,0.000000,0.000000}%
\pgfsetfillcolor{currentfill}%
\pgfsetlinewidth{0.803000pt}%
\definecolor{currentstroke}{rgb}{0.000000,0.000000,0.000000}%
\pgfsetstrokecolor{currentstroke}%
\pgfsetdash{}{0pt}%
\pgfsys@defobject{currentmarker}{\pgfqpoint{-0.048611in}{0.000000in}}{\pgfqpoint{-0.000000in}{0.000000in}}{%
\pgfpathmoveto{\pgfqpoint{-0.000000in}{0.000000in}}%
\pgfpathlineto{\pgfqpoint{-0.048611in}{0.000000in}}%
\pgfusepath{stroke,fill}%
}%
\begin{pgfscope}%
\pgfsys@transformshift{0.875000in}{3.489470in}%
\pgfsys@useobject{currentmarker}{}%
\end{pgfscope}%
\end{pgfscope}%
\begin{pgfscope}%
\definecolor{textcolor}{rgb}{0.000000,0.000000,0.000000}%
\pgfsetstrokecolor{textcolor}%
\pgfsetfillcolor{textcolor}%
\pgftext[x=0.468533in, y=3.436709in, left, base]{\color{textcolor}\sffamily\fontsize{10.000000}{12.000000}\selectfont 0.15}%
\end{pgfscope}%
\begin{pgfscope}%
\pgfpathrectangle{\pgfqpoint{0.875000in}{0.440000in}}{\pgfqpoint{5.425000in}{3.080000in}}%
\pgfusepath{clip}%
\pgfsetrectcap%
\pgfsetroundjoin%
\pgfsetlinewidth{1.505625pt}%
\definecolor{currentstroke}{rgb}{0.000000,0.000000,0.000000}%
\pgfsetstrokecolor{currentstroke}%
\pgfsetstrokeopacity{0.200000}%
\pgfsetdash{}{0pt}%
\pgfpathmoveto{\pgfqpoint{1.121591in}{2.058022in}}%
\pgfpathlineto{\pgfqpoint{1.144008in}{2.058022in}}%
\pgfpathlineto{\pgfqpoint{1.144008in}{2.084450in}}%
\pgfpathlineto{\pgfqpoint{1.166426in}{2.084450in}}%
\pgfpathlineto{\pgfqpoint{1.166426in}{2.234479in}}%
\pgfpathlineto{\pgfqpoint{1.188843in}{2.234479in}}%
\pgfpathlineto{\pgfqpoint{1.188843in}{2.334473in}}%
\pgfpathlineto{\pgfqpoint{1.211260in}{2.334473in}}%
\pgfpathlineto{\pgfqpoint{1.211260in}{2.214030in}}%
\pgfpathlineto{\pgfqpoint{1.233678in}{2.214030in}}%
\pgfpathlineto{\pgfqpoint{1.233678in}{2.289679in}}%
\pgfpathlineto{\pgfqpoint{1.256095in}{2.289679in}}%
\pgfpathlineto{\pgfqpoint{1.256095in}{2.292441in}}%
\pgfpathlineto{\pgfqpoint{1.278512in}{2.292441in}}%
\pgfpathlineto{\pgfqpoint{1.278512in}{2.518721in}}%
\pgfpathlineto{\pgfqpoint{1.300930in}{2.518721in}}%
\pgfpathlineto{\pgfqpoint{1.300930in}{2.693096in}}%
\pgfpathlineto{\pgfqpoint{1.323347in}{2.693096in}}%
\pgfpathlineto{\pgfqpoint{1.323347in}{2.539911in}}%
\pgfpathlineto{\pgfqpoint{1.345764in}{2.539911in}}%
\pgfpathlineto{\pgfqpoint{1.345764in}{2.258216in}}%
\pgfpathlineto{\pgfqpoint{1.368182in}{2.258216in}}%
\pgfpathlineto{\pgfqpoint{1.368182in}{2.022593in}}%
\pgfpathlineto{\pgfqpoint{1.390599in}{2.022593in}}%
\pgfpathlineto{\pgfqpoint{1.390599in}{1.691430in}}%
\pgfpathlineto{\pgfqpoint{1.413017in}{1.691430in}}%
\pgfpathlineto{\pgfqpoint{1.413017in}{1.387736in}}%
\pgfpathlineto{\pgfqpoint{1.435434in}{1.387736in}}%
\pgfpathlineto{\pgfqpoint{1.435434in}{1.364084in}}%
\pgfpathlineto{\pgfqpoint{1.457851in}{1.364084in}}%
\pgfpathlineto{\pgfqpoint{1.457851in}{1.374447in}}%
\pgfpathlineto{\pgfqpoint{1.480269in}{1.374447in}}%
\pgfpathlineto{\pgfqpoint{1.480269in}{1.668828in}}%
\pgfpathlineto{\pgfqpoint{1.502686in}{1.668828in}}%
\pgfpathlineto{\pgfqpoint{1.502686in}{2.113240in}}%
\pgfpathlineto{\pgfqpoint{1.525103in}{2.113240in}}%
\pgfpathlineto{\pgfqpoint{1.525103in}{2.490036in}}%
\pgfpathlineto{\pgfqpoint{1.547521in}{2.490036in}}%
\pgfpathlineto{\pgfqpoint{1.547521in}{2.372557in}}%
\pgfpathlineto{\pgfqpoint{1.569938in}{2.372557in}}%
\pgfpathlineto{\pgfqpoint{1.569938in}{1.858670in}}%
\pgfpathlineto{\pgfqpoint{1.592355in}{1.858670in}}%
\pgfpathlineto{\pgfqpoint{1.592355in}{1.525524in}}%
\pgfpathlineto{\pgfqpoint{1.614773in}{1.525524in}}%
\pgfpathlineto{\pgfqpoint{1.614773in}{1.377401in}}%
\pgfpathlineto{\pgfqpoint{1.637190in}{1.377401in}}%
\pgfpathlineto{\pgfqpoint{1.637190in}{1.426161in}}%
\pgfpathlineto{\pgfqpoint{1.659607in}{1.426161in}}%
\pgfpathlineto{\pgfqpoint{1.659607in}{1.584538in}}%
\pgfpathlineto{\pgfqpoint{1.682025in}{1.584538in}}%
\pgfpathlineto{\pgfqpoint{1.682025in}{1.858629in}}%
\pgfpathlineto{\pgfqpoint{1.704442in}{1.858629in}}%
\pgfpathlineto{\pgfqpoint{1.704442in}{2.213921in}}%
\pgfpathlineto{\pgfqpoint{1.726860in}{2.213921in}}%
\pgfpathlineto{\pgfqpoint{1.726860in}{2.313546in}}%
\pgfpathlineto{\pgfqpoint{1.749277in}{2.313546in}}%
\pgfpathlineto{\pgfqpoint{1.749277in}{2.251492in}}%
\pgfpathlineto{\pgfqpoint{1.771694in}{2.251492in}}%
\pgfpathlineto{\pgfqpoint{1.771694in}{2.106918in}}%
\pgfpathlineto{\pgfqpoint{1.794112in}{2.106918in}}%
\pgfpathlineto{\pgfqpoint{1.794112in}{1.940703in}}%
\pgfpathlineto{\pgfqpoint{1.816529in}{1.940703in}}%
\pgfpathlineto{\pgfqpoint{1.816529in}{1.830157in}}%
\pgfpathlineto{\pgfqpoint{1.838946in}{1.830157in}}%
\pgfpathlineto{\pgfqpoint{1.838946in}{1.593971in}}%
\pgfpathlineto{\pgfqpoint{1.861364in}{1.593971in}}%
\pgfpathlineto{\pgfqpoint{1.861364in}{1.495126in}}%
\pgfpathlineto{\pgfqpoint{1.883781in}{1.495126in}}%
\pgfpathlineto{\pgfqpoint{1.883781in}{1.608910in}}%
\pgfpathlineto{\pgfqpoint{1.906198in}{1.608910in}}%
\pgfpathlineto{\pgfqpoint{1.906198in}{1.911871in}}%
\pgfpathlineto{\pgfqpoint{1.928616in}{1.911871in}}%
\pgfpathlineto{\pgfqpoint{1.928616in}{2.157539in}}%
\pgfpathlineto{\pgfqpoint{1.951033in}{2.157539in}}%
\pgfpathlineto{\pgfqpoint{1.951033in}{2.333096in}}%
\pgfpathlineto{\pgfqpoint{1.973450in}{2.333096in}}%
\pgfpathlineto{\pgfqpoint{1.973450in}{2.553085in}}%
\pgfpathlineto{\pgfqpoint{1.995868in}{2.553085in}}%
\pgfpathlineto{\pgfqpoint{1.995868in}{2.609304in}}%
\pgfpathlineto{\pgfqpoint{2.018285in}{2.609304in}}%
\pgfpathlineto{\pgfqpoint{2.018285in}{2.536976in}}%
\pgfpathlineto{\pgfqpoint{2.040702in}{2.536976in}}%
\pgfpathlineto{\pgfqpoint{2.040702in}{2.304702in}}%
\pgfpathlineto{\pgfqpoint{2.063120in}{2.304702in}}%
\pgfpathlineto{\pgfqpoint{2.063120in}{2.156759in}}%
\pgfpathlineto{\pgfqpoint{2.085537in}{2.156759in}}%
\pgfpathlineto{\pgfqpoint{2.085537in}{2.369195in}}%
\pgfpathlineto{\pgfqpoint{2.107955in}{2.369195in}}%
\pgfpathlineto{\pgfqpoint{2.107955in}{2.697345in}}%
\pgfpathlineto{\pgfqpoint{2.130372in}{2.697345in}}%
\pgfpathlineto{\pgfqpoint{2.130372in}{2.913581in}}%
\pgfpathlineto{\pgfqpoint{2.152789in}{2.913581in}}%
\pgfpathlineto{\pgfqpoint{2.152789in}{3.018638in}}%
\pgfpathlineto{\pgfqpoint{2.175207in}{3.018638in}}%
\pgfpathlineto{\pgfqpoint{2.175207in}{2.811905in}}%
\pgfpathlineto{\pgfqpoint{2.197624in}{2.811905in}}%
\pgfpathlineto{\pgfqpoint{2.197624in}{2.537266in}}%
\pgfpathlineto{\pgfqpoint{2.220041in}{2.537266in}}%
\pgfpathlineto{\pgfqpoint{2.220041in}{2.298059in}}%
\pgfpathlineto{\pgfqpoint{2.242459in}{2.298059in}}%
\pgfpathlineto{\pgfqpoint{2.242459in}{2.226243in}}%
\pgfpathlineto{\pgfqpoint{2.264876in}{2.226243in}}%
\pgfpathlineto{\pgfqpoint{2.264876in}{2.167852in}}%
\pgfpathlineto{\pgfqpoint{2.287293in}{2.167852in}}%
\pgfpathlineto{\pgfqpoint{2.287293in}{1.910965in}}%
\pgfpathlineto{\pgfqpoint{2.309711in}{1.910965in}}%
\pgfpathlineto{\pgfqpoint{2.309711in}{1.586608in}}%
\pgfpathlineto{\pgfqpoint{2.332128in}{1.586608in}}%
\pgfpathlineto{\pgfqpoint{2.332128in}{1.226944in}}%
\pgfpathlineto{\pgfqpoint{2.354545in}{1.226944in}}%
\pgfpathlineto{\pgfqpoint{2.354545in}{1.342153in}}%
\pgfpathlineto{\pgfqpoint{2.376963in}{1.342153in}}%
\pgfpathlineto{\pgfqpoint{2.376963in}{1.463454in}}%
\pgfpathlineto{\pgfqpoint{2.399380in}{1.463454in}}%
\pgfpathlineto{\pgfqpoint{2.399380in}{1.551943in}}%
\pgfpathlineto{\pgfqpoint{2.421798in}{1.551943in}}%
\pgfpathlineto{\pgfqpoint{2.421798in}{1.764077in}}%
\pgfpathlineto{\pgfqpoint{2.444215in}{1.764077in}}%
\pgfpathlineto{\pgfqpoint{2.444215in}{1.690474in}}%
\pgfpathlineto{\pgfqpoint{2.466632in}{1.690474in}}%
\pgfpathlineto{\pgfqpoint{2.466632in}{1.443981in}}%
\pgfpathlineto{\pgfqpoint{2.489050in}{1.443981in}}%
\pgfpathlineto{\pgfqpoint{2.489050in}{1.438443in}}%
\pgfpathlineto{\pgfqpoint{2.511467in}{1.438443in}}%
\pgfpathlineto{\pgfqpoint{2.511467in}{1.436858in}}%
\pgfpathlineto{\pgfqpoint{2.533884in}{1.436858in}}%
\pgfpathlineto{\pgfqpoint{2.533884in}{1.255702in}}%
\pgfpathlineto{\pgfqpoint{2.556302in}{1.255702in}}%
\pgfpathlineto{\pgfqpoint{2.556302in}{1.369052in}}%
\pgfpathlineto{\pgfqpoint{2.578719in}{1.369052in}}%
\pgfpathlineto{\pgfqpoint{2.578719in}{1.661540in}}%
\pgfpathlineto{\pgfqpoint{2.601136in}{1.661540in}}%
\pgfpathlineto{\pgfqpoint{2.601136in}{1.994127in}}%
\pgfpathlineto{\pgfqpoint{2.623554in}{1.994127in}}%
\pgfpathlineto{\pgfqpoint{2.623554in}{2.048489in}}%
\pgfpathlineto{\pgfqpoint{2.645971in}{2.048489in}}%
\pgfpathlineto{\pgfqpoint{2.645971in}{2.062718in}}%
\pgfpathlineto{\pgfqpoint{2.668388in}{2.062718in}}%
\pgfpathlineto{\pgfqpoint{2.668388in}{2.012058in}}%
\pgfpathlineto{\pgfqpoint{2.690806in}{2.012058in}}%
\pgfpathlineto{\pgfqpoint{2.690806in}{1.721260in}}%
\pgfpathlineto{\pgfqpoint{2.713223in}{1.721260in}}%
\pgfpathlineto{\pgfqpoint{2.713223in}{1.593578in}}%
\pgfpathlineto{\pgfqpoint{2.735640in}{1.593578in}}%
\pgfpathlineto{\pgfqpoint{2.735640in}{1.519563in}}%
\pgfpathlineto{\pgfqpoint{2.758058in}{1.519563in}}%
\pgfpathlineto{\pgfqpoint{2.758058in}{1.446827in}}%
\pgfpathlineto{\pgfqpoint{2.780475in}{1.446827in}}%
\pgfpathlineto{\pgfqpoint{2.780475in}{1.471692in}}%
\pgfpathlineto{\pgfqpoint{2.802893in}{1.471692in}}%
\pgfpathlineto{\pgfqpoint{2.802893in}{1.518703in}}%
\pgfpathlineto{\pgfqpoint{2.825310in}{1.518703in}}%
\pgfpathlineto{\pgfqpoint{2.825310in}{1.595729in}}%
\pgfpathlineto{\pgfqpoint{2.847727in}{1.595729in}}%
\pgfpathlineto{\pgfqpoint{2.847727in}{1.624964in}}%
\pgfpathlineto{\pgfqpoint{2.870145in}{1.624964in}}%
\pgfpathlineto{\pgfqpoint{2.870145in}{1.503693in}}%
\pgfpathlineto{\pgfqpoint{2.892562in}{1.503693in}}%
\pgfpathlineto{\pgfqpoint{2.892562in}{1.537683in}}%
\pgfpathlineto{\pgfqpoint{2.914979in}{1.537683in}}%
\pgfpathlineto{\pgfqpoint{2.914979in}{1.657519in}}%
\pgfpathlineto{\pgfqpoint{2.937397in}{1.657519in}}%
\pgfpathlineto{\pgfqpoint{2.937397in}{1.770854in}}%
\pgfpathlineto{\pgfqpoint{2.959814in}{1.770854in}}%
\pgfpathlineto{\pgfqpoint{2.959814in}{1.657959in}}%
\pgfpathlineto{\pgfqpoint{2.982231in}{1.657959in}}%
\pgfpathlineto{\pgfqpoint{2.982231in}{1.476293in}}%
\pgfpathlineto{\pgfqpoint{3.004649in}{1.476293in}}%
\pgfpathlineto{\pgfqpoint{3.004649in}{1.622567in}}%
\pgfpathlineto{\pgfqpoint{3.027066in}{1.622567in}}%
\pgfpathlineto{\pgfqpoint{3.027066in}{1.858093in}}%
\pgfpathlineto{\pgfqpoint{3.049483in}{1.858093in}}%
\pgfpathlineto{\pgfqpoint{3.049483in}{2.070187in}}%
\pgfpathlineto{\pgfqpoint{3.071901in}{2.070187in}}%
\pgfpathlineto{\pgfqpoint{3.071901in}{2.110904in}}%
\pgfpathlineto{\pgfqpoint{3.094318in}{2.110904in}}%
\pgfpathlineto{\pgfqpoint{3.094318in}{2.079712in}}%
\pgfpathlineto{\pgfqpoint{3.116736in}{2.079712in}}%
\pgfpathlineto{\pgfqpoint{3.116736in}{2.023613in}}%
\pgfpathlineto{\pgfqpoint{3.139153in}{2.023613in}}%
\pgfpathlineto{\pgfqpoint{3.139153in}{2.061392in}}%
\pgfpathlineto{\pgfqpoint{3.161570in}{2.061392in}}%
\pgfpathlineto{\pgfqpoint{3.161570in}{2.197707in}}%
\pgfpathlineto{\pgfqpoint{3.183988in}{2.197707in}}%
\pgfpathlineto{\pgfqpoint{3.183988in}{2.345383in}}%
\pgfpathlineto{\pgfqpoint{3.206405in}{2.345383in}}%
\pgfpathlineto{\pgfqpoint{3.206405in}{2.655528in}}%
\pgfpathlineto{\pgfqpoint{3.228822in}{2.655528in}}%
\pgfpathlineto{\pgfqpoint{3.228822in}{2.939749in}}%
\pgfpathlineto{\pgfqpoint{3.251240in}{2.939749in}}%
\pgfpathlineto{\pgfqpoint{3.251240in}{2.995062in}}%
\pgfpathlineto{\pgfqpoint{3.273657in}{2.995062in}}%
\pgfpathlineto{\pgfqpoint{3.273657in}{2.850281in}}%
\pgfpathlineto{\pgfqpoint{3.296074in}{2.850281in}}%
\pgfpathlineto{\pgfqpoint{3.296074in}{2.726310in}}%
\pgfpathlineto{\pgfqpoint{3.318492in}{2.726310in}}%
\pgfpathlineto{\pgfqpoint{3.318492in}{2.707283in}}%
\pgfpathlineto{\pgfqpoint{3.340909in}{2.707283in}}%
\pgfpathlineto{\pgfqpoint{3.340909in}{2.667446in}}%
\pgfpathlineto{\pgfqpoint{3.363326in}{2.667446in}}%
\pgfpathlineto{\pgfqpoint{3.363326in}{2.794307in}}%
\pgfpathlineto{\pgfqpoint{3.385744in}{2.794307in}}%
\pgfpathlineto{\pgfqpoint{3.385744in}{3.038542in}}%
\pgfpathlineto{\pgfqpoint{3.408161in}{3.038542in}}%
\pgfpathlineto{\pgfqpoint{3.408161in}{2.966235in}}%
\pgfpathlineto{\pgfqpoint{3.430579in}{2.966235in}}%
\pgfpathlineto{\pgfqpoint{3.430579in}{2.749982in}}%
\pgfpathlineto{\pgfqpoint{3.452996in}{2.749982in}}%
\pgfpathlineto{\pgfqpoint{3.452996in}{2.488174in}}%
\pgfpathlineto{\pgfqpoint{3.475413in}{2.488174in}}%
\pgfpathlineto{\pgfqpoint{3.475413in}{2.476029in}}%
\pgfpathlineto{\pgfqpoint{3.497831in}{2.476029in}}%
\pgfpathlineto{\pgfqpoint{3.497831in}{2.671480in}}%
\pgfpathlineto{\pgfqpoint{3.520248in}{2.671480in}}%
\pgfpathlineto{\pgfqpoint{3.520248in}{2.796016in}}%
\pgfpathlineto{\pgfqpoint{3.542665in}{2.796016in}}%
\pgfpathlineto{\pgfqpoint{3.542665in}{3.030074in}}%
\pgfpathlineto{\pgfqpoint{3.565083in}{3.030074in}}%
\pgfpathlineto{\pgfqpoint{3.565083in}{3.036443in}}%
\pgfpathlineto{\pgfqpoint{3.587500in}{3.036443in}}%
\pgfpathlineto{\pgfqpoint{3.587500in}{2.704764in}}%
\pgfpathlineto{\pgfqpoint{3.609917in}{2.704764in}}%
\pgfpathlineto{\pgfqpoint{3.609917in}{2.459241in}}%
\pgfpathlineto{\pgfqpoint{3.632335in}{2.459241in}}%
\pgfpathlineto{\pgfqpoint{3.632335in}{2.328464in}}%
\pgfpathlineto{\pgfqpoint{3.654752in}{2.328464in}}%
\pgfpathlineto{\pgfqpoint{3.654752in}{1.969379in}}%
\pgfpathlineto{\pgfqpoint{3.677169in}{1.969379in}}%
\pgfpathlineto{\pgfqpoint{3.677169in}{1.821972in}}%
\pgfpathlineto{\pgfqpoint{3.699587in}{1.821972in}}%
\pgfpathlineto{\pgfqpoint{3.699587in}{1.828324in}}%
\pgfpathlineto{\pgfqpoint{3.722004in}{1.828324in}}%
\pgfpathlineto{\pgfqpoint{3.722004in}{1.630574in}}%
\pgfpathlineto{\pgfqpoint{3.744421in}{1.630574in}}%
\pgfpathlineto{\pgfqpoint{3.744421in}{1.432419in}}%
\pgfpathlineto{\pgfqpoint{3.766839in}{1.432419in}}%
\pgfpathlineto{\pgfqpoint{3.766839in}{1.176141in}}%
\pgfpathlineto{\pgfqpoint{3.789256in}{1.176141in}}%
\pgfpathlineto{\pgfqpoint{3.789256in}{0.923626in}}%
\pgfpathlineto{\pgfqpoint{3.811674in}{0.923626in}}%
\pgfpathlineto{\pgfqpoint{3.811674in}{0.717661in}}%
\pgfpathlineto{\pgfqpoint{3.834091in}{0.717661in}}%
\pgfpathlineto{\pgfqpoint{3.834091in}{0.580000in}}%
\pgfpathlineto{\pgfqpoint{3.856508in}{0.580000in}}%
\pgfpathlineto{\pgfqpoint{3.856508in}{0.652722in}}%
\pgfpathlineto{\pgfqpoint{3.878926in}{0.652722in}}%
\pgfpathlineto{\pgfqpoint{3.878926in}{0.778918in}}%
\pgfpathlineto{\pgfqpoint{3.901343in}{0.778918in}}%
\pgfpathlineto{\pgfqpoint{3.901343in}{0.884488in}}%
\pgfpathlineto{\pgfqpoint{3.923760in}{0.884488in}}%
\pgfpathlineto{\pgfqpoint{3.923760in}{0.967680in}}%
\pgfpathlineto{\pgfqpoint{3.946178in}{0.967680in}}%
\pgfpathlineto{\pgfqpoint{3.946178in}{0.982952in}}%
\pgfpathlineto{\pgfqpoint{3.968595in}{0.982952in}}%
\pgfpathlineto{\pgfqpoint{3.968595in}{1.243037in}}%
\pgfpathlineto{\pgfqpoint{3.991012in}{1.243037in}}%
\pgfpathlineto{\pgfqpoint{3.991012in}{1.429357in}}%
\pgfpathlineto{\pgfqpoint{4.013430in}{1.429357in}}%
\pgfpathlineto{\pgfqpoint{4.013430in}{1.522711in}}%
\pgfpathlineto{\pgfqpoint{4.035847in}{1.522711in}}%
\pgfpathlineto{\pgfqpoint{4.035847in}{1.596793in}}%
\pgfpathlineto{\pgfqpoint{4.058264in}{1.596793in}}%
\pgfpathlineto{\pgfqpoint{4.058264in}{1.392497in}}%
\pgfpathlineto{\pgfqpoint{4.080682in}{1.392497in}}%
\pgfpathlineto{\pgfqpoint{4.080682in}{1.197288in}}%
\pgfpathlineto{\pgfqpoint{4.103099in}{1.197288in}}%
\pgfpathlineto{\pgfqpoint{4.103099in}{1.186204in}}%
\pgfpathlineto{\pgfqpoint{4.125517in}{1.186204in}}%
\pgfpathlineto{\pgfqpoint{4.125517in}{1.524415in}}%
\pgfpathlineto{\pgfqpoint{4.147934in}{1.524415in}}%
\pgfpathlineto{\pgfqpoint{4.147934in}{1.983388in}}%
\pgfpathlineto{\pgfqpoint{4.170351in}{1.983388in}}%
\pgfpathlineto{\pgfqpoint{4.170351in}{2.188373in}}%
\pgfpathlineto{\pgfqpoint{4.192769in}{2.188373in}}%
\pgfpathlineto{\pgfqpoint{4.192769in}{2.066593in}}%
\pgfpathlineto{\pgfqpoint{4.215186in}{2.066593in}}%
\pgfpathlineto{\pgfqpoint{4.215186in}{1.902309in}}%
\pgfpathlineto{\pgfqpoint{4.237603in}{1.902309in}}%
\pgfpathlineto{\pgfqpoint{4.237603in}{1.724023in}}%
\pgfpathlineto{\pgfqpoint{4.260021in}{1.724023in}}%
\pgfpathlineto{\pgfqpoint{4.260021in}{1.613265in}}%
\pgfpathlineto{\pgfqpoint{4.282438in}{1.613265in}}%
\pgfpathlineto{\pgfqpoint{4.282438in}{1.760736in}}%
\pgfpathlineto{\pgfqpoint{4.304855in}{1.760736in}}%
\pgfpathlineto{\pgfqpoint{4.304855in}{1.894372in}}%
\pgfpathlineto{\pgfqpoint{4.327273in}{1.894372in}}%
\pgfpathlineto{\pgfqpoint{4.327273in}{2.179590in}}%
\pgfpathlineto{\pgfqpoint{4.349690in}{2.179590in}}%
\pgfpathlineto{\pgfqpoint{4.349690in}{2.632555in}}%
\pgfpathlineto{\pgfqpoint{4.372107in}{2.632555in}}%
\pgfpathlineto{\pgfqpoint{4.372107in}{2.843803in}}%
\pgfpathlineto{\pgfqpoint{4.394525in}{2.843803in}}%
\pgfpathlineto{\pgfqpoint{4.394525in}{2.790534in}}%
\pgfpathlineto{\pgfqpoint{4.416942in}{2.790534in}}%
\pgfpathlineto{\pgfqpoint{4.416942in}{2.702782in}}%
\pgfpathlineto{\pgfqpoint{4.439360in}{2.702782in}}%
\pgfpathlineto{\pgfqpoint{4.439360in}{2.609916in}}%
\pgfpathlineto{\pgfqpoint{4.461777in}{2.609916in}}%
\pgfpathlineto{\pgfqpoint{4.461777in}{2.585518in}}%
\pgfpathlineto{\pgfqpoint{4.484194in}{2.585518in}}%
\pgfpathlineto{\pgfqpoint{4.484194in}{2.503995in}}%
\pgfpathlineto{\pgfqpoint{4.506612in}{2.503995in}}%
\pgfpathlineto{\pgfqpoint{4.506612in}{2.370865in}}%
\pgfpathlineto{\pgfqpoint{4.529029in}{2.370865in}}%
\pgfpathlineto{\pgfqpoint{4.529029in}{2.208628in}}%
\pgfpathlineto{\pgfqpoint{4.551446in}{2.208628in}}%
\pgfpathlineto{\pgfqpoint{4.551446in}{2.105463in}}%
\pgfpathlineto{\pgfqpoint{4.573864in}{2.105463in}}%
\pgfpathlineto{\pgfqpoint{4.573864in}{1.995566in}}%
\pgfpathlineto{\pgfqpoint{4.596281in}{1.995566in}}%
\pgfpathlineto{\pgfqpoint{4.596281in}{1.953952in}}%
\pgfpathlineto{\pgfqpoint{4.618698in}{1.953952in}}%
\pgfpathlineto{\pgfqpoint{4.618698in}{2.272984in}}%
\pgfpathlineto{\pgfqpoint{4.641116in}{2.272984in}}%
\pgfpathlineto{\pgfqpoint{4.641116in}{2.490249in}}%
\pgfpathlineto{\pgfqpoint{4.663533in}{2.490249in}}%
\pgfpathlineto{\pgfqpoint{4.663533in}{2.591152in}}%
\pgfpathlineto{\pgfqpoint{4.685950in}{2.591152in}}%
\pgfpathlineto{\pgfqpoint{4.685950in}{2.464620in}}%
\pgfpathlineto{\pgfqpoint{4.708368in}{2.464620in}}%
\pgfpathlineto{\pgfqpoint{4.708368in}{2.047797in}}%
\pgfpathlineto{\pgfqpoint{4.730785in}{2.047797in}}%
\pgfpathlineto{\pgfqpoint{4.730785in}{1.707453in}}%
\pgfpathlineto{\pgfqpoint{4.753202in}{1.707453in}}%
\pgfpathlineto{\pgfqpoint{4.753202in}{1.740183in}}%
\pgfpathlineto{\pgfqpoint{4.775620in}{1.740183in}}%
\pgfpathlineto{\pgfqpoint{4.775620in}{1.931578in}}%
\pgfpathlineto{\pgfqpoint{4.798037in}{1.931578in}}%
\pgfpathlineto{\pgfqpoint{4.798037in}{2.030160in}}%
\pgfpathlineto{\pgfqpoint{4.820455in}{2.030160in}}%
\pgfpathlineto{\pgfqpoint{4.820455in}{2.163025in}}%
\pgfpathlineto{\pgfqpoint{4.842872in}{2.163025in}}%
\pgfpathlineto{\pgfqpoint{4.842872in}{2.201612in}}%
\pgfpathlineto{\pgfqpoint{4.865289in}{2.201612in}}%
\pgfpathlineto{\pgfqpoint{4.865289in}{2.165345in}}%
\pgfpathlineto{\pgfqpoint{4.887707in}{2.165345in}}%
\pgfpathlineto{\pgfqpoint{4.887707in}{2.186260in}}%
\pgfpathlineto{\pgfqpoint{4.910124in}{2.186260in}}%
\pgfpathlineto{\pgfqpoint{4.910124in}{2.288747in}}%
\pgfpathlineto{\pgfqpoint{4.932541in}{2.288747in}}%
\pgfpathlineto{\pgfqpoint{4.932541in}{2.354617in}}%
\pgfpathlineto{\pgfqpoint{4.954959in}{2.354617in}}%
\pgfpathlineto{\pgfqpoint{4.954959in}{2.318984in}}%
\pgfpathlineto{\pgfqpoint{4.977376in}{2.318984in}}%
\pgfpathlineto{\pgfqpoint{4.977376in}{2.120344in}}%
\pgfpathlineto{\pgfqpoint{4.999793in}{2.120344in}}%
\pgfpathlineto{\pgfqpoint{4.999793in}{1.977933in}}%
\pgfpathlineto{\pgfqpoint{5.022211in}{1.977933in}}%
\pgfpathlineto{\pgfqpoint{5.022211in}{1.896426in}}%
\pgfpathlineto{\pgfqpoint{5.044628in}{1.896426in}}%
\pgfpathlineto{\pgfqpoint{5.044628in}{1.852990in}}%
\pgfpathlineto{\pgfqpoint{5.067045in}{1.852990in}}%
\pgfpathlineto{\pgfqpoint{5.067045in}{1.898183in}}%
\pgfpathlineto{\pgfqpoint{5.089463in}{1.898183in}}%
\pgfpathlineto{\pgfqpoint{5.089463in}{2.012342in}}%
\pgfpathlineto{\pgfqpoint{5.111880in}{2.012342in}}%
\pgfpathlineto{\pgfqpoint{5.111880in}{2.149772in}}%
\pgfpathlineto{\pgfqpoint{5.134298in}{2.149772in}}%
\pgfpathlineto{\pgfqpoint{5.134298in}{2.027736in}}%
\pgfpathlineto{\pgfqpoint{5.156715in}{2.027736in}}%
\pgfpathlineto{\pgfqpoint{5.156715in}{1.726498in}}%
\pgfpathlineto{\pgfqpoint{5.179132in}{1.726498in}}%
\pgfpathlineto{\pgfqpoint{5.179132in}{1.331941in}}%
\pgfpathlineto{\pgfqpoint{5.201550in}{1.331941in}}%
\pgfpathlineto{\pgfqpoint{5.201550in}{1.100421in}}%
\pgfpathlineto{\pgfqpoint{5.223967in}{1.100421in}}%
\pgfpathlineto{\pgfqpoint{5.223967in}{1.093527in}}%
\pgfpathlineto{\pgfqpoint{5.246384in}{1.093527in}}%
\pgfpathlineto{\pgfqpoint{5.246384in}{1.259635in}}%
\pgfpathlineto{\pgfqpoint{5.268802in}{1.259635in}}%
\pgfpathlineto{\pgfqpoint{5.268802in}{1.627939in}}%
\pgfpathlineto{\pgfqpoint{5.291219in}{1.627939in}}%
\pgfpathlineto{\pgfqpoint{5.291219in}{1.926206in}}%
\pgfpathlineto{\pgfqpoint{5.313636in}{1.926206in}}%
\pgfpathlineto{\pgfqpoint{5.313636in}{1.990775in}}%
\pgfpathlineto{\pgfqpoint{5.336054in}{1.990775in}}%
\pgfpathlineto{\pgfqpoint{5.336054in}{1.889110in}}%
\pgfpathlineto{\pgfqpoint{5.358471in}{1.889110in}}%
\pgfpathlineto{\pgfqpoint{5.358471in}{1.799930in}}%
\pgfpathlineto{\pgfqpoint{5.380888in}{1.799930in}}%
\pgfpathlineto{\pgfqpoint{5.380888in}{1.711618in}}%
\pgfpathlineto{\pgfqpoint{5.403306in}{1.711618in}}%
\pgfpathlineto{\pgfqpoint{5.403306in}{1.743163in}}%
\pgfpathlineto{\pgfqpoint{5.425723in}{1.743163in}}%
\pgfpathlineto{\pgfqpoint{5.425723in}{1.921925in}}%
\pgfpathlineto{\pgfqpoint{5.448140in}{1.921925in}}%
\pgfpathlineto{\pgfqpoint{5.448140in}{2.149133in}}%
\pgfpathlineto{\pgfqpoint{5.470558in}{2.149133in}}%
\pgfpathlineto{\pgfqpoint{5.470558in}{2.349884in}}%
\pgfpathlineto{\pgfqpoint{5.492975in}{2.349884in}}%
\pgfpathlineto{\pgfqpoint{5.492975in}{2.321403in}}%
\pgfpathlineto{\pgfqpoint{5.515393in}{2.321403in}}%
\pgfpathlineto{\pgfqpoint{5.515393in}{2.293351in}}%
\pgfpathlineto{\pgfqpoint{5.537810in}{2.293351in}}%
\pgfpathlineto{\pgfqpoint{5.537810in}{2.490788in}}%
\pgfpathlineto{\pgfqpoint{5.560227in}{2.490788in}}%
\pgfpathlineto{\pgfqpoint{5.560227in}{2.580157in}}%
\pgfpathlineto{\pgfqpoint{5.582645in}{2.580157in}}%
\pgfpathlineto{\pgfqpoint{5.582645in}{2.623500in}}%
\pgfpathlineto{\pgfqpoint{5.605062in}{2.623500in}}%
\pgfpathlineto{\pgfqpoint{5.605062in}{2.561411in}}%
\pgfpathlineto{\pgfqpoint{5.627479in}{2.561411in}}%
\pgfpathlineto{\pgfqpoint{5.627479in}{2.336427in}}%
\pgfpathlineto{\pgfqpoint{5.649897in}{2.336427in}}%
\pgfpathlineto{\pgfqpoint{5.649897in}{2.006344in}}%
\pgfpathlineto{\pgfqpoint{5.672314in}{2.006344in}}%
\pgfpathlineto{\pgfqpoint{5.672314in}{1.489450in}}%
\pgfpathlineto{\pgfqpoint{5.694731in}{1.489450in}}%
\pgfpathlineto{\pgfqpoint{5.694731in}{1.216123in}}%
\pgfpathlineto{\pgfqpoint{5.717149in}{1.216123in}}%
\pgfpathlineto{\pgfqpoint{5.717149in}{1.183774in}}%
\pgfpathlineto{\pgfqpoint{5.739566in}{1.183774in}}%
\pgfpathlineto{\pgfqpoint{5.739566in}{1.538189in}}%
\pgfpathlineto{\pgfqpoint{5.761983in}{1.538189in}}%
\pgfpathlineto{\pgfqpoint{5.761983in}{1.986205in}}%
\pgfpathlineto{\pgfqpoint{5.784401in}{1.986205in}}%
\pgfpathlineto{\pgfqpoint{5.784401in}{2.315928in}}%
\pgfpathlineto{\pgfqpoint{5.806818in}{2.315928in}}%
\pgfpathlineto{\pgfqpoint{5.806818in}{2.471752in}}%
\pgfpathlineto{\pgfqpoint{5.829236in}{2.471752in}}%
\pgfpathlineto{\pgfqpoint{5.829236in}{2.151395in}}%
\pgfpathlineto{\pgfqpoint{5.851653in}{2.151395in}}%
\pgfpathlineto{\pgfqpoint{5.851653in}{1.860001in}}%
\pgfpathlineto{\pgfqpoint{5.874070in}{1.860001in}}%
\pgfpathlineto{\pgfqpoint{5.874070in}{1.773168in}}%
\pgfpathlineto{\pgfqpoint{5.896488in}{1.773168in}}%
\pgfpathlineto{\pgfqpoint{5.896488in}{1.749116in}}%
\pgfpathlineto{\pgfqpoint{5.918905in}{1.749116in}}%
\pgfpathlineto{\pgfqpoint{5.918905in}{1.597052in}}%
\pgfpathlineto{\pgfqpoint{5.941322in}{1.597052in}}%
\pgfpathlineto{\pgfqpoint{5.941322in}{1.642989in}}%
\pgfpathlineto{\pgfqpoint{5.963740in}{1.642989in}}%
\pgfpathlineto{\pgfqpoint{5.963740in}{1.837673in}}%
\pgfpathlineto{\pgfqpoint{5.986157in}{1.837673in}}%
\pgfpathlineto{\pgfqpoint{5.986157in}{1.840384in}}%
\pgfpathlineto{\pgfqpoint{6.008574in}{1.840384in}}%
\pgfpathlineto{\pgfqpoint{6.008574in}{1.867273in}}%
\pgfpathlineto{\pgfqpoint{6.030992in}{1.867273in}}%
\pgfpathlineto{\pgfqpoint{6.030992in}{1.911252in}}%
\pgfpathlineto{\pgfqpoint{6.053409in}{1.911252in}}%
\pgfpathlineto{\pgfqpoint{6.053409in}{1.980000in}}%
\pgfpathlineto{\pgfqpoint{6.053409in}{1.980000in}}%
\pgfusepath{stroke}%
\end{pgfscope}%
\begin{pgfscope}%
\pgfpathrectangle{\pgfqpoint{0.875000in}{0.440000in}}{\pgfqpoint{5.425000in}{3.080000in}}%
\pgfusepath{clip}%
\pgfsetbuttcap%
\pgfsetroundjoin%
\pgfsetlinewidth{1.505625pt}%
\definecolor{currentstroke}{rgb}{1.000000,0.000000,0.000000}%
\pgfsetstrokecolor{currentstroke}%
\pgfsetstrokeopacity{0.900000}%
\pgfsetdash{{5.550000pt}{2.400000pt}}{0.000000pt}%
\pgfpathmoveto{\pgfqpoint{1.121591in}{1.980000in}}%
\pgfpathlineto{\pgfqpoint{1.121591in}{2.058022in}}%
\pgfusepath{stroke}%
\end{pgfscope}%
\begin{pgfscope}%
\pgfpathrectangle{\pgfqpoint{0.875000in}{0.440000in}}{\pgfqpoint{5.425000in}{3.080000in}}%
\pgfusepath{clip}%
\pgfsetbuttcap%
\pgfsetroundjoin%
\pgfsetlinewidth{1.505625pt}%
\definecolor{currentstroke}{rgb}{1.000000,0.000000,0.000000}%
\pgfsetstrokecolor{currentstroke}%
\pgfsetstrokeopacity{0.900000}%
\pgfsetdash{{5.550000pt}{2.400000pt}}{0.000000pt}%
\pgfpathmoveto{\pgfqpoint{1.682025in}{1.980000in}}%
\pgfpathlineto{\pgfqpoint{1.682025in}{1.858629in}}%
\pgfusepath{stroke}%
\end{pgfscope}%
\begin{pgfscope}%
\pgfpathrectangle{\pgfqpoint{0.875000in}{0.440000in}}{\pgfqpoint{5.425000in}{3.080000in}}%
\pgfusepath{clip}%
\pgfsetbuttcap%
\pgfsetroundjoin%
\pgfsetlinewidth{1.505625pt}%
\definecolor{currentstroke}{rgb}{1.000000,0.000000,0.000000}%
\pgfsetstrokecolor{currentstroke}%
\pgfsetstrokeopacity{0.900000}%
\pgfsetdash{{5.550000pt}{2.400000pt}}{0.000000pt}%
\pgfpathmoveto{\pgfqpoint{2.242459in}{1.980000in}}%
\pgfpathlineto{\pgfqpoint{2.242459in}{2.226243in}}%
\pgfusepath{stroke}%
\end{pgfscope}%
\begin{pgfscope}%
\pgfpathrectangle{\pgfqpoint{0.875000in}{0.440000in}}{\pgfqpoint{5.425000in}{3.080000in}}%
\pgfusepath{clip}%
\pgfsetbuttcap%
\pgfsetroundjoin%
\pgfsetlinewidth{1.505625pt}%
\definecolor{currentstroke}{rgb}{1.000000,0.000000,0.000000}%
\pgfsetstrokecolor{currentstroke}%
\pgfsetstrokeopacity{0.900000}%
\pgfsetdash{{5.550000pt}{2.400000pt}}{0.000000pt}%
\pgfpathmoveto{\pgfqpoint{2.802893in}{1.980000in}}%
\pgfpathlineto{\pgfqpoint{2.802893in}{1.518703in}}%
\pgfusepath{stroke}%
\end{pgfscope}%
\begin{pgfscope}%
\pgfpathrectangle{\pgfqpoint{0.875000in}{0.440000in}}{\pgfqpoint{5.425000in}{3.080000in}}%
\pgfusepath{clip}%
\pgfsetbuttcap%
\pgfsetroundjoin%
\pgfsetlinewidth{1.505625pt}%
\definecolor{currentstroke}{rgb}{1.000000,0.000000,0.000000}%
\pgfsetstrokecolor{currentstroke}%
\pgfsetstrokeopacity{0.900000}%
\pgfsetdash{{5.550000pt}{2.400000pt}}{0.000000pt}%
\pgfpathmoveto{\pgfqpoint{3.363326in}{1.980000in}}%
\pgfpathlineto{\pgfqpoint{3.363326in}{2.794307in}}%
\pgfusepath{stroke}%
\end{pgfscope}%
\begin{pgfscope}%
\pgfpathrectangle{\pgfqpoint{0.875000in}{0.440000in}}{\pgfqpoint{5.425000in}{3.080000in}}%
\pgfusepath{clip}%
\pgfsetbuttcap%
\pgfsetroundjoin%
\pgfsetlinewidth{1.505625pt}%
\definecolor{currentstroke}{rgb}{1.000000,0.000000,0.000000}%
\pgfsetstrokecolor{currentstroke}%
\pgfsetstrokeopacity{0.900000}%
\pgfsetdash{{5.550000pt}{2.400000pt}}{0.000000pt}%
\pgfpathmoveto{\pgfqpoint{3.923760in}{1.980000in}}%
\pgfpathlineto{\pgfqpoint{3.923760in}{0.967680in}}%
\pgfusepath{stroke}%
\end{pgfscope}%
\begin{pgfscope}%
\pgfpathrectangle{\pgfqpoint{0.875000in}{0.440000in}}{\pgfqpoint{5.425000in}{3.080000in}}%
\pgfusepath{clip}%
\pgfsetbuttcap%
\pgfsetroundjoin%
\pgfsetlinewidth{1.505625pt}%
\definecolor{currentstroke}{rgb}{1.000000,0.000000,0.000000}%
\pgfsetstrokecolor{currentstroke}%
\pgfsetstrokeopacity{0.900000}%
\pgfsetdash{{5.550000pt}{2.400000pt}}{0.000000pt}%
\pgfpathmoveto{\pgfqpoint{4.484194in}{1.980000in}}%
\pgfpathlineto{\pgfqpoint{4.484194in}{2.503995in}}%
\pgfusepath{stroke}%
\end{pgfscope}%
\begin{pgfscope}%
\pgfpathrectangle{\pgfqpoint{0.875000in}{0.440000in}}{\pgfqpoint{5.425000in}{3.080000in}}%
\pgfusepath{clip}%
\pgfsetbuttcap%
\pgfsetroundjoin%
\pgfsetlinewidth{1.505625pt}%
\definecolor{currentstroke}{rgb}{1.000000,0.000000,0.000000}%
\pgfsetstrokecolor{currentstroke}%
\pgfsetstrokeopacity{0.900000}%
\pgfsetdash{{5.550000pt}{2.400000pt}}{0.000000pt}%
\pgfpathmoveto{\pgfqpoint{5.044628in}{1.980000in}}%
\pgfpathlineto{\pgfqpoint{5.044628in}{1.852990in}}%
\pgfusepath{stroke}%
\end{pgfscope}%
\begin{pgfscope}%
\pgfpathrectangle{\pgfqpoint{0.875000in}{0.440000in}}{\pgfqpoint{5.425000in}{3.080000in}}%
\pgfusepath{clip}%
\pgfsetbuttcap%
\pgfsetroundjoin%
\pgfsetlinewidth{1.505625pt}%
\definecolor{currentstroke}{rgb}{1.000000,0.000000,0.000000}%
\pgfsetstrokecolor{currentstroke}%
\pgfsetstrokeopacity{0.900000}%
\pgfsetdash{{5.550000pt}{2.400000pt}}{0.000000pt}%
\pgfpathmoveto{\pgfqpoint{5.605062in}{1.980000in}}%
\pgfpathlineto{\pgfqpoint{5.605062in}{2.561411in}}%
\pgfusepath{stroke}%
\end{pgfscope}%
\begin{pgfscope}%
\pgfsetrectcap%
\pgfsetmiterjoin%
\pgfsetlinewidth{0.803000pt}%
\definecolor{currentstroke}{rgb}{0.000000,0.000000,0.000000}%
\pgfsetstrokecolor{currentstroke}%
\pgfsetdash{}{0pt}%
\pgfpathmoveto{\pgfqpoint{0.875000in}{0.440000in}}%
\pgfpathlineto{\pgfqpoint{0.875000in}{3.520000in}}%
\pgfusepath{stroke}%
\end{pgfscope}%
\begin{pgfscope}%
\pgfsetrectcap%
\pgfsetmiterjoin%
\pgfsetlinewidth{0.803000pt}%
\definecolor{currentstroke}{rgb}{0.000000,0.000000,0.000000}%
\pgfsetstrokecolor{currentstroke}%
\pgfsetdash{}{0pt}%
\pgfpathmoveto{\pgfqpoint{6.300000in}{0.440000in}}%
\pgfpathlineto{\pgfqpoint{6.300000in}{3.520000in}}%
\pgfusepath{stroke}%
\end{pgfscope}%
\begin{pgfscope}%
\pgfsetrectcap%
\pgfsetmiterjoin%
\pgfsetlinewidth{0.803000pt}%
\definecolor{currentstroke}{rgb}{0.000000,0.000000,0.000000}%
\pgfsetstrokecolor{currentstroke}%
\pgfsetdash{}{0pt}%
\pgfpathmoveto{\pgfqpoint{0.875000in}{0.440000in}}%
\pgfpathlineto{\pgfqpoint{6.300000in}{0.440000in}}%
\pgfusepath{stroke}%
\end{pgfscope}%
\begin{pgfscope}%
\pgfsetrectcap%
\pgfsetmiterjoin%
\pgfsetlinewidth{0.803000pt}%
\definecolor{currentstroke}{rgb}{0.000000,0.000000,0.000000}%
\pgfsetstrokecolor{currentstroke}%
\pgfsetdash{}{0pt}%
\pgfpathmoveto{\pgfqpoint{0.875000in}{3.520000in}}%
\pgfpathlineto{\pgfqpoint{6.300000in}{3.520000in}}%
\pgfusepath{stroke}%
\end{pgfscope}%
\begin{pgfscope}%
\definecolor{textcolor}{rgb}{0.000000,0.000000,0.000000}%
\pgfsetstrokecolor{textcolor}%
\pgfsetfillcolor{textcolor}%
\pgftext[x=1.121591in,y=2.058022in,left,base]{\color{textcolor}\sffamily\fontsize{10.000000}{12.000000}\selectfont 0.01}%
\end{pgfscope}%
\begin{pgfscope}%
\definecolor{textcolor}{rgb}{0.000000,0.000000,0.000000}%
\pgfsetstrokecolor{textcolor}%
\pgfsetfillcolor{textcolor}%
\pgftext[x=1.682025in,y=1.858629in,left,base]{\color{textcolor}\sffamily\fontsize{10.000000}{12.000000}\selectfont -0.01}%
\end{pgfscope}%
\begin{pgfscope}%
\definecolor{textcolor}{rgb}{0.000000,0.000000,0.000000}%
\pgfsetstrokecolor{textcolor}%
\pgfsetfillcolor{textcolor}%
\pgftext[x=2.242459in,y=2.226243in,left,base]{\color{textcolor}\sffamily\fontsize{10.000000}{12.000000}\selectfont 0.02}%
\end{pgfscope}%
\begin{pgfscope}%
\definecolor{textcolor}{rgb}{0.000000,0.000000,0.000000}%
\pgfsetstrokecolor{textcolor}%
\pgfsetfillcolor{textcolor}%
\pgftext[x=2.802893in,y=1.518703in,left,base]{\color{textcolor}\sffamily\fontsize{10.000000}{12.000000}\selectfont -0.05}%
\end{pgfscope}%
\begin{pgfscope}%
\definecolor{textcolor}{rgb}{0.000000,0.000000,0.000000}%
\pgfsetstrokecolor{textcolor}%
\pgfsetfillcolor{textcolor}%
\pgftext[x=3.363326in,y=2.794307in,left,base]{\color{textcolor}\sffamily\fontsize{10.000000}{12.000000}\selectfont 0.08}%
\end{pgfscope}%
\begin{pgfscope}%
\definecolor{textcolor}{rgb}{0.000000,0.000000,0.000000}%
\pgfsetstrokecolor{textcolor}%
\pgfsetfillcolor{textcolor}%
\pgftext[x=3.923760in,y=0.967680in,left,base]{\color{textcolor}\sffamily\fontsize{10.000000}{12.000000}\selectfont -0.1}%
\end{pgfscope}%
\begin{pgfscope}%
\definecolor{textcolor}{rgb}{0.000000,0.000000,0.000000}%
\pgfsetstrokecolor{textcolor}%
\pgfsetfillcolor{textcolor}%
\pgftext[x=4.484194in,y=2.503995in,left,base]{\color{textcolor}\sffamily\fontsize{10.000000}{12.000000}\selectfont 0.05}%
\end{pgfscope}%
\begin{pgfscope}%
\definecolor{textcolor}{rgb}{0.000000,0.000000,0.000000}%
\pgfsetstrokecolor{textcolor}%
\pgfsetfillcolor{textcolor}%
\pgftext[x=5.044628in,y=1.852990in,left,base]{\color{textcolor}\sffamily\fontsize{10.000000}{12.000000}\selectfont -0.01}%
\end{pgfscope}%
\begin{pgfscope}%
\definecolor{textcolor}{rgb}{0.000000,0.000000,0.000000}%
\pgfsetstrokecolor{textcolor}%
\pgfsetfillcolor{textcolor}%
\pgftext[x=5.605062in,y=2.561411in,left,base]{\color{textcolor}\sffamily\fontsize{10.000000}{12.000000}\selectfont 0.06}%
\end{pgfscope}%
\end{pgfpicture}%
\makeatother%
\endgroup%
}
    \end{center}
    \caption{\emph{Ampiezza istantanea}, misurata ad intervalli regolari di 100 samples con frequenza di campionamento 22050Hz}
\end{figure}


L'\emph{ampiezza istantanea} è l'ampiezza misurata a un istante specifico del fenomeno sonoro. È un valore reale%
\footnote{I numeri reali, appartenenti all'insieme $\mathbb{R}$, sono tutti i numeri rappresentabili su una retta. Per chi non ha mai incontrato i numeri complessi, appartenenti all'insieme $\mathbb{C}$ e rappresentabili su un piano, l'insieme $\mathbb{R}$ è, intuitivamente, l'insieme di \emph{tutti i numeri}. Purtroppo o per fortuna, la matematica moderna è un po' più complicata di così. Dovremo accennare altre volte a $\mathbb{C}$, perché i numeri complessi sono molto utili per rappresentare certi aspetti dei segnali audio.}
\emph{segnato}%
\footnote{Cioè, potenzialmente negativo.}
: convenzionalmente, è positiva se rappresenta una pressione nella direzione nella direzione dall'emittente al recettore; negativa se rappresenta una pressione nella direzione dal recettore al ricevente (cioè, dal punto di vista del ricevente, una depressione). 

Le ragioni per la scelta dell'istante a cui compiere la misurazione possono essere varie, ma in generale l'ampiezza istantanea viene misurata ripetutamente, a intervalli di tempo regolari, nelle operazioni di campionamento del segnale: se le misure sono abbastanza frequenti, la successione delle ampiezze istantanee può costituire una buona approssimazione numerica dell'andamento del segnale. Di fatto, virtualmente tutte le tecniche di campionamento digitale del segnale audio sono basate su questo principio. 

Di per sé, l'ampiezza istantanea non ci dà un'informazione affidabile sul livello dinamico percepito, perché qualsiasi fenomeno sonoro è costituito dalla rapida alternanza di pressioni relativamente alte (cioè pressioni che, in una direzione o nell'altra, deviano in maniera importante dalla pressione atmosferica media) e pressioni relativamente basse (cioè pressioni prossime alla pressione atmosferica media). 


\subsection{Ampiezza di picco}

\begin{figure}
    \begin{center}
       \scalebox{0.6} {%% Creator: Matplotlib, PGF backend
%%
%% To include the figure in your LaTeX document, write
%%   \input{<filename>.pgf}
%%
%% Make sure the required packages are loaded in your preamble
%%   \usepackage{pgf}
%%
%% Also ensure that all the required font packages are loaded; for instance,
%% the lmodern package is sometimes necessary when using math font.
%%   \usepackage{lmodern}
%%
%% Figures using additional raster images can only be included by \input if
%% they are in the same directory as the main LaTeX file. For loading figures
%% from other directories you can use the `import` package
%%   \usepackage{import}
%%
%% and then include the figures with
%%   \import{<path to file>}{<filename>.pgf}
%%
%% Matplotlib used the following preamble
%%   
%%   \usepackage{fontspec}
%%   \setmainfont{DejaVuSerif.ttf}[Path=\detokenize{/opt/homebrew/Caskroom/miniconda/base/envs/label-studio/lib/python3.9/site-packages/matplotlib/mpl-data/fonts/ttf/}]
%%   \setsansfont{DejaVuSans.ttf}[Path=\detokenize{/opt/homebrew/Caskroom/miniconda/base/envs/label-studio/lib/python3.9/site-packages/matplotlib/mpl-data/fonts/ttf/}]
%%   \setmonofont{DejaVuSansMono.ttf}[Path=\detokenize{/opt/homebrew/Caskroom/miniconda/base/envs/label-studio/lib/python3.9/site-packages/matplotlib/mpl-data/fonts/ttf/}]
%%   \makeatletter\@ifpackageloaded{underscore}{}{\usepackage[strings]{underscore}}\makeatother
%%
\begingroup%
\makeatletter%
\begin{pgfpicture}%
\pgfpathrectangle{\pgfpointorigin}{\pgfqpoint{7.000000in}{4.000000in}}%
\pgfusepath{use as bounding box, clip}%
\begin{pgfscope}%
\pgfsetbuttcap%
\pgfsetmiterjoin%
\definecolor{currentfill}{rgb}{1.000000,1.000000,1.000000}%
\pgfsetfillcolor{currentfill}%
\pgfsetlinewidth{0.000000pt}%
\definecolor{currentstroke}{rgb}{1.000000,1.000000,1.000000}%
\pgfsetstrokecolor{currentstroke}%
\pgfsetdash{}{0pt}%
\pgfpathmoveto{\pgfqpoint{0.000000in}{0.000000in}}%
\pgfpathlineto{\pgfqpoint{7.000000in}{0.000000in}}%
\pgfpathlineto{\pgfqpoint{7.000000in}{4.000000in}}%
\pgfpathlineto{\pgfqpoint{0.000000in}{4.000000in}}%
\pgfpathlineto{\pgfqpoint{0.000000in}{0.000000in}}%
\pgfpathclose%
\pgfusepath{fill}%
\end{pgfscope}%
\begin{pgfscope}%
\pgfsetbuttcap%
\pgfsetmiterjoin%
\definecolor{currentfill}{rgb}{1.000000,1.000000,1.000000}%
\pgfsetfillcolor{currentfill}%
\pgfsetlinewidth{0.000000pt}%
\definecolor{currentstroke}{rgb}{0.000000,0.000000,0.000000}%
\pgfsetstrokecolor{currentstroke}%
\pgfsetstrokeopacity{0.000000}%
\pgfsetdash{}{0pt}%
\pgfpathmoveto{\pgfqpoint{0.875000in}{0.440000in}}%
\pgfpathlineto{\pgfqpoint{6.300000in}{0.440000in}}%
\pgfpathlineto{\pgfqpoint{6.300000in}{3.520000in}}%
\pgfpathlineto{\pgfqpoint{0.875000in}{3.520000in}}%
\pgfpathlineto{\pgfqpoint{0.875000in}{0.440000in}}%
\pgfpathclose%
\pgfusepath{fill}%
\end{pgfscope}%
\begin{pgfscope}%
\pgfsetbuttcap%
\pgfsetroundjoin%
\definecolor{currentfill}{rgb}{0.000000,0.000000,0.000000}%
\pgfsetfillcolor{currentfill}%
\pgfsetlinewidth{0.803000pt}%
\definecolor{currentstroke}{rgb}{0.000000,0.000000,0.000000}%
\pgfsetstrokecolor{currentstroke}%
\pgfsetdash{}{0pt}%
\pgfsys@defobject{currentmarker}{\pgfqpoint{0.000000in}{-0.048611in}}{\pgfqpoint{0.000000in}{0.000000in}}{%
\pgfpathmoveto{\pgfqpoint{0.000000in}{0.000000in}}%
\pgfpathlineto{\pgfqpoint{0.000000in}{-0.048611in}}%
\pgfusepath{stroke,fill}%
}%
\begin{pgfscope}%
\pgfsys@transformshift{1.121591in}{0.440000in}%
\pgfsys@useobject{currentmarker}{}%
\end{pgfscope}%
\end{pgfscope}%
\begin{pgfscope}%
\definecolor{textcolor}{rgb}{0.000000,0.000000,0.000000}%
\pgfsetstrokecolor{textcolor}%
\pgfsetfillcolor{textcolor}%
\pgftext[x=1.121591in,y=0.342778in,,top]{\color{textcolor}\sffamily\fontsize{10.000000}{12.000000}\selectfont 0.000}%
\end{pgfscope}%
\begin{pgfscope}%
\pgfsetbuttcap%
\pgfsetroundjoin%
\definecolor{currentfill}{rgb}{0.000000,0.000000,0.000000}%
\pgfsetfillcolor{currentfill}%
\pgfsetlinewidth{0.803000pt}%
\definecolor{currentstroke}{rgb}{0.000000,0.000000,0.000000}%
\pgfsetstrokecolor{currentstroke}%
\pgfsetdash{}{0pt}%
\pgfsys@defobject{currentmarker}{\pgfqpoint{0.000000in}{-0.048611in}}{\pgfqpoint{0.000000in}{0.000000in}}{%
\pgfpathmoveto{\pgfqpoint{0.000000in}{0.000000in}}%
\pgfpathlineto{\pgfqpoint{0.000000in}{-0.048611in}}%
\pgfusepath{stroke,fill}%
}%
\begin{pgfscope}%
\pgfsys@transformshift{1.863045in}{0.440000in}%
\pgfsys@useobject{currentmarker}{}%
\end{pgfscope}%
\end{pgfscope}%
\begin{pgfscope}%
\definecolor{textcolor}{rgb}{0.000000,0.000000,0.000000}%
\pgfsetstrokecolor{textcolor}%
\pgfsetfillcolor{textcolor}%
\pgftext[x=1.863045in,y=0.342778in,,top]{\color{textcolor}\sffamily\fontsize{10.000000}{12.000000}\selectfont 0.002}%
\end{pgfscope}%
\begin{pgfscope}%
\pgfsetbuttcap%
\pgfsetroundjoin%
\definecolor{currentfill}{rgb}{0.000000,0.000000,0.000000}%
\pgfsetfillcolor{currentfill}%
\pgfsetlinewidth{0.803000pt}%
\definecolor{currentstroke}{rgb}{0.000000,0.000000,0.000000}%
\pgfsetstrokecolor{currentstroke}%
\pgfsetdash{}{0pt}%
\pgfsys@defobject{currentmarker}{\pgfqpoint{0.000000in}{-0.048611in}}{\pgfqpoint{0.000000in}{0.000000in}}{%
\pgfpathmoveto{\pgfqpoint{0.000000in}{0.000000in}}%
\pgfpathlineto{\pgfqpoint{0.000000in}{-0.048611in}}%
\pgfusepath{stroke,fill}%
}%
\begin{pgfscope}%
\pgfsys@transformshift{2.604499in}{0.440000in}%
\pgfsys@useobject{currentmarker}{}%
\end{pgfscope}%
\end{pgfscope}%
\begin{pgfscope}%
\definecolor{textcolor}{rgb}{0.000000,0.000000,0.000000}%
\pgfsetstrokecolor{textcolor}%
\pgfsetfillcolor{textcolor}%
\pgftext[x=2.604499in,y=0.342778in,,top]{\color{textcolor}\sffamily\fontsize{10.000000}{12.000000}\selectfont 0.003}%
\end{pgfscope}%
\begin{pgfscope}%
\pgfsetbuttcap%
\pgfsetroundjoin%
\definecolor{currentfill}{rgb}{0.000000,0.000000,0.000000}%
\pgfsetfillcolor{currentfill}%
\pgfsetlinewidth{0.803000pt}%
\definecolor{currentstroke}{rgb}{0.000000,0.000000,0.000000}%
\pgfsetstrokecolor{currentstroke}%
\pgfsetdash{}{0pt}%
\pgfsys@defobject{currentmarker}{\pgfqpoint{0.000000in}{-0.048611in}}{\pgfqpoint{0.000000in}{0.000000in}}{%
\pgfpathmoveto{\pgfqpoint{0.000000in}{0.000000in}}%
\pgfpathlineto{\pgfqpoint{0.000000in}{-0.048611in}}%
\pgfusepath{stroke,fill}%
}%
\begin{pgfscope}%
\pgfsys@transformshift{3.345953in}{0.440000in}%
\pgfsys@useobject{currentmarker}{}%
\end{pgfscope}%
\end{pgfscope}%
\begin{pgfscope}%
\definecolor{textcolor}{rgb}{0.000000,0.000000,0.000000}%
\pgfsetstrokecolor{textcolor}%
\pgfsetfillcolor{textcolor}%
\pgftext[x=3.345953in,y=0.342778in,,top]{\color{textcolor}\sffamily\fontsize{10.000000}{12.000000}\selectfont 0.005}%
\end{pgfscope}%
\begin{pgfscope}%
\pgfsetbuttcap%
\pgfsetroundjoin%
\definecolor{currentfill}{rgb}{0.000000,0.000000,0.000000}%
\pgfsetfillcolor{currentfill}%
\pgfsetlinewidth{0.803000pt}%
\definecolor{currentstroke}{rgb}{0.000000,0.000000,0.000000}%
\pgfsetstrokecolor{currentstroke}%
\pgfsetdash{}{0pt}%
\pgfsys@defobject{currentmarker}{\pgfqpoint{0.000000in}{-0.048611in}}{\pgfqpoint{0.000000in}{0.000000in}}{%
\pgfpathmoveto{\pgfqpoint{0.000000in}{0.000000in}}%
\pgfpathlineto{\pgfqpoint{0.000000in}{-0.048611in}}%
\pgfusepath{stroke,fill}%
}%
\begin{pgfscope}%
\pgfsys@transformshift{4.087407in}{0.440000in}%
\pgfsys@useobject{currentmarker}{}%
\end{pgfscope}%
\end{pgfscope}%
\begin{pgfscope}%
\definecolor{textcolor}{rgb}{0.000000,0.000000,0.000000}%
\pgfsetstrokecolor{textcolor}%
\pgfsetfillcolor{textcolor}%
\pgftext[x=4.087407in,y=0.342778in,,top]{\color{textcolor}\sffamily\fontsize{10.000000}{12.000000}\selectfont 0.006}%
\end{pgfscope}%
\begin{pgfscope}%
\pgfsetbuttcap%
\pgfsetroundjoin%
\definecolor{currentfill}{rgb}{0.000000,0.000000,0.000000}%
\pgfsetfillcolor{currentfill}%
\pgfsetlinewidth{0.803000pt}%
\definecolor{currentstroke}{rgb}{0.000000,0.000000,0.000000}%
\pgfsetstrokecolor{currentstroke}%
\pgfsetdash{}{0pt}%
\pgfsys@defobject{currentmarker}{\pgfqpoint{0.000000in}{-0.048611in}}{\pgfqpoint{0.000000in}{0.000000in}}{%
\pgfpathmoveto{\pgfqpoint{0.000000in}{0.000000in}}%
\pgfpathlineto{\pgfqpoint{0.000000in}{-0.048611in}}%
\pgfusepath{stroke,fill}%
}%
\begin{pgfscope}%
\pgfsys@transformshift{4.828861in}{0.440000in}%
\pgfsys@useobject{currentmarker}{}%
\end{pgfscope}%
\end{pgfscope}%
\begin{pgfscope}%
\definecolor{textcolor}{rgb}{0.000000,0.000000,0.000000}%
\pgfsetstrokecolor{textcolor}%
\pgfsetfillcolor{textcolor}%
\pgftext[x=4.828861in,y=0.342778in,,top]{\color{textcolor}\sffamily\fontsize{10.000000}{12.000000}\selectfont 0.008}%
\end{pgfscope}%
\begin{pgfscope}%
\pgfsetbuttcap%
\pgfsetroundjoin%
\definecolor{currentfill}{rgb}{0.000000,0.000000,0.000000}%
\pgfsetfillcolor{currentfill}%
\pgfsetlinewidth{0.803000pt}%
\definecolor{currentstroke}{rgb}{0.000000,0.000000,0.000000}%
\pgfsetstrokecolor{currentstroke}%
\pgfsetdash{}{0pt}%
\pgfsys@defobject{currentmarker}{\pgfqpoint{0.000000in}{-0.048611in}}{\pgfqpoint{0.000000in}{0.000000in}}{%
\pgfpathmoveto{\pgfqpoint{0.000000in}{0.000000in}}%
\pgfpathlineto{\pgfqpoint{0.000000in}{-0.048611in}}%
\pgfusepath{stroke,fill}%
}%
\begin{pgfscope}%
\pgfsys@transformshift{5.570315in}{0.440000in}%
\pgfsys@useobject{currentmarker}{}%
\end{pgfscope}%
\end{pgfscope}%
\begin{pgfscope}%
\definecolor{textcolor}{rgb}{0.000000,0.000000,0.000000}%
\pgfsetstrokecolor{textcolor}%
\pgfsetfillcolor{textcolor}%
\pgftext[x=5.570315in,y=0.342778in,,top]{\color{textcolor}\sffamily\fontsize{10.000000}{12.000000}\selectfont 0.009}%
\end{pgfscope}%
\begin{pgfscope}%
\definecolor{textcolor}{rgb}{0.000000,0.000000,0.000000}%
\pgfsetstrokecolor{textcolor}%
\pgfsetfillcolor{textcolor}%
\pgftext[x=3.587500in,y=0.152809in,,top]{\color{textcolor}\sffamily\fontsize{10.000000}{12.000000}\selectfont Time}%
\end{pgfscope}%
\begin{pgfscope}%
\pgfsetbuttcap%
\pgfsetroundjoin%
\definecolor{currentfill}{rgb}{0.000000,0.000000,0.000000}%
\pgfsetfillcolor{currentfill}%
\pgfsetlinewidth{0.803000pt}%
\definecolor{currentstroke}{rgb}{0.000000,0.000000,0.000000}%
\pgfsetstrokecolor{currentstroke}%
\pgfsetdash{}{0pt}%
\pgfsys@defobject{currentmarker}{\pgfqpoint{-0.048611in}{0.000000in}}{\pgfqpoint{-0.000000in}{0.000000in}}{%
\pgfpathmoveto{\pgfqpoint{-0.000000in}{0.000000in}}%
\pgfpathlineto{\pgfqpoint{-0.048611in}{0.000000in}}%
\pgfusepath{stroke,fill}%
}%
\begin{pgfscope}%
\pgfsys@transformshift{0.875000in}{0.470530in}%
\pgfsys@useobject{currentmarker}{}%
\end{pgfscope}%
\end{pgfscope}%
\begin{pgfscope}%
\definecolor{textcolor}{rgb}{0.000000,0.000000,0.000000}%
\pgfsetstrokecolor{textcolor}%
\pgfsetfillcolor{textcolor}%
\pgftext[x=0.360508in, y=0.417768in, left, base]{\color{textcolor}\sffamily\fontsize{10.000000}{12.000000}\selectfont \ensuremath{-}0.15}%
\end{pgfscope}%
\begin{pgfscope}%
\pgfsetbuttcap%
\pgfsetroundjoin%
\definecolor{currentfill}{rgb}{0.000000,0.000000,0.000000}%
\pgfsetfillcolor{currentfill}%
\pgfsetlinewidth{0.803000pt}%
\definecolor{currentstroke}{rgb}{0.000000,0.000000,0.000000}%
\pgfsetstrokecolor{currentstroke}%
\pgfsetdash{}{0pt}%
\pgfsys@defobject{currentmarker}{\pgfqpoint{-0.048611in}{0.000000in}}{\pgfqpoint{-0.000000in}{0.000000in}}{%
\pgfpathmoveto{\pgfqpoint{-0.000000in}{0.000000in}}%
\pgfpathlineto{\pgfqpoint{-0.048611in}{0.000000in}}%
\pgfusepath{stroke,fill}%
}%
\begin{pgfscope}%
\pgfsys@transformshift{0.875000in}{0.973686in}%
\pgfsys@useobject{currentmarker}{}%
\end{pgfscope}%
\end{pgfscope}%
\begin{pgfscope}%
\definecolor{textcolor}{rgb}{0.000000,0.000000,0.000000}%
\pgfsetstrokecolor{textcolor}%
\pgfsetfillcolor{textcolor}%
\pgftext[x=0.360508in, y=0.920925in, left, base]{\color{textcolor}\sffamily\fontsize{10.000000}{12.000000}\selectfont \ensuremath{-}0.10}%
\end{pgfscope}%
\begin{pgfscope}%
\pgfsetbuttcap%
\pgfsetroundjoin%
\definecolor{currentfill}{rgb}{0.000000,0.000000,0.000000}%
\pgfsetfillcolor{currentfill}%
\pgfsetlinewidth{0.803000pt}%
\definecolor{currentstroke}{rgb}{0.000000,0.000000,0.000000}%
\pgfsetstrokecolor{currentstroke}%
\pgfsetdash{}{0pt}%
\pgfsys@defobject{currentmarker}{\pgfqpoint{-0.048611in}{0.000000in}}{\pgfqpoint{-0.000000in}{0.000000in}}{%
\pgfpathmoveto{\pgfqpoint{-0.000000in}{0.000000in}}%
\pgfpathlineto{\pgfqpoint{-0.048611in}{0.000000in}}%
\pgfusepath{stroke,fill}%
}%
\begin{pgfscope}%
\pgfsys@transformshift{0.875000in}{1.476843in}%
\pgfsys@useobject{currentmarker}{}%
\end{pgfscope}%
\end{pgfscope}%
\begin{pgfscope}%
\definecolor{textcolor}{rgb}{0.000000,0.000000,0.000000}%
\pgfsetstrokecolor{textcolor}%
\pgfsetfillcolor{textcolor}%
\pgftext[x=0.360508in, y=1.424082in, left, base]{\color{textcolor}\sffamily\fontsize{10.000000}{12.000000}\selectfont \ensuremath{-}0.05}%
\end{pgfscope}%
\begin{pgfscope}%
\pgfsetbuttcap%
\pgfsetroundjoin%
\definecolor{currentfill}{rgb}{0.000000,0.000000,0.000000}%
\pgfsetfillcolor{currentfill}%
\pgfsetlinewidth{0.803000pt}%
\definecolor{currentstroke}{rgb}{0.000000,0.000000,0.000000}%
\pgfsetstrokecolor{currentstroke}%
\pgfsetdash{}{0pt}%
\pgfsys@defobject{currentmarker}{\pgfqpoint{-0.048611in}{0.000000in}}{\pgfqpoint{-0.000000in}{0.000000in}}{%
\pgfpathmoveto{\pgfqpoint{-0.000000in}{0.000000in}}%
\pgfpathlineto{\pgfqpoint{-0.048611in}{0.000000in}}%
\pgfusepath{stroke,fill}%
}%
\begin{pgfscope}%
\pgfsys@transformshift{0.875000in}{1.980000in}%
\pgfsys@useobject{currentmarker}{}%
\end{pgfscope}%
\end{pgfscope}%
\begin{pgfscope}%
\definecolor{textcolor}{rgb}{0.000000,0.000000,0.000000}%
\pgfsetstrokecolor{textcolor}%
\pgfsetfillcolor{textcolor}%
\pgftext[x=0.468533in, y=1.927238in, left, base]{\color{textcolor}\sffamily\fontsize{10.000000}{12.000000}\selectfont 0.00}%
\end{pgfscope}%
\begin{pgfscope}%
\pgfsetbuttcap%
\pgfsetroundjoin%
\definecolor{currentfill}{rgb}{0.000000,0.000000,0.000000}%
\pgfsetfillcolor{currentfill}%
\pgfsetlinewidth{0.803000pt}%
\definecolor{currentstroke}{rgb}{0.000000,0.000000,0.000000}%
\pgfsetstrokecolor{currentstroke}%
\pgfsetdash{}{0pt}%
\pgfsys@defobject{currentmarker}{\pgfqpoint{-0.048611in}{0.000000in}}{\pgfqpoint{-0.000000in}{0.000000in}}{%
\pgfpathmoveto{\pgfqpoint{-0.000000in}{0.000000in}}%
\pgfpathlineto{\pgfqpoint{-0.048611in}{0.000000in}}%
\pgfusepath{stroke,fill}%
}%
\begin{pgfscope}%
\pgfsys@transformshift{0.875000in}{2.483157in}%
\pgfsys@useobject{currentmarker}{}%
\end{pgfscope}%
\end{pgfscope}%
\begin{pgfscope}%
\definecolor{textcolor}{rgb}{0.000000,0.000000,0.000000}%
\pgfsetstrokecolor{textcolor}%
\pgfsetfillcolor{textcolor}%
\pgftext[x=0.468533in, y=2.430395in, left, base]{\color{textcolor}\sffamily\fontsize{10.000000}{12.000000}\selectfont 0.05}%
\end{pgfscope}%
\begin{pgfscope}%
\pgfsetbuttcap%
\pgfsetroundjoin%
\definecolor{currentfill}{rgb}{0.000000,0.000000,0.000000}%
\pgfsetfillcolor{currentfill}%
\pgfsetlinewidth{0.803000pt}%
\definecolor{currentstroke}{rgb}{0.000000,0.000000,0.000000}%
\pgfsetstrokecolor{currentstroke}%
\pgfsetdash{}{0pt}%
\pgfsys@defobject{currentmarker}{\pgfqpoint{-0.048611in}{0.000000in}}{\pgfqpoint{-0.000000in}{0.000000in}}{%
\pgfpathmoveto{\pgfqpoint{-0.000000in}{0.000000in}}%
\pgfpathlineto{\pgfqpoint{-0.048611in}{0.000000in}}%
\pgfusepath{stroke,fill}%
}%
\begin{pgfscope}%
\pgfsys@transformshift{0.875000in}{2.986314in}%
\pgfsys@useobject{currentmarker}{}%
\end{pgfscope}%
\end{pgfscope}%
\begin{pgfscope}%
\definecolor{textcolor}{rgb}{0.000000,0.000000,0.000000}%
\pgfsetstrokecolor{textcolor}%
\pgfsetfillcolor{textcolor}%
\pgftext[x=0.468533in, y=2.933552in, left, base]{\color{textcolor}\sffamily\fontsize{10.000000}{12.000000}\selectfont 0.10}%
\end{pgfscope}%
\begin{pgfscope}%
\pgfsetbuttcap%
\pgfsetroundjoin%
\definecolor{currentfill}{rgb}{0.000000,0.000000,0.000000}%
\pgfsetfillcolor{currentfill}%
\pgfsetlinewidth{0.803000pt}%
\definecolor{currentstroke}{rgb}{0.000000,0.000000,0.000000}%
\pgfsetstrokecolor{currentstroke}%
\pgfsetdash{}{0pt}%
\pgfsys@defobject{currentmarker}{\pgfqpoint{-0.048611in}{0.000000in}}{\pgfqpoint{-0.000000in}{0.000000in}}{%
\pgfpathmoveto{\pgfqpoint{-0.000000in}{0.000000in}}%
\pgfpathlineto{\pgfqpoint{-0.048611in}{0.000000in}}%
\pgfusepath{stroke,fill}%
}%
\begin{pgfscope}%
\pgfsys@transformshift{0.875000in}{3.489470in}%
\pgfsys@useobject{currentmarker}{}%
\end{pgfscope}%
\end{pgfscope}%
\begin{pgfscope}%
\definecolor{textcolor}{rgb}{0.000000,0.000000,0.000000}%
\pgfsetstrokecolor{textcolor}%
\pgfsetfillcolor{textcolor}%
\pgftext[x=0.468533in, y=3.436709in, left, base]{\color{textcolor}\sffamily\fontsize{10.000000}{12.000000}\selectfont 0.15}%
\end{pgfscope}%
\begin{pgfscope}%
\pgfpathrectangle{\pgfqpoint{0.875000in}{0.440000in}}{\pgfqpoint{5.425000in}{3.080000in}}%
\pgfusepath{clip}%
\pgfsetrectcap%
\pgfsetroundjoin%
\pgfsetlinewidth{1.505625pt}%
\definecolor{currentstroke}{rgb}{0.000000,0.000000,0.000000}%
\pgfsetstrokecolor{currentstroke}%
\pgfsetstrokeopacity{0.200000}%
\pgfsetdash{}{0pt}%
\pgfpathmoveto{\pgfqpoint{1.121591in}{2.058022in}}%
\pgfpathlineto{\pgfqpoint{1.144008in}{2.058022in}}%
\pgfpathlineto{\pgfqpoint{1.144008in}{2.084450in}}%
\pgfpathlineto{\pgfqpoint{1.166426in}{2.084450in}}%
\pgfpathlineto{\pgfqpoint{1.166426in}{2.234479in}}%
\pgfpathlineto{\pgfqpoint{1.188843in}{2.234479in}}%
\pgfpathlineto{\pgfqpoint{1.188843in}{2.334473in}}%
\pgfpathlineto{\pgfqpoint{1.211260in}{2.334473in}}%
\pgfpathlineto{\pgfqpoint{1.211260in}{2.214030in}}%
\pgfpathlineto{\pgfqpoint{1.233678in}{2.214030in}}%
\pgfpathlineto{\pgfqpoint{1.233678in}{2.289679in}}%
\pgfpathlineto{\pgfqpoint{1.256095in}{2.289679in}}%
\pgfpathlineto{\pgfqpoint{1.256095in}{2.292441in}}%
\pgfpathlineto{\pgfqpoint{1.278512in}{2.292441in}}%
\pgfpathlineto{\pgfqpoint{1.278512in}{2.518721in}}%
\pgfpathlineto{\pgfqpoint{1.300930in}{2.518721in}}%
\pgfpathlineto{\pgfqpoint{1.300930in}{2.693096in}}%
\pgfpathlineto{\pgfqpoint{1.323347in}{2.693096in}}%
\pgfpathlineto{\pgfqpoint{1.323347in}{2.539911in}}%
\pgfpathlineto{\pgfqpoint{1.345764in}{2.539911in}}%
\pgfpathlineto{\pgfqpoint{1.345764in}{2.258216in}}%
\pgfpathlineto{\pgfqpoint{1.368182in}{2.258216in}}%
\pgfpathlineto{\pgfqpoint{1.368182in}{2.022593in}}%
\pgfpathlineto{\pgfqpoint{1.390599in}{2.022593in}}%
\pgfpathlineto{\pgfqpoint{1.390599in}{1.691430in}}%
\pgfpathlineto{\pgfqpoint{1.413017in}{1.691430in}}%
\pgfpathlineto{\pgfqpoint{1.413017in}{1.387736in}}%
\pgfpathlineto{\pgfqpoint{1.435434in}{1.387736in}}%
\pgfpathlineto{\pgfqpoint{1.435434in}{1.364084in}}%
\pgfpathlineto{\pgfqpoint{1.457851in}{1.364084in}}%
\pgfpathlineto{\pgfqpoint{1.457851in}{1.374447in}}%
\pgfpathlineto{\pgfqpoint{1.480269in}{1.374447in}}%
\pgfpathlineto{\pgfqpoint{1.480269in}{1.668828in}}%
\pgfpathlineto{\pgfqpoint{1.502686in}{1.668828in}}%
\pgfpathlineto{\pgfqpoint{1.502686in}{2.113240in}}%
\pgfpathlineto{\pgfqpoint{1.525103in}{2.113240in}}%
\pgfpathlineto{\pgfqpoint{1.525103in}{2.490036in}}%
\pgfpathlineto{\pgfqpoint{1.547521in}{2.490036in}}%
\pgfpathlineto{\pgfqpoint{1.547521in}{2.372557in}}%
\pgfpathlineto{\pgfqpoint{1.569938in}{2.372557in}}%
\pgfpathlineto{\pgfqpoint{1.569938in}{1.858670in}}%
\pgfpathlineto{\pgfqpoint{1.592355in}{1.858670in}}%
\pgfpathlineto{\pgfqpoint{1.592355in}{1.525524in}}%
\pgfpathlineto{\pgfqpoint{1.614773in}{1.525524in}}%
\pgfpathlineto{\pgfqpoint{1.614773in}{1.377401in}}%
\pgfpathlineto{\pgfqpoint{1.637190in}{1.377401in}}%
\pgfpathlineto{\pgfqpoint{1.637190in}{1.426161in}}%
\pgfpathlineto{\pgfqpoint{1.659607in}{1.426161in}}%
\pgfpathlineto{\pgfqpoint{1.659607in}{1.584538in}}%
\pgfpathlineto{\pgfqpoint{1.682025in}{1.584538in}}%
\pgfpathlineto{\pgfqpoint{1.682025in}{1.858629in}}%
\pgfpathlineto{\pgfqpoint{1.704442in}{1.858629in}}%
\pgfpathlineto{\pgfqpoint{1.704442in}{2.213921in}}%
\pgfpathlineto{\pgfqpoint{1.726860in}{2.213921in}}%
\pgfpathlineto{\pgfqpoint{1.726860in}{2.313546in}}%
\pgfpathlineto{\pgfqpoint{1.749277in}{2.313546in}}%
\pgfpathlineto{\pgfqpoint{1.749277in}{2.251492in}}%
\pgfpathlineto{\pgfqpoint{1.771694in}{2.251492in}}%
\pgfpathlineto{\pgfqpoint{1.771694in}{2.106918in}}%
\pgfpathlineto{\pgfqpoint{1.794112in}{2.106918in}}%
\pgfpathlineto{\pgfqpoint{1.794112in}{1.940703in}}%
\pgfpathlineto{\pgfqpoint{1.816529in}{1.940703in}}%
\pgfpathlineto{\pgfqpoint{1.816529in}{1.830157in}}%
\pgfpathlineto{\pgfqpoint{1.838946in}{1.830157in}}%
\pgfpathlineto{\pgfqpoint{1.838946in}{1.593971in}}%
\pgfpathlineto{\pgfqpoint{1.861364in}{1.593971in}}%
\pgfpathlineto{\pgfqpoint{1.861364in}{1.495126in}}%
\pgfpathlineto{\pgfqpoint{1.883781in}{1.495126in}}%
\pgfpathlineto{\pgfqpoint{1.883781in}{1.608910in}}%
\pgfpathlineto{\pgfqpoint{1.906198in}{1.608910in}}%
\pgfpathlineto{\pgfqpoint{1.906198in}{1.911871in}}%
\pgfpathlineto{\pgfqpoint{1.928616in}{1.911871in}}%
\pgfpathlineto{\pgfqpoint{1.928616in}{2.157539in}}%
\pgfpathlineto{\pgfqpoint{1.951033in}{2.157539in}}%
\pgfpathlineto{\pgfqpoint{1.951033in}{2.333096in}}%
\pgfpathlineto{\pgfqpoint{1.973450in}{2.333096in}}%
\pgfpathlineto{\pgfqpoint{1.973450in}{2.553085in}}%
\pgfpathlineto{\pgfqpoint{1.995868in}{2.553085in}}%
\pgfpathlineto{\pgfqpoint{1.995868in}{2.609304in}}%
\pgfpathlineto{\pgfqpoint{2.018285in}{2.609304in}}%
\pgfpathlineto{\pgfqpoint{2.018285in}{2.536976in}}%
\pgfpathlineto{\pgfqpoint{2.040702in}{2.536976in}}%
\pgfpathlineto{\pgfqpoint{2.040702in}{2.304702in}}%
\pgfpathlineto{\pgfqpoint{2.063120in}{2.304702in}}%
\pgfpathlineto{\pgfqpoint{2.063120in}{2.156759in}}%
\pgfpathlineto{\pgfqpoint{2.085537in}{2.156759in}}%
\pgfpathlineto{\pgfqpoint{2.085537in}{2.369195in}}%
\pgfpathlineto{\pgfqpoint{2.107955in}{2.369195in}}%
\pgfpathlineto{\pgfqpoint{2.107955in}{2.697345in}}%
\pgfpathlineto{\pgfqpoint{2.130372in}{2.697345in}}%
\pgfpathlineto{\pgfqpoint{2.130372in}{2.913581in}}%
\pgfpathlineto{\pgfqpoint{2.152789in}{2.913581in}}%
\pgfpathlineto{\pgfqpoint{2.152789in}{3.018638in}}%
\pgfpathlineto{\pgfqpoint{2.175207in}{3.018638in}}%
\pgfpathlineto{\pgfqpoint{2.175207in}{2.811905in}}%
\pgfpathlineto{\pgfqpoint{2.197624in}{2.811905in}}%
\pgfpathlineto{\pgfqpoint{2.197624in}{2.537266in}}%
\pgfpathlineto{\pgfqpoint{2.220041in}{2.537266in}}%
\pgfpathlineto{\pgfqpoint{2.220041in}{2.298059in}}%
\pgfpathlineto{\pgfqpoint{2.242459in}{2.298059in}}%
\pgfpathlineto{\pgfqpoint{2.242459in}{2.226243in}}%
\pgfpathlineto{\pgfqpoint{2.264876in}{2.226243in}}%
\pgfpathlineto{\pgfqpoint{2.264876in}{2.167852in}}%
\pgfpathlineto{\pgfqpoint{2.287293in}{2.167852in}}%
\pgfpathlineto{\pgfqpoint{2.287293in}{1.910965in}}%
\pgfpathlineto{\pgfqpoint{2.309711in}{1.910965in}}%
\pgfpathlineto{\pgfqpoint{2.309711in}{1.586608in}}%
\pgfpathlineto{\pgfqpoint{2.332128in}{1.586608in}}%
\pgfpathlineto{\pgfqpoint{2.332128in}{1.226944in}}%
\pgfpathlineto{\pgfqpoint{2.354545in}{1.226944in}}%
\pgfpathlineto{\pgfqpoint{2.354545in}{1.342153in}}%
\pgfpathlineto{\pgfqpoint{2.376963in}{1.342153in}}%
\pgfpathlineto{\pgfqpoint{2.376963in}{1.463454in}}%
\pgfpathlineto{\pgfqpoint{2.399380in}{1.463454in}}%
\pgfpathlineto{\pgfqpoint{2.399380in}{1.551943in}}%
\pgfpathlineto{\pgfqpoint{2.421798in}{1.551943in}}%
\pgfpathlineto{\pgfqpoint{2.421798in}{1.764077in}}%
\pgfpathlineto{\pgfqpoint{2.444215in}{1.764077in}}%
\pgfpathlineto{\pgfqpoint{2.444215in}{1.690474in}}%
\pgfpathlineto{\pgfqpoint{2.466632in}{1.690474in}}%
\pgfpathlineto{\pgfqpoint{2.466632in}{1.443981in}}%
\pgfpathlineto{\pgfqpoint{2.489050in}{1.443981in}}%
\pgfpathlineto{\pgfqpoint{2.489050in}{1.438443in}}%
\pgfpathlineto{\pgfqpoint{2.511467in}{1.438443in}}%
\pgfpathlineto{\pgfqpoint{2.511467in}{1.436858in}}%
\pgfpathlineto{\pgfqpoint{2.533884in}{1.436858in}}%
\pgfpathlineto{\pgfqpoint{2.533884in}{1.255702in}}%
\pgfpathlineto{\pgfqpoint{2.556302in}{1.255702in}}%
\pgfpathlineto{\pgfqpoint{2.556302in}{1.369052in}}%
\pgfpathlineto{\pgfqpoint{2.578719in}{1.369052in}}%
\pgfpathlineto{\pgfqpoint{2.578719in}{1.661540in}}%
\pgfpathlineto{\pgfqpoint{2.601136in}{1.661540in}}%
\pgfpathlineto{\pgfqpoint{2.601136in}{1.994127in}}%
\pgfpathlineto{\pgfqpoint{2.623554in}{1.994127in}}%
\pgfpathlineto{\pgfqpoint{2.623554in}{2.048489in}}%
\pgfpathlineto{\pgfqpoint{2.645971in}{2.048489in}}%
\pgfpathlineto{\pgfqpoint{2.645971in}{2.062718in}}%
\pgfpathlineto{\pgfqpoint{2.668388in}{2.062718in}}%
\pgfpathlineto{\pgfqpoint{2.668388in}{2.012058in}}%
\pgfpathlineto{\pgfqpoint{2.690806in}{2.012058in}}%
\pgfpathlineto{\pgfqpoint{2.690806in}{1.721260in}}%
\pgfpathlineto{\pgfqpoint{2.713223in}{1.721260in}}%
\pgfpathlineto{\pgfqpoint{2.713223in}{1.593578in}}%
\pgfpathlineto{\pgfqpoint{2.735640in}{1.593578in}}%
\pgfpathlineto{\pgfqpoint{2.735640in}{1.519563in}}%
\pgfpathlineto{\pgfqpoint{2.758058in}{1.519563in}}%
\pgfpathlineto{\pgfqpoint{2.758058in}{1.446827in}}%
\pgfpathlineto{\pgfqpoint{2.780475in}{1.446827in}}%
\pgfpathlineto{\pgfqpoint{2.780475in}{1.471692in}}%
\pgfpathlineto{\pgfqpoint{2.802893in}{1.471692in}}%
\pgfpathlineto{\pgfqpoint{2.802893in}{1.518703in}}%
\pgfpathlineto{\pgfqpoint{2.825310in}{1.518703in}}%
\pgfpathlineto{\pgfqpoint{2.825310in}{1.595729in}}%
\pgfpathlineto{\pgfqpoint{2.847727in}{1.595729in}}%
\pgfpathlineto{\pgfqpoint{2.847727in}{1.624964in}}%
\pgfpathlineto{\pgfqpoint{2.870145in}{1.624964in}}%
\pgfpathlineto{\pgfqpoint{2.870145in}{1.503693in}}%
\pgfpathlineto{\pgfqpoint{2.892562in}{1.503693in}}%
\pgfpathlineto{\pgfqpoint{2.892562in}{1.537683in}}%
\pgfpathlineto{\pgfqpoint{2.914979in}{1.537683in}}%
\pgfpathlineto{\pgfqpoint{2.914979in}{1.657519in}}%
\pgfpathlineto{\pgfqpoint{2.937397in}{1.657519in}}%
\pgfpathlineto{\pgfqpoint{2.937397in}{1.770854in}}%
\pgfpathlineto{\pgfqpoint{2.959814in}{1.770854in}}%
\pgfpathlineto{\pgfqpoint{2.959814in}{1.657959in}}%
\pgfpathlineto{\pgfqpoint{2.982231in}{1.657959in}}%
\pgfpathlineto{\pgfqpoint{2.982231in}{1.476293in}}%
\pgfpathlineto{\pgfqpoint{3.004649in}{1.476293in}}%
\pgfpathlineto{\pgfqpoint{3.004649in}{1.622567in}}%
\pgfpathlineto{\pgfqpoint{3.027066in}{1.622567in}}%
\pgfpathlineto{\pgfqpoint{3.027066in}{1.858093in}}%
\pgfpathlineto{\pgfqpoint{3.049483in}{1.858093in}}%
\pgfpathlineto{\pgfqpoint{3.049483in}{2.070187in}}%
\pgfpathlineto{\pgfqpoint{3.071901in}{2.070187in}}%
\pgfpathlineto{\pgfqpoint{3.071901in}{2.110904in}}%
\pgfpathlineto{\pgfqpoint{3.094318in}{2.110904in}}%
\pgfpathlineto{\pgfqpoint{3.094318in}{2.079712in}}%
\pgfpathlineto{\pgfqpoint{3.116736in}{2.079712in}}%
\pgfpathlineto{\pgfqpoint{3.116736in}{2.023613in}}%
\pgfpathlineto{\pgfqpoint{3.139153in}{2.023613in}}%
\pgfpathlineto{\pgfqpoint{3.139153in}{2.061392in}}%
\pgfpathlineto{\pgfqpoint{3.161570in}{2.061392in}}%
\pgfpathlineto{\pgfqpoint{3.161570in}{2.197707in}}%
\pgfpathlineto{\pgfqpoint{3.183988in}{2.197707in}}%
\pgfpathlineto{\pgfqpoint{3.183988in}{2.345383in}}%
\pgfpathlineto{\pgfqpoint{3.206405in}{2.345383in}}%
\pgfpathlineto{\pgfqpoint{3.206405in}{2.655528in}}%
\pgfpathlineto{\pgfqpoint{3.228822in}{2.655528in}}%
\pgfpathlineto{\pgfqpoint{3.228822in}{2.939749in}}%
\pgfpathlineto{\pgfqpoint{3.251240in}{2.939749in}}%
\pgfpathlineto{\pgfqpoint{3.251240in}{2.995062in}}%
\pgfpathlineto{\pgfqpoint{3.273657in}{2.995062in}}%
\pgfpathlineto{\pgfqpoint{3.273657in}{2.850281in}}%
\pgfpathlineto{\pgfqpoint{3.296074in}{2.850281in}}%
\pgfpathlineto{\pgfqpoint{3.296074in}{2.726310in}}%
\pgfpathlineto{\pgfqpoint{3.318492in}{2.726310in}}%
\pgfpathlineto{\pgfqpoint{3.318492in}{2.707283in}}%
\pgfpathlineto{\pgfqpoint{3.340909in}{2.707283in}}%
\pgfpathlineto{\pgfqpoint{3.340909in}{2.667446in}}%
\pgfpathlineto{\pgfqpoint{3.363326in}{2.667446in}}%
\pgfpathlineto{\pgfqpoint{3.363326in}{2.794307in}}%
\pgfpathlineto{\pgfqpoint{3.385744in}{2.794307in}}%
\pgfpathlineto{\pgfqpoint{3.385744in}{3.038542in}}%
\pgfpathlineto{\pgfqpoint{3.408161in}{3.038542in}}%
\pgfpathlineto{\pgfqpoint{3.408161in}{2.966235in}}%
\pgfpathlineto{\pgfqpoint{3.430579in}{2.966235in}}%
\pgfpathlineto{\pgfqpoint{3.430579in}{2.749982in}}%
\pgfpathlineto{\pgfqpoint{3.452996in}{2.749982in}}%
\pgfpathlineto{\pgfqpoint{3.452996in}{2.488174in}}%
\pgfpathlineto{\pgfqpoint{3.475413in}{2.488174in}}%
\pgfpathlineto{\pgfqpoint{3.475413in}{2.476029in}}%
\pgfpathlineto{\pgfqpoint{3.497831in}{2.476029in}}%
\pgfpathlineto{\pgfqpoint{3.497831in}{2.671480in}}%
\pgfpathlineto{\pgfqpoint{3.520248in}{2.671480in}}%
\pgfpathlineto{\pgfqpoint{3.520248in}{2.796016in}}%
\pgfpathlineto{\pgfqpoint{3.542665in}{2.796016in}}%
\pgfpathlineto{\pgfqpoint{3.542665in}{3.030074in}}%
\pgfpathlineto{\pgfqpoint{3.565083in}{3.030074in}}%
\pgfpathlineto{\pgfqpoint{3.565083in}{3.036443in}}%
\pgfpathlineto{\pgfqpoint{3.587500in}{3.036443in}}%
\pgfpathlineto{\pgfqpoint{3.587500in}{2.704764in}}%
\pgfpathlineto{\pgfqpoint{3.609917in}{2.704764in}}%
\pgfpathlineto{\pgfqpoint{3.609917in}{2.459241in}}%
\pgfpathlineto{\pgfqpoint{3.632335in}{2.459241in}}%
\pgfpathlineto{\pgfqpoint{3.632335in}{2.328464in}}%
\pgfpathlineto{\pgfqpoint{3.654752in}{2.328464in}}%
\pgfpathlineto{\pgfqpoint{3.654752in}{1.969379in}}%
\pgfpathlineto{\pgfqpoint{3.677169in}{1.969379in}}%
\pgfpathlineto{\pgfqpoint{3.677169in}{1.821972in}}%
\pgfpathlineto{\pgfqpoint{3.699587in}{1.821972in}}%
\pgfpathlineto{\pgfqpoint{3.699587in}{1.828324in}}%
\pgfpathlineto{\pgfqpoint{3.722004in}{1.828324in}}%
\pgfpathlineto{\pgfqpoint{3.722004in}{1.630574in}}%
\pgfpathlineto{\pgfqpoint{3.744421in}{1.630574in}}%
\pgfpathlineto{\pgfqpoint{3.744421in}{1.432419in}}%
\pgfpathlineto{\pgfqpoint{3.766839in}{1.432419in}}%
\pgfpathlineto{\pgfqpoint{3.766839in}{1.176141in}}%
\pgfpathlineto{\pgfqpoint{3.789256in}{1.176141in}}%
\pgfpathlineto{\pgfqpoint{3.789256in}{0.923626in}}%
\pgfpathlineto{\pgfqpoint{3.811674in}{0.923626in}}%
\pgfpathlineto{\pgfqpoint{3.811674in}{0.717661in}}%
\pgfpathlineto{\pgfqpoint{3.834091in}{0.717661in}}%
\pgfpathlineto{\pgfqpoint{3.834091in}{0.580000in}}%
\pgfpathlineto{\pgfqpoint{3.856508in}{0.580000in}}%
\pgfpathlineto{\pgfqpoint{3.856508in}{0.652722in}}%
\pgfpathlineto{\pgfqpoint{3.878926in}{0.652722in}}%
\pgfpathlineto{\pgfqpoint{3.878926in}{0.778918in}}%
\pgfpathlineto{\pgfqpoint{3.901343in}{0.778918in}}%
\pgfpathlineto{\pgfqpoint{3.901343in}{0.884488in}}%
\pgfpathlineto{\pgfqpoint{3.923760in}{0.884488in}}%
\pgfpathlineto{\pgfqpoint{3.923760in}{0.967680in}}%
\pgfpathlineto{\pgfqpoint{3.946178in}{0.967680in}}%
\pgfpathlineto{\pgfqpoint{3.946178in}{0.982952in}}%
\pgfpathlineto{\pgfqpoint{3.968595in}{0.982952in}}%
\pgfpathlineto{\pgfqpoint{3.968595in}{1.243037in}}%
\pgfpathlineto{\pgfqpoint{3.991012in}{1.243037in}}%
\pgfpathlineto{\pgfqpoint{3.991012in}{1.429357in}}%
\pgfpathlineto{\pgfqpoint{4.013430in}{1.429357in}}%
\pgfpathlineto{\pgfqpoint{4.013430in}{1.522711in}}%
\pgfpathlineto{\pgfqpoint{4.035847in}{1.522711in}}%
\pgfpathlineto{\pgfqpoint{4.035847in}{1.596793in}}%
\pgfpathlineto{\pgfqpoint{4.058264in}{1.596793in}}%
\pgfpathlineto{\pgfqpoint{4.058264in}{1.392497in}}%
\pgfpathlineto{\pgfqpoint{4.080682in}{1.392497in}}%
\pgfpathlineto{\pgfqpoint{4.080682in}{1.197288in}}%
\pgfpathlineto{\pgfqpoint{4.103099in}{1.197288in}}%
\pgfpathlineto{\pgfqpoint{4.103099in}{1.186204in}}%
\pgfpathlineto{\pgfqpoint{4.125517in}{1.186204in}}%
\pgfpathlineto{\pgfqpoint{4.125517in}{1.524415in}}%
\pgfpathlineto{\pgfqpoint{4.147934in}{1.524415in}}%
\pgfpathlineto{\pgfqpoint{4.147934in}{1.983388in}}%
\pgfpathlineto{\pgfqpoint{4.170351in}{1.983388in}}%
\pgfpathlineto{\pgfqpoint{4.170351in}{2.188373in}}%
\pgfpathlineto{\pgfqpoint{4.192769in}{2.188373in}}%
\pgfpathlineto{\pgfqpoint{4.192769in}{2.066593in}}%
\pgfpathlineto{\pgfqpoint{4.215186in}{2.066593in}}%
\pgfpathlineto{\pgfqpoint{4.215186in}{1.902309in}}%
\pgfpathlineto{\pgfqpoint{4.237603in}{1.902309in}}%
\pgfpathlineto{\pgfqpoint{4.237603in}{1.724023in}}%
\pgfpathlineto{\pgfqpoint{4.260021in}{1.724023in}}%
\pgfpathlineto{\pgfqpoint{4.260021in}{1.613265in}}%
\pgfpathlineto{\pgfqpoint{4.282438in}{1.613265in}}%
\pgfpathlineto{\pgfqpoint{4.282438in}{1.760736in}}%
\pgfpathlineto{\pgfqpoint{4.304855in}{1.760736in}}%
\pgfpathlineto{\pgfqpoint{4.304855in}{1.894372in}}%
\pgfpathlineto{\pgfqpoint{4.327273in}{1.894372in}}%
\pgfpathlineto{\pgfqpoint{4.327273in}{2.179590in}}%
\pgfpathlineto{\pgfqpoint{4.349690in}{2.179590in}}%
\pgfpathlineto{\pgfqpoint{4.349690in}{2.632555in}}%
\pgfpathlineto{\pgfqpoint{4.372107in}{2.632555in}}%
\pgfpathlineto{\pgfqpoint{4.372107in}{2.843803in}}%
\pgfpathlineto{\pgfqpoint{4.394525in}{2.843803in}}%
\pgfpathlineto{\pgfqpoint{4.394525in}{2.790534in}}%
\pgfpathlineto{\pgfqpoint{4.416942in}{2.790534in}}%
\pgfpathlineto{\pgfqpoint{4.416942in}{2.702782in}}%
\pgfpathlineto{\pgfqpoint{4.439360in}{2.702782in}}%
\pgfpathlineto{\pgfqpoint{4.439360in}{2.609916in}}%
\pgfpathlineto{\pgfqpoint{4.461777in}{2.609916in}}%
\pgfpathlineto{\pgfqpoint{4.461777in}{2.585518in}}%
\pgfpathlineto{\pgfqpoint{4.484194in}{2.585518in}}%
\pgfpathlineto{\pgfqpoint{4.484194in}{2.503995in}}%
\pgfpathlineto{\pgfqpoint{4.506612in}{2.503995in}}%
\pgfpathlineto{\pgfqpoint{4.506612in}{2.370865in}}%
\pgfpathlineto{\pgfqpoint{4.529029in}{2.370865in}}%
\pgfpathlineto{\pgfqpoint{4.529029in}{2.208628in}}%
\pgfpathlineto{\pgfqpoint{4.551446in}{2.208628in}}%
\pgfpathlineto{\pgfqpoint{4.551446in}{2.105463in}}%
\pgfpathlineto{\pgfqpoint{4.573864in}{2.105463in}}%
\pgfpathlineto{\pgfqpoint{4.573864in}{1.995566in}}%
\pgfpathlineto{\pgfqpoint{4.596281in}{1.995566in}}%
\pgfpathlineto{\pgfqpoint{4.596281in}{1.953952in}}%
\pgfpathlineto{\pgfqpoint{4.618698in}{1.953952in}}%
\pgfpathlineto{\pgfqpoint{4.618698in}{2.272984in}}%
\pgfpathlineto{\pgfqpoint{4.641116in}{2.272984in}}%
\pgfpathlineto{\pgfqpoint{4.641116in}{2.490249in}}%
\pgfpathlineto{\pgfqpoint{4.663533in}{2.490249in}}%
\pgfpathlineto{\pgfqpoint{4.663533in}{2.591152in}}%
\pgfpathlineto{\pgfqpoint{4.685950in}{2.591152in}}%
\pgfpathlineto{\pgfqpoint{4.685950in}{2.464620in}}%
\pgfpathlineto{\pgfqpoint{4.708368in}{2.464620in}}%
\pgfpathlineto{\pgfqpoint{4.708368in}{2.047797in}}%
\pgfpathlineto{\pgfqpoint{4.730785in}{2.047797in}}%
\pgfpathlineto{\pgfqpoint{4.730785in}{1.707453in}}%
\pgfpathlineto{\pgfqpoint{4.753202in}{1.707453in}}%
\pgfpathlineto{\pgfqpoint{4.753202in}{1.740183in}}%
\pgfpathlineto{\pgfqpoint{4.775620in}{1.740183in}}%
\pgfpathlineto{\pgfqpoint{4.775620in}{1.931578in}}%
\pgfpathlineto{\pgfqpoint{4.798037in}{1.931578in}}%
\pgfpathlineto{\pgfqpoint{4.798037in}{2.030160in}}%
\pgfpathlineto{\pgfqpoint{4.820455in}{2.030160in}}%
\pgfpathlineto{\pgfqpoint{4.820455in}{2.163025in}}%
\pgfpathlineto{\pgfqpoint{4.842872in}{2.163025in}}%
\pgfpathlineto{\pgfqpoint{4.842872in}{2.201612in}}%
\pgfpathlineto{\pgfqpoint{4.865289in}{2.201612in}}%
\pgfpathlineto{\pgfqpoint{4.865289in}{2.165345in}}%
\pgfpathlineto{\pgfqpoint{4.887707in}{2.165345in}}%
\pgfpathlineto{\pgfqpoint{4.887707in}{2.186260in}}%
\pgfpathlineto{\pgfqpoint{4.910124in}{2.186260in}}%
\pgfpathlineto{\pgfqpoint{4.910124in}{2.288747in}}%
\pgfpathlineto{\pgfqpoint{4.932541in}{2.288747in}}%
\pgfpathlineto{\pgfqpoint{4.932541in}{2.354617in}}%
\pgfpathlineto{\pgfqpoint{4.954959in}{2.354617in}}%
\pgfpathlineto{\pgfqpoint{4.954959in}{2.318984in}}%
\pgfpathlineto{\pgfqpoint{4.977376in}{2.318984in}}%
\pgfpathlineto{\pgfqpoint{4.977376in}{2.120344in}}%
\pgfpathlineto{\pgfqpoint{4.999793in}{2.120344in}}%
\pgfpathlineto{\pgfqpoint{4.999793in}{1.977933in}}%
\pgfpathlineto{\pgfqpoint{5.022211in}{1.977933in}}%
\pgfpathlineto{\pgfqpoint{5.022211in}{1.896426in}}%
\pgfpathlineto{\pgfqpoint{5.044628in}{1.896426in}}%
\pgfpathlineto{\pgfqpoint{5.044628in}{1.852990in}}%
\pgfpathlineto{\pgfqpoint{5.067045in}{1.852990in}}%
\pgfpathlineto{\pgfqpoint{5.067045in}{1.898183in}}%
\pgfpathlineto{\pgfqpoint{5.089463in}{1.898183in}}%
\pgfpathlineto{\pgfqpoint{5.089463in}{2.012342in}}%
\pgfpathlineto{\pgfqpoint{5.111880in}{2.012342in}}%
\pgfpathlineto{\pgfqpoint{5.111880in}{2.149772in}}%
\pgfpathlineto{\pgfqpoint{5.134298in}{2.149772in}}%
\pgfpathlineto{\pgfqpoint{5.134298in}{2.027736in}}%
\pgfpathlineto{\pgfqpoint{5.156715in}{2.027736in}}%
\pgfpathlineto{\pgfqpoint{5.156715in}{1.726498in}}%
\pgfpathlineto{\pgfqpoint{5.179132in}{1.726498in}}%
\pgfpathlineto{\pgfqpoint{5.179132in}{1.331941in}}%
\pgfpathlineto{\pgfqpoint{5.201550in}{1.331941in}}%
\pgfpathlineto{\pgfqpoint{5.201550in}{1.100421in}}%
\pgfpathlineto{\pgfqpoint{5.223967in}{1.100421in}}%
\pgfpathlineto{\pgfqpoint{5.223967in}{1.093527in}}%
\pgfpathlineto{\pgfqpoint{5.246384in}{1.093527in}}%
\pgfpathlineto{\pgfqpoint{5.246384in}{1.259635in}}%
\pgfpathlineto{\pgfqpoint{5.268802in}{1.259635in}}%
\pgfpathlineto{\pgfqpoint{5.268802in}{1.627939in}}%
\pgfpathlineto{\pgfqpoint{5.291219in}{1.627939in}}%
\pgfpathlineto{\pgfqpoint{5.291219in}{1.926206in}}%
\pgfpathlineto{\pgfqpoint{5.313636in}{1.926206in}}%
\pgfpathlineto{\pgfqpoint{5.313636in}{1.990775in}}%
\pgfpathlineto{\pgfqpoint{5.336054in}{1.990775in}}%
\pgfpathlineto{\pgfqpoint{5.336054in}{1.889110in}}%
\pgfpathlineto{\pgfqpoint{5.358471in}{1.889110in}}%
\pgfpathlineto{\pgfqpoint{5.358471in}{1.799930in}}%
\pgfpathlineto{\pgfqpoint{5.380888in}{1.799930in}}%
\pgfpathlineto{\pgfqpoint{5.380888in}{1.711618in}}%
\pgfpathlineto{\pgfqpoint{5.403306in}{1.711618in}}%
\pgfpathlineto{\pgfqpoint{5.403306in}{1.743163in}}%
\pgfpathlineto{\pgfqpoint{5.425723in}{1.743163in}}%
\pgfpathlineto{\pgfqpoint{5.425723in}{1.921925in}}%
\pgfpathlineto{\pgfqpoint{5.448140in}{1.921925in}}%
\pgfpathlineto{\pgfqpoint{5.448140in}{2.149133in}}%
\pgfpathlineto{\pgfqpoint{5.470558in}{2.149133in}}%
\pgfpathlineto{\pgfqpoint{5.470558in}{2.349884in}}%
\pgfpathlineto{\pgfqpoint{5.492975in}{2.349884in}}%
\pgfpathlineto{\pgfqpoint{5.492975in}{2.321403in}}%
\pgfpathlineto{\pgfqpoint{5.515393in}{2.321403in}}%
\pgfpathlineto{\pgfqpoint{5.515393in}{2.293351in}}%
\pgfpathlineto{\pgfqpoint{5.537810in}{2.293351in}}%
\pgfpathlineto{\pgfqpoint{5.537810in}{2.490788in}}%
\pgfpathlineto{\pgfqpoint{5.560227in}{2.490788in}}%
\pgfpathlineto{\pgfqpoint{5.560227in}{2.580157in}}%
\pgfpathlineto{\pgfqpoint{5.582645in}{2.580157in}}%
\pgfpathlineto{\pgfqpoint{5.582645in}{2.623500in}}%
\pgfpathlineto{\pgfqpoint{5.605062in}{2.623500in}}%
\pgfpathlineto{\pgfqpoint{5.605062in}{2.561411in}}%
\pgfpathlineto{\pgfqpoint{5.627479in}{2.561411in}}%
\pgfpathlineto{\pgfqpoint{5.627479in}{2.336427in}}%
\pgfpathlineto{\pgfqpoint{5.649897in}{2.336427in}}%
\pgfpathlineto{\pgfqpoint{5.649897in}{2.006344in}}%
\pgfpathlineto{\pgfqpoint{5.672314in}{2.006344in}}%
\pgfpathlineto{\pgfqpoint{5.672314in}{1.489450in}}%
\pgfpathlineto{\pgfqpoint{5.694731in}{1.489450in}}%
\pgfpathlineto{\pgfqpoint{5.694731in}{1.216123in}}%
\pgfpathlineto{\pgfqpoint{5.717149in}{1.216123in}}%
\pgfpathlineto{\pgfqpoint{5.717149in}{1.183774in}}%
\pgfpathlineto{\pgfqpoint{5.739566in}{1.183774in}}%
\pgfpathlineto{\pgfqpoint{5.739566in}{1.538189in}}%
\pgfpathlineto{\pgfqpoint{5.761983in}{1.538189in}}%
\pgfpathlineto{\pgfqpoint{5.761983in}{1.986205in}}%
\pgfpathlineto{\pgfqpoint{5.784401in}{1.986205in}}%
\pgfpathlineto{\pgfqpoint{5.784401in}{2.315928in}}%
\pgfpathlineto{\pgfqpoint{5.806818in}{2.315928in}}%
\pgfpathlineto{\pgfqpoint{5.806818in}{2.471752in}}%
\pgfpathlineto{\pgfqpoint{5.829236in}{2.471752in}}%
\pgfpathlineto{\pgfqpoint{5.829236in}{2.151395in}}%
\pgfpathlineto{\pgfqpoint{5.851653in}{2.151395in}}%
\pgfpathlineto{\pgfqpoint{5.851653in}{1.860001in}}%
\pgfpathlineto{\pgfqpoint{5.874070in}{1.860001in}}%
\pgfpathlineto{\pgfqpoint{5.874070in}{1.773168in}}%
\pgfpathlineto{\pgfqpoint{5.896488in}{1.773168in}}%
\pgfpathlineto{\pgfqpoint{5.896488in}{1.749116in}}%
\pgfpathlineto{\pgfqpoint{5.918905in}{1.749116in}}%
\pgfpathlineto{\pgfqpoint{5.918905in}{1.597052in}}%
\pgfpathlineto{\pgfqpoint{5.941322in}{1.597052in}}%
\pgfpathlineto{\pgfqpoint{5.941322in}{1.642989in}}%
\pgfpathlineto{\pgfqpoint{5.963740in}{1.642989in}}%
\pgfpathlineto{\pgfqpoint{5.963740in}{1.837673in}}%
\pgfpathlineto{\pgfqpoint{5.986157in}{1.837673in}}%
\pgfpathlineto{\pgfqpoint{5.986157in}{1.840384in}}%
\pgfpathlineto{\pgfqpoint{6.008574in}{1.840384in}}%
\pgfpathlineto{\pgfqpoint{6.008574in}{1.867273in}}%
\pgfpathlineto{\pgfqpoint{6.030992in}{1.867273in}}%
\pgfpathlineto{\pgfqpoint{6.030992in}{1.911252in}}%
\pgfpathlineto{\pgfqpoint{6.053409in}{1.911252in}}%
\pgfpathlineto{\pgfqpoint{6.053409in}{1.980000in}}%
\pgfpathlineto{\pgfqpoint{6.053409in}{1.980000in}}%
\pgfusepath{stroke}%
\end{pgfscope}%
\begin{pgfscope}%
\pgfpathrectangle{\pgfqpoint{0.875000in}{0.440000in}}{\pgfqpoint{5.425000in}{3.080000in}}%
\pgfusepath{clip}%
\pgfsetbuttcap%
\pgfsetroundjoin%
\pgfsetlinewidth{1.505625pt}%
\definecolor{currentstroke}{rgb}{1.000000,0.000000,0.000000}%
\pgfsetstrokecolor{currentstroke}%
\pgfsetstrokeopacity{0.900000}%
\pgfsetdash{{5.550000pt}{2.400000pt}}{0.000000pt}%
\pgfpathmoveto{\pgfqpoint{1.121591in}{0.580000in}}%
\pgfpathlineto{\pgfqpoint{1.121591in}{3.380000in}}%
\pgfusepath{stroke}%
\end{pgfscope}%
\begin{pgfscope}%
\pgfpathrectangle{\pgfqpoint{0.875000in}{0.440000in}}{\pgfqpoint{5.425000in}{3.080000in}}%
\pgfusepath{clip}%
\pgfsetbuttcap%
\pgfsetroundjoin%
\pgfsetlinewidth{1.505625pt}%
\definecolor{currentstroke}{rgb}{1.000000,0.000000,0.000000}%
\pgfsetstrokecolor{currentstroke}%
\pgfsetstrokeopacity{0.900000}%
\pgfsetdash{{5.550000pt}{2.400000pt}}{0.000000pt}%
\pgfpathmoveto{\pgfqpoint{1.682025in}{0.580000in}}%
\pgfpathlineto{\pgfqpoint{1.682025in}{3.380000in}}%
\pgfusepath{stroke}%
\end{pgfscope}%
\begin{pgfscope}%
\pgfpathrectangle{\pgfqpoint{0.875000in}{0.440000in}}{\pgfqpoint{5.425000in}{3.080000in}}%
\pgfusepath{clip}%
\pgfsetbuttcap%
\pgfsetroundjoin%
\pgfsetlinewidth{1.505625pt}%
\definecolor{currentstroke}{rgb}{1.000000,0.000000,0.000000}%
\pgfsetstrokecolor{currentstroke}%
\pgfsetstrokeopacity{0.900000}%
\pgfsetdash{{5.550000pt}{2.400000pt}}{0.000000pt}%
\pgfpathmoveto{\pgfqpoint{2.242459in}{0.580000in}}%
\pgfpathlineto{\pgfqpoint{2.242459in}{3.380000in}}%
\pgfusepath{stroke}%
\end{pgfscope}%
\begin{pgfscope}%
\pgfpathrectangle{\pgfqpoint{0.875000in}{0.440000in}}{\pgfqpoint{5.425000in}{3.080000in}}%
\pgfusepath{clip}%
\pgfsetbuttcap%
\pgfsetroundjoin%
\pgfsetlinewidth{1.505625pt}%
\definecolor{currentstroke}{rgb}{1.000000,0.000000,0.000000}%
\pgfsetstrokecolor{currentstroke}%
\pgfsetstrokeopacity{0.900000}%
\pgfsetdash{{5.550000pt}{2.400000pt}}{0.000000pt}%
\pgfpathmoveto{\pgfqpoint{2.802893in}{0.580000in}}%
\pgfpathlineto{\pgfqpoint{2.802893in}{3.380000in}}%
\pgfusepath{stroke}%
\end{pgfscope}%
\begin{pgfscope}%
\pgfpathrectangle{\pgfqpoint{0.875000in}{0.440000in}}{\pgfqpoint{5.425000in}{3.080000in}}%
\pgfusepath{clip}%
\pgfsetbuttcap%
\pgfsetroundjoin%
\pgfsetlinewidth{1.505625pt}%
\definecolor{currentstroke}{rgb}{1.000000,0.000000,0.000000}%
\pgfsetstrokecolor{currentstroke}%
\pgfsetstrokeopacity{0.900000}%
\pgfsetdash{{5.550000pt}{2.400000pt}}{0.000000pt}%
\pgfpathmoveto{\pgfqpoint{3.363326in}{0.580000in}}%
\pgfpathlineto{\pgfqpoint{3.363326in}{3.380000in}}%
\pgfusepath{stroke}%
\end{pgfscope}%
\begin{pgfscope}%
\pgfpathrectangle{\pgfqpoint{0.875000in}{0.440000in}}{\pgfqpoint{5.425000in}{3.080000in}}%
\pgfusepath{clip}%
\pgfsetbuttcap%
\pgfsetroundjoin%
\pgfsetlinewidth{1.505625pt}%
\definecolor{currentstroke}{rgb}{1.000000,0.000000,0.000000}%
\pgfsetstrokecolor{currentstroke}%
\pgfsetstrokeopacity{0.900000}%
\pgfsetdash{{5.550000pt}{2.400000pt}}{0.000000pt}%
\pgfpathmoveto{\pgfqpoint{3.923760in}{0.580000in}}%
\pgfpathlineto{\pgfqpoint{3.923760in}{3.380000in}}%
\pgfusepath{stroke}%
\end{pgfscope}%
\begin{pgfscope}%
\pgfpathrectangle{\pgfqpoint{0.875000in}{0.440000in}}{\pgfqpoint{5.425000in}{3.080000in}}%
\pgfusepath{clip}%
\pgfsetbuttcap%
\pgfsetroundjoin%
\pgfsetlinewidth{1.505625pt}%
\definecolor{currentstroke}{rgb}{1.000000,0.000000,0.000000}%
\pgfsetstrokecolor{currentstroke}%
\pgfsetstrokeopacity{0.900000}%
\pgfsetdash{{5.550000pt}{2.400000pt}}{0.000000pt}%
\pgfpathmoveto{\pgfqpoint{4.484194in}{0.580000in}}%
\pgfpathlineto{\pgfqpoint{4.484194in}{3.380000in}}%
\pgfusepath{stroke}%
\end{pgfscope}%
\begin{pgfscope}%
\pgfpathrectangle{\pgfqpoint{0.875000in}{0.440000in}}{\pgfqpoint{5.425000in}{3.080000in}}%
\pgfusepath{clip}%
\pgfsetbuttcap%
\pgfsetroundjoin%
\pgfsetlinewidth{1.505625pt}%
\definecolor{currentstroke}{rgb}{1.000000,0.000000,0.000000}%
\pgfsetstrokecolor{currentstroke}%
\pgfsetstrokeopacity{0.900000}%
\pgfsetdash{{5.550000pt}{2.400000pt}}{0.000000pt}%
\pgfpathmoveto{\pgfqpoint{5.044628in}{0.580000in}}%
\pgfpathlineto{\pgfqpoint{5.044628in}{3.380000in}}%
\pgfusepath{stroke}%
\end{pgfscope}%
\begin{pgfscope}%
\pgfpathrectangle{\pgfqpoint{0.875000in}{0.440000in}}{\pgfqpoint{5.425000in}{3.080000in}}%
\pgfusepath{clip}%
\pgfsetbuttcap%
\pgfsetroundjoin%
\pgfsetlinewidth{1.505625pt}%
\definecolor{currentstroke}{rgb}{1.000000,0.000000,0.000000}%
\pgfsetstrokecolor{currentstroke}%
\pgfsetstrokeopacity{0.900000}%
\pgfsetdash{{5.550000pt}{2.400000pt}}{0.000000pt}%
\pgfpathmoveto{\pgfqpoint{5.605062in}{0.580000in}}%
\pgfpathlineto{\pgfqpoint{5.605062in}{3.380000in}}%
\pgfusepath{stroke}%
\end{pgfscope}%
\begin{pgfscope}%
\pgfpathrectangle{\pgfqpoint{0.875000in}{0.440000in}}{\pgfqpoint{5.425000in}{3.080000in}}%
\pgfusepath{clip}%
\pgfsetbuttcap%
\pgfsetroundjoin%
\pgfsetlinewidth{1.505625pt}%
\definecolor{currentstroke}{rgb}{0.000000,0.000000,1.000000}%
\pgfsetstrokecolor{currentstroke}%
\pgfsetstrokeopacity{0.500000}%
\pgfsetdash{{5.550000pt}{2.400000pt}}{0.000000pt}%
\pgfpathmoveto{\pgfqpoint{1.300930in}{1.980000in}}%
\pgfpathlineto{\pgfqpoint{1.300930in}{2.693096in}}%
\pgfusepath{stroke}%
\end{pgfscope}%
\begin{pgfscope}%
\pgfpathrectangle{\pgfqpoint{0.875000in}{0.440000in}}{\pgfqpoint{5.425000in}{3.080000in}}%
\pgfusepath{clip}%
\pgfsetbuttcap%
\pgfsetroundjoin%
\pgfsetlinewidth{1.505625pt}%
\definecolor{currentstroke}{rgb}{0.000000,0.000000,1.000000}%
\pgfsetstrokecolor{currentstroke}%
\pgfsetstrokeopacity{0.500000}%
\pgfsetdash{{5.550000pt}{2.400000pt}}{0.000000pt}%
\pgfpathmoveto{\pgfqpoint{2.152789in}{1.980000in}}%
\pgfpathlineto{\pgfqpoint{2.152789in}{3.018638in}}%
\pgfusepath{stroke}%
\end{pgfscope}%
\begin{pgfscope}%
\pgfpathrectangle{\pgfqpoint{0.875000in}{0.440000in}}{\pgfqpoint{5.425000in}{3.080000in}}%
\pgfusepath{clip}%
\pgfsetbuttcap%
\pgfsetroundjoin%
\pgfsetlinewidth{1.505625pt}%
\definecolor{currentstroke}{rgb}{0.000000,0.000000,1.000000}%
\pgfsetstrokecolor{currentstroke}%
\pgfsetstrokeopacity{0.500000}%
\pgfsetdash{{5.550000pt}{2.400000pt}}{0.000000pt}%
\pgfpathmoveto{\pgfqpoint{2.332128in}{1.980000in}}%
\pgfpathlineto{\pgfqpoint{2.332128in}{1.226944in}}%
\pgfusepath{stroke}%
\end{pgfscope}%
\begin{pgfscope}%
\pgfpathrectangle{\pgfqpoint{0.875000in}{0.440000in}}{\pgfqpoint{5.425000in}{3.080000in}}%
\pgfusepath{clip}%
\pgfsetbuttcap%
\pgfsetroundjoin%
\pgfsetlinewidth{1.505625pt}%
\definecolor{currentstroke}{rgb}{0.000000,0.000000,1.000000}%
\pgfsetstrokecolor{currentstroke}%
\pgfsetstrokeopacity{0.500000}%
\pgfsetdash{{5.550000pt}{2.400000pt}}{0.000000pt}%
\pgfpathmoveto{\pgfqpoint{3.251240in}{1.980000in}}%
\pgfpathlineto{\pgfqpoint{3.251240in}{2.995062in}}%
\pgfusepath{stroke}%
\end{pgfscope}%
\begin{pgfscope}%
\pgfpathrectangle{\pgfqpoint{0.875000in}{0.440000in}}{\pgfqpoint{5.425000in}{3.080000in}}%
\pgfusepath{clip}%
\pgfsetbuttcap%
\pgfsetroundjoin%
\pgfsetlinewidth{1.505625pt}%
\definecolor{currentstroke}{rgb}{0.000000,0.000000,1.000000}%
\pgfsetstrokecolor{currentstroke}%
\pgfsetstrokeopacity{0.500000}%
\pgfsetdash{{5.550000pt}{2.400000pt}}{0.000000pt}%
\pgfpathmoveto{\pgfqpoint{3.834091in}{1.980000in}}%
\pgfpathlineto{\pgfqpoint{3.834091in}{0.580000in}}%
\pgfusepath{stroke}%
\end{pgfscope}%
\begin{pgfscope}%
\pgfpathrectangle{\pgfqpoint{0.875000in}{0.440000in}}{\pgfqpoint{5.425000in}{3.080000in}}%
\pgfusepath{clip}%
\pgfsetbuttcap%
\pgfsetroundjoin%
\pgfsetlinewidth{1.505625pt}%
\definecolor{currentstroke}{rgb}{0.000000,0.000000,1.000000}%
\pgfsetstrokecolor{currentstroke}%
\pgfsetstrokeopacity{0.500000}%
\pgfsetdash{{5.550000pt}{2.400000pt}}{0.000000pt}%
\pgfpathmoveto{\pgfqpoint{3.923760in}{1.980000in}}%
\pgfpathlineto{\pgfqpoint{3.923760in}{0.967680in}}%
\pgfusepath{stroke}%
\end{pgfscope}%
\begin{pgfscope}%
\pgfpathrectangle{\pgfqpoint{0.875000in}{0.440000in}}{\pgfqpoint{5.425000in}{3.080000in}}%
\pgfusepath{clip}%
\pgfsetbuttcap%
\pgfsetroundjoin%
\pgfsetlinewidth{1.505625pt}%
\definecolor{currentstroke}{rgb}{0.000000,0.000000,1.000000}%
\pgfsetstrokecolor{currentstroke}%
\pgfsetstrokeopacity{0.500000}%
\pgfsetdash{{5.550000pt}{2.400000pt}}{0.000000pt}%
\pgfpathmoveto{\pgfqpoint{4.663533in}{1.980000in}}%
\pgfpathlineto{\pgfqpoint{4.663533in}{2.591152in}}%
\pgfusepath{stroke}%
\end{pgfscope}%
\begin{pgfscope}%
\pgfpathrectangle{\pgfqpoint{0.875000in}{0.440000in}}{\pgfqpoint{5.425000in}{3.080000in}}%
\pgfusepath{clip}%
\pgfsetbuttcap%
\pgfsetroundjoin%
\pgfsetlinewidth{1.505625pt}%
\definecolor{currentstroke}{rgb}{0.000000,0.000000,1.000000}%
\pgfsetstrokecolor{currentstroke}%
\pgfsetstrokeopacity{0.500000}%
\pgfsetdash{{5.550000pt}{2.400000pt}}{0.000000pt}%
\pgfpathmoveto{\pgfqpoint{5.223967in}{1.980000in}}%
\pgfpathlineto{\pgfqpoint{5.223967in}{1.093527in}}%
\pgfusepath{stroke}%
\end{pgfscope}%
\begin{pgfscope}%
\pgfpathrectangle{\pgfqpoint{0.875000in}{0.440000in}}{\pgfqpoint{5.425000in}{3.080000in}}%
\pgfusepath{clip}%
\pgfsetbuttcap%
\pgfsetroundjoin%
\pgfsetlinewidth{1.505625pt}%
\definecolor{currentstroke}{rgb}{0.000000,0.000000,1.000000}%
\pgfsetstrokecolor{currentstroke}%
\pgfsetstrokeopacity{0.500000}%
\pgfsetdash{{5.550000pt}{2.400000pt}}{0.000000pt}%
\pgfpathmoveto{\pgfqpoint{5.717149in}{1.980000in}}%
\pgfpathlineto{\pgfqpoint{5.717149in}{1.183774in}}%
\pgfusepath{stroke}%
\end{pgfscope}%
\begin{pgfscope}%
\pgfsetrectcap%
\pgfsetmiterjoin%
\pgfsetlinewidth{0.803000pt}%
\definecolor{currentstroke}{rgb}{0.000000,0.000000,0.000000}%
\pgfsetstrokecolor{currentstroke}%
\pgfsetdash{}{0pt}%
\pgfpathmoveto{\pgfqpoint{0.875000in}{0.440000in}}%
\pgfpathlineto{\pgfqpoint{0.875000in}{3.520000in}}%
\pgfusepath{stroke}%
\end{pgfscope}%
\begin{pgfscope}%
\pgfsetrectcap%
\pgfsetmiterjoin%
\pgfsetlinewidth{0.803000pt}%
\definecolor{currentstroke}{rgb}{0.000000,0.000000,0.000000}%
\pgfsetstrokecolor{currentstroke}%
\pgfsetdash{}{0pt}%
\pgfpathmoveto{\pgfqpoint{6.300000in}{0.440000in}}%
\pgfpathlineto{\pgfqpoint{6.300000in}{3.520000in}}%
\pgfusepath{stroke}%
\end{pgfscope}%
\begin{pgfscope}%
\pgfsetrectcap%
\pgfsetmiterjoin%
\pgfsetlinewidth{0.803000pt}%
\definecolor{currentstroke}{rgb}{0.000000,0.000000,0.000000}%
\pgfsetstrokecolor{currentstroke}%
\pgfsetdash{}{0pt}%
\pgfpathmoveto{\pgfqpoint{0.875000in}{0.440000in}}%
\pgfpathlineto{\pgfqpoint{6.300000in}{0.440000in}}%
\pgfusepath{stroke}%
\end{pgfscope}%
\begin{pgfscope}%
\pgfsetrectcap%
\pgfsetmiterjoin%
\pgfsetlinewidth{0.803000pt}%
\definecolor{currentstroke}{rgb}{0.000000,0.000000,0.000000}%
\pgfsetstrokecolor{currentstroke}%
\pgfsetdash{}{0pt}%
\pgfpathmoveto{\pgfqpoint{0.875000in}{3.520000in}}%
\pgfpathlineto{\pgfqpoint{6.300000in}{3.520000in}}%
\pgfusepath{stroke}%
\end{pgfscope}%
\begin{pgfscope}%
\definecolor{textcolor}{rgb}{0.000000,0.000000,0.000000}%
\pgfsetstrokecolor{textcolor}%
\pgfsetfillcolor{textcolor}%
\pgftext[x=1.300930in,y=2.693096in,left,base]{\color{textcolor}\sffamily\fontsize{10.000000}{12.000000}\selectfont 0.07}%
\end{pgfscope}%
\begin{pgfscope}%
\definecolor{textcolor}{rgb}{0.000000,0.000000,0.000000}%
\pgfsetstrokecolor{textcolor}%
\pgfsetfillcolor{textcolor}%
\pgftext[x=2.152789in,y=3.018638in,left,base]{\color{textcolor}\sffamily\fontsize{10.000000}{12.000000}\selectfont 0.1}%
\end{pgfscope}%
\begin{pgfscope}%
\definecolor{textcolor}{rgb}{0.000000,0.000000,0.000000}%
\pgfsetstrokecolor{textcolor}%
\pgfsetfillcolor{textcolor}%
\pgftext[x=2.332128in,y=1.226944in,left,base]{\color{textcolor}\sffamily\fontsize{10.000000}{12.000000}\selectfont -0.07}%
\end{pgfscope}%
\begin{pgfscope}%
\definecolor{textcolor}{rgb}{0.000000,0.000000,0.000000}%
\pgfsetstrokecolor{textcolor}%
\pgfsetfillcolor{textcolor}%
\pgftext[x=3.251240in,y=2.995062in,left,base]{\color{textcolor}\sffamily\fontsize{10.000000}{12.000000}\selectfont 0.1}%
\end{pgfscope}%
\begin{pgfscope}%
\definecolor{textcolor}{rgb}{0.000000,0.000000,0.000000}%
\pgfsetstrokecolor{textcolor}%
\pgfsetfillcolor{textcolor}%
\pgftext[x=3.834091in,y=0.580000in,left,base]{\color{textcolor}\sffamily\fontsize{10.000000}{12.000000}\selectfont -0.14}%
\end{pgfscope}%
\begin{pgfscope}%
\definecolor{textcolor}{rgb}{0.000000,0.000000,0.000000}%
\pgfsetstrokecolor{textcolor}%
\pgfsetfillcolor{textcolor}%
\pgftext[x=3.923760in,y=0.967680in,left,base]{\color{textcolor}\sffamily\fontsize{10.000000}{12.000000}\selectfont -0.1}%
\end{pgfscope}%
\begin{pgfscope}%
\definecolor{textcolor}{rgb}{0.000000,0.000000,0.000000}%
\pgfsetstrokecolor{textcolor}%
\pgfsetfillcolor{textcolor}%
\pgftext[x=4.663533in,y=2.591152in,left,base]{\color{textcolor}\sffamily\fontsize{10.000000}{12.000000}\selectfont 0.06}%
\end{pgfscope}%
\begin{pgfscope}%
\definecolor{textcolor}{rgb}{0.000000,0.000000,0.000000}%
\pgfsetstrokecolor{textcolor}%
\pgfsetfillcolor{textcolor}%
\pgftext[x=5.223967in,y=1.093527in,left,base]{\color{textcolor}\sffamily\fontsize{10.000000}{12.000000}\selectfont -0.09}%
\end{pgfscope}%
\begin{pgfscope}%
\definecolor{textcolor}{rgb}{0.000000,0.000000,0.000000}%
\pgfsetstrokecolor{textcolor}%
\pgfsetfillcolor{textcolor}%
\pgftext[x=5.717149in,y=1.183774in,left,base]{\color{textcolor}\sffamily\fontsize{10.000000}{12.000000}\selectfont -0.08}%
\end{pgfscope}%
\end{pgfpicture}%
\makeatother%
\endgroup%
}
    \end{center}
    \caption{\emph{Ampiezza di picco}, misurata ad intervalli regolari di 100 samples con frequenza di campionamento 22050Hz}
\end{figure}

L'\emph{ampiezza di picco} è riferita non a un istante ma a un intervallo di tempo, ed è l'ampiezza di valore assoluto massimo entro tale intervallo. Se l'intervallo è ragionevolmente ampio, può costituire una prima stima del livello dinamico percepito. L'ampiezza di picco è, in linea di principio, un valore segnato, per le stesse ragioni per cui lo è l'ampiezza istantanea: è, a tutti gli effetti, l'ampiezza istantanea rilevata all'istante in cui questa è massima in valore assoluto.

Un caso tipico in cui è importante conoscere l'ampiezza di picco è la regolazione del guadagno in un sistema di registrazione: si producono segnali forti e si regola il livello di premplificazione in maniera che l'ampiezza di picco ricevuta dal registratore (o, in un sistema digitale, dal convertitore) non ecceda il livello massimo che questo può trattare. In questo caso, come in molti altri, non è interessante sapere in quale direzione è stata rilevata la massima ampiezza: per questa ragione, viene normalmente considerato il valore assoluto dell'ampiezza di picco. 

L'ampiezza di picco può essere raggiunta più volte nell'intervallo di tempo considerato, anche con segni diversi: per esempio, una sinusoide correttamente simmetrica rispetto allo 0 raggiunge la sua ampiezza di picco due volte per ogni ciclo, una volta dal lato positivo e una dal lato negativo.



\subsection{Ampiezza picco-picco}

\begin{figure}
    \begin{center}
       \scalebox{0.6} {%% Creator: Matplotlib, PGF backend
%%
%% To include the figure in your LaTeX document, write
%%   \input{<filename>.pgf}
%%
%% Make sure the required packages are loaded in your preamble
%%   \usepackage{pgf}
%%
%% Also ensure that all the required font packages are loaded; for instance,
%% the lmodern package is sometimes necessary when using math font.
%%   \usepackage{lmodern}
%%
%% Figures using additional raster images can only be included by \input if
%% they are in the same directory as the main LaTeX file. For loading figures
%% from other directories you can use the `import` package
%%   \usepackage{import}
%%
%% and then include the figures with
%%   \import{<path to file>}{<filename>.pgf}
%%
%% Matplotlib used the following preamble
%%   
%%   \usepackage{fontspec}
%%   \setmainfont{DejaVuSerif.ttf}[Path=\detokenize{/opt/homebrew/Caskroom/miniconda/base/envs/label-studio/lib/python3.9/site-packages/matplotlib/mpl-data/fonts/ttf/}]
%%   \setsansfont{DejaVuSans.ttf}[Path=\detokenize{/opt/homebrew/Caskroom/miniconda/base/envs/label-studio/lib/python3.9/site-packages/matplotlib/mpl-data/fonts/ttf/}]
%%   \setmonofont{DejaVuSansMono.ttf}[Path=\detokenize{/opt/homebrew/Caskroom/miniconda/base/envs/label-studio/lib/python3.9/site-packages/matplotlib/mpl-data/fonts/ttf/}]
%%   \makeatletter\@ifpackageloaded{underscore}{}{\usepackage[strings]{underscore}}\makeatother
%%
\begingroup%
\makeatletter%
\begin{pgfpicture}%
\pgfpathrectangle{\pgfpointorigin}{\pgfqpoint{7.000000in}{4.000000in}}%
\pgfusepath{use as bounding box, clip}%
\begin{pgfscope}%
\pgfsetbuttcap%
\pgfsetmiterjoin%
\definecolor{currentfill}{rgb}{1.000000,1.000000,1.000000}%
\pgfsetfillcolor{currentfill}%
\pgfsetlinewidth{0.000000pt}%
\definecolor{currentstroke}{rgb}{1.000000,1.000000,1.000000}%
\pgfsetstrokecolor{currentstroke}%
\pgfsetdash{}{0pt}%
\pgfpathmoveto{\pgfqpoint{0.000000in}{0.000000in}}%
\pgfpathlineto{\pgfqpoint{7.000000in}{0.000000in}}%
\pgfpathlineto{\pgfqpoint{7.000000in}{4.000000in}}%
\pgfpathlineto{\pgfqpoint{0.000000in}{4.000000in}}%
\pgfpathlineto{\pgfqpoint{0.000000in}{0.000000in}}%
\pgfpathclose%
\pgfusepath{fill}%
\end{pgfscope}%
\begin{pgfscope}%
\pgfsetbuttcap%
\pgfsetmiterjoin%
\definecolor{currentfill}{rgb}{1.000000,1.000000,1.000000}%
\pgfsetfillcolor{currentfill}%
\pgfsetlinewidth{0.000000pt}%
\definecolor{currentstroke}{rgb}{0.000000,0.000000,0.000000}%
\pgfsetstrokecolor{currentstroke}%
\pgfsetstrokeopacity{0.000000}%
\pgfsetdash{}{0pt}%
\pgfpathmoveto{\pgfqpoint{0.875000in}{0.440000in}}%
\pgfpathlineto{\pgfqpoint{6.300000in}{0.440000in}}%
\pgfpathlineto{\pgfqpoint{6.300000in}{3.520000in}}%
\pgfpathlineto{\pgfqpoint{0.875000in}{3.520000in}}%
\pgfpathlineto{\pgfqpoint{0.875000in}{0.440000in}}%
\pgfpathclose%
\pgfusepath{fill}%
\end{pgfscope}%
\begin{pgfscope}%
\pgfsetbuttcap%
\pgfsetroundjoin%
\definecolor{currentfill}{rgb}{0.000000,0.000000,0.000000}%
\pgfsetfillcolor{currentfill}%
\pgfsetlinewidth{0.803000pt}%
\definecolor{currentstroke}{rgb}{0.000000,0.000000,0.000000}%
\pgfsetstrokecolor{currentstroke}%
\pgfsetdash{}{0pt}%
\pgfsys@defobject{currentmarker}{\pgfqpoint{0.000000in}{-0.048611in}}{\pgfqpoint{0.000000in}{0.000000in}}{%
\pgfpathmoveto{\pgfqpoint{0.000000in}{0.000000in}}%
\pgfpathlineto{\pgfqpoint{0.000000in}{-0.048611in}}%
\pgfusepath{stroke,fill}%
}%
\begin{pgfscope}%
\pgfsys@transformshift{1.121591in}{0.440000in}%
\pgfsys@useobject{currentmarker}{}%
\end{pgfscope}%
\end{pgfscope}%
\begin{pgfscope}%
\definecolor{textcolor}{rgb}{0.000000,0.000000,0.000000}%
\pgfsetstrokecolor{textcolor}%
\pgfsetfillcolor{textcolor}%
\pgftext[x=1.121591in,y=0.342778in,,top]{\color{textcolor}\sffamily\fontsize{10.000000}{12.000000}\selectfont 0.000}%
\end{pgfscope}%
\begin{pgfscope}%
\pgfsetbuttcap%
\pgfsetroundjoin%
\definecolor{currentfill}{rgb}{0.000000,0.000000,0.000000}%
\pgfsetfillcolor{currentfill}%
\pgfsetlinewidth{0.803000pt}%
\definecolor{currentstroke}{rgb}{0.000000,0.000000,0.000000}%
\pgfsetstrokecolor{currentstroke}%
\pgfsetdash{}{0pt}%
\pgfsys@defobject{currentmarker}{\pgfqpoint{0.000000in}{-0.048611in}}{\pgfqpoint{0.000000in}{0.000000in}}{%
\pgfpathmoveto{\pgfqpoint{0.000000in}{0.000000in}}%
\pgfpathlineto{\pgfqpoint{0.000000in}{-0.048611in}}%
\pgfusepath{stroke,fill}%
}%
\begin{pgfscope}%
\pgfsys@transformshift{1.861364in}{0.440000in}%
\pgfsys@useobject{currentmarker}{}%
\end{pgfscope}%
\end{pgfscope}%
\begin{pgfscope}%
\definecolor{textcolor}{rgb}{0.000000,0.000000,0.000000}%
\pgfsetstrokecolor{textcolor}%
\pgfsetfillcolor{textcolor}%
\pgftext[x=1.861364in,y=0.342778in,,top]{\color{textcolor}\sffamily\fontsize{10.000000}{12.000000}\selectfont 0.002}%
\end{pgfscope}%
\begin{pgfscope}%
\pgfsetbuttcap%
\pgfsetroundjoin%
\definecolor{currentfill}{rgb}{0.000000,0.000000,0.000000}%
\pgfsetfillcolor{currentfill}%
\pgfsetlinewidth{0.803000pt}%
\definecolor{currentstroke}{rgb}{0.000000,0.000000,0.000000}%
\pgfsetstrokecolor{currentstroke}%
\pgfsetdash{}{0pt}%
\pgfsys@defobject{currentmarker}{\pgfqpoint{0.000000in}{-0.048611in}}{\pgfqpoint{0.000000in}{0.000000in}}{%
\pgfpathmoveto{\pgfqpoint{0.000000in}{0.000000in}}%
\pgfpathlineto{\pgfqpoint{0.000000in}{-0.048611in}}%
\pgfusepath{stroke,fill}%
}%
\begin{pgfscope}%
\pgfsys@transformshift{2.601136in}{0.440000in}%
\pgfsys@useobject{currentmarker}{}%
\end{pgfscope}%
\end{pgfscope}%
\begin{pgfscope}%
\definecolor{textcolor}{rgb}{0.000000,0.000000,0.000000}%
\pgfsetstrokecolor{textcolor}%
\pgfsetfillcolor{textcolor}%
\pgftext[x=2.601136in,y=0.342778in,,top]{\color{textcolor}\sffamily\fontsize{10.000000}{12.000000}\selectfont 0.003}%
\end{pgfscope}%
\begin{pgfscope}%
\pgfsetbuttcap%
\pgfsetroundjoin%
\definecolor{currentfill}{rgb}{0.000000,0.000000,0.000000}%
\pgfsetfillcolor{currentfill}%
\pgfsetlinewidth{0.803000pt}%
\definecolor{currentstroke}{rgb}{0.000000,0.000000,0.000000}%
\pgfsetstrokecolor{currentstroke}%
\pgfsetdash{}{0pt}%
\pgfsys@defobject{currentmarker}{\pgfqpoint{0.000000in}{-0.048611in}}{\pgfqpoint{0.000000in}{0.000000in}}{%
\pgfpathmoveto{\pgfqpoint{0.000000in}{0.000000in}}%
\pgfpathlineto{\pgfqpoint{0.000000in}{-0.048611in}}%
\pgfusepath{stroke,fill}%
}%
\begin{pgfscope}%
\pgfsys@transformshift{3.340909in}{0.440000in}%
\pgfsys@useobject{currentmarker}{}%
\end{pgfscope}%
\end{pgfscope}%
\begin{pgfscope}%
\definecolor{textcolor}{rgb}{0.000000,0.000000,0.000000}%
\pgfsetstrokecolor{textcolor}%
\pgfsetfillcolor{textcolor}%
\pgftext[x=3.340909in,y=0.342778in,,top]{\color{textcolor}\sffamily\fontsize{10.000000}{12.000000}\selectfont 0.005}%
\end{pgfscope}%
\begin{pgfscope}%
\pgfsetbuttcap%
\pgfsetroundjoin%
\definecolor{currentfill}{rgb}{0.000000,0.000000,0.000000}%
\pgfsetfillcolor{currentfill}%
\pgfsetlinewidth{0.803000pt}%
\definecolor{currentstroke}{rgb}{0.000000,0.000000,0.000000}%
\pgfsetstrokecolor{currentstroke}%
\pgfsetdash{}{0pt}%
\pgfsys@defobject{currentmarker}{\pgfqpoint{0.000000in}{-0.048611in}}{\pgfqpoint{0.000000in}{0.000000in}}{%
\pgfpathmoveto{\pgfqpoint{0.000000in}{0.000000in}}%
\pgfpathlineto{\pgfqpoint{0.000000in}{-0.048611in}}%
\pgfusepath{stroke,fill}%
}%
\begin{pgfscope}%
\pgfsys@transformshift{4.080682in}{0.440000in}%
\pgfsys@useobject{currentmarker}{}%
\end{pgfscope}%
\end{pgfscope}%
\begin{pgfscope}%
\definecolor{textcolor}{rgb}{0.000000,0.000000,0.000000}%
\pgfsetstrokecolor{textcolor}%
\pgfsetfillcolor{textcolor}%
\pgftext[x=4.080682in,y=0.342778in,,top]{\color{textcolor}\sffamily\fontsize{10.000000}{12.000000}\selectfont 0.006}%
\end{pgfscope}%
\begin{pgfscope}%
\pgfsetbuttcap%
\pgfsetroundjoin%
\definecolor{currentfill}{rgb}{0.000000,0.000000,0.000000}%
\pgfsetfillcolor{currentfill}%
\pgfsetlinewidth{0.803000pt}%
\definecolor{currentstroke}{rgb}{0.000000,0.000000,0.000000}%
\pgfsetstrokecolor{currentstroke}%
\pgfsetdash{}{0pt}%
\pgfsys@defobject{currentmarker}{\pgfqpoint{0.000000in}{-0.048611in}}{\pgfqpoint{0.000000in}{0.000000in}}{%
\pgfpathmoveto{\pgfqpoint{0.000000in}{0.000000in}}%
\pgfpathlineto{\pgfqpoint{0.000000in}{-0.048611in}}%
\pgfusepath{stroke,fill}%
}%
\begin{pgfscope}%
\pgfsys@transformshift{4.820455in}{0.440000in}%
\pgfsys@useobject{currentmarker}{}%
\end{pgfscope}%
\end{pgfscope}%
\begin{pgfscope}%
\definecolor{textcolor}{rgb}{0.000000,0.000000,0.000000}%
\pgfsetstrokecolor{textcolor}%
\pgfsetfillcolor{textcolor}%
\pgftext[x=4.820455in,y=0.342778in,,top]{\color{textcolor}\sffamily\fontsize{10.000000}{12.000000}\selectfont 0.008}%
\end{pgfscope}%
\begin{pgfscope}%
\pgfsetbuttcap%
\pgfsetroundjoin%
\definecolor{currentfill}{rgb}{0.000000,0.000000,0.000000}%
\pgfsetfillcolor{currentfill}%
\pgfsetlinewidth{0.803000pt}%
\definecolor{currentstroke}{rgb}{0.000000,0.000000,0.000000}%
\pgfsetstrokecolor{currentstroke}%
\pgfsetdash{}{0pt}%
\pgfsys@defobject{currentmarker}{\pgfqpoint{0.000000in}{-0.048611in}}{\pgfqpoint{0.000000in}{0.000000in}}{%
\pgfpathmoveto{\pgfqpoint{0.000000in}{0.000000in}}%
\pgfpathlineto{\pgfqpoint{0.000000in}{-0.048611in}}%
\pgfusepath{stroke,fill}%
}%
\begin{pgfscope}%
\pgfsys@transformshift{5.560227in}{0.440000in}%
\pgfsys@useobject{currentmarker}{}%
\end{pgfscope}%
\end{pgfscope}%
\begin{pgfscope}%
\definecolor{textcolor}{rgb}{0.000000,0.000000,0.000000}%
\pgfsetstrokecolor{textcolor}%
\pgfsetfillcolor{textcolor}%
\pgftext[x=5.560227in,y=0.342778in,,top]{\color{textcolor}\sffamily\fontsize{10.000000}{12.000000}\selectfont 0.009}%
\end{pgfscope}%
\begin{pgfscope}%
\pgfsetbuttcap%
\pgfsetroundjoin%
\definecolor{currentfill}{rgb}{0.000000,0.000000,0.000000}%
\pgfsetfillcolor{currentfill}%
\pgfsetlinewidth{0.803000pt}%
\definecolor{currentstroke}{rgb}{0.000000,0.000000,0.000000}%
\pgfsetstrokecolor{currentstroke}%
\pgfsetdash{}{0pt}%
\pgfsys@defobject{currentmarker}{\pgfqpoint{0.000000in}{-0.048611in}}{\pgfqpoint{0.000000in}{0.000000in}}{%
\pgfpathmoveto{\pgfqpoint{0.000000in}{0.000000in}}%
\pgfpathlineto{\pgfqpoint{0.000000in}{-0.048611in}}%
\pgfusepath{stroke,fill}%
}%
\begin{pgfscope}%
\pgfsys@transformshift{6.300000in}{0.440000in}%
\pgfsys@useobject{currentmarker}{}%
\end{pgfscope}%
\end{pgfscope}%
\begin{pgfscope}%
\definecolor{textcolor}{rgb}{0.000000,0.000000,0.000000}%
\pgfsetstrokecolor{textcolor}%
\pgfsetfillcolor{textcolor}%
\pgftext[x=6.300000in,y=0.342778in,,top]{\color{textcolor}\sffamily\fontsize{10.000000}{12.000000}\selectfont 0.011}%
\end{pgfscope}%
\begin{pgfscope}%
\definecolor{textcolor}{rgb}{0.000000,0.000000,0.000000}%
\pgfsetstrokecolor{textcolor}%
\pgfsetfillcolor{textcolor}%
\pgftext[x=3.587500in,y=0.152809in,,top]{\color{textcolor}\sffamily\fontsize{10.000000}{12.000000}\selectfont Time}%
\end{pgfscope}%
\begin{pgfscope}%
\pgfsetbuttcap%
\pgfsetroundjoin%
\definecolor{currentfill}{rgb}{0.000000,0.000000,0.000000}%
\pgfsetfillcolor{currentfill}%
\pgfsetlinewidth{0.803000pt}%
\definecolor{currentstroke}{rgb}{0.000000,0.000000,0.000000}%
\pgfsetstrokecolor{currentstroke}%
\pgfsetdash{}{0pt}%
\pgfsys@defobject{currentmarker}{\pgfqpoint{-0.048611in}{0.000000in}}{\pgfqpoint{-0.000000in}{0.000000in}}{%
\pgfpathmoveto{\pgfqpoint{-0.000000in}{0.000000in}}%
\pgfpathlineto{\pgfqpoint{-0.048611in}{0.000000in}}%
\pgfusepath{stroke,fill}%
}%
\begin{pgfscope}%
\pgfsys@transformshift{0.875000in}{0.470530in}%
\pgfsys@useobject{currentmarker}{}%
\end{pgfscope}%
\end{pgfscope}%
\begin{pgfscope}%
\definecolor{textcolor}{rgb}{0.000000,0.000000,0.000000}%
\pgfsetstrokecolor{textcolor}%
\pgfsetfillcolor{textcolor}%
\pgftext[x=0.360508in, y=0.417768in, left, base]{\color{textcolor}\sffamily\fontsize{10.000000}{12.000000}\selectfont \ensuremath{-}0.15}%
\end{pgfscope}%
\begin{pgfscope}%
\pgfsetbuttcap%
\pgfsetroundjoin%
\definecolor{currentfill}{rgb}{0.000000,0.000000,0.000000}%
\pgfsetfillcolor{currentfill}%
\pgfsetlinewidth{0.803000pt}%
\definecolor{currentstroke}{rgb}{0.000000,0.000000,0.000000}%
\pgfsetstrokecolor{currentstroke}%
\pgfsetdash{}{0pt}%
\pgfsys@defobject{currentmarker}{\pgfqpoint{-0.048611in}{0.000000in}}{\pgfqpoint{-0.000000in}{0.000000in}}{%
\pgfpathmoveto{\pgfqpoint{-0.000000in}{0.000000in}}%
\pgfpathlineto{\pgfqpoint{-0.048611in}{0.000000in}}%
\pgfusepath{stroke,fill}%
}%
\begin{pgfscope}%
\pgfsys@transformshift{0.875000in}{0.973686in}%
\pgfsys@useobject{currentmarker}{}%
\end{pgfscope}%
\end{pgfscope}%
\begin{pgfscope}%
\definecolor{textcolor}{rgb}{0.000000,0.000000,0.000000}%
\pgfsetstrokecolor{textcolor}%
\pgfsetfillcolor{textcolor}%
\pgftext[x=0.360508in, y=0.920925in, left, base]{\color{textcolor}\sffamily\fontsize{10.000000}{12.000000}\selectfont \ensuremath{-}0.10}%
\end{pgfscope}%
\begin{pgfscope}%
\pgfsetbuttcap%
\pgfsetroundjoin%
\definecolor{currentfill}{rgb}{0.000000,0.000000,0.000000}%
\pgfsetfillcolor{currentfill}%
\pgfsetlinewidth{0.803000pt}%
\definecolor{currentstroke}{rgb}{0.000000,0.000000,0.000000}%
\pgfsetstrokecolor{currentstroke}%
\pgfsetdash{}{0pt}%
\pgfsys@defobject{currentmarker}{\pgfqpoint{-0.048611in}{0.000000in}}{\pgfqpoint{-0.000000in}{0.000000in}}{%
\pgfpathmoveto{\pgfqpoint{-0.000000in}{0.000000in}}%
\pgfpathlineto{\pgfqpoint{-0.048611in}{0.000000in}}%
\pgfusepath{stroke,fill}%
}%
\begin{pgfscope}%
\pgfsys@transformshift{0.875000in}{1.476843in}%
\pgfsys@useobject{currentmarker}{}%
\end{pgfscope}%
\end{pgfscope}%
\begin{pgfscope}%
\definecolor{textcolor}{rgb}{0.000000,0.000000,0.000000}%
\pgfsetstrokecolor{textcolor}%
\pgfsetfillcolor{textcolor}%
\pgftext[x=0.360508in, y=1.424082in, left, base]{\color{textcolor}\sffamily\fontsize{10.000000}{12.000000}\selectfont \ensuremath{-}0.05}%
\end{pgfscope}%
\begin{pgfscope}%
\pgfsetbuttcap%
\pgfsetroundjoin%
\definecolor{currentfill}{rgb}{0.000000,0.000000,0.000000}%
\pgfsetfillcolor{currentfill}%
\pgfsetlinewidth{0.803000pt}%
\definecolor{currentstroke}{rgb}{0.000000,0.000000,0.000000}%
\pgfsetstrokecolor{currentstroke}%
\pgfsetdash{}{0pt}%
\pgfsys@defobject{currentmarker}{\pgfqpoint{-0.048611in}{0.000000in}}{\pgfqpoint{-0.000000in}{0.000000in}}{%
\pgfpathmoveto{\pgfqpoint{-0.000000in}{0.000000in}}%
\pgfpathlineto{\pgfqpoint{-0.048611in}{0.000000in}}%
\pgfusepath{stroke,fill}%
}%
\begin{pgfscope}%
\pgfsys@transformshift{0.875000in}{1.980000in}%
\pgfsys@useobject{currentmarker}{}%
\end{pgfscope}%
\end{pgfscope}%
\begin{pgfscope}%
\definecolor{textcolor}{rgb}{0.000000,0.000000,0.000000}%
\pgfsetstrokecolor{textcolor}%
\pgfsetfillcolor{textcolor}%
\pgftext[x=0.468533in, y=1.927238in, left, base]{\color{textcolor}\sffamily\fontsize{10.000000}{12.000000}\selectfont 0.00}%
\end{pgfscope}%
\begin{pgfscope}%
\pgfsetbuttcap%
\pgfsetroundjoin%
\definecolor{currentfill}{rgb}{0.000000,0.000000,0.000000}%
\pgfsetfillcolor{currentfill}%
\pgfsetlinewidth{0.803000pt}%
\definecolor{currentstroke}{rgb}{0.000000,0.000000,0.000000}%
\pgfsetstrokecolor{currentstroke}%
\pgfsetdash{}{0pt}%
\pgfsys@defobject{currentmarker}{\pgfqpoint{-0.048611in}{0.000000in}}{\pgfqpoint{-0.000000in}{0.000000in}}{%
\pgfpathmoveto{\pgfqpoint{-0.000000in}{0.000000in}}%
\pgfpathlineto{\pgfqpoint{-0.048611in}{0.000000in}}%
\pgfusepath{stroke,fill}%
}%
\begin{pgfscope}%
\pgfsys@transformshift{0.875000in}{2.483157in}%
\pgfsys@useobject{currentmarker}{}%
\end{pgfscope}%
\end{pgfscope}%
\begin{pgfscope}%
\definecolor{textcolor}{rgb}{0.000000,0.000000,0.000000}%
\pgfsetstrokecolor{textcolor}%
\pgfsetfillcolor{textcolor}%
\pgftext[x=0.468533in, y=2.430395in, left, base]{\color{textcolor}\sffamily\fontsize{10.000000}{12.000000}\selectfont 0.05}%
\end{pgfscope}%
\begin{pgfscope}%
\pgfsetbuttcap%
\pgfsetroundjoin%
\definecolor{currentfill}{rgb}{0.000000,0.000000,0.000000}%
\pgfsetfillcolor{currentfill}%
\pgfsetlinewidth{0.803000pt}%
\definecolor{currentstroke}{rgb}{0.000000,0.000000,0.000000}%
\pgfsetstrokecolor{currentstroke}%
\pgfsetdash{}{0pt}%
\pgfsys@defobject{currentmarker}{\pgfqpoint{-0.048611in}{0.000000in}}{\pgfqpoint{-0.000000in}{0.000000in}}{%
\pgfpathmoveto{\pgfqpoint{-0.000000in}{0.000000in}}%
\pgfpathlineto{\pgfqpoint{-0.048611in}{0.000000in}}%
\pgfusepath{stroke,fill}%
}%
\begin{pgfscope}%
\pgfsys@transformshift{0.875000in}{2.986314in}%
\pgfsys@useobject{currentmarker}{}%
\end{pgfscope}%
\end{pgfscope}%
\begin{pgfscope}%
\definecolor{textcolor}{rgb}{0.000000,0.000000,0.000000}%
\pgfsetstrokecolor{textcolor}%
\pgfsetfillcolor{textcolor}%
\pgftext[x=0.468533in, y=2.933552in, left, base]{\color{textcolor}\sffamily\fontsize{10.000000}{12.000000}\selectfont 0.10}%
\end{pgfscope}%
\begin{pgfscope}%
\pgfsetbuttcap%
\pgfsetroundjoin%
\definecolor{currentfill}{rgb}{0.000000,0.000000,0.000000}%
\pgfsetfillcolor{currentfill}%
\pgfsetlinewidth{0.803000pt}%
\definecolor{currentstroke}{rgb}{0.000000,0.000000,0.000000}%
\pgfsetstrokecolor{currentstroke}%
\pgfsetdash{}{0pt}%
\pgfsys@defobject{currentmarker}{\pgfqpoint{-0.048611in}{0.000000in}}{\pgfqpoint{-0.000000in}{0.000000in}}{%
\pgfpathmoveto{\pgfqpoint{-0.000000in}{0.000000in}}%
\pgfpathlineto{\pgfqpoint{-0.048611in}{0.000000in}}%
\pgfusepath{stroke,fill}%
}%
\begin{pgfscope}%
\pgfsys@transformshift{0.875000in}{3.489470in}%
\pgfsys@useobject{currentmarker}{}%
\end{pgfscope}%
\end{pgfscope}%
\begin{pgfscope}%
\definecolor{textcolor}{rgb}{0.000000,0.000000,0.000000}%
\pgfsetstrokecolor{textcolor}%
\pgfsetfillcolor{textcolor}%
\pgftext[x=0.468533in, y=3.436709in, left, base]{\color{textcolor}\sffamily\fontsize{10.000000}{12.000000}\selectfont 0.15}%
\end{pgfscope}%
\begin{pgfscope}%
\pgfpathrectangle{\pgfqpoint{0.875000in}{0.440000in}}{\pgfqpoint{5.425000in}{3.080000in}}%
\pgfusepath{clip}%
\pgfsetrectcap%
\pgfsetroundjoin%
\pgfsetlinewidth{1.505625pt}%
\definecolor{currentstroke}{rgb}{0.000000,0.000000,0.000000}%
\pgfsetstrokecolor{currentstroke}%
\pgfsetstrokeopacity{0.200000}%
\pgfsetdash{}{0pt}%
\pgfpathmoveto{\pgfqpoint{1.121591in}{2.058022in}}%
\pgfpathlineto{\pgfqpoint{1.143957in}{2.058022in}}%
\pgfpathlineto{\pgfqpoint{1.143957in}{2.084450in}}%
\pgfpathlineto{\pgfqpoint{1.166324in}{2.084450in}}%
\pgfpathlineto{\pgfqpoint{1.166324in}{2.234479in}}%
\pgfpathlineto{\pgfqpoint{1.188690in}{2.234479in}}%
\pgfpathlineto{\pgfqpoint{1.188690in}{2.334473in}}%
\pgfpathlineto{\pgfqpoint{1.211057in}{2.334473in}}%
\pgfpathlineto{\pgfqpoint{1.211057in}{2.214030in}}%
\pgfpathlineto{\pgfqpoint{1.233424in}{2.214030in}}%
\pgfpathlineto{\pgfqpoint{1.233424in}{2.289679in}}%
\pgfpathlineto{\pgfqpoint{1.255790in}{2.289679in}}%
\pgfpathlineto{\pgfqpoint{1.255790in}{2.292441in}}%
\pgfpathlineto{\pgfqpoint{1.278157in}{2.292441in}}%
\pgfpathlineto{\pgfqpoint{1.278157in}{2.518721in}}%
\pgfpathlineto{\pgfqpoint{1.300523in}{2.518721in}}%
\pgfpathlineto{\pgfqpoint{1.300523in}{2.693096in}}%
\pgfpathlineto{\pgfqpoint{1.322890in}{2.693096in}}%
\pgfpathlineto{\pgfqpoint{1.322890in}{2.539911in}}%
\pgfpathlineto{\pgfqpoint{1.345256in}{2.539911in}}%
\pgfpathlineto{\pgfqpoint{1.345256in}{2.258216in}}%
\pgfpathlineto{\pgfqpoint{1.367623in}{2.258216in}}%
\pgfpathlineto{\pgfqpoint{1.367623in}{2.022593in}}%
\pgfpathlineto{\pgfqpoint{1.389989in}{2.022593in}}%
\pgfpathlineto{\pgfqpoint{1.389989in}{1.691430in}}%
\pgfpathlineto{\pgfqpoint{1.412356in}{1.691430in}}%
\pgfpathlineto{\pgfqpoint{1.412356in}{1.387736in}}%
\pgfpathlineto{\pgfqpoint{1.434722in}{1.387736in}}%
\pgfpathlineto{\pgfqpoint{1.434722in}{1.364084in}}%
\pgfpathlineto{\pgfqpoint{1.457089in}{1.364084in}}%
\pgfpathlineto{\pgfqpoint{1.457089in}{1.374447in}}%
\pgfpathlineto{\pgfqpoint{1.479455in}{1.374447in}}%
\pgfpathlineto{\pgfqpoint{1.479455in}{1.668828in}}%
\pgfpathlineto{\pgfqpoint{1.501822in}{1.668828in}}%
\pgfpathlineto{\pgfqpoint{1.501822in}{2.113240in}}%
\pgfpathlineto{\pgfqpoint{1.524188in}{2.113240in}}%
\pgfpathlineto{\pgfqpoint{1.524188in}{2.490036in}}%
\pgfpathlineto{\pgfqpoint{1.546555in}{2.490036in}}%
\pgfpathlineto{\pgfqpoint{1.546555in}{2.372557in}}%
\pgfpathlineto{\pgfqpoint{1.568921in}{2.372557in}}%
\pgfpathlineto{\pgfqpoint{1.568921in}{1.858670in}}%
\pgfpathlineto{\pgfqpoint{1.591288in}{1.858670in}}%
\pgfpathlineto{\pgfqpoint{1.591288in}{1.525524in}}%
\pgfpathlineto{\pgfqpoint{1.613654in}{1.525524in}}%
\pgfpathlineto{\pgfqpoint{1.613654in}{1.377401in}}%
\pgfpathlineto{\pgfqpoint{1.636021in}{1.377401in}}%
\pgfpathlineto{\pgfqpoint{1.636021in}{1.426161in}}%
\pgfpathlineto{\pgfqpoint{1.658387in}{1.426161in}}%
\pgfpathlineto{\pgfqpoint{1.658387in}{1.584538in}}%
\pgfpathlineto{\pgfqpoint{1.680754in}{1.584538in}}%
\pgfpathlineto{\pgfqpoint{1.680754in}{1.858629in}}%
\pgfpathlineto{\pgfqpoint{1.703120in}{1.858629in}}%
\pgfpathlineto{\pgfqpoint{1.703120in}{2.213921in}}%
\pgfpathlineto{\pgfqpoint{1.725487in}{2.213921in}}%
\pgfpathlineto{\pgfqpoint{1.725487in}{2.313546in}}%
\pgfpathlineto{\pgfqpoint{1.747854in}{2.313546in}}%
\pgfpathlineto{\pgfqpoint{1.747854in}{2.251492in}}%
\pgfpathlineto{\pgfqpoint{1.770220in}{2.251492in}}%
\pgfpathlineto{\pgfqpoint{1.770220in}{2.106918in}}%
\pgfpathlineto{\pgfqpoint{1.792587in}{2.106918in}}%
\pgfpathlineto{\pgfqpoint{1.792587in}{1.940703in}}%
\pgfpathlineto{\pgfqpoint{1.814953in}{1.940703in}}%
\pgfpathlineto{\pgfqpoint{1.814953in}{1.830157in}}%
\pgfpathlineto{\pgfqpoint{1.837320in}{1.830157in}}%
\pgfpathlineto{\pgfqpoint{1.837320in}{1.593971in}}%
\pgfpathlineto{\pgfqpoint{1.859686in}{1.593971in}}%
\pgfpathlineto{\pgfqpoint{1.859686in}{1.495126in}}%
\pgfpathlineto{\pgfqpoint{1.882053in}{1.495126in}}%
\pgfpathlineto{\pgfqpoint{1.882053in}{1.608910in}}%
\pgfpathlineto{\pgfqpoint{1.904419in}{1.608910in}}%
\pgfpathlineto{\pgfqpoint{1.904419in}{1.911871in}}%
\pgfpathlineto{\pgfqpoint{1.926786in}{1.911871in}}%
\pgfpathlineto{\pgfqpoint{1.926786in}{2.157539in}}%
\pgfpathlineto{\pgfqpoint{1.949152in}{2.157539in}}%
\pgfpathlineto{\pgfqpoint{1.949152in}{2.333096in}}%
\pgfpathlineto{\pgfqpoint{1.971519in}{2.333096in}}%
\pgfpathlineto{\pgfqpoint{1.971519in}{2.553085in}}%
\pgfpathlineto{\pgfqpoint{1.993885in}{2.553085in}}%
\pgfpathlineto{\pgfqpoint{1.993885in}{2.609304in}}%
\pgfpathlineto{\pgfqpoint{2.016252in}{2.609304in}}%
\pgfpathlineto{\pgfqpoint{2.016252in}{2.536976in}}%
\pgfpathlineto{\pgfqpoint{2.038618in}{2.536976in}}%
\pgfpathlineto{\pgfqpoint{2.038618in}{2.304702in}}%
\pgfpathlineto{\pgfqpoint{2.060985in}{2.304702in}}%
\pgfpathlineto{\pgfqpoint{2.060985in}{2.156759in}}%
\pgfpathlineto{\pgfqpoint{2.083351in}{2.156759in}}%
\pgfpathlineto{\pgfqpoint{2.083351in}{2.369195in}}%
\pgfpathlineto{\pgfqpoint{2.105718in}{2.369195in}}%
\pgfpathlineto{\pgfqpoint{2.105718in}{2.697345in}}%
\pgfpathlineto{\pgfqpoint{2.128084in}{2.697345in}}%
\pgfpathlineto{\pgfqpoint{2.128084in}{2.913581in}}%
\pgfpathlineto{\pgfqpoint{2.150451in}{2.913581in}}%
\pgfpathlineto{\pgfqpoint{2.150451in}{3.018638in}}%
\pgfpathlineto{\pgfqpoint{2.172817in}{3.018638in}}%
\pgfpathlineto{\pgfqpoint{2.172817in}{2.811905in}}%
\pgfpathlineto{\pgfqpoint{2.195184in}{2.811905in}}%
\pgfpathlineto{\pgfqpoint{2.195184in}{2.537266in}}%
\pgfpathlineto{\pgfqpoint{2.217551in}{2.537266in}}%
\pgfpathlineto{\pgfqpoint{2.217551in}{2.298059in}}%
\pgfpathlineto{\pgfqpoint{2.239917in}{2.298059in}}%
\pgfpathlineto{\pgfqpoint{2.239917in}{2.226243in}}%
\pgfpathlineto{\pgfqpoint{2.262284in}{2.226243in}}%
\pgfpathlineto{\pgfqpoint{2.262284in}{2.167852in}}%
\pgfpathlineto{\pgfqpoint{2.284650in}{2.167852in}}%
\pgfpathlineto{\pgfqpoint{2.284650in}{1.910965in}}%
\pgfpathlineto{\pgfqpoint{2.307017in}{1.910965in}}%
\pgfpathlineto{\pgfqpoint{2.307017in}{1.586608in}}%
\pgfpathlineto{\pgfqpoint{2.329383in}{1.586608in}}%
\pgfpathlineto{\pgfqpoint{2.329383in}{1.226944in}}%
\pgfpathlineto{\pgfqpoint{2.351750in}{1.226944in}}%
\pgfpathlineto{\pgfqpoint{2.351750in}{1.342153in}}%
\pgfpathlineto{\pgfqpoint{2.374116in}{1.342153in}}%
\pgfpathlineto{\pgfqpoint{2.374116in}{1.463454in}}%
\pgfpathlineto{\pgfqpoint{2.396483in}{1.463454in}}%
\pgfpathlineto{\pgfqpoint{2.396483in}{1.551943in}}%
\pgfpathlineto{\pgfqpoint{2.418849in}{1.551943in}}%
\pgfpathlineto{\pgfqpoint{2.418849in}{1.764077in}}%
\pgfpathlineto{\pgfqpoint{2.441216in}{1.764077in}}%
\pgfpathlineto{\pgfqpoint{2.441216in}{1.690474in}}%
\pgfpathlineto{\pgfqpoint{2.463582in}{1.690474in}}%
\pgfpathlineto{\pgfqpoint{2.463582in}{1.443981in}}%
\pgfpathlineto{\pgfqpoint{2.485949in}{1.443981in}}%
\pgfpathlineto{\pgfqpoint{2.485949in}{1.438443in}}%
\pgfpathlineto{\pgfqpoint{2.508315in}{1.438443in}}%
\pgfpathlineto{\pgfqpoint{2.508315in}{1.436858in}}%
\pgfpathlineto{\pgfqpoint{2.530682in}{1.436858in}}%
\pgfpathlineto{\pgfqpoint{2.530682in}{1.255702in}}%
\pgfpathlineto{\pgfqpoint{2.553048in}{1.255702in}}%
\pgfpathlineto{\pgfqpoint{2.553048in}{1.369052in}}%
\pgfpathlineto{\pgfqpoint{2.575415in}{1.369052in}}%
\pgfpathlineto{\pgfqpoint{2.575415in}{1.661540in}}%
\pgfpathlineto{\pgfqpoint{2.597781in}{1.661540in}}%
\pgfpathlineto{\pgfqpoint{2.597781in}{1.994127in}}%
\pgfpathlineto{\pgfqpoint{2.620148in}{1.994127in}}%
\pgfpathlineto{\pgfqpoint{2.620148in}{2.048489in}}%
\pgfpathlineto{\pgfqpoint{2.642514in}{2.048489in}}%
\pgfpathlineto{\pgfqpoint{2.642514in}{2.062718in}}%
\pgfpathlineto{\pgfqpoint{2.664881in}{2.062718in}}%
\pgfpathlineto{\pgfqpoint{2.664881in}{2.012058in}}%
\pgfpathlineto{\pgfqpoint{2.687247in}{2.012058in}}%
\pgfpathlineto{\pgfqpoint{2.687247in}{1.721260in}}%
\pgfpathlineto{\pgfqpoint{2.709614in}{1.721260in}}%
\pgfpathlineto{\pgfqpoint{2.709614in}{1.593578in}}%
\pgfpathlineto{\pgfqpoint{2.731981in}{1.593578in}}%
\pgfpathlineto{\pgfqpoint{2.731981in}{1.519563in}}%
\pgfpathlineto{\pgfqpoint{2.754347in}{1.519563in}}%
\pgfpathlineto{\pgfqpoint{2.754347in}{1.446827in}}%
\pgfpathlineto{\pgfqpoint{2.776714in}{1.446827in}}%
\pgfpathlineto{\pgfqpoint{2.776714in}{1.471692in}}%
\pgfpathlineto{\pgfqpoint{2.799080in}{1.471692in}}%
\pgfpathlineto{\pgfqpoint{2.799080in}{1.518703in}}%
\pgfpathlineto{\pgfqpoint{2.821447in}{1.518703in}}%
\pgfpathlineto{\pgfqpoint{2.821447in}{1.595729in}}%
\pgfpathlineto{\pgfqpoint{2.843813in}{1.595729in}}%
\pgfpathlineto{\pgfqpoint{2.843813in}{1.624964in}}%
\pgfpathlineto{\pgfqpoint{2.866180in}{1.624964in}}%
\pgfpathlineto{\pgfqpoint{2.866180in}{1.503693in}}%
\pgfpathlineto{\pgfqpoint{2.888546in}{1.503693in}}%
\pgfpathlineto{\pgfqpoint{2.888546in}{1.537683in}}%
\pgfpathlineto{\pgfqpoint{2.910913in}{1.537683in}}%
\pgfpathlineto{\pgfqpoint{2.910913in}{1.657519in}}%
\pgfpathlineto{\pgfqpoint{2.933279in}{1.657519in}}%
\pgfpathlineto{\pgfqpoint{2.933279in}{1.770854in}}%
\pgfpathlineto{\pgfqpoint{2.955646in}{1.770854in}}%
\pgfpathlineto{\pgfqpoint{2.955646in}{1.657959in}}%
\pgfpathlineto{\pgfqpoint{2.978012in}{1.657959in}}%
\pgfpathlineto{\pgfqpoint{2.978012in}{1.476293in}}%
\pgfpathlineto{\pgfqpoint{3.000379in}{1.476293in}}%
\pgfpathlineto{\pgfqpoint{3.000379in}{1.622567in}}%
\pgfpathlineto{\pgfqpoint{3.022745in}{1.622567in}}%
\pgfpathlineto{\pgfqpoint{3.022745in}{1.858093in}}%
\pgfpathlineto{\pgfqpoint{3.045112in}{1.858093in}}%
\pgfpathlineto{\pgfqpoint{3.045112in}{2.070187in}}%
\pgfpathlineto{\pgfqpoint{3.067478in}{2.070187in}}%
\pgfpathlineto{\pgfqpoint{3.067478in}{2.110904in}}%
\pgfpathlineto{\pgfqpoint{3.089845in}{2.110904in}}%
\pgfpathlineto{\pgfqpoint{3.089845in}{2.079712in}}%
\pgfpathlineto{\pgfqpoint{3.112211in}{2.079712in}}%
\pgfpathlineto{\pgfqpoint{3.112211in}{2.023613in}}%
\pgfpathlineto{\pgfqpoint{3.134578in}{2.023613in}}%
\pgfpathlineto{\pgfqpoint{3.134578in}{2.061392in}}%
\pgfpathlineto{\pgfqpoint{3.156944in}{2.061392in}}%
\pgfpathlineto{\pgfqpoint{3.156944in}{2.197707in}}%
\pgfpathlineto{\pgfqpoint{3.179311in}{2.197707in}}%
\pgfpathlineto{\pgfqpoint{3.179311in}{2.345383in}}%
\pgfpathlineto{\pgfqpoint{3.201677in}{2.345383in}}%
\pgfpathlineto{\pgfqpoint{3.201677in}{2.655528in}}%
\pgfpathlineto{\pgfqpoint{3.224044in}{2.655528in}}%
\pgfpathlineto{\pgfqpoint{3.224044in}{2.939749in}}%
\pgfpathlineto{\pgfqpoint{3.246411in}{2.939749in}}%
\pgfpathlineto{\pgfqpoint{3.246411in}{2.995062in}}%
\pgfpathlineto{\pgfqpoint{3.268777in}{2.995062in}}%
\pgfpathlineto{\pgfqpoint{3.268777in}{2.850281in}}%
\pgfpathlineto{\pgfqpoint{3.291144in}{2.850281in}}%
\pgfpathlineto{\pgfqpoint{3.291144in}{2.726310in}}%
\pgfpathlineto{\pgfqpoint{3.313510in}{2.726310in}}%
\pgfpathlineto{\pgfqpoint{3.313510in}{2.707283in}}%
\pgfpathlineto{\pgfqpoint{3.335877in}{2.707283in}}%
\pgfpathlineto{\pgfqpoint{3.335877in}{2.667446in}}%
\pgfpathlineto{\pgfqpoint{3.358243in}{2.667446in}}%
\pgfpathlineto{\pgfqpoint{3.358243in}{2.794307in}}%
\pgfpathlineto{\pgfqpoint{3.380610in}{2.794307in}}%
\pgfpathlineto{\pgfqpoint{3.380610in}{3.038542in}}%
\pgfpathlineto{\pgfqpoint{3.402976in}{3.038542in}}%
\pgfpathlineto{\pgfqpoint{3.402976in}{2.966235in}}%
\pgfpathlineto{\pgfqpoint{3.425343in}{2.966235in}}%
\pgfpathlineto{\pgfqpoint{3.425343in}{2.749982in}}%
\pgfpathlineto{\pgfqpoint{3.447709in}{2.749982in}}%
\pgfpathlineto{\pgfqpoint{3.447709in}{2.488174in}}%
\pgfpathlineto{\pgfqpoint{3.470076in}{2.488174in}}%
\pgfpathlineto{\pgfqpoint{3.470076in}{2.476029in}}%
\pgfpathlineto{\pgfqpoint{3.492442in}{2.476029in}}%
\pgfpathlineto{\pgfqpoint{3.492442in}{2.671480in}}%
\pgfpathlineto{\pgfqpoint{3.514809in}{2.671480in}}%
\pgfpathlineto{\pgfqpoint{3.514809in}{2.796016in}}%
\pgfpathlineto{\pgfqpoint{3.537175in}{2.796016in}}%
\pgfpathlineto{\pgfqpoint{3.537175in}{3.030074in}}%
\pgfpathlineto{\pgfqpoint{3.559542in}{3.030074in}}%
\pgfpathlineto{\pgfqpoint{3.559542in}{3.036443in}}%
\pgfpathlineto{\pgfqpoint{3.581908in}{3.036443in}}%
\pgfpathlineto{\pgfqpoint{3.581908in}{2.704764in}}%
\pgfpathlineto{\pgfqpoint{3.604275in}{2.704764in}}%
\pgfpathlineto{\pgfqpoint{3.604275in}{2.459241in}}%
\pgfpathlineto{\pgfqpoint{3.626641in}{2.459241in}}%
\pgfpathlineto{\pgfqpoint{3.626641in}{2.328464in}}%
\pgfpathlineto{\pgfqpoint{3.649008in}{2.328464in}}%
\pgfpathlineto{\pgfqpoint{3.649008in}{1.969379in}}%
\pgfpathlineto{\pgfqpoint{3.671374in}{1.969379in}}%
\pgfpathlineto{\pgfqpoint{3.671374in}{1.821972in}}%
\pgfpathlineto{\pgfqpoint{3.693741in}{1.821972in}}%
\pgfpathlineto{\pgfqpoint{3.693741in}{1.828324in}}%
\pgfpathlineto{\pgfqpoint{3.716108in}{1.828324in}}%
\pgfpathlineto{\pgfqpoint{3.716108in}{1.630574in}}%
\pgfpathlineto{\pgfqpoint{3.738474in}{1.630574in}}%
\pgfpathlineto{\pgfqpoint{3.738474in}{1.432419in}}%
\pgfpathlineto{\pgfqpoint{3.760841in}{1.432419in}}%
\pgfpathlineto{\pgfqpoint{3.760841in}{1.176141in}}%
\pgfpathlineto{\pgfqpoint{3.783207in}{1.176141in}}%
\pgfpathlineto{\pgfqpoint{3.783207in}{0.923626in}}%
\pgfpathlineto{\pgfqpoint{3.805574in}{0.923626in}}%
\pgfpathlineto{\pgfqpoint{3.805574in}{0.717661in}}%
\pgfpathlineto{\pgfqpoint{3.827940in}{0.717661in}}%
\pgfpathlineto{\pgfqpoint{3.827940in}{0.580000in}}%
\pgfpathlineto{\pgfqpoint{3.850307in}{0.580000in}}%
\pgfpathlineto{\pgfqpoint{3.850307in}{0.652722in}}%
\pgfpathlineto{\pgfqpoint{3.872673in}{0.652722in}}%
\pgfpathlineto{\pgfqpoint{3.872673in}{0.778918in}}%
\pgfpathlineto{\pgfqpoint{3.895040in}{0.778918in}}%
\pgfpathlineto{\pgfqpoint{3.895040in}{0.884488in}}%
\pgfpathlineto{\pgfqpoint{3.917406in}{0.884488in}}%
\pgfpathlineto{\pgfqpoint{3.917406in}{0.967680in}}%
\pgfpathlineto{\pgfqpoint{3.939773in}{0.967680in}}%
\pgfpathlineto{\pgfqpoint{3.939773in}{0.982952in}}%
\pgfpathlineto{\pgfqpoint{3.962139in}{0.982952in}}%
\pgfpathlineto{\pgfqpoint{3.962139in}{1.243037in}}%
\pgfpathlineto{\pgfqpoint{3.984506in}{1.243037in}}%
\pgfpathlineto{\pgfqpoint{3.984506in}{1.429357in}}%
\pgfpathlineto{\pgfqpoint{4.006872in}{1.429357in}}%
\pgfpathlineto{\pgfqpoint{4.006872in}{1.522711in}}%
\pgfpathlineto{\pgfqpoint{4.029239in}{1.522711in}}%
\pgfpathlineto{\pgfqpoint{4.029239in}{1.596793in}}%
\pgfpathlineto{\pgfqpoint{4.051605in}{1.596793in}}%
\pgfpathlineto{\pgfqpoint{4.051605in}{1.392497in}}%
\pgfpathlineto{\pgfqpoint{4.073972in}{1.392497in}}%
\pgfpathlineto{\pgfqpoint{4.073972in}{1.197288in}}%
\pgfpathlineto{\pgfqpoint{4.096338in}{1.197288in}}%
\pgfpathlineto{\pgfqpoint{4.096338in}{1.186204in}}%
\pgfpathlineto{\pgfqpoint{4.118705in}{1.186204in}}%
\pgfpathlineto{\pgfqpoint{4.118705in}{1.524415in}}%
\pgfpathlineto{\pgfqpoint{4.141071in}{1.524415in}}%
\pgfpathlineto{\pgfqpoint{4.141071in}{1.983388in}}%
\pgfpathlineto{\pgfqpoint{4.163438in}{1.983388in}}%
\pgfpathlineto{\pgfqpoint{4.163438in}{2.188373in}}%
\pgfpathlineto{\pgfqpoint{4.185804in}{2.188373in}}%
\pgfpathlineto{\pgfqpoint{4.185804in}{2.066593in}}%
\pgfpathlineto{\pgfqpoint{4.208171in}{2.066593in}}%
\pgfpathlineto{\pgfqpoint{4.208171in}{1.902309in}}%
\pgfpathlineto{\pgfqpoint{4.230538in}{1.902309in}}%
\pgfpathlineto{\pgfqpoint{4.230538in}{1.724023in}}%
\pgfpathlineto{\pgfqpoint{4.252904in}{1.724023in}}%
\pgfpathlineto{\pgfqpoint{4.252904in}{1.613265in}}%
\pgfpathlineto{\pgfqpoint{4.275271in}{1.613265in}}%
\pgfpathlineto{\pgfqpoint{4.275271in}{1.760736in}}%
\pgfpathlineto{\pgfqpoint{4.297637in}{1.760736in}}%
\pgfpathlineto{\pgfqpoint{4.297637in}{1.894372in}}%
\pgfpathlineto{\pgfqpoint{4.320004in}{1.894372in}}%
\pgfpathlineto{\pgfqpoint{4.320004in}{2.179590in}}%
\pgfpathlineto{\pgfqpoint{4.342370in}{2.179590in}}%
\pgfpathlineto{\pgfqpoint{4.342370in}{2.632555in}}%
\pgfpathlineto{\pgfqpoint{4.364737in}{2.632555in}}%
\pgfpathlineto{\pgfqpoint{4.364737in}{2.843803in}}%
\pgfpathlineto{\pgfqpoint{4.387103in}{2.843803in}}%
\pgfpathlineto{\pgfqpoint{4.387103in}{2.790534in}}%
\pgfpathlineto{\pgfqpoint{4.409470in}{2.790534in}}%
\pgfpathlineto{\pgfqpoint{4.409470in}{2.702782in}}%
\pgfpathlineto{\pgfqpoint{4.431836in}{2.702782in}}%
\pgfpathlineto{\pgfqpoint{4.431836in}{2.609916in}}%
\pgfpathlineto{\pgfqpoint{4.454203in}{2.609916in}}%
\pgfpathlineto{\pgfqpoint{4.454203in}{2.585518in}}%
\pgfpathlineto{\pgfqpoint{4.476569in}{2.585518in}}%
\pgfpathlineto{\pgfqpoint{4.476569in}{2.503995in}}%
\pgfpathlineto{\pgfqpoint{4.498936in}{2.503995in}}%
\pgfpathlineto{\pgfqpoint{4.498936in}{2.370865in}}%
\pgfpathlineto{\pgfqpoint{4.521302in}{2.370865in}}%
\pgfpathlineto{\pgfqpoint{4.521302in}{2.208628in}}%
\pgfpathlineto{\pgfqpoint{4.543669in}{2.208628in}}%
\pgfpathlineto{\pgfqpoint{4.543669in}{2.105463in}}%
\pgfpathlineto{\pgfqpoint{4.566035in}{2.105463in}}%
\pgfpathlineto{\pgfqpoint{4.566035in}{1.995566in}}%
\pgfpathlineto{\pgfqpoint{4.588402in}{1.995566in}}%
\pgfpathlineto{\pgfqpoint{4.588402in}{1.953952in}}%
\pgfpathlineto{\pgfqpoint{4.610768in}{1.953952in}}%
\pgfpathlineto{\pgfqpoint{4.610768in}{2.272984in}}%
\pgfpathlineto{\pgfqpoint{4.633135in}{2.272984in}}%
\pgfpathlineto{\pgfqpoint{4.633135in}{2.490249in}}%
\pgfpathlineto{\pgfqpoint{4.655501in}{2.490249in}}%
\pgfpathlineto{\pgfqpoint{4.655501in}{2.591152in}}%
\pgfpathlineto{\pgfqpoint{4.677868in}{2.591152in}}%
\pgfpathlineto{\pgfqpoint{4.677868in}{2.464620in}}%
\pgfpathlineto{\pgfqpoint{4.700234in}{2.464620in}}%
\pgfpathlineto{\pgfqpoint{4.700234in}{2.047797in}}%
\pgfpathlineto{\pgfqpoint{4.722601in}{2.047797in}}%
\pgfpathlineto{\pgfqpoint{4.722601in}{1.707453in}}%
\pgfpathlineto{\pgfqpoint{4.744968in}{1.707453in}}%
\pgfpathlineto{\pgfqpoint{4.744968in}{1.740183in}}%
\pgfpathlineto{\pgfqpoint{4.767334in}{1.740183in}}%
\pgfpathlineto{\pgfqpoint{4.767334in}{1.931578in}}%
\pgfpathlineto{\pgfqpoint{4.789701in}{1.931578in}}%
\pgfpathlineto{\pgfqpoint{4.789701in}{2.030160in}}%
\pgfpathlineto{\pgfqpoint{4.812067in}{2.030160in}}%
\pgfpathlineto{\pgfqpoint{4.812067in}{2.163025in}}%
\pgfpathlineto{\pgfqpoint{4.834434in}{2.163025in}}%
\pgfpathlineto{\pgfqpoint{4.834434in}{2.201612in}}%
\pgfpathlineto{\pgfqpoint{4.856800in}{2.201612in}}%
\pgfpathlineto{\pgfqpoint{4.856800in}{2.165345in}}%
\pgfpathlineto{\pgfqpoint{4.879167in}{2.165345in}}%
\pgfpathlineto{\pgfqpoint{4.879167in}{2.186260in}}%
\pgfpathlineto{\pgfqpoint{4.901533in}{2.186260in}}%
\pgfpathlineto{\pgfqpoint{4.901533in}{2.288747in}}%
\pgfpathlineto{\pgfqpoint{4.923900in}{2.288747in}}%
\pgfpathlineto{\pgfqpoint{4.923900in}{2.354617in}}%
\pgfpathlineto{\pgfqpoint{4.946266in}{2.354617in}}%
\pgfpathlineto{\pgfqpoint{4.946266in}{2.318984in}}%
\pgfpathlineto{\pgfqpoint{4.968633in}{2.318984in}}%
\pgfpathlineto{\pgfqpoint{4.968633in}{2.120344in}}%
\pgfpathlineto{\pgfqpoint{4.990999in}{2.120344in}}%
\pgfpathlineto{\pgfqpoint{4.990999in}{1.977933in}}%
\pgfpathlineto{\pgfqpoint{5.013366in}{1.977933in}}%
\pgfpathlineto{\pgfqpoint{5.013366in}{1.896426in}}%
\pgfpathlineto{\pgfqpoint{5.035732in}{1.896426in}}%
\pgfpathlineto{\pgfqpoint{5.035732in}{1.852990in}}%
\pgfpathlineto{\pgfqpoint{5.058099in}{1.852990in}}%
\pgfpathlineto{\pgfqpoint{5.058099in}{1.898183in}}%
\pgfpathlineto{\pgfqpoint{5.080465in}{1.898183in}}%
\pgfpathlineto{\pgfqpoint{5.080465in}{2.012342in}}%
\pgfpathlineto{\pgfqpoint{5.102832in}{2.012342in}}%
\pgfpathlineto{\pgfqpoint{5.102832in}{2.149772in}}%
\pgfpathlineto{\pgfqpoint{5.125198in}{2.149772in}}%
\pgfpathlineto{\pgfqpoint{5.125198in}{2.027736in}}%
\pgfpathlineto{\pgfqpoint{5.147565in}{2.027736in}}%
\pgfpathlineto{\pgfqpoint{5.147565in}{1.726498in}}%
\pgfpathlineto{\pgfqpoint{5.169931in}{1.726498in}}%
\pgfpathlineto{\pgfqpoint{5.169931in}{1.331941in}}%
\pgfpathlineto{\pgfqpoint{5.192298in}{1.331941in}}%
\pgfpathlineto{\pgfqpoint{5.192298in}{1.100421in}}%
\pgfpathlineto{\pgfqpoint{5.214665in}{1.100421in}}%
\pgfpathlineto{\pgfqpoint{5.214665in}{1.093527in}}%
\pgfpathlineto{\pgfqpoint{5.237031in}{1.093527in}}%
\pgfpathlineto{\pgfqpoint{5.237031in}{1.259635in}}%
\pgfpathlineto{\pgfqpoint{5.259398in}{1.259635in}}%
\pgfpathlineto{\pgfqpoint{5.259398in}{1.627939in}}%
\pgfpathlineto{\pgfqpoint{5.281764in}{1.627939in}}%
\pgfpathlineto{\pgfqpoint{5.281764in}{1.926206in}}%
\pgfpathlineto{\pgfqpoint{5.304131in}{1.926206in}}%
\pgfpathlineto{\pgfqpoint{5.304131in}{1.990775in}}%
\pgfpathlineto{\pgfqpoint{5.326497in}{1.990775in}}%
\pgfpathlineto{\pgfqpoint{5.326497in}{1.889110in}}%
\pgfpathlineto{\pgfqpoint{5.348864in}{1.889110in}}%
\pgfpathlineto{\pgfqpoint{5.348864in}{1.799930in}}%
\pgfpathlineto{\pgfqpoint{5.371230in}{1.799930in}}%
\pgfpathlineto{\pgfqpoint{5.371230in}{1.711618in}}%
\pgfpathlineto{\pgfqpoint{5.393597in}{1.711618in}}%
\pgfpathlineto{\pgfqpoint{5.393597in}{1.743163in}}%
\pgfpathlineto{\pgfqpoint{5.415963in}{1.743163in}}%
\pgfpathlineto{\pgfqpoint{5.415963in}{1.921925in}}%
\pgfpathlineto{\pgfqpoint{5.438330in}{1.921925in}}%
\pgfpathlineto{\pgfqpoint{5.438330in}{2.149133in}}%
\pgfpathlineto{\pgfqpoint{5.460696in}{2.149133in}}%
\pgfpathlineto{\pgfqpoint{5.460696in}{2.349884in}}%
\pgfpathlineto{\pgfqpoint{5.483063in}{2.349884in}}%
\pgfpathlineto{\pgfqpoint{5.483063in}{2.321403in}}%
\pgfpathlineto{\pgfqpoint{5.505429in}{2.321403in}}%
\pgfpathlineto{\pgfqpoint{5.505429in}{2.293351in}}%
\pgfpathlineto{\pgfqpoint{5.527796in}{2.293351in}}%
\pgfpathlineto{\pgfqpoint{5.527796in}{2.490788in}}%
\pgfpathlineto{\pgfqpoint{5.550162in}{2.490788in}}%
\pgfpathlineto{\pgfqpoint{5.550162in}{2.580157in}}%
\pgfpathlineto{\pgfqpoint{5.572529in}{2.580157in}}%
\pgfpathlineto{\pgfqpoint{5.572529in}{2.623500in}}%
\pgfpathlineto{\pgfqpoint{5.594895in}{2.623500in}}%
\pgfpathlineto{\pgfqpoint{5.594895in}{2.561411in}}%
\pgfpathlineto{\pgfqpoint{5.617262in}{2.561411in}}%
\pgfpathlineto{\pgfqpoint{5.617262in}{2.336427in}}%
\pgfpathlineto{\pgfqpoint{5.639628in}{2.336427in}}%
\pgfpathlineto{\pgfqpoint{5.639628in}{2.006344in}}%
\pgfpathlineto{\pgfqpoint{5.661995in}{2.006344in}}%
\pgfpathlineto{\pgfqpoint{5.661995in}{1.489450in}}%
\pgfpathlineto{\pgfqpoint{5.684361in}{1.489450in}}%
\pgfpathlineto{\pgfqpoint{5.684361in}{1.216123in}}%
\pgfpathlineto{\pgfqpoint{5.706728in}{1.216123in}}%
\pgfpathlineto{\pgfqpoint{5.706728in}{1.183774in}}%
\pgfpathlineto{\pgfqpoint{5.729095in}{1.183774in}}%
\pgfpathlineto{\pgfqpoint{5.729095in}{1.538189in}}%
\pgfpathlineto{\pgfqpoint{5.751461in}{1.538189in}}%
\pgfpathlineto{\pgfqpoint{5.751461in}{1.986205in}}%
\pgfpathlineto{\pgfqpoint{5.773828in}{1.986205in}}%
\pgfpathlineto{\pgfqpoint{5.773828in}{2.315928in}}%
\pgfpathlineto{\pgfqpoint{5.796194in}{2.315928in}}%
\pgfpathlineto{\pgfqpoint{5.796194in}{2.471752in}}%
\pgfpathlineto{\pgfqpoint{5.818561in}{2.471752in}}%
\pgfpathlineto{\pgfqpoint{5.818561in}{2.151395in}}%
\pgfpathlineto{\pgfqpoint{5.840927in}{2.151395in}}%
\pgfpathlineto{\pgfqpoint{5.840927in}{1.860001in}}%
\pgfpathlineto{\pgfqpoint{5.863294in}{1.860001in}}%
\pgfpathlineto{\pgfqpoint{5.863294in}{1.773168in}}%
\pgfpathlineto{\pgfqpoint{5.885660in}{1.773168in}}%
\pgfpathlineto{\pgfqpoint{5.885660in}{1.749116in}}%
\pgfpathlineto{\pgfqpoint{5.908027in}{1.749116in}}%
\pgfpathlineto{\pgfqpoint{5.908027in}{1.597052in}}%
\pgfpathlineto{\pgfqpoint{5.930393in}{1.597052in}}%
\pgfpathlineto{\pgfqpoint{5.930393in}{1.642989in}}%
\pgfpathlineto{\pgfqpoint{5.952760in}{1.642989in}}%
\pgfpathlineto{\pgfqpoint{5.952760in}{1.837673in}}%
\pgfpathlineto{\pgfqpoint{5.975126in}{1.837673in}}%
\pgfpathlineto{\pgfqpoint{5.975126in}{1.840384in}}%
\pgfpathlineto{\pgfqpoint{5.997493in}{1.840384in}}%
\pgfpathlineto{\pgfqpoint{5.997493in}{1.867273in}}%
\pgfpathlineto{\pgfqpoint{6.019859in}{1.867273in}}%
\pgfpathlineto{\pgfqpoint{6.019859in}{1.911252in}}%
\pgfpathlineto{\pgfqpoint{6.042226in}{1.911252in}}%
\pgfpathlineto{\pgfqpoint{6.042226in}{1.980000in}}%
\pgfpathlineto{\pgfqpoint{6.042226in}{1.980000in}}%
\pgfusepath{stroke}%
\end{pgfscope}%
\begin{pgfscope}%
\pgfpathrectangle{\pgfqpoint{0.875000in}{0.440000in}}{\pgfqpoint{5.425000in}{3.080000in}}%
\pgfusepath{clip}%
\pgfsetbuttcap%
\pgfsetroundjoin%
\pgfsetlinewidth{1.505625pt}%
\definecolor{currentstroke}{rgb}{1.000000,0.000000,0.000000}%
\pgfsetstrokecolor{currentstroke}%
\pgfsetstrokeopacity{0.900000}%
\pgfsetdash{{5.550000pt}{2.400000pt}}{0.000000pt}%
\pgfpathmoveto{\pgfqpoint{1.121591in}{0.580000in}}%
\pgfpathlineto{\pgfqpoint{1.121591in}{3.380000in}}%
\pgfusepath{stroke}%
\end{pgfscope}%
\begin{pgfscope}%
\pgfpathrectangle{\pgfqpoint{0.875000in}{0.440000in}}{\pgfqpoint{5.425000in}{3.080000in}}%
\pgfusepath{clip}%
\pgfsetbuttcap%
\pgfsetroundjoin%
\pgfsetlinewidth{1.505625pt}%
\definecolor{currentstroke}{rgb}{1.000000,0.000000,0.000000}%
\pgfsetstrokecolor{currentstroke}%
\pgfsetstrokeopacity{0.900000}%
\pgfsetdash{{5.550000pt}{2.400000pt}}{0.000000pt}%
\pgfpathmoveto{\pgfqpoint{1.680754in}{0.580000in}}%
\pgfpathlineto{\pgfqpoint{1.680754in}{3.380000in}}%
\pgfusepath{stroke}%
\end{pgfscope}%
\begin{pgfscope}%
\pgfpathrectangle{\pgfqpoint{0.875000in}{0.440000in}}{\pgfqpoint{5.425000in}{3.080000in}}%
\pgfusepath{clip}%
\pgfsetbuttcap%
\pgfsetroundjoin%
\pgfsetlinewidth{1.505625pt}%
\definecolor{currentstroke}{rgb}{1.000000,0.000000,0.000000}%
\pgfsetstrokecolor{currentstroke}%
\pgfsetstrokeopacity{0.900000}%
\pgfsetdash{{5.550000pt}{2.400000pt}}{0.000000pt}%
\pgfpathmoveto{\pgfqpoint{2.239917in}{0.580000in}}%
\pgfpathlineto{\pgfqpoint{2.239917in}{3.380000in}}%
\pgfusepath{stroke}%
\end{pgfscope}%
\begin{pgfscope}%
\pgfpathrectangle{\pgfqpoint{0.875000in}{0.440000in}}{\pgfqpoint{5.425000in}{3.080000in}}%
\pgfusepath{clip}%
\pgfsetbuttcap%
\pgfsetroundjoin%
\pgfsetlinewidth{1.505625pt}%
\definecolor{currentstroke}{rgb}{1.000000,0.000000,0.000000}%
\pgfsetstrokecolor{currentstroke}%
\pgfsetstrokeopacity{0.900000}%
\pgfsetdash{{5.550000pt}{2.400000pt}}{0.000000pt}%
\pgfpathmoveto{\pgfqpoint{2.799080in}{0.580000in}}%
\pgfpathlineto{\pgfqpoint{2.799080in}{3.380000in}}%
\pgfusepath{stroke}%
\end{pgfscope}%
\begin{pgfscope}%
\pgfpathrectangle{\pgfqpoint{0.875000in}{0.440000in}}{\pgfqpoint{5.425000in}{3.080000in}}%
\pgfusepath{clip}%
\pgfsetbuttcap%
\pgfsetroundjoin%
\pgfsetlinewidth{1.505625pt}%
\definecolor{currentstroke}{rgb}{1.000000,0.000000,0.000000}%
\pgfsetstrokecolor{currentstroke}%
\pgfsetstrokeopacity{0.900000}%
\pgfsetdash{{5.550000pt}{2.400000pt}}{0.000000pt}%
\pgfpathmoveto{\pgfqpoint{3.358243in}{0.580000in}}%
\pgfpathlineto{\pgfqpoint{3.358243in}{3.380000in}}%
\pgfusepath{stroke}%
\end{pgfscope}%
\begin{pgfscope}%
\pgfpathrectangle{\pgfqpoint{0.875000in}{0.440000in}}{\pgfqpoint{5.425000in}{3.080000in}}%
\pgfusepath{clip}%
\pgfsetbuttcap%
\pgfsetroundjoin%
\pgfsetlinewidth{1.505625pt}%
\definecolor{currentstroke}{rgb}{1.000000,0.000000,0.000000}%
\pgfsetstrokecolor{currentstroke}%
\pgfsetstrokeopacity{0.900000}%
\pgfsetdash{{5.550000pt}{2.400000pt}}{0.000000pt}%
\pgfpathmoveto{\pgfqpoint{3.917406in}{0.580000in}}%
\pgfpathlineto{\pgfqpoint{3.917406in}{3.380000in}}%
\pgfusepath{stroke}%
\end{pgfscope}%
\begin{pgfscope}%
\pgfpathrectangle{\pgfqpoint{0.875000in}{0.440000in}}{\pgfqpoint{5.425000in}{3.080000in}}%
\pgfusepath{clip}%
\pgfsetbuttcap%
\pgfsetroundjoin%
\pgfsetlinewidth{1.505625pt}%
\definecolor{currentstroke}{rgb}{1.000000,0.000000,0.000000}%
\pgfsetstrokecolor{currentstroke}%
\pgfsetstrokeopacity{0.900000}%
\pgfsetdash{{5.550000pt}{2.400000pt}}{0.000000pt}%
\pgfpathmoveto{\pgfqpoint{4.476569in}{0.580000in}}%
\pgfpathlineto{\pgfqpoint{4.476569in}{3.380000in}}%
\pgfusepath{stroke}%
\end{pgfscope}%
\begin{pgfscope}%
\pgfpathrectangle{\pgfqpoint{0.875000in}{0.440000in}}{\pgfqpoint{5.425000in}{3.080000in}}%
\pgfusepath{clip}%
\pgfsetbuttcap%
\pgfsetroundjoin%
\pgfsetlinewidth{1.505625pt}%
\definecolor{currentstroke}{rgb}{1.000000,0.000000,0.000000}%
\pgfsetstrokecolor{currentstroke}%
\pgfsetstrokeopacity{0.900000}%
\pgfsetdash{{5.550000pt}{2.400000pt}}{0.000000pt}%
\pgfpathmoveto{\pgfqpoint{5.035732in}{0.580000in}}%
\pgfpathlineto{\pgfqpoint{5.035732in}{3.380000in}}%
\pgfusepath{stroke}%
\end{pgfscope}%
\begin{pgfscope}%
\pgfpathrectangle{\pgfqpoint{0.875000in}{0.440000in}}{\pgfqpoint{5.425000in}{3.080000in}}%
\pgfusepath{clip}%
\pgfsetbuttcap%
\pgfsetroundjoin%
\pgfsetlinewidth{1.505625pt}%
\definecolor{currentstroke}{rgb}{1.000000,0.000000,0.000000}%
\pgfsetstrokecolor{currentstroke}%
\pgfsetstrokeopacity{0.900000}%
\pgfsetdash{{5.550000pt}{2.400000pt}}{0.000000pt}%
\pgfpathmoveto{\pgfqpoint{5.594895in}{0.580000in}}%
\pgfpathlineto{\pgfqpoint{5.594895in}{3.380000in}}%
\pgfusepath{stroke}%
\end{pgfscope}%
\begin{pgfscope}%
\pgfpathrectangle{\pgfqpoint{0.875000in}{0.440000in}}{\pgfqpoint{5.425000in}{3.080000in}}%
\pgfusepath{clip}%
\pgfsetbuttcap%
\pgfsetroundjoin%
\pgfsetlinewidth{1.505625pt}%
\definecolor{currentstroke}{rgb}{0.000000,0.000000,0.000000}%
\pgfsetstrokecolor{currentstroke}%
\pgfsetstrokeopacity{0.200000}%
\pgfsetdash{}{0pt}%
\pgfpathmoveto{\pgfqpoint{1.121591in}{1.980000in}}%
\pgfpathlineto{\pgfqpoint{6.053409in}{1.980000in}}%
\pgfusepath{stroke}%
\end{pgfscope}%
\begin{pgfscope}%
\pgfpathrectangle{\pgfqpoint{0.875000in}{0.440000in}}{\pgfqpoint{5.425000in}{3.080000in}}%
\pgfusepath{clip}%
\pgfsetbuttcap%
\pgfsetroundjoin%
\pgfsetlinewidth{1.505625pt}%
\definecolor{currentstroke}{rgb}{0.000000,0.000000,1.000000}%
\pgfsetstrokecolor{currentstroke}%
\pgfsetdash{}{0pt}%
\pgfpathmoveto{\pgfqpoint{1.121591in}{1.980000in}}%
\pgfpathlineto{\pgfqpoint{1.121591in}{1.980000in}}%
\pgfusepath{stroke}%
\end{pgfscope}%
\begin{pgfscope}%
\pgfpathrectangle{\pgfqpoint{0.875000in}{0.440000in}}{\pgfqpoint{5.425000in}{3.080000in}}%
\pgfusepath{clip}%
\pgfsetbuttcap%
\pgfsetroundjoin%
\pgfsetlinewidth{1.505625pt}%
\definecolor{currentstroke}{rgb}{0.000000,0.000000,1.000000}%
\pgfsetstrokecolor{currentstroke}%
\pgfsetdash{}{0pt}%
\pgfpathmoveto{\pgfqpoint{1.300523in}{1.980000in}}%
\pgfpathlineto{\pgfqpoint{1.300523in}{2.693096in}}%
\pgfusepath{stroke}%
\end{pgfscope}%
\begin{pgfscope}%
\pgfpathrectangle{\pgfqpoint{0.875000in}{0.440000in}}{\pgfqpoint{5.425000in}{3.080000in}}%
\pgfusepath{clip}%
\pgfsetbuttcap%
\pgfsetroundjoin%
\pgfsetlinewidth{1.505625pt}%
\definecolor{currentstroke}{rgb}{0.000000,0.000000,1.000000}%
\pgfsetstrokecolor{currentstroke}%
\pgfsetdash{}{0pt}%
\pgfpathmoveto{\pgfqpoint{2.150451in}{1.980000in}}%
\pgfpathlineto{\pgfqpoint{2.150451in}{3.018638in}}%
\pgfusepath{stroke}%
\end{pgfscope}%
\begin{pgfscope}%
\pgfpathrectangle{\pgfqpoint{0.875000in}{0.440000in}}{\pgfqpoint{5.425000in}{3.080000in}}%
\pgfusepath{clip}%
\pgfsetbuttcap%
\pgfsetroundjoin%
\pgfsetlinewidth{1.505625pt}%
\definecolor{currentstroke}{rgb}{0.000000,0.000000,1.000000}%
\pgfsetstrokecolor{currentstroke}%
\pgfsetdash{}{0pt}%
\pgfpathmoveto{\pgfqpoint{2.239917in}{1.980000in}}%
\pgfpathlineto{\pgfqpoint{2.239917in}{2.226243in}}%
\pgfusepath{stroke}%
\end{pgfscope}%
\begin{pgfscope}%
\pgfpathrectangle{\pgfqpoint{0.875000in}{0.440000in}}{\pgfqpoint{5.425000in}{3.080000in}}%
\pgfusepath{clip}%
\pgfsetbuttcap%
\pgfsetroundjoin%
\pgfsetlinewidth{1.505625pt}%
\definecolor{currentstroke}{rgb}{0.000000,0.000000,1.000000}%
\pgfsetstrokecolor{currentstroke}%
\pgfsetdash{}{0pt}%
\pgfpathmoveto{\pgfqpoint{3.246411in}{1.980000in}}%
\pgfpathlineto{\pgfqpoint{3.246411in}{2.995062in}}%
\pgfusepath{stroke}%
\end{pgfscope}%
\begin{pgfscope}%
\pgfpathrectangle{\pgfqpoint{0.875000in}{0.440000in}}{\pgfqpoint{5.425000in}{3.080000in}}%
\pgfusepath{clip}%
\pgfsetbuttcap%
\pgfsetroundjoin%
\pgfsetlinewidth{1.505625pt}%
\definecolor{currentstroke}{rgb}{0.000000,0.000000,1.000000}%
\pgfsetstrokecolor{currentstroke}%
\pgfsetdash{}{0pt}%
\pgfpathmoveto{\pgfqpoint{3.380610in}{1.980000in}}%
\pgfpathlineto{\pgfqpoint{3.380610in}{3.038542in}}%
\pgfusepath{stroke}%
\end{pgfscope}%
\begin{pgfscope}%
\pgfpathrectangle{\pgfqpoint{0.875000in}{0.440000in}}{\pgfqpoint{5.425000in}{3.080000in}}%
\pgfusepath{clip}%
\pgfsetbuttcap%
\pgfsetroundjoin%
\pgfsetlinewidth{1.505625pt}%
\definecolor{currentstroke}{rgb}{0.000000,0.000000,1.000000}%
\pgfsetstrokecolor{currentstroke}%
\pgfsetdash{}{0pt}%
\pgfpathmoveto{\pgfqpoint{4.364737in}{1.980000in}}%
\pgfpathlineto{\pgfqpoint{4.364737in}{2.843803in}}%
\pgfusepath{stroke}%
\end{pgfscope}%
\begin{pgfscope}%
\pgfpathrectangle{\pgfqpoint{0.875000in}{0.440000in}}{\pgfqpoint{5.425000in}{3.080000in}}%
\pgfusepath{clip}%
\pgfsetbuttcap%
\pgfsetroundjoin%
\pgfsetlinewidth{1.505625pt}%
\definecolor{currentstroke}{rgb}{0.000000,0.000000,1.000000}%
\pgfsetstrokecolor{currentstroke}%
\pgfsetdash{}{0pt}%
\pgfpathmoveto{\pgfqpoint{4.655501in}{1.980000in}}%
\pgfpathlineto{\pgfqpoint{4.655501in}{2.591152in}}%
\pgfusepath{stroke}%
\end{pgfscope}%
\begin{pgfscope}%
\pgfpathrectangle{\pgfqpoint{0.875000in}{0.440000in}}{\pgfqpoint{5.425000in}{3.080000in}}%
\pgfusepath{clip}%
\pgfsetbuttcap%
\pgfsetroundjoin%
\pgfsetlinewidth{1.505625pt}%
\definecolor{currentstroke}{rgb}{0.000000,0.000000,1.000000}%
\pgfsetstrokecolor{currentstroke}%
\pgfsetdash{}{0pt}%
\pgfpathmoveto{\pgfqpoint{5.572529in}{1.980000in}}%
\pgfpathlineto{\pgfqpoint{5.572529in}{2.623500in}}%
\pgfusepath{stroke}%
\end{pgfscope}%
\begin{pgfscope}%
\pgfpathrectangle{\pgfqpoint{0.875000in}{0.440000in}}{\pgfqpoint{5.425000in}{3.080000in}}%
\pgfusepath{clip}%
\pgfsetbuttcap%
\pgfsetroundjoin%
\pgfsetlinewidth{1.505625pt}%
\definecolor{currentstroke}{rgb}{0.000000,0.000000,1.000000}%
\pgfsetstrokecolor{currentstroke}%
\pgfsetdash{}{0pt}%
\pgfpathmoveto{\pgfqpoint{5.594895in}{1.980000in}}%
\pgfpathlineto{\pgfqpoint{5.594895in}{2.561411in}}%
\pgfusepath{stroke}%
\end{pgfscope}%
\begin{pgfscope}%
\pgfpathrectangle{\pgfqpoint{0.875000in}{0.440000in}}{\pgfqpoint{5.425000in}{3.080000in}}%
\pgfusepath{clip}%
\pgfsetbuttcap%
\pgfsetroundjoin%
\pgfsetlinewidth{1.505625pt}%
\definecolor{currentstroke}{rgb}{0.000000,0.000000,0.000000}%
\pgfsetstrokecolor{currentstroke}%
\pgfsetdash{}{0pt}%
\pgfpathmoveto{\pgfqpoint{1.121591in}{1.980000in}}%
\pgfpathlineto{\pgfqpoint{1.121591in}{1.980000in}}%
\pgfusepath{stroke}%
\end{pgfscope}%
\begin{pgfscope}%
\pgfpathrectangle{\pgfqpoint{0.875000in}{0.440000in}}{\pgfqpoint{5.425000in}{3.080000in}}%
\pgfusepath{clip}%
\pgfsetbuttcap%
\pgfsetroundjoin%
\pgfsetlinewidth{1.505625pt}%
\definecolor{currentstroke}{rgb}{0.000000,0.000000,0.000000}%
\pgfsetstrokecolor{currentstroke}%
\pgfsetdash{}{0pt}%
\pgfpathmoveto{\pgfqpoint{1.434722in}{1.980000in}}%
\pgfpathlineto{\pgfqpoint{1.434722in}{1.364084in}}%
\pgfusepath{stroke}%
\end{pgfscope}%
\begin{pgfscope}%
\pgfpathrectangle{\pgfqpoint{0.875000in}{0.440000in}}{\pgfqpoint{5.425000in}{3.080000in}}%
\pgfusepath{clip}%
\pgfsetbuttcap%
\pgfsetroundjoin%
\pgfsetlinewidth{1.505625pt}%
\definecolor{currentstroke}{rgb}{0.000000,0.000000,0.000000}%
\pgfsetstrokecolor{currentstroke}%
\pgfsetdash{}{0pt}%
\pgfpathmoveto{\pgfqpoint{1.859686in}{1.980000in}}%
\pgfpathlineto{\pgfqpoint{1.859686in}{1.495126in}}%
\pgfusepath{stroke}%
\end{pgfscope}%
\begin{pgfscope}%
\pgfpathrectangle{\pgfqpoint{0.875000in}{0.440000in}}{\pgfqpoint{5.425000in}{3.080000in}}%
\pgfusepath{clip}%
\pgfsetbuttcap%
\pgfsetroundjoin%
\pgfsetlinewidth{1.505625pt}%
\definecolor{currentstroke}{rgb}{0.000000,0.000000,0.000000}%
\pgfsetstrokecolor{currentstroke}%
\pgfsetdash{}{0pt}%
\pgfpathmoveto{\pgfqpoint{2.329383in}{1.980000in}}%
\pgfpathlineto{\pgfqpoint{2.329383in}{1.226944in}}%
\pgfusepath{stroke}%
\end{pgfscope}%
\begin{pgfscope}%
\pgfpathrectangle{\pgfqpoint{0.875000in}{0.440000in}}{\pgfqpoint{5.425000in}{3.080000in}}%
\pgfusepath{clip}%
\pgfsetbuttcap%
\pgfsetroundjoin%
\pgfsetlinewidth{1.505625pt}%
\definecolor{currentstroke}{rgb}{0.000000,0.000000,0.000000}%
\pgfsetstrokecolor{currentstroke}%
\pgfsetdash{}{0pt}%
\pgfpathmoveto{\pgfqpoint{2.978012in}{1.980000in}}%
\pgfpathlineto{\pgfqpoint{2.978012in}{1.476293in}}%
\pgfusepath{stroke}%
\end{pgfscope}%
\begin{pgfscope}%
\pgfpathrectangle{\pgfqpoint{0.875000in}{0.440000in}}{\pgfqpoint{5.425000in}{3.080000in}}%
\pgfusepath{clip}%
\pgfsetbuttcap%
\pgfsetroundjoin%
\pgfsetlinewidth{1.505625pt}%
\definecolor{currentstroke}{rgb}{0.000000,0.000000,0.000000}%
\pgfsetstrokecolor{currentstroke}%
\pgfsetdash{}{0pt}%
\pgfpathmoveto{\pgfqpoint{3.827940in}{1.980000in}}%
\pgfpathlineto{\pgfqpoint{3.827940in}{0.580000in}}%
\pgfusepath{stroke}%
\end{pgfscope}%
\begin{pgfscope}%
\pgfpathrectangle{\pgfqpoint{0.875000in}{0.440000in}}{\pgfqpoint{5.425000in}{3.080000in}}%
\pgfusepath{clip}%
\pgfsetbuttcap%
\pgfsetroundjoin%
\pgfsetlinewidth{1.505625pt}%
\definecolor{currentstroke}{rgb}{0.000000,0.000000,0.000000}%
\pgfsetstrokecolor{currentstroke}%
\pgfsetdash{}{0pt}%
\pgfpathmoveto{\pgfqpoint{3.917406in}{1.980000in}}%
\pgfpathlineto{\pgfqpoint{3.917406in}{0.967680in}}%
\pgfusepath{stroke}%
\end{pgfscope}%
\begin{pgfscope}%
\pgfpathrectangle{\pgfqpoint{0.875000in}{0.440000in}}{\pgfqpoint{5.425000in}{3.080000in}}%
\pgfusepath{clip}%
\pgfsetbuttcap%
\pgfsetroundjoin%
\pgfsetlinewidth{1.505625pt}%
\definecolor{currentstroke}{rgb}{0.000000,0.000000,0.000000}%
\pgfsetstrokecolor{currentstroke}%
\pgfsetdash{}{0pt}%
\pgfpathmoveto{\pgfqpoint{4.722601in}{1.980000in}}%
\pgfpathlineto{\pgfqpoint{4.722601in}{1.707453in}}%
\pgfusepath{stroke}%
\end{pgfscope}%
\begin{pgfscope}%
\pgfpathrectangle{\pgfqpoint{0.875000in}{0.440000in}}{\pgfqpoint{5.425000in}{3.080000in}}%
\pgfusepath{clip}%
\pgfsetbuttcap%
\pgfsetroundjoin%
\pgfsetlinewidth{1.505625pt}%
\definecolor{currentstroke}{rgb}{0.000000,0.000000,0.000000}%
\pgfsetstrokecolor{currentstroke}%
\pgfsetdash{}{0pt}%
\pgfpathmoveto{\pgfqpoint{5.214665in}{1.980000in}}%
\pgfpathlineto{\pgfqpoint{5.214665in}{1.093527in}}%
\pgfusepath{stroke}%
\end{pgfscope}%
\begin{pgfscope}%
\pgfpathrectangle{\pgfqpoint{0.875000in}{0.440000in}}{\pgfqpoint{5.425000in}{3.080000in}}%
\pgfusepath{clip}%
\pgfsetbuttcap%
\pgfsetroundjoin%
\pgfsetlinewidth{1.505625pt}%
\definecolor{currentstroke}{rgb}{0.000000,0.000000,0.000000}%
\pgfsetstrokecolor{currentstroke}%
\pgfsetdash{}{0pt}%
\pgfpathmoveto{\pgfqpoint{5.706728in}{1.980000in}}%
\pgfpathlineto{\pgfqpoint{5.706728in}{1.183774in}}%
\pgfusepath{stroke}%
\end{pgfscope}%
\begin{pgfscope}%
\pgfsetrectcap%
\pgfsetmiterjoin%
\pgfsetlinewidth{0.803000pt}%
\definecolor{currentstroke}{rgb}{0.000000,0.000000,0.000000}%
\pgfsetstrokecolor{currentstroke}%
\pgfsetdash{}{0pt}%
\pgfpathmoveto{\pgfqpoint{0.875000in}{0.440000in}}%
\pgfpathlineto{\pgfqpoint{0.875000in}{3.520000in}}%
\pgfusepath{stroke}%
\end{pgfscope}%
\begin{pgfscope}%
\pgfsetrectcap%
\pgfsetmiterjoin%
\pgfsetlinewidth{0.803000pt}%
\definecolor{currentstroke}{rgb}{0.000000,0.000000,0.000000}%
\pgfsetstrokecolor{currentstroke}%
\pgfsetdash{}{0pt}%
\pgfpathmoveto{\pgfqpoint{6.300000in}{0.440000in}}%
\pgfpathlineto{\pgfqpoint{6.300000in}{3.520000in}}%
\pgfusepath{stroke}%
\end{pgfscope}%
\begin{pgfscope}%
\pgfsetrectcap%
\pgfsetmiterjoin%
\pgfsetlinewidth{0.803000pt}%
\definecolor{currentstroke}{rgb}{0.000000,0.000000,0.000000}%
\pgfsetstrokecolor{currentstroke}%
\pgfsetdash{}{0pt}%
\pgfpathmoveto{\pgfqpoint{0.875000in}{0.440000in}}%
\pgfpathlineto{\pgfqpoint{6.300000in}{0.440000in}}%
\pgfusepath{stroke}%
\end{pgfscope}%
\begin{pgfscope}%
\pgfsetrectcap%
\pgfsetmiterjoin%
\pgfsetlinewidth{0.803000pt}%
\definecolor{currentstroke}{rgb}{0.000000,0.000000,0.000000}%
\pgfsetstrokecolor{currentstroke}%
\pgfsetdash{}{0pt}%
\pgfpathmoveto{\pgfqpoint{0.875000in}{3.520000in}}%
\pgfpathlineto{\pgfqpoint{6.300000in}{3.520000in}}%
\pgfusepath{stroke}%
\end{pgfscope}%
\begin{pgfscope}%
\definecolor{textcolor}{rgb}{0.000000,0.000000,0.000000}%
\pgfsetstrokecolor{textcolor}%
\pgfsetfillcolor{textcolor}%
\pgftext[x=1.121591in,y=1.980000in,left,base]{\color{textcolor}\sffamily\fontsize{10.000000}{12.000000}\selectfont 0.0}%
\end{pgfscope}%
\begin{pgfscope}%
\definecolor{textcolor}{rgb}{0.000000,0.000000,0.000000}%
\pgfsetstrokecolor{textcolor}%
\pgfsetfillcolor{textcolor}%
\pgftext[x=1.300523in,y=2.693096in,left,base]{\color{textcolor}\sffamily\fontsize{10.000000}{12.000000}\selectfont 0.07}%
\end{pgfscope}%
\begin{pgfscope}%
\definecolor{textcolor}{rgb}{0.000000,0.000000,0.000000}%
\pgfsetstrokecolor{textcolor}%
\pgfsetfillcolor{textcolor}%
\pgftext[x=2.150451in,y=3.018638in,left,base]{\color{textcolor}\sffamily\fontsize{10.000000}{12.000000}\selectfont 0.1}%
\end{pgfscope}%
\begin{pgfscope}%
\definecolor{textcolor}{rgb}{0.000000,0.000000,0.000000}%
\pgfsetstrokecolor{textcolor}%
\pgfsetfillcolor{textcolor}%
\pgftext[x=2.239917in,y=2.226243in,left,base]{\color{textcolor}\sffamily\fontsize{10.000000}{12.000000}\selectfont 0.02}%
\end{pgfscope}%
\begin{pgfscope}%
\definecolor{textcolor}{rgb}{0.000000,0.000000,0.000000}%
\pgfsetstrokecolor{textcolor}%
\pgfsetfillcolor{textcolor}%
\pgftext[x=3.246411in,y=2.995062in,left,base]{\color{textcolor}\sffamily\fontsize{10.000000}{12.000000}\selectfont 0.1}%
\end{pgfscope}%
\begin{pgfscope}%
\definecolor{textcolor}{rgb}{0.000000,0.000000,0.000000}%
\pgfsetstrokecolor{textcolor}%
\pgfsetfillcolor{textcolor}%
\pgftext[x=3.380610in,y=3.038542in,left,base]{\color{textcolor}\sffamily\fontsize{10.000000}{12.000000}\selectfont 0.11}%
\end{pgfscope}%
\begin{pgfscope}%
\definecolor{textcolor}{rgb}{0.000000,0.000000,0.000000}%
\pgfsetstrokecolor{textcolor}%
\pgfsetfillcolor{textcolor}%
\pgftext[x=4.364737in,y=2.843803in,left,base]{\color{textcolor}\sffamily\fontsize{10.000000}{12.000000}\selectfont 0.09}%
\end{pgfscope}%
\begin{pgfscope}%
\definecolor{textcolor}{rgb}{0.000000,0.000000,0.000000}%
\pgfsetstrokecolor{textcolor}%
\pgfsetfillcolor{textcolor}%
\pgftext[x=4.655501in,y=2.591152in,left,base]{\color{textcolor}\sffamily\fontsize{10.000000}{12.000000}\selectfont 0.06}%
\end{pgfscope}%
\begin{pgfscope}%
\definecolor{textcolor}{rgb}{0.000000,0.000000,0.000000}%
\pgfsetstrokecolor{textcolor}%
\pgfsetfillcolor{textcolor}%
\pgftext[x=5.572529in,y=2.623500in,left,base]{\color{textcolor}\sffamily\fontsize{10.000000}{12.000000}\selectfont 0.06}%
\end{pgfscope}%
\begin{pgfscope}%
\definecolor{textcolor}{rgb}{0.000000,0.000000,0.000000}%
\pgfsetstrokecolor{textcolor}%
\pgfsetfillcolor{textcolor}%
\pgftext[x=5.594895in,y=2.561411in,left,base]{\color{textcolor}\sffamily\fontsize{10.000000}{12.000000}\selectfont 0.06}%
\end{pgfscope}%
\begin{pgfscope}%
\definecolor{textcolor}{rgb}{0.000000,0.000000,0.000000}%
\pgfsetstrokecolor{textcolor}%
\pgfsetfillcolor{textcolor}%
\pgftext[x=1.121591in,y=1.980000in,left,base]{\color{textcolor}\sffamily\fontsize{10.000000}{12.000000}\selectfont 0.0}%
\end{pgfscope}%
\begin{pgfscope}%
\definecolor{textcolor}{rgb}{0.000000,0.000000,0.000000}%
\pgfsetstrokecolor{textcolor}%
\pgfsetfillcolor{textcolor}%
\pgftext[x=1.434722in,y=1.364084in,left,base]{\color{textcolor}\sffamily\fontsize{10.000000}{12.000000}\selectfont -0.06}%
\end{pgfscope}%
\begin{pgfscope}%
\definecolor{textcolor}{rgb}{0.000000,0.000000,0.000000}%
\pgfsetstrokecolor{textcolor}%
\pgfsetfillcolor{textcolor}%
\pgftext[x=1.859686in,y=1.495126in,left,base]{\color{textcolor}\sffamily\fontsize{10.000000}{12.000000}\selectfont -0.05}%
\end{pgfscope}%
\begin{pgfscope}%
\definecolor{textcolor}{rgb}{0.000000,0.000000,0.000000}%
\pgfsetstrokecolor{textcolor}%
\pgfsetfillcolor{textcolor}%
\pgftext[x=2.329383in,y=1.226944in,left,base]{\color{textcolor}\sffamily\fontsize{10.000000}{12.000000}\selectfont -0.07}%
\end{pgfscope}%
\begin{pgfscope}%
\definecolor{textcolor}{rgb}{0.000000,0.000000,0.000000}%
\pgfsetstrokecolor{textcolor}%
\pgfsetfillcolor{textcolor}%
\pgftext[x=2.978012in,y=1.476293in,left,base]{\color{textcolor}\sffamily\fontsize{10.000000}{12.000000}\selectfont -0.05}%
\end{pgfscope}%
\begin{pgfscope}%
\definecolor{textcolor}{rgb}{0.000000,0.000000,0.000000}%
\pgfsetstrokecolor{textcolor}%
\pgfsetfillcolor{textcolor}%
\pgftext[x=3.827940in,y=0.580000in,left,base]{\color{textcolor}\sffamily\fontsize{10.000000}{12.000000}\selectfont -0.14}%
\end{pgfscope}%
\begin{pgfscope}%
\definecolor{textcolor}{rgb}{0.000000,0.000000,0.000000}%
\pgfsetstrokecolor{textcolor}%
\pgfsetfillcolor{textcolor}%
\pgftext[x=3.917406in,y=0.967680in,left,base]{\color{textcolor}\sffamily\fontsize{10.000000}{12.000000}\selectfont -0.1}%
\end{pgfscope}%
\begin{pgfscope}%
\definecolor{textcolor}{rgb}{0.000000,0.000000,0.000000}%
\pgfsetstrokecolor{textcolor}%
\pgfsetfillcolor{textcolor}%
\pgftext[x=4.722601in,y=1.707453in,left,base]{\color{textcolor}\sffamily\fontsize{10.000000}{12.000000}\selectfont -0.03}%
\end{pgfscope}%
\begin{pgfscope}%
\definecolor{textcolor}{rgb}{0.000000,0.000000,0.000000}%
\pgfsetstrokecolor{textcolor}%
\pgfsetfillcolor{textcolor}%
\pgftext[x=5.214665in,y=1.093527in,left,base]{\color{textcolor}\sffamily\fontsize{10.000000}{12.000000}\selectfont -0.09}%
\end{pgfscope}%
\begin{pgfscope}%
\definecolor{textcolor}{rgb}{0.000000,0.000000,0.000000}%
\pgfsetstrokecolor{textcolor}%
\pgfsetfillcolor{textcolor}%
\pgftext[x=5.706728in,y=1.183774in,left,base]{\color{textcolor}\sffamily\fontsize{10.000000}{12.000000}\selectfont -0.08}%
\end{pgfscope}%
\end{pgfpicture}%
\makeatother%
\endgroup%
}
    \end{center}
    \caption{\emph{Ampiezza picco-picco}, misurata ad intervalli regolari di 100 samples con frequenza di campionamento 22050Hz}
\end{figure}

L'\emph{ampiezza picco-picco} è il valore assoluto della differenza tra la massima e la minima ampiezza entro un dato intervallo di tempo. In questo caso, ``massima'' e ``minima'' vanno intese in senso numerico, considerando il segno, e non in valore assoluto: cioè, le ampiezze negative saranno tutte considerate più piccole delle ampiezze positive (e quindi, per esempio, un'ampiezza di -0.5 è considerata più piccola di un'ampiezza di 1; e un'ampiezza di 0.1 è considerata più grande di una di -0.5). 

Intuitivamente, l'ampiezza picco-picco calcola la distanza tra i valori estremi di ampiezza istantanea che il segnale tocca entro l'intervallo temporale. Nel caso di una forma d'onda perfettamente centrata sullo 0, essa coincide con il doppio del valore assoluto dell'ampiezza di picco. Per esempio, il segnale nell'intervallo considerato potrebbe avere ampiezza massima a 0.7 e minima a -0.7: entrambi i valori possono essere considerati ampiezza di picco, visto che il loro valore assoluto 0.7 è identico. In questo caso, l'ampiezza picco-picco, cioè il valore assoluto della differenza tra massimo e minimo è $0.7 - (-0.7) = 0.7 + 0.7 = 1.4$. 

Se però la forma d'onda non è centrata sullo 0, o se la posizione dello 0 non è nota con esattezza (per esempio perché esiste una componente di tensione continua),%
\footnote{Una componente di tensione continua provoca una traslazione sull'asse verticale della rappresentazione nel dominio del tempo.}
l'ampiezza picco-picco fornisce l'unica possibile misura affidabile. Ad esempio, se il segnale descritto sopra si trovasse traslato di -0.5 sull'asse verticale, il suo massimo sarebbe $0.7 - 0.5 = 0.2$ e il minimo $-0.7 - 0.5 = -1.2$. Questo però non modificherebbe l'ampiezza picco-picco, che sarebbe $0.2 - (-1.2) = 1.4$. L'ampiezza picco-picco, secondo la definizione data sopra, non è mai negativa. Inoltre, in un segnale bipolare (cioè, un segnale che attraversa la linea dello 0) è sempre maggiore dell'ampiezza di picco.




\subsection{Ampiezza RMS}

L'\emph{ampiezza RMS} (da \emph{root-mean-square}: vedremo tra poco perché) è una misura presa rispetto a un intervallo di tempo, e tiene conto dell'andamento dell'ampiezza nell'intervallo. Di conseguenza, può rappresentare in maniera più accurata il livello dinamico percepito, che a parità di ampiezza di picco (o picco-picco) sarà maggiore se, su scale di durata di frazioni di secondo, il segnale rimane forte più a lungo. L'ampiezza RMS non è mai maggiore dell'ampiezza di picco, e in generale --- salvo i casi di forme d'onda rettangolari e di segnali costanti --- è minore di essa.

La definizione di ampiezza RMS per un segnale campionato in ampiezza a intervalli di tempo regolari, come i segnali digitali che trattiamo abitualmente, è
\begin{equation}\label{eq:rms}
A_{\text{RMS}} = \sqrt{\frac{A_1^2+A_2^2+\ldots+A_n^2}{n}}
\end{equation}
dove $n$ è il numero totale di campioni e $A_1$ ... $A_n$ sono i valori di ampiezza istantanea dei singoli campioni entro la finestra temporale considerata. L'ampiezza RMS è quindi la radice quadrata della media dei quadrati delle ampiezze istantanee, da cui il nome. 

Se consideriamo invece un segnale continuo nel tempo, com'è il caso per segnali 
la definizione di RMS deve fare ricorso al concetto di integrale:%
\footnote{Un integrale può essere visto come la somma di infiniti campioni di durata infinitesimale: ci serve quindi a calcolare il valore totalizzato da un fenomeno continuo. Se non sei familiare con questo formalismo ignora tranquillamente i dettagli di questa formula.}
\begin{equation*}
A_{\textrm{RMS}} = \sqrt{\frac{1}{t_2-t_1}\int_{t_1}^{t_2}[A(t)]^2\textrm{d}t}
\end{equation*}
dove $t$ indica il tempo, $t_1$ e $t_2$ sono rispettivamente il tempo iniziale e finale considerati e $A(t)$ è l'ampiezza nel tempo.

L'ampiezza RMS non è mai negativa.%
\footnote{Volendo essere pignoli dovremmo dire che ogni numero reale positivo ha due radici quadrate, una positiva e una negativa: ma per il calcolo della RMS consideriamo solo quella positiva. D'altra parte, solo numeri reali non negativi hanno una radice quadrata reale: ma la formula della RMS mette sotto radice una somma di quadrati, e il quadrato di qualsiasi numero reale è non negativo. Tutto questo per dire che la RMS è ben definita.}


\section{Intensità sonora}

Dal punto di vista percettivo, più importante del concetto di pressione è quello di \emph{intensità sonora}. Per definirla correttamente abbiamo bisogno di introdurre alcune nozioni:


\subsection{Energia e lavoro}
Energia e lavoro sono due grandezze strettamente correlate, rappresentate dalle stesse unità di misura come il Joule e la caloria. Il lavoro è definito come la forza che sposta un corpo moltiplicata per lo spostamento effettuato:
\begin{equation}
W = F \cdot s
\end{equation}
L'energia può essere vista come la capacità di un corpo di svolgere un lavoro: si misura in termini del lavoro che in virtù di essa un corpo è in grado di svolgere. È una grandezza fisica fondamentale che obbedisce alla legge di conservazione: viene trasferita, non si crea e non si distrugge.%
\footnote{Questa è una semplificazione: nelle reazioni nucleari, la materia viene trasformata in energia; ed è teoricamente possibile trasformare energia in materia. Nella meccanica classica, però, l'energia è sempre conservata.}
In senso proprio, quindi, ciò che viene trasferito da un corpo vibrante alla membrana del nostro timpano è l'energia sonora, non la pressione sonora. In altri termini, il fatto che una pressione sonora si applichi al timpano è concettualmente una conseguenza del trasferimento di energia, e non viceversa.

L'energia è proporzionale al quadrato della pressione sonora e, di conseguenza, al quadrato dell'ampiezza e di tutti gli altri fenomeni direttamente proporzionali all'ampiezza:%
\footnote{La ragione di questa relazione discende dalle definizioni formali di pressione e di energia. Potresti chiederti: ma \emph{perché} è così? La fisica dovrebbe rispondere che la scienza si occupa del come, non del perché che è invece appannaggio della filosofia; ma qui puoi trovare alcuni tentativi, più o meno fantasiosi, di spiegazioni intuitive: \url{https://languagelog.ldc.upenn.edu/nll/?p=6508}. E, a dire il vero, anche il concetto di \emph{conseguenza}, impiegato in maniera un po' infingarda qualche riga sopra, sarebbe una questione filosofica più che scientifica.}
\begin{equation}
E \propto \Delta p^2 \propto \Delta A^2
\end{equation}



\subsection{Potenza}

La \emph{potenza} (misurata in Watt e indicata qui dalla lettera $P$ maiuscola, da non confondersi con la $p$ minuscola con cui indichiamo la pressione) è una misura della quantità di lavoro svolto, o di energia dispiegata, per unità di tempo: 
\begin{equation}
P = \frac{W}{t}
\end{equation}
A parità di lavoro o di energia, la potenza è maggiore se il tempo è più breve. La stessa quantità di energia sonora dispiegata in un secondo o in un'ora darà luogo a fenomeni sonori di potenza diversa, maggiore nel primo caso (perché il lavoro è concentrato in un tempo più breve), minore nel secondo. A parità di tempo, la potenza è proporzionale all'energia e quindi al quadrato dell'ampiezza:
\begin{equation}
P \propto E \propto \Delta p^2 \propto A^2
\end{equation}



\subsection{Intensità}

L'intensità sonora è una misura della potenza riferita all'area su cui questa è distribuita:
\begin{equation}
I = \frac{P}{a}
\end{equation}

L'intensità sonora non ha una sua specifica unità di misura, ma è misurata in Watt per metro quadrato ($\frac{W}{m^2}$ o $W \cdot m^{-2}$).

Possiamo assumere che il suono, in assenza di ostacoli, si propaghi su un fronte d'onda di forma corrispondente a una superficie sferica a partire dall'emittente, che consideriamo puntiforme. La superficie sferica naturalmente sarà sempre più ampia via via che il fronte d'onda si allontana dalla sorgente, secondo la formula
\begin{equation}
a = 4 \pi r^2
\end{equation}
dove $a$ è l'area della sfera e $r$ il raggio, che in questo caso corrisponde alla distanza tra l'emittente e il fronte d'onda che stiamo considerando. Di conseguenza, possiamo dire che l'energia dispiegata dal fenomeno sonoro in un certo lasso di tempo si distribuisce su una superficie la cui area aumenta in maniera proporzionale al quadrato della distanza. In altri termini, posso pensare l'intensità sonora come la potenza portata dall'onda sonora su una certa superficie, in senso perpendicolare a essa, e a una certa distanza dall'emittente. E dunque, a parità di tutti gli altri parametri, l'intensità sonora è inversamente proporzionale al quadrato della distanza:
\begin{equation}
I \propto \frac{1}{r^2}
\end{equation}

Questa relazione è chiamata \emph{legge dell'inverso del quadrato} e si applica a un vasto insieme di altri fenomeni, tra cui l'intensità luminosa e l'attrazione gravitazionale.

D'altra parte, l'intensità sonora è direttamente proporzionale alla potenza, all'energia e al quadrato della pressione e dell'ampiezza:
\begin{equation}
I \propto P \propto E \propto \Delta p^2 \propto A^2
\end{equation}

Di conseguenza,
\begin{equation}
\frac{1}{r^2} \propto \Delta p^2 \propto A^2
\end{equation}
e quindi
\begin{equation}
\frac{1}{r} \propto \Delta p \propto A
\end{equation}

Questo è un risultato molto importante: pressione, ampiezza e spostamento sono inversamente proporzionali alla distanza, non al suo quadrato. Si dice quindi che seguono la \emph{legge dell'inverso della distanza}.



\section{Riepilogo e classificazione delle grandezze}

Abbiamo quindi definito due categorie di grandezze fisiche nelle quali ci imbatteremo ancora, e sulle quali non dev'esserci confusione:

\subsubsection{Quantità proporzionali alla potenza (\emph{power quantities})}

La prima categoria include l'energia, il lavoro, la potenza e l'intensità sonora. È importante ricordare che da un punto di vista concettuale la quantità fondamentale tra queste è l'energia, che gode del principio di conservazione. L'energia, il lavoro e la potenza non variano con la distanza: la quantità di energia (e quindi di lavoro e di potenza) in gioco è quella trasferita o trasferibile dall'emittente ai corpi circostanti, e rispetto a essa la distanza non è un parametro rilevante. La distanza entra in gioco quando si considera l'intensità sonora, che segue la \emph{legge dell'inverso del quadrato}.


\subsubsection{Quantità proporzionali alla radice quadrata della potenza (\emph{root-power quantities})}

La seconda categoria include la pressione sonora e la tensione elettrica. La potenza è proporzionale al quadrato di queste grandezze: di conseguenza, queste saranno proporzionali alla radice quadrata della potenza, da cui il nome:
\begin{equation}
W \propto \Delta p^2 \implies \Delta p \propto \sqrt{W}
\end{equation}

Si tratta di grandezze a cui non si applica principio di conservazione e che variano in maniera inversamente proporzionale alla distanza. Concettualmente, possiamo considerare i fenomeni rappresentati da queste grandezze come conseguenze del trasferimento di energia. D'altra parte la pressione sonora è il fenomeno che viene direttamente misurato da una membrana di microfono o dal nostro timpano.

L'ampiezza, in senso proprio, non è una grandezza fisica perché non descrive un fenomeno specifico; possiamo però considerarla un'astrazione delle \emph{root-power quantities}, utile quando vogliamo considerare il fenomeno sonoro in maniera indipendente dal fenomeno fisico che l'ha prodotto; questo è particolarmente rilevante nel momento in cui trattiamo il suono dal punto di vista elettroacustico e, di conseguenza, numerico.



\section{Intermezzo: logaritmi --- un ripasso}

Il prossimo concetto che introdurremo richiede uno strumento matematico che dovresti già conoscere, ma che potrebbe meritare un veloce ripasso: il \emph{logaritmo}.

Il logaritmo è definito come l'operazione inversa rispetto all'elevamento a potenza: cioè, se $b^x = y$ allora $\log _b y = x$. Chiamiamo $b$ la \emph{base} del logaritmo.

I logaritmi hanno alcune proprietà molto utili, che sono facilmente dimostrabili ma che mi limiterò a enunciare:

\begin{itemize}

\item Il logaritmo di 1 in qualsiasi base è 0, perché qualsiasi numero elevato a potenza 0 dà 1.

\item Il logaritmo di qualsiasi numero maggiore di 1 è positivo in qualsiasi base; il logaritmo di qualsiasi numero maggiore di 0 e minore di 1 è negativo in qualsiasi base.

\item Il logaritmo di 0 e di numeri negativi non esiste.

\item La somma di due logaritmi di uguale base è uguale al logaritmo (nella stessa base) del prodotto: $\log _b x + \log _b y = \log _b (xy)$. Questa identità rende tra l'altro evidente il fatto che il logaritmo non è una funzione lineare. 

\item È possibile trasformare un esponente dentro il logaritmo in un prodotto fuori dal logaritmo: $\log _b x^y = y \log _b x$. Queste due ultime affermazioni possono essere intese intuitivamente pensando che il logaritmo ci permette di ``scendere di livello'' con le operazioni aritmetiche: trasforma l'elevamento a potenza in moltiplicazione e la moltiplicazione in addizione. Per lungo tempo, questa semplificazione dei calcoli è stata la prima ragione per usare i logaritmi.

\item I logaritmi in tutte le basi sono direttamente proporzionali tra loro: cioè, è possibile passare da una base all'altra applicando al logaritmo un coefficiente --- più specificatamente, $\log _b x = \frac{\log _c b}{\log _c a}$, dove $c$ è una base scelta arbitrariamente. Questa formula, detta \emph{formula del cambio di base}, ci dice tra l'altro che se conosciamo i logaritmi in una base qualsiasi possiamo calcolare logaritmi in qualsiasi altra base.

\item Il logaritmo ci dà una misura dei rapporti tra numeri: la differenza dei logaritmi di coppie di numeri in rapporto 2:1 sarà sempre la stessa (il valore esatto della differenza dipende dalla base del logaritmo). Quindi $\frac{x}{y} = \frac{z}{w} \implies \log _b x - \log _b y = \log _b z - \log _b w$. Per esempio, in base 10 un rapporto di 2:1 corrisponderà a una differenza tra i logaritmi di circa 0.301 (incontreremo di nuovo molto presto questo valore). 

\item Il logaritmo del reciproco di un numero è il negativo del logaritmo di quel numero: $log _b \frac{1}{x} = -log _b x$.

\item Il logaritmo in base 10 ci dà una misura della ``lunghezza tipografica'' di un numero intero: $log_{10}10 = 1$, $log_{10}100 = 2$, $log_{10}1000 = 3$ e così via; per argomenti intermedi avremo valori di logaritmo intermedi (ma, ovviamente, non \emph{linearmente} intermedi: $log_{10}55 \approx 1,74$, che è molto più vicino a $log_{10}100$ che a $log_{10}10$ anche se 55 è la media esatta tra 10 e 100). Generalizzando ulteriormente, il logaritmo in base $b$ ci dà una misura della ``lunghezza tipografica'' di un numero rappresentato in base di numerazione $b$.

\end{itemize}

%\begin{figure}
%    \begin{center}
%       \scalebox{0.6} {%% Creator: Matplotlib, PGF backend
%%
%% To include the figure in your LaTeX document, write
%%   \input{<filename>.pgf}
%%
%% Make sure the required packages are loaded in your preamble
%%   \usepackage{pgf}
%%
%% Also ensure that all the required font packages are loaded; for instance,
%% the lmodern package is sometimes necessary when using math font.
%%   \usepackage{lmodern}
%%
%% Figures using additional raster images can only be included by \input if
%% they are in the same directory as the main LaTeX file. For loading figures
%% from other directories you can use the `import` package
%%   \usepackage{import}
%%
%% and then include the figures with
%%   \import{<path to file>}{<filename>.pgf}
%%
%% Matplotlib used the following preamble
%%   
%%   \makeatletter\@ifpackageloaded{underscore}{}{\usepackage[strings]{underscore}}\makeatother
%%
\begingroup%
\makeatletter%
\begin{pgfpicture}%
\pgfpathrectangle{\pgfpointorigin}{\pgfqpoint{6.400000in}{4.800000in}}%
\pgfusepath{use as bounding box, clip}%
\begin{pgfscope}%
\pgfsetbuttcap%
\pgfsetmiterjoin%
\definecolor{currentfill}{rgb}{1.000000,1.000000,1.000000}%
\pgfsetfillcolor{currentfill}%
\pgfsetlinewidth{0.000000pt}%
\definecolor{currentstroke}{rgb}{1.000000,1.000000,1.000000}%
\pgfsetstrokecolor{currentstroke}%
\pgfsetdash{}{0pt}%
\pgfpathmoveto{\pgfqpoint{0.000000in}{0.000000in}}%
\pgfpathlineto{\pgfqpoint{6.400000in}{0.000000in}}%
\pgfpathlineto{\pgfqpoint{6.400000in}{4.800000in}}%
\pgfpathlineto{\pgfqpoint{0.000000in}{4.800000in}}%
\pgfpathlineto{\pgfqpoint{0.000000in}{0.000000in}}%
\pgfpathclose%
\pgfusepath{fill}%
\end{pgfscope}%
\begin{pgfscope}%
\pgfsetbuttcap%
\pgfsetmiterjoin%
\definecolor{currentfill}{rgb}{1.000000,1.000000,1.000000}%
\pgfsetfillcolor{currentfill}%
\pgfsetlinewidth{0.000000pt}%
\definecolor{currentstroke}{rgb}{0.000000,0.000000,0.000000}%
\pgfsetstrokecolor{currentstroke}%
\pgfsetstrokeopacity{0.000000}%
\pgfsetdash{}{0pt}%
\pgfpathmoveto{\pgfqpoint{0.800000in}{0.528000in}}%
\pgfpathlineto{\pgfqpoint{5.760000in}{0.528000in}}%
\pgfpathlineto{\pgfqpoint{5.760000in}{4.224000in}}%
\pgfpathlineto{\pgfqpoint{0.800000in}{4.224000in}}%
\pgfpathlineto{\pgfqpoint{0.800000in}{0.528000in}}%
\pgfpathclose%
\pgfusepath{fill}%
\end{pgfscope}%
\begin{pgfscope}%
\pgfsetbuttcap%
\pgfsetroundjoin%
\definecolor{currentfill}{rgb}{0.000000,0.000000,0.000000}%
\pgfsetfillcolor{currentfill}%
\pgfsetlinewidth{0.803000pt}%
\definecolor{currentstroke}{rgb}{0.000000,0.000000,0.000000}%
\pgfsetstrokecolor{currentstroke}%
\pgfsetdash{}{0pt}%
\pgfsys@defobject{currentmarker}{\pgfqpoint{0.000000in}{-0.048611in}}{\pgfqpoint{0.000000in}{0.000000in}}{%
\pgfpathmoveto{\pgfqpoint{0.000000in}{0.000000in}}%
\pgfpathlineto{\pgfqpoint{0.000000in}{-0.048611in}}%
\pgfusepath{stroke,fill}%
}%
\begin{pgfscope}%
\pgfsys@transformshift{1.025455in}{0.528000in}%
\pgfsys@useobject{currentmarker}{}%
\end{pgfscope}%
\end{pgfscope}%
\begin{pgfscope}%
\definecolor{textcolor}{rgb}{0.000000,0.000000,0.000000}%
\pgfsetstrokecolor{textcolor}%
\pgfsetfillcolor{textcolor}%
\pgftext[x=1.025455in,y=0.430778in,,top]{\color{textcolor}\rmfamily\fontsize{10.000000}{12.000000}\selectfont \(\displaystyle {0}\)}%
\end{pgfscope}%
\begin{pgfscope}%
\pgfsetbuttcap%
\pgfsetroundjoin%
\definecolor{currentfill}{rgb}{0.000000,0.000000,0.000000}%
\pgfsetfillcolor{currentfill}%
\pgfsetlinewidth{0.803000pt}%
\definecolor{currentstroke}{rgb}{0.000000,0.000000,0.000000}%
\pgfsetstrokecolor{currentstroke}%
\pgfsetdash{}{0pt}%
\pgfsys@defobject{currentmarker}{\pgfqpoint{0.000000in}{-0.048611in}}{\pgfqpoint{0.000000in}{0.000000in}}{%
\pgfpathmoveto{\pgfqpoint{0.000000in}{0.000000in}}%
\pgfpathlineto{\pgfqpoint{0.000000in}{-0.048611in}}%
\pgfusepath{stroke,fill}%
}%
\begin{pgfscope}%
\pgfsys@transformshift{2.027475in}{0.528000in}%
\pgfsys@useobject{currentmarker}{}%
\end{pgfscope}%
\end{pgfscope}%
\begin{pgfscope}%
\definecolor{textcolor}{rgb}{0.000000,0.000000,0.000000}%
\pgfsetstrokecolor{textcolor}%
\pgfsetfillcolor{textcolor}%
\pgftext[x=2.027475in,y=0.430778in,,top]{\color{textcolor}\rmfamily\fontsize{10.000000}{12.000000}\selectfont \(\displaystyle {2}\)}%
\end{pgfscope}%
\begin{pgfscope}%
\pgfsetbuttcap%
\pgfsetroundjoin%
\definecolor{currentfill}{rgb}{0.000000,0.000000,0.000000}%
\pgfsetfillcolor{currentfill}%
\pgfsetlinewidth{0.803000pt}%
\definecolor{currentstroke}{rgb}{0.000000,0.000000,0.000000}%
\pgfsetstrokecolor{currentstroke}%
\pgfsetdash{}{0pt}%
\pgfsys@defobject{currentmarker}{\pgfqpoint{0.000000in}{-0.048611in}}{\pgfqpoint{0.000000in}{0.000000in}}{%
\pgfpathmoveto{\pgfqpoint{0.000000in}{0.000000in}}%
\pgfpathlineto{\pgfqpoint{0.000000in}{-0.048611in}}%
\pgfusepath{stroke,fill}%
}%
\begin{pgfscope}%
\pgfsys@transformshift{3.029495in}{0.528000in}%
\pgfsys@useobject{currentmarker}{}%
\end{pgfscope}%
\end{pgfscope}%
\begin{pgfscope}%
\definecolor{textcolor}{rgb}{0.000000,0.000000,0.000000}%
\pgfsetstrokecolor{textcolor}%
\pgfsetfillcolor{textcolor}%
\pgftext[x=3.029495in,y=0.430778in,,top]{\color{textcolor}\rmfamily\fontsize{10.000000}{12.000000}\selectfont \(\displaystyle {4}\)}%
\end{pgfscope}%
\begin{pgfscope}%
\pgfsetbuttcap%
\pgfsetroundjoin%
\definecolor{currentfill}{rgb}{0.000000,0.000000,0.000000}%
\pgfsetfillcolor{currentfill}%
\pgfsetlinewidth{0.803000pt}%
\definecolor{currentstroke}{rgb}{0.000000,0.000000,0.000000}%
\pgfsetstrokecolor{currentstroke}%
\pgfsetdash{}{0pt}%
\pgfsys@defobject{currentmarker}{\pgfqpoint{0.000000in}{-0.048611in}}{\pgfqpoint{0.000000in}{0.000000in}}{%
\pgfpathmoveto{\pgfqpoint{0.000000in}{0.000000in}}%
\pgfpathlineto{\pgfqpoint{0.000000in}{-0.048611in}}%
\pgfusepath{stroke,fill}%
}%
\begin{pgfscope}%
\pgfsys@transformshift{4.031515in}{0.528000in}%
\pgfsys@useobject{currentmarker}{}%
\end{pgfscope}%
\end{pgfscope}%
\begin{pgfscope}%
\definecolor{textcolor}{rgb}{0.000000,0.000000,0.000000}%
\pgfsetstrokecolor{textcolor}%
\pgfsetfillcolor{textcolor}%
\pgftext[x=4.031515in,y=0.430778in,,top]{\color{textcolor}\rmfamily\fontsize{10.000000}{12.000000}\selectfont \(\displaystyle {6}\)}%
\end{pgfscope}%
\begin{pgfscope}%
\pgfsetbuttcap%
\pgfsetroundjoin%
\definecolor{currentfill}{rgb}{0.000000,0.000000,0.000000}%
\pgfsetfillcolor{currentfill}%
\pgfsetlinewidth{0.803000pt}%
\definecolor{currentstroke}{rgb}{0.000000,0.000000,0.000000}%
\pgfsetstrokecolor{currentstroke}%
\pgfsetdash{}{0pt}%
\pgfsys@defobject{currentmarker}{\pgfqpoint{0.000000in}{-0.048611in}}{\pgfqpoint{0.000000in}{0.000000in}}{%
\pgfpathmoveto{\pgfqpoint{0.000000in}{0.000000in}}%
\pgfpathlineto{\pgfqpoint{0.000000in}{-0.048611in}}%
\pgfusepath{stroke,fill}%
}%
\begin{pgfscope}%
\pgfsys@transformshift{5.033535in}{0.528000in}%
\pgfsys@useobject{currentmarker}{}%
\end{pgfscope}%
\end{pgfscope}%
\begin{pgfscope}%
\definecolor{textcolor}{rgb}{0.000000,0.000000,0.000000}%
\pgfsetstrokecolor{textcolor}%
\pgfsetfillcolor{textcolor}%
\pgftext[x=5.033535in,y=0.430778in,,top]{\color{textcolor}\rmfamily\fontsize{10.000000}{12.000000}\selectfont \(\displaystyle {8}\)}%
\end{pgfscope}%
\begin{pgfscope}%
\pgfsetbuttcap%
\pgfsetroundjoin%
\definecolor{currentfill}{rgb}{0.000000,0.000000,0.000000}%
\pgfsetfillcolor{currentfill}%
\pgfsetlinewidth{0.803000pt}%
\definecolor{currentstroke}{rgb}{0.000000,0.000000,0.000000}%
\pgfsetstrokecolor{currentstroke}%
\pgfsetdash{}{0pt}%
\pgfsys@defobject{currentmarker}{\pgfqpoint{-0.048611in}{0.000000in}}{\pgfqpoint{-0.000000in}{0.000000in}}{%
\pgfpathmoveto{\pgfqpoint{-0.000000in}{0.000000in}}%
\pgfpathlineto{\pgfqpoint{-0.048611in}{0.000000in}}%
\pgfusepath{stroke,fill}%
}%
\begin{pgfscope}%
\pgfsys@transformshift{0.800000in}{0.696000in}%
\pgfsys@useobject{currentmarker}{}%
\end{pgfscope}%
\end{pgfscope}%
\begin{pgfscope}%
\definecolor{textcolor}{rgb}{0.000000,0.000000,0.000000}%
\pgfsetstrokecolor{textcolor}%
\pgfsetfillcolor{textcolor}%
\pgftext[x=0.525308in, y=0.647775in, left, base]{\color{textcolor}\rmfamily\fontsize{10.000000}{12.000000}\selectfont \(\displaystyle {0.0}\)}%
\end{pgfscope}%
\begin{pgfscope}%
\pgfsetbuttcap%
\pgfsetroundjoin%
\definecolor{currentfill}{rgb}{0.000000,0.000000,0.000000}%
\pgfsetfillcolor{currentfill}%
\pgfsetlinewidth{0.803000pt}%
\definecolor{currentstroke}{rgb}{0.000000,0.000000,0.000000}%
\pgfsetstrokecolor{currentstroke}%
\pgfsetdash{}{0pt}%
\pgfsys@defobject{currentmarker}{\pgfqpoint{-0.048611in}{0.000000in}}{\pgfqpoint{-0.000000in}{0.000000in}}{%
\pgfpathmoveto{\pgfqpoint{-0.000000in}{0.000000in}}%
\pgfpathlineto{\pgfqpoint{-0.048611in}{0.000000in}}%
\pgfusepath{stroke,fill}%
}%
\begin{pgfscope}%
\pgfsys@transformshift{0.800000in}{1.425615in}%
\pgfsys@useobject{currentmarker}{}%
\end{pgfscope}%
\end{pgfscope}%
\begin{pgfscope}%
\definecolor{textcolor}{rgb}{0.000000,0.000000,0.000000}%
\pgfsetstrokecolor{textcolor}%
\pgfsetfillcolor{textcolor}%
\pgftext[x=0.525308in, y=1.377389in, left, base]{\color{textcolor}\rmfamily\fontsize{10.000000}{12.000000}\selectfont \(\displaystyle {0.5}\)}%
\end{pgfscope}%
\begin{pgfscope}%
\pgfsetbuttcap%
\pgfsetroundjoin%
\definecolor{currentfill}{rgb}{0.000000,0.000000,0.000000}%
\pgfsetfillcolor{currentfill}%
\pgfsetlinewidth{0.803000pt}%
\definecolor{currentstroke}{rgb}{0.000000,0.000000,0.000000}%
\pgfsetstrokecolor{currentstroke}%
\pgfsetdash{}{0pt}%
\pgfsys@defobject{currentmarker}{\pgfqpoint{-0.048611in}{0.000000in}}{\pgfqpoint{-0.000000in}{0.000000in}}{%
\pgfpathmoveto{\pgfqpoint{-0.000000in}{0.000000in}}%
\pgfpathlineto{\pgfqpoint{-0.048611in}{0.000000in}}%
\pgfusepath{stroke,fill}%
}%
\begin{pgfscope}%
\pgfsys@transformshift{0.800000in}{2.155229in}%
\pgfsys@useobject{currentmarker}{}%
\end{pgfscope}%
\end{pgfscope}%
\begin{pgfscope}%
\definecolor{textcolor}{rgb}{0.000000,0.000000,0.000000}%
\pgfsetstrokecolor{textcolor}%
\pgfsetfillcolor{textcolor}%
\pgftext[x=0.525308in, y=2.107004in, left, base]{\color{textcolor}\rmfamily\fontsize{10.000000}{12.000000}\selectfont \(\displaystyle {1.0}\)}%
\end{pgfscope}%
\begin{pgfscope}%
\pgfsetbuttcap%
\pgfsetroundjoin%
\definecolor{currentfill}{rgb}{0.000000,0.000000,0.000000}%
\pgfsetfillcolor{currentfill}%
\pgfsetlinewidth{0.803000pt}%
\definecolor{currentstroke}{rgb}{0.000000,0.000000,0.000000}%
\pgfsetstrokecolor{currentstroke}%
\pgfsetdash{}{0pt}%
\pgfsys@defobject{currentmarker}{\pgfqpoint{-0.048611in}{0.000000in}}{\pgfqpoint{-0.000000in}{0.000000in}}{%
\pgfpathmoveto{\pgfqpoint{-0.000000in}{0.000000in}}%
\pgfpathlineto{\pgfqpoint{-0.048611in}{0.000000in}}%
\pgfusepath{stroke,fill}%
}%
\begin{pgfscope}%
\pgfsys@transformshift{0.800000in}{2.884844in}%
\pgfsys@useobject{currentmarker}{}%
\end{pgfscope}%
\end{pgfscope}%
\begin{pgfscope}%
\definecolor{textcolor}{rgb}{0.000000,0.000000,0.000000}%
\pgfsetstrokecolor{textcolor}%
\pgfsetfillcolor{textcolor}%
\pgftext[x=0.525308in, y=2.836619in, left, base]{\color{textcolor}\rmfamily\fontsize{10.000000}{12.000000}\selectfont \(\displaystyle {1.5}\)}%
\end{pgfscope}%
\begin{pgfscope}%
\pgfsetbuttcap%
\pgfsetroundjoin%
\definecolor{currentfill}{rgb}{0.000000,0.000000,0.000000}%
\pgfsetfillcolor{currentfill}%
\pgfsetlinewidth{0.803000pt}%
\definecolor{currentstroke}{rgb}{0.000000,0.000000,0.000000}%
\pgfsetstrokecolor{currentstroke}%
\pgfsetdash{}{0pt}%
\pgfsys@defobject{currentmarker}{\pgfqpoint{-0.048611in}{0.000000in}}{\pgfqpoint{-0.000000in}{0.000000in}}{%
\pgfpathmoveto{\pgfqpoint{-0.000000in}{0.000000in}}%
\pgfpathlineto{\pgfqpoint{-0.048611in}{0.000000in}}%
\pgfusepath{stroke,fill}%
}%
\begin{pgfscope}%
\pgfsys@transformshift{0.800000in}{3.614459in}%
\pgfsys@useobject{currentmarker}{}%
\end{pgfscope}%
\end{pgfscope}%
\begin{pgfscope}%
\definecolor{textcolor}{rgb}{0.000000,0.000000,0.000000}%
\pgfsetstrokecolor{textcolor}%
\pgfsetfillcolor{textcolor}%
\pgftext[x=0.525308in, y=3.566234in, left, base]{\color{textcolor}\rmfamily\fontsize{10.000000}{12.000000}\selectfont \(\displaystyle {2.0}\)}%
\end{pgfscope}%
\begin{pgfscope}%
\pgfpathrectangle{\pgfqpoint{0.800000in}{0.528000in}}{\pgfqpoint{4.960000in}{3.696000in}}%
\pgfusepath{clip}%
\pgfsetrectcap%
\pgfsetroundjoin%
\pgfsetlinewidth{1.505625pt}%
\definecolor{currentstroke}{rgb}{0.121569,0.466667,0.705882}%
\pgfsetstrokecolor{currentstroke}%
\pgfsetdash{}{0pt}%
\pgfpathmoveto{\pgfqpoint{1.025455in}{0.696000in}}%
\pgfpathlineto{\pgfqpoint{1.526465in}{1.707461in}}%
\pgfpathlineto{\pgfqpoint{2.027475in}{2.299127in}}%
\pgfpathlineto{\pgfqpoint{2.528485in}{2.718922in}}%
\pgfpathlineto{\pgfqpoint{3.029495in}{3.044539in}}%
\pgfpathlineto{\pgfqpoint{3.530505in}{3.310588in}}%
\pgfpathlineto{\pgfqpoint{4.031515in}{3.535529in}}%
\pgfpathlineto{\pgfqpoint{4.532525in}{3.730382in}}%
\pgfpathlineto{\pgfqpoint{5.033535in}{3.902255in}}%
\pgfpathlineto{\pgfqpoint{5.534545in}{4.056000in}}%
\pgfusepath{stroke}%
\end{pgfscope}%
\begin{pgfscope}%
\pgfpathrectangle{\pgfqpoint{0.800000in}{0.528000in}}{\pgfqpoint{4.960000in}{3.696000in}}%
\pgfusepath{clip}%
\pgfsetrectcap%
\pgfsetroundjoin%
\pgfsetlinewidth{1.505625pt}%
\definecolor{currentstroke}{rgb}{1.000000,0.498039,0.054902}%
\pgfsetstrokecolor{currentstroke}%
\pgfsetdash{}{0pt}%
\pgfpathmoveto{\pgfqpoint{1.025455in}{0.696000in}}%
\pgfpathlineto{\pgfqpoint{1.526465in}{1.135272in}}%
\pgfpathlineto{\pgfqpoint{2.027475in}{1.392229in}}%
\pgfpathlineto{\pgfqpoint{2.528485in}{1.574544in}}%
\pgfpathlineto{\pgfqpoint{3.029495in}{1.715958in}}%
\pgfpathlineto{\pgfqpoint{3.530505in}{1.831501in}}%
\pgfpathlineto{\pgfqpoint{4.031515in}{1.929192in}}%
\pgfpathlineto{\pgfqpoint{4.532525in}{2.013816in}}%
\pgfpathlineto{\pgfqpoint{5.033535in}{2.088459in}}%
\pgfpathlineto{\pgfqpoint{5.534545in}{2.155229in}}%
\pgfusepath{stroke}%
\end{pgfscope}%
\begin{pgfscope}%
\pgfsetrectcap%
\pgfsetmiterjoin%
\pgfsetlinewidth{0.803000pt}%
\definecolor{currentstroke}{rgb}{0.000000,0.000000,0.000000}%
\pgfsetstrokecolor{currentstroke}%
\pgfsetdash{}{0pt}%
\pgfpathmoveto{\pgfqpoint{0.800000in}{0.528000in}}%
\pgfpathlineto{\pgfqpoint{0.800000in}{4.224000in}}%
\pgfusepath{stroke}%
\end{pgfscope}%
\begin{pgfscope}%
\pgfsetrectcap%
\pgfsetmiterjoin%
\pgfsetlinewidth{0.803000pt}%
\definecolor{currentstroke}{rgb}{0.000000,0.000000,0.000000}%
\pgfsetstrokecolor{currentstroke}%
\pgfsetdash{}{0pt}%
\pgfpathmoveto{\pgfqpoint{5.760000in}{0.528000in}}%
\pgfpathlineto{\pgfqpoint{5.760000in}{4.224000in}}%
\pgfusepath{stroke}%
\end{pgfscope}%
\begin{pgfscope}%
\pgfsetrectcap%
\pgfsetmiterjoin%
\pgfsetlinewidth{0.803000pt}%
\definecolor{currentstroke}{rgb}{0.000000,0.000000,0.000000}%
\pgfsetstrokecolor{currentstroke}%
\pgfsetdash{}{0pt}%
\pgfpathmoveto{\pgfqpoint{0.800000in}{0.528000in}}%
\pgfpathlineto{\pgfqpoint{5.760000in}{0.528000in}}%
\pgfusepath{stroke}%
\end{pgfscope}%
\begin{pgfscope}%
\pgfsetrectcap%
\pgfsetmiterjoin%
\pgfsetlinewidth{0.803000pt}%
\definecolor{currentstroke}{rgb}{0.000000,0.000000,0.000000}%
\pgfsetstrokecolor{currentstroke}%
\pgfsetdash{}{0pt}%
\pgfpathmoveto{\pgfqpoint{0.800000in}{4.224000in}}%
\pgfpathlineto{\pgfqpoint{5.760000in}{4.224000in}}%
\pgfusepath{stroke}%
\end{pgfscope}%
\end{pgfpicture}%
\makeatother%
\endgroup%
}
%    \end{center}
%    \caption{\emph{Log in base 10} e \emph{Log naturale}.}
%\end{figure}


\section{Decibel}

Le pressioni (o, dovremmo dire, i $\Delta p$) che possiamo percepire come fenomeni sonori --- abbastanza forti da essere udite, non così forti da danneggiare le nostre orecchie --- si situano approssimativamente in un ambito compreso tra i \qty{20}{\micro\pascal} e i \qty{20}{\Pa}. La nostra scala percettiva della dinamica, però, è tutt'altro che lineare rispetto all'ambito di pressione così misurato; la distanza percettiva tra un suono A di pressione \qty{100}{\micro\pascal} e un suono B di \qty{200}{\micro\pascal} sarà analoga a quella tra un suono C di pressione \qty{1000}{\micro\pascal} e un suono D di \qty{1100}{\micro\pascal}, nonostante la differenza tra A e B sia la stessa che tra C e D, e cioè di \qty{100}{\micro\pascal}.

Invece, la distanza percettiva tra un suono A di pressione \qty{100}{\micro\pascal} e un suono B di \qty{200}{\micro\pascal} sarà decisamente diversa rispetto a quella tra un suono C di pressione \qty{1000}{\micro\pascal} e un suono D di \qty{2000}{\micro\pascal}: l'osservazione rilevante è che il rapporto di pressione tra A e B è lo stesso che tra C e D, e cioè $\frac{1}{2}$. 

Il fatto che tendiamo a graduare molti fenomeni secondo i rapporti moltiplicativi tra le grandezze fisiche coinvolte piuttosto che secondo le loro differenze è talvolta considerata una legge fondamentale della percezione umana, chiamata \emph{legge di Fechner}. Guardando questa relazione moltiplicativa da un punto di vista leggermente diverso, possiamo dire che la nostra scala percettiva è tendenzialmente \emph{logaritmica} rispetto alla grandezza fisica osservata, dal momento che lo stesso rapporto moltiplicativo corrisponde alla stessa differenza tra i logaritmi. Tornando all'ultimo esempio,
\begin{equation}
\begin{aligned}
log_{10}100 = 2\\
log_{10}200 \approx 2.301\\
log_{10}1000 = 3\\
log_{10}2000 \approx 3.301\\
log_{10}200 - log_{10}100 = log_{10}2000 - log_{10}1000 \approx 0.301
\end{aligned}
\end{equation}
e anche, per la definizione stessa di logaritmo,
\begin{equation}
log_{10}\frac{200}{100} = log_{10}\frac{2000}{1000} = log_{10}2 \approx 0.301 
\end{equation}
e, già che ci siamo,
\begin{equation}
log_{10}\frac{100}{200} = log_{10}\frac{1000}{2000} = log_{10}\frac{1}{2} \approx -0.301 
\end{equation}

Da questa constatazione discende la definizione di bel (\unit{B}), un'unità di misura adimensionale che, in una delle sue possibili definizioni, esprime il logaritmo decimale del rapporto tra due intensità sonore:
\begin{equation}
L_{B} = log_{10}\frac{I}{I_0}
\end{equation}
dove $I$ è l'intensità sonora che vogliamo misurare rispetto a un'intensità di riferimento $I_0$. 

Molto più spesso del bel, in realtà, è usato il decibel, corrispondente alla decima parte del bel:
\begin{equation}
L_{dB} = 10 \cdot log_{10}\frac{I}{I_0}
\end{equation}

Formule analoghe possono essere usate per esprimere la relazione tra due livelli di energia sonora o di potenza sonora, dal momento che si tratta di grandezze proporzionali tra loro:
\begin{equation}
L_{dB} = 10 \cdot log_{10}\frac{E}{E_0}
\end{equation}
\begin{equation}
L_{dB} = 10 \cdot log_{10}\frac{P}{P_0}
\end{equation}

Una proprietà fondamentale di questa definizione di decibel è la sua coerenza con un'altra definizione equivalente, ma riferita al rapporto tra due pressioni sonore:
\begin{equation}
L_{dB} = 20 \cdot log_{10}\frac{\Delta p}{\Delta p_0}
\end{equation}

L'equivalenza tra queste due definizioni discende dalla relazione quadratica tra pressione e intensità sonora e dalla proprietà che permette di ``tramutare'' un esponente dentro il logaritmo in un coefficiente fuori dal logaritmo:
\begin{equation}
log x^a = a \cdot log x
\end{equation}
e quindi:
\begin{equation}
\begin{aligned}
L_{dB} = 10 \cdot log_{10}\frac{I}{I_0}\\
= 10 \cdot log_{10}\frac{\Delta p^2}{\Delta p_0^2}\\
= 10 \cdot log_{10}(\frac{\Delta p}{\Delta p_0})^2\\
= 2 \cdot 10 log_{10}\frac{\Delta p}{\Delta p_0}\\
= 20 log_{10}\frac{\Delta p}{\Delta p_0}
\end{aligned}
\end{equation}

Formule analoghe a quella che definisce il decibel in riferimento alla pressione sonora possono essere usate considerando altre grandezze a essa proporzionali, come l'ampiezza e la tensione elettrica:
\begin{equation}
L_{dB} = 20 \cdot log_{10}\frac{A}{A_0}
\end{equation}
\begin{equation}
L_{dB} = 20 \cdot log_{10}\frac{V}{V_0}
\end{equation}

Quale che sia il valore di riferimento di pressione sonora, intensità, potenza, voltaggio, ampiezza o altro, ci sono alcune considerazioni generali che possono essere fatte:

\begin{itemize}

\item A un suono al livello di riferimento corrisponde sempre un livello di \qty{0}{dB}: se per esempio $I$ è uguale a $I_0$, e poiché il logaritmo di 1 in qualsiasi base è 0, allora
\begin{equation}
L_{dB} = 10 \cdot log_{10}\frac{I}{I_0} = 10 \cdot log_{10}1 = 10 \cdot 0 = 0
\end{equation}
e lo stesso vale se invece dell'intensità si considera la pressione acustica, o l'ampiezza, o qualsiasi altra delle grandezze che abbiamo citato.

\item A suoni più forti del valore di riferimento (cioè con ampiezza, pressione, intensità ecc. superiori a esso) corrispondono livelli in decibel positivi; a suoni più deboli del valore di riferimento corrispondono livelli in decibel negativi. Infatti nel primo caso il rapporto tra valore misurato e valore di riferimento sarà maggiore di 1, e quindi il suo logaritmo maggiore di 0; nel secondo caso il rapporto sarà maggiore di 0 e minore di 1, e quindi il suo logaritmo minore di 0.

\item Il valore di riferimento non può essere il silenzio perfetto (corrispondente a un valore di intensità sonora, pressione sonora, ampiezza ecc. pari a 0), perché in questo caso metteremmo uno 0 sotto la linea di frazione e la divisione per 0 ha risultato indefinito.

\item In senso proprio, neppure il valore misurato può essere il silenzio perfetto, perché questo vorrebbe dire calcolare il logaritmo di 0 che pure non è definito. Con un leggero abuso di terminologia, si usa però spesso scrivere che il silenzio perfetto corrisponde a $-\infty$ dB. Ad ogni modo, il silenzio perfetto non esiste nel mondo fisico, ma solo nel contesto astratto delle rappresentazioni numeriche.

\item In ogni caso, anche quando il valore misurato o il valore di riferimento derivano da misurazioni con segno come la misurazione dell'ampiezza istantanea, nel calcolo del livello in dB si considera sempre il loro valore assoluto, perché il logaritmo dei numeri negativi non è definito.

\end{itemize}

La scala dei decibel, come dicevamo, rappresenta una modellizzazione semplice della maniera in cui la percezione umana gradua l'intensità del fenomeno sonoro. In maniera estremamente approssimativa, ma non completamente sbagliata, possiamo considerare che una differenza di 1, o 10, o 50 dB tra due suoni altrimenti identici abbia sempre lo stesso ``peso'' percettivo. Questo è particolarmente importante in elettroacustica e informatica musicale, contesti nei quali è semplice controllare in maniera precisa l'ampiezza di un suono senza modificare gli altri suoi parametri. 



\section{Livelli di riferimento convenzionali per i decibel}

Il decibel è una misura relativa, che però può essere impiegata in maniera assoluta se viene fissato un valore di riferimento per $A_0$, $\Delta p_0$, $V_0$, $W_0$ e così via. Esistono alcuni valori di riferimento convenzionali che danno luogo a scale assolute utili in contesti diversi. Queste comprendono, tra le altre:

\begin{itemize}

\item{$dB_{SPL}$}, dove SPL sta per \emph{sound pressure level}: misura la pressione acustica rispetto a una $P_0$ di \qty{20}{\micro\pascal}, corrispondenti approssimativamente alla soglia di udibilità umana. Quando nel linguaggio comune si dice ``in quel locale la musica era a 90 dB'' si sottintende che si sta parlando di $dB_{SPL}$. Tutti i suoni udibili avranno livelli non negativi in $dB_{SPL}$. La pressione sonora di \qty{20}{\pascal} che viene spesso indicata come la massima che l'orecchio umano può tollerare senza subire danni permanenti corrisponderà quindi a un livello sonoro di 120 dB\ped{SPL}:
\begin{equation}
L_{dB} = 20 \cdot log_{10}\frac{20 \unit{Pa}}{20 \unit{\micro\pascal}} = 20 \cdot log_{10}10^6 = 20 \cdot 6 = 120
\end{equation}

\item{$dB_{FS}$}, dove FS sta per \emph{full scale}: misura l'ampiezza rispetto al massimo livello rappresentabile in un sistema audio digitale a rappresentazione intera o in virgola fissa.%
\footnote{Maggiori dettagli su queste rappresentazioni saranno dati più avanti. Per il momento, possiamo considerare come minimo che si tratta delle rappresentazioni usate dalle interfacce audio e dalla maggior parte dei formati di file audio.}
Tutti i suoni correttamente rappresentabili in un tale sistema avranno livelli non positivi in $dB_{FS}$.

\end{itemize}

Esistono numerose altre scale di riferimento usate in elettroacustica e psicoacustica, ma non le approfondiremo qui.




\section{Piccola aritmetica dei decibel}

Succede spesso di dover convertire in decibel un rapporto di ampiezze, o viceversa. Questo può essere facilmente fatto con Max tramite gli oggetti \emph{atodb} e \emph{dbtoa}, ma è utile avere in mente almeno alcune identità semplici:

\begin{itemize}

\item Se un dato rapporto di ampiezza corrisponde a un guadagno pari a $L_{dB}$, allora l'inverso di tale rapporto corrisponderà a un guadagno pari a $-L_{dB}$: se

\begin{equation}
20 \cdot log_{10} \frac{A}{A_0} = L_{dB}
\end{equation}
allora
\begin{equation}
\begin{aligned}
20 \cdot log_{10} \frac{A_0}{A} =\\
20 \cdot log_{10} \frac{1}{\frac{A_0}{A}} =\\
20 \cdot log_{10} (\frac{A}{A_0})^{-1} =\\
-1 \cdot 20 \cdot log_{10} \frac{A}{A_0} =\\
-1 \cdot L_{dB} =\\
-L_{dB}
\end{aligned}
\end{equation}

\item Raddoppiare l'ampiezza corrisponde a un guadagno di circa \qty{6}{dB}, dal momento che $log_{10} 2 \approx 0.301$ e quindi $20 \cdot log_{10} 2 \approx 6.02$. Poiché uno scarto di \qty{0.02}{dB} è nella maggior parte dei casi del tutto irrilevante, spesso si considera l'identità come esatta anziché approssimata.

\item Dimezzare l'ampiezza corrisponde a un guadagno di circa \qty{-6}{dB}.

\item Moltiplicare per 10 l'ampiezza corrisponde a un guadagno di esattamente \qty{20}{dB}, poiché $log_{10} 10 = 1$ e quindi $20 \cdot log_{10} 10 = 20$.

\item Dividere per 10 l'ampiezza corrisponde a un guadagno di esattamente \qty{-20}{dB}.

\end{itemize}

Considerando che, in virtù delle proprietà dei logaritmi che abbiamo visto sopra, moltiplicazioni dell'ampiezza corrispondono sempre a somme algebriche di decibel ed elevamenti a potenza dell'ampiezza corrispondono a moltiplicazioni di decibel, le identità riportate sopra possono essere utili per fare alcuni calcoli in maniera semplice. Per esempio:

\begin{itemize}

\item Un guadagno di 100, cioè $10 \cdot 10$, corrisponde a $20 + 20 = 40$ dB.

\item Un guadagno di 20, cioè $2 \cdot 10$, corrisponde a circa $6 + 20 = 26$ dB.

\item Un'attenuazione di 1000, cioè un guadagno di $\frac{1}{1000}$ o $10^{-3}$ corrisponde a $20 \cdot -3 = -60$ dB.

\item \qty{8}{dB}, cioè $20 - 2 \cdot 6$ dB, corrispondono a un guadagno di circa $10 \cdot 2^{-2} = 10 \cdot \frac{1}{4} = 2.5$

\item \qty{10}{dB}, cioè $20 \cdot \frac{1}{2}$ dB, corrispondono a un guadagno di $10^\frac{1}{2} = \sqrt{10} \approx {3.16}$

\end{itemize}

Queste operazioni corrispondono, tra l'altro, al collegamento in serie di amplificatori o attenuatori, come avviene normalmente nel percorso del segnale lungo una catena elettroacustica o all'interno di un mixer. Se per esempio lo stadio di guadagno di un mixer è regolato a \qty{40}{dB}, il fader del corrispondente canale a \qty{-10}{dB} e il livello di uscita a \qty{-20}{dB} (assumendo che non ci siano altre trasformazioni del segnale, come equalizzazioni o altro) sapremo che il segnale in uscita dal mixer sarà $40-10-20 = 10$ dB più forte rispetto a quello in entrata, e quindi che il voltaggio in uscita sarà circa pari al voltaggio in entrata moltiplicato per 3.16.



  